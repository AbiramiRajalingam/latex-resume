% \iffalse meta-comment
%
% Copyright (C) 1993-2021
%
% The LaTeX Project and any individual authors listed elsewhere
% in this file.
%
% This file is part of the Standard LaTeX `Tools Bundle'.
% -------------------------------------------------------
%
% It may be distributed and/or modified under the
% conditions of the LaTeX Project Public License, either version 1.3c
% of this license or (at your option) any later version.
% The latest version of this license is in
%    https://www.latex-project.org/lppl.txt
% and version 1.3c or later is part of all distributions of LaTeX
% version 2005/12/01 or later.
%
% The list of all files belonging to the LaTeX `Tools Bundle' is
% given in the file `manifest.txt'.
%
% \fi
%
%%
%
%
\def\FMithmInfo{2014/10/28 v2.2c Theorem extension package (FMi)}
%
% \ProvidesFile{theorem.dtx}[\FMithmInfo]
%\iffalse   % this is a METACOMMENT !
%%
%% Package `theorem' to use with LaTeX2e
%% Copyright (C) 1989-1995 Frank Mittelbach, all rights reserved.
%<+package>\@ifundefined{theorem@style}{}{\endinput}
%<+package>\NeedsTeXFormat{LaTeX2e}[1995/06/01]
%<+package>\ProvidesPackage{theorem}
%<+package>         [\FMithmInfo]
%
% \fi
%
% \changes{v1.0}{87/05/07}{First draft.}
% \changes{v1.0a}{87/05/07}{`thistheoremstyle' added.}
% \changes{v1.1a}{87/07/29}{New concept: redefine the internal
%                          \protect\LaTeX{} commands `@xnthm' etc.}
% \changes{v1.1b}{87/07/31}{`theoremfont' used to specify font.}
% \changes{v1.2a}{88/01/08}{`@endtheorem' now globally defined.}
% \changes{v1.2b}{88/01/11}{`newtheorem' only allowed in the preamble.}
% \changes{v1.3a}{88/02/14}{`theoremfont' now a toks register.}
%
% \changes{v2.0a}{88/11/16}{New implementation and new documentation.}
% \changes{v2.0a}{88/11/16}{No layout changes but internal modifications
%                           old substyles won't work.}
% \changes{v2.0b}{89/01/30}{`global' used for all definitions.}
% \changes{v2.0c}{89/06/15}{Documentation translated to English.
%                            Thanks to LLL and Nina}
% \changes{v2.1a}{90/02/15}{Support for new font selection.}
% \changes{v2.2a}{94/02/03}{Cleanup for LaTeX2e}
%
% \DoNotIndex{\@endparenv,\@ifdefinable,\@ifnextchar,\@ifundefined}
% \DoNotIndex{\@namedef,\@nameuse,\@spaces,\@tempa,\@warning}
% \DoNotIndex{\ }
% \DoNotIndex{\begingroup,\bfseries}
% \DoNotIndex{\csname}
% \DoNotIndex{\def,\do,\docdate}
% \DoNotIndex{\endcsname,\endgroup,\endinput,\endtrivlist,\expandafter}
% \DoNotIndex{\filedate,\fileversion}
% \DoNotIndex{\gdef,\global}
% \DoNotIndex{\hbox,\hskip}
% \DoNotIndex{\ignorespaces,\input,\itshape,\item}
% \DoNotIndex{\labelsep,\let,\llap}
% \DoNotIndex{\makeatletter}
% \DoNotIndex{\noexpand}
% \DoNotIndex{\refstepcounter,\relax,\rlap}
% \DoNotIndex{\setlength,\slshape,\space,\string,\strut}
% \DoNotIndex{\the,\trivlist,\typeout}
% \DoNotIndex{\vbox}
% \DoNotIndex{\xdef}
%
% \def\deflabel#1{\ttfamily #1\hfill}
% \def\deflist#1{\begin{list}{}{\settowidth\labelwidth{\ttfamily #1}%
%                               \setlength\leftmargin\labelwidth
%                               \addtolength\leftmargin\labelsep
%                               \let\makelabel\deflabel}}
% \def\enddeflist{\end{list}}
% \def\envfont{\normalfont\ttfamily}
%
% \MakeShortVerb{\|}
%
% \newcommand{\TUB}{{\sl TUGboat\/}}
%
% \GetFileInfo{theorem.dtx}
% \title{An Extension of the
%        \LaTeX{} theorem environment\thanks{This file
%        has version number \fileversion, last
%        revised \filedate.}}
% \author{Frank Mittelbach}
%
% \MaintainedByLaTeXTeam{tools}
% \maketitle
%
% \begin{abstract}
%    The macros described in this paper yield an extension of the
%    \LaTeX{} theorem mechanism. It is designed is to satisfy the
%    different requirements of various journals. Thus, the
%    layout of the ``theorems'' can be manipulated by determining a
%    ``style''. This article describes not only the use, but
%    also the definition, of the necessary macros.
% \end{abstract}
%
% \section*{Preface to version 2.2}
%
% For \LaTeXe{} this package did not need any fundamental changes. I
% only modified the messages generated so that theorem layout styles
% will show up with the |\listfiles| command and cleaned the section
% on the New Font Selection Scheme since this is now included in
% \LaTeX.
%
% \section*{Preface to version 2.1}
%
% This version is identical to 2.0g described in \TUB\ 10\#3 except
% for some internal defaults which are now set depending on the used
% font selection scheme.
%
% This was done to avoid unpleasant surprises if the new font selection
% scheme is in force. For further details see
% section~\ref{sec:newfonts} and \cite{bk:GMS94}.
%
% \section{Introduction}
%
% For our purposes here, ``theorems'' are labelled enunciations,
% often set off from the main text by extra space and a font change.
% Theorems, corollaries, conjectures, definitions, and remarks
% are all instances of ``theorems''.  The ``header'' of these structures
% is composed of a label (such as \textsc{Theorem} or \textsc{Remark})
% and a number which serializes an item in the sequence of items
% with the same label.
%
% Shortly after the introduction of \LaTeX{} at the Fachbereich
% Mathematik in Mainz, the desire to manipulate the layout of
% ``theorems'' arose.  In Mainz, the following two conventions came
% into general use:
% \begin{enumerate}
%   \item The number of the theorem is shown in the margin.
%   \item There is a line break at
%         the end of the theorem header.
% \end{enumerate}
%
% Additionally, some journals require different formats which depend on
% the ``sort of theorem'': e.g.\ often remarks and definitions are set
% in |\upshape|, while |\itshape| is employed for main theorems.
%
% Confronted with these requirements, a theorem environment was
% developed in Mainz which allows separate determination of the
% layout of the ``theorems sets'', comparable to |\pagestyle|.
%
%
% \section{The user interface}
% \subsection{Defining new theorem sets}
%
% \DescribeMacro\newtheorem
% As in the original \LaTeX{} version, the command |\newtheorem|
% defines a new ``theorem set'' or ``theorem-like structure''.
% Two required arguments name the new environment and give the text
% to be typeset with each instance of the new ``set'', while
% an optional argument determines how
% the ``set'' is enumerated:
% \begin{description}
%    \item[\ttfamily \bslash newtheorem\{foo\}\{bar\}]
%       The theorem set {\envfont foo} (whose name is \texttt{bar})
%       uses its own counter.
%    \item[\ttfamily \bslash newtheorem\{foo2\}{[foo]}\{bar2\}]
%       The theorem set {\envfont foo2} (printed name \texttt{bar2})
%       uses the same counter as the theorem set \texttt{foo}.
%    \item[\ttfamily \bslash newtheorem\{foo3\}\{bar3\}{[section]}]
%       The theorem set {\envfont foo3} (printed name \texttt{bar3}) is
%       enumerated within the counter \texttt{section}, i.e.\ with every
%       new |\section| the enumeration begins again with 1, and
%       the enumeration is composed from the section-number and the
%       theorem counter itself.
% \end{description}
%
% \DescribeMacro\theoremstyle
% Additionally, the command |\theoremstyle| can define the layout
% of various, or all, theorem sets.  It should be noted that any
% theorem set defined by |\newtheorem| is typeset in the
% |\theoremstyle| that is current at the time of the definition.
% Thus, the following
% \begin{verbatim}
%    \theoremstyle{break}        \newtheorem{Cor}{Corollary}
%    \theoremstyle{plain}        \newtheorem{Exa}{Example}[section]
%\end{verbatim}                             ^^A No blank at line start
% leads to the result that the set {\envfont Cor} is formatted in the
% style \texttt{break}, while the set {\envfont Exa} and all the
% following ones are formatted in the style \texttt{plain}, unless
% another |\theoremstyle| follows.  Since the definitions installed by
% |\newtheorem| are global, one also can limit |\theoremstyle| locally
% by grouping braces.
%
% \DescribeMacro\theorembodyfont
% The choice of the font for the theorem body is completely
% independent of the chosen |\theoremstyle|; this has proven to
% be very advantageous.  For example,
% \begin{verbatim}
%   {\theorembodyfont{\upshape}          \newtheorem{Rem}{Remark}}
%\end{verbatim}
% defines a theorem set {\envfont Rem}, which will be set in
% |\upshape| in the current layout (which in our example is
% \texttt{plain}).  As with |\theoremstyle|, the |\theorembodyfont|
% chosen is that current at the time of |\newtheorem|.  If
% |\theorembodyfont| is not specified or one defines
% |\theorembodyfont{}|, then the font used will be that defined by the
% |\theoremstyle|.
%
% \DescribeMacro\theoremheaderfont It is also possible to customize
% the font used for the theorem headers.  This is, however, a global
% declaration, and therefore there should be at most one
% |\theoremheaderfont| declaration in the preamble.\footnote{If it is
% actually necessary to have different header fonts, one has to define
% new theorem styles (substituting the desired font) or specify the
% information directly in the \texttt{\bslash newtheorem} declaration
% (the unclean variant).}
%
% \DescribeMacro\theorempreskipamount
% \DescribeMacro\theorempostskipamount Two additional parameters
% affect the vertical space around the theorem environments:
% |\theorempreskipamount| and |\theorempostskipamount| define,
% respectively, the spacing before and after such an environment.
% These parameters apply for all theorem sets and can be manipulated
% with the ordinary length macros.  They are rubber lengths,
% (`\textsf{skips}'), and therefore can contain \texttt{plus} and
% \texttt{minus} parts.
%
% Since the definition of theorem sets should---most sensibly---be
% placed in the preamble, we only allow installation there.
% It is therefore possible to release the memory used here after
% |\begin{document}|, in order to make room for other
% applications.
%
%
% \subsection{Existing theorem styles}
%
% The following theorem styles exist to date:
% \begin{deflist}{marginbreak}
%    \item[plain]
%       This theorem style emulates the original \LaTeX{} definition,
%       except that additionally the parameters
%       |\theorem...skipamount| are used.
%    \item[break]
%       In this style, the theorem header is followed by a line
%       break.
%    \item[marginbreak]
%       The theorem number is set in the margin, and there is a line
%       break as in \texttt{break}.
%    \item[changebreak]
%       Like \texttt{break}, but with header number and text
%       interchanged.
%    \item[change]
%       Header number and text are interchanged, without a line break.
%    \item[margin]
%       The number is set in the left margin, without a line break.
% \end{deflist}
% All styles (except \texttt{plain}) select |\slshape| as the default
% |\theorembodyfont|.
%
%
% \subsection{Examples}
%
% Given the above theorem sets {\envfont Cor}, {\envfont Exa}
% and {\envfont Rem},
% suppose that the preamble also contains the declarations:
% \begin{verbatim}
%    \theoremstyle{marginbreak}   \newtheorem{Lem}[Cor]{Lemma}
%    \theoremstyle{change}
%    \theorembodyfont{\itshape}        \newtheorem{Def}[Cor]{Definition}
%
%    \theoremheaderfont{\scshape}
%\end{verbatim}
% Then the following are some typical examples of the typeset output
% resulting from their use.
%
% \begin{Cor}
%    This is a sentence typeset in the theorem environment {\envfont
%    Cor}.
% \end{Cor}
%
% \begin{Exa}
%    This is a sentence typeset in the theorem environment {\envfont
%    Exa}.
% \end{Exa}
%
% \begin{Rem}
%    This is a sentence typeset in the theorem environment {\envfont
%    Rem}.
% \end{Rem}
%
% \begin{Lem}[Ben User]
%    This is a sentence typeset in the theorem environment {\envfont
%    Lem}.
% \end{Lem}
%
% \begin{Def}[Very impressive Definition]
%    This is a sentence typeset in the theorem environment {\envfont
%    Def}.
% \end{Def}
%
% The last two examples show the effect of the optional argument to
% a theorem environment (it is the text typeset in parentheses).
%
%
% \section{Special Considerations}
% \label{sec:newfonts}
%
% Theoremheader and body are implemented as a unit. This means that
% the |\theoremheaderfont|  will inherit characteristics of the
% |\theorembodyfont| in \LaTeXe. Thus, if for example
% |\theorembodyfont| is |\itshape| and |\theoremheaderfont|
% is |\bfseries| the font selected for the header will have the
% characteristics `bold extended italic'. If this is not desired one
% should set the |\theoremheaderfont| to something like
% \begin{verbatim}
%      \theoremheaderfont{\normalfont\bfseries}
%\end{verbatim}
% i.e.\ supplying all necessary font information explicitly.
%
%
%
% \section{Acknowledgements}
%
% The publication of this set of macros was only possible with the
% help of Christina Busse (translating the manuscript into English),
% Joachim Pense (playing the r\^ole of typist), Chris Rowley
% (looking everything over) and many others providing useful
% suggestions.
%
% \MaybeStop{%
%    \begin{thebibliography}{9}
%       \bibitem{bk:GMS94} \textsc{M.~Goossens}, \textsc{F.~Mittelbach}
%         and \textsc{A.~Samarin}.
%         \newblock The \LaTeX{} Companion.
%         \newblock
%         Addison-Wesley, Reading, Massachusetts, 1994.
%       \bibitem{latex:source} \textsc{Lamport, Leslie.} \newblock
%                               \texttt{latex.tex}, version 2.09, date
%                               Feb. 1990.
%    \end{thebibliography}
%    \PrintIndex
%    \PrintChanges}
%
%
% \section{The documentation driver file}
%
% The next bit of code contains the documentation driver file for
% \TeX{}, i.e., the file that will produce the documentation you are
% currently reading. It will be extracted from this file by the
% \texttt{docstrip} program. Since it is the first code in the file
% one can alternatively process this file directly with \LaTeXe{} to
% obtain the documentation.
%
%    \begin{macrocode}
%<*driver>
\documentclass{ltxdoc}

\usepackage{theorem}

%<+driver>% The next few lines define theorem sets which are used
%<+driver>% in the example section of the documentation.

\theoremstyle{break}       \newtheorem{Cor}{Corollary}
\theoremstyle{plain}       \newtheorem{Exa}{Example}[section]
{\theorembodyfont{\upshape}\newtheorem{Rem}{Remark}}
\theoremstyle{marginbreak} \newtheorem{Lem}[Cor]{Lemma}
\theoremstyle{change}
\theorembodyfont{\itshape} \newtheorem{Def}[Cor]{Definition}

\theoremheaderfont{\scshape}

\RecordChanges

\begin{document}
   \DocInput{theorem.dtx}
\end{document}
%</driver>
%    \end{macrocode}
%
% \section{Definition of the Macros}
%
% If the file has been loaded before, we abort immediately. If not the
% package announces itself (this is actually done at the very top if
% the file---the way it is done isn't good style so don't copy it).
% \changes{v2.0e}{89/07/19}{Spaces removed from `typeout'.}
% \changes{v2.2b}{95/11/19}{Announce moved to top of file}
%    \begin{macrocode}
%<*package>
%\@ifundefined{theorem@style}{}{\endinput}
%\def\FMithmInfo{1995/11/19 v2.2b Theorem extension package (FMi)}
%\ProvidesPackage{theorem}[\FMithmInfo]
%    \end{macrocode}
%
%
% \subsection{Definition of theorem styles and fonts}
%
% All the definitions in this file are done globally to allow
% inputting this file inside a group.
%
% \begin{macro}{\theoremstyle}
%    Before a theorem style can be installed, the chosen style must be
%    known. For that reason, we must test to see that
%    |\th@|\meta{style} is known or, more precisely, that it is
%    different from |\relax|. If the style is not known then
%    |\th@plain| is used.
%    \begin{macrocode}
\gdef\theoremstyle#1{%
   \@ifundefined{th@#1}{\@warning
          {Unknown theoremstyle `#1'. Using `plain'}%
          \theorem@style{plain}}%
%    \end{macrocode}
%    We save the theorem style to be used in the token register
%    |\theorem@style|.
%    \begin{macrocode}
      {\theorem@style{#1}}%
%    \end{macrocode}
%    Now we ``evaluate'' the theorem style: this means, we call the
%    macro |\th@|\meta{style} which will activate the relevant
%    definitions which are contained in a separate file.  This is done
%    in a group to suppress changes to the current font.  This could
%    otherwise pose problems together with the new font selection
%    scheme\footnote{When I printed the original article using the new
%    font selection scheme I ended with a document with slanted
%    typefaces (text headings and all) simply because one of the
%    theorem styles used \texttt{\bslash sl} at toplevel.} if the
%    |\th@|\meta{style} is evaluated a second time.
% \changes{v2.1a}{90/02/15}{Group added around `th@..}
% \changes{v2.1a}{90/02/15}{Reseting of `..@begintheorem removed.}
%    \begin{macrocode}
      \begingroup
        \csname th@\the\theorem@style \endcsname
      \endgroup}
%    \end{macrocode}
% \end{macro}
%
%
% \begin{macro}{\@begintheorem}
% \changes{v2.1a}{90/02/15}{Relaxed at toplevel.}
% \begin{macro}{\@opargbegintheorem}
% \changes{v2.1a}{90/02/15}{Relaxed at toplevel.}
%    We reset |\@begintheorem| and |\@opargbegintheorem| to
%    |\relax| since these commands are no longer necessary at
%    toplevel. This will save a few tokens.
%    \begin{macrocode}
\global\let\@begintheorem\relax
\global\let\@opargbegintheorem\relax
%    \end{macrocode}
% \end{macro}
% \end{macro}
%
% \begin{macro}{\theorem@style}
%    Obviously the \textsf{token} register used above has to be
%    allocated. To assure the utmost compatibility with the
%    original \LaTeX{} definition, we set the default theorem style to
%    \texttt{plain}, which implements the usual \LaTeX{} convention.
% \changes{v2.0f}{89/07/20}{Unnecessary `global' removed.}
%    \begin{macrocode}
\newtoks\theorem@style
\global\theorem@style{plain}
%    \end{macrocode}
% \end{macro}
%
% \begin{macro}{\theorembodyfont}
% \begin{macro}{\theorem@bodyfont}
%    If the |\theorembodyfont| is set by the user then it should not
%    interact with the default font set in the theorem style. When the
%    new font selection is in force this may happen if, for example,
%    the default is |\itshape| and the new |\theorembodyfont| is
%    |\sffamily|. So we add a |\reset@font| command in front of the
%    user definition.
% \changes{v2.0f}{89/07/20}{Unnecessary `global' removed.}
% \changes{v2.0g}{89/09/19}{`theoremfont to `theorembodyfont.}
% \changes{v2.1c}{92/06/22}{`reset@font added and logic changed.}
% \changes{v2.1d}{92/10/28}{Reset to empty if arg is empty}
%    \begin{macrocode}
\gdef\theorembodyfont#1{%
%    \end{macrocode}
%    We check if the argument supplied is empty and if so put nothing
%    into the |\theorem@bodyfont| token register to allow for
%    |\theorembodyfont{}| as a mean of using the default of the
%    current |\theoremstyle|.
%    \begin{macrocode}
   \def\@tempa{#1}%
   \ifx\@tempa\@empty
    \theorem@bodyfont{}%
   \else
    \theorem@bodyfont{\reset@font#1}%
   \fi
   }
\newtoks\theorem@bodyfont
\global\theorem@bodyfont{}
%    \end{macrocode}
% \end{macro}
% \end{macro}
%
%
% \begin{macro}{\theoremheaderfont}
% \changes{v2.0g}{89/09/19}{Macro added.}
%    The font for the theorem headers is handled differently because
%    this definition applies to all theorem styles.
%    \begin{macrocode}
\gdef\theoremheaderfont#1{\gdef\theorem@headerfont{#1}%
%    \end{macrocode}
%    After using the macro once it is redefined to produce an error
%    message.
%    \begin{macrocode}
       \gdef\theoremheaderfont##1{%
        \typeout{\string\theoremheaderfont\space should be used
                 only once.}}}
%    \end{macrocode}
% \end{macro}
%
% \begin{macro}{\theorem@headerfont}
% \changes{v2.0g}{89/09/19}{Macro added.}
%    To set the |\theorem@headerfont| default we first test if
%    the new fontselection scheme is in force.
% \changes{v2.1a}{90/02/15}{Support for new font selection.}
%    \begin{macrocode}
\ifx\upshape\undefined
%    \end{macrocode}
%    If not we define it to expand into |\bfseries|. We don't use
%    |\let| just in case a following style option redefines this
%    macro.
%    \begin{macrocode}
\gdef\theorem@headerfont{\bfseries}
%    \end{macrocode}
%    Otherwise we reset the current shape before calling |\bfseries|.
%    \begin{macrocode}
\else \gdef\theorem@headerfont{\normalfont\bfseries}\fi
%    \end{macrocode}
% \end{macro}
%
% \begin{macro}{\th@plain}
% \begin{macro}{\th@break}
% \begin{macro}{\th@marginbreak}
% \begin{macro}{\th@changebreak}
% \begin{macro}{\th@change}
% \begin{macro}{\th@margin}
%    The different styles are defined in macros such as
%    |\th@plain|.  Since memory space is precious in
%    ``non-Big-versions'', we have to avoid offering too many unused
%    definitions.  Therefore we define these styles in separate files
%    that can be loaded on demand.  Thus the commands themselves only
%    load these files. We use |\@input@| a \LaTeXe{} internal command
%    that ensures that the file will be listed with |\listfiles|
% \changes{v2.2a}{94/02/02}{Use `@input@ to load theorem layout files}
%    \begin{macrocode}
\gdef\th@plain{\@input@{thp.sty}}
\gdef\th@break{\@input@{thb.sty}}
\gdef\th@marginbreak{\@input@{thmb.sty}}
\gdef\th@changebreak{\@input@{thcb.sty}}
\gdef\th@change{\@input@{thc.sty}}
\gdef\th@margin{\@input@{thm.sty}}
%    \end{macrocode}
%    This list will be expanded when new styles become
%    available.  For testing,  just append new theorem substyles
%    as document options.
% \end{macro}
% \end{macro}
% \end{macro}
% \end{macro}
% \end{macro}
% \end{macro}
%
%
%
% \subsection{Definition of a new theorem set}
%
% As already pointed out, a new theorem environment can be defined in
% three different ways:
% \begin{verbatim}
%    \newtheorem{Lem}{Lemma}
%    \newtheorem{Lem}{Lemma}[section]
%    \newtheorem{Lem}[Theorem]{Lemma}
%\end{verbatim}
% The function of the macro |\newtheorem| is to recognize these cases
% and then to branch into one of the three macros |\@ynthm|, |\@xnthm|
% or |\@othm|.  This mechanism is adopted unchanged from
% \cite{latex:source}; the essential point here is that, for example,
% in the second case, the arguments \texttt{Lem}, \texttt{Lemma} and
% \texttt{section} are passed over to the macro |\@xnthm|.
%
% We inspect this case first because the others present fewer problems,
% and thus are easily derived from this one.
%
% \begin{macro}{\@xnthm}
%    For our example arguments, the macro |\@xnthm| must fulfill
%    the following:
%    \begin{itemize}
%       \item
%          Define a new \LaTeX-counter `Lem'
%       \item
%          reset this counter within a |\section|
%       \item
%          define the macro |\theLem|
%       \item
%          define the environment macros |\Lem| and
%          |\endLem| using the current
%          |\theoremstyle| and |\theorem@bodyfont|.
%    \end{itemize}
%    Obviously, all this should happen only if the first argument
%    of |\@xnthm| (i.e.\ \texttt{Lem} in our example) is chosen so
%    as not to conflict with any previously defined commands or
%    environments.  This test is performed by the \LaTeX{} macro
%    |\@ifdefinable|.
%    \begin{macrocode}
\gdef\@xnthm#1#2[#3]{\expandafter\@ifdefinable\csname #1\endcsname
%    \end{macrocode}
%    Therefore, the first argument of |\@ifdefinable| is the
%    expansion (in the example, |\Lem|) of
%    |\csname||#1\endcsname|.  The second argument is
%    executed only if the test has been completed successfully.
%    \begin{macrocode}
   {%
%    \end{macrocode}
%    Now we define the new counter. The names of the \LaTeX{}
%    macros employed should speak for themselves:
% \changes{v2.2b}{1995/10/07}{Use \cs{@newctr}}
% \changes{v2.2c}{1995/11/23}{Use \cs{@newctr} properly.}
%    \begin{macrocode}
    \@definecounter{#1}\@newctr{#1}[#3]%
%    \end{macrocode}
%    Using |\@newctr| will give a proper error message if the counter
%    in |#3| is not defined.
%    In defining `|\theLem|' we must generate the desired macro
%    name by use of |\expandafter| and |\csname|.
%    \begin{macrocode}
    \expandafter\xdef\csname the#1\endcsname
%    \end{macrocode}
%    An |\xdef| is used in order to make the definition global,
%    and to ensure that it contains the replacement texts of
%    |\@thmcountersep| and |\@thmcounter|.\footnote{These
%    two macros can be defined by the document style.  Their default
%    values produce a `\texttt{.}' as separation and an arabic
%    representation of the number.} However, not everything should be
%    expanded. For example, it saves space to use |\thesection|
%    instead of its---at times---lengthy expansion.
%    \begin{macrocode}
      {\expandafter \noexpand \csname the#3\endcsname
       \@thmcountersep \@thmcounter{#1}}%
%    \end{macrocode}
%    Thus with the defaults of \LaTeX{}, |\theLem| would be
%    replaced by the command sequence |\thesection.\arabic{Lem}|.
%
%    We will now look at the definition of the macro which is executed
%    at the beginning of the actual environment (in our example this
%    macro is |\Lem|). It should be noted that we use an
%    ``|\expandafter| trick'' to expand only certain parts of the
%    replacement text at the time of the definition.
%    \begin{macrocode}
    \def\@tempa{\global\@namedef{#1}}%
    \expandafter \@tempa \expandafter{%
%    \end{macrocode}
%    First, the macro that contains the current definitions of
%    |\@begintheorem| and |\@opargtheorem| should be called
%    up.  The name of this macro---as is already known---has the form
%    |\th@|\meta{theorem\ style}; therefore, it must be called by
%    \begin{macrocode}
      \csname th@\the \theorem@style
%    \end{macrocode}
%    In addition the default theorem font should be changeable, i.e.\
%    we have to insert the contents of |\theorem@bodyfont|. For
%    that reason, we expand even further, beyond |\endcsname|,
%    and thus insert the contents of the \textsf{token} register
%    |\theorem@bodyfont| in the replacement text.
%    \begin{macrocode}
            \expandafter \endcsname \the \theorem@bodyfont
%    \end{macrocode}
%    Now it is time to call the macro |\@thm| which takes over
%    the further processing.  It has two arguments: the current
%    counter name (in our example, \texttt{Lem}), and the text of the
%    label (in our example, \texttt{Lemma}).
%    \begin{macrocode}
     \@thm{#1}{#2}}%
%    \end{macrocode}
%    With this, the `sub-definition' is complete.  The macro
%    |\@endtheorem| ends a theorem environment and is, so far, nothing
%    but an |\endtrivlist|.  (Hence it is defined globally, and not
%    within the theorem styles.\footnote{This has to be changed as
%    soon as theorem styles that change \texttt{\bslash @endtheorem}
%    exist.  In such a case, all existing styles must be changed as
%    well since they will have to reset the macro.}) Therefore, we can
%    set it equivalent to the macro that ends the theorem set (in our
%    example, |\endLem|).  However, if some day theorem styles exist
%    that do change |\@endtheorem|, we would have to use the
%    commented-out line instead.
%    \begin{macrocode}
    \global \expandafter \let \csname end#1\endcsname \@endtheorem
%   \global\@namedef{end#1}{\@endtheorem}%
%    \end{macrocode}
%    With these commands all the required definitions are employed,
%    unless the test |\@ifdefinable| has failed.  Therefore, we
%    end the second argument of this macro and with it the definition
%    of |\@xnthm|.
%    \begin{macrocode}
   }}
%    \end{macrocode}
% \end{macro}
%
% \begin{macro}{\@ynthm}
%    The definition of |\@ynthm| is completely analogous. In
%    this case the new counter that is defined is not reset within
%    another counter; thus the definition of |\the...|
%    is simplified:
%    \begin{macrocode}
\gdef\@ynthm#1#2{\expandafter\@ifdefinable\csname #1\endcsname
   {\@definecounter{#1}%
    \expandafter\xdef\csname the#1\endcsname{\@thmcounter{#1}}%
%    \end{macrocode}
%    The rest of the definition corresponds literally to that of
%    |\@xnthm|:
%    \begin{macrocode}
    \def\@tempa{\global\@namedef{#1}}\expandafter \@tempa
     \expandafter{\csname th@\the \theorem@style \expandafter
     \endcsname \the\theorem@bodyfont \@thm{#1}{#2}}%
    \global \expandafter \let \csname end#1\endcsname \@endtheorem}}
%    \end{macrocode}
% \end{macro}
%
% \begin{macro}{\@othm}
%    The definition of |\@othm| does not contain anything new.
%    \begin{macrocode}
\gdef\@othm#1[#2]#3{%
%    \end{macrocode}
% \changes{v2.2b}{95/10/07}{Build in check for counter defined}
%    We do not define a new counter but instead use one that has
%    already been defined.  Thus the only definition we need is that
%    of this pseudo-counter (i.e.\ |\the|\meta{env.\ name}).
%    First we check if |#2| corresponds to a known counter name.
%    \begin{macrocode}
  \expandafter\ifx\csname c@#2\endcsname\relax
   \@nocounterr{#2}%
  \else
   \expandafter\@ifdefinable\csname #1\endcsname
   {\expandafter \xdef \csname the#1\endcsname
     {\expandafter \noexpand \csname the#2\endcsname}%
%    \end{macrocode}
%    All other parts of the definition can be adopted from
%    |\@xnthm|. We have to remember, though, that in this case
%    the name of the current counter and the theorem label have moved
%    to the second and third arguments.
%    \begin{macrocode}
    \def\@tempa{\global\@namedef{#1}}\expandafter \@tempa
     \expandafter{\csname th@\the \theorem@style \expandafter
     \endcsname \the\theorem@bodyfont \@thm{#2}{#3}}%
    \global \expandafter \let \csname end#1\endcsname \@endtheorem}%
  \fi}
%    \end{macrocode}
% \end{macro}
%
% \subsection{Macros that are employed in a {\envfont theorem}
%    environment}
%
% \begin{macro}{\@thm}
%    The macro |\@thm| has to increase the current counter.  Then,
%    depending on whether the environment has (or does not have) an
%    optional argument, it has to branch into either |\@begintheorem|
%    or |\@opargtheorem|.
%    \begin{macrocode}
\gdef\@thm#1#2{\refstepcounter{#1}%
%    \end{macrocode}
%    Now we start a {\envfont trivlist} environment, and give
%    |\@topsep| and |\@topsepadd| the values of the \textsf{skip}
%    registers |\theorempreskipamount| and |\theorempostskipamount|.
%    The value in |\@topsep| is the vertical space that is inserted by
%    the first (and only) |\item| in our |\trivlist| whilst
%    |\@topsepadd| is inserted by |\@endparenv| at the end of that
%    {\envfont trivlist} environment.  By using these registers, we
%    obtain the desired space around a {\envfont theorem} environment.
%    \begin{macrocode}
   \trivlist
   \@topsep \theorempreskipamount               % used by first \item
   \@topsepadd \theorempostskipamount           % used by \@endparenv
%    \end{macrocode}
%    Now we have to test whether an optional argument has been given.
% \changes{v2.0d}{89/07/18}{Forgotten percent added.}
%    \begin{macrocode}
   \@ifnextchar [%
%    \end{macrocode}
%    If there is an optional argument, we will call |\@ythm|, and
%    move the arguments read back into the input stream.
%    \begin{macrocode}
   {\@ythm{#1}{#2}}%
%    \end{macrocode}
%    If not, we call |\@begintheorem|. Its first argument is the
%    name of the theorem set (hence the second argument of
%    |\@thm|). Its second argument is the macro that produces the
%    current number.
%    \begin{macrocode}
   {\@begintheorem{#2}{\csname the#1\endcsname}\ignorespaces}}
%    \end{macrocode}
% \end{macro}
%
% \begin{macro}{\@xthm}
% \begin{macro}{\@ythm}
%    Both these macros were originally called by |\@thm|.  We do
%    not need |\@xthm| anymore, hence we reset it to
%    |\relax|. The definition of |\@ythm| has not changed at
%    all from its definition in \LaTeX. In order to make the macros
%    easier to understand, we will
%    nevertheless present it (commented out).
%    \begin{macrocode}
\global\let\@xthm\relax
% \def\@ythm#1#2[#3]{\@opargbegintheorem{#2}{\csname
%           the#1\endcsname}{#3}\ignorespaces}
%</package>
%    \end{macrocode}
%    The primitive |\ignorespaces| in |\@ythm| and
%    |\@thm| is needed to remove the spaces between the
%    |\begin{...}| and the actual text.
% \end{macro}
% \end{macro}
%
%
% \subsection{Definition of the theorem substyles}
%
% As already pointed out, the theorem substyles, defined below, are
% only loaded when necessary.  Note that all these substyles, except
% \texttt{plain}, have |\slshape| as the default body font.
%
% \subsubsection{The \texttt{plain} style}
%
% \label{dok:th@plain} As the following macros use |@|, we have
% to locally set the |\catcode| of this symbol to ``letter''.
% This happens within a group, so that we do not have to worry about
% which |\catcode| that symbol had before.
%    \begin{macrocode}
%<*thp>
\begingroup \makeatletter
%    \end{macrocode}
% Since we are now within a group, we must make all definitions
% globally.  First we make sure that \texttt{theorem.sty} is loaded.
% This will allow us to use this file as a document style option
% without having to call \texttt{theorem} itself as an option.  At the
% same time, we assure that at least version 2 is loaded, since
% |\theorem@style| was not defined in earlier versions.
%    \begin{macrocode}
\@ifundefined{theorem@style}{% \iffalse meta-comment
%
% Copyright (C) 1993-2024
%
% The LaTeX Project and any individual authors listed elsewhere
% in this file.
%
% This file is part of the Standard LaTeX `Tools Bundle'.
% -------------------------------------------------------
%
% It may be distributed and/or modified under the
% conditions of the LaTeX Project Public License, either version 1.3c
% of this license or (at your option) any later version.
% The latest version of this license is in
%    https://www.latex-project.org/lppl.txt
% and version 1.3c or later is part of all distributions of LaTeX
% version 2005/12/01 or later.
%
% The list of all files belonging to the LaTeX `Tools Bundle' is
% given in the file `manifest.txt'.
%
% \fi
%
%%
%
%
\def\FMithmInfo{2023/07/05 v2.2c Theorem extension package (FMi)}
%
% \ProvidesFile{theorem.dtx}[\FMithmInfo]
%\iffalse   % this is a METACOMMENT !
%%
%% Package `theorem' to use with LaTeX2e
%% Copyright (C) 1989-1995 Frank Mittelbach, all rights reserved.
%<+package>\@ifundefined{theorem@style}{}{\endinput}
%<+package>\NeedsTeXFormat{LaTeX2e}[1995/06/01]
%<+package>\ProvidesPackage{theorem}
%<+package>         [\FMithmInfo]
%
% \fi
%
% \changes{v1.0}{87/05/07}{First draft.}
% \changes{v1.0a}{87/05/07}{`thistheoremstyle' added.}
% \changes{v1.1a}{87/07/29}{New concept: redefine the internal
%                          \protect\LaTeX{} commands `@xnthm' etc.}
% \changes{v1.1b}{87/07/31}{`theoremfont' used to specify font.}
% \changes{v1.2a}{88/01/08}{`@endtheorem' now globally defined.}
% \changes{v1.2b}{88/01/11}{`newtheorem' only allowed in the preamble.}
% \changes{v1.3a}{88/02/14}{`theoremfont' now a toks register.}
%
% \changes{v2.0a}{88/11/16}{New implementation and new documentation.}
% \changes{v2.0a}{88/11/16}{No layout changes but internal modifications
%                           old substyles won't work.}
% \changes{v2.0b}{89/01/30}{`global' used for all definitions.}
% \changes{v2.0c}{89/06/15}{Documentation translated to English.
%                            Thanks to LLL and Nina}
% \changes{v2.1a}{90/02/15}{Support for new font selection.}
% \changes{v2.2a}{94/02/03}{Cleanup for LaTeX2e}
%
% \DoNotIndex{\@endparenv,\@ifdefinable,\@ifnextchar,\@ifundefined}
% \DoNotIndex{\@namedef,\@nameuse,\@spaces,\@tempa,\@warning}
% \DoNotIndex{\ }
% \DoNotIndex{\begingroup,\bfseries}
% \DoNotIndex{\csname}
% \DoNotIndex{\def,\do,\docdate}
% \DoNotIndex{\endcsname,\endgroup,\endinput,\endtrivlist,\expandafter}
% \DoNotIndex{\filedate,\fileversion}
% \DoNotIndex{\gdef,\global}
% \DoNotIndex{\hbox,\hskip}
% \DoNotIndex{\ignorespaces,\input,\itshape,\item}
% \DoNotIndex{\labelsep,\let,\llap}
% \DoNotIndex{\makeatletter}
% \DoNotIndex{\noexpand}
% \DoNotIndex{\refstepcounter,\relax,\rlap}
% \DoNotIndex{\setlength,\slshape,\space,\string,\strut}
% \DoNotIndex{\the,\trivlist,\typeout}
% \DoNotIndex{\vbox}
% \DoNotIndex{\xdef}
%
% \def\deflabel#1{\ttfamily #1\hfill}
% \def\deflist#1{\begin{list}{}{\settowidth\labelwidth{\ttfamily #1}%
%                               \setlength\leftmargin\labelwidth
%                               \addtolength\leftmargin\labelsep
%                               \let\makelabel\deflabel}}
% \def\enddeflist{\end{list}}
% \def\envfont{\normalfont\ttfamily}
%
% \MakeShortVerb{\|}
%
% \newcommand{\TUB}{{\sl TUGboat\/}}
%
% \GetFileInfo{theorem.dtx}
% \title{An Extension of the
%        \LaTeX{} theorem environment\thanks{This file
%        has version number \fileversion, last
%        revised \filedate.}}
% \author{Frank Mittelbach}
%
% \MaintainedByLaTeXTeam{tools}
% \maketitle
%
% \begin{abstract}
%    The macros described in this paper yield an extension of the
%    \LaTeX{} theorem mechanism. It is designed to satisfy the
%    different requirements of various journals. Thus, the
%    layout of the ``theorems'' can be manipulated by determining a
%    ``style''. This article describes not only the use, but
%    also the definition, of the necessary macros.
% \end{abstract}
%
% \section*{Preface to version 2.2}
%
% For \LaTeXe{} this package did not need any fundamental changes. I
% only modified the messages generated so that theorem layout styles
% will show up with the |\listfiles| command and cleaned the section
% on the New Font Selection Scheme since this is now included in
% \LaTeX.
%
% \section*{Preface to version 2.1}
%
% This version is identical to 2.0g described in \TUB\ 10\#3 except
% for some internal defaults which are now set depending on the used
% font selection scheme.
%
% This was done to avoid unpleasant surprises if the new font selection
% scheme is in force. For further details see
% section~\ref{sec:newfonts} and \cite{bk:GMS94}.
%
% \section{Introduction}
%
% For our purposes here, ``theorems'' are labelled enunciations,
% often set off from the main text by extra space and a font change.
% Theorems, corollaries, conjectures, definitions, and remarks
% are all instances of ``theorems''.  The ``header'' of these structures
% is composed of a label (such as \textsc{Theorem} or \textsc{Remark})
% and a number which serializes an item in the sequence of items
% with the same label.
%
% Shortly after the introduction of \LaTeX{} at the Fachbereich
% Mathematik in Mainz, the desire to manipulate the layout of
% ``theorems'' arose.  In Mainz, the following two conventions came
% into general use:
% \begin{enumerate}
%   \item The number of the theorem is shown in the margin.
%   \item There is a line break at
%         the end of the theorem header.
% \end{enumerate}
%
% Additionally, some journals require different formats which depend on
% the ``sort of theorem'': e.g.\ often remarks and definitions are set
% in |\upshape|, while |\itshape| is employed for main theorems.
%
% Confronted with these requirements, a theorem environment was
% developed in Mainz which allows separate determination of the
% layout of the ``theorems sets'', comparable to |\pagestyle|.
%
%
% \section{The user interface}
% \subsection{Defining new theorem sets}
%
% \DescribeMacro\newtheorem
% As in the original \LaTeX{} version, the command |\newtheorem|
% defines a new ``theorem set'' or ``theorem-like structure''.
% Two required arguments name the new environment and give the text
% to be typeset with each instance of the new ``set'', while
% an optional argument determines how
% the ``set'' is enumerated:
% \begin{description}
%    \item[\ttfamily \bslash newtheorem\{foo\}\{bar\}]
%       The theorem set {\envfont foo} (whose name is \texttt{bar})
%       uses its own counter.
%    \item[\ttfamily \bslash newtheorem\{foo2\}{[foo]}\{bar2\}]
%       The theorem set {\envfont foo2} (printed name \texttt{bar2})
%       uses the same counter as the theorem set \texttt{foo}.
%    \item[\ttfamily \bslash newtheorem\{foo3\}\{bar3\}{[section]}]
%       The theorem set {\envfont foo3} (printed name \texttt{bar3}) is
%       enumerated within the counter \texttt{section}, i.e.\ with every
%       new |\section| the enumeration begins again with 1, and
%       the enumeration is composed from the section-number and the
%       theorem counter itself.
% \end{description}
%
% \DescribeMacro\theoremstyle
% Additionally, the command |\theoremstyle| can define the layout
% of various, or all, theorem sets.  It should be noted that any
% theorem set defined by |\newtheorem| is typeset in the
% |\theoremstyle| that is current at the time of the definition.
% Thus, the following
% \begin{verbatim}
%    \theoremstyle{break}        \newtheorem{Cor}{Corollary}
%    \theoremstyle{plain}        \newtheorem{Exa}{Example}[section]
%\end{verbatim}                             ^^A No blank at line start
% leads to the result that the set {\envfont Cor} is formatted in the
% style \texttt{break}, while the set {\envfont Exa} and all the
% following ones are formatted in the style \texttt{plain}, unless
% another |\theoremstyle| follows.  Since the definitions installed by
% |\newtheorem| are global, one also can limit |\theoremstyle| locally
% by grouping braces.
%
% \DescribeMacro\theorembodyfont
% The choice of the font for the theorem body is completely
% independent of the chosen |\theoremstyle|; this has proven to
% be very advantageous.  For example,
% \begin{verbatim}
%   {\theorembodyfont{\upshape}          \newtheorem{Rem}{Remark}}
%\end{verbatim}
% defines a theorem set {\envfont Rem}, which will be set in
% |\upshape| in the current layout (which in our example is
% \texttt{plain}).  As with |\theoremstyle|, the |\theorembodyfont|
% chosen is that current at the time of |\newtheorem|.  If
% |\theorembodyfont| is not specified or one defines
% |\theorembodyfont{}|, then the font used will be that defined by the
% |\theoremstyle|.
%
% \DescribeMacro\theoremheaderfont It is also possible to customize
% the font used for the theorem headers.  This is, however, a global
% declaration, and therefore there should be at most one
% |\theoremheaderfont| declaration in the preamble.\footnote{If it is
% actually necessary to have different header fonts, one has to define
% new theorem styles (substituting the desired font) or specify the
% information directly in the \texttt{\bslash newtheorem} declaration
% (the unclean variant).}
%
% \DescribeMacro\theorempreskipamount
% \DescribeMacro\theorempostskipamount Two additional parameters
% affect the vertical space around the theorem environments:
% |\theorempreskipamount| and |\theorempostskipamount| define,
% respectively, the spacing before and after such an environment.
% These parameters apply for all theorem sets and can be manipulated
% with the ordinary length macros.  They are rubber lengths
% (`\textsf{skips}'), and therefore can contain \texttt{plus} and
% \texttt{minus} parts.
%
% Since the definition of theorem sets should---most sensibly---be
% placed in the preamble, we only allow installation there.
% It is therefore possible to release the memory used here after
% |\begin{document}|, in order to make room for other
% applications.
%
%
% \subsection{Existing theorem styles}
%
% The following theorem styles exist to date:
% \begin{deflist}{marginbreak}
%    \item[plain]
%       This theorem style emulates the original \LaTeX{} definition,
%       except that additionally the parameters
%       |\theorem...skipamount| are used.
%    \item[break]
%       In this style, the theorem header is followed by a line
%       break.
%    \item[marginbreak]
%       The theorem number is set in the margin, and there is a line
%       break as in \texttt{break}.
%    \item[changebreak]
%       Like \texttt{break}, but with header number and text
%       interchanged.
%    \item[change]
%       Header number and text are interchanged, without a line break.
%    \item[margin]
%       The number is set in the left margin, without a line break.
% \end{deflist}
% All styles (except \texttt{plain}) select |\slshape| as the default
% |\theorembodyfont|.
%
%
% \subsection{Examples}
%
% Given the above theorem sets {\envfont Cor}, {\envfont Exa}
% and {\envfont Rem},
% suppose that the preamble also contains the declarations:
% \begin{verbatim}
%    \theoremstyle{marginbreak}   \newtheorem{Lem}[Cor]{Lemma}
%    \theoremstyle{change}
%    \theorembodyfont{\itshape}        \newtheorem{Def}[Cor]{Definition}
%
%    \theoremheaderfont{\scshape}
%\end{verbatim}
% Then the following are some typical examples of the typeset output
% resulting from their use.
%
% \begin{Cor}
%    This is a sentence typeset in the theorem environment {\envfont
%    Cor}.
% \end{Cor}
%
% \begin{Exa}
%    This is a sentence typeset in the theorem environment {\envfont
%    Exa}.
% \end{Exa}
%
% \begin{Rem}
%    This is a sentence typeset in the theorem environment {\envfont
%    Rem}.
% \end{Rem}
%
% \begin{Lem}[Ben User]
%    This is a sentence typeset in the theorem environment {\envfont
%    Lem}.
% \end{Lem}
%
% \begin{Def}[Very impressive Definition]
%    This is a sentence typeset in the theorem environment {\envfont
%    Def}.
% \end{Def}
%
% The last two examples show the effect of the optional argument to
% a theorem environment (it is the text typeset in parentheses).
%
%
% \section{Special Considerations}
% \label{sec:newfonts}
%
% Theoremheader and body are implemented as a unit. This means that
% the |\theoremheaderfont|  will inherit characteristics of the
% |\theorembodyfont| in \LaTeXe. Thus, if for example
% |\theorembodyfont| is |\itshape| and |\theoremheaderfont|
% is |\bfseries| the font selected for the header will have the
% characteristics `bold extended italic'. If this is not desired one
% should set the |\theoremheaderfont| to something like
% \begin{verbatim}
%      \theoremheaderfont{\normalfont\bfseries}
%\end{verbatim}
% i.e.\ supplying all necessary font information explicitly.
%
%
%
% \section{Acknowledgements}
%
% The publication of this set of macros was only possible with the
% help of Christina Busse (translating the manuscript into English),
% Joachim Pense (playing the r\^ole of typist), Chris Rowley
% (looking everything over) and many others providing useful
% suggestions.
%
% \MaybeStop{%
%    \begin{thebibliography}{9}
%       \bibitem{bk:GMS94} \textsc{M.~Goossens}, \textsc{F.~Mittelbach}
%         and \textsc{A.~Samarin}.
%         \newblock The \LaTeX{} Companion.
%         \newblock
%         Addison-Wesley, Reading, Massachusetts, 1994.
%       \bibitem{latex:source} \textsc{Lamport, Leslie.} \newblock
%                               \texttt{latex.tex}, version 2.09, date
%                               Feb. 1990.
%    \end{thebibliography}
%    \PrintIndex
%    \PrintChanges}
%
%
% \section{The documentation driver file}
%
% The next bit of code contains the documentation driver file for
% \TeX{}, i.e., the file that will produce the documentation you are
% currently reading. It will be extracted from this file by the
% \texttt{docstrip} program. Since it is the first code in the file
% one can alternatively process this file directly with \LaTeXe{} to
% obtain the documentation.
%
%    \begin{macrocode}
%<*driver>
\documentclass{ltxdoc}

\usepackage{theorem}

%<+driver>% The next few lines define theorem sets which are used
%<+driver>% in the example section of the documentation.

\theoremstyle{break}       \newtheorem{Cor}{Corollary}
\theoremstyle{plain}       \newtheorem{Exa}{Example}[section]
{\theorembodyfont{\upshape}\newtheorem{Rem}{Remark}}
\theoremstyle{marginbreak} \newtheorem{Lem}[Cor]{Lemma}
\theoremstyle{change}
\theorembodyfont{\itshape} \newtheorem{Def}[Cor]{Definition}

\theoremheaderfont{\scshape}

\RecordChanges

\begin{document}
   \DocInput{theorem.dtx}
\end{document}
%</driver>
%    \end{macrocode}
%
% \section{Definition of the Macros}
%
% If the file has been loaded before, we abort immediately. If not the
% package announces itself (this is actually done at the very top of
% the file---the way it is done isn't good style so don't copy it).
% \changes{v2.0e}{89/07/19}{Spaces removed from `typeout'.}
% \changes{v2.2b}{95/11/19}{Announce moved to top of file}
%    \begin{macrocode}
%<*package>
%\@ifundefined{theorem@style}{}{\endinput}
%\def\FMithmInfo{1995/11/19 v2.2b Theorem extension package (FMi)}
%\ProvidesPackage{theorem}[\FMithmInfo]
%    \end{macrocode}
%
%
% \subsection{Definition of theorem styles and fonts}
%
% All the definitions in this file are done globally to allow
% inputting this file inside a group.
%
% \begin{macro}{\theoremstyle}
%    Before a theorem style can be installed, the chosen style must be
%    known. For that reason, we must test to see that
%    |\th@|\meta{style} is known or, more precisely, that it is
%    different from |\relax|. If the style is not known then
%    |\th@plain| is used.
%    \begin{macrocode}
\gdef\theoremstyle#1{%
   \@ifundefined{th@#1}{\@warning
          {Unknown theoremstyle `#1'. Using `plain'}%
          \theorem@style{plain}}%
%    \end{macrocode}
%    We save the theorem style to be used in the token register
%    |\theorem@style|.
%    \begin{macrocode}
      {\theorem@style{#1}}%
%    \end{macrocode}
%    Now we ``evaluate'' the theorem style: this means, we call the
%    macro |\th@|\meta{style} which will activate the relevant
%    definitions which are contained in a separate file.  This is done
%    in a group to suppress changes to the current font.  This could
%    otherwise pose problems together with the new font selection
%    scheme\footnote{When I printed the original article using the new
%    font selection scheme I ended with a document with slanted
%    typefaces (text headings and all) simply because one of the
%    theorem styles used \texttt{\bslash sl} at toplevel.} if the
%    |\th@|\meta{style} is evaluated a second time.
% \changes{v2.1a}{90/02/15}{Group added around `th@..}
% \changes{v2.1a}{90/02/15}{Reseting of `..@begintheorem removed.}
%    \begin{macrocode}
      \begingroup
        \csname th@\the\theorem@style \endcsname
      \endgroup}
%    \end{macrocode}
% \end{macro}
%
%
% \begin{macro}{\@begintheorem}
% \changes{v2.1a}{90/02/15}{Relaxed at toplevel.}
% \begin{macro}{\@opargbegintheorem}
% \changes{v2.1a}{90/02/15}{Relaxed at toplevel.}
%    We reset |\@begintheorem| and |\@opargbegintheorem| to
%    |\relax| since these commands are no longer necessary at
%    toplevel. This will save a few tokens.
%    \begin{macrocode}
\global\let\@begintheorem\relax
\global\let\@opargbegintheorem\relax
%    \end{macrocode}
% \end{macro}
% \end{macro}
%
% \begin{macro}{\theorem@style}
%    Obviously the \textsf{token} register used above has to be
%    allocated. To assure the utmost compatibility with the
%    original \LaTeX{} definition, we set the default theorem style to
%    \texttt{plain}, which implements the usual \LaTeX{} convention.
% \changes{v2.0f}{89/07/20}{Unnecessary `global' removed.}
%    \begin{macrocode}
\newtoks\theorem@style
\global\theorem@style{plain}
%    \end{macrocode}
% \end{macro}
%
% \begin{macro}{\theorembodyfont}
% \begin{macro}{\theorem@bodyfont}
%    If the |\theorembodyfont| is set by the user then it should not
%    interact with the default font set in the theorem style. When the
%    new font selection is in force this may happen if, for example,
%    the default is |\itshape| and the new |\theorembodyfont| is
%    |\sffamily|. So we add a |\reset@font| command in front of the
%    user definition.
% \changes{v2.0f}{89/07/20}{Unnecessary `global' removed.}
% \changes{v2.0g}{89/09/19}{`theoremfont to `theorembodyfont.}
% \changes{v2.1c}{92/06/22}{`reset@font added and logic changed.}
% \changes{v2.1d}{92/10/28}{Reset to empty if arg is empty}
%    \begin{macrocode}
\gdef\theorembodyfont#1{%
%    \end{macrocode}
%    We check if the argument supplied is empty and if so put nothing
%    into the |\theorem@bodyfont| token register to allow for
%    |\theorembodyfont{}| as a mean of using the default of the
%    current |\theoremstyle|.
%    \begin{macrocode}
   \def\@tempa{#1}%
   \ifx\@tempa\@empty
    \theorem@bodyfont{}%
   \else
    \theorem@bodyfont{\reset@font#1}%
   \fi
   }
\newtoks\theorem@bodyfont
\global\theorem@bodyfont{}
%    \end{macrocode}
% \end{macro}
% \end{macro}
%
%
% \begin{macro}{\theoremheaderfont}
% \changes{v2.0g}{89/09/19}{Macro added.}
%    The font for the theorem headers is handled differently because
%    this definition applies to all theorem styles.
%    \begin{macrocode}
\gdef\theoremheaderfont#1{\gdef\theorem@headerfont{#1}%
%    \end{macrocode}
%    After using the macro once it is redefined to produce an error
%    message.
%    \begin{macrocode}
       \gdef\theoremheaderfont##1{%
        \typeout{\string\theoremheaderfont\space should be used
                 only once.}}}
%    \end{macrocode}
% \end{macro}
%
% \begin{macro}{\theorem@headerfont}
% \changes{v2.0g}{89/09/19}{Macro added.}
%    To set the |\theorem@headerfont| default we first test if
%    the new fontselection scheme is in force.
% \changes{v2.1a}{90/02/15}{Support for new font selection.}
%    \begin{macrocode}
\ifx\upshape\undefined
%    \end{macrocode}
%    If not we define it to expand into |\bfseries|. We don't use
%    |\let| just in case a following style option redefines this
%    macro.
%    \begin{macrocode}
\gdef\theorem@headerfont{\bfseries}
%    \end{macrocode}
%    Otherwise we reset the current shape before calling |\bfseries|.
%    \begin{macrocode}
\else \gdef\theorem@headerfont{\normalfont\bfseries}\fi
%    \end{macrocode}
% \end{macro}
%
% \begin{macro}{\th@plain}
% \begin{macro}{\th@break}
% \begin{macro}{\th@marginbreak}
% \begin{macro}{\th@changebreak}
% \begin{macro}{\th@change}
% \begin{macro}{\th@margin}
%    The different styles are defined in macros such as
%    |\th@plain|.  Since memory space is precious in
%    ``non-Big-versions'', we have to avoid offering too many unused
%    definitions.  Therefore we define these styles in separate files
%    that can be loaded on demand.  Thus the commands themselves only
%    load these files. We use |\@input@|, a \LaTeXe{} internal command
%    that ensures that the file will be listed with |\listfiles|.
% \changes{v2.2a}{94/02/02}{Use `@input@ to load theorem layout files}
%    \begin{macrocode}
\gdef\th@plain{\@input@{thp.sty}}
\gdef\th@break{\@input@{thb.sty}}
\gdef\th@marginbreak{\@input@{thmb.sty}}
\gdef\th@changebreak{\@input@{thcb.sty}}
\gdef\th@change{\@input@{thc.sty}}
\gdef\th@margin{\@input@{thm.sty}}
%    \end{macrocode}
%    This list will be expanded when new styles become
%    available.  For testing,  just append new theorem substyles
%    as document options.
% \end{macro}
% \end{macro}
% \end{macro}
% \end{macro}
% \end{macro}
% \end{macro}
%
%
%
% \subsection{Definition of a new theorem set}
%
% As already pointed out, a new theorem environment can be defined in
% three different ways:
% \begin{verbatim}
%    \newtheorem{Lem}{Lemma}
%    \newtheorem{Lem}{Lemma}[section]
%    \newtheorem{Lem}[Theorem]{Lemma}
%\end{verbatim}
% The function of the macro |\newtheorem| is to recognize these cases
% and then to branch into one of the three macros |\@ynthm|, |\@xnthm|
% or |\@othm|.  This mechanism is adopted unchanged from
% \cite{latex:source}; the essential point here is that, for example,
% in the second case, the arguments \texttt{Lem}, \texttt{Lemma} and
% \texttt{section} are passed over to the macro |\@xnthm|.
%
% We inspect this case first because the others present fewer problems,
% and thus are easily derived from this one.
%
% \begin{macro}{\@xnthm}
%    For our example arguments, the macro |\@xnthm| must fulfill
%    the following:
%    \begin{itemize}
%       \item
%          Define a new \LaTeX-counter `Lem'
%       \item
%          reset this counter within a |\section|
%       \item
%          define the macro |\theLem|
%       \item
%          define the environment macros |\Lem| and
%          |\endLem| using the current
%          |\theoremstyle| and |\theorem@bodyfont|.
%    \end{itemize}
%    Obviously, all this should happen only if the first argument
%    of |\@xnthm| (i.e.\ \texttt{Lem} in our example) is chosen so
%    as not to conflict with any previously defined commands or
%    environments.  This test is performed by the \LaTeX{} macro
%    |\@ifdefinable|.
%    \begin{macrocode}
\gdef\@xnthm#1#2[#3]{\expandafter\@ifdefinable\csname #1\endcsname
%    \end{macrocode}
%    Therefore, the first argument of |\@ifdefinable| is the
%    expansion (in the example, |\Lem|) of
%    |\csname||#1\endcsname|.  The second argument is
%    executed only if the test has been completed successfully.
%    \begin{macrocode}
   {%
%    \end{macrocode}
%    Now we define the new counter. The names of the \LaTeX{}
%    macros employed should speak for themselves:
% \changes{v2.2b}{1995/10/07}{Use \cs{@newctr}}
% \changes{v2.2c}{1995/11/23}{Use \cs{@newctr} properly.}
%    \begin{macrocode}
    \@definecounter{#1}\@newctr{#1}[#3]%
%    \end{macrocode}
%    Using |\@newctr| will give a proper error message if the counter
%    in |#3| is not defined.
%    In defining `|\theLem|' we must generate the desired macro
%    name by use of |\expandafter| and |\csname|.
%    \begin{macrocode}
    \expandafter\xdef\csname the#1\endcsname
%    \end{macrocode}
%    An |\xdef| is used in order to make the definition global,
%    and to ensure that it contains the replacement texts of
%    |\@thmcountersep| and |\@thmcounter|.\footnote{These
%    two macros can be defined by the document style.  Their default
%    values produce a `\texttt{.}' as separation and an arabic
%    representation of the number.} However, not everything should be
%    expanded. For example, it saves space to use |\thesection|
%    instead of its---at times---lengthy expansion.
%    \begin{macrocode}
      {\expandafter \noexpand \csname the#3\endcsname
       \@thmcountersep \@thmcounter{#1}}%
%    \end{macrocode}
%    Thus with the defaults of \LaTeX{}, |\theLem| would be
%    replaced by the command sequence |\thesection.\arabic{Lem}|.
%
%    We will now look at the definition of the macro which is executed
%    at the beginning of the actual environment (in our example this
%    macro is |\Lem|). It should be noted that we use an
%    ``|\expandafter| trick'' to expand only certain parts of the
%    replacement text at the time of the definition.
%    \begin{macrocode}
    \def\@tempa{\global\@namedef{#1}}%
    \expandafter \@tempa \expandafter{%
%    \end{macrocode}
%    First, the macro that contains the current definitions of
%    |\@begintheorem| and |\@opargtheorem| should be called
%    up.  The name of this macro---as is already known---has the form
%    |\th@|\meta{theorem\ style}; therefore, it must be called by
%    \begin{macrocode}
      \csname th@\the \theorem@style
%    \end{macrocode}
%    In addition the default theorem font should be changeable, i.e.\
%    we have to insert the contents of |\theorem@bodyfont|. For
%    that reason, we expand even further, beyond |\endcsname|,
%    and thus insert the contents of the \textsf{token} register
%    |\theorem@bodyfont| in the replacement text.
%    \begin{macrocode}
            \expandafter \endcsname \the \theorem@bodyfont
%    \end{macrocode}
%    Now it is time to call the macro |\@thm| which takes over
%    the further processing.  It has two arguments: the current
%    counter name (in our example, \texttt{Lem}), and the text of the
%    label (in our example, \texttt{Lemma}).
%    \begin{macrocode}
     \@thm{#1}{#2}}%
%    \end{macrocode}
%    With this, the `sub-definition' is complete.  The macro
%    |\@endtheorem| ends a theorem environment and is, so far, nothing
%    but an |\endtrivlist|.  (Hence it is defined globally, and not
%    within the theorem styles.\footnote{This has to be changed as
%    soon as theorem styles that change \texttt{\bslash @endtheorem}
%    exist.  In such a case, all existing styles must be changed as
%    well since they will have to reset the macro.}) Therefore, we can
%    set it equivalent to the macro that ends the theorem set (in our
%    example, |\endLem|).  However, if some day theorem styles exist
%    that do change |\@endtheorem|, we would have to use the
%    commented-out line instead.
%    \begin{macrocode}
    \global \expandafter \let \csname end#1\endcsname \@endtheorem
%   \global\@namedef{end#1}{\@endtheorem}%
%    \end{macrocode}
%    With these commands all the required definitions are employed,
%    unless the test |\@ifdefinable| has failed.  Therefore, we
%    end the second argument of this macro and with it the definition
%    of |\@xnthm|.
%    \begin{macrocode}
   }}
%    \end{macrocode}
% \end{macro}
%
% \begin{macro}{\@ynthm}
%    The definition of |\@ynthm| is completely analogous. In
%    this case the new counter that is defined is not reset within
%    another counter; thus the definition of |\the...|
%    is simplified:
%    \begin{macrocode}
\gdef\@ynthm#1#2{\expandafter\@ifdefinable\csname #1\endcsname
   {\@definecounter{#1}%
    \expandafter\xdef\csname the#1\endcsname{\@thmcounter{#1}}%
%    \end{macrocode}
%    The rest of the definition corresponds literally to that of
%    |\@xnthm|:
%    \begin{macrocode}
    \def\@tempa{\global\@namedef{#1}}\expandafter \@tempa
     \expandafter{\csname th@\the \theorem@style \expandafter
     \endcsname \the\theorem@bodyfont \@thm{#1}{#2}}%
    \global \expandafter \let \csname end#1\endcsname \@endtheorem}}
%    \end{macrocode}
% \end{macro}
%
% \begin{macro}{\@othm}
%    The definition of |\@othm| does not contain anything new.
%    \begin{macrocode}
\gdef\@othm#1[#2]#3{%
%    \end{macrocode}
% \changes{v2.2b}{95/10/07}{Build in check for counter defined}
%    We do not define a new counter but instead use one that has
%    already been defined.  Thus the only definition we need is that
%    of this pseudo-counter (i.e.\ |\the|\meta{env.\ name}).
%    First we check if |#2| corresponds to a known counter name.
%    \begin{macrocode}
  \expandafter\ifx\csname c@#2\endcsname\relax
   \@nocounterr{#2}%
  \else
   \expandafter\@ifdefinable\csname #1\endcsname
   {\expandafter \xdef \csname the#1\endcsname
     {\expandafter \noexpand \csname the#2\endcsname}%
%    \end{macrocode}
%    All other parts of the definition can be adopted from
%    |\@xnthm|. We have to remember, though, that in this case
%    the name of the current counter and the theorem label have moved
%    to the second and third arguments.
%    \begin{macrocode}
    \def\@tempa{\global\@namedef{#1}}\expandafter \@tempa
     \expandafter{\csname th@\the \theorem@style \expandafter
     \endcsname \the\theorem@bodyfont \@thm{#2}{#3}}%
    \global \expandafter \let \csname end#1\endcsname \@endtheorem}%
  \fi}
%    \end{macrocode}
% \end{macro}
%
% \subsection{Macros that are employed in a {\envfont theorem}
%    environment}
%
% \begin{macro}{\@thm}
%    The macro |\@thm| has to increase the current counter.  Then,
%    depending on whether the environment has (or does not have) an
%    optional argument, it has to branch into either |\@begintheorem|
%    or |\@opargtheorem|.
%    \begin{macrocode}
\gdef\@thm#1#2{\refstepcounter{#1}%
%    \end{macrocode}
%    Now we start a {\envfont trivlist} environment, and give
%    |\@topsep| and |\@topsepadd| the values of the \textsf{skip}
%    registers |\theorempreskipamount| and |\theorempostskipamount|.
%    The value in |\@topsep| is the vertical space that is inserted by
%    the first (and only) |\item| in our |\trivlist| whilst
%    |\@topsepadd| is inserted by |\@endparenv| at the end of that
%    {\envfont trivlist} environment.  By using these registers, we
%    obtain the desired space around a {\envfont theorem} environment.
%    \begin{macrocode}
   \trivlist
   \@topsep \theorempreskipamount               % used by first \item
   \@topsepadd \theorempostskipamount           % used by \@endparenv
%    \end{macrocode}
%    Now we have to test whether an optional argument has been given.
% \changes{v2.0d}{89/07/18}{Forgotten percent added.}
%    \begin{macrocode}
   \@ifnextchar [%
%    \end{macrocode}
%    If there is an optional argument, we will call |\@ythm|, and
%    move the arguments read back into the input stream.
%    \begin{macrocode}
   {\@ythm{#1}{#2}}%
%    \end{macrocode}
%    If not, we call |\@begintheorem|. Its first argument is the
%    name of the theorem set (hence the second argument of
%    |\@thm|). Its second argument is the macro that produces the
%    current number.
%    \begin{macrocode}
   {\@begintheorem{#2}{\csname the#1\endcsname}\ignorespaces}}
%    \end{macrocode}
% \end{macro}
%
% \begin{macro}{\@xthm}
% \begin{macro}{\@ythm}
%    Both these macros were originally called by |\@thm|.  We do
%    not need |\@xthm| anymore, hence we reset it to
%    |\relax|. The definition of |\@ythm| has not changed at
%    all from its definition in \LaTeX. In order to make the macros
%    easier to understand, we will
%    nevertheless present it (commented out).
%    \begin{macrocode}
\global\let\@xthm\relax
% \def\@ythm#1#2[#3]{\@opargbegintheorem{#2}{\csname
%           the#1\endcsname}{#3}\ignorespaces}
%</package>
%    \end{macrocode}
%    The primitive |\ignorespaces| in |\@ythm| and
%    |\@thm| is needed to remove the spaces between the
%    |\begin{...}| and the actual text.
% \end{macro}
% \end{macro}
%
%
% \subsection{Definition of the theorem substyles}
%
% As already pointed out, the theorem substyles, defined below, are
% only loaded when necessary.  Note that all these substyles, except
% \texttt{plain}, have |\slshape| as the default body font.
%
% \subsubsection{The \texttt{plain} style}
%
% \label{dok:th@plain} As the following macros use |@|, we have
% to locally set the |\catcode| of this symbol to ``letter''.
% This happens within a group, so that we do not have to worry about
% which |\catcode| that symbol had before.
%    \begin{macrocode}
%<*thp>
\begingroup \makeatletter
%    \end{macrocode}
% Since we are now within a group, we must make all definitions
% globally.  First we make sure that \texttt{theorem.sty} is loaded.
% This will allow us to use this file as a document style option
% without having to call \texttt{theorem} itself as an option.  At the
% same time, we assure that at least version 2 is loaded, since
% |\theorem@style| was not defined in earlier versions.
%    \begin{macrocode}
\@ifundefined{theorem@style}{% \iffalse meta-comment
%
% Copyright (C) 1993-2024
%
% The LaTeX Project and any individual authors listed elsewhere
% in this file.
%
% This file is part of the Standard LaTeX `Tools Bundle'.
% -------------------------------------------------------
%
% It may be distributed and/or modified under the
% conditions of the LaTeX Project Public License, either version 1.3c
% of this license or (at your option) any later version.
% The latest version of this license is in
%    https://www.latex-project.org/lppl.txt
% and version 1.3c or later is part of all distributions of LaTeX
% version 2005/12/01 or later.
%
% The list of all files belonging to the LaTeX `Tools Bundle' is
% given in the file `manifest.txt'.
%
% \fi
%
%%
%
%
\def\FMithmInfo{2023/07/05 v2.2c Theorem extension package (FMi)}
%
% \ProvidesFile{theorem.dtx}[\FMithmInfo]
%\iffalse   % this is a METACOMMENT !
%%
%% Package `theorem' to use with LaTeX2e
%% Copyright (C) 1989-1995 Frank Mittelbach, all rights reserved.
%<+package>\@ifundefined{theorem@style}{}{\endinput}
%<+package>\NeedsTeXFormat{LaTeX2e}[1995/06/01]
%<+package>\ProvidesPackage{theorem}
%<+package>         [\FMithmInfo]
%
% \fi
%
% \changes{v1.0}{87/05/07}{First draft.}
% \changes{v1.0a}{87/05/07}{`thistheoremstyle' added.}
% \changes{v1.1a}{87/07/29}{New concept: redefine the internal
%                          \protect\LaTeX{} commands `@xnthm' etc.}
% \changes{v1.1b}{87/07/31}{`theoremfont' used to specify font.}
% \changes{v1.2a}{88/01/08}{`@endtheorem' now globally defined.}
% \changes{v1.2b}{88/01/11}{`newtheorem' only allowed in the preamble.}
% \changes{v1.3a}{88/02/14}{`theoremfont' now a toks register.}
%
% \changes{v2.0a}{88/11/16}{New implementation and new documentation.}
% \changes{v2.0a}{88/11/16}{No layout changes but internal modifications
%                           old substyles won't work.}
% \changes{v2.0b}{89/01/30}{`global' used for all definitions.}
% \changes{v2.0c}{89/06/15}{Documentation translated to English.
%                            Thanks to LLL and Nina}
% \changes{v2.1a}{90/02/15}{Support for new font selection.}
% \changes{v2.2a}{94/02/03}{Cleanup for LaTeX2e}
%
% \DoNotIndex{\@endparenv,\@ifdefinable,\@ifnextchar,\@ifundefined}
% \DoNotIndex{\@namedef,\@nameuse,\@spaces,\@tempa,\@warning}
% \DoNotIndex{\ }
% \DoNotIndex{\begingroup,\bfseries}
% \DoNotIndex{\csname}
% \DoNotIndex{\def,\do,\docdate}
% \DoNotIndex{\endcsname,\endgroup,\endinput,\endtrivlist,\expandafter}
% \DoNotIndex{\filedate,\fileversion}
% \DoNotIndex{\gdef,\global}
% \DoNotIndex{\hbox,\hskip}
% \DoNotIndex{\ignorespaces,\input,\itshape,\item}
% \DoNotIndex{\labelsep,\let,\llap}
% \DoNotIndex{\makeatletter}
% \DoNotIndex{\noexpand}
% \DoNotIndex{\refstepcounter,\relax,\rlap}
% \DoNotIndex{\setlength,\slshape,\space,\string,\strut}
% \DoNotIndex{\the,\trivlist,\typeout}
% \DoNotIndex{\vbox}
% \DoNotIndex{\xdef}
%
% \def\deflabel#1{\ttfamily #1\hfill}
% \def\deflist#1{\begin{list}{}{\settowidth\labelwidth{\ttfamily #1}%
%                               \setlength\leftmargin\labelwidth
%                               \addtolength\leftmargin\labelsep
%                               \let\makelabel\deflabel}}
% \def\enddeflist{\end{list}}
% \def\envfont{\normalfont\ttfamily}
%
% \MakeShortVerb{\|}
%
% \newcommand{\TUB}{{\sl TUGboat\/}}
%
% \GetFileInfo{theorem.dtx}
% \title{An Extension of the
%        \LaTeX{} theorem environment\thanks{This file
%        has version number \fileversion, last
%        revised \filedate.}}
% \author{Frank Mittelbach}
%
% \MaintainedByLaTeXTeam{tools}
% \maketitle
%
% \begin{abstract}
%    The macros described in this paper yield an extension of the
%    \LaTeX{} theorem mechanism. It is designed to satisfy the
%    different requirements of various journals. Thus, the
%    layout of the ``theorems'' can be manipulated by determining a
%    ``style''. This article describes not only the use, but
%    also the definition, of the necessary macros.
% \end{abstract}
%
% \section*{Preface to version 2.2}
%
% For \LaTeXe{} this package did not need any fundamental changes. I
% only modified the messages generated so that theorem layout styles
% will show up with the |\listfiles| command and cleaned the section
% on the New Font Selection Scheme since this is now included in
% \LaTeX.
%
% \section*{Preface to version 2.1}
%
% This version is identical to 2.0g described in \TUB\ 10\#3 except
% for some internal defaults which are now set depending on the used
% font selection scheme.
%
% This was done to avoid unpleasant surprises if the new font selection
% scheme is in force. For further details see
% section~\ref{sec:newfonts} and \cite{bk:GMS94}.
%
% \section{Introduction}
%
% For our purposes here, ``theorems'' are labelled enunciations,
% often set off from the main text by extra space and a font change.
% Theorems, corollaries, conjectures, definitions, and remarks
% are all instances of ``theorems''.  The ``header'' of these structures
% is composed of a label (such as \textsc{Theorem} or \textsc{Remark})
% and a number which serializes an item in the sequence of items
% with the same label.
%
% Shortly after the introduction of \LaTeX{} at the Fachbereich
% Mathematik in Mainz, the desire to manipulate the layout of
% ``theorems'' arose.  In Mainz, the following two conventions came
% into general use:
% \begin{enumerate}
%   \item The number of the theorem is shown in the margin.
%   \item There is a line break at
%         the end of the theorem header.
% \end{enumerate}
%
% Additionally, some journals require different formats which depend on
% the ``sort of theorem'': e.g.\ often remarks and definitions are set
% in |\upshape|, while |\itshape| is employed for main theorems.
%
% Confronted with these requirements, a theorem environment was
% developed in Mainz which allows separate determination of the
% layout of the ``theorems sets'', comparable to |\pagestyle|.
%
%
% \section{The user interface}
% \subsection{Defining new theorem sets}
%
% \DescribeMacro\newtheorem
% As in the original \LaTeX{} version, the command |\newtheorem|
% defines a new ``theorem set'' or ``theorem-like structure''.
% Two required arguments name the new environment and give the text
% to be typeset with each instance of the new ``set'', while
% an optional argument determines how
% the ``set'' is enumerated:
% \begin{description}
%    \item[\ttfamily \bslash newtheorem\{foo\}\{bar\}]
%       The theorem set {\envfont foo} (whose name is \texttt{bar})
%       uses its own counter.
%    \item[\ttfamily \bslash newtheorem\{foo2\}{[foo]}\{bar2\}]
%       The theorem set {\envfont foo2} (printed name \texttt{bar2})
%       uses the same counter as the theorem set \texttt{foo}.
%    \item[\ttfamily \bslash newtheorem\{foo3\}\{bar3\}{[section]}]
%       The theorem set {\envfont foo3} (printed name \texttt{bar3}) is
%       enumerated within the counter \texttt{section}, i.e.\ with every
%       new |\section| the enumeration begins again with 1, and
%       the enumeration is composed from the section-number and the
%       theorem counter itself.
% \end{description}
%
% \DescribeMacro\theoremstyle
% Additionally, the command |\theoremstyle| can define the layout
% of various, or all, theorem sets.  It should be noted that any
% theorem set defined by |\newtheorem| is typeset in the
% |\theoremstyle| that is current at the time of the definition.
% Thus, the following
% \begin{verbatim}
%    \theoremstyle{break}        \newtheorem{Cor}{Corollary}
%    \theoremstyle{plain}        \newtheorem{Exa}{Example}[section]
%\end{verbatim}                             ^^A No blank at line start
% leads to the result that the set {\envfont Cor} is formatted in the
% style \texttt{break}, while the set {\envfont Exa} and all the
% following ones are formatted in the style \texttt{plain}, unless
% another |\theoremstyle| follows.  Since the definitions installed by
% |\newtheorem| are global, one also can limit |\theoremstyle| locally
% by grouping braces.
%
% \DescribeMacro\theorembodyfont
% The choice of the font for the theorem body is completely
% independent of the chosen |\theoremstyle|; this has proven to
% be very advantageous.  For example,
% \begin{verbatim}
%   {\theorembodyfont{\upshape}          \newtheorem{Rem}{Remark}}
%\end{verbatim}
% defines a theorem set {\envfont Rem}, which will be set in
% |\upshape| in the current layout (which in our example is
% \texttt{plain}).  As with |\theoremstyle|, the |\theorembodyfont|
% chosen is that current at the time of |\newtheorem|.  If
% |\theorembodyfont| is not specified or one defines
% |\theorembodyfont{}|, then the font used will be that defined by the
% |\theoremstyle|.
%
% \DescribeMacro\theoremheaderfont It is also possible to customize
% the font used for the theorem headers.  This is, however, a global
% declaration, and therefore there should be at most one
% |\theoremheaderfont| declaration in the preamble.\footnote{If it is
% actually necessary to have different header fonts, one has to define
% new theorem styles (substituting the desired font) or specify the
% information directly in the \texttt{\bslash newtheorem} declaration
% (the unclean variant).}
%
% \DescribeMacro\theorempreskipamount
% \DescribeMacro\theorempostskipamount Two additional parameters
% affect the vertical space around the theorem environments:
% |\theorempreskipamount| and |\theorempostskipamount| define,
% respectively, the spacing before and after such an environment.
% These parameters apply for all theorem sets and can be manipulated
% with the ordinary length macros.  They are rubber lengths
% (`\textsf{skips}'), and therefore can contain \texttt{plus} and
% \texttt{minus} parts.
%
% Since the definition of theorem sets should---most sensibly---be
% placed in the preamble, we only allow installation there.
% It is therefore possible to release the memory used here after
% |\begin{document}|, in order to make room for other
% applications.
%
%
% \subsection{Existing theorem styles}
%
% The following theorem styles exist to date:
% \begin{deflist}{marginbreak}
%    \item[plain]
%       This theorem style emulates the original \LaTeX{} definition,
%       except that additionally the parameters
%       |\theorem...skipamount| are used.
%    \item[break]
%       In this style, the theorem header is followed by a line
%       break.
%    \item[marginbreak]
%       The theorem number is set in the margin, and there is a line
%       break as in \texttt{break}.
%    \item[changebreak]
%       Like \texttt{break}, but with header number and text
%       interchanged.
%    \item[change]
%       Header number and text are interchanged, without a line break.
%    \item[margin]
%       The number is set in the left margin, without a line break.
% \end{deflist}
% All styles (except \texttt{plain}) select |\slshape| as the default
% |\theorembodyfont|.
%
%
% \subsection{Examples}
%
% Given the above theorem sets {\envfont Cor}, {\envfont Exa}
% and {\envfont Rem},
% suppose that the preamble also contains the declarations:
% \begin{verbatim}
%    \theoremstyle{marginbreak}   \newtheorem{Lem}[Cor]{Lemma}
%    \theoremstyle{change}
%    \theorembodyfont{\itshape}        \newtheorem{Def}[Cor]{Definition}
%
%    \theoremheaderfont{\scshape}
%\end{verbatim}
% Then the following are some typical examples of the typeset output
% resulting from their use.
%
% \begin{Cor}
%    This is a sentence typeset in the theorem environment {\envfont
%    Cor}.
% \end{Cor}
%
% \begin{Exa}
%    This is a sentence typeset in the theorem environment {\envfont
%    Exa}.
% \end{Exa}
%
% \begin{Rem}
%    This is a sentence typeset in the theorem environment {\envfont
%    Rem}.
% \end{Rem}
%
% \begin{Lem}[Ben User]
%    This is a sentence typeset in the theorem environment {\envfont
%    Lem}.
% \end{Lem}
%
% \begin{Def}[Very impressive Definition]
%    This is a sentence typeset in the theorem environment {\envfont
%    Def}.
% \end{Def}
%
% The last two examples show the effect of the optional argument to
% a theorem environment (it is the text typeset in parentheses).
%
%
% \section{Special Considerations}
% \label{sec:newfonts}
%
% Theoremheader and body are implemented as a unit. This means that
% the |\theoremheaderfont|  will inherit characteristics of the
% |\theorembodyfont| in \LaTeXe. Thus, if for example
% |\theorembodyfont| is |\itshape| and |\theoremheaderfont|
% is |\bfseries| the font selected for the header will have the
% characteristics `bold extended italic'. If this is not desired one
% should set the |\theoremheaderfont| to something like
% \begin{verbatim}
%      \theoremheaderfont{\normalfont\bfseries}
%\end{verbatim}
% i.e.\ supplying all necessary font information explicitly.
%
%
%
% \section{Acknowledgements}
%
% The publication of this set of macros was only possible with the
% help of Christina Busse (translating the manuscript into English),
% Joachim Pense (playing the r\^ole of typist), Chris Rowley
% (looking everything over) and many others providing useful
% suggestions.
%
% \MaybeStop{%
%    \begin{thebibliography}{9}
%       \bibitem{bk:GMS94} \textsc{M.~Goossens}, \textsc{F.~Mittelbach}
%         and \textsc{A.~Samarin}.
%         \newblock The \LaTeX{} Companion.
%         \newblock
%         Addison-Wesley, Reading, Massachusetts, 1994.
%       \bibitem{latex:source} \textsc{Lamport, Leslie.} \newblock
%                               \texttt{latex.tex}, version 2.09, date
%                               Feb. 1990.
%    \end{thebibliography}
%    \PrintIndex
%    \PrintChanges}
%
%
% \section{The documentation driver file}
%
% The next bit of code contains the documentation driver file for
% \TeX{}, i.e., the file that will produce the documentation you are
% currently reading. It will be extracted from this file by the
% \texttt{docstrip} program. Since it is the first code in the file
% one can alternatively process this file directly with \LaTeXe{} to
% obtain the documentation.
%
%    \begin{macrocode}
%<*driver>
\documentclass{ltxdoc}

\usepackage{theorem}

%<+driver>% The next few lines define theorem sets which are used
%<+driver>% in the example section of the documentation.

\theoremstyle{break}       \newtheorem{Cor}{Corollary}
\theoremstyle{plain}       \newtheorem{Exa}{Example}[section]
{\theorembodyfont{\upshape}\newtheorem{Rem}{Remark}}
\theoremstyle{marginbreak} \newtheorem{Lem}[Cor]{Lemma}
\theoremstyle{change}
\theorembodyfont{\itshape} \newtheorem{Def}[Cor]{Definition}

\theoremheaderfont{\scshape}

\RecordChanges

\begin{document}
   \DocInput{theorem.dtx}
\end{document}
%</driver>
%    \end{macrocode}
%
% \section{Definition of the Macros}
%
% If the file has been loaded before, we abort immediately. If not the
% package announces itself (this is actually done at the very top of
% the file---the way it is done isn't good style so don't copy it).
% \changes{v2.0e}{89/07/19}{Spaces removed from `typeout'.}
% \changes{v2.2b}{95/11/19}{Announce moved to top of file}
%    \begin{macrocode}
%<*package>
%\@ifundefined{theorem@style}{}{\endinput}
%\def\FMithmInfo{1995/11/19 v2.2b Theorem extension package (FMi)}
%\ProvidesPackage{theorem}[\FMithmInfo]
%    \end{macrocode}
%
%
% \subsection{Definition of theorem styles and fonts}
%
% All the definitions in this file are done globally to allow
% inputting this file inside a group.
%
% \begin{macro}{\theoremstyle}
%    Before a theorem style can be installed, the chosen style must be
%    known. For that reason, we must test to see that
%    |\th@|\meta{style} is known or, more precisely, that it is
%    different from |\relax|. If the style is not known then
%    |\th@plain| is used.
%    \begin{macrocode}
\gdef\theoremstyle#1{%
   \@ifundefined{th@#1}{\@warning
          {Unknown theoremstyle `#1'. Using `plain'}%
          \theorem@style{plain}}%
%    \end{macrocode}
%    We save the theorem style to be used in the token register
%    |\theorem@style|.
%    \begin{macrocode}
      {\theorem@style{#1}}%
%    \end{macrocode}
%    Now we ``evaluate'' the theorem style: this means, we call the
%    macro |\th@|\meta{style} which will activate the relevant
%    definitions which are contained in a separate file.  This is done
%    in a group to suppress changes to the current font.  This could
%    otherwise pose problems together with the new font selection
%    scheme\footnote{When I printed the original article using the new
%    font selection scheme I ended with a document with slanted
%    typefaces (text headings and all) simply because one of the
%    theorem styles used \texttt{\bslash sl} at toplevel.} if the
%    |\th@|\meta{style} is evaluated a second time.
% \changes{v2.1a}{90/02/15}{Group added around `th@..}
% \changes{v2.1a}{90/02/15}{Reseting of `..@begintheorem removed.}
%    \begin{macrocode}
      \begingroup
        \csname th@\the\theorem@style \endcsname
      \endgroup}
%    \end{macrocode}
% \end{macro}
%
%
% \begin{macro}{\@begintheorem}
% \changes{v2.1a}{90/02/15}{Relaxed at toplevel.}
% \begin{macro}{\@opargbegintheorem}
% \changes{v2.1a}{90/02/15}{Relaxed at toplevel.}
%    We reset |\@begintheorem| and |\@opargbegintheorem| to
%    |\relax| since these commands are no longer necessary at
%    toplevel. This will save a few tokens.
%    \begin{macrocode}
\global\let\@begintheorem\relax
\global\let\@opargbegintheorem\relax
%    \end{macrocode}
% \end{macro}
% \end{macro}
%
% \begin{macro}{\theorem@style}
%    Obviously the \textsf{token} register used above has to be
%    allocated. To assure the utmost compatibility with the
%    original \LaTeX{} definition, we set the default theorem style to
%    \texttt{plain}, which implements the usual \LaTeX{} convention.
% \changes{v2.0f}{89/07/20}{Unnecessary `global' removed.}
%    \begin{macrocode}
\newtoks\theorem@style
\global\theorem@style{plain}
%    \end{macrocode}
% \end{macro}
%
% \begin{macro}{\theorembodyfont}
% \begin{macro}{\theorem@bodyfont}
%    If the |\theorembodyfont| is set by the user then it should not
%    interact with the default font set in the theorem style. When the
%    new font selection is in force this may happen if, for example,
%    the default is |\itshape| and the new |\theorembodyfont| is
%    |\sffamily|. So we add a |\reset@font| command in front of the
%    user definition.
% \changes{v2.0f}{89/07/20}{Unnecessary `global' removed.}
% \changes{v2.0g}{89/09/19}{`theoremfont to `theorembodyfont.}
% \changes{v2.1c}{92/06/22}{`reset@font added and logic changed.}
% \changes{v2.1d}{92/10/28}{Reset to empty if arg is empty}
%    \begin{macrocode}
\gdef\theorembodyfont#1{%
%    \end{macrocode}
%    We check if the argument supplied is empty and if so put nothing
%    into the |\theorem@bodyfont| token register to allow for
%    |\theorembodyfont{}| as a mean of using the default of the
%    current |\theoremstyle|.
%    \begin{macrocode}
   \def\@tempa{#1}%
   \ifx\@tempa\@empty
    \theorem@bodyfont{}%
   \else
    \theorem@bodyfont{\reset@font#1}%
   \fi
   }
\newtoks\theorem@bodyfont
\global\theorem@bodyfont{}
%    \end{macrocode}
% \end{macro}
% \end{macro}
%
%
% \begin{macro}{\theoremheaderfont}
% \changes{v2.0g}{89/09/19}{Macro added.}
%    The font for the theorem headers is handled differently because
%    this definition applies to all theorem styles.
%    \begin{macrocode}
\gdef\theoremheaderfont#1{\gdef\theorem@headerfont{#1}%
%    \end{macrocode}
%    After using the macro once it is redefined to produce an error
%    message.
%    \begin{macrocode}
       \gdef\theoremheaderfont##1{%
        \typeout{\string\theoremheaderfont\space should be used
                 only once.}}}
%    \end{macrocode}
% \end{macro}
%
% \begin{macro}{\theorem@headerfont}
% \changes{v2.0g}{89/09/19}{Macro added.}
%    To set the |\theorem@headerfont| default we first test if
%    the new fontselection scheme is in force.
% \changes{v2.1a}{90/02/15}{Support for new font selection.}
%    \begin{macrocode}
\ifx\upshape\undefined
%    \end{macrocode}
%    If not we define it to expand into |\bfseries|. We don't use
%    |\let| just in case a following style option redefines this
%    macro.
%    \begin{macrocode}
\gdef\theorem@headerfont{\bfseries}
%    \end{macrocode}
%    Otherwise we reset the current shape before calling |\bfseries|.
%    \begin{macrocode}
\else \gdef\theorem@headerfont{\normalfont\bfseries}\fi
%    \end{macrocode}
% \end{macro}
%
% \begin{macro}{\th@plain}
% \begin{macro}{\th@break}
% \begin{macro}{\th@marginbreak}
% \begin{macro}{\th@changebreak}
% \begin{macro}{\th@change}
% \begin{macro}{\th@margin}
%    The different styles are defined in macros such as
%    |\th@plain|.  Since memory space is precious in
%    ``non-Big-versions'', we have to avoid offering too many unused
%    definitions.  Therefore we define these styles in separate files
%    that can be loaded on demand.  Thus the commands themselves only
%    load these files. We use |\@input@|, a \LaTeXe{} internal command
%    that ensures that the file will be listed with |\listfiles|.
% \changes{v2.2a}{94/02/02}{Use `@input@ to load theorem layout files}
%    \begin{macrocode}
\gdef\th@plain{\@input@{thp.sty}}
\gdef\th@break{\@input@{thb.sty}}
\gdef\th@marginbreak{\@input@{thmb.sty}}
\gdef\th@changebreak{\@input@{thcb.sty}}
\gdef\th@change{\@input@{thc.sty}}
\gdef\th@margin{\@input@{thm.sty}}
%    \end{macrocode}
%    This list will be expanded when new styles become
%    available.  For testing,  just append new theorem substyles
%    as document options.
% \end{macro}
% \end{macro}
% \end{macro}
% \end{macro}
% \end{macro}
% \end{macro}
%
%
%
% \subsection{Definition of a new theorem set}
%
% As already pointed out, a new theorem environment can be defined in
% three different ways:
% \begin{verbatim}
%    \newtheorem{Lem}{Lemma}
%    \newtheorem{Lem}{Lemma}[section]
%    \newtheorem{Lem}[Theorem]{Lemma}
%\end{verbatim}
% The function of the macro |\newtheorem| is to recognize these cases
% and then to branch into one of the three macros |\@ynthm|, |\@xnthm|
% or |\@othm|.  This mechanism is adopted unchanged from
% \cite{latex:source}; the essential point here is that, for example,
% in the second case, the arguments \texttt{Lem}, \texttt{Lemma} and
% \texttt{section} are passed over to the macro |\@xnthm|.
%
% We inspect this case first because the others present fewer problems,
% and thus are easily derived from this one.
%
% \begin{macro}{\@xnthm}
%    For our example arguments, the macro |\@xnthm| must fulfill
%    the following:
%    \begin{itemize}
%       \item
%          Define a new \LaTeX-counter `Lem'
%       \item
%          reset this counter within a |\section|
%       \item
%          define the macro |\theLem|
%       \item
%          define the environment macros |\Lem| and
%          |\endLem| using the current
%          |\theoremstyle| and |\theorem@bodyfont|.
%    \end{itemize}
%    Obviously, all this should happen only if the first argument
%    of |\@xnthm| (i.e.\ \texttt{Lem} in our example) is chosen so
%    as not to conflict with any previously defined commands or
%    environments.  This test is performed by the \LaTeX{} macro
%    |\@ifdefinable|.
%    \begin{macrocode}
\gdef\@xnthm#1#2[#3]{\expandafter\@ifdefinable\csname #1\endcsname
%    \end{macrocode}
%    Therefore, the first argument of |\@ifdefinable| is the
%    expansion (in the example, |\Lem|) of
%    |\csname||#1\endcsname|.  The second argument is
%    executed only if the test has been completed successfully.
%    \begin{macrocode}
   {%
%    \end{macrocode}
%    Now we define the new counter. The names of the \LaTeX{}
%    macros employed should speak for themselves:
% \changes{v2.2b}{1995/10/07}{Use \cs{@newctr}}
% \changes{v2.2c}{1995/11/23}{Use \cs{@newctr} properly.}
%    \begin{macrocode}
    \@definecounter{#1}\@newctr{#1}[#3]%
%    \end{macrocode}
%    Using |\@newctr| will give a proper error message if the counter
%    in |#3| is not defined.
%    In defining `|\theLem|' we must generate the desired macro
%    name by use of |\expandafter| and |\csname|.
%    \begin{macrocode}
    \expandafter\xdef\csname the#1\endcsname
%    \end{macrocode}
%    An |\xdef| is used in order to make the definition global,
%    and to ensure that it contains the replacement texts of
%    |\@thmcountersep| and |\@thmcounter|.\footnote{These
%    two macros can be defined by the document style.  Their default
%    values produce a `\texttt{.}' as separation and an arabic
%    representation of the number.} However, not everything should be
%    expanded. For example, it saves space to use |\thesection|
%    instead of its---at times---lengthy expansion.
%    \begin{macrocode}
      {\expandafter \noexpand \csname the#3\endcsname
       \@thmcountersep \@thmcounter{#1}}%
%    \end{macrocode}
%    Thus with the defaults of \LaTeX{}, |\theLem| would be
%    replaced by the command sequence |\thesection.\arabic{Lem}|.
%
%    We will now look at the definition of the macro which is executed
%    at the beginning of the actual environment (in our example this
%    macro is |\Lem|). It should be noted that we use an
%    ``|\expandafter| trick'' to expand only certain parts of the
%    replacement text at the time of the definition.
%    \begin{macrocode}
    \def\@tempa{\global\@namedef{#1}}%
    \expandafter \@tempa \expandafter{%
%    \end{macrocode}
%    First, the macro that contains the current definitions of
%    |\@begintheorem| and |\@opargtheorem| should be called
%    up.  The name of this macro---as is already known---has the form
%    |\th@|\meta{theorem\ style}; therefore, it must be called by
%    \begin{macrocode}
      \csname th@\the \theorem@style
%    \end{macrocode}
%    In addition the default theorem font should be changeable, i.e.\
%    we have to insert the contents of |\theorem@bodyfont|. For
%    that reason, we expand even further, beyond |\endcsname|,
%    and thus insert the contents of the \textsf{token} register
%    |\theorem@bodyfont| in the replacement text.
%    \begin{macrocode}
            \expandafter \endcsname \the \theorem@bodyfont
%    \end{macrocode}
%    Now it is time to call the macro |\@thm| which takes over
%    the further processing.  It has two arguments: the current
%    counter name (in our example, \texttt{Lem}), and the text of the
%    label (in our example, \texttt{Lemma}).
%    \begin{macrocode}
     \@thm{#1}{#2}}%
%    \end{macrocode}
%    With this, the `sub-definition' is complete.  The macro
%    |\@endtheorem| ends a theorem environment and is, so far, nothing
%    but an |\endtrivlist|.  (Hence it is defined globally, and not
%    within the theorem styles.\footnote{This has to be changed as
%    soon as theorem styles that change \texttt{\bslash @endtheorem}
%    exist.  In such a case, all existing styles must be changed as
%    well since they will have to reset the macro.}) Therefore, we can
%    set it equivalent to the macro that ends the theorem set (in our
%    example, |\endLem|).  However, if some day theorem styles exist
%    that do change |\@endtheorem|, we would have to use the
%    commented-out line instead.
%    \begin{macrocode}
    \global \expandafter \let \csname end#1\endcsname \@endtheorem
%   \global\@namedef{end#1}{\@endtheorem}%
%    \end{macrocode}
%    With these commands all the required definitions are employed,
%    unless the test |\@ifdefinable| has failed.  Therefore, we
%    end the second argument of this macro and with it the definition
%    of |\@xnthm|.
%    \begin{macrocode}
   }}
%    \end{macrocode}
% \end{macro}
%
% \begin{macro}{\@ynthm}
%    The definition of |\@ynthm| is completely analogous. In
%    this case the new counter that is defined is not reset within
%    another counter; thus the definition of |\the...|
%    is simplified:
%    \begin{macrocode}
\gdef\@ynthm#1#2{\expandafter\@ifdefinable\csname #1\endcsname
   {\@definecounter{#1}%
    \expandafter\xdef\csname the#1\endcsname{\@thmcounter{#1}}%
%    \end{macrocode}
%    The rest of the definition corresponds literally to that of
%    |\@xnthm|:
%    \begin{macrocode}
    \def\@tempa{\global\@namedef{#1}}\expandafter \@tempa
     \expandafter{\csname th@\the \theorem@style \expandafter
     \endcsname \the\theorem@bodyfont \@thm{#1}{#2}}%
    \global \expandafter \let \csname end#1\endcsname \@endtheorem}}
%    \end{macrocode}
% \end{macro}
%
% \begin{macro}{\@othm}
%    The definition of |\@othm| does not contain anything new.
%    \begin{macrocode}
\gdef\@othm#1[#2]#3{%
%    \end{macrocode}
% \changes{v2.2b}{95/10/07}{Build in check for counter defined}
%    We do not define a new counter but instead use one that has
%    already been defined.  Thus the only definition we need is that
%    of this pseudo-counter (i.e.\ |\the|\meta{env.\ name}).
%    First we check if |#2| corresponds to a known counter name.
%    \begin{macrocode}
  \expandafter\ifx\csname c@#2\endcsname\relax
   \@nocounterr{#2}%
  \else
   \expandafter\@ifdefinable\csname #1\endcsname
   {\expandafter \xdef \csname the#1\endcsname
     {\expandafter \noexpand \csname the#2\endcsname}%
%    \end{macrocode}
%    All other parts of the definition can be adopted from
%    |\@xnthm|. We have to remember, though, that in this case
%    the name of the current counter and the theorem label have moved
%    to the second and third arguments.
%    \begin{macrocode}
    \def\@tempa{\global\@namedef{#1}}\expandafter \@tempa
     \expandafter{\csname th@\the \theorem@style \expandafter
     \endcsname \the\theorem@bodyfont \@thm{#2}{#3}}%
    \global \expandafter \let \csname end#1\endcsname \@endtheorem}%
  \fi}
%    \end{macrocode}
% \end{macro}
%
% \subsection{Macros that are employed in a {\envfont theorem}
%    environment}
%
% \begin{macro}{\@thm}
%    The macro |\@thm| has to increase the current counter.  Then,
%    depending on whether the environment has (or does not have) an
%    optional argument, it has to branch into either |\@begintheorem|
%    or |\@opargtheorem|.
%    \begin{macrocode}
\gdef\@thm#1#2{\refstepcounter{#1}%
%    \end{macrocode}
%    Now we start a {\envfont trivlist} environment, and give
%    |\@topsep| and |\@topsepadd| the values of the \textsf{skip}
%    registers |\theorempreskipamount| and |\theorempostskipamount|.
%    The value in |\@topsep| is the vertical space that is inserted by
%    the first (and only) |\item| in our |\trivlist| whilst
%    |\@topsepadd| is inserted by |\@endparenv| at the end of that
%    {\envfont trivlist} environment.  By using these registers, we
%    obtain the desired space around a {\envfont theorem} environment.
%    \begin{macrocode}
   \trivlist
   \@topsep \theorempreskipamount               % used by first \item
   \@topsepadd \theorempostskipamount           % used by \@endparenv
%    \end{macrocode}
%    Now we have to test whether an optional argument has been given.
% \changes{v2.0d}{89/07/18}{Forgotten percent added.}
%    \begin{macrocode}
   \@ifnextchar [%
%    \end{macrocode}
%    If there is an optional argument, we will call |\@ythm|, and
%    move the arguments read back into the input stream.
%    \begin{macrocode}
   {\@ythm{#1}{#2}}%
%    \end{macrocode}
%    If not, we call |\@begintheorem|. Its first argument is the
%    name of the theorem set (hence the second argument of
%    |\@thm|). Its second argument is the macro that produces the
%    current number.
%    \begin{macrocode}
   {\@begintheorem{#2}{\csname the#1\endcsname}\ignorespaces}}
%    \end{macrocode}
% \end{macro}
%
% \begin{macro}{\@xthm}
% \begin{macro}{\@ythm}
%    Both these macros were originally called by |\@thm|.  We do
%    not need |\@xthm| anymore, hence we reset it to
%    |\relax|. The definition of |\@ythm| has not changed at
%    all from its definition in \LaTeX. In order to make the macros
%    easier to understand, we will
%    nevertheless present it (commented out).
%    \begin{macrocode}
\global\let\@xthm\relax
% \def\@ythm#1#2[#3]{\@opargbegintheorem{#2}{\csname
%           the#1\endcsname}{#3}\ignorespaces}
%</package>
%    \end{macrocode}
%    The primitive |\ignorespaces| in |\@ythm| and
%    |\@thm| is needed to remove the spaces between the
%    |\begin{...}| and the actual text.
% \end{macro}
% \end{macro}
%
%
% \subsection{Definition of the theorem substyles}
%
% As already pointed out, the theorem substyles, defined below, are
% only loaded when necessary.  Note that all these substyles, except
% \texttt{plain}, have |\slshape| as the default body font.
%
% \subsubsection{The \texttt{plain} style}
%
% \label{dok:th@plain} As the following macros use |@|, we have
% to locally set the |\catcode| of this symbol to ``letter''.
% This happens within a group, so that we do not have to worry about
% which |\catcode| that symbol had before.
%    \begin{macrocode}
%<*thp>
\begingroup \makeatletter
%    \end{macrocode}
% Since we are now within a group, we must make all definitions
% globally.  First we make sure that \texttt{theorem.sty} is loaded.
% This will allow us to use this file as a document style option
% without having to call \texttt{theorem} itself as an option.  At the
% same time, we assure that at least version 2 is loaded, since
% |\theorem@style| was not defined in earlier versions.
%    \begin{macrocode}
\@ifundefined{theorem@style}{% \iffalse meta-comment
%
% Copyright (C) 1993-2024
%
% The LaTeX Project and any individual authors listed elsewhere
% in this file.
%
% This file is part of the Standard LaTeX `Tools Bundle'.
% -------------------------------------------------------
%
% It may be distributed and/or modified under the
% conditions of the LaTeX Project Public License, either version 1.3c
% of this license or (at your option) any later version.
% The latest version of this license is in
%    https://www.latex-project.org/lppl.txt
% and version 1.3c or later is part of all distributions of LaTeX
% version 2005/12/01 or later.
%
% The list of all files belonging to the LaTeX `Tools Bundle' is
% given in the file `manifest.txt'.
%
% \fi
%
%%
%
%
\def\FMithmInfo{2023/07/05 v2.2c Theorem extension package (FMi)}
%
% \ProvidesFile{theorem.dtx}[\FMithmInfo]
%\iffalse   % this is a METACOMMENT !
%%
%% Package `theorem' to use with LaTeX2e
%% Copyright (C) 1989-1995 Frank Mittelbach, all rights reserved.
%<+package>\@ifundefined{theorem@style}{}{\endinput}
%<+package>\NeedsTeXFormat{LaTeX2e}[1995/06/01]
%<+package>\ProvidesPackage{theorem}
%<+package>         [\FMithmInfo]
%
% \fi
%
% \changes{v1.0}{87/05/07}{First draft.}
% \changes{v1.0a}{87/05/07}{`thistheoremstyle' added.}
% \changes{v1.1a}{87/07/29}{New concept: redefine the internal
%                          \protect\LaTeX{} commands `@xnthm' etc.}
% \changes{v1.1b}{87/07/31}{`theoremfont' used to specify font.}
% \changes{v1.2a}{88/01/08}{`@endtheorem' now globally defined.}
% \changes{v1.2b}{88/01/11}{`newtheorem' only allowed in the preamble.}
% \changes{v1.3a}{88/02/14}{`theoremfont' now a toks register.}
%
% \changes{v2.0a}{88/11/16}{New implementation and new documentation.}
% \changes{v2.0a}{88/11/16}{No layout changes but internal modifications
%                           old substyles won't work.}
% \changes{v2.0b}{89/01/30}{`global' used for all definitions.}
% \changes{v2.0c}{89/06/15}{Documentation translated to English.
%                            Thanks to LLL and Nina}
% \changes{v2.1a}{90/02/15}{Support for new font selection.}
% \changes{v2.2a}{94/02/03}{Cleanup for LaTeX2e}
%
% \DoNotIndex{\@endparenv,\@ifdefinable,\@ifnextchar,\@ifundefined}
% \DoNotIndex{\@namedef,\@nameuse,\@spaces,\@tempa,\@warning}
% \DoNotIndex{\ }
% \DoNotIndex{\begingroup,\bfseries}
% \DoNotIndex{\csname}
% \DoNotIndex{\def,\do,\docdate}
% \DoNotIndex{\endcsname,\endgroup,\endinput,\endtrivlist,\expandafter}
% \DoNotIndex{\filedate,\fileversion}
% \DoNotIndex{\gdef,\global}
% \DoNotIndex{\hbox,\hskip}
% \DoNotIndex{\ignorespaces,\input,\itshape,\item}
% \DoNotIndex{\labelsep,\let,\llap}
% \DoNotIndex{\makeatletter}
% \DoNotIndex{\noexpand}
% \DoNotIndex{\refstepcounter,\relax,\rlap}
% \DoNotIndex{\setlength,\slshape,\space,\string,\strut}
% \DoNotIndex{\the,\trivlist,\typeout}
% \DoNotIndex{\vbox}
% \DoNotIndex{\xdef}
%
% \def\deflabel#1{\ttfamily #1\hfill}
% \def\deflist#1{\begin{list}{}{\settowidth\labelwidth{\ttfamily #1}%
%                               \setlength\leftmargin\labelwidth
%                               \addtolength\leftmargin\labelsep
%                               \let\makelabel\deflabel}}
% \def\enddeflist{\end{list}}
% \def\envfont{\normalfont\ttfamily}
%
% \MakeShortVerb{\|}
%
% \newcommand{\TUB}{{\sl TUGboat\/}}
%
% \GetFileInfo{theorem.dtx}
% \title{An Extension of the
%        \LaTeX{} theorem environment\thanks{This file
%        has version number \fileversion, last
%        revised \filedate.}}
% \author{Frank Mittelbach}
%
% \MaintainedByLaTeXTeam{tools}
% \maketitle
%
% \begin{abstract}
%    The macros described in this paper yield an extension of the
%    \LaTeX{} theorem mechanism. It is designed to satisfy the
%    different requirements of various journals. Thus, the
%    layout of the ``theorems'' can be manipulated by determining a
%    ``style''. This article describes not only the use, but
%    also the definition, of the necessary macros.
% \end{abstract}
%
% \section*{Preface to version 2.2}
%
% For \LaTeXe{} this package did not need any fundamental changes. I
% only modified the messages generated so that theorem layout styles
% will show up with the |\listfiles| command and cleaned the section
% on the New Font Selection Scheme since this is now included in
% \LaTeX.
%
% \section*{Preface to version 2.1}
%
% This version is identical to 2.0g described in \TUB\ 10\#3 except
% for some internal defaults which are now set depending on the used
% font selection scheme.
%
% This was done to avoid unpleasant surprises if the new font selection
% scheme is in force. For further details see
% section~\ref{sec:newfonts} and \cite{bk:GMS94}.
%
% \section{Introduction}
%
% For our purposes here, ``theorems'' are labelled enunciations,
% often set off from the main text by extra space and a font change.
% Theorems, corollaries, conjectures, definitions, and remarks
% are all instances of ``theorems''.  The ``header'' of these structures
% is composed of a label (such as \textsc{Theorem} or \textsc{Remark})
% and a number which serializes an item in the sequence of items
% with the same label.
%
% Shortly after the introduction of \LaTeX{} at the Fachbereich
% Mathematik in Mainz, the desire to manipulate the layout of
% ``theorems'' arose.  In Mainz, the following two conventions came
% into general use:
% \begin{enumerate}
%   \item The number of the theorem is shown in the margin.
%   \item There is a line break at
%         the end of the theorem header.
% \end{enumerate}
%
% Additionally, some journals require different formats which depend on
% the ``sort of theorem'': e.g.\ often remarks and definitions are set
% in |\upshape|, while |\itshape| is employed for main theorems.
%
% Confronted with these requirements, a theorem environment was
% developed in Mainz which allows separate determination of the
% layout of the ``theorems sets'', comparable to |\pagestyle|.
%
%
% \section{The user interface}
% \subsection{Defining new theorem sets}
%
% \DescribeMacro\newtheorem
% As in the original \LaTeX{} version, the command |\newtheorem|
% defines a new ``theorem set'' or ``theorem-like structure''.
% Two required arguments name the new environment and give the text
% to be typeset with each instance of the new ``set'', while
% an optional argument determines how
% the ``set'' is enumerated:
% \begin{description}
%    \item[\ttfamily \bslash newtheorem\{foo\}\{bar\}]
%       The theorem set {\envfont foo} (whose name is \texttt{bar})
%       uses its own counter.
%    \item[\ttfamily \bslash newtheorem\{foo2\}{[foo]}\{bar2\}]
%       The theorem set {\envfont foo2} (printed name \texttt{bar2})
%       uses the same counter as the theorem set \texttt{foo}.
%    \item[\ttfamily \bslash newtheorem\{foo3\}\{bar3\}{[section]}]
%       The theorem set {\envfont foo3} (printed name \texttt{bar3}) is
%       enumerated within the counter \texttt{section}, i.e.\ with every
%       new |\section| the enumeration begins again with 1, and
%       the enumeration is composed from the section-number and the
%       theorem counter itself.
% \end{description}
%
% \DescribeMacro\theoremstyle
% Additionally, the command |\theoremstyle| can define the layout
% of various, or all, theorem sets.  It should be noted that any
% theorem set defined by |\newtheorem| is typeset in the
% |\theoremstyle| that is current at the time of the definition.
% Thus, the following
% \begin{verbatim}
%    \theoremstyle{break}        \newtheorem{Cor}{Corollary}
%    \theoremstyle{plain}        \newtheorem{Exa}{Example}[section]
%\end{verbatim}                             ^^A No blank at line start
% leads to the result that the set {\envfont Cor} is formatted in the
% style \texttt{break}, while the set {\envfont Exa} and all the
% following ones are formatted in the style \texttt{plain}, unless
% another |\theoremstyle| follows.  Since the definitions installed by
% |\newtheorem| are global, one also can limit |\theoremstyle| locally
% by grouping braces.
%
% \DescribeMacro\theorembodyfont
% The choice of the font for the theorem body is completely
% independent of the chosen |\theoremstyle|; this has proven to
% be very advantageous.  For example,
% \begin{verbatim}
%   {\theorembodyfont{\upshape}          \newtheorem{Rem}{Remark}}
%\end{verbatim}
% defines a theorem set {\envfont Rem}, which will be set in
% |\upshape| in the current layout (which in our example is
% \texttt{plain}).  As with |\theoremstyle|, the |\theorembodyfont|
% chosen is that current at the time of |\newtheorem|.  If
% |\theorembodyfont| is not specified or one defines
% |\theorembodyfont{}|, then the font used will be that defined by the
% |\theoremstyle|.
%
% \DescribeMacro\theoremheaderfont It is also possible to customize
% the font used for the theorem headers.  This is, however, a global
% declaration, and therefore there should be at most one
% |\theoremheaderfont| declaration in the preamble.\footnote{If it is
% actually necessary to have different header fonts, one has to define
% new theorem styles (substituting the desired font) or specify the
% information directly in the \texttt{\bslash newtheorem} declaration
% (the unclean variant).}
%
% \DescribeMacro\theorempreskipamount
% \DescribeMacro\theorempostskipamount Two additional parameters
% affect the vertical space around the theorem environments:
% |\theorempreskipamount| and |\theorempostskipamount| define,
% respectively, the spacing before and after such an environment.
% These parameters apply for all theorem sets and can be manipulated
% with the ordinary length macros.  They are rubber lengths
% (`\textsf{skips}'), and therefore can contain \texttt{plus} and
% \texttt{minus} parts.
%
% Since the definition of theorem sets should---most sensibly---be
% placed in the preamble, we only allow installation there.
% It is therefore possible to release the memory used here after
% |\begin{document}|, in order to make room for other
% applications.
%
%
% \subsection{Existing theorem styles}
%
% The following theorem styles exist to date:
% \begin{deflist}{marginbreak}
%    \item[plain]
%       This theorem style emulates the original \LaTeX{} definition,
%       except that additionally the parameters
%       |\theorem...skipamount| are used.
%    \item[break]
%       In this style, the theorem header is followed by a line
%       break.
%    \item[marginbreak]
%       The theorem number is set in the margin, and there is a line
%       break as in \texttt{break}.
%    \item[changebreak]
%       Like \texttt{break}, but with header number and text
%       interchanged.
%    \item[change]
%       Header number and text are interchanged, without a line break.
%    \item[margin]
%       The number is set in the left margin, without a line break.
% \end{deflist}
% All styles (except \texttt{plain}) select |\slshape| as the default
% |\theorembodyfont|.
%
%
% \subsection{Examples}
%
% Given the above theorem sets {\envfont Cor}, {\envfont Exa}
% and {\envfont Rem},
% suppose that the preamble also contains the declarations:
% \begin{verbatim}
%    \theoremstyle{marginbreak}   \newtheorem{Lem}[Cor]{Lemma}
%    \theoremstyle{change}
%    \theorembodyfont{\itshape}        \newtheorem{Def}[Cor]{Definition}
%
%    \theoremheaderfont{\scshape}
%\end{verbatim}
% Then the following are some typical examples of the typeset output
% resulting from their use.
%
% \begin{Cor}
%    This is a sentence typeset in the theorem environment {\envfont
%    Cor}.
% \end{Cor}
%
% \begin{Exa}
%    This is a sentence typeset in the theorem environment {\envfont
%    Exa}.
% \end{Exa}
%
% \begin{Rem}
%    This is a sentence typeset in the theorem environment {\envfont
%    Rem}.
% \end{Rem}
%
% \begin{Lem}[Ben User]
%    This is a sentence typeset in the theorem environment {\envfont
%    Lem}.
% \end{Lem}
%
% \begin{Def}[Very impressive Definition]
%    This is a sentence typeset in the theorem environment {\envfont
%    Def}.
% \end{Def}
%
% The last two examples show the effect of the optional argument to
% a theorem environment (it is the text typeset in parentheses).
%
%
% \section{Special Considerations}
% \label{sec:newfonts}
%
% Theoremheader and body are implemented as a unit. This means that
% the |\theoremheaderfont|  will inherit characteristics of the
% |\theorembodyfont| in \LaTeXe. Thus, if for example
% |\theorembodyfont| is |\itshape| and |\theoremheaderfont|
% is |\bfseries| the font selected for the header will have the
% characteristics `bold extended italic'. If this is not desired one
% should set the |\theoremheaderfont| to something like
% \begin{verbatim}
%      \theoremheaderfont{\normalfont\bfseries}
%\end{verbatim}
% i.e.\ supplying all necessary font information explicitly.
%
%
%
% \section{Acknowledgements}
%
% The publication of this set of macros was only possible with the
% help of Christina Busse (translating the manuscript into English),
% Joachim Pense (playing the r\^ole of typist), Chris Rowley
% (looking everything over) and many others providing useful
% suggestions.
%
% \MaybeStop{%
%    \begin{thebibliography}{9}
%       \bibitem{bk:GMS94} \textsc{M.~Goossens}, \textsc{F.~Mittelbach}
%         and \textsc{A.~Samarin}.
%         \newblock The \LaTeX{} Companion.
%         \newblock
%         Addison-Wesley, Reading, Massachusetts, 1994.
%       \bibitem{latex:source} \textsc{Lamport, Leslie.} \newblock
%                               \texttt{latex.tex}, version 2.09, date
%                               Feb. 1990.
%    \end{thebibliography}
%    \PrintIndex
%    \PrintChanges}
%
%
% \section{The documentation driver file}
%
% The next bit of code contains the documentation driver file for
% \TeX{}, i.e., the file that will produce the documentation you are
% currently reading. It will be extracted from this file by the
% \texttt{docstrip} program. Since it is the first code in the file
% one can alternatively process this file directly with \LaTeXe{} to
% obtain the documentation.
%
%    \begin{macrocode}
%<*driver>
\documentclass{ltxdoc}

\usepackage{theorem}

%<+driver>% The next few lines define theorem sets which are used
%<+driver>% in the example section of the documentation.

\theoremstyle{break}       \newtheorem{Cor}{Corollary}
\theoremstyle{plain}       \newtheorem{Exa}{Example}[section]
{\theorembodyfont{\upshape}\newtheorem{Rem}{Remark}}
\theoremstyle{marginbreak} \newtheorem{Lem}[Cor]{Lemma}
\theoremstyle{change}
\theorembodyfont{\itshape} \newtheorem{Def}[Cor]{Definition}

\theoremheaderfont{\scshape}

\RecordChanges

\begin{document}
   \DocInput{theorem.dtx}
\end{document}
%</driver>
%    \end{macrocode}
%
% \section{Definition of the Macros}
%
% If the file has been loaded before, we abort immediately. If not the
% package announces itself (this is actually done at the very top of
% the file---the way it is done isn't good style so don't copy it).
% \changes{v2.0e}{89/07/19}{Spaces removed from `typeout'.}
% \changes{v2.2b}{95/11/19}{Announce moved to top of file}
%    \begin{macrocode}
%<*package>
%\@ifundefined{theorem@style}{}{\endinput}
%\def\FMithmInfo{1995/11/19 v2.2b Theorem extension package (FMi)}
%\ProvidesPackage{theorem}[\FMithmInfo]
%    \end{macrocode}
%
%
% \subsection{Definition of theorem styles and fonts}
%
% All the definitions in this file are done globally to allow
% inputting this file inside a group.
%
% \begin{macro}{\theoremstyle}
%    Before a theorem style can be installed, the chosen style must be
%    known. For that reason, we must test to see that
%    |\th@|\meta{style} is known or, more precisely, that it is
%    different from |\relax|. If the style is not known then
%    |\th@plain| is used.
%    \begin{macrocode}
\gdef\theoremstyle#1{%
   \@ifundefined{th@#1}{\@warning
          {Unknown theoremstyle `#1'. Using `plain'}%
          \theorem@style{plain}}%
%    \end{macrocode}
%    We save the theorem style to be used in the token register
%    |\theorem@style|.
%    \begin{macrocode}
      {\theorem@style{#1}}%
%    \end{macrocode}
%    Now we ``evaluate'' the theorem style: this means, we call the
%    macro |\th@|\meta{style} which will activate the relevant
%    definitions which are contained in a separate file.  This is done
%    in a group to suppress changes to the current font.  This could
%    otherwise pose problems together with the new font selection
%    scheme\footnote{When I printed the original article using the new
%    font selection scheme I ended with a document with slanted
%    typefaces (text headings and all) simply because one of the
%    theorem styles used \texttt{\bslash sl} at toplevel.} if the
%    |\th@|\meta{style} is evaluated a second time.
% \changes{v2.1a}{90/02/15}{Group added around `th@..}
% \changes{v2.1a}{90/02/15}{Reseting of `..@begintheorem removed.}
%    \begin{macrocode}
      \begingroup
        \csname th@\the\theorem@style \endcsname
      \endgroup}
%    \end{macrocode}
% \end{macro}
%
%
% \begin{macro}{\@begintheorem}
% \changes{v2.1a}{90/02/15}{Relaxed at toplevel.}
% \begin{macro}{\@opargbegintheorem}
% \changes{v2.1a}{90/02/15}{Relaxed at toplevel.}
%    We reset |\@begintheorem| and |\@opargbegintheorem| to
%    |\relax| since these commands are no longer necessary at
%    toplevel. This will save a few tokens.
%    \begin{macrocode}
\global\let\@begintheorem\relax
\global\let\@opargbegintheorem\relax
%    \end{macrocode}
% \end{macro}
% \end{macro}
%
% \begin{macro}{\theorem@style}
%    Obviously the \textsf{token} register used above has to be
%    allocated. To assure the utmost compatibility with the
%    original \LaTeX{} definition, we set the default theorem style to
%    \texttt{plain}, which implements the usual \LaTeX{} convention.
% \changes{v2.0f}{89/07/20}{Unnecessary `global' removed.}
%    \begin{macrocode}
\newtoks\theorem@style
\global\theorem@style{plain}
%    \end{macrocode}
% \end{macro}
%
% \begin{macro}{\theorembodyfont}
% \begin{macro}{\theorem@bodyfont}
%    If the |\theorembodyfont| is set by the user then it should not
%    interact with the default font set in the theorem style. When the
%    new font selection is in force this may happen if, for example,
%    the default is |\itshape| and the new |\theorembodyfont| is
%    |\sffamily|. So we add a |\reset@font| command in front of the
%    user definition.
% \changes{v2.0f}{89/07/20}{Unnecessary `global' removed.}
% \changes{v2.0g}{89/09/19}{`theoremfont to `theorembodyfont.}
% \changes{v2.1c}{92/06/22}{`reset@font added and logic changed.}
% \changes{v2.1d}{92/10/28}{Reset to empty if arg is empty}
%    \begin{macrocode}
\gdef\theorembodyfont#1{%
%    \end{macrocode}
%    We check if the argument supplied is empty and if so put nothing
%    into the |\theorem@bodyfont| token register to allow for
%    |\theorembodyfont{}| as a mean of using the default of the
%    current |\theoremstyle|.
%    \begin{macrocode}
   \def\@tempa{#1}%
   \ifx\@tempa\@empty
    \theorem@bodyfont{}%
   \else
    \theorem@bodyfont{\reset@font#1}%
   \fi
   }
\newtoks\theorem@bodyfont
\global\theorem@bodyfont{}
%    \end{macrocode}
% \end{macro}
% \end{macro}
%
%
% \begin{macro}{\theoremheaderfont}
% \changes{v2.0g}{89/09/19}{Macro added.}
%    The font for the theorem headers is handled differently because
%    this definition applies to all theorem styles.
%    \begin{macrocode}
\gdef\theoremheaderfont#1{\gdef\theorem@headerfont{#1}%
%    \end{macrocode}
%    After using the macro once it is redefined to produce an error
%    message.
%    \begin{macrocode}
       \gdef\theoremheaderfont##1{%
        \typeout{\string\theoremheaderfont\space should be used
                 only once.}}}
%    \end{macrocode}
% \end{macro}
%
% \begin{macro}{\theorem@headerfont}
% \changes{v2.0g}{89/09/19}{Macro added.}
%    To set the |\theorem@headerfont| default we first test if
%    the new fontselection scheme is in force.
% \changes{v2.1a}{90/02/15}{Support for new font selection.}
%    \begin{macrocode}
\ifx\upshape\undefined
%    \end{macrocode}
%    If not we define it to expand into |\bfseries|. We don't use
%    |\let| just in case a following style option redefines this
%    macro.
%    \begin{macrocode}
\gdef\theorem@headerfont{\bfseries}
%    \end{macrocode}
%    Otherwise we reset the current shape before calling |\bfseries|.
%    \begin{macrocode}
\else \gdef\theorem@headerfont{\normalfont\bfseries}\fi
%    \end{macrocode}
% \end{macro}
%
% \begin{macro}{\th@plain}
% \begin{macro}{\th@break}
% \begin{macro}{\th@marginbreak}
% \begin{macro}{\th@changebreak}
% \begin{macro}{\th@change}
% \begin{macro}{\th@margin}
%    The different styles are defined in macros such as
%    |\th@plain|.  Since memory space is precious in
%    ``non-Big-versions'', we have to avoid offering too many unused
%    definitions.  Therefore we define these styles in separate files
%    that can be loaded on demand.  Thus the commands themselves only
%    load these files. We use |\@input@|, a \LaTeXe{} internal command
%    that ensures that the file will be listed with |\listfiles|.
% \changes{v2.2a}{94/02/02}{Use `@input@ to load theorem layout files}
%    \begin{macrocode}
\gdef\th@plain{\@input@{thp.sty}}
\gdef\th@break{\@input@{thb.sty}}
\gdef\th@marginbreak{\@input@{thmb.sty}}
\gdef\th@changebreak{\@input@{thcb.sty}}
\gdef\th@change{\@input@{thc.sty}}
\gdef\th@margin{\@input@{thm.sty}}
%    \end{macrocode}
%    This list will be expanded when new styles become
%    available.  For testing,  just append new theorem substyles
%    as document options.
% \end{macro}
% \end{macro}
% \end{macro}
% \end{macro}
% \end{macro}
% \end{macro}
%
%
%
% \subsection{Definition of a new theorem set}
%
% As already pointed out, a new theorem environment can be defined in
% three different ways:
% \begin{verbatim}
%    \newtheorem{Lem}{Lemma}
%    \newtheorem{Lem}{Lemma}[section]
%    \newtheorem{Lem}[Theorem]{Lemma}
%\end{verbatim}
% The function of the macro |\newtheorem| is to recognize these cases
% and then to branch into one of the three macros |\@ynthm|, |\@xnthm|
% or |\@othm|.  This mechanism is adopted unchanged from
% \cite{latex:source}; the essential point here is that, for example,
% in the second case, the arguments \texttt{Lem}, \texttt{Lemma} and
% \texttt{section} are passed over to the macro |\@xnthm|.
%
% We inspect this case first because the others present fewer problems,
% and thus are easily derived from this one.
%
% \begin{macro}{\@xnthm}
%    For our example arguments, the macro |\@xnthm| must fulfill
%    the following:
%    \begin{itemize}
%       \item
%          Define a new \LaTeX-counter `Lem'
%       \item
%          reset this counter within a |\section|
%       \item
%          define the macro |\theLem|
%       \item
%          define the environment macros |\Lem| and
%          |\endLem| using the current
%          |\theoremstyle| and |\theorem@bodyfont|.
%    \end{itemize}
%    Obviously, all this should happen only if the first argument
%    of |\@xnthm| (i.e.\ \texttt{Lem} in our example) is chosen so
%    as not to conflict with any previously defined commands or
%    environments.  This test is performed by the \LaTeX{} macro
%    |\@ifdefinable|.
%    \begin{macrocode}
\gdef\@xnthm#1#2[#3]{\expandafter\@ifdefinable\csname #1\endcsname
%    \end{macrocode}
%    Therefore, the first argument of |\@ifdefinable| is the
%    expansion (in the example, |\Lem|) of
%    |\csname||#1\endcsname|.  The second argument is
%    executed only if the test has been completed successfully.
%    \begin{macrocode}
   {%
%    \end{macrocode}
%    Now we define the new counter. The names of the \LaTeX{}
%    macros employed should speak for themselves:
% \changes{v2.2b}{1995/10/07}{Use \cs{@newctr}}
% \changes{v2.2c}{1995/11/23}{Use \cs{@newctr} properly.}
%    \begin{macrocode}
    \@definecounter{#1}\@newctr{#1}[#3]%
%    \end{macrocode}
%    Using |\@newctr| will give a proper error message if the counter
%    in |#3| is not defined.
%    In defining `|\theLem|' we must generate the desired macro
%    name by use of |\expandafter| and |\csname|.
%    \begin{macrocode}
    \expandafter\xdef\csname the#1\endcsname
%    \end{macrocode}
%    An |\xdef| is used in order to make the definition global,
%    and to ensure that it contains the replacement texts of
%    |\@thmcountersep| and |\@thmcounter|.\footnote{These
%    two macros can be defined by the document style.  Their default
%    values produce a `\texttt{.}' as separation and an arabic
%    representation of the number.} However, not everything should be
%    expanded. For example, it saves space to use |\thesection|
%    instead of its---at times---lengthy expansion.
%    \begin{macrocode}
      {\expandafter \noexpand \csname the#3\endcsname
       \@thmcountersep \@thmcounter{#1}}%
%    \end{macrocode}
%    Thus with the defaults of \LaTeX{}, |\theLem| would be
%    replaced by the command sequence |\thesection.\arabic{Lem}|.
%
%    We will now look at the definition of the macro which is executed
%    at the beginning of the actual environment (in our example this
%    macro is |\Lem|). It should be noted that we use an
%    ``|\expandafter| trick'' to expand only certain parts of the
%    replacement text at the time of the definition.
%    \begin{macrocode}
    \def\@tempa{\global\@namedef{#1}}%
    \expandafter \@tempa \expandafter{%
%    \end{macrocode}
%    First, the macro that contains the current definitions of
%    |\@begintheorem| and |\@opargtheorem| should be called
%    up.  The name of this macro---as is already known---has the form
%    |\th@|\meta{theorem\ style}; therefore, it must be called by
%    \begin{macrocode}
      \csname th@\the \theorem@style
%    \end{macrocode}
%    In addition the default theorem font should be changeable, i.e.\
%    we have to insert the contents of |\theorem@bodyfont|. For
%    that reason, we expand even further, beyond |\endcsname|,
%    and thus insert the contents of the \textsf{token} register
%    |\theorem@bodyfont| in the replacement text.
%    \begin{macrocode}
            \expandafter \endcsname \the \theorem@bodyfont
%    \end{macrocode}
%    Now it is time to call the macro |\@thm| which takes over
%    the further processing.  It has two arguments: the current
%    counter name (in our example, \texttt{Lem}), and the text of the
%    label (in our example, \texttt{Lemma}).
%    \begin{macrocode}
     \@thm{#1}{#2}}%
%    \end{macrocode}
%    With this, the `sub-definition' is complete.  The macro
%    |\@endtheorem| ends a theorem environment and is, so far, nothing
%    but an |\endtrivlist|.  (Hence it is defined globally, and not
%    within the theorem styles.\footnote{This has to be changed as
%    soon as theorem styles that change \texttt{\bslash @endtheorem}
%    exist.  In such a case, all existing styles must be changed as
%    well since they will have to reset the macro.}) Therefore, we can
%    set it equivalent to the macro that ends the theorem set (in our
%    example, |\endLem|).  However, if some day theorem styles exist
%    that do change |\@endtheorem|, we would have to use the
%    commented-out line instead.
%    \begin{macrocode}
    \global \expandafter \let \csname end#1\endcsname \@endtheorem
%   \global\@namedef{end#1}{\@endtheorem}%
%    \end{macrocode}
%    With these commands all the required definitions are employed,
%    unless the test |\@ifdefinable| has failed.  Therefore, we
%    end the second argument of this macro and with it the definition
%    of |\@xnthm|.
%    \begin{macrocode}
   }}
%    \end{macrocode}
% \end{macro}
%
% \begin{macro}{\@ynthm}
%    The definition of |\@ynthm| is completely analogous. In
%    this case the new counter that is defined is not reset within
%    another counter; thus the definition of |\the...|
%    is simplified:
%    \begin{macrocode}
\gdef\@ynthm#1#2{\expandafter\@ifdefinable\csname #1\endcsname
   {\@definecounter{#1}%
    \expandafter\xdef\csname the#1\endcsname{\@thmcounter{#1}}%
%    \end{macrocode}
%    The rest of the definition corresponds literally to that of
%    |\@xnthm|:
%    \begin{macrocode}
    \def\@tempa{\global\@namedef{#1}}\expandafter \@tempa
     \expandafter{\csname th@\the \theorem@style \expandafter
     \endcsname \the\theorem@bodyfont \@thm{#1}{#2}}%
    \global \expandafter \let \csname end#1\endcsname \@endtheorem}}
%    \end{macrocode}
% \end{macro}
%
% \begin{macro}{\@othm}
%    The definition of |\@othm| does not contain anything new.
%    \begin{macrocode}
\gdef\@othm#1[#2]#3{%
%    \end{macrocode}
% \changes{v2.2b}{95/10/07}{Build in check for counter defined}
%    We do not define a new counter but instead use one that has
%    already been defined.  Thus the only definition we need is that
%    of this pseudo-counter (i.e.\ |\the|\meta{env.\ name}).
%    First we check if |#2| corresponds to a known counter name.
%    \begin{macrocode}
  \expandafter\ifx\csname c@#2\endcsname\relax
   \@nocounterr{#2}%
  \else
   \expandafter\@ifdefinable\csname #1\endcsname
   {\expandafter \xdef \csname the#1\endcsname
     {\expandafter \noexpand \csname the#2\endcsname}%
%    \end{macrocode}
%    All other parts of the definition can be adopted from
%    |\@xnthm|. We have to remember, though, that in this case
%    the name of the current counter and the theorem label have moved
%    to the second and third arguments.
%    \begin{macrocode}
    \def\@tempa{\global\@namedef{#1}}\expandafter \@tempa
     \expandafter{\csname th@\the \theorem@style \expandafter
     \endcsname \the\theorem@bodyfont \@thm{#2}{#3}}%
    \global \expandafter \let \csname end#1\endcsname \@endtheorem}%
  \fi}
%    \end{macrocode}
% \end{macro}
%
% \subsection{Macros that are employed in a {\envfont theorem}
%    environment}
%
% \begin{macro}{\@thm}
%    The macro |\@thm| has to increase the current counter.  Then,
%    depending on whether the environment has (or does not have) an
%    optional argument, it has to branch into either |\@begintheorem|
%    or |\@opargtheorem|.
%    \begin{macrocode}
\gdef\@thm#1#2{\refstepcounter{#1}%
%    \end{macrocode}
%    Now we start a {\envfont trivlist} environment, and give
%    |\@topsep| and |\@topsepadd| the values of the \textsf{skip}
%    registers |\theorempreskipamount| and |\theorempostskipamount|.
%    The value in |\@topsep| is the vertical space that is inserted by
%    the first (and only) |\item| in our |\trivlist| whilst
%    |\@topsepadd| is inserted by |\@endparenv| at the end of that
%    {\envfont trivlist} environment.  By using these registers, we
%    obtain the desired space around a {\envfont theorem} environment.
%    \begin{macrocode}
   \trivlist
   \@topsep \theorempreskipamount               % used by first \item
   \@topsepadd \theorempostskipamount           % used by \@endparenv
%    \end{macrocode}
%    Now we have to test whether an optional argument has been given.
% \changes{v2.0d}{89/07/18}{Forgotten percent added.}
%    \begin{macrocode}
   \@ifnextchar [%
%    \end{macrocode}
%    If there is an optional argument, we will call |\@ythm|, and
%    move the arguments read back into the input stream.
%    \begin{macrocode}
   {\@ythm{#1}{#2}}%
%    \end{macrocode}
%    If not, we call |\@begintheorem|. Its first argument is the
%    name of the theorem set (hence the second argument of
%    |\@thm|). Its second argument is the macro that produces the
%    current number.
%    \begin{macrocode}
   {\@begintheorem{#2}{\csname the#1\endcsname}\ignorespaces}}
%    \end{macrocode}
% \end{macro}
%
% \begin{macro}{\@xthm}
% \begin{macro}{\@ythm}
%    Both these macros were originally called by |\@thm|.  We do
%    not need |\@xthm| anymore, hence we reset it to
%    |\relax|. The definition of |\@ythm| has not changed at
%    all from its definition in \LaTeX. In order to make the macros
%    easier to understand, we will
%    nevertheless present it (commented out).
%    \begin{macrocode}
\global\let\@xthm\relax
% \def\@ythm#1#2[#3]{\@opargbegintheorem{#2}{\csname
%           the#1\endcsname}{#3}\ignorespaces}
%</package>
%    \end{macrocode}
%    The primitive |\ignorespaces| in |\@ythm| and
%    |\@thm| is needed to remove the spaces between the
%    |\begin{...}| and the actual text.
% \end{macro}
% \end{macro}
%
%
% \subsection{Definition of the theorem substyles}
%
% As already pointed out, the theorem substyles, defined below, are
% only loaded when necessary.  Note that all these substyles, except
% \texttt{plain}, have |\slshape| as the default body font.
%
% \subsubsection{The \texttt{plain} style}
%
% \label{dok:th@plain} As the following macros use |@|, we have
% to locally set the |\catcode| of this symbol to ``letter''.
% This happens within a group, so that we do not have to worry about
% which |\catcode| that symbol had before.
%    \begin{macrocode}
%<*thp>
\begingroup \makeatletter
%    \end{macrocode}
% Since we are now within a group, we must make all definitions
% globally.  First we make sure that \texttt{theorem.sty} is loaded.
% This will allow us to use this file as a document style option
% without having to call \texttt{theorem} itself as an option.  At the
% same time, we assure that at least version 2 is loaded, since
% |\theorem@style| was not defined in earlier versions.
%    \begin{macrocode}
\@ifundefined{theorem@style}{\input{theorem.sty}}{}
\ProvidesFile{thp.sty}
             [\FMithmInfo]
%    \end{macrocode}
%
%
% \begin{macro}{\th@plain}
% \begin{macro}{\@begintheorem}
% \begin{macro}{\@opargbegintheorem}
%    |\theoremstyle{plain}| corresponds to the original
%    definition, except that the distances to the surrounding text are
%    determined by the parameters |\theorempreskipamount| and
%    |\theorempostskipamount|.
%    First we set the default body font.
%    \begin{macrocode}
\gdef\th@plain{\normalfont\itshape
%    \end{macrocode}
%    Then we define |\@begintheorem| and
%    |\@opargbegintheorem|. These two macros define how the
%    header of a theorem is typeset.  |\@opargbegintheorem| will
%    be called if a \textsf{theorem} environment with an optional
%    argument is encountered; otherwise, the header is constructed by
%    calling |\@begintheorem|.  If one of these macros is
%    executed, we are within a \textsf{trivlist} environment started by
%    |\@thm|.  So the theorem header is produced with an
%    |\item| command.
%
%    Instead of specifying the header font directly, all standard
%    theorem styles use the |\theorem@headerfont| macro to allow
%    customization.  The extra space (|\labelsep|) is necessary
%    because of problems in the \textsf{trivlist} environment.
%    \begin{macrocode}
  \def\@begintheorem##1##2{%
        \item[\hskip\labelsep \theorem@headerfont ##1\ ##2]}%
%    \end{macrocode}
%    The definition of |\@opargbegintheorem| is completely
%    analogous. The only difference is the fact that there exists a
%    third argument (which is the optional parameter of the
%    environment and contains additional information about the
%    theorem).  Customarily we enclose it in parentheses.
%    \begin{macrocode}
\def\@opargbegintheorem##1##2##3{%
   \item[\hskip\labelsep \theorem@headerfont ##1\ ##2\ (##3)]}}
%    \end{macrocode}
% \end{macro}
% \end{macro}
% \end{macro}
%
% \noindent
% We conclude with an |\endgroup| to restore the
% |\catcode| of |@|.
%    \begin{macrocode}
\endgroup
%</thp>
%    \end{macrocode}
%
%
% \subsubsection{The \texttt{break} style}
%
% This style option is stored in the file |thb.sty|.
% For the next two lines see the documentation for |\th@plain| on
% page~\pageref{dok:th@plain}.
%    \begin{macrocode}
%<*thb>
\begingroup \makeatletter
\@ifundefined{theorem@style}{\input{theorem.sty}}{}
\ProvidesFile{thb.sty}
             [\FMithmInfo]
%    \end{macrocode}
%
% \begin{macro}{\th@break}
%    |\theoremstyle{break}| produces a line break after the name
%    of the theorem.  The font is |\slshape|.  Hence, we define
%    |\th@break| as follows:
%    \begin{macrocode}
\gdef\th@break{\normalfont\slshape
  \def\@begintheorem##1##2{\item[%
%    \end{macrocode}
%    We run into the following problem: it is not possible to create
%    the header with |\item[|\meta{title}|]| and then start
%    a new line by, for example, |\mbox{}\\|. Such a definition
%    will fail whenever a list environment follows immediately.  With
%    the above construction, the |\mbox{}| causes the switch
%    |@inlabel| (cf.\ definition of |\list| and
%    |\trivlist| in \cite{latex:source}) to be set to \texttt{false}
%    and so the following list will insert additional vertical
%    space (|\topskip|). This is quite annoying. Therefore, we
%    create the line break within the |\item|. In order to ensure
%    that the text will begin at the proper position in the following
%    line, we simply pretend that the label does not take any
%    room.\footnote{This will lead to problems whenever very high
%    symbols occurring in the line tower into the
%    heading. So, something else has to be done here sometime.}
%    \begin{macrocode}
       \rlap{\vbox{\hbox{\hskip \labelsep\theorem@headerfont ##1\ ##2}%
                   \hbox{\strut}}}]}%
%    \end{macrocode}
%    Again, the definition of |\@opargbegintheorem| is
%    completely analogous.
%    \begin{macrocode}
\def\@opargbegintheorem##1##2##3{%
  \item[\rlap{\vbox{\hbox{\hskip \labelsep \theorem@headerfont
                     ##1\ ##2\ (##3)}%
                    \hbox{\strut}}}]}}
\endgroup
%</thb>
%    \end{macrocode}
% \end{macro}
%
%
% \subsubsection{The \texttt{changebreak} style}
%
%    \begin{macrocode}
% This style option is stored in the file |thcb.sty|.
%    \begin{macrocode}
%<*thcb>
\begingroup \makeatletter
\@ifundefined{theorem@style}{\input{theorem.sty}}{}
\ProvidesFile{thcb.sty}
             [\FMithmInfo]
%    \end{macrocode}
%
% \begin{macro}{\th@changebreak}
%    The \texttt{change-break} theorem style is like \texttt{break}
%    but with interchange of theorem name and theorem number.  Thus we
%    define |\th@changebreak| as follows:
%    \begin{macrocode}
\gdef\th@changebreak{\normalfont\slshape
  \def\@begintheorem##1##2{\item
       [\rlap{\vbox{\hbox{\hskip\labelsep \theorem@headerfont ##2\ ##1}%
                    \hbox{\strut}}}]}%
\def\@opargbegintheorem##1##2##3{%
  \item[\rlap{\vbox{\hbox{\hskip\labelsep \theorem@headerfont
                     ##2\ ##1\ (##3)}%
                    \hbox{\strut}}}]}}
\endgroup
%</thcb>
%    \end{macrocode}
% \end{macro}
%
%
% \subsubsection{The \texttt{change} style}
%
% This style option is stored in the file |thc.sty|.
%    \begin{macrocode}
%<*thc>
\begingroup \makeatletter
\@ifundefined{theorem@style}{\input{theorem.sty}}{}
\ProvidesFile{thc.sty}
             [\FMithmInfo]
%    \end{macrocode}
%
% \begin{macro}{\th@change}
%    The \texttt{change} theorem style corresponds to the
%    \texttt{change break} style without a linebreak after the header.
%    To say it in another way, it's the same as the \texttt{plain}
%    style but with number and name interchanged and |\slshape| as the
%    default font.
%    \begin{macrocode}
\gdef\th@change{\normalfont\slshape
  \def\@begintheorem##1##2{\item
       [\hskip\labelsep \theorem@headerfont ##2\ ##1]}%
\def\@opargbegintheorem##1##2##3{%
  \item[\hskip\labelsep \theorem@headerfont ##2\ ##1\ (##3)]}}
\endgroup
%</thc>
%    \end{macrocode}
% \end{macro}
%
%
% \subsubsection{The \texttt{marginbreak} style}
%
% This style option is the one used most often at Mainz.  It is saved
% in the file |thmb.sty|.
%    \begin{macrocode}
%<*thmb>
\begingroup \makeatletter
\@ifundefined{theorem@style}{\input{theorem.sty}}{}
\ProvidesFile{thmb.sty}
             [\FMithmInfo]
%    \end{macrocode}
%
% \begin{macro}{\th@marginbreak}
%    The \texttt{margin break} style is nearly the same as the
%    \texttt{change break} style. The only difference is the placement
%    of the theorem number.  We use |\llap| to place it in the left
%    margin.
%
%    In this style |\labelsep| denotes the separation between the
%    number and the text.
%    \begin{macrocode}
\gdef\th@marginbreak{\normalfont\slshape
  \def\@begintheorem##1##2{\item
       [\rlap{\vbox{\theorem@headerfont
                    \hbox{\llap{##2}\hskip\labelsep ##1}%
                    \hbox{\strut}}}]}%
\def\@opargbegintheorem##1##2##3{%
  \item[\rlap{\vbox{\theorem@headerfont
                    \hbox{\llap{##2}\hskip\labelsep ##1\ (##3)}%
                    \hbox{\strut}}}]}}
\endgroup
%</thmb>
%    \end{macrocode}
% \changes{v2.1c}{90/02/17}{`endgroup outside macrocode env.}
% \end{macro}
%
%
%
% \subsubsection{The \texttt{margin} style}
%
% This style option is stored in the file |thm.sty|.
%    \begin{macrocode}
%<*thm>
\begingroup \makeatletter
\@ifundefined{theorem@style}{\input{theorem.sty}}{}
\ProvidesFile{thm.sty}
             [\FMithmInfo]
%    \end{macrocode}
%
% \begin{macro}{\th@margin}
%    Again this is only a variant of the theorem styles described
%    above without any new ideas.
%    \begin{macrocode}
\gdef\th@margin{\normalfont\slshape
  \def\@begintheorem##1##2{\item
       [\theorem@headerfont \llap{##2}\hskip\labelsep ##1]}%
\def\@opargbegintheorem##1##2##3{%
  \item[\theorem@headerfont \llap{##2}\hskip\labelsep ##1\ (##3)]}}
\endgroup
%</thm>
%    \end{macrocode}
% \end{macro}
%
%
%
%
% \subsection{Final Definitions}
%
% \begin{macro}{\theorempreskipamount}
% \begin{macro}{\theorempostskipamount}
%    The \textsf{skip} parameters that regulate the vertical empty space
%    before and after the {\envfont theorem} environment have
%    to be allocated as well.
% \changes{v2.0f}{89/07/20}{Unnecessary `global' removed.}
%    \begin{macrocode}
%<*package>
\newskip\theorempreskipamount
\newskip\theorempostskipamount
%    \end{macrocode}
%    Since we have used the same values for all theorem sets, we now can
%    assign them.
%    \begin{macrocode}
\global\setlength\theorempreskipamount{12pt plus 5pt minus 3pt}
\global\setlength\theorempostskipamount{8pt plus 3pt minus 1.5pt}
%    \end{macrocode}
% \end{macro}
% \end{macro}
%
%
% \begin{macro}{\@endtheorem}
%    The same holds for the macro |\@endtheorem|, which ends a
%    {\envfont theorem} environment. Since it is the same for all
%    theorem sets, it is removed from the macros |\th@|\meta{style}.
%    It simply ends the {\envfont trivlist} environment, which was
%    begun in |\@thm|.
%    \begin{macrocode}
\global\let\@endtheorem=\endtrivlist
%    \end{macrocode}
% \end{macro}
%
%
%
% \begin{macro}{\@preamblecmds}
%    All macros defined above are to be used only in the preamble.
%    Therefore, we insert them in |\@preamblecmds| which will disable
%    them at begin document. This is done by the internal \LaTeXe{}
%    command |\@onlypreamble|.
% \changes{v2.0g}{89/09/19}{`theoremheaderfont and `theorembodyfont
%    added.}
% \changes{v2.2a}{94/02/03}{Use \LaTeXe{} interface}
%    \begin{macrocode}
\@onlypreamble\@xnthm
\@onlypreamble\@ynthm
\@onlypreamble\@othm
\@onlypreamble\newtheorem
\@onlypreamble\theoremstyle
\@onlypreamble\theorembodyfont
\@onlypreamble\theoremheaderfont
%    \end{macrocode}
% \end{macro}
%
% Finally we declare the \texttt{plain} theorem style to be the default.
% \changes{v2.1b}{90/06/04}{Fixed missing `item by declaring default
%    style.}
%    \begin{macrocode}
\theoremstyle{plain}
%</package>
%    \end{macrocode}
%
% \Finale

\endinput
}{}
\ProvidesFile{thp.sty}
             [\FMithmInfo]
%    \end{macrocode}
%
%
% \begin{macro}{\th@plain}
% \begin{macro}{\@begintheorem}
% \begin{macro}{\@opargbegintheorem}
%    |\theoremstyle{plain}| corresponds to the original
%    definition, except that the distances to the surrounding text are
%    determined by the parameters |\theorempreskipamount| and
%    |\theorempostskipamount|.
%    First we set the default body font.
%    \begin{macrocode}
\gdef\th@plain{\normalfont\itshape
%    \end{macrocode}
%    Then we define |\@begintheorem| and
%    |\@opargbegintheorem|. These two macros define how the
%    header of a theorem is typeset.  |\@opargbegintheorem| will
%    be called if a \textsf{theorem} environment with an optional
%    argument is encountered; otherwise, the header is constructed by
%    calling |\@begintheorem|.  If one of these macros is
%    executed, we are within a \textsf{trivlist} environment started by
%    |\@thm|.  So the theorem header is produced with an
%    |\item| command.
%
%    Instead of specifying the header font directly, all standard
%    theorem styles use the |\theorem@headerfont| macro to allow
%    customization.  The extra space (|\labelsep|) is necessary
%    because of problems in the \textsf{trivlist} environment.
%    \begin{macrocode}
  \def\@begintheorem##1##2{%
        \item[\hskip\labelsep \theorem@headerfont ##1\ ##2]}%
%    \end{macrocode}
%    The definition of |\@opargbegintheorem| is completely
%    analogous. The only difference is the fact that there exists a
%    third argument (which is the optional parameter of the
%    environment and contains additional information about the
%    theorem).  Customarily we enclose it in parentheses.
%    \begin{macrocode}
\def\@opargbegintheorem##1##2##3{%
   \item[\hskip\labelsep \theorem@headerfont ##1\ ##2\ (##3)]}}
%    \end{macrocode}
% \end{macro}
% \end{macro}
% \end{macro}
%
% \noindent
% We conclude with an |\endgroup| to restore the
% |\catcode| of |@|.
%    \begin{macrocode}
\endgroup
%</thp>
%    \end{macrocode}
%
%
% \subsubsection{The \texttt{break} style}
%
% This style option is stored in the file |thb.sty|.
% For the next two lines see the documentation for |\th@plain| on
% page~\pageref{dok:th@plain}.
%    \begin{macrocode}
%<*thb>
\begingroup \makeatletter
\@ifundefined{theorem@style}{% \iffalse meta-comment
%
% Copyright (C) 1993-2024
%
% The LaTeX Project and any individual authors listed elsewhere
% in this file.
%
% This file is part of the Standard LaTeX `Tools Bundle'.
% -------------------------------------------------------
%
% It may be distributed and/or modified under the
% conditions of the LaTeX Project Public License, either version 1.3c
% of this license or (at your option) any later version.
% The latest version of this license is in
%    https://www.latex-project.org/lppl.txt
% and version 1.3c or later is part of all distributions of LaTeX
% version 2005/12/01 or later.
%
% The list of all files belonging to the LaTeX `Tools Bundle' is
% given in the file `manifest.txt'.
%
% \fi
%
%%
%
%
\def\FMithmInfo{2023/07/05 v2.2c Theorem extension package (FMi)}
%
% \ProvidesFile{theorem.dtx}[\FMithmInfo]
%\iffalse   % this is a METACOMMENT !
%%
%% Package `theorem' to use with LaTeX2e
%% Copyright (C) 1989-1995 Frank Mittelbach, all rights reserved.
%<+package>\@ifundefined{theorem@style}{}{\endinput}
%<+package>\NeedsTeXFormat{LaTeX2e}[1995/06/01]
%<+package>\ProvidesPackage{theorem}
%<+package>         [\FMithmInfo]
%
% \fi
%
% \changes{v1.0}{87/05/07}{First draft.}
% \changes{v1.0a}{87/05/07}{`thistheoremstyle' added.}
% \changes{v1.1a}{87/07/29}{New concept: redefine the internal
%                          \protect\LaTeX{} commands `@xnthm' etc.}
% \changes{v1.1b}{87/07/31}{`theoremfont' used to specify font.}
% \changes{v1.2a}{88/01/08}{`@endtheorem' now globally defined.}
% \changes{v1.2b}{88/01/11}{`newtheorem' only allowed in the preamble.}
% \changes{v1.3a}{88/02/14}{`theoremfont' now a toks register.}
%
% \changes{v2.0a}{88/11/16}{New implementation and new documentation.}
% \changes{v2.0a}{88/11/16}{No layout changes but internal modifications
%                           old substyles won't work.}
% \changes{v2.0b}{89/01/30}{`global' used for all definitions.}
% \changes{v2.0c}{89/06/15}{Documentation translated to English.
%                            Thanks to LLL and Nina}
% \changes{v2.1a}{90/02/15}{Support for new font selection.}
% \changes{v2.2a}{94/02/03}{Cleanup for LaTeX2e}
%
% \DoNotIndex{\@endparenv,\@ifdefinable,\@ifnextchar,\@ifundefined}
% \DoNotIndex{\@namedef,\@nameuse,\@spaces,\@tempa,\@warning}
% \DoNotIndex{\ }
% \DoNotIndex{\begingroup,\bfseries}
% \DoNotIndex{\csname}
% \DoNotIndex{\def,\do,\docdate}
% \DoNotIndex{\endcsname,\endgroup,\endinput,\endtrivlist,\expandafter}
% \DoNotIndex{\filedate,\fileversion}
% \DoNotIndex{\gdef,\global}
% \DoNotIndex{\hbox,\hskip}
% \DoNotIndex{\ignorespaces,\input,\itshape,\item}
% \DoNotIndex{\labelsep,\let,\llap}
% \DoNotIndex{\makeatletter}
% \DoNotIndex{\noexpand}
% \DoNotIndex{\refstepcounter,\relax,\rlap}
% \DoNotIndex{\setlength,\slshape,\space,\string,\strut}
% \DoNotIndex{\the,\trivlist,\typeout}
% \DoNotIndex{\vbox}
% \DoNotIndex{\xdef}
%
% \def\deflabel#1{\ttfamily #1\hfill}
% \def\deflist#1{\begin{list}{}{\settowidth\labelwidth{\ttfamily #1}%
%                               \setlength\leftmargin\labelwidth
%                               \addtolength\leftmargin\labelsep
%                               \let\makelabel\deflabel}}
% \def\enddeflist{\end{list}}
% \def\envfont{\normalfont\ttfamily}
%
% \MakeShortVerb{\|}
%
% \newcommand{\TUB}{{\sl TUGboat\/}}
%
% \GetFileInfo{theorem.dtx}
% \title{An Extension of the
%        \LaTeX{} theorem environment\thanks{This file
%        has version number \fileversion, last
%        revised \filedate.}}
% \author{Frank Mittelbach}
%
% \MaintainedByLaTeXTeam{tools}
% \maketitle
%
% \begin{abstract}
%    The macros described in this paper yield an extension of the
%    \LaTeX{} theorem mechanism. It is designed to satisfy the
%    different requirements of various journals. Thus, the
%    layout of the ``theorems'' can be manipulated by determining a
%    ``style''. This article describes not only the use, but
%    also the definition, of the necessary macros.
% \end{abstract}
%
% \section*{Preface to version 2.2}
%
% For \LaTeXe{} this package did not need any fundamental changes. I
% only modified the messages generated so that theorem layout styles
% will show up with the |\listfiles| command and cleaned the section
% on the New Font Selection Scheme since this is now included in
% \LaTeX.
%
% \section*{Preface to version 2.1}
%
% This version is identical to 2.0g described in \TUB\ 10\#3 except
% for some internal defaults which are now set depending on the used
% font selection scheme.
%
% This was done to avoid unpleasant surprises if the new font selection
% scheme is in force. For further details see
% section~\ref{sec:newfonts} and \cite{bk:GMS94}.
%
% \section{Introduction}
%
% For our purposes here, ``theorems'' are labelled enunciations,
% often set off from the main text by extra space and a font change.
% Theorems, corollaries, conjectures, definitions, and remarks
% are all instances of ``theorems''.  The ``header'' of these structures
% is composed of a label (such as \textsc{Theorem} or \textsc{Remark})
% and a number which serializes an item in the sequence of items
% with the same label.
%
% Shortly after the introduction of \LaTeX{} at the Fachbereich
% Mathematik in Mainz, the desire to manipulate the layout of
% ``theorems'' arose.  In Mainz, the following two conventions came
% into general use:
% \begin{enumerate}
%   \item The number of the theorem is shown in the margin.
%   \item There is a line break at
%         the end of the theorem header.
% \end{enumerate}
%
% Additionally, some journals require different formats which depend on
% the ``sort of theorem'': e.g.\ often remarks and definitions are set
% in |\upshape|, while |\itshape| is employed for main theorems.
%
% Confronted with these requirements, a theorem environment was
% developed in Mainz which allows separate determination of the
% layout of the ``theorems sets'', comparable to |\pagestyle|.
%
%
% \section{The user interface}
% \subsection{Defining new theorem sets}
%
% \DescribeMacro\newtheorem
% As in the original \LaTeX{} version, the command |\newtheorem|
% defines a new ``theorem set'' or ``theorem-like structure''.
% Two required arguments name the new environment and give the text
% to be typeset with each instance of the new ``set'', while
% an optional argument determines how
% the ``set'' is enumerated:
% \begin{description}
%    \item[\ttfamily \bslash newtheorem\{foo\}\{bar\}]
%       The theorem set {\envfont foo} (whose name is \texttt{bar})
%       uses its own counter.
%    \item[\ttfamily \bslash newtheorem\{foo2\}{[foo]}\{bar2\}]
%       The theorem set {\envfont foo2} (printed name \texttt{bar2})
%       uses the same counter as the theorem set \texttt{foo}.
%    \item[\ttfamily \bslash newtheorem\{foo3\}\{bar3\}{[section]}]
%       The theorem set {\envfont foo3} (printed name \texttt{bar3}) is
%       enumerated within the counter \texttt{section}, i.e.\ with every
%       new |\section| the enumeration begins again with 1, and
%       the enumeration is composed from the section-number and the
%       theorem counter itself.
% \end{description}
%
% \DescribeMacro\theoremstyle
% Additionally, the command |\theoremstyle| can define the layout
% of various, or all, theorem sets.  It should be noted that any
% theorem set defined by |\newtheorem| is typeset in the
% |\theoremstyle| that is current at the time of the definition.
% Thus, the following
% \begin{verbatim}
%    \theoremstyle{break}        \newtheorem{Cor}{Corollary}
%    \theoremstyle{plain}        \newtheorem{Exa}{Example}[section]
%\end{verbatim}                             ^^A No blank at line start
% leads to the result that the set {\envfont Cor} is formatted in the
% style \texttt{break}, while the set {\envfont Exa} and all the
% following ones are formatted in the style \texttt{plain}, unless
% another |\theoremstyle| follows.  Since the definitions installed by
% |\newtheorem| are global, one also can limit |\theoremstyle| locally
% by grouping braces.
%
% \DescribeMacro\theorembodyfont
% The choice of the font for the theorem body is completely
% independent of the chosen |\theoremstyle|; this has proven to
% be very advantageous.  For example,
% \begin{verbatim}
%   {\theorembodyfont{\upshape}          \newtheorem{Rem}{Remark}}
%\end{verbatim}
% defines a theorem set {\envfont Rem}, which will be set in
% |\upshape| in the current layout (which in our example is
% \texttt{plain}).  As with |\theoremstyle|, the |\theorembodyfont|
% chosen is that current at the time of |\newtheorem|.  If
% |\theorembodyfont| is not specified or one defines
% |\theorembodyfont{}|, then the font used will be that defined by the
% |\theoremstyle|.
%
% \DescribeMacro\theoremheaderfont It is also possible to customize
% the font used for the theorem headers.  This is, however, a global
% declaration, and therefore there should be at most one
% |\theoremheaderfont| declaration in the preamble.\footnote{If it is
% actually necessary to have different header fonts, one has to define
% new theorem styles (substituting the desired font) or specify the
% information directly in the \texttt{\bslash newtheorem} declaration
% (the unclean variant).}
%
% \DescribeMacro\theorempreskipamount
% \DescribeMacro\theorempostskipamount Two additional parameters
% affect the vertical space around the theorem environments:
% |\theorempreskipamount| and |\theorempostskipamount| define,
% respectively, the spacing before and after such an environment.
% These parameters apply for all theorem sets and can be manipulated
% with the ordinary length macros.  They are rubber lengths
% (`\textsf{skips}'), and therefore can contain \texttt{plus} and
% \texttt{minus} parts.
%
% Since the definition of theorem sets should---most sensibly---be
% placed in the preamble, we only allow installation there.
% It is therefore possible to release the memory used here after
% |\begin{document}|, in order to make room for other
% applications.
%
%
% \subsection{Existing theorem styles}
%
% The following theorem styles exist to date:
% \begin{deflist}{marginbreak}
%    \item[plain]
%       This theorem style emulates the original \LaTeX{} definition,
%       except that additionally the parameters
%       |\theorem...skipamount| are used.
%    \item[break]
%       In this style, the theorem header is followed by a line
%       break.
%    \item[marginbreak]
%       The theorem number is set in the margin, and there is a line
%       break as in \texttt{break}.
%    \item[changebreak]
%       Like \texttt{break}, but with header number and text
%       interchanged.
%    \item[change]
%       Header number and text are interchanged, without a line break.
%    \item[margin]
%       The number is set in the left margin, without a line break.
% \end{deflist}
% All styles (except \texttt{plain}) select |\slshape| as the default
% |\theorembodyfont|.
%
%
% \subsection{Examples}
%
% Given the above theorem sets {\envfont Cor}, {\envfont Exa}
% and {\envfont Rem},
% suppose that the preamble also contains the declarations:
% \begin{verbatim}
%    \theoremstyle{marginbreak}   \newtheorem{Lem}[Cor]{Lemma}
%    \theoremstyle{change}
%    \theorembodyfont{\itshape}        \newtheorem{Def}[Cor]{Definition}
%
%    \theoremheaderfont{\scshape}
%\end{verbatim}
% Then the following are some typical examples of the typeset output
% resulting from their use.
%
% \begin{Cor}
%    This is a sentence typeset in the theorem environment {\envfont
%    Cor}.
% \end{Cor}
%
% \begin{Exa}
%    This is a sentence typeset in the theorem environment {\envfont
%    Exa}.
% \end{Exa}
%
% \begin{Rem}
%    This is a sentence typeset in the theorem environment {\envfont
%    Rem}.
% \end{Rem}
%
% \begin{Lem}[Ben User]
%    This is a sentence typeset in the theorem environment {\envfont
%    Lem}.
% \end{Lem}
%
% \begin{Def}[Very impressive Definition]
%    This is a sentence typeset in the theorem environment {\envfont
%    Def}.
% \end{Def}
%
% The last two examples show the effect of the optional argument to
% a theorem environment (it is the text typeset in parentheses).
%
%
% \section{Special Considerations}
% \label{sec:newfonts}
%
% Theoremheader and body are implemented as a unit. This means that
% the |\theoremheaderfont|  will inherit characteristics of the
% |\theorembodyfont| in \LaTeXe. Thus, if for example
% |\theorembodyfont| is |\itshape| and |\theoremheaderfont|
% is |\bfseries| the font selected for the header will have the
% characteristics `bold extended italic'. If this is not desired one
% should set the |\theoremheaderfont| to something like
% \begin{verbatim}
%      \theoremheaderfont{\normalfont\bfseries}
%\end{verbatim}
% i.e.\ supplying all necessary font information explicitly.
%
%
%
% \section{Acknowledgements}
%
% The publication of this set of macros was only possible with the
% help of Christina Busse (translating the manuscript into English),
% Joachim Pense (playing the r\^ole of typist), Chris Rowley
% (looking everything over) and many others providing useful
% suggestions.
%
% \MaybeStop{%
%    \begin{thebibliography}{9}
%       \bibitem{bk:GMS94} \textsc{M.~Goossens}, \textsc{F.~Mittelbach}
%         and \textsc{A.~Samarin}.
%         \newblock The \LaTeX{} Companion.
%         \newblock
%         Addison-Wesley, Reading, Massachusetts, 1994.
%       \bibitem{latex:source} \textsc{Lamport, Leslie.} \newblock
%                               \texttt{latex.tex}, version 2.09, date
%                               Feb. 1990.
%    \end{thebibliography}
%    \PrintIndex
%    \PrintChanges}
%
%
% \section{The documentation driver file}
%
% The next bit of code contains the documentation driver file for
% \TeX{}, i.e., the file that will produce the documentation you are
% currently reading. It will be extracted from this file by the
% \texttt{docstrip} program. Since it is the first code in the file
% one can alternatively process this file directly with \LaTeXe{} to
% obtain the documentation.
%
%    \begin{macrocode}
%<*driver>
\documentclass{ltxdoc}

\usepackage{theorem}

%<+driver>% The next few lines define theorem sets which are used
%<+driver>% in the example section of the documentation.

\theoremstyle{break}       \newtheorem{Cor}{Corollary}
\theoremstyle{plain}       \newtheorem{Exa}{Example}[section]
{\theorembodyfont{\upshape}\newtheorem{Rem}{Remark}}
\theoremstyle{marginbreak} \newtheorem{Lem}[Cor]{Lemma}
\theoremstyle{change}
\theorembodyfont{\itshape} \newtheorem{Def}[Cor]{Definition}

\theoremheaderfont{\scshape}

\RecordChanges

\begin{document}
   \DocInput{theorem.dtx}
\end{document}
%</driver>
%    \end{macrocode}
%
% \section{Definition of the Macros}
%
% If the file has been loaded before, we abort immediately. If not the
% package announces itself (this is actually done at the very top of
% the file---the way it is done isn't good style so don't copy it).
% \changes{v2.0e}{89/07/19}{Spaces removed from `typeout'.}
% \changes{v2.2b}{95/11/19}{Announce moved to top of file}
%    \begin{macrocode}
%<*package>
%\@ifundefined{theorem@style}{}{\endinput}
%\def\FMithmInfo{1995/11/19 v2.2b Theorem extension package (FMi)}
%\ProvidesPackage{theorem}[\FMithmInfo]
%    \end{macrocode}
%
%
% \subsection{Definition of theorem styles and fonts}
%
% All the definitions in this file are done globally to allow
% inputting this file inside a group.
%
% \begin{macro}{\theoremstyle}
%    Before a theorem style can be installed, the chosen style must be
%    known. For that reason, we must test to see that
%    |\th@|\meta{style} is known or, more precisely, that it is
%    different from |\relax|. If the style is not known then
%    |\th@plain| is used.
%    \begin{macrocode}
\gdef\theoremstyle#1{%
   \@ifundefined{th@#1}{\@warning
          {Unknown theoremstyle `#1'. Using `plain'}%
          \theorem@style{plain}}%
%    \end{macrocode}
%    We save the theorem style to be used in the token register
%    |\theorem@style|.
%    \begin{macrocode}
      {\theorem@style{#1}}%
%    \end{macrocode}
%    Now we ``evaluate'' the theorem style: this means, we call the
%    macro |\th@|\meta{style} which will activate the relevant
%    definitions which are contained in a separate file.  This is done
%    in a group to suppress changes to the current font.  This could
%    otherwise pose problems together with the new font selection
%    scheme\footnote{When I printed the original article using the new
%    font selection scheme I ended with a document with slanted
%    typefaces (text headings and all) simply because one of the
%    theorem styles used \texttt{\bslash sl} at toplevel.} if the
%    |\th@|\meta{style} is evaluated a second time.
% \changes{v2.1a}{90/02/15}{Group added around `th@..}
% \changes{v2.1a}{90/02/15}{Reseting of `..@begintheorem removed.}
%    \begin{macrocode}
      \begingroup
        \csname th@\the\theorem@style \endcsname
      \endgroup}
%    \end{macrocode}
% \end{macro}
%
%
% \begin{macro}{\@begintheorem}
% \changes{v2.1a}{90/02/15}{Relaxed at toplevel.}
% \begin{macro}{\@opargbegintheorem}
% \changes{v2.1a}{90/02/15}{Relaxed at toplevel.}
%    We reset |\@begintheorem| and |\@opargbegintheorem| to
%    |\relax| since these commands are no longer necessary at
%    toplevel. This will save a few tokens.
%    \begin{macrocode}
\global\let\@begintheorem\relax
\global\let\@opargbegintheorem\relax
%    \end{macrocode}
% \end{macro}
% \end{macro}
%
% \begin{macro}{\theorem@style}
%    Obviously the \textsf{token} register used above has to be
%    allocated. To assure the utmost compatibility with the
%    original \LaTeX{} definition, we set the default theorem style to
%    \texttt{plain}, which implements the usual \LaTeX{} convention.
% \changes{v2.0f}{89/07/20}{Unnecessary `global' removed.}
%    \begin{macrocode}
\newtoks\theorem@style
\global\theorem@style{plain}
%    \end{macrocode}
% \end{macro}
%
% \begin{macro}{\theorembodyfont}
% \begin{macro}{\theorem@bodyfont}
%    If the |\theorembodyfont| is set by the user then it should not
%    interact with the default font set in the theorem style. When the
%    new font selection is in force this may happen if, for example,
%    the default is |\itshape| and the new |\theorembodyfont| is
%    |\sffamily|. So we add a |\reset@font| command in front of the
%    user definition.
% \changes{v2.0f}{89/07/20}{Unnecessary `global' removed.}
% \changes{v2.0g}{89/09/19}{`theoremfont to `theorembodyfont.}
% \changes{v2.1c}{92/06/22}{`reset@font added and logic changed.}
% \changes{v2.1d}{92/10/28}{Reset to empty if arg is empty}
%    \begin{macrocode}
\gdef\theorembodyfont#1{%
%    \end{macrocode}
%    We check if the argument supplied is empty and if so put nothing
%    into the |\theorem@bodyfont| token register to allow for
%    |\theorembodyfont{}| as a mean of using the default of the
%    current |\theoremstyle|.
%    \begin{macrocode}
   \def\@tempa{#1}%
   \ifx\@tempa\@empty
    \theorem@bodyfont{}%
   \else
    \theorem@bodyfont{\reset@font#1}%
   \fi
   }
\newtoks\theorem@bodyfont
\global\theorem@bodyfont{}
%    \end{macrocode}
% \end{macro}
% \end{macro}
%
%
% \begin{macro}{\theoremheaderfont}
% \changes{v2.0g}{89/09/19}{Macro added.}
%    The font for the theorem headers is handled differently because
%    this definition applies to all theorem styles.
%    \begin{macrocode}
\gdef\theoremheaderfont#1{\gdef\theorem@headerfont{#1}%
%    \end{macrocode}
%    After using the macro once it is redefined to produce an error
%    message.
%    \begin{macrocode}
       \gdef\theoremheaderfont##1{%
        \typeout{\string\theoremheaderfont\space should be used
                 only once.}}}
%    \end{macrocode}
% \end{macro}
%
% \begin{macro}{\theorem@headerfont}
% \changes{v2.0g}{89/09/19}{Macro added.}
%    To set the |\theorem@headerfont| default we first test if
%    the new fontselection scheme is in force.
% \changes{v2.1a}{90/02/15}{Support for new font selection.}
%    \begin{macrocode}
\ifx\upshape\undefined
%    \end{macrocode}
%    If not we define it to expand into |\bfseries|. We don't use
%    |\let| just in case a following style option redefines this
%    macro.
%    \begin{macrocode}
\gdef\theorem@headerfont{\bfseries}
%    \end{macrocode}
%    Otherwise we reset the current shape before calling |\bfseries|.
%    \begin{macrocode}
\else \gdef\theorem@headerfont{\normalfont\bfseries}\fi
%    \end{macrocode}
% \end{macro}
%
% \begin{macro}{\th@plain}
% \begin{macro}{\th@break}
% \begin{macro}{\th@marginbreak}
% \begin{macro}{\th@changebreak}
% \begin{macro}{\th@change}
% \begin{macro}{\th@margin}
%    The different styles are defined in macros such as
%    |\th@plain|.  Since memory space is precious in
%    ``non-Big-versions'', we have to avoid offering too many unused
%    definitions.  Therefore we define these styles in separate files
%    that can be loaded on demand.  Thus the commands themselves only
%    load these files. We use |\@input@|, a \LaTeXe{} internal command
%    that ensures that the file will be listed with |\listfiles|.
% \changes{v2.2a}{94/02/02}{Use `@input@ to load theorem layout files}
%    \begin{macrocode}
\gdef\th@plain{\@input@{thp.sty}}
\gdef\th@break{\@input@{thb.sty}}
\gdef\th@marginbreak{\@input@{thmb.sty}}
\gdef\th@changebreak{\@input@{thcb.sty}}
\gdef\th@change{\@input@{thc.sty}}
\gdef\th@margin{\@input@{thm.sty}}
%    \end{macrocode}
%    This list will be expanded when new styles become
%    available.  For testing,  just append new theorem substyles
%    as document options.
% \end{macro}
% \end{macro}
% \end{macro}
% \end{macro}
% \end{macro}
% \end{macro}
%
%
%
% \subsection{Definition of a new theorem set}
%
% As already pointed out, a new theorem environment can be defined in
% three different ways:
% \begin{verbatim}
%    \newtheorem{Lem}{Lemma}
%    \newtheorem{Lem}{Lemma}[section]
%    \newtheorem{Lem}[Theorem]{Lemma}
%\end{verbatim}
% The function of the macro |\newtheorem| is to recognize these cases
% and then to branch into one of the three macros |\@ynthm|, |\@xnthm|
% or |\@othm|.  This mechanism is adopted unchanged from
% \cite{latex:source}; the essential point here is that, for example,
% in the second case, the arguments \texttt{Lem}, \texttt{Lemma} and
% \texttt{section} are passed over to the macro |\@xnthm|.
%
% We inspect this case first because the others present fewer problems,
% and thus are easily derived from this one.
%
% \begin{macro}{\@xnthm}
%    For our example arguments, the macro |\@xnthm| must fulfill
%    the following:
%    \begin{itemize}
%       \item
%          Define a new \LaTeX-counter `Lem'
%       \item
%          reset this counter within a |\section|
%       \item
%          define the macro |\theLem|
%       \item
%          define the environment macros |\Lem| and
%          |\endLem| using the current
%          |\theoremstyle| and |\theorem@bodyfont|.
%    \end{itemize}
%    Obviously, all this should happen only if the first argument
%    of |\@xnthm| (i.e.\ \texttt{Lem} in our example) is chosen so
%    as not to conflict with any previously defined commands or
%    environments.  This test is performed by the \LaTeX{} macro
%    |\@ifdefinable|.
%    \begin{macrocode}
\gdef\@xnthm#1#2[#3]{\expandafter\@ifdefinable\csname #1\endcsname
%    \end{macrocode}
%    Therefore, the first argument of |\@ifdefinable| is the
%    expansion (in the example, |\Lem|) of
%    |\csname||#1\endcsname|.  The second argument is
%    executed only if the test has been completed successfully.
%    \begin{macrocode}
   {%
%    \end{macrocode}
%    Now we define the new counter. The names of the \LaTeX{}
%    macros employed should speak for themselves:
% \changes{v2.2b}{1995/10/07}{Use \cs{@newctr}}
% \changes{v2.2c}{1995/11/23}{Use \cs{@newctr} properly.}
%    \begin{macrocode}
    \@definecounter{#1}\@newctr{#1}[#3]%
%    \end{macrocode}
%    Using |\@newctr| will give a proper error message if the counter
%    in |#3| is not defined.
%    In defining `|\theLem|' we must generate the desired macro
%    name by use of |\expandafter| and |\csname|.
%    \begin{macrocode}
    \expandafter\xdef\csname the#1\endcsname
%    \end{macrocode}
%    An |\xdef| is used in order to make the definition global,
%    and to ensure that it contains the replacement texts of
%    |\@thmcountersep| and |\@thmcounter|.\footnote{These
%    two macros can be defined by the document style.  Their default
%    values produce a `\texttt{.}' as separation and an arabic
%    representation of the number.} However, not everything should be
%    expanded. For example, it saves space to use |\thesection|
%    instead of its---at times---lengthy expansion.
%    \begin{macrocode}
      {\expandafter \noexpand \csname the#3\endcsname
       \@thmcountersep \@thmcounter{#1}}%
%    \end{macrocode}
%    Thus with the defaults of \LaTeX{}, |\theLem| would be
%    replaced by the command sequence |\thesection.\arabic{Lem}|.
%
%    We will now look at the definition of the macro which is executed
%    at the beginning of the actual environment (in our example this
%    macro is |\Lem|). It should be noted that we use an
%    ``|\expandafter| trick'' to expand only certain parts of the
%    replacement text at the time of the definition.
%    \begin{macrocode}
    \def\@tempa{\global\@namedef{#1}}%
    \expandafter \@tempa \expandafter{%
%    \end{macrocode}
%    First, the macro that contains the current definitions of
%    |\@begintheorem| and |\@opargtheorem| should be called
%    up.  The name of this macro---as is already known---has the form
%    |\th@|\meta{theorem\ style}; therefore, it must be called by
%    \begin{macrocode}
      \csname th@\the \theorem@style
%    \end{macrocode}
%    In addition the default theorem font should be changeable, i.e.\
%    we have to insert the contents of |\theorem@bodyfont|. For
%    that reason, we expand even further, beyond |\endcsname|,
%    and thus insert the contents of the \textsf{token} register
%    |\theorem@bodyfont| in the replacement text.
%    \begin{macrocode}
            \expandafter \endcsname \the \theorem@bodyfont
%    \end{macrocode}
%    Now it is time to call the macro |\@thm| which takes over
%    the further processing.  It has two arguments: the current
%    counter name (in our example, \texttt{Lem}), and the text of the
%    label (in our example, \texttt{Lemma}).
%    \begin{macrocode}
     \@thm{#1}{#2}}%
%    \end{macrocode}
%    With this, the `sub-definition' is complete.  The macro
%    |\@endtheorem| ends a theorem environment and is, so far, nothing
%    but an |\endtrivlist|.  (Hence it is defined globally, and not
%    within the theorem styles.\footnote{This has to be changed as
%    soon as theorem styles that change \texttt{\bslash @endtheorem}
%    exist.  In such a case, all existing styles must be changed as
%    well since they will have to reset the macro.}) Therefore, we can
%    set it equivalent to the macro that ends the theorem set (in our
%    example, |\endLem|).  However, if some day theorem styles exist
%    that do change |\@endtheorem|, we would have to use the
%    commented-out line instead.
%    \begin{macrocode}
    \global \expandafter \let \csname end#1\endcsname \@endtheorem
%   \global\@namedef{end#1}{\@endtheorem}%
%    \end{macrocode}
%    With these commands all the required definitions are employed,
%    unless the test |\@ifdefinable| has failed.  Therefore, we
%    end the second argument of this macro and with it the definition
%    of |\@xnthm|.
%    \begin{macrocode}
   }}
%    \end{macrocode}
% \end{macro}
%
% \begin{macro}{\@ynthm}
%    The definition of |\@ynthm| is completely analogous. In
%    this case the new counter that is defined is not reset within
%    another counter; thus the definition of |\the...|
%    is simplified:
%    \begin{macrocode}
\gdef\@ynthm#1#2{\expandafter\@ifdefinable\csname #1\endcsname
   {\@definecounter{#1}%
    \expandafter\xdef\csname the#1\endcsname{\@thmcounter{#1}}%
%    \end{macrocode}
%    The rest of the definition corresponds literally to that of
%    |\@xnthm|:
%    \begin{macrocode}
    \def\@tempa{\global\@namedef{#1}}\expandafter \@tempa
     \expandafter{\csname th@\the \theorem@style \expandafter
     \endcsname \the\theorem@bodyfont \@thm{#1}{#2}}%
    \global \expandafter \let \csname end#1\endcsname \@endtheorem}}
%    \end{macrocode}
% \end{macro}
%
% \begin{macro}{\@othm}
%    The definition of |\@othm| does not contain anything new.
%    \begin{macrocode}
\gdef\@othm#1[#2]#3{%
%    \end{macrocode}
% \changes{v2.2b}{95/10/07}{Build in check for counter defined}
%    We do not define a new counter but instead use one that has
%    already been defined.  Thus the only definition we need is that
%    of this pseudo-counter (i.e.\ |\the|\meta{env.\ name}).
%    First we check if |#2| corresponds to a known counter name.
%    \begin{macrocode}
  \expandafter\ifx\csname c@#2\endcsname\relax
   \@nocounterr{#2}%
  \else
   \expandafter\@ifdefinable\csname #1\endcsname
   {\expandafter \xdef \csname the#1\endcsname
     {\expandafter \noexpand \csname the#2\endcsname}%
%    \end{macrocode}
%    All other parts of the definition can be adopted from
%    |\@xnthm|. We have to remember, though, that in this case
%    the name of the current counter and the theorem label have moved
%    to the second and third arguments.
%    \begin{macrocode}
    \def\@tempa{\global\@namedef{#1}}\expandafter \@tempa
     \expandafter{\csname th@\the \theorem@style \expandafter
     \endcsname \the\theorem@bodyfont \@thm{#2}{#3}}%
    \global \expandafter \let \csname end#1\endcsname \@endtheorem}%
  \fi}
%    \end{macrocode}
% \end{macro}
%
% \subsection{Macros that are employed in a {\envfont theorem}
%    environment}
%
% \begin{macro}{\@thm}
%    The macro |\@thm| has to increase the current counter.  Then,
%    depending on whether the environment has (or does not have) an
%    optional argument, it has to branch into either |\@begintheorem|
%    or |\@opargtheorem|.
%    \begin{macrocode}
\gdef\@thm#1#2{\refstepcounter{#1}%
%    \end{macrocode}
%    Now we start a {\envfont trivlist} environment, and give
%    |\@topsep| and |\@topsepadd| the values of the \textsf{skip}
%    registers |\theorempreskipamount| and |\theorempostskipamount|.
%    The value in |\@topsep| is the vertical space that is inserted by
%    the first (and only) |\item| in our |\trivlist| whilst
%    |\@topsepadd| is inserted by |\@endparenv| at the end of that
%    {\envfont trivlist} environment.  By using these registers, we
%    obtain the desired space around a {\envfont theorem} environment.
%    \begin{macrocode}
   \trivlist
   \@topsep \theorempreskipamount               % used by first \item
   \@topsepadd \theorempostskipamount           % used by \@endparenv
%    \end{macrocode}
%    Now we have to test whether an optional argument has been given.
% \changes{v2.0d}{89/07/18}{Forgotten percent added.}
%    \begin{macrocode}
   \@ifnextchar [%
%    \end{macrocode}
%    If there is an optional argument, we will call |\@ythm|, and
%    move the arguments read back into the input stream.
%    \begin{macrocode}
   {\@ythm{#1}{#2}}%
%    \end{macrocode}
%    If not, we call |\@begintheorem|. Its first argument is the
%    name of the theorem set (hence the second argument of
%    |\@thm|). Its second argument is the macro that produces the
%    current number.
%    \begin{macrocode}
   {\@begintheorem{#2}{\csname the#1\endcsname}\ignorespaces}}
%    \end{macrocode}
% \end{macro}
%
% \begin{macro}{\@xthm}
% \begin{macro}{\@ythm}
%    Both these macros were originally called by |\@thm|.  We do
%    not need |\@xthm| anymore, hence we reset it to
%    |\relax|. The definition of |\@ythm| has not changed at
%    all from its definition in \LaTeX. In order to make the macros
%    easier to understand, we will
%    nevertheless present it (commented out).
%    \begin{macrocode}
\global\let\@xthm\relax
% \def\@ythm#1#2[#3]{\@opargbegintheorem{#2}{\csname
%           the#1\endcsname}{#3}\ignorespaces}
%</package>
%    \end{macrocode}
%    The primitive |\ignorespaces| in |\@ythm| and
%    |\@thm| is needed to remove the spaces between the
%    |\begin{...}| and the actual text.
% \end{macro}
% \end{macro}
%
%
% \subsection{Definition of the theorem substyles}
%
% As already pointed out, the theorem substyles, defined below, are
% only loaded when necessary.  Note that all these substyles, except
% \texttt{plain}, have |\slshape| as the default body font.
%
% \subsubsection{The \texttt{plain} style}
%
% \label{dok:th@plain} As the following macros use |@|, we have
% to locally set the |\catcode| of this symbol to ``letter''.
% This happens within a group, so that we do not have to worry about
% which |\catcode| that symbol had before.
%    \begin{macrocode}
%<*thp>
\begingroup \makeatletter
%    \end{macrocode}
% Since we are now within a group, we must make all definitions
% globally.  First we make sure that \texttt{theorem.sty} is loaded.
% This will allow us to use this file as a document style option
% without having to call \texttt{theorem} itself as an option.  At the
% same time, we assure that at least version 2 is loaded, since
% |\theorem@style| was not defined in earlier versions.
%    \begin{macrocode}
\@ifundefined{theorem@style}{\input{theorem.sty}}{}
\ProvidesFile{thp.sty}
             [\FMithmInfo]
%    \end{macrocode}
%
%
% \begin{macro}{\th@plain}
% \begin{macro}{\@begintheorem}
% \begin{macro}{\@opargbegintheorem}
%    |\theoremstyle{plain}| corresponds to the original
%    definition, except that the distances to the surrounding text are
%    determined by the parameters |\theorempreskipamount| and
%    |\theorempostskipamount|.
%    First we set the default body font.
%    \begin{macrocode}
\gdef\th@plain{\normalfont\itshape
%    \end{macrocode}
%    Then we define |\@begintheorem| and
%    |\@opargbegintheorem|. These two macros define how the
%    header of a theorem is typeset.  |\@opargbegintheorem| will
%    be called if a \textsf{theorem} environment with an optional
%    argument is encountered; otherwise, the header is constructed by
%    calling |\@begintheorem|.  If one of these macros is
%    executed, we are within a \textsf{trivlist} environment started by
%    |\@thm|.  So the theorem header is produced with an
%    |\item| command.
%
%    Instead of specifying the header font directly, all standard
%    theorem styles use the |\theorem@headerfont| macro to allow
%    customization.  The extra space (|\labelsep|) is necessary
%    because of problems in the \textsf{trivlist} environment.
%    \begin{macrocode}
  \def\@begintheorem##1##2{%
        \item[\hskip\labelsep \theorem@headerfont ##1\ ##2]}%
%    \end{macrocode}
%    The definition of |\@opargbegintheorem| is completely
%    analogous. The only difference is the fact that there exists a
%    third argument (which is the optional parameter of the
%    environment and contains additional information about the
%    theorem).  Customarily we enclose it in parentheses.
%    \begin{macrocode}
\def\@opargbegintheorem##1##2##3{%
   \item[\hskip\labelsep \theorem@headerfont ##1\ ##2\ (##3)]}}
%    \end{macrocode}
% \end{macro}
% \end{macro}
% \end{macro}
%
% \noindent
% We conclude with an |\endgroup| to restore the
% |\catcode| of |@|.
%    \begin{macrocode}
\endgroup
%</thp>
%    \end{macrocode}
%
%
% \subsubsection{The \texttt{break} style}
%
% This style option is stored in the file |thb.sty|.
% For the next two lines see the documentation for |\th@plain| on
% page~\pageref{dok:th@plain}.
%    \begin{macrocode}
%<*thb>
\begingroup \makeatletter
\@ifundefined{theorem@style}{\input{theorem.sty}}{}
\ProvidesFile{thb.sty}
             [\FMithmInfo]
%    \end{macrocode}
%
% \begin{macro}{\th@break}
%    |\theoremstyle{break}| produces a line break after the name
%    of the theorem.  The font is |\slshape|.  Hence, we define
%    |\th@break| as follows:
%    \begin{macrocode}
\gdef\th@break{\normalfont\slshape
  \def\@begintheorem##1##2{\item[%
%    \end{macrocode}
%    We run into the following problem: it is not possible to create
%    the header with |\item[|\meta{title}|]| and then start
%    a new line by, for example, |\mbox{}\\|. Such a definition
%    will fail whenever a list environment follows immediately.  With
%    the above construction, the |\mbox{}| causes the switch
%    |@inlabel| (cf.\ definition of |\list| and
%    |\trivlist| in \cite{latex:source}) to be set to \texttt{false}
%    and so the following list will insert additional vertical
%    space (|\topskip|). This is quite annoying. Therefore, we
%    create the line break within the |\item|. In order to ensure
%    that the text will begin at the proper position in the following
%    line, we simply pretend that the label does not take any
%    room.\footnote{This will lead to problems whenever very high
%    symbols occurring in the line tower into the
%    heading. So, something else has to be done here sometime.}
%    \begin{macrocode}
       \rlap{\vbox{\hbox{\hskip \labelsep\theorem@headerfont ##1\ ##2}%
                   \hbox{\strut}}}]}%
%    \end{macrocode}
%    Again, the definition of |\@opargbegintheorem| is
%    completely analogous.
%    \begin{macrocode}
\def\@opargbegintheorem##1##2##3{%
  \item[\rlap{\vbox{\hbox{\hskip \labelsep \theorem@headerfont
                     ##1\ ##2\ (##3)}%
                    \hbox{\strut}}}]}}
\endgroup
%</thb>
%    \end{macrocode}
% \end{macro}
%
%
% \subsubsection{The \texttt{changebreak} style}
%
%    \begin{macrocode}
% This style option is stored in the file |thcb.sty|.
%    \begin{macrocode}
%<*thcb>
\begingroup \makeatletter
\@ifundefined{theorem@style}{\input{theorem.sty}}{}
\ProvidesFile{thcb.sty}
             [\FMithmInfo]
%    \end{macrocode}
%
% \begin{macro}{\th@changebreak}
%    The \texttt{change-break} theorem style is like \texttt{break}
%    but with interchange of theorem name and theorem number.  Thus we
%    define |\th@changebreak| as follows:
%    \begin{macrocode}
\gdef\th@changebreak{\normalfont\slshape
  \def\@begintheorem##1##2{\item
       [\rlap{\vbox{\hbox{\hskip\labelsep \theorem@headerfont ##2\ ##1}%
                    \hbox{\strut}}}]}%
\def\@opargbegintheorem##1##2##3{%
  \item[\rlap{\vbox{\hbox{\hskip\labelsep \theorem@headerfont
                     ##2\ ##1\ (##3)}%
                    \hbox{\strut}}}]}}
\endgroup
%</thcb>
%    \end{macrocode}
% \end{macro}
%
%
% \subsubsection{The \texttt{change} style}
%
% This style option is stored in the file |thc.sty|.
%    \begin{macrocode}
%<*thc>
\begingroup \makeatletter
\@ifundefined{theorem@style}{\input{theorem.sty}}{}
\ProvidesFile{thc.sty}
             [\FMithmInfo]
%    \end{macrocode}
%
% \begin{macro}{\th@change}
%    The \texttt{change} theorem style corresponds to the
%    \texttt{change break} style without a linebreak after the header.
%    To say it in another way, it's the same as the \texttt{plain}
%    style but with number and name interchanged and |\slshape| as the
%    default font.
%    \begin{macrocode}
\gdef\th@change{\normalfont\slshape
  \def\@begintheorem##1##2{\item
       [\hskip\labelsep \theorem@headerfont ##2\ ##1]}%
\def\@opargbegintheorem##1##2##3{%
  \item[\hskip\labelsep \theorem@headerfont ##2\ ##1\ (##3)]}}
\endgroup
%</thc>
%    \end{macrocode}
% \end{macro}
%
%
% \subsubsection{The \texttt{marginbreak} style}
%
% This style option is the one used most often at Mainz.  It is saved
% in the file |thmb.sty|.
%    \begin{macrocode}
%<*thmb>
\begingroup \makeatletter
\@ifundefined{theorem@style}{\input{theorem.sty}}{}
\ProvidesFile{thmb.sty}
             [\FMithmInfo]
%    \end{macrocode}
%
% \begin{macro}{\th@marginbreak}
%    The \texttt{margin break} style is nearly the same as the
%    \texttt{change break} style. The only difference is the placement
%    of the theorem number.  We use |\llap| to place it in the left
%    margin.
%
%    In this style |\labelsep| denotes the separation between the
%    number and the text.
%    \begin{macrocode}
\gdef\th@marginbreak{\normalfont\slshape
  \def\@begintheorem##1##2{\item
       [\rlap{\vbox{\theorem@headerfont
                    \hbox{\llap{##2}\hskip\labelsep ##1}%
                    \hbox{\strut}}}]}%
\def\@opargbegintheorem##1##2##3{%
  \item[\rlap{\vbox{\theorem@headerfont
                    \hbox{\llap{##2}\hskip\labelsep ##1\ (##3)}%
                    \hbox{\strut}}}]}}
\endgroup
%</thmb>
%    \end{macrocode}
% \changes{v2.1c}{90/02/17}{`endgroup outside macrocode env.}
% \end{macro}
%
%
%
% \subsubsection{The \texttt{margin} style}
%
% This style option is stored in the file |thm.sty|.
%    \begin{macrocode}
%<*thm>
\begingroup \makeatletter
\@ifundefined{theorem@style}{\input{theorem.sty}}{}
\ProvidesFile{thm.sty}
             [\FMithmInfo]
%    \end{macrocode}
%
% \begin{macro}{\th@margin}
%    Again this is only a variant of the theorem styles described
%    above without any new ideas.
%    \begin{macrocode}
\gdef\th@margin{\normalfont\slshape
  \def\@begintheorem##1##2{\item
       [\theorem@headerfont \llap{##2}\hskip\labelsep ##1]}%
\def\@opargbegintheorem##1##2##3{%
  \item[\theorem@headerfont \llap{##2}\hskip\labelsep ##1\ (##3)]}}
\endgroup
%</thm>
%    \end{macrocode}
% \end{macro}
%
%
%
%
% \subsection{Final Definitions}
%
% \begin{macro}{\theorempreskipamount}
% \begin{macro}{\theorempostskipamount}
%    The \textsf{skip} parameters that regulate the vertical empty space
%    before and after the {\envfont theorem} environment have
%    to be allocated as well.
% \changes{v2.0f}{89/07/20}{Unnecessary `global' removed.}
%    \begin{macrocode}
%<*package>
\newskip\theorempreskipamount
\newskip\theorempostskipamount
%    \end{macrocode}
%    Since we have used the same values for all theorem sets, we now can
%    assign them.
%    \begin{macrocode}
\global\setlength\theorempreskipamount{12pt plus 5pt minus 3pt}
\global\setlength\theorempostskipamount{8pt plus 3pt minus 1.5pt}
%    \end{macrocode}
% \end{macro}
% \end{macro}
%
%
% \begin{macro}{\@endtheorem}
%    The same holds for the macro |\@endtheorem|, which ends a
%    {\envfont theorem} environment. Since it is the same for all
%    theorem sets, it is removed from the macros |\th@|\meta{style}.
%    It simply ends the {\envfont trivlist} environment, which was
%    begun in |\@thm|.
%    \begin{macrocode}
\global\let\@endtheorem=\endtrivlist
%    \end{macrocode}
% \end{macro}
%
%
%
% \begin{macro}{\@preamblecmds}
%    All macros defined above are to be used only in the preamble.
%    Therefore, we insert them in |\@preamblecmds| which will disable
%    them at begin document. This is done by the internal \LaTeXe{}
%    command |\@onlypreamble|.
% \changes{v2.0g}{89/09/19}{`theoremheaderfont and `theorembodyfont
%    added.}
% \changes{v2.2a}{94/02/03}{Use \LaTeXe{} interface}
%    \begin{macrocode}
\@onlypreamble\@xnthm
\@onlypreamble\@ynthm
\@onlypreamble\@othm
\@onlypreamble\newtheorem
\@onlypreamble\theoremstyle
\@onlypreamble\theorembodyfont
\@onlypreamble\theoremheaderfont
%    \end{macrocode}
% \end{macro}
%
% Finally we declare the \texttt{plain} theorem style to be the default.
% \changes{v2.1b}{90/06/04}{Fixed missing `item by declaring default
%    style.}
%    \begin{macrocode}
\theoremstyle{plain}
%</package>
%    \end{macrocode}
%
% \Finale

\endinput
}{}
\ProvidesFile{thb.sty}
             [\FMithmInfo]
%    \end{macrocode}
%
% \begin{macro}{\th@break}
%    |\theoremstyle{break}| produces a line break after the name
%    of the theorem.  The font is |\slshape|.  Hence, we define
%    |\th@break| as follows:
%    \begin{macrocode}
\gdef\th@break{\normalfont\slshape
  \def\@begintheorem##1##2{\item[%
%    \end{macrocode}
%    We run into the following problem: it is not possible to create
%    the header with |\item[|\meta{title}|]| and then start
%    a new line by, for example, |\mbox{}\\|. Such a definition
%    will fail whenever a list environment follows immediately.  With
%    the above construction, the |\mbox{}| causes the switch
%    |@inlabel| (cf.\ definition of |\list| and
%    |\trivlist| in \cite{latex:source}) to be set to \texttt{false}
%    and so the following list will insert additional vertical
%    space (|\topskip|). This is quite annoying. Therefore, we
%    create the line break within the |\item|. In order to ensure
%    that the text will begin at the proper position in the following
%    line, we simply pretend that the label does not take any
%    room.\footnote{This will lead to problems whenever very high
%    symbols occurring in the line tower into the
%    heading. So, something else has to be done here sometime.}
%    \begin{macrocode}
       \rlap{\vbox{\hbox{\hskip \labelsep\theorem@headerfont ##1\ ##2}%
                   \hbox{\strut}}}]}%
%    \end{macrocode}
%    Again, the definition of |\@opargbegintheorem| is
%    completely analogous.
%    \begin{macrocode}
\def\@opargbegintheorem##1##2##3{%
  \item[\rlap{\vbox{\hbox{\hskip \labelsep \theorem@headerfont
                     ##1\ ##2\ (##3)}%
                    \hbox{\strut}}}]}}
\endgroup
%</thb>
%    \end{macrocode}
% \end{macro}
%
%
% \subsubsection{The \texttt{changebreak} style}
%
%    \begin{macrocode}
% This style option is stored in the file |thcb.sty|.
%    \begin{macrocode}
%<*thcb>
\begingroup \makeatletter
\@ifundefined{theorem@style}{% \iffalse meta-comment
%
% Copyright (C) 1993-2024
%
% The LaTeX Project and any individual authors listed elsewhere
% in this file.
%
% This file is part of the Standard LaTeX `Tools Bundle'.
% -------------------------------------------------------
%
% It may be distributed and/or modified under the
% conditions of the LaTeX Project Public License, either version 1.3c
% of this license or (at your option) any later version.
% The latest version of this license is in
%    https://www.latex-project.org/lppl.txt
% and version 1.3c or later is part of all distributions of LaTeX
% version 2005/12/01 or later.
%
% The list of all files belonging to the LaTeX `Tools Bundle' is
% given in the file `manifest.txt'.
%
% \fi
%
%%
%
%
\def\FMithmInfo{2023/07/05 v2.2c Theorem extension package (FMi)}
%
% \ProvidesFile{theorem.dtx}[\FMithmInfo]
%\iffalse   % this is a METACOMMENT !
%%
%% Package `theorem' to use with LaTeX2e
%% Copyright (C) 1989-1995 Frank Mittelbach, all rights reserved.
%<+package>\@ifundefined{theorem@style}{}{\endinput}
%<+package>\NeedsTeXFormat{LaTeX2e}[1995/06/01]
%<+package>\ProvidesPackage{theorem}
%<+package>         [\FMithmInfo]
%
% \fi
%
% \changes{v1.0}{87/05/07}{First draft.}
% \changes{v1.0a}{87/05/07}{`thistheoremstyle' added.}
% \changes{v1.1a}{87/07/29}{New concept: redefine the internal
%                          \protect\LaTeX{} commands `@xnthm' etc.}
% \changes{v1.1b}{87/07/31}{`theoremfont' used to specify font.}
% \changes{v1.2a}{88/01/08}{`@endtheorem' now globally defined.}
% \changes{v1.2b}{88/01/11}{`newtheorem' only allowed in the preamble.}
% \changes{v1.3a}{88/02/14}{`theoremfont' now a toks register.}
%
% \changes{v2.0a}{88/11/16}{New implementation and new documentation.}
% \changes{v2.0a}{88/11/16}{No layout changes but internal modifications
%                           old substyles won't work.}
% \changes{v2.0b}{89/01/30}{`global' used for all definitions.}
% \changes{v2.0c}{89/06/15}{Documentation translated to English.
%                            Thanks to LLL and Nina}
% \changes{v2.1a}{90/02/15}{Support for new font selection.}
% \changes{v2.2a}{94/02/03}{Cleanup for LaTeX2e}
%
% \DoNotIndex{\@endparenv,\@ifdefinable,\@ifnextchar,\@ifundefined}
% \DoNotIndex{\@namedef,\@nameuse,\@spaces,\@tempa,\@warning}
% \DoNotIndex{\ }
% \DoNotIndex{\begingroup,\bfseries}
% \DoNotIndex{\csname}
% \DoNotIndex{\def,\do,\docdate}
% \DoNotIndex{\endcsname,\endgroup,\endinput,\endtrivlist,\expandafter}
% \DoNotIndex{\filedate,\fileversion}
% \DoNotIndex{\gdef,\global}
% \DoNotIndex{\hbox,\hskip}
% \DoNotIndex{\ignorespaces,\input,\itshape,\item}
% \DoNotIndex{\labelsep,\let,\llap}
% \DoNotIndex{\makeatletter}
% \DoNotIndex{\noexpand}
% \DoNotIndex{\refstepcounter,\relax,\rlap}
% \DoNotIndex{\setlength,\slshape,\space,\string,\strut}
% \DoNotIndex{\the,\trivlist,\typeout}
% \DoNotIndex{\vbox}
% \DoNotIndex{\xdef}
%
% \def\deflabel#1{\ttfamily #1\hfill}
% \def\deflist#1{\begin{list}{}{\settowidth\labelwidth{\ttfamily #1}%
%                               \setlength\leftmargin\labelwidth
%                               \addtolength\leftmargin\labelsep
%                               \let\makelabel\deflabel}}
% \def\enddeflist{\end{list}}
% \def\envfont{\normalfont\ttfamily}
%
% \MakeShortVerb{\|}
%
% \newcommand{\TUB}{{\sl TUGboat\/}}
%
% \GetFileInfo{theorem.dtx}
% \title{An Extension of the
%        \LaTeX{} theorem environment\thanks{This file
%        has version number \fileversion, last
%        revised \filedate.}}
% \author{Frank Mittelbach}
%
% \MaintainedByLaTeXTeam{tools}
% \maketitle
%
% \begin{abstract}
%    The macros described in this paper yield an extension of the
%    \LaTeX{} theorem mechanism. It is designed to satisfy the
%    different requirements of various journals. Thus, the
%    layout of the ``theorems'' can be manipulated by determining a
%    ``style''. This article describes not only the use, but
%    also the definition, of the necessary macros.
% \end{abstract}
%
% \section*{Preface to version 2.2}
%
% For \LaTeXe{} this package did not need any fundamental changes. I
% only modified the messages generated so that theorem layout styles
% will show up with the |\listfiles| command and cleaned the section
% on the New Font Selection Scheme since this is now included in
% \LaTeX.
%
% \section*{Preface to version 2.1}
%
% This version is identical to 2.0g described in \TUB\ 10\#3 except
% for some internal defaults which are now set depending on the used
% font selection scheme.
%
% This was done to avoid unpleasant surprises if the new font selection
% scheme is in force. For further details see
% section~\ref{sec:newfonts} and \cite{bk:GMS94}.
%
% \section{Introduction}
%
% For our purposes here, ``theorems'' are labelled enunciations,
% often set off from the main text by extra space and a font change.
% Theorems, corollaries, conjectures, definitions, and remarks
% are all instances of ``theorems''.  The ``header'' of these structures
% is composed of a label (such as \textsc{Theorem} or \textsc{Remark})
% and a number which serializes an item in the sequence of items
% with the same label.
%
% Shortly after the introduction of \LaTeX{} at the Fachbereich
% Mathematik in Mainz, the desire to manipulate the layout of
% ``theorems'' arose.  In Mainz, the following two conventions came
% into general use:
% \begin{enumerate}
%   \item The number of the theorem is shown in the margin.
%   \item There is a line break at
%         the end of the theorem header.
% \end{enumerate}
%
% Additionally, some journals require different formats which depend on
% the ``sort of theorem'': e.g.\ often remarks and definitions are set
% in |\upshape|, while |\itshape| is employed for main theorems.
%
% Confronted with these requirements, a theorem environment was
% developed in Mainz which allows separate determination of the
% layout of the ``theorems sets'', comparable to |\pagestyle|.
%
%
% \section{The user interface}
% \subsection{Defining new theorem sets}
%
% \DescribeMacro\newtheorem
% As in the original \LaTeX{} version, the command |\newtheorem|
% defines a new ``theorem set'' or ``theorem-like structure''.
% Two required arguments name the new environment and give the text
% to be typeset with each instance of the new ``set'', while
% an optional argument determines how
% the ``set'' is enumerated:
% \begin{description}
%    \item[\ttfamily \bslash newtheorem\{foo\}\{bar\}]
%       The theorem set {\envfont foo} (whose name is \texttt{bar})
%       uses its own counter.
%    \item[\ttfamily \bslash newtheorem\{foo2\}{[foo]}\{bar2\}]
%       The theorem set {\envfont foo2} (printed name \texttt{bar2})
%       uses the same counter as the theorem set \texttt{foo}.
%    \item[\ttfamily \bslash newtheorem\{foo3\}\{bar3\}{[section]}]
%       The theorem set {\envfont foo3} (printed name \texttt{bar3}) is
%       enumerated within the counter \texttt{section}, i.e.\ with every
%       new |\section| the enumeration begins again with 1, and
%       the enumeration is composed from the section-number and the
%       theorem counter itself.
% \end{description}
%
% \DescribeMacro\theoremstyle
% Additionally, the command |\theoremstyle| can define the layout
% of various, or all, theorem sets.  It should be noted that any
% theorem set defined by |\newtheorem| is typeset in the
% |\theoremstyle| that is current at the time of the definition.
% Thus, the following
% \begin{verbatim}
%    \theoremstyle{break}        \newtheorem{Cor}{Corollary}
%    \theoremstyle{plain}        \newtheorem{Exa}{Example}[section]
%\end{verbatim}                             ^^A No blank at line start
% leads to the result that the set {\envfont Cor} is formatted in the
% style \texttt{break}, while the set {\envfont Exa} and all the
% following ones are formatted in the style \texttt{plain}, unless
% another |\theoremstyle| follows.  Since the definitions installed by
% |\newtheorem| are global, one also can limit |\theoremstyle| locally
% by grouping braces.
%
% \DescribeMacro\theorembodyfont
% The choice of the font for the theorem body is completely
% independent of the chosen |\theoremstyle|; this has proven to
% be very advantageous.  For example,
% \begin{verbatim}
%   {\theorembodyfont{\upshape}          \newtheorem{Rem}{Remark}}
%\end{verbatim}
% defines a theorem set {\envfont Rem}, which will be set in
% |\upshape| in the current layout (which in our example is
% \texttt{plain}).  As with |\theoremstyle|, the |\theorembodyfont|
% chosen is that current at the time of |\newtheorem|.  If
% |\theorembodyfont| is not specified or one defines
% |\theorembodyfont{}|, then the font used will be that defined by the
% |\theoremstyle|.
%
% \DescribeMacro\theoremheaderfont It is also possible to customize
% the font used for the theorem headers.  This is, however, a global
% declaration, and therefore there should be at most one
% |\theoremheaderfont| declaration in the preamble.\footnote{If it is
% actually necessary to have different header fonts, one has to define
% new theorem styles (substituting the desired font) or specify the
% information directly in the \texttt{\bslash newtheorem} declaration
% (the unclean variant).}
%
% \DescribeMacro\theorempreskipamount
% \DescribeMacro\theorempostskipamount Two additional parameters
% affect the vertical space around the theorem environments:
% |\theorempreskipamount| and |\theorempostskipamount| define,
% respectively, the spacing before and after such an environment.
% These parameters apply for all theorem sets and can be manipulated
% with the ordinary length macros.  They are rubber lengths
% (`\textsf{skips}'), and therefore can contain \texttt{plus} and
% \texttt{minus} parts.
%
% Since the definition of theorem sets should---most sensibly---be
% placed in the preamble, we only allow installation there.
% It is therefore possible to release the memory used here after
% |\begin{document}|, in order to make room for other
% applications.
%
%
% \subsection{Existing theorem styles}
%
% The following theorem styles exist to date:
% \begin{deflist}{marginbreak}
%    \item[plain]
%       This theorem style emulates the original \LaTeX{} definition,
%       except that additionally the parameters
%       |\theorem...skipamount| are used.
%    \item[break]
%       In this style, the theorem header is followed by a line
%       break.
%    \item[marginbreak]
%       The theorem number is set in the margin, and there is a line
%       break as in \texttt{break}.
%    \item[changebreak]
%       Like \texttt{break}, but with header number and text
%       interchanged.
%    \item[change]
%       Header number and text are interchanged, without a line break.
%    \item[margin]
%       The number is set in the left margin, without a line break.
% \end{deflist}
% All styles (except \texttt{plain}) select |\slshape| as the default
% |\theorembodyfont|.
%
%
% \subsection{Examples}
%
% Given the above theorem sets {\envfont Cor}, {\envfont Exa}
% and {\envfont Rem},
% suppose that the preamble also contains the declarations:
% \begin{verbatim}
%    \theoremstyle{marginbreak}   \newtheorem{Lem}[Cor]{Lemma}
%    \theoremstyle{change}
%    \theorembodyfont{\itshape}        \newtheorem{Def}[Cor]{Definition}
%
%    \theoremheaderfont{\scshape}
%\end{verbatim}
% Then the following are some typical examples of the typeset output
% resulting from their use.
%
% \begin{Cor}
%    This is a sentence typeset in the theorem environment {\envfont
%    Cor}.
% \end{Cor}
%
% \begin{Exa}
%    This is a sentence typeset in the theorem environment {\envfont
%    Exa}.
% \end{Exa}
%
% \begin{Rem}
%    This is a sentence typeset in the theorem environment {\envfont
%    Rem}.
% \end{Rem}
%
% \begin{Lem}[Ben User]
%    This is a sentence typeset in the theorem environment {\envfont
%    Lem}.
% \end{Lem}
%
% \begin{Def}[Very impressive Definition]
%    This is a sentence typeset in the theorem environment {\envfont
%    Def}.
% \end{Def}
%
% The last two examples show the effect of the optional argument to
% a theorem environment (it is the text typeset in parentheses).
%
%
% \section{Special Considerations}
% \label{sec:newfonts}
%
% Theoremheader and body are implemented as a unit. This means that
% the |\theoremheaderfont|  will inherit characteristics of the
% |\theorembodyfont| in \LaTeXe. Thus, if for example
% |\theorembodyfont| is |\itshape| and |\theoremheaderfont|
% is |\bfseries| the font selected for the header will have the
% characteristics `bold extended italic'. If this is not desired one
% should set the |\theoremheaderfont| to something like
% \begin{verbatim}
%      \theoremheaderfont{\normalfont\bfseries}
%\end{verbatim}
% i.e.\ supplying all necessary font information explicitly.
%
%
%
% \section{Acknowledgements}
%
% The publication of this set of macros was only possible with the
% help of Christina Busse (translating the manuscript into English),
% Joachim Pense (playing the r\^ole of typist), Chris Rowley
% (looking everything over) and many others providing useful
% suggestions.
%
% \MaybeStop{%
%    \begin{thebibliography}{9}
%       \bibitem{bk:GMS94} \textsc{M.~Goossens}, \textsc{F.~Mittelbach}
%         and \textsc{A.~Samarin}.
%         \newblock The \LaTeX{} Companion.
%         \newblock
%         Addison-Wesley, Reading, Massachusetts, 1994.
%       \bibitem{latex:source} \textsc{Lamport, Leslie.} \newblock
%                               \texttt{latex.tex}, version 2.09, date
%                               Feb. 1990.
%    \end{thebibliography}
%    \PrintIndex
%    \PrintChanges}
%
%
% \section{The documentation driver file}
%
% The next bit of code contains the documentation driver file for
% \TeX{}, i.e., the file that will produce the documentation you are
% currently reading. It will be extracted from this file by the
% \texttt{docstrip} program. Since it is the first code in the file
% one can alternatively process this file directly with \LaTeXe{} to
% obtain the documentation.
%
%    \begin{macrocode}
%<*driver>
\documentclass{ltxdoc}

\usepackage{theorem}

%<+driver>% The next few lines define theorem sets which are used
%<+driver>% in the example section of the documentation.

\theoremstyle{break}       \newtheorem{Cor}{Corollary}
\theoremstyle{plain}       \newtheorem{Exa}{Example}[section]
{\theorembodyfont{\upshape}\newtheorem{Rem}{Remark}}
\theoremstyle{marginbreak} \newtheorem{Lem}[Cor]{Lemma}
\theoremstyle{change}
\theorembodyfont{\itshape} \newtheorem{Def}[Cor]{Definition}

\theoremheaderfont{\scshape}

\RecordChanges

\begin{document}
   \DocInput{theorem.dtx}
\end{document}
%</driver>
%    \end{macrocode}
%
% \section{Definition of the Macros}
%
% If the file has been loaded before, we abort immediately. If not the
% package announces itself (this is actually done at the very top of
% the file---the way it is done isn't good style so don't copy it).
% \changes{v2.0e}{89/07/19}{Spaces removed from `typeout'.}
% \changes{v2.2b}{95/11/19}{Announce moved to top of file}
%    \begin{macrocode}
%<*package>
%\@ifundefined{theorem@style}{}{\endinput}
%\def\FMithmInfo{1995/11/19 v2.2b Theorem extension package (FMi)}
%\ProvidesPackage{theorem}[\FMithmInfo]
%    \end{macrocode}
%
%
% \subsection{Definition of theorem styles and fonts}
%
% All the definitions in this file are done globally to allow
% inputting this file inside a group.
%
% \begin{macro}{\theoremstyle}
%    Before a theorem style can be installed, the chosen style must be
%    known. For that reason, we must test to see that
%    |\th@|\meta{style} is known or, more precisely, that it is
%    different from |\relax|. If the style is not known then
%    |\th@plain| is used.
%    \begin{macrocode}
\gdef\theoremstyle#1{%
   \@ifundefined{th@#1}{\@warning
          {Unknown theoremstyle `#1'. Using `plain'}%
          \theorem@style{plain}}%
%    \end{macrocode}
%    We save the theorem style to be used in the token register
%    |\theorem@style|.
%    \begin{macrocode}
      {\theorem@style{#1}}%
%    \end{macrocode}
%    Now we ``evaluate'' the theorem style: this means, we call the
%    macro |\th@|\meta{style} which will activate the relevant
%    definitions which are contained in a separate file.  This is done
%    in a group to suppress changes to the current font.  This could
%    otherwise pose problems together with the new font selection
%    scheme\footnote{When I printed the original article using the new
%    font selection scheme I ended with a document with slanted
%    typefaces (text headings and all) simply because one of the
%    theorem styles used \texttt{\bslash sl} at toplevel.} if the
%    |\th@|\meta{style} is evaluated a second time.
% \changes{v2.1a}{90/02/15}{Group added around `th@..}
% \changes{v2.1a}{90/02/15}{Reseting of `..@begintheorem removed.}
%    \begin{macrocode}
      \begingroup
        \csname th@\the\theorem@style \endcsname
      \endgroup}
%    \end{macrocode}
% \end{macro}
%
%
% \begin{macro}{\@begintheorem}
% \changes{v2.1a}{90/02/15}{Relaxed at toplevel.}
% \begin{macro}{\@opargbegintheorem}
% \changes{v2.1a}{90/02/15}{Relaxed at toplevel.}
%    We reset |\@begintheorem| and |\@opargbegintheorem| to
%    |\relax| since these commands are no longer necessary at
%    toplevel. This will save a few tokens.
%    \begin{macrocode}
\global\let\@begintheorem\relax
\global\let\@opargbegintheorem\relax
%    \end{macrocode}
% \end{macro}
% \end{macro}
%
% \begin{macro}{\theorem@style}
%    Obviously the \textsf{token} register used above has to be
%    allocated. To assure the utmost compatibility with the
%    original \LaTeX{} definition, we set the default theorem style to
%    \texttt{plain}, which implements the usual \LaTeX{} convention.
% \changes{v2.0f}{89/07/20}{Unnecessary `global' removed.}
%    \begin{macrocode}
\newtoks\theorem@style
\global\theorem@style{plain}
%    \end{macrocode}
% \end{macro}
%
% \begin{macro}{\theorembodyfont}
% \begin{macro}{\theorem@bodyfont}
%    If the |\theorembodyfont| is set by the user then it should not
%    interact with the default font set in the theorem style. When the
%    new font selection is in force this may happen if, for example,
%    the default is |\itshape| and the new |\theorembodyfont| is
%    |\sffamily|. So we add a |\reset@font| command in front of the
%    user definition.
% \changes{v2.0f}{89/07/20}{Unnecessary `global' removed.}
% \changes{v2.0g}{89/09/19}{`theoremfont to `theorembodyfont.}
% \changes{v2.1c}{92/06/22}{`reset@font added and logic changed.}
% \changes{v2.1d}{92/10/28}{Reset to empty if arg is empty}
%    \begin{macrocode}
\gdef\theorembodyfont#1{%
%    \end{macrocode}
%    We check if the argument supplied is empty and if so put nothing
%    into the |\theorem@bodyfont| token register to allow for
%    |\theorembodyfont{}| as a mean of using the default of the
%    current |\theoremstyle|.
%    \begin{macrocode}
   \def\@tempa{#1}%
   \ifx\@tempa\@empty
    \theorem@bodyfont{}%
   \else
    \theorem@bodyfont{\reset@font#1}%
   \fi
   }
\newtoks\theorem@bodyfont
\global\theorem@bodyfont{}
%    \end{macrocode}
% \end{macro}
% \end{macro}
%
%
% \begin{macro}{\theoremheaderfont}
% \changes{v2.0g}{89/09/19}{Macro added.}
%    The font for the theorem headers is handled differently because
%    this definition applies to all theorem styles.
%    \begin{macrocode}
\gdef\theoremheaderfont#1{\gdef\theorem@headerfont{#1}%
%    \end{macrocode}
%    After using the macro once it is redefined to produce an error
%    message.
%    \begin{macrocode}
       \gdef\theoremheaderfont##1{%
        \typeout{\string\theoremheaderfont\space should be used
                 only once.}}}
%    \end{macrocode}
% \end{macro}
%
% \begin{macro}{\theorem@headerfont}
% \changes{v2.0g}{89/09/19}{Macro added.}
%    To set the |\theorem@headerfont| default we first test if
%    the new fontselection scheme is in force.
% \changes{v2.1a}{90/02/15}{Support for new font selection.}
%    \begin{macrocode}
\ifx\upshape\undefined
%    \end{macrocode}
%    If not we define it to expand into |\bfseries|. We don't use
%    |\let| just in case a following style option redefines this
%    macro.
%    \begin{macrocode}
\gdef\theorem@headerfont{\bfseries}
%    \end{macrocode}
%    Otherwise we reset the current shape before calling |\bfseries|.
%    \begin{macrocode}
\else \gdef\theorem@headerfont{\normalfont\bfseries}\fi
%    \end{macrocode}
% \end{macro}
%
% \begin{macro}{\th@plain}
% \begin{macro}{\th@break}
% \begin{macro}{\th@marginbreak}
% \begin{macro}{\th@changebreak}
% \begin{macro}{\th@change}
% \begin{macro}{\th@margin}
%    The different styles are defined in macros such as
%    |\th@plain|.  Since memory space is precious in
%    ``non-Big-versions'', we have to avoid offering too many unused
%    definitions.  Therefore we define these styles in separate files
%    that can be loaded on demand.  Thus the commands themselves only
%    load these files. We use |\@input@|, a \LaTeXe{} internal command
%    that ensures that the file will be listed with |\listfiles|.
% \changes{v2.2a}{94/02/02}{Use `@input@ to load theorem layout files}
%    \begin{macrocode}
\gdef\th@plain{\@input@{thp.sty}}
\gdef\th@break{\@input@{thb.sty}}
\gdef\th@marginbreak{\@input@{thmb.sty}}
\gdef\th@changebreak{\@input@{thcb.sty}}
\gdef\th@change{\@input@{thc.sty}}
\gdef\th@margin{\@input@{thm.sty}}
%    \end{macrocode}
%    This list will be expanded when new styles become
%    available.  For testing,  just append new theorem substyles
%    as document options.
% \end{macro}
% \end{macro}
% \end{macro}
% \end{macro}
% \end{macro}
% \end{macro}
%
%
%
% \subsection{Definition of a new theorem set}
%
% As already pointed out, a new theorem environment can be defined in
% three different ways:
% \begin{verbatim}
%    \newtheorem{Lem}{Lemma}
%    \newtheorem{Lem}{Lemma}[section]
%    \newtheorem{Lem}[Theorem]{Lemma}
%\end{verbatim}
% The function of the macro |\newtheorem| is to recognize these cases
% and then to branch into one of the three macros |\@ynthm|, |\@xnthm|
% or |\@othm|.  This mechanism is adopted unchanged from
% \cite{latex:source}; the essential point here is that, for example,
% in the second case, the arguments \texttt{Lem}, \texttt{Lemma} and
% \texttt{section} are passed over to the macro |\@xnthm|.
%
% We inspect this case first because the others present fewer problems,
% and thus are easily derived from this one.
%
% \begin{macro}{\@xnthm}
%    For our example arguments, the macro |\@xnthm| must fulfill
%    the following:
%    \begin{itemize}
%       \item
%          Define a new \LaTeX-counter `Lem'
%       \item
%          reset this counter within a |\section|
%       \item
%          define the macro |\theLem|
%       \item
%          define the environment macros |\Lem| and
%          |\endLem| using the current
%          |\theoremstyle| and |\theorem@bodyfont|.
%    \end{itemize}
%    Obviously, all this should happen only if the first argument
%    of |\@xnthm| (i.e.\ \texttt{Lem} in our example) is chosen so
%    as not to conflict with any previously defined commands or
%    environments.  This test is performed by the \LaTeX{} macro
%    |\@ifdefinable|.
%    \begin{macrocode}
\gdef\@xnthm#1#2[#3]{\expandafter\@ifdefinable\csname #1\endcsname
%    \end{macrocode}
%    Therefore, the first argument of |\@ifdefinable| is the
%    expansion (in the example, |\Lem|) of
%    |\csname||#1\endcsname|.  The second argument is
%    executed only if the test has been completed successfully.
%    \begin{macrocode}
   {%
%    \end{macrocode}
%    Now we define the new counter. The names of the \LaTeX{}
%    macros employed should speak for themselves:
% \changes{v2.2b}{1995/10/07}{Use \cs{@newctr}}
% \changes{v2.2c}{1995/11/23}{Use \cs{@newctr} properly.}
%    \begin{macrocode}
    \@definecounter{#1}\@newctr{#1}[#3]%
%    \end{macrocode}
%    Using |\@newctr| will give a proper error message if the counter
%    in |#3| is not defined.
%    In defining `|\theLem|' we must generate the desired macro
%    name by use of |\expandafter| and |\csname|.
%    \begin{macrocode}
    \expandafter\xdef\csname the#1\endcsname
%    \end{macrocode}
%    An |\xdef| is used in order to make the definition global,
%    and to ensure that it contains the replacement texts of
%    |\@thmcountersep| and |\@thmcounter|.\footnote{These
%    two macros can be defined by the document style.  Their default
%    values produce a `\texttt{.}' as separation and an arabic
%    representation of the number.} However, not everything should be
%    expanded. For example, it saves space to use |\thesection|
%    instead of its---at times---lengthy expansion.
%    \begin{macrocode}
      {\expandafter \noexpand \csname the#3\endcsname
       \@thmcountersep \@thmcounter{#1}}%
%    \end{macrocode}
%    Thus with the defaults of \LaTeX{}, |\theLem| would be
%    replaced by the command sequence |\thesection.\arabic{Lem}|.
%
%    We will now look at the definition of the macro which is executed
%    at the beginning of the actual environment (in our example this
%    macro is |\Lem|). It should be noted that we use an
%    ``|\expandafter| trick'' to expand only certain parts of the
%    replacement text at the time of the definition.
%    \begin{macrocode}
    \def\@tempa{\global\@namedef{#1}}%
    \expandafter \@tempa \expandafter{%
%    \end{macrocode}
%    First, the macro that contains the current definitions of
%    |\@begintheorem| and |\@opargtheorem| should be called
%    up.  The name of this macro---as is already known---has the form
%    |\th@|\meta{theorem\ style}; therefore, it must be called by
%    \begin{macrocode}
      \csname th@\the \theorem@style
%    \end{macrocode}
%    In addition the default theorem font should be changeable, i.e.\
%    we have to insert the contents of |\theorem@bodyfont|. For
%    that reason, we expand even further, beyond |\endcsname|,
%    and thus insert the contents of the \textsf{token} register
%    |\theorem@bodyfont| in the replacement text.
%    \begin{macrocode}
            \expandafter \endcsname \the \theorem@bodyfont
%    \end{macrocode}
%    Now it is time to call the macro |\@thm| which takes over
%    the further processing.  It has two arguments: the current
%    counter name (in our example, \texttt{Lem}), and the text of the
%    label (in our example, \texttt{Lemma}).
%    \begin{macrocode}
     \@thm{#1}{#2}}%
%    \end{macrocode}
%    With this, the `sub-definition' is complete.  The macro
%    |\@endtheorem| ends a theorem environment and is, so far, nothing
%    but an |\endtrivlist|.  (Hence it is defined globally, and not
%    within the theorem styles.\footnote{This has to be changed as
%    soon as theorem styles that change \texttt{\bslash @endtheorem}
%    exist.  In such a case, all existing styles must be changed as
%    well since they will have to reset the macro.}) Therefore, we can
%    set it equivalent to the macro that ends the theorem set (in our
%    example, |\endLem|).  However, if some day theorem styles exist
%    that do change |\@endtheorem|, we would have to use the
%    commented-out line instead.
%    \begin{macrocode}
    \global \expandafter \let \csname end#1\endcsname \@endtheorem
%   \global\@namedef{end#1}{\@endtheorem}%
%    \end{macrocode}
%    With these commands all the required definitions are employed,
%    unless the test |\@ifdefinable| has failed.  Therefore, we
%    end the second argument of this macro and with it the definition
%    of |\@xnthm|.
%    \begin{macrocode}
   }}
%    \end{macrocode}
% \end{macro}
%
% \begin{macro}{\@ynthm}
%    The definition of |\@ynthm| is completely analogous. In
%    this case the new counter that is defined is not reset within
%    another counter; thus the definition of |\the...|
%    is simplified:
%    \begin{macrocode}
\gdef\@ynthm#1#2{\expandafter\@ifdefinable\csname #1\endcsname
   {\@definecounter{#1}%
    \expandafter\xdef\csname the#1\endcsname{\@thmcounter{#1}}%
%    \end{macrocode}
%    The rest of the definition corresponds literally to that of
%    |\@xnthm|:
%    \begin{macrocode}
    \def\@tempa{\global\@namedef{#1}}\expandafter \@tempa
     \expandafter{\csname th@\the \theorem@style \expandafter
     \endcsname \the\theorem@bodyfont \@thm{#1}{#2}}%
    \global \expandafter \let \csname end#1\endcsname \@endtheorem}}
%    \end{macrocode}
% \end{macro}
%
% \begin{macro}{\@othm}
%    The definition of |\@othm| does not contain anything new.
%    \begin{macrocode}
\gdef\@othm#1[#2]#3{%
%    \end{macrocode}
% \changes{v2.2b}{95/10/07}{Build in check for counter defined}
%    We do not define a new counter but instead use one that has
%    already been defined.  Thus the only definition we need is that
%    of this pseudo-counter (i.e.\ |\the|\meta{env.\ name}).
%    First we check if |#2| corresponds to a known counter name.
%    \begin{macrocode}
  \expandafter\ifx\csname c@#2\endcsname\relax
   \@nocounterr{#2}%
  \else
   \expandafter\@ifdefinable\csname #1\endcsname
   {\expandafter \xdef \csname the#1\endcsname
     {\expandafter \noexpand \csname the#2\endcsname}%
%    \end{macrocode}
%    All other parts of the definition can be adopted from
%    |\@xnthm|. We have to remember, though, that in this case
%    the name of the current counter and the theorem label have moved
%    to the second and third arguments.
%    \begin{macrocode}
    \def\@tempa{\global\@namedef{#1}}\expandafter \@tempa
     \expandafter{\csname th@\the \theorem@style \expandafter
     \endcsname \the\theorem@bodyfont \@thm{#2}{#3}}%
    \global \expandafter \let \csname end#1\endcsname \@endtheorem}%
  \fi}
%    \end{macrocode}
% \end{macro}
%
% \subsection{Macros that are employed in a {\envfont theorem}
%    environment}
%
% \begin{macro}{\@thm}
%    The macro |\@thm| has to increase the current counter.  Then,
%    depending on whether the environment has (or does not have) an
%    optional argument, it has to branch into either |\@begintheorem|
%    or |\@opargtheorem|.
%    \begin{macrocode}
\gdef\@thm#1#2{\refstepcounter{#1}%
%    \end{macrocode}
%    Now we start a {\envfont trivlist} environment, and give
%    |\@topsep| and |\@topsepadd| the values of the \textsf{skip}
%    registers |\theorempreskipamount| and |\theorempostskipamount|.
%    The value in |\@topsep| is the vertical space that is inserted by
%    the first (and only) |\item| in our |\trivlist| whilst
%    |\@topsepadd| is inserted by |\@endparenv| at the end of that
%    {\envfont trivlist} environment.  By using these registers, we
%    obtain the desired space around a {\envfont theorem} environment.
%    \begin{macrocode}
   \trivlist
   \@topsep \theorempreskipamount               % used by first \item
   \@topsepadd \theorempostskipamount           % used by \@endparenv
%    \end{macrocode}
%    Now we have to test whether an optional argument has been given.
% \changes{v2.0d}{89/07/18}{Forgotten percent added.}
%    \begin{macrocode}
   \@ifnextchar [%
%    \end{macrocode}
%    If there is an optional argument, we will call |\@ythm|, and
%    move the arguments read back into the input stream.
%    \begin{macrocode}
   {\@ythm{#1}{#2}}%
%    \end{macrocode}
%    If not, we call |\@begintheorem|. Its first argument is the
%    name of the theorem set (hence the second argument of
%    |\@thm|). Its second argument is the macro that produces the
%    current number.
%    \begin{macrocode}
   {\@begintheorem{#2}{\csname the#1\endcsname}\ignorespaces}}
%    \end{macrocode}
% \end{macro}
%
% \begin{macro}{\@xthm}
% \begin{macro}{\@ythm}
%    Both these macros were originally called by |\@thm|.  We do
%    not need |\@xthm| anymore, hence we reset it to
%    |\relax|. The definition of |\@ythm| has not changed at
%    all from its definition in \LaTeX. In order to make the macros
%    easier to understand, we will
%    nevertheless present it (commented out).
%    \begin{macrocode}
\global\let\@xthm\relax
% \def\@ythm#1#2[#3]{\@opargbegintheorem{#2}{\csname
%           the#1\endcsname}{#3}\ignorespaces}
%</package>
%    \end{macrocode}
%    The primitive |\ignorespaces| in |\@ythm| and
%    |\@thm| is needed to remove the spaces between the
%    |\begin{...}| and the actual text.
% \end{macro}
% \end{macro}
%
%
% \subsection{Definition of the theorem substyles}
%
% As already pointed out, the theorem substyles, defined below, are
% only loaded when necessary.  Note that all these substyles, except
% \texttt{plain}, have |\slshape| as the default body font.
%
% \subsubsection{The \texttt{plain} style}
%
% \label{dok:th@plain} As the following macros use |@|, we have
% to locally set the |\catcode| of this symbol to ``letter''.
% This happens within a group, so that we do not have to worry about
% which |\catcode| that symbol had before.
%    \begin{macrocode}
%<*thp>
\begingroup \makeatletter
%    \end{macrocode}
% Since we are now within a group, we must make all definitions
% globally.  First we make sure that \texttt{theorem.sty} is loaded.
% This will allow us to use this file as a document style option
% without having to call \texttt{theorem} itself as an option.  At the
% same time, we assure that at least version 2 is loaded, since
% |\theorem@style| was not defined in earlier versions.
%    \begin{macrocode}
\@ifundefined{theorem@style}{\input{theorem.sty}}{}
\ProvidesFile{thp.sty}
             [\FMithmInfo]
%    \end{macrocode}
%
%
% \begin{macro}{\th@plain}
% \begin{macro}{\@begintheorem}
% \begin{macro}{\@opargbegintheorem}
%    |\theoremstyle{plain}| corresponds to the original
%    definition, except that the distances to the surrounding text are
%    determined by the parameters |\theorempreskipamount| and
%    |\theorempostskipamount|.
%    First we set the default body font.
%    \begin{macrocode}
\gdef\th@plain{\normalfont\itshape
%    \end{macrocode}
%    Then we define |\@begintheorem| and
%    |\@opargbegintheorem|. These two macros define how the
%    header of a theorem is typeset.  |\@opargbegintheorem| will
%    be called if a \textsf{theorem} environment with an optional
%    argument is encountered; otherwise, the header is constructed by
%    calling |\@begintheorem|.  If one of these macros is
%    executed, we are within a \textsf{trivlist} environment started by
%    |\@thm|.  So the theorem header is produced with an
%    |\item| command.
%
%    Instead of specifying the header font directly, all standard
%    theorem styles use the |\theorem@headerfont| macro to allow
%    customization.  The extra space (|\labelsep|) is necessary
%    because of problems in the \textsf{trivlist} environment.
%    \begin{macrocode}
  \def\@begintheorem##1##2{%
        \item[\hskip\labelsep \theorem@headerfont ##1\ ##2]}%
%    \end{macrocode}
%    The definition of |\@opargbegintheorem| is completely
%    analogous. The only difference is the fact that there exists a
%    third argument (which is the optional parameter of the
%    environment and contains additional information about the
%    theorem).  Customarily we enclose it in parentheses.
%    \begin{macrocode}
\def\@opargbegintheorem##1##2##3{%
   \item[\hskip\labelsep \theorem@headerfont ##1\ ##2\ (##3)]}}
%    \end{macrocode}
% \end{macro}
% \end{macro}
% \end{macro}
%
% \noindent
% We conclude with an |\endgroup| to restore the
% |\catcode| of |@|.
%    \begin{macrocode}
\endgroup
%</thp>
%    \end{macrocode}
%
%
% \subsubsection{The \texttt{break} style}
%
% This style option is stored in the file |thb.sty|.
% For the next two lines see the documentation for |\th@plain| on
% page~\pageref{dok:th@plain}.
%    \begin{macrocode}
%<*thb>
\begingroup \makeatletter
\@ifundefined{theorem@style}{\input{theorem.sty}}{}
\ProvidesFile{thb.sty}
             [\FMithmInfo]
%    \end{macrocode}
%
% \begin{macro}{\th@break}
%    |\theoremstyle{break}| produces a line break after the name
%    of the theorem.  The font is |\slshape|.  Hence, we define
%    |\th@break| as follows:
%    \begin{macrocode}
\gdef\th@break{\normalfont\slshape
  \def\@begintheorem##1##2{\item[%
%    \end{macrocode}
%    We run into the following problem: it is not possible to create
%    the header with |\item[|\meta{title}|]| and then start
%    a new line by, for example, |\mbox{}\\|. Such a definition
%    will fail whenever a list environment follows immediately.  With
%    the above construction, the |\mbox{}| causes the switch
%    |@inlabel| (cf.\ definition of |\list| and
%    |\trivlist| in \cite{latex:source}) to be set to \texttt{false}
%    and so the following list will insert additional vertical
%    space (|\topskip|). This is quite annoying. Therefore, we
%    create the line break within the |\item|. In order to ensure
%    that the text will begin at the proper position in the following
%    line, we simply pretend that the label does not take any
%    room.\footnote{This will lead to problems whenever very high
%    symbols occurring in the line tower into the
%    heading. So, something else has to be done here sometime.}
%    \begin{macrocode}
       \rlap{\vbox{\hbox{\hskip \labelsep\theorem@headerfont ##1\ ##2}%
                   \hbox{\strut}}}]}%
%    \end{macrocode}
%    Again, the definition of |\@opargbegintheorem| is
%    completely analogous.
%    \begin{macrocode}
\def\@opargbegintheorem##1##2##3{%
  \item[\rlap{\vbox{\hbox{\hskip \labelsep \theorem@headerfont
                     ##1\ ##2\ (##3)}%
                    \hbox{\strut}}}]}}
\endgroup
%</thb>
%    \end{macrocode}
% \end{macro}
%
%
% \subsubsection{The \texttt{changebreak} style}
%
%    \begin{macrocode}
% This style option is stored in the file |thcb.sty|.
%    \begin{macrocode}
%<*thcb>
\begingroup \makeatletter
\@ifundefined{theorem@style}{\input{theorem.sty}}{}
\ProvidesFile{thcb.sty}
             [\FMithmInfo]
%    \end{macrocode}
%
% \begin{macro}{\th@changebreak}
%    The \texttt{change-break} theorem style is like \texttt{break}
%    but with interchange of theorem name and theorem number.  Thus we
%    define |\th@changebreak| as follows:
%    \begin{macrocode}
\gdef\th@changebreak{\normalfont\slshape
  \def\@begintheorem##1##2{\item
       [\rlap{\vbox{\hbox{\hskip\labelsep \theorem@headerfont ##2\ ##1}%
                    \hbox{\strut}}}]}%
\def\@opargbegintheorem##1##2##3{%
  \item[\rlap{\vbox{\hbox{\hskip\labelsep \theorem@headerfont
                     ##2\ ##1\ (##3)}%
                    \hbox{\strut}}}]}}
\endgroup
%</thcb>
%    \end{macrocode}
% \end{macro}
%
%
% \subsubsection{The \texttt{change} style}
%
% This style option is stored in the file |thc.sty|.
%    \begin{macrocode}
%<*thc>
\begingroup \makeatletter
\@ifundefined{theorem@style}{\input{theorem.sty}}{}
\ProvidesFile{thc.sty}
             [\FMithmInfo]
%    \end{macrocode}
%
% \begin{macro}{\th@change}
%    The \texttt{change} theorem style corresponds to the
%    \texttt{change break} style without a linebreak after the header.
%    To say it in another way, it's the same as the \texttt{plain}
%    style but with number and name interchanged and |\slshape| as the
%    default font.
%    \begin{macrocode}
\gdef\th@change{\normalfont\slshape
  \def\@begintheorem##1##2{\item
       [\hskip\labelsep \theorem@headerfont ##2\ ##1]}%
\def\@opargbegintheorem##1##2##3{%
  \item[\hskip\labelsep \theorem@headerfont ##2\ ##1\ (##3)]}}
\endgroup
%</thc>
%    \end{macrocode}
% \end{macro}
%
%
% \subsubsection{The \texttt{marginbreak} style}
%
% This style option is the one used most often at Mainz.  It is saved
% in the file |thmb.sty|.
%    \begin{macrocode}
%<*thmb>
\begingroup \makeatletter
\@ifundefined{theorem@style}{\input{theorem.sty}}{}
\ProvidesFile{thmb.sty}
             [\FMithmInfo]
%    \end{macrocode}
%
% \begin{macro}{\th@marginbreak}
%    The \texttt{margin break} style is nearly the same as the
%    \texttt{change break} style. The only difference is the placement
%    of the theorem number.  We use |\llap| to place it in the left
%    margin.
%
%    In this style |\labelsep| denotes the separation between the
%    number and the text.
%    \begin{macrocode}
\gdef\th@marginbreak{\normalfont\slshape
  \def\@begintheorem##1##2{\item
       [\rlap{\vbox{\theorem@headerfont
                    \hbox{\llap{##2}\hskip\labelsep ##1}%
                    \hbox{\strut}}}]}%
\def\@opargbegintheorem##1##2##3{%
  \item[\rlap{\vbox{\theorem@headerfont
                    \hbox{\llap{##2}\hskip\labelsep ##1\ (##3)}%
                    \hbox{\strut}}}]}}
\endgroup
%</thmb>
%    \end{macrocode}
% \changes{v2.1c}{90/02/17}{`endgroup outside macrocode env.}
% \end{macro}
%
%
%
% \subsubsection{The \texttt{margin} style}
%
% This style option is stored in the file |thm.sty|.
%    \begin{macrocode}
%<*thm>
\begingroup \makeatletter
\@ifundefined{theorem@style}{\input{theorem.sty}}{}
\ProvidesFile{thm.sty}
             [\FMithmInfo]
%    \end{macrocode}
%
% \begin{macro}{\th@margin}
%    Again this is only a variant of the theorem styles described
%    above without any new ideas.
%    \begin{macrocode}
\gdef\th@margin{\normalfont\slshape
  \def\@begintheorem##1##2{\item
       [\theorem@headerfont \llap{##2}\hskip\labelsep ##1]}%
\def\@opargbegintheorem##1##2##3{%
  \item[\theorem@headerfont \llap{##2}\hskip\labelsep ##1\ (##3)]}}
\endgroup
%</thm>
%    \end{macrocode}
% \end{macro}
%
%
%
%
% \subsection{Final Definitions}
%
% \begin{macro}{\theorempreskipamount}
% \begin{macro}{\theorempostskipamount}
%    The \textsf{skip} parameters that regulate the vertical empty space
%    before and after the {\envfont theorem} environment have
%    to be allocated as well.
% \changes{v2.0f}{89/07/20}{Unnecessary `global' removed.}
%    \begin{macrocode}
%<*package>
\newskip\theorempreskipamount
\newskip\theorempostskipamount
%    \end{macrocode}
%    Since we have used the same values for all theorem sets, we now can
%    assign them.
%    \begin{macrocode}
\global\setlength\theorempreskipamount{12pt plus 5pt minus 3pt}
\global\setlength\theorempostskipamount{8pt plus 3pt minus 1.5pt}
%    \end{macrocode}
% \end{macro}
% \end{macro}
%
%
% \begin{macro}{\@endtheorem}
%    The same holds for the macro |\@endtheorem|, which ends a
%    {\envfont theorem} environment. Since it is the same for all
%    theorem sets, it is removed from the macros |\th@|\meta{style}.
%    It simply ends the {\envfont trivlist} environment, which was
%    begun in |\@thm|.
%    \begin{macrocode}
\global\let\@endtheorem=\endtrivlist
%    \end{macrocode}
% \end{macro}
%
%
%
% \begin{macro}{\@preamblecmds}
%    All macros defined above are to be used only in the preamble.
%    Therefore, we insert them in |\@preamblecmds| which will disable
%    them at begin document. This is done by the internal \LaTeXe{}
%    command |\@onlypreamble|.
% \changes{v2.0g}{89/09/19}{`theoremheaderfont and `theorembodyfont
%    added.}
% \changes{v2.2a}{94/02/03}{Use \LaTeXe{} interface}
%    \begin{macrocode}
\@onlypreamble\@xnthm
\@onlypreamble\@ynthm
\@onlypreamble\@othm
\@onlypreamble\newtheorem
\@onlypreamble\theoremstyle
\@onlypreamble\theorembodyfont
\@onlypreamble\theoremheaderfont
%    \end{macrocode}
% \end{macro}
%
% Finally we declare the \texttt{plain} theorem style to be the default.
% \changes{v2.1b}{90/06/04}{Fixed missing `item by declaring default
%    style.}
%    \begin{macrocode}
\theoremstyle{plain}
%</package>
%    \end{macrocode}
%
% \Finale

\endinput
}{}
\ProvidesFile{thcb.sty}
             [\FMithmInfo]
%    \end{macrocode}
%
% \begin{macro}{\th@changebreak}
%    The \texttt{change-break} theorem style is like \texttt{break}
%    but with interchange of theorem name and theorem number.  Thus we
%    define |\th@changebreak| as follows:
%    \begin{macrocode}
\gdef\th@changebreak{\normalfont\slshape
  \def\@begintheorem##1##2{\item
       [\rlap{\vbox{\hbox{\hskip\labelsep \theorem@headerfont ##2\ ##1}%
                    \hbox{\strut}}}]}%
\def\@opargbegintheorem##1##2##3{%
  \item[\rlap{\vbox{\hbox{\hskip\labelsep \theorem@headerfont
                     ##2\ ##1\ (##3)}%
                    \hbox{\strut}}}]}}
\endgroup
%</thcb>
%    \end{macrocode}
% \end{macro}
%
%
% \subsubsection{The \texttt{change} style}
%
% This style option is stored in the file |thc.sty|.
%    \begin{macrocode}
%<*thc>
\begingroup \makeatletter
\@ifundefined{theorem@style}{% \iffalse meta-comment
%
% Copyright (C) 1993-2024
%
% The LaTeX Project and any individual authors listed elsewhere
% in this file.
%
% This file is part of the Standard LaTeX `Tools Bundle'.
% -------------------------------------------------------
%
% It may be distributed and/or modified under the
% conditions of the LaTeX Project Public License, either version 1.3c
% of this license or (at your option) any later version.
% The latest version of this license is in
%    https://www.latex-project.org/lppl.txt
% and version 1.3c or later is part of all distributions of LaTeX
% version 2005/12/01 or later.
%
% The list of all files belonging to the LaTeX `Tools Bundle' is
% given in the file `manifest.txt'.
%
% \fi
%
%%
%
%
\def\FMithmInfo{2023/07/05 v2.2c Theorem extension package (FMi)}
%
% \ProvidesFile{theorem.dtx}[\FMithmInfo]
%\iffalse   % this is a METACOMMENT !
%%
%% Package `theorem' to use with LaTeX2e
%% Copyright (C) 1989-1995 Frank Mittelbach, all rights reserved.
%<+package>\@ifundefined{theorem@style}{}{\endinput}
%<+package>\NeedsTeXFormat{LaTeX2e}[1995/06/01]
%<+package>\ProvidesPackage{theorem}
%<+package>         [\FMithmInfo]
%
% \fi
%
% \changes{v1.0}{87/05/07}{First draft.}
% \changes{v1.0a}{87/05/07}{`thistheoremstyle' added.}
% \changes{v1.1a}{87/07/29}{New concept: redefine the internal
%                          \protect\LaTeX{} commands `@xnthm' etc.}
% \changes{v1.1b}{87/07/31}{`theoremfont' used to specify font.}
% \changes{v1.2a}{88/01/08}{`@endtheorem' now globally defined.}
% \changes{v1.2b}{88/01/11}{`newtheorem' only allowed in the preamble.}
% \changes{v1.3a}{88/02/14}{`theoremfont' now a toks register.}
%
% \changes{v2.0a}{88/11/16}{New implementation and new documentation.}
% \changes{v2.0a}{88/11/16}{No layout changes but internal modifications
%                           old substyles won't work.}
% \changes{v2.0b}{89/01/30}{`global' used for all definitions.}
% \changes{v2.0c}{89/06/15}{Documentation translated to English.
%                            Thanks to LLL and Nina}
% \changes{v2.1a}{90/02/15}{Support for new font selection.}
% \changes{v2.2a}{94/02/03}{Cleanup for LaTeX2e}
%
% \DoNotIndex{\@endparenv,\@ifdefinable,\@ifnextchar,\@ifundefined}
% \DoNotIndex{\@namedef,\@nameuse,\@spaces,\@tempa,\@warning}
% \DoNotIndex{\ }
% \DoNotIndex{\begingroup,\bfseries}
% \DoNotIndex{\csname}
% \DoNotIndex{\def,\do,\docdate}
% \DoNotIndex{\endcsname,\endgroup,\endinput,\endtrivlist,\expandafter}
% \DoNotIndex{\filedate,\fileversion}
% \DoNotIndex{\gdef,\global}
% \DoNotIndex{\hbox,\hskip}
% \DoNotIndex{\ignorespaces,\input,\itshape,\item}
% \DoNotIndex{\labelsep,\let,\llap}
% \DoNotIndex{\makeatletter}
% \DoNotIndex{\noexpand}
% \DoNotIndex{\refstepcounter,\relax,\rlap}
% \DoNotIndex{\setlength,\slshape,\space,\string,\strut}
% \DoNotIndex{\the,\trivlist,\typeout}
% \DoNotIndex{\vbox}
% \DoNotIndex{\xdef}
%
% \def\deflabel#1{\ttfamily #1\hfill}
% \def\deflist#1{\begin{list}{}{\settowidth\labelwidth{\ttfamily #1}%
%                               \setlength\leftmargin\labelwidth
%                               \addtolength\leftmargin\labelsep
%                               \let\makelabel\deflabel}}
% \def\enddeflist{\end{list}}
% \def\envfont{\normalfont\ttfamily}
%
% \MakeShortVerb{\|}
%
% \newcommand{\TUB}{{\sl TUGboat\/}}
%
% \GetFileInfo{theorem.dtx}
% \title{An Extension of the
%        \LaTeX{} theorem environment\thanks{This file
%        has version number \fileversion, last
%        revised \filedate.}}
% \author{Frank Mittelbach}
%
% \MaintainedByLaTeXTeam{tools}
% \maketitle
%
% \begin{abstract}
%    The macros described in this paper yield an extension of the
%    \LaTeX{} theorem mechanism. It is designed to satisfy the
%    different requirements of various journals. Thus, the
%    layout of the ``theorems'' can be manipulated by determining a
%    ``style''. This article describes not only the use, but
%    also the definition, of the necessary macros.
% \end{abstract}
%
% \section*{Preface to version 2.2}
%
% For \LaTeXe{} this package did not need any fundamental changes. I
% only modified the messages generated so that theorem layout styles
% will show up with the |\listfiles| command and cleaned the section
% on the New Font Selection Scheme since this is now included in
% \LaTeX.
%
% \section*{Preface to version 2.1}
%
% This version is identical to 2.0g described in \TUB\ 10\#3 except
% for some internal defaults which are now set depending on the used
% font selection scheme.
%
% This was done to avoid unpleasant surprises if the new font selection
% scheme is in force. For further details see
% section~\ref{sec:newfonts} and \cite{bk:GMS94}.
%
% \section{Introduction}
%
% For our purposes here, ``theorems'' are labelled enunciations,
% often set off from the main text by extra space and a font change.
% Theorems, corollaries, conjectures, definitions, and remarks
% are all instances of ``theorems''.  The ``header'' of these structures
% is composed of a label (such as \textsc{Theorem} or \textsc{Remark})
% and a number which serializes an item in the sequence of items
% with the same label.
%
% Shortly after the introduction of \LaTeX{} at the Fachbereich
% Mathematik in Mainz, the desire to manipulate the layout of
% ``theorems'' arose.  In Mainz, the following two conventions came
% into general use:
% \begin{enumerate}
%   \item The number of the theorem is shown in the margin.
%   \item There is a line break at
%         the end of the theorem header.
% \end{enumerate}
%
% Additionally, some journals require different formats which depend on
% the ``sort of theorem'': e.g.\ often remarks and definitions are set
% in |\upshape|, while |\itshape| is employed for main theorems.
%
% Confronted with these requirements, a theorem environment was
% developed in Mainz which allows separate determination of the
% layout of the ``theorems sets'', comparable to |\pagestyle|.
%
%
% \section{The user interface}
% \subsection{Defining new theorem sets}
%
% \DescribeMacro\newtheorem
% As in the original \LaTeX{} version, the command |\newtheorem|
% defines a new ``theorem set'' or ``theorem-like structure''.
% Two required arguments name the new environment and give the text
% to be typeset with each instance of the new ``set'', while
% an optional argument determines how
% the ``set'' is enumerated:
% \begin{description}
%    \item[\ttfamily \bslash newtheorem\{foo\}\{bar\}]
%       The theorem set {\envfont foo} (whose name is \texttt{bar})
%       uses its own counter.
%    \item[\ttfamily \bslash newtheorem\{foo2\}{[foo]}\{bar2\}]
%       The theorem set {\envfont foo2} (printed name \texttt{bar2})
%       uses the same counter as the theorem set \texttt{foo}.
%    \item[\ttfamily \bslash newtheorem\{foo3\}\{bar3\}{[section]}]
%       The theorem set {\envfont foo3} (printed name \texttt{bar3}) is
%       enumerated within the counter \texttt{section}, i.e.\ with every
%       new |\section| the enumeration begins again with 1, and
%       the enumeration is composed from the section-number and the
%       theorem counter itself.
% \end{description}
%
% \DescribeMacro\theoremstyle
% Additionally, the command |\theoremstyle| can define the layout
% of various, or all, theorem sets.  It should be noted that any
% theorem set defined by |\newtheorem| is typeset in the
% |\theoremstyle| that is current at the time of the definition.
% Thus, the following
% \begin{verbatim}
%    \theoremstyle{break}        \newtheorem{Cor}{Corollary}
%    \theoremstyle{plain}        \newtheorem{Exa}{Example}[section]
%\end{verbatim}                             ^^A No blank at line start
% leads to the result that the set {\envfont Cor} is formatted in the
% style \texttt{break}, while the set {\envfont Exa} and all the
% following ones are formatted in the style \texttt{plain}, unless
% another |\theoremstyle| follows.  Since the definitions installed by
% |\newtheorem| are global, one also can limit |\theoremstyle| locally
% by grouping braces.
%
% \DescribeMacro\theorembodyfont
% The choice of the font for the theorem body is completely
% independent of the chosen |\theoremstyle|; this has proven to
% be very advantageous.  For example,
% \begin{verbatim}
%   {\theorembodyfont{\upshape}          \newtheorem{Rem}{Remark}}
%\end{verbatim}
% defines a theorem set {\envfont Rem}, which will be set in
% |\upshape| in the current layout (which in our example is
% \texttt{plain}).  As with |\theoremstyle|, the |\theorembodyfont|
% chosen is that current at the time of |\newtheorem|.  If
% |\theorembodyfont| is not specified or one defines
% |\theorembodyfont{}|, then the font used will be that defined by the
% |\theoremstyle|.
%
% \DescribeMacro\theoremheaderfont It is also possible to customize
% the font used for the theorem headers.  This is, however, a global
% declaration, and therefore there should be at most one
% |\theoremheaderfont| declaration in the preamble.\footnote{If it is
% actually necessary to have different header fonts, one has to define
% new theorem styles (substituting the desired font) or specify the
% information directly in the \texttt{\bslash newtheorem} declaration
% (the unclean variant).}
%
% \DescribeMacro\theorempreskipamount
% \DescribeMacro\theorempostskipamount Two additional parameters
% affect the vertical space around the theorem environments:
% |\theorempreskipamount| and |\theorempostskipamount| define,
% respectively, the spacing before and after such an environment.
% These parameters apply for all theorem sets and can be manipulated
% with the ordinary length macros.  They are rubber lengths
% (`\textsf{skips}'), and therefore can contain \texttt{plus} and
% \texttt{minus} parts.
%
% Since the definition of theorem sets should---most sensibly---be
% placed in the preamble, we only allow installation there.
% It is therefore possible to release the memory used here after
% |\begin{document}|, in order to make room for other
% applications.
%
%
% \subsection{Existing theorem styles}
%
% The following theorem styles exist to date:
% \begin{deflist}{marginbreak}
%    \item[plain]
%       This theorem style emulates the original \LaTeX{} definition,
%       except that additionally the parameters
%       |\theorem...skipamount| are used.
%    \item[break]
%       In this style, the theorem header is followed by a line
%       break.
%    \item[marginbreak]
%       The theorem number is set in the margin, and there is a line
%       break as in \texttt{break}.
%    \item[changebreak]
%       Like \texttt{break}, but with header number and text
%       interchanged.
%    \item[change]
%       Header number and text are interchanged, without a line break.
%    \item[margin]
%       The number is set in the left margin, without a line break.
% \end{deflist}
% All styles (except \texttt{plain}) select |\slshape| as the default
% |\theorembodyfont|.
%
%
% \subsection{Examples}
%
% Given the above theorem sets {\envfont Cor}, {\envfont Exa}
% and {\envfont Rem},
% suppose that the preamble also contains the declarations:
% \begin{verbatim}
%    \theoremstyle{marginbreak}   \newtheorem{Lem}[Cor]{Lemma}
%    \theoremstyle{change}
%    \theorembodyfont{\itshape}        \newtheorem{Def}[Cor]{Definition}
%
%    \theoremheaderfont{\scshape}
%\end{verbatim}
% Then the following are some typical examples of the typeset output
% resulting from their use.
%
% \begin{Cor}
%    This is a sentence typeset in the theorem environment {\envfont
%    Cor}.
% \end{Cor}
%
% \begin{Exa}
%    This is a sentence typeset in the theorem environment {\envfont
%    Exa}.
% \end{Exa}
%
% \begin{Rem}
%    This is a sentence typeset in the theorem environment {\envfont
%    Rem}.
% \end{Rem}
%
% \begin{Lem}[Ben User]
%    This is a sentence typeset in the theorem environment {\envfont
%    Lem}.
% \end{Lem}
%
% \begin{Def}[Very impressive Definition]
%    This is a sentence typeset in the theorem environment {\envfont
%    Def}.
% \end{Def}
%
% The last two examples show the effect of the optional argument to
% a theorem environment (it is the text typeset in parentheses).
%
%
% \section{Special Considerations}
% \label{sec:newfonts}
%
% Theoremheader and body are implemented as a unit. This means that
% the |\theoremheaderfont|  will inherit characteristics of the
% |\theorembodyfont| in \LaTeXe. Thus, if for example
% |\theorembodyfont| is |\itshape| and |\theoremheaderfont|
% is |\bfseries| the font selected for the header will have the
% characteristics `bold extended italic'. If this is not desired one
% should set the |\theoremheaderfont| to something like
% \begin{verbatim}
%      \theoremheaderfont{\normalfont\bfseries}
%\end{verbatim}
% i.e.\ supplying all necessary font information explicitly.
%
%
%
% \section{Acknowledgements}
%
% The publication of this set of macros was only possible with the
% help of Christina Busse (translating the manuscript into English),
% Joachim Pense (playing the r\^ole of typist), Chris Rowley
% (looking everything over) and many others providing useful
% suggestions.
%
% \MaybeStop{%
%    \begin{thebibliography}{9}
%       \bibitem{bk:GMS94} \textsc{M.~Goossens}, \textsc{F.~Mittelbach}
%         and \textsc{A.~Samarin}.
%         \newblock The \LaTeX{} Companion.
%         \newblock
%         Addison-Wesley, Reading, Massachusetts, 1994.
%       \bibitem{latex:source} \textsc{Lamport, Leslie.} \newblock
%                               \texttt{latex.tex}, version 2.09, date
%                               Feb. 1990.
%    \end{thebibliography}
%    \PrintIndex
%    \PrintChanges}
%
%
% \section{The documentation driver file}
%
% The next bit of code contains the documentation driver file for
% \TeX{}, i.e., the file that will produce the documentation you are
% currently reading. It will be extracted from this file by the
% \texttt{docstrip} program. Since it is the first code in the file
% one can alternatively process this file directly with \LaTeXe{} to
% obtain the documentation.
%
%    \begin{macrocode}
%<*driver>
\documentclass{ltxdoc}

\usepackage{theorem}

%<+driver>% The next few lines define theorem sets which are used
%<+driver>% in the example section of the documentation.

\theoremstyle{break}       \newtheorem{Cor}{Corollary}
\theoremstyle{plain}       \newtheorem{Exa}{Example}[section]
{\theorembodyfont{\upshape}\newtheorem{Rem}{Remark}}
\theoremstyle{marginbreak} \newtheorem{Lem}[Cor]{Lemma}
\theoremstyle{change}
\theorembodyfont{\itshape} \newtheorem{Def}[Cor]{Definition}

\theoremheaderfont{\scshape}

\RecordChanges

\begin{document}
   \DocInput{theorem.dtx}
\end{document}
%</driver>
%    \end{macrocode}
%
% \section{Definition of the Macros}
%
% If the file has been loaded before, we abort immediately. If not the
% package announces itself (this is actually done at the very top of
% the file---the way it is done isn't good style so don't copy it).
% \changes{v2.0e}{89/07/19}{Spaces removed from `typeout'.}
% \changes{v2.2b}{95/11/19}{Announce moved to top of file}
%    \begin{macrocode}
%<*package>
%\@ifundefined{theorem@style}{}{\endinput}
%\def\FMithmInfo{1995/11/19 v2.2b Theorem extension package (FMi)}
%\ProvidesPackage{theorem}[\FMithmInfo]
%    \end{macrocode}
%
%
% \subsection{Definition of theorem styles and fonts}
%
% All the definitions in this file are done globally to allow
% inputting this file inside a group.
%
% \begin{macro}{\theoremstyle}
%    Before a theorem style can be installed, the chosen style must be
%    known. For that reason, we must test to see that
%    |\th@|\meta{style} is known or, more precisely, that it is
%    different from |\relax|. If the style is not known then
%    |\th@plain| is used.
%    \begin{macrocode}
\gdef\theoremstyle#1{%
   \@ifundefined{th@#1}{\@warning
          {Unknown theoremstyle `#1'. Using `plain'}%
          \theorem@style{plain}}%
%    \end{macrocode}
%    We save the theorem style to be used in the token register
%    |\theorem@style|.
%    \begin{macrocode}
      {\theorem@style{#1}}%
%    \end{macrocode}
%    Now we ``evaluate'' the theorem style: this means, we call the
%    macro |\th@|\meta{style} which will activate the relevant
%    definitions which are contained in a separate file.  This is done
%    in a group to suppress changes to the current font.  This could
%    otherwise pose problems together with the new font selection
%    scheme\footnote{When I printed the original article using the new
%    font selection scheme I ended with a document with slanted
%    typefaces (text headings and all) simply because one of the
%    theorem styles used \texttt{\bslash sl} at toplevel.} if the
%    |\th@|\meta{style} is evaluated a second time.
% \changes{v2.1a}{90/02/15}{Group added around `th@..}
% \changes{v2.1a}{90/02/15}{Reseting of `..@begintheorem removed.}
%    \begin{macrocode}
      \begingroup
        \csname th@\the\theorem@style \endcsname
      \endgroup}
%    \end{macrocode}
% \end{macro}
%
%
% \begin{macro}{\@begintheorem}
% \changes{v2.1a}{90/02/15}{Relaxed at toplevel.}
% \begin{macro}{\@opargbegintheorem}
% \changes{v2.1a}{90/02/15}{Relaxed at toplevel.}
%    We reset |\@begintheorem| and |\@opargbegintheorem| to
%    |\relax| since these commands are no longer necessary at
%    toplevel. This will save a few tokens.
%    \begin{macrocode}
\global\let\@begintheorem\relax
\global\let\@opargbegintheorem\relax
%    \end{macrocode}
% \end{macro}
% \end{macro}
%
% \begin{macro}{\theorem@style}
%    Obviously the \textsf{token} register used above has to be
%    allocated. To assure the utmost compatibility with the
%    original \LaTeX{} definition, we set the default theorem style to
%    \texttt{plain}, which implements the usual \LaTeX{} convention.
% \changes{v2.0f}{89/07/20}{Unnecessary `global' removed.}
%    \begin{macrocode}
\newtoks\theorem@style
\global\theorem@style{plain}
%    \end{macrocode}
% \end{macro}
%
% \begin{macro}{\theorembodyfont}
% \begin{macro}{\theorem@bodyfont}
%    If the |\theorembodyfont| is set by the user then it should not
%    interact with the default font set in the theorem style. When the
%    new font selection is in force this may happen if, for example,
%    the default is |\itshape| and the new |\theorembodyfont| is
%    |\sffamily|. So we add a |\reset@font| command in front of the
%    user definition.
% \changes{v2.0f}{89/07/20}{Unnecessary `global' removed.}
% \changes{v2.0g}{89/09/19}{`theoremfont to `theorembodyfont.}
% \changes{v2.1c}{92/06/22}{`reset@font added and logic changed.}
% \changes{v2.1d}{92/10/28}{Reset to empty if arg is empty}
%    \begin{macrocode}
\gdef\theorembodyfont#1{%
%    \end{macrocode}
%    We check if the argument supplied is empty and if so put nothing
%    into the |\theorem@bodyfont| token register to allow for
%    |\theorembodyfont{}| as a mean of using the default of the
%    current |\theoremstyle|.
%    \begin{macrocode}
   \def\@tempa{#1}%
   \ifx\@tempa\@empty
    \theorem@bodyfont{}%
   \else
    \theorem@bodyfont{\reset@font#1}%
   \fi
   }
\newtoks\theorem@bodyfont
\global\theorem@bodyfont{}
%    \end{macrocode}
% \end{macro}
% \end{macro}
%
%
% \begin{macro}{\theoremheaderfont}
% \changes{v2.0g}{89/09/19}{Macro added.}
%    The font for the theorem headers is handled differently because
%    this definition applies to all theorem styles.
%    \begin{macrocode}
\gdef\theoremheaderfont#1{\gdef\theorem@headerfont{#1}%
%    \end{macrocode}
%    After using the macro once it is redefined to produce an error
%    message.
%    \begin{macrocode}
       \gdef\theoremheaderfont##1{%
        \typeout{\string\theoremheaderfont\space should be used
                 only once.}}}
%    \end{macrocode}
% \end{macro}
%
% \begin{macro}{\theorem@headerfont}
% \changes{v2.0g}{89/09/19}{Macro added.}
%    To set the |\theorem@headerfont| default we first test if
%    the new fontselection scheme is in force.
% \changes{v2.1a}{90/02/15}{Support for new font selection.}
%    \begin{macrocode}
\ifx\upshape\undefined
%    \end{macrocode}
%    If not we define it to expand into |\bfseries|. We don't use
%    |\let| just in case a following style option redefines this
%    macro.
%    \begin{macrocode}
\gdef\theorem@headerfont{\bfseries}
%    \end{macrocode}
%    Otherwise we reset the current shape before calling |\bfseries|.
%    \begin{macrocode}
\else \gdef\theorem@headerfont{\normalfont\bfseries}\fi
%    \end{macrocode}
% \end{macro}
%
% \begin{macro}{\th@plain}
% \begin{macro}{\th@break}
% \begin{macro}{\th@marginbreak}
% \begin{macro}{\th@changebreak}
% \begin{macro}{\th@change}
% \begin{macro}{\th@margin}
%    The different styles are defined in macros such as
%    |\th@plain|.  Since memory space is precious in
%    ``non-Big-versions'', we have to avoid offering too many unused
%    definitions.  Therefore we define these styles in separate files
%    that can be loaded on demand.  Thus the commands themselves only
%    load these files. We use |\@input@|, a \LaTeXe{} internal command
%    that ensures that the file will be listed with |\listfiles|.
% \changes{v2.2a}{94/02/02}{Use `@input@ to load theorem layout files}
%    \begin{macrocode}
\gdef\th@plain{\@input@{thp.sty}}
\gdef\th@break{\@input@{thb.sty}}
\gdef\th@marginbreak{\@input@{thmb.sty}}
\gdef\th@changebreak{\@input@{thcb.sty}}
\gdef\th@change{\@input@{thc.sty}}
\gdef\th@margin{\@input@{thm.sty}}
%    \end{macrocode}
%    This list will be expanded when new styles become
%    available.  For testing,  just append new theorem substyles
%    as document options.
% \end{macro}
% \end{macro}
% \end{macro}
% \end{macro}
% \end{macro}
% \end{macro}
%
%
%
% \subsection{Definition of a new theorem set}
%
% As already pointed out, a new theorem environment can be defined in
% three different ways:
% \begin{verbatim}
%    \newtheorem{Lem}{Lemma}
%    \newtheorem{Lem}{Lemma}[section]
%    \newtheorem{Lem}[Theorem]{Lemma}
%\end{verbatim}
% The function of the macro |\newtheorem| is to recognize these cases
% and then to branch into one of the three macros |\@ynthm|, |\@xnthm|
% or |\@othm|.  This mechanism is adopted unchanged from
% \cite{latex:source}; the essential point here is that, for example,
% in the second case, the arguments \texttt{Lem}, \texttt{Lemma} and
% \texttt{section} are passed over to the macro |\@xnthm|.
%
% We inspect this case first because the others present fewer problems,
% and thus are easily derived from this one.
%
% \begin{macro}{\@xnthm}
%    For our example arguments, the macro |\@xnthm| must fulfill
%    the following:
%    \begin{itemize}
%       \item
%          Define a new \LaTeX-counter `Lem'
%       \item
%          reset this counter within a |\section|
%       \item
%          define the macro |\theLem|
%       \item
%          define the environment macros |\Lem| and
%          |\endLem| using the current
%          |\theoremstyle| and |\theorem@bodyfont|.
%    \end{itemize}
%    Obviously, all this should happen only if the first argument
%    of |\@xnthm| (i.e.\ \texttt{Lem} in our example) is chosen so
%    as not to conflict with any previously defined commands or
%    environments.  This test is performed by the \LaTeX{} macro
%    |\@ifdefinable|.
%    \begin{macrocode}
\gdef\@xnthm#1#2[#3]{\expandafter\@ifdefinable\csname #1\endcsname
%    \end{macrocode}
%    Therefore, the first argument of |\@ifdefinable| is the
%    expansion (in the example, |\Lem|) of
%    |\csname||#1\endcsname|.  The second argument is
%    executed only if the test has been completed successfully.
%    \begin{macrocode}
   {%
%    \end{macrocode}
%    Now we define the new counter. The names of the \LaTeX{}
%    macros employed should speak for themselves:
% \changes{v2.2b}{1995/10/07}{Use \cs{@newctr}}
% \changes{v2.2c}{1995/11/23}{Use \cs{@newctr} properly.}
%    \begin{macrocode}
    \@definecounter{#1}\@newctr{#1}[#3]%
%    \end{macrocode}
%    Using |\@newctr| will give a proper error message if the counter
%    in |#3| is not defined.
%    In defining `|\theLem|' we must generate the desired macro
%    name by use of |\expandafter| and |\csname|.
%    \begin{macrocode}
    \expandafter\xdef\csname the#1\endcsname
%    \end{macrocode}
%    An |\xdef| is used in order to make the definition global,
%    and to ensure that it contains the replacement texts of
%    |\@thmcountersep| and |\@thmcounter|.\footnote{These
%    two macros can be defined by the document style.  Their default
%    values produce a `\texttt{.}' as separation and an arabic
%    representation of the number.} However, not everything should be
%    expanded. For example, it saves space to use |\thesection|
%    instead of its---at times---lengthy expansion.
%    \begin{macrocode}
      {\expandafter \noexpand \csname the#3\endcsname
       \@thmcountersep \@thmcounter{#1}}%
%    \end{macrocode}
%    Thus with the defaults of \LaTeX{}, |\theLem| would be
%    replaced by the command sequence |\thesection.\arabic{Lem}|.
%
%    We will now look at the definition of the macro which is executed
%    at the beginning of the actual environment (in our example this
%    macro is |\Lem|). It should be noted that we use an
%    ``|\expandafter| trick'' to expand only certain parts of the
%    replacement text at the time of the definition.
%    \begin{macrocode}
    \def\@tempa{\global\@namedef{#1}}%
    \expandafter \@tempa \expandafter{%
%    \end{macrocode}
%    First, the macro that contains the current definitions of
%    |\@begintheorem| and |\@opargtheorem| should be called
%    up.  The name of this macro---as is already known---has the form
%    |\th@|\meta{theorem\ style}; therefore, it must be called by
%    \begin{macrocode}
      \csname th@\the \theorem@style
%    \end{macrocode}
%    In addition the default theorem font should be changeable, i.e.\
%    we have to insert the contents of |\theorem@bodyfont|. For
%    that reason, we expand even further, beyond |\endcsname|,
%    and thus insert the contents of the \textsf{token} register
%    |\theorem@bodyfont| in the replacement text.
%    \begin{macrocode}
            \expandafter \endcsname \the \theorem@bodyfont
%    \end{macrocode}
%    Now it is time to call the macro |\@thm| which takes over
%    the further processing.  It has two arguments: the current
%    counter name (in our example, \texttt{Lem}), and the text of the
%    label (in our example, \texttt{Lemma}).
%    \begin{macrocode}
     \@thm{#1}{#2}}%
%    \end{macrocode}
%    With this, the `sub-definition' is complete.  The macro
%    |\@endtheorem| ends a theorem environment and is, so far, nothing
%    but an |\endtrivlist|.  (Hence it is defined globally, and not
%    within the theorem styles.\footnote{This has to be changed as
%    soon as theorem styles that change \texttt{\bslash @endtheorem}
%    exist.  In such a case, all existing styles must be changed as
%    well since they will have to reset the macro.}) Therefore, we can
%    set it equivalent to the macro that ends the theorem set (in our
%    example, |\endLem|).  However, if some day theorem styles exist
%    that do change |\@endtheorem|, we would have to use the
%    commented-out line instead.
%    \begin{macrocode}
    \global \expandafter \let \csname end#1\endcsname \@endtheorem
%   \global\@namedef{end#1}{\@endtheorem}%
%    \end{macrocode}
%    With these commands all the required definitions are employed,
%    unless the test |\@ifdefinable| has failed.  Therefore, we
%    end the second argument of this macro and with it the definition
%    of |\@xnthm|.
%    \begin{macrocode}
   }}
%    \end{macrocode}
% \end{macro}
%
% \begin{macro}{\@ynthm}
%    The definition of |\@ynthm| is completely analogous. In
%    this case the new counter that is defined is not reset within
%    another counter; thus the definition of |\the...|
%    is simplified:
%    \begin{macrocode}
\gdef\@ynthm#1#2{\expandafter\@ifdefinable\csname #1\endcsname
   {\@definecounter{#1}%
    \expandafter\xdef\csname the#1\endcsname{\@thmcounter{#1}}%
%    \end{macrocode}
%    The rest of the definition corresponds literally to that of
%    |\@xnthm|:
%    \begin{macrocode}
    \def\@tempa{\global\@namedef{#1}}\expandafter \@tempa
     \expandafter{\csname th@\the \theorem@style \expandafter
     \endcsname \the\theorem@bodyfont \@thm{#1}{#2}}%
    \global \expandafter \let \csname end#1\endcsname \@endtheorem}}
%    \end{macrocode}
% \end{macro}
%
% \begin{macro}{\@othm}
%    The definition of |\@othm| does not contain anything new.
%    \begin{macrocode}
\gdef\@othm#1[#2]#3{%
%    \end{macrocode}
% \changes{v2.2b}{95/10/07}{Build in check for counter defined}
%    We do not define a new counter but instead use one that has
%    already been defined.  Thus the only definition we need is that
%    of this pseudo-counter (i.e.\ |\the|\meta{env.\ name}).
%    First we check if |#2| corresponds to a known counter name.
%    \begin{macrocode}
  \expandafter\ifx\csname c@#2\endcsname\relax
   \@nocounterr{#2}%
  \else
   \expandafter\@ifdefinable\csname #1\endcsname
   {\expandafter \xdef \csname the#1\endcsname
     {\expandafter \noexpand \csname the#2\endcsname}%
%    \end{macrocode}
%    All other parts of the definition can be adopted from
%    |\@xnthm|. We have to remember, though, that in this case
%    the name of the current counter and the theorem label have moved
%    to the second and third arguments.
%    \begin{macrocode}
    \def\@tempa{\global\@namedef{#1}}\expandafter \@tempa
     \expandafter{\csname th@\the \theorem@style \expandafter
     \endcsname \the\theorem@bodyfont \@thm{#2}{#3}}%
    \global \expandafter \let \csname end#1\endcsname \@endtheorem}%
  \fi}
%    \end{macrocode}
% \end{macro}
%
% \subsection{Macros that are employed in a {\envfont theorem}
%    environment}
%
% \begin{macro}{\@thm}
%    The macro |\@thm| has to increase the current counter.  Then,
%    depending on whether the environment has (or does not have) an
%    optional argument, it has to branch into either |\@begintheorem|
%    or |\@opargtheorem|.
%    \begin{macrocode}
\gdef\@thm#1#2{\refstepcounter{#1}%
%    \end{macrocode}
%    Now we start a {\envfont trivlist} environment, and give
%    |\@topsep| and |\@topsepadd| the values of the \textsf{skip}
%    registers |\theorempreskipamount| and |\theorempostskipamount|.
%    The value in |\@topsep| is the vertical space that is inserted by
%    the first (and only) |\item| in our |\trivlist| whilst
%    |\@topsepadd| is inserted by |\@endparenv| at the end of that
%    {\envfont trivlist} environment.  By using these registers, we
%    obtain the desired space around a {\envfont theorem} environment.
%    \begin{macrocode}
   \trivlist
   \@topsep \theorempreskipamount               % used by first \item
   \@topsepadd \theorempostskipamount           % used by \@endparenv
%    \end{macrocode}
%    Now we have to test whether an optional argument has been given.
% \changes{v2.0d}{89/07/18}{Forgotten percent added.}
%    \begin{macrocode}
   \@ifnextchar [%
%    \end{macrocode}
%    If there is an optional argument, we will call |\@ythm|, and
%    move the arguments read back into the input stream.
%    \begin{macrocode}
   {\@ythm{#1}{#2}}%
%    \end{macrocode}
%    If not, we call |\@begintheorem|. Its first argument is the
%    name of the theorem set (hence the second argument of
%    |\@thm|). Its second argument is the macro that produces the
%    current number.
%    \begin{macrocode}
   {\@begintheorem{#2}{\csname the#1\endcsname}\ignorespaces}}
%    \end{macrocode}
% \end{macro}
%
% \begin{macro}{\@xthm}
% \begin{macro}{\@ythm}
%    Both these macros were originally called by |\@thm|.  We do
%    not need |\@xthm| anymore, hence we reset it to
%    |\relax|. The definition of |\@ythm| has not changed at
%    all from its definition in \LaTeX. In order to make the macros
%    easier to understand, we will
%    nevertheless present it (commented out).
%    \begin{macrocode}
\global\let\@xthm\relax
% \def\@ythm#1#2[#3]{\@opargbegintheorem{#2}{\csname
%           the#1\endcsname}{#3}\ignorespaces}
%</package>
%    \end{macrocode}
%    The primitive |\ignorespaces| in |\@ythm| and
%    |\@thm| is needed to remove the spaces between the
%    |\begin{...}| and the actual text.
% \end{macro}
% \end{macro}
%
%
% \subsection{Definition of the theorem substyles}
%
% As already pointed out, the theorem substyles, defined below, are
% only loaded when necessary.  Note that all these substyles, except
% \texttt{plain}, have |\slshape| as the default body font.
%
% \subsubsection{The \texttt{plain} style}
%
% \label{dok:th@plain} As the following macros use |@|, we have
% to locally set the |\catcode| of this symbol to ``letter''.
% This happens within a group, so that we do not have to worry about
% which |\catcode| that symbol had before.
%    \begin{macrocode}
%<*thp>
\begingroup \makeatletter
%    \end{macrocode}
% Since we are now within a group, we must make all definitions
% globally.  First we make sure that \texttt{theorem.sty} is loaded.
% This will allow us to use this file as a document style option
% without having to call \texttt{theorem} itself as an option.  At the
% same time, we assure that at least version 2 is loaded, since
% |\theorem@style| was not defined in earlier versions.
%    \begin{macrocode}
\@ifundefined{theorem@style}{\input{theorem.sty}}{}
\ProvidesFile{thp.sty}
             [\FMithmInfo]
%    \end{macrocode}
%
%
% \begin{macro}{\th@plain}
% \begin{macro}{\@begintheorem}
% \begin{macro}{\@opargbegintheorem}
%    |\theoremstyle{plain}| corresponds to the original
%    definition, except that the distances to the surrounding text are
%    determined by the parameters |\theorempreskipamount| and
%    |\theorempostskipamount|.
%    First we set the default body font.
%    \begin{macrocode}
\gdef\th@plain{\normalfont\itshape
%    \end{macrocode}
%    Then we define |\@begintheorem| and
%    |\@opargbegintheorem|. These two macros define how the
%    header of a theorem is typeset.  |\@opargbegintheorem| will
%    be called if a \textsf{theorem} environment with an optional
%    argument is encountered; otherwise, the header is constructed by
%    calling |\@begintheorem|.  If one of these macros is
%    executed, we are within a \textsf{trivlist} environment started by
%    |\@thm|.  So the theorem header is produced with an
%    |\item| command.
%
%    Instead of specifying the header font directly, all standard
%    theorem styles use the |\theorem@headerfont| macro to allow
%    customization.  The extra space (|\labelsep|) is necessary
%    because of problems in the \textsf{trivlist} environment.
%    \begin{macrocode}
  \def\@begintheorem##1##2{%
        \item[\hskip\labelsep \theorem@headerfont ##1\ ##2]}%
%    \end{macrocode}
%    The definition of |\@opargbegintheorem| is completely
%    analogous. The only difference is the fact that there exists a
%    third argument (which is the optional parameter of the
%    environment and contains additional information about the
%    theorem).  Customarily we enclose it in parentheses.
%    \begin{macrocode}
\def\@opargbegintheorem##1##2##3{%
   \item[\hskip\labelsep \theorem@headerfont ##1\ ##2\ (##3)]}}
%    \end{macrocode}
% \end{macro}
% \end{macro}
% \end{macro}
%
% \noindent
% We conclude with an |\endgroup| to restore the
% |\catcode| of |@|.
%    \begin{macrocode}
\endgroup
%</thp>
%    \end{macrocode}
%
%
% \subsubsection{The \texttt{break} style}
%
% This style option is stored in the file |thb.sty|.
% For the next two lines see the documentation for |\th@plain| on
% page~\pageref{dok:th@plain}.
%    \begin{macrocode}
%<*thb>
\begingroup \makeatletter
\@ifundefined{theorem@style}{\input{theorem.sty}}{}
\ProvidesFile{thb.sty}
             [\FMithmInfo]
%    \end{macrocode}
%
% \begin{macro}{\th@break}
%    |\theoremstyle{break}| produces a line break after the name
%    of the theorem.  The font is |\slshape|.  Hence, we define
%    |\th@break| as follows:
%    \begin{macrocode}
\gdef\th@break{\normalfont\slshape
  \def\@begintheorem##1##2{\item[%
%    \end{macrocode}
%    We run into the following problem: it is not possible to create
%    the header with |\item[|\meta{title}|]| and then start
%    a new line by, for example, |\mbox{}\\|. Such a definition
%    will fail whenever a list environment follows immediately.  With
%    the above construction, the |\mbox{}| causes the switch
%    |@inlabel| (cf.\ definition of |\list| and
%    |\trivlist| in \cite{latex:source}) to be set to \texttt{false}
%    and so the following list will insert additional vertical
%    space (|\topskip|). This is quite annoying. Therefore, we
%    create the line break within the |\item|. In order to ensure
%    that the text will begin at the proper position in the following
%    line, we simply pretend that the label does not take any
%    room.\footnote{This will lead to problems whenever very high
%    symbols occurring in the line tower into the
%    heading. So, something else has to be done here sometime.}
%    \begin{macrocode}
       \rlap{\vbox{\hbox{\hskip \labelsep\theorem@headerfont ##1\ ##2}%
                   \hbox{\strut}}}]}%
%    \end{macrocode}
%    Again, the definition of |\@opargbegintheorem| is
%    completely analogous.
%    \begin{macrocode}
\def\@opargbegintheorem##1##2##3{%
  \item[\rlap{\vbox{\hbox{\hskip \labelsep \theorem@headerfont
                     ##1\ ##2\ (##3)}%
                    \hbox{\strut}}}]}}
\endgroup
%</thb>
%    \end{macrocode}
% \end{macro}
%
%
% \subsubsection{The \texttt{changebreak} style}
%
%    \begin{macrocode}
% This style option is stored in the file |thcb.sty|.
%    \begin{macrocode}
%<*thcb>
\begingroup \makeatletter
\@ifundefined{theorem@style}{\input{theorem.sty}}{}
\ProvidesFile{thcb.sty}
             [\FMithmInfo]
%    \end{macrocode}
%
% \begin{macro}{\th@changebreak}
%    The \texttt{change-break} theorem style is like \texttt{break}
%    but with interchange of theorem name and theorem number.  Thus we
%    define |\th@changebreak| as follows:
%    \begin{macrocode}
\gdef\th@changebreak{\normalfont\slshape
  \def\@begintheorem##1##2{\item
       [\rlap{\vbox{\hbox{\hskip\labelsep \theorem@headerfont ##2\ ##1}%
                    \hbox{\strut}}}]}%
\def\@opargbegintheorem##1##2##3{%
  \item[\rlap{\vbox{\hbox{\hskip\labelsep \theorem@headerfont
                     ##2\ ##1\ (##3)}%
                    \hbox{\strut}}}]}}
\endgroup
%</thcb>
%    \end{macrocode}
% \end{macro}
%
%
% \subsubsection{The \texttt{change} style}
%
% This style option is stored in the file |thc.sty|.
%    \begin{macrocode}
%<*thc>
\begingroup \makeatletter
\@ifundefined{theorem@style}{\input{theorem.sty}}{}
\ProvidesFile{thc.sty}
             [\FMithmInfo]
%    \end{macrocode}
%
% \begin{macro}{\th@change}
%    The \texttt{change} theorem style corresponds to the
%    \texttt{change break} style without a linebreak after the header.
%    To say it in another way, it's the same as the \texttt{plain}
%    style but with number and name interchanged and |\slshape| as the
%    default font.
%    \begin{macrocode}
\gdef\th@change{\normalfont\slshape
  \def\@begintheorem##1##2{\item
       [\hskip\labelsep \theorem@headerfont ##2\ ##1]}%
\def\@opargbegintheorem##1##2##3{%
  \item[\hskip\labelsep \theorem@headerfont ##2\ ##1\ (##3)]}}
\endgroup
%</thc>
%    \end{macrocode}
% \end{macro}
%
%
% \subsubsection{The \texttt{marginbreak} style}
%
% This style option is the one used most often at Mainz.  It is saved
% in the file |thmb.sty|.
%    \begin{macrocode}
%<*thmb>
\begingroup \makeatletter
\@ifundefined{theorem@style}{\input{theorem.sty}}{}
\ProvidesFile{thmb.sty}
             [\FMithmInfo]
%    \end{macrocode}
%
% \begin{macro}{\th@marginbreak}
%    The \texttt{margin break} style is nearly the same as the
%    \texttt{change break} style. The only difference is the placement
%    of the theorem number.  We use |\llap| to place it in the left
%    margin.
%
%    In this style |\labelsep| denotes the separation between the
%    number and the text.
%    \begin{macrocode}
\gdef\th@marginbreak{\normalfont\slshape
  \def\@begintheorem##1##2{\item
       [\rlap{\vbox{\theorem@headerfont
                    \hbox{\llap{##2}\hskip\labelsep ##1}%
                    \hbox{\strut}}}]}%
\def\@opargbegintheorem##1##2##3{%
  \item[\rlap{\vbox{\theorem@headerfont
                    \hbox{\llap{##2}\hskip\labelsep ##1\ (##3)}%
                    \hbox{\strut}}}]}}
\endgroup
%</thmb>
%    \end{macrocode}
% \changes{v2.1c}{90/02/17}{`endgroup outside macrocode env.}
% \end{macro}
%
%
%
% \subsubsection{The \texttt{margin} style}
%
% This style option is stored in the file |thm.sty|.
%    \begin{macrocode}
%<*thm>
\begingroup \makeatletter
\@ifundefined{theorem@style}{\input{theorem.sty}}{}
\ProvidesFile{thm.sty}
             [\FMithmInfo]
%    \end{macrocode}
%
% \begin{macro}{\th@margin}
%    Again this is only a variant of the theorem styles described
%    above without any new ideas.
%    \begin{macrocode}
\gdef\th@margin{\normalfont\slshape
  \def\@begintheorem##1##2{\item
       [\theorem@headerfont \llap{##2}\hskip\labelsep ##1]}%
\def\@opargbegintheorem##1##2##3{%
  \item[\theorem@headerfont \llap{##2}\hskip\labelsep ##1\ (##3)]}}
\endgroup
%</thm>
%    \end{macrocode}
% \end{macro}
%
%
%
%
% \subsection{Final Definitions}
%
% \begin{macro}{\theorempreskipamount}
% \begin{macro}{\theorempostskipamount}
%    The \textsf{skip} parameters that regulate the vertical empty space
%    before and after the {\envfont theorem} environment have
%    to be allocated as well.
% \changes{v2.0f}{89/07/20}{Unnecessary `global' removed.}
%    \begin{macrocode}
%<*package>
\newskip\theorempreskipamount
\newskip\theorempostskipamount
%    \end{macrocode}
%    Since we have used the same values for all theorem sets, we now can
%    assign them.
%    \begin{macrocode}
\global\setlength\theorempreskipamount{12pt plus 5pt minus 3pt}
\global\setlength\theorempostskipamount{8pt plus 3pt minus 1.5pt}
%    \end{macrocode}
% \end{macro}
% \end{macro}
%
%
% \begin{macro}{\@endtheorem}
%    The same holds for the macro |\@endtheorem|, which ends a
%    {\envfont theorem} environment. Since it is the same for all
%    theorem sets, it is removed from the macros |\th@|\meta{style}.
%    It simply ends the {\envfont trivlist} environment, which was
%    begun in |\@thm|.
%    \begin{macrocode}
\global\let\@endtheorem=\endtrivlist
%    \end{macrocode}
% \end{macro}
%
%
%
% \begin{macro}{\@preamblecmds}
%    All macros defined above are to be used only in the preamble.
%    Therefore, we insert them in |\@preamblecmds| which will disable
%    them at begin document. This is done by the internal \LaTeXe{}
%    command |\@onlypreamble|.
% \changes{v2.0g}{89/09/19}{`theoremheaderfont and `theorembodyfont
%    added.}
% \changes{v2.2a}{94/02/03}{Use \LaTeXe{} interface}
%    \begin{macrocode}
\@onlypreamble\@xnthm
\@onlypreamble\@ynthm
\@onlypreamble\@othm
\@onlypreamble\newtheorem
\@onlypreamble\theoremstyle
\@onlypreamble\theorembodyfont
\@onlypreamble\theoremheaderfont
%    \end{macrocode}
% \end{macro}
%
% Finally we declare the \texttt{plain} theorem style to be the default.
% \changes{v2.1b}{90/06/04}{Fixed missing `item by declaring default
%    style.}
%    \begin{macrocode}
\theoremstyle{plain}
%</package>
%    \end{macrocode}
%
% \Finale

\endinput
}{}
\ProvidesFile{thc.sty}
             [\FMithmInfo]
%    \end{macrocode}
%
% \begin{macro}{\th@change}
%    The \texttt{change} theorem style corresponds to the
%    \texttt{change break} style without a linebreak after the header.
%    To say it in another way, it's the same as the \texttt{plain}
%    style but with number and name interchanged and |\slshape| as the
%    default font.
%    \begin{macrocode}
\gdef\th@change{\normalfont\slshape
  \def\@begintheorem##1##2{\item
       [\hskip\labelsep \theorem@headerfont ##2\ ##1]}%
\def\@opargbegintheorem##1##2##3{%
  \item[\hskip\labelsep \theorem@headerfont ##2\ ##1\ (##3)]}}
\endgroup
%</thc>
%    \end{macrocode}
% \end{macro}
%
%
% \subsubsection{The \texttt{marginbreak} style}
%
% This style option is the one used most often at Mainz.  It is saved
% in the file |thmb.sty|.
%    \begin{macrocode}
%<*thmb>
\begingroup \makeatletter
\@ifundefined{theorem@style}{% \iffalse meta-comment
%
% Copyright (C) 1993-2024
%
% The LaTeX Project and any individual authors listed elsewhere
% in this file.
%
% This file is part of the Standard LaTeX `Tools Bundle'.
% -------------------------------------------------------
%
% It may be distributed and/or modified under the
% conditions of the LaTeX Project Public License, either version 1.3c
% of this license or (at your option) any later version.
% The latest version of this license is in
%    https://www.latex-project.org/lppl.txt
% and version 1.3c or later is part of all distributions of LaTeX
% version 2005/12/01 or later.
%
% The list of all files belonging to the LaTeX `Tools Bundle' is
% given in the file `manifest.txt'.
%
% \fi
%
%%
%
%
\def\FMithmInfo{2023/07/05 v2.2c Theorem extension package (FMi)}
%
% \ProvidesFile{theorem.dtx}[\FMithmInfo]
%\iffalse   % this is a METACOMMENT !
%%
%% Package `theorem' to use with LaTeX2e
%% Copyright (C) 1989-1995 Frank Mittelbach, all rights reserved.
%<+package>\@ifundefined{theorem@style}{}{\endinput}
%<+package>\NeedsTeXFormat{LaTeX2e}[1995/06/01]
%<+package>\ProvidesPackage{theorem}
%<+package>         [\FMithmInfo]
%
% \fi
%
% \changes{v1.0}{87/05/07}{First draft.}
% \changes{v1.0a}{87/05/07}{`thistheoremstyle' added.}
% \changes{v1.1a}{87/07/29}{New concept: redefine the internal
%                          \protect\LaTeX{} commands `@xnthm' etc.}
% \changes{v1.1b}{87/07/31}{`theoremfont' used to specify font.}
% \changes{v1.2a}{88/01/08}{`@endtheorem' now globally defined.}
% \changes{v1.2b}{88/01/11}{`newtheorem' only allowed in the preamble.}
% \changes{v1.3a}{88/02/14}{`theoremfont' now a toks register.}
%
% \changes{v2.0a}{88/11/16}{New implementation and new documentation.}
% \changes{v2.0a}{88/11/16}{No layout changes but internal modifications
%                           old substyles won't work.}
% \changes{v2.0b}{89/01/30}{`global' used for all definitions.}
% \changes{v2.0c}{89/06/15}{Documentation translated to English.
%                            Thanks to LLL and Nina}
% \changes{v2.1a}{90/02/15}{Support for new font selection.}
% \changes{v2.2a}{94/02/03}{Cleanup for LaTeX2e}
%
% \DoNotIndex{\@endparenv,\@ifdefinable,\@ifnextchar,\@ifundefined}
% \DoNotIndex{\@namedef,\@nameuse,\@spaces,\@tempa,\@warning}
% \DoNotIndex{\ }
% \DoNotIndex{\begingroup,\bfseries}
% \DoNotIndex{\csname}
% \DoNotIndex{\def,\do,\docdate}
% \DoNotIndex{\endcsname,\endgroup,\endinput,\endtrivlist,\expandafter}
% \DoNotIndex{\filedate,\fileversion}
% \DoNotIndex{\gdef,\global}
% \DoNotIndex{\hbox,\hskip}
% \DoNotIndex{\ignorespaces,\input,\itshape,\item}
% \DoNotIndex{\labelsep,\let,\llap}
% \DoNotIndex{\makeatletter}
% \DoNotIndex{\noexpand}
% \DoNotIndex{\refstepcounter,\relax,\rlap}
% \DoNotIndex{\setlength,\slshape,\space,\string,\strut}
% \DoNotIndex{\the,\trivlist,\typeout}
% \DoNotIndex{\vbox}
% \DoNotIndex{\xdef}
%
% \def\deflabel#1{\ttfamily #1\hfill}
% \def\deflist#1{\begin{list}{}{\settowidth\labelwidth{\ttfamily #1}%
%                               \setlength\leftmargin\labelwidth
%                               \addtolength\leftmargin\labelsep
%                               \let\makelabel\deflabel}}
% \def\enddeflist{\end{list}}
% \def\envfont{\normalfont\ttfamily}
%
% \MakeShortVerb{\|}
%
% \newcommand{\TUB}{{\sl TUGboat\/}}
%
% \GetFileInfo{theorem.dtx}
% \title{An Extension of the
%        \LaTeX{} theorem environment\thanks{This file
%        has version number \fileversion, last
%        revised \filedate.}}
% \author{Frank Mittelbach}
%
% \MaintainedByLaTeXTeam{tools}
% \maketitle
%
% \begin{abstract}
%    The macros described in this paper yield an extension of the
%    \LaTeX{} theorem mechanism. It is designed to satisfy the
%    different requirements of various journals. Thus, the
%    layout of the ``theorems'' can be manipulated by determining a
%    ``style''. This article describes not only the use, but
%    also the definition, of the necessary macros.
% \end{abstract}
%
% \section*{Preface to version 2.2}
%
% For \LaTeXe{} this package did not need any fundamental changes. I
% only modified the messages generated so that theorem layout styles
% will show up with the |\listfiles| command and cleaned the section
% on the New Font Selection Scheme since this is now included in
% \LaTeX.
%
% \section*{Preface to version 2.1}
%
% This version is identical to 2.0g described in \TUB\ 10\#3 except
% for some internal defaults which are now set depending on the used
% font selection scheme.
%
% This was done to avoid unpleasant surprises if the new font selection
% scheme is in force. For further details see
% section~\ref{sec:newfonts} and \cite{bk:GMS94}.
%
% \section{Introduction}
%
% For our purposes here, ``theorems'' are labelled enunciations,
% often set off from the main text by extra space and a font change.
% Theorems, corollaries, conjectures, definitions, and remarks
% are all instances of ``theorems''.  The ``header'' of these structures
% is composed of a label (such as \textsc{Theorem} or \textsc{Remark})
% and a number which serializes an item in the sequence of items
% with the same label.
%
% Shortly after the introduction of \LaTeX{} at the Fachbereich
% Mathematik in Mainz, the desire to manipulate the layout of
% ``theorems'' arose.  In Mainz, the following two conventions came
% into general use:
% \begin{enumerate}
%   \item The number of the theorem is shown in the margin.
%   \item There is a line break at
%         the end of the theorem header.
% \end{enumerate}
%
% Additionally, some journals require different formats which depend on
% the ``sort of theorem'': e.g.\ often remarks and definitions are set
% in |\upshape|, while |\itshape| is employed for main theorems.
%
% Confronted with these requirements, a theorem environment was
% developed in Mainz which allows separate determination of the
% layout of the ``theorems sets'', comparable to |\pagestyle|.
%
%
% \section{The user interface}
% \subsection{Defining new theorem sets}
%
% \DescribeMacro\newtheorem
% As in the original \LaTeX{} version, the command |\newtheorem|
% defines a new ``theorem set'' or ``theorem-like structure''.
% Two required arguments name the new environment and give the text
% to be typeset with each instance of the new ``set'', while
% an optional argument determines how
% the ``set'' is enumerated:
% \begin{description}
%    \item[\ttfamily \bslash newtheorem\{foo\}\{bar\}]
%       The theorem set {\envfont foo} (whose name is \texttt{bar})
%       uses its own counter.
%    \item[\ttfamily \bslash newtheorem\{foo2\}{[foo]}\{bar2\}]
%       The theorem set {\envfont foo2} (printed name \texttt{bar2})
%       uses the same counter as the theorem set \texttt{foo}.
%    \item[\ttfamily \bslash newtheorem\{foo3\}\{bar3\}{[section]}]
%       The theorem set {\envfont foo3} (printed name \texttt{bar3}) is
%       enumerated within the counter \texttt{section}, i.e.\ with every
%       new |\section| the enumeration begins again with 1, and
%       the enumeration is composed from the section-number and the
%       theorem counter itself.
% \end{description}
%
% \DescribeMacro\theoremstyle
% Additionally, the command |\theoremstyle| can define the layout
% of various, or all, theorem sets.  It should be noted that any
% theorem set defined by |\newtheorem| is typeset in the
% |\theoremstyle| that is current at the time of the definition.
% Thus, the following
% \begin{verbatim}
%    \theoremstyle{break}        \newtheorem{Cor}{Corollary}
%    \theoremstyle{plain}        \newtheorem{Exa}{Example}[section]
%\end{verbatim}                             ^^A No blank at line start
% leads to the result that the set {\envfont Cor} is formatted in the
% style \texttt{break}, while the set {\envfont Exa} and all the
% following ones are formatted in the style \texttt{plain}, unless
% another |\theoremstyle| follows.  Since the definitions installed by
% |\newtheorem| are global, one also can limit |\theoremstyle| locally
% by grouping braces.
%
% \DescribeMacro\theorembodyfont
% The choice of the font for the theorem body is completely
% independent of the chosen |\theoremstyle|; this has proven to
% be very advantageous.  For example,
% \begin{verbatim}
%   {\theorembodyfont{\upshape}          \newtheorem{Rem}{Remark}}
%\end{verbatim}
% defines a theorem set {\envfont Rem}, which will be set in
% |\upshape| in the current layout (which in our example is
% \texttt{plain}).  As with |\theoremstyle|, the |\theorembodyfont|
% chosen is that current at the time of |\newtheorem|.  If
% |\theorembodyfont| is not specified or one defines
% |\theorembodyfont{}|, then the font used will be that defined by the
% |\theoremstyle|.
%
% \DescribeMacro\theoremheaderfont It is also possible to customize
% the font used for the theorem headers.  This is, however, a global
% declaration, and therefore there should be at most one
% |\theoremheaderfont| declaration in the preamble.\footnote{If it is
% actually necessary to have different header fonts, one has to define
% new theorem styles (substituting the desired font) or specify the
% information directly in the \texttt{\bslash newtheorem} declaration
% (the unclean variant).}
%
% \DescribeMacro\theorempreskipamount
% \DescribeMacro\theorempostskipamount Two additional parameters
% affect the vertical space around the theorem environments:
% |\theorempreskipamount| and |\theorempostskipamount| define,
% respectively, the spacing before and after such an environment.
% These parameters apply for all theorem sets and can be manipulated
% with the ordinary length macros.  They are rubber lengths
% (`\textsf{skips}'), and therefore can contain \texttt{plus} and
% \texttt{minus} parts.
%
% Since the definition of theorem sets should---most sensibly---be
% placed in the preamble, we only allow installation there.
% It is therefore possible to release the memory used here after
% |\begin{document}|, in order to make room for other
% applications.
%
%
% \subsection{Existing theorem styles}
%
% The following theorem styles exist to date:
% \begin{deflist}{marginbreak}
%    \item[plain]
%       This theorem style emulates the original \LaTeX{} definition,
%       except that additionally the parameters
%       |\theorem...skipamount| are used.
%    \item[break]
%       In this style, the theorem header is followed by a line
%       break.
%    \item[marginbreak]
%       The theorem number is set in the margin, and there is a line
%       break as in \texttt{break}.
%    \item[changebreak]
%       Like \texttt{break}, but with header number and text
%       interchanged.
%    \item[change]
%       Header number and text are interchanged, without a line break.
%    \item[margin]
%       The number is set in the left margin, without a line break.
% \end{deflist}
% All styles (except \texttt{plain}) select |\slshape| as the default
% |\theorembodyfont|.
%
%
% \subsection{Examples}
%
% Given the above theorem sets {\envfont Cor}, {\envfont Exa}
% and {\envfont Rem},
% suppose that the preamble also contains the declarations:
% \begin{verbatim}
%    \theoremstyle{marginbreak}   \newtheorem{Lem}[Cor]{Lemma}
%    \theoremstyle{change}
%    \theorembodyfont{\itshape}        \newtheorem{Def}[Cor]{Definition}
%
%    \theoremheaderfont{\scshape}
%\end{verbatim}
% Then the following are some typical examples of the typeset output
% resulting from their use.
%
% \begin{Cor}
%    This is a sentence typeset in the theorem environment {\envfont
%    Cor}.
% \end{Cor}
%
% \begin{Exa}
%    This is a sentence typeset in the theorem environment {\envfont
%    Exa}.
% \end{Exa}
%
% \begin{Rem}
%    This is a sentence typeset in the theorem environment {\envfont
%    Rem}.
% \end{Rem}
%
% \begin{Lem}[Ben User]
%    This is a sentence typeset in the theorem environment {\envfont
%    Lem}.
% \end{Lem}
%
% \begin{Def}[Very impressive Definition]
%    This is a sentence typeset in the theorem environment {\envfont
%    Def}.
% \end{Def}
%
% The last two examples show the effect of the optional argument to
% a theorem environment (it is the text typeset in parentheses).
%
%
% \section{Special Considerations}
% \label{sec:newfonts}
%
% Theoremheader and body are implemented as a unit. This means that
% the |\theoremheaderfont|  will inherit characteristics of the
% |\theorembodyfont| in \LaTeXe. Thus, if for example
% |\theorembodyfont| is |\itshape| and |\theoremheaderfont|
% is |\bfseries| the font selected for the header will have the
% characteristics `bold extended italic'. If this is not desired one
% should set the |\theoremheaderfont| to something like
% \begin{verbatim}
%      \theoremheaderfont{\normalfont\bfseries}
%\end{verbatim}
% i.e.\ supplying all necessary font information explicitly.
%
%
%
% \section{Acknowledgements}
%
% The publication of this set of macros was only possible with the
% help of Christina Busse (translating the manuscript into English),
% Joachim Pense (playing the r\^ole of typist), Chris Rowley
% (looking everything over) and many others providing useful
% suggestions.
%
% \MaybeStop{%
%    \begin{thebibliography}{9}
%       \bibitem{bk:GMS94} \textsc{M.~Goossens}, \textsc{F.~Mittelbach}
%         and \textsc{A.~Samarin}.
%         \newblock The \LaTeX{} Companion.
%         \newblock
%         Addison-Wesley, Reading, Massachusetts, 1994.
%       \bibitem{latex:source} \textsc{Lamport, Leslie.} \newblock
%                               \texttt{latex.tex}, version 2.09, date
%                               Feb. 1990.
%    \end{thebibliography}
%    \PrintIndex
%    \PrintChanges}
%
%
% \section{The documentation driver file}
%
% The next bit of code contains the documentation driver file for
% \TeX{}, i.e., the file that will produce the documentation you are
% currently reading. It will be extracted from this file by the
% \texttt{docstrip} program. Since it is the first code in the file
% one can alternatively process this file directly with \LaTeXe{} to
% obtain the documentation.
%
%    \begin{macrocode}
%<*driver>
\documentclass{ltxdoc}

\usepackage{theorem}

%<+driver>% The next few lines define theorem sets which are used
%<+driver>% in the example section of the documentation.

\theoremstyle{break}       \newtheorem{Cor}{Corollary}
\theoremstyle{plain}       \newtheorem{Exa}{Example}[section]
{\theorembodyfont{\upshape}\newtheorem{Rem}{Remark}}
\theoremstyle{marginbreak} \newtheorem{Lem}[Cor]{Lemma}
\theoremstyle{change}
\theorembodyfont{\itshape} \newtheorem{Def}[Cor]{Definition}

\theoremheaderfont{\scshape}

\RecordChanges

\begin{document}
   \DocInput{theorem.dtx}
\end{document}
%</driver>
%    \end{macrocode}
%
% \section{Definition of the Macros}
%
% If the file has been loaded before, we abort immediately. If not the
% package announces itself (this is actually done at the very top of
% the file---the way it is done isn't good style so don't copy it).
% \changes{v2.0e}{89/07/19}{Spaces removed from `typeout'.}
% \changes{v2.2b}{95/11/19}{Announce moved to top of file}
%    \begin{macrocode}
%<*package>
%\@ifundefined{theorem@style}{}{\endinput}
%\def\FMithmInfo{1995/11/19 v2.2b Theorem extension package (FMi)}
%\ProvidesPackage{theorem}[\FMithmInfo]
%    \end{macrocode}
%
%
% \subsection{Definition of theorem styles and fonts}
%
% All the definitions in this file are done globally to allow
% inputting this file inside a group.
%
% \begin{macro}{\theoremstyle}
%    Before a theorem style can be installed, the chosen style must be
%    known. For that reason, we must test to see that
%    |\th@|\meta{style} is known or, more precisely, that it is
%    different from |\relax|. If the style is not known then
%    |\th@plain| is used.
%    \begin{macrocode}
\gdef\theoremstyle#1{%
   \@ifundefined{th@#1}{\@warning
          {Unknown theoremstyle `#1'. Using `plain'}%
          \theorem@style{plain}}%
%    \end{macrocode}
%    We save the theorem style to be used in the token register
%    |\theorem@style|.
%    \begin{macrocode}
      {\theorem@style{#1}}%
%    \end{macrocode}
%    Now we ``evaluate'' the theorem style: this means, we call the
%    macro |\th@|\meta{style} which will activate the relevant
%    definitions which are contained in a separate file.  This is done
%    in a group to suppress changes to the current font.  This could
%    otherwise pose problems together with the new font selection
%    scheme\footnote{When I printed the original article using the new
%    font selection scheme I ended with a document with slanted
%    typefaces (text headings and all) simply because one of the
%    theorem styles used \texttt{\bslash sl} at toplevel.} if the
%    |\th@|\meta{style} is evaluated a second time.
% \changes{v2.1a}{90/02/15}{Group added around `th@..}
% \changes{v2.1a}{90/02/15}{Reseting of `..@begintheorem removed.}
%    \begin{macrocode}
      \begingroup
        \csname th@\the\theorem@style \endcsname
      \endgroup}
%    \end{macrocode}
% \end{macro}
%
%
% \begin{macro}{\@begintheorem}
% \changes{v2.1a}{90/02/15}{Relaxed at toplevel.}
% \begin{macro}{\@opargbegintheorem}
% \changes{v2.1a}{90/02/15}{Relaxed at toplevel.}
%    We reset |\@begintheorem| and |\@opargbegintheorem| to
%    |\relax| since these commands are no longer necessary at
%    toplevel. This will save a few tokens.
%    \begin{macrocode}
\global\let\@begintheorem\relax
\global\let\@opargbegintheorem\relax
%    \end{macrocode}
% \end{macro}
% \end{macro}
%
% \begin{macro}{\theorem@style}
%    Obviously the \textsf{token} register used above has to be
%    allocated. To assure the utmost compatibility with the
%    original \LaTeX{} definition, we set the default theorem style to
%    \texttt{plain}, which implements the usual \LaTeX{} convention.
% \changes{v2.0f}{89/07/20}{Unnecessary `global' removed.}
%    \begin{macrocode}
\newtoks\theorem@style
\global\theorem@style{plain}
%    \end{macrocode}
% \end{macro}
%
% \begin{macro}{\theorembodyfont}
% \begin{macro}{\theorem@bodyfont}
%    If the |\theorembodyfont| is set by the user then it should not
%    interact with the default font set in the theorem style. When the
%    new font selection is in force this may happen if, for example,
%    the default is |\itshape| and the new |\theorembodyfont| is
%    |\sffamily|. So we add a |\reset@font| command in front of the
%    user definition.
% \changes{v2.0f}{89/07/20}{Unnecessary `global' removed.}
% \changes{v2.0g}{89/09/19}{`theoremfont to `theorembodyfont.}
% \changes{v2.1c}{92/06/22}{`reset@font added and logic changed.}
% \changes{v2.1d}{92/10/28}{Reset to empty if arg is empty}
%    \begin{macrocode}
\gdef\theorembodyfont#1{%
%    \end{macrocode}
%    We check if the argument supplied is empty and if so put nothing
%    into the |\theorem@bodyfont| token register to allow for
%    |\theorembodyfont{}| as a mean of using the default of the
%    current |\theoremstyle|.
%    \begin{macrocode}
   \def\@tempa{#1}%
   \ifx\@tempa\@empty
    \theorem@bodyfont{}%
   \else
    \theorem@bodyfont{\reset@font#1}%
   \fi
   }
\newtoks\theorem@bodyfont
\global\theorem@bodyfont{}
%    \end{macrocode}
% \end{macro}
% \end{macro}
%
%
% \begin{macro}{\theoremheaderfont}
% \changes{v2.0g}{89/09/19}{Macro added.}
%    The font for the theorem headers is handled differently because
%    this definition applies to all theorem styles.
%    \begin{macrocode}
\gdef\theoremheaderfont#1{\gdef\theorem@headerfont{#1}%
%    \end{macrocode}
%    After using the macro once it is redefined to produce an error
%    message.
%    \begin{macrocode}
       \gdef\theoremheaderfont##1{%
        \typeout{\string\theoremheaderfont\space should be used
                 only once.}}}
%    \end{macrocode}
% \end{macro}
%
% \begin{macro}{\theorem@headerfont}
% \changes{v2.0g}{89/09/19}{Macro added.}
%    To set the |\theorem@headerfont| default we first test if
%    the new fontselection scheme is in force.
% \changes{v2.1a}{90/02/15}{Support for new font selection.}
%    \begin{macrocode}
\ifx\upshape\undefined
%    \end{macrocode}
%    If not we define it to expand into |\bfseries|. We don't use
%    |\let| just in case a following style option redefines this
%    macro.
%    \begin{macrocode}
\gdef\theorem@headerfont{\bfseries}
%    \end{macrocode}
%    Otherwise we reset the current shape before calling |\bfseries|.
%    \begin{macrocode}
\else \gdef\theorem@headerfont{\normalfont\bfseries}\fi
%    \end{macrocode}
% \end{macro}
%
% \begin{macro}{\th@plain}
% \begin{macro}{\th@break}
% \begin{macro}{\th@marginbreak}
% \begin{macro}{\th@changebreak}
% \begin{macro}{\th@change}
% \begin{macro}{\th@margin}
%    The different styles are defined in macros such as
%    |\th@plain|.  Since memory space is precious in
%    ``non-Big-versions'', we have to avoid offering too many unused
%    definitions.  Therefore we define these styles in separate files
%    that can be loaded on demand.  Thus the commands themselves only
%    load these files. We use |\@input@|, a \LaTeXe{} internal command
%    that ensures that the file will be listed with |\listfiles|.
% \changes{v2.2a}{94/02/02}{Use `@input@ to load theorem layout files}
%    \begin{macrocode}
\gdef\th@plain{\@input@{thp.sty}}
\gdef\th@break{\@input@{thb.sty}}
\gdef\th@marginbreak{\@input@{thmb.sty}}
\gdef\th@changebreak{\@input@{thcb.sty}}
\gdef\th@change{\@input@{thc.sty}}
\gdef\th@margin{\@input@{thm.sty}}
%    \end{macrocode}
%    This list will be expanded when new styles become
%    available.  For testing,  just append new theorem substyles
%    as document options.
% \end{macro}
% \end{macro}
% \end{macro}
% \end{macro}
% \end{macro}
% \end{macro}
%
%
%
% \subsection{Definition of a new theorem set}
%
% As already pointed out, a new theorem environment can be defined in
% three different ways:
% \begin{verbatim}
%    \newtheorem{Lem}{Lemma}
%    \newtheorem{Lem}{Lemma}[section]
%    \newtheorem{Lem}[Theorem]{Lemma}
%\end{verbatim}
% The function of the macro |\newtheorem| is to recognize these cases
% and then to branch into one of the three macros |\@ynthm|, |\@xnthm|
% or |\@othm|.  This mechanism is adopted unchanged from
% \cite{latex:source}; the essential point here is that, for example,
% in the second case, the arguments \texttt{Lem}, \texttt{Lemma} and
% \texttt{section} are passed over to the macro |\@xnthm|.
%
% We inspect this case first because the others present fewer problems,
% and thus are easily derived from this one.
%
% \begin{macro}{\@xnthm}
%    For our example arguments, the macro |\@xnthm| must fulfill
%    the following:
%    \begin{itemize}
%       \item
%          Define a new \LaTeX-counter `Lem'
%       \item
%          reset this counter within a |\section|
%       \item
%          define the macro |\theLem|
%       \item
%          define the environment macros |\Lem| and
%          |\endLem| using the current
%          |\theoremstyle| and |\theorem@bodyfont|.
%    \end{itemize}
%    Obviously, all this should happen only if the first argument
%    of |\@xnthm| (i.e.\ \texttt{Lem} in our example) is chosen so
%    as not to conflict with any previously defined commands or
%    environments.  This test is performed by the \LaTeX{} macro
%    |\@ifdefinable|.
%    \begin{macrocode}
\gdef\@xnthm#1#2[#3]{\expandafter\@ifdefinable\csname #1\endcsname
%    \end{macrocode}
%    Therefore, the first argument of |\@ifdefinable| is the
%    expansion (in the example, |\Lem|) of
%    |\csname||#1\endcsname|.  The second argument is
%    executed only if the test has been completed successfully.
%    \begin{macrocode}
   {%
%    \end{macrocode}
%    Now we define the new counter. The names of the \LaTeX{}
%    macros employed should speak for themselves:
% \changes{v2.2b}{1995/10/07}{Use \cs{@newctr}}
% \changes{v2.2c}{1995/11/23}{Use \cs{@newctr} properly.}
%    \begin{macrocode}
    \@definecounter{#1}\@newctr{#1}[#3]%
%    \end{macrocode}
%    Using |\@newctr| will give a proper error message if the counter
%    in |#3| is not defined.
%    In defining `|\theLem|' we must generate the desired macro
%    name by use of |\expandafter| and |\csname|.
%    \begin{macrocode}
    \expandafter\xdef\csname the#1\endcsname
%    \end{macrocode}
%    An |\xdef| is used in order to make the definition global,
%    and to ensure that it contains the replacement texts of
%    |\@thmcountersep| and |\@thmcounter|.\footnote{These
%    two macros can be defined by the document style.  Their default
%    values produce a `\texttt{.}' as separation and an arabic
%    representation of the number.} However, not everything should be
%    expanded. For example, it saves space to use |\thesection|
%    instead of its---at times---lengthy expansion.
%    \begin{macrocode}
      {\expandafter \noexpand \csname the#3\endcsname
       \@thmcountersep \@thmcounter{#1}}%
%    \end{macrocode}
%    Thus with the defaults of \LaTeX{}, |\theLem| would be
%    replaced by the command sequence |\thesection.\arabic{Lem}|.
%
%    We will now look at the definition of the macro which is executed
%    at the beginning of the actual environment (in our example this
%    macro is |\Lem|). It should be noted that we use an
%    ``|\expandafter| trick'' to expand only certain parts of the
%    replacement text at the time of the definition.
%    \begin{macrocode}
    \def\@tempa{\global\@namedef{#1}}%
    \expandafter \@tempa \expandafter{%
%    \end{macrocode}
%    First, the macro that contains the current definitions of
%    |\@begintheorem| and |\@opargtheorem| should be called
%    up.  The name of this macro---as is already known---has the form
%    |\th@|\meta{theorem\ style}; therefore, it must be called by
%    \begin{macrocode}
      \csname th@\the \theorem@style
%    \end{macrocode}
%    In addition the default theorem font should be changeable, i.e.\
%    we have to insert the contents of |\theorem@bodyfont|. For
%    that reason, we expand even further, beyond |\endcsname|,
%    and thus insert the contents of the \textsf{token} register
%    |\theorem@bodyfont| in the replacement text.
%    \begin{macrocode}
            \expandafter \endcsname \the \theorem@bodyfont
%    \end{macrocode}
%    Now it is time to call the macro |\@thm| which takes over
%    the further processing.  It has two arguments: the current
%    counter name (in our example, \texttt{Lem}), and the text of the
%    label (in our example, \texttt{Lemma}).
%    \begin{macrocode}
     \@thm{#1}{#2}}%
%    \end{macrocode}
%    With this, the `sub-definition' is complete.  The macro
%    |\@endtheorem| ends a theorem environment and is, so far, nothing
%    but an |\endtrivlist|.  (Hence it is defined globally, and not
%    within the theorem styles.\footnote{This has to be changed as
%    soon as theorem styles that change \texttt{\bslash @endtheorem}
%    exist.  In such a case, all existing styles must be changed as
%    well since they will have to reset the macro.}) Therefore, we can
%    set it equivalent to the macro that ends the theorem set (in our
%    example, |\endLem|).  However, if some day theorem styles exist
%    that do change |\@endtheorem|, we would have to use the
%    commented-out line instead.
%    \begin{macrocode}
    \global \expandafter \let \csname end#1\endcsname \@endtheorem
%   \global\@namedef{end#1}{\@endtheorem}%
%    \end{macrocode}
%    With these commands all the required definitions are employed,
%    unless the test |\@ifdefinable| has failed.  Therefore, we
%    end the second argument of this macro and with it the definition
%    of |\@xnthm|.
%    \begin{macrocode}
   }}
%    \end{macrocode}
% \end{macro}
%
% \begin{macro}{\@ynthm}
%    The definition of |\@ynthm| is completely analogous. In
%    this case the new counter that is defined is not reset within
%    another counter; thus the definition of |\the...|
%    is simplified:
%    \begin{macrocode}
\gdef\@ynthm#1#2{\expandafter\@ifdefinable\csname #1\endcsname
   {\@definecounter{#1}%
    \expandafter\xdef\csname the#1\endcsname{\@thmcounter{#1}}%
%    \end{macrocode}
%    The rest of the definition corresponds literally to that of
%    |\@xnthm|:
%    \begin{macrocode}
    \def\@tempa{\global\@namedef{#1}}\expandafter \@tempa
     \expandafter{\csname th@\the \theorem@style \expandafter
     \endcsname \the\theorem@bodyfont \@thm{#1}{#2}}%
    \global \expandafter \let \csname end#1\endcsname \@endtheorem}}
%    \end{macrocode}
% \end{macro}
%
% \begin{macro}{\@othm}
%    The definition of |\@othm| does not contain anything new.
%    \begin{macrocode}
\gdef\@othm#1[#2]#3{%
%    \end{macrocode}
% \changes{v2.2b}{95/10/07}{Build in check for counter defined}
%    We do not define a new counter but instead use one that has
%    already been defined.  Thus the only definition we need is that
%    of this pseudo-counter (i.e.\ |\the|\meta{env.\ name}).
%    First we check if |#2| corresponds to a known counter name.
%    \begin{macrocode}
  \expandafter\ifx\csname c@#2\endcsname\relax
   \@nocounterr{#2}%
  \else
   \expandafter\@ifdefinable\csname #1\endcsname
   {\expandafter \xdef \csname the#1\endcsname
     {\expandafter \noexpand \csname the#2\endcsname}%
%    \end{macrocode}
%    All other parts of the definition can be adopted from
%    |\@xnthm|. We have to remember, though, that in this case
%    the name of the current counter and the theorem label have moved
%    to the second and third arguments.
%    \begin{macrocode}
    \def\@tempa{\global\@namedef{#1}}\expandafter \@tempa
     \expandafter{\csname th@\the \theorem@style \expandafter
     \endcsname \the\theorem@bodyfont \@thm{#2}{#3}}%
    \global \expandafter \let \csname end#1\endcsname \@endtheorem}%
  \fi}
%    \end{macrocode}
% \end{macro}
%
% \subsection{Macros that are employed in a {\envfont theorem}
%    environment}
%
% \begin{macro}{\@thm}
%    The macro |\@thm| has to increase the current counter.  Then,
%    depending on whether the environment has (or does not have) an
%    optional argument, it has to branch into either |\@begintheorem|
%    or |\@opargtheorem|.
%    \begin{macrocode}
\gdef\@thm#1#2{\refstepcounter{#1}%
%    \end{macrocode}
%    Now we start a {\envfont trivlist} environment, and give
%    |\@topsep| and |\@topsepadd| the values of the \textsf{skip}
%    registers |\theorempreskipamount| and |\theorempostskipamount|.
%    The value in |\@topsep| is the vertical space that is inserted by
%    the first (and only) |\item| in our |\trivlist| whilst
%    |\@topsepadd| is inserted by |\@endparenv| at the end of that
%    {\envfont trivlist} environment.  By using these registers, we
%    obtain the desired space around a {\envfont theorem} environment.
%    \begin{macrocode}
   \trivlist
   \@topsep \theorempreskipamount               % used by first \item
   \@topsepadd \theorempostskipamount           % used by \@endparenv
%    \end{macrocode}
%    Now we have to test whether an optional argument has been given.
% \changes{v2.0d}{89/07/18}{Forgotten percent added.}
%    \begin{macrocode}
   \@ifnextchar [%
%    \end{macrocode}
%    If there is an optional argument, we will call |\@ythm|, and
%    move the arguments read back into the input stream.
%    \begin{macrocode}
   {\@ythm{#1}{#2}}%
%    \end{macrocode}
%    If not, we call |\@begintheorem|. Its first argument is the
%    name of the theorem set (hence the second argument of
%    |\@thm|). Its second argument is the macro that produces the
%    current number.
%    \begin{macrocode}
   {\@begintheorem{#2}{\csname the#1\endcsname}\ignorespaces}}
%    \end{macrocode}
% \end{macro}
%
% \begin{macro}{\@xthm}
% \begin{macro}{\@ythm}
%    Both these macros were originally called by |\@thm|.  We do
%    not need |\@xthm| anymore, hence we reset it to
%    |\relax|. The definition of |\@ythm| has not changed at
%    all from its definition in \LaTeX. In order to make the macros
%    easier to understand, we will
%    nevertheless present it (commented out).
%    \begin{macrocode}
\global\let\@xthm\relax
% \def\@ythm#1#2[#3]{\@opargbegintheorem{#2}{\csname
%           the#1\endcsname}{#3}\ignorespaces}
%</package>
%    \end{macrocode}
%    The primitive |\ignorespaces| in |\@ythm| and
%    |\@thm| is needed to remove the spaces between the
%    |\begin{...}| and the actual text.
% \end{macro}
% \end{macro}
%
%
% \subsection{Definition of the theorem substyles}
%
% As already pointed out, the theorem substyles, defined below, are
% only loaded when necessary.  Note that all these substyles, except
% \texttt{plain}, have |\slshape| as the default body font.
%
% \subsubsection{The \texttt{plain} style}
%
% \label{dok:th@plain} As the following macros use |@|, we have
% to locally set the |\catcode| of this symbol to ``letter''.
% This happens within a group, so that we do not have to worry about
% which |\catcode| that symbol had before.
%    \begin{macrocode}
%<*thp>
\begingroup \makeatletter
%    \end{macrocode}
% Since we are now within a group, we must make all definitions
% globally.  First we make sure that \texttt{theorem.sty} is loaded.
% This will allow us to use this file as a document style option
% without having to call \texttt{theorem} itself as an option.  At the
% same time, we assure that at least version 2 is loaded, since
% |\theorem@style| was not defined in earlier versions.
%    \begin{macrocode}
\@ifundefined{theorem@style}{\input{theorem.sty}}{}
\ProvidesFile{thp.sty}
             [\FMithmInfo]
%    \end{macrocode}
%
%
% \begin{macro}{\th@plain}
% \begin{macro}{\@begintheorem}
% \begin{macro}{\@opargbegintheorem}
%    |\theoremstyle{plain}| corresponds to the original
%    definition, except that the distances to the surrounding text are
%    determined by the parameters |\theorempreskipamount| and
%    |\theorempostskipamount|.
%    First we set the default body font.
%    \begin{macrocode}
\gdef\th@plain{\normalfont\itshape
%    \end{macrocode}
%    Then we define |\@begintheorem| and
%    |\@opargbegintheorem|. These two macros define how the
%    header of a theorem is typeset.  |\@opargbegintheorem| will
%    be called if a \textsf{theorem} environment with an optional
%    argument is encountered; otherwise, the header is constructed by
%    calling |\@begintheorem|.  If one of these macros is
%    executed, we are within a \textsf{trivlist} environment started by
%    |\@thm|.  So the theorem header is produced with an
%    |\item| command.
%
%    Instead of specifying the header font directly, all standard
%    theorem styles use the |\theorem@headerfont| macro to allow
%    customization.  The extra space (|\labelsep|) is necessary
%    because of problems in the \textsf{trivlist} environment.
%    \begin{macrocode}
  \def\@begintheorem##1##2{%
        \item[\hskip\labelsep \theorem@headerfont ##1\ ##2]}%
%    \end{macrocode}
%    The definition of |\@opargbegintheorem| is completely
%    analogous. The only difference is the fact that there exists a
%    third argument (which is the optional parameter of the
%    environment and contains additional information about the
%    theorem).  Customarily we enclose it in parentheses.
%    \begin{macrocode}
\def\@opargbegintheorem##1##2##3{%
   \item[\hskip\labelsep \theorem@headerfont ##1\ ##2\ (##3)]}}
%    \end{macrocode}
% \end{macro}
% \end{macro}
% \end{macro}
%
% \noindent
% We conclude with an |\endgroup| to restore the
% |\catcode| of |@|.
%    \begin{macrocode}
\endgroup
%</thp>
%    \end{macrocode}
%
%
% \subsubsection{The \texttt{break} style}
%
% This style option is stored in the file |thb.sty|.
% For the next two lines see the documentation for |\th@plain| on
% page~\pageref{dok:th@plain}.
%    \begin{macrocode}
%<*thb>
\begingroup \makeatletter
\@ifundefined{theorem@style}{\input{theorem.sty}}{}
\ProvidesFile{thb.sty}
             [\FMithmInfo]
%    \end{macrocode}
%
% \begin{macro}{\th@break}
%    |\theoremstyle{break}| produces a line break after the name
%    of the theorem.  The font is |\slshape|.  Hence, we define
%    |\th@break| as follows:
%    \begin{macrocode}
\gdef\th@break{\normalfont\slshape
  \def\@begintheorem##1##2{\item[%
%    \end{macrocode}
%    We run into the following problem: it is not possible to create
%    the header with |\item[|\meta{title}|]| and then start
%    a new line by, for example, |\mbox{}\\|. Such a definition
%    will fail whenever a list environment follows immediately.  With
%    the above construction, the |\mbox{}| causes the switch
%    |@inlabel| (cf.\ definition of |\list| and
%    |\trivlist| in \cite{latex:source}) to be set to \texttt{false}
%    and so the following list will insert additional vertical
%    space (|\topskip|). This is quite annoying. Therefore, we
%    create the line break within the |\item|. In order to ensure
%    that the text will begin at the proper position in the following
%    line, we simply pretend that the label does not take any
%    room.\footnote{This will lead to problems whenever very high
%    symbols occurring in the line tower into the
%    heading. So, something else has to be done here sometime.}
%    \begin{macrocode}
       \rlap{\vbox{\hbox{\hskip \labelsep\theorem@headerfont ##1\ ##2}%
                   \hbox{\strut}}}]}%
%    \end{macrocode}
%    Again, the definition of |\@opargbegintheorem| is
%    completely analogous.
%    \begin{macrocode}
\def\@opargbegintheorem##1##2##3{%
  \item[\rlap{\vbox{\hbox{\hskip \labelsep \theorem@headerfont
                     ##1\ ##2\ (##3)}%
                    \hbox{\strut}}}]}}
\endgroup
%</thb>
%    \end{macrocode}
% \end{macro}
%
%
% \subsubsection{The \texttt{changebreak} style}
%
%    \begin{macrocode}
% This style option is stored in the file |thcb.sty|.
%    \begin{macrocode}
%<*thcb>
\begingroup \makeatletter
\@ifundefined{theorem@style}{\input{theorem.sty}}{}
\ProvidesFile{thcb.sty}
             [\FMithmInfo]
%    \end{macrocode}
%
% \begin{macro}{\th@changebreak}
%    The \texttt{change-break} theorem style is like \texttt{break}
%    but with interchange of theorem name and theorem number.  Thus we
%    define |\th@changebreak| as follows:
%    \begin{macrocode}
\gdef\th@changebreak{\normalfont\slshape
  \def\@begintheorem##1##2{\item
       [\rlap{\vbox{\hbox{\hskip\labelsep \theorem@headerfont ##2\ ##1}%
                    \hbox{\strut}}}]}%
\def\@opargbegintheorem##1##2##3{%
  \item[\rlap{\vbox{\hbox{\hskip\labelsep \theorem@headerfont
                     ##2\ ##1\ (##3)}%
                    \hbox{\strut}}}]}}
\endgroup
%</thcb>
%    \end{macrocode}
% \end{macro}
%
%
% \subsubsection{The \texttt{change} style}
%
% This style option is stored in the file |thc.sty|.
%    \begin{macrocode}
%<*thc>
\begingroup \makeatletter
\@ifundefined{theorem@style}{\input{theorem.sty}}{}
\ProvidesFile{thc.sty}
             [\FMithmInfo]
%    \end{macrocode}
%
% \begin{macro}{\th@change}
%    The \texttt{change} theorem style corresponds to the
%    \texttt{change break} style without a linebreak after the header.
%    To say it in another way, it's the same as the \texttt{plain}
%    style but with number and name interchanged and |\slshape| as the
%    default font.
%    \begin{macrocode}
\gdef\th@change{\normalfont\slshape
  \def\@begintheorem##1##2{\item
       [\hskip\labelsep \theorem@headerfont ##2\ ##1]}%
\def\@opargbegintheorem##1##2##3{%
  \item[\hskip\labelsep \theorem@headerfont ##2\ ##1\ (##3)]}}
\endgroup
%</thc>
%    \end{macrocode}
% \end{macro}
%
%
% \subsubsection{The \texttt{marginbreak} style}
%
% This style option is the one used most often at Mainz.  It is saved
% in the file |thmb.sty|.
%    \begin{macrocode}
%<*thmb>
\begingroup \makeatletter
\@ifundefined{theorem@style}{\input{theorem.sty}}{}
\ProvidesFile{thmb.sty}
             [\FMithmInfo]
%    \end{macrocode}
%
% \begin{macro}{\th@marginbreak}
%    The \texttt{margin break} style is nearly the same as the
%    \texttt{change break} style. The only difference is the placement
%    of the theorem number.  We use |\llap| to place it in the left
%    margin.
%
%    In this style |\labelsep| denotes the separation between the
%    number and the text.
%    \begin{macrocode}
\gdef\th@marginbreak{\normalfont\slshape
  \def\@begintheorem##1##2{\item
       [\rlap{\vbox{\theorem@headerfont
                    \hbox{\llap{##2}\hskip\labelsep ##1}%
                    \hbox{\strut}}}]}%
\def\@opargbegintheorem##1##2##3{%
  \item[\rlap{\vbox{\theorem@headerfont
                    \hbox{\llap{##2}\hskip\labelsep ##1\ (##3)}%
                    \hbox{\strut}}}]}}
\endgroup
%</thmb>
%    \end{macrocode}
% \changes{v2.1c}{90/02/17}{`endgroup outside macrocode env.}
% \end{macro}
%
%
%
% \subsubsection{The \texttt{margin} style}
%
% This style option is stored in the file |thm.sty|.
%    \begin{macrocode}
%<*thm>
\begingroup \makeatletter
\@ifundefined{theorem@style}{\input{theorem.sty}}{}
\ProvidesFile{thm.sty}
             [\FMithmInfo]
%    \end{macrocode}
%
% \begin{macro}{\th@margin}
%    Again this is only a variant of the theorem styles described
%    above without any new ideas.
%    \begin{macrocode}
\gdef\th@margin{\normalfont\slshape
  \def\@begintheorem##1##2{\item
       [\theorem@headerfont \llap{##2}\hskip\labelsep ##1]}%
\def\@opargbegintheorem##1##2##3{%
  \item[\theorem@headerfont \llap{##2}\hskip\labelsep ##1\ (##3)]}}
\endgroup
%</thm>
%    \end{macrocode}
% \end{macro}
%
%
%
%
% \subsection{Final Definitions}
%
% \begin{macro}{\theorempreskipamount}
% \begin{macro}{\theorempostskipamount}
%    The \textsf{skip} parameters that regulate the vertical empty space
%    before and after the {\envfont theorem} environment have
%    to be allocated as well.
% \changes{v2.0f}{89/07/20}{Unnecessary `global' removed.}
%    \begin{macrocode}
%<*package>
\newskip\theorempreskipamount
\newskip\theorempostskipamount
%    \end{macrocode}
%    Since we have used the same values for all theorem sets, we now can
%    assign them.
%    \begin{macrocode}
\global\setlength\theorempreskipamount{12pt plus 5pt minus 3pt}
\global\setlength\theorempostskipamount{8pt plus 3pt minus 1.5pt}
%    \end{macrocode}
% \end{macro}
% \end{macro}
%
%
% \begin{macro}{\@endtheorem}
%    The same holds for the macro |\@endtheorem|, which ends a
%    {\envfont theorem} environment. Since it is the same for all
%    theorem sets, it is removed from the macros |\th@|\meta{style}.
%    It simply ends the {\envfont trivlist} environment, which was
%    begun in |\@thm|.
%    \begin{macrocode}
\global\let\@endtheorem=\endtrivlist
%    \end{macrocode}
% \end{macro}
%
%
%
% \begin{macro}{\@preamblecmds}
%    All macros defined above are to be used only in the preamble.
%    Therefore, we insert them in |\@preamblecmds| which will disable
%    them at begin document. This is done by the internal \LaTeXe{}
%    command |\@onlypreamble|.
% \changes{v2.0g}{89/09/19}{`theoremheaderfont and `theorembodyfont
%    added.}
% \changes{v2.2a}{94/02/03}{Use \LaTeXe{} interface}
%    \begin{macrocode}
\@onlypreamble\@xnthm
\@onlypreamble\@ynthm
\@onlypreamble\@othm
\@onlypreamble\newtheorem
\@onlypreamble\theoremstyle
\@onlypreamble\theorembodyfont
\@onlypreamble\theoremheaderfont
%    \end{macrocode}
% \end{macro}
%
% Finally we declare the \texttt{plain} theorem style to be the default.
% \changes{v2.1b}{90/06/04}{Fixed missing `item by declaring default
%    style.}
%    \begin{macrocode}
\theoremstyle{plain}
%</package>
%    \end{macrocode}
%
% \Finale

\endinput
}{}
\ProvidesFile{thmb.sty}
             [\FMithmInfo]
%    \end{macrocode}
%
% \begin{macro}{\th@marginbreak}
%    The \texttt{margin break} style is nearly the same as the
%    \texttt{change break} style. The only difference is the placement
%    of the theorem number.  We use |\llap| to place it in the left
%    margin.
%
%    In this style |\labelsep| denotes the separation between the
%    number and the text.
%    \begin{macrocode}
\gdef\th@marginbreak{\normalfont\slshape
  \def\@begintheorem##1##2{\item
       [\rlap{\vbox{\theorem@headerfont
                    \hbox{\llap{##2}\hskip\labelsep ##1}%
                    \hbox{\strut}}}]}%
\def\@opargbegintheorem##1##2##3{%
  \item[\rlap{\vbox{\theorem@headerfont
                    \hbox{\llap{##2}\hskip\labelsep ##1\ (##3)}%
                    \hbox{\strut}}}]}}
\endgroup
%</thmb>
%    \end{macrocode}
% \changes{v2.1c}{90/02/17}{`endgroup outside macrocode env.}
% \end{macro}
%
%
%
% \subsubsection{The \texttt{margin} style}
%
% This style option is stored in the file |thm.sty|.
%    \begin{macrocode}
%<*thm>
\begingroup \makeatletter
\@ifundefined{theorem@style}{% \iffalse meta-comment
%
% Copyright (C) 1993-2024
%
% The LaTeX Project and any individual authors listed elsewhere
% in this file.
%
% This file is part of the Standard LaTeX `Tools Bundle'.
% -------------------------------------------------------
%
% It may be distributed and/or modified under the
% conditions of the LaTeX Project Public License, either version 1.3c
% of this license or (at your option) any later version.
% The latest version of this license is in
%    https://www.latex-project.org/lppl.txt
% and version 1.3c or later is part of all distributions of LaTeX
% version 2005/12/01 or later.
%
% The list of all files belonging to the LaTeX `Tools Bundle' is
% given in the file `manifest.txt'.
%
% \fi
%
%%
%
%
\def\FMithmInfo{2023/07/05 v2.2c Theorem extension package (FMi)}
%
% \ProvidesFile{theorem.dtx}[\FMithmInfo]
%\iffalse   % this is a METACOMMENT !
%%
%% Package `theorem' to use with LaTeX2e
%% Copyright (C) 1989-1995 Frank Mittelbach, all rights reserved.
%<+package>\@ifundefined{theorem@style}{}{\endinput}
%<+package>\NeedsTeXFormat{LaTeX2e}[1995/06/01]
%<+package>\ProvidesPackage{theorem}
%<+package>         [\FMithmInfo]
%
% \fi
%
% \changes{v1.0}{87/05/07}{First draft.}
% \changes{v1.0a}{87/05/07}{`thistheoremstyle' added.}
% \changes{v1.1a}{87/07/29}{New concept: redefine the internal
%                          \protect\LaTeX{} commands `@xnthm' etc.}
% \changes{v1.1b}{87/07/31}{`theoremfont' used to specify font.}
% \changes{v1.2a}{88/01/08}{`@endtheorem' now globally defined.}
% \changes{v1.2b}{88/01/11}{`newtheorem' only allowed in the preamble.}
% \changes{v1.3a}{88/02/14}{`theoremfont' now a toks register.}
%
% \changes{v2.0a}{88/11/16}{New implementation and new documentation.}
% \changes{v2.0a}{88/11/16}{No layout changes but internal modifications
%                           old substyles won't work.}
% \changes{v2.0b}{89/01/30}{`global' used for all definitions.}
% \changes{v2.0c}{89/06/15}{Documentation translated to English.
%                            Thanks to LLL and Nina}
% \changes{v2.1a}{90/02/15}{Support for new font selection.}
% \changes{v2.2a}{94/02/03}{Cleanup for LaTeX2e}
%
% \DoNotIndex{\@endparenv,\@ifdefinable,\@ifnextchar,\@ifundefined}
% \DoNotIndex{\@namedef,\@nameuse,\@spaces,\@tempa,\@warning}
% \DoNotIndex{\ }
% \DoNotIndex{\begingroup,\bfseries}
% \DoNotIndex{\csname}
% \DoNotIndex{\def,\do,\docdate}
% \DoNotIndex{\endcsname,\endgroup,\endinput,\endtrivlist,\expandafter}
% \DoNotIndex{\filedate,\fileversion}
% \DoNotIndex{\gdef,\global}
% \DoNotIndex{\hbox,\hskip}
% \DoNotIndex{\ignorespaces,\input,\itshape,\item}
% \DoNotIndex{\labelsep,\let,\llap}
% \DoNotIndex{\makeatletter}
% \DoNotIndex{\noexpand}
% \DoNotIndex{\refstepcounter,\relax,\rlap}
% \DoNotIndex{\setlength,\slshape,\space,\string,\strut}
% \DoNotIndex{\the,\trivlist,\typeout}
% \DoNotIndex{\vbox}
% \DoNotIndex{\xdef}
%
% \def\deflabel#1{\ttfamily #1\hfill}
% \def\deflist#1{\begin{list}{}{\settowidth\labelwidth{\ttfamily #1}%
%                               \setlength\leftmargin\labelwidth
%                               \addtolength\leftmargin\labelsep
%                               \let\makelabel\deflabel}}
% \def\enddeflist{\end{list}}
% \def\envfont{\normalfont\ttfamily}
%
% \MakeShortVerb{\|}
%
% \newcommand{\TUB}{{\sl TUGboat\/}}
%
% \GetFileInfo{theorem.dtx}
% \title{An Extension of the
%        \LaTeX{} theorem environment\thanks{This file
%        has version number \fileversion, last
%        revised \filedate.}}
% \author{Frank Mittelbach}
%
% \MaintainedByLaTeXTeam{tools}
% \maketitle
%
% \begin{abstract}
%    The macros described in this paper yield an extension of the
%    \LaTeX{} theorem mechanism. It is designed to satisfy the
%    different requirements of various journals. Thus, the
%    layout of the ``theorems'' can be manipulated by determining a
%    ``style''. This article describes not only the use, but
%    also the definition, of the necessary macros.
% \end{abstract}
%
% \section*{Preface to version 2.2}
%
% For \LaTeXe{} this package did not need any fundamental changes. I
% only modified the messages generated so that theorem layout styles
% will show up with the |\listfiles| command and cleaned the section
% on the New Font Selection Scheme since this is now included in
% \LaTeX.
%
% \section*{Preface to version 2.1}
%
% This version is identical to 2.0g described in \TUB\ 10\#3 except
% for some internal defaults which are now set depending on the used
% font selection scheme.
%
% This was done to avoid unpleasant surprises if the new font selection
% scheme is in force. For further details see
% section~\ref{sec:newfonts} and \cite{bk:GMS94}.
%
% \section{Introduction}
%
% For our purposes here, ``theorems'' are labelled enunciations,
% often set off from the main text by extra space and a font change.
% Theorems, corollaries, conjectures, definitions, and remarks
% are all instances of ``theorems''.  The ``header'' of these structures
% is composed of a label (such as \textsc{Theorem} or \textsc{Remark})
% and a number which serializes an item in the sequence of items
% with the same label.
%
% Shortly after the introduction of \LaTeX{} at the Fachbereich
% Mathematik in Mainz, the desire to manipulate the layout of
% ``theorems'' arose.  In Mainz, the following two conventions came
% into general use:
% \begin{enumerate}
%   \item The number of the theorem is shown in the margin.
%   \item There is a line break at
%         the end of the theorem header.
% \end{enumerate}
%
% Additionally, some journals require different formats which depend on
% the ``sort of theorem'': e.g.\ often remarks and definitions are set
% in |\upshape|, while |\itshape| is employed for main theorems.
%
% Confronted with these requirements, a theorem environment was
% developed in Mainz which allows separate determination of the
% layout of the ``theorems sets'', comparable to |\pagestyle|.
%
%
% \section{The user interface}
% \subsection{Defining new theorem sets}
%
% \DescribeMacro\newtheorem
% As in the original \LaTeX{} version, the command |\newtheorem|
% defines a new ``theorem set'' or ``theorem-like structure''.
% Two required arguments name the new environment and give the text
% to be typeset with each instance of the new ``set'', while
% an optional argument determines how
% the ``set'' is enumerated:
% \begin{description}
%    \item[\ttfamily \bslash newtheorem\{foo\}\{bar\}]
%       The theorem set {\envfont foo} (whose name is \texttt{bar})
%       uses its own counter.
%    \item[\ttfamily \bslash newtheorem\{foo2\}{[foo]}\{bar2\}]
%       The theorem set {\envfont foo2} (printed name \texttt{bar2})
%       uses the same counter as the theorem set \texttt{foo}.
%    \item[\ttfamily \bslash newtheorem\{foo3\}\{bar3\}{[section]}]
%       The theorem set {\envfont foo3} (printed name \texttt{bar3}) is
%       enumerated within the counter \texttt{section}, i.e.\ with every
%       new |\section| the enumeration begins again with 1, and
%       the enumeration is composed from the section-number and the
%       theorem counter itself.
% \end{description}
%
% \DescribeMacro\theoremstyle
% Additionally, the command |\theoremstyle| can define the layout
% of various, or all, theorem sets.  It should be noted that any
% theorem set defined by |\newtheorem| is typeset in the
% |\theoremstyle| that is current at the time of the definition.
% Thus, the following
% \begin{verbatim}
%    \theoremstyle{break}        \newtheorem{Cor}{Corollary}
%    \theoremstyle{plain}        \newtheorem{Exa}{Example}[section]
%\end{verbatim}                             ^^A No blank at line start
% leads to the result that the set {\envfont Cor} is formatted in the
% style \texttt{break}, while the set {\envfont Exa} and all the
% following ones are formatted in the style \texttt{plain}, unless
% another |\theoremstyle| follows.  Since the definitions installed by
% |\newtheorem| are global, one also can limit |\theoremstyle| locally
% by grouping braces.
%
% \DescribeMacro\theorembodyfont
% The choice of the font for the theorem body is completely
% independent of the chosen |\theoremstyle|; this has proven to
% be very advantageous.  For example,
% \begin{verbatim}
%   {\theorembodyfont{\upshape}          \newtheorem{Rem}{Remark}}
%\end{verbatim}
% defines a theorem set {\envfont Rem}, which will be set in
% |\upshape| in the current layout (which in our example is
% \texttt{plain}).  As with |\theoremstyle|, the |\theorembodyfont|
% chosen is that current at the time of |\newtheorem|.  If
% |\theorembodyfont| is not specified or one defines
% |\theorembodyfont{}|, then the font used will be that defined by the
% |\theoremstyle|.
%
% \DescribeMacro\theoremheaderfont It is also possible to customize
% the font used for the theorem headers.  This is, however, a global
% declaration, and therefore there should be at most one
% |\theoremheaderfont| declaration in the preamble.\footnote{If it is
% actually necessary to have different header fonts, one has to define
% new theorem styles (substituting the desired font) or specify the
% information directly in the \texttt{\bslash newtheorem} declaration
% (the unclean variant).}
%
% \DescribeMacro\theorempreskipamount
% \DescribeMacro\theorempostskipamount Two additional parameters
% affect the vertical space around the theorem environments:
% |\theorempreskipamount| and |\theorempostskipamount| define,
% respectively, the spacing before and after such an environment.
% These parameters apply for all theorem sets and can be manipulated
% with the ordinary length macros.  They are rubber lengths
% (`\textsf{skips}'), and therefore can contain \texttt{plus} and
% \texttt{minus} parts.
%
% Since the definition of theorem sets should---most sensibly---be
% placed in the preamble, we only allow installation there.
% It is therefore possible to release the memory used here after
% |\begin{document}|, in order to make room for other
% applications.
%
%
% \subsection{Existing theorem styles}
%
% The following theorem styles exist to date:
% \begin{deflist}{marginbreak}
%    \item[plain]
%       This theorem style emulates the original \LaTeX{} definition,
%       except that additionally the parameters
%       |\theorem...skipamount| are used.
%    \item[break]
%       In this style, the theorem header is followed by a line
%       break.
%    \item[marginbreak]
%       The theorem number is set in the margin, and there is a line
%       break as in \texttt{break}.
%    \item[changebreak]
%       Like \texttt{break}, but with header number and text
%       interchanged.
%    \item[change]
%       Header number and text are interchanged, without a line break.
%    \item[margin]
%       The number is set in the left margin, without a line break.
% \end{deflist}
% All styles (except \texttt{plain}) select |\slshape| as the default
% |\theorembodyfont|.
%
%
% \subsection{Examples}
%
% Given the above theorem sets {\envfont Cor}, {\envfont Exa}
% and {\envfont Rem},
% suppose that the preamble also contains the declarations:
% \begin{verbatim}
%    \theoremstyle{marginbreak}   \newtheorem{Lem}[Cor]{Lemma}
%    \theoremstyle{change}
%    \theorembodyfont{\itshape}        \newtheorem{Def}[Cor]{Definition}
%
%    \theoremheaderfont{\scshape}
%\end{verbatim}
% Then the following are some typical examples of the typeset output
% resulting from their use.
%
% \begin{Cor}
%    This is a sentence typeset in the theorem environment {\envfont
%    Cor}.
% \end{Cor}
%
% \begin{Exa}
%    This is a sentence typeset in the theorem environment {\envfont
%    Exa}.
% \end{Exa}
%
% \begin{Rem}
%    This is a sentence typeset in the theorem environment {\envfont
%    Rem}.
% \end{Rem}
%
% \begin{Lem}[Ben User]
%    This is a sentence typeset in the theorem environment {\envfont
%    Lem}.
% \end{Lem}
%
% \begin{Def}[Very impressive Definition]
%    This is a sentence typeset in the theorem environment {\envfont
%    Def}.
% \end{Def}
%
% The last two examples show the effect of the optional argument to
% a theorem environment (it is the text typeset in parentheses).
%
%
% \section{Special Considerations}
% \label{sec:newfonts}
%
% Theoremheader and body are implemented as a unit. This means that
% the |\theoremheaderfont|  will inherit characteristics of the
% |\theorembodyfont| in \LaTeXe. Thus, if for example
% |\theorembodyfont| is |\itshape| and |\theoremheaderfont|
% is |\bfseries| the font selected for the header will have the
% characteristics `bold extended italic'. If this is not desired one
% should set the |\theoremheaderfont| to something like
% \begin{verbatim}
%      \theoremheaderfont{\normalfont\bfseries}
%\end{verbatim}
% i.e.\ supplying all necessary font information explicitly.
%
%
%
% \section{Acknowledgements}
%
% The publication of this set of macros was only possible with the
% help of Christina Busse (translating the manuscript into English),
% Joachim Pense (playing the r\^ole of typist), Chris Rowley
% (looking everything over) and many others providing useful
% suggestions.
%
% \MaybeStop{%
%    \begin{thebibliography}{9}
%       \bibitem{bk:GMS94} \textsc{M.~Goossens}, \textsc{F.~Mittelbach}
%         and \textsc{A.~Samarin}.
%         \newblock The \LaTeX{} Companion.
%         \newblock
%         Addison-Wesley, Reading, Massachusetts, 1994.
%       \bibitem{latex:source} \textsc{Lamport, Leslie.} \newblock
%                               \texttt{latex.tex}, version 2.09, date
%                               Feb. 1990.
%    \end{thebibliography}
%    \PrintIndex
%    \PrintChanges}
%
%
% \section{The documentation driver file}
%
% The next bit of code contains the documentation driver file for
% \TeX{}, i.e., the file that will produce the documentation you are
% currently reading. It will be extracted from this file by the
% \texttt{docstrip} program. Since it is the first code in the file
% one can alternatively process this file directly with \LaTeXe{} to
% obtain the documentation.
%
%    \begin{macrocode}
%<*driver>
\documentclass{ltxdoc}

\usepackage{theorem}

%<+driver>% The next few lines define theorem sets which are used
%<+driver>% in the example section of the documentation.

\theoremstyle{break}       \newtheorem{Cor}{Corollary}
\theoremstyle{plain}       \newtheorem{Exa}{Example}[section]
{\theorembodyfont{\upshape}\newtheorem{Rem}{Remark}}
\theoremstyle{marginbreak} \newtheorem{Lem}[Cor]{Lemma}
\theoremstyle{change}
\theorembodyfont{\itshape} \newtheorem{Def}[Cor]{Definition}

\theoremheaderfont{\scshape}

\RecordChanges

\begin{document}
   \DocInput{theorem.dtx}
\end{document}
%</driver>
%    \end{macrocode}
%
% \section{Definition of the Macros}
%
% If the file has been loaded before, we abort immediately. If not the
% package announces itself (this is actually done at the very top of
% the file---the way it is done isn't good style so don't copy it).
% \changes{v2.0e}{89/07/19}{Spaces removed from `typeout'.}
% \changes{v2.2b}{95/11/19}{Announce moved to top of file}
%    \begin{macrocode}
%<*package>
%\@ifundefined{theorem@style}{}{\endinput}
%\def\FMithmInfo{1995/11/19 v2.2b Theorem extension package (FMi)}
%\ProvidesPackage{theorem}[\FMithmInfo]
%    \end{macrocode}
%
%
% \subsection{Definition of theorem styles and fonts}
%
% All the definitions in this file are done globally to allow
% inputting this file inside a group.
%
% \begin{macro}{\theoremstyle}
%    Before a theorem style can be installed, the chosen style must be
%    known. For that reason, we must test to see that
%    |\th@|\meta{style} is known or, more precisely, that it is
%    different from |\relax|. If the style is not known then
%    |\th@plain| is used.
%    \begin{macrocode}
\gdef\theoremstyle#1{%
   \@ifundefined{th@#1}{\@warning
          {Unknown theoremstyle `#1'. Using `plain'}%
          \theorem@style{plain}}%
%    \end{macrocode}
%    We save the theorem style to be used in the token register
%    |\theorem@style|.
%    \begin{macrocode}
      {\theorem@style{#1}}%
%    \end{macrocode}
%    Now we ``evaluate'' the theorem style: this means, we call the
%    macro |\th@|\meta{style} which will activate the relevant
%    definitions which are contained in a separate file.  This is done
%    in a group to suppress changes to the current font.  This could
%    otherwise pose problems together with the new font selection
%    scheme\footnote{When I printed the original article using the new
%    font selection scheme I ended with a document with slanted
%    typefaces (text headings and all) simply because one of the
%    theorem styles used \texttt{\bslash sl} at toplevel.} if the
%    |\th@|\meta{style} is evaluated a second time.
% \changes{v2.1a}{90/02/15}{Group added around `th@..}
% \changes{v2.1a}{90/02/15}{Reseting of `..@begintheorem removed.}
%    \begin{macrocode}
      \begingroup
        \csname th@\the\theorem@style \endcsname
      \endgroup}
%    \end{macrocode}
% \end{macro}
%
%
% \begin{macro}{\@begintheorem}
% \changes{v2.1a}{90/02/15}{Relaxed at toplevel.}
% \begin{macro}{\@opargbegintheorem}
% \changes{v2.1a}{90/02/15}{Relaxed at toplevel.}
%    We reset |\@begintheorem| and |\@opargbegintheorem| to
%    |\relax| since these commands are no longer necessary at
%    toplevel. This will save a few tokens.
%    \begin{macrocode}
\global\let\@begintheorem\relax
\global\let\@opargbegintheorem\relax
%    \end{macrocode}
% \end{macro}
% \end{macro}
%
% \begin{macro}{\theorem@style}
%    Obviously the \textsf{token} register used above has to be
%    allocated. To assure the utmost compatibility with the
%    original \LaTeX{} definition, we set the default theorem style to
%    \texttt{plain}, which implements the usual \LaTeX{} convention.
% \changes{v2.0f}{89/07/20}{Unnecessary `global' removed.}
%    \begin{macrocode}
\newtoks\theorem@style
\global\theorem@style{plain}
%    \end{macrocode}
% \end{macro}
%
% \begin{macro}{\theorembodyfont}
% \begin{macro}{\theorem@bodyfont}
%    If the |\theorembodyfont| is set by the user then it should not
%    interact with the default font set in the theorem style. When the
%    new font selection is in force this may happen if, for example,
%    the default is |\itshape| and the new |\theorembodyfont| is
%    |\sffamily|. So we add a |\reset@font| command in front of the
%    user definition.
% \changes{v2.0f}{89/07/20}{Unnecessary `global' removed.}
% \changes{v2.0g}{89/09/19}{`theoremfont to `theorembodyfont.}
% \changes{v2.1c}{92/06/22}{`reset@font added and logic changed.}
% \changes{v2.1d}{92/10/28}{Reset to empty if arg is empty}
%    \begin{macrocode}
\gdef\theorembodyfont#1{%
%    \end{macrocode}
%    We check if the argument supplied is empty and if so put nothing
%    into the |\theorem@bodyfont| token register to allow for
%    |\theorembodyfont{}| as a mean of using the default of the
%    current |\theoremstyle|.
%    \begin{macrocode}
   \def\@tempa{#1}%
   \ifx\@tempa\@empty
    \theorem@bodyfont{}%
   \else
    \theorem@bodyfont{\reset@font#1}%
   \fi
   }
\newtoks\theorem@bodyfont
\global\theorem@bodyfont{}
%    \end{macrocode}
% \end{macro}
% \end{macro}
%
%
% \begin{macro}{\theoremheaderfont}
% \changes{v2.0g}{89/09/19}{Macro added.}
%    The font for the theorem headers is handled differently because
%    this definition applies to all theorem styles.
%    \begin{macrocode}
\gdef\theoremheaderfont#1{\gdef\theorem@headerfont{#1}%
%    \end{macrocode}
%    After using the macro once it is redefined to produce an error
%    message.
%    \begin{macrocode}
       \gdef\theoremheaderfont##1{%
        \typeout{\string\theoremheaderfont\space should be used
                 only once.}}}
%    \end{macrocode}
% \end{macro}
%
% \begin{macro}{\theorem@headerfont}
% \changes{v2.0g}{89/09/19}{Macro added.}
%    To set the |\theorem@headerfont| default we first test if
%    the new fontselection scheme is in force.
% \changes{v2.1a}{90/02/15}{Support for new font selection.}
%    \begin{macrocode}
\ifx\upshape\undefined
%    \end{macrocode}
%    If not we define it to expand into |\bfseries|. We don't use
%    |\let| just in case a following style option redefines this
%    macro.
%    \begin{macrocode}
\gdef\theorem@headerfont{\bfseries}
%    \end{macrocode}
%    Otherwise we reset the current shape before calling |\bfseries|.
%    \begin{macrocode}
\else \gdef\theorem@headerfont{\normalfont\bfseries}\fi
%    \end{macrocode}
% \end{macro}
%
% \begin{macro}{\th@plain}
% \begin{macro}{\th@break}
% \begin{macro}{\th@marginbreak}
% \begin{macro}{\th@changebreak}
% \begin{macro}{\th@change}
% \begin{macro}{\th@margin}
%    The different styles are defined in macros such as
%    |\th@plain|.  Since memory space is precious in
%    ``non-Big-versions'', we have to avoid offering too many unused
%    definitions.  Therefore we define these styles in separate files
%    that can be loaded on demand.  Thus the commands themselves only
%    load these files. We use |\@input@|, a \LaTeXe{} internal command
%    that ensures that the file will be listed with |\listfiles|.
% \changes{v2.2a}{94/02/02}{Use `@input@ to load theorem layout files}
%    \begin{macrocode}
\gdef\th@plain{\@input@{thp.sty}}
\gdef\th@break{\@input@{thb.sty}}
\gdef\th@marginbreak{\@input@{thmb.sty}}
\gdef\th@changebreak{\@input@{thcb.sty}}
\gdef\th@change{\@input@{thc.sty}}
\gdef\th@margin{\@input@{thm.sty}}
%    \end{macrocode}
%    This list will be expanded when new styles become
%    available.  For testing,  just append new theorem substyles
%    as document options.
% \end{macro}
% \end{macro}
% \end{macro}
% \end{macro}
% \end{macro}
% \end{macro}
%
%
%
% \subsection{Definition of a new theorem set}
%
% As already pointed out, a new theorem environment can be defined in
% three different ways:
% \begin{verbatim}
%    \newtheorem{Lem}{Lemma}
%    \newtheorem{Lem}{Lemma}[section]
%    \newtheorem{Lem}[Theorem]{Lemma}
%\end{verbatim}
% The function of the macro |\newtheorem| is to recognize these cases
% and then to branch into one of the three macros |\@ynthm|, |\@xnthm|
% or |\@othm|.  This mechanism is adopted unchanged from
% \cite{latex:source}; the essential point here is that, for example,
% in the second case, the arguments \texttt{Lem}, \texttt{Lemma} and
% \texttt{section} are passed over to the macro |\@xnthm|.
%
% We inspect this case first because the others present fewer problems,
% and thus are easily derived from this one.
%
% \begin{macro}{\@xnthm}
%    For our example arguments, the macro |\@xnthm| must fulfill
%    the following:
%    \begin{itemize}
%       \item
%          Define a new \LaTeX-counter `Lem'
%       \item
%          reset this counter within a |\section|
%       \item
%          define the macro |\theLem|
%       \item
%          define the environment macros |\Lem| and
%          |\endLem| using the current
%          |\theoremstyle| and |\theorem@bodyfont|.
%    \end{itemize}
%    Obviously, all this should happen only if the first argument
%    of |\@xnthm| (i.e.\ \texttt{Lem} in our example) is chosen so
%    as not to conflict with any previously defined commands or
%    environments.  This test is performed by the \LaTeX{} macro
%    |\@ifdefinable|.
%    \begin{macrocode}
\gdef\@xnthm#1#2[#3]{\expandafter\@ifdefinable\csname #1\endcsname
%    \end{macrocode}
%    Therefore, the first argument of |\@ifdefinable| is the
%    expansion (in the example, |\Lem|) of
%    |\csname||#1\endcsname|.  The second argument is
%    executed only if the test has been completed successfully.
%    \begin{macrocode}
   {%
%    \end{macrocode}
%    Now we define the new counter. The names of the \LaTeX{}
%    macros employed should speak for themselves:
% \changes{v2.2b}{1995/10/07}{Use \cs{@newctr}}
% \changes{v2.2c}{1995/11/23}{Use \cs{@newctr} properly.}
%    \begin{macrocode}
    \@definecounter{#1}\@newctr{#1}[#3]%
%    \end{macrocode}
%    Using |\@newctr| will give a proper error message if the counter
%    in |#3| is not defined.
%    In defining `|\theLem|' we must generate the desired macro
%    name by use of |\expandafter| and |\csname|.
%    \begin{macrocode}
    \expandafter\xdef\csname the#1\endcsname
%    \end{macrocode}
%    An |\xdef| is used in order to make the definition global,
%    and to ensure that it contains the replacement texts of
%    |\@thmcountersep| and |\@thmcounter|.\footnote{These
%    two macros can be defined by the document style.  Their default
%    values produce a `\texttt{.}' as separation and an arabic
%    representation of the number.} However, not everything should be
%    expanded. For example, it saves space to use |\thesection|
%    instead of its---at times---lengthy expansion.
%    \begin{macrocode}
      {\expandafter \noexpand \csname the#3\endcsname
       \@thmcountersep \@thmcounter{#1}}%
%    \end{macrocode}
%    Thus with the defaults of \LaTeX{}, |\theLem| would be
%    replaced by the command sequence |\thesection.\arabic{Lem}|.
%
%    We will now look at the definition of the macro which is executed
%    at the beginning of the actual environment (in our example this
%    macro is |\Lem|). It should be noted that we use an
%    ``|\expandafter| trick'' to expand only certain parts of the
%    replacement text at the time of the definition.
%    \begin{macrocode}
    \def\@tempa{\global\@namedef{#1}}%
    \expandafter \@tempa \expandafter{%
%    \end{macrocode}
%    First, the macro that contains the current definitions of
%    |\@begintheorem| and |\@opargtheorem| should be called
%    up.  The name of this macro---as is already known---has the form
%    |\th@|\meta{theorem\ style}; therefore, it must be called by
%    \begin{macrocode}
      \csname th@\the \theorem@style
%    \end{macrocode}
%    In addition the default theorem font should be changeable, i.e.\
%    we have to insert the contents of |\theorem@bodyfont|. For
%    that reason, we expand even further, beyond |\endcsname|,
%    and thus insert the contents of the \textsf{token} register
%    |\theorem@bodyfont| in the replacement text.
%    \begin{macrocode}
            \expandafter \endcsname \the \theorem@bodyfont
%    \end{macrocode}
%    Now it is time to call the macro |\@thm| which takes over
%    the further processing.  It has two arguments: the current
%    counter name (in our example, \texttt{Lem}), and the text of the
%    label (in our example, \texttt{Lemma}).
%    \begin{macrocode}
     \@thm{#1}{#2}}%
%    \end{macrocode}
%    With this, the `sub-definition' is complete.  The macro
%    |\@endtheorem| ends a theorem environment and is, so far, nothing
%    but an |\endtrivlist|.  (Hence it is defined globally, and not
%    within the theorem styles.\footnote{This has to be changed as
%    soon as theorem styles that change \texttt{\bslash @endtheorem}
%    exist.  In such a case, all existing styles must be changed as
%    well since they will have to reset the macro.}) Therefore, we can
%    set it equivalent to the macro that ends the theorem set (in our
%    example, |\endLem|).  However, if some day theorem styles exist
%    that do change |\@endtheorem|, we would have to use the
%    commented-out line instead.
%    \begin{macrocode}
    \global \expandafter \let \csname end#1\endcsname \@endtheorem
%   \global\@namedef{end#1}{\@endtheorem}%
%    \end{macrocode}
%    With these commands all the required definitions are employed,
%    unless the test |\@ifdefinable| has failed.  Therefore, we
%    end the second argument of this macro and with it the definition
%    of |\@xnthm|.
%    \begin{macrocode}
   }}
%    \end{macrocode}
% \end{macro}
%
% \begin{macro}{\@ynthm}
%    The definition of |\@ynthm| is completely analogous. In
%    this case the new counter that is defined is not reset within
%    another counter; thus the definition of |\the...|
%    is simplified:
%    \begin{macrocode}
\gdef\@ynthm#1#2{\expandafter\@ifdefinable\csname #1\endcsname
   {\@definecounter{#1}%
    \expandafter\xdef\csname the#1\endcsname{\@thmcounter{#1}}%
%    \end{macrocode}
%    The rest of the definition corresponds literally to that of
%    |\@xnthm|:
%    \begin{macrocode}
    \def\@tempa{\global\@namedef{#1}}\expandafter \@tempa
     \expandafter{\csname th@\the \theorem@style \expandafter
     \endcsname \the\theorem@bodyfont \@thm{#1}{#2}}%
    \global \expandafter \let \csname end#1\endcsname \@endtheorem}}
%    \end{macrocode}
% \end{macro}
%
% \begin{macro}{\@othm}
%    The definition of |\@othm| does not contain anything new.
%    \begin{macrocode}
\gdef\@othm#1[#2]#3{%
%    \end{macrocode}
% \changes{v2.2b}{95/10/07}{Build in check for counter defined}
%    We do not define a new counter but instead use one that has
%    already been defined.  Thus the only definition we need is that
%    of this pseudo-counter (i.e.\ |\the|\meta{env.\ name}).
%    First we check if |#2| corresponds to a known counter name.
%    \begin{macrocode}
  \expandafter\ifx\csname c@#2\endcsname\relax
   \@nocounterr{#2}%
  \else
   \expandafter\@ifdefinable\csname #1\endcsname
   {\expandafter \xdef \csname the#1\endcsname
     {\expandafter \noexpand \csname the#2\endcsname}%
%    \end{macrocode}
%    All other parts of the definition can be adopted from
%    |\@xnthm|. We have to remember, though, that in this case
%    the name of the current counter and the theorem label have moved
%    to the second and third arguments.
%    \begin{macrocode}
    \def\@tempa{\global\@namedef{#1}}\expandafter \@tempa
     \expandafter{\csname th@\the \theorem@style \expandafter
     \endcsname \the\theorem@bodyfont \@thm{#2}{#3}}%
    \global \expandafter \let \csname end#1\endcsname \@endtheorem}%
  \fi}
%    \end{macrocode}
% \end{macro}
%
% \subsection{Macros that are employed in a {\envfont theorem}
%    environment}
%
% \begin{macro}{\@thm}
%    The macro |\@thm| has to increase the current counter.  Then,
%    depending on whether the environment has (or does not have) an
%    optional argument, it has to branch into either |\@begintheorem|
%    or |\@opargtheorem|.
%    \begin{macrocode}
\gdef\@thm#1#2{\refstepcounter{#1}%
%    \end{macrocode}
%    Now we start a {\envfont trivlist} environment, and give
%    |\@topsep| and |\@topsepadd| the values of the \textsf{skip}
%    registers |\theorempreskipamount| and |\theorempostskipamount|.
%    The value in |\@topsep| is the vertical space that is inserted by
%    the first (and only) |\item| in our |\trivlist| whilst
%    |\@topsepadd| is inserted by |\@endparenv| at the end of that
%    {\envfont trivlist} environment.  By using these registers, we
%    obtain the desired space around a {\envfont theorem} environment.
%    \begin{macrocode}
   \trivlist
   \@topsep \theorempreskipamount               % used by first \item
   \@topsepadd \theorempostskipamount           % used by \@endparenv
%    \end{macrocode}
%    Now we have to test whether an optional argument has been given.
% \changes{v2.0d}{89/07/18}{Forgotten percent added.}
%    \begin{macrocode}
   \@ifnextchar [%
%    \end{macrocode}
%    If there is an optional argument, we will call |\@ythm|, and
%    move the arguments read back into the input stream.
%    \begin{macrocode}
   {\@ythm{#1}{#2}}%
%    \end{macrocode}
%    If not, we call |\@begintheorem|. Its first argument is the
%    name of the theorem set (hence the second argument of
%    |\@thm|). Its second argument is the macro that produces the
%    current number.
%    \begin{macrocode}
   {\@begintheorem{#2}{\csname the#1\endcsname}\ignorespaces}}
%    \end{macrocode}
% \end{macro}
%
% \begin{macro}{\@xthm}
% \begin{macro}{\@ythm}
%    Both these macros were originally called by |\@thm|.  We do
%    not need |\@xthm| anymore, hence we reset it to
%    |\relax|. The definition of |\@ythm| has not changed at
%    all from its definition in \LaTeX. In order to make the macros
%    easier to understand, we will
%    nevertheless present it (commented out).
%    \begin{macrocode}
\global\let\@xthm\relax
% \def\@ythm#1#2[#3]{\@opargbegintheorem{#2}{\csname
%           the#1\endcsname}{#3}\ignorespaces}
%</package>
%    \end{macrocode}
%    The primitive |\ignorespaces| in |\@ythm| and
%    |\@thm| is needed to remove the spaces between the
%    |\begin{...}| and the actual text.
% \end{macro}
% \end{macro}
%
%
% \subsection{Definition of the theorem substyles}
%
% As already pointed out, the theorem substyles, defined below, are
% only loaded when necessary.  Note that all these substyles, except
% \texttt{plain}, have |\slshape| as the default body font.
%
% \subsubsection{The \texttt{plain} style}
%
% \label{dok:th@plain} As the following macros use |@|, we have
% to locally set the |\catcode| of this symbol to ``letter''.
% This happens within a group, so that we do not have to worry about
% which |\catcode| that symbol had before.
%    \begin{macrocode}
%<*thp>
\begingroup \makeatletter
%    \end{macrocode}
% Since we are now within a group, we must make all definitions
% globally.  First we make sure that \texttt{theorem.sty} is loaded.
% This will allow us to use this file as a document style option
% without having to call \texttt{theorem} itself as an option.  At the
% same time, we assure that at least version 2 is loaded, since
% |\theorem@style| was not defined in earlier versions.
%    \begin{macrocode}
\@ifundefined{theorem@style}{\input{theorem.sty}}{}
\ProvidesFile{thp.sty}
             [\FMithmInfo]
%    \end{macrocode}
%
%
% \begin{macro}{\th@plain}
% \begin{macro}{\@begintheorem}
% \begin{macro}{\@opargbegintheorem}
%    |\theoremstyle{plain}| corresponds to the original
%    definition, except that the distances to the surrounding text are
%    determined by the parameters |\theorempreskipamount| and
%    |\theorempostskipamount|.
%    First we set the default body font.
%    \begin{macrocode}
\gdef\th@plain{\normalfont\itshape
%    \end{macrocode}
%    Then we define |\@begintheorem| and
%    |\@opargbegintheorem|. These two macros define how the
%    header of a theorem is typeset.  |\@opargbegintheorem| will
%    be called if a \textsf{theorem} environment with an optional
%    argument is encountered; otherwise, the header is constructed by
%    calling |\@begintheorem|.  If one of these macros is
%    executed, we are within a \textsf{trivlist} environment started by
%    |\@thm|.  So the theorem header is produced with an
%    |\item| command.
%
%    Instead of specifying the header font directly, all standard
%    theorem styles use the |\theorem@headerfont| macro to allow
%    customization.  The extra space (|\labelsep|) is necessary
%    because of problems in the \textsf{trivlist} environment.
%    \begin{macrocode}
  \def\@begintheorem##1##2{%
        \item[\hskip\labelsep \theorem@headerfont ##1\ ##2]}%
%    \end{macrocode}
%    The definition of |\@opargbegintheorem| is completely
%    analogous. The only difference is the fact that there exists a
%    third argument (which is the optional parameter of the
%    environment and contains additional information about the
%    theorem).  Customarily we enclose it in parentheses.
%    \begin{macrocode}
\def\@opargbegintheorem##1##2##3{%
   \item[\hskip\labelsep \theorem@headerfont ##1\ ##2\ (##3)]}}
%    \end{macrocode}
% \end{macro}
% \end{macro}
% \end{macro}
%
% \noindent
% We conclude with an |\endgroup| to restore the
% |\catcode| of |@|.
%    \begin{macrocode}
\endgroup
%</thp>
%    \end{macrocode}
%
%
% \subsubsection{The \texttt{break} style}
%
% This style option is stored in the file |thb.sty|.
% For the next two lines see the documentation for |\th@plain| on
% page~\pageref{dok:th@plain}.
%    \begin{macrocode}
%<*thb>
\begingroup \makeatletter
\@ifundefined{theorem@style}{\input{theorem.sty}}{}
\ProvidesFile{thb.sty}
             [\FMithmInfo]
%    \end{macrocode}
%
% \begin{macro}{\th@break}
%    |\theoremstyle{break}| produces a line break after the name
%    of the theorem.  The font is |\slshape|.  Hence, we define
%    |\th@break| as follows:
%    \begin{macrocode}
\gdef\th@break{\normalfont\slshape
  \def\@begintheorem##1##2{\item[%
%    \end{macrocode}
%    We run into the following problem: it is not possible to create
%    the header with |\item[|\meta{title}|]| and then start
%    a new line by, for example, |\mbox{}\\|. Such a definition
%    will fail whenever a list environment follows immediately.  With
%    the above construction, the |\mbox{}| causes the switch
%    |@inlabel| (cf.\ definition of |\list| and
%    |\trivlist| in \cite{latex:source}) to be set to \texttt{false}
%    and so the following list will insert additional vertical
%    space (|\topskip|). This is quite annoying. Therefore, we
%    create the line break within the |\item|. In order to ensure
%    that the text will begin at the proper position in the following
%    line, we simply pretend that the label does not take any
%    room.\footnote{This will lead to problems whenever very high
%    symbols occurring in the line tower into the
%    heading. So, something else has to be done here sometime.}
%    \begin{macrocode}
       \rlap{\vbox{\hbox{\hskip \labelsep\theorem@headerfont ##1\ ##2}%
                   \hbox{\strut}}}]}%
%    \end{macrocode}
%    Again, the definition of |\@opargbegintheorem| is
%    completely analogous.
%    \begin{macrocode}
\def\@opargbegintheorem##1##2##3{%
  \item[\rlap{\vbox{\hbox{\hskip \labelsep \theorem@headerfont
                     ##1\ ##2\ (##3)}%
                    \hbox{\strut}}}]}}
\endgroup
%</thb>
%    \end{macrocode}
% \end{macro}
%
%
% \subsubsection{The \texttt{changebreak} style}
%
%    \begin{macrocode}
% This style option is stored in the file |thcb.sty|.
%    \begin{macrocode}
%<*thcb>
\begingroup \makeatletter
\@ifundefined{theorem@style}{\input{theorem.sty}}{}
\ProvidesFile{thcb.sty}
             [\FMithmInfo]
%    \end{macrocode}
%
% \begin{macro}{\th@changebreak}
%    The \texttt{change-break} theorem style is like \texttt{break}
%    but with interchange of theorem name and theorem number.  Thus we
%    define |\th@changebreak| as follows:
%    \begin{macrocode}
\gdef\th@changebreak{\normalfont\slshape
  \def\@begintheorem##1##2{\item
       [\rlap{\vbox{\hbox{\hskip\labelsep \theorem@headerfont ##2\ ##1}%
                    \hbox{\strut}}}]}%
\def\@opargbegintheorem##1##2##3{%
  \item[\rlap{\vbox{\hbox{\hskip\labelsep \theorem@headerfont
                     ##2\ ##1\ (##3)}%
                    \hbox{\strut}}}]}}
\endgroup
%</thcb>
%    \end{macrocode}
% \end{macro}
%
%
% \subsubsection{The \texttt{change} style}
%
% This style option is stored in the file |thc.sty|.
%    \begin{macrocode}
%<*thc>
\begingroup \makeatletter
\@ifundefined{theorem@style}{\input{theorem.sty}}{}
\ProvidesFile{thc.sty}
             [\FMithmInfo]
%    \end{macrocode}
%
% \begin{macro}{\th@change}
%    The \texttt{change} theorem style corresponds to the
%    \texttt{change break} style without a linebreak after the header.
%    To say it in another way, it's the same as the \texttt{plain}
%    style but with number and name interchanged and |\slshape| as the
%    default font.
%    \begin{macrocode}
\gdef\th@change{\normalfont\slshape
  \def\@begintheorem##1##2{\item
       [\hskip\labelsep \theorem@headerfont ##2\ ##1]}%
\def\@opargbegintheorem##1##2##3{%
  \item[\hskip\labelsep \theorem@headerfont ##2\ ##1\ (##3)]}}
\endgroup
%</thc>
%    \end{macrocode}
% \end{macro}
%
%
% \subsubsection{The \texttt{marginbreak} style}
%
% This style option is the one used most often at Mainz.  It is saved
% in the file |thmb.sty|.
%    \begin{macrocode}
%<*thmb>
\begingroup \makeatletter
\@ifundefined{theorem@style}{\input{theorem.sty}}{}
\ProvidesFile{thmb.sty}
             [\FMithmInfo]
%    \end{macrocode}
%
% \begin{macro}{\th@marginbreak}
%    The \texttt{margin break} style is nearly the same as the
%    \texttt{change break} style. The only difference is the placement
%    of the theorem number.  We use |\llap| to place it in the left
%    margin.
%
%    In this style |\labelsep| denotes the separation between the
%    number and the text.
%    \begin{macrocode}
\gdef\th@marginbreak{\normalfont\slshape
  \def\@begintheorem##1##2{\item
       [\rlap{\vbox{\theorem@headerfont
                    \hbox{\llap{##2}\hskip\labelsep ##1}%
                    \hbox{\strut}}}]}%
\def\@opargbegintheorem##1##2##3{%
  \item[\rlap{\vbox{\theorem@headerfont
                    \hbox{\llap{##2}\hskip\labelsep ##1\ (##3)}%
                    \hbox{\strut}}}]}}
\endgroup
%</thmb>
%    \end{macrocode}
% \changes{v2.1c}{90/02/17}{`endgroup outside macrocode env.}
% \end{macro}
%
%
%
% \subsubsection{The \texttt{margin} style}
%
% This style option is stored in the file |thm.sty|.
%    \begin{macrocode}
%<*thm>
\begingroup \makeatletter
\@ifundefined{theorem@style}{\input{theorem.sty}}{}
\ProvidesFile{thm.sty}
             [\FMithmInfo]
%    \end{macrocode}
%
% \begin{macro}{\th@margin}
%    Again this is only a variant of the theorem styles described
%    above without any new ideas.
%    \begin{macrocode}
\gdef\th@margin{\normalfont\slshape
  \def\@begintheorem##1##2{\item
       [\theorem@headerfont \llap{##2}\hskip\labelsep ##1]}%
\def\@opargbegintheorem##1##2##3{%
  \item[\theorem@headerfont \llap{##2}\hskip\labelsep ##1\ (##3)]}}
\endgroup
%</thm>
%    \end{macrocode}
% \end{macro}
%
%
%
%
% \subsection{Final Definitions}
%
% \begin{macro}{\theorempreskipamount}
% \begin{macro}{\theorempostskipamount}
%    The \textsf{skip} parameters that regulate the vertical empty space
%    before and after the {\envfont theorem} environment have
%    to be allocated as well.
% \changes{v2.0f}{89/07/20}{Unnecessary `global' removed.}
%    \begin{macrocode}
%<*package>
\newskip\theorempreskipamount
\newskip\theorempostskipamount
%    \end{macrocode}
%    Since we have used the same values for all theorem sets, we now can
%    assign them.
%    \begin{macrocode}
\global\setlength\theorempreskipamount{12pt plus 5pt minus 3pt}
\global\setlength\theorempostskipamount{8pt plus 3pt minus 1.5pt}
%    \end{macrocode}
% \end{macro}
% \end{macro}
%
%
% \begin{macro}{\@endtheorem}
%    The same holds for the macro |\@endtheorem|, which ends a
%    {\envfont theorem} environment. Since it is the same for all
%    theorem sets, it is removed from the macros |\th@|\meta{style}.
%    It simply ends the {\envfont trivlist} environment, which was
%    begun in |\@thm|.
%    \begin{macrocode}
\global\let\@endtheorem=\endtrivlist
%    \end{macrocode}
% \end{macro}
%
%
%
% \begin{macro}{\@preamblecmds}
%    All macros defined above are to be used only in the preamble.
%    Therefore, we insert them in |\@preamblecmds| which will disable
%    them at begin document. This is done by the internal \LaTeXe{}
%    command |\@onlypreamble|.
% \changes{v2.0g}{89/09/19}{`theoremheaderfont and `theorembodyfont
%    added.}
% \changes{v2.2a}{94/02/03}{Use \LaTeXe{} interface}
%    \begin{macrocode}
\@onlypreamble\@xnthm
\@onlypreamble\@ynthm
\@onlypreamble\@othm
\@onlypreamble\newtheorem
\@onlypreamble\theoremstyle
\@onlypreamble\theorembodyfont
\@onlypreamble\theoremheaderfont
%    \end{macrocode}
% \end{macro}
%
% Finally we declare the \texttt{plain} theorem style to be the default.
% \changes{v2.1b}{90/06/04}{Fixed missing `item by declaring default
%    style.}
%    \begin{macrocode}
\theoremstyle{plain}
%</package>
%    \end{macrocode}
%
% \Finale

\endinput
}{}
\ProvidesFile{thm.sty}
             [\FMithmInfo]
%    \end{macrocode}
%
% \begin{macro}{\th@margin}
%    Again this is only a variant of the theorem styles described
%    above without any new ideas.
%    \begin{macrocode}
\gdef\th@margin{\normalfont\slshape
  \def\@begintheorem##1##2{\item
       [\theorem@headerfont \llap{##2}\hskip\labelsep ##1]}%
\def\@opargbegintheorem##1##2##3{%
  \item[\theorem@headerfont \llap{##2}\hskip\labelsep ##1\ (##3)]}}
\endgroup
%</thm>
%    \end{macrocode}
% \end{macro}
%
%
%
%
% \subsection{Final Definitions}
%
% \begin{macro}{\theorempreskipamount}
% \begin{macro}{\theorempostskipamount}
%    The \textsf{skip} parameters that regulate the vertical empty space
%    before and after the {\envfont theorem} environment have
%    to be allocated as well.
% \changes{v2.0f}{89/07/20}{Unnecessary `global' removed.}
%    \begin{macrocode}
%<*package>
\newskip\theorempreskipamount
\newskip\theorempostskipamount
%    \end{macrocode}
%    Since we have used the same values for all theorem sets, we now can
%    assign them.
%    \begin{macrocode}
\global\setlength\theorempreskipamount{12pt plus 5pt minus 3pt}
\global\setlength\theorempostskipamount{8pt plus 3pt minus 1.5pt}
%    \end{macrocode}
% \end{macro}
% \end{macro}
%
%
% \begin{macro}{\@endtheorem}
%    The same holds for the macro |\@endtheorem|, which ends a
%    {\envfont theorem} environment. Since it is the same for all
%    theorem sets, it is removed from the macros |\th@|\meta{style}.
%    It simply ends the {\envfont trivlist} environment, which was
%    begun in |\@thm|.
%    \begin{macrocode}
\global\let\@endtheorem=\endtrivlist
%    \end{macrocode}
% \end{macro}
%
%
%
% \begin{macro}{\@preamblecmds}
%    All macros defined above are to be used only in the preamble.
%    Therefore, we insert them in |\@preamblecmds| which will disable
%    them at begin document. This is done by the internal \LaTeXe{}
%    command |\@onlypreamble|.
% \changes{v2.0g}{89/09/19}{`theoremheaderfont and `theorembodyfont
%    added.}
% \changes{v2.2a}{94/02/03}{Use \LaTeXe{} interface}
%    \begin{macrocode}
\@onlypreamble\@xnthm
\@onlypreamble\@ynthm
\@onlypreamble\@othm
\@onlypreamble\newtheorem
\@onlypreamble\theoremstyle
\@onlypreamble\theorembodyfont
\@onlypreamble\theoremheaderfont
%    \end{macrocode}
% \end{macro}
%
% Finally we declare the \texttt{plain} theorem style to be the default.
% \changes{v2.1b}{90/06/04}{Fixed missing `item by declaring default
%    style.}
%    \begin{macrocode}
\theoremstyle{plain}
%</package>
%    \end{macrocode}
%
% \Finale

\endinput
}{}
\ProvidesFile{thp.sty}
             [\FMithmInfo]
%    \end{macrocode}
%
%
% \begin{macro}{\th@plain}
% \begin{macro}{\@begintheorem}
% \begin{macro}{\@opargbegintheorem}
%    |\theoremstyle{plain}| corresponds to the original
%    definition, except that the distances to the surrounding text are
%    determined by the parameters |\theorempreskipamount| and
%    |\theorempostskipamount|.
%    First we set the default body font.
%    \begin{macrocode}
\gdef\th@plain{\normalfont\itshape
%    \end{macrocode}
%    Then we define |\@begintheorem| and
%    |\@opargbegintheorem|. These two macros define how the
%    header of a theorem is typeset.  |\@opargbegintheorem| will
%    be called if a \textsf{theorem} environment with an optional
%    argument is encountered; otherwise, the header is constructed by
%    calling |\@begintheorem|.  If one of these macros is
%    executed, we are within a \textsf{trivlist} environment started by
%    |\@thm|.  So the theorem header is produced with an
%    |\item| command.
%
%    Instead of specifying the header font directly, all standard
%    theorem styles use the |\theorem@headerfont| macro to allow
%    customization.  The extra space (|\labelsep|) is necessary
%    because of problems in the \textsf{trivlist} environment.
%    \begin{macrocode}
  \def\@begintheorem##1##2{%
        \item[\hskip\labelsep \theorem@headerfont ##1\ ##2]}%
%    \end{macrocode}
%    The definition of |\@opargbegintheorem| is completely
%    analogous. The only difference is the fact that there exists a
%    third argument (which is the optional parameter of the
%    environment and contains additional information about the
%    theorem).  Customarily we enclose it in parentheses.
%    \begin{macrocode}
\def\@opargbegintheorem##1##2##3{%
   \item[\hskip\labelsep \theorem@headerfont ##1\ ##2\ (##3)]}}
%    \end{macrocode}
% \end{macro}
% \end{macro}
% \end{macro}
%
% \noindent
% We conclude with an |\endgroup| to restore the
% |\catcode| of |@|.
%    \begin{macrocode}
\endgroup
%</thp>
%    \end{macrocode}
%
%
% \subsubsection{The \texttt{break} style}
%
% This style option is stored in the file |thb.sty|.
% For the next two lines see the documentation for |\th@plain| on
% page~\pageref{dok:th@plain}.
%    \begin{macrocode}
%<*thb>
\begingroup \makeatletter
\@ifundefined{theorem@style}{% \iffalse meta-comment
%
% Copyright (C) 1993-2024
%
% The LaTeX Project and any individual authors listed elsewhere
% in this file.
%
% This file is part of the Standard LaTeX `Tools Bundle'.
% -------------------------------------------------------
%
% It may be distributed and/or modified under the
% conditions of the LaTeX Project Public License, either version 1.3c
% of this license or (at your option) any later version.
% The latest version of this license is in
%    https://www.latex-project.org/lppl.txt
% and version 1.3c or later is part of all distributions of LaTeX
% version 2005/12/01 or later.
%
% The list of all files belonging to the LaTeX `Tools Bundle' is
% given in the file `manifest.txt'.
%
% \fi
%
%%
%
%
\def\FMithmInfo{2023/07/05 v2.2c Theorem extension package (FMi)}
%
% \ProvidesFile{theorem.dtx}[\FMithmInfo]
%\iffalse   % this is a METACOMMENT !
%%
%% Package `theorem' to use with LaTeX2e
%% Copyright (C) 1989-1995 Frank Mittelbach, all rights reserved.
%<+package>\@ifundefined{theorem@style}{}{\endinput}
%<+package>\NeedsTeXFormat{LaTeX2e}[1995/06/01]
%<+package>\ProvidesPackage{theorem}
%<+package>         [\FMithmInfo]
%
% \fi
%
% \changes{v1.0}{87/05/07}{First draft.}
% \changes{v1.0a}{87/05/07}{`thistheoremstyle' added.}
% \changes{v1.1a}{87/07/29}{New concept: redefine the internal
%                          \protect\LaTeX{} commands `@xnthm' etc.}
% \changes{v1.1b}{87/07/31}{`theoremfont' used to specify font.}
% \changes{v1.2a}{88/01/08}{`@endtheorem' now globally defined.}
% \changes{v1.2b}{88/01/11}{`newtheorem' only allowed in the preamble.}
% \changes{v1.3a}{88/02/14}{`theoremfont' now a toks register.}
%
% \changes{v2.0a}{88/11/16}{New implementation and new documentation.}
% \changes{v2.0a}{88/11/16}{No layout changes but internal modifications
%                           old substyles won't work.}
% \changes{v2.0b}{89/01/30}{`global' used for all definitions.}
% \changes{v2.0c}{89/06/15}{Documentation translated to English.
%                            Thanks to LLL and Nina}
% \changes{v2.1a}{90/02/15}{Support for new font selection.}
% \changes{v2.2a}{94/02/03}{Cleanup for LaTeX2e}
%
% \DoNotIndex{\@endparenv,\@ifdefinable,\@ifnextchar,\@ifundefined}
% \DoNotIndex{\@namedef,\@nameuse,\@spaces,\@tempa,\@warning}
% \DoNotIndex{\ }
% \DoNotIndex{\begingroup,\bfseries}
% \DoNotIndex{\csname}
% \DoNotIndex{\def,\do,\docdate}
% \DoNotIndex{\endcsname,\endgroup,\endinput,\endtrivlist,\expandafter}
% \DoNotIndex{\filedate,\fileversion}
% \DoNotIndex{\gdef,\global}
% \DoNotIndex{\hbox,\hskip}
% \DoNotIndex{\ignorespaces,\input,\itshape,\item}
% \DoNotIndex{\labelsep,\let,\llap}
% \DoNotIndex{\makeatletter}
% \DoNotIndex{\noexpand}
% \DoNotIndex{\refstepcounter,\relax,\rlap}
% \DoNotIndex{\setlength,\slshape,\space,\string,\strut}
% \DoNotIndex{\the,\trivlist,\typeout}
% \DoNotIndex{\vbox}
% \DoNotIndex{\xdef}
%
% \def\deflabel#1{\ttfamily #1\hfill}
% \def\deflist#1{\begin{list}{}{\settowidth\labelwidth{\ttfamily #1}%
%                               \setlength\leftmargin\labelwidth
%                               \addtolength\leftmargin\labelsep
%                               \let\makelabel\deflabel}}
% \def\enddeflist{\end{list}}
% \def\envfont{\normalfont\ttfamily}
%
% \MakeShortVerb{\|}
%
% \newcommand{\TUB}{{\sl TUGboat\/}}
%
% \GetFileInfo{theorem.dtx}
% \title{An Extension of the
%        \LaTeX{} theorem environment\thanks{This file
%        has version number \fileversion, last
%        revised \filedate.}}
% \author{Frank Mittelbach}
%
% \MaintainedByLaTeXTeam{tools}
% \maketitle
%
% \begin{abstract}
%    The macros described in this paper yield an extension of the
%    \LaTeX{} theorem mechanism. It is designed to satisfy the
%    different requirements of various journals. Thus, the
%    layout of the ``theorems'' can be manipulated by determining a
%    ``style''. This article describes not only the use, but
%    also the definition, of the necessary macros.
% \end{abstract}
%
% \section*{Preface to version 2.2}
%
% For \LaTeXe{} this package did not need any fundamental changes. I
% only modified the messages generated so that theorem layout styles
% will show up with the |\listfiles| command and cleaned the section
% on the New Font Selection Scheme since this is now included in
% \LaTeX.
%
% \section*{Preface to version 2.1}
%
% This version is identical to 2.0g described in \TUB\ 10\#3 except
% for some internal defaults which are now set depending on the used
% font selection scheme.
%
% This was done to avoid unpleasant surprises if the new font selection
% scheme is in force. For further details see
% section~\ref{sec:newfonts} and \cite{bk:GMS94}.
%
% \section{Introduction}
%
% For our purposes here, ``theorems'' are labelled enunciations,
% often set off from the main text by extra space and a font change.
% Theorems, corollaries, conjectures, definitions, and remarks
% are all instances of ``theorems''.  The ``header'' of these structures
% is composed of a label (such as \textsc{Theorem} or \textsc{Remark})
% and a number which serializes an item in the sequence of items
% with the same label.
%
% Shortly after the introduction of \LaTeX{} at the Fachbereich
% Mathematik in Mainz, the desire to manipulate the layout of
% ``theorems'' arose.  In Mainz, the following two conventions came
% into general use:
% \begin{enumerate}
%   \item The number of the theorem is shown in the margin.
%   \item There is a line break at
%         the end of the theorem header.
% \end{enumerate}
%
% Additionally, some journals require different formats which depend on
% the ``sort of theorem'': e.g.\ often remarks and definitions are set
% in |\upshape|, while |\itshape| is employed for main theorems.
%
% Confronted with these requirements, a theorem environment was
% developed in Mainz which allows separate determination of the
% layout of the ``theorems sets'', comparable to |\pagestyle|.
%
%
% \section{The user interface}
% \subsection{Defining new theorem sets}
%
% \DescribeMacro\newtheorem
% As in the original \LaTeX{} version, the command |\newtheorem|
% defines a new ``theorem set'' or ``theorem-like structure''.
% Two required arguments name the new environment and give the text
% to be typeset with each instance of the new ``set'', while
% an optional argument determines how
% the ``set'' is enumerated:
% \begin{description}
%    \item[\ttfamily \bslash newtheorem\{foo\}\{bar\}]
%       The theorem set {\envfont foo} (whose name is \texttt{bar})
%       uses its own counter.
%    \item[\ttfamily \bslash newtheorem\{foo2\}{[foo]}\{bar2\}]
%       The theorem set {\envfont foo2} (printed name \texttt{bar2})
%       uses the same counter as the theorem set \texttt{foo}.
%    \item[\ttfamily \bslash newtheorem\{foo3\}\{bar3\}{[section]}]
%       The theorem set {\envfont foo3} (printed name \texttt{bar3}) is
%       enumerated within the counter \texttt{section}, i.e.\ with every
%       new |\section| the enumeration begins again with 1, and
%       the enumeration is composed from the section-number and the
%       theorem counter itself.
% \end{description}
%
% \DescribeMacro\theoremstyle
% Additionally, the command |\theoremstyle| can define the layout
% of various, or all, theorem sets.  It should be noted that any
% theorem set defined by |\newtheorem| is typeset in the
% |\theoremstyle| that is current at the time of the definition.
% Thus, the following
% \begin{verbatim}
%    \theoremstyle{break}        \newtheorem{Cor}{Corollary}
%    \theoremstyle{plain}        \newtheorem{Exa}{Example}[section]
%\end{verbatim}                             ^^A No blank at line start
% leads to the result that the set {\envfont Cor} is formatted in the
% style \texttt{break}, while the set {\envfont Exa} and all the
% following ones are formatted in the style \texttt{plain}, unless
% another |\theoremstyle| follows.  Since the definitions installed by
% |\newtheorem| are global, one also can limit |\theoremstyle| locally
% by grouping braces.
%
% \DescribeMacro\theorembodyfont
% The choice of the font for the theorem body is completely
% independent of the chosen |\theoremstyle|; this has proven to
% be very advantageous.  For example,
% \begin{verbatim}
%   {\theorembodyfont{\upshape}          \newtheorem{Rem}{Remark}}
%\end{verbatim}
% defines a theorem set {\envfont Rem}, which will be set in
% |\upshape| in the current layout (which in our example is
% \texttt{plain}).  As with |\theoremstyle|, the |\theorembodyfont|
% chosen is that current at the time of |\newtheorem|.  If
% |\theorembodyfont| is not specified or one defines
% |\theorembodyfont{}|, then the font used will be that defined by the
% |\theoremstyle|.
%
% \DescribeMacro\theoremheaderfont It is also possible to customize
% the font used for the theorem headers.  This is, however, a global
% declaration, and therefore there should be at most one
% |\theoremheaderfont| declaration in the preamble.\footnote{If it is
% actually necessary to have different header fonts, one has to define
% new theorem styles (substituting the desired font) or specify the
% information directly in the \texttt{\bslash newtheorem} declaration
% (the unclean variant).}
%
% \DescribeMacro\theorempreskipamount
% \DescribeMacro\theorempostskipamount Two additional parameters
% affect the vertical space around the theorem environments:
% |\theorempreskipamount| and |\theorempostskipamount| define,
% respectively, the spacing before and after such an environment.
% These parameters apply for all theorem sets and can be manipulated
% with the ordinary length macros.  They are rubber lengths
% (`\textsf{skips}'), and therefore can contain \texttt{plus} and
% \texttt{minus} parts.
%
% Since the definition of theorem sets should---most sensibly---be
% placed in the preamble, we only allow installation there.
% It is therefore possible to release the memory used here after
% |\begin{document}|, in order to make room for other
% applications.
%
%
% \subsection{Existing theorem styles}
%
% The following theorem styles exist to date:
% \begin{deflist}{marginbreak}
%    \item[plain]
%       This theorem style emulates the original \LaTeX{} definition,
%       except that additionally the parameters
%       |\theorem...skipamount| are used.
%    \item[break]
%       In this style, the theorem header is followed by a line
%       break.
%    \item[marginbreak]
%       The theorem number is set in the margin, and there is a line
%       break as in \texttt{break}.
%    \item[changebreak]
%       Like \texttt{break}, but with header number and text
%       interchanged.
%    \item[change]
%       Header number and text are interchanged, without a line break.
%    \item[margin]
%       The number is set in the left margin, without a line break.
% \end{deflist}
% All styles (except \texttt{plain}) select |\slshape| as the default
% |\theorembodyfont|.
%
%
% \subsection{Examples}
%
% Given the above theorem sets {\envfont Cor}, {\envfont Exa}
% and {\envfont Rem},
% suppose that the preamble also contains the declarations:
% \begin{verbatim}
%    \theoremstyle{marginbreak}   \newtheorem{Lem}[Cor]{Lemma}
%    \theoremstyle{change}
%    \theorembodyfont{\itshape}        \newtheorem{Def}[Cor]{Definition}
%
%    \theoremheaderfont{\scshape}
%\end{verbatim}
% Then the following are some typical examples of the typeset output
% resulting from their use.
%
% \begin{Cor}
%    This is a sentence typeset in the theorem environment {\envfont
%    Cor}.
% \end{Cor}
%
% \begin{Exa}
%    This is a sentence typeset in the theorem environment {\envfont
%    Exa}.
% \end{Exa}
%
% \begin{Rem}
%    This is a sentence typeset in the theorem environment {\envfont
%    Rem}.
% \end{Rem}
%
% \begin{Lem}[Ben User]
%    This is a sentence typeset in the theorem environment {\envfont
%    Lem}.
% \end{Lem}
%
% \begin{Def}[Very impressive Definition]
%    This is a sentence typeset in the theorem environment {\envfont
%    Def}.
% \end{Def}
%
% The last two examples show the effect of the optional argument to
% a theorem environment (it is the text typeset in parentheses).
%
%
% \section{Special Considerations}
% \label{sec:newfonts}
%
% Theoremheader and body are implemented as a unit. This means that
% the |\theoremheaderfont|  will inherit characteristics of the
% |\theorembodyfont| in \LaTeXe. Thus, if for example
% |\theorembodyfont| is |\itshape| and |\theoremheaderfont|
% is |\bfseries| the font selected for the header will have the
% characteristics `bold extended italic'. If this is not desired one
% should set the |\theoremheaderfont| to something like
% \begin{verbatim}
%      \theoremheaderfont{\normalfont\bfseries}
%\end{verbatim}
% i.e.\ supplying all necessary font information explicitly.
%
%
%
% \section{Acknowledgements}
%
% The publication of this set of macros was only possible with the
% help of Christina Busse (translating the manuscript into English),
% Joachim Pense (playing the r\^ole of typist), Chris Rowley
% (looking everything over) and many others providing useful
% suggestions.
%
% \MaybeStop{%
%    \begin{thebibliography}{9}
%       \bibitem{bk:GMS94} \textsc{M.~Goossens}, \textsc{F.~Mittelbach}
%         and \textsc{A.~Samarin}.
%         \newblock The \LaTeX{} Companion.
%         \newblock
%         Addison-Wesley, Reading, Massachusetts, 1994.
%       \bibitem{latex:source} \textsc{Lamport, Leslie.} \newblock
%                               \texttt{latex.tex}, version 2.09, date
%                               Feb. 1990.
%    \end{thebibliography}
%    \PrintIndex
%    \PrintChanges}
%
%
% \section{The documentation driver file}
%
% The next bit of code contains the documentation driver file for
% \TeX{}, i.e., the file that will produce the documentation you are
% currently reading. It will be extracted from this file by the
% \texttt{docstrip} program. Since it is the first code in the file
% one can alternatively process this file directly with \LaTeXe{} to
% obtain the documentation.
%
%    \begin{macrocode}
%<*driver>
\documentclass{ltxdoc}

\usepackage{theorem}

%<+driver>% The next few lines define theorem sets which are used
%<+driver>% in the example section of the documentation.

\theoremstyle{break}       \newtheorem{Cor}{Corollary}
\theoremstyle{plain}       \newtheorem{Exa}{Example}[section]
{\theorembodyfont{\upshape}\newtheorem{Rem}{Remark}}
\theoremstyle{marginbreak} \newtheorem{Lem}[Cor]{Lemma}
\theoremstyle{change}
\theorembodyfont{\itshape} \newtheorem{Def}[Cor]{Definition}

\theoremheaderfont{\scshape}

\RecordChanges

\begin{document}
   \DocInput{theorem.dtx}
\end{document}
%</driver>
%    \end{macrocode}
%
% \section{Definition of the Macros}
%
% If the file has been loaded before, we abort immediately. If not the
% package announces itself (this is actually done at the very top of
% the file---the way it is done isn't good style so don't copy it).
% \changes{v2.0e}{89/07/19}{Spaces removed from `typeout'.}
% \changes{v2.2b}{95/11/19}{Announce moved to top of file}
%    \begin{macrocode}
%<*package>
%\@ifundefined{theorem@style}{}{\endinput}
%\def\FMithmInfo{1995/11/19 v2.2b Theorem extension package (FMi)}
%\ProvidesPackage{theorem}[\FMithmInfo]
%    \end{macrocode}
%
%
% \subsection{Definition of theorem styles and fonts}
%
% All the definitions in this file are done globally to allow
% inputting this file inside a group.
%
% \begin{macro}{\theoremstyle}
%    Before a theorem style can be installed, the chosen style must be
%    known. For that reason, we must test to see that
%    |\th@|\meta{style} is known or, more precisely, that it is
%    different from |\relax|. If the style is not known then
%    |\th@plain| is used.
%    \begin{macrocode}
\gdef\theoremstyle#1{%
   \@ifundefined{th@#1}{\@warning
          {Unknown theoremstyle `#1'. Using `plain'}%
          \theorem@style{plain}}%
%    \end{macrocode}
%    We save the theorem style to be used in the token register
%    |\theorem@style|.
%    \begin{macrocode}
      {\theorem@style{#1}}%
%    \end{macrocode}
%    Now we ``evaluate'' the theorem style: this means, we call the
%    macro |\th@|\meta{style} which will activate the relevant
%    definitions which are contained in a separate file.  This is done
%    in a group to suppress changes to the current font.  This could
%    otherwise pose problems together with the new font selection
%    scheme\footnote{When I printed the original article using the new
%    font selection scheme I ended with a document with slanted
%    typefaces (text headings and all) simply because one of the
%    theorem styles used \texttt{\bslash sl} at toplevel.} if the
%    |\th@|\meta{style} is evaluated a second time.
% \changes{v2.1a}{90/02/15}{Group added around `th@..}
% \changes{v2.1a}{90/02/15}{Reseting of `..@begintheorem removed.}
%    \begin{macrocode}
      \begingroup
        \csname th@\the\theorem@style \endcsname
      \endgroup}
%    \end{macrocode}
% \end{macro}
%
%
% \begin{macro}{\@begintheorem}
% \changes{v2.1a}{90/02/15}{Relaxed at toplevel.}
% \begin{macro}{\@opargbegintheorem}
% \changes{v2.1a}{90/02/15}{Relaxed at toplevel.}
%    We reset |\@begintheorem| and |\@opargbegintheorem| to
%    |\relax| since these commands are no longer necessary at
%    toplevel. This will save a few tokens.
%    \begin{macrocode}
\global\let\@begintheorem\relax
\global\let\@opargbegintheorem\relax
%    \end{macrocode}
% \end{macro}
% \end{macro}
%
% \begin{macro}{\theorem@style}
%    Obviously the \textsf{token} register used above has to be
%    allocated. To assure the utmost compatibility with the
%    original \LaTeX{} definition, we set the default theorem style to
%    \texttt{plain}, which implements the usual \LaTeX{} convention.
% \changes{v2.0f}{89/07/20}{Unnecessary `global' removed.}
%    \begin{macrocode}
\newtoks\theorem@style
\global\theorem@style{plain}
%    \end{macrocode}
% \end{macro}
%
% \begin{macro}{\theorembodyfont}
% \begin{macro}{\theorem@bodyfont}
%    If the |\theorembodyfont| is set by the user then it should not
%    interact with the default font set in the theorem style. When the
%    new font selection is in force this may happen if, for example,
%    the default is |\itshape| and the new |\theorembodyfont| is
%    |\sffamily|. So we add a |\reset@font| command in front of the
%    user definition.
% \changes{v2.0f}{89/07/20}{Unnecessary `global' removed.}
% \changes{v2.0g}{89/09/19}{`theoremfont to `theorembodyfont.}
% \changes{v2.1c}{92/06/22}{`reset@font added and logic changed.}
% \changes{v2.1d}{92/10/28}{Reset to empty if arg is empty}
%    \begin{macrocode}
\gdef\theorembodyfont#1{%
%    \end{macrocode}
%    We check if the argument supplied is empty and if so put nothing
%    into the |\theorem@bodyfont| token register to allow for
%    |\theorembodyfont{}| as a mean of using the default of the
%    current |\theoremstyle|.
%    \begin{macrocode}
   \def\@tempa{#1}%
   \ifx\@tempa\@empty
    \theorem@bodyfont{}%
   \else
    \theorem@bodyfont{\reset@font#1}%
   \fi
   }
\newtoks\theorem@bodyfont
\global\theorem@bodyfont{}
%    \end{macrocode}
% \end{macro}
% \end{macro}
%
%
% \begin{macro}{\theoremheaderfont}
% \changes{v2.0g}{89/09/19}{Macro added.}
%    The font for the theorem headers is handled differently because
%    this definition applies to all theorem styles.
%    \begin{macrocode}
\gdef\theoremheaderfont#1{\gdef\theorem@headerfont{#1}%
%    \end{macrocode}
%    After using the macro once it is redefined to produce an error
%    message.
%    \begin{macrocode}
       \gdef\theoremheaderfont##1{%
        \typeout{\string\theoremheaderfont\space should be used
                 only once.}}}
%    \end{macrocode}
% \end{macro}
%
% \begin{macro}{\theorem@headerfont}
% \changes{v2.0g}{89/09/19}{Macro added.}
%    To set the |\theorem@headerfont| default we first test if
%    the new fontselection scheme is in force.
% \changes{v2.1a}{90/02/15}{Support for new font selection.}
%    \begin{macrocode}
\ifx\upshape\undefined
%    \end{macrocode}
%    If not we define it to expand into |\bfseries|. We don't use
%    |\let| just in case a following style option redefines this
%    macro.
%    \begin{macrocode}
\gdef\theorem@headerfont{\bfseries}
%    \end{macrocode}
%    Otherwise we reset the current shape before calling |\bfseries|.
%    \begin{macrocode}
\else \gdef\theorem@headerfont{\normalfont\bfseries}\fi
%    \end{macrocode}
% \end{macro}
%
% \begin{macro}{\th@plain}
% \begin{macro}{\th@break}
% \begin{macro}{\th@marginbreak}
% \begin{macro}{\th@changebreak}
% \begin{macro}{\th@change}
% \begin{macro}{\th@margin}
%    The different styles are defined in macros such as
%    |\th@plain|.  Since memory space is precious in
%    ``non-Big-versions'', we have to avoid offering too many unused
%    definitions.  Therefore we define these styles in separate files
%    that can be loaded on demand.  Thus the commands themselves only
%    load these files. We use |\@input@|, a \LaTeXe{} internal command
%    that ensures that the file will be listed with |\listfiles|.
% \changes{v2.2a}{94/02/02}{Use `@input@ to load theorem layout files}
%    \begin{macrocode}
\gdef\th@plain{\@input@{thp.sty}}
\gdef\th@break{\@input@{thb.sty}}
\gdef\th@marginbreak{\@input@{thmb.sty}}
\gdef\th@changebreak{\@input@{thcb.sty}}
\gdef\th@change{\@input@{thc.sty}}
\gdef\th@margin{\@input@{thm.sty}}
%    \end{macrocode}
%    This list will be expanded when new styles become
%    available.  For testing,  just append new theorem substyles
%    as document options.
% \end{macro}
% \end{macro}
% \end{macro}
% \end{macro}
% \end{macro}
% \end{macro}
%
%
%
% \subsection{Definition of a new theorem set}
%
% As already pointed out, a new theorem environment can be defined in
% three different ways:
% \begin{verbatim}
%    \newtheorem{Lem}{Lemma}
%    \newtheorem{Lem}{Lemma}[section]
%    \newtheorem{Lem}[Theorem]{Lemma}
%\end{verbatim}
% The function of the macro |\newtheorem| is to recognize these cases
% and then to branch into one of the three macros |\@ynthm|, |\@xnthm|
% or |\@othm|.  This mechanism is adopted unchanged from
% \cite{latex:source}; the essential point here is that, for example,
% in the second case, the arguments \texttt{Lem}, \texttt{Lemma} and
% \texttt{section} are passed over to the macro |\@xnthm|.
%
% We inspect this case first because the others present fewer problems,
% and thus are easily derived from this one.
%
% \begin{macro}{\@xnthm}
%    For our example arguments, the macro |\@xnthm| must fulfill
%    the following:
%    \begin{itemize}
%       \item
%          Define a new \LaTeX-counter `Lem'
%       \item
%          reset this counter within a |\section|
%       \item
%          define the macro |\theLem|
%       \item
%          define the environment macros |\Lem| and
%          |\endLem| using the current
%          |\theoremstyle| and |\theorem@bodyfont|.
%    \end{itemize}
%    Obviously, all this should happen only if the first argument
%    of |\@xnthm| (i.e.\ \texttt{Lem} in our example) is chosen so
%    as not to conflict with any previously defined commands or
%    environments.  This test is performed by the \LaTeX{} macro
%    |\@ifdefinable|.
%    \begin{macrocode}
\gdef\@xnthm#1#2[#3]{\expandafter\@ifdefinable\csname #1\endcsname
%    \end{macrocode}
%    Therefore, the first argument of |\@ifdefinable| is the
%    expansion (in the example, |\Lem|) of
%    |\csname||#1\endcsname|.  The second argument is
%    executed only if the test has been completed successfully.
%    \begin{macrocode}
   {%
%    \end{macrocode}
%    Now we define the new counter. The names of the \LaTeX{}
%    macros employed should speak for themselves:
% \changes{v2.2b}{1995/10/07}{Use \cs{@newctr}}
% \changes{v2.2c}{1995/11/23}{Use \cs{@newctr} properly.}
%    \begin{macrocode}
    \@definecounter{#1}\@newctr{#1}[#3]%
%    \end{macrocode}
%    Using |\@newctr| will give a proper error message if the counter
%    in |#3| is not defined.
%    In defining `|\theLem|' we must generate the desired macro
%    name by use of |\expandafter| and |\csname|.
%    \begin{macrocode}
    \expandafter\xdef\csname the#1\endcsname
%    \end{macrocode}
%    An |\xdef| is used in order to make the definition global,
%    and to ensure that it contains the replacement texts of
%    |\@thmcountersep| and |\@thmcounter|.\footnote{These
%    two macros can be defined by the document style.  Their default
%    values produce a `\texttt{.}' as separation and an arabic
%    representation of the number.} However, not everything should be
%    expanded. For example, it saves space to use |\thesection|
%    instead of its---at times---lengthy expansion.
%    \begin{macrocode}
      {\expandafter \noexpand \csname the#3\endcsname
       \@thmcountersep \@thmcounter{#1}}%
%    \end{macrocode}
%    Thus with the defaults of \LaTeX{}, |\theLem| would be
%    replaced by the command sequence |\thesection.\arabic{Lem}|.
%
%    We will now look at the definition of the macro which is executed
%    at the beginning of the actual environment (in our example this
%    macro is |\Lem|). It should be noted that we use an
%    ``|\expandafter| trick'' to expand only certain parts of the
%    replacement text at the time of the definition.
%    \begin{macrocode}
    \def\@tempa{\global\@namedef{#1}}%
    \expandafter \@tempa \expandafter{%
%    \end{macrocode}
%    First, the macro that contains the current definitions of
%    |\@begintheorem| and |\@opargtheorem| should be called
%    up.  The name of this macro---as is already known---has the form
%    |\th@|\meta{theorem\ style}; therefore, it must be called by
%    \begin{macrocode}
      \csname th@\the \theorem@style
%    \end{macrocode}
%    In addition the default theorem font should be changeable, i.e.\
%    we have to insert the contents of |\theorem@bodyfont|. For
%    that reason, we expand even further, beyond |\endcsname|,
%    and thus insert the contents of the \textsf{token} register
%    |\theorem@bodyfont| in the replacement text.
%    \begin{macrocode}
            \expandafter \endcsname \the \theorem@bodyfont
%    \end{macrocode}
%    Now it is time to call the macro |\@thm| which takes over
%    the further processing.  It has two arguments: the current
%    counter name (in our example, \texttt{Lem}), and the text of the
%    label (in our example, \texttt{Lemma}).
%    \begin{macrocode}
     \@thm{#1}{#2}}%
%    \end{macrocode}
%    With this, the `sub-definition' is complete.  The macro
%    |\@endtheorem| ends a theorem environment and is, so far, nothing
%    but an |\endtrivlist|.  (Hence it is defined globally, and not
%    within the theorem styles.\footnote{This has to be changed as
%    soon as theorem styles that change \texttt{\bslash @endtheorem}
%    exist.  In such a case, all existing styles must be changed as
%    well since they will have to reset the macro.}) Therefore, we can
%    set it equivalent to the macro that ends the theorem set (in our
%    example, |\endLem|).  However, if some day theorem styles exist
%    that do change |\@endtheorem|, we would have to use the
%    commented-out line instead.
%    \begin{macrocode}
    \global \expandafter \let \csname end#1\endcsname \@endtheorem
%   \global\@namedef{end#1}{\@endtheorem}%
%    \end{macrocode}
%    With these commands all the required definitions are employed,
%    unless the test |\@ifdefinable| has failed.  Therefore, we
%    end the second argument of this macro and with it the definition
%    of |\@xnthm|.
%    \begin{macrocode}
   }}
%    \end{macrocode}
% \end{macro}
%
% \begin{macro}{\@ynthm}
%    The definition of |\@ynthm| is completely analogous. In
%    this case the new counter that is defined is not reset within
%    another counter; thus the definition of |\the...|
%    is simplified:
%    \begin{macrocode}
\gdef\@ynthm#1#2{\expandafter\@ifdefinable\csname #1\endcsname
   {\@definecounter{#1}%
    \expandafter\xdef\csname the#1\endcsname{\@thmcounter{#1}}%
%    \end{macrocode}
%    The rest of the definition corresponds literally to that of
%    |\@xnthm|:
%    \begin{macrocode}
    \def\@tempa{\global\@namedef{#1}}\expandafter \@tempa
     \expandafter{\csname th@\the \theorem@style \expandafter
     \endcsname \the\theorem@bodyfont \@thm{#1}{#2}}%
    \global \expandafter \let \csname end#1\endcsname \@endtheorem}}
%    \end{macrocode}
% \end{macro}
%
% \begin{macro}{\@othm}
%    The definition of |\@othm| does not contain anything new.
%    \begin{macrocode}
\gdef\@othm#1[#2]#3{%
%    \end{macrocode}
% \changes{v2.2b}{95/10/07}{Build in check for counter defined}
%    We do not define a new counter but instead use one that has
%    already been defined.  Thus the only definition we need is that
%    of this pseudo-counter (i.e.\ |\the|\meta{env.\ name}).
%    First we check if |#2| corresponds to a known counter name.
%    \begin{macrocode}
  \expandafter\ifx\csname c@#2\endcsname\relax
   \@nocounterr{#2}%
  \else
   \expandafter\@ifdefinable\csname #1\endcsname
   {\expandafter \xdef \csname the#1\endcsname
     {\expandafter \noexpand \csname the#2\endcsname}%
%    \end{macrocode}
%    All other parts of the definition can be adopted from
%    |\@xnthm|. We have to remember, though, that in this case
%    the name of the current counter and the theorem label have moved
%    to the second and third arguments.
%    \begin{macrocode}
    \def\@tempa{\global\@namedef{#1}}\expandafter \@tempa
     \expandafter{\csname th@\the \theorem@style \expandafter
     \endcsname \the\theorem@bodyfont \@thm{#2}{#3}}%
    \global \expandafter \let \csname end#1\endcsname \@endtheorem}%
  \fi}
%    \end{macrocode}
% \end{macro}
%
% \subsection{Macros that are employed in a {\envfont theorem}
%    environment}
%
% \begin{macro}{\@thm}
%    The macro |\@thm| has to increase the current counter.  Then,
%    depending on whether the environment has (or does not have) an
%    optional argument, it has to branch into either |\@begintheorem|
%    or |\@opargtheorem|.
%    \begin{macrocode}
\gdef\@thm#1#2{\refstepcounter{#1}%
%    \end{macrocode}
%    Now we start a {\envfont trivlist} environment, and give
%    |\@topsep| and |\@topsepadd| the values of the \textsf{skip}
%    registers |\theorempreskipamount| and |\theorempostskipamount|.
%    The value in |\@topsep| is the vertical space that is inserted by
%    the first (and only) |\item| in our |\trivlist| whilst
%    |\@topsepadd| is inserted by |\@endparenv| at the end of that
%    {\envfont trivlist} environment.  By using these registers, we
%    obtain the desired space around a {\envfont theorem} environment.
%    \begin{macrocode}
   \trivlist
   \@topsep \theorempreskipamount               % used by first \item
   \@topsepadd \theorempostskipamount           % used by \@endparenv
%    \end{macrocode}
%    Now we have to test whether an optional argument has been given.
% \changes{v2.0d}{89/07/18}{Forgotten percent added.}
%    \begin{macrocode}
   \@ifnextchar [%
%    \end{macrocode}
%    If there is an optional argument, we will call |\@ythm|, and
%    move the arguments read back into the input stream.
%    \begin{macrocode}
   {\@ythm{#1}{#2}}%
%    \end{macrocode}
%    If not, we call |\@begintheorem|. Its first argument is the
%    name of the theorem set (hence the second argument of
%    |\@thm|). Its second argument is the macro that produces the
%    current number.
%    \begin{macrocode}
   {\@begintheorem{#2}{\csname the#1\endcsname}\ignorespaces}}
%    \end{macrocode}
% \end{macro}
%
% \begin{macro}{\@xthm}
% \begin{macro}{\@ythm}
%    Both these macros were originally called by |\@thm|.  We do
%    not need |\@xthm| anymore, hence we reset it to
%    |\relax|. The definition of |\@ythm| has not changed at
%    all from its definition in \LaTeX. In order to make the macros
%    easier to understand, we will
%    nevertheless present it (commented out).
%    \begin{macrocode}
\global\let\@xthm\relax
% \def\@ythm#1#2[#3]{\@opargbegintheorem{#2}{\csname
%           the#1\endcsname}{#3}\ignorespaces}
%</package>
%    \end{macrocode}
%    The primitive |\ignorespaces| in |\@ythm| and
%    |\@thm| is needed to remove the spaces between the
%    |\begin{...}| and the actual text.
% \end{macro}
% \end{macro}
%
%
% \subsection{Definition of the theorem substyles}
%
% As already pointed out, the theorem substyles, defined below, are
% only loaded when necessary.  Note that all these substyles, except
% \texttt{plain}, have |\slshape| as the default body font.
%
% \subsubsection{The \texttt{plain} style}
%
% \label{dok:th@plain} As the following macros use |@|, we have
% to locally set the |\catcode| of this symbol to ``letter''.
% This happens within a group, so that we do not have to worry about
% which |\catcode| that symbol had before.
%    \begin{macrocode}
%<*thp>
\begingroup \makeatletter
%    \end{macrocode}
% Since we are now within a group, we must make all definitions
% globally.  First we make sure that \texttt{theorem.sty} is loaded.
% This will allow us to use this file as a document style option
% without having to call \texttt{theorem} itself as an option.  At the
% same time, we assure that at least version 2 is loaded, since
% |\theorem@style| was not defined in earlier versions.
%    \begin{macrocode}
\@ifundefined{theorem@style}{% \iffalse meta-comment
%
% Copyright (C) 1993-2024
%
% The LaTeX Project and any individual authors listed elsewhere
% in this file.
%
% This file is part of the Standard LaTeX `Tools Bundle'.
% -------------------------------------------------------
%
% It may be distributed and/or modified under the
% conditions of the LaTeX Project Public License, either version 1.3c
% of this license or (at your option) any later version.
% The latest version of this license is in
%    https://www.latex-project.org/lppl.txt
% and version 1.3c or later is part of all distributions of LaTeX
% version 2005/12/01 or later.
%
% The list of all files belonging to the LaTeX `Tools Bundle' is
% given in the file `manifest.txt'.
%
% \fi
%
%%
%
%
\def\FMithmInfo{2023/07/05 v2.2c Theorem extension package (FMi)}
%
% \ProvidesFile{theorem.dtx}[\FMithmInfo]
%\iffalse   % this is a METACOMMENT !
%%
%% Package `theorem' to use with LaTeX2e
%% Copyright (C) 1989-1995 Frank Mittelbach, all rights reserved.
%<+package>\@ifundefined{theorem@style}{}{\endinput}
%<+package>\NeedsTeXFormat{LaTeX2e}[1995/06/01]
%<+package>\ProvidesPackage{theorem}
%<+package>         [\FMithmInfo]
%
% \fi
%
% \changes{v1.0}{87/05/07}{First draft.}
% \changes{v1.0a}{87/05/07}{`thistheoremstyle' added.}
% \changes{v1.1a}{87/07/29}{New concept: redefine the internal
%                          \protect\LaTeX{} commands `@xnthm' etc.}
% \changes{v1.1b}{87/07/31}{`theoremfont' used to specify font.}
% \changes{v1.2a}{88/01/08}{`@endtheorem' now globally defined.}
% \changes{v1.2b}{88/01/11}{`newtheorem' only allowed in the preamble.}
% \changes{v1.3a}{88/02/14}{`theoremfont' now a toks register.}
%
% \changes{v2.0a}{88/11/16}{New implementation and new documentation.}
% \changes{v2.0a}{88/11/16}{No layout changes but internal modifications
%                           old substyles won't work.}
% \changes{v2.0b}{89/01/30}{`global' used for all definitions.}
% \changes{v2.0c}{89/06/15}{Documentation translated to English.
%                            Thanks to LLL and Nina}
% \changes{v2.1a}{90/02/15}{Support for new font selection.}
% \changes{v2.2a}{94/02/03}{Cleanup for LaTeX2e}
%
% \DoNotIndex{\@endparenv,\@ifdefinable,\@ifnextchar,\@ifundefined}
% \DoNotIndex{\@namedef,\@nameuse,\@spaces,\@tempa,\@warning}
% \DoNotIndex{\ }
% \DoNotIndex{\begingroup,\bfseries}
% \DoNotIndex{\csname}
% \DoNotIndex{\def,\do,\docdate}
% \DoNotIndex{\endcsname,\endgroup,\endinput,\endtrivlist,\expandafter}
% \DoNotIndex{\filedate,\fileversion}
% \DoNotIndex{\gdef,\global}
% \DoNotIndex{\hbox,\hskip}
% \DoNotIndex{\ignorespaces,\input,\itshape,\item}
% \DoNotIndex{\labelsep,\let,\llap}
% \DoNotIndex{\makeatletter}
% \DoNotIndex{\noexpand}
% \DoNotIndex{\refstepcounter,\relax,\rlap}
% \DoNotIndex{\setlength,\slshape,\space,\string,\strut}
% \DoNotIndex{\the,\trivlist,\typeout}
% \DoNotIndex{\vbox}
% \DoNotIndex{\xdef}
%
% \def\deflabel#1{\ttfamily #1\hfill}
% \def\deflist#1{\begin{list}{}{\settowidth\labelwidth{\ttfamily #1}%
%                               \setlength\leftmargin\labelwidth
%                               \addtolength\leftmargin\labelsep
%                               \let\makelabel\deflabel}}
% \def\enddeflist{\end{list}}
% \def\envfont{\normalfont\ttfamily}
%
% \MakeShortVerb{\|}
%
% \newcommand{\TUB}{{\sl TUGboat\/}}
%
% \GetFileInfo{theorem.dtx}
% \title{An Extension of the
%        \LaTeX{} theorem environment\thanks{This file
%        has version number \fileversion, last
%        revised \filedate.}}
% \author{Frank Mittelbach}
%
% \MaintainedByLaTeXTeam{tools}
% \maketitle
%
% \begin{abstract}
%    The macros described in this paper yield an extension of the
%    \LaTeX{} theorem mechanism. It is designed to satisfy the
%    different requirements of various journals. Thus, the
%    layout of the ``theorems'' can be manipulated by determining a
%    ``style''. This article describes not only the use, but
%    also the definition, of the necessary macros.
% \end{abstract}
%
% \section*{Preface to version 2.2}
%
% For \LaTeXe{} this package did not need any fundamental changes. I
% only modified the messages generated so that theorem layout styles
% will show up with the |\listfiles| command and cleaned the section
% on the New Font Selection Scheme since this is now included in
% \LaTeX.
%
% \section*{Preface to version 2.1}
%
% This version is identical to 2.0g described in \TUB\ 10\#3 except
% for some internal defaults which are now set depending on the used
% font selection scheme.
%
% This was done to avoid unpleasant surprises if the new font selection
% scheme is in force. For further details see
% section~\ref{sec:newfonts} and \cite{bk:GMS94}.
%
% \section{Introduction}
%
% For our purposes here, ``theorems'' are labelled enunciations,
% often set off from the main text by extra space and a font change.
% Theorems, corollaries, conjectures, definitions, and remarks
% are all instances of ``theorems''.  The ``header'' of these structures
% is composed of a label (such as \textsc{Theorem} or \textsc{Remark})
% and a number which serializes an item in the sequence of items
% with the same label.
%
% Shortly after the introduction of \LaTeX{} at the Fachbereich
% Mathematik in Mainz, the desire to manipulate the layout of
% ``theorems'' arose.  In Mainz, the following two conventions came
% into general use:
% \begin{enumerate}
%   \item The number of the theorem is shown in the margin.
%   \item There is a line break at
%         the end of the theorem header.
% \end{enumerate}
%
% Additionally, some journals require different formats which depend on
% the ``sort of theorem'': e.g.\ often remarks and definitions are set
% in |\upshape|, while |\itshape| is employed for main theorems.
%
% Confronted with these requirements, a theorem environment was
% developed in Mainz which allows separate determination of the
% layout of the ``theorems sets'', comparable to |\pagestyle|.
%
%
% \section{The user interface}
% \subsection{Defining new theorem sets}
%
% \DescribeMacro\newtheorem
% As in the original \LaTeX{} version, the command |\newtheorem|
% defines a new ``theorem set'' or ``theorem-like structure''.
% Two required arguments name the new environment and give the text
% to be typeset with each instance of the new ``set'', while
% an optional argument determines how
% the ``set'' is enumerated:
% \begin{description}
%    \item[\ttfamily \bslash newtheorem\{foo\}\{bar\}]
%       The theorem set {\envfont foo} (whose name is \texttt{bar})
%       uses its own counter.
%    \item[\ttfamily \bslash newtheorem\{foo2\}{[foo]}\{bar2\}]
%       The theorem set {\envfont foo2} (printed name \texttt{bar2})
%       uses the same counter as the theorem set \texttt{foo}.
%    \item[\ttfamily \bslash newtheorem\{foo3\}\{bar3\}{[section]}]
%       The theorem set {\envfont foo3} (printed name \texttt{bar3}) is
%       enumerated within the counter \texttt{section}, i.e.\ with every
%       new |\section| the enumeration begins again with 1, and
%       the enumeration is composed from the section-number and the
%       theorem counter itself.
% \end{description}
%
% \DescribeMacro\theoremstyle
% Additionally, the command |\theoremstyle| can define the layout
% of various, or all, theorem sets.  It should be noted that any
% theorem set defined by |\newtheorem| is typeset in the
% |\theoremstyle| that is current at the time of the definition.
% Thus, the following
% \begin{verbatim}
%    \theoremstyle{break}        \newtheorem{Cor}{Corollary}
%    \theoremstyle{plain}        \newtheorem{Exa}{Example}[section]
%\end{verbatim}                             ^^A No blank at line start
% leads to the result that the set {\envfont Cor} is formatted in the
% style \texttt{break}, while the set {\envfont Exa} and all the
% following ones are formatted in the style \texttt{plain}, unless
% another |\theoremstyle| follows.  Since the definitions installed by
% |\newtheorem| are global, one also can limit |\theoremstyle| locally
% by grouping braces.
%
% \DescribeMacro\theorembodyfont
% The choice of the font for the theorem body is completely
% independent of the chosen |\theoremstyle|; this has proven to
% be very advantageous.  For example,
% \begin{verbatim}
%   {\theorembodyfont{\upshape}          \newtheorem{Rem}{Remark}}
%\end{verbatim}
% defines a theorem set {\envfont Rem}, which will be set in
% |\upshape| in the current layout (which in our example is
% \texttt{plain}).  As with |\theoremstyle|, the |\theorembodyfont|
% chosen is that current at the time of |\newtheorem|.  If
% |\theorembodyfont| is not specified or one defines
% |\theorembodyfont{}|, then the font used will be that defined by the
% |\theoremstyle|.
%
% \DescribeMacro\theoremheaderfont It is also possible to customize
% the font used for the theorem headers.  This is, however, a global
% declaration, and therefore there should be at most one
% |\theoremheaderfont| declaration in the preamble.\footnote{If it is
% actually necessary to have different header fonts, one has to define
% new theorem styles (substituting the desired font) or specify the
% information directly in the \texttt{\bslash newtheorem} declaration
% (the unclean variant).}
%
% \DescribeMacro\theorempreskipamount
% \DescribeMacro\theorempostskipamount Two additional parameters
% affect the vertical space around the theorem environments:
% |\theorempreskipamount| and |\theorempostskipamount| define,
% respectively, the spacing before and after such an environment.
% These parameters apply for all theorem sets and can be manipulated
% with the ordinary length macros.  They are rubber lengths
% (`\textsf{skips}'), and therefore can contain \texttt{plus} and
% \texttt{minus} parts.
%
% Since the definition of theorem sets should---most sensibly---be
% placed in the preamble, we only allow installation there.
% It is therefore possible to release the memory used here after
% |\begin{document}|, in order to make room for other
% applications.
%
%
% \subsection{Existing theorem styles}
%
% The following theorem styles exist to date:
% \begin{deflist}{marginbreak}
%    \item[plain]
%       This theorem style emulates the original \LaTeX{} definition,
%       except that additionally the parameters
%       |\theorem...skipamount| are used.
%    \item[break]
%       In this style, the theorem header is followed by a line
%       break.
%    \item[marginbreak]
%       The theorem number is set in the margin, and there is a line
%       break as in \texttt{break}.
%    \item[changebreak]
%       Like \texttt{break}, but with header number and text
%       interchanged.
%    \item[change]
%       Header number and text are interchanged, without a line break.
%    \item[margin]
%       The number is set in the left margin, without a line break.
% \end{deflist}
% All styles (except \texttt{plain}) select |\slshape| as the default
% |\theorembodyfont|.
%
%
% \subsection{Examples}
%
% Given the above theorem sets {\envfont Cor}, {\envfont Exa}
% and {\envfont Rem},
% suppose that the preamble also contains the declarations:
% \begin{verbatim}
%    \theoremstyle{marginbreak}   \newtheorem{Lem}[Cor]{Lemma}
%    \theoremstyle{change}
%    \theorembodyfont{\itshape}        \newtheorem{Def}[Cor]{Definition}
%
%    \theoremheaderfont{\scshape}
%\end{verbatim}
% Then the following are some typical examples of the typeset output
% resulting from their use.
%
% \begin{Cor}
%    This is a sentence typeset in the theorem environment {\envfont
%    Cor}.
% \end{Cor}
%
% \begin{Exa}
%    This is a sentence typeset in the theorem environment {\envfont
%    Exa}.
% \end{Exa}
%
% \begin{Rem}
%    This is a sentence typeset in the theorem environment {\envfont
%    Rem}.
% \end{Rem}
%
% \begin{Lem}[Ben User]
%    This is a sentence typeset in the theorem environment {\envfont
%    Lem}.
% \end{Lem}
%
% \begin{Def}[Very impressive Definition]
%    This is a sentence typeset in the theorem environment {\envfont
%    Def}.
% \end{Def}
%
% The last two examples show the effect of the optional argument to
% a theorem environment (it is the text typeset in parentheses).
%
%
% \section{Special Considerations}
% \label{sec:newfonts}
%
% Theoremheader and body are implemented as a unit. This means that
% the |\theoremheaderfont|  will inherit characteristics of the
% |\theorembodyfont| in \LaTeXe. Thus, if for example
% |\theorembodyfont| is |\itshape| and |\theoremheaderfont|
% is |\bfseries| the font selected for the header will have the
% characteristics `bold extended italic'. If this is not desired one
% should set the |\theoremheaderfont| to something like
% \begin{verbatim}
%      \theoremheaderfont{\normalfont\bfseries}
%\end{verbatim}
% i.e.\ supplying all necessary font information explicitly.
%
%
%
% \section{Acknowledgements}
%
% The publication of this set of macros was only possible with the
% help of Christina Busse (translating the manuscript into English),
% Joachim Pense (playing the r\^ole of typist), Chris Rowley
% (looking everything over) and many others providing useful
% suggestions.
%
% \MaybeStop{%
%    \begin{thebibliography}{9}
%       \bibitem{bk:GMS94} \textsc{M.~Goossens}, \textsc{F.~Mittelbach}
%         and \textsc{A.~Samarin}.
%         \newblock The \LaTeX{} Companion.
%         \newblock
%         Addison-Wesley, Reading, Massachusetts, 1994.
%       \bibitem{latex:source} \textsc{Lamport, Leslie.} \newblock
%                               \texttt{latex.tex}, version 2.09, date
%                               Feb. 1990.
%    \end{thebibliography}
%    \PrintIndex
%    \PrintChanges}
%
%
% \section{The documentation driver file}
%
% The next bit of code contains the documentation driver file for
% \TeX{}, i.e., the file that will produce the documentation you are
% currently reading. It will be extracted from this file by the
% \texttt{docstrip} program. Since it is the first code in the file
% one can alternatively process this file directly with \LaTeXe{} to
% obtain the documentation.
%
%    \begin{macrocode}
%<*driver>
\documentclass{ltxdoc}

\usepackage{theorem}

%<+driver>% The next few lines define theorem sets which are used
%<+driver>% in the example section of the documentation.

\theoremstyle{break}       \newtheorem{Cor}{Corollary}
\theoremstyle{plain}       \newtheorem{Exa}{Example}[section]
{\theorembodyfont{\upshape}\newtheorem{Rem}{Remark}}
\theoremstyle{marginbreak} \newtheorem{Lem}[Cor]{Lemma}
\theoremstyle{change}
\theorembodyfont{\itshape} \newtheorem{Def}[Cor]{Definition}

\theoremheaderfont{\scshape}

\RecordChanges

\begin{document}
   \DocInput{theorem.dtx}
\end{document}
%</driver>
%    \end{macrocode}
%
% \section{Definition of the Macros}
%
% If the file has been loaded before, we abort immediately. If not the
% package announces itself (this is actually done at the very top of
% the file---the way it is done isn't good style so don't copy it).
% \changes{v2.0e}{89/07/19}{Spaces removed from `typeout'.}
% \changes{v2.2b}{95/11/19}{Announce moved to top of file}
%    \begin{macrocode}
%<*package>
%\@ifundefined{theorem@style}{}{\endinput}
%\def\FMithmInfo{1995/11/19 v2.2b Theorem extension package (FMi)}
%\ProvidesPackage{theorem}[\FMithmInfo]
%    \end{macrocode}
%
%
% \subsection{Definition of theorem styles and fonts}
%
% All the definitions in this file are done globally to allow
% inputting this file inside a group.
%
% \begin{macro}{\theoremstyle}
%    Before a theorem style can be installed, the chosen style must be
%    known. For that reason, we must test to see that
%    |\th@|\meta{style} is known or, more precisely, that it is
%    different from |\relax|. If the style is not known then
%    |\th@plain| is used.
%    \begin{macrocode}
\gdef\theoremstyle#1{%
   \@ifundefined{th@#1}{\@warning
          {Unknown theoremstyle `#1'. Using `plain'}%
          \theorem@style{plain}}%
%    \end{macrocode}
%    We save the theorem style to be used in the token register
%    |\theorem@style|.
%    \begin{macrocode}
      {\theorem@style{#1}}%
%    \end{macrocode}
%    Now we ``evaluate'' the theorem style: this means, we call the
%    macro |\th@|\meta{style} which will activate the relevant
%    definitions which are contained in a separate file.  This is done
%    in a group to suppress changes to the current font.  This could
%    otherwise pose problems together with the new font selection
%    scheme\footnote{When I printed the original article using the new
%    font selection scheme I ended with a document with slanted
%    typefaces (text headings and all) simply because one of the
%    theorem styles used \texttt{\bslash sl} at toplevel.} if the
%    |\th@|\meta{style} is evaluated a second time.
% \changes{v2.1a}{90/02/15}{Group added around `th@..}
% \changes{v2.1a}{90/02/15}{Reseting of `..@begintheorem removed.}
%    \begin{macrocode}
      \begingroup
        \csname th@\the\theorem@style \endcsname
      \endgroup}
%    \end{macrocode}
% \end{macro}
%
%
% \begin{macro}{\@begintheorem}
% \changes{v2.1a}{90/02/15}{Relaxed at toplevel.}
% \begin{macro}{\@opargbegintheorem}
% \changes{v2.1a}{90/02/15}{Relaxed at toplevel.}
%    We reset |\@begintheorem| and |\@opargbegintheorem| to
%    |\relax| since these commands are no longer necessary at
%    toplevel. This will save a few tokens.
%    \begin{macrocode}
\global\let\@begintheorem\relax
\global\let\@opargbegintheorem\relax
%    \end{macrocode}
% \end{macro}
% \end{macro}
%
% \begin{macro}{\theorem@style}
%    Obviously the \textsf{token} register used above has to be
%    allocated. To assure the utmost compatibility with the
%    original \LaTeX{} definition, we set the default theorem style to
%    \texttt{plain}, which implements the usual \LaTeX{} convention.
% \changes{v2.0f}{89/07/20}{Unnecessary `global' removed.}
%    \begin{macrocode}
\newtoks\theorem@style
\global\theorem@style{plain}
%    \end{macrocode}
% \end{macro}
%
% \begin{macro}{\theorembodyfont}
% \begin{macro}{\theorem@bodyfont}
%    If the |\theorembodyfont| is set by the user then it should not
%    interact with the default font set in the theorem style. When the
%    new font selection is in force this may happen if, for example,
%    the default is |\itshape| and the new |\theorembodyfont| is
%    |\sffamily|. So we add a |\reset@font| command in front of the
%    user definition.
% \changes{v2.0f}{89/07/20}{Unnecessary `global' removed.}
% \changes{v2.0g}{89/09/19}{`theoremfont to `theorembodyfont.}
% \changes{v2.1c}{92/06/22}{`reset@font added and logic changed.}
% \changes{v2.1d}{92/10/28}{Reset to empty if arg is empty}
%    \begin{macrocode}
\gdef\theorembodyfont#1{%
%    \end{macrocode}
%    We check if the argument supplied is empty and if so put nothing
%    into the |\theorem@bodyfont| token register to allow for
%    |\theorembodyfont{}| as a mean of using the default of the
%    current |\theoremstyle|.
%    \begin{macrocode}
   \def\@tempa{#1}%
   \ifx\@tempa\@empty
    \theorem@bodyfont{}%
   \else
    \theorem@bodyfont{\reset@font#1}%
   \fi
   }
\newtoks\theorem@bodyfont
\global\theorem@bodyfont{}
%    \end{macrocode}
% \end{macro}
% \end{macro}
%
%
% \begin{macro}{\theoremheaderfont}
% \changes{v2.0g}{89/09/19}{Macro added.}
%    The font for the theorem headers is handled differently because
%    this definition applies to all theorem styles.
%    \begin{macrocode}
\gdef\theoremheaderfont#1{\gdef\theorem@headerfont{#1}%
%    \end{macrocode}
%    After using the macro once it is redefined to produce an error
%    message.
%    \begin{macrocode}
       \gdef\theoremheaderfont##1{%
        \typeout{\string\theoremheaderfont\space should be used
                 only once.}}}
%    \end{macrocode}
% \end{macro}
%
% \begin{macro}{\theorem@headerfont}
% \changes{v2.0g}{89/09/19}{Macro added.}
%    To set the |\theorem@headerfont| default we first test if
%    the new fontselection scheme is in force.
% \changes{v2.1a}{90/02/15}{Support for new font selection.}
%    \begin{macrocode}
\ifx\upshape\undefined
%    \end{macrocode}
%    If not we define it to expand into |\bfseries|. We don't use
%    |\let| just in case a following style option redefines this
%    macro.
%    \begin{macrocode}
\gdef\theorem@headerfont{\bfseries}
%    \end{macrocode}
%    Otherwise we reset the current shape before calling |\bfseries|.
%    \begin{macrocode}
\else \gdef\theorem@headerfont{\normalfont\bfseries}\fi
%    \end{macrocode}
% \end{macro}
%
% \begin{macro}{\th@plain}
% \begin{macro}{\th@break}
% \begin{macro}{\th@marginbreak}
% \begin{macro}{\th@changebreak}
% \begin{macro}{\th@change}
% \begin{macro}{\th@margin}
%    The different styles are defined in macros such as
%    |\th@plain|.  Since memory space is precious in
%    ``non-Big-versions'', we have to avoid offering too many unused
%    definitions.  Therefore we define these styles in separate files
%    that can be loaded on demand.  Thus the commands themselves only
%    load these files. We use |\@input@|, a \LaTeXe{} internal command
%    that ensures that the file will be listed with |\listfiles|.
% \changes{v2.2a}{94/02/02}{Use `@input@ to load theorem layout files}
%    \begin{macrocode}
\gdef\th@plain{\@input@{thp.sty}}
\gdef\th@break{\@input@{thb.sty}}
\gdef\th@marginbreak{\@input@{thmb.sty}}
\gdef\th@changebreak{\@input@{thcb.sty}}
\gdef\th@change{\@input@{thc.sty}}
\gdef\th@margin{\@input@{thm.sty}}
%    \end{macrocode}
%    This list will be expanded when new styles become
%    available.  For testing,  just append new theorem substyles
%    as document options.
% \end{macro}
% \end{macro}
% \end{macro}
% \end{macro}
% \end{macro}
% \end{macro}
%
%
%
% \subsection{Definition of a new theorem set}
%
% As already pointed out, a new theorem environment can be defined in
% three different ways:
% \begin{verbatim}
%    \newtheorem{Lem}{Lemma}
%    \newtheorem{Lem}{Lemma}[section]
%    \newtheorem{Lem}[Theorem]{Lemma}
%\end{verbatim}
% The function of the macro |\newtheorem| is to recognize these cases
% and then to branch into one of the three macros |\@ynthm|, |\@xnthm|
% or |\@othm|.  This mechanism is adopted unchanged from
% \cite{latex:source}; the essential point here is that, for example,
% in the second case, the arguments \texttt{Lem}, \texttt{Lemma} and
% \texttt{section} are passed over to the macro |\@xnthm|.
%
% We inspect this case first because the others present fewer problems,
% and thus are easily derived from this one.
%
% \begin{macro}{\@xnthm}
%    For our example arguments, the macro |\@xnthm| must fulfill
%    the following:
%    \begin{itemize}
%       \item
%          Define a new \LaTeX-counter `Lem'
%       \item
%          reset this counter within a |\section|
%       \item
%          define the macro |\theLem|
%       \item
%          define the environment macros |\Lem| and
%          |\endLem| using the current
%          |\theoremstyle| and |\theorem@bodyfont|.
%    \end{itemize}
%    Obviously, all this should happen only if the first argument
%    of |\@xnthm| (i.e.\ \texttt{Lem} in our example) is chosen so
%    as not to conflict with any previously defined commands or
%    environments.  This test is performed by the \LaTeX{} macro
%    |\@ifdefinable|.
%    \begin{macrocode}
\gdef\@xnthm#1#2[#3]{\expandafter\@ifdefinable\csname #1\endcsname
%    \end{macrocode}
%    Therefore, the first argument of |\@ifdefinable| is the
%    expansion (in the example, |\Lem|) of
%    |\csname||#1\endcsname|.  The second argument is
%    executed only if the test has been completed successfully.
%    \begin{macrocode}
   {%
%    \end{macrocode}
%    Now we define the new counter. The names of the \LaTeX{}
%    macros employed should speak for themselves:
% \changes{v2.2b}{1995/10/07}{Use \cs{@newctr}}
% \changes{v2.2c}{1995/11/23}{Use \cs{@newctr} properly.}
%    \begin{macrocode}
    \@definecounter{#1}\@newctr{#1}[#3]%
%    \end{macrocode}
%    Using |\@newctr| will give a proper error message if the counter
%    in |#3| is not defined.
%    In defining `|\theLem|' we must generate the desired macro
%    name by use of |\expandafter| and |\csname|.
%    \begin{macrocode}
    \expandafter\xdef\csname the#1\endcsname
%    \end{macrocode}
%    An |\xdef| is used in order to make the definition global,
%    and to ensure that it contains the replacement texts of
%    |\@thmcountersep| and |\@thmcounter|.\footnote{These
%    two macros can be defined by the document style.  Their default
%    values produce a `\texttt{.}' as separation and an arabic
%    representation of the number.} However, not everything should be
%    expanded. For example, it saves space to use |\thesection|
%    instead of its---at times---lengthy expansion.
%    \begin{macrocode}
      {\expandafter \noexpand \csname the#3\endcsname
       \@thmcountersep \@thmcounter{#1}}%
%    \end{macrocode}
%    Thus with the defaults of \LaTeX{}, |\theLem| would be
%    replaced by the command sequence |\thesection.\arabic{Lem}|.
%
%    We will now look at the definition of the macro which is executed
%    at the beginning of the actual environment (in our example this
%    macro is |\Lem|). It should be noted that we use an
%    ``|\expandafter| trick'' to expand only certain parts of the
%    replacement text at the time of the definition.
%    \begin{macrocode}
    \def\@tempa{\global\@namedef{#1}}%
    \expandafter \@tempa \expandafter{%
%    \end{macrocode}
%    First, the macro that contains the current definitions of
%    |\@begintheorem| and |\@opargtheorem| should be called
%    up.  The name of this macro---as is already known---has the form
%    |\th@|\meta{theorem\ style}; therefore, it must be called by
%    \begin{macrocode}
      \csname th@\the \theorem@style
%    \end{macrocode}
%    In addition the default theorem font should be changeable, i.e.\
%    we have to insert the contents of |\theorem@bodyfont|. For
%    that reason, we expand even further, beyond |\endcsname|,
%    and thus insert the contents of the \textsf{token} register
%    |\theorem@bodyfont| in the replacement text.
%    \begin{macrocode}
            \expandafter \endcsname \the \theorem@bodyfont
%    \end{macrocode}
%    Now it is time to call the macro |\@thm| which takes over
%    the further processing.  It has two arguments: the current
%    counter name (in our example, \texttt{Lem}), and the text of the
%    label (in our example, \texttt{Lemma}).
%    \begin{macrocode}
     \@thm{#1}{#2}}%
%    \end{macrocode}
%    With this, the `sub-definition' is complete.  The macro
%    |\@endtheorem| ends a theorem environment and is, so far, nothing
%    but an |\endtrivlist|.  (Hence it is defined globally, and not
%    within the theorem styles.\footnote{This has to be changed as
%    soon as theorem styles that change \texttt{\bslash @endtheorem}
%    exist.  In such a case, all existing styles must be changed as
%    well since they will have to reset the macro.}) Therefore, we can
%    set it equivalent to the macro that ends the theorem set (in our
%    example, |\endLem|).  However, if some day theorem styles exist
%    that do change |\@endtheorem|, we would have to use the
%    commented-out line instead.
%    \begin{macrocode}
    \global \expandafter \let \csname end#1\endcsname \@endtheorem
%   \global\@namedef{end#1}{\@endtheorem}%
%    \end{macrocode}
%    With these commands all the required definitions are employed,
%    unless the test |\@ifdefinable| has failed.  Therefore, we
%    end the second argument of this macro and with it the definition
%    of |\@xnthm|.
%    \begin{macrocode}
   }}
%    \end{macrocode}
% \end{macro}
%
% \begin{macro}{\@ynthm}
%    The definition of |\@ynthm| is completely analogous. In
%    this case the new counter that is defined is not reset within
%    another counter; thus the definition of |\the...|
%    is simplified:
%    \begin{macrocode}
\gdef\@ynthm#1#2{\expandafter\@ifdefinable\csname #1\endcsname
   {\@definecounter{#1}%
    \expandafter\xdef\csname the#1\endcsname{\@thmcounter{#1}}%
%    \end{macrocode}
%    The rest of the definition corresponds literally to that of
%    |\@xnthm|:
%    \begin{macrocode}
    \def\@tempa{\global\@namedef{#1}}\expandafter \@tempa
     \expandafter{\csname th@\the \theorem@style \expandafter
     \endcsname \the\theorem@bodyfont \@thm{#1}{#2}}%
    \global \expandafter \let \csname end#1\endcsname \@endtheorem}}
%    \end{macrocode}
% \end{macro}
%
% \begin{macro}{\@othm}
%    The definition of |\@othm| does not contain anything new.
%    \begin{macrocode}
\gdef\@othm#1[#2]#3{%
%    \end{macrocode}
% \changes{v2.2b}{95/10/07}{Build in check for counter defined}
%    We do not define a new counter but instead use one that has
%    already been defined.  Thus the only definition we need is that
%    of this pseudo-counter (i.e.\ |\the|\meta{env.\ name}).
%    First we check if |#2| corresponds to a known counter name.
%    \begin{macrocode}
  \expandafter\ifx\csname c@#2\endcsname\relax
   \@nocounterr{#2}%
  \else
   \expandafter\@ifdefinable\csname #1\endcsname
   {\expandafter \xdef \csname the#1\endcsname
     {\expandafter \noexpand \csname the#2\endcsname}%
%    \end{macrocode}
%    All other parts of the definition can be adopted from
%    |\@xnthm|. We have to remember, though, that in this case
%    the name of the current counter and the theorem label have moved
%    to the second and third arguments.
%    \begin{macrocode}
    \def\@tempa{\global\@namedef{#1}}\expandafter \@tempa
     \expandafter{\csname th@\the \theorem@style \expandafter
     \endcsname \the\theorem@bodyfont \@thm{#2}{#3}}%
    \global \expandafter \let \csname end#1\endcsname \@endtheorem}%
  \fi}
%    \end{macrocode}
% \end{macro}
%
% \subsection{Macros that are employed in a {\envfont theorem}
%    environment}
%
% \begin{macro}{\@thm}
%    The macro |\@thm| has to increase the current counter.  Then,
%    depending on whether the environment has (or does not have) an
%    optional argument, it has to branch into either |\@begintheorem|
%    or |\@opargtheorem|.
%    \begin{macrocode}
\gdef\@thm#1#2{\refstepcounter{#1}%
%    \end{macrocode}
%    Now we start a {\envfont trivlist} environment, and give
%    |\@topsep| and |\@topsepadd| the values of the \textsf{skip}
%    registers |\theorempreskipamount| and |\theorempostskipamount|.
%    The value in |\@topsep| is the vertical space that is inserted by
%    the first (and only) |\item| in our |\trivlist| whilst
%    |\@topsepadd| is inserted by |\@endparenv| at the end of that
%    {\envfont trivlist} environment.  By using these registers, we
%    obtain the desired space around a {\envfont theorem} environment.
%    \begin{macrocode}
   \trivlist
   \@topsep \theorempreskipamount               % used by first \item
   \@topsepadd \theorempostskipamount           % used by \@endparenv
%    \end{macrocode}
%    Now we have to test whether an optional argument has been given.
% \changes{v2.0d}{89/07/18}{Forgotten percent added.}
%    \begin{macrocode}
   \@ifnextchar [%
%    \end{macrocode}
%    If there is an optional argument, we will call |\@ythm|, and
%    move the arguments read back into the input stream.
%    \begin{macrocode}
   {\@ythm{#1}{#2}}%
%    \end{macrocode}
%    If not, we call |\@begintheorem|. Its first argument is the
%    name of the theorem set (hence the second argument of
%    |\@thm|). Its second argument is the macro that produces the
%    current number.
%    \begin{macrocode}
   {\@begintheorem{#2}{\csname the#1\endcsname}\ignorespaces}}
%    \end{macrocode}
% \end{macro}
%
% \begin{macro}{\@xthm}
% \begin{macro}{\@ythm}
%    Both these macros were originally called by |\@thm|.  We do
%    not need |\@xthm| anymore, hence we reset it to
%    |\relax|. The definition of |\@ythm| has not changed at
%    all from its definition in \LaTeX. In order to make the macros
%    easier to understand, we will
%    nevertheless present it (commented out).
%    \begin{macrocode}
\global\let\@xthm\relax
% \def\@ythm#1#2[#3]{\@opargbegintheorem{#2}{\csname
%           the#1\endcsname}{#3}\ignorespaces}
%</package>
%    \end{macrocode}
%    The primitive |\ignorespaces| in |\@ythm| and
%    |\@thm| is needed to remove the spaces between the
%    |\begin{...}| and the actual text.
% \end{macro}
% \end{macro}
%
%
% \subsection{Definition of the theorem substyles}
%
% As already pointed out, the theorem substyles, defined below, are
% only loaded when necessary.  Note that all these substyles, except
% \texttt{plain}, have |\slshape| as the default body font.
%
% \subsubsection{The \texttt{plain} style}
%
% \label{dok:th@plain} As the following macros use |@|, we have
% to locally set the |\catcode| of this symbol to ``letter''.
% This happens within a group, so that we do not have to worry about
% which |\catcode| that symbol had before.
%    \begin{macrocode}
%<*thp>
\begingroup \makeatletter
%    \end{macrocode}
% Since we are now within a group, we must make all definitions
% globally.  First we make sure that \texttt{theorem.sty} is loaded.
% This will allow us to use this file as a document style option
% without having to call \texttt{theorem} itself as an option.  At the
% same time, we assure that at least version 2 is loaded, since
% |\theorem@style| was not defined in earlier versions.
%    \begin{macrocode}
\@ifundefined{theorem@style}{\input{theorem.sty}}{}
\ProvidesFile{thp.sty}
             [\FMithmInfo]
%    \end{macrocode}
%
%
% \begin{macro}{\th@plain}
% \begin{macro}{\@begintheorem}
% \begin{macro}{\@opargbegintheorem}
%    |\theoremstyle{plain}| corresponds to the original
%    definition, except that the distances to the surrounding text are
%    determined by the parameters |\theorempreskipamount| and
%    |\theorempostskipamount|.
%    First we set the default body font.
%    \begin{macrocode}
\gdef\th@plain{\normalfont\itshape
%    \end{macrocode}
%    Then we define |\@begintheorem| and
%    |\@opargbegintheorem|. These two macros define how the
%    header of a theorem is typeset.  |\@opargbegintheorem| will
%    be called if a \textsf{theorem} environment with an optional
%    argument is encountered; otherwise, the header is constructed by
%    calling |\@begintheorem|.  If one of these macros is
%    executed, we are within a \textsf{trivlist} environment started by
%    |\@thm|.  So the theorem header is produced with an
%    |\item| command.
%
%    Instead of specifying the header font directly, all standard
%    theorem styles use the |\theorem@headerfont| macro to allow
%    customization.  The extra space (|\labelsep|) is necessary
%    because of problems in the \textsf{trivlist} environment.
%    \begin{macrocode}
  \def\@begintheorem##1##2{%
        \item[\hskip\labelsep \theorem@headerfont ##1\ ##2]}%
%    \end{macrocode}
%    The definition of |\@opargbegintheorem| is completely
%    analogous. The only difference is the fact that there exists a
%    third argument (which is the optional parameter of the
%    environment and contains additional information about the
%    theorem).  Customarily we enclose it in parentheses.
%    \begin{macrocode}
\def\@opargbegintheorem##1##2##3{%
   \item[\hskip\labelsep \theorem@headerfont ##1\ ##2\ (##3)]}}
%    \end{macrocode}
% \end{macro}
% \end{macro}
% \end{macro}
%
% \noindent
% We conclude with an |\endgroup| to restore the
% |\catcode| of |@|.
%    \begin{macrocode}
\endgroup
%</thp>
%    \end{macrocode}
%
%
% \subsubsection{The \texttt{break} style}
%
% This style option is stored in the file |thb.sty|.
% For the next two lines see the documentation for |\th@plain| on
% page~\pageref{dok:th@plain}.
%    \begin{macrocode}
%<*thb>
\begingroup \makeatletter
\@ifundefined{theorem@style}{\input{theorem.sty}}{}
\ProvidesFile{thb.sty}
             [\FMithmInfo]
%    \end{macrocode}
%
% \begin{macro}{\th@break}
%    |\theoremstyle{break}| produces a line break after the name
%    of the theorem.  The font is |\slshape|.  Hence, we define
%    |\th@break| as follows:
%    \begin{macrocode}
\gdef\th@break{\normalfont\slshape
  \def\@begintheorem##1##2{\item[%
%    \end{macrocode}
%    We run into the following problem: it is not possible to create
%    the header with |\item[|\meta{title}|]| and then start
%    a new line by, for example, |\mbox{}\\|. Such a definition
%    will fail whenever a list environment follows immediately.  With
%    the above construction, the |\mbox{}| causes the switch
%    |@inlabel| (cf.\ definition of |\list| and
%    |\trivlist| in \cite{latex:source}) to be set to \texttt{false}
%    and so the following list will insert additional vertical
%    space (|\topskip|). This is quite annoying. Therefore, we
%    create the line break within the |\item|. In order to ensure
%    that the text will begin at the proper position in the following
%    line, we simply pretend that the label does not take any
%    room.\footnote{This will lead to problems whenever very high
%    symbols occurring in the line tower into the
%    heading. So, something else has to be done here sometime.}
%    \begin{macrocode}
       \rlap{\vbox{\hbox{\hskip \labelsep\theorem@headerfont ##1\ ##2}%
                   \hbox{\strut}}}]}%
%    \end{macrocode}
%    Again, the definition of |\@opargbegintheorem| is
%    completely analogous.
%    \begin{macrocode}
\def\@opargbegintheorem##1##2##3{%
  \item[\rlap{\vbox{\hbox{\hskip \labelsep \theorem@headerfont
                     ##1\ ##2\ (##3)}%
                    \hbox{\strut}}}]}}
\endgroup
%</thb>
%    \end{macrocode}
% \end{macro}
%
%
% \subsubsection{The \texttt{changebreak} style}
%
%    \begin{macrocode}
% This style option is stored in the file |thcb.sty|.
%    \begin{macrocode}
%<*thcb>
\begingroup \makeatletter
\@ifundefined{theorem@style}{\input{theorem.sty}}{}
\ProvidesFile{thcb.sty}
             [\FMithmInfo]
%    \end{macrocode}
%
% \begin{macro}{\th@changebreak}
%    The \texttt{change-break} theorem style is like \texttt{break}
%    but with interchange of theorem name and theorem number.  Thus we
%    define |\th@changebreak| as follows:
%    \begin{macrocode}
\gdef\th@changebreak{\normalfont\slshape
  \def\@begintheorem##1##2{\item
       [\rlap{\vbox{\hbox{\hskip\labelsep \theorem@headerfont ##2\ ##1}%
                    \hbox{\strut}}}]}%
\def\@opargbegintheorem##1##2##3{%
  \item[\rlap{\vbox{\hbox{\hskip\labelsep \theorem@headerfont
                     ##2\ ##1\ (##3)}%
                    \hbox{\strut}}}]}}
\endgroup
%</thcb>
%    \end{macrocode}
% \end{macro}
%
%
% \subsubsection{The \texttt{change} style}
%
% This style option is stored in the file |thc.sty|.
%    \begin{macrocode}
%<*thc>
\begingroup \makeatletter
\@ifundefined{theorem@style}{\input{theorem.sty}}{}
\ProvidesFile{thc.sty}
             [\FMithmInfo]
%    \end{macrocode}
%
% \begin{macro}{\th@change}
%    The \texttt{change} theorem style corresponds to the
%    \texttt{change break} style without a linebreak after the header.
%    To say it in another way, it's the same as the \texttt{plain}
%    style but with number and name interchanged and |\slshape| as the
%    default font.
%    \begin{macrocode}
\gdef\th@change{\normalfont\slshape
  \def\@begintheorem##1##2{\item
       [\hskip\labelsep \theorem@headerfont ##2\ ##1]}%
\def\@opargbegintheorem##1##2##3{%
  \item[\hskip\labelsep \theorem@headerfont ##2\ ##1\ (##3)]}}
\endgroup
%</thc>
%    \end{macrocode}
% \end{macro}
%
%
% \subsubsection{The \texttt{marginbreak} style}
%
% This style option is the one used most often at Mainz.  It is saved
% in the file |thmb.sty|.
%    \begin{macrocode}
%<*thmb>
\begingroup \makeatletter
\@ifundefined{theorem@style}{\input{theorem.sty}}{}
\ProvidesFile{thmb.sty}
             [\FMithmInfo]
%    \end{macrocode}
%
% \begin{macro}{\th@marginbreak}
%    The \texttt{margin break} style is nearly the same as the
%    \texttt{change break} style. The only difference is the placement
%    of the theorem number.  We use |\llap| to place it in the left
%    margin.
%
%    In this style |\labelsep| denotes the separation between the
%    number and the text.
%    \begin{macrocode}
\gdef\th@marginbreak{\normalfont\slshape
  \def\@begintheorem##1##2{\item
       [\rlap{\vbox{\theorem@headerfont
                    \hbox{\llap{##2}\hskip\labelsep ##1}%
                    \hbox{\strut}}}]}%
\def\@opargbegintheorem##1##2##3{%
  \item[\rlap{\vbox{\theorem@headerfont
                    \hbox{\llap{##2}\hskip\labelsep ##1\ (##3)}%
                    \hbox{\strut}}}]}}
\endgroup
%</thmb>
%    \end{macrocode}
% \changes{v2.1c}{90/02/17}{`endgroup outside macrocode env.}
% \end{macro}
%
%
%
% \subsubsection{The \texttt{margin} style}
%
% This style option is stored in the file |thm.sty|.
%    \begin{macrocode}
%<*thm>
\begingroup \makeatletter
\@ifundefined{theorem@style}{\input{theorem.sty}}{}
\ProvidesFile{thm.sty}
             [\FMithmInfo]
%    \end{macrocode}
%
% \begin{macro}{\th@margin}
%    Again this is only a variant of the theorem styles described
%    above without any new ideas.
%    \begin{macrocode}
\gdef\th@margin{\normalfont\slshape
  \def\@begintheorem##1##2{\item
       [\theorem@headerfont \llap{##2}\hskip\labelsep ##1]}%
\def\@opargbegintheorem##1##2##3{%
  \item[\theorem@headerfont \llap{##2}\hskip\labelsep ##1\ (##3)]}}
\endgroup
%</thm>
%    \end{macrocode}
% \end{macro}
%
%
%
%
% \subsection{Final Definitions}
%
% \begin{macro}{\theorempreskipamount}
% \begin{macro}{\theorempostskipamount}
%    The \textsf{skip} parameters that regulate the vertical empty space
%    before and after the {\envfont theorem} environment have
%    to be allocated as well.
% \changes{v2.0f}{89/07/20}{Unnecessary `global' removed.}
%    \begin{macrocode}
%<*package>
\newskip\theorempreskipamount
\newskip\theorempostskipamount
%    \end{macrocode}
%    Since we have used the same values for all theorem sets, we now can
%    assign them.
%    \begin{macrocode}
\global\setlength\theorempreskipamount{12pt plus 5pt minus 3pt}
\global\setlength\theorempostskipamount{8pt plus 3pt minus 1.5pt}
%    \end{macrocode}
% \end{macro}
% \end{macro}
%
%
% \begin{macro}{\@endtheorem}
%    The same holds for the macro |\@endtheorem|, which ends a
%    {\envfont theorem} environment. Since it is the same for all
%    theorem sets, it is removed from the macros |\th@|\meta{style}.
%    It simply ends the {\envfont trivlist} environment, which was
%    begun in |\@thm|.
%    \begin{macrocode}
\global\let\@endtheorem=\endtrivlist
%    \end{macrocode}
% \end{macro}
%
%
%
% \begin{macro}{\@preamblecmds}
%    All macros defined above are to be used only in the preamble.
%    Therefore, we insert them in |\@preamblecmds| which will disable
%    them at begin document. This is done by the internal \LaTeXe{}
%    command |\@onlypreamble|.
% \changes{v2.0g}{89/09/19}{`theoremheaderfont and `theorembodyfont
%    added.}
% \changes{v2.2a}{94/02/03}{Use \LaTeXe{} interface}
%    \begin{macrocode}
\@onlypreamble\@xnthm
\@onlypreamble\@ynthm
\@onlypreamble\@othm
\@onlypreamble\newtheorem
\@onlypreamble\theoremstyle
\@onlypreamble\theorembodyfont
\@onlypreamble\theoremheaderfont
%    \end{macrocode}
% \end{macro}
%
% Finally we declare the \texttt{plain} theorem style to be the default.
% \changes{v2.1b}{90/06/04}{Fixed missing `item by declaring default
%    style.}
%    \begin{macrocode}
\theoremstyle{plain}
%</package>
%    \end{macrocode}
%
% \Finale

\endinput
}{}
\ProvidesFile{thp.sty}
             [\FMithmInfo]
%    \end{macrocode}
%
%
% \begin{macro}{\th@plain}
% \begin{macro}{\@begintheorem}
% \begin{macro}{\@opargbegintheorem}
%    |\theoremstyle{plain}| corresponds to the original
%    definition, except that the distances to the surrounding text are
%    determined by the parameters |\theorempreskipamount| and
%    |\theorempostskipamount|.
%    First we set the default body font.
%    \begin{macrocode}
\gdef\th@plain{\normalfont\itshape
%    \end{macrocode}
%    Then we define |\@begintheorem| and
%    |\@opargbegintheorem|. These two macros define how the
%    header of a theorem is typeset.  |\@opargbegintheorem| will
%    be called if a \textsf{theorem} environment with an optional
%    argument is encountered; otherwise, the header is constructed by
%    calling |\@begintheorem|.  If one of these macros is
%    executed, we are within a \textsf{trivlist} environment started by
%    |\@thm|.  So the theorem header is produced with an
%    |\item| command.
%
%    Instead of specifying the header font directly, all standard
%    theorem styles use the |\theorem@headerfont| macro to allow
%    customization.  The extra space (|\labelsep|) is necessary
%    because of problems in the \textsf{trivlist} environment.
%    \begin{macrocode}
  \def\@begintheorem##1##2{%
        \item[\hskip\labelsep \theorem@headerfont ##1\ ##2]}%
%    \end{macrocode}
%    The definition of |\@opargbegintheorem| is completely
%    analogous. The only difference is the fact that there exists a
%    third argument (which is the optional parameter of the
%    environment and contains additional information about the
%    theorem).  Customarily we enclose it in parentheses.
%    \begin{macrocode}
\def\@opargbegintheorem##1##2##3{%
   \item[\hskip\labelsep \theorem@headerfont ##1\ ##2\ (##3)]}}
%    \end{macrocode}
% \end{macro}
% \end{macro}
% \end{macro}
%
% \noindent
% We conclude with an |\endgroup| to restore the
% |\catcode| of |@|.
%    \begin{macrocode}
\endgroup
%</thp>
%    \end{macrocode}
%
%
% \subsubsection{The \texttt{break} style}
%
% This style option is stored in the file |thb.sty|.
% For the next two lines see the documentation for |\th@plain| on
% page~\pageref{dok:th@plain}.
%    \begin{macrocode}
%<*thb>
\begingroup \makeatletter
\@ifundefined{theorem@style}{% \iffalse meta-comment
%
% Copyright (C) 1993-2024
%
% The LaTeX Project and any individual authors listed elsewhere
% in this file.
%
% This file is part of the Standard LaTeX `Tools Bundle'.
% -------------------------------------------------------
%
% It may be distributed and/or modified under the
% conditions of the LaTeX Project Public License, either version 1.3c
% of this license or (at your option) any later version.
% The latest version of this license is in
%    https://www.latex-project.org/lppl.txt
% and version 1.3c or later is part of all distributions of LaTeX
% version 2005/12/01 or later.
%
% The list of all files belonging to the LaTeX `Tools Bundle' is
% given in the file `manifest.txt'.
%
% \fi
%
%%
%
%
\def\FMithmInfo{2023/07/05 v2.2c Theorem extension package (FMi)}
%
% \ProvidesFile{theorem.dtx}[\FMithmInfo]
%\iffalse   % this is a METACOMMENT !
%%
%% Package `theorem' to use with LaTeX2e
%% Copyright (C) 1989-1995 Frank Mittelbach, all rights reserved.
%<+package>\@ifundefined{theorem@style}{}{\endinput}
%<+package>\NeedsTeXFormat{LaTeX2e}[1995/06/01]
%<+package>\ProvidesPackage{theorem}
%<+package>         [\FMithmInfo]
%
% \fi
%
% \changes{v1.0}{87/05/07}{First draft.}
% \changes{v1.0a}{87/05/07}{`thistheoremstyle' added.}
% \changes{v1.1a}{87/07/29}{New concept: redefine the internal
%                          \protect\LaTeX{} commands `@xnthm' etc.}
% \changes{v1.1b}{87/07/31}{`theoremfont' used to specify font.}
% \changes{v1.2a}{88/01/08}{`@endtheorem' now globally defined.}
% \changes{v1.2b}{88/01/11}{`newtheorem' only allowed in the preamble.}
% \changes{v1.3a}{88/02/14}{`theoremfont' now a toks register.}
%
% \changes{v2.0a}{88/11/16}{New implementation and new documentation.}
% \changes{v2.0a}{88/11/16}{No layout changes but internal modifications
%                           old substyles won't work.}
% \changes{v2.0b}{89/01/30}{`global' used for all definitions.}
% \changes{v2.0c}{89/06/15}{Documentation translated to English.
%                            Thanks to LLL and Nina}
% \changes{v2.1a}{90/02/15}{Support for new font selection.}
% \changes{v2.2a}{94/02/03}{Cleanup for LaTeX2e}
%
% \DoNotIndex{\@endparenv,\@ifdefinable,\@ifnextchar,\@ifundefined}
% \DoNotIndex{\@namedef,\@nameuse,\@spaces,\@tempa,\@warning}
% \DoNotIndex{\ }
% \DoNotIndex{\begingroup,\bfseries}
% \DoNotIndex{\csname}
% \DoNotIndex{\def,\do,\docdate}
% \DoNotIndex{\endcsname,\endgroup,\endinput,\endtrivlist,\expandafter}
% \DoNotIndex{\filedate,\fileversion}
% \DoNotIndex{\gdef,\global}
% \DoNotIndex{\hbox,\hskip}
% \DoNotIndex{\ignorespaces,\input,\itshape,\item}
% \DoNotIndex{\labelsep,\let,\llap}
% \DoNotIndex{\makeatletter}
% \DoNotIndex{\noexpand}
% \DoNotIndex{\refstepcounter,\relax,\rlap}
% \DoNotIndex{\setlength,\slshape,\space,\string,\strut}
% \DoNotIndex{\the,\trivlist,\typeout}
% \DoNotIndex{\vbox}
% \DoNotIndex{\xdef}
%
% \def\deflabel#1{\ttfamily #1\hfill}
% \def\deflist#1{\begin{list}{}{\settowidth\labelwidth{\ttfamily #1}%
%                               \setlength\leftmargin\labelwidth
%                               \addtolength\leftmargin\labelsep
%                               \let\makelabel\deflabel}}
% \def\enddeflist{\end{list}}
% \def\envfont{\normalfont\ttfamily}
%
% \MakeShortVerb{\|}
%
% \newcommand{\TUB}{{\sl TUGboat\/}}
%
% \GetFileInfo{theorem.dtx}
% \title{An Extension of the
%        \LaTeX{} theorem environment\thanks{This file
%        has version number \fileversion, last
%        revised \filedate.}}
% \author{Frank Mittelbach}
%
% \MaintainedByLaTeXTeam{tools}
% \maketitle
%
% \begin{abstract}
%    The macros described in this paper yield an extension of the
%    \LaTeX{} theorem mechanism. It is designed to satisfy the
%    different requirements of various journals. Thus, the
%    layout of the ``theorems'' can be manipulated by determining a
%    ``style''. This article describes not only the use, but
%    also the definition, of the necessary macros.
% \end{abstract}
%
% \section*{Preface to version 2.2}
%
% For \LaTeXe{} this package did not need any fundamental changes. I
% only modified the messages generated so that theorem layout styles
% will show up with the |\listfiles| command and cleaned the section
% on the New Font Selection Scheme since this is now included in
% \LaTeX.
%
% \section*{Preface to version 2.1}
%
% This version is identical to 2.0g described in \TUB\ 10\#3 except
% for some internal defaults which are now set depending on the used
% font selection scheme.
%
% This was done to avoid unpleasant surprises if the new font selection
% scheme is in force. For further details see
% section~\ref{sec:newfonts} and \cite{bk:GMS94}.
%
% \section{Introduction}
%
% For our purposes here, ``theorems'' are labelled enunciations,
% often set off from the main text by extra space and a font change.
% Theorems, corollaries, conjectures, definitions, and remarks
% are all instances of ``theorems''.  The ``header'' of these structures
% is composed of a label (such as \textsc{Theorem} or \textsc{Remark})
% and a number which serializes an item in the sequence of items
% with the same label.
%
% Shortly after the introduction of \LaTeX{} at the Fachbereich
% Mathematik in Mainz, the desire to manipulate the layout of
% ``theorems'' arose.  In Mainz, the following two conventions came
% into general use:
% \begin{enumerate}
%   \item The number of the theorem is shown in the margin.
%   \item There is a line break at
%         the end of the theorem header.
% \end{enumerate}
%
% Additionally, some journals require different formats which depend on
% the ``sort of theorem'': e.g.\ often remarks and definitions are set
% in |\upshape|, while |\itshape| is employed for main theorems.
%
% Confronted with these requirements, a theorem environment was
% developed in Mainz which allows separate determination of the
% layout of the ``theorems sets'', comparable to |\pagestyle|.
%
%
% \section{The user interface}
% \subsection{Defining new theorem sets}
%
% \DescribeMacro\newtheorem
% As in the original \LaTeX{} version, the command |\newtheorem|
% defines a new ``theorem set'' or ``theorem-like structure''.
% Two required arguments name the new environment and give the text
% to be typeset with each instance of the new ``set'', while
% an optional argument determines how
% the ``set'' is enumerated:
% \begin{description}
%    \item[\ttfamily \bslash newtheorem\{foo\}\{bar\}]
%       The theorem set {\envfont foo} (whose name is \texttt{bar})
%       uses its own counter.
%    \item[\ttfamily \bslash newtheorem\{foo2\}{[foo]}\{bar2\}]
%       The theorem set {\envfont foo2} (printed name \texttt{bar2})
%       uses the same counter as the theorem set \texttt{foo}.
%    \item[\ttfamily \bslash newtheorem\{foo3\}\{bar3\}{[section]}]
%       The theorem set {\envfont foo3} (printed name \texttt{bar3}) is
%       enumerated within the counter \texttt{section}, i.e.\ with every
%       new |\section| the enumeration begins again with 1, and
%       the enumeration is composed from the section-number and the
%       theorem counter itself.
% \end{description}
%
% \DescribeMacro\theoremstyle
% Additionally, the command |\theoremstyle| can define the layout
% of various, or all, theorem sets.  It should be noted that any
% theorem set defined by |\newtheorem| is typeset in the
% |\theoremstyle| that is current at the time of the definition.
% Thus, the following
% \begin{verbatim}
%    \theoremstyle{break}        \newtheorem{Cor}{Corollary}
%    \theoremstyle{plain}        \newtheorem{Exa}{Example}[section]
%\end{verbatim}                             ^^A No blank at line start
% leads to the result that the set {\envfont Cor} is formatted in the
% style \texttt{break}, while the set {\envfont Exa} and all the
% following ones are formatted in the style \texttt{plain}, unless
% another |\theoremstyle| follows.  Since the definitions installed by
% |\newtheorem| are global, one also can limit |\theoremstyle| locally
% by grouping braces.
%
% \DescribeMacro\theorembodyfont
% The choice of the font for the theorem body is completely
% independent of the chosen |\theoremstyle|; this has proven to
% be very advantageous.  For example,
% \begin{verbatim}
%   {\theorembodyfont{\upshape}          \newtheorem{Rem}{Remark}}
%\end{verbatim}
% defines a theorem set {\envfont Rem}, which will be set in
% |\upshape| in the current layout (which in our example is
% \texttt{plain}).  As with |\theoremstyle|, the |\theorembodyfont|
% chosen is that current at the time of |\newtheorem|.  If
% |\theorembodyfont| is not specified or one defines
% |\theorembodyfont{}|, then the font used will be that defined by the
% |\theoremstyle|.
%
% \DescribeMacro\theoremheaderfont It is also possible to customize
% the font used for the theorem headers.  This is, however, a global
% declaration, and therefore there should be at most one
% |\theoremheaderfont| declaration in the preamble.\footnote{If it is
% actually necessary to have different header fonts, one has to define
% new theorem styles (substituting the desired font) or specify the
% information directly in the \texttt{\bslash newtheorem} declaration
% (the unclean variant).}
%
% \DescribeMacro\theorempreskipamount
% \DescribeMacro\theorempostskipamount Two additional parameters
% affect the vertical space around the theorem environments:
% |\theorempreskipamount| and |\theorempostskipamount| define,
% respectively, the spacing before and after such an environment.
% These parameters apply for all theorem sets and can be manipulated
% with the ordinary length macros.  They are rubber lengths
% (`\textsf{skips}'), and therefore can contain \texttt{plus} and
% \texttt{minus} parts.
%
% Since the definition of theorem sets should---most sensibly---be
% placed in the preamble, we only allow installation there.
% It is therefore possible to release the memory used here after
% |\begin{document}|, in order to make room for other
% applications.
%
%
% \subsection{Existing theorem styles}
%
% The following theorem styles exist to date:
% \begin{deflist}{marginbreak}
%    \item[plain]
%       This theorem style emulates the original \LaTeX{} definition,
%       except that additionally the parameters
%       |\theorem...skipamount| are used.
%    \item[break]
%       In this style, the theorem header is followed by a line
%       break.
%    \item[marginbreak]
%       The theorem number is set in the margin, and there is a line
%       break as in \texttt{break}.
%    \item[changebreak]
%       Like \texttt{break}, but with header number and text
%       interchanged.
%    \item[change]
%       Header number and text are interchanged, without a line break.
%    \item[margin]
%       The number is set in the left margin, without a line break.
% \end{deflist}
% All styles (except \texttt{plain}) select |\slshape| as the default
% |\theorembodyfont|.
%
%
% \subsection{Examples}
%
% Given the above theorem sets {\envfont Cor}, {\envfont Exa}
% and {\envfont Rem},
% suppose that the preamble also contains the declarations:
% \begin{verbatim}
%    \theoremstyle{marginbreak}   \newtheorem{Lem}[Cor]{Lemma}
%    \theoremstyle{change}
%    \theorembodyfont{\itshape}        \newtheorem{Def}[Cor]{Definition}
%
%    \theoremheaderfont{\scshape}
%\end{verbatim}
% Then the following are some typical examples of the typeset output
% resulting from their use.
%
% \begin{Cor}
%    This is a sentence typeset in the theorem environment {\envfont
%    Cor}.
% \end{Cor}
%
% \begin{Exa}
%    This is a sentence typeset in the theorem environment {\envfont
%    Exa}.
% \end{Exa}
%
% \begin{Rem}
%    This is a sentence typeset in the theorem environment {\envfont
%    Rem}.
% \end{Rem}
%
% \begin{Lem}[Ben User]
%    This is a sentence typeset in the theorem environment {\envfont
%    Lem}.
% \end{Lem}
%
% \begin{Def}[Very impressive Definition]
%    This is a sentence typeset in the theorem environment {\envfont
%    Def}.
% \end{Def}
%
% The last two examples show the effect of the optional argument to
% a theorem environment (it is the text typeset in parentheses).
%
%
% \section{Special Considerations}
% \label{sec:newfonts}
%
% Theoremheader and body are implemented as a unit. This means that
% the |\theoremheaderfont|  will inherit characteristics of the
% |\theorembodyfont| in \LaTeXe. Thus, if for example
% |\theorembodyfont| is |\itshape| and |\theoremheaderfont|
% is |\bfseries| the font selected for the header will have the
% characteristics `bold extended italic'. If this is not desired one
% should set the |\theoremheaderfont| to something like
% \begin{verbatim}
%      \theoremheaderfont{\normalfont\bfseries}
%\end{verbatim}
% i.e.\ supplying all necessary font information explicitly.
%
%
%
% \section{Acknowledgements}
%
% The publication of this set of macros was only possible with the
% help of Christina Busse (translating the manuscript into English),
% Joachim Pense (playing the r\^ole of typist), Chris Rowley
% (looking everything over) and many others providing useful
% suggestions.
%
% \MaybeStop{%
%    \begin{thebibliography}{9}
%       \bibitem{bk:GMS94} \textsc{M.~Goossens}, \textsc{F.~Mittelbach}
%         and \textsc{A.~Samarin}.
%         \newblock The \LaTeX{} Companion.
%         \newblock
%         Addison-Wesley, Reading, Massachusetts, 1994.
%       \bibitem{latex:source} \textsc{Lamport, Leslie.} \newblock
%                               \texttt{latex.tex}, version 2.09, date
%                               Feb. 1990.
%    \end{thebibliography}
%    \PrintIndex
%    \PrintChanges}
%
%
% \section{The documentation driver file}
%
% The next bit of code contains the documentation driver file for
% \TeX{}, i.e., the file that will produce the documentation you are
% currently reading. It will be extracted from this file by the
% \texttt{docstrip} program. Since it is the first code in the file
% one can alternatively process this file directly with \LaTeXe{} to
% obtain the documentation.
%
%    \begin{macrocode}
%<*driver>
\documentclass{ltxdoc}

\usepackage{theorem}

%<+driver>% The next few lines define theorem sets which are used
%<+driver>% in the example section of the documentation.

\theoremstyle{break}       \newtheorem{Cor}{Corollary}
\theoremstyle{plain}       \newtheorem{Exa}{Example}[section]
{\theorembodyfont{\upshape}\newtheorem{Rem}{Remark}}
\theoremstyle{marginbreak} \newtheorem{Lem}[Cor]{Lemma}
\theoremstyle{change}
\theorembodyfont{\itshape} \newtheorem{Def}[Cor]{Definition}

\theoremheaderfont{\scshape}

\RecordChanges

\begin{document}
   \DocInput{theorem.dtx}
\end{document}
%</driver>
%    \end{macrocode}
%
% \section{Definition of the Macros}
%
% If the file has been loaded before, we abort immediately. If not the
% package announces itself (this is actually done at the very top of
% the file---the way it is done isn't good style so don't copy it).
% \changes{v2.0e}{89/07/19}{Spaces removed from `typeout'.}
% \changes{v2.2b}{95/11/19}{Announce moved to top of file}
%    \begin{macrocode}
%<*package>
%\@ifundefined{theorem@style}{}{\endinput}
%\def\FMithmInfo{1995/11/19 v2.2b Theorem extension package (FMi)}
%\ProvidesPackage{theorem}[\FMithmInfo]
%    \end{macrocode}
%
%
% \subsection{Definition of theorem styles and fonts}
%
% All the definitions in this file are done globally to allow
% inputting this file inside a group.
%
% \begin{macro}{\theoremstyle}
%    Before a theorem style can be installed, the chosen style must be
%    known. For that reason, we must test to see that
%    |\th@|\meta{style} is known or, more precisely, that it is
%    different from |\relax|. If the style is not known then
%    |\th@plain| is used.
%    \begin{macrocode}
\gdef\theoremstyle#1{%
   \@ifundefined{th@#1}{\@warning
          {Unknown theoremstyle `#1'. Using `plain'}%
          \theorem@style{plain}}%
%    \end{macrocode}
%    We save the theorem style to be used in the token register
%    |\theorem@style|.
%    \begin{macrocode}
      {\theorem@style{#1}}%
%    \end{macrocode}
%    Now we ``evaluate'' the theorem style: this means, we call the
%    macro |\th@|\meta{style} which will activate the relevant
%    definitions which are contained in a separate file.  This is done
%    in a group to suppress changes to the current font.  This could
%    otherwise pose problems together with the new font selection
%    scheme\footnote{When I printed the original article using the new
%    font selection scheme I ended with a document with slanted
%    typefaces (text headings and all) simply because one of the
%    theorem styles used \texttt{\bslash sl} at toplevel.} if the
%    |\th@|\meta{style} is evaluated a second time.
% \changes{v2.1a}{90/02/15}{Group added around `th@..}
% \changes{v2.1a}{90/02/15}{Reseting of `..@begintheorem removed.}
%    \begin{macrocode}
      \begingroup
        \csname th@\the\theorem@style \endcsname
      \endgroup}
%    \end{macrocode}
% \end{macro}
%
%
% \begin{macro}{\@begintheorem}
% \changes{v2.1a}{90/02/15}{Relaxed at toplevel.}
% \begin{macro}{\@opargbegintheorem}
% \changes{v2.1a}{90/02/15}{Relaxed at toplevel.}
%    We reset |\@begintheorem| and |\@opargbegintheorem| to
%    |\relax| since these commands are no longer necessary at
%    toplevel. This will save a few tokens.
%    \begin{macrocode}
\global\let\@begintheorem\relax
\global\let\@opargbegintheorem\relax
%    \end{macrocode}
% \end{macro}
% \end{macro}
%
% \begin{macro}{\theorem@style}
%    Obviously the \textsf{token} register used above has to be
%    allocated. To assure the utmost compatibility with the
%    original \LaTeX{} definition, we set the default theorem style to
%    \texttt{plain}, which implements the usual \LaTeX{} convention.
% \changes{v2.0f}{89/07/20}{Unnecessary `global' removed.}
%    \begin{macrocode}
\newtoks\theorem@style
\global\theorem@style{plain}
%    \end{macrocode}
% \end{macro}
%
% \begin{macro}{\theorembodyfont}
% \begin{macro}{\theorem@bodyfont}
%    If the |\theorembodyfont| is set by the user then it should not
%    interact with the default font set in the theorem style. When the
%    new font selection is in force this may happen if, for example,
%    the default is |\itshape| and the new |\theorembodyfont| is
%    |\sffamily|. So we add a |\reset@font| command in front of the
%    user definition.
% \changes{v2.0f}{89/07/20}{Unnecessary `global' removed.}
% \changes{v2.0g}{89/09/19}{`theoremfont to `theorembodyfont.}
% \changes{v2.1c}{92/06/22}{`reset@font added and logic changed.}
% \changes{v2.1d}{92/10/28}{Reset to empty if arg is empty}
%    \begin{macrocode}
\gdef\theorembodyfont#1{%
%    \end{macrocode}
%    We check if the argument supplied is empty and if so put nothing
%    into the |\theorem@bodyfont| token register to allow for
%    |\theorembodyfont{}| as a mean of using the default of the
%    current |\theoremstyle|.
%    \begin{macrocode}
   \def\@tempa{#1}%
   \ifx\@tempa\@empty
    \theorem@bodyfont{}%
   \else
    \theorem@bodyfont{\reset@font#1}%
   \fi
   }
\newtoks\theorem@bodyfont
\global\theorem@bodyfont{}
%    \end{macrocode}
% \end{macro}
% \end{macro}
%
%
% \begin{macro}{\theoremheaderfont}
% \changes{v2.0g}{89/09/19}{Macro added.}
%    The font for the theorem headers is handled differently because
%    this definition applies to all theorem styles.
%    \begin{macrocode}
\gdef\theoremheaderfont#1{\gdef\theorem@headerfont{#1}%
%    \end{macrocode}
%    After using the macro once it is redefined to produce an error
%    message.
%    \begin{macrocode}
       \gdef\theoremheaderfont##1{%
        \typeout{\string\theoremheaderfont\space should be used
                 only once.}}}
%    \end{macrocode}
% \end{macro}
%
% \begin{macro}{\theorem@headerfont}
% \changes{v2.0g}{89/09/19}{Macro added.}
%    To set the |\theorem@headerfont| default we first test if
%    the new fontselection scheme is in force.
% \changes{v2.1a}{90/02/15}{Support for new font selection.}
%    \begin{macrocode}
\ifx\upshape\undefined
%    \end{macrocode}
%    If not we define it to expand into |\bfseries|. We don't use
%    |\let| just in case a following style option redefines this
%    macro.
%    \begin{macrocode}
\gdef\theorem@headerfont{\bfseries}
%    \end{macrocode}
%    Otherwise we reset the current shape before calling |\bfseries|.
%    \begin{macrocode}
\else \gdef\theorem@headerfont{\normalfont\bfseries}\fi
%    \end{macrocode}
% \end{macro}
%
% \begin{macro}{\th@plain}
% \begin{macro}{\th@break}
% \begin{macro}{\th@marginbreak}
% \begin{macro}{\th@changebreak}
% \begin{macro}{\th@change}
% \begin{macro}{\th@margin}
%    The different styles are defined in macros such as
%    |\th@plain|.  Since memory space is precious in
%    ``non-Big-versions'', we have to avoid offering too many unused
%    definitions.  Therefore we define these styles in separate files
%    that can be loaded on demand.  Thus the commands themselves only
%    load these files. We use |\@input@|, a \LaTeXe{} internal command
%    that ensures that the file will be listed with |\listfiles|.
% \changes{v2.2a}{94/02/02}{Use `@input@ to load theorem layout files}
%    \begin{macrocode}
\gdef\th@plain{\@input@{thp.sty}}
\gdef\th@break{\@input@{thb.sty}}
\gdef\th@marginbreak{\@input@{thmb.sty}}
\gdef\th@changebreak{\@input@{thcb.sty}}
\gdef\th@change{\@input@{thc.sty}}
\gdef\th@margin{\@input@{thm.sty}}
%    \end{macrocode}
%    This list will be expanded when new styles become
%    available.  For testing,  just append new theorem substyles
%    as document options.
% \end{macro}
% \end{macro}
% \end{macro}
% \end{macro}
% \end{macro}
% \end{macro}
%
%
%
% \subsection{Definition of a new theorem set}
%
% As already pointed out, a new theorem environment can be defined in
% three different ways:
% \begin{verbatim}
%    \newtheorem{Lem}{Lemma}
%    \newtheorem{Lem}{Lemma}[section]
%    \newtheorem{Lem}[Theorem]{Lemma}
%\end{verbatim}
% The function of the macro |\newtheorem| is to recognize these cases
% and then to branch into one of the three macros |\@ynthm|, |\@xnthm|
% or |\@othm|.  This mechanism is adopted unchanged from
% \cite{latex:source}; the essential point here is that, for example,
% in the second case, the arguments \texttt{Lem}, \texttt{Lemma} and
% \texttt{section} are passed over to the macro |\@xnthm|.
%
% We inspect this case first because the others present fewer problems,
% and thus are easily derived from this one.
%
% \begin{macro}{\@xnthm}
%    For our example arguments, the macro |\@xnthm| must fulfill
%    the following:
%    \begin{itemize}
%       \item
%          Define a new \LaTeX-counter `Lem'
%       \item
%          reset this counter within a |\section|
%       \item
%          define the macro |\theLem|
%       \item
%          define the environment macros |\Lem| and
%          |\endLem| using the current
%          |\theoremstyle| and |\theorem@bodyfont|.
%    \end{itemize}
%    Obviously, all this should happen only if the first argument
%    of |\@xnthm| (i.e.\ \texttt{Lem} in our example) is chosen so
%    as not to conflict with any previously defined commands or
%    environments.  This test is performed by the \LaTeX{} macro
%    |\@ifdefinable|.
%    \begin{macrocode}
\gdef\@xnthm#1#2[#3]{\expandafter\@ifdefinable\csname #1\endcsname
%    \end{macrocode}
%    Therefore, the first argument of |\@ifdefinable| is the
%    expansion (in the example, |\Lem|) of
%    |\csname||#1\endcsname|.  The second argument is
%    executed only if the test has been completed successfully.
%    \begin{macrocode}
   {%
%    \end{macrocode}
%    Now we define the new counter. The names of the \LaTeX{}
%    macros employed should speak for themselves:
% \changes{v2.2b}{1995/10/07}{Use \cs{@newctr}}
% \changes{v2.2c}{1995/11/23}{Use \cs{@newctr} properly.}
%    \begin{macrocode}
    \@definecounter{#1}\@newctr{#1}[#3]%
%    \end{macrocode}
%    Using |\@newctr| will give a proper error message if the counter
%    in |#3| is not defined.
%    In defining `|\theLem|' we must generate the desired macro
%    name by use of |\expandafter| and |\csname|.
%    \begin{macrocode}
    \expandafter\xdef\csname the#1\endcsname
%    \end{macrocode}
%    An |\xdef| is used in order to make the definition global,
%    and to ensure that it contains the replacement texts of
%    |\@thmcountersep| and |\@thmcounter|.\footnote{These
%    two macros can be defined by the document style.  Their default
%    values produce a `\texttt{.}' as separation and an arabic
%    representation of the number.} However, not everything should be
%    expanded. For example, it saves space to use |\thesection|
%    instead of its---at times---lengthy expansion.
%    \begin{macrocode}
      {\expandafter \noexpand \csname the#3\endcsname
       \@thmcountersep \@thmcounter{#1}}%
%    \end{macrocode}
%    Thus with the defaults of \LaTeX{}, |\theLem| would be
%    replaced by the command sequence |\thesection.\arabic{Lem}|.
%
%    We will now look at the definition of the macro which is executed
%    at the beginning of the actual environment (in our example this
%    macro is |\Lem|). It should be noted that we use an
%    ``|\expandafter| trick'' to expand only certain parts of the
%    replacement text at the time of the definition.
%    \begin{macrocode}
    \def\@tempa{\global\@namedef{#1}}%
    \expandafter \@tempa \expandafter{%
%    \end{macrocode}
%    First, the macro that contains the current definitions of
%    |\@begintheorem| and |\@opargtheorem| should be called
%    up.  The name of this macro---as is already known---has the form
%    |\th@|\meta{theorem\ style}; therefore, it must be called by
%    \begin{macrocode}
      \csname th@\the \theorem@style
%    \end{macrocode}
%    In addition the default theorem font should be changeable, i.e.\
%    we have to insert the contents of |\theorem@bodyfont|. For
%    that reason, we expand even further, beyond |\endcsname|,
%    and thus insert the contents of the \textsf{token} register
%    |\theorem@bodyfont| in the replacement text.
%    \begin{macrocode}
            \expandafter \endcsname \the \theorem@bodyfont
%    \end{macrocode}
%    Now it is time to call the macro |\@thm| which takes over
%    the further processing.  It has two arguments: the current
%    counter name (in our example, \texttt{Lem}), and the text of the
%    label (in our example, \texttt{Lemma}).
%    \begin{macrocode}
     \@thm{#1}{#2}}%
%    \end{macrocode}
%    With this, the `sub-definition' is complete.  The macro
%    |\@endtheorem| ends a theorem environment and is, so far, nothing
%    but an |\endtrivlist|.  (Hence it is defined globally, and not
%    within the theorem styles.\footnote{This has to be changed as
%    soon as theorem styles that change \texttt{\bslash @endtheorem}
%    exist.  In such a case, all existing styles must be changed as
%    well since they will have to reset the macro.}) Therefore, we can
%    set it equivalent to the macro that ends the theorem set (in our
%    example, |\endLem|).  However, if some day theorem styles exist
%    that do change |\@endtheorem|, we would have to use the
%    commented-out line instead.
%    \begin{macrocode}
    \global \expandafter \let \csname end#1\endcsname \@endtheorem
%   \global\@namedef{end#1}{\@endtheorem}%
%    \end{macrocode}
%    With these commands all the required definitions are employed,
%    unless the test |\@ifdefinable| has failed.  Therefore, we
%    end the second argument of this macro and with it the definition
%    of |\@xnthm|.
%    \begin{macrocode}
   }}
%    \end{macrocode}
% \end{macro}
%
% \begin{macro}{\@ynthm}
%    The definition of |\@ynthm| is completely analogous. In
%    this case the new counter that is defined is not reset within
%    another counter; thus the definition of |\the...|
%    is simplified:
%    \begin{macrocode}
\gdef\@ynthm#1#2{\expandafter\@ifdefinable\csname #1\endcsname
   {\@definecounter{#1}%
    \expandafter\xdef\csname the#1\endcsname{\@thmcounter{#1}}%
%    \end{macrocode}
%    The rest of the definition corresponds literally to that of
%    |\@xnthm|:
%    \begin{macrocode}
    \def\@tempa{\global\@namedef{#1}}\expandafter \@tempa
     \expandafter{\csname th@\the \theorem@style \expandafter
     \endcsname \the\theorem@bodyfont \@thm{#1}{#2}}%
    \global \expandafter \let \csname end#1\endcsname \@endtheorem}}
%    \end{macrocode}
% \end{macro}
%
% \begin{macro}{\@othm}
%    The definition of |\@othm| does not contain anything new.
%    \begin{macrocode}
\gdef\@othm#1[#2]#3{%
%    \end{macrocode}
% \changes{v2.2b}{95/10/07}{Build in check for counter defined}
%    We do not define a new counter but instead use one that has
%    already been defined.  Thus the only definition we need is that
%    of this pseudo-counter (i.e.\ |\the|\meta{env.\ name}).
%    First we check if |#2| corresponds to a known counter name.
%    \begin{macrocode}
  \expandafter\ifx\csname c@#2\endcsname\relax
   \@nocounterr{#2}%
  \else
   \expandafter\@ifdefinable\csname #1\endcsname
   {\expandafter \xdef \csname the#1\endcsname
     {\expandafter \noexpand \csname the#2\endcsname}%
%    \end{macrocode}
%    All other parts of the definition can be adopted from
%    |\@xnthm|. We have to remember, though, that in this case
%    the name of the current counter and the theorem label have moved
%    to the second and third arguments.
%    \begin{macrocode}
    \def\@tempa{\global\@namedef{#1}}\expandafter \@tempa
     \expandafter{\csname th@\the \theorem@style \expandafter
     \endcsname \the\theorem@bodyfont \@thm{#2}{#3}}%
    \global \expandafter \let \csname end#1\endcsname \@endtheorem}%
  \fi}
%    \end{macrocode}
% \end{macro}
%
% \subsection{Macros that are employed in a {\envfont theorem}
%    environment}
%
% \begin{macro}{\@thm}
%    The macro |\@thm| has to increase the current counter.  Then,
%    depending on whether the environment has (or does not have) an
%    optional argument, it has to branch into either |\@begintheorem|
%    or |\@opargtheorem|.
%    \begin{macrocode}
\gdef\@thm#1#2{\refstepcounter{#1}%
%    \end{macrocode}
%    Now we start a {\envfont trivlist} environment, and give
%    |\@topsep| and |\@topsepadd| the values of the \textsf{skip}
%    registers |\theorempreskipamount| and |\theorempostskipamount|.
%    The value in |\@topsep| is the vertical space that is inserted by
%    the first (and only) |\item| in our |\trivlist| whilst
%    |\@topsepadd| is inserted by |\@endparenv| at the end of that
%    {\envfont trivlist} environment.  By using these registers, we
%    obtain the desired space around a {\envfont theorem} environment.
%    \begin{macrocode}
   \trivlist
   \@topsep \theorempreskipamount               % used by first \item
   \@topsepadd \theorempostskipamount           % used by \@endparenv
%    \end{macrocode}
%    Now we have to test whether an optional argument has been given.
% \changes{v2.0d}{89/07/18}{Forgotten percent added.}
%    \begin{macrocode}
   \@ifnextchar [%
%    \end{macrocode}
%    If there is an optional argument, we will call |\@ythm|, and
%    move the arguments read back into the input stream.
%    \begin{macrocode}
   {\@ythm{#1}{#2}}%
%    \end{macrocode}
%    If not, we call |\@begintheorem|. Its first argument is the
%    name of the theorem set (hence the second argument of
%    |\@thm|). Its second argument is the macro that produces the
%    current number.
%    \begin{macrocode}
   {\@begintheorem{#2}{\csname the#1\endcsname}\ignorespaces}}
%    \end{macrocode}
% \end{macro}
%
% \begin{macro}{\@xthm}
% \begin{macro}{\@ythm}
%    Both these macros were originally called by |\@thm|.  We do
%    not need |\@xthm| anymore, hence we reset it to
%    |\relax|. The definition of |\@ythm| has not changed at
%    all from its definition in \LaTeX. In order to make the macros
%    easier to understand, we will
%    nevertheless present it (commented out).
%    \begin{macrocode}
\global\let\@xthm\relax
% \def\@ythm#1#2[#3]{\@opargbegintheorem{#2}{\csname
%           the#1\endcsname}{#3}\ignorespaces}
%</package>
%    \end{macrocode}
%    The primitive |\ignorespaces| in |\@ythm| and
%    |\@thm| is needed to remove the spaces between the
%    |\begin{...}| and the actual text.
% \end{macro}
% \end{macro}
%
%
% \subsection{Definition of the theorem substyles}
%
% As already pointed out, the theorem substyles, defined below, are
% only loaded when necessary.  Note that all these substyles, except
% \texttt{plain}, have |\slshape| as the default body font.
%
% \subsubsection{The \texttt{plain} style}
%
% \label{dok:th@plain} As the following macros use |@|, we have
% to locally set the |\catcode| of this symbol to ``letter''.
% This happens within a group, so that we do not have to worry about
% which |\catcode| that symbol had before.
%    \begin{macrocode}
%<*thp>
\begingroup \makeatletter
%    \end{macrocode}
% Since we are now within a group, we must make all definitions
% globally.  First we make sure that \texttt{theorem.sty} is loaded.
% This will allow us to use this file as a document style option
% without having to call \texttt{theorem} itself as an option.  At the
% same time, we assure that at least version 2 is loaded, since
% |\theorem@style| was not defined in earlier versions.
%    \begin{macrocode}
\@ifundefined{theorem@style}{\input{theorem.sty}}{}
\ProvidesFile{thp.sty}
             [\FMithmInfo]
%    \end{macrocode}
%
%
% \begin{macro}{\th@plain}
% \begin{macro}{\@begintheorem}
% \begin{macro}{\@opargbegintheorem}
%    |\theoremstyle{plain}| corresponds to the original
%    definition, except that the distances to the surrounding text are
%    determined by the parameters |\theorempreskipamount| and
%    |\theorempostskipamount|.
%    First we set the default body font.
%    \begin{macrocode}
\gdef\th@plain{\normalfont\itshape
%    \end{macrocode}
%    Then we define |\@begintheorem| and
%    |\@opargbegintheorem|. These two macros define how the
%    header of a theorem is typeset.  |\@opargbegintheorem| will
%    be called if a \textsf{theorem} environment with an optional
%    argument is encountered; otherwise, the header is constructed by
%    calling |\@begintheorem|.  If one of these macros is
%    executed, we are within a \textsf{trivlist} environment started by
%    |\@thm|.  So the theorem header is produced with an
%    |\item| command.
%
%    Instead of specifying the header font directly, all standard
%    theorem styles use the |\theorem@headerfont| macro to allow
%    customization.  The extra space (|\labelsep|) is necessary
%    because of problems in the \textsf{trivlist} environment.
%    \begin{macrocode}
  \def\@begintheorem##1##2{%
        \item[\hskip\labelsep \theorem@headerfont ##1\ ##2]}%
%    \end{macrocode}
%    The definition of |\@opargbegintheorem| is completely
%    analogous. The only difference is the fact that there exists a
%    third argument (which is the optional parameter of the
%    environment and contains additional information about the
%    theorem).  Customarily we enclose it in parentheses.
%    \begin{macrocode}
\def\@opargbegintheorem##1##2##3{%
   \item[\hskip\labelsep \theorem@headerfont ##1\ ##2\ (##3)]}}
%    \end{macrocode}
% \end{macro}
% \end{macro}
% \end{macro}
%
% \noindent
% We conclude with an |\endgroup| to restore the
% |\catcode| of |@|.
%    \begin{macrocode}
\endgroup
%</thp>
%    \end{macrocode}
%
%
% \subsubsection{The \texttt{break} style}
%
% This style option is stored in the file |thb.sty|.
% For the next two lines see the documentation for |\th@plain| on
% page~\pageref{dok:th@plain}.
%    \begin{macrocode}
%<*thb>
\begingroup \makeatletter
\@ifundefined{theorem@style}{\input{theorem.sty}}{}
\ProvidesFile{thb.sty}
             [\FMithmInfo]
%    \end{macrocode}
%
% \begin{macro}{\th@break}
%    |\theoremstyle{break}| produces a line break after the name
%    of the theorem.  The font is |\slshape|.  Hence, we define
%    |\th@break| as follows:
%    \begin{macrocode}
\gdef\th@break{\normalfont\slshape
  \def\@begintheorem##1##2{\item[%
%    \end{macrocode}
%    We run into the following problem: it is not possible to create
%    the header with |\item[|\meta{title}|]| and then start
%    a new line by, for example, |\mbox{}\\|. Such a definition
%    will fail whenever a list environment follows immediately.  With
%    the above construction, the |\mbox{}| causes the switch
%    |@inlabel| (cf.\ definition of |\list| and
%    |\trivlist| in \cite{latex:source}) to be set to \texttt{false}
%    and so the following list will insert additional vertical
%    space (|\topskip|). This is quite annoying. Therefore, we
%    create the line break within the |\item|. In order to ensure
%    that the text will begin at the proper position in the following
%    line, we simply pretend that the label does not take any
%    room.\footnote{This will lead to problems whenever very high
%    symbols occurring in the line tower into the
%    heading. So, something else has to be done here sometime.}
%    \begin{macrocode}
       \rlap{\vbox{\hbox{\hskip \labelsep\theorem@headerfont ##1\ ##2}%
                   \hbox{\strut}}}]}%
%    \end{macrocode}
%    Again, the definition of |\@opargbegintheorem| is
%    completely analogous.
%    \begin{macrocode}
\def\@opargbegintheorem##1##2##3{%
  \item[\rlap{\vbox{\hbox{\hskip \labelsep \theorem@headerfont
                     ##1\ ##2\ (##3)}%
                    \hbox{\strut}}}]}}
\endgroup
%</thb>
%    \end{macrocode}
% \end{macro}
%
%
% \subsubsection{The \texttt{changebreak} style}
%
%    \begin{macrocode}
% This style option is stored in the file |thcb.sty|.
%    \begin{macrocode}
%<*thcb>
\begingroup \makeatletter
\@ifundefined{theorem@style}{\input{theorem.sty}}{}
\ProvidesFile{thcb.sty}
             [\FMithmInfo]
%    \end{macrocode}
%
% \begin{macro}{\th@changebreak}
%    The \texttt{change-break} theorem style is like \texttt{break}
%    but with interchange of theorem name and theorem number.  Thus we
%    define |\th@changebreak| as follows:
%    \begin{macrocode}
\gdef\th@changebreak{\normalfont\slshape
  \def\@begintheorem##1##2{\item
       [\rlap{\vbox{\hbox{\hskip\labelsep \theorem@headerfont ##2\ ##1}%
                    \hbox{\strut}}}]}%
\def\@opargbegintheorem##1##2##3{%
  \item[\rlap{\vbox{\hbox{\hskip\labelsep \theorem@headerfont
                     ##2\ ##1\ (##3)}%
                    \hbox{\strut}}}]}}
\endgroup
%</thcb>
%    \end{macrocode}
% \end{macro}
%
%
% \subsubsection{The \texttt{change} style}
%
% This style option is stored in the file |thc.sty|.
%    \begin{macrocode}
%<*thc>
\begingroup \makeatletter
\@ifundefined{theorem@style}{\input{theorem.sty}}{}
\ProvidesFile{thc.sty}
             [\FMithmInfo]
%    \end{macrocode}
%
% \begin{macro}{\th@change}
%    The \texttt{change} theorem style corresponds to the
%    \texttt{change break} style without a linebreak after the header.
%    To say it in another way, it's the same as the \texttt{plain}
%    style but with number and name interchanged and |\slshape| as the
%    default font.
%    \begin{macrocode}
\gdef\th@change{\normalfont\slshape
  \def\@begintheorem##1##2{\item
       [\hskip\labelsep \theorem@headerfont ##2\ ##1]}%
\def\@opargbegintheorem##1##2##3{%
  \item[\hskip\labelsep \theorem@headerfont ##2\ ##1\ (##3)]}}
\endgroup
%</thc>
%    \end{macrocode}
% \end{macro}
%
%
% \subsubsection{The \texttt{marginbreak} style}
%
% This style option is the one used most often at Mainz.  It is saved
% in the file |thmb.sty|.
%    \begin{macrocode}
%<*thmb>
\begingroup \makeatletter
\@ifundefined{theorem@style}{\input{theorem.sty}}{}
\ProvidesFile{thmb.sty}
             [\FMithmInfo]
%    \end{macrocode}
%
% \begin{macro}{\th@marginbreak}
%    The \texttt{margin break} style is nearly the same as the
%    \texttt{change break} style. The only difference is the placement
%    of the theorem number.  We use |\llap| to place it in the left
%    margin.
%
%    In this style |\labelsep| denotes the separation between the
%    number and the text.
%    \begin{macrocode}
\gdef\th@marginbreak{\normalfont\slshape
  \def\@begintheorem##1##2{\item
       [\rlap{\vbox{\theorem@headerfont
                    \hbox{\llap{##2}\hskip\labelsep ##1}%
                    \hbox{\strut}}}]}%
\def\@opargbegintheorem##1##2##3{%
  \item[\rlap{\vbox{\theorem@headerfont
                    \hbox{\llap{##2}\hskip\labelsep ##1\ (##3)}%
                    \hbox{\strut}}}]}}
\endgroup
%</thmb>
%    \end{macrocode}
% \changes{v2.1c}{90/02/17}{`endgroup outside macrocode env.}
% \end{macro}
%
%
%
% \subsubsection{The \texttt{margin} style}
%
% This style option is stored in the file |thm.sty|.
%    \begin{macrocode}
%<*thm>
\begingroup \makeatletter
\@ifundefined{theorem@style}{\input{theorem.sty}}{}
\ProvidesFile{thm.sty}
             [\FMithmInfo]
%    \end{macrocode}
%
% \begin{macro}{\th@margin}
%    Again this is only a variant of the theorem styles described
%    above without any new ideas.
%    \begin{macrocode}
\gdef\th@margin{\normalfont\slshape
  \def\@begintheorem##1##2{\item
       [\theorem@headerfont \llap{##2}\hskip\labelsep ##1]}%
\def\@opargbegintheorem##1##2##3{%
  \item[\theorem@headerfont \llap{##2}\hskip\labelsep ##1\ (##3)]}}
\endgroup
%</thm>
%    \end{macrocode}
% \end{macro}
%
%
%
%
% \subsection{Final Definitions}
%
% \begin{macro}{\theorempreskipamount}
% \begin{macro}{\theorempostskipamount}
%    The \textsf{skip} parameters that regulate the vertical empty space
%    before and after the {\envfont theorem} environment have
%    to be allocated as well.
% \changes{v2.0f}{89/07/20}{Unnecessary `global' removed.}
%    \begin{macrocode}
%<*package>
\newskip\theorempreskipamount
\newskip\theorempostskipamount
%    \end{macrocode}
%    Since we have used the same values for all theorem sets, we now can
%    assign them.
%    \begin{macrocode}
\global\setlength\theorempreskipamount{12pt plus 5pt minus 3pt}
\global\setlength\theorempostskipamount{8pt plus 3pt minus 1.5pt}
%    \end{macrocode}
% \end{macro}
% \end{macro}
%
%
% \begin{macro}{\@endtheorem}
%    The same holds for the macro |\@endtheorem|, which ends a
%    {\envfont theorem} environment. Since it is the same for all
%    theorem sets, it is removed from the macros |\th@|\meta{style}.
%    It simply ends the {\envfont trivlist} environment, which was
%    begun in |\@thm|.
%    \begin{macrocode}
\global\let\@endtheorem=\endtrivlist
%    \end{macrocode}
% \end{macro}
%
%
%
% \begin{macro}{\@preamblecmds}
%    All macros defined above are to be used only in the preamble.
%    Therefore, we insert them in |\@preamblecmds| which will disable
%    them at begin document. This is done by the internal \LaTeXe{}
%    command |\@onlypreamble|.
% \changes{v2.0g}{89/09/19}{`theoremheaderfont and `theorembodyfont
%    added.}
% \changes{v2.2a}{94/02/03}{Use \LaTeXe{} interface}
%    \begin{macrocode}
\@onlypreamble\@xnthm
\@onlypreamble\@ynthm
\@onlypreamble\@othm
\@onlypreamble\newtheorem
\@onlypreamble\theoremstyle
\@onlypreamble\theorembodyfont
\@onlypreamble\theoremheaderfont
%    \end{macrocode}
% \end{macro}
%
% Finally we declare the \texttt{plain} theorem style to be the default.
% \changes{v2.1b}{90/06/04}{Fixed missing `item by declaring default
%    style.}
%    \begin{macrocode}
\theoremstyle{plain}
%</package>
%    \end{macrocode}
%
% \Finale

\endinput
}{}
\ProvidesFile{thb.sty}
             [\FMithmInfo]
%    \end{macrocode}
%
% \begin{macro}{\th@break}
%    |\theoremstyle{break}| produces a line break after the name
%    of the theorem.  The font is |\slshape|.  Hence, we define
%    |\th@break| as follows:
%    \begin{macrocode}
\gdef\th@break{\normalfont\slshape
  \def\@begintheorem##1##2{\item[%
%    \end{macrocode}
%    We run into the following problem: it is not possible to create
%    the header with |\item[|\meta{title}|]| and then start
%    a new line by, for example, |\mbox{}\\|. Such a definition
%    will fail whenever a list environment follows immediately.  With
%    the above construction, the |\mbox{}| causes the switch
%    |@inlabel| (cf.\ definition of |\list| and
%    |\trivlist| in \cite{latex:source}) to be set to \texttt{false}
%    and so the following list will insert additional vertical
%    space (|\topskip|). This is quite annoying. Therefore, we
%    create the line break within the |\item|. In order to ensure
%    that the text will begin at the proper position in the following
%    line, we simply pretend that the label does not take any
%    room.\footnote{This will lead to problems whenever very high
%    symbols occurring in the line tower into the
%    heading. So, something else has to be done here sometime.}
%    \begin{macrocode}
       \rlap{\vbox{\hbox{\hskip \labelsep\theorem@headerfont ##1\ ##2}%
                   \hbox{\strut}}}]}%
%    \end{macrocode}
%    Again, the definition of |\@opargbegintheorem| is
%    completely analogous.
%    \begin{macrocode}
\def\@opargbegintheorem##1##2##3{%
  \item[\rlap{\vbox{\hbox{\hskip \labelsep \theorem@headerfont
                     ##1\ ##2\ (##3)}%
                    \hbox{\strut}}}]}}
\endgroup
%</thb>
%    \end{macrocode}
% \end{macro}
%
%
% \subsubsection{The \texttt{changebreak} style}
%
%    \begin{macrocode}
% This style option is stored in the file |thcb.sty|.
%    \begin{macrocode}
%<*thcb>
\begingroup \makeatletter
\@ifundefined{theorem@style}{% \iffalse meta-comment
%
% Copyright (C) 1993-2024
%
% The LaTeX Project and any individual authors listed elsewhere
% in this file.
%
% This file is part of the Standard LaTeX `Tools Bundle'.
% -------------------------------------------------------
%
% It may be distributed and/or modified under the
% conditions of the LaTeX Project Public License, either version 1.3c
% of this license or (at your option) any later version.
% The latest version of this license is in
%    https://www.latex-project.org/lppl.txt
% and version 1.3c or later is part of all distributions of LaTeX
% version 2005/12/01 or later.
%
% The list of all files belonging to the LaTeX `Tools Bundle' is
% given in the file `manifest.txt'.
%
% \fi
%
%%
%
%
\def\FMithmInfo{2023/07/05 v2.2c Theorem extension package (FMi)}
%
% \ProvidesFile{theorem.dtx}[\FMithmInfo]
%\iffalse   % this is a METACOMMENT !
%%
%% Package `theorem' to use with LaTeX2e
%% Copyright (C) 1989-1995 Frank Mittelbach, all rights reserved.
%<+package>\@ifundefined{theorem@style}{}{\endinput}
%<+package>\NeedsTeXFormat{LaTeX2e}[1995/06/01]
%<+package>\ProvidesPackage{theorem}
%<+package>         [\FMithmInfo]
%
% \fi
%
% \changes{v1.0}{87/05/07}{First draft.}
% \changes{v1.0a}{87/05/07}{`thistheoremstyle' added.}
% \changes{v1.1a}{87/07/29}{New concept: redefine the internal
%                          \protect\LaTeX{} commands `@xnthm' etc.}
% \changes{v1.1b}{87/07/31}{`theoremfont' used to specify font.}
% \changes{v1.2a}{88/01/08}{`@endtheorem' now globally defined.}
% \changes{v1.2b}{88/01/11}{`newtheorem' only allowed in the preamble.}
% \changes{v1.3a}{88/02/14}{`theoremfont' now a toks register.}
%
% \changes{v2.0a}{88/11/16}{New implementation and new documentation.}
% \changes{v2.0a}{88/11/16}{No layout changes but internal modifications
%                           old substyles won't work.}
% \changes{v2.0b}{89/01/30}{`global' used for all definitions.}
% \changes{v2.0c}{89/06/15}{Documentation translated to English.
%                            Thanks to LLL and Nina}
% \changes{v2.1a}{90/02/15}{Support for new font selection.}
% \changes{v2.2a}{94/02/03}{Cleanup for LaTeX2e}
%
% \DoNotIndex{\@endparenv,\@ifdefinable,\@ifnextchar,\@ifundefined}
% \DoNotIndex{\@namedef,\@nameuse,\@spaces,\@tempa,\@warning}
% \DoNotIndex{\ }
% \DoNotIndex{\begingroup,\bfseries}
% \DoNotIndex{\csname}
% \DoNotIndex{\def,\do,\docdate}
% \DoNotIndex{\endcsname,\endgroup,\endinput,\endtrivlist,\expandafter}
% \DoNotIndex{\filedate,\fileversion}
% \DoNotIndex{\gdef,\global}
% \DoNotIndex{\hbox,\hskip}
% \DoNotIndex{\ignorespaces,\input,\itshape,\item}
% \DoNotIndex{\labelsep,\let,\llap}
% \DoNotIndex{\makeatletter}
% \DoNotIndex{\noexpand}
% \DoNotIndex{\refstepcounter,\relax,\rlap}
% \DoNotIndex{\setlength,\slshape,\space,\string,\strut}
% \DoNotIndex{\the,\trivlist,\typeout}
% \DoNotIndex{\vbox}
% \DoNotIndex{\xdef}
%
% \def\deflabel#1{\ttfamily #1\hfill}
% \def\deflist#1{\begin{list}{}{\settowidth\labelwidth{\ttfamily #1}%
%                               \setlength\leftmargin\labelwidth
%                               \addtolength\leftmargin\labelsep
%                               \let\makelabel\deflabel}}
% \def\enddeflist{\end{list}}
% \def\envfont{\normalfont\ttfamily}
%
% \MakeShortVerb{\|}
%
% \newcommand{\TUB}{{\sl TUGboat\/}}
%
% \GetFileInfo{theorem.dtx}
% \title{An Extension of the
%        \LaTeX{} theorem environment\thanks{This file
%        has version number \fileversion, last
%        revised \filedate.}}
% \author{Frank Mittelbach}
%
% \MaintainedByLaTeXTeam{tools}
% \maketitle
%
% \begin{abstract}
%    The macros described in this paper yield an extension of the
%    \LaTeX{} theorem mechanism. It is designed to satisfy the
%    different requirements of various journals. Thus, the
%    layout of the ``theorems'' can be manipulated by determining a
%    ``style''. This article describes not only the use, but
%    also the definition, of the necessary macros.
% \end{abstract}
%
% \section*{Preface to version 2.2}
%
% For \LaTeXe{} this package did not need any fundamental changes. I
% only modified the messages generated so that theorem layout styles
% will show up with the |\listfiles| command and cleaned the section
% on the New Font Selection Scheme since this is now included in
% \LaTeX.
%
% \section*{Preface to version 2.1}
%
% This version is identical to 2.0g described in \TUB\ 10\#3 except
% for some internal defaults which are now set depending on the used
% font selection scheme.
%
% This was done to avoid unpleasant surprises if the new font selection
% scheme is in force. For further details see
% section~\ref{sec:newfonts} and \cite{bk:GMS94}.
%
% \section{Introduction}
%
% For our purposes here, ``theorems'' are labelled enunciations,
% often set off from the main text by extra space and a font change.
% Theorems, corollaries, conjectures, definitions, and remarks
% are all instances of ``theorems''.  The ``header'' of these structures
% is composed of a label (such as \textsc{Theorem} or \textsc{Remark})
% and a number which serializes an item in the sequence of items
% with the same label.
%
% Shortly after the introduction of \LaTeX{} at the Fachbereich
% Mathematik in Mainz, the desire to manipulate the layout of
% ``theorems'' arose.  In Mainz, the following two conventions came
% into general use:
% \begin{enumerate}
%   \item The number of the theorem is shown in the margin.
%   \item There is a line break at
%         the end of the theorem header.
% \end{enumerate}
%
% Additionally, some journals require different formats which depend on
% the ``sort of theorem'': e.g.\ often remarks and definitions are set
% in |\upshape|, while |\itshape| is employed for main theorems.
%
% Confronted with these requirements, a theorem environment was
% developed in Mainz which allows separate determination of the
% layout of the ``theorems sets'', comparable to |\pagestyle|.
%
%
% \section{The user interface}
% \subsection{Defining new theorem sets}
%
% \DescribeMacro\newtheorem
% As in the original \LaTeX{} version, the command |\newtheorem|
% defines a new ``theorem set'' or ``theorem-like structure''.
% Two required arguments name the new environment and give the text
% to be typeset with each instance of the new ``set'', while
% an optional argument determines how
% the ``set'' is enumerated:
% \begin{description}
%    \item[\ttfamily \bslash newtheorem\{foo\}\{bar\}]
%       The theorem set {\envfont foo} (whose name is \texttt{bar})
%       uses its own counter.
%    \item[\ttfamily \bslash newtheorem\{foo2\}{[foo]}\{bar2\}]
%       The theorem set {\envfont foo2} (printed name \texttt{bar2})
%       uses the same counter as the theorem set \texttt{foo}.
%    \item[\ttfamily \bslash newtheorem\{foo3\}\{bar3\}{[section]}]
%       The theorem set {\envfont foo3} (printed name \texttt{bar3}) is
%       enumerated within the counter \texttt{section}, i.e.\ with every
%       new |\section| the enumeration begins again with 1, and
%       the enumeration is composed from the section-number and the
%       theorem counter itself.
% \end{description}
%
% \DescribeMacro\theoremstyle
% Additionally, the command |\theoremstyle| can define the layout
% of various, or all, theorem sets.  It should be noted that any
% theorem set defined by |\newtheorem| is typeset in the
% |\theoremstyle| that is current at the time of the definition.
% Thus, the following
% \begin{verbatim}
%    \theoremstyle{break}        \newtheorem{Cor}{Corollary}
%    \theoremstyle{plain}        \newtheorem{Exa}{Example}[section]
%\end{verbatim}                             ^^A No blank at line start
% leads to the result that the set {\envfont Cor} is formatted in the
% style \texttt{break}, while the set {\envfont Exa} and all the
% following ones are formatted in the style \texttt{plain}, unless
% another |\theoremstyle| follows.  Since the definitions installed by
% |\newtheorem| are global, one also can limit |\theoremstyle| locally
% by grouping braces.
%
% \DescribeMacro\theorembodyfont
% The choice of the font for the theorem body is completely
% independent of the chosen |\theoremstyle|; this has proven to
% be very advantageous.  For example,
% \begin{verbatim}
%   {\theorembodyfont{\upshape}          \newtheorem{Rem}{Remark}}
%\end{verbatim}
% defines a theorem set {\envfont Rem}, which will be set in
% |\upshape| in the current layout (which in our example is
% \texttt{plain}).  As with |\theoremstyle|, the |\theorembodyfont|
% chosen is that current at the time of |\newtheorem|.  If
% |\theorembodyfont| is not specified or one defines
% |\theorembodyfont{}|, then the font used will be that defined by the
% |\theoremstyle|.
%
% \DescribeMacro\theoremheaderfont It is also possible to customize
% the font used for the theorem headers.  This is, however, a global
% declaration, and therefore there should be at most one
% |\theoremheaderfont| declaration in the preamble.\footnote{If it is
% actually necessary to have different header fonts, one has to define
% new theorem styles (substituting the desired font) or specify the
% information directly in the \texttt{\bslash newtheorem} declaration
% (the unclean variant).}
%
% \DescribeMacro\theorempreskipamount
% \DescribeMacro\theorempostskipamount Two additional parameters
% affect the vertical space around the theorem environments:
% |\theorempreskipamount| and |\theorempostskipamount| define,
% respectively, the spacing before and after such an environment.
% These parameters apply for all theorem sets and can be manipulated
% with the ordinary length macros.  They are rubber lengths
% (`\textsf{skips}'), and therefore can contain \texttt{plus} and
% \texttt{minus} parts.
%
% Since the definition of theorem sets should---most sensibly---be
% placed in the preamble, we only allow installation there.
% It is therefore possible to release the memory used here after
% |\begin{document}|, in order to make room for other
% applications.
%
%
% \subsection{Existing theorem styles}
%
% The following theorem styles exist to date:
% \begin{deflist}{marginbreak}
%    \item[plain]
%       This theorem style emulates the original \LaTeX{} definition,
%       except that additionally the parameters
%       |\theorem...skipamount| are used.
%    \item[break]
%       In this style, the theorem header is followed by a line
%       break.
%    \item[marginbreak]
%       The theorem number is set in the margin, and there is a line
%       break as in \texttt{break}.
%    \item[changebreak]
%       Like \texttt{break}, but with header number and text
%       interchanged.
%    \item[change]
%       Header number and text are interchanged, without a line break.
%    \item[margin]
%       The number is set in the left margin, without a line break.
% \end{deflist}
% All styles (except \texttt{plain}) select |\slshape| as the default
% |\theorembodyfont|.
%
%
% \subsection{Examples}
%
% Given the above theorem sets {\envfont Cor}, {\envfont Exa}
% and {\envfont Rem},
% suppose that the preamble also contains the declarations:
% \begin{verbatim}
%    \theoremstyle{marginbreak}   \newtheorem{Lem}[Cor]{Lemma}
%    \theoremstyle{change}
%    \theorembodyfont{\itshape}        \newtheorem{Def}[Cor]{Definition}
%
%    \theoremheaderfont{\scshape}
%\end{verbatim}
% Then the following are some typical examples of the typeset output
% resulting from their use.
%
% \begin{Cor}
%    This is a sentence typeset in the theorem environment {\envfont
%    Cor}.
% \end{Cor}
%
% \begin{Exa}
%    This is a sentence typeset in the theorem environment {\envfont
%    Exa}.
% \end{Exa}
%
% \begin{Rem}
%    This is a sentence typeset in the theorem environment {\envfont
%    Rem}.
% \end{Rem}
%
% \begin{Lem}[Ben User]
%    This is a sentence typeset in the theorem environment {\envfont
%    Lem}.
% \end{Lem}
%
% \begin{Def}[Very impressive Definition]
%    This is a sentence typeset in the theorem environment {\envfont
%    Def}.
% \end{Def}
%
% The last two examples show the effect of the optional argument to
% a theorem environment (it is the text typeset in parentheses).
%
%
% \section{Special Considerations}
% \label{sec:newfonts}
%
% Theoremheader and body are implemented as a unit. This means that
% the |\theoremheaderfont|  will inherit characteristics of the
% |\theorembodyfont| in \LaTeXe. Thus, if for example
% |\theorembodyfont| is |\itshape| and |\theoremheaderfont|
% is |\bfseries| the font selected for the header will have the
% characteristics `bold extended italic'. If this is not desired one
% should set the |\theoremheaderfont| to something like
% \begin{verbatim}
%      \theoremheaderfont{\normalfont\bfseries}
%\end{verbatim}
% i.e.\ supplying all necessary font information explicitly.
%
%
%
% \section{Acknowledgements}
%
% The publication of this set of macros was only possible with the
% help of Christina Busse (translating the manuscript into English),
% Joachim Pense (playing the r\^ole of typist), Chris Rowley
% (looking everything over) and many others providing useful
% suggestions.
%
% \MaybeStop{%
%    \begin{thebibliography}{9}
%       \bibitem{bk:GMS94} \textsc{M.~Goossens}, \textsc{F.~Mittelbach}
%         and \textsc{A.~Samarin}.
%         \newblock The \LaTeX{} Companion.
%         \newblock
%         Addison-Wesley, Reading, Massachusetts, 1994.
%       \bibitem{latex:source} \textsc{Lamport, Leslie.} \newblock
%                               \texttt{latex.tex}, version 2.09, date
%                               Feb. 1990.
%    \end{thebibliography}
%    \PrintIndex
%    \PrintChanges}
%
%
% \section{The documentation driver file}
%
% The next bit of code contains the documentation driver file for
% \TeX{}, i.e., the file that will produce the documentation you are
% currently reading. It will be extracted from this file by the
% \texttt{docstrip} program. Since it is the first code in the file
% one can alternatively process this file directly with \LaTeXe{} to
% obtain the documentation.
%
%    \begin{macrocode}
%<*driver>
\documentclass{ltxdoc}

\usepackage{theorem}

%<+driver>% The next few lines define theorem sets which are used
%<+driver>% in the example section of the documentation.

\theoremstyle{break}       \newtheorem{Cor}{Corollary}
\theoremstyle{plain}       \newtheorem{Exa}{Example}[section]
{\theorembodyfont{\upshape}\newtheorem{Rem}{Remark}}
\theoremstyle{marginbreak} \newtheorem{Lem}[Cor]{Lemma}
\theoremstyle{change}
\theorembodyfont{\itshape} \newtheorem{Def}[Cor]{Definition}

\theoremheaderfont{\scshape}

\RecordChanges

\begin{document}
   \DocInput{theorem.dtx}
\end{document}
%</driver>
%    \end{macrocode}
%
% \section{Definition of the Macros}
%
% If the file has been loaded before, we abort immediately. If not the
% package announces itself (this is actually done at the very top of
% the file---the way it is done isn't good style so don't copy it).
% \changes{v2.0e}{89/07/19}{Spaces removed from `typeout'.}
% \changes{v2.2b}{95/11/19}{Announce moved to top of file}
%    \begin{macrocode}
%<*package>
%\@ifundefined{theorem@style}{}{\endinput}
%\def\FMithmInfo{1995/11/19 v2.2b Theorem extension package (FMi)}
%\ProvidesPackage{theorem}[\FMithmInfo]
%    \end{macrocode}
%
%
% \subsection{Definition of theorem styles and fonts}
%
% All the definitions in this file are done globally to allow
% inputting this file inside a group.
%
% \begin{macro}{\theoremstyle}
%    Before a theorem style can be installed, the chosen style must be
%    known. For that reason, we must test to see that
%    |\th@|\meta{style} is known or, more precisely, that it is
%    different from |\relax|. If the style is not known then
%    |\th@plain| is used.
%    \begin{macrocode}
\gdef\theoremstyle#1{%
   \@ifundefined{th@#1}{\@warning
          {Unknown theoremstyle `#1'. Using `plain'}%
          \theorem@style{plain}}%
%    \end{macrocode}
%    We save the theorem style to be used in the token register
%    |\theorem@style|.
%    \begin{macrocode}
      {\theorem@style{#1}}%
%    \end{macrocode}
%    Now we ``evaluate'' the theorem style: this means, we call the
%    macro |\th@|\meta{style} which will activate the relevant
%    definitions which are contained in a separate file.  This is done
%    in a group to suppress changes to the current font.  This could
%    otherwise pose problems together with the new font selection
%    scheme\footnote{When I printed the original article using the new
%    font selection scheme I ended with a document with slanted
%    typefaces (text headings and all) simply because one of the
%    theorem styles used \texttt{\bslash sl} at toplevel.} if the
%    |\th@|\meta{style} is evaluated a second time.
% \changes{v2.1a}{90/02/15}{Group added around `th@..}
% \changes{v2.1a}{90/02/15}{Reseting of `..@begintheorem removed.}
%    \begin{macrocode}
      \begingroup
        \csname th@\the\theorem@style \endcsname
      \endgroup}
%    \end{macrocode}
% \end{macro}
%
%
% \begin{macro}{\@begintheorem}
% \changes{v2.1a}{90/02/15}{Relaxed at toplevel.}
% \begin{macro}{\@opargbegintheorem}
% \changes{v2.1a}{90/02/15}{Relaxed at toplevel.}
%    We reset |\@begintheorem| and |\@opargbegintheorem| to
%    |\relax| since these commands are no longer necessary at
%    toplevel. This will save a few tokens.
%    \begin{macrocode}
\global\let\@begintheorem\relax
\global\let\@opargbegintheorem\relax
%    \end{macrocode}
% \end{macro}
% \end{macro}
%
% \begin{macro}{\theorem@style}
%    Obviously the \textsf{token} register used above has to be
%    allocated. To assure the utmost compatibility with the
%    original \LaTeX{} definition, we set the default theorem style to
%    \texttt{plain}, which implements the usual \LaTeX{} convention.
% \changes{v2.0f}{89/07/20}{Unnecessary `global' removed.}
%    \begin{macrocode}
\newtoks\theorem@style
\global\theorem@style{plain}
%    \end{macrocode}
% \end{macro}
%
% \begin{macro}{\theorembodyfont}
% \begin{macro}{\theorem@bodyfont}
%    If the |\theorembodyfont| is set by the user then it should not
%    interact with the default font set in the theorem style. When the
%    new font selection is in force this may happen if, for example,
%    the default is |\itshape| and the new |\theorembodyfont| is
%    |\sffamily|. So we add a |\reset@font| command in front of the
%    user definition.
% \changes{v2.0f}{89/07/20}{Unnecessary `global' removed.}
% \changes{v2.0g}{89/09/19}{`theoremfont to `theorembodyfont.}
% \changes{v2.1c}{92/06/22}{`reset@font added and logic changed.}
% \changes{v2.1d}{92/10/28}{Reset to empty if arg is empty}
%    \begin{macrocode}
\gdef\theorembodyfont#1{%
%    \end{macrocode}
%    We check if the argument supplied is empty and if so put nothing
%    into the |\theorem@bodyfont| token register to allow for
%    |\theorembodyfont{}| as a mean of using the default of the
%    current |\theoremstyle|.
%    \begin{macrocode}
   \def\@tempa{#1}%
   \ifx\@tempa\@empty
    \theorem@bodyfont{}%
   \else
    \theorem@bodyfont{\reset@font#1}%
   \fi
   }
\newtoks\theorem@bodyfont
\global\theorem@bodyfont{}
%    \end{macrocode}
% \end{macro}
% \end{macro}
%
%
% \begin{macro}{\theoremheaderfont}
% \changes{v2.0g}{89/09/19}{Macro added.}
%    The font for the theorem headers is handled differently because
%    this definition applies to all theorem styles.
%    \begin{macrocode}
\gdef\theoremheaderfont#1{\gdef\theorem@headerfont{#1}%
%    \end{macrocode}
%    After using the macro once it is redefined to produce an error
%    message.
%    \begin{macrocode}
       \gdef\theoremheaderfont##1{%
        \typeout{\string\theoremheaderfont\space should be used
                 only once.}}}
%    \end{macrocode}
% \end{macro}
%
% \begin{macro}{\theorem@headerfont}
% \changes{v2.0g}{89/09/19}{Macro added.}
%    To set the |\theorem@headerfont| default we first test if
%    the new fontselection scheme is in force.
% \changes{v2.1a}{90/02/15}{Support for new font selection.}
%    \begin{macrocode}
\ifx\upshape\undefined
%    \end{macrocode}
%    If not we define it to expand into |\bfseries|. We don't use
%    |\let| just in case a following style option redefines this
%    macro.
%    \begin{macrocode}
\gdef\theorem@headerfont{\bfseries}
%    \end{macrocode}
%    Otherwise we reset the current shape before calling |\bfseries|.
%    \begin{macrocode}
\else \gdef\theorem@headerfont{\normalfont\bfseries}\fi
%    \end{macrocode}
% \end{macro}
%
% \begin{macro}{\th@plain}
% \begin{macro}{\th@break}
% \begin{macro}{\th@marginbreak}
% \begin{macro}{\th@changebreak}
% \begin{macro}{\th@change}
% \begin{macro}{\th@margin}
%    The different styles are defined in macros such as
%    |\th@plain|.  Since memory space is precious in
%    ``non-Big-versions'', we have to avoid offering too many unused
%    definitions.  Therefore we define these styles in separate files
%    that can be loaded on demand.  Thus the commands themselves only
%    load these files. We use |\@input@|, a \LaTeXe{} internal command
%    that ensures that the file will be listed with |\listfiles|.
% \changes{v2.2a}{94/02/02}{Use `@input@ to load theorem layout files}
%    \begin{macrocode}
\gdef\th@plain{\@input@{thp.sty}}
\gdef\th@break{\@input@{thb.sty}}
\gdef\th@marginbreak{\@input@{thmb.sty}}
\gdef\th@changebreak{\@input@{thcb.sty}}
\gdef\th@change{\@input@{thc.sty}}
\gdef\th@margin{\@input@{thm.sty}}
%    \end{macrocode}
%    This list will be expanded when new styles become
%    available.  For testing,  just append new theorem substyles
%    as document options.
% \end{macro}
% \end{macro}
% \end{macro}
% \end{macro}
% \end{macro}
% \end{macro}
%
%
%
% \subsection{Definition of a new theorem set}
%
% As already pointed out, a new theorem environment can be defined in
% three different ways:
% \begin{verbatim}
%    \newtheorem{Lem}{Lemma}
%    \newtheorem{Lem}{Lemma}[section]
%    \newtheorem{Lem}[Theorem]{Lemma}
%\end{verbatim}
% The function of the macro |\newtheorem| is to recognize these cases
% and then to branch into one of the three macros |\@ynthm|, |\@xnthm|
% or |\@othm|.  This mechanism is adopted unchanged from
% \cite{latex:source}; the essential point here is that, for example,
% in the second case, the arguments \texttt{Lem}, \texttt{Lemma} and
% \texttt{section} are passed over to the macro |\@xnthm|.
%
% We inspect this case first because the others present fewer problems,
% and thus are easily derived from this one.
%
% \begin{macro}{\@xnthm}
%    For our example arguments, the macro |\@xnthm| must fulfill
%    the following:
%    \begin{itemize}
%       \item
%          Define a new \LaTeX-counter `Lem'
%       \item
%          reset this counter within a |\section|
%       \item
%          define the macro |\theLem|
%       \item
%          define the environment macros |\Lem| and
%          |\endLem| using the current
%          |\theoremstyle| and |\theorem@bodyfont|.
%    \end{itemize}
%    Obviously, all this should happen only if the first argument
%    of |\@xnthm| (i.e.\ \texttt{Lem} in our example) is chosen so
%    as not to conflict with any previously defined commands or
%    environments.  This test is performed by the \LaTeX{} macro
%    |\@ifdefinable|.
%    \begin{macrocode}
\gdef\@xnthm#1#2[#3]{\expandafter\@ifdefinable\csname #1\endcsname
%    \end{macrocode}
%    Therefore, the first argument of |\@ifdefinable| is the
%    expansion (in the example, |\Lem|) of
%    |\csname||#1\endcsname|.  The second argument is
%    executed only if the test has been completed successfully.
%    \begin{macrocode}
   {%
%    \end{macrocode}
%    Now we define the new counter. The names of the \LaTeX{}
%    macros employed should speak for themselves:
% \changes{v2.2b}{1995/10/07}{Use \cs{@newctr}}
% \changes{v2.2c}{1995/11/23}{Use \cs{@newctr} properly.}
%    \begin{macrocode}
    \@definecounter{#1}\@newctr{#1}[#3]%
%    \end{macrocode}
%    Using |\@newctr| will give a proper error message if the counter
%    in |#3| is not defined.
%    In defining `|\theLem|' we must generate the desired macro
%    name by use of |\expandafter| and |\csname|.
%    \begin{macrocode}
    \expandafter\xdef\csname the#1\endcsname
%    \end{macrocode}
%    An |\xdef| is used in order to make the definition global,
%    and to ensure that it contains the replacement texts of
%    |\@thmcountersep| and |\@thmcounter|.\footnote{These
%    two macros can be defined by the document style.  Their default
%    values produce a `\texttt{.}' as separation and an arabic
%    representation of the number.} However, not everything should be
%    expanded. For example, it saves space to use |\thesection|
%    instead of its---at times---lengthy expansion.
%    \begin{macrocode}
      {\expandafter \noexpand \csname the#3\endcsname
       \@thmcountersep \@thmcounter{#1}}%
%    \end{macrocode}
%    Thus with the defaults of \LaTeX{}, |\theLem| would be
%    replaced by the command sequence |\thesection.\arabic{Lem}|.
%
%    We will now look at the definition of the macro which is executed
%    at the beginning of the actual environment (in our example this
%    macro is |\Lem|). It should be noted that we use an
%    ``|\expandafter| trick'' to expand only certain parts of the
%    replacement text at the time of the definition.
%    \begin{macrocode}
    \def\@tempa{\global\@namedef{#1}}%
    \expandafter \@tempa \expandafter{%
%    \end{macrocode}
%    First, the macro that contains the current definitions of
%    |\@begintheorem| and |\@opargtheorem| should be called
%    up.  The name of this macro---as is already known---has the form
%    |\th@|\meta{theorem\ style}; therefore, it must be called by
%    \begin{macrocode}
      \csname th@\the \theorem@style
%    \end{macrocode}
%    In addition the default theorem font should be changeable, i.e.\
%    we have to insert the contents of |\theorem@bodyfont|. For
%    that reason, we expand even further, beyond |\endcsname|,
%    and thus insert the contents of the \textsf{token} register
%    |\theorem@bodyfont| in the replacement text.
%    \begin{macrocode}
            \expandafter \endcsname \the \theorem@bodyfont
%    \end{macrocode}
%    Now it is time to call the macro |\@thm| which takes over
%    the further processing.  It has two arguments: the current
%    counter name (in our example, \texttt{Lem}), and the text of the
%    label (in our example, \texttt{Lemma}).
%    \begin{macrocode}
     \@thm{#1}{#2}}%
%    \end{macrocode}
%    With this, the `sub-definition' is complete.  The macro
%    |\@endtheorem| ends a theorem environment and is, so far, nothing
%    but an |\endtrivlist|.  (Hence it is defined globally, and not
%    within the theorem styles.\footnote{This has to be changed as
%    soon as theorem styles that change \texttt{\bslash @endtheorem}
%    exist.  In such a case, all existing styles must be changed as
%    well since they will have to reset the macro.}) Therefore, we can
%    set it equivalent to the macro that ends the theorem set (in our
%    example, |\endLem|).  However, if some day theorem styles exist
%    that do change |\@endtheorem|, we would have to use the
%    commented-out line instead.
%    \begin{macrocode}
    \global \expandafter \let \csname end#1\endcsname \@endtheorem
%   \global\@namedef{end#1}{\@endtheorem}%
%    \end{macrocode}
%    With these commands all the required definitions are employed,
%    unless the test |\@ifdefinable| has failed.  Therefore, we
%    end the second argument of this macro and with it the definition
%    of |\@xnthm|.
%    \begin{macrocode}
   }}
%    \end{macrocode}
% \end{macro}
%
% \begin{macro}{\@ynthm}
%    The definition of |\@ynthm| is completely analogous. In
%    this case the new counter that is defined is not reset within
%    another counter; thus the definition of |\the...|
%    is simplified:
%    \begin{macrocode}
\gdef\@ynthm#1#2{\expandafter\@ifdefinable\csname #1\endcsname
   {\@definecounter{#1}%
    \expandafter\xdef\csname the#1\endcsname{\@thmcounter{#1}}%
%    \end{macrocode}
%    The rest of the definition corresponds literally to that of
%    |\@xnthm|:
%    \begin{macrocode}
    \def\@tempa{\global\@namedef{#1}}\expandafter \@tempa
     \expandafter{\csname th@\the \theorem@style \expandafter
     \endcsname \the\theorem@bodyfont \@thm{#1}{#2}}%
    \global \expandafter \let \csname end#1\endcsname \@endtheorem}}
%    \end{macrocode}
% \end{macro}
%
% \begin{macro}{\@othm}
%    The definition of |\@othm| does not contain anything new.
%    \begin{macrocode}
\gdef\@othm#1[#2]#3{%
%    \end{macrocode}
% \changes{v2.2b}{95/10/07}{Build in check for counter defined}
%    We do not define a new counter but instead use one that has
%    already been defined.  Thus the only definition we need is that
%    of this pseudo-counter (i.e.\ |\the|\meta{env.\ name}).
%    First we check if |#2| corresponds to a known counter name.
%    \begin{macrocode}
  \expandafter\ifx\csname c@#2\endcsname\relax
   \@nocounterr{#2}%
  \else
   \expandafter\@ifdefinable\csname #1\endcsname
   {\expandafter \xdef \csname the#1\endcsname
     {\expandafter \noexpand \csname the#2\endcsname}%
%    \end{macrocode}
%    All other parts of the definition can be adopted from
%    |\@xnthm|. We have to remember, though, that in this case
%    the name of the current counter and the theorem label have moved
%    to the second and third arguments.
%    \begin{macrocode}
    \def\@tempa{\global\@namedef{#1}}\expandafter \@tempa
     \expandafter{\csname th@\the \theorem@style \expandafter
     \endcsname \the\theorem@bodyfont \@thm{#2}{#3}}%
    \global \expandafter \let \csname end#1\endcsname \@endtheorem}%
  \fi}
%    \end{macrocode}
% \end{macro}
%
% \subsection{Macros that are employed in a {\envfont theorem}
%    environment}
%
% \begin{macro}{\@thm}
%    The macro |\@thm| has to increase the current counter.  Then,
%    depending on whether the environment has (or does not have) an
%    optional argument, it has to branch into either |\@begintheorem|
%    or |\@opargtheorem|.
%    \begin{macrocode}
\gdef\@thm#1#2{\refstepcounter{#1}%
%    \end{macrocode}
%    Now we start a {\envfont trivlist} environment, and give
%    |\@topsep| and |\@topsepadd| the values of the \textsf{skip}
%    registers |\theorempreskipamount| and |\theorempostskipamount|.
%    The value in |\@topsep| is the vertical space that is inserted by
%    the first (and only) |\item| in our |\trivlist| whilst
%    |\@topsepadd| is inserted by |\@endparenv| at the end of that
%    {\envfont trivlist} environment.  By using these registers, we
%    obtain the desired space around a {\envfont theorem} environment.
%    \begin{macrocode}
   \trivlist
   \@topsep \theorempreskipamount               % used by first \item
   \@topsepadd \theorempostskipamount           % used by \@endparenv
%    \end{macrocode}
%    Now we have to test whether an optional argument has been given.
% \changes{v2.0d}{89/07/18}{Forgotten percent added.}
%    \begin{macrocode}
   \@ifnextchar [%
%    \end{macrocode}
%    If there is an optional argument, we will call |\@ythm|, and
%    move the arguments read back into the input stream.
%    \begin{macrocode}
   {\@ythm{#1}{#2}}%
%    \end{macrocode}
%    If not, we call |\@begintheorem|. Its first argument is the
%    name of the theorem set (hence the second argument of
%    |\@thm|). Its second argument is the macro that produces the
%    current number.
%    \begin{macrocode}
   {\@begintheorem{#2}{\csname the#1\endcsname}\ignorespaces}}
%    \end{macrocode}
% \end{macro}
%
% \begin{macro}{\@xthm}
% \begin{macro}{\@ythm}
%    Both these macros were originally called by |\@thm|.  We do
%    not need |\@xthm| anymore, hence we reset it to
%    |\relax|. The definition of |\@ythm| has not changed at
%    all from its definition in \LaTeX. In order to make the macros
%    easier to understand, we will
%    nevertheless present it (commented out).
%    \begin{macrocode}
\global\let\@xthm\relax
% \def\@ythm#1#2[#3]{\@opargbegintheorem{#2}{\csname
%           the#1\endcsname}{#3}\ignorespaces}
%</package>
%    \end{macrocode}
%    The primitive |\ignorespaces| in |\@ythm| and
%    |\@thm| is needed to remove the spaces between the
%    |\begin{...}| and the actual text.
% \end{macro}
% \end{macro}
%
%
% \subsection{Definition of the theorem substyles}
%
% As already pointed out, the theorem substyles, defined below, are
% only loaded when necessary.  Note that all these substyles, except
% \texttt{plain}, have |\slshape| as the default body font.
%
% \subsubsection{The \texttt{plain} style}
%
% \label{dok:th@plain} As the following macros use |@|, we have
% to locally set the |\catcode| of this symbol to ``letter''.
% This happens within a group, so that we do not have to worry about
% which |\catcode| that symbol had before.
%    \begin{macrocode}
%<*thp>
\begingroup \makeatletter
%    \end{macrocode}
% Since we are now within a group, we must make all definitions
% globally.  First we make sure that \texttt{theorem.sty} is loaded.
% This will allow us to use this file as a document style option
% without having to call \texttt{theorem} itself as an option.  At the
% same time, we assure that at least version 2 is loaded, since
% |\theorem@style| was not defined in earlier versions.
%    \begin{macrocode}
\@ifundefined{theorem@style}{\input{theorem.sty}}{}
\ProvidesFile{thp.sty}
             [\FMithmInfo]
%    \end{macrocode}
%
%
% \begin{macro}{\th@plain}
% \begin{macro}{\@begintheorem}
% \begin{macro}{\@opargbegintheorem}
%    |\theoremstyle{plain}| corresponds to the original
%    definition, except that the distances to the surrounding text are
%    determined by the parameters |\theorempreskipamount| and
%    |\theorempostskipamount|.
%    First we set the default body font.
%    \begin{macrocode}
\gdef\th@plain{\normalfont\itshape
%    \end{macrocode}
%    Then we define |\@begintheorem| and
%    |\@opargbegintheorem|. These two macros define how the
%    header of a theorem is typeset.  |\@opargbegintheorem| will
%    be called if a \textsf{theorem} environment with an optional
%    argument is encountered; otherwise, the header is constructed by
%    calling |\@begintheorem|.  If one of these macros is
%    executed, we are within a \textsf{trivlist} environment started by
%    |\@thm|.  So the theorem header is produced with an
%    |\item| command.
%
%    Instead of specifying the header font directly, all standard
%    theorem styles use the |\theorem@headerfont| macro to allow
%    customization.  The extra space (|\labelsep|) is necessary
%    because of problems in the \textsf{trivlist} environment.
%    \begin{macrocode}
  \def\@begintheorem##1##2{%
        \item[\hskip\labelsep \theorem@headerfont ##1\ ##2]}%
%    \end{macrocode}
%    The definition of |\@opargbegintheorem| is completely
%    analogous. The only difference is the fact that there exists a
%    third argument (which is the optional parameter of the
%    environment and contains additional information about the
%    theorem).  Customarily we enclose it in parentheses.
%    \begin{macrocode}
\def\@opargbegintheorem##1##2##3{%
   \item[\hskip\labelsep \theorem@headerfont ##1\ ##2\ (##3)]}}
%    \end{macrocode}
% \end{macro}
% \end{macro}
% \end{macro}
%
% \noindent
% We conclude with an |\endgroup| to restore the
% |\catcode| of |@|.
%    \begin{macrocode}
\endgroup
%</thp>
%    \end{macrocode}
%
%
% \subsubsection{The \texttt{break} style}
%
% This style option is stored in the file |thb.sty|.
% For the next two lines see the documentation for |\th@plain| on
% page~\pageref{dok:th@plain}.
%    \begin{macrocode}
%<*thb>
\begingroup \makeatletter
\@ifundefined{theorem@style}{\input{theorem.sty}}{}
\ProvidesFile{thb.sty}
             [\FMithmInfo]
%    \end{macrocode}
%
% \begin{macro}{\th@break}
%    |\theoremstyle{break}| produces a line break after the name
%    of the theorem.  The font is |\slshape|.  Hence, we define
%    |\th@break| as follows:
%    \begin{macrocode}
\gdef\th@break{\normalfont\slshape
  \def\@begintheorem##1##2{\item[%
%    \end{macrocode}
%    We run into the following problem: it is not possible to create
%    the header with |\item[|\meta{title}|]| and then start
%    a new line by, for example, |\mbox{}\\|. Such a definition
%    will fail whenever a list environment follows immediately.  With
%    the above construction, the |\mbox{}| causes the switch
%    |@inlabel| (cf.\ definition of |\list| and
%    |\trivlist| in \cite{latex:source}) to be set to \texttt{false}
%    and so the following list will insert additional vertical
%    space (|\topskip|). This is quite annoying. Therefore, we
%    create the line break within the |\item|. In order to ensure
%    that the text will begin at the proper position in the following
%    line, we simply pretend that the label does not take any
%    room.\footnote{This will lead to problems whenever very high
%    symbols occurring in the line tower into the
%    heading. So, something else has to be done here sometime.}
%    \begin{macrocode}
       \rlap{\vbox{\hbox{\hskip \labelsep\theorem@headerfont ##1\ ##2}%
                   \hbox{\strut}}}]}%
%    \end{macrocode}
%    Again, the definition of |\@opargbegintheorem| is
%    completely analogous.
%    \begin{macrocode}
\def\@opargbegintheorem##1##2##3{%
  \item[\rlap{\vbox{\hbox{\hskip \labelsep \theorem@headerfont
                     ##1\ ##2\ (##3)}%
                    \hbox{\strut}}}]}}
\endgroup
%</thb>
%    \end{macrocode}
% \end{macro}
%
%
% \subsubsection{The \texttt{changebreak} style}
%
%    \begin{macrocode}
% This style option is stored in the file |thcb.sty|.
%    \begin{macrocode}
%<*thcb>
\begingroup \makeatletter
\@ifundefined{theorem@style}{\input{theorem.sty}}{}
\ProvidesFile{thcb.sty}
             [\FMithmInfo]
%    \end{macrocode}
%
% \begin{macro}{\th@changebreak}
%    The \texttt{change-break} theorem style is like \texttt{break}
%    but with interchange of theorem name and theorem number.  Thus we
%    define |\th@changebreak| as follows:
%    \begin{macrocode}
\gdef\th@changebreak{\normalfont\slshape
  \def\@begintheorem##1##2{\item
       [\rlap{\vbox{\hbox{\hskip\labelsep \theorem@headerfont ##2\ ##1}%
                    \hbox{\strut}}}]}%
\def\@opargbegintheorem##1##2##3{%
  \item[\rlap{\vbox{\hbox{\hskip\labelsep \theorem@headerfont
                     ##2\ ##1\ (##3)}%
                    \hbox{\strut}}}]}}
\endgroup
%</thcb>
%    \end{macrocode}
% \end{macro}
%
%
% \subsubsection{The \texttt{change} style}
%
% This style option is stored in the file |thc.sty|.
%    \begin{macrocode}
%<*thc>
\begingroup \makeatletter
\@ifundefined{theorem@style}{\input{theorem.sty}}{}
\ProvidesFile{thc.sty}
             [\FMithmInfo]
%    \end{macrocode}
%
% \begin{macro}{\th@change}
%    The \texttt{change} theorem style corresponds to the
%    \texttt{change break} style without a linebreak after the header.
%    To say it in another way, it's the same as the \texttt{plain}
%    style but with number and name interchanged and |\slshape| as the
%    default font.
%    \begin{macrocode}
\gdef\th@change{\normalfont\slshape
  \def\@begintheorem##1##2{\item
       [\hskip\labelsep \theorem@headerfont ##2\ ##1]}%
\def\@opargbegintheorem##1##2##3{%
  \item[\hskip\labelsep \theorem@headerfont ##2\ ##1\ (##3)]}}
\endgroup
%</thc>
%    \end{macrocode}
% \end{macro}
%
%
% \subsubsection{The \texttt{marginbreak} style}
%
% This style option is the one used most often at Mainz.  It is saved
% in the file |thmb.sty|.
%    \begin{macrocode}
%<*thmb>
\begingroup \makeatletter
\@ifundefined{theorem@style}{\input{theorem.sty}}{}
\ProvidesFile{thmb.sty}
             [\FMithmInfo]
%    \end{macrocode}
%
% \begin{macro}{\th@marginbreak}
%    The \texttt{margin break} style is nearly the same as the
%    \texttt{change break} style. The only difference is the placement
%    of the theorem number.  We use |\llap| to place it in the left
%    margin.
%
%    In this style |\labelsep| denotes the separation between the
%    number and the text.
%    \begin{macrocode}
\gdef\th@marginbreak{\normalfont\slshape
  \def\@begintheorem##1##2{\item
       [\rlap{\vbox{\theorem@headerfont
                    \hbox{\llap{##2}\hskip\labelsep ##1}%
                    \hbox{\strut}}}]}%
\def\@opargbegintheorem##1##2##3{%
  \item[\rlap{\vbox{\theorem@headerfont
                    \hbox{\llap{##2}\hskip\labelsep ##1\ (##3)}%
                    \hbox{\strut}}}]}}
\endgroup
%</thmb>
%    \end{macrocode}
% \changes{v2.1c}{90/02/17}{`endgroup outside macrocode env.}
% \end{macro}
%
%
%
% \subsubsection{The \texttt{margin} style}
%
% This style option is stored in the file |thm.sty|.
%    \begin{macrocode}
%<*thm>
\begingroup \makeatletter
\@ifundefined{theorem@style}{\input{theorem.sty}}{}
\ProvidesFile{thm.sty}
             [\FMithmInfo]
%    \end{macrocode}
%
% \begin{macro}{\th@margin}
%    Again this is only a variant of the theorem styles described
%    above without any new ideas.
%    \begin{macrocode}
\gdef\th@margin{\normalfont\slshape
  \def\@begintheorem##1##2{\item
       [\theorem@headerfont \llap{##2}\hskip\labelsep ##1]}%
\def\@opargbegintheorem##1##2##3{%
  \item[\theorem@headerfont \llap{##2}\hskip\labelsep ##1\ (##3)]}}
\endgroup
%</thm>
%    \end{macrocode}
% \end{macro}
%
%
%
%
% \subsection{Final Definitions}
%
% \begin{macro}{\theorempreskipamount}
% \begin{macro}{\theorempostskipamount}
%    The \textsf{skip} parameters that regulate the vertical empty space
%    before and after the {\envfont theorem} environment have
%    to be allocated as well.
% \changes{v2.0f}{89/07/20}{Unnecessary `global' removed.}
%    \begin{macrocode}
%<*package>
\newskip\theorempreskipamount
\newskip\theorempostskipamount
%    \end{macrocode}
%    Since we have used the same values for all theorem sets, we now can
%    assign them.
%    \begin{macrocode}
\global\setlength\theorempreskipamount{12pt plus 5pt minus 3pt}
\global\setlength\theorempostskipamount{8pt plus 3pt minus 1.5pt}
%    \end{macrocode}
% \end{macro}
% \end{macro}
%
%
% \begin{macro}{\@endtheorem}
%    The same holds for the macro |\@endtheorem|, which ends a
%    {\envfont theorem} environment. Since it is the same for all
%    theorem sets, it is removed from the macros |\th@|\meta{style}.
%    It simply ends the {\envfont trivlist} environment, which was
%    begun in |\@thm|.
%    \begin{macrocode}
\global\let\@endtheorem=\endtrivlist
%    \end{macrocode}
% \end{macro}
%
%
%
% \begin{macro}{\@preamblecmds}
%    All macros defined above are to be used only in the preamble.
%    Therefore, we insert them in |\@preamblecmds| which will disable
%    them at begin document. This is done by the internal \LaTeXe{}
%    command |\@onlypreamble|.
% \changes{v2.0g}{89/09/19}{`theoremheaderfont and `theorembodyfont
%    added.}
% \changes{v2.2a}{94/02/03}{Use \LaTeXe{} interface}
%    \begin{macrocode}
\@onlypreamble\@xnthm
\@onlypreamble\@ynthm
\@onlypreamble\@othm
\@onlypreamble\newtheorem
\@onlypreamble\theoremstyle
\@onlypreamble\theorembodyfont
\@onlypreamble\theoremheaderfont
%    \end{macrocode}
% \end{macro}
%
% Finally we declare the \texttt{plain} theorem style to be the default.
% \changes{v2.1b}{90/06/04}{Fixed missing `item by declaring default
%    style.}
%    \begin{macrocode}
\theoremstyle{plain}
%</package>
%    \end{macrocode}
%
% \Finale

\endinput
}{}
\ProvidesFile{thcb.sty}
             [\FMithmInfo]
%    \end{macrocode}
%
% \begin{macro}{\th@changebreak}
%    The \texttt{change-break} theorem style is like \texttt{break}
%    but with interchange of theorem name and theorem number.  Thus we
%    define |\th@changebreak| as follows:
%    \begin{macrocode}
\gdef\th@changebreak{\normalfont\slshape
  \def\@begintheorem##1##2{\item
       [\rlap{\vbox{\hbox{\hskip\labelsep \theorem@headerfont ##2\ ##1}%
                    \hbox{\strut}}}]}%
\def\@opargbegintheorem##1##2##3{%
  \item[\rlap{\vbox{\hbox{\hskip\labelsep \theorem@headerfont
                     ##2\ ##1\ (##3)}%
                    \hbox{\strut}}}]}}
\endgroup
%</thcb>
%    \end{macrocode}
% \end{macro}
%
%
% \subsubsection{The \texttt{change} style}
%
% This style option is stored in the file |thc.sty|.
%    \begin{macrocode}
%<*thc>
\begingroup \makeatletter
\@ifundefined{theorem@style}{% \iffalse meta-comment
%
% Copyright (C) 1993-2024
%
% The LaTeX Project and any individual authors listed elsewhere
% in this file.
%
% This file is part of the Standard LaTeX `Tools Bundle'.
% -------------------------------------------------------
%
% It may be distributed and/or modified under the
% conditions of the LaTeX Project Public License, either version 1.3c
% of this license or (at your option) any later version.
% The latest version of this license is in
%    https://www.latex-project.org/lppl.txt
% and version 1.3c or later is part of all distributions of LaTeX
% version 2005/12/01 or later.
%
% The list of all files belonging to the LaTeX `Tools Bundle' is
% given in the file `manifest.txt'.
%
% \fi
%
%%
%
%
\def\FMithmInfo{2023/07/05 v2.2c Theorem extension package (FMi)}
%
% \ProvidesFile{theorem.dtx}[\FMithmInfo]
%\iffalse   % this is a METACOMMENT !
%%
%% Package `theorem' to use with LaTeX2e
%% Copyright (C) 1989-1995 Frank Mittelbach, all rights reserved.
%<+package>\@ifundefined{theorem@style}{}{\endinput}
%<+package>\NeedsTeXFormat{LaTeX2e}[1995/06/01]
%<+package>\ProvidesPackage{theorem}
%<+package>         [\FMithmInfo]
%
% \fi
%
% \changes{v1.0}{87/05/07}{First draft.}
% \changes{v1.0a}{87/05/07}{`thistheoremstyle' added.}
% \changes{v1.1a}{87/07/29}{New concept: redefine the internal
%                          \protect\LaTeX{} commands `@xnthm' etc.}
% \changes{v1.1b}{87/07/31}{`theoremfont' used to specify font.}
% \changes{v1.2a}{88/01/08}{`@endtheorem' now globally defined.}
% \changes{v1.2b}{88/01/11}{`newtheorem' only allowed in the preamble.}
% \changes{v1.3a}{88/02/14}{`theoremfont' now a toks register.}
%
% \changes{v2.0a}{88/11/16}{New implementation and new documentation.}
% \changes{v2.0a}{88/11/16}{No layout changes but internal modifications
%                           old substyles won't work.}
% \changes{v2.0b}{89/01/30}{`global' used for all definitions.}
% \changes{v2.0c}{89/06/15}{Documentation translated to English.
%                            Thanks to LLL and Nina}
% \changes{v2.1a}{90/02/15}{Support for new font selection.}
% \changes{v2.2a}{94/02/03}{Cleanup for LaTeX2e}
%
% \DoNotIndex{\@endparenv,\@ifdefinable,\@ifnextchar,\@ifundefined}
% \DoNotIndex{\@namedef,\@nameuse,\@spaces,\@tempa,\@warning}
% \DoNotIndex{\ }
% \DoNotIndex{\begingroup,\bfseries}
% \DoNotIndex{\csname}
% \DoNotIndex{\def,\do,\docdate}
% \DoNotIndex{\endcsname,\endgroup,\endinput,\endtrivlist,\expandafter}
% \DoNotIndex{\filedate,\fileversion}
% \DoNotIndex{\gdef,\global}
% \DoNotIndex{\hbox,\hskip}
% \DoNotIndex{\ignorespaces,\input,\itshape,\item}
% \DoNotIndex{\labelsep,\let,\llap}
% \DoNotIndex{\makeatletter}
% \DoNotIndex{\noexpand}
% \DoNotIndex{\refstepcounter,\relax,\rlap}
% \DoNotIndex{\setlength,\slshape,\space,\string,\strut}
% \DoNotIndex{\the,\trivlist,\typeout}
% \DoNotIndex{\vbox}
% \DoNotIndex{\xdef}
%
% \def\deflabel#1{\ttfamily #1\hfill}
% \def\deflist#1{\begin{list}{}{\settowidth\labelwidth{\ttfamily #1}%
%                               \setlength\leftmargin\labelwidth
%                               \addtolength\leftmargin\labelsep
%                               \let\makelabel\deflabel}}
% \def\enddeflist{\end{list}}
% \def\envfont{\normalfont\ttfamily}
%
% \MakeShortVerb{\|}
%
% \newcommand{\TUB}{{\sl TUGboat\/}}
%
% \GetFileInfo{theorem.dtx}
% \title{An Extension of the
%        \LaTeX{} theorem environment\thanks{This file
%        has version number \fileversion, last
%        revised \filedate.}}
% \author{Frank Mittelbach}
%
% \MaintainedByLaTeXTeam{tools}
% \maketitle
%
% \begin{abstract}
%    The macros described in this paper yield an extension of the
%    \LaTeX{} theorem mechanism. It is designed to satisfy the
%    different requirements of various journals. Thus, the
%    layout of the ``theorems'' can be manipulated by determining a
%    ``style''. This article describes not only the use, but
%    also the definition, of the necessary macros.
% \end{abstract}
%
% \section*{Preface to version 2.2}
%
% For \LaTeXe{} this package did not need any fundamental changes. I
% only modified the messages generated so that theorem layout styles
% will show up with the |\listfiles| command and cleaned the section
% on the New Font Selection Scheme since this is now included in
% \LaTeX.
%
% \section*{Preface to version 2.1}
%
% This version is identical to 2.0g described in \TUB\ 10\#3 except
% for some internal defaults which are now set depending on the used
% font selection scheme.
%
% This was done to avoid unpleasant surprises if the new font selection
% scheme is in force. For further details see
% section~\ref{sec:newfonts} and \cite{bk:GMS94}.
%
% \section{Introduction}
%
% For our purposes here, ``theorems'' are labelled enunciations,
% often set off from the main text by extra space and a font change.
% Theorems, corollaries, conjectures, definitions, and remarks
% are all instances of ``theorems''.  The ``header'' of these structures
% is composed of a label (such as \textsc{Theorem} or \textsc{Remark})
% and a number which serializes an item in the sequence of items
% with the same label.
%
% Shortly after the introduction of \LaTeX{} at the Fachbereich
% Mathematik in Mainz, the desire to manipulate the layout of
% ``theorems'' arose.  In Mainz, the following two conventions came
% into general use:
% \begin{enumerate}
%   \item The number of the theorem is shown in the margin.
%   \item There is a line break at
%         the end of the theorem header.
% \end{enumerate}
%
% Additionally, some journals require different formats which depend on
% the ``sort of theorem'': e.g.\ often remarks and definitions are set
% in |\upshape|, while |\itshape| is employed for main theorems.
%
% Confronted with these requirements, a theorem environment was
% developed in Mainz which allows separate determination of the
% layout of the ``theorems sets'', comparable to |\pagestyle|.
%
%
% \section{The user interface}
% \subsection{Defining new theorem sets}
%
% \DescribeMacro\newtheorem
% As in the original \LaTeX{} version, the command |\newtheorem|
% defines a new ``theorem set'' or ``theorem-like structure''.
% Two required arguments name the new environment and give the text
% to be typeset with each instance of the new ``set'', while
% an optional argument determines how
% the ``set'' is enumerated:
% \begin{description}
%    \item[\ttfamily \bslash newtheorem\{foo\}\{bar\}]
%       The theorem set {\envfont foo} (whose name is \texttt{bar})
%       uses its own counter.
%    \item[\ttfamily \bslash newtheorem\{foo2\}{[foo]}\{bar2\}]
%       The theorem set {\envfont foo2} (printed name \texttt{bar2})
%       uses the same counter as the theorem set \texttt{foo}.
%    \item[\ttfamily \bslash newtheorem\{foo3\}\{bar3\}{[section]}]
%       The theorem set {\envfont foo3} (printed name \texttt{bar3}) is
%       enumerated within the counter \texttt{section}, i.e.\ with every
%       new |\section| the enumeration begins again with 1, and
%       the enumeration is composed from the section-number and the
%       theorem counter itself.
% \end{description}
%
% \DescribeMacro\theoremstyle
% Additionally, the command |\theoremstyle| can define the layout
% of various, or all, theorem sets.  It should be noted that any
% theorem set defined by |\newtheorem| is typeset in the
% |\theoremstyle| that is current at the time of the definition.
% Thus, the following
% \begin{verbatim}
%    \theoremstyle{break}        \newtheorem{Cor}{Corollary}
%    \theoremstyle{plain}        \newtheorem{Exa}{Example}[section]
%\end{verbatim}                             ^^A No blank at line start
% leads to the result that the set {\envfont Cor} is formatted in the
% style \texttt{break}, while the set {\envfont Exa} and all the
% following ones are formatted in the style \texttt{plain}, unless
% another |\theoremstyle| follows.  Since the definitions installed by
% |\newtheorem| are global, one also can limit |\theoremstyle| locally
% by grouping braces.
%
% \DescribeMacro\theorembodyfont
% The choice of the font for the theorem body is completely
% independent of the chosen |\theoremstyle|; this has proven to
% be very advantageous.  For example,
% \begin{verbatim}
%   {\theorembodyfont{\upshape}          \newtheorem{Rem}{Remark}}
%\end{verbatim}
% defines a theorem set {\envfont Rem}, which will be set in
% |\upshape| in the current layout (which in our example is
% \texttt{plain}).  As with |\theoremstyle|, the |\theorembodyfont|
% chosen is that current at the time of |\newtheorem|.  If
% |\theorembodyfont| is not specified or one defines
% |\theorembodyfont{}|, then the font used will be that defined by the
% |\theoremstyle|.
%
% \DescribeMacro\theoremheaderfont It is also possible to customize
% the font used for the theorem headers.  This is, however, a global
% declaration, and therefore there should be at most one
% |\theoremheaderfont| declaration in the preamble.\footnote{If it is
% actually necessary to have different header fonts, one has to define
% new theorem styles (substituting the desired font) or specify the
% information directly in the \texttt{\bslash newtheorem} declaration
% (the unclean variant).}
%
% \DescribeMacro\theorempreskipamount
% \DescribeMacro\theorempostskipamount Two additional parameters
% affect the vertical space around the theorem environments:
% |\theorempreskipamount| and |\theorempostskipamount| define,
% respectively, the spacing before and after such an environment.
% These parameters apply for all theorem sets and can be manipulated
% with the ordinary length macros.  They are rubber lengths
% (`\textsf{skips}'), and therefore can contain \texttt{plus} and
% \texttt{minus} parts.
%
% Since the definition of theorem sets should---most sensibly---be
% placed in the preamble, we only allow installation there.
% It is therefore possible to release the memory used here after
% |\begin{document}|, in order to make room for other
% applications.
%
%
% \subsection{Existing theorem styles}
%
% The following theorem styles exist to date:
% \begin{deflist}{marginbreak}
%    \item[plain]
%       This theorem style emulates the original \LaTeX{} definition,
%       except that additionally the parameters
%       |\theorem...skipamount| are used.
%    \item[break]
%       In this style, the theorem header is followed by a line
%       break.
%    \item[marginbreak]
%       The theorem number is set in the margin, and there is a line
%       break as in \texttt{break}.
%    \item[changebreak]
%       Like \texttt{break}, but with header number and text
%       interchanged.
%    \item[change]
%       Header number and text are interchanged, without a line break.
%    \item[margin]
%       The number is set in the left margin, without a line break.
% \end{deflist}
% All styles (except \texttt{plain}) select |\slshape| as the default
% |\theorembodyfont|.
%
%
% \subsection{Examples}
%
% Given the above theorem sets {\envfont Cor}, {\envfont Exa}
% and {\envfont Rem},
% suppose that the preamble also contains the declarations:
% \begin{verbatim}
%    \theoremstyle{marginbreak}   \newtheorem{Lem}[Cor]{Lemma}
%    \theoremstyle{change}
%    \theorembodyfont{\itshape}        \newtheorem{Def}[Cor]{Definition}
%
%    \theoremheaderfont{\scshape}
%\end{verbatim}
% Then the following are some typical examples of the typeset output
% resulting from their use.
%
% \begin{Cor}
%    This is a sentence typeset in the theorem environment {\envfont
%    Cor}.
% \end{Cor}
%
% \begin{Exa}
%    This is a sentence typeset in the theorem environment {\envfont
%    Exa}.
% \end{Exa}
%
% \begin{Rem}
%    This is a sentence typeset in the theorem environment {\envfont
%    Rem}.
% \end{Rem}
%
% \begin{Lem}[Ben User]
%    This is a sentence typeset in the theorem environment {\envfont
%    Lem}.
% \end{Lem}
%
% \begin{Def}[Very impressive Definition]
%    This is a sentence typeset in the theorem environment {\envfont
%    Def}.
% \end{Def}
%
% The last two examples show the effect of the optional argument to
% a theorem environment (it is the text typeset in parentheses).
%
%
% \section{Special Considerations}
% \label{sec:newfonts}
%
% Theoremheader and body are implemented as a unit. This means that
% the |\theoremheaderfont|  will inherit characteristics of the
% |\theorembodyfont| in \LaTeXe. Thus, if for example
% |\theorembodyfont| is |\itshape| and |\theoremheaderfont|
% is |\bfseries| the font selected for the header will have the
% characteristics `bold extended italic'. If this is not desired one
% should set the |\theoremheaderfont| to something like
% \begin{verbatim}
%      \theoremheaderfont{\normalfont\bfseries}
%\end{verbatim}
% i.e.\ supplying all necessary font information explicitly.
%
%
%
% \section{Acknowledgements}
%
% The publication of this set of macros was only possible with the
% help of Christina Busse (translating the manuscript into English),
% Joachim Pense (playing the r\^ole of typist), Chris Rowley
% (looking everything over) and many others providing useful
% suggestions.
%
% \MaybeStop{%
%    \begin{thebibliography}{9}
%       \bibitem{bk:GMS94} \textsc{M.~Goossens}, \textsc{F.~Mittelbach}
%         and \textsc{A.~Samarin}.
%         \newblock The \LaTeX{} Companion.
%         \newblock
%         Addison-Wesley, Reading, Massachusetts, 1994.
%       \bibitem{latex:source} \textsc{Lamport, Leslie.} \newblock
%                               \texttt{latex.tex}, version 2.09, date
%                               Feb. 1990.
%    \end{thebibliography}
%    \PrintIndex
%    \PrintChanges}
%
%
% \section{The documentation driver file}
%
% The next bit of code contains the documentation driver file for
% \TeX{}, i.e., the file that will produce the documentation you are
% currently reading. It will be extracted from this file by the
% \texttt{docstrip} program. Since it is the first code in the file
% one can alternatively process this file directly with \LaTeXe{} to
% obtain the documentation.
%
%    \begin{macrocode}
%<*driver>
\documentclass{ltxdoc}

\usepackage{theorem}

%<+driver>% The next few lines define theorem sets which are used
%<+driver>% in the example section of the documentation.

\theoremstyle{break}       \newtheorem{Cor}{Corollary}
\theoremstyle{plain}       \newtheorem{Exa}{Example}[section]
{\theorembodyfont{\upshape}\newtheorem{Rem}{Remark}}
\theoremstyle{marginbreak} \newtheorem{Lem}[Cor]{Lemma}
\theoremstyle{change}
\theorembodyfont{\itshape} \newtheorem{Def}[Cor]{Definition}

\theoremheaderfont{\scshape}

\RecordChanges

\begin{document}
   \DocInput{theorem.dtx}
\end{document}
%</driver>
%    \end{macrocode}
%
% \section{Definition of the Macros}
%
% If the file has been loaded before, we abort immediately. If not the
% package announces itself (this is actually done at the very top of
% the file---the way it is done isn't good style so don't copy it).
% \changes{v2.0e}{89/07/19}{Spaces removed from `typeout'.}
% \changes{v2.2b}{95/11/19}{Announce moved to top of file}
%    \begin{macrocode}
%<*package>
%\@ifundefined{theorem@style}{}{\endinput}
%\def\FMithmInfo{1995/11/19 v2.2b Theorem extension package (FMi)}
%\ProvidesPackage{theorem}[\FMithmInfo]
%    \end{macrocode}
%
%
% \subsection{Definition of theorem styles and fonts}
%
% All the definitions in this file are done globally to allow
% inputting this file inside a group.
%
% \begin{macro}{\theoremstyle}
%    Before a theorem style can be installed, the chosen style must be
%    known. For that reason, we must test to see that
%    |\th@|\meta{style} is known or, more precisely, that it is
%    different from |\relax|. If the style is not known then
%    |\th@plain| is used.
%    \begin{macrocode}
\gdef\theoremstyle#1{%
   \@ifundefined{th@#1}{\@warning
          {Unknown theoremstyle `#1'. Using `plain'}%
          \theorem@style{plain}}%
%    \end{macrocode}
%    We save the theorem style to be used in the token register
%    |\theorem@style|.
%    \begin{macrocode}
      {\theorem@style{#1}}%
%    \end{macrocode}
%    Now we ``evaluate'' the theorem style: this means, we call the
%    macro |\th@|\meta{style} which will activate the relevant
%    definitions which are contained in a separate file.  This is done
%    in a group to suppress changes to the current font.  This could
%    otherwise pose problems together with the new font selection
%    scheme\footnote{When I printed the original article using the new
%    font selection scheme I ended with a document with slanted
%    typefaces (text headings and all) simply because one of the
%    theorem styles used \texttt{\bslash sl} at toplevel.} if the
%    |\th@|\meta{style} is evaluated a second time.
% \changes{v2.1a}{90/02/15}{Group added around `th@..}
% \changes{v2.1a}{90/02/15}{Reseting of `..@begintheorem removed.}
%    \begin{macrocode}
      \begingroup
        \csname th@\the\theorem@style \endcsname
      \endgroup}
%    \end{macrocode}
% \end{macro}
%
%
% \begin{macro}{\@begintheorem}
% \changes{v2.1a}{90/02/15}{Relaxed at toplevel.}
% \begin{macro}{\@opargbegintheorem}
% \changes{v2.1a}{90/02/15}{Relaxed at toplevel.}
%    We reset |\@begintheorem| and |\@opargbegintheorem| to
%    |\relax| since these commands are no longer necessary at
%    toplevel. This will save a few tokens.
%    \begin{macrocode}
\global\let\@begintheorem\relax
\global\let\@opargbegintheorem\relax
%    \end{macrocode}
% \end{macro}
% \end{macro}
%
% \begin{macro}{\theorem@style}
%    Obviously the \textsf{token} register used above has to be
%    allocated. To assure the utmost compatibility with the
%    original \LaTeX{} definition, we set the default theorem style to
%    \texttt{plain}, which implements the usual \LaTeX{} convention.
% \changes{v2.0f}{89/07/20}{Unnecessary `global' removed.}
%    \begin{macrocode}
\newtoks\theorem@style
\global\theorem@style{plain}
%    \end{macrocode}
% \end{macro}
%
% \begin{macro}{\theorembodyfont}
% \begin{macro}{\theorem@bodyfont}
%    If the |\theorembodyfont| is set by the user then it should not
%    interact with the default font set in the theorem style. When the
%    new font selection is in force this may happen if, for example,
%    the default is |\itshape| and the new |\theorembodyfont| is
%    |\sffamily|. So we add a |\reset@font| command in front of the
%    user definition.
% \changes{v2.0f}{89/07/20}{Unnecessary `global' removed.}
% \changes{v2.0g}{89/09/19}{`theoremfont to `theorembodyfont.}
% \changes{v2.1c}{92/06/22}{`reset@font added and logic changed.}
% \changes{v2.1d}{92/10/28}{Reset to empty if arg is empty}
%    \begin{macrocode}
\gdef\theorembodyfont#1{%
%    \end{macrocode}
%    We check if the argument supplied is empty and if so put nothing
%    into the |\theorem@bodyfont| token register to allow for
%    |\theorembodyfont{}| as a mean of using the default of the
%    current |\theoremstyle|.
%    \begin{macrocode}
   \def\@tempa{#1}%
   \ifx\@tempa\@empty
    \theorem@bodyfont{}%
   \else
    \theorem@bodyfont{\reset@font#1}%
   \fi
   }
\newtoks\theorem@bodyfont
\global\theorem@bodyfont{}
%    \end{macrocode}
% \end{macro}
% \end{macro}
%
%
% \begin{macro}{\theoremheaderfont}
% \changes{v2.0g}{89/09/19}{Macro added.}
%    The font for the theorem headers is handled differently because
%    this definition applies to all theorem styles.
%    \begin{macrocode}
\gdef\theoremheaderfont#1{\gdef\theorem@headerfont{#1}%
%    \end{macrocode}
%    After using the macro once it is redefined to produce an error
%    message.
%    \begin{macrocode}
       \gdef\theoremheaderfont##1{%
        \typeout{\string\theoremheaderfont\space should be used
                 only once.}}}
%    \end{macrocode}
% \end{macro}
%
% \begin{macro}{\theorem@headerfont}
% \changes{v2.0g}{89/09/19}{Macro added.}
%    To set the |\theorem@headerfont| default we first test if
%    the new fontselection scheme is in force.
% \changes{v2.1a}{90/02/15}{Support for new font selection.}
%    \begin{macrocode}
\ifx\upshape\undefined
%    \end{macrocode}
%    If not we define it to expand into |\bfseries|. We don't use
%    |\let| just in case a following style option redefines this
%    macro.
%    \begin{macrocode}
\gdef\theorem@headerfont{\bfseries}
%    \end{macrocode}
%    Otherwise we reset the current shape before calling |\bfseries|.
%    \begin{macrocode}
\else \gdef\theorem@headerfont{\normalfont\bfseries}\fi
%    \end{macrocode}
% \end{macro}
%
% \begin{macro}{\th@plain}
% \begin{macro}{\th@break}
% \begin{macro}{\th@marginbreak}
% \begin{macro}{\th@changebreak}
% \begin{macro}{\th@change}
% \begin{macro}{\th@margin}
%    The different styles are defined in macros such as
%    |\th@plain|.  Since memory space is precious in
%    ``non-Big-versions'', we have to avoid offering too many unused
%    definitions.  Therefore we define these styles in separate files
%    that can be loaded on demand.  Thus the commands themselves only
%    load these files. We use |\@input@|, a \LaTeXe{} internal command
%    that ensures that the file will be listed with |\listfiles|.
% \changes{v2.2a}{94/02/02}{Use `@input@ to load theorem layout files}
%    \begin{macrocode}
\gdef\th@plain{\@input@{thp.sty}}
\gdef\th@break{\@input@{thb.sty}}
\gdef\th@marginbreak{\@input@{thmb.sty}}
\gdef\th@changebreak{\@input@{thcb.sty}}
\gdef\th@change{\@input@{thc.sty}}
\gdef\th@margin{\@input@{thm.sty}}
%    \end{macrocode}
%    This list will be expanded when new styles become
%    available.  For testing,  just append new theorem substyles
%    as document options.
% \end{macro}
% \end{macro}
% \end{macro}
% \end{macro}
% \end{macro}
% \end{macro}
%
%
%
% \subsection{Definition of a new theorem set}
%
% As already pointed out, a new theorem environment can be defined in
% three different ways:
% \begin{verbatim}
%    \newtheorem{Lem}{Lemma}
%    \newtheorem{Lem}{Lemma}[section]
%    \newtheorem{Lem}[Theorem]{Lemma}
%\end{verbatim}
% The function of the macro |\newtheorem| is to recognize these cases
% and then to branch into one of the three macros |\@ynthm|, |\@xnthm|
% or |\@othm|.  This mechanism is adopted unchanged from
% \cite{latex:source}; the essential point here is that, for example,
% in the second case, the arguments \texttt{Lem}, \texttt{Lemma} and
% \texttt{section} are passed over to the macro |\@xnthm|.
%
% We inspect this case first because the others present fewer problems,
% and thus are easily derived from this one.
%
% \begin{macro}{\@xnthm}
%    For our example arguments, the macro |\@xnthm| must fulfill
%    the following:
%    \begin{itemize}
%       \item
%          Define a new \LaTeX-counter `Lem'
%       \item
%          reset this counter within a |\section|
%       \item
%          define the macro |\theLem|
%       \item
%          define the environment macros |\Lem| and
%          |\endLem| using the current
%          |\theoremstyle| and |\theorem@bodyfont|.
%    \end{itemize}
%    Obviously, all this should happen only if the first argument
%    of |\@xnthm| (i.e.\ \texttt{Lem} in our example) is chosen so
%    as not to conflict with any previously defined commands or
%    environments.  This test is performed by the \LaTeX{} macro
%    |\@ifdefinable|.
%    \begin{macrocode}
\gdef\@xnthm#1#2[#3]{\expandafter\@ifdefinable\csname #1\endcsname
%    \end{macrocode}
%    Therefore, the first argument of |\@ifdefinable| is the
%    expansion (in the example, |\Lem|) of
%    |\csname||#1\endcsname|.  The second argument is
%    executed only if the test has been completed successfully.
%    \begin{macrocode}
   {%
%    \end{macrocode}
%    Now we define the new counter. The names of the \LaTeX{}
%    macros employed should speak for themselves:
% \changes{v2.2b}{1995/10/07}{Use \cs{@newctr}}
% \changes{v2.2c}{1995/11/23}{Use \cs{@newctr} properly.}
%    \begin{macrocode}
    \@definecounter{#1}\@newctr{#1}[#3]%
%    \end{macrocode}
%    Using |\@newctr| will give a proper error message if the counter
%    in |#3| is not defined.
%    In defining `|\theLem|' we must generate the desired macro
%    name by use of |\expandafter| and |\csname|.
%    \begin{macrocode}
    \expandafter\xdef\csname the#1\endcsname
%    \end{macrocode}
%    An |\xdef| is used in order to make the definition global,
%    and to ensure that it contains the replacement texts of
%    |\@thmcountersep| and |\@thmcounter|.\footnote{These
%    two macros can be defined by the document style.  Their default
%    values produce a `\texttt{.}' as separation and an arabic
%    representation of the number.} However, not everything should be
%    expanded. For example, it saves space to use |\thesection|
%    instead of its---at times---lengthy expansion.
%    \begin{macrocode}
      {\expandafter \noexpand \csname the#3\endcsname
       \@thmcountersep \@thmcounter{#1}}%
%    \end{macrocode}
%    Thus with the defaults of \LaTeX{}, |\theLem| would be
%    replaced by the command sequence |\thesection.\arabic{Lem}|.
%
%    We will now look at the definition of the macro which is executed
%    at the beginning of the actual environment (in our example this
%    macro is |\Lem|). It should be noted that we use an
%    ``|\expandafter| trick'' to expand only certain parts of the
%    replacement text at the time of the definition.
%    \begin{macrocode}
    \def\@tempa{\global\@namedef{#1}}%
    \expandafter \@tempa \expandafter{%
%    \end{macrocode}
%    First, the macro that contains the current definitions of
%    |\@begintheorem| and |\@opargtheorem| should be called
%    up.  The name of this macro---as is already known---has the form
%    |\th@|\meta{theorem\ style}; therefore, it must be called by
%    \begin{macrocode}
      \csname th@\the \theorem@style
%    \end{macrocode}
%    In addition the default theorem font should be changeable, i.e.\
%    we have to insert the contents of |\theorem@bodyfont|. For
%    that reason, we expand even further, beyond |\endcsname|,
%    and thus insert the contents of the \textsf{token} register
%    |\theorem@bodyfont| in the replacement text.
%    \begin{macrocode}
            \expandafter \endcsname \the \theorem@bodyfont
%    \end{macrocode}
%    Now it is time to call the macro |\@thm| which takes over
%    the further processing.  It has two arguments: the current
%    counter name (in our example, \texttt{Lem}), and the text of the
%    label (in our example, \texttt{Lemma}).
%    \begin{macrocode}
     \@thm{#1}{#2}}%
%    \end{macrocode}
%    With this, the `sub-definition' is complete.  The macro
%    |\@endtheorem| ends a theorem environment and is, so far, nothing
%    but an |\endtrivlist|.  (Hence it is defined globally, and not
%    within the theorem styles.\footnote{This has to be changed as
%    soon as theorem styles that change \texttt{\bslash @endtheorem}
%    exist.  In such a case, all existing styles must be changed as
%    well since they will have to reset the macro.}) Therefore, we can
%    set it equivalent to the macro that ends the theorem set (in our
%    example, |\endLem|).  However, if some day theorem styles exist
%    that do change |\@endtheorem|, we would have to use the
%    commented-out line instead.
%    \begin{macrocode}
    \global \expandafter \let \csname end#1\endcsname \@endtheorem
%   \global\@namedef{end#1}{\@endtheorem}%
%    \end{macrocode}
%    With these commands all the required definitions are employed,
%    unless the test |\@ifdefinable| has failed.  Therefore, we
%    end the second argument of this macro and with it the definition
%    of |\@xnthm|.
%    \begin{macrocode}
   }}
%    \end{macrocode}
% \end{macro}
%
% \begin{macro}{\@ynthm}
%    The definition of |\@ynthm| is completely analogous. In
%    this case the new counter that is defined is not reset within
%    another counter; thus the definition of |\the...|
%    is simplified:
%    \begin{macrocode}
\gdef\@ynthm#1#2{\expandafter\@ifdefinable\csname #1\endcsname
   {\@definecounter{#1}%
    \expandafter\xdef\csname the#1\endcsname{\@thmcounter{#1}}%
%    \end{macrocode}
%    The rest of the definition corresponds literally to that of
%    |\@xnthm|:
%    \begin{macrocode}
    \def\@tempa{\global\@namedef{#1}}\expandafter \@tempa
     \expandafter{\csname th@\the \theorem@style \expandafter
     \endcsname \the\theorem@bodyfont \@thm{#1}{#2}}%
    \global \expandafter \let \csname end#1\endcsname \@endtheorem}}
%    \end{macrocode}
% \end{macro}
%
% \begin{macro}{\@othm}
%    The definition of |\@othm| does not contain anything new.
%    \begin{macrocode}
\gdef\@othm#1[#2]#3{%
%    \end{macrocode}
% \changes{v2.2b}{95/10/07}{Build in check for counter defined}
%    We do not define a new counter but instead use one that has
%    already been defined.  Thus the only definition we need is that
%    of this pseudo-counter (i.e.\ |\the|\meta{env.\ name}).
%    First we check if |#2| corresponds to a known counter name.
%    \begin{macrocode}
  \expandafter\ifx\csname c@#2\endcsname\relax
   \@nocounterr{#2}%
  \else
   \expandafter\@ifdefinable\csname #1\endcsname
   {\expandafter \xdef \csname the#1\endcsname
     {\expandafter \noexpand \csname the#2\endcsname}%
%    \end{macrocode}
%    All other parts of the definition can be adopted from
%    |\@xnthm|. We have to remember, though, that in this case
%    the name of the current counter and the theorem label have moved
%    to the second and third arguments.
%    \begin{macrocode}
    \def\@tempa{\global\@namedef{#1}}\expandafter \@tempa
     \expandafter{\csname th@\the \theorem@style \expandafter
     \endcsname \the\theorem@bodyfont \@thm{#2}{#3}}%
    \global \expandafter \let \csname end#1\endcsname \@endtheorem}%
  \fi}
%    \end{macrocode}
% \end{macro}
%
% \subsection{Macros that are employed in a {\envfont theorem}
%    environment}
%
% \begin{macro}{\@thm}
%    The macro |\@thm| has to increase the current counter.  Then,
%    depending on whether the environment has (or does not have) an
%    optional argument, it has to branch into either |\@begintheorem|
%    or |\@opargtheorem|.
%    \begin{macrocode}
\gdef\@thm#1#2{\refstepcounter{#1}%
%    \end{macrocode}
%    Now we start a {\envfont trivlist} environment, and give
%    |\@topsep| and |\@topsepadd| the values of the \textsf{skip}
%    registers |\theorempreskipamount| and |\theorempostskipamount|.
%    The value in |\@topsep| is the vertical space that is inserted by
%    the first (and only) |\item| in our |\trivlist| whilst
%    |\@topsepadd| is inserted by |\@endparenv| at the end of that
%    {\envfont trivlist} environment.  By using these registers, we
%    obtain the desired space around a {\envfont theorem} environment.
%    \begin{macrocode}
   \trivlist
   \@topsep \theorempreskipamount               % used by first \item
   \@topsepadd \theorempostskipamount           % used by \@endparenv
%    \end{macrocode}
%    Now we have to test whether an optional argument has been given.
% \changes{v2.0d}{89/07/18}{Forgotten percent added.}
%    \begin{macrocode}
   \@ifnextchar [%
%    \end{macrocode}
%    If there is an optional argument, we will call |\@ythm|, and
%    move the arguments read back into the input stream.
%    \begin{macrocode}
   {\@ythm{#1}{#2}}%
%    \end{macrocode}
%    If not, we call |\@begintheorem|. Its first argument is the
%    name of the theorem set (hence the second argument of
%    |\@thm|). Its second argument is the macro that produces the
%    current number.
%    \begin{macrocode}
   {\@begintheorem{#2}{\csname the#1\endcsname}\ignorespaces}}
%    \end{macrocode}
% \end{macro}
%
% \begin{macro}{\@xthm}
% \begin{macro}{\@ythm}
%    Both these macros were originally called by |\@thm|.  We do
%    not need |\@xthm| anymore, hence we reset it to
%    |\relax|. The definition of |\@ythm| has not changed at
%    all from its definition in \LaTeX. In order to make the macros
%    easier to understand, we will
%    nevertheless present it (commented out).
%    \begin{macrocode}
\global\let\@xthm\relax
% \def\@ythm#1#2[#3]{\@opargbegintheorem{#2}{\csname
%           the#1\endcsname}{#3}\ignorespaces}
%</package>
%    \end{macrocode}
%    The primitive |\ignorespaces| in |\@ythm| and
%    |\@thm| is needed to remove the spaces between the
%    |\begin{...}| and the actual text.
% \end{macro}
% \end{macro}
%
%
% \subsection{Definition of the theorem substyles}
%
% As already pointed out, the theorem substyles, defined below, are
% only loaded when necessary.  Note that all these substyles, except
% \texttt{plain}, have |\slshape| as the default body font.
%
% \subsubsection{The \texttt{plain} style}
%
% \label{dok:th@plain} As the following macros use |@|, we have
% to locally set the |\catcode| of this symbol to ``letter''.
% This happens within a group, so that we do not have to worry about
% which |\catcode| that symbol had before.
%    \begin{macrocode}
%<*thp>
\begingroup \makeatletter
%    \end{macrocode}
% Since we are now within a group, we must make all definitions
% globally.  First we make sure that \texttt{theorem.sty} is loaded.
% This will allow us to use this file as a document style option
% without having to call \texttt{theorem} itself as an option.  At the
% same time, we assure that at least version 2 is loaded, since
% |\theorem@style| was not defined in earlier versions.
%    \begin{macrocode}
\@ifundefined{theorem@style}{\input{theorem.sty}}{}
\ProvidesFile{thp.sty}
             [\FMithmInfo]
%    \end{macrocode}
%
%
% \begin{macro}{\th@plain}
% \begin{macro}{\@begintheorem}
% \begin{macro}{\@opargbegintheorem}
%    |\theoremstyle{plain}| corresponds to the original
%    definition, except that the distances to the surrounding text are
%    determined by the parameters |\theorempreskipamount| and
%    |\theorempostskipamount|.
%    First we set the default body font.
%    \begin{macrocode}
\gdef\th@plain{\normalfont\itshape
%    \end{macrocode}
%    Then we define |\@begintheorem| and
%    |\@opargbegintheorem|. These two macros define how the
%    header of a theorem is typeset.  |\@opargbegintheorem| will
%    be called if a \textsf{theorem} environment with an optional
%    argument is encountered; otherwise, the header is constructed by
%    calling |\@begintheorem|.  If one of these macros is
%    executed, we are within a \textsf{trivlist} environment started by
%    |\@thm|.  So the theorem header is produced with an
%    |\item| command.
%
%    Instead of specifying the header font directly, all standard
%    theorem styles use the |\theorem@headerfont| macro to allow
%    customization.  The extra space (|\labelsep|) is necessary
%    because of problems in the \textsf{trivlist} environment.
%    \begin{macrocode}
  \def\@begintheorem##1##2{%
        \item[\hskip\labelsep \theorem@headerfont ##1\ ##2]}%
%    \end{macrocode}
%    The definition of |\@opargbegintheorem| is completely
%    analogous. The only difference is the fact that there exists a
%    third argument (which is the optional parameter of the
%    environment and contains additional information about the
%    theorem).  Customarily we enclose it in parentheses.
%    \begin{macrocode}
\def\@opargbegintheorem##1##2##3{%
   \item[\hskip\labelsep \theorem@headerfont ##1\ ##2\ (##3)]}}
%    \end{macrocode}
% \end{macro}
% \end{macro}
% \end{macro}
%
% \noindent
% We conclude with an |\endgroup| to restore the
% |\catcode| of |@|.
%    \begin{macrocode}
\endgroup
%</thp>
%    \end{macrocode}
%
%
% \subsubsection{The \texttt{break} style}
%
% This style option is stored in the file |thb.sty|.
% For the next two lines see the documentation for |\th@plain| on
% page~\pageref{dok:th@plain}.
%    \begin{macrocode}
%<*thb>
\begingroup \makeatletter
\@ifundefined{theorem@style}{\input{theorem.sty}}{}
\ProvidesFile{thb.sty}
             [\FMithmInfo]
%    \end{macrocode}
%
% \begin{macro}{\th@break}
%    |\theoremstyle{break}| produces a line break after the name
%    of the theorem.  The font is |\slshape|.  Hence, we define
%    |\th@break| as follows:
%    \begin{macrocode}
\gdef\th@break{\normalfont\slshape
  \def\@begintheorem##1##2{\item[%
%    \end{macrocode}
%    We run into the following problem: it is not possible to create
%    the header with |\item[|\meta{title}|]| and then start
%    a new line by, for example, |\mbox{}\\|. Such a definition
%    will fail whenever a list environment follows immediately.  With
%    the above construction, the |\mbox{}| causes the switch
%    |@inlabel| (cf.\ definition of |\list| and
%    |\trivlist| in \cite{latex:source}) to be set to \texttt{false}
%    and so the following list will insert additional vertical
%    space (|\topskip|). This is quite annoying. Therefore, we
%    create the line break within the |\item|. In order to ensure
%    that the text will begin at the proper position in the following
%    line, we simply pretend that the label does not take any
%    room.\footnote{This will lead to problems whenever very high
%    symbols occurring in the line tower into the
%    heading. So, something else has to be done here sometime.}
%    \begin{macrocode}
       \rlap{\vbox{\hbox{\hskip \labelsep\theorem@headerfont ##1\ ##2}%
                   \hbox{\strut}}}]}%
%    \end{macrocode}
%    Again, the definition of |\@opargbegintheorem| is
%    completely analogous.
%    \begin{macrocode}
\def\@opargbegintheorem##1##2##3{%
  \item[\rlap{\vbox{\hbox{\hskip \labelsep \theorem@headerfont
                     ##1\ ##2\ (##3)}%
                    \hbox{\strut}}}]}}
\endgroup
%</thb>
%    \end{macrocode}
% \end{macro}
%
%
% \subsubsection{The \texttt{changebreak} style}
%
%    \begin{macrocode}
% This style option is stored in the file |thcb.sty|.
%    \begin{macrocode}
%<*thcb>
\begingroup \makeatletter
\@ifundefined{theorem@style}{\input{theorem.sty}}{}
\ProvidesFile{thcb.sty}
             [\FMithmInfo]
%    \end{macrocode}
%
% \begin{macro}{\th@changebreak}
%    The \texttt{change-break} theorem style is like \texttt{break}
%    but with interchange of theorem name and theorem number.  Thus we
%    define |\th@changebreak| as follows:
%    \begin{macrocode}
\gdef\th@changebreak{\normalfont\slshape
  \def\@begintheorem##1##2{\item
       [\rlap{\vbox{\hbox{\hskip\labelsep \theorem@headerfont ##2\ ##1}%
                    \hbox{\strut}}}]}%
\def\@opargbegintheorem##1##2##3{%
  \item[\rlap{\vbox{\hbox{\hskip\labelsep \theorem@headerfont
                     ##2\ ##1\ (##3)}%
                    \hbox{\strut}}}]}}
\endgroup
%</thcb>
%    \end{macrocode}
% \end{macro}
%
%
% \subsubsection{The \texttt{change} style}
%
% This style option is stored in the file |thc.sty|.
%    \begin{macrocode}
%<*thc>
\begingroup \makeatletter
\@ifundefined{theorem@style}{\input{theorem.sty}}{}
\ProvidesFile{thc.sty}
             [\FMithmInfo]
%    \end{macrocode}
%
% \begin{macro}{\th@change}
%    The \texttt{change} theorem style corresponds to the
%    \texttt{change break} style without a linebreak after the header.
%    To say it in another way, it's the same as the \texttt{plain}
%    style but with number and name interchanged and |\slshape| as the
%    default font.
%    \begin{macrocode}
\gdef\th@change{\normalfont\slshape
  \def\@begintheorem##1##2{\item
       [\hskip\labelsep \theorem@headerfont ##2\ ##1]}%
\def\@opargbegintheorem##1##2##3{%
  \item[\hskip\labelsep \theorem@headerfont ##2\ ##1\ (##3)]}}
\endgroup
%</thc>
%    \end{macrocode}
% \end{macro}
%
%
% \subsubsection{The \texttt{marginbreak} style}
%
% This style option is the one used most often at Mainz.  It is saved
% in the file |thmb.sty|.
%    \begin{macrocode}
%<*thmb>
\begingroup \makeatletter
\@ifundefined{theorem@style}{\input{theorem.sty}}{}
\ProvidesFile{thmb.sty}
             [\FMithmInfo]
%    \end{macrocode}
%
% \begin{macro}{\th@marginbreak}
%    The \texttt{margin break} style is nearly the same as the
%    \texttt{change break} style. The only difference is the placement
%    of the theorem number.  We use |\llap| to place it in the left
%    margin.
%
%    In this style |\labelsep| denotes the separation between the
%    number and the text.
%    \begin{macrocode}
\gdef\th@marginbreak{\normalfont\slshape
  \def\@begintheorem##1##2{\item
       [\rlap{\vbox{\theorem@headerfont
                    \hbox{\llap{##2}\hskip\labelsep ##1}%
                    \hbox{\strut}}}]}%
\def\@opargbegintheorem##1##2##3{%
  \item[\rlap{\vbox{\theorem@headerfont
                    \hbox{\llap{##2}\hskip\labelsep ##1\ (##3)}%
                    \hbox{\strut}}}]}}
\endgroup
%</thmb>
%    \end{macrocode}
% \changes{v2.1c}{90/02/17}{`endgroup outside macrocode env.}
% \end{macro}
%
%
%
% \subsubsection{The \texttt{margin} style}
%
% This style option is stored in the file |thm.sty|.
%    \begin{macrocode}
%<*thm>
\begingroup \makeatletter
\@ifundefined{theorem@style}{\input{theorem.sty}}{}
\ProvidesFile{thm.sty}
             [\FMithmInfo]
%    \end{macrocode}
%
% \begin{macro}{\th@margin}
%    Again this is only a variant of the theorem styles described
%    above without any new ideas.
%    \begin{macrocode}
\gdef\th@margin{\normalfont\slshape
  \def\@begintheorem##1##2{\item
       [\theorem@headerfont \llap{##2}\hskip\labelsep ##1]}%
\def\@opargbegintheorem##1##2##3{%
  \item[\theorem@headerfont \llap{##2}\hskip\labelsep ##1\ (##3)]}}
\endgroup
%</thm>
%    \end{macrocode}
% \end{macro}
%
%
%
%
% \subsection{Final Definitions}
%
% \begin{macro}{\theorempreskipamount}
% \begin{macro}{\theorempostskipamount}
%    The \textsf{skip} parameters that regulate the vertical empty space
%    before and after the {\envfont theorem} environment have
%    to be allocated as well.
% \changes{v2.0f}{89/07/20}{Unnecessary `global' removed.}
%    \begin{macrocode}
%<*package>
\newskip\theorempreskipamount
\newskip\theorempostskipamount
%    \end{macrocode}
%    Since we have used the same values for all theorem sets, we now can
%    assign them.
%    \begin{macrocode}
\global\setlength\theorempreskipamount{12pt plus 5pt minus 3pt}
\global\setlength\theorempostskipamount{8pt plus 3pt minus 1.5pt}
%    \end{macrocode}
% \end{macro}
% \end{macro}
%
%
% \begin{macro}{\@endtheorem}
%    The same holds for the macro |\@endtheorem|, which ends a
%    {\envfont theorem} environment. Since it is the same for all
%    theorem sets, it is removed from the macros |\th@|\meta{style}.
%    It simply ends the {\envfont trivlist} environment, which was
%    begun in |\@thm|.
%    \begin{macrocode}
\global\let\@endtheorem=\endtrivlist
%    \end{macrocode}
% \end{macro}
%
%
%
% \begin{macro}{\@preamblecmds}
%    All macros defined above are to be used only in the preamble.
%    Therefore, we insert them in |\@preamblecmds| which will disable
%    them at begin document. This is done by the internal \LaTeXe{}
%    command |\@onlypreamble|.
% \changes{v2.0g}{89/09/19}{`theoremheaderfont and `theorembodyfont
%    added.}
% \changes{v2.2a}{94/02/03}{Use \LaTeXe{} interface}
%    \begin{macrocode}
\@onlypreamble\@xnthm
\@onlypreamble\@ynthm
\@onlypreamble\@othm
\@onlypreamble\newtheorem
\@onlypreamble\theoremstyle
\@onlypreamble\theorembodyfont
\@onlypreamble\theoremheaderfont
%    \end{macrocode}
% \end{macro}
%
% Finally we declare the \texttt{plain} theorem style to be the default.
% \changes{v2.1b}{90/06/04}{Fixed missing `item by declaring default
%    style.}
%    \begin{macrocode}
\theoremstyle{plain}
%</package>
%    \end{macrocode}
%
% \Finale

\endinput
}{}
\ProvidesFile{thc.sty}
             [\FMithmInfo]
%    \end{macrocode}
%
% \begin{macro}{\th@change}
%    The \texttt{change} theorem style corresponds to the
%    \texttt{change break} style without a linebreak after the header.
%    To say it in another way, it's the same as the \texttt{plain}
%    style but with number and name interchanged and |\slshape| as the
%    default font.
%    \begin{macrocode}
\gdef\th@change{\normalfont\slshape
  \def\@begintheorem##1##2{\item
       [\hskip\labelsep \theorem@headerfont ##2\ ##1]}%
\def\@opargbegintheorem##1##2##3{%
  \item[\hskip\labelsep \theorem@headerfont ##2\ ##1\ (##3)]}}
\endgroup
%</thc>
%    \end{macrocode}
% \end{macro}
%
%
% \subsubsection{The \texttt{marginbreak} style}
%
% This style option is the one used most often at Mainz.  It is saved
% in the file |thmb.sty|.
%    \begin{macrocode}
%<*thmb>
\begingroup \makeatletter
\@ifundefined{theorem@style}{% \iffalse meta-comment
%
% Copyright (C) 1993-2024
%
% The LaTeX Project and any individual authors listed elsewhere
% in this file.
%
% This file is part of the Standard LaTeX `Tools Bundle'.
% -------------------------------------------------------
%
% It may be distributed and/or modified under the
% conditions of the LaTeX Project Public License, either version 1.3c
% of this license or (at your option) any later version.
% The latest version of this license is in
%    https://www.latex-project.org/lppl.txt
% and version 1.3c or later is part of all distributions of LaTeX
% version 2005/12/01 or later.
%
% The list of all files belonging to the LaTeX `Tools Bundle' is
% given in the file `manifest.txt'.
%
% \fi
%
%%
%
%
\def\FMithmInfo{2023/07/05 v2.2c Theorem extension package (FMi)}
%
% \ProvidesFile{theorem.dtx}[\FMithmInfo]
%\iffalse   % this is a METACOMMENT !
%%
%% Package `theorem' to use with LaTeX2e
%% Copyright (C) 1989-1995 Frank Mittelbach, all rights reserved.
%<+package>\@ifundefined{theorem@style}{}{\endinput}
%<+package>\NeedsTeXFormat{LaTeX2e}[1995/06/01]
%<+package>\ProvidesPackage{theorem}
%<+package>         [\FMithmInfo]
%
% \fi
%
% \changes{v1.0}{87/05/07}{First draft.}
% \changes{v1.0a}{87/05/07}{`thistheoremstyle' added.}
% \changes{v1.1a}{87/07/29}{New concept: redefine the internal
%                          \protect\LaTeX{} commands `@xnthm' etc.}
% \changes{v1.1b}{87/07/31}{`theoremfont' used to specify font.}
% \changes{v1.2a}{88/01/08}{`@endtheorem' now globally defined.}
% \changes{v1.2b}{88/01/11}{`newtheorem' only allowed in the preamble.}
% \changes{v1.3a}{88/02/14}{`theoremfont' now a toks register.}
%
% \changes{v2.0a}{88/11/16}{New implementation and new documentation.}
% \changes{v2.0a}{88/11/16}{No layout changes but internal modifications
%                           old substyles won't work.}
% \changes{v2.0b}{89/01/30}{`global' used for all definitions.}
% \changes{v2.0c}{89/06/15}{Documentation translated to English.
%                            Thanks to LLL and Nina}
% \changes{v2.1a}{90/02/15}{Support for new font selection.}
% \changes{v2.2a}{94/02/03}{Cleanup for LaTeX2e}
%
% \DoNotIndex{\@endparenv,\@ifdefinable,\@ifnextchar,\@ifundefined}
% \DoNotIndex{\@namedef,\@nameuse,\@spaces,\@tempa,\@warning}
% \DoNotIndex{\ }
% \DoNotIndex{\begingroup,\bfseries}
% \DoNotIndex{\csname}
% \DoNotIndex{\def,\do,\docdate}
% \DoNotIndex{\endcsname,\endgroup,\endinput,\endtrivlist,\expandafter}
% \DoNotIndex{\filedate,\fileversion}
% \DoNotIndex{\gdef,\global}
% \DoNotIndex{\hbox,\hskip}
% \DoNotIndex{\ignorespaces,\input,\itshape,\item}
% \DoNotIndex{\labelsep,\let,\llap}
% \DoNotIndex{\makeatletter}
% \DoNotIndex{\noexpand}
% \DoNotIndex{\refstepcounter,\relax,\rlap}
% \DoNotIndex{\setlength,\slshape,\space,\string,\strut}
% \DoNotIndex{\the,\trivlist,\typeout}
% \DoNotIndex{\vbox}
% \DoNotIndex{\xdef}
%
% \def\deflabel#1{\ttfamily #1\hfill}
% \def\deflist#1{\begin{list}{}{\settowidth\labelwidth{\ttfamily #1}%
%                               \setlength\leftmargin\labelwidth
%                               \addtolength\leftmargin\labelsep
%                               \let\makelabel\deflabel}}
% \def\enddeflist{\end{list}}
% \def\envfont{\normalfont\ttfamily}
%
% \MakeShortVerb{\|}
%
% \newcommand{\TUB}{{\sl TUGboat\/}}
%
% \GetFileInfo{theorem.dtx}
% \title{An Extension of the
%        \LaTeX{} theorem environment\thanks{This file
%        has version number \fileversion, last
%        revised \filedate.}}
% \author{Frank Mittelbach}
%
% \MaintainedByLaTeXTeam{tools}
% \maketitle
%
% \begin{abstract}
%    The macros described in this paper yield an extension of the
%    \LaTeX{} theorem mechanism. It is designed to satisfy the
%    different requirements of various journals. Thus, the
%    layout of the ``theorems'' can be manipulated by determining a
%    ``style''. This article describes not only the use, but
%    also the definition, of the necessary macros.
% \end{abstract}
%
% \section*{Preface to version 2.2}
%
% For \LaTeXe{} this package did not need any fundamental changes. I
% only modified the messages generated so that theorem layout styles
% will show up with the |\listfiles| command and cleaned the section
% on the New Font Selection Scheme since this is now included in
% \LaTeX.
%
% \section*{Preface to version 2.1}
%
% This version is identical to 2.0g described in \TUB\ 10\#3 except
% for some internal defaults which are now set depending on the used
% font selection scheme.
%
% This was done to avoid unpleasant surprises if the new font selection
% scheme is in force. For further details see
% section~\ref{sec:newfonts} and \cite{bk:GMS94}.
%
% \section{Introduction}
%
% For our purposes here, ``theorems'' are labelled enunciations,
% often set off from the main text by extra space and a font change.
% Theorems, corollaries, conjectures, definitions, and remarks
% are all instances of ``theorems''.  The ``header'' of these structures
% is composed of a label (such as \textsc{Theorem} or \textsc{Remark})
% and a number which serializes an item in the sequence of items
% with the same label.
%
% Shortly after the introduction of \LaTeX{} at the Fachbereich
% Mathematik in Mainz, the desire to manipulate the layout of
% ``theorems'' arose.  In Mainz, the following two conventions came
% into general use:
% \begin{enumerate}
%   \item The number of the theorem is shown in the margin.
%   \item There is a line break at
%         the end of the theorem header.
% \end{enumerate}
%
% Additionally, some journals require different formats which depend on
% the ``sort of theorem'': e.g.\ often remarks and definitions are set
% in |\upshape|, while |\itshape| is employed for main theorems.
%
% Confronted with these requirements, a theorem environment was
% developed in Mainz which allows separate determination of the
% layout of the ``theorems sets'', comparable to |\pagestyle|.
%
%
% \section{The user interface}
% \subsection{Defining new theorem sets}
%
% \DescribeMacro\newtheorem
% As in the original \LaTeX{} version, the command |\newtheorem|
% defines a new ``theorem set'' or ``theorem-like structure''.
% Two required arguments name the new environment and give the text
% to be typeset with each instance of the new ``set'', while
% an optional argument determines how
% the ``set'' is enumerated:
% \begin{description}
%    \item[\ttfamily \bslash newtheorem\{foo\}\{bar\}]
%       The theorem set {\envfont foo} (whose name is \texttt{bar})
%       uses its own counter.
%    \item[\ttfamily \bslash newtheorem\{foo2\}{[foo]}\{bar2\}]
%       The theorem set {\envfont foo2} (printed name \texttt{bar2})
%       uses the same counter as the theorem set \texttt{foo}.
%    \item[\ttfamily \bslash newtheorem\{foo3\}\{bar3\}{[section]}]
%       The theorem set {\envfont foo3} (printed name \texttt{bar3}) is
%       enumerated within the counter \texttt{section}, i.e.\ with every
%       new |\section| the enumeration begins again with 1, and
%       the enumeration is composed from the section-number and the
%       theorem counter itself.
% \end{description}
%
% \DescribeMacro\theoremstyle
% Additionally, the command |\theoremstyle| can define the layout
% of various, or all, theorem sets.  It should be noted that any
% theorem set defined by |\newtheorem| is typeset in the
% |\theoremstyle| that is current at the time of the definition.
% Thus, the following
% \begin{verbatim}
%    \theoremstyle{break}        \newtheorem{Cor}{Corollary}
%    \theoremstyle{plain}        \newtheorem{Exa}{Example}[section]
%\end{verbatim}                             ^^A No blank at line start
% leads to the result that the set {\envfont Cor} is formatted in the
% style \texttt{break}, while the set {\envfont Exa} and all the
% following ones are formatted in the style \texttt{plain}, unless
% another |\theoremstyle| follows.  Since the definitions installed by
% |\newtheorem| are global, one also can limit |\theoremstyle| locally
% by grouping braces.
%
% \DescribeMacro\theorembodyfont
% The choice of the font for the theorem body is completely
% independent of the chosen |\theoremstyle|; this has proven to
% be very advantageous.  For example,
% \begin{verbatim}
%   {\theorembodyfont{\upshape}          \newtheorem{Rem}{Remark}}
%\end{verbatim}
% defines a theorem set {\envfont Rem}, which will be set in
% |\upshape| in the current layout (which in our example is
% \texttt{plain}).  As with |\theoremstyle|, the |\theorembodyfont|
% chosen is that current at the time of |\newtheorem|.  If
% |\theorembodyfont| is not specified or one defines
% |\theorembodyfont{}|, then the font used will be that defined by the
% |\theoremstyle|.
%
% \DescribeMacro\theoremheaderfont It is also possible to customize
% the font used for the theorem headers.  This is, however, a global
% declaration, and therefore there should be at most one
% |\theoremheaderfont| declaration in the preamble.\footnote{If it is
% actually necessary to have different header fonts, one has to define
% new theorem styles (substituting the desired font) or specify the
% information directly in the \texttt{\bslash newtheorem} declaration
% (the unclean variant).}
%
% \DescribeMacro\theorempreskipamount
% \DescribeMacro\theorempostskipamount Two additional parameters
% affect the vertical space around the theorem environments:
% |\theorempreskipamount| and |\theorempostskipamount| define,
% respectively, the spacing before and after such an environment.
% These parameters apply for all theorem sets and can be manipulated
% with the ordinary length macros.  They are rubber lengths
% (`\textsf{skips}'), and therefore can contain \texttt{plus} and
% \texttt{minus} parts.
%
% Since the definition of theorem sets should---most sensibly---be
% placed in the preamble, we only allow installation there.
% It is therefore possible to release the memory used here after
% |\begin{document}|, in order to make room for other
% applications.
%
%
% \subsection{Existing theorem styles}
%
% The following theorem styles exist to date:
% \begin{deflist}{marginbreak}
%    \item[plain]
%       This theorem style emulates the original \LaTeX{} definition,
%       except that additionally the parameters
%       |\theorem...skipamount| are used.
%    \item[break]
%       In this style, the theorem header is followed by a line
%       break.
%    \item[marginbreak]
%       The theorem number is set in the margin, and there is a line
%       break as in \texttt{break}.
%    \item[changebreak]
%       Like \texttt{break}, but with header number and text
%       interchanged.
%    \item[change]
%       Header number and text are interchanged, without a line break.
%    \item[margin]
%       The number is set in the left margin, without a line break.
% \end{deflist}
% All styles (except \texttt{plain}) select |\slshape| as the default
% |\theorembodyfont|.
%
%
% \subsection{Examples}
%
% Given the above theorem sets {\envfont Cor}, {\envfont Exa}
% and {\envfont Rem},
% suppose that the preamble also contains the declarations:
% \begin{verbatim}
%    \theoremstyle{marginbreak}   \newtheorem{Lem}[Cor]{Lemma}
%    \theoremstyle{change}
%    \theorembodyfont{\itshape}        \newtheorem{Def}[Cor]{Definition}
%
%    \theoremheaderfont{\scshape}
%\end{verbatim}
% Then the following are some typical examples of the typeset output
% resulting from their use.
%
% \begin{Cor}
%    This is a sentence typeset in the theorem environment {\envfont
%    Cor}.
% \end{Cor}
%
% \begin{Exa}
%    This is a sentence typeset in the theorem environment {\envfont
%    Exa}.
% \end{Exa}
%
% \begin{Rem}
%    This is a sentence typeset in the theorem environment {\envfont
%    Rem}.
% \end{Rem}
%
% \begin{Lem}[Ben User]
%    This is a sentence typeset in the theorem environment {\envfont
%    Lem}.
% \end{Lem}
%
% \begin{Def}[Very impressive Definition]
%    This is a sentence typeset in the theorem environment {\envfont
%    Def}.
% \end{Def}
%
% The last two examples show the effect of the optional argument to
% a theorem environment (it is the text typeset in parentheses).
%
%
% \section{Special Considerations}
% \label{sec:newfonts}
%
% Theoremheader and body are implemented as a unit. This means that
% the |\theoremheaderfont|  will inherit characteristics of the
% |\theorembodyfont| in \LaTeXe. Thus, if for example
% |\theorembodyfont| is |\itshape| and |\theoremheaderfont|
% is |\bfseries| the font selected for the header will have the
% characteristics `bold extended italic'. If this is not desired one
% should set the |\theoremheaderfont| to something like
% \begin{verbatim}
%      \theoremheaderfont{\normalfont\bfseries}
%\end{verbatim}
% i.e.\ supplying all necessary font information explicitly.
%
%
%
% \section{Acknowledgements}
%
% The publication of this set of macros was only possible with the
% help of Christina Busse (translating the manuscript into English),
% Joachim Pense (playing the r\^ole of typist), Chris Rowley
% (looking everything over) and many others providing useful
% suggestions.
%
% \MaybeStop{%
%    \begin{thebibliography}{9}
%       \bibitem{bk:GMS94} \textsc{M.~Goossens}, \textsc{F.~Mittelbach}
%         and \textsc{A.~Samarin}.
%         \newblock The \LaTeX{} Companion.
%         \newblock
%         Addison-Wesley, Reading, Massachusetts, 1994.
%       \bibitem{latex:source} \textsc{Lamport, Leslie.} \newblock
%                               \texttt{latex.tex}, version 2.09, date
%                               Feb. 1990.
%    \end{thebibliography}
%    \PrintIndex
%    \PrintChanges}
%
%
% \section{The documentation driver file}
%
% The next bit of code contains the documentation driver file for
% \TeX{}, i.e., the file that will produce the documentation you are
% currently reading. It will be extracted from this file by the
% \texttt{docstrip} program. Since it is the first code in the file
% one can alternatively process this file directly with \LaTeXe{} to
% obtain the documentation.
%
%    \begin{macrocode}
%<*driver>
\documentclass{ltxdoc}

\usepackage{theorem}

%<+driver>% The next few lines define theorem sets which are used
%<+driver>% in the example section of the documentation.

\theoremstyle{break}       \newtheorem{Cor}{Corollary}
\theoremstyle{plain}       \newtheorem{Exa}{Example}[section]
{\theorembodyfont{\upshape}\newtheorem{Rem}{Remark}}
\theoremstyle{marginbreak} \newtheorem{Lem}[Cor]{Lemma}
\theoremstyle{change}
\theorembodyfont{\itshape} \newtheorem{Def}[Cor]{Definition}

\theoremheaderfont{\scshape}

\RecordChanges

\begin{document}
   \DocInput{theorem.dtx}
\end{document}
%</driver>
%    \end{macrocode}
%
% \section{Definition of the Macros}
%
% If the file has been loaded before, we abort immediately. If not the
% package announces itself (this is actually done at the very top of
% the file---the way it is done isn't good style so don't copy it).
% \changes{v2.0e}{89/07/19}{Spaces removed from `typeout'.}
% \changes{v2.2b}{95/11/19}{Announce moved to top of file}
%    \begin{macrocode}
%<*package>
%\@ifundefined{theorem@style}{}{\endinput}
%\def\FMithmInfo{1995/11/19 v2.2b Theorem extension package (FMi)}
%\ProvidesPackage{theorem}[\FMithmInfo]
%    \end{macrocode}
%
%
% \subsection{Definition of theorem styles and fonts}
%
% All the definitions in this file are done globally to allow
% inputting this file inside a group.
%
% \begin{macro}{\theoremstyle}
%    Before a theorem style can be installed, the chosen style must be
%    known. For that reason, we must test to see that
%    |\th@|\meta{style} is known or, more precisely, that it is
%    different from |\relax|. If the style is not known then
%    |\th@plain| is used.
%    \begin{macrocode}
\gdef\theoremstyle#1{%
   \@ifundefined{th@#1}{\@warning
          {Unknown theoremstyle `#1'. Using `plain'}%
          \theorem@style{plain}}%
%    \end{macrocode}
%    We save the theorem style to be used in the token register
%    |\theorem@style|.
%    \begin{macrocode}
      {\theorem@style{#1}}%
%    \end{macrocode}
%    Now we ``evaluate'' the theorem style: this means, we call the
%    macro |\th@|\meta{style} which will activate the relevant
%    definitions which are contained in a separate file.  This is done
%    in a group to suppress changes to the current font.  This could
%    otherwise pose problems together with the new font selection
%    scheme\footnote{When I printed the original article using the new
%    font selection scheme I ended with a document with slanted
%    typefaces (text headings and all) simply because one of the
%    theorem styles used \texttt{\bslash sl} at toplevel.} if the
%    |\th@|\meta{style} is evaluated a second time.
% \changes{v2.1a}{90/02/15}{Group added around `th@..}
% \changes{v2.1a}{90/02/15}{Reseting of `..@begintheorem removed.}
%    \begin{macrocode}
      \begingroup
        \csname th@\the\theorem@style \endcsname
      \endgroup}
%    \end{macrocode}
% \end{macro}
%
%
% \begin{macro}{\@begintheorem}
% \changes{v2.1a}{90/02/15}{Relaxed at toplevel.}
% \begin{macro}{\@opargbegintheorem}
% \changes{v2.1a}{90/02/15}{Relaxed at toplevel.}
%    We reset |\@begintheorem| and |\@opargbegintheorem| to
%    |\relax| since these commands are no longer necessary at
%    toplevel. This will save a few tokens.
%    \begin{macrocode}
\global\let\@begintheorem\relax
\global\let\@opargbegintheorem\relax
%    \end{macrocode}
% \end{macro}
% \end{macro}
%
% \begin{macro}{\theorem@style}
%    Obviously the \textsf{token} register used above has to be
%    allocated. To assure the utmost compatibility with the
%    original \LaTeX{} definition, we set the default theorem style to
%    \texttt{plain}, which implements the usual \LaTeX{} convention.
% \changes{v2.0f}{89/07/20}{Unnecessary `global' removed.}
%    \begin{macrocode}
\newtoks\theorem@style
\global\theorem@style{plain}
%    \end{macrocode}
% \end{macro}
%
% \begin{macro}{\theorembodyfont}
% \begin{macro}{\theorem@bodyfont}
%    If the |\theorembodyfont| is set by the user then it should not
%    interact with the default font set in the theorem style. When the
%    new font selection is in force this may happen if, for example,
%    the default is |\itshape| and the new |\theorembodyfont| is
%    |\sffamily|. So we add a |\reset@font| command in front of the
%    user definition.
% \changes{v2.0f}{89/07/20}{Unnecessary `global' removed.}
% \changes{v2.0g}{89/09/19}{`theoremfont to `theorembodyfont.}
% \changes{v2.1c}{92/06/22}{`reset@font added and logic changed.}
% \changes{v2.1d}{92/10/28}{Reset to empty if arg is empty}
%    \begin{macrocode}
\gdef\theorembodyfont#1{%
%    \end{macrocode}
%    We check if the argument supplied is empty and if so put nothing
%    into the |\theorem@bodyfont| token register to allow for
%    |\theorembodyfont{}| as a mean of using the default of the
%    current |\theoremstyle|.
%    \begin{macrocode}
   \def\@tempa{#1}%
   \ifx\@tempa\@empty
    \theorem@bodyfont{}%
   \else
    \theorem@bodyfont{\reset@font#1}%
   \fi
   }
\newtoks\theorem@bodyfont
\global\theorem@bodyfont{}
%    \end{macrocode}
% \end{macro}
% \end{macro}
%
%
% \begin{macro}{\theoremheaderfont}
% \changes{v2.0g}{89/09/19}{Macro added.}
%    The font for the theorem headers is handled differently because
%    this definition applies to all theorem styles.
%    \begin{macrocode}
\gdef\theoremheaderfont#1{\gdef\theorem@headerfont{#1}%
%    \end{macrocode}
%    After using the macro once it is redefined to produce an error
%    message.
%    \begin{macrocode}
       \gdef\theoremheaderfont##1{%
        \typeout{\string\theoremheaderfont\space should be used
                 only once.}}}
%    \end{macrocode}
% \end{macro}
%
% \begin{macro}{\theorem@headerfont}
% \changes{v2.0g}{89/09/19}{Macro added.}
%    To set the |\theorem@headerfont| default we first test if
%    the new fontselection scheme is in force.
% \changes{v2.1a}{90/02/15}{Support for new font selection.}
%    \begin{macrocode}
\ifx\upshape\undefined
%    \end{macrocode}
%    If not we define it to expand into |\bfseries|. We don't use
%    |\let| just in case a following style option redefines this
%    macro.
%    \begin{macrocode}
\gdef\theorem@headerfont{\bfseries}
%    \end{macrocode}
%    Otherwise we reset the current shape before calling |\bfseries|.
%    \begin{macrocode}
\else \gdef\theorem@headerfont{\normalfont\bfseries}\fi
%    \end{macrocode}
% \end{macro}
%
% \begin{macro}{\th@plain}
% \begin{macro}{\th@break}
% \begin{macro}{\th@marginbreak}
% \begin{macro}{\th@changebreak}
% \begin{macro}{\th@change}
% \begin{macro}{\th@margin}
%    The different styles are defined in macros such as
%    |\th@plain|.  Since memory space is precious in
%    ``non-Big-versions'', we have to avoid offering too many unused
%    definitions.  Therefore we define these styles in separate files
%    that can be loaded on demand.  Thus the commands themselves only
%    load these files. We use |\@input@|, a \LaTeXe{} internal command
%    that ensures that the file will be listed with |\listfiles|.
% \changes{v2.2a}{94/02/02}{Use `@input@ to load theorem layout files}
%    \begin{macrocode}
\gdef\th@plain{\@input@{thp.sty}}
\gdef\th@break{\@input@{thb.sty}}
\gdef\th@marginbreak{\@input@{thmb.sty}}
\gdef\th@changebreak{\@input@{thcb.sty}}
\gdef\th@change{\@input@{thc.sty}}
\gdef\th@margin{\@input@{thm.sty}}
%    \end{macrocode}
%    This list will be expanded when new styles become
%    available.  For testing,  just append new theorem substyles
%    as document options.
% \end{macro}
% \end{macro}
% \end{macro}
% \end{macro}
% \end{macro}
% \end{macro}
%
%
%
% \subsection{Definition of a new theorem set}
%
% As already pointed out, a new theorem environment can be defined in
% three different ways:
% \begin{verbatim}
%    \newtheorem{Lem}{Lemma}
%    \newtheorem{Lem}{Lemma}[section]
%    \newtheorem{Lem}[Theorem]{Lemma}
%\end{verbatim}
% The function of the macro |\newtheorem| is to recognize these cases
% and then to branch into one of the three macros |\@ynthm|, |\@xnthm|
% or |\@othm|.  This mechanism is adopted unchanged from
% \cite{latex:source}; the essential point here is that, for example,
% in the second case, the arguments \texttt{Lem}, \texttt{Lemma} and
% \texttt{section} are passed over to the macro |\@xnthm|.
%
% We inspect this case first because the others present fewer problems,
% and thus are easily derived from this one.
%
% \begin{macro}{\@xnthm}
%    For our example arguments, the macro |\@xnthm| must fulfill
%    the following:
%    \begin{itemize}
%       \item
%          Define a new \LaTeX-counter `Lem'
%       \item
%          reset this counter within a |\section|
%       \item
%          define the macro |\theLem|
%       \item
%          define the environment macros |\Lem| and
%          |\endLem| using the current
%          |\theoremstyle| and |\theorem@bodyfont|.
%    \end{itemize}
%    Obviously, all this should happen only if the first argument
%    of |\@xnthm| (i.e.\ \texttt{Lem} in our example) is chosen so
%    as not to conflict with any previously defined commands or
%    environments.  This test is performed by the \LaTeX{} macro
%    |\@ifdefinable|.
%    \begin{macrocode}
\gdef\@xnthm#1#2[#3]{\expandafter\@ifdefinable\csname #1\endcsname
%    \end{macrocode}
%    Therefore, the first argument of |\@ifdefinable| is the
%    expansion (in the example, |\Lem|) of
%    |\csname||#1\endcsname|.  The second argument is
%    executed only if the test has been completed successfully.
%    \begin{macrocode}
   {%
%    \end{macrocode}
%    Now we define the new counter. The names of the \LaTeX{}
%    macros employed should speak for themselves:
% \changes{v2.2b}{1995/10/07}{Use \cs{@newctr}}
% \changes{v2.2c}{1995/11/23}{Use \cs{@newctr} properly.}
%    \begin{macrocode}
    \@definecounter{#1}\@newctr{#1}[#3]%
%    \end{macrocode}
%    Using |\@newctr| will give a proper error message if the counter
%    in |#3| is not defined.
%    In defining `|\theLem|' we must generate the desired macro
%    name by use of |\expandafter| and |\csname|.
%    \begin{macrocode}
    \expandafter\xdef\csname the#1\endcsname
%    \end{macrocode}
%    An |\xdef| is used in order to make the definition global,
%    and to ensure that it contains the replacement texts of
%    |\@thmcountersep| and |\@thmcounter|.\footnote{These
%    two macros can be defined by the document style.  Their default
%    values produce a `\texttt{.}' as separation and an arabic
%    representation of the number.} However, not everything should be
%    expanded. For example, it saves space to use |\thesection|
%    instead of its---at times---lengthy expansion.
%    \begin{macrocode}
      {\expandafter \noexpand \csname the#3\endcsname
       \@thmcountersep \@thmcounter{#1}}%
%    \end{macrocode}
%    Thus with the defaults of \LaTeX{}, |\theLem| would be
%    replaced by the command sequence |\thesection.\arabic{Lem}|.
%
%    We will now look at the definition of the macro which is executed
%    at the beginning of the actual environment (in our example this
%    macro is |\Lem|). It should be noted that we use an
%    ``|\expandafter| trick'' to expand only certain parts of the
%    replacement text at the time of the definition.
%    \begin{macrocode}
    \def\@tempa{\global\@namedef{#1}}%
    \expandafter \@tempa \expandafter{%
%    \end{macrocode}
%    First, the macro that contains the current definitions of
%    |\@begintheorem| and |\@opargtheorem| should be called
%    up.  The name of this macro---as is already known---has the form
%    |\th@|\meta{theorem\ style}; therefore, it must be called by
%    \begin{macrocode}
      \csname th@\the \theorem@style
%    \end{macrocode}
%    In addition the default theorem font should be changeable, i.e.\
%    we have to insert the contents of |\theorem@bodyfont|. For
%    that reason, we expand even further, beyond |\endcsname|,
%    and thus insert the contents of the \textsf{token} register
%    |\theorem@bodyfont| in the replacement text.
%    \begin{macrocode}
            \expandafter \endcsname \the \theorem@bodyfont
%    \end{macrocode}
%    Now it is time to call the macro |\@thm| which takes over
%    the further processing.  It has two arguments: the current
%    counter name (in our example, \texttt{Lem}), and the text of the
%    label (in our example, \texttt{Lemma}).
%    \begin{macrocode}
     \@thm{#1}{#2}}%
%    \end{macrocode}
%    With this, the `sub-definition' is complete.  The macro
%    |\@endtheorem| ends a theorem environment and is, so far, nothing
%    but an |\endtrivlist|.  (Hence it is defined globally, and not
%    within the theorem styles.\footnote{This has to be changed as
%    soon as theorem styles that change \texttt{\bslash @endtheorem}
%    exist.  In such a case, all existing styles must be changed as
%    well since they will have to reset the macro.}) Therefore, we can
%    set it equivalent to the macro that ends the theorem set (in our
%    example, |\endLem|).  However, if some day theorem styles exist
%    that do change |\@endtheorem|, we would have to use the
%    commented-out line instead.
%    \begin{macrocode}
    \global \expandafter \let \csname end#1\endcsname \@endtheorem
%   \global\@namedef{end#1}{\@endtheorem}%
%    \end{macrocode}
%    With these commands all the required definitions are employed,
%    unless the test |\@ifdefinable| has failed.  Therefore, we
%    end the second argument of this macro and with it the definition
%    of |\@xnthm|.
%    \begin{macrocode}
   }}
%    \end{macrocode}
% \end{macro}
%
% \begin{macro}{\@ynthm}
%    The definition of |\@ynthm| is completely analogous. In
%    this case the new counter that is defined is not reset within
%    another counter; thus the definition of |\the...|
%    is simplified:
%    \begin{macrocode}
\gdef\@ynthm#1#2{\expandafter\@ifdefinable\csname #1\endcsname
   {\@definecounter{#1}%
    \expandafter\xdef\csname the#1\endcsname{\@thmcounter{#1}}%
%    \end{macrocode}
%    The rest of the definition corresponds literally to that of
%    |\@xnthm|:
%    \begin{macrocode}
    \def\@tempa{\global\@namedef{#1}}\expandafter \@tempa
     \expandafter{\csname th@\the \theorem@style \expandafter
     \endcsname \the\theorem@bodyfont \@thm{#1}{#2}}%
    \global \expandafter \let \csname end#1\endcsname \@endtheorem}}
%    \end{macrocode}
% \end{macro}
%
% \begin{macro}{\@othm}
%    The definition of |\@othm| does not contain anything new.
%    \begin{macrocode}
\gdef\@othm#1[#2]#3{%
%    \end{macrocode}
% \changes{v2.2b}{95/10/07}{Build in check for counter defined}
%    We do not define a new counter but instead use one that has
%    already been defined.  Thus the only definition we need is that
%    of this pseudo-counter (i.e.\ |\the|\meta{env.\ name}).
%    First we check if |#2| corresponds to a known counter name.
%    \begin{macrocode}
  \expandafter\ifx\csname c@#2\endcsname\relax
   \@nocounterr{#2}%
  \else
   \expandafter\@ifdefinable\csname #1\endcsname
   {\expandafter \xdef \csname the#1\endcsname
     {\expandafter \noexpand \csname the#2\endcsname}%
%    \end{macrocode}
%    All other parts of the definition can be adopted from
%    |\@xnthm|. We have to remember, though, that in this case
%    the name of the current counter and the theorem label have moved
%    to the second and third arguments.
%    \begin{macrocode}
    \def\@tempa{\global\@namedef{#1}}\expandafter \@tempa
     \expandafter{\csname th@\the \theorem@style \expandafter
     \endcsname \the\theorem@bodyfont \@thm{#2}{#3}}%
    \global \expandafter \let \csname end#1\endcsname \@endtheorem}%
  \fi}
%    \end{macrocode}
% \end{macro}
%
% \subsection{Macros that are employed in a {\envfont theorem}
%    environment}
%
% \begin{macro}{\@thm}
%    The macro |\@thm| has to increase the current counter.  Then,
%    depending on whether the environment has (or does not have) an
%    optional argument, it has to branch into either |\@begintheorem|
%    or |\@opargtheorem|.
%    \begin{macrocode}
\gdef\@thm#1#2{\refstepcounter{#1}%
%    \end{macrocode}
%    Now we start a {\envfont trivlist} environment, and give
%    |\@topsep| and |\@topsepadd| the values of the \textsf{skip}
%    registers |\theorempreskipamount| and |\theorempostskipamount|.
%    The value in |\@topsep| is the vertical space that is inserted by
%    the first (and only) |\item| in our |\trivlist| whilst
%    |\@topsepadd| is inserted by |\@endparenv| at the end of that
%    {\envfont trivlist} environment.  By using these registers, we
%    obtain the desired space around a {\envfont theorem} environment.
%    \begin{macrocode}
   \trivlist
   \@topsep \theorempreskipamount               % used by first \item
   \@topsepadd \theorempostskipamount           % used by \@endparenv
%    \end{macrocode}
%    Now we have to test whether an optional argument has been given.
% \changes{v2.0d}{89/07/18}{Forgotten percent added.}
%    \begin{macrocode}
   \@ifnextchar [%
%    \end{macrocode}
%    If there is an optional argument, we will call |\@ythm|, and
%    move the arguments read back into the input stream.
%    \begin{macrocode}
   {\@ythm{#1}{#2}}%
%    \end{macrocode}
%    If not, we call |\@begintheorem|. Its first argument is the
%    name of the theorem set (hence the second argument of
%    |\@thm|). Its second argument is the macro that produces the
%    current number.
%    \begin{macrocode}
   {\@begintheorem{#2}{\csname the#1\endcsname}\ignorespaces}}
%    \end{macrocode}
% \end{macro}
%
% \begin{macro}{\@xthm}
% \begin{macro}{\@ythm}
%    Both these macros were originally called by |\@thm|.  We do
%    not need |\@xthm| anymore, hence we reset it to
%    |\relax|. The definition of |\@ythm| has not changed at
%    all from its definition in \LaTeX. In order to make the macros
%    easier to understand, we will
%    nevertheless present it (commented out).
%    \begin{macrocode}
\global\let\@xthm\relax
% \def\@ythm#1#2[#3]{\@opargbegintheorem{#2}{\csname
%           the#1\endcsname}{#3}\ignorespaces}
%</package>
%    \end{macrocode}
%    The primitive |\ignorespaces| in |\@ythm| and
%    |\@thm| is needed to remove the spaces between the
%    |\begin{...}| and the actual text.
% \end{macro}
% \end{macro}
%
%
% \subsection{Definition of the theorem substyles}
%
% As already pointed out, the theorem substyles, defined below, are
% only loaded when necessary.  Note that all these substyles, except
% \texttt{plain}, have |\slshape| as the default body font.
%
% \subsubsection{The \texttt{plain} style}
%
% \label{dok:th@plain} As the following macros use |@|, we have
% to locally set the |\catcode| of this symbol to ``letter''.
% This happens within a group, so that we do not have to worry about
% which |\catcode| that symbol had before.
%    \begin{macrocode}
%<*thp>
\begingroup \makeatletter
%    \end{macrocode}
% Since we are now within a group, we must make all definitions
% globally.  First we make sure that \texttt{theorem.sty} is loaded.
% This will allow us to use this file as a document style option
% without having to call \texttt{theorem} itself as an option.  At the
% same time, we assure that at least version 2 is loaded, since
% |\theorem@style| was not defined in earlier versions.
%    \begin{macrocode}
\@ifundefined{theorem@style}{\input{theorem.sty}}{}
\ProvidesFile{thp.sty}
             [\FMithmInfo]
%    \end{macrocode}
%
%
% \begin{macro}{\th@plain}
% \begin{macro}{\@begintheorem}
% \begin{macro}{\@opargbegintheorem}
%    |\theoremstyle{plain}| corresponds to the original
%    definition, except that the distances to the surrounding text are
%    determined by the parameters |\theorempreskipamount| and
%    |\theorempostskipamount|.
%    First we set the default body font.
%    \begin{macrocode}
\gdef\th@plain{\normalfont\itshape
%    \end{macrocode}
%    Then we define |\@begintheorem| and
%    |\@opargbegintheorem|. These two macros define how the
%    header of a theorem is typeset.  |\@opargbegintheorem| will
%    be called if a \textsf{theorem} environment with an optional
%    argument is encountered; otherwise, the header is constructed by
%    calling |\@begintheorem|.  If one of these macros is
%    executed, we are within a \textsf{trivlist} environment started by
%    |\@thm|.  So the theorem header is produced with an
%    |\item| command.
%
%    Instead of specifying the header font directly, all standard
%    theorem styles use the |\theorem@headerfont| macro to allow
%    customization.  The extra space (|\labelsep|) is necessary
%    because of problems in the \textsf{trivlist} environment.
%    \begin{macrocode}
  \def\@begintheorem##1##2{%
        \item[\hskip\labelsep \theorem@headerfont ##1\ ##2]}%
%    \end{macrocode}
%    The definition of |\@opargbegintheorem| is completely
%    analogous. The only difference is the fact that there exists a
%    third argument (which is the optional parameter of the
%    environment and contains additional information about the
%    theorem).  Customarily we enclose it in parentheses.
%    \begin{macrocode}
\def\@opargbegintheorem##1##2##3{%
   \item[\hskip\labelsep \theorem@headerfont ##1\ ##2\ (##3)]}}
%    \end{macrocode}
% \end{macro}
% \end{macro}
% \end{macro}
%
% \noindent
% We conclude with an |\endgroup| to restore the
% |\catcode| of |@|.
%    \begin{macrocode}
\endgroup
%</thp>
%    \end{macrocode}
%
%
% \subsubsection{The \texttt{break} style}
%
% This style option is stored in the file |thb.sty|.
% For the next two lines see the documentation for |\th@plain| on
% page~\pageref{dok:th@plain}.
%    \begin{macrocode}
%<*thb>
\begingroup \makeatletter
\@ifundefined{theorem@style}{\input{theorem.sty}}{}
\ProvidesFile{thb.sty}
             [\FMithmInfo]
%    \end{macrocode}
%
% \begin{macro}{\th@break}
%    |\theoremstyle{break}| produces a line break after the name
%    of the theorem.  The font is |\slshape|.  Hence, we define
%    |\th@break| as follows:
%    \begin{macrocode}
\gdef\th@break{\normalfont\slshape
  \def\@begintheorem##1##2{\item[%
%    \end{macrocode}
%    We run into the following problem: it is not possible to create
%    the header with |\item[|\meta{title}|]| and then start
%    a new line by, for example, |\mbox{}\\|. Such a definition
%    will fail whenever a list environment follows immediately.  With
%    the above construction, the |\mbox{}| causes the switch
%    |@inlabel| (cf.\ definition of |\list| and
%    |\trivlist| in \cite{latex:source}) to be set to \texttt{false}
%    and so the following list will insert additional vertical
%    space (|\topskip|). This is quite annoying. Therefore, we
%    create the line break within the |\item|. In order to ensure
%    that the text will begin at the proper position in the following
%    line, we simply pretend that the label does not take any
%    room.\footnote{This will lead to problems whenever very high
%    symbols occurring in the line tower into the
%    heading. So, something else has to be done here sometime.}
%    \begin{macrocode}
       \rlap{\vbox{\hbox{\hskip \labelsep\theorem@headerfont ##1\ ##2}%
                   \hbox{\strut}}}]}%
%    \end{macrocode}
%    Again, the definition of |\@opargbegintheorem| is
%    completely analogous.
%    \begin{macrocode}
\def\@opargbegintheorem##1##2##3{%
  \item[\rlap{\vbox{\hbox{\hskip \labelsep \theorem@headerfont
                     ##1\ ##2\ (##3)}%
                    \hbox{\strut}}}]}}
\endgroup
%</thb>
%    \end{macrocode}
% \end{macro}
%
%
% \subsubsection{The \texttt{changebreak} style}
%
%    \begin{macrocode}
% This style option is stored in the file |thcb.sty|.
%    \begin{macrocode}
%<*thcb>
\begingroup \makeatletter
\@ifundefined{theorem@style}{\input{theorem.sty}}{}
\ProvidesFile{thcb.sty}
             [\FMithmInfo]
%    \end{macrocode}
%
% \begin{macro}{\th@changebreak}
%    The \texttt{change-break} theorem style is like \texttt{break}
%    but with interchange of theorem name and theorem number.  Thus we
%    define |\th@changebreak| as follows:
%    \begin{macrocode}
\gdef\th@changebreak{\normalfont\slshape
  \def\@begintheorem##1##2{\item
       [\rlap{\vbox{\hbox{\hskip\labelsep \theorem@headerfont ##2\ ##1}%
                    \hbox{\strut}}}]}%
\def\@opargbegintheorem##1##2##3{%
  \item[\rlap{\vbox{\hbox{\hskip\labelsep \theorem@headerfont
                     ##2\ ##1\ (##3)}%
                    \hbox{\strut}}}]}}
\endgroup
%</thcb>
%    \end{macrocode}
% \end{macro}
%
%
% \subsubsection{The \texttt{change} style}
%
% This style option is stored in the file |thc.sty|.
%    \begin{macrocode}
%<*thc>
\begingroup \makeatletter
\@ifundefined{theorem@style}{\input{theorem.sty}}{}
\ProvidesFile{thc.sty}
             [\FMithmInfo]
%    \end{macrocode}
%
% \begin{macro}{\th@change}
%    The \texttt{change} theorem style corresponds to the
%    \texttt{change break} style without a linebreak after the header.
%    To say it in another way, it's the same as the \texttt{plain}
%    style but with number and name interchanged and |\slshape| as the
%    default font.
%    \begin{macrocode}
\gdef\th@change{\normalfont\slshape
  \def\@begintheorem##1##2{\item
       [\hskip\labelsep \theorem@headerfont ##2\ ##1]}%
\def\@opargbegintheorem##1##2##3{%
  \item[\hskip\labelsep \theorem@headerfont ##2\ ##1\ (##3)]}}
\endgroup
%</thc>
%    \end{macrocode}
% \end{macro}
%
%
% \subsubsection{The \texttt{marginbreak} style}
%
% This style option is the one used most often at Mainz.  It is saved
% in the file |thmb.sty|.
%    \begin{macrocode}
%<*thmb>
\begingroup \makeatletter
\@ifundefined{theorem@style}{\input{theorem.sty}}{}
\ProvidesFile{thmb.sty}
             [\FMithmInfo]
%    \end{macrocode}
%
% \begin{macro}{\th@marginbreak}
%    The \texttt{margin break} style is nearly the same as the
%    \texttt{change break} style. The only difference is the placement
%    of the theorem number.  We use |\llap| to place it in the left
%    margin.
%
%    In this style |\labelsep| denotes the separation between the
%    number and the text.
%    \begin{macrocode}
\gdef\th@marginbreak{\normalfont\slshape
  \def\@begintheorem##1##2{\item
       [\rlap{\vbox{\theorem@headerfont
                    \hbox{\llap{##2}\hskip\labelsep ##1}%
                    \hbox{\strut}}}]}%
\def\@opargbegintheorem##1##2##3{%
  \item[\rlap{\vbox{\theorem@headerfont
                    \hbox{\llap{##2}\hskip\labelsep ##1\ (##3)}%
                    \hbox{\strut}}}]}}
\endgroup
%</thmb>
%    \end{macrocode}
% \changes{v2.1c}{90/02/17}{`endgroup outside macrocode env.}
% \end{macro}
%
%
%
% \subsubsection{The \texttt{margin} style}
%
% This style option is stored in the file |thm.sty|.
%    \begin{macrocode}
%<*thm>
\begingroup \makeatletter
\@ifundefined{theorem@style}{\input{theorem.sty}}{}
\ProvidesFile{thm.sty}
             [\FMithmInfo]
%    \end{macrocode}
%
% \begin{macro}{\th@margin}
%    Again this is only a variant of the theorem styles described
%    above without any new ideas.
%    \begin{macrocode}
\gdef\th@margin{\normalfont\slshape
  \def\@begintheorem##1##2{\item
       [\theorem@headerfont \llap{##2}\hskip\labelsep ##1]}%
\def\@opargbegintheorem##1##2##3{%
  \item[\theorem@headerfont \llap{##2}\hskip\labelsep ##1\ (##3)]}}
\endgroup
%</thm>
%    \end{macrocode}
% \end{macro}
%
%
%
%
% \subsection{Final Definitions}
%
% \begin{macro}{\theorempreskipamount}
% \begin{macro}{\theorempostskipamount}
%    The \textsf{skip} parameters that regulate the vertical empty space
%    before and after the {\envfont theorem} environment have
%    to be allocated as well.
% \changes{v2.0f}{89/07/20}{Unnecessary `global' removed.}
%    \begin{macrocode}
%<*package>
\newskip\theorempreskipamount
\newskip\theorempostskipamount
%    \end{macrocode}
%    Since we have used the same values for all theorem sets, we now can
%    assign them.
%    \begin{macrocode}
\global\setlength\theorempreskipamount{12pt plus 5pt minus 3pt}
\global\setlength\theorempostskipamount{8pt plus 3pt minus 1.5pt}
%    \end{macrocode}
% \end{macro}
% \end{macro}
%
%
% \begin{macro}{\@endtheorem}
%    The same holds for the macro |\@endtheorem|, which ends a
%    {\envfont theorem} environment. Since it is the same for all
%    theorem sets, it is removed from the macros |\th@|\meta{style}.
%    It simply ends the {\envfont trivlist} environment, which was
%    begun in |\@thm|.
%    \begin{macrocode}
\global\let\@endtheorem=\endtrivlist
%    \end{macrocode}
% \end{macro}
%
%
%
% \begin{macro}{\@preamblecmds}
%    All macros defined above are to be used only in the preamble.
%    Therefore, we insert them in |\@preamblecmds| which will disable
%    them at begin document. This is done by the internal \LaTeXe{}
%    command |\@onlypreamble|.
% \changes{v2.0g}{89/09/19}{`theoremheaderfont and `theorembodyfont
%    added.}
% \changes{v2.2a}{94/02/03}{Use \LaTeXe{} interface}
%    \begin{macrocode}
\@onlypreamble\@xnthm
\@onlypreamble\@ynthm
\@onlypreamble\@othm
\@onlypreamble\newtheorem
\@onlypreamble\theoremstyle
\@onlypreamble\theorembodyfont
\@onlypreamble\theoremheaderfont
%    \end{macrocode}
% \end{macro}
%
% Finally we declare the \texttt{plain} theorem style to be the default.
% \changes{v2.1b}{90/06/04}{Fixed missing `item by declaring default
%    style.}
%    \begin{macrocode}
\theoremstyle{plain}
%</package>
%    \end{macrocode}
%
% \Finale

\endinput
}{}
\ProvidesFile{thmb.sty}
             [\FMithmInfo]
%    \end{macrocode}
%
% \begin{macro}{\th@marginbreak}
%    The \texttt{margin break} style is nearly the same as the
%    \texttt{change break} style. The only difference is the placement
%    of the theorem number.  We use |\llap| to place it in the left
%    margin.
%
%    In this style |\labelsep| denotes the separation between the
%    number and the text.
%    \begin{macrocode}
\gdef\th@marginbreak{\normalfont\slshape
  \def\@begintheorem##1##2{\item
       [\rlap{\vbox{\theorem@headerfont
                    \hbox{\llap{##2}\hskip\labelsep ##1}%
                    \hbox{\strut}}}]}%
\def\@opargbegintheorem##1##2##3{%
  \item[\rlap{\vbox{\theorem@headerfont
                    \hbox{\llap{##2}\hskip\labelsep ##1\ (##3)}%
                    \hbox{\strut}}}]}}
\endgroup
%</thmb>
%    \end{macrocode}
% \changes{v2.1c}{90/02/17}{`endgroup outside macrocode env.}
% \end{macro}
%
%
%
% \subsubsection{The \texttt{margin} style}
%
% This style option is stored in the file |thm.sty|.
%    \begin{macrocode}
%<*thm>
\begingroup \makeatletter
\@ifundefined{theorem@style}{% \iffalse meta-comment
%
% Copyright (C) 1993-2024
%
% The LaTeX Project and any individual authors listed elsewhere
% in this file.
%
% This file is part of the Standard LaTeX `Tools Bundle'.
% -------------------------------------------------------
%
% It may be distributed and/or modified under the
% conditions of the LaTeX Project Public License, either version 1.3c
% of this license or (at your option) any later version.
% The latest version of this license is in
%    https://www.latex-project.org/lppl.txt
% and version 1.3c or later is part of all distributions of LaTeX
% version 2005/12/01 or later.
%
% The list of all files belonging to the LaTeX `Tools Bundle' is
% given in the file `manifest.txt'.
%
% \fi
%
%%
%
%
\def\FMithmInfo{2023/07/05 v2.2c Theorem extension package (FMi)}
%
% \ProvidesFile{theorem.dtx}[\FMithmInfo]
%\iffalse   % this is a METACOMMENT !
%%
%% Package `theorem' to use with LaTeX2e
%% Copyright (C) 1989-1995 Frank Mittelbach, all rights reserved.
%<+package>\@ifundefined{theorem@style}{}{\endinput}
%<+package>\NeedsTeXFormat{LaTeX2e}[1995/06/01]
%<+package>\ProvidesPackage{theorem}
%<+package>         [\FMithmInfo]
%
% \fi
%
% \changes{v1.0}{87/05/07}{First draft.}
% \changes{v1.0a}{87/05/07}{`thistheoremstyle' added.}
% \changes{v1.1a}{87/07/29}{New concept: redefine the internal
%                          \protect\LaTeX{} commands `@xnthm' etc.}
% \changes{v1.1b}{87/07/31}{`theoremfont' used to specify font.}
% \changes{v1.2a}{88/01/08}{`@endtheorem' now globally defined.}
% \changes{v1.2b}{88/01/11}{`newtheorem' only allowed in the preamble.}
% \changes{v1.3a}{88/02/14}{`theoremfont' now a toks register.}
%
% \changes{v2.0a}{88/11/16}{New implementation and new documentation.}
% \changes{v2.0a}{88/11/16}{No layout changes but internal modifications
%                           old substyles won't work.}
% \changes{v2.0b}{89/01/30}{`global' used for all definitions.}
% \changes{v2.0c}{89/06/15}{Documentation translated to English.
%                            Thanks to LLL and Nina}
% \changes{v2.1a}{90/02/15}{Support for new font selection.}
% \changes{v2.2a}{94/02/03}{Cleanup for LaTeX2e}
%
% \DoNotIndex{\@endparenv,\@ifdefinable,\@ifnextchar,\@ifundefined}
% \DoNotIndex{\@namedef,\@nameuse,\@spaces,\@tempa,\@warning}
% \DoNotIndex{\ }
% \DoNotIndex{\begingroup,\bfseries}
% \DoNotIndex{\csname}
% \DoNotIndex{\def,\do,\docdate}
% \DoNotIndex{\endcsname,\endgroup,\endinput,\endtrivlist,\expandafter}
% \DoNotIndex{\filedate,\fileversion}
% \DoNotIndex{\gdef,\global}
% \DoNotIndex{\hbox,\hskip}
% \DoNotIndex{\ignorespaces,\input,\itshape,\item}
% \DoNotIndex{\labelsep,\let,\llap}
% \DoNotIndex{\makeatletter}
% \DoNotIndex{\noexpand}
% \DoNotIndex{\refstepcounter,\relax,\rlap}
% \DoNotIndex{\setlength,\slshape,\space,\string,\strut}
% \DoNotIndex{\the,\trivlist,\typeout}
% \DoNotIndex{\vbox}
% \DoNotIndex{\xdef}
%
% \def\deflabel#1{\ttfamily #1\hfill}
% \def\deflist#1{\begin{list}{}{\settowidth\labelwidth{\ttfamily #1}%
%                               \setlength\leftmargin\labelwidth
%                               \addtolength\leftmargin\labelsep
%                               \let\makelabel\deflabel}}
% \def\enddeflist{\end{list}}
% \def\envfont{\normalfont\ttfamily}
%
% \MakeShortVerb{\|}
%
% \newcommand{\TUB}{{\sl TUGboat\/}}
%
% \GetFileInfo{theorem.dtx}
% \title{An Extension of the
%        \LaTeX{} theorem environment\thanks{This file
%        has version number \fileversion, last
%        revised \filedate.}}
% \author{Frank Mittelbach}
%
% \MaintainedByLaTeXTeam{tools}
% \maketitle
%
% \begin{abstract}
%    The macros described in this paper yield an extension of the
%    \LaTeX{} theorem mechanism. It is designed to satisfy the
%    different requirements of various journals. Thus, the
%    layout of the ``theorems'' can be manipulated by determining a
%    ``style''. This article describes not only the use, but
%    also the definition, of the necessary macros.
% \end{abstract}
%
% \section*{Preface to version 2.2}
%
% For \LaTeXe{} this package did not need any fundamental changes. I
% only modified the messages generated so that theorem layout styles
% will show up with the |\listfiles| command and cleaned the section
% on the New Font Selection Scheme since this is now included in
% \LaTeX.
%
% \section*{Preface to version 2.1}
%
% This version is identical to 2.0g described in \TUB\ 10\#3 except
% for some internal defaults which are now set depending on the used
% font selection scheme.
%
% This was done to avoid unpleasant surprises if the new font selection
% scheme is in force. For further details see
% section~\ref{sec:newfonts} and \cite{bk:GMS94}.
%
% \section{Introduction}
%
% For our purposes here, ``theorems'' are labelled enunciations,
% often set off from the main text by extra space and a font change.
% Theorems, corollaries, conjectures, definitions, and remarks
% are all instances of ``theorems''.  The ``header'' of these structures
% is composed of a label (such as \textsc{Theorem} or \textsc{Remark})
% and a number which serializes an item in the sequence of items
% with the same label.
%
% Shortly after the introduction of \LaTeX{} at the Fachbereich
% Mathematik in Mainz, the desire to manipulate the layout of
% ``theorems'' arose.  In Mainz, the following two conventions came
% into general use:
% \begin{enumerate}
%   \item The number of the theorem is shown in the margin.
%   \item There is a line break at
%         the end of the theorem header.
% \end{enumerate}
%
% Additionally, some journals require different formats which depend on
% the ``sort of theorem'': e.g.\ often remarks and definitions are set
% in |\upshape|, while |\itshape| is employed for main theorems.
%
% Confronted with these requirements, a theorem environment was
% developed in Mainz which allows separate determination of the
% layout of the ``theorems sets'', comparable to |\pagestyle|.
%
%
% \section{The user interface}
% \subsection{Defining new theorem sets}
%
% \DescribeMacro\newtheorem
% As in the original \LaTeX{} version, the command |\newtheorem|
% defines a new ``theorem set'' or ``theorem-like structure''.
% Two required arguments name the new environment and give the text
% to be typeset with each instance of the new ``set'', while
% an optional argument determines how
% the ``set'' is enumerated:
% \begin{description}
%    \item[\ttfamily \bslash newtheorem\{foo\}\{bar\}]
%       The theorem set {\envfont foo} (whose name is \texttt{bar})
%       uses its own counter.
%    \item[\ttfamily \bslash newtheorem\{foo2\}{[foo]}\{bar2\}]
%       The theorem set {\envfont foo2} (printed name \texttt{bar2})
%       uses the same counter as the theorem set \texttt{foo}.
%    \item[\ttfamily \bslash newtheorem\{foo3\}\{bar3\}{[section]}]
%       The theorem set {\envfont foo3} (printed name \texttt{bar3}) is
%       enumerated within the counter \texttt{section}, i.e.\ with every
%       new |\section| the enumeration begins again with 1, and
%       the enumeration is composed from the section-number and the
%       theorem counter itself.
% \end{description}
%
% \DescribeMacro\theoremstyle
% Additionally, the command |\theoremstyle| can define the layout
% of various, or all, theorem sets.  It should be noted that any
% theorem set defined by |\newtheorem| is typeset in the
% |\theoremstyle| that is current at the time of the definition.
% Thus, the following
% \begin{verbatim}
%    \theoremstyle{break}        \newtheorem{Cor}{Corollary}
%    \theoremstyle{plain}        \newtheorem{Exa}{Example}[section]
%\end{verbatim}                             ^^A No blank at line start
% leads to the result that the set {\envfont Cor} is formatted in the
% style \texttt{break}, while the set {\envfont Exa} and all the
% following ones are formatted in the style \texttt{plain}, unless
% another |\theoremstyle| follows.  Since the definitions installed by
% |\newtheorem| are global, one also can limit |\theoremstyle| locally
% by grouping braces.
%
% \DescribeMacro\theorembodyfont
% The choice of the font for the theorem body is completely
% independent of the chosen |\theoremstyle|; this has proven to
% be very advantageous.  For example,
% \begin{verbatim}
%   {\theorembodyfont{\upshape}          \newtheorem{Rem}{Remark}}
%\end{verbatim}
% defines a theorem set {\envfont Rem}, which will be set in
% |\upshape| in the current layout (which in our example is
% \texttt{plain}).  As with |\theoremstyle|, the |\theorembodyfont|
% chosen is that current at the time of |\newtheorem|.  If
% |\theorembodyfont| is not specified or one defines
% |\theorembodyfont{}|, then the font used will be that defined by the
% |\theoremstyle|.
%
% \DescribeMacro\theoremheaderfont It is also possible to customize
% the font used for the theorem headers.  This is, however, a global
% declaration, and therefore there should be at most one
% |\theoremheaderfont| declaration in the preamble.\footnote{If it is
% actually necessary to have different header fonts, one has to define
% new theorem styles (substituting the desired font) or specify the
% information directly in the \texttt{\bslash newtheorem} declaration
% (the unclean variant).}
%
% \DescribeMacro\theorempreskipamount
% \DescribeMacro\theorempostskipamount Two additional parameters
% affect the vertical space around the theorem environments:
% |\theorempreskipamount| and |\theorempostskipamount| define,
% respectively, the spacing before and after such an environment.
% These parameters apply for all theorem sets and can be manipulated
% with the ordinary length macros.  They are rubber lengths
% (`\textsf{skips}'), and therefore can contain \texttt{plus} and
% \texttt{minus} parts.
%
% Since the definition of theorem sets should---most sensibly---be
% placed in the preamble, we only allow installation there.
% It is therefore possible to release the memory used here after
% |\begin{document}|, in order to make room for other
% applications.
%
%
% \subsection{Existing theorem styles}
%
% The following theorem styles exist to date:
% \begin{deflist}{marginbreak}
%    \item[plain]
%       This theorem style emulates the original \LaTeX{} definition,
%       except that additionally the parameters
%       |\theorem...skipamount| are used.
%    \item[break]
%       In this style, the theorem header is followed by a line
%       break.
%    \item[marginbreak]
%       The theorem number is set in the margin, and there is a line
%       break as in \texttt{break}.
%    \item[changebreak]
%       Like \texttt{break}, but with header number and text
%       interchanged.
%    \item[change]
%       Header number and text are interchanged, without a line break.
%    \item[margin]
%       The number is set in the left margin, without a line break.
% \end{deflist}
% All styles (except \texttt{plain}) select |\slshape| as the default
% |\theorembodyfont|.
%
%
% \subsection{Examples}
%
% Given the above theorem sets {\envfont Cor}, {\envfont Exa}
% and {\envfont Rem},
% suppose that the preamble also contains the declarations:
% \begin{verbatim}
%    \theoremstyle{marginbreak}   \newtheorem{Lem}[Cor]{Lemma}
%    \theoremstyle{change}
%    \theorembodyfont{\itshape}        \newtheorem{Def}[Cor]{Definition}
%
%    \theoremheaderfont{\scshape}
%\end{verbatim}
% Then the following are some typical examples of the typeset output
% resulting from their use.
%
% \begin{Cor}
%    This is a sentence typeset in the theorem environment {\envfont
%    Cor}.
% \end{Cor}
%
% \begin{Exa}
%    This is a sentence typeset in the theorem environment {\envfont
%    Exa}.
% \end{Exa}
%
% \begin{Rem}
%    This is a sentence typeset in the theorem environment {\envfont
%    Rem}.
% \end{Rem}
%
% \begin{Lem}[Ben User]
%    This is a sentence typeset in the theorem environment {\envfont
%    Lem}.
% \end{Lem}
%
% \begin{Def}[Very impressive Definition]
%    This is a sentence typeset in the theorem environment {\envfont
%    Def}.
% \end{Def}
%
% The last two examples show the effect of the optional argument to
% a theorem environment (it is the text typeset in parentheses).
%
%
% \section{Special Considerations}
% \label{sec:newfonts}
%
% Theoremheader and body are implemented as a unit. This means that
% the |\theoremheaderfont|  will inherit characteristics of the
% |\theorembodyfont| in \LaTeXe. Thus, if for example
% |\theorembodyfont| is |\itshape| and |\theoremheaderfont|
% is |\bfseries| the font selected for the header will have the
% characteristics `bold extended italic'. If this is not desired one
% should set the |\theoremheaderfont| to something like
% \begin{verbatim}
%      \theoremheaderfont{\normalfont\bfseries}
%\end{verbatim}
% i.e.\ supplying all necessary font information explicitly.
%
%
%
% \section{Acknowledgements}
%
% The publication of this set of macros was only possible with the
% help of Christina Busse (translating the manuscript into English),
% Joachim Pense (playing the r\^ole of typist), Chris Rowley
% (looking everything over) and many others providing useful
% suggestions.
%
% \MaybeStop{%
%    \begin{thebibliography}{9}
%       \bibitem{bk:GMS94} \textsc{M.~Goossens}, \textsc{F.~Mittelbach}
%         and \textsc{A.~Samarin}.
%         \newblock The \LaTeX{} Companion.
%         \newblock
%         Addison-Wesley, Reading, Massachusetts, 1994.
%       \bibitem{latex:source} \textsc{Lamport, Leslie.} \newblock
%                               \texttt{latex.tex}, version 2.09, date
%                               Feb. 1990.
%    \end{thebibliography}
%    \PrintIndex
%    \PrintChanges}
%
%
% \section{The documentation driver file}
%
% The next bit of code contains the documentation driver file for
% \TeX{}, i.e., the file that will produce the documentation you are
% currently reading. It will be extracted from this file by the
% \texttt{docstrip} program. Since it is the first code in the file
% one can alternatively process this file directly with \LaTeXe{} to
% obtain the documentation.
%
%    \begin{macrocode}
%<*driver>
\documentclass{ltxdoc}

\usepackage{theorem}

%<+driver>% The next few lines define theorem sets which are used
%<+driver>% in the example section of the documentation.

\theoremstyle{break}       \newtheorem{Cor}{Corollary}
\theoremstyle{plain}       \newtheorem{Exa}{Example}[section]
{\theorembodyfont{\upshape}\newtheorem{Rem}{Remark}}
\theoremstyle{marginbreak} \newtheorem{Lem}[Cor]{Lemma}
\theoremstyle{change}
\theorembodyfont{\itshape} \newtheorem{Def}[Cor]{Definition}

\theoremheaderfont{\scshape}

\RecordChanges

\begin{document}
   \DocInput{theorem.dtx}
\end{document}
%</driver>
%    \end{macrocode}
%
% \section{Definition of the Macros}
%
% If the file has been loaded before, we abort immediately. If not the
% package announces itself (this is actually done at the very top of
% the file---the way it is done isn't good style so don't copy it).
% \changes{v2.0e}{89/07/19}{Spaces removed from `typeout'.}
% \changes{v2.2b}{95/11/19}{Announce moved to top of file}
%    \begin{macrocode}
%<*package>
%\@ifundefined{theorem@style}{}{\endinput}
%\def\FMithmInfo{1995/11/19 v2.2b Theorem extension package (FMi)}
%\ProvidesPackage{theorem}[\FMithmInfo]
%    \end{macrocode}
%
%
% \subsection{Definition of theorem styles and fonts}
%
% All the definitions in this file are done globally to allow
% inputting this file inside a group.
%
% \begin{macro}{\theoremstyle}
%    Before a theorem style can be installed, the chosen style must be
%    known. For that reason, we must test to see that
%    |\th@|\meta{style} is known or, more precisely, that it is
%    different from |\relax|. If the style is not known then
%    |\th@plain| is used.
%    \begin{macrocode}
\gdef\theoremstyle#1{%
   \@ifundefined{th@#1}{\@warning
          {Unknown theoremstyle `#1'. Using `plain'}%
          \theorem@style{plain}}%
%    \end{macrocode}
%    We save the theorem style to be used in the token register
%    |\theorem@style|.
%    \begin{macrocode}
      {\theorem@style{#1}}%
%    \end{macrocode}
%    Now we ``evaluate'' the theorem style: this means, we call the
%    macro |\th@|\meta{style} which will activate the relevant
%    definitions which are contained in a separate file.  This is done
%    in a group to suppress changes to the current font.  This could
%    otherwise pose problems together with the new font selection
%    scheme\footnote{When I printed the original article using the new
%    font selection scheme I ended with a document with slanted
%    typefaces (text headings and all) simply because one of the
%    theorem styles used \texttt{\bslash sl} at toplevel.} if the
%    |\th@|\meta{style} is evaluated a second time.
% \changes{v2.1a}{90/02/15}{Group added around `th@..}
% \changes{v2.1a}{90/02/15}{Reseting of `..@begintheorem removed.}
%    \begin{macrocode}
      \begingroup
        \csname th@\the\theorem@style \endcsname
      \endgroup}
%    \end{macrocode}
% \end{macro}
%
%
% \begin{macro}{\@begintheorem}
% \changes{v2.1a}{90/02/15}{Relaxed at toplevel.}
% \begin{macro}{\@opargbegintheorem}
% \changes{v2.1a}{90/02/15}{Relaxed at toplevel.}
%    We reset |\@begintheorem| and |\@opargbegintheorem| to
%    |\relax| since these commands are no longer necessary at
%    toplevel. This will save a few tokens.
%    \begin{macrocode}
\global\let\@begintheorem\relax
\global\let\@opargbegintheorem\relax
%    \end{macrocode}
% \end{macro}
% \end{macro}
%
% \begin{macro}{\theorem@style}
%    Obviously the \textsf{token} register used above has to be
%    allocated. To assure the utmost compatibility with the
%    original \LaTeX{} definition, we set the default theorem style to
%    \texttt{plain}, which implements the usual \LaTeX{} convention.
% \changes{v2.0f}{89/07/20}{Unnecessary `global' removed.}
%    \begin{macrocode}
\newtoks\theorem@style
\global\theorem@style{plain}
%    \end{macrocode}
% \end{macro}
%
% \begin{macro}{\theorembodyfont}
% \begin{macro}{\theorem@bodyfont}
%    If the |\theorembodyfont| is set by the user then it should not
%    interact with the default font set in the theorem style. When the
%    new font selection is in force this may happen if, for example,
%    the default is |\itshape| and the new |\theorembodyfont| is
%    |\sffamily|. So we add a |\reset@font| command in front of the
%    user definition.
% \changes{v2.0f}{89/07/20}{Unnecessary `global' removed.}
% \changes{v2.0g}{89/09/19}{`theoremfont to `theorembodyfont.}
% \changes{v2.1c}{92/06/22}{`reset@font added and logic changed.}
% \changes{v2.1d}{92/10/28}{Reset to empty if arg is empty}
%    \begin{macrocode}
\gdef\theorembodyfont#1{%
%    \end{macrocode}
%    We check if the argument supplied is empty and if so put nothing
%    into the |\theorem@bodyfont| token register to allow for
%    |\theorembodyfont{}| as a mean of using the default of the
%    current |\theoremstyle|.
%    \begin{macrocode}
   \def\@tempa{#1}%
   \ifx\@tempa\@empty
    \theorem@bodyfont{}%
   \else
    \theorem@bodyfont{\reset@font#1}%
   \fi
   }
\newtoks\theorem@bodyfont
\global\theorem@bodyfont{}
%    \end{macrocode}
% \end{macro}
% \end{macro}
%
%
% \begin{macro}{\theoremheaderfont}
% \changes{v2.0g}{89/09/19}{Macro added.}
%    The font for the theorem headers is handled differently because
%    this definition applies to all theorem styles.
%    \begin{macrocode}
\gdef\theoremheaderfont#1{\gdef\theorem@headerfont{#1}%
%    \end{macrocode}
%    After using the macro once it is redefined to produce an error
%    message.
%    \begin{macrocode}
       \gdef\theoremheaderfont##1{%
        \typeout{\string\theoremheaderfont\space should be used
                 only once.}}}
%    \end{macrocode}
% \end{macro}
%
% \begin{macro}{\theorem@headerfont}
% \changes{v2.0g}{89/09/19}{Macro added.}
%    To set the |\theorem@headerfont| default we first test if
%    the new fontselection scheme is in force.
% \changes{v2.1a}{90/02/15}{Support for new font selection.}
%    \begin{macrocode}
\ifx\upshape\undefined
%    \end{macrocode}
%    If not we define it to expand into |\bfseries|. We don't use
%    |\let| just in case a following style option redefines this
%    macro.
%    \begin{macrocode}
\gdef\theorem@headerfont{\bfseries}
%    \end{macrocode}
%    Otherwise we reset the current shape before calling |\bfseries|.
%    \begin{macrocode}
\else \gdef\theorem@headerfont{\normalfont\bfseries}\fi
%    \end{macrocode}
% \end{macro}
%
% \begin{macro}{\th@plain}
% \begin{macro}{\th@break}
% \begin{macro}{\th@marginbreak}
% \begin{macro}{\th@changebreak}
% \begin{macro}{\th@change}
% \begin{macro}{\th@margin}
%    The different styles are defined in macros such as
%    |\th@plain|.  Since memory space is precious in
%    ``non-Big-versions'', we have to avoid offering too many unused
%    definitions.  Therefore we define these styles in separate files
%    that can be loaded on demand.  Thus the commands themselves only
%    load these files. We use |\@input@|, a \LaTeXe{} internal command
%    that ensures that the file will be listed with |\listfiles|.
% \changes{v2.2a}{94/02/02}{Use `@input@ to load theorem layout files}
%    \begin{macrocode}
\gdef\th@plain{\@input@{thp.sty}}
\gdef\th@break{\@input@{thb.sty}}
\gdef\th@marginbreak{\@input@{thmb.sty}}
\gdef\th@changebreak{\@input@{thcb.sty}}
\gdef\th@change{\@input@{thc.sty}}
\gdef\th@margin{\@input@{thm.sty}}
%    \end{macrocode}
%    This list will be expanded when new styles become
%    available.  For testing,  just append new theorem substyles
%    as document options.
% \end{macro}
% \end{macro}
% \end{macro}
% \end{macro}
% \end{macro}
% \end{macro}
%
%
%
% \subsection{Definition of a new theorem set}
%
% As already pointed out, a new theorem environment can be defined in
% three different ways:
% \begin{verbatim}
%    \newtheorem{Lem}{Lemma}
%    \newtheorem{Lem}{Lemma}[section]
%    \newtheorem{Lem}[Theorem]{Lemma}
%\end{verbatim}
% The function of the macro |\newtheorem| is to recognize these cases
% and then to branch into one of the three macros |\@ynthm|, |\@xnthm|
% or |\@othm|.  This mechanism is adopted unchanged from
% \cite{latex:source}; the essential point here is that, for example,
% in the second case, the arguments \texttt{Lem}, \texttt{Lemma} and
% \texttt{section} are passed over to the macro |\@xnthm|.
%
% We inspect this case first because the others present fewer problems,
% and thus are easily derived from this one.
%
% \begin{macro}{\@xnthm}
%    For our example arguments, the macro |\@xnthm| must fulfill
%    the following:
%    \begin{itemize}
%       \item
%          Define a new \LaTeX-counter `Lem'
%       \item
%          reset this counter within a |\section|
%       \item
%          define the macro |\theLem|
%       \item
%          define the environment macros |\Lem| and
%          |\endLem| using the current
%          |\theoremstyle| and |\theorem@bodyfont|.
%    \end{itemize}
%    Obviously, all this should happen only if the first argument
%    of |\@xnthm| (i.e.\ \texttt{Lem} in our example) is chosen so
%    as not to conflict with any previously defined commands or
%    environments.  This test is performed by the \LaTeX{} macro
%    |\@ifdefinable|.
%    \begin{macrocode}
\gdef\@xnthm#1#2[#3]{\expandafter\@ifdefinable\csname #1\endcsname
%    \end{macrocode}
%    Therefore, the first argument of |\@ifdefinable| is the
%    expansion (in the example, |\Lem|) of
%    |\csname||#1\endcsname|.  The second argument is
%    executed only if the test has been completed successfully.
%    \begin{macrocode}
   {%
%    \end{macrocode}
%    Now we define the new counter. The names of the \LaTeX{}
%    macros employed should speak for themselves:
% \changes{v2.2b}{1995/10/07}{Use \cs{@newctr}}
% \changes{v2.2c}{1995/11/23}{Use \cs{@newctr} properly.}
%    \begin{macrocode}
    \@definecounter{#1}\@newctr{#1}[#3]%
%    \end{macrocode}
%    Using |\@newctr| will give a proper error message if the counter
%    in |#3| is not defined.
%    In defining `|\theLem|' we must generate the desired macro
%    name by use of |\expandafter| and |\csname|.
%    \begin{macrocode}
    \expandafter\xdef\csname the#1\endcsname
%    \end{macrocode}
%    An |\xdef| is used in order to make the definition global,
%    and to ensure that it contains the replacement texts of
%    |\@thmcountersep| and |\@thmcounter|.\footnote{These
%    two macros can be defined by the document style.  Their default
%    values produce a `\texttt{.}' as separation and an arabic
%    representation of the number.} However, not everything should be
%    expanded. For example, it saves space to use |\thesection|
%    instead of its---at times---lengthy expansion.
%    \begin{macrocode}
      {\expandafter \noexpand \csname the#3\endcsname
       \@thmcountersep \@thmcounter{#1}}%
%    \end{macrocode}
%    Thus with the defaults of \LaTeX{}, |\theLem| would be
%    replaced by the command sequence |\thesection.\arabic{Lem}|.
%
%    We will now look at the definition of the macro which is executed
%    at the beginning of the actual environment (in our example this
%    macro is |\Lem|). It should be noted that we use an
%    ``|\expandafter| trick'' to expand only certain parts of the
%    replacement text at the time of the definition.
%    \begin{macrocode}
    \def\@tempa{\global\@namedef{#1}}%
    \expandafter \@tempa \expandafter{%
%    \end{macrocode}
%    First, the macro that contains the current definitions of
%    |\@begintheorem| and |\@opargtheorem| should be called
%    up.  The name of this macro---as is already known---has the form
%    |\th@|\meta{theorem\ style}; therefore, it must be called by
%    \begin{macrocode}
      \csname th@\the \theorem@style
%    \end{macrocode}
%    In addition the default theorem font should be changeable, i.e.\
%    we have to insert the contents of |\theorem@bodyfont|. For
%    that reason, we expand even further, beyond |\endcsname|,
%    and thus insert the contents of the \textsf{token} register
%    |\theorem@bodyfont| in the replacement text.
%    \begin{macrocode}
            \expandafter \endcsname \the \theorem@bodyfont
%    \end{macrocode}
%    Now it is time to call the macro |\@thm| which takes over
%    the further processing.  It has two arguments: the current
%    counter name (in our example, \texttt{Lem}), and the text of the
%    label (in our example, \texttt{Lemma}).
%    \begin{macrocode}
     \@thm{#1}{#2}}%
%    \end{macrocode}
%    With this, the `sub-definition' is complete.  The macro
%    |\@endtheorem| ends a theorem environment and is, so far, nothing
%    but an |\endtrivlist|.  (Hence it is defined globally, and not
%    within the theorem styles.\footnote{This has to be changed as
%    soon as theorem styles that change \texttt{\bslash @endtheorem}
%    exist.  In such a case, all existing styles must be changed as
%    well since they will have to reset the macro.}) Therefore, we can
%    set it equivalent to the macro that ends the theorem set (in our
%    example, |\endLem|).  However, if some day theorem styles exist
%    that do change |\@endtheorem|, we would have to use the
%    commented-out line instead.
%    \begin{macrocode}
    \global \expandafter \let \csname end#1\endcsname \@endtheorem
%   \global\@namedef{end#1}{\@endtheorem}%
%    \end{macrocode}
%    With these commands all the required definitions are employed,
%    unless the test |\@ifdefinable| has failed.  Therefore, we
%    end the second argument of this macro and with it the definition
%    of |\@xnthm|.
%    \begin{macrocode}
   }}
%    \end{macrocode}
% \end{macro}
%
% \begin{macro}{\@ynthm}
%    The definition of |\@ynthm| is completely analogous. In
%    this case the new counter that is defined is not reset within
%    another counter; thus the definition of |\the...|
%    is simplified:
%    \begin{macrocode}
\gdef\@ynthm#1#2{\expandafter\@ifdefinable\csname #1\endcsname
   {\@definecounter{#1}%
    \expandafter\xdef\csname the#1\endcsname{\@thmcounter{#1}}%
%    \end{macrocode}
%    The rest of the definition corresponds literally to that of
%    |\@xnthm|:
%    \begin{macrocode}
    \def\@tempa{\global\@namedef{#1}}\expandafter \@tempa
     \expandafter{\csname th@\the \theorem@style \expandafter
     \endcsname \the\theorem@bodyfont \@thm{#1}{#2}}%
    \global \expandafter \let \csname end#1\endcsname \@endtheorem}}
%    \end{macrocode}
% \end{macro}
%
% \begin{macro}{\@othm}
%    The definition of |\@othm| does not contain anything new.
%    \begin{macrocode}
\gdef\@othm#1[#2]#3{%
%    \end{macrocode}
% \changes{v2.2b}{95/10/07}{Build in check for counter defined}
%    We do not define a new counter but instead use one that has
%    already been defined.  Thus the only definition we need is that
%    of this pseudo-counter (i.e.\ |\the|\meta{env.\ name}).
%    First we check if |#2| corresponds to a known counter name.
%    \begin{macrocode}
  \expandafter\ifx\csname c@#2\endcsname\relax
   \@nocounterr{#2}%
  \else
   \expandafter\@ifdefinable\csname #1\endcsname
   {\expandafter \xdef \csname the#1\endcsname
     {\expandafter \noexpand \csname the#2\endcsname}%
%    \end{macrocode}
%    All other parts of the definition can be adopted from
%    |\@xnthm|. We have to remember, though, that in this case
%    the name of the current counter and the theorem label have moved
%    to the second and third arguments.
%    \begin{macrocode}
    \def\@tempa{\global\@namedef{#1}}\expandafter \@tempa
     \expandafter{\csname th@\the \theorem@style \expandafter
     \endcsname \the\theorem@bodyfont \@thm{#2}{#3}}%
    \global \expandafter \let \csname end#1\endcsname \@endtheorem}%
  \fi}
%    \end{macrocode}
% \end{macro}
%
% \subsection{Macros that are employed in a {\envfont theorem}
%    environment}
%
% \begin{macro}{\@thm}
%    The macro |\@thm| has to increase the current counter.  Then,
%    depending on whether the environment has (or does not have) an
%    optional argument, it has to branch into either |\@begintheorem|
%    or |\@opargtheorem|.
%    \begin{macrocode}
\gdef\@thm#1#2{\refstepcounter{#1}%
%    \end{macrocode}
%    Now we start a {\envfont trivlist} environment, and give
%    |\@topsep| and |\@topsepadd| the values of the \textsf{skip}
%    registers |\theorempreskipamount| and |\theorempostskipamount|.
%    The value in |\@topsep| is the vertical space that is inserted by
%    the first (and only) |\item| in our |\trivlist| whilst
%    |\@topsepadd| is inserted by |\@endparenv| at the end of that
%    {\envfont trivlist} environment.  By using these registers, we
%    obtain the desired space around a {\envfont theorem} environment.
%    \begin{macrocode}
   \trivlist
   \@topsep \theorempreskipamount               % used by first \item
   \@topsepadd \theorempostskipamount           % used by \@endparenv
%    \end{macrocode}
%    Now we have to test whether an optional argument has been given.
% \changes{v2.0d}{89/07/18}{Forgotten percent added.}
%    \begin{macrocode}
   \@ifnextchar [%
%    \end{macrocode}
%    If there is an optional argument, we will call |\@ythm|, and
%    move the arguments read back into the input stream.
%    \begin{macrocode}
   {\@ythm{#1}{#2}}%
%    \end{macrocode}
%    If not, we call |\@begintheorem|. Its first argument is the
%    name of the theorem set (hence the second argument of
%    |\@thm|). Its second argument is the macro that produces the
%    current number.
%    \begin{macrocode}
   {\@begintheorem{#2}{\csname the#1\endcsname}\ignorespaces}}
%    \end{macrocode}
% \end{macro}
%
% \begin{macro}{\@xthm}
% \begin{macro}{\@ythm}
%    Both these macros were originally called by |\@thm|.  We do
%    not need |\@xthm| anymore, hence we reset it to
%    |\relax|. The definition of |\@ythm| has not changed at
%    all from its definition in \LaTeX. In order to make the macros
%    easier to understand, we will
%    nevertheless present it (commented out).
%    \begin{macrocode}
\global\let\@xthm\relax
% \def\@ythm#1#2[#3]{\@opargbegintheorem{#2}{\csname
%           the#1\endcsname}{#3}\ignorespaces}
%</package>
%    \end{macrocode}
%    The primitive |\ignorespaces| in |\@ythm| and
%    |\@thm| is needed to remove the spaces between the
%    |\begin{...}| and the actual text.
% \end{macro}
% \end{macro}
%
%
% \subsection{Definition of the theorem substyles}
%
% As already pointed out, the theorem substyles, defined below, are
% only loaded when necessary.  Note that all these substyles, except
% \texttt{plain}, have |\slshape| as the default body font.
%
% \subsubsection{The \texttt{plain} style}
%
% \label{dok:th@plain} As the following macros use |@|, we have
% to locally set the |\catcode| of this symbol to ``letter''.
% This happens within a group, so that we do not have to worry about
% which |\catcode| that symbol had before.
%    \begin{macrocode}
%<*thp>
\begingroup \makeatletter
%    \end{macrocode}
% Since we are now within a group, we must make all definitions
% globally.  First we make sure that \texttt{theorem.sty} is loaded.
% This will allow us to use this file as a document style option
% without having to call \texttt{theorem} itself as an option.  At the
% same time, we assure that at least version 2 is loaded, since
% |\theorem@style| was not defined in earlier versions.
%    \begin{macrocode}
\@ifundefined{theorem@style}{\input{theorem.sty}}{}
\ProvidesFile{thp.sty}
             [\FMithmInfo]
%    \end{macrocode}
%
%
% \begin{macro}{\th@plain}
% \begin{macro}{\@begintheorem}
% \begin{macro}{\@opargbegintheorem}
%    |\theoremstyle{plain}| corresponds to the original
%    definition, except that the distances to the surrounding text are
%    determined by the parameters |\theorempreskipamount| and
%    |\theorempostskipamount|.
%    First we set the default body font.
%    \begin{macrocode}
\gdef\th@plain{\normalfont\itshape
%    \end{macrocode}
%    Then we define |\@begintheorem| and
%    |\@opargbegintheorem|. These two macros define how the
%    header of a theorem is typeset.  |\@opargbegintheorem| will
%    be called if a \textsf{theorem} environment with an optional
%    argument is encountered; otherwise, the header is constructed by
%    calling |\@begintheorem|.  If one of these macros is
%    executed, we are within a \textsf{trivlist} environment started by
%    |\@thm|.  So the theorem header is produced with an
%    |\item| command.
%
%    Instead of specifying the header font directly, all standard
%    theorem styles use the |\theorem@headerfont| macro to allow
%    customization.  The extra space (|\labelsep|) is necessary
%    because of problems in the \textsf{trivlist} environment.
%    \begin{macrocode}
  \def\@begintheorem##1##2{%
        \item[\hskip\labelsep \theorem@headerfont ##1\ ##2]}%
%    \end{macrocode}
%    The definition of |\@opargbegintheorem| is completely
%    analogous. The only difference is the fact that there exists a
%    third argument (which is the optional parameter of the
%    environment and contains additional information about the
%    theorem).  Customarily we enclose it in parentheses.
%    \begin{macrocode}
\def\@opargbegintheorem##1##2##3{%
   \item[\hskip\labelsep \theorem@headerfont ##1\ ##2\ (##3)]}}
%    \end{macrocode}
% \end{macro}
% \end{macro}
% \end{macro}
%
% \noindent
% We conclude with an |\endgroup| to restore the
% |\catcode| of |@|.
%    \begin{macrocode}
\endgroup
%</thp>
%    \end{macrocode}
%
%
% \subsubsection{The \texttt{break} style}
%
% This style option is stored in the file |thb.sty|.
% For the next two lines see the documentation for |\th@plain| on
% page~\pageref{dok:th@plain}.
%    \begin{macrocode}
%<*thb>
\begingroup \makeatletter
\@ifundefined{theorem@style}{\input{theorem.sty}}{}
\ProvidesFile{thb.sty}
             [\FMithmInfo]
%    \end{macrocode}
%
% \begin{macro}{\th@break}
%    |\theoremstyle{break}| produces a line break after the name
%    of the theorem.  The font is |\slshape|.  Hence, we define
%    |\th@break| as follows:
%    \begin{macrocode}
\gdef\th@break{\normalfont\slshape
  \def\@begintheorem##1##2{\item[%
%    \end{macrocode}
%    We run into the following problem: it is not possible to create
%    the header with |\item[|\meta{title}|]| and then start
%    a new line by, for example, |\mbox{}\\|. Such a definition
%    will fail whenever a list environment follows immediately.  With
%    the above construction, the |\mbox{}| causes the switch
%    |@inlabel| (cf.\ definition of |\list| and
%    |\trivlist| in \cite{latex:source}) to be set to \texttt{false}
%    and so the following list will insert additional vertical
%    space (|\topskip|). This is quite annoying. Therefore, we
%    create the line break within the |\item|. In order to ensure
%    that the text will begin at the proper position in the following
%    line, we simply pretend that the label does not take any
%    room.\footnote{This will lead to problems whenever very high
%    symbols occurring in the line tower into the
%    heading. So, something else has to be done here sometime.}
%    \begin{macrocode}
       \rlap{\vbox{\hbox{\hskip \labelsep\theorem@headerfont ##1\ ##2}%
                   \hbox{\strut}}}]}%
%    \end{macrocode}
%    Again, the definition of |\@opargbegintheorem| is
%    completely analogous.
%    \begin{macrocode}
\def\@opargbegintheorem##1##2##3{%
  \item[\rlap{\vbox{\hbox{\hskip \labelsep \theorem@headerfont
                     ##1\ ##2\ (##3)}%
                    \hbox{\strut}}}]}}
\endgroup
%</thb>
%    \end{macrocode}
% \end{macro}
%
%
% \subsubsection{The \texttt{changebreak} style}
%
%    \begin{macrocode}
% This style option is stored in the file |thcb.sty|.
%    \begin{macrocode}
%<*thcb>
\begingroup \makeatletter
\@ifundefined{theorem@style}{\input{theorem.sty}}{}
\ProvidesFile{thcb.sty}
             [\FMithmInfo]
%    \end{macrocode}
%
% \begin{macro}{\th@changebreak}
%    The \texttt{change-break} theorem style is like \texttt{break}
%    but with interchange of theorem name and theorem number.  Thus we
%    define |\th@changebreak| as follows:
%    \begin{macrocode}
\gdef\th@changebreak{\normalfont\slshape
  \def\@begintheorem##1##2{\item
       [\rlap{\vbox{\hbox{\hskip\labelsep \theorem@headerfont ##2\ ##1}%
                    \hbox{\strut}}}]}%
\def\@opargbegintheorem##1##2##3{%
  \item[\rlap{\vbox{\hbox{\hskip\labelsep \theorem@headerfont
                     ##2\ ##1\ (##3)}%
                    \hbox{\strut}}}]}}
\endgroup
%</thcb>
%    \end{macrocode}
% \end{macro}
%
%
% \subsubsection{The \texttt{change} style}
%
% This style option is stored in the file |thc.sty|.
%    \begin{macrocode}
%<*thc>
\begingroup \makeatletter
\@ifundefined{theorem@style}{\input{theorem.sty}}{}
\ProvidesFile{thc.sty}
             [\FMithmInfo]
%    \end{macrocode}
%
% \begin{macro}{\th@change}
%    The \texttt{change} theorem style corresponds to the
%    \texttt{change break} style without a linebreak after the header.
%    To say it in another way, it's the same as the \texttt{plain}
%    style but with number and name interchanged and |\slshape| as the
%    default font.
%    \begin{macrocode}
\gdef\th@change{\normalfont\slshape
  \def\@begintheorem##1##2{\item
       [\hskip\labelsep \theorem@headerfont ##2\ ##1]}%
\def\@opargbegintheorem##1##2##3{%
  \item[\hskip\labelsep \theorem@headerfont ##2\ ##1\ (##3)]}}
\endgroup
%</thc>
%    \end{macrocode}
% \end{macro}
%
%
% \subsubsection{The \texttt{marginbreak} style}
%
% This style option is the one used most often at Mainz.  It is saved
% in the file |thmb.sty|.
%    \begin{macrocode}
%<*thmb>
\begingroup \makeatletter
\@ifundefined{theorem@style}{\input{theorem.sty}}{}
\ProvidesFile{thmb.sty}
             [\FMithmInfo]
%    \end{macrocode}
%
% \begin{macro}{\th@marginbreak}
%    The \texttt{margin break} style is nearly the same as the
%    \texttt{change break} style. The only difference is the placement
%    of the theorem number.  We use |\llap| to place it in the left
%    margin.
%
%    In this style |\labelsep| denotes the separation between the
%    number and the text.
%    \begin{macrocode}
\gdef\th@marginbreak{\normalfont\slshape
  \def\@begintheorem##1##2{\item
       [\rlap{\vbox{\theorem@headerfont
                    \hbox{\llap{##2}\hskip\labelsep ##1}%
                    \hbox{\strut}}}]}%
\def\@opargbegintheorem##1##2##3{%
  \item[\rlap{\vbox{\theorem@headerfont
                    \hbox{\llap{##2}\hskip\labelsep ##1\ (##3)}%
                    \hbox{\strut}}}]}}
\endgroup
%</thmb>
%    \end{macrocode}
% \changes{v2.1c}{90/02/17}{`endgroup outside macrocode env.}
% \end{macro}
%
%
%
% \subsubsection{The \texttt{margin} style}
%
% This style option is stored in the file |thm.sty|.
%    \begin{macrocode}
%<*thm>
\begingroup \makeatletter
\@ifundefined{theorem@style}{\input{theorem.sty}}{}
\ProvidesFile{thm.sty}
             [\FMithmInfo]
%    \end{macrocode}
%
% \begin{macro}{\th@margin}
%    Again this is only a variant of the theorem styles described
%    above without any new ideas.
%    \begin{macrocode}
\gdef\th@margin{\normalfont\slshape
  \def\@begintheorem##1##2{\item
       [\theorem@headerfont \llap{##2}\hskip\labelsep ##1]}%
\def\@opargbegintheorem##1##2##3{%
  \item[\theorem@headerfont \llap{##2}\hskip\labelsep ##1\ (##3)]}}
\endgroup
%</thm>
%    \end{macrocode}
% \end{macro}
%
%
%
%
% \subsection{Final Definitions}
%
% \begin{macro}{\theorempreskipamount}
% \begin{macro}{\theorempostskipamount}
%    The \textsf{skip} parameters that regulate the vertical empty space
%    before and after the {\envfont theorem} environment have
%    to be allocated as well.
% \changes{v2.0f}{89/07/20}{Unnecessary `global' removed.}
%    \begin{macrocode}
%<*package>
\newskip\theorempreskipamount
\newskip\theorempostskipamount
%    \end{macrocode}
%    Since we have used the same values for all theorem sets, we now can
%    assign them.
%    \begin{macrocode}
\global\setlength\theorempreskipamount{12pt plus 5pt minus 3pt}
\global\setlength\theorempostskipamount{8pt plus 3pt minus 1.5pt}
%    \end{macrocode}
% \end{macro}
% \end{macro}
%
%
% \begin{macro}{\@endtheorem}
%    The same holds for the macro |\@endtheorem|, which ends a
%    {\envfont theorem} environment. Since it is the same for all
%    theorem sets, it is removed from the macros |\th@|\meta{style}.
%    It simply ends the {\envfont trivlist} environment, which was
%    begun in |\@thm|.
%    \begin{macrocode}
\global\let\@endtheorem=\endtrivlist
%    \end{macrocode}
% \end{macro}
%
%
%
% \begin{macro}{\@preamblecmds}
%    All macros defined above are to be used only in the preamble.
%    Therefore, we insert them in |\@preamblecmds| which will disable
%    them at begin document. This is done by the internal \LaTeXe{}
%    command |\@onlypreamble|.
% \changes{v2.0g}{89/09/19}{`theoremheaderfont and `theorembodyfont
%    added.}
% \changes{v2.2a}{94/02/03}{Use \LaTeXe{} interface}
%    \begin{macrocode}
\@onlypreamble\@xnthm
\@onlypreamble\@ynthm
\@onlypreamble\@othm
\@onlypreamble\newtheorem
\@onlypreamble\theoremstyle
\@onlypreamble\theorembodyfont
\@onlypreamble\theoremheaderfont
%    \end{macrocode}
% \end{macro}
%
% Finally we declare the \texttt{plain} theorem style to be the default.
% \changes{v2.1b}{90/06/04}{Fixed missing `item by declaring default
%    style.}
%    \begin{macrocode}
\theoremstyle{plain}
%</package>
%    \end{macrocode}
%
% \Finale

\endinput
}{}
\ProvidesFile{thm.sty}
             [\FMithmInfo]
%    \end{macrocode}
%
% \begin{macro}{\th@margin}
%    Again this is only a variant of the theorem styles described
%    above without any new ideas.
%    \begin{macrocode}
\gdef\th@margin{\normalfont\slshape
  \def\@begintheorem##1##2{\item
       [\theorem@headerfont \llap{##2}\hskip\labelsep ##1]}%
\def\@opargbegintheorem##1##2##3{%
  \item[\theorem@headerfont \llap{##2}\hskip\labelsep ##1\ (##3)]}}
\endgroup
%</thm>
%    \end{macrocode}
% \end{macro}
%
%
%
%
% \subsection{Final Definitions}
%
% \begin{macro}{\theorempreskipamount}
% \begin{macro}{\theorempostskipamount}
%    The \textsf{skip} parameters that regulate the vertical empty space
%    before and after the {\envfont theorem} environment have
%    to be allocated as well.
% \changes{v2.0f}{89/07/20}{Unnecessary `global' removed.}
%    \begin{macrocode}
%<*package>
\newskip\theorempreskipamount
\newskip\theorempostskipamount
%    \end{macrocode}
%    Since we have used the same values for all theorem sets, we now can
%    assign them.
%    \begin{macrocode}
\global\setlength\theorempreskipamount{12pt plus 5pt minus 3pt}
\global\setlength\theorempostskipamount{8pt plus 3pt minus 1.5pt}
%    \end{macrocode}
% \end{macro}
% \end{macro}
%
%
% \begin{macro}{\@endtheorem}
%    The same holds for the macro |\@endtheorem|, which ends a
%    {\envfont theorem} environment. Since it is the same for all
%    theorem sets, it is removed from the macros |\th@|\meta{style}.
%    It simply ends the {\envfont trivlist} environment, which was
%    begun in |\@thm|.
%    \begin{macrocode}
\global\let\@endtheorem=\endtrivlist
%    \end{macrocode}
% \end{macro}
%
%
%
% \begin{macro}{\@preamblecmds}
%    All macros defined above are to be used only in the preamble.
%    Therefore, we insert them in |\@preamblecmds| which will disable
%    them at begin document. This is done by the internal \LaTeXe{}
%    command |\@onlypreamble|.
% \changes{v2.0g}{89/09/19}{`theoremheaderfont and `theorembodyfont
%    added.}
% \changes{v2.2a}{94/02/03}{Use \LaTeXe{} interface}
%    \begin{macrocode}
\@onlypreamble\@xnthm
\@onlypreamble\@ynthm
\@onlypreamble\@othm
\@onlypreamble\newtheorem
\@onlypreamble\theoremstyle
\@onlypreamble\theorembodyfont
\@onlypreamble\theoremheaderfont
%    \end{macrocode}
% \end{macro}
%
% Finally we declare the \texttt{plain} theorem style to be the default.
% \changes{v2.1b}{90/06/04}{Fixed missing `item by declaring default
%    style.}
%    \begin{macrocode}
\theoremstyle{plain}
%</package>
%    \end{macrocode}
%
% \Finale

\endinput
}{}
\ProvidesFile{thb.sty}
             [\FMithmInfo]
%    \end{macrocode}
%
% \begin{macro}{\th@break}
%    |\theoremstyle{break}| produces a line break after the name
%    of the theorem.  The font is |\slshape|.  Hence, we define
%    |\th@break| as follows:
%    \begin{macrocode}
\gdef\th@break{\normalfont\slshape
  \def\@begintheorem##1##2{\item[%
%    \end{macrocode}
%    We run into the following problem: it is not possible to create
%    the header with |\item[|\meta{title}|]| and then start
%    a new line by, for example, |\mbox{}\\|. Such a definition
%    will fail whenever a list environment follows immediately.  With
%    the above construction, the |\mbox{}| causes the switch
%    |@inlabel| (cf.\ definition of |\list| and
%    |\trivlist| in \cite{latex:source}) to be set to \texttt{false}
%    and so the following list will insert additional vertical
%    space (|\topskip|). This is quite annoying. Therefore, we
%    create the line break within the |\item|. In order to ensure
%    that the text will begin at the proper position in the following
%    line, we simply pretend that the label does not take any
%    room.\footnote{This will lead to problems whenever very high
%    symbols occurring in the line tower into the
%    heading. So, something else has to be done here sometime.}
%    \begin{macrocode}
       \rlap{\vbox{\hbox{\hskip \labelsep\theorem@headerfont ##1\ ##2}%
                   \hbox{\strut}}}]}%
%    \end{macrocode}
%    Again, the definition of |\@opargbegintheorem| is
%    completely analogous.
%    \begin{macrocode}
\def\@opargbegintheorem##1##2##3{%
  \item[\rlap{\vbox{\hbox{\hskip \labelsep \theorem@headerfont
                     ##1\ ##2\ (##3)}%
                    \hbox{\strut}}}]}}
\endgroup
%</thb>
%    \end{macrocode}
% \end{macro}
%
%
% \subsubsection{The \texttt{changebreak} style}
%
%    \begin{macrocode}
% This style option is stored in the file |thcb.sty|.
%    \begin{macrocode}
%<*thcb>
\begingroup \makeatletter
\@ifundefined{theorem@style}{% \iffalse meta-comment
%
% Copyright (C) 1993-2024
%
% The LaTeX Project and any individual authors listed elsewhere
% in this file.
%
% This file is part of the Standard LaTeX `Tools Bundle'.
% -------------------------------------------------------
%
% It may be distributed and/or modified under the
% conditions of the LaTeX Project Public License, either version 1.3c
% of this license or (at your option) any later version.
% The latest version of this license is in
%    https://www.latex-project.org/lppl.txt
% and version 1.3c or later is part of all distributions of LaTeX
% version 2005/12/01 or later.
%
% The list of all files belonging to the LaTeX `Tools Bundle' is
% given in the file `manifest.txt'.
%
% \fi
%
%%
%
%
\def\FMithmInfo{2023/07/05 v2.2c Theorem extension package (FMi)}
%
% \ProvidesFile{theorem.dtx}[\FMithmInfo]
%\iffalse   % this is a METACOMMENT !
%%
%% Package `theorem' to use with LaTeX2e
%% Copyright (C) 1989-1995 Frank Mittelbach, all rights reserved.
%<+package>\@ifundefined{theorem@style}{}{\endinput}
%<+package>\NeedsTeXFormat{LaTeX2e}[1995/06/01]
%<+package>\ProvidesPackage{theorem}
%<+package>         [\FMithmInfo]
%
% \fi
%
% \changes{v1.0}{87/05/07}{First draft.}
% \changes{v1.0a}{87/05/07}{`thistheoremstyle' added.}
% \changes{v1.1a}{87/07/29}{New concept: redefine the internal
%                          \protect\LaTeX{} commands `@xnthm' etc.}
% \changes{v1.1b}{87/07/31}{`theoremfont' used to specify font.}
% \changes{v1.2a}{88/01/08}{`@endtheorem' now globally defined.}
% \changes{v1.2b}{88/01/11}{`newtheorem' only allowed in the preamble.}
% \changes{v1.3a}{88/02/14}{`theoremfont' now a toks register.}
%
% \changes{v2.0a}{88/11/16}{New implementation and new documentation.}
% \changes{v2.0a}{88/11/16}{No layout changes but internal modifications
%                           old substyles won't work.}
% \changes{v2.0b}{89/01/30}{`global' used for all definitions.}
% \changes{v2.0c}{89/06/15}{Documentation translated to English.
%                            Thanks to LLL and Nina}
% \changes{v2.1a}{90/02/15}{Support for new font selection.}
% \changes{v2.2a}{94/02/03}{Cleanup for LaTeX2e}
%
% \DoNotIndex{\@endparenv,\@ifdefinable,\@ifnextchar,\@ifundefined}
% \DoNotIndex{\@namedef,\@nameuse,\@spaces,\@tempa,\@warning}
% \DoNotIndex{\ }
% \DoNotIndex{\begingroup,\bfseries}
% \DoNotIndex{\csname}
% \DoNotIndex{\def,\do,\docdate}
% \DoNotIndex{\endcsname,\endgroup,\endinput,\endtrivlist,\expandafter}
% \DoNotIndex{\filedate,\fileversion}
% \DoNotIndex{\gdef,\global}
% \DoNotIndex{\hbox,\hskip}
% \DoNotIndex{\ignorespaces,\input,\itshape,\item}
% \DoNotIndex{\labelsep,\let,\llap}
% \DoNotIndex{\makeatletter}
% \DoNotIndex{\noexpand}
% \DoNotIndex{\refstepcounter,\relax,\rlap}
% \DoNotIndex{\setlength,\slshape,\space,\string,\strut}
% \DoNotIndex{\the,\trivlist,\typeout}
% \DoNotIndex{\vbox}
% \DoNotIndex{\xdef}
%
% \def\deflabel#1{\ttfamily #1\hfill}
% \def\deflist#1{\begin{list}{}{\settowidth\labelwidth{\ttfamily #1}%
%                               \setlength\leftmargin\labelwidth
%                               \addtolength\leftmargin\labelsep
%                               \let\makelabel\deflabel}}
% \def\enddeflist{\end{list}}
% \def\envfont{\normalfont\ttfamily}
%
% \MakeShortVerb{\|}
%
% \newcommand{\TUB}{{\sl TUGboat\/}}
%
% \GetFileInfo{theorem.dtx}
% \title{An Extension of the
%        \LaTeX{} theorem environment\thanks{This file
%        has version number \fileversion, last
%        revised \filedate.}}
% \author{Frank Mittelbach}
%
% \MaintainedByLaTeXTeam{tools}
% \maketitle
%
% \begin{abstract}
%    The macros described in this paper yield an extension of the
%    \LaTeX{} theorem mechanism. It is designed to satisfy the
%    different requirements of various journals. Thus, the
%    layout of the ``theorems'' can be manipulated by determining a
%    ``style''. This article describes not only the use, but
%    also the definition, of the necessary macros.
% \end{abstract}
%
% \section*{Preface to version 2.2}
%
% For \LaTeXe{} this package did not need any fundamental changes. I
% only modified the messages generated so that theorem layout styles
% will show up with the |\listfiles| command and cleaned the section
% on the New Font Selection Scheme since this is now included in
% \LaTeX.
%
% \section*{Preface to version 2.1}
%
% This version is identical to 2.0g described in \TUB\ 10\#3 except
% for some internal defaults which are now set depending on the used
% font selection scheme.
%
% This was done to avoid unpleasant surprises if the new font selection
% scheme is in force. For further details see
% section~\ref{sec:newfonts} and \cite{bk:GMS94}.
%
% \section{Introduction}
%
% For our purposes here, ``theorems'' are labelled enunciations,
% often set off from the main text by extra space and a font change.
% Theorems, corollaries, conjectures, definitions, and remarks
% are all instances of ``theorems''.  The ``header'' of these structures
% is composed of a label (such as \textsc{Theorem} or \textsc{Remark})
% and a number which serializes an item in the sequence of items
% with the same label.
%
% Shortly after the introduction of \LaTeX{} at the Fachbereich
% Mathematik in Mainz, the desire to manipulate the layout of
% ``theorems'' arose.  In Mainz, the following two conventions came
% into general use:
% \begin{enumerate}
%   \item The number of the theorem is shown in the margin.
%   \item There is a line break at
%         the end of the theorem header.
% \end{enumerate}
%
% Additionally, some journals require different formats which depend on
% the ``sort of theorem'': e.g.\ often remarks and definitions are set
% in |\upshape|, while |\itshape| is employed for main theorems.
%
% Confronted with these requirements, a theorem environment was
% developed in Mainz which allows separate determination of the
% layout of the ``theorems sets'', comparable to |\pagestyle|.
%
%
% \section{The user interface}
% \subsection{Defining new theorem sets}
%
% \DescribeMacro\newtheorem
% As in the original \LaTeX{} version, the command |\newtheorem|
% defines a new ``theorem set'' or ``theorem-like structure''.
% Two required arguments name the new environment and give the text
% to be typeset with each instance of the new ``set'', while
% an optional argument determines how
% the ``set'' is enumerated:
% \begin{description}
%    \item[\ttfamily \bslash newtheorem\{foo\}\{bar\}]
%       The theorem set {\envfont foo} (whose name is \texttt{bar})
%       uses its own counter.
%    \item[\ttfamily \bslash newtheorem\{foo2\}{[foo]}\{bar2\}]
%       The theorem set {\envfont foo2} (printed name \texttt{bar2})
%       uses the same counter as the theorem set \texttt{foo}.
%    \item[\ttfamily \bslash newtheorem\{foo3\}\{bar3\}{[section]}]
%       The theorem set {\envfont foo3} (printed name \texttt{bar3}) is
%       enumerated within the counter \texttt{section}, i.e.\ with every
%       new |\section| the enumeration begins again with 1, and
%       the enumeration is composed from the section-number and the
%       theorem counter itself.
% \end{description}
%
% \DescribeMacro\theoremstyle
% Additionally, the command |\theoremstyle| can define the layout
% of various, or all, theorem sets.  It should be noted that any
% theorem set defined by |\newtheorem| is typeset in the
% |\theoremstyle| that is current at the time of the definition.
% Thus, the following
% \begin{verbatim}
%    \theoremstyle{break}        \newtheorem{Cor}{Corollary}
%    \theoremstyle{plain}        \newtheorem{Exa}{Example}[section]
%\end{verbatim}                             ^^A No blank at line start
% leads to the result that the set {\envfont Cor} is formatted in the
% style \texttt{break}, while the set {\envfont Exa} and all the
% following ones are formatted in the style \texttt{plain}, unless
% another |\theoremstyle| follows.  Since the definitions installed by
% |\newtheorem| are global, one also can limit |\theoremstyle| locally
% by grouping braces.
%
% \DescribeMacro\theorembodyfont
% The choice of the font for the theorem body is completely
% independent of the chosen |\theoremstyle|; this has proven to
% be very advantageous.  For example,
% \begin{verbatim}
%   {\theorembodyfont{\upshape}          \newtheorem{Rem}{Remark}}
%\end{verbatim}
% defines a theorem set {\envfont Rem}, which will be set in
% |\upshape| in the current layout (which in our example is
% \texttt{plain}).  As with |\theoremstyle|, the |\theorembodyfont|
% chosen is that current at the time of |\newtheorem|.  If
% |\theorembodyfont| is not specified or one defines
% |\theorembodyfont{}|, then the font used will be that defined by the
% |\theoremstyle|.
%
% \DescribeMacro\theoremheaderfont It is also possible to customize
% the font used for the theorem headers.  This is, however, a global
% declaration, and therefore there should be at most one
% |\theoremheaderfont| declaration in the preamble.\footnote{If it is
% actually necessary to have different header fonts, one has to define
% new theorem styles (substituting the desired font) or specify the
% information directly in the \texttt{\bslash newtheorem} declaration
% (the unclean variant).}
%
% \DescribeMacro\theorempreskipamount
% \DescribeMacro\theorempostskipamount Two additional parameters
% affect the vertical space around the theorem environments:
% |\theorempreskipamount| and |\theorempostskipamount| define,
% respectively, the spacing before and after such an environment.
% These parameters apply for all theorem sets and can be manipulated
% with the ordinary length macros.  They are rubber lengths
% (`\textsf{skips}'), and therefore can contain \texttt{plus} and
% \texttt{minus} parts.
%
% Since the definition of theorem sets should---most sensibly---be
% placed in the preamble, we only allow installation there.
% It is therefore possible to release the memory used here after
% |\begin{document}|, in order to make room for other
% applications.
%
%
% \subsection{Existing theorem styles}
%
% The following theorem styles exist to date:
% \begin{deflist}{marginbreak}
%    \item[plain]
%       This theorem style emulates the original \LaTeX{} definition,
%       except that additionally the parameters
%       |\theorem...skipamount| are used.
%    \item[break]
%       In this style, the theorem header is followed by a line
%       break.
%    \item[marginbreak]
%       The theorem number is set in the margin, and there is a line
%       break as in \texttt{break}.
%    \item[changebreak]
%       Like \texttt{break}, but with header number and text
%       interchanged.
%    \item[change]
%       Header number and text are interchanged, without a line break.
%    \item[margin]
%       The number is set in the left margin, without a line break.
% \end{deflist}
% All styles (except \texttt{plain}) select |\slshape| as the default
% |\theorembodyfont|.
%
%
% \subsection{Examples}
%
% Given the above theorem sets {\envfont Cor}, {\envfont Exa}
% and {\envfont Rem},
% suppose that the preamble also contains the declarations:
% \begin{verbatim}
%    \theoremstyle{marginbreak}   \newtheorem{Lem}[Cor]{Lemma}
%    \theoremstyle{change}
%    \theorembodyfont{\itshape}        \newtheorem{Def}[Cor]{Definition}
%
%    \theoremheaderfont{\scshape}
%\end{verbatim}
% Then the following are some typical examples of the typeset output
% resulting from their use.
%
% \begin{Cor}
%    This is a sentence typeset in the theorem environment {\envfont
%    Cor}.
% \end{Cor}
%
% \begin{Exa}
%    This is a sentence typeset in the theorem environment {\envfont
%    Exa}.
% \end{Exa}
%
% \begin{Rem}
%    This is a sentence typeset in the theorem environment {\envfont
%    Rem}.
% \end{Rem}
%
% \begin{Lem}[Ben User]
%    This is a sentence typeset in the theorem environment {\envfont
%    Lem}.
% \end{Lem}
%
% \begin{Def}[Very impressive Definition]
%    This is a sentence typeset in the theorem environment {\envfont
%    Def}.
% \end{Def}
%
% The last two examples show the effect of the optional argument to
% a theorem environment (it is the text typeset in parentheses).
%
%
% \section{Special Considerations}
% \label{sec:newfonts}
%
% Theoremheader and body are implemented as a unit. This means that
% the |\theoremheaderfont|  will inherit characteristics of the
% |\theorembodyfont| in \LaTeXe. Thus, if for example
% |\theorembodyfont| is |\itshape| and |\theoremheaderfont|
% is |\bfseries| the font selected for the header will have the
% characteristics `bold extended italic'. If this is not desired one
% should set the |\theoremheaderfont| to something like
% \begin{verbatim}
%      \theoremheaderfont{\normalfont\bfseries}
%\end{verbatim}
% i.e.\ supplying all necessary font information explicitly.
%
%
%
% \section{Acknowledgements}
%
% The publication of this set of macros was only possible with the
% help of Christina Busse (translating the manuscript into English),
% Joachim Pense (playing the r\^ole of typist), Chris Rowley
% (looking everything over) and many others providing useful
% suggestions.
%
% \MaybeStop{%
%    \begin{thebibliography}{9}
%       \bibitem{bk:GMS94} \textsc{M.~Goossens}, \textsc{F.~Mittelbach}
%         and \textsc{A.~Samarin}.
%         \newblock The \LaTeX{} Companion.
%         \newblock
%         Addison-Wesley, Reading, Massachusetts, 1994.
%       \bibitem{latex:source} \textsc{Lamport, Leslie.} \newblock
%                               \texttt{latex.tex}, version 2.09, date
%                               Feb. 1990.
%    \end{thebibliography}
%    \PrintIndex
%    \PrintChanges}
%
%
% \section{The documentation driver file}
%
% The next bit of code contains the documentation driver file for
% \TeX{}, i.e., the file that will produce the documentation you are
% currently reading. It will be extracted from this file by the
% \texttt{docstrip} program. Since it is the first code in the file
% one can alternatively process this file directly with \LaTeXe{} to
% obtain the documentation.
%
%    \begin{macrocode}
%<*driver>
\documentclass{ltxdoc}

\usepackage{theorem}

%<+driver>% The next few lines define theorem sets which are used
%<+driver>% in the example section of the documentation.

\theoremstyle{break}       \newtheorem{Cor}{Corollary}
\theoremstyle{plain}       \newtheorem{Exa}{Example}[section]
{\theorembodyfont{\upshape}\newtheorem{Rem}{Remark}}
\theoremstyle{marginbreak} \newtheorem{Lem}[Cor]{Lemma}
\theoremstyle{change}
\theorembodyfont{\itshape} \newtheorem{Def}[Cor]{Definition}

\theoremheaderfont{\scshape}

\RecordChanges

\begin{document}
   \DocInput{theorem.dtx}
\end{document}
%</driver>
%    \end{macrocode}
%
% \section{Definition of the Macros}
%
% If the file has been loaded before, we abort immediately. If not the
% package announces itself (this is actually done at the very top of
% the file---the way it is done isn't good style so don't copy it).
% \changes{v2.0e}{89/07/19}{Spaces removed from `typeout'.}
% \changes{v2.2b}{95/11/19}{Announce moved to top of file}
%    \begin{macrocode}
%<*package>
%\@ifundefined{theorem@style}{}{\endinput}
%\def\FMithmInfo{1995/11/19 v2.2b Theorem extension package (FMi)}
%\ProvidesPackage{theorem}[\FMithmInfo]
%    \end{macrocode}
%
%
% \subsection{Definition of theorem styles and fonts}
%
% All the definitions in this file are done globally to allow
% inputting this file inside a group.
%
% \begin{macro}{\theoremstyle}
%    Before a theorem style can be installed, the chosen style must be
%    known. For that reason, we must test to see that
%    |\th@|\meta{style} is known or, more precisely, that it is
%    different from |\relax|. If the style is not known then
%    |\th@plain| is used.
%    \begin{macrocode}
\gdef\theoremstyle#1{%
   \@ifundefined{th@#1}{\@warning
          {Unknown theoremstyle `#1'. Using `plain'}%
          \theorem@style{plain}}%
%    \end{macrocode}
%    We save the theorem style to be used in the token register
%    |\theorem@style|.
%    \begin{macrocode}
      {\theorem@style{#1}}%
%    \end{macrocode}
%    Now we ``evaluate'' the theorem style: this means, we call the
%    macro |\th@|\meta{style} which will activate the relevant
%    definitions which are contained in a separate file.  This is done
%    in a group to suppress changes to the current font.  This could
%    otherwise pose problems together with the new font selection
%    scheme\footnote{When I printed the original article using the new
%    font selection scheme I ended with a document with slanted
%    typefaces (text headings and all) simply because one of the
%    theorem styles used \texttt{\bslash sl} at toplevel.} if the
%    |\th@|\meta{style} is evaluated a second time.
% \changes{v2.1a}{90/02/15}{Group added around `th@..}
% \changes{v2.1a}{90/02/15}{Reseting of `..@begintheorem removed.}
%    \begin{macrocode}
      \begingroup
        \csname th@\the\theorem@style \endcsname
      \endgroup}
%    \end{macrocode}
% \end{macro}
%
%
% \begin{macro}{\@begintheorem}
% \changes{v2.1a}{90/02/15}{Relaxed at toplevel.}
% \begin{macro}{\@opargbegintheorem}
% \changes{v2.1a}{90/02/15}{Relaxed at toplevel.}
%    We reset |\@begintheorem| and |\@opargbegintheorem| to
%    |\relax| since these commands are no longer necessary at
%    toplevel. This will save a few tokens.
%    \begin{macrocode}
\global\let\@begintheorem\relax
\global\let\@opargbegintheorem\relax
%    \end{macrocode}
% \end{macro}
% \end{macro}
%
% \begin{macro}{\theorem@style}
%    Obviously the \textsf{token} register used above has to be
%    allocated. To assure the utmost compatibility with the
%    original \LaTeX{} definition, we set the default theorem style to
%    \texttt{plain}, which implements the usual \LaTeX{} convention.
% \changes{v2.0f}{89/07/20}{Unnecessary `global' removed.}
%    \begin{macrocode}
\newtoks\theorem@style
\global\theorem@style{plain}
%    \end{macrocode}
% \end{macro}
%
% \begin{macro}{\theorembodyfont}
% \begin{macro}{\theorem@bodyfont}
%    If the |\theorembodyfont| is set by the user then it should not
%    interact with the default font set in the theorem style. When the
%    new font selection is in force this may happen if, for example,
%    the default is |\itshape| and the new |\theorembodyfont| is
%    |\sffamily|. So we add a |\reset@font| command in front of the
%    user definition.
% \changes{v2.0f}{89/07/20}{Unnecessary `global' removed.}
% \changes{v2.0g}{89/09/19}{`theoremfont to `theorembodyfont.}
% \changes{v2.1c}{92/06/22}{`reset@font added and logic changed.}
% \changes{v2.1d}{92/10/28}{Reset to empty if arg is empty}
%    \begin{macrocode}
\gdef\theorembodyfont#1{%
%    \end{macrocode}
%    We check if the argument supplied is empty and if so put nothing
%    into the |\theorem@bodyfont| token register to allow for
%    |\theorembodyfont{}| as a mean of using the default of the
%    current |\theoremstyle|.
%    \begin{macrocode}
   \def\@tempa{#1}%
   \ifx\@tempa\@empty
    \theorem@bodyfont{}%
   \else
    \theorem@bodyfont{\reset@font#1}%
   \fi
   }
\newtoks\theorem@bodyfont
\global\theorem@bodyfont{}
%    \end{macrocode}
% \end{macro}
% \end{macro}
%
%
% \begin{macro}{\theoremheaderfont}
% \changes{v2.0g}{89/09/19}{Macro added.}
%    The font for the theorem headers is handled differently because
%    this definition applies to all theorem styles.
%    \begin{macrocode}
\gdef\theoremheaderfont#1{\gdef\theorem@headerfont{#1}%
%    \end{macrocode}
%    After using the macro once it is redefined to produce an error
%    message.
%    \begin{macrocode}
       \gdef\theoremheaderfont##1{%
        \typeout{\string\theoremheaderfont\space should be used
                 only once.}}}
%    \end{macrocode}
% \end{macro}
%
% \begin{macro}{\theorem@headerfont}
% \changes{v2.0g}{89/09/19}{Macro added.}
%    To set the |\theorem@headerfont| default we first test if
%    the new fontselection scheme is in force.
% \changes{v2.1a}{90/02/15}{Support for new font selection.}
%    \begin{macrocode}
\ifx\upshape\undefined
%    \end{macrocode}
%    If not we define it to expand into |\bfseries|. We don't use
%    |\let| just in case a following style option redefines this
%    macro.
%    \begin{macrocode}
\gdef\theorem@headerfont{\bfseries}
%    \end{macrocode}
%    Otherwise we reset the current shape before calling |\bfseries|.
%    \begin{macrocode}
\else \gdef\theorem@headerfont{\normalfont\bfseries}\fi
%    \end{macrocode}
% \end{macro}
%
% \begin{macro}{\th@plain}
% \begin{macro}{\th@break}
% \begin{macro}{\th@marginbreak}
% \begin{macro}{\th@changebreak}
% \begin{macro}{\th@change}
% \begin{macro}{\th@margin}
%    The different styles are defined in macros such as
%    |\th@plain|.  Since memory space is precious in
%    ``non-Big-versions'', we have to avoid offering too many unused
%    definitions.  Therefore we define these styles in separate files
%    that can be loaded on demand.  Thus the commands themselves only
%    load these files. We use |\@input@|, a \LaTeXe{} internal command
%    that ensures that the file will be listed with |\listfiles|.
% \changes{v2.2a}{94/02/02}{Use `@input@ to load theorem layout files}
%    \begin{macrocode}
\gdef\th@plain{\@input@{thp.sty}}
\gdef\th@break{\@input@{thb.sty}}
\gdef\th@marginbreak{\@input@{thmb.sty}}
\gdef\th@changebreak{\@input@{thcb.sty}}
\gdef\th@change{\@input@{thc.sty}}
\gdef\th@margin{\@input@{thm.sty}}
%    \end{macrocode}
%    This list will be expanded when new styles become
%    available.  For testing,  just append new theorem substyles
%    as document options.
% \end{macro}
% \end{macro}
% \end{macro}
% \end{macro}
% \end{macro}
% \end{macro}
%
%
%
% \subsection{Definition of a new theorem set}
%
% As already pointed out, a new theorem environment can be defined in
% three different ways:
% \begin{verbatim}
%    \newtheorem{Lem}{Lemma}
%    \newtheorem{Lem}{Lemma}[section]
%    \newtheorem{Lem}[Theorem]{Lemma}
%\end{verbatim}
% The function of the macro |\newtheorem| is to recognize these cases
% and then to branch into one of the three macros |\@ynthm|, |\@xnthm|
% or |\@othm|.  This mechanism is adopted unchanged from
% \cite{latex:source}; the essential point here is that, for example,
% in the second case, the arguments \texttt{Lem}, \texttt{Lemma} and
% \texttt{section} are passed over to the macro |\@xnthm|.
%
% We inspect this case first because the others present fewer problems,
% and thus are easily derived from this one.
%
% \begin{macro}{\@xnthm}
%    For our example arguments, the macro |\@xnthm| must fulfill
%    the following:
%    \begin{itemize}
%       \item
%          Define a new \LaTeX-counter `Lem'
%       \item
%          reset this counter within a |\section|
%       \item
%          define the macro |\theLem|
%       \item
%          define the environment macros |\Lem| and
%          |\endLem| using the current
%          |\theoremstyle| and |\theorem@bodyfont|.
%    \end{itemize}
%    Obviously, all this should happen only if the first argument
%    of |\@xnthm| (i.e.\ \texttt{Lem} in our example) is chosen so
%    as not to conflict with any previously defined commands or
%    environments.  This test is performed by the \LaTeX{} macro
%    |\@ifdefinable|.
%    \begin{macrocode}
\gdef\@xnthm#1#2[#3]{\expandafter\@ifdefinable\csname #1\endcsname
%    \end{macrocode}
%    Therefore, the first argument of |\@ifdefinable| is the
%    expansion (in the example, |\Lem|) of
%    |\csname||#1\endcsname|.  The second argument is
%    executed only if the test has been completed successfully.
%    \begin{macrocode}
   {%
%    \end{macrocode}
%    Now we define the new counter. The names of the \LaTeX{}
%    macros employed should speak for themselves:
% \changes{v2.2b}{1995/10/07}{Use \cs{@newctr}}
% \changes{v2.2c}{1995/11/23}{Use \cs{@newctr} properly.}
%    \begin{macrocode}
    \@definecounter{#1}\@newctr{#1}[#3]%
%    \end{macrocode}
%    Using |\@newctr| will give a proper error message if the counter
%    in |#3| is not defined.
%    In defining `|\theLem|' we must generate the desired macro
%    name by use of |\expandafter| and |\csname|.
%    \begin{macrocode}
    \expandafter\xdef\csname the#1\endcsname
%    \end{macrocode}
%    An |\xdef| is used in order to make the definition global,
%    and to ensure that it contains the replacement texts of
%    |\@thmcountersep| and |\@thmcounter|.\footnote{These
%    two macros can be defined by the document style.  Their default
%    values produce a `\texttt{.}' as separation and an arabic
%    representation of the number.} However, not everything should be
%    expanded. For example, it saves space to use |\thesection|
%    instead of its---at times---lengthy expansion.
%    \begin{macrocode}
      {\expandafter \noexpand \csname the#3\endcsname
       \@thmcountersep \@thmcounter{#1}}%
%    \end{macrocode}
%    Thus with the defaults of \LaTeX{}, |\theLem| would be
%    replaced by the command sequence |\thesection.\arabic{Lem}|.
%
%    We will now look at the definition of the macro which is executed
%    at the beginning of the actual environment (in our example this
%    macro is |\Lem|). It should be noted that we use an
%    ``|\expandafter| trick'' to expand only certain parts of the
%    replacement text at the time of the definition.
%    \begin{macrocode}
    \def\@tempa{\global\@namedef{#1}}%
    \expandafter \@tempa \expandafter{%
%    \end{macrocode}
%    First, the macro that contains the current definitions of
%    |\@begintheorem| and |\@opargtheorem| should be called
%    up.  The name of this macro---as is already known---has the form
%    |\th@|\meta{theorem\ style}; therefore, it must be called by
%    \begin{macrocode}
      \csname th@\the \theorem@style
%    \end{macrocode}
%    In addition the default theorem font should be changeable, i.e.\
%    we have to insert the contents of |\theorem@bodyfont|. For
%    that reason, we expand even further, beyond |\endcsname|,
%    and thus insert the contents of the \textsf{token} register
%    |\theorem@bodyfont| in the replacement text.
%    \begin{macrocode}
            \expandafter \endcsname \the \theorem@bodyfont
%    \end{macrocode}
%    Now it is time to call the macro |\@thm| which takes over
%    the further processing.  It has two arguments: the current
%    counter name (in our example, \texttt{Lem}), and the text of the
%    label (in our example, \texttt{Lemma}).
%    \begin{macrocode}
     \@thm{#1}{#2}}%
%    \end{macrocode}
%    With this, the `sub-definition' is complete.  The macro
%    |\@endtheorem| ends a theorem environment and is, so far, nothing
%    but an |\endtrivlist|.  (Hence it is defined globally, and not
%    within the theorem styles.\footnote{This has to be changed as
%    soon as theorem styles that change \texttt{\bslash @endtheorem}
%    exist.  In such a case, all existing styles must be changed as
%    well since they will have to reset the macro.}) Therefore, we can
%    set it equivalent to the macro that ends the theorem set (in our
%    example, |\endLem|).  However, if some day theorem styles exist
%    that do change |\@endtheorem|, we would have to use the
%    commented-out line instead.
%    \begin{macrocode}
    \global \expandafter \let \csname end#1\endcsname \@endtheorem
%   \global\@namedef{end#1}{\@endtheorem}%
%    \end{macrocode}
%    With these commands all the required definitions are employed,
%    unless the test |\@ifdefinable| has failed.  Therefore, we
%    end the second argument of this macro and with it the definition
%    of |\@xnthm|.
%    \begin{macrocode}
   }}
%    \end{macrocode}
% \end{macro}
%
% \begin{macro}{\@ynthm}
%    The definition of |\@ynthm| is completely analogous. In
%    this case the new counter that is defined is not reset within
%    another counter; thus the definition of |\the...|
%    is simplified:
%    \begin{macrocode}
\gdef\@ynthm#1#2{\expandafter\@ifdefinable\csname #1\endcsname
   {\@definecounter{#1}%
    \expandafter\xdef\csname the#1\endcsname{\@thmcounter{#1}}%
%    \end{macrocode}
%    The rest of the definition corresponds literally to that of
%    |\@xnthm|:
%    \begin{macrocode}
    \def\@tempa{\global\@namedef{#1}}\expandafter \@tempa
     \expandafter{\csname th@\the \theorem@style \expandafter
     \endcsname \the\theorem@bodyfont \@thm{#1}{#2}}%
    \global \expandafter \let \csname end#1\endcsname \@endtheorem}}
%    \end{macrocode}
% \end{macro}
%
% \begin{macro}{\@othm}
%    The definition of |\@othm| does not contain anything new.
%    \begin{macrocode}
\gdef\@othm#1[#2]#3{%
%    \end{macrocode}
% \changes{v2.2b}{95/10/07}{Build in check for counter defined}
%    We do not define a new counter but instead use one that has
%    already been defined.  Thus the only definition we need is that
%    of this pseudo-counter (i.e.\ |\the|\meta{env.\ name}).
%    First we check if |#2| corresponds to a known counter name.
%    \begin{macrocode}
  \expandafter\ifx\csname c@#2\endcsname\relax
   \@nocounterr{#2}%
  \else
   \expandafter\@ifdefinable\csname #1\endcsname
   {\expandafter \xdef \csname the#1\endcsname
     {\expandafter \noexpand \csname the#2\endcsname}%
%    \end{macrocode}
%    All other parts of the definition can be adopted from
%    |\@xnthm|. We have to remember, though, that in this case
%    the name of the current counter and the theorem label have moved
%    to the second and third arguments.
%    \begin{macrocode}
    \def\@tempa{\global\@namedef{#1}}\expandafter \@tempa
     \expandafter{\csname th@\the \theorem@style \expandafter
     \endcsname \the\theorem@bodyfont \@thm{#2}{#3}}%
    \global \expandafter \let \csname end#1\endcsname \@endtheorem}%
  \fi}
%    \end{macrocode}
% \end{macro}
%
% \subsection{Macros that are employed in a {\envfont theorem}
%    environment}
%
% \begin{macro}{\@thm}
%    The macro |\@thm| has to increase the current counter.  Then,
%    depending on whether the environment has (or does not have) an
%    optional argument, it has to branch into either |\@begintheorem|
%    or |\@opargtheorem|.
%    \begin{macrocode}
\gdef\@thm#1#2{\refstepcounter{#1}%
%    \end{macrocode}
%    Now we start a {\envfont trivlist} environment, and give
%    |\@topsep| and |\@topsepadd| the values of the \textsf{skip}
%    registers |\theorempreskipamount| and |\theorempostskipamount|.
%    The value in |\@topsep| is the vertical space that is inserted by
%    the first (and only) |\item| in our |\trivlist| whilst
%    |\@topsepadd| is inserted by |\@endparenv| at the end of that
%    {\envfont trivlist} environment.  By using these registers, we
%    obtain the desired space around a {\envfont theorem} environment.
%    \begin{macrocode}
   \trivlist
   \@topsep \theorempreskipamount               % used by first \item
   \@topsepadd \theorempostskipamount           % used by \@endparenv
%    \end{macrocode}
%    Now we have to test whether an optional argument has been given.
% \changes{v2.0d}{89/07/18}{Forgotten percent added.}
%    \begin{macrocode}
   \@ifnextchar [%
%    \end{macrocode}
%    If there is an optional argument, we will call |\@ythm|, and
%    move the arguments read back into the input stream.
%    \begin{macrocode}
   {\@ythm{#1}{#2}}%
%    \end{macrocode}
%    If not, we call |\@begintheorem|. Its first argument is the
%    name of the theorem set (hence the second argument of
%    |\@thm|). Its second argument is the macro that produces the
%    current number.
%    \begin{macrocode}
   {\@begintheorem{#2}{\csname the#1\endcsname}\ignorespaces}}
%    \end{macrocode}
% \end{macro}
%
% \begin{macro}{\@xthm}
% \begin{macro}{\@ythm}
%    Both these macros were originally called by |\@thm|.  We do
%    not need |\@xthm| anymore, hence we reset it to
%    |\relax|. The definition of |\@ythm| has not changed at
%    all from its definition in \LaTeX. In order to make the macros
%    easier to understand, we will
%    nevertheless present it (commented out).
%    \begin{macrocode}
\global\let\@xthm\relax
% \def\@ythm#1#2[#3]{\@opargbegintheorem{#2}{\csname
%           the#1\endcsname}{#3}\ignorespaces}
%</package>
%    \end{macrocode}
%    The primitive |\ignorespaces| in |\@ythm| and
%    |\@thm| is needed to remove the spaces between the
%    |\begin{...}| and the actual text.
% \end{macro}
% \end{macro}
%
%
% \subsection{Definition of the theorem substyles}
%
% As already pointed out, the theorem substyles, defined below, are
% only loaded when necessary.  Note that all these substyles, except
% \texttt{plain}, have |\slshape| as the default body font.
%
% \subsubsection{The \texttt{plain} style}
%
% \label{dok:th@plain} As the following macros use |@|, we have
% to locally set the |\catcode| of this symbol to ``letter''.
% This happens within a group, so that we do not have to worry about
% which |\catcode| that symbol had before.
%    \begin{macrocode}
%<*thp>
\begingroup \makeatletter
%    \end{macrocode}
% Since we are now within a group, we must make all definitions
% globally.  First we make sure that \texttt{theorem.sty} is loaded.
% This will allow us to use this file as a document style option
% without having to call \texttt{theorem} itself as an option.  At the
% same time, we assure that at least version 2 is loaded, since
% |\theorem@style| was not defined in earlier versions.
%    \begin{macrocode}
\@ifundefined{theorem@style}{% \iffalse meta-comment
%
% Copyright (C) 1993-2024
%
% The LaTeX Project and any individual authors listed elsewhere
% in this file.
%
% This file is part of the Standard LaTeX `Tools Bundle'.
% -------------------------------------------------------
%
% It may be distributed and/or modified under the
% conditions of the LaTeX Project Public License, either version 1.3c
% of this license or (at your option) any later version.
% The latest version of this license is in
%    https://www.latex-project.org/lppl.txt
% and version 1.3c or later is part of all distributions of LaTeX
% version 2005/12/01 or later.
%
% The list of all files belonging to the LaTeX `Tools Bundle' is
% given in the file `manifest.txt'.
%
% \fi
%
%%
%
%
\def\FMithmInfo{2023/07/05 v2.2c Theorem extension package (FMi)}
%
% \ProvidesFile{theorem.dtx}[\FMithmInfo]
%\iffalse   % this is a METACOMMENT !
%%
%% Package `theorem' to use with LaTeX2e
%% Copyright (C) 1989-1995 Frank Mittelbach, all rights reserved.
%<+package>\@ifundefined{theorem@style}{}{\endinput}
%<+package>\NeedsTeXFormat{LaTeX2e}[1995/06/01]
%<+package>\ProvidesPackage{theorem}
%<+package>         [\FMithmInfo]
%
% \fi
%
% \changes{v1.0}{87/05/07}{First draft.}
% \changes{v1.0a}{87/05/07}{`thistheoremstyle' added.}
% \changes{v1.1a}{87/07/29}{New concept: redefine the internal
%                          \protect\LaTeX{} commands `@xnthm' etc.}
% \changes{v1.1b}{87/07/31}{`theoremfont' used to specify font.}
% \changes{v1.2a}{88/01/08}{`@endtheorem' now globally defined.}
% \changes{v1.2b}{88/01/11}{`newtheorem' only allowed in the preamble.}
% \changes{v1.3a}{88/02/14}{`theoremfont' now a toks register.}
%
% \changes{v2.0a}{88/11/16}{New implementation and new documentation.}
% \changes{v2.0a}{88/11/16}{No layout changes but internal modifications
%                           old substyles won't work.}
% \changes{v2.0b}{89/01/30}{`global' used for all definitions.}
% \changes{v2.0c}{89/06/15}{Documentation translated to English.
%                            Thanks to LLL and Nina}
% \changes{v2.1a}{90/02/15}{Support for new font selection.}
% \changes{v2.2a}{94/02/03}{Cleanup for LaTeX2e}
%
% \DoNotIndex{\@endparenv,\@ifdefinable,\@ifnextchar,\@ifundefined}
% \DoNotIndex{\@namedef,\@nameuse,\@spaces,\@tempa,\@warning}
% \DoNotIndex{\ }
% \DoNotIndex{\begingroup,\bfseries}
% \DoNotIndex{\csname}
% \DoNotIndex{\def,\do,\docdate}
% \DoNotIndex{\endcsname,\endgroup,\endinput,\endtrivlist,\expandafter}
% \DoNotIndex{\filedate,\fileversion}
% \DoNotIndex{\gdef,\global}
% \DoNotIndex{\hbox,\hskip}
% \DoNotIndex{\ignorespaces,\input,\itshape,\item}
% \DoNotIndex{\labelsep,\let,\llap}
% \DoNotIndex{\makeatletter}
% \DoNotIndex{\noexpand}
% \DoNotIndex{\refstepcounter,\relax,\rlap}
% \DoNotIndex{\setlength,\slshape,\space,\string,\strut}
% \DoNotIndex{\the,\trivlist,\typeout}
% \DoNotIndex{\vbox}
% \DoNotIndex{\xdef}
%
% \def\deflabel#1{\ttfamily #1\hfill}
% \def\deflist#1{\begin{list}{}{\settowidth\labelwidth{\ttfamily #1}%
%                               \setlength\leftmargin\labelwidth
%                               \addtolength\leftmargin\labelsep
%                               \let\makelabel\deflabel}}
% \def\enddeflist{\end{list}}
% \def\envfont{\normalfont\ttfamily}
%
% \MakeShortVerb{\|}
%
% \newcommand{\TUB}{{\sl TUGboat\/}}
%
% \GetFileInfo{theorem.dtx}
% \title{An Extension of the
%        \LaTeX{} theorem environment\thanks{This file
%        has version number \fileversion, last
%        revised \filedate.}}
% \author{Frank Mittelbach}
%
% \MaintainedByLaTeXTeam{tools}
% \maketitle
%
% \begin{abstract}
%    The macros described in this paper yield an extension of the
%    \LaTeX{} theorem mechanism. It is designed to satisfy the
%    different requirements of various journals. Thus, the
%    layout of the ``theorems'' can be manipulated by determining a
%    ``style''. This article describes not only the use, but
%    also the definition, of the necessary macros.
% \end{abstract}
%
% \section*{Preface to version 2.2}
%
% For \LaTeXe{} this package did not need any fundamental changes. I
% only modified the messages generated so that theorem layout styles
% will show up with the |\listfiles| command and cleaned the section
% on the New Font Selection Scheme since this is now included in
% \LaTeX.
%
% \section*{Preface to version 2.1}
%
% This version is identical to 2.0g described in \TUB\ 10\#3 except
% for some internal defaults which are now set depending on the used
% font selection scheme.
%
% This was done to avoid unpleasant surprises if the new font selection
% scheme is in force. For further details see
% section~\ref{sec:newfonts} and \cite{bk:GMS94}.
%
% \section{Introduction}
%
% For our purposes here, ``theorems'' are labelled enunciations,
% often set off from the main text by extra space and a font change.
% Theorems, corollaries, conjectures, definitions, and remarks
% are all instances of ``theorems''.  The ``header'' of these structures
% is composed of a label (such as \textsc{Theorem} or \textsc{Remark})
% and a number which serializes an item in the sequence of items
% with the same label.
%
% Shortly after the introduction of \LaTeX{} at the Fachbereich
% Mathematik in Mainz, the desire to manipulate the layout of
% ``theorems'' arose.  In Mainz, the following two conventions came
% into general use:
% \begin{enumerate}
%   \item The number of the theorem is shown in the margin.
%   \item There is a line break at
%         the end of the theorem header.
% \end{enumerate}
%
% Additionally, some journals require different formats which depend on
% the ``sort of theorem'': e.g.\ often remarks and definitions are set
% in |\upshape|, while |\itshape| is employed for main theorems.
%
% Confronted with these requirements, a theorem environment was
% developed in Mainz which allows separate determination of the
% layout of the ``theorems sets'', comparable to |\pagestyle|.
%
%
% \section{The user interface}
% \subsection{Defining new theorem sets}
%
% \DescribeMacro\newtheorem
% As in the original \LaTeX{} version, the command |\newtheorem|
% defines a new ``theorem set'' or ``theorem-like structure''.
% Two required arguments name the new environment and give the text
% to be typeset with each instance of the new ``set'', while
% an optional argument determines how
% the ``set'' is enumerated:
% \begin{description}
%    \item[\ttfamily \bslash newtheorem\{foo\}\{bar\}]
%       The theorem set {\envfont foo} (whose name is \texttt{bar})
%       uses its own counter.
%    \item[\ttfamily \bslash newtheorem\{foo2\}{[foo]}\{bar2\}]
%       The theorem set {\envfont foo2} (printed name \texttt{bar2})
%       uses the same counter as the theorem set \texttt{foo}.
%    \item[\ttfamily \bslash newtheorem\{foo3\}\{bar3\}{[section]}]
%       The theorem set {\envfont foo3} (printed name \texttt{bar3}) is
%       enumerated within the counter \texttt{section}, i.e.\ with every
%       new |\section| the enumeration begins again with 1, and
%       the enumeration is composed from the section-number and the
%       theorem counter itself.
% \end{description}
%
% \DescribeMacro\theoremstyle
% Additionally, the command |\theoremstyle| can define the layout
% of various, or all, theorem sets.  It should be noted that any
% theorem set defined by |\newtheorem| is typeset in the
% |\theoremstyle| that is current at the time of the definition.
% Thus, the following
% \begin{verbatim}
%    \theoremstyle{break}        \newtheorem{Cor}{Corollary}
%    \theoremstyle{plain}        \newtheorem{Exa}{Example}[section]
%\end{verbatim}                             ^^A No blank at line start
% leads to the result that the set {\envfont Cor} is formatted in the
% style \texttt{break}, while the set {\envfont Exa} and all the
% following ones are formatted in the style \texttt{plain}, unless
% another |\theoremstyle| follows.  Since the definitions installed by
% |\newtheorem| are global, one also can limit |\theoremstyle| locally
% by grouping braces.
%
% \DescribeMacro\theorembodyfont
% The choice of the font for the theorem body is completely
% independent of the chosen |\theoremstyle|; this has proven to
% be very advantageous.  For example,
% \begin{verbatim}
%   {\theorembodyfont{\upshape}          \newtheorem{Rem}{Remark}}
%\end{verbatim}
% defines a theorem set {\envfont Rem}, which will be set in
% |\upshape| in the current layout (which in our example is
% \texttt{plain}).  As with |\theoremstyle|, the |\theorembodyfont|
% chosen is that current at the time of |\newtheorem|.  If
% |\theorembodyfont| is not specified or one defines
% |\theorembodyfont{}|, then the font used will be that defined by the
% |\theoremstyle|.
%
% \DescribeMacro\theoremheaderfont It is also possible to customize
% the font used for the theorem headers.  This is, however, a global
% declaration, and therefore there should be at most one
% |\theoremheaderfont| declaration in the preamble.\footnote{If it is
% actually necessary to have different header fonts, one has to define
% new theorem styles (substituting the desired font) or specify the
% information directly in the \texttt{\bslash newtheorem} declaration
% (the unclean variant).}
%
% \DescribeMacro\theorempreskipamount
% \DescribeMacro\theorempostskipamount Two additional parameters
% affect the vertical space around the theorem environments:
% |\theorempreskipamount| and |\theorempostskipamount| define,
% respectively, the spacing before and after such an environment.
% These parameters apply for all theorem sets and can be manipulated
% with the ordinary length macros.  They are rubber lengths
% (`\textsf{skips}'), and therefore can contain \texttt{plus} and
% \texttt{minus} parts.
%
% Since the definition of theorem sets should---most sensibly---be
% placed in the preamble, we only allow installation there.
% It is therefore possible to release the memory used here after
% |\begin{document}|, in order to make room for other
% applications.
%
%
% \subsection{Existing theorem styles}
%
% The following theorem styles exist to date:
% \begin{deflist}{marginbreak}
%    \item[plain]
%       This theorem style emulates the original \LaTeX{} definition,
%       except that additionally the parameters
%       |\theorem...skipamount| are used.
%    \item[break]
%       In this style, the theorem header is followed by a line
%       break.
%    \item[marginbreak]
%       The theorem number is set in the margin, and there is a line
%       break as in \texttt{break}.
%    \item[changebreak]
%       Like \texttt{break}, but with header number and text
%       interchanged.
%    \item[change]
%       Header number and text are interchanged, without a line break.
%    \item[margin]
%       The number is set in the left margin, without a line break.
% \end{deflist}
% All styles (except \texttt{plain}) select |\slshape| as the default
% |\theorembodyfont|.
%
%
% \subsection{Examples}
%
% Given the above theorem sets {\envfont Cor}, {\envfont Exa}
% and {\envfont Rem},
% suppose that the preamble also contains the declarations:
% \begin{verbatim}
%    \theoremstyle{marginbreak}   \newtheorem{Lem}[Cor]{Lemma}
%    \theoremstyle{change}
%    \theorembodyfont{\itshape}        \newtheorem{Def}[Cor]{Definition}
%
%    \theoremheaderfont{\scshape}
%\end{verbatim}
% Then the following are some typical examples of the typeset output
% resulting from their use.
%
% \begin{Cor}
%    This is a sentence typeset in the theorem environment {\envfont
%    Cor}.
% \end{Cor}
%
% \begin{Exa}
%    This is a sentence typeset in the theorem environment {\envfont
%    Exa}.
% \end{Exa}
%
% \begin{Rem}
%    This is a sentence typeset in the theorem environment {\envfont
%    Rem}.
% \end{Rem}
%
% \begin{Lem}[Ben User]
%    This is a sentence typeset in the theorem environment {\envfont
%    Lem}.
% \end{Lem}
%
% \begin{Def}[Very impressive Definition]
%    This is a sentence typeset in the theorem environment {\envfont
%    Def}.
% \end{Def}
%
% The last two examples show the effect of the optional argument to
% a theorem environment (it is the text typeset in parentheses).
%
%
% \section{Special Considerations}
% \label{sec:newfonts}
%
% Theoremheader and body are implemented as a unit. This means that
% the |\theoremheaderfont|  will inherit characteristics of the
% |\theorembodyfont| in \LaTeXe. Thus, if for example
% |\theorembodyfont| is |\itshape| and |\theoremheaderfont|
% is |\bfseries| the font selected for the header will have the
% characteristics `bold extended italic'. If this is not desired one
% should set the |\theoremheaderfont| to something like
% \begin{verbatim}
%      \theoremheaderfont{\normalfont\bfseries}
%\end{verbatim}
% i.e.\ supplying all necessary font information explicitly.
%
%
%
% \section{Acknowledgements}
%
% The publication of this set of macros was only possible with the
% help of Christina Busse (translating the manuscript into English),
% Joachim Pense (playing the r\^ole of typist), Chris Rowley
% (looking everything over) and many others providing useful
% suggestions.
%
% \MaybeStop{%
%    \begin{thebibliography}{9}
%       \bibitem{bk:GMS94} \textsc{M.~Goossens}, \textsc{F.~Mittelbach}
%         and \textsc{A.~Samarin}.
%         \newblock The \LaTeX{} Companion.
%         \newblock
%         Addison-Wesley, Reading, Massachusetts, 1994.
%       \bibitem{latex:source} \textsc{Lamport, Leslie.} \newblock
%                               \texttt{latex.tex}, version 2.09, date
%                               Feb. 1990.
%    \end{thebibliography}
%    \PrintIndex
%    \PrintChanges}
%
%
% \section{The documentation driver file}
%
% The next bit of code contains the documentation driver file for
% \TeX{}, i.e., the file that will produce the documentation you are
% currently reading. It will be extracted from this file by the
% \texttt{docstrip} program. Since it is the first code in the file
% one can alternatively process this file directly with \LaTeXe{} to
% obtain the documentation.
%
%    \begin{macrocode}
%<*driver>
\documentclass{ltxdoc}

\usepackage{theorem}

%<+driver>% The next few lines define theorem sets which are used
%<+driver>% in the example section of the documentation.

\theoremstyle{break}       \newtheorem{Cor}{Corollary}
\theoremstyle{plain}       \newtheorem{Exa}{Example}[section]
{\theorembodyfont{\upshape}\newtheorem{Rem}{Remark}}
\theoremstyle{marginbreak} \newtheorem{Lem}[Cor]{Lemma}
\theoremstyle{change}
\theorembodyfont{\itshape} \newtheorem{Def}[Cor]{Definition}

\theoremheaderfont{\scshape}

\RecordChanges

\begin{document}
   \DocInput{theorem.dtx}
\end{document}
%</driver>
%    \end{macrocode}
%
% \section{Definition of the Macros}
%
% If the file has been loaded before, we abort immediately. If not the
% package announces itself (this is actually done at the very top of
% the file---the way it is done isn't good style so don't copy it).
% \changes{v2.0e}{89/07/19}{Spaces removed from `typeout'.}
% \changes{v2.2b}{95/11/19}{Announce moved to top of file}
%    \begin{macrocode}
%<*package>
%\@ifundefined{theorem@style}{}{\endinput}
%\def\FMithmInfo{1995/11/19 v2.2b Theorem extension package (FMi)}
%\ProvidesPackage{theorem}[\FMithmInfo]
%    \end{macrocode}
%
%
% \subsection{Definition of theorem styles and fonts}
%
% All the definitions in this file are done globally to allow
% inputting this file inside a group.
%
% \begin{macro}{\theoremstyle}
%    Before a theorem style can be installed, the chosen style must be
%    known. For that reason, we must test to see that
%    |\th@|\meta{style} is known or, more precisely, that it is
%    different from |\relax|. If the style is not known then
%    |\th@plain| is used.
%    \begin{macrocode}
\gdef\theoremstyle#1{%
   \@ifundefined{th@#1}{\@warning
          {Unknown theoremstyle `#1'. Using `plain'}%
          \theorem@style{plain}}%
%    \end{macrocode}
%    We save the theorem style to be used in the token register
%    |\theorem@style|.
%    \begin{macrocode}
      {\theorem@style{#1}}%
%    \end{macrocode}
%    Now we ``evaluate'' the theorem style: this means, we call the
%    macro |\th@|\meta{style} which will activate the relevant
%    definitions which are contained in a separate file.  This is done
%    in a group to suppress changes to the current font.  This could
%    otherwise pose problems together with the new font selection
%    scheme\footnote{When I printed the original article using the new
%    font selection scheme I ended with a document with slanted
%    typefaces (text headings and all) simply because one of the
%    theorem styles used \texttt{\bslash sl} at toplevel.} if the
%    |\th@|\meta{style} is evaluated a second time.
% \changes{v2.1a}{90/02/15}{Group added around `th@..}
% \changes{v2.1a}{90/02/15}{Reseting of `..@begintheorem removed.}
%    \begin{macrocode}
      \begingroup
        \csname th@\the\theorem@style \endcsname
      \endgroup}
%    \end{macrocode}
% \end{macro}
%
%
% \begin{macro}{\@begintheorem}
% \changes{v2.1a}{90/02/15}{Relaxed at toplevel.}
% \begin{macro}{\@opargbegintheorem}
% \changes{v2.1a}{90/02/15}{Relaxed at toplevel.}
%    We reset |\@begintheorem| and |\@opargbegintheorem| to
%    |\relax| since these commands are no longer necessary at
%    toplevel. This will save a few tokens.
%    \begin{macrocode}
\global\let\@begintheorem\relax
\global\let\@opargbegintheorem\relax
%    \end{macrocode}
% \end{macro}
% \end{macro}
%
% \begin{macro}{\theorem@style}
%    Obviously the \textsf{token} register used above has to be
%    allocated. To assure the utmost compatibility with the
%    original \LaTeX{} definition, we set the default theorem style to
%    \texttt{plain}, which implements the usual \LaTeX{} convention.
% \changes{v2.0f}{89/07/20}{Unnecessary `global' removed.}
%    \begin{macrocode}
\newtoks\theorem@style
\global\theorem@style{plain}
%    \end{macrocode}
% \end{macro}
%
% \begin{macro}{\theorembodyfont}
% \begin{macro}{\theorem@bodyfont}
%    If the |\theorembodyfont| is set by the user then it should not
%    interact with the default font set in the theorem style. When the
%    new font selection is in force this may happen if, for example,
%    the default is |\itshape| and the new |\theorembodyfont| is
%    |\sffamily|. So we add a |\reset@font| command in front of the
%    user definition.
% \changes{v2.0f}{89/07/20}{Unnecessary `global' removed.}
% \changes{v2.0g}{89/09/19}{`theoremfont to `theorembodyfont.}
% \changes{v2.1c}{92/06/22}{`reset@font added and logic changed.}
% \changes{v2.1d}{92/10/28}{Reset to empty if arg is empty}
%    \begin{macrocode}
\gdef\theorembodyfont#1{%
%    \end{macrocode}
%    We check if the argument supplied is empty and if so put nothing
%    into the |\theorem@bodyfont| token register to allow for
%    |\theorembodyfont{}| as a mean of using the default of the
%    current |\theoremstyle|.
%    \begin{macrocode}
   \def\@tempa{#1}%
   \ifx\@tempa\@empty
    \theorem@bodyfont{}%
   \else
    \theorem@bodyfont{\reset@font#1}%
   \fi
   }
\newtoks\theorem@bodyfont
\global\theorem@bodyfont{}
%    \end{macrocode}
% \end{macro}
% \end{macro}
%
%
% \begin{macro}{\theoremheaderfont}
% \changes{v2.0g}{89/09/19}{Macro added.}
%    The font for the theorem headers is handled differently because
%    this definition applies to all theorem styles.
%    \begin{macrocode}
\gdef\theoremheaderfont#1{\gdef\theorem@headerfont{#1}%
%    \end{macrocode}
%    After using the macro once it is redefined to produce an error
%    message.
%    \begin{macrocode}
       \gdef\theoremheaderfont##1{%
        \typeout{\string\theoremheaderfont\space should be used
                 only once.}}}
%    \end{macrocode}
% \end{macro}
%
% \begin{macro}{\theorem@headerfont}
% \changes{v2.0g}{89/09/19}{Macro added.}
%    To set the |\theorem@headerfont| default we first test if
%    the new fontselection scheme is in force.
% \changes{v2.1a}{90/02/15}{Support for new font selection.}
%    \begin{macrocode}
\ifx\upshape\undefined
%    \end{macrocode}
%    If not we define it to expand into |\bfseries|. We don't use
%    |\let| just in case a following style option redefines this
%    macro.
%    \begin{macrocode}
\gdef\theorem@headerfont{\bfseries}
%    \end{macrocode}
%    Otherwise we reset the current shape before calling |\bfseries|.
%    \begin{macrocode}
\else \gdef\theorem@headerfont{\normalfont\bfseries}\fi
%    \end{macrocode}
% \end{macro}
%
% \begin{macro}{\th@plain}
% \begin{macro}{\th@break}
% \begin{macro}{\th@marginbreak}
% \begin{macro}{\th@changebreak}
% \begin{macro}{\th@change}
% \begin{macro}{\th@margin}
%    The different styles are defined in macros such as
%    |\th@plain|.  Since memory space is precious in
%    ``non-Big-versions'', we have to avoid offering too many unused
%    definitions.  Therefore we define these styles in separate files
%    that can be loaded on demand.  Thus the commands themselves only
%    load these files. We use |\@input@|, a \LaTeXe{} internal command
%    that ensures that the file will be listed with |\listfiles|.
% \changes{v2.2a}{94/02/02}{Use `@input@ to load theorem layout files}
%    \begin{macrocode}
\gdef\th@plain{\@input@{thp.sty}}
\gdef\th@break{\@input@{thb.sty}}
\gdef\th@marginbreak{\@input@{thmb.sty}}
\gdef\th@changebreak{\@input@{thcb.sty}}
\gdef\th@change{\@input@{thc.sty}}
\gdef\th@margin{\@input@{thm.sty}}
%    \end{macrocode}
%    This list will be expanded when new styles become
%    available.  For testing,  just append new theorem substyles
%    as document options.
% \end{macro}
% \end{macro}
% \end{macro}
% \end{macro}
% \end{macro}
% \end{macro}
%
%
%
% \subsection{Definition of a new theorem set}
%
% As already pointed out, a new theorem environment can be defined in
% three different ways:
% \begin{verbatim}
%    \newtheorem{Lem}{Lemma}
%    \newtheorem{Lem}{Lemma}[section]
%    \newtheorem{Lem}[Theorem]{Lemma}
%\end{verbatim}
% The function of the macro |\newtheorem| is to recognize these cases
% and then to branch into one of the three macros |\@ynthm|, |\@xnthm|
% or |\@othm|.  This mechanism is adopted unchanged from
% \cite{latex:source}; the essential point here is that, for example,
% in the second case, the arguments \texttt{Lem}, \texttt{Lemma} and
% \texttt{section} are passed over to the macro |\@xnthm|.
%
% We inspect this case first because the others present fewer problems,
% and thus are easily derived from this one.
%
% \begin{macro}{\@xnthm}
%    For our example arguments, the macro |\@xnthm| must fulfill
%    the following:
%    \begin{itemize}
%       \item
%          Define a new \LaTeX-counter `Lem'
%       \item
%          reset this counter within a |\section|
%       \item
%          define the macro |\theLem|
%       \item
%          define the environment macros |\Lem| and
%          |\endLem| using the current
%          |\theoremstyle| and |\theorem@bodyfont|.
%    \end{itemize}
%    Obviously, all this should happen only if the first argument
%    of |\@xnthm| (i.e.\ \texttt{Lem} in our example) is chosen so
%    as not to conflict with any previously defined commands or
%    environments.  This test is performed by the \LaTeX{} macro
%    |\@ifdefinable|.
%    \begin{macrocode}
\gdef\@xnthm#1#2[#3]{\expandafter\@ifdefinable\csname #1\endcsname
%    \end{macrocode}
%    Therefore, the first argument of |\@ifdefinable| is the
%    expansion (in the example, |\Lem|) of
%    |\csname||#1\endcsname|.  The second argument is
%    executed only if the test has been completed successfully.
%    \begin{macrocode}
   {%
%    \end{macrocode}
%    Now we define the new counter. The names of the \LaTeX{}
%    macros employed should speak for themselves:
% \changes{v2.2b}{1995/10/07}{Use \cs{@newctr}}
% \changes{v2.2c}{1995/11/23}{Use \cs{@newctr} properly.}
%    \begin{macrocode}
    \@definecounter{#1}\@newctr{#1}[#3]%
%    \end{macrocode}
%    Using |\@newctr| will give a proper error message if the counter
%    in |#3| is not defined.
%    In defining `|\theLem|' we must generate the desired macro
%    name by use of |\expandafter| and |\csname|.
%    \begin{macrocode}
    \expandafter\xdef\csname the#1\endcsname
%    \end{macrocode}
%    An |\xdef| is used in order to make the definition global,
%    and to ensure that it contains the replacement texts of
%    |\@thmcountersep| and |\@thmcounter|.\footnote{These
%    two macros can be defined by the document style.  Their default
%    values produce a `\texttt{.}' as separation and an arabic
%    representation of the number.} However, not everything should be
%    expanded. For example, it saves space to use |\thesection|
%    instead of its---at times---lengthy expansion.
%    \begin{macrocode}
      {\expandafter \noexpand \csname the#3\endcsname
       \@thmcountersep \@thmcounter{#1}}%
%    \end{macrocode}
%    Thus with the defaults of \LaTeX{}, |\theLem| would be
%    replaced by the command sequence |\thesection.\arabic{Lem}|.
%
%    We will now look at the definition of the macro which is executed
%    at the beginning of the actual environment (in our example this
%    macro is |\Lem|). It should be noted that we use an
%    ``|\expandafter| trick'' to expand only certain parts of the
%    replacement text at the time of the definition.
%    \begin{macrocode}
    \def\@tempa{\global\@namedef{#1}}%
    \expandafter \@tempa \expandafter{%
%    \end{macrocode}
%    First, the macro that contains the current definitions of
%    |\@begintheorem| and |\@opargtheorem| should be called
%    up.  The name of this macro---as is already known---has the form
%    |\th@|\meta{theorem\ style}; therefore, it must be called by
%    \begin{macrocode}
      \csname th@\the \theorem@style
%    \end{macrocode}
%    In addition the default theorem font should be changeable, i.e.\
%    we have to insert the contents of |\theorem@bodyfont|. For
%    that reason, we expand even further, beyond |\endcsname|,
%    and thus insert the contents of the \textsf{token} register
%    |\theorem@bodyfont| in the replacement text.
%    \begin{macrocode}
            \expandafter \endcsname \the \theorem@bodyfont
%    \end{macrocode}
%    Now it is time to call the macro |\@thm| which takes over
%    the further processing.  It has two arguments: the current
%    counter name (in our example, \texttt{Lem}), and the text of the
%    label (in our example, \texttt{Lemma}).
%    \begin{macrocode}
     \@thm{#1}{#2}}%
%    \end{macrocode}
%    With this, the `sub-definition' is complete.  The macro
%    |\@endtheorem| ends a theorem environment and is, so far, nothing
%    but an |\endtrivlist|.  (Hence it is defined globally, and not
%    within the theorem styles.\footnote{This has to be changed as
%    soon as theorem styles that change \texttt{\bslash @endtheorem}
%    exist.  In such a case, all existing styles must be changed as
%    well since they will have to reset the macro.}) Therefore, we can
%    set it equivalent to the macro that ends the theorem set (in our
%    example, |\endLem|).  However, if some day theorem styles exist
%    that do change |\@endtheorem|, we would have to use the
%    commented-out line instead.
%    \begin{macrocode}
    \global \expandafter \let \csname end#1\endcsname \@endtheorem
%   \global\@namedef{end#1}{\@endtheorem}%
%    \end{macrocode}
%    With these commands all the required definitions are employed,
%    unless the test |\@ifdefinable| has failed.  Therefore, we
%    end the second argument of this macro and with it the definition
%    of |\@xnthm|.
%    \begin{macrocode}
   }}
%    \end{macrocode}
% \end{macro}
%
% \begin{macro}{\@ynthm}
%    The definition of |\@ynthm| is completely analogous. In
%    this case the new counter that is defined is not reset within
%    another counter; thus the definition of |\the...|
%    is simplified:
%    \begin{macrocode}
\gdef\@ynthm#1#2{\expandafter\@ifdefinable\csname #1\endcsname
   {\@definecounter{#1}%
    \expandafter\xdef\csname the#1\endcsname{\@thmcounter{#1}}%
%    \end{macrocode}
%    The rest of the definition corresponds literally to that of
%    |\@xnthm|:
%    \begin{macrocode}
    \def\@tempa{\global\@namedef{#1}}\expandafter \@tempa
     \expandafter{\csname th@\the \theorem@style \expandafter
     \endcsname \the\theorem@bodyfont \@thm{#1}{#2}}%
    \global \expandafter \let \csname end#1\endcsname \@endtheorem}}
%    \end{macrocode}
% \end{macro}
%
% \begin{macro}{\@othm}
%    The definition of |\@othm| does not contain anything new.
%    \begin{macrocode}
\gdef\@othm#1[#2]#3{%
%    \end{macrocode}
% \changes{v2.2b}{95/10/07}{Build in check for counter defined}
%    We do not define a new counter but instead use one that has
%    already been defined.  Thus the only definition we need is that
%    of this pseudo-counter (i.e.\ |\the|\meta{env.\ name}).
%    First we check if |#2| corresponds to a known counter name.
%    \begin{macrocode}
  \expandafter\ifx\csname c@#2\endcsname\relax
   \@nocounterr{#2}%
  \else
   \expandafter\@ifdefinable\csname #1\endcsname
   {\expandafter \xdef \csname the#1\endcsname
     {\expandafter \noexpand \csname the#2\endcsname}%
%    \end{macrocode}
%    All other parts of the definition can be adopted from
%    |\@xnthm|. We have to remember, though, that in this case
%    the name of the current counter and the theorem label have moved
%    to the second and third arguments.
%    \begin{macrocode}
    \def\@tempa{\global\@namedef{#1}}\expandafter \@tempa
     \expandafter{\csname th@\the \theorem@style \expandafter
     \endcsname \the\theorem@bodyfont \@thm{#2}{#3}}%
    \global \expandafter \let \csname end#1\endcsname \@endtheorem}%
  \fi}
%    \end{macrocode}
% \end{macro}
%
% \subsection{Macros that are employed in a {\envfont theorem}
%    environment}
%
% \begin{macro}{\@thm}
%    The macro |\@thm| has to increase the current counter.  Then,
%    depending on whether the environment has (or does not have) an
%    optional argument, it has to branch into either |\@begintheorem|
%    or |\@opargtheorem|.
%    \begin{macrocode}
\gdef\@thm#1#2{\refstepcounter{#1}%
%    \end{macrocode}
%    Now we start a {\envfont trivlist} environment, and give
%    |\@topsep| and |\@topsepadd| the values of the \textsf{skip}
%    registers |\theorempreskipamount| and |\theorempostskipamount|.
%    The value in |\@topsep| is the vertical space that is inserted by
%    the first (and only) |\item| in our |\trivlist| whilst
%    |\@topsepadd| is inserted by |\@endparenv| at the end of that
%    {\envfont trivlist} environment.  By using these registers, we
%    obtain the desired space around a {\envfont theorem} environment.
%    \begin{macrocode}
   \trivlist
   \@topsep \theorempreskipamount               % used by first \item
   \@topsepadd \theorempostskipamount           % used by \@endparenv
%    \end{macrocode}
%    Now we have to test whether an optional argument has been given.
% \changes{v2.0d}{89/07/18}{Forgotten percent added.}
%    \begin{macrocode}
   \@ifnextchar [%
%    \end{macrocode}
%    If there is an optional argument, we will call |\@ythm|, and
%    move the arguments read back into the input stream.
%    \begin{macrocode}
   {\@ythm{#1}{#2}}%
%    \end{macrocode}
%    If not, we call |\@begintheorem|. Its first argument is the
%    name of the theorem set (hence the second argument of
%    |\@thm|). Its second argument is the macro that produces the
%    current number.
%    \begin{macrocode}
   {\@begintheorem{#2}{\csname the#1\endcsname}\ignorespaces}}
%    \end{macrocode}
% \end{macro}
%
% \begin{macro}{\@xthm}
% \begin{macro}{\@ythm}
%    Both these macros were originally called by |\@thm|.  We do
%    not need |\@xthm| anymore, hence we reset it to
%    |\relax|. The definition of |\@ythm| has not changed at
%    all from its definition in \LaTeX. In order to make the macros
%    easier to understand, we will
%    nevertheless present it (commented out).
%    \begin{macrocode}
\global\let\@xthm\relax
% \def\@ythm#1#2[#3]{\@opargbegintheorem{#2}{\csname
%           the#1\endcsname}{#3}\ignorespaces}
%</package>
%    \end{macrocode}
%    The primitive |\ignorespaces| in |\@ythm| and
%    |\@thm| is needed to remove the spaces between the
%    |\begin{...}| and the actual text.
% \end{macro}
% \end{macro}
%
%
% \subsection{Definition of the theorem substyles}
%
% As already pointed out, the theorem substyles, defined below, are
% only loaded when necessary.  Note that all these substyles, except
% \texttt{plain}, have |\slshape| as the default body font.
%
% \subsubsection{The \texttt{plain} style}
%
% \label{dok:th@plain} As the following macros use |@|, we have
% to locally set the |\catcode| of this symbol to ``letter''.
% This happens within a group, so that we do not have to worry about
% which |\catcode| that symbol had before.
%    \begin{macrocode}
%<*thp>
\begingroup \makeatletter
%    \end{macrocode}
% Since we are now within a group, we must make all definitions
% globally.  First we make sure that \texttt{theorem.sty} is loaded.
% This will allow us to use this file as a document style option
% without having to call \texttt{theorem} itself as an option.  At the
% same time, we assure that at least version 2 is loaded, since
% |\theorem@style| was not defined in earlier versions.
%    \begin{macrocode}
\@ifundefined{theorem@style}{\input{theorem.sty}}{}
\ProvidesFile{thp.sty}
             [\FMithmInfo]
%    \end{macrocode}
%
%
% \begin{macro}{\th@plain}
% \begin{macro}{\@begintheorem}
% \begin{macro}{\@opargbegintheorem}
%    |\theoremstyle{plain}| corresponds to the original
%    definition, except that the distances to the surrounding text are
%    determined by the parameters |\theorempreskipamount| and
%    |\theorempostskipamount|.
%    First we set the default body font.
%    \begin{macrocode}
\gdef\th@plain{\normalfont\itshape
%    \end{macrocode}
%    Then we define |\@begintheorem| and
%    |\@opargbegintheorem|. These two macros define how the
%    header of a theorem is typeset.  |\@opargbegintheorem| will
%    be called if a \textsf{theorem} environment with an optional
%    argument is encountered; otherwise, the header is constructed by
%    calling |\@begintheorem|.  If one of these macros is
%    executed, we are within a \textsf{trivlist} environment started by
%    |\@thm|.  So the theorem header is produced with an
%    |\item| command.
%
%    Instead of specifying the header font directly, all standard
%    theorem styles use the |\theorem@headerfont| macro to allow
%    customization.  The extra space (|\labelsep|) is necessary
%    because of problems in the \textsf{trivlist} environment.
%    \begin{macrocode}
  \def\@begintheorem##1##2{%
        \item[\hskip\labelsep \theorem@headerfont ##1\ ##2]}%
%    \end{macrocode}
%    The definition of |\@opargbegintheorem| is completely
%    analogous. The only difference is the fact that there exists a
%    third argument (which is the optional parameter of the
%    environment and contains additional information about the
%    theorem).  Customarily we enclose it in parentheses.
%    \begin{macrocode}
\def\@opargbegintheorem##1##2##3{%
   \item[\hskip\labelsep \theorem@headerfont ##1\ ##2\ (##3)]}}
%    \end{macrocode}
% \end{macro}
% \end{macro}
% \end{macro}
%
% \noindent
% We conclude with an |\endgroup| to restore the
% |\catcode| of |@|.
%    \begin{macrocode}
\endgroup
%</thp>
%    \end{macrocode}
%
%
% \subsubsection{The \texttt{break} style}
%
% This style option is stored in the file |thb.sty|.
% For the next two lines see the documentation for |\th@plain| on
% page~\pageref{dok:th@plain}.
%    \begin{macrocode}
%<*thb>
\begingroup \makeatletter
\@ifundefined{theorem@style}{\input{theorem.sty}}{}
\ProvidesFile{thb.sty}
             [\FMithmInfo]
%    \end{macrocode}
%
% \begin{macro}{\th@break}
%    |\theoremstyle{break}| produces a line break after the name
%    of the theorem.  The font is |\slshape|.  Hence, we define
%    |\th@break| as follows:
%    \begin{macrocode}
\gdef\th@break{\normalfont\slshape
  \def\@begintheorem##1##2{\item[%
%    \end{macrocode}
%    We run into the following problem: it is not possible to create
%    the header with |\item[|\meta{title}|]| and then start
%    a new line by, for example, |\mbox{}\\|. Such a definition
%    will fail whenever a list environment follows immediately.  With
%    the above construction, the |\mbox{}| causes the switch
%    |@inlabel| (cf.\ definition of |\list| and
%    |\trivlist| in \cite{latex:source}) to be set to \texttt{false}
%    and so the following list will insert additional vertical
%    space (|\topskip|). This is quite annoying. Therefore, we
%    create the line break within the |\item|. In order to ensure
%    that the text will begin at the proper position in the following
%    line, we simply pretend that the label does not take any
%    room.\footnote{This will lead to problems whenever very high
%    symbols occurring in the line tower into the
%    heading. So, something else has to be done here sometime.}
%    \begin{macrocode}
       \rlap{\vbox{\hbox{\hskip \labelsep\theorem@headerfont ##1\ ##2}%
                   \hbox{\strut}}}]}%
%    \end{macrocode}
%    Again, the definition of |\@opargbegintheorem| is
%    completely analogous.
%    \begin{macrocode}
\def\@opargbegintheorem##1##2##3{%
  \item[\rlap{\vbox{\hbox{\hskip \labelsep \theorem@headerfont
                     ##1\ ##2\ (##3)}%
                    \hbox{\strut}}}]}}
\endgroup
%</thb>
%    \end{macrocode}
% \end{macro}
%
%
% \subsubsection{The \texttt{changebreak} style}
%
%    \begin{macrocode}
% This style option is stored in the file |thcb.sty|.
%    \begin{macrocode}
%<*thcb>
\begingroup \makeatletter
\@ifundefined{theorem@style}{\input{theorem.sty}}{}
\ProvidesFile{thcb.sty}
             [\FMithmInfo]
%    \end{macrocode}
%
% \begin{macro}{\th@changebreak}
%    The \texttt{change-break} theorem style is like \texttt{break}
%    but with interchange of theorem name and theorem number.  Thus we
%    define |\th@changebreak| as follows:
%    \begin{macrocode}
\gdef\th@changebreak{\normalfont\slshape
  \def\@begintheorem##1##2{\item
       [\rlap{\vbox{\hbox{\hskip\labelsep \theorem@headerfont ##2\ ##1}%
                    \hbox{\strut}}}]}%
\def\@opargbegintheorem##1##2##3{%
  \item[\rlap{\vbox{\hbox{\hskip\labelsep \theorem@headerfont
                     ##2\ ##1\ (##3)}%
                    \hbox{\strut}}}]}}
\endgroup
%</thcb>
%    \end{macrocode}
% \end{macro}
%
%
% \subsubsection{The \texttt{change} style}
%
% This style option is stored in the file |thc.sty|.
%    \begin{macrocode}
%<*thc>
\begingroup \makeatletter
\@ifundefined{theorem@style}{\input{theorem.sty}}{}
\ProvidesFile{thc.sty}
             [\FMithmInfo]
%    \end{macrocode}
%
% \begin{macro}{\th@change}
%    The \texttt{change} theorem style corresponds to the
%    \texttt{change break} style without a linebreak after the header.
%    To say it in another way, it's the same as the \texttt{plain}
%    style but with number and name interchanged and |\slshape| as the
%    default font.
%    \begin{macrocode}
\gdef\th@change{\normalfont\slshape
  \def\@begintheorem##1##2{\item
       [\hskip\labelsep \theorem@headerfont ##2\ ##1]}%
\def\@opargbegintheorem##1##2##3{%
  \item[\hskip\labelsep \theorem@headerfont ##2\ ##1\ (##3)]}}
\endgroup
%</thc>
%    \end{macrocode}
% \end{macro}
%
%
% \subsubsection{The \texttt{marginbreak} style}
%
% This style option is the one used most often at Mainz.  It is saved
% in the file |thmb.sty|.
%    \begin{macrocode}
%<*thmb>
\begingroup \makeatletter
\@ifundefined{theorem@style}{\input{theorem.sty}}{}
\ProvidesFile{thmb.sty}
             [\FMithmInfo]
%    \end{macrocode}
%
% \begin{macro}{\th@marginbreak}
%    The \texttt{margin break} style is nearly the same as the
%    \texttt{change break} style. The only difference is the placement
%    of the theorem number.  We use |\llap| to place it in the left
%    margin.
%
%    In this style |\labelsep| denotes the separation between the
%    number and the text.
%    \begin{macrocode}
\gdef\th@marginbreak{\normalfont\slshape
  \def\@begintheorem##1##2{\item
       [\rlap{\vbox{\theorem@headerfont
                    \hbox{\llap{##2}\hskip\labelsep ##1}%
                    \hbox{\strut}}}]}%
\def\@opargbegintheorem##1##2##3{%
  \item[\rlap{\vbox{\theorem@headerfont
                    \hbox{\llap{##2}\hskip\labelsep ##1\ (##3)}%
                    \hbox{\strut}}}]}}
\endgroup
%</thmb>
%    \end{macrocode}
% \changes{v2.1c}{90/02/17}{`endgroup outside macrocode env.}
% \end{macro}
%
%
%
% \subsubsection{The \texttt{margin} style}
%
% This style option is stored in the file |thm.sty|.
%    \begin{macrocode}
%<*thm>
\begingroup \makeatletter
\@ifundefined{theorem@style}{\input{theorem.sty}}{}
\ProvidesFile{thm.sty}
             [\FMithmInfo]
%    \end{macrocode}
%
% \begin{macro}{\th@margin}
%    Again this is only a variant of the theorem styles described
%    above without any new ideas.
%    \begin{macrocode}
\gdef\th@margin{\normalfont\slshape
  \def\@begintheorem##1##2{\item
       [\theorem@headerfont \llap{##2}\hskip\labelsep ##1]}%
\def\@opargbegintheorem##1##2##3{%
  \item[\theorem@headerfont \llap{##2}\hskip\labelsep ##1\ (##3)]}}
\endgroup
%</thm>
%    \end{macrocode}
% \end{macro}
%
%
%
%
% \subsection{Final Definitions}
%
% \begin{macro}{\theorempreskipamount}
% \begin{macro}{\theorempostskipamount}
%    The \textsf{skip} parameters that regulate the vertical empty space
%    before and after the {\envfont theorem} environment have
%    to be allocated as well.
% \changes{v2.0f}{89/07/20}{Unnecessary `global' removed.}
%    \begin{macrocode}
%<*package>
\newskip\theorempreskipamount
\newskip\theorempostskipamount
%    \end{macrocode}
%    Since we have used the same values for all theorem sets, we now can
%    assign them.
%    \begin{macrocode}
\global\setlength\theorempreskipamount{12pt plus 5pt minus 3pt}
\global\setlength\theorempostskipamount{8pt plus 3pt minus 1.5pt}
%    \end{macrocode}
% \end{macro}
% \end{macro}
%
%
% \begin{macro}{\@endtheorem}
%    The same holds for the macro |\@endtheorem|, which ends a
%    {\envfont theorem} environment. Since it is the same for all
%    theorem sets, it is removed from the macros |\th@|\meta{style}.
%    It simply ends the {\envfont trivlist} environment, which was
%    begun in |\@thm|.
%    \begin{macrocode}
\global\let\@endtheorem=\endtrivlist
%    \end{macrocode}
% \end{macro}
%
%
%
% \begin{macro}{\@preamblecmds}
%    All macros defined above are to be used only in the preamble.
%    Therefore, we insert them in |\@preamblecmds| which will disable
%    them at begin document. This is done by the internal \LaTeXe{}
%    command |\@onlypreamble|.
% \changes{v2.0g}{89/09/19}{`theoremheaderfont and `theorembodyfont
%    added.}
% \changes{v2.2a}{94/02/03}{Use \LaTeXe{} interface}
%    \begin{macrocode}
\@onlypreamble\@xnthm
\@onlypreamble\@ynthm
\@onlypreamble\@othm
\@onlypreamble\newtheorem
\@onlypreamble\theoremstyle
\@onlypreamble\theorembodyfont
\@onlypreamble\theoremheaderfont
%    \end{macrocode}
% \end{macro}
%
% Finally we declare the \texttt{plain} theorem style to be the default.
% \changes{v2.1b}{90/06/04}{Fixed missing `item by declaring default
%    style.}
%    \begin{macrocode}
\theoremstyle{plain}
%</package>
%    \end{macrocode}
%
% \Finale

\endinput
}{}
\ProvidesFile{thp.sty}
             [\FMithmInfo]
%    \end{macrocode}
%
%
% \begin{macro}{\th@plain}
% \begin{macro}{\@begintheorem}
% \begin{macro}{\@opargbegintheorem}
%    |\theoremstyle{plain}| corresponds to the original
%    definition, except that the distances to the surrounding text are
%    determined by the parameters |\theorempreskipamount| and
%    |\theorempostskipamount|.
%    First we set the default body font.
%    \begin{macrocode}
\gdef\th@plain{\normalfont\itshape
%    \end{macrocode}
%    Then we define |\@begintheorem| and
%    |\@opargbegintheorem|. These two macros define how the
%    header of a theorem is typeset.  |\@opargbegintheorem| will
%    be called if a \textsf{theorem} environment with an optional
%    argument is encountered; otherwise, the header is constructed by
%    calling |\@begintheorem|.  If one of these macros is
%    executed, we are within a \textsf{trivlist} environment started by
%    |\@thm|.  So the theorem header is produced with an
%    |\item| command.
%
%    Instead of specifying the header font directly, all standard
%    theorem styles use the |\theorem@headerfont| macro to allow
%    customization.  The extra space (|\labelsep|) is necessary
%    because of problems in the \textsf{trivlist} environment.
%    \begin{macrocode}
  \def\@begintheorem##1##2{%
        \item[\hskip\labelsep \theorem@headerfont ##1\ ##2]}%
%    \end{macrocode}
%    The definition of |\@opargbegintheorem| is completely
%    analogous. The only difference is the fact that there exists a
%    third argument (which is the optional parameter of the
%    environment and contains additional information about the
%    theorem).  Customarily we enclose it in parentheses.
%    \begin{macrocode}
\def\@opargbegintheorem##1##2##3{%
   \item[\hskip\labelsep \theorem@headerfont ##1\ ##2\ (##3)]}}
%    \end{macrocode}
% \end{macro}
% \end{macro}
% \end{macro}
%
% \noindent
% We conclude with an |\endgroup| to restore the
% |\catcode| of |@|.
%    \begin{macrocode}
\endgroup
%</thp>
%    \end{macrocode}
%
%
% \subsubsection{The \texttt{break} style}
%
% This style option is stored in the file |thb.sty|.
% For the next two lines see the documentation for |\th@plain| on
% page~\pageref{dok:th@plain}.
%    \begin{macrocode}
%<*thb>
\begingroup \makeatletter
\@ifundefined{theorem@style}{% \iffalse meta-comment
%
% Copyright (C) 1993-2024
%
% The LaTeX Project and any individual authors listed elsewhere
% in this file.
%
% This file is part of the Standard LaTeX `Tools Bundle'.
% -------------------------------------------------------
%
% It may be distributed and/or modified under the
% conditions of the LaTeX Project Public License, either version 1.3c
% of this license or (at your option) any later version.
% The latest version of this license is in
%    https://www.latex-project.org/lppl.txt
% and version 1.3c or later is part of all distributions of LaTeX
% version 2005/12/01 or later.
%
% The list of all files belonging to the LaTeX `Tools Bundle' is
% given in the file `manifest.txt'.
%
% \fi
%
%%
%
%
\def\FMithmInfo{2023/07/05 v2.2c Theorem extension package (FMi)}
%
% \ProvidesFile{theorem.dtx}[\FMithmInfo]
%\iffalse   % this is a METACOMMENT !
%%
%% Package `theorem' to use with LaTeX2e
%% Copyright (C) 1989-1995 Frank Mittelbach, all rights reserved.
%<+package>\@ifundefined{theorem@style}{}{\endinput}
%<+package>\NeedsTeXFormat{LaTeX2e}[1995/06/01]
%<+package>\ProvidesPackage{theorem}
%<+package>         [\FMithmInfo]
%
% \fi
%
% \changes{v1.0}{87/05/07}{First draft.}
% \changes{v1.0a}{87/05/07}{`thistheoremstyle' added.}
% \changes{v1.1a}{87/07/29}{New concept: redefine the internal
%                          \protect\LaTeX{} commands `@xnthm' etc.}
% \changes{v1.1b}{87/07/31}{`theoremfont' used to specify font.}
% \changes{v1.2a}{88/01/08}{`@endtheorem' now globally defined.}
% \changes{v1.2b}{88/01/11}{`newtheorem' only allowed in the preamble.}
% \changes{v1.3a}{88/02/14}{`theoremfont' now a toks register.}
%
% \changes{v2.0a}{88/11/16}{New implementation and new documentation.}
% \changes{v2.0a}{88/11/16}{No layout changes but internal modifications
%                           old substyles won't work.}
% \changes{v2.0b}{89/01/30}{`global' used for all definitions.}
% \changes{v2.0c}{89/06/15}{Documentation translated to English.
%                            Thanks to LLL and Nina}
% \changes{v2.1a}{90/02/15}{Support for new font selection.}
% \changes{v2.2a}{94/02/03}{Cleanup for LaTeX2e}
%
% \DoNotIndex{\@endparenv,\@ifdefinable,\@ifnextchar,\@ifundefined}
% \DoNotIndex{\@namedef,\@nameuse,\@spaces,\@tempa,\@warning}
% \DoNotIndex{\ }
% \DoNotIndex{\begingroup,\bfseries}
% \DoNotIndex{\csname}
% \DoNotIndex{\def,\do,\docdate}
% \DoNotIndex{\endcsname,\endgroup,\endinput,\endtrivlist,\expandafter}
% \DoNotIndex{\filedate,\fileversion}
% \DoNotIndex{\gdef,\global}
% \DoNotIndex{\hbox,\hskip}
% \DoNotIndex{\ignorespaces,\input,\itshape,\item}
% \DoNotIndex{\labelsep,\let,\llap}
% \DoNotIndex{\makeatletter}
% \DoNotIndex{\noexpand}
% \DoNotIndex{\refstepcounter,\relax,\rlap}
% \DoNotIndex{\setlength,\slshape,\space,\string,\strut}
% \DoNotIndex{\the,\trivlist,\typeout}
% \DoNotIndex{\vbox}
% \DoNotIndex{\xdef}
%
% \def\deflabel#1{\ttfamily #1\hfill}
% \def\deflist#1{\begin{list}{}{\settowidth\labelwidth{\ttfamily #1}%
%                               \setlength\leftmargin\labelwidth
%                               \addtolength\leftmargin\labelsep
%                               \let\makelabel\deflabel}}
% \def\enddeflist{\end{list}}
% \def\envfont{\normalfont\ttfamily}
%
% \MakeShortVerb{\|}
%
% \newcommand{\TUB}{{\sl TUGboat\/}}
%
% \GetFileInfo{theorem.dtx}
% \title{An Extension of the
%        \LaTeX{} theorem environment\thanks{This file
%        has version number \fileversion, last
%        revised \filedate.}}
% \author{Frank Mittelbach}
%
% \MaintainedByLaTeXTeam{tools}
% \maketitle
%
% \begin{abstract}
%    The macros described in this paper yield an extension of the
%    \LaTeX{} theorem mechanism. It is designed to satisfy the
%    different requirements of various journals. Thus, the
%    layout of the ``theorems'' can be manipulated by determining a
%    ``style''. This article describes not only the use, but
%    also the definition, of the necessary macros.
% \end{abstract}
%
% \section*{Preface to version 2.2}
%
% For \LaTeXe{} this package did not need any fundamental changes. I
% only modified the messages generated so that theorem layout styles
% will show up with the |\listfiles| command and cleaned the section
% on the New Font Selection Scheme since this is now included in
% \LaTeX.
%
% \section*{Preface to version 2.1}
%
% This version is identical to 2.0g described in \TUB\ 10\#3 except
% for some internal defaults which are now set depending on the used
% font selection scheme.
%
% This was done to avoid unpleasant surprises if the new font selection
% scheme is in force. For further details see
% section~\ref{sec:newfonts} and \cite{bk:GMS94}.
%
% \section{Introduction}
%
% For our purposes here, ``theorems'' are labelled enunciations,
% often set off from the main text by extra space and a font change.
% Theorems, corollaries, conjectures, definitions, and remarks
% are all instances of ``theorems''.  The ``header'' of these structures
% is composed of a label (such as \textsc{Theorem} or \textsc{Remark})
% and a number which serializes an item in the sequence of items
% with the same label.
%
% Shortly after the introduction of \LaTeX{} at the Fachbereich
% Mathematik in Mainz, the desire to manipulate the layout of
% ``theorems'' arose.  In Mainz, the following two conventions came
% into general use:
% \begin{enumerate}
%   \item The number of the theorem is shown in the margin.
%   \item There is a line break at
%         the end of the theorem header.
% \end{enumerate}
%
% Additionally, some journals require different formats which depend on
% the ``sort of theorem'': e.g.\ often remarks and definitions are set
% in |\upshape|, while |\itshape| is employed for main theorems.
%
% Confronted with these requirements, a theorem environment was
% developed in Mainz which allows separate determination of the
% layout of the ``theorems sets'', comparable to |\pagestyle|.
%
%
% \section{The user interface}
% \subsection{Defining new theorem sets}
%
% \DescribeMacro\newtheorem
% As in the original \LaTeX{} version, the command |\newtheorem|
% defines a new ``theorem set'' or ``theorem-like structure''.
% Two required arguments name the new environment and give the text
% to be typeset with each instance of the new ``set'', while
% an optional argument determines how
% the ``set'' is enumerated:
% \begin{description}
%    \item[\ttfamily \bslash newtheorem\{foo\}\{bar\}]
%       The theorem set {\envfont foo} (whose name is \texttt{bar})
%       uses its own counter.
%    \item[\ttfamily \bslash newtheorem\{foo2\}{[foo]}\{bar2\}]
%       The theorem set {\envfont foo2} (printed name \texttt{bar2})
%       uses the same counter as the theorem set \texttt{foo}.
%    \item[\ttfamily \bslash newtheorem\{foo3\}\{bar3\}{[section]}]
%       The theorem set {\envfont foo3} (printed name \texttt{bar3}) is
%       enumerated within the counter \texttt{section}, i.e.\ with every
%       new |\section| the enumeration begins again with 1, and
%       the enumeration is composed from the section-number and the
%       theorem counter itself.
% \end{description}
%
% \DescribeMacro\theoremstyle
% Additionally, the command |\theoremstyle| can define the layout
% of various, or all, theorem sets.  It should be noted that any
% theorem set defined by |\newtheorem| is typeset in the
% |\theoremstyle| that is current at the time of the definition.
% Thus, the following
% \begin{verbatim}
%    \theoremstyle{break}        \newtheorem{Cor}{Corollary}
%    \theoremstyle{plain}        \newtheorem{Exa}{Example}[section]
%\end{verbatim}                             ^^A No blank at line start
% leads to the result that the set {\envfont Cor} is formatted in the
% style \texttt{break}, while the set {\envfont Exa} and all the
% following ones are formatted in the style \texttt{plain}, unless
% another |\theoremstyle| follows.  Since the definitions installed by
% |\newtheorem| are global, one also can limit |\theoremstyle| locally
% by grouping braces.
%
% \DescribeMacro\theorembodyfont
% The choice of the font for the theorem body is completely
% independent of the chosen |\theoremstyle|; this has proven to
% be very advantageous.  For example,
% \begin{verbatim}
%   {\theorembodyfont{\upshape}          \newtheorem{Rem}{Remark}}
%\end{verbatim}
% defines a theorem set {\envfont Rem}, which will be set in
% |\upshape| in the current layout (which in our example is
% \texttt{plain}).  As with |\theoremstyle|, the |\theorembodyfont|
% chosen is that current at the time of |\newtheorem|.  If
% |\theorembodyfont| is not specified or one defines
% |\theorembodyfont{}|, then the font used will be that defined by the
% |\theoremstyle|.
%
% \DescribeMacro\theoremheaderfont It is also possible to customize
% the font used for the theorem headers.  This is, however, a global
% declaration, and therefore there should be at most one
% |\theoremheaderfont| declaration in the preamble.\footnote{If it is
% actually necessary to have different header fonts, one has to define
% new theorem styles (substituting the desired font) or specify the
% information directly in the \texttt{\bslash newtheorem} declaration
% (the unclean variant).}
%
% \DescribeMacro\theorempreskipamount
% \DescribeMacro\theorempostskipamount Two additional parameters
% affect the vertical space around the theorem environments:
% |\theorempreskipamount| and |\theorempostskipamount| define,
% respectively, the spacing before and after such an environment.
% These parameters apply for all theorem sets and can be manipulated
% with the ordinary length macros.  They are rubber lengths
% (`\textsf{skips}'), and therefore can contain \texttt{plus} and
% \texttt{minus} parts.
%
% Since the definition of theorem sets should---most sensibly---be
% placed in the preamble, we only allow installation there.
% It is therefore possible to release the memory used here after
% |\begin{document}|, in order to make room for other
% applications.
%
%
% \subsection{Existing theorem styles}
%
% The following theorem styles exist to date:
% \begin{deflist}{marginbreak}
%    \item[plain]
%       This theorem style emulates the original \LaTeX{} definition,
%       except that additionally the parameters
%       |\theorem...skipamount| are used.
%    \item[break]
%       In this style, the theorem header is followed by a line
%       break.
%    \item[marginbreak]
%       The theorem number is set in the margin, and there is a line
%       break as in \texttt{break}.
%    \item[changebreak]
%       Like \texttt{break}, but with header number and text
%       interchanged.
%    \item[change]
%       Header number and text are interchanged, without a line break.
%    \item[margin]
%       The number is set in the left margin, without a line break.
% \end{deflist}
% All styles (except \texttt{plain}) select |\slshape| as the default
% |\theorembodyfont|.
%
%
% \subsection{Examples}
%
% Given the above theorem sets {\envfont Cor}, {\envfont Exa}
% and {\envfont Rem},
% suppose that the preamble also contains the declarations:
% \begin{verbatim}
%    \theoremstyle{marginbreak}   \newtheorem{Lem}[Cor]{Lemma}
%    \theoremstyle{change}
%    \theorembodyfont{\itshape}        \newtheorem{Def}[Cor]{Definition}
%
%    \theoremheaderfont{\scshape}
%\end{verbatim}
% Then the following are some typical examples of the typeset output
% resulting from their use.
%
% \begin{Cor}
%    This is a sentence typeset in the theorem environment {\envfont
%    Cor}.
% \end{Cor}
%
% \begin{Exa}
%    This is a sentence typeset in the theorem environment {\envfont
%    Exa}.
% \end{Exa}
%
% \begin{Rem}
%    This is a sentence typeset in the theorem environment {\envfont
%    Rem}.
% \end{Rem}
%
% \begin{Lem}[Ben User]
%    This is a sentence typeset in the theorem environment {\envfont
%    Lem}.
% \end{Lem}
%
% \begin{Def}[Very impressive Definition]
%    This is a sentence typeset in the theorem environment {\envfont
%    Def}.
% \end{Def}
%
% The last two examples show the effect of the optional argument to
% a theorem environment (it is the text typeset in parentheses).
%
%
% \section{Special Considerations}
% \label{sec:newfonts}
%
% Theoremheader and body are implemented as a unit. This means that
% the |\theoremheaderfont|  will inherit characteristics of the
% |\theorembodyfont| in \LaTeXe. Thus, if for example
% |\theorembodyfont| is |\itshape| and |\theoremheaderfont|
% is |\bfseries| the font selected for the header will have the
% characteristics `bold extended italic'. If this is not desired one
% should set the |\theoremheaderfont| to something like
% \begin{verbatim}
%      \theoremheaderfont{\normalfont\bfseries}
%\end{verbatim}
% i.e.\ supplying all necessary font information explicitly.
%
%
%
% \section{Acknowledgements}
%
% The publication of this set of macros was only possible with the
% help of Christina Busse (translating the manuscript into English),
% Joachim Pense (playing the r\^ole of typist), Chris Rowley
% (looking everything over) and many others providing useful
% suggestions.
%
% \MaybeStop{%
%    \begin{thebibliography}{9}
%       \bibitem{bk:GMS94} \textsc{M.~Goossens}, \textsc{F.~Mittelbach}
%         and \textsc{A.~Samarin}.
%         \newblock The \LaTeX{} Companion.
%         \newblock
%         Addison-Wesley, Reading, Massachusetts, 1994.
%       \bibitem{latex:source} \textsc{Lamport, Leslie.} \newblock
%                               \texttt{latex.tex}, version 2.09, date
%                               Feb. 1990.
%    \end{thebibliography}
%    \PrintIndex
%    \PrintChanges}
%
%
% \section{The documentation driver file}
%
% The next bit of code contains the documentation driver file for
% \TeX{}, i.e., the file that will produce the documentation you are
% currently reading. It will be extracted from this file by the
% \texttt{docstrip} program. Since it is the first code in the file
% one can alternatively process this file directly with \LaTeXe{} to
% obtain the documentation.
%
%    \begin{macrocode}
%<*driver>
\documentclass{ltxdoc}

\usepackage{theorem}

%<+driver>% The next few lines define theorem sets which are used
%<+driver>% in the example section of the documentation.

\theoremstyle{break}       \newtheorem{Cor}{Corollary}
\theoremstyle{plain}       \newtheorem{Exa}{Example}[section]
{\theorembodyfont{\upshape}\newtheorem{Rem}{Remark}}
\theoremstyle{marginbreak} \newtheorem{Lem}[Cor]{Lemma}
\theoremstyle{change}
\theorembodyfont{\itshape} \newtheorem{Def}[Cor]{Definition}

\theoremheaderfont{\scshape}

\RecordChanges

\begin{document}
   \DocInput{theorem.dtx}
\end{document}
%</driver>
%    \end{macrocode}
%
% \section{Definition of the Macros}
%
% If the file has been loaded before, we abort immediately. If not the
% package announces itself (this is actually done at the very top of
% the file---the way it is done isn't good style so don't copy it).
% \changes{v2.0e}{89/07/19}{Spaces removed from `typeout'.}
% \changes{v2.2b}{95/11/19}{Announce moved to top of file}
%    \begin{macrocode}
%<*package>
%\@ifundefined{theorem@style}{}{\endinput}
%\def\FMithmInfo{1995/11/19 v2.2b Theorem extension package (FMi)}
%\ProvidesPackage{theorem}[\FMithmInfo]
%    \end{macrocode}
%
%
% \subsection{Definition of theorem styles and fonts}
%
% All the definitions in this file are done globally to allow
% inputting this file inside a group.
%
% \begin{macro}{\theoremstyle}
%    Before a theorem style can be installed, the chosen style must be
%    known. For that reason, we must test to see that
%    |\th@|\meta{style} is known or, more precisely, that it is
%    different from |\relax|. If the style is not known then
%    |\th@plain| is used.
%    \begin{macrocode}
\gdef\theoremstyle#1{%
   \@ifundefined{th@#1}{\@warning
          {Unknown theoremstyle `#1'. Using `plain'}%
          \theorem@style{plain}}%
%    \end{macrocode}
%    We save the theorem style to be used in the token register
%    |\theorem@style|.
%    \begin{macrocode}
      {\theorem@style{#1}}%
%    \end{macrocode}
%    Now we ``evaluate'' the theorem style: this means, we call the
%    macro |\th@|\meta{style} which will activate the relevant
%    definitions which are contained in a separate file.  This is done
%    in a group to suppress changes to the current font.  This could
%    otherwise pose problems together with the new font selection
%    scheme\footnote{When I printed the original article using the new
%    font selection scheme I ended with a document with slanted
%    typefaces (text headings and all) simply because one of the
%    theorem styles used \texttt{\bslash sl} at toplevel.} if the
%    |\th@|\meta{style} is evaluated a second time.
% \changes{v2.1a}{90/02/15}{Group added around `th@..}
% \changes{v2.1a}{90/02/15}{Reseting of `..@begintheorem removed.}
%    \begin{macrocode}
      \begingroup
        \csname th@\the\theorem@style \endcsname
      \endgroup}
%    \end{macrocode}
% \end{macro}
%
%
% \begin{macro}{\@begintheorem}
% \changes{v2.1a}{90/02/15}{Relaxed at toplevel.}
% \begin{macro}{\@opargbegintheorem}
% \changes{v2.1a}{90/02/15}{Relaxed at toplevel.}
%    We reset |\@begintheorem| and |\@opargbegintheorem| to
%    |\relax| since these commands are no longer necessary at
%    toplevel. This will save a few tokens.
%    \begin{macrocode}
\global\let\@begintheorem\relax
\global\let\@opargbegintheorem\relax
%    \end{macrocode}
% \end{macro}
% \end{macro}
%
% \begin{macro}{\theorem@style}
%    Obviously the \textsf{token} register used above has to be
%    allocated. To assure the utmost compatibility with the
%    original \LaTeX{} definition, we set the default theorem style to
%    \texttt{plain}, which implements the usual \LaTeX{} convention.
% \changes{v2.0f}{89/07/20}{Unnecessary `global' removed.}
%    \begin{macrocode}
\newtoks\theorem@style
\global\theorem@style{plain}
%    \end{macrocode}
% \end{macro}
%
% \begin{macro}{\theorembodyfont}
% \begin{macro}{\theorem@bodyfont}
%    If the |\theorembodyfont| is set by the user then it should not
%    interact with the default font set in the theorem style. When the
%    new font selection is in force this may happen if, for example,
%    the default is |\itshape| and the new |\theorembodyfont| is
%    |\sffamily|. So we add a |\reset@font| command in front of the
%    user definition.
% \changes{v2.0f}{89/07/20}{Unnecessary `global' removed.}
% \changes{v2.0g}{89/09/19}{`theoremfont to `theorembodyfont.}
% \changes{v2.1c}{92/06/22}{`reset@font added and logic changed.}
% \changes{v2.1d}{92/10/28}{Reset to empty if arg is empty}
%    \begin{macrocode}
\gdef\theorembodyfont#1{%
%    \end{macrocode}
%    We check if the argument supplied is empty and if so put nothing
%    into the |\theorem@bodyfont| token register to allow for
%    |\theorembodyfont{}| as a mean of using the default of the
%    current |\theoremstyle|.
%    \begin{macrocode}
   \def\@tempa{#1}%
   \ifx\@tempa\@empty
    \theorem@bodyfont{}%
   \else
    \theorem@bodyfont{\reset@font#1}%
   \fi
   }
\newtoks\theorem@bodyfont
\global\theorem@bodyfont{}
%    \end{macrocode}
% \end{macro}
% \end{macro}
%
%
% \begin{macro}{\theoremheaderfont}
% \changes{v2.0g}{89/09/19}{Macro added.}
%    The font for the theorem headers is handled differently because
%    this definition applies to all theorem styles.
%    \begin{macrocode}
\gdef\theoremheaderfont#1{\gdef\theorem@headerfont{#1}%
%    \end{macrocode}
%    After using the macro once it is redefined to produce an error
%    message.
%    \begin{macrocode}
       \gdef\theoremheaderfont##1{%
        \typeout{\string\theoremheaderfont\space should be used
                 only once.}}}
%    \end{macrocode}
% \end{macro}
%
% \begin{macro}{\theorem@headerfont}
% \changes{v2.0g}{89/09/19}{Macro added.}
%    To set the |\theorem@headerfont| default we first test if
%    the new fontselection scheme is in force.
% \changes{v2.1a}{90/02/15}{Support for new font selection.}
%    \begin{macrocode}
\ifx\upshape\undefined
%    \end{macrocode}
%    If not we define it to expand into |\bfseries|. We don't use
%    |\let| just in case a following style option redefines this
%    macro.
%    \begin{macrocode}
\gdef\theorem@headerfont{\bfseries}
%    \end{macrocode}
%    Otherwise we reset the current shape before calling |\bfseries|.
%    \begin{macrocode}
\else \gdef\theorem@headerfont{\normalfont\bfseries}\fi
%    \end{macrocode}
% \end{macro}
%
% \begin{macro}{\th@plain}
% \begin{macro}{\th@break}
% \begin{macro}{\th@marginbreak}
% \begin{macro}{\th@changebreak}
% \begin{macro}{\th@change}
% \begin{macro}{\th@margin}
%    The different styles are defined in macros such as
%    |\th@plain|.  Since memory space is precious in
%    ``non-Big-versions'', we have to avoid offering too many unused
%    definitions.  Therefore we define these styles in separate files
%    that can be loaded on demand.  Thus the commands themselves only
%    load these files. We use |\@input@|, a \LaTeXe{} internal command
%    that ensures that the file will be listed with |\listfiles|.
% \changes{v2.2a}{94/02/02}{Use `@input@ to load theorem layout files}
%    \begin{macrocode}
\gdef\th@plain{\@input@{thp.sty}}
\gdef\th@break{\@input@{thb.sty}}
\gdef\th@marginbreak{\@input@{thmb.sty}}
\gdef\th@changebreak{\@input@{thcb.sty}}
\gdef\th@change{\@input@{thc.sty}}
\gdef\th@margin{\@input@{thm.sty}}
%    \end{macrocode}
%    This list will be expanded when new styles become
%    available.  For testing,  just append new theorem substyles
%    as document options.
% \end{macro}
% \end{macro}
% \end{macro}
% \end{macro}
% \end{macro}
% \end{macro}
%
%
%
% \subsection{Definition of a new theorem set}
%
% As already pointed out, a new theorem environment can be defined in
% three different ways:
% \begin{verbatim}
%    \newtheorem{Lem}{Lemma}
%    \newtheorem{Lem}{Lemma}[section]
%    \newtheorem{Lem}[Theorem]{Lemma}
%\end{verbatim}
% The function of the macro |\newtheorem| is to recognize these cases
% and then to branch into one of the three macros |\@ynthm|, |\@xnthm|
% or |\@othm|.  This mechanism is adopted unchanged from
% \cite{latex:source}; the essential point here is that, for example,
% in the second case, the arguments \texttt{Lem}, \texttt{Lemma} and
% \texttt{section} are passed over to the macro |\@xnthm|.
%
% We inspect this case first because the others present fewer problems,
% and thus are easily derived from this one.
%
% \begin{macro}{\@xnthm}
%    For our example arguments, the macro |\@xnthm| must fulfill
%    the following:
%    \begin{itemize}
%       \item
%          Define a new \LaTeX-counter `Lem'
%       \item
%          reset this counter within a |\section|
%       \item
%          define the macro |\theLem|
%       \item
%          define the environment macros |\Lem| and
%          |\endLem| using the current
%          |\theoremstyle| and |\theorem@bodyfont|.
%    \end{itemize}
%    Obviously, all this should happen only if the first argument
%    of |\@xnthm| (i.e.\ \texttt{Lem} in our example) is chosen so
%    as not to conflict with any previously defined commands or
%    environments.  This test is performed by the \LaTeX{} macro
%    |\@ifdefinable|.
%    \begin{macrocode}
\gdef\@xnthm#1#2[#3]{\expandafter\@ifdefinable\csname #1\endcsname
%    \end{macrocode}
%    Therefore, the first argument of |\@ifdefinable| is the
%    expansion (in the example, |\Lem|) of
%    |\csname||#1\endcsname|.  The second argument is
%    executed only if the test has been completed successfully.
%    \begin{macrocode}
   {%
%    \end{macrocode}
%    Now we define the new counter. The names of the \LaTeX{}
%    macros employed should speak for themselves:
% \changes{v2.2b}{1995/10/07}{Use \cs{@newctr}}
% \changes{v2.2c}{1995/11/23}{Use \cs{@newctr} properly.}
%    \begin{macrocode}
    \@definecounter{#1}\@newctr{#1}[#3]%
%    \end{macrocode}
%    Using |\@newctr| will give a proper error message if the counter
%    in |#3| is not defined.
%    In defining `|\theLem|' we must generate the desired macro
%    name by use of |\expandafter| and |\csname|.
%    \begin{macrocode}
    \expandafter\xdef\csname the#1\endcsname
%    \end{macrocode}
%    An |\xdef| is used in order to make the definition global,
%    and to ensure that it contains the replacement texts of
%    |\@thmcountersep| and |\@thmcounter|.\footnote{These
%    two macros can be defined by the document style.  Their default
%    values produce a `\texttt{.}' as separation and an arabic
%    representation of the number.} However, not everything should be
%    expanded. For example, it saves space to use |\thesection|
%    instead of its---at times---lengthy expansion.
%    \begin{macrocode}
      {\expandafter \noexpand \csname the#3\endcsname
       \@thmcountersep \@thmcounter{#1}}%
%    \end{macrocode}
%    Thus with the defaults of \LaTeX{}, |\theLem| would be
%    replaced by the command sequence |\thesection.\arabic{Lem}|.
%
%    We will now look at the definition of the macro which is executed
%    at the beginning of the actual environment (in our example this
%    macro is |\Lem|). It should be noted that we use an
%    ``|\expandafter| trick'' to expand only certain parts of the
%    replacement text at the time of the definition.
%    \begin{macrocode}
    \def\@tempa{\global\@namedef{#1}}%
    \expandafter \@tempa \expandafter{%
%    \end{macrocode}
%    First, the macro that contains the current definitions of
%    |\@begintheorem| and |\@opargtheorem| should be called
%    up.  The name of this macro---as is already known---has the form
%    |\th@|\meta{theorem\ style}; therefore, it must be called by
%    \begin{macrocode}
      \csname th@\the \theorem@style
%    \end{macrocode}
%    In addition the default theorem font should be changeable, i.e.\
%    we have to insert the contents of |\theorem@bodyfont|. For
%    that reason, we expand even further, beyond |\endcsname|,
%    and thus insert the contents of the \textsf{token} register
%    |\theorem@bodyfont| in the replacement text.
%    \begin{macrocode}
            \expandafter \endcsname \the \theorem@bodyfont
%    \end{macrocode}
%    Now it is time to call the macro |\@thm| which takes over
%    the further processing.  It has two arguments: the current
%    counter name (in our example, \texttt{Lem}), and the text of the
%    label (in our example, \texttt{Lemma}).
%    \begin{macrocode}
     \@thm{#1}{#2}}%
%    \end{macrocode}
%    With this, the `sub-definition' is complete.  The macro
%    |\@endtheorem| ends a theorem environment and is, so far, nothing
%    but an |\endtrivlist|.  (Hence it is defined globally, and not
%    within the theorem styles.\footnote{This has to be changed as
%    soon as theorem styles that change \texttt{\bslash @endtheorem}
%    exist.  In such a case, all existing styles must be changed as
%    well since they will have to reset the macro.}) Therefore, we can
%    set it equivalent to the macro that ends the theorem set (in our
%    example, |\endLem|).  However, if some day theorem styles exist
%    that do change |\@endtheorem|, we would have to use the
%    commented-out line instead.
%    \begin{macrocode}
    \global \expandafter \let \csname end#1\endcsname \@endtheorem
%   \global\@namedef{end#1}{\@endtheorem}%
%    \end{macrocode}
%    With these commands all the required definitions are employed,
%    unless the test |\@ifdefinable| has failed.  Therefore, we
%    end the second argument of this macro and with it the definition
%    of |\@xnthm|.
%    \begin{macrocode}
   }}
%    \end{macrocode}
% \end{macro}
%
% \begin{macro}{\@ynthm}
%    The definition of |\@ynthm| is completely analogous. In
%    this case the new counter that is defined is not reset within
%    another counter; thus the definition of |\the...|
%    is simplified:
%    \begin{macrocode}
\gdef\@ynthm#1#2{\expandafter\@ifdefinable\csname #1\endcsname
   {\@definecounter{#1}%
    \expandafter\xdef\csname the#1\endcsname{\@thmcounter{#1}}%
%    \end{macrocode}
%    The rest of the definition corresponds literally to that of
%    |\@xnthm|:
%    \begin{macrocode}
    \def\@tempa{\global\@namedef{#1}}\expandafter \@tempa
     \expandafter{\csname th@\the \theorem@style \expandafter
     \endcsname \the\theorem@bodyfont \@thm{#1}{#2}}%
    \global \expandafter \let \csname end#1\endcsname \@endtheorem}}
%    \end{macrocode}
% \end{macro}
%
% \begin{macro}{\@othm}
%    The definition of |\@othm| does not contain anything new.
%    \begin{macrocode}
\gdef\@othm#1[#2]#3{%
%    \end{macrocode}
% \changes{v2.2b}{95/10/07}{Build in check for counter defined}
%    We do not define a new counter but instead use one that has
%    already been defined.  Thus the only definition we need is that
%    of this pseudo-counter (i.e.\ |\the|\meta{env.\ name}).
%    First we check if |#2| corresponds to a known counter name.
%    \begin{macrocode}
  \expandafter\ifx\csname c@#2\endcsname\relax
   \@nocounterr{#2}%
  \else
   \expandafter\@ifdefinable\csname #1\endcsname
   {\expandafter \xdef \csname the#1\endcsname
     {\expandafter \noexpand \csname the#2\endcsname}%
%    \end{macrocode}
%    All other parts of the definition can be adopted from
%    |\@xnthm|. We have to remember, though, that in this case
%    the name of the current counter and the theorem label have moved
%    to the second and third arguments.
%    \begin{macrocode}
    \def\@tempa{\global\@namedef{#1}}\expandafter \@tempa
     \expandafter{\csname th@\the \theorem@style \expandafter
     \endcsname \the\theorem@bodyfont \@thm{#2}{#3}}%
    \global \expandafter \let \csname end#1\endcsname \@endtheorem}%
  \fi}
%    \end{macrocode}
% \end{macro}
%
% \subsection{Macros that are employed in a {\envfont theorem}
%    environment}
%
% \begin{macro}{\@thm}
%    The macro |\@thm| has to increase the current counter.  Then,
%    depending on whether the environment has (or does not have) an
%    optional argument, it has to branch into either |\@begintheorem|
%    or |\@opargtheorem|.
%    \begin{macrocode}
\gdef\@thm#1#2{\refstepcounter{#1}%
%    \end{macrocode}
%    Now we start a {\envfont trivlist} environment, and give
%    |\@topsep| and |\@topsepadd| the values of the \textsf{skip}
%    registers |\theorempreskipamount| and |\theorempostskipamount|.
%    The value in |\@topsep| is the vertical space that is inserted by
%    the first (and only) |\item| in our |\trivlist| whilst
%    |\@topsepadd| is inserted by |\@endparenv| at the end of that
%    {\envfont trivlist} environment.  By using these registers, we
%    obtain the desired space around a {\envfont theorem} environment.
%    \begin{macrocode}
   \trivlist
   \@topsep \theorempreskipamount               % used by first \item
   \@topsepadd \theorempostskipamount           % used by \@endparenv
%    \end{macrocode}
%    Now we have to test whether an optional argument has been given.
% \changes{v2.0d}{89/07/18}{Forgotten percent added.}
%    \begin{macrocode}
   \@ifnextchar [%
%    \end{macrocode}
%    If there is an optional argument, we will call |\@ythm|, and
%    move the arguments read back into the input stream.
%    \begin{macrocode}
   {\@ythm{#1}{#2}}%
%    \end{macrocode}
%    If not, we call |\@begintheorem|. Its first argument is the
%    name of the theorem set (hence the second argument of
%    |\@thm|). Its second argument is the macro that produces the
%    current number.
%    \begin{macrocode}
   {\@begintheorem{#2}{\csname the#1\endcsname}\ignorespaces}}
%    \end{macrocode}
% \end{macro}
%
% \begin{macro}{\@xthm}
% \begin{macro}{\@ythm}
%    Both these macros were originally called by |\@thm|.  We do
%    not need |\@xthm| anymore, hence we reset it to
%    |\relax|. The definition of |\@ythm| has not changed at
%    all from its definition in \LaTeX. In order to make the macros
%    easier to understand, we will
%    nevertheless present it (commented out).
%    \begin{macrocode}
\global\let\@xthm\relax
% \def\@ythm#1#2[#3]{\@opargbegintheorem{#2}{\csname
%           the#1\endcsname}{#3}\ignorespaces}
%</package>
%    \end{macrocode}
%    The primitive |\ignorespaces| in |\@ythm| and
%    |\@thm| is needed to remove the spaces between the
%    |\begin{...}| and the actual text.
% \end{macro}
% \end{macro}
%
%
% \subsection{Definition of the theorem substyles}
%
% As already pointed out, the theorem substyles, defined below, are
% only loaded when necessary.  Note that all these substyles, except
% \texttt{plain}, have |\slshape| as the default body font.
%
% \subsubsection{The \texttt{plain} style}
%
% \label{dok:th@plain} As the following macros use |@|, we have
% to locally set the |\catcode| of this symbol to ``letter''.
% This happens within a group, so that we do not have to worry about
% which |\catcode| that symbol had before.
%    \begin{macrocode}
%<*thp>
\begingroup \makeatletter
%    \end{macrocode}
% Since we are now within a group, we must make all definitions
% globally.  First we make sure that \texttt{theorem.sty} is loaded.
% This will allow us to use this file as a document style option
% without having to call \texttt{theorem} itself as an option.  At the
% same time, we assure that at least version 2 is loaded, since
% |\theorem@style| was not defined in earlier versions.
%    \begin{macrocode}
\@ifundefined{theorem@style}{\input{theorem.sty}}{}
\ProvidesFile{thp.sty}
             [\FMithmInfo]
%    \end{macrocode}
%
%
% \begin{macro}{\th@plain}
% \begin{macro}{\@begintheorem}
% \begin{macro}{\@opargbegintheorem}
%    |\theoremstyle{plain}| corresponds to the original
%    definition, except that the distances to the surrounding text are
%    determined by the parameters |\theorempreskipamount| and
%    |\theorempostskipamount|.
%    First we set the default body font.
%    \begin{macrocode}
\gdef\th@plain{\normalfont\itshape
%    \end{macrocode}
%    Then we define |\@begintheorem| and
%    |\@opargbegintheorem|. These two macros define how the
%    header of a theorem is typeset.  |\@opargbegintheorem| will
%    be called if a \textsf{theorem} environment with an optional
%    argument is encountered; otherwise, the header is constructed by
%    calling |\@begintheorem|.  If one of these macros is
%    executed, we are within a \textsf{trivlist} environment started by
%    |\@thm|.  So the theorem header is produced with an
%    |\item| command.
%
%    Instead of specifying the header font directly, all standard
%    theorem styles use the |\theorem@headerfont| macro to allow
%    customization.  The extra space (|\labelsep|) is necessary
%    because of problems in the \textsf{trivlist} environment.
%    \begin{macrocode}
  \def\@begintheorem##1##2{%
        \item[\hskip\labelsep \theorem@headerfont ##1\ ##2]}%
%    \end{macrocode}
%    The definition of |\@opargbegintheorem| is completely
%    analogous. The only difference is the fact that there exists a
%    third argument (which is the optional parameter of the
%    environment and contains additional information about the
%    theorem).  Customarily we enclose it in parentheses.
%    \begin{macrocode}
\def\@opargbegintheorem##1##2##3{%
   \item[\hskip\labelsep \theorem@headerfont ##1\ ##2\ (##3)]}}
%    \end{macrocode}
% \end{macro}
% \end{macro}
% \end{macro}
%
% \noindent
% We conclude with an |\endgroup| to restore the
% |\catcode| of |@|.
%    \begin{macrocode}
\endgroup
%</thp>
%    \end{macrocode}
%
%
% \subsubsection{The \texttt{break} style}
%
% This style option is stored in the file |thb.sty|.
% For the next two lines see the documentation for |\th@plain| on
% page~\pageref{dok:th@plain}.
%    \begin{macrocode}
%<*thb>
\begingroup \makeatletter
\@ifundefined{theorem@style}{\input{theorem.sty}}{}
\ProvidesFile{thb.sty}
             [\FMithmInfo]
%    \end{macrocode}
%
% \begin{macro}{\th@break}
%    |\theoremstyle{break}| produces a line break after the name
%    of the theorem.  The font is |\slshape|.  Hence, we define
%    |\th@break| as follows:
%    \begin{macrocode}
\gdef\th@break{\normalfont\slshape
  \def\@begintheorem##1##2{\item[%
%    \end{macrocode}
%    We run into the following problem: it is not possible to create
%    the header with |\item[|\meta{title}|]| and then start
%    a new line by, for example, |\mbox{}\\|. Such a definition
%    will fail whenever a list environment follows immediately.  With
%    the above construction, the |\mbox{}| causes the switch
%    |@inlabel| (cf.\ definition of |\list| and
%    |\trivlist| in \cite{latex:source}) to be set to \texttt{false}
%    and so the following list will insert additional vertical
%    space (|\topskip|). This is quite annoying. Therefore, we
%    create the line break within the |\item|. In order to ensure
%    that the text will begin at the proper position in the following
%    line, we simply pretend that the label does not take any
%    room.\footnote{This will lead to problems whenever very high
%    symbols occurring in the line tower into the
%    heading. So, something else has to be done here sometime.}
%    \begin{macrocode}
       \rlap{\vbox{\hbox{\hskip \labelsep\theorem@headerfont ##1\ ##2}%
                   \hbox{\strut}}}]}%
%    \end{macrocode}
%    Again, the definition of |\@opargbegintheorem| is
%    completely analogous.
%    \begin{macrocode}
\def\@opargbegintheorem##1##2##3{%
  \item[\rlap{\vbox{\hbox{\hskip \labelsep \theorem@headerfont
                     ##1\ ##2\ (##3)}%
                    \hbox{\strut}}}]}}
\endgroup
%</thb>
%    \end{macrocode}
% \end{macro}
%
%
% \subsubsection{The \texttt{changebreak} style}
%
%    \begin{macrocode}
% This style option is stored in the file |thcb.sty|.
%    \begin{macrocode}
%<*thcb>
\begingroup \makeatletter
\@ifundefined{theorem@style}{\input{theorem.sty}}{}
\ProvidesFile{thcb.sty}
             [\FMithmInfo]
%    \end{macrocode}
%
% \begin{macro}{\th@changebreak}
%    The \texttt{change-break} theorem style is like \texttt{break}
%    but with interchange of theorem name and theorem number.  Thus we
%    define |\th@changebreak| as follows:
%    \begin{macrocode}
\gdef\th@changebreak{\normalfont\slshape
  \def\@begintheorem##1##2{\item
       [\rlap{\vbox{\hbox{\hskip\labelsep \theorem@headerfont ##2\ ##1}%
                    \hbox{\strut}}}]}%
\def\@opargbegintheorem##1##2##3{%
  \item[\rlap{\vbox{\hbox{\hskip\labelsep \theorem@headerfont
                     ##2\ ##1\ (##3)}%
                    \hbox{\strut}}}]}}
\endgroup
%</thcb>
%    \end{macrocode}
% \end{macro}
%
%
% \subsubsection{The \texttt{change} style}
%
% This style option is stored in the file |thc.sty|.
%    \begin{macrocode}
%<*thc>
\begingroup \makeatletter
\@ifundefined{theorem@style}{\input{theorem.sty}}{}
\ProvidesFile{thc.sty}
             [\FMithmInfo]
%    \end{macrocode}
%
% \begin{macro}{\th@change}
%    The \texttt{change} theorem style corresponds to the
%    \texttt{change break} style without a linebreak after the header.
%    To say it in another way, it's the same as the \texttt{plain}
%    style but with number and name interchanged and |\slshape| as the
%    default font.
%    \begin{macrocode}
\gdef\th@change{\normalfont\slshape
  \def\@begintheorem##1##2{\item
       [\hskip\labelsep \theorem@headerfont ##2\ ##1]}%
\def\@opargbegintheorem##1##2##3{%
  \item[\hskip\labelsep \theorem@headerfont ##2\ ##1\ (##3)]}}
\endgroup
%</thc>
%    \end{macrocode}
% \end{macro}
%
%
% \subsubsection{The \texttt{marginbreak} style}
%
% This style option is the one used most often at Mainz.  It is saved
% in the file |thmb.sty|.
%    \begin{macrocode}
%<*thmb>
\begingroup \makeatletter
\@ifundefined{theorem@style}{\input{theorem.sty}}{}
\ProvidesFile{thmb.sty}
             [\FMithmInfo]
%    \end{macrocode}
%
% \begin{macro}{\th@marginbreak}
%    The \texttt{margin break} style is nearly the same as the
%    \texttt{change break} style. The only difference is the placement
%    of the theorem number.  We use |\llap| to place it in the left
%    margin.
%
%    In this style |\labelsep| denotes the separation between the
%    number and the text.
%    \begin{macrocode}
\gdef\th@marginbreak{\normalfont\slshape
  \def\@begintheorem##1##2{\item
       [\rlap{\vbox{\theorem@headerfont
                    \hbox{\llap{##2}\hskip\labelsep ##1}%
                    \hbox{\strut}}}]}%
\def\@opargbegintheorem##1##2##3{%
  \item[\rlap{\vbox{\theorem@headerfont
                    \hbox{\llap{##2}\hskip\labelsep ##1\ (##3)}%
                    \hbox{\strut}}}]}}
\endgroup
%</thmb>
%    \end{macrocode}
% \changes{v2.1c}{90/02/17}{`endgroup outside macrocode env.}
% \end{macro}
%
%
%
% \subsubsection{The \texttt{margin} style}
%
% This style option is stored in the file |thm.sty|.
%    \begin{macrocode}
%<*thm>
\begingroup \makeatletter
\@ifundefined{theorem@style}{\input{theorem.sty}}{}
\ProvidesFile{thm.sty}
             [\FMithmInfo]
%    \end{macrocode}
%
% \begin{macro}{\th@margin}
%    Again this is only a variant of the theorem styles described
%    above without any new ideas.
%    \begin{macrocode}
\gdef\th@margin{\normalfont\slshape
  \def\@begintheorem##1##2{\item
       [\theorem@headerfont \llap{##2}\hskip\labelsep ##1]}%
\def\@opargbegintheorem##1##2##3{%
  \item[\theorem@headerfont \llap{##2}\hskip\labelsep ##1\ (##3)]}}
\endgroup
%</thm>
%    \end{macrocode}
% \end{macro}
%
%
%
%
% \subsection{Final Definitions}
%
% \begin{macro}{\theorempreskipamount}
% \begin{macro}{\theorempostskipamount}
%    The \textsf{skip} parameters that regulate the vertical empty space
%    before and after the {\envfont theorem} environment have
%    to be allocated as well.
% \changes{v2.0f}{89/07/20}{Unnecessary `global' removed.}
%    \begin{macrocode}
%<*package>
\newskip\theorempreskipamount
\newskip\theorempostskipamount
%    \end{macrocode}
%    Since we have used the same values for all theorem sets, we now can
%    assign them.
%    \begin{macrocode}
\global\setlength\theorempreskipamount{12pt plus 5pt minus 3pt}
\global\setlength\theorempostskipamount{8pt plus 3pt minus 1.5pt}
%    \end{macrocode}
% \end{macro}
% \end{macro}
%
%
% \begin{macro}{\@endtheorem}
%    The same holds for the macro |\@endtheorem|, which ends a
%    {\envfont theorem} environment. Since it is the same for all
%    theorem sets, it is removed from the macros |\th@|\meta{style}.
%    It simply ends the {\envfont trivlist} environment, which was
%    begun in |\@thm|.
%    \begin{macrocode}
\global\let\@endtheorem=\endtrivlist
%    \end{macrocode}
% \end{macro}
%
%
%
% \begin{macro}{\@preamblecmds}
%    All macros defined above are to be used only in the preamble.
%    Therefore, we insert them in |\@preamblecmds| which will disable
%    them at begin document. This is done by the internal \LaTeXe{}
%    command |\@onlypreamble|.
% \changes{v2.0g}{89/09/19}{`theoremheaderfont and `theorembodyfont
%    added.}
% \changes{v2.2a}{94/02/03}{Use \LaTeXe{} interface}
%    \begin{macrocode}
\@onlypreamble\@xnthm
\@onlypreamble\@ynthm
\@onlypreamble\@othm
\@onlypreamble\newtheorem
\@onlypreamble\theoremstyle
\@onlypreamble\theorembodyfont
\@onlypreamble\theoremheaderfont
%    \end{macrocode}
% \end{macro}
%
% Finally we declare the \texttt{plain} theorem style to be the default.
% \changes{v2.1b}{90/06/04}{Fixed missing `item by declaring default
%    style.}
%    \begin{macrocode}
\theoremstyle{plain}
%</package>
%    \end{macrocode}
%
% \Finale

\endinput
}{}
\ProvidesFile{thb.sty}
             [\FMithmInfo]
%    \end{macrocode}
%
% \begin{macro}{\th@break}
%    |\theoremstyle{break}| produces a line break after the name
%    of the theorem.  The font is |\slshape|.  Hence, we define
%    |\th@break| as follows:
%    \begin{macrocode}
\gdef\th@break{\normalfont\slshape
  \def\@begintheorem##1##2{\item[%
%    \end{macrocode}
%    We run into the following problem: it is not possible to create
%    the header with |\item[|\meta{title}|]| and then start
%    a new line by, for example, |\mbox{}\\|. Such a definition
%    will fail whenever a list environment follows immediately.  With
%    the above construction, the |\mbox{}| causes the switch
%    |@inlabel| (cf.\ definition of |\list| and
%    |\trivlist| in \cite{latex:source}) to be set to \texttt{false}
%    and so the following list will insert additional vertical
%    space (|\topskip|). This is quite annoying. Therefore, we
%    create the line break within the |\item|. In order to ensure
%    that the text will begin at the proper position in the following
%    line, we simply pretend that the label does not take any
%    room.\footnote{This will lead to problems whenever very high
%    symbols occurring in the line tower into the
%    heading. So, something else has to be done here sometime.}
%    \begin{macrocode}
       \rlap{\vbox{\hbox{\hskip \labelsep\theorem@headerfont ##1\ ##2}%
                   \hbox{\strut}}}]}%
%    \end{macrocode}
%    Again, the definition of |\@opargbegintheorem| is
%    completely analogous.
%    \begin{macrocode}
\def\@opargbegintheorem##1##2##3{%
  \item[\rlap{\vbox{\hbox{\hskip \labelsep \theorem@headerfont
                     ##1\ ##2\ (##3)}%
                    \hbox{\strut}}}]}}
\endgroup
%</thb>
%    \end{macrocode}
% \end{macro}
%
%
% \subsubsection{The \texttt{changebreak} style}
%
%    \begin{macrocode}
% This style option is stored in the file |thcb.sty|.
%    \begin{macrocode}
%<*thcb>
\begingroup \makeatletter
\@ifundefined{theorem@style}{% \iffalse meta-comment
%
% Copyright (C) 1993-2024
%
% The LaTeX Project and any individual authors listed elsewhere
% in this file.
%
% This file is part of the Standard LaTeX `Tools Bundle'.
% -------------------------------------------------------
%
% It may be distributed and/or modified under the
% conditions of the LaTeX Project Public License, either version 1.3c
% of this license or (at your option) any later version.
% The latest version of this license is in
%    https://www.latex-project.org/lppl.txt
% and version 1.3c or later is part of all distributions of LaTeX
% version 2005/12/01 or later.
%
% The list of all files belonging to the LaTeX `Tools Bundle' is
% given in the file `manifest.txt'.
%
% \fi
%
%%
%
%
\def\FMithmInfo{2023/07/05 v2.2c Theorem extension package (FMi)}
%
% \ProvidesFile{theorem.dtx}[\FMithmInfo]
%\iffalse   % this is a METACOMMENT !
%%
%% Package `theorem' to use with LaTeX2e
%% Copyright (C) 1989-1995 Frank Mittelbach, all rights reserved.
%<+package>\@ifundefined{theorem@style}{}{\endinput}
%<+package>\NeedsTeXFormat{LaTeX2e}[1995/06/01]
%<+package>\ProvidesPackage{theorem}
%<+package>         [\FMithmInfo]
%
% \fi
%
% \changes{v1.0}{87/05/07}{First draft.}
% \changes{v1.0a}{87/05/07}{`thistheoremstyle' added.}
% \changes{v1.1a}{87/07/29}{New concept: redefine the internal
%                          \protect\LaTeX{} commands `@xnthm' etc.}
% \changes{v1.1b}{87/07/31}{`theoremfont' used to specify font.}
% \changes{v1.2a}{88/01/08}{`@endtheorem' now globally defined.}
% \changes{v1.2b}{88/01/11}{`newtheorem' only allowed in the preamble.}
% \changes{v1.3a}{88/02/14}{`theoremfont' now a toks register.}
%
% \changes{v2.0a}{88/11/16}{New implementation and new documentation.}
% \changes{v2.0a}{88/11/16}{No layout changes but internal modifications
%                           old substyles won't work.}
% \changes{v2.0b}{89/01/30}{`global' used for all definitions.}
% \changes{v2.0c}{89/06/15}{Documentation translated to English.
%                            Thanks to LLL and Nina}
% \changes{v2.1a}{90/02/15}{Support for new font selection.}
% \changes{v2.2a}{94/02/03}{Cleanup for LaTeX2e}
%
% \DoNotIndex{\@endparenv,\@ifdefinable,\@ifnextchar,\@ifundefined}
% \DoNotIndex{\@namedef,\@nameuse,\@spaces,\@tempa,\@warning}
% \DoNotIndex{\ }
% \DoNotIndex{\begingroup,\bfseries}
% \DoNotIndex{\csname}
% \DoNotIndex{\def,\do,\docdate}
% \DoNotIndex{\endcsname,\endgroup,\endinput,\endtrivlist,\expandafter}
% \DoNotIndex{\filedate,\fileversion}
% \DoNotIndex{\gdef,\global}
% \DoNotIndex{\hbox,\hskip}
% \DoNotIndex{\ignorespaces,\input,\itshape,\item}
% \DoNotIndex{\labelsep,\let,\llap}
% \DoNotIndex{\makeatletter}
% \DoNotIndex{\noexpand}
% \DoNotIndex{\refstepcounter,\relax,\rlap}
% \DoNotIndex{\setlength,\slshape,\space,\string,\strut}
% \DoNotIndex{\the,\trivlist,\typeout}
% \DoNotIndex{\vbox}
% \DoNotIndex{\xdef}
%
% \def\deflabel#1{\ttfamily #1\hfill}
% \def\deflist#1{\begin{list}{}{\settowidth\labelwidth{\ttfamily #1}%
%                               \setlength\leftmargin\labelwidth
%                               \addtolength\leftmargin\labelsep
%                               \let\makelabel\deflabel}}
% \def\enddeflist{\end{list}}
% \def\envfont{\normalfont\ttfamily}
%
% \MakeShortVerb{\|}
%
% \newcommand{\TUB}{{\sl TUGboat\/}}
%
% \GetFileInfo{theorem.dtx}
% \title{An Extension of the
%        \LaTeX{} theorem environment\thanks{This file
%        has version number \fileversion, last
%        revised \filedate.}}
% \author{Frank Mittelbach}
%
% \MaintainedByLaTeXTeam{tools}
% \maketitle
%
% \begin{abstract}
%    The macros described in this paper yield an extension of the
%    \LaTeX{} theorem mechanism. It is designed to satisfy the
%    different requirements of various journals. Thus, the
%    layout of the ``theorems'' can be manipulated by determining a
%    ``style''. This article describes not only the use, but
%    also the definition, of the necessary macros.
% \end{abstract}
%
% \section*{Preface to version 2.2}
%
% For \LaTeXe{} this package did not need any fundamental changes. I
% only modified the messages generated so that theorem layout styles
% will show up with the |\listfiles| command and cleaned the section
% on the New Font Selection Scheme since this is now included in
% \LaTeX.
%
% \section*{Preface to version 2.1}
%
% This version is identical to 2.0g described in \TUB\ 10\#3 except
% for some internal defaults which are now set depending on the used
% font selection scheme.
%
% This was done to avoid unpleasant surprises if the new font selection
% scheme is in force. For further details see
% section~\ref{sec:newfonts} and \cite{bk:GMS94}.
%
% \section{Introduction}
%
% For our purposes here, ``theorems'' are labelled enunciations,
% often set off from the main text by extra space and a font change.
% Theorems, corollaries, conjectures, definitions, and remarks
% are all instances of ``theorems''.  The ``header'' of these structures
% is composed of a label (such as \textsc{Theorem} or \textsc{Remark})
% and a number which serializes an item in the sequence of items
% with the same label.
%
% Shortly after the introduction of \LaTeX{} at the Fachbereich
% Mathematik in Mainz, the desire to manipulate the layout of
% ``theorems'' arose.  In Mainz, the following two conventions came
% into general use:
% \begin{enumerate}
%   \item The number of the theorem is shown in the margin.
%   \item There is a line break at
%         the end of the theorem header.
% \end{enumerate}
%
% Additionally, some journals require different formats which depend on
% the ``sort of theorem'': e.g.\ often remarks and definitions are set
% in |\upshape|, while |\itshape| is employed for main theorems.
%
% Confronted with these requirements, a theorem environment was
% developed in Mainz which allows separate determination of the
% layout of the ``theorems sets'', comparable to |\pagestyle|.
%
%
% \section{The user interface}
% \subsection{Defining new theorem sets}
%
% \DescribeMacro\newtheorem
% As in the original \LaTeX{} version, the command |\newtheorem|
% defines a new ``theorem set'' or ``theorem-like structure''.
% Two required arguments name the new environment and give the text
% to be typeset with each instance of the new ``set'', while
% an optional argument determines how
% the ``set'' is enumerated:
% \begin{description}
%    \item[\ttfamily \bslash newtheorem\{foo\}\{bar\}]
%       The theorem set {\envfont foo} (whose name is \texttt{bar})
%       uses its own counter.
%    \item[\ttfamily \bslash newtheorem\{foo2\}{[foo]}\{bar2\}]
%       The theorem set {\envfont foo2} (printed name \texttt{bar2})
%       uses the same counter as the theorem set \texttt{foo}.
%    \item[\ttfamily \bslash newtheorem\{foo3\}\{bar3\}{[section]}]
%       The theorem set {\envfont foo3} (printed name \texttt{bar3}) is
%       enumerated within the counter \texttt{section}, i.e.\ with every
%       new |\section| the enumeration begins again with 1, and
%       the enumeration is composed from the section-number and the
%       theorem counter itself.
% \end{description}
%
% \DescribeMacro\theoremstyle
% Additionally, the command |\theoremstyle| can define the layout
% of various, or all, theorem sets.  It should be noted that any
% theorem set defined by |\newtheorem| is typeset in the
% |\theoremstyle| that is current at the time of the definition.
% Thus, the following
% \begin{verbatim}
%    \theoremstyle{break}        \newtheorem{Cor}{Corollary}
%    \theoremstyle{plain}        \newtheorem{Exa}{Example}[section]
%\end{verbatim}                             ^^A No blank at line start
% leads to the result that the set {\envfont Cor} is formatted in the
% style \texttt{break}, while the set {\envfont Exa} and all the
% following ones are formatted in the style \texttt{plain}, unless
% another |\theoremstyle| follows.  Since the definitions installed by
% |\newtheorem| are global, one also can limit |\theoremstyle| locally
% by grouping braces.
%
% \DescribeMacro\theorembodyfont
% The choice of the font for the theorem body is completely
% independent of the chosen |\theoremstyle|; this has proven to
% be very advantageous.  For example,
% \begin{verbatim}
%   {\theorembodyfont{\upshape}          \newtheorem{Rem}{Remark}}
%\end{verbatim}
% defines a theorem set {\envfont Rem}, which will be set in
% |\upshape| in the current layout (which in our example is
% \texttt{plain}).  As with |\theoremstyle|, the |\theorembodyfont|
% chosen is that current at the time of |\newtheorem|.  If
% |\theorembodyfont| is not specified or one defines
% |\theorembodyfont{}|, then the font used will be that defined by the
% |\theoremstyle|.
%
% \DescribeMacro\theoremheaderfont It is also possible to customize
% the font used for the theorem headers.  This is, however, a global
% declaration, and therefore there should be at most one
% |\theoremheaderfont| declaration in the preamble.\footnote{If it is
% actually necessary to have different header fonts, one has to define
% new theorem styles (substituting the desired font) or specify the
% information directly in the \texttt{\bslash newtheorem} declaration
% (the unclean variant).}
%
% \DescribeMacro\theorempreskipamount
% \DescribeMacro\theorempostskipamount Two additional parameters
% affect the vertical space around the theorem environments:
% |\theorempreskipamount| and |\theorempostskipamount| define,
% respectively, the spacing before and after such an environment.
% These parameters apply for all theorem sets and can be manipulated
% with the ordinary length macros.  They are rubber lengths
% (`\textsf{skips}'), and therefore can contain \texttt{plus} and
% \texttt{minus} parts.
%
% Since the definition of theorem sets should---most sensibly---be
% placed in the preamble, we only allow installation there.
% It is therefore possible to release the memory used here after
% |\begin{document}|, in order to make room for other
% applications.
%
%
% \subsection{Existing theorem styles}
%
% The following theorem styles exist to date:
% \begin{deflist}{marginbreak}
%    \item[plain]
%       This theorem style emulates the original \LaTeX{} definition,
%       except that additionally the parameters
%       |\theorem...skipamount| are used.
%    \item[break]
%       In this style, the theorem header is followed by a line
%       break.
%    \item[marginbreak]
%       The theorem number is set in the margin, and there is a line
%       break as in \texttt{break}.
%    \item[changebreak]
%       Like \texttt{break}, but with header number and text
%       interchanged.
%    \item[change]
%       Header number and text are interchanged, without a line break.
%    \item[margin]
%       The number is set in the left margin, without a line break.
% \end{deflist}
% All styles (except \texttt{plain}) select |\slshape| as the default
% |\theorembodyfont|.
%
%
% \subsection{Examples}
%
% Given the above theorem sets {\envfont Cor}, {\envfont Exa}
% and {\envfont Rem},
% suppose that the preamble also contains the declarations:
% \begin{verbatim}
%    \theoremstyle{marginbreak}   \newtheorem{Lem}[Cor]{Lemma}
%    \theoremstyle{change}
%    \theorembodyfont{\itshape}        \newtheorem{Def}[Cor]{Definition}
%
%    \theoremheaderfont{\scshape}
%\end{verbatim}
% Then the following are some typical examples of the typeset output
% resulting from their use.
%
% \begin{Cor}
%    This is a sentence typeset in the theorem environment {\envfont
%    Cor}.
% \end{Cor}
%
% \begin{Exa}
%    This is a sentence typeset in the theorem environment {\envfont
%    Exa}.
% \end{Exa}
%
% \begin{Rem}
%    This is a sentence typeset in the theorem environment {\envfont
%    Rem}.
% \end{Rem}
%
% \begin{Lem}[Ben User]
%    This is a sentence typeset in the theorem environment {\envfont
%    Lem}.
% \end{Lem}
%
% \begin{Def}[Very impressive Definition]
%    This is a sentence typeset in the theorem environment {\envfont
%    Def}.
% \end{Def}
%
% The last two examples show the effect of the optional argument to
% a theorem environment (it is the text typeset in parentheses).
%
%
% \section{Special Considerations}
% \label{sec:newfonts}
%
% Theoremheader and body are implemented as a unit. This means that
% the |\theoremheaderfont|  will inherit characteristics of the
% |\theorembodyfont| in \LaTeXe. Thus, if for example
% |\theorembodyfont| is |\itshape| and |\theoremheaderfont|
% is |\bfseries| the font selected for the header will have the
% characteristics `bold extended italic'. If this is not desired one
% should set the |\theoremheaderfont| to something like
% \begin{verbatim}
%      \theoremheaderfont{\normalfont\bfseries}
%\end{verbatim}
% i.e.\ supplying all necessary font information explicitly.
%
%
%
% \section{Acknowledgements}
%
% The publication of this set of macros was only possible with the
% help of Christina Busse (translating the manuscript into English),
% Joachim Pense (playing the r\^ole of typist), Chris Rowley
% (looking everything over) and many others providing useful
% suggestions.
%
% \MaybeStop{%
%    \begin{thebibliography}{9}
%       \bibitem{bk:GMS94} \textsc{M.~Goossens}, \textsc{F.~Mittelbach}
%         and \textsc{A.~Samarin}.
%         \newblock The \LaTeX{} Companion.
%         \newblock
%         Addison-Wesley, Reading, Massachusetts, 1994.
%       \bibitem{latex:source} \textsc{Lamport, Leslie.} \newblock
%                               \texttt{latex.tex}, version 2.09, date
%                               Feb. 1990.
%    \end{thebibliography}
%    \PrintIndex
%    \PrintChanges}
%
%
% \section{The documentation driver file}
%
% The next bit of code contains the documentation driver file for
% \TeX{}, i.e., the file that will produce the documentation you are
% currently reading. It will be extracted from this file by the
% \texttt{docstrip} program. Since it is the first code in the file
% one can alternatively process this file directly with \LaTeXe{} to
% obtain the documentation.
%
%    \begin{macrocode}
%<*driver>
\documentclass{ltxdoc}

\usepackage{theorem}

%<+driver>% The next few lines define theorem sets which are used
%<+driver>% in the example section of the documentation.

\theoremstyle{break}       \newtheorem{Cor}{Corollary}
\theoremstyle{plain}       \newtheorem{Exa}{Example}[section]
{\theorembodyfont{\upshape}\newtheorem{Rem}{Remark}}
\theoremstyle{marginbreak} \newtheorem{Lem}[Cor]{Lemma}
\theoremstyle{change}
\theorembodyfont{\itshape} \newtheorem{Def}[Cor]{Definition}

\theoremheaderfont{\scshape}

\RecordChanges

\begin{document}
   \DocInput{theorem.dtx}
\end{document}
%</driver>
%    \end{macrocode}
%
% \section{Definition of the Macros}
%
% If the file has been loaded before, we abort immediately. If not the
% package announces itself (this is actually done at the very top of
% the file---the way it is done isn't good style so don't copy it).
% \changes{v2.0e}{89/07/19}{Spaces removed from `typeout'.}
% \changes{v2.2b}{95/11/19}{Announce moved to top of file}
%    \begin{macrocode}
%<*package>
%\@ifundefined{theorem@style}{}{\endinput}
%\def\FMithmInfo{1995/11/19 v2.2b Theorem extension package (FMi)}
%\ProvidesPackage{theorem}[\FMithmInfo]
%    \end{macrocode}
%
%
% \subsection{Definition of theorem styles and fonts}
%
% All the definitions in this file are done globally to allow
% inputting this file inside a group.
%
% \begin{macro}{\theoremstyle}
%    Before a theorem style can be installed, the chosen style must be
%    known. For that reason, we must test to see that
%    |\th@|\meta{style} is known or, more precisely, that it is
%    different from |\relax|. If the style is not known then
%    |\th@plain| is used.
%    \begin{macrocode}
\gdef\theoremstyle#1{%
   \@ifundefined{th@#1}{\@warning
          {Unknown theoremstyle `#1'. Using `plain'}%
          \theorem@style{plain}}%
%    \end{macrocode}
%    We save the theorem style to be used in the token register
%    |\theorem@style|.
%    \begin{macrocode}
      {\theorem@style{#1}}%
%    \end{macrocode}
%    Now we ``evaluate'' the theorem style: this means, we call the
%    macro |\th@|\meta{style} which will activate the relevant
%    definitions which are contained in a separate file.  This is done
%    in a group to suppress changes to the current font.  This could
%    otherwise pose problems together with the new font selection
%    scheme\footnote{When I printed the original article using the new
%    font selection scheme I ended with a document with slanted
%    typefaces (text headings and all) simply because one of the
%    theorem styles used \texttt{\bslash sl} at toplevel.} if the
%    |\th@|\meta{style} is evaluated a second time.
% \changes{v2.1a}{90/02/15}{Group added around `th@..}
% \changes{v2.1a}{90/02/15}{Reseting of `..@begintheorem removed.}
%    \begin{macrocode}
      \begingroup
        \csname th@\the\theorem@style \endcsname
      \endgroup}
%    \end{macrocode}
% \end{macro}
%
%
% \begin{macro}{\@begintheorem}
% \changes{v2.1a}{90/02/15}{Relaxed at toplevel.}
% \begin{macro}{\@opargbegintheorem}
% \changes{v2.1a}{90/02/15}{Relaxed at toplevel.}
%    We reset |\@begintheorem| and |\@opargbegintheorem| to
%    |\relax| since these commands are no longer necessary at
%    toplevel. This will save a few tokens.
%    \begin{macrocode}
\global\let\@begintheorem\relax
\global\let\@opargbegintheorem\relax
%    \end{macrocode}
% \end{macro}
% \end{macro}
%
% \begin{macro}{\theorem@style}
%    Obviously the \textsf{token} register used above has to be
%    allocated. To assure the utmost compatibility with the
%    original \LaTeX{} definition, we set the default theorem style to
%    \texttt{plain}, which implements the usual \LaTeX{} convention.
% \changes{v2.0f}{89/07/20}{Unnecessary `global' removed.}
%    \begin{macrocode}
\newtoks\theorem@style
\global\theorem@style{plain}
%    \end{macrocode}
% \end{macro}
%
% \begin{macro}{\theorembodyfont}
% \begin{macro}{\theorem@bodyfont}
%    If the |\theorembodyfont| is set by the user then it should not
%    interact with the default font set in the theorem style. When the
%    new font selection is in force this may happen if, for example,
%    the default is |\itshape| and the new |\theorembodyfont| is
%    |\sffamily|. So we add a |\reset@font| command in front of the
%    user definition.
% \changes{v2.0f}{89/07/20}{Unnecessary `global' removed.}
% \changes{v2.0g}{89/09/19}{`theoremfont to `theorembodyfont.}
% \changes{v2.1c}{92/06/22}{`reset@font added and logic changed.}
% \changes{v2.1d}{92/10/28}{Reset to empty if arg is empty}
%    \begin{macrocode}
\gdef\theorembodyfont#1{%
%    \end{macrocode}
%    We check if the argument supplied is empty and if so put nothing
%    into the |\theorem@bodyfont| token register to allow for
%    |\theorembodyfont{}| as a mean of using the default of the
%    current |\theoremstyle|.
%    \begin{macrocode}
   \def\@tempa{#1}%
   \ifx\@tempa\@empty
    \theorem@bodyfont{}%
   \else
    \theorem@bodyfont{\reset@font#1}%
   \fi
   }
\newtoks\theorem@bodyfont
\global\theorem@bodyfont{}
%    \end{macrocode}
% \end{macro}
% \end{macro}
%
%
% \begin{macro}{\theoremheaderfont}
% \changes{v2.0g}{89/09/19}{Macro added.}
%    The font for the theorem headers is handled differently because
%    this definition applies to all theorem styles.
%    \begin{macrocode}
\gdef\theoremheaderfont#1{\gdef\theorem@headerfont{#1}%
%    \end{macrocode}
%    After using the macro once it is redefined to produce an error
%    message.
%    \begin{macrocode}
       \gdef\theoremheaderfont##1{%
        \typeout{\string\theoremheaderfont\space should be used
                 only once.}}}
%    \end{macrocode}
% \end{macro}
%
% \begin{macro}{\theorem@headerfont}
% \changes{v2.0g}{89/09/19}{Macro added.}
%    To set the |\theorem@headerfont| default we first test if
%    the new fontselection scheme is in force.
% \changes{v2.1a}{90/02/15}{Support for new font selection.}
%    \begin{macrocode}
\ifx\upshape\undefined
%    \end{macrocode}
%    If not we define it to expand into |\bfseries|. We don't use
%    |\let| just in case a following style option redefines this
%    macro.
%    \begin{macrocode}
\gdef\theorem@headerfont{\bfseries}
%    \end{macrocode}
%    Otherwise we reset the current shape before calling |\bfseries|.
%    \begin{macrocode}
\else \gdef\theorem@headerfont{\normalfont\bfseries}\fi
%    \end{macrocode}
% \end{macro}
%
% \begin{macro}{\th@plain}
% \begin{macro}{\th@break}
% \begin{macro}{\th@marginbreak}
% \begin{macro}{\th@changebreak}
% \begin{macro}{\th@change}
% \begin{macro}{\th@margin}
%    The different styles are defined in macros such as
%    |\th@plain|.  Since memory space is precious in
%    ``non-Big-versions'', we have to avoid offering too many unused
%    definitions.  Therefore we define these styles in separate files
%    that can be loaded on demand.  Thus the commands themselves only
%    load these files. We use |\@input@|, a \LaTeXe{} internal command
%    that ensures that the file will be listed with |\listfiles|.
% \changes{v2.2a}{94/02/02}{Use `@input@ to load theorem layout files}
%    \begin{macrocode}
\gdef\th@plain{\@input@{thp.sty}}
\gdef\th@break{\@input@{thb.sty}}
\gdef\th@marginbreak{\@input@{thmb.sty}}
\gdef\th@changebreak{\@input@{thcb.sty}}
\gdef\th@change{\@input@{thc.sty}}
\gdef\th@margin{\@input@{thm.sty}}
%    \end{macrocode}
%    This list will be expanded when new styles become
%    available.  For testing,  just append new theorem substyles
%    as document options.
% \end{macro}
% \end{macro}
% \end{macro}
% \end{macro}
% \end{macro}
% \end{macro}
%
%
%
% \subsection{Definition of a new theorem set}
%
% As already pointed out, a new theorem environment can be defined in
% three different ways:
% \begin{verbatim}
%    \newtheorem{Lem}{Lemma}
%    \newtheorem{Lem}{Lemma}[section]
%    \newtheorem{Lem}[Theorem]{Lemma}
%\end{verbatim}
% The function of the macro |\newtheorem| is to recognize these cases
% and then to branch into one of the three macros |\@ynthm|, |\@xnthm|
% or |\@othm|.  This mechanism is adopted unchanged from
% \cite{latex:source}; the essential point here is that, for example,
% in the second case, the arguments \texttt{Lem}, \texttt{Lemma} and
% \texttt{section} are passed over to the macro |\@xnthm|.
%
% We inspect this case first because the others present fewer problems,
% and thus are easily derived from this one.
%
% \begin{macro}{\@xnthm}
%    For our example arguments, the macro |\@xnthm| must fulfill
%    the following:
%    \begin{itemize}
%       \item
%          Define a new \LaTeX-counter `Lem'
%       \item
%          reset this counter within a |\section|
%       \item
%          define the macro |\theLem|
%       \item
%          define the environment macros |\Lem| and
%          |\endLem| using the current
%          |\theoremstyle| and |\theorem@bodyfont|.
%    \end{itemize}
%    Obviously, all this should happen only if the first argument
%    of |\@xnthm| (i.e.\ \texttt{Lem} in our example) is chosen so
%    as not to conflict with any previously defined commands or
%    environments.  This test is performed by the \LaTeX{} macro
%    |\@ifdefinable|.
%    \begin{macrocode}
\gdef\@xnthm#1#2[#3]{\expandafter\@ifdefinable\csname #1\endcsname
%    \end{macrocode}
%    Therefore, the first argument of |\@ifdefinable| is the
%    expansion (in the example, |\Lem|) of
%    |\csname||#1\endcsname|.  The second argument is
%    executed only if the test has been completed successfully.
%    \begin{macrocode}
   {%
%    \end{macrocode}
%    Now we define the new counter. The names of the \LaTeX{}
%    macros employed should speak for themselves:
% \changes{v2.2b}{1995/10/07}{Use \cs{@newctr}}
% \changes{v2.2c}{1995/11/23}{Use \cs{@newctr} properly.}
%    \begin{macrocode}
    \@definecounter{#1}\@newctr{#1}[#3]%
%    \end{macrocode}
%    Using |\@newctr| will give a proper error message if the counter
%    in |#3| is not defined.
%    In defining `|\theLem|' we must generate the desired macro
%    name by use of |\expandafter| and |\csname|.
%    \begin{macrocode}
    \expandafter\xdef\csname the#1\endcsname
%    \end{macrocode}
%    An |\xdef| is used in order to make the definition global,
%    and to ensure that it contains the replacement texts of
%    |\@thmcountersep| and |\@thmcounter|.\footnote{These
%    two macros can be defined by the document style.  Their default
%    values produce a `\texttt{.}' as separation and an arabic
%    representation of the number.} However, not everything should be
%    expanded. For example, it saves space to use |\thesection|
%    instead of its---at times---lengthy expansion.
%    \begin{macrocode}
      {\expandafter \noexpand \csname the#3\endcsname
       \@thmcountersep \@thmcounter{#1}}%
%    \end{macrocode}
%    Thus with the defaults of \LaTeX{}, |\theLem| would be
%    replaced by the command sequence |\thesection.\arabic{Lem}|.
%
%    We will now look at the definition of the macro which is executed
%    at the beginning of the actual environment (in our example this
%    macro is |\Lem|). It should be noted that we use an
%    ``|\expandafter| trick'' to expand only certain parts of the
%    replacement text at the time of the definition.
%    \begin{macrocode}
    \def\@tempa{\global\@namedef{#1}}%
    \expandafter \@tempa \expandafter{%
%    \end{macrocode}
%    First, the macro that contains the current definitions of
%    |\@begintheorem| and |\@opargtheorem| should be called
%    up.  The name of this macro---as is already known---has the form
%    |\th@|\meta{theorem\ style}; therefore, it must be called by
%    \begin{macrocode}
      \csname th@\the \theorem@style
%    \end{macrocode}
%    In addition the default theorem font should be changeable, i.e.\
%    we have to insert the contents of |\theorem@bodyfont|. For
%    that reason, we expand even further, beyond |\endcsname|,
%    and thus insert the contents of the \textsf{token} register
%    |\theorem@bodyfont| in the replacement text.
%    \begin{macrocode}
            \expandafter \endcsname \the \theorem@bodyfont
%    \end{macrocode}
%    Now it is time to call the macro |\@thm| which takes over
%    the further processing.  It has two arguments: the current
%    counter name (in our example, \texttt{Lem}), and the text of the
%    label (in our example, \texttt{Lemma}).
%    \begin{macrocode}
     \@thm{#1}{#2}}%
%    \end{macrocode}
%    With this, the `sub-definition' is complete.  The macro
%    |\@endtheorem| ends a theorem environment and is, so far, nothing
%    but an |\endtrivlist|.  (Hence it is defined globally, and not
%    within the theorem styles.\footnote{This has to be changed as
%    soon as theorem styles that change \texttt{\bslash @endtheorem}
%    exist.  In such a case, all existing styles must be changed as
%    well since they will have to reset the macro.}) Therefore, we can
%    set it equivalent to the macro that ends the theorem set (in our
%    example, |\endLem|).  However, if some day theorem styles exist
%    that do change |\@endtheorem|, we would have to use the
%    commented-out line instead.
%    \begin{macrocode}
    \global \expandafter \let \csname end#1\endcsname \@endtheorem
%   \global\@namedef{end#1}{\@endtheorem}%
%    \end{macrocode}
%    With these commands all the required definitions are employed,
%    unless the test |\@ifdefinable| has failed.  Therefore, we
%    end the second argument of this macro and with it the definition
%    of |\@xnthm|.
%    \begin{macrocode}
   }}
%    \end{macrocode}
% \end{macro}
%
% \begin{macro}{\@ynthm}
%    The definition of |\@ynthm| is completely analogous. In
%    this case the new counter that is defined is not reset within
%    another counter; thus the definition of |\the...|
%    is simplified:
%    \begin{macrocode}
\gdef\@ynthm#1#2{\expandafter\@ifdefinable\csname #1\endcsname
   {\@definecounter{#1}%
    \expandafter\xdef\csname the#1\endcsname{\@thmcounter{#1}}%
%    \end{macrocode}
%    The rest of the definition corresponds literally to that of
%    |\@xnthm|:
%    \begin{macrocode}
    \def\@tempa{\global\@namedef{#1}}\expandafter \@tempa
     \expandafter{\csname th@\the \theorem@style \expandafter
     \endcsname \the\theorem@bodyfont \@thm{#1}{#2}}%
    \global \expandafter \let \csname end#1\endcsname \@endtheorem}}
%    \end{macrocode}
% \end{macro}
%
% \begin{macro}{\@othm}
%    The definition of |\@othm| does not contain anything new.
%    \begin{macrocode}
\gdef\@othm#1[#2]#3{%
%    \end{macrocode}
% \changes{v2.2b}{95/10/07}{Build in check for counter defined}
%    We do not define a new counter but instead use one that has
%    already been defined.  Thus the only definition we need is that
%    of this pseudo-counter (i.e.\ |\the|\meta{env.\ name}).
%    First we check if |#2| corresponds to a known counter name.
%    \begin{macrocode}
  \expandafter\ifx\csname c@#2\endcsname\relax
   \@nocounterr{#2}%
  \else
   \expandafter\@ifdefinable\csname #1\endcsname
   {\expandafter \xdef \csname the#1\endcsname
     {\expandafter \noexpand \csname the#2\endcsname}%
%    \end{macrocode}
%    All other parts of the definition can be adopted from
%    |\@xnthm|. We have to remember, though, that in this case
%    the name of the current counter and the theorem label have moved
%    to the second and third arguments.
%    \begin{macrocode}
    \def\@tempa{\global\@namedef{#1}}\expandafter \@tempa
     \expandafter{\csname th@\the \theorem@style \expandafter
     \endcsname \the\theorem@bodyfont \@thm{#2}{#3}}%
    \global \expandafter \let \csname end#1\endcsname \@endtheorem}%
  \fi}
%    \end{macrocode}
% \end{macro}
%
% \subsection{Macros that are employed in a {\envfont theorem}
%    environment}
%
% \begin{macro}{\@thm}
%    The macro |\@thm| has to increase the current counter.  Then,
%    depending on whether the environment has (or does not have) an
%    optional argument, it has to branch into either |\@begintheorem|
%    or |\@opargtheorem|.
%    \begin{macrocode}
\gdef\@thm#1#2{\refstepcounter{#1}%
%    \end{macrocode}
%    Now we start a {\envfont trivlist} environment, and give
%    |\@topsep| and |\@topsepadd| the values of the \textsf{skip}
%    registers |\theorempreskipamount| and |\theorempostskipamount|.
%    The value in |\@topsep| is the vertical space that is inserted by
%    the first (and only) |\item| in our |\trivlist| whilst
%    |\@topsepadd| is inserted by |\@endparenv| at the end of that
%    {\envfont trivlist} environment.  By using these registers, we
%    obtain the desired space around a {\envfont theorem} environment.
%    \begin{macrocode}
   \trivlist
   \@topsep \theorempreskipamount               % used by first \item
   \@topsepadd \theorempostskipamount           % used by \@endparenv
%    \end{macrocode}
%    Now we have to test whether an optional argument has been given.
% \changes{v2.0d}{89/07/18}{Forgotten percent added.}
%    \begin{macrocode}
   \@ifnextchar [%
%    \end{macrocode}
%    If there is an optional argument, we will call |\@ythm|, and
%    move the arguments read back into the input stream.
%    \begin{macrocode}
   {\@ythm{#1}{#2}}%
%    \end{macrocode}
%    If not, we call |\@begintheorem|. Its first argument is the
%    name of the theorem set (hence the second argument of
%    |\@thm|). Its second argument is the macro that produces the
%    current number.
%    \begin{macrocode}
   {\@begintheorem{#2}{\csname the#1\endcsname}\ignorespaces}}
%    \end{macrocode}
% \end{macro}
%
% \begin{macro}{\@xthm}
% \begin{macro}{\@ythm}
%    Both these macros were originally called by |\@thm|.  We do
%    not need |\@xthm| anymore, hence we reset it to
%    |\relax|. The definition of |\@ythm| has not changed at
%    all from its definition in \LaTeX. In order to make the macros
%    easier to understand, we will
%    nevertheless present it (commented out).
%    \begin{macrocode}
\global\let\@xthm\relax
% \def\@ythm#1#2[#3]{\@opargbegintheorem{#2}{\csname
%           the#1\endcsname}{#3}\ignorespaces}
%</package>
%    \end{macrocode}
%    The primitive |\ignorespaces| in |\@ythm| and
%    |\@thm| is needed to remove the spaces between the
%    |\begin{...}| and the actual text.
% \end{macro}
% \end{macro}
%
%
% \subsection{Definition of the theorem substyles}
%
% As already pointed out, the theorem substyles, defined below, are
% only loaded when necessary.  Note that all these substyles, except
% \texttt{plain}, have |\slshape| as the default body font.
%
% \subsubsection{The \texttt{plain} style}
%
% \label{dok:th@plain} As the following macros use |@|, we have
% to locally set the |\catcode| of this symbol to ``letter''.
% This happens within a group, so that we do not have to worry about
% which |\catcode| that symbol had before.
%    \begin{macrocode}
%<*thp>
\begingroup \makeatletter
%    \end{macrocode}
% Since we are now within a group, we must make all definitions
% globally.  First we make sure that \texttt{theorem.sty} is loaded.
% This will allow us to use this file as a document style option
% without having to call \texttt{theorem} itself as an option.  At the
% same time, we assure that at least version 2 is loaded, since
% |\theorem@style| was not defined in earlier versions.
%    \begin{macrocode}
\@ifundefined{theorem@style}{\input{theorem.sty}}{}
\ProvidesFile{thp.sty}
             [\FMithmInfo]
%    \end{macrocode}
%
%
% \begin{macro}{\th@plain}
% \begin{macro}{\@begintheorem}
% \begin{macro}{\@opargbegintheorem}
%    |\theoremstyle{plain}| corresponds to the original
%    definition, except that the distances to the surrounding text are
%    determined by the parameters |\theorempreskipamount| and
%    |\theorempostskipamount|.
%    First we set the default body font.
%    \begin{macrocode}
\gdef\th@plain{\normalfont\itshape
%    \end{macrocode}
%    Then we define |\@begintheorem| and
%    |\@opargbegintheorem|. These two macros define how the
%    header of a theorem is typeset.  |\@opargbegintheorem| will
%    be called if a \textsf{theorem} environment with an optional
%    argument is encountered; otherwise, the header is constructed by
%    calling |\@begintheorem|.  If one of these macros is
%    executed, we are within a \textsf{trivlist} environment started by
%    |\@thm|.  So the theorem header is produced with an
%    |\item| command.
%
%    Instead of specifying the header font directly, all standard
%    theorem styles use the |\theorem@headerfont| macro to allow
%    customization.  The extra space (|\labelsep|) is necessary
%    because of problems in the \textsf{trivlist} environment.
%    \begin{macrocode}
  \def\@begintheorem##1##2{%
        \item[\hskip\labelsep \theorem@headerfont ##1\ ##2]}%
%    \end{macrocode}
%    The definition of |\@opargbegintheorem| is completely
%    analogous. The only difference is the fact that there exists a
%    third argument (which is the optional parameter of the
%    environment and contains additional information about the
%    theorem).  Customarily we enclose it in parentheses.
%    \begin{macrocode}
\def\@opargbegintheorem##1##2##3{%
   \item[\hskip\labelsep \theorem@headerfont ##1\ ##2\ (##3)]}}
%    \end{macrocode}
% \end{macro}
% \end{macro}
% \end{macro}
%
% \noindent
% We conclude with an |\endgroup| to restore the
% |\catcode| of |@|.
%    \begin{macrocode}
\endgroup
%</thp>
%    \end{macrocode}
%
%
% \subsubsection{The \texttt{break} style}
%
% This style option is stored in the file |thb.sty|.
% For the next two lines see the documentation for |\th@plain| on
% page~\pageref{dok:th@plain}.
%    \begin{macrocode}
%<*thb>
\begingroup \makeatletter
\@ifundefined{theorem@style}{\input{theorem.sty}}{}
\ProvidesFile{thb.sty}
             [\FMithmInfo]
%    \end{macrocode}
%
% \begin{macro}{\th@break}
%    |\theoremstyle{break}| produces a line break after the name
%    of the theorem.  The font is |\slshape|.  Hence, we define
%    |\th@break| as follows:
%    \begin{macrocode}
\gdef\th@break{\normalfont\slshape
  \def\@begintheorem##1##2{\item[%
%    \end{macrocode}
%    We run into the following problem: it is not possible to create
%    the header with |\item[|\meta{title}|]| and then start
%    a new line by, for example, |\mbox{}\\|. Such a definition
%    will fail whenever a list environment follows immediately.  With
%    the above construction, the |\mbox{}| causes the switch
%    |@inlabel| (cf.\ definition of |\list| and
%    |\trivlist| in \cite{latex:source}) to be set to \texttt{false}
%    and so the following list will insert additional vertical
%    space (|\topskip|). This is quite annoying. Therefore, we
%    create the line break within the |\item|. In order to ensure
%    that the text will begin at the proper position in the following
%    line, we simply pretend that the label does not take any
%    room.\footnote{This will lead to problems whenever very high
%    symbols occurring in the line tower into the
%    heading. So, something else has to be done here sometime.}
%    \begin{macrocode}
       \rlap{\vbox{\hbox{\hskip \labelsep\theorem@headerfont ##1\ ##2}%
                   \hbox{\strut}}}]}%
%    \end{macrocode}
%    Again, the definition of |\@opargbegintheorem| is
%    completely analogous.
%    \begin{macrocode}
\def\@opargbegintheorem##1##2##3{%
  \item[\rlap{\vbox{\hbox{\hskip \labelsep \theorem@headerfont
                     ##1\ ##2\ (##3)}%
                    \hbox{\strut}}}]}}
\endgroup
%</thb>
%    \end{macrocode}
% \end{macro}
%
%
% \subsubsection{The \texttt{changebreak} style}
%
%    \begin{macrocode}
% This style option is stored in the file |thcb.sty|.
%    \begin{macrocode}
%<*thcb>
\begingroup \makeatletter
\@ifundefined{theorem@style}{\input{theorem.sty}}{}
\ProvidesFile{thcb.sty}
             [\FMithmInfo]
%    \end{macrocode}
%
% \begin{macro}{\th@changebreak}
%    The \texttt{change-break} theorem style is like \texttt{break}
%    but with interchange of theorem name and theorem number.  Thus we
%    define |\th@changebreak| as follows:
%    \begin{macrocode}
\gdef\th@changebreak{\normalfont\slshape
  \def\@begintheorem##1##2{\item
       [\rlap{\vbox{\hbox{\hskip\labelsep \theorem@headerfont ##2\ ##1}%
                    \hbox{\strut}}}]}%
\def\@opargbegintheorem##1##2##3{%
  \item[\rlap{\vbox{\hbox{\hskip\labelsep \theorem@headerfont
                     ##2\ ##1\ (##3)}%
                    \hbox{\strut}}}]}}
\endgroup
%</thcb>
%    \end{macrocode}
% \end{macro}
%
%
% \subsubsection{The \texttt{change} style}
%
% This style option is stored in the file |thc.sty|.
%    \begin{macrocode}
%<*thc>
\begingroup \makeatletter
\@ifundefined{theorem@style}{\input{theorem.sty}}{}
\ProvidesFile{thc.sty}
             [\FMithmInfo]
%    \end{macrocode}
%
% \begin{macro}{\th@change}
%    The \texttt{change} theorem style corresponds to the
%    \texttt{change break} style without a linebreak after the header.
%    To say it in another way, it's the same as the \texttt{plain}
%    style but with number and name interchanged and |\slshape| as the
%    default font.
%    \begin{macrocode}
\gdef\th@change{\normalfont\slshape
  \def\@begintheorem##1##2{\item
       [\hskip\labelsep \theorem@headerfont ##2\ ##1]}%
\def\@opargbegintheorem##1##2##3{%
  \item[\hskip\labelsep \theorem@headerfont ##2\ ##1\ (##3)]}}
\endgroup
%</thc>
%    \end{macrocode}
% \end{macro}
%
%
% \subsubsection{The \texttt{marginbreak} style}
%
% This style option is the one used most often at Mainz.  It is saved
% in the file |thmb.sty|.
%    \begin{macrocode}
%<*thmb>
\begingroup \makeatletter
\@ifundefined{theorem@style}{\input{theorem.sty}}{}
\ProvidesFile{thmb.sty}
             [\FMithmInfo]
%    \end{macrocode}
%
% \begin{macro}{\th@marginbreak}
%    The \texttt{margin break} style is nearly the same as the
%    \texttt{change break} style. The only difference is the placement
%    of the theorem number.  We use |\llap| to place it in the left
%    margin.
%
%    In this style |\labelsep| denotes the separation between the
%    number and the text.
%    \begin{macrocode}
\gdef\th@marginbreak{\normalfont\slshape
  \def\@begintheorem##1##2{\item
       [\rlap{\vbox{\theorem@headerfont
                    \hbox{\llap{##2}\hskip\labelsep ##1}%
                    \hbox{\strut}}}]}%
\def\@opargbegintheorem##1##2##3{%
  \item[\rlap{\vbox{\theorem@headerfont
                    \hbox{\llap{##2}\hskip\labelsep ##1\ (##3)}%
                    \hbox{\strut}}}]}}
\endgroup
%</thmb>
%    \end{macrocode}
% \changes{v2.1c}{90/02/17}{`endgroup outside macrocode env.}
% \end{macro}
%
%
%
% \subsubsection{The \texttt{margin} style}
%
% This style option is stored in the file |thm.sty|.
%    \begin{macrocode}
%<*thm>
\begingroup \makeatletter
\@ifundefined{theorem@style}{\input{theorem.sty}}{}
\ProvidesFile{thm.sty}
             [\FMithmInfo]
%    \end{macrocode}
%
% \begin{macro}{\th@margin}
%    Again this is only a variant of the theorem styles described
%    above without any new ideas.
%    \begin{macrocode}
\gdef\th@margin{\normalfont\slshape
  \def\@begintheorem##1##2{\item
       [\theorem@headerfont \llap{##2}\hskip\labelsep ##1]}%
\def\@opargbegintheorem##1##2##3{%
  \item[\theorem@headerfont \llap{##2}\hskip\labelsep ##1\ (##3)]}}
\endgroup
%</thm>
%    \end{macrocode}
% \end{macro}
%
%
%
%
% \subsection{Final Definitions}
%
% \begin{macro}{\theorempreskipamount}
% \begin{macro}{\theorempostskipamount}
%    The \textsf{skip} parameters that regulate the vertical empty space
%    before and after the {\envfont theorem} environment have
%    to be allocated as well.
% \changes{v2.0f}{89/07/20}{Unnecessary `global' removed.}
%    \begin{macrocode}
%<*package>
\newskip\theorempreskipamount
\newskip\theorempostskipamount
%    \end{macrocode}
%    Since we have used the same values for all theorem sets, we now can
%    assign them.
%    \begin{macrocode}
\global\setlength\theorempreskipamount{12pt plus 5pt minus 3pt}
\global\setlength\theorempostskipamount{8pt plus 3pt minus 1.5pt}
%    \end{macrocode}
% \end{macro}
% \end{macro}
%
%
% \begin{macro}{\@endtheorem}
%    The same holds for the macro |\@endtheorem|, which ends a
%    {\envfont theorem} environment. Since it is the same for all
%    theorem sets, it is removed from the macros |\th@|\meta{style}.
%    It simply ends the {\envfont trivlist} environment, which was
%    begun in |\@thm|.
%    \begin{macrocode}
\global\let\@endtheorem=\endtrivlist
%    \end{macrocode}
% \end{macro}
%
%
%
% \begin{macro}{\@preamblecmds}
%    All macros defined above are to be used only in the preamble.
%    Therefore, we insert them in |\@preamblecmds| which will disable
%    them at begin document. This is done by the internal \LaTeXe{}
%    command |\@onlypreamble|.
% \changes{v2.0g}{89/09/19}{`theoremheaderfont and `theorembodyfont
%    added.}
% \changes{v2.2a}{94/02/03}{Use \LaTeXe{} interface}
%    \begin{macrocode}
\@onlypreamble\@xnthm
\@onlypreamble\@ynthm
\@onlypreamble\@othm
\@onlypreamble\newtheorem
\@onlypreamble\theoremstyle
\@onlypreamble\theorembodyfont
\@onlypreamble\theoremheaderfont
%    \end{macrocode}
% \end{macro}
%
% Finally we declare the \texttt{plain} theorem style to be the default.
% \changes{v2.1b}{90/06/04}{Fixed missing `item by declaring default
%    style.}
%    \begin{macrocode}
\theoremstyle{plain}
%</package>
%    \end{macrocode}
%
% \Finale

\endinput
}{}
\ProvidesFile{thcb.sty}
             [\FMithmInfo]
%    \end{macrocode}
%
% \begin{macro}{\th@changebreak}
%    The \texttt{change-break} theorem style is like \texttt{break}
%    but with interchange of theorem name and theorem number.  Thus we
%    define |\th@changebreak| as follows:
%    \begin{macrocode}
\gdef\th@changebreak{\normalfont\slshape
  \def\@begintheorem##1##2{\item
       [\rlap{\vbox{\hbox{\hskip\labelsep \theorem@headerfont ##2\ ##1}%
                    \hbox{\strut}}}]}%
\def\@opargbegintheorem##1##2##3{%
  \item[\rlap{\vbox{\hbox{\hskip\labelsep \theorem@headerfont
                     ##2\ ##1\ (##3)}%
                    \hbox{\strut}}}]}}
\endgroup
%</thcb>
%    \end{macrocode}
% \end{macro}
%
%
% \subsubsection{The \texttt{change} style}
%
% This style option is stored in the file |thc.sty|.
%    \begin{macrocode}
%<*thc>
\begingroup \makeatletter
\@ifundefined{theorem@style}{% \iffalse meta-comment
%
% Copyright (C) 1993-2024
%
% The LaTeX Project and any individual authors listed elsewhere
% in this file.
%
% This file is part of the Standard LaTeX `Tools Bundle'.
% -------------------------------------------------------
%
% It may be distributed and/or modified under the
% conditions of the LaTeX Project Public License, either version 1.3c
% of this license or (at your option) any later version.
% The latest version of this license is in
%    https://www.latex-project.org/lppl.txt
% and version 1.3c or later is part of all distributions of LaTeX
% version 2005/12/01 or later.
%
% The list of all files belonging to the LaTeX `Tools Bundle' is
% given in the file `manifest.txt'.
%
% \fi
%
%%
%
%
\def\FMithmInfo{2023/07/05 v2.2c Theorem extension package (FMi)}
%
% \ProvidesFile{theorem.dtx}[\FMithmInfo]
%\iffalse   % this is a METACOMMENT !
%%
%% Package `theorem' to use with LaTeX2e
%% Copyright (C) 1989-1995 Frank Mittelbach, all rights reserved.
%<+package>\@ifundefined{theorem@style}{}{\endinput}
%<+package>\NeedsTeXFormat{LaTeX2e}[1995/06/01]
%<+package>\ProvidesPackage{theorem}
%<+package>         [\FMithmInfo]
%
% \fi
%
% \changes{v1.0}{87/05/07}{First draft.}
% \changes{v1.0a}{87/05/07}{`thistheoremstyle' added.}
% \changes{v1.1a}{87/07/29}{New concept: redefine the internal
%                          \protect\LaTeX{} commands `@xnthm' etc.}
% \changes{v1.1b}{87/07/31}{`theoremfont' used to specify font.}
% \changes{v1.2a}{88/01/08}{`@endtheorem' now globally defined.}
% \changes{v1.2b}{88/01/11}{`newtheorem' only allowed in the preamble.}
% \changes{v1.3a}{88/02/14}{`theoremfont' now a toks register.}
%
% \changes{v2.0a}{88/11/16}{New implementation and new documentation.}
% \changes{v2.0a}{88/11/16}{No layout changes but internal modifications
%                           old substyles won't work.}
% \changes{v2.0b}{89/01/30}{`global' used for all definitions.}
% \changes{v2.0c}{89/06/15}{Documentation translated to English.
%                            Thanks to LLL and Nina}
% \changes{v2.1a}{90/02/15}{Support for new font selection.}
% \changes{v2.2a}{94/02/03}{Cleanup for LaTeX2e}
%
% \DoNotIndex{\@endparenv,\@ifdefinable,\@ifnextchar,\@ifundefined}
% \DoNotIndex{\@namedef,\@nameuse,\@spaces,\@tempa,\@warning}
% \DoNotIndex{\ }
% \DoNotIndex{\begingroup,\bfseries}
% \DoNotIndex{\csname}
% \DoNotIndex{\def,\do,\docdate}
% \DoNotIndex{\endcsname,\endgroup,\endinput,\endtrivlist,\expandafter}
% \DoNotIndex{\filedate,\fileversion}
% \DoNotIndex{\gdef,\global}
% \DoNotIndex{\hbox,\hskip}
% \DoNotIndex{\ignorespaces,\input,\itshape,\item}
% \DoNotIndex{\labelsep,\let,\llap}
% \DoNotIndex{\makeatletter}
% \DoNotIndex{\noexpand}
% \DoNotIndex{\refstepcounter,\relax,\rlap}
% \DoNotIndex{\setlength,\slshape,\space,\string,\strut}
% \DoNotIndex{\the,\trivlist,\typeout}
% \DoNotIndex{\vbox}
% \DoNotIndex{\xdef}
%
% \def\deflabel#1{\ttfamily #1\hfill}
% \def\deflist#1{\begin{list}{}{\settowidth\labelwidth{\ttfamily #1}%
%                               \setlength\leftmargin\labelwidth
%                               \addtolength\leftmargin\labelsep
%                               \let\makelabel\deflabel}}
% \def\enddeflist{\end{list}}
% \def\envfont{\normalfont\ttfamily}
%
% \MakeShortVerb{\|}
%
% \newcommand{\TUB}{{\sl TUGboat\/}}
%
% \GetFileInfo{theorem.dtx}
% \title{An Extension of the
%        \LaTeX{} theorem environment\thanks{This file
%        has version number \fileversion, last
%        revised \filedate.}}
% \author{Frank Mittelbach}
%
% \MaintainedByLaTeXTeam{tools}
% \maketitle
%
% \begin{abstract}
%    The macros described in this paper yield an extension of the
%    \LaTeX{} theorem mechanism. It is designed to satisfy the
%    different requirements of various journals. Thus, the
%    layout of the ``theorems'' can be manipulated by determining a
%    ``style''. This article describes not only the use, but
%    also the definition, of the necessary macros.
% \end{abstract}
%
% \section*{Preface to version 2.2}
%
% For \LaTeXe{} this package did not need any fundamental changes. I
% only modified the messages generated so that theorem layout styles
% will show up with the |\listfiles| command and cleaned the section
% on the New Font Selection Scheme since this is now included in
% \LaTeX.
%
% \section*{Preface to version 2.1}
%
% This version is identical to 2.0g described in \TUB\ 10\#3 except
% for some internal defaults which are now set depending on the used
% font selection scheme.
%
% This was done to avoid unpleasant surprises if the new font selection
% scheme is in force. For further details see
% section~\ref{sec:newfonts} and \cite{bk:GMS94}.
%
% \section{Introduction}
%
% For our purposes here, ``theorems'' are labelled enunciations,
% often set off from the main text by extra space and a font change.
% Theorems, corollaries, conjectures, definitions, and remarks
% are all instances of ``theorems''.  The ``header'' of these structures
% is composed of a label (such as \textsc{Theorem} or \textsc{Remark})
% and a number which serializes an item in the sequence of items
% with the same label.
%
% Shortly after the introduction of \LaTeX{} at the Fachbereich
% Mathematik in Mainz, the desire to manipulate the layout of
% ``theorems'' arose.  In Mainz, the following two conventions came
% into general use:
% \begin{enumerate}
%   \item The number of the theorem is shown in the margin.
%   \item There is a line break at
%         the end of the theorem header.
% \end{enumerate}
%
% Additionally, some journals require different formats which depend on
% the ``sort of theorem'': e.g.\ often remarks and definitions are set
% in |\upshape|, while |\itshape| is employed for main theorems.
%
% Confronted with these requirements, a theorem environment was
% developed in Mainz which allows separate determination of the
% layout of the ``theorems sets'', comparable to |\pagestyle|.
%
%
% \section{The user interface}
% \subsection{Defining new theorem sets}
%
% \DescribeMacro\newtheorem
% As in the original \LaTeX{} version, the command |\newtheorem|
% defines a new ``theorem set'' or ``theorem-like structure''.
% Two required arguments name the new environment and give the text
% to be typeset with each instance of the new ``set'', while
% an optional argument determines how
% the ``set'' is enumerated:
% \begin{description}
%    \item[\ttfamily \bslash newtheorem\{foo\}\{bar\}]
%       The theorem set {\envfont foo} (whose name is \texttt{bar})
%       uses its own counter.
%    \item[\ttfamily \bslash newtheorem\{foo2\}{[foo]}\{bar2\}]
%       The theorem set {\envfont foo2} (printed name \texttt{bar2})
%       uses the same counter as the theorem set \texttt{foo}.
%    \item[\ttfamily \bslash newtheorem\{foo3\}\{bar3\}{[section]}]
%       The theorem set {\envfont foo3} (printed name \texttt{bar3}) is
%       enumerated within the counter \texttt{section}, i.e.\ with every
%       new |\section| the enumeration begins again with 1, and
%       the enumeration is composed from the section-number and the
%       theorem counter itself.
% \end{description}
%
% \DescribeMacro\theoremstyle
% Additionally, the command |\theoremstyle| can define the layout
% of various, or all, theorem sets.  It should be noted that any
% theorem set defined by |\newtheorem| is typeset in the
% |\theoremstyle| that is current at the time of the definition.
% Thus, the following
% \begin{verbatim}
%    \theoremstyle{break}        \newtheorem{Cor}{Corollary}
%    \theoremstyle{plain}        \newtheorem{Exa}{Example}[section]
%\end{verbatim}                             ^^A No blank at line start
% leads to the result that the set {\envfont Cor} is formatted in the
% style \texttt{break}, while the set {\envfont Exa} and all the
% following ones are formatted in the style \texttt{plain}, unless
% another |\theoremstyle| follows.  Since the definitions installed by
% |\newtheorem| are global, one also can limit |\theoremstyle| locally
% by grouping braces.
%
% \DescribeMacro\theorembodyfont
% The choice of the font for the theorem body is completely
% independent of the chosen |\theoremstyle|; this has proven to
% be very advantageous.  For example,
% \begin{verbatim}
%   {\theorembodyfont{\upshape}          \newtheorem{Rem}{Remark}}
%\end{verbatim}
% defines a theorem set {\envfont Rem}, which will be set in
% |\upshape| in the current layout (which in our example is
% \texttt{plain}).  As with |\theoremstyle|, the |\theorembodyfont|
% chosen is that current at the time of |\newtheorem|.  If
% |\theorembodyfont| is not specified or one defines
% |\theorembodyfont{}|, then the font used will be that defined by the
% |\theoremstyle|.
%
% \DescribeMacro\theoremheaderfont It is also possible to customize
% the font used for the theorem headers.  This is, however, a global
% declaration, and therefore there should be at most one
% |\theoremheaderfont| declaration in the preamble.\footnote{If it is
% actually necessary to have different header fonts, one has to define
% new theorem styles (substituting the desired font) or specify the
% information directly in the \texttt{\bslash newtheorem} declaration
% (the unclean variant).}
%
% \DescribeMacro\theorempreskipamount
% \DescribeMacro\theorempostskipamount Two additional parameters
% affect the vertical space around the theorem environments:
% |\theorempreskipamount| and |\theorempostskipamount| define,
% respectively, the spacing before and after such an environment.
% These parameters apply for all theorem sets and can be manipulated
% with the ordinary length macros.  They are rubber lengths
% (`\textsf{skips}'), and therefore can contain \texttt{plus} and
% \texttt{minus} parts.
%
% Since the definition of theorem sets should---most sensibly---be
% placed in the preamble, we only allow installation there.
% It is therefore possible to release the memory used here after
% |\begin{document}|, in order to make room for other
% applications.
%
%
% \subsection{Existing theorem styles}
%
% The following theorem styles exist to date:
% \begin{deflist}{marginbreak}
%    \item[plain]
%       This theorem style emulates the original \LaTeX{} definition,
%       except that additionally the parameters
%       |\theorem...skipamount| are used.
%    \item[break]
%       In this style, the theorem header is followed by a line
%       break.
%    \item[marginbreak]
%       The theorem number is set in the margin, and there is a line
%       break as in \texttt{break}.
%    \item[changebreak]
%       Like \texttt{break}, but with header number and text
%       interchanged.
%    \item[change]
%       Header number and text are interchanged, without a line break.
%    \item[margin]
%       The number is set in the left margin, without a line break.
% \end{deflist}
% All styles (except \texttt{plain}) select |\slshape| as the default
% |\theorembodyfont|.
%
%
% \subsection{Examples}
%
% Given the above theorem sets {\envfont Cor}, {\envfont Exa}
% and {\envfont Rem},
% suppose that the preamble also contains the declarations:
% \begin{verbatim}
%    \theoremstyle{marginbreak}   \newtheorem{Lem}[Cor]{Lemma}
%    \theoremstyle{change}
%    \theorembodyfont{\itshape}        \newtheorem{Def}[Cor]{Definition}
%
%    \theoremheaderfont{\scshape}
%\end{verbatim}
% Then the following are some typical examples of the typeset output
% resulting from their use.
%
% \begin{Cor}
%    This is a sentence typeset in the theorem environment {\envfont
%    Cor}.
% \end{Cor}
%
% \begin{Exa}
%    This is a sentence typeset in the theorem environment {\envfont
%    Exa}.
% \end{Exa}
%
% \begin{Rem}
%    This is a sentence typeset in the theorem environment {\envfont
%    Rem}.
% \end{Rem}
%
% \begin{Lem}[Ben User]
%    This is a sentence typeset in the theorem environment {\envfont
%    Lem}.
% \end{Lem}
%
% \begin{Def}[Very impressive Definition]
%    This is a sentence typeset in the theorem environment {\envfont
%    Def}.
% \end{Def}
%
% The last two examples show the effect of the optional argument to
% a theorem environment (it is the text typeset in parentheses).
%
%
% \section{Special Considerations}
% \label{sec:newfonts}
%
% Theoremheader and body are implemented as a unit. This means that
% the |\theoremheaderfont|  will inherit characteristics of the
% |\theorembodyfont| in \LaTeXe. Thus, if for example
% |\theorembodyfont| is |\itshape| and |\theoremheaderfont|
% is |\bfseries| the font selected for the header will have the
% characteristics `bold extended italic'. If this is not desired one
% should set the |\theoremheaderfont| to something like
% \begin{verbatim}
%      \theoremheaderfont{\normalfont\bfseries}
%\end{verbatim}
% i.e.\ supplying all necessary font information explicitly.
%
%
%
% \section{Acknowledgements}
%
% The publication of this set of macros was only possible with the
% help of Christina Busse (translating the manuscript into English),
% Joachim Pense (playing the r\^ole of typist), Chris Rowley
% (looking everything over) and many others providing useful
% suggestions.
%
% \MaybeStop{%
%    \begin{thebibliography}{9}
%       \bibitem{bk:GMS94} \textsc{M.~Goossens}, \textsc{F.~Mittelbach}
%         and \textsc{A.~Samarin}.
%         \newblock The \LaTeX{} Companion.
%         \newblock
%         Addison-Wesley, Reading, Massachusetts, 1994.
%       \bibitem{latex:source} \textsc{Lamport, Leslie.} \newblock
%                               \texttt{latex.tex}, version 2.09, date
%                               Feb. 1990.
%    \end{thebibliography}
%    \PrintIndex
%    \PrintChanges}
%
%
% \section{The documentation driver file}
%
% The next bit of code contains the documentation driver file for
% \TeX{}, i.e., the file that will produce the documentation you are
% currently reading. It will be extracted from this file by the
% \texttt{docstrip} program. Since it is the first code in the file
% one can alternatively process this file directly with \LaTeXe{} to
% obtain the documentation.
%
%    \begin{macrocode}
%<*driver>
\documentclass{ltxdoc}

\usepackage{theorem}

%<+driver>% The next few lines define theorem sets which are used
%<+driver>% in the example section of the documentation.

\theoremstyle{break}       \newtheorem{Cor}{Corollary}
\theoremstyle{plain}       \newtheorem{Exa}{Example}[section]
{\theorembodyfont{\upshape}\newtheorem{Rem}{Remark}}
\theoremstyle{marginbreak} \newtheorem{Lem}[Cor]{Lemma}
\theoremstyle{change}
\theorembodyfont{\itshape} \newtheorem{Def}[Cor]{Definition}

\theoremheaderfont{\scshape}

\RecordChanges

\begin{document}
   \DocInput{theorem.dtx}
\end{document}
%</driver>
%    \end{macrocode}
%
% \section{Definition of the Macros}
%
% If the file has been loaded before, we abort immediately. If not the
% package announces itself (this is actually done at the very top of
% the file---the way it is done isn't good style so don't copy it).
% \changes{v2.0e}{89/07/19}{Spaces removed from `typeout'.}
% \changes{v2.2b}{95/11/19}{Announce moved to top of file}
%    \begin{macrocode}
%<*package>
%\@ifundefined{theorem@style}{}{\endinput}
%\def\FMithmInfo{1995/11/19 v2.2b Theorem extension package (FMi)}
%\ProvidesPackage{theorem}[\FMithmInfo]
%    \end{macrocode}
%
%
% \subsection{Definition of theorem styles and fonts}
%
% All the definitions in this file are done globally to allow
% inputting this file inside a group.
%
% \begin{macro}{\theoremstyle}
%    Before a theorem style can be installed, the chosen style must be
%    known. For that reason, we must test to see that
%    |\th@|\meta{style} is known or, more precisely, that it is
%    different from |\relax|. If the style is not known then
%    |\th@plain| is used.
%    \begin{macrocode}
\gdef\theoremstyle#1{%
   \@ifundefined{th@#1}{\@warning
          {Unknown theoremstyle `#1'. Using `plain'}%
          \theorem@style{plain}}%
%    \end{macrocode}
%    We save the theorem style to be used in the token register
%    |\theorem@style|.
%    \begin{macrocode}
      {\theorem@style{#1}}%
%    \end{macrocode}
%    Now we ``evaluate'' the theorem style: this means, we call the
%    macro |\th@|\meta{style} which will activate the relevant
%    definitions which are contained in a separate file.  This is done
%    in a group to suppress changes to the current font.  This could
%    otherwise pose problems together with the new font selection
%    scheme\footnote{When I printed the original article using the new
%    font selection scheme I ended with a document with slanted
%    typefaces (text headings and all) simply because one of the
%    theorem styles used \texttt{\bslash sl} at toplevel.} if the
%    |\th@|\meta{style} is evaluated a second time.
% \changes{v2.1a}{90/02/15}{Group added around `th@..}
% \changes{v2.1a}{90/02/15}{Reseting of `..@begintheorem removed.}
%    \begin{macrocode}
      \begingroup
        \csname th@\the\theorem@style \endcsname
      \endgroup}
%    \end{macrocode}
% \end{macro}
%
%
% \begin{macro}{\@begintheorem}
% \changes{v2.1a}{90/02/15}{Relaxed at toplevel.}
% \begin{macro}{\@opargbegintheorem}
% \changes{v2.1a}{90/02/15}{Relaxed at toplevel.}
%    We reset |\@begintheorem| and |\@opargbegintheorem| to
%    |\relax| since these commands are no longer necessary at
%    toplevel. This will save a few tokens.
%    \begin{macrocode}
\global\let\@begintheorem\relax
\global\let\@opargbegintheorem\relax
%    \end{macrocode}
% \end{macro}
% \end{macro}
%
% \begin{macro}{\theorem@style}
%    Obviously the \textsf{token} register used above has to be
%    allocated. To assure the utmost compatibility with the
%    original \LaTeX{} definition, we set the default theorem style to
%    \texttt{plain}, which implements the usual \LaTeX{} convention.
% \changes{v2.0f}{89/07/20}{Unnecessary `global' removed.}
%    \begin{macrocode}
\newtoks\theorem@style
\global\theorem@style{plain}
%    \end{macrocode}
% \end{macro}
%
% \begin{macro}{\theorembodyfont}
% \begin{macro}{\theorem@bodyfont}
%    If the |\theorembodyfont| is set by the user then it should not
%    interact with the default font set in the theorem style. When the
%    new font selection is in force this may happen if, for example,
%    the default is |\itshape| and the new |\theorembodyfont| is
%    |\sffamily|. So we add a |\reset@font| command in front of the
%    user definition.
% \changes{v2.0f}{89/07/20}{Unnecessary `global' removed.}
% \changes{v2.0g}{89/09/19}{`theoremfont to `theorembodyfont.}
% \changes{v2.1c}{92/06/22}{`reset@font added and logic changed.}
% \changes{v2.1d}{92/10/28}{Reset to empty if arg is empty}
%    \begin{macrocode}
\gdef\theorembodyfont#1{%
%    \end{macrocode}
%    We check if the argument supplied is empty and if so put nothing
%    into the |\theorem@bodyfont| token register to allow for
%    |\theorembodyfont{}| as a mean of using the default of the
%    current |\theoremstyle|.
%    \begin{macrocode}
   \def\@tempa{#1}%
   \ifx\@tempa\@empty
    \theorem@bodyfont{}%
   \else
    \theorem@bodyfont{\reset@font#1}%
   \fi
   }
\newtoks\theorem@bodyfont
\global\theorem@bodyfont{}
%    \end{macrocode}
% \end{macro}
% \end{macro}
%
%
% \begin{macro}{\theoremheaderfont}
% \changes{v2.0g}{89/09/19}{Macro added.}
%    The font for the theorem headers is handled differently because
%    this definition applies to all theorem styles.
%    \begin{macrocode}
\gdef\theoremheaderfont#1{\gdef\theorem@headerfont{#1}%
%    \end{macrocode}
%    After using the macro once it is redefined to produce an error
%    message.
%    \begin{macrocode}
       \gdef\theoremheaderfont##1{%
        \typeout{\string\theoremheaderfont\space should be used
                 only once.}}}
%    \end{macrocode}
% \end{macro}
%
% \begin{macro}{\theorem@headerfont}
% \changes{v2.0g}{89/09/19}{Macro added.}
%    To set the |\theorem@headerfont| default we first test if
%    the new fontselection scheme is in force.
% \changes{v2.1a}{90/02/15}{Support for new font selection.}
%    \begin{macrocode}
\ifx\upshape\undefined
%    \end{macrocode}
%    If not we define it to expand into |\bfseries|. We don't use
%    |\let| just in case a following style option redefines this
%    macro.
%    \begin{macrocode}
\gdef\theorem@headerfont{\bfseries}
%    \end{macrocode}
%    Otherwise we reset the current shape before calling |\bfseries|.
%    \begin{macrocode}
\else \gdef\theorem@headerfont{\normalfont\bfseries}\fi
%    \end{macrocode}
% \end{macro}
%
% \begin{macro}{\th@plain}
% \begin{macro}{\th@break}
% \begin{macro}{\th@marginbreak}
% \begin{macro}{\th@changebreak}
% \begin{macro}{\th@change}
% \begin{macro}{\th@margin}
%    The different styles are defined in macros such as
%    |\th@plain|.  Since memory space is precious in
%    ``non-Big-versions'', we have to avoid offering too many unused
%    definitions.  Therefore we define these styles in separate files
%    that can be loaded on demand.  Thus the commands themselves only
%    load these files. We use |\@input@|, a \LaTeXe{} internal command
%    that ensures that the file will be listed with |\listfiles|.
% \changes{v2.2a}{94/02/02}{Use `@input@ to load theorem layout files}
%    \begin{macrocode}
\gdef\th@plain{\@input@{thp.sty}}
\gdef\th@break{\@input@{thb.sty}}
\gdef\th@marginbreak{\@input@{thmb.sty}}
\gdef\th@changebreak{\@input@{thcb.sty}}
\gdef\th@change{\@input@{thc.sty}}
\gdef\th@margin{\@input@{thm.sty}}
%    \end{macrocode}
%    This list will be expanded when new styles become
%    available.  For testing,  just append new theorem substyles
%    as document options.
% \end{macro}
% \end{macro}
% \end{macro}
% \end{macro}
% \end{macro}
% \end{macro}
%
%
%
% \subsection{Definition of a new theorem set}
%
% As already pointed out, a new theorem environment can be defined in
% three different ways:
% \begin{verbatim}
%    \newtheorem{Lem}{Lemma}
%    \newtheorem{Lem}{Lemma}[section]
%    \newtheorem{Lem}[Theorem]{Lemma}
%\end{verbatim}
% The function of the macro |\newtheorem| is to recognize these cases
% and then to branch into one of the three macros |\@ynthm|, |\@xnthm|
% or |\@othm|.  This mechanism is adopted unchanged from
% \cite{latex:source}; the essential point here is that, for example,
% in the second case, the arguments \texttt{Lem}, \texttt{Lemma} and
% \texttt{section} are passed over to the macro |\@xnthm|.
%
% We inspect this case first because the others present fewer problems,
% and thus are easily derived from this one.
%
% \begin{macro}{\@xnthm}
%    For our example arguments, the macro |\@xnthm| must fulfill
%    the following:
%    \begin{itemize}
%       \item
%          Define a new \LaTeX-counter `Lem'
%       \item
%          reset this counter within a |\section|
%       \item
%          define the macro |\theLem|
%       \item
%          define the environment macros |\Lem| and
%          |\endLem| using the current
%          |\theoremstyle| and |\theorem@bodyfont|.
%    \end{itemize}
%    Obviously, all this should happen only if the first argument
%    of |\@xnthm| (i.e.\ \texttt{Lem} in our example) is chosen so
%    as not to conflict with any previously defined commands or
%    environments.  This test is performed by the \LaTeX{} macro
%    |\@ifdefinable|.
%    \begin{macrocode}
\gdef\@xnthm#1#2[#3]{\expandafter\@ifdefinable\csname #1\endcsname
%    \end{macrocode}
%    Therefore, the first argument of |\@ifdefinable| is the
%    expansion (in the example, |\Lem|) of
%    |\csname||#1\endcsname|.  The second argument is
%    executed only if the test has been completed successfully.
%    \begin{macrocode}
   {%
%    \end{macrocode}
%    Now we define the new counter. The names of the \LaTeX{}
%    macros employed should speak for themselves:
% \changes{v2.2b}{1995/10/07}{Use \cs{@newctr}}
% \changes{v2.2c}{1995/11/23}{Use \cs{@newctr} properly.}
%    \begin{macrocode}
    \@definecounter{#1}\@newctr{#1}[#3]%
%    \end{macrocode}
%    Using |\@newctr| will give a proper error message if the counter
%    in |#3| is not defined.
%    In defining `|\theLem|' we must generate the desired macro
%    name by use of |\expandafter| and |\csname|.
%    \begin{macrocode}
    \expandafter\xdef\csname the#1\endcsname
%    \end{macrocode}
%    An |\xdef| is used in order to make the definition global,
%    and to ensure that it contains the replacement texts of
%    |\@thmcountersep| and |\@thmcounter|.\footnote{These
%    two macros can be defined by the document style.  Their default
%    values produce a `\texttt{.}' as separation and an arabic
%    representation of the number.} However, not everything should be
%    expanded. For example, it saves space to use |\thesection|
%    instead of its---at times---lengthy expansion.
%    \begin{macrocode}
      {\expandafter \noexpand \csname the#3\endcsname
       \@thmcountersep \@thmcounter{#1}}%
%    \end{macrocode}
%    Thus with the defaults of \LaTeX{}, |\theLem| would be
%    replaced by the command sequence |\thesection.\arabic{Lem}|.
%
%    We will now look at the definition of the macro which is executed
%    at the beginning of the actual environment (in our example this
%    macro is |\Lem|). It should be noted that we use an
%    ``|\expandafter| trick'' to expand only certain parts of the
%    replacement text at the time of the definition.
%    \begin{macrocode}
    \def\@tempa{\global\@namedef{#1}}%
    \expandafter \@tempa \expandafter{%
%    \end{macrocode}
%    First, the macro that contains the current definitions of
%    |\@begintheorem| and |\@opargtheorem| should be called
%    up.  The name of this macro---as is already known---has the form
%    |\th@|\meta{theorem\ style}; therefore, it must be called by
%    \begin{macrocode}
      \csname th@\the \theorem@style
%    \end{macrocode}
%    In addition the default theorem font should be changeable, i.e.\
%    we have to insert the contents of |\theorem@bodyfont|. For
%    that reason, we expand even further, beyond |\endcsname|,
%    and thus insert the contents of the \textsf{token} register
%    |\theorem@bodyfont| in the replacement text.
%    \begin{macrocode}
            \expandafter \endcsname \the \theorem@bodyfont
%    \end{macrocode}
%    Now it is time to call the macro |\@thm| which takes over
%    the further processing.  It has two arguments: the current
%    counter name (in our example, \texttt{Lem}), and the text of the
%    label (in our example, \texttt{Lemma}).
%    \begin{macrocode}
     \@thm{#1}{#2}}%
%    \end{macrocode}
%    With this, the `sub-definition' is complete.  The macro
%    |\@endtheorem| ends a theorem environment and is, so far, nothing
%    but an |\endtrivlist|.  (Hence it is defined globally, and not
%    within the theorem styles.\footnote{This has to be changed as
%    soon as theorem styles that change \texttt{\bslash @endtheorem}
%    exist.  In such a case, all existing styles must be changed as
%    well since they will have to reset the macro.}) Therefore, we can
%    set it equivalent to the macro that ends the theorem set (in our
%    example, |\endLem|).  However, if some day theorem styles exist
%    that do change |\@endtheorem|, we would have to use the
%    commented-out line instead.
%    \begin{macrocode}
    \global \expandafter \let \csname end#1\endcsname \@endtheorem
%   \global\@namedef{end#1}{\@endtheorem}%
%    \end{macrocode}
%    With these commands all the required definitions are employed,
%    unless the test |\@ifdefinable| has failed.  Therefore, we
%    end the second argument of this macro and with it the definition
%    of |\@xnthm|.
%    \begin{macrocode}
   }}
%    \end{macrocode}
% \end{macro}
%
% \begin{macro}{\@ynthm}
%    The definition of |\@ynthm| is completely analogous. In
%    this case the new counter that is defined is not reset within
%    another counter; thus the definition of |\the...|
%    is simplified:
%    \begin{macrocode}
\gdef\@ynthm#1#2{\expandafter\@ifdefinable\csname #1\endcsname
   {\@definecounter{#1}%
    \expandafter\xdef\csname the#1\endcsname{\@thmcounter{#1}}%
%    \end{macrocode}
%    The rest of the definition corresponds literally to that of
%    |\@xnthm|:
%    \begin{macrocode}
    \def\@tempa{\global\@namedef{#1}}\expandafter \@tempa
     \expandafter{\csname th@\the \theorem@style \expandafter
     \endcsname \the\theorem@bodyfont \@thm{#1}{#2}}%
    \global \expandafter \let \csname end#1\endcsname \@endtheorem}}
%    \end{macrocode}
% \end{macro}
%
% \begin{macro}{\@othm}
%    The definition of |\@othm| does not contain anything new.
%    \begin{macrocode}
\gdef\@othm#1[#2]#3{%
%    \end{macrocode}
% \changes{v2.2b}{95/10/07}{Build in check for counter defined}
%    We do not define a new counter but instead use one that has
%    already been defined.  Thus the only definition we need is that
%    of this pseudo-counter (i.e.\ |\the|\meta{env.\ name}).
%    First we check if |#2| corresponds to a known counter name.
%    \begin{macrocode}
  \expandafter\ifx\csname c@#2\endcsname\relax
   \@nocounterr{#2}%
  \else
   \expandafter\@ifdefinable\csname #1\endcsname
   {\expandafter \xdef \csname the#1\endcsname
     {\expandafter \noexpand \csname the#2\endcsname}%
%    \end{macrocode}
%    All other parts of the definition can be adopted from
%    |\@xnthm|. We have to remember, though, that in this case
%    the name of the current counter and the theorem label have moved
%    to the second and third arguments.
%    \begin{macrocode}
    \def\@tempa{\global\@namedef{#1}}\expandafter \@tempa
     \expandafter{\csname th@\the \theorem@style \expandafter
     \endcsname \the\theorem@bodyfont \@thm{#2}{#3}}%
    \global \expandafter \let \csname end#1\endcsname \@endtheorem}%
  \fi}
%    \end{macrocode}
% \end{macro}
%
% \subsection{Macros that are employed in a {\envfont theorem}
%    environment}
%
% \begin{macro}{\@thm}
%    The macro |\@thm| has to increase the current counter.  Then,
%    depending on whether the environment has (or does not have) an
%    optional argument, it has to branch into either |\@begintheorem|
%    or |\@opargtheorem|.
%    \begin{macrocode}
\gdef\@thm#1#2{\refstepcounter{#1}%
%    \end{macrocode}
%    Now we start a {\envfont trivlist} environment, and give
%    |\@topsep| and |\@topsepadd| the values of the \textsf{skip}
%    registers |\theorempreskipamount| and |\theorempostskipamount|.
%    The value in |\@topsep| is the vertical space that is inserted by
%    the first (and only) |\item| in our |\trivlist| whilst
%    |\@topsepadd| is inserted by |\@endparenv| at the end of that
%    {\envfont trivlist} environment.  By using these registers, we
%    obtain the desired space around a {\envfont theorem} environment.
%    \begin{macrocode}
   \trivlist
   \@topsep \theorempreskipamount               % used by first \item
   \@topsepadd \theorempostskipamount           % used by \@endparenv
%    \end{macrocode}
%    Now we have to test whether an optional argument has been given.
% \changes{v2.0d}{89/07/18}{Forgotten percent added.}
%    \begin{macrocode}
   \@ifnextchar [%
%    \end{macrocode}
%    If there is an optional argument, we will call |\@ythm|, and
%    move the arguments read back into the input stream.
%    \begin{macrocode}
   {\@ythm{#1}{#2}}%
%    \end{macrocode}
%    If not, we call |\@begintheorem|. Its first argument is the
%    name of the theorem set (hence the second argument of
%    |\@thm|). Its second argument is the macro that produces the
%    current number.
%    \begin{macrocode}
   {\@begintheorem{#2}{\csname the#1\endcsname}\ignorespaces}}
%    \end{macrocode}
% \end{macro}
%
% \begin{macro}{\@xthm}
% \begin{macro}{\@ythm}
%    Both these macros were originally called by |\@thm|.  We do
%    not need |\@xthm| anymore, hence we reset it to
%    |\relax|. The definition of |\@ythm| has not changed at
%    all from its definition in \LaTeX. In order to make the macros
%    easier to understand, we will
%    nevertheless present it (commented out).
%    \begin{macrocode}
\global\let\@xthm\relax
% \def\@ythm#1#2[#3]{\@opargbegintheorem{#2}{\csname
%           the#1\endcsname}{#3}\ignorespaces}
%</package>
%    \end{macrocode}
%    The primitive |\ignorespaces| in |\@ythm| and
%    |\@thm| is needed to remove the spaces between the
%    |\begin{...}| and the actual text.
% \end{macro}
% \end{macro}
%
%
% \subsection{Definition of the theorem substyles}
%
% As already pointed out, the theorem substyles, defined below, are
% only loaded when necessary.  Note that all these substyles, except
% \texttt{plain}, have |\slshape| as the default body font.
%
% \subsubsection{The \texttt{plain} style}
%
% \label{dok:th@plain} As the following macros use |@|, we have
% to locally set the |\catcode| of this symbol to ``letter''.
% This happens within a group, so that we do not have to worry about
% which |\catcode| that symbol had before.
%    \begin{macrocode}
%<*thp>
\begingroup \makeatletter
%    \end{macrocode}
% Since we are now within a group, we must make all definitions
% globally.  First we make sure that \texttt{theorem.sty} is loaded.
% This will allow us to use this file as a document style option
% without having to call \texttt{theorem} itself as an option.  At the
% same time, we assure that at least version 2 is loaded, since
% |\theorem@style| was not defined in earlier versions.
%    \begin{macrocode}
\@ifundefined{theorem@style}{\input{theorem.sty}}{}
\ProvidesFile{thp.sty}
             [\FMithmInfo]
%    \end{macrocode}
%
%
% \begin{macro}{\th@plain}
% \begin{macro}{\@begintheorem}
% \begin{macro}{\@opargbegintheorem}
%    |\theoremstyle{plain}| corresponds to the original
%    definition, except that the distances to the surrounding text are
%    determined by the parameters |\theorempreskipamount| and
%    |\theorempostskipamount|.
%    First we set the default body font.
%    \begin{macrocode}
\gdef\th@plain{\normalfont\itshape
%    \end{macrocode}
%    Then we define |\@begintheorem| and
%    |\@opargbegintheorem|. These two macros define how the
%    header of a theorem is typeset.  |\@opargbegintheorem| will
%    be called if a \textsf{theorem} environment with an optional
%    argument is encountered; otherwise, the header is constructed by
%    calling |\@begintheorem|.  If one of these macros is
%    executed, we are within a \textsf{trivlist} environment started by
%    |\@thm|.  So the theorem header is produced with an
%    |\item| command.
%
%    Instead of specifying the header font directly, all standard
%    theorem styles use the |\theorem@headerfont| macro to allow
%    customization.  The extra space (|\labelsep|) is necessary
%    because of problems in the \textsf{trivlist} environment.
%    \begin{macrocode}
  \def\@begintheorem##1##2{%
        \item[\hskip\labelsep \theorem@headerfont ##1\ ##2]}%
%    \end{macrocode}
%    The definition of |\@opargbegintheorem| is completely
%    analogous. The only difference is the fact that there exists a
%    third argument (which is the optional parameter of the
%    environment and contains additional information about the
%    theorem).  Customarily we enclose it in parentheses.
%    \begin{macrocode}
\def\@opargbegintheorem##1##2##3{%
   \item[\hskip\labelsep \theorem@headerfont ##1\ ##2\ (##3)]}}
%    \end{macrocode}
% \end{macro}
% \end{macro}
% \end{macro}
%
% \noindent
% We conclude with an |\endgroup| to restore the
% |\catcode| of |@|.
%    \begin{macrocode}
\endgroup
%</thp>
%    \end{macrocode}
%
%
% \subsubsection{The \texttt{break} style}
%
% This style option is stored in the file |thb.sty|.
% For the next two lines see the documentation for |\th@plain| on
% page~\pageref{dok:th@plain}.
%    \begin{macrocode}
%<*thb>
\begingroup \makeatletter
\@ifundefined{theorem@style}{\input{theorem.sty}}{}
\ProvidesFile{thb.sty}
             [\FMithmInfo]
%    \end{macrocode}
%
% \begin{macro}{\th@break}
%    |\theoremstyle{break}| produces a line break after the name
%    of the theorem.  The font is |\slshape|.  Hence, we define
%    |\th@break| as follows:
%    \begin{macrocode}
\gdef\th@break{\normalfont\slshape
  \def\@begintheorem##1##2{\item[%
%    \end{macrocode}
%    We run into the following problem: it is not possible to create
%    the header with |\item[|\meta{title}|]| and then start
%    a new line by, for example, |\mbox{}\\|. Such a definition
%    will fail whenever a list environment follows immediately.  With
%    the above construction, the |\mbox{}| causes the switch
%    |@inlabel| (cf.\ definition of |\list| and
%    |\trivlist| in \cite{latex:source}) to be set to \texttt{false}
%    and so the following list will insert additional vertical
%    space (|\topskip|). This is quite annoying. Therefore, we
%    create the line break within the |\item|. In order to ensure
%    that the text will begin at the proper position in the following
%    line, we simply pretend that the label does not take any
%    room.\footnote{This will lead to problems whenever very high
%    symbols occurring in the line tower into the
%    heading. So, something else has to be done here sometime.}
%    \begin{macrocode}
       \rlap{\vbox{\hbox{\hskip \labelsep\theorem@headerfont ##1\ ##2}%
                   \hbox{\strut}}}]}%
%    \end{macrocode}
%    Again, the definition of |\@opargbegintheorem| is
%    completely analogous.
%    \begin{macrocode}
\def\@opargbegintheorem##1##2##3{%
  \item[\rlap{\vbox{\hbox{\hskip \labelsep \theorem@headerfont
                     ##1\ ##2\ (##3)}%
                    \hbox{\strut}}}]}}
\endgroup
%</thb>
%    \end{macrocode}
% \end{macro}
%
%
% \subsubsection{The \texttt{changebreak} style}
%
%    \begin{macrocode}
% This style option is stored in the file |thcb.sty|.
%    \begin{macrocode}
%<*thcb>
\begingroup \makeatletter
\@ifundefined{theorem@style}{\input{theorem.sty}}{}
\ProvidesFile{thcb.sty}
             [\FMithmInfo]
%    \end{macrocode}
%
% \begin{macro}{\th@changebreak}
%    The \texttt{change-break} theorem style is like \texttt{break}
%    but with interchange of theorem name and theorem number.  Thus we
%    define |\th@changebreak| as follows:
%    \begin{macrocode}
\gdef\th@changebreak{\normalfont\slshape
  \def\@begintheorem##1##2{\item
       [\rlap{\vbox{\hbox{\hskip\labelsep \theorem@headerfont ##2\ ##1}%
                    \hbox{\strut}}}]}%
\def\@opargbegintheorem##1##2##3{%
  \item[\rlap{\vbox{\hbox{\hskip\labelsep \theorem@headerfont
                     ##2\ ##1\ (##3)}%
                    \hbox{\strut}}}]}}
\endgroup
%</thcb>
%    \end{macrocode}
% \end{macro}
%
%
% \subsubsection{The \texttt{change} style}
%
% This style option is stored in the file |thc.sty|.
%    \begin{macrocode}
%<*thc>
\begingroup \makeatletter
\@ifundefined{theorem@style}{\input{theorem.sty}}{}
\ProvidesFile{thc.sty}
             [\FMithmInfo]
%    \end{macrocode}
%
% \begin{macro}{\th@change}
%    The \texttt{change} theorem style corresponds to the
%    \texttt{change break} style without a linebreak after the header.
%    To say it in another way, it's the same as the \texttt{plain}
%    style but with number and name interchanged and |\slshape| as the
%    default font.
%    \begin{macrocode}
\gdef\th@change{\normalfont\slshape
  \def\@begintheorem##1##2{\item
       [\hskip\labelsep \theorem@headerfont ##2\ ##1]}%
\def\@opargbegintheorem##1##2##3{%
  \item[\hskip\labelsep \theorem@headerfont ##2\ ##1\ (##3)]}}
\endgroup
%</thc>
%    \end{macrocode}
% \end{macro}
%
%
% \subsubsection{The \texttt{marginbreak} style}
%
% This style option is the one used most often at Mainz.  It is saved
% in the file |thmb.sty|.
%    \begin{macrocode}
%<*thmb>
\begingroup \makeatletter
\@ifundefined{theorem@style}{\input{theorem.sty}}{}
\ProvidesFile{thmb.sty}
             [\FMithmInfo]
%    \end{macrocode}
%
% \begin{macro}{\th@marginbreak}
%    The \texttt{margin break} style is nearly the same as the
%    \texttt{change break} style. The only difference is the placement
%    of the theorem number.  We use |\llap| to place it in the left
%    margin.
%
%    In this style |\labelsep| denotes the separation between the
%    number and the text.
%    \begin{macrocode}
\gdef\th@marginbreak{\normalfont\slshape
  \def\@begintheorem##1##2{\item
       [\rlap{\vbox{\theorem@headerfont
                    \hbox{\llap{##2}\hskip\labelsep ##1}%
                    \hbox{\strut}}}]}%
\def\@opargbegintheorem##1##2##3{%
  \item[\rlap{\vbox{\theorem@headerfont
                    \hbox{\llap{##2}\hskip\labelsep ##1\ (##3)}%
                    \hbox{\strut}}}]}}
\endgroup
%</thmb>
%    \end{macrocode}
% \changes{v2.1c}{90/02/17}{`endgroup outside macrocode env.}
% \end{macro}
%
%
%
% \subsubsection{The \texttt{margin} style}
%
% This style option is stored in the file |thm.sty|.
%    \begin{macrocode}
%<*thm>
\begingroup \makeatletter
\@ifundefined{theorem@style}{\input{theorem.sty}}{}
\ProvidesFile{thm.sty}
             [\FMithmInfo]
%    \end{macrocode}
%
% \begin{macro}{\th@margin}
%    Again this is only a variant of the theorem styles described
%    above without any new ideas.
%    \begin{macrocode}
\gdef\th@margin{\normalfont\slshape
  \def\@begintheorem##1##2{\item
       [\theorem@headerfont \llap{##2}\hskip\labelsep ##1]}%
\def\@opargbegintheorem##1##2##3{%
  \item[\theorem@headerfont \llap{##2}\hskip\labelsep ##1\ (##3)]}}
\endgroup
%</thm>
%    \end{macrocode}
% \end{macro}
%
%
%
%
% \subsection{Final Definitions}
%
% \begin{macro}{\theorempreskipamount}
% \begin{macro}{\theorempostskipamount}
%    The \textsf{skip} parameters that regulate the vertical empty space
%    before and after the {\envfont theorem} environment have
%    to be allocated as well.
% \changes{v2.0f}{89/07/20}{Unnecessary `global' removed.}
%    \begin{macrocode}
%<*package>
\newskip\theorempreskipamount
\newskip\theorempostskipamount
%    \end{macrocode}
%    Since we have used the same values for all theorem sets, we now can
%    assign them.
%    \begin{macrocode}
\global\setlength\theorempreskipamount{12pt plus 5pt minus 3pt}
\global\setlength\theorempostskipamount{8pt plus 3pt minus 1.5pt}
%    \end{macrocode}
% \end{macro}
% \end{macro}
%
%
% \begin{macro}{\@endtheorem}
%    The same holds for the macro |\@endtheorem|, which ends a
%    {\envfont theorem} environment. Since it is the same for all
%    theorem sets, it is removed from the macros |\th@|\meta{style}.
%    It simply ends the {\envfont trivlist} environment, which was
%    begun in |\@thm|.
%    \begin{macrocode}
\global\let\@endtheorem=\endtrivlist
%    \end{macrocode}
% \end{macro}
%
%
%
% \begin{macro}{\@preamblecmds}
%    All macros defined above are to be used only in the preamble.
%    Therefore, we insert them in |\@preamblecmds| which will disable
%    them at begin document. This is done by the internal \LaTeXe{}
%    command |\@onlypreamble|.
% \changes{v2.0g}{89/09/19}{`theoremheaderfont and `theorembodyfont
%    added.}
% \changes{v2.2a}{94/02/03}{Use \LaTeXe{} interface}
%    \begin{macrocode}
\@onlypreamble\@xnthm
\@onlypreamble\@ynthm
\@onlypreamble\@othm
\@onlypreamble\newtheorem
\@onlypreamble\theoremstyle
\@onlypreamble\theorembodyfont
\@onlypreamble\theoremheaderfont
%    \end{macrocode}
% \end{macro}
%
% Finally we declare the \texttt{plain} theorem style to be the default.
% \changes{v2.1b}{90/06/04}{Fixed missing `item by declaring default
%    style.}
%    \begin{macrocode}
\theoremstyle{plain}
%</package>
%    \end{macrocode}
%
% \Finale

\endinput
}{}
\ProvidesFile{thc.sty}
             [\FMithmInfo]
%    \end{macrocode}
%
% \begin{macro}{\th@change}
%    The \texttt{change} theorem style corresponds to the
%    \texttt{change break} style without a linebreak after the header.
%    To say it in another way, it's the same as the \texttt{plain}
%    style but with number and name interchanged and |\slshape| as the
%    default font.
%    \begin{macrocode}
\gdef\th@change{\normalfont\slshape
  \def\@begintheorem##1##2{\item
       [\hskip\labelsep \theorem@headerfont ##2\ ##1]}%
\def\@opargbegintheorem##1##2##3{%
  \item[\hskip\labelsep \theorem@headerfont ##2\ ##1\ (##3)]}}
\endgroup
%</thc>
%    \end{macrocode}
% \end{macro}
%
%
% \subsubsection{The \texttt{marginbreak} style}
%
% This style option is the one used most often at Mainz.  It is saved
% in the file |thmb.sty|.
%    \begin{macrocode}
%<*thmb>
\begingroup \makeatletter
\@ifundefined{theorem@style}{% \iffalse meta-comment
%
% Copyright (C) 1993-2024
%
% The LaTeX Project and any individual authors listed elsewhere
% in this file.
%
% This file is part of the Standard LaTeX `Tools Bundle'.
% -------------------------------------------------------
%
% It may be distributed and/or modified under the
% conditions of the LaTeX Project Public License, either version 1.3c
% of this license or (at your option) any later version.
% The latest version of this license is in
%    https://www.latex-project.org/lppl.txt
% and version 1.3c or later is part of all distributions of LaTeX
% version 2005/12/01 or later.
%
% The list of all files belonging to the LaTeX `Tools Bundle' is
% given in the file `manifest.txt'.
%
% \fi
%
%%
%
%
\def\FMithmInfo{2023/07/05 v2.2c Theorem extension package (FMi)}
%
% \ProvidesFile{theorem.dtx}[\FMithmInfo]
%\iffalse   % this is a METACOMMENT !
%%
%% Package `theorem' to use with LaTeX2e
%% Copyright (C) 1989-1995 Frank Mittelbach, all rights reserved.
%<+package>\@ifundefined{theorem@style}{}{\endinput}
%<+package>\NeedsTeXFormat{LaTeX2e}[1995/06/01]
%<+package>\ProvidesPackage{theorem}
%<+package>         [\FMithmInfo]
%
% \fi
%
% \changes{v1.0}{87/05/07}{First draft.}
% \changes{v1.0a}{87/05/07}{`thistheoremstyle' added.}
% \changes{v1.1a}{87/07/29}{New concept: redefine the internal
%                          \protect\LaTeX{} commands `@xnthm' etc.}
% \changes{v1.1b}{87/07/31}{`theoremfont' used to specify font.}
% \changes{v1.2a}{88/01/08}{`@endtheorem' now globally defined.}
% \changes{v1.2b}{88/01/11}{`newtheorem' only allowed in the preamble.}
% \changes{v1.3a}{88/02/14}{`theoremfont' now a toks register.}
%
% \changes{v2.0a}{88/11/16}{New implementation and new documentation.}
% \changes{v2.0a}{88/11/16}{No layout changes but internal modifications
%                           old substyles won't work.}
% \changes{v2.0b}{89/01/30}{`global' used for all definitions.}
% \changes{v2.0c}{89/06/15}{Documentation translated to English.
%                            Thanks to LLL and Nina}
% \changes{v2.1a}{90/02/15}{Support for new font selection.}
% \changes{v2.2a}{94/02/03}{Cleanup for LaTeX2e}
%
% \DoNotIndex{\@endparenv,\@ifdefinable,\@ifnextchar,\@ifundefined}
% \DoNotIndex{\@namedef,\@nameuse,\@spaces,\@tempa,\@warning}
% \DoNotIndex{\ }
% \DoNotIndex{\begingroup,\bfseries}
% \DoNotIndex{\csname}
% \DoNotIndex{\def,\do,\docdate}
% \DoNotIndex{\endcsname,\endgroup,\endinput,\endtrivlist,\expandafter}
% \DoNotIndex{\filedate,\fileversion}
% \DoNotIndex{\gdef,\global}
% \DoNotIndex{\hbox,\hskip}
% \DoNotIndex{\ignorespaces,\input,\itshape,\item}
% \DoNotIndex{\labelsep,\let,\llap}
% \DoNotIndex{\makeatletter}
% \DoNotIndex{\noexpand}
% \DoNotIndex{\refstepcounter,\relax,\rlap}
% \DoNotIndex{\setlength,\slshape,\space,\string,\strut}
% \DoNotIndex{\the,\trivlist,\typeout}
% \DoNotIndex{\vbox}
% \DoNotIndex{\xdef}
%
% \def\deflabel#1{\ttfamily #1\hfill}
% \def\deflist#1{\begin{list}{}{\settowidth\labelwidth{\ttfamily #1}%
%                               \setlength\leftmargin\labelwidth
%                               \addtolength\leftmargin\labelsep
%                               \let\makelabel\deflabel}}
% \def\enddeflist{\end{list}}
% \def\envfont{\normalfont\ttfamily}
%
% \MakeShortVerb{\|}
%
% \newcommand{\TUB}{{\sl TUGboat\/}}
%
% \GetFileInfo{theorem.dtx}
% \title{An Extension of the
%        \LaTeX{} theorem environment\thanks{This file
%        has version number \fileversion, last
%        revised \filedate.}}
% \author{Frank Mittelbach}
%
% \MaintainedByLaTeXTeam{tools}
% \maketitle
%
% \begin{abstract}
%    The macros described in this paper yield an extension of the
%    \LaTeX{} theorem mechanism. It is designed to satisfy the
%    different requirements of various journals. Thus, the
%    layout of the ``theorems'' can be manipulated by determining a
%    ``style''. This article describes not only the use, but
%    also the definition, of the necessary macros.
% \end{abstract}
%
% \section*{Preface to version 2.2}
%
% For \LaTeXe{} this package did not need any fundamental changes. I
% only modified the messages generated so that theorem layout styles
% will show up with the |\listfiles| command and cleaned the section
% on the New Font Selection Scheme since this is now included in
% \LaTeX.
%
% \section*{Preface to version 2.1}
%
% This version is identical to 2.0g described in \TUB\ 10\#3 except
% for some internal defaults which are now set depending on the used
% font selection scheme.
%
% This was done to avoid unpleasant surprises if the new font selection
% scheme is in force. For further details see
% section~\ref{sec:newfonts} and \cite{bk:GMS94}.
%
% \section{Introduction}
%
% For our purposes here, ``theorems'' are labelled enunciations,
% often set off from the main text by extra space and a font change.
% Theorems, corollaries, conjectures, definitions, and remarks
% are all instances of ``theorems''.  The ``header'' of these structures
% is composed of a label (such as \textsc{Theorem} or \textsc{Remark})
% and a number which serializes an item in the sequence of items
% with the same label.
%
% Shortly after the introduction of \LaTeX{} at the Fachbereich
% Mathematik in Mainz, the desire to manipulate the layout of
% ``theorems'' arose.  In Mainz, the following two conventions came
% into general use:
% \begin{enumerate}
%   \item The number of the theorem is shown in the margin.
%   \item There is a line break at
%         the end of the theorem header.
% \end{enumerate}
%
% Additionally, some journals require different formats which depend on
% the ``sort of theorem'': e.g.\ often remarks and definitions are set
% in |\upshape|, while |\itshape| is employed for main theorems.
%
% Confronted with these requirements, a theorem environment was
% developed in Mainz which allows separate determination of the
% layout of the ``theorems sets'', comparable to |\pagestyle|.
%
%
% \section{The user interface}
% \subsection{Defining new theorem sets}
%
% \DescribeMacro\newtheorem
% As in the original \LaTeX{} version, the command |\newtheorem|
% defines a new ``theorem set'' or ``theorem-like structure''.
% Two required arguments name the new environment and give the text
% to be typeset with each instance of the new ``set'', while
% an optional argument determines how
% the ``set'' is enumerated:
% \begin{description}
%    \item[\ttfamily \bslash newtheorem\{foo\}\{bar\}]
%       The theorem set {\envfont foo} (whose name is \texttt{bar})
%       uses its own counter.
%    \item[\ttfamily \bslash newtheorem\{foo2\}{[foo]}\{bar2\}]
%       The theorem set {\envfont foo2} (printed name \texttt{bar2})
%       uses the same counter as the theorem set \texttt{foo}.
%    \item[\ttfamily \bslash newtheorem\{foo3\}\{bar3\}{[section]}]
%       The theorem set {\envfont foo3} (printed name \texttt{bar3}) is
%       enumerated within the counter \texttt{section}, i.e.\ with every
%       new |\section| the enumeration begins again with 1, and
%       the enumeration is composed from the section-number and the
%       theorem counter itself.
% \end{description}
%
% \DescribeMacro\theoremstyle
% Additionally, the command |\theoremstyle| can define the layout
% of various, or all, theorem sets.  It should be noted that any
% theorem set defined by |\newtheorem| is typeset in the
% |\theoremstyle| that is current at the time of the definition.
% Thus, the following
% \begin{verbatim}
%    \theoremstyle{break}        \newtheorem{Cor}{Corollary}
%    \theoremstyle{plain}        \newtheorem{Exa}{Example}[section]
%\end{verbatim}                             ^^A No blank at line start
% leads to the result that the set {\envfont Cor} is formatted in the
% style \texttt{break}, while the set {\envfont Exa} and all the
% following ones are formatted in the style \texttt{plain}, unless
% another |\theoremstyle| follows.  Since the definitions installed by
% |\newtheorem| are global, one also can limit |\theoremstyle| locally
% by grouping braces.
%
% \DescribeMacro\theorembodyfont
% The choice of the font for the theorem body is completely
% independent of the chosen |\theoremstyle|; this has proven to
% be very advantageous.  For example,
% \begin{verbatim}
%   {\theorembodyfont{\upshape}          \newtheorem{Rem}{Remark}}
%\end{verbatim}
% defines a theorem set {\envfont Rem}, which will be set in
% |\upshape| in the current layout (which in our example is
% \texttt{plain}).  As with |\theoremstyle|, the |\theorembodyfont|
% chosen is that current at the time of |\newtheorem|.  If
% |\theorembodyfont| is not specified or one defines
% |\theorembodyfont{}|, then the font used will be that defined by the
% |\theoremstyle|.
%
% \DescribeMacro\theoremheaderfont It is also possible to customize
% the font used for the theorem headers.  This is, however, a global
% declaration, and therefore there should be at most one
% |\theoremheaderfont| declaration in the preamble.\footnote{If it is
% actually necessary to have different header fonts, one has to define
% new theorem styles (substituting the desired font) or specify the
% information directly in the \texttt{\bslash newtheorem} declaration
% (the unclean variant).}
%
% \DescribeMacro\theorempreskipamount
% \DescribeMacro\theorempostskipamount Two additional parameters
% affect the vertical space around the theorem environments:
% |\theorempreskipamount| and |\theorempostskipamount| define,
% respectively, the spacing before and after such an environment.
% These parameters apply for all theorem sets and can be manipulated
% with the ordinary length macros.  They are rubber lengths
% (`\textsf{skips}'), and therefore can contain \texttt{plus} and
% \texttt{minus} parts.
%
% Since the definition of theorem sets should---most sensibly---be
% placed in the preamble, we only allow installation there.
% It is therefore possible to release the memory used here after
% |\begin{document}|, in order to make room for other
% applications.
%
%
% \subsection{Existing theorem styles}
%
% The following theorem styles exist to date:
% \begin{deflist}{marginbreak}
%    \item[plain]
%       This theorem style emulates the original \LaTeX{} definition,
%       except that additionally the parameters
%       |\theorem...skipamount| are used.
%    \item[break]
%       In this style, the theorem header is followed by a line
%       break.
%    \item[marginbreak]
%       The theorem number is set in the margin, and there is a line
%       break as in \texttt{break}.
%    \item[changebreak]
%       Like \texttt{break}, but with header number and text
%       interchanged.
%    \item[change]
%       Header number and text are interchanged, without a line break.
%    \item[margin]
%       The number is set in the left margin, without a line break.
% \end{deflist}
% All styles (except \texttt{plain}) select |\slshape| as the default
% |\theorembodyfont|.
%
%
% \subsection{Examples}
%
% Given the above theorem sets {\envfont Cor}, {\envfont Exa}
% and {\envfont Rem},
% suppose that the preamble also contains the declarations:
% \begin{verbatim}
%    \theoremstyle{marginbreak}   \newtheorem{Lem}[Cor]{Lemma}
%    \theoremstyle{change}
%    \theorembodyfont{\itshape}        \newtheorem{Def}[Cor]{Definition}
%
%    \theoremheaderfont{\scshape}
%\end{verbatim}
% Then the following are some typical examples of the typeset output
% resulting from their use.
%
% \begin{Cor}
%    This is a sentence typeset in the theorem environment {\envfont
%    Cor}.
% \end{Cor}
%
% \begin{Exa}
%    This is a sentence typeset in the theorem environment {\envfont
%    Exa}.
% \end{Exa}
%
% \begin{Rem}
%    This is a sentence typeset in the theorem environment {\envfont
%    Rem}.
% \end{Rem}
%
% \begin{Lem}[Ben User]
%    This is a sentence typeset in the theorem environment {\envfont
%    Lem}.
% \end{Lem}
%
% \begin{Def}[Very impressive Definition]
%    This is a sentence typeset in the theorem environment {\envfont
%    Def}.
% \end{Def}
%
% The last two examples show the effect of the optional argument to
% a theorem environment (it is the text typeset in parentheses).
%
%
% \section{Special Considerations}
% \label{sec:newfonts}
%
% Theoremheader and body are implemented as a unit. This means that
% the |\theoremheaderfont|  will inherit characteristics of the
% |\theorembodyfont| in \LaTeXe. Thus, if for example
% |\theorembodyfont| is |\itshape| and |\theoremheaderfont|
% is |\bfseries| the font selected for the header will have the
% characteristics `bold extended italic'. If this is not desired one
% should set the |\theoremheaderfont| to something like
% \begin{verbatim}
%      \theoremheaderfont{\normalfont\bfseries}
%\end{verbatim}
% i.e.\ supplying all necessary font information explicitly.
%
%
%
% \section{Acknowledgements}
%
% The publication of this set of macros was only possible with the
% help of Christina Busse (translating the manuscript into English),
% Joachim Pense (playing the r\^ole of typist), Chris Rowley
% (looking everything over) and many others providing useful
% suggestions.
%
% \MaybeStop{%
%    \begin{thebibliography}{9}
%       \bibitem{bk:GMS94} \textsc{M.~Goossens}, \textsc{F.~Mittelbach}
%         and \textsc{A.~Samarin}.
%         \newblock The \LaTeX{} Companion.
%         \newblock
%         Addison-Wesley, Reading, Massachusetts, 1994.
%       \bibitem{latex:source} \textsc{Lamport, Leslie.} \newblock
%                               \texttt{latex.tex}, version 2.09, date
%                               Feb. 1990.
%    \end{thebibliography}
%    \PrintIndex
%    \PrintChanges}
%
%
% \section{The documentation driver file}
%
% The next bit of code contains the documentation driver file for
% \TeX{}, i.e., the file that will produce the documentation you are
% currently reading. It will be extracted from this file by the
% \texttt{docstrip} program. Since it is the first code in the file
% one can alternatively process this file directly with \LaTeXe{} to
% obtain the documentation.
%
%    \begin{macrocode}
%<*driver>
\documentclass{ltxdoc}

\usepackage{theorem}

%<+driver>% The next few lines define theorem sets which are used
%<+driver>% in the example section of the documentation.

\theoremstyle{break}       \newtheorem{Cor}{Corollary}
\theoremstyle{plain}       \newtheorem{Exa}{Example}[section]
{\theorembodyfont{\upshape}\newtheorem{Rem}{Remark}}
\theoremstyle{marginbreak} \newtheorem{Lem}[Cor]{Lemma}
\theoremstyle{change}
\theorembodyfont{\itshape} \newtheorem{Def}[Cor]{Definition}

\theoremheaderfont{\scshape}

\RecordChanges

\begin{document}
   \DocInput{theorem.dtx}
\end{document}
%</driver>
%    \end{macrocode}
%
% \section{Definition of the Macros}
%
% If the file has been loaded before, we abort immediately. If not the
% package announces itself (this is actually done at the very top of
% the file---the way it is done isn't good style so don't copy it).
% \changes{v2.0e}{89/07/19}{Spaces removed from `typeout'.}
% \changes{v2.2b}{95/11/19}{Announce moved to top of file}
%    \begin{macrocode}
%<*package>
%\@ifundefined{theorem@style}{}{\endinput}
%\def\FMithmInfo{1995/11/19 v2.2b Theorem extension package (FMi)}
%\ProvidesPackage{theorem}[\FMithmInfo]
%    \end{macrocode}
%
%
% \subsection{Definition of theorem styles and fonts}
%
% All the definitions in this file are done globally to allow
% inputting this file inside a group.
%
% \begin{macro}{\theoremstyle}
%    Before a theorem style can be installed, the chosen style must be
%    known. For that reason, we must test to see that
%    |\th@|\meta{style} is known or, more precisely, that it is
%    different from |\relax|. If the style is not known then
%    |\th@plain| is used.
%    \begin{macrocode}
\gdef\theoremstyle#1{%
   \@ifundefined{th@#1}{\@warning
          {Unknown theoremstyle `#1'. Using `plain'}%
          \theorem@style{plain}}%
%    \end{macrocode}
%    We save the theorem style to be used in the token register
%    |\theorem@style|.
%    \begin{macrocode}
      {\theorem@style{#1}}%
%    \end{macrocode}
%    Now we ``evaluate'' the theorem style: this means, we call the
%    macro |\th@|\meta{style} which will activate the relevant
%    definitions which are contained in a separate file.  This is done
%    in a group to suppress changes to the current font.  This could
%    otherwise pose problems together with the new font selection
%    scheme\footnote{When I printed the original article using the new
%    font selection scheme I ended with a document with slanted
%    typefaces (text headings and all) simply because one of the
%    theorem styles used \texttt{\bslash sl} at toplevel.} if the
%    |\th@|\meta{style} is evaluated a second time.
% \changes{v2.1a}{90/02/15}{Group added around `th@..}
% \changes{v2.1a}{90/02/15}{Reseting of `..@begintheorem removed.}
%    \begin{macrocode}
      \begingroup
        \csname th@\the\theorem@style \endcsname
      \endgroup}
%    \end{macrocode}
% \end{macro}
%
%
% \begin{macro}{\@begintheorem}
% \changes{v2.1a}{90/02/15}{Relaxed at toplevel.}
% \begin{macro}{\@opargbegintheorem}
% \changes{v2.1a}{90/02/15}{Relaxed at toplevel.}
%    We reset |\@begintheorem| and |\@opargbegintheorem| to
%    |\relax| since these commands are no longer necessary at
%    toplevel. This will save a few tokens.
%    \begin{macrocode}
\global\let\@begintheorem\relax
\global\let\@opargbegintheorem\relax
%    \end{macrocode}
% \end{macro}
% \end{macro}
%
% \begin{macro}{\theorem@style}
%    Obviously the \textsf{token} register used above has to be
%    allocated. To assure the utmost compatibility with the
%    original \LaTeX{} definition, we set the default theorem style to
%    \texttt{plain}, which implements the usual \LaTeX{} convention.
% \changes{v2.0f}{89/07/20}{Unnecessary `global' removed.}
%    \begin{macrocode}
\newtoks\theorem@style
\global\theorem@style{plain}
%    \end{macrocode}
% \end{macro}
%
% \begin{macro}{\theorembodyfont}
% \begin{macro}{\theorem@bodyfont}
%    If the |\theorembodyfont| is set by the user then it should not
%    interact with the default font set in the theorem style. When the
%    new font selection is in force this may happen if, for example,
%    the default is |\itshape| and the new |\theorembodyfont| is
%    |\sffamily|. So we add a |\reset@font| command in front of the
%    user definition.
% \changes{v2.0f}{89/07/20}{Unnecessary `global' removed.}
% \changes{v2.0g}{89/09/19}{`theoremfont to `theorembodyfont.}
% \changes{v2.1c}{92/06/22}{`reset@font added and logic changed.}
% \changes{v2.1d}{92/10/28}{Reset to empty if arg is empty}
%    \begin{macrocode}
\gdef\theorembodyfont#1{%
%    \end{macrocode}
%    We check if the argument supplied is empty and if so put nothing
%    into the |\theorem@bodyfont| token register to allow for
%    |\theorembodyfont{}| as a mean of using the default of the
%    current |\theoremstyle|.
%    \begin{macrocode}
   \def\@tempa{#1}%
   \ifx\@tempa\@empty
    \theorem@bodyfont{}%
   \else
    \theorem@bodyfont{\reset@font#1}%
   \fi
   }
\newtoks\theorem@bodyfont
\global\theorem@bodyfont{}
%    \end{macrocode}
% \end{macro}
% \end{macro}
%
%
% \begin{macro}{\theoremheaderfont}
% \changes{v2.0g}{89/09/19}{Macro added.}
%    The font for the theorem headers is handled differently because
%    this definition applies to all theorem styles.
%    \begin{macrocode}
\gdef\theoremheaderfont#1{\gdef\theorem@headerfont{#1}%
%    \end{macrocode}
%    After using the macro once it is redefined to produce an error
%    message.
%    \begin{macrocode}
       \gdef\theoremheaderfont##1{%
        \typeout{\string\theoremheaderfont\space should be used
                 only once.}}}
%    \end{macrocode}
% \end{macro}
%
% \begin{macro}{\theorem@headerfont}
% \changes{v2.0g}{89/09/19}{Macro added.}
%    To set the |\theorem@headerfont| default we first test if
%    the new fontselection scheme is in force.
% \changes{v2.1a}{90/02/15}{Support for new font selection.}
%    \begin{macrocode}
\ifx\upshape\undefined
%    \end{macrocode}
%    If not we define it to expand into |\bfseries|. We don't use
%    |\let| just in case a following style option redefines this
%    macro.
%    \begin{macrocode}
\gdef\theorem@headerfont{\bfseries}
%    \end{macrocode}
%    Otherwise we reset the current shape before calling |\bfseries|.
%    \begin{macrocode}
\else \gdef\theorem@headerfont{\normalfont\bfseries}\fi
%    \end{macrocode}
% \end{macro}
%
% \begin{macro}{\th@plain}
% \begin{macro}{\th@break}
% \begin{macro}{\th@marginbreak}
% \begin{macro}{\th@changebreak}
% \begin{macro}{\th@change}
% \begin{macro}{\th@margin}
%    The different styles are defined in macros such as
%    |\th@plain|.  Since memory space is precious in
%    ``non-Big-versions'', we have to avoid offering too many unused
%    definitions.  Therefore we define these styles in separate files
%    that can be loaded on demand.  Thus the commands themselves only
%    load these files. We use |\@input@|, a \LaTeXe{} internal command
%    that ensures that the file will be listed with |\listfiles|.
% \changes{v2.2a}{94/02/02}{Use `@input@ to load theorem layout files}
%    \begin{macrocode}
\gdef\th@plain{\@input@{thp.sty}}
\gdef\th@break{\@input@{thb.sty}}
\gdef\th@marginbreak{\@input@{thmb.sty}}
\gdef\th@changebreak{\@input@{thcb.sty}}
\gdef\th@change{\@input@{thc.sty}}
\gdef\th@margin{\@input@{thm.sty}}
%    \end{macrocode}
%    This list will be expanded when new styles become
%    available.  For testing,  just append new theorem substyles
%    as document options.
% \end{macro}
% \end{macro}
% \end{macro}
% \end{macro}
% \end{macro}
% \end{macro}
%
%
%
% \subsection{Definition of a new theorem set}
%
% As already pointed out, a new theorem environment can be defined in
% three different ways:
% \begin{verbatim}
%    \newtheorem{Lem}{Lemma}
%    \newtheorem{Lem}{Lemma}[section]
%    \newtheorem{Lem}[Theorem]{Lemma}
%\end{verbatim}
% The function of the macro |\newtheorem| is to recognize these cases
% and then to branch into one of the three macros |\@ynthm|, |\@xnthm|
% or |\@othm|.  This mechanism is adopted unchanged from
% \cite{latex:source}; the essential point here is that, for example,
% in the second case, the arguments \texttt{Lem}, \texttt{Lemma} and
% \texttt{section} are passed over to the macro |\@xnthm|.
%
% We inspect this case first because the others present fewer problems,
% and thus are easily derived from this one.
%
% \begin{macro}{\@xnthm}
%    For our example arguments, the macro |\@xnthm| must fulfill
%    the following:
%    \begin{itemize}
%       \item
%          Define a new \LaTeX-counter `Lem'
%       \item
%          reset this counter within a |\section|
%       \item
%          define the macro |\theLem|
%       \item
%          define the environment macros |\Lem| and
%          |\endLem| using the current
%          |\theoremstyle| and |\theorem@bodyfont|.
%    \end{itemize}
%    Obviously, all this should happen only if the first argument
%    of |\@xnthm| (i.e.\ \texttt{Lem} in our example) is chosen so
%    as not to conflict with any previously defined commands or
%    environments.  This test is performed by the \LaTeX{} macro
%    |\@ifdefinable|.
%    \begin{macrocode}
\gdef\@xnthm#1#2[#3]{\expandafter\@ifdefinable\csname #1\endcsname
%    \end{macrocode}
%    Therefore, the first argument of |\@ifdefinable| is the
%    expansion (in the example, |\Lem|) of
%    |\csname||#1\endcsname|.  The second argument is
%    executed only if the test has been completed successfully.
%    \begin{macrocode}
   {%
%    \end{macrocode}
%    Now we define the new counter. The names of the \LaTeX{}
%    macros employed should speak for themselves:
% \changes{v2.2b}{1995/10/07}{Use \cs{@newctr}}
% \changes{v2.2c}{1995/11/23}{Use \cs{@newctr} properly.}
%    \begin{macrocode}
    \@definecounter{#1}\@newctr{#1}[#3]%
%    \end{macrocode}
%    Using |\@newctr| will give a proper error message if the counter
%    in |#3| is not defined.
%    In defining `|\theLem|' we must generate the desired macro
%    name by use of |\expandafter| and |\csname|.
%    \begin{macrocode}
    \expandafter\xdef\csname the#1\endcsname
%    \end{macrocode}
%    An |\xdef| is used in order to make the definition global,
%    and to ensure that it contains the replacement texts of
%    |\@thmcountersep| and |\@thmcounter|.\footnote{These
%    two macros can be defined by the document style.  Their default
%    values produce a `\texttt{.}' as separation and an arabic
%    representation of the number.} However, not everything should be
%    expanded. For example, it saves space to use |\thesection|
%    instead of its---at times---lengthy expansion.
%    \begin{macrocode}
      {\expandafter \noexpand \csname the#3\endcsname
       \@thmcountersep \@thmcounter{#1}}%
%    \end{macrocode}
%    Thus with the defaults of \LaTeX{}, |\theLem| would be
%    replaced by the command sequence |\thesection.\arabic{Lem}|.
%
%    We will now look at the definition of the macro which is executed
%    at the beginning of the actual environment (in our example this
%    macro is |\Lem|). It should be noted that we use an
%    ``|\expandafter| trick'' to expand only certain parts of the
%    replacement text at the time of the definition.
%    \begin{macrocode}
    \def\@tempa{\global\@namedef{#1}}%
    \expandafter \@tempa \expandafter{%
%    \end{macrocode}
%    First, the macro that contains the current definitions of
%    |\@begintheorem| and |\@opargtheorem| should be called
%    up.  The name of this macro---as is already known---has the form
%    |\th@|\meta{theorem\ style}; therefore, it must be called by
%    \begin{macrocode}
      \csname th@\the \theorem@style
%    \end{macrocode}
%    In addition the default theorem font should be changeable, i.e.\
%    we have to insert the contents of |\theorem@bodyfont|. For
%    that reason, we expand even further, beyond |\endcsname|,
%    and thus insert the contents of the \textsf{token} register
%    |\theorem@bodyfont| in the replacement text.
%    \begin{macrocode}
            \expandafter \endcsname \the \theorem@bodyfont
%    \end{macrocode}
%    Now it is time to call the macro |\@thm| which takes over
%    the further processing.  It has two arguments: the current
%    counter name (in our example, \texttt{Lem}), and the text of the
%    label (in our example, \texttt{Lemma}).
%    \begin{macrocode}
     \@thm{#1}{#2}}%
%    \end{macrocode}
%    With this, the `sub-definition' is complete.  The macro
%    |\@endtheorem| ends a theorem environment and is, so far, nothing
%    but an |\endtrivlist|.  (Hence it is defined globally, and not
%    within the theorem styles.\footnote{This has to be changed as
%    soon as theorem styles that change \texttt{\bslash @endtheorem}
%    exist.  In such a case, all existing styles must be changed as
%    well since they will have to reset the macro.}) Therefore, we can
%    set it equivalent to the macro that ends the theorem set (in our
%    example, |\endLem|).  However, if some day theorem styles exist
%    that do change |\@endtheorem|, we would have to use the
%    commented-out line instead.
%    \begin{macrocode}
    \global \expandafter \let \csname end#1\endcsname \@endtheorem
%   \global\@namedef{end#1}{\@endtheorem}%
%    \end{macrocode}
%    With these commands all the required definitions are employed,
%    unless the test |\@ifdefinable| has failed.  Therefore, we
%    end the second argument of this macro and with it the definition
%    of |\@xnthm|.
%    \begin{macrocode}
   }}
%    \end{macrocode}
% \end{macro}
%
% \begin{macro}{\@ynthm}
%    The definition of |\@ynthm| is completely analogous. In
%    this case the new counter that is defined is not reset within
%    another counter; thus the definition of |\the...|
%    is simplified:
%    \begin{macrocode}
\gdef\@ynthm#1#2{\expandafter\@ifdefinable\csname #1\endcsname
   {\@definecounter{#1}%
    \expandafter\xdef\csname the#1\endcsname{\@thmcounter{#1}}%
%    \end{macrocode}
%    The rest of the definition corresponds literally to that of
%    |\@xnthm|:
%    \begin{macrocode}
    \def\@tempa{\global\@namedef{#1}}\expandafter \@tempa
     \expandafter{\csname th@\the \theorem@style \expandafter
     \endcsname \the\theorem@bodyfont \@thm{#1}{#2}}%
    \global \expandafter \let \csname end#1\endcsname \@endtheorem}}
%    \end{macrocode}
% \end{macro}
%
% \begin{macro}{\@othm}
%    The definition of |\@othm| does not contain anything new.
%    \begin{macrocode}
\gdef\@othm#1[#2]#3{%
%    \end{macrocode}
% \changes{v2.2b}{95/10/07}{Build in check for counter defined}
%    We do not define a new counter but instead use one that has
%    already been defined.  Thus the only definition we need is that
%    of this pseudo-counter (i.e.\ |\the|\meta{env.\ name}).
%    First we check if |#2| corresponds to a known counter name.
%    \begin{macrocode}
  \expandafter\ifx\csname c@#2\endcsname\relax
   \@nocounterr{#2}%
  \else
   \expandafter\@ifdefinable\csname #1\endcsname
   {\expandafter \xdef \csname the#1\endcsname
     {\expandafter \noexpand \csname the#2\endcsname}%
%    \end{macrocode}
%    All other parts of the definition can be adopted from
%    |\@xnthm|. We have to remember, though, that in this case
%    the name of the current counter and the theorem label have moved
%    to the second and third arguments.
%    \begin{macrocode}
    \def\@tempa{\global\@namedef{#1}}\expandafter \@tempa
     \expandafter{\csname th@\the \theorem@style \expandafter
     \endcsname \the\theorem@bodyfont \@thm{#2}{#3}}%
    \global \expandafter \let \csname end#1\endcsname \@endtheorem}%
  \fi}
%    \end{macrocode}
% \end{macro}
%
% \subsection{Macros that are employed in a {\envfont theorem}
%    environment}
%
% \begin{macro}{\@thm}
%    The macro |\@thm| has to increase the current counter.  Then,
%    depending on whether the environment has (or does not have) an
%    optional argument, it has to branch into either |\@begintheorem|
%    or |\@opargtheorem|.
%    \begin{macrocode}
\gdef\@thm#1#2{\refstepcounter{#1}%
%    \end{macrocode}
%    Now we start a {\envfont trivlist} environment, and give
%    |\@topsep| and |\@topsepadd| the values of the \textsf{skip}
%    registers |\theorempreskipamount| and |\theorempostskipamount|.
%    The value in |\@topsep| is the vertical space that is inserted by
%    the first (and only) |\item| in our |\trivlist| whilst
%    |\@topsepadd| is inserted by |\@endparenv| at the end of that
%    {\envfont trivlist} environment.  By using these registers, we
%    obtain the desired space around a {\envfont theorem} environment.
%    \begin{macrocode}
   \trivlist
   \@topsep \theorempreskipamount               % used by first \item
   \@topsepadd \theorempostskipamount           % used by \@endparenv
%    \end{macrocode}
%    Now we have to test whether an optional argument has been given.
% \changes{v2.0d}{89/07/18}{Forgotten percent added.}
%    \begin{macrocode}
   \@ifnextchar [%
%    \end{macrocode}
%    If there is an optional argument, we will call |\@ythm|, and
%    move the arguments read back into the input stream.
%    \begin{macrocode}
   {\@ythm{#1}{#2}}%
%    \end{macrocode}
%    If not, we call |\@begintheorem|. Its first argument is the
%    name of the theorem set (hence the second argument of
%    |\@thm|). Its second argument is the macro that produces the
%    current number.
%    \begin{macrocode}
   {\@begintheorem{#2}{\csname the#1\endcsname}\ignorespaces}}
%    \end{macrocode}
% \end{macro}
%
% \begin{macro}{\@xthm}
% \begin{macro}{\@ythm}
%    Both these macros were originally called by |\@thm|.  We do
%    not need |\@xthm| anymore, hence we reset it to
%    |\relax|. The definition of |\@ythm| has not changed at
%    all from its definition in \LaTeX. In order to make the macros
%    easier to understand, we will
%    nevertheless present it (commented out).
%    \begin{macrocode}
\global\let\@xthm\relax
% \def\@ythm#1#2[#3]{\@opargbegintheorem{#2}{\csname
%           the#1\endcsname}{#3}\ignorespaces}
%</package>
%    \end{macrocode}
%    The primitive |\ignorespaces| in |\@ythm| and
%    |\@thm| is needed to remove the spaces between the
%    |\begin{...}| and the actual text.
% \end{macro}
% \end{macro}
%
%
% \subsection{Definition of the theorem substyles}
%
% As already pointed out, the theorem substyles, defined below, are
% only loaded when necessary.  Note that all these substyles, except
% \texttt{plain}, have |\slshape| as the default body font.
%
% \subsubsection{The \texttt{plain} style}
%
% \label{dok:th@plain} As the following macros use |@|, we have
% to locally set the |\catcode| of this symbol to ``letter''.
% This happens within a group, so that we do not have to worry about
% which |\catcode| that symbol had before.
%    \begin{macrocode}
%<*thp>
\begingroup \makeatletter
%    \end{macrocode}
% Since we are now within a group, we must make all definitions
% globally.  First we make sure that \texttt{theorem.sty} is loaded.
% This will allow us to use this file as a document style option
% without having to call \texttt{theorem} itself as an option.  At the
% same time, we assure that at least version 2 is loaded, since
% |\theorem@style| was not defined in earlier versions.
%    \begin{macrocode}
\@ifundefined{theorem@style}{\input{theorem.sty}}{}
\ProvidesFile{thp.sty}
             [\FMithmInfo]
%    \end{macrocode}
%
%
% \begin{macro}{\th@plain}
% \begin{macro}{\@begintheorem}
% \begin{macro}{\@opargbegintheorem}
%    |\theoremstyle{plain}| corresponds to the original
%    definition, except that the distances to the surrounding text are
%    determined by the parameters |\theorempreskipamount| and
%    |\theorempostskipamount|.
%    First we set the default body font.
%    \begin{macrocode}
\gdef\th@plain{\normalfont\itshape
%    \end{macrocode}
%    Then we define |\@begintheorem| and
%    |\@opargbegintheorem|. These two macros define how the
%    header of a theorem is typeset.  |\@opargbegintheorem| will
%    be called if a \textsf{theorem} environment with an optional
%    argument is encountered; otherwise, the header is constructed by
%    calling |\@begintheorem|.  If one of these macros is
%    executed, we are within a \textsf{trivlist} environment started by
%    |\@thm|.  So the theorem header is produced with an
%    |\item| command.
%
%    Instead of specifying the header font directly, all standard
%    theorem styles use the |\theorem@headerfont| macro to allow
%    customization.  The extra space (|\labelsep|) is necessary
%    because of problems in the \textsf{trivlist} environment.
%    \begin{macrocode}
  \def\@begintheorem##1##2{%
        \item[\hskip\labelsep \theorem@headerfont ##1\ ##2]}%
%    \end{macrocode}
%    The definition of |\@opargbegintheorem| is completely
%    analogous. The only difference is the fact that there exists a
%    third argument (which is the optional parameter of the
%    environment and contains additional information about the
%    theorem).  Customarily we enclose it in parentheses.
%    \begin{macrocode}
\def\@opargbegintheorem##1##2##3{%
   \item[\hskip\labelsep \theorem@headerfont ##1\ ##2\ (##3)]}}
%    \end{macrocode}
% \end{macro}
% \end{macro}
% \end{macro}
%
% \noindent
% We conclude with an |\endgroup| to restore the
% |\catcode| of |@|.
%    \begin{macrocode}
\endgroup
%</thp>
%    \end{macrocode}
%
%
% \subsubsection{The \texttt{break} style}
%
% This style option is stored in the file |thb.sty|.
% For the next two lines see the documentation for |\th@plain| on
% page~\pageref{dok:th@plain}.
%    \begin{macrocode}
%<*thb>
\begingroup \makeatletter
\@ifundefined{theorem@style}{\input{theorem.sty}}{}
\ProvidesFile{thb.sty}
             [\FMithmInfo]
%    \end{macrocode}
%
% \begin{macro}{\th@break}
%    |\theoremstyle{break}| produces a line break after the name
%    of the theorem.  The font is |\slshape|.  Hence, we define
%    |\th@break| as follows:
%    \begin{macrocode}
\gdef\th@break{\normalfont\slshape
  \def\@begintheorem##1##2{\item[%
%    \end{macrocode}
%    We run into the following problem: it is not possible to create
%    the header with |\item[|\meta{title}|]| and then start
%    a new line by, for example, |\mbox{}\\|. Such a definition
%    will fail whenever a list environment follows immediately.  With
%    the above construction, the |\mbox{}| causes the switch
%    |@inlabel| (cf.\ definition of |\list| and
%    |\trivlist| in \cite{latex:source}) to be set to \texttt{false}
%    and so the following list will insert additional vertical
%    space (|\topskip|). This is quite annoying. Therefore, we
%    create the line break within the |\item|. In order to ensure
%    that the text will begin at the proper position in the following
%    line, we simply pretend that the label does not take any
%    room.\footnote{This will lead to problems whenever very high
%    symbols occurring in the line tower into the
%    heading. So, something else has to be done here sometime.}
%    \begin{macrocode}
       \rlap{\vbox{\hbox{\hskip \labelsep\theorem@headerfont ##1\ ##2}%
                   \hbox{\strut}}}]}%
%    \end{macrocode}
%    Again, the definition of |\@opargbegintheorem| is
%    completely analogous.
%    \begin{macrocode}
\def\@opargbegintheorem##1##2##3{%
  \item[\rlap{\vbox{\hbox{\hskip \labelsep \theorem@headerfont
                     ##1\ ##2\ (##3)}%
                    \hbox{\strut}}}]}}
\endgroup
%</thb>
%    \end{macrocode}
% \end{macro}
%
%
% \subsubsection{The \texttt{changebreak} style}
%
%    \begin{macrocode}
% This style option is stored in the file |thcb.sty|.
%    \begin{macrocode}
%<*thcb>
\begingroup \makeatletter
\@ifundefined{theorem@style}{\input{theorem.sty}}{}
\ProvidesFile{thcb.sty}
             [\FMithmInfo]
%    \end{macrocode}
%
% \begin{macro}{\th@changebreak}
%    The \texttt{change-break} theorem style is like \texttt{break}
%    but with interchange of theorem name and theorem number.  Thus we
%    define |\th@changebreak| as follows:
%    \begin{macrocode}
\gdef\th@changebreak{\normalfont\slshape
  \def\@begintheorem##1##2{\item
       [\rlap{\vbox{\hbox{\hskip\labelsep \theorem@headerfont ##2\ ##1}%
                    \hbox{\strut}}}]}%
\def\@opargbegintheorem##1##2##3{%
  \item[\rlap{\vbox{\hbox{\hskip\labelsep \theorem@headerfont
                     ##2\ ##1\ (##3)}%
                    \hbox{\strut}}}]}}
\endgroup
%</thcb>
%    \end{macrocode}
% \end{macro}
%
%
% \subsubsection{The \texttt{change} style}
%
% This style option is stored in the file |thc.sty|.
%    \begin{macrocode}
%<*thc>
\begingroup \makeatletter
\@ifundefined{theorem@style}{\input{theorem.sty}}{}
\ProvidesFile{thc.sty}
             [\FMithmInfo]
%    \end{macrocode}
%
% \begin{macro}{\th@change}
%    The \texttt{change} theorem style corresponds to the
%    \texttt{change break} style without a linebreak after the header.
%    To say it in another way, it's the same as the \texttt{plain}
%    style but with number and name interchanged and |\slshape| as the
%    default font.
%    \begin{macrocode}
\gdef\th@change{\normalfont\slshape
  \def\@begintheorem##1##2{\item
       [\hskip\labelsep \theorem@headerfont ##2\ ##1]}%
\def\@opargbegintheorem##1##2##3{%
  \item[\hskip\labelsep \theorem@headerfont ##2\ ##1\ (##3)]}}
\endgroup
%</thc>
%    \end{macrocode}
% \end{macro}
%
%
% \subsubsection{The \texttt{marginbreak} style}
%
% This style option is the one used most often at Mainz.  It is saved
% in the file |thmb.sty|.
%    \begin{macrocode}
%<*thmb>
\begingroup \makeatletter
\@ifundefined{theorem@style}{\input{theorem.sty}}{}
\ProvidesFile{thmb.sty}
             [\FMithmInfo]
%    \end{macrocode}
%
% \begin{macro}{\th@marginbreak}
%    The \texttt{margin break} style is nearly the same as the
%    \texttt{change break} style. The only difference is the placement
%    of the theorem number.  We use |\llap| to place it in the left
%    margin.
%
%    In this style |\labelsep| denotes the separation between the
%    number and the text.
%    \begin{macrocode}
\gdef\th@marginbreak{\normalfont\slshape
  \def\@begintheorem##1##2{\item
       [\rlap{\vbox{\theorem@headerfont
                    \hbox{\llap{##2}\hskip\labelsep ##1}%
                    \hbox{\strut}}}]}%
\def\@opargbegintheorem##1##2##3{%
  \item[\rlap{\vbox{\theorem@headerfont
                    \hbox{\llap{##2}\hskip\labelsep ##1\ (##3)}%
                    \hbox{\strut}}}]}}
\endgroup
%</thmb>
%    \end{macrocode}
% \changes{v2.1c}{90/02/17}{`endgroup outside macrocode env.}
% \end{macro}
%
%
%
% \subsubsection{The \texttt{margin} style}
%
% This style option is stored in the file |thm.sty|.
%    \begin{macrocode}
%<*thm>
\begingroup \makeatletter
\@ifundefined{theorem@style}{\input{theorem.sty}}{}
\ProvidesFile{thm.sty}
             [\FMithmInfo]
%    \end{macrocode}
%
% \begin{macro}{\th@margin}
%    Again this is only a variant of the theorem styles described
%    above without any new ideas.
%    \begin{macrocode}
\gdef\th@margin{\normalfont\slshape
  \def\@begintheorem##1##2{\item
       [\theorem@headerfont \llap{##2}\hskip\labelsep ##1]}%
\def\@opargbegintheorem##1##2##3{%
  \item[\theorem@headerfont \llap{##2}\hskip\labelsep ##1\ (##3)]}}
\endgroup
%</thm>
%    \end{macrocode}
% \end{macro}
%
%
%
%
% \subsection{Final Definitions}
%
% \begin{macro}{\theorempreskipamount}
% \begin{macro}{\theorempostskipamount}
%    The \textsf{skip} parameters that regulate the vertical empty space
%    before and after the {\envfont theorem} environment have
%    to be allocated as well.
% \changes{v2.0f}{89/07/20}{Unnecessary `global' removed.}
%    \begin{macrocode}
%<*package>
\newskip\theorempreskipamount
\newskip\theorempostskipamount
%    \end{macrocode}
%    Since we have used the same values for all theorem sets, we now can
%    assign them.
%    \begin{macrocode}
\global\setlength\theorempreskipamount{12pt plus 5pt minus 3pt}
\global\setlength\theorempostskipamount{8pt plus 3pt minus 1.5pt}
%    \end{macrocode}
% \end{macro}
% \end{macro}
%
%
% \begin{macro}{\@endtheorem}
%    The same holds for the macro |\@endtheorem|, which ends a
%    {\envfont theorem} environment. Since it is the same for all
%    theorem sets, it is removed from the macros |\th@|\meta{style}.
%    It simply ends the {\envfont trivlist} environment, which was
%    begun in |\@thm|.
%    \begin{macrocode}
\global\let\@endtheorem=\endtrivlist
%    \end{macrocode}
% \end{macro}
%
%
%
% \begin{macro}{\@preamblecmds}
%    All macros defined above are to be used only in the preamble.
%    Therefore, we insert them in |\@preamblecmds| which will disable
%    them at begin document. This is done by the internal \LaTeXe{}
%    command |\@onlypreamble|.
% \changes{v2.0g}{89/09/19}{`theoremheaderfont and `theorembodyfont
%    added.}
% \changes{v2.2a}{94/02/03}{Use \LaTeXe{} interface}
%    \begin{macrocode}
\@onlypreamble\@xnthm
\@onlypreamble\@ynthm
\@onlypreamble\@othm
\@onlypreamble\newtheorem
\@onlypreamble\theoremstyle
\@onlypreamble\theorembodyfont
\@onlypreamble\theoremheaderfont
%    \end{macrocode}
% \end{macro}
%
% Finally we declare the \texttt{plain} theorem style to be the default.
% \changes{v2.1b}{90/06/04}{Fixed missing `item by declaring default
%    style.}
%    \begin{macrocode}
\theoremstyle{plain}
%</package>
%    \end{macrocode}
%
% \Finale

\endinput
}{}
\ProvidesFile{thmb.sty}
             [\FMithmInfo]
%    \end{macrocode}
%
% \begin{macro}{\th@marginbreak}
%    The \texttt{margin break} style is nearly the same as the
%    \texttt{change break} style. The only difference is the placement
%    of the theorem number.  We use |\llap| to place it in the left
%    margin.
%
%    In this style |\labelsep| denotes the separation between the
%    number and the text.
%    \begin{macrocode}
\gdef\th@marginbreak{\normalfont\slshape
  \def\@begintheorem##1##2{\item
       [\rlap{\vbox{\theorem@headerfont
                    \hbox{\llap{##2}\hskip\labelsep ##1}%
                    \hbox{\strut}}}]}%
\def\@opargbegintheorem##1##2##3{%
  \item[\rlap{\vbox{\theorem@headerfont
                    \hbox{\llap{##2}\hskip\labelsep ##1\ (##3)}%
                    \hbox{\strut}}}]}}
\endgroup
%</thmb>
%    \end{macrocode}
% \changes{v2.1c}{90/02/17}{`endgroup outside macrocode env.}
% \end{macro}
%
%
%
% \subsubsection{The \texttt{margin} style}
%
% This style option is stored in the file |thm.sty|.
%    \begin{macrocode}
%<*thm>
\begingroup \makeatletter
\@ifundefined{theorem@style}{% \iffalse meta-comment
%
% Copyright (C) 1993-2024
%
% The LaTeX Project and any individual authors listed elsewhere
% in this file.
%
% This file is part of the Standard LaTeX `Tools Bundle'.
% -------------------------------------------------------
%
% It may be distributed and/or modified under the
% conditions of the LaTeX Project Public License, either version 1.3c
% of this license or (at your option) any later version.
% The latest version of this license is in
%    https://www.latex-project.org/lppl.txt
% and version 1.3c or later is part of all distributions of LaTeX
% version 2005/12/01 or later.
%
% The list of all files belonging to the LaTeX `Tools Bundle' is
% given in the file `manifest.txt'.
%
% \fi
%
%%
%
%
\def\FMithmInfo{2023/07/05 v2.2c Theorem extension package (FMi)}
%
% \ProvidesFile{theorem.dtx}[\FMithmInfo]
%\iffalse   % this is a METACOMMENT !
%%
%% Package `theorem' to use with LaTeX2e
%% Copyright (C) 1989-1995 Frank Mittelbach, all rights reserved.
%<+package>\@ifundefined{theorem@style}{}{\endinput}
%<+package>\NeedsTeXFormat{LaTeX2e}[1995/06/01]
%<+package>\ProvidesPackage{theorem}
%<+package>         [\FMithmInfo]
%
% \fi
%
% \changes{v1.0}{87/05/07}{First draft.}
% \changes{v1.0a}{87/05/07}{`thistheoremstyle' added.}
% \changes{v1.1a}{87/07/29}{New concept: redefine the internal
%                          \protect\LaTeX{} commands `@xnthm' etc.}
% \changes{v1.1b}{87/07/31}{`theoremfont' used to specify font.}
% \changes{v1.2a}{88/01/08}{`@endtheorem' now globally defined.}
% \changes{v1.2b}{88/01/11}{`newtheorem' only allowed in the preamble.}
% \changes{v1.3a}{88/02/14}{`theoremfont' now a toks register.}
%
% \changes{v2.0a}{88/11/16}{New implementation and new documentation.}
% \changes{v2.0a}{88/11/16}{No layout changes but internal modifications
%                           old substyles won't work.}
% \changes{v2.0b}{89/01/30}{`global' used for all definitions.}
% \changes{v2.0c}{89/06/15}{Documentation translated to English.
%                            Thanks to LLL and Nina}
% \changes{v2.1a}{90/02/15}{Support for new font selection.}
% \changes{v2.2a}{94/02/03}{Cleanup for LaTeX2e}
%
% \DoNotIndex{\@endparenv,\@ifdefinable,\@ifnextchar,\@ifundefined}
% \DoNotIndex{\@namedef,\@nameuse,\@spaces,\@tempa,\@warning}
% \DoNotIndex{\ }
% \DoNotIndex{\begingroup,\bfseries}
% \DoNotIndex{\csname}
% \DoNotIndex{\def,\do,\docdate}
% \DoNotIndex{\endcsname,\endgroup,\endinput,\endtrivlist,\expandafter}
% \DoNotIndex{\filedate,\fileversion}
% \DoNotIndex{\gdef,\global}
% \DoNotIndex{\hbox,\hskip}
% \DoNotIndex{\ignorespaces,\input,\itshape,\item}
% \DoNotIndex{\labelsep,\let,\llap}
% \DoNotIndex{\makeatletter}
% \DoNotIndex{\noexpand}
% \DoNotIndex{\refstepcounter,\relax,\rlap}
% \DoNotIndex{\setlength,\slshape,\space,\string,\strut}
% \DoNotIndex{\the,\trivlist,\typeout}
% \DoNotIndex{\vbox}
% \DoNotIndex{\xdef}
%
% \def\deflabel#1{\ttfamily #1\hfill}
% \def\deflist#1{\begin{list}{}{\settowidth\labelwidth{\ttfamily #1}%
%                               \setlength\leftmargin\labelwidth
%                               \addtolength\leftmargin\labelsep
%                               \let\makelabel\deflabel}}
% \def\enddeflist{\end{list}}
% \def\envfont{\normalfont\ttfamily}
%
% \MakeShortVerb{\|}
%
% \newcommand{\TUB}{{\sl TUGboat\/}}
%
% \GetFileInfo{theorem.dtx}
% \title{An Extension of the
%        \LaTeX{} theorem environment\thanks{This file
%        has version number \fileversion, last
%        revised \filedate.}}
% \author{Frank Mittelbach}
%
% \MaintainedByLaTeXTeam{tools}
% \maketitle
%
% \begin{abstract}
%    The macros described in this paper yield an extension of the
%    \LaTeX{} theorem mechanism. It is designed to satisfy the
%    different requirements of various journals. Thus, the
%    layout of the ``theorems'' can be manipulated by determining a
%    ``style''. This article describes not only the use, but
%    also the definition, of the necessary macros.
% \end{abstract}
%
% \section*{Preface to version 2.2}
%
% For \LaTeXe{} this package did not need any fundamental changes. I
% only modified the messages generated so that theorem layout styles
% will show up with the |\listfiles| command and cleaned the section
% on the New Font Selection Scheme since this is now included in
% \LaTeX.
%
% \section*{Preface to version 2.1}
%
% This version is identical to 2.0g described in \TUB\ 10\#3 except
% for some internal defaults which are now set depending on the used
% font selection scheme.
%
% This was done to avoid unpleasant surprises if the new font selection
% scheme is in force. For further details see
% section~\ref{sec:newfonts} and \cite{bk:GMS94}.
%
% \section{Introduction}
%
% For our purposes here, ``theorems'' are labelled enunciations,
% often set off from the main text by extra space and a font change.
% Theorems, corollaries, conjectures, definitions, and remarks
% are all instances of ``theorems''.  The ``header'' of these structures
% is composed of a label (such as \textsc{Theorem} or \textsc{Remark})
% and a number which serializes an item in the sequence of items
% with the same label.
%
% Shortly after the introduction of \LaTeX{} at the Fachbereich
% Mathematik in Mainz, the desire to manipulate the layout of
% ``theorems'' arose.  In Mainz, the following two conventions came
% into general use:
% \begin{enumerate}
%   \item The number of the theorem is shown in the margin.
%   \item There is a line break at
%         the end of the theorem header.
% \end{enumerate}
%
% Additionally, some journals require different formats which depend on
% the ``sort of theorem'': e.g.\ often remarks and definitions are set
% in |\upshape|, while |\itshape| is employed for main theorems.
%
% Confronted with these requirements, a theorem environment was
% developed in Mainz which allows separate determination of the
% layout of the ``theorems sets'', comparable to |\pagestyle|.
%
%
% \section{The user interface}
% \subsection{Defining new theorem sets}
%
% \DescribeMacro\newtheorem
% As in the original \LaTeX{} version, the command |\newtheorem|
% defines a new ``theorem set'' or ``theorem-like structure''.
% Two required arguments name the new environment and give the text
% to be typeset with each instance of the new ``set'', while
% an optional argument determines how
% the ``set'' is enumerated:
% \begin{description}
%    \item[\ttfamily \bslash newtheorem\{foo\}\{bar\}]
%       The theorem set {\envfont foo} (whose name is \texttt{bar})
%       uses its own counter.
%    \item[\ttfamily \bslash newtheorem\{foo2\}{[foo]}\{bar2\}]
%       The theorem set {\envfont foo2} (printed name \texttt{bar2})
%       uses the same counter as the theorem set \texttt{foo}.
%    \item[\ttfamily \bslash newtheorem\{foo3\}\{bar3\}{[section]}]
%       The theorem set {\envfont foo3} (printed name \texttt{bar3}) is
%       enumerated within the counter \texttt{section}, i.e.\ with every
%       new |\section| the enumeration begins again with 1, and
%       the enumeration is composed from the section-number and the
%       theorem counter itself.
% \end{description}
%
% \DescribeMacro\theoremstyle
% Additionally, the command |\theoremstyle| can define the layout
% of various, or all, theorem sets.  It should be noted that any
% theorem set defined by |\newtheorem| is typeset in the
% |\theoremstyle| that is current at the time of the definition.
% Thus, the following
% \begin{verbatim}
%    \theoremstyle{break}        \newtheorem{Cor}{Corollary}
%    \theoremstyle{plain}        \newtheorem{Exa}{Example}[section]
%\end{verbatim}                             ^^A No blank at line start
% leads to the result that the set {\envfont Cor} is formatted in the
% style \texttt{break}, while the set {\envfont Exa} and all the
% following ones are formatted in the style \texttt{plain}, unless
% another |\theoremstyle| follows.  Since the definitions installed by
% |\newtheorem| are global, one also can limit |\theoremstyle| locally
% by grouping braces.
%
% \DescribeMacro\theorembodyfont
% The choice of the font for the theorem body is completely
% independent of the chosen |\theoremstyle|; this has proven to
% be very advantageous.  For example,
% \begin{verbatim}
%   {\theorembodyfont{\upshape}          \newtheorem{Rem}{Remark}}
%\end{verbatim}
% defines a theorem set {\envfont Rem}, which will be set in
% |\upshape| in the current layout (which in our example is
% \texttt{plain}).  As with |\theoremstyle|, the |\theorembodyfont|
% chosen is that current at the time of |\newtheorem|.  If
% |\theorembodyfont| is not specified or one defines
% |\theorembodyfont{}|, then the font used will be that defined by the
% |\theoremstyle|.
%
% \DescribeMacro\theoremheaderfont It is also possible to customize
% the font used for the theorem headers.  This is, however, a global
% declaration, and therefore there should be at most one
% |\theoremheaderfont| declaration in the preamble.\footnote{If it is
% actually necessary to have different header fonts, one has to define
% new theorem styles (substituting the desired font) or specify the
% information directly in the \texttt{\bslash newtheorem} declaration
% (the unclean variant).}
%
% \DescribeMacro\theorempreskipamount
% \DescribeMacro\theorempostskipamount Two additional parameters
% affect the vertical space around the theorem environments:
% |\theorempreskipamount| and |\theorempostskipamount| define,
% respectively, the spacing before and after such an environment.
% These parameters apply for all theorem sets and can be manipulated
% with the ordinary length macros.  They are rubber lengths
% (`\textsf{skips}'), and therefore can contain \texttt{plus} and
% \texttt{minus} parts.
%
% Since the definition of theorem sets should---most sensibly---be
% placed in the preamble, we only allow installation there.
% It is therefore possible to release the memory used here after
% |\begin{document}|, in order to make room for other
% applications.
%
%
% \subsection{Existing theorem styles}
%
% The following theorem styles exist to date:
% \begin{deflist}{marginbreak}
%    \item[plain]
%       This theorem style emulates the original \LaTeX{} definition,
%       except that additionally the parameters
%       |\theorem...skipamount| are used.
%    \item[break]
%       In this style, the theorem header is followed by a line
%       break.
%    \item[marginbreak]
%       The theorem number is set in the margin, and there is a line
%       break as in \texttt{break}.
%    \item[changebreak]
%       Like \texttt{break}, but with header number and text
%       interchanged.
%    \item[change]
%       Header number and text are interchanged, without a line break.
%    \item[margin]
%       The number is set in the left margin, without a line break.
% \end{deflist}
% All styles (except \texttt{plain}) select |\slshape| as the default
% |\theorembodyfont|.
%
%
% \subsection{Examples}
%
% Given the above theorem sets {\envfont Cor}, {\envfont Exa}
% and {\envfont Rem},
% suppose that the preamble also contains the declarations:
% \begin{verbatim}
%    \theoremstyle{marginbreak}   \newtheorem{Lem}[Cor]{Lemma}
%    \theoremstyle{change}
%    \theorembodyfont{\itshape}        \newtheorem{Def}[Cor]{Definition}
%
%    \theoremheaderfont{\scshape}
%\end{verbatim}
% Then the following are some typical examples of the typeset output
% resulting from their use.
%
% \begin{Cor}
%    This is a sentence typeset in the theorem environment {\envfont
%    Cor}.
% \end{Cor}
%
% \begin{Exa}
%    This is a sentence typeset in the theorem environment {\envfont
%    Exa}.
% \end{Exa}
%
% \begin{Rem}
%    This is a sentence typeset in the theorem environment {\envfont
%    Rem}.
% \end{Rem}
%
% \begin{Lem}[Ben User]
%    This is a sentence typeset in the theorem environment {\envfont
%    Lem}.
% \end{Lem}
%
% \begin{Def}[Very impressive Definition]
%    This is a sentence typeset in the theorem environment {\envfont
%    Def}.
% \end{Def}
%
% The last two examples show the effect of the optional argument to
% a theorem environment (it is the text typeset in parentheses).
%
%
% \section{Special Considerations}
% \label{sec:newfonts}
%
% Theoremheader and body are implemented as a unit. This means that
% the |\theoremheaderfont|  will inherit characteristics of the
% |\theorembodyfont| in \LaTeXe. Thus, if for example
% |\theorembodyfont| is |\itshape| and |\theoremheaderfont|
% is |\bfseries| the font selected for the header will have the
% characteristics `bold extended italic'. If this is not desired one
% should set the |\theoremheaderfont| to something like
% \begin{verbatim}
%      \theoremheaderfont{\normalfont\bfseries}
%\end{verbatim}
% i.e.\ supplying all necessary font information explicitly.
%
%
%
% \section{Acknowledgements}
%
% The publication of this set of macros was only possible with the
% help of Christina Busse (translating the manuscript into English),
% Joachim Pense (playing the r\^ole of typist), Chris Rowley
% (looking everything over) and many others providing useful
% suggestions.
%
% \MaybeStop{%
%    \begin{thebibliography}{9}
%       \bibitem{bk:GMS94} \textsc{M.~Goossens}, \textsc{F.~Mittelbach}
%         and \textsc{A.~Samarin}.
%         \newblock The \LaTeX{} Companion.
%         \newblock
%         Addison-Wesley, Reading, Massachusetts, 1994.
%       \bibitem{latex:source} \textsc{Lamport, Leslie.} \newblock
%                               \texttt{latex.tex}, version 2.09, date
%                               Feb. 1990.
%    \end{thebibliography}
%    \PrintIndex
%    \PrintChanges}
%
%
% \section{The documentation driver file}
%
% The next bit of code contains the documentation driver file for
% \TeX{}, i.e., the file that will produce the documentation you are
% currently reading. It will be extracted from this file by the
% \texttt{docstrip} program. Since it is the first code in the file
% one can alternatively process this file directly with \LaTeXe{} to
% obtain the documentation.
%
%    \begin{macrocode}
%<*driver>
\documentclass{ltxdoc}

\usepackage{theorem}

%<+driver>% The next few lines define theorem sets which are used
%<+driver>% in the example section of the documentation.

\theoremstyle{break}       \newtheorem{Cor}{Corollary}
\theoremstyle{plain}       \newtheorem{Exa}{Example}[section]
{\theorembodyfont{\upshape}\newtheorem{Rem}{Remark}}
\theoremstyle{marginbreak} \newtheorem{Lem}[Cor]{Lemma}
\theoremstyle{change}
\theorembodyfont{\itshape} \newtheorem{Def}[Cor]{Definition}

\theoremheaderfont{\scshape}

\RecordChanges

\begin{document}
   \DocInput{theorem.dtx}
\end{document}
%</driver>
%    \end{macrocode}
%
% \section{Definition of the Macros}
%
% If the file has been loaded before, we abort immediately. If not the
% package announces itself (this is actually done at the very top of
% the file---the way it is done isn't good style so don't copy it).
% \changes{v2.0e}{89/07/19}{Spaces removed from `typeout'.}
% \changes{v2.2b}{95/11/19}{Announce moved to top of file}
%    \begin{macrocode}
%<*package>
%\@ifundefined{theorem@style}{}{\endinput}
%\def\FMithmInfo{1995/11/19 v2.2b Theorem extension package (FMi)}
%\ProvidesPackage{theorem}[\FMithmInfo]
%    \end{macrocode}
%
%
% \subsection{Definition of theorem styles and fonts}
%
% All the definitions in this file are done globally to allow
% inputting this file inside a group.
%
% \begin{macro}{\theoremstyle}
%    Before a theorem style can be installed, the chosen style must be
%    known. For that reason, we must test to see that
%    |\th@|\meta{style} is known or, more precisely, that it is
%    different from |\relax|. If the style is not known then
%    |\th@plain| is used.
%    \begin{macrocode}
\gdef\theoremstyle#1{%
   \@ifundefined{th@#1}{\@warning
          {Unknown theoremstyle `#1'. Using `plain'}%
          \theorem@style{plain}}%
%    \end{macrocode}
%    We save the theorem style to be used in the token register
%    |\theorem@style|.
%    \begin{macrocode}
      {\theorem@style{#1}}%
%    \end{macrocode}
%    Now we ``evaluate'' the theorem style: this means, we call the
%    macro |\th@|\meta{style} which will activate the relevant
%    definitions which are contained in a separate file.  This is done
%    in a group to suppress changes to the current font.  This could
%    otherwise pose problems together with the new font selection
%    scheme\footnote{When I printed the original article using the new
%    font selection scheme I ended with a document with slanted
%    typefaces (text headings and all) simply because one of the
%    theorem styles used \texttt{\bslash sl} at toplevel.} if the
%    |\th@|\meta{style} is evaluated a second time.
% \changes{v2.1a}{90/02/15}{Group added around `th@..}
% \changes{v2.1a}{90/02/15}{Reseting of `..@begintheorem removed.}
%    \begin{macrocode}
      \begingroup
        \csname th@\the\theorem@style \endcsname
      \endgroup}
%    \end{macrocode}
% \end{macro}
%
%
% \begin{macro}{\@begintheorem}
% \changes{v2.1a}{90/02/15}{Relaxed at toplevel.}
% \begin{macro}{\@opargbegintheorem}
% \changes{v2.1a}{90/02/15}{Relaxed at toplevel.}
%    We reset |\@begintheorem| and |\@opargbegintheorem| to
%    |\relax| since these commands are no longer necessary at
%    toplevel. This will save a few tokens.
%    \begin{macrocode}
\global\let\@begintheorem\relax
\global\let\@opargbegintheorem\relax
%    \end{macrocode}
% \end{macro}
% \end{macro}
%
% \begin{macro}{\theorem@style}
%    Obviously the \textsf{token} register used above has to be
%    allocated. To assure the utmost compatibility with the
%    original \LaTeX{} definition, we set the default theorem style to
%    \texttt{plain}, which implements the usual \LaTeX{} convention.
% \changes{v2.0f}{89/07/20}{Unnecessary `global' removed.}
%    \begin{macrocode}
\newtoks\theorem@style
\global\theorem@style{plain}
%    \end{macrocode}
% \end{macro}
%
% \begin{macro}{\theorembodyfont}
% \begin{macro}{\theorem@bodyfont}
%    If the |\theorembodyfont| is set by the user then it should not
%    interact with the default font set in the theorem style. When the
%    new font selection is in force this may happen if, for example,
%    the default is |\itshape| and the new |\theorembodyfont| is
%    |\sffamily|. So we add a |\reset@font| command in front of the
%    user definition.
% \changes{v2.0f}{89/07/20}{Unnecessary `global' removed.}
% \changes{v2.0g}{89/09/19}{`theoremfont to `theorembodyfont.}
% \changes{v2.1c}{92/06/22}{`reset@font added and logic changed.}
% \changes{v2.1d}{92/10/28}{Reset to empty if arg is empty}
%    \begin{macrocode}
\gdef\theorembodyfont#1{%
%    \end{macrocode}
%    We check if the argument supplied is empty and if so put nothing
%    into the |\theorem@bodyfont| token register to allow for
%    |\theorembodyfont{}| as a mean of using the default of the
%    current |\theoremstyle|.
%    \begin{macrocode}
   \def\@tempa{#1}%
   \ifx\@tempa\@empty
    \theorem@bodyfont{}%
   \else
    \theorem@bodyfont{\reset@font#1}%
   \fi
   }
\newtoks\theorem@bodyfont
\global\theorem@bodyfont{}
%    \end{macrocode}
% \end{macro}
% \end{macro}
%
%
% \begin{macro}{\theoremheaderfont}
% \changes{v2.0g}{89/09/19}{Macro added.}
%    The font for the theorem headers is handled differently because
%    this definition applies to all theorem styles.
%    \begin{macrocode}
\gdef\theoremheaderfont#1{\gdef\theorem@headerfont{#1}%
%    \end{macrocode}
%    After using the macro once it is redefined to produce an error
%    message.
%    \begin{macrocode}
       \gdef\theoremheaderfont##1{%
        \typeout{\string\theoremheaderfont\space should be used
                 only once.}}}
%    \end{macrocode}
% \end{macro}
%
% \begin{macro}{\theorem@headerfont}
% \changes{v2.0g}{89/09/19}{Macro added.}
%    To set the |\theorem@headerfont| default we first test if
%    the new fontselection scheme is in force.
% \changes{v2.1a}{90/02/15}{Support for new font selection.}
%    \begin{macrocode}
\ifx\upshape\undefined
%    \end{macrocode}
%    If not we define it to expand into |\bfseries|. We don't use
%    |\let| just in case a following style option redefines this
%    macro.
%    \begin{macrocode}
\gdef\theorem@headerfont{\bfseries}
%    \end{macrocode}
%    Otherwise we reset the current shape before calling |\bfseries|.
%    \begin{macrocode}
\else \gdef\theorem@headerfont{\normalfont\bfseries}\fi
%    \end{macrocode}
% \end{macro}
%
% \begin{macro}{\th@plain}
% \begin{macro}{\th@break}
% \begin{macro}{\th@marginbreak}
% \begin{macro}{\th@changebreak}
% \begin{macro}{\th@change}
% \begin{macro}{\th@margin}
%    The different styles are defined in macros such as
%    |\th@plain|.  Since memory space is precious in
%    ``non-Big-versions'', we have to avoid offering too many unused
%    definitions.  Therefore we define these styles in separate files
%    that can be loaded on demand.  Thus the commands themselves only
%    load these files. We use |\@input@|, a \LaTeXe{} internal command
%    that ensures that the file will be listed with |\listfiles|.
% \changes{v2.2a}{94/02/02}{Use `@input@ to load theorem layout files}
%    \begin{macrocode}
\gdef\th@plain{\@input@{thp.sty}}
\gdef\th@break{\@input@{thb.sty}}
\gdef\th@marginbreak{\@input@{thmb.sty}}
\gdef\th@changebreak{\@input@{thcb.sty}}
\gdef\th@change{\@input@{thc.sty}}
\gdef\th@margin{\@input@{thm.sty}}
%    \end{macrocode}
%    This list will be expanded when new styles become
%    available.  For testing,  just append new theorem substyles
%    as document options.
% \end{macro}
% \end{macro}
% \end{macro}
% \end{macro}
% \end{macro}
% \end{macro}
%
%
%
% \subsection{Definition of a new theorem set}
%
% As already pointed out, a new theorem environment can be defined in
% three different ways:
% \begin{verbatim}
%    \newtheorem{Lem}{Lemma}
%    \newtheorem{Lem}{Lemma}[section]
%    \newtheorem{Lem}[Theorem]{Lemma}
%\end{verbatim}
% The function of the macro |\newtheorem| is to recognize these cases
% and then to branch into one of the three macros |\@ynthm|, |\@xnthm|
% or |\@othm|.  This mechanism is adopted unchanged from
% \cite{latex:source}; the essential point here is that, for example,
% in the second case, the arguments \texttt{Lem}, \texttt{Lemma} and
% \texttt{section} are passed over to the macro |\@xnthm|.
%
% We inspect this case first because the others present fewer problems,
% and thus are easily derived from this one.
%
% \begin{macro}{\@xnthm}
%    For our example arguments, the macro |\@xnthm| must fulfill
%    the following:
%    \begin{itemize}
%       \item
%          Define a new \LaTeX-counter `Lem'
%       \item
%          reset this counter within a |\section|
%       \item
%          define the macro |\theLem|
%       \item
%          define the environment macros |\Lem| and
%          |\endLem| using the current
%          |\theoremstyle| and |\theorem@bodyfont|.
%    \end{itemize}
%    Obviously, all this should happen only if the first argument
%    of |\@xnthm| (i.e.\ \texttt{Lem} in our example) is chosen so
%    as not to conflict with any previously defined commands or
%    environments.  This test is performed by the \LaTeX{} macro
%    |\@ifdefinable|.
%    \begin{macrocode}
\gdef\@xnthm#1#2[#3]{\expandafter\@ifdefinable\csname #1\endcsname
%    \end{macrocode}
%    Therefore, the first argument of |\@ifdefinable| is the
%    expansion (in the example, |\Lem|) of
%    |\csname||#1\endcsname|.  The second argument is
%    executed only if the test has been completed successfully.
%    \begin{macrocode}
   {%
%    \end{macrocode}
%    Now we define the new counter. The names of the \LaTeX{}
%    macros employed should speak for themselves:
% \changes{v2.2b}{1995/10/07}{Use \cs{@newctr}}
% \changes{v2.2c}{1995/11/23}{Use \cs{@newctr} properly.}
%    \begin{macrocode}
    \@definecounter{#1}\@newctr{#1}[#3]%
%    \end{macrocode}
%    Using |\@newctr| will give a proper error message if the counter
%    in |#3| is not defined.
%    In defining `|\theLem|' we must generate the desired macro
%    name by use of |\expandafter| and |\csname|.
%    \begin{macrocode}
    \expandafter\xdef\csname the#1\endcsname
%    \end{macrocode}
%    An |\xdef| is used in order to make the definition global,
%    and to ensure that it contains the replacement texts of
%    |\@thmcountersep| and |\@thmcounter|.\footnote{These
%    two macros can be defined by the document style.  Their default
%    values produce a `\texttt{.}' as separation and an arabic
%    representation of the number.} However, not everything should be
%    expanded. For example, it saves space to use |\thesection|
%    instead of its---at times---lengthy expansion.
%    \begin{macrocode}
      {\expandafter \noexpand \csname the#3\endcsname
       \@thmcountersep \@thmcounter{#1}}%
%    \end{macrocode}
%    Thus with the defaults of \LaTeX{}, |\theLem| would be
%    replaced by the command sequence |\thesection.\arabic{Lem}|.
%
%    We will now look at the definition of the macro which is executed
%    at the beginning of the actual environment (in our example this
%    macro is |\Lem|). It should be noted that we use an
%    ``|\expandafter| trick'' to expand only certain parts of the
%    replacement text at the time of the definition.
%    \begin{macrocode}
    \def\@tempa{\global\@namedef{#1}}%
    \expandafter \@tempa \expandafter{%
%    \end{macrocode}
%    First, the macro that contains the current definitions of
%    |\@begintheorem| and |\@opargtheorem| should be called
%    up.  The name of this macro---as is already known---has the form
%    |\th@|\meta{theorem\ style}; therefore, it must be called by
%    \begin{macrocode}
      \csname th@\the \theorem@style
%    \end{macrocode}
%    In addition the default theorem font should be changeable, i.e.\
%    we have to insert the contents of |\theorem@bodyfont|. For
%    that reason, we expand even further, beyond |\endcsname|,
%    and thus insert the contents of the \textsf{token} register
%    |\theorem@bodyfont| in the replacement text.
%    \begin{macrocode}
            \expandafter \endcsname \the \theorem@bodyfont
%    \end{macrocode}
%    Now it is time to call the macro |\@thm| which takes over
%    the further processing.  It has two arguments: the current
%    counter name (in our example, \texttt{Lem}), and the text of the
%    label (in our example, \texttt{Lemma}).
%    \begin{macrocode}
     \@thm{#1}{#2}}%
%    \end{macrocode}
%    With this, the `sub-definition' is complete.  The macro
%    |\@endtheorem| ends a theorem environment and is, so far, nothing
%    but an |\endtrivlist|.  (Hence it is defined globally, and not
%    within the theorem styles.\footnote{This has to be changed as
%    soon as theorem styles that change \texttt{\bslash @endtheorem}
%    exist.  In such a case, all existing styles must be changed as
%    well since they will have to reset the macro.}) Therefore, we can
%    set it equivalent to the macro that ends the theorem set (in our
%    example, |\endLem|).  However, if some day theorem styles exist
%    that do change |\@endtheorem|, we would have to use the
%    commented-out line instead.
%    \begin{macrocode}
    \global \expandafter \let \csname end#1\endcsname \@endtheorem
%   \global\@namedef{end#1}{\@endtheorem}%
%    \end{macrocode}
%    With these commands all the required definitions are employed,
%    unless the test |\@ifdefinable| has failed.  Therefore, we
%    end the second argument of this macro and with it the definition
%    of |\@xnthm|.
%    \begin{macrocode}
   }}
%    \end{macrocode}
% \end{macro}
%
% \begin{macro}{\@ynthm}
%    The definition of |\@ynthm| is completely analogous. In
%    this case the new counter that is defined is not reset within
%    another counter; thus the definition of |\the...|
%    is simplified:
%    \begin{macrocode}
\gdef\@ynthm#1#2{\expandafter\@ifdefinable\csname #1\endcsname
   {\@definecounter{#1}%
    \expandafter\xdef\csname the#1\endcsname{\@thmcounter{#1}}%
%    \end{macrocode}
%    The rest of the definition corresponds literally to that of
%    |\@xnthm|:
%    \begin{macrocode}
    \def\@tempa{\global\@namedef{#1}}\expandafter \@tempa
     \expandafter{\csname th@\the \theorem@style \expandafter
     \endcsname \the\theorem@bodyfont \@thm{#1}{#2}}%
    \global \expandafter \let \csname end#1\endcsname \@endtheorem}}
%    \end{macrocode}
% \end{macro}
%
% \begin{macro}{\@othm}
%    The definition of |\@othm| does not contain anything new.
%    \begin{macrocode}
\gdef\@othm#1[#2]#3{%
%    \end{macrocode}
% \changes{v2.2b}{95/10/07}{Build in check for counter defined}
%    We do not define a new counter but instead use one that has
%    already been defined.  Thus the only definition we need is that
%    of this pseudo-counter (i.e.\ |\the|\meta{env.\ name}).
%    First we check if |#2| corresponds to a known counter name.
%    \begin{macrocode}
  \expandafter\ifx\csname c@#2\endcsname\relax
   \@nocounterr{#2}%
  \else
   \expandafter\@ifdefinable\csname #1\endcsname
   {\expandafter \xdef \csname the#1\endcsname
     {\expandafter \noexpand \csname the#2\endcsname}%
%    \end{macrocode}
%    All other parts of the definition can be adopted from
%    |\@xnthm|. We have to remember, though, that in this case
%    the name of the current counter and the theorem label have moved
%    to the second and third arguments.
%    \begin{macrocode}
    \def\@tempa{\global\@namedef{#1}}\expandafter \@tempa
     \expandafter{\csname th@\the \theorem@style \expandafter
     \endcsname \the\theorem@bodyfont \@thm{#2}{#3}}%
    \global \expandafter \let \csname end#1\endcsname \@endtheorem}%
  \fi}
%    \end{macrocode}
% \end{macro}
%
% \subsection{Macros that are employed in a {\envfont theorem}
%    environment}
%
% \begin{macro}{\@thm}
%    The macro |\@thm| has to increase the current counter.  Then,
%    depending on whether the environment has (or does not have) an
%    optional argument, it has to branch into either |\@begintheorem|
%    or |\@opargtheorem|.
%    \begin{macrocode}
\gdef\@thm#1#2{\refstepcounter{#1}%
%    \end{macrocode}
%    Now we start a {\envfont trivlist} environment, and give
%    |\@topsep| and |\@topsepadd| the values of the \textsf{skip}
%    registers |\theorempreskipamount| and |\theorempostskipamount|.
%    The value in |\@topsep| is the vertical space that is inserted by
%    the first (and only) |\item| in our |\trivlist| whilst
%    |\@topsepadd| is inserted by |\@endparenv| at the end of that
%    {\envfont trivlist} environment.  By using these registers, we
%    obtain the desired space around a {\envfont theorem} environment.
%    \begin{macrocode}
   \trivlist
   \@topsep \theorempreskipamount               % used by first \item
   \@topsepadd \theorempostskipamount           % used by \@endparenv
%    \end{macrocode}
%    Now we have to test whether an optional argument has been given.
% \changes{v2.0d}{89/07/18}{Forgotten percent added.}
%    \begin{macrocode}
   \@ifnextchar [%
%    \end{macrocode}
%    If there is an optional argument, we will call |\@ythm|, and
%    move the arguments read back into the input stream.
%    \begin{macrocode}
   {\@ythm{#1}{#2}}%
%    \end{macrocode}
%    If not, we call |\@begintheorem|. Its first argument is the
%    name of the theorem set (hence the second argument of
%    |\@thm|). Its second argument is the macro that produces the
%    current number.
%    \begin{macrocode}
   {\@begintheorem{#2}{\csname the#1\endcsname}\ignorespaces}}
%    \end{macrocode}
% \end{macro}
%
% \begin{macro}{\@xthm}
% \begin{macro}{\@ythm}
%    Both these macros were originally called by |\@thm|.  We do
%    not need |\@xthm| anymore, hence we reset it to
%    |\relax|. The definition of |\@ythm| has not changed at
%    all from its definition in \LaTeX. In order to make the macros
%    easier to understand, we will
%    nevertheless present it (commented out).
%    \begin{macrocode}
\global\let\@xthm\relax
% \def\@ythm#1#2[#3]{\@opargbegintheorem{#2}{\csname
%           the#1\endcsname}{#3}\ignorespaces}
%</package>
%    \end{macrocode}
%    The primitive |\ignorespaces| in |\@ythm| and
%    |\@thm| is needed to remove the spaces between the
%    |\begin{...}| and the actual text.
% \end{macro}
% \end{macro}
%
%
% \subsection{Definition of the theorem substyles}
%
% As already pointed out, the theorem substyles, defined below, are
% only loaded when necessary.  Note that all these substyles, except
% \texttt{plain}, have |\slshape| as the default body font.
%
% \subsubsection{The \texttt{plain} style}
%
% \label{dok:th@plain} As the following macros use |@|, we have
% to locally set the |\catcode| of this symbol to ``letter''.
% This happens within a group, so that we do not have to worry about
% which |\catcode| that symbol had before.
%    \begin{macrocode}
%<*thp>
\begingroup \makeatletter
%    \end{macrocode}
% Since we are now within a group, we must make all definitions
% globally.  First we make sure that \texttt{theorem.sty} is loaded.
% This will allow us to use this file as a document style option
% without having to call \texttt{theorem} itself as an option.  At the
% same time, we assure that at least version 2 is loaded, since
% |\theorem@style| was not defined in earlier versions.
%    \begin{macrocode}
\@ifundefined{theorem@style}{\input{theorem.sty}}{}
\ProvidesFile{thp.sty}
             [\FMithmInfo]
%    \end{macrocode}
%
%
% \begin{macro}{\th@plain}
% \begin{macro}{\@begintheorem}
% \begin{macro}{\@opargbegintheorem}
%    |\theoremstyle{plain}| corresponds to the original
%    definition, except that the distances to the surrounding text are
%    determined by the parameters |\theorempreskipamount| and
%    |\theorempostskipamount|.
%    First we set the default body font.
%    \begin{macrocode}
\gdef\th@plain{\normalfont\itshape
%    \end{macrocode}
%    Then we define |\@begintheorem| and
%    |\@opargbegintheorem|. These two macros define how the
%    header of a theorem is typeset.  |\@opargbegintheorem| will
%    be called if a \textsf{theorem} environment with an optional
%    argument is encountered; otherwise, the header is constructed by
%    calling |\@begintheorem|.  If one of these macros is
%    executed, we are within a \textsf{trivlist} environment started by
%    |\@thm|.  So the theorem header is produced with an
%    |\item| command.
%
%    Instead of specifying the header font directly, all standard
%    theorem styles use the |\theorem@headerfont| macro to allow
%    customization.  The extra space (|\labelsep|) is necessary
%    because of problems in the \textsf{trivlist} environment.
%    \begin{macrocode}
  \def\@begintheorem##1##2{%
        \item[\hskip\labelsep \theorem@headerfont ##1\ ##2]}%
%    \end{macrocode}
%    The definition of |\@opargbegintheorem| is completely
%    analogous. The only difference is the fact that there exists a
%    third argument (which is the optional parameter of the
%    environment and contains additional information about the
%    theorem).  Customarily we enclose it in parentheses.
%    \begin{macrocode}
\def\@opargbegintheorem##1##2##3{%
   \item[\hskip\labelsep \theorem@headerfont ##1\ ##2\ (##3)]}}
%    \end{macrocode}
% \end{macro}
% \end{macro}
% \end{macro}
%
% \noindent
% We conclude with an |\endgroup| to restore the
% |\catcode| of |@|.
%    \begin{macrocode}
\endgroup
%</thp>
%    \end{macrocode}
%
%
% \subsubsection{The \texttt{break} style}
%
% This style option is stored in the file |thb.sty|.
% For the next two lines see the documentation for |\th@plain| on
% page~\pageref{dok:th@plain}.
%    \begin{macrocode}
%<*thb>
\begingroup \makeatletter
\@ifundefined{theorem@style}{\input{theorem.sty}}{}
\ProvidesFile{thb.sty}
             [\FMithmInfo]
%    \end{macrocode}
%
% \begin{macro}{\th@break}
%    |\theoremstyle{break}| produces a line break after the name
%    of the theorem.  The font is |\slshape|.  Hence, we define
%    |\th@break| as follows:
%    \begin{macrocode}
\gdef\th@break{\normalfont\slshape
  \def\@begintheorem##1##2{\item[%
%    \end{macrocode}
%    We run into the following problem: it is not possible to create
%    the header with |\item[|\meta{title}|]| and then start
%    a new line by, for example, |\mbox{}\\|. Such a definition
%    will fail whenever a list environment follows immediately.  With
%    the above construction, the |\mbox{}| causes the switch
%    |@inlabel| (cf.\ definition of |\list| and
%    |\trivlist| in \cite{latex:source}) to be set to \texttt{false}
%    and so the following list will insert additional vertical
%    space (|\topskip|). This is quite annoying. Therefore, we
%    create the line break within the |\item|. In order to ensure
%    that the text will begin at the proper position in the following
%    line, we simply pretend that the label does not take any
%    room.\footnote{This will lead to problems whenever very high
%    symbols occurring in the line tower into the
%    heading. So, something else has to be done here sometime.}
%    \begin{macrocode}
       \rlap{\vbox{\hbox{\hskip \labelsep\theorem@headerfont ##1\ ##2}%
                   \hbox{\strut}}}]}%
%    \end{macrocode}
%    Again, the definition of |\@opargbegintheorem| is
%    completely analogous.
%    \begin{macrocode}
\def\@opargbegintheorem##1##2##3{%
  \item[\rlap{\vbox{\hbox{\hskip \labelsep \theorem@headerfont
                     ##1\ ##2\ (##3)}%
                    \hbox{\strut}}}]}}
\endgroup
%</thb>
%    \end{macrocode}
% \end{macro}
%
%
% \subsubsection{The \texttt{changebreak} style}
%
%    \begin{macrocode}
% This style option is stored in the file |thcb.sty|.
%    \begin{macrocode}
%<*thcb>
\begingroup \makeatletter
\@ifundefined{theorem@style}{\input{theorem.sty}}{}
\ProvidesFile{thcb.sty}
             [\FMithmInfo]
%    \end{macrocode}
%
% \begin{macro}{\th@changebreak}
%    The \texttt{change-break} theorem style is like \texttt{break}
%    but with interchange of theorem name and theorem number.  Thus we
%    define |\th@changebreak| as follows:
%    \begin{macrocode}
\gdef\th@changebreak{\normalfont\slshape
  \def\@begintheorem##1##2{\item
       [\rlap{\vbox{\hbox{\hskip\labelsep \theorem@headerfont ##2\ ##1}%
                    \hbox{\strut}}}]}%
\def\@opargbegintheorem##1##2##3{%
  \item[\rlap{\vbox{\hbox{\hskip\labelsep \theorem@headerfont
                     ##2\ ##1\ (##3)}%
                    \hbox{\strut}}}]}}
\endgroup
%</thcb>
%    \end{macrocode}
% \end{macro}
%
%
% \subsubsection{The \texttt{change} style}
%
% This style option is stored in the file |thc.sty|.
%    \begin{macrocode}
%<*thc>
\begingroup \makeatletter
\@ifundefined{theorem@style}{\input{theorem.sty}}{}
\ProvidesFile{thc.sty}
             [\FMithmInfo]
%    \end{macrocode}
%
% \begin{macro}{\th@change}
%    The \texttt{change} theorem style corresponds to the
%    \texttt{change break} style without a linebreak after the header.
%    To say it in another way, it's the same as the \texttt{plain}
%    style but with number and name interchanged and |\slshape| as the
%    default font.
%    \begin{macrocode}
\gdef\th@change{\normalfont\slshape
  \def\@begintheorem##1##2{\item
       [\hskip\labelsep \theorem@headerfont ##2\ ##1]}%
\def\@opargbegintheorem##1##2##3{%
  \item[\hskip\labelsep \theorem@headerfont ##2\ ##1\ (##3)]}}
\endgroup
%</thc>
%    \end{macrocode}
% \end{macro}
%
%
% \subsubsection{The \texttt{marginbreak} style}
%
% This style option is the one used most often at Mainz.  It is saved
% in the file |thmb.sty|.
%    \begin{macrocode}
%<*thmb>
\begingroup \makeatletter
\@ifundefined{theorem@style}{\input{theorem.sty}}{}
\ProvidesFile{thmb.sty}
             [\FMithmInfo]
%    \end{macrocode}
%
% \begin{macro}{\th@marginbreak}
%    The \texttt{margin break} style is nearly the same as the
%    \texttt{change break} style. The only difference is the placement
%    of the theorem number.  We use |\llap| to place it in the left
%    margin.
%
%    In this style |\labelsep| denotes the separation between the
%    number and the text.
%    \begin{macrocode}
\gdef\th@marginbreak{\normalfont\slshape
  \def\@begintheorem##1##2{\item
       [\rlap{\vbox{\theorem@headerfont
                    \hbox{\llap{##2}\hskip\labelsep ##1}%
                    \hbox{\strut}}}]}%
\def\@opargbegintheorem##1##2##3{%
  \item[\rlap{\vbox{\theorem@headerfont
                    \hbox{\llap{##2}\hskip\labelsep ##1\ (##3)}%
                    \hbox{\strut}}}]}}
\endgroup
%</thmb>
%    \end{macrocode}
% \changes{v2.1c}{90/02/17}{`endgroup outside macrocode env.}
% \end{macro}
%
%
%
% \subsubsection{The \texttt{margin} style}
%
% This style option is stored in the file |thm.sty|.
%    \begin{macrocode}
%<*thm>
\begingroup \makeatletter
\@ifundefined{theorem@style}{\input{theorem.sty}}{}
\ProvidesFile{thm.sty}
             [\FMithmInfo]
%    \end{macrocode}
%
% \begin{macro}{\th@margin}
%    Again this is only a variant of the theorem styles described
%    above without any new ideas.
%    \begin{macrocode}
\gdef\th@margin{\normalfont\slshape
  \def\@begintheorem##1##2{\item
       [\theorem@headerfont \llap{##2}\hskip\labelsep ##1]}%
\def\@opargbegintheorem##1##2##3{%
  \item[\theorem@headerfont \llap{##2}\hskip\labelsep ##1\ (##3)]}}
\endgroup
%</thm>
%    \end{macrocode}
% \end{macro}
%
%
%
%
% \subsection{Final Definitions}
%
% \begin{macro}{\theorempreskipamount}
% \begin{macro}{\theorempostskipamount}
%    The \textsf{skip} parameters that regulate the vertical empty space
%    before and after the {\envfont theorem} environment have
%    to be allocated as well.
% \changes{v2.0f}{89/07/20}{Unnecessary `global' removed.}
%    \begin{macrocode}
%<*package>
\newskip\theorempreskipamount
\newskip\theorempostskipamount
%    \end{macrocode}
%    Since we have used the same values for all theorem sets, we now can
%    assign them.
%    \begin{macrocode}
\global\setlength\theorempreskipamount{12pt plus 5pt minus 3pt}
\global\setlength\theorempostskipamount{8pt plus 3pt minus 1.5pt}
%    \end{macrocode}
% \end{macro}
% \end{macro}
%
%
% \begin{macro}{\@endtheorem}
%    The same holds for the macro |\@endtheorem|, which ends a
%    {\envfont theorem} environment. Since it is the same for all
%    theorem sets, it is removed from the macros |\th@|\meta{style}.
%    It simply ends the {\envfont trivlist} environment, which was
%    begun in |\@thm|.
%    \begin{macrocode}
\global\let\@endtheorem=\endtrivlist
%    \end{macrocode}
% \end{macro}
%
%
%
% \begin{macro}{\@preamblecmds}
%    All macros defined above are to be used only in the preamble.
%    Therefore, we insert them in |\@preamblecmds| which will disable
%    them at begin document. This is done by the internal \LaTeXe{}
%    command |\@onlypreamble|.
% \changes{v2.0g}{89/09/19}{`theoremheaderfont and `theorembodyfont
%    added.}
% \changes{v2.2a}{94/02/03}{Use \LaTeXe{} interface}
%    \begin{macrocode}
\@onlypreamble\@xnthm
\@onlypreamble\@ynthm
\@onlypreamble\@othm
\@onlypreamble\newtheorem
\@onlypreamble\theoremstyle
\@onlypreamble\theorembodyfont
\@onlypreamble\theoremheaderfont
%    \end{macrocode}
% \end{macro}
%
% Finally we declare the \texttt{plain} theorem style to be the default.
% \changes{v2.1b}{90/06/04}{Fixed missing `item by declaring default
%    style.}
%    \begin{macrocode}
\theoremstyle{plain}
%</package>
%    \end{macrocode}
%
% \Finale

\endinput
}{}
\ProvidesFile{thm.sty}
             [\FMithmInfo]
%    \end{macrocode}
%
% \begin{macro}{\th@margin}
%    Again this is only a variant of the theorem styles described
%    above without any new ideas.
%    \begin{macrocode}
\gdef\th@margin{\normalfont\slshape
  \def\@begintheorem##1##2{\item
       [\theorem@headerfont \llap{##2}\hskip\labelsep ##1]}%
\def\@opargbegintheorem##1##2##3{%
  \item[\theorem@headerfont \llap{##2}\hskip\labelsep ##1\ (##3)]}}
\endgroup
%</thm>
%    \end{macrocode}
% \end{macro}
%
%
%
%
% \subsection{Final Definitions}
%
% \begin{macro}{\theorempreskipamount}
% \begin{macro}{\theorempostskipamount}
%    The \textsf{skip} parameters that regulate the vertical empty space
%    before and after the {\envfont theorem} environment have
%    to be allocated as well.
% \changes{v2.0f}{89/07/20}{Unnecessary `global' removed.}
%    \begin{macrocode}
%<*package>
\newskip\theorempreskipamount
\newskip\theorempostskipamount
%    \end{macrocode}
%    Since we have used the same values for all theorem sets, we now can
%    assign them.
%    \begin{macrocode}
\global\setlength\theorempreskipamount{12pt plus 5pt minus 3pt}
\global\setlength\theorempostskipamount{8pt plus 3pt minus 1.5pt}
%    \end{macrocode}
% \end{macro}
% \end{macro}
%
%
% \begin{macro}{\@endtheorem}
%    The same holds for the macro |\@endtheorem|, which ends a
%    {\envfont theorem} environment. Since it is the same for all
%    theorem sets, it is removed from the macros |\th@|\meta{style}.
%    It simply ends the {\envfont trivlist} environment, which was
%    begun in |\@thm|.
%    \begin{macrocode}
\global\let\@endtheorem=\endtrivlist
%    \end{macrocode}
% \end{macro}
%
%
%
% \begin{macro}{\@preamblecmds}
%    All macros defined above are to be used only in the preamble.
%    Therefore, we insert them in |\@preamblecmds| which will disable
%    them at begin document. This is done by the internal \LaTeXe{}
%    command |\@onlypreamble|.
% \changes{v2.0g}{89/09/19}{`theoremheaderfont and `theorembodyfont
%    added.}
% \changes{v2.2a}{94/02/03}{Use \LaTeXe{} interface}
%    \begin{macrocode}
\@onlypreamble\@xnthm
\@onlypreamble\@ynthm
\@onlypreamble\@othm
\@onlypreamble\newtheorem
\@onlypreamble\theoremstyle
\@onlypreamble\theorembodyfont
\@onlypreamble\theoremheaderfont
%    \end{macrocode}
% \end{macro}
%
% Finally we declare the \texttt{plain} theorem style to be the default.
% \changes{v2.1b}{90/06/04}{Fixed missing `item by declaring default
%    style.}
%    \begin{macrocode}
\theoremstyle{plain}
%</package>
%    \end{macrocode}
%
% \Finale

\endinput
}{}
\ProvidesFile{thcb.sty}
             [\FMithmInfo]
%    \end{macrocode}
%
% \begin{macro}{\th@changebreak}
%    The \texttt{change-break} theorem style is like \texttt{break}
%    but with interchange of theorem name and theorem number.  Thus we
%    define |\th@changebreak| as follows:
%    \begin{macrocode}
\gdef\th@changebreak{\normalfont\slshape
  \def\@begintheorem##1##2{\item
       [\rlap{\vbox{\hbox{\hskip\labelsep \theorem@headerfont ##2\ ##1}%
                    \hbox{\strut}}}]}%
\def\@opargbegintheorem##1##2##3{%
  \item[\rlap{\vbox{\hbox{\hskip\labelsep \theorem@headerfont
                     ##2\ ##1\ (##3)}%
                    \hbox{\strut}}}]}}
\endgroup
%</thcb>
%    \end{macrocode}
% \end{macro}
%
%
% \subsubsection{The \texttt{change} style}
%
% This style option is stored in the file |thc.sty|.
%    \begin{macrocode}
%<*thc>
\begingroup \makeatletter
\@ifundefined{theorem@style}{% \iffalse meta-comment
%
% Copyright (C) 1993-2024
%
% The LaTeX Project and any individual authors listed elsewhere
% in this file.
%
% This file is part of the Standard LaTeX `Tools Bundle'.
% -------------------------------------------------------
%
% It may be distributed and/or modified under the
% conditions of the LaTeX Project Public License, either version 1.3c
% of this license or (at your option) any later version.
% The latest version of this license is in
%    https://www.latex-project.org/lppl.txt
% and version 1.3c or later is part of all distributions of LaTeX
% version 2005/12/01 or later.
%
% The list of all files belonging to the LaTeX `Tools Bundle' is
% given in the file `manifest.txt'.
%
% \fi
%
%%
%
%
\def\FMithmInfo{2023/07/05 v2.2c Theorem extension package (FMi)}
%
% \ProvidesFile{theorem.dtx}[\FMithmInfo]
%\iffalse   % this is a METACOMMENT !
%%
%% Package `theorem' to use with LaTeX2e
%% Copyright (C) 1989-1995 Frank Mittelbach, all rights reserved.
%<+package>\@ifundefined{theorem@style}{}{\endinput}
%<+package>\NeedsTeXFormat{LaTeX2e}[1995/06/01]
%<+package>\ProvidesPackage{theorem}
%<+package>         [\FMithmInfo]
%
% \fi
%
% \changes{v1.0}{87/05/07}{First draft.}
% \changes{v1.0a}{87/05/07}{`thistheoremstyle' added.}
% \changes{v1.1a}{87/07/29}{New concept: redefine the internal
%                          \protect\LaTeX{} commands `@xnthm' etc.}
% \changes{v1.1b}{87/07/31}{`theoremfont' used to specify font.}
% \changes{v1.2a}{88/01/08}{`@endtheorem' now globally defined.}
% \changes{v1.2b}{88/01/11}{`newtheorem' only allowed in the preamble.}
% \changes{v1.3a}{88/02/14}{`theoremfont' now a toks register.}
%
% \changes{v2.0a}{88/11/16}{New implementation and new documentation.}
% \changes{v2.0a}{88/11/16}{No layout changes but internal modifications
%                           old substyles won't work.}
% \changes{v2.0b}{89/01/30}{`global' used for all definitions.}
% \changes{v2.0c}{89/06/15}{Documentation translated to English.
%                            Thanks to LLL and Nina}
% \changes{v2.1a}{90/02/15}{Support for new font selection.}
% \changes{v2.2a}{94/02/03}{Cleanup for LaTeX2e}
%
% \DoNotIndex{\@endparenv,\@ifdefinable,\@ifnextchar,\@ifundefined}
% \DoNotIndex{\@namedef,\@nameuse,\@spaces,\@tempa,\@warning}
% \DoNotIndex{\ }
% \DoNotIndex{\begingroup,\bfseries}
% \DoNotIndex{\csname}
% \DoNotIndex{\def,\do,\docdate}
% \DoNotIndex{\endcsname,\endgroup,\endinput,\endtrivlist,\expandafter}
% \DoNotIndex{\filedate,\fileversion}
% \DoNotIndex{\gdef,\global}
% \DoNotIndex{\hbox,\hskip}
% \DoNotIndex{\ignorespaces,\input,\itshape,\item}
% \DoNotIndex{\labelsep,\let,\llap}
% \DoNotIndex{\makeatletter}
% \DoNotIndex{\noexpand}
% \DoNotIndex{\refstepcounter,\relax,\rlap}
% \DoNotIndex{\setlength,\slshape,\space,\string,\strut}
% \DoNotIndex{\the,\trivlist,\typeout}
% \DoNotIndex{\vbox}
% \DoNotIndex{\xdef}
%
% \def\deflabel#1{\ttfamily #1\hfill}
% \def\deflist#1{\begin{list}{}{\settowidth\labelwidth{\ttfamily #1}%
%                               \setlength\leftmargin\labelwidth
%                               \addtolength\leftmargin\labelsep
%                               \let\makelabel\deflabel}}
% \def\enddeflist{\end{list}}
% \def\envfont{\normalfont\ttfamily}
%
% \MakeShortVerb{\|}
%
% \newcommand{\TUB}{{\sl TUGboat\/}}
%
% \GetFileInfo{theorem.dtx}
% \title{An Extension of the
%        \LaTeX{} theorem environment\thanks{This file
%        has version number \fileversion, last
%        revised \filedate.}}
% \author{Frank Mittelbach}
%
% \MaintainedByLaTeXTeam{tools}
% \maketitle
%
% \begin{abstract}
%    The macros described in this paper yield an extension of the
%    \LaTeX{} theorem mechanism. It is designed to satisfy the
%    different requirements of various journals. Thus, the
%    layout of the ``theorems'' can be manipulated by determining a
%    ``style''. This article describes not only the use, but
%    also the definition, of the necessary macros.
% \end{abstract}
%
% \section*{Preface to version 2.2}
%
% For \LaTeXe{} this package did not need any fundamental changes. I
% only modified the messages generated so that theorem layout styles
% will show up with the |\listfiles| command and cleaned the section
% on the New Font Selection Scheme since this is now included in
% \LaTeX.
%
% \section*{Preface to version 2.1}
%
% This version is identical to 2.0g described in \TUB\ 10\#3 except
% for some internal defaults which are now set depending on the used
% font selection scheme.
%
% This was done to avoid unpleasant surprises if the new font selection
% scheme is in force. For further details see
% section~\ref{sec:newfonts} and \cite{bk:GMS94}.
%
% \section{Introduction}
%
% For our purposes here, ``theorems'' are labelled enunciations,
% often set off from the main text by extra space and a font change.
% Theorems, corollaries, conjectures, definitions, and remarks
% are all instances of ``theorems''.  The ``header'' of these structures
% is composed of a label (such as \textsc{Theorem} or \textsc{Remark})
% and a number which serializes an item in the sequence of items
% with the same label.
%
% Shortly after the introduction of \LaTeX{} at the Fachbereich
% Mathematik in Mainz, the desire to manipulate the layout of
% ``theorems'' arose.  In Mainz, the following two conventions came
% into general use:
% \begin{enumerate}
%   \item The number of the theorem is shown in the margin.
%   \item There is a line break at
%         the end of the theorem header.
% \end{enumerate}
%
% Additionally, some journals require different formats which depend on
% the ``sort of theorem'': e.g.\ often remarks and definitions are set
% in |\upshape|, while |\itshape| is employed for main theorems.
%
% Confronted with these requirements, a theorem environment was
% developed in Mainz which allows separate determination of the
% layout of the ``theorems sets'', comparable to |\pagestyle|.
%
%
% \section{The user interface}
% \subsection{Defining new theorem sets}
%
% \DescribeMacro\newtheorem
% As in the original \LaTeX{} version, the command |\newtheorem|
% defines a new ``theorem set'' or ``theorem-like structure''.
% Two required arguments name the new environment and give the text
% to be typeset with each instance of the new ``set'', while
% an optional argument determines how
% the ``set'' is enumerated:
% \begin{description}
%    \item[\ttfamily \bslash newtheorem\{foo\}\{bar\}]
%       The theorem set {\envfont foo} (whose name is \texttt{bar})
%       uses its own counter.
%    \item[\ttfamily \bslash newtheorem\{foo2\}{[foo]}\{bar2\}]
%       The theorem set {\envfont foo2} (printed name \texttt{bar2})
%       uses the same counter as the theorem set \texttt{foo}.
%    \item[\ttfamily \bslash newtheorem\{foo3\}\{bar3\}{[section]}]
%       The theorem set {\envfont foo3} (printed name \texttt{bar3}) is
%       enumerated within the counter \texttt{section}, i.e.\ with every
%       new |\section| the enumeration begins again with 1, and
%       the enumeration is composed from the section-number and the
%       theorem counter itself.
% \end{description}
%
% \DescribeMacro\theoremstyle
% Additionally, the command |\theoremstyle| can define the layout
% of various, or all, theorem sets.  It should be noted that any
% theorem set defined by |\newtheorem| is typeset in the
% |\theoremstyle| that is current at the time of the definition.
% Thus, the following
% \begin{verbatim}
%    \theoremstyle{break}        \newtheorem{Cor}{Corollary}
%    \theoremstyle{plain}        \newtheorem{Exa}{Example}[section]
%\end{verbatim}                             ^^A No blank at line start
% leads to the result that the set {\envfont Cor} is formatted in the
% style \texttt{break}, while the set {\envfont Exa} and all the
% following ones are formatted in the style \texttt{plain}, unless
% another |\theoremstyle| follows.  Since the definitions installed by
% |\newtheorem| are global, one also can limit |\theoremstyle| locally
% by grouping braces.
%
% \DescribeMacro\theorembodyfont
% The choice of the font for the theorem body is completely
% independent of the chosen |\theoremstyle|; this has proven to
% be very advantageous.  For example,
% \begin{verbatim}
%   {\theorembodyfont{\upshape}          \newtheorem{Rem}{Remark}}
%\end{verbatim}
% defines a theorem set {\envfont Rem}, which will be set in
% |\upshape| in the current layout (which in our example is
% \texttt{plain}).  As with |\theoremstyle|, the |\theorembodyfont|
% chosen is that current at the time of |\newtheorem|.  If
% |\theorembodyfont| is not specified or one defines
% |\theorembodyfont{}|, then the font used will be that defined by the
% |\theoremstyle|.
%
% \DescribeMacro\theoremheaderfont It is also possible to customize
% the font used for the theorem headers.  This is, however, a global
% declaration, and therefore there should be at most one
% |\theoremheaderfont| declaration in the preamble.\footnote{If it is
% actually necessary to have different header fonts, one has to define
% new theorem styles (substituting the desired font) or specify the
% information directly in the \texttt{\bslash newtheorem} declaration
% (the unclean variant).}
%
% \DescribeMacro\theorempreskipamount
% \DescribeMacro\theorempostskipamount Two additional parameters
% affect the vertical space around the theorem environments:
% |\theorempreskipamount| and |\theorempostskipamount| define,
% respectively, the spacing before and after such an environment.
% These parameters apply for all theorem sets and can be manipulated
% with the ordinary length macros.  They are rubber lengths
% (`\textsf{skips}'), and therefore can contain \texttt{plus} and
% \texttt{minus} parts.
%
% Since the definition of theorem sets should---most sensibly---be
% placed in the preamble, we only allow installation there.
% It is therefore possible to release the memory used here after
% |\begin{document}|, in order to make room for other
% applications.
%
%
% \subsection{Existing theorem styles}
%
% The following theorem styles exist to date:
% \begin{deflist}{marginbreak}
%    \item[plain]
%       This theorem style emulates the original \LaTeX{} definition,
%       except that additionally the parameters
%       |\theorem...skipamount| are used.
%    \item[break]
%       In this style, the theorem header is followed by a line
%       break.
%    \item[marginbreak]
%       The theorem number is set in the margin, and there is a line
%       break as in \texttt{break}.
%    \item[changebreak]
%       Like \texttt{break}, but with header number and text
%       interchanged.
%    \item[change]
%       Header number and text are interchanged, without a line break.
%    \item[margin]
%       The number is set in the left margin, without a line break.
% \end{deflist}
% All styles (except \texttt{plain}) select |\slshape| as the default
% |\theorembodyfont|.
%
%
% \subsection{Examples}
%
% Given the above theorem sets {\envfont Cor}, {\envfont Exa}
% and {\envfont Rem},
% suppose that the preamble also contains the declarations:
% \begin{verbatim}
%    \theoremstyle{marginbreak}   \newtheorem{Lem}[Cor]{Lemma}
%    \theoremstyle{change}
%    \theorembodyfont{\itshape}        \newtheorem{Def}[Cor]{Definition}
%
%    \theoremheaderfont{\scshape}
%\end{verbatim}
% Then the following are some typical examples of the typeset output
% resulting from their use.
%
% \begin{Cor}
%    This is a sentence typeset in the theorem environment {\envfont
%    Cor}.
% \end{Cor}
%
% \begin{Exa}
%    This is a sentence typeset in the theorem environment {\envfont
%    Exa}.
% \end{Exa}
%
% \begin{Rem}
%    This is a sentence typeset in the theorem environment {\envfont
%    Rem}.
% \end{Rem}
%
% \begin{Lem}[Ben User]
%    This is a sentence typeset in the theorem environment {\envfont
%    Lem}.
% \end{Lem}
%
% \begin{Def}[Very impressive Definition]
%    This is a sentence typeset in the theorem environment {\envfont
%    Def}.
% \end{Def}
%
% The last two examples show the effect of the optional argument to
% a theorem environment (it is the text typeset in parentheses).
%
%
% \section{Special Considerations}
% \label{sec:newfonts}
%
% Theoremheader and body are implemented as a unit. This means that
% the |\theoremheaderfont|  will inherit characteristics of the
% |\theorembodyfont| in \LaTeXe. Thus, if for example
% |\theorembodyfont| is |\itshape| and |\theoremheaderfont|
% is |\bfseries| the font selected for the header will have the
% characteristics `bold extended italic'. If this is not desired one
% should set the |\theoremheaderfont| to something like
% \begin{verbatim}
%      \theoremheaderfont{\normalfont\bfseries}
%\end{verbatim}
% i.e.\ supplying all necessary font information explicitly.
%
%
%
% \section{Acknowledgements}
%
% The publication of this set of macros was only possible with the
% help of Christina Busse (translating the manuscript into English),
% Joachim Pense (playing the r\^ole of typist), Chris Rowley
% (looking everything over) and many others providing useful
% suggestions.
%
% \MaybeStop{%
%    \begin{thebibliography}{9}
%       \bibitem{bk:GMS94} \textsc{M.~Goossens}, \textsc{F.~Mittelbach}
%         and \textsc{A.~Samarin}.
%         \newblock The \LaTeX{} Companion.
%         \newblock
%         Addison-Wesley, Reading, Massachusetts, 1994.
%       \bibitem{latex:source} \textsc{Lamport, Leslie.} \newblock
%                               \texttt{latex.tex}, version 2.09, date
%                               Feb. 1990.
%    \end{thebibliography}
%    \PrintIndex
%    \PrintChanges}
%
%
% \section{The documentation driver file}
%
% The next bit of code contains the documentation driver file for
% \TeX{}, i.e., the file that will produce the documentation you are
% currently reading. It will be extracted from this file by the
% \texttt{docstrip} program. Since it is the first code in the file
% one can alternatively process this file directly with \LaTeXe{} to
% obtain the documentation.
%
%    \begin{macrocode}
%<*driver>
\documentclass{ltxdoc}

\usepackage{theorem}

%<+driver>% The next few lines define theorem sets which are used
%<+driver>% in the example section of the documentation.

\theoremstyle{break}       \newtheorem{Cor}{Corollary}
\theoremstyle{plain}       \newtheorem{Exa}{Example}[section]
{\theorembodyfont{\upshape}\newtheorem{Rem}{Remark}}
\theoremstyle{marginbreak} \newtheorem{Lem}[Cor]{Lemma}
\theoremstyle{change}
\theorembodyfont{\itshape} \newtheorem{Def}[Cor]{Definition}

\theoremheaderfont{\scshape}

\RecordChanges

\begin{document}
   \DocInput{theorem.dtx}
\end{document}
%</driver>
%    \end{macrocode}
%
% \section{Definition of the Macros}
%
% If the file has been loaded before, we abort immediately. If not the
% package announces itself (this is actually done at the very top of
% the file---the way it is done isn't good style so don't copy it).
% \changes{v2.0e}{89/07/19}{Spaces removed from `typeout'.}
% \changes{v2.2b}{95/11/19}{Announce moved to top of file}
%    \begin{macrocode}
%<*package>
%\@ifundefined{theorem@style}{}{\endinput}
%\def\FMithmInfo{1995/11/19 v2.2b Theorem extension package (FMi)}
%\ProvidesPackage{theorem}[\FMithmInfo]
%    \end{macrocode}
%
%
% \subsection{Definition of theorem styles and fonts}
%
% All the definitions in this file are done globally to allow
% inputting this file inside a group.
%
% \begin{macro}{\theoremstyle}
%    Before a theorem style can be installed, the chosen style must be
%    known. For that reason, we must test to see that
%    |\th@|\meta{style} is known or, more precisely, that it is
%    different from |\relax|. If the style is not known then
%    |\th@plain| is used.
%    \begin{macrocode}
\gdef\theoremstyle#1{%
   \@ifundefined{th@#1}{\@warning
          {Unknown theoremstyle `#1'. Using `plain'}%
          \theorem@style{plain}}%
%    \end{macrocode}
%    We save the theorem style to be used in the token register
%    |\theorem@style|.
%    \begin{macrocode}
      {\theorem@style{#1}}%
%    \end{macrocode}
%    Now we ``evaluate'' the theorem style: this means, we call the
%    macro |\th@|\meta{style} which will activate the relevant
%    definitions which are contained in a separate file.  This is done
%    in a group to suppress changes to the current font.  This could
%    otherwise pose problems together with the new font selection
%    scheme\footnote{When I printed the original article using the new
%    font selection scheme I ended with a document with slanted
%    typefaces (text headings and all) simply because one of the
%    theorem styles used \texttt{\bslash sl} at toplevel.} if the
%    |\th@|\meta{style} is evaluated a second time.
% \changes{v2.1a}{90/02/15}{Group added around `th@..}
% \changes{v2.1a}{90/02/15}{Reseting of `..@begintheorem removed.}
%    \begin{macrocode}
      \begingroup
        \csname th@\the\theorem@style \endcsname
      \endgroup}
%    \end{macrocode}
% \end{macro}
%
%
% \begin{macro}{\@begintheorem}
% \changes{v2.1a}{90/02/15}{Relaxed at toplevel.}
% \begin{macro}{\@opargbegintheorem}
% \changes{v2.1a}{90/02/15}{Relaxed at toplevel.}
%    We reset |\@begintheorem| and |\@opargbegintheorem| to
%    |\relax| since these commands are no longer necessary at
%    toplevel. This will save a few tokens.
%    \begin{macrocode}
\global\let\@begintheorem\relax
\global\let\@opargbegintheorem\relax
%    \end{macrocode}
% \end{macro}
% \end{macro}
%
% \begin{macro}{\theorem@style}
%    Obviously the \textsf{token} register used above has to be
%    allocated. To assure the utmost compatibility with the
%    original \LaTeX{} definition, we set the default theorem style to
%    \texttt{plain}, which implements the usual \LaTeX{} convention.
% \changes{v2.0f}{89/07/20}{Unnecessary `global' removed.}
%    \begin{macrocode}
\newtoks\theorem@style
\global\theorem@style{plain}
%    \end{macrocode}
% \end{macro}
%
% \begin{macro}{\theorembodyfont}
% \begin{macro}{\theorem@bodyfont}
%    If the |\theorembodyfont| is set by the user then it should not
%    interact with the default font set in the theorem style. When the
%    new font selection is in force this may happen if, for example,
%    the default is |\itshape| and the new |\theorembodyfont| is
%    |\sffamily|. So we add a |\reset@font| command in front of the
%    user definition.
% \changes{v2.0f}{89/07/20}{Unnecessary `global' removed.}
% \changes{v2.0g}{89/09/19}{`theoremfont to `theorembodyfont.}
% \changes{v2.1c}{92/06/22}{`reset@font added and logic changed.}
% \changes{v2.1d}{92/10/28}{Reset to empty if arg is empty}
%    \begin{macrocode}
\gdef\theorembodyfont#1{%
%    \end{macrocode}
%    We check if the argument supplied is empty and if so put nothing
%    into the |\theorem@bodyfont| token register to allow for
%    |\theorembodyfont{}| as a mean of using the default of the
%    current |\theoremstyle|.
%    \begin{macrocode}
   \def\@tempa{#1}%
   \ifx\@tempa\@empty
    \theorem@bodyfont{}%
   \else
    \theorem@bodyfont{\reset@font#1}%
   \fi
   }
\newtoks\theorem@bodyfont
\global\theorem@bodyfont{}
%    \end{macrocode}
% \end{macro}
% \end{macro}
%
%
% \begin{macro}{\theoremheaderfont}
% \changes{v2.0g}{89/09/19}{Macro added.}
%    The font for the theorem headers is handled differently because
%    this definition applies to all theorem styles.
%    \begin{macrocode}
\gdef\theoremheaderfont#1{\gdef\theorem@headerfont{#1}%
%    \end{macrocode}
%    After using the macro once it is redefined to produce an error
%    message.
%    \begin{macrocode}
       \gdef\theoremheaderfont##1{%
        \typeout{\string\theoremheaderfont\space should be used
                 only once.}}}
%    \end{macrocode}
% \end{macro}
%
% \begin{macro}{\theorem@headerfont}
% \changes{v2.0g}{89/09/19}{Macro added.}
%    To set the |\theorem@headerfont| default we first test if
%    the new fontselection scheme is in force.
% \changes{v2.1a}{90/02/15}{Support for new font selection.}
%    \begin{macrocode}
\ifx\upshape\undefined
%    \end{macrocode}
%    If not we define it to expand into |\bfseries|. We don't use
%    |\let| just in case a following style option redefines this
%    macro.
%    \begin{macrocode}
\gdef\theorem@headerfont{\bfseries}
%    \end{macrocode}
%    Otherwise we reset the current shape before calling |\bfseries|.
%    \begin{macrocode}
\else \gdef\theorem@headerfont{\normalfont\bfseries}\fi
%    \end{macrocode}
% \end{macro}
%
% \begin{macro}{\th@plain}
% \begin{macro}{\th@break}
% \begin{macro}{\th@marginbreak}
% \begin{macro}{\th@changebreak}
% \begin{macro}{\th@change}
% \begin{macro}{\th@margin}
%    The different styles are defined in macros such as
%    |\th@plain|.  Since memory space is precious in
%    ``non-Big-versions'', we have to avoid offering too many unused
%    definitions.  Therefore we define these styles in separate files
%    that can be loaded on demand.  Thus the commands themselves only
%    load these files. We use |\@input@|, a \LaTeXe{} internal command
%    that ensures that the file will be listed with |\listfiles|.
% \changes{v2.2a}{94/02/02}{Use `@input@ to load theorem layout files}
%    \begin{macrocode}
\gdef\th@plain{\@input@{thp.sty}}
\gdef\th@break{\@input@{thb.sty}}
\gdef\th@marginbreak{\@input@{thmb.sty}}
\gdef\th@changebreak{\@input@{thcb.sty}}
\gdef\th@change{\@input@{thc.sty}}
\gdef\th@margin{\@input@{thm.sty}}
%    \end{macrocode}
%    This list will be expanded when new styles become
%    available.  For testing,  just append new theorem substyles
%    as document options.
% \end{macro}
% \end{macro}
% \end{macro}
% \end{macro}
% \end{macro}
% \end{macro}
%
%
%
% \subsection{Definition of a new theorem set}
%
% As already pointed out, a new theorem environment can be defined in
% three different ways:
% \begin{verbatim}
%    \newtheorem{Lem}{Lemma}
%    \newtheorem{Lem}{Lemma}[section]
%    \newtheorem{Lem}[Theorem]{Lemma}
%\end{verbatim}
% The function of the macro |\newtheorem| is to recognize these cases
% and then to branch into one of the three macros |\@ynthm|, |\@xnthm|
% or |\@othm|.  This mechanism is adopted unchanged from
% \cite{latex:source}; the essential point here is that, for example,
% in the second case, the arguments \texttt{Lem}, \texttt{Lemma} and
% \texttt{section} are passed over to the macro |\@xnthm|.
%
% We inspect this case first because the others present fewer problems,
% and thus are easily derived from this one.
%
% \begin{macro}{\@xnthm}
%    For our example arguments, the macro |\@xnthm| must fulfill
%    the following:
%    \begin{itemize}
%       \item
%          Define a new \LaTeX-counter `Lem'
%       \item
%          reset this counter within a |\section|
%       \item
%          define the macro |\theLem|
%       \item
%          define the environment macros |\Lem| and
%          |\endLem| using the current
%          |\theoremstyle| and |\theorem@bodyfont|.
%    \end{itemize}
%    Obviously, all this should happen only if the first argument
%    of |\@xnthm| (i.e.\ \texttt{Lem} in our example) is chosen so
%    as not to conflict with any previously defined commands or
%    environments.  This test is performed by the \LaTeX{} macro
%    |\@ifdefinable|.
%    \begin{macrocode}
\gdef\@xnthm#1#2[#3]{\expandafter\@ifdefinable\csname #1\endcsname
%    \end{macrocode}
%    Therefore, the first argument of |\@ifdefinable| is the
%    expansion (in the example, |\Lem|) of
%    |\csname||#1\endcsname|.  The second argument is
%    executed only if the test has been completed successfully.
%    \begin{macrocode}
   {%
%    \end{macrocode}
%    Now we define the new counter. The names of the \LaTeX{}
%    macros employed should speak for themselves:
% \changes{v2.2b}{1995/10/07}{Use \cs{@newctr}}
% \changes{v2.2c}{1995/11/23}{Use \cs{@newctr} properly.}
%    \begin{macrocode}
    \@definecounter{#1}\@newctr{#1}[#3]%
%    \end{macrocode}
%    Using |\@newctr| will give a proper error message if the counter
%    in |#3| is not defined.
%    In defining `|\theLem|' we must generate the desired macro
%    name by use of |\expandafter| and |\csname|.
%    \begin{macrocode}
    \expandafter\xdef\csname the#1\endcsname
%    \end{macrocode}
%    An |\xdef| is used in order to make the definition global,
%    and to ensure that it contains the replacement texts of
%    |\@thmcountersep| and |\@thmcounter|.\footnote{These
%    two macros can be defined by the document style.  Their default
%    values produce a `\texttt{.}' as separation and an arabic
%    representation of the number.} However, not everything should be
%    expanded. For example, it saves space to use |\thesection|
%    instead of its---at times---lengthy expansion.
%    \begin{macrocode}
      {\expandafter \noexpand \csname the#3\endcsname
       \@thmcountersep \@thmcounter{#1}}%
%    \end{macrocode}
%    Thus with the defaults of \LaTeX{}, |\theLem| would be
%    replaced by the command sequence |\thesection.\arabic{Lem}|.
%
%    We will now look at the definition of the macro which is executed
%    at the beginning of the actual environment (in our example this
%    macro is |\Lem|). It should be noted that we use an
%    ``|\expandafter| trick'' to expand only certain parts of the
%    replacement text at the time of the definition.
%    \begin{macrocode}
    \def\@tempa{\global\@namedef{#1}}%
    \expandafter \@tempa \expandafter{%
%    \end{macrocode}
%    First, the macro that contains the current definitions of
%    |\@begintheorem| and |\@opargtheorem| should be called
%    up.  The name of this macro---as is already known---has the form
%    |\th@|\meta{theorem\ style}; therefore, it must be called by
%    \begin{macrocode}
      \csname th@\the \theorem@style
%    \end{macrocode}
%    In addition the default theorem font should be changeable, i.e.\
%    we have to insert the contents of |\theorem@bodyfont|. For
%    that reason, we expand even further, beyond |\endcsname|,
%    and thus insert the contents of the \textsf{token} register
%    |\theorem@bodyfont| in the replacement text.
%    \begin{macrocode}
            \expandafter \endcsname \the \theorem@bodyfont
%    \end{macrocode}
%    Now it is time to call the macro |\@thm| which takes over
%    the further processing.  It has two arguments: the current
%    counter name (in our example, \texttt{Lem}), and the text of the
%    label (in our example, \texttt{Lemma}).
%    \begin{macrocode}
     \@thm{#1}{#2}}%
%    \end{macrocode}
%    With this, the `sub-definition' is complete.  The macro
%    |\@endtheorem| ends a theorem environment and is, so far, nothing
%    but an |\endtrivlist|.  (Hence it is defined globally, and not
%    within the theorem styles.\footnote{This has to be changed as
%    soon as theorem styles that change \texttt{\bslash @endtheorem}
%    exist.  In such a case, all existing styles must be changed as
%    well since they will have to reset the macro.}) Therefore, we can
%    set it equivalent to the macro that ends the theorem set (in our
%    example, |\endLem|).  However, if some day theorem styles exist
%    that do change |\@endtheorem|, we would have to use the
%    commented-out line instead.
%    \begin{macrocode}
    \global \expandafter \let \csname end#1\endcsname \@endtheorem
%   \global\@namedef{end#1}{\@endtheorem}%
%    \end{macrocode}
%    With these commands all the required definitions are employed,
%    unless the test |\@ifdefinable| has failed.  Therefore, we
%    end the second argument of this macro and with it the definition
%    of |\@xnthm|.
%    \begin{macrocode}
   }}
%    \end{macrocode}
% \end{macro}
%
% \begin{macro}{\@ynthm}
%    The definition of |\@ynthm| is completely analogous. In
%    this case the new counter that is defined is not reset within
%    another counter; thus the definition of |\the...|
%    is simplified:
%    \begin{macrocode}
\gdef\@ynthm#1#2{\expandafter\@ifdefinable\csname #1\endcsname
   {\@definecounter{#1}%
    \expandafter\xdef\csname the#1\endcsname{\@thmcounter{#1}}%
%    \end{macrocode}
%    The rest of the definition corresponds literally to that of
%    |\@xnthm|:
%    \begin{macrocode}
    \def\@tempa{\global\@namedef{#1}}\expandafter \@tempa
     \expandafter{\csname th@\the \theorem@style \expandafter
     \endcsname \the\theorem@bodyfont \@thm{#1}{#2}}%
    \global \expandafter \let \csname end#1\endcsname \@endtheorem}}
%    \end{macrocode}
% \end{macro}
%
% \begin{macro}{\@othm}
%    The definition of |\@othm| does not contain anything new.
%    \begin{macrocode}
\gdef\@othm#1[#2]#3{%
%    \end{macrocode}
% \changes{v2.2b}{95/10/07}{Build in check for counter defined}
%    We do not define a new counter but instead use one that has
%    already been defined.  Thus the only definition we need is that
%    of this pseudo-counter (i.e.\ |\the|\meta{env.\ name}).
%    First we check if |#2| corresponds to a known counter name.
%    \begin{macrocode}
  \expandafter\ifx\csname c@#2\endcsname\relax
   \@nocounterr{#2}%
  \else
   \expandafter\@ifdefinable\csname #1\endcsname
   {\expandafter \xdef \csname the#1\endcsname
     {\expandafter \noexpand \csname the#2\endcsname}%
%    \end{macrocode}
%    All other parts of the definition can be adopted from
%    |\@xnthm|. We have to remember, though, that in this case
%    the name of the current counter and the theorem label have moved
%    to the second and third arguments.
%    \begin{macrocode}
    \def\@tempa{\global\@namedef{#1}}\expandafter \@tempa
     \expandafter{\csname th@\the \theorem@style \expandafter
     \endcsname \the\theorem@bodyfont \@thm{#2}{#3}}%
    \global \expandafter \let \csname end#1\endcsname \@endtheorem}%
  \fi}
%    \end{macrocode}
% \end{macro}
%
% \subsection{Macros that are employed in a {\envfont theorem}
%    environment}
%
% \begin{macro}{\@thm}
%    The macro |\@thm| has to increase the current counter.  Then,
%    depending on whether the environment has (or does not have) an
%    optional argument, it has to branch into either |\@begintheorem|
%    or |\@opargtheorem|.
%    \begin{macrocode}
\gdef\@thm#1#2{\refstepcounter{#1}%
%    \end{macrocode}
%    Now we start a {\envfont trivlist} environment, and give
%    |\@topsep| and |\@topsepadd| the values of the \textsf{skip}
%    registers |\theorempreskipamount| and |\theorempostskipamount|.
%    The value in |\@topsep| is the vertical space that is inserted by
%    the first (and only) |\item| in our |\trivlist| whilst
%    |\@topsepadd| is inserted by |\@endparenv| at the end of that
%    {\envfont trivlist} environment.  By using these registers, we
%    obtain the desired space around a {\envfont theorem} environment.
%    \begin{macrocode}
   \trivlist
   \@topsep \theorempreskipamount               % used by first \item
   \@topsepadd \theorempostskipamount           % used by \@endparenv
%    \end{macrocode}
%    Now we have to test whether an optional argument has been given.
% \changes{v2.0d}{89/07/18}{Forgotten percent added.}
%    \begin{macrocode}
   \@ifnextchar [%
%    \end{macrocode}
%    If there is an optional argument, we will call |\@ythm|, and
%    move the arguments read back into the input stream.
%    \begin{macrocode}
   {\@ythm{#1}{#2}}%
%    \end{macrocode}
%    If not, we call |\@begintheorem|. Its first argument is the
%    name of the theorem set (hence the second argument of
%    |\@thm|). Its second argument is the macro that produces the
%    current number.
%    \begin{macrocode}
   {\@begintheorem{#2}{\csname the#1\endcsname}\ignorespaces}}
%    \end{macrocode}
% \end{macro}
%
% \begin{macro}{\@xthm}
% \begin{macro}{\@ythm}
%    Both these macros were originally called by |\@thm|.  We do
%    not need |\@xthm| anymore, hence we reset it to
%    |\relax|. The definition of |\@ythm| has not changed at
%    all from its definition in \LaTeX. In order to make the macros
%    easier to understand, we will
%    nevertheless present it (commented out).
%    \begin{macrocode}
\global\let\@xthm\relax
% \def\@ythm#1#2[#3]{\@opargbegintheorem{#2}{\csname
%           the#1\endcsname}{#3}\ignorespaces}
%</package>
%    \end{macrocode}
%    The primitive |\ignorespaces| in |\@ythm| and
%    |\@thm| is needed to remove the spaces between the
%    |\begin{...}| and the actual text.
% \end{macro}
% \end{macro}
%
%
% \subsection{Definition of the theorem substyles}
%
% As already pointed out, the theorem substyles, defined below, are
% only loaded when necessary.  Note that all these substyles, except
% \texttt{plain}, have |\slshape| as the default body font.
%
% \subsubsection{The \texttt{plain} style}
%
% \label{dok:th@plain} As the following macros use |@|, we have
% to locally set the |\catcode| of this symbol to ``letter''.
% This happens within a group, so that we do not have to worry about
% which |\catcode| that symbol had before.
%    \begin{macrocode}
%<*thp>
\begingroup \makeatletter
%    \end{macrocode}
% Since we are now within a group, we must make all definitions
% globally.  First we make sure that \texttt{theorem.sty} is loaded.
% This will allow us to use this file as a document style option
% without having to call \texttt{theorem} itself as an option.  At the
% same time, we assure that at least version 2 is loaded, since
% |\theorem@style| was not defined in earlier versions.
%    \begin{macrocode}
\@ifundefined{theorem@style}{% \iffalse meta-comment
%
% Copyright (C) 1993-2024
%
% The LaTeX Project and any individual authors listed elsewhere
% in this file.
%
% This file is part of the Standard LaTeX `Tools Bundle'.
% -------------------------------------------------------
%
% It may be distributed and/or modified under the
% conditions of the LaTeX Project Public License, either version 1.3c
% of this license or (at your option) any later version.
% The latest version of this license is in
%    https://www.latex-project.org/lppl.txt
% and version 1.3c or later is part of all distributions of LaTeX
% version 2005/12/01 or later.
%
% The list of all files belonging to the LaTeX `Tools Bundle' is
% given in the file `manifest.txt'.
%
% \fi
%
%%
%
%
\def\FMithmInfo{2023/07/05 v2.2c Theorem extension package (FMi)}
%
% \ProvidesFile{theorem.dtx}[\FMithmInfo]
%\iffalse   % this is a METACOMMENT !
%%
%% Package `theorem' to use with LaTeX2e
%% Copyright (C) 1989-1995 Frank Mittelbach, all rights reserved.
%<+package>\@ifundefined{theorem@style}{}{\endinput}
%<+package>\NeedsTeXFormat{LaTeX2e}[1995/06/01]
%<+package>\ProvidesPackage{theorem}
%<+package>         [\FMithmInfo]
%
% \fi
%
% \changes{v1.0}{87/05/07}{First draft.}
% \changes{v1.0a}{87/05/07}{`thistheoremstyle' added.}
% \changes{v1.1a}{87/07/29}{New concept: redefine the internal
%                          \protect\LaTeX{} commands `@xnthm' etc.}
% \changes{v1.1b}{87/07/31}{`theoremfont' used to specify font.}
% \changes{v1.2a}{88/01/08}{`@endtheorem' now globally defined.}
% \changes{v1.2b}{88/01/11}{`newtheorem' only allowed in the preamble.}
% \changes{v1.3a}{88/02/14}{`theoremfont' now a toks register.}
%
% \changes{v2.0a}{88/11/16}{New implementation and new documentation.}
% \changes{v2.0a}{88/11/16}{No layout changes but internal modifications
%                           old substyles won't work.}
% \changes{v2.0b}{89/01/30}{`global' used for all definitions.}
% \changes{v2.0c}{89/06/15}{Documentation translated to English.
%                            Thanks to LLL and Nina}
% \changes{v2.1a}{90/02/15}{Support for new font selection.}
% \changes{v2.2a}{94/02/03}{Cleanup for LaTeX2e}
%
% \DoNotIndex{\@endparenv,\@ifdefinable,\@ifnextchar,\@ifundefined}
% \DoNotIndex{\@namedef,\@nameuse,\@spaces,\@tempa,\@warning}
% \DoNotIndex{\ }
% \DoNotIndex{\begingroup,\bfseries}
% \DoNotIndex{\csname}
% \DoNotIndex{\def,\do,\docdate}
% \DoNotIndex{\endcsname,\endgroup,\endinput,\endtrivlist,\expandafter}
% \DoNotIndex{\filedate,\fileversion}
% \DoNotIndex{\gdef,\global}
% \DoNotIndex{\hbox,\hskip}
% \DoNotIndex{\ignorespaces,\input,\itshape,\item}
% \DoNotIndex{\labelsep,\let,\llap}
% \DoNotIndex{\makeatletter}
% \DoNotIndex{\noexpand}
% \DoNotIndex{\refstepcounter,\relax,\rlap}
% \DoNotIndex{\setlength,\slshape,\space,\string,\strut}
% \DoNotIndex{\the,\trivlist,\typeout}
% \DoNotIndex{\vbox}
% \DoNotIndex{\xdef}
%
% \def\deflabel#1{\ttfamily #1\hfill}
% \def\deflist#1{\begin{list}{}{\settowidth\labelwidth{\ttfamily #1}%
%                               \setlength\leftmargin\labelwidth
%                               \addtolength\leftmargin\labelsep
%                               \let\makelabel\deflabel}}
% \def\enddeflist{\end{list}}
% \def\envfont{\normalfont\ttfamily}
%
% \MakeShortVerb{\|}
%
% \newcommand{\TUB}{{\sl TUGboat\/}}
%
% \GetFileInfo{theorem.dtx}
% \title{An Extension of the
%        \LaTeX{} theorem environment\thanks{This file
%        has version number \fileversion, last
%        revised \filedate.}}
% \author{Frank Mittelbach}
%
% \MaintainedByLaTeXTeam{tools}
% \maketitle
%
% \begin{abstract}
%    The macros described in this paper yield an extension of the
%    \LaTeX{} theorem mechanism. It is designed to satisfy the
%    different requirements of various journals. Thus, the
%    layout of the ``theorems'' can be manipulated by determining a
%    ``style''. This article describes not only the use, but
%    also the definition, of the necessary macros.
% \end{abstract}
%
% \section*{Preface to version 2.2}
%
% For \LaTeXe{} this package did not need any fundamental changes. I
% only modified the messages generated so that theorem layout styles
% will show up with the |\listfiles| command and cleaned the section
% on the New Font Selection Scheme since this is now included in
% \LaTeX.
%
% \section*{Preface to version 2.1}
%
% This version is identical to 2.0g described in \TUB\ 10\#3 except
% for some internal defaults which are now set depending on the used
% font selection scheme.
%
% This was done to avoid unpleasant surprises if the new font selection
% scheme is in force. For further details see
% section~\ref{sec:newfonts} and \cite{bk:GMS94}.
%
% \section{Introduction}
%
% For our purposes here, ``theorems'' are labelled enunciations,
% often set off from the main text by extra space and a font change.
% Theorems, corollaries, conjectures, definitions, and remarks
% are all instances of ``theorems''.  The ``header'' of these structures
% is composed of a label (such as \textsc{Theorem} or \textsc{Remark})
% and a number which serializes an item in the sequence of items
% with the same label.
%
% Shortly after the introduction of \LaTeX{} at the Fachbereich
% Mathematik in Mainz, the desire to manipulate the layout of
% ``theorems'' arose.  In Mainz, the following two conventions came
% into general use:
% \begin{enumerate}
%   \item The number of the theorem is shown in the margin.
%   \item There is a line break at
%         the end of the theorem header.
% \end{enumerate}
%
% Additionally, some journals require different formats which depend on
% the ``sort of theorem'': e.g.\ often remarks and definitions are set
% in |\upshape|, while |\itshape| is employed for main theorems.
%
% Confronted with these requirements, a theorem environment was
% developed in Mainz which allows separate determination of the
% layout of the ``theorems sets'', comparable to |\pagestyle|.
%
%
% \section{The user interface}
% \subsection{Defining new theorem sets}
%
% \DescribeMacro\newtheorem
% As in the original \LaTeX{} version, the command |\newtheorem|
% defines a new ``theorem set'' or ``theorem-like structure''.
% Two required arguments name the new environment and give the text
% to be typeset with each instance of the new ``set'', while
% an optional argument determines how
% the ``set'' is enumerated:
% \begin{description}
%    \item[\ttfamily \bslash newtheorem\{foo\}\{bar\}]
%       The theorem set {\envfont foo} (whose name is \texttt{bar})
%       uses its own counter.
%    \item[\ttfamily \bslash newtheorem\{foo2\}{[foo]}\{bar2\}]
%       The theorem set {\envfont foo2} (printed name \texttt{bar2})
%       uses the same counter as the theorem set \texttt{foo}.
%    \item[\ttfamily \bslash newtheorem\{foo3\}\{bar3\}{[section]}]
%       The theorem set {\envfont foo3} (printed name \texttt{bar3}) is
%       enumerated within the counter \texttt{section}, i.e.\ with every
%       new |\section| the enumeration begins again with 1, and
%       the enumeration is composed from the section-number and the
%       theorem counter itself.
% \end{description}
%
% \DescribeMacro\theoremstyle
% Additionally, the command |\theoremstyle| can define the layout
% of various, or all, theorem sets.  It should be noted that any
% theorem set defined by |\newtheorem| is typeset in the
% |\theoremstyle| that is current at the time of the definition.
% Thus, the following
% \begin{verbatim}
%    \theoremstyle{break}        \newtheorem{Cor}{Corollary}
%    \theoremstyle{plain}        \newtheorem{Exa}{Example}[section]
%\end{verbatim}                             ^^A No blank at line start
% leads to the result that the set {\envfont Cor} is formatted in the
% style \texttt{break}, while the set {\envfont Exa} and all the
% following ones are formatted in the style \texttt{plain}, unless
% another |\theoremstyle| follows.  Since the definitions installed by
% |\newtheorem| are global, one also can limit |\theoremstyle| locally
% by grouping braces.
%
% \DescribeMacro\theorembodyfont
% The choice of the font for the theorem body is completely
% independent of the chosen |\theoremstyle|; this has proven to
% be very advantageous.  For example,
% \begin{verbatim}
%   {\theorembodyfont{\upshape}          \newtheorem{Rem}{Remark}}
%\end{verbatim}
% defines a theorem set {\envfont Rem}, which will be set in
% |\upshape| in the current layout (which in our example is
% \texttt{plain}).  As with |\theoremstyle|, the |\theorembodyfont|
% chosen is that current at the time of |\newtheorem|.  If
% |\theorembodyfont| is not specified or one defines
% |\theorembodyfont{}|, then the font used will be that defined by the
% |\theoremstyle|.
%
% \DescribeMacro\theoremheaderfont It is also possible to customize
% the font used for the theorem headers.  This is, however, a global
% declaration, and therefore there should be at most one
% |\theoremheaderfont| declaration in the preamble.\footnote{If it is
% actually necessary to have different header fonts, one has to define
% new theorem styles (substituting the desired font) or specify the
% information directly in the \texttt{\bslash newtheorem} declaration
% (the unclean variant).}
%
% \DescribeMacro\theorempreskipamount
% \DescribeMacro\theorempostskipamount Two additional parameters
% affect the vertical space around the theorem environments:
% |\theorempreskipamount| and |\theorempostskipamount| define,
% respectively, the spacing before and after such an environment.
% These parameters apply for all theorem sets and can be manipulated
% with the ordinary length macros.  They are rubber lengths
% (`\textsf{skips}'), and therefore can contain \texttt{plus} and
% \texttt{minus} parts.
%
% Since the definition of theorem sets should---most sensibly---be
% placed in the preamble, we only allow installation there.
% It is therefore possible to release the memory used here after
% |\begin{document}|, in order to make room for other
% applications.
%
%
% \subsection{Existing theorem styles}
%
% The following theorem styles exist to date:
% \begin{deflist}{marginbreak}
%    \item[plain]
%       This theorem style emulates the original \LaTeX{} definition,
%       except that additionally the parameters
%       |\theorem...skipamount| are used.
%    \item[break]
%       In this style, the theorem header is followed by a line
%       break.
%    \item[marginbreak]
%       The theorem number is set in the margin, and there is a line
%       break as in \texttt{break}.
%    \item[changebreak]
%       Like \texttt{break}, but with header number and text
%       interchanged.
%    \item[change]
%       Header number and text are interchanged, without a line break.
%    \item[margin]
%       The number is set in the left margin, without a line break.
% \end{deflist}
% All styles (except \texttt{plain}) select |\slshape| as the default
% |\theorembodyfont|.
%
%
% \subsection{Examples}
%
% Given the above theorem sets {\envfont Cor}, {\envfont Exa}
% and {\envfont Rem},
% suppose that the preamble also contains the declarations:
% \begin{verbatim}
%    \theoremstyle{marginbreak}   \newtheorem{Lem}[Cor]{Lemma}
%    \theoremstyle{change}
%    \theorembodyfont{\itshape}        \newtheorem{Def}[Cor]{Definition}
%
%    \theoremheaderfont{\scshape}
%\end{verbatim}
% Then the following are some typical examples of the typeset output
% resulting from their use.
%
% \begin{Cor}
%    This is a sentence typeset in the theorem environment {\envfont
%    Cor}.
% \end{Cor}
%
% \begin{Exa}
%    This is a sentence typeset in the theorem environment {\envfont
%    Exa}.
% \end{Exa}
%
% \begin{Rem}
%    This is a sentence typeset in the theorem environment {\envfont
%    Rem}.
% \end{Rem}
%
% \begin{Lem}[Ben User]
%    This is a sentence typeset in the theorem environment {\envfont
%    Lem}.
% \end{Lem}
%
% \begin{Def}[Very impressive Definition]
%    This is a sentence typeset in the theorem environment {\envfont
%    Def}.
% \end{Def}
%
% The last two examples show the effect of the optional argument to
% a theorem environment (it is the text typeset in parentheses).
%
%
% \section{Special Considerations}
% \label{sec:newfonts}
%
% Theoremheader and body are implemented as a unit. This means that
% the |\theoremheaderfont|  will inherit characteristics of the
% |\theorembodyfont| in \LaTeXe. Thus, if for example
% |\theorembodyfont| is |\itshape| and |\theoremheaderfont|
% is |\bfseries| the font selected for the header will have the
% characteristics `bold extended italic'. If this is not desired one
% should set the |\theoremheaderfont| to something like
% \begin{verbatim}
%      \theoremheaderfont{\normalfont\bfseries}
%\end{verbatim}
% i.e.\ supplying all necessary font information explicitly.
%
%
%
% \section{Acknowledgements}
%
% The publication of this set of macros was only possible with the
% help of Christina Busse (translating the manuscript into English),
% Joachim Pense (playing the r\^ole of typist), Chris Rowley
% (looking everything over) and many others providing useful
% suggestions.
%
% \MaybeStop{%
%    \begin{thebibliography}{9}
%       \bibitem{bk:GMS94} \textsc{M.~Goossens}, \textsc{F.~Mittelbach}
%         and \textsc{A.~Samarin}.
%         \newblock The \LaTeX{} Companion.
%         \newblock
%         Addison-Wesley, Reading, Massachusetts, 1994.
%       \bibitem{latex:source} \textsc{Lamport, Leslie.} \newblock
%                               \texttt{latex.tex}, version 2.09, date
%                               Feb. 1990.
%    \end{thebibliography}
%    \PrintIndex
%    \PrintChanges}
%
%
% \section{The documentation driver file}
%
% The next bit of code contains the documentation driver file for
% \TeX{}, i.e., the file that will produce the documentation you are
% currently reading. It will be extracted from this file by the
% \texttt{docstrip} program. Since it is the first code in the file
% one can alternatively process this file directly with \LaTeXe{} to
% obtain the documentation.
%
%    \begin{macrocode}
%<*driver>
\documentclass{ltxdoc}

\usepackage{theorem}

%<+driver>% The next few lines define theorem sets which are used
%<+driver>% in the example section of the documentation.

\theoremstyle{break}       \newtheorem{Cor}{Corollary}
\theoremstyle{plain}       \newtheorem{Exa}{Example}[section]
{\theorembodyfont{\upshape}\newtheorem{Rem}{Remark}}
\theoremstyle{marginbreak} \newtheorem{Lem}[Cor]{Lemma}
\theoremstyle{change}
\theorembodyfont{\itshape} \newtheorem{Def}[Cor]{Definition}

\theoremheaderfont{\scshape}

\RecordChanges

\begin{document}
   \DocInput{theorem.dtx}
\end{document}
%</driver>
%    \end{macrocode}
%
% \section{Definition of the Macros}
%
% If the file has been loaded before, we abort immediately. If not the
% package announces itself (this is actually done at the very top of
% the file---the way it is done isn't good style so don't copy it).
% \changes{v2.0e}{89/07/19}{Spaces removed from `typeout'.}
% \changes{v2.2b}{95/11/19}{Announce moved to top of file}
%    \begin{macrocode}
%<*package>
%\@ifundefined{theorem@style}{}{\endinput}
%\def\FMithmInfo{1995/11/19 v2.2b Theorem extension package (FMi)}
%\ProvidesPackage{theorem}[\FMithmInfo]
%    \end{macrocode}
%
%
% \subsection{Definition of theorem styles and fonts}
%
% All the definitions in this file are done globally to allow
% inputting this file inside a group.
%
% \begin{macro}{\theoremstyle}
%    Before a theorem style can be installed, the chosen style must be
%    known. For that reason, we must test to see that
%    |\th@|\meta{style} is known or, more precisely, that it is
%    different from |\relax|. If the style is not known then
%    |\th@plain| is used.
%    \begin{macrocode}
\gdef\theoremstyle#1{%
   \@ifundefined{th@#1}{\@warning
          {Unknown theoremstyle `#1'. Using `plain'}%
          \theorem@style{plain}}%
%    \end{macrocode}
%    We save the theorem style to be used in the token register
%    |\theorem@style|.
%    \begin{macrocode}
      {\theorem@style{#1}}%
%    \end{macrocode}
%    Now we ``evaluate'' the theorem style: this means, we call the
%    macro |\th@|\meta{style} which will activate the relevant
%    definitions which are contained in a separate file.  This is done
%    in a group to suppress changes to the current font.  This could
%    otherwise pose problems together with the new font selection
%    scheme\footnote{When I printed the original article using the new
%    font selection scheme I ended with a document with slanted
%    typefaces (text headings and all) simply because one of the
%    theorem styles used \texttt{\bslash sl} at toplevel.} if the
%    |\th@|\meta{style} is evaluated a second time.
% \changes{v2.1a}{90/02/15}{Group added around `th@..}
% \changes{v2.1a}{90/02/15}{Reseting of `..@begintheorem removed.}
%    \begin{macrocode}
      \begingroup
        \csname th@\the\theorem@style \endcsname
      \endgroup}
%    \end{macrocode}
% \end{macro}
%
%
% \begin{macro}{\@begintheorem}
% \changes{v2.1a}{90/02/15}{Relaxed at toplevel.}
% \begin{macro}{\@opargbegintheorem}
% \changes{v2.1a}{90/02/15}{Relaxed at toplevel.}
%    We reset |\@begintheorem| and |\@opargbegintheorem| to
%    |\relax| since these commands are no longer necessary at
%    toplevel. This will save a few tokens.
%    \begin{macrocode}
\global\let\@begintheorem\relax
\global\let\@opargbegintheorem\relax
%    \end{macrocode}
% \end{macro}
% \end{macro}
%
% \begin{macro}{\theorem@style}
%    Obviously the \textsf{token} register used above has to be
%    allocated. To assure the utmost compatibility with the
%    original \LaTeX{} definition, we set the default theorem style to
%    \texttt{plain}, which implements the usual \LaTeX{} convention.
% \changes{v2.0f}{89/07/20}{Unnecessary `global' removed.}
%    \begin{macrocode}
\newtoks\theorem@style
\global\theorem@style{plain}
%    \end{macrocode}
% \end{macro}
%
% \begin{macro}{\theorembodyfont}
% \begin{macro}{\theorem@bodyfont}
%    If the |\theorembodyfont| is set by the user then it should not
%    interact with the default font set in the theorem style. When the
%    new font selection is in force this may happen if, for example,
%    the default is |\itshape| and the new |\theorembodyfont| is
%    |\sffamily|. So we add a |\reset@font| command in front of the
%    user definition.
% \changes{v2.0f}{89/07/20}{Unnecessary `global' removed.}
% \changes{v2.0g}{89/09/19}{`theoremfont to `theorembodyfont.}
% \changes{v2.1c}{92/06/22}{`reset@font added and logic changed.}
% \changes{v2.1d}{92/10/28}{Reset to empty if arg is empty}
%    \begin{macrocode}
\gdef\theorembodyfont#1{%
%    \end{macrocode}
%    We check if the argument supplied is empty and if so put nothing
%    into the |\theorem@bodyfont| token register to allow for
%    |\theorembodyfont{}| as a mean of using the default of the
%    current |\theoremstyle|.
%    \begin{macrocode}
   \def\@tempa{#1}%
   \ifx\@tempa\@empty
    \theorem@bodyfont{}%
   \else
    \theorem@bodyfont{\reset@font#1}%
   \fi
   }
\newtoks\theorem@bodyfont
\global\theorem@bodyfont{}
%    \end{macrocode}
% \end{macro}
% \end{macro}
%
%
% \begin{macro}{\theoremheaderfont}
% \changes{v2.0g}{89/09/19}{Macro added.}
%    The font for the theorem headers is handled differently because
%    this definition applies to all theorem styles.
%    \begin{macrocode}
\gdef\theoremheaderfont#1{\gdef\theorem@headerfont{#1}%
%    \end{macrocode}
%    After using the macro once it is redefined to produce an error
%    message.
%    \begin{macrocode}
       \gdef\theoremheaderfont##1{%
        \typeout{\string\theoremheaderfont\space should be used
                 only once.}}}
%    \end{macrocode}
% \end{macro}
%
% \begin{macro}{\theorem@headerfont}
% \changes{v2.0g}{89/09/19}{Macro added.}
%    To set the |\theorem@headerfont| default we first test if
%    the new fontselection scheme is in force.
% \changes{v2.1a}{90/02/15}{Support for new font selection.}
%    \begin{macrocode}
\ifx\upshape\undefined
%    \end{macrocode}
%    If not we define it to expand into |\bfseries|. We don't use
%    |\let| just in case a following style option redefines this
%    macro.
%    \begin{macrocode}
\gdef\theorem@headerfont{\bfseries}
%    \end{macrocode}
%    Otherwise we reset the current shape before calling |\bfseries|.
%    \begin{macrocode}
\else \gdef\theorem@headerfont{\normalfont\bfseries}\fi
%    \end{macrocode}
% \end{macro}
%
% \begin{macro}{\th@plain}
% \begin{macro}{\th@break}
% \begin{macro}{\th@marginbreak}
% \begin{macro}{\th@changebreak}
% \begin{macro}{\th@change}
% \begin{macro}{\th@margin}
%    The different styles are defined in macros such as
%    |\th@plain|.  Since memory space is precious in
%    ``non-Big-versions'', we have to avoid offering too many unused
%    definitions.  Therefore we define these styles in separate files
%    that can be loaded on demand.  Thus the commands themselves only
%    load these files. We use |\@input@|, a \LaTeXe{} internal command
%    that ensures that the file will be listed with |\listfiles|.
% \changes{v2.2a}{94/02/02}{Use `@input@ to load theorem layout files}
%    \begin{macrocode}
\gdef\th@plain{\@input@{thp.sty}}
\gdef\th@break{\@input@{thb.sty}}
\gdef\th@marginbreak{\@input@{thmb.sty}}
\gdef\th@changebreak{\@input@{thcb.sty}}
\gdef\th@change{\@input@{thc.sty}}
\gdef\th@margin{\@input@{thm.sty}}
%    \end{macrocode}
%    This list will be expanded when new styles become
%    available.  For testing,  just append new theorem substyles
%    as document options.
% \end{macro}
% \end{macro}
% \end{macro}
% \end{macro}
% \end{macro}
% \end{macro}
%
%
%
% \subsection{Definition of a new theorem set}
%
% As already pointed out, a new theorem environment can be defined in
% three different ways:
% \begin{verbatim}
%    \newtheorem{Lem}{Lemma}
%    \newtheorem{Lem}{Lemma}[section]
%    \newtheorem{Lem}[Theorem]{Lemma}
%\end{verbatim}
% The function of the macro |\newtheorem| is to recognize these cases
% and then to branch into one of the three macros |\@ynthm|, |\@xnthm|
% or |\@othm|.  This mechanism is adopted unchanged from
% \cite{latex:source}; the essential point here is that, for example,
% in the second case, the arguments \texttt{Lem}, \texttt{Lemma} and
% \texttt{section} are passed over to the macro |\@xnthm|.
%
% We inspect this case first because the others present fewer problems,
% and thus are easily derived from this one.
%
% \begin{macro}{\@xnthm}
%    For our example arguments, the macro |\@xnthm| must fulfill
%    the following:
%    \begin{itemize}
%       \item
%          Define a new \LaTeX-counter `Lem'
%       \item
%          reset this counter within a |\section|
%       \item
%          define the macro |\theLem|
%       \item
%          define the environment macros |\Lem| and
%          |\endLem| using the current
%          |\theoremstyle| and |\theorem@bodyfont|.
%    \end{itemize}
%    Obviously, all this should happen only if the first argument
%    of |\@xnthm| (i.e.\ \texttt{Lem} in our example) is chosen so
%    as not to conflict with any previously defined commands or
%    environments.  This test is performed by the \LaTeX{} macro
%    |\@ifdefinable|.
%    \begin{macrocode}
\gdef\@xnthm#1#2[#3]{\expandafter\@ifdefinable\csname #1\endcsname
%    \end{macrocode}
%    Therefore, the first argument of |\@ifdefinable| is the
%    expansion (in the example, |\Lem|) of
%    |\csname||#1\endcsname|.  The second argument is
%    executed only if the test has been completed successfully.
%    \begin{macrocode}
   {%
%    \end{macrocode}
%    Now we define the new counter. The names of the \LaTeX{}
%    macros employed should speak for themselves:
% \changes{v2.2b}{1995/10/07}{Use \cs{@newctr}}
% \changes{v2.2c}{1995/11/23}{Use \cs{@newctr} properly.}
%    \begin{macrocode}
    \@definecounter{#1}\@newctr{#1}[#3]%
%    \end{macrocode}
%    Using |\@newctr| will give a proper error message if the counter
%    in |#3| is not defined.
%    In defining `|\theLem|' we must generate the desired macro
%    name by use of |\expandafter| and |\csname|.
%    \begin{macrocode}
    \expandafter\xdef\csname the#1\endcsname
%    \end{macrocode}
%    An |\xdef| is used in order to make the definition global,
%    and to ensure that it contains the replacement texts of
%    |\@thmcountersep| and |\@thmcounter|.\footnote{These
%    two macros can be defined by the document style.  Their default
%    values produce a `\texttt{.}' as separation and an arabic
%    representation of the number.} However, not everything should be
%    expanded. For example, it saves space to use |\thesection|
%    instead of its---at times---lengthy expansion.
%    \begin{macrocode}
      {\expandafter \noexpand \csname the#3\endcsname
       \@thmcountersep \@thmcounter{#1}}%
%    \end{macrocode}
%    Thus with the defaults of \LaTeX{}, |\theLem| would be
%    replaced by the command sequence |\thesection.\arabic{Lem}|.
%
%    We will now look at the definition of the macro which is executed
%    at the beginning of the actual environment (in our example this
%    macro is |\Lem|). It should be noted that we use an
%    ``|\expandafter| trick'' to expand only certain parts of the
%    replacement text at the time of the definition.
%    \begin{macrocode}
    \def\@tempa{\global\@namedef{#1}}%
    \expandafter \@tempa \expandafter{%
%    \end{macrocode}
%    First, the macro that contains the current definitions of
%    |\@begintheorem| and |\@opargtheorem| should be called
%    up.  The name of this macro---as is already known---has the form
%    |\th@|\meta{theorem\ style}; therefore, it must be called by
%    \begin{macrocode}
      \csname th@\the \theorem@style
%    \end{macrocode}
%    In addition the default theorem font should be changeable, i.e.\
%    we have to insert the contents of |\theorem@bodyfont|. For
%    that reason, we expand even further, beyond |\endcsname|,
%    and thus insert the contents of the \textsf{token} register
%    |\theorem@bodyfont| in the replacement text.
%    \begin{macrocode}
            \expandafter \endcsname \the \theorem@bodyfont
%    \end{macrocode}
%    Now it is time to call the macro |\@thm| which takes over
%    the further processing.  It has two arguments: the current
%    counter name (in our example, \texttt{Lem}), and the text of the
%    label (in our example, \texttt{Lemma}).
%    \begin{macrocode}
     \@thm{#1}{#2}}%
%    \end{macrocode}
%    With this, the `sub-definition' is complete.  The macro
%    |\@endtheorem| ends a theorem environment and is, so far, nothing
%    but an |\endtrivlist|.  (Hence it is defined globally, and not
%    within the theorem styles.\footnote{This has to be changed as
%    soon as theorem styles that change \texttt{\bslash @endtheorem}
%    exist.  In such a case, all existing styles must be changed as
%    well since they will have to reset the macro.}) Therefore, we can
%    set it equivalent to the macro that ends the theorem set (in our
%    example, |\endLem|).  However, if some day theorem styles exist
%    that do change |\@endtheorem|, we would have to use the
%    commented-out line instead.
%    \begin{macrocode}
    \global \expandafter \let \csname end#1\endcsname \@endtheorem
%   \global\@namedef{end#1}{\@endtheorem}%
%    \end{macrocode}
%    With these commands all the required definitions are employed,
%    unless the test |\@ifdefinable| has failed.  Therefore, we
%    end the second argument of this macro and with it the definition
%    of |\@xnthm|.
%    \begin{macrocode}
   }}
%    \end{macrocode}
% \end{macro}
%
% \begin{macro}{\@ynthm}
%    The definition of |\@ynthm| is completely analogous. In
%    this case the new counter that is defined is not reset within
%    another counter; thus the definition of |\the...|
%    is simplified:
%    \begin{macrocode}
\gdef\@ynthm#1#2{\expandafter\@ifdefinable\csname #1\endcsname
   {\@definecounter{#1}%
    \expandafter\xdef\csname the#1\endcsname{\@thmcounter{#1}}%
%    \end{macrocode}
%    The rest of the definition corresponds literally to that of
%    |\@xnthm|:
%    \begin{macrocode}
    \def\@tempa{\global\@namedef{#1}}\expandafter \@tempa
     \expandafter{\csname th@\the \theorem@style \expandafter
     \endcsname \the\theorem@bodyfont \@thm{#1}{#2}}%
    \global \expandafter \let \csname end#1\endcsname \@endtheorem}}
%    \end{macrocode}
% \end{macro}
%
% \begin{macro}{\@othm}
%    The definition of |\@othm| does not contain anything new.
%    \begin{macrocode}
\gdef\@othm#1[#2]#3{%
%    \end{macrocode}
% \changes{v2.2b}{95/10/07}{Build in check for counter defined}
%    We do not define a new counter but instead use one that has
%    already been defined.  Thus the only definition we need is that
%    of this pseudo-counter (i.e.\ |\the|\meta{env.\ name}).
%    First we check if |#2| corresponds to a known counter name.
%    \begin{macrocode}
  \expandafter\ifx\csname c@#2\endcsname\relax
   \@nocounterr{#2}%
  \else
   \expandafter\@ifdefinable\csname #1\endcsname
   {\expandafter \xdef \csname the#1\endcsname
     {\expandafter \noexpand \csname the#2\endcsname}%
%    \end{macrocode}
%    All other parts of the definition can be adopted from
%    |\@xnthm|. We have to remember, though, that in this case
%    the name of the current counter and the theorem label have moved
%    to the second and third arguments.
%    \begin{macrocode}
    \def\@tempa{\global\@namedef{#1}}\expandafter \@tempa
     \expandafter{\csname th@\the \theorem@style \expandafter
     \endcsname \the\theorem@bodyfont \@thm{#2}{#3}}%
    \global \expandafter \let \csname end#1\endcsname \@endtheorem}%
  \fi}
%    \end{macrocode}
% \end{macro}
%
% \subsection{Macros that are employed in a {\envfont theorem}
%    environment}
%
% \begin{macro}{\@thm}
%    The macro |\@thm| has to increase the current counter.  Then,
%    depending on whether the environment has (or does not have) an
%    optional argument, it has to branch into either |\@begintheorem|
%    or |\@opargtheorem|.
%    \begin{macrocode}
\gdef\@thm#1#2{\refstepcounter{#1}%
%    \end{macrocode}
%    Now we start a {\envfont trivlist} environment, and give
%    |\@topsep| and |\@topsepadd| the values of the \textsf{skip}
%    registers |\theorempreskipamount| and |\theorempostskipamount|.
%    The value in |\@topsep| is the vertical space that is inserted by
%    the first (and only) |\item| in our |\trivlist| whilst
%    |\@topsepadd| is inserted by |\@endparenv| at the end of that
%    {\envfont trivlist} environment.  By using these registers, we
%    obtain the desired space around a {\envfont theorem} environment.
%    \begin{macrocode}
   \trivlist
   \@topsep \theorempreskipamount               % used by first \item
   \@topsepadd \theorempostskipamount           % used by \@endparenv
%    \end{macrocode}
%    Now we have to test whether an optional argument has been given.
% \changes{v2.0d}{89/07/18}{Forgotten percent added.}
%    \begin{macrocode}
   \@ifnextchar [%
%    \end{macrocode}
%    If there is an optional argument, we will call |\@ythm|, and
%    move the arguments read back into the input stream.
%    \begin{macrocode}
   {\@ythm{#1}{#2}}%
%    \end{macrocode}
%    If not, we call |\@begintheorem|. Its first argument is the
%    name of the theorem set (hence the second argument of
%    |\@thm|). Its second argument is the macro that produces the
%    current number.
%    \begin{macrocode}
   {\@begintheorem{#2}{\csname the#1\endcsname}\ignorespaces}}
%    \end{macrocode}
% \end{macro}
%
% \begin{macro}{\@xthm}
% \begin{macro}{\@ythm}
%    Both these macros were originally called by |\@thm|.  We do
%    not need |\@xthm| anymore, hence we reset it to
%    |\relax|. The definition of |\@ythm| has not changed at
%    all from its definition in \LaTeX. In order to make the macros
%    easier to understand, we will
%    nevertheless present it (commented out).
%    \begin{macrocode}
\global\let\@xthm\relax
% \def\@ythm#1#2[#3]{\@opargbegintheorem{#2}{\csname
%           the#1\endcsname}{#3}\ignorespaces}
%</package>
%    \end{macrocode}
%    The primitive |\ignorespaces| in |\@ythm| and
%    |\@thm| is needed to remove the spaces between the
%    |\begin{...}| and the actual text.
% \end{macro}
% \end{macro}
%
%
% \subsection{Definition of the theorem substyles}
%
% As already pointed out, the theorem substyles, defined below, are
% only loaded when necessary.  Note that all these substyles, except
% \texttt{plain}, have |\slshape| as the default body font.
%
% \subsubsection{The \texttt{plain} style}
%
% \label{dok:th@plain} As the following macros use |@|, we have
% to locally set the |\catcode| of this symbol to ``letter''.
% This happens within a group, so that we do not have to worry about
% which |\catcode| that symbol had before.
%    \begin{macrocode}
%<*thp>
\begingroup \makeatletter
%    \end{macrocode}
% Since we are now within a group, we must make all definitions
% globally.  First we make sure that \texttt{theorem.sty} is loaded.
% This will allow us to use this file as a document style option
% without having to call \texttt{theorem} itself as an option.  At the
% same time, we assure that at least version 2 is loaded, since
% |\theorem@style| was not defined in earlier versions.
%    \begin{macrocode}
\@ifundefined{theorem@style}{\input{theorem.sty}}{}
\ProvidesFile{thp.sty}
             [\FMithmInfo]
%    \end{macrocode}
%
%
% \begin{macro}{\th@plain}
% \begin{macro}{\@begintheorem}
% \begin{macro}{\@opargbegintheorem}
%    |\theoremstyle{plain}| corresponds to the original
%    definition, except that the distances to the surrounding text are
%    determined by the parameters |\theorempreskipamount| and
%    |\theorempostskipamount|.
%    First we set the default body font.
%    \begin{macrocode}
\gdef\th@plain{\normalfont\itshape
%    \end{macrocode}
%    Then we define |\@begintheorem| and
%    |\@opargbegintheorem|. These two macros define how the
%    header of a theorem is typeset.  |\@opargbegintheorem| will
%    be called if a \textsf{theorem} environment with an optional
%    argument is encountered; otherwise, the header is constructed by
%    calling |\@begintheorem|.  If one of these macros is
%    executed, we are within a \textsf{trivlist} environment started by
%    |\@thm|.  So the theorem header is produced with an
%    |\item| command.
%
%    Instead of specifying the header font directly, all standard
%    theorem styles use the |\theorem@headerfont| macro to allow
%    customization.  The extra space (|\labelsep|) is necessary
%    because of problems in the \textsf{trivlist} environment.
%    \begin{macrocode}
  \def\@begintheorem##1##2{%
        \item[\hskip\labelsep \theorem@headerfont ##1\ ##2]}%
%    \end{macrocode}
%    The definition of |\@opargbegintheorem| is completely
%    analogous. The only difference is the fact that there exists a
%    third argument (which is the optional parameter of the
%    environment and contains additional information about the
%    theorem).  Customarily we enclose it in parentheses.
%    \begin{macrocode}
\def\@opargbegintheorem##1##2##3{%
   \item[\hskip\labelsep \theorem@headerfont ##1\ ##2\ (##3)]}}
%    \end{macrocode}
% \end{macro}
% \end{macro}
% \end{macro}
%
% \noindent
% We conclude with an |\endgroup| to restore the
% |\catcode| of |@|.
%    \begin{macrocode}
\endgroup
%</thp>
%    \end{macrocode}
%
%
% \subsubsection{The \texttt{break} style}
%
% This style option is stored in the file |thb.sty|.
% For the next two lines see the documentation for |\th@plain| on
% page~\pageref{dok:th@plain}.
%    \begin{macrocode}
%<*thb>
\begingroup \makeatletter
\@ifundefined{theorem@style}{\input{theorem.sty}}{}
\ProvidesFile{thb.sty}
             [\FMithmInfo]
%    \end{macrocode}
%
% \begin{macro}{\th@break}
%    |\theoremstyle{break}| produces a line break after the name
%    of the theorem.  The font is |\slshape|.  Hence, we define
%    |\th@break| as follows:
%    \begin{macrocode}
\gdef\th@break{\normalfont\slshape
  \def\@begintheorem##1##2{\item[%
%    \end{macrocode}
%    We run into the following problem: it is not possible to create
%    the header with |\item[|\meta{title}|]| and then start
%    a new line by, for example, |\mbox{}\\|. Such a definition
%    will fail whenever a list environment follows immediately.  With
%    the above construction, the |\mbox{}| causes the switch
%    |@inlabel| (cf.\ definition of |\list| and
%    |\trivlist| in \cite{latex:source}) to be set to \texttt{false}
%    and so the following list will insert additional vertical
%    space (|\topskip|). This is quite annoying. Therefore, we
%    create the line break within the |\item|. In order to ensure
%    that the text will begin at the proper position in the following
%    line, we simply pretend that the label does not take any
%    room.\footnote{This will lead to problems whenever very high
%    symbols occurring in the line tower into the
%    heading. So, something else has to be done here sometime.}
%    \begin{macrocode}
       \rlap{\vbox{\hbox{\hskip \labelsep\theorem@headerfont ##1\ ##2}%
                   \hbox{\strut}}}]}%
%    \end{macrocode}
%    Again, the definition of |\@opargbegintheorem| is
%    completely analogous.
%    \begin{macrocode}
\def\@opargbegintheorem##1##2##3{%
  \item[\rlap{\vbox{\hbox{\hskip \labelsep \theorem@headerfont
                     ##1\ ##2\ (##3)}%
                    \hbox{\strut}}}]}}
\endgroup
%</thb>
%    \end{macrocode}
% \end{macro}
%
%
% \subsubsection{The \texttt{changebreak} style}
%
%    \begin{macrocode}
% This style option is stored in the file |thcb.sty|.
%    \begin{macrocode}
%<*thcb>
\begingroup \makeatletter
\@ifundefined{theorem@style}{\input{theorem.sty}}{}
\ProvidesFile{thcb.sty}
             [\FMithmInfo]
%    \end{macrocode}
%
% \begin{macro}{\th@changebreak}
%    The \texttt{change-break} theorem style is like \texttt{break}
%    but with interchange of theorem name and theorem number.  Thus we
%    define |\th@changebreak| as follows:
%    \begin{macrocode}
\gdef\th@changebreak{\normalfont\slshape
  \def\@begintheorem##1##2{\item
       [\rlap{\vbox{\hbox{\hskip\labelsep \theorem@headerfont ##2\ ##1}%
                    \hbox{\strut}}}]}%
\def\@opargbegintheorem##1##2##3{%
  \item[\rlap{\vbox{\hbox{\hskip\labelsep \theorem@headerfont
                     ##2\ ##1\ (##3)}%
                    \hbox{\strut}}}]}}
\endgroup
%</thcb>
%    \end{macrocode}
% \end{macro}
%
%
% \subsubsection{The \texttt{change} style}
%
% This style option is stored in the file |thc.sty|.
%    \begin{macrocode}
%<*thc>
\begingroup \makeatletter
\@ifundefined{theorem@style}{\input{theorem.sty}}{}
\ProvidesFile{thc.sty}
             [\FMithmInfo]
%    \end{macrocode}
%
% \begin{macro}{\th@change}
%    The \texttt{change} theorem style corresponds to the
%    \texttt{change break} style without a linebreak after the header.
%    To say it in another way, it's the same as the \texttt{plain}
%    style but with number and name interchanged and |\slshape| as the
%    default font.
%    \begin{macrocode}
\gdef\th@change{\normalfont\slshape
  \def\@begintheorem##1##2{\item
       [\hskip\labelsep \theorem@headerfont ##2\ ##1]}%
\def\@opargbegintheorem##1##2##3{%
  \item[\hskip\labelsep \theorem@headerfont ##2\ ##1\ (##3)]}}
\endgroup
%</thc>
%    \end{macrocode}
% \end{macro}
%
%
% \subsubsection{The \texttt{marginbreak} style}
%
% This style option is the one used most often at Mainz.  It is saved
% in the file |thmb.sty|.
%    \begin{macrocode}
%<*thmb>
\begingroup \makeatletter
\@ifundefined{theorem@style}{\input{theorem.sty}}{}
\ProvidesFile{thmb.sty}
             [\FMithmInfo]
%    \end{macrocode}
%
% \begin{macro}{\th@marginbreak}
%    The \texttt{margin break} style is nearly the same as the
%    \texttt{change break} style. The only difference is the placement
%    of the theorem number.  We use |\llap| to place it in the left
%    margin.
%
%    In this style |\labelsep| denotes the separation between the
%    number and the text.
%    \begin{macrocode}
\gdef\th@marginbreak{\normalfont\slshape
  \def\@begintheorem##1##2{\item
       [\rlap{\vbox{\theorem@headerfont
                    \hbox{\llap{##2}\hskip\labelsep ##1}%
                    \hbox{\strut}}}]}%
\def\@opargbegintheorem##1##2##3{%
  \item[\rlap{\vbox{\theorem@headerfont
                    \hbox{\llap{##2}\hskip\labelsep ##1\ (##3)}%
                    \hbox{\strut}}}]}}
\endgroup
%</thmb>
%    \end{macrocode}
% \changes{v2.1c}{90/02/17}{`endgroup outside macrocode env.}
% \end{macro}
%
%
%
% \subsubsection{The \texttt{margin} style}
%
% This style option is stored in the file |thm.sty|.
%    \begin{macrocode}
%<*thm>
\begingroup \makeatletter
\@ifundefined{theorem@style}{\input{theorem.sty}}{}
\ProvidesFile{thm.sty}
             [\FMithmInfo]
%    \end{macrocode}
%
% \begin{macro}{\th@margin}
%    Again this is only a variant of the theorem styles described
%    above without any new ideas.
%    \begin{macrocode}
\gdef\th@margin{\normalfont\slshape
  \def\@begintheorem##1##2{\item
       [\theorem@headerfont \llap{##2}\hskip\labelsep ##1]}%
\def\@opargbegintheorem##1##2##3{%
  \item[\theorem@headerfont \llap{##2}\hskip\labelsep ##1\ (##3)]}}
\endgroup
%</thm>
%    \end{macrocode}
% \end{macro}
%
%
%
%
% \subsection{Final Definitions}
%
% \begin{macro}{\theorempreskipamount}
% \begin{macro}{\theorempostskipamount}
%    The \textsf{skip} parameters that regulate the vertical empty space
%    before and after the {\envfont theorem} environment have
%    to be allocated as well.
% \changes{v2.0f}{89/07/20}{Unnecessary `global' removed.}
%    \begin{macrocode}
%<*package>
\newskip\theorempreskipamount
\newskip\theorempostskipamount
%    \end{macrocode}
%    Since we have used the same values for all theorem sets, we now can
%    assign them.
%    \begin{macrocode}
\global\setlength\theorempreskipamount{12pt plus 5pt minus 3pt}
\global\setlength\theorempostskipamount{8pt plus 3pt minus 1.5pt}
%    \end{macrocode}
% \end{macro}
% \end{macro}
%
%
% \begin{macro}{\@endtheorem}
%    The same holds for the macro |\@endtheorem|, which ends a
%    {\envfont theorem} environment. Since it is the same for all
%    theorem sets, it is removed from the macros |\th@|\meta{style}.
%    It simply ends the {\envfont trivlist} environment, which was
%    begun in |\@thm|.
%    \begin{macrocode}
\global\let\@endtheorem=\endtrivlist
%    \end{macrocode}
% \end{macro}
%
%
%
% \begin{macro}{\@preamblecmds}
%    All macros defined above are to be used only in the preamble.
%    Therefore, we insert them in |\@preamblecmds| which will disable
%    them at begin document. This is done by the internal \LaTeXe{}
%    command |\@onlypreamble|.
% \changes{v2.0g}{89/09/19}{`theoremheaderfont and `theorembodyfont
%    added.}
% \changes{v2.2a}{94/02/03}{Use \LaTeXe{} interface}
%    \begin{macrocode}
\@onlypreamble\@xnthm
\@onlypreamble\@ynthm
\@onlypreamble\@othm
\@onlypreamble\newtheorem
\@onlypreamble\theoremstyle
\@onlypreamble\theorembodyfont
\@onlypreamble\theoremheaderfont
%    \end{macrocode}
% \end{macro}
%
% Finally we declare the \texttt{plain} theorem style to be the default.
% \changes{v2.1b}{90/06/04}{Fixed missing `item by declaring default
%    style.}
%    \begin{macrocode}
\theoremstyle{plain}
%</package>
%    \end{macrocode}
%
% \Finale

\endinput
}{}
\ProvidesFile{thp.sty}
             [\FMithmInfo]
%    \end{macrocode}
%
%
% \begin{macro}{\th@plain}
% \begin{macro}{\@begintheorem}
% \begin{macro}{\@opargbegintheorem}
%    |\theoremstyle{plain}| corresponds to the original
%    definition, except that the distances to the surrounding text are
%    determined by the parameters |\theorempreskipamount| and
%    |\theorempostskipamount|.
%    First we set the default body font.
%    \begin{macrocode}
\gdef\th@plain{\normalfont\itshape
%    \end{macrocode}
%    Then we define |\@begintheorem| and
%    |\@opargbegintheorem|. These two macros define how the
%    header of a theorem is typeset.  |\@opargbegintheorem| will
%    be called if a \textsf{theorem} environment with an optional
%    argument is encountered; otherwise, the header is constructed by
%    calling |\@begintheorem|.  If one of these macros is
%    executed, we are within a \textsf{trivlist} environment started by
%    |\@thm|.  So the theorem header is produced with an
%    |\item| command.
%
%    Instead of specifying the header font directly, all standard
%    theorem styles use the |\theorem@headerfont| macro to allow
%    customization.  The extra space (|\labelsep|) is necessary
%    because of problems in the \textsf{trivlist} environment.
%    \begin{macrocode}
  \def\@begintheorem##1##2{%
        \item[\hskip\labelsep \theorem@headerfont ##1\ ##2]}%
%    \end{macrocode}
%    The definition of |\@opargbegintheorem| is completely
%    analogous. The only difference is the fact that there exists a
%    third argument (which is the optional parameter of the
%    environment and contains additional information about the
%    theorem).  Customarily we enclose it in parentheses.
%    \begin{macrocode}
\def\@opargbegintheorem##1##2##3{%
   \item[\hskip\labelsep \theorem@headerfont ##1\ ##2\ (##3)]}}
%    \end{macrocode}
% \end{macro}
% \end{macro}
% \end{macro}
%
% \noindent
% We conclude with an |\endgroup| to restore the
% |\catcode| of |@|.
%    \begin{macrocode}
\endgroup
%</thp>
%    \end{macrocode}
%
%
% \subsubsection{The \texttt{break} style}
%
% This style option is stored in the file |thb.sty|.
% For the next two lines see the documentation for |\th@plain| on
% page~\pageref{dok:th@plain}.
%    \begin{macrocode}
%<*thb>
\begingroup \makeatletter
\@ifundefined{theorem@style}{% \iffalse meta-comment
%
% Copyright (C) 1993-2024
%
% The LaTeX Project and any individual authors listed elsewhere
% in this file.
%
% This file is part of the Standard LaTeX `Tools Bundle'.
% -------------------------------------------------------
%
% It may be distributed and/or modified under the
% conditions of the LaTeX Project Public License, either version 1.3c
% of this license or (at your option) any later version.
% The latest version of this license is in
%    https://www.latex-project.org/lppl.txt
% and version 1.3c or later is part of all distributions of LaTeX
% version 2005/12/01 or later.
%
% The list of all files belonging to the LaTeX `Tools Bundle' is
% given in the file `manifest.txt'.
%
% \fi
%
%%
%
%
\def\FMithmInfo{2023/07/05 v2.2c Theorem extension package (FMi)}
%
% \ProvidesFile{theorem.dtx}[\FMithmInfo]
%\iffalse   % this is a METACOMMENT !
%%
%% Package `theorem' to use with LaTeX2e
%% Copyright (C) 1989-1995 Frank Mittelbach, all rights reserved.
%<+package>\@ifundefined{theorem@style}{}{\endinput}
%<+package>\NeedsTeXFormat{LaTeX2e}[1995/06/01]
%<+package>\ProvidesPackage{theorem}
%<+package>         [\FMithmInfo]
%
% \fi
%
% \changes{v1.0}{87/05/07}{First draft.}
% \changes{v1.0a}{87/05/07}{`thistheoremstyle' added.}
% \changes{v1.1a}{87/07/29}{New concept: redefine the internal
%                          \protect\LaTeX{} commands `@xnthm' etc.}
% \changes{v1.1b}{87/07/31}{`theoremfont' used to specify font.}
% \changes{v1.2a}{88/01/08}{`@endtheorem' now globally defined.}
% \changes{v1.2b}{88/01/11}{`newtheorem' only allowed in the preamble.}
% \changes{v1.3a}{88/02/14}{`theoremfont' now a toks register.}
%
% \changes{v2.0a}{88/11/16}{New implementation and new documentation.}
% \changes{v2.0a}{88/11/16}{No layout changes but internal modifications
%                           old substyles won't work.}
% \changes{v2.0b}{89/01/30}{`global' used for all definitions.}
% \changes{v2.0c}{89/06/15}{Documentation translated to English.
%                            Thanks to LLL and Nina}
% \changes{v2.1a}{90/02/15}{Support for new font selection.}
% \changes{v2.2a}{94/02/03}{Cleanup for LaTeX2e}
%
% \DoNotIndex{\@endparenv,\@ifdefinable,\@ifnextchar,\@ifundefined}
% \DoNotIndex{\@namedef,\@nameuse,\@spaces,\@tempa,\@warning}
% \DoNotIndex{\ }
% \DoNotIndex{\begingroup,\bfseries}
% \DoNotIndex{\csname}
% \DoNotIndex{\def,\do,\docdate}
% \DoNotIndex{\endcsname,\endgroup,\endinput,\endtrivlist,\expandafter}
% \DoNotIndex{\filedate,\fileversion}
% \DoNotIndex{\gdef,\global}
% \DoNotIndex{\hbox,\hskip}
% \DoNotIndex{\ignorespaces,\input,\itshape,\item}
% \DoNotIndex{\labelsep,\let,\llap}
% \DoNotIndex{\makeatletter}
% \DoNotIndex{\noexpand}
% \DoNotIndex{\refstepcounter,\relax,\rlap}
% \DoNotIndex{\setlength,\slshape,\space,\string,\strut}
% \DoNotIndex{\the,\trivlist,\typeout}
% \DoNotIndex{\vbox}
% \DoNotIndex{\xdef}
%
% \def\deflabel#1{\ttfamily #1\hfill}
% \def\deflist#1{\begin{list}{}{\settowidth\labelwidth{\ttfamily #1}%
%                               \setlength\leftmargin\labelwidth
%                               \addtolength\leftmargin\labelsep
%                               \let\makelabel\deflabel}}
% \def\enddeflist{\end{list}}
% \def\envfont{\normalfont\ttfamily}
%
% \MakeShortVerb{\|}
%
% \newcommand{\TUB}{{\sl TUGboat\/}}
%
% \GetFileInfo{theorem.dtx}
% \title{An Extension of the
%        \LaTeX{} theorem environment\thanks{This file
%        has version number \fileversion, last
%        revised \filedate.}}
% \author{Frank Mittelbach}
%
% \MaintainedByLaTeXTeam{tools}
% \maketitle
%
% \begin{abstract}
%    The macros described in this paper yield an extension of the
%    \LaTeX{} theorem mechanism. It is designed to satisfy the
%    different requirements of various journals. Thus, the
%    layout of the ``theorems'' can be manipulated by determining a
%    ``style''. This article describes not only the use, but
%    also the definition, of the necessary macros.
% \end{abstract}
%
% \section*{Preface to version 2.2}
%
% For \LaTeXe{} this package did not need any fundamental changes. I
% only modified the messages generated so that theorem layout styles
% will show up with the |\listfiles| command and cleaned the section
% on the New Font Selection Scheme since this is now included in
% \LaTeX.
%
% \section*{Preface to version 2.1}
%
% This version is identical to 2.0g described in \TUB\ 10\#3 except
% for some internal defaults which are now set depending on the used
% font selection scheme.
%
% This was done to avoid unpleasant surprises if the new font selection
% scheme is in force. For further details see
% section~\ref{sec:newfonts} and \cite{bk:GMS94}.
%
% \section{Introduction}
%
% For our purposes here, ``theorems'' are labelled enunciations,
% often set off from the main text by extra space and a font change.
% Theorems, corollaries, conjectures, definitions, and remarks
% are all instances of ``theorems''.  The ``header'' of these structures
% is composed of a label (such as \textsc{Theorem} or \textsc{Remark})
% and a number which serializes an item in the sequence of items
% with the same label.
%
% Shortly after the introduction of \LaTeX{} at the Fachbereich
% Mathematik in Mainz, the desire to manipulate the layout of
% ``theorems'' arose.  In Mainz, the following two conventions came
% into general use:
% \begin{enumerate}
%   \item The number of the theorem is shown in the margin.
%   \item There is a line break at
%         the end of the theorem header.
% \end{enumerate}
%
% Additionally, some journals require different formats which depend on
% the ``sort of theorem'': e.g.\ often remarks and definitions are set
% in |\upshape|, while |\itshape| is employed for main theorems.
%
% Confronted with these requirements, a theorem environment was
% developed in Mainz which allows separate determination of the
% layout of the ``theorems sets'', comparable to |\pagestyle|.
%
%
% \section{The user interface}
% \subsection{Defining new theorem sets}
%
% \DescribeMacro\newtheorem
% As in the original \LaTeX{} version, the command |\newtheorem|
% defines a new ``theorem set'' or ``theorem-like structure''.
% Two required arguments name the new environment and give the text
% to be typeset with each instance of the new ``set'', while
% an optional argument determines how
% the ``set'' is enumerated:
% \begin{description}
%    \item[\ttfamily \bslash newtheorem\{foo\}\{bar\}]
%       The theorem set {\envfont foo} (whose name is \texttt{bar})
%       uses its own counter.
%    \item[\ttfamily \bslash newtheorem\{foo2\}{[foo]}\{bar2\}]
%       The theorem set {\envfont foo2} (printed name \texttt{bar2})
%       uses the same counter as the theorem set \texttt{foo}.
%    \item[\ttfamily \bslash newtheorem\{foo3\}\{bar3\}{[section]}]
%       The theorem set {\envfont foo3} (printed name \texttt{bar3}) is
%       enumerated within the counter \texttt{section}, i.e.\ with every
%       new |\section| the enumeration begins again with 1, and
%       the enumeration is composed from the section-number and the
%       theorem counter itself.
% \end{description}
%
% \DescribeMacro\theoremstyle
% Additionally, the command |\theoremstyle| can define the layout
% of various, or all, theorem sets.  It should be noted that any
% theorem set defined by |\newtheorem| is typeset in the
% |\theoremstyle| that is current at the time of the definition.
% Thus, the following
% \begin{verbatim}
%    \theoremstyle{break}        \newtheorem{Cor}{Corollary}
%    \theoremstyle{plain}        \newtheorem{Exa}{Example}[section]
%\end{verbatim}                             ^^A No blank at line start
% leads to the result that the set {\envfont Cor} is formatted in the
% style \texttt{break}, while the set {\envfont Exa} and all the
% following ones are formatted in the style \texttt{plain}, unless
% another |\theoremstyle| follows.  Since the definitions installed by
% |\newtheorem| are global, one also can limit |\theoremstyle| locally
% by grouping braces.
%
% \DescribeMacro\theorembodyfont
% The choice of the font for the theorem body is completely
% independent of the chosen |\theoremstyle|; this has proven to
% be very advantageous.  For example,
% \begin{verbatim}
%   {\theorembodyfont{\upshape}          \newtheorem{Rem}{Remark}}
%\end{verbatim}
% defines a theorem set {\envfont Rem}, which will be set in
% |\upshape| in the current layout (which in our example is
% \texttt{plain}).  As with |\theoremstyle|, the |\theorembodyfont|
% chosen is that current at the time of |\newtheorem|.  If
% |\theorembodyfont| is not specified or one defines
% |\theorembodyfont{}|, then the font used will be that defined by the
% |\theoremstyle|.
%
% \DescribeMacro\theoremheaderfont It is also possible to customize
% the font used for the theorem headers.  This is, however, a global
% declaration, and therefore there should be at most one
% |\theoremheaderfont| declaration in the preamble.\footnote{If it is
% actually necessary to have different header fonts, one has to define
% new theorem styles (substituting the desired font) or specify the
% information directly in the \texttt{\bslash newtheorem} declaration
% (the unclean variant).}
%
% \DescribeMacro\theorempreskipamount
% \DescribeMacro\theorempostskipamount Two additional parameters
% affect the vertical space around the theorem environments:
% |\theorempreskipamount| and |\theorempostskipamount| define,
% respectively, the spacing before and after such an environment.
% These parameters apply for all theorem sets and can be manipulated
% with the ordinary length macros.  They are rubber lengths
% (`\textsf{skips}'), and therefore can contain \texttt{plus} and
% \texttt{minus} parts.
%
% Since the definition of theorem sets should---most sensibly---be
% placed in the preamble, we only allow installation there.
% It is therefore possible to release the memory used here after
% |\begin{document}|, in order to make room for other
% applications.
%
%
% \subsection{Existing theorem styles}
%
% The following theorem styles exist to date:
% \begin{deflist}{marginbreak}
%    \item[plain]
%       This theorem style emulates the original \LaTeX{} definition,
%       except that additionally the parameters
%       |\theorem...skipamount| are used.
%    \item[break]
%       In this style, the theorem header is followed by a line
%       break.
%    \item[marginbreak]
%       The theorem number is set in the margin, and there is a line
%       break as in \texttt{break}.
%    \item[changebreak]
%       Like \texttt{break}, but with header number and text
%       interchanged.
%    \item[change]
%       Header number and text are interchanged, without a line break.
%    \item[margin]
%       The number is set in the left margin, without a line break.
% \end{deflist}
% All styles (except \texttt{plain}) select |\slshape| as the default
% |\theorembodyfont|.
%
%
% \subsection{Examples}
%
% Given the above theorem sets {\envfont Cor}, {\envfont Exa}
% and {\envfont Rem},
% suppose that the preamble also contains the declarations:
% \begin{verbatim}
%    \theoremstyle{marginbreak}   \newtheorem{Lem}[Cor]{Lemma}
%    \theoremstyle{change}
%    \theorembodyfont{\itshape}        \newtheorem{Def}[Cor]{Definition}
%
%    \theoremheaderfont{\scshape}
%\end{verbatim}
% Then the following are some typical examples of the typeset output
% resulting from their use.
%
% \begin{Cor}
%    This is a sentence typeset in the theorem environment {\envfont
%    Cor}.
% \end{Cor}
%
% \begin{Exa}
%    This is a sentence typeset in the theorem environment {\envfont
%    Exa}.
% \end{Exa}
%
% \begin{Rem}
%    This is a sentence typeset in the theorem environment {\envfont
%    Rem}.
% \end{Rem}
%
% \begin{Lem}[Ben User]
%    This is a sentence typeset in the theorem environment {\envfont
%    Lem}.
% \end{Lem}
%
% \begin{Def}[Very impressive Definition]
%    This is a sentence typeset in the theorem environment {\envfont
%    Def}.
% \end{Def}
%
% The last two examples show the effect of the optional argument to
% a theorem environment (it is the text typeset in parentheses).
%
%
% \section{Special Considerations}
% \label{sec:newfonts}
%
% Theoremheader and body are implemented as a unit. This means that
% the |\theoremheaderfont|  will inherit characteristics of the
% |\theorembodyfont| in \LaTeXe. Thus, if for example
% |\theorembodyfont| is |\itshape| and |\theoremheaderfont|
% is |\bfseries| the font selected for the header will have the
% characteristics `bold extended italic'. If this is not desired one
% should set the |\theoremheaderfont| to something like
% \begin{verbatim}
%      \theoremheaderfont{\normalfont\bfseries}
%\end{verbatim}
% i.e.\ supplying all necessary font information explicitly.
%
%
%
% \section{Acknowledgements}
%
% The publication of this set of macros was only possible with the
% help of Christina Busse (translating the manuscript into English),
% Joachim Pense (playing the r\^ole of typist), Chris Rowley
% (looking everything over) and many others providing useful
% suggestions.
%
% \MaybeStop{%
%    \begin{thebibliography}{9}
%       \bibitem{bk:GMS94} \textsc{M.~Goossens}, \textsc{F.~Mittelbach}
%         and \textsc{A.~Samarin}.
%         \newblock The \LaTeX{} Companion.
%         \newblock
%         Addison-Wesley, Reading, Massachusetts, 1994.
%       \bibitem{latex:source} \textsc{Lamport, Leslie.} \newblock
%                               \texttt{latex.tex}, version 2.09, date
%                               Feb. 1990.
%    \end{thebibliography}
%    \PrintIndex
%    \PrintChanges}
%
%
% \section{The documentation driver file}
%
% The next bit of code contains the documentation driver file for
% \TeX{}, i.e., the file that will produce the documentation you are
% currently reading. It will be extracted from this file by the
% \texttt{docstrip} program. Since it is the first code in the file
% one can alternatively process this file directly with \LaTeXe{} to
% obtain the documentation.
%
%    \begin{macrocode}
%<*driver>
\documentclass{ltxdoc}

\usepackage{theorem}

%<+driver>% The next few lines define theorem sets which are used
%<+driver>% in the example section of the documentation.

\theoremstyle{break}       \newtheorem{Cor}{Corollary}
\theoremstyle{plain}       \newtheorem{Exa}{Example}[section]
{\theorembodyfont{\upshape}\newtheorem{Rem}{Remark}}
\theoremstyle{marginbreak} \newtheorem{Lem}[Cor]{Lemma}
\theoremstyle{change}
\theorembodyfont{\itshape} \newtheorem{Def}[Cor]{Definition}

\theoremheaderfont{\scshape}

\RecordChanges

\begin{document}
   \DocInput{theorem.dtx}
\end{document}
%</driver>
%    \end{macrocode}
%
% \section{Definition of the Macros}
%
% If the file has been loaded before, we abort immediately. If not the
% package announces itself (this is actually done at the very top of
% the file---the way it is done isn't good style so don't copy it).
% \changes{v2.0e}{89/07/19}{Spaces removed from `typeout'.}
% \changes{v2.2b}{95/11/19}{Announce moved to top of file}
%    \begin{macrocode}
%<*package>
%\@ifundefined{theorem@style}{}{\endinput}
%\def\FMithmInfo{1995/11/19 v2.2b Theorem extension package (FMi)}
%\ProvidesPackage{theorem}[\FMithmInfo]
%    \end{macrocode}
%
%
% \subsection{Definition of theorem styles and fonts}
%
% All the definitions in this file are done globally to allow
% inputting this file inside a group.
%
% \begin{macro}{\theoremstyle}
%    Before a theorem style can be installed, the chosen style must be
%    known. For that reason, we must test to see that
%    |\th@|\meta{style} is known or, more precisely, that it is
%    different from |\relax|. If the style is not known then
%    |\th@plain| is used.
%    \begin{macrocode}
\gdef\theoremstyle#1{%
   \@ifundefined{th@#1}{\@warning
          {Unknown theoremstyle `#1'. Using `plain'}%
          \theorem@style{plain}}%
%    \end{macrocode}
%    We save the theorem style to be used in the token register
%    |\theorem@style|.
%    \begin{macrocode}
      {\theorem@style{#1}}%
%    \end{macrocode}
%    Now we ``evaluate'' the theorem style: this means, we call the
%    macro |\th@|\meta{style} which will activate the relevant
%    definitions which are contained in a separate file.  This is done
%    in a group to suppress changes to the current font.  This could
%    otherwise pose problems together with the new font selection
%    scheme\footnote{When I printed the original article using the new
%    font selection scheme I ended with a document with slanted
%    typefaces (text headings and all) simply because one of the
%    theorem styles used \texttt{\bslash sl} at toplevel.} if the
%    |\th@|\meta{style} is evaluated a second time.
% \changes{v2.1a}{90/02/15}{Group added around `th@..}
% \changes{v2.1a}{90/02/15}{Reseting of `..@begintheorem removed.}
%    \begin{macrocode}
      \begingroup
        \csname th@\the\theorem@style \endcsname
      \endgroup}
%    \end{macrocode}
% \end{macro}
%
%
% \begin{macro}{\@begintheorem}
% \changes{v2.1a}{90/02/15}{Relaxed at toplevel.}
% \begin{macro}{\@opargbegintheorem}
% \changes{v2.1a}{90/02/15}{Relaxed at toplevel.}
%    We reset |\@begintheorem| and |\@opargbegintheorem| to
%    |\relax| since these commands are no longer necessary at
%    toplevel. This will save a few tokens.
%    \begin{macrocode}
\global\let\@begintheorem\relax
\global\let\@opargbegintheorem\relax
%    \end{macrocode}
% \end{macro}
% \end{macro}
%
% \begin{macro}{\theorem@style}
%    Obviously the \textsf{token} register used above has to be
%    allocated. To assure the utmost compatibility with the
%    original \LaTeX{} definition, we set the default theorem style to
%    \texttt{plain}, which implements the usual \LaTeX{} convention.
% \changes{v2.0f}{89/07/20}{Unnecessary `global' removed.}
%    \begin{macrocode}
\newtoks\theorem@style
\global\theorem@style{plain}
%    \end{macrocode}
% \end{macro}
%
% \begin{macro}{\theorembodyfont}
% \begin{macro}{\theorem@bodyfont}
%    If the |\theorembodyfont| is set by the user then it should not
%    interact with the default font set in the theorem style. When the
%    new font selection is in force this may happen if, for example,
%    the default is |\itshape| and the new |\theorembodyfont| is
%    |\sffamily|. So we add a |\reset@font| command in front of the
%    user definition.
% \changes{v2.0f}{89/07/20}{Unnecessary `global' removed.}
% \changes{v2.0g}{89/09/19}{`theoremfont to `theorembodyfont.}
% \changes{v2.1c}{92/06/22}{`reset@font added and logic changed.}
% \changes{v2.1d}{92/10/28}{Reset to empty if arg is empty}
%    \begin{macrocode}
\gdef\theorembodyfont#1{%
%    \end{macrocode}
%    We check if the argument supplied is empty and if so put nothing
%    into the |\theorem@bodyfont| token register to allow for
%    |\theorembodyfont{}| as a mean of using the default of the
%    current |\theoremstyle|.
%    \begin{macrocode}
   \def\@tempa{#1}%
   \ifx\@tempa\@empty
    \theorem@bodyfont{}%
   \else
    \theorem@bodyfont{\reset@font#1}%
   \fi
   }
\newtoks\theorem@bodyfont
\global\theorem@bodyfont{}
%    \end{macrocode}
% \end{macro}
% \end{macro}
%
%
% \begin{macro}{\theoremheaderfont}
% \changes{v2.0g}{89/09/19}{Macro added.}
%    The font for the theorem headers is handled differently because
%    this definition applies to all theorem styles.
%    \begin{macrocode}
\gdef\theoremheaderfont#1{\gdef\theorem@headerfont{#1}%
%    \end{macrocode}
%    After using the macro once it is redefined to produce an error
%    message.
%    \begin{macrocode}
       \gdef\theoremheaderfont##1{%
        \typeout{\string\theoremheaderfont\space should be used
                 only once.}}}
%    \end{macrocode}
% \end{macro}
%
% \begin{macro}{\theorem@headerfont}
% \changes{v2.0g}{89/09/19}{Macro added.}
%    To set the |\theorem@headerfont| default we first test if
%    the new fontselection scheme is in force.
% \changes{v2.1a}{90/02/15}{Support for new font selection.}
%    \begin{macrocode}
\ifx\upshape\undefined
%    \end{macrocode}
%    If not we define it to expand into |\bfseries|. We don't use
%    |\let| just in case a following style option redefines this
%    macro.
%    \begin{macrocode}
\gdef\theorem@headerfont{\bfseries}
%    \end{macrocode}
%    Otherwise we reset the current shape before calling |\bfseries|.
%    \begin{macrocode}
\else \gdef\theorem@headerfont{\normalfont\bfseries}\fi
%    \end{macrocode}
% \end{macro}
%
% \begin{macro}{\th@plain}
% \begin{macro}{\th@break}
% \begin{macro}{\th@marginbreak}
% \begin{macro}{\th@changebreak}
% \begin{macro}{\th@change}
% \begin{macro}{\th@margin}
%    The different styles are defined in macros such as
%    |\th@plain|.  Since memory space is precious in
%    ``non-Big-versions'', we have to avoid offering too many unused
%    definitions.  Therefore we define these styles in separate files
%    that can be loaded on demand.  Thus the commands themselves only
%    load these files. We use |\@input@|, a \LaTeXe{} internal command
%    that ensures that the file will be listed with |\listfiles|.
% \changes{v2.2a}{94/02/02}{Use `@input@ to load theorem layout files}
%    \begin{macrocode}
\gdef\th@plain{\@input@{thp.sty}}
\gdef\th@break{\@input@{thb.sty}}
\gdef\th@marginbreak{\@input@{thmb.sty}}
\gdef\th@changebreak{\@input@{thcb.sty}}
\gdef\th@change{\@input@{thc.sty}}
\gdef\th@margin{\@input@{thm.sty}}
%    \end{macrocode}
%    This list will be expanded when new styles become
%    available.  For testing,  just append new theorem substyles
%    as document options.
% \end{macro}
% \end{macro}
% \end{macro}
% \end{macro}
% \end{macro}
% \end{macro}
%
%
%
% \subsection{Definition of a new theorem set}
%
% As already pointed out, a new theorem environment can be defined in
% three different ways:
% \begin{verbatim}
%    \newtheorem{Lem}{Lemma}
%    \newtheorem{Lem}{Lemma}[section]
%    \newtheorem{Lem}[Theorem]{Lemma}
%\end{verbatim}
% The function of the macro |\newtheorem| is to recognize these cases
% and then to branch into one of the three macros |\@ynthm|, |\@xnthm|
% or |\@othm|.  This mechanism is adopted unchanged from
% \cite{latex:source}; the essential point here is that, for example,
% in the second case, the arguments \texttt{Lem}, \texttt{Lemma} and
% \texttt{section} are passed over to the macro |\@xnthm|.
%
% We inspect this case first because the others present fewer problems,
% and thus are easily derived from this one.
%
% \begin{macro}{\@xnthm}
%    For our example arguments, the macro |\@xnthm| must fulfill
%    the following:
%    \begin{itemize}
%       \item
%          Define a new \LaTeX-counter `Lem'
%       \item
%          reset this counter within a |\section|
%       \item
%          define the macro |\theLem|
%       \item
%          define the environment macros |\Lem| and
%          |\endLem| using the current
%          |\theoremstyle| and |\theorem@bodyfont|.
%    \end{itemize}
%    Obviously, all this should happen only if the first argument
%    of |\@xnthm| (i.e.\ \texttt{Lem} in our example) is chosen so
%    as not to conflict with any previously defined commands or
%    environments.  This test is performed by the \LaTeX{} macro
%    |\@ifdefinable|.
%    \begin{macrocode}
\gdef\@xnthm#1#2[#3]{\expandafter\@ifdefinable\csname #1\endcsname
%    \end{macrocode}
%    Therefore, the first argument of |\@ifdefinable| is the
%    expansion (in the example, |\Lem|) of
%    |\csname||#1\endcsname|.  The second argument is
%    executed only if the test has been completed successfully.
%    \begin{macrocode}
   {%
%    \end{macrocode}
%    Now we define the new counter. The names of the \LaTeX{}
%    macros employed should speak for themselves:
% \changes{v2.2b}{1995/10/07}{Use \cs{@newctr}}
% \changes{v2.2c}{1995/11/23}{Use \cs{@newctr} properly.}
%    \begin{macrocode}
    \@definecounter{#1}\@newctr{#1}[#3]%
%    \end{macrocode}
%    Using |\@newctr| will give a proper error message if the counter
%    in |#3| is not defined.
%    In defining `|\theLem|' we must generate the desired macro
%    name by use of |\expandafter| and |\csname|.
%    \begin{macrocode}
    \expandafter\xdef\csname the#1\endcsname
%    \end{macrocode}
%    An |\xdef| is used in order to make the definition global,
%    and to ensure that it contains the replacement texts of
%    |\@thmcountersep| and |\@thmcounter|.\footnote{These
%    two macros can be defined by the document style.  Their default
%    values produce a `\texttt{.}' as separation and an arabic
%    representation of the number.} However, not everything should be
%    expanded. For example, it saves space to use |\thesection|
%    instead of its---at times---lengthy expansion.
%    \begin{macrocode}
      {\expandafter \noexpand \csname the#3\endcsname
       \@thmcountersep \@thmcounter{#1}}%
%    \end{macrocode}
%    Thus with the defaults of \LaTeX{}, |\theLem| would be
%    replaced by the command sequence |\thesection.\arabic{Lem}|.
%
%    We will now look at the definition of the macro which is executed
%    at the beginning of the actual environment (in our example this
%    macro is |\Lem|). It should be noted that we use an
%    ``|\expandafter| trick'' to expand only certain parts of the
%    replacement text at the time of the definition.
%    \begin{macrocode}
    \def\@tempa{\global\@namedef{#1}}%
    \expandafter \@tempa \expandafter{%
%    \end{macrocode}
%    First, the macro that contains the current definitions of
%    |\@begintheorem| and |\@opargtheorem| should be called
%    up.  The name of this macro---as is already known---has the form
%    |\th@|\meta{theorem\ style}; therefore, it must be called by
%    \begin{macrocode}
      \csname th@\the \theorem@style
%    \end{macrocode}
%    In addition the default theorem font should be changeable, i.e.\
%    we have to insert the contents of |\theorem@bodyfont|. For
%    that reason, we expand even further, beyond |\endcsname|,
%    and thus insert the contents of the \textsf{token} register
%    |\theorem@bodyfont| in the replacement text.
%    \begin{macrocode}
            \expandafter \endcsname \the \theorem@bodyfont
%    \end{macrocode}
%    Now it is time to call the macro |\@thm| which takes over
%    the further processing.  It has two arguments: the current
%    counter name (in our example, \texttt{Lem}), and the text of the
%    label (in our example, \texttt{Lemma}).
%    \begin{macrocode}
     \@thm{#1}{#2}}%
%    \end{macrocode}
%    With this, the `sub-definition' is complete.  The macro
%    |\@endtheorem| ends a theorem environment and is, so far, nothing
%    but an |\endtrivlist|.  (Hence it is defined globally, and not
%    within the theorem styles.\footnote{This has to be changed as
%    soon as theorem styles that change \texttt{\bslash @endtheorem}
%    exist.  In such a case, all existing styles must be changed as
%    well since they will have to reset the macro.}) Therefore, we can
%    set it equivalent to the macro that ends the theorem set (in our
%    example, |\endLem|).  However, if some day theorem styles exist
%    that do change |\@endtheorem|, we would have to use the
%    commented-out line instead.
%    \begin{macrocode}
    \global \expandafter \let \csname end#1\endcsname \@endtheorem
%   \global\@namedef{end#1}{\@endtheorem}%
%    \end{macrocode}
%    With these commands all the required definitions are employed,
%    unless the test |\@ifdefinable| has failed.  Therefore, we
%    end the second argument of this macro and with it the definition
%    of |\@xnthm|.
%    \begin{macrocode}
   }}
%    \end{macrocode}
% \end{macro}
%
% \begin{macro}{\@ynthm}
%    The definition of |\@ynthm| is completely analogous. In
%    this case the new counter that is defined is not reset within
%    another counter; thus the definition of |\the...|
%    is simplified:
%    \begin{macrocode}
\gdef\@ynthm#1#2{\expandafter\@ifdefinable\csname #1\endcsname
   {\@definecounter{#1}%
    \expandafter\xdef\csname the#1\endcsname{\@thmcounter{#1}}%
%    \end{macrocode}
%    The rest of the definition corresponds literally to that of
%    |\@xnthm|:
%    \begin{macrocode}
    \def\@tempa{\global\@namedef{#1}}\expandafter \@tempa
     \expandafter{\csname th@\the \theorem@style \expandafter
     \endcsname \the\theorem@bodyfont \@thm{#1}{#2}}%
    \global \expandafter \let \csname end#1\endcsname \@endtheorem}}
%    \end{macrocode}
% \end{macro}
%
% \begin{macro}{\@othm}
%    The definition of |\@othm| does not contain anything new.
%    \begin{macrocode}
\gdef\@othm#1[#2]#3{%
%    \end{macrocode}
% \changes{v2.2b}{95/10/07}{Build in check for counter defined}
%    We do not define a new counter but instead use one that has
%    already been defined.  Thus the only definition we need is that
%    of this pseudo-counter (i.e.\ |\the|\meta{env.\ name}).
%    First we check if |#2| corresponds to a known counter name.
%    \begin{macrocode}
  \expandafter\ifx\csname c@#2\endcsname\relax
   \@nocounterr{#2}%
  \else
   \expandafter\@ifdefinable\csname #1\endcsname
   {\expandafter \xdef \csname the#1\endcsname
     {\expandafter \noexpand \csname the#2\endcsname}%
%    \end{macrocode}
%    All other parts of the definition can be adopted from
%    |\@xnthm|. We have to remember, though, that in this case
%    the name of the current counter and the theorem label have moved
%    to the second and third arguments.
%    \begin{macrocode}
    \def\@tempa{\global\@namedef{#1}}\expandafter \@tempa
     \expandafter{\csname th@\the \theorem@style \expandafter
     \endcsname \the\theorem@bodyfont \@thm{#2}{#3}}%
    \global \expandafter \let \csname end#1\endcsname \@endtheorem}%
  \fi}
%    \end{macrocode}
% \end{macro}
%
% \subsection{Macros that are employed in a {\envfont theorem}
%    environment}
%
% \begin{macro}{\@thm}
%    The macro |\@thm| has to increase the current counter.  Then,
%    depending on whether the environment has (or does not have) an
%    optional argument, it has to branch into either |\@begintheorem|
%    or |\@opargtheorem|.
%    \begin{macrocode}
\gdef\@thm#1#2{\refstepcounter{#1}%
%    \end{macrocode}
%    Now we start a {\envfont trivlist} environment, and give
%    |\@topsep| and |\@topsepadd| the values of the \textsf{skip}
%    registers |\theorempreskipamount| and |\theorempostskipamount|.
%    The value in |\@topsep| is the vertical space that is inserted by
%    the first (and only) |\item| in our |\trivlist| whilst
%    |\@topsepadd| is inserted by |\@endparenv| at the end of that
%    {\envfont trivlist} environment.  By using these registers, we
%    obtain the desired space around a {\envfont theorem} environment.
%    \begin{macrocode}
   \trivlist
   \@topsep \theorempreskipamount               % used by first \item
   \@topsepadd \theorempostskipamount           % used by \@endparenv
%    \end{macrocode}
%    Now we have to test whether an optional argument has been given.
% \changes{v2.0d}{89/07/18}{Forgotten percent added.}
%    \begin{macrocode}
   \@ifnextchar [%
%    \end{macrocode}
%    If there is an optional argument, we will call |\@ythm|, and
%    move the arguments read back into the input stream.
%    \begin{macrocode}
   {\@ythm{#1}{#2}}%
%    \end{macrocode}
%    If not, we call |\@begintheorem|. Its first argument is the
%    name of the theorem set (hence the second argument of
%    |\@thm|). Its second argument is the macro that produces the
%    current number.
%    \begin{macrocode}
   {\@begintheorem{#2}{\csname the#1\endcsname}\ignorespaces}}
%    \end{macrocode}
% \end{macro}
%
% \begin{macro}{\@xthm}
% \begin{macro}{\@ythm}
%    Both these macros were originally called by |\@thm|.  We do
%    not need |\@xthm| anymore, hence we reset it to
%    |\relax|. The definition of |\@ythm| has not changed at
%    all from its definition in \LaTeX. In order to make the macros
%    easier to understand, we will
%    nevertheless present it (commented out).
%    \begin{macrocode}
\global\let\@xthm\relax
% \def\@ythm#1#2[#3]{\@opargbegintheorem{#2}{\csname
%           the#1\endcsname}{#3}\ignorespaces}
%</package>
%    \end{macrocode}
%    The primitive |\ignorespaces| in |\@ythm| and
%    |\@thm| is needed to remove the spaces between the
%    |\begin{...}| and the actual text.
% \end{macro}
% \end{macro}
%
%
% \subsection{Definition of the theorem substyles}
%
% As already pointed out, the theorem substyles, defined below, are
% only loaded when necessary.  Note that all these substyles, except
% \texttt{plain}, have |\slshape| as the default body font.
%
% \subsubsection{The \texttt{plain} style}
%
% \label{dok:th@plain} As the following macros use |@|, we have
% to locally set the |\catcode| of this symbol to ``letter''.
% This happens within a group, so that we do not have to worry about
% which |\catcode| that symbol had before.
%    \begin{macrocode}
%<*thp>
\begingroup \makeatletter
%    \end{macrocode}
% Since we are now within a group, we must make all definitions
% globally.  First we make sure that \texttt{theorem.sty} is loaded.
% This will allow us to use this file as a document style option
% without having to call \texttt{theorem} itself as an option.  At the
% same time, we assure that at least version 2 is loaded, since
% |\theorem@style| was not defined in earlier versions.
%    \begin{macrocode}
\@ifundefined{theorem@style}{\input{theorem.sty}}{}
\ProvidesFile{thp.sty}
             [\FMithmInfo]
%    \end{macrocode}
%
%
% \begin{macro}{\th@plain}
% \begin{macro}{\@begintheorem}
% \begin{macro}{\@opargbegintheorem}
%    |\theoremstyle{plain}| corresponds to the original
%    definition, except that the distances to the surrounding text are
%    determined by the parameters |\theorempreskipamount| and
%    |\theorempostskipamount|.
%    First we set the default body font.
%    \begin{macrocode}
\gdef\th@plain{\normalfont\itshape
%    \end{macrocode}
%    Then we define |\@begintheorem| and
%    |\@opargbegintheorem|. These two macros define how the
%    header of a theorem is typeset.  |\@opargbegintheorem| will
%    be called if a \textsf{theorem} environment with an optional
%    argument is encountered; otherwise, the header is constructed by
%    calling |\@begintheorem|.  If one of these macros is
%    executed, we are within a \textsf{trivlist} environment started by
%    |\@thm|.  So the theorem header is produced with an
%    |\item| command.
%
%    Instead of specifying the header font directly, all standard
%    theorem styles use the |\theorem@headerfont| macro to allow
%    customization.  The extra space (|\labelsep|) is necessary
%    because of problems in the \textsf{trivlist} environment.
%    \begin{macrocode}
  \def\@begintheorem##1##2{%
        \item[\hskip\labelsep \theorem@headerfont ##1\ ##2]}%
%    \end{macrocode}
%    The definition of |\@opargbegintheorem| is completely
%    analogous. The only difference is the fact that there exists a
%    third argument (which is the optional parameter of the
%    environment and contains additional information about the
%    theorem).  Customarily we enclose it in parentheses.
%    \begin{macrocode}
\def\@opargbegintheorem##1##2##3{%
   \item[\hskip\labelsep \theorem@headerfont ##1\ ##2\ (##3)]}}
%    \end{macrocode}
% \end{macro}
% \end{macro}
% \end{macro}
%
% \noindent
% We conclude with an |\endgroup| to restore the
% |\catcode| of |@|.
%    \begin{macrocode}
\endgroup
%</thp>
%    \end{macrocode}
%
%
% \subsubsection{The \texttt{break} style}
%
% This style option is stored in the file |thb.sty|.
% For the next two lines see the documentation for |\th@plain| on
% page~\pageref{dok:th@plain}.
%    \begin{macrocode}
%<*thb>
\begingroup \makeatletter
\@ifundefined{theorem@style}{\input{theorem.sty}}{}
\ProvidesFile{thb.sty}
             [\FMithmInfo]
%    \end{macrocode}
%
% \begin{macro}{\th@break}
%    |\theoremstyle{break}| produces a line break after the name
%    of the theorem.  The font is |\slshape|.  Hence, we define
%    |\th@break| as follows:
%    \begin{macrocode}
\gdef\th@break{\normalfont\slshape
  \def\@begintheorem##1##2{\item[%
%    \end{macrocode}
%    We run into the following problem: it is not possible to create
%    the header with |\item[|\meta{title}|]| and then start
%    a new line by, for example, |\mbox{}\\|. Such a definition
%    will fail whenever a list environment follows immediately.  With
%    the above construction, the |\mbox{}| causes the switch
%    |@inlabel| (cf.\ definition of |\list| and
%    |\trivlist| in \cite{latex:source}) to be set to \texttt{false}
%    and so the following list will insert additional vertical
%    space (|\topskip|). This is quite annoying. Therefore, we
%    create the line break within the |\item|. In order to ensure
%    that the text will begin at the proper position in the following
%    line, we simply pretend that the label does not take any
%    room.\footnote{This will lead to problems whenever very high
%    symbols occurring in the line tower into the
%    heading. So, something else has to be done here sometime.}
%    \begin{macrocode}
       \rlap{\vbox{\hbox{\hskip \labelsep\theorem@headerfont ##1\ ##2}%
                   \hbox{\strut}}}]}%
%    \end{macrocode}
%    Again, the definition of |\@opargbegintheorem| is
%    completely analogous.
%    \begin{macrocode}
\def\@opargbegintheorem##1##2##3{%
  \item[\rlap{\vbox{\hbox{\hskip \labelsep \theorem@headerfont
                     ##1\ ##2\ (##3)}%
                    \hbox{\strut}}}]}}
\endgroup
%</thb>
%    \end{macrocode}
% \end{macro}
%
%
% \subsubsection{The \texttt{changebreak} style}
%
%    \begin{macrocode}
% This style option is stored in the file |thcb.sty|.
%    \begin{macrocode}
%<*thcb>
\begingroup \makeatletter
\@ifundefined{theorem@style}{\input{theorem.sty}}{}
\ProvidesFile{thcb.sty}
             [\FMithmInfo]
%    \end{macrocode}
%
% \begin{macro}{\th@changebreak}
%    The \texttt{change-break} theorem style is like \texttt{break}
%    but with interchange of theorem name and theorem number.  Thus we
%    define |\th@changebreak| as follows:
%    \begin{macrocode}
\gdef\th@changebreak{\normalfont\slshape
  \def\@begintheorem##1##2{\item
       [\rlap{\vbox{\hbox{\hskip\labelsep \theorem@headerfont ##2\ ##1}%
                    \hbox{\strut}}}]}%
\def\@opargbegintheorem##1##2##3{%
  \item[\rlap{\vbox{\hbox{\hskip\labelsep \theorem@headerfont
                     ##2\ ##1\ (##3)}%
                    \hbox{\strut}}}]}}
\endgroup
%</thcb>
%    \end{macrocode}
% \end{macro}
%
%
% \subsubsection{The \texttt{change} style}
%
% This style option is stored in the file |thc.sty|.
%    \begin{macrocode}
%<*thc>
\begingroup \makeatletter
\@ifundefined{theorem@style}{\input{theorem.sty}}{}
\ProvidesFile{thc.sty}
             [\FMithmInfo]
%    \end{macrocode}
%
% \begin{macro}{\th@change}
%    The \texttt{change} theorem style corresponds to the
%    \texttt{change break} style without a linebreak after the header.
%    To say it in another way, it's the same as the \texttt{plain}
%    style but with number and name interchanged and |\slshape| as the
%    default font.
%    \begin{macrocode}
\gdef\th@change{\normalfont\slshape
  \def\@begintheorem##1##2{\item
       [\hskip\labelsep \theorem@headerfont ##2\ ##1]}%
\def\@opargbegintheorem##1##2##3{%
  \item[\hskip\labelsep \theorem@headerfont ##2\ ##1\ (##3)]}}
\endgroup
%</thc>
%    \end{macrocode}
% \end{macro}
%
%
% \subsubsection{The \texttt{marginbreak} style}
%
% This style option is the one used most often at Mainz.  It is saved
% in the file |thmb.sty|.
%    \begin{macrocode}
%<*thmb>
\begingroup \makeatletter
\@ifundefined{theorem@style}{\input{theorem.sty}}{}
\ProvidesFile{thmb.sty}
             [\FMithmInfo]
%    \end{macrocode}
%
% \begin{macro}{\th@marginbreak}
%    The \texttt{margin break} style is nearly the same as the
%    \texttt{change break} style. The only difference is the placement
%    of the theorem number.  We use |\llap| to place it in the left
%    margin.
%
%    In this style |\labelsep| denotes the separation between the
%    number and the text.
%    \begin{macrocode}
\gdef\th@marginbreak{\normalfont\slshape
  \def\@begintheorem##1##2{\item
       [\rlap{\vbox{\theorem@headerfont
                    \hbox{\llap{##2}\hskip\labelsep ##1}%
                    \hbox{\strut}}}]}%
\def\@opargbegintheorem##1##2##3{%
  \item[\rlap{\vbox{\theorem@headerfont
                    \hbox{\llap{##2}\hskip\labelsep ##1\ (##3)}%
                    \hbox{\strut}}}]}}
\endgroup
%</thmb>
%    \end{macrocode}
% \changes{v2.1c}{90/02/17}{`endgroup outside macrocode env.}
% \end{macro}
%
%
%
% \subsubsection{The \texttt{margin} style}
%
% This style option is stored in the file |thm.sty|.
%    \begin{macrocode}
%<*thm>
\begingroup \makeatletter
\@ifundefined{theorem@style}{\input{theorem.sty}}{}
\ProvidesFile{thm.sty}
             [\FMithmInfo]
%    \end{macrocode}
%
% \begin{macro}{\th@margin}
%    Again this is only a variant of the theorem styles described
%    above without any new ideas.
%    \begin{macrocode}
\gdef\th@margin{\normalfont\slshape
  \def\@begintheorem##1##2{\item
       [\theorem@headerfont \llap{##2}\hskip\labelsep ##1]}%
\def\@opargbegintheorem##1##2##3{%
  \item[\theorem@headerfont \llap{##2}\hskip\labelsep ##1\ (##3)]}}
\endgroup
%</thm>
%    \end{macrocode}
% \end{macro}
%
%
%
%
% \subsection{Final Definitions}
%
% \begin{macro}{\theorempreskipamount}
% \begin{macro}{\theorempostskipamount}
%    The \textsf{skip} parameters that regulate the vertical empty space
%    before and after the {\envfont theorem} environment have
%    to be allocated as well.
% \changes{v2.0f}{89/07/20}{Unnecessary `global' removed.}
%    \begin{macrocode}
%<*package>
\newskip\theorempreskipamount
\newskip\theorempostskipamount
%    \end{macrocode}
%    Since we have used the same values for all theorem sets, we now can
%    assign them.
%    \begin{macrocode}
\global\setlength\theorempreskipamount{12pt plus 5pt minus 3pt}
\global\setlength\theorempostskipamount{8pt plus 3pt minus 1.5pt}
%    \end{macrocode}
% \end{macro}
% \end{macro}
%
%
% \begin{macro}{\@endtheorem}
%    The same holds for the macro |\@endtheorem|, which ends a
%    {\envfont theorem} environment. Since it is the same for all
%    theorem sets, it is removed from the macros |\th@|\meta{style}.
%    It simply ends the {\envfont trivlist} environment, which was
%    begun in |\@thm|.
%    \begin{macrocode}
\global\let\@endtheorem=\endtrivlist
%    \end{macrocode}
% \end{macro}
%
%
%
% \begin{macro}{\@preamblecmds}
%    All macros defined above are to be used only in the preamble.
%    Therefore, we insert them in |\@preamblecmds| which will disable
%    them at begin document. This is done by the internal \LaTeXe{}
%    command |\@onlypreamble|.
% \changes{v2.0g}{89/09/19}{`theoremheaderfont and `theorembodyfont
%    added.}
% \changes{v2.2a}{94/02/03}{Use \LaTeXe{} interface}
%    \begin{macrocode}
\@onlypreamble\@xnthm
\@onlypreamble\@ynthm
\@onlypreamble\@othm
\@onlypreamble\newtheorem
\@onlypreamble\theoremstyle
\@onlypreamble\theorembodyfont
\@onlypreamble\theoremheaderfont
%    \end{macrocode}
% \end{macro}
%
% Finally we declare the \texttt{plain} theorem style to be the default.
% \changes{v2.1b}{90/06/04}{Fixed missing `item by declaring default
%    style.}
%    \begin{macrocode}
\theoremstyle{plain}
%</package>
%    \end{macrocode}
%
% \Finale

\endinput
}{}
\ProvidesFile{thb.sty}
             [\FMithmInfo]
%    \end{macrocode}
%
% \begin{macro}{\th@break}
%    |\theoremstyle{break}| produces a line break after the name
%    of the theorem.  The font is |\slshape|.  Hence, we define
%    |\th@break| as follows:
%    \begin{macrocode}
\gdef\th@break{\normalfont\slshape
  \def\@begintheorem##1##2{\item[%
%    \end{macrocode}
%    We run into the following problem: it is not possible to create
%    the header with |\item[|\meta{title}|]| and then start
%    a new line by, for example, |\mbox{}\\|. Such a definition
%    will fail whenever a list environment follows immediately.  With
%    the above construction, the |\mbox{}| causes the switch
%    |@inlabel| (cf.\ definition of |\list| and
%    |\trivlist| in \cite{latex:source}) to be set to \texttt{false}
%    and so the following list will insert additional vertical
%    space (|\topskip|). This is quite annoying. Therefore, we
%    create the line break within the |\item|. In order to ensure
%    that the text will begin at the proper position in the following
%    line, we simply pretend that the label does not take any
%    room.\footnote{This will lead to problems whenever very high
%    symbols occurring in the line tower into the
%    heading. So, something else has to be done here sometime.}
%    \begin{macrocode}
       \rlap{\vbox{\hbox{\hskip \labelsep\theorem@headerfont ##1\ ##2}%
                   \hbox{\strut}}}]}%
%    \end{macrocode}
%    Again, the definition of |\@opargbegintheorem| is
%    completely analogous.
%    \begin{macrocode}
\def\@opargbegintheorem##1##2##3{%
  \item[\rlap{\vbox{\hbox{\hskip \labelsep \theorem@headerfont
                     ##1\ ##2\ (##3)}%
                    \hbox{\strut}}}]}}
\endgroup
%</thb>
%    \end{macrocode}
% \end{macro}
%
%
% \subsubsection{The \texttt{changebreak} style}
%
%    \begin{macrocode}
% This style option is stored in the file |thcb.sty|.
%    \begin{macrocode}
%<*thcb>
\begingroup \makeatletter
\@ifundefined{theorem@style}{% \iffalse meta-comment
%
% Copyright (C) 1993-2024
%
% The LaTeX Project and any individual authors listed elsewhere
% in this file.
%
% This file is part of the Standard LaTeX `Tools Bundle'.
% -------------------------------------------------------
%
% It may be distributed and/or modified under the
% conditions of the LaTeX Project Public License, either version 1.3c
% of this license or (at your option) any later version.
% The latest version of this license is in
%    https://www.latex-project.org/lppl.txt
% and version 1.3c or later is part of all distributions of LaTeX
% version 2005/12/01 or later.
%
% The list of all files belonging to the LaTeX `Tools Bundle' is
% given in the file `manifest.txt'.
%
% \fi
%
%%
%
%
\def\FMithmInfo{2023/07/05 v2.2c Theorem extension package (FMi)}
%
% \ProvidesFile{theorem.dtx}[\FMithmInfo]
%\iffalse   % this is a METACOMMENT !
%%
%% Package `theorem' to use with LaTeX2e
%% Copyright (C) 1989-1995 Frank Mittelbach, all rights reserved.
%<+package>\@ifundefined{theorem@style}{}{\endinput}
%<+package>\NeedsTeXFormat{LaTeX2e}[1995/06/01]
%<+package>\ProvidesPackage{theorem}
%<+package>         [\FMithmInfo]
%
% \fi
%
% \changes{v1.0}{87/05/07}{First draft.}
% \changes{v1.0a}{87/05/07}{`thistheoremstyle' added.}
% \changes{v1.1a}{87/07/29}{New concept: redefine the internal
%                          \protect\LaTeX{} commands `@xnthm' etc.}
% \changes{v1.1b}{87/07/31}{`theoremfont' used to specify font.}
% \changes{v1.2a}{88/01/08}{`@endtheorem' now globally defined.}
% \changes{v1.2b}{88/01/11}{`newtheorem' only allowed in the preamble.}
% \changes{v1.3a}{88/02/14}{`theoremfont' now a toks register.}
%
% \changes{v2.0a}{88/11/16}{New implementation and new documentation.}
% \changes{v2.0a}{88/11/16}{No layout changes but internal modifications
%                           old substyles won't work.}
% \changes{v2.0b}{89/01/30}{`global' used for all definitions.}
% \changes{v2.0c}{89/06/15}{Documentation translated to English.
%                            Thanks to LLL and Nina}
% \changes{v2.1a}{90/02/15}{Support for new font selection.}
% \changes{v2.2a}{94/02/03}{Cleanup for LaTeX2e}
%
% \DoNotIndex{\@endparenv,\@ifdefinable,\@ifnextchar,\@ifundefined}
% \DoNotIndex{\@namedef,\@nameuse,\@spaces,\@tempa,\@warning}
% \DoNotIndex{\ }
% \DoNotIndex{\begingroup,\bfseries}
% \DoNotIndex{\csname}
% \DoNotIndex{\def,\do,\docdate}
% \DoNotIndex{\endcsname,\endgroup,\endinput,\endtrivlist,\expandafter}
% \DoNotIndex{\filedate,\fileversion}
% \DoNotIndex{\gdef,\global}
% \DoNotIndex{\hbox,\hskip}
% \DoNotIndex{\ignorespaces,\input,\itshape,\item}
% \DoNotIndex{\labelsep,\let,\llap}
% \DoNotIndex{\makeatletter}
% \DoNotIndex{\noexpand}
% \DoNotIndex{\refstepcounter,\relax,\rlap}
% \DoNotIndex{\setlength,\slshape,\space,\string,\strut}
% \DoNotIndex{\the,\trivlist,\typeout}
% \DoNotIndex{\vbox}
% \DoNotIndex{\xdef}
%
% \def\deflabel#1{\ttfamily #1\hfill}
% \def\deflist#1{\begin{list}{}{\settowidth\labelwidth{\ttfamily #1}%
%                               \setlength\leftmargin\labelwidth
%                               \addtolength\leftmargin\labelsep
%                               \let\makelabel\deflabel}}
% \def\enddeflist{\end{list}}
% \def\envfont{\normalfont\ttfamily}
%
% \MakeShortVerb{\|}
%
% \newcommand{\TUB}{{\sl TUGboat\/}}
%
% \GetFileInfo{theorem.dtx}
% \title{An Extension of the
%        \LaTeX{} theorem environment\thanks{This file
%        has version number \fileversion, last
%        revised \filedate.}}
% \author{Frank Mittelbach}
%
% \MaintainedByLaTeXTeam{tools}
% \maketitle
%
% \begin{abstract}
%    The macros described in this paper yield an extension of the
%    \LaTeX{} theorem mechanism. It is designed to satisfy the
%    different requirements of various journals. Thus, the
%    layout of the ``theorems'' can be manipulated by determining a
%    ``style''. This article describes not only the use, but
%    also the definition, of the necessary macros.
% \end{abstract}
%
% \section*{Preface to version 2.2}
%
% For \LaTeXe{} this package did not need any fundamental changes. I
% only modified the messages generated so that theorem layout styles
% will show up with the |\listfiles| command and cleaned the section
% on the New Font Selection Scheme since this is now included in
% \LaTeX.
%
% \section*{Preface to version 2.1}
%
% This version is identical to 2.0g described in \TUB\ 10\#3 except
% for some internal defaults which are now set depending on the used
% font selection scheme.
%
% This was done to avoid unpleasant surprises if the new font selection
% scheme is in force. For further details see
% section~\ref{sec:newfonts} and \cite{bk:GMS94}.
%
% \section{Introduction}
%
% For our purposes here, ``theorems'' are labelled enunciations,
% often set off from the main text by extra space and a font change.
% Theorems, corollaries, conjectures, definitions, and remarks
% are all instances of ``theorems''.  The ``header'' of these structures
% is composed of a label (such as \textsc{Theorem} or \textsc{Remark})
% and a number which serializes an item in the sequence of items
% with the same label.
%
% Shortly after the introduction of \LaTeX{} at the Fachbereich
% Mathematik in Mainz, the desire to manipulate the layout of
% ``theorems'' arose.  In Mainz, the following two conventions came
% into general use:
% \begin{enumerate}
%   \item The number of the theorem is shown in the margin.
%   \item There is a line break at
%         the end of the theorem header.
% \end{enumerate}
%
% Additionally, some journals require different formats which depend on
% the ``sort of theorem'': e.g.\ often remarks and definitions are set
% in |\upshape|, while |\itshape| is employed for main theorems.
%
% Confronted with these requirements, a theorem environment was
% developed in Mainz which allows separate determination of the
% layout of the ``theorems sets'', comparable to |\pagestyle|.
%
%
% \section{The user interface}
% \subsection{Defining new theorem sets}
%
% \DescribeMacro\newtheorem
% As in the original \LaTeX{} version, the command |\newtheorem|
% defines a new ``theorem set'' or ``theorem-like structure''.
% Two required arguments name the new environment and give the text
% to be typeset with each instance of the new ``set'', while
% an optional argument determines how
% the ``set'' is enumerated:
% \begin{description}
%    \item[\ttfamily \bslash newtheorem\{foo\}\{bar\}]
%       The theorem set {\envfont foo} (whose name is \texttt{bar})
%       uses its own counter.
%    \item[\ttfamily \bslash newtheorem\{foo2\}{[foo]}\{bar2\}]
%       The theorem set {\envfont foo2} (printed name \texttt{bar2})
%       uses the same counter as the theorem set \texttt{foo}.
%    \item[\ttfamily \bslash newtheorem\{foo3\}\{bar3\}{[section]}]
%       The theorem set {\envfont foo3} (printed name \texttt{bar3}) is
%       enumerated within the counter \texttt{section}, i.e.\ with every
%       new |\section| the enumeration begins again with 1, and
%       the enumeration is composed from the section-number and the
%       theorem counter itself.
% \end{description}
%
% \DescribeMacro\theoremstyle
% Additionally, the command |\theoremstyle| can define the layout
% of various, or all, theorem sets.  It should be noted that any
% theorem set defined by |\newtheorem| is typeset in the
% |\theoremstyle| that is current at the time of the definition.
% Thus, the following
% \begin{verbatim}
%    \theoremstyle{break}        \newtheorem{Cor}{Corollary}
%    \theoremstyle{plain}        \newtheorem{Exa}{Example}[section]
%\end{verbatim}                             ^^A No blank at line start
% leads to the result that the set {\envfont Cor} is formatted in the
% style \texttt{break}, while the set {\envfont Exa} and all the
% following ones are formatted in the style \texttt{plain}, unless
% another |\theoremstyle| follows.  Since the definitions installed by
% |\newtheorem| are global, one also can limit |\theoremstyle| locally
% by grouping braces.
%
% \DescribeMacro\theorembodyfont
% The choice of the font for the theorem body is completely
% independent of the chosen |\theoremstyle|; this has proven to
% be very advantageous.  For example,
% \begin{verbatim}
%   {\theorembodyfont{\upshape}          \newtheorem{Rem}{Remark}}
%\end{verbatim}
% defines a theorem set {\envfont Rem}, which will be set in
% |\upshape| in the current layout (which in our example is
% \texttt{plain}).  As with |\theoremstyle|, the |\theorembodyfont|
% chosen is that current at the time of |\newtheorem|.  If
% |\theorembodyfont| is not specified or one defines
% |\theorembodyfont{}|, then the font used will be that defined by the
% |\theoremstyle|.
%
% \DescribeMacro\theoremheaderfont It is also possible to customize
% the font used for the theorem headers.  This is, however, a global
% declaration, and therefore there should be at most one
% |\theoremheaderfont| declaration in the preamble.\footnote{If it is
% actually necessary to have different header fonts, one has to define
% new theorem styles (substituting the desired font) or specify the
% information directly in the \texttt{\bslash newtheorem} declaration
% (the unclean variant).}
%
% \DescribeMacro\theorempreskipamount
% \DescribeMacro\theorempostskipamount Two additional parameters
% affect the vertical space around the theorem environments:
% |\theorempreskipamount| and |\theorempostskipamount| define,
% respectively, the spacing before and after such an environment.
% These parameters apply for all theorem sets and can be manipulated
% with the ordinary length macros.  They are rubber lengths
% (`\textsf{skips}'), and therefore can contain \texttt{plus} and
% \texttt{minus} parts.
%
% Since the definition of theorem sets should---most sensibly---be
% placed in the preamble, we only allow installation there.
% It is therefore possible to release the memory used here after
% |\begin{document}|, in order to make room for other
% applications.
%
%
% \subsection{Existing theorem styles}
%
% The following theorem styles exist to date:
% \begin{deflist}{marginbreak}
%    \item[plain]
%       This theorem style emulates the original \LaTeX{} definition,
%       except that additionally the parameters
%       |\theorem...skipamount| are used.
%    \item[break]
%       In this style, the theorem header is followed by a line
%       break.
%    \item[marginbreak]
%       The theorem number is set in the margin, and there is a line
%       break as in \texttt{break}.
%    \item[changebreak]
%       Like \texttt{break}, but with header number and text
%       interchanged.
%    \item[change]
%       Header number and text are interchanged, without a line break.
%    \item[margin]
%       The number is set in the left margin, without a line break.
% \end{deflist}
% All styles (except \texttt{plain}) select |\slshape| as the default
% |\theorembodyfont|.
%
%
% \subsection{Examples}
%
% Given the above theorem sets {\envfont Cor}, {\envfont Exa}
% and {\envfont Rem},
% suppose that the preamble also contains the declarations:
% \begin{verbatim}
%    \theoremstyle{marginbreak}   \newtheorem{Lem}[Cor]{Lemma}
%    \theoremstyle{change}
%    \theorembodyfont{\itshape}        \newtheorem{Def}[Cor]{Definition}
%
%    \theoremheaderfont{\scshape}
%\end{verbatim}
% Then the following are some typical examples of the typeset output
% resulting from their use.
%
% \begin{Cor}
%    This is a sentence typeset in the theorem environment {\envfont
%    Cor}.
% \end{Cor}
%
% \begin{Exa}
%    This is a sentence typeset in the theorem environment {\envfont
%    Exa}.
% \end{Exa}
%
% \begin{Rem}
%    This is a sentence typeset in the theorem environment {\envfont
%    Rem}.
% \end{Rem}
%
% \begin{Lem}[Ben User]
%    This is a sentence typeset in the theorem environment {\envfont
%    Lem}.
% \end{Lem}
%
% \begin{Def}[Very impressive Definition]
%    This is a sentence typeset in the theorem environment {\envfont
%    Def}.
% \end{Def}
%
% The last two examples show the effect of the optional argument to
% a theorem environment (it is the text typeset in parentheses).
%
%
% \section{Special Considerations}
% \label{sec:newfonts}
%
% Theoremheader and body are implemented as a unit. This means that
% the |\theoremheaderfont|  will inherit characteristics of the
% |\theorembodyfont| in \LaTeXe. Thus, if for example
% |\theorembodyfont| is |\itshape| and |\theoremheaderfont|
% is |\bfseries| the font selected for the header will have the
% characteristics `bold extended italic'. If this is not desired one
% should set the |\theoremheaderfont| to something like
% \begin{verbatim}
%      \theoremheaderfont{\normalfont\bfseries}
%\end{verbatim}
% i.e.\ supplying all necessary font information explicitly.
%
%
%
% \section{Acknowledgements}
%
% The publication of this set of macros was only possible with the
% help of Christina Busse (translating the manuscript into English),
% Joachim Pense (playing the r\^ole of typist), Chris Rowley
% (looking everything over) and many others providing useful
% suggestions.
%
% \MaybeStop{%
%    \begin{thebibliography}{9}
%       \bibitem{bk:GMS94} \textsc{M.~Goossens}, \textsc{F.~Mittelbach}
%         and \textsc{A.~Samarin}.
%         \newblock The \LaTeX{} Companion.
%         \newblock
%         Addison-Wesley, Reading, Massachusetts, 1994.
%       \bibitem{latex:source} \textsc{Lamport, Leslie.} \newblock
%                               \texttt{latex.tex}, version 2.09, date
%                               Feb. 1990.
%    \end{thebibliography}
%    \PrintIndex
%    \PrintChanges}
%
%
% \section{The documentation driver file}
%
% The next bit of code contains the documentation driver file for
% \TeX{}, i.e., the file that will produce the documentation you are
% currently reading. It will be extracted from this file by the
% \texttt{docstrip} program. Since it is the first code in the file
% one can alternatively process this file directly with \LaTeXe{} to
% obtain the documentation.
%
%    \begin{macrocode}
%<*driver>
\documentclass{ltxdoc}

\usepackage{theorem}

%<+driver>% The next few lines define theorem sets which are used
%<+driver>% in the example section of the documentation.

\theoremstyle{break}       \newtheorem{Cor}{Corollary}
\theoremstyle{plain}       \newtheorem{Exa}{Example}[section]
{\theorembodyfont{\upshape}\newtheorem{Rem}{Remark}}
\theoremstyle{marginbreak} \newtheorem{Lem}[Cor]{Lemma}
\theoremstyle{change}
\theorembodyfont{\itshape} \newtheorem{Def}[Cor]{Definition}

\theoremheaderfont{\scshape}

\RecordChanges

\begin{document}
   \DocInput{theorem.dtx}
\end{document}
%</driver>
%    \end{macrocode}
%
% \section{Definition of the Macros}
%
% If the file has been loaded before, we abort immediately. If not the
% package announces itself (this is actually done at the very top of
% the file---the way it is done isn't good style so don't copy it).
% \changes{v2.0e}{89/07/19}{Spaces removed from `typeout'.}
% \changes{v2.2b}{95/11/19}{Announce moved to top of file}
%    \begin{macrocode}
%<*package>
%\@ifundefined{theorem@style}{}{\endinput}
%\def\FMithmInfo{1995/11/19 v2.2b Theorem extension package (FMi)}
%\ProvidesPackage{theorem}[\FMithmInfo]
%    \end{macrocode}
%
%
% \subsection{Definition of theorem styles and fonts}
%
% All the definitions in this file are done globally to allow
% inputting this file inside a group.
%
% \begin{macro}{\theoremstyle}
%    Before a theorem style can be installed, the chosen style must be
%    known. For that reason, we must test to see that
%    |\th@|\meta{style} is known or, more precisely, that it is
%    different from |\relax|. If the style is not known then
%    |\th@plain| is used.
%    \begin{macrocode}
\gdef\theoremstyle#1{%
   \@ifundefined{th@#1}{\@warning
          {Unknown theoremstyle `#1'. Using `plain'}%
          \theorem@style{plain}}%
%    \end{macrocode}
%    We save the theorem style to be used in the token register
%    |\theorem@style|.
%    \begin{macrocode}
      {\theorem@style{#1}}%
%    \end{macrocode}
%    Now we ``evaluate'' the theorem style: this means, we call the
%    macro |\th@|\meta{style} which will activate the relevant
%    definitions which are contained in a separate file.  This is done
%    in a group to suppress changes to the current font.  This could
%    otherwise pose problems together with the new font selection
%    scheme\footnote{When I printed the original article using the new
%    font selection scheme I ended with a document with slanted
%    typefaces (text headings and all) simply because one of the
%    theorem styles used \texttt{\bslash sl} at toplevel.} if the
%    |\th@|\meta{style} is evaluated a second time.
% \changes{v2.1a}{90/02/15}{Group added around `th@..}
% \changes{v2.1a}{90/02/15}{Reseting of `..@begintheorem removed.}
%    \begin{macrocode}
      \begingroup
        \csname th@\the\theorem@style \endcsname
      \endgroup}
%    \end{macrocode}
% \end{macro}
%
%
% \begin{macro}{\@begintheorem}
% \changes{v2.1a}{90/02/15}{Relaxed at toplevel.}
% \begin{macro}{\@opargbegintheorem}
% \changes{v2.1a}{90/02/15}{Relaxed at toplevel.}
%    We reset |\@begintheorem| and |\@opargbegintheorem| to
%    |\relax| since these commands are no longer necessary at
%    toplevel. This will save a few tokens.
%    \begin{macrocode}
\global\let\@begintheorem\relax
\global\let\@opargbegintheorem\relax
%    \end{macrocode}
% \end{macro}
% \end{macro}
%
% \begin{macro}{\theorem@style}
%    Obviously the \textsf{token} register used above has to be
%    allocated. To assure the utmost compatibility with the
%    original \LaTeX{} definition, we set the default theorem style to
%    \texttt{plain}, which implements the usual \LaTeX{} convention.
% \changes{v2.0f}{89/07/20}{Unnecessary `global' removed.}
%    \begin{macrocode}
\newtoks\theorem@style
\global\theorem@style{plain}
%    \end{macrocode}
% \end{macro}
%
% \begin{macro}{\theorembodyfont}
% \begin{macro}{\theorem@bodyfont}
%    If the |\theorembodyfont| is set by the user then it should not
%    interact with the default font set in the theorem style. When the
%    new font selection is in force this may happen if, for example,
%    the default is |\itshape| and the new |\theorembodyfont| is
%    |\sffamily|. So we add a |\reset@font| command in front of the
%    user definition.
% \changes{v2.0f}{89/07/20}{Unnecessary `global' removed.}
% \changes{v2.0g}{89/09/19}{`theoremfont to `theorembodyfont.}
% \changes{v2.1c}{92/06/22}{`reset@font added and logic changed.}
% \changes{v2.1d}{92/10/28}{Reset to empty if arg is empty}
%    \begin{macrocode}
\gdef\theorembodyfont#1{%
%    \end{macrocode}
%    We check if the argument supplied is empty and if so put nothing
%    into the |\theorem@bodyfont| token register to allow for
%    |\theorembodyfont{}| as a mean of using the default of the
%    current |\theoremstyle|.
%    \begin{macrocode}
   \def\@tempa{#1}%
   \ifx\@tempa\@empty
    \theorem@bodyfont{}%
   \else
    \theorem@bodyfont{\reset@font#1}%
   \fi
   }
\newtoks\theorem@bodyfont
\global\theorem@bodyfont{}
%    \end{macrocode}
% \end{macro}
% \end{macro}
%
%
% \begin{macro}{\theoremheaderfont}
% \changes{v2.0g}{89/09/19}{Macro added.}
%    The font for the theorem headers is handled differently because
%    this definition applies to all theorem styles.
%    \begin{macrocode}
\gdef\theoremheaderfont#1{\gdef\theorem@headerfont{#1}%
%    \end{macrocode}
%    After using the macro once it is redefined to produce an error
%    message.
%    \begin{macrocode}
       \gdef\theoremheaderfont##1{%
        \typeout{\string\theoremheaderfont\space should be used
                 only once.}}}
%    \end{macrocode}
% \end{macro}
%
% \begin{macro}{\theorem@headerfont}
% \changes{v2.0g}{89/09/19}{Macro added.}
%    To set the |\theorem@headerfont| default we first test if
%    the new fontselection scheme is in force.
% \changes{v2.1a}{90/02/15}{Support for new font selection.}
%    \begin{macrocode}
\ifx\upshape\undefined
%    \end{macrocode}
%    If not we define it to expand into |\bfseries|. We don't use
%    |\let| just in case a following style option redefines this
%    macro.
%    \begin{macrocode}
\gdef\theorem@headerfont{\bfseries}
%    \end{macrocode}
%    Otherwise we reset the current shape before calling |\bfseries|.
%    \begin{macrocode}
\else \gdef\theorem@headerfont{\normalfont\bfseries}\fi
%    \end{macrocode}
% \end{macro}
%
% \begin{macro}{\th@plain}
% \begin{macro}{\th@break}
% \begin{macro}{\th@marginbreak}
% \begin{macro}{\th@changebreak}
% \begin{macro}{\th@change}
% \begin{macro}{\th@margin}
%    The different styles are defined in macros such as
%    |\th@plain|.  Since memory space is precious in
%    ``non-Big-versions'', we have to avoid offering too many unused
%    definitions.  Therefore we define these styles in separate files
%    that can be loaded on demand.  Thus the commands themselves only
%    load these files. We use |\@input@|, a \LaTeXe{} internal command
%    that ensures that the file will be listed with |\listfiles|.
% \changes{v2.2a}{94/02/02}{Use `@input@ to load theorem layout files}
%    \begin{macrocode}
\gdef\th@plain{\@input@{thp.sty}}
\gdef\th@break{\@input@{thb.sty}}
\gdef\th@marginbreak{\@input@{thmb.sty}}
\gdef\th@changebreak{\@input@{thcb.sty}}
\gdef\th@change{\@input@{thc.sty}}
\gdef\th@margin{\@input@{thm.sty}}
%    \end{macrocode}
%    This list will be expanded when new styles become
%    available.  For testing,  just append new theorem substyles
%    as document options.
% \end{macro}
% \end{macro}
% \end{macro}
% \end{macro}
% \end{macro}
% \end{macro}
%
%
%
% \subsection{Definition of a new theorem set}
%
% As already pointed out, a new theorem environment can be defined in
% three different ways:
% \begin{verbatim}
%    \newtheorem{Lem}{Lemma}
%    \newtheorem{Lem}{Lemma}[section]
%    \newtheorem{Lem}[Theorem]{Lemma}
%\end{verbatim}
% The function of the macro |\newtheorem| is to recognize these cases
% and then to branch into one of the three macros |\@ynthm|, |\@xnthm|
% or |\@othm|.  This mechanism is adopted unchanged from
% \cite{latex:source}; the essential point here is that, for example,
% in the second case, the arguments \texttt{Lem}, \texttt{Lemma} and
% \texttt{section} are passed over to the macro |\@xnthm|.
%
% We inspect this case first because the others present fewer problems,
% and thus are easily derived from this one.
%
% \begin{macro}{\@xnthm}
%    For our example arguments, the macro |\@xnthm| must fulfill
%    the following:
%    \begin{itemize}
%       \item
%          Define a new \LaTeX-counter `Lem'
%       \item
%          reset this counter within a |\section|
%       \item
%          define the macro |\theLem|
%       \item
%          define the environment macros |\Lem| and
%          |\endLem| using the current
%          |\theoremstyle| and |\theorem@bodyfont|.
%    \end{itemize}
%    Obviously, all this should happen only if the first argument
%    of |\@xnthm| (i.e.\ \texttt{Lem} in our example) is chosen so
%    as not to conflict with any previously defined commands or
%    environments.  This test is performed by the \LaTeX{} macro
%    |\@ifdefinable|.
%    \begin{macrocode}
\gdef\@xnthm#1#2[#3]{\expandafter\@ifdefinable\csname #1\endcsname
%    \end{macrocode}
%    Therefore, the first argument of |\@ifdefinable| is the
%    expansion (in the example, |\Lem|) of
%    |\csname||#1\endcsname|.  The second argument is
%    executed only if the test has been completed successfully.
%    \begin{macrocode}
   {%
%    \end{macrocode}
%    Now we define the new counter. The names of the \LaTeX{}
%    macros employed should speak for themselves:
% \changes{v2.2b}{1995/10/07}{Use \cs{@newctr}}
% \changes{v2.2c}{1995/11/23}{Use \cs{@newctr} properly.}
%    \begin{macrocode}
    \@definecounter{#1}\@newctr{#1}[#3]%
%    \end{macrocode}
%    Using |\@newctr| will give a proper error message if the counter
%    in |#3| is not defined.
%    In defining `|\theLem|' we must generate the desired macro
%    name by use of |\expandafter| and |\csname|.
%    \begin{macrocode}
    \expandafter\xdef\csname the#1\endcsname
%    \end{macrocode}
%    An |\xdef| is used in order to make the definition global,
%    and to ensure that it contains the replacement texts of
%    |\@thmcountersep| and |\@thmcounter|.\footnote{These
%    two macros can be defined by the document style.  Their default
%    values produce a `\texttt{.}' as separation and an arabic
%    representation of the number.} However, not everything should be
%    expanded. For example, it saves space to use |\thesection|
%    instead of its---at times---lengthy expansion.
%    \begin{macrocode}
      {\expandafter \noexpand \csname the#3\endcsname
       \@thmcountersep \@thmcounter{#1}}%
%    \end{macrocode}
%    Thus with the defaults of \LaTeX{}, |\theLem| would be
%    replaced by the command sequence |\thesection.\arabic{Lem}|.
%
%    We will now look at the definition of the macro which is executed
%    at the beginning of the actual environment (in our example this
%    macro is |\Lem|). It should be noted that we use an
%    ``|\expandafter| trick'' to expand only certain parts of the
%    replacement text at the time of the definition.
%    \begin{macrocode}
    \def\@tempa{\global\@namedef{#1}}%
    \expandafter \@tempa \expandafter{%
%    \end{macrocode}
%    First, the macro that contains the current definitions of
%    |\@begintheorem| and |\@opargtheorem| should be called
%    up.  The name of this macro---as is already known---has the form
%    |\th@|\meta{theorem\ style}; therefore, it must be called by
%    \begin{macrocode}
      \csname th@\the \theorem@style
%    \end{macrocode}
%    In addition the default theorem font should be changeable, i.e.\
%    we have to insert the contents of |\theorem@bodyfont|. For
%    that reason, we expand even further, beyond |\endcsname|,
%    and thus insert the contents of the \textsf{token} register
%    |\theorem@bodyfont| in the replacement text.
%    \begin{macrocode}
            \expandafter \endcsname \the \theorem@bodyfont
%    \end{macrocode}
%    Now it is time to call the macro |\@thm| which takes over
%    the further processing.  It has two arguments: the current
%    counter name (in our example, \texttt{Lem}), and the text of the
%    label (in our example, \texttt{Lemma}).
%    \begin{macrocode}
     \@thm{#1}{#2}}%
%    \end{macrocode}
%    With this, the `sub-definition' is complete.  The macro
%    |\@endtheorem| ends a theorem environment and is, so far, nothing
%    but an |\endtrivlist|.  (Hence it is defined globally, and not
%    within the theorem styles.\footnote{This has to be changed as
%    soon as theorem styles that change \texttt{\bslash @endtheorem}
%    exist.  In such a case, all existing styles must be changed as
%    well since they will have to reset the macro.}) Therefore, we can
%    set it equivalent to the macro that ends the theorem set (in our
%    example, |\endLem|).  However, if some day theorem styles exist
%    that do change |\@endtheorem|, we would have to use the
%    commented-out line instead.
%    \begin{macrocode}
    \global \expandafter \let \csname end#1\endcsname \@endtheorem
%   \global\@namedef{end#1}{\@endtheorem}%
%    \end{macrocode}
%    With these commands all the required definitions are employed,
%    unless the test |\@ifdefinable| has failed.  Therefore, we
%    end the second argument of this macro and with it the definition
%    of |\@xnthm|.
%    \begin{macrocode}
   }}
%    \end{macrocode}
% \end{macro}
%
% \begin{macro}{\@ynthm}
%    The definition of |\@ynthm| is completely analogous. In
%    this case the new counter that is defined is not reset within
%    another counter; thus the definition of |\the...|
%    is simplified:
%    \begin{macrocode}
\gdef\@ynthm#1#2{\expandafter\@ifdefinable\csname #1\endcsname
   {\@definecounter{#1}%
    \expandafter\xdef\csname the#1\endcsname{\@thmcounter{#1}}%
%    \end{macrocode}
%    The rest of the definition corresponds literally to that of
%    |\@xnthm|:
%    \begin{macrocode}
    \def\@tempa{\global\@namedef{#1}}\expandafter \@tempa
     \expandafter{\csname th@\the \theorem@style \expandafter
     \endcsname \the\theorem@bodyfont \@thm{#1}{#2}}%
    \global \expandafter \let \csname end#1\endcsname \@endtheorem}}
%    \end{macrocode}
% \end{macro}
%
% \begin{macro}{\@othm}
%    The definition of |\@othm| does not contain anything new.
%    \begin{macrocode}
\gdef\@othm#1[#2]#3{%
%    \end{macrocode}
% \changes{v2.2b}{95/10/07}{Build in check for counter defined}
%    We do not define a new counter but instead use one that has
%    already been defined.  Thus the only definition we need is that
%    of this pseudo-counter (i.e.\ |\the|\meta{env.\ name}).
%    First we check if |#2| corresponds to a known counter name.
%    \begin{macrocode}
  \expandafter\ifx\csname c@#2\endcsname\relax
   \@nocounterr{#2}%
  \else
   \expandafter\@ifdefinable\csname #1\endcsname
   {\expandafter \xdef \csname the#1\endcsname
     {\expandafter \noexpand \csname the#2\endcsname}%
%    \end{macrocode}
%    All other parts of the definition can be adopted from
%    |\@xnthm|. We have to remember, though, that in this case
%    the name of the current counter and the theorem label have moved
%    to the second and third arguments.
%    \begin{macrocode}
    \def\@tempa{\global\@namedef{#1}}\expandafter \@tempa
     \expandafter{\csname th@\the \theorem@style \expandafter
     \endcsname \the\theorem@bodyfont \@thm{#2}{#3}}%
    \global \expandafter \let \csname end#1\endcsname \@endtheorem}%
  \fi}
%    \end{macrocode}
% \end{macro}
%
% \subsection{Macros that are employed in a {\envfont theorem}
%    environment}
%
% \begin{macro}{\@thm}
%    The macro |\@thm| has to increase the current counter.  Then,
%    depending on whether the environment has (or does not have) an
%    optional argument, it has to branch into either |\@begintheorem|
%    or |\@opargtheorem|.
%    \begin{macrocode}
\gdef\@thm#1#2{\refstepcounter{#1}%
%    \end{macrocode}
%    Now we start a {\envfont trivlist} environment, and give
%    |\@topsep| and |\@topsepadd| the values of the \textsf{skip}
%    registers |\theorempreskipamount| and |\theorempostskipamount|.
%    The value in |\@topsep| is the vertical space that is inserted by
%    the first (and only) |\item| in our |\trivlist| whilst
%    |\@topsepadd| is inserted by |\@endparenv| at the end of that
%    {\envfont trivlist} environment.  By using these registers, we
%    obtain the desired space around a {\envfont theorem} environment.
%    \begin{macrocode}
   \trivlist
   \@topsep \theorempreskipamount               % used by first \item
   \@topsepadd \theorempostskipamount           % used by \@endparenv
%    \end{macrocode}
%    Now we have to test whether an optional argument has been given.
% \changes{v2.0d}{89/07/18}{Forgotten percent added.}
%    \begin{macrocode}
   \@ifnextchar [%
%    \end{macrocode}
%    If there is an optional argument, we will call |\@ythm|, and
%    move the arguments read back into the input stream.
%    \begin{macrocode}
   {\@ythm{#1}{#2}}%
%    \end{macrocode}
%    If not, we call |\@begintheorem|. Its first argument is the
%    name of the theorem set (hence the second argument of
%    |\@thm|). Its second argument is the macro that produces the
%    current number.
%    \begin{macrocode}
   {\@begintheorem{#2}{\csname the#1\endcsname}\ignorespaces}}
%    \end{macrocode}
% \end{macro}
%
% \begin{macro}{\@xthm}
% \begin{macro}{\@ythm}
%    Both these macros were originally called by |\@thm|.  We do
%    not need |\@xthm| anymore, hence we reset it to
%    |\relax|. The definition of |\@ythm| has not changed at
%    all from its definition in \LaTeX. In order to make the macros
%    easier to understand, we will
%    nevertheless present it (commented out).
%    \begin{macrocode}
\global\let\@xthm\relax
% \def\@ythm#1#2[#3]{\@opargbegintheorem{#2}{\csname
%           the#1\endcsname}{#3}\ignorespaces}
%</package>
%    \end{macrocode}
%    The primitive |\ignorespaces| in |\@ythm| and
%    |\@thm| is needed to remove the spaces between the
%    |\begin{...}| and the actual text.
% \end{macro}
% \end{macro}
%
%
% \subsection{Definition of the theorem substyles}
%
% As already pointed out, the theorem substyles, defined below, are
% only loaded when necessary.  Note that all these substyles, except
% \texttt{plain}, have |\slshape| as the default body font.
%
% \subsubsection{The \texttt{plain} style}
%
% \label{dok:th@plain} As the following macros use |@|, we have
% to locally set the |\catcode| of this symbol to ``letter''.
% This happens within a group, so that we do not have to worry about
% which |\catcode| that symbol had before.
%    \begin{macrocode}
%<*thp>
\begingroup \makeatletter
%    \end{macrocode}
% Since we are now within a group, we must make all definitions
% globally.  First we make sure that \texttt{theorem.sty} is loaded.
% This will allow us to use this file as a document style option
% without having to call \texttt{theorem} itself as an option.  At the
% same time, we assure that at least version 2 is loaded, since
% |\theorem@style| was not defined in earlier versions.
%    \begin{macrocode}
\@ifundefined{theorem@style}{\input{theorem.sty}}{}
\ProvidesFile{thp.sty}
             [\FMithmInfo]
%    \end{macrocode}
%
%
% \begin{macro}{\th@plain}
% \begin{macro}{\@begintheorem}
% \begin{macro}{\@opargbegintheorem}
%    |\theoremstyle{plain}| corresponds to the original
%    definition, except that the distances to the surrounding text are
%    determined by the parameters |\theorempreskipamount| and
%    |\theorempostskipamount|.
%    First we set the default body font.
%    \begin{macrocode}
\gdef\th@plain{\normalfont\itshape
%    \end{macrocode}
%    Then we define |\@begintheorem| and
%    |\@opargbegintheorem|. These two macros define how the
%    header of a theorem is typeset.  |\@opargbegintheorem| will
%    be called if a \textsf{theorem} environment with an optional
%    argument is encountered; otherwise, the header is constructed by
%    calling |\@begintheorem|.  If one of these macros is
%    executed, we are within a \textsf{trivlist} environment started by
%    |\@thm|.  So the theorem header is produced with an
%    |\item| command.
%
%    Instead of specifying the header font directly, all standard
%    theorem styles use the |\theorem@headerfont| macro to allow
%    customization.  The extra space (|\labelsep|) is necessary
%    because of problems in the \textsf{trivlist} environment.
%    \begin{macrocode}
  \def\@begintheorem##1##2{%
        \item[\hskip\labelsep \theorem@headerfont ##1\ ##2]}%
%    \end{macrocode}
%    The definition of |\@opargbegintheorem| is completely
%    analogous. The only difference is the fact that there exists a
%    third argument (which is the optional parameter of the
%    environment and contains additional information about the
%    theorem).  Customarily we enclose it in parentheses.
%    \begin{macrocode}
\def\@opargbegintheorem##1##2##3{%
   \item[\hskip\labelsep \theorem@headerfont ##1\ ##2\ (##3)]}}
%    \end{macrocode}
% \end{macro}
% \end{macro}
% \end{macro}
%
% \noindent
% We conclude with an |\endgroup| to restore the
% |\catcode| of |@|.
%    \begin{macrocode}
\endgroup
%</thp>
%    \end{macrocode}
%
%
% \subsubsection{The \texttt{break} style}
%
% This style option is stored in the file |thb.sty|.
% For the next two lines see the documentation for |\th@plain| on
% page~\pageref{dok:th@plain}.
%    \begin{macrocode}
%<*thb>
\begingroup \makeatletter
\@ifundefined{theorem@style}{\input{theorem.sty}}{}
\ProvidesFile{thb.sty}
             [\FMithmInfo]
%    \end{macrocode}
%
% \begin{macro}{\th@break}
%    |\theoremstyle{break}| produces a line break after the name
%    of the theorem.  The font is |\slshape|.  Hence, we define
%    |\th@break| as follows:
%    \begin{macrocode}
\gdef\th@break{\normalfont\slshape
  \def\@begintheorem##1##2{\item[%
%    \end{macrocode}
%    We run into the following problem: it is not possible to create
%    the header with |\item[|\meta{title}|]| and then start
%    a new line by, for example, |\mbox{}\\|. Such a definition
%    will fail whenever a list environment follows immediately.  With
%    the above construction, the |\mbox{}| causes the switch
%    |@inlabel| (cf.\ definition of |\list| and
%    |\trivlist| in \cite{latex:source}) to be set to \texttt{false}
%    and so the following list will insert additional vertical
%    space (|\topskip|). This is quite annoying. Therefore, we
%    create the line break within the |\item|. In order to ensure
%    that the text will begin at the proper position in the following
%    line, we simply pretend that the label does not take any
%    room.\footnote{This will lead to problems whenever very high
%    symbols occurring in the line tower into the
%    heading. So, something else has to be done here sometime.}
%    \begin{macrocode}
       \rlap{\vbox{\hbox{\hskip \labelsep\theorem@headerfont ##1\ ##2}%
                   \hbox{\strut}}}]}%
%    \end{macrocode}
%    Again, the definition of |\@opargbegintheorem| is
%    completely analogous.
%    \begin{macrocode}
\def\@opargbegintheorem##1##2##3{%
  \item[\rlap{\vbox{\hbox{\hskip \labelsep \theorem@headerfont
                     ##1\ ##2\ (##3)}%
                    \hbox{\strut}}}]}}
\endgroup
%</thb>
%    \end{macrocode}
% \end{macro}
%
%
% \subsubsection{The \texttt{changebreak} style}
%
%    \begin{macrocode}
% This style option is stored in the file |thcb.sty|.
%    \begin{macrocode}
%<*thcb>
\begingroup \makeatletter
\@ifundefined{theorem@style}{\input{theorem.sty}}{}
\ProvidesFile{thcb.sty}
             [\FMithmInfo]
%    \end{macrocode}
%
% \begin{macro}{\th@changebreak}
%    The \texttt{change-break} theorem style is like \texttt{break}
%    but with interchange of theorem name and theorem number.  Thus we
%    define |\th@changebreak| as follows:
%    \begin{macrocode}
\gdef\th@changebreak{\normalfont\slshape
  \def\@begintheorem##1##2{\item
       [\rlap{\vbox{\hbox{\hskip\labelsep \theorem@headerfont ##2\ ##1}%
                    \hbox{\strut}}}]}%
\def\@opargbegintheorem##1##2##3{%
  \item[\rlap{\vbox{\hbox{\hskip\labelsep \theorem@headerfont
                     ##2\ ##1\ (##3)}%
                    \hbox{\strut}}}]}}
\endgroup
%</thcb>
%    \end{macrocode}
% \end{macro}
%
%
% \subsubsection{The \texttt{change} style}
%
% This style option is stored in the file |thc.sty|.
%    \begin{macrocode}
%<*thc>
\begingroup \makeatletter
\@ifundefined{theorem@style}{\input{theorem.sty}}{}
\ProvidesFile{thc.sty}
             [\FMithmInfo]
%    \end{macrocode}
%
% \begin{macro}{\th@change}
%    The \texttt{change} theorem style corresponds to the
%    \texttt{change break} style without a linebreak after the header.
%    To say it in another way, it's the same as the \texttt{plain}
%    style but with number and name interchanged and |\slshape| as the
%    default font.
%    \begin{macrocode}
\gdef\th@change{\normalfont\slshape
  \def\@begintheorem##1##2{\item
       [\hskip\labelsep \theorem@headerfont ##2\ ##1]}%
\def\@opargbegintheorem##1##2##3{%
  \item[\hskip\labelsep \theorem@headerfont ##2\ ##1\ (##3)]}}
\endgroup
%</thc>
%    \end{macrocode}
% \end{macro}
%
%
% \subsubsection{The \texttt{marginbreak} style}
%
% This style option is the one used most often at Mainz.  It is saved
% in the file |thmb.sty|.
%    \begin{macrocode}
%<*thmb>
\begingroup \makeatletter
\@ifundefined{theorem@style}{\input{theorem.sty}}{}
\ProvidesFile{thmb.sty}
             [\FMithmInfo]
%    \end{macrocode}
%
% \begin{macro}{\th@marginbreak}
%    The \texttt{margin break} style is nearly the same as the
%    \texttt{change break} style. The only difference is the placement
%    of the theorem number.  We use |\llap| to place it in the left
%    margin.
%
%    In this style |\labelsep| denotes the separation between the
%    number and the text.
%    \begin{macrocode}
\gdef\th@marginbreak{\normalfont\slshape
  \def\@begintheorem##1##2{\item
       [\rlap{\vbox{\theorem@headerfont
                    \hbox{\llap{##2}\hskip\labelsep ##1}%
                    \hbox{\strut}}}]}%
\def\@opargbegintheorem##1##2##3{%
  \item[\rlap{\vbox{\theorem@headerfont
                    \hbox{\llap{##2}\hskip\labelsep ##1\ (##3)}%
                    \hbox{\strut}}}]}}
\endgroup
%</thmb>
%    \end{macrocode}
% \changes{v2.1c}{90/02/17}{`endgroup outside macrocode env.}
% \end{macro}
%
%
%
% \subsubsection{The \texttt{margin} style}
%
% This style option is stored in the file |thm.sty|.
%    \begin{macrocode}
%<*thm>
\begingroup \makeatletter
\@ifundefined{theorem@style}{\input{theorem.sty}}{}
\ProvidesFile{thm.sty}
             [\FMithmInfo]
%    \end{macrocode}
%
% \begin{macro}{\th@margin}
%    Again this is only a variant of the theorem styles described
%    above without any new ideas.
%    \begin{macrocode}
\gdef\th@margin{\normalfont\slshape
  \def\@begintheorem##1##2{\item
       [\theorem@headerfont \llap{##2}\hskip\labelsep ##1]}%
\def\@opargbegintheorem##1##2##3{%
  \item[\theorem@headerfont \llap{##2}\hskip\labelsep ##1\ (##3)]}}
\endgroup
%</thm>
%    \end{macrocode}
% \end{macro}
%
%
%
%
% \subsection{Final Definitions}
%
% \begin{macro}{\theorempreskipamount}
% \begin{macro}{\theorempostskipamount}
%    The \textsf{skip} parameters that regulate the vertical empty space
%    before and after the {\envfont theorem} environment have
%    to be allocated as well.
% \changes{v2.0f}{89/07/20}{Unnecessary `global' removed.}
%    \begin{macrocode}
%<*package>
\newskip\theorempreskipamount
\newskip\theorempostskipamount
%    \end{macrocode}
%    Since we have used the same values for all theorem sets, we now can
%    assign them.
%    \begin{macrocode}
\global\setlength\theorempreskipamount{12pt plus 5pt minus 3pt}
\global\setlength\theorempostskipamount{8pt plus 3pt minus 1.5pt}
%    \end{macrocode}
% \end{macro}
% \end{macro}
%
%
% \begin{macro}{\@endtheorem}
%    The same holds for the macro |\@endtheorem|, which ends a
%    {\envfont theorem} environment. Since it is the same for all
%    theorem sets, it is removed from the macros |\th@|\meta{style}.
%    It simply ends the {\envfont trivlist} environment, which was
%    begun in |\@thm|.
%    \begin{macrocode}
\global\let\@endtheorem=\endtrivlist
%    \end{macrocode}
% \end{macro}
%
%
%
% \begin{macro}{\@preamblecmds}
%    All macros defined above are to be used only in the preamble.
%    Therefore, we insert them in |\@preamblecmds| which will disable
%    them at begin document. This is done by the internal \LaTeXe{}
%    command |\@onlypreamble|.
% \changes{v2.0g}{89/09/19}{`theoremheaderfont and `theorembodyfont
%    added.}
% \changes{v2.2a}{94/02/03}{Use \LaTeXe{} interface}
%    \begin{macrocode}
\@onlypreamble\@xnthm
\@onlypreamble\@ynthm
\@onlypreamble\@othm
\@onlypreamble\newtheorem
\@onlypreamble\theoremstyle
\@onlypreamble\theorembodyfont
\@onlypreamble\theoremheaderfont
%    \end{macrocode}
% \end{macro}
%
% Finally we declare the \texttt{plain} theorem style to be the default.
% \changes{v2.1b}{90/06/04}{Fixed missing `item by declaring default
%    style.}
%    \begin{macrocode}
\theoremstyle{plain}
%</package>
%    \end{macrocode}
%
% \Finale

\endinput
}{}
\ProvidesFile{thcb.sty}
             [\FMithmInfo]
%    \end{macrocode}
%
% \begin{macro}{\th@changebreak}
%    The \texttt{change-break} theorem style is like \texttt{break}
%    but with interchange of theorem name and theorem number.  Thus we
%    define |\th@changebreak| as follows:
%    \begin{macrocode}
\gdef\th@changebreak{\normalfont\slshape
  \def\@begintheorem##1##2{\item
       [\rlap{\vbox{\hbox{\hskip\labelsep \theorem@headerfont ##2\ ##1}%
                    \hbox{\strut}}}]}%
\def\@opargbegintheorem##1##2##3{%
  \item[\rlap{\vbox{\hbox{\hskip\labelsep \theorem@headerfont
                     ##2\ ##1\ (##3)}%
                    \hbox{\strut}}}]}}
\endgroup
%</thcb>
%    \end{macrocode}
% \end{macro}
%
%
% \subsubsection{The \texttt{change} style}
%
% This style option is stored in the file |thc.sty|.
%    \begin{macrocode}
%<*thc>
\begingroup \makeatletter
\@ifundefined{theorem@style}{% \iffalse meta-comment
%
% Copyright (C) 1993-2024
%
% The LaTeX Project and any individual authors listed elsewhere
% in this file.
%
% This file is part of the Standard LaTeX `Tools Bundle'.
% -------------------------------------------------------
%
% It may be distributed and/or modified under the
% conditions of the LaTeX Project Public License, either version 1.3c
% of this license or (at your option) any later version.
% The latest version of this license is in
%    https://www.latex-project.org/lppl.txt
% and version 1.3c or later is part of all distributions of LaTeX
% version 2005/12/01 or later.
%
% The list of all files belonging to the LaTeX `Tools Bundle' is
% given in the file `manifest.txt'.
%
% \fi
%
%%
%
%
\def\FMithmInfo{2023/07/05 v2.2c Theorem extension package (FMi)}
%
% \ProvidesFile{theorem.dtx}[\FMithmInfo]
%\iffalse   % this is a METACOMMENT !
%%
%% Package `theorem' to use with LaTeX2e
%% Copyright (C) 1989-1995 Frank Mittelbach, all rights reserved.
%<+package>\@ifundefined{theorem@style}{}{\endinput}
%<+package>\NeedsTeXFormat{LaTeX2e}[1995/06/01]
%<+package>\ProvidesPackage{theorem}
%<+package>         [\FMithmInfo]
%
% \fi
%
% \changes{v1.0}{87/05/07}{First draft.}
% \changes{v1.0a}{87/05/07}{`thistheoremstyle' added.}
% \changes{v1.1a}{87/07/29}{New concept: redefine the internal
%                          \protect\LaTeX{} commands `@xnthm' etc.}
% \changes{v1.1b}{87/07/31}{`theoremfont' used to specify font.}
% \changes{v1.2a}{88/01/08}{`@endtheorem' now globally defined.}
% \changes{v1.2b}{88/01/11}{`newtheorem' only allowed in the preamble.}
% \changes{v1.3a}{88/02/14}{`theoremfont' now a toks register.}
%
% \changes{v2.0a}{88/11/16}{New implementation and new documentation.}
% \changes{v2.0a}{88/11/16}{No layout changes but internal modifications
%                           old substyles won't work.}
% \changes{v2.0b}{89/01/30}{`global' used for all definitions.}
% \changes{v2.0c}{89/06/15}{Documentation translated to English.
%                            Thanks to LLL and Nina}
% \changes{v2.1a}{90/02/15}{Support for new font selection.}
% \changes{v2.2a}{94/02/03}{Cleanup for LaTeX2e}
%
% \DoNotIndex{\@endparenv,\@ifdefinable,\@ifnextchar,\@ifundefined}
% \DoNotIndex{\@namedef,\@nameuse,\@spaces,\@tempa,\@warning}
% \DoNotIndex{\ }
% \DoNotIndex{\begingroup,\bfseries}
% \DoNotIndex{\csname}
% \DoNotIndex{\def,\do,\docdate}
% \DoNotIndex{\endcsname,\endgroup,\endinput,\endtrivlist,\expandafter}
% \DoNotIndex{\filedate,\fileversion}
% \DoNotIndex{\gdef,\global}
% \DoNotIndex{\hbox,\hskip}
% \DoNotIndex{\ignorespaces,\input,\itshape,\item}
% \DoNotIndex{\labelsep,\let,\llap}
% \DoNotIndex{\makeatletter}
% \DoNotIndex{\noexpand}
% \DoNotIndex{\refstepcounter,\relax,\rlap}
% \DoNotIndex{\setlength,\slshape,\space,\string,\strut}
% \DoNotIndex{\the,\trivlist,\typeout}
% \DoNotIndex{\vbox}
% \DoNotIndex{\xdef}
%
% \def\deflabel#1{\ttfamily #1\hfill}
% \def\deflist#1{\begin{list}{}{\settowidth\labelwidth{\ttfamily #1}%
%                               \setlength\leftmargin\labelwidth
%                               \addtolength\leftmargin\labelsep
%                               \let\makelabel\deflabel}}
% \def\enddeflist{\end{list}}
% \def\envfont{\normalfont\ttfamily}
%
% \MakeShortVerb{\|}
%
% \newcommand{\TUB}{{\sl TUGboat\/}}
%
% \GetFileInfo{theorem.dtx}
% \title{An Extension of the
%        \LaTeX{} theorem environment\thanks{This file
%        has version number \fileversion, last
%        revised \filedate.}}
% \author{Frank Mittelbach}
%
% \MaintainedByLaTeXTeam{tools}
% \maketitle
%
% \begin{abstract}
%    The macros described in this paper yield an extension of the
%    \LaTeX{} theorem mechanism. It is designed to satisfy the
%    different requirements of various journals. Thus, the
%    layout of the ``theorems'' can be manipulated by determining a
%    ``style''. This article describes not only the use, but
%    also the definition, of the necessary macros.
% \end{abstract}
%
% \section*{Preface to version 2.2}
%
% For \LaTeXe{} this package did not need any fundamental changes. I
% only modified the messages generated so that theorem layout styles
% will show up with the |\listfiles| command and cleaned the section
% on the New Font Selection Scheme since this is now included in
% \LaTeX.
%
% \section*{Preface to version 2.1}
%
% This version is identical to 2.0g described in \TUB\ 10\#3 except
% for some internal defaults which are now set depending on the used
% font selection scheme.
%
% This was done to avoid unpleasant surprises if the new font selection
% scheme is in force. For further details see
% section~\ref{sec:newfonts} and \cite{bk:GMS94}.
%
% \section{Introduction}
%
% For our purposes here, ``theorems'' are labelled enunciations,
% often set off from the main text by extra space and a font change.
% Theorems, corollaries, conjectures, definitions, and remarks
% are all instances of ``theorems''.  The ``header'' of these structures
% is composed of a label (such as \textsc{Theorem} or \textsc{Remark})
% and a number which serializes an item in the sequence of items
% with the same label.
%
% Shortly after the introduction of \LaTeX{} at the Fachbereich
% Mathematik in Mainz, the desire to manipulate the layout of
% ``theorems'' arose.  In Mainz, the following two conventions came
% into general use:
% \begin{enumerate}
%   \item The number of the theorem is shown in the margin.
%   \item There is a line break at
%         the end of the theorem header.
% \end{enumerate}
%
% Additionally, some journals require different formats which depend on
% the ``sort of theorem'': e.g.\ often remarks and definitions are set
% in |\upshape|, while |\itshape| is employed for main theorems.
%
% Confronted with these requirements, a theorem environment was
% developed in Mainz which allows separate determination of the
% layout of the ``theorems sets'', comparable to |\pagestyle|.
%
%
% \section{The user interface}
% \subsection{Defining new theorem sets}
%
% \DescribeMacro\newtheorem
% As in the original \LaTeX{} version, the command |\newtheorem|
% defines a new ``theorem set'' or ``theorem-like structure''.
% Two required arguments name the new environment and give the text
% to be typeset with each instance of the new ``set'', while
% an optional argument determines how
% the ``set'' is enumerated:
% \begin{description}
%    \item[\ttfamily \bslash newtheorem\{foo\}\{bar\}]
%       The theorem set {\envfont foo} (whose name is \texttt{bar})
%       uses its own counter.
%    \item[\ttfamily \bslash newtheorem\{foo2\}{[foo]}\{bar2\}]
%       The theorem set {\envfont foo2} (printed name \texttt{bar2})
%       uses the same counter as the theorem set \texttt{foo}.
%    \item[\ttfamily \bslash newtheorem\{foo3\}\{bar3\}{[section]}]
%       The theorem set {\envfont foo3} (printed name \texttt{bar3}) is
%       enumerated within the counter \texttt{section}, i.e.\ with every
%       new |\section| the enumeration begins again with 1, and
%       the enumeration is composed from the section-number and the
%       theorem counter itself.
% \end{description}
%
% \DescribeMacro\theoremstyle
% Additionally, the command |\theoremstyle| can define the layout
% of various, or all, theorem sets.  It should be noted that any
% theorem set defined by |\newtheorem| is typeset in the
% |\theoremstyle| that is current at the time of the definition.
% Thus, the following
% \begin{verbatim}
%    \theoremstyle{break}        \newtheorem{Cor}{Corollary}
%    \theoremstyle{plain}        \newtheorem{Exa}{Example}[section]
%\end{verbatim}                             ^^A No blank at line start
% leads to the result that the set {\envfont Cor} is formatted in the
% style \texttt{break}, while the set {\envfont Exa} and all the
% following ones are formatted in the style \texttt{plain}, unless
% another |\theoremstyle| follows.  Since the definitions installed by
% |\newtheorem| are global, one also can limit |\theoremstyle| locally
% by grouping braces.
%
% \DescribeMacro\theorembodyfont
% The choice of the font for the theorem body is completely
% independent of the chosen |\theoremstyle|; this has proven to
% be very advantageous.  For example,
% \begin{verbatim}
%   {\theorembodyfont{\upshape}          \newtheorem{Rem}{Remark}}
%\end{verbatim}
% defines a theorem set {\envfont Rem}, which will be set in
% |\upshape| in the current layout (which in our example is
% \texttt{plain}).  As with |\theoremstyle|, the |\theorembodyfont|
% chosen is that current at the time of |\newtheorem|.  If
% |\theorembodyfont| is not specified or one defines
% |\theorembodyfont{}|, then the font used will be that defined by the
% |\theoremstyle|.
%
% \DescribeMacro\theoremheaderfont It is also possible to customize
% the font used for the theorem headers.  This is, however, a global
% declaration, and therefore there should be at most one
% |\theoremheaderfont| declaration in the preamble.\footnote{If it is
% actually necessary to have different header fonts, one has to define
% new theorem styles (substituting the desired font) or specify the
% information directly in the \texttt{\bslash newtheorem} declaration
% (the unclean variant).}
%
% \DescribeMacro\theorempreskipamount
% \DescribeMacro\theorempostskipamount Two additional parameters
% affect the vertical space around the theorem environments:
% |\theorempreskipamount| and |\theorempostskipamount| define,
% respectively, the spacing before and after such an environment.
% These parameters apply for all theorem sets and can be manipulated
% with the ordinary length macros.  They are rubber lengths
% (`\textsf{skips}'), and therefore can contain \texttt{plus} and
% \texttt{minus} parts.
%
% Since the definition of theorem sets should---most sensibly---be
% placed in the preamble, we only allow installation there.
% It is therefore possible to release the memory used here after
% |\begin{document}|, in order to make room for other
% applications.
%
%
% \subsection{Existing theorem styles}
%
% The following theorem styles exist to date:
% \begin{deflist}{marginbreak}
%    \item[plain]
%       This theorem style emulates the original \LaTeX{} definition,
%       except that additionally the parameters
%       |\theorem...skipamount| are used.
%    \item[break]
%       In this style, the theorem header is followed by a line
%       break.
%    \item[marginbreak]
%       The theorem number is set in the margin, and there is a line
%       break as in \texttt{break}.
%    \item[changebreak]
%       Like \texttt{break}, but with header number and text
%       interchanged.
%    \item[change]
%       Header number and text are interchanged, without a line break.
%    \item[margin]
%       The number is set in the left margin, without a line break.
% \end{deflist}
% All styles (except \texttt{plain}) select |\slshape| as the default
% |\theorembodyfont|.
%
%
% \subsection{Examples}
%
% Given the above theorem sets {\envfont Cor}, {\envfont Exa}
% and {\envfont Rem},
% suppose that the preamble also contains the declarations:
% \begin{verbatim}
%    \theoremstyle{marginbreak}   \newtheorem{Lem}[Cor]{Lemma}
%    \theoremstyle{change}
%    \theorembodyfont{\itshape}        \newtheorem{Def}[Cor]{Definition}
%
%    \theoremheaderfont{\scshape}
%\end{verbatim}
% Then the following are some typical examples of the typeset output
% resulting from their use.
%
% \begin{Cor}
%    This is a sentence typeset in the theorem environment {\envfont
%    Cor}.
% \end{Cor}
%
% \begin{Exa}
%    This is a sentence typeset in the theorem environment {\envfont
%    Exa}.
% \end{Exa}
%
% \begin{Rem}
%    This is a sentence typeset in the theorem environment {\envfont
%    Rem}.
% \end{Rem}
%
% \begin{Lem}[Ben User]
%    This is a sentence typeset in the theorem environment {\envfont
%    Lem}.
% \end{Lem}
%
% \begin{Def}[Very impressive Definition]
%    This is a sentence typeset in the theorem environment {\envfont
%    Def}.
% \end{Def}
%
% The last two examples show the effect of the optional argument to
% a theorem environment (it is the text typeset in parentheses).
%
%
% \section{Special Considerations}
% \label{sec:newfonts}
%
% Theoremheader and body are implemented as a unit. This means that
% the |\theoremheaderfont|  will inherit characteristics of the
% |\theorembodyfont| in \LaTeXe. Thus, if for example
% |\theorembodyfont| is |\itshape| and |\theoremheaderfont|
% is |\bfseries| the font selected for the header will have the
% characteristics `bold extended italic'. If this is not desired one
% should set the |\theoremheaderfont| to something like
% \begin{verbatim}
%      \theoremheaderfont{\normalfont\bfseries}
%\end{verbatim}
% i.e.\ supplying all necessary font information explicitly.
%
%
%
% \section{Acknowledgements}
%
% The publication of this set of macros was only possible with the
% help of Christina Busse (translating the manuscript into English),
% Joachim Pense (playing the r\^ole of typist), Chris Rowley
% (looking everything over) and many others providing useful
% suggestions.
%
% \MaybeStop{%
%    \begin{thebibliography}{9}
%       \bibitem{bk:GMS94} \textsc{M.~Goossens}, \textsc{F.~Mittelbach}
%         and \textsc{A.~Samarin}.
%         \newblock The \LaTeX{} Companion.
%         \newblock
%         Addison-Wesley, Reading, Massachusetts, 1994.
%       \bibitem{latex:source} \textsc{Lamport, Leslie.} \newblock
%                               \texttt{latex.tex}, version 2.09, date
%                               Feb. 1990.
%    \end{thebibliography}
%    \PrintIndex
%    \PrintChanges}
%
%
% \section{The documentation driver file}
%
% The next bit of code contains the documentation driver file for
% \TeX{}, i.e., the file that will produce the documentation you are
% currently reading. It will be extracted from this file by the
% \texttt{docstrip} program. Since it is the first code in the file
% one can alternatively process this file directly with \LaTeXe{} to
% obtain the documentation.
%
%    \begin{macrocode}
%<*driver>
\documentclass{ltxdoc}

\usepackage{theorem}

%<+driver>% The next few lines define theorem sets which are used
%<+driver>% in the example section of the documentation.

\theoremstyle{break}       \newtheorem{Cor}{Corollary}
\theoremstyle{plain}       \newtheorem{Exa}{Example}[section]
{\theorembodyfont{\upshape}\newtheorem{Rem}{Remark}}
\theoremstyle{marginbreak} \newtheorem{Lem}[Cor]{Lemma}
\theoremstyle{change}
\theorembodyfont{\itshape} \newtheorem{Def}[Cor]{Definition}

\theoremheaderfont{\scshape}

\RecordChanges

\begin{document}
   \DocInput{theorem.dtx}
\end{document}
%</driver>
%    \end{macrocode}
%
% \section{Definition of the Macros}
%
% If the file has been loaded before, we abort immediately. If not the
% package announces itself (this is actually done at the very top of
% the file---the way it is done isn't good style so don't copy it).
% \changes{v2.0e}{89/07/19}{Spaces removed from `typeout'.}
% \changes{v2.2b}{95/11/19}{Announce moved to top of file}
%    \begin{macrocode}
%<*package>
%\@ifundefined{theorem@style}{}{\endinput}
%\def\FMithmInfo{1995/11/19 v2.2b Theorem extension package (FMi)}
%\ProvidesPackage{theorem}[\FMithmInfo]
%    \end{macrocode}
%
%
% \subsection{Definition of theorem styles and fonts}
%
% All the definitions in this file are done globally to allow
% inputting this file inside a group.
%
% \begin{macro}{\theoremstyle}
%    Before a theorem style can be installed, the chosen style must be
%    known. For that reason, we must test to see that
%    |\th@|\meta{style} is known or, more precisely, that it is
%    different from |\relax|. If the style is not known then
%    |\th@plain| is used.
%    \begin{macrocode}
\gdef\theoremstyle#1{%
   \@ifundefined{th@#1}{\@warning
          {Unknown theoremstyle `#1'. Using `plain'}%
          \theorem@style{plain}}%
%    \end{macrocode}
%    We save the theorem style to be used in the token register
%    |\theorem@style|.
%    \begin{macrocode}
      {\theorem@style{#1}}%
%    \end{macrocode}
%    Now we ``evaluate'' the theorem style: this means, we call the
%    macro |\th@|\meta{style} which will activate the relevant
%    definitions which are contained in a separate file.  This is done
%    in a group to suppress changes to the current font.  This could
%    otherwise pose problems together with the new font selection
%    scheme\footnote{When I printed the original article using the new
%    font selection scheme I ended with a document with slanted
%    typefaces (text headings and all) simply because one of the
%    theorem styles used \texttt{\bslash sl} at toplevel.} if the
%    |\th@|\meta{style} is evaluated a second time.
% \changes{v2.1a}{90/02/15}{Group added around `th@..}
% \changes{v2.1a}{90/02/15}{Reseting of `..@begintheorem removed.}
%    \begin{macrocode}
      \begingroup
        \csname th@\the\theorem@style \endcsname
      \endgroup}
%    \end{macrocode}
% \end{macro}
%
%
% \begin{macro}{\@begintheorem}
% \changes{v2.1a}{90/02/15}{Relaxed at toplevel.}
% \begin{macro}{\@opargbegintheorem}
% \changes{v2.1a}{90/02/15}{Relaxed at toplevel.}
%    We reset |\@begintheorem| and |\@opargbegintheorem| to
%    |\relax| since these commands are no longer necessary at
%    toplevel. This will save a few tokens.
%    \begin{macrocode}
\global\let\@begintheorem\relax
\global\let\@opargbegintheorem\relax
%    \end{macrocode}
% \end{macro}
% \end{macro}
%
% \begin{macro}{\theorem@style}
%    Obviously the \textsf{token} register used above has to be
%    allocated. To assure the utmost compatibility with the
%    original \LaTeX{} definition, we set the default theorem style to
%    \texttt{plain}, which implements the usual \LaTeX{} convention.
% \changes{v2.0f}{89/07/20}{Unnecessary `global' removed.}
%    \begin{macrocode}
\newtoks\theorem@style
\global\theorem@style{plain}
%    \end{macrocode}
% \end{macro}
%
% \begin{macro}{\theorembodyfont}
% \begin{macro}{\theorem@bodyfont}
%    If the |\theorembodyfont| is set by the user then it should not
%    interact with the default font set in the theorem style. When the
%    new font selection is in force this may happen if, for example,
%    the default is |\itshape| and the new |\theorembodyfont| is
%    |\sffamily|. So we add a |\reset@font| command in front of the
%    user definition.
% \changes{v2.0f}{89/07/20}{Unnecessary `global' removed.}
% \changes{v2.0g}{89/09/19}{`theoremfont to `theorembodyfont.}
% \changes{v2.1c}{92/06/22}{`reset@font added and logic changed.}
% \changes{v2.1d}{92/10/28}{Reset to empty if arg is empty}
%    \begin{macrocode}
\gdef\theorembodyfont#1{%
%    \end{macrocode}
%    We check if the argument supplied is empty and if so put nothing
%    into the |\theorem@bodyfont| token register to allow for
%    |\theorembodyfont{}| as a mean of using the default of the
%    current |\theoremstyle|.
%    \begin{macrocode}
   \def\@tempa{#1}%
   \ifx\@tempa\@empty
    \theorem@bodyfont{}%
   \else
    \theorem@bodyfont{\reset@font#1}%
   \fi
   }
\newtoks\theorem@bodyfont
\global\theorem@bodyfont{}
%    \end{macrocode}
% \end{macro}
% \end{macro}
%
%
% \begin{macro}{\theoremheaderfont}
% \changes{v2.0g}{89/09/19}{Macro added.}
%    The font for the theorem headers is handled differently because
%    this definition applies to all theorem styles.
%    \begin{macrocode}
\gdef\theoremheaderfont#1{\gdef\theorem@headerfont{#1}%
%    \end{macrocode}
%    After using the macro once it is redefined to produce an error
%    message.
%    \begin{macrocode}
       \gdef\theoremheaderfont##1{%
        \typeout{\string\theoremheaderfont\space should be used
                 only once.}}}
%    \end{macrocode}
% \end{macro}
%
% \begin{macro}{\theorem@headerfont}
% \changes{v2.0g}{89/09/19}{Macro added.}
%    To set the |\theorem@headerfont| default we first test if
%    the new fontselection scheme is in force.
% \changes{v2.1a}{90/02/15}{Support for new font selection.}
%    \begin{macrocode}
\ifx\upshape\undefined
%    \end{macrocode}
%    If not we define it to expand into |\bfseries|. We don't use
%    |\let| just in case a following style option redefines this
%    macro.
%    \begin{macrocode}
\gdef\theorem@headerfont{\bfseries}
%    \end{macrocode}
%    Otherwise we reset the current shape before calling |\bfseries|.
%    \begin{macrocode}
\else \gdef\theorem@headerfont{\normalfont\bfseries}\fi
%    \end{macrocode}
% \end{macro}
%
% \begin{macro}{\th@plain}
% \begin{macro}{\th@break}
% \begin{macro}{\th@marginbreak}
% \begin{macro}{\th@changebreak}
% \begin{macro}{\th@change}
% \begin{macro}{\th@margin}
%    The different styles are defined in macros such as
%    |\th@plain|.  Since memory space is precious in
%    ``non-Big-versions'', we have to avoid offering too many unused
%    definitions.  Therefore we define these styles in separate files
%    that can be loaded on demand.  Thus the commands themselves only
%    load these files. We use |\@input@|, a \LaTeXe{} internal command
%    that ensures that the file will be listed with |\listfiles|.
% \changes{v2.2a}{94/02/02}{Use `@input@ to load theorem layout files}
%    \begin{macrocode}
\gdef\th@plain{\@input@{thp.sty}}
\gdef\th@break{\@input@{thb.sty}}
\gdef\th@marginbreak{\@input@{thmb.sty}}
\gdef\th@changebreak{\@input@{thcb.sty}}
\gdef\th@change{\@input@{thc.sty}}
\gdef\th@margin{\@input@{thm.sty}}
%    \end{macrocode}
%    This list will be expanded when new styles become
%    available.  For testing,  just append new theorem substyles
%    as document options.
% \end{macro}
% \end{macro}
% \end{macro}
% \end{macro}
% \end{macro}
% \end{macro}
%
%
%
% \subsection{Definition of a new theorem set}
%
% As already pointed out, a new theorem environment can be defined in
% three different ways:
% \begin{verbatim}
%    \newtheorem{Lem}{Lemma}
%    \newtheorem{Lem}{Lemma}[section]
%    \newtheorem{Lem}[Theorem]{Lemma}
%\end{verbatim}
% The function of the macro |\newtheorem| is to recognize these cases
% and then to branch into one of the three macros |\@ynthm|, |\@xnthm|
% or |\@othm|.  This mechanism is adopted unchanged from
% \cite{latex:source}; the essential point here is that, for example,
% in the second case, the arguments \texttt{Lem}, \texttt{Lemma} and
% \texttt{section} are passed over to the macro |\@xnthm|.
%
% We inspect this case first because the others present fewer problems,
% and thus are easily derived from this one.
%
% \begin{macro}{\@xnthm}
%    For our example arguments, the macro |\@xnthm| must fulfill
%    the following:
%    \begin{itemize}
%       \item
%          Define a new \LaTeX-counter `Lem'
%       \item
%          reset this counter within a |\section|
%       \item
%          define the macro |\theLem|
%       \item
%          define the environment macros |\Lem| and
%          |\endLem| using the current
%          |\theoremstyle| and |\theorem@bodyfont|.
%    \end{itemize}
%    Obviously, all this should happen only if the first argument
%    of |\@xnthm| (i.e.\ \texttt{Lem} in our example) is chosen so
%    as not to conflict with any previously defined commands or
%    environments.  This test is performed by the \LaTeX{} macro
%    |\@ifdefinable|.
%    \begin{macrocode}
\gdef\@xnthm#1#2[#3]{\expandafter\@ifdefinable\csname #1\endcsname
%    \end{macrocode}
%    Therefore, the first argument of |\@ifdefinable| is the
%    expansion (in the example, |\Lem|) of
%    |\csname||#1\endcsname|.  The second argument is
%    executed only if the test has been completed successfully.
%    \begin{macrocode}
   {%
%    \end{macrocode}
%    Now we define the new counter. The names of the \LaTeX{}
%    macros employed should speak for themselves:
% \changes{v2.2b}{1995/10/07}{Use \cs{@newctr}}
% \changes{v2.2c}{1995/11/23}{Use \cs{@newctr} properly.}
%    \begin{macrocode}
    \@definecounter{#1}\@newctr{#1}[#3]%
%    \end{macrocode}
%    Using |\@newctr| will give a proper error message if the counter
%    in |#3| is not defined.
%    In defining `|\theLem|' we must generate the desired macro
%    name by use of |\expandafter| and |\csname|.
%    \begin{macrocode}
    \expandafter\xdef\csname the#1\endcsname
%    \end{macrocode}
%    An |\xdef| is used in order to make the definition global,
%    and to ensure that it contains the replacement texts of
%    |\@thmcountersep| and |\@thmcounter|.\footnote{These
%    two macros can be defined by the document style.  Their default
%    values produce a `\texttt{.}' as separation and an arabic
%    representation of the number.} However, not everything should be
%    expanded. For example, it saves space to use |\thesection|
%    instead of its---at times---lengthy expansion.
%    \begin{macrocode}
      {\expandafter \noexpand \csname the#3\endcsname
       \@thmcountersep \@thmcounter{#1}}%
%    \end{macrocode}
%    Thus with the defaults of \LaTeX{}, |\theLem| would be
%    replaced by the command sequence |\thesection.\arabic{Lem}|.
%
%    We will now look at the definition of the macro which is executed
%    at the beginning of the actual environment (in our example this
%    macro is |\Lem|). It should be noted that we use an
%    ``|\expandafter| trick'' to expand only certain parts of the
%    replacement text at the time of the definition.
%    \begin{macrocode}
    \def\@tempa{\global\@namedef{#1}}%
    \expandafter \@tempa \expandafter{%
%    \end{macrocode}
%    First, the macro that contains the current definitions of
%    |\@begintheorem| and |\@opargtheorem| should be called
%    up.  The name of this macro---as is already known---has the form
%    |\th@|\meta{theorem\ style}; therefore, it must be called by
%    \begin{macrocode}
      \csname th@\the \theorem@style
%    \end{macrocode}
%    In addition the default theorem font should be changeable, i.e.\
%    we have to insert the contents of |\theorem@bodyfont|. For
%    that reason, we expand even further, beyond |\endcsname|,
%    and thus insert the contents of the \textsf{token} register
%    |\theorem@bodyfont| in the replacement text.
%    \begin{macrocode}
            \expandafter \endcsname \the \theorem@bodyfont
%    \end{macrocode}
%    Now it is time to call the macro |\@thm| which takes over
%    the further processing.  It has two arguments: the current
%    counter name (in our example, \texttt{Lem}), and the text of the
%    label (in our example, \texttt{Lemma}).
%    \begin{macrocode}
     \@thm{#1}{#2}}%
%    \end{macrocode}
%    With this, the `sub-definition' is complete.  The macro
%    |\@endtheorem| ends a theorem environment and is, so far, nothing
%    but an |\endtrivlist|.  (Hence it is defined globally, and not
%    within the theorem styles.\footnote{This has to be changed as
%    soon as theorem styles that change \texttt{\bslash @endtheorem}
%    exist.  In such a case, all existing styles must be changed as
%    well since they will have to reset the macro.}) Therefore, we can
%    set it equivalent to the macro that ends the theorem set (in our
%    example, |\endLem|).  However, if some day theorem styles exist
%    that do change |\@endtheorem|, we would have to use the
%    commented-out line instead.
%    \begin{macrocode}
    \global \expandafter \let \csname end#1\endcsname \@endtheorem
%   \global\@namedef{end#1}{\@endtheorem}%
%    \end{macrocode}
%    With these commands all the required definitions are employed,
%    unless the test |\@ifdefinable| has failed.  Therefore, we
%    end the second argument of this macro and with it the definition
%    of |\@xnthm|.
%    \begin{macrocode}
   }}
%    \end{macrocode}
% \end{macro}
%
% \begin{macro}{\@ynthm}
%    The definition of |\@ynthm| is completely analogous. In
%    this case the new counter that is defined is not reset within
%    another counter; thus the definition of |\the...|
%    is simplified:
%    \begin{macrocode}
\gdef\@ynthm#1#2{\expandafter\@ifdefinable\csname #1\endcsname
   {\@definecounter{#1}%
    \expandafter\xdef\csname the#1\endcsname{\@thmcounter{#1}}%
%    \end{macrocode}
%    The rest of the definition corresponds literally to that of
%    |\@xnthm|:
%    \begin{macrocode}
    \def\@tempa{\global\@namedef{#1}}\expandafter \@tempa
     \expandafter{\csname th@\the \theorem@style \expandafter
     \endcsname \the\theorem@bodyfont \@thm{#1}{#2}}%
    \global \expandafter \let \csname end#1\endcsname \@endtheorem}}
%    \end{macrocode}
% \end{macro}
%
% \begin{macro}{\@othm}
%    The definition of |\@othm| does not contain anything new.
%    \begin{macrocode}
\gdef\@othm#1[#2]#3{%
%    \end{macrocode}
% \changes{v2.2b}{95/10/07}{Build in check for counter defined}
%    We do not define a new counter but instead use one that has
%    already been defined.  Thus the only definition we need is that
%    of this pseudo-counter (i.e.\ |\the|\meta{env.\ name}).
%    First we check if |#2| corresponds to a known counter name.
%    \begin{macrocode}
  \expandafter\ifx\csname c@#2\endcsname\relax
   \@nocounterr{#2}%
  \else
   \expandafter\@ifdefinable\csname #1\endcsname
   {\expandafter \xdef \csname the#1\endcsname
     {\expandafter \noexpand \csname the#2\endcsname}%
%    \end{macrocode}
%    All other parts of the definition can be adopted from
%    |\@xnthm|. We have to remember, though, that in this case
%    the name of the current counter and the theorem label have moved
%    to the second and third arguments.
%    \begin{macrocode}
    \def\@tempa{\global\@namedef{#1}}\expandafter \@tempa
     \expandafter{\csname th@\the \theorem@style \expandafter
     \endcsname \the\theorem@bodyfont \@thm{#2}{#3}}%
    \global \expandafter \let \csname end#1\endcsname \@endtheorem}%
  \fi}
%    \end{macrocode}
% \end{macro}
%
% \subsection{Macros that are employed in a {\envfont theorem}
%    environment}
%
% \begin{macro}{\@thm}
%    The macro |\@thm| has to increase the current counter.  Then,
%    depending on whether the environment has (or does not have) an
%    optional argument, it has to branch into either |\@begintheorem|
%    or |\@opargtheorem|.
%    \begin{macrocode}
\gdef\@thm#1#2{\refstepcounter{#1}%
%    \end{macrocode}
%    Now we start a {\envfont trivlist} environment, and give
%    |\@topsep| and |\@topsepadd| the values of the \textsf{skip}
%    registers |\theorempreskipamount| and |\theorempostskipamount|.
%    The value in |\@topsep| is the vertical space that is inserted by
%    the first (and only) |\item| in our |\trivlist| whilst
%    |\@topsepadd| is inserted by |\@endparenv| at the end of that
%    {\envfont trivlist} environment.  By using these registers, we
%    obtain the desired space around a {\envfont theorem} environment.
%    \begin{macrocode}
   \trivlist
   \@topsep \theorempreskipamount               % used by first \item
   \@topsepadd \theorempostskipamount           % used by \@endparenv
%    \end{macrocode}
%    Now we have to test whether an optional argument has been given.
% \changes{v2.0d}{89/07/18}{Forgotten percent added.}
%    \begin{macrocode}
   \@ifnextchar [%
%    \end{macrocode}
%    If there is an optional argument, we will call |\@ythm|, and
%    move the arguments read back into the input stream.
%    \begin{macrocode}
   {\@ythm{#1}{#2}}%
%    \end{macrocode}
%    If not, we call |\@begintheorem|. Its first argument is the
%    name of the theorem set (hence the second argument of
%    |\@thm|). Its second argument is the macro that produces the
%    current number.
%    \begin{macrocode}
   {\@begintheorem{#2}{\csname the#1\endcsname}\ignorespaces}}
%    \end{macrocode}
% \end{macro}
%
% \begin{macro}{\@xthm}
% \begin{macro}{\@ythm}
%    Both these macros were originally called by |\@thm|.  We do
%    not need |\@xthm| anymore, hence we reset it to
%    |\relax|. The definition of |\@ythm| has not changed at
%    all from its definition in \LaTeX. In order to make the macros
%    easier to understand, we will
%    nevertheless present it (commented out).
%    \begin{macrocode}
\global\let\@xthm\relax
% \def\@ythm#1#2[#3]{\@opargbegintheorem{#2}{\csname
%           the#1\endcsname}{#3}\ignorespaces}
%</package>
%    \end{macrocode}
%    The primitive |\ignorespaces| in |\@ythm| and
%    |\@thm| is needed to remove the spaces between the
%    |\begin{...}| and the actual text.
% \end{macro}
% \end{macro}
%
%
% \subsection{Definition of the theorem substyles}
%
% As already pointed out, the theorem substyles, defined below, are
% only loaded when necessary.  Note that all these substyles, except
% \texttt{plain}, have |\slshape| as the default body font.
%
% \subsubsection{The \texttt{plain} style}
%
% \label{dok:th@plain} As the following macros use |@|, we have
% to locally set the |\catcode| of this symbol to ``letter''.
% This happens within a group, so that we do not have to worry about
% which |\catcode| that symbol had before.
%    \begin{macrocode}
%<*thp>
\begingroup \makeatletter
%    \end{macrocode}
% Since we are now within a group, we must make all definitions
% globally.  First we make sure that \texttt{theorem.sty} is loaded.
% This will allow us to use this file as a document style option
% without having to call \texttt{theorem} itself as an option.  At the
% same time, we assure that at least version 2 is loaded, since
% |\theorem@style| was not defined in earlier versions.
%    \begin{macrocode}
\@ifundefined{theorem@style}{\input{theorem.sty}}{}
\ProvidesFile{thp.sty}
             [\FMithmInfo]
%    \end{macrocode}
%
%
% \begin{macro}{\th@plain}
% \begin{macro}{\@begintheorem}
% \begin{macro}{\@opargbegintheorem}
%    |\theoremstyle{plain}| corresponds to the original
%    definition, except that the distances to the surrounding text are
%    determined by the parameters |\theorempreskipamount| and
%    |\theorempostskipamount|.
%    First we set the default body font.
%    \begin{macrocode}
\gdef\th@plain{\normalfont\itshape
%    \end{macrocode}
%    Then we define |\@begintheorem| and
%    |\@opargbegintheorem|. These two macros define how the
%    header of a theorem is typeset.  |\@opargbegintheorem| will
%    be called if a \textsf{theorem} environment with an optional
%    argument is encountered; otherwise, the header is constructed by
%    calling |\@begintheorem|.  If one of these macros is
%    executed, we are within a \textsf{trivlist} environment started by
%    |\@thm|.  So the theorem header is produced with an
%    |\item| command.
%
%    Instead of specifying the header font directly, all standard
%    theorem styles use the |\theorem@headerfont| macro to allow
%    customization.  The extra space (|\labelsep|) is necessary
%    because of problems in the \textsf{trivlist} environment.
%    \begin{macrocode}
  \def\@begintheorem##1##2{%
        \item[\hskip\labelsep \theorem@headerfont ##1\ ##2]}%
%    \end{macrocode}
%    The definition of |\@opargbegintheorem| is completely
%    analogous. The only difference is the fact that there exists a
%    third argument (which is the optional parameter of the
%    environment and contains additional information about the
%    theorem).  Customarily we enclose it in parentheses.
%    \begin{macrocode}
\def\@opargbegintheorem##1##2##3{%
   \item[\hskip\labelsep \theorem@headerfont ##1\ ##2\ (##3)]}}
%    \end{macrocode}
% \end{macro}
% \end{macro}
% \end{macro}
%
% \noindent
% We conclude with an |\endgroup| to restore the
% |\catcode| of |@|.
%    \begin{macrocode}
\endgroup
%</thp>
%    \end{macrocode}
%
%
% \subsubsection{The \texttt{break} style}
%
% This style option is stored in the file |thb.sty|.
% For the next two lines see the documentation for |\th@plain| on
% page~\pageref{dok:th@plain}.
%    \begin{macrocode}
%<*thb>
\begingroup \makeatletter
\@ifundefined{theorem@style}{\input{theorem.sty}}{}
\ProvidesFile{thb.sty}
             [\FMithmInfo]
%    \end{macrocode}
%
% \begin{macro}{\th@break}
%    |\theoremstyle{break}| produces a line break after the name
%    of the theorem.  The font is |\slshape|.  Hence, we define
%    |\th@break| as follows:
%    \begin{macrocode}
\gdef\th@break{\normalfont\slshape
  \def\@begintheorem##1##2{\item[%
%    \end{macrocode}
%    We run into the following problem: it is not possible to create
%    the header with |\item[|\meta{title}|]| and then start
%    a new line by, for example, |\mbox{}\\|. Such a definition
%    will fail whenever a list environment follows immediately.  With
%    the above construction, the |\mbox{}| causes the switch
%    |@inlabel| (cf.\ definition of |\list| and
%    |\trivlist| in \cite{latex:source}) to be set to \texttt{false}
%    and so the following list will insert additional vertical
%    space (|\topskip|). This is quite annoying. Therefore, we
%    create the line break within the |\item|. In order to ensure
%    that the text will begin at the proper position in the following
%    line, we simply pretend that the label does not take any
%    room.\footnote{This will lead to problems whenever very high
%    symbols occurring in the line tower into the
%    heading. So, something else has to be done here sometime.}
%    \begin{macrocode}
       \rlap{\vbox{\hbox{\hskip \labelsep\theorem@headerfont ##1\ ##2}%
                   \hbox{\strut}}}]}%
%    \end{macrocode}
%    Again, the definition of |\@opargbegintheorem| is
%    completely analogous.
%    \begin{macrocode}
\def\@opargbegintheorem##1##2##3{%
  \item[\rlap{\vbox{\hbox{\hskip \labelsep \theorem@headerfont
                     ##1\ ##2\ (##3)}%
                    \hbox{\strut}}}]}}
\endgroup
%</thb>
%    \end{macrocode}
% \end{macro}
%
%
% \subsubsection{The \texttt{changebreak} style}
%
%    \begin{macrocode}
% This style option is stored in the file |thcb.sty|.
%    \begin{macrocode}
%<*thcb>
\begingroup \makeatletter
\@ifundefined{theorem@style}{\input{theorem.sty}}{}
\ProvidesFile{thcb.sty}
             [\FMithmInfo]
%    \end{macrocode}
%
% \begin{macro}{\th@changebreak}
%    The \texttt{change-break} theorem style is like \texttt{break}
%    but with interchange of theorem name and theorem number.  Thus we
%    define |\th@changebreak| as follows:
%    \begin{macrocode}
\gdef\th@changebreak{\normalfont\slshape
  \def\@begintheorem##1##2{\item
       [\rlap{\vbox{\hbox{\hskip\labelsep \theorem@headerfont ##2\ ##1}%
                    \hbox{\strut}}}]}%
\def\@opargbegintheorem##1##2##3{%
  \item[\rlap{\vbox{\hbox{\hskip\labelsep \theorem@headerfont
                     ##2\ ##1\ (##3)}%
                    \hbox{\strut}}}]}}
\endgroup
%</thcb>
%    \end{macrocode}
% \end{macro}
%
%
% \subsubsection{The \texttt{change} style}
%
% This style option is stored in the file |thc.sty|.
%    \begin{macrocode}
%<*thc>
\begingroup \makeatletter
\@ifundefined{theorem@style}{\input{theorem.sty}}{}
\ProvidesFile{thc.sty}
             [\FMithmInfo]
%    \end{macrocode}
%
% \begin{macro}{\th@change}
%    The \texttt{change} theorem style corresponds to the
%    \texttt{change break} style without a linebreak after the header.
%    To say it in another way, it's the same as the \texttt{plain}
%    style but with number and name interchanged and |\slshape| as the
%    default font.
%    \begin{macrocode}
\gdef\th@change{\normalfont\slshape
  \def\@begintheorem##1##2{\item
       [\hskip\labelsep \theorem@headerfont ##2\ ##1]}%
\def\@opargbegintheorem##1##2##3{%
  \item[\hskip\labelsep \theorem@headerfont ##2\ ##1\ (##3)]}}
\endgroup
%</thc>
%    \end{macrocode}
% \end{macro}
%
%
% \subsubsection{The \texttt{marginbreak} style}
%
% This style option is the one used most often at Mainz.  It is saved
% in the file |thmb.sty|.
%    \begin{macrocode}
%<*thmb>
\begingroup \makeatletter
\@ifundefined{theorem@style}{\input{theorem.sty}}{}
\ProvidesFile{thmb.sty}
             [\FMithmInfo]
%    \end{macrocode}
%
% \begin{macro}{\th@marginbreak}
%    The \texttt{margin break} style is nearly the same as the
%    \texttt{change break} style. The only difference is the placement
%    of the theorem number.  We use |\llap| to place it in the left
%    margin.
%
%    In this style |\labelsep| denotes the separation between the
%    number and the text.
%    \begin{macrocode}
\gdef\th@marginbreak{\normalfont\slshape
  \def\@begintheorem##1##2{\item
       [\rlap{\vbox{\theorem@headerfont
                    \hbox{\llap{##2}\hskip\labelsep ##1}%
                    \hbox{\strut}}}]}%
\def\@opargbegintheorem##1##2##3{%
  \item[\rlap{\vbox{\theorem@headerfont
                    \hbox{\llap{##2}\hskip\labelsep ##1\ (##3)}%
                    \hbox{\strut}}}]}}
\endgroup
%</thmb>
%    \end{macrocode}
% \changes{v2.1c}{90/02/17}{`endgroup outside macrocode env.}
% \end{macro}
%
%
%
% \subsubsection{The \texttt{margin} style}
%
% This style option is stored in the file |thm.sty|.
%    \begin{macrocode}
%<*thm>
\begingroup \makeatletter
\@ifundefined{theorem@style}{\input{theorem.sty}}{}
\ProvidesFile{thm.sty}
             [\FMithmInfo]
%    \end{macrocode}
%
% \begin{macro}{\th@margin}
%    Again this is only a variant of the theorem styles described
%    above without any new ideas.
%    \begin{macrocode}
\gdef\th@margin{\normalfont\slshape
  \def\@begintheorem##1##2{\item
       [\theorem@headerfont \llap{##2}\hskip\labelsep ##1]}%
\def\@opargbegintheorem##1##2##3{%
  \item[\theorem@headerfont \llap{##2}\hskip\labelsep ##1\ (##3)]}}
\endgroup
%</thm>
%    \end{macrocode}
% \end{macro}
%
%
%
%
% \subsection{Final Definitions}
%
% \begin{macro}{\theorempreskipamount}
% \begin{macro}{\theorempostskipamount}
%    The \textsf{skip} parameters that regulate the vertical empty space
%    before and after the {\envfont theorem} environment have
%    to be allocated as well.
% \changes{v2.0f}{89/07/20}{Unnecessary `global' removed.}
%    \begin{macrocode}
%<*package>
\newskip\theorempreskipamount
\newskip\theorempostskipamount
%    \end{macrocode}
%    Since we have used the same values for all theorem sets, we now can
%    assign them.
%    \begin{macrocode}
\global\setlength\theorempreskipamount{12pt plus 5pt minus 3pt}
\global\setlength\theorempostskipamount{8pt plus 3pt minus 1.5pt}
%    \end{macrocode}
% \end{macro}
% \end{macro}
%
%
% \begin{macro}{\@endtheorem}
%    The same holds for the macro |\@endtheorem|, which ends a
%    {\envfont theorem} environment. Since it is the same for all
%    theorem sets, it is removed from the macros |\th@|\meta{style}.
%    It simply ends the {\envfont trivlist} environment, which was
%    begun in |\@thm|.
%    \begin{macrocode}
\global\let\@endtheorem=\endtrivlist
%    \end{macrocode}
% \end{macro}
%
%
%
% \begin{macro}{\@preamblecmds}
%    All macros defined above are to be used only in the preamble.
%    Therefore, we insert them in |\@preamblecmds| which will disable
%    them at begin document. This is done by the internal \LaTeXe{}
%    command |\@onlypreamble|.
% \changes{v2.0g}{89/09/19}{`theoremheaderfont and `theorembodyfont
%    added.}
% \changes{v2.2a}{94/02/03}{Use \LaTeXe{} interface}
%    \begin{macrocode}
\@onlypreamble\@xnthm
\@onlypreamble\@ynthm
\@onlypreamble\@othm
\@onlypreamble\newtheorem
\@onlypreamble\theoremstyle
\@onlypreamble\theorembodyfont
\@onlypreamble\theoremheaderfont
%    \end{macrocode}
% \end{macro}
%
% Finally we declare the \texttt{plain} theorem style to be the default.
% \changes{v2.1b}{90/06/04}{Fixed missing `item by declaring default
%    style.}
%    \begin{macrocode}
\theoremstyle{plain}
%</package>
%    \end{macrocode}
%
% \Finale

\endinput
}{}
\ProvidesFile{thc.sty}
             [\FMithmInfo]
%    \end{macrocode}
%
% \begin{macro}{\th@change}
%    The \texttt{change} theorem style corresponds to the
%    \texttt{change break} style without a linebreak after the header.
%    To say it in another way, it's the same as the \texttt{plain}
%    style but with number and name interchanged and |\slshape| as the
%    default font.
%    \begin{macrocode}
\gdef\th@change{\normalfont\slshape
  \def\@begintheorem##1##2{\item
       [\hskip\labelsep \theorem@headerfont ##2\ ##1]}%
\def\@opargbegintheorem##1##2##3{%
  \item[\hskip\labelsep \theorem@headerfont ##2\ ##1\ (##3)]}}
\endgroup
%</thc>
%    \end{macrocode}
% \end{macro}
%
%
% \subsubsection{The \texttt{marginbreak} style}
%
% This style option is the one used most often at Mainz.  It is saved
% in the file |thmb.sty|.
%    \begin{macrocode}
%<*thmb>
\begingroup \makeatletter
\@ifundefined{theorem@style}{% \iffalse meta-comment
%
% Copyright (C) 1993-2024
%
% The LaTeX Project and any individual authors listed elsewhere
% in this file.
%
% This file is part of the Standard LaTeX `Tools Bundle'.
% -------------------------------------------------------
%
% It may be distributed and/or modified under the
% conditions of the LaTeX Project Public License, either version 1.3c
% of this license or (at your option) any later version.
% The latest version of this license is in
%    https://www.latex-project.org/lppl.txt
% and version 1.3c or later is part of all distributions of LaTeX
% version 2005/12/01 or later.
%
% The list of all files belonging to the LaTeX `Tools Bundle' is
% given in the file `manifest.txt'.
%
% \fi
%
%%
%
%
\def\FMithmInfo{2023/07/05 v2.2c Theorem extension package (FMi)}
%
% \ProvidesFile{theorem.dtx}[\FMithmInfo]
%\iffalse   % this is a METACOMMENT !
%%
%% Package `theorem' to use with LaTeX2e
%% Copyright (C) 1989-1995 Frank Mittelbach, all rights reserved.
%<+package>\@ifundefined{theorem@style}{}{\endinput}
%<+package>\NeedsTeXFormat{LaTeX2e}[1995/06/01]
%<+package>\ProvidesPackage{theorem}
%<+package>         [\FMithmInfo]
%
% \fi
%
% \changes{v1.0}{87/05/07}{First draft.}
% \changes{v1.0a}{87/05/07}{`thistheoremstyle' added.}
% \changes{v1.1a}{87/07/29}{New concept: redefine the internal
%                          \protect\LaTeX{} commands `@xnthm' etc.}
% \changes{v1.1b}{87/07/31}{`theoremfont' used to specify font.}
% \changes{v1.2a}{88/01/08}{`@endtheorem' now globally defined.}
% \changes{v1.2b}{88/01/11}{`newtheorem' only allowed in the preamble.}
% \changes{v1.3a}{88/02/14}{`theoremfont' now a toks register.}
%
% \changes{v2.0a}{88/11/16}{New implementation and new documentation.}
% \changes{v2.0a}{88/11/16}{No layout changes but internal modifications
%                           old substyles won't work.}
% \changes{v2.0b}{89/01/30}{`global' used for all definitions.}
% \changes{v2.0c}{89/06/15}{Documentation translated to English.
%                            Thanks to LLL and Nina}
% \changes{v2.1a}{90/02/15}{Support for new font selection.}
% \changes{v2.2a}{94/02/03}{Cleanup for LaTeX2e}
%
% \DoNotIndex{\@endparenv,\@ifdefinable,\@ifnextchar,\@ifundefined}
% \DoNotIndex{\@namedef,\@nameuse,\@spaces,\@tempa,\@warning}
% \DoNotIndex{\ }
% \DoNotIndex{\begingroup,\bfseries}
% \DoNotIndex{\csname}
% \DoNotIndex{\def,\do,\docdate}
% \DoNotIndex{\endcsname,\endgroup,\endinput,\endtrivlist,\expandafter}
% \DoNotIndex{\filedate,\fileversion}
% \DoNotIndex{\gdef,\global}
% \DoNotIndex{\hbox,\hskip}
% \DoNotIndex{\ignorespaces,\input,\itshape,\item}
% \DoNotIndex{\labelsep,\let,\llap}
% \DoNotIndex{\makeatletter}
% \DoNotIndex{\noexpand}
% \DoNotIndex{\refstepcounter,\relax,\rlap}
% \DoNotIndex{\setlength,\slshape,\space,\string,\strut}
% \DoNotIndex{\the,\trivlist,\typeout}
% \DoNotIndex{\vbox}
% \DoNotIndex{\xdef}
%
% \def\deflabel#1{\ttfamily #1\hfill}
% \def\deflist#1{\begin{list}{}{\settowidth\labelwidth{\ttfamily #1}%
%                               \setlength\leftmargin\labelwidth
%                               \addtolength\leftmargin\labelsep
%                               \let\makelabel\deflabel}}
% \def\enddeflist{\end{list}}
% \def\envfont{\normalfont\ttfamily}
%
% \MakeShortVerb{\|}
%
% \newcommand{\TUB}{{\sl TUGboat\/}}
%
% \GetFileInfo{theorem.dtx}
% \title{An Extension of the
%        \LaTeX{} theorem environment\thanks{This file
%        has version number \fileversion, last
%        revised \filedate.}}
% \author{Frank Mittelbach}
%
% \MaintainedByLaTeXTeam{tools}
% \maketitle
%
% \begin{abstract}
%    The macros described in this paper yield an extension of the
%    \LaTeX{} theorem mechanism. It is designed to satisfy the
%    different requirements of various journals. Thus, the
%    layout of the ``theorems'' can be manipulated by determining a
%    ``style''. This article describes not only the use, but
%    also the definition, of the necessary macros.
% \end{abstract}
%
% \section*{Preface to version 2.2}
%
% For \LaTeXe{} this package did not need any fundamental changes. I
% only modified the messages generated so that theorem layout styles
% will show up with the |\listfiles| command and cleaned the section
% on the New Font Selection Scheme since this is now included in
% \LaTeX.
%
% \section*{Preface to version 2.1}
%
% This version is identical to 2.0g described in \TUB\ 10\#3 except
% for some internal defaults which are now set depending on the used
% font selection scheme.
%
% This was done to avoid unpleasant surprises if the new font selection
% scheme is in force. For further details see
% section~\ref{sec:newfonts} and \cite{bk:GMS94}.
%
% \section{Introduction}
%
% For our purposes here, ``theorems'' are labelled enunciations,
% often set off from the main text by extra space and a font change.
% Theorems, corollaries, conjectures, definitions, and remarks
% are all instances of ``theorems''.  The ``header'' of these structures
% is composed of a label (such as \textsc{Theorem} or \textsc{Remark})
% and a number which serializes an item in the sequence of items
% with the same label.
%
% Shortly after the introduction of \LaTeX{} at the Fachbereich
% Mathematik in Mainz, the desire to manipulate the layout of
% ``theorems'' arose.  In Mainz, the following two conventions came
% into general use:
% \begin{enumerate}
%   \item The number of the theorem is shown in the margin.
%   \item There is a line break at
%         the end of the theorem header.
% \end{enumerate}
%
% Additionally, some journals require different formats which depend on
% the ``sort of theorem'': e.g.\ often remarks and definitions are set
% in |\upshape|, while |\itshape| is employed for main theorems.
%
% Confronted with these requirements, a theorem environment was
% developed in Mainz which allows separate determination of the
% layout of the ``theorems sets'', comparable to |\pagestyle|.
%
%
% \section{The user interface}
% \subsection{Defining new theorem sets}
%
% \DescribeMacro\newtheorem
% As in the original \LaTeX{} version, the command |\newtheorem|
% defines a new ``theorem set'' or ``theorem-like structure''.
% Two required arguments name the new environment and give the text
% to be typeset with each instance of the new ``set'', while
% an optional argument determines how
% the ``set'' is enumerated:
% \begin{description}
%    \item[\ttfamily \bslash newtheorem\{foo\}\{bar\}]
%       The theorem set {\envfont foo} (whose name is \texttt{bar})
%       uses its own counter.
%    \item[\ttfamily \bslash newtheorem\{foo2\}{[foo]}\{bar2\}]
%       The theorem set {\envfont foo2} (printed name \texttt{bar2})
%       uses the same counter as the theorem set \texttt{foo}.
%    \item[\ttfamily \bslash newtheorem\{foo3\}\{bar3\}{[section]}]
%       The theorem set {\envfont foo3} (printed name \texttt{bar3}) is
%       enumerated within the counter \texttt{section}, i.e.\ with every
%       new |\section| the enumeration begins again with 1, and
%       the enumeration is composed from the section-number and the
%       theorem counter itself.
% \end{description}
%
% \DescribeMacro\theoremstyle
% Additionally, the command |\theoremstyle| can define the layout
% of various, or all, theorem sets.  It should be noted that any
% theorem set defined by |\newtheorem| is typeset in the
% |\theoremstyle| that is current at the time of the definition.
% Thus, the following
% \begin{verbatim}
%    \theoremstyle{break}        \newtheorem{Cor}{Corollary}
%    \theoremstyle{plain}        \newtheorem{Exa}{Example}[section]
%\end{verbatim}                             ^^A No blank at line start
% leads to the result that the set {\envfont Cor} is formatted in the
% style \texttt{break}, while the set {\envfont Exa} and all the
% following ones are formatted in the style \texttt{plain}, unless
% another |\theoremstyle| follows.  Since the definitions installed by
% |\newtheorem| are global, one also can limit |\theoremstyle| locally
% by grouping braces.
%
% \DescribeMacro\theorembodyfont
% The choice of the font for the theorem body is completely
% independent of the chosen |\theoremstyle|; this has proven to
% be very advantageous.  For example,
% \begin{verbatim}
%   {\theorembodyfont{\upshape}          \newtheorem{Rem}{Remark}}
%\end{verbatim}
% defines a theorem set {\envfont Rem}, which will be set in
% |\upshape| in the current layout (which in our example is
% \texttt{plain}).  As with |\theoremstyle|, the |\theorembodyfont|
% chosen is that current at the time of |\newtheorem|.  If
% |\theorembodyfont| is not specified or one defines
% |\theorembodyfont{}|, then the font used will be that defined by the
% |\theoremstyle|.
%
% \DescribeMacro\theoremheaderfont It is also possible to customize
% the font used for the theorem headers.  This is, however, a global
% declaration, and therefore there should be at most one
% |\theoremheaderfont| declaration in the preamble.\footnote{If it is
% actually necessary to have different header fonts, one has to define
% new theorem styles (substituting the desired font) or specify the
% information directly in the \texttt{\bslash newtheorem} declaration
% (the unclean variant).}
%
% \DescribeMacro\theorempreskipamount
% \DescribeMacro\theorempostskipamount Two additional parameters
% affect the vertical space around the theorem environments:
% |\theorempreskipamount| and |\theorempostskipamount| define,
% respectively, the spacing before and after such an environment.
% These parameters apply for all theorem sets and can be manipulated
% with the ordinary length macros.  They are rubber lengths
% (`\textsf{skips}'), and therefore can contain \texttt{plus} and
% \texttt{minus} parts.
%
% Since the definition of theorem sets should---most sensibly---be
% placed in the preamble, we only allow installation there.
% It is therefore possible to release the memory used here after
% |\begin{document}|, in order to make room for other
% applications.
%
%
% \subsection{Existing theorem styles}
%
% The following theorem styles exist to date:
% \begin{deflist}{marginbreak}
%    \item[plain]
%       This theorem style emulates the original \LaTeX{} definition,
%       except that additionally the parameters
%       |\theorem...skipamount| are used.
%    \item[break]
%       In this style, the theorem header is followed by a line
%       break.
%    \item[marginbreak]
%       The theorem number is set in the margin, and there is a line
%       break as in \texttt{break}.
%    \item[changebreak]
%       Like \texttt{break}, but with header number and text
%       interchanged.
%    \item[change]
%       Header number and text are interchanged, without a line break.
%    \item[margin]
%       The number is set in the left margin, without a line break.
% \end{deflist}
% All styles (except \texttt{plain}) select |\slshape| as the default
% |\theorembodyfont|.
%
%
% \subsection{Examples}
%
% Given the above theorem sets {\envfont Cor}, {\envfont Exa}
% and {\envfont Rem},
% suppose that the preamble also contains the declarations:
% \begin{verbatim}
%    \theoremstyle{marginbreak}   \newtheorem{Lem}[Cor]{Lemma}
%    \theoremstyle{change}
%    \theorembodyfont{\itshape}        \newtheorem{Def}[Cor]{Definition}
%
%    \theoremheaderfont{\scshape}
%\end{verbatim}
% Then the following are some typical examples of the typeset output
% resulting from their use.
%
% \begin{Cor}
%    This is a sentence typeset in the theorem environment {\envfont
%    Cor}.
% \end{Cor}
%
% \begin{Exa}
%    This is a sentence typeset in the theorem environment {\envfont
%    Exa}.
% \end{Exa}
%
% \begin{Rem}
%    This is a sentence typeset in the theorem environment {\envfont
%    Rem}.
% \end{Rem}
%
% \begin{Lem}[Ben User]
%    This is a sentence typeset in the theorem environment {\envfont
%    Lem}.
% \end{Lem}
%
% \begin{Def}[Very impressive Definition]
%    This is a sentence typeset in the theorem environment {\envfont
%    Def}.
% \end{Def}
%
% The last two examples show the effect of the optional argument to
% a theorem environment (it is the text typeset in parentheses).
%
%
% \section{Special Considerations}
% \label{sec:newfonts}
%
% Theoremheader and body are implemented as a unit. This means that
% the |\theoremheaderfont|  will inherit characteristics of the
% |\theorembodyfont| in \LaTeXe. Thus, if for example
% |\theorembodyfont| is |\itshape| and |\theoremheaderfont|
% is |\bfseries| the font selected for the header will have the
% characteristics `bold extended italic'. If this is not desired one
% should set the |\theoremheaderfont| to something like
% \begin{verbatim}
%      \theoremheaderfont{\normalfont\bfseries}
%\end{verbatim}
% i.e.\ supplying all necessary font information explicitly.
%
%
%
% \section{Acknowledgements}
%
% The publication of this set of macros was only possible with the
% help of Christina Busse (translating the manuscript into English),
% Joachim Pense (playing the r\^ole of typist), Chris Rowley
% (looking everything over) and many others providing useful
% suggestions.
%
% \MaybeStop{%
%    \begin{thebibliography}{9}
%       \bibitem{bk:GMS94} \textsc{M.~Goossens}, \textsc{F.~Mittelbach}
%         and \textsc{A.~Samarin}.
%         \newblock The \LaTeX{} Companion.
%         \newblock
%         Addison-Wesley, Reading, Massachusetts, 1994.
%       \bibitem{latex:source} \textsc{Lamport, Leslie.} \newblock
%                               \texttt{latex.tex}, version 2.09, date
%                               Feb. 1990.
%    \end{thebibliography}
%    \PrintIndex
%    \PrintChanges}
%
%
% \section{The documentation driver file}
%
% The next bit of code contains the documentation driver file for
% \TeX{}, i.e., the file that will produce the documentation you are
% currently reading. It will be extracted from this file by the
% \texttt{docstrip} program. Since it is the first code in the file
% one can alternatively process this file directly with \LaTeXe{} to
% obtain the documentation.
%
%    \begin{macrocode}
%<*driver>
\documentclass{ltxdoc}

\usepackage{theorem}

%<+driver>% The next few lines define theorem sets which are used
%<+driver>% in the example section of the documentation.

\theoremstyle{break}       \newtheorem{Cor}{Corollary}
\theoremstyle{plain}       \newtheorem{Exa}{Example}[section]
{\theorembodyfont{\upshape}\newtheorem{Rem}{Remark}}
\theoremstyle{marginbreak} \newtheorem{Lem}[Cor]{Lemma}
\theoremstyle{change}
\theorembodyfont{\itshape} \newtheorem{Def}[Cor]{Definition}

\theoremheaderfont{\scshape}

\RecordChanges

\begin{document}
   \DocInput{theorem.dtx}
\end{document}
%</driver>
%    \end{macrocode}
%
% \section{Definition of the Macros}
%
% If the file has been loaded before, we abort immediately. If not the
% package announces itself (this is actually done at the very top of
% the file---the way it is done isn't good style so don't copy it).
% \changes{v2.0e}{89/07/19}{Spaces removed from `typeout'.}
% \changes{v2.2b}{95/11/19}{Announce moved to top of file}
%    \begin{macrocode}
%<*package>
%\@ifundefined{theorem@style}{}{\endinput}
%\def\FMithmInfo{1995/11/19 v2.2b Theorem extension package (FMi)}
%\ProvidesPackage{theorem}[\FMithmInfo]
%    \end{macrocode}
%
%
% \subsection{Definition of theorem styles and fonts}
%
% All the definitions in this file are done globally to allow
% inputting this file inside a group.
%
% \begin{macro}{\theoremstyle}
%    Before a theorem style can be installed, the chosen style must be
%    known. For that reason, we must test to see that
%    |\th@|\meta{style} is known or, more precisely, that it is
%    different from |\relax|. If the style is not known then
%    |\th@plain| is used.
%    \begin{macrocode}
\gdef\theoremstyle#1{%
   \@ifundefined{th@#1}{\@warning
          {Unknown theoremstyle `#1'. Using `plain'}%
          \theorem@style{plain}}%
%    \end{macrocode}
%    We save the theorem style to be used in the token register
%    |\theorem@style|.
%    \begin{macrocode}
      {\theorem@style{#1}}%
%    \end{macrocode}
%    Now we ``evaluate'' the theorem style: this means, we call the
%    macro |\th@|\meta{style} which will activate the relevant
%    definitions which are contained in a separate file.  This is done
%    in a group to suppress changes to the current font.  This could
%    otherwise pose problems together with the new font selection
%    scheme\footnote{When I printed the original article using the new
%    font selection scheme I ended with a document with slanted
%    typefaces (text headings and all) simply because one of the
%    theorem styles used \texttt{\bslash sl} at toplevel.} if the
%    |\th@|\meta{style} is evaluated a second time.
% \changes{v2.1a}{90/02/15}{Group added around `th@..}
% \changes{v2.1a}{90/02/15}{Reseting of `..@begintheorem removed.}
%    \begin{macrocode}
      \begingroup
        \csname th@\the\theorem@style \endcsname
      \endgroup}
%    \end{macrocode}
% \end{macro}
%
%
% \begin{macro}{\@begintheorem}
% \changes{v2.1a}{90/02/15}{Relaxed at toplevel.}
% \begin{macro}{\@opargbegintheorem}
% \changes{v2.1a}{90/02/15}{Relaxed at toplevel.}
%    We reset |\@begintheorem| and |\@opargbegintheorem| to
%    |\relax| since these commands are no longer necessary at
%    toplevel. This will save a few tokens.
%    \begin{macrocode}
\global\let\@begintheorem\relax
\global\let\@opargbegintheorem\relax
%    \end{macrocode}
% \end{macro}
% \end{macro}
%
% \begin{macro}{\theorem@style}
%    Obviously the \textsf{token} register used above has to be
%    allocated. To assure the utmost compatibility with the
%    original \LaTeX{} definition, we set the default theorem style to
%    \texttt{plain}, which implements the usual \LaTeX{} convention.
% \changes{v2.0f}{89/07/20}{Unnecessary `global' removed.}
%    \begin{macrocode}
\newtoks\theorem@style
\global\theorem@style{plain}
%    \end{macrocode}
% \end{macro}
%
% \begin{macro}{\theorembodyfont}
% \begin{macro}{\theorem@bodyfont}
%    If the |\theorembodyfont| is set by the user then it should not
%    interact with the default font set in the theorem style. When the
%    new font selection is in force this may happen if, for example,
%    the default is |\itshape| and the new |\theorembodyfont| is
%    |\sffamily|. So we add a |\reset@font| command in front of the
%    user definition.
% \changes{v2.0f}{89/07/20}{Unnecessary `global' removed.}
% \changes{v2.0g}{89/09/19}{`theoremfont to `theorembodyfont.}
% \changes{v2.1c}{92/06/22}{`reset@font added and logic changed.}
% \changes{v2.1d}{92/10/28}{Reset to empty if arg is empty}
%    \begin{macrocode}
\gdef\theorembodyfont#1{%
%    \end{macrocode}
%    We check if the argument supplied is empty and if so put nothing
%    into the |\theorem@bodyfont| token register to allow for
%    |\theorembodyfont{}| as a mean of using the default of the
%    current |\theoremstyle|.
%    \begin{macrocode}
   \def\@tempa{#1}%
   \ifx\@tempa\@empty
    \theorem@bodyfont{}%
   \else
    \theorem@bodyfont{\reset@font#1}%
   \fi
   }
\newtoks\theorem@bodyfont
\global\theorem@bodyfont{}
%    \end{macrocode}
% \end{macro}
% \end{macro}
%
%
% \begin{macro}{\theoremheaderfont}
% \changes{v2.0g}{89/09/19}{Macro added.}
%    The font for the theorem headers is handled differently because
%    this definition applies to all theorem styles.
%    \begin{macrocode}
\gdef\theoremheaderfont#1{\gdef\theorem@headerfont{#1}%
%    \end{macrocode}
%    After using the macro once it is redefined to produce an error
%    message.
%    \begin{macrocode}
       \gdef\theoremheaderfont##1{%
        \typeout{\string\theoremheaderfont\space should be used
                 only once.}}}
%    \end{macrocode}
% \end{macro}
%
% \begin{macro}{\theorem@headerfont}
% \changes{v2.0g}{89/09/19}{Macro added.}
%    To set the |\theorem@headerfont| default we first test if
%    the new fontselection scheme is in force.
% \changes{v2.1a}{90/02/15}{Support for new font selection.}
%    \begin{macrocode}
\ifx\upshape\undefined
%    \end{macrocode}
%    If not we define it to expand into |\bfseries|. We don't use
%    |\let| just in case a following style option redefines this
%    macro.
%    \begin{macrocode}
\gdef\theorem@headerfont{\bfseries}
%    \end{macrocode}
%    Otherwise we reset the current shape before calling |\bfseries|.
%    \begin{macrocode}
\else \gdef\theorem@headerfont{\normalfont\bfseries}\fi
%    \end{macrocode}
% \end{macro}
%
% \begin{macro}{\th@plain}
% \begin{macro}{\th@break}
% \begin{macro}{\th@marginbreak}
% \begin{macro}{\th@changebreak}
% \begin{macro}{\th@change}
% \begin{macro}{\th@margin}
%    The different styles are defined in macros such as
%    |\th@plain|.  Since memory space is precious in
%    ``non-Big-versions'', we have to avoid offering too many unused
%    definitions.  Therefore we define these styles in separate files
%    that can be loaded on demand.  Thus the commands themselves only
%    load these files. We use |\@input@|, a \LaTeXe{} internal command
%    that ensures that the file will be listed with |\listfiles|.
% \changes{v2.2a}{94/02/02}{Use `@input@ to load theorem layout files}
%    \begin{macrocode}
\gdef\th@plain{\@input@{thp.sty}}
\gdef\th@break{\@input@{thb.sty}}
\gdef\th@marginbreak{\@input@{thmb.sty}}
\gdef\th@changebreak{\@input@{thcb.sty}}
\gdef\th@change{\@input@{thc.sty}}
\gdef\th@margin{\@input@{thm.sty}}
%    \end{macrocode}
%    This list will be expanded when new styles become
%    available.  For testing,  just append new theorem substyles
%    as document options.
% \end{macro}
% \end{macro}
% \end{macro}
% \end{macro}
% \end{macro}
% \end{macro}
%
%
%
% \subsection{Definition of a new theorem set}
%
% As already pointed out, a new theorem environment can be defined in
% three different ways:
% \begin{verbatim}
%    \newtheorem{Lem}{Lemma}
%    \newtheorem{Lem}{Lemma}[section]
%    \newtheorem{Lem}[Theorem]{Lemma}
%\end{verbatim}
% The function of the macro |\newtheorem| is to recognize these cases
% and then to branch into one of the three macros |\@ynthm|, |\@xnthm|
% or |\@othm|.  This mechanism is adopted unchanged from
% \cite{latex:source}; the essential point here is that, for example,
% in the second case, the arguments \texttt{Lem}, \texttt{Lemma} and
% \texttt{section} are passed over to the macro |\@xnthm|.
%
% We inspect this case first because the others present fewer problems,
% and thus are easily derived from this one.
%
% \begin{macro}{\@xnthm}
%    For our example arguments, the macro |\@xnthm| must fulfill
%    the following:
%    \begin{itemize}
%       \item
%          Define a new \LaTeX-counter `Lem'
%       \item
%          reset this counter within a |\section|
%       \item
%          define the macro |\theLem|
%       \item
%          define the environment macros |\Lem| and
%          |\endLem| using the current
%          |\theoremstyle| and |\theorem@bodyfont|.
%    \end{itemize}
%    Obviously, all this should happen only if the first argument
%    of |\@xnthm| (i.e.\ \texttt{Lem} in our example) is chosen so
%    as not to conflict with any previously defined commands or
%    environments.  This test is performed by the \LaTeX{} macro
%    |\@ifdefinable|.
%    \begin{macrocode}
\gdef\@xnthm#1#2[#3]{\expandafter\@ifdefinable\csname #1\endcsname
%    \end{macrocode}
%    Therefore, the first argument of |\@ifdefinable| is the
%    expansion (in the example, |\Lem|) of
%    |\csname||#1\endcsname|.  The second argument is
%    executed only if the test has been completed successfully.
%    \begin{macrocode}
   {%
%    \end{macrocode}
%    Now we define the new counter. The names of the \LaTeX{}
%    macros employed should speak for themselves:
% \changes{v2.2b}{1995/10/07}{Use \cs{@newctr}}
% \changes{v2.2c}{1995/11/23}{Use \cs{@newctr} properly.}
%    \begin{macrocode}
    \@definecounter{#1}\@newctr{#1}[#3]%
%    \end{macrocode}
%    Using |\@newctr| will give a proper error message if the counter
%    in |#3| is not defined.
%    In defining `|\theLem|' we must generate the desired macro
%    name by use of |\expandafter| and |\csname|.
%    \begin{macrocode}
    \expandafter\xdef\csname the#1\endcsname
%    \end{macrocode}
%    An |\xdef| is used in order to make the definition global,
%    and to ensure that it contains the replacement texts of
%    |\@thmcountersep| and |\@thmcounter|.\footnote{These
%    two macros can be defined by the document style.  Their default
%    values produce a `\texttt{.}' as separation and an arabic
%    representation of the number.} However, not everything should be
%    expanded. For example, it saves space to use |\thesection|
%    instead of its---at times---lengthy expansion.
%    \begin{macrocode}
      {\expandafter \noexpand \csname the#3\endcsname
       \@thmcountersep \@thmcounter{#1}}%
%    \end{macrocode}
%    Thus with the defaults of \LaTeX{}, |\theLem| would be
%    replaced by the command sequence |\thesection.\arabic{Lem}|.
%
%    We will now look at the definition of the macro which is executed
%    at the beginning of the actual environment (in our example this
%    macro is |\Lem|). It should be noted that we use an
%    ``|\expandafter| trick'' to expand only certain parts of the
%    replacement text at the time of the definition.
%    \begin{macrocode}
    \def\@tempa{\global\@namedef{#1}}%
    \expandafter \@tempa \expandafter{%
%    \end{macrocode}
%    First, the macro that contains the current definitions of
%    |\@begintheorem| and |\@opargtheorem| should be called
%    up.  The name of this macro---as is already known---has the form
%    |\th@|\meta{theorem\ style}; therefore, it must be called by
%    \begin{macrocode}
      \csname th@\the \theorem@style
%    \end{macrocode}
%    In addition the default theorem font should be changeable, i.e.\
%    we have to insert the contents of |\theorem@bodyfont|. For
%    that reason, we expand even further, beyond |\endcsname|,
%    and thus insert the contents of the \textsf{token} register
%    |\theorem@bodyfont| in the replacement text.
%    \begin{macrocode}
            \expandafter \endcsname \the \theorem@bodyfont
%    \end{macrocode}
%    Now it is time to call the macro |\@thm| which takes over
%    the further processing.  It has two arguments: the current
%    counter name (in our example, \texttt{Lem}), and the text of the
%    label (in our example, \texttt{Lemma}).
%    \begin{macrocode}
     \@thm{#1}{#2}}%
%    \end{macrocode}
%    With this, the `sub-definition' is complete.  The macro
%    |\@endtheorem| ends a theorem environment and is, so far, nothing
%    but an |\endtrivlist|.  (Hence it is defined globally, and not
%    within the theorem styles.\footnote{This has to be changed as
%    soon as theorem styles that change \texttt{\bslash @endtheorem}
%    exist.  In such a case, all existing styles must be changed as
%    well since they will have to reset the macro.}) Therefore, we can
%    set it equivalent to the macro that ends the theorem set (in our
%    example, |\endLem|).  However, if some day theorem styles exist
%    that do change |\@endtheorem|, we would have to use the
%    commented-out line instead.
%    \begin{macrocode}
    \global \expandafter \let \csname end#1\endcsname \@endtheorem
%   \global\@namedef{end#1}{\@endtheorem}%
%    \end{macrocode}
%    With these commands all the required definitions are employed,
%    unless the test |\@ifdefinable| has failed.  Therefore, we
%    end the second argument of this macro and with it the definition
%    of |\@xnthm|.
%    \begin{macrocode}
   }}
%    \end{macrocode}
% \end{macro}
%
% \begin{macro}{\@ynthm}
%    The definition of |\@ynthm| is completely analogous. In
%    this case the new counter that is defined is not reset within
%    another counter; thus the definition of |\the...|
%    is simplified:
%    \begin{macrocode}
\gdef\@ynthm#1#2{\expandafter\@ifdefinable\csname #1\endcsname
   {\@definecounter{#1}%
    \expandafter\xdef\csname the#1\endcsname{\@thmcounter{#1}}%
%    \end{macrocode}
%    The rest of the definition corresponds literally to that of
%    |\@xnthm|:
%    \begin{macrocode}
    \def\@tempa{\global\@namedef{#1}}\expandafter \@tempa
     \expandafter{\csname th@\the \theorem@style \expandafter
     \endcsname \the\theorem@bodyfont \@thm{#1}{#2}}%
    \global \expandafter \let \csname end#1\endcsname \@endtheorem}}
%    \end{macrocode}
% \end{macro}
%
% \begin{macro}{\@othm}
%    The definition of |\@othm| does not contain anything new.
%    \begin{macrocode}
\gdef\@othm#1[#2]#3{%
%    \end{macrocode}
% \changes{v2.2b}{95/10/07}{Build in check for counter defined}
%    We do not define a new counter but instead use one that has
%    already been defined.  Thus the only definition we need is that
%    of this pseudo-counter (i.e.\ |\the|\meta{env.\ name}).
%    First we check if |#2| corresponds to a known counter name.
%    \begin{macrocode}
  \expandafter\ifx\csname c@#2\endcsname\relax
   \@nocounterr{#2}%
  \else
   \expandafter\@ifdefinable\csname #1\endcsname
   {\expandafter \xdef \csname the#1\endcsname
     {\expandafter \noexpand \csname the#2\endcsname}%
%    \end{macrocode}
%    All other parts of the definition can be adopted from
%    |\@xnthm|. We have to remember, though, that in this case
%    the name of the current counter and the theorem label have moved
%    to the second and third arguments.
%    \begin{macrocode}
    \def\@tempa{\global\@namedef{#1}}\expandafter \@tempa
     \expandafter{\csname th@\the \theorem@style \expandafter
     \endcsname \the\theorem@bodyfont \@thm{#2}{#3}}%
    \global \expandafter \let \csname end#1\endcsname \@endtheorem}%
  \fi}
%    \end{macrocode}
% \end{macro}
%
% \subsection{Macros that are employed in a {\envfont theorem}
%    environment}
%
% \begin{macro}{\@thm}
%    The macro |\@thm| has to increase the current counter.  Then,
%    depending on whether the environment has (or does not have) an
%    optional argument, it has to branch into either |\@begintheorem|
%    or |\@opargtheorem|.
%    \begin{macrocode}
\gdef\@thm#1#2{\refstepcounter{#1}%
%    \end{macrocode}
%    Now we start a {\envfont trivlist} environment, and give
%    |\@topsep| and |\@topsepadd| the values of the \textsf{skip}
%    registers |\theorempreskipamount| and |\theorempostskipamount|.
%    The value in |\@topsep| is the vertical space that is inserted by
%    the first (and only) |\item| in our |\trivlist| whilst
%    |\@topsepadd| is inserted by |\@endparenv| at the end of that
%    {\envfont trivlist} environment.  By using these registers, we
%    obtain the desired space around a {\envfont theorem} environment.
%    \begin{macrocode}
   \trivlist
   \@topsep \theorempreskipamount               % used by first \item
   \@topsepadd \theorempostskipamount           % used by \@endparenv
%    \end{macrocode}
%    Now we have to test whether an optional argument has been given.
% \changes{v2.0d}{89/07/18}{Forgotten percent added.}
%    \begin{macrocode}
   \@ifnextchar [%
%    \end{macrocode}
%    If there is an optional argument, we will call |\@ythm|, and
%    move the arguments read back into the input stream.
%    \begin{macrocode}
   {\@ythm{#1}{#2}}%
%    \end{macrocode}
%    If not, we call |\@begintheorem|. Its first argument is the
%    name of the theorem set (hence the second argument of
%    |\@thm|). Its second argument is the macro that produces the
%    current number.
%    \begin{macrocode}
   {\@begintheorem{#2}{\csname the#1\endcsname}\ignorespaces}}
%    \end{macrocode}
% \end{macro}
%
% \begin{macro}{\@xthm}
% \begin{macro}{\@ythm}
%    Both these macros were originally called by |\@thm|.  We do
%    not need |\@xthm| anymore, hence we reset it to
%    |\relax|. The definition of |\@ythm| has not changed at
%    all from its definition in \LaTeX. In order to make the macros
%    easier to understand, we will
%    nevertheless present it (commented out).
%    \begin{macrocode}
\global\let\@xthm\relax
% \def\@ythm#1#2[#3]{\@opargbegintheorem{#2}{\csname
%           the#1\endcsname}{#3}\ignorespaces}
%</package>
%    \end{macrocode}
%    The primitive |\ignorespaces| in |\@ythm| and
%    |\@thm| is needed to remove the spaces between the
%    |\begin{...}| and the actual text.
% \end{macro}
% \end{macro}
%
%
% \subsection{Definition of the theorem substyles}
%
% As already pointed out, the theorem substyles, defined below, are
% only loaded when necessary.  Note that all these substyles, except
% \texttt{plain}, have |\slshape| as the default body font.
%
% \subsubsection{The \texttt{plain} style}
%
% \label{dok:th@plain} As the following macros use |@|, we have
% to locally set the |\catcode| of this symbol to ``letter''.
% This happens within a group, so that we do not have to worry about
% which |\catcode| that symbol had before.
%    \begin{macrocode}
%<*thp>
\begingroup \makeatletter
%    \end{macrocode}
% Since we are now within a group, we must make all definitions
% globally.  First we make sure that \texttt{theorem.sty} is loaded.
% This will allow us to use this file as a document style option
% without having to call \texttt{theorem} itself as an option.  At the
% same time, we assure that at least version 2 is loaded, since
% |\theorem@style| was not defined in earlier versions.
%    \begin{macrocode}
\@ifundefined{theorem@style}{\input{theorem.sty}}{}
\ProvidesFile{thp.sty}
             [\FMithmInfo]
%    \end{macrocode}
%
%
% \begin{macro}{\th@plain}
% \begin{macro}{\@begintheorem}
% \begin{macro}{\@opargbegintheorem}
%    |\theoremstyle{plain}| corresponds to the original
%    definition, except that the distances to the surrounding text are
%    determined by the parameters |\theorempreskipamount| and
%    |\theorempostskipamount|.
%    First we set the default body font.
%    \begin{macrocode}
\gdef\th@plain{\normalfont\itshape
%    \end{macrocode}
%    Then we define |\@begintheorem| and
%    |\@opargbegintheorem|. These two macros define how the
%    header of a theorem is typeset.  |\@opargbegintheorem| will
%    be called if a \textsf{theorem} environment with an optional
%    argument is encountered; otherwise, the header is constructed by
%    calling |\@begintheorem|.  If one of these macros is
%    executed, we are within a \textsf{trivlist} environment started by
%    |\@thm|.  So the theorem header is produced with an
%    |\item| command.
%
%    Instead of specifying the header font directly, all standard
%    theorem styles use the |\theorem@headerfont| macro to allow
%    customization.  The extra space (|\labelsep|) is necessary
%    because of problems in the \textsf{trivlist} environment.
%    \begin{macrocode}
  \def\@begintheorem##1##2{%
        \item[\hskip\labelsep \theorem@headerfont ##1\ ##2]}%
%    \end{macrocode}
%    The definition of |\@opargbegintheorem| is completely
%    analogous. The only difference is the fact that there exists a
%    third argument (which is the optional parameter of the
%    environment and contains additional information about the
%    theorem).  Customarily we enclose it in parentheses.
%    \begin{macrocode}
\def\@opargbegintheorem##1##2##3{%
   \item[\hskip\labelsep \theorem@headerfont ##1\ ##2\ (##3)]}}
%    \end{macrocode}
% \end{macro}
% \end{macro}
% \end{macro}
%
% \noindent
% We conclude with an |\endgroup| to restore the
% |\catcode| of |@|.
%    \begin{macrocode}
\endgroup
%</thp>
%    \end{macrocode}
%
%
% \subsubsection{The \texttt{break} style}
%
% This style option is stored in the file |thb.sty|.
% For the next two lines see the documentation for |\th@plain| on
% page~\pageref{dok:th@plain}.
%    \begin{macrocode}
%<*thb>
\begingroup \makeatletter
\@ifundefined{theorem@style}{\input{theorem.sty}}{}
\ProvidesFile{thb.sty}
             [\FMithmInfo]
%    \end{macrocode}
%
% \begin{macro}{\th@break}
%    |\theoremstyle{break}| produces a line break after the name
%    of the theorem.  The font is |\slshape|.  Hence, we define
%    |\th@break| as follows:
%    \begin{macrocode}
\gdef\th@break{\normalfont\slshape
  \def\@begintheorem##1##2{\item[%
%    \end{macrocode}
%    We run into the following problem: it is not possible to create
%    the header with |\item[|\meta{title}|]| and then start
%    a new line by, for example, |\mbox{}\\|. Such a definition
%    will fail whenever a list environment follows immediately.  With
%    the above construction, the |\mbox{}| causes the switch
%    |@inlabel| (cf.\ definition of |\list| and
%    |\trivlist| in \cite{latex:source}) to be set to \texttt{false}
%    and so the following list will insert additional vertical
%    space (|\topskip|). This is quite annoying. Therefore, we
%    create the line break within the |\item|. In order to ensure
%    that the text will begin at the proper position in the following
%    line, we simply pretend that the label does not take any
%    room.\footnote{This will lead to problems whenever very high
%    symbols occurring in the line tower into the
%    heading. So, something else has to be done here sometime.}
%    \begin{macrocode}
       \rlap{\vbox{\hbox{\hskip \labelsep\theorem@headerfont ##1\ ##2}%
                   \hbox{\strut}}}]}%
%    \end{macrocode}
%    Again, the definition of |\@opargbegintheorem| is
%    completely analogous.
%    \begin{macrocode}
\def\@opargbegintheorem##1##2##3{%
  \item[\rlap{\vbox{\hbox{\hskip \labelsep \theorem@headerfont
                     ##1\ ##2\ (##3)}%
                    \hbox{\strut}}}]}}
\endgroup
%</thb>
%    \end{macrocode}
% \end{macro}
%
%
% \subsubsection{The \texttt{changebreak} style}
%
%    \begin{macrocode}
% This style option is stored in the file |thcb.sty|.
%    \begin{macrocode}
%<*thcb>
\begingroup \makeatletter
\@ifundefined{theorem@style}{\input{theorem.sty}}{}
\ProvidesFile{thcb.sty}
             [\FMithmInfo]
%    \end{macrocode}
%
% \begin{macro}{\th@changebreak}
%    The \texttt{change-break} theorem style is like \texttt{break}
%    but with interchange of theorem name and theorem number.  Thus we
%    define |\th@changebreak| as follows:
%    \begin{macrocode}
\gdef\th@changebreak{\normalfont\slshape
  \def\@begintheorem##1##2{\item
       [\rlap{\vbox{\hbox{\hskip\labelsep \theorem@headerfont ##2\ ##1}%
                    \hbox{\strut}}}]}%
\def\@opargbegintheorem##1##2##3{%
  \item[\rlap{\vbox{\hbox{\hskip\labelsep \theorem@headerfont
                     ##2\ ##1\ (##3)}%
                    \hbox{\strut}}}]}}
\endgroup
%</thcb>
%    \end{macrocode}
% \end{macro}
%
%
% \subsubsection{The \texttt{change} style}
%
% This style option is stored in the file |thc.sty|.
%    \begin{macrocode}
%<*thc>
\begingroup \makeatletter
\@ifundefined{theorem@style}{\input{theorem.sty}}{}
\ProvidesFile{thc.sty}
             [\FMithmInfo]
%    \end{macrocode}
%
% \begin{macro}{\th@change}
%    The \texttt{change} theorem style corresponds to the
%    \texttt{change break} style without a linebreak after the header.
%    To say it in another way, it's the same as the \texttt{plain}
%    style but with number and name interchanged and |\slshape| as the
%    default font.
%    \begin{macrocode}
\gdef\th@change{\normalfont\slshape
  \def\@begintheorem##1##2{\item
       [\hskip\labelsep \theorem@headerfont ##2\ ##1]}%
\def\@opargbegintheorem##1##2##3{%
  \item[\hskip\labelsep \theorem@headerfont ##2\ ##1\ (##3)]}}
\endgroup
%</thc>
%    \end{macrocode}
% \end{macro}
%
%
% \subsubsection{The \texttt{marginbreak} style}
%
% This style option is the one used most often at Mainz.  It is saved
% in the file |thmb.sty|.
%    \begin{macrocode}
%<*thmb>
\begingroup \makeatletter
\@ifundefined{theorem@style}{\input{theorem.sty}}{}
\ProvidesFile{thmb.sty}
             [\FMithmInfo]
%    \end{macrocode}
%
% \begin{macro}{\th@marginbreak}
%    The \texttt{margin break} style is nearly the same as the
%    \texttt{change break} style. The only difference is the placement
%    of the theorem number.  We use |\llap| to place it in the left
%    margin.
%
%    In this style |\labelsep| denotes the separation between the
%    number and the text.
%    \begin{macrocode}
\gdef\th@marginbreak{\normalfont\slshape
  \def\@begintheorem##1##2{\item
       [\rlap{\vbox{\theorem@headerfont
                    \hbox{\llap{##2}\hskip\labelsep ##1}%
                    \hbox{\strut}}}]}%
\def\@opargbegintheorem##1##2##3{%
  \item[\rlap{\vbox{\theorem@headerfont
                    \hbox{\llap{##2}\hskip\labelsep ##1\ (##3)}%
                    \hbox{\strut}}}]}}
\endgroup
%</thmb>
%    \end{macrocode}
% \changes{v2.1c}{90/02/17}{`endgroup outside macrocode env.}
% \end{macro}
%
%
%
% \subsubsection{The \texttt{margin} style}
%
% This style option is stored in the file |thm.sty|.
%    \begin{macrocode}
%<*thm>
\begingroup \makeatletter
\@ifundefined{theorem@style}{\input{theorem.sty}}{}
\ProvidesFile{thm.sty}
             [\FMithmInfo]
%    \end{macrocode}
%
% \begin{macro}{\th@margin}
%    Again this is only a variant of the theorem styles described
%    above without any new ideas.
%    \begin{macrocode}
\gdef\th@margin{\normalfont\slshape
  \def\@begintheorem##1##2{\item
       [\theorem@headerfont \llap{##2}\hskip\labelsep ##1]}%
\def\@opargbegintheorem##1##2##3{%
  \item[\theorem@headerfont \llap{##2}\hskip\labelsep ##1\ (##3)]}}
\endgroup
%</thm>
%    \end{macrocode}
% \end{macro}
%
%
%
%
% \subsection{Final Definitions}
%
% \begin{macro}{\theorempreskipamount}
% \begin{macro}{\theorempostskipamount}
%    The \textsf{skip} parameters that regulate the vertical empty space
%    before and after the {\envfont theorem} environment have
%    to be allocated as well.
% \changes{v2.0f}{89/07/20}{Unnecessary `global' removed.}
%    \begin{macrocode}
%<*package>
\newskip\theorempreskipamount
\newskip\theorempostskipamount
%    \end{macrocode}
%    Since we have used the same values for all theorem sets, we now can
%    assign them.
%    \begin{macrocode}
\global\setlength\theorempreskipamount{12pt plus 5pt minus 3pt}
\global\setlength\theorempostskipamount{8pt plus 3pt minus 1.5pt}
%    \end{macrocode}
% \end{macro}
% \end{macro}
%
%
% \begin{macro}{\@endtheorem}
%    The same holds for the macro |\@endtheorem|, which ends a
%    {\envfont theorem} environment. Since it is the same for all
%    theorem sets, it is removed from the macros |\th@|\meta{style}.
%    It simply ends the {\envfont trivlist} environment, which was
%    begun in |\@thm|.
%    \begin{macrocode}
\global\let\@endtheorem=\endtrivlist
%    \end{macrocode}
% \end{macro}
%
%
%
% \begin{macro}{\@preamblecmds}
%    All macros defined above are to be used only in the preamble.
%    Therefore, we insert them in |\@preamblecmds| which will disable
%    them at begin document. This is done by the internal \LaTeXe{}
%    command |\@onlypreamble|.
% \changes{v2.0g}{89/09/19}{`theoremheaderfont and `theorembodyfont
%    added.}
% \changes{v2.2a}{94/02/03}{Use \LaTeXe{} interface}
%    \begin{macrocode}
\@onlypreamble\@xnthm
\@onlypreamble\@ynthm
\@onlypreamble\@othm
\@onlypreamble\newtheorem
\@onlypreamble\theoremstyle
\@onlypreamble\theorembodyfont
\@onlypreamble\theoremheaderfont
%    \end{macrocode}
% \end{macro}
%
% Finally we declare the \texttt{plain} theorem style to be the default.
% \changes{v2.1b}{90/06/04}{Fixed missing `item by declaring default
%    style.}
%    \begin{macrocode}
\theoremstyle{plain}
%</package>
%    \end{macrocode}
%
% \Finale

\endinput
}{}
\ProvidesFile{thmb.sty}
             [\FMithmInfo]
%    \end{macrocode}
%
% \begin{macro}{\th@marginbreak}
%    The \texttt{margin break} style is nearly the same as the
%    \texttt{change break} style. The only difference is the placement
%    of the theorem number.  We use |\llap| to place it in the left
%    margin.
%
%    In this style |\labelsep| denotes the separation between the
%    number and the text.
%    \begin{macrocode}
\gdef\th@marginbreak{\normalfont\slshape
  \def\@begintheorem##1##2{\item
       [\rlap{\vbox{\theorem@headerfont
                    \hbox{\llap{##2}\hskip\labelsep ##1}%
                    \hbox{\strut}}}]}%
\def\@opargbegintheorem##1##2##3{%
  \item[\rlap{\vbox{\theorem@headerfont
                    \hbox{\llap{##2}\hskip\labelsep ##1\ (##3)}%
                    \hbox{\strut}}}]}}
\endgroup
%</thmb>
%    \end{macrocode}
% \changes{v2.1c}{90/02/17}{`endgroup outside macrocode env.}
% \end{macro}
%
%
%
% \subsubsection{The \texttt{margin} style}
%
% This style option is stored in the file |thm.sty|.
%    \begin{macrocode}
%<*thm>
\begingroup \makeatletter
\@ifundefined{theorem@style}{% \iffalse meta-comment
%
% Copyright (C) 1993-2024
%
% The LaTeX Project and any individual authors listed elsewhere
% in this file.
%
% This file is part of the Standard LaTeX `Tools Bundle'.
% -------------------------------------------------------
%
% It may be distributed and/or modified under the
% conditions of the LaTeX Project Public License, either version 1.3c
% of this license or (at your option) any later version.
% The latest version of this license is in
%    https://www.latex-project.org/lppl.txt
% and version 1.3c or later is part of all distributions of LaTeX
% version 2005/12/01 or later.
%
% The list of all files belonging to the LaTeX `Tools Bundle' is
% given in the file `manifest.txt'.
%
% \fi
%
%%
%
%
\def\FMithmInfo{2023/07/05 v2.2c Theorem extension package (FMi)}
%
% \ProvidesFile{theorem.dtx}[\FMithmInfo]
%\iffalse   % this is a METACOMMENT !
%%
%% Package `theorem' to use with LaTeX2e
%% Copyright (C) 1989-1995 Frank Mittelbach, all rights reserved.
%<+package>\@ifundefined{theorem@style}{}{\endinput}
%<+package>\NeedsTeXFormat{LaTeX2e}[1995/06/01]
%<+package>\ProvidesPackage{theorem}
%<+package>         [\FMithmInfo]
%
% \fi
%
% \changes{v1.0}{87/05/07}{First draft.}
% \changes{v1.0a}{87/05/07}{`thistheoremstyle' added.}
% \changes{v1.1a}{87/07/29}{New concept: redefine the internal
%                          \protect\LaTeX{} commands `@xnthm' etc.}
% \changes{v1.1b}{87/07/31}{`theoremfont' used to specify font.}
% \changes{v1.2a}{88/01/08}{`@endtheorem' now globally defined.}
% \changes{v1.2b}{88/01/11}{`newtheorem' only allowed in the preamble.}
% \changes{v1.3a}{88/02/14}{`theoremfont' now a toks register.}
%
% \changes{v2.0a}{88/11/16}{New implementation and new documentation.}
% \changes{v2.0a}{88/11/16}{No layout changes but internal modifications
%                           old substyles won't work.}
% \changes{v2.0b}{89/01/30}{`global' used for all definitions.}
% \changes{v2.0c}{89/06/15}{Documentation translated to English.
%                            Thanks to LLL and Nina}
% \changes{v2.1a}{90/02/15}{Support for new font selection.}
% \changes{v2.2a}{94/02/03}{Cleanup for LaTeX2e}
%
% \DoNotIndex{\@endparenv,\@ifdefinable,\@ifnextchar,\@ifundefined}
% \DoNotIndex{\@namedef,\@nameuse,\@spaces,\@tempa,\@warning}
% \DoNotIndex{\ }
% \DoNotIndex{\begingroup,\bfseries}
% \DoNotIndex{\csname}
% \DoNotIndex{\def,\do,\docdate}
% \DoNotIndex{\endcsname,\endgroup,\endinput,\endtrivlist,\expandafter}
% \DoNotIndex{\filedate,\fileversion}
% \DoNotIndex{\gdef,\global}
% \DoNotIndex{\hbox,\hskip}
% \DoNotIndex{\ignorespaces,\input,\itshape,\item}
% \DoNotIndex{\labelsep,\let,\llap}
% \DoNotIndex{\makeatletter}
% \DoNotIndex{\noexpand}
% \DoNotIndex{\refstepcounter,\relax,\rlap}
% \DoNotIndex{\setlength,\slshape,\space,\string,\strut}
% \DoNotIndex{\the,\trivlist,\typeout}
% \DoNotIndex{\vbox}
% \DoNotIndex{\xdef}
%
% \def\deflabel#1{\ttfamily #1\hfill}
% \def\deflist#1{\begin{list}{}{\settowidth\labelwidth{\ttfamily #1}%
%                               \setlength\leftmargin\labelwidth
%                               \addtolength\leftmargin\labelsep
%                               \let\makelabel\deflabel}}
% \def\enddeflist{\end{list}}
% \def\envfont{\normalfont\ttfamily}
%
% \MakeShortVerb{\|}
%
% \newcommand{\TUB}{{\sl TUGboat\/}}
%
% \GetFileInfo{theorem.dtx}
% \title{An Extension of the
%        \LaTeX{} theorem environment\thanks{This file
%        has version number \fileversion, last
%        revised \filedate.}}
% \author{Frank Mittelbach}
%
% \MaintainedByLaTeXTeam{tools}
% \maketitle
%
% \begin{abstract}
%    The macros described in this paper yield an extension of the
%    \LaTeX{} theorem mechanism. It is designed to satisfy the
%    different requirements of various journals. Thus, the
%    layout of the ``theorems'' can be manipulated by determining a
%    ``style''. This article describes not only the use, but
%    also the definition, of the necessary macros.
% \end{abstract}
%
% \section*{Preface to version 2.2}
%
% For \LaTeXe{} this package did not need any fundamental changes. I
% only modified the messages generated so that theorem layout styles
% will show up with the |\listfiles| command and cleaned the section
% on the New Font Selection Scheme since this is now included in
% \LaTeX.
%
% \section*{Preface to version 2.1}
%
% This version is identical to 2.0g described in \TUB\ 10\#3 except
% for some internal defaults which are now set depending on the used
% font selection scheme.
%
% This was done to avoid unpleasant surprises if the new font selection
% scheme is in force. For further details see
% section~\ref{sec:newfonts} and \cite{bk:GMS94}.
%
% \section{Introduction}
%
% For our purposes here, ``theorems'' are labelled enunciations,
% often set off from the main text by extra space and a font change.
% Theorems, corollaries, conjectures, definitions, and remarks
% are all instances of ``theorems''.  The ``header'' of these structures
% is composed of a label (such as \textsc{Theorem} or \textsc{Remark})
% and a number which serializes an item in the sequence of items
% with the same label.
%
% Shortly after the introduction of \LaTeX{} at the Fachbereich
% Mathematik in Mainz, the desire to manipulate the layout of
% ``theorems'' arose.  In Mainz, the following two conventions came
% into general use:
% \begin{enumerate}
%   \item The number of the theorem is shown in the margin.
%   \item There is a line break at
%         the end of the theorem header.
% \end{enumerate}
%
% Additionally, some journals require different formats which depend on
% the ``sort of theorem'': e.g.\ often remarks and definitions are set
% in |\upshape|, while |\itshape| is employed for main theorems.
%
% Confronted with these requirements, a theorem environment was
% developed in Mainz which allows separate determination of the
% layout of the ``theorems sets'', comparable to |\pagestyle|.
%
%
% \section{The user interface}
% \subsection{Defining new theorem sets}
%
% \DescribeMacro\newtheorem
% As in the original \LaTeX{} version, the command |\newtheorem|
% defines a new ``theorem set'' or ``theorem-like structure''.
% Two required arguments name the new environment and give the text
% to be typeset with each instance of the new ``set'', while
% an optional argument determines how
% the ``set'' is enumerated:
% \begin{description}
%    \item[\ttfamily \bslash newtheorem\{foo\}\{bar\}]
%       The theorem set {\envfont foo} (whose name is \texttt{bar})
%       uses its own counter.
%    \item[\ttfamily \bslash newtheorem\{foo2\}{[foo]}\{bar2\}]
%       The theorem set {\envfont foo2} (printed name \texttt{bar2})
%       uses the same counter as the theorem set \texttt{foo}.
%    \item[\ttfamily \bslash newtheorem\{foo3\}\{bar3\}{[section]}]
%       The theorem set {\envfont foo3} (printed name \texttt{bar3}) is
%       enumerated within the counter \texttt{section}, i.e.\ with every
%       new |\section| the enumeration begins again with 1, and
%       the enumeration is composed from the section-number and the
%       theorem counter itself.
% \end{description}
%
% \DescribeMacro\theoremstyle
% Additionally, the command |\theoremstyle| can define the layout
% of various, or all, theorem sets.  It should be noted that any
% theorem set defined by |\newtheorem| is typeset in the
% |\theoremstyle| that is current at the time of the definition.
% Thus, the following
% \begin{verbatim}
%    \theoremstyle{break}        \newtheorem{Cor}{Corollary}
%    \theoremstyle{plain}        \newtheorem{Exa}{Example}[section]
%\end{verbatim}                             ^^A No blank at line start
% leads to the result that the set {\envfont Cor} is formatted in the
% style \texttt{break}, while the set {\envfont Exa} and all the
% following ones are formatted in the style \texttt{plain}, unless
% another |\theoremstyle| follows.  Since the definitions installed by
% |\newtheorem| are global, one also can limit |\theoremstyle| locally
% by grouping braces.
%
% \DescribeMacro\theorembodyfont
% The choice of the font for the theorem body is completely
% independent of the chosen |\theoremstyle|; this has proven to
% be very advantageous.  For example,
% \begin{verbatim}
%   {\theorembodyfont{\upshape}          \newtheorem{Rem}{Remark}}
%\end{verbatim}
% defines a theorem set {\envfont Rem}, which will be set in
% |\upshape| in the current layout (which in our example is
% \texttt{plain}).  As with |\theoremstyle|, the |\theorembodyfont|
% chosen is that current at the time of |\newtheorem|.  If
% |\theorembodyfont| is not specified or one defines
% |\theorembodyfont{}|, then the font used will be that defined by the
% |\theoremstyle|.
%
% \DescribeMacro\theoremheaderfont It is also possible to customize
% the font used for the theorem headers.  This is, however, a global
% declaration, and therefore there should be at most one
% |\theoremheaderfont| declaration in the preamble.\footnote{If it is
% actually necessary to have different header fonts, one has to define
% new theorem styles (substituting the desired font) or specify the
% information directly in the \texttt{\bslash newtheorem} declaration
% (the unclean variant).}
%
% \DescribeMacro\theorempreskipamount
% \DescribeMacro\theorempostskipamount Two additional parameters
% affect the vertical space around the theorem environments:
% |\theorempreskipamount| and |\theorempostskipamount| define,
% respectively, the spacing before and after such an environment.
% These parameters apply for all theorem sets and can be manipulated
% with the ordinary length macros.  They are rubber lengths
% (`\textsf{skips}'), and therefore can contain \texttt{plus} and
% \texttt{minus} parts.
%
% Since the definition of theorem sets should---most sensibly---be
% placed in the preamble, we only allow installation there.
% It is therefore possible to release the memory used here after
% |\begin{document}|, in order to make room for other
% applications.
%
%
% \subsection{Existing theorem styles}
%
% The following theorem styles exist to date:
% \begin{deflist}{marginbreak}
%    \item[plain]
%       This theorem style emulates the original \LaTeX{} definition,
%       except that additionally the parameters
%       |\theorem...skipamount| are used.
%    \item[break]
%       In this style, the theorem header is followed by a line
%       break.
%    \item[marginbreak]
%       The theorem number is set in the margin, and there is a line
%       break as in \texttt{break}.
%    \item[changebreak]
%       Like \texttt{break}, but with header number and text
%       interchanged.
%    \item[change]
%       Header number and text are interchanged, without a line break.
%    \item[margin]
%       The number is set in the left margin, without a line break.
% \end{deflist}
% All styles (except \texttt{plain}) select |\slshape| as the default
% |\theorembodyfont|.
%
%
% \subsection{Examples}
%
% Given the above theorem sets {\envfont Cor}, {\envfont Exa}
% and {\envfont Rem},
% suppose that the preamble also contains the declarations:
% \begin{verbatim}
%    \theoremstyle{marginbreak}   \newtheorem{Lem}[Cor]{Lemma}
%    \theoremstyle{change}
%    \theorembodyfont{\itshape}        \newtheorem{Def}[Cor]{Definition}
%
%    \theoremheaderfont{\scshape}
%\end{verbatim}
% Then the following are some typical examples of the typeset output
% resulting from their use.
%
% \begin{Cor}
%    This is a sentence typeset in the theorem environment {\envfont
%    Cor}.
% \end{Cor}
%
% \begin{Exa}
%    This is a sentence typeset in the theorem environment {\envfont
%    Exa}.
% \end{Exa}
%
% \begin{Rem}
%    This is a sentence typeset in the theorem environment {\envfont
%    Rem}.
% \end{Rem}
%
% \begin{Lem}[Ben User]
%    This is a sentence typeset in the theorem environment {\envfont
%    Lem}.
% \end{Lem}
%
% \begin{Def}[Very impressive Definition]
%    This is a sentence typeset in the theorem environment {\envfont
%    Def}.
% \end{Def}
%
% The last two examples show the effect of the optional argument to
% a theorem environment (it is the text typeset in parentheses).
%
%
% \section{Special Considerations}
% \label{sec:newfonts}
%
% Theoremheader and body are implemented as a unit. This means that
% the |\theoremheaderfont|  will inherit characteristics of the
% |\theorembodyfont| in \LaTeXe. Thus, if for example
% |\theorembodyfont| is |\itshape| and |\theoremheaderfont|
% is |\bfseries| the font selected for the header will have the
% characteristics `bold extended italic'. If this is not desired one
% should set the |\theoremheaderfont| to something like
% \begin{verbatim}
%      \theoremheaderfont{\normalfont\bfseries}
%\end{verbatim}
% i.e.\ supplying all necessary font information explicitly.
%
%
%
% \section{Acknowledgements}
%
% The publication of this set of macros was only possible with the
% help of Christina Busse (translating the manuscript into English),
% Joachim Pense (playing the r\^ole of typist), Chris Rowley
% (looking everything over) and many others providing useful
% suggestions.
%
% \MaybeStop{%
%    \begin{thebibliography}{9}
%       \bibitem{bk:GMS94} \textsc{M.~Goossens}, \textsc{F.~Mittelbach}
%         and \textsc{A.~Samarin}.
%         \newblock The \LaTeX{} Companion.
%         \newblock
%         Addison-Wesley, Reading, Massachusetts, 1994.
%       \bibitem{latex:source} \textsc{Lamport, Leslie.} \newblock
%                               \texttt{latex.tex}, version 2.09, date
%                               Feb. 1990.
%    \end{thebibliography}
%    \PrintIndex
%    \PrintChanges}
%
%
% \section{The documentation driver file}
%
% The next bit of code contains the documentation driver file for
% \TeX{}, i.e., the file that will produce the documentation you are
% currently reading. It will be extracted from this file by the
% \texttt{docstrip} program. Since it is the first code in the file
% one can alternatively process this file directly with \LaTeXe{} to
% obtain the documentation.
%
%    \begin{macrocode}
%<*driver>
\documentclass{ltxdoc}

\usepackage{theorem}

%<+driver>% The next few lines define theorem sets which are used
%<+driver>% in the example section of the documentation.

\theoremstyle{break}       \newtheorem{Cor}{Corollary}
\theoremstyle{plain}       \newtheorem{Exa}{Example}[section]
{\theorembodyfont{\upshape}\newtheorem{Rem}{Remark}}
\theoremstyle{marginbreak} \newtheorem{Lem}[Cor]{Lemma}
\theoremstyle{change}
\theorembodyfont{\itshape} \newtheorem{Def}[Cor]{Definition}

\theoremheaderfont{\scshape}

\RecordChanges

\begin{document}
   \DocInput{theorem.dtx}
\end{document}
%</driver>
%    \end{macrocode}
%
% \section{Definition of the Macros}
%
% If the file has been loaded before, we abort immediately. If not the
% package announces itself (this is actually done at the very top of
% the file---the way it is done isn't good style so don't copy it).
% \changes{v2.0e}{89/07/19}{Spaces removed from `typeout'.}
% \changes{v2.2b}{95/11/19}{Announce moved to top of file}
%    \begin{macrocode}
%<*package>
%\@ifundefined{theorem@style}{}{\endinput}
%\def\FMithmInfo{1995/11/19 v2.2b Theorem extension package (FMi)}
%\ProvidesPackage{theorem}[\FMithmInfo]
%    \end{macrocode}
%
%
% \subsection{Definition of theorem styles and fonts}
%
% All the definitions in this file are done globally to allow
% inputting this file inside a group.
%
% \begin{macro}{\theoremstyle}
%    Before a theorem style can be installed, the chosen style must be
%    known. For that reason, we must test to see that
%    |\th@|\meta{style} is known or, more precisely, that it is
%    different from |\relax|. If the style is not known then
%    |\th@plain| is used.
%    \begin{macrocode}
\gdef\theoremstyle#1{%
   \@ifundefined{th@#1}{\@warning
          {Unknown theoremstyle `#1'. Using `plain'}%
          \theorem@style{plain}}%
%    \end{macrocode}
%    We save the theorem style to be used in the token register
%    |\theorem@style|.
%    \begin{macrocode}
      {\theorem@style{#1}}%
%    \end{macrocode}
%    Now we ``evaluate'' the theorem style: this means, we call the
%    macro |\th@|\meta{style} which will activate the relevant
%    definitions which are contained in a separate file.  This is done
%    in a group to suppress changes to the current font.  This could
%    otherwise pose problems together with the new font selection
%    scheme\footnote{When I printed the original article using the new
%    font selection scheme I ended with a document with slanted
%    typefaces (text headings and all) simply because one of the
%    theorem styles used \texttt{\bslash sl} at toplevel.} if the
%    |\th@|\meta{style} is evaluated a second time.
% \changes{v2.1a}{90/02/15}{Group added around `th@..}
% \changes{v2.1a}{90/02/15}{Reseting of `..@begintheorem removed.}
%    \begin{macrocode}
      \begingroup
        \csname th@\the\theorem@style \endcsname
      \endgroup}
%    \end{macrocode}
% \end{macro}
%
%
% \begin{macro}{\@begintheorem}
% \changes{v2.1a}{90/02/15}{Relaxed at toplevel.}
% \begin{macro}{\@opargbegintheorem}
% \changes{v2.1a}{90/02/15}{Relaxed at toplevel.}
%    We reset |\@begintheorem| and |\@opargbegintheorem| to
%    |\relax| since these commands are no longer necessary at
%    toplevel. This will save a few tokens.
%    \begin{macrocode}
\global\let\@begintheorem\relax
\global\let\@opargbegintheorem\relax
%    \end{macrocode}
% \end{macro}
% \end{macro}
%
% \begin{macro}{\theorem@style}
%    Obviously the \textsf{token} register used above has to be
%    allocated. To assure the utmost compatibility with the
%    original \LaTeX{} definition, we set the default theorem style to
%    \texttt{plain}, which implements the usual \LaTeX{} convention.
% \changes{v2.0f}{89/07/20}{Unnecessary `global' removed.}
%    \begin{macrocode}
\newtoks\theorem@style
\global\theorem@style{plain}
%    \end{macrocode}
% \end{macro}
%
% \begin{macro}{\theorembodyfont}
% \begin{macro}{\theorem@bodyfont}
%    If the |\theorembodyfont| is set by the user then it should not
%    interact with the default font set in the theorem style. When the
%    new font selection is in force this may happen if, for example,
%    the default is |\itshape| and the new |\theorembodyfont| is
%    |\sffamily|. So we add a |\reset@font| command in front of the
%    user definition.
% \changes{v2.0f}{89/07/20}{Unnecessary `global' removed.}
% \changes{v2.0g}{89/09/19}{`theoremfont to `theorembodyfont.}
% \changes{v2.1c}{92/06/22}{`reset@font added and logic changed.}
% \changes{v2.1d}{92/10/28}{Reset to empty if arg is empty}
%    \begin{macrocode}
\gdef\theorembodyfont#1{%
%    \end{macrocode}
%    We check if the argument supplied is empty and if so put nothing
%    into the |\theorem@bodyfont| token register to allow for
%    |\theorembodyfont{}| as a mean of using the default of the
%    current |\theoremstyle|.
%    \begin{macrocode}
   \def\@tempa{#1}%
   \ifx\@tempa\@empty
    \theorem@bodyfont{}%
   \else
    \theorem@bodyfont{\reset@font#1}%
   \fi
   }
\newtoks\theorem@bodyfont
\global\theorem@bodyfont{}
%    \end{macrocode}
% \end{macro}
% \end{macro}
%
%
% \begin{macro}{\theoremheaderfont}
% \changes{v2.0g}{89/09/19}{Macro added.}
%    The font for the theorem headers is handled differently because
%    this definition applies to all theorem styles.
%    \begin{macrocode}
\gdef\theoremheaderfont#1{\gdef\theorem@headerfont{#1}%
%    \end{macrocode}
%    After using the macro once it is redefined to produce an error
%    message.
%    \begin{macrocode}
       \gdef\theoremheaderfont##1{%
        \typeout{\string\theoremheaderfont\space should be used
                 only once.}}}
%    \end{macrocode}
% \end{macro}
%
% \begin{macro}{\theorem@headerfont}
% \changes{v2.0g}{89/09/19}{Macro added.}
%    To set the |\theorem@headerfont| default we first test if
%    the new fontselection scheme is in force.
% \changes{v2.1a}{90/02/15}{Support for new font selection.}
%    \begin{macrocode}
\ifx\upshape\undefined
%    \end{macrocode}
%    If not we define it to expand into |\bfseries|. We don't use
%    |\let| just in case a following style option redefines this
%    macro.
%    \begin{macrocode}
\gdef\theorem@headerfont{\bfseries}
%    \end{macrocode}
%    Otherwise we reset the current shape before calling |\bfseries|.
%    \begin{macrocode}
\else \gdef\theorem@headerfont{\normalfont\bfseries}\fi
%    \end{macrocode}
% \end{macro}
%
% \begin{macro}{\th@plain}
% \begin{macro}{\th@break}
% \begin{macro}{\th@marginbreak}
% \begin{macro}{\th@changebreak}
% \begin{macro}{\th@change}
% \begin{macro}{\th@margin}
%    The different styles are defined in macros such as
%    |\th@plain|.  Since memory space is precious in
%    ``non-Big-versions'', we have to avoid offering too many unused
%    definitions.  Therefore we define these styles in separate files
%    that can be loaded on demand.  Thus the commands themselves only
%    load these files. We use |\@input@|, a \LaTeXe{} internal command
%    that ensures that the file will be listed with |\listfiles|.
% \changes{v2.2a}{94/02/02}{Use `@input@ to load theorem layout files}
%    \begin{macrocode}
\gdef\th@plain{\@input@{thp.sty}}
\gdef\th@break{\@input@{thb.sty}}
\gdef\th@marginbreak{\@input@{thmb.sty}}
\gdef\th@changebreak{\@input@{thcb.sty}}
\gdef\th@change{\@input@{thc.sty}}
\gdef\th@margin{\@input@{thm.sty}}
%    \end{macrocode}
%    This list will be expanded when new styles become
%    available.  For testing,  just append new theorem substyles
%    as document options.
% \end{macro}
% \end{macro}
% \end{macro}
% \end{macro}
% \end{macro}
% \end{macro}
%
%
%
% \subsection{Definition of a new theorem set}
%
% As already pointed out, a new theorem environment can be defined in
% three different ways:
% \begin{verbatim}
%    \newtheorem{Lem}{Lemma}
%    \newtheorem{Lem}{Lemma}[section]
%    \newtheorem{Lem}[Theorem]{Lemma}
%\end{verbatim}
% The function of the macro |\newtheorem| is to recognize these cases
% and then to branch into one of the three macros |\@ynthm|, |\@xnthm|
% or |\@othm|.  This mechanism is adopted unchanged from
% \cite{latex:source}; the essential point here is that, for example,
% in the second case, the arguments \texttt{Lem}, \texttt{Lemma} and
% \texttt{section} are passed over to the macro |\@xnthm|.
%
% We inspect this case first because the others present fewer problems,
% and thus are easily derived from this one.
%
% \begin{macro}{\@xnthm}
%    For our example arguments, the macro |\@xnthm| must fulfill
%    the following:
%    \begin{itemize}
%       \item
%          Define a new \LaTeX-counter `Lem'
%       \item
%          reset this counter within a |\section|
%       \item
%          define the macro |\theLem|
%       \item
%          define the environment macros |\Lem| and
%          |\endLem| using the current
%          |\theoremstyle| and |\theorem@bodyfont|.
%    \end{itemize}
%    Obviously, all this should happen only if the first argument
%    of |\@xnthm| (i.e.\ \texttt{Lem} in our example) is chosen so
%    as not to conflict with any previously defined commands or
%    environments.  This test is performed by the \LaTeX{} macro
%    |\@ifdefinable|.
%    \begin{macrocode}
\gdef\@xnthm#1#2[#3]{\expandafter\@ifdefinable\csname #1\endcsname
%    \end{macrocode}
%    Therefore, the first argument of |\@ifdefinable| is the
%    expansion (in the example, |\Lem|) of
%    |\csname||#1\endcsname|.  The second argument is
%    executed only if the test has been completed successfully.
%    \begin{macrocode}
   {%
%    \end{macrocode}
%    Now we define the new counter. The names of the \LaTeX{}
%    macros employed should speak for themselves:
% \changes{v2.2b}{1995/10/07}{Use \cs{@newctr}}
% \changes{v2.2c}{1995/11/23}{Use \cs{@newctr} properly.}
%    \begin{macrocode}
    \@definecounter{#1}\@newctr{#1}[#3]%
%    \end{macrocode}
%    Using |\@newctr| will give a proper error message if the counter
%    in |#3| is not defined.
%    In defining `|\theLem|' we must generate the desired macro
%    name by use of |\expandafter| and |\csname|.
%    \begin{macrocode}
    \expandafter\xdef\csname the#1\endcsname
%    \end{macrocode}
%    An |\xdef| is used in order to make the definition global,
%    and to ensure that it contains the replacement texts of
%    |\@thmcountersep| and |\@thmcounter|.\footnote{These
%    two macros can be defined by the document style.  Their default
%    values produce a `\texttt{.}' as separation and an arabic
%    representation of the number.} However, not everything should be
%    expanded. For example, it saves space to use |\thesection|
%    instead of its---at times---lengthy expansion.
%    \begin{macrocode}
      {\expandafter \noexpand \csname the#3\endcsname
       \@thmcountersep \@thmcounter{#1}}%
%    \end{macrocode}
%    Thus with the defaults of \LaTeX{}, |\theLem| would be
%    replaced by the command sequence |\thesection.\arabic{Lem}|.
%
%    We will now look at the definition of the macro which is executed
%    at the beginning of the actual environment (in our example this
%    macro is |\Lem|). It should be noted that we use an
%    ``|\expandafter| trick'' to expand only certain parts of the
%    replacement text at the time of the definition.
%    \begin{macrocode}
    \def\@tempa{\global\@namedef{#1}}%
    \expandafter \@tempa \expandafter{%
%    \end{macrocode}
%    First, the macro that contains the current definitions of
%    |\@begintheorem| and |\@opargtheorem| should be called
%    up.  The name of this macro---as is already known---has the form
%    |\th@|\meta{theorem\ style}; therefore, it must be called by
%    \begin{macrocode}
      \csname th@\the \theorem@style
%    \end{macrocode}
%    In addition the default theorem font should be changeable, i.e.\
%    we have to insert the contents of |\theorem@bodyfont|. For
%    that reason, we expand even further, beyond |\endcsname|,
%    and thus insert the contents of the \textsf{token} register
%    |\theorem@bodyfont| in the replacement text.
%    \begin{macrocode}
            \expandafter \endcsname \the \theorem@bodyfont
%    \end{macrocode}
%    Now it is time to call the macro |\@thm| which takes over
%    the further processing.  It has two arguments: the current
%    counter name (in our example, \texttt{Lem}), and the text of the
%    label (in our example, \texttt{Lemma}).
%    \begin{macrocode}
     \@thm{#1}{#2}}%
%    \end{macrocode}
%    With this, the `sub-definition' is complete.  The macro
%    |\@endtheorem| ends a theorem environment and is, so far, nothing
%    but an |\endtrivlist|.  (Hence it is defined globally, and not
%    within the theorem styles.\footnote{This has to be changed as
%    soon as theorem styles that change \texttt{\bslash @endtheorem}
%    exist.  In such a case, all existing styles must be changed as
%    well since they will have to reset the macro.}) Therefore, we can
%    set it equivalent to the macro that ends the theorem set (in our
%    example, |\endLem|).  However, if some day theorem styles exist
%    that do change |\@endtheorem|, we would have to use the
%    commented-out line instead.
%    \begin{macrocode}
    \global \expandafter \let \csname end#1\endcsname \@endtheorem
%   \global\@namedef{end#1}{\@endtheorem}%
%    \end{macrocode}
%    With these commands all the required definitions are employed,
%    unless the test |\@ifdefinable| has failed.  Therefore, we
%    end the second argument of this macro and with it the definition
%    of |\@xnthm|.
%    \begin{macrocode}
   }}
%    \end{macrocode}
% \end{macro}
%
% \begin{macro}{\@ynthm}
%    The definition of |\@ynthm| is completely analogous. In
%    this case the new counter that is defined is not reset within
%    another counter; thus the definition of |\the...|
%    is simplified:
%    \begin{macrocode}
\gdef\@ynthm#1#2{\expandafter\@ifdefinable\csname #1\endcsname
   {\@definecounter{#1}%
    \expandafter\xdef\csname the#1\endcsname{\@thmcounter{#1}}%
%    \end{macrocode}
%    The rest of the definition corresponds literally to that of
%    |\@xnthm|:
%    \begin{macrocode}
    \def\@tempa{\global\@namedef{#1}}\expandafter \@tempa
     \expandafter{\csname th@\the \theorem@style \expandafter
     \endcsname \the\theorem@bodyfont \@thm{#1}{#2}}%
    \global \expandafter \let \csname end#1\endcsname \@endtheorem}}
%    \end{macrocode}
% \end{macro}
%
% \begin{macro}{\@othm}
%    The definition of |\@othm| does not contain anything new.
%    \begin{macrocode}
\gdef\@othm#1[#2]#3{%
%    \end{macrocode}
% \changes{v2.2b}{95/10/07}{Build in check for counter defined}
%    We do not define a new counter but instead use one that has
%    already been defined.  Thus the only definition we need is that
%    of this pseudo-counter (i.e.\ |\the|\meta{env.\ name}).
%    First we check if |#2| corresponds to a known counter name.
%    \begin{macrocode}
  \expandafter\ifx\csname c@#2\endcsname\relax
   \@nocounterr{#2}%
  \else
   \expandafter\@ifdefinable\csname #1\endcsname
   {\expandafter \xdef \csname the#1\endcsname
     {\expandafter \noexpand \csname the#2\endcsname}%
%    \end{macrocode}
%    All other parts of the definition can be adopted from
%    |\@xnthm|. We have to remember, though, that in this case
%    the name of the current counter and the theorem label have moved
%    to the second and third arguments.
%    \begin{macrocode}
    \def\@tempa{\global\@namedef{#1}}\expandafter \@tempa
     \expandafter{\csname th@\the \theorem@style \expandafter
     \endcsname \the\theorem@bodyfont \@thm{#2}{#3}}%
    \global \expandafter \let \csname end#1\endcsname \@endtheorem}%
  \fi}
%    \end{macrocode}
% \end{macro}
%
% \subsection{Macros that are employed in a {\envfont theorem}
%    environment}
%
% \begin{macro}{\@thm}
%    The macro |\@thm| has to increase the current counter.  Then,
%    depending on whether the environment has (or does not have) an
%    optional argument, it has to branch into either |\@begintheorem|
%    or |\@opargtheorem|.
%    \begin{macrocode}
\gdef\@thm#1#2{\refstepcounter{#1}%
%    \end{macrocode}
%    Now we start a {\envfont trivlist} environment, and give
%    |\@topsep| and |\@topsepadd| the values of the \textsf{skip}
%    registers |\theorempreskipamount| and |\theorempostskipamount|.
%    The value in |\@topsep| is the vertical space that is inserted by
%    the first (and only) |\item| in our |\trivlist| whilst
%    |\@topsepadd| is inserted by |\@endparenv| at the end of that
%    {\envfont trivlist} environment.  By using these registers, we
%    obtain the desired space around a {\envfont theorem} environment.
%    \begin{macrocode}
   \trivlist
   \@topsep \theorempreskipamount               % used by first \item
   \@topsepadd \theorempostskipamount           % used by \@endparenv
%    \end{macrocode}
%    Now we have to test whether an optional argument has been given.
% \changes{v2.0d}{89/07/18}{Forgotten percent added.}
%    \begin{macrocode}
   \@ifnextchar [%
%    \end{macrocode}
%    If there is an optional argument, we will call |\@ythm|, and
%    move the arguments read back into the input stream.
%    \begin{macrocode}
   {\@ythm{#1}{#2}}%
%    \end{macrocode}
%    If not, we call |\@begintheorem|. Its first argument is the
%    name of the theorem set (hence the second argument of
%    |\@thm|). Its second argument is the macro that produces the
%    current number.
%    \begin{macrocode}
   {\@begintheorem{#2}{\csname the#1\endcsname}\ignorespaces}}
%    \end{macrocode}
% \end{macro}
%
% \begin{macro}{\@xthm}
% \begin{macro}{\@ythm}
%    Both these macros were originally called by |\@thm|.  We do
%    not need |\@xthm| anymore, hence we reset it to
%    |\relax|. The definition of |\@ythm| has not changed at
%    all from its definition in \LaTeX. In order to make the macros
%    easier to understand, we will
%    nevertheless present it (commented out).
%    \begin{macrocode}
\global\let\@xthm\relax
% \def\@ythm#1#2[#3]{\@opargbegintheorem{#2}{\csname
%           the#1\endcsname}{#3}\ignorespaces}
%</package>
%    \end{macrocode}
%    The primitive |\ignorespaces| in |\@ythm| and
%    |\@thm| is needed to remove the spaces between the
%    |\begin{...}| and the actual text.
% \end{macro}
% \end{macro}
%
%
% \subsection{Definition of the theorem substyles}
%
% As already pointed out, the theorem substyles, defined below, are
% only loaded when necessary.  Note that all these substyles, except
% \texttt{plain}, have |\slshape| as the default body font.
%
% \subsubsection{The \texttt{plain} style}
%
% \label{dok:th@plain} As the following macros use |@|, we have
% to locally set the |\catcode| of this symbol to ``letter''.
% This happens within a group, so that we do not have to worry about
% which |\catcode| that symbol had before.
%    \begin{macrocode}
%<*thp>
\begingroup \makeatletter
%    \end{macrocode}
% Since we are now within a group, we must make all definitions
% globally.  First we make sure that \texttt{theorem.sty} is loaded.
% This will allow us to use this file as a document style option
% without having to call \texttt{theorem} itself as an option.  At the
% same time, we assure that at least version 2 is loaded, since
% |\theorem@style| was not defined in earlier versions.
%    \begin{macrocode}
\@ifundefined{theorem@style}{\input{theorem.sty}}{}
\ProvidesFile{thp.sty}
             [\FMithmInfo]
%    \end{macrocode}
%
%
% \begin{macro}{\th@plain}
% \begin{macro}{\@begintheorem}
% \begin{macro}{\@opargbegintheorem}
%    |\theoremstyle{plain}| corresponds to the original
%    definition, except that the distances to the surrounding text are
%    determined by the parameters |\theorempreskipamount| and
%    |\theorempostskipamount|.
%    First we set the default body font.
%    \begin{macrocode}
\gdef\th@plain{\normalfont\itshape
%    \end{macrocode}
%    Then we define |\@begintheorem| and
%    |\@opargbegintheorem|. These two macros define how the
%    header of a theorem is typeset.  |\@opargbegintheorem| will
%    be called if a \textsf{theorem} environment with an optional
%    argument is encountered; otherwise, the header is constructed by
%    calling |\@begintheorem|.  If one of these macros is
%    executed, we are within a \textsf{trivlist} environment started by
%    |\@thm|.  So the theorem header is produced with an
%    |\item| command.
%
%    Instead of specifying the header font directly, all standard
%    theorem styles use the |\theorem@headerfont| macro to allow
%    customization.  The extra space (|\labelsep|) is necessary
%    because of problems in the \textsf{trivlist} environment.
%    \begin{macrocode}
  \def\@begintheorem##1##2{%
        \item[\hskip\labelsep \theorem@headerfont ##1\ ##2]}%
%    \end{macrocode}
%    The definition of |\@opargbegintheorem| is completely
%    analogous. The only difference is the fact that there exists a
%    third argument (which is the optional parameter of the
%    environment and contains additional information about the
%    theorem).  Customarily we enclose it in parentheses.
%    \begin{macrocode}
\def\@opargbegintheorem##1##2##3{%
   \item[\hskip\labelsep \theorem@headerfont ##1\ ##2\ (##3)]}}
%    \end{macrocode}
% \end{macro}
% \end{macro}
% \end{macro}
%
% \noindent
% We conclude with an |\endgroup| to restore the
% |\catcode| of |@|.
%    \begin{macrocode}
\endgroup
%</thp>
%    \end{macrocode}
%
%
% \subsubsection{The \texttt{break} style}
%
% This style option is stored in the file |thb.sty|.
% For the next two lines see the documentation for |\th@plain| on
% page~\pageref{dok:th@plain}.
%    \begin{macrocode}
%<*thb>
\begingroup \makeatletter
\@ifundefined{theorem@style}{\input{theorem.sty}}{}
\ProvidesFile{thb.sty}
             [\FMithmInfo]
%    \end{macrocode}
%
% \begin{macro}{\th@break}
%    |\theoremstyle{break}| produces a line break after the name
%    of the theorem.  The font is |\slshape|.  Hence, we define
%    |\th@break| as follows:
%    \begin{macrocode}
\gdef\th@break{\normalfont\slshape
  \def\@begintheorem##1##2{\item[%
%    \end{macrocode}
%    We run into the following problem: it is not possible to create
%    the header with |\item[|\meta{title}|]| and then start
%    a new line by, for example, |\mbox{}\\|. Such a definition
%    will fail whenever a list environment follows immediately.  With
%    the above construction, the |\mbox{}| causes the switch
%    |@inlabel| (cf.\ definition of |\list| and
%    |\trivlist| in \cite{latex:source}) to be set to \texttt{false}
%    and so the following list will insert additional vertical
%    space (|\topskip|). This is quite annoying. Therefore, we
%    create the line break within the |\item|. In order to ensure
%    that the text will begin at the proper position in the following
%    line, we simply pretend that the label does not take any
%    room.\footnote{This will lead to problems whenever very high
%    symbols occurring in the line tower into the
%    heading. So, something else has to be done here sometime.}
%    \begin{macrocode}
       \rlap{\vbox{\hbox{\hskip \labelsep\theorem@headerfont ##1\ ##2}%
                   \hbox{\strut}}}]}%
%    \end{macrocode}
%    Again, the definition of |\@opargbegintheorem| is
%    completely analogous.
%    \begin{macrocode}
\def\@opargbegintheorem##1##2##3{%
  \item[\rlap{\vbox{\hbox{\hskip \labelsep \theorem@headerfont
                     ##1\ ##2\ (##3)}%
                    \hbox{\strut}}}]}}
\endgroup
%</thb>
%    \end{macrocode}
% \end{macro}
%
%
% \subsubsection{The \texttt{changebreak} style}
%
%    \begin{macrocode}
% This style option is stored in the file |thcb.sty|.
%    \begin{macrocode}
%<*thcb>
\begingroup \makeatletter
\@ifundefined{theorem@style}{\input{theorem.sty}}{}
\ProvidesFile{thcb.sty}
             [\FMithmInfo]
%    \end{macrocode}
%
% \begin{macro}{\th@changebreak}
%    The \texttt{change-break} theorem style is like \texttt{break}
%    but with interchange of theorem name and theorem number.  Thus we
%    define |\th@changebreak| as follows:
%    \begin{macrocode}
\gdef\th@changebreak{\normalfont\slshape
  \def\@begintheorem##1##2{\item
       [\rlap{\vbox{\hbox{\hskip\labelsep \theorem@headerfont ##2\ ##1}%
                    \hbox{\strut}}}]}%
\def\@opargbegintheorem##1##2##3{%
  \item[\rlap{\vbox{\hbox{\hskip\labelsep \theorem@headerfont
                     ##2\ ##1\ (##3)}%
                    \hbox{\strut}}}]}}
\endgroup
%</thcb>
%    \end{macrocode}
% \end{macro}
%
%
% \subsubsection{The \texttt{change} style}
%
% This style option is stored in the file |thc.sty|.
%    \begin{macrocode}
%<*thc>
\begingroup \makeatletter
\@ifundefined{theorem@style}{\input{theorem.sty}}{}
\ProvidesFile{thc.sty}
             [\FMithmInfo]
%    \end{macrocode}
%
% \begin{macro}{\th@change}
%    The \texttt{change} theorem style corresponds to the
%    \texttt{change break} style without a linebreak after the header.
%    To say it in another way, it's the same as the \texttt{plain}
%    style but with number and name interchanged and |\slshape| as the
%    default font.
%    \begin{macrocode}
\gdef\th@change{\normalfont\slshape
  \def\@begintheorem##1##2{\item
       [\hskip\labelsep \theorem@headerfont ##2\ ##1]}%
\def\@opargbegintheorem##1##2##3{%
  \item[\hskip\labelsep \theorem@headerfont ##2\ ##1\ (##3)]}}
\endgroup
%</thc>
%    \end{macrocode}
% \end{macro}
%
%
% \subsubsection{The \texttt{marginbreak} style}
%
% This style option is the one used most often at Mainz.  It is saved
% in the file |thmb.sty|.
%    \begin{macrocode}
%<*thmb>
\begingroup \makeatletter
\@ifundefined{theorem@style}{\input{theorem.sty}}{}
\ProvidesFile{thmb.sty}
             [\FMithmInfo]
%    \end{macrocode}
%
% \begin{macro}{\th@marginbreak}
%    The \texttt{margin break} style is nearly the same as the
%    \texttt{change break} style. The only difference is the placement
%    of the theorem number.  We use |\llap| to place it in the left
%    margin.
%
%    In this style |\labelsep| denotes the separation between the
%    number and the text.
%    \begin{macrocode}
\gdef\th@marginbreak{\normalfont\slshape
  \def\@begintheorem##1##2{\item
       [\rlap{\vbox{\theorem@headerfont
                    \hbox{\llap{##2}\hskip\labelsep ##1}%
                    \hbox{\strut}}}]}%
\def\@opargbegintheorem##1##2##3{%
  \item[\rlap{\vbox{\theorem@headerfont
                    \hbox{\llap{##2}\hskip\labelsep ##1\ (##3)}%
                    \hbox{\strut}}}]}}
\endgroup
%</thmb>
%    \end{macrocode}
% \changes{v2.1c}{90/02/17}{`endgroup outside macrocode env.}
% \end{macro}
%
%
%
% \subsubsection{The \texttt{margin} style}
%
% This style option is stored in the file |thm.sty|.
%    \begin{macrocode}
%<*thm>
\begingroup \makeatletter
\@ifundefined{theorem@style}{\input{theorem.sty}}{}
\ProvidesFile{thm.sty}
             [\FMithmInfo]
%    \end{macrocode}
%
% \begin{macro}{\th@margin}
%    Again this is only a variant of the theorem styles described
%    above without any new ideas.
%    \begin{macrocode}
\gdef\th@margin{\normalfont\slshape
  \def\@begintheorem##1##2{\item
       [\theorem@headerfont \llap{##2}\hskip\labelsep ##1]}%
\def\@opargbegintheorem##1##2##3{%
  \item[\theorem@headerfont \llap{##2}\hskip\labelsep ##1\ (##3)]}}
\endgroup
%</thm>
%    \end{macrocode}
% \end{macro}
%
%
%
%
% \subsection{Final Definitions}
%
% \begin{macro}{\theorempreskipamount}
% \begin{macro}{\theorempostskipamount}
%    The \textsf{skip} parameters that regulate the vertical empty space
%    before and after the {\envfont theorem} environment have
%    to be allocated as well.
% \changes{v2.0f}{89/07/20}{Unnecessary `global' removed.}
%    \begin{macrocode}
%<*package>
\newskip\theorempreskipamount
\newskip\theorempostskipamount
%    \end{macrocode}
%    Since we have used the same values for all theorem sets, we now can
%    assign them.
%    \begin{macrocode}
\global\setlength\theorempreskipamount{12pt plus 5pt minus 3pt}
\global\setlength\theorempostskipamount{8pt plus 3pt minus 1.5pt}
%    \end{macrocode}
% \end{macro}
% \end{macro}
%
%
% \begin{macro}{\@endtheorem}
%    The same holds for the macro |\@endtheorem|, which ends a
%    {\envfont theorem} environment. Since it is the same for all
%    theorem sets, it is removed from the macros |\th@|\meta{style}.
%    It simply ends the {\envfont trivlist} environment, which was
%    begun in |\@thm|.
%    \begin{macrocode}
\global\let\@endtheorem=\endtrivlist
%    \end{macrocode}
% \end{macro}
%
%
%
% \begin{macro}{\@preamblecmds}
%    All macros defined above are to be used only in the preamble.
%    Therefore, we insert them in |\@preamblecmds| which will disable
%    them at begin document. This is done by the internal \LaTeXe{}
%    command |\@onlypreamble|.
% \changes{v2.0g}{89/09/19}{`theoremheaderfont and `theorembodyfont
%    added.}
% \changes{v2.2a}{94/02/03}{Use \LaTeXe{} interface}
%    \begin{macrocode}
\@onlypreamble\@xnthm
\@onlypreamble\@ynthm
\@onlypreamble\@othm
\@onlypreamble\newtheorem
\@onlypreamble\theoremstyle
\@onlypreamble\theorembodyfont
\@onlypreamble\theoremheaderfont
%    \end{macrocode}
% \end{macro}
%
% Finally we declare the \texttt{plain} theorem style to be the default.
% \changes{v2.1b}{90/06/04}{Fixed missing `item by declaring default
%    style.}
%    \begin{macrocode}
\theoremstyle{plain}
%</package>
%    \end{macrocode}
%
% \Finale

\endinput
}{}
\ProvidesFile{thm.sty}
             [\FMithmInfo]
%    \end{macrocode}
%
% \begin{macro}{\th@margin}
%    Again this is only a variant of the theorem styles described
%    above without any new ideas.
%    \begin{macrocode}
\gdef\th@margin{\normalfont\slshape
  \def\@begintheorem##1##2{\item
       [\theorem@headerfont \llap{##2}\hskip\labelsep ##1]}%
\def\@opargbegintheorem##1##2##3{%
  \item[\theorem@headerfont \llap{##2}\hskip\labelsep ##1\ (##3)]}}
\endgroup
%</thm>
%    \end{macrocode}
% \end{macro}
%
%
%
%
% \subsection{Final Definitions}
%
% \begin{macro}{\theorempreskipamount}
% \begin{macro}{\theorempostskipamount}
%    The \textsf{skip} parameters that regulate the vertical empty space
%    before and after the {\envfont theorem} environment have
%    to be allocated as well.
% \changes{v2.0f}{89/07/20}{Unnecessary `global' removed.}
%    \begin{macrocode}
%<*package>
\newskip\theorempreskipamount
\newskip\theorempostskipamount
%    \end{macrocode}
%    Since we have used the same values for all theorem sets, we now can
%    assign them.
%    \begin{macrocode}
\global\setlength\theorempreskipamount{12pt plus 5pt minus 3pt}
\global\setlength\theorempostskipamount{8pt plus 3pt minus 1.5pt}
%    \end{macrocode}
% \end{macro}
% \end{macro}
%
%
% \begin{macro}{\@endtheorem}
%    The same holds for the macro |\@endtheorem|, which ends a
%    {\envfont theorem} environment. Since it is the same for all
%    theorem sets, it is removed from the macros |\th@|\meta{style}.
%    It simply ends the {\envfont trivlist} environment, which was
%    begun in |\@thm|.
%    \begin{macrocode}
\global\let\@endtheorem=\endtrivlist
%    \end{macrocode}
% \end{macro}
%
%
%
% \begin{macro}{\@preamblecmds}
%    All macros defined above are to be used only in the preamble.
%    Therefore, we insert them in |\@preamblecmds| which will disable
%    them at begin document. This is done by the internal \LaTeXe{}
%    command |\@onlypreamble|.
% \changes{v2.0g}{89/09/19}{`theoremheaderfont and `theorembodyfont
%    added.}
% \changes{v2.2a}{94/02/03}{Use \LaTeXe{} interface}
%    \begin{macrocode}
\@onlypreamble\@xnthm
\@onlypreamble\@ynthm
\@onlypreamble\@othm
\@onlypreamble\newtheorem
\@onlypreamble\theoremstyle
\@onlypreamble\theorembodyfont
\@onlypreamble\theoremheaderfont
%    \end{macrocode}
% \end{macro}
%
% Finally we declare the \texttt{plain} theorem style to be the default.
% \changes{v2.1b}{90/06/04}{Fixed missing `item by declaring default
%    style.}
%    \begin{macrocode}
\theoremstyle{plain}
%</package>
%    \end{macrocode}
%
% \Finale

\endinput
}{}
\ProvidesFile{thc.sty}
             [\FMithmInfo]
%    \end{macrocode}
%
% \begin{macro}{\th@change}
%    The \texttt{change} theorem style corresponds to the
%    \texttt{change break} style without a linebreak after the header.
%    To say it in another way, it's the same as the \texttt{plain}
%    style but with number and name interchanged and |\slshape| as the
%    default font.
%    \begin{macrocode}
\gdef\th@change{\normalfont\slshape
  \def\@begintheorem##1##2{\item
       [\hskip\labelsep \theorem@headerfont ##2\ ##1]}%
\def\@opargbegintheorem##1##2##3{%
  \item[\hskip\labelsep \theorem@headerfont ##2\ ##1\ (##3)]}}
\endgroup
%</thc>
%    \end{macrocode}
% \end{macro}
%
%
% \subsubsection{The \texttt{marginbreak} style}
%
% This style option is the one used most often at Mainz.  It is saved
% in the file |thmb.sty|.
%    \begin{macrocode}
%<*thmb>
\begingroup \makeatletter
\@ifundefined{theorem@style}{% \iffalse meta-comment
%
% Copyright (C) 1993-2024
%
% The LaTeX Project and any individual authors listed elsewhere
% in this file.
%
% This file is part of the Standard LaTeX `Tools Bundle'.
% -------------------------------------------------------
%
% It may be distributed and/or modified under the
% conditions of the LaTeX Project Public License, either version 1.3c
% of this license or (at your option) any later version.
% The latest version of this license is in
%    https://www.latex-project.org/lppl.txt
% and version 1.3c or later is part of all distributions of LaTeX
% version 2005/12/01 or later.
%
% The list of all files belonging to the LaTeX `Tools Bundle' is
% given in the file `manifest.txt'.
%
% \fi
%
%%
%
%
\def\FMithmInfo{2023/07/05 v2.2c Theorem extension package (FMi)}
%
% \ProvidesFile{theorem.dtx}[\FMithmInfo]
%\iffalse   % this is a METACOMMENT !
%%
%% Package `theorem' to use with LaTeX2e
%% Copyright (C) 1989-1995 Frank Mittelbach, all rights reserved.
%<+package>\@ifundefined{theorem@style}{}{\endinput}
%<+package>\NeedsTeXFormat{LaTeX2e}[1995/06/01]
%<+package>\ProvidesPackage{theorem}
%<+package>         [\FMithmInfo]
%
% \fi
%
% \changes{v1.0}{87/05/07}{First draft.}
% \changes{v1.0a}{87/05/07}{`thistheoremstyle' added.}
% \changes{v1.1a}{87/07/29}{New concept: redefine the internal
%                          \protect\LaTeX{} commands `@xnthm' etc.}
% \changes{v1.1b}{87/07/31}{`theoremfont' used to specify font.}
% \changes{v1.2a}{88/01/08}{`@endtheorem' now globally defined.}
% \changes{v1.2b}{88/01/11}{`newtheorem' only allowed in the preamble.}
% \changes{v1.3a}{88/02/14}{`theoremfont' now a toks register.}
%
% \changes{v2.0a}{88/11/16}{New implementation and new documentation.}
% \changes{v2.0a}{88/11/16}{No layout changes but internal modifications
%                           old substyles won't work.}
% \changes{v2.0b}{89/01/30}{`global' used for all definitions.}
% \changes{v2.0c}{89/06/15}{Documentation translated to English.
%                            Thanks to LLL and Nina}
% \changes{v2.1a}{90/02/15}{Support for new font selection.}
% \changes{v2.2a}{94/02/03}{Cleanup for LaTeX2e}
%
% \DoNotIndex{\@endparenv,\@ifdefinable,\@ifnextchar,\@ifundefined}
% \DoNotIndex{\@namedef,\@nameuse,\@spaces,\@tempa,\@warning}
% \DoNotIndex{\ }
% \DoNotIndex{\begingroup,\bfseries}
% \DoNotIndex{\csname}
% \DoNotIndex{\def,\do,\docdate}
% \DoNotIndex{\endcsname,\endgroup,\endinput,\endtrivlist,\expandafter}
% \DoNotIndex{\filedate,\fileversion}
% \DoNotIndex{\gdef,\global}
% \DoNotIndex{\hbox,\hskip}
% \DoNotIndex{\ignorespaces,\input,\itshape,\item}
% \DoNotIndex{\labelsep,\let,\llap}
% \DoNotIndex{\makeatletter}
% \DoNotIndex{\noexpand}
% \DoNotIndex{\refstepcounter,\relax,\rlap}
% \DoNotIndex{\setlength,\slshape,\space,\string,\strut}
% \DoNotIndex{\the,\trivlist,\typeout}
% \DoNotIndex{\vbox}
% \DoNotIndex{\xdef}
%
% \def\deflabel#1{\ttfamily #1\hfill}
% \def\deflist#1{\begin{list}{}{\settowidth\labelwidth{\ttfamily #1}%
%                               \setlength\leftmargin\labelwidth
%                               \addtolength\leftmargin\labelsep
%                               \let\makelabel\deflabel}}
% \def\enddeflist{\end{list}}
% \def\envfont{\normalfont\ttfamily}
%
% \MakeShortVerb{\|}
%
% \newcommand{\TUB}{{\sl TUGboat\/}}
%
% \GetFileInfo{theorem.dtx}
% \title{An Extension of the
%        \LaTeX{} theorem environment\thanks{This file
%        has version number \fileversion, last
%        revised \filedate.}}
% \author{Frank Mittelbach}
%
% \MaintainedByLaTeXTeam{tools}
% \maketitle
%
% \begin{abstract}
%    The macros described in this paper yield an extension of the
%    \LaTeX{} theorem mechanism. It is designed to satisfy the
%    different requirements of various journals. Thus, the
%    layout of the ``theorems'' can be manipulated by determining a
%    ``style''. This article describes not only the use, but
%    also the definition, of the necessary macros.
% \end{abstract}
%
% \section*{Preface to version 2.2}
%
% For \LaTeXe{} this package did not need any fundamental changes. I
% only modified the messages generated so that theorem layout styles
% will show up with the |\listfiles| command and cleaned the section
% on the New Font Selection Scheme since this is now included in
% \LaTeX.
%
% \section*{Preface to version 2.1}
%
% This version is identical to 2.0g described in \TUB\ 10\#3 except
% for some internal defaults which are now set depending on the used
% font selection scheme.
%
% This was done to avoid unpleasant surprises if the new font selection
% scheme is in force. For further details see
% section~\ref{sec:newfonts} and \cite{bk:GMS94}.
%
% \section{Introduction}
%
% For our purposes here, ``theorems'' are labelled enunciations,
% often set off from the main text by extra space and a font change.
% Theorems, corollaries, conjectures, definitions, and remarks
% are all instances of ``theorems''.  The ``header'' of these structures
% is composed of a label (such as \textsc{Theorem} or \textsc{Remark})
% and a number which serializes an item in the sequence of items
% with the same label.
%
% Shortly after the introduction of \LaTeX{} at the Fachbereich
% Mathematik in Mainz, the desire to manipulate the layout of
% ``theorems'' arose.  In Mainz, the following two conventions came
% into general use:
% \begin{enumerate}
%   \item The number of the theorem is shown in the margin.
%   \item There is a line break at
%         the end of the theorem header.
% \end{enumerate}
%
% Additionally, some journals require different formats which depend on
% the ``sort of theorem'': e.g.\ often remarks and definitions are set
% in |\upshape|, while |\itshape| is employed for main theorems.
%
% Confronted with these requirements, a theorem environment was
% developed in Mainz which allows separate determination of the
% layout of the ``theorems sets'', comparable to |\pagestyle|.
%
%
% \section{The user interface}
% \subsection{Defining new theorem sets}
%
% \DescribeMacro\newtheorem
% As in the original \LaTeX{} version, the command |\newtheorem|
% defines a new ``theorem set'' or ``theorem-like structure''.
% Two required arguments name the new environment and give the text
% to be typeset with each instance of the new ``set'', while
% an optional argument determines how
% the ``set'' is enumerated:
% \begin{description}
%    \item[\ttfamily \bslash newtheorem\{foo\}\{bar\}]
%       The theorem set {\envfont foo} (whose name is \texttt{bar})
%       uses its own counter.
%    \item[\ttfamily \bslash newtheorem\{foo2\}{[foo]}\{bar2\}]
%       The theorem set {\envfont foo2} (printed name \texttt{bar2})
%       uses the same counter as the theorem set \texttt{foo}.
%    \item[\ttfamily \bslash newtheorem\{foo3\}\{bar3\}{[section]}]
%       The theorem set {\envfont foo3} (printed name \texttt{bar3}) is
%       enumerated within the counter \texttt{section}, i.e.\ with every
%       new |\section| the enumeration begins again with 1, and
%       the enumeration is composed from the section-number and the
%       theorem counter itself.
% \end{description}
%
% \DescribeMacro\theoremstyle
% Additionally, the command |\theoremstyle| can define the layout
% of various, or all, theorem sets.  It should be noted that any
% theorem set defined by |\newtheorem| is typeset in the
% |\theoremstyle| that is current at the time of the definition.
% Thus, the following
% \begin{verbatim}
%    \theoremstyle{break}        \newtheorem{Cor}{Corollary}
%    \theoremstyle{plain}        \newtheorem{Exa}{Example}[section]
%\end{verbatim}                             ^^A No blank at line start
% leads to the result that the set {\envfont Cor} is formatted in the
% style \texttt{break}, while the set {\envfont Exa} and all the
% following ones are formatted in the style \texttt{plain}, unless
% another |\theoremstyle| follows.  Since the definitions installed by
% |\newtheorem| are global, one also can limit |\theoremstyle| locally
% by grouping braces.
%
% \DescribeMacro\theorembodyfont
% The choice of the font for the theorem body is completely
% independent of the chosen |\theoremstyle|; this has proven to
% be very advantageous.  For example,
% \begin{verbatim}
%   {\theorembodyfont{\upshape}          \newtheorem{Rem}{Remark}}
%\end{verbatim}
% defines a theorem set {\envfont Rem}, which will be set in
% |\upshape| in the current layout (which in our example is
% \texttt{plain}).  As with |\theoremstyle|, the |\theorembodyfont|
% chosen is that current at the time of |\newtheorem|.  If
% |\theorembodyfont| is not specified or one defines
% |\theorembodyfont{}|, then the font used will be that defined by the
% |\theoremstyle|.
%
% \DescribeMacro\theoremheaderfont It is also possible to customize
% the font used for the theorem headers.  This is, however, a global
% declaration, and therefore there should be at most one
% |\theoremheaderfont| declaration in the preamble.\footnote{If it is
% actually necessary to have different header fonts, one has to define
% new theorem styles (substituting the desired font) or specify the
% information directly in the \texttt{\bslash newtheorem} declaration
% (the unclean variant).}
%
% \DescribeMacro\theorempreskipamount
% \DescribeMacro\theorempostskipamount Two additional parameters
% affect the vertical space around the theorem environments:
% |\theorempreskipamount| and |\theorempostskipamount| define,
% respectively, the spacing before and after such an environment.
% These parameters apply for all theorem sets and can be manipulated
% with the ordinary length macros.  They are rubber lengths
% (`\textsf{skips}'), and therefore can contain \texttt{plus} and
% \texttt{minus} parts.
%
% Since the definition of theorem sets should---most sensibly---be
% placed in the preamble, we only allow installation there.
% It is therefore possible to release the memory used here after
% |\begin{document}|, in order to make room for other
% applications.
%
%
% \subsection{Existing theorem styles}
%
% The following theorem styles exist to date:
% \begin{deflist}{marginbreak}
%    \item[plain]
%       This theorem style emulates the original \LaTeX{} definition,
%       except that additionally the parameters
%       |\theorem...skipamount| are used.
%    \item[break]
%       In this style, the theorem header is followed by a line
%       break.
%    \item[marginbreak]
%       The theorem number is set in the margin, and there is a line
%       break as in \texttt{break}.
%    \item[changebreak]
%       Like \texttt{break}, but with header number and text
%       interchanged.
%    \item[change]
%       Header number and text are interchanged, without a line break.
%    \item[margin]
%       The number is set in the left margin, without a line break.
% \end{deflist}
% All styles (except \texttt{plain}) select |\slshape| as the default
% |\theorembodyfont|.
%
%
% \subsection{Examples}
%
% Given the above theorem sets {\envfont Cor}, {\envfont Exa}
% and {\envfont Rem},
% suppose that the preamble also contains the declarations:
% \begin{verbatim}
%    \theoremstyle{marginbreak}   \newtheorem{Lem}[Cor]{Lemma}
%    \theoremstyle{change}
%    \theorembodyfont{\itshape}        \newtheorem{Def}[Cor]{Definition}
%
%    \theoremheaderfont{\scshape}
%\end{verbatim}
% Then the following are some typical examples of the typeset output
% resulting from their use.
%
% \begin{Cor}
%    This is a sentence typeset in the theorem environment {\envfont
%    Cor}.
% \end{Cor}
%
% \begin{Exa}
%    This is a sentence typeset in the theorem environment {\envfont
%    Exa}.
% \end{Exa}
%
% \begin{Rem}
%    This is a sentence typeset in the theorem environment {\envfont
%    Rem}.
% \end{Rem}
%
% \begin{Lem}[Ben User]
%    This is a sentence typeset in the theorem environment {\envfont
%    Lem}.
% \end{Lem}
%
% \begin{Def}[Very impressive Definition]
%    This is a sentence typeset in the theorem environment {\envfont
%    Def}.
% \end{Def}
%
% The last two examples show the effect of the optional argument to
% a theorem environment (it is the text typeset in parentheses).
%
%
% \section{Special Considerations}
% \label{sec:newfonts}
%
% Theoremheader and body are implemented as a unit. This means that
% the |\theoremheaderfont|  will inherit characteristics of the
% |\theorembodyfont| in \LaTeXe. Thus, if for example
% |\theorembodyfont| is |\itshape| and |\theoremheaderfont|
% is |\bfseries| the font selected for the header will have the
% characteristics `bold extended italic'. If this is not desired one
% should set the |\theoremheaderfont| to something like
% \begin{verbatim}
%      \theoremheaderfont{\normalfont\bfseries}
%\end{verbatim}
% i.e.\ supplying all necessary font information explicitly.
%
%
%
% \section{Acknowledgements}
%
% The publication of this set of macros was only possible with the
% help of Christina Busse (translating the manuscript into English),
% Joachim Pense (playing the r\^ole of typist), Chris Rowley
% (looking everything over) and many others providing useful
% suggestions.
%
% \MaybeStop{%
%    \begin{thebibliography}{9}
%       \bibitem{bk:GMS94} \textsc{M.~Goossens}, \textsc{F.~Mittelbach}
%         and \textsc{A.~Samarin}.
%         \newblock The \LaTeX{} Companion.
%         \newblock
%         Addison-Wesley, Reading, Massachusetts, 1994.
%       \bibitem{latex:source} \textsc{Lamport, Leslie.} \newblock
%                               \texttt{latex.tex}, version 2.09, date
%                               Feb. 1990.
%    \end{thebibliography}
%    \PrintIndex
%    \PrintChanges}
%
%
% \section{The documentation driver file}
%
% The next bit of code contains the documentation driver file for
% \TeX{}, i.e., the file that will produce the documentation you are
% currently reading. It will be extracted from this file by the
% \texttt{docstrip} program. Since it is the first code in the file
% one can alternatively process this file directly with \LaTeXe{} to
% obtain the documentation.
%
%    \begin{macrocode}
%<*driver>
\documentclass{ltxdoc}

\usepackage{theorem}

%<+driver>% The next few lines define theorem sets which are used
%<+driver>% in the example section of the documentation.

\theoremstyle{break}       \newtheorem{Cor}{Corollary}
\theoremstyle{plain}       \newtheorem{Exa}{Example}[section]
{\theorembodyfont{\upshape}\newtheorem{Rem}{Remark}}
\theoremstyle{marginbreak} \newtheorem{Lem}[Cor]{Lemma}
\theoremstyle{change}
\theorembodyfont{\itshape} \newtheorem{Def}[Cor]{Definition}

\theoremheaderfont{\scshape}

\RecordChanges

\begin{document}
   \DocInput{theorem.dtx}
\end{document}
%</driver>
%    \end{macrocode}
%
% \section{Definition of the Macros}
%
% If the file has been loaded before, we abort immediately. If not the
% package announces itself (this is actually done at the very top of
% the file---the way it is done isn't good style so don't copy it).
% \changes{v2.0e}{89/07/19}{Spaces removed from `typeout'.}
% \changes{v2.2b}{95/11/19}{Announce moved to top of file}
%    \begin{macrocode}
%<*package>
%\@ifundefined{theorem@style}{}{\endinput}
%\def\FMithmInfo{1995/11/19 v2.2b Theorem extension package (FMi)}
%\ProvidesPackage{theorem}[\FMithmInfo]
%    \end{macrocode}
%
%
% \subsection{Definition of theorem styles and fonts}
%
% All the definitions in this file are done globally to allow
% inputting this file inside a group.
%
% \begin{macro}{\theoremstyle}
%    Before a theorem style can be installed, the chosen style must be
%    known. For that reason, we must test to see that
%    |\th@|\meta{style} is known or, more precisely, that it is
%    different from |\relax|. If the style is not known then
%    |\th@plain| is used.
%    \begin{macrocode}
\gdef\theoremstyle#1{%
   \@ifundefined{th@#1}{\@warning
          {Unknown theoremstyle `#1'. Using `plain'}%
          \theorem@style{plain}}%
%    \end{macrocode}
%    We save the theorem style to be used in the token register
%    |\theorem@style|.
%    \begin{macrocode}
      {\theorem@style{#1}}%
%    \end{macrocode}
%    Now we ``evaluate'' the theorem style: this means, we call the
%    macro |\th@|\meta{style} which will activate the relevant
%    definitions which are contained in a separate file.  This is done
%    in a group to suppress changes to the current font.  This could
%    otherwise pose problems together with the new font selection
%    scheme\footnote{When I printed the original article using the new
%    font selection scheme I ended with a document with slanted
%    typefaces (text headings and all) simply because one of the
%    theorem styles used \texttt{\bslash sl} at toplevel.} if the
%    |\th@|\meta{style} is evaluated a second time.
% \changes{v2.1a}{90/02/15}{Group added around `th@..}
% \changes{v2.1a}{90/02/15}{Reseting of `..@begintheorem removed.}
%    \begin{macrocode}
      \begingroup
        \csname th@\the\theorem@style \endcsname
      \endgroup}
%    \end{macrocode}
% \end{macro}
%
%
% \begin{macro}{\@begintheorem}
% \changes{v2.1a}{90/02/15}{Relaxed at toplevel.}
% \begin{macro}{\@opargbegintheorem}
% \changes{v2.1a}{90/02/15}{Relaxed at toplevel.}
%    We reset |\@begintheorem| and |\@opargbegintheorem| to
%    |\relax| since these commands are no longer necessary at
%    toplevel. This will save a few tokens.
%    \begin{macrocode}
\global\let\@begintheorem\relax
\global\let\@opargbegintheorem\relax
%    \end{macrocode}
% \end{macro}
% \end{macro}
%
% \begin{macro}{\theorem@style}
%    Obviously the \textsf{token} register used above has to be
%    allocated. To assure the utmost compatibility with the
%    original \LaTeX{} definition, we set the default theorem style to
%    \texttt{plain}, which implements the usual \LaTeX{} convention.
% \changes{v2.0f}{89/07/20}{Unnecessary `global' removed.}
%    \begin{macrocode}
\newtoks\theorem@style
\global\theorem@style{plain}
%    \end{macrocode}
% \end{macro}
%
% \begin{macro}{\theorembodyfont}
% \begin{macro}{\theorem@bodyfont}
%    If the |\theorembodyfont| is set by the user then it should not
%    interact with the default font set in the theorem style. When the
%    new font selection is in force this may happen if, for example,
%    the default is |\itshape| and the new |\theorembodyfont| is
%    |\sffamily|. So we add a |\reset@font| command in front of the
%    user definition.
% \changes{v2.0f}{89/07/20}{Unnecessary `global' removed.}
% \changes{v2.0g}{89/09/19}{`theoremfont to `theorembodyfont.}
% \changes{v2.1c}{92/06/22}{`reset@font added and logic changed.}
% \changes{v2.1d}{92/10/28}{Reset to empty if arg is empty}
%    \begin{macrocode}
\gdef\theorembodyfont#1{%
%    \end{macrocode}
%    We check if the argument supplied is empty and if so put nothing
%    into the |\theorem@bodyfont| token register to allow for
%    |\theorembodyfont{}| as a mean of using the default of the
%    current |\theoremstyle|.
%    \begin{macrocode}
   \def\@tempa{#1}%
   \ifx\@tempa\@empty
    \theorem@bodyfont{}%
   \else
    \theorem@bodyfont{\reset@font#1}%
   \fi
   }
\newtoks\theorem@bodyfont
\global\theorem@bodyfont{}
%    \end{macrocode}
% \end{macro}
% \end{macro}
%
%
% \begin{macro}{\theoremheaderfont}
% \changes{v2.0g}{89/09/19}{Macro added.}
%    The font for the theorem headers is handled differently because
%    this definition applies to all theorem styles.
%    \begin{macrocode}
\gdef\theoremheaderfont#1{\gdef\theorem@headerfont{#1}%
%    \end{macrocode}
%    After using the macro once it is redefined to produce an error
%    message.
%    \begin{macrocode}
       \gdef\theoremheaderfont##1{%
        \typeout{\string\theoremheaderfont\space should be used
                 only once.}}}
%    \end{macrocode}
% \end{macro}
%
% \begin{macro}{\theorem@headerfont}
% \changes{v2.0g}{89/09/19}{Macro added.}
%    To set the |\theorem@headerfont| default we first test if
%    the new fontselection scheme is in force.
% \changes{v2.1a}{90/02/15}{Support for new font selection.}
%    \begin{macrocode}
\ifx\upshape\undefined
%    \end{macrocode}
%    If not we define it to expand into |\bfseries|. We don't use
%    |\let| just in case a following style option redefines this
%    macro.
%    \begin{macrocode}
\gdef\theorem@headerfont{\bfseries}
%    \end{macrocode}
%    Otherwise we reset the current shape before calling |\bfseries|.
%    \begin{macrocode}
\else \gdef\theorem@headerfont{\normalfont\bfseries}\fi
%    \end{macrocode}
% \end{macro}
%
% \begin{macro}{\th@plain}
% \begin{macro}{\th@break}
% \begin{macro}{\th@marginbreak}
% \begin{macro}{\th@changebreak}
% \begin{macro}{\th@change}
% \begin{macro}{\th@margin}
%    The different styles are defined in macros such as
%    |\th@plain|.  Since memory space is precious in
%    ``non-Big-versions'', we have to avoid offering too many unused
%    definitions.  Therefore we define these styles in separate files
%    that can be loaded on demand.  Thus the commands themselves only
%    load these files. We use |\@input@|, a \LaTeXe{} internal command
%    that ensures that the file will be listed with |\listfiles|.
% \changes{v2.2a}{94/02/02}{Use `@input@ to load theorem layout files}
%    \begin{macrocode}
\gdef\th@plain{\@input@{thp.sty}}
\gdef\th@break{\@input@{thb.sty}}
\gdef\th@marginbreak{\@input@{thmb.sty}}
\gdef\th@changebreak{\@input@{thcb.sty}}
\gdef\th@change{\@input@{thc.sty}}
\gdef\th@margin{\@input@{thm.sty}}
%    \end{macrocode}
%    This list will be expanded when new styles become
%    available.  For testing,  just append new theorem substyles
%    as document options.
% \end{macro}
% \end{macro}
% \end{macro}
% \end{macro}
% \end{macro}
% \end{macro}
%
%
%
% \subsection{Definition of a new theorem set}
%
% As already pointed out, a new theorem environment can be defined in
% three different ways:
% \begin{verbatim}
%    \newtheorem{Lem}{Lemma}
%    \newtheorem{Lem}{Lemma}[section]
%    \newtheorem{Lem}[Theorem]{Lemma}
%\end{verbatim}
% The function of the macro |\newtheorem| is to recognize these cases
% and then to branch into one of the three macros |\@ynthm|, |\@xnthm|
% or |\@othm|.  This mechanism is adopted unchanged from
% \cite{latex:source}; the essential point here is that, for example,
% in the second case, the arguments \texttt{Lem}, \texttt{Lemma} and
% \texttt{section} are passed over to the macro |\@xnthm|.
%
% We inspect this case first because the others present fewer problems,
% and thus are easily derived from this one.
%
% \begin{macro}{\@xnthm}
%    For our example arguments, the macro |\@xnthm| must fulfill
%    the following:
%    \begin{itemize}
%       \item
%          Define a new \LaTeX-counter `Lem'
%       \item
%          reset this counter within a |\section|
%       \item
%          define the macro |\theLem|
%       \item
%          define the environment macros |\Lem| and
%          |\endLem| using the current
%          |\theoremstyle| and |\theorem@bodyfont|.
%    \end{itemize}
%    Obviously, all this should happen only if the first argument
%    of |\@xnthm| (i.e.\ \texttt{Lem} in our example) is chosen so
%    as not to conflict with any previously defined commands or
%    environments.  This test is performed by the \LaTeX{} macro
%    |\@ifdefinable|.
%    \begin{macrocode}
\gdef\@xnthm#1#2[#3]{\expandafter\@ifdefinable\csname #1\endcsname
%    \end{macrocode}
%    Therefore, the first argument of |\@ifdefinable| is the
%    expansion (in the example, |\Lem|) of
%    |\csname||#1\endcsname|.  The second argument is
%    executed only if the test has been completed successfully.
%    \begin{macrocode}
   {%
%    \end{macrocode}
%    Now we define the new counter. The names of the \LaTeX{}
%    macros employed should speak for themselves:
% \changes{v2.2b}{1995/10/07}{Use \cs{@newctr}}
% \changes{v2.2c}{1995/11/23}{Use \cs{@newctr} properly.}
%    \begin{macrocode}
    \@definecounter{#1}\@newctr{#1}[#3]%
%    \end{macrocode}
%    Using |\@newctr| will give a proper error message if the counter
%    in |#3| is not defined.
%    In defining `|\theLem|' we must generate the desired macro
%    name by use of |\expandafter| and |\csname|.
%    \begin{macrocode}
    \expandafter\xdef\csname the#1\endcsname
%    \end{macrocode}
%    An |\xdef| is used in order to make the definition global,
%    and to ensure that it contains the replacement texts of
%    |\@thmcountersep| and |\@thmcounter|.\footnote{These
%    two macros can be defined by the document style.  Their default
%    values produce a `\texttt{.}' as separation and an arabic
%    representation of the number.} However, not everything should be
%    expanded. For example, it saves space to use |\thesection|
%    instead of its---at times---lengthy expansion.
%    \begin{macrocode}
      {\expandafter \noexpand \csname the#3\endcsname
       \@thmcountersep \@thmcounter{#1}}%
%    \end{macrocode}
%    Thus with the defaults of \LaTeX{}, |\theLem| would be
%    replaced by the command sequence |\thesection.\arabic{Lem}|.
%
%    We will now look at the definition of the macro which is executed
%    at the beginning of the actual environment (in our example this
%    macro is |\Lem|). It should be noted that we use an
%    ``|\expandafter| trick'' to expand only certain parts of the
%    replacement text at the time of the definition.
%    \begin{macrocode}
    \def\@tempa{\global\@namedef{#1}}%
    \expandafter \@tempa \expandafter{%
%    \end{macrocode}
%    First, the macro that contains the current definitions of
%    |\@begintheorem| and |\@opargtheorem| should be called
%    up.  The name of this macro---as is already known---has the form
%    |\th@|\meta{theorem\ style}; therefore, it must be called by
%    \begin{macrocode}
      \csname th@\the \theorem@style
%    \end{macrocode}
%    In addition the default theorem font should be changeable, i.e.\
%    we have to insert the contents of |\theorem@bodyfont|. For
%    that reason, we expand even further, beyond |\endcsname|,
%    and thus insert the contents of the \textsf{token} register
%    |\theorem@bodyfont| in the replacement text.
%    \begin{macrocode}
            \expandafter \endcsname \the \theorem@bodyfont
%    \end{macrocode}
%    Now it is time to call the macro |\@thm| which takes over
%    the further processing.  It has two arguments: the current
%    counter name (in our example, \texttt{Lem}), and the text of the
%    label (in our example, \texttt{Lemma}).
%    \begin{macrocode}
     \@thm{#1}{#2}}%
%    \end{macrocode}
%    With this, the `sub-definition' is complete.  The macro
%    |\@endtheorem| ends a theorem environment and is, so far, nothing
%    but an |\endtrivlist|.  (Hence it is defined globally, and not
%    within the theorem styles.\footnote{This has to be changed as
%    soon as theorem styles that change \texttt{\bslash @endtheorem}
%    exist.  In such a case, all existing styles must be changed as
%    well since they will have to reset the macro.}) Therefore, we can
%    set it equivalent to the macro that ends the theorem set (in our
%    example, |\endLem|).  However, if some day theorem styles exist
%    that do change |\@endtheorem|, we would have to use the
%    commented-out line instead.
%    \begin{macrocode}
    \global \expandafter \let \csname end#1\endcsname \@endtheorem
%   \global\@namedef{end#1}{\@endtheorem}%
%    \end{macrocode}
%    With these commands all the required definitions are employed,
%    unless the test |\@ifdefinable| has failed.  Therefore, we
%    end the second argument of this macro and with it the definition
%    of |\@xnthm|.
%    \begin{macrocode}
   }}
%    \end{macrocode}
% \end{macro}
%
% \begin{macro}{\@ynthm}
%    The definition of |\@ynthm| is completely analogous. In
%    this case the new counter that is defined is not reset within
%    another counter; thus the definition of |\the...|
%    is simplified:
%    \begin{macrocode}
\gdef\@ynthm#1#2{\expandafter\@ifdefinable\csname #1\endcsname
   {\@definecounter{#1}%
    \expandafter\xdef\csname the#1\endcsname{\@thmcounter{#1}}%
%    \end{macrocode}
%    The rest of the definition corresponds literally to that of
%    |\@xnthm|:
%    \begin{macrocode}
    \def\@tempa{\global\@namedef{#1}}\expandafter \@tempa
     \expandafter{\csname th@\the \theorem@style \expandafter
     \endcsname \the\theorem@bodyfont \@thm{#1}{#2}}%
    \global \expandafter \let \csname end#1\endcsname \@endtheorem}}
%    \end{macrocode}
% \end{macro}
%
% \begin{macro}{\@othm}
%    The definition of |\@othm| does not contain anything new.
%    \begin{macrocode}
\gdef\@othm#1[#2]#3{%
%    \end{macrocode}
% \changes{v2.2b}{95/10/07}{Build in check for counter defined}
%    We do not define a new counter but instead use one that has
%    already been defined.  Thus the only definition we need is that
%    of this pseudo-counter (i.e.\ |\the|\meta{env.\ name}).
%    First we check if |#2| corresponds to a known counter name.
%    \begin{macrocode}
  \expandafter\ifx\csname c@#2\endcsname\relax
   \@nocounterr{#2}%
  \else
   \expandafter\@ifdefinable\csname #1\endcsname
   {\expandafter \xdef \csname the#1\endcsname
     {\expandafter \noexpand \csname the#2\endcsname}%
%    \end{macrocode}
%    All other parts of the definition can be adopted from
%    |\@xnthm|. We have to remember, though, that in this case
%    the name of the current counter and the theorem label have moved
%    to the second and third arguments.
%    \begin{macrocode}
    \def\@tempa{\global\@namedef{#1}}\expandafter \@tempa
     \expandafter{\csname th@\the \theorem@style \expandafter
     \endcsname \the\theorem@bodyfont \@thm{#2}{#3}}%
    \global \expandafter \let \csname end#1\endcsname \@endtheorem}%
  \fi}
%    \end{macrocode}
% \end{macro}
%
% \subsection{Macros that are employed in a {\envfont theorem}
%    environment}
%
% \begin{macro}{\@thm}
%    The macro |\@thm| has to increase the current counter.  Then,
%    depending on whether the environment has (or does not have) an
%    optional argument, it has to branch into either |\@begintheorem|
%    or |\@opargtheorem|.
%    \begin{macrocode}
\gdef\@thm#1#2{\refstepcounter{#1}%
%    \end{macrocode}
%    Now we start a {\envfont trivlist} environment, and give
%    |\@topsep| and |\@topsepadd| the values of the \textsf{skip}
%    registers |\theorempreskipamount| and |\theorempostskipamount|.
%    The value in |\@topsep| is the vertical space that is inserted by
%    the first (and only) |\item| in our |\trivlist| whilst
%    |\@topsepadd| is inserted by |\@endparenv| at the end of that
%    {\envfont trivlist} environment.  By using these registers, we
%    obtain the desired space around a {\envfont theorem} environment.
%    \begin{macrocode}
   \trivlist
   \@topsep \theorempreskipamount               % used by first \item
   \@topsepadd \theorempostskipamount           % used by \@endparenv
%    \end{macrocode}
%    Now we have to test whether an optional argument has been given.
% \changes{v2.0d}{89/07/18}{Forgotten percent added.}
%    \begin{macrocode}
   \@ifnextchar [%
%    \end{macrocode}
%    If there is an optional argument, we will call |\@ythm|, and
%    move the arguments read back into the input stream.
%    \begin{macrocode}
   {\@ythm{#1}{#2}}%
%    \end{macrocode}
%    If not, we call |\@begintheorem|. Its first argument is the
%    name of the theorem set (hence the second argument of
%    |\@thm|). Its second argument is the macro that produces the
%    current number.
%    \begin{macrocode}
   {\@begintheorem{#2}{\csname the#1\endcsname}\ignorespaces}}
%    \end{macrocode}
% \end{macro}
%
% \begin{macro}{\@xthm}
% \begin{macro}{\@ythm}
%    Both these macros were originally called by |\@thm|.  We do
%    not need |\@xthm| anymore, hence we reset it to
%    |\relax|. The definition of |\@ythm| has not changed at
%    all from its definition in \LaTeX. In order to make the macros
%    easier to understand, we will
%    nevertheless present it (commented out).
%    \begin{macrocode}
\global\let\@xthm\relax
% \def\@ythm#1#2[#3]{\@opargbegintheorem{#2}{\csname
%           the#1\endcsname}{#3}\ignorespaces}
%</package>
%    \end{macrocode}
%    The primitive |\ignorespaces| in |\@ythm| and
%    |\@thm| is needed to remove the spaces between the
%    |\begin{...}| and the actual text.
% \end{macro}
% \end{macro}
%
%
% \subsection{Definition of the theorem substyles}
%
% As already pointed out, the theorem substyles, defined below, are
% only loaded when necessary.  Note that all these substyles, except
% \texttt{plain}, have |\slshape| as the default body font.
%
% \subsubsection{The \texttt{plain} style}
%
% \label{dok:th@plain} As the following macros use |@|, we have
% to locally set the |\catcode| of this symbol to ``letter''.
% This happens within a group, so that we do not have to worry about
% which |\catcode| that symbol had before.
%    \begin{macrocode}
%<*thp>
\begingroup \makeatletter
%    \end{macrocode}
% Since we are now within a group, we must make all definitions
% globally.  First we make sure that \texttt{theorem.sty} is loaded.
% This will allow us to use this file as a document style option
% without having to call \texttt{theorem} itself as an option.  At the
% same time, we assure that at least version 2 is loaded, since
% |\theorem@style| was not defined in earlier versions.
%    \begin{macrocode}
\@ifundefined{theorem@style}{% \iffalse meta-comment
%
% Copyright (C) 1993-2024
%
% The LaTeX Project and any individual authors listed elsewhere
% in this file.
%
% This file is part of the Standard LaTeX `Tools Bundle'.
% -------------------------------------------------------
%
% It may be distributed and/or modified under the
% conditions of the LaTeX Project Public License, either version 1.3c
% of this license or (at your option) any later version.
% The latest version of this license is in
%    https://www.latex-project.org/lppl.txt
% and version 1.3c or later is part of all distributions of LaTeX
% version 2005/12/01 or later.
%
% The list of all files belonging to the LaTeX `Tools Bundle' is
% given in the file `manifest.txt'.
%
% \fi
%
%%
%
%
\def\FMithmInfo{2023/07/05 v2.2c Theorem extension package (FMi)}
%
% \ProvidesFile{theorem.dtx}[\FMithmInfo]
%\iffalse   % this is a METACOMMENT !
%%
%% Package `theorem' to use with LaTeX2e
%% Copyright (C) 1989-1995 Frank Mittelbach, all rights reserved.
%<+package>\@ifundefined{theorem@style}{}{\endinput}
%<+package>\NeedsTeXFormat{LaTeX2e}[1995/06/01]
%<+package>\ProvidesPackage{theorem}
%<+package>         [\FMithmInfo]
%
% \fi
%
% \changes{v1.0}{87/05/07}{First draft.}
% \changes{v1.0a}{87/05/07}{`thistheoremstyle' added.}
% \changes{v1.1a}{87/07/29}{New concept: redefine the internal
%                          \protect\LaTeX{} commands `@xnthm' etc.}
% \changes{v1.1b}{87/07/31}{`theoremfont' used to specify font.}
% \changes{v1.2a}{88/01/08}{`@endtheorem' now globally defined.}
% \changes{v1.2b}{88/01/11}{`newtheorem' only allowed in the preamble.}
% \changes{v1.3a}{88/02/14}{`theoremfont' now a toks register.}
%
% \changes{v2.0a}{88/11/16}{New implementation and new documentation.}
% \changes{v2.0a}{88/11/16}{No layout changes but internal modifications
%                           old substyles won't work.}
% \changes{v2.0b}{89/01/30}{`global' used for all definitions.}
% \changes{v2.0c}{89/06/15}{Documentation translated to English.
%                            Thanks to LLL and Nina}
% \changes{v2.1a}{90/02/15}{Support for new font selection.}
% \changes{v2.2a}{94/02/03}{Cleanup for LaTeX2e}
%
% \DoNotIndex{\@endparenv,\@ifdefinable,\@ifnextchar,\@ifundefined}
% \DoNotIndex{\@namedef,\@nameuse,\@spaces,\@tempa,\@warning}
% \DoNotIndex{\ }
% \DoNotIndex{\begingroup,\bfseries}
% \DoNotIndex{\csname}
% \DoNotIndex{\def,\do,\docdate}
% \DoNotIndex{\endcsname,\endgroup,\endinput,\endtrivlist,\expandafter}
% \DoNotIndex{\filedate,\fileversion}
% \DoNotIndex{\gdef,\global}
% \DoNotIndex{\hbox,\hskip}
% \DoNotIndex{\ignorespaces,\input,\itshape,\item}
% \DoNotIndex{\labelsep,\let,\llap}
% \DoNotIndex{\makeatletter}
% \DoNotIndex{\noexpand}
% \DoNotIndex{\refstepcounter,\relax,\rlap}
% \DoNotIndex{\setlength,\slshape,\space,\string,\strut}
% \DoNotIndex{\the,\trivlist,\typeout}
% \DoNotIndex{\vbox}
% \DoNotIndex{\xdef}
%
% \def\deflabel#1{\ttfamily #1\hfill}
% \def\deflist#1{\begin{list}{}{\settowidth\labelwidth{\ttfamily #1}%
%                               \setlength\leftmargin\labelwidth
%                               \addtolength\leftmargin\labelsep
%                               \let\makelabel\deflabel}}
% \def\enddeflist{\end{list}}
% \def\envfont{\normalfont\ttfamily}
%
% \MakeShortVerb{\|}
%
% \newcommand{\TUB}{{\sl TUGboat\/}}
%
% \GetFileInfo{theorem.dtx}
% \title{An Extension of the
%        \LaTeX{} theorem environment\thanks{This file
%        has version number \fileversion, last
%        revised \filedate.}}
% \author{Frank Mittelbach}
%
% \MaintainedByLaTeXTeam{tools}
% \maketitle
%
% \begin{abstract}
%    The macros described in this paper yield an extension of the
%    \LaTeX{} theorem mechanism. It is designed to satisfy the
%    different requirements of various journals. Thus, the
%    layout of the ``theorems'' can be manipulated by determining a
%    ``style''. This article describes not only the use, but
%    also the definition, of the necessary macros.
% \end{abstract}
%
% \section*{Preface to version 2.2}
%
% For \LaTeXe{} this package did not need any fundamental changes. I
% only modified the messages generated so that theorem layout styles
% will show up with the |\listfiles| command and cleaned the section
% on the New Font Selection Scheme since this is now included in
% \LaTeX.
%
% \section*{Preface to version 2.1}
%
% This version is identical to 2.0g described in \TUB\ 10\#3 except
% for some internal defaults which are now set depending on the used
% font selection scheme.
%
% This was done to avoid unpleasant surprises if the new font selection
% scheme is in force. For further details see
% section~\ref{sec:newfonts} and \cite{bk:GMS94}.
%
% \section{Introduction}
%
% For our purposes here, ``theorems'' are labelled enunciations,
% often set off from the main text by extra space and a font change.
% Theorems, corollaries, conjectures, definitions, and remarks
% are all instances of ``theorems''.  The ``header'' of these structures
% is composed of a label (such as \textsc{Theorem} or \textsc{Remark})
% and a number which serializes an item in the sequence of items
% with the same label.
%
% Shortly after the introduction of \LaTeX{} at the Fachbereich
% Mathematik in Mainz, the desire to manipulate the layout of
% ``theorems'' arose.  In Mainz, the following two conventions came
% into general use:
% \begin{enumerate}
%   \item The number of the theorem is shown in the margin.
%   \item There is a line break at
%         the end of the theorem header.
% \end{enumerate}
%
% Additionally, some journals require different formats which depend on
% the ``sort of theorem'': e.g.\ often remarks and definitions are set
% in |\upshape|, while |\itshape| is employed for main theorems.
%
% Confronted with these requirements, a theorem environment was
% developed in Mainz which allows separate determination of the
% layout of the ``theorems sets'', comparable to |\pagestyle|.
%
%
% \section{The user interface}
% \subsection{Defining new theorem sets}
%
% \DescribeMacro\newtheorem
% As in the original \LaTeX{} version, the command |\newtheorem|
% defines a new ``theorem set'' or ``theorem-like structure''.
% Two required arguments name the new environment and give the text
% to be typeset with each instance of the new ``set'', while
% an optional argument determines how
% the ``set'' is enumerated:
% \begin{description}
%    \item[\ttfamily \bslash newtheorem\{foo\}\{bar\}]
%       The theorem set {\envfont foo} (whose name is \texttt{bar})
%       uses its own counter.
%    \item[\ttfamily \bslash newtheorem\{foo2\}{[foo]}\{bar2\}]
%       The theorem set {\envfont foo2} (printed name \texttt{bar2})
%       uses the same counter as the theorem set \texttt{foo}.
%    \item[\ttfamily \bslash newtheorem\{foo3\}\{bar3\}{[section]}]
%       The theorem set {\envfont foo3} (printed name \texttt{bar3}) is
%       enumerated within the counter \texttt{section}, i.e.\ with every
%       new |\section| the enumeration begins again with 1, and
%       the enumeration is composed from the section-number and the
%       theorem counter itself.
% \end{description}
%
% \DescribeMacro\theoremstyle
% Additionally, the command |\theoremstyle| can define the layout
% of various, or all, theorem sets.  It should be noted that any
% theorem set defined by |\newtheorem| is typeset in the
% |\theoremstyle| that is current at the time of the definition.
% Thus, the following
% \begin{verbatim}
%    \theoremstyle{break}        \newtheorem{Cor}{Corollary}
%    \theoremstyle{plain}        \newtheorem{Exa}{Example}[section]
%\end{verbatim}                             ^^A No blank at line start
% leads to the result that the set {\envfont Cor} is formatted in the
% style \texttt{break}, while the set {\envfont Exa} and all the
% following ones are formatted in the style \texttt{plain}, unless
% another |\theoremstyle| follows.  Since the definitions installed by
% |\newtheorem| are global, one also can limit |\theoremstyle| locally
% by grouping braces.
%
% \DescribeMacro\theorembodyfont
% The choice of the font for the theorem body is completely
% independent of the chosen |\theoremstyle|; this has proven to
% be very advantageous.  For example,
% \begin{verbatim}
%   {\theorembodyfont{\upshape}          \newtheorem{Rem}{Remark}}
%\end{verbatim}
% defines a theorem set {\envfont Rem}, which will be set in
% |\upshape| in the current layout (which in our example is
% \texttt{plain}).  As with |\theoremstyle|, the |\theorembodyfont|
% chosen is that current at the time of |\newtheorem|.  If
% |\theorembodyfont| is not specified or one defines
% |\theorembodyfont{}|, then the font used will be that defined by the
% |\theoremstyle|.
%
% \DescribeMacro\theoremheaderfont It is also possible to customize
% the font used for the theorem headers.  This is, however, a global
% declaration, and therefore there should be at most one
% |\theoremheaderfont| declaration in the preamble.\footnote{If it is
% actually necessary to have different header fonts, one has to define
% new theorem styles (substituting the desired font) or specify the
% information directly in the \texttt{\bslash newtheorem} declaration
% (the unclean variant).}
%
% \DescribeMacro\theorempreskipamount
% \DescribeMacro\theorempostskipamount Two additional parameters
% affect the vertical space around the theorem environments:
% |\theorempreskipamount| and |\theorempostskipamount| define,
% respectively, the spacing before and after such an environment.
% These parameters apply for all theorem sets and can be manipulated
% with the ordinary length macros.  They are rubber lengths
% (`\textsf{skips}'), and therefore can contain \texttt{plus} and
% \texttt{minus} parts.
%
% Since the definition of theorem sets should---most sensibly---be
% placed in the preamble, we only allow installation there.
% It is therefore possible to release the memory used here after
% |\begin{document}|, in order to make room for other
% applications.
%
%
% \subsection{Existing theorem styles}
%
% The following theorem styles exist to date:
% \begin{deflist}{marginbreak}
%    \item[plain]
%       This theorem style emulates the original \LaTeX{} definition,
%       except that additionally the parameters
%       |\theorem...skipamount| are used.
%    \item[break]
%       In this style, the theorem header is followed by a line
%       break.
%    \item[marginbreak]
%       The theorem number is set in the margin, and there is a line
%       break as in \texttt{break}.
%    \item[changebreak]
%       Like \texttt{break}, but with header number and text
%       interchanged.
%    \item[change]
%       Header number and text are interchanged, without a line break.
%    \item[margin]
%       The number is set in the left margin, without a line break.
% \end{deflist}
% All styles (except \texttt{plain}) select |\slshape| as the default
% |\theorembodyfont|.
%
%
% \subsection{Examples}
%
% Given the above theorem sets {\envfont Cor}, {\envfont Exa}
% and {\envfont Rem},
% suppose that the preamble also contains the declarations:
% \begin{verbatim}
%    \theoremstyle{marginbreak}   \newtheorem{Lem}[Cor]{Lemma}
%    \theoremstyle{change}
%    \theorembodyfont{\itshape}        \newtheorem{Def}[Cor]{Definition}
%
%    \theoremheaderfont{\scshape}
%\end{verbatim}
% Then the following are some typical examples of the typeset output
% resulting from their use.
%
% \begin{Cor}
%    This is a sentence typeset in the theorem environment {\envfont
%    Cor}.
% \end{Cor}
%
% \begin{Exa}
%    This is a sentence typeset in the theorem environment {\envfont
%    Exa}.
% \end{Exa}
%
% \begin{Rem}
%    This is a sentence typeset in the theorem environment {\envfont
%    Rem}.
% \end{Rem}
%
% \begin{Lem}[Ben User]
%    This is a sentence typeset in the theorem environment {\envfont
%    Lem}.
% \end{Lem}
%
% \begin{Def}[Very impressive Definition]
%    This is a sentence typeset in the theorem environment {\envfont
%    Def}.
% \end{Def}
%
% The last two examples show the effect of the optional argument to
% a theorem environment (it is the text typeset in parentheses).
%
%
% \section{Special Considerations}
% \label{sec:newfonts}
%
% Theoremheader and body are implemented as a unit. This means that
% the |\theoremheaderfont|  will inherit characteristics of the
% |\theorembodyfont| in \LaTeXe. Thus, if for example
% |\theorembodyfont| is |\itshape| and |\theoremheaderfont|
% is |\bfseries| the font selected for the header will have the
% characteristics `bold extended italic'. If this is not desired one
% should set the |\theoremheaderfont| to something like
% \begin{verbatim}
%      \theoremheaderfont{\normalfont\bfseries}
%\end{verbatim}
% i.e.\ supplying all necessary font information explicitly.
%
%
%
% \section{Acknowledgements}
%
% The publication of this set of macros was only possible with the
% help of Christina Busse (translating the manuscript into English),
% Joachim Pense (playing the r\^ole of typist), Chris Rowley
% (looking everything over) and many others providing useful
% suggestions.
%
% \MaybeStop{%
%    \begin{thebibliography}{9}
%       \bibitem{bk:GMS94} \textsc{M.~Goossens}, \textsc{F.~Mittelbach}
%         and \textsc{A.~Samarin}.
%         \newblock The \LaTeX{} Companion.
%         \newblock
%         Addison-Wesley, Reading, Massachusetts, 1994.
%       \bibitem{latex:source} \textsc{Lamport, Leslie.} \newblock
%                               \texttt{latex.tex}, version 2.09, date
%                               Feb. 1990.
%    \end{thebibliography}
%    \PrintIndex
%    \PrintChanges}
%
%
% \section{The documentation driver file}
%
% The next bit of code contains the documentation driver file for
% \TeX{}, i.e., the file that will produce the documentation you are
% currently reading. It will be extracted from this file by the
% \texttt{docstrip} program. Since it is the first code in the file
% one can alternatively process this file directly with \LaTeXe{} to
% obtain the documentation.
%
%    \begin{macrocode}
%<*driver>
\documentclass{ltxdoc}

\usepackage{theorem}

%<+driver>% The next few lines define theorem sets which are used
%<+driver>% in the example section of the documentation.

\theoremstyle{break}       \newtheorem{Cor}{Corollary}
\theoremstyle{plain}       \newtheorem{Exa}{Example}[section]
{\theorembodyfont{\upshape}\newtheorem{Rem}{Remark}}
\theoremstyle{marginbreak} \newtheorem{Lem}[Cor]{Lemma}
\theoremstyle{change}
\theorembodyfont{\itshape} \newtheorem{Def}[Cor]{Definition}

\theoremheaderfont{\scshape}

\RecordChanges

\begin{document}
   \DocInput{theorem.dtx}
\end{document}
%</driver>
%    \end{macrocode}
%
% \section{Definition of the Macros}
%
% If the file has been loaded before, we abort immediately. If not the
% package announces itself (this is actually done at the very top of
% the file---the way it is done isn't good style so don't copy it).
% \changes{v2.0e}{89/07/19}{Spaces removed from `typeout'.}
% \changes{v2.2b}{95/11/19}{Announce moved to top of file}
%    \begin{macrocode}
%<*package>
%\@ifundefined{theorem@style}{}{\endinput}
%\def\FMithmInfo{1995/11/19 v2.2b Theorem extension package (FMi)}
%\ProvidesPackage{theorem}[\FMithmInfo]
%    \end{macrocode}
%
%
% \subsection{Definition of theorem styles and fonts}
%
% All the definitions in this file are done globally to allow
% inputting this file inside a group.
%
% \begin{macro}{\theoremstyle}
%    Before a theorem style can be installed, the chosen style must be
%    known. For that reason, we must test to see that
%    |\th@|\meta{style} is known or, more precisely, that it is
%    different from |\relax|. If the style is not known then
%    |\th@plain| is used.
%    \begin{macrocode}
\gdef\theoremstyle#1{%
   \@ifundefined{th@#1}{\@warning
          {Unknown theoremstyle `#1'. Using `plain'}%
          \theorem@style{plain}}%
%    \end{macrocode}
%    We save the theorem style to be used in the token register
%    |\theorem@style|.
%    \begin{macrocode}
      {\theorem@style{#1}}%
%    \end{macrocode}
%    Now we ``evaluate'' the theorem style: this means, we call the
%    macro |\th@|\meta{style} which will activate the relevant
%    definitions which are contained in a separate file.  This is done
%    in a group to suppress changes to the current font.  This could
%    otherwise pose problems together with the new font selection
%    scheme\footnote{When I printed the original article using the new
%    font selection scheme I ended with a document with slanted
%    typefaces (text headings and all) simply because one of the
%    theorem styles used \texttt{\bslash sl} at toplevel.} if the
%    |\th@|\meta{style} is evaluated a second time.
% \changes{v2.1a}{90/02/15}{Group added around `th@..}
% \changes{v2.1a}{90/02/15}{Reseting of `..@begintheorem removed.}
%    \begin{macrocode}
      \begingroup
        \csname th@\the\theorem@style \endcsname
      \endgroup}
%    \end{macrocode}
% \end{macro}
%
%
% \begin{macro}{\@begintheorem}
% \changes{v2.1a}{90/02/15}{Relaxed at toplevel.}
% \begin{macro}{\@opargbegintheorem}
% \changes{v2.1a}{90/02/15}{Relaxed at toplevel.}
%    We reset |\@begintheorem| and |\@opargbegintheorem| to
%    |\relax| since these commands are no longer necessary at
%    toplevel. This will save a few tokens.
%    \begin{macrocode}
\global\let\@begintheorem\relax
\global\let\@opargbegintheorem\relax
%    \end{macrocode}
% \end{macro}
% \end{macro}
%
% \begin{macro}{\theorem@style}
%    Obviously the \textsf{token} register used above has to be
%    allocated. To assure the utmost compatibility with the
%    original \LaTeX{} definition, we set the default theorem style to
%    \texttt{plain}, which implements the usual \LaTeX{} convention.
% \changes{v2.0f}{89/07/20}{Unnecessary `global' removed.}
%    \begin{macrocode}
\newtoks\theorem@style
\global\theorem@style{plain}
%    \end{macrocode}
% \end{macro}
%
% \begin{macro}{\theorembodyfont}
% \begin{macro}{\theorem@bodyfont}
%    If the |\theorembodyfont| is set by the user then it should not
%    interact with the default font set in the theorem style. When the
%    new font selection is in force this may happen if, for example,
%    the default is |\itshape| and the new |\theorembodyfont| is
%    |\sffamily|. So we add a |\reset@font| command in front of the
%    user definition.
% \changes{v2.0f}{89/07/20}{Unnecessary `global' removed.}
% \changes{v2.0g}{89/09/19}{`theoremfont to `theorembodyfont.}
% \changes{v2.1c}{92/06/22}{`reset@font added and logic changed.}
% \changes{v2.1d}{92/10/28}{Reset to empty if arg is empty}
%    \begin{macrocode}
\gdef\theorembodyfont#1{%
%    \end{macrocode}
%    We check if the argument supplied is empty and if so put nothing
%    into the |\theorem@bodyfont| token register to allow for
%    |\theorembodyfont{}| as a mean of using the default of the
%    current |\theoremstyle|.
%    \begin{macrocode}
   \def\@tempa{#1}%
   \ifx\@tempa\@empty
    \theorem@bodyfont{}%
   \else
    \theorem@bodyfont{\reset@font#1}%
   \fi
   }
\newtoks\theorem@bodyfont
\global\theorem@bodyfont{}
%    \end{macrocode}
% \end{macro}
% \end{macro}
%
%
% \begin{macro}{\theoremheaderfont}
% \changes{v2.0g}{89/09/19}{Macro added.}
%    The font for the theorem headers is handled differently because
%    this definition applies to all theorem styles.
%    \begin{macrocode}
\gdef\theoremheaderfont#1{\gdef\theorem@headerfont{#1}%
%    \end{macrocode}
%    After using the macro once it is redefined to produce an error
%    message.
%    \begin{macrocode}
       \gdef\theoremheaderfont##1{%
        \typeout{\string\theoremheaderfont\space should be used
                 only once.}}}
%    \end{macrocode}
% \end{macro}
%
% \begin{macro}{\theorem@headerfont}
% \changes{v2.0g}{89/09/19}{Macro added.}
%    To set the |\theorem@headerfont| default we first test if
%    the new fontselection scheme is in force.
% \changes{v2.1a}{90/02/15}{Support for new font selection.}
%    \begin{macrocode}
\ifx\upshape\undefined
%    \end{macrocode}
%    If not we define it to expand into |\bfseries|. We don't use
%    |\let| just in case a following style option redefines this
%    macro.
%    \begin{macrocode}
\gdef\theorem@headerfont{\bfseries}
%    \end{macrocode}
%    Otherwise we reset the current shape before calling |\bfseries|.
%    \begin{macrocode}
\else \gdef\theorem@headerfont{\normalfont\bfseries}\fi
%    \end{macrocode}
% \end{macro}
%
% \begin{macro}{\th@plain}
% \begin{macro}{\th@break}
% \begin{macro}{\th@marginbreak}
% \begin{macro}{\th@changebreak}
% \begin{macro}{\th@change}
% \begin{macro}{\th@margin}
%    The different styles are defined in macros such as
%    |\th@plain|.  Since memory space is precious in
%    ``non-Big-versions'', we have to avoid offering too many unused
%    definitions.  Therefore we define these styles in separate files
%    that can be loaded on demand.  Thus the commands themselves only
%    load these files. We use |\@input@|, a \LaTeXe{} internal command
%    that ensures that the file will be listed with |\listfiles|.
% \changes{v2.2a}{94/02/02}{Use `@input@ to load theorem layout files}
%    \begin{macrocode}
\gdef\th@plain{\@input@{thp.sty}}
\gdef\th@break{\@input@{thb.sty}}
\gdef\th@marginbreak{\@input@{thmb.sty}}
\gdef\th@changebreak{\@input@{thcb.sty}}
\gdef\th@change{\@input@{thc.sty}}
\gdef\th@margin{\@input@{thm.sty}}
%    \end{macrocode}
%    This list will be expanded when new styles become
%    available.  For testing,  just append new theorem substyles
%    as document options.
% \end{macro}
% \end{macro}
% \end{macro}
% \end{macro}
% \end{macro}
% \end{macro}
%
%
%
% \subsection{Definition of a new theorem set}
%
% As already pointed out, a new theorem environment can be defined in
% three different ways:
% \begin{verbatim}
%    \newtheorem{Lem}{Lemma}
%    \newtheorem{Lem}{Lemma}[section]
%    \newtheorem{Lem}[Theorem]{Lemma}
%\end{verbatim}
% The function of the macro |\newtheorem| is to recognize these cases
% and then to branch into one of the three macros |\@ynthm|, |\@xnthm|
% or |\@othm|.  This mechanism is adopted unchanged from
% \cite{latex:source}; the essential point here is that, for example,
% in the second case, the arguments \texttt{Lem}, \texttt{Lemma} and
% \texttt{section} are passed over to the macro |\@xnthm|.
%
% We inspect this case first because the others present fewer problems,
% and thus are easily derived from this one.
%
% \begin{macro}{\@xnthm}
%    For our example arguments, the macro |\@xnthm| must fulfill
%    the following:
%    \begin{itemize}
%       \item
%          Define a new \LaTeX-counter `Lem'
%       \item
%          reset this counter within a |\section|
%       \item
%          define the macro |\theLem|
%       \item
%          define the environment macros |\Lem| and
%          |\endLem| using the current
%          |\theoremstyle| and |\theorem@bodyfont|.
%    \end{itemize}
%    Obviously, all this should happen only if the first argument
%    of |\@xnthm| (i.e.\ \texttt{Lem} in our example) is chosen so
%    as not to conflict with any previously defined commands or
%    environments.  This test is performed by the \LaTeX{} macro
%    |\@ifdefinable|.
%    \begin{macrocode}
\gdef\@xnthm#1#2[#3]{\expandafter\@ifdefinable\csname #1\endcsname
%    \end{macrocode}
%    Therefore, the first argument of |\@ifdefinable| is the
%    expansion (in the example, |\Lem|) of
%    |\csname||#1\endcsname|.  The second argument is
%    executed only if the test has been completed successfully.
%    \begin{macrocode}
   {%
%    \end{macrocode}
%    Now we define the new counter. The names of the \LaTeX{}
%    macros employed should speak for themselves:
% \changes{v2.2b}{1995/10/07}{Use \cs{@newctr}}
% \changes{v2.2c}{1995/11/23}{Use \cs{@newctr} properly.}
%    \begin{macrocode}
    \@definecounter{#1}\@newctr{#1}[#3]%
%    \end{macrocode}
%    Using |\@newctr| will give a proper error message if the counter
%    in |#3| is not defined.
%    In defining `|\theLem|' we must generate the desired macro
%    name by use of |\expandafter| and |\csname|.
%    \begin{macrocode}
    \expandafter\xdef\csname the#1\endcsname
%    \end{macrocode}
%    An |\xdef| is used in order to make the definition global,
%    and to ensure that it contains the replacement texts of
%    |\@thmcountersep| and |\@thmcounter|.\footnote{These
%    two macros can be defined by the document style.  Their default
%    values produce a `\texttt{.}' as separation and an arabic
%    representation of the number.} However, not everything should be
%    expanded. For example, it saves space to use |\thesection|
%    instead of its---at times---lengthy expansion.
%    \begin{macrocode}
      {\expandafter \noexpand \csname the#3\endcsname
       \@thmcountersep \@thmcounter{#1}}%
%    \end{macrocode}
%    Thus with the defaults of \LaTeX{}, |\theLem| would be
%    replaced by the command sequence |\thesection.\arabic{Lem}|.
%
%    We will now look at the definition of the macro which is executed
%    at the beginning of the actual environment (in our example this
%    macro is |\Lem|). It should be noted that we use an
%    ``|\expandafter| trick'' to expand only certain parts of the
%    replacement text at the time of the definition.
%    \begin{macrocode}
    \def\@tempa{\global\@namedef{#1}}%
    \expandafter \@tempa \expandafter{%
%    \end{macrocode}
%    First, the macro that contains the current definitions of
%    |\@begintheorem| and |\@opargtheorem| should be called
%    up.  The name of this macro---as is already known---has the form
%    |\th@|\meta{theorem\ style}; therefore, it must be called by
%    \begin{macrocode}
      \csname th@\the \theorem@style
%    \end{macrocode}
%    In addition the default theorem font should be changeable, i.e.\
%    we have to insert the contents of |\theorem@bodyfont|. For
%    that reason, we expand even further, beyond |\endcsname|,
%    and thus insert the contents of the \textsf{token} register
%    |\theorem@bodyfont| in the replacement text.
%    \begin{macrocode}
            \expandafter \endcsname \the \theorem@bodyfont
%    \end{macrocode}
%    Now it is time to call the macro |\@thm| which takes over
%    the further processing.  It has two arguments: the current
%    counter name (in our example, \texttt{Lem}), and the text of the
%    label (in our example, \texttt{Lemma}).
%    \begin{macrocode}
     \@thm{#1}{#2}}%
%    \end{macrocode}
%    With this, the `sub-definition' is complete.  The macro
%    |\@endtheorem| ends a theorem environment and is, so far, nothing
%    but an |\endtrivlist|.  (Hence it is defined globally, and not
%    within the theorem styles.\footnote{This has to be changed as
%    soon as theorem styles that change \texttt{\bslash @endtheorem}
%    exist.  In such a case, all existing styles must be changed as
%    well since they will have to reset the macro.}) Therefore, we can
%    set it equivalent to the macro that ends the theorem set (in our
%    example, |\endLem|).  However, if some day theorem styles exist
%    that do change |\@endtheorem|, we would have to use the
%    commented-out line instead.
%    \begin{macrocode}
    \global \expandafter \let \csname end#1\endcsname \@endtheorem
%   \global\@namedef{end#1}{\@endtheorem}%
%    \end{macrocode}
%    With these commands all the required definitions are employed,
%    unless the test |\@ifdefinable| has failed.  Therefore, we
%    end the second argument of this macro and with it the definition
%    of |\@xnthm|.
%    \begin{macrocode}
   }}
%    \end{macrocode}
% \end{macro}
%
% \begin{macro}{\@ynthm}
%    The definition of |\@ynthm| is completely analogous. In
%    this case the new counter that is defined is not reset within
%    another counter; thus the definition of |\the...|
%    is simplified:
%    \begin{macrocode}
\gdef\@ynthm#1#2{\expandafter\@ifdefinable\csname #1\endcsname
   {\@definecounter{#1}%
    \expandafter\xdef\csname the#1\endcsname{\@thmcounter{#1}}%
%    \end{macrocode}
%    The rest of the definition corresponds literally to that of
%    |\@xnthm|:
%    \begin{macrocode}
    \def\@tempa{\global\@namedef{#1}}\expandafter \@tempa
     \expandafter{\csname th@\the \theorem@style \expandafter
     \endcsname \the\theorem@bodyfont \@thm{#1}{#2}}%
    \global \expandafter \let \csname end#1\endcsname \@endtheorem}}
%    \end{macrocode}
% \end{macro}
%
% \begin{macro}{\@othm}
%    The definition of |\@othm| does not contain anything new.
%    \begin{macrocode}
\gdef\@othm#1[#2]#3{%
%    \end{macrocode}
% \changes{v2.2b}{95/10/07}{Build in check for counter defined}
%    We do not define a new counter but instead use one that has
%    already been defined.  Thus the only definition we need is that
%    of this pseudo-counter (i.e.\ |\the|\meta{env.\ name}).
%    First we check if |#2| corresponds to a known counter name.
%    \begin{macrocode}
  \expandafter\ifx\csname c@#2\endcsname\relax
   \@nocounterr{#2}%
  \else
   \expandafter\@ifdefinable\csname #1\endcsname
   {\expandafter \xdef \csname the#1\endcsname
     {\expandafter \noexpand \csname the#2\endcsname}%
%    \end{macrocode}
%    All other parts of the definition can be adopted from
%    |\@xnthm|. We have to remember, though, that in this case
%    the name of the current counter and the theorem label have moved
%    to the second and third arguments.
%    \begin{macrocode}
    \def\@tempa{\global\@namedef{#1}}\expandafter \@tempa
     \expandafter{\csname th@\the \theorem@style \expandafter
     \endcsname \the\theorem@bodyfont \@thm{#2}{#3}}%
    \global \expandafter \let \csname end#1\endcsname \@endtheorem}%
  \fi}
%    \end{macrocode}
% \end{macro}
%
% \subsection{Macros that are employed in a {\envfont theorem}
%    environment}
%
% \begin{macro}{\@thm}
%    The macro |\@thm| has to increase the current counter.  Then,
%    depending on whether the environment has (or does not have) an
%    optional argument, it has to branch into either |\@begintheorem|
%    or |\@opargtheorem|.
%    \begin{macrocode}
\gdef\@thm#1#2{\refstepcounter{#1}%
%    \end{macrocode}
%    Now we start a {\envfont trivlist} environment, and give
%    |\@topsep| and |\@topsepadd| the values of the \textsf{skip}
%    registers |\theorempreskipamount| and |\theorempostskipamount|.
%    The value in |\@topsep| is the vertical space that is inserted by
%    the first (and only) |\item| in our |\trivlist| whilst
%    |\@topsepadd| is inserted by |\@endparenv| at the end of that
%    {\envfont trivlist} environment.  By using these registers, we
%    obtain the desired space around a {\envfont theorem} environment.
%    \begin{macrocode}
   \trivlist
   \@topsep \theorempreskipamount               % used by first \item
   \@topsepadd \theorempostskipamount           % used by \@endparenv
%    \end{macrocode}
%    Now we have to test whether an optional argument has been given.
% \changes{v2.0d}{89/07/18}{Forgotten percent added.}
%    \begin{macrocode}
   \@ifnextchar [%
%    \end{macrocode}
%    If there is an optional argument, we will call |\@ythm|, and
%    move the arguments read back into the input stream.
%    \begin{macrocode}
   {\@ythm{#1}{#2}}%
%    \end{macrocode}
%    If not, we call |\@begintheorem|. Its first argument is the
%    name of the theorem set (hence the second argument of
%    |\@thm|). Its second argument is the macro that produces the
%    current number.
%    \begin{macrocode}
   {\@begintheorem{#2}{\csname the#1\endcsname}\ignorespaces}}
%    \end{macrocode}
% \end{macro}
%
% \begin{macro}{\@xthm}
% \begin{macro}{\@ythm}
%    Both these macros were originally called by |\@thm|.  We do
%    not need |\@xthm| anymore, hence we reset it to
%    |\relax|. The definition of |\@ythm| has not changed at
%    all from its definition in \LaTeX. In order to make the macros
%    easier to understand, we will
%    nevertheless present it (commented out).
%    \begin{macrocode}
\global\let\@xthm\relax
% \def\@ythm#1#2[#3]{\@opargbegintheorem{#2}{\csname
%           the#1\endcsname}{#3}\ignorespaces}
%</package>
%    \end{macrocode}
%    The primitive |\ignorespaces| in |\@ythm| and
%    |\@thm| is needed to remove the spaces between the
%    |\begin{...}| and the actual text.
% \end{macro}
% \end{macro}
%
%
% \subsection{Definition of the theorem substyles}
%
% As already pointed out, the theorem substyles, defined below, are
% only loaded when necessary.  Note that all these substyles, except
% \texttt{plain}, have |\slshape| as the default body font.
%
% \subsubsection{The \texttt{plain} style}
%
% \label{dok:th@plain} As the following macros use |@|, we have
% to locally set the |\catcode| of this symbol to ``letter''.
% This happens within a group, so that we do not have to worry about
% which |\catcode| that symbol had before.
%    \begin{macrocode}
%<*thp>
\begingroup \makeatletter
%    \end{macrocode}
% Since we are now within a group, we must make all definitions
% globally.  First we make sure that \texttt{theorem.sty} is loaded.
% This will allow us to use this file as a document style option
% without having to call \texttt{theorem} itself as an option.  At the
% same time, we assure that at least version 2 is loaded, since
% |\theorem@style| was not defined in earlier versions.
%    \begin{macrocode}
\@ifundefined{theorem@style}{\input{theorem.sty}}{}
\ProvidesFile{thp.sty}
             [\FMithmInfo]
%    \end{macrocode}
%
%
% \begin{macro}{\th@plain}
% \begin{macro}{\@begintheorem}
% \begin{macro}{\@opargbegintheorem}
%    |\theoremstyle{plain}| corresponds to the original
%    definition, except that the distances to the surrounding text are
%    determined by the parameters |\theorempreskipamount| and
%    |\theorempostskipamount|.
%    First we set the default body font.
%    \begin{macrocode}
\gdef\th@plain{\normalfont\itshape
%    \end{macrocode}
%    Then we define |\@begintheorem| and
%    |\@opargbegintheorem|. These two macros define how the
%    header of a theorem is typeset.  |\@opargbegintheorem| will
%    be called if a \textsf{theorem} environment with an optional
%    argument is encountered; otherwise, the header is constructed by
%    calling |\@begintheorem|.  If one of these macros is
%    executed, we are within a \textsf{trivlist} environment started by
%    |\@thm|.  So the theorem header is produced with an
%    |\item| command.
%
%    Instead of specifying the header font directly, all standard
%    theorem styles use the |\theorem@headerfont| macro to allow
%    customization.  The extra space (|\labelsep|) is necessary
%    because of problems in the \textsf{trivlist} environment.
%    \begin{macrocode}
  \def\@begintheorem##1##2{%
        \item[\hskip\labelsep \theorem@headerfont ##1\ ##2]}%
%    \end{macrocode}
%    The definition of |\@opargbegintheorem| is completely
%    analogous. The only difference is the fact that there exists a
%    third argument (which is the optional parameter of the
%    environment and contains additional information about the
%    theorem).  Customarily we enclose it in parentheses.
%    \begin{macrocode}
\def\@opargbegintheorem##1##2##3{%
   \item[\hskip\labelsep \theorem@headerfont ##1\ ##2\ (##3)]}}
%    \end{macrocode}
% \end{macro}
% \end{macro}
% \end{macro}
%
% \noindent
% We conclude with an |\endgroup| to restore the
% |\catcode| of |@|.
%    \begin{macrocode}
\endgroup
%</thp>
%    \end{macrocode}
%
%
% \subsubsection{The \texttt{break} style}
%
% This style option is stored in the file |thb.sty|.
% For the next two lines see the documentation for |\th@plain| on
% page~\pageref{dok:th@plain}.
%    \begin{macrocode}
%<*thb>
\begingroup \makeatletter
\@ifundefined{theorem@style}{\input{theorem.sty}}{}
\ProvidesFile{thb.sty}
             [\FMithmInfo]
%    \end{macrocode}
%
% \begin{macro}{\th@break}
%    |\theoremstyle{break}| produces a line break after the name
%    of the theorem.  The font is |\slshape|.  Hence, we define
%    |\th@break| as follows:
%    \begin{macrocode}
\gdef\th@break{\normalfont\slshape
  \def\@begintheorem##1##2{\item[%
%    \end{macrocode}
%    We run into the following problem: it is not possible to create
%    the header with |\item[|\meta{title}|]| and then start
%    a new line by, for example, |\mbox{}\\|. Such a definition
%    will fail whenever a list environment follows immediately.  With
%    the above construction, the |\mbox{}| causes the switch
%    |@inlabel| (cf.\ definition of |\list| and
%    |\trivlist| in \cite{latex:source}) to be set to \texttt{false}
%    and so the following list will insert additional vertical
%    space (|\topskip|). This is quite annoying. Therefore, we
%    create the line break within the |\item|. In order to ensure
%    that the text will begin at the proper position in the following
%    line, we simply pretend that the label does not take any
%    room.\footnote{This will lead to problems whenever very high
%    symbols occurring in the line tower into the
%    heading. So, something else has to be done here sometime.}
%    \begin{macrocode}
       \rlap{\vbox{\hbox{\hskip \labelsep\theorem@headerfont ##1\ ##2}%
                   \hbox{\strut}}}]}%
%    \end{macrocode}
%    Again, the definition of |\@opargbegintheorem| is
%    completely analogous.
%    \begin{macrocode}
\def\@opargbegintheorem##1##2##3{%
  \item[\rlap{\vbox{\hbox{\hskip \labelsep \theorem@headerfont
                     ##1\ ##2\ (##3)}%
                    \hbox{\strut}}}]}}
\endgroup
%</thb>
%    \end{macrocode}
% \end{macro}
%
%
% \subsubsection{The \texttt{changebreak} style}
%
%    \begin{macrocode}
% This style option is stored in the file |thcb.sty|.
%    \begin{macrocode}
%<*thcb>
\begingroup \makeatletter
\@ifundefined{theorem@style}{\input{theorem.sty}}{}
\ProvidesFile{thcb.sty}
             [\FMithmInfo]
%    \end{macrocode}
%
% \begin{macro}{\th@changebreak}
%    The \texttt{change-break} theorem style is like \texttt{break}
%    but with interchange of theorem name and theorem number.  Thus we
%    define |\th@changebreak| as follows:
%    \begin{macrocode}
\gdef\th@changebreak{\normalfont\slshape
  \def\@begintheorem##1##2{\item
       [\rlap{\vbox{\hbox{\hskip\labelsep \theorem@headerfont ##2\ ##1}%
                    \hbox{\strut}}}]}%
\def\@opargbegintheorem##1##2##3{%
  \item[\rlap{\vbox{\hbox{\hskip\labelsep \theorem@headerfont
                     ##2\ ##1\ (##3)}%
                    \hbox{\strut}}}]}}
\endgroup
%</thcb>
%    \end{macrocode}
% \end{macro}
%
%
% \subsubsection{The \texttt{change} style}
%
% This style option is stored in the file |thc.sty|.
%    \begin{macrocode}
%<*thc>
\begingroup \makeatletter
\@ifundefined{theorem@style}{\input{theorem.sty}}{}
\ProvidesFile{thc.sty}
             [\FMithmInfo]
%    \end{macrocode}
%
% \begin{macro}{\th@change}
%    The \texttt{change} theorem style corresponds to the
%    \texttt{change break} style without a linebreak after the header.
%    To say it in another way, it's the same as the \texttt{plain}
%    style but with number and name interchanged and |\slshape| as the
%    default font.
%    \begin{macrocode}
\gdef\th@change{\normalfont\slshape
  \def\@begintheorem##1##2{\item
       [\hskip\labelsep \theorem@headerfont ##2\ ##1]}%
\def\@opargbegintheorem##1##2##3{%
  \item[\hskip\labelsep \theorem@headerfont ##2\ ##1\ (##3)]}}
\endgroup
%</thc>
%    \end{macrocode}
% \end{macro}
%
%
% \subsubsection{The \texttt{marginbreak} style}
%
% This style option is the one used most often at Mainz.  It is saved
% in the file |thmb.sty|.
%    \begin{macrocode}
%<*thmb>
\begingroup \makeatletter
\@ifundefined{theorem@style}{\input{theorem.sty}}{}
\ProvidesFile{thmb.sty}
             [\FMithmInfo]
%    \end{macrocode}
%
% \begin{macro}{\th@marginbreak}
%    The \texttt{margin break} style is nearly the same as the
%    \texttt{change break} style. The only difference is the placement
%    of the theorem number.  We use |\llap| to place it in the left
%    margin.
%
%    In this style |\labelsep| denotes the separation between the
%    number and the text.
%    \begin{macrocode}
\gdef\th@marginbreak{\normalfont\slshape
  \def\@begintheorem##1##2{\item
       [\rlap{\vbox{\theorem@headerfont
                    \hbox{\llap{##2}\hskip\labelsep ##1}%
                    \hbox{\strut}}}]}%
\def\@opargbegintheorem##1##2##3{%
  \item[\rlap{\vbox{\theorem@headerfont
                    \hbox{\llap{##2}\hskip\labelsep ##1\ (##3)}%
                    \hbox{\strut}}}]}}
\endgroup
%</thmb>
%    \end{macrocode}
% \changes{v2.1c}{90/02/17}{`endgroup outside macrocode env.}
% \end{macro}
%
%
%
% \subsubsection{The \texttt{margin} style}
%
% This style option is stored in the file |thm.sty|.
%    \begin{macrocode}
%<*thm>
\begingroup \makeatletter
\@ifundefined{theorem@style}{\input{theorem.sty}}{}
\ProvidesFile{thm.sty}
             [\FMithmInfo]
%    \end{macrocode}
%
% \begin{macro}{\th@margin}
%    Again this is only a variant of the theorem styles described
%    above without any new ideas.
%    \begin{macrocode}
\gdef\th@margin{\normalfont\slshape
  \def\@begintheorem##1##2{\item
       [\theorem@headerfont \llap{##2}\hskip\labelsep ##1]}%
\def\@opargbegintheorem##1##2##3{%
  \item[\theorem@headerfont \llap{##2}\hskip\labelsep ##1\ (##3)]}}
\endgroup
%</thm>
%    \end{macrocode}
% \end{macro}
%
%
%
%
% \subsection{Final Definitions}
%
% \begin{macro}{\theorempreskipamount}
% \begin{macro}{\theorempostskipamount}
%    The \textsf{skip} parameters that regulate the vertical empty space
%    before and after the {\envfont theorem} environment have
%    to be allocated as well.
% \changes{v2.0f}{89/07/20}{Unnecessary `global' removed.}
%    \begin{macrocode}
%<*package>
\newskip\theorempreskipamount
\newskip\theorempostskipamount
%    \end{macrocode}
%    Since we have used the same values for all theorem sets, we now can
%    assign them.
%    \begin{macrocode}
\global\setlength\theorempreskipamount{12pt plus 5pt minus 3pt}
\global\setlength\theorempostskipamount{8pt plus 3pt minus 1.5pt}
%    \end{macrocode}
% \end{macro}
% \end{macro}
%
%
% \begin{macro}{\@endtheorem}
%    The same holds for the macro |\@endtheorem|, which ends a
%    {\envfont theorem} environment. Since it is the same for all
%    theorem sets, it is removed from the macros |\th@|\meta{style}.
%    It simply ends the {\envfont trivlist} environment, which was
%    begun in |\@thm|.
%    \begin{macrocode}
\global\let\@endtheorem=\endtrivlist
%    \end{macrocode}
% \end{macro}
%
%
%
% \begin{macro}{\@preamblecmds}
%    All macros defined above are to be used only in the preamble.
%    Therefore, we insert them in |\@preamblecmds| which will disable
%    them at begin document. This is done by the internal \LaTeXe{}
%    command |\@onlypreamble|.
% \changes{v2.0g}{89/09/19}{`theoremheaderfont and `theorembodyfont
%    added.}
% \changes{v2.2a}{94/02/03}{Use \LaTeXe{} interface}
%    \begin{macrocode}
\@onlypreamble\@xnthm
\@onlypreamble\@ynthm
\@onlypreamble\@othm
\@onlypreamble\newtheorem
\@onlypreamble\theoremstyle
\@onlypreamble\theorembodyfont
\@onlypreamble\theoremheaderfont
%    \end{macrocode}
% \end{macro}
%
% Finally we declare the \texttt{plain} theorem style to be the default.
% \changes{v2.1b}{90/06/04}{Fixed missing `item by declaring default
%    style.}
%    \begin{macrocode}
\theoremstyle{plain}
%</package>
%    \end{macrocode}
%
% \Finale

\endinput
}{}
\ProvidesFile{thp.sty}
             [\FMithmInfo]
%    \end{macrocode}
%
%
% \begin{macro}{\th@plain}
% \begin{macro}{\@begintheorem}
% \begin{macro}{\@opargbegintheorem}
%    |\theoremstyle{plain}| corresponds to the original
%    definition, except that the distances to the surrounding text are
%    determined by the parameters |\theorempreskipamount| and
%    |\theorempostskipamount|.
%    First we set the default body font.
%    \begin{macrocode}
\gdef\th@plain{\normalfont\itshape
%    \end{macrocode}
%    Then we define |\@begintheorem| and
%    |\@opargbegintheorem|. These two macros define how the
%    header of a theorem is typeset.  |\@opargbegintheorem| will
%    be called if a \textsf{theorem} environment with an optional
%    argument is encountered; otherwise, the header is constructed by
%    calling |\@begintheorem|.  If one of these macros is
%    executed, we are within a \textsf{trivlist} environment started by
%    |\@thm|.  So the theorem header is produced with an
%    |\item| command.
%
%    Instead of specifying the header font directly, all standard
%    theorem styles use the |\theorem@headerfont| macro to allow
%    customization.  The extra space (|\labelsep|) is necessary
%    because of problems in the \textsf{trivlist} environment.
%    \begin{macrocode}
  \def\@begintheorem##1##2{%
        \item[\hskip\labelsep \theorem@headerfont ##1\ ##2]}%
%    \end{macrocode}
%    The definition of |\@opargbegintheorem| is completely
%    analogous. The only difference is the fact that there exists a
%    third argument (which is the optional parameter of the
%    environment and contains additional information about the
%    theorem).  Customarily we enclose it in parentheses.
%    \begin{macrocode}
\def\@opargbegintheorem##1##2##3{%
   \item[\hskip\labelsep \theorem@headerfont ##1\ ##2\ (##3)]}}
%    \end{macrocode}
% \end{macro}
% \end{macro}
% \end{macro}
%
% \noindent
% We conclude with an |\endgroup| to restore the
% |\catcode| of |@|.
%    \begin{macrocode}
\endgroup
%</thp>
%    \end{macrocode}
%
%
% \subsubsection{The \texttt{break} style}
%
% This style option is stored in the file |thb.sty|.
% For the next two lines see the documentation for |\th@plain| on
% page~\pageref{dok:th@plain}.
%    \begin{macrocode}
%<*thb>
\begingroup \makeatletter
\@ifundefined{theorem@style}{% \iffalse meta-comment
%
% Copyright (C) 1993-2024
%
% The LaTeX Project and any individual authors listed elsewhere
% in this file.
%
% This file is part of the Standard LaTeX `Tools Bundle'.
% -------------------------------------------------------
%
% It may be distributed and/or modified under the
% conditions of the LaTeX Project Public License, either version 1.3c
% of this license or (at your option) any later version.
% The latest version of this license is in
%    https://www.latex-project.org/lppl.txt
% and version 1.3c or later is part of all distributions of LaTeX
% version 2005/12/01 or later.
%
% The list of all files belonging to the LaTeX `Tools Bundle' is
% given in the file `manifest.txt'.
%
% \fi
%
%%
%
%
\def\FMithmInfo{2023/07/05 v2.2c Theorem extension package (FMi)}
%
% \ProvidesFile{theorem.dtx}[\FMithmInfo]
%\iffalse   % this is a METACOMMENT !
%%
%% Package `theorem' to use with LaTeX2e
%% Copyright (C) 1989-1995 Frank Mittelbach, all rights reserved.
%<+package>\@ifundefined{theorem@style}{}{\endinput}
%<+package>\NeedsTeXFormat{LaTeX2e}[1995/06/01]
%<+package>\ProvidesPackage{theorem}
%<+package>         [\FMithmInfo]
%
% \fi
%
% \changes{v1.0}{87/05/07}{First draft.}
% \changes{v1.0a}{87/05/07}{`thistheoremstyle' added.}
% \changes{v1.1a}{87/07/29}{New concept: redefine the internal
%                          \protect\LaTeX{} commands `@xnthm' etc.}
% \changes{v1.1b}{87/07/31}{`theoremfont' used to specify font.}
% \changes{v1.2a}{88/01/08}{`@endtheorem' now globally defined.}
% \changes{v1.2b}{88/01/11}{`newtheorem' only allowed in the preamble.}
% \changes{v1.3a}{88/02/14}{`theoremfont' now a toks register.}
%
% \changes{v2.0a}{88/11/16}{New implementation and new documentation.}
% \changes{v2.0a}{88/11/16}{No layout changes but internal modifications
%                           old substyles won't work.}
% \changes{v2.0b}{89/01/30}{`global' used for all definitions.}
% \changes{v2.0c}{89/06/15}{Documentation translated to English.
%                            Thanks to LLL and Nina}
% \changes{v2.1a}{90/02/15}{Support for new font selection.}
% \changes{v2.2a}{94/02/03}{Cleanup for LaTeX2e}
%
% \DoNotIndex{\@endparenv,\@ifdefinable,\@ifnextchar,\@ifundefined}
% \DoNotIndex{\@namedef,\@nameuse,\@spaces,\@tempa,\@warning}
% \DoNotIndex{\ }
% \DoNotIndex{\begingroup,\bfseries}
% \DoNotIndex{\csname}
% \DoNotIndex{\def,\do,\docdate}
% \DoNotIndex{\endcsname,\endgroup,\endinput,\endtrivlist,\expandafter}
% \DoNotIndex{\filedate,\fileversion}
% \DoNotIndex{\gdef,\global}
% \DoNotIndex{\hbox,\hskip}
% \DoNotIndex{\ignorespaces,\input,\itshape,\item}
% \DoNotIndex{\labelsep,\let,\llap}
% \DoNotIndex{\makeatletter}
% \DoNotIndex{\noexpand}
% \DoNotIndex{\refstepcounter,\relax,\rlap}
% \DoNotIndex{\setlength,\slshape,\space,\string,\strut}
% \DoNotIndex{\the,\trivlist,\typeout}
% \DoNotIndex{\vbox}
% \DoNotIndex{\xdef}
%
% \def\deflabel#1{\ttfamily #1\hfill}
% \def\deflist#1{\begin{list}{}{\settowidth\labelwidth{\ttfamily #1}%
%                               \setlength\leftmargin\labelwidth
%                               \addtolength\leftmargin\labelsep
%                               \let\makelabel\deflabel}}
% \def\enddeflist{\end{list}}
% \def\envfont{\normalfont\ttfamily}
%
% \MakeShortVerb{\|}
%
% \newcommand{\TUB}{{\sl TUGboat\/}}
%
% \GetFileInfo{theorem.dtx}
% \title{An Extension of the
%        \LaTeX{} theorem environment\thanks{This file
%        has version number \fileversion, last
%        revised \filedate.}}
% \author{Frank Mittelbach}
%
% \MaintainedByLaTeXTeam{tools}
% \maketitle
%
% \begin{abstract}
%    The macros described in this paper yield an extension of the
%    \LaTeX{} theorem mechanism. It is designed to satisfy the
%    different requirements of various journals. Thus, the
%    layout of the ``theorems'' can be manipulated by determining a
%    ``style''. This article describes not only the use, but
%    also the definition, of the necessary macros.
% \end{abstract}
%
% \section*{Preface to version 2.2}
%
% For \LaTeXe{} this package did not need any fundamental changes. I
% only modified the messages generated so that theorem layout styles
% will show up with the |\listfiles| command and cleaned the section
% on the New Font Selection Scheme since this is now included in
% \LaTeX.
%
% \section*{Preface to version 2.1}
%
% This version is identical to 2.0g described in \TUB\ 10\#3 except
% for some internal defaults which are now set depending on the used
% font selection scheme.
%
% This was done to avoid unpleasant surprises if the new font selection
% scheme is in force. For further details see
% section~\ref{sec:newfonts} and \cite{bk:GMS94}.
%
% \section{Introduction}
%
% For our purposes here, ``theorems'' are labelled enunciations,
% often set off from the main text by extra space and a font change.
% Theorems, corollaries, conjectures, definitions, and remarks
% are all instances of ``theorems''.  The ``header'' of these structures
% is composed of a label (such as \textsc{Theorem} or \textsc{Remark})
% and a number which serializes an item in the sequence of items
% with the same label.
%
% Shortly after the introduction of \LaTeX{} at the Fachbereich
% Mathematik in Mainz, the desire to manipulate the layout of
% ``theorems'' arose.  In Mainz, the following two conventions came
% into general use:
% \begin{enumerate}
%   \item The number of the theorem is shown in the margin.
%   \item There is a line break at
%         the end of the theorem header.
% \end{enumerate}
%
% Additionally, some journals require different formats which depend on
% the ``sort of theorem'': e.g.\ often remarks and definitions are set
% in |\upshape|, while |\itshape| is employed for main theorems.
%
% Confronted with these requirements, a theorem environment was
% developed in Mainz which allows separate determination of the
% layout of the ``theorems sets'', comparable to |\pagestyle|.
%
%
% \section{The user interface}
% \subsection{Defining new theorem sets}
%
% \DescribeMacro\newtheorem
% As in the original \LaTeX{} version, the command |\newtheorem|
% defines a new ``theorem set'' or ``theorem-like structure''.
% Two required arguments name the new environment and give the text
% to be typeset with each instance of the new ``set'', while
% an optional argument determines how
% the ``set'' is enumerated:
% \begin{description}
%    \item[\ttfamily \bslash newtheorem\{foo\}\{bar\}]
%       The theorem set {\envfont foo} (whose name is \texttt{bar})
%       uses its own counter.
%    \item[\ttfamily \bslash newtheorem\{foo2\}{[foo]}\{bar2\}]
%       The theorem set {\envfont foo2} (printed name \texttt{bar2})
%       uses the same counter as the theorem set \texttt{foo}.
%    \item[\ttfamily \bslash newtheorem\{foo3\}\{bar3\}{[section]}]
%       The theorem set {\envfont foo3} (printed name \texttt{bar3}) is
%       enumerated within the counter \texttt{section}, i.e.\ with every
%       new |\section| the enumeration begins again with 1, and
%       the enumeration is composed from the section-number and the
%       theorem counter itself.
% \end{description}
%
% \DescribeMacro\theoremstyle
% Additionally, the command |\theoremstyle| can define the layout
% of various, or all, theorem sets.  It should be noted that any
% theorem set defined by |\newtheorem| is typeset in the
% |\theoremstyle| that is current at the time of the definition.
% Thus, the following
% \begin{verbatim}
%    \theoremstyle{break}        \newtheorem{Cor}{Corollary}
%    \theoremstyle{plain}        \newtheorem{Exa}{Example}[section]
%\end{verbatim}                             ^^A No blank at line start
% leads to the result that the set {\envfont Cor} is formatted in the
% style \texttt{break}, while the set {\envfont Exa} and all the
% following ones are formatted in the style \texttt{plain}, unless
% another |\theoremstyle| follows.  Since the definitions installed by
% |\newtheorem| are global, one also can limit |\theoremstyle| locally
% by grouping braces.
%
% \DescribeMacro\theorembodyfont
% The choice of the font for the theorem body is completely
% independent of the chosen |\theoremstyle|; this has proven to
% be very advantageous.  For example,
% \begin{verbatim}
%   {\theorembodyfont{\upshape}          \newtheorem{Rem}{Remark}}
%\end{verbatim}
% defines a theorem set {\envfont Rem}, which will be set in
% |\upshape| in the current layout (which in our example is
% \texttt{plain}).  As with |\theoremstyle|, the |\theorembodyfont|
% chosen is that current at the time of |\newtheorem|.  If
% |\theorembodyfont| is not specified or one defines
% |\theorembodyfont{}|, then the font used will be that defined by the
% |\theoremstyle|.
%
% \DescribeMacro\theoremheaderfont It is also possible to customize
% the font used for the theorem headers.  This is, however, a global
% declaration, and therefore there should be at most one
% |\theoremheaderfont| declaration in the preamble.\footnote{If it is
% actually necessary to have different header fonts, one has to define
% new theorem styles (substituting the desired font) or specify the
% information directly in the \texttt{\bslash newtheorem} declaration
% (the unclean variant).}
%
% \DescribeMacro\theorempreskipamount
% \DescribeMacro\theorempostskipamount Two additional parameters
% affect the vertical space around the theorem environments:
% |\theorempreskipamount| and |\theorempostskipamount| define,
% respectively, the spacing before and after such an environment.
% These parameters apply for all theorem sets and can be manipulated
% with the ordinary length macros.  They are rubber lengths
% (`\textsf{skips}'), and therefore can contain \texttt{plus} and
% \texttt{minus} parts.
%
% Since the definition of theorem sets should---most sensibly---be
% placed in the preamble, we only allow installation there.
% It is therefore possible to release the memory used here after
% |\begin{document}|, in order to make room for other
% applications.
%
%
% \subsection{Existing theorem styles}
%
% The following theorem styles exist to date:
% \begin{deflist}{marginbreak}
%    \item[plain]
%       This theorem style emulates the original \LaTeX{} definition,
%       except that additionally the parameters
%       |\theorem...skipamount| are used.
%    \item[break]
%       In this style, the theorem header is followed by a line
%       break.
%    \item[marginbreak]
%       The theorem number is set in the margin, and there is a line
%       break as in \texttt{break}.
%    \item[changebreak]
%       Like \texttt{break}, but with header number and text
%       interchanged.
%    \item[change]
%       Header number and text are interchanged, without a line break.
%    \item[margin]
%       The number is set in the left margin, without a line break.
% \end{deflist}
% All styles (except \texttt{plain}) select |\slshape| as the default
% |\theorembodyfont|.
%
%
% \subsection{Examples}
%
% Given the above theorem sets {\envfont Cor}, {\envfont Exa}
% and {\envfont Rem},
% suppose that the preamble also contains the declarations:
% \begin{verbatim}
%    \theoremstyle{marginbreak}   \newtheorem{Lem}[Cor]{Lemma}
%    \theoremstyle{change}
%    \theorembodyfont{\itshape}        \newtheorem{Def}[Cor]{Definition}
%
%    \theoremheaderfont{\scshape}
%\end{verbatim}
% Then the following are some typical examples of the typeset output
% resulting from their use.
%
% \begin{Cor}
%    This is a sentence typeset in the theorem environment {\envfont
%    Cor}.
% \end{Cor}
%
% \begin{Exa}
%    This is a sentence typeset in the theorem environment {\envfont
%    Exa}.
% \end{Exa}
%
% \begin{Rem}
%    This is a sentence typeset in the theorem environment {\envfont
%    Rem}.
% \end{Rem}
%
% \begin{Lem}[Ben User]
%    This is a sentence typeset in the theorem environment {\envfont
%    Lem}.
% \end{Lem}
%
% \begin{Def}[Very impressive Definition]
%    This is a sentence typeset in the theorem environment {\envfont
%    Def}.
% \end{Def}
%
% The last two examples show the effect of the optional argument to
% a theorem environment (it is the text typeset in parentheses).
%
%
% \section{Special Considerations}
% \label{sec:newfonts}
%
% Theoremheader and body are implemented as a unit. This means that
% the |\theoremheaderfont|  will inherit characteristics of the
% |\theorembodyfont| in \LaTeXe. Thus, if for example
% |\theorembodyfont| is |\itshape| and |\theoremheaderfont|
% is |\bfseries| the font selected for the header will have the
% characteristics `bold extended italic'. If this is not desired one
% should set the |\theoremheaderfont| to something like
% \begin{verbatim}
%      \theoremheaderfont{\normalfont\bfseries}
%\end{verbatim}
% i.e.\ supplying all necessary font information explicitly.
%
%
%
% \section{Acknowledgements}
%
% The publication of this set of macros was only possible with the
% help of Christina Busse (translating the manuscript into English),
% Joachim Pense (playing the r\^ole of typist), Chris Rowley
% (looking everything over) and many others providing useful
% suggestions.
%
% \MaybeStop{%
%    \begin{thebibliography}{9}
%       \bibitem{bk:GMS94} \textsc{M.~Goossens}, \textsc{F.~Mittelbach}
%         and \textsc{A.~Samarin}.
%         \newblock The \LaTeX{} Companion.
%         \newblock
%         Addison-Wesley, Reading, Massachusetts, 1994.
%       \bibitem{latex:source} \textsc{Lamport, Leslie.} \newblock
%                               \texttt{latex.tex}, version 2.09, date
%                               Feb. 1990.
%    \end{thebibliography}
%    \PrintIndex
%    \PrintChanges}
%
%
% \section{The documentation driver file}
%
% The next bit of code contains the documentation driver file for
% \TeX{}, i.e., the file that will produce the documentation you are
% currently reading. It will be extracted from this file by the
% \texttt{docstrip} program. Since it is the first code in the file
% one can alternatively process this file directly with \LaTeXe{} to
% obtain the documentation.
%
%    \begin{macrocode}
%<*driver>
\documentclass{ltxdoc}

\usepackage{theorem}

%<+driver>% The next few lines define theorem sets which are used
%<+driver>% in the example section of the documentation.

\theoremstyle{break}       \newtheorem{Cor}{Corollary}
\theoremstyle{plain}       \newtheorem{Exa}{Example}[section]
{\theorembodyfont{\upshape}\newtheorem{Rem}{Remark}}
\theoremstyle{marginbreak} \newtheorem{Lem}[Cor]{Lemma}
\theoremstyle{change}
\theorembodyfont{\itshape} \newtheorem{Def}[Cor]{Definition}

\theoremheaderfont{\scshape}

\RecordChanges

\begin{document}
   \DocInput{theorem.dtx}
\end{document}
%</driver>
%    \end{macrocode}
%
% \section{Definition of the Macros}
%
% If the file has been loaded before, we abort immediately. If not the
% package announces itself (this is actually done at the very top of
% the file---the way it is done isn't good style so don't copy it).
% \changes{v2.0e}{89/07/19}{Spaces removed from `typeout'.}
% \changes{v2.2b}{95/11/19}{Announce moved to top of file}
%    \begin{macrocode}
%<*package>
%\@ifundefined{theorem@style}{}{\endinput}
%\def\FMithmInfo{1995/11/19 v2.2b Theorem extension package (FMi)}
%\ProvidesPackage{theorem}[\FMithmInfo]
%    \end{macrocode}
%
%
% \subsection{Definition of theorem styles and fonts}
%
% All the definitions in this file are done globally to allow
% inputting this file inside a group.
%
% \begin{macro}{\theoremstyle}
%    Before a theorem style can be installed, the chosen style must be
%    known. For that reason, we must test to see that
%    |\th@|\meta{style} is known or, more precisely, that it is
%    different from |\relax|. If the style is not known then
%    |\th@plain| is used.
%    \begin{macrocode}
\gdef\theoremstyle#1{%
   \@ifundefined{th@#1}{\@warning
          {Unknown theoremstyle `#1'. Using `plain'}%
          \theorem@style{plain}}%
%    \end{macrocode}
%    We save the theorem style to be used in the token register
%    |\theorem@style|.
%    \begin{macrocode}
      {\theorem@style{#1}}%
%    \end{macrocode}
%    Now we ``evaluate'' the theorem style: this means, we call the
%    macro |\th@|\meta{style} which will activate the relevant
%    definitions which are contained in a separate file.  This is done
%    in a group to suppress changes to the current font.  This could
%    otherwise pose problems together with the new font selection
%    scheme\footnote{When I printed the original article using the new
%    font selection scheme I ended with a document with slanted
%    typefaces (text headings and all) simply because one of the
%    theorem styles used \texttt{\bslash sl} at toplevel.} if the
%    |\th@|\meta{style} is evaluated a second time.
% \changes{v2.1a}{90/02/15}{Group added around `th@..}
% \changes{v2.1a}{90/02/15}{Reseting of `..@begintheorem removed.}
%    \begin{macrocode}
      \begingroup
        \csname th@\the\theorem@style \endcsname
      \endgroup}
%    \end{macrocode}
% \end{macro}
%
%
% \begin{macro}{\@begintheorem}
% \changes{v2.1a}{90/02/15}{Relaxed at toplevel.}
% \begin{macro}{\@opargbegintheorem}
% \changes{v2.1a}{90/02/15}{Relaxed at toplevel.}
%    We reset |\@begintheorem| and |\@opargbegintheorem| to
%    |\relax| since these commands are no longer necessary at
%    toplevel. This will save a few tokens.
%    \begin{macrocode}
\global\let\@begintheorem\relax
\global\let\@opargbegintheorem\relax
%    \end{macrocode}
% \end{macro}
% \end{macro}
%
% \begin{macro}{\theorem@style}
%    Obviously the \textsf{token} register used above has to be
%    allocated. To assure the utmost compatibility with the
%    original \LaTeX{} definition, we set the default theorem style to
%    \texttt{plain}, which implements the usual \LaTeX{} convention.
% \changes{v2.0f}{89/07/20}{Unnecessary `global' removed.}
%    \begin{macrocode}
\newtoks\theorem@style
\global\theorem@style{plain}
%    \end{macrocode}
% \end{macro}
%
% \begin{macro}{\theorembodyfont}
% \begin{macro}{\theorem@bodyfont}
%    If the |\theorembodyfont| is set by the user then it should not
%    interact with the default font set in the theorem style. When the
%    new font selection is in force this may happen if, for example,
%    the default is |\itshape| and the new |\theorembodyfont| is
%    |\sffamily|. So we add a |\reset@font| command in front of the
%    user definition.
% \changes{v2.0f}{89/07/20}{Unnecessary `global' removed.}
% \changes{v2.0g}{89/09/19}{`theoremfont to `theorembodyfont.}
% \changes{v2.1c}{92/06/22}{`reset@font added and logic changed.}
% \changes{v2.1d}{92/10/28}{Reset to empty if arg is empty}
%    \begin{macrocode}
\gdef\theorembodyfont#1{%
%    \end{macrocode}
%    We check if the argument supplied is empty and if so put nothing
%    into the |\theorem@bodyfont| token register to allow for
%    |\theorembodyfont{}| as a mean of using the default of the
%    current |\theoremstyle|.
%    \begin{macrocode}
   \def\@tempa{#1}%
   \ifx\@tempa\@empty
    \theorem@bodyfont{}%
   \else
    \theorem@bodyfont{\reset@font#1}%
   \fi
   }
\newtoks\theorem@bodyfont
\global\theorem@bodyfont{}
%    \end{macrocode}
% \end{macro}
% \end{macro}
%
%
% \begin{macro}{\theoremheaderfont}
% \changes{v2.0g}{89/09/19}{Macro added.}
%    The font for the theorem headers is handled differently because
%    this definition applies to all theorem styles.
%    \begin{macrocode}
\gdef\theoremheaderfont#1{\gdef\theorem@headerfont{#1}%
%    \end{macrocode}
%    After using the macro once it is redefined to produce an error
%    message.
%    \begin{macrocode}
       \gdef\theoremheaderfont##1{%
        \typeout{\string\theoremheaderfont\space should be used
                 only once.}}}
%    \end{macrocode}
% \end{macro}
%
% \begin{macro}{\theorem@headerfont}
% \changes{v2.0g}{89/09/19}{Macro added.}
%    To set the |\theorem@headerfont| default we first test if
%    the new fontselection scheme is in force.
% \changes{v2.1a}{90/02/15}{Support for new font selection.}
%    \begin{macrocode}
\ifx\upshape\undefined
%    \end{macrocode}
%    If not we define it to expand into |\bfseries|. We don't use
%    |\let| just in case a following style option redefines this
%    macro.
%    \begin{macrocode}
\gdef\theorem@headerfont{\bfseries}
%    \end{macrocode}
%    Otherwise we reset the current shape before calling |\bfseries|.
%    \begin{macrocode}
\else \gdef\theorem@headerfont{\normalfont\bfseries}\fi
%    \end{macrocode}
% \end{macro}
%
% \begin{macro}{\th@plain}
% \begin{macro}{\th@break}
% \begin{macro}{\th@marginbreak}
% \begin{macro}{\th@changebreak}
% \begin{macro}{\th@change}
% \begin{macro}{\th@margin}
%    The different styles are defined in macros such as
%    |\th@plain|.  Since memory space is precious in
%    ``non-Big-versions'', we have to avoid offering too many unused
%    definitions.  Therefore we define these styles in separate files
%    that can be loaded on demand.  Thus the commands themselves only
%    load these files. We use |\@input@|, a \LaTeXe{} internal command
%    that ensures that the file will be listed with |\listfiles|.
% \changes{v2.2a}{94/02/02}{Use `@input@ to load theorem layout files}
%    \begin{macrocode}
\gdef\th@plain{\@input@{thp.sty}}
\gdef\th@break{\@input@{thb.sty}}
\gdef\th@marginbreak{\@input@{thmb.sty}}
\gdef\th@changebreak{\@input@{thcb.sty}}
\gdef\th@change{\@input@{thc.sty}}
\gdef\th@margin{\@input@{thm.sty}}
%    \end{macrocode}
%    This list will be expanded when new styles become
%    available.  For testing,  just append new theorem substyles
%    as document options.
% \end{macro}
% \end{macro}
% \end{macro}
% \end{macro}
% \end{macro}
% \end{macro}
%
%
%
% \subsection{Definition of a new theorem set}
%
% As already pointed out, a new theorem environment can be defined in
% three different ways:
% \begin{verbatim}
%    \newtheorem{Lem}{Lemma}
%    \newtheorem{Lem}{Lemma}[section]
%    \newtheorem{Lem}[Theorem]{Lemma}
%\end{verbatim}
% The function of the macro |\newtheorem| is to recognize these cases
% and then to branch into one of the three macros |\@ynthm|, |\@xnthm|
% or |\@othm|.  This mechanism is adopted unchanged from
% \cite{latex:source}; the essential point here is that, for example,
% in the second case, the arguments \texttt{Lem}, \texttt{Lemma} and
% \texttt{section} are passed over to the macro |\@xnthm|.
%
% We inspect this case first because the others present fewer problems,
% and thus are easily derived from this one.
%
% \begin{macro}{\@xnthm}
%    For our example arguments, the macro |\@xnthm| must fulfill
%    the following:
%    \begin{itemize}
%       \item
%          Define a new \LaTeX-counter `Lem'
%       \item
%          reset this counter within a |\section|
%       \item
%          define the macro |\theLem|
%       \item
%          define the environment macros |\Lem| and
%          |\endLem| using the current
%          |\theoremstyle| and |\theorem@bodyfont|.
%    \end{itemize}
%    Obviously, all this should happen only if the first argument
%    of |\@xnthm| (i.e.\ \texttt{Lem} in our example) is chosen so
%    as not to conflict with any previously defined commands or
%    environments.  This test is performed by the \LaTeX{} macro
%    |\@ifdefinable|.
%    \begin{macrocode}
\gdef\@xnthm#1#2[#3]{\expandafter\@ifdefinable\csname #1\endcsname
%    \end{macrocode}
%    Therefore, the first argument of |\@ifdefinable| is the
%    expansion (in the example, |\Lem|) of
%    |\csname||#1\endcsname|.  The second argument is
%    executed only if the test has been completed successfully.
%    \begin{macrocode}
   {%
%    \end{macrocode}
%    Now we define the new counter. The names of the \LaTeX{}
%    macros employed should speak for themselves:
% \changes{v2.2b}{1995/10/07}{Use \cs{@newctr}}
% \changes{v2.2c}{1995/11/23}{Use \cs{@newctr} properly.}
%    \begin{macrocode}
    \@definecounter{#1}\@newctr{#1}[#3]%
%    \end{macrocode}
%    Using |\@newctr| will give a proper error message if the counter
%    in |#3| is not defined.
%    In defining `|\theLem|' we must generate the desired macro
%    name by use of |\expandafter| and |\csname|.
%    \begin{macrocode}
    \expandafter\xdef\csname the#1\endcsname
%    \end{macrocode}
%    An |\xdef| is used in order to make the definition global,
%    and to ensure that it contains the replacement texts of
%    |\@thmcountersep| and |\@thmcounter|.\footnote{These
%    two macros can be defined by the document style.  Their default
%    values produce a `\texttt{.}' as separation and an arabic
%    representation of the number.} However, not everything should be
%    expanded. For example, it saves space to use |\thesection|
%    instead of its---at times---lengthy expansion.
%    \begin{macrocode}
      {\expandafter \noexpand \csname the#3\endcsname
       \@thmcountersep \@thmcounter{#1}}%
%    \end{macrocode}
%    Thus with the defaults of \LaTeX{}, |\theLem| would be
%    replaced by the command sequence |\thesection.\arabic{Lem}|.
%
%    We will now look at the definition of the macro which is executed
%    at the beginning of the actual environment (in our example this
%    macro is |\Lem|). It should be noted that we use an
%    ``|\expandafter| trick'' to expand only certain parts of the
%    replacement text at the time of the definition.
%    \begin{macrocode}
    \def\@tempa{\global\@namedef{#1}}%
    \expandafter \@tempa \expandafter{%
%    \end{macrocode}
%    First, the macro that contains the current definitions of
%    |\@begintheorem| and |\@opargtheorem| should be called
%    up.  The name of this macro---as is already known---has the form
%    |\th@|\meta{theorem\ style}; therefore, it must be called by
%    \begin{macrocode}
      \csname th@\the \theorem@style
%    \end{macrocode}
%    In addition the default theorem font should be changeable, i.e.\
%    we have to insert the contents of |\theorem@bodyfont|. For
%    that reason, we expand even further, beyond |\endcsname|,
%    and thus insert the contents of the \textsf{token} register
%    |\theorem@bodyfont| in the replacement text.
%    \begin{macrocode}
            \expandafter \endcsname \the \theorem@bodyfont
%    \end{macrocode}
%    Now it is time to call the macro |\@thm| which takes over
%    the further processing.  It has two arguments: the current
%    counter name (in our example, \texttt{Lem}), and the text of the
%    label (in our example, \texttt{Lemma}).
%    \begin{macrocode}
     \@thm{#1}{#2}}%
%    \end{macrocode}
%    With this, the `sub-definition' is complete.  The macro
%    |\@endtheorem| ends a theorem environment and is, so far, nothing
%    but an |\endtrivlist|.  (Hence it is defined globally, and not
%    within the theorem styles.\footnote{This has to be changed as
%    soon as theorem styles that change \texttt{\bslash @endtheorem}
%    exist.  In such a case, all existing styles must be changed as
%    well since they will have to reset the macro.}) Therefore, we can
%    set it equivalent to the macro that ends the theorem set (in our
%    example, |\endLem|).  However, if some day theorem styles exist
%    that do change |\@endtheorem|, we would have to use the
%    commented-out line instead.
%    \begin{macrocode}
    \global \expandafter \let \csname end#1\endcsname \@endtheorem
%   \global\@namedef{end#1}{\@endtheorem}%
%    \end{macrocode}
%    With these commands all the required definitions are employed,
%    unless the test |\@ifdefinable| has failed.  Therefore, we
%    end the second argument of this macro and with it the definition
%    of |\@xnthm|.
%    \begin{macrocode}
   }}
%    \end{macrocode}
% \end{macro}
%
% \begin{macro}{\@ynthm}
%    The definition of |\@ynthm| is completely analogous. In
%    this case the new counter that is defined is not reset within
%    another counter; thus the definition of |\the...|
%    is simplified:
%    \begin{macrocode}
\gdef\@ynthm#1#2{\expandafter\@ifdefinable\csname #1\endcsname
   {\@definecounter{#1}%
    \expandafter\xdef\csname the#1\endcsname{\@thmcounter{#1}}%
%    \end{macrocode}
%    The rest of the definition corresponds literally to that of
%    |\@xnthm|:
%    \begin{macrocode}
    \def\@tempa{\global\@namedef{#1}}\expandafter \@tempa
     \expandafter{\csname th@\the \theorem@style \expandafter
     \endcsname \the\theorem@bodyfont \@thm{#1}{#2}}%
    \global \expandafter \let \csname end#1\endcsname \@endtheorem}}
%    \end{macrocode}
% \end{macro}
%
% \begin{macro}{\@othm}
%    The definition of |\@othm| does not contain anything new.
%    \begin{macrocode}
\gdef\@othm#1[#2]#3{%
%    \end{macrocode}
% \changes{v2.2b}{95/10/07}{Build in check for counter defined}
%    We do not define a new counter but instead use one that has
%    already been defined.  Thus the only definition we need is that
%    of this pseudo-counter (i.e.\ |\the|\meta{env.\ name}).
%    First we check if |#2| corresponds to a known counter name.
%    \begin{macrocode}
  \expandafter\ifx\csname c@#2\endcsname\relax
   \@nocounterr{#2}%
  \else
   \expandafter\@ifdefinable\csname #1\endcsname
   {\expandafter \xdef \csname the#1\endcsname
     {\expandafter \noexpand \csname the#2\endcsname}%
%    \end{macrocode}
%    All other parts of the definition can be adopted from
%    |\@xnthm|. We have to remember, though, that in this case
%    the name of the current counter and the theorem label have moved
%    to the second and third arguments.
%    \begin{macrocode}
    \def\@tempa{\global\@namedef{#1}}\expandafter \@tempa
     \expandafter{\csname th@\the \theorem@style \expandafter
     \endcsname \the\theorem@bodyfont \@thm{#2}{#3}}%
    \global \expandafter \let \csname end#1\endcsname \@endtheorem}%
  \fi}
%    \end{macrocode}
% \end{macro}
%
% \subsection{Macros that are employed in a {\envfont theorem}
%    environment}
%
% \begin{macro}{\@thm}
%    The macro |\@thm| has to increase the current counter.  Then,
%    depending on whether the environment has (or does not have) an
%    optional argument, it has to branch into either |\@begintheorem|
%    or |\@opargtheorem|.
%    \begin{macrocode}
\gdef\@thm#1#2{\refstepcounter{#1}%
%    \end{macrocode}
%    Now we start a {\envfont trivlist} environment, and give
%    |\@topsep| and |\@topsepadd| the values of the \textsf{skip}
%    registers |\theorempreskipamount| and |\theorempostskipamount|.
%    The value in |\@topsep| is the vertical space that is inserted by
%    the first (and only) |\item| in our |\trivlist| whilst
%    |\@topsepadd| is inserted by |\@endparenv| at the end of that
%    {\envfont trivlist} environment.  By using these registers, we
%    obtain the desired space around a {\envfont theorem} environment.
%    \begin{macrocode}
   \trivlist
   \@topsep \theorempreskipamount               % used by first \item
   \@topsepadd \theorempostskipamount           % used by \@endparenv
%    \end{macrocode}
%    Now we have to test whether an optional argument has been given.
% \changes{v2.0d}{89/07/18}{Forgotten percent added.}
%    \begin{macrocode}
   \@ifnextchar [%
%    \end{macrocode}
%    If there is an optional argument, we will call |\@ythm|, and
%    move the arguments read back into the input stream.
%    \begin{macrocode}
   {\@ythm{#1}{#2}}%
%    \end{macrocode}
%    If not, we call |\@begintheorem|. Its first argument is the
%    name of the theorem set (hence the second argument of
%    |\@thm|). Its second argument is the macro that produces the
%    current number.
%    \begin{macrocode}
   {\@begintheorem{#2}{\csname the#1\endcsname}\ignorespaces}}
%    \end{macrocode}
% \end{macro}
%
% \begin{macro}{\@xthm}
% \begin{macro}{\@ythm}
%    Both these macros were originally called by |\@thm|.  We do
%    not need |\@xthm| anymore, hence we reset it to
%    |\relax|. The definition of |\@ythm| has not changed at
%    all from its definition in \LaTeX. In order to make the macros
%    easier to understand, we will
%    nevertheless present it (commented out).
%    \begin{macrocode}
\global\let\@xthm\relax
% \def\@ythm#1#2[#3]{\@opargbegintheorem{#2}{\csname
%           the#1\endcsname}{#3}\ignorespaces}
%</package>
%    \end{macrocode}
%    The primitive |\ignorespaces| in |\@ythm| and
%    |\@thm| is needed to remove the spaces between the
%    |\begin{...}| and the actual text.
% \end{macro}
% \end{macro}
%
%
% \subsection{Definition of the theorem substyles}
%
% As already pointed out, the theorem substyles, defined below, are
% only loaded when necessary.  Note that all these substyles, except
% \texttt{plain}, have |\slshape| as the default body font.
%
% \subsubsection{The \texttt{plain} style}
%
% \label{dok:th@plain} As the following macros use |@|, we have
% to locally set the |\catcode| of this symbol to ``letter''.
% This happens within a group, so that we do not have to worry about
% which |\catcode| that symbol had before.
%    \begin{macrocode}
%<*thp>
\begingroup \makeatletter
%    \end{macrocode}
% Since we are now within a group, we must make all definitions
% globally.  First we make sure that \texttt{theorem.sty} is loaded.
% This will allow us to use this file as a document style option
% without having to call \texttt{theorem} itself as an option.  At the
% same time, we assure that at least version 2 is loaded, since
% |\theorem@style| was not defined in earlier versions.
%    \begin{macrocode}
\@ifundefined{theorem@style}{\input{theorem.sty}}{}
\ProvidesFile{thp.sty}
             [\FMithmInfo]
%    \end{macrocode}
%
%
% \begin{macro}{\th@plain}
% \begin{macro}{\@begintheorem}
% \begin{macro}{\@opargbegintheorem}
%    |\theoremstyle{plain}| corresponds to the original
%    definition, except that the distances to the surrounding text are
%    determined by the parameters |\theorempreskipamount| and
%    |\theorempostskipamount|.
%    First we set the default body font.
%    \begin{macrocode}
\gdef\th@plain{\normalfont\itshape
%    \end{macrocode}
%    Then we define |\@begintheorem| and
%    |\@opargbegintheorem|. These two macros define how the
%    header of a theorem is typeset.  |\@opargbegintheorem| will
%    be called if a \textsf{theorem} environment with an optional
%    argument is encountered; otherwise, the header is constructed by
%    calling |\@begintheorem|.  If one of these macros is
%    executed, we are within a \textsf{trivlist} environment started by
%    |\@thm|.  So the theorem header is produced with an
%    |\item| command.
%
%    Instead of specifying the header font directly, all standard
%    theorem styles use the |\theorem@headerfont| macro to allow
%    customization.  The extra space (|\labelsep|) is necessary
%    because of problems in the \textsf{trivlist} environment.
%    \begin{macrocode}
  \def\@begintheorem##1##2{%
        \item[\hskip\labelsep \theorem@headerfont ##1\ ##2]}%
%    \end{macrocode}
%    The definition of |\@opargbegintheorem| is completely
%    analogous. The only difference is the fact that there exists a
%    third argument (which is the optional parameter of the
%    environment and contains additional information about the
%    theorem).  Customarily we enclose it in parentheses.
%    \begin{macrocode}
\def\@opargbegintheorem##1##2##3{%
   \item[\hskip\labelsep \theorem@headerfont ##1\ ##2\ (##3)]}}
%    \end{macrocode}
% \end{macro}
% \end{macro}
% \end{macro}
%
% \noindent
% We conclude with an |\endgroup| to restore the
% |\catcode| of |@|.
%    \begin{macrocode}
\endgroup
%</thp>
%    \end{macrocode}
%
%
% \subsubsection{The \texttt{break} style}
%
% This style option is stored in the file |thb.sty|.
% For the next two lines see the documentation for |\th@plain| on
% page~\pageref{dok:th@plain}.
%    \begin{macrocode}
%<*thb>
\begingroup \makeatletter
\@ifundefined{theorem@style}{\input{theorem.sty}}{}
\ProvidesFile{thb.sty}
             [\FMithmInfo]
%    \end{macrocode}
%
% \begin{macro}{\th@break}
%    |\theoremstyle{break}| produces a line break after the name
%    of the theorem.  The font is |\slshape|.  Hence, we define
%    |\th@break| as follows:
%    \begin{macrocode}
\gdef\th@break{\normalfont\slshape
  \def\@begintheorem##1##2{\item[%
%    \end{macrocode}
%    We run into the following problem: it is not possible to create
%    the header with |\item[|\meta{title}|]| and then start
%    a new line by, for example, |\mbox{}\\|. Such a definition
%    will fail whenever a list environment follows immediately.  With
%    the above construction, the |\mbox{}| causes the switch
%    |@inlabel| (cf.\ definition of |\list| and
%    |\trivlist| in \cite{latex:source}) to be set to \texttt{false}
%    and so the following list will insert additional vertical
%    space (|\topskip|). This is quite annoying. Therefore, we
%    create the line break within the |\item|. In order to ensure
%    that the text will begin at the proper position in the following
%    line, we simply pretend that the label does not take any
%    room.\footnote{This will lead to problems whenever very high
%    symbols occurring in the line tower into the
%    heading. So, something else has to be done here sometime.}
%    \begin{macrocode}
       \rlap{\vbox{\hbox{\hskip \labelsep\theorem@headerfont ##1\ ##2}%
                   \hbox{\strut}}}]}%
%    \end{macrocode}
%    Again, the definition of |\@opargbegintheorem| is
%    completely analogous.
%    \begin{macrocode}
\def\@opargbegintheorem##1##2##3{%
  \item[\rlap{\vbox{\hbox{\hskip \labelsep \theorem@headerfont
                     ##1\ ##2\ (##3)}%
                    \hbox{\strut}}}]}}
\endgroup
%</thb>
%    \end{macrocode}
% \end{macro}
%
%
% \subsubsection{The \texttt{changebreak} style}
%
%    \begin{macrocode}
% This style option is stored in the file |thcb.sty|.
%    \begin{macrocode}
%<*thcb>
\begingroup \makeatletter
\@ifundefined{theorem@style}{\input{theorem.sty}}{}
\ProvidesFile{thcb.sty}
             [\FMithmInfo]
%    \end{macrocode}
%
% \begin{macro}{\th@changebreak}
%    The \texttt{change-break} theorem style is like \texttt{break}
%    but with interchange of theorem name and theorem number.  Thus we
%    define |\th@changebreak| as follows:
%    \begin{macrocode}
\gdef\th@changebreak{\normalfont\slshape
  \def\@begintheorem##1##2{\item
       [\rlap{\vbox{\hbox{\hskip\labelsep \theorem@headerfont ##2\ ##1}%
                    \hbox{\strut}}}]}%
\def\@opargbegintheorem##1##2##3{%
  \item[\rlap{\vbox{\hbox{\hskip\labelsep \theorem@headerfont
                     ##2\ ##1\ (##3)}%
                    \hbox{\strut}}}]}}
\endgroup
%</thcb>
%    \end{macrocode}
% \end{macro}
%
%
% \subsubsection{The \texttt{change} style}
%
% This style option is stored in the file |thc.sty|.
%    \begin{macrocode}
%<*thc>
\begingroup \makeatletter
\@ifundefined{theorem@style}{\input{theorem.sty}}{}
\ProvidesFile{thc.sty}
             [\FMithmInfo]
%    \end{macrocode}
%
% \begin{macro}{\th@change}
%    The \texttt{change} theorem style corresponds to the
%    \texttt{change break} style without a linebreak after the header.
%    To say it in another way, it's the same as the \texttt{plain}
%    style but with number and name interchanged and |\slshape| as the
%    default font.
%    \begin{macrocode}
\gdef\th@change{\normalfont\slshape
  \def\@begintheorem##1##2{\item
       [\hskip\labelsep \theorem@headerfont ##2\ ##1]}%
\def\@opargbegintheorem##1##2##3{%
  \item[\hskip\labelsep \theorem@headerfont ##2\ ##1\ (##3)]}}
\endgroup
%</thc>
%    \end{macrocode}
% \end{macro}
%
%
% \subsubsection{The \texttt{marginbreak} style}
%
% This style option is the one used most often at Mainz.  It is saved
% in the file |thmb.sty|.
%    \begin{macrocode}
%<*thmb>
\begingroup \makeatletter
\@ifundefined{theorem@style}{\input{theorem.sty}}{}
\ProvidesFile{thmb.sty}
             [\FMithmInfo]
%    \end{macrocode}
%
% \begin{macro}{\th@marginbreak}
%    The \texttt{margin break} style is nearly the same as the
%    \texttt{change break} style. The only difference is the placement
%    of the theorem number.  We use |\llap| to place it in the left
%    margin.
%
%    In this style |\labelsep| denotes the separation between the
%    number and the text.
%    \begin{macrocode}
\gdef\th@marginbreak{\normalfont\slshape
  \def\@begintheorem##1##2{\item
       [\rlap{\vbox{\theorem@headerfont
                    \hbox{\llap{##2}\hskip\labelsep ##1}%
                    \hbox{\strut}}}]}%
\def\@opargbegintheorem##1##2##3{%
  \item[\rlap{\vbox{\theorem@headerfont
                    \hbox{\llap{##2}\hskip\labelsep ##1\ (##3)}%
                    \hbox{\strut}}}]}}
\endgroup
%</thmb>
%    \end{macrocode}
% \changes{v2.1c}{90/02/17}{`endgroup outside macrocode env.}
% \end{macro}
%
%
%
% \subsubsection{The \texttt{margin} style}
%
% This style option is stored in the file |thm.sty|.
%    \begin{macrocode}
%<*thm>
\begingroup \makeatletter
\@ifundefined{theorem@style}{\input{theorem.sty}}{}
\ProvidesFile{thm.sty}
             [\FMithmInfo]
%    \end{macrocode}
%
% \begin{macro}{\th@margin}
%    Again this is only a variant of the theorem styles described
%    above without any new ideas.
%    \begin{macrocode}
\gdef\th@margin{\normalfont\slshape
  \def\@begintheorem##1##2{\item
       [\theorem@headerfont \llap{##2}\hskip\labelsep ##1]}%
\def\@opargbegintheorem##1##2##3{%
  \item[\theorem@headerfont \llap{##2}\hskip\labelsep ##1\ (##3)]}}
\endgroup
%</thm>
%    \end{macrocode}
% \end{macro}
%
%
%
%
% \subsection{Final Definitions}
%
% \begin{macro}{\theorempreskipamount}
% \begin{macro}{\theorempostskipamount}
%    The \textsf{skip} parameters that regulate the vertical empty space
%    before and after the {\envfont theorem} environment have
%    to be allocated as well.
% \changes{v2.0f}{89/07/20}{Unnecessary `global' removed.}
%    \begin{macrocode}
%<*package>
\newskip\theorempreskipamount
\newskip\theorempostskipamount
%    \end{macrocode}
%    Since we have used the same values for all theorem sets, we now can
%    assign them.
%    \begin{macrocode}
\global\setlength\theorempreskipamount{12pt plus 5pt minus 3pt}
\global\setlength\theorempostskipamount{8pt plus 3pt minus 1.5pt}
%    \end{macrocode}
% \end{macro}
% \end{macro}
%
%
% \begin{macro}{\@endtheorem}
%    The same holds for the macro |\@endtheorem|, which ends a
%    {\envfont theorem} environment. Since it is the same for all
%    theorem sets, it is removed from the macros |\th@|\meta{style}.
%    It simply ends the {\envfont trivlist} environment, which was
%    begun in |\@thm|.
%    \begin{macrocode}
\global\let\@endtheorem=\endtrivlist
%    \end{macrocode}
% \end{macro}
%
%
%
% \begin{macro}{\@preamblecmds}
%    All macros defined above are to be used only in the preamble.
%    Therefore, we insert them in |\@preamblecmds| which will disable
%    them at begin document. This is done by the internal \LaTeXe{}
%    command |\@onlypreamble|.
% \changes{v2.0g}{89/09/19}{`theoremheaderfont and `theorembodyfont
%    added.}
% \changes{v2.2a}{94/02/03}{Use \LaTeXe{} interface}
%    \begin{macrocode}
\@onlypreamble\@xnthm
\@onlypreamble\@ynthm
\@onlypreamble\@othm
\@onlypreamble\newtheorem
\@onlypreamble\theoremstyle
\@onlypreamble\theorembodyfont
\@onlypreamble\theoremheaderfont
%    \end{macrocode}
% \end{macro}
%
% Finally we declare the \texttt{plain} theorem style to be the default.
% \changes{v2.1b}{90/06/04}{Fixed missing `item by declaring default
%    style.}
%    \begin{macrocode}
\theoremstyle{plain}
%</package>
%    \end{macrocode}
%
% \Finale

\endinput
}{}
\ProvidesFile{thb.sty}
             [\FMithmInfo]
%    \end{macrocode}
%
% \begin{macro}{\th@break}
%    |\theoremstyle{break}| produces a line break after the name
%    of the theorem.  The font is |\slshape|.  Hence, we define
%    |\th@break| as follows:
%    \begin{macrocode}
\gdef\th@break{\normalfont\slshape
  \def\@begintheorem##1##2{\item[%
%    \end{macrocode}
%    We run into the following problem: it is not possible to create
%    the header with |\item[|\meta{title}|]| and then start
%    a new line by, for example, |\mbox{}\\|. Such a definition
%    will fail whenever a list environment follows immediately.  With
%    the above construction, the |\mbox{}| causes the switch
%    |@inlabel| (cf.\ definition of |\list| and
%    |\trivlist| in \cite{latex:source}) to be set to \texttt{false}
%    and so the following list will insert additional vertical
%    space (|\topskip|). This is quite annoying. Therefore, we
%    create the line break within the |\item|. In order to ensure
%    that the text will begin at the proper position in the following
%    line, we simply pretend that the label does not take any
%    room.\footnote{This will lead to problems whenever very high
%    symbols occurring in the line tower into the
%    heading. So, something else has to be done here sometime.}
%    \begin{macrocode}
       \rlap{\vbox{\hbox{\hskip \labelsep\theorem@headerfont ##1\ ##2}%
                   \hbox{\strut}}}]}%
%    \end{macrocode}
%    Again, the definition of |\@opargbegintheorem| is
%    completely analogous.
%    \begin{macrocode}
\def\@opargbegintheorem##1##2##3{%
  \item[\rlap{\vbox{\hbox{\hskip \labelsep \theorem@headerfont
                     ##1\ ##2\ (##3)}%
                    \hbox{\strut}}}]}}
\endgroup
%</thb>
%    \end{macrocode}
% \end{macro}
%
%
% \subsubsection{The \texttt{changebreak} style}
%
%    \begin{macrocode}
% This style option is stored in the file |thcb.sty|.
%    \begin{macrocode}
%<*thcb>
\begingroup \makeatletter
\@ifundefined{theorem@style}{% \iffalse meta-comment
%
% Copyright (C) 1993-2024
%
% The LaTeX Project and any individual authors listed elsewhere
% in this file.
%
% This file is part of the Standard LaTeX `Tools Bundle'.
% -------------------------------------------------------
%
% It may be distributed and/or modified under the
% conditions of the LaTeX Project Public License, either version 1.3c
% of this license or (at your option) any later version.
% The latest version of this license is in
%    https://www.latex-project.org/lppl.txt
% and version 1.3c or later is part of all distributions of LaTeX
% version 2005/12/01 or later.
%
% The list of all files belonging to the LaTeX `Tools Bundle' is
% given in the file `manifest.txt'.
%
% \fi
%
%%
%
%
\def\FMithmInfo{2023/07/05 v2.2c Theorem extension package (FMi)}
%
% \ProvidesFile{theorem.dtx}[\FMithmInfo]
%\iffalse   % this is a METACOMMENT !
%%
%% Package `theorem' to use with LaTeX2e
%% Copyright (C) 1989-1995 Frank Mittelbach, all rights reserved.
%<+package>\@ifundefined{theorem@style}{}{\endinput}
%<+package>\NeedsTeXFormat{LaTeX2e}[1995/06/01]
%<+package>\ProvidesPackage{theorem}
%<+package>         [\FMithmInfo]
%
% \fi
%
% \changes{v1.0}{87/05/07}{First draft.}
% \changes{v1.0a}{87/05/07}{`thistheoremstyle' added.}
% \changes{v1.1a}{87/07/29}{New concept: redefine the internal
%                          \protect\LaTeX{} commands `@xnthm' etc.}
% \changes{v1.1b}{87/07/31}{`theoremfont' used to specify font.}
% \changes{v1.2a}{88/01/08}{`@endtheorem' now globally defined.}
% \changes{v1.2b}{88/01/11}{`newtheorem' only allowed in the preamble.}
% \changes{v1.3a}{88/02/14}{`theoremfont' now a toks register.}
%
% \changes{v2.0a}{88/11/16}{New implementation and new documentation.}
% \changes{v2.0a}{88/11/16}{No layout changes but internal modifications
%                           old substyles won't work.}
% \changes{v2.0b}{89/01/30}{`global' used for all definitions.}
% \changes{v2.0c}{89/06/15}{Documentation translated to English.
%                            Thanks to LLL and Nina}
% \changes{v2.1a}{90/02/15}{Support for new font selection.}
% \changes{v2.2a}{94/02/03}{Cleanup for LaTeX2e}
%
% \DoNotIndex{\@endparenv,\@ifdefinable,\@ifnextchar,\@ifundefined}
% \DoNotIndex{\@namedef,\@nameuse,\@spaces,\@tempa,\@warning}
% \DoNotIndex{\ }
% \DoNotIndex{\begingroup,\bfseries}
% \DoNotIndex{\csname}
% \DoNotIndex{\def,\do,\docdate}
% \DoNotIndex{\endcsname,\endgroup,\endinput,\endtrivlist,\expandafter}
% \DoNotIndex{\filedate,\fileversion}
% \DoNotIndex{\gdef,\global}
% \DoNotIndex{\hbox,\hskip}
% \DoNotIndex{\ignorespaces,\input,\itshape,\item}
% \DoNotIndex{\labelsep,\let,\llap}
% \DoNotIndex{\makeatletter}
% \DoNotIndex{\noexpand}
% \DoNotIndex{\refstepcounter,\relax,\rlap}
% \DoNotIndex{\setlength,\slshape,\space,\string,\strut}
% \DoNotIndex{\the,\trivlist,\typeout}
% \DoNotIndex{\vbox}
% \DoNotIndex{\xdef}
%
% \def\deflabel#1{\ttfamily #1\hfill}
% \def\deflist#1{\begin{list}{}{\settowidth\labelwidth{\ttfamily #1}%
%                               \setlength\leftmargin\labelwidth
%                               \addtolength\leftmargin\labelsep
%                               \let\makelabel\deflabel}}
% \def\enddeflist{\end{list}}
% \def\envfont{\normalfont\ttfamily}
%
% \MakeShortVerb{\|}
%
% \newcommand{\TUB}{{\sl TUGboat\/}}
%
% \GetFileInfo{theorem.dtx}
% \title{An Extension of the
%        \LaTeX{} theorem environment\thanks{This file
%        has version number \fileversion, last
%        revised \filedate.}}
% \author{Frank Mittelbach}
%
% \MaintainedByLaTeXTeam{tools}
% \maketitle
%
% \begin{abstract}
%    The macros described in this paper yield an extension of the
%    \LaTeX{} theorem mechanism. It is designed to satisfy the
%    different requirements of various journals. Thus, the
%    layout of the ``theorems'' can be manipulated by determining a
%    ``style''. This article describes not only the use, but
%    also the definition, of the necessary macros.
% \end{abstract}
%
% \section*{Preface to version 2.2}
%
% For \LaTeXe{} this package did not need any fundamental changes. I
% only modified the messages generated so that theorem layout styles
% will show up with the |\listfiles| command and cleaned the section
% on the New Font Selection Scheme since this is now included in
% \LaTeX.
%
% \section*{Preface to version 2.1}
%
% This version is identical to 2.0g described in \TUB\ 10\#3 except
% for some internal defaults which are now set depending on the used
% font selection scheme.
%
% This was done to avoid unpleasant surprises if the new font selection
% scheme is in force. For further details see
% section~\ref{sec:newfonts} and \cite{bk:GMS94}.
%
% \section{Introduction}
%
% For our purposes here, ``theorems'' are labelled enunciations,
% often set off from the main text by extra space and a font change.
% Theorems, corollaries, conjectures, definitions, and remarks
% are all instances of ``theorems''.  The ``header'' of these structures
% is composed of a label (such as \textsc{Theorem} or \textsc{Remark})
% and a number which serializes an item in the sequence of items
% with the same label.
%
% Shortly after the introduction of \LaTeX{} at the Fachbereich
% Mathematik in Mainz, the desire to manipulate the layout of
% ``theorems'' arose.  In Mainz, the following two conventions came
% into general use:
% \begin{enumerate}
%   \item The number of the theorem is shown in the margin.
%   \item There is a line break at
%         the end of the theorem header.
% \end{enumerate}
%
% Additionally, some journals require different formats which depend on
% the ``sort of theorem'': e.g.\ often remarks and definitions are set
% in |\upshape|, while |\itshape| is employed for main theorems.
%
% Confronted with these requirements, a theorem environment was
% developed in Mainz which allows separate determination of the
% layout of the ``theorems sets'', comparable to |\pagestyle|.
%
%
% \section{The user interface}
% \subsection{Defining new theorem sets}
%
% \DescribeMacro\newtheorem
% As in the original \LaTeX{} version, the command |\newtheorem|
% defines a new ``theorem set'' or ``theorem-like structure''.
% Two required arguments name the new environment and give the text
% to be typeset with each instance of the new ``set'', while
% an optional argument determines how
% the ``set'' is enumerated:
% \begin{description}
%    \item[\ttfamily \bslash newtheorem\{foo\}\{bar\}]
%       The theorem set {\envfont foo} (whose name is \texttt{bar})
%       uses its own counter.
%    \item[\ttfamily \bslash newtheorem\{foo2\}{[foo]}\{bar2\}]
%       The theorem set {\envfont foo2} (printed name \texttt{bar2})
%       uses the same counter as the theorem set \texttt{foo}.
%    \item[\ttfamily \bslash newtheorem\{foo3\}\{bar3\}{[section]}]
%       The theorem set {\envfont foo3} (printed name \texttt{bar3}) is
%       enumerated within the counter \texttt{section}, i.e.\ with every
%       new |\section| the enumeration begins again with 1, and
%       the enumeration is composed from the section-number and the
%       theorem counter itself.
% \end{description}
%
% \DescribeMacro\theoremstyle
% Additionally, the command |\theoremstyle| can define the layout
% of various, or all, theorem sets.  It should be noted that any
% theorem set defined by |\newtheorem| is typeset in the
% |\theoremstyle| that is current at the time of the definition.
% Thus, the following
% \begin{verbatim}
%    \theoremstyle{break}        \newtheorem{Cor}{Corollary}
%    \theoremstyle{plain}        \newtheorem{Exa}{Example}[section]
%\end{verbatim}                             ^^A No blank at line start
% leads to the result that the set {\envfont Cor} is formatted in the
% style \texttt{break}, while the set {\envfont Exa} and all the
% following ones are formatted in the style \texttt{plain}, unless
% another |\theoremstyle| follows.  Since the definitions installed by
% |\newtheorem| are global, one also can limit |\theoremstyle| locally
% by grouping braces.
%
% \DescribeMacro\theorembodyfont
% The choice of the font for the theorem body is completely
% independent of the chosen |\theoremstyle|; this has proven to
% be very advantageous.  For example,
% \begin{verbatim}
%   {\theorembodyfont{\upshape}          \newtheorem{Rem}{Remark}}
%\end{verbatim}
% defines a theorem set {\envfont Rem}, which will be set in
% |\upshape| in the current layout (which in our example is
% \texttt{plain}).  As with |\theoremstyle|, the |\theorembodyfont|
% chosen is that current at the time of |\newtheorem|.  If
% |\theorembodyfont| is not specified or one defines
% |\theorembodyfont{}|, then the font used will be that defined by the
% |\theoremstyle|.
%
% \DescribeMacro\theoremheaderfont It is also possible to customize
% the font used for the theorem headers.  This is, however, a global
% declaration, and therefore there should be at most one
% |\theoremheaderfont| declaration in the preamble.\footnote{If it is
% actually necessary to have different header fonts, one has to define
% new theorem styles (substituting the desired font) or specify the
% information directly in the \texttt{\bslash newtheorem} declaration
% (the unclean variant).}
%
% \DescribeMacro\theorempreskipamount
% \DescribeMacro\theorempostskipamount Two additional parameters
% affect the vertical space around the theorem environments:
% |\theorempreskipamount| and |\theorempostskipamount| define,
% respectively, the spacing before and after such an environment.
% These parameters apply for all theorem sets and can be manipulated
% with the ordinary length macros.  They are rubber lengths
% (`\textsf{skips}'), and therefore can contain \texttt{plus} and
% \texttt{minus} parts.
%
% Since the definition of theorem sets should---most sensibly---be
% placed in the preamble, we only allow installation there.
% It is therefore possible to release the memory used here after
% |\begin{document}|, in order to make room for other
% applications.
%
%
% \subsection{Existing theorem styles}
%
% The following theorem styles exist to date:
% \begin{deflist}{marginbreak}
%    \item[plain]
%       This theorem style emulates the original \LaTeX{} definition,
%       except that additionally the parameters
%       |\theorem...skipamount| are used.
%    \item[break]
%       In this style, the theorem header is followed by a line
%       break.
%    \item[marginbreak]
%       The theorem number is set in the margin, and there is a line
%       break as in \texttt{break}.
%    \item[changebreak]
%       Like \texttt{break}, but with header number and text
%       interchanged.
%    \item[change]
%       Header number and text are interchanged, without a line break.
%    \item[margin]
%       The number is set in the left margin, without a line break.
% \end{deflist}
% All styles (except \texttt{plain}) select |\slshape| as the default
% |\theorembodyfont|.
%
%
% \subsection{Examples}
%
% Given the above theorem sets {\envfont Cor}, {\envfont Exa}
% and {\envfont Rem},
% suppose that the preamble also contains the declarations:
% \begin{verbatim}
%    \theoremstyle{marginbreak}   \newtheorem{Lem}[Cor]{Lemma}
%    \theoremstyle{change}
%    \theorembodyfont{\itshape}        \newtheorem{Def}[Cor]{Definition}
%
%    \theoremheaderfont{\scshape}
%\end{verbatim}
% Then the following are some typical examples of the typeset output
% resulting from their use.
%
% \begin{Cor}
%    This is a sentence typeset in the theorem environment {\envfont
%    Cor}.
% \end{Cor}
%
% \begin{Exa}
%    This is a sentence typeset in the theorem environment {\envfont
%    Exa}.
% \end{Exa}
%
% \begin{Rem}
%    This is a sentence typeset in the theorem environment {\envfont
%    Rem}.
% \end{Rem}
%
% \begin{Lem}[Ben User]
%    This is a sentence typeset in the theorem environment {\envfont
%    Lem}.
% \end{Lem}
%
% \begin{Def}[Very impressive Definition]
%    This is a sentence typeset in the theorem environment {\envfont
%    Def}.
% \end{Def}
%
% The last two examples show the effect of the optional argument to
% a theorem environment (it is the text typeset in parentheses).
%
%
% \section{Special Considerations}
% \label{sec:newfonts}
%
% Theoremheader and body are implemented as a unit. This means that
% the |\theoremheaderfont|  will inherit characteristics of the
% |\theorembodyfont| in \LaTeXe. Thus, if for example
% |\theorembodyfont| is |\itshape| and |\theoremheaderfont|
% is |\bfseries| the font selected for the header will have the
% characteristics `bold extended italic'. If this is not desired one
% should set the |\theoremheaderfont| to something like
% \begin{verbatim}
%      \theoremheaderfont{\normalfont\bfseries}
%\end{verbatim}
% i.e.\ supplying all necessary font information explicitly.
%
%
%
% \section{Acknowledgements}
%
% The publication of this set of macros was only possible with the
% help of Christina Busse (translating the manuscript into English),
% Joachim Pense (playing the r\^ole of typist), Chris Rowley
% (looking everything over) and many others providing useful
% suggestions.
%
% \MaybeStop{%
%    \begin{thebibliography}{9}
%       \bibitem{bk:GMS94} \textsc{M.~Goossens}, \textsc{F.~Mittelbach}
%         and \textsc{A.~Samarin}.
%         \newblock The \LaTeX{} Companion.
%         \newblock
%         Addison-Wesley, Reading, Massachusetts, 1994.
%       \bibitem{latex:source} \textsc{Lamport, Leslie.} \newblock
%                               \texttt{latex.tex}, version 2.09, date
%                               Feb. 1990.
%    \end{thebibliography}
%    \PrintIndex
%    \PrintChanges}
%
%
% \section{The documentation driver file}
%
% The next bit of code contains the documentation driver file for
% \TeX{}, i.e., the file that will produce the documentation you are
% currently reading. It will be extracted from this file by the
% \texttt{docstrip} program. Since it is the first code in the file
% one can alternatively process this file directly with \LaTeXe{} to
% obtain the documentation.
%
%    \begin{macrocode}
%<*driver>
\documentclass{ltxdoc}

\usepackage{theorem}

%<+driver>% The next few lines define theorem sets which are used
%<+driver>% in the example section of the documentation.

\theoremstyle{break}       \newtheorem{Cor}{Corollary}
\theoremstyle{plain}       \newtheorem{Exa}{Example}[section]
{\theorembodyfont{\upshape}\newtheorem{Rem}{Remark}}
\theoremstyle{marginbreak} \newtheorem{Lem}[Cor]{Lemma}
\theoremstyle{change}
\theorembodyfont{\itshape} \newtheorem{Def}[Cor]{Definition}

\theoremheaderfont{\scshape}

\RecordChanges

\begin{document}
   \DocInput{theorem.dtx}
\end{document}
%</driver>
%    \end{macrocode}
%
% \section{Definition of the Macros}
%
% If the file has been loaded before, we abort immediately. If not the
% package announces itself (this is actually done at the very top of
% the file---the way it is done isn't good style so don't copy it).
% \changes{v2.0e}{89/07/19}{Spaces removed from `typeout'.}
% \changes{v2.2b}{95/11/19}{Announce moved to top of file}
%    \begin{macrocode}
%<*package>
%\@ifundefined{theorem@style}{}{\endinput}
%\def\FMithmInfo{1995/11/19 v2.2b Theorem extension package (FMi)}
%\ProvidesPackage{theorem}[\FMithmInfo]
%    \end{macrocode}
%
%
% \subsection{Definition of theorem styles and fonts}
%
% All the definitions in this file are done globally to allow
% inputting this file inside a group.
%
% \begin{macro}{\theoremstyle}
%    Before a theorem style can be installed, the chosen style must be
%    known. For that reason, we must test to see that
%    |\th@|\meta{style} is known or, more precisely, that it is
%    different from |\relax|. If the style is not known then
%    |\th@plain| is used.
%    \begin{macrocode}
\gdef\theoremstyle#1{%
   \@ifundefined{th@#1}{\@warning
          {Unknown theoremstyle `#1'. Using `plain'}%
          \theorem@style{plain}}%
%    \end{macrocode}
%    We save the theorem style to be used in the token register
%    |\theorem@style|.
%    \begin{macrocode}
      {\theorem@style{#1}}%
%    \end{macrocode}
%    Now we ``evaluate'' the theorem style: this means, we call the
%    macro |\th@|\meta{style} which will activate the relevant
%    definitions which are contained in a separate file.  This is done
%    in a group to suppress changes to the current font.  This could
%    otherwise pose problems together with the new font selection
%    scheme\footnote{When I printed the original article using the new
%    font selection scheme I ended with a document with slanted
%    typefaces (text headings and all) simply because one of the
%    theorem styles used \texttt{\bslash sl} at toplevel.} if the
%    |\th@|\meta{style} is evaluated a second time.
% \changes{v2.1a}{90/02/15}{Group added around `th@..}
% \changes{v2.1a}{90/02/15}{Reseting of `..@begintheorem removed.}
%    \begin{macrocode}
      \begingroup
        \csname th@\the\theorem@style \endcsname
      \endgroup}
%    \end{macrocode}
% \end{macro}
%
%
% \begin{macro}{\@begintheorem}
% \changes{v2.1a}{90/02/15}{Relaxed at toplevel.}
% \begin{macro}{\@opargbegintheorem}
% \changes{v2.1a}{90/02/15}{Relaxed at toplevel.}
%    We reset |\@begintheorem| and |\@opargbegintheorem| to
%    |\relax| since these commands are no longer necessary at
%    toplevel. This will save a few tokens.
%    \begin{macrocode}
\global\let\@begintheorem\relax
\global\let\@opargbegintheorem\relax
%    \end{macrocode}
% \end{macro}
% \end{macro}
%
% \begin{macro}{\theorem@style}
%    Obviously the \textsf{token} register used above has to be
%    allocated. To assure the utmost compatibility with the
%    original \LaTeX{} definition, we set the default theorem style to
%    \texttt{plain}, which implements the usual \LaTeX{} convention.
% \changes{v2.0f}{89/07/20}{Unnecessary `global' removed.}
%    \begin{macrocode}
\newtoks\theorem@style
\global\theorem@style{plain}
%    \end{macrocode}
% \end{macro}
%
% \begin{macro}{\theorembodyfont}
% \begin{macro}{\theorem@bodyfont}
%    If the |\theorembodyfont| is set by the user then it should not
%    interact with the default font set in the theorem style. When the
%    new font selection is in force this may happen if, for example,
%    the default is |\itshape| and the new |\theorembodyfont| is
%    |\sffamily|. So we add a |\reset@font| command in front of the
%    user definition.
% \changes{v2.0f}{89/07/20}{Unnecessary `global' removed.}
% \changes{v2.0g}{89/09/19}{`theoremfont to `theorembodyfont.}
% \changes{v2.1c}{92/06/22}{`reset@font added and logic changed.}
% \changes{v2.1d}{92/10/28}{Reset to empty if arg is empty}
%    \begin{macrocode}
\gdef\theorembodyfont#1{%
%    \end{macrocode}
%    We check if the argument supplied is empty and if so put nothing
%    into the |\theorem@bodyfont| token register to allow for
%    |\theorembodyfont{}| as a mean of using the default of the
%    current |\theoremstyle|.
%    \begin{macrocode}
   \def\@tempa{#1}%
   \ifx\@tempa\@empty
    \theorem@bodyfont{}%
   \else
    \theorem@bodyfont{\reset@font#1}%
   \fi
   }
\newtoks\theorem@bodyfont
\global\theorem@bodyfont{}
%    \end{macrocode}
% \end{macro}
% \end{macro}
%
%
% \begin{macro}{\theoremheaderfont}
% \changes{v2.0g}{89/09/19}{Macro added.}
%    The font for the theorem headers is handled differently because
%    this definition applies to all theorem styles.
%    \begin{macrocode}
\gdef\theoremheaderfont#1{\gdef\theorem@headerfont{#1}%
%    \end{macrocode}
%    After using the macro once it is redefined to produce an error
%    message.
%    \begin{macrocode}
       \gdef\theoremheaderfont##1{%
        \typeout{\string\theoremheaderfont\space should be used
                 only once.}}}
%    \end{macrocode}
% \end{macro}
%
% \begin{macro}{\theorem@headerfont}
% \changes{v2.0g}{89/09/19}{Macro added.}
%    To set the |\theorem@headerfont| default we first test if
%    the new fontselection scheme is in force.
% \changes{v2.1a}{90/02/15}{Support for new font selection.}
%    \begin{macrocode}
\ifx\upshape\undefined
%    \end{macrocode}
%    If not we define it to expand into |\bfseries|. We don't use
%    |\let| just in case a following style option redefines this
%    macro.
%    \begin{macrocode}
\gdef\theorem@headerfont{\bfseries}
%    \end{macrocode}
%    Otherwise we reset the current shape before calling |\bfseries|.
%    \begin{macrocode}
\else \gdef\theorem@headerfont{\normalfont\bfseries}\fi
%    \end{macrocode}
% \end{macro}
%
% \begin{macro}{\th@plain}
% \begin{macro}{\th@break}
% \begin{macro}{\th@marginbreak}
% \begin{macro}{\th@changebreak}
% \begin{macro}{\th@change}
% \begin{macro}{\th@margin}
%    The different styles are defined in macros such as
%    |\th@plain|.  Since memory space is precious in
%    ``non-Big-versions'', we have to avoid offering too many unused
%    definitions.  Therefore we define these styles in separate files
%    that can be loaded on demand.  Thus the commands themselves only
%    load these files. We use |\@input@|, a \LaTeXe{} internal command
%    that ensures that the file will be listed with |\listfiles|.
% \changes{v2.2a}{94/02/02}{Use `@input@ to load theorem layout files}
%    \begin{macrocode}
\gdef\th@plain{\@input@{thp.sty}}
\gdef\th@break{\@input@{thb.sty}}
\gdef\th@marginbreak{\@input@{thmb.sty}}
\gdef\th@changebreak{\@input@{thcb.sty}}
\gdef\th@change{\@input@{thc.sty}}
\gdef\th@margin{\@input@{thm.sty}}
%    \end{macrocode}
%    This list will be expanded when new styles become
%    available.  For testing,  just append new theorem substyles
%    as document options.
% \end{macro}
% \end{macro}
% \end{macro}
% \end{macro}
% \end{macro}
% \end{macro}
%
%
%
% \subsection{Definition of a new theorem set}
%
% As already pointed out, a new theorem environment can be defined in
% three different ways:
% \begin{verbatim}
%    \newtheorem{Lem}{Lemma}
%    \newtheorem{Lem}{Lemma}[section]
%    \newtheorem{Lem}[Theorem]{Lemma}
%\end{verbatim}
% The function of the macro |\newtheorem| is to recognize these cases
% and then to branch into one of the three macros |\@ynthm|, |\@xnthm|
% or |\@othm|.  This mechanism is adopted unchanged from
% \cite{latex:source}; the essential point here is that, for example,
% in the second case, the arguments \texttt{Lem}, \texttt{Lemma} and
% \texttt{section} are passed over to the macro |\@xnthm|.
%
% We inspect this case first because the others present fewer problems,
% and thus are easily derived from this one.
%
% \begin{macro}{\@xnthm}
%    For our example arguments, the macro |\@xnthm| must fulfill
%    the following:
%    \begin{itemize}
%       \item
%          Define a new \LaTeX-counter `Lem'
%       \item
%          reset this counter within a |\section|
%       \item
%          define the macro |\theLem|
%       \item
%          define the environment macros |\Lem| and
%          |\endLem| using the current
%          |\theoremstyle| and |\theorem@bodyfont|.
%    \end{itemize}
%    Obviously, all this should happen only if the first argument
%    of |\@xnthm| (i.e.\ \texttt{Lem} in our example) is chosen so
%    as not to conflict with any previously defined commands or
%    environments.  This test is performed by the \LaTeX{} macro
%    |\@ifdefinable|.
%    \begin{macrocode}
\gdef\@xnthm#1#2[#3]{\expandafter\@ifdefinable\csname #1\endcsname
%    \end{macrocode}
%    Therefore, the first argument of |\@ifdefinable| is the
%    expansion (in the example, |\Lem|) of
%    |\csname||#1\endcsname|.  The second argument is
%    executed only if the test has been completed successfully.
%    \begin{macrocode}
   {%
%    \end{macrocode}
%    Now we define the new counter. The names of the \LaTeX{}
%    macros employed should speak for themselves:
% \changes{v2.2b}{1995/10/07}{Use \cs{@newctr}}
% \changes{v2.2c}{1995/11/23}{Use \cs{@newctr} properly.}
%    \begin{macrocode}
    \@definecounter{#1}\@newctr{#1}[#3]%
%    \end{macrocode}
%    Using |\@newctr| will give a proper error message if the counter
%    in |#3| is not defined.
%    In defining `|\theLem|' we must generate the desired macro
%    name by use of |\expandafter| and |\csname|.
%    \begin{macrocode}
    \expandafter\xdef\csname the#1\endcsname
%    \end{macrocode}
%    An |\xdef| is used in order to make the definition global,
%    and to ensure that it contains the replacement texts of
%    |\@thmcountersep| and |\@thmcounter|.\footnote{These
%    two macros can be defined by the document style.  Their default
%    values produce a `\texttt{.}' as separation and an arabic
%    representation of the number.} However, not everything should be
%    expanded. For example, it saves space to use |\thesection|
%    instead of its---at times---lengthy expansion.
%    \begin{macrocode}
      {\expandafter \noexpand \csname the#3\endcsname
       \@thmcountersep \@thmcounter{#1}}%
%    \end{macrocode}
%    Thus with the defaults of \LaTeX{}, |\theLem| would be
%    replaced by the command sequence |\thesection.\arabic{Lem}|.
%
%    We will now look at the definition of the macro which is executed
%    at the beginning of the actual environment (in our example this
%    macro is |\Lem|). It should be noted that we use an
%    ``|\expandafter| trick'' to expand only certain parts of the
%    replacement text at the time of the definition.
%    \begin{macrocode}
    \def\@tempa{\global\@namedef{#1}}%
    \expandafter \@tempa \expandafter{%
%    \end{macrocode}
%    First, the macro that contains the current definitions of
%    |\@begintheorem| and |\@opargtheorem| should be called
%    up.  The name of this macro---as is already known---has the form
%    |\th@|\meta{theorem\ style}; therefore, it must be called by
%    \begin{macrocode}
      \csname th@\the \theorem@style
%    \end{macrocode}
%    In addition the default theorem font should be changeable, i.e.\
%    we have to insert the contents of |\theorem@bodyfont|. For
%    that reason, we expand even further, beyond |\endcsname|,
%    and thus insert the contents of the \textsf{token} register
%    |\theorem@bodyfont| in the replacement text.
%    \begin{macrocode}
            \expandafter \endcsname \the \theorem@bodyfont
%    \end{macrocode}
%    Now it is time to call the macro |\@thm| which takes over
%    the further processing.  It has two arguments: the current
%    counter name (in our example, \texttt{Lem}), and the text of the
%    label (in our example, \texttt{Lemma}).
%    \begin{macrocode}
     \@thm{#1}{#2}}%
%    \end{macrocode}
%    With this, the `sub-definition' is complete.  The macro
%    |\@endtheorem| ends a theorem environment and is, so far, nothing
%    but an |\endtrivlist|.  (Hence it is defined globally, and not
%    within the theorem styles.\footnote{This has to be changed as
%    soon as theorem styles that change \texttt{\bslash @endtheorem}
%    exist.  In such a case, all existing styles must be changed as
%    well since they will have to reset the macro.}) Therefore, we can
%    set it equivalent to the macro that ends the theorem set (in our
%    example, |\endLem|).  However, if some day theorem styles exist
%    that do change |\@endtheorem|, we would have to use the
%    commented-out line instead.
%    \begin{macrocode}
    \global \expandafter \let \csname end#1\endcsname \@endtheorem
%   \global\@namedef{end#1}{\@endtheorem}%
%    \end{macrocode}
%    With these commands all the required definitions are employed,
%    unless the test |\@ifdefinable| has failed.  Therefore, we
%    end the second argument of this macro and with it the definition
%    of |\@xnthm|.
%    \begin{macrocode}
   }}
%    \end{macrocode}
% \end{macro}
%
% \begin{macro}{\@ynthm}
%    The definition of |\@ynthm| is completely analogous. In
%    this case the new counter that is defined is not reset within
%    another counter; thus the definition of |\the...|
%    is simplified:
%    \begin{macrocode}
\gdef\@ynthm#1#2{\expandafter\@ifdefinable\csname #1\endcsname
   {\@definecounter{#1}%
    \expandafter\xdef\csname the#1\endcsname{\@thmcounter{#1}}%
%    \end{macrocode}
%    The rest of the definition corresponds literally to that of
%    |\@xnthm|:
%    \begin{macrocode}
    \def\@tempa{\global\@namedef{#1}}\expandafter \@tempa
     \expandafter{\csname th@\the \theorem@style \expandafter
     \endcsname \the\theorem@bodyfont \@thm{#1}{#2}}%
    \global \expandafter \let \csname end#1\endcsname \@endtheorem}}
%    \end{macrocode}
% \end{macro}
%
% \begin{macro}{\@othm}
%    The definition of |\@othm| does not contain anything new.
%    \begin{macrocode}
\gdef\@othm#1[#2]#3{%
%    \end{macrocode}
% \changes{v2.2b}{95/10/07}{Build in check for counter defined}
%    We do not define a new counter but instead use one that has
%    already been defined.  Thus the only definition we need is that
%    of this pseudo-counter (i.e.\ |\the|\meta{env.\ name}).
%    First we check if |#2| corresponds to a known counter name.
%    \begin{macrocode}
  \expandafter\ifx\csname c@#2\endcsname\relax
   \@nocounterr{#2}%
  \else
   \expandafter\@ifdefinable\csname #1\endcsname
   {\expandafter \xdef \csname the#1\endcsname
     {\expandafter \noexpand \csname the#2\endcsname}%
%    \end{macrocode}
%    All other parts of the definition can be adopted from
%    |\@xnthm|. We have to remember, though, that in this case
%    the name of the current counter and the theorem label have moved
%    to the second and third arguments.
%    \begin{macrocode}
    \def\@tempa{\global\@namedef{#1}}\expandafter \@tempa
     \expandafter{\csname th@\the \theorem@style \expandafter
     \endcsname \the\theorem@bodyfont \@thm{#2}{#3}}%
    \global \expandafter \let \csname end#1\endcsname \@endtheorem}%
  \fi}
%    \end{macrocode}
% \end{macro}
%
% \subsection{Macros that are employed in a {\envfont theorem}
%    environment}
%
% \begin{macro}{\@thm}
%    The macro |\@thm| has to increase the current counter.  Then,
%    depending on whether the environment has (or does not have) an
%    optional argument, it has to branch into either |\@begintheorem|
%    or |\@opargtheorem|.
%    \begin{macrocode}
\gdef\@thm#1#2{\refstepcounter{#1}%
%    \end{macrocode}
%    Now we start a {\envfont trivlist} environment, and give
%    |\@topsep| and |\@topsepadd| the values of the \textsf{skip}
%    registers |\theorempreskipamount| and |\theorempostskipamount|.
%    The value in |\@topsep| is the vertical space that is inserted by
%    the first (and only) |\item| in our |\trivlist| whilst
%    |\@topsepadd| is inserted by |\@endparenv| at the end of that
%    {\envfont trivlist} environment.  By using these registers, we
%    obtain the desired space around a {\envfont theorem} environment.
%    \begin{macrocode}
   \trivlist
   \@topsep \theorempreskipamount               % used by first \item
   \@topsepadd \theorempostskipamount           % used by \@endparenv
%    \end{macrocode}
%    Now we have to test whether an optional argument has been given.
% \changes{v2.0d}{89/07/18}{Forgotten percent added.}
%    \begin{macrocode}
   \@ifnextchar [%
%    \end{macrocode}
%    If there is an optional argument, we will call |\@ythm|, and
%    move the arguments read back into the input stream.
%    \begin{macrocode}
   {\@ythm{#1}{#2}}%
%    \end{macrocode}
%    If not, we call |\@begintheorem|. Its first argument is the
%    name of the theorem set (hence the second argument of
%    |\@thm|). Its second argument is the macro that produces the
%    current number.
%    \begin{macrocode}
   {\@begintheorem{#2}{\csname the#1\endcsname}\ignorespaces}}
%    \end{macrocode}
% \end{macro}
%
% \begin{macro}{\@xthm}
% \begin{macro}{\@ythm}
%    Both these macros were originally called by |\@thm|.  We do
%    not need |\@xthm| anymore, hence we reset it to
%    |\relax|. The definition of |\@ythm| has not changed at
%    all from its definition in \LaTeX. In order to make the macros
%    easier to understand, we will
%    nevertheless present it (commented out).
%    \begin{macrocode}
\global\let\@xthm\relax
% \def\@ythm#1#2[#3]{\@opargbegintheorem{#2}{\csname
%           the#1\endcsname}{#3}\ignorespaces}
%</package>
%    \end{macrocode}
%    The primitive |\ignorespaces| in |\@ythm| and
%    |\@thm| is needed to remove the spaces between the
%    |\begin{...}| and the actual text.
% \end{macro}
% \end{macro}
%
%
% \subsection{Definition of the theorem substyles}
%
% As already pointed out, the theorem substyles, defined below, are
% only loaded when necessary.  Note that all these substyles, except
% \texttt{plain}, have |\slshape| as the default body font.
%
% \subsubsection{The \texttt{plain} style}
%
% \label{dok:th@plain} As the following macros use |@|, we have
% to locally set the |\catcode| of this symbol to ``letter''.
% This happens within a group, so that we do not have to worry about
% which |\catcode| that symbol had before.
%    \begin{macrocode}
%<*thp>
\begingroup \makeatletter
%    \end{macrocode}
% Since we are now within a group, we must make all definitions
% globally.  First we make sure that \texttt{theorem.sty} is loaded.
% This will allow us to use this file as a document style option
% without having to call \texttt{theorem} itself as an option.  At the
% same time, we assure that at least version 2 is loaded, since
% |\theorem@style| was not defined in earlier versions.
%    \begin{macrocode}
\@ifundefined{theorem@style}{\input{theorem.sty}}{}
\ProvidesFile{thp.sty}
             [\FMithmInfo]
%    \end{macrocode}
%
%
% \begin{macro}{\th@plain}
% \begin{macro}{\@begintheorem}
% \begin{macro}{\@opargbegintheorem}
%    |\theoremstyle{plain}| corresponds to the original
%    definition, except that the distances to the surrounding text are
%    determined by the parameters |\theorempreskipamount| and
%    |\theorempostskipamount|.
%    First we set the default body font.
%    \begin{macrocode}
\gdef\th@plain{\normalfont\itshape
%    \end{macrocode}
%    Then we define |\@begintheorem| and
%    |\@opargbegintheorem|. These two macros define how the
%    header of a theorem is typeset.  |\@opargbegintheorem| will
%    be called if a \textsf{theorem} environment with an optional
%    argument is encountered; otherwise, the header is constructed by
%    calling |\@begintheorem|.  If one of these macros is
%    executed, we are within a \textsf{trivlist} environment started by
%    |\@thm|.  So the theorem header is produced with an
%    |\item| command.
%
%    Instead of specifying the header font directly, all standard
%    theorem styles use the |\theorem@headerfont| macro to allow
%    customization.  The extra space (|\labelsep|) is necessary
%    because of problems in the \textsf{trivlist} environment.
%    \begin{macrocode}
  \def\@begintheorem##1##2{%
        \item[\hskip\labelsep \theorem@headerfont ##1\ ##2]}%
%    \end{macrocode}
%    The definition of |\@opargbegintheorem| is completely
%    analogous. The only difference is the fact that there exists a
%    third argument (which is the optional parameter of the
%    environment and contains additional information about the
%    theorem).  Customarily we enclose it in parentheses.
%    \begin{macrocode}
\def\@opargbegintheorem##1##2##3{%
   \item[\hskip\labelsep \theorem@headerfont ##1\ ##2\ (##3)]}}
%    \end{macrocode}
% \end{macro}
% \end{macro}
% \end{macro}
%
% \noindent
% We conclude with an |\endgroup| to restore the
% |\catcode| of |@|.
%    \begin{macrocode}
\endgroup
%</thp>
%    \end{macrocode}
%
%
% \subsubsection{The \texttt{break} style}
%
% This style option is stored in the file |thb.sty|.
% For the next two lines see the documentation for |\th@plain| on
% page~\pageref{dok:th@plain}.
%    \begin{macrocode}
%<*thb>
\begingroup \makeatletter
\@ifundefined{theorem@style}{\input{theorem.sty}}{}
\ProvidesFile{thb.sty}
             [\FMithmInfo]
%    \end{macrocode}
%
% \begin{macro}{\th@break}
%    |\theoremstyle{break}| produces a line break after the name
%    of the theorem.  The font is |\slshape|.  Hence, we define
%    |\th@break| as follows:
%    \begin{macrocode}
\gdef\th@break{\normalfont\slshape
  \def\@begintheorem##1##2{\item[%
%    \end{macrocode}
%    We run into the following problem: it is not possible to create
%    the header with |\item[|\meta{title}|]| and then start
%    a new line by, for example, |\mbox{}\\|. Such a definition
%    will fail whenever a list environment follows immediately.  With
%    the above construction, the |\mbox{}| causes the switch
%    |@inlabel| (cf.\ definition of |\list| and
%    |\trivlist| in \cite{latex:source}) to be set to \texttt{false}
%    and so the following list will insert additional vertical
%    space (|\topskip|). This is quite annoying. Therefore, we
%    create the line break within the |\item|. In order to ensure
%    that the text will begin at the proper position in the following
%    line, we simply pretend that the label does not take any
%    room.\footnote{This will lead to problems whenever very high
%    symbols occurring in the line tower into the
%    heading. So, something else has to be done here sometime.}
%    \begin{macrocode}
       \rlap{\vbox{\hbox{\hskip \labelsep\theorem@headerfont ##1\ ##2}%
                   \hbox{\strut}}}]}%
%    \end{macrocode}
%    Again, the definition of |\@opargbegintheorem| is
%    completely analogous.
%    \begin{macrocode}
\def\@opargbegintheorem##1##2##3{%
  \item[\rlap{\vbox{\hbox{\hskip \labelsep \theorem@headerfont
                     ##1\ ##2\ (##3)}%
                    \hbox{\strut}}}]}}
\endgroup
%</thb>
%    \end{macrocode}
% \end{macro}
%
%
% \subsubsection{The \texttt{changebreak} style}
%
%    \begin{macrocode}
% This style option is stored in the file |thcb.sty|.
%    \begin{macrocode}
%<*thcb>
\begingroup \makeatletter
\@ifundefined{theorem@style}{\input{theorem.sty}}{}
\ProvidesFile{thcb.sty}
             [\FMithmInfo]
%    \end{macrocode}
%
% \begin{macro}{\th@changebreak}
%    The \texttt{change-break} theorem style is like \texttt{break}
%    but with interchange of theorem name and theorem number.  Thus we
%    define |\th@changebreak| as follows:
%    \begin{macrocode}
\gdef\th@changebreak{\normalfont\slshape
  \def\@begintheorem##1##2{\item
       [\rlap{\vbox{\hbox{\hskip\labelsep \theorem@headerfont ##2\ ##1}%
                    \hbox{\strut}}}]}%
\def\@opargbegintheorem##1##2##3{%
  \item[\rlap{\vbox{\hbox{\hskip\labelsep \theorem@headerfont
                     ##2\ ##1\ (##3)}%
                    \hbox{\strut}}}]}}
\endgroup
%</thcb>
%    \end{macrocode}
% \end{macro}
%
%
% \subsubsection{The \texttt{change} style}
%
% This style option is stored in the file |thc.sty|.
%    \begin{macrocode}
%<*thc>
\begingroup \makeatletter
\@ifundefined{theorem@style}{\input{theorem.sty}}{}
\ProvidesFile{thc.sty}
             [\FMithmInfo]
%    \end{macrocode}
%
% \begin{macro}{\th@change}
%    The \texttt{change} theorem style corresponds to the
%    \texttt{change break} style without a linebreak after the header.
%    To say it in another way, it's the same as the \texttt{plain}
%    style but with number and name interchanged and |\slshape| as the
%    default font.
%    \begin{macrocode}
\gdef\th@change{\normalfont\slshape
  \def\@begintheorem##1##2{\item
       [\hskip\labelsep \theorem@headerfont ##2\ ##1]}%
\def\@opargbegintheorem##1##2##3{%
  \item[\hskip\labelsep \theorem@headerfont ##2\ ##1\ (##3)]}}
\endgroup
%</thc>
%    \end{macrocode}
% \end{macro}
%
%
% \subsubsection{The \texttt{marginbreak} style}
%
% This style option is the one used most often at Mainz.  It is saved
% in the file |thmb.sty|.
%    \begin{macrocode}
%<*thmb>
\begingroup \makeatletter
\@ifundefined{theorem@style}{\input{theorem.sty}}{}
\ProvidesFile{thmb.sty}
             [\FMithmInfo]
%    \end{macrocode}
%
% \begin{macro}{\th@marginbreak}
%    The \texttt{margin break} style is nearly the same as the
%    \texttt{change break} style. The only difference is the placement
%    of the theorem number.  We use |\llap| to place it in the left
%    margin.
%
%    In this style |\labelsep| denotes the separation between the
%    number and the text.
%    \begin{macrocode}
\gdef\th@marginbreak{\normalfont\slshape
  \def\@begintheorem##1##2{\item
       [\rlap{\vbox{\theorem@headerfont
                    \hbox{\llap{##2}\hskip\labelsep ##1}%
                    \hbox{\strut}}}]}%
\def\@opargbegintheorem##1##2##3{%
  \item[\rlap{\vbox{\theorem@headerfont
                    \hbox{\llap{##2}\hskip\labelsep ##1\ (##3)}%
                    \hbox{\strut}}}]}}
\endgroup
%</thmb>
%    \end{macrocode}
% \changes{v2.1c}{90/02/17}{`endgroup outside macrocode env.}
% \end{macro}
%
%
%
% \subsubsection{The \texttt{margin} style}
%
% This style option is stored in the file |thm.sty|.
%    \begin{macrocode}
%<*thm>
\begingroup \makeatletter
\@ifundefined{theorem@style}{\input{theorem.sty}}{}
\ProvidesFile{thm.sty}
             [\FMithmInfo]
%    \end{macrocode}
%
% \begin{macro}{\th@margin}
%    Again this is only a variant of the theorem styles described
%    above without any new ideas.
%    \begin{macrocode}
\gdef\th@margin{\normalfont\slshape
  \def\@begintheorem##1##2{\item
       [\theorem@headerfont \llap{##2}\hskip\labelsep ##1]}%
\def\@opargbegintheorem##1##2##3{%
  \item[\theorem@headerfont \llap{##2}\hskip\labelsep ##1\ (##3)]}}
\endgroup
%</thm>
%    \end{macrocode}
% \end{macro}
%
%
%
%
% \subsection{Final Definitions}
%
% \begin{macro}{\theorempreskipamount}
% \begin{macro}{\theorempostskipamount}
%    The \textsf{skip} parameters that regulate the vertical empty space
%    before and after the {\envfont theorem} environment have
%    to be allocated as well.
% \changes{v2.0f}{89/07/20}{Unnecessary `global' removed.}
%    \begin{macrocode}
%<*package>
\newskip\theorempreskipamount
\newskip\theorempostskipamount
%    \end{macrocode}
%    Since we have used the same values for all theorem sets, we now can
%    assign them.
%    \begin{macrocode}
\global\setlength\theorempreskipamount{12pt plus 5pt minus 3pt}
\global\setlength\theorempostskipamount{8pt plus 3pt minus 1.5pt}
%    \end{macrocode}
% \end{macro}
% \end{macro}
%
%
% \begin{macro}{\@endtheorem}
%    The same holds for the macro |\@endtheorem|, which ends a
%    {\envfont theorem} environment. Since it is the same for all
%    theorem sets, it is removed from the macros |\th@|\meta{style}.
%    It simply ends the {\envfont trivlist} environment, which was
%    begun in |\@thm|.
%    \begin{macrocode}
\global\let\@endtheorem=\endtrivlist
%    \end{macrocode}
% \end{macro}
%
%
%
% \begin{macro}{\@preamblecmds}
%    All macros defined above are to be used only in the preamble.
%    Therefore, we insert them in |\@preamblecmds| which will disable
%    them at begin document. This is done by the internal \LaTeXe{}
%    command |\@onlypreamble|.
% \changes{v2.0g}{89/09/19}{`theoremheaderfont and `theorembodyfont
%    added.}
% \changes{v2.2a}{94/02/03}{Use \LaTeXe{} interface}
%    \begin{macrocode}
\@onlypreamble\@xnthm
\@onlypreamble\@ynthm
\@onlypreamble\@othm
\@onlypreamble\newtheorem
\@onlypreamble\theoremstyle
\@onlypreamble\theorembodyfont
\@onlypreamble\theoremheaderfont
%    \end{macrocode}
% \end{macro}
%
% Finally we declare the \texttt{plain} theorem style to be the default.
% \changes{v2.1b}{90/06/04}{Fixed missing `item by declaring default
%    style.}
%    \begin{macrocode}
\theoremstyle{plain}
%</package>
%    \end{macrocode}
%
% \Finale

\endinput
}{}
\ProvidesFile{thcb.sty}
             [\FMithmInfo]
%    \end{macrocode}
%
% \begin{macro}{\th@changebreak}
%    The \texttt{change-break} theorem style is like \texttt{break}
%    but with interchange of theorem name and theorem number.  Thus we
%    define |\th@changebreak| as follows:
%    \begin{macrocode}
\gdef\th@changebreak{\normalfont\slshape
  \def\@begintheorem##1##2{\item
       [\rlap{\vbox{\hbox{\hskip\labelsep \theorem@headerfont ##2\ ##1}%
                    \hbox{\strut}}}]}%
\def\@opargbegintheorem##1##2##3{%
  \item[\rlap{\vbox{\hbox{\hskip\labelsep \theorem@headerfont
                     ##2\ ##1\ (##3)}%
                    \hbox{\strut}}}]}}
\endgroup
%</thcb>
%    \end{macrocode}
% \end{macro}
%
%
% \subsubsection{The \texttt{change} style}
%
% This style option is stored in the file |thc.sty|.
%    \begin{macrocode}
%<*thc>
\begingroup \makeatletter
\@ifundefined{theorem@style}{% \iffalse meta-comment
%
% Copyright (C) 1993-2024
%
% The LaTeX Project and any individual authors listed elsewhere
% in this file.
%
% This file is part of the Standard LaTeX `Tools Bundle'.
% -------------------------------------------------------
%
% It may be distributed and/or modified under the
% conditions of the LaTeX Project Public License, either version 1.3c
% of this license or (at your option) any later version.
% The latest version of this license is in
%    https://www.latex-project.org/lppl.txt
% and version 1.3c or later is part of all distributions of LaTeX
% version 2005/12/01 or later.
%
% The list of all files belonging to the LaTeX `Tools Bundle' is
% given in the file `manifest.txt'.
%
% \fi
%
%%
%
%
\def\FMithmInfo{2023/07/05 v2.2c Theorem extension package (FMi)}
%
% \ProvidesFile{theorem.dtx}[\FMithmInfo]
%\iffalse   % this is a METACOMMENT !
%%
%% Package `theorem' to use with LaTeX2e
%% Copyright (C) 1989-1995 Frank Mittelbach, all rights reserved.
%<+package>\@ifundefined{theorem@style}{}{\endinput}
%<+package>\NeedsTeXFormat{LaTeX2e}[1995/06/01]
%<+package>\ProvidesPackage{theorem}
%<+package>         [\FMithmInfo]
%
% \fi
%
% \changes{v1.0}{87/05/07}{First draft.}
% \changes{v1.0a}{87/05/07}{`thistheoremstyle' added.}
% \changes{v1.1a}{87/07/29}{New concept: redefine the internal
%                          \protect\LaTeX{} commands `@xnthm' etc.}
% \changes{v1.1b}{87/07/31}{`theoremfont' used to specify font.}
% \changes{v1.2a}{88/01/08}{`@endtheorem' now globally defined.}
% \changes{v1.2b}{88/01/11}{`newtheorem' only allowed in the preamble.}
% \changes{v1.3a}{88/02/14}{`theoremfont' now a toks register.}
%
% \changes{v2.0a}{88/11/16}{New implementation and new documentation.}
% \changes{v2.0a}{88/11/16}{No layout changes but internal modifications
%                           old substyles won't work.}
% \changes{v2.0b}{89/01/30}{`global' used for all definitions.}
% \changes{v2.0c}{89/06/15}{Documentation translated to English.
%                            Thanks to LLL and Nina}
% \changes{v2.1a}{90/02/15}{Support for new font selection.}
% \changes{v2.2a}{94/02/03}{Cleanup for LaTeX2e}
%
% \DoNotIndex{\@endparenv,\@ifdefinable,\@ifnextchar,\@ifundefined}
% \DoNotIndex{\@namedef,\@nameuse,\@spaces,\@tempa,\@warning}
% \DoNotIndex{\ }
% \DoNotIndex{\begingroup,\bfseries}
% \DoNotIndex{\csname}
% \DoNotIndex{\def,\do,\docdate}
% \DoNotIndex{\endcsname,\endgroup,\endinput,\endtrivlist,\expandafter}
% \DoNotIndex{\filedate,\fileversion}
% \DoNotIndex{\gdef,\global}
% \DoNotIndex{\hbox,\hskip}
% \DoNotIndex{\ignorespaces,\input,\itshape,\item}
% \DoNotIndex{\labelsep,\let,\llap}
% \DoNotIndex{\makeatletter}
% \DoNotIndex{\noexpand}
% \DoNotIndex{\refstepcounter,\relax,\rlap}
% \DoNotIndex{\setlength,\slshape,\space,\string,\strut}
% \DoNotIndex{\the,\trivlist,\typeout}
% \DoNotIndex{\vbox}
% \DoNotIndex{\xdef}
%
% \def\deflabel#1{\ttfamily #1\hfill}
% \def\deflist#1{\begin{list}{}{\settowidth\labelwidth{\ttfamily #1}%
%                               \setlength\leftmargin\labelwidth
%                               \addtolength\leftmargin\labelsep
%                               \let\makelabel\deflabel}}
% \def\enddeflist{\end{list}}
% \def\envfont{\normalfont\ttfamily}
%
% \MakeShortVerb{\|}
%
% \newcommand{\TUB}{{\sl TUGboat\/}}
%
% \GetFileInfo{theorem.dtx}
% \title{An Extension of the
%        \LaTeX{} theorem environment\thanks{This file
%        has version number \fileversion, last
%        revised \filedate.}}
% \author{Frank Mittelbach}
%
% \MaintainedByLaTeXTeam{tools}
% \maketitle
%
% \begin{abstract}
%    The macros described in this paper yield an extension of the
%    \LaTeX{} theorem mechanism. It is designed to satisfy the
%    different requirements of various journals. Thus, the
%    layout of the ``theorems'' can be manipulated by determining a
%    ``style''. This article describes not only the use, but
%    also the definition, of the necessary macros.
% \end{abstract}
%
% \section*{Preface to version 2.2}
%
% For \LaTeXe{} this package did not need any fundamental changes. I
% only modified the messages generated so that theorem layout styles
% will show up with the |\listfiles| command and cleaned the section
% on the New Font Selection Scheme since this is now included in
% \LaTeX.
%
% \section*{Preface to version 2.1}
%
% This version is identical to 2.0g described in \TUB\ 10\#3 except
% for some internal defaults which are now set depending on the used
% font selection scheme.
%
% This was done to avoid unpleasant surprises if the new font selection
% scheme is in force. For further details see
% section~\ref{sec:newfonts} and \cite{bk:GMS94}.
%
% \section{Introduction}
%
% For our purposes here, ``theorems'' are labelled enunciations,
% often set off from the main text by extra space and a font change.
% Theorems, corollaries, conjectures, definitions, and remarks
% are all instances of ``theorems''.  The ``header'' of these structures
% is composed of a label (such as \textsc{Theorem} or \textsc{Remark})
% and a number which serializes an item in the sequence of items
% with the same label.
%
% Shortly after the introduction of \LaTeX{} at the Fachbereich
% Mathematik in Mainz, the desire to manipulate the layout of
% ``theorems'' arose.  In Mainz, the following two conventions came
% into general use:
% \begin{enumerate}
%   \item The number of the theorem is shown in the margin.
%   \item There is a line break at
%         the end of the theorem header.
% \end{enumerate}
%
% Additionally, some journals require different formats which depend on
% the ``sort of theorem'': e.g.\ often remarks and definitions are set
% in |\upshape|, while |\itshape| is employed for main theorems.
%
% Confronted with these requirements, a theorem environment was
% developed in Mainz which allows separate determination of the
% layout of the ``theorems sets'', comparable to |\pagestyle|.
%
%
% \section{The user interface}
% \subsection{Defining new theorem sets}
%
% \DescribeMacro\newtheorem
% As in the original \LaTeX{} version, the command |\newtheorem|
% defines a new ``theorem set'' or ``theorem-like structure''.
% Two required arguments name the new environment and give the text
% to be typeset with each instance of the new ``set'', while
% an optional argument determines how
% the ``set'' is enumerated:
% \begin{description}
%    \item[\ttfamily \bslash newtheorem\{foo\}\{bar\}]
%       The theorem set {\envfont foo} (whose name is \texttt{bar})
%       uses its own counter.
%    \item[\ttfamily \bslash newtheorem\{foo2\}{[foo]}\{bar2\}]
%       The theorem set {\envfont foo2} (printed name \texttt{bar2})
%       uses the same counter as the theorem set \texttt{foo}.
%    \item[\ttfamily \bslash newtheorem\{foo3\}\{bar3\}{[section]}]
%       The theorem set {\envfont foo3} (printed name \texttt{bar3}) is
%       enumerated within the counter \texttt{section}, i.e.\ with every
%       new |\section| the enumeration begins again with 1, and
%       the enumeration is composed from the section-number and the
%       theorem counter itself.
% \end{description}
%
% \DescribeMacro\theoremstyle
% Additionally, the command |\theoremstyle| can define the layout
% of various, or all, theorem sets.  It should be noted that any
% theorem set defined by |\newtheorem| is typeset in the
% |\theoremstyle| that is current at the time of the definition.
% Thus, the following
% \begin{verbatim}
%    \theoremstyle{break}        \newtheorem{Cor}{Corollary}
%    \theoremstyle{plain}        \newtheorem{Exa}{Example}[section]
%\end{verbatim}                             ^^A No blank at line start
% leads to the result that the set {\envfont Cor} is formatted in the
% style \texttt{break}, while the set {\envfont Exa} and all the
% following ones are formatted in the style \texttt{plain}, unless
% another |\theoremstyle| follows.  Since the definitions installed by
% |\newtheorem| are global, one also can limit |\theoremstyle| locally
% by grouping braces.
%
% \DescribeMacro\theorembodyfont
% The choice of the font for the theorem body is completely
% independent of the chosen |\theoremstyle|; this has proven to
% be very advantageous.  For example,
% \begin{verbatim}
%   {\theorembodyfont{\upshape}          \newtheorem{Rem}{Remark}}
%\end{verbatim}
% defines a theorem set {\envfont Rem}, which will be set in
% |\upshape| in the current layout (which in our example is
% \texttt{plain}).  As with |\theoremstyle|, the |\theorembodyfont|
% chosen is that current at the time of |\newtheorem|.  If
% |\theorembodyfont| is not specified or one defines
% |\theorembodyfont{}|, then the font used will be that defined by the
% |\theoremstyle|.
%
% \DescribeMacro\theoremheaderfont It is also possible to customize
% the font used for the theorem headers.  This is, however, a global
% declaration, and therefore there should be at most one
% |\theoremheaderfont| declaration in the preamble.\footnote{If it is
% actually necessary to have different header fonts, one has to define
% new theorem styles (substituting the desired font) or specify the
% information directly in the \texttt{\bslash newtheorem} declaration
% (the unclean variant).}
%
% \DescribeMacro\theorempreskipamount
% \DescribeMacro\theorempostskipamount Two additional parameters
% affect the vertical space around the theorem environments:
% |\theorempreskipamount| and |\theorempostskipamount| define,
% respectively, the spacing before and after such an environment.
% These parameters apply for all theorem sets and can be manipulated
% with the ordinary length macros.  They are rubber lengths
% (`\textsf{skips}'), and therefore can contain \texttt{plus} and
% \texttt{minus} parts.
%
% Since the definition of theorem sets should---most sensibly---be
% placed in the preamble, we only allow installation there.
% It is therefore possible to release the memory used here after
% |\begin{document}|, in order to make room for other
% applications.
%
%
% \subsection{Existing theorem styles}
%
% The following theorem styles exist to date:
% \begin{deflist}{marginbreak}
%    \item[plain]
%       This theorem style emulates the original \LaTeX{} definition,
%       except that additionally the parameters
%       |\theorem...skipamount| are used.
%    \item[break]
%       In this style, the theorem header is followed by a line
%       break.
%    \item[marginbreak]
%       The theorem number is set in the margin, and there is a line
%       break as in \texttt{break}.
%    \item[changebreak]
%       Like \texttt{break}, but with header number and text
%       interchanged.
%    \item[change]
%       Header number and text are interchanged, without a line break.
%    \item[margin]
%       The number is set in the left margin, without a line break.
% \end{deflist}
% All styles (except \texttt{plain}) select |\slshape| as the default
% |\theorembodyfont|.
%
%
% \subsection{Examples}
%
% Given the above theorem sets {\envfont Cor}, {\envfont Exa}
% and {\envfont Rem},
% suppose that the preamble also contains the declarations:
% \begin{verbatim}
%    \theoremstyle{marginbreak}   \newtheorem{Lem}[Cor]{Lemma}
%    \theoremstyle{change}
%    \theorembodyfont{\itshape}        \newtheorem{Def}[Cor]{Definition}
%
%    \theoremheaderfont{\scshape}
%\end{verbatim}
% Then the following are some typical examples of the typeset output
% resulting from their use.
%
% \begin{Cor}
%    This is a sentence typeset in the theorem environment {\envfont
%    Cor}.
% \end{Cor}
%
% \begin{Exa}
%    This is a sentence typeset in the theorem environment {\envfont
%    Exa}.
% \end{Exa}
%
% \begin{Rem}
%    This is a sentence typeset in the theorem environment {\envfont
%    Rem}.
% \end{Rem}
%
% \begin{Lem}[Ben User]
%    This is a sentence typeset in the theorem environment {\envfont
%    Lem}.
% \end{Lem}
%
% \begin{Def}[Very impressive Definition]
%    This is a sentence typeset in the theorem environment {\envfont
%    Def}.
% \end{Def}
%
% The last two examples show the effect of the optional argument to
% a theorem environment (it is the text typeset in parentheses).
%
%
% \section{Special Considerations}
% \label{sec:newfonts}
%
% Theoremheader and body are implemented as a unit. This means that
% the |\theoremheaderfont|  will inherit characteristics of the
% |\theorembodyfont| in \LaTeXe. Thus, if for example
% |\theorembodyfont| is |\itshape| and |\theoremheaderfont|
% is |\bfseries| the font selected for the header will have the
% characteristics `bold extended italic'. If this is not desired one
% should set the |\theoremheaderfont| to something like
% \begin{verbatim}
%      \theoremheaderfont{\normalfont\bfseries}
%\end{verbatim}
% i.e.\ supplying all necessary font information explicitly.
%
%
%
% \section{Acknowledgements}
%
% The publication of this set of macros was only possible with the
% help of Christina Busse (translating the manuscript into English),
% Joachim Pense (playing the r\^ole of typist), Chris Rowley
% (looking everything over) and many others providing useful
% suggestions.
%
% \MaybeStop{%
%    \begin{thebibliography}{9}
%       \bibitem{bk:GMS94} \textsc{M.~Goossens}, \textsc{F.~Mittelbach}
%         and \textsc{A.~Samarin}.
%         \newblock The \LaTeX{} Companion.
%         \newblock
%         Addison-Wesley, Reading, Massachusetts, 1994.
%       \bibitem{latex:source} \textsc{Lamport, Leslie.} \newblock
%                               \texttt{latex.tex}, version 2.09, date
%                               Feb. 1990.
%    \end{thebibliography}
%    \PrintIndex
%    \PrintChanges}
%
%
% \section{The documentation driver file}
%
% The next bit of code contains the documentation driver file for
% \TeX{}, i.e., the file that will produce the documentation you are
% currently reading. It will be extracted from this file by the
% \texttt{docstrip} program. Since it is the first code in the file
% one can alternatively process this file directly with \LaTeXe{} to
% obtain the documentation.
%
%    \begin{macrocode}
%<*driver>
\documentclass{ltxdoc}

\usepackage{theorem}

%<+driver>% The next few lines define theorem sets which are used
%<+driver>% in the example section of the documentation.

\theoremstyle{break}       \newtheorem{Cor}{Corollary}
\theoremstyle{plain}       \newtheorem{Exa}{Example}[section]
{\theorembodyfont{\upshape}\newtheorem{Rem}{Remark}}
\theoremstyle{marginbreak} \newtheorem{Lem}[Cor]{Lemma}
\theoremstyle{change}
\theorembodyfont{\itshape} \newtheorem{Def}[Cor]{Definition}

\theoremheaderfont{\scshape}

\RecordChanges

\begin{document}
   \DocInput{theorem.dtx}
\end{document}
%</driver>
%    \end{macrocode}
%
% \section{Definition of the Macros}
%
% If the file has been loaded before, we abort immediately. If not the
% package announces itself (this is actually done at the very top of
% the file---the way it is done isn't good style so don't copy it).
% \changes{v2.0e}{89/07/19}{Spaces removed from `typeout'.}
% \changes{v2.2b}{95/11/19}{Announce moved to top of file}
%    \begin{macrocode}
%<*package>
%\@ifundefined{theorem@style}{}{\endinput}
%\def\FMithmInfo{1995/11/19 v2.2b Theorem extension package (FMi)}
%\ProvidesPackage{theorem}[\FMithmInfo]
%    \end{macrocode}
%
%
% \subsection{Definition of theorem styles and fonts}
%
% All the definitions in this file are done globally to allow
% inputting this file inside a group.
%
% \begin{macro}{\theoremstyle}
%    Before a theorem style can be installed, the chosen style must be
%    known. For that reason, we must test to see that
%    |\th@|\meta{style} is known or, more precisely, that it is
%    different from |\relax|. If the style is not known then
%    |\th@plain| is used.
%    \begin{macrocode}
\gdef\theoremstyle#1{%
   \@ifundefined{th@#1}{\@warning
          {Unknown theoremstyle `#1'. Using `plain'}%
          \theorem@style{plain}}%
%    \end{macrocode}
%    We save the theorem style to be used in the token register
%    |\theorem@style|.
%    \begin{macrocode}
      {\theorem@style{#1}}%
%    \end{macrocode}
%    Now we ``evaluate'' the theorem style: this means, we call the
%    macro |\th@|\meta{style} which will activate the relevant
%    definitions which are contained in a separate file.  This is done
%    in a group to suppress changes to the current font.  This could
%    otherwise pose problems together with the new font selection
%    scheme\footnote{When I printed the original article using the new
%    font selection scheme I ended with a document with slanted
%    typefaces (text headings and all) simply because one of the
%    theorem styles used \texttt{\bslash sl} at toplevel.} if the
%    |\th@|\meta{style} is evaluated a second time.
% \changes{v2.1a}{90/02/15}{Group added around `th@..}
% \changes{v2.1a}{90/02/15}{Reseting of `..@begintheorem removed.}
%    \begin{macrocode}
      \begingroup
        \csname th@\the\theorem@style \endcsname
      \endgroup}
%    \end{macrocode}
% \end{macro}
%
%
% \begin{macro}{\@begintheorem}
% \changes{v2.1a}{90/02/15}{Relaxed at toplevel.}
% \begin{macro}{\@opargbegintheorem}
% \changes{v2.1a}{90/02/15}{Relaxed at toplevel.}
%    We reset |\@begintheorem| and |\@opargbegintheorem| to
%    |\relax| since these commands are no longer necessary at
%    toplevel. This will save a few tokens.
%    \begin{macrocode}
\global\let\@begintheorem\relax
\global\let\@opargbegintheorem\relax
%    \end{macrocode}
% \end{macro}
% \end{macro}
%
% \begin{macro}{\theorem@style}
%    Obviously the \textsf{token} register used above has to be
%    allocated. To assure the utmost compatibility with the
%    original \LaTeX{} definition, we set the default theorem style to
%    \texttt{plain}, which implements the usual \LaTeX{} convention.
% \changes{v2.0f}{89/07/20}{Unnecessary `global' removed.}
%    \begin{macrocode}
\newtoks\theorem@style
\global\theorem@style{plain}
%    \end{macrocode}
% \end{macro}
%
% \begin{macro}{\theorembodyfont}
% \begin{macro}{\theorem@bodyfont}
%    If the |\theorembodyfont| is set by the user then it should not
%    interact with the default font set in the theorem style. When the
%    new font selection is in force this may happen if, for example,
%    the default is |\itshape| and the new |\theorembodyfont| is
%    |\sffamily|. So we add a |\reset@font| command in front of the
%    user definition.
% \changes{v2.0f}{89/07/20}{Unnecessary `global' removed.}
% \changes{v2.0g}{89/09/19}{`theoremfont to `theorembodyfont.}
% \changes{v2.1c}{92/06/22}{`reset@font added and logic changed.}
% \changes{v2.1d}{92/10/28}{Reset to empty if arg is empty}
%    \begin{macrocode}
\gdef\theorembodyfont#1{%
%    \end{macrocode}
%    We check if the argument supplied is empty and if so put nothing
%    into the |\theorem@bodyfont| token register to allow for
%    |\theorembodyfont{}| as a mean of using the default of the
%    current |\theoremstyle|.
%    \begin{macrocode}
   \def\@tempa{#1}%
   \ifx\@tempa\@empty
    \theorem@bodyfont{}%
   \else
    \theorem@bodyfont{\reset@font#1}%
   \fi
   }
\newtoks\theorem@bodyfont
\global\theorem@bodyfont{}
%    \end{macrocode}
% \end{macro}
% \end{macro}
%
%
% \begin{macro}{\theoremheaderfont}
% \changes{v2.0g}{89/09/19}{Macro added.}
%    The font for the theorem headers is handled differently because
%    this definition applies to all theorem styles.
%    \begin{macrocode}
\gdef\theoremheaderfont#1{\gdef\theorem@headerfont{#1}%
%    \end{macrocode}
%    After using the macro once it is redefined to produce an error
%    message.
%    \begin{macrocode}
       \gdef\theoremheaderfont##1{%
        \typeout{\string\theoremheaderfont\space should be used
                 only once.}}}
%    \end{macrocode}
% \end{macro}
%
% \begin{macro}{\theorem@headerfont}
% \changes{v2.0g}{89/09/19}{Macro added.}
%    To set the |\theorem@headerfont| default we first test if
%    the new fontselection scheme is in force.
% \changes{v2.1a}{90/02/15}{Support for new font selection.}
%    \begin{macrocode}
\ifx\upshape\undefined
%    \end{macrocode}
%    If not we define it to expand into |\bfseries|. We don't use
%    |\let| just in case a following style option redefines this
%    macro.
%    \begin{macrocode}
\gdef\theorem@headerfont{\bfseries}
%    \end{macrocode}
%    Otherwise we reset the current shape before calling |\bfseries|.
%    \begin{macrocode}
\else \gdef\theorem@headerfont{\normalfont\bfseries}\fi
%    \end{macrocode}
% \end{macro}
%
% \begin{macro}{\th@plain}
% \begin{macro}{\th@break}
% \begin{macro}{\th@marginbreak}
% \begin{macro}{\th@changebreak}
% \begin{macro}{\th@change}
% \begin{macro}{\th@margin}
%    The different styles are defined in macros such as
%    |\th@plain|.  Since memory space is precious in
%    ``non-Big-versions'', we have to avoid offering too many unused
%    definitions.  Therefore we define these styles in separate files
%    that can be loaded on demand.  Thus the commands themselves only
%    load these files. We use |\@input@|, a \LaTeXe{} internal command
%    that ensures that the file will be listed with |\listfiles|.
% \changes{v2.2a}{94/02/02}{Use `@input@ to load theorem layout files}
%    \begin{macrocode}
\gdef\th@plain{\@input@{thp.sty}}
\gdef\th@break{\@input@{thb.sty}}
\gdef\th@marginbreak{\@input@{thmb.sty}}
\gdef\th@changebreak{\@input@{thcb.sty}}
\gdef\th@change{\@input@{thc.sty}}
\gdef\th@margin{\@input@{thm.sty}}
%    \end{macrocode}
%    This list will be expanded when new styles become
%    available.  For testing,  just append new theorem substyles
%    as document options.
% \end{macro}
% \end{macro}
% \end{macro}
% \end{macro}
% \end{macro}
% \end{macro}
%
%
%
% \subsection{Definition of a new theorem set}
%
% As already pointed out, a new theorem environment can be defined in
% three different ways:
% \begin{verbatim}
%    \newtheorem{Lem}{Lemma}
%    \newtheorem{Lem}{Lemma}[section]
%    \newtheorem{Lem}[Theorem]{Lemma}
%\end{verbatim}
% The function of the macro |\newtheorem| is to recognize these cases
% and then to branch into one of the three macros |\@ynthm|, |\@xnthm|
% or |\@othm|.  This mechanism is adopted unchanged from
% \cite{latex:source}; the essential point here is that, for example,
% in the second case, the arguments \texttt{Lem}, \texttt{Lemma} and
% \texttt{section} are passed over to the macro |\@xnthm|.
%
% We inspect this case first because the others present fewer problems,
% and thus are easily derived from this one.
%
% \begin{macro}{\@xnthm}
%    For our example arguments, the macro |\@xnthm| must fulfill
%    the following:
%    \begin{itemize}
%       \item
%          Define a new \LaTeX-counter `Lem'
%       \item
%          reset this counter within a |\section|
%       \item
%          define the macro |\theLem|
%       \item
%          define the environment macros |\Lem| and
%          |\endLem| using the current
%          |\theoremstyle| and |\theorem@bodyfont|.
%    \end{itemize}
%    Obviously, all this should happen only if the first argument
%    of |\@xnthm| (i.e.\ \texttt{Lem} in our example) is chosen so
%    as not to conflict with any previously defined commands or
%    environments.  This test is performed by the \LaTeX{} macro
%    |\@ifdefinable|.
%    \begin{macrocode}
\gdef\@xnthm#1#2[#3]{\expandafter\@ifdefinable\csname #1\endcsname
%    \end{macrocode}
%    Therefore, the first argument of |\@ifdefinable| is the
%    expansion (in the example, |\Lem|) of
%    |\csname||#1\endcsname|.  The second argument is
%    executed only if the test has been completed successfully.
%    \begin{macrocode}
   {%
%    \end{macrocode}
%    Now we define the new counter. The names of the \LaTeX{}
%    macros employed should speak for themselves:
% \changes{v2.2b}{1995/10/07}{Use \cs{@newctr}}
% \changes{v2.2c}{1995/11/23}{Use \cs{@newctr} properly.}
%    \begin{macrocode}
    \@definecounter{#1}\@newctr{#1}[#3]%
%    \end{macrocode}
%    Using |\@newctr| will give a proper error message if the counter
%    in |#3| is not defined.
%    In defining `|\theLem|' we must generate the desired macro
%    name by use of |\expandafter| and |\csname|.
%    \begin{macrocode}
    \expandafter\xdef\csname the#1\endcsname
%    \end{macrocode}
%    An |\xdef| is used in order to make the definition global,
%    and to ensure that it contains the replacement texts of
%    |\@thmcountersep| and |\@thmcounter|.\footnote{These
%    two macros can be defined by the document style.  Their default
%    values produce a `\texttt{.}' as separation and an arabic
%    representation of the number.} However, not everything should be
%    expanded. For example, it saves space to use |\thesection|
%    instead of its---at times---lengthy expansion.
%    \begin{macrocode}
      {\expandafter \noexpand \csname the#3\endcsname
       \@thmcountersep \@thmcounter{#1}}%
%    \end{macrocode}
%    Thus with the defaults of \LaTeX{}, |\theLem| would be
%    replaced by the command sequence |\thesection.\arabic{Lem}|.
%
%    We will now look at the definition of the macro which is executed
%    at the beginning of the actual environment (in our example this
%    macro is |\Lem|). It should be noted that we use an
%    ``|\expandafter| trick'' to expand only certain parts of the
%    replacement text at the time of the definition.
%    \begin{macrocode}
    \def\@tempa{\global\@namedef{#1}}%
    \expandafter \@tempa \expandafter{%
%    \end{macrocode}
%    First, the macro that contains the current definitions of
%    |\@begintheorem| and |\@opargtheorem| should be called
%    up.  The name of this macro---as is already known---has the form
%    |\th@|\meta{theorem\ style}; therefore, it must be called by
%    \begin{macrocode}
      \csname th@\the \theorem@style
%    \end{macrocode}
%    In addition the default theorem font should be changeable, i.e.\
%    we have to insert the contents of |\theorem@bodyfont|. For
%    that reason, we expand even further, beyond |\endcsname|,
%    and thus insert the contents of the \textsf{token} register
%    |\theorem@bodyfont| in the replacement text.
%    \begin{macrocode}
            \expandafter \endcsname \the \theorem@bodyfont
%    \end{macrocode}
%    Now it is time to call the macro |\@thm| which takes over
%    the further processing.  It has two arguments: the current
%    counter name (in our example, \texttt{Lem}), and the text of the
%    label (in our example, \texttt{Lemma}).
%    \begin{macrocode}
     \@thm{#1}{#2}}%
%    \end{macrocode}
%    With this, the `sub-definition' is complete.  The macro
%    |\@endtheorem| ends a theorem environment and is, so far, nothing
%    but an |\endtrivlist|.  (Hence it is defined globally, and not
%    within the theorem styles.\footnote{This has to be changed as
%    soon as theorem styles that change \texttt{\bslash @endtheorem}
%    exist.  In such a case, all existing styles must be changed as
%    well since they will have to reset the macro.}) Therefore, we can
%    set it equivalent to the macro that ends the theorem set (in our
%    example, |\endLem|).  However, if some day theorem styles exist
%    that do change |\@endtheorem|, we would have to use the
%    commented-out line instead.
%    \begin{macrocode}
    \global \expandafter \let \csname end#1\endcsname \@endtheorem
%   \global\@namedef{end#1}{\@endtheorem}%
%    \end{macrocode}
%    With these commands all the required definitions are employed,
%    unless the test |\@ifdefinable| has failed.  Therefore, we
%    end the second argument of this macro and with it the definition
%    of |\@xnthm|.
%    \begin{macrocode}
   }}
%    \end{macrocode}
% \end{macro}
%
% \begin{macro}{\@ynthm}
%    The definition of |\@ynthm| is completely analogous. In
%    this case the new counter that is defined is not reset within
%    another counter; thus the definition of |\the...|
%    is simplified:
%    \begin{macrocode}
\gdef\@ynthm#1#2{\expandafter\@ifdefinable\csname #1\endcsname
   {\@definecounter{#1}%
    \expandafter\xdef\csname the#1\endcsname{\@thmcounter{#1}}%
%    \end{macrocode}
%    The rest of the definition corresponds literally to that of
%    |\@xnthm|:
%    \begin{macrocode}
    \def\@tempa{\global\@namedef{#1}}\expandafter \@tempa
     \expandafter{\csname th@\the \theorem@style \expandafter
     \endcsname \the\theorem@bodyfont \@thm{#1}{#2}}%
    \global \expandafter \let \csname end#1\endcsname \@endtheorem}}
%    \end{macrocode}
% \end{macro}
%
% \begin{macro}{\@othm}
%    The definition of |\@othm| does not contain anything new.
%    \begin{macrocode}
\gdef\@othm#1[#2]#3{%
%    \end{macrocode}
% \changes{v2.2b}{95/10/07}{Build in check for counter defined}
%    We do not define a new counter but instead use one that has
%    already been defined.  Thus the only definition we need is that
%    of this pseudo-counter (i.e.\ |\the|\meta{env.\ name}).
%    First we check if |#2| corresponds to a known counter name.
%    \begin{macrocode}
  \expandafter\ifx\csname c@#2\endcsname\relax
   \@nocounterr{#2}%
  \else
   \expandafter\@ifdefinable\csname #1\endcsname
   {\expandafter \xdef \csname the#1\endcsname
     {\expandafter \noexpand \csname the#2\endcsname}%
%    \end{macrocode}
%    All other parts of the definition can be adopted from
%    |\@xnthm|. We have to remember, though, that in this case
%    the name of the current counter and the theorem label have moved
%    to the second and third arguments.
%    \begin{macrocode}
    \def\@tempa{\global\@namedef{#1}}\expandafter \@tempa
     \expandafter{\csname th@\the \theorem@style \expandafter
     \endcsname \the\theorem@bodyfont \@thm{#2}{#3}}%
    \global \expandafter \let \csname end#1\endcsname \@endtheorem}%
  \fi}
%    \end{macrocode}
% \end{macro}
%
% \subsection{Macros that are employed in a {\envfont theorem}
%    environment}
%
% \begin{macro}{\@thm}
%    The macro |\@thm| has to increase the current counter.  Then,
%    depending on whether the environment has (or does not have) an
%    optional argument, it has to branch into either |\@begintheorem|
%    or |\@opargtheorem|.
%    \begin{macrocode}
\gdef\@thm#1#2{\refstepcounter{#1}%
%    \end{macrocode}
%    Now we start a {\envfont trivlist} environment, and give
%    |\@topsep| and |\@topsepadd| the values of the \textsf{skip}
%    registers |\theorempreskipamount| and |\theorempostskipamount|.
%    The value in |\@topsep| is the vertical space that is inserted by
%    the first (and only) |\item| in our |\trivlist| whilst
%    |\@topsepadd| is inserted by |\@endparenv| at the end of that
%    {\envfont trivlist} environment.  By using these registers, we
%    obtain the desired space around a {\envfont theorem} environment.
%    \begin{macrocode}
   \trivlist
   \@topsep \theorempreskipamount               % used by first \item
   \@topsepadd \theorempostskipamount           % used by \@endparenv
%    \end{macrocode}
%    Now we have to test whether an optional argument has been given.
% \changes{v2.0d}{89/07/18}{Forgotten percent added.}
%    \begin{macrocode}
   \@ifnextchar [%
%    \end{macrocode}
%    If there is an optional argument, we will call |\@ythm|, and
%    move the arguments read back into the input stream.
%    \begin{macrocode}
   {\@ythm{#1}{#2}}%
%    \end{macrocode}
%    If not, we call |\@begintheorem|. Its first argument is the
%    name of the theorem set (hence the second argument of
%    |\@thm|). Its second argument is the macro that produces the
%    current number.
%    \begin{macrocode}
   {\@begintheorem{#2}{\csname the#1\endcsname}\ignorespaces}}
%    \end{macrocode}
% \end{macro}
%
% \begin{macro}{\@xthm}
% \begin{macro}{\@ythm}
%    Both these macros were originally called by |\@thm|.  We do
%    not need |\@xthm| anymore, hence we reset it to
%    |\relax|. The definition of |\@ythm| has not changed at
%    all from its definition in \LaTeX. In order to make the macros
%    easier to understand, we will
%    nevertheless present it (commented out).
%    \begin{macrocode}
\global\let\@xthm\relax
% \def\@ythm#1#2[#3]{\@opargbegintheorem{#2}{\csname
%           the#1\endcsname}{#3}\ignorespaces}
%</package>
%    \end{macrocode}
%    The primitive |\ignorespaces| in |\@ythm| and
%    |\@thm| is needed to remove the spaces between the
%    |\begin{...}| and the actual text.
% \end{macro}
% \end{macro}
%
%
% \subsection{Definition of the theorem substyles}
%
% As already pointed out, the theorem substyles, defined below, are
% only loaded when necessary.  Note that all these substyles, except
% \texttt{plain}, have |\slshape| as the default body font.
%
% \subsubsection{The \texttt{plain} style}
%
% \label{dok:th@plain} As the following macros use |@|, we have
% to locally set the |\catcode| of this symbol to ``letter''.
% This happens within a group, so that we do not have to worry about
% which |\catcode| that symbol had before.
%    \begin{macrocode}
%<*thp>
\begingroup \makeatletter
%    \end{macrocode}
% Since we are now within a group, we must make all definitions
% globally.  First we make sure that \texttt{theorem.sty} is loaded.
% This will allow us to use this file as a document style option
% without having to call \texttt{theorem} itself as an option.  At the
% same time, we assure that at least version 2 is loaded, since
% |\theorem@style| was not defined in earlier versions.
%    \begin{macrocode}
\@ifundefined{theorem@style}{\input{theorem.sty}}{}
\ProvidesFile{thp.sty}
             [\FMithmInfo]
%    \end{macrocode}
%
%
% \begin{macro}{\th@plain}
% \begin{macro}{\@begintheorem}
% \begin{macro}{\@opargbegintheorem}
%    |\theoremstyle{plain}| corresponds to the original
%    definition, except that the distances to the surrounding text are
%    determined by the parameters |\theorempreskipamount| and
%    |\theorempostskipamount|.
%    First we set the default body font.
%    \begin{macrocode}
\gdef\th@plain{\normalfont\itshape
%    \end{macrocode}
%    Then we define |\@begintheorem| and
%    |\@opargbegintheorem|. These two macros define how the
%    header of a theorem is typeset.  |\@opargbegintheorem| will
%    be called if a \textsf{theorem} environment with an optional
%    argument is encountered; otherwise, the header is constructed by
%    calling |\@begintheorem|.  If one of these macros is
%    executed, we are within a \textsf{trivlist} environment started by
%    |\@thm|.  So the theorem header is produced with an
%    |\item| command.
%
%    Instead of specifying the header font directly, all standard
%    theorem styles use the |\theorem@headerfont| macro to allow
%    customization.  The extra space (|\labelsep|) is necessary
%    because of problems in the \textsf{trivlist} environment.
%    \begin{macrocode}
  \def\@begintheorem##1##2{%
        \item[\hskip\labelsep \theorem@headerfont ##1\ ##2]}%
%    \end{macrocode}
%    The definition of |\@opargbegintheorem| is completely
%    analogous. The only difference is the fact that there exists a
%    third argument (which is the optional parameter of the
%    environment and contains additional information about the
%    theorem).  Customarily we enclose it in parentheses.
%    \begin{macrocode}
\def\@opargbegintheorem##1##2##3{%
   \item[\hskip\labelsep \theorem@headerfont ##1\ ##2\ (##3)]}}
%    \end{macrocode}
% \end{macro}
% \end{macro}
% \end{macro}
%
% \noindent
% We conclude with an |\endgroup| to restore the
% |\catcode| of |@|.
%    \begin{macrocode}
\endgroup
%</thp>
%    \end{macrocode}
%
%
% \subsubsection{The \texttt{break} style}
%
% This style option is stored in the file |thb.sty|.
% For the next two lines see the documentation for |\th@plain| on
% page~\pageref{dok:th@plain}.
%    \begin{macrocode}
%<*thb>
\begingroup \makeatletter
\@ifundefined{theorem@style}{\input{theorem.sty}}{}
\ProvidesFile{thb.sty}
             [\FMithmInfo]
%    \end{macrocode}
%
% \begin{macro}{\th@break}
%    |\theoremstyle{break}| produces a line break after the name
%    of the theorem.  The font is |\slshape|.  Hence, we define
%    |\th@break| as follows:
%    \begin{macrocode}
\gdef\th@break{\normalfont\slshape
  \def\@begintheorem##1##2{\item[%
%    \end{macrocode}
%    We run into the following problem: it is not possible to create
%    the header with |\item[|\meta{title}|]| and then start
%    a new line by, for example, |\mbox{}\\|. Such a definition
%    will fail whenever a list environment follows immediately.  With
%    the above construction, the |\mbox{}| causes the switch
%    |@inlabel| (cf.\ definition of |\list| and
%    |\trivlist| in \cite{latex:source}) to be set to \texttt{false}
%    and so the following list will insert additional vertical
%    space (|\topskip|). This is quite annoying. Therefore, we
%    create the line break within the |\item|. In order to ensure
%    that the text will begin at the proper position in the following
%    line, we simply pretend that the label does not take any
%    room.\footnote{This will lead to problems whenever very high
%    symbols occurring in the line tower into the
%    heading. So, something else has to be done here sometime.}
%    \begin{macrocode}
       \rlap{\vbox{\hbox{\hskip \labelsep\theorem@headerfont ##1\ ##2}%
                   \hbox{\strut}}}]}%
%    \end{macrocode}
%    Again, the definition of |\@opargbegintheorem| is
%    completely analogous.
%    \begin{macrocode}
\def\@opargbegintheorem##1##2##3{%
  \item[\rlap{\vbox{\hbox{\hskip \labelsep \theorem@headerfont
                     ##1\ ##2\ (##3)}%
                    \hbox{\strut}}}]}}
\endgroup
%</thb>
%    \end{macrocode}
% \end{macro}
%
%
% \subsubsection{The \texttt{changebreak} style}
%
%    \begin{macrocode}
% This style option is stored in the file |thcb.sty|.
%    \begin{macrocode}
%<*thcb>
\begingroup \makeatletter
\@ifundefined{theorem@style}{\input{theorem.sty}}{}
\ProvidesFile{thcb.sty}
             [\FMithmInfo]
%    \end{macrocode}
%
% \begin{macro}{\th@changebreak}
%    The \texttt{change-break} theorem style is like \texttt{break}
%    but with interchange of theorem name and theorem number.  Thus we
%    define |\th@changebreak| as follows:
%    \begin{macrocode}
\gdef\th@changebreak{\normalfont\slshape
  \def\@begintheorem##1##2{\item
       [\rlap{\vbox{\hbox{\hskip\labelsep \theorem@headerfont ##2\ ##1}%
                    \hbox{\strut}}}]}%
\def\@opargbegintheorem##1##2##3{%
  \item[\rlap{\vbox{\hbox{\hskip\labelsep \theorem@headerfont
                     ##2\ ##1\ (##3)}%
                    \hbox{\strut}}}]}}
\endgroup
%</thcb>
%    \end{macrocode}
% \end{macro}
%
%
% \subsubsection{The \texttt{change} style}
%
% This style option is stored in the file |thc.sty|.
%    \begin{macrocode}
%<*thc>
\begingroup \makeatletter
\@ifundefined{theorem@style}{\input{theorem.sty}}{}
\ProvidesFile{thc.sty}
             [\FMithmInfo]
%    \end{macrocode}
%
% \begin{macro}{\th@change}
%    The \texttt{change} theorem style corresponds to the
%    \texttt{change break} style without a linebreak after the header.
%    To say it in another way, it's the same as the \texttt{plain}
%    style but with number and name interchanged and |\slshape| as the
%    default font.
%    \begin{macrocode}
\gdef\th@change{\normalfont\slshape
  \def\@begintheorem##1##2{\item
       [\hskip\labelsep \theorem@headerfont ##2\ ##1]}%
\def\@opargbegintheorem##1##2##3{%
  \item[\hskip\labelsep \theorem@headerfont ##2\ ##1\ (##3)]}}
\endgroup
%</thc>
%    \end{macrocode}
% \end{macro}
%
%
% \subsubsection{The \texttt{marginbreak} style}
%
% This style option is the one used most often at Mainz.  It is saved
% in the file |thmb.sty|.
%    \begin{macrocode}
%<*thmb>
\begingroup \makeatletter
\@ifundefined{theorem@style}{\input{theorem.sty}}{}
\ProvidesFile{thmb.sty}
             [\FMithmInfo]
%    \end{macrocode}
%
% \begin{macro}{\th@marginbreak}
%    The \texttt{margin break} style is nearly the same as the
%    \texttt{change break} style. The only difference is the placement
%    of the theorem number.  We use |\llap| to place it in the left
%    margin.
%
%    In this style |\labelsep| denotes the separation between the
%    number and the text.
%    \begin{macrocode}
\gdef\th@marginbreak{\normalfont\slshape
  \def\@begintheorem##1##2{\item
       [\rlap{\vbox{\theorem@headerfont
                    \hbox{\llap{##2}\hskip\labelsep ##1}%
                    \hbox{\strut}}}]}%
\def\@opargbegintheorem##1##2##3{%
  \item[\rlap{\vbox{\theorem@headerfont
                    \hbox{\llap{##2}\hskip\labelsep ##1\ (##3)}%
                    \hbox{\strut}}}]}}
\endgroup
%</thmb>
%    \end{macrocode}
% \changes{v2.1c}{90/02/17}{`endgroup outside macrocode env.}
% \end{macro}
%
%
%
% \subsubsection{The \texttt{margin} style}
%
% This style option is stored in the file |thm.sty|.
%    \begin{macrocode}
%<*thm>
\begingroup \makeatletter
\@ifundefined{theorem@style}{\input{theorem.sty}}{}
\ProvidesFile{thm.sty}
             [\FMithmInfo]
%    \end{macrocode}
%
% \begin{macro}{\th@margin}
%    Again this is only a variant of the theorem styles described
%    above without any new ideas.
%    \begin{macrocode}
\gdef\th@margin{\normalfont\slshape
  \def\@begintheorem##1##2{\item
       [\theorem@headerfont \llap{##2}\hskip\labelsep ##1]}%
\def\@opargbegintheorem##1##2##3{%
  \item[\theorem@headerfont \llap{##2}\hskip\labelsep ##1\ (##3)]}}
\endgroup
%</thm>
%    \end{macrocode}
% \end{macro}
%
%
%
%
% \subsection{Final Definitions}
%
% \begin{macro}{\theorempreskipamount}
% \begin{macro}{\theorempostskipamount}
%    The \textsf{skip} parameters that regulate the vertical empty space
%    before and after the {\envfont theorem} environment have
%    to be allocated as well.
% \changes{v2.0f}{89/07/20}{Unnecessary `global' removed.}
%    \begin{macrocode}
%<*package>
\newskip\theorempreskipamount
\newskip\theorempostskipamount
%    \end{macrocode}
%    Since we have used the same values for all theorem sets, we now can
%    assign them.
%    \begin{macrocode}
\global\setlength\theorempreskipamount{12pt plus 5pt minus 3pt}
\global\setlength\theorempostskipamount{8pt plus 3pt minus 1.5pt}
%    \end{macrocode}
% \end{macro}
% \end{macro}
%
%
% \begin{macro}{\@endtheorem}
%    The same holds for the macro |\@endtheorem|, which ends a
%    {\envfont theorem} environment. Since it is the same for all
%    theorem sets, it is removed from the macros |\th@|\meta{style}.
%    It simply ends the {\envfont trivlist} environment, which was
%    begun in |\@thm|.
%    \begin{macrocode}
\global\let\@endtheorem=\endtrivlist
%    \end{macrocode}
% \end{macro}
%
%
%
% \begin{macro}{\@preamblecmds}
%    All macros defined above are to be used only in the preamble.
%    Therefore, we insert them in |\@preamblecmds| which will disable
%    them at begin document. This is done by the internal \LaTeXe{}
%    command |\@onlypreamble|.
% \changes{v2.0g}{89/09/19}{`theoremheaderfont and `theorembodyfont
%    added.}
% \changes{v2.2a}{94/02/03}{Use \LaTeXe{} interface}
%    \begin{macrocode}
\@onlypreamble\@xnthm
\@onlypreamble\@ynthm
\@onlypreamble\@othm
\@onlypreamble\newtheorem
\@onlypreamble\theoremstyle
\@onlypreamble\theorembodyfont
\@onlypreamble\theoremheaderfont
%    \end{macrocode}
% \end{macro}
%
% Finally we declare the \texttt{plain} theorem style to be the default.
% \changes{v2.1b}{90/06/04}{Fixed missing `item by declaring default
%    style.}
%    \begin{macrocode}
\theoremstyle{plain}
%</package>
%    \end{macrocode}
%
% \Finale

\endinput
}{}
\ProvidesFile{thc.sty}
             [\FMithmInfo]
%    \end{macrocode}
%
% \begin{macro}{\th@change}
%    The \texttt{change} theorem style corresponds to the
%    \texttt{change break} style without a linebreak after the header.
%    To say it in another way, it's the same as the \texttt{plain}
%    style but with number and name interchanged and |\slshape| as the
%    default font.
%    \begin{macrocode}
\gdef\th@change{\normalfont\slshape
  \def\@begintheorem##1##2{\item
       [\hskip\labelsep \theorem@headerfont ##2\ ##1]}%
\def\@opargbegintheorem##1##2##3{%
  \item[\hskip\labelsep \theorem@headerfont ##2\ ##1\ (##3)]}}
\endgroup
%</thc>
%    \end{macrocode}
% \end{macro}
%
%
% \subsubsection{The \texttt{marginbreak} style}
%
% This style option is the one used most often at Mainz.  It is saved
% in the file |thmb.sty|.
%    \begin{macrocode}
%<*thmb>
\begingroup \makeatletter
\@ifundefined{theorem@style}{% \iffalse meta-comment
%
% Copyright (C) 1993-2024
%
% The LaTeX Project and any individual authors listed elsewhere
% in this file.
%
% This file is part of the Standard LaTeX `Tools Bundle'.
% -------------------------------------------------------
%
% It may be distributed and/or modified under the
% conditions of the LaTeX Project Public License, either version 1.3c
% of this license or (at your option) any later version.
% The latest version of this license is in
%    https://www.latex-project.org/lppl.txt
% and version 1.3c or later is part of all distributions of LaTeX
% version 2005/12/01 or later.
%
% The list of all files belonging to the LaTeX `Tools Bundle' is
% given in the file `manifest.txt'.
%
% \fi
%
%%
%
%
\def\FMithmInfo{2023/07/05 v2.2c Theorem extension package (FMi)}
%
% \ProvidesFile{theorem.dtx}[\FMithmInfo]
%\iffalse   % this is a METACOMMENT !
%%
%% Package `theorem' to use with LaTeX2e
%% Copyright (C) 1989-1995 Frank Mittelbach, all rights reserved.
%<+package>\@ifundefined{theorem@style}{}{\endinput}
%<+package>\NeedsTeXFormat{LaTeX2e}[1995/06/01]
%<+package>\ProvidesPackage{theorem}
%<+package>         [\FMithmInfo]
%
% \fi
%
% \changes{v1.0}{87/05/07}{First draft.}
% \changes{v1.0a}{87/05/07}{`thistheoremstyle' added.}
% \changes{v1.1a}{87/07/29}{New concept: redefine the internal
%                          \protect\LaTeX{} commands `@xnthm' etc.}
% \changes{v1.1b}{87/07/31}{`theoremfont' used to specify font.}
% \changes{v1.2a}{88/01/08}{`@endtheorem' now globally defined.}
% \changes{v1.2b}{88/01/11}{`newtheorem' only allowed in the preamble.}
% \changes{v1.3a}{88/02/14}{`theoremfont' now a toks register.}
%
% \changes{v2.0a}{88/11/16}{New implementation and new documentation.}
% \changes{v2.0a}{88/11/16}{No layout changes but internal modifications
%                           old substyles won't work.}
% \changes{v2.0b}{89/01/30}{`global' used for all definitions.}
% \changes{v2.0c}{89/06/15}{Documentation translated to English.
%                            Thanks to LLL and Nina}
% \changes{v2.1a}{90/02/15}{Support for new font selection.}
% \changes{v2.2a}{94/02/03}{Cleanup for LaTeX2e}
%
% \DoNotIndex{\@endparenv,\@ifdefinable,\@ifnextchar,\@ifundefined}
% \DoNotIndex{\@namedef,\@nameuse,\@spaces,\@tempa,\@warning}
% \DoNotIndex{\ }
% \DoNotIndex{\begingroup,\bfseries}
% \DoNotIndex{\csname}
% \DoNotIndex{\def,\do,\docdate}
% \DoNotIndex{\endcsname,\endgroup,\endinput,\endtrivlist,\expandafter}
% \DoNotIndex{\filedate,\fileversion}
% \DoNotIndex{\gdef,\global}
% \DoNotIndex{\hbox,\hskip}
% \DoNotIndex{\ignorespaces,\input,\itshape,\item}
% \DoNotIndex{\labelsep,\let,\llap}
% \DoNotIndex{\makeatletter}
% \DoNotIndex{\noexpand}
% \DoNotIndex{\refstepcounter,\relax,\rlap}
% \DoNotIndex{\setlength,\slshape,\space,\string,\strut}
% \DoNotIndex{\the,\trivlist,\typeout}
% \DoNotIndex{\vbox}
% \DoNotIndex{\xdef}
%
% \def\deflabel#1{\ttfamily #1\hfill}
% \def\deflist#1{\begin{list}{}{\settowidth\labelwidth{\ttfamily #1}%
%                               \setlength\leftmargin\labelwidth
%                               \addtolength\leftmargin\labelsep
%                               \let\makelabel\deflabel}}
% \def\enddeflist{\end{list}}
% \def\envfont{\normalfont\ttfamily}
%
% \MakeShortVerb{\|}
%
% \newcommand{\TUB}{{\sl TUGboat\/}}
%
% \GetFileInfo{theorem.dtx}
% \title{An Extension of the
%        \LaTeX{} theorem environment\thanks{This file
%        has version number \fileversion, last
%        revised \filedate.}}
% \author{Frank Mittelbach}
%
% \MaintainedByLaTeXTeam{tools}
% \maketitle
%
% \begin{abstract}
%    The macros described in this paper yield an extension of the
%    \LaTeX{} theorem mechanism. It is designed to satisfy the
%    different requirements of various journals. Thus, the
%    layout of the ``theorems'' can be manipulated by determining a
%    ``style''. This article describes not only the use, but
%    also the definition, of the necessary macros.
% \end{abstract}
%
% \section*{Preface to version 2.2}
%
% For \LaTeXe{} this package did not need any fundamental changes. I
% only modified the messages generated so that theorem layout styles
% will show up with the |\listfiles| command and cleaned the section
% on the New Font Selection Scheme since this is now included in
% \LaTeX.
%
% \section*{Preface to version 2.1}
%
% This version is identical to 2.0g described in \TUB\ 10\#3 except
% for some internal defaults which are now set depending on the used
% font selection scheme.
%
% This was done to avoid unpleasant surprises if the new font selection
% scheme is in force. For further details see
% section~\ref{sec:newfonts} and \cite{bk:GMS94}.
%
% \section{Introduction}
%
% For our purposes here, ``theorems'' are labelled enunciations,
% often set off from the main text by extra space and a font change.
% Theorems, corollaries, conjectures, definitions, and remarks
% are all instances of ``theorems''.  The ``header'' of these structures
% is composed of a label (such as \textsc{Theorem} or \textsc{Remark})
% and a number which serializes an item in the sequence of items
% with the same label.
%
% Shortly after the introduction of \LaTeX{} at the Fachbereich
% Mathematik in Mainz, the desire to manipulate the layout of
% ``theorems'' arose.  In Mainz, the following two conventions came
% into general use:
% \begin{enumerate}
%   \item The number of the theorem is shown in the margin.
%   \item There is a line break at
%         the end of the theorem header.
% \end{enumerate}
%
% Additionally, some journals require different formats which depend on
% the ``sort of theorem'': e.g.\ often remarks and definitions are set
% in |\upshape|, while |\itshape| is employed for main theorems.
%
% Confronted with these requirements, a theorem environment was
% developed in Mainz which allows separate determination of the
% layout of the ``theorems sets'', comparable to |\pagestyle|.
%
%
% \section{The user interface}
% \subsection{Defining new theorem sets}
%
% \DescribeMacro\newtheorem
% As in the original \LaTeX{} version, the command |\newtheorem|
% defines a new ``theorem set'' or ``theorem-like structure''.
% Two required arguments name the new environment and give the text
% to be typeset with each instance of the new ``set'', while
% an optional argument determines how
% the ``set'' is enumerated:
% \begin{description}
%    \item[\ttfamily \bslash newtheorem\{foo\}\{bar\}]
%       The theorem set {\envfont foo} (whose name is \texttt{bar})
%       uses its own counter.
%    \item[\ttfamily \bslash newtheorem\{foo2\}{[foo]}\{bar2\}]
%       The theorem set {\envfont foo2} (printed name \texttt{bar2})
%       uses the same counter as the theorem set \texttt{foo}.
%    \item[\ttfamily \bslash newtheorem\{foo3\}\{bar3\}{[section]}]
%       The theorem set {\envfont foo3} (printed name \texttt{bar3}) is
%       enumerated within the counter \texttt{section}, i.e.\ with every
%       new |\section| the enumeration begins again with 1, and
%       the enumeration is composed from the section-number and the
%       theorem counter itself.
% \end{description}
%
% \DescribeMacro\theoremstyle
% Additionally, the command |\theoremstyle| can define the layout
% of various, or all, theorem sets.  It should be noted that any
% theorem set defined by |\newtheorem| is typeset in the
% |\theoremstyle| that is current at the time of the definition.
% Thus, the following
% \begin{verbatim}
%    \theoremstyle{break}        \newtheorem{Cor}{Corollary}
%    \theoremstyle{plain}        \newtheorem{Exa}{Example}[section]
%\end{verbatim}                             ^^A No blank at line start
% leads to the result that the set {\envfont Cor} is formatted in the
% style \texttt{break}, while the set {\envfont Exa} and all the
% following ones are formatted in the style \texttt{plain}, unless
% another |\theoremstyle| follows.  Since the definitions installed by
% |\newtheorem| are global, one also can limit |\theoremstyle| locally
% by grouping braces.
%
% \DescribeMacro\theorembodyfont
% The choice of the font for the theorem body is completely
% independent of the chosen |\theoremstyle|; this has proven to
% be very advantageous.  For example,
% \begin{verbatim}
%   {\theorembodyfont{\upshape}          \newtheorem{Rem}{Remark}}
%\end{verbatim}
% defines a theorem set {\envfont Rem}, which will be set in
% |\upshape| in the current layout (which in our example is
% \texttt{plain}).  As with |\theoremstyle|, the |\theorembodyfont|
% chosen is that current at the time of |\newtheorem|.  If
% |\theorembodyfont| is not specified or one defines
% |\theorembodyfont{}|, then the font used will be that defined by the
% |\theoremstyle|.
%
% \DescribeMacro\theoremheaderfont It is also possible to customize
% the font used for the theorem headers.  This is, however, a global
% declaration, and therefore there should be at most one
% |\theoremheaderfont| declaration in the preamble.\footnote{If it is
% actually necessary to have different header fonts, one has to define
% new theorem styles (substituting the desired font) or specify the
% information directly in the \texttt{\bslash newtheorem} declaration
% (the unclean variant).}
%
% \DescribeMacro\theorempreskipamount
% \DescribeMacro\theorempostskipamount Two additional parameters
% affect the vertical space around the theorem environments:
% |\theorempreskipamount| and |\theorempostskipamount| define,
% respectively, the spacing before and after such an environment.
% These parameters apply for all theorem sets and can be manipulated
% with the ordinary length macros.  They are rubber lengths
% (`\textsf{skips}'), and therefore can contain \texttt{plus} and
% \texttt{minus} parts.
%
% Since the definition of theorem sets should---most sensibly---be
% placed in the preamble, we only allow installation there.
% It is therefore possible to release the memory used here after
% |\begin{document}|, in order to make room for other
% applications.
%
%
% \subsection{Existing theorem styles}
%
% The following theorem styles exist to date:
% \begin{deflist}{marginbreak}
%    \item[plain]
%       This theorem style emulates the original \LaTeX{} definition,
%       except that additionally the parameters
%       |\theorem...skipamount| are used.
%    \item[break]
%       In this style, the theorem header is followed by a line
%       break.
%    \item[marginbreak]
%       The theorem number is set in the margin, and there is a line
%       break as in \texttt{break}.
%    \item[changebreak]
%       Like \texttt{break}, but with header number and text
%       interchanged.
%    \item[change]
%       Header number and text are interchanged, without a line break.
%    \item[margin]
%       The number is set in the left margin, without a line break.
% \end{deflist}
% All styles (except \texttt{plain}) select |\slshape| as the default
% |\theorembodyfont|.
%
%
% \subsection{Examples}
%
% Given the above theorem sets {\envfont Cor}, {\envfont Exa}
% and {\envfont Rem},
% suppose that the preamble also contains the declarations:
% \begin{verbatim}
%    \theoremstyle{marginbreak}   \newtheorem{Lem}[Cor]{Lemma}
%    \theoremstyle{change}
%    \theorembodyfont{\itshape}        \newtheorem{Def}[Cor]{Definition}
%
%    \theoremheaderfont{\scshape}
%\end{verbatim}
% Then the following are some typical examples of the typeset output
% resulting from their use.
%
% \begin{Cor}
%    This is a sentence typeset in the theorem environment {\envfont
%    Cor}.
% \end{Cor}
%
% \begin{Exa}
%    This is a sentence typeset in the theorem environment {\envfont
%    Exa}.
% \end{Exa}
%
% \begin{Rem}
%    This is a sentence typeset in the theorem environment {\envfont
%    Rem}.
% \end{Rem}
%
% \begin{Lem}[Ben User]
%    This is a sentence typeset in the theorem environment {\envfont
%    Lem}.
% \end{Lem}
%
% \begin{Def}[Very impressive Definition]
%    This is a sentence typeset in the theorem environment {\envfont
%    Def}.
% \end{Def}
%
% The last two examples show the effect of the optional argument to
% a theorem environment (it is the text typeset in parentheses).
%
%
% \section{Special Considerations}
% \label{sec:newfonts}
%
% Theoremheader and body are implemented as a unit. This means that
% the |\theoremheaderfont|  will inherit characteristics of the
% |\theorembodyfont| in \LaTeXe. Thus, if for example
% |\theorembodyfont| is |\itshape| and |\theoremheaderfont|
% is |\bfseries| the font selected for the header will have the
% characteristics `bold extended italic'. If this is not desired one
% should set the |\theoremheaderfont| to something like
% \begin{verbatim}
%      \theoremheaderfont{\normalfont\bfseries}
%\end{verbatim}
% i.e.\ supplying all necessary font information explicitly.
%
%
%
% \section{Acknowledgements}
%
% The publication of this set of macros was only possible with the
% help of Christina Busse (translating the manuscript into English),
% Joachim Pense (playing the r\^ole of typist), Chris Rowley
% (looking everything over) and many others providing useful
% suggestions.
%
% \MaybeStop{%
%    \begin{thebibliography}{9}
%       \bibitem{bk:GMS94} \textsc{M.~Goossens}, \textsc{F.~Mittelbach}
%         and \textsc{A.~Samarin}.
%         \newblock The \LaTeX{} Companion.
%         \newblock
%         Addison-Wesley, Reading, Massachusetts, 1994.
%       \bibitem{latex:source} \textsc{Lamport, Leslie.} \newblock
%                               \texttt{latex.tex}, version 2.09, date
%                               Feb. 1990.
%    \end{thebibliography}
%    \PrintIndex
%    \PrintChanges}
%
%
% \section{The documentation driver file}
%
% The next bit of code contains the documentation driver file for
% \TeX{}, i.e., the file that will produce the documentation you are
% currently reading. It will be extracted from this file by the
% \texttt{docstrip} program. Since it is the first code in the file
% one can alternatively process this file directly with \LaTeXe{} to
% obtain the documentation.
%
%    \begin{macrocode}
%<*driver>
\documentclass{ltxdoc}

\usepackage{theorem}

%<+driver>% The next few lines define theorem sets which are used
%<+driver>% in the example section of the documentation.

\theoremstyle{break}       \newtheorem{Cor}{Corollary}
\theoremstyle{plain}       \newtheorem{Exa}{Example}[section]
{\theorembodyfont{\upshape}\newtheorem{Rem}{Remark}}
\theoremstyle{marginbreak} \newtheorem{Lem}[Cor]{Lemma}
\theoremstyle{change}
\theorembodyfont{\itshape} \newtheorem{Def}[Cor]{Definition}

\theoremheaderfont{\scshape}

\RecordChanges

\begin{document}
   \DocInput{theorem.dtx}
\end{document}
%</driver>
%    \end{macrocode}
%
% \section{Definition of the Macros}
%
% If the file has been loaded before, we abort immediately. If not the
% package announces itself (this is actually done at the very top of
% the file---the way it is done isn't good style so don't copy it).
% \changes{v2.0e}{89/07/19}{Spaces removed from `typeout'.}
% \changes{v2.2b}{95/11/19}{Announce moved to top of file}
%    \begin{macrocode}
%<*package>
%\@ifundefined{theorem@style}{}{\endinput}
%\def\FMithmInfo{1995/11/19 v2.2b Theorem extension package (FMi)}
%\ProvidesPackage{theorem}[\FMithmInfo]
%    \end{macrocode}
%
%
% \subsection{Definition of theorem styles and fonts}
%
% All the definitions in this file are done globally to allow
% inputting this file inside a group.
%
% \begin{macro}{\theoremstyle}
%    Before a theorem style can be installed, the chosen style must be
%    known. For that reason, we must test to see that
%    |\th@|\meta{style} is known or, more precisely, that it is
%    different from |\relax|. If the style is not known then
%    |\th@plain| is used.
%    \begin{macrocode}
\gdef\theoremstyle#1{%
   \@ifundefined{th@#1}{\@warning
          {Unknown theoremstyle `#1'. Using `plain'}%
          \theorem@style{plain}}%
%    \end{macrocode}
%    We save the theorem style to be used in the token register
%    |\theorem@style|.
%    \begin{macrocode}
      {\theorem@style{#1}}%
%    \end{macrocode}
%    Now we ``evaluate'' the theorem style: this means, we call the
%    macro |\th@|\meta{style} which will activate the relevant
%    definitions which are contained in a separate file.  This is done
%    in a group to suppress changes to the current font.  This could
%    otherwise pose problems together with the new font selection
%    scheme\footnote{When I printed the original article using the new
%    font selection scheme I ended with a document with slanted
%    typefaces (text headings and all) simply because one of the
%    theorem styles used \texttt{\bslash sl} at toplevel.} if the
%    |\th@|\meta{style} is evaluated a second time.
% \changes{v2.1a}{90/02/15}{Group added around `th@..}
% \changes{v2.1a}{90/02/15}{Reseting of `..@begintheorem removed.}
%    \begin{macrocode}
      \begingroup
        \csname th@\the\theorem@style \endcsname
      \endgroup}
%    \end{macrocode}
% \end{macro}
%
%
% \begin{macro}{\@begintheorem}
% \changes{v2.1a}{90/02/15}{Relaxed at toplevel.}
% \begin{macro}{\@opargbegintheorem}
% \changes{v2.1a}{90/02/15}{Relaxed at toplevel.}
%    We reset |\@begintheorem| and |\@opargbegintheorem| to
%    |\relax| since these commands are no longer necessary at
%    toplevel. This will save a few tokens.
%    \begin{macrocode}
\global\let\@begintheorem\relax
\global\let\@opargbegintheorem\relax
%    \end{macrocode}
% \end{macro}
% \end{macro}
%
% \begin{macro}{\theorem@style}
%    Obviously the \textsf{token} register used above has to be
%    allocated. To assure the utmost compatibility with the
%    original \LaTeX{} definition, we set the default theorem style to
%    \texttt{plain}, which implements the usual \LaTeX{} convention.
% \changes{v2.0f}{89/07/20}{Unnecessary `global' removed.}
%    \begin{macrocode}
\newtoks\theorem@style
\global\theorem@style{plain}
%    \end{macrocode}
% \end{macro}
%
% \begin{macro}{\theorembodyfont}
% \begin{macro}{\theorem@bodyfont}
%    If the |\theorembodyfont| is set by the user then it should not
%    interact with the default font set in the theorem style. When the
%    new font selection is in force this may happen if, for example,
%    the default is |\itshape| and the new |\theorembodyfont| is
%    |\sffamily|. So we add a |\reset@font| command in front of the
%    user definition.
% \changes{v2.0f}{89/07/20}{Unnecessary `global' removed.}
% \changes{v2.0g}{89/09/19}{`theoremfont to `theorembodyfont.}
% \changes{v2.1c}{92/06/22}{`reset@font added and logic changed.}
% \changes{v2.1d}{92/10/28}{Reset to empty if arg is empty}
%    \begin{macrocode}
\gdef\theorembodyfont#1{%
%    \end{macrocode}
%    We check if the argument supplied is empty and if so put nothing
%    into the |\theorem@bodyfont| token register to allow for
%    |\theorembodyfont{}| as a mean of using the default of the
%    current |\theoremstyle|.
%    \begin{macrocode}
   \def\@tempa{#1}%
   \ifx\@tempa\@empty
    \theorem@bodyfont{}%
   \else
    \theorem@bodyfont{\reset@font#1}%
   \fi
   }
\newtoks\theorem@bodyfont
\global\theorem@bodyfont{}
%    \end{macrocode}
% \end{macro}
% \end{macro}
%
%
% \begin{macro}{\theoremheaderfont}
% \changes{v2.0g}{89/09/19}{Macro added.}
%    The font for the theorem headers is handled differently because
%    this definition applies to all theorem styles.
%    \begin{macrocode}
\gdef\theoremheaderfont#1{\gdef\theorem@headerfont{#1}%
%    \end{macrocode}
%    After using the macro once it is redefined to produce an error
%    message.
%    \begin{macrocode}
       \gdef\theoremheaderfont##1{%
        \typeout{\string\theoremheaderfont\space should be used
                 only once.}}}
%    \end{macrocode}
% \end{macro}
%
% \begin{macro}{\theorem@headerfont}
% \changes{v2.0g}{89/09/19}{Macro added.}
%    To set the |\theorem@headerfont| default we first test if
%    the new fontselection scheme is in force.
% \changes{v2.1a}{90/02/15}{Support for new font selection.}
%    \begin{macrocode}
\ifx\upshape\undefined
%    \end{macrocode}
%    If not we define it to expand into |\bfseries|. We don't use
%    |\let| just in case a following style option redefines this
%    macro.
%    \begin{macrocode}
\gdef\theorem@headerfont{\bfseries}
%    \end{macrocode}
%    Otherwise we reset the current shape before calling |\bfseries|.
%    \begin{macrocode}
\else \gdef\theorem@headerfont{\normalfont\bfseries}\fi
%    \end{macrocode}
% \end{macro}
%
% \begin{macro}{\th@plain}
% \begin{macro}{\th@break}
% \begin{macro}{\th@marginbreak}
% \begin{macro}{\th@changebreak}
% \begin{macro}{\th@change}
% \begin{macro}{\th@margin}
%    The different styles are defined in macros such as
%    |\th@plain|.  Since memory space is precious in
%    ``non-Big-versions'', we have to avoid offering too many unused
%    definitions.  Therefore we define these styles in separate files
%    that can be loaded on demand.  Thus the commands themselves only
%    load these files. We use |\@input@|, a \LaTeXe{} internal command
%    that ensures that the file will be listed with |\listfiles|.
% \changes{v2.2a}{94/02/02}{Use `@input@ to load theorem layout files}
%    \begin{macrocode}
\gdef\th@plain{\@input@{thp.sty}}
\gdef\th@break{\@input@{thb.sty}}
\gdef\th@marginbreak{\@input@{thmb.sty}}
\gdef\th@changebreak{\@input@{thcb.sty}}
\gdef\th@change{\@input@{thc.sty}}
\gdef\th@margin{\@input@{thm.sty}}
%    \end{macrocode}
%    This list will be expanded when new styles become
%    available.  For testing,  just append new theorem substyles
%    as document options.
% \end{macro}
% \end{macro}
% \end{macro}
% \end{macro}
% \end{macro}
% \end{macro}
%
%
%
% \subsection{Definition of a new theorem set}
%
% As already pointed out, a new theorem environment can be defined in
% three different ways:
% \begin{verbatim}
%    \newtheorem{Lem}{Lemma}
%    \newtheorem{Lem}{Lemma}[section]
%    \newtheorem{Lem}[Theorem]{Lemma}
%\end{verbatim}
% The function of the macro |\newtheorem| is to recognize these cases
% and then to branch into one of the three macros |\@ynthm|, |\@xnthm|
% or |\@othm|.  This mechanism is adopted unchanged from
% \cite{latex:source}; the essential point here is that, for example,
% in the second case, the arguments \texttt{Lem}, \texttt{Lemma} and
% \texttt{section} are passed over to the macro |\@xnthm|.
%
% We inspect this case first because the others present fewer problems,
% and thus are easily derived from this one.
%
% \begin{macro}{\@xnthm}
%    For our example arguments, the macro |\@xnthm| must fulfill
%    the following:
%    \begin{itemize}
%       \item
%          Define a new \LaTeX-counter `Lem'
%       \item
%          reset this counter within a |\section|
%       \item
%          define the macro |\theLem|
%       \item
%          define the environment macros |\Lem| and
%          |\endLem| using the current
%          |\theoremstyle| and |\theorem@bodyfont|.
%    \end{itemize}
%    Obviously, all this should happen only if the first argument
%    of |\@xnthm| (i.e.\ \texttt{Lem} in our example) is chosen so
%    as not to conflict with any previously defined commands or
%    environments.  This test is performed by the \LaTeX{} macro
%    |\@ifdefinable|.
%    \begin{macrocode}
\gdef\@xnthm#1#2[#3]{\expandafter\@ifdefinable\csname #1\endcsname
%    \end{macrocode}
%    Therefore, the first argument of |\@ifdefinable| is the
%    expansion (in the example, |\Lem|) of
%    |\csname||#1\endcsname|.  The second argument is
%    executed only if the test has been completed successfully.
%    \begin{macrocode}
   {%
%    \end{macrocode}
%    Now we define the new counter. The names of the \LaTeX{}
%    macros employed should speak for themselves:
% \changes{v2.2b}{1995/10/07}{Use \cs{@newctr}}
% \changes{v2.2c}{1995/11/23}{Use \cs{@newctr} properly.}
%    \begin{macrocode}
    \@definecounter{#1}\@newctr{#1}[#3]%
%    \end{macrocode}
%    Using |\@newctr| will give a proper error message if the counter
%    in |#3| is not defined.
%    In defining `|\theLem|' we must generate the desired macro
%    name by use of |\expandafter| and |\csname|.
%    \begin{macrocode}
    \expandafter\xdef\csname the#1\endcsname
%    \end{macrocode}
%    An |\xdef| is used in order to make the definition global,
%    and to ensure that it contains the replacement texts of
%    |\@thmcountersep| and |\@thmcounter|.\footnote{These
%    two macros can be defined by the document style.  Their default
%    values produce a `\texttt{.}' as separation and an arabic
%    representation of the number.} However, not everything should be
%    expanded. For example, it saves space to use |\thesection|
%    instead of its---at times---lengthy expansion.
%    \begin{macrocode}
      {\expandafter \noexpand \csname the#3\endcsname
       \@thmcountersep \@thmcounter{#1}}%
%    \end{macrocode}
%    Thus with the defaults of \LaTeX{}, |\theLem| would be
%    replaced by the command sequence |\thesection.\arabic{Lem}|.
%
%    We will now look at the definition of the macro which is executed
%    at the beginning of the actual environment (in our example this
%    macro is |\Lem|). It should be noted that we use an
%    ``|\expandafter| trick'' to expand only certain parts of the
%    replacement text at the time of the definition.
%    \begin{macrocode}
    \def\@tempa{\global\@namedef{#1}}%
    \expandafter \@tempa \expandafter{%
%    \end{macrocode}
%    First, the macro that contains the current definitions of
%    |\@begintheorem| and |\@opargtheorem| should be called
%    up.  The name of this macro---as is already known---has the form
%    |\th@|\meta{theorem\ style}; therefore, it must be called by
%    \begin{macrocode}
      \csname th@\the \theorem@style
%    \end{macrocode}
%    In addition the default theorem font should be changeable, i.e.\
%    we have to insert the contents of |\theorem@bodyfont|. For
%    that reason, we expand even further, beyond |\endcsname|,
%    and thus insert the contents of the \textsf{token} register
%    |\theorem@bodyfont| in the replacement text.
%    \begin{macrocode}
            \expandafter \endcsname \the \theorem@bodyfont
%    \end{macrocode}
%    Now it is time to call the macro |\@thm| which takes over
%    the further processing.  It has two arguments: the current
%    counter name (in our example, \texttt{Lem}), and the text of the
%    label (in our example, \texttt{Lemma}).
%    \begin{macrocode}
     \@thm{#1}{#2}}%
%    \end{macrocode}
%    With this, the `sub-definition' is complete.  The macro
%    |\@endtheorem| ends a theorem environment and is, so far, nothing
%    but an |\endtrivlist|.  (Hence it is defined globally, and not
%    within the theorem styles.\footnote{This has to be changed as
%    soon as theorem styles that change \texttt{\bslash @endtheorem}
%    exist.  In such a case, all existing styles must be changed as
%    well since they will have to reset the macro.}) Therefore, we can
%    set it equivalent to the macro that ends the theorem set (in our
%    example, |\endLem|).  However, if some day theorem styles exist
%    that do change |\@endtheorem|, we would have to use the
%    commented-out line instead.
%    \begin{macrocode}
    \global \expandafter \let \csname end#1\endcsname \@endtheorem
%   \global\@namedef{end#1}{\@endtheorem}%
%    \end{macrocode}
%    With these commands all the required definitions are employed,
%    unless the test |\@ifdefinable| has failed.  Therefore, we
%    end the second argument of this macro and with it the definition
%    of |\@xnthm|.
%    \begin{macrocode}
   }}
%    \end{macrocode}
% \end{macro}
%
% \begin{macro}{\@ynthm}
%    The definition of |\@ynthm| is completely analogous. In
%    this case the new counter that is defined is not reset within
%    another counter; thus the definition of |\the...|
%    is simplified:
%    \begin{macrocode}
\gdef\@ynthm#1#2{\expandafter\@ifdefinable\csname #1\endcsname
   {\@definecounter{#1}%
    \expandafter\xdef\csname the#1\endcsname{\@thmcounter{#1}}%
%    \end{macrocode}
%    The rest of the definition corresponds literally to that of
%    |\@xnthm|:
%    \begin{macrocode}
    \def\@tempa{\global\@namedef{#1}}\expandafter \@tempa
     \expandafter{\csname th@\the \theorem@style \expandafter
     \endcsname \the\theorem@bodyfont \@thm{#1}{#2}}%
    \global \expandafter \let \csname end#1\endcsname \@endtheorem}}
%    \end{macrocode}
% \end{macro}
%
% \begin{macro}{\@othm}
%    The definition of |\@othm| does not contain anything new.
%    \begin{macrocode}
\gdef\@othm#1[#2]#3{%
%    \end{macrocode}
% \changes{v2.2b}{95/10/07}{Build in check for counter defined}
%    We do not define a new counter but instead use one that has
%    already been defined.  Thus the only definition we need is that
%    of this pseudo-counter (i.e.\ |\the|\meta{env.\ name}).
%    First we check if |#2| corresponds to a known counter name.
%    \begin{macrocode}
  \expandafter\ifx\csname c@#2\endcsname\relax
   \@nocounterr{#2}%
  \else
   \expandafter\@ifdefinable\csname #1\endcsname
   {\expandafter \xdef \csname the#1\endcsname
     {\expandafter \noexpand \csname the#2\endcsname}%
%    \end{macrocode}
%    All other parts of the definition can be adopted from
%    |\@xnthm|. We have to remember, though, that in this case
%    the name of the current counter and the theorem label have moved
%    to the second and third arguments.
%    \begin{macrocode}
    \def\@tempa{\global\@namedef{#1}}\expandafter \@tempa
     \expandafter{\csname th@\the \theorem@style \expandafter
     \endcsname \the\theorem@bodyfont \@thm{#2}{#3}}%
    \global \expandafter \let \csname end#1\endcsname \@endtheorem}%
  \fi}
%    \end{macrocode}
% \end{macro}
%
% \subsection{Macros that are employed in a {\envfont theorem}
%    environment}
%
% \begin{macro}{\@thm}
%    The macro |\@thm| has to increase the current counter.  Then,
%    depending on whether the environment has (or does not have) an
%    optional argument, it has to branch into either |\@begintheorem|
%    or |\@opargtheorem|.
%    \begin{macrocode}
\gdef\@thm#1#2{\refstepcounter{#1}%
%    \end{macrocode}
%    Now we start a {\envfont trivlist} environment, and give
%    |\@topsep| and |\@topsepadd| the values of the \textsf{skip}
%    registers |\theorempreskipamount| and |\theorempostskipamount|.
%    The value in |\@topsep| is the vertical space that is inserted by
%    the first (and only) |\item| in our |\trivlist| whilst
%    |\@topsepadd| is inserted by |\@endparenv| at the end of that
%    {\envfont trivlist} environment.  By using these registers, we
%    obtain the desired space around a {\envfont theorem} environment.
%    \begin{macrocode}
   \trivlist
   \@topsep \theorempreskipamount               % used by first \item
   \@topsepadd \theorempostskipamount           % used by \@endparenv
%    \end{macrocode}
%    Now we have to test whether an optional argument has been given.
% \changes{v2.0d}{89/07/18}{Forgotten percent added.}
%    \begin{macrocode}
   \@ifnextchar [%
%    \end{macrocode}
%    If there is an optional argument, we will call |\@ythm|, and
%    move the arguments read back into the input stream.
%    \begin{macrocode}
   {\@ythm{#1}{#2}}%
%    \end{macrocode}
%    If not, we call |\@begintheorem|. Its first argument is the
%    name of the theorem set (hence the second argument of
%    |\@thm|). Its second argument is the macro that produces the
%    current number.
%    \begin{macrocode}
   {\@begintheorem{#2}{\csname the#1\endcsname}\ignorespaces}}
%    \end{macrocode}
% \end{macro}
%
% \begin{macro}{\@xthm}
% \begin{macro}{\@ythm}
%    Both these macros were originally called by |\@thm|.  We do
%    not need |\@xthm| anymore, hence we reset it to
%    |\relax|. The definition of |\@ythm| has not changed at
%    all from its definition in \LaTeX. In order to make the macros
%    easier to understand, we will
%    nevertheless present it (commented out).
%    \begin{macrocode}
\global\let\@xthm\relax
% \def\@ythm#1#2[#3]{\@opargbegintheorem{#2}{\csname
%           the#1\endcsname}{#3}\ignorespaces}
%</package>
%    \end{macrocode}
%    The primitive |\ignorespaces| in |\@ythm| and
%    |\@thm| is needed to remove the spaces between the
%    |\begin{...}| and the actual text.
% \end{macro}
% \end{macro}
%
%
% \subsection{Definition of the theorem substyles}
%
% As already pointed out, the theorem substyles, defined below, are
% only loaded when necessary.  Note that all these substyles, except
% \texttt{plain}, have |\slshape| as the default body font.
%
% \subsubsection{The \texttt{plain} style}
%
% \label{dok:th@plain} As the following macros use |@|, we have
% to locally set the |\catcode| of this symbol to ``letter''.
% This happens within a group, so that we do not have to worry about
% which |\catcode| that symbol had before.
%    \begin{macrocode}
%<*thp>
\begingroup \makeatletter
%    \end{macrocode}
% Since we are now within a group, we must make all definitions
% globally.  First we make sure that \texttt{theorem.sty} is loaded.
% This will allow us to use this file as a document style option
% without having to call \texttt{theorem} itself as an option.  At the
% same time, we assure that at least version 2 is loaded, since
% |\theorem@style| was not defined in earlier versions.
%    \begin{macrocode}
\@ifundefined{theorem@style}{\input{theorem.sty}}{}
\ProvidesFile{thp.sty}
             [\FMithmInfo]
%    \end{macrocode}
%
%
% \begin{macro}{\th@plain}
% \begin{macro}{\@begintheorem}
% \begin{macro}{\@opargbegintheorem}
%    |\theoremstyle{plain}| corresponds to the original
%    definition, except that the distances to the surrounding text are
%    determined by the parameters |\theorempreskipamount| and
%    |\theorempostskipamount|.
%    First we set the default body font.
%    \begin{macrocode}
\gdef\th@plain{\normalfont\itshape
%    \end{macrocode}
%    Then we define |\@begintheorem| and
%    |\@opargbegintheorem|. These two macros define how the
%    header of a theorem is typeset.  |\@opargbegintheorem| will
%    be called if a \textsf{theorem} environment with an optional
%    argument is encountered; otherwise, the header is constructed by
%    calling |\@begintheorem|.  If one of these macros is
%    executed, we are within a \textsf{trivlist} environment started by
%    |\@thm|.  So the theorem header is produced with an
%    |\item| command.
%
%    Instead of specifying the header font directly, all standard
%    theorem styles use the |\theorem@headerfont| macro to allow
%    customization.  The extra space (|\labelsep|) is necessary
%    because of problems in the \textsf{trivlist} environment.
%    \begin{macrocode}
  \def\@begintheorem##1##2{%
        \item[\hskip\labelsep \theorem@headerfont ##1\ ##2]}%
%    \end{macrocode}
%    The definition of |\@opargbegintheorem| is completely
%    analogous. The only difference is the fact that there exists a
%    third argument (which is the optional parameter of the
%    environment and contains additional information about the
%    theorem).  Customarily we enclose it in parentheses.
%    \begin{macrocode}
\def\@opargbegintheorem##1##2##3{%
   \item[\hskip\labelsep \theorem@headerfont ##1\ ##2\ (##3)]}}
%    \end{macrocode}
% \end{macro}
% \end{macro}
% \end{macro}
%
% \noindent
% We conclude with an |\endgroup| to restore the
% |\catcode| of |@|.
%    \begin{macrocode}
\endgroup
%</thp>
%    \end{macrocode}
%
%
% \subsubsection{The \texttt{break} style}
%
% This style option is stored in the file |thb.sty|.
% For the next two lines see the documentation for |\th@plain| on
% page~\pageref{dok:th@plain}.
%    \begin{macrocode}
%<*thb>
\begingroup \makeatletter
\@ifundefined{theorem@style}{\input{theorem.sty}}{}
\ProvidesFile{thb.sty}
             [\FMithmInfo]
%    \end{macrocode}
%
% \begin{macro}{\th@break}
%    |\theoremstyle{break}| produces a line break after the name
%    of the theorem.  The font is |\slshape|.  Hence, we define
%    |\th@break| as follows:
%    \begin{macrocode}
\gdef\th@break{\normalfont\slshape
  \def\@begintheorem##1##2{\item[%
%    \end{macrocode}
%    We run into the following problem: it is not possible to create
%    the header with |\item[|\meta{title}|]| and then start
%    a new line by, for example, |\mbox{}\\|. Such a definition
%    will fail whenever a list environment follows immediately.  With
%    the above construction, the |\mbox{}| causes the switch
%    |@inlabel| (cf.\ definition of |\list| and
%    |\trivlist| in \cite{latex:source}) to be set to \texttt{false}
%    and so the following list will insert additional vertical
%    space (|\topskip|). This is quite annoying. Therefore, we
%    create the line break within the |\item|. In order to ensure
%    that the text will begin at the proper position in the following
%    line, we simply pretend that the label does not take any
%    room.\footnote{This will lead to problems whenever very high
%    symbols occurring in the line tower into the
%    heading. So, something else has to be done here sometime.}
%    \begin{macrocode}
       \rlap{\vbox{\hbox{\hskip \labelsep\theorem@headerfont ##1\ ##2}%
                   \hbox{\strut}}}]}%
%    \end{macrocode}
%    Again, the definition of |\@opargbegintheorem| is
%    completely analogous.
%    \begin{macrocode}
\def\@opargbegintheorem##1##2##3{%
  \item[\rlap{\vbox{\hbox{\hskip \labelsep \theorem@headerfont
                     ##1\ ##2\ (##3)}%
                    \hbox{\strut}}}]}}
\endgroup
%</thb>
%    \end{macrocode}
% \end{macro}
%
%
% \subsubsection{The \texttt{changebreak} style}
%
%    \begin{macrocode}
% This style option is stored in the file |thcb.sty|.
%    \begin{macrocode}
%<*thcb>
\begingroup \makeatletter
\@ifundefined{theorem@style}{\input{theorem.sty}}{}
\ProvidesFile{thcb.sty}
             [\FMithmInfo]
%    \end{macrocode}
%
% \begin{macro}{\th@changebreak}
%    The \texttt{change-break} theorem style is like \texttt{break}
%    but with interchange of theorem name and theorem number.  Thus we
%    define |\th@changebreak| as follows:
%    \begin{macrocode}
\gdef\th@changebreak{\normalfont\slshape
  \def\@begintheorem##1##2{\item
       [\rlap{\vbox{\hbox{\hskip\labelsep \theorem@headerfont ##2\ ##1}%
                    \hbox{\strut}}}]}%
\def\@opargbegintheorem##1##2##3{%
  \item[\rlap{\vbox{\hbox{\hskip\labelsep \theorem@headerfont
                     ##2\ ##1\ (##3)}%
                    \hbox{\strut}}}]}}
\endgroup
%</thcb>
%    \end{macrocode}
% \end{macro}
%
%
% \subsubsection{The \texttt{change} style}
%
% This style option is stored in the file |thc.sty|.
%    \begin{macrocode}
%<*thc>
\begingroup \makeatletter
\@ifundefined{theorem@style}{\input{theorem.sty}}{}
\ProvidesFile{thc.sty}
             [\FMithmInfo]
%    \end{macrocode}
%
% \begin{macro}{\th@change}
%    The \texttt{change} theorem style corresponds to the
%    \texttt{change break} style without a linebreak after the header.
%    To say it in another way, it's the same as the \texttt{plain}
%    style but with number and name interchanged and |\slshape| as the
%    default font.
%    \begin{macrocode}
\gdef\th@change{\normalfont\slshape
  \def\@begintheorem##1##2{\item
       [\hskip\labelsep \theorem@headerfont ##2\ ##1]}%
\def\@opargbegintheorem##1##2##3{%
  \item[\hskip\labelsep \theorem@headerfont ##2\ ##1\ (##3)]}}
\endgroup
%</thc>
%    \end{macrocode}
% \end{macro}
%
%
% \subsubsection{The \texttt{marginbreak} style}
%
% This style option is the one used most often at Mainz.  It is saved
% in the file |thmb.sty|.
%    \begin{macrocode}
%<*thmb>
\begingroup \makeatletter
\@ifundefined{theorem@style}{\input{theorem.sty}}{}
\ProvidesFile{thmb.sty}
             [\FMithmInfo]
%    \end{macrocode}
%
% \begin{macro}{\th@marginbreak}
%    The \texttt{margin break} style is nearly the same as the
%    \texttt{change break} style. The only difference is the placement
%    of the theorem number.  We use |\llap| to place it in the left
%    margin.
%
%    In this style |\labelsep| denotes the separation between the
%    number and the text.
%    \begin{macrocode}
\gdef\th@marginbreak{\normalfont\slshape
  \def\@begintheorem##1##2{\item
       [\rlap{\vbox{\theorem@headerfont
                    \hbox{\llap{##2}\hskip\labelsep ##1}%
                    \hbox{\strut}}}]}%
\def\@opargbegintheorem##1##2##3{%
  \item[\rlap{\vbox{\theorem@headerfont
                    \hbox{\llap{##2}\hskip\labelsep ##1\ (##3)}%
                    \hbox{\strut}}}]}}
\endgroup
%</thmb>
%    \end{macrocode}
% \changes{v2.1c}{90/02/17}{`endgroup outside macrocode env.}
% \end{macro}
%
%
%
% \subsubsection{The \texttt{margin} style}
%
% This style option is stored in the file |thm.sty|.
%    \begin{macrocode}
%<*thm>
\begingroup \makeatletter
\@ifundefined{theorem@style}{\input{theorem.sty}}{}
\ProvidesFile{thm.sty}
             [\FMithmInfo]
%    \end{macrocode}
%
% \begin{macro}{\th@margin}
%    Again this is only a variant of the theorem styles described
%    above without any new ideas.
%    \begin{macrocode}
\gdef\th@margin{\normalfont\slshape
  \def\@begintheorem##1##2{\item
       [\theorem@headerfont \llap{##2}\hskip\labelsep ##1]}%
\def\@opargbegintheorem##1##2##3{%
  \item[\theorem@headerfont \llap{##2}\hskip\labelsep ##1\ (##3)]}}
\endgroup
%</thm>
%    \end{macrocode}
% \end{macro}
%
%
%
%
% \subsection{Final Definitions}
%
% \begin{macro}{\theorempreskipamount}
% \begin{macro}{\theorempostskipamount}
%    The \textsf{skip} parameters that regulate the vertical empty space
%    before and after the {\envfont theorem} environment have
%    to be allocated as well.
% \changes{v2.0f}{89/07/20}{Unnecessary `global' removed.}
%    \begin{macrocode}
%<*package>
\newskip\theorempreskipamount
\newskip\theorempostskipamount
%    \end{macrocode}
%    Since we have used the same values for all theorem sets, we now can
%    assign them.
%    \begin{macrocode}
\global\setlength\theorempreskipamount{12pt plus 5pt minus 3pt}
\global\setlength\theorempostskipamount{8pt plus 3pt minus 1.5pt}
%    \end{macrocode}
% \end{macro}
% \end{macro}
%
%
% \begin{macro}{\@endtheorem}
%    The same holds for the macro |\@endtheorem|, which ends a
%    {\envfont theorem} environment. Since it is the same for all
%    theorem sets, it is removed from the macros |\th@|\meta{style}.
%    It simply ends the {\envfont trivlist} environment, which was
%    begun in |\@thm|.
%    \begin{macrocode}
\global\let\@endtheorem=\endtrivlist
%    \end{macrocode}
% \end{macro}
%
%
%
% \begin{macro}{\@preamblecmds}
%    All macros defined above are to be used only in the preamble.
%    Therefore, we insert them in |\@preamblecmds| which will disable
%    them at begin document. This is done by the internal \LaTeXe{}
%    command |\@onlypreamble|.
% \changes{v2.0g}{89/09/19}{`theoremheaderfont and `theorembodyfont
%    added.}
% \changes{v2.2a}{94/02/03}{Use \LaTeXe{} interface}
%    \begin{macrocode}
\@onlypreamble\@xnthm
\@onlypreamble\@ynthm
\@onlypreamble\@othm
\@onlypreamble\newtheorem
\@onlypreamble\theoremstyle
\@onlypreamble\theorembodyfont
\@onlypreamble\theoremheaderfont
%    \end{macrocode}
% \end{macro}
%
% Finally we declare the \texttt{plain} theorem style to be the default.
% \changes{v2.1b}{90/06/04}{Fixed missing `item by declaring default
%    style.}
%    \begin{macrocode}
\theoremstyle{plain}
%</package>
%    \end{macrocode}
%
% \Finale

\endinput
}{}
\ProvidesFile{thmb.sty}
             [\FMithmInfo]
%    \end{macrocode}
%
% \begin{macro}{\th@marginbreak}
%    The \texttt{margin break} style is nearly the same as the
%    \texttt{change break} style. The only difference is the placement
%    of the theorem number.  We use |\llap| to place it in the left
%    margin.
%
%    In this style |\labelsep| denotes the separation between the
%    number and the text.
%    \begin{macrocode}
\gdef\th@marginbreak{\normalfont\slshape
  \def\@begintheorem##1##2{\item
       [\rlap{\vbox{\theorem@headerfont
                    \hbox{\llap{##2}\hskip\labelsep ##1}%
                    \hbox{\strut}}}]}%
\def\@opargbegintheorem##1##2##3{%
  \item[\rlap{\vbox{\theorem@headerfont
                    \hbox{\llap{##2}\hskip\labelsep ##1\ (##3)}%
                    \hbox{\strut}}}]}}
\endgroup
%</thmb>
%    \end{macrocode}
% \changes{v2.1c}{90/02/17}{`endgroup outside macrocode env.}
% \end{macro}
%
%
%
% \subsubsection{The \texttt{margin} style}
%
% This style option is stored in the file |thm.sty|.
%    \begin{macrocode}
%<*thm>
\begingroup \makeatletter
\@ifundefined{theorem@style}{% \iffalse meta-comment
%
% Copyright (C) 1993-2024
%
% The LaTeX Project and any individual authors listed elsewhere
% in this file.
%
% This file is part of the Standard LaTeX `Tools Bundle'.
% -------------------------------------------------------
%
% It may be distributed and/or modified under the
% conditions of the LaTeX Project Public License, either version 1.3c
% of this license or (at your option) any later version.
% The latest version of this license is in
%    https://www.latex-project.org/lppl.txt
% and version 1.3c or later is part of all distributions of LaTeX
% version 2005/12/01 or later.
%
% The list of all files belonging to the LaTeX `Tools Bundle' is
% given in the file `manifest.txt'.
%
% \fi
%
%%
%
%
\def\FMithmInfo{2023/07/05 v2.2c Theorem extension package (FMi)}
%
% \ProvidesFile{theorem.dtx}[\FMithmInfo]
%\iffalse   % this is a METACOMMENT !
%%
%% Package `theorem' to use with LaTeX2e
%% Copyright (C) 1989-1995 Frank Mittelbach, all rights reserved.
%<+package>\@ifundefined{theorem@style}{}{\endinput}
%<+package>\NeedsTeXFormat{LaTeX2e}[1995/06/01]
%<+package>\ProvidesPackage{theorem}
%<+package>         [\FMithmInfo]
%
% \fi
%
% \changes{v1.0}{87/05/07}{First draft.}
% \changes{v1.0a}{87/05/07}{`thistheoremstyle' added.}
% \changes{v1.1a}{87/07/29}{New concept: redefine the internal
%                          \protect\LaTeX{} commands `@xnthm' etc.}
% \changes{v1.1b}{87/07/31}{`theoremfont' used to specify font.}
% \changes{v1.2a}{88/01/08}{`@endtheorem' now globally defined.}
% \changes{v1.2b}{88/01/11}{`newtheorem' only allowed in the preamble.}
% \changes{v1.3a}{88/02/14}{`theoremfont' now a toks register.}
%
% \changes{v2.0a}{88/11/16}{New implementation and new documentation.}
% \changes{v2.0a}{88/11/16}{No layout changes but internal modifications
%                           old substyles won't work.}
% \changes{v2.0b}{89/01/30}{`global' used for all definitions.}
% \changes{v2.0c}{89/06/15}{Documentation translated to English.
%                            Thanks to LLL and Nina}
% \changes{v2.1a}{90/02/15}{Support for new font selection.}
% \changes{v2.2a}{94/02/03}{Cleanup for LaTeX2e}
%
% \DoNotIndex{\@endparenv,\@ifdefinable,\@ifnextchar,\@ifundefined}
% \DoNotIndex{\@namedef,\@nameuse,\@spaces,\@tempa,\@warning}
% \DoNotIndex{\ }
% \DoNotIndex{\begingroup,\bfseries}
% \DoNotIndex{\csname}
% \DoNotIndex{\def,\do,\docdate}
% \DoNotIndex{\endcsname,\endgroup,\endinput,\endtrivlist,\expandafter}
% \DoNotIndex{\filedate,\fileversion}
% \DoNotIndex{\gdef,\global}
% \DoNotIndex{\hbox,\hskip}
% \DoNotIndex{\ignorespaces,\input,\itshape,\item}
% \DoNotIndex{\labelsep,\let,\llap}
% \DoNotIndex{\makeatletter}
% \DoNotIndex{\noexpand}
% \DoNotIndex{\refstepcounter,\relax,\rlap}
% \DoNotIndex{\setlength,\slshape,\space,\string,\strut}
% \DoNotIndex{\the,\trivlist,\typeout}
% \DoNotIndex{\vbox}
% \DoNotIndex{\xdef}
%
% \def\deflabel#1{\ttfamily #1\hfill}
% \def\deflist#1{\begin{list}{}{\settowidth\labelwidth{\ttfamily #1}%
%                               \setlength\leftmargin\labelwidth
%                               \addtolength\leftmargin\labelsep
%                               \let\makelabel\deflabel}}
% \def\enddeflist{\end{list}}
% \def\envfont{\normalfont\ttfamily}
%
% \MakeShortVerb{\|}
%
% \newcommand{\TUB}{{\sl TUGboat\/}}
%
% \GetFileInfo{theorem.dtx}
% \title{An Extension of the
%        \LaTeX{} theorem environment\thanks{This file
%        has version number \fileversion, last
%        revised \filedate.}}
% \author{Frank Mittelbach}
%
% \MaintainedByLaTeXTeam{tools}
% \maketitle
%
% \begin{abstract}
%    The macros described in this paper yield an extension of the
%    \LaTeX{} theorem mechanism. It is designed to satisfy the
%    different requirements of various journals. Thus, the
%    layout of the ``theorems'' can be manipulated by determining a
%    ``style''. This article describes not only the use, but
%    also the definition, of the necessary macros.
% \end{abstract}
%
% \section*{Preface to version 2.2}
%
% For \LaTeXe{} this package did not need any fundamental changes. I
% only modified the messages generated so that theorem layout styles
% will show up with the |\listfiles| command and cleaned the section
% on the New Font Selection Scheme since this is now included in
% \LaTeX.
%
% \section*{Preface to version 2.1}
%
% This version is identical to 2.0g described in \TUB\ 10\#3 except
% for some internal defaults which are now set depending on the used
% font selection scheme.
%
% This was done to avoid unpleasant surprises if the new font selection
% scheme is in force. For further details see
% section~\ref{sec:newfonts} and \cite{bk:GMS94}.
%
% \section{Introduction}
%
% For our purposes here, ``theorems'' are labelled enunciations,
% often set off from the main text by extra space and a font change.
% Theorems, corollaries, conjectures, definitions, and remarks
% are all instances of ``theorems''.  The ``header'' of these structures
% is composed of a label (such as \textsc{Theorem} or \textsc{Remark})
% and a number which serializes an item in the sequence of items
% with the same label.
%
% Shortly after the introduction of \LaTeX{} at the Fachbereich
% Mathematik in Mainz, the desire to manipulate the layout of
% ``theorems'' arose.  In Mainz, the following two conventions came
% into general use:
% \begin{enumerate}
%   \item The number of the theorem is shown in the margin.
%   \item There is a line break at
%         the end of the theorem header.
% \end{enumerate}
%
% Additionally, some journals require different formats which depend on
% the ``sort of theorem'': e.g.\ often remarks and definitions are set
% in |\upshape|, while |\itshape| is employed for main theorems.
%
% Confronted with these requirements, a theorem environment was
% developed in Mainz which allows separate determination of the
% layout of the ``theorems sets'', comparable to |\pagestyle|.
%
%
% \section{The user interface}
% \subsection{Defining new theorem sets}
%
% \DescribeMacro\newtheorem
% As in the original \LaTeX{} version, the command |\newtheorem|
% defines a new ``theorem set'' or ``theorem-like structure''.
% Two required arguments name the new environment and give the text
% to be typeset with each instance of the new ``set'', while
% an optional argument determines how
% the ``set'' is enumerated:
% \begin{description}
%    \item[\ttfamily \bslash newtheorem\{foo\}\{bar\}]
%       The theorem set {\envfont foo} (whose name is \texttt{bar})
%       uses its own counter.
%    \item[\ttfamily \bslash newtheorem\{foo2\}{[foo]}\{bar2\}]
%       The theorem set {\envfont foo2} (printed name \texttt{bar2})
%       uses the same counter as the theorem set \texttt{foo}.
%    \item[\ttfamily \bslash newtheorem\{foo3\}\{bar3\}{[section]}]
%       The theorem set {\envfont foo3} (printed name \texttt{bar3}) is
%       enumerated within the counter \texttt{section}, i.e.\ with every
%       new |\section| the enumeration begins again with 1, and
%       the enumeration is composed from the section-number and the
%       theorem counter itself.
% \end{description}
%
% \DescribeMacro\theoremstyle
% Additionally, the command |\theoremstyle| can define the layout
% of various, or all, theorem sets.  It should be noted that any
% theorem set defined by |\newtheorem| is typeset in the
% |\theoremstyle| that is current at the time of the definition.
% Thus, the following
% \begin{verbatim}
%    \theoremstyle{break}        \newtheorem{Cor}{Corollary}
%    \theoremstyle{plain}        \newtheorem{Exa}{Example}[section]
%\end{verbatim}                             ^^A No blank at line start
% leads to the result that the set {\envfont Cor} is formatted in the
% style \texttt{break}, while the set {\envfont Exa} and all the
% following ones are formatted in the style \texttt{plain}, unless
% another |\theoremstyle| follows.  Since the definitions installed by
% |\newtheorem| are global, one also can limit |\theoremstyle| locally
% by grouping braces.
%
% \DescribeMacro\theorembodyfont
% The choice of the font for the theorem body is completely
% independent of the chosen |\theoremstyle|; this has proven to
% be very advantageous.  For example,
% \begin{verbatim}
%   {\theorembodyfont{\upshape}          \newtheorem{Rem}{Remark}}
%\end{verbatim}
% defines a theorem set {\envfont Rem}, which will be set in
% |\upshape| in the current layout (which in our example is
% \texttt{plain}).  As with |\theoremstyle|, the |\theorembodyfont|
% chosen is that current at the time of |\newtheorem|.  If
% |\theorembodyfont| is not specified or one defines
% |\theorembodyfont{}|, then the font used will be that defined by the
% |\theoremstyle|.
%
% \DescribeMacro\theoremheaderfont It is also possible to customize
% the font used for the theorem headers.  This is, however, a global
% declaration, and therefore there should be at most one
% |\theoremheaderfont| declaration in the preamble.\footnote{If it is
% actually necessary to have different header fonts, one has to define
% new theorem styles (substituting the desired font) or specify the
% information directly in the \texttt{\bslash newtheorem} declaration
% (the unclean variant).}
%
% \DescribeMacro\theorempreskipamount
% \DescribeMacro\theorempostskipamount Two additional parameters
% affect the vertical space around the theorem environments:
% |\theorempreskipamount| and |\theorempostskipamount| define,
% respectively, the spacing before and after such an environment.
% These parameters apply for all theorem sets and can be manipulated
% with the ordinary length macros.  They are rubber lengths
% (`\textsf{skips}'), and therefore can contain \texttt{plus} and
% \texttt{minus} parts.
%
% Since the definition of theorem sets should---most sensibly---be
% placed in the preamble, we only allow installation there.
% It is therefore possible to release the memory used here after
% |\begin{document}|, in order to make room for other
% applications.
%
%
% \subsection{Existing theorem styles}
%
% The following theorem styles exist to date:
% \begin{deflist}{marginbreak}
%    \item[plain]
%       This theorem style emulates the original \LaTeX{} definition,
%       except that additionally the parameters
%       |\theorem...skipamount| are used.
%    \item[break]
%       In this style, the theorem header is followed by a line
%       break.
%    \item[marginbreak]
%       The theorem number is set in the margin, and there is a line
%       break as in \texttt{break}.
%    \item[changebreak]
%       Like \texttt{break}, but with header number and text
%       interchanged.
%    \item[change]
%       Header number and text are interchanged, without a line break.
%    \item[margin]
%       The number is set in the left margin, without a line break.
% \end{deflist}
% All styles (except \texttt{plain}) select |\slshape| as the default
% |\theorembodyfont|.
%
%
% \subsection{Examples}
%
% Given the above theorem sets {\envfont Cor}, {\envfont Exa}
% and {\envfont Rem},
% suppose that the preamble also contains the declarations:
% \begin{verbatim}
%    \theoremstyle{marginbreak}   \newtheorem{Lem}[Cor]{Lemma}
%    \theoremstyle{change}
%    \theorembodyfont{\itshape}        \newtheorem{Def}[Cor]{Definition}
%
%    \theoremheaderfont{\scshape}
%\end{verbatim}
% Then the following are some typical examples of the typeset output
% resulting from their use.
%
% \begin{Cor}
%    This is a sentence typeset in the theorem environment {\envfont
%    Cor}.
% \end{Cor}
%
% \begin{Exa}
%    This is a sentence typeset in the theorem environment {\envfont
%    Exa}.
% \end{Exa}
%
% \begin{Rem}
%    This is a sentence typeset in the theorem environment {\envfont
%    Rem}.
% \end{Rem}
%
% \begin{Lem}[Ben User]
%    This is a sentence typeset in the theorem environment {\envfont
%    Lem}.
% \end{Lem}
%
% \begin{Def}[Very impressive Definition]
%    This is a sentence typeset in the theorem environment {\envfont
%    Def}.
% \end{Def}
%
% The last two examples show the effect of the optional argument to
% a theorem environment (it is the text typeset in parentheses).
%
%
% \section{Special Considerations}
% \label{sec:newfonts}
%
% Theoremheader and body are implemented as a unit. This means that
% the |\theoremheaderfont|  will inherit characteristics of the
% |\theorembodyfont| in \LaTeXe. Thus, if for example
% |\theorembodyfont| is |\itshape| and |\theoremheaderfont|
% is |\bfseries| the font selected for the header will have the
% characteristics `bold extended italic'. If this is not desired one
% should set the |\theoremheaderfont| to something like
% \begin{verbatim}
%      \theoremheaderfont{\normalfont\bfseries}
%\end{verbatim}
% i.e.\ supplying all necessary font information explicitly.
%
%
%
% \section{Acknowledgements}
%
% The publication of this set of macros was only possible with the
% help of Christina Busse (translating the manuscript into English),
% Joachim Pense (playing the r\^ole of typist), Chris Rowley
% (looking everything over) and many others providing useful
% suggestions.
%
% \MaybeStop{%
%    \begin{thebibliography}{9}
%       \bibitem{bk:GMS94} \textsc{M.~Goossens}, \textsc{F.~Mittelbach}
%         and \textsc{A.~Samarin}.
%         \newblock The \LaTeX{} Companion.
%         \newblock
%         Addison-Wesley, Reading, Massachusetts, 1994.
%       \bibitem{latex:source} \textsc{Lamport, Leslie.} \newblock
%                               \texttt{latex.tex}, version 2.09, date
%                               Feb. 1990.
%    \end{thebibliography}
%    \PrintIndex
%    \PrintChanges}
%
%
% \section{The documentation driver file}
%
% The next bit of code contains the documentation driver file for
% \TeX{}, i.e., the file that will produce the documentation you are
% currently reading. It will be extracted from this file by the
% \texttt{docstrip} program. Since it is the first code in the file
% one can alternatively process this file directly with \LaTeXe{} to
% obtain the documentation.
%
%    \begin{macrocode}
%<*driver>
\documentclass{ltxdoc}

\usepackage{theorem}

%<+driver>% The next few lines define theorem sets which are used
%<+driver>% in the example section of the documentation.

\theoremstyle{break}       \newtheorem{Cor}{Corollary}
\theoremstyle{plain}       \newtheorem{Exa}{Example}[section]
{\theorembodyfont{\upshape}\newtheorem{Rem}{Remark}}
\theoremstyle{marginbreak} \newtheorem{Lem}[Cor]{Lemma}
\theoremstyle{change}
\theorembodyfont{\itshape} \newtheorem{Def}[Cor]{Definition}

\theoremheaderfont{\scshape}

\RecordChanges

\begin{document}
   \DocInput{theorem.dtx}
\end{document}
%</driver>
%    \end{macrocode}
%
% \section{Definition of the Macros}
%
% If the file has been loaded before, we abort immediately. If not the
% package announces itself (this is actually done at the very top of
% the file---the way it is done isn't good style so don't copy it).
% \changes{v2.0e}{89/07/19}{Spaces removed from `typeout'.}
% \changes{v2.2b}{95/11/19}{Announce moved to top of file}
%    \begin{macrocode}
%<*package>
%\@ifundefined{theorem@style}{}{\endinput}
%\def\FMithmInfo{1995/11/19 v2.2b Theorem extension package (FMi)}
%\ProvidesPackage{theorem}[\FMithmInfo]
%    \end{macrocode}
%
%
% \subsection{Definition of theorem styles and fonts}
%
% All the definitions in this file are done globally to allow
% inputting this file inside a group.
%
% \begin{macro}{\theoremstyle}
%    Before a theorem style can be installed, the chosen style must be
%    known. For that reason, we must test to see that
%    |\th@|\meta{style} is known or, more precisely, that it is
%    different from |\relax|. If the style is not known then
%    |\th@plain| is used.
%    \begin{macrocode}
\gdef\theoremstyle#1{%
   \@ifundefined{th@#1}{\@warning
          {Unknown theoremstyle `#1'. Using `plain'}%
          \theorem@style{plain}}%
%    \end{macrocode}
%    We save the theorem style to be used in the token register
%    |\theorem@style|.
%    \begin{macrocode}
      {\theorem@style{#1}}%
%    \end{macrocode}
%    Now we ``evaluate'' the theorem style: this means, we call the
%    macro |\th@|\meta{style} which will activate the relevant
%    definitions which are contained in a separate file.  This is done
%    in a group to suppress changes to the current font.  This could
%    otherwise pose problems together with the new font selection
%    scheme\footnote{When I printed the original article using the new
%    font selection scheme I ended with a document with slanted
%    typefaces (text headings and all) simply because one of the
%    theorem styles used \texttt{\bslash sl} at toplevel.} if the
%    |\th@|\meta{style} is evaluated a second time.
% \changes{v2.1a}{90/02/15}{Group added around `th@..}
% \changes{v2.1a}{90/02/15}{Reseting of `..@begintheorem removed.}
%    \begin{macrocode}
      \begingroup
        \csname th@\the\theorem@style \endcsname
      \endgroup}
%    \end{macrocode}
% \end{macro}
%
%
% \begin{macro}{\@begintheorem}
% \changes{v2.1a}{90/02/15}{Relaxed at toplevel.}
% \begin{macro}{\@opargbegintheorem}
% \changes{v2.1a}{90/02/15}{Relaxed at toplevel.}
%    We reset |\@begintheorem| and |\@opargbegintheorem| to
%    |\relax| since these commands are no longer necessary at
%    toplevel. This will save a few tokens.
%    \begin{macrocode}
\global\let\@begintheorem\relax
\global\let\@opargbegintheorem\relax
%    \end{macrocode}
% \end{macro}
% \end{macro}
%
% \begin{macro}{\theorem@style}
%    Obviously the \textsf{token} register used above has to be
%    allocated. To assure the utmost compatibility with the
%    original \LaTeX{} definition, we set the default theorem style to
%    \texttt{plain}, which implements the usual \LaTeX{} convention.
% \changes{v2.0f}{89/07/20}{Unnecessary `global' removed.}
%    \begin{macrocode}
\newtoks\theorem@style
\global\theorem@style{plain}
%    \end{macrocode}
% \end{macro}
%
% \begin{macro}{\theorembodyfont}
% \begin{macro}{\theorem@bodyfont}
%    If the |\theorembodyfont| is set by the user then it should not
%    interact with the default font set in the theorem style. When the
%    new font selection is in force this may happen if, for example,
%    the default is |\itshape| and the new |\theorembodyfont| is
%    |\sffamily|. So we add a |\reset@font| command in front of the
%    user definition.
% \changes{v2.0f}{89/07/20}{Unnecessary `global' removed.}
% \changes{v2.0g}{89/09/19}{`theoremfont to `theorembodyfont.}
% \changes{v2.1c}{92/06/22}{`reset@font added and logic changed.}
% \changes{v2.1d}{92/10/28}{Reset to empty if arg is empty}
%    \begin{macrocode}
\gdef\theorembodyfont#1{%
%    \end{macrocode}
%    We check if the argument supplied is empty and if so put nothing
%    into the |\theorem@bodyfont| token register to allow for
%    |\theorembodyfont{}| as a mean of using the default of the
%    current |\theoremstyle|.
%    \begin{macrocode}
   \def\@tempa{#1}%
   \ifx\@tempa\@empty
    \theorem@bodyfont{}%
   \else
    \theorem@bodyfont{\reset@font#1}%
   \fi
   }
\newtoks\theorem@bodyfont
\global\theorem@bodyfont{}
%    \end{macrocode}
% \end{macro}
% \end{macro}
%
%
% \begin{macro}{\theoremheaderfont}
% \changes{v2.0g}{89/09/19}{Macro added.}
%    The font for the theorem headers is handled differently because
%    this definition applies to all theorem styles.
%    \begin{macrocode}
\gdef\theoremheaderfont#1{\gdef\theorem@headerfont{#1}%
%    \end{macrocode}
%    After using the macro once it is redefined to produce an error
%    message.
%    \begin{macrocode}
       \gdef\theoremheaderfont##1{%
        \typeout{\string\theoremheaderfont\space should be used
                 only once.}}}
%    \end{macrocode}
% \end{macro}
%
% \begin{macro}{\theorem@headerfont}
% \changes{v2.0g}{89/09/19}{Macro added.}
%    To set the |\theorem@headerfont| default we first test if
%    the new fontselection scheme is in force.
% \changes{v2.1a}{90/02/15}{Support for new font selection.}
%    \begin{macrocode}
\ifx\upshape\undefined
%    \end{macrocode}
%    If not we define it to expand into |\bfseries|. We don't use
%    |\let| just in case a following style option redefines this
%    macro.
%    \begin{macrocode}
\gdef\theorem@headerfont{\bfseries}
%    \end{macrocode}
%    Otherwise we reset the current shape before calling |\bfseries|.
%    \begin{macrocode}
\else \gdef\theorem@headerfont{\normalfont\bfseries}\fi
%    \end{macrocode}
% \end{macro}
%
% \begin{macro}{\th@plain}
% \begin{macro}{\th@break}
% \begin{macro}{\th@marginbreak}
% \begin{macro}{\th@changebreak}
% \begin{macro}{\th@change}
% \begin{macro}{\th@margin}
%    The different styles are defined in macros such as
%    |\th@plain|.  Since memory space is precious in
%    ``non-Big-versions'', we have to avoid offering too many unused
%    definitions.  Therefore we define these styles in separate files
%    that can be loaded on demand.  Thus the commands themselves only
%    load these files. We use |\@input@|, a \LaTeXe{} internal command
%    that ensures that the file will be listed with |\listfiles|.
% \changes{v2.2a}{94/02/02}{Use `@input@ to load theorem layout files}
%    \begin{macrocode}
\gdef\th@plain{\@input@{thp.sty}}
\gdef\th@break{\@input@{thb.sty}}
\gdef\th@marginbreak{\@input@{thmb.sty}}
\gdef\th@changebreak{\@input@{thcb.sty}}
\gdef\th@change{\@input@{thc.sty}}
\gdef\th@margin{\@input@{thm.sty}}
%    \end{macrocode}
%    This list will be expanded when new styles become
%    available.  For testing,  just append new theorem substyles
%    as document options.
% \end{macro}
% \end{macro}
% \end{macro}
% \end{macro}
% \end{macro}
% \end{macro}
%
%
%
% \subsection{Definition of a new theorem set}
%
% As already pointed out, a new theorem environment can be defined in
% three different ways:
% \begin{verbatim}
%    \newtheorem{Lem}{Lemma}
%    \newtheorem{Lem}{Lemma}[section]
%    \newtheorem{Lem}[Theorem]{Lemma}
%\end{verbatim}
% The function of the macro |\newtheorem| is to recognize these cases
% and then to branch into one of the three macros |\@ynthm|, |\@xnthm|
% or |\@othm|.  This mechanism is adopted unchanged from
% \cite{latex:source}; the essential point here is that, for example,
% in the second case, the arguments \texttt{Lem}, \texttt{Lemma} and
% \texttt{section} are passed over to the macro |\@xnthm|.
%
% We inspect this case first because the others present fewer problems,
% and thus are easily derived from this one.
%
% \begin{macro}{\@xnthm}
%    For our example arguments, the macro |\@xnthm| must fulfill
%    the following:
%    \begin{itemize}
%       \item
%          Define a new \LaTeX-counter `Lem'
%       \item
%          reset this counter within a |\section|
%       \item
%          define the macro |\theLem|
%       \item
%          define the environment macros |\Lem| and
%          |\endLem| using the current
%          |\theoremstyle| and |\theorem@bodyfont|.
%    \end{itemize}
%    Obviously, all this should happen only if the first argument
%    of |\@xnthm| (i.e.\ \texttt{Lem} in our example) is chosen so
%    as not to conflict with any previously defined commands or
%    environments.  This test is performed by the \LaTeX{} macro
%    |\@ifdefinable|.
%    \begin{macrocode}
\gdef\@xnthm#1#2[#3]{\expandafter\@ifdefinable\csname #1\endcsname
%    \end{macrocode}
%    Therefore, the first argument of |\@ifdefinable| is the
%    expansion (in the example, |\Lem|) of
%    |\csname||#1\endcsname|.  The second argument is
%    executed only if the test has been completed successfully.
%    \begin{macrocode}
   {%
%    \end{macrocode}
%    Now we define the new counter. The names of the \LaTeX{}
%    macros employed should speak for themselves:
% \changes{v2.2b}{1995/10/07}{Use \cs{@newctr}}
% \changes{v2.2c}{1995/11/23}{Use \cs{@newctr} properly.}
%    \begin{macrocode}
    \@definecounter{#1}\@newctr{#1}[#3]%
%    \end{macrocode}
%    Using |\@newctr| will give a proper error message if the counter
%    in |#3| is not defined.
%    In defining `|\theLem|' we must generate the desired macro
%    name by use of |\expandafter| and |\csname|.
%    \begin{macrocode}
    \expandafter\xdef\csname the#1\endcsname
%    \end{macrocode}
%    An |\xdef| is used in order to make the definition global,
%    and to ensure that it contains the replacement texts of
%    |\@thmcountersep| and |\@thmcounter|.\footnote{These
%    two macros can be defined by the document style.  Their default
%    values produce a `\texttt{.}' as separation and an arabic
%    representation of the number.} However, not everything should be
%    expanded. For example, it saves space to use |\thesection|
%    instead of its---at times---lengthy expansion.
%    \begin{macrocode}
      {\expandafter \noexpand \csname the#3\endcsname
       \@thmcountersep \@thmcounter{#1}}%
%    \end{macrocode}
%    Thus with the defaults of \LaTeX{}, |\theLem| would be
%    replaced by the command sequence |\thesection.\arabic{Lem}|.
%
%    We will now look at the definition of the macro which is executed
%    at the beginning of the actual environment (in our example this
%    macro is |\Lem|). It should be noted that we use an
%    ``|\expandafter| trick'' to expand only certain parts of the
%    replacement text at the time of the definition.
%    \begin{macrocode}
    \def\@tempa{\global\@namedef{#1}}%
    \expandafter \@tempa \expandafter{%
%    \end{macrocode}
%    First, the macro that contains the current definitions of
%    |\@begintheorem| and |\@opargtheorem| should be called
%    up.  The name of this macro---as is already known---has the form
%    |\th@|\meta{theorem\ style}; therefore, it must be called by
%    \begin{macrocode}
      \csname th@\the \theorem@style
%    \end{macrocode}
%    In addition the default theorem font should be changeable, i.e.\
%    we have to insert the contents of |\theorem@bodyfont|. For
%    that reason, we expand even further, beyond |\endcsname|,
%    and thus insert the contents of the \textsf{token} register
%    |\theorem@bodyfont| in the replacement text.
%    \begin{macrocode}
            \expandafter \endcsname \the \theorem@bodyfont
%    \end{macrocode}
%    Now it is time to call the macro |\@thm| which takes over
%    the further processing.  It has two arguments: the current
%    counter name (in our example, \texttt{Lem}), and the text of the
%    label (in our example, \texttt{Lemma}).
%    \begin{macrocode}
     \@thm{#1}{#2}}%
%    \end{macrocode}
%    With this, the `sub-definition' is complete.  The macro
%    |\@endtheorem| ends a theorem environment and is, so far, nothing
%    but an |\endtrivlist|.  (Hence it is defined globally, and not
%    within the theorem styles.\footnote{This has to be changed as
%    soon as theorem styles that change \texttt{\bslash @endtheorem}
%    exist.  In such a case, all existing styles must be changed as
%    well since they will have to reset the macro.}) Therefore, we can
%    set it equivalent to the macro that ends the theorem set (in our
%    example, |\endLem|).  However, if some day theorem styles exist
%    that do change |\@endtheorem|, we would have to use the
%    commented-out line instead.
%    \begin{macrocode}
    \global \expandafter \let \csname end#1\endcsname \@endtheorem
%   \global\@namedef{end#1}{\@endtheorem}%
%    \end{macrocode}
%    With these commands all the required definitions are employed,
%    unless the test |\@ifdefinable| has failed.  Therefore, we
%    end the second argument of this macro and with it the definition
%    of |\@xnthm|.
%    \begin{macrocode}
   }}
%    \end{macrocode}
% \end{macro}
%
% \begin{macro}{\@ynthm}
%    The definition of |\@ynthm| is completely analogous. In
%    this case the new counter that is defined is not reset within
%    another counter; thus the definition of |\the...|
%    is simplified:
%    \begin{macrocode}
\gdef\@ynthm#1#2{\expandafter\@ifdefinable\csname #1\endcsname
   {\@definecounter{#1}%
    \expandafter\xdef\csname the#1\endcsname{\@thmcounter{#1}}%
%    \end{macrocode}
%    The rest of the definition corresponds literally to that of
%    |\@xnthm|:
%    \begin{macrocode}
    \def\@tempa{\global\@namedef{#1}}\expandafter \@tempa
     \expandafter{\csname th@\the \theorem@style \expandafter
     \endcsname \the\theorem@bodyfont \@thm{#1}{#2}}%
    \global \expandafter \let \csname end#1\endcsname \@endtheorem}}
%    \end{macrocode}
% \end{macro}
%
% \begin{macro}{\@othm}
%    The definition of |\@othm| does not contain anything new.
%    \begin{macrocode}
\gdef\@othm#1[#2]#3{%
%    \end{macrocode}
% \changes{v2.2b}{95/10/07}{Build in check for counter defined}
%    We do not define a new counter but instead use one that has
%    already been defined.  Thus the only definition we need is that
%    of this pseudo-counter (i.e.\ |\the|\meta{env.\ name}).
%    First we check if |#2| corresponds to a known counter name.
%    \begin{macrocode}
  \expandafter\ifx\csname c@#2\endcsname\relax
   \@nocounterr{#2}%
  \else
   \expandafter\@ifdefinable\csname #1\endcsname
   {\expandafter \xdef \csname the#1\endcsname
     {\expandafter \noexpand \csname the#2\endcsname}%
%    \end{macrocode}
%    All other parts of the definition can be adopted from
%    |\@xnthm|. We have to remember, though, that in this case
%    the name of the current counter and the theorem label have moved
%    to the second and third arguments.
%    \begin{macrocode}
    \def\@tempa{\global\@namedef{#1}}\expandafter \@tempa
     \expandafter{\csname th@\the \theorem@style \expandafter
     \endcsname \the\theorem@bodyfont \@thm{#2}{#3}}%
    \global \expandafter \let \csname end#1\endcsname \@endtheorem}%
  \fi}
%    \end{macrocode}
% \end{macro}
%
% \subsection{Macros that are employed in a {\envfont theorem}
%    environment}
%
% \begin{macro}{\@thm}
%    The macro |\@thm| has to increase the current counter.  Then,
%    depending on whether the environment has (or does not have) an
%    optional argument, it has to branch into either |\@begintheorem|
%    or |\@opargtheorem|.
%    \begin{macrocode}
\gdef\@thm#1#2{\refstepcounter{#1}%
%    \end{macrocode}
%    Now we start a {\envfont trivlist} environment, and give
%    |\@topsep| and |\@topsepadd| the values of the \textsf{skip}
%    registers |\theorempreskipamount| and |\theorempostskipamount|.
%    The value in |\@topsep| is the vertical space that is inserted by
%    the first (and only) |\item| in our |\trivlist| whilst
%    |\@topsepadd| is inserted by |\@endparenv| at the end of that
%    {\envfont trivlist} environment.  By using these registers, we
%    obtain the desired space around a {\envfont theorem} environment.
%    \begin{macrocode}
   \trivlist
   \@topsep \theorempreskipamount               % used by first \item
   \@topsepadd \theorempostskipamount           % used by \@endparenv
%    \end{macrocode}
%    Now we have to test whether an optional argument has been given.
% \changes{v2.0d}{89/07/18}{Forgotten percent added.}
%    \begin{macrocode}
   \@ifnextchar [%
%    \end{macrocode}
%    If there is an optional argument, we will call |\@ythm|, and
%    move the arguments read back into the input stream.
%    \begin{macrocode}
   {\@ythm{#1}{#2}}%
%    \end{macrocode}
%    If not, we call |\@begintheorem|. Its first argument is the
%    name of the theorem set (hence the second argument of
%    |\@thm|). Its second argument is the macro that produces the
%    current number.
%    \begin{macrocode}
   {\@begintheorem{#2}{\csname the#1\endcsname}\ignorespaces}}
%    \end{macrocode}
% \end{macro}
%
% \begin{macro}{\@xthm}
% \begin{macro}{\@ythm}
%    Both these macros were originally called by |\@thm|.  We do
%    not need |\@xthm| anymore, hence we reset it to
%    |\relax|. The definition of |\@ythm| has not changed at
%    all from its definition in \LaTeX. In order to make the macros
%    easier to understand, we will
%    nevertheless present it (commented out).
%    \begin{macrocode}
\global\let\@xthm\relax
% \def\@ythm#1#2[#3]{\@opargbegintheorem{#2}{\csname
%           the#1\endcsname}{#3}\ignorespaces}
%</package>
%    \end{macrocode}
%    The primitive |\ignorespaces| in |\@ythm| and
%    |\@thm| is needed to remove the spaces between the
%    |\begin{...}| and the actual text.
% \end{macro}
% \end{macro}
%
%
% \subsection{Definition of the theorem substyles}
%
% As already pointed out, the theorem substyles, defined below, are
% only loaded when necessary.  Note that all these substyles, except
% \texttt{plain}, have |\slshape| as the default body font.
%
% \subsubsection{The \texttt{plain} style}
%
% \label{dok:th@plain} As the following macros use |@|, we have
% to locally set the |\catcode| of this symbol to ``letter''.
% This happens within a group, so that we do not have to worry about
% which |\catcode| that symbol had before.
%    \begin{macrocode}
%<*thp>
\begingroup \makeatletter
%    \end{macrocode}
% Since we are now within a group, we must make all definitions
% globally.  First we make sure that \texttt{theorem.sty} is loaded.
% This will allow us to use this file as a document style option
% without having to call \texttt{theorem} itself as an option.  At the
% same time, we assure that at least version 2 is loaded, since
% |\theorem@style| was not defined in earlier versions.
%    \begin{macrocode}
\@ifundefined{theorem@style}{\input{theorem.sty}}{}
\ProvidesFile{thp.sty}
             [\FMithmInfo]
%    \end{macrocode}
%
%
% \begin{macro}{\th@plain}
% \begin{macro}{\@begintheorem}
% \begin{macro}{\@opargbegintheorem}
%    |\theoremstyle{plain}| corresponds to the original
%    definition, except that the distances to the surrounding text are
%    determined by the parameters |\theorempreskipamount| and
%    |\theorempostskipamount|.
%    First we set the default body font.
%    \begin{macrocode}
\gdef\th@plain{\normalfont\itshape
%    \end{macrocode}
%    Then we define |\@begintheorem| and
%    |\@opargbegintheorem|. These two macros define how the
%    header of a theorem is typeset.  |\@opargbegintheorem| will
%    be called if a \textsf{theorem} environment with an optional
%    argument is encountered; otherwise, the header is constructed by
%    calling |\@begintheorem|.  If one of these macros is
%    executed, we are within a \textsf{trivlist} environment started by
%    |\@thm|.  So the theorem header is produced with an
%    |\item| command.
%
%    Instead of specifying the header font directly, all standard
%    theorem styles use the |\theorem@headerfont| macro to allow
%    customization.  The extra space (|\labelsep|) is necessary
%    because of problems in the \textsf{trivlist} environment.
%    \begin{macrocode}
  \def\@begintheorem##1##2{%
        \item[\hskip\labelsep \theorem@headerfont ##1\ ##2]}%
%    \end{macrocode}
%    The definition of |\@opargbegintheorem| is completely
%    analogous. The only difference is the fact that there exists a
%    third argument (which is the optional parameter of the
%    environment and contains additional information about the
%    theorem).  Customarily we enclose it in parentheses.
%    \begin{macrocode}
\def\@opargbegintheorem##1##2##3{%
   \item[\hskip\labelsep \theorem@headerfont ##1\ ##2\ (##3)]}}
%    \end{macrocode}
% \end{macro}
% \end{macro}
% \end{macro}
%
% \noindent
% We conclude with an |\endgroup| to restore the
% |\catcode| of |@|.
%    \begin{macrocode}
\endgroup
%</thp>
%    \end{macrocode}
%
%
% \subsubsection{The \texttt{break} style}
%
% This style option is stored in the file |thb.sty|.
% For the next two lines see the documentation for |\th@plain| on
% page~\pageref{dok:th@plain}.
%    \begin{macrocode}
%<*thb>
\begingroup \makeatletter
\@ifundefined{theorem@style}{\input{theorem.sty}}{}
\ProvidesFile{thb.sty}
             [\FMithmInfo]
%    \end{macrocode}
%
% \begin{macro}{\th@break}
%    |\theoremstyle{break}| produces a line break after the name
%    of the theorem.  The font is |\slshape|.  Hence, we define
%    |\th@break| as follows:
%    \begin{macrocode}
\gdef\th@break{\normalfont\slshape
  \def\@begintheorem##1##2{\item[%
%    \end{macrocode}
%    We run into the following problem: it is not possible to create
%    the header with |\item[|\meta{title}|]| and then start
%    a new line by, for example, |\mbox{}\\|. Such a definition
%    will fail whenever a list environment follows immediately.  With
%    the above construction, the |\mbox{}| causes the switch
%    |@inlabel| (cf.\ definition of |\list| and
%    |\trivlist| in \cite{latex:source}) to be set to \texttt{false}
%    and so the following list will insert additional vertical
%    space (|\topskip|). This is quite annoying. Therefore, we
%    create the line break within the |\item|. In order to ensure
%    that the text will begin at the proper position in the following
%    line, we simply pretend that the label does not take any
%    room.\footnote{This will lead to problems whenever very high
%    symbols occurring in the line tower into the
%    heading. So, something else has to be done here sometime.}
%    \begin{macrocode}
       \rlap{\vbox{\hbox{\hskip \labelsep\theorem@headerfont ##1\ ##2}%
                   \hbox{\strut}}}]}%
%    \end{macrocode}
%    Again, the definition of |\@opargbegintheorem| is
%    completely analogous.
%    \begin{macrocode}
\def\@opargbegintheorem##1##2##3{%
  \item[\rlap{\vbox{\hbox{\hskip \labelsep \theorem@headerfont
                     ##1\ ##2\ (##3)}%
                    \hbox{\strut}}}]}}
\endgroup
%</thb>
%    \end{macrocode}
% \end{macro}
%
%
% \subsubsection{The \texttt{changebreak} style}
%
%    \begin{macrocode}
% This style option is stored in the file |thcb.sty|.
%    \begin{macrocode}
%<*thcb>
\begingroup \makeatletter
\@ifundefined{theorem@style}{\input{theorem.sty}}{}
\ProvidesFile{thcb.sty}
             [\FMithmInfo]
%    \end{macrocode}
%
% \begin{macro}{\th@changebreak}
%    The \texttt{change-break} theorem style is like \texttt{break}
%    but with interchange of theorem name and theorem number.  Thus we
%    define |\th@changebreak| as follows:
%    \begin{macrocode}
\gdef\th@changebreak{\normalfont\slshape
  \def\@begintheorem##1##2{\item
       [\rlap{\vbox{\hbox{\hskip\labelsep \theorem@headerfont ##2\ ##1}%
                    \hbox{\strut}}}]}%
\def\@opargbegintheorem##1##2##3{%
  \item[\rlap{\vbox{\hbox{\hskip\labelsep \theorem@headerfont
                     ##2\ ##1\ (##3)}%
                    \hbox{\strut}}}]}}
\endgroup
%</thcb>
%    \end{macrocode}
% \end{macro}
%
%
% \subsubsection{The \texttt{change} style}
%
% This style option is stored in the file |thc.sty|.
%    \begin{macrocode}
%<*thc>
\begingroup \makeatletter
\@ifundefined{theorem@style}{\input{theorem.sty}}{}
\ProvidesFile{thc.sty}
             [\FMithmInfo]
%    \end{macrocode}
%
% \begin{macro}{\th@change}
%    The \texttt{change} theorem style corresponds to the
%    \texttt{change break} style without a linebreak after the header.
%    To say it in another way, it's the same as the \texttt{plain}
%    style but with number and name interchanged and |\slshape| as the
%    default font.
%    \begin{macrocode}
\gdef\th@change{\normalfont\slshape
  \def\@begintheorem##1##2{\item
       [\hskip\labelsep \theorem@headerfont ##2\ ##1]}%
\def\@opargbegintheorem##1##2##3{%
  \item[\hskip\labelsep \theorem@headerfont ##2\ ##1\ (##3)]}}
\endgroup
%</thc>
%    \end{macrocode}
% \end{macro}
%
%
% \subsubsection{The \texttt{marginbreak} style}
%
% This style option is the one used most often at Mainz.  It is saved
% in the file |thmb.sty|.
%    \begin{macrocode}
%<*thmb>
\begingroup \makeatletter
\@ifundefined{theorem@style}{\input{theorem.sty}}{}
\ProvidesFile{thmb.sty}
             [\FMithmInfo]
%    \end{macrocode}
%
% \begin{macro}{\th@marginbreak}
%    The \texttt{margin break} style is nearly the same as the
%    \texttt{change break} style. The only difference is the placement
%    of the theorem number.  We use |\llap| to place it in the left
%    margin.
%
%    In this style |\labelsep| denotes the separation between the
%    number and the text.
%    \begin{macrocode}
\gdef\th@marginbreak{\normalfont\slshape
  \def\@begintheorem##1##2{\item
       [\rlap{\vbox{\theorem@headerfont
                    \hbox{\llap{##2}\hskip\labelsep ##1}%
                    \hbox{\strut}}}]}%
\def\@opargbegintheorem##1##2##3{%
  \item[\rlap{\vbox{\theorem@headerfont
                    \hbox{\llap{##2}\hskip\labelsep ##1\ (##3)}%
                    \hbox{\strut}}}]}}
\endgroup
%</thmb>
%    \end{macrocode}
% \changes{v2.1c}{90/02/17}{`endgroup outside macrocode env.}
% \end{macro}
%
%
%
% \subsubsection{The \texttt{margin} style}
%
% This style option is stored in the file |thm.sty|.
%    \begin{macrocode}
%<*thm>
\begingroup \makeatletter
\@ifundefined{theorem@style}{\input{theorem.sty}}{}
\ProvidesFile{thm.sty}
             [\FMithmInfo]
%    \end{macrocode}
%
% \begin{macro}{\th@margin}
%    Again this is only a variant of the theorem styles described
%    above without any new ideas.
%    \begin{macrocode}
\gdef\th@margin{\normalfont\slshape
  \def\@begintheorem##1##2{\item
       [\theorem@headerfont \llap{##2}\hskip\labelsep ##1]}%
\def\@opargbegintheorem##1##2##3{%
  \item[\theorem@headerfont \llap{##2}\hskip\labelsep ##1\ (##3)]}}
\endgroup
%</thm>
%    \end{macrocode}
% \end{macro}
%
%
%
%
% \subsection{Final Definitions}
%
% \begin{macro}{\theorempreskipamount}
% \begin{macro}{\theorempostskipamount}
%    The \textsf{skip} parameters that regulate the vertical empty space
%    before and after the {\envfont theorem} environment have
%    to be allocated as well.
% \changes{v2.0f}{89/07/20}{Unnecessary `global' removed.}
%    \begin{macrocode}
%<*package>
\newskip\theorempreskipamount
\newskip\theorempostskipamount
%    \end{macrocode}
%    Since we have used the same values for all theorem sets, we now can
%    assign them.
%    \begin{macrocode}
\global\setlength\theorempreskipamount{12pt plus 5pt minus 3pt}
\global\setlength\theorempostskipamount{8pt plus 3pt minus 1.5pt}
%    \end{macrocode}
% \end{macro}
% \end{macro}
%
%
% \begin{macro}{\@endtheorem}
%    The same holds for the macro |\@endtheorem|, which ends a
%    {\envfont theorem} environment. Since it is the same for all
%    theorem sets, it is removed from the macros |\th@|\meta{style}.
%    It simply ends the {\envfont trivlist} environment, which was
%    begun in |\@thm|.
%    \begin{macrocode}
\global\let\@endtheorem=\endtrivlist
%    \end{macrocode}
% \end{macro}
%
%
%
% \begin{macro}{\@preamblecmds}
%    All macros defined above are to be used only in the preamble.
%    Therefore, we insert them in |\@preamblecmds| which will disable
%    them at begin document. This is done by the internal \LaTeXe{}
%    command |\@onlypreamble|.
% \changes{v2.0g}{89/09/19}{`theoremheaderfont and `theorembodyfont
%    added.}
% \changes{v2.2a}{94/02/03}{Use \LaTeXe{} interface}
%    \begin{macrocode}
\@onlypreamble\@xnthm
\@onlypreamble\@ynthm
\@onlypreamble\@othm
\@onlypreamble\newtheorem
\@onlypreamble\theoremstyle
\@onlypreamble\theorembodyfont
\@onlypreamble\theoremheaderfont
%    \end{macrocode}
% \end{macro}
%
% Finally we declare the \texttt{plain} theorem style to be the default.
% \changes{v2.1b}{90/06/04}{Fixed missing `item by declaring default
%    style.}
%    \begin{macrocode}
\theoremstyle{plain}
%</package>
%    \end{macrocode}
%
% \Finale

\endinput
}{}
\ProvidesFile{thm.sty}
             [\FMithmInfo]
%    \end{macrocode}
%
% \begin{macro}{\th@margin}
%    Again this is only a variant of the theorem styles described
%    above without any new ideas.
%    \begin{macrocode}
\gdef\th@margin{\normalfont\slshape
  \def\@begintheorem##1##2{\item
       [\theorem@headerfont \llap{##2}\hskip\labelsep ##1]}%
\def\@opargbegintheorem##1##2##3{%
  \item[\theorem@headerfont \llap{##2}\hskip\labelsep ##1\ (##3)]}}
\endgroup
%</thm>
%    \end{macrocode}
% \end{macro}
%
%
%
%
% \subsection{Final Definitions}
%
% \begin{macro}{\theorempreskipamount}
% \begin{macro}{\theorempostskipamount}
%    The \textsf{skip} parameters that regulate the vertical empty space
%    before and after the {\envfont theorem} environment have
%    to be allocated as well.
% \changes{v2.0f}{89/07/20}{Unnecessary `global' removed.}
%    \begin{macrocode}
%<*package>
\newskip\theorempreskipamount
\newskip\theorempostskipamount
%    \end{macrocode}
%    Since we have used the same values for all theorem sets, we now can
%    assign them.
%    \begin{macrocode}
\global\setlength\theorempreskipamount{12pt plus 5pt minus 3pt}
\global\setlength\theorempostskipamount{8pt plus 3pt minus 1.5pt}
%    \end{macrocode}
% \end{macro}
% \end{macro}
%
%
% \begin{macro}{\@endtheorem}
%    The same holds for the macro |\@endtheorem|, which ends a
%    {\envfont theorem} environment. Since it is the same for all
%    theorem sets, it is removed from the macros |\th@|\meta{style}.
%    It simply ends the {\envfont trivlist} environment, which was
%    begun in |\@thm|.
%    \begin{macrocode}
\global\let\@endtheorem=\endtrivlist
%    \end{macrocode}
% \end{macro}
%
%
%
% \begin{macro}{\@preamblecmds}
%    All macros defined above are to be used only in the preamble.
%    Therefore, we insert them in |\@preamblecmds| which will disable
%    them at begin document. This is done by the internal \LaTeXe{}
%    command |\@onlypreamble|.
% \changes{v2.0g}{89/09/19}{`theoremheaderfont and `theorembodyfont
%    added.}
% \changes{v2.2a}{94/02/03}{Use \LaTeXe{} interface}
%    \begin{macrocode}
\@onlypreamble\@xnthm
\@onlypreamble\@ynthm
\@onlypreamble\@othm
\@onlypreamble\newtheorem
\@onlypreamble\theoremstyle
\@onlypreamble\theorembodyfont
\@onlypreamble\theoremheaderfont
%    \end{macrocode}
% \end{macro}
%
% Finally we declare the \texttt{plain} theorem style to be the default.
% \changes{v2.1b}{90/06/04}{Fixed missing `item by declaring default
%    style.}
%    \begin{macrocode}
\theoremstyle{plain}
%</package>
%    \end{macrocode}
%
% \Finale

\endinput
}{}
\ProvidesFile{thmb.sty}
             [\FMithmInfo]
%    \end{macrocode}
%
% \begin{macro}{\th@marginbreak}
%    The \texttt{margin break} style is nearly the same as the
%    \texttt{change break} style. The only difference is the placement
%    of the theorem number.  We use |\llap| to place it in the left
%    margin.
%
%    In this style |\labelsep| denotes the separation between the
%    number and the text.
%    \begin{macrocode}
\gdef\th@marginbreak{\normalfont\slshape
  \def\@begintheorem##1##2{\item
       [\rlap{\vbox{\theorem@headerfont
                    \hbox{\llap{##2}\hskip\labelsep ##1}%
                    \hbox{\strut}}}]}%
\def\@opargbegintheorem##1##2##3{%
  \item[\rlap{\vbox{\theorem@headerfont
                    \hbox{\llap{##2}\hskip\labelsep ##1\ (##3)}%
                    \hbox{\strut}}}]}}
\endgroup
%</thmb>
%    \end{macrocode}
% \changes{v2.1c}{90/02/17}{`endgroup outside macrocode env.}
% \end{macro}
%
%
%
% \subsubsection{The \texttt{margin} style}
%
% This style option is stored in the file |thm.sty|.
%    \begin{macrocode}
%<*thm>
\begingroup \makeatletter
\@ifundefined{theorem@style}{% \iffalse meta-comment
%
% Copyright (C) 1993-2024
%
% The LaTeX Project and any individual authors listed elsewhere
% in this file.
%
% This file is part of the Standard LaTeX `Tools Bundle'.
% -------------------------------------------------------
%
% It may be distributed and/or modified under the
% conditions of the LaTeX Project Public License, either version 1.3c
% of this license or (at your option) any later version.
% The latest version of this license is in
%    https://www.latex-project.org/lppl.txt
% and version 1.3c or later is part of all distributions of LaTeX
% version 2005/12/01 or later.
%
% The list of all files belonging to the LaTeX `Tools Bundle' is
% given in the file `manifest.txt'.
%
% \fi
%
%%
%
%
\def\FMithmInfo{2023/07/05 v2.2c Theorem extension package (FMi)}
%
% \ProvidesFile{theorem.dtx}[\FMithmInfo]
%\iffalse   % this is a METACOMMENT !
%%
%% Package `theorem' to use with LaTeX2e
%% Copyright (C) 1989-1995 Frank Mittelbach, all rights reserved.
%<+package>\@ifundefined{theorem@style}{}{\endinput}
%<+package>\NeedsTeXFormat{LaTeX2e}[1995/06/01]
%<+package>\ProvidesPackage{theorem}
%<+package>         [\FMithmInfo]
%
% \fi
%
% \changes{v1.0}{87/05/07}{First draft.}
% \changes{v1.0a}{87/05/07}{`thistheoremstyle' added.}
% \changes{v1.1a}{87/07/29}{New concept: redefine the internal
%                          \protect\LaTeX{} commands `@xnthm' etc.}
% \changes{v1.1b}{87/07/31}{`theoremfont' used to specify font.}
% \changes{v1.2a}{88/01/08}{`@endtheorem' now globally defined.}
% \changes{v1.2b}{88/01/11}{`newtheorem' only allowed in the preamble.}
% \changes{v1.3a}{88/02/14}{`theoremfont' now a toks register.}
%
% \changes{v2.0a}{88/11/16}{New implementation and new documentation.}
% \changes{v2.0a}{88/11/16}{No layout changes but internal modifications
%                           old substyles won't work.}
% \changes{v2.0b}{89/01/30}{`global' used for all definitions.}
% \changes{v2.0c}{89/06/15}{Documentation translated to English.
%                            Thanks to LLL and Nina}
% \changes{v2.1a}{90/02/15}{Support for new font selection.}
% \changes{v2.2a}{94/02/03}{Cleanup for LaTeX2e}
%
% \DoNotIndex{\@endparenv,\@ifdefinable,\@ifnextchar,\@ifundefined}
% \DoNotIndex{\@namedef,\@nameuse,\@spaces,\@tempa,\@warning}
% \DoNotIndex{\ }
% \DoNotIndex{\begingroup,\bfseries}
% \DoNotIndex{\csname}
% \DoNotIndex{\def,\do,\docdate}
% \DoNotIndex{\endcsname,\endgroup,\endinput,\endtrivlist,\expandafter}
% \DoNotIndex{\filedate,\fileversion}
% \DoNotIndex{\gdef,\global}
% \DoNotIndex{\hbox,\hskip}
% \DoNotIndex{\ignorespaces,\input,\itshape,\item}
% \DoNotIndex{\labelsep,\let,\llap}
% \DoNotIndex{\makeatletter}
% \DoNotIndex{\noexpand}
% \DoNotIndex{\refstepcounter,\relax,\rlap}
% \DoNotIndex{\setlength,\slshape,\space,\string,\strut}
% \DoNotIndex{\the,\trivlist,\typeout}
% \DoNotIndex{\vbox}
% \DoNotIndex{\xdef}
%
% \def\deflabel#1{\ttfamily #1\hfill}
% \def\deflist#1{\begin{list}{}{\settowidth\labelwidth{\ttfamily #1}%
%                               \setlength\leftmargin\labelwidth
%                               \addtolength\leftmargin\labelsep
%                               \let\makelabel\deflabel}}
% \def\enddeflist{\end{list}}
% \def\envfont{\normalfont\ttfamily}
%
% \MakeShortVerb{\|}
%
% \newcommand{\TUB}{{\sl TUGboat\/}}
%
% \GetFileInfo{theorem.dtx}
% \title{An Extension of the
%        \LaTeX{} theorem environment\thanks{This file
%        has version number \fileversion, last
%        revised \filedate.}}
% \author{Frank Mittelbach}
%
% \MaintainedByLaTeXTeam{tools}
% \maketitle
%
% \begin{abstract}
%    The macros described in this paper yield an extension of the
%    \LaTeX{} theorem mechanism. It is designed to satisfy the
%    different requirements of various journals. Thus, the
%    layout of the ``theorems'' can be manipulated by determining a
%    ``style''. This article describes not only the use, but
%    also the definition, of the necessary macros.
% \end{abstract}
%
% \section*{Preface to version 2.2}
%
% For \LaTeXe{} this package did not need any fundamental changes. I
% only modified the messages generated so that theorem layout styles
% will show up with the |\listfiles| command and cleaned the section
% on the New Font Selection Scheme since this is now included in
% \LaTeX.
%
% \section*{Preface to version 2.1}
%
% This version is identical to 2.0g described in \TUB\ 10\#3 except
% for some internal defaults which are now set depending on the used
% font selection scheme.
%
% This was done to avoid unpleasant surprises if the new font selection
% scheme is in force. For further details see
% section~\ref{sec:newfonts} and \cite{bk:GMS94}.
%
% \section{Introduction}
%
% For our purposes here, ``theorems'' are labelled enunciations,
% often set off from the main text by extra space and a font change.
% Theorems, corollaries, conjectures, definitions, and remarks
% are all instances of ``theorems''.  The ``header'' of these structures
% is composed of a label (such as \textsc{Theorem} or \textsc{Remark})
% and a number which serializes an item in the sequence of items
% with the same label.
%
% Shortly after the introduction of \LaTeX{} at the Fachbereich
% Mathematik in Mainz, the desire to manipulate the layout of
% ``theorems'' arose.  In Mainz, the following two conventions came
% into general use:
% \begin{enumerate}
%   \item The number of the theorem is shown in the margin.
%   \item There is a line break at
%         the end of the theorem header.
% \end{enumerate}
%
% Additionally, some journals require different formats which depend on
% the ``sort of theorem'': e.g.\ often remarks and definitions are set
% in |\upshape|, while |\itshape| is employed for main theorems.
%
% Confronted with these requirements, a theorem environment was
% developed in Mainz which allows separate determination of the
% layout of the ``theorems sets'', comparable to |\pagestyle|.
%
%
% \section{The user interface}
% \subsection{Defining new theorem sets}
%
% \DescribeMacro\newtheorem
% As in the original \LaTeX{} version, the command |\newtheorem|
% defines a new ``theorem set'' or ``theorem-like structure''.
% Two required arguments name the new environment and give the text
% to be typeset with each instance of the new ``set'', while
% an optional argument determines how
% the ``set'' is enumerated:
% \begin{description}
%    \item[\ttfamily \bslash newtheorem\{foo\}\{bar\}]
%       The theorem set {\envfont foo} (whose name is \texttt{bar})
%       uses its own counter.
%    \item[\ttfamily \bslash newtheorem\{foo2\}{[foo]}\{bar2\}]
%       The theorem set {\envfont foo2} (printed name \texttt{bar2})
%       uses the same counter as the theorem set \texttt{foo}.
%    \item[\ttfamily \bslash newtheorem\{foo3\}\{bar3\}{[section]}]
%       The theorem set {\envfont foo3} (printed name \texttt{bar3}) is
%       enumerated within the counter \texttt{section}, i.e.\ with every
%       new |\section| the enumeration begins again with 1, and
%       the enumeration is composed from the section-number and the
%       theorem counter itself.
% \end{description}
%
% \DescribeMacro\theoremstyle
% Additionally, the command |\theoremstyle| can define the layout
% of various, or all, theorem sets.  It should be noted that any
% theorem set defined by |\newtheorem| is typeset in the
% |\theoremstyle| that is current at the time of the definition.
% Thus, the following
% \begin{verbatim}
%    \theoremstyle{break}        \newtheorem{Cor}{Corollary}
%    \theoremstyle{plain}        \newtheorem{Exa}{Example}[section]
%\end{verbatim}                             ^^A No blank at line start
% leads to the result that the set {\envfont Cor} is formatted in the
% style \texttt{break}, while the set {\envfont Exa} and all the
% following ones are formatted in the style \texttt{plain}, unless
% another |\theoremstyle| follows.  Since the definitions installed by
% |\newtheorem| are global, one also can limit |\theoremstyle| locally
% by grouping braces.
%
% \DescribeMacro\theorembodyfont
% The choice of the font for the theorem body is completely
% independent of the chosen |\theoremstyle|; this has proven to
% be very advantageous.  For example,
% \begin{verbatim}
%   {\theorembodyfont{\upshape}          \newtheorem{Rem}{Remark}}
%\end{verbatim}
% defines a theorem set {\envfont Rem}, which will be set in
% |\upshape| in the current layout (which in our example is
% \texttt{plain}).  As with |\theoremstyle|, the |\theorembodyfont|
% chosen is that current at the time of |\newtheorem|.  If
% |\theorembodyfont| is not specified or one defines
% |\theorembodyfont{}|, then the font used will be that defined by the
% |\theoremstyle|.
%
% \DescribeMacro\theoremheaderfont It is also possible to customize
% the font used for the theorem headers.  This is, however, a global
% declaration, and therefore there should be at most one
% |\theoremheaderfont| declaration in the preamble.\footnote{If it is
% actually necessary to have different header fonts, one has to define
% new theorem styles (substituting the desired font) or specify the
% information directly in the \texttt{\bslash newtheorem} declaration
% (the unclean variant).}
%
% \DescribeMacro\theorempreskipamount
% \DescribeMacro\theorempostskipamount Two additional parameters
% affect the vertical space around the theorem environments:
% |\theorempreskipamount| and |\theorempostskipamount| define,
% respectively, the spacing before and after such an environment.
% These parameters apply for all theorem sets and can be manipulated
% with the ordinary length macros.  They are rubber lengths
% (`\textsf{skips}'), and therefore can contain \texttt{plus} and
% \texttt{minus} parts.
%
% Since the definition of theorem sets should---most sensibly---be
% placed in the preamble, we only allow installation there.
% It is therefore possible to release the memory used here after
% |\begin{document}|, in order to make room for other
% applications.
%
%
% \subsection{Existing theorem styles}
%
% The following theorem styles exist to date:
% \begin{deflist}{marginbreak}
%    \item[plain]
%       This theorem style emulates the original \LaTeX{} definition,
%       except that additionally the parameters
%       |\theorem...skipamount| are used.
%    \item[break]
%       In this style, the theorem header is followed by a line
%       break.
%    \item[marginbreak]
%       The theorem number is set in the margin, and there is a line
%       break as in \texttt{break}.
%    \item[changebreak]
%       Like \texttt{break}, but with header number and text
%       interchanged.
%    \item[change]
%       Header number and text are interchanged, without a line break.
%    \item[margin]
%       The number is set in the left margin, without a line break.
% \end{deflist}
% All styles (except \texttt{plain}) select |\slshape| as the default
% |\theorembodyfont|.
%
%
% \subsection{Examples}
%
% Given the above theorem sets {\envfont Cor}, {\envfont Exa}
% and {\envfont Rem},
% suppose that the preamble also contains the declarations:
% \begin{verbatim}
%    \theoremstyle{marginbreak}   \newtheorem{Lem}[Cor]{Lemma}
%    \theoremstyle{change}
%    \theorembodyfont{\itshape}        \newtheorem{Def}[Cor]{Definition}
%
%    \theoremheaderfont{\scshape}
%\end{verbatim}
% Then the following are some typical examples of the typeset output
% resulting from their use.
%
% \begin{Cor}
%    This is a sentence typeset in the theorem environment {\envfont
%    Cor}.
% \end{Cor}
%
% \begin{Exa}
%    This is a sentence typeset in the theorem environment {\envfont
%    Exa}.
% \end{Exa}
%
% \begin{Rem}
%    This is a sentence typeset in the theorem environment {\envfont
%    Rem}.
% \end{Rem}
%
% \begin{Lem}[Ben User]
%    This is a sentence typeset in the theorem environment {\envfont
%    Lem}.
% \end{Lem}
%
% \begin{Def}[Very impressive Definition]
%    This is a sentence typeset in the theorem environment {\envfont
%    Def}.
% \end{Def}
%
% The last two examples show the effect of the optional argument to
% a theorem environment (it is the text typeset in parentheses).
%
%
% \section{Special Considerations}
% \label{sec:newfonts}
%
% Theoremheader and body are implemented as a unit. This means that
% the |\theoremheaderfont|  will inherit characteristics of the
% |\theorembodyfont| in \LaTeXe. Thus, if for example
% |\theorembodyfont| is |\itshape| and |\theoremheaderfont|
% is |\bfseries| the font selected for the header will have the
% characteristics `bold extended italic'. If this is not desired one
% should set the |\theoremheaderfont| to something like
% \begin{verbatim}
%      \theoremheaderfont{\normalfont\bfseries}
%\end{verbatim}
% i.e.\ supplying all necessary font information explicitly.
%
%
%
% \section{Acknowledgements}
%
% The publication of this set of macros was only possible with the
% help of Christina Busse (translating the manuscript into English),
% Joachim Pense (playing the r\^ole of typist), Chris Rowley
% (looking everything over) and many others providing useful
% suggestions.
%
% \MaybeStop{%
%    \begin{thebibliography}{9}
%       \bibitem{bk:GMS94} \textsc{M.~Goossens}, \textsc{F.~Mittelbach}
%         and \textsc{A.~Samarin}.
%         \newblock The \LaTeX{} Companion.
%         \newblock
%         Addison-Wesley, Reading, Massachusetts, 1994.
%       \bibitem{latex:source} \textsc{Lamport, Leslie.} \newblock
%                               \texttt{latex.tex}, version 2.09, date
%                               Feb. 1990.
%    \end{thebibliography}
%    \PrintIndex
%    \PrintChanges}
%
%
% \section{The documentation driver file}
%
% The next bit of code contains the documentation driver file for
% \TeX{}, i.e., the file that will produce the documentation you are
% currently reading. It will be extracted from this file by the
% \texttt{docstrip} program. Since it is the first code in the file
% one can alternatively process this file directly with \LaTeXe{} to
% obtain the documentation.
%
%    \begin{macrocode}
%<*driver>
\documentclass{ltxdoc}

\usepackage{theorem}

%<+driver>% The next few lines define theorem sets which are used
%<+driver>% in the example section of the documentation.

\theoremstyle{break}       \newtheorem{Cor}{Corollary}
\theoremstyle{plain}       \newtheorem{Exa}{Example}[section]
{\theorembodyfont{\upshape}\newtheorem{Rem}{Remark}}
\theoremstyle{marginbreak} \newtheorem{Lem}[Cor]{Lemma}
\theoremstyle{change}
\theorembodyfont{\itshape} \newtheorem{Def}[Cor]{Definition}

\theoremheaderfont{\scshape}

\RecordChanges

\begin{document}
   \DocInput{theorem.dtx}
\end{document}
%</driver>
%    \end{macrocode}
%
% \section{Definition of the Macros}
%
% If the file has been loaded before, we abort immediately. If not the
% package announces itself (this is actually done at the very top of
% the file---the way it is done isn't good style so don't copy it).
% \changes{v2.0e}{89/07/19}{Spaces removed from `typeout'.}
% \changes{v2.2b}{95/11/19}{Announce moved to top of file}
%    \begin{macrocode}
%<*package>
%\@ifundefined{theorem@style}{}{\endinput}
%\def\FMithmInfo{1995/11/19 v2.2b Theorem extension package (FMi)}
%\ProvidesPackage{theorem}[\FMithmInfo]
%    \end{macrocode}
%
%
% \subsection{Definition of theorem styles and fonts}
%
% All the definitions in this file are done globally to allow
% inputting this file inside a group.
%
% \begin{macro}{\theoremstyle}
%    Before a theorem style can be installed, the chosen style must be
%    known. For that reason, we must test to see that
%    |\th@|\meta{style} is known or, more precisely, that it is
%    different from |\relax|. If the style is not known then
%    |\th@plain| is used.
%    \begin{macrocode}
\gdef\theoremstyle#1{%
   \@ifundefined{th@#1}{\@warning
          {Unknown theoremstyle `#1'. Using `plain'}%
          \theorem@style{plain}}%
%    \end{macrocode}
%    We save the theorem style to be used in the token register
%    |\theorem@style|.
%    \begin{macrocode}
      {\theorem@style{#1}}%
%    \end{macrocode}
%    Now we ``evaluate'' the theorem style: this means, we call the
%    macro |\th@|\meta{style} which will activate the relevant
%    definitions which are contained in a separate file.  This is done
%    in a group to suppress changes to the current font.  This could
%    otherwise pose problems together with the new font selection
%    scheme\footnote{When I printed the original article using the new
%    font selection scheme I ended with a document with slanted
%    typefaces (text headings and all) simply because one of the
%    theorem styles used \texttt{\bslash sl} at toplevel.} if the
%    |\th@|\meta{style} is evaluated a second time.
% \changes{v2.1a}{90/02/15}{Group added around `th@..}
% \changes{v2.1a}{90/02/15}{Reseting of `..@begintheorem removed.}
%    \begin{macrocode}
      \begingroup
        \csname th@\the\theorem@style \endcsname
      \endgroup}
%    \end{macrocode}
% \end{macro}
%
%
% \begin{macro}{\@begintheorem}
% \changes{v2.1a}{90/02/15}{Relaxed at toplevel.}
% \begin{macro}{\@opargbegintheorem}
% \changes{v2.1a}{90/02/15}{Relaxed at toplevel.}
%    We reset |\@begintheorem| and |\@opargbegintheorem| to
%    |\relax| since these commands are no longer necessary at
%    toplevel. This will save a few tokens.
%    \begin{macrocode}
\global\let\@begintheorem\relax
\global\let\@opargbegintheorem\relax
%    \end{macrocode}
% \end{macro}
% \end{macro}
%
% \begin{macro}{\theorem@style}
%    Obviously the \textsf{token} register used above has to be
%    allocated. To assure the utmost compatibility with the
%    original \LaTeX{} definition, we set the default theorem style to
%    \texttt{plain}, which implements the usual \LaTeX{} convention.
% \changes{v2.0f}{89/07/20}{Unnecessary `global' removed.}
%    \begin{macrocode}
\newtoks\theorem@style
\global\theorem@style{plain}
%    \end{macrocode}
% \end{macro}
%
% \begin{macro}{\theorembodyfont}
% \begin{macro}{\theorem@bodyfont}
%    If the |\theorembodyfont| is set by the user then it should not
%    interact with the default font set in the theorem style. When the
%    new font selection is in force this may happen if, for example,
%    the default is |\itshape| and the new |\theorembodyfont| is
%    |\sffamily|. So we add a |\reset@font| command in front of the
%    user definition.
% \changes{v2.0f}{89/07/20}{Unnecessary `global' removed.}
% \changes{v2.0g}{89/09/19}{`theoremfont to `theorembodyfont.}
% \changes{v2.1c}{92/06/22}{`reset@font added and logic changed.}
% \changes{v2.1d}{92/10/28}{Reset to empty if arg is empty}
%    \begin{macrocode}
\gdef\theorembodyfont#1{%
%    \end{macrocode}
%    We check if the argument supplied is empty and if so put nothing
%    into the |\theorem@bodyfont| token register to allow for
%    |\theorembodyfont{}| as a mean of using the default of the
%    current |\theoremstyle|.
%    \begin{macrocode}
   \def\@tempa{#1}%
   \ifx\@tempa\@empty
    \theorem@bodyfont{}%
   \else
    \theorem@bodyfont{\reset@font#1}%
   \fi
   }
\newtoks\theorem@bodyfont
\global\theorem@bodyfont{}
%    \end{macrocode}
% \end{macro}
% \end{macro}
%
%
% \begin{macro}{\theoremheaderfont}
% \changes{v2.0g}{89/09/19}{Macro added.}
%    The font for the theorem headers is handled differently because
%    this definition applies to all theorem styles.
%    \begin{macrocode}
\gdef\theoremheaderfont#1{\gdef\theorem@headerfont{#1}%
%    \end{macrocode}
%    After using the macro once it is redefined to produce an error
%    message.
%    \begin{macrocode}
       \gdef\theoremheaderfont##1{%
        \typeout{\string\theoremheaderfont\space should be used
                 only once.}}}
%    \end{macrocode}
% \end{macro}
%
% \begin{macro}{\theorem@headerfont}
% \changes{v2.0g}{89/09/19}{Macro added.}
%    To set the |\theorem@headerfont| default we first test if
%    the new fontselection scheme is in force.
% \changes{v2.1a}{90/02/15}{Support for new font selection.}
%    \begin{macrocode}
\ifx\upshape\undefined
%    \end{macrocode}
%    If not we define it to expand into |\bfseries|. We don't use
%    |\let| just in case a following style option redefines this
%    macro.
%    \begin{macrocode}
\gdef\theorem@headerfont{\bfseries}
%    \end{macrocode}
%    Otherwise we reset the current shape before calling |\bfseries|.
%    \begin{macrocode}
\else \gdef\theorem@headerfont{\normalfont\bfseries}\fi
%    \end{macrocode}
% \end{macro}
%
% \begin{macro}{\th@plain}
% \begin{macro}{\th@break}
% \begin{macro}{\th@marginbreak}
% \begin{macro}{\th@changebreak}
% \begin{macro}{\th@change}
% \begin{macro}{\th@margin}
%    The different styles are defined in macros such as
%    |\th@plain|.  Since memory space is precious in
%    ``non-Big-versions'', we have to avoid offering too many unused
%    definitions.  Therefore we define these styles in separate files
%    that can be loaded on demand.  Thus the commands themselves only
%    load these files. We use |\@input@|, a \LaTeXe{} internal command
%    that ensures that the file will be listed with |\listfiles|.
% \changes{v2.2a}{94/02/02}{Use `@input@ to load theorem layout files}
%    \begin{macrocode}
\gdef\th@plain{\@input@{thp.sty}}
\gdef\th@break{\@input@{thb.sty}}
\gdef\th@marginbreak{\@input@{thmb.sty}}
\gdef\th@changebreak{\@input@{thcb.sty}}
\gdef\th@change{\@input@{thc.sty}}
\gdef\th@margin{\@input@{thm.sty}}
%    \end{macrocode}
%    This list will be expanded when new styles become
%    available.  For testing,  just append new theorem substyles
%    as document options.
% \end{macro}
% \end{macro}
% \end{macro}
% \end{macro}
% \end{macro}
% \end{macro}
%
%
%
% \subsection{Definition of a new theorem set}
%
% As already pointed out, a new theorem environment can be defined in
% three different ways:
% \begin{verbatim}
%    \newtheorem{Lem}{Lemma}
%    \newtheorem{Lem}{Lemma}[section]
%    \newtheorem{Lem}[Theorem]{Lemma}
%\end{verbatim}
% The function of the macro |\newtheorem| is to recognize these cases
% and then to branch into one of the three macros |\@ynthm|, |\@xnthm|
% or |\@othm|.  This mechanism is adopted unchanged from
% \cite{latex:source}; the essential point here is that, for example,
% in the second case, the arguments \texttt{Lem}, \texttt{Lemma} and
% \texttt{section} are passed over to the macro |\@xnthm|.
%
% We inspect this case first because the others present fewer problems,
% and thus are easily derived from this one.
%
% \begin{macro}{\@xnthm}
%    For our example arguments, the macro |\@xnthm| must fulfill
%    the following:
%    \begin{itemize}
%       \item
%          Define a new \LaTeX-counter `Lem'
%       \item
%          reset this counter within a |\section|
%       \item
%          define the macro |\theLem|
%       \item
%          define the environment macros |\Lem| and
%          |\endLem| using the current
%          |\theoremstyle| and |\theorem@bodyfont|.
%    \end{itemize}
%    Obviously, all this should happen only if the first argument
%    of |\@xnthm| (i.e.\ \texttt{Lem} in our example) is chosen so
%    as not to conflict with any previously defined commands or
%    environments.  This test is performed by the \LaTeX{} macro
%    |\@ifdefinable|.
%    \begin{macrocode}
\gdef\@xnthm#1#2[#3]{\expandafter\@ifdefinable\csname #1\endcsname
%    \end{macrocode}
%    Therefore, the first argument of |\@ifdefinable| is the
%    expansion (in the example, |\Lem|) of
%    |\csname||#1\endcsname|.  The second argument is
%    executed only if the test has been completed successfully.
%    \begin{macrocode}
   {%
%    \end{macrocode}
%    Now we define the new counter. The names of the \LaTeX{}
%    macros employed should speak for themselves:
% \changes{v2.2b}{1995/10/07}{Use \cs{@newctr}}
% \changes{v2.2c}{1995/11/23}{Use \cs{@newctr} properly.}
%    \begin{macrocode}
    \@definecounter{#1}\@newctr{#1}[#3]%
%    \end{macrocode}
%    Using |\@newctr| will give a proper error message if the counter
%    in |#3| is not defined.
%    In defining `|\theLem|' we must generate the desired macro
%    name by use of |\expandafter| and |\csname|.
%    \begin{macrocode}
    \expandafter\xdef\csname the#1\endcsname
%    \end{macrocode}
%    An |\xdef| is used in order to make the definition global,
%    and to ensure that it contains the replacement texts of
%    |\@thmcountersep| and |\@thmcounter|.\footnote{These
%    two macros can be defined by the document style.  Their default
%    values produce a `\texttt{.}' as separation and an arabic
%    representation of the number.} However, not everything should be
%    expanded. For example, it saves space to use |\thesection|
%    instead of its---at times---lengthy expansion.
%    \begin{macrocode}
      {\expandafter \noexpand \csname the#3\endcsname
       \@thmcountersep \@thmcounter{#1}}%
%    \end{macrocode}
%    Thus with the defaults of \LaTeX{}, |\theLem| would be
%    replaced by the command sequence |\thesection.\arabic{Lem}|.
%
%    We will now look at the definition of the macro which is executed
%    at the beginning of the actual environment (in our example this
%    macro is |\Lem|). It should be noted that we use an
%    ``|\expandafter| trick'' to expand only certain parts of the
%    replacement text at the time of the definition.
%    \begin{macrocode}
    \def\@tempa{\global\@namedef{#1}}%
    \expandafter \@tempa \expandafter{%
%    \end{macrocode}
%    First, the macro that contains the current definitions of
%    |\@begintheorem| and |\@opargtheorem| should be called
%    up.  The name of this macro---as is already known---has the form
%    |\th@|\meta{theorem\ style}; therefore, it must be called by
%    \begin{macrocode}
      \csname th@\the \theorem@style
%    \end{macrocode}
%    In addition the default theorem font should be changeable, i.e.\
%    we have to insert the contents of |\theorem@bodyfont|. For
%    that reason, we expand even further, beyond |\endcsname|,
%    and thus insert the contents of the \textsf{token} register
%    |\theorem@bodyfont| in the replacement text.
%    \begin{macrocode}
            \expandafter \endcsname \the \theorem@bodyfont
%    \end{macrocode}
%    Now it is time to call the macro |\@thm| which takes over
%    the further processing.  It has two arguments: the current
%    counter name (in our example, \texttt{Lem}), and the text of the
%    label (in our example, \texttt{Lemma}).
%    \begin{macrocode}
     \@thm{#1}{#2}}%
%    \end{macrocode}
%    With this, the `sub-definition' is complete.  The macro
%    |\@endtheorem| ends a theorem environment and is, so far, nothing
%    but an |\endtrivlist|.  (Hence it is defined globally, and not
%    within the theorem styles.\footnote{This has to be changed as
%    soon as theorem styles that change \texttt{\bslash @endtheorem}
%    exist.  In such a case, all existing styles must be changed as
%    well since they will have to reset the macro.}) Therefore, we can
%    set it equivalent to the macro that ends the theorem set (in our
%    example, |\endLem|).  However, if some day theorem styles exist
%    that do change |\@endtheorem|, we would have to use the
%    commented-out line instead.
%    \begin{macrocode}
    \global \expandafter \let \csname end#1\endcsname \@endtheorem
%   \global\@namedef{end#1}{\@endtheorem}%
%    \end{macrocode}
%    With these commands all the required definitions are employed,
%    unless the test |\@ifdefinable| has failed.  Therefore, we
%    end the second argument of this macro and with it the definition
%    of |\@xnthm|.
%    \begin{macrocode}
   }}
%    \end{macrocode}
% \end{macro}
%
% \begin{macro}{\@ynthm}
%    The definition of |\@ynthm| is completely analogous. In
%    this case the new counter that is defined is not reset within
%    another counter; thus the definition of |\the...|
%    is simplified:
%    \begin{macrocode}
\gdef\@ynthm#1#2{\expandafter\@ifdefinable\csname #1\endcsname
   {\@definecounter{#1}%
    \expandafter\xdef\csname the#1\endcsname{\@thmcounter{#1}}%
%    \end{macrocode}
%    The rest of the definition corresponds literally to that of
%    |\@xnthm|:
%    \begin{macrocode}
    \def\@tempa{\global\@namedef{#1}}\expandafter \@tempa
     \expandafter{\csname th@\the \theorem@style \expandafter
     \endcsname \the\theorem@bodyfont \@thm{#1}{#2}}%
    \global \expandafter \let \csname end#1\endcsname \@endtheorem}}
%    \end{macrocode}
% \end{macro}
%
% \begin{macro}{\@othm}
%    The definition of |\@othm| does not contain anything new.
%    \begin{macrocode}
\gdef\@othm#1[#2]#3{%
%    \end{macrocode}
% \changes{v2.2b}{95/10/07}{Build in check for counter defined}
%    We do not define a new counter but instead use one that has
%    already been defined.  Thus the only definition we need is that
%    of this pseudo-counter (i.e.\ |\the|\meta{env.\ name}).
%    First we check if |#2| corresponds to a known counter name.
%    \begin{macrocode}
  \expandafter\ifx\csname c@#2\endcsname\relax
   \@nocounterr{#2}%
  \else
   \expandafter\@ifdefinable\csname #1\endcsname
   {\expandafter \xdef \csname the#1\endcsname
     {\expandafter \noexpand \csname the#2\endcsname}%
%    \end{macrocode}
%    All other parts of the definition can be adopted from
%    |\@xnthm|. We have to remember, though, that in this case
%    the name of the current counter and the theorem label have moved
%    to the second and third arguments.
%    \begin{macrocode}
    \def\@tempa{\global\@namedef{#1}}\expandafter \@tempa
     \expandafter{\csname th@\the \theorem@style \expandafter
     \endcsname \the\theorem@bodyfont \@thm{#2}{#3}}%
    \global \expandafter \let \csname end#1\endcsname \@endtheorem}%
  \fi}
%    \end{macrocode}
% \end{macro}
%
% \subsection{Macros that are employed in a {\envfont theorem}
%    environment}
%
% \begin{macro}{\@thm}
%    The macro |\@thm| has to increase the current counter.  Then,
%    depending on whether the environment has (or does not have) an
%    optional argument, it has to branch into either |\@begintheorem|
%    or |\@opargtheorem|.
%    \begin{macrocode}
\gdef\@thm#1#2{\refstepcounter{#1}%
%    \end{macrocode}
%    Now we start a {\envfont trivlist} environment, and give
%    |\@topsep| and |\@topsepadd| the values of the \textsf{skip}
%    registers |\theorempreskipamount| and |\theorempostskipamount|.
%    The value in |\@topsep| is the vertical space that is inserted by
%    the first (and only) |\item| in our |\trivlist| whilst
%    |\@topsepadd| is inserted by |\@endparenv| at the end of that
%    {\envfont trivlist} environment.  By using these registers, we
%    obtain the desired space around a {\envfont theorem} environment.
%    \begin{macrocode}
   \trivlist
   \@topsep \theorempreskipamount               % used by first \item
   \@topsepadd \theorempostskipamount           % used by \@endparenv
%    \end{macrocode}
%    Now we have to test whether an optional argument has been given.
% \changes{v2.0d}{89/07/18}{Forgotten percent added.}
%    \begin{macrocode}
   \@ifnextchar [%
%    \end{macrocode}
%    If there is an optional argument, we will call |\@ythm|, and
%    move the arguments read back into the input stream.
%    \begin{macrocode}
   {\@ythm{#1}{#2}}%
%    \end{macrocode}
%    If not, we call |\@begintheorem|. Its first argument is the
%    name of the theorem set (hence the second argument of
%    |\@thm|). Its second argument is the macro that produces the
%    current number.
%    \begin{macrocode}
   {\@begintheorem{#2}{\csname the#1\endcsname}\ignorespaces}}
%    \end{macrocode}
% \end{macro}
%
% \begin{macro}{\@xthm}
% \begin{macro}{\@ythm}
%    Both these macros were originally called by |\@thm|.  We do
%    not need |\@xthm| anymore, hence we reset it to
%    |\relax|. The definition of |\@ythm| has not changed at
%    all from its definition in \LaTeX. In order to make the macros
%    easier to understand, we will
%    nevertheless present it (commented out).
%    \begin{macrocode}
\global\let\@xthm\relax
% \def\@ythm#1#2[#3]{\@opargbegintheorem{#2}{\csname
%           the#1\endcsname}{#3}\ignorespaces}
%</package>
%    \end{macrocode}
%    The primitive |\ignorespaces| in |\@ythm| and
%    |\@thm| is needed to remove the spaces between the
%    |\begin{...}| and the actual text.
% \end{macro}
% \end{macro}
%
%
% \subsection{Definition of the theorem substyles}
%
% As already pointed out, the theorem substyles, defined below, are
% only loaded when necessary.  Note that all these substyles, except
% \texttt{plain}, have |\slshape| as the default body font.
%
% \subsubsection{The \texttt{plain} style}
%
% \label{dok:th@plain} As the following macros use |@|, we have
% to locally set the |\catcode| of this symbol to ``letter''.
% This happens within a group, so that we do not have to worry about
% which |\catcode| that symbol had before.
%    \begin{macrocode}
%<*thp>
\begingroup \makeatletter
%    \end{macrocode}
% Since we are now within a group, we must make all definitions
% globally.  First we make sure that \texttt{theorem.sty} is loaded.
% This will allow us to use this file as a document style option
% without having to call \texttt{theorem} itself as an option.  At the
% same time, we assure that at least version 2 is loaded, since
% |\theorem@style| was not defined in earlier versions.
%    \begin{macrocode}
\@ifundefined{theorem@style}{% \iffalse meta-comment
%
% Copyright (C) 1993-2024
%
% The LaTeX Project and any individual authors listed elsewhere
% in this file.
%
% This file is part of the Standard LaTeX `Tools Bundle'.
% -------------------------------------------------------
%
% It may be distributed and/or modified under the
% conditions of the LaTeX Project Public License, either version 1.3c
% of this license or (at your option) any later version.
% The latest version of this license is in
%    https://www.latex-project.org/lppl.txt
% and version 1.3c or later is part of all distributions of LaTeX
% version 2005/12/01 or later.
%
% The list of all files belonging to the LaTeX `Tools Bundle' is
% given in the file `manifest.txt'.
%
% \fi
%
%%
%
%
\def\FMithmInfo{2023/07/05 v2.2c Theorem extension package (FMi)}
%
% \ProvidesFile{theorem.dtx}[\FMithmInfo]
%\iffalse   % this is a METACOMMENT !
%%
%% Package `theorem' to use with LaTeX2e
%% Copyright (C) 1989-1995 Frank Mittelbach, all rights reserved.
%<+package>\@ifundefined{theorem@style}{}{\endinput}
%<+package>\NeedsTeXFormat{LaTeX2e}[1995/06/01]
%<+package>\ProvidesPackage{theorem}
%<+package>         [\FMithmInfo]
%
% \fi
%
% \changes{v1.0}{87/05/07}{First draft.}
% \changes{v1.0a}{87/05/07}{`thistheoremstyle' added.}
% \changes{v1.1a}{87/07/29}{New concept: redefine the internal
%                          \protect\LaTeX{} commands `@xnthm' etc.}
% \changes{v1.1b}{87/07/31}{`theoremfont' used to specify font.}
% \changes{v1.2a}{88/01/08}{`@endtheorem' now globally defined.}
% \changes{v1.2b}{88/01/11}{`newtheorem' only allowed in the preamble.}
% \changes{v1.3a}{88/02/14}{`theoremfont' now a toks register.}
%
% \changes{v2.0a}{88/11/16}{New implementation and new documentation.}
% \changes{v2.0a}{88/11/16}{No layout changes but internal modifications
%                           old substyles won't work.}
% \changes{v2.0b}{89/01/30}{`global' used for all definitions.}
% \changes{v2.0c}{89/06/15}{Documentation translated to English.
%                            Thanks to LLL and Nina}
% \changes{v2.1a}{90/02/15}{Support for new font selection.}
% \changes{v2.2a}{94/02/03}{Cleanup for LaTeX2e}
%
% \DoNotIndex{\@endparenv,\@ifdefinable,\@ifnextchar,\@ifundefined}
% \DoNotIndex{\@namedef,\@nameuse,\@spaces,\@tempa,\@warning}
% \DoNotIndex{\ }
% \DoNotIndex{\begingroup,\bfseries}
% \DoNotIndex{\csname}
% \DoNotIndex{\def,\do,\docdate}
% \DoNotIndex{\endcsname,\endgroup,\endinput,\endtrivlist,\expandafter}
% \DoNotIndex{\filedate,\fileversion}
% \DoNotIndex{\gdef,\global}
% \DoNotIndex{\hbox,\hskip}
% \DoNotIndex{\ignorespaces,\input,\itshape,\item}
% \DoNotIndex{\labelsep,\let,\llap}
% \DoNotIndex{\makeatletter}
% \DoNotIndex{\noexpand}
% \DoNotIndex{\refstepcounter,\relax,\rlap}
% \DoNotIndex{\setlength,\slshape,\space,\string,\strut}
% \DoNotIndex{\the,\trivlist,\typeout}
% \DoNotIndex{\vbox}
% \DoNotIndex{\xdef}
%
% \def\deflabel#1{\ttfamily #1\hfill}
% \def\deflist#1{\begin{list}{}{\settowidth\labelwidth{\ttfamily #1}%
%                               \setlength\leftmargin\labelwidth
%                               \addtolength\leftmargin\labelsep
%                               \let\makelabel\deflabel}}
% \def\enddeflist{\end{list}}
% \def\envfont{\normalfont\ttfamily}
%
% \MakeShortVerb{\|}
%
% \newcommand{\TUB}{{\sl TUGboat\/}}
%
% \GetFileInfo{theorem.dtx}
% \title{An Extension of the
%        \LaTeX{} theorem environment\thanks{This file
%        has version number \fileversion, last
%        revised \filedate.}}
% \author{Frank Mittelbach}
%
% \MaintainedByLaTeXTeam{tools}
% \maketitle
%
% \begin{abstract}
%    The macros described in this paper yield an extension of the
%    \LaTeX{} theorem mechanism. It is designed to satisfy the
%    different requirements of various journals. Thus, the
%    layout of the ``theorems'' can be manipulated by determining a
%    ``style''. This article describes not only the use, but
%    also the definition, of the necessary macros.
% \end{abstract}
%
% \section*{Preface to version 2.2}
%
% For \LaTeXe{} this package did not need any fundamental changes. I
% only modified the messages generated so that theorem layout styles
% will show up with the |\listfiles| command and cleaned the section
% on the New Font Selection Scheme since this is now included in
% \LaTeX.
%
% \section*{Preface to version 2.1}
%
% This version is identical to 2.0g described in \TUB\ 10\#3 except
% for some internal defaults which are now set depending on the used
% font selection scheme.
%
% This was done to avoid unpleasant surprises if the new font selection
% scheme is in force. For further details see
% section~\ref{sec:newfonts} and \cite{bk:GMS94}.
%
% \section{Introduction}
%
% For our purposes here, ``theorems'' are labelled enunciations,
% often set off from the main text by extra space and a font change.
% Theorems, corollaries, conjectures, definitions, and remarks
% are all instances of ``theorems''.  The ``header'' of these structures
% is composed of a label (such as \textsc{Theorem} or \textsc{Remark})
% and a number which serializes an item in the sequence of items
% with the same label.
%
% Shortly after the introduction of \LaTeX{} at the Fachbereich
% Mathematik in Mainz, the desire to manipulate the layout of
% ``theorems'' arose.  In Mainz, the following two conventions came
% into general use:
% \begin{enumerate}
%   \item The number of the theorem is shown in the margin.
%   \item There is a line break at
%         the end of the theorem header.
% \end{enumerate}
%
% Additionally, some journals require different formats which depend on
% the ``sort of theorem'': e.g.\ often remarks and definitions are set
% in |\upshape|, while |\itshape| is employed for main theorems.
%
% Confronted with these requirements, a theorem environment was
% developed in Mainz which allows separate determination of the
% layout of the ``theorems sets'', comparable to |\pagestyle|.
%
%
% \section{The user interface}
% \subsection{Defining new theorem sets}
%
% \DescribeMacro\newtheorem
% As in the original \LaTeX{} version, the command |\newtheorem|
% defines a new ``theorem set'' or ``theorem-like structure''.
% Two required arguments name the new environment and give the text
% to be typeset with each instance of the new ``set'', while
% an optional argument determines how
% the ``set'' is enumerated:
% \begin{description}
%    \item[\ttfamily \bslash newtheorem\{foo\}\{bar\}]
%       The theorem set {\envfont foo} (whose name is \texttt{bar})
%       uses its own counter.
%    \item[\ttfamily \bslash newtheorem\{foo2\}{[foo]}\{bar2\}]
%       The theorem set {\envfont foo2} (printed name \texttt{bar2})
%       uses the same counter as the theorem set \texttt{foo}.
%    \item[\ttfamily \bslash newtheorem\{foo3\}\{bar3\}{[section]}]
%       The theorem set {\envfont foo3} (printed name \texttt{bar3}) is
%       enumerated within the counter \texttt{section}, i.e.\ with every
%       new |\section| the enumeration begins again with 1, and
%       the enumeration is composed from the section-number and the
%       theorem counter itself.
% \end{description}
%
% \DescribeMacro\theoremstyle
% Additionally, the command |\theoremstyle| can define the layout
% of various, or all, theorem sets.  It should be noted that any
% theorem set defined by |\newtheorem| is typeset in the
% |\theoremstyle| that is current at the time of the definition.
% Thus, the following
% \begin{verbatim}
%    \theoremstyle{break}        \newtheorem{Cor}{Corollary}
%    \theoremstyle{plain}        \newtheorem{Exa}{Example}[section]
%\end{verbatim}                             ^^A No blank at line start
% leads to the result that the set {\envfont Cor} is formatted in the
% style \texttt{break}, while the set {\envfont Exa} and all the
% following ones are formatted in the style \texttt{plain}, unless
% another |\theoremstyle| follows.  Since the definitions installed by
% |\newtheorem| are global, one also can limit |\theoremstyle| locally
% by grouping braces.
%
% \DescribeMacro\theorembodyfont
% The choice of the font for the theorem body is completely
% independent of the chosen |\theoremstyle|; this has proven to
% be very advantageous.  For example,
% \begin{verbatim}
%   {\theorembodyfont{\upshape}          \newtheorem{Rem}{Remark}}
%\end{verbatim}
% defines a theorem set {\envfont Rem}, which will be set in
% |\upshape| in the current layout (which in our example is
% \texttt{plain}).  As with |\theoremstyle|, the |\theorembodyfont|
% chosen is that current at the time of |\newtheorem|.  If
% |\theorembodyfont| is not specified or one defines
% |\theorembodyfont{}|, then the font used will be that defined by the
% |\theoremstyle|.
%
% \DescribeMacro\theoremheaderfont It is also possible to customize
% the font used for the theorem headers.  This is, however, a global
% declaration, and therefore there should be at most one
% |\theoremheaderfont| declaration in the preamble.\footnote{If it is
% actually necessary to have different header fonts, one has to define
% new theorem styles (substituting the desired font) or specify the
% information directly in the \texttt{\bslash newtheorem} declaration
% (the unclean variant).}
%
% \DescribeMacro\theorempreskipamount
% \DescribeMacro\theorempostskipamount Two additional parameters
% affect the vertical space around the theorem environments:
% |\theorempreskipamount| and |\theorempostskipamount| define,
% respectively, the spacing before and after such an environment.
% These parameters apply for all theorem sets and can be manipulated
% with the ordinary length macros.  They are rubber lengths
% (`\textsf{skips}'), and therefore can contain \texttt{plus} and
% \texttt{minus} parts.
%
% Since the definition of theorem sets should---most sensibly---be
% placed in the preamble, we only allow installation there.
% It is therefore possible to release the memory used here after
% |\begin{document}|, in order to make room for other
% applications.
%
%
% \subsection{Existing theorem styles}
%
% The following theorem styles exist to date:
% \begin{deflist}{marginbreak}
%    \item[plain]
%       This theorem style emulates the original \LaTeX{} definition,
%       except that additionally the parameters
%       |\theorem...skipamount| are used.
%    \item[break]
%       In this style, the theorem header is followed by a line
%       break.
%    \item[marginbreak]
%       The theorem number is set in the margin, and there is a line
%       break as in \texttt{break}.
%    \item[changebreak]
%       Like \texttt{break}, but with header number and text
%       interchanged.
%    \item[change]
%       Header number and text are interchanged, without a line break.
%    \item[margin]
%       The number is set in the left margin, without a line break.
% \end{deflist}
% All styles (except \texttt{plain}) select |\slshape| as the default
% |\theorembodyfont|.
%
%
% \subsection{Examples}
%
% Given the above theorem sets {\envfont Cor}, {\envfont Exa}
% and {\envfont Rem},
% suppose that the preamble also contains the declarations:
% \begin{verbatim}
%    \theoremstyle{marginbreak}   \newtheorem{Lem}[Cor]{Lemma}
%    \theoremstyle{change}
%    \theorembodyfont{\itshape}        \newtheorem{Def}[Cor]{Definition}
%
%    \theoremheaderfont{\scshape}
%\end{verbatim}
% Then the following are some typical examples of the typeset output
% resulting from their use.
%
% \begin{Cor}
%    This is a sentence typeset in the theorem environment {\envfont
%    Cor}.
% \end{Cor}
%
% \begin{Exa}
%    This is a sentence typeset in the theorem environment {\envfont
%    Exa}.
% \end{Exa}
%
% \begin{Rem}
%    This is a sentence typeset in the theorem environment {\envfont
%    Rem}.
% \end{Rem}
%
% \begin{Lem}[Ben User]
%    This is a sentence typeset in the theorem environment {\envfont
%    Lem}.
% \end{Lem}
%
% \begin{Def}[Very impressive Definition]
%    This is a sentence typeset in the theorem environment {\envfont
%    Def}.
% \end{Def}
%
% The last two examples show the effect of the optional argument to
% a theorem environment (it is the text typeset in parentheses).
%
%
% \section{Special Considerations}
% \label{sec:newfonts}
%
% Theoremheader and body are implemented as a unit. This means that
% the |\theoremheaderfont|  will inherit characteristics of the
% |\theorembodyfont| in \LaTeXe. Thus, if for example
% |\theorembodyfont| is |\itshape| and |\theoremheaderfont|
% is |\bfseries| the font selected for the header will have the
% characteristics `bold extended italic'. If this is not desired one
% should set the |\theoremheaderfont| to something like
% \begin{verbatim}
%      \theoremheaderfont{\normalfont\bfseries}
%\end{verbatim}
% i.e.\ supplying all necessary font information explicitly.
%
%
%
% \section{Acknowledgements}
%
% The publication of this set of macros was only possible with the
% help of Christina Busse (translating the manuscript into English),
% Joachim Pense (playing the r\^ole of typist), Chris Rowley
% (looking everything over) and many others providing useful
% suggestions.
%
% \MaybeStop{%
%    \begin{thebibliography}{9}
%       \bibitem{bk:GMS94} \textsc{M.~Goossens}, \textsc{F.~Mittelbach}
%         and \textsc{A.~Samarin}.
%         \newblock The \LaTeX{} Companion.
%         \newblock
%         Addison-Wesley, Reading, Massachusetts, 1994.
%       \bibitem{latex:source} \textsc{Lamport, Leslie.} \newblock
%                               \texttt{latex.tex}, version 2.09, date
%                               Feb. 1990.
%    \end{thebibliography}
%    \PrintIndex
%    \PrintChanges}
%
%
% \section{The documentation driver file}
%
% The next bit of code contains the documentation driver file for
% \TeX{}, i.e., the file that will produce the documentation you are
% currently reading. It will be extracted from this file by the
% \texttt{docstrip} program. Since it is the first code in the file
% one can alternatively process this file directly with \LaTeXe{} to
% obtain the documentation.
%
%    \begin{macrocode}
%<*driver>
\documentclass{ltxdoc}

\usepackage{theorem}

%<+driver>% The next few lines define theorem sets which are used
%<+driver>% in the example section of the documentation.

\theoremstyle{break}       \newtheorem{Cor}{Corollary}
\theoremstyle{plain}       \newtheorem{Exa}{Example}[section]
{\theorembodyfont{\upshape}\newtheorem{Rem}{Remark}}
\theoremstyle{marginbreak} \newtheorem{Lem}[Cor]{Lemma}
\theoremstyle{change}
\theorembodyfont{\itshape} \newtheorem{Def}[Cor]{Definition}

\theoremheaderfont{\scshape}

\RecordChanges

\begin{document}
   \DocInput{theorem.dtx}
\end{document}
%</driver>
%    \end{macrocode}
%
% \section{Definition of the Macros}
%
% If the file has been loaded before, we abort immediately. If not the
% package announces itself (this is actually done at the very top of
% the file---the way it is done isn't good style so don't copy it).
% \changes{v2.0e}{89/07/19}{Spaces removed from `typeout'.}
% \changes{v2.2b}{95/11/19}{Announce moved to top of file}
%    \begin{macrocode}
%<*package>
%\@ifundefined{theorem@style}{}{\endinput}
%\def\FMithmInfo{1995/11/19 v2.2b Theorem extension package (FMi)}
%\ProvidesPackage{theorem}[\FMithmInfo]
%    \end{macrocode}
%
%
% \subsection{Definition of theorem styles and fonts}
%
% All the definitions in this file are done globally to allow
% inputting this file inside a group.
%
% \begin{macro}{\theoremstyle}
%    Before a theorem style can be installed, the chosen style must be
%    known. For that reason, we must test to see that
%    |\th@|\meta{style} is known or, more precisely, that it is
%    different from |\relax|. If the style is not known then
%    |\th@plain| is used.
%    \begin{macrocode}
\gdef\theoremstyle#1{%
   \@ifundefined{th@#1}{\@warning
          {Unknown theoremstyle `#1'. Using `plain'}%
          \theorem@style{plain}}%
%    \end{macrocode}
%    We save the theorem style to be used in the token register
%    |\theorem@style|.
%    \begin{macrocode}
      {\theorem@style{#1}}%
%    \end{macrocode}
%    Now we ``evaluate'' the theorem style: this means, we call the
%    macro |\th@|\meta{style} which will activate the relevant
%    definitions which are contained in a separate file.  This is done
%    in a group to suppress changes to the current font.  This could
%    otherwise pose problems together with the new font selection
%    scheme\footnote{When I printed the original article using the new
%    font selection scheme I ended with a document with slanted
%    typefaces (text headings and all) simply because one of the
%    theorem styles used \texttt{\bslash sl} at toplevel.} if the
%    |\th@|\meta{style} is evaluated a second time.
% \changes{v2.1a}{90/02/15}{Group added around `th@..}
% \changes{v2.1a}{90/02/15}{Reseting of `..@begintheorem removed.}
%    \begin{macrocode}
      \begingroup
        \csname th@\the\theorem@style \endcsname
      \endgroup}
%    \end{macrocode}
% \end{macro}
%
%
% \begin{macro}{\@begintheorem}
% \changes{v2.1a}{90/02/15}{Relaxed at toplevel.}
% \begin{macro}{\@opargbegintheorem}
% \changes{v2.1a}{90/02/15}{Relaxed at toplevel.}
%    We reset |\@begintheorem| and |\@opargbegintheorem| to
%    |\relax| since these commands are no longer necessary at
%    toplevel. This will save a few tokens.
%    \begin{macrocode}
\global\let\@begintheorem\relax
\global\let\@opargbegintheorem\relax
%    \end{macrocode}
% \end{macro}
% \end{macro}
%
% \begin{macro}{\theorem@style}
%    Obviously the \textsf{token} register used above has to be
%    allocated. To assure the utmost compatibility with the
%    original \LaTeX{} definition, we set the default theorem style to
%    \texttt{plain}, which implements the usual \LaTeX{} convention.
% \changes{v2.0f}{89/07/20}{Unnecessary `global' removed.}
%    \begin{macrocode}
\newtoks\theorem@style
\global\theorem@style{plain}
%    \end{macrocode}
% \end{macro}
%
% \begin{macro}{\theorembodyfont}
% \begin{macro}{\theorem@bodyfont}
%    If the |\theorembodyfont| is set by the user then it should not
%    interact with the default font set in the theorem style. When the
%    new font selection is in force this may happen if, for example,
%    the default is |\itshape| and the new |\theorembodyfont| is
%    |\sffamily|. So we add a |\reset@font| command in front of the
%    user definition.
% \changes{v2.0f}{89/07/20}{Unnecessary `global' removed.}
% \changes{v2.0g}{89/09/19}{`theoremfont to `theorembodyfont.}
% \changes{v2.1c}{92/06/22}{`reset@font added and logic changed.}
% \changes{v2.1d}{92/10/28}{Reset to empty if arg is empty}
%    \begin{macrocode}
\gdef\theorembodyfont#1{%
%    \end{macrocode}
%    We check if the argument supplied is empty and if so put nothing
%    into the |\theorem@bodyfont| token register to allow for
%    |\theorembodyfont{}| as a mean of using the default of the
%    current |\theoremstyle|.
%    \begin{macrocode}
   \def\@tempa{#1}%
   \ifx\@tempa\@empty
    \theorem@bodyfont{}%
   \else
    \theorem@bodyfont{\reset@font#1}%
   \fi
   }
\newtoks\theorem@bodyfont
\global\theorem@bodyfont{}
%    \end{macrocode}
% \end{macro}
% \end{macro}
%
%
% \begin{macro}{\theoremheaderfont}
% \changes{v2.0g}{89/09/19}{Macro added.}
%    The font for the theorem headers is handled differently because
%    this definition applies to all theorem styles.
%    \begin{macrocode}
\gdef\theoremheaderfont#1{\gdef\theorem@headerfont{#1}%
%    \end{macrocode}
%    After using the macro once it is redefined to produce an error
%    message.
%    \begin{macrocode}
       \gdef\theoremheaderfont##1{%
        \typeout{\string\theoremheaderfont\space should be used
                 only once.}}}
%    \end{macrocode}
% \end{macro}
%
% \begin{macro}{\theorem@headerfont}
% \changes{v2.0g}{89/09/19}{Macro added.}
%    To set the |\theorem@headerfont| default we first test if
%    the new fontselection scheme is in force.
% \changes{v2.1a}{90/02/15}{Support for new font selection.}
%    \begin{macrocode}
\ifx\upshape\undefined
%    \end{macrocode}
%    If not we define it to expand into |\bfseries|. We don't use
%    |\let| just in case a following style option redefines this
%    macro.
%    \begin{macrocode}
\gdef\theorem@headerfont{\bfseries}
%    \end{macrocode}
%    Otherwise we reset the current shape before calling |\bfseries|.
%    \begin{macrocode}
\else \gdef\theorem@headerfont{\normalfont\bfseries}\fi
%    \end{macrocode}
% \end{macro}
%
% \begin{macro}{\th@plain}
% \begin{macro}{\th@break}
% \begin{macro}{\th@marginbreak}
% \begin{macro}{\th@changebreak}
% \begin{macro}{\th@change}
% \begin{macro}{\th@margin}
%    The different styles are defined in macros such as
%    |\th@plain|.  Since memory space is precious in
%    ``non-Big-versions'', we have to avoid offering too many unused
%    definitions.  Therefore we define these styles in separate files
%    that can be loaded on demand.  Thus the commands themselves only
%    load these files. We use |\@input@|, a \LaTeXe{} internal command
%    that ensures that the file will be listed with |\listfiles|.
% \changes{v2.2a}{94/02/02}{Use `@input@ to load theorem layout files}
%    \begin{macrocode}
\gdef\th@plain{\@input@{thp.sty}}
\gdef\th@break{\@input@{thb.sty}}
\gdef\th@marginbreak{\@input@{thmb.sty}}
\gdef\th@changebreak{\@input@{thcb.sty}}
\gdef\th@change{\@input@{thc.sty}}
\gdef\th@margin{\@input@{thm.sty}}
%    \end{macrocode}
%    This list will be expanded when new styles become
%    available.  For testing,  just append new theorem substyles
%    as document options.
% \end{macro}
% \end{macro}
% \end{macro}
% \end{macro}
% \end{macro}
% \end{macro}
%
%
%
% \subsection{Definition of a new theorem set}
%
% As already pointed out, a new theorem environment can be defined in
% three different ways:
% \begin{verbatim}
%    \newtheorem{Lem}{Lemma}
%    \newtheorem{Lem}{Lemma}[section]
%    \newtheorem{Lem}[Theorem]{Lemma}
%\end{verbatim}
% The function of the macro |\newtheorem| is to recognize these cases
% and then to branch into one of the three macros |\@ynthm|, |\@xnthm|
% or |\@othm|.  This mechanism is adopted unchanged from
% \cite{latex:source}; the essential point here is that, for example,
% in the second case, the arguments \texttt{Lem}, \texttt{Lemma} and
% \texttt{section} are passed over to the macro |\@xnthm|.
%
% We inspect this case first because the others present fewer problems,
% and thus are easily derived from this one.
%
% \begin{macro}{\@xnthm}
%    For our example arguments, the macro |\@xnthm| must fulfill
%    the following:
%    \begin{itemize}
%       \item
%          Define a new \LaTeX-counter `Lem'
%       \item
%          reset this counter within a |\section|
%       \item
%          define the macro |\theLem|
%       \item
%          define the environment macros |\Lem| and
%          |\endLem| using the current
%          |\theoremstyle| and |\theorem@bodyfont|.
%    \end{itemize}
%    Obviously, all this should happen only if the first argument
%    of |\@xnthm| (i.e.\ \texttt{Lem} in our example) is chosen so
%    as not to conflict with any previously defined commands or
%    environments.  This test is performed by the \LaTeX{} macro
%    |\@ifdefinable|.
%    \begin{macrocode}
\gdef\@xnthm#1#2[#3]{\expandafter\@ifdefinable\csname #1\endcsname
%    \end{macrocode}
%    Therefore, the first argument of |\@ifdefinable| is the
%    expansion (in the example, |\Lem|) of
%    |\csname||#1\endcsname|.  The second argument is
%    executed only if the test has been completed successfully.
%    \begin{macrocode}
   {%
%    \end{macrocode}
%    Now we define the new counter. The names of the \LaTeX{}
%    macros employed should speak for themselves:
% \changes{v2.2b}{1995/10/07}{Use \cs{@newctr}}
% \changes{v2.2c}{1995/11/23}{Use \cs{@newctr} properly.}
%    \begin{macrocode}
    \@definecounter{#1}\@newctr{#1}[#3]%
%    \end{macrocode}
%    Using |\@newctr| will give a proper error message if the counter
%    in |#3| is not defined.
%    In defining `|\theLem|' we must generate the desired macro
%    name by use of |\expandafter| and |\csname|.
%    \begin{macrocode}
    \expandafter\xdef\csname the#1\endcsname
%    \end{macrocode}
%    An |\xdef| is used in order to make the definition global,
%    and to ensure that it contains the replacement texts of
%    |\@thmcountersep| and |\@thmcounter|.\footnote{These
%    two macros can be defined by the document style.  Their default
%    values produce a `\texttt{.}' as separation and an arabic
%    representation of the number.} However, not everything should be
%    expanded. For example, it saves space to use |\thesection|
%    instead of its---at times---lengthy expansion.
%    \begin{macrocode}
      {\expandafter \noexpand \csname the#3\endcsname
       \@thmcountersep \@thmcounter{#1}}%
%    \end{macrocode}
%    Thus with the defaults of \LaTeX{}, |\theLem| would be
%    replaced by the command sequence |\thesection.\arabic{Lem}|.
%
%    We will now look at the definition of the macro which is executed
%    at the beginning of the actual environment (in our example this
%    macro is |\Lem|). It should be noted that we use an
%    ``|\expandafter| trick'' to expand only certain parts of the
%    replacement text at the time of the definition.
%    \begin{macrocode}
    \def\@tempa{\global\@namedef{#1}}%
    \expandafter \@tempa \expandafter{%
%    \end{macrocode}
%    First, the macro that contains the current definitions of
%    |\@begintheorem| and |\@opargtheorem| should be called
%    up.  The name of this macro---as is already known---has the form
%    |\th@|\meta{theorem\ style}; therefore, it must be called by
%    \begin{macrocode}
      \csname th@\the \theorem@style
%    \end{macrocode}
%    In addition the default theorem font should be changeable, i.e.\
%    we have to insert the contents of |\theorem@bodyfont|. For
%    that reason, we expand even further, beyond |\endcsname|,
%    and thus insert the contents of the \textsf{token} register
%    |\theorem@bodyfont| in the replacement text.
%    \begin{macrocode}
            \expandafter \endcsname \the \theorem@bodyfont
%    \end{macrocode}
%    Now it is time to call the macro |\@thm| which takes over
%    the further processing.  It has two arguments: the current
%    counter name (in our example, \texttt{Lem}), and the text of the
%    label (in our example, \texttt{Lemma}).
%    \begin{macrocode}
     \@thm{#1}{#2}}%
%    \end{macrocode}
%    With this, the `sub-definition' is complete.  The macro
%    |\@endtheorem| ends a theorem environment and is, so far, nothing
%    but an |\endtrivlist|.  (Hence it is defined globally, and not
%    within the theorem styles.\footnote{This has to be changed as
%    soon as theorem styles that change \texttt{\bslash @endtheorem}
%    exist.  In such a case, all existing styles must be changed as
%    well since they will have to reset the macro.}) Therefore, we can
%    set it equivalent to the macro that ends the theorem set (in our
%    example, |\endLem|).  However, if some day theorem styles exist
%    that do change |\@endtheorem|, we would have to use the
%    commented-out line instead.
%    \begin{macrocode}
    \global \expandafter \let \csname end#1\endcsname \@endtheorem
%   \global\@namedef{end#1}{\@endtheorem}%
%    \end{macrocode}
%    With these commands all the required definitions are employed,
%    unless the test |\@ifdefinable| has failed.  Therefore, we
%    end the second argument of this macro and with it the definition
%    of |\@xnthm|.
%    \begin{macrocode}
   }}
%    \end{macrocode}
% \end{macro}
%
% \begin{macro}{\@ynthm}
%    The definition of |\@ynthm| is completely analogous. In
%    this case the new counter that is defined is not reset within
%    another counter; thus the definition of |\the...|
%    is simplified:
%    \begin{macrocode}
\gdef\@ynthm#1#2{\expandafter\@ifdefinable\csname #1\endcsname
   {\@definecounter{#1}%
    \expandafter\xdef\csname the#1\endcsname{\@thmcounter{#1}}%
%    \end{macrocode}
%    The rest of the definition corresponds literally to that of
%    |\@xnthm|:
%    \begin{macrocode}
    \def\@tempa{\global\@namedef{#1}}\expandafter \@tempa
     \expandafter{\csname th@\the \theorem@style \expandafter
     \endcsname \the\theorem@bodyfont \@thm{#1}{#2}}%
    \global \expandafter \let \csname end#1\endcsname \@endtheorem}}
%    \end{macrocode}
% \end{macro}
%
% \begin{macro}{\@othm}
%    The definition of |\@othm| does not contain anything new.
%    \begin{macrocode}
\gdef\@othm#1[#2]#3{%
%    \end{macrocode}
% \changes{v2.2b}{95/10/07}{Build in check for counter defined}
%    We do not define a new counter but instead use one that has
%    already been defined.  Thus the only definition we need is that
%    of this pseudo-counter (i.e.\ |\the|\meta{env.\ name}).
%    First we check if |#2| corresponds to a known counter name.
%    \begin{macrocode}
  \expandafter\ifx\csname c@#2\endcsname\relax
   \@nocounterr{#2}%
  \else
   \expandafter\@ifdefinable\csname #1\endcsname
   {\expandafter \xdef \csname the#1\endcsname
     {\expandafter \noexpand \csname the#2\endcsname}%
%    \end{macrocode}
%    All other parts of the definition can be adopted from
%    |\@xnthm|. We have to remember, though, that in this case
%    the name of the current counter and the theorem label have moved
%    to the second and third arguments.
%    \begin{macrocode}
    \def\@tempa{\global\@namedef{#1}}\expandafter \@tempa
     \expandafter{\csname th@\the \theorem@style \expandafter
     \endcsname \the\theorem@bodyfont \@thm{#2}{#3}}%
    \global \expandafter \let \csname end#1\endcsname \@endtheorem}%
  \fi}
%    \end{macrocode}
% \end{macro}
%
% \subsection{Macros that are employed in a {\envfont theorem}
%    environment}
%
% \begin{macro}{\@thm}
%    The macro |\@thm| has to increase the current counter.  Then,
%    depending on whether the environment has (or does not have) an
%    optional argument, it has to branch into either |\@begintheorem|
%    or |\@opargtheorem|.
%    \begin{macrocode}
\gdef\@thm#1#2{\refstepcounter{#1}%
%    \end{macrocode}
%    Now we start a {\envfont trivlist} environment, and give
%    |\@topsep| and |\@topsepadd| the values of the \textsf{skip}
%    registers |\theorempreskipamount| and |\theorempostskipamount|.
%    The value in |\@topsep| is the vertical space that is inserted by
%    the first (and only) |\item| in our |\trivlist| whilst
%    |\@topsepadd| is inserted by |\@endparenv| at the end of that
%    {\envfont trivlist} environment.  By using these registers, we
%    obtain the desired space around a {\envfont theorem} environment.
%    \begin{macrocode}
   \trivlist
   \@topsep \theorempreskipamount               % used by first \item
   \@topsepadd \theorempostskipamount           % used by \@endparenv
%    \end{macrocode}
%    Now we have to test whether an optional argument has been given.
% \changes{v2.0d}{89/07/18}{Forgotten percent added.}
%    \begin{macrocode}
   \@ifnextchar [%
%    \end{macrocode}
%    If there is an optional argument, we will call |\@ythm|, and
%    move the arguments read back into the input stream.
%    \begin{macrocode}
   {\@ythm{#1}{#2}}%
%    \end{macrocode}
%    If not, we call |\@begintheorem|. Its first argument is the
%    name of the theorem set (hence the second argument of
%    |\@thm|). Its second argument is the macro that produces the
%    current number.
%    \begin{macrocode}
   {\@begintheorem{#2}{\csname the#1\endcsname}\ignorespaces}}
%    \end{macrocode}
% \end{macro}
%
% \begin{macro}{\@xthm}
% \begin{macro}{\@ythm}
%    Both these macros were originally called by |\@thm|.  We do
%    not need |\@xthm| anymore, hence we reset it to
%    |\relax|. The definition of |\@ythm| has not changed at
%    all from its definition in \LaTeX. In order to make the macros
%    easier to understand, we will
%    nevertheless present it (commented out).
%    \begin{macrocode}
\global\let\@xthm\relax
% \def\@ythm#1#2[#3]{\@opargbegintheorem{#2}{\csname
%           the#1\endcsname}{#3}\ignorespaces}
%</package>
%    \end{macrocode}
%    The primitive |\ignorespaces| in |\@ythm| and
%    |\@thm| is needed to remove the spaces between the
%    |\begin{...}| and the actual text.
% \end{macro}
% \end{macro}
%
%
% \subsection{Definition of the theorem substyles}
%
% As already pointed out, the theorem substyles, defined below, are
% only loaded when necessary.  Note that all these substyles, except
% \texttt{plain}, have |\slshape| as the default body font.
%
% \subsubsection{The \texttt{plain} style}
%
% \label{dok:th@plain} As the following macros use |@|, we have
% to locally set the |\catcode| of this symbol to ``letter''.
% This happens within a group, so that we do not have to worry about
% which |\catcode| that symbol had before.
%    \begin{macrocode}
%<*thp>
\begingroup \makeatletter
%    \end{macrocode}
% Since we are now within a group, we must make all definitions
% globally.  First we make sure that \texttt{theorem.sty} is loaded.
% This will allow us to use this file as a document style option
% without having to call \texttt{theorem} itself as an option.  At the
% same time, we assure that at least version 2 is loaded, since
% |\theorem@style| was not defined in earlier versions.
%    \begin{macrocode}
\@ifundefined{theorem@style}{\input{theorem.sty}}{}
\ProvidesFile{thp.sty}
             [\FMithmInfo]
%    \end{macrocode}
%
%
% \begin{macro}{\th@plain}
% \begin{macro}{\@begintheorem}
% \begin{macro}{\@opargbegintheorem}
%    |\theoremstyle{plain}| corresponds to the original
%    definition, except that the distances to the surrounding text are
%    determined by the parameters |\theorempreskipamount| and
%    |\theorempostskipamount|.
%    First we set the default body font.
%    \begin{macrocode}
\gdef\th@plain{\normalfont\itshape
%    \end{macrocode}
%    Then we define |\@begintheorem| and
%    |\@opargbegintheorem|. These two macros define how the
%    header of a theorem is typeset.  |\@opargbegintheorem| will
%    be called if a \textsf{theorem} environment with an optional
%    argument is encountered; otherwise, the header is constructed by
%    calling |\@begintheorem|.  If one of these macros is
%    executed, we are within a \textsf{trivlist} environment started by
%    |\@thm|.  So the theorem header is produced with an
%    |\item| command.
%
%    Instead of specifying the header font directly, all standard
%    theorem styles use the |\theorem@headerfont| macro to allow
%    customization.  The extra space (|\labelsep|) is necessary
%    because of problems in the \textsf{trivlist} environment.
%    \begin{macrocode}
  \def\@begintheorem##1##2{%
        \item[\hskip\labelsep \theorem@headerfont ##1\ ##2]}%
%    \end{macrocode}
%    The definition of |\@opargbegintheorem| is completely
%    analogous. The only difference is the fact that there exists a
%    third argument (which is the optional parameter of the
%    environment and contains additional information about the
%    theorem).  Customarily we enclose it in parentheses.
%    \begin{macrocode}
\def\@opargbegintheorem##1##2##3{%
   \item[\hskip\labelsep \theorem@headerfont ##1\ ##2\ (##3)]}}
%    \end{macrocode}
% \end{macro}
% \end{macro}
% \end{macro}
%
% \noindent
% We conclude with an |\endgroup| to restore the
% |\catcode| of |@|.
%    \begin{macrocode}
\endgroup
%</thp>
%    \end{macrocode}
%
%
% \subsubsection{The \texttt{break} style}
%
% This style option is stored in the file |thb.sty|.
% For the next two lines see the documentation for |\th@plain| on
% page~\pageref{dok:th@plain}.
%    \begin{macrocode}
%<*thb>
\begingroup \makeatletter
\@ifundefined{theorem@style}{\input{theorem.sty}}{}
\ProvidesFile{thb.sty}
             [\FMithmInfo]
%    \end{macrocode}
%
% \begin{macro}{\th@break}
%    |\theoremstyle{break}| produces a line break after the name
%    of the theorem.  The font is |\slshape|.  Hence, we define
%    |\th@break| as follows:
%    \begin{macrocode}
\gdef\th@break{\normalfont\slshape
  \def\@begintheorem##1##2{\item[%
%    \end{macrocode}
%    We run into the following problem: it is not possible to create
%    the header with |\item[|\meta{title}|]| and then start
%    a new line by, for example, |\mbox{}\\|. Such a definition
%    will fail whenever a list environment follows immediately.  With
%    the above construction, the |\mbox{}| causes the switch
%    |@inlabel| (cf.\ definition of |\list| and
%    |\trivlist| in \cite{latex:source}) to be set to \texttt{false}
%    and so the following list will insert additional vertical
%    space (|\topskip|). This is quite annoying. Therefore, we
%    create the line break within the |\item|. In order to ensure
%    that the text will begin at the proper position in the following
%    line, we simply pretend that the label does not take any
%    room.\footnote{This will lead to problems whenever very high
%    symbols occurring in the line tower into the
%    heading. So, something else has to be done here sometime.}
%    \begin{macrocode}
       \rlap{\vbox{\hbox{\hskip \labelsep\theorem@headerfont ##1\ ##2}%
                   \hbox{\strut}}}]}%
%    \end{macrocode}
%    Again, the definition of |\@opargbegintheorem| is
%    completely analogous.
%    \begin{macrocode}
\def\@opargbegintheorem##1##2##3{%
  \item[\rlap{\vbox{\hbox{\hskip \labelsep \theorem@headerfont
                     ##1\ ##2\ (##3)}%
                    \hbox{\strut}}}]}}
\endgroup
%</thb>
%    \end{macrocode}
% \end{macro}
%
%
% \subsubsection{The \texttt{changebreak} style}
%
%    \begin{macrocode}
% This style option is stored in the file |thcb.sty|.
%    \begin{macrocode}
%<*thcb>
\begingroup \makeatletter
\@ifundefined{theorem@style}{\input{theorem.sty}}{}
\ProvidesFile{thcb.sty}
             [\FMithmInfo]
%    \end{macrocode}
%
% \begin{macro}{\th@changebreak}
%    The \texttt{change-break} theorem style is like \texttt{break}
%    but with interchange of theorem name and theorem number.  Thus we
%    define |\th@changebreak| as follows:
%    \begin{macrocode}
\gdef\th@changebreak{\normalfont\slshape
  \def\@begintheorem##1##2{\item
       [\rlap{\vbox{\hbox{\hskip\labelsep \theorem@headerfont ##2\ ##1}%
                    \hbox{\strut}}}]}%
\def\@opargbegintheorem##1##2##3{%
  \item[\rlap{\vbox{\hbox{\hskip\labelsep \theorem@headerfont
                     ##2\ ##1\ (##3)}%
                    \hbox{\strut}}}]}}
\endgroup
%</thcb>
%    \end{macrocode}
% \end{macro}
%
%
% \subsubsection{The \texttt{change} style}
%
% This style option is stored in the file |thc.sty|.
%    \begin{macrocode}
%<*thc>
\begingroup \makeatletter
\@ifundefined{theorem@style}{\input{theorem.sty}}{}
\ProvidesFile{thc.sty}
             [\FMithmInfo]
%    \end{macrocode}
%
% \begin{macro}{\th@change}
%    The \texttt{change} theorem style corresponds to the
%    \texttt{change break} style without a linebreak after the header.
%    To say it in another way, it's the same as the \texttt{plain}
%    style but with number and name interchanged and |\slshape| as the
%    default font.
%    \begin{macrocode}
\gdef\th@change{\normalfont\slshape
  \def\@begintheorem##1##2{\item
       [\hskip\labelsep \theorem@headerfont ##2\ ##1]}%
\def\@opargbegintheorem##1##2##3{%
  \item[\hskip\labelsep \theorem@headerfont ##2\ ##1\ (##3)]}}
\endgroup
%</thc>
%    \end{macrocode}
% \end{macro}
%
%
% \subsubsection{The \texttt{marginbreak} style}
%
% This style option is the one used most often at Mainz.  It is saved
% in the file |thmb.sty|.
%    \begin{macrocode}
%<*thmb>
\begingroup \makeatletter
\@ifundefined{theorem@style}{\input{theorem.sty}}{}
\ProvidesFile{thmb.sty}
             [\FMithmInfo]
%    \end{macrocode}
%
% \begin{macro}{\th@marginbreak}
%    The \texttt{margin break} style is nearly the same as the
%    \texttt{change break} style. The only difference is the placement
%    of the theorem number.  We use |\llap| to place it in the left
%    margin.
%
%    In this style |\labelsep| denotes the separation between the
%    number and the text.
%    \begin{macrocode}
\gdef\th@marginbreak{\normalfont\slshape
  \def\@begintheorem##1##2{\item
       [\rlap{\vbox{\theorem@headerfont
                    \hbox{\llap{##2}\hskip\labelsep ##1}%
                    \hbox{\strut}}}]}%
\def\@opargbegintheorem##1##2##3{%
  \item[\rlap{\vbox{\theorem@headerfont
                    \hbox{\llap{##2}\hskip\labelsep ##1\ (##3)}%
                    \hbox{\strut}}}]}}
\endgroup
%</thmb>
%    \end{macrocode}
% \changes{v2.1c}{90/02/17}{`endgroup outside macrocode env.}
% \end{macro}
%
%
%
% \subsubsection{The \texttt{margin} style}
%
% This style option is stored in the file |thm.sty|.
%    \begin{macrocode}
%<*thm>
\begingroup \makeatletter
\@ifundefined{theorem@style}{\input{theorem.sty}}{}
\ProvidesFile{thm.sty}
             [\FMithmInfo]
%    \end{macrocode}
%
% \begin{macro}{\th@margin}
%    Again this is only a variant of the theorem styles described
%    above without any new ideas.
%    \begin{macrocode}
\gdef\th@margin{\normalfont\slshape
  \def\@begintheorem##1##2{\item
       [\theorem@headerfont \llap{##2}\hskip\labelsep ##1]}%
\def\@opargbegintheorem##1##2##3{%
  \item[\theorem@headerfont \llap{##2}\hskip\labelsep ##1\ (##3)]}}
\endgroup
%</thm>
%    \end{macrocode}
% \end{macro}
%
%
%
%
% \subsection{Final Definitions}
%
% \begin{macro}{\theorempreskipamount}
% \begin{macro}{\theorempostskipamount}
%    The \textsf{skip} parameters that regulate the vertical empty space
%    before and after the {\envfont theorem} environment have
%    to be allocated as well.
% \changes{v2.0f}{89/07/20}{Unnecessary `global' removed.}
%    \begin{macrocode}
%<*package>
\newskip\theorempreskipamount
\newskip\theorempostskipamount
%    \end{macrocode}
%    Since we have used the same values for all theorem sets, we now can
%    assign them.
%    \begin{macrocode}
\global\setlength\theorempreskipamount{12pt plus 5pt minus 3pt}
\global\setlength\theorempostskipamount{8pt plus 3pt minus 1.5pt}
%    \end{macrocode}
% \end{macro}
% \end{macro}
%
%
% \begin{macro}{\@endtheorem}
%    The same holds for the macro |\@endtheorem|, which ends a
%    {\envfont theorem} environment. Since it is the same for all
%    theorem sets, it is removed from the macros |\th@|\meta{style}.
%    It simply ends the {\envfont trivlist} environment, which was
%    begun in |\@thm|.
%    \begin{macrocode}
\global\let\@endtheorem=\endtrivlist
%    \end{macrocode}
% \end{macro}
%
%
%
% \begin{macro}{\@preamblecmds}
%    All macros defined above are to be used only in the preamble.
%    Therefore, we insert them in |\@preamblecmds| which will disable
%    them at begin document. This is done by the internal \LaTeXe{}
%    command |\@onlypreamble|.
% \changes{v2.0g}{89/09/19}{`theoremheaderfont and `theorembodyfont
%    added.}
% \changes{v2.2a}{94/02/03}{Use \LaTeXe{} interface}
%    \begin{macrocode}
\@onlypreamble\@xnthm
\@onlypreamble\@ynthm
\@onlypreamble\@othm
\@onlypreamble\newtheorem
\@onlypreamble\theoremstyle
\@onlypreamble\theorembodyfont
\@onlypreamble\theoremheaderfont
%    \end{macrocode}
% \end{macro}
%
% Finally we declare the \texttt{plain} theorem style to be the default.
% \changes{v2.1b}{90/06/04}{Fixed missing `item by declaring default
%    style.}
%    \begin{macrocode}
\theoremstyle{plain}
%</package>
%    \end{macrocode}
%
% \Finale

\endinput
}{}
\ProvidesFile{thp.sty}
             [\FMithmInfo]
%    \end{macrocode}
%
%
% \begin{macro}{\th@plain}
% \begin{macro}{\@begintheorem}
% \begin{macro}{\@opargbegintheorem}
%    |\theoremstyle{plain}| corresponds to the original
%    definition, except that the distances to the surrounding text are
%    determined by the parameters |\theorempreskipamount| and
%    |\theorempostskipamount|.
%    First we set the default body font.
%    \begin{macrocode}
\gdef\th@plain{\normalfont\itshape
%    \end{macrocode}
%    Then we define |\@begintheorem| and
%    |\@opargbegintheorem|. These two macros define how the
%    header of a theorem is typeset.  |\@opargbegintheorem| will
%    be called if a \textsf{theorem} environment with an optional
%    argument is encountered; otherwise, the header is constructed by
%    calling |\@begintheorem|.  If one of these macros is
%    executed, we are within a \textsf{trivlist} environment started by
%    |\@thm|.  So the theorem header is produced with an
%    |\item| command.
%
%    Instead of specifying the header font directly, all standard
%    theorem styles use the |\theorem@headerfont| macro to allow
%    customization.  The extra space (|\labelsep|) is necessary
%    because of problems in the \textsf{trivlist} environment.
%    \begin{macrocode}
  \def\@begintheorem##1##2{%
        \item[\hskip\labelsep \theorem@headerfont ##1\ ##2]}%
%    \end{macrocode}
%    The definition of |\@opargbegintheorem| is completely
%    analogous. The only difference is the fact that there exists a
%    third argument (which is the optional parameter of the
%    environment and contains additional information about the
%    theorem).  Customarily we enclose it in parentheses.
%    \begin{macrocode}
\def\@opargbegintheorem##1##2##3{%
   \item[\hskip\labelsep \theorem@headerfont ##1\ ##2\ (##3)]}}
%    \end{macrocode}
% \end{macro}
% \end{macro}
% \end{macro}
%
% \noindent
% We conclude with an |\endgroup| to restore the
% |\catcode| of |@|.
%    \begin{macrocode}
\endgroup
%</thp>
%    \end{macrocode}
%
%
% \subsubsection{The \texttt{break} style}
%
% This style option is stored in the file |thb.sty|.
% For the next two lines see the documentation for |\th@plain| on
% page~\pageref{dok:th@plain}.
%    \begin{macrocode}
%<*thb>
\begingroup \makeatletter
\@ifundefined{theorem@style}{% \iffalse meta-comment
%
% Copyright (C) 1993-2024
%
% The LaTeX Project and any individual authors listed elsewhere
% in this file.
%
% This file is part of the Standard LaTeX `Tools Bundle'.
% -------------------------------------------------------
%
% It may be distributed and/or modified under the
% conditions of the LaTeX Project Public License, either version 1.3c
% of this license or (at your option) any later version.
% The latest version of this license is in
%    https://www.latex-project.org/lppl.txt
% and version 1.3c or later is part of all distributions of LaTeX
% version 2005/12/01 or later.
%
% The list of all files belonging to the LaTeX `Tools Bundle' is
% given in the file `manifest.txt'.
%
% \fi
%
%%
%
%
\def\FMithmInfo{2023/07/05 v2.2c Theorem extension package (FMi)}
%
% \ProvidesFile{theorem.dtx}[\FMithmInfo]
%\iffalse   % this is a METACOMMENT !
%%
%% Package `theorem' to use with LaTeX2e
%% Copyright (C) 1989-1995 Frank Mittelbach, all rights reserved.
%<+package>\@ifundefined{theorem@style}{}{\endinput}
%<+package>\NeedsTeXFormat{LaTeX2e}[1995/06/01]
%<+package>\ProvidesPackage{theorem}
%<+package>         [\FMithmInfo]
%
% \fi
%
% \changes{v1.0}{87/05/07}{First draft.}
% \changes{v1.0a}{87/05/07}{`thistheoremstyle' added.}
% \changes{v1.1a}{87/07/29}{New concept: redefine the internal
%                          \protect\LaTeX{} commands `@xnthm' etc.}
% \changes{v1.1b}{87/07/31}{`theoremfont' used to specify font.}
% \changes{v1.2a}{88/01/08}{`@endtheorem' now globally defined.}
% \changes{v1.2b}{88/01/11}{`newtheorem' only allowed in the preamble.}
% \changes{v1.3a}{88/02/14}{`theoremfont' now a toks register.}
%
% \changes{v2.0a}{88/11/16}{New implementation and new documentation.}
% \changes{v2.0a}{88/11/16}{No layout changes but internal modifications
%                           old substyles won't work.}
% \changes{v2.0b}{89/01/30}{`global' used for all definitions.}
% \changes{v2.0c}{89/06/15}{Documentation translated to English.
%                            Thanks to LLL and Nina}
% \changes{v2.1a}{90/02/15}{Support for new font selection.}
% \changes{v2.2a}{94/02/03}{Cleanup for LaTeX2e}
%
% \DoNotIndex{\@endparenv,\@ifdefinable,\@ifnextchar,\@ifundefined}
% \DoNotIndex{\@namedef,\@nameuse,\@spaces,\@tempa,\@warning}
% \DoNotIndex{\ }
% \DoNotIndex{\begingroup,\bfseries}
% \DoNotIndex{\csname}
% \DoNotIndex{\def,\do,\docdate}
% \DoNotIndex{\endcsname,\endgroup,\endinput,\endtrivlist,\expandafter}
% \DoNotIndex{\filedate,\fileversion}
% \DoNotIndex{\gdef,\global}
% \DoNotIndex{\hbox,\hskip}
% \DoNotIndex{\ignorespaces,\input,\itshape,\item}
% \DoNotIndex{\labelsep,\let,\llap}
% \DoNotIndex{\makeatletter}
% \DoNotIndex{\noexpand}
% \DoNotIndex{\refstepcounter,\relax,\rlap}
% \DoNotIndex{\setlength,\slshape,\space,\string,\strut}
% \DoNotIndex{\the,\trivlist,\typeout}
% \DoNotIndex{\vbox}
% \DoNotIndex{\xdef}
%
% \def\deflabel#1{\ttfamily #1\hfill}
% \def\deflist#1{\begin{list}{}{\settowidth\labelwidth{\ttfamily #1}%
%                               \setlength\leftmargin\labelwidth
%                               \addtolength\leftmargin\labelsep
%                               \let\makelabel\deflabel}}
% \def\enddeflist{\end{list}}
% \def\envfont{\normalfont\ttfamily}
%
% \MakeShortVerb{\|}
%
% \newcommand{\TUB}{{\sl TUGboat\/}}
%
% \GetFileInfo{theorem.dtx}
% \title{An Extension of the
%        \LaTeX{} theorem environment\thanks{This file
%        has version number \fileversion, last
%        revised \filedate.}}
% \author{Frank Mittelbach}
%
% \MaintainedByLaTeXTeam{tools}
% \maketitle
%
% \begin{abstract}
%    The macros described in this paper yield an extension of the
%    \LaTeX{} theorem mechanism. It is designed to satisfy the
%    different requirements of various journals. Thus, the
%    layout of the ``theorems'' can be manipulated by determining a
%    ``style''. This article describes not only the use, but
%    also the definition, of the necessary macros.
% \end{abstract}
%
% \section*{Preface to version 2.2}
%
% For \LaTeXe{} this package did not need any fundamental changes. I
% only modified the messages generated so that theorem layout styles
% will show up with the |\listfiles| command and cleaned the section
% on the New Font Selection Scheme since this is now included in
% \LaTeX.
%
% \section*{Preface to version 2.1}
%
% This version is identical to 2.0g described in \TUB\ 10\#3 except
% for some internal defaults which are now set depending on the used
% font selection scheme.
%
% This was done to avoid unpleasant surprises if the new font selection
% scheme is in force. For further details see
% section~\ref{sec:newfonts} and \cite{bk:GMS94}.
%
% \section{Introduction}
%
% For our purposes here, ``theorems'' are labelled enunciations,
% often set off from the main text by extra space and a font change.
% Theorems, corollaries, conjectures, definitions, and remarks
% are all instances of ``theorems''.  The ``header'' of these structures
% is composed of a label (such as \textsc{Theorem} or \textsc{Remark})
% and a number which serializes an item in the sequence of items
% with the same label.
%
% Shortly after the introduction of \LaTeX{} at the Fachbereich
% Mathematik in Mainz, the desire to manipulate the layout of
% ``theorems'' arose.  In Mainz, the following two conventions came
% into general use:
% \begin{enumerate}
%   \item The number of the theorem is shown in the margin.
%   \item There is a line break at
%         the end of the theorem header.
% \end{enumerate}
%
% Additionally, some journals require different formats which depend on
% the ``sort of theorem'': e.g.\ often remarks and definitions are set
% in |\upshape|, while |\itshape| is employed for main theorems.
%
% Confronted with these requirements, a theorem environment was
% developed in Mainz which allows separate determination of the
% layout of the ``theorems sets'', comparable to |\pagestyle|.
%
%
% \section{The user interface}
% \subsection{Defining new theorem sets}
%
% \DescribeMacro\newtheorem
% As in the original \LaTeX{} version, the command |\newtheorem|
% defines a new ``theorem set'' or ``theorem-like structure''.
% Two required arguments name the new environment and give the text
% to be typeset with each instance of the new ``set'', while
% an optional argument determines how
% the ``set'' is enumerated:
% \begin{description}
%    \item[\ttfamily \bslash newtheorem\{foo\}\{bar\}]
%       The theorem set {\envfont foo} (whose name is \texttt{bar})
%       uses its own counter.
%    \item[\ttfamily \bslash newtheorem\{foo2\}{[foo]}\{bar2\}]
%       The theorem set {\envfont foo2} (printed name \texttt{bar2})
%       uses the same counter as the theorem set \texttt{foo}.
%    \item[\ttfamily \bslash newtheorem\{foo3\}\{bar3\}{[section]}]
%       The theorem set {\envfont foo3} (printed name \texttt{bar3}) is
%       enumerated within the counter \texttt{section}, i.e.\ with every
%       new |\section| the enumeration begins again with 1, and
%       the enumeration is composed from the section-number and the
%       theorem counter itself.
% \end{description}
%
% \DescribeMacro\theoremstyle
% Additionally, the command |\theoremstyle| can define the layout
% of various, or all, theorem sets.  It should be noted that any
% theorem set defined by |\newtheorem| is typeset in the
% |\theoremstyle| that is current at the time of the definition.
% Thus, the following
% \begin{verbatim}
%    \theoremstyle{break}        \newtheorem{Cor}{Corollary}
%    \theoremstyle{plain}        \newtheorem{Exa}{Example}[section]
%\end{verbatim}                             ^^A No blank at line start
% leads to the result that the set {\envfont Cor} is formatted in the
% style \texttt{break}, while the set {\envfont Exa} and all the
% following ones are formatted in the style \texttt{plain}, unless
% another |\theoremstyle| follows.  Since the definitions installed by
% |\newtheorem| are global, one also can limit |\theoremstyle| locally
% by grouping braces.
%
% \DescribeMacro\theorembodyfont
% The choice of the font for the theorem body is completely
% independent of the chosen |\theoremstyle|; this has proven to
% be very advantageous.  For example,
% \begin{verbatim}
%   {\theorembodyfont{\upshape}          \newtheorem{Rem}{Remark}}
%\end{verbatim}
% defines a theorem set {\envfont Rem}, which will be set in
% |\upshape| in the current layout (which in our example is
% \texttt{plain}).  As with |\theoremstyle|, the |\theorembodyfont|
% chosen is that current at the time of |\newtheorem|.  If
% |\theorembodyfont| is not specified or one defines
% |\theorembodyfont{}|, then the font used will be that defined by the
% |\theoremstyle|.
%
% \DescribeMacro\theoremheaderfont It is also possible to customize
% the font used for the theorem headers.  This is, however, a global
% declaration, and therefore there should be at most one
% |\theoremheaderfont| declaration in the preamble.\footnote{If it is
% actually necessary to have different header fonts, one has to define
% new theorem styles (substituting the desired font) or specify the
% information directly in the \texttt{\bslash newtheorem} declaration
% (the unclean variant).}
%
% \DescribeMacro\theorempreskipamount
% \DescribeMacro\theorempostskipamount Two additional parameters
% affect the vertical space around the theorem environments:
% |\theorempreskipamount| and |\theorempostskipamount| define,
% respectively, the spacing before and after such an environment.
% These parameters apply for all theorem sets and can be manipulated
% with the ordinary length macros.  They are rubber lengths
% (`\textsf{skips}'), and therefore can contain \texttt{plus} and
% \texttt{minus} parts.
%
% Since the definition of theorem sets should---most sensibly---be
% placed in the preamble, we only allow installation there.
% It is therefore possible to release the memory used here after
% |\begin{document}|, in order to make room for other
% applications.
%
%
% \subsection{Existing theorem styles}
%
% The following theorem styles exist to date:
% \begin{deflist}{marginbreak}
%    \item[plain]
%       This theorem style emulates the original \LaTeX{} definition,
%       except that additionally the parameters
%       |\theorem...skipamount| are used.
%    \item[break]
%       In this style, the theorem header is followed by a line
%       break.
%    \item[marginbreak]
%       The theorem number is set in the margin, and there is a line
%       break as in \texttt{break}.
%    \item[changebreak]
%       Like \texttt{break}, but with header number and text
%       interchanged.
%    \item[change]
%       Header number and text are interchanged, without a line break.
%    \item[margin]
%       The number is set in the left margin, without a line break.
% \end{deflist}
% All styles (except \texttt{plain}) select |\slshape| as the default
% |\theorembodyfont|.
%
%
% \subsection{Examples}
%
% Given the above theorem sets {\envfont Cor}, {\envfont Exa}
% and {\envfont Rem},
% suppose that the preamble also contains the declarations:
% \begin{verbatim}
%    \theoremstyle{marginbreak}   \newtheorem{Lem}[Cor]{Lemma}
%    \theoremstyle{change}
%    \theorembodyfont{\itshape}        \newtheorem{Def}[Cor]{Definition}
%
%    \theoremheaderfont{\scshape}
%\end{verbatim}
% Then the following are some typical examples of the typeset output
% resulting from their use.
%
% \begin{Cor}
%    This is a sentence typeset in the theorem environment {\envfont
%    Cor}.
% \end{Cor}
%
% \begin{Exa}
%    This is a sentence typeset in the theorem environment {\envfont
%    Exa}.
% \end{Exa}
%
% \begin{Rem}
%    This is a sentence typeset in the theorem environment {\envfont
%    Rem}.
% \end{Rem}
%
% \begin{Lem}[Ben User]
%    This is a sentence typeset in the theorem environment {\envfont
%    Lem}.
% \end{Lem}
%
% \begin{Def}[Very impressive Definition]
%    This is a sentence typeset in the theorem environment {\envfont
%    Def}.
% \end{Def}
%
% The last two examples show the effect of the optional argument to
% a theorem environment (it is the text typeset in parentheses).
%
%
% \section{Special Considerations}
% \label{sec:newfonts}
%
% Theoremheader and body are implemented as a unit. This means that
% the |\theoremheaderfont|  will inherit characteristics of the
% |\theorembodyfont| in \LaTeXe. Thus, if for example
% |\theorembodyfont| is |\itshape| and |\theoremheaderfont|
% is |\bfseries| the font selected for the header will have the
% characteristics `bold extended italic'. If this is not desired one
% should set the |\theoremheaderfont| to something like
% \begin{verbatim}
%      \theoremheaderfont{\normalfont\bfseries}
%\end{verbatim}
% i.e.\ supplying all necessary font information explicitly.
%
%
%
% \section{Acknowledgements}
%
% The publication of this set of macros was only possible with the
% help of Christina Busse (translating the manuscript into English),
% Joachim Pense (playing the r\^ole of typist), Chris Rowley
% (looking everything over) and many others providing useful
% suggestions.
%
% \MaybeStop{%
%    \begin{thebibliography}{9}
%       \bibitem{bk:GMS94} \textsc{M.~Goossens}, \textsc{F.~Mittelbach}
%         and \textsc{A.~Samarin}.
%         \newblock The \LaTeX{} Companion.
%         \newblock
%         Addison-Wesley, Reading, Massachusetts, 1994.
%       \bibitem{latex:source} \textsc{Lamport, Leslie.} \newblock
%                               \texttt{latex.tex}, version 2.09, date
%                               Feb. 1990.
%    \end{thebibliography}
%    \PrintIndex
%    \PrintChanges}
%
%
% \section{The documentation driver file}
%
% The next bit of code contains the documentation driver file for
% \TeX{}, i.e., the file that will produce the documentation you are
% currently reading. It will be extracted from this file by the
% \texttt{docstrip} program. Since it is the first code in the file
% one can alternatively process this file directly with \LaTeXe{} to
% obtain the documentation.
%
%    \begin{macrocode}
%<*driver>
\documentclass{ltxdoc}

\usepackage{theorem}

%<+driver>% The next few lines define theorem sets which are used
%<+driver>% in the example section of the documentation.

\theoremstyle{break}       \newtheorem{Cor}{Corollary}
\theoremstyle{plain}       \newtheorem{Exa}{Example}[section]
{\theorembodyfont{\upshape}\newtheorem{Rem}{Remark}}
\theoremstyle{marginbreak} \newtheorem{Lem}[Cor]{Lemma}
\theoremstyle{change}
\theorembodyfont{\itshape} \newtheorem{Def}[Cor]{Definition}

\theoremheaderfont{\scshape}

\RecordChanges

\begin{document}
   \DocInput{theorem.dtx}
\end{document}
%</driver>
%    \end{macrocode}
%
% \section{Definition of the Macros}
%
% If the file has been loaded before, we abort immediately. If not the
% package announces itself (this is actually done at the very top of
% the file---the way it is done isn't good style so don't copy it).
% \changes{v2.0e}{89/07/19}{Spaces removed from `typeout'.}
% \changes{v2.2b}{95/11/19}{Announce moved to top of file}
%    \begin{macrocode}
%<*package>
%\@ifundefined{theorem@style}{}{\endinput}
%\def\FMithmInfo{1995/11/19 v2.2b Theorem extension package (FMi)}
%\ProvidesPackage{theorem}[\FMithmInfo]
%    \end{macrocode}
%
%
% \subsection{Definition of theorem styles and fonts}
%
% All the definitions in this file are done globally to allow
% inputting this file inside a group.
%
% \begin{macro}{\theoremstyle}
%    Before a theorem style can be installed, the chosen style must be
%    known. For that reason, we must test to see that
%    |\th@|\meta{style} is known or, more precisely, that it is
%    different from |\relax|. If the style is not known then
%    |\th@plain| is used.
%    \begin{macrocode}
\gdef\theoremstyle#1{%
   \@ifundefined{th@#1}{\@warning
          {Unknown theoremstyle `#1'. Using `plain'}%
          \theorem@style{plain}}%
%    \end{macrocode}
%    We save the theorem style to be used in the token register
%    |\theorem@style|.
%    \begin{macrocode}
      {\theorem@style{#1}}%
%    \end{macrocode}
%    Now we ``evaluate'' the theorem style: this means, we call the
%    macro |\th@|\meta{style} which will activate the relevant
%    definitions which are contained in a separate file.  This is done
%    in a group to suppress changes to the current font.  This could
%    otherwise pose problems together with the new font selection
%    scheme\footnote{When I printed the original article using the new
%    font selection scheme I ended with a document with slanted
%    typefaces (text headings and all) simply because one of the
%    theorem styles used \texttt{\bslash sl} at toplevel.} if the
%    |\th@|\meta{style} is evaluated a second time.
% \changes{v2.1a}{90/02/15}{Group added around `th@..}
% \changes{v2.1a}{90/02/15}{Reseting of `..@begintheorem removed.}
%    \begin{macrocode}
      \begingroup
        \csname th@\the\theorem@style \endcsname
      \endgroup}
%    \end{macrocode}
% \end{macro}
%
%
% \begin{macro}{\@begintheorem}
% \changes{v2.1a}{90/02/15}{Relaxed at toplevel.}
% \begin{macro}{\@opargbegintheorem}
% \changes{v2.1a}{90/02/15}{Relaxed at toplevel.}
%    We reset |\@begintheorem| and |\@opargbegintheorem| to
%    |\relax| since these commands are no longer necessary at
%    toplevel. This will save a few tokens.
%    \begin{macrocode}
\global\let\@begintheorem\relax
\global\let\@opargbegintheorem\relax
%    \end{macrocode}
% \end{macro}
% \end{macro}
%
% \begin{macro}{\theorem@style}
%    Obviously the \textsf{token} register used above has to be
%    allocated. To assure the utmost compatibility with the
%    original \LaTeX{} definition, we set the default theorem style to
%    \texttt{plain}, which implements the usual \LaTeX{} convention.
% \changes{v2.0f}{89/07/20}{Unnecessary `global' removed.}
%    \begin{macrocode}
\newtoks\theorem@style
\global\theorem@style{plain}
%    \end{macrocode}
% \end{macro}
%
% \begin{macro}{\theorembodyfont}
% \begin{macro}{\theorem@bodyfont}
%    If the |\theorembodyfont| is set by the user then it should not
%    interact with the default font set in the theorem style. When the
%    new font selection is in force this may happen if, for example,
%    the default is |\itshape| and the new |\theorembodyfont| is
%    |\sffamily|. So we add a |\reset@font| command in front of the
%    user definition.
% \changes{v2.0f}{89/07/20}{Unnecessary `global' removed.}
% \changes{v2.0g}{89/09/19}{`theoremfont to `theorembodyfont.}
% \changes{v2.1c}{92/06/22}{`reset@font added and logic changed.}
% \changes{v2.1d}{92/10/28}{Reset to empty if arg is empty}
%    \begin{macrocode}
\gdef\theorembodyfont#1{%
%    \end{macrocode}
%    We check if the argument supplied is empty and if so put nothing
%    into the |\theorem@bodyfont| token register to allow for
%    |\theorembodyfont{}| as a mean of using the default of the
%    current |\theoremstyle|.
%    \begin{macrocode}
   \def\@tempa{#1}%
   \ifx\@tempa\@empty
    \theorem@bodyfont{}%
   \else
    \theorem@bodyfont{\reset@font#1}%
   \fi
   }
\newtoks\theorem@bodyfont
\global\theorem@bodyfont{}
%    \end{macrocode}
% \end{macro}
% \end{macro}
%
%
% \begin{macro}{\theoremheaderfont}
% \changes{v2.0g}{89/09/19}{Macro added.}
%    The font for the theorem headers is handled differently because
%    this definition applies to all theorem styles.
%    \begin{macrocode}
\gdef\theoremheaderfont#1{\gdef\theorem@headerfont{#1}%
%    \end{macrocode}
%    After using the macro once it is redefined to produce an error
%    message.
%    \begin{macrocode}
       \gdef\theoremheaderfont##1{%
        \typeout{\string\theoremheaderfont\space should be used
                 only once.}}}
%    \end{macrocode}
% \end{macro}
%
% \begin{macro}{\theorem@headerfont}
% \changes{v2.0g}{89/09/19}{Macro added.}
%    To set the |\theorem@headerfont| default we first test if
%    the new fontselection scheme is in force.
% \changes{v2.1a}{90/02/15}{Support for new font selection.}
%    \begin{macrocode}
\ifx\upshape\undefined
%    \end{macrocode}
%    If not we define it to expand into |\bfseries|. We don't use
%    |\let| just in case a following style option redefines this
%    macro.
%    \begin{macrocode}
\gdef\theorem@headerfont{\bfseries}
%    \end{macrocode}
%    Otherwise we reset the current shape before calling |\bfseries|.
%    \begin{macrocode}
\else \gdef\theorem@headerfont{\normalfont\bfseries}\fi
%    \end{macrocode}
% \end{macro}
%
% \begin{macro}{\th@plain}
% \begin{macro}{\th@break}
% \begin{macro}{\th@marginbreak}
% \begin{macro}{\th@changebreak}
% \begin{macro}{\th@change}
% \begin{macro}{\th@margin}
%    The different styles are defined in macros such as
%    |\th@plain|.  Since memory space is precious in
%    ``non-Big-versions'', we have to avoid offering too many unused
%    definitions.  Therefore we define these styles in separate files
%    that can be loaded on demand.  Thus the commands themselves only
%    load these files. We use |\@input@|, a \LaTeXe{} internal command
%    that ensures that the file will be listed with |\listfiles|.
% \changes{v2.2a}{94/02/02}{Use `@input@ to load theorem layout files}
%    \begin{macrocode}
\gdef\th@plain{\@input@{thp.sty}}
\gdef\th@break{\@input@{thb.sty}}
\gdef\th@marginbreak{\@input@{thmb.sty}}
\gdef\th@changebreak{\@input@{thcb.sty}}
\gdef\th@change{\@input@{thc.sty}}
\gdef\th@margin{\@input@{thm.sty}}
%    \end{macrocode}
%    This list will be expanded when new styles become
%    available.  For testing,  just append new theorem substyles
%    as document options.
% \end{macro}
% \end{macro}
% \end{macro}
% \end{macro}
% \end{macro}
% \end{macro}
%
%
%
% \subsection{Definition of a new theorem set}
%
% As already pointed out, a new theorem environment can be defined in
% three different ways:
% \begin{verbatim}
%    \newtheorem{Lem}{Lemma}
%    \newtheorem{Lem}{Lemma}[section]
%    \newtheorem{Lem}[Theorem]{Lemma}
%\end{verbatim}
% The function of the macro |\newtheorem| is to recognize these cases
% and then to branch into one of the three macros |\@ynthm|, |\@xnthm|
% or |\@othm|.  This mechanism is adopted unchanged from
% \cite{latex:source}; the essential point here is that, for example,
% in the second case, the arguments \texttt{Lem}, \texttt{Lemma} and
% \texttt{section} are passed over to the macro |\@xnthm|.
%
% We inspect this case first because the others present fewer problems,
% and thus are easily derived from this one.
%
% \begin{macro}{\@xnthm}
%    For our example arguments, the macro |\@xnthm| must fulfill
%    the following:
%    \begin{itemize}
%       \item
%          Define a new \LaTeX-counter `Lem'
%       \item
%          reset this counter within a |\section|
%       \item
%          define the macro |\theLem|
%       \item
%          define the environment macros |\Lem| and
%          |\endLem| using the current
%          |\theoremstyle| and |\theorem@bodyfont|.
%    \end{itemize}
%    Obviously, all this should happen only if the first argument
%    of |\@xnthm| (i.e.\ \texttt{Lem} in our example) is chosen so
%    as not to conflict with any previously defined commands or
%    environments.  This test is performed by the \LaTeX{} macro
%    |\@ifdefinable|.
%    \begin{macrocode}
\gdef\@xnthm#1#2[#3]{\expandafter\@ifdefinable\csname #1\endcsname
%    \end{macrocode}
%    Therefore, the first argument of |\@ifdefinable| is the
%    expansion (in the example, |\Lem|) of
%    |\csname||#1\endcsname|.  The second argument is
%    executed only if the test has been completed successfully.
%    \begin{macrocode}
   {%
%    \end{macrocode}
%    Now we define the new counter. The names of the \LaTeX{}
%    macros employed should speak for themselves:
% \changes{v2.2b}{1995/10/07}{Use \cs{@newctr}}
% \changes{v2.2c}{1995/11/23}{Use \cs{@newctr} properly.}
%    \begin{macrocode}
    \@definecounter{#1}\@newctr{#1}[#3]%
%    \end{macrocode}
%    Using |\@newctr| will give a proper error message if the counter
%    in |#3| is not defined.
%    In defining `|\theLem|' we must generate the desired macro
%    name by use of |\expandafter| and |\csname|.
%    \begin{macrocode}
    \expandafter\xdef\csname the#1\endcsname
%    \end{macrocode}
%    An |\xdef| is used in order to make the definition global,
%    and to ensure that it contains the replacement texts of
%    |\@thmcountersep| and |\@thmcounter|.\footnote{These
%    two macros can be defined by the document style.  Their default
%    values produce a `\texttt{.}' as separation and an arabic
%    representation of the number.} However, not everything should be
%    expanded. For example, it saves space to use |\thesection|
%    instead of its---at times---lengthy expansion.
%    \begin{macrocode}
      {\expandafter \noexpand \csname the#3\endcsname
       \@thmcountersep \@thmcounter{#1}}%
%    \end{macrocode}
%    Thus with the defaults of \LaTeX{}, |\theLem| would be
%    replaced by the command sequence |\thesection.\arabic{Lem}|.
%
%    We will now look at the definition of the macro which is executed
%    at the beginning of the actual environment (in our example this
%    macro is |\Lem|). It should be noted that we use an
%    ``|\expandafter| trick'' to expand only certain parts of the
%    replacement text at the time of the definition.
%    \begin{macrocode}
    \def\@tempa{\global\@namedef{#1}}%
    \expandafter \@tempa \expandafter{%
%    \end{macrocode}
%    First, the macro that contains the current definitions of
%    |\@begintheorem| and |\@opargtheorem| should be called
%    up.  The name of this macro---as is already known---has the form
%    |\th@|\meta{theorem\ style}; therefore, it must be called by
%    \begin{macrocode}
      \csname th@\the \theorem@style
%    \end{macrocode}
%    In addition the default theorem font should be changeable, i.e.\
%    we have to insert the contents of |\theorem@bodyfont|. For
%    that reason, we expand even further, beyond |\endcsname|,
%    and thus insert the contents of the \textsf{token} register
%    |\theorem@bodyfont| in the replacement text.
%    \begin{macrocode}
            \expandafter \endcsname \the \theorem@bodyfont
%    \end{macrocode}
%    Now it is time to call the macro |\@thm| which takes over
%    the further processing.  It has two arguments: the current
%    counter name (in our example, \texttt{Lem}), and the text of the
%    label (in our example, \texttt{Lemma}).
%    \begin{macrocode}
     \@thm{#1}{#2}}%
%    \end{macrocode}
%    With this, the `sub-definition' is complete.  The macro
%    |\@endtheorem| ends a theorem environment and is, so far, nothing
%    but an |\endtrivlist|.  (Hence it is defined globally, and not
%    within the theorem styles.\footnote{This has to be changed as
%    soon as theorem styles that change \texttt{\bslash @endtheorem}
%    exist.  In such a case, all existing styles must be changed as
%    well since they will have to reset the macro.}) Therefore, we can
%    set it equivalent to the macro that ends the theorem set (in our
%    example, |\endLem|).  However, if some day theorem styles exist
%    that do change |\@endtheorem|, we would have to use the
%    commented-out line instead.
%    \begin{macrocode}
    \global \expandafter \let \csname end#1\endcsname \@endtheorem
%   \global\@namedef{end#1}{\@endtheorem}%
%    \end{macrocode}
%    With these commands all the required definitions are employed,
%    unless the test |\@ifdefinable| has failed.  Therefore, we
%    end the second argument of this macro and with it the definition
%    of |\@xnthm|.
%    \begin{macrocode}
   }}
%    \end{macrocode}
% \end{macro}
%
% \begin{macro}{\@ynthm}
%    The definition of |\@ynthm| is completely analogous. In
%    this case the new counter that is defined is not reset within
%    another counter; thus the definition of |\the...|
%    is simplified:
%    \begin{macrocode}
\gdef\@ynthm#1#2{\expandafter\@ifdefinable\csname #1\endcsname
   {\@definecounter{#1}%
    \expandafter\xdef\csname the#1\endcsname{\@thmcounter{#1}}%
%    \end{macrocode}
%    The rest of the definition corresponds literally to that of
%    |\@xnthm|:
%    \begin{macrocode}
    \def\@tempa{\global\@namedef{#1}}\expandafter \@tempa
     \expandafter{\csname th@\the \theorem@style \expandafter
     \endcsname \the\theorem@bodyfont \@thm{#1}{#2}}%
    \global \expandafter \let \csname end#1\endcsname \@endtheorem}}
%    \end{macrocode}
% \end{macro}
%
% \begin{macro}{\@othm}
%    The definition of |\@othm| does not contain anything new.
%    \begin{macrocode}
\gdef\@othm#1[#2]#3{%
%    \end{macrocode}
% \changes{v2.2b}{95/10/07}{Build in check for counter defined}
%    We do not define a new counter but instead use one that has
%    already been defined.  Thus the only definition we need is that
%    of this pseudo-counter (i.e.\ |\the|\meta{env.\ name}).
%    First we check if |#2| corresponds to a known counter name.
%    \begin{macrocode}
  \expandafter\ifx\csname c@#2\endcsname\relax
   \@nocounterr{#2}%
  \else
   \expandafter\@ifdefinable\csname #1\endcsname
   {\expandafter \xdef \csname the#1\endcsname
     {\expandafter \noexpand \csname the#2\endcsname}%
%    \end{macrocode}
%    All other parts of the definition can be adopted from
%    |\@xnthm|. We have to remember, though, that in this case
%    the name of the current counter and the theorem label have moved
%    to the second and third arguments.
%    \begin{macrocode}
    \def\@tempa{\global\@namedef{#1}}\expandafter \@tempa
     \expandafter{\csname th@\the \theorem@style \expandafter
     \endcsname \the\theorem@bodyfont \@thm{#2}{#3}}%
    \global \expandafter \let \csname end#1\endcsname \@endtheorem}%
  \fi}
%    \end{macrocode}
% \end{macro}
%
% \subsection{Macros that are employed in a {\envfont theorem}
%    environment}
%
% \begin{macro}{\@thm}
%    The macro |\@thm| has to increase the current counter.  Then,
%    depending on whether the environment has (or does not have) an
%    optional argument, it has to branch into either |\@begintheorem|
%    or |\@opargtheorem|.
%    \begin{macrocode}
\gdef\@thm#1#2{\refstepcounter{#1}%
%    \end{macrocode}
%    Now we start a {\envfont trivlist} environment, and give
%    |\@topsep| and |\@topsepadd| the values of the \textsf{skip}
%    registers |\theorempreskipamount| and |\theorempostskipamount|.
%    The value in |\@topsep| is the vertical space that is inserted by
%    the first (and only) |\item| in our |\trivlist| whilst
%    |\@topsepadd| is inserted by |\@endparenv| at the end of that
%    {\envfont trivlist} environment.  By using these registers, we
%    obtain the desired space around a {\envfont theorem} environment.
%    \begin{macrocode}
   \trivlist
   \@topsep \theorempreskipamount               % used by first \item
   \@topsepadd \theorempostskipamount           % used by \@endparenv
%    \end{macrocode}
%    Now we have to test whether an optional argument has been given.
% \changes{v2.0d}{89/07/18}{Forgotten percent added.}
%    \begin{macrocode}
   \@ifnextchar [%
%    \end{macrocode}
%    If there is an optional argument, we will call |\@ythm|, and
%    move the arguments read back into the input stream.
%    \begin{macrocode}
   {\@ythm{#1}{#2}}%
%    \end{macrocode}
%    If not, we call |\@begintheorem|. Its first argument is the
%    name of the theorem set (hence the second argument of
%    |\@thm|). Its second argument is the macro that produces the
%    current number.
%    \begin{macrocode}
   {\@begintheorem{#2}{\csname the#1\endcsname}\ignorespaces}}
%    \end{macrocode}
% \end{macro}
%
% \begin{macro}{\@xthm}
% \begin{macro}{\@ythm}
%    Both these macros were originally called by |\@thm|.  We do
%    not need |\@xthm| anymore, hence we reset it to
%    |\relax|. The definition of |\@ythm| has not changed at
%    all from its definition in \LaTeX. In order to make the macros
%    easier to understand, we will
%    nevertheless present it (commented out).
%    \begin{macrocode}
\global\let\@xthm\relax
% \def\@ythm#1#2[#3]{\@opargbegintheorem{#2}{\csname
%           the#1\endcsname}{#3}\ignorespaces}
%</package>
%    \end{macrocode}
%    The primitive |\ignorespaces| in |\@ythm| and
%    |\@thm| is needed to remove the spaces between the
%    |\begin{...}| and the actual text.
% \end{macro}
% \end{macro}
%
%
% \subsection{Definition of the theorem substyles}
%
% As already pointed out, the theorem substyles, defined below, are
% only loaded when necessary.  Note that all these substyles, except
% \texttt{plain}, have |\slshape| as the default body font.
%
% \subsubsection{The \texttt{plain} style}
%
% \label{dok:th@plain} As the following macros use |@|, we have
% to locally set the |\catcode| of this symbol to ``letter''.
% This happens within a group, so that we do not have to worry about
% which |\catcode| that symbol had before.
%    \begin{macrocode}
%<*thp>
\begingroup \makeatletter
%    \end{macrocode}
% Since we are now within a group, we must make all definitions
% globally.  First we make sure that \texttt{theorem.sty} is loaded.
% This will allow us to use this file as a document style option
% without having to call \texttt{theorem} itself as an option.  At the
% same time, we assure that at least version 2 is loaded, since
% |\theorem@style| was not defined in earlier versions.
%    \begin{macrocode}
\@ifundefined{theorem@style}{\input{theorem.sty}}{}
\ProvidesFile{thp.sty}
             [\FMithmInfo]
%    \end{macrocode}
%
%
% \begin{macro}{\th@plain}
% \begin{macro}{\@begintheorem}
% \begin{macro}{\@opargbegintheorem}
%    |\theoremstyle{plain}| corresponds to the original
%    definition, except that the distances to the surrounding text are
%    determined by the parameters |\theorempreskipamount| and
%    |\theorempostskipamount|.
%    First we set the default body font.
%    \begin{macrocode}
\gdef\th@plain{\normalfont\itshape
%    \end{macrocode}
%    Then we define |\@begintheorem| and
%    |\@opargbegintheorem|. These two macros define how the
%    header of a theorem is typeset.  |\@opargbegintheorem| will
%    be called if a \textsf{theorem} environment with an optional
%    argument is encountered; otherwise, the header is constructed by
%    calling |\@begintheorem|.  If one of these macros is
%    executed, we are within a \textsf{trivlist} environment started by
%    |\@thm|.  So the theorem header is produced with an
%    |\item| command.
%
%    Instead of specifying the header font directly, all standard
%    theorem styles use the |\theorem@headerfont| macro to allow
%    customization.  The extra space (|\labelsep|) is necessary
%    because of problems in the \textsf{trivlist} environment.
%    \begin{macrocode}
  \def\@begintheorem##1##2{%
        \item[\hskip\labelsep \theorem@headerfont ##1\ ##2]}%
%    \end{macrocode}
%    The definition of |\@opargbegintheorem| is completely
%    analogous. The only difference is the fact that there exists a
%    third argument (which is the optional parameter of the
%    environment and contains additional information about the
%    theorem).  Customarily we enclose it in parentheses.
%    \begin{macrocode}
\def\@opargbegintheorem##1##2##3{%
   \item[\hskip\labelsep \theorem@headerfont ##1\ ##2\ (##3)]}}
%    \end{macrocode}
% \end{macro}
% \end{macro}
% \end{macro}
%
% \noindent
% We conclude with an |\endgroup| to restore the
% |\catcode| of |@|.
%    \begin{macrocode}
\endgroup
%</thp>
%    \end{macrocode}
%
%
% \subsubsection{The \texttt{break} style}
%
% This style option is stored in the file |thb.sty|.
% For the next two lines see the documentation for |\th@plain| on
% page~\pageref{dok:th@plain}.
%    \begin{macrocode}
%<*thb>
\begingroup \makeatletter
\@ifundefined{theorem@style}{\input{theorem.sty}}{}
\ProvidesFile{thb.sty}
             [\FMithmInfo]
%    \end{macrocode}
%
% \begin{macro}{\th@break}
%    |\theoremstyle{break}| produces a line break after the name
%    of the theorem.  The font is |\slshape|.  Hence, we define
%    |\th@break| as follows:
%    \begin{macrocode}
\gdef\th@break{\normalfont\slshape
  \def\@begintheorem##1##2{\item[%
%    \end{macrocode}
%    We run into the following problem: it is not possible to create
%    the header with |\item[|\meta{title}|]| and then start
%    a new line by, for example, |\mbox{}\\|. Such a definition
%    will fail whenever a list environment follows immediately.  With
%    the above construction, the |\mbox{}| causes the switch
%    |@inlabel| (cf.\ definition of |\list| and
%    |\trivlist| in \cite{latex:source}) to be set to \texttt{false}
%    and so the following list will insert additional vertical
%    space (|\topskip|). This is quite annoying. Therefore, we
%    create the line break within the |\item|. In order to ensure
%    that the text will begin at the proper position in the following
%    line, we simply pretend that the label does not take any
%    room.\footnote{This will lead to problems whenever very high
%    symbols occurring in the line tower into the
%    heading. So, something else has to be done here sometime.}
%    \begin{macrocode}
       \rlap{\vbox{\hbox{\hskip \labelsep\theorem@headerfont ##1\ ##2}%
                   \hbox{\strut}}}]}%
%    \end{macrocode}
%    Again, the definition of |\@opargbegintheorem| is
%    completely analogous.
%    \begin{macrocode}
\def\@opargbegintheorem##1##2##3{%
  \item[\rlap{\vbox{\hbox{\hskip \labelsep \theorem@headerfont
                     ##1\ ##2\ (##3)}%
                    \hbox{\strut}}}]}}
\endgroup
%</thb>
%    \end{macrocode}
% \end{macro}
%
%
% \subsubsection{The \texttt{changebreak} style}
%
%    \begin{macrocode}
% This style option is stored in the file |thcb.sty|.
%    \begin{macrocode}
%<*thcb>
\begingroup \makeatletter
\@ifundefined{theorem@style}{\input{theorem.sty}}{}
\ProvidesFile{thcb.sty}
             [\FMithmInfo]
%    \end{macrocode}
%
% \begin{macro}{\th@changebreak}
%    The \texttt{change-break} theorem style is like \texttt{break}
%    but with interchange of theorem name and theorem number.  Thus we
%    define |\th@changebreak| as follows:
%    \begin{macrocode}
\gdef\th@changebreak{\normalfont\slshape
  \def\@begintheorem##1##2{\item
       [\rlap{\vbox{\hbox{\hskip\labelsep \theorem@headerfont ##2\ ##1}%
                    \hbox{\strut}}}]}%
\def\@opargbegintheorem##1##2##3{%
  \item[\rlap{\vbox{\hbox{\hskip\labelsep \theorem@headerfont
                     ##2\ ##1\ (##3)}%
                    \hbox{\strut}}}]}}
\endgroup
%</thcb>
%    \end{macrocode}
% \end{macro}
%
%
% \subsubsection{The \texttt{change} style}
%
% This style option is stored in the file |thc.sty|.
%    \begin{macrocode}
%<*thc>
\begingroup \makeatletter
\@ifundefined{theorem@style}{\input{theorem.sty}}{}
\ProvidesFile{thc.sty}
             [\FMithmInfo]
%    \end{macrocode}
%
% \begin{macro}{\th@change}
%    The \texttt{change} theorem style corresponds to the
%    \texttt{change break} style without a linebreak after the header.
%    To say it in another way, it's the same as the \texttt{plain}
%    style but with number and name interchanged and |\slshape| as the
%    default font.
%    \begin{macrocode}
\gdef\th@change{\normalfont\slshape
  \def\@begintheorem##1##2{\item
       [\hskip\labelsep \theorem@headerfont ##2\ ##1]}%
\def\@opargbegintheorem##1##2##3{%
  \item[\hskip\labelsep \theorem@headerfont ##2\ ##1\ (##3)]}}
\endgroup
%</thc>
%    \end{macrocode}
% \end{macro}
%
%
% \subsubsection{The \texttt{marginbreak} style}
%
% This style option is the one used most often at Mainz.  It is saved
% in the file |thmb.sty|.
%    \begin{macrocode}
%<*thmb>
\begingroup \makeatletter
\@ifundefined{theorem@style}{\input{theorem.sty}}{}
\ProvidesFile{thmb.sty}
             [\FMithmInfo]
%    \end{macrocode}
%
% \begin{macro}{\th@marginbreak}
%    The \texttt{margin break} style is nearly the same as the
%    \texttt{change break} style. The only difference is the placement
%    of the theorem number.  We use |\llap| to place it in the left
%    margin.
%
%    In this style |\labelsep| denotes the separation between the
%    number and the text.
%    \begin{macrocode}
\gdef\th@marginbreak{\normalfont\slshape
  \def\@begintheorem##1##2{\item
       [\rlap{\vbox{\theorem@headerfont
                    \hbox{\llap{##2}\hskip\labelsep ##1}%
                    \hbox{\strut}}}]}%
\def\@opargbegintheorem##1##2##3{%
  \item[\rlap{\vbox{\theorem@headerfont
                    \hbox{\llap{##2}\hskip\labelsep ##1\ (##3)}%
                    \hbox{\strut}}}]}}
\endgroup
%</thmb>
%    \end{macrocode}
% \changes{v2.1c}{90/02/17}{`endgroup outside macrocode env.}
% \end{macro}
%
%
%
% \subsubsection{The \texttt{margin} style}
%
% This style option is stored in the file |thm.sty|.
%    \begin{macrocode}
%<*thm>
\begingroup \makeatletter
\@ifundefined{theorem@style}{\input{theorem.sty}}{}
\ProvidesFile{thm.sty}
             [\FMithmInfo]
%    \end{macrocode}
%
% \begin{macro}{\th@margin}
%    Again this is only a variant of the theorem styles described
%    above without any new ideas.
%    \begin{macrocode}
\gdef\th@margin{\normalfont\slshape
  \def\@begintheorem##1##2{\item
       [\theorem@headerfont \llap{##2}\hskip\labelsep ##1]}%
\def\@opargbegintheorem##1##2##3{%
  \item[\theorem@headerfont \llap{##2}\hskip\labelsep ##1\ (##3)]}}
\endgroup
%</thm>
%    \end{macrocode}
% \end{macro}
%
%
%
%
% \subsection{Final Definitions}
%
% \begin{macro}{\theorempreskipamount}
% \begin{macro}{\theorempostskipamount}
%    The \textsf{skip} parameters that regulate the vertical empty space
%    before and after the {\envfont theorem} environment have
%    to be allocated as well.
% \changes{v2.0f}{89/07/20}{Unnecessary `global' removed.}
%    \begin{macrocode}
%<*package>
\newskip\theorempreskipamount
\newskip\theorempostskipamount
%    \end{macrocode}
%    Since we have used the same values for all theorem sets, we now can
%    assign them.
%    \begin{macrocode}
\global\setlength\theorempreskipamount{12pt plus 5pt minus 3pt}
\global\setlength\theorempostskipamount{8pt plus 3pt minus 1.5pt}
%    \end{macrocode}
% \end{macro}
% \end{macro}
%
%
% \begin{macro}{\@endtheorem}
%    The same holds for the macro |\@endtheorem|, which ends a
%    {\envfont theorem} environment. Since it is the same for all
%    theorem sets, it is removed from the macros |\th@|\meta{style}.
%    It simply ends the {\envfont trivlist} environment, which was
%    begun in |\@thm|.
%    \begin{macrocode}
\global\let\@endtheorem=\endtrivlist
%    \end{macrocode}
% \end{macro}
%
%
%
% \begin{macro}{\@preamblecmds}
%    All macros defined above are to be used only in the preamble.
%    Therefore, we insert them in |\@preamblecmds| which will disable
%    them at begin document. This is done by the internal \LaTeXe{}
%    command |\@onlypreamble|.
% \changes{v2.0g}{89/09/19}{`theoremheaderfont and `theorembodyfont
%    added.}
% \changes{v2.2a}{94/02/03}{Use \LaTeXe{} interface}
%    \begin{macrocode}
\@onlypreamble\@xnthm
\@onlypreamble\@ynthm
\@onlypreamble\@othm
\@onlypreamble\newtheorem
\@onlypreamble\theoremstyle
\@onlypreamble\theorembodyfont
\@onlypreamble\theoremheaderfont
%    \end{macrocode}
% \end{macro}
%
% Finally we declare the \texttt{plain} theorem style to be the default.
% \changes{v2.1b}{90/06/04}{Fixed missing `item by declaring default
%    style.}
%    \begin{macrocode}
\theoremstyle{plain}
%</package>
%    \end{macrocode}
%
% \Finale

\endinput
}{}
\ProvidesFile{thb.sty}
             [\FMithmInfo]
%    \end{macrocode}
%
% \begin{macro}{\th@break}
%    |\theoremstyle{break}| produces a line break after the name
%    of the theorem.  The font is |\slshape|.  Hence, we define
%    |\th@break| as follows:
%    \begin{macrocode}
\gdef\th@break{\normalfont\slshape
  \def\@begintheorem##1##2{\item[%
%    \end{macrocode}
%    We run into the following problem: it is not possible to create
%    the header with |\item[|\meta{title}|]| and then start
%    a new line by, for example, |\mbox{}\\|. Such a definition
%    will fail whenever a list environment follows immediately.  With
%    the above construction, the |\mbox{}| causes the switch
%    |@inlabel| (cf.\ definition of |\list| and
%    |\trivlist| in \cite{latex:source}) to be set to \texttt{false}
%    and so the following list will insert additional vertical
%    space (|\topskip|). This is quite annoying. Therefore, we
%    create the line break within the |\item|. In order to ensure
%    that the text will begin at the proper position in the following
%    line, we simply pretend that the label does not take any
%    room.\footnote{This will lead to problems whenever very high
%    symbols occurring in the line tower into the
%    heading. So, something else has to be done here sometime.}
%    \begin{macrocode}
       \rlap{\vbox{\hbox{\hskip \labelsep\theorem@headerfont ##1\ ##2}%
                   \hbox{\strut}}}]}%
%    \end{macrocode}
%    Again, the definition of |\@opargbegintheorem| is
%    completely analogous.
%    \begin{macrocode}
\def\@opargbegintheorem##1##2##3{%
  \item[\rlap{\vbox{\hbox{\hskip \labelsep \theorem@headerfont
                     ##1\ ##2\ (##3)}%
                    \hbox{\strut}}}]}}
\endgroup
%</thb>
%    \end{macrocode}
% \end{macro}
%
%
% \subsubsection{The \texttt{changebreak} style}
%
%    \begin{macrocode}
% This style option is stored in the file |thcb.sty|.
%    \begin{macrocode}
%<*thcb>
\begingroup \makeatletter
\@ifundefined{theorem@style}{% \iffalse meta-comment
%
% Copyright (C) 1993-2024
%
% The LaTeX Project and any individual authors listed elsewhere
% in this file.
%
% This file is part of the Standard LaTeX `Tools Bundle'.
% -------------------------------------------------------
%
% It may be distributed and/or modified under the
% conditions of the LaTeX Project Public License, either version 1.3c
% of this license or (at your option) any later version.
% The latest version of this license is in
%    https://www.latex-project.org/lppl.txt
% and version 1.3c or later is part of all distributions of LaTeX
% version 2005/12/01 or later.
%
% The list of all files belonging to the LaTeX `Tools Bundle' is
% given in the file `manifest.txt'.
%
% \fi
%
%%
%
%
\def\FMithmInfo{2023/07/05 v2.2c Theorem extension package (FMi)}
%
% \ProvidesFile{theorem.dtx}[\FMithmInfo]
%\iffalse   % this is a METACOMMENT !
%%
%% Package `theorem' to use with LaTeX2e
%% Copyright (C) 1989-1995 Frank Mittelbach, all rights reserved.
%<+package>\@ifundefined{theorem@style}{}{\endinput}
%<+package>\NeedsTeXFormat{LaTeX2e}[1995/06/01]
%<+package>\ProvidesPackage{theorem}
%<+package>         [\FMithmInfo]
%
% \fi
%
% \changes{v1.0}{87/05/07}{First draft.}
% \changes{v1.0a}{87/05/07}{`thistheoremstyle' added.}
% \changes{v1.1a}{87/07/29}{New concept: redefine the internal
%                          \protect\LaTeX{} commands `@xnthm' etc.}
% \changes{v1.1b}{87/07/31}{`theoremfont' used to specify font.}
% \changes{v1.2a}{88/01/08}{`@endtheorem' now globally defined.}
% \changes{v1.2b}{88/01/11}{`newtheorem' only allowed in the preamble.}
% \changes{v1.3a}{88/02/14}{`theoremfont' now a toks register.}
%
% \changes{v2.0a}{88/11/16}{New implementation and new documentation.}
% \changes{v2.0a}{88/11/16}{No layout changes but internal modifications
%                           old substyles won't work.}
% \changes{v2.0b}{89/01/30}{`global' used for all definitions.}
% \changes{v2.0c}{89/06/15}{Documentation translated to English.
%                            Thanks to LLL and Nina}
% \changes{v2.1a}{90/02/15}{Support for new font selection.}
% \changes{v2.2a}{94/02/03}{Cleanup for LaTeX2e}
%
% \DoNotIndex{\@endparenv,\@ifdefinable,\@ifnextchar,\@ifundefined}
% \DoNotIndex{\@namedef,\@nameuse,\@spaces,\@tempa,\@warning}
% \DoNotIndex{\ }
% \DoNotIndex{\begingroup,\bfseries}
% \DoNotIndex{\csname}
% \DoNotIndex{\def,\do,\docdate}
% \DoNotIndex{\endcsname,\endgroup,\endinput,\endtrivlist,\expandafter}
% \DoNotIndex{\filedate,\fileversion}
% \DoNotIndex{\gdef,\global}
% \DoNotIndex{\hbox,\hskip}
% \DoNotIndex{\ignorespaces,\input,\itshape,\item}
% \DoNotIndex{\labelsep,\let,\llap}
% \DoNotIndex{\makeatletter}
% \DoNotIndex{\noexpand}
% \DoNotIndex{\refstepcounter,\relax,\rlap}
% \DoNotIndex{\setlength,\slshape,\space,\string,\strut}
% \DoNotIndex{\the,\trivlist,\typeout}
% \DoNotIndex{\vbox}
% \DoNotIndex{\xdef}
%
% \def\deflabel#1{\ttfamily #1\hfill}
% \def\deflist#1{\begin{list}{}{\settowidth\labelwidth{\ttfamily #1}%
%                               \setlength\leftmargin\labelwidth
%                               \addtolength\leftmargin\labelsep
%                               \let\makelabel\deflabel}}
% \def\enddeflist{\end{list}}
% \def\envfont{\normalfont\ttfamily}
%
% \MakeShortVerb{\|}
%
% \newcommand{\TUB}{{\sl TUGboat\/}}
%
% \GetFileInfo{theorem.dtx}
% \title{An Extension of the
%        \LaTeX{} theorem environment\thanks{This file
%        has version number \fileversion, last
%        revised \filedate.}}
% \author{Frank Mittelbach}
%
% \MaintainedByLaTeXTeam{tools}
% \maketitle
%
% \begin{abstract}
%    The macros described in this paper yield an extension of the
%    \LaTeX{} theorem mechanism. It is designed to satisfy the
%    different requirements of various journals. Thus, the
%    layout of the ``theorems'' can be manipulated by determining a
%    ``style''. This article describes not only the use, but
%    also the definition, of the necessary macros.
% \end{abstract}
%
% \section*{Preface to version 2.2}
%
% For \LaTeXe{} this package did not need any fundamental changes. I
% only modified the messages generated so that theorem layout styles
% will show up with the |\listfiles| command and cleaned the section
% on the New Font Selection Scheme since this is now included in
% \LaTeX.
%
% \section*{Preface to version 2.1}
%
% This version is identical to 2.0g described in \TUB\ 10\#3 except
% for some internal defaults which are now set depending on the used
% font selection scheme.
%
% This was done to avoid unpleasant surprises if the new font selection
% scheme is in force. For further details see
% section~\ref{sec:newfonts} and \cite{bk:GMS94}.
%
% \section{Introduction}
%
% For our purposes here, ``theorems'' are labelled enunciations,
% often set off from the main text by extra space and a font change.
% Theorems, corollaries, conjectures, definitions, and remarks
% are all instances of ``theorems''.  The ``header'' of these structures
% is composed of a label (such as \textsc{Theorem} or \textsc{Remark})
% and a number which serializes an item in the sequence of items
% with the same label.
%
% Shortly after the introduction of \LaTeX{} at the Fachbereich
% Mathematik in Mainz, the desire to manipulate the layout of
% ``theorems'' arose.  In Mainz, the following two conventions came
% into general use:
% \begin{enumerate}
%   \item The number of the theorem is shown in the margin.
%   \item There is a line break at
%         the end of the theorem header.
% \end{enumerate}
%
% Additionally, some journals require different formats which depend on
% the ``sort of theorem'': e.g.\ often remarks and definitions are set
% in |\upshape|, while |\itshape| is employed for main theorems.
%
% Confronted with these requirements, a theorem environment was
% developed in Mainz which allows separate determination of the
% layout of the ``theorems sets'', comparable to |\pagestyle|.
%
%
% \section{The user interface}
% \subsection{Defining new theorem sets}
%
% \DescribeMacro\newtheorem
% As in the original \LaTeX{} version, the command |\newtheorem|
% defines a new ``theorem set'' or ``theorem-like structure''.
% Two required arguments name the new environment and give the text
% to be typeset with each instance of the new ``set'', while
% an optional argument determines how
% the ``set'' is enumerated:
% \begin{description}
%    \item[\ttfamily \bslash newtheorem\{foo\}\{bar\}]
%       The theorem set {\envfont foo} (whose name is \texttt{bar})
%       uses its own counter.
%    \item[\ttfamily \bslash newtheorem\{foo2\}{[foo]}\{bar2\}]
%       The theorem set {\envfont foo2} (printed name \texttt{bar2})
%       uses the same counter as the theorem set \texttt{foo}.
%    \item[\ttfamily \bslash newtheorem\{foo3\}\{bar3\}{[section]}]
%       The theorem set {\envfont foo3} (printed name \texttt{bar3}) is
%       enumerated within the counter \texttt{section}, i.e.\ with every
%       new |\section| the enumeration begins again with 1, and
%       the enumeration is composed from the section-number and the
%       theorem counter itself.
% \end{description}
%
% \DescribeMacro\theoremstyle
% Additionally, the command |\theoremstyle| can define the layout
% of various, or all, theorem sets.  It should be noted that any
% theorem set defined by |\newtheorem| is typeset in the
% |\theoremstyle| that is current at the time of the definition.
% Thus, the following
% \begin{verbatim}
%    \theoremstyle{break}        \newtheorem{Cor}{Corollary}
%    \theoremstyle{plain}        \newtheorem{Exa}{Example}[section]
%\end{verbatim}                             ^^A No blank at line start
% leads to the result that the set {\envfont Cor} is formatted in the
% style \texttt{break}, while the set {\envfont Exa} and all the
% following ones are formatted in the style \texttt{plain}, unless
% another |\theoremstyle| follows.  Since the definitions installed by
% |\newtheorem| are global, one also can limit |\theoremstyle| locally
% by grouping braces.
%
% \DescribeMacro\theorembodyfont
% The choice of the font for the theorem body is completely
% independent of the chosen |\theoremstyle|; this has proven to
% be very advantageous.  For example,
% \begin{verbatim}
%   {\theorembodyfont{\upshape}          \newtheorem{Rem}{Remark}}
%\end{verbatim}
% defines a theorem set {\envfont Rem}, which will be set in
% |\upshape| in the current layout (which in our example is
% \texttt{plain}).  As with |\theoremstyle|, the |\theorembodyfont|
% chosen is that current at the time of |\newtheorem|.  If
% |\theorembodyfont| is not specified or one defines
% |\theorembodyfont{}|, then the font used will be that defined by the
% |\theoremstyle|.
%
% \DescribeMacro\theoremheaderfont It is also possible to customize
% the font used for the theorem headers.  This is, however, a global
% declaration, and therefore there should be at most one
% |\theoremheaderfont| declaration in the preamble.\footnote{If it is
% actually necessary to have different header fonts, one has to define
% new theorem styles (substituting the desired font) or specify the
% information directly in the \texttt{\bslash newtheorem} declaration
% (the unclean variant).}
%
% \DescribeMacro\theorempreskipamount
% \DescribeMacro\theorempostskipamount Two additional parameters
% affect the vertical space around the theorem environments:
% |\theorempreskipamount| and |\theorempostskipamount| define,
% respectively, the spacing before and after such an environment.
% These parameters apply for all theorem sets and can be manipulated
% with the ordinary length macros.  They are rubber lengths
% (`\textsf{skips}'), and therefore can contain \texttt{plus} and
% \texttt{minus} parts.
%
% Since the definition of theorem sets should---most sensibly---be
% placed in the preamble, we only allow installation there.
% It is therefore possible to release the memory used here after
% |\begin{document}|, in order to make room for other
% applications.
%
%
% \subsection{Existing theorem styles}
%
% The following theorem styles exist to date:
% \begin{deflist}{marginbreak}
%    \item[plain]
%       This theorem style emulates the original \LaTeX{} definition,
%       except that additionally the parameters
%       |\theorem...skipamount| are used.
%    \item[break]
%       In this style, the theorem header is followed by a line
%       break.
%    \item[marginbreak]
%       The theorem number is set in the margin, and there is a line
%       break as in \texttt{break}.
%    \item[changebreak]
%       Like \texttt{break}, but with header number and text
%       interchanged.
%    \item[change]
%       Header number and text are interchanged, without a line break.
%    \item[margin]
%       The number is set in the left margin, without a line break.
% \end{deflist}
% All styles (except \texttt{plain}) select |\slshape| as the default
% |\theorembodyfont|.
%
%
% \subsection{Examples}
%
% Given the above theorem sets {\envfont Cor}, {\envfont Exa}
% and {\envfont Rem},
% suppose that the preamble also contains the declarations:
% \begin{verbatim}
%    \theoremstyle{marginbreak}   \newtheorem{Lem}[Cor]{Lemma}
%    \theoremstyle{change}
%    \theorembodyfont{\itshape}        \newtheorem{Def}[Cor]{Definition}
%
%    \theoremheaderfont{\scshape}
%\end{verbatim}
% Then the following are some typical examples of the typeset output
% resulting from their use.
%
% \begin{Cor}
%    This is a sentence typeset in the theorem environment {\envfont
%    Cor}.
% \end{Cor}
%
% \begin{Exa}
%    This is a sentence typeset in the theorem environment {\envfont
%    Exa}.
% \end{Exa}
%
% \begin{Rem}
%    This is a sentence typeset in the theorem environment {\envfont
%    Rem}.
% \end{Rem}
%
% \begin{Lem}[Ben User]
%    This is a sentence typeset in the theorem environment {\envfont
%    Lem}.
% \end{Lem}
%
% \begin{Def}[Very impressive Definition]
%    This is a sentence typeset in the theorem environment {\envfont
%    Def}.
% \end{Def}
%
% The last two examples show the effect of the optional argument to
% a theorem environment (it is the text typeset in parentheses).
%
%
% \section{Special Considerations}
% \label{sec:newfonts}
%
% Theoremheader and body are implemented as a unit. This means that
% the |\theoremheaderfont|  will inherit characteristics of the
% |\theorembodyfont| in \LaTeXe. Thus, if for example
% |\theorembodyfont| is |\itshape| and |\theoremheaderfont|
% is |\bfseries| the font selected for the header will have the
% characteristics `bold extended italic'. If this is not desired one
% should set the |\theoremheaderfont| to something like
% \begin{verbatim}
%      \theoremheaderfont{\normalfont\bfseries}
%\end{verbatim}
% i.e.\ supplying all necessary font information explicitly.
%
%
%
% \section{Acknowledgements}
%
% The publication of this set of macros was only possible with the
% help of Christina Busse (translating the manuscript into English),
% Joachim Pense (playing the r\^ole of typist), Chris Rowley
% (looking everything over) and many others providing useful
% suggestions.
%
% \MaybeStop{%
%    \begin{thebibliography}{9}
%       \bibitem{bk:GMS94} \textsc{M.~Goossens}, \textsc{F.~Mittelbach}
%         and \textsc{A.~Samarin}.
%         \newblock The \LaTeX{} Companion.
%         \newblock
%         Addison-Wesley, Reading, Massachusetts, 1994.
%       \bibitem{latex:source} \textsc{Lamport, Leslie.} \newblock
%                               \texttt{latex.tex}, version 2.09, date
%                               Feb. 1990.
%    \end{thebibliography}
%    \PrintIndex
%    \PrintChanges}
%
%
% \section{The documentation driver file}
%
% The next bit of code contains the documentation driver file for
% \TeX{}, i.e., the file that will produce the documentation you are
% currently reading. It will be extracted from this file by the
% \texttt{docstrip} program. Since it is the first code in the file
% one can alternatively process this file directly with \LaTeXe{} to
% obtain the documentation.
%
%    \begin{macrocode}
%<*driver>
\documentclass{ltxdoc}

\usepackage{theorem}

%<+driver>% The next few lines define theorem sets which are used
%<+driver>% in the example section of the documentation.

\theoremstyle{break}       \newtheorem{Cor}{Corollary}
\theoremstyle{plain}       \newtheorem{Exa}{Example}[section]
{\theorembodyfont{\upshape}\newtheorem{Rem}{Remark}}
\theoremstyle{marginbreak} \newtheorem{Lem}[Cor]{Lemma}
\theoremstyle{change}
\theorembodyfont{\itshape} \newtheorem{Def}[Cor]{Definition}

\theoremheaderfont{\scshape}

\RecordChanges

\begin{document}
   \DocInput{theorem.dtx}
\end{document}
%</driver>
%    \end{macrocode}
%
% \section{Definition of the Macros}
%
% If the file has been loaded before, we abort immediately. If not the
% package announces itself (this is actually done at the very top of
% the file---the way it is done isn't good style so don't copy it).
% \changes{v2.0e}{89/07/19}{Spaces removed from `typeout'.}
% \changes{v2.2b}{95/11/19}{Announce moved to top of file}
%    \begin{macrocode}
%<*package>
%\@ifundefined{theorem@style}{}{\endinput}
%\def\FMithmInfo{1995/11/19 v2.2b Theorem extension package (FMi)}
%\ProvidesPackage{theorem}[\FMithmInfo]
%    \end{macrocode}
%
%
% \subsection{Definition of theorem styles and fonts}
%
% All the definitions in this file are done globally to allow
% inputting this file inside a group.
%
% \begin{macro}{\theoremstyle}
%    Before a theorem style can be installed, the chosen style must be
%    known. For that reason, we must test to see that
%    |\th@|\meta{style} is known or, more precisely, that it is
%    different from |\relax|. If the style is not known then
%    |\th@plain| is used.
%    \begin{macrocode}
\gdef\theoremstyle#1{%
   \@ifundefined{th@#1}{\@warning
          {Unknown theoremstyle `#1'. Using `plain'}%
          \theorem@style{plain}}%
%    \end{macrocode}
%    We save the theorem style to be used in the token register
%    |\theorem@style|.
%    \begin{macrocode}
      {\theorem@style{#1}}%
%    \end{macrocode}
%    Now we ``evaluate'' the theorem style: this means, we call the
%    macro |\th@|\meta{style} which will activate the relevant
%    definitions which are contained in a separate file.  This is done
%    in a group to suppress changes to the current font.  This could
%    otherwise pose problems together with the new font selection
%    scheme\footnote{When I printed the original article using the new
%    font selection scheme I ended with a document with slanted
%    typefaces (text headings and all) simply because one of the
%    theorem styles used \texttt{\bslash sl} at toplevel.} if the
%    |\th@|\meta{style} is evaluated a second time.
% \changes{v2.1a}{90/02/15}{Group added around `th@..}
% \changes{v2.1a}{90/02/15}{Reseting of `..@begintheorem removed.}
%    \begin{macrocode}
      \begingroup
        \csname th@\the\theorem@style \endcsname
      \endgroup}
%    \end{macrocode}
% \end{macro}
%
%
% \begin{macro}{\@begintheorem}
% \changes{v2.1a}{90/02/15}{Relaxed at toplevel.}
% \begin{macro}{\@opargbegintheorem}
% \changes{v2.1a}{90/02/15}{Relaxed at toplevel.}
%    We reset |\@begintheorem| and |\@opargbegintheorem| to
%    |\relax| since these commands are no longer necessary at
%    toplevel. This will save a few tokens.
%    \begin{macrocode}
\global\let\@begintheorem\relax
\global\let\@opargbegintheorem\relax
%    \end{macrocode}
% \end{macro}
% \end{macro}
%
% \begin{macro}{\theorem@style}
%    Obviously the \textsf{token} register used above has to be
%    allocated. To assure the utmost compatibility with the
%    original \LaTeX{} definition, we set the default theorem style to
%    \texttt{plain}, which implements the usual \LaTeX{} convention.
% \changes{v2.0f}{89/07/20}{Unnecessary `global' removed.}
%    \begin{macrocode}
\newtoks\theorem@style
\global\theorem@style{plain}
%    \end{macrocode}
% \end{macro}
%
% \begin{macro}{\theorembodyfont}
% \begin{macro}{\theorem@bodyfont}
%    If the |\theorembodyfont| is set by the user then it should not
%    interact with the default font set in the theorem style. When the
%    new font selection is in force this may happen if, for example,
%    the default is |\itshape| and the new |\theorembodyfont| is
%    |\sffamily|. So we add a |\reset@font| command in front of the
%    user definition.
% \changes{v2.0f}{89/07/20}{Unnecessary `global' removed.}
% \changes{v2.0g}{89/09/19}{`theoremfont to `theorembodyfont.}
% \changes{v2.1c}{92/06/22}{`reset@font added and logic changed.}
% \changes{v2.1d}{92/10/28}{Reset to empty if arg is empty}
%    \begin{macrocode}
\gdef\theorembodyfont#1{%
%    \end{macrocode}
%    We check if the argument supplied is empty and if so put nothing
%    into the |\theorem@bodyfont| token register to allow for
%    |\theorembodyfont{}| as a mean of using the default of the
%    current |\theoremstyle|.
%    \begin{macrocode}
   \def\@tempa{#1}%
   \ifx\@tempa\@empty
    \theorem@bodyfont{}%
   \else
    \theorem@bodyfont{\reset@font#1}%
   \fi
   }
\newtoks\theorem@bodyfont
\global\theorem@bodyfont{}
%    \end{macrocode}
% \end{macro}
% \end{macro}
%
%
% \begin{macro}{\theoremheaderfont}
% \changes{v2.0g}{89/09/19}{Macro added.}
%    The font for the theorem headers is handled differently because
%    this definition applies to all theorem styles.
%    \begin{macrocode}
\gdef\theoremheaderfont#1{\gdef\theorem@headerfont{#1}%
%    \end{macrocode}
%    After using the macro once it is redefined to produce an error
%    message.
%    \begin{macrocode}
       \gdef\theoremheaderfont##1{%
        \typeout{\string\theoremheaderfont\space should be used
                 only once.}}}
%    \end{macrocode}
% \end{macro}
%
% \begin{macro}{\theorem@headerfont}
% \changes{v2.0g}{89/09/19}{Macro added.}
%    To set the |\theorem@headerfont| default we first test if
%    the new fontselection scheme is in force.
% \changes{v2.1a}{90/02/15}{Support for new font selection.}
%    \begin{macrocode}
\ifx\upshape\undefined
%    \end{macrocode}
%    If not we define it to expand into |\bfseries|. We don't use
%    |\let| just in case a following style option redefines this
%    macro.
%    \begin{macrocode}
\gdef\theorem@headerfont{\bfseries}
%    \end{macrocode}
%    Otherwise we reset the current shape before calling |\bfseries|.
%    \begin{macrocode}
\else \gdef\theorem@headerfont{\normalfont\bfseries}\fi
%    \end{macrocode}
% \end{macro}
%
% \begin{macro}{\th@plain}
% \begin{macro}{\th@break}
% \begin{macro}{\th@marginbreak}
% \begin{macro}{\th@changebreak}
% \begin{macro}{\th@change}
% \begin{macro}{\th@margin}
%    The different styles are defined in macros such as
%    |\th@plain|.  Since memory space is precious in
%    ``non-Big-versions'', we have to avoid offering too many unused
%    definitions.  Therefore we define these styles in separate files
%    that can be loaded on demand.  Thus the commands themselves only
%    load these files. We use |\@input@|, a \LaTeXe{} internal command
%    that ensures that the file will be listed with |\listfiles|.
% \changes{v2.2a}{94/02/02}{Use `@input@ to load theorem layout files}
%    \begin{macrocode}
\gdef\th@plain{\@input@{thp.sty}}
\gdef\th@break{\@input@{thb.sty}}
\gdef\th@marginbreak{\@input@{thmb.sty}}
\gdef\th@changebreak{\@input@{thcb.sty}}
\gdef\th@change{\@input@{thc.sty}}
\gdef\th@margin{\@input@{thm.sty}}
%    \end{macrocode}
%    This list will be expanded when new styles become
%    available.  For testing,  just append new theorem substyles
%    as document options.
% \end{macro}
% \end{macro}
% \end{macro}
% \end{macro}
% \end{macro}
% \end{macro}
%
%
%
% \subsection{Definition of a new theorem set}
%
% As already pointed out, a new theorem environment can be defined in
% three different ways:
% \begin{verbatim}
%    \newtheorem{Lem}{Lemma}
%    \newtheorem{Lem}{Lemma}[section]
%    \newtheorem{Lem}[Theorem]{Lemma}
%\end{verbatim}
% The function of the macro |\newtheorem| is to recognize these cases
% and then to branch into one of the three macros |\@ynthm|, |\@xnthm|
% or |\@othm|.  This mechanism is adopted unchanged from
% \cite{latex:source}; the essential point here is that, for example,
% in the second case, the arguments \texttt{Lem}, \texttt{Lemma} and
% \texttt{section} are passed over to the macro |\@xnthm|.
%
% We inspect this case first because the others present fewer problems,
% and thus are easily derived from this one.
%
% \begin{macro}{\@xnthm}
%    For our example arguments, the macro |\@xnthm| must fulfill
%    the following:
%    \begin{itemize}
%       \item
%          Define a new \LaTeX-counter `Lem'
%       \item
%          reset this counter within a |\section|
%       \item
%          define the macro |\theLem|
%       \item
%          define the environment macros |\Lem| and
%          |\endLem| using the current
%          |\theoremstyle| and |\theorem@bodyfont|.
%    \end{itemize}
%    Obviously, all this should happen only if the first argument
%    of |\@xnthm| (i.e.\ \texttt{Lem} in our example) is chosen so
%    as not to conflict with any previously defined commands or
%    environments.  This test is performed by the \LaTeX{} macro
%    |\@ifdefinable|.
%    \begin{macrocode}
\gdef\@xnthm#1#2[#3]{\expandafter\@ifdefinable\csname #1\endcsname
%    \end{macrocode}
%    Therefore, the first argument of |\@ifdefinable| is the
%    expansion (in the example, |\Lem|) of
%    |\csname||#1\endcsname|.  The second argument is
%    executed only if the test has been completed successfully.
%    \begin{macrocode}
   {%
%    \end{macrocode}
%    Now we define the new counter. The names of the \LaTeX{}
%    macros employed should speak for themselves:
% \changes{v2.2b}{1995/10/07}{Use \cs{@newctr}}
% \changes{v2.2c}{1995/11/23}{Use \cs{@newctr} properly.}
%    \begin{macrocode}
    \@definecounter{#1}\@newctr{#1}[#3]%
%    \end{macrocode}
%    Using |\@newctr| will give a proper error message if the counter
%    in |#3| is not defined.
%    In defining `|\theLem|' we must generate the desired macro
%    name by use of |\expandafter| and |\csname|.
%    \begin{macrocode}
    \expandafter\xdef\csname the#1\endcsname
%    \end{macrocode}
%    An |\xdef| is used in order to make the definition global,
%    and to ensure that it contains the replacement texts of
%    |\@thmcountersep| and |\@thmcounter|.\footnote{These
%    two macros can be defined by the document style.  Their default
%    values produce a `\texttt{.}' as separation and an arabic
%    representation of the number.} However, not everything should be
%    expanded. For example, it saves space to use |\thesection|
%    instead of its---at times---lengthy expansion.
%    \begin{macrocode}
      {\expandafter \noexpand \csname the#3\endcsname
       \@thmcountersep \@thmcounter{#1}}%
%    \end{macrocode}
%    Thus with the defaults of \LaTeX{}, |\theLem| would be
%    replaced by the command sequence |\thesection.\arabic{Lem}|.
%
%    We will now look at the definition of the macro which is executed
%    at the beginning of the actual environment (in our example this
%    macro is |\Lem|). It should be noted that we use an
%    ``|\expandafter| trick'' to expand only certain parts of the
%    replacement text at the time of the definition.
%    \begin{macrocode}
    \def\@tempa{\global\@namedef{#1}}%
    \expandafter \@tempa \expandafter{%
%    \end{macrocode}
%    First, the macro that contains the current definitions of
%    |\@begintheorem| and |\@opargtheorem| should be called
%    up.  The name of this macro---as is already known---has the form
%    |\th@|\meta{theorem\ style}; therefore, it must be called by
%    \begin{macrocode}
      \csname th@\the \theorem@style
%    \end{macrocode}
%    In addition the default theorem font should be changeable, i.e.\
%    we have to insert the contents of |\theorem@bodyfont|. For
%    that reason, we expand even further, beyond |\endcsname|,
%    and thus insert the contents of the \textsf{token} register
%    |\theorem@bodyfont| in the replacement text.
%    \begin{macrocode}
            \expandafter \endcsname \the \theorem@bodyfont
%    \end{macrocode}
%    Now it is time to call the macro |\@thm| which takes over
%    the further processing.  It has two arguments: the current
%    counter name (in our example, \texttt{Lem}), and the text of the
%    label (in our example, \texttt{Lemma}).
%    \begin{macrocode}
     \@thm{#1}{#2}}%
%    \end{macrocode}
%    With this, the `sub-definition' is complete.  The macro
%    |\@endtheorem| ends a theorem environment and is, so far, nothing
%    but an |\endtrivlist|.  (Hence it is defined globally, and not
%    within the theorem styles.\footnote{This has to be changed as
%    soon as theorem styles that change \texttt{\bslash @endtheorem}
%    exist.  In such a case, all existing styles must be changed as
%    well since they will have to reset the macro.}) Therefore, we can
%    set it equivalent to the macro that ends the theorem set (in our
%    example, |\endLem|).  However, if some day theorem styles exist
%    that do change |\@endtheorem|, we would have to use the
%    commented-out line instead.
%    \begin{macrocode}
    \global \expandafter \let \csname end#1\endcsname \@endtheorem
%   \global\@namedef{end#1}{\@endtheorem}%
%    \end{macrocode}
%    With these commands all the required definitions are employed,
%    unless the test |\@ifdefinable| has failed.  Therefore, we
%    end the second argument of this macro and with it the definition
%    of |\@xnthm|.
%    \begin{macrocode}
   }}
%    \end{macrocode}
% \end{macro}
%
% \begin{macro}{\@ynthm}
%    The definition of |\@ynthm| is completely analogous. In
%    this case the new counter that is defined is not reset within
%    another counter; thus the definition of |\the...|
%    is simplified:
%    \begin{macrocode}
\gdef\@ynthm#1#2{\expandafter\@ifdefinable\csname #1\endcsname
   {\@definecounter{#1}%
    \expandafter\xdef\csname the#1\endcsname{\@thmcounter{#1}}%
%    \end{macrocode}
%    The rest of the definition corresponds literally to that of
%    |\@xnthm|:
%    \begin{macrocode}
    \def\@tempa{\global\@namedef{#1}}\expandafter \@tempa
     \expandafter{\csname th@\the \theorem@style \expandafter
     \endcsname \the\theorem@bodyfont \@thm{#1}{#2}}%
    \global \expandafter \let \csname end#1\endcsname \@endtheorem}}
%    \end{macrocode}
% \end{macro}
%
% \begin{macro}{\@othm}
%    The definition of |\@othm| does not contain anything new.
%    \begin{macrocode}
\gdef\@othm#1[#2]#3{%
%    \end{macrocode}
% \changes{v2.2b}{95/10/07}{Build in check for counter defined}
%    We do not define a new counter but instead use one that has
%    already been defined.  Thus the only definition we need is that
%    of this pseudo-counter (i.e.\ |\the|\meta{env.\ name}).
%    First we check if |#2| corresponds to a known counter name.
%    \begin{macrocode}
  \expandafter\ifx\csname c@#2\endcsname\relax
   \@nocounterr{#2}%
  \else
   \expandafter\@ifdefinable\csname #1\endcsname
   {\expandafter \xdef \csname the#1\endcsname
     {\expandafter \noexpand \csname the#2\endcsname}%
%    \end{macrocode}
%    All other parts of the definition can be adopted from
%    |\@xnthm|. We have to remember, though, that in this case
%    the name of the current counter and the theorem label have moved
%    to the second and third arguments.
%    \begin{macrocode}
    \def\@tempa{\global\@namedef{#1}}\expandafter \@tempa
     \expandafter{\csname th@\the \theorem@style \expandafter
     \endcsname \the\theorem@bodyfont \@thm{#2}{#3}}%
    \global \expandafter \let \csname end#1\endcsname \@endtheorem}%
  \fi}
%    \end{macrocode}
% \end{macro}
%
% \subsection{Macros that are employed in a {\envfont theorem}
%    environment}
%
% \begin{macro}{\@thm}
%    The macro |\@thm| has to increase the current counter.  Then,
%    depending on whether the environment has (or does not have) an
%    optional argument, it has to branch into either |\@begintheorem|
%    or |\@opargtheorem|.
%    \begin{macrocode}
\gdef\@thm#1#2{\refstepcounter{#1}%
%    \end{macrocode}
%    Now we start a {\envfont trivlist} environment, and give
%    |\@topsep| and |\@topsepadd| the values of the \textsf{skip}
%    registers |\theorempreskipamount| and |\theorempostskipamount|.
%    The value in |\@topsep| is the vertical space that is inserted by
%    the first (and only) |\item| in our |\trivlist| whilst
%    |\@topsepadd| is inserted by |\@endparenv| at the end of that
%    {\envfont trivlist} environment.  By using these registers, we
%    obtain the desired space around a {\envfont theorem} environment.
%    \begin{macrocode}
   \trivlist
   \@topsep \theorempreskipamount               % used by first \item
   \@topsepadd \theorempostskipamount           % used by \@endparenv
%    \end{macrocode}
%    Now we have to test whether an optional argument has been given.
% \changes{v2.0d}{89/07/18}{Forgotten percent added.}
%    \begin{macrocode}
   \@ifnextchar [%
%    \end{macrocode}
%    If there is an optional argument, we will call |\@ythm|, and
%    move the arguments read back into the input stream.
%    \begin{macrocode}
   {\@ythm{#1}{#2}}%
%    \end{macrocode}
%    If not, we call |\@begintheorem|. Its first argument is the
%    name of the theorem set (hence the second argument of
%    |\@thm|). Its second argument is the macro that produces the
%    current number.
%    \begin{macrocode}
   {\@begintheorem{#2}{\csname the#1\endcsname}\ignorespaces}}
%    \end{macrocode}
% \end{macro}
%
% \begin{macro}{\@xthm}
% \begin{macro}{\@ythm}
%    Both these macros were originally called by |\@thm|.  We do
%    not need |\@xthm| anymore, hence we reset it to
%    |\relax|. The definition of |\@ythm| has not changed at
%    all from its definition in \LaTeX. In order to make the macros
%    easier to understand, we will
%    nevertheless present it (commented out).
%    \begin{macrocode}
\global\let\@xthm\relax
% \def\@ythm#1#2[#3]{\@opargbegintheorem{#2}{\csname
%           the#1\endcsname}{#3}\ignorespaces}
%</package>
%    \end{macrocode}
%    The primitive |\ignorespaces| in |\@ythm| and
%    |\@thm| is needed to remove the spaces between the
%    |\begin{...}| and the actual text.
% \end{macro}
% \end{macro}
%
%
% \subsection{Definition of the theorem substyles}
%
% As already pointed out, the theorem substyles, defined below, are
% only loaded when necessary.  Note that all these substyles, except
% \texttt{plain}, have |\slshape| as the default body font.
%
% \subsubsection{The \texttt{plain} style}
%
% \label{dok:th@plain} As the following macros use |@|, we have
% to locally set the |\catcode| of this symbol to ``letter''.
% This happens within a group, so that we do not have to worry about
% which |\catcode| that symbol had before.
%    \begin{macrocode}
%<*thp>
\begingroup \makeatletter
%    \end{macrocode}
% Since we are now within a group, we must make all definitions
% globally.  First we make sure that \texttt{theorem.sty} is loaded.
% This will allow us to use this file as a document style option
% without having to call \texttt{theorem} itself as an option.  At the
% same time, we assure that at least version 2 is loaded, since
% |\theorem@style| was not defined in earlier versions.
%    \begin{macrocode}
\@ifundefined{theorem@style}{\input{theorem.sty}}{}
\ProvidesFile{thp.sty}
             [\FMithmInfo]
%    \end{macrocode}
%
%
% \begin{macro}{\th@plain}
% \begin{macro}{\@begintheorem}
% \begin{macro}{\@opargbegintheorem}
%    |\theoremstyle{plain}| corresponds to the original
%    definition, except that the distances to the surrounding text are
%    determined by the parameters |\theorempreskipamount| and
%    |\theorempostskipamount|.
%    First we set the default body font.
%    \begin{macrocode}
\gdef\th@plain{\normalfont\itshape
%    \end{macrocode}
%    Then we define |\@begintheorem| and
%    |\@opargbegintheorem|. These two macros define how the
%    header of a theorem is typeset.  |\@opargbegintheorem| will
%    be called if a \textsf{theorem} environment with an optional
%    argument is encountered; otherwise, the header is constructed by
%    calling |\@begintheorem|.  If one of these macros is
%    executed, we are within a \textsf{trivlist} environment started by
%    |\@thm|.  So the theorem header is produced with an
%    |\item| command.
%
%    Instead of specifying the header font directly, all standard
%    theorem styles use the |\theorem@headerfont| macro to allow
%    customization.  The extra space (|\labelsep|) is necessary
%    because of problems in the \textsf{trivlist} environment.
%    \begin{macrocode}
  \def\@begintheorem##1##2{%
        \item[\hskip\labelsep \theorem@headerfont ##1\ ##2]}%
%    \end{macrocode}
%    The definition of |\@opargbegintheorem| is completely
%    analogous. The only difference is the fact that there exists a
%    third argument (which is the optional parameter of the
%    environment and contains additional information about the
%    theorem).  Customarily we enclose it in parentheses.
%    \begin{macrocode}
\def\@opargbegintheorem##1##2##3{%
   \item[\hskip\labelsep \theorem@headerfont ##1\ ##2\ (##3)]}}
%    \end{macrocode}
% \end{macro}
% \end{macro}
% \end{macro}
%
% \noindent
% We conclude with an |\endgroup| to restore the
% |\catcode| of |@|.
%    \begin{macrocode}
\endgroup
%</thp>
%    \end{macrocode}
%
%
% \subsubsection{The \texttt{break} style}
%
% This style option is stored in the file |thb.sty|.
% For the next two lines see the documentation for |\th@plain| on
% page~\pageref{dok:th@plain}.
%    \begin{macrocode}
%<*thb>
\begingroup \makeatletter
\@ifundefined{theorem@style}{\input{theorem.sty}}{}
\ProvidesFile{thb.sty}
             [\FMithmInfo]
%    \end{macrocode}
%
% \begin{macro}{\th@break}
%    |\theoremstyle{break}| produces a line break after the name
%    of the theorem.  The font is |\slshape|.  Hence, we define
%    |\th@break| as follows:
%    \begin{macrocode}
\gdef\th@break{\normalfont\slshape
  \def\@begintheorem##1##2{\item[%
%    \end{macrocode}
%    We run into the following problem: it is not possible to create
%    the header with |\item[|\meta{title}|]| and then start
%    a new line by, for example, |\mbox{}\\|. Such a definition
%    will fail whenever a list environment follows immediately.  With
%    the above construction, the |\mbox{}| causes the switch
%    |@inlabel| (cf.\ definition of |\list| and
%    |\trivlist| in \cite{latex:source}) to be set to \texttt{false}
%    and so the following list will insert additional vertical
%    space (|\topskip|). This is quite annoying. Therefore, we
%    create the line break within the |\item|. In order to ensure
%    that the text will begin at the proper position in the following
%    line, we simply pretend that the label does not take any
%    room.\footnote{This will lead to problems whenever very high
%    symbols occurring in the line tower into the
%    heading. So, something else has to be done here sometime.}
%    \begin{macrocode}
       \rlap{\vbox{\hbox{\hskip \labelsep\theorem@headerfont ##1\ ##2}%
                   \hbox{\strut}}}]}%
%    \end{macrocode}
%    Again, the definition of |\@opargbegintheorem| is
%    completely analogous.
%    \begin{macrocode}
\def\@opargbegintheorem##1##2##3{%
  \item[\rlap{\vbox{\hbox{\hskip \labelsep \theorem@headerfont
                     ##1\ ##2\ (##3)}%
                    \hbox{\strut}}}]}}
\endgroup
%</thb>
%    \end{macrocode}
% \end{macro}
%
%
% \subsubsection{The \texttt{changebreak} style}
%
%    \begin{macrocode}
% This style option is stored in the file |thcb.sty|.
%    \begin{macrocode}
%<*thcb>
\begingroup \makeatletter
\@ifundefined{theorem@style}{\input{theorem.sty}}{}
\ProvidesFile{thcb.sty}
             [\FMithmInfo]
%    \end{macrocode}
%
% \begin{macro}{\th@changebreak}
%    The \texttt{change-break} theorem style is like \texttt{break}
%    but with interchange of theorem name and theorem number.  Thus we
%    define |\th@changebreak| as follows:
%    \begin{macrocode}
\gdef\th@changebreak{\normalfont\slshape
  \def\@begintheorem##1##2{\item
       [\rlap{\vbox{\hbox{\hskip\labelsep \theorem@headerfont ##2\ ##1}%
                    \hbox{\strut}}}]}%
\def\@opargbegintheorem##1##2##3{%
  \item[\rlap{\vbox{\hbox{\hskip\labelsep \theorem@headerfont
                     ##2\ ##1\ (##3)}%
                    \hbox{\strut}}}]}}
\endgroup
%</thcb>
%    \end{macrocode}
% \end{macro}
%
%
% \subsubsection{The \texttt{change} style}
%
% This style option is stored in the file |thc.sty|.
%    \begin{macrocode}
%<*thc>
\begingroup \makeatletter
\@ifundefined{theorem@style}{\input{theorem.sty}}{}
\ProvidesFile{thc.sty}
             [\FMithmInfo]
%    \end{macrocode}
%
% \begin{macro}{\th@change}
%    The \texttt{change} theorem style corresponds to the
%    \texttt{change break} style without a linebreak after the header.
%    To say it in another way, it's the same as the \texttt{plain}
%    style but with number and name interchanged and |\slshape| as the
%    default font.
%    \begin{macrocode}
\gdef\th@change{\normalfont\slshape
  \def\@begintheorem##1##2{\item
       [\hskip\labelsep \theorem@headerfont ##2\ ##1]}%
\def\@opargbegintheorem##1##2##3{%
  \item[\hskip\labelsep \theorem@headerfont ##2\ ##1\ (##3)]}}
\endgroup
%</thc>
%    \end{macrocode}
% \end{macro}
%
%
% \subsubsection{The \texttt{marginbreak} style}
%
% This style option is the one used most often at Mainz.  It is saved
% in the file |thmb.sty|.
%    \begin{macrocode}
%<*thmb>
\begingroup \makeatletter
\@ifundefined{theorem@style}{\input{theorem.sty}}{}
\ProvidesFile{thmb.sty}
             [\FMithmInfo]
%    \end{macrocode}
%
% \begin{macro}{\th@marginbreak}
%    The \texttt{margin break} style is nearly the same as the
%    \texttt{change break} style. The only difference is the placement
%    of the theorem number.  We use |\llap| to place it in the left
%    margin.
%
%    In this style |\labelsep| denotes the separation between the
%    number and the text.
%    \begin{macrocode}
\gdef\th@marginbreak{\normalfont\slshape
  \def\@begintheorem##1##2{\item
       [\rlap{\vbox{\theorem@headerfont
                    \hbox{\llap{##2}\hskip\labelsep ##1}%
                    \hbox{\strut}}}]}%
\def\@opargbegintheorem##1##2##3{%
  \item[\rlap{\vbox{\theorem@headerfont
                    \hbox{\llap{##2}\hskip\labelsep ##1\ (##3)}%
                    \hbox{\strut}}}]}}
\endgroup
%</thmb>
%    \end{macrocode}
% \changes{v2.1c}{90/02/17}{`endgroup outside macrocode env.}
% \end{macro}
%
%
%
% \subsubsection{The \texttt{margin} style}
%
% This style option is stored in the file |thm.sty|.
%    \begin{macrocode}
%<*thm>
\begingroup \makeatletter
\@ifundefined{theorem@style}{\input{theorem.sty}}{}
\ProvidesFile{thm.sty}
             [\FMithmInfo]
%    \end{macrocode}
%
% \begin{macro}{\th@margin}
%    Again this is only a variant of the theorem styles described
%    above without any new ideas.
%    \begin{macrocode}
\gdef\th@margin{\normalfont\slshape
  \def\@begintheorem##1##2{\item
       [\theorem@headerfont \llap{##2}\hskip\labelsep ##1]}%
\def\@opargbegintheorem##1##2##3{%
  \item[\theorem@headerfont \llap{##2}\hskip\labelsep ##1\ (##3)]}}
\endgroup
%</thm>
%    \end{macrocode}
% \end{macro}
%
%
%
%
% \subsection{Final Definitions}
%
% \begin{macro}{\theorempreskipamount}
% \begin{macro}{\theorempostskipamount}
%    The \textsf{skip} parameters that regulate the vertical empty space
%    before and after the {\envfont theorem} environment have
%    to be allocated as well.
% \changes{v2.0f}{89/07/20}{Unnecessary `global' removed.}
%    \begin{macrocode}
%<*package>
\newskip\theorempreskipamount
\newskip\theorempostskipamount
%    \end{macrocode}
%    Since we have used the same values for all theorem sets, we now can
%    assign them.
%    \begin{macrocode}
\global\setlength\theorempreskipamount{12pt plus 5pt minus 3pt}
\global\setlength\theorempostskipamount{8pt plus 3pt minus 1.5pt}
%    \end{macrocode}
% \end{macro}
% \end{macro}
%
%
% \begin{macro}{\@endtheorem}
%    The same holds for the macro |\@endtheorem|, which ends a
%    {\envfont theorem} environment. Since it is the same for all
%    theorem sets, it is removed from the macros |\th@|\meta{style}.
%    It simply ends the {\envfont trivlist} environment, which was
%    begun in |\@thm|.
%    \begin{macrocode}
\global\let\@endtheorem=\endtrivlist
%    \end{macrocode}
% \end{macro}
%
%
%
% \begin{macro}{\@preamblecmds}
%    All macros defined above are to be used only in the preamble.
%    Therefore, we insert them in |\@preamblecmds| which will disable
%    them at begin document. This is done by the internal \LaTeXe{}
%    command |\@onlypreamble|.
% \changes{v2.0g}{89/09/19}{`theoremheaderfont and `theorembodyfont
%    added.}
% \changes{v2.2a}{94/02/03}{Use \LaTeXe{} interface}
%    \begin{macrocode}
\@onlypreamble\@xnthm
\@onlypreamble\@ynthm
\@onlypreamble\@othm
\@onlypreamble\newtheorem
\@onlypreamble\theoremstyle
\@onlypreamble\theorembodyfont
\@onlypreamble\theoremheaderfont
%    \end{macrocode}
% \end{macro}
%
% Finally we declare the \texttt{plain} theorem style to be the default.
% \changes{v2.1b}{90/06/04}{Fixed missing `item by declaring default
%    style.}
%    \begin{macrocode}
\theoremstyle{plain}
%</package>
%    \end{macrocode}
%
% \Finale

\endinput
}{}
\ProvidesFile{thcb.sty}
             [\FMithmInfo]
%    \end{macrocode}
%
% \begin{macro}{\th@changebreak}
%    The \texttt{change-break} theorem style is like \texttt{break}
%    but with interchange of theorem name and theorem number.  Thus we
%    define |\th@changebreak| as follows:
%    \begin{macrocode}
\gdef\th@changebreak{\normalfont\slshape
  \def\@begintheorem##1##2{\item
       [\rlap{\vbox{\hbox{\hskip\labelsep \theorem@headerfont ##2\ ##1}%
                    \hbox{\strut}}}]}%
\def\@opargbegintheorem##1##2##3{%
  \item[\rlap{\vbox{\hbox{\hskip\labelsep \theorem@headerfont
                     ##2\ ##1\ (##3)}%
                    \hbox{\strut}}}]}}
\endgroup
%</thcb>
%    \end{macrocode}
% \end{macro}
%
%
% \subsubsection{The \texttt{change} style}
%
% This style option is stored in the file |thc.sty|.
%    \begin{macrocode}
%<*thc>
\begingroup \makeatletter
\@ifundefined{theorem@style}{% \iffalse meta-comment
%
% Copyright (C) 1993-2024
%
% The LaTeX Project and any individual authors listed elsewhere
% in this file.
%
% This file is part of the Standard LaTeX `Tools Bundle'.
% -------------------------------------------------------
%
% It may be distributed and/or modified under the
% conditions of the LaTeX Project Public License, either version 1.3c
% of this license or (at your option) any later version.
% The latest version of this license is in
%    https://www.latex-project.org/lppl.txt
% and version 1.3c or later is part of all distributions of LaTeX
% version 2005/12/01 or later.
%
% The list of all files belonging to the LaTeX `Tools Bundle' is
% given in the file `manifest.txt'.
%
% \fi
%
%%
%
%
\def\FMithmInfo{2023/07/05 v2.2c Theorem extension package (FMi)}
%
% \ProvidesFile{theorem.dtx}[\FMithmInfo]
%\iffalse   % this is a METACOMMENT !
%%
%% Package `theorem' to use with LaTeX2e
%% Copyright (C) 1989-1995 Frank Mittelbach, all rights reserved.
%<+package>\@ifundefined{theorem@style}{}{\endinput}
%<+package>\NeedsTeXFormat{LaTeX2e}[1995/06/01]
%<+package>\ProvidesPackage{theorem}
%<+package>         [\FMithmInfo]
%
% \fi
%
% \changes{v1.0}{87/05/07}{First draft.}
% \changes{v1.0a}{87/05/07}{`thistheoremstyle' added.}
% \changes{v1.1a}{87/07/29}{New concept: redefine the internal
%                          \protect\LaTeX{} commands `@xnthm' etc.}
% \changes{v1.1b}{87/07/31}{`theoremfont' used to specify font.}
% \changes{v1.2a}{88/01/08}{`@endtheorem' now globally defined.}
% \changes{v1.2b}{88/01/11}{`newtheorem' only allowed in the preamble.}
% \changes{v1.3a}{88/02/14}{`theoremfont' now a toks register.}
%
% \changes{v2.0a}{88/11/16}{New implementation and new documentation.}
% \changes{v2.0a}{88/11/16}{No layout changes but internal modifications
%                           old substyles won't work.}
% \changes{v2.0b}{89/01/30}{`global' used for all definitions.}
% \changes{v2.0c}{89/06/15}{Documentation translated to English.
%                            Thanks to LLL and Nina}
% \changes{v2.1a}{90/02/15}{Support for new font selection.}
% \changes{v2.2a}{94/02/03}{Cleanup for LaTeX2e}
%
% \DoNotIndex{\@endparenv,\@ifdefinable,\@ifnextchar,\@ifundefined}
% \DoNotIndex{\@namedef,\@nameuse,\@spaces,\@tempa,\@warning}
% \DoNotIndex{\ }
% \DoNotIndex{\begingroup,\bfseries}
% \DoNotIndex{\csname}
% \DoNotIndex{\def,\do,\docdate}
% \DoNotIndex{\endcsname,\endgroup,\endinput,\endtrivlist,\expandafter}
% \DoNotIndex{\filedate,\fileversion}
% \DoNotIndex{\gdef,\global}
% \DoNotIndex{\hbox,\hskip}
% \DoNotIndex{\ignorespaces,\input,\itshape,\item}
% \DoNotIndex{\labelsep,\let,\llap}
% \DoNotIndex{\makeatletter}
% \DoNotIndex{\noexpand}
% \DoNotIndex{\refstepcounter,\relax,\rlap}
% \DoNotIndex{\setlength,\slshape,\space,\string,\strut}
% \DoNotIndex{\the,\trivlist,\typeout}
% \DoNotIndex{\vbox}
% \DoNotIndex{\xdef}
%
% \def\deflabel#1{\ttfamily #1\hfill}
% \def\deflist#1{\begin{list}{}{\settowidth\labelwidth{\ttfamily #1}%
%                               \setlength\leftmargin\labelwidth
%                               \addtolength\leftmargin\labelsep
%                               \let\makelabel\deflabel}}
% \def\enddeflist{\end{list}}
% \def\envfont{\normalfont\ttfamily}
%
% \MakeShortVerb{\|}
%
% \newcommand{\TUB}{{\sl TUGboat\/}}
%
% \GetFileInfo{theorem.dtx}
% \title{An Extension of the
%        \LaTeX{} theorem environment\thanks{This file
%        has version number \fileversion, last
%        revised \filedate.}}
% \author{Frank Mittelbach}
%
% \MaintainedByLaTeXTeam{tools}
% \maketitle
%
% \begin{abstract}
%    The macros described in this paper yield an extension of the
%    \LaTeX{} theorem mechanism. It is designed to satisfy the
%    different requirements of various journals. Thus, the
%    layout of the ``theorems'' can be manipulated by determining a
%    ``style''. This article describes not only the use, but
%    also the definition, of the necessary macros.
% \end{abstract}
%
% \section*{Preface to version 2.2}
%
% For \LaTeXe{} this package did not need any fundamental changes. I
% only modified the messages generated so that theorem layout styles
% will show up with the |\listfiles| command and cleaned the section
% on the New Font Selection Scheme since this is now included in
% \LaTeX.
%
% \section*{Preface to version 2.1}
%
% This version is identical to 2.0g described in \TUB\ 10\#3 except
% for some internal defaults which are now set depending on the used
% font selection scheme.
%
% This was done to avoid unpleasant surprises if the new font selection
% scheme is in force. For further details see
% section~\ref{sec:newfonts} and \cite{bk:GMS94}.
%
% \section{Introduction}
%
% For our purposes here, ``theorems'' are labelled enunciations,
% often set off from the main text by extra space and a font change.
% Theorems, corollaries, conjectures, definitions, and remarks
% are all instances of ``theorems''.  The ``header'' of these structures
% is composed of a label (such as \textsc{Theorem} or \textsc{Remark})
% and a number which serializes an item in the sequence of items
% with the same label.
%
% Shortly after the introduction of \LaTeX{} at the Fachbereich
% Mathematik in Mainz, the desire to manipulate the layout of
% ``theorems'' arose.  In Mainz, the following two conventions came
% into general use:
% \begin{enumerate}
%   \item The number of the theorem is shown in the margin.
%   \item There is a line break at
%         the end of the theorem header.
% \end{enumerate}
%
% Additionally, some journals require different formats which depend on
% the ``sort of theorem'': e.g.\ often remarks and definitions are set
% in |\upshape|, while |\itshape| is employed for main theorems.
%
% Confronted with these requirements, a theorem environment was
% developed in Mainz which allows separate determination of the
% layout of the ``theorems sets'', comparable to |\pagestyle|.
%
%
% \section{The user interface}
% \subsection{Defining new theorem sets}
%
% \DescribeMacro\newtheorem
% As in the original \LaTeX{} version, the command |\newtheorem|
% defines a new ``theorem set'' or ``theorem-like structure''.
% Two required arguments name the new environment and give the text
% to be typeset with each instance of the new ``set'', while
% an optional argument determines how
% the ``set'' is enumerated:
% \begin{description}
%    \item[\ttfamily \bslash newtheorem\{foo\}\{bar\}]
%       The theorem set {\envfont foo} (whose name is \texttt{bar})
%       uses its own counter.
%    \item[\ttfamily \bslash newtheorem\{foo2\}{[foo]}\{bar2\}]
%       The theorem set {\envfont foo2} (printed name \texttt{bar2})
%       uses the same counter as the theorem set \texttt{foo}.
%    \item[\ttfamily \bslash newtheorem\{foo3\}\{bar3\}{[section]}]
%       The theorem set {\envfont foo3} (printed name \texttt{bar3}) is
%       enumerated within the counter \texttt{section}, i.e.\ with every
%       new |\section| the enumeration begins again with 1, and
%       the enumeration is composed from the section-number and the
%       theorem counter itself.
% \end{description}
%
% \DescribeMacro\theoremstyle
% Additionally, the command |\theoremstyle| can define the layout
% of various, or all, theorem sets.  It should be noted that any
% theorem set defined by |\newtheorem| is typeset in the
% |\theoremstyle| that is current at the time of the definition.
% Thus, the following
% \begin{verbatim}
%    \theoremstyle{break}        \newtheorem{Cor}{Corollary}
%    \theoremstyle{plain}        \newtheorem{Exa}{Example}[section]
%\end{verbatim}                             ^^A No blank at line start
% leads to the result that the set {\envfont Cor} is formatted in the
% style \texttt{break}, while the set {\envfont Exa} and all the
% following ones are formatted in the style \texttt{plain}, unless
% another |\theoremstyle| follows.  Since the definitions installed by
% |\newtheorem| are global, one also can limit |\theoremstyle| locally
% by grouping braces.
%
% \DescribeMacro\theorembodyfont
% The choice of the font for the theorem body is completely
% independent of the chosen |\theoremstyle|; this has proven to
% be very advantageous.  For example,
% \begin{verbatim}
%   {\theorembodyfont{\upshape}          \newtheorem{Rem}{Remark}}
%\end{verbatim}
% defines a theorem set {\envfont Rem}, which will be set in
% |\upshape| in the current layout (which in our example is
% \texttt{plain}).  As with |\theoremstyle|, the |\theorembodyfont|
% chosen is that current at the time of |\newtheorem|.  If
% |\theorembodyfont| is not specified or one defines
% |\theorembodyfont{}|, then the font used will be that defined by the
% |\theoremstyle|.
%
% \DescribeMacro\theoremheaderfont It is also possible to customize
% the font used for the theorem headers.  This is, however, a global
% declaration, and therefore there should be at most one
% |\theoremheaderfont| declaration in the preamble.\footnote{If it is
% actually necessary to have different header fonts, one has to define
% new theorem styles (substituting the desired font) or specify the
% information directly in the \texttt{\bslash newtheorem} declaration
% (the unclean variant).}
%
% \DescribeMacro\theorempreskipamount
% \DescribeMacro\theorempostskipamount Two additional parameters
% affect the vertical space around the theorem environments:
% |\theorempreskipamount| and |\theorempostskipamount| define,
% respectively, the spacing before and after such an environment.
% These parameters apply for all theorem sets and can be manipulated
% with the ordinary length macros.  They are rubber lengths
% (`\textsf{skips}'), and therefore can contain \texttt{plus} and
% \texttt{minus} parts.
%
% Since the definition of theorem sets should---most sensibly---be
% placed in the preamble, we only allow installation there.
% It is therefore possible to release the memory used here after
% |\begin{document}|, in order to make room for other
% applications.
%
%
% \subsection{Existing theorem styles}
%
% The following theorem styles exist to date:
% \begin{deflist}{marginbreak}
%    \item[plain]
%       This theorem style emulates the original \LaTeX{} definition,
%       except that additionally the parameters
%       |\theorem...skipamount| are used.
%    \item[break]
%       In this style, the theorem header is followed by a line
%       break.
%    \item[marginbreak]
%       The theorem number is set in the margin, and there is a line
%       break as in \texttt{break}.
%    \item[changebreak]
%       Like \texttt{break}, but with header number and text
%       interchanged.
%    \item[change]
%       Header number and text are interchanged, without a line break.
%    \item[margin]
%       The number is set in the left margin, without a line break.
% \end{deflist}
% All styles (except \texttt{plain}) select |\slshape| as the default
% |\theorembodyfont|.
%
%
% \subsection{Examples}
%
% Given the above theorem sets {\envfont Cor}, {\envfont Exa}
% and {\envfont Rem},
% suppose that the preamble also contains the declarations:
% \begin{verbatim}
%    \theoremstyle{marginbreak}   \newtheorem{Lem}[Cor]{Lemma}
%    \theoremstyle{change}
%    \theorembodyfont{\itshape}        \newtheorem{Def}[Cor]{Definition}
%
%    \theoremheaderfont{\scshape}
%\end{verbatim}
% Then the following are some typical examples of the typeset output
% resulting from their use.
%
% \begin{Cor}
%    This is a sentence typeset in the theorem environment {\envfont
%    Cor}.
% \end{Cor}
%
% \begin{Exa}
%    This is a sentence typeset in the theorem environment {\envfont
%    Exa}.
% \end{Exa}
%
% \begin{Rem}
%    This is a sentence typeset in the theorem environment {\envfont
%    Rem}.
% \end{Rem}
%
% \begin{Lem}[Ben User]
%    This is a sentence typeset in the theorem environment {\envfont
%    Lem}.
% \end{Lem}
%
% \begin{Def}[Very impressive Definition]
%    This is a sentence typeset in the theorem environment {\envfont
%    Def}.
% \end{Def}
%
% The last two examples show the effect of the optional argument to
% a theorem environment (it is the text typeset in parentheses).
%
%
% \section{Special Considerations}
% \label{sec:newfonts}
%
% Theoremheader and body are implemented as a unit. This means that
% the |\theoremheaderfont|  will inherit characteristics of the
% |\theorembodyfont| in \LaTeXe. Thus, if for example
% |\theorembodyfont| is |\itshape| and |\theoremheaderfont|
% is |\bfseries| the font selected for the header will have the
% characteristics `bold extended italic'. If this is not desired one
% should set the |\theoremheaderfont| to something like
% \begin{verbatim}
%      \theoremheaderfont{\normalfont\bfseries}
%\end{verbatim}
% i.e.\ supplying all necessary font information explicitly.
%
%
%
% \section{Acknowledgements}
%
% The publication of this set of macros was only possible with the
% help of Christina Busse (translating the manuscript into English),
% Joachim Pense (playing the r\^ole of typist), Chris Rowley
% (looking everything over) and many others providing useful
% suggestions.
%
% \MaybeStop{%
%    \begin{thebibliography}{9}
%       \bibitem{bk:GMS94} \textsc{M.~Goossens}, \textsc{F.~Mittelbach}
%         and \textsc{A.~Samarin}.
%         \newblock The \LaTeX{} Companion.
%         \newblock
%         Addison-Wesley, Reading, Massachusetts, 1994.
%       \bibitem{latex:source} \textsc{Lamport, Leslie.} \newblock
%                               \texttt{latex.tex}, version 2.09, date
%                               Feb. 1990.
%    \end{thebibliography}
%    \PrintIndex
%    \PrintChanges}
%
%
% \section{The documentation driver file}
%
% The next bit of code contains the documentation driver file for
% \TeX{}, i.e., the file that will produce the documentation you are
% currently reading. It will be extracted from this file by the
% \texttt{docstrip} program. Since it is the first code in the file
% one can alternatively process this file directly with \LaTeXe{} to
% obtain the documentation.
%
%    \begin{macrocode}
%<*driver>
\documentclass{ltxdoc}

\usepackage{theorem}

%<+driver>% The next few lines define theorem sets which are used
%<+driver>% in the example section of the documentation.

\theoremstyle{break}       \newtheorem{Cor}{Corollary}
\theoremstyle{plain}       \newtheorem{Exa}{Example}[section]
{\theorembodyfont{\upshape}\newtheorem{Rem}{Remark}}
\theoremstyle{marginbreak} \newtheorem{Lem}[Cor]{Lemma}
\theoremstyle{change}
\theorembodyfont{\itshape} \newtheorem{Def}[Cor]{Definition}

\theoremheaderfont{\scshape}

\RecordChanges

\begin{document}
   \DocInput{theorem.dtx}
\end{document}
%</driver>
%    \end{macrocode}
%
% \section{Definition of the Macros}
%
% If the file has been loaded before, we abort immediately. If not the
% package announces itself (this is actually done at the very top of
% the file---the way it is done isn't good style so don't copy it).
% \changes{v2.0e}{89/07/19}{Spaces removed from `typeout'.}
% \changes{v2.2b}{95/11/19}{Announce moved to top of file}
%    \begin{macrocode}
%<*package>
%\@ifundefined{theorem@style}{}{\endinput}
%\def\FMithmInfo{1995/11/19 v2.2b Theorem extension package (FMi)}
%\ProvidesPackage{theorem}[\FMithmInfo]
%    \end{macrocode}
%
%
% \subsection{Definition of theorem styles and fonts}
%
% All the definitions in this file are done globally to allow
% inputting this file inside a group.
%
% \begin{macro}{\theoremstyle}
%    Before a theorem style can be installed, the chosen style must be
%    known. For that reason, we must test to see that
%    |\th@|\meta{style} is known or, more precisely, that it is
%    different from |\relax|. If the style is not known then
%    |\th@plain| is used.
%    \begin{macrocode}
\gdef\theoremstyle#1{%
   \@ifundefined{th@#1}{\@warning
          {Unknown theoremstyle `#1'. Using `plain'}%
          \theorem@style{plain}}%
%    \end{macrocode}
%    We save the theorem style to be used in the token register
%    |\theorem@style|.
%    \begin{macrocode}
      {\theorem@style{#1}}%
%    \end{macrocode}
%    Now we ``evaluate'' the theorem style: this means, we call the
%    macro |\th@|\meta{style} which will activate the relevant
%    definitions which are contained in a separate file.  This is done
%    in a group to suppress changes to the current font.  This could
%    otherwise pose problems together with the new font selection
%    scheme\footnote{When I printed the original article using the new
%    font selection scheme I ended with a document with slanted
%    typefaces (text headings and all) simply because one of the
%    theorem styles used \texttt{\bslash sl} at toplevel.} if the
%    |\th@|\meta{style} is evaluated a second time.
% \changes{v2.1a}{90/02/15}{Group added around `th@..}
% \changes{v2.1a}{90/02/15}{Reseting of `..@begintheorem removed.}
%    \begin{macrocode}
      \begingroup
        \csname th@\the\theorem@style \endcsname
      \endgroup}
%    \end{macrocode}
% \end{macro}
%
%
% \begin{macro}{\@begintheorem}
% \changes{v2.1a}{90/02/15}{Relaxed at toplevel.}
% \begin{macro}{\@opargbegintheorem}
% \changes{v2.1a}{90/02/15}{Relaxed at toplevel.}
%    We reset |\@begintheorem| and |\@opargbegintheorem| to
%    |\relax| since these commands are no longer necessary at
%    toplevel. This will save a few tokens.
%    \begin{macrocode}
\global\let\@begintheorem\relax
\global\let\@opargbegintheorem\relax
%    \end{macrocode}
% \end{macro}
% \end{macro}
%
% \begin{macro}{\theorem@style}
%    Obviously the \textsf{token} register used above has to be
%    allocated. To assure the utmost compatibility with the
%    original \LaTeX{} definition, we set the default theorem style to
%    \texttt{plain}, which implements the usual \LaTeX{} convention.
% \changes{v2.0f}{89/07/20}{Unnecessary `global' removed.}
%    \begin{macrocode}
\newtoks\theorem@style
\global\theorem@style{plain}
%    \end{macrocode}
% \end{macro}
%
% \begin{macro}{\theorembodyfont}
% \begin{macro}{\theorem@bodyfont}
%    If the |\theorembodyfont| is set by the user then it should not
%    interact with the default font set in the theorem style. When the
%    new font selection is in force this may happen if, for example,
%    the default is |\itshape| and the new |\theorembodyfont| is
%    |\sffamily|. So we add a |\reset@font| command in front of the
%    user definition.
% \changes{v2.0f}{89/07/20}{Unnecessary `global' removed.}
% \changes{v2.0g}{89/09/19}{`theoremfont to `theorembodyfont.}
% \changes{v2.1c}{92/06/22}{`reset@font added and logic changed.}
% \changes{v2.1d}{92/10/28}{Reset to empty if arg is empty}
%    \begin{macrocode}
\gdef\theorembodyfont#1{%
%    \end{macrocode}
%    We check if the argument supplied is empty and if so put nothing
%    into the |\theorem@bodyfont| token register to allow for
%    |\theorembodyfont{}| as a mean of using the default of the
%    current |\theoremstyle|.
%    \begin{macrocode}
   \def\@tempa{#1}%
   \ifx\@tempa\@empty
    \theorem@bodyfont{}%
   \else
    \theorem@bodyfont{\reset@font#1}%
   \fi
   }
\newtoks\theorem@bodyfont
\global\theorem@bodyfont{}
%    \end{macrocode}
% \end{macro}
% \end{macro}
%
%
% \begin{macro}{\theoremheaderfont}
% \changes{v2.0g}{89/09/19}{Macro added.}
%    The font for the theorem headers is handled differently because
%    this definition applies to all theorem styles.
%    \begin{macrocode}
\gdef\theoremheaderfont#1{\gdef\theorem@headerfont{#1}%
%    \end{macrocode}
%    After using the macro once it is redefined to produce an error
%    message.
%    \begin{macrocode}
       \gdef\theoremheaderfont##1{%
        \typeout{\string\theoremheaderfont\space should be used
                 only once.}}}
%    \end{macrocode}
% \end{macro}
%
% \begin{macro}{\theorem@headerfont}
% \changes{v2.0g}{89/09/19}{Macro added.}
%    To set the |\theorem@headerfont| default we first test if
%    the new fontselection scheme is in force.
% \changes{v2.1a}{90/02/15}{Support for new font selection.}
%    \begin{macrocode}
\ifx\upshape\undefined
%    \end{macrocode}
%    If not we define it to expand into |\bfseries|. We don't use
%    |\let| just in case a following style option redefines this
%    macro.
%    \begin{macrocode}
\gdef\theorem@headerfont{\bfseries}
%    \end{macrocode}
%    Otherwise we reset the current shape before calling |\bfseries|.
%    \begin{macrocode}
\else \gdef\theorem@headerfont{\normalfont\bfseries}\fi
%    \end{macrocode}
% \end{macro}
%
% \begin{macro}{\th@plain}
% \begin{macro}{\th@break}
% \begin{macro}{\th@marginbreak}
% \begin{macro}{\th@changebreak}
% \begin{macro}{\th@change}
% \begin{macro}{\th@margin}
%    The different styles are defined in macros such as
%    |\th@plain|.  Since memory space is precious in
%    ``non-Big-versions'', we have to avoid offering too many unused
%    definitions.  Therefore we define these styles in separate files
%    that can be loaded on demand.  Thus the commands themselves only
%    load these files. We use |\@input@|, a \LaTeXe{} internal command
%    that ensures that the file will be listed with |\listfiles|.
% \changes{v2.2a}{94/02/02}{Use `@input@ to load theorem layout files}
%    \begin{macrocode}
\gdef\th@plain{\@input@{thp.sty}}
\gdef\th@break{\@input@{thb.sty}}
\gdef\th@marginbreak{\@input@{thmb.sty}}
\gdef\th@changebreak{\@input@{thcb.sty}}
\gdef\th@change{\@input@{thc.sty}}
\gdef\th@margin{\@input@{thm.sty}}
%    \end{macrocode}
%    This list will be expanded when new styles become
%    available.  For testing,  just append new theorem substyles
%    as document options.
% \end{macro}
% \end{macro}
% \end{macro}
% \end{macro}
% \end{macro}
% \end{macro}
%
%
%
% \subsection{Definition of a new theorem set}
%
% As already pointed out, a new theorem environment can be defined in
% three different ways:
% \begin{verbatim}
%    \newtheorem{Lem}{Lemma}
%    \newtheorem{Lem}{Lemma}[section]
%    \newtheorem{Lem}[Theorem]{Lemma}
%\end{verbatim}
% The function of the macro |\newtheorem| is to recognize these cases
% and then to branch into one of the three macros |\@ynthm|, |\@xnthm|
% or |\@othm|.  This mechanism is adopted unchanged from
% \cite{latex:source}; the essential point here is that, for example,
% in the second case, the arguments \texttt{Lem}, \texttt{Lemma} and
% \texttt{section} are passed over to the macro |\@xnthm|.
%
% We inspect this case first because the others present fewer problems,
% and thus are easily derived from this one.
%
% \begin{macro}{\@xnthm}
%    For our example arguments, the macro |\@xnthm| must fulfill
%    the following:
%    \begin{itemize}
%       \item
%          Define a new \LaTeX-counter `Lem'
%       \item
%          reset this counter within a |\section|
%       \item
%          define the macro |\theLem|
%       \item
%          define the environment macros |\Lem| and
%          |\endLem| using the current
%          |\theoremstyle| and |\theorem@bodyfont|.
%    \end{itemize}
%    Obviously, all this should happen only if the first argument
%    of |\@xnthm| (i.e.\ \texttt{Lem} in our example) is chosen so
%    as not to conflict with any previously defined commands or
%    environments.  This test is performed by the \LaTeX{} macro
%    |\@ifdefinable|.
%    \begin{macrocode}
\gdef\@xnthm#1#2[#3]{\expandafter\@ifdefinable\csname #1\endcsname
%    \end{macrocode}
%    Therefore, the first argument of |\@ifdefinable| is the
%    expansion (in the example, |\Lem|) of
%    |\csname||#1\endcsname|.  The second argument is
%    executed only if the test has been completed successfully.
%    \begin{macrocode}
   {%
%    \end{macrocode}
%    Now we define the new counter. The names of the \LaTeX{}
%    macros employed should speak for themselves:
% \changes{v2.2b}{1995/10/07}{Use \cs{@newctr}}
% \changes{v2.2c}{1995/11/23}{Use \cs{@newctr} properly.}
%    \begin{macrocode}
    \@definecounter{#1}\@newctr{#1}[#3]%
%    \end{macrocode}
%    Using |\@newctr| will give a proper error message if the counter
%    in |#3| is not defined.
%    In defining `|\theLem|' we must generate the desired macro
%    name by use of |\expandafter| and |\csname|.
%    \begin{macrocode}
    \expandafter\xdef\csname the#1\endcsname
%    \end{macrocode}
%    An |\xdef| is used in order to make the definition global,
%    and to ensure that it contains the replacement texts of
%    |\@thmcountersep| and |\@thmcounter|.\footnote{These
%    two macros can be defined by the document style.  Their default
%    values produce a `\texttt{.}' as separation and an arabic
%    representation of the number.} However, not everything should be
%    expanded. For example, it saves space to use |\thesection|
%    instead of its---at times---lengthy expansion.
%    \begin{macrocode}
      {\expandafter \noexpand \csname the#3\endcsname
       \@thmcountersep \@thmcounter{#1}}%
%    \end{macrocode}
%    Thus with the defaults of \LaTeX{}, |\theLem| would be
%    replaced by the command sequence |\thesection.\arabic{Lem}|.
%
%    We will now look at the definition of the macro which is executed
%    at the beginning of the actual environment (in our example this
%    macro is |\Lem|). It should be noted that we use an
%    ``|\expandafter| trick'' to expand only certain parts of the
%    replacement text at the time of the definition.
%    \begin{macrocode}
    \def\@tempa{\global\@namedef{#1}}%
    \expandafter \@tempa \expandafter{%
%    \end{macrocode}
%    First, the macro that contains the current definitions of
%    |\@begintheorem| and |\@opargtheorem| should be called
%    up.  The name of this macro---as is already known---has the form
%    |\th@|\meta{theorem\ style}; therefore, it must be called by
%    \begin{macrocode}
      \csname th@\the \theorem@style
%    \end{macrocode}
%    In addition the default theorem font should be changeable, i.e.\
%    we have to insert the contents of |\theorem@bodyfont|. For
%    that reason, we expand even further, beyond |\endcsname|,
%    and thus insert the contents of the \textsf{token} register
%    |\theorem@bodyfont| in the replacement text.
%    \begin{macrocode}
            \expandafter \endcsname \the \theorem@bodyfont
%    \end{macrocode}
%    Now it is time to call the macro |\@thm| which takes over
%    the further processing.  It has two arguments: the current
%    counter name (in our example, \texttt{Lem}), and the text of the
%    label (in our example, \texttt{Lemma}).
%    \begin{macrocode}
     \@thm{#1}{#2}}%
%    \end{macrocode}
%    With this, the `sub-definition' is complete.  The macro
%    |\@endtheorem| ends a theorem environment and is, so far, nothing
%    but an |\endtrivlist|.  (Hence it is defined globally, and not
%    within the theorem styles.\footnote{This has to be changed as
%    soon as theorem styles that change \texttt{\bslash @endtheorem}
%    exist.  In such a case, all existing styles must be changed as
%    well since they will have to reset the macro.}) Therefore, we can
%    set it equivalent to the macro that ends the theorem set (in our
%    example, |\endLem|).  However, if some day theorem styles exist
%    that do change |\@endtheorem|, we would have to use the
%    commented-out line instead.
%    \begin{macrocode}
    \global \expandafter \let \csname end#1\endcsname \@endtheorem
%   \global\@namedef{end#1}{\@endtheorem}%
%    \end{macrocode}
%    With these commands all the required definitions are employed,
%    unless the test |\@ifdefinable| has failed.  Therefore, we
%    end the second argument of this macro and with it the definition
%    of |\@xnthm|.
%    \begin{macrocode}
   }}
%    \end{macrocode}
% \end{macro}
%
% \begin{macro}{\@ynthm}
%    The definition of |\@ynthm| is completely analogous. In
%    this case the new counter that is defined is not reset within
%    another counter; thus the definition of |\the...|
%    is simplified:
%    \begin{macrocode}
\gdef\@ynthm#1#2{\expandafter\@ifdefinable\csname #1\endcsname
   {\@definecounter{#1}%
    \expandafter\xdef\csname the#1\endcsname{\@thmcounter{#1}}%
%    \end{macrocode}
%    The rest of the definition corresponds literally to that of
%    |\@xnthm|:
%    \begin{macrocode}
    \def\@tempa{\global\@namedef{#1}}\expandafter \@tempa
     \expandafter{\csname th@\the \theorem@style \expandafter
     \endcsname \the\theorem@bodyfont \@thm{#1}{#2}}%
    \global \expandafter \let \csname end#1\endcsname \@endtheorem}}
%    \end{macrocode}
% \end{macro}
%
% \begin{macro}{\@othm}
%    The definition of |\@othm| does not contain anything new.
%    \begin{macrocode}
\gdef\@othm#1[#2]#3{%
%    \end{macrocode}
% \changes{v2.2b}{95/10/07}{Build in check for counter defined}
%    We do not define a new counter but instead use one that has
%    already been defined.  Thus the only definition we need is that
%    of this pseudo-counter (i.e.\ |\the|\meta{env.\ name}).
%    First we check if |#2| corresponds to a known counter name.
%    \begin{macrocode}
  \expandafter\ifx\csname c@#2\endcsname\relax
   \@nocounterr{#2}%
  \else
   \expandafter\@ifdefinable\csname #1\endcsname
   {\expandafter \xdef \csname the#1\endcsname
     {\expandafter \noexpand \csname the#2\endcsname}%
%    \end{macrocode}
%    All other parts of the definition can be adopted from
%    |\@xnthm|. We have to remember, though, that in this case
%    the name of the current counter and the theorem label have moved
%    to the second and third arguments.
%    \begin{macrocode}
    \def\@tempa{\global\@namedef{#1}}\expandafter \@tempa
     \expandafter{\csname th@\the \theorem@style \expandafter
     \endcsname \the\theorem@bodyfont \@thm{#2}{#3}}%
    \global \expandafter \let \csname end#1\endcsname \@endtheorem}%
  \fi}
%    \end{macrocode}
% \end{macro}
%
% \subsection{Macros that are employed in a {\envfont theorem}
%    environment}
%
% \begin{macro}{\@thm}
%    The macro |\@thm| has to increase the current counter.  Then,
%    depending on whether the environment has (or does not have) an
%    optional argument, it has to branch into either |\@begintheorem|
%    or |\@opargtheorem|.
%    \begin{macrocode}
\gdef\@thm#1#2{\refstepcounter{#1}%
%    \end{macrocode}
%    Now we start a {\envfont trivlist} environment, and give
%    |\@topsep| and |\@topsepadd| the values of the \textsf{skip}
%    registers |\theorempreskipamount| and |\theorempostskipamount|.
%    The value in |\@topsep| is the vertical space that is inserted by
%    the first (and only) |\item| in our |\trivlist| whilst
%    |\@topsepadd| is inserted by |\@endparenv| at the end of that
%    {\envfont trivlist} environment.  By using these registers, we
%    obtain the desired space around a {\envfont theorem} environment.
%    \begin{macrocode}
   \trivlist
   \@topsep \theorempreskipamount               % used by first \item
   \@topsepadd \theorempostskipamount           % used by \@endparenv
%    \end{macrocode}
%    Now we have to test whether an optional argument has been given.
% \changes{v2.0d}{89/07/18}{Forgotten percent added.}
%    \begin{macrocode}
   \@ifnextchar [%
%    \end{macrocode}
%    If there is an optional argument, we will call |\@ythm|, and
%    move the arguments read back into the input stream.
%    \begin{macrocode}
   {\@ythm{#1}{#2}}%
%    \end{macrocode}
%    If not, we call |\@begintheorem|. Its first argument is the
%    name of the theorem set (hence the second argument of
%    |\@thm|). Its second argument is the macro that produces the
%    current number.
%    \begin{macrocode}
   {\@begintheorem{#2}{\csname the#1\endcsname}\ignorespaces}}
%    \end{macrocode}
% \end{macro}
%
% \begin{macro}{\@xthm}
% \begin{macro}{\@ythm}
%    Both these macros were originally called by |\@thm|.  We do
%    not need |\@xthm| anymore, hence we reset it to
%    |\relax|. The definition of |\@ythm| has not changed at
%    all from its definition in \LaTeX. In order to make the macros
%    easier to understand, we will
%    nevertheless present it (commented out).
%    \begin{macrocode}
\global\let\@xthm\relax
% \def\@ythm#1#2[#3]{\@opargbegintheorem{#2}{\csname
%           the#1\endcsname}{#3}\ignorespaces}
%</package>
%    \end{macrocode}
%    The primitive |\ignorespaces| in |\@ythm| and
%    |\@thm| is needed to remove the spaces between the
%    |\begin{...}| and the actual text.
% \end{macro}
% \end{macro}
%
%
% \subsection{Definition of the theorem substyles}
%
% As already pointed out, the theorem substyles, defined below, are
% only loaded when necessary.  Note that all these substyles, except
% \texttt{plain}, have |\slshape| as the default body font.
%
% \subsubsection{The \texttt{plain} style}
%
% \label{dok:th@plain} As the following macros use |@|, we have
% to locally set the |\catcode| of this symbol to ``letter''.
% This happens within a group, so that we do not have to worry about
% which |\catcode| that symbol had before.
%    \begin{macrocode}
%<*thp>
\begingroup \makeatletter
%    \end{macrocode}
% Since we are now within a group, we must make all definitions
% globally.  First we make sure that \texttt{theorem.sty} is loaded.
% This will allow us to use this file as a document style option
% without having to call \texttt{theorem} itself as an option.  At the
% same time, we assure that at least version 2 is loaded, since
% |\theorem@style| was not defined in earlier versions.
%    \begin{macrocode}
\@ifundefined{theorem@style}{\input{theorem.sty}}{}
\ProvidesFile{thp.sty}
             [\FMithmInfo]
%    \end{macrocode}
%
%
% \begin{macro}{\th@plain}
% \begin{macro}{\@begintheorem}
% \begin{macro}{\@opargbegintheorem}
%    |\theoremstyle{plain}| corresponds to the original
%    definition, except that the distances to the surrounding text are
%    determined by the parameters |\theorempreskipamount| and
%    |\theorempostskipamount|.
%    First we set the default body font.
%    \begin{macrocode}
\gdef\th@plain{\normalfont\itshape
%    \end{macrocode}
%    Then we define |\@begintheorem| and
%    |\@opargbegintheorem|. These two macros define how the
%    header of a theorem is typeset.  |\@opargbegintheorem| will
%    be called if a \textsf{theorem} environment with an optional
%    argument is encountered; otherwise, the header is constructed by
%    calling |\@begintheorem|.  If one of these macros is
%    executed, we are within a \textsf{trivlist} environment started by
%    |\@thm|.  So the theorem header is produced with an
%    |\item| command.
%
%    Instead of specifying the header font directly, all standard
%    theorem styles use the |\theorem@headerfont| macro to allow
%    customization.  The extra space (|\labelsep|) is necessary
%    because of problems in the \textsf{trivlist} environment.
%    \begin{macrocode}
  \def\@begintheorem##1##2{%
        \item[\hskip\labelsep \theorem@headerfont ##1\ ##2]}%
%    \end{macrocode}
%    The definition of |\@opargbegintheorem| is completely
%    analogous. The only difference is the fact that there exists a
%    third argument (which is the optional parameter of the
%    environment and contains additional information about the
%    theorem).  Customarily we enclose it in parentheses.
%    \begin{macrocode}
\def\@opargbegintheorem##1##2##3{%
   \item[\hskip\labelsep \theorem@headerfont ##1\ ##2\ (##3)]}}
%    \end{macrocode}
% \end{macro}
% \end{macro}
% \end{macro}
%
% \noindent
% We conclude with an |\endgroup| to restore the
% |\catcode| of |@|.
%    \begin{macrocode}
\endgroup
%</thp>
%    \end{macrocode}
%
%
% \subsubsection{The \texttt{break} style}
%
% This style option is stored in the file |thb.sty|.
% For the next two lines see the documentation for |\th@plain| on
% page~\pageref{dok:th@plain}.
%    \begin{macrocode}
%<*thb>
\begingroup \makeatletter
\@ifundefined{theorem@style}{\input{theorem.sty}}{}
\ProvidesFile{thb.sty}
             [\FMithmInfo]
%    \end{macrocode}
%
% \begin{macro}{\th@break}
%    |\theoremstyle{break}| produces a line break after the name
%    of the theorem.  The font is |\slshape|.  Hence, we define
%    |\th@break| as follows:
%    \begin{macrocode}
\gdef\th@break{\normalfont\slshape
  \def\@begintheorem##1##2{\item[%
%    \end{macrocode}
%    We run into the following problem: it is not possible to create
%    the header with |\item[|\meta{title}|]| and then start
%    a new line by, for example, |\mbox{}\\|. Such a definition
%    will fail whenever a list environment follows immediately.  With
%    the above construction, the |\mbox{}| causes the switch
%    |@inlabel| (cf.\ definition of |\list| and
%    |\trivlist| in \cite{latex:source}) to be set to \texttt{false}
%    and so the following list will insert additional vertical
%    space (|\topskip|). This is quite annoying. Therefore, we
%    create the line break within the |\item|. In order to ensure
%    that the text will begin at the proper position in the following
%    line, we simply pretend that the label does not take any
%    room.\footnote{This will lead to problems whenever very high
%    symbols occurring in the line tower into the
%    heading. So, something else has to be done here sometime.}
%    \begin{macrocode}
       \rlap{\vbox{\hbox{\hskip \labelsep\theorem@headerfont ##1\ ##2}%
                   \hbox{\strut}}}]}%
%    \end{macrocode}
%    Again, the definition of |\@opargbegintheorem| is
%    completely analogous.
%    \begin{macrocode}
\def\@opargbegintheorem##1##2##3{%
  \item[\rlap{\vbox{\hbox{\hskip \labelsep \theorem@headerfont
                     ##1\ ##2\ (##3)}%
                    \hbox{\strut}}}]}}
\endgroup
%</thb>
%    \end{macrocode}
% \end{macro}
%
%
% \subsubsection{The \texttt{changebreak} style}
%
%    \begin{macrocode}
% This style option is stored in the file |thcb.sty|.
%    \begin{macrocode}
%<*thcb>
\begingroup \makeatletter
\@ifundefined{theorem@style}{\input{theorem.sty}}{}
\ProvidesFile{thcb.sty}
             [\FMithmInfo]
%    \end{macrocode}
%
% \begin{macro}{\th@changebreak}
%    The \texttt{change-break} theorem style is like \texttt{break}
%    but with interchange of theorem name and theorem number.  Thus we
%    define |\th@changebreak| as follows:
%    \begin{macrocode}
\gdef\th@changebreak{\normalfont\slshape
  \def\@begintheorem##1##2{\item
       [\rlap{\vbox{\hbox{\hskip\labelsep \theorem@headerfont ##2\ ##1}%
                    \hbox{\strut}}}]}%
\def\@opargbegintheorem##1##2##3{%
  \item[\rlap{\vbox{\hbox{\hskip\labelsep \theorem@headerfont
                     ##2\ ##1\ (##3)}%
                    \hbox{\strut}}}]}}
\endgroup
%</thcb>
%    \end{macrocode}
% \end{macro}
%
%
% \subsubsection{The \texttt{change} style}
%
% This style option is stored in the file |thc.sty|.
%    \begin{macrocode}
%<*thc>
\begingroup \makeatletter
\@ifundefined{theorem@style}{\input{theorem.sty}}{}
\ProvidesFile{thc.sty}
             [\FMithmInfo]
%    \end{macrocode}
%
% \begin{macro}{\th@change}
%    The \texttt{change} theorem style corresponds to the
%    \texttt{change break} style without a linebreak after the header.
%    To say it in another way, it's the same as the \texttt{plain}
%    style but with number and name interchanged and |\slshape| as the
%    default font.
%    \begin{macrocode}
\gdef\th@change{\normalfont\slshape
  \def\@begintheorem##1##2{\item
       [\hskip\labelsep \theorem@headerfont ##2\ ##1]}%
\def\@opargbegintheorem##1##2##3{%
  \item[\hskip\labelsep \theorem@headerfont ##2\ ##1\ (##3)]}}
\endgroup
%</thc>
%    \end{macrocode}
% \end{macro}
%
%
% \subsubsection{The \texttt{marginbreak} style}
%
% This style option is the one used most often at Mainz.  It is saved
% in the file |thmb.sty|.
%    \begin{macrocode}
%<*thmb>
\begingroup \makeatletter
\@ifundefined{theorem@style}{\input{theorem.sty}}{}
\ProvidesFile{thmb.sty}
             [\FMithmInfo]
%    \end{macrocode}
%
% \begin{macro}{\th@marginbreak}
%    The \texttt{margin break} style is nearly the same as the
%    \texttt{change break} style. The only difference is the placement
%    of the theorem number.  We use |\llap| to place it in the left
%    margin.
%
%    In this style |\labelsep| denotes the separation between the
%    number and the text.
%    \begin{macrocode}
\gdef\th@marginbreak{\normalfont\slshape
  \def\@begintheorem##1##2{\item
       [\rlap{\vbox{\theorem@headerfont
                    \hbox{\llap{##2}\hskip\labelsep ##1}%
                    \hbox{\strut}}}]}%
\def\@opargbegintheorem##1##2##3{%
  \item[\rlap{\vbox{\theorem@headerfont
                    \hbox{\llap{##2}\hskip\labelsep ##1\ (##3)}%
                    \hbox{\strut}}}]}}
\endgroup
%</thmb>
%    \end{macrocode}
% \changes{v2.1c}{90/02/17}{`endgroup outside macrocode env.}
% \end{macro}
%
%
%
% \subsubsection{The \texttt{margin} style}
%
% This style option is stored in the file |thm.sty|.
%    \begin{macrocode}
%<*thm>
\begingroup \makeatletter
\@ifundefined{theorem@style}{\input{theorem.sty}}{}
\ProvidesFile{thm.sty}
             [\FMithmInfo]
%    \end{macrocode}
%
% \begin{macro}{\th@margin}
%    Again this is only a variant of the theorem styles described
%    above without any new ideas.
%    \begin{macrocode}
\gdef\th@margin{\normalfont\slshape
  \def\@begintheorem##1##2{\item
       [\theorem@headerfont \llap{##2}\hskip\labelsep ##1]}%
\def\@opargbegintheorem##1##2##3{%
  \item[\theorem@headerfont \llap{##2}\hskip\labelsep ##1\ (##3)]}}
\endgroup
%</thm>
%    \end{macrocode}
% \end{macro}
%
%
%
%
% \subsection{Final Definitions}
%
% \begin{macro}{\theorempreskipamount}
% \begin{macro}{\theorempostskipamount}
%    The \textsf{skip} parameters that regulate the vertical empty space
%    before and after the {\envfont theorem} environment have
%    to be allocated as well.
% \changes{v2.0f}{89/07/20}{Unnecessary `global' removed.}
%    \begin{macrocode}
%<*package>
\newskip\theorempreskipamount
\newskip\theorempostskipamount
%    \end{macrocode}
%    Since we have used the same values for all theorem sets, we now can
%    assign them.
%    \begin{macrocode}
\global\setlength\theorempreskipamount{12pt plus 5pt minus 3pt}
\global\setlength\theorempostskipamount{8pt plus 3pt minus 1.5pt}
%    \end{macrocode}
% \end{macro}
% \end{macro}
%
%
% \begin{macro}{\@endtheorem}
%    The same holds for the macro |\@endtheorem|, which ends a
%    {\envfont theorem} environment. Since it is the same for all
%    theorem sets, it is removed from the macros |\th@|\meta{style}.
%    It simply ends the {\envfont trivlist} environment, which was
%    begun in |\@thm|.
%    \begin{macrocode}
\global\let\@endtheorem=\endtrivlist
%    \end{macrocode}
% \end{macro}
%
%
%
% \begin{macro}{\@preamblecmds}
%    All macros defined above are to be used only in the preamble.
%    Therefore, we insert them in |\@preamblecmds| which will disable
%    them at begin document. This is done by the internal \LaTeXe{}
%    command |\@onlypreamble|.
% \changes{v2.0g}{89/09/19}{`theoremheaderfont and `theorembodyfont
%    added.}
% \changes{v2.2a}{94/02/03}{Use \LaTeXe{} interface}
%    \begin{macrocode}
\@onlypreamble\@xnthm
\@onlypreamble\@ynthm
\@onlypreamble\@othm
\@onlypreamble\newtheorem
\@onlypreamble\theoremstyle
\@onlypreamble\theorembodyfont
\@onlypreamble\theoremheaderfont
%    \end{macrocode}
% \end{macro}
%
% Finally we declare the \texttt{plain} theorem style to be the default.
% \changes{v2.1b}{90/06/04}{Fixed missing `item by declaring default
%    style.}
%    \begin{macrocode}
\theoremstyle{plain}
%</package>
%    \end{macrocode}
%
% \Finale

\endinput
}{}
\ProvidesFile{thc.sty}
             [\FMithmInfo]
%    \end{macrocode}
%
% \begin{macro}{\th@change}
%    The \texttt{change} theorem style corresponds to the
%    \texttt{change break} style without a linebreak after the header.
%    To say it in another way, it's the same as the \texttt{plain}
%    style but with number and name interchanged and |\slshape| as the
%    default font.
%    \begin{macrocode}
\gdef\th@change{\normalfont\slshape
  \def\@begintheorem##1##2{\item
       [\hskip\labelsep \theorem@headerfont ##2\ ##1]}%
\def\@opargbegintheorem##1##2##3{%
  \item[\hskip\labelsep \theorem@headerfont ##2\ ##1\ (##3)]}}
\endgroup
%</thc>
%    \end{macrocode}
% \end{macro}
%
%
% \subsubsection{The \texttt{marginbreak} style}
%
% This style option is the one used most often at Mainz.  It is saved
% in the file |thmb.sty|.
%    \begin{macrocode}
%<*thmb>
\begingroup \makeatletter
\@ifundefined{theorem@style}{% \iffalse meta-comment
%
% Copyright (C) 1993-2024
%
% The LaTeX Project and any individual authors listed elsewhere
% in this file.
%
% This file is part of the Standard LaTeX `Tools Bundle'.
% -------------------------------------------------------
%
% It may be distributed and/or modified under the
% conditions of the LaTeX Project Public License, either version 1.3c
% of this license or (at your option) any later version.
% The latest version of this license is in
%    https://www.latex-project.org/lppl.txt
% and version 1.3c or later is part of all distributions of LaTeX
% version 2005/12/01 or later.
%
% The list of all files belonging to the LaTeX `Tools Bundle' is
% given in the file `manifest.txt'.
%
% \fi
%
%%
%
%
\def\FMithmInfo{2023/07/05 v2.2c Theorem extension package (FMi)}
%
% \ProvidesFile{theorem.dtx}[\FMithmInfo]
%\iffalse   % this is a METACOMMENT !
%%
%% Package `theorem' to use with LaTeX2e
%% Copyright (C) 1989-1995 Frank Mittelbach, all rights reserved.
%<+package>\@ifundefined{theorem@style}{}{\endinput}
%<+package>\NeedsTeXFormat{LaTeX2e}[1995/06/01]
%<+package>\ProvidesPackage{theorem}
%<+package>         [\FMithmInfo]
%
% \fi
%
% \changes{v1.0}{87/05/07}{First draft.}
% \changes{v1.0a}{87/05/07}{`thistheoremstyle' added.}
% \changes{v1.1a}{87/07/29}{New concept: redefine the internal
%                          \protect\LaTeX{} commands `@xnthm' etc.}
% \changes{v1.1b}{87/07/31}{`theoremfont' used to specify font.}
% \changes{v1.2a}{88/01/08}{`@endtheorem' now globally defined.}
% \changes{v1.2b}{88/01/11}{`newtheorem' only allowed in the preamble.}
% \changes{v1.3a}{88/02/14}{`theoremfont' now a toks register.}
%
% \changes{v2.0a}{88/11/16}{New implementation and new documentation.}
% \changes{v2.0a}{88/11/16}{No layout changes but internal modifications
%                           old substyles won't work.}
% \changes{v2.0b}{89/01/30}{`global' used for all definitions.}
% \changes{v2.0c}{89/06/15}{Documentation translated to English.
%                            Thanks to LLL and Nina}
% \changes{v2.1a}{90/02/15}{Support for new font selection.}
% \changes{v2.2a}{94/02/03}{Cleanup for LaTeX2e}
%
% \DoNotIndex{\@endparenv,\@ifdefinable,\@ifnextchar,\@ifundefined}
% \DoNotIndex{\@namedef,\@nameuse,\@spaces,\@tempa,\@warning}
% \DoNotIndex{\ }
% \DoNotIndex{\begingroup,\bfseries}
% \DoNotIndex{\csname}
% \DoNotIndex{\def,\do,\docdate}
% \DoNotIndex{\endcsname,\endgroup,\endinput,\endtrivlist,\expandafter}
% \DoNotIndex{\filedate,\fileversion}
% \DoNotIndex{\gdef,\global}
% \DoNotIndex{\hbox,\hskip}
% \DoNotIndex{\ignorespaces,\input,\itshape,\item}
% \DoNotIndex{\labelsep,\let,\llap}
% \DoNotIndex{\makeatletter}
% \DoNotIndex{\noexpand}
% \DoNotIndex{\refstepcounter,\relax,\rlap}
% \DoNotIndex{\setlength,\slshape,\space,\string,\strut}
% \DoNotIndex{\the,\trivlist,\typeout}
% \DoNotIndex{\vbox}
% \DoNotIndex{\xdef}
%
% \def\deflabel#1{\ttfamily #1\hfill}
% \def\deflist#1{\begin{list}{}{\settowidth\labelwidth{\ttfamily #1}%
%                               \setlength\leftmargin\labelwidth
%                               \addtolength\leftmargin\labelsep
%                               \let\makelabel\deflabel}}
% \def\enddeflist{\end{list}}
% \def\envfont{\normalfont\ttfamily}
%
% \MakeShortVerb{\|}
%
% \newcommand{\TUB}{{\sl TUGboat\/}}
%
% \GetFileInfo{theorem.dtx}
% \title{An Extension of the
%        \LaTeX{} theorem environment\thanks{This file
%        has version number \fileversion, last
%        revised \filedate.}}
% \author{Frank Mittelbach}
%
% \MaintainedByLaTeXTeam{tools}
% \maketitle
%
% \begin{abstract}
%    The macros described in this paper yield an extension of the
%    \LaTeX{} theorem mechanism. It is designed to satisfy the
%    different requirements of various journals. Thus, the
%    layout of the ``theorems'' can be manipulated by determining a
%    ``style''. This article describes not only the use, but
%    also the definition, of the necessary macros.
% \end{abstract}
%
% \section*{Preface to version 2.2}
%
% For \LaTeXe{} this package did not need any fundamental changes. I
% only modified the messages generated so that theorem layout styles
% will show up with the |\listfiles| command and cleaned the section
% on the New Font Selection Scheme since this is now included in
% \LaTeX.
%
% \section*{Preface to version 2.1}
%
% This version is identical to 2.0g described in \TUB\ 10\#3 except
% for some internal defaults which are now set depending on the used
% font selection scheme.
%
% This was done to avoid unpleasant surprises if the new font selection
% scheme is in force. For further details see
% section~\ref{sec:newfonts} and \cite{bk:GMS94}.
%
% \section{Introduction}
%
% For our purposes here, ``theorems'' are labelled enunciations,
% often set off from the main text by extra space and a font change.
% Theorems, corollaries, conjectures, definitions, and remarks
% are all instances of ``theorems''.  The ``header'' of these structures
% is composed of a label (such as \textsc{Theorem} or \textsc{Remark})
% and a number which serializes an item in the sequence of items
% with the same label.
%
% Shortly after the introduction of \LaTeX{} at the Fachbereich
% Mathematik in Mainz, the desire to manipulate the layout of
% ``theorems'' arose.  In Mainz, the following two conventions came
% into general use:
% \begin{enumerate}
%   \item The number of the theorem is shown in the margin.
%   \item There is a line break at
%         the end of the theorem header.
% \end{enumerate}
%
% Additionally, some journals require different formats which depend on
% the ``sort of theorem'': e.g.\ often remarks and definitions are set
% in |\upshape|, while |\itshape| is employed for main theorems.
%
% Confronted with these requirements, a theorem environment was
% developed in Mainz which allows separate determination of the
% layout of the ``theorems sets'', comparable to |\pagestyle|.
%
%
% \section{The user interface}
% \subsection{Defining new theorem sets}
%
% \DescribeMacro\newtheorem
% As in the original \LaTeX{} version, the command |\newtheorem|
% defines a new ``theorem set'' or ``theorem-like structure''.
% Two required arguments name the new environment and give the text
% to be typeset with each instance of the new ``set'', while
% an optional argument determines how
% the ``set'' is enumerated:
% \begin{description}
%    \item[\ttfamily \bslash newtheorem\{foo\}\{bar\}]
%       The theorem set {\envfont foo} (whose name is \texttt{bar})
%       uses its own counter.
%    \item[\ttfamily \bslash newtheorem\{foo2\}{[foo]}\{bar2\}]
%       The theorem set {\envfont foo2} (printed name \texttt{bar2})
%       uses the same counter as the theorem set \texttt{foo}.
%    \item[\ttfamily \bslash newtheorem\{foo3\}\{bar3\}{[section]}]
%       The theorem set {\envfont foo3} (printed name \texttt{bar3}) is
%       enumerated within the counter \texttt{section}, i.e.\ with every
%       new |\section| the enumeration begins again with 1, and
%       the enumeration is composed from the section-number and the
%       theorem counter itself.
% \end{description}
%
% \DescribeMacro\theoremstyle
% Additionally, the command |\theoremstyle| can define the layout
% of various, or all, theorem sets.  It should be noted that any
% theorem set defined by |\newtheorem| is typeset in the
% |\theoremstyle| that is current at the time of the definition.
% Thus, the following
% \begin{verbatim}
%    \theoremstyle{break}        \newtheorem{Cor}{Corollary}
%    \theoremstyle{plain}        \newtheorem{Exa}{Example}[section]
%\end{verbatim}                             ^^A No blank at line start
% leads to the result that the set {\envfont Cor} is formatted in the
% style \texttt{break}, while the set {\envfont Exa} and all the
% following ones are formatted in the style \texttt{plain}, unless
% another |\theoremstyle| follows.  Since the definitions installed by
% |\newtheorem| are global, one also can limit |\theoremstyle| locally
% by grouping braces.
%
% \DescribeMacro\theorembodyfont
% The choice of the font for the theorem body is completely
% independent of the chosen |\theoremstyle|; this has proven to
% be very advantageous.  For example,
% \begin{verbatim}
%   {\theorembodyfont{\upshape}          \newtheorem{Rem}{Remark}}
%\end{verbatim}
% defines a theorem set {\envfont Rem}, which will be set in
% |\upshape| in the current layout (which in our example is
% \texttt{plain}).  As with |\theoremstyle|, the |\theorembodyfont|
% chosen is that current at the time of |\newtheorem|.  If
% |\theorembodyfont| is not specified or one defines
% |\theorembodyfont{}|, then the font used will be that defined by the
% |\theoremstyle|.
%
% \DescribeMacro\theoremheaderfont It is also possible to customize
% the font used for the theorem headers.  This is, however, a global
% declaration, and therefore there should be at most one
% |\theoremheaderfont| declaration in the preamble.\footnote{If it is
% actually necessary to have different header fonts, one has to define
% new theorem styles (substituting the desired font) or specify the
% information directly in the \texttt{\bslash newtheorem} declaration
% (the unclean variant).}
%
% \DescribeMacro\theorempreskipamount
% \DescribeMacro\theorempostskipamount Two additional parameters
% affect the vertical space around the theorem environments:
% |\theorempreskipamount| and |\theorempostskipamount| define,
% respectively, the spacing before and after such an environment.
% These parameters apply for all theorem sets and can be manipulated
% with the ordinary length macros.  They are rubber lengths
% (`\textsf{skips}'), and therefore can contain \texttt{plus} and
% \texttt{minus} parts.
%
% Since the definition of theorem sets should---most sensibly---be
% placed in the preamble, we only allow installation there.
% It is therefore possible to release the memory used here after
% |\begin{document}|, in order to make room for other
% applications.
%
%
% \subsection{Existing theorem styles}
%
% The following theorem styles exist to date:
% \begin{deflist}{marginbreak}
%    \item[plain]
%       This theorem style emulates the original \LaTeX{} definition,
%       except that additionally the parameters
%       |\theorem...skipamount| are used.
%    \item[break]
%       In this style, the theorem header is followed by a line
%       break.
%    \item[marginbreak]
%       The theorem number is set in the margin, and there is a line
%       break as in \texttt{break}.
%    \item[changebreak]
%       Like \texttt{break}, but with header number and text
%       interchanged.
%    \item[change]
%       Header number and text are interchanged, without a line break.
%    \item[margin]
%       The number is set in the left margin, without a line break.
% \end{deflist}
% All styles (except \texttt{plain}) select |\slshape| as the default
% |\theorembodyfont|.
%
%
% \subsection{Examples}
%
% Given the above theorem sets {\envfont Cor}, {\envfont Exa}
% and {\envfont Rem},
% suppose that the preamble also contains the declarations:
% \begin{verbatim}
%    \theoremstyle{marginbreak}   \newtheorem{Lem}[Cor]{Lemma}
%    \theoremstyle{change}
%    \theorembodyfont{\itshape}        \newtheorem{Def}[Cor]{Definition}
%
%    \theoremheaderfont{\scshape}
%\end{verbatim}
% Then the following are some typical examples of the typeset output
% resulting from their use.
%
% \begin{Cor}
%    This is a sentence typeset in the theorem environment {\envfont
%    Cor}.
% \end{Cor}
%
% \begin{Exa}
%    This is a sentence typeset in the theorem environment {\envfont
%    Exa}.
% \end{Exa}
%
% \begin{Rem}
%    This is a sentence typeset in the theorem environment {\envfont
%    Rem}.
% \end{Rem}
%
% \begin{Lem}[Ben User]
%    This is a sentence typeset in the theorem environment {\envfont
%    Lem}.
% \end{Lem}
%
% \begin{Def}[Very impressive Definition]
%    This is a sentence typeset in the theorem environment {\envfont
%    Def}.
% \end{Def}
%
% The last two examples show the effect of the optional argument to
% a theorem environment (it is the text typeset in parentheses).
%
%
% \section{Special Considerations}
% \label{sec:newfonts}
%
% Theoremheader and body are implemented as a unit. This means that
% the |\theoremheaderfont|  will inherit characteristics of the
% |\theorembodyfont| in \LaTeXe. Thus, if for example
% |\theorembodyfont| is |\itshape| and |\theoremheaderfont|
% is |\bfseries| the font selected for the header will have the
% characteristics `bold extended italic'. If this is not desired one
% should set the |\theoremheaderfont| to something like
% \begin{verbatim}
%      \theoremheaderfont{\normalfont\bfseries}
%\end{verbatim}
% i.e.\ supplying all necessary font information explicitly.
%
%
%
% \section{Acknowledgements}
%
% The publication of this set of macros was only possible with the
% help of Christina Busse (translating the manuscript into English),
% Joachim Pense (playing the r\^ole of typist), Chris Rowley
% (looking everything over) and many others providing useful
% suggestions.
%
% \MaybeStop{%
%    \begin{thebibliography}{9}
%       \bibitem{bk:GMS94} \textsc{M.~Goossens}, \textsc{F.~Mittelbach}
%         and \textsc{A.~Samarin}.
%         \newblock The \LaTeX{} Companion.
%         \newblock
%         Addison-Wesley, Reading, Massachusetts, 1994.
%       \bibitem{latex:source} \textsc{Lamport, Leslie.} \newblock
%                               \texttt{latex.tex}, version 2.09, date
%                               Feb. 1990.
%    \end{thebibliography}
%    \PrintIndex
%    \PrintChanges}
%
%
% \section{The documentation driver file}
%
% The next bit of code contains the documentation driver file for
% \TeX{}, i.e., the file that will produce the documentation you are
% currently reading. It will be extracted from this file by the
% \texttt{docstrip} program. Since it is the first code in the file
% one can alternatively process this file directly with \LaTeXe{} to
% obtain the documentation.
%
%    \begin{macrocode}
%<*driver>
\documentclass{ltxdoc}

\usepackage{theorem}

%<+driver>% The next few lines define theorem sets which are used
%<+driver>% in the example section of the documentation.

\theoremstyle{break}       \newtheorem{Cor}{Corollary}
\theoremstyle{plain}       \newtheorem{Exa}{Example}[section]
{\theorembodyfont{\upshape}\newtheorem{Rem}{Remark}}
\theoremstyle{marginbreak} \newtheorem{Lem}[Cor]{Lemma}
\theoremstyle{change}
\theorembodyfont{\itshape} \newtheorem{Def}[Cor]{Definition}

\theoremheaderfont{\scshape}

\RecordChanges

\begin{document}
   \DocInput{theorem.dtx}
\end{document}
%</driver>
%    \end{macrocode}
%
% \section{Definition of the Macros}
%
% If the file has been loaded before, we abort immediately. If not the
% package announces itself (this is actually done at the very top of
% the file---the way it is done isn't good style so don't copy it).
% \changes{v2.0e}{89/07/19}{Spaces removed from `typeout'.}
% \changes{v2.2b}{95/11/19}{Announce moved to top of file}
%    \begin{macrocode}
%<*package>
%\@ifundefined{theorem@style}{}{\endinput}
%\def\FMithmInfo{1995/11/19 v2.2b Theorem extension package (FMi)}
%\ProvidesPackage{theorem}[\FMithmInfo]
%    \end{macrocode}
%
%
% \subsection{Definition of theorem styles and fonts}
%
% All the definitions in this file are done globally to allow
% inputting this file inside a group.
%
% \begin{macro}{\theoremstyle}
%    Before a theorem style can be installed, the chosen style must be
%    known. For that reason, we must test to see that
%    |\th@|\meta{style} is known or, more precisely, that it is
%    different from |\relax|. If the style is not known then
%    |\th@plain| is used.
%    \begin{macrocode}
\gdef\theoremstyle#1{%
   \@ifundefined{th@#1}{\@warning
          {Unknown theoremstyle `#1'. Using `plain'}%
          \theorem@style{plain}}%
%    \end{macrocode}
%    We save the theorem style to be used in the token register
%    |\theorem@style|.
%    \begin{macrocode}
      {\theorem@style{#1}}%
%    \end{macrocode}
%    Now we ``evaluate'' the theorem style: this means, we call the
%    macro |\th@|\meta{style} which will activate the relevant
%    definitions which are contained in a separate file.  This is done
%    in a group to suppress changes to the current font.  This could
%    otherwise pose problems together with the new font selection
%    scheme\footnote{When I printed the original article using the new
%    font selection scheme I ended with a document with slanted
%    typefaces (text headings and all) simply because one of the
%    theorem styles used \texttt{\bslash sl} at toplevel.} if the
%    |\th@|\meta{style} is evaluated a second time.
% \changes{v2.1a}{90/02/15}{Group added around `th@..}
% \changes{v2.1a}{90/02/15}{Reseting of `..@begintheorem removed.}
%    \begin{macrocode}
      \begingroup
        \csname th@\the\theorem@style \endcsname
      \endgroup}
%    \end{macrocode}
% \end{macro}
%
%
% \begin{macro}{\@begintheorem}
% \changes{v2.1a}{90/02/15}{Relaxed at toplevel.}
% \begin{macro}{\@opargbegintheorem}
% \changes{v2.1a}{90/02/15}{Relaxed at toplevel.}
%    We reset |\@begintheorem| and |\@opargbegintheorem| to
%    |\relax| since these commands are no longer necessary at
%    toplevel. This will save a few tokens.
%    \begin{macrocode}
\global\let\@begintheorem\relax
\global\let\@opargbegintheorem\relax
%    \end{macrocode}
% \end{macro}
% \end{macro}
%
% \begin{macro}{\theorem@style}
%    Obviously the \textsf{token} register used above has to be
%    allocated. To assure the utmost compatibility with the
%    original \LaTeX{} definition, we set the default theorem style to
%    \texttt{plain}, which implements the usual \LaTeX{} convention.
% \changes{v2.0f}{89/07/20}{Unnecessary `global' removed.}
%    \begin{macrocode}
\newtoks\theorem@style
\global\theorem@style{plain}
%    \end{macrocode}
% \end{macro}
%
% \begin{macro}{\theorembodyfont}
% \begin{macro}{\theorem@bodyfont}
%    If the |\theorembodyfont| is set by the user then it should not
%    interact with the default font set in the theorem style. When the
%    new font selection is in force this may happen if, for example,
%    the default is |\itshape| and the new |\theorembodyfont| is
%    |\sffamily|. So we add a |\reset@font| command in front of the
%    user definition.
% \changes{v2.0f}{89/07/20}{Unnecessary `global' removed.}
% \changes{v2.0g}{89/09/19}{`theoremfont to `theorembodyfont.}
% \changes{v2.1c}{92/06/22}{`reset@font added and logic changed.}
% \changes{v2.1d}{92/10/28}{Reset to empty if arg is empty}
%    \begin{macrocode}
\gdef\theorembodyfont#1{%
%    \end{macrocode}
%    We check if the argument supplied is empty and if so put nothing
%    into the |\theorem@bodyfont| token register to allow for
%    |\theorembodyfont{}| as a mean of using the default of the
%    current |\theoremstyle|.
%    \begin{macrocode}
   \def\@tempa{#1}%
   \ifx\@tempa\@empty
    \theorem@bodyfont{}%
   \else
    \theorem@bodyfont{\reset@font#1}%
   \fi
   }
\newtoks\theorem@bodyfont
\global\theorem@bodyfont{}
%    \end{macrocode}
% \end{macro}
% \end{macro}
%
%
% \begin{macro}{\theoremheaderfont}
% \changes{v2.0g}{89/09/19}{Macro added.}
%    The font for the theorem headers is handled differently because
%    this definition applies to all theorem styles.
%    \begin{macrocode}
\gdef\theoremheaderfont#1{\gdef\theorem@headerfont{#1}%
%    \end{macrocode}
%    After using the macro once it is redefined to produce an error
%    message.
%    \begin{macrocode}
       \gdef\theoremheaderfont##1{%
        \typeout{\string\theoremheaderfont\space should be used
                 only once.}}}
%    \end{macrocode}
% \end{macro}
%
% \begin{macro}{\theorem@headerfont}
% \changes{v2.0g}{89/09/19}{Macro added.}
%    To set the |\theorem@headerfont| default we first test if
%    the new fontselection scheme is in force.
% \changes{v2.1a}{90/02/15}{Support for new font selection.}
%    \begin{macrocode}
\ifx\upshape\undefined
%    \end{macrocode}
%    If not we define it to expand into |\bfseries|. We don't use
%    |\let| just in case a following style option redefines this
%    macro.
%    \begin{macrocode}
\gdef\theorem@headerfont{\bfseries}
%    \end{macrocode}
%    Otherwise we reset the current shape before calling |\bfseries|.
%    \begin{macrocode}
\else \gdef\theorem@headerfont{\normalfont\bfseries}\fi
%    \end{macrocode}
% \end{macro}
%
% \begin{macro}{\th@plain}
% \begin{macro}{\th@break}
% \begin{macro}{\th@marginbreak}
% \begin{macro}{\th@changebreak}
% \begin{macro}{\th@change}
% \begin{macro}{\th@margin}
%    The different styles are defined in macros such as
%    |\th@plain|.  Since memory space is precious in
%    ``non-Big-versions'', we have to avoid offering too many unused
%    definitions.  Therefore we define these styles in separate files
%    that can be loaded on demand.  Thus the commands themselves only
%    load these files. We use |\@input@|, a \LaTeXe{} internal command
%    that ensures that the file will be listed with |\listfiles|.
% \changes{v2.2a}{94/02/02}{Use `@input@ to load theorem layout files}
%    \begin{macrocode}
\gdef\th@plain{\@input@{thp.sty}}
\gdef\th@break{\@input@{thb.sty}}
\gdef\th@marginbreak{\@input@{thmb.sty}}
\gdef\th@changebreak{\@input@{thcb.sty}}
\gdef\th@change{\@input@{thc.sty}}
\gdef\th@margin{\@input@{thm.sty}}
%    \end{macrocode}
%    This list will be expanded when new styles become
%    available.  For testing,  just append new theorem substyles
%    as document options.
% \end{macro}
% \end{macro}
% \end{macro}
% \end{macro}
% \end{macro}
% \end{macro}
%
%
%
% \subsection{Definition of a new theorem set}
%
% As already pointed out, a new theorem environment can be defined in
% three different ways:
% \begin{verbatim}
%    \newtheorem{Lem}{Lemma}
%    \newtheorem{Lem}{Lemma}[section]
%    \newtheorem{Lem}[Theorem]{Lemma}
%\end{verbatim}
% The function of the macro |\newtheorem| is to recognize these cases
% and then to branch into one of the three macros |\@ynthm|, |\@xnthm|
% or |\@othm|.  This mechanism is adopted unchanged from
% \cite{latex:source}; the essential point here is that, for example,
% in the second case, the arguments \texttt{Lem}, \texttt{Lemma} and
% \texttt{section} are passed over to the macro |\@xnthm|.
%
% We inspect this case first because the others present fewer problems,
% and thus are easily derived from this one.
%
% \begin{macro}{\@xnthm}
%    For our example arguments, the macro |\@xnthm| must fulfill
%    the following:
%    \begin{itemize}
%       \item
%          Define a new \LaTeX-counter `Lem'
%       \item
%          reset this counter within a |\section|
%       \item
%          define the macro |\theLem|
%       \item
%          define the environment macros |\Lem| and
%          |\endLem| using the current
%          |\theoremstyle| and |\theorem@bodyfont|.
%    \end{itemize}
%    Obviously, all this should happen only if the first argument
%    of |\@xnthm| (i.e.\ \texttt{Lem} in our example) is chosen so
%    as not to conflict with any previously defined commands or
%    environments.  This test is performed by the \LaTeX{} macro
%    |\@ifdefinable|.
%    \begin{macrocode}
\gdef\@xnthm#1#2[#3]{\expandafter\@ifdefinable\csname #1\endcsname
%    \end{macrocode}
%    Therefore, the first argument of |\@ifdefinable| is the
%    expansion (in the example, |\Lem|) of
%    |\csname||#1\endcsname|.  The second argument is
%    executed only if the test has been completed successfully.
%    \begin{macrocode}
   {%
%    \end{macrocode}
%    Now we define the new counter. The names of the \LaTeX{}
%    macros employed should speak for themselves:
% \changes{v2.2b}{1995/10/07}{Use \cs{@newctr}}
% \changes{v2.2c}{1995/11/23}{Use \cs{@newctr} properly.}
%    \begin{macrocode}
    \@definecounter{#1}\@newctr{#1}[#3]%
%    \end{macrocode}
%    Using |\@newctr| will give a proper error message if the counter
%    in |#3| is not defined.
%    In defining `|\theLem|' we must generate the desired macro
%    name by use of |\expandafter| and |\csname|.
%    \begin{macrocode}
    \expandafter\xdef\csname the#1\endcsname
%    \end{macrocode}
%    An |\xdef| is used in order to make the definition global,
%    and to ensure that it contains the replacement texts of
%    |\@thmcountersep| and |\@thmcounter|.\footnote{These
%    two macros can be defined by the document style.  Their default
%    values produce a `\texttt{.}' as separation and an arabic
%    representation of the number.} However, not everything should be
%    expanded. For example, it saves space to use |\thesection|
%    instead of its---at times---lengthy expansion.
%    \begin{macrocode}
      {\expandafter \noexpand \csname the#3\endcsname
       \@thmcountersep \@thmcounter{#1}}%
%    \end{macrocode}
%    Thus with the defaults of \LaTeX{}, |\theLem| would be
%    replaced by the command sequence |\thesection.\arabic{Lem}|.
%
%    We will now look at the definition of the macro which is executed
%    at the beginning of the actual environment (in our example this
%    macro is |\Lem|). It should be noted that we use an
%    ``|\expandafter| trick'' to expand only certain parts of the
%    replacement text at the time of the definition.
%    \begin{macrocode}
    \def\@tempa{\global\@namedef{#1}}%
    \expandafter \@tempa \expandafter{%
%    \end{macrocode}
%    First, the macro that contains the current definitions of
%    |\@begintheorem| and |\@opargtheorem| should be called
%    up.  The name of this macro---as is already known---has the form
%    |\th@|\meta{theorem\ style}; therefore, it must be called by
%    \begin{macrocode}
      \csname th@\the \theorem@style
%    \end{macrocode}
%    In addition the default theorem font should be changeable, i.e.\
%    we have to insert the contents of |\theorem@bodyfont|. For
%    that reason, we expand even further, beyond |\endcsname|,
%    and thus insert the contents of the \textsf{token} register
%    |\theorem@bodyfont| in the replacement text.
%    \begin{macrocode}
            \expandafter \endcsname \the \theorem@bodyfont
%    \end{macrocode}
%    Now it is time to call the macro |\@thm| which takes over
%    the further processing.  It has two arguments: the current
%    counter name (in our example, \texttt{Lem}), and the text of the
%    label (in our example, \texttt{Lemma}).
%    \begin{macrocode}
     \@thm{#1}{#2}}%
%    \end{macrocode}
%    With this, the `sub-definition' is complete.  The macro
%    |\@endtheorem| ends a theorem environment and is, so far, nothing
%    but an |\endtrivlist|.  (Hence it is defined globally, and not
%    within the theorem styles.\footnote{This has to be changed as
%    soon as theorem styles that change \texttt{\bslash @endtheorem}
%    exist.  In such a case, all existing styles must be changed as
%    well since they will have to reset the macro.}) Therefore, we can
%    set it equivalent to the macro that ends the theorem set (in our
%    example, |\endLem|).  However, if some day theorem styles exist
%    that do change |\@endtheorem|, we would have to use the
%    commented-out line instead.
%    \begin{macrocode}
    \global \expandafter \let \csname end#1\endcsname \@endtheorem
%   \global\@namedef{end#1}{\@endtheorem}%
%    \end{macrocode}
%    With these commands all the required definitions are employed,
%    unless the test |\@ifdefinable| has failed.  Therefore, we
%    end the second argument of this macro and with it the definition
%    of |\@xnthm|.
%    \begin{macrocode}
   }}
%    \end{macrocode}
% \end{macro}
%
% \begin{macro}{\@ynthm}
%    The definition of |\@ynthm| is completely analogous. In
%    this case the new counter that is defined is not reset within
%    another counter; thus the definition of |\the...|
%    is simplified:
%    \begin{macrocode}
\gdef\@ynthm#1#2{\expandafter\@ifdefinable\csname #1\endcsname
   {\@definecounter{#1}%
    \expandafter\xdef\csname the#1\endcsname{\@thmcounter{#1}}%
%    \end{macrocode}
%    The rest of the definition corresponds literally to that of
%    |\@xnthm|:
%    \begin{macrocode}
    \def\@tempa{\global\@namedef{#1}}\expandafter \@tempa
     \expandafter{\csname th@\the \theorem@style \expandafter
     \endcsname \the\theorem@bodyfont \@thm{#1}{#2}}%
    \global \expandafter \let \csname end#1\endcsname \@endtheorem}}
%    \end{macrocode}
% \end{macro}
%
% \begin{macro}{\@othm}
%    The definition of |\@othm| does not contain anything new.
%    \begin{macrocode}
\gdef\@othm#1[#2]#3{%
%    \end{macrocode}
% \changes{v2.2b}{95/10/07}{Build in check for counter defined}
%    We do not define a new counter but instead use one that has
%    already been defined.  Thus the only definition we need is that
%    of this pseudo-counter (i.e.\ |\the|\meta{env.\ name}).
%    First we check if |#2| corresponds to a known counter name.
%    \begin{macrocode}
  \expandafter\ifx\csname c@#2\endcsname\relax
   \@nocounterr{#2}%
  \else
   \expandafter\@ifdefinable\csname #1\endcsname
   {\expandafter \xdef \csname the#1\endcsname
     {\expandafter \noexpand \csname the#2\endcsname}%
%    \end{macrocode}
%    All other parts of the definition can be adopted from
%    |\@xnthm|. We have to remember, though, that in this case
%    the name of the current counter and the theorem label have moved
%    to the second and third arguments.
%    \begin{macrocode}
    \def\@tempa{\global\@namedef{#1}}\expandafter \@tempa
     \expandafter{\csname th@\the \theorem@style \expandafter
     \endcsname \the\theorem@bodyfont \@thm{#2}{#3}}%
    \global \expandafter \let \csname end#1\endcsname \@endtheorem}%
  \fi}
%    \end{macrocode}
% \end{macro}
%
% \subsection{Macros that are employed in a {\envfont theorem}
%    environment}
%
% \begin{macro}{\@thm}
%    The macro |\@thm| has to increase the current counter.  Then,
%    depending on whether the environment has (or does not have) an
%    optional argument, it has to branch into either |\@begintheorem|
%    or |\@opargtheorem|.
%    \begin{macrocode}
\gdef\@thm#1#2{\refstepcounter{#1}%
%    \end{macrocode}
%    Now we start a {\envfont trivlist} environment, and give
%    |\@topsep| and |\@topsepadd| the values of the \textsf{skip}
%    registers |\theorempreskipamount| and |\theorempostskipamount|.
%    The value in |\@topsep| is the vertical space that is inserted by
%    the first (and only) |\item| in our |\trivlist| whilst
%    |\@topsepadd| is inserted by |\@endparenv| at the end of that
%    {\envfont trivlist} environment.  By using these registers, we
%    obtain the desired space around a {\envfont theorem} environment.
%    \begin{macrocode}
   \trivlist
   \@topsep \theorempreskipamount               % used by first \item
   \@topsepadd \theorempostskipamount           % used by \@endparenv
%    \end{macrocode}
%    Now we have to test whether an optional argument has been given.
% \changes{v2.0d}{89/07/18}{Forgotten percent added.}
%    \begin{macrocode}
   \@ifnextchar [%
%    \end{macrocode}
%    If there is an optional argument, we will call |\@ythm|, and
%    move the arguments read back into the input stream.
%    \begin{macrocode}
   {\@ythm{#1}{#2}}%
%    \end{macrocode}
%    If not, we call |\@begintheorem|. Its first argument is the
%    name of the theorem set (hence the second argument of
%    |\@thm|). Its second argument is the macro that produces the
%    current number.
%    \begin{macrocode}
   {\@begintheorem{#2}{\csname the#1\endcsname}\ignorespaces}}
%    \end{macrocode}
% \end{macro}
%
% \begin{macro}{\@xthm}
% \begin{macro}{\@ythm}
%    Both these macros were originally called by |\@thm|.  We do
%    not need |\@xthm| anymore, hence we reset it to
%    |\relax|. The definition of |\@ythm| has not changed at
%    all from its definition in \LaTeX. In order to make the macros
%    easier to understand, we will
%    nevertheless present it (commented out).
%    \begin{macrocode}
\global\let\@xthm\relax
% \def\@ythm#1#2[#3]{\@opargbegintheorem{#2}{\csname
%           the#1\endcsname}{#3}\ignorespaces}
%</package>
%    \end{macrocode}
%    The primitive |\ignorespaces| in |\@ythm| and
%    |\@thm| is needed to remove the spaces between the
%    |\begin{...}| and the actual text.
% \end{macro}
% \end{macro}
%
%
% \subsection{Definition of the theorem substyles}
%
% As already pointed out, the theorem substyles, defined below, are
% only loaded when necessary.  Note that all these substyles, except
% \texttt{plain}, have |\slshape| as the default body font.
%
% \subsubsection{The \texttt{plain} style}
%
% \label{dok:th@plain} As the following macros use |@|, we have
% to locally set the |\catcode| of this symbol to ``letter''.
% This happens within a group, so that we do not have to worry about
% which |\catcode| that symbol had before.
%    \begin{macrocode}
%<*thp>
\begingroup \makeatletter
%    \end{macrocode}
% Since we are now within a group, we must make all definitions
% globally.  First we make sure that \texttt{theorem.sty} is loaded.
% This will allow us to use this file as a document style option
% without having to call \texttt{theorem} itself as an option.  At the
% same time, we assure that at least version 2 is loaded, since
% |\theorem@style| was not defined in earlier versions.
%    \begin{macrocode}
\@ifundefined{theorem@style}{\input{theorem.sty}}{}
\ProvidesFile{thp.sty}
             [\FMithmInfo]
%    \end{macrocode}
%
%
% \begin{macro}{\th@plain}
% \begin{macro}{\@begintheorem}
% \begin{macro}{\@opargbegintheorem}
%    |\theoremstyle{plain}| corresponds to the original
%    definition, except that the distances to the surrounding text are
%    determined by the parameters |\theorempreskipamount| and
%    |\theorempostskipamount|.
%    First we set the default body font.
%    \begin{macrocode}
\gdef\th@plain{\normalfont\itshape
%    \end{macrocode}
%    Then we define |\@begintheorem| and
%    |\@opargbegintheorem|. These two macros define how the
%    header of a theorem is typeset.  |\@opargbegintheorem| will
%    be called if a \textsf{theorem} environment with an optional
%    argument is encountered; otherwise, the header is constructed by
%    calling |\@begintheorem|.  If one of these macros is
%    executed, we are within a \textsf{trivlist} environment started by
%    |\@thm|.  So the theorem header is produced with an
%    |\item| command.
%
%    Instead of specifying the header font directly, all standard
%    theorem styles use the |\theorem@headerfont| macro to allow
%    customization.  The extra space (|\labelsep|) is necessary
%    because of problems in the \textsf{trivlist} environment.
%    \begin{macrocode}
  \def\@begintheorem##1##2{%
        \item[\hskip\labelsep \theorem@headerfont ##1\ ##2]}%
%    \end{macrocode}
%    The definition of |\@opargbegintheorem| is completely
%    analogous. The only difference is the fact that there exists a
%    third argument (which is the optional parameter of the
%    environment and contains additional information about the
%    theorem).  Customarily we enclose it in parentheses.
%    \begin{macrocode}
\def\@opargbegintheorem##1##2##3{%
   \item[\hskip\labelsep \theorem@headerfont ##1\ ##2\ (##3)]}}
%    \end{macrocode}
% \end{macro}
% \end{macro}
% \end{macro}
%
% \noindent
% We conclude with an |\endgroup| to restore the
% |\catcode| of |@|.
%    \begin{macrocode}
\endgroup
%</thp>
%    \end{macrocode}
%
%
% \subsubsection{The \texttt{break} style}
%
% This style option is stored in the file |thb.sty|.
% For the next two lines see the documentation for |\th@plain| on
% page~\pageref{dok:th@plain}.
%    \begin{macrocode}
%<*thb>
\begingroup \makeatletter
\@ifundefined{theorem@style}{\input{theorem.sty}}{}
\ProvidesFile{thb.sty}
             [\FMithmInfo]
%    \end{macrocode}
%
% \begin{macro}{\th@break}
%    |\theoremstyle{break}| produces a line break after the name
%    of the theorem.  The font is |\slshape|.  Hence, we define
%    |\th@break| as follows:
%    \begin{macrocode}
\gdef\th@break{\normalfont\slshape
  \def\@begintheorem##1##2{\item[%
%    \end{macrocode}
%    We run into the following problem: it is not possible to create
%    the header with |\item[|\meta{title}|]| and then start
%    a new line by, for example, |\mbox{}\\|. Such a definition
%    will fail whenever a list environment follows immediately.  With
%    the above construction, the |\mbox{}| causes the switch
%    |@inlabel| (cf.\ definition of |\list| and
%    |\trivlist| in \cite{latex:source}) to be set to \texttt{false}
%    and so the following list will insert additional vertical
%    space (|\topskip|). This is quite annoying. Therefore, we
%    create the line break within the |\item|. In order to ensure
%    that the text will begin at the proper position in the following
%    line, we simply pretend that the label does not take any
%    room.\footnote{This will lead to problems whenever very high
%    symbols occurring in the line tower into the
%    heading. So, something else has to be done here sometime.}
%    \begin{macrocode}
       \rlap{\vbox{\hbox{\hskip \labelsep\theorem@headerfont ##1\ ##2}%
                   \hbox{\strut}}}]}%
%    \end{macrocode}
%    Again, the definition of |\@opargbegintheorem| is
%    completely analogous.
%    \begin{macrocode}
\def\@opargbegintheorem##1##2##3{%
  \item[\rlap{\vbox{\hbox{\hskip \labelsep \theorem@headerfont
                     ##1\ ##2\ (##3)}%
                    \hbox{\strut}}}]}}
\endgroup
%</thb>
%    \end{macrocode}
% \end{macro}
%
%
% \subsubsection{The \texttt{changebreak} style}
%
%    \begin{macrocode}
% This style option is stored in the file |thcb.sty|.
%    \begin{macrocode}
%<*thcb>
\begingroup \makeatletter
\@ifundefined{theorem@style}{\input{theorem.sty}}{}
\ProvidesFile{thcb.sty}
             [\FMithmInfo]
%    \end{macrocode}
%
% \begin{macro}{\th@changebreak}
%    The \texttt{change-break} theorem style is like \texttt{break}
%    but with interchange of theorem name and theorem number.  Thus we
%    define |\th@changebreak| as follows:
%    \begin{macrocode}
\gdef\th@changebreak{\normalfont\slshape
  \def\@begintheorem##1##2{\item
       [\rlap{\vbox{\hbox{\hskip\labelsep \theorem@headerfont ##2\ ##1}%
                    \hbox{\strut}}}]}%
\def\@opargbegintheorem##1##2##3{%
  \item[\rlap{\vbox{\hbox{\hskip\labelsep \theorem@headerfont
                     ##2\ ##1\ (##3)}%
                    \hbox{\strut}}}]}}
\endgroup
%</thcb>
%    \end{macrocode}
% \end{macro}
%
%
% \subsubsection{The \texttt{change} style}
%
% This style option is stored in the file |thc.sty|.
%    \begin{macrocode}
%<*thc>
\begingroup \makeatletter
\@ifundefined{theorem@style}{\input{theorem.sty}}{}
\ProvidesFile{thc.sty}
             [\FMithmInfo]
%    \end{macrocode}
%
% \begin{macro}{\th@change}
%    The \texttt{change} theorem style corresponds to the
%    \texttt{change break} style without a linebreak after the header.
%    To say it in another way, it's the same as the \texttt{plain}
%    style but with number and name interchanged and |\slshape| as the
%    default font.
%    \begin{macrocode}
\gdef\th@change{\normalfont\slshape
  \def\@begintheorem##1##2{\item
       [\hskip\labelsep \theorem@headerfont ##2\ ##1]}%
\def\@opargbegintheorem##1##2##3{%
  \item[\hskip\labelsep \theorem@headerfont ##2\ ##1\ (##3)]}}
\endgroup
%</thc>
%    \end{macrocode}
% \end{macro}
%
%
% \subsubsection{The \texttt{marginbreak} style}
%
% This style option is the one used most often at Mainz.  It is saved
% in the file |thmb.sty|.
%    \begin{macrocode}
%<*thmb>
\begingroup \makeatletter
\@ifundefined{theorem@style}{\input{theorem.sty}}{}
\ProvidesFile{thmb.sty}
             [\FMithmInfo]
%    \end{macrocode}
%
% \begin{macro}{\th@marginbreak}
%    The \texttt{margin break} style is nearly the same as the
%    \texttt{change break} style. The only difference is the placement
%    of the theorem number.  We use |\llap| to place it in the left
%    margin.
%
%    In this style |\labelsep| denotes the separation between the
%    number and the text.
%    \begin{macrocode}
\gdef\th@marginbreak{\normalfont\slshape
  \def\@begintheorem##1##2{\item
       [\rlap{\vbox{\theorem@headerfont
                    \hbox{\llap{##2}\hskip\labelsep ##1}%
                    \hbox{\strut}}}]}%
\def\@opargbegintheorem##1##2##3{%
  \item[\rlap{\vbox{\theorem@headerfont
                    \hbox{\llap{##2}\hskip\labelsep ##1\ (##3)}%
                    \hbox{\strut}}}]}}
\endgroup
%</thmb>
%    \end{macrocode}
% \changes{v2.1c}{90/02/17}{`endgroup outside macrocode env.}
% \end{macro}
%
%
%
% \subsubsection{The \texttt{margin} style}
%
% This style option is stored in the file |thm.sty|.
%    \begin{macrocode}
%<*thm>
\begingroup \makeatletter
\@ifundefined{theorem@style}{\input{theorem.sty}}{}
\ProvidesFile{thm.sty}
             [\FMithmInfo]
%    \end{macrocode}
%
% \begin{macro}{\th@margin}
%    Again this is only a variant of the theorem styles described
%    above without any new ideas.
%    \begin{macrocode}
\gdef\th@margin{\normalfont\slshape
  \def\@begintheorem##1##2{\item
       [\theorem@headerfont \llap{##2}\hskip\labelsep ##1]}%
\def\@opargbegintheorem##1##2##3{%
  \item[\theorem@headerfont \llap{##2}\hskip\labelsep ##1\ (##3)]}}
\endgroup
%</thm>
%    \end{macrocode}
% \end{macro}
%
%
%
%
% \subsection{Final Definitions}
%
% \begin{macro}{\theorempreskipamount}
% \begin{macro}{\theorempostskipamount}
%    The \textsf{skip} parameters that regulate the vertical empty space
%    before and after the {\envfont theorem} environment have
%    to be allocated as well.
% \changes{v2.0f}{89/07/20}{Unnecessary `global' removed.}
%    \begin{macrocode}
%<*package>
\newskip\theorempreskipamount
\newskip\theorempostskipamount
%    \end{macrocode}
%    Since we have used the same values for all theorem sets, we now can
%    assign them.
%    \begin{macrocode}
\global\setlength\theorempreskipamount{12pt plus 5pt minus 3pt}
\global\setlength\theorempostskipamount{8pt plus 3pt minus 1.5pt}
%    \end{macrocode}
% \end{macro}
% \end{macro}
%
%
% \begin{macro}{\@endtheorem}
%    The same holds for the macro |\@endtheorem|, which ends a
%    {\envfont theorem} environment. Since it is the same for all
%    theorem sets, it is removed from the macros |\th@|\meta{style}.
%    It simply ends the {\envfont trivlist} environment, which was
%    begun in |\@thm|.
%    \begin{macrocode}
\global\let\@endtheorem=\endtrivlist
%    \end{macrocode}
% \end{macro}
%
%
%
% \begin{macro}{\@preamblecmds}
%    All macros defined above are to be used only in the preamble.
%    Therefore, we insert them in |\@preamblecmds| which will disable
%    them at begin document. This is done by the internal \LaTeXe{}
%    command |\@onlypreamble|.
% \changes{v2.0g}{89/09/19}{`theoremheaderfont and `theorembodyfont
%    added.}
% \changes{v2.2a}{94/02/03}{Use \LaTeXe{} interface}
%    \begin{macrocode}
\@onlypreamble\@xnthm
\@onlypreamble\@ynthm
\@onlypreamble\@othm
\@onlypreamble\newtheorem
\@onlypreamble\theoremstyle
\@onlypreamble\theorembodyfont
\@onlypreamble\theoremheaderfont
%    \end{macrocode}
% \end{macro}
%
% Finally we declare the \texttt{plain} theorem style to be the default.
% \changes{v2.1b}{90/06/04}{Fixed missing `item by declaring default
%    style.}
%    \begin{macrocode}
\theoremstyle{plain}
%</package>
%    \end{macrocode}
%
% \Finale

\endinput
}{}
\ProvidesFile{thmb.sty}
             [\FMithmInfo]
%    \end{macrocode}
%
% \begin{macro}{\th@marginbreak}
%    The \texttt{margin break} style is nearly the same as the
%    \texttt{change break} style. The only difference is the placement
%    of the theorem number.  We use |\llap| to place it in the left
%    margin.
%
%    In this style |\labelsep| denotes the separation between the
%    number and the text.
%    \begin{macrocode}
\gdef\th@marginbreak{\normalfont\slshape
  \def\@begintheorem##1##2{\item
       [\rlap{\vbox{\theorem@headerfont
                    \hbox{\llap{##2}\hskip\labelsep ##1}%
                    \hbox{\strut}}}]}%
\def\@opargbegintheorem##1##2##3{%
  \item[\rlap{\vbox{\theorem@headerfont
                    \hbox{\llap{##2}\hskip\labelsep ##1\ (##3)}%
                    \hbox{\strut}}}]}}
\endgroup
%</thmb>
%    \end{macrocode}
% \changes{v2.1c}{90/02/17}{`endgroup outside macrocode env.}
% \end{macro}
%
%
%
% \subsubsection{The \texttt{margin} style}
%
% This style option is stored in the file |thm.sty|.
%    \begin{macrocode}
%<*thm>
\begingroup \makeatletter
\@ifundefined{theorem@style}{% \iffalse meta-comment
%
% Copyright (C) 1993-2024
%
% The LaTeX Project and any individual authors listed elsewhere
% in this file.
%
% This file is part of the Standard LaTeX `Tools Bundle'.
% -------------------------------------------------------
%
% It may be distributed and/or modified under the
% conditions of the LaTeX Project Public License, either version 1.3c
% of this license or (at your option) any later version.
% The latest version of this license is in
%    https://www.latex-project.org/lppl.txt
% and version 1.3c or later is part of all distributions of LaTeX
% version 2005/12/01 or later.
%
% The list of all files belonging to the LaTeX `Tools Bundle' is
% given in the file `manifest.txt'.
%
% \fi
%
%%
%
%
\def\FMithmInfo{2023/07/05 v2.2c Theorem extension package (FMi)}
%
% \ProvidesFile{theorem.dtx}[\FMithmInfo]
%\iffalse   % this is a METACOMMENT !
%%
%% Package `theorem' to use with LaTeX2e
%% Copyright (C) 1989-1995 Frank Mittelbach, all rights reserved.
%<+package>\@ifundefined{theorem@style}{}{\endinput}
%<+package>\NeedsTeXFormat{LaTeX2e}[1995/06/01]
%<+package>\ProvidesPackage{theorem}
%<+package>         [\FMithmInfo]
%
% \fi
%
% \changes{v1.0}{87/05/07}{First draft.}
% \changes{v1.0a}{87/05/07}{`thistheoremstyle' added.}
% \changes{v1.1a}{87/07/29}{New concept: redefine the internal
%                          \protect\LaTeX{} commands `@xnthm' etc.}
% \changes{v1.1b}{87/07/31}{`theoremfont' used to specify font.}
% \changes{v1.2a}{88/01/08}{`@endtheorem' now globally defined.}
% \changes{v1.2b}{88/01/11}{`newtheorem' only allowed in the preamble.}
% \changes{v1.3a}{88/02/14}{`theoremfont' now a toks register.}
%
% \changes{v2.0a}{88/11/16}{New implementation and new documentation.}
% \changes{v2.0a}{88/11/16}{No layout changes but internal modifications
%                           old substyles won't work.}
% \changes{v2.0b}{89/01/30}{`global' used for all definitions.}
% \changes{v2.0c}{89/06/15}{Documentation translated to English.
%                            Thanks to LLL and Nina}
% \changes{v2.1a}{90/02/15}{Support for new font selection.}
% \changes{v2.2a}{94/02/03}{Cleanup for LaTeX2e}
%
% \DoNotIndex{\@endparenv,\@ifdefinable,\@ifnextchar,\@ifundefined}
% \DoNotIndex{\@namedef,\@nameuse,\@spaces,\@tempa,\@warning}
% \DoNotIndex{\ }
% \DoNotIndex{\begingroup,\bfseries}
% \DoNotIndex{\csname}
% \DoNotIndex{\def,\do,\docdate}
% \DoNotIndex{\endcsname,\endgroup,\endinput,\endtrivlist,\expandafter}
% \DoNotIndex{\filedate,\fileversion}
% \DoNotIndex{\gdef,\global}
% \DoNotIndex{\hbox,\hskip}
% \DoNotIndex{\ignorespaces,\input,\itshape,\item}
% \DoNotIndex{\labelsep,\let,\llap}
% \DoNotIndex{\makeatletter}
% \DoNotIndex{\noexpand}
% \DoNotIndex{\refstepcounter,\relax,\rlap}
% \DoNotIndex{\setlength,\slshape,\space,\string,\strut}
% \DoNotIndex{\the,\trivlist,\typeout}
% \DoNotIndex{\vbox}
% \DoNotIndex{\xdef}
%
% \def\deflabel#1{\ttfamily #1\hfill}
% \def\deflist#1{\begin{list}{}{\settowidth\labelwidth{\ttfamily #1}%
%                               \setlength\leftmargin\labelwidth
%                               \addtolength\leftmargin\labelsep
%                               \let\makelabel\deflabel}}
% \def\enddeflist{\end{list}}
% \def\envfont{\normalfont\ttfamily}
%
% \MakeShortVerb{\|}
%
% \newcommand{\TUB}{{\sl TUGboat\/}}
%
% \GetFileInfo{theorem.dtx}
% \title{An Extension of the
%        \LaTeX{} theorem environment\thanks{This file
%        has version number \fileversion, last
%        revised \filedate.}}
% \author{Frank Mittelbach}
%
% \MaintainedByLaTeXTeam{tools}
% \maketitle
%
% \begin{abstract}
%    The macros described in this paper yield an extension of the
%    \LaTeX{} theorem mechanism. It is designed to satisfy the
%    different requirements of various journals. Thus, the
%    layout of the ``theorems'' can be manipulated by determining a
%    ``style''. This article describes not only the use, but
%    also the definition, of the necessary macros.
% \end{abstract}
%
% \section*{Preface to version 2.2}
%
% For \LaTeXe{} this package did not need any fundamental changes. I
% only modified the messages generated so that theorem layout styles
% will show up with the |\listfiles| command and cleaned the section
% on the New Font Selection Scheme since this is now included in
% \LaTeX.
%
% \section*{Preface to version 2.1}
%
% This version is identical to 2.0g described in \TUB\ 10\#3 except
% for some internal defaults which are now set depending on the used
% font selection scheme.
%
% This was done to avoid unpleasant surprises if the new font selection
% scheme is in force. For further details see
% section~\ref{sec:newfonts} and \cite{bk:GMS94}.
%
% \section{Introduction}
%
% For our purposes here, ``theorems'' are labelled enunciations,
% often set off from the main text by extra space and a font change.
% Theorems, corollaries, conjectures, definitions, and remarks
% are all instances of ``theorems''.  The ``header'' of these structures
% is composed of a label (such as \textsc{Theorem} or \textsc{Remark})
% and a number which serializes an item in the sequence of items
% with the same label.
%
% Shortly after the introduction of \LaTeX{} at the Fachbereich
% Mathematik in Mainz, the desire to manipulate the layout of
% ``theorems'' arose.  In Mainz, the following two conventions came
% into general use:
% \begin{enumerate}
%   \item The number of the theorem is shown in the margin.
%   \item There is a line break at
%         the end of the theorem header.
% \end{enumerate}
%
% Additionally, some journals require different formats which depend on
% the ``sort of theorem'': e.g.\ often remarks and definitions are set
% in |\upshape|, while |\itshape| is employed for main theorems.
%
% Confronted with these requirements, a theorem environment was
% developed in Mainz which allows separate determination of the
% layout of the ``theorems sets'', comparable to |\pagestyle|.
%
%
% \section{The user interface}
% \subsection{Defining new theorem sets}
%
% \DescribeMacro\newtheorem
% As in the original \LaTeX{} version, the command |\newtheorem|
% defines a new ``theorem set'' or ``theorem-like structure''.
% Two required arguments name the new environment and give the text
% to be typeset with each instance of the new ``set'', while
% an optional argument determines how
% the ``set'' is enumerated:
% \begin{description}
%    \item[\ttfamily \bslash newtheorem\{foo\}\{bar\}]
%       The theorem set {\envfont foo} (whose name is \texttt{bar})
%       uses its own counter.
%    \item[\ttfamily \bslash newtheorem\{foo2\}{[foo]}\{bar2\}]
%       The theorem set {\envfont foo2} (printed name \texttt{bar2})
%       uses the same counter as the theorem set \texttt{foo}.
%    \item[\ttfamily \bslash newtheorem\{foo3\}\{bar3\}{[section]}]
%       The theorem set {\envfont foo3} (printed name \texttt{bar3}) is
%       enumerated within the counter \texttt{section}, i.e.\ with every
%       new |\section| the enumeration begins again with 1, and
%       the enumeration is composed from the section-number and the
%       theorem counter itself.
% \end{description}
%
% \DescribeMacro\theoremstyle
% Additionally, the command |\theoremstyle| can define the layout
% of various, or all, theorem sets.  It should be noted that any
% theorem set defined by |\newtheorem| is typeset in the
% |\theoremstyle| that is current at the time of the definition.
% Thus, the following
% \begin{verbatim}
%    \theoremstyle{break}        \newtheorem{Cor}{Corollary}
%    \theoremstyle{plain}        \newtheorem{Exa}{Example}[section]
%\end{verbatim}                             ^^A No blank at line start
% leads to the result that the set {\envfont Cor} is formatted in the
% style \texttt{break}, while the set {\envfont Exa} and all the
% following ones are formatted in the style \texttt{plain}, unless
% another |\theoremstyle| follows.  Since the definitions installed by
% |\newtheorem| are global, one also can limit |\theoremstyle| locally
% by grouping braces.
%
% \DescribeMacro\theorembodyfont
% The choice of the font for the theorem body is completely
% independent of the chosen |\theoremstyle|; this has proven to
% be very advantageous.  For example,
% \begin{verbatim}
%   {\theorembodyfont{\upshape}          \newtheorem{Rem}{Remark}}
%\end{verbatim}
% defines a theorem set {\envfont Rem}, which will be set in
% |\upshape| in the current layout (which in our example is
% \texttt{plain}).  As with |\theoremstyle|, the |\theorembodyfont|
% chosen is that current at the time of |\newtheorem|.  If
% |\theorembodyfont| is not specified or one defines
% |\theorembodyfont{}|, then the font used will be that defined by the
% |\theoremstyle|.
%
% \DescribeMacro\theoremheaderfont It is also possible to customize
% the font used for the theorem headers.  This is, however, a global
% declaration, and therefore there should be at most one
% |\theoremheaderfont| declaration in the preamble.\footnote{If it is
% actually necessary to have different header fonts, one has to define
% new theorem styles (substituting the desired font) or specify the
% information directly in the \texttt{\bslash newtheorem} declaration
% (the unclean variant).}
%
% \DescribeMacro\theorempreskipamount
% \DescribeMacro\theorempostskipamount Two additional parameters
% affect the vertical space around the theorem environments:
% |\theorempreskipamount| and |\theorempostskipamount| define,
% respectively, the spacing before and after such an environment.
% These parameters apply for all theorem sets and can be manipulated
% with the ordinary length macros.  They are rubber lengths
% (`\textsf{skips}'), and therefore can contain \texttt{plus} and
% \texttt{minus} parts.
%
% Since the definition of theorem sets should---most sensibly---be
% placed in the preamble, we only allow installation there.
% It is therefore possible to release the memory used here after
% |\begin{document}|, in order to make room for other
% applications.
%
%
% \subsection{Existing theorem styles}
%
% The following theorem styles exist to date:
% \begin{deflist}{marginbreak}
%    \item[plain]
%       This theorem style emulates the original \LaTeX{} definition,
%       except that additionally the parameters
%       |\theorem...skipamount| are used.
%    \item[break]
%       In this style, the theorem header is followed by a line
%       break.
%    \item[marginbreak]
%       The theorem number is set in the margin, and there is a line
%       break as in \texttt{break}.
%    \item[changebreak]
%       Like \texttt{break}, but with header number and text
%       interchanged.
%    \item[change]
%       Header number and text are interchanged, without a line break.
%    \item[margin]
%       The number is set in the left margin, without a line break.
% \end{deflist}
% All styles (except \texttt{plain}) select |\slshape| as the default
% |\theorembodyfont|.
%
%
% \subsection{Examples}
%
% Given the above theorem sets {\envfont Cor}, {\envfont Exa}
% and {\envfont Rem},
% suppose that the preamble also contains the declarations:
% \begin{verbatim}
%    \theoremstyle{marginbreak}   \newtheorem{Lem}[Cor]{Lemma}
%    \theoremstyle{change}
%    \theorembodyfont{\itshape}        \newtheorem{Def}[Cor]{Definition}
%
%    \theoremheaderfont{\scshape}
%\end{verbatim}
% Then the following are some typical examples of the typeset output
% resulting from their use.
%
% \begin{Cor}
%    This is a sentence typeset in the theorem environment {\envfont
%    Cor}.
% \end{Cor}
%
% \begin{Exa}
%    This is a sentence typeset in the theorem environment {\envfont
%    Exa}.
% \end{Exa}
%
% \begin{Rem}
%    This is a sentence typeset in the theorem environment {\envfont
%    Rem}.
% \end{Rem}
%
% \begin{Lem}[Ben User]
%    This is a sentence typeset in the theorem environment {\envfont
%    Lem}.
% \end{Lem}
%
% \begin{Def}[Very impressive Definition]
%    This is a sentence typeset in the theorem environment {\envfont
%    Def}.
% \end{Def}
%
% The last two examples show the effect of the optional argument to
% a theorem environment (it is the text typeset in parentheses).
%
%
% \section{Special Considerations}
% \label{sec:newfonts}
%
% Theoremheader and body are implemented as a unit. This means that
% the |\theoremheaderfont|  will inherit characteristics of the
% |\theorembodyfont| in \LaTeXe. Thus, if for example
% |\theorembodyfont| is |\itshape| and |\theoremheaderfont|
% is |\bfseries| the font selected for the header will have the
% characteristics `bold extended italic'. If this is not desired one
% should set the |\theoremheaderfont| to something like
% \begin{verbatim}
%      \theoremheaderfont{\normalfont\bfseries}
%\end{verbatim}
% i.e.\ supplying all necessary font information explicitly.
%
%
%
% \section{Acknowledgements}
%
% The publication of this set of macros was only possible with the
% help of Christina Busse (translating the manuscript into English),
% Joachim Pense (playing the r\^ole of typist), Chris Rowley
% (looking everything over) and many others providing useful
% suggestions.
%
% \MaybeStop{%
%    \begin{thebibliography}{9}
%       \bibitem{bk:GMS94} \textsc{M.~Goossens}, \textsc{F.~Mittelbach}
%         and \textsc{A.~Samarin}.
%         \newblock The \LaTeX{} Companion.
%         \newblock
%         Addison-Wesley, Reading, Massachusetts, 1994.
%       \bibitem{latex:source} \textsc{Lamport, Leslie.} \newblock
%                               \texttt{latex.tex}, version 2.09, date
%                               Feb. 1990.
%    \end{thebibliography}
%    \PrintIndex
%    \PrintChanges}
%
%
% \section{The documentation driver file}
%
% The next bit of code contains the documentation driver file for
% \TeX{}, i.e., the file that will produce the documentation you are
% currently reading. It will be extracted from this file by the
% \texttt{docstrip} program. Since it is the first code in the file
% one can alternatively process this file directly with \LaTeXe{} to
% obtain the documentation.
%
%    \begin{macrocode}
%<*driver>
\documentclass{ltxdoc}

\usepackage{theorem}

%<+driver>% The next few lines define theorem sets which are used
%<+driver>% in the example section of the documentation.

\theoremstyle{break}       \newtheorem{Cor}{Corollary}
\theoremstyle{plain}       \newtheorem{Exa}{Example}[section]
{\theorembodyfont{\upshape}\newtheorem{Rem}{Remark}}
\theoremstyle{marginbreak} \newtheorem{Lem}[Cor]{Lemma}
\theoremstyle{change}
\theorembodyfont{\itshape} \newtheorem{Def}[Cor]{Definition}

\theoremheaderfont{\scshape}

\RecordChanges

\begin{document}
   \DocInput{theorem.dtx}
\end{document}
%</driver>
%    \end{macrocode}
%
% \section{Definition of the Macros}
%
% If the file has been loaded before, we abort immediately. If not the
% package announces itself (this is actually done at the very top of
% the file---the way it is done isn't good style so don't copy it).
% \changes{v2.0e}{89/07/19}{Spaces removed from `typeout'.}
% \changes{v2.2b}{95/11/19}{Announce moved to top of file}
%    \begin{macrocode}
%<*package>
%\@ifundefined{theorem@style}{}{\endinput}
%\def\FMithmInfo{1995/11/19 v2.2b Theorem extension package (FMi)}
%\ProvidesPackage{theorem}[\FMithmInfo]
%    \end{macrocode}
%
%
% \subsection{Definition of theorem styles and fonts}
%
% All the definitions in this file are done globally to allow
% inputting this file inside a group.
%
% \begin{macro}{\theoremstyle}
%    Before a theorem style can be installed, the chosen style must be
%    known. For that reason, we must test to see that
%    |\th@|\meta{style} is known or, more precisely, that it is
%    different from |\relax|. If the style is not known then
%    |\th@plain| is used.
%    \begin{macrocode}
\gdef\theoremstyle#1{%
   \@ifundefined{th@#1}{\@warning
          {Unknown theoremstyle `#1'. Using `plain'}%
          \theorem@style{plain}}%
%    \end{macrocode}
%    We save the theorem style to be used in the token register
%    |\theorem@style|.
%    \begin{macrocode}
      {\theorem@style{#1}}%
%    \end{macrocode}
%    Now we ``evaluate'' the theorem style: this means, we call the
%    macro |\th@|\meta{style} which will activate the relevant
%    definitions which are contained in a separate file.  This is done
%    in a group to suppress changes to the current font.  This could
%    otherwise pose problems together with the new font selection
%    scheme\footnote{When I printed the original article using the new
%    font selection scheme I ended with a document with slanted
%    typefaces (text headings and all) simply because one of the
%    theorem styles used \texttt{\bslash sl} at toplevel.} if the
%    |\th@|\meta{style} is evaluated a second time.
% \changes{v2.1a}{90/02/15}{Group added around `th@..}
% \changes{v2.1a}{90/02/15}{Reseting of `..@begintheorem removed.}
%    \begin{macrocode}
      \begingroup
        \csname th@\the\theorem@style \endcsname
      \endgroup}
%    \end{macrocode}
% \end{macro}
%
%
% \begin{macro}{\@begintheorem}
% \changes{v2.1a}{90/02/15}{Relaxed at toplevel.}
% \begin{macro}{\@opargbegintheorem}
% \changes{v2.1a}{90/02/15}{Relaxed at toplevel.}
%    We reset |\@begintheorem| and |\@opargbegintheorem| to
%    |\relax| since these commands are no longer necessary at
%    toplevel. This will save a few tokens.
%    \begin{macrocode}
\global\let\@begintheorem\relax
\global\let\@opargbegintheorem\relax
%    \end{macrocode}
% \end{macro}
% \end{macro}
%
% \begin{macro}{\theorem@style}
%    Obviously the \textsf{token} register used above has to be
%    allocated. To assure the utmost compatibility with the
%    original \LaTeX{} definition, we set the default theorem style to
%    \texttt{plain}, which implements the usual \LaTeX{} convention.
% \changes{v2.0f}{89/07/20}{Unnecessary `global' removed.}
%    \begin{macrocode}
\newtoks\theorem@style
\global\theorem@style{plain}
%    \end{macrocode}
% \end{macro}
%
% \begin{macro}{\theorembodyfont}
% \begin{macro}{\theorem@bodyfont}
%    If the |\theorembodyfont| is set by the user then it should not
%    interact with the default font set in the theorem style. When the
%    new font selection is in force this may happen if, for example,
%    the default is |\itshape| and the new |\theorembodyfont| is
%    |\sffamily|. So we add a |\reset@font| command in front of the
%    user definition.
% \changes{v2.0f}{89/07/20}{Unnecessary `global' removed.}
% \changes{v2.0g}{89/09/19}{`theoremfont to `theorembodyfont.}
% \changes{v2.1c}{92/06/22}{`reset@font added and logic changed.}
% \changes{v2.1d}{92/10/28}{Reset to empty if arg is empty}
%    \begin{macrocode}
\gdef\theorembodyfont#1{%
%    \end{macrocode}
%    We check if the argument supplied is empty and if so put nothing
%    into the |\theorem@bodyfont| token register to allow for
%    |\theorembodyfont{}| as a mean of using the default of the
%    current |\theoremstyle|.
%    \begin{macrocode}
   \def\@tempa{#1}%
   \ifx\@tempa\@empty
    \theorem@bodyfont{}%
   \else
    \theorem@bodyfont{\reset@font#1}%
   \fi
   }
\newtoks\theorem@bodyfont
\global\theorem@bodyfont{}
%    \end{macrocode}
% \end{macro}
% \end{macro}
%
%
% \begin{macro}{\theoremheaderfont}
% \changes{v2.0g}{89/09/19}{Macro added.}
%    The font for the theorem headers is handled differently because
%    this definition applies to all theorem styles.
%    \begin{macrocode}
\gdef\theoremheaderfont#1{\gdef\theorem@headerfont{#1}%
%    \end{macrocode}
%    After using the macro once it is redefined to produce an error
%    message.
%    \begin{macrocode}
       \gdef\theoremheaderfont##1{%
        \typeout{\string\theoremheaderfont\space should be used
                 only once.}}}
%    \end{macrocode}
% \end{macro}
%
% \begin{macro}{\theorem@headerfont}
% \changes{v2.0g}{89/09/19}{Macro added.}
%    To set the |\theorem@headerfont| default we first test if
%    the new fontselection scheme is in force.
% \changes{v2.1a}{90/02/15}{Support for new font selection.}
%    \begin{macrocode}
\ifx\upshape\undefined
%    \end{macrocode}
%    If not we define it to expand into |\bfseries|. We don't use
%    |\let| just in case a following style option redefines this
%    macro.
%    \begin{macrocode}
\gdef\theorem@headerfont{\bfseries}
%    \end{macrocode}
%    Otherwise we reset the current shape before calling |\bfseries|.
%    \begin{macrocode}
\else \gdef\theorem@headerfont{\normalfont\bfseries}\fi
%    \end{macrocode}
% \end{macro}
%
% \begin{macro}{\th@plain}
% \begin{macro}{\th@break}
% \begin{macro}{\th@marginbreak}
% \begin{macro}{\th@changebreak}
% \begin{macro}{\th@change}
% \begin{macro}{\th@margin}
%    The different styles are defined in macros such as
%    |\th@plain|.  Since memory space is precious in
%    ``non-Big-versions'', we have to avoid offering too many unused
%    definitions.  Therefore we define these styles in separate files
%    that can be loaded on demand.  Thus the commands themselves only
%    load these files. We use |\@input@|, a \LaTeXe{} internal command
%    that ensures that the file will be listed with |\listfiles|.
% \changes{v2.2a}{94/02/02}{Use `@input@ to load theorem layout files}
%    \begin{macrocode}
\gdef\th@plain{\@input@{thp.sty}}
\gdef\th@break{\@input@{thb.sty}}
\gdef\th@marginbreak{\@input@{thmb.sty}}
\gdef\th@changebreak{\@input@{thcb.sty}}
\gdef\th@change{\@input@{thc.sty}}
\gdef\th@margin{\@input@{thm.sty}}
%    \end{macrocode}
%    This list will be expanded when new styles become
%    available.  For testing,  just append new theorem substyles
%    as document options.
% \end{macro}
% \end{macro}
% \end{macro}
% \end{macro}
% \end{macro}
% \end{macro}
%
%
%
% \subsection{Definition of a new theorem set}
%
% As already pointed out, a new theorem environment can be defined in
% three different ways:
% \begin{verbatim}
%    \newtheorem{Lem}{Lemma}
%    \newtheorem{Lem}{Lemma}[section]
%    \newtheorem{Lem}[Theorem]{Lemma}
%\end{verbatim}
% The function of the macro |\newtheorem| is to recognize these cases
% and then to branch into one of the three macros |\@ynthm|, |\@xnthm|
% or |\@othm|.  This mechanism is adopted unchanged from
% \cite{latex:source}; the essential point here is that, for example,
% in the second case, the arguments \texttt{Lem}, \texttt{Lemma} and
% \texttt{section} are passed over to the macro |\@xnthm|.
%
% We inspect this case first because the others present fewer problems,
% and thus are easily derived from this one.
%
% \begin{macro}{\@xnthm}
%    For our example arguments, the macro |\@xnthm| must fulfill
%    the following:
%    \begin{itemize}
%       \item
%          Define a new \LaTeX-counter `Lem'
%       \item
%          reset this counter within a |\section|
%       \item
%          define the macro |\theLem|
%       \item
%          define the environment macros |\Lem| and
%          |\endLem| using the current
%          |\theoremstyle| and |\theorem@bodyfont|.
%    \end{itemize}
%    Obviously, all this should happen only if the first argument
%    of |\@xnthm| (i.e.\ \texttt{Lem} in our example) is chosen so
%    as not to conflict with any previously defined commands or
%    environments.  This test is performed by the \LaTeX{} macro
%    |\@ifdefinable|.
%    \begin{macrocode}
\gdef\@xnthm#1#2[#3]{\expandafter\@ifdefinable\csname #1\endcsname
%    \end{macrocode}
%    Therefore, the first argument of |\@ifdefinable| is the
%    expansion (in the example, |\Lem|) of
%    |\csname||#1\endcsname|.  The second argument is
%    executed only if the test has been completed successfully.
%    \begin{macrocode}
   {%
%    \end{macrocode}
%    Now we define the new counter. The names of the \LaTeX{}
%    macros employed should speak for themselves:
% \changes{v2.2b}{1995/10/07}{Use \cs{@newctr}}
% \changes{v2.2c}{1995/11/23}{Use \cs{@newctr} properly.}
%    \begin{macrocode}
    \@definecounter{#1}\@newctr{#1}[#3]%
%    \end{macrocode}
%    Using |\@newctr| will give a proper error message if the counter
%    in |#3| is not defined.
%    In defining `|\theLem|' we must generate the desired macro
%    name by use of |\expandafter| and |\csname|.
%    \begin{macrocode}
    \expandafter\xdef\csname the#1\endcsname
%    \end{macrocode}
%    An |\xdef| is used in order to make the definition global,
%    and to ensure that it contains the replacement texts of
%    |\@thmcountersep| and |\@thmcounter|.\footnote{These
%    two macros can be defined by the document style.  Their default
%    values produce a `\texttt{.}' as separation and an arabic
%    representation of the number.} However, not everything should be
%    expanded. For example, it saves space to use |\thesection|
%    instead of its---at times---lengthy expansion.
%    \begin{macrocode}
      {\expandafter \noexpand \csname the#3\endcsname
       \@thmcountersep \@thmcounter{#1}}%
%    \end{macrocode}
%    Thus with the defaults of \LaTeX{}, |\theLem| would be
%    replaced by the command sequence |\thesection.\arabic{Lem}|.
%
%    We will now look at the definition of the macro which is executed
%    at the beginning of the actual environment (in our example this
%    macro is |\Lem|). It should be noted that we use an
%    ``|\expandafter| trick'' to expand only certain parts of the
%    replacement text at the time of the definition.
%    \begin{macrocode}
    \def\@tempa{\global\@namedef{#1}}%
    \expandafter \@tempa \expandafter{%
%    \end{macrocode}
%    First, the macro that contains the current definitions of
%    |\@begintheorem| and |\@opargtheorem| should be called
%    up.  The name of this macro---as is already known---has the form
%    |\th@|\meta{theorem\ style}; therefore, it must be called by
%    \begin{macrocode}
      \csname th@\the \theorem@style
%    \end{macrocode}
%    In addition the default theorem font should be changeable, i.e.\
%    we have to insert the contents of |\theorem@bodyfont|. For
%    that reason, we expand even further, beyond |\endcsname|,
%    and thus insert the contents of the \textsf{token} register
%    |\theorem@bodyfont| in the replacement text.
%    \begin{macrocode}
            \expandafter \endcsname \the \theorem@bodyfont
%    \end{macrocode}
%    Now it is time to call the macro |\@thm| which takes over
%    the further processing.  It has two arguments: the current
%    counter name (in our example, \texttt{Lem}), and the text of the
%    label (in our example, \texttt{Lemma}).
%    \begin{macrocode}
     \@thm{#1}{#2}}%
%    \end{macrocode}
%    With this, the `sub-definition' is complete.  The macro
%    |\@endtheorem| ends a theorem environment and is, so far, nothing
%    but an |\endtrivlist|.  (Hence it is defined globally, and not
%    within the theorem styles.\footnote{This has to be changed as
%    soon as theorem styles that change \texttt{\bslash @endtheorem}
%    exist.  In such a case, all existing styles must be changed as
%    well since they will have to reset the macro.}) Therefore, we can
%    set it equivalent to the macro that ends the theorem set (in our
%    example, |\endLem|).  However, if some day theorem styles exist
%    that do change |\@endtheorem|, we would have to use the
%    commented-out line instead.
%    \begin{macrocode}
    \global \expandafter \let \csname end#1\endcsname \@endtheorem
%   \global\@namedef{end#1}{\@endtheorem}%
%    \end{macrocode}
%    With these commands all the required definitions are employed,
%    unless the test |\@ifdefinable| has failed.  Therefore, we
%    end the second argument of this macro and with it the definition
%    of |\@xnthm|.
%    \begin{macrocode}
   }}
%    \end{macrocode}
% \end{macro}
%
% \begin{macro}{\@ynthm}
%    The definition of |\@ynthm| is completely analogous. In
%    this case the new counter that is defined is not reset within
%    another counter; thus the definition of |\the...|
%    is simplified:
%    \begin{macrocode}
\gdef\@ynthm#1#2{\expandafter\@ifdefinable\csname #1\endcsname
   {\@definecounter{#1}%
    \expandafter\xdef\csname the#1\endcsname{\@thmcounter{#1}}%
%    \end{macrocode}
%    The rest of the definition corresponds literally to that of
%    |\@xnthm|:
%    \begin{macrocode}
    \def\@tempa{\global\@namedef{#1}}\expandafter \@tempa
     \expandafter{\csname th@\the \theorem@style \expandafter
     \endcsname \the\theorem@bodyfont \@thm{#1}{#2}}%
    \global \expandafter \let \csname end#1\endcsname \@endtheorem}}
%    \end{macrocode}
% \end{macro}
%
% \begin{macro}{\@othm}
%    The definition of |\@othm| does not contain anything new.
%    \begin{macrocode}
\gdef\@othm#1[#2]#3{%
%    \end{macrocode}
% \changes{v2.2b}{95/10/07}{Build in check for counter defined}
%    We do not define a new counter but instead use one that has
%    already been defined.  Thus the only definition we need is that
%    of this pseudo-counter (i.e.\ |\the|\meta{env.\ name}).
%    First we check if |#2| corresponds to a known counter name.
%    \begin{macrocode}
  \expandafter\ifx\csname c@#2\endcsname\relax
   \@nocounterr{#2}%
  \else
   \expandafter\@ifdefinable\csname #1\endcsname
   {\expandafter \xdef \csname the#1\endcsname
     {\expandafter \noexpand \csname the#2\endcsname}%
%    \end{macrocode}
%    All other parts of the definition can be adopted from
%    |\@xnthm|. We have to remember, though, that in this case
%    the name of the current counter and the theorem label have moved
%    to the second and third arguments.
%    \begin{macrocode}
    \def\@tempa{\global\@namedef{#1}}\expandafter \@tempa
     \expandafter{\csname th@\the \theorem@style \expandafter
     \endcsname \the\theorem@bodyfont \@thm{#2}{#3}}%
    \global \expandafter \let \csname end#1\endcsname \@endtheorem}%
  \fi}
%    \end{macrocode}
% \end{macro}
%
% \subsection{Macros that are employed in a {\envfont theorem}
%    environment}
%
% \begin{macro}{\@thm}
%    The macro |\@thm| has to increase the current counter.  Then,
%    depending on whether the environment has (or does not have) an
%    optional argument, it has to branch into either |\@begintheorem|
%    or |\@opargtheorem|.
%    \begin{macrocode}
\gdef\@thm#1#2{\refstepcounter{#1}%
%    \end{macrocode}
%    Now we start a {\envfont trivlist} environment, and give
%    |\@topsep| and |\@topsepadd| the values of the \textsf{skip}
%    registers |\theorempreskipamount| and |\theorempostskipamount|.
%    The value in |\@topsep| is the vertical space that is inserted by
%    the first (and only) |\item| in our |\trivlist| whilst
%    |\@topsepadd| is inserted by |\@endparenv| at the end of that
%    {\envfont trivlist} environment.  By using these registers, we
%    obtain the desired space around a {\envfont theorem} environment.
%    \begin{macrocode}
   \trivlist
   \@topsep \theorempreskipamount               % used by first \item
   \@topsepadd \theorempostskipamount           % used by \@endparenv
%    \end{macrocode}
%    Now we have to test whether an optional argument has been given.
% \changes{v2.0d}{89/07/18}{Forgotten percent added.}
%    \begin{macrocode}
   \@ifnextchar [%
%    \end{macrocode}
%    If there is an optional argument, we will call |\@ythm|, and
%    move the arguments read back into the input stream.
%    \begin{macrocode}
   {\@ythm{#1}{#2}}%
%    \end{macrocode}
%    If not, we call |\@begintheorem|. Its first argument is the
%    name of the theorem set (hence the second argument of
%    |\@thm|). Its second argument is the macro that produces the
%    current number.
%    \begin{macrocode}
   {\@begintheorem{#2}{\csname the#1\endcsname}\ignorespaces}}
%    \end{macrocode}
% \end{macro}
%
% \begin{macro}{\@xthm}
% \begin{macro}{\@ythm}
%    Both these macros were originally called by |\@thm|.  We do
%    not need |\@xthm| anymore, hence we reset it to
%    |\relax|. The definition of |\@ythm| has not changed at
%    all from its definition in \LaTeX. In order to make the macros
%    easier to understand, we will
%    nevertheless present it (commented out).
%    \begin{macrocode}
\global\let\@xthm\relax
% \def\@ythm#1#2[#3]{\@opargbegintheorem{#2}{\csname
%           the#1\endcsname}{#3}\ignorespaces}
%</package>
%    \end{macrocode}
%    The primitive |\ignorespaces| in |\@ythm| and
%    |\@thm| is needed to remove the spaces between the
%    |\begin{...}| and the actual text.
% \end{macro}
% \end{macro}
%
%
% \subsection{Definition of the theorem substyles}
%
% As already pointed out, the theorem substyles, defined below, are
% only loaded when necessary.  Note that all these substyles, except
% \texttt{plain}, have |\slshape| as the default body font.
%
% \subsubsection{The \texttt{plain} style}
%
% \label{dok:th@plain} As the following macros use |@|, we have
% to locally set the |\catcode| of this symbol to ``letter''.
% This happens within a group, so that we do not have to worry about
% which |\catcode| that symbol had before.
%    \begin{macrocode}
%<*thp>
\begingroup \makeatletter
%    \end{macrocode}
% Since we are now within a group, we must make all definitions
% globally.  First we make sure that \texttt{theorem.sty} is loaded.
% This will allow us to use this file as a document style option
% without having to call \texttt{theorem} itself as an option.  At the
% same time, we assure that at least version 2 is loaded, since
% |\theorem@style| was not defined in earlier versions.
%    \begin{macrocode}
\@ifundefined{theorem@style}{\input{theorem.sty}}{}
\ProvidesFile{thp.sty}
             [\FMithmInfo]
%    \end{macrocode}
%
%
% \begin{macro}{\th@plain}
% \begin{macro}{\@begintheorem}
% \begin{macro}{\@opargbegintheorem}
%    |\theoremstyle{plain}| corresponds to the original
%    definition, except that the distances to the surrounding text are
%    determined by the parameters |\theorempreskipamount| and
%    |\theorempostskipamount|.
%    First we set the default body font.
%    \begin{macrocode}
\gdef\th@plain{\normalfont\itshape
%    \end{macrocode}
%    Then we define |\@begintheorem| and
%    |\@opargbegintheorem|. These two macros define how the
%    header of a theorem is typeset.  |\@opargbegintheorem| will
%    be called if a \textsf{theorem} environment with an optional
%    argument is encountered; otherwise, the header is constructed by
%    calling |\@begintheorem|.  If one of these macros is
%    executed, we are within a \textsf{trivlist} environment started by
%    |\@thm|.  So the theorem header is produced with an
%    |\item| command.
%
%    Instead of specifying the header font directly, all standard
%    theorem styles use the |\theorem@headerfont| macro to allow
%    customization.  The extra space (|\labelsep|) is necessary
%    because of problems in the \textsf{trivlist} environment.
%    \begin{macrocode}
  \def\@begintheorem##1##2{%
        \item[\hskip\labelsep \theorem@headerfont ##1\ ##2]}%
%    \end{macrocode}
%    The definition of |\@opargbegintheorem| is completely
%    analogous. The only difference is the fact that there exists a
%    third argument (which is the optional parameter of the
%    environment and contains additional information about the
%    theorem).  Customarily we enclose it in parentheses.
%    \begin{macrocode}
\def\@opargbegintheorem##1##2##3{%
   \item[\hskip\labelsep \theorem@headerfont ##1\ ##2\ (##3)]}}
%    \end{macrocode}
% \end{macro}
% \end{macro}
% \end{macro}
%
% \noindent
% We conclude with an |\endgroup| to restore the
% |\catcode| of |@|.
%    \begin{macrocode}
\endgroup
%</thp>
%    \end{macrocode}
%
%
% \subsubsection{The \texttt{break} style}
%
% This style option is stored in the file |thb.sty|.
% For the next two lines see the documentation for |\th@plain| on
% page~\pageref{dok:th@plain}.
%    \begin{macrocode}
%<*thb>
\begingroup \makeatletter
\@ifundefined{theorem@style}{\input{theorem.sty}}{}
\ProvidesFile{thb.sty}
             [\FMithmInfo]
%    \end{macrocode}
%
% \begin{macro}{\th@break}
%    |\theoremstyle{break}| produces a line break after the name
%    of the theorem.  The font is |\slshape|.  Hence, we define
%    |\th@break| as follows:
%    \begin{macrocode}
\gdef\th@break{\normalfont\slshape
  \def\@begintheorem##1##2{\item[%
%    \end{macrocode}
%    We run into the following problem: it is not possible to create
%    the header with |\item[|\meta{title}|]| and then start
%    a new line by, for example, |\mbox{}\\|. Such a definition
%    will fail whenever a list environment follows immediately.  With
%    the above construction, the |\mbox{}| causes the switch
%    |@inlabel| (cf.\ definition of |\list| and
%    |\trivlist| in \cite{latex:source}) to be set to \texttt{false}
%    and so the following list will insert additional vertical
%    space (|\topskip|). This is quite annoying. Therefore, we
%    create the line break within the |\item|. In order to ensure
%    that the text will begin at the proper position in the following
%    line, we simply pretend that the label does not take any
%    room.\footnote{This will lead to problems whenever very high
%    symbols occurring in the line tower into the
%    heading. So, something else has to be done here sometime.}
%    \begin{macrocode}
       \rlap{\vbox{\hbox{\hskip \labelsep\theorem@headerfont ##1\ ##2}%
                   \hbox{\strut}}}]}%
%    \end{macrocode}
%    Again, the definition of |\@opargbegintheorem| is
%    completely analogous.
%    \begin{macrocode}
\def\@opargbegintheorem##1##2##3{%
  \item[\rlap{\vbox{\hbox{\hskip \labelsep \theorem@headerfont
                     ##1\ ##2\ (##3)}%
                    \hbox{\strut}}}]}}
\endgroup
%</thb>
%    \end{macrocode}
% \end{macro}
%
%
% \subsubsection{The \texttt{changebreak} style}
%
%    \begin{macrocode}
% This style option is stored in the file |thcb.sty|.
%    \begin{macrocode}
%<*thcb>
\begingroup \makeatletter
\@ifundefined{theorem@style}{\input{theorem.sty}}{}
\ProvidesFile{thcb.sty}
             [\FMithmInfo]
%    \end{macrocode}
%
% \begin{macro}{\th@changebreak}
%    The \texttt{change-break} theorem style is like \texttt{break}
%    but with interchange of theorem name and theorem number.  Thus we
%    define |\th@changebreak| as follows:
%    \begin{macrocode}
\gdef\th@changebreak{\normalfont\slshape
  \def\@begintheorem##1##2{\item
       [\rlap{\vbox{\hbox{\hskip\labelsep \theorem@headerfont ##2\ ##1}%
                    \hbox{\strut}}}]}%
\def\@opargbegintheorem##1##2##3{%
  \item[\rlap{\vbox{\hbox{\hskip\labelsep \theorem@headerfont
                     ##2\ ##1\ (##3)}%
                    \hbox{\strut}}}]}}
\endgroup
%</thcb>
%    \end{macrocode}
% \end{macro}
%
%
% \subsubsection{The \texttt{change} style}
%
% This style option is stored in the file |thc.sty|.
%    \begin{macrocode}
%<*thc>
\begingroup \makeatletter
\@ifundefined{theorem@style}{\input{theorem.sty}}{}
\ProvidesFile{thc.sty}
             [\FMithmInfo]
%    \end{macrocode}
%
% \begin{macro}{\th@change}
%    The \texttt{change} theorem style corresponds to the
%    \texttt{change break} style without a linebreak after the header.
%    To say it in another way, it's the same as the \texttt{plain}
%    style but with number and name interchanged and |\slshape| as the
%    default font.
%    \begin{macrocode}
\gdef\th@change{\normalfont\slshape
  \def\@begintheorem##1##2{\item
       [\hskip\labelsep \theorem@headerfont ##2\ ##1]}%
\def\@opargbegintheorem##1##2##3{%
  \item[\hskip\labelsep \theorem@headerfont ##2\ ##1\ (##3)]}}
\endgroup
%</thc>
%    \end{macrocode}
% \end{macro}
%
%
% \subsubsection{The \texttt{marginbreak} style}
%
% This style option is the one used most often at Mainz.  It is saved
% in the file |thmb.sty|.
%    \begin{macrocode}
%<*thmb>
\begingroup \makeatletter
\@ifundefined{theorem@style}{\input{theorem.sty}}{}
\ProvidesFile{thmb.sty}
             [\FMithmInfo]
%    \end{macrocode}
%
% \begin{macro}{\th@marginbreak}
%    The \texttt{margin break} style is nearly the same as the
%    \texttt{change break} style. The only difference is the placement
%    of the theorem number.  We use |\llap| to place it in the left
%    margin.
%
%    In this style |\labelsep| denotes the separation between the
%    number and the text.
%    \begin{macrocode}
\gdef\th@marginbreak{\normalfont\slshape
  \def\@begintheorem##1##2{\item
       [\rlap{\vbox{\theorem@headerfont
                    \hbox{\llap{##2}\hskip\labelsep ##1}%
                    \hbox{\strut}}}]}%
\def\@opargbegintheorem##1##2##3{%
  \item[\rlap{\vbox{\theorem@headerfont
                    \hbox{\llap{##2}\hskip\labelsep ##1\ (##3)}%
                    \hbox{\strut}}}]}}
\endgroup
%</thmb>
%    \end{macrocode}
% \changes{v2.1c}{90/02/17}{`endgroup outside macrocode env.}
% \end{macro}
%
%
%
% \subsubsection{The \texttt{margin} style}
%
% This style option is stored in the file |thm.sty|.
%    \begin{macrocode}
%<*thm>
\begingroup \makeatletter
\@ifundefined{theorem@style}{\input{theorem.sty}}{}
\ProvidesFile{thm.sty}
             [\FMithmInfo]
%    \end{macrocode}
%
% \begin{macro}{\th@margin}
%    Again this is only a variant of the theorem styles described
%    above without any new ideas.
%    \begin{macrocode}
\gdef\th@margin{\normalfont\slshape
  \def\@begintheorem##1##2{\item
       [\theorem@headerfont \llap{##2}\hskip\labelsep ##1]}%
\def\@opargbegintheorem##1##2##3{%
  \item[\theorem@headerfont \llap{##2}\hskip\labelsep ##1\ (##3)]}}
\endgroup
%</thm>
%    \end{macrocode}
% \end{macro}
%
%
%
%
% \subsection{Final Definitions}
%
% \begin{macro}{\theorempreskipamount}
% \begin{macro}{\theorempostskipamount}
%    The \textsf{skip} parameters that regulate the vertical empty space
%    before and after the {\envfont theorem} environment have
%    to be allocated as well.
% \changes{v2.0f}{89/07/20}{Unnecessary `global' removed.}
%    \begin{macrocode}
%<*package>
\newskip\theorempreskipamount
\newskip\theorempostskipamount
%    \end{macrocode}
%    Since we have used the same values for all theorem sets, we now can
%    assign them.
%    \begin{macrocode}
\global\setlength\theorempreskipamount{12pt plus 5pt minus 3pt}
\global\setlength\theorempostskipamount{8pt plus 3pt minus 1.5pt}
%    \end{macrocode}
% \end{macro}
% \end{macro}
%
%
% \begin{macro}{\@endtheorem}
%    The same holds for the macro |\@endtheorem|, which ends a
%    {\envfont theorem} environment. Since it is the same for all
%    theorem sets, it is removed from the macros |\th@|\meta{style}.
%    It simply ends the {\envfont trivlist} environment, which was
%    begun in |\@thm|.
%    \begin{macrocode}
\global\let\@endtheorem=\endtrivlist
%    \end{macrocode}
% \end{macro}
%
%
%
% \begin{macro}{\@preamblecmds}
%    All macros defined above are to be used only in the preamble.
%    Therefore, we insert them in |\@preamblecmds| which will disable
%    them at begin document. This is done by the internal \LaTeXe{}
%    command |\@onlypreamble|.
% \changes{v2.0g}{89/09/19}{`theoremheaderfont and `theorembodyfont
%    added.}
% \changes{v2.2a}{94/02/03}{Use \LaTeXe{} interface}
%    \begin{macrocode}
\@onlypreamble\@xnthm
\@onlypreamble\@ynthm
\@onlypreamble\@othm
\@onlypreamble\newtheorem
\@onlypreamble\theoremstyle
\@onlypreamble\theorembodyfont
\@onlypreamble\theoremheaderfont
%    \end{macrocode}
% \end{macro}
%
% Finally we declare the \texttt{plain} theorem style to be the default.
% \changes{v2.1b}{90/06/04}{Fixed missing `item by declaring default
%    style.}
%    \begin{macrocode}
\theoremstyle{plain}
%</package>
%    \end{macrocode}
%
% \Finale

\endinput
}{}
\ProvidesFile{thm.sty}
             [\FMithmInfo]
%    \end{macrocode}
%
% \begin{macro}{\th@margin}
%    Again this is only a variant of the theorem styles described
%    above without any new ideas.
%    \begin{macrocode}
\gdef\th@margin{\normalfont\slshape
  \def\@begintheorem##1##2{\item
       [\theorem@headerfont \llap{##2}\hskip\labelsep ##1]}%
\def\@opargbegintheorem##1##2##3{%
  \item[\theorem@headerfont \llap{##2}\hskip\labelsep ##1\ (##3)]}}
\endgroup
%</thm>
%    \end{macrocode}
% \end{macro}
%
%
%
%
% \subsection{Final Definitions}
%
% \begin{macro}{\theorempreskipamount}
% \begin{macro}{\theorempostskipamount}
%    The \textsf{skip} parameters that regulate the vertical empty space
%    before and after the {\envfont theorem} environment have
%    to be allocated as well.
% \changes{v2.0f}{89/07/20}{Unnecessary `global' removed.}
%    \begin{macrocode}
%<*package>
\newskip\theorempreskipamount
\newskip\theorempostskipamount
%    \end{macrocode}
%    Since we have used the same values for all theorem sets, we now can
%    assign them.
%    \begin{macrocode}
\global\setlength\theorempreskipamount{12pt plus 5pt minus 3pt}
\global\setlength\theorempostskipamount{8pt plus 3pt minus 1.5pt}
%    \end{macrocode}
% \end{macro}
% \end{macro}
%
%
% \begin{macro}{\@endtheorem}
%    The same holds for the macro |\@endtheorem|, which ends a
%    {\envfont theorem} environment. Since it is the same for all
%    theorem sets, it is removed from the macros |\th@|\meta{style}.
%    It simply ends the {\envfont trivlist} environment, which was
%    begun in |\@thm|.
%    \begin{macrocode}
\global\let\@endtheorem=\endtrivlist
%    \end{macrocode}
% \end{macro}
%
%
%
% \begin{macro}{\@preamblecmds}
%    All macros defined above are to be used only in the preamble.
%    Therefore, we insert them in |\@preamblecmds| which will disable
%    them at begin document. This is done by the internal \LaTeXe{}
%    command |\@onlypreamble|.
% \changes{v2.0g}{89/09/19}{`theoremheaderfont and `theorembodyfont
%    added.}
% \changes{v2.2a}{94/02/03}{Use \LaTeXe{} interface}
%    \begin{macrocode}
\@onlypreamble\@xnthm
\@onlypreamble\@ynthm
\@onlypreamble\@othm
\@onlypreamble\newtheorem
\@onlypreamble\theoremstyle
\@onlypreamble\theorembodyfont
\@onlypreamble\theoremheaderfont
%    \end{macrocode}
% \end{macro}
%
% Finally we declare the \texttt{plain} theorem style to be the default.
% \changes{v2.1b}{90/06/04}{Fixed missing `item by declaring default
%    style.}
%    \begin{macrocode}
\theoremstyle{plain}
%</package>
%    \end{macrocode}
%
% \Finale

\endinput
}{}
\ProvidesFile{thm.sty}
             [\FMithmInfo]
%    \end{macrocode}
%
% \begin{macro}{\th@margin}
%    Again this is only a variant of the theorem styles described
%    above without any new ideas.
%    \begin{macrocode}
\gdef\th@margin{\normalfont\slshape
  \def\@begintheorem##1##2{\item
       [\theorem@headerfont \llap{##2}\hskip\labelsep ##1]}%
\def\@opargbegintheorem##1##2##3{%
  \item[\theorem@headerfont \llap{##2}\hskip\labelsep ##1\ (##3)]}}
\endgroup
%</thm>
%    \end{macrocode}
% \end{macro}
%
%
%
%
% \subsection{Final Definitions}
%
% \begin{macro}{\theorempreskipamount}
% \begin{macro}{\theorempostskipamount}
%    The \textsf{skip} parameters that regulate the vertical empty space
%    before and after the {\envfont theorem} environment have
%    to be allocated as well.
% \changes{v2.0f}{89/07/20}{Unnecessary `global' removed.}
%    \begin{macrocode}
%<*package>
\newskip\theorempreskipamount
\newskip\theorempostskipamount
%    \end{macrocode}
%    Since we have used the same values for all theorem sets, we now can
%    assign them.
%    \begin{macrocode}
\global\setlength\theorempreskipamount{12pt plus 5pt minus 3pt}
\global\setlength\theorempostskipamount{8pt plus 3pt minus 1.5pt}
%    \end{macrocode}
% \end{macro}
% \end{macro}
%
%
% \begin{macro}{\@endtheorem}
%    The same holds for the macro |\@endtheorem|, which ends a
%    {\envfont theorem} environment. Since it is the same for all
%    theorem sets, it is removed from the macros |\th@|\meta{style}.
%    It simply ends the {\envfont trivlist} environment, which was
%    begun in |\@thm|.
%    \begin{macrocode}
\global\let\@endtheorem=\endtrivlist
%    \end{macrocode}
% \end{macro}
%
%
%
% \begin{macro}{\@preamblecmds}
%    All macros defined above are to be used only in the preamble.
%    Therefore, we insert them in |\@preamblecmds| which will disable
%    them at begin document. This is done by the internal \LaTeXe{}
%    command |\@onlypreamble|.
% \changes{v2.0g}{89/09/19}{`theoremheaderfont and `theorembodyfont
%    added.}
% \changes{v2.2a}{94/02/03}{Use \LaTeXe{} interface}
%    \begin{macrocode}
\@onlypreamble\@xnthm
\@onlypreamble\@ynthm
\@onlypreamble\@othm
\@onlypreamble\newtheorem
\@onlypreamble\theoremstyle
\@onlypreamble\theorembodyfont
\@onlypreamble\theoremheaderfont
%    \end{macrocode}
% \end{macro}
%
% Finally we declare the \texttt{plain} theorem style to be the default.
% \changes{v2.1b}{90/06/04}{Fixed missing `item by declaring default
%    style.}
%    \begin{macrocode}
\theoremstyle{plain}
%</package>
%    \end{macrocode}
%
% \Finale

\endinput
}{}
\ProvidesFile{thp.sty}
             [\FMithmInfo]
%    \end{macrocode}
%
%
% \begin{macro}{\th@plain}
% \begin{macro}{\@begintheorem}
% \begin{macro}{\@opargbegintheorem}
%    |\theoremstyle{plain}| corresponds to the original
%    definition, except that the distances to the surrounding text are
%    determined by the parameters |\theorempreskipamount| and
%    |\theorempostskipamount|.
%    First we set the default body font.
%    \begin{macrocode}
\gdef\th@plain{\normalfont\itshape
%    \end{macrocode}
%    Then we define |\@begintheorem| and
%    |\@opargbegintheorem|. These two macros define how the
%    header of a theorem is typeset.  |\@opargbegintheorem| will
%    be called if a \textsf{theorem} environment with an optional
%    argument is encountered; otherwise, the header is constructed by
%    calling |\@begintheorem|.  If one of these macros is
%    executed, we are within a \textsf{trivlist} environment started by
%    |\@thm|.  So the theorem header is produced with an
%    |\item| command.
%
%    Instead of specifying the header font directly, all standard
%    theorem styles use the |\theorem@headerfont| macro to allow
%    customization.  The extra space (|\labelsep|) is necessary
%    because of problems in the \textsf{trivlist} environment.
%    \begin{macrocode}
  \def\@begintheorem##1##2{%
        \item[\hskip\labelsep \theorem@headerfont ##1\ ##2]}%
%    \end{macrocode}
%    The definition of |\@opargbegintheorem| is completely
%    analogous. The only difference is the fact that there exists a
%    third argument (which is the optional parameter of the
%    environment and contains additional information about the
%    theorem).  Customarily we enclose it in parentheses.
%    \begin{macrocode}
\def\@opargbegintheorem##1##2##3{%
   \item[\hskip\labelsep \theorem@headerfont ##1\ ##2\ (##3)]}}
%    \end{macrocode}
% \end{macro}
% \end{macro}
% \end{macro}
%
% \noindent
% We conclude with an |\endgroup| to restore the
% |\catcode| of |@|.
%    \begin{macrocode}
\endgroup
%</thp>
%    \end{macrocode}
%
%
% \subsubsection{The \texttt{break} style}
%
% This style option is stored in the file |thb.sty|.
% For the next two lines see the documentation for |\th@plain| on
% page~\pageref{dok:th@plain}.
%    \begin{macrocode}
%<*thb>
\begingroup \makeatletter
\@ifundefined{theorem@style}{% \iffalse meta-comment
%
% Copyright (C) 1993-2024
%
% The LaTeX Project and any individual authors listed elsewhere
% in this file.
%
% This file is part of the Standard LaTeX `Tools Bundle'.
% -------------------------------------------------------
%
% It may be distributed and/or modified under the
% conditions of the LaTeX Project Public License, either version 1.3c
% of this license or (at your option) any later version.
% The latest version of this license is in
%    https://www.latex-project.org/lppl.txt
% and version 1.3c or later is part of all distributions of LaTeX
% version 2005/12/01 or later.
%
% The list of all files belonging to the LaTeX `Tools Bundle' is
% given in the file `manifest.txt'.
%
% \fi
%
%%
%
%
\def\FMithmInfo{2023/07/05 v2.2c Theorem extension package (FMi)}
%
% \ProvidesFile{theorem.dtx}[\FMithmInfo]
%\iffalse   % this is a METACOMMENT !
%%
%% Package `theorem' to use with LaTeX2e
%% Copyright (C) 1989-1995 Frank Mittelbach, all rights reserved.
%<+package>\@ifundefined{theorem@style}{}{\endinput}
%<+package>\NeedsTeXFormat{LaTeX2e}[1995/06/01]
%<+package>\ProvidesPackage{theorem}
%<+package>         [\FMithmInfo]
%
% \fi
%
% \changes{v1.0}{87/05/07}{First draft.}
% \changes{v1.0a}{87/05/07}{`thistheoremstyle' added.}
% \changes{v1.1a}{87/07/29}{New concept: redefine the internal
%                          \protect\LaTeX{} commands `@xnthm' etc.}
% \changes{v1.1b}{87/07/31}{`theoremfont' used to specify font.}
% \changes{v1.2a}{88/01/08}{`@endtheorem' now globally defined.}
% \changes{v1.2b}{88/01/11}{`newtheorem' only allowed in the preamble.}
% \changes{v1.3a}{88/02/14}{`theoremfont' now a toks register.}
%
% \changes{v2.0a}{88/11/16}{New implementation and new documentation.}
% \changes{v2.0a}{88/11/16}{No layout changes but internal modifications
%                           old substyles won't work.}
% \changes{v2.0b}{89/01/30}{`global' used for all definitions.}
% \changes{v2.0c}{89/06/15}{Documentation translated to English.
%                            Thanks to LLL and Nina}
% \changes{v2.1a}{90/02/15}{Support for new font selection.}
% \changes{v2.2a}{94/02/03}{Cleanup for LaTeX2e}
%
% \DoNotIndex{\@endparenv,\@ifdefinable,\@ifnextchar,\@ifundefined}
% \DoNotIndex{\@namedef,\@nameuse,\@spaces,\@tempa,\@warning}
% \DoNotIndex{\ }
% \DoNotIndex{\begingroup,\bfseries}
% \DoNotIndex{\csname}
% \DoNotIndex{\def,\do,\docdate}
% \DoNotIndex{\endcsname,\endgroup,\endinput,\endtrivlist,\expandafter}
% \DoNotIndex{\filedate,\fileversion}
% \DoNotIndex{\gdef,\global}
% \DoNotIndex{\hbox,\hskip}
% \DoNotIndex{\ignorespaces,\input,\itshape,\item}
% \DoNotIndex{\labelsep,\let,\llap}
% \DoNotIndex{\makeatletter}
% \DoNotIndex{\noexpand}
% \DoNotIndex{\refstepcounter,\relax,\rlap}
% \DoNotIndex{\setlength,\slshape,\space,\string,\strut}
% \DoNotIndex{\the,\trivlist,\typeout}
% \DoNotIndex{\vbox}
% \DoNotIndex{\xdef}
%
% \def\deflabel#1{\ttfamily #1\hfill}
% \def\deflist#1{\begin{list}{}{\settowidth\labelwidth{\ttfamily #1}%
%                               \setlength\leftmargin\labelwidth
%                               \addtolength\leftmargin\labelsep
%                               \let\makelabel\deflabel}}
% \def\enddeflist{\end{list}}
% \def\envfont{\normalfont\ttfamily}
%
% \MakeShortVerb{\|}
%
% \newcommand{\TUB}{{\sl TUGboat\/}}
%
% \GetFileInfo{theorem.dtx}
% \title{An Extension of the
%        \LaTeX{} theorem environment\thanks{This file
%        has version number \fileversion, last
%        revised \filedate.}}
% \author{Frank Mittelbach}
%
% \MaintainedByLaTeXTeam{tools}
% \maketitle
%
% \begin{abstract}
%    The macros described in this paper yield an extension of the
%    \LaTeX{} theorem mechanism. It is designed to satisfy the
%    different requirements of various journals. Thus, the
%    layout of the ``theorems'' can be manipulated by determining a
%    ``style''. This article describes not only the use, but
%    also the definition, of the necessary macros.
% \end{abstract}
%
% \section*{Preface to version 2.2}
%
% For \LaTeXe{} this package did not need any fundamental changes. I
% only modified the messages generated so that theorem layout styles
% will show up with the |\listfiles| command and cleaned the section
% on the New Font Selection Scheme since this is now included in
% \LaTeX.
%
% \section*{Preface to version 2.1}
%
% This version is identical to 2.0g described in \TUB\ 10\#3 except
% for some internal defaults which are now set depending on the used
% font selection scheme.
%
% This was done to avoid unpleasant surprises if the new font selection
% scheme is in force. For further details see
% section~\ref{sec:newfonts} and \cite{bk:GMS94}.
%
% \section{Introduction}
%
% For our purposes here, ``theorems'' are labelled enunciations,
% often set off from the main text by extra space and a font change.
% Theorems, corollaries, conjectures, definitions, and remarks
% are all instances of ``theorems''.  The ``header'' of these structures
% is composed of a label (such as \textsc{Theorem} or \textsc{Remark})
% and a number which serializes an item in the sequence of items
% with the same label.
%
% Shortly after the introduction of \LaTeX{} at the Fachbereich
% Mathematik in Mainz, the desire to manipulate the layout of
% ``theorems'' arose.  In Mainz, the following two conventions came
% into general use:
% \begin{enumerate}
%   \item The number of the theorem is shown in the margin.
%   \item There is a line break at
%         the end of the theorem header.
% \end{enumerate}
%
% Additionally, some journals require different formats which depend on
% the ``sort of theorem'': e.g.\ often remarks and definitions are set
% in |\upshape|, while |\itshape| is employed for main theorems.
%
% Confronted with these requirements, a theorem environment was
% developed in Mainz which allows separate determination of the
% layout of the ``theorems sets'', comparable to |\pagestyle|.
%
%
% \section{The user interface}
% \subsection{Defining new theorem sets}
%
% \DescribeMacro\newtheorem
% As in the original \LaTeX{} version, the command |\newtheorem|
% defines a new ``theorem set'' or ``theorem-like structure''.
% Two required arguments name the new environment and give the text
% to be typeset with each instance of the new ``set'', while
% an optional argument determines how
% the ``set'' is enumerated:
% \begin{description}
%    \item[\ttfamily \bslash newtheorem\{foo\}\{bar\}]
%       The theorem set {\envfont foo} (whose name is \texttt{bar})
%       uses its own counter.
%    \item[\ttfamily \bslash newtheorem\{foo2\}{[foo]}\{bar2\}]
%       The theorem set {\envfont foo2} (printed name \texttt{bar2})
%       uses the same counter as the theorem set \texttt{foo}.
%    \item[\ttfamily \bslash newtheorem\{foo3\}\{bar3\}{[section]}]
%       The theorem set {\envfont foo3} (printed name \texttt{bar3}) is
%       enumerated within the counter \texttt{section}, i.e.\ with every
%       new |\section| the enumeration begins again with 1, and
%       the enumeration is composed from the section-number and the
%       theorem counter itself.
% \end{description}
%
% \DescribeMacro\theoremstyle
% Additionally, the command |\theoremstyle| can define the layout
% of various, or all, theorem sets.  It should be noted that any
% theorem set defined by |\newtheorem| is typeset in the
% |\theoremstyle| that is current at the time of the definition.
% Thus, the following
% \begin{verbatim}
%    \theoremstyle{break}        \newtheorem{Cor}{Corollary}
%    \theoremstyle{plain}        \newtheorem{Exa}{Example}[section]
%\end{verbatim}                             ^^A No blank at line start
% leads to the result that the set {\envfont Cor} is formatted in the
% style \texttt{break}, while the set {\envfont Exa} and all the
% following ones are formatted in the style \texttt{plain}, unless
% another |\theoremstyle| follows.  Since the definitions installed by
% |\newtheorem| are global, one also can limit |\theoremstyle| locally
% by grouping braces.
%
% \DescribeMacro\theorembodyfont
% The choice of the font for the theorem body is completely
% independent of the chosen |\theoremstyle|; this has proven to
% be very advantageous.  For example,
% \begin{verbatim}
%   {\theorembodyfont{\upshape}          \newtheorem{Rem}{Remark}}
%\end{verbatim}
% defines a theorem set {\envfont Rem}, which will be set in
% |\upshape| in the current layout (which in our example is
% \texttt{plain}).  As with |\theoremstyle|, the |\theorembodyfont|
% chosen is that current at the time of |\newtheorem|.  If
% |\theorembodyfont| is not specified or one defines
% |\theorembodyfont{}|, then the font used will be that defined by the
% |\theoremstyle|.
%
% \DescribeMacro\theoremheaderfont It is also possible to customize
% the font used for the theorem headers.  This is, however, a global
% declaration, and therefore there should be at most one
% |\theoremheaderfont| declaration in the preamble.\footnote{If it is
% actually necessary to have different header fonts, one has to define
% new theorem styles (substituting the desired font) or specify the
% information directly in the \texttt{\bslash newtheorem} declaration
% (the unclean variant).}
%
% \DescribeMacro\theorempreskipamount
% \DescribeMacro\theorempostskipamount Two additional parameters
% affect the vertical space around the theorem environments:
% |\theorempreskipamount| and |\theorempostskipamount| define,
% respectively, the spacing before and after such an environment.
% These parameters apply for all theorem sets and can be manipulated
% with the ordinary length macros.  They are rubber lengths
% (`\textsf{skips}'), and therefore can contain \texttt{plus} and
% \texttt{minus} parts.
%
% Since the definition of theorem sets should---most sensibly---be
% placed in the preamble, we only allow installation there.
% It is therefore possible to release the memory used here after
% |\begin{document}|, in order to make room for other
% applications.
%
%
% \subsection{Existing theorem styles}
%
% The following theorem styles exist to date:
% \begin{deflist}{marginbreak}
%    \item[plain]
%       This theorem style emulates the original \LaTeX{} definition,
%       except that additionally the parameters
%       |\theorem...skipamount| are used.
%    \item[break]
%       In this style, the theorem header is followed by a line
%       break.
%    \item[marginbreak]
%       The theorem number is set in the margin, and there is a line
%       break as in \texttt{break}.
%    \item[changebreak]
%       Like \texttt{break}, but with header number and text
%       interchanged.
%    \item[change]
%       Header number and text are interchanged, without a line break.
%    \item[margin]
%       The number is set in the left margin, without a line break.
% \end{deflist}
% All styles (except \texttt{plain}) select |\slshape| as the default
% |\theorembodyfont|.
%
%
% \subsection{Examples}
%
% Given the above theorem sets {\envfont Cor}, {\envfont Exa}
% and {\envfont Rem},
% suppose that the preamble also contains the declarations:
% \begin{verbatim}
%    \theoremstyle{marginbreak}   \newtheorem{Lem}[Cor]{Lemma}
%    \theoremstyle{change}
%    \theorembodyfont{\itshape}        \newtheorem{Def}[Cor]{Definition}
%
%    \theoremheaderfont{\scshape}
%\end{verbatim}
% Then the following are some typical examples of the typeset output
% resulting from their use.
%
% \begin{Cor}
%    This is a sentence typeset in the theorem environment {\envfont
%    Cor}.
% \end{Cor}
%
% \begin{Exa}
%    This is a sentence typeset in the theorem environment {\envfont
%    Exa}.
% \end{Exa}
%
% \begin{Rem}
%    This is a sentence typeset in the theorem environment {\envfont
%    Rem}.
% \end{Rem}
%
% \begin{Lem}[Ben User]
%    This is a sentence typeset in the theorem environment {\envfont
%    Lem}.
% \end{Lem}
%
% \begin{Def}[Very impressive Definition]
%    This is a sentence typeset in the theorem environment {\envfont
%    Def}.
% \end{Def}
%
% The last two examples show the effect of the optional argument to
% a theorem environment (it is the text typeset in parentheses).
%
%
% \section{Special Considerations}
% \label{sec:newfonts}
%
% Theoremheader and body are implemented as a unit. This means that
% the |\theoremheaderfont|  will inherit characteristics of the
% |\theorembodyfont| in \LaTeXe. Thus, if for example
% |\theorembodyfont| is |\itshape| and |\theoremheaderfont|
% is |\bfseries| the font selected for the header will have the
% characteristics `bold extended italic'. If this is not desired one
% should set the |\theoremheaderfont| to something like
% \begin{verbatim}
%      \theoremheaderfont{\normalfont\bfseries}
%\end{verbatim}
% i.e.\ supplying all necessary font information explicitly.
%
%
%
% \section{Acknowledgements}
%
% The publication of this set of macros was only possible with the
% help of Christina Busse (translating the manuscript into English),
% Joachim Pense (playing the r\^ole of typist), Chris Rowley
% (looking everything over) and many others providing useful
% suggestions.
%
% \MaybeStop{%
%    \begin{thebibliography}{9}
%       \bibitem{bk:GMS94} \textsc{M.~Goossens}, \textsc{F.~Mittelbach}
%         and \textsc{A.~Samarin}.
%         \newblock The \LaTeX{} Companion.
%         \newblock
%         Addison-Wesley, Reading, Massachusetts, 1994.
%       \bibitem{latex:source} \textsc{Lamport, Leslie.} \newblock
%                               \texttt{latex.tex}, version 2.09, date
%                               Feb. 1990.
%    \end{thebibliography}
%    \PrintIndex
%    \PrintChanges}
%
%
% \section{The documentation driver file}
%
% The next bit of code contains the documentation driver file for
% \TeX{}, i.e., the file that will produce the documentation you are
% currently reading. It will be extracted from this file by the
% \texttt{docstrip} program. Since it is the first code in the file
% one can alternatively process this file directly with \LaTeXe{} to
% obtain the documentation.
%
%    \begin{macrocode}
%<*driver>
\documentclass{ltxdoc}

\usepackage{theorem}

%<+driver>% The next few lines define theorem sets which are used
%<+driver>% in the example section of the documentation.

\theoremstyle{break}       \newtheorem{Cor}{Corollary}
\theoremstyle{plain}       \newtheorem{Exa}{Example}[section]
{\theorembodyfont{\upshape}\newtheorem{Rem}{Remark}}
\theoremstyle{marginbreak} \newtheorem{Lem}[Cor]{Lemma}
\theoremstyle{change}
\theorembodyfont{\itshape} \newtheorem{Def}[Cor]{Definition}

\theoremheaderfont{\scshape}

\RecordChanges

\begin{document}
   \DocInput{theorem.dtx}
\end{document}
%</driver>
%    \end{macrocode}
%
% \section{Definition of the Macros}
%
% If the file has been loaded before, we abort immediately. If not the
% package announces itself (this is actually done at the very top of
% the file---the way it is done isn't good style so don't copy it).
% \changes{v2.0e}{89/07/19}{Spaces removed from `typeout'.}
% \changes{v2.2b}{95/11/19}{Announce moved to top of file}
%    \begin{macrocode}
%<*package>
%\@ifundefined{theorem@style}{}{\endinput}
%\def\FMithmInfo{1995/11/19 v2.2b Theorem extension package (FMi)}
%\ProvidesPackage{theorem}[\FMithmInfo]
%    \end{macrocode}
%
%
% \subsection{Definition of theorem styles and fonts}
%
% All the definitions in this file are done globally to allow
% inputting this file inside a group.
%
% \begin{macro}{\theoremstyle}
%    Before a theorem style can be installed, the chosen style must be
%    known. For that reason, we must test to see that
%    |\th@|\meta{style} is known or, more precisely, that it is
%    different from |\relax|. If the style is not known then
%    |\th@plain| is used.
%    \begin{macrocode}
\gdef\theoremstyle#1{%
   \@ifundefined{th@#1}{\@warning
          {Unknown theoremstyle `#1'. Using `plain'}%
          \theorem@style{plain}}%
%    \end{macrocode}
%    We save the theorem style to be used in the token register
%    |\theorem@style|.
%    \begin{macrocode}
      {\theorem@style{#1}}%
%    \end{macrocode}
%    Now we ``evaluate'' the theorem style: this means, we call the
%    macro |\th@|\meta{style} which will activate the relevant
%    definitions which are contained in a separate file.  This is done
%    in a group to suppress changes to the current font.  This could
%    otherwise pose problems together with the new font selection
%    scheme\footnote{When I printed the original article using the new
%    font selection scheme I ended with a document with slanted
%    typefaces (text headings and all) simply because one of the
%    theorem styles used \texttt{\bslash sl} at toplevel.} if the
%    |\th@|\meta{style} is evaluated a second time.
% \changes{v2.1a}{90/02/15}{Group added around `th@..}
% \changes{v2.1a}{90/02/15}{Reseting of `..@begintheorem removed.}
%    \begin{macrocode}
      \begingroup
        \csname th@\the\theorem@style \endcsname
      \endgroup}
%    \end{macrocode}
% \end{macro}
%
%
% \begin{macro}{\@begintheorem}
% \changes{v2.1a}{90/02/15}{Relaxed at toplevel.}
% \begin{macro}{\@opargbegintheorem}
% \changes{v2.1a}{90/02/15}{Relaxed at toplevel.}
%    We reset |\@begintheorem| and |\@opargbegintheorem| to
%    |\relax| since these commands are no longer necessary at
%    toplevel. This will save a few tokens.
%    \begin{macrocode}
\global\let\@begintheorem\relax
\global\let\@opargbegintheorem\relax
%    \end{macrocode}
% \end{macro}
% \end{macro}
%
% \begin{macro}{\theorem@style}
%    Obviously the \textsf{token} register used above has to be
%    allocated. To assure the utmost compatibility with the
%    original \LaTeX{} definition, we set the default theorem style to
%    \texttt{plain}, which implements the usual \LaTeX{} convention.
% \changes{v2.0f}{89/07/20}{Unnecessary `global' removed.}
%    \begin{macrocode}
\newtoks\theorem@style
\global\theorem@style{plain}
%    \end{macrocode}
% \end{macro}
%
% \begin{macro}{\theorembodyfont}
% \begin{macro}{\theorem@bodyfont}
%    If the |\theorembodyfont| is set by the user then it should not
%    interact with the default font set in the theorem style. When the
%    new font selection is in force this may happen if, for example,
%    the default is |\itshape| and the new |\theorembodyfont| is
%    |\sffamily|. So we add a |\reset@font| command in front of the
%    user definition.
% \changes{v2.0f}{89/07/20}{Unnecessary `global' removed.}
% \changes{v2.0g}{89/09/19}{`theoremfont to `theorembodyfont.}
% \changes{v2.1c}{92/06/22}{`reset@font added and logic changed.}
% \changes{v2.1d}{92/10/28}{Reset to empty if arg is empty}
%    \begin{macrocode}
\gdef\theorembodyfont#1{%
%    \end{macrocode}
%    We check if the argument supplied is empty and if so put nothing
%    into the |\theorem@bodyfont| token register to allow for
%    |\theorembodyfont{}| as a mean of using the default of the
%    current |\theoremstyle|.
%    \begin{macrocode}
   \def\@tempa{#1}%
   \ifx\@tempa\@empty
    \theorem@bodyfont{}%
   \else
    \theorem@bodyfont{\reset@font#1}%
   \fi
   }
\newtoks\theorem@bodyfont
\global\theorem@bodyfont{}
%    \end{macrocode}
% \end{macro}
% \end{macro}
%
%
% \begin{macro}{\theoremheaderfont}
% \changes{v2.0g}{89/09/19}{Macro added.}
%    The font for the theorem headers is handled differently because
%    this definition applies to all theorem styles.
%    \begin{macrocode}
\gdef\theoremheaderfont#1{\gdef\theorem@headerfont{#1}%
%    \end{macrocode}
%    After using the macro once it is redefined to produce an error
%    message.
%    \begin{macrocode}
       \gdef\theoremheaderfont##1{%
        \typeout{\string\theoremheaderfont\space should be used
                 only once.}}}
%    \end{macrocode}
% \end{macro}
%
% \begin{macro}{\theorem@headerfont}
% \changes{v2.0g}{89/09/19}{Macro added.}
%    To set the |\theorem@headerfont| default we first test if
%    the new fontselection scheme is in force.
% \changes{v2.1a}{90/02/15}{Support for new font selection.}
%    \begin{macrocode}
\ifx\upshape\undefined
%    \end{macrocode}
%    If not we define it to expand into |\bfseries|. We don't use
%    |\let| just in case a following style option redefines this
%    macro.
%    \begin{macrocode}
\gdef\theorem@headerfont{\bfseries}
%    \end{macrocode}
%    Otherwise we reset the current shape before calling |\bfseries|.
%    \begin{macrocode}
\else \gdef\theorem@headerfont{\normalfont\bfseries}\fi
%    \end{macrocode}
% \end{macro}
%
% \begin{macro}{\th@plain}
% \begin{macro}{\th@break}
% \begin{macro}{\th@marginbreak}
% \begin{macro}{\th@changebreak}
% \begin{macro}{\th@change}
% \begin{macro}{\th@margin}
%    The different styles are defined in macros such as
%    |\th@plain|.  Since memory space is precious in
%    ``non-Big-versions'', we have to avoid offering too many unused
%    definitions.  Therefore we define these styles in separate files
%    that can be loaded on demand.  Thus the commands themselves only
%    load these files. We use |\@input@|, a \LaTeXe{} internal command
%    that ensures that the file will be listed with |\listfiles|.
% \changes{v2.2a}{94/02/02}{Use `@input@ to load theorem layout files}
%    \begin{macrocode}
\gdef\th@plain{\@input@{thp.sty}}
\gdef\th@break{\@input@{thb.sty}}
\gdef\th@marginbreak{\@input@{thmb.sty}}
\gdef\th@changebreak{\@input@{thcb.sty}}
\gdef\th@change{\@input@{thc.sty}}
\gdef\th@margin{\@input@{thm.sty}}
%    \end{macrocode}
%    This list will be expanded when new styles become
%    available.  For testing,  just append new theorem substyles
%    as document options.
% \end{macro}
% \end{macro}
% \end{macro}
% \end{macro}
% \end{macro}
% \end{macro}
%
%
%
% \subsection{Definition of a new theorem set}
%
% As already pointed out, a new theorem environment can be defined in
% three different ways:
% \begin{verbatim}
%    \newtheorem{Lem}{Lemma}
%    \newtheorem{Lem}{Lemma}[section]
%    \newtheorem{Lem}[Theorem]{Lemma}
%\end{verbatim}
% The function of the macro |\newtheorem| is to recognize these cases
% and then to branch into one of the three macros |\@ynthm|, |\@xnthm|
% or |\@othm|.  This mechanism is adopted unchanged from
% \cite{latex:source}; the essential point here is that, for example,
% in the second case, the arguments \texttt{Lem}, \texttt{Lemma} and
% \texttt{section} are passed over to the macro |\@xnthm|.
%
% We inspect this case first because the others present fewer problems,
% and thus are easily derived from this one.
%
% \begin{macro}{\@xnthm}
%    For our example arguments, the macro |\@xnthm| must fulfill
%    the following:
%    \begin{itemize}
%       \item
%          Define a new \LaTeX-counter `Lem'
%       \item
%          reset this counter within a |\section|
%       \item
%          define the macro |\theLem|
%       \item
%          define the environment macros |\Lem| and
%          |\endLem| using the current
%          |\theoremstyle| and |\theorem@bodyfont|.
%    \end{itemize}
%    Obviously, all this should happen only if the first argument
%    of |\@xnthm| (i.e.\ \texttt{Lem} in our example) is chosen so
%    as not to conflict with any previously defined commands or
%    environments.  This test is performed by the \LaTeX{} macro
%    |\@ifdefinable|.
%    \begin{macrocode}
\gdef\@xnthm#1#2[#3]{\expandafter\@ifdefinable\csname #1\endcsname
%    \end{macrocode}
%    Therefore, the first argument of |\@ifdefinable| is the
%    expansion (in the example, |\Lem|) of
%    |\csname||#1\endcsname|.  The second argument is
%    executed only if the test has been completed successfully.
%    \begin{macrocode}
   {%
%    \end{macrocode}
%    Now we define the new counter. The names of the \LaTeX{}
%    macros employed should speak for themselves:
% \changes{v2.2b}{1995/10/07}{Use \cs{@newctr}}
% \changes{v2.2c}{1995/11/23}{Use \cs{@newctr} properly.}
%    \begin{macrocode}
    \@definecounter{#1}\@newctr{#1}[#3]%
%    \end{macrocode}
%    Using |\@newctr| will give a proper error message if the counter
%    in |#3| is not defined.
%    In defining `|\theLem|' we must generate the desired macro
%    name by use of |\expandafter| and |\csname|.
%    \begin{macrocode}
    \expandafter\xdef\csname the#1\endcsname
%    \end{macrocode}
%    An |\xdef| is used in order to make the definition global,
%    and to ensure that it contains the replacement texts of
%    |\@thmcountersep| and |\@thmcounter|.\footnote{These
%    two macros can be defined by the document style.  Their default
%    values produce a `\texttt{.}' as separation and an arabic
%    representation of the number.} However, not everything should be
%    expanded. For example, it saves space to use |\thesection|
%    instead of its---at times---lengthy expansion.
%    \begin{macrocode}
      {\expandafter \noexpand \csname the#3\endcsname
       \@thmcountersep \@thmcounter{#1}}%
%    \end{macrocode}
%    Thus with the defaults of \LaTeX{}, |\theLem| would be
%    replaced by the command sequence |\thesection.\arabic{Lem}|.
%
%    We will now look at the definition of the macro which is executed
%    at the beginning of the actual environment (in our example this
%    macro is |\Lem|). It should be noted that we use an
%    ``|\expandafter| trick'' to expand only certain parts of the
%    replacement text at the time of the definition.
%    \begin{macrocode}
    \def\@tempa{\global\@namedef{#1}}%
    \expandafter \@tempa \expandafter{%
%    \end{macrocode}
%    First, the macro that contains the current definitions of
%    |\@begintheorem| and |\@opargtheorem| should be called
%    up.  The name of this macro---as is already known---has the form
%    |\th@|\meta{theorem\ style}; therefore, it must be called by
%    \begin{macrocode}
      \csname th@\the \theorem@style
%    \end{macrocode}
%    In addition the default theorem font should be changeable, i.e.\
%    we have to insert the contents of |\theorem@bodyfont|. For
%    that reason, we expand even further, beyond |\endcsname|,
%    and thus insert the contents of the \textsf{token} register
%    |\theorem@bodyfont| in the replacement text.
%    \begin{macrocode}
            \expandafter \endcsname \the \theorem@bodyfont
%    \end{macrocode}
%    Now it is time to call the macro |\@thm| which takes over
%    the further processing.  It has two arguments: the current
%    counter name (in our example, \texttt{Lem}), and the text of the
%    label (in our example, \texttt{Lemma}).
%    \begin{macrocode}
     \@thm{#1}{#2}}%
%    \end{macrocode}
%    With this, the `sub-definition' is complete.  The macro
%    |\@endtheorem| ends a theorem environment and is, so far, nothing
%    but an |\endtrivlist|.  (Hence it is defined globally, and not
%    within the theorem styles.\footnote{This has to be changed as
%    soon as theorem styles that change \texttt{\bslash @endtheorem}
%    exist.  In such a case, all existing styles must be changed as
%    well since they will have to reset the macro.}) Therefore, we can
%    set it equivalent to the macro that ends the theorem set (in our
%    example, |\endLem|).  However, if some day theorem styles exist
%    that do change |\@endtheorem|, we would have to use the
%    commented-out line instead.
%    \begin{macrocode}
    \global \expandafter \let \csname end#1\endcsname \@endtheorem
%   \global\@namedef{end#1}{\@endtheorem}%
%    \end{macrocode}
%    With these commands all the required definitions are employed,
%    unless the test |\@ifdefinable| has failed.  Therefore, we
%    end the second argument of this macro and with it the definition
%    of |\@xnthm|.
%    \begin{macrocode}
   }}
%    \end{macrocode}
% \end{macro}
%
% \begin{macro}{\@ynthm}
%    The definition of |\@ynthm| is completely analogous. In
%    this case the new counter that is defined is not reset within
%    another counter; thus the definition of |\the...|
%    is simplified:
%    \begin{macrocode}
\gdef\@ynthm#1#2{\expandafter\@ifdefinable\csname #1\endcsname
   {\@definecounter{#1}%
    \expandafter\xdef\csname the#1\endcsname{\@thmcounter{#1}}%
%    \end{macrocode}
%    The rest of the definition corresponds literally to that of
%    |\@xnthm|:
%    \begin{macrocode}
    \def\@tempa{\global\@namedef{#1}}\expandafter \@tempa
     \expandafter{\csname th@\the \theorem@style \expandafter
     \endcsname \the\theorem@bodyfont \@thm{#1}{#2}}%
    \global \expandafter \let \csname end#1\endcsname \@endtheorem}}
%    \end{macrocode}
% \end{macro}
%
% \begin{macro}{\@othm}
%    The definition of |\@othm| does not contain anything new.
%    \begin{macrocode}
\gdef\@othm#1[#2]#3{%
%    \end{macrocode}
% \changes{v2.2b}{95/10/07}{Build in check for counter defined}
%    We do not define a new counter but instead use one that has
%    already been defined.  Thus the only definition we need is that
%    of this pseudo-counter (i.e.\ |\the|\meta{env.\ name}).
%    First we check if |#2| corresponds to a known counter name.
%    \begin{macrocode}
  \expandafter\ifx\csname c@#2\endcsname\relax
   \@nocounterr{#2}%
  \else
   \expandafter\@ifdefinable\csname #1\endcsname
   {\expandafter \xdef \csname the#1\endcsname
     {\expandafter \noexpand \csname the#2\endcsname}%
%    \end{macrocode}
%    All other parts of the definition can be adopted from
%    |\@xnthm|. We have to remember, though, that in this case
%    the name of the current counter and the theorem label have moved
%    to the second and third arguments.
%    \begin{macrocode}
    \def\@tempa{\global\@namedef{#1}}\expandafter \@tempa
     \expandafter{\csname th@\the \theorem@style \expandafter
     \endcsname \the\theorem@bodyfont \@thm{#2}{#3}}%
    \global \expandafter \let \csname end#1\endcsname \@endtheorem}%
  \fi}
%    \end{macrocode}
% \end{macro}
%
% \subsection{Macros that are employed in a {\envfont theorem}
%    environment}
%
% \begin{macro}{\@thm}
%    The macro |\@thm| has to increase the current counter.  Then,
%    depending on whether the environment has (or does not have) an
%    optional argument, it has to branch into either |\@begintheorem|
%    or |\@opargtheorem|.
%    \begin{macrocode}
\gdef\@thm#1#2{\refstepcounter{#1}%
%    \end{macrocode}
%    Now we start a {\envfont trivlist} environment, and give
%    |\@topsep| and |\@topsepadd| the values of the \textsf{skip}
%    registers |\theorempreskipamount| and |\theorempostskipamount|.
%    The value in |\@topsep| is the vertical space that is inserted by
%    the first (and only) |\item| in our |\trivlist| whilst
%    |\@topsepadd| is inserted by |\@endparenv| at the end of that
%    {\envfont trivlist} environment.  By using these registers, we
%    obtain the desired space around a {\envfont theorem} environment.
%    \begin{macrocode}
   \trivlist
   \@topsep \theorempreskipamount               % used by first \item
   \@topsepadd \theorempostskipamount           % used by \@endparenv
%    \end{macrocode}
%    Now we have to test whether an optional argument has been given.
% \changes{v2.0d}{89/07/18}{Forgotten percent added.}
%    \begin{macrocode}
   \@ifnextchar [%
%    \end{macrocode}
%    If there is an optional argument, we will call |\@ythm|, and
%    move the arguments read back into the input stream.
%    \begin{macrocode}
   {\@ythm{#1}{#2}}%
%    \end{macrocode}
%    If not, we call |\@begintheorem|. Its first argument is the
%    name of the theorem set (hence the second argument of
%    |\@thm|). Its second argument is the macro that produces the
%    current number.
%    \begin{macrocode}
   {\@begintheorem{#2}{\csname the#1\endcsname}\ignorespaces}}
%    \end{macrocode}
% \end{macro}
%
% \begin{macro}{\@xthm}
% \begin{macro}{\@ythm}
%    Both these macros were originally called by |\@thm|.  We do
%    not need |\@xthm| anymore, hence we reset it to
%    |\relax|. The definition of |\@ythm| has not changed at
%    all from its definition in \LaTeX. In order to make the macros
%    easier to understand, we will
%    nevertheless present it (commented out).
%    \begin{macrocode}
\global\let\@xthm\relax
% \def\@ythm#1#2[#3]{\@opargbegintheorem{#2}{\csname
%           the#1\endcsname}{#3}\ignorespaces}
%</package>
%    \end{macrocode}
%    The primitive |\ignorespaces| in |\@ythm| and
%    |\@thm| is needed to remove the spaces between the
%    |\begin{...}| and the actual text.
% \end{macro}
% \end{macro}
%
%
% \subsection{Definition of the theorem substyles}
%
% As already pointed out, the theorem substyles, defined below, are
% only loaded when necessary.  Note that all these substyles, except
% \texttt{plain}, have |\slshape| as the default body font.
%
% \subsubsection{The \texttt{plain} style}
%
% \label{dok:th@plain} As the following macros use |@|, we have
% to locally set the |\catcode| of this symbol to ``letter''.
% This happens within a group, so that we do not have to worry about
% which |\catcode| that symbol had before.
%    \begin{macrocode}
%<*thp>
\begingroup \makeatletter
%    \end{macrocode}
% Since we are now within a group, we must make all definitions
% globally.  First we make sure that \texttt{theorem.sty} is loaded.
% This will allow us to use this file as a document style option
% without having to call \texttt{theorem} itself as an option.  At the
% same time, we assure that at least version 2 is loaded, since
% |\theorem@style| was not defined in earlier versions.
%    \begin{macrocode}
\@ifundefined{theorem@style}{% \iffalse meta-comment
%
% Copyright (C) 1993-2024
%
% The LaTeX Project and any individual authors listed elsewhere
% in this file.
%
% This file is part of the Standard LaTeX `Tools Bundle'.
% -------------------------------------------------------
%
% It may be distributed and/or modified under the
% conditions of the LaTeX Project Public License, either version 1.3c
% of this license or (at your option) any later version.
% The latest version of this license is in
%    https://www.latex-project.org/lppl.txt
% and version 1.3c or later is part of all distributions of LaTeX
% version 2005/12/01 or later.
%
% The list of all files belonging to the LaTeX `Tools Bundle' is
% given in the file `manifest.txt'.
%
% \fi
%
%%
%
%
\def\FMithmInfo{2023/07/05 v2.2c Theorem extension package (FMi)}
%
% \ProvidesFile{theorem.dtx}[\FMithmInfo]
%\iffalse   % this is a METACOMMENT !
%%
%% Package `theorem' to use with LaTeX2e
%% Copyright (C) 1989-1995 Frank Mittelbach, all rights reserved.
%<+package>\@ifundefined{theorem@style}{}{\endinput}
%<+package>\NeedsTeXFormat{LaTeX2e}[1995/06/01]
%<+package>\ProvidesPackage{theorem}
%<+package>         [\FMithmInfo]
%
% \fi
%
% \changes{v1.0}{87/05/07}{First draft.}
% \changes{v1.0a}{87/05/07}{`thistheoremstyle' added.}
% \changes{v1.1a}{87/07/29}{New concept: redefine the internal
%                          \protect\LaTeX{} commands `@xnthm' etc.}
% \changes{v1.1b}{87/07/31}{`theoremfont' used to specify font.}
% \changes{v1.2a}{88/01/08}{`@endtheorem' now globally defined.}
% \changes{v1.2b}{88/01/11}{`newtheorem' only allowed in the preamble.}
% \changes{v1.3a}{88/02/14}{`theoremfont' now a toks register.}
%
% \changes{v2.0a}{88/11/16}{New implementation and new documentation.}
% \changes{v2.0a}{88/11/16}{No layout changes but internal modifications
%                           old substyles won't work.}
% \changes{v2.0b}{89/01/30}{`global' used for all definitions.}
% \changes{v2.0c}{89/06/15}{Documentation translated to English.
%                            Thanks to LLL and Nina}
% \changes{v2.1a}{90/02/15}{Support for new font selection.}
% \changes{v2.2a}{94/02/03}{Cleanup for LaTeX2e}
%
% \DoNotIndex{\@endparenv,\@ifdefinable,\@ifnextchar,\@ifundefined}
% \DoNotIndex{\@namedef,\@nameuse,\@spaces,\@tempa,\@warning}
% \DoNotIndex{\ }
% \DoNotIndex{\begingroup,\bfseries}
% \DoNotIndex{\csname}
% \DoNotIndex{\def,\do,\docdate}
% \DoNotIndex{\endcsname,\endgroup,\endinput,\endtrivlist,\expandafter}
% \DoNotIndex{\filedate,\fileversion}
% \DoNotIndex{\gdef,\global}
% \DoNotIndex{\hbox,\hskip}
% \DoNotIndex{\ignorespaces,\input,\itshape,\item}
% \DoNotIndex{\labelsep,\let,\llap}
% \DoNotIndex{\makeatletter}
% \DoNotIndex{\noexpand}
% \DoNotIndex{\refstepcounter,\relax,\rlap}
% \DoNotIndex{\setlength,\slshape,\space,\string,\strut}
% \DoNotIndex{\the,\trivlist,\typeout}
% \DoNotIndex{\vbox}
% \DoNotIndex{\xdef}
%
% \def\deflabel#1{\ttfamily #1\hfill}
% \def\deflist#1{\begin{list}{}{\settowidth\labelwidth{\ttfamily #1}%
%                               \setlength\leftmargin\labelwidth
%                               \addtolength\leftmargin\labelsep
%                               \let\makelabel\deflabel}}
% \def\enddeflist{\end{list}}
% \def\envfont{\normalfont\ttfamily}
%
% \MakeShortVerb{\|}
%
% \newcommand{\TUB}{{\sl TUGboat\/}}
%
% \GetFileInfo{theorem.dtx}
% \title{An Extension of the
%        \LaTeX{} theorem environment\thanks{This file
%        has version number \fileversion, last
%        revised \filedate.}}
% \author{Frank Mittelbach}
%
% \MaintainedByLaTeXTeam{tools}
% \maketitle
%
% \begin{abstract}
%    The macros described in this paper yield an extension of the
%    \LaTeX{} theorem mechanism. It is designed to satisfy the
%    different requirements of various journals. Thus, the
%    layout of the ``theorems'' can be manipulated by determining a
%    ``style''. This article describes not only the use, but
%    also the definition, of the necessary macros.
% \end{abstract}
%
% \section*{Preface to version 2.2}
%
% For \LaTeXe{} this package did not need any fundamental changes. I
% only modified the messages generated so that theorem layout styles
% will show up with the |\listfiles| command and cleaned the section
% on the New Font Selection Scheme since this is now included in
% \LaTeX.
%
% \section*{Preface to version 2.1}
%
% This version is identical to 2.0g described in \TUB\ 10\#3 except
% for some internal defaults which are now set depending on the used
% font selection scheme.
%
% This was done to avoid unpleasant surprises if the new font selection
% scheme is in force. For further details see
% section~\ref{sec:newfonts} and \cite{bk:GMS94}.
%
% \section{Introduction}
%
% For our purposes here, ``theorems'' are labelled enunciations,
% often set off from the main text by extra space and a font change.
% Theorems, corollaries, conjectures, definitions, and remarks
% are all instances of ``theorems''.  The ``header'' of these structures
% is composed of a label (such as \textsc{Theorem} or \textsc{Remark})
% and a number which serializes an item in the sequence of items
% with the same label.
%
% Shortly after the introduction of \LaTeX{} at the Fachbereich
% Mathematik in Mainz, the desire to manipulate the layout of
% ``theorems'' arose.  In Mainz, the following two conventions came
% into general use:
% \begin{enumerate}
%   \item The number of the theorem is shown in the margin.
%   \item There is a line break at
%         the end of the theorem header.
% \end{enumerate}
%
% Additionally, some journals require different formats which depend on
% the ``sort of theorem'': e.g.\ often remarks and definitions are set
% in |\upshape|, while |\itshape| is employed for main theorems.
%
% Confronted with these requirements, a theorem environment was
% developed in Mainz which allows separate determination of the
% layout of the ``theorems sets'', comparable to |\pagestyle|.
%
%
% \section{The user interface}
% \subsection{Defining new theorem sets}
%
% \DescribeMacro\newtheorem
% As in the original \LaTeX{} version, the command |\newtheorem|
% defines a new ``theorem set'' or ``theorem-like structure''.
% Two required arguments name the new environment and give the text
% to be typeset with each instance of the new ``set'', while
% an optional argument determines how
% the ``set'' is enumerated:
% \begin{description}
%    \item[\ttfamily \bslash newtheorem\{foo\}\{bar\}]
%       The theorem set {\envfont foo} (whose name is \texttt{bar})
%       uses its own counter.
%    \item[\ttfamily \bslash newtheorem\{foo2\}{[foo]}\{bar2\}]
%       The theorem set {\envfont foo2} (printed name \texttt{bar2})
%       uses the same counter as the theorem set \texttt{foo}.
%    \item[\ttfamily \bslash newtheorem\{foo3\}\{bar3\}{[section]}]
%       The theorem set {\envfont foo3} (printed name \texttt{bar3}) is
%       enumerated within the counter \texttt{section}, i.e.\ with every
%       new |\section| the enumeration begins again with 1, and
%       the enumeration is composed from the section-number and the
%       theorem counter itself.
% \end{description}
%
% \DescribeMacro\theoremstyle
% Additionally, the command |\theoremstyle| can define the layout
% of various, or all, theorem sets.  It should be noted that any
% theorem set defined by |\newtheorem| is typeset in the
% |\theoremstyle| that is current at the time of the definition.
% Thus, the following
% \begin{verbatim}
%    \theoremstyle{break}        \newtheorem{Cor}{Corollary}
%    \theoremstyle{plain}        \newtheorem{Exa}{Example}[section]
%\end{verbatim}                             ^^A No blank at line start
% leads to the result that the set {\envfont Cor} is formatted in the
% style \texttt{break}, while the set {\envfont Exa} and all the
% following ones are formatted in the style \texttt{plain}, unless
% another |\theoremstyle| follows.  Since the definitions installed by
% |\newtheorem| are global, one also can limit |\theoremstyle| locally
% by grouping braces.
%
% \DescribeMacro\theorembodyfont
% The choice of the font for the theorem body is completely
% independent of the chosen |\theoremstyle|; this has proven to
% be very advantageous.  For example,
% \begin{verbatim}
%   {\theorembodyfont{\upshape}          \newtheorem{Rem}{Remark}}
%\end{verbatim}
% defines a theorem set {\envfont Rem}, which will be set in
% |\upshape| in the current layout (which in our example is
% \texttt{plain}).  As with |\theoremstyle|, the |\theorembodyfont|
% chosen is that current at the time of |\newtheorem|.  If
% |\theorembodyfont| is not specified or one defines
% |\theorembodyfont{}|, then the font used will be that defined by the
% |\theoremstyle|.
%
% \DescribeMacro\theoremheaderfont It is also possible to customize
% the font used for the theorem headers.  This is, however, a global
% declaration, and therefore there should be at most one
% |\theoremheaderfont| declaration in the preamble.\footnote{If it is
% actually necessary to have different header fonts, one has to define
% new theorem styles (substituting the desired font) or specify the
% information directly in the \texttt{\bslash newtheorem} declaration
% (the unclean variant).}
%
% \DescribeMacro\theorempreskipamount
% \DescribeMacro\theorempostskipamount Two additional parameters
% affect the vertical space around the theorem environments:
% |\theorempreskipamount| and |\theorempostskipamount| define,
% respectively, the spacing before and after such an environment.
% These parameters apply for all theorem sets and can be manipulated
% with the ordinary length macros.  They are rubber lengths
% (`\textsf{skips}'), and therefore can contain \texttt{plus} and
% \texttt{minus} parts.
%
% Since the definition of theorem sets should---most sensibly---be
% placed in the preamble, we only allow installation there.
% It is therefore possible to release the memory used here after
% |\begin{document}|, in order to make room for other
% applications.
%
%
% \subsection{Existing theorem styles}
%
% The following theorem styles exist to date:
% \begin{deflist}{marginbreak}
%    \item[plain]
%       This theorem style emulates the original \LaTeX{} definition,
%       except that additionally the parameters
%       |\theorem...skipamount| are used.
%    \item[break]
%       In this style, the theorem header is followed by a line
%       break.
%    \item[marginbreak]
%       The theorem number is set in the margin, and there is a line
%       break as in \texttt{break}.
%    \item[changebreak]
%       Like \texttt{break}, but with header number and text
%       interchanged.
%    \item[change]
%       Header number and text are interchanged, without a line break.
%    \item[margin]
%       The number is set in the left margin, without a line break.
% \end{deflist}
% All styles (except \texttt{plain}) select |\slshape| as the default
% |\theorembodyfont|.
%
%
% \subsection{Examples}
%
% Given the above theorem sets {\envfont Cor}, {\envfont Exa}
% and {\envfont Rem},
% suppose that the preamble also contains the declarations:
% \begin{verbatim}
%    \theoremstyle{marginbreak}   \newtheorem{Lem}[Cor]{Lemma}
%    \theoremstyle{change}
%    \theorembodyfont{\itshape}        \newtheorem{Def}[Cor]{Definition}
%
%    \theoremheaderfont{\scshape}
%\end{verbatim}
% Then the following are some typical examples of the typeset output
% resulting from their use.
%
% \begin{Cor}
%    This is a sentence typeset in the theorem environment {\envfont
%    Cor}.
% \end{Cor}
%
% \begin{Exa}
%    This is a sentence typeset in the theorem environment {\envfont
%    Exa}.
% \end{Exa}
%
% \begin{Rem}
%    This is a sentence typeset in the theorem environment {\envfont
%    Rem}.
% \end{Rem}
%
% \begin{Lem}[Ben User]
%    This is a sentence typeset in the theorem environment {\envfont
%    Lem}.
% \end{Lem}
%
% \begin{Def}[Very impressive Definition]
%    This is a sentence typeset in the theorem environment {\envfont
%    Def}.
% \end{Def}
%
% The last two examples show the effect of the optional argument to
% a theorem environment (it is the text typeset in parentheses).
%
%
% \section{Special Considerations}
% \label{sec:newfonts}
%
% Theoremheader and body are implemented as a unit. This means that
% the |\theoremheaderfont|  will inherit characteristics of the
% |\theorembodyfont| in \LaTeXe. Thus, if for example
% |\theorembodyfont| is |\itshape| and |\theoremheaderfont|
% is |\bfseries| the font selected for the header will have the
% characteristics `bold extended italic'. If this is not desired one
% should set the |\theoremheaderfont| to something like
% \begin{verbatim}
%      \theoremheaderfont{\normalfont\bfseries}
%\end{verbatim}
% i.e.\ supplying all necessary font information explicitly.
%
%
%
% \section{Acknowledgements}
%
% The publication of this set of macros was only possible with the
% help of Christina Busse (translating the manuscript into English),
% Joachim Pense (playing the r\^ole of typist), Chris Rowley
% (looking everything over) and many others providing useful
% suggestions.
%
% \MaybeStop{%
%    \begin{thebibliography}{9}
%       \bibitem{bk:GMS94} \textsc{M.~Goossens}, \textsc{F.~Mittelbach}
%         and \textsc{A.~Samarin}.
%         \newblock The \LaTeX{} Companion.
%         \newblock
%         Addison-Wesley, Reading, Massachusetts, 1994.
%       \bibitem{latex:source} \textsc{Lamport, Leslie.} \newblock
%                               \texttt{latex.tex}, version 2.09, date
%                               Feb. 1990.
%    \end{thebibliography}
%    \PrintIndex
%    \PrintChanges}
%
%
% \section{The documentation driver file}
%
% The next bit of code contains the documentation driver file for
% \TeX{}, i.e., the file that will produce the documentation you are
% currently reading. It will be extracted from this file by the
% \texttt{docstrip} program. Since it is the first code in the file
% one can alternatively process this file directly with \LaTeXe{} to
% obtain the documentation.
%
%    \begin{macrocode}
%<*driver>
\documentclass{ltxdoc}

\usepackage{theorem}

%<+driver>% The next few lines define theorem sets which are used
%<+driver>% in the example section of the documentation.

\theoremstyle{break}       \newtheorem{Cor}{Corollary}
\theoremstyle{plain}       \newtheorem{Exa}{Example}[section]
{\theorembodyfont{\upshape}\newtheorem{Rem}{Remark}}
\theoremstyle{marginbreak} \newtheorem{Lem}[Cor]{Lemma}
\theoremstyle{change}
\theorembodyfont{\itshape} \newtheorem{Def}[Cor]{Definition}

\theoremheaderfont{\scshape}

\RecordChanges

\begin{document}
   \DocInput{theorem.dtx}
\end{document}
%</driver>
%    \end{macrocode}
%
% \section{Definition of the Macros}
%
% If the file has been loaded before, we abort immediately. If not the
% package announces itself (this is actually done at the very top of
% the file---the way it is done isn't good style so don't copy it).
% \changes{v2.0e}{89/07/19}{Spaces removed from `typeout'.}
% \changes{v2.2b}{95/11/19}{Announce moved to top of file}
%    \begin{macrocode}
%<*package>
%\@ifundefined{theorem@style}{}{\endinput}
%\def\FMithmInfo{1995/11/19 v2.2b Theorem extension package (FMi)}
%\ProvidesPackage{theorem}[\FMithmInfo]
%    \end{macrocode}
%
%
% \subsection{Definition of theorem styles and fonts}
%
% All the definitions in this file are done globally to allow
% inputting this file inside a group.
%
% \begin{macro}{\theoremstyle}
%    Before a theorem style can be installed, the chosen style must be
%    known. For that reason, we must test to see that
%    |\th@|\meta{style} is known or, more precisely, that it is
%    different from |\relax|. If the style is not known then
%    |\th@plain| is used.
%    \begin{macrocode}
\gdef\theoremstyle#1{%
   \@ifundefined{th@#1}{\@warning
          {Unknown theoremstyle `#1'. Using `plain'}%
          \theorem@style{plain}}%
%    \end{macrocode}
%    We save the theorem style to be used in the token register
%    |\theorem@style|.
%    \begin{macrocode}
      {\theorem@style{#1}}%
%    \end{macrocode}
%    Now we ``evaluate'' the theorem style: this means, we call the
%    macro |\th@|\meta{style} which will activate the relevant
%    definitions which are contained in a separate file.  This is done
%    in a group to suppress changes to the current font.  This could
%    otherwise pose problems together with the new font selection
%    scheme\footnote{When I printed the original article using the new
%    font selection scheme I ended with a document with slanted
%    typefaces (text headings and all) simply because one of the
%    theorem styles used \texttt{\bslash sl} at toplevel.} if the
%    |\th@|\meta{style} is evaluated a second time.
% \changes{v2.1a}{90/02/15}{Group added around `th@..}
% \changes{v2.1a}{90/02/15}{Reseting of `..@begintheorem removed.}
%    \begin{macrocode}
      \begingroup
        \csname th@\the\theorem@style \endcsname
      \endgroup}
%    \end{macrocode}
% \end{macro}
%
%
% \begin{macro}{\@begintheorem}
% \changes{v2.1a}{90/02/15}{Relaxed at toplevel.}
% \begin{macro}{\@opargbegintheorem}
% \changes{v2.1a}{90/02/15}{Relaxed at toplevel.}
%    We reset |\@begintheorem| and |\@opargbegintheorem| to
%    |\relax| since these commands are no longer necessary at
%    toplevel. This will save a few tokens.
%    \begin{macrocode}
\global\let\@begintheorem\relax
\global\let\@opargbegintheorem\relax
%    \end{macrocode}
% \end{macro}
% \end{macro}
%
% \begin{macro}{\theorem@style}
%    Obviously the \textsf{token} register used above has to be
%    allocated. To assure the utmost compatibility with the
%    original \LaTeX{} definition, we set the default theorem style to
%    \texttt{plain}, which implements the usual \LaTeX{} convention.
% \changes{v2.0f}{89/07/20}{Unnecessary `global' removed.}
%    \begin{macrocode}
\newtoks\theorem@style
\global\theorem@style{plain}
%    \end{macrocode}
% \end{macro}
%
% \begin{macro}{\theorembodyfont}
% \begin{macro}{\theorem@bodyfont}
%    If the |\theorembodyfont| is set by the user then it should not
%    interact with the default font set in the theorem style. When the
%    new font selection is in force this may happen if, for example,
%    the default is |\itshape| and the new |\theorembodyfont| is
%    |\sffamily|. So we add a |\reset@font| command in front of the
%    user definition.
% \changes{v2.0f}{89/07/20}{Unnecessary `global' removed.}
% \changes{v2.0g}{89/09/19}{`theoremfont to `theorembodyfont.}
% \changes{v2.1c}{92/06/22}{`reset@font added and logic changed.}
% \changes{v2.1d}{92/10/28}{Reset to empty if arg is empty}
%    \begin{macrocode}
\gdef\theorembodyfont#1{%
%    \end{macrocode}
%    We check if the argument supplied is empty and if so put nothing
%    into the |\theorem@bodyfont| token register to allow for
%    |\theorembodyfont{}| as a mean of using the default of the
%    current |\theoremstyle|.
%    \begin{macrocode}
   \def\@tempa{#1}%
   \ifx\@tempa\@empty
    \theorem@bodyfont{}%
   \else
    \theorem@bodyfont{\reset@font#1}%
   \fi
   }
\newtoks\theorem@bodyfont
\global\theorem@bodyfont{}
%    \end{macrocode}
% \end{macro}
% \end{macro}
%
%
% \begin{macro}{\theoremheaderfont}
% \changes{v2.0g}{89/09/19}{Macro added.}
%    The font for the theorem headers is handled differently because
%    this definition applies to all theorem styles.
%    \begin{macrocode}
\gdef\theoremheaderfont#1{\gdef\theorem@headerfont{#1}%
%    \end{macrocode}
%    After using the macro once it is redefined to produce an error
%    message.
%    \begin{macrocode}
       \gdef\theoremheaderfont##1{%
        \typeout{\string\theoremheaderfont\space should be used
                 only once.}}}
%    \end{macrocode}
% \end{macro}
%
% \begin{macro}{\theorem@headerfont}
% \changes{v2.0g}{89/09/19}{Macro added.}
%    To set the |\theorem@headerfont| default we first test if
%    the new fontselection scheme is in force.
% \changes{v2.1a}{90/02/15}{Support for new font selection.}
%    \begin{macrocode}
\ifx\upshape\undefined
%    \end{macrocode}
%    If not we define it to expand into |\bfseries|. We don't use
%    |\let| just in case a following style option redefines this
%    macro.
%    \begin{macrocode}
\gdef\theorem@headerfont{\bfseries}
%    \end{macrocode}
%    Otherwise we reset the current shape before calling |\bfseries|.
%    \begin{macrocode}
\else \gdef\theorem@headerfont{\normalfont\bfseries}\fi
%    \end{macrocode}
% \end{macro}
%
% \begin{macro}{\th@plain}
% \begin{macro}{\th@break}
% \begin{macro}{\th@marginbreak}
% \begin{macro}{\th@changebreak}
% \begin{macro}{\th@change}
% \begin{macro}{\th@margin}
%    The different styles are defined in macros such as
%    |\th@plain|.  Since memory space is precious in
%    ``non-Big-versions'', we have to avoid offering too many unused
%    definitions.  Therefore we define these styles in separate files
%    that can be loaded on demand.  Thus the commands themselves only
%    load these files. We use |\@input@|, a \LaTeXe{} internal command
%    that ensures that the file will be listed with |\listfiles|.
% \changes{v2.2a}{94/02/02}{Use `@input@ to load theorem layout files}
%    \begin{macrocode}
\gdef\th@plain{\@input@{thp.sty}}
\gdef\th@break{\@input@{thb.sty}}
\gdef\th@marginbreak{\@input@{thmb.sty}}
\gdef\th@changebreak{\@input@{thcb.sty}}
\gdef\th@change{\@input@{thc.sty}}
\gdef\th@margin{\@input@{thm.sty}}
%    \end{macrocode}
%    This list will be expanded when new styles become
%    available.  For testing,  just append new theorem substyles
%    as document options.
% \end{macro}
% \end{macro}
% \end{macro}
% \end{macro}
% \end{macro}
% \end{macro}
%
%
%
% \subsection{Definition of a new theorem set}
%
% As already pointed out, a new theorem environment can be defined in
% three different ways:
% \begin{verbatim}
%    \newtheorem{Lem}{Lemma}
%    \newtheorem{Lem}{Lemma}[section]
%    \newtheorem{Lem}[Theorem]{Lemma}
%\end{verbatim}
% The function of the macro |\newtheorem| is to recognize these cases
% and then to branch into one of the three macros |\@ynthm|, |\@xnthm|
% or |\@othm|.  This mechanism is adopted unchanged from
% \cite{latex:source}; the essential point here is that, for example,
% in the second case, the arguments \texttt{Lem}, \texttt{Lemma} and
% \texttt{section} are passed over to the macro |\@xnthm|.
%
% We inspect this case first because the others present fewer problems,
% and thus are easily derived from this one.
%
% \begin{macro}{\@xnthm}
%    For our example arguments, the macro |\@xnthm| must fulfill
%    the following:
%    \begin{itemize}
%       \item
%          Define a new \LaTeX-counter `Lem'
%       \item
%          reset this counter within a |\section|
%       \item
%          define the macro |\theLem|
%       \item
%          define the environment macros |\Lem| and
%          |\endLem| using the current
%          |\theoremstyle| and |\theorem@bodyfont|.
%    \end{itemize}
%    Obviously, all this should happen only if the first argument
%    of |\@xnthm| (i.e.\ \texttt{Lem} in our example) is chosen so
%    as not to conflict with any previously defined commands or
%    environments.  This test is performed by the \LaTeX{} macro
%    |\@ifdefinable|.
%    \begin{macrocode}
\gdef\@xnthm#1#2[#3]{\expandafter\@ifdefinable\csname #1\endcsname
%    \end{macrocode}
%    Therefore, the first argument of |\@ifdefinable| is the
%    expansion (in the example, |\Lem|) of
%    |\csname||#1\endcsname|.  The second argument is
%    executed only if the test has been completed successfully.
%    \begin{macrocode}
   {%
%    \end{macrocode}
%    Now we define the new counter. The names of the \LaTeX{}
%    macros employed should speak for themselves:
% \changes{v2.2b}{1995/10/07}{Use \cs{@newctr}}
% \changes{v2.2c}{1995/11/23}{Use \cs{@newctr} properly.}
%    \begin{macrocode}
    \@definecounter{#1}\@newctr{#1}[#3]%
%    \end{macrocode}
%    Using |\@newctr| will give a proper error message if the counter
%    in |#3| is not defined.
%    In defining `|\theLem|' we must generate the desired macro
%    name by use of |\expandafter| and |\csname|.
%    \begin{macrocode}
    \expandafter\xdef\csname the#1\endcsname
%    \end{macrocode}
%    An |\xdef| is used in order to make the definition global,
%    and to ensure that it contains the replacement texts of
%    |\@thmcountersep| and |\@thmcounter|.\footnote{These
%    two macros can be defined by the document style.  Their default
%    values produce a `\texttt{.}' as separation and an arabic
%    representation of the number.} However, not everything should be
%    expanded. For example, it saves space to use |\thesection|
%    instead of its---at times---lengthy expansion.
%    \begin{macrocode}
      {\expandafter \noexpand \csname the#3\endcsname
       \@thmcountersep \@thmcounter{#1}}%
%    \end{macrocode}
%    Thus with the defaults of \LaTeX{}, |\theLem| would be
%    replaced by the command sequence |\thesection.\arabic{Lem}|.
%
%    We will now look at the definition of the macro which is executed
%    at the beginning of the actual environment (in our example this
%    macro is |\Lem|). It should be noted that we use an
%    ``|\expandafter| trick'' to expand only certain parts of the
%    replacement text at the time of the definition.
%    \begin{macrocode}
    \def\@tempa{\global\@namedef{#1}}%
    \expandafter \@tempa \expandafter{%
%    \end{macrocode}
%    First, the macro that contains the current definitions of
%    |\@begintheorem| and |\@opargtheorem| should be called
%    up.  The name of this macro---as is already known---has the form
%    |\th@|\meta{theorem\ style}; therefore, it must be called by
%    \begin{macrocode}
      \csname th@\the \theorem@style
%    \end{macrocode}
%    In addition the default theorem font should be changeable, i.e.\
%    we have to insert the contents of |\theorem@bodyfont|. For
%    that reason, we expand even further, beyond |\endcsname|,
%    and thus insert the contents of the \textsf{token} register
%    |\theorem@bodyfont| in the replacement text.
%    \begin{macrocode}
            \expandafter \endcsname \the \theorem@bodyfont
%    \end{macrocode}
%    Now it is time to call the macro |\@thm| which takes over
%    the further processing.  It has two arguments: the current
%    counter name (in our example, \texttt{Lem}), and the text of the
%    label (in our example, \texttt{Lemma}).
%    \begin{macrocode}
     \@thm{#1}{#2}}%
%    \end{macrocode}
%    With this, the `sub-definition' is complete.  The macro
%    |\@endtheorem| ends a theorem environment and is, so far, nothing
%    but an |\endtrivlist|.  (Hence it is defined globally, and not
%    within the theorem styles.\footnote{This has to be changed as
%    soon as theorem styles that change \texttt{\bslash @endtheorem}
%    exist.  In such a case, all existing styles must be changed as
%    well since they will have to reset the macro.}) Therefore, we can
%    set it equivalent to the macro that ends the theorem set (in our
%    example, |\endLem|).  However, if some day theorem styles exist
%    that do change |\@endtheorem|, we would have to use the
%    commented-out line instead.
%    \begin{macrocode}
    \global \expandafter \let \csname end#1\endcsname \@endtheorem
%   \global\@namedef{end#1}{\@endtheorem}%
%    \end{macrocode}
%    With these commands all the required definitions are employed,
%    unless the test |\@ifdefinable| has failed.  Therefore, we
%    end the second argument of this macro and with it the definition
%    of |\@xnthm|.
%    \begin{macrocode}
   }}
%    \end{macrocode}
% \end{macro}
%
% \begin{macro}{\@ynthm}
%    The definition of |\@ynthm| is completely analogous. In
%    this case the new counter that is defined is not reset within
%    another counter; thus the definition of |\the...|
%    is simplified:
%    \begin{macrocode}
\gdef\@ynthm#1#2{\expandafter\@ifdefinable\csname #1\endcsname
   {\@definecounter{#1}%
    \expandafter\xdef\csname the#1\endcsname{\@thmcounter{#1}}%
%    \end{macrocode}
%    The rest of the definition corresponds literally to that of
%    |\@xnthm|:
%    \begin{macrocode}
    \def\@tempa{\global\@namedef{#1}}\expandafter \@tempa
     \expandafter{\csname th@\the \theorem@style \expandafter
     \endcsname \the\theorem@bodyfont \@thm{#1}{#2}}%
    \global \expandafter \let \csname end#1\endcsname \@endtheorem}}
%    \end{macrocode}
% \end{macro}
%
% \begin{macro}{\@othm}
%    The definition of |\@othm| does not contain anything new.
%    \begin{macrocode}
\gdef\@othm#1[#2]#3{%
%    \end{macrocode}
% \changes{v2.2b}{95/10/07}{Build in check for counter defined}
%    We do not define a new counter but instead use one that has
%    already been defined.  Thus the only definition we need is that
%    of this pseudo-counter (i.e.\ |\the|\meta{env.\ name}).
%    First we check if |#2| corresponds to a known counter name.
%    \begin{macrocode}
  \expandafter\ifx\csname c@#2\endcsname\relax
   \@nocounterr{#2}%
  \else
   \expandafter\@ifdefinable\csname #1\endcsname
   {\expandafter \xdef \csname the#1\endcsname
     {\expandafter \noexpand \csname the#2\endcsname}%
%    \end{macrocode}
%    All other parts of the definition can be adopted from
%    |\@xnthm|. We have to remember, though, that in this case
%    the name of the current counter and the theorem label have moved
%    to the second and third arguments.
%    \begin{macrocode}
    \def\@tempa{\global\@namedef{#1}}\expandafter \@tempa
     \expandafter{\csname th@\the \theorem@style \expandafter
     \endcsname \the\theorem@bodyfont \@thm{#2}{#3}}%
    \global \expandafter \let \csname end#1\endcsname \@endtheorem}%
  \fi}
%    \end{macrocode}
% \end{macro}
%
% \subsection{Macros that are employed in a {\envfont theorem}
%    environment}
%
% \begin{macro}{\@thm}
%    The macro |\@thm| has to increase the current counter.  Then,
%    depending on whether the environment has (or does not have) an
%    optional argument, it has to branch into either |\@begintheorem|
%    or |\@opargtheorem|.
%    \begin{macrocode}
\gdef\@thm#1#2{\refstepcounter{#1}%
%    \end{macrocode}
%    Now we start a {\envfont trivlist} environment, and give
%    |\@topsep| and |\@topsepadd| the values of the \textsf{skip}
%    registers |\theorempreskipamount| and |\theorempostskipamount|.
%    The value in |\@topsep| is the vertical space that is inserted by
%    the first (and only) |\item| in our |\trivlist| whilst
%    |\@topsepadd| is inserted by |\@endparenv| at the end of that
%    {\envfont trivlist} environment.  By using these registers, we
%    obtain the desired space around a {\envfont theorem} environment.
%    \begin{macrocode}
   \trivlist
   \@topsep \theorempreskipamount               % used by first \item
   \@topsepadd \theorempostskipamount           % used by \@endparenv
%    \end{macrocode}
%    Now we have to test whether an optional argument has been given.
% \changes{v2.0d}{89/07/18}{Forgotten percent added.}
%    \begin{macrocode}
   \@ifnextchar [%
%    \end{macrocode}
%    If there is an optional argument, we will call |\@ythm|, and
%    move the arguments read back into the input stream.
%    \begin{macrocode}
   {\@ythm{#1}{#2}}%
%    \end{macrocode}
%    If not, we call |\@begintheorem|. Its first argument is the
%    name of the theorem set (hence the second argument of
%    |\@thm|). Its second argument is the macro that produces the
%    current number.
%    \begin{macrocode}
   {\@begintheorem{#2}{\csname the#1\endcsname}\ignorespaces}}
%    \end{macrocode}
% \end{macro}
%
% \begin{macro}{\@xthm}
% \begin{macro}{\@ythm}
%    Both these macros were originally called by |\@thm|.  We do
%    not need |\@xthm| anymore, hence we reset it to
%    |\relax|. The definition of |\@ythm| has not changed at
%    all from its definition in \LaTeX. In order to make the macros
%    easier to understand, we will
%    nevertheless present it (commented out).
%    \begin{macrocode}
\global\let\@xthm\relax
% \def\@ythm#1#2[#3]{\@opargbegintheorem{#2}{\csname
%           the#1\endcsname}{#3}\ignorespaces}
%</package>
%    \end{macrocode}
%    The primitive |\ignorespaces| in |\@ythm| and
%    |\@thm| is needed to remove the spaces between the
%    |\begin{...}| and the actual text.
% \end{macro}
% \end{macro}
%
%
% \subsection{Definition of the theorem substyles}
%
% As already pointed out, the theorem substyles, defined below, are
% only loaded when necessary.  Note that all these substyles, except
% \texttt{plain}, have |\slshape| as the default body font.
%
% \subsubsection{The \texttt{plain} style}
%
% \label{dok:th@plain} As the following macros use |@|, we have
% to locally set the |\catcode| of this symbol to ``letter''.
% This happens within a group, so that we do not have to worry about
% which |\catcode| that symbol had before.
%    \begin{macrocode}
%<*thp>
\begingroup \makeatletter
%    \end{macrocode}
% Since we are now within a group, we must make all definitions
% globally.  First we make sure that \texttt{theorem.sty} is loaded.
% This will allow us to use this file as a document style option
% without having to call \texttt{theorem} itself as an option.  At the
% same time, we assure that at least version 2 is loaded, since
% |\theorem@style| was not defined in earlier versions.
%    \begin{macrocode}
\@ifundefined{theorem@style}{% \iffalse meta-comment
%
% Copyright (C) 1993-2024
%
% The LaTeX Project and any individual authors listed elsewhere
% in this file.
%
% This file is part of the Standard LaTeX `Tools Bundle'.
% -------------------------------------------------------
%
% It may be distributed and/or modified under the
% conditions of the LaTeX Project Public License, either version 1.3c
% of this license or (at your option) any later version.
% The latest version of this license is in
%    https://www.latex-project.org/lppl.txt
% and version 1.3c or later is part of all distributions of LaTeX
% version 2005/12/01 or later.
%
% The list of all files belonging to the LaTeX `Tools Bundle' is
% given in the file `manifest.txt'.
%
% \fi
%
%%
%
%
\def\FMithmInfo{2023/07/05 v2.2c Theorem extension package (FMi)}
%
% \ProvidesFile{theorem.dtx}[\FMithmInfo]
%\iffalse   % this is a METACOMMENT !
%%
%% Package `theorem' to use with LaTeX2e
%% Copyright (C) 1989-1995 Frank Mittelbach, all rights reserved.
%<+package>\@ifundefined{theorem@style}{}{\endinput}
%<+package>\NeedsTeXFormat{LaTeX2e}[1995/06/01]
%<+package>\ProvidesPackage{theorem}
%<+package>         [\FMithmInfo]
%
% \fi
%
% \changes{v1.0}{87/05/07}{First draft.}
% \changes{v1.0a}{87/05/07}{`thistheoremstyle' added.}
% \changes{v1.1a}{87/07/29}{New concept: redefine the internal
%                          \protect\LaTeX{} commands `@xnthm' etc.}
% \changes{v1.1b}{87/07/31}{`theoremfont' used to specify font.}
% \changes{v1.2a}{88/01/08}{`@endtheorem' now globally defined.}
% \changes{v1.2b}{88/01/11}{`newtheorem' only allowed in the preamble.}
% \changes{v1.3a}{88/02/14}{`theoremfont' now a toks register.}
%
% \changes{v2.0a}{88/11/16}{New implementation and new documentation.}
% \changes{v2.0a}{88/11/16}{No layout changes but internal modifications
%                           old substyles won't work.}
% \changes{v2.0b}{89/01/30}{`global' used for all definitions.}
% \changes{v2.0c}{89/06/15}{Documentation translated to English.
%                            Thanks to LLL and Nina}
% \changes{v2.1a}{90/02/15}{Support for new font selection.}
% \changes{v2.2a}{94/02/03}{Cleanup for LaTeX2e}
%
% \DoNotIndex{\@endparenv,\@ifdefinable,\@ifnextchar,\@ifundefined}
% \DoNotIndex{\@namedef,\@nameuse,\@spaces,\@tempa,\@warning}
% \DoNotIndex{\ }
% \DoNotIndex{\begingroup,\bfseries}
% \DoNotIndex{\csname}
% \DoNotIndex{\def,\do,\docdate}
% \DoNotIndex{\endcsname,\endgroup,\endinput,\endtrivlist,\expandafter}
% \DoNotIndex{\filedate,\fileversion}
% \DoNotIndex{\gdef,\global}
% \DoNotIndex{\hbox,\hskip}
% \DoNotIndex{\ignorespaces,\input,\itshape,\item}
% \DoNotIndex{\labelsep,\let,\llap}
% \DoNotIndex{\makeatletter}
% \DoNotIndex{\noexpand}
% \DoNotIndex{\refstepcounter,\relax,\rlap}
% \DoNotIndex{\setlength,\slshape,\space,\string,\strut}
% \DoNotIndex{\the,\trivlist,\typeout}
% \DoNotIndex{\vbox}
% \DoNotIndex{\xdef}
%
% \def\deflabel#1{\ttfamily #1\hfill}
% \def\deflist#1{\begin{list}{}{\settowidth\labelwidth{\ttfamily #1}%
%                               \setlength\leftmargin\labelwidth
%                               \addtolength\leftmargin\labelsep
%                               \let\makelabel\deflabel}}
% \def\enddeflist{\end{list}}
% \def\envfont{\normalfont\ttfamily}
%
% \MakeShortVerb{\|}
%
% \newcommand{\TUB}{{\sl TUGboat\/}}
%
% \GetFileInfo{theorem.dtx}
% \title{An Extension of the
%        \LaTeX{} theorem environment\thanks{This file
%        has version number \fileversion, last
%        revised \filedate.}}
% \author{Frank Mittelbach}
%
% \MaintainedByLaTeXTeam{tools}
% \maketitle
%
% \begin{abstract}
%    The macros described in this paper yield an extension of the
%    \LaTeX{} theorem mechanism. It is designed to satisfy the
%    different requirements of various journals. Thus, the
%    layout of the ``theorems'' can be manipulated by determining a
%    ``style''. This article describes not only the use, but
%    also the definition, of the necessary macros.
% \end{abstract}
%
% \section*{Preface to version 2.2}
%
% For \LaTeXe{} this package did not need any fundamental changes. I
% only modified the messages generated so that theorem layout styles
% will show up with the |\listfiles| command and cleaned the section
% on the New Font Selection Scheme since this is now included in
% \LaTeX.
%
% \section*{Preface to version 2.1}
%
% This version is identical to 2.0g described in \TUB\ 10\#3 except
% for some internal defaults which are now set depending on the used
% font selection scheme.
%
% This was done to avoid unpleasant surprises if the new font selection
% scheme is in force. For further details see
% section~\ref{sec:newfonts} and \cite{bk:GMS94}.
%
% \section{Introduction}
%
% For our purposes here, ``theorems'' are labelled enunciations,
% often set off from the main text by extra space and a font change.
% Theorems, corollaries, conjectures, definitions, and remarks
% are all instances of ``theorems''.  The ``header'' of these structures
% is composed of a label (such as \textsc{Theorem} or \textsc{Remark})
% and a number which serializes an item in the sequence of items
% with the same label.
%
% Shortly after the introduction of \LaTeX{} at the Fachbereich
% Mathematik in Mainz, the desire to manipulate the layout of
% ``theorems'' arose.  In Mainz, the following two conventions came
% into general use:
% \begin{enumerate}
%   \item The number of the theorem is shown in the margin.
%   \item There is a line break at
%         the end of the theorem header.
% \end{enumerate}
%
% Additionally, some journals require different formats which depend on
% the ``sort of theorem'': e.g.\ often remarks and definitions are set
% in |\upshape|, while |\itshape| is employed for main theorems.
%
% Confronted with these requirements, a theorem environment was
% developed in Mainz which allows separate determination of the
% layout of the ``theorems sets'', comparable to |\pagestyle|.
%
%
% \section{The user interface}
% \subsection{Defining new theorem sets}
%
% \DescribeMacro\newtheorem
% As in the original \LaTeX{} version, the command |\newtheorem|
% defines a new ``theorem set'' or ``theorem-like structure''.
% Two required arguments name the new environment and give the text
% to be typeset with each instance of the new ``set'', while
% an optional argument determines how
% the ``set'' is enumerated:
% \begin{description}
%    \item[\ttfamily \bslash newtheorem\{foo\}\{bar\}]
%       The theorem set {\envfont foo} (whose name is \texttt{bar})
%       uses its own counter.
%    \item[\ttfamily \bslash newtheorem\{foo2\}{[foo]}\{bar2\}]
%       The theorem set {\envfont foo2} (printed name \texttt{bar2})
%       uses the same counter as the theorem set \texttt{foo}.
%    \item[\ttfamily \bslash newtheorem\{foo3\}\{bar3\}{[section]}]
%       The theorem set {\envfont foo3} (printed name \texttt{bar3}) is
%       enumerated within the counter \texttt{section}, i.e.\ with every
%       new |\section| the enumeration begins again with 1, and
%       the enumeration is composed from the section-number and the
%       theorem counter itself.
% \end{description}
%
% \DescribeMacro\theoremstyle
% Additionally, the command |\theoremstyle| can define the layout
% of various, or all, theorem sets.  It should be noted that any
% theorem set defined by |\newtheorem| is typeset in the
% |\theoremstyle| that is current at the time of the definition.
% Thus, the following
% \begin{verbatim}
%    \theoremstyle{break}        \newtheorem{Cor}{Corollary}
%    \theoremstyle{plain}        \newtheorem{Exa}{Example}[section]
%\end{verbatim}                             ^^A No blank at line start
% leads to the result that the set {\envfont Cor} is formatted in the
% style \texttt{break}, while the set {\envfont Exa} and all the
% following ones are formatted in the style \texttt{plain}, unless
% another |\theoremstyle| follows.  Since the definitions installed by
% |\newtheorem| are global, one also can limit |\theoremstyle| locally
% by grouping braces.
%
% \DescribeMacro\theorembodyfont
% The choice of the font for the theorem body is completely
% independent of the chosen |\theoremstyle|; this has proven to
% be very advantageous.  For example,
% \begin{verbatim}
%   {\theorembodyfont{\upshape}          \newtheorem{Rem}{Remark}}
%\end{verbatim}
% defines a theorem set {\envfont Rem}, which will be set in
% |\upshape| in the current layout (which in our example is
% \texttt{plain}).  As with |\theoremstyle|, the |\theorembodyfont|
% chosen is that current at the time of |\newtheorem|.  If
% |\theorembodyfont| is not specified or one defines
% |\theorembodyfont{}|, then the font used will be that defined by the
% |\theoremstyle|.
%
% \DescribeMacro\theoremheaderfont It is also possible to customize
% the font used for the theorem headers.  This is, however, a global
% declaration, and therefore there should be at most one
% |\theoremheaderfont| declaration in the preamble.\footnote{If it is
% actually necessary to have different header fonts, one has to define
% new theorem styles (substituting the desired font) or specify the
% information directly in the \texttt{\bslash newtheorem} declaration
% (the unclean variant).}
%
% \DescribeMacro\theorempreskipamount
% \DescribeMacro\theorempostskipamount Two additional parameters
% affect the vertical space around the theorem environments:
% |\theorempreskipamount| and |\theorempostskipamount| define,
% respectively, the spacing before and after such an environment.
% These parameters apply for all theorem sets and can be manipulated
% with the ordinary length macros.  They are rubber lengths
% (`\textsf{skips}'), and therefore can contain \texttt{plus} and
% \texttt{minus} parts.
%
% Since the definition of theorem sets should---most sensibly---be
% placed in the preamble, we only allow installation there.
% It is therefore possible to release the memory used here after
% |\begin{document}|, in order to make room for other
% applications.
%
%
% \subsection{Existing theorem styles}
%
% The following theorem styles exist to date:
% \begin{deflist}{marginbreak}
%    \item[plain]
%       This theorem style emulates the original \LaTeX{} definition,
%       except that additionally the parameters
%       |\theorem...skipamount| are used.
%    \item[break]
%       In this style, the theorem header is followed by a line
%       break.
%    \item[marginbreak]
%       The theorem number is set in the margin, and there is a line
%       break as in \texttt{break}.
%    \item[changebreak]
%       Like \texttt{break}, but with header number and text
%       interchanged.
%    \item[change]
%       Header number and text are interchanged, without a line break.
%    \item[margin]
%       The number is set in the left margin, without a line break.
% \end{deflist}
% All styles (except \texttt{plain}) select |\slshape| as the default
% |\theorembodyfont|.
%
%
% \subsection{Examples}
%
% Given the above theorem sets {\envfont Cor}, {\envfont Exa}
% and {\envfont Rem},
% suppose that the preamble also contains the declarations:
% \begin{verbatim}
%    \theoremstyle{marginbreak}   \newtheorem{Lem}[Cor]{Lemma}
%    \theoremstyle{change}
%    \theorembodyfont{\itshape}        \newtheorem{Def}[Cor]{Definition}
%
%    \theoremheaderfont{\scshape}
%\end{verbatim}
% Then the following are some typical examples of the typeset output
% resulting from their use.
%
% \begin{Cor}
%    This is a sentence typeset in the theorem environment {\envfont
%    Cor}.
% \end{Cor}
%
% \begin{Exa}
%    This is a sentence typeset in the theorem environment {\envfont
%    Exa}.
% \end{Exa}
%
% \begin{Rem}
%    This is a sentence typeset in the theorem environment {\envfont
%    Rem}.
% \end{Rem}
%
% \begin{Lem}[Ben User]
%    This is a sentence typeset in the theorem environment {\envfont
%    Lem}.
% \end{Lem}
%
% \begin{Def}[Very impressive Definition]
%    This is a sentence typeset in the theorem environment {\envfont
%    Def}.
% \end{Def}
%
% The last two examples show the effect of the optional argument to
% a theorem environment (it is the text typeset in parentheses).
%
%
% \section{Special Considerations}
% \label{sec:newfonts}
%
% Theoremheader and body are implemented as a unit. This means that
% the |\theoremheaderfont|  will inherit characteristics of the
% |\theorembodyfont| in \LaTeXe. Thus, if for example
% |\theorembodyfont| is |\itshape| and |\theoremheaderfont|
% is |\bfseries| the font selected for the header will have the
% characteristics `bold extended italic'. If this is not desired one
% should set the |\theoremheaderfont| to something like
% \begin{verbatim}
%      \theoremheaderfont{\normalfont\bfseries}
%\end{verbatim}
% i.e.\ supplying all necessary font information explicitly.
%
%
%
% \section{Acknowledgements}
%
% The publication of this set of macros was only possible with the
% help of Christina Busse (translating the manuscript into English),
% Joachim Pense (playing the r\^ole of typist), Chris Rowley
% (looking everything over) and many others providing useful
% suggestions.
%
% \MaybeStop{%
%    \begin{thebibliography}{9}
%       \bibitem{bk:GMS94} \textsc{M.~Goossens}, \textsc{F.~Mittelbach}
%         and \textsc{A.~Samarin}.
%         \newblock The \LaTeX{} Companion.
%         \newblock
%         Addison-Wesley, Reading, Massachusetts, 1994.
%       \bibitem{latex:source} \textsc{Lamport, Leslie.} \newblock
%                               \texttt{latex.tex}, version 2.09, date
%                               Feb. 1990.
%    \end{thebibliography}
%    \PrintIndex
%    \PrintChanges}
%
%
% \section{The documentation driver file}
%
% The next bit of code contains the documentation driver file for
% \TeX{}, i.e., the file that will produce the documentation you are
% currently reading. It will be extracted from this file by the
% \texttt{docstrip} program. Since it is the first code in the file
% one can alternatively process this file directly with \LaTeXe{} to
% obtain the documentation.
%
%    \begin{macrocode}
%<*driver>
\documentclass{ltxdoc}

\usepackage{theorem}

%<+driver>% The next few lines define theorem sets which are used
%<+driver>% in the example section of the documentation.

\theoremstyle{break}       \newtheorem{Cor}{Corollary}
\theoremstyle{plain}       \newtheorem{Exa}{Example}[section]
{\theorembodyfont{\upshape}\newtheorem{Rem}{Remark}}
\theoremstyle{marginbreak} \newtheorem{Lem}[Cor]{Lemma}
\theoremstyle{change}
\theorembodyfont{\itshape} \newtheorem{Def}[Cor]{Definition}

\theoremheaderfont{\scshape}

\RecordChanges

\begin{document}
   \DocInput{theorem.dtx}
\end{document}
%</driver>
%    \end{macrocode}
%
% \section{Definition of the Macros}
%
% If the file has been loaded before, we abort immediately. If not the
% package announces itself (this is actually done at the very top of
% the file---the way it is done isn't good style so don't copy it).
% \changes{v2.0e}{89/07/19}{Spaces removed from `typeout'.}
% \changes{v2.2b}{95/11/19}{Announce moved to top of file}
%    \begin{macrocode}
%<*package>
%\@ifundefined{theorem@style}{}{\endinput}
%\def\FMithmInfo{1995/11/19 v2.2b Theorem extension package (FMi)}
%\ProvidesPackage{theorem}[\FMithmInfo]
%    \end{macrocode}
%
%
% \subsection{Definition of theorem styles and fonts}
%
% All the definitions in this file are done globally to allow
% inputting this file inside a group.
%
% \begin{macro}{\theoremstyle}
%    Before a theorem style can be installed, the chosen style must be
%    known. For that reason, we must test to see that
%    |\th@|\meta{style} is known or, more precisely, that it is
%    different from |\relax|. If the style is not known then
%    |\th@plain| is used.
%    \begin{macrocode}
\gdef\theoremstyle#1{%
   \@ifundefined{th@#1}{\@warning
          {Unknown theoremstyle `#1'. Using `plain'}%
          \theorem@style{plain}}%
%    \end{macrocode}
%    We save the theorem style to be used in the token register
%    |\theorem@style|.
%    \begin{macrocode}
      {\theorem@style{#1}}%
%    \end{macrocode}
%    Now we ``evaluate'' the theorem style: this means, we call the
%    macro |\th@|\meta{style} which will activate the relevant
%    definitions which are contained in a separate file.  This is done
%    in a group to suppress changes to the current font.  This could
%    otherwise pose problems together with the new font selection
%    scheme\footnote{When I printed the original article using the new
%    font selection scheme I ended with a document with slanted
%    typefaces (text headings and all) simply because one of the
%    theorem styles used \texttt{\bslash sl} at toplevel.} if the
%    |\th@|\meta{style} is evaluated a second time.
% \changes{v2.1a}{90/02/15}{Group added around `th@..}
% \changes{v2.1a}{90/02/15}{Reseting of `..@begintheorem removed.}
%    \begin{macrocode}
      \begingroup
        \csname th@\the\theorem@style \endcsname
      \endgroup}
%    \end{macrocode}
% \end{macro}
%
%
% \begin{macro}{\@begintheorem}
% \changes{v2.1a}{90/02/15}{Relaxed at toplevel.}
% \begin{macro}{\@opargbegintheorem}
% \changes{v2.1a}{90/02/15}{Relaxed at toplevel.}
%    We reset |\@begintheorem| and |\@opargbegintheorem| to
%    |\relax| since these commands are no longer necessary at
%    toplevel. This will save a few tokens.
%    \begin{macrocode}
\global\let\@begintheorem\relax
\global\let\@opargbegintheorem\relax
%    \end{macrocode}
% \end{macro}
% \end{macro}
%
% \begin{macro}{\theorem@style}
%    Obviously the \textsf{token} register used above has to be
%    allocated. To assure the utmost compatibility with the
%    original \LaTeX{} definition, we set the default theorem style to
%    \texttt{plain}, which implements the usual \LaTeX{} convention.
% \changes{v2.0f}{89/07/20}{Unnecessary `global' removed.}
%    \begin{macrocode}
\newtoks\theorem@style
\global\theorem@style{plain}
%    \end{macrocode}
% \end{macro}
%
% \begin{macro}{\theorembodyfont}
% \begin{macro}{\theorem@bodyfont}
%    If the |\theorembodyfont| is set by the user then it should not
%    interact with the default font set in the theorem style. When the
%    new font selection is in force this may happen if, for example,
%    the default is |\itshape| and the new |\theorembodyfont| is
%    |\sffamily|. So we add a |\reset@font| command in front of the
%    user definition.
% \changes{v2.0f}{89/07/20}{Unnecessary `global' removed.}
% \changes{v2.0g}{89/09/19}{`theoremfont to `theorembodyfont.}
% \changes{v2.1c}{92/06/22}{`reset@font added and logic changed.}
% \changes{v2.1d}{92/10/28}{Reset to empty if arg is empty}
%    \begin{macrocode}
\gdef\theorembodyfont#1{%
%    \end{macrocode}
%    We check if the argument supplied is empty and if so put nothing
%    into the |\theorem@bodyfont| token register to allow for
%    |\theorembodyfont{}| as a mean of using the default of the
%    current |\theoremstyle|.
%    \begin{macrocode}
   \def\@tempa{#1}%
   \ifx\@tempa\@empty
    \theorem@bodyfont{}%
   \else
    \theorem@bodyfont{\reset@font#1}%
   \fi
   }
\newtoks\theorem@bodyfont
\global\theorem@bodyfont{}
%    \end{macrocode}
% \end{macro}
% \end{macro}
%
%
% \begin{macro}{\theoremheaderfont}
% \changes{v2.0g}{89/09/19}{Macro added.}
%    The font for the theorem headers is handled differently because
%    this definition applies to all theorem styles.
%    \begin{macrocode}
\gdef\theoremheaderfont#1{\gdef\theorem@headerfont{#1}%
%    \end{macrocode}
%    After using the macro once it is redefined to produce an error
%    message.
%    \begin{macrocode}
       \gdef\theoremheaderfont##1{%
        \typeout{\string\theoremheaderfont\space should be used
                 only once.}}}
%    \end{macrocode}
% \end{macro}
%
% \begin{macro}{\theorem@headerfont}
% \changes{v2.0g}{89/09/19}{Macro added.}
%    To set the |\theorem@headerfont| default we first test if
%    the new fontselection scheme is in force.
% \changes{v2.1a}{90/02/15}{Support for new font selection.}
%    \begin{macrocode}
\ifx\upshape\undefined
%    \end{macrocode}
%    If not we define it to expand into |\bfseries|. We don't use
%    |\let| just in case a following style option redefines this
%    macro.
%    \begin{macrocode}
\gdef\theorem@headerfont{\bfseries}
%    \end{macrocode}
%    Otherwise we reset the current shape before calling |\bfseries|.
%    \begin{macrocode}
\else \gdef\theorem@headerfont{\normalfont\bfseries}\fi
%    \end{macrocode}
% \end{macro}
%
% \begin{macro}{\th@plain}
% \begin{macro}{\th@break}
% \begin{macro}{\th@marginbreak}
% \begin{macro}{\th@changebreak}
% \begin{macro}{\th@change}
% \begin{macro}{\th@margin}
%    The different styles are defined in macros such as
%    |\th@plain|.  Since memory space is precious in
%    ``non-Big-versions'', we have to avoid offering too many unused
%    definitions.  Therefore we define these styles in separate files
%    that can be loaded on demand.  Thus the commands themselves only
%    load these files. We use |\@input@|, a \LaTeXe{} internal command
%    that ensures that the file will be listed with |\listfiles|.
% \changes{v2.2a}{94/02/02}{Use `@input@ to load theorem layout files}
%    \begin{macrocode}
\gdef\th@plain{\@input@{thp.sty}}
\gdef\th@break{\@input@{thb.sty}}
\gdef\th@marginbreak{\@input@{thmb.sty}}
\gdef\th@changebreak{\@input@{thcb.sty}}
\gdef\th@change{\@input@{thc.sty}}
\gdef\th@margin{\@input@{thm.sty}}
%    \end{macrocode}
%    This list will be expanded when new styles become
%    available.  For testing,  just append new theorem substyles
%    as document options.
% \end{macro}
% \end{macro}
% \end{macro}
% \end{macro}
% \end{macro}
% \end{macro}
%
%
%
% \subsection{Definition of a new theorem set}
%
% As already pointed out, a new theorem environment can be defined in
% three different ways:
% \begin{verbatim}
%    \newtheorem{Lem}{Lemma}
%    \newtheorem{Lem}{Lemma}[section]
%    \newtheorem{Lem}[Theorem]{Lemma}
%\end{verbatim}
% The function of the macro |\newtheorem| is to recognize these cases
% and then to branch into one of the three macros |\@ynthm|, |\@xnthm|
% or |\@othm|.  This mechanism is adopted unchanged from
% \cite{latex:source}; the essential point here is that, for example,
% in the second case, the arguments \texttt{Lem}, \texttt{Lemma} and
% \texttt{section} are passed over to the macro |\@xnthm|.
%
% We inspect this case first because the others present fewer problems,
% and thus are easily derived from this one.
%
% \begin{macro}{\@xnthm}
%    For our example arguments, the macro |\@xnthm| must fulfill
%    the following:
%    \begin{itemize}
%       \item
%          Define a new \LaTeX-counter `Lem'
%       \item
%          reset this counter within a |\section|
%       \item
%          define the macro |\theLem|
%       \item
%          define the environment macros |\Lem| and
%          |\endLem| using the current
%          |\theoremstyle| and |\theorem@bodyfont|.
%    \end{itemize}
%    Obviously, all this should happen only if the first argument
%    of |\@xnthm| (i.e.\ \texttt{Lem} in our example) is chosen so
%    as not to conflict with any previously defined commands or
%    environments.  This test is performed by the \LaTeX{} macro
%    |\@ifdefinable|.
%    \begin{macrocode}
\gdef\@xnthm#1#2[#3]{\expandafter\@ifdefinable\csname #1\endcsname
%    \end{macrocode}
%    Therefore, the first argument of |\@ifdefinable| is the
%    expansion (in the example, |\Lem|) of
%    |\csname||#1\endcsname|.  The second argument is
%    executed only if the test has been completed successfully.
%    \begin{macrocode}
   {%
%    \end{macrocode}
%    Now we define the new counter. The names of the \LaTeX{}
%    macros employed should speak for themselves:
% \changes{v2.2b}{1995/10/07}{Use \cs{@newctr}}
% \changes{v2.2c}{1995/11/23}{Use \cs{@newctr} properly.}
%    \begin{macrocode}
    \@definecounter{#1}\@newctr{#1}[#3]%
%    \end{macrocode}
%    Using |\@newctr| will give a proper error message if the counter
%    in |#3| is not defined.
%    In defining `|\theLem|' we must generate the desired macro
%    name by use of |\expandafter| and |\csname|.
%    \begin{macrocode}
    \expandafter\xdef\csname the#1\endcsname
%    \end{macrocode}
%    An |\xdef| is used in order to make the definition global,
%    and to ensure that it contains the replacement texts of
%    |\@thmcountersep| and |\@thmcounter|.\footnote{These
%    two macros can be defined by the document style.  Their default
%    values produce a `\texttt{.}' as separation and an arabic
%    representation of the number.} However, not everything should be
%    expanded. For example, it saves space to use |\thesection|
%    instead of its---at times---lengthy expansion.
%    \begin{macrocode}
      {\expandafter \noexpand \csname the#3\endcsname
       \@thmcountersep \@thmcounter{#1}}%
%    \end{macrocode}
%    Thus with the defaults of \LaTeX{}, |\theLem| would be
%    replaced by the command sequence |\thesection.\arabic{Lem}|.
%
%    We will now look at the definition of the macro which is executed
%    at the beginning of the actual environment (in our example this
%    macro is |\Lem|). It should be noted that we use an
%    ``|\expandafter| trick'' to expand only certain parts of the
%    replacement text at the time of the definition.
%    \begin{macrocode}
    \def\@tempa{\global\@namedef{#1}}%
    \expandafter \@tempa \expandafter{%
%    \end{macrocode}
%    First, the macro that contains the current definitions of
%    |\@begintheorem| and |\@opargtheorem| should be called
%    up.  The name of this macro---as is already known---has the form
%    |\th@|\meta{theorem\ style}; therefore, it must be called by
%    \begin{macrocode}
      \csname th@\the \theorem@style
%    \end{macrocode}
%    In addition the default theorem font should be changeable, i.e.\
%    we have to insert the contents of |\theorem@bodyfont|. For
%    that reason, we expand even further, beyond |\endcsname|,
%    and thus insert the contents of the \textsf{token} register
%    |\theorem@bodyfont| in the replacement text.
%    \begin{macrocode}
            \expandafter \endcsname \the \theorem@bodyfont
%    \end{macrocode}
%    Now it is time to call the macro |\@thm| which takes over
%    the further processing.  It has two arguments: the current
%    counter name (in our example, \texttt{Lem}), and the text of the
%    label (in our example, \texttt{Lemma}).
%    \begin{macrocode}
     \@thm{#1}{#2}}%
%    \end{macrocode}
%    With this, the `sub-definition' is complete.  The macro
%    |\@endtheorem| ends a theorem environment and is, so far, nothing
%    but an |\endtrivlist|.  (Hence it is defined globally, and not
%    within the theorem styles.\footnote{This has to be changed as
%    soon as theorem styles that change \texttt{\bslash @endtheorem}
%    exist.  In such a case, all existing styles must be changed as
%    well since they will have to reset the macro.}) Therefore, we can
%    set it equivalent to the macro that ends the theorem set (in our
%    example, |\endLem|).  However, if some day theorem styles exist
%    that do change |\@endtheorem|, we would have to use the
%    commented-out line instead.
%    \begin{macrocode}
    \global \expandafter \let \csname end#1\endcsname \@endtheorem
%   \global\@namedef{end#1}{\@endtheorem}%
%    \end{macrocode}
%    With these commands all the required definitions are employed,
%    unless the test |\@ifdefinable| has failed.  Therefore, we
%    end the second argument of this macro and with it the definition
%    of |\@xnthm|.
%    \begin{macrocode}
   }}
%    \end{macrocode}
% \end{macro}
%
% \begin{macro}{\@ynthm}
%    The definition of |\@ynthm| is completely analogous. In
%    this case the new counter that is defined is not reset within
%    another counter; thus the definition of |\the...|
%    is simplified:
%    \begin{macrocode}
\gdef\@ynthm#1#2{\expandafter\@ifdefinable\csname #1\endcsname
   {\@definecounter{#1}%
    \expandafter\xdef\csname the#1\endcsname{\@thmcounter{#1}}%
%    \end{macrocode}
%    The rest of the definition corresponds literally to that of
%    |\@xnthm|:
%    \begin{macrocode}
    \def\@tempa{\global\@namedef{#1}}\expandafter \@tempa
     \expandafter{\csname th@\the \theorem@style \expandafter
     \endcsname \the\theorem@bodyfont \@thm{#1}{#2}}%
    \global \expandafter \let \csname end#1\endcsname \@endtheorem}}
%    \end{macrocode}
% \end{macro}
%
% \begin{macro}{\@othm}
%    The definition of |\@othm| does not contain anything new.
%    \begin{macrocode}
\gdef\@othm#1[#2]#3{%
%    \end{macrocode}
% \changes{v2.2b}{95/10/07}{Build in check for counter defined}
%    We do not define a new counter but instead use one that has
%    already been defined.  Thus the only definition we need is that
%    of this pseudo-counter (i.e.\ |\the|\meta{env.\ name}).
%    First we check if |#2| corresponds to a known counter name.
%    \begin{macrocode}
  \expandafter\ifx\csname c@#2\endcsname\relax
   \@nocounterr{#2}%
  \else
   \expandafter\@ifdefinable\csname #1\endcsname
   {\expandafter \xdef \csname the#1\endcsname
     {\expandafter \noexpand \csname the#2\endcsname}%
%    \end{macrocode}
%    All other parts of the definition can be adopted from
%    |\@xnthm|. We have to remember, though, that in this case
%    the name of the current counter and the theorem label have moved
%    to the second and third arguments.
%    \begin{macrocode}
    \def\@tempa{\global\@namedef{#1}}\expandafter \@tempa
     \expandafter{\csname th@\the \theorem@style \expandafter
     \endcsname \the\theorem@bodyfont \@thm{#2}{#3}}%
    \global \expandafter \let \csname end#1\endcsname \@endtheorem}%
  \fi}
%    \end{macrocode}
% \end{macro}
%
% \subsection{Macros that are employed in a {\envfont theorem}
%    environment}
%
% \begin{macro}{\@thm}
%    The macro |\@thm| has to increase the current counter.  Then,
%    depending on whether the environment has (or does not have) an
%    optional argument, it has to branch into either |\@begintheorem|
%    or |\@opargtheorem|.
%    \begin{macrocode}
\gdef\@thm#1#2{\refstepcounter{#1}%
%    \end{macrocode}
%    Now we start a {\envfont trivlist} environment, and give
%    |\@topsep| and |\@topsepadd| the values of the \textsf{skip}
%    registers |\theorempreskipamount| and |\theorempostskipamount|.
%    The value in |\@topsep| is the vertical space that is inserted by
%    the first (and only) |\item| in our |\trivlist| whilst
%    |\@topsepadd| is inserted by |\@endparenv| at the end of that
%    {\envfont trivlist} environment.  By using these registers, we
%    obtain the desired space around a {\envfont theorem} environment.
%    \begin{macrocode}
   \trivlist
   \@topsep \theorempreskipamount               % used by first \item
   \@topsepadd \theorempostskipamount           % used by \@endparenv
%    \end{macrocode}
%    Now we have to test whether an optional argument has been given.
% \changes{v2.0d}{89/07/18}{Forgotten percent added.}
%    \begin{macrocode}
   \@ifnextchar [%
%    \end{macrocode}
%    If there is an optional argument, we will call |\@ythm|, and
%    move the arguments read back into the input stream.
%    \begin{macrocode}
   {\@ythm{#1}{#2}}%
%    \end{macrocode}
%    If not, we call |\@begintheorem|. Its first argument is the
%    name of the theorem set (hence the second argument of
%    |\@thm|). Its second argument is the macro that produces the
%    current number.
%    \begin{macrocode}
   {\@begintheorem{#2}{\csname the#1\endcsname}\ignorespaces}}
%    \end{macrocode}
% \end{macro}
%
% \begin{macro}{\@xthm}
% \begin{macro}{\@ythm}
%    Both these macros were originally called by |\@thm|.  We do
%    not need |\@xthm| anymore, hence we reset it to
%    |\relax|. The definition of |\@ythm| has not changed at
%    all from its definition in \LaTeX. In order to make the macros
%    easier to understand, we will
%    nevertheless present it (commented out).
%    \begin{macrocode}
\global\let\@xthm\relax
% \def\@ythm#1#2[#3]{\@opargbegintheorem{#2}{\csname
%           the#1\endcsname}{#3}\ignorespaces}
%</package>
%    \end{macrocode}
%    The primitive |\ignorespaces| in |\@ythm| and
%    |\@thm| is needed to remove the spaces between the
%    |\begin{...}| and the actual text.
% \end{macro}
% \end{macro}
%
%
% \subsection{Definition of the theorem substyles}
%
% As already pointed out, the theorem substyles, defined below, are
% only loaded when necessary.  Note that all these substyles, except
% \texttt{plain}, have |\slshape| as the default body font.
%
% \subsubsection{The \texttt{plain} style}
%
% \label{dok:th@plain} As the following macros use |@|, we have
% to locally set the |\catcode| of this symbol to ``letter''.
% This happens within a group, so that we do not have to worry about
% which |\catcode| that symbol had before.
%    \begin{macrocode}
%<*thp>
\begingroup \makeatletter
%    \end{macrocode}
% Since we are now within a group, we must make all definitions
% globally.  First we make sure that \texttt{theorem.sty} is loaded.
% This will allow us to use this file as a document style option
% without having to call \texttt{theorem} itself as an option.  At the
% same time, we assure that at least version 2 is loaded, since
% |\theorem@style| was not defined in earlier versions.
%    \begin{macrocode}
\@ifundefined{theorem@style}{\input{theorem.sty}}{}
\ProvidesFile{thp.sty}
             [\FMithmInfo]
%    \end{macrocode}
%
%
% \begin{macro}{\th@plain}
% \begin{macro}{\@begintheorem}
% \begin{macro}{\@opargbegintheorem}
%    |\theoremstyle{plain}| corresponds to the original
%    definition, except that the distances to the surrounding text are
%    determined by the parameters |\theorempreskipamount| and
%    |\theorempostskipamount|.
%    First we set the default body font.
%    \begin{macrocode}
\gdef\th@plain{\normalfont\itshape
%    \end{macrocode}
%    Then we define |\@begintheorem| and
%    |\@opargbegintheorem|. These two macros define how the
%    header of a theorem is typeset.  |\@opargbegintheorem| will
%    be called if a \textsf{theorem} environment with an optional
%    argument is encountered; otherwise, the header is constructed by
%    calling |\@begintheorem|.  If one of these macros is
%    executed, we are within a \textsf{trivlist} environment started by
%    |\@thm|.  So the theorem header is produced with an
%    |\item| command.
%
%    Instead of specifying the header font directly, all standard
%    theorem styles use the |\theorem@headerfont| macro to allow
%    customization.  The extra space (|\labelsep|) is necessary
%    because of problems in the \textsf{trivlist} environment.
%    \begin{macrocode}
  \def\@begintheorem##1##2{%
        \item[\hskip\labelsep \theorem@headerfont ##1\ ##2]}%
%    \end{macrocode}
%    The definition of |\@opargbegintheorem| is completely
%    analogous. The only difference is the fact that there exists a
%    third argument (which is the optional parameter of the
%    environment and contains additional information about the
%    theorem).  Customarily we enclose it in parentheses.
%    \begin{macrocode}
\def\@opargbegintheorem##1##2##3{%
   \item[\hskip\labelsep \theorem@headerfont ##1\ ##2\ (##3)]}}
%    \end{macrocode}
% \end{macro}
% \end{macro}
% \end{macro}
%
% \noindent
% We conclude with an |\endgroup| to restore the
% |\catcode| of |@|.
%    \begin{macrocode}
\endgroup
%</thp>
%    \end{macrocode}
%
%
% \subsubsection{The \texttt{break} style}
%
% This style option is stored in the file |thb.sty|.
% For the next two lines see the documentation for |\th@plain| on
% page~\pageref{dok:th@plain}.
%    \begin{macrocode}
%<*thb>
\begingroup \makeatletter
\@ifundefined{theorem@style}{\input{theorem.sty}}{}
\ProvidesFile{thb.sty}
             [\FMithmInfo]
%    \end{macrocode}
%
% \begin{macro}{\th@break}
%    |\theoremstyle{break}| produces a line break after the name
%    of the theorem.  The font is |\slshape|.  Hence, we define
%    |\th@break| as follows:
%    \begin{macrocode}
\gdef\th@break{\normalfont\slshape
  \def\@begintheorem##1##2{\item[%
%    \end{macrocode}
%    We run into the following problem: it is not possible to create
%    the header with |\item[|\meta{title}|]| and then start
%    a new line by, for example, |\mbox{}\\|. Such a definition
%    will fail whenever a list environment follows immediately.  With
%    the above construction, the |\mbox{}| causes the switch
%    |@inlabel| (cf.\ definition of |\list| and
%    |\trivlist| in \cite{latex:source}) to be set to \texttt{false}
%    and so the following list will insert additional vertical
%    space (|\topskip|). This is quite annoying. Therefore, we
%    create the line break within the |\item|. In order to ensure
%    that the text will begin at the proper position in the following
%    line, we simply pretend that the label does not take any
%    room.\footnote{This will lead to problems whenever very high
%    symbols occurring in the line tower into the
%    heading. So, something else has to be done here sometime.}
%    \begin{macrocode}
       \rlap{\vbox{\hbox{\hskip \labelsep\theorem@headerfont ##1\ ##2}%
                   \hbox{\strut}}}]}%
%    \end{macrocode}
%    Again, the definition of |\@opargbegintheorem| is
%    completely analogous.
%    \begin{macrocode}
\def\@opargbegintheorem##1##2##3{%
  \item[\rlap{\vbox{\hbox{\hskip \labelsep \theorem@headerfont
                     ##1\ ##2\ (##3)}%
                    \hbox{\strut}}}]}}
\endgroup
%</thb>
%    \end{macrocode}
% \end{macro}
%
%
% \subsubsection{The \texttt{changebreak} style}
%
%    \begin{macrocode}
% This style option is stored in the file |thcb.sty|.
%    \begin{macrocode}
%<*thcb>
\begingroup \makeatletter
\@ifundefined{theorem@style}{\input{theorem.sty}}{}
\ProvidesFile{thcb.sty}
             [\FMithmInfo]
%    \end{macrocode}
%
% \begin{macro}{\th@changebreak}
%    The \texttt{change-break} theorem style is like \texttt{break}
%    but with interchange of theorem name and theorem number.  Thus we
%    define |\th@changebreak| as follows:
%    \begin{macrocode}
\gdef\th@changebreak{\normalfont\slshape
  \def\@begintheorem##1##2{\item
       [\rlap{\vbox{\hbox{\hskip\labelsep \theorem@headerfont ##2\ ##1}%
                    \hbox{\strut}}}]}%
\def\@opargbegintheorem##1##2##3{%
  \item[\rlap{\vbox{\hbox{\hskip\labelsep \theorem@headerfont
                     ##2\ ##1\ (##3)}%
                    \hbox{\strut}}}]}}
\endgroup
%</thcb>
%    \end{macrocode}
% \end{macro}
%
%
% \subsubsection{The \texttt{change} style}
%
% This style option is stored in the file |thc.sty|.
%    \begin{macrocode}
%<*thc>
\begingroup \makeatletter
\@ifundefined{theorem@style}{\input{theorem.sty}}{}
\ProvidesFile{thc.sty}
             [\FMithmInfo]
%    \end{macrocode}
%
% \begin{macro}{\th@change}
%    The \texttt{change} theorem style corresponds to the
%    \texttt{change break} style without a linebreak after the header.
%    To say it in another way, it's the same as the \texttt{plain}
%    style but with number and name interchanged and |\slshape| as the
%    default font.
%    \begin{macrocode}
\gdef\th@change{\normalfont\slshape
  \def\@begintheorem##1##2{\item
       [\hskip\labelsep \theorem@headerfont ##2\ ##1]}%
\def\@opargbegintheorem##1##2##3{%
  \item[\hskip\labelsep \theorem@headerfont ##2\ ##1\ (##3)]}}
\endgroup
%</thc>
%    \end{macrocode}
% \end{macro}
%
%
% \subsubsection{The \texttt{marginbreak} style}
%
% This style option is the one used most often at Mainz.  It is saved
% in the file |thmb.sty|.
%    \begin{macrocode}
%<*thmb>
\begingroup \makeatletter
\@ifundefined{theorem@style}{\input{theorem.sty}}{}
\ProvidesFile{thmb.sty}
             [\FMithmInfo]
%    \end{macrocode}
%
% \begin{macro}{\th@marginbreak}
%    The \texttt{margin break} style is nearly the same as the
%    \texttt{change break} style. The only difference is the placement
%    of the theorem number.  We use |\llap| to place it in the left
%    margin.
%
%    In this style |\labelsep| denotes the separation between the
%    number and the text.
%    \begin{macrocode}
\gdef\th@marginbreak{\normalfont\slshape
  \def\@begintheorem##1##2{\item
       [\rlap{\vbox{\theorem@headerfont
                    \hbox{\llap{##2}\hskip\labelsep ##1}%
                    \hbox{\strut}}}]}%
\def\@opargbegintheorem##1##2##3{%
  \item[\rlap{\vbox{\theorem@headerfont
                    \hbox{\llap{##2}\hskip\labelsep ##1\ (##3)}%
                    \hbox{\strut}}}]}}
\endgroup
%</thmb>
%    \end{macrocode}
% \changes{v2.1c}{90/02/17}{`endgroup outside macrocode env.}
% \end{macro}
%
%
%
% \subsubsection{The \texttt{margin} style}
%
% This style option is stored in the file |thm.sty|.
%    \begin{macrocode}
%<*thm>
\begingroup \makeatletter
\@ifundefined{theorem@style}{\input{theorem.sty}}{}
\ProvidesFile{thm.sty}
             [\FMithmInfo]
%    \end{macrocode}
%
% \begin{macro}{\th@margin}
%    Again this is only a variant of the theorem styles described
%    above without any new ideas.
%    \begin{macrocode}
\gdef\th@margin{\normalfont\slshape
  \def\@begintheorem##1##2{\item
       [\theorem@headerfont \llap{##2}\hskip\labelsep ##1]}%
\def\@opargbegintheorem##1##2##3{%
  \item[\theorem@headerfont \llap{##2}\hskip\labelsep ##1\ (##3)]}}
\endgroup
%</thm>
%    \end{macrocode}
% \end{macro}
%
%
%
%
% \subsection{Final Definitions}
%
% \begin{macro}{\theorempreskipamount}
% \begin{macro}{\theorempostskipamount}
%    The \textsf{skip} parameters that regulate the vertical empty space
%    before and after the {\envfont theorem} environment have
%    to be allocated as well.
% \changes{v2.0f}{89/07/20}{Unnecessary `global' removed.}
%    \begin{macrocode}
%<*package>
\newskip\theorempreskipamount
\newskip\theorempostskipamount
%    \end{macrocode}
%    Since we have used the same values for all theorem sets, we now can
%    assign them.
%    \begin{macrocode}
\global\setlength\theorempreskipamount{12pt plus 5pt minus 3pt}
\global\setlength\theorempostskipamount{8pt plus 3pt minus 1.5pt}
%    \end{macrocode}
% \end{macro}
% \end{macro}
%
%
% \begin{macro}{\@endtheorem}
%    The same holds for the macro |\@endtheorem|, which ends a
%    {\envfont theorem} environment. Since it is the same for all
%    theorem sets, it is removed from the macros |\th@|\meta{style}.
%    It simply ends the {\envfont trivlist} environment, which was
%    begun in |\@thm|.
%    \begin{macrocode}
\global\let\@endtheorem=\endtrivlist
%    \end{macrocode}
% \end{macro}
%
%
%
% \begin{macro}{\@preamblecmds}
%    All macros defined above are to be used only in the preamble.
%    Therefore, we insert them in |\@preamblecmds| which will disable
%    them at begin document. This is done by the internal \LaTeXe{}
%    command |\@onlypreamble|.
% \changes{v2.0g}{89/09/19}{`theoremheaderfont and `theorembodyfont
%    added.}
% \changes{v2.2a}{94/02/03}{Use \LaTeXe{} interface}
%    \begin{macrocode}
\@onlypreamble\@xnthm
\@onlypreamble\@ynthm
\@onlypreamble\@othm
\@onlypreamble\newtheorem
\@onlypreamble\theoremstyle
\@onlypreamble\theorembodyfont
\@onlypreamble\theoremheaderfont
%    \end{macrocode}
% \end{macro}
%
% Finally we declare the \texttt{plain} theorem style to be the default.
% \changes{v2.1b}{90/06/04}{Fixed missing `item by declaring default
%    style.}
%    \begin{macrocode}
\theoremstyle{plain}
%</package>
%    \end{macrocode}
%
% \Finale

\endinput
}{}
\ProvidesFile{thp.sty}
             [\FMithmInfo]
%    \end{macrocode}
%
%
% \begin{macro}{\th@plain}
% \begin{macro}{\@begintheorem}
% \begin{macro}{\@opargbegintheorem}
%    |\theoremstyle{plain}| corresponds to the original
%    definition, except that the distances to the surrounding text are
%    determined by the parameters |\theorempreskipamount| and
%    |\theorempostskipamount|.
%    First we set the default body font.
%    \begin{macrocode}
\gdef\th@plain{\normalfont\itshape
%    \end{macrocode}
%    Then we define |\@begintheorem| and
%    |\@opargbegintheorem|. These two macros define how the
%    header of a theorem is typeset.  |\@opargbegintheorem| will
%    be called if a \textsf{theorem} environment with an optional
%    argument is encountered; otherwise, the header is constructed by
%    calling |\@begintheorem|.  If one of these macros is
%    executed, we are within a \textsf{trivlist} environment started by
%    |\@thm|.  So the theorem header is produced with an
%    |\item| command.
%
%    Instead of specifying the header font directly, all standard
%    theorem styles use the |\theorem@headerfont| macro to allow
%    customization.  The extra space (|\labelsep|) is necessary
%    because of problems in the \textsf{trivlist} environment.
%    \begin{macrocode}
  \def\@begintheorem##1##2{%
        \item[\hskip\labelsep \theorem@headerfont ##1\ ##2]}%
%    \end{macrocode}
%    The definition of |\@opargbegintheorem| is completely
%    analogous. The only difference is the fact that there exists a
%    third argument (which is the optional parameter of the
%    environment and contains additional information about the
%    theorem).  Customarily we enclose it in parentheses.
%    \begin{macrocode}
\def\@opargbegintheorem##1##2##3{%
   \item[\hskip\labelsep \theorem@headerfont ##1\ ##2\ (##3)]}}
%    \end{macrocode}
% \end{macro}
% \end{macro}
% \end{macro}
%
% \noindent
% We conclude with an |\endgroup| to restore the
% |\catcode| of |@|.
%    \begin{macrocode}
\endgroup
%</thp>
%    \end{macrocode}
%
%
% \subsubsection{The \texttt{break} style}
%
% This style option is stored in the file |thb.sty|.
% For the next two lines see the documentation for |\th@plain| on
% page~\pageref{dok:th@plain}.
%    \begin{macrocode}
%<*thb>
\begingroup \makeatletter
\@ifundefined{theorem@style}{% \iffalse meta-comment
%
% Copyright (C) 1993-2024
%
% The LaTeX Project and any individual authors listed elsewhere
% in this file.
%
% This file is part of the Standard LaTeX `Tools Bundle'.
% -------------------------------------------------------
%
% It may be distributed and/or modified under the
% conditions of the LaTeX Project Public License, either version 1.3c
% of this license or (at your option) any later version.
% The latest version of this license is in
%    https://www.latex-project.org/lppl.txt
% and version 1.3c or later is part of all distributions of LaTeX
% version 2005/12/01 or later.
%
% The list of all files belonging to the LaTeX `Tools Bundle' is
% given in the file `manifest.txt'.
%
% \fi
%
%%
%
%
\def\FMithmInfo{2023/07/05 v2.2c Theorem extension package (FMi)}
%
% \ProvidesFile{theorem.dtx}[\FMithmInfo]
%\iffalse   % this is a METACOMMENT !
%%
%% Package `theorem' to use with LaTeX2e
%% Copyright (C) 1989-1995 Frank Mittelbach, all rights reserved.
%<+package>\@ifundefined{theorem@style}{}{\endinput}
%<+package>\NeedsTeXFormat{LaTeX2e}[1995/06/01]
%<+package>\ProvidesPackage{theorem}
%<+package>         [\FMithmInfo]
%
% \fi
%
% \changes{v1.0}{87/05/07}{First draft.}
% \changes{v1.0a}{87/05/07}{`thistheoremstyle' added.}
% \changes{v1.1a}{87/07/29}{New concept: redefine the internal
%                          \protect\LaTeX{} commands `@xnthm' etc.}
% \changes{v1.1b}{87/07/31}{`theoremfont' used to specify font.}
% \changes{v1.2a}{88/01/08}{`@endtheorem' now globally defined.}
% \changes{v1.2b}{88/01/11}{`newtheorem' only allowed in the preamble.}
% \changes{v1.3a}{88/02/14}{`theoremfont' now a toks register.}
%
% \changes{v2.0a}{88/11/16}{New implementation and new documentation.}
% \changes{v2.0a}{88/11/16}{No layout changes but internal modifications
%                           old substyles won't work.}
% \changes{v2.0b}{89/01/30}{`global' used for all definitions.}
% \changes{v2.0c}{89/06/15}{Documentation translated to English.
%                            Thanks to LLL and Nina}
% \changes{v2.1a}{90/02/15}{Support for new font selection.}
% \changes{v2.2a}{94/02/03}{Cleanup for LaTeX2e}
%
% \DoNotIndex{\@endparenv,\@ifdefinable,\@ifnextchar,\@ifundefined}
% \DoNotIndex{\@namedef,\@nameuse,\@spaces,\@tempa,\@warning}
% \DoNotIndex{\ }
% \DoNotIndex{\begingroup,\bfseries}
% \DoNotIndex{\csname}
% \DoNotIndex{\def,\do,\docdate}
% \DoNotIndex{\endcsname,\endgroup,\endinput,\endtrivlist,\expandafter}
% \DoNotIndex{\filedate,\fileversion}
% \DoNotIndex{\gdef,\global}
% \DoNotIndex{\hbox,\hskip}
% \DoNotIndex{\ignorespaces,\input,\itshape,\item}
% \DoNotIndex{\labelsep,\let,\llap}
% \DoNotIndex{\makeatletter}
% \DoNotIndex{\noexpand}
% \DoNotIndex{\refstepcounter,\relax,\rlap}
% \DoNotIndex{\setlength,\slshape,\space,\string,\strut}
% \DoNotIndex{\the,\trivlist,\typeout}
% \DoNotIndex{\vbox}
% \DoNotIndex{\xdef}
%
% \def\deflabel#1{\ttfamily #1\hfill}
% \def\deflist#1{\begin{list}{}{\settowidth\labelwidth{\ttfamily #1}%
%                               \setlength\leftmargin\labelwidth
%                               \addtolength\leftmargin\labelsep
%                               \let\makelabel\deflabel}}
% \def\enddeflist{\end{list}}
% \def\envfont{\normalfont\ttfamily}
%
% \MakeShortVerb{\|}
%
% \newcommand{\TUB}{{\sl TUGboat\/}}
%
% \GetFileInfo{theorem.dtx}
% \title{An Extension of the
%        \LaTeX{} theorem environment\thanks{This file
%        has version number \fileversion, last
%        revised \filedate.}}
% \author{Frank Mittelbach}
%
% \MaintainedByLaTeXTeam{tools}
% \maketitle
%
% \begin{abstract}
%    The macros described in this paper yield an extension of the
%    \LaTeX{} theorem mechanism. It is designed to satisfy the
%    different requirements of various journals. Thus, the
%    layout of the ``theorems'' can be manipulated by determining a
%    ``style''. This article describes not only the use, but
%    also the definition, of the necessary macros.
% \end{abstract}
%
% \section*{Preface to version 2.2}
%
% For \LaTeXe{} this package did not need any fundamental changes. I
% only modified the messages generated so that theorem layout styles
% will show up with the |\listfiles| command and cleaned the section
% on the New Font Selection Scheme since this is now included in
% \LaTeX.
%
% \section*{Preface to version 2.1}
%
% This version is identical to 2.0g described in \TUB\ 10\#3 except
% for some internal defaults which are now set depending on the used
% font selection scheme.
%
% This was done to avoid unpleasant surprises if the new font selection
% scheme is in force. For further details see
% section~\ref{sec:newfonts} and \cite{bk:GMS94}.
%
% \section{Introduction}
%
% For our purposes here, ``theorems'' are labelled enunciations,
% often set off from the main text by extra space and a font change.
% Theorems, corollaries, conjectures, definitions, and remarks
% are all instances of ``theorems''.  The ``header'' of these structures
% is composed of a label (such as \textsc{Theorem} or \textsc{Remark})
% and a number which serializes an item in the sequence of items
% with the same label.
%
% Shortly after the introduction of \LaTeX{} at the Fachbereich
% Mathematik in Mainz, the desire to manipulate the layout of
% ``theorems'' arose.  In Mainz, the following two conventions came
% into general use:
% \begin{enumerate}
%   \item The number of the theorem is shown in the margin.
%   \item There is a line break at
%         the end of the theorem header.
% \end{enumerate}
%
% Additionally, some journals require different formats which depend on
% the ``sort of theorem'': e.g.\ often remarks and definitions are set
% in |\upshape|, while |\itshape| is employed for main theorems.
%
% Confronted with these requirements, a theorem environment was
% developed in Mainz which allows separate determination of the
% layout of the ``theorems sets'', comparable to |\pagestyle|.
%
%
% \section{The user interface}
% \subsection{Defining new theorem sets}
%
% \DescribeMacro\newtheorem
% As in the original \LaTeX{} version, the command |\newtheorem|
% defines a new ``theorem set'' or ``theorem-like structure''.
% Two required arguments name the new environment and give the text
% to be typeset with each instance of the new ``set'', while
% an optional argument determines how
% the ``set'' is enumerated:
% \begin{description}
%    \item[\ttfamily \bslash newtheorem\{foo\}\{bar\}]
%       The theorem set {\envfont foo} (whose name is \texttt{bar})
%       uses its own counter.
%    \item[\ttfamily \bslash newtheorem\{foo2\}{[foo]}\{bar2\}]
%       The theorem set {\envfont foo2} (printed name \texttt{bar2})
%       uses the same counter as the theorem set \texttt{foo}.
%    \item[\ttfamily \bslash newtheorem\{foo3\}\{bar3\}{[section]}]
%       The theorem set {\envfont foo3} (printed name \texttt{bar3}) is
%       enumerated within the counter \texttt{section}, i.e.\ with every
%       new |\section| the enumeration begins again with 1, and
%       the enumeration is composed from the section-number and the
%       theorem counter itself.
% \end{description}
%
% \DescribeMacro\theoremstyle
% Additionally, the command |\theoremstyle| can define the layout
% of various, or all, theorem sets.  It should be noted that any
% theorem set defined by |\newtheorem| is typeset in the
% |\theoremstyle| that is current at the time of the definition.
% Thus, the following
% \begin{verbatim}
%    \theoremstyle{break}        \newtheorem{Cor}{Corollary}
%    \theoremstyle{plain}        \newtheorem{Exa}{Example}[section]
%\end{verbatim}                             ^^A No blank at line start
% leads to the result that the set {\envfont Cor} is formatted in the
% style \texttt{break}, while the set {\envfont Exa} and all the
% following ones are formatted in the style \texttt{plain}, unless
% another |\theoremstyle| follows.  Since the definitions installed by
% |\newtheorem| are global, one also can limit |\theoremstyle| locally
% by grouping braces.
%
% \DescribeMacro\theorembodyfont
% The choice of the font for the theorem body is completely
% independent of the chosen |\theoremstyle|; this has proven to
% be very advantageous.  For example,
% \begin{verbatim}
%   {\theorembodyfont{\upshape}          \newtheorem{Rem}{Remark}}
%\end{verbatim}
% defines a theorem set {\envfont Rem}, which will be set in
% |\upshape| in the current layout (which in our example is
% \texttt{plain}).  As with |\theoremstyle|, the |\theorembodyfont|
% chosen is that current at the time of |\newtheorem|.  If
% |\theorembodyfont| is not specified or one defines
% |\theorembodyfont{}|, then the font used will be that defined by the
% |\theoremstyle|.
%
% \DescribeMacro\theoremheaderfont It is also possible to customize
% the font used for the theorem headers.  This is, however, a global
% declaration, and therefore there should be at most one
% |\theoremheaderfont| declaration in the preamble.\footnote{If it is
% actually necessary to have different header fonts, one has to define
% new theorem styles (substituting the desired font) or specify the
% information directly in the \texttt{\bslash newtheorem} declaration
% (the unclean variant).}
%
% \DescribeMacro\theorempreskipamount
% \DescribeMacro\theorempostskipamount Two additional parameters
% affect the vertical space around the theorem environments:
% |\theorempreskipamount| and |\theorempostskipamount| define,
% respectively, the spacing before and after such an environment.
% These parameters apply for all theorem sets and can be manipulated
% with the ordinary length macros.  They are rubber lengths
% (`\textsf{skips}'), and therefore can contain \texttt{plus} and
% \texttt{minus} parts.
%
% Since the definition of theorem sets should---most sensibly---be
% placed in the preamble, we only allow installation there.
% It is therefore possible to release the memory used here after
% |\begin{document}|, in order to make room for other
% applications.
%
%
% \subsection{Existing theorem styles}
%
% The following theorem styles exist to date:
% \begin{deflist}{marginbreak}
%    \item[plain]
%       This theorem style emulates the original \LaTeX{} definition,
%       except that additionally the parameters
%       |\theorem...skipamount| are used.
%    \item[break]
%       In this style, the theorem header is followed by a line
%       break.
%    \item[marginbreak]
%       The theorem number is set in the margin, and there is a line
%       break as in \texttt{break}.
%    \item[changebreak]
%       Like \texttt{break}, but with header number and text
%       interchanged.
%    \item[change]
%       Header number and text are interchanged, without a line break.
%    \item[margin]
%       The number is set in the left margin, without a line break.
% \end{deflist}
% All styles (except \texttt{plain}) select |\slshape| as the default
% |\theorembodyfont|.
%
%
% \subsection{Examples}
%
% Given the above theorem sets {\envfont Cor}, {\envfont Exa}
% and {\envfont Rem},
% suppose that the preamble also contains the declarations:
% \begin{verbatim}
%    \theoremstyle{marginbreak}   \newtheorem{Lem}[Cor]{Lemma}
%    \theoremstyle{change}
%    \theorembodyfont{\itshape}        \newtheorem{Def}[Cor]{Definition}
%
%    \theoremheaderfont{\scshape}
%\end{verbatim}
% Then the following are some typical examples of the typeset output
% resulting from their use.
%
% \begin{Cor}
%    This is a sentence typeset in the theorem environment {\envfont
%    Cor}.
% \end{Cor}
%
% \begin{Exa}
%    This is a sentence typeset in the theorem environment {\envfont
%    Exa}.
% \end{Exa}
%
% \begin{Rem}
%    This is a sentence typeset in the theorem environment {\envfont
%    Rem}.
% \end{Rem}
%
% \begin{Lem}[Ben User]
%    This is a sentence typeset in the theorem environment {\envfont
%    Lem}.
% \end{Lem}
%
% \begin{Def}[Very impressive Definition]
%    This is a sentence typeset in the theorem environment {\envfont
%    Def}.
% \end{Def}
%
% The last two examples show the effect of the optional argument to
% a theorem environment (it is the text typeset in parentheses).
%
%
% \section{Special Considerations}
% \label{sec:newfonts}
%
% Theoremheader and body are implemented as a unit. This means that
% the |\theoremheaderfont|  will inherit characteristics of the
% |\theorembodyfont| in \LaTeXe. Thus, if for example
% |\theorembodyfont| is |\itshape| and |\theoremheaderfont|
% is |\bfseries| the font selected for the header will have the
% characteristics `bold extended italic'. If this is not desired one
% should set the |\theoremheaderfont| to something like
% \begin{verbatim}
%      \theoremheaderfont{\normalfont\bfseries}
%\end{verbatim}
% i.e.\ supplying all necessary font information explicitly.
%
%
%
% \section{Acknowledgements}
%
% The publication of this set of macros was only possible with the
% help of Christina Busse (translating the manuscript into English),
% Joachim Pense (playing the r\^ole of typist), Chris Rowley
% (looking everything over) and many others providing useful
% suggestions.
%
% \MaybeStop{%
%    \begin{thebibliography}{9}
%       \bibitem{bk:GMS94} \textsc{M.~Goossens}, \textsc{F.~Mittelbach}
%         and \textsc{A.~Samarin}.
%         \newblock The \LaTeX{} Companion.
%         \newblock
%         Addison-Wesley, Reading, Massachusetts, 1994.
%       \bibitem{latex:source} \textsc{Lamport, Leslie.} \newblock
%                               \texttt{latex.tex}, version 2.09, date
%                               Feb. 1990.
%    \end{thebibliography}
%    \PrintIndex
%    \PrintChanges}
%
%
% \section{The documentation driver file}
%
% The next bit of code contains the documentation driver file for
% \TeX{}, i.e., the file that will produce the documentation you are
% currently reading. It will be extracted from this file by the
% \texttt{docstrip} program. Since it is the first code in the file
% one can alternatively process this file directly with \LaTeXe{} to
% obtain the documentation.
%
%    \begin{macrocode}
%<*driver>
\documentclass{ltxdoc}

\usepackage{theorem}

%<+driver>% The next few lines define theorem sets which are used
%<+driver>% in the example section of the documentation.

\theoremstyle{break}       \newtheorem{Cor}{Corollary}
\theoremstyle{plain}       \newtheorem{Exa}{Example}[section]
{\theorembodyfont{\upshape}\newtheorem{Rem}{Remark}}
\theoremstyle{marginbreak} \newtheorem{Lem}[Cor]{Lemma}
\theoremstyle{change}
\theorembodyfont{\itshape} \newtheorem{Def}[Cor]{Definition}

\theoremheaderfont{\scshape}

\RecordChanges

\begin{document}
   \DocInput{theorem.dtx}
\end{document}
%</driver>
%    \end{macrocode}
%
% \section{Definition of the Macros}
%
% If the file has been loaded before, we abort immediately. If not the
% package announces itself (this is actually done at the very top of
% the file---the way it is done isn't good style so don't copy it).
% \changes{v2.0e}{89/07/19}{Spaces removed from `typeout'.}
% \changes{v2.2b}{95/11/19}{Announce moved to top of file}
%    \begin{macrocode}
%<*package>
%\@ifundefined{theorem@style}{}{\endinput}
%\def\FMithmInfo{1995/11/19 v2.2b Theorem extension package (FMi)}
%\ProvidesPackage{theorem}[\FMithmInfo]
%    \end{macrocode}
%
%
% \subsection{Definition of theorem styles and fonts}
%
% All the definitions in this file are done globally to allow
% inputting this file inside a group.
%
% \begin{macro}{\theoremstyle}
%    Before a theorem style can be installed, the chosen style must be
%    known. For that reason, we must test to see that
%    |\th@|\meta{style} is known or, more precisely, that it is
%    different from |\relax|. If the style is not known then
%    |\th@plain| is used.
%    \begin{macrocode}
\gdef\theoremstyle#1{%
   \@ifundefined{th@#1}{\@warning
          {Unknown theoremstyle `#1'. Using `plain'}%
          \theorem@style{plain}}%
%    \end{macrocode}
%    We save the theorem style to be used in the token register
%    |\theorem@style|.
%    \begin{macrocode}
      {\theorem@style{#1}}%
%    \end{macrocode}
%    Now we ``evaluate'' the theorem style: this means, we call the
%    macro |\th@|\meta{style} which will activate the relevant
%    definitions which are contained in a separate file.  This is done
%    in a group to suppress changes to the current font.  This could
%    otherwise pose problems together with the new font selection
%    scheme\footnote{When I printed the original article using the new
%    font selection scheme I ended with a document with slanted
%    typefaces (text headings and all) simply because one of the
%    theorem styles used \texttt{\bslash sl} at toplevel.} if the
%    |\th@|\meta{style} is evaluated a second time.
% \changes{v2.1a}{90/02/15}{Group added around `th@..}
% \changes{v2.1a}{90/02/15}{Reseting of `..@begintheorem removed.}
%    \begin{macrocode}
      \begingroup
        \csname th@\the\theorem@style \endcsname
      \endgroup}
%    \end{macrocode}
% \end{macro}
%
%
% \begin{macro}{\@begintheorem}
% \changes{v2.1a}{90/02/15}{Relaxed at toplevel.}
% \begin{macro}{\@opargbegintheorem}
% \changes{v2.1a}{90/02/15}{Relaxed at toplevel.}
%    We reset |\@begintheorem| and |\@opargbegintheorem| to
%    |\relax| since these commands are no longer necessary at
%    toplevel. This will save a few tokens.
%    \begin{macrocode}
\global\let\@begintheorem\relax
\global\let\@opargbegintheorem\relax
%    \end{macrocode}
% \end{macro}
% \end{macro}
%
% \begin{macro}{\theorem@style}
%    Obviously the \textsf{token} register used above has to be
%    allocated. To assure the utmost compatibility with the
%    original \LaTeX{} definition, we set the default theorem style to
%    \texttt{plain}, which implements the usual \LaTeX{} convention.
% \changes{v2.0f}{89/07/20}{Unnecessary `global' removed.}
%    \begin{macrocode}
\newtoks\theorem@style
\global\theorem@style{plain}
%    \end{macrocode}
% \end{macro}
%
% \begin{macro}{\theorembodyfont}
% \begin{macro}{\theorem@bodyfont}
%    If the |\theorembodyfont| is set by the user then it should not
%    interact with the default font set in the theorem style. When the
%    new font selection is in force this may happen if, for example,
%    the default is |\itshape| and the new |\theorembodyfont| is
%    |\sffamily|. So we add a |\reset@font| command in front of the
%    user definition.
% \changes{v2.0f}{89/07/20}{Unnecessary `global' removed.}
% \changes{v2.0g}{89/09/19}{`theoremfont to `theorembodyfont.}
% \changes{v2.1c}{92/06/22}{`reset@font added and logic changed.}
% \changes{v2.1d}{92/10/28}{Reset to empty if arg is empty}
%    \begin{macrocode}
\gdef\theorembodyfont#1{%
%    \end{macrocode}
%    We check if the argument supplied is empty and if so put nothing
%    into the |\theorem@bodyfont| token register to allow for
%    |\theorembodyfont{}| as a mean of using the default of the
%    current |\theoremstyle|.
%    \begin{macrocode}
   \def\@tempa{#1}%
   \ifx\@tempa\@empty
    \theorem@bodyfont{}%
   \else
    \theorem@bodyfont{\reset@font#1}%
   \fi
   }
\newtoks\theorem@bodyfont
\global\theorem@bodyfont{}
%    \end{macrocode}
% \end{macro}
% \end{macro}
%
%
% \begin{macro}{\theoremheaderfont}
% \changes{v2.0g}{89/09/19}{Macro added.}
%    The font for the theorem headers is handled differently because
%    this definition applies to all theorem styles.
%    \begin{macrocode}
\gdef\theoremheaderfont#1{\gdef\theorem@headerfont{#1}%
%    \end{macrocode}
%    After using the macro once it is redefined to produce an error
%    message.
%    \begin{macrocode}
       \gdef\theoremheaderfont##1{%
        \typeout{\string\theoremheaderfont\space should be used
                 only once.}}}
%    \end{macrocode}
% \end{macro}
%
% \begin{macro}{\theorem@headerfont}
% \changes{v2.0g}{89/09/19}{Macro added.}
%    To set the |\theorem@headerfont| default we first test if
%    the new fontselection scheme is in force.
% \changes{v2.1a}{90/02/15}{Support for new font selection.}
%    \begin{macrocode}
\ifx\upshape\undefined
%    \end{macrocode}
%    If not we define it to expand into |\bfseries|. We don't use
%    |\let| just in case a following style option redefines this
%    macro.
%    \begin{macrocode}
\gdef\theorem@headerfont{\bfseries}
%    \end{macrocode}
%    Otherwise we reset the current shape before calling |\bfseries|.
%    \begin{macrocode}
\else \gdef\theorem@headerfont{\normalfont\bfseries}\fi
%    \end{macrocode}
% \end{macro}
%
% \begin{macro}{\th@plain}
% \begin{macro}{\th@break}
% \begin{macro}{\th@marginbreak}
% \begin{macro}{\th@changebreak}
% \begin{macro}{\th@change}
% \begin{macro}{\th@margin}
%    The different styles are defined in macros such as
%    |\th@plain|.  Since memory space is precious in
%    ``non-Big-versions'', we have to avoid offering too many unused
%    definitions.  Therefore we define these styles in separate files
%    that can be loaded on demand.  Thus the commands themselves only
%    load these files. We use |\@input@|, a \LaTeXe{} internal command
%    that ensures that the file will be listed with |\listfiles|.
% \changes{v2.2a}{94/02/02}{Use `@input@ to load theorem layout files}
%    \begin{macrocode}
\gdef\th@plain{\@input@{thp.sty}}
\gdef\th@break{\@input@{thb.sty}}
\gdef\th@marginbreak{\@input@{thmb.sty}}
\gdef\th@changebreak{\@input@{thcb.sty}}
\gdef\th@change{\@input@{thc.sty}}
\gdef\th@margin{\@input@{thm.sty}}
%    \end{macrocode}
%    This list will be expanded when new styles become
%    available.  For testing,  just append new theorem substyles
%    as document options.
% \end{macro}
% \end{macro}
% \end{macro}
% \end{macro}
% \end{macro}
% \end{macro}
%
%
%
% \subsection{Definition of a new theorem set}
%
% As already pointed out, a new theorem environment can be defined in
% three different ways:
% \begin{verbatim}
%    \newtheorem{Lem}{Lemma}
%    \newtheorem{Lem}{Lemma}[section]
%    \newtheorem{Lem}[Theorem]{Lemma}
%\end{verbatim}
% The function of the macro |\newtheorem| is to recognize these cases
% and then to branch into one of the three macros |\@ynthm|, |\@xnthm|
% or |\@othm|.  This mechanism is adopted unchanged from
% \cite{latex:source}; the essential point here is that, for example,
% in the second case, the arguments \texttt{Lem}, \texttt{Lemma} and
% \texttt{section} are passed over to the macro |\@xnthm|.
%
% We inspect this case first because the others present fewer problems,
% and thus are easily derived from this one.
%
% \begin{macro}{\@xnthm}
%    For our example arguments, the macro |\@xnthm| must fulfill
%    the following:
%    \begin{itemize}
%       \item
%          Define a new \LaTeX-counter `Lem'
%       \item
%          reset this counter within a |\section|
%       \item
%          define the macro |\theLem|
%       \item
%          define the environment macros |\Lem| and
%          |\endLem| using the current
%          |\theoremstyle| and |\theorem@bodyfont|.
%    \end{itemize}
%    Obviously, all this should happen only if the first argument
%    of |\@xnthm| (i.e.\ \texttt{Lem} in our example) is chosen so
%    as not to conflict with any previously defined commands or
%    environments.  This test is performed by the \LaTeX{} macro
%    |\@ifdefinable|.
%    \begin{macrocode}
\gdef\@xnthm#1#2[#3]{\expandafter\@ifdefinable\csname #1\endcsname
%    \end{macrocode}
%    Therefore, the first argument of |\@ifdefinable| is the
%    expansion (in the example, |\Lem|) of
%    |\csname||#1\endcsname|.  The second argument is
%    executed only if the test has been completed successfully.
%    \begin{macrocode}
   {%
%    \end{macrocode}
%    Now we define the new counter. The names of the \LaTeX{}
%    macros employed should speak for themselves:
% \changes{v2.2b}{1995/10/07}{Use \cs{@newctr}}
% \changes{v2.2c}{1995/11/23}{Use \cs{@newctr} properly.}
%    \begin{macrocode}
    \@definecounter{#1}\@newctr{#1}[#3]%
%    \end{macrocode}
%    Using |\@newctr| will give a proper error message if the counter
%    in |#3| is not defined.
%    In defining `|\theLem|' we must generate the desired macro
%    name by use of |\expandafter| and |\csname|.
%    \begin{macrocode}
    \expandafter\xdef\csname the#1\endcsname
%    \end{macrocode}
%    An |\xdef| is used in order to make the definition global,
%    and to ensure that it contains the replacement texts of
%    |\@thmcountersep| and |\@thmcounter|.\footnote{These
%    two macros can be defined by the document style.  Their default
%    values produce a `\texttt{.}' as separation and an arabic
%    representation of the number.} However, not everything should be
%    expanded. For example, it saves space to use |\thesection|
%    instead of its---at times---lengthy expansion.
%    \begin{macrocode}
      {\expandafter \noexpand \csname the#3\endcsname
       \@thmcountersep \@thmcounter{#1}}%
%    \end{macrocode}
%    Thus with the defaults of \LaTeX{}, |\theLem| would be
%    replaced by the command sequence |\thesection.\arabic{Lem}|.
%
%    We will now look at the definition of the macro which is executed
%    at the beginning of the actual environment (in our example this
%    macro is |\Lem|). It should be noted that we use an
%    ``|\expandafter| trick'' to expand only certain parts of the
%    replacement text at the time of the definition.
%    \begin{macrocode}
    \def\@tempa{\global\@namedef{#1}}%
    \expandafter \@tempa \expandafter{%
%    \end{macrocode}
%    First, the macro that contains the current definitions of
%    |\@begintheorem| and |\@opargtheorem| should be called
%    up.  The name of this macro---as is already known---has the form
%    |\th@|\meta{theorem\ style}; therefore, it must be called by
%    \begin{macrocode}
      \csname th@\the \theorem@style
%    \end{macrocode}
%    In addition the default theorem font should be changeable, i.e.\
%    we have to insert the contents of |\theorem@bodyfont|. For
%    that reason, we expand even further, beyond |\endcsname|,
%    and thus insert the contents of the \textsf{token} register
%    |\theorem@bodyfont| in the replacement text.
%    \begin{macrocode}
            \expandafter \endcsname \the \theorem@bodyfont
%    \end{macrocode}
%    Now it is time to call the macro |\@thm| which takes over
%    the further processing.  It has two arguments: the current
%    counter name (in our example, \texttt{Lem}), and the text of the
%    label (in our example, \texttt{Lemma}).
%    \begin{macrocode}
     \@thm{#1}{#2}}%
%    \end{macrocode}
%    With this, the `sub-definition' is complete.  The macro
%    |\@endtheorem| ends a theorem environment and is, so far, nothing
%    but an |\endtrivlist|.  (Hence it is defined globally, and not
%    within the theorem styles.\footnote{This has to be changed as
%    soon as theorem styles that change \texttt{\bslash @endtheorem}
%    exist.  In such a case, all existing styles must be changed as
%    well since they will have to reset the macro.}) Therefore, we can
%    set it equivalent to the macro that ends the theorem set (in our
%    example, |\endLem|).  However, if some day theorem styles exist
%    that do change |\@endtheorem|, we would have to use the
%    commented-out line instead.
%    \begin{macrocode}
    \global \expandafter \let \csname end#1\endcsname \@endtheorem
%   \global\@namedef{end#1}{\@endtheorem}%
%    \end{macrocode}
%    With these commands all the required definitions are employed,
%    unless the test |\@ifdefinable| has failed.  Therefore, we
%    end the second argument of this macro and with it the definition
%    of |\@xnthm|.
%    \begin{macrocode}
   }}
%    \end{macrocode}
% \end{macro}
%
% \begin{macro}{\@ynthm}
%    The definition of |\@ynthm| is completely analogous. In
%    this case the new counter that is defined is not reset within
%    another counter; thus the definition of |\the...|
%    is simplified:
%    \begin{macrocode}
\gdef\@ynthm#1#2{\expandafter\@ifdefinable\csname #1\endcsname
   {\@definecounter{#1}%
    \expandafter\xdef\csname the#1\endcsname{\@thmcounter{#1}}%
%    \end{macrocode}
%    The rest of the definition corresponds literally to that of
%    |\@xnthm|:
%    \begin{macrocode}
    \def\@tempa{\global\@namedef{#1}}\expandafter \@tempa
     \expandafter{\csname th@\the \theorem@style \expandafter
     \endcsname \the\theorem@bodyfont \@thm{#1}{#2}}%
    \global \expandafter \let \csname end#1\endcsname \@endtheorem}}
%    \end{macrocode}
% \end{macro}
%
% \begin{macro}{\@othm}
%    The definition of |\@othm| does not contain anything new.
%    \begin{macrocode}
\gdef\@othm#1[#2]#3{%
%    \end{macrocode}
% \changes{v2.2b}{95/10/07}{Build in check for counter defined}
%    We do not define a new counter but instead use one that has
%    already been defined.  Thus the only definition we need is that
%    of this pseudo-counter (i.e.\ |\the|\meta{env.\ name}).
%    First we check if |#2| corresponds to a known counter name.
%    \begin{macrocode}
  \expandafter\ifx\csname c@#2\endcsname\relax
   \@nocounterr{#2}%
  \else
   \expandafter\@ifdefinable\csname #1\endcsname
   {\expandafter \xdef \csname the#1\endcsname
     {\expandafter \noexpand \csname the#2\endcsname}%
%    \end{macrocode}
%    All other parts of the definition can be adopted from
%    |\@xnthm|. We have to remember, though, that in this case
%    the name of the current counter and the theorem label have moved
%    to the second and third arguments.
%    \begin{macrocode}
    \def\@tempa{\global\@namedef{#1}}\expandafter \@tempa
     \expandafter{\csname th@\the \theorem@style \expandafter
     \endcsname \the\theorem@bodyfont \@thm{#2}{#3}}%
    \global \expandafter \let \csname end#1\endcsname \@endtheorem}%
  \fi}
%    \end{macrocode}
% \end{macro}
%
% \subsection{Macros that are employed in a {\envfont theorem}
%    environment}
%
% \begin{macro}{\@thm}
%    The macro |\@thm| has to increase the current counter.  Then,
%    depending on whether the environment has (or does not have) an
%    optional argument, it has to branch into either |\@begintheorem|
%    or |\@opargtheorem|.
%    \begin{macrocode}
\gdef\@thm#1#2{\refstepcounter{#1}%
%    \end{macrocode}
%    Now we start a {\envfont trivlist} environment, and give
%    |\@topsep| and |\@topsepadd| the values of the \textsf{skip}
%    registers |\theorempreskipamount| and |\theorempostskipamount|.
%    The value in |\@topsep| is the vertical space that is inserted by
%    the first (and only) |\item| in our |\trivlist| whilst
%    |\@topsepadd| is inserted by |\@endparenv| at the end of that
%    {\envfont trivlist} environment.  By using these registers, we
%    obtain the desired space around a {\envfont theorem} environment.
%    \begin{macrocode}
   \trivlist
   \@topsep \theorempreskipamount               % used by first \item
   \@topsepadd \theorempostskipamount           % used by \@endparenv
%    \end{macrocode}
%    Now we have to test whether an optional argument has been given.
% \changes{v2.0d}{89/07/18}{Forgotten percent added.}
%    \begin{macrocode}
   \@ifnextchar [%
%    \end{macrocode}
%    If there is an optional argument, we will call |\@ythm|, and
%    move the arguments read back into the input stream.
%    \begin{macrocode}
   {\@ythm{#1}{#2}}%
%    \end{macrocode}
%    If not, we call |\@begintheorem|. Its first argument is the
%    name of the theorem set (hence the second argument of
%    |\@thm|). Its second argument is the macro that produces the
%    current number.
%    \begin{macrocode}
   {\@begintheorem{#2}{\csname the#1\endcsname}\ignorespaces}}
%    \end{macrocode}
% \end{macro}
%
% \begin{macro}{\@xthm}
% \begin{macro}{\@ythm}
%    Both these macros were originally called by |\@thm|.  We do
%    not need |\@xthm| anymore, hence we reset it to
%    |\relax|. The definition of |\@ythm| has not changed at
%    all from its definition in \LaTeX. In order to make the macros
%    easier to understand, we will
%    nevertheless present it (commented out).
%    \begin{macrocode}
\global\let\@xthm\relax
% \def\@ythm#1#2[#3]{\@opargbegintheorem{#2}{\csname
%           the#1\endcsname}{#3}\ignorespaces}
%</package>
%    \end{macrocode}
%    The primitive |\ignorespaces| in |\@ythm| and
%    |\@thm| is needed to remove the spaces between the
%    |\begin{...}| and the actual text.
% \end{macro}
% \end{macro}
%
%
% \subsection{Definition of the theorem substyles}
%
% As already pointed out, the theorem substyles, defined below, are
% only loaded when necessary.  Note that all these substyles, except
% \texttt{plain}, have |\slshape| as the default body font.
%
% \subsubsection{The \texttt{plain} style}
%
% \label{dok:th@plain} As the following macros use |@|, we have
% to locally set the |\catcode| of this symbol to ``letter''.
% This happens within a group, so that we do not have to worry about
% which |\catcode| that symbol had before.
%    \begin{macrocode}
%<*thp>
\begingroup \makeatletter
%    \end{macrocode}
% Since we are now within a group, we must make all definitions
% globally.  First we make sure that \texttt{theorem.sty} is loaded.
% This will allow us to use this file as a document style option
% without having to call \texttt{theorem} itself as an option.  At the
% same time, we assure that at least version 2 is loaded, since
% |\theorem@style| was not defined in earlier versions.
%    \begin{macrocode}
\@ifundefined{theorem@style}{\input{theorem.sty}}{}
\ProvidesFile{thp.sty}
             [\FMithmInfo]
%    \end{macrocode}
%
%
% \begin{macro}{\th@plain}
% \begin{macro}{\@begintheorem}
% \begin{macro}{\@opargbegintheorem}
%    |\theoremstyle{plain}| corresponds to the original
%    definition, except that the distances to the surrounding text are
%    determined by the parameters |\theorempreskipamount| and
%    |\theorempostskipamount|.
%    First we set the default body font.
%    \begin{macrocode}
\gdef\th@plain{\normalfont\itshape
%    \end{macrocode}
%    Then we define |\@begintheorem| and
%    |\@opargbegintheorem|. These two macros define how the
%    header of a theorem is typeset.  |\@opargbegintheorem| will
%    be called if a \textsf{theorem} environment with an optional
%    argument is encountered; otherwise, the header is constructed by
%    calling |\@begintheorem|.  If one of these macros is
%    executed, we are within a \textsf{trivlist} environment started by
%    |\@thm|.  So the theorem header is produced with an
%    |\item| command.
%
%    Instead of specifying the header font directly, all standard
%    theorem styles use the |\theorem@headerfont| macro to allow
%    customization.  The extra space (|\labelsep|) is necessary
%    because of problems in the \textsf{trivlist} environment.
%    \begin{macrocode}
  \def\@begintheorem##1##2{%
        \item[\hskip\labelsep \theorem@headerfont ##1\ ##2]}%
%    \end{macrocode}
%    The definition of |\@opargbegintheorem| is completely
%    analogous. The only difference is the fact that there exists a
%    third argument (which is the optional parameter of the
%    environment and contains additional information about the
%    theorem).  Customarily we enclose it in parentheses.
%    \begin{macrocode}
\def\@opargbegintheorem##1##2##3{%
   \item[\hskip\labelsep \theorem@headerfont ##1\ ##2\ (##3)]}}
%    \end{macrocode}
% \end{macro}
% \end{macro}
% \end{macro}
%
% \noindent
% We conclude with an |\endgroup| to restore the
% |\catcode| of |@|.
%    \begin{macrocode}
\endgroup
%</thp>
%    \end{macrocode}
%
%
% \subsubsection{The \texttt{break} style}
%
% This style option is stored in the file |thb.sty|.
% For the next two lines see the documentation for |\th@plain| on
% page~\pageref{dok:th@plain}.
%    \begin{macrocode}
%<*thb>
\begingroup \makeatletter
\@ifundefined{theorem@style}{\input{theorem.sty}}{}
\ProvidesFile{thb.sty}
             [\FMithmInfo]
%    \end{macrocode}
%
% \begin{macro}{\th@break}
%    |\theoremstyle{break}| produces a line break after the name
%    of the theorem.  The font is |\slshape|.  Hence, we define
%    |\th@break| as follows:
%    \begin{macrocode}
\gdef\th@break{\normalfont\slshape
  \def\@begintheorem##1##2{\item[%
%    \end{macrocode}
%    We run into the following problem: it is not possible to create
%    the header with |\item[|\meta{title}|]| and then start
%    a new line by, for example, |\mbox{}\\|. Such a definition
%    will fail whenever a list environment follows immediately.  With
%    the above construction, the |\mbox{}| causes the switch
%    |@inlabel| (cf.\ definition of |\list| and
%    |\trivlist| in \cite{latex:source}) to be set to \texttt{false}
%    and so the following list will insert additional vertical
%    space (|\topskip|). This is quite annoying. Therefore, we
%    create the line break within the |\item|. In order to ensure
%    that the text will begin at the proper position in the following
%    line, we simply pretend that the label does not take any
%    room.\footnote{This will lead to problems whenever very high
%    symbols occurring in the line tower into the
%    heading. So, something else has to be done here sometime.}
%    \begin{macrocode}
       \rlap{\vbox{\hbox{\hskip \labelsep\theorem@headerfont ##1\ ##2}%
                   \hbox{\strut}}}]}%
%    \end{macrocode}
%    Again, the definition of |\@opargbegintheorem| is
%    completely analogous.
%    \begin{macrocode}
\def\@opargbegintheorem##1##2##3{%
  \item[\rlap{\vbox{\hbox{\hskip \labelsep \theorem@headerfont
                     ##1\ ##2\ (##3)}%
                    \hbox{\strut}}}]}}
\endgroup
%</thb>
%    \end{macrocode}
% \end{macro}
%
%
% \subsubsection{The \texttt{changebreak} style}
%
%    \begin{macrocode}
% This style option is stored in the file |thcb.sty|.
%    \begin{macrocode}
%<*thcb>
\begingroup \makeatletter
\@ifundefined{theorem@style}{\input{theorem.sty}}{}
\ProvidesFile{thcb.sty}
             [\FMithmInfo]
%    \end{macrocode}
%
% \begin{macro}{\th@changebreak}
%    The \texttt{change-break} theorem style is like \texttt{break}
%    but with interchange of theorem name and theorem number.  Thus we
%    define |\th@changebreak| as follows:
%    \begin{macrocode}
\gdef\th@changebreak{\normalfont\slshape
  \def\@begintheorem##1##2{\item
       [\rlap{\vbox{\hbox{\hskip\labelsep \theorem@headerfont ##2\ ##1}%
                    \hbox{\strut}}}]}%
\def\@opargbegintheorem##1##2##3{%
  \item[\rlap{\vbox{\hbox{\hskip\labelsep \theorem@headerfont
                     ##2\ ##1\ (##3)}%
                    \hbox{\strut}}}]}}
\endgroup
%</thcb>
%    \end{macrocode}
% \end{macro}
%
%
% \subsubsection{The \texttt{change} style}
%
% This style option is stored in the file |thc.sty|.
%    \begin{macrocode}
%<*thc>
\begingroup \makeatletter
\@ifundefined{theorem@style}{\input{theorem.sty}}{}
\ProvidesFile{thc.sty}
             [\FMithmInfo]
%    \end{macrocode}
%
% \begin{macro}{\th@change}
%    The \texttt{change} theorem style corresponds to the
%    \texttt{change break} style without a linebreak after the header.
%    To say it in another way, it's the same as the \texttt{plain}
%    style but with number and name interchanged and |\slshape| as the
%    default font.
%    \begin{macrocode}
\gdef\th@change{\normalfont\slshape
  \def\@begintheorem##1##2{\item
       [\hskip\labelsep \theorem@headerfont ##2\ ##1]}%
\def\@opargbegintheorem##1##2##3{%
  \item[\hskip\labelsep \theorem@headerfont ##2\ ##1\ (##3)]}}
\endgroup
%</thc>
%    \end{macrocode}
% \end{macro}
%
%
% \subsubsection{The \texttt{marginbreak} style}
%
% This style option is the one used most often at Mainz.  It is saved
% in the file |thmb.sty|.
%    \begin{macrocode}
%<*thmb>
\begingroup \makeatletter
\@ifundefined{theorem@style}{\input{theorem.sty}}{}
\ProvidesFile{thmb.sty}
             [\FMithmInfo]
%    \end{macrocode}
%
% \begin{macro}{\th@marginbreak}
%    The \texttt{margin break} style is nearly the same as the
%    \texttt{change break} style. The only difference is the placement
%    of the theorem number.  We use |\llap| to place it in the left
%    margin.
%
%    In this style |\labelsep| denotes the separation between the
%    number and the text.
%    \begin{macrocode}
\gdef\th@marginbreak{\normalfont\slshape
  \def\@begintheorem##1##2{\item
       [\rlap{\vbox{\theorem@headerfont
                    \hbox{\llap{##2}\hskip\labelsep ##1}%
                    \hbox{\strut}}}]}%
\def\@opargbegintheorem##1##2##3{%
  \item[\rlap{\vbox{\theorem@headerfont
                    \hbox{\llap{##2}\hskip\labelsep ##1\ (##3)}%
                    \hbox{\strut}}}]}}
\endgroup
%</thmb>
%    \end{macrocode}
% \changes{v2.1c}{90/02/17}{`endgroup outside macrocode env.}
% \end{macro}
%
%
%
% \subsubsection{The \texttt{margin} style}
%
% This style option is stored in the file |thm.sty|.
%    \begin{macrocode}
%<*thm>
\begingroup \makeatletter
\@ifundefined{theorem@style}{\input{theorem.sty}}{}
\ProvidesFile{thm.sty}
             [\FMithmInfo]
%    \end{macrocode}
%
% \begin{macro}{\th@margin}
%    Again this is only a variant of the theorem styles described
%    above without any new ideas.
%    \begin{macrocode}
\gdef\th@margin{\normalfont\slshape
  \def\@begintheorem##1##2{\item
       [\theorem@headerfont \llap{##2}\hskip\labelsep ##1]}%
\def\@opargbegintheorem##1##2##3{%
  \item[\theorem@headerfont \llap{##2}\hskip\labelsep ##1\ (##3)]}}
\endgroup
%</thm>
%    \end{macrocode}
% \end{macro}
%
%
%
%
% \subsection{Final Definitions}
%
% \begin{macro}{\theorempreskipamount}
% \begin{macro}{\theorempostskipamount}
%    The \textsf{skip} parameters that regulate the vertical empty space
%    before and after the {\envfont theorem} environment have
%    to be allocated as well.
% \changes{v2.0f}{89/07/20}{Unnecessary `global' removed.}
%    \begin{macrocode}
%<*package>
\newskip\theorempreskipamount
\newskip\theorempostskipamount
%    \end{macrocode}
%    Since we have used the same values for all theorem sets, we now can
%    assign them.
%    \begin{macrocode}
\global\setlength\theorempreskipamount{12pt plus 5pt minus 3pt}
\global\setlength\theorempostskipamount{8pt plus 3pt minus 1.5pt}
%    \end{macrocode}
% \end{macro}
% \end{macro}
%
%
% \begin{macro}{\@endtheorem}
%    The same holds for the macro |\@endtheorem|, which ends a
%    {\envfont theorem} environment. Since it is the same for all
%    theorem sets, it is removed from the macros |\th@|\meta{style}.
%    It simply ends the {\envfont trivlist} environment, which was
%    begun in |\@thm|.
%    \begin{macrocode}
\global\let\@endtheorem=\endtrivlist
%    \end{macrocode}
% \end{macro}
%
%
%
% \begin{macro}{\@preamblecmds}
%    All macros defined above are to be used only in the preamble.
%    Therefore, we insert them in |\@preamblecmds| which will disable
%    them at begin document. This is done by the internal \LaTeXe{}
%    command |\@onlypreamble|.
% \changes{v2.0g}{89/09/19}{`theoremheaderfont and `theorembodyfont
%    added.}
% \changes{v2.2a}{94/02/03}{Use \LaTeXe{} interface}
%    \begin{macrocode}
\@onlypreamble\@xnthm
\@onlypreamble\@ynthm
\@onlypreamble\@othm
\@onlypreamble\newtheorem
\@onlypreamble\theoremstyle
\@onlypreamble\theorembodyfont
\@onlypreamble\theoremheaderfont
%    \end{macrocode}
% \end{macro}
%
% Finally we declare the \texttt{plain} theorem style to be the default.
% \changes{v2.1b}{90/06/04}{Fixed missing `item by declaring default
%    style.}
%    \begin{macrocode}
\theoremstyle{plain}
%</package>
%    \end{macrocode}
%
% \Finale

\endinput
}{}
\ProvidesFile{thb.sty}
             [\FMithmInfo]
%    \end{macrocode}
%
% \begin{macro}{\th@break}
%    |\theoremstyle{break}| produces a line break after the name
%    of the theorem.  The font is |\slshape|.  Hence, we define
%    |\th@break| as follows:
%    \begin{macrocode}
\gdef\th@break{\normalfont\slshape
  \def\@begintheorem##1##2{\item[%
%    \end{macrocode}
%    We run into the following problem: it is not possible to create
%    the header with |\item[|\meta{title}|]| and then start
%    a new line by, for example, |\mbox{}\\|. Such a definition
%    will fail whenever a list environment follows immediately.  With
%    the above construction, the |\mbox{}| causes the switch
%    |@inlabel| (cf.\ definition of |\list| and
%    |\trivlist| in \cite{latex:source}) to be set to \texttt{false}
%    and so the following list will insert additional vertical
%    space (|\topskip|). This is quite annoying. Therefore, we
%    create the line break within the |\item|. In order to ensure
%    that the text will begin at the proper position in the following
%    line, we simply pretend that the label does not take any
%    room.\footnote{This will lead to problems whenever very high
%    symbols occurring in the line tower into the
%    heading. So, something else has to be done here sometime.}
%    \begin{macrocode}
       \rlap{\vbox{\hbox{\hskip \labelsep\theorem@headerfont ##1\ ##2}%
                   \hbox{\strut}}}]}%
%    \end{macrocode}
%    Again, the definition of |\@opargbegintheorem| is
%    completely analogous.
%    \begin{macrocode}
\def\@opargbegintheorem##1##2##3{%
  \item[\rlap{\vbox{\hbox{\hskip \labelsep \theorem@headerfont
                     ##1\ ##2\ (##3)}%
                    \hbox{\strut}}}]}}
\endgroup
%</thb>
%    \end{macrocode}
% \end{macro}
%
%
% \subsubsection{The \texttt{changebreak} style}
%
%    \begin{macrocode}
% This style option is stored in the file |thcb.sty|.
%    \begin{macrocode}
%<*thcb>
\begingroup \makeatletter
\@ifundefined{theorem@style}{% \iffalse meta-comment
%
% Copyright (C) 1993-2024
%
% The LaTeX Project and any individual authors listed elsewhere
% in this file.
%
% This file is part of the Standard LaTeX `Tools Bundle'.
% -------------------------------------------------------
%
% It may be distributed and/or modified under the
% conditions of the LaTeX Project Public License, either version 1.3c
% of this license or (at your option) any later version.
% The latest version of this license is in
%    https://www.latex-project.org/lppl.txt
% and version 1.3c or later is part of all distributions of LaTeX
% version 2005/12/01 or later.
%
% The list of all files belonging to the LaTeX `Tools Bundle' is
% given in the file `manifest.txt'.
%
% \fi
%
%%
%
%
\def\FMithmInfo{2023/07/05 v2.2c Theorem extension package (FMi)}
%
% \ProvidesFile{theorem.dtx}[\FMithmInfo]
%\iffalse   % this is a METACOMMENT !
%%
%% Package `theorem' to use with LaTeX2e
%% Copyright (C) 1989-1995 Frank Mittelbach, all rights reserved.
%<+package>\@ifundefined{theorem@style}{}{\endinput}
%<+package>\NeedsTeXFormat{LaTeX2e}[1995/06/01]
%<+package>\ProvidesPackage{theorem}
%<+package>         [\FMithmInfo]
%
% \fi
%
% \changes{v1.0}{87/05/07}{First draft.}
% \changes{v1.0a}{87/05/07}{`thistheoremstyle' added.}
% \changes{v1.1a}{87/07/29}{New concept: redefine the internal
%                          \protect\LaTeX{} commands `@xnthm' etc.}
% \changes{v1.1b}{87/07/31}{`theoremfont' used to specify font.}
% \changes{v1.2a}{88/01/08}{`@endtheorem' now globally defined.}
% \changes{v1.2b}{88/01/11}{`newtheorem' only allowed in the preamble.}
% \changes{v1.3a}{88/02/14}{`theoremfont' now a toks register.}
%
% \changes{v2.0a}{88/11/16}{New implementation and new documentation.}
% \changes{v2.0a}{88/11/16}{No layout changes but internal modifications
%                           old substyles won't work.}
% \changes{v2.0b}{89/01/30}{`global' used for all definitions.}
% \changes{v2.0c}{89/06/15}{Documentation translated to English.
%                            Thanks to LLL and Nina}
% \changes{v2.1a}{90/02/15}{Support for new font selection.}
% \changes{v2.2a}{94/02/03}{Cleanup for LaTeX2e}
%
% \DoNotIndex{\@endparenv,\@ifdefinable,\@ifnextchar,\@ifundefined}
% \DoNotIndex{\@namedef,\@nameuse,\@spaces,\@tempa,\@warning}
% \DoNotIndex{\ }
% \DoNotIndex{\begingroup,\bfseries}
% \DoNotIndex{\csname}
% \DoNotIndex{\def,\do,\docdate}
% \DoNotIndex{\endcsname,\endgroup,\endinput,\endtrivlist,\expandafter}
% \DoNotIndex{\filedate,\fileversion}
% \DoNotIndex{\gdef,\global}
% \DoNotIndex{\hbox,\hskip}
% \DoNotIndex{\ignorespaces,\input,\itshape,\item}
% \DoNotIndex{\labelsep,\let,\llap}
% \DoNotIndex{\makeatletter}
% \DoNotIndex{\noexpand}
% \DoNotIndex{\refstepcounter,\relax,\rlap}
% \DoNotIndex{\setlength,\slshape,\space,\string,\strut}
% \DoNotIndex{\the,\trivlist,\typeout}
% \DoNotIndex{\vbox}
% \DoNotIndex{\xdef}
%
% \def\deflabel#1{\ttfamily #1\hfill}
% \def\deflist#1{\begin{list}{}{\settowidth\labelwidth{\ttfamily #1}%
%                               \setlength\leftmargin\labelwidth
%                               \addtolength\leftmargin\labelsep
%                               \let\makelabel\deflabel}}
% \def\enddeflist{\end{list}}
% \def\envfont{\normalfont\ttfamily}
%
% \MakeShortVerb{\|}
%
% \newcommand{\TUB}{{\sl TUGboat\/}}
%
% \GetFileInfo{theorem.dtx}
% \title{An Extension of the
%        \LaTeX{} theorem environment\thanks{This file
%        has version number \fileversion, last
%        revised \filedate.}}
% \author{Frank Mittelbach}
%
% \MaintainedByLaTeXTeam{tools}
% \maketitle
%
% \begin{abstract}
%    The macros described in this paper yield an extension of the
%    \LaTeX{} theorem mechanism. It is designed to satisfy the
%    different requirements of various journals. Thus, the
%    layout of the ``theorems'' can be manipulated by determining a
%    ``style''. This article describes not only the use, but
%    also the definition, of the necessary macros.
% \end{abstract}
%
% \section*{Preface to version 2.2}
%
% For \LaTeXe{} this package did not need any fundamental changes. I
% only modified the messages generated so that theorem layout styles
% will show up with the |\listfiles| command and cleaned the section
% on the New Font Selection Scheme since this is now included in
% \LaTeX.
%
% \section*{Preface to version 2.1}
%
% This version is identical to 2.0g described in \TUB\ 10\#3 except
% for some internal defaults which are now set depending on the used
% font selection scheme.
%
% This was done to avoid unpleasant surprises if the new font selection
% scheme is in force. For further details see
% section~\ref{sec:newfonts} and \cite{bk:GMS94}.
%
% \section{Introduction}
%
% For our purposes here, ``theorems'' are labelled enunciations,
% often set off from the main text by extra space and a font change.
% Theorems, corollaries, conjectures, definitions, and remarks
% are all instances of ``theorems''.  The ``header'' of these structures
% is composed of a label (such as \textsc{Theorem} or \textsc{Remark})
% and a number which serializes an item in the sequence of items
% with the same label.
%
% Shortly after the introduction of \LaTeX{} at the Fachbereich
% Mathematik in Mainz, the desire to manipulate the layout of
% ``theorems'' arose.  In Mainz, the following two conventions came
% into general use:
% \begin{enumerate}
%   \item The number of the theorem is shown in the margin.
%   \item There is a line break at
%         the end of the theorem header.
% \end{enumerate}
%
% Additionally, some journals require different formats which depend on
% the ``sort of theorem'': e.g.\ often remarks and definitions are set
% in |\upshape|, while |\itshape| is employed for main theorems.
%
% Confronted with these requirements, a theorem environment was
% developed in Mainz which allows separate determination of the
% layout of the ``theorems sets'', comparable to |\pagestyle|.
%
%
% \section{The user interface}
% \subsection{Defining new theorem sets}
%
% \DescribeMacro\newtheorem
% As in the original \LaTeX{} version, the command |\newtheorem|
% defines a new ``theorem set'' or ``theorem-like structure''.
% Two required arguments name the new environment and give the text
% to be typeset with each instance of the new ``set'', while
% an optional argument determines how
% the ``set'' is enumerated:
% \begin{description}
%    \item[\ttfamily \bslash newtheorem\{foo\}\{bar\}]
%       The theorem set {\envfont foo} (whose name is \texttt{bar})
%       uses its own counter.
%    \item[\ttfamily \bslash newtheorem\{foo2\}{[foo]}\{bar2\}]
%       The theorem set {\envfont foo2} (printed name \texttt{bar2})
%       uses the same counter as the theorem set \texttt{foo}.
%    \item[\ttfamily \bslash newtheorem\{foo3\}\{bar3\}{[section]}]
%       The theorem set {\envfont foo3} (printed name \texttt{bar3}) is
%       enumerated within the counter \texttt{section}, i.e.\ with every
%       new |\section| the enumeration begins again with 1, and
%       the enumeration is composed from the section-number and the
%       theorem counter itself.
% \end{description}
%
% \DescribeMacro\theoremstyle
% Additionally, the command |\theoremstyle| can define the layout
% of various, or all, theorem sets.  It should be noted that any
% theorem set defined by |\newtheorem| is typeset in the
% |\theoremstyle| that is current at the time of the definition.
% Thus, the following
% \begin{verbatim}
%    \theoremstyle{break}        \newtheorem{Cor}{Corollary}
%    \theoremstyle{plain}        \newtheorem{Exa}{Example}[section]
%\end{verbatim}                             ^^A No blank at line start
% leads to the result that the set {\envfont Cor} is formatted in the
% style \texttt{break}, while the set {\envfont Exa} and all the
% following ones are formatted in the style \texttt{plain}, unless
% another |\theoremstyle| follows.  Since the definitions installed by
% |\newtheorem| are global, one also can limit |\theoremstyle| locally
% by grouping braces.
%
% \DescribeMacro\theorembodyfont
% The choice of the font for the theorem body is completely
% independent of the chosen |\theoremstyle|; this has proven to
% be very advantageous.  For example,
% \begin{verbatim}
%   {\theorembodyfont{\upshape}          \newtheorem{Rem}{Remark}}
%\end{verbatim}
% defines a theorem set {\envfont Rem}, which will be set in
% |\upshape| in the current layout (which in our example is
% \texttt{plain}).  As with |\theoremstyle|, the |\theorembodyfont|
% chosen is that current at the time of |\newtheorem|.  If
% |\theorembodyfont| is not specified or one defines
% |\theorembodyfont{}|, then the font used will be that defined by the
% |\theoremstyle|.
%
% \DescribeMacro\theoremheaderfont It is also possible to customize
% the font used for the theorem headers.  This is, however, a global
% declaration, and therefore there should be at most one
% |\theoremheaderfont| declaration in the preamble.\footnote{If it is
% actually necessary to have different header fonts, one has to define
% new theorem styles (substituting the desired font) or specify the
% information directly in the \texttt{\bslash newtheorem} declaration
% (the unclean variant).}
%
% \DescribeMacro\theorempreskipamount
% \DescribeMacro\theorempostskipamount Two additional parameters
% affect the vertical space around the theorem environments:
% |\theorempreskipamount| and |\theorempostskipamount| define,
% respectively, the spacing before and after such an environment.
% These parameters apply for all theorem sets and can be manipulated
% with the ordinary length macros.  They are rubber lengths
% (`\textsf{skips}'), and therefore can contain \texttt{plus} and
% \texttt{minus} parts.
%
% Since the definition of theorem sets should---most sensibly---be
% placed in the preamble, we only allow installation there.
% It is therefore possible to release the memory used here after
% |\begin{document}|, in order to make room for other
% applications.
%
%
% \subsection{Existing theorem styles}
%
% The following theorem styles exist to date:
% \begin{deflist}{marginbreak}
%    \item[plain]
%       This theorem style emulates the original \LaTeX{} definition,
%       except that additionally the parameters
%       |\theorem...skipamount| are used.
%    \item[break]
%       In this style, the theorem header is followed by a line
%       break.
%    \item[marginbreak]
%       The theorem number is set in the margin, and there is a line
%       break as in \texttt{break}.
%    \item[changebreak]
%       Like \texttt{break}, but with header number and text
%       interchanged.
%    \item[change]
%       Header number and text are interchanged, without a line break.
%    \item[margin]
%       The number is set in the left margin, without a line break.
% \end{deflist}
% All styles (except \texttt{plain}) select |\slshape| as the default
% |\theorembodyfont|.
%
%
% \subsection{Examples}
%
% Given the above theorem sets {\envfont Cor}, {\envfont Exa}
% and {\envfont Rem},
% suppose that the preamble also contains the declarations:
% \begin{verbatim}
%    \theoremstyle{marginbreak}   \newtheorem{Lem}[Cor]{Lemma}
%    \theoremstyle{change}
%    \theorembodyfont{\itshape}        \newtheorem{Def}[Cor]{Definition}
%
%    \theoremheaderfont{\scshape}
%\end{verbatim}
% Then the following are some typical examples of the typeset output
% resulting from their use.
%
% \begin{Cor}
%    This is a sentence typeset in the theorem environment {\envfont
%    Cor}.
% \end{Cor}
%
% \begin{Exa}
%    This is a sentence typeset in the theorem environment {\envfont
%    Exa}.
% \end{Exa}
%
% \begin{Rem}
%    This is a sentence typeset in the theorem environment {\envfont
%    Rem}.
% \end{Rem}
%
% \begin{Lem}[Ben User]
%    This is a sentence typeset in the theorem environment {\envfont
%    Lem}.
% \end{Lem}
%
% \begin{Def}[Very impressive Definition]
%    This is a sentence typeset in the theorem environment {\envfont
%    Def}.
% \end{Def}
%
% The last two examples show the effect of the optional argument to
% a theorem environment (it is the text typeset in parentheses).
%
%
% \section{Special Considerations}
% \label{sec:newfonts}
%
% Theoremheader and body are implemented as a unit. This means that
% the |\theoremheaderfont|  will inherit characteristics of the
% |\theorembodyfont| in \LaTeXe. Thus, if for example
% |\theorembodyfont| is |\itshape| and |\theoremheaderfont|
% is |\bfseries| the font selected for the header will have the
% characteristics `bold extended italic'. If this is not desired one
% should set the |\theoremheaderfont| to something like
% \begin{verbatim}
%      \theoremheaderfont{\normalfont\bfseries}
%\end{verbatim}
% i.e.\ supplying all necessary font information explicitly.
%
%
%
% \section{Acknowledgements}
%
% The publication of this set of macros was only possible with the
% help of Christina Busse (translating the manuscript into English),
% Joachim Pense (playing the r\^ole of typist), Chris Rowley
% (looking everything over) and many others providing useful
% suggestions.
%
% \MaybeStop{%
%    \begin{thebibliography}{9}
%       \bibitem{bk:GMS94} \textsc{M.~Goossens}, \textsc{F.~Mittelbach}
%         and \textsc{A.~Samarin}.
%         \newblock The \LaTeX{} Companion.
%         \newblock
%         Addison-Wesley, Reading, Massachusetts, 1994.
%       \bibitem{latex:source} \textsc{Lamport, Leslie.} \newblock
%                               \texttt{latex.tex}, version 2.09, date
%                               Feb. 1990.
%    \end{thebibliography}
%    \PrintIndex
%    \PrintChanges}
%
%
% \section{The documentation driver file}
%
% The next bit of code contains the documentation driver file for
% \TeX{}, i.e., the file that will produce the documentation you are
% currently reading. It will be extracted from this file by the
% \texttt{docstrip} program. Since it is the first code in the file
% one can alternatively process this file directly with \LaTeXe{} to
% obtain the documentation.
%
%    \begin{macrocode}
%<*driver>
\documentclass{ltxdoc}

\usepackage{theorem}

%<+driver>% The next few lines define theorem sets which are used
%<+driver>% in the example section of the documentation.

\theoremstyle{break}       \newtheorem{Cor}{Corollary}
\theoremstyle{plain}       \newtheorem{Exa}{Example}[section]
{\theorembodyfont{\upshape}\newtheorem{Rem}{Remark}}
\theoremstyle{marginbreak} \newtheorem{Lem}[Cor]{Lemma}
\theoremstyle{change}
\theorembodyfont{\itshape} \newtheorem{Def}[Cor]{Definition}

\theoremheaderfont{\scshape}

\RecordChanges

\begin{document}
   \DocInput{theorem.dtx}
\end{document}
%</driver>
%    \end{macrocode}
%
% \section{Definition of the Macros}
%
% If the file has been loaded before, we abort immediately. If not the
% package announces itself (this is actually done at the very top of
% the file---the way it is done isn't good style so don't copy it).
% \changes{v2.0e}{89/07/19}{Spaces removed from `typeout'.}
% \changes{v2.2b}{95/11/19}{Announce moved to top of file}
%    \begin{macrocode}
%<*package>
%\@ifundefined{theorem@style}{}{\endinput}
%\def\FMithmInfo{1995/11/19 v2.2b Theorem extension package (FMi)}
%\ProvidesPackage{theorem}[\FMithmInfo]
%    \end{macrocode}
%
%
% \subsection{Definition of theorem styles and fonts}
%
% All the definitions in this file are done globally to allow
% inputting this file inside a group.
%
% \begin{macro}{\theoremstyle}
%    Before a theorem style can be installed, the chosen style must be
%    known. For that reason, we must test to see that
%    |\th@|\meta{style} is known or, more precisely, that it is
%    different from |\relax|. If the style is not known then
%    |\th@plain| is used.
%    \begin{macrocode}
\gdef\theoremstyle#1{%
   \@ifundefined{th@#1}{\@warning
          {Unknown theoremstyle `#1'. Using `plain'}%
          \theorem@style{plain}}%
%    \end{macrocode}
%    We save the theorem style to be used in the token register
%    |\theorem@style|.
%    \begin{macrocode}
      {\theorem@style{#1}}%
%    \end{macrocode}
%    Now we ``evaluate'' the theorem style: this means, we call the
%    macro |\th@|\meta{style} which will activate the relevant
%    definitions which are contained in a separate file.  This is done
%    in a group to suppress changes to the current font.  This could
%    otherwise pose problems together with the new font selection
%    scheme\footnote{When I printed the original article using the new
%    font selection scheme I ended with a document with slanted
%    typefaces (text headings and all) simply because one of the
%    theorem styles used \texttt{\bslash sl} at toplevel.} if the
%    |\th@|\meta{style} is evaluated a second time.
% \changes{v2.1a}{90/02/15}{Group added around `th@..}
% \changes{v2.1a}{90/02/15}{Reseting of `..@begintheorem removed.}
%    \begin{macrocode}
      \begingroup
        \csname th@\the\theorem@style \endcsname
      \endgroup}
%    \end{macrocode}
% \end{macro}
%
%
% \begin{macro}{\@begintheorem}
% \changes{v2.1a}{90/02/15}{Relaxed at toplevel.}
% \begin{macro}{\@opargbegintheorem}
% \changes{v2.1a}{90/02/15}{Relaxed at toplevel.}
%    We reset |\@begintheorem| and |\@opargbegintheorem| to
%    |\relax| since these commands are no longer necessary at
%    toplevel. This will save a few tokens.
%    \begin{macrocode}
\global\let\@begintheorem\relax
\global\let\@opargbegintheorem\relax
%    \end{macrocode}
% \end{macro}
% \end{macro}
%
% \begin{macro}{\theorem@style}
%    Obviously the \textsf{token} register used above has to be
%    allocated. To assure the utmost compatibility with the
%    original \LaTeX{} definition, we set the default theorem style to
%    \texttt{plain}, which implements the usual \LaTeX{} convention.
% \changes{v2.0f}{89/07/20}{Unnecessary `global' removed.}
%    \begin{macrocode}
\newtoks\theorem@style
\global\theorem@style{plain}
%    \end{macrocode}
% \end{macro}
%
% \begin{macro}{\theorembodyfont}
% \begin{macro}{\theorem@bodyfont}
%    If the |\theorembodyfont| is set by the user then it should not
%    interact with the default font set in the theorem style. When the
%    new font selection is in force this may happen if, for example,
%    the default is |\itshape| and the new |\theorembodyfont| is
%    |\sffamily|. So we add a |\reset@font| command in front of the
%    user definition.
% \changes{v2.0f}{89/07/20}{Unnecessary `global' removed.}
% \changes{v2.0g}{89/09/19}{`theoremfont to `theorembodyfont.}
% \changes{v2.1c}{92/06/22}{`reset@font added and logic changed.}
% \changes{v2.1d}{92/10/28}{Reset to empty if arg is empty}
%    \begin{macrocode}
\gdef\theorembodyfont#1{%
%    \end{macrocode}
%    We check if the argument supplied is empty and if so put nothing
%    into the |\theorem@bodyfont| token register to allow for
%    |\theorembodyfont{}| as a mean of using the default of the
%    current |\theoremstyle|.
%    \begin{macrocode}
   \def\@tempa{#1}%
   \ifx\@tempa\@empty
    \theorem@bodyfont{}%
   \else
    \theorem@bodyfont{\reset@font#1}%
   \fi
   }
\newtoks\theorem@bodyfont
\global\theorem@bodyfont{}
%    \end{macrocode}
% \end{macro}
% \end{macro}
%
%
% \begin{macro}{\theoremheaderfont}
% \changes{v2.0g}{89/09/19}{Macro added.}
%    The font for the theorem headers is handled differently because
%    this definition applies to all theorem styles.
%    \begin{macrocode}
\gdef\theoremheaderfont#1{\gdef\theorem@headerfont{#1}%
%    \end{macrocode}
%    After using the macro once it is redefined to produce an error
%    message.
%    \begin{macrocode}
       \gdef\theoremheaderfont##1{%
        \typeout{\string\theoremheaderfont\space should be used
                 only once.}}}
%    \end{macrocode}
% \end{macro}
%
% \begin{macro}{\theorem@headerfont}
% \changes{v2.0g}{89/09/19}{Macro added.}
%    To set the |\theorem@headerfont| default we first test if
%    the new fontselection scheme is in force.
% \changes{v2.1a}{90/02/15}{Support for new font selection.}
%    \begin{macrocode}
\ifx\upshape\undefined
%    \end{macrocode}
%    If not we define it to expand into |\bfseries|. We don't use
%    |\let| just in case a following style option redefines this
%    macro.
%    \begin{macrocode}
\gdef\theorem@headerfont{\bfseries}
%    \end{macrocode}
%    Otherwise we reset the current shape before calling |\bfseries|.
%    \begin{macrocode}
\else \gdef\theorem@headerfont{\normalfont\bfseries}\fi
%    \end{macrocode}
% \end{macro}
%
% \begin{macro}{\th@plain}
% \begin{macro}{\th@break}
% \begin{macro}{\th@marginbreak}
% \begin{macro}{\th@changebreak}
% \begin{macro}{\th@change}
% \begin{macro}{\th@margin}
%    The different styles are defined in macros such as
%    |\th@plain|.  Since memory space is precious in
%    ``non-Big-versions'', we have to avoid offering too many unused
%    definitions.  Therefore we define these styles in separate files
%    that can be loaded on demand.  Thus the commands themselves only
%    load these files. We use |\@input@|, a \LaTeXe{} internal command
%    that ensures that the file will be listed with |\listfiles|.
% \changes{v2.2a}{94/02/02}{Use `@input@ to load theorem layout files}
%    \begin{macrocode}
\gdef\th@plain{\@input@{thp.sty}}
\gdef\th@break{\@input@{thb.sty}}
\gdef\th@marginbreak{\@input@{thmb.sty}}
\gdef\th@changebreak{\@input@{thcb.sty}}
\gdef\th@change{\@input@{thc.sty}}
\gdef\th@margin{\@input@{thm.sty}}
%    \end{macrocode}
%    This list will be expanded when new styles become
%    available.  For testing,  just append new theorem substyles
%    as document options.
% \end{macro}
% \end{macro}
% \end{macro}
% \end{macro}
% \end{macro}
% \end{macro}
%
%
%
% \subsection{Definition of a new theorem set}
%
% As already pointed out, a new theorem environment can be defined in
% three different ways:
% \begin{verbatim}
%    \newtheorem{Lem}{Lemma}
%    \newtheorem{Lem}{Lemma}[section]
%    \newtheorem{Lem}[Theorem]{Lemma}
%\end{verbatim}
% The function of the macro |\newtheorem| is to recognize these cases
% and then to branch into one of the three macros |\@ynthm|, |\@xnthm|
% or |\@othm|.  This mechanism is adopted unchanged from
% \cite{latex:source}; the essential point here is that, for example,
% in the second case, the arguments \texttt{Lem}, \texttt{Lemma} and
% \texttt{section} are passed over to the macro |\@xnthm|.
%
% We inspect this case first because the others present fewer problems,
% and thus are easily derived from this one.
%
% \begin{macro}{\@xnthm}
%    For our example arguments, the macro |\@xnthm| must fulfill
%    the following:
%    \begin{itemize}
%       \item
%          Define a new \LaTeX-counter `Lem'
%       \item
%          reset this counter within a |\section|
%       \item
%          define the macro |\theLem|
%       \item
%          define the environment macros |\Lem| and
%          |\endLem| using the current
%          |\theoremstyle| and |\theorem@bodyfont|.
%    \end{itemize}
%    Obviously, all this should happen only if the first argument
%    of |\@xnthm| (i.e.\ \texttt{Lem} in our example) is chosen so
%    as not to conflict with any previously defined commands or
%    environments.  This test is performed by the \LaTeX{} macro
%    |\@ifdefinable|.
%    \begin{macrocode}
\gdef\@xnthm#1#2[#3]{\expandafter\@ifdefinable\csname #1\endcsname
%    \end{macrocode}
%    Therefore, the first argument of |\@ifdefinable| is the
%    expansion (in the example, |\Lem|) of
%    |\csname||#1\endcsname|.  The second argument is
%    executed only if the test has been completed successfully.
%    \begin{macrocode}
   {%
%    \end{macrocode}
%    Now we define the new counter. The names of the \LaTeX{}
%    macros employed should speak for themselves:
% \changes{v2.2b}{1995/10/07}{Use \cs{@newctr}}
% \changes{v2.2c}{1995/11/23}{Use \cs{@newctr} properly.}
%    \begin{macrocode}
    \@definecounter{#1}\@newctr{#1}[#3]%
%    \end{macrocode}
%    Using |\@newctr| will give a proper error message if the counter
%    in |#3| is not defined.
%    In defining `|\theLem|' we must generate the desired macro
%    name by use of |\expandafter| and |\csname|.
%    \begin{macrocode}
    \expandafter\xdef\csname the#1\endcsname
%    \end{macrocode}
%    An |\xdef| is used in order to make the definition global,
%    and to ensure that it contains the replacement texts of
%    |\@thmcountersep| and |\@thmcounter|.\footnote{These
%    two macros can be defined by the document style.  Their default
%    values produce a `\texttt{.}' as separation and an arabic
%    representation of the number.} However, not everything should be
%    expanded. For example, it saves space to use |\thesection|
%    instead of its---at times---lengthy expansion.
%    \begin{macrocode}
      {\expandafter \noexpand \csname the#3\endcsname
       \@thmcountersep \@thmcounter{#1}}%
%    \end{macrocode}
%    Thus with the defaults of \LaTeX{}, |\theLem| would be
%    replaced by the command sequence |\thesection.\arabic{Lem}|.
%
%    We will now look at the definition of the macro which is executed
%    at the beginning of the actual environment (in our example this
%    macro is |\Lem|). It should be noted that we use an
%    ``|\expandafter| trick'' to expand only certain parts of the
%    replacement text at the time of the definition.
%    \begin{macrocode}
    \def\@tempa{\global\@namedef{#1}}%
    \expandafter \@tempa \expandafter{%
%    \end{macrocode}
%    First, the macro that contains the current definitions of
%    |\@begintheorem| and |\@opargtheorem| should be called
%    up.  The name of this macro---as is already known---has the form
%    |\th@|\meta{theorem\ style}; therefore, it must be called by
%    \begin{macrocode}
      \csname th@\the \theorem@style
%    \end{macrocode}
%    In addition the default theorem font should be changeable, i.e.\
%    we have to insert the contents of |\theorem@bodyfont|. For
%    that reason, we expand even further, beyond |\endcsname|,
%    and thus insert the contents of the \textsf{token} register
%    |\theorem@bodyfont| in the replacement text.
%    \begin{macrocode}
            \expandafter \endcsname \the \theorem@bodyfont
%    \end{macrocode}
%    Now it is time to call the macro |\@thm| which takes over
%    the further processing.  It has two arguments: the current
%    counter name (in our example, \texttt{Lem}), and the text of the
%    label (in our example, \texttt{Lemma}).
%    \begin{macrocode}
     \@thm{#1}{#2}}%
%    \end{macrocode}
%    With this, the `sub-definition' is complete.  The macro
%    |\@endtheorem| ends a theorem environment and is, so far, nothing
%    but an |\endtrivlist|.  (Hence it is defined globally, and not
%    within the theorem styles.\footnote{This has to be changed as
%    soon as theorem styles that change \texttt{\bslash @endtheorem}
%    exist.  In such a case, all existing styles must be changed as
%    well since they will have to reset the macro.}) Therefore, we can
%    set it equivalent to the macro that ends the theorem set (in our
%    example, |\endLem|).  However, if some day theorem styles exist
%    that do change |\@endtheorem|, we would have to use the
%    commented-out line instead.
%    \begin{macrocode}
    \global \expandafter \let \csname end#1\endcsname \@endtheorem
%   \global\@namedef{end#1}{\@endtheorem}%
%    \end{macrocode}
%    With these commands all the required definitions are employed,
%    unless the test |\@ifdefinable| has failed.  Therefore, we
%    end the second argument of this macro and with it the definition
%    of |\@xnthm|.
%    \begin{macrocode}
   }}
%    \end{macrocode}
% \end{macro}
%
% \begin{macro}{\@ynthm}
%    The definition of |\@ynthm| is completely analogous. In
%    this case the new counter that is defined is not reset within
%    another counter; thus the definition of |\the...|
%    is simplified:
%    \begin{macrocode}
\gdef\@ynthm#1#2{\expandafter\@ifdefinable\csname #1\endcsname
   {\@definecounter{#1}%
    \expandafter\xdef\csname the#1\endcsname{\@thmcounter{#1}}%
%    \end{macrocode}
%    The rest of the definition corresponds literally to that of
%    |\@xnthm|:
%    \begin{macrocode}
    \def\@tempa{\global\@namedef{#1}}\expandafter \@tempa
     \expandafter{\csname th@\the \theorem@style \expandafter
     \endcsname \the\theorem@bodyfont \@thm{#1}{#2}}%
    \global \expandafter \let \csname end#1\endcsname \@endtheorem}}
%    \end{macrocode}
% \end{macro}
%
% \begin{macro}{\@othm}
%    The definition of |\@othm| does not contain anything new.
%    \begin{macrocode}
\gdef\@othm#1[#2]#3{%
%    \end{macrocode}
% \changes{v2.2b}{95/10/07}{Build in check for counter defined}
%    We do not define a new counter but instead use one that has
%    already been defined.  Thus the only definition we need is that
%    of this pseudo-counter (i.e.\ |\the|\meta{env.\ name}).
%    First we check if |#2| corresponds to a known counter name.
%    \begin{macrocode}
  \expandafter\ifx\csname c@#2\endcsname\relax
   \@nocounterr{#2}%
  \else
   \expandafter\@ifdefinable\csname #1\endcsname
   {\expandafter \xdef \csname the#1\endcsname
     {\expandafter \noexpand \csname the#2\endcsname}%
%    \end{macrocode}
%    All other parts of the definition can be adopted from
%    |\@xnthm|. We have to remember, though, that in this case
%    the name of the current counter and the theorem label have moved
%    to the second and third arguments.
%    \begin{macrocode}
    \def\@tempa{\global\@namedef{#1}}\expandafter \@tempa
     \expandafter{\csname th@\the \theorem@style \expandafter
     \endcsname \the\theorem@bodyfont \@thm{#2}{#3}}%
    \global \expandafter \let \csname end#1\endcsname \@endtheorem}%
  \fi}
%    \end{macrocode}
% \end{macro}
%
% \subsection{Macros that are employed in a {\envfont theorem}
%    environment}
%
% \begin{macro}{\@thm}
%    The macro |\@thm| has to increase the current counter.  Then,
%    depending on whether the environment has (or does not have) an
%    optional argument, it has to branch into either |\@begintheorem|
%    or |\@opargtheorem|.
%    \begin{macrocode}
\gdef\@thm#1#2{\refstepcounter{#1}%
%    \end{macrocode}
%    Now we start a {\envfont trivlist} environment, and give
%    |\@topsep| and |\@topsepadd| the values of the \textsf{skip}
%    registers |\theorempreskipamount| and |\theorempostskipamount|.
%    The value in |\@topsep| is the vertical space that is inserted by
%    the first (and only) |\item| in our |\trivlist| whilst
%    |\@topsepadd| is inserted by |\@endparenv| at the end of that
%    {\envfont trivlist} environment.  By using these registers, we
%    obtain the desired space around a {\envfont theorem} environment.
%    \begin{macrocode}
   \trivlist
   \@topsep \theorempreskipamount               % used by first \item
   \@topsepadd \theorempostskipamount           % used by \@endparenv
%    \end{macrocode}
%    Now we have to test whether an optional argument has been given.
% \changes{v2.0d}{89/07/18}{Forgotten percent added.}
%    \begin{macrocode}
   \@ifnextchar [%
%    \end{macrocode}
%    If there is an optional argument, we will call |\@ythm|, and
%    move the arguments read back into the input stream.
%    \begin{macrocode}
   {\@ythm{#1}{#2}}%
%    \end{macrocode}
%    If not, we call |\@begintheorem|. Its first argument is the
%    name of the theorem set (hence the second argument of
%    |\@thm|). Its second argument is the macro that produces the
%    current number.
%    \begin{macrocode}
   {\@begintheorem{#2}{\csname the#1\endcsname}\ignorespaces}}
%    \end{macrocode}
% \end{macro}
%
% \begin{macro}{\@xthm}
% \begin{macro}{\@ythm}
%    Both these macros were originally called by |\@thm|.  We do
%    not need |\@xthm| anymore, hence we reset it to
%    |\relax|. The definition of |\@ythm| has not changed at
%    all from its definition in \LaTeX. In order to make the macros
%    easier to understand, we will
%    nevertheless present it (commented out).
%    \begin{macrocode}
\global\let\@xthm\relax
% \def\@ythm#1#2[#3]{\@opargbegintheorem{#2}{\csname
%           the#1\endcsname}{#3}\ignorespaces}
%</package>
%    \end{macrocode}
%    The primitive |\ignorespaces| in |\@ythm| and
%    |\@thm| is needed to remove the spaces between the
%    |\begin{...}| and the actual text.
% \end{macro}
% \end{macro}
%
%
% \subsection{Definition of the theorem substyles}
%
% As already pointed out, the theorem substyles, defined below, are
% only loaded when necessary.  Note that all these substyles, except
% \texttt{plain}, have |\slshape| as the default body font.
%
% \subsubsection{The \texttt{plain} style}
%
% \label{dok:th@plain} As the following macros use |@|, we have
% to locally set the |\catcode| of this symbol to ``letter''.
% This happens within a group, so that we do not have to worry about
% which |\catcode| that symbol had before.
%    \begin{macrocode}
%<*thp>
\begingroup \makeatletter
%    \end{macrocode}
% Since we are now within a group, we must make all definitions
% globally.  First we make sure that \texttt{theorem.sty} is loaded.
% This will allow us to use this file as a document style option
% without having to call \texttt{theorem} itself as an option.  At the
% same time, we assure that at least version 2 is loaded, since
% |\theorem@style| was not defined in earlier versions.
%    \begin{macrocode}
\@ifundefined{theorem@style}{\input{theorem.sty}}{}
\ProvidesFile{thp.sty}
             [\FMithmInfo]
%    \end{macrocode}
%
%
% \begin{macro}{\th@plain}
% \begin{macro}{\@begintheorem}
% \begin{macro}{\@opargbegintheorem}
%    |\theoremstyle{plain}| corresponds to the original
%    definition, except that the distances to the surrounding text are
%    determined by the parameters |\theorempreskipamount| and
%    |\theorempostskipamount|.
%    First we set the default body font.
%    \begin{macrocode}
\gdef\th@plain{\normalfont\itshape
%    \end{macrocode}
%    Then we define |\@begintheorem| and
%    |\@opargbegintheorem|. These two macros define how the
%    header of a theorem is typeset.  |\@opargbegintheorem| will
%    be called if a \textsf{theorem} environment with an optional
%    argument is encountered; otherwise, the header is constructed by
%    calling |\@begintheorem|.  If one of these macros is
%    executed, we are within a \textsf{trivlist} environment started by
%    |\@thm|.  So the theorem header is produced with an
%    |\item| command.
%
%    Instead of specifying the header font directly, all standard
%    theorem styles use the |\theorem@headerfont| macro to allow
%    customization.  The extra space (|\labelsep|) is necessary
%    because of problems in the \textsf{trivlist} environment.
%    \begin{macrocode}
  \def\@begintheorem##1##2{%
        \item[\hskip\labelsep \theorem@headerfont ##1\ ##2]}%
%    \end{macrocode}
%    The definition of |\@opargbegintheorem| is completely
%    analogous. The only difference is the fact that there exists a
%    third argument (which is the optional parameter of the
%    environment and contains additional information about the
%    theorem).  Customarily we enclose it in parentheses.
%    \begin{macrocode}
\def\@opargbegintheorem##1##2##3{%
   \item[\hskip\labelsep \theorem@headerfont ##1\ ##2\ (##3)]}}
%    \end{macrocode}
% \end{macro}
% \end{macro}
% \end{macro}
%
% \noindent
% We conclude with an |\endgroup| to restore the
% |\catcode| of |@|.
%    \begin{macrocode}
\endgroup
%</thp>
%    \end{macrocode}
%
%
% \subsubsection{The \texttt{break} style}
%
% This style option is stored in the file |thb.sty|.
% For the next two lines see the documentation for |\th@plain| on
% page~\pageref{dok:th@plain}.
%    \begin{macrocode}
%<*thb>
\begingroup \makeatletter
\@ifundefined{theorem@style}{\input{theorem.sty}}{}
\ProvidesFile{thb.sty}
             [\FMithmInfo]
%    \end{macrocode}
%
% \begin{macro}{\th@break}
%    |\theoremstyle{break}| produces a line break after the name
%    of the theorem.  The font is |\slshape|.  Hence, we define
%    |\th@break| as follows:
%    \begin{macrocode}
\gdef\th@break{\normalfont\slshape
  \def\@begintheorem##1##2{\item[%
%    \end{macrocode}
%    We run into the following problem: it is not possible to create
%    the header with |\item[|\meta{title}|]| and then start
%    a new line by, for example, |\mbox{}\\|. Such a definition
%    will fail whenever a list environment follows immediately.  With
%    the above construction, the |\mbox{}| causes the switch
%    |@inlabel| (cf.\ definition of |\list| and
%    |\trivlist| in \cite{latex:source}) to be set to \texttt{false}
%    and so the following list will insert additional vertical
%    space (|\topskip|). This is quite annoying. Therefore, we
%    create the line break within the |\item|. In order to ensure
%    that the text will begin at the proper position in the following
%    line, we simply pretend that the label does not take any
%    room.\footnote{This will lead to problems whenever very high
%    symbols occurring in the line tower into the
%    heading. So, something else has to be done here sometime.}
%    \begin{macrocode}
       \rlap{\vbox{\hbox{\hskip \labelsep\theorem@headerfont ##1\ ##2}%
                   \hbox{\strut}}}]}%
%    \end{macrocode}
%    Again, the definition of |\@opargbegintheorem| is
%    completely analogous.
%    \begin{macrocode}
\def\@opargbegintheorem##1##2##3{%
  \item[\rlap{\vbox{\hbox{\hskip \labelsep \theorem@headerfont
                     ##1\ ##2\ (##3)}%
                    \hbox{\strut}}}]}}
\endgroup
%</thb>
%    \end{macrocode}
% \end{macro}
%
%
% \subsubsection{The \texttt{changebreak} style}
%
%    \begin{macrocode}
% This style option is stored in the file |thcb.sty|.
%    \begin{macrocode}
%<*thcb>
\begingroup \makeatletter
\@ifundefined{theorem@style}{\input{theorem.sty}}{}
\ProvidesFile{thcb.sty}
             [\FMithmInfo]
%    \end{macrocode}
%
% \begin{macro}{\th@changebreak}
%    The \texttt{change-break} theorem style is like \texttt{break}
%    but with interchange of theorem name and theorem number.  Thus we
%    define |\th@changebreak| as follows:
%    \begin{macrocode}
\gdef\th@changebreak{\normalfont\slshape
  \def\@begintheorem##1##2{\item
       [\rlap{\vbox{\hbox{\hskip\labelsep \theorem@headerfont ##2\ ##1}%
                    \hbox{\strut}}}]}%
\def\@opargbegintheorem##1##2##3{%
  \item[\rlap{\vbox{\hbox{\hskip\labelsep \theorem@headerfont
                     ##2\ ##1\ (##3)}%
                    \hbox{\strut}}}]}}
\endgroup
%</thcb>
%    \end{macrocode}
% \end{macro}
%
%
% \subsubsection{The \texttt{change} style}
%
% This style option is stored in the file |thc.sty|.
%    \begin{macrocode}
%<*thc>
\begingroup \makeatletter
\@ifundefined{theorem@style}{\input{theorem.sty}}{}
\ProvidesFile{thc.sty}
             [\FMithmInfo]
%    \end{macrocode}
%
% \begin{macro}{\th@change}
%    The \texttt{change} theorem style corresponds to the
%    \texttt{change break} style without a linebreak after the header.
%    To say it in another way, it's the same as the \texttt{plain}
%    style but with number and name interchanged and |\slshape| as the
%    default font.
%    \begin{macrocode}
\gdef\th@change{\normalfont\slshape
  \def\@begintheorem##1##2{\item
       [\hskip\labelsep \theorem@headerfont ##2\ ##1]}%
\def\@opargbegintheorem##1##2##3{%
  \item[\hskip\labelsep \theorem@headerfont ##2\ ##1\ (##3)]}}
\endgroup
%</thc>
%    \end{macrocode}
% \end{macro}
%
%
% \subsubsection{The \texttt{marginbreak} style}
%
% This style option is the one used most often at Mainz.  It is saved
% in the file |thmb.sty|.
%    \begin{macrocode}
%<*thmb>
\begingroup \makeatletter
\@ifundefined{theorem@style}{\input{theorem.sty}}{}
\ProvidesFile{thmb.sty}
             [\FMithmInfo]
%    \end{macrocode}
%
% \begin{macro}{\th@marginbreak}
%    The \texttt{margin break} style is nearly the same as the
%    \texttt{change break} style. The only difference is the placement
%    of the theorem number.  We use |\llap| to place it in the left
%    margin.
%
%    In this style |\labelsep| denotes the separation between the
%    number and the text.
%    \begin{macrocode}
\gdef\th@marginbreak{\normalfont\slshape
  \def\@begintheorem##1##2{\item
       [\rlap{\vbox{\theorem@headerfont
                    \hbox{\llap{##2}\hskip\labelsep ##1}%
                    \hbox{\strut}}}]}%
\def\@opargbegintheorem##1##2##3{%
  \item[\rlap{\vbox{\theorem@headerfont
                    \hbox{\llap{##2}\hskip\labelsep ##1\ (##3)}%
                    \hbox{\strut}}}]}}
\endgroup
%</thmb>
%    \end{macrocode}
% \changes{v2.1c}{90/02/17}{`endgroup outside macrocode env.}
% \end{macro}
%
%
%
% \subsubsection{The \texttt{margin} style}
%
% This style option is stored in the file |thm.sty|.
%    \begin{macrocode}
%<*thm>
\begingroup \makeatletter
\@ifundefined{theorem@style}{\input{theorem.sty}}{}
\ProvidesFile{thm.sty}
             [\FMithmInfo]
%    \end{macrocode}
%
% \begin{macro}{\th@margin}
%    Again this is only a variant of the theorem styles described
%    above without any new ideas.
%    \begin{macrocode}
\gdef\th@margin{\normalfont\slshape
  \def\@begintheorem##1##2{\item
       [\theorem@headerfont \llap{##2}\hskip\labelsep ##1]}%
\def\@opargbegintheorem##1##2##3{%
  \item[\theorem@headerfont \llap{##2}\hskip\labelsep ##1\ (##3)]}}
\endgroup
%</thm>
%    \end{macrocode}
% \end{macro}
%
%
%
%
% \subsection{Final Definitions}
%
% \begin{macro}{\theorempreskipamount}
% \begin{macro}{\theorempostskipamount}
%    The \textsf{skip} parameters that regulate the vertical empty space
%    before and after the {\envfont theorem} environment have
%    to be allocated as well.
% \changes{v2.0f}{89/07/20}{Unnecessary `global' removed.}
%    \begin{macrocode}
%<*package>
\newskip\theorempreskipamount
\newskip\theorempostskipamount
%    \end{macrocode}
%    Since we have used the same values for all theorem sets, we now can
%    assign them.
%    \begin{macrocode}
\global\setlength\theorempreskipamount{12pt plus 5pt minus 3pt}
\global\setlength\theorempostskipamount{8pt plus 3pt minus 1.5pt}
%    \end{macrocode}
% \end{macro}
% \end{macro}
%
%
% \begin{macro}{\@endtheorem}
%    The same holds for the macro |\@endtheorem|, which ends a
%    {\envfont theorem} environment. Since it is the same for all
%    theorem sets, it is removed from the macros |\th@|\meta{style}.
%    It simply ends the {\envfont trivlist} environment, which was
%    begun in |\@thm|.
%    \begin{macrocode}
\global\let\@endtheorem=\endtrivlist
%    \end{macrocode}
% \end{macro}
%
%
%
% \begin{macro}{\@preamblecmds}
%    All macros defined above are to be used only in the preamble.
%    Therefore, we insert them in |\@preamblecmds| which will disable
%    them at begin document. This is done by the internal \LaTeXe{}
%    command |\@onlypreamble|.
% \changes{v2.0g}{89/09/19}{`theoremheaderfont and `theorembodyfont
%    added.}
% \changes{v2.2a}{94/02/03}{Use \LaTeXe{} interface}
%    \begin{macrocode}
\@onlypreamble\@xnthm
\@onlypreamble\@ynthm
\@onlypreamble\@othm
\@onlypreamble\newtheorem
\@onlypreamble\theoremstyle
\@onlypreamble\theorembodyfont
\@onlypreamble\theoremheaderfont
%    \end{macrocode}
% \end{macro}
%
% Finally we declare the \texttt{plain} theorem style to be the default.
% \changes{v2.1b}{90/06/04}{Fixed missing `item by declaring default
%    style.}
%    \begin{macrocode}
\theoremstyle{plain}
%</package>
%    \end{macrocode}
%
% \Finale

\endinput
}{}
\ProvidesFile{thcb.sty}
             [\FMithmInfo]
%    \end{macrocode}
%
% \begin{macro}{\th@changebreak}
%    The \texttt{change-break} theorem style is like \texttt{break}
%    but with interchange of theorem name and theorem number.  Thus we
%    define |\th@changebreak| as follows:
%    \begin{macrocode}
\gdef\th@changebreak{\normalfont\slshape
  \def\@begintheorem##1##2{\item
       [\rlap{\vbox{\hbox{\hskip\labelsep \theorem@headerfont ##2\ ##1}%
                    \hbox{\strut}}}]}%
\def\@opargbegintheorem##1##2##3{%
  \item[\rlap{\vbox{\hbox{\hskip\labelsep \theorem@headerfont
                     ##2\ ##1\ (##3)}%
                    \hbox{\strut}}}]}}
\endgroup
%</thcb>
%    \end{macrocode}
% \end{macro}
%
%
% \subsubsection{The \texttt{change} style}
%
% This style option is stored in the file |thc.sty|.
%    \begin{macrocode}
%<*thc>
\begingroup \makeatletter
\@ifundefined{theorem@style}{% \iffalse meta-comment
%
% Copyright (C) 1993-2024
%
% The LaTeX Project and any individual authors listed elsewhere
% in this file.
%
% This file is part of the Standard LaTeX `Tools Bundle'.
% -------------------------------------------------------
%
% It may be distributed and/or modified under the
% conditions of the LaTeX Project Public License, either version 1.3c
% of this license or (at your option) any later version.
% The latest version of this license is in
%    https://www.latex-project.org/lppl.txt
% and version 1.3c or later is part of all distributions of LaTeX
% version 2005/12/01 or later.
%
% The list of all files belonging to the LaTeX `Tools Bundle' is
% given in the file `manifest.txt'.
%
% \fi
%
%%
%
%
\def\FMithmInfo{2023/07/05 v2.2c Theorem extension package (FMi)}
%
% \ProvidesFile{theorem.dtx}[\FMithmInfo]
%\iffalse   % this is a METACOMMENT !
%%
%% Package `theorem' to use with LaTeX2e
%% Copyright (C) 1989-1995 Frank Mittelbach, all rights reserved.
%<+package>\@ifundefined{theorem@style}{}{\endinput}
%<+package>\NeedsTeXFormat{LaTeX2e}[1995/06/01]
%<+package>\ProvidesPackage{theorem}
%<+package>         [\FMithmInfo]
%
% \fi
%
% \changes{v1.0}{87/05/07}{First draft.}
% \changes{v1.0a}{87/05/07}{`thistheoremstyle' added.}
% \changes{v1.1a}{87/07/29}{New concept: redefine the internal
%                          \protect\LaTeX{} commands `@xnthm' etc.}
% \changes{v1.1b}{87/07/31}{`theoremfont' used to specify font.}
% \changes{v1.2a}{88/01/08}{`@endtheorem' now globally defined.}
% \changes{v1.2b}{88/01/11}{`newtheorem' only allowed in the preamble.}
% \changes{v1.3a}{88/02/14}{`theoremfont' now a toks register.}
%
% \changes{v2.0a}{88/11/16}{New implementation and new documentation.}
% \changes{v2.0a}{88/11/16}{No layout changes but internal modifications
%                           old substyles won't work.}
% \changes{v2.0b}{89/01/30}{`global' used for all definitions.}
% \changes{v2.0c}{89/06/15}{Documentation translated to English.
%                            Thanks to LLL and Nina}
% \changes{v2.1a}{90/02/15}{Support for new font selection.}
% \changes{v2.2a}{94/02/03}{Cleanup for LaTeX2e}
%
% \DoNotIndex{\@endparenv,\@ifdefinable,\@ifnextchar,\@ifundefined}
% \DoNotIndex{\@namedef,\@nameuse,\@spaces,\@tempa,\@warning}
% \DoNotIndex{\ }
% \DoNotIndex{\begingroup,\bfseries}
% \DoNotIndex{\csname}
% \DoNotIndex{\def,\do,\docdate}
% \DoNotIndex{\endcsname,\endgroup,\endinput,\endtrivlist,\expandafter}
% \DoNotIndex{\filedate,\fileversion}
% \DoNotIndex{\gdef,\global}
% \DoNotIndex{\hbox,\hskip}
% \DoNotIndex{\ignorespaces,\input,\itshape,\item}
% \DoNotIndex{\labelsep,\let,\llap}
% \DoNotIndex{\makeatletter}
% \DoNotIndex{\noexpand}
% \DoNotIndex{\refstepcounter,\relax,\rlap}
% \DoNotIndex{\setlength,\slshape,\space,\string,\strut}
% \DoNotIndex{\the,\trivlist,\typeout}
% \DoNotIndex{\vbox}
% \DoNotIndex{\xdef}
%
% \def\deflabel#1{\ttfamily #1\hfill}
% \def\deflist#1{\begin{list}{}{\settowidth\labelwidth{\ttfamily #1}%
%                               \setlength\leftmargin\labelwidth
%                               \addtolength\leftmargin\labelsep
%                               \let\makelabel\deflabel}}
% \def\enddeflist{\end{list}}
% \def\envfont{\normalfont\ttfamily}
%
% \MakeShortVerb{\|}
%
% \newcommand{\TUB}{{\sl TUGboat\/}}
%
% \GetFileInfo{theorem.dtx}
% \title{An Extension of the
%        \LaTeX{} theorem environment\thanks{This file
%        has version number \fileversion, last
%        revised \filedate.}}
% \author{Frank Mittelbach}
%
% \MaintainedByLaTeXTeam{tools}
% \maketitle
%
% \begin{abstract}
%    The macros described in this paper yield an extension of the
%    \LaTeX{} theorem mechanism. It is designed to satisfy the
%    different requirements of various journals. Thus, the
%    layout of the ``theorems'' can be manipulated by determining a
%    ``style''. This article describes not only the use, but
%    also the definition, of the necessary macros.
% \end{abstract}
%
% \section*{Preface to version 2.2}
%
% For \LaTeXe{} this package did not need any fundamental changes. I
% only modified the messages generated so that theorem layout styles
% will show up with the |\listfiles| command and cleaned the section
% on the New Font Selection Scheme since this is now included in
% \LaTeX.
%
% \section*{Preface to version 2.1}
%
% This version is identical to 2.0g described in \TUB\ 10\#3 except
% for some internal defaults which are now set depending on the used
% font selection scheme.
%
% This was done to avoid unpleasant surprises if the new font selection
% scheme is in force. For further details see
% section~\ref{sec:newfonts} and \cite{bk:GMS94}.
%
% \section{Introduction}
%
% For our purposes here, ``theorems'' are labelled enunciations,
% often set off from the main text by extra space and a font change.
% Theorems, corollaries, conjectures, definitions, and remarks
% are all instances of ``theorems''.  The ``header'' of these structures
% is composed of a label (such as \textsc{Theorem} or \textsc{Remark})
% and a number which serializes an item in the sequence of items
% with the same label.
%
% Shortly after the introduction of \LaTeX{} at the Fachbereich
% Mathematik in Mainz, the desire to manipulate the layout of
% ``theorems'' arose.  In Mainz, the following two conventions came
% into general use:
% \begin{enumerate}
%   \item The number of the theorem is shown in the margin.
%   \item There is a line break at
%         the end of the theorem header.
% \end{enumerate}
%
% Additionally, some journals require different formats which depend on
% the ``sort of theorem'': e.g.\ often remarks and definitions are set
% in |\upshape|, while |\itshape| is employed for main theorems.
%
% Confronted with these requirements, a theorem environment was
% developed in Mainz which allows separate determination of the
% layout of the ``theorems sets'', comparable to |\pagestyle|.
%
%
% \section{The user interface}
% \subsection{Defining new theorem sets}
%
% \DescribeMacro\newtheorem
% As in the original \LaTeX{} version, the command |\newtheorem|
% defines a new ``theorem set'' or ``theorem-like structure''.
% Two required arguments name the new environment and give the text
% to be typeset with each instance of the new ``set'', while
% an optional argument determines how
% the ``set'' is enumerated:
% \begin{description}
%    \item[\ttfamily \bslash newtheorem\{foo\}\{bar\}]
%       The theorem set {\envfont foo} (whose name is \texttt{bar})
%       uses its own counter.
%    \item[\ttfamily \bslash newtheorem\{foo2\}{[foo]}\{bar2\}]
%       The theorem set {\envfont foo2} (printed name \texttt{bar2})
%       uses the same counter as the theorem set \texttt{foo}.
%    \item[\ttfamily \bslash newtheorem\{foo3\}\{bar3\}{[section]}]
%       The theorem set {\envfont foo3} (printed name \texttt{bar3}) is
%       enumerated within the counter \texttt{section}, i.e.\ with every
%       new |\section| the enumeration begins again with 1, and
%       the enumeration is composed from the section-number and the
%       theorem counter itself.
% \end{description}
%
% \DescribeMacro\theoremstyle
% Additionally, the command |\theoremstyle| can define the layout
% of various, or all, theorem sets.  It should be noted that any
% theorem set defined by |\newtheorem| is typeset in the
% |\theoremstyle| that is current at the time of the definition.
% Thus, the following
% \begin{verbatim}
%    \theoremstyle{break}        \newtheorem{Cor}{Corollary}
%    \theoremstyle{plain}        \newtheorem{Exa}{Example}[section]
%\end{verbatim}                             ^^A No blank at line start
% leads to the result that the set {\envfont Cor} is formatted in the
% style \texttt{break}, while the set {\envfont Exa} and all the
% following ones are formatted in the style \texttt{plain}, unless
% another |\theoremstyle| follows.  Since the definitions installed by
% |\newtheorem| are global, one also can limit |\theoremstyle| locally
% by grouping braces.
%
% \DescribeMacro\theorembodyfont
% The choice of the font for the theorem body is completely
% independent of the chosen |\theoremstyle|; this has proven to
% be very advantageous.  For example,
% \begin{verbatim}
%   {\theorembodyfont{\upshape}          \newtheorem{Rem}{Remark}}
%\end{verbatim}
% defines a theorem set {\envfont Rem}, which will be set in
% |\upshape| in the current layout (which in our example is
% \texttt{plain}).  As with |\theoremstyle|, the |\theorembodyfont|
% chosen is that current at the time of |\newtheorem|.  If
% |\theorembodyfont| is not specified or one defines
% |\theorembodyfont{}|, then the font used will be that defined by the
% |\theoremstyle|.
%
% \DescribeMacro\theoremheaderfont It is also possible to customize
% the font used for the theorem headers.  This is, however, a global
% declaration, and therefore there should be at most one
% |\theoremheaderfont| declaration in the preamble.\footnote{If it is
% actually necessary to have different header fonts, one has to define
% new theorem styles (substituting the desired font) or specify the
% information directly in the \texttt{\bslash newtheorem} declaration
% (the unclean variant).}
%
% \DescribeMacro\theorempreskipamount
% \DescribeMacro\theorempostskipamount Two additional parameters
% affect the vertical space around the theorem environments:
% |\theorempreskipamount| and |\theorempostskipamount| define,
% respectively, the spacing before and after such an environment.
% These parameters apply for all theorem sets and can be manipulated
% with the ordinary length macros.  They are rubber lengths
% (`\textsf{skips}'), and therefore can contain \texttt{plus} and
% \texttt{minus} parts.
%
% Since the definition of theorem sets should---most sensibly---be
% placed in the preamble, we only allow installation there.
% It is therefore possible to release the memory used here after
% |\begin{document}|, in order to make room for other
% applications.
%
%
% \subsection{Existing theorem styles}
%
% The following theorem styles exist to date:
% \begin{deflist}{marginbreak}
%    \item[plain]
%       This theorem style emulates the original \LaTeX{} definition,
%       except that additionally the parameters
%       |\theorem...skipamount| are used.
%    \item[break]
%       In this style, the theorem header is followed by a line
%       break.
%    \item[marginbreak]
%       The theorem number is set in the margin, and there is a line
%       break as in \texttt{break}.
%    \item[changebreak]
%       Like \texttt{break}, but with header number and text
%       interchanged.
%    \item[change]
%       Header number and text are interchanged, without a line break.
%    \item[margin]
%       The number is set in the left margin, without a line break.
% \end{deflist}
% All styles (except \texttt{plain}) select |\slshape| as the default
% |\theorembodyfont|.
%
%
% \subsection{Examples}
%
% Given the above theorem sets {\envfont Cor}, {\envfont Exa}
% and {\envfont Rem},
% suppose that the preamble also contains the declarations:
% \begin{verbatim}
%    \theoremstyle{marginbreak}   \newtheorem{Lem}[Cor]{Lemma}
%    \theoremstyle{change}
%    \theorembodyfont{\itshape}        \newtheorem{Def}[Cor]{Definition}
%
%    \theoremheaderfont{\scshape}
%\end{verbatim}
% Then the following are some typical examples of the typeset output
% resulting from their use.
%
% \begin{Cor}
%    This is a sentence typeset in the theorem environment {\envfont
%    Cor}.
% \end{Cor}
%
% \begin{Exa}
%    This is a sentence typeset in the theorem environment {\envfont
%    Exa}.
% \end{Exa}
%
% \begin{Rem}
%    This is a sentence typeset in the theorem environment {\envfont
%    Rem}.
% \end{Rem}
%
% \begin{Lem}[Ben User]
%    This is a sentence typeset in the theorem environment {\envfont
%    Lem}.
% \end{Lem}
%
% \begin{Def}[Very impressive Definition]
%    This is a sentence typeset in the theorem environment {\envfont
%    Def}.
% \end{Def}
%
% The last two examples show the effect of the optional argument to
% a theorem environment (it is the text typeset in parentheses).
%
%
% \section{Special Considerations}
% \label{sec:newfonts}
%
% Theoremheader and body are implemented as a unit. This means that
% the |\theoremheaderfont|  will inherit characteristics of the
% |\theorembodyfont| in \LaTeXe. Thus, if for example
% |\theorembodyfont| is |\itshape| and |\theoremheaderfont|
% is |\bfseries| the font selected for the header will have the
% characteristics `bold extended italic'. If this is not desired one
% should set the |\theoremheaderfont| to something like
% \begin{verbatim}
%      \theoremheaderfont{\normalfont\bfseries}
%\end{verbatim}
% i.e.\ supplying all necessary font information explicitly.
%
%
%
% \section{Acknowledgements}
%
% The publication of this set of macros was only possible with the
% help of Christina Busse (translating the manuscript into English),
% Joachim Pense (playing the r\^ole of typist), Chris Rowley
% (looking everything over) and many others providing useful
% suggestions.
%
% \MaybeStop{%
%    \begin{thebibliography}{9}
%       \bibitem{bk:GMS94} \textsc{M.~Goossens}, \textsc{F.~Mittelbach}
%         and \textsc{A.~Samarin}.
%         \newblock The \LaTeX{} Companion.
%         \newblock
%         Addison-Wesley, Reading, Massachusetts, 1994.
%       \bibitem{latex:source} \textsc{Lamport, Leslie.} \newblock
%                               \texttt{latex.tex}, version 2.09, date
%                               Feb. 1990.
%    \end{thebibliography}
%    \PrintIndex
%    \PrintChanges}
%
%
% \section{The documentation driver file}
%
% The next bit of code contains the documentation driver file for
% \TeX{}, i.e., the file that will produce the documentation you are
% currently reading. It will be extracted from this file by the
% \texttt{docstrip} program. Since it is the first code in the file
% one can alternatively process this file directly with \LaTeXe{} to
% obtain the documentation.
%
%    \begin{macrocode}
%<*driver>
\documentclass{ltxdoc}

\usepackage{theorem}

%<+driver>% The next few lines define theorem sets which are used
%<+driver>% in the example section of the documentation.

\theoremstyle{break}       \newtheorem{Cor}{Corollary}
\theoremstyle{plain}       \newtheorem{Exa}{Example}[section]
{\theorembodyfont{\upshape}\newtheorem{Rem}{Remark}}
\theoremstyle{marginbreak} \newtheorem{Lem}[Cor]{Lemma}
\theoremstyle{change}
\theorembodyfont{\itshape} \newtheorem{Def}[Cor]{Definition}

\theoremheaderfont{\scshape}

\RecordChanges

\begin{document}
   \DocInput{theorem.dtx}
\end{document}
%</driver>
%    \end{macrocode}
%
% \section{Definition of the Macros}
%
% If the file has been loaded before, we abort immediately. If not the
% package announces itself (this is actually done at the very top of
% the file---the way it is done isn't good style so don't copy it).
% \changes{v2.0e}{89/07/19}{Spaces removed from `typeout'.}
% \changes{v2.2b}{95/11/19}{Announce moved to top of file}
%    \begin{macrocode}
%<*package>
%\@ifundefined{theorem@style}{}{\endinput}
%\def\FMithmInfo{1995/11/19 v2.2b Theorem extension package (FMi)}
%\ProvidesPackage{theorem}[\FMithmInfo]
%    \end{macrocode}
%
%
% \subsection{Definition of theorem styles and fonts}
%
% All the definitions in this file are done globally to allow
% inputting this file inside a group.
%
% \begin{macro}{\theoremstyle}
%    Before a theorem style can be installed, the chosen style must be
%    known. For that reason, we must test to see that
%    |\th@|\meta{style} is known or, more precisely, that it is
%    different from |\relax|. If the style is not known then
%    |\th@plain| is used.
%    \begin{macrocode}
\gdef\theoremstyle#1{%
   \@ifundefined{th@#1}{\@warning
          {Unknown theoremstyle `#1'. Using `plain'}%
          \theorem@style{plain}}%
%    \end{macrocode}
%    We save the theorem style to be used in the token register
%    |\theorem@style|.
%    \begin{macrocode}
      {\theorem@style{#1}}%
%    \end{macrocode}
%    Now we ``evaluate'' the theorem style: this means, we call the
%    macro |\th@|\meta{style} which will activate the relevant
%    definitions which are contained in a separate file.  This is done
%    in a group to suppress changes to the current font.  This could
%    otherwise pose problems together with the new font selection
%    scheme\footnote{When I printed the original article using the new
%    font selection scheme I ended with a document with slanted
%    typefaces (text headings and all) simply because one of the
%    theorem styles used \texttt{\bslash sl} at toplevel.} if the
%    |\th@|\meta{style} is evaluated a second time.
% \changes{v2.1a}{90/02/15}{Group added around `th@..}
% \changes{v2.1a}{90/02/15}{Reseting of `..@begintheorem removed.}
%    \begin{macrocode}
      \begingroup
        \csname th@\the\theorem@style \endcsname
      \endgroup}
%    \end{macrocode}
% \end{macro}
%
%
% \begin{macro}{\@begintheorem}
% \changes{v2.1a}{90/02/15}{Relaxed at toplevel.}
% \begin{macro}{\@opargbegintheorem}
% \changes{v2.1a}{90/02/15}{Relaxed at toplevel.}
%    We reset |\@begintheorem| and |\@opargbegintheorem| to
%    |\relax| since these commands are no longer necessary at
%    toplevel. This will save a few tokens.
%    \begin{macrocode}
\global\let\@begintheorem\relax
\global\let\@opargbegintheorem\relax
%    \end{macrocode}
% \end{macro}
% \end{macro}
%
% \begin{macro}{\theorem@style}
%    Obviously the \textsf{token} register used above has to be
%    allocated. To assure the utmost compatibility with the
%    original \LaTeX{} definition, we set the default theorem style to
%    \texttt{plain}, which implements the usual \LaTeX{} convention.
% \changes{v2.0f}{89/07/20}{Unnecessary `global' removed.}
%    \begin{macrocode}
\newtoks\theorem@style
\global\theorem@style{plain}
%    \end{macrocode}
% \end{macro}
%
% \begin{macro}{\theorembodyfont}
% \begin{macro}{\theorem@bodyfont}
%    If the |\theorembodyfont| is set by the user then it should not
%    interact with the default font set in the theorem style. When the
%    new font selection is in force this may happen if, for example,
%    the default is |\itshape| and the new |\theorembodyfont| is
%    |\sffamily|. So we add a |\reset@font| command in front of the
%    user definition.
% \changes{v2.0f}{89/07/20}{Unnecessary `global' removed.}
% \changes{v2.0g}{89/09/19}{`theoremfont to `theorembodyfont.}
% \changes{v2.1c}{92/06/22}{`reset@font added and logic changed.}
% \changes{v2.1d}{92/10/28}{Reset to empty if arg is empty}
%    \begin{macrocode}
\gdef\theorembodyfont#1{%
%    \end{macrocode}
%    We check if the argument supplied is empty and if so put nothing
%    into the |\theorem@bodyfont| token register to allow for
%    |\theorembodyfont{}| as a mean of using the default of the
%    current |\theoremstyle|.
%    \begin{macrocode}
   \def\@tempa{#1}%
   \ifx\@tempa\@empty
    \theorem@bodyfont{}%
   \else
    \theorem@bodyfont{\reset@font#1}%
   \fi
   }
\newtoks\theorem@bodyfont
\global\theorem@bodyfont{}
%    \end{macrocode}
% \end{macro}
% \end{macro}
%
%
% \begin{macro}{\theoremheaderfont}
% \changes{v2.0g}{89/09/19}{Macro added.}
%    The font for the theorem headers is handled differently because
%    this definition applies to all theorem styles.
%    \begin{macrocode}
\gdef\theoremheaderfont#1{\gdef\theorem@headerfont{#1}%
%    \end{macrocode}
%    After using the macro once it is redefined to produce an error
%    message.
%    \begin{macrocode}
       \gdef\theoremheaderfont##1{%
        \typeout{\string\theoremheaderfont\space should be used
                 only once.}}}
%    \end{macrocode}
% \end{macro}
%
% \begin{macro}{\theorem@headerfont}
% \changes{v2.0g}{89/09/19}{Macro added.}
%    To set the |\theorem@headerfont| default we first test if
%    the new fontselection scheme is in force.
% \changes{v2.1a}{90/02/15}{Support for new font selection.}
%    \begin{macrocode}
\ifx\upshape\undefined
%    \end{macrocode}
%    If not we define it to expand into |\bfseries|. We don't use
%    |\let| just in case a following style option redefines this
%    macro.
%    \begin{macrocode}
\gdef\theorem@headerfont{\bfseries}
%    \end{macrocode}
%    Otherwise we reset the current shape before calling |\bfseries|.
%    \begin{macrocode}
\else \gdef\theorem@headerfont{\normalfont\bfseries}\fi
%    \end{macrocode}
% \end{macro}
%
% \begin{macro}{\th@plain}
% \begin{macro}{\th@break}
% \begin{macro}{\th@marginbreak}
% \begin{macro}{\th@changebreak}
% \begin{macro}{\th@change}
% \begin{macro}{\th@margin}
%    The different styles are defined in macros such as
%    |\th@plain|.  Since memory space is precious in
%    ``non-Big-versions'', we have to avoid offering too many unused
%    definitions.  Therefore we define these styles in separate files
%    that can be loaded on demand.  Thus the commands themselves only
%    load these files. We use |\@input@|, a \LaTeXe{} internal command
%    that ensures that the file will be listed with |\listfiles|.
% \changes{v2.2a}{94/02/02}{Use `@input@ to load theorem layout files}
%    \begin{macrocode}
\gdef\th@plain{\@input@{thp.sty}}
\gdef\th@break{\@input@{thb.sty}}
\gdef\th@marginbreak{\@input@{thmb.sty}}
\gdef\th@changebreak{\@input@{thcb.sty}}
\gdef\th@change{\@input@{thc.sty}}
\gdef\th@margin{\@input@{thm.sty}}
%    \end{macrocode}
%    This list will be expanded when new styles become
%    available.  For testing,  just append new theorem substyles
%    as document options.
% \end{macro}
% \end{macro}
% \end{macro}
% \end{macro}
% \end{macro}
% \end{macro}
%
%
%
% \subsection{Definition of a new theorem set}
%
% As already pointed out, a new theorem environment can be defined in
% three different ways:
% \begin{verbatim}
%    \newtheorem{Lem}{Lemma}
%    \newtheorem{Lem}{Lemma}[section]
%    \newtheorem{Lem}[Theorem]{Lemma}
%\end{verbatim}
% The function of the macro |\newtheorem| is to recognize these cases
% and then to branch into one of the three macros |\@ynthm|, |\@xnthm|
% or |\@othm|.  This mechanism is adopted unchanged from
% \cite{latex:source}; the essential point here is that, for example,
% in the second case, the arguments \texttt{Lem}, \texttt{Lemma} and
% \texttt{section} are passed over to the macro |\@xnthm|.
%
% We inspect this case first because the others present fewer problems,
% and thus are easily derived from this one.
%
% \begin{macro}{\@xnthm}
%    For our example arguments, the macro |\@xnthm| must fulfill
%    the following:
%    \begin{itemize}
%       \item
%          Define a new \LaTeX-counter `Lem'
%       \item
%          reset this counter within a |\section|
%       \item
%          define the macro |\theLem|
%       \item
%          define the environment macros |\Lem| and
%          |\endLem| using the current
%          |\theoremstyle| and |\theorem@bodyfont|.
%    \end{itemize}
%    Obviously, all this should happen only if the first argument
%    of |\@xnthm| (i.e.\ \texttt{Lem} in our example) is chosen so
%    as not to conflict with any previously defined commands or
%    environments.  This test is performed by the \LaTeX{} macro
%    |\@ifdefinable|.
%    \begin{macrocode}
\gdef\@xnthm#1#2[#3]{\expandafter\@ifdefinable\csname #1\endcsname
%    \end{macrocode}
%    Therefore, the first argument of |\@ifdefinable| is the
%    expansion (in the example, |\Lem|) of
%    |\csname||#1\endcsname|.  The second argument is
%    executed only if the test has been completed successfully.
%    \begin{macrocode}
   {%
%    \end{macrocode}
%    Now we define the new counter. The names of the \LaTeX{}
%    macros employed should speak for themselves:
% \changes{v2.2b}{1995/10/07}{Use \cs{@newctr}}
% \changes{v2.2c}{1995/11/23}{Use \cs{@newctr} properly.}
%    \begin{macrocode}
    \@definecounter{#1}\@newctr{#1}[#3]%
%    \end{macrocode}
%    Using |\@newctr| will give a proper error message if the counter
%    in |#3| is not defined.
%    In defining `|\theLem|' we must generate the desired macro
%    name by use of |\expandafter| and |\csname|.
%    \begin{macrocode}
    \expandafter\xdef\csname the#1\endcsname
%    \end{macrocode}
%    An |\xdef| is used in order to make the definition global,
%    and to ensure that it contains the replacement texts of
%    |\@thmcountersep| and |\@thmcounter|.\footnote{These
%    two macros can be defined by the document style.  Their default
%    values produce a `\texttt{.}' as separation and an arabic
%    representation of the number.} However, not everything should be
%    expanded. For example, it saves space to use |\thesection|
%    instead of its---at times---lengthy expansion.
%    \begin{macrocode}
      {\expandafter \noexpand \csname the#3\endcsname
       \@thmcountersep \@thmcounter{#1}}%
%    \end{macrocode}
%    Thus with the defaults of \LaTeX{}, |\theLem| would be
%    replaced by the command sequence |\thesection.\arabic{Lem}|.
%
%    We will now look at the definition of the macro which is executed
%    at the beginning of the actual environment (in our example this
%    macro is |\Lem|). It should be noted that we use an
%    ``|\expandafter| trick'' to expand only certain parts of the
%    replacement text at the time of the definition.
%    \begin{macrocode}
    \def\@tempa{\global\@namedef{#1}}%
    \expandafter \@tempa \expandafter{%
%    \end{macrocode}
%    First, the macro that contains the current definitions of
%    |\@begintheorem| and |\@opargtheorem| should be called
%    up.  The name of this macro---as is already known---has the form
%    |\th@|\meta{theorem\ style}; therefore, it must be called by
%    \begin{macrocode}
      \csname th@\the \theorem@style
%    \end{macrocode}
%    In addition the default theorem font should be changeable, i.e.\
%    we have to insert the contents of |\theorem@bodyfont|. For
%    that reason, we expand even further, beyond |\endcsname|,
%    and thus insert the contents of the \textsf{token} register
%    |\theorem@bodyfont| in the replacement text.
%    \begin{macrocode}
            \expandafter \endcsname \the \theorem@bodyfont
%    \end{macrocode}
%    Now it is time to call the macro |\@thm| which takes over
%    the further processing.  It has two arguments: the current
%    counter name (in our example, \texttt{Lem}), and the text of the
%    label (in our example, \texttt{Lemma}).
%    \begin{macrocode}
     \@thm{#1}{#2}}%
%    \end{macrocode}
%    With this, the `sub-definition' is complete.  The macro
%    |\@endtheorem| ends a theorem environment and is, so far, nothing
%    but an |\endtrivlist|.  (Hence it is defined globally, and not
%    within the theorem styles.\footnote{This has to be changed as
%    soon as theorem styles that change \texttt{\bslash @endtheorem}
%    exist.  In such a case, all existing styles must be changed as
%    well since they will have to reset the macro.}) Therefore, we can
%    set it equivalent to the macro that ends the theorem set (in our
%    example, |\endLem|).  However, if some day theorem styles exist
%    that do change |\@endtheorem|, we would have to use the
%    commented-out line instead.
%    \begin{macrocode}
    \global \expandafter \let \csname end#1\endcsname \@endtheorem
%   \global\@namedef{end#1}{\@endtheorem}%
%    \end{macrocode}
%    With these commands all the required definitions are employed,
%    unless the test |\@ifdefinable| has failed.  Therefore, we
%    end the second argument of this macro and with it the definition
%    of |\@xnthm|.
%    \begin{macrocode}
   }}
%    \end{macrocode}
% \end{macro}
%
% \begin{macro}{\@ynthm}
%    The definition of |\@ynthm| is completely analogous. In
%    this case the new counter that is defined is not reset within
%    another counter; thus the definition of |\the...|
%    is simplified:
%    \begin{macrocode}
\gdef\@ynthm#1#2{\expandafter\@ifdefinable\csname #1\endcsname
   {\@definecounter{#1}%
    \expandafter\xdef\csname the#1\endcsname{\@thmcounter{#1}}%
%    \end{macrocode}
%    The rest of the definition corresponds literally to that of
%    |\@xnthm|:
%    \begin{macrocode}
    \def\@tempa{\global\@namedef{#1}}\expandafter \@tempa
     \expandafter{\csname th@\the \theorem@style \expandafter
     \endcsname \the\theorem@bodyfont \@thm{#1}{#2}}%
    \global \expandafter \let \csname end#1\endcsname \@endtheorem}}
%    \end{macrocode}
% \end{macro}
%
% \begin{macro}{\@othm}
%    The definition of |\@othm| does not contain anything new.
%    \begin{macrocode}
\gdef\@othm#1[#2]#3{%
%    \end{macrocode}
% \changes{v2.2b}{95/10/07}{Build in check for counter defined}
%    We do not define a new counter but instead use one that has
%    already been defined.  Thus the only definition we need is that
%    of this pseudo-counter (i.e.\ |\the|\meta{env.\ name}).
%    First we check if |#2| corresponds to a known counter name.
%    \begin{macrocode}
  \expandafter\ifx\csname c@#2\endcsname\relax
   \@nocounterr{#2}%
  \else
   \expandafter\@ifdefinable\csname #1\endcsname
   {\expandafter \xdef \csname the#1\endcsname
     {\expandafter \noexpand \csname the#2\endcsname}%
%    \end{macrocode}
%    All other parts of the definition can be adopted from
%    |\@xnthm|. We have to remember, though, that in this case
%    the name of the current counter and the theorem label have moved
%    to the second and third arguments.
%    \begin{macrocode}
    \def\@tempa{\global\@namedef{#1}}\expandafter \@tempa
     \expandafter{\csname th@\the \theorem@style \expandafter
     \endcsname \the\theorem@bodyfont \@thm{#2}{#3}}%
    \global \expandafter \let \csname end#1\endcsname \@endtheorem}%
  \fi}
%    \end{macrocode}
% \end{macro}
%
% \subsection{Macros that are employed in a {\envfont theorem}
%    environment}
%
% \begin{macro}{\@thm}
%    The macro |\@thm| has to increase the current counter.  Then,
%    depending on whether the environment has (or does not have) an
%    optional argument, it has to branch into either |\@begintheorem|
%    or |\@opargtheorem|.
%    \begin{macrocode}
\gdef\@thm#1#2{\refstepcounter{#1}%
%    \end{macrocode}
%    Now we start a {\envfont trivlist} environment, and give
%    |\@topsep| and |\@topsepadd| the values of the \textsf{skip}
%    registers |\theorempreskipamount| and |\theorempostskipamount|.
%    The value in |\@topsep| is the vertical space that is inserted by
%    the first (and only) |\item| in our |\trivlist| whilst
%    |\@topsepadd| is inserted by |\@endparenv| at the end of that
%    {\envfont trivlist} environment.  By using these registers, we
%    obtain the desired space around a {\envfont theorem} environment.
%    \begin{macrocode}
   \trivlist
   \@topsep \theorempreskipamount               % used by first \item
   \@topsepadd \theorempostskipamount           % used by \@endparenv
%    \end{macrocode}
%    Now we have to test whether an optional argument has been given.
% \changes{v2.0d}{89/07/18}{Forgotten percent added.}
%    \begin{macrocode}
   \@ifnextchar [%
%    \end{macrocode}
%    If there is an optional argument, we will call |\@ythm|, and
%    move the arguments read back into the input stream.
%    \begin{macrocode}
   {\@ythm{#1}{#2}}%
%    \end{macrocode}
%    If not, we call |\@begintheorem|. Its first argument is the
%    name of the theorem set (hence the second argument of
%    |\@thm|). Its second argument is the macro that produces the
%    current number.
%    \begin{macrocode}
   {\@begintheorem{#2}{\csname the#1\endcsname}\ignorespaces}}
%    \end{macrocode}
% \end{macro}
%
% \begin{macro}{\@xthm}
% \begin{macro}{\@ythm}
%    Both these macros were originally called by |\@thm|.  We do
%    not need |\@xthm| anymore, hence we reset it to
%    |\relax|. The definition of |\@ythm| has not changed at
%    all from its definition in \LaTeX. In order to make the macros
%    easier to understand, we will
%    nevertheless present it (commented out).
%    \begin{macrocode}
\global\let\@xthm\relax
% \def\@ythm#1#2[#3]{\@opargbegintheorem{#2}{\csname
%           the#1\endcsname}{#3}\ignorespaces}
%</package>
%    \end{macrocode}
%    The primitive |\ignorespaces| in |\@ythm| and
%    |\@thm| is needed to remove the spaces between the
%    |\begin{...}| and the actual text.
% \end{macro}
% \end{macro}
%
%
% \subsection{Definition of the theorem substyles}
%
% As already pointed out, the theorem substyles, defined below, are
% only loaded when necessary.  Note that all these substyles, except
% \texttt{plain}, have |\slshape| as the default body font.
%
% \subsubsection{The \texttt{plain} style}
%
% \label{dok:th@plain} As the following macros use |@|, we have
% to locally set the |\catcode| of this symbol to ``letter''.
% This happens within a group, so that we do not have to worry about
% which |\catcode| that symbol had before.
%    \begin{macrocode}
%<*thp>
\begingroup \makeatletter
%    \end{macrocode}
% Since we are now within a group, we must make all definitions
% globally.  First we make sure that \texttt{theorem.sty} is loaded.
% This will allow us to use this file as a document style option
% without having to call \texttt{theorem} itself as an option.  At the
% same time, we assure that at least version 2 is loaded, since
% |\theorem@style| was not defined in earlier versions.
%    \begin{macrocode}
\@ifundefined{theorem@style}{\input{theorem.sty}}{}
\ProvidesFile{thp.sty}
             [\FMithmInfo]
%    \end{macrocode}
%
%
% \begin{macro}{\th@plain}
% \begin{macro}{\@begintheorem}
% \begin{macro}{\@opargbegintheorem}
%    |\theoremstyle{plain}| corresponds to the original
%    definition, except that the distances to the surrounding text are
%    determined by the parameters |\theorempreskipamount| and
%    |\theorempostskipamount|.
%    First we set the default body font.
%    \begin{macrocode}
\gdef\th@plain{\normalfont\itshape
%    \end{macrocode}
%    Then we define |\@begintheorem| and
%    |\@opargbegintheorem|. These two macros define how the
%    header of a theorem is typeset.  |\@opargbegintheorem| will
%    be called if a \textsf{theorem} environment with an optional
%    argument is encountered; otherwise, the header is constructed by
%    calling |\@begintheorem|.  If one of these macros is
%    executed, we are within a \textsf{trivlist} environment started by
%    |\@thm|.  So the theorem header is produced with an
%    |\item| command.
%
%    Instead of specifying the header font directly, all standard
%    theorem styles use the |\theorem@headerfont| macro to allow
%    customization.  The extra space (|\labelsep|) is necessary
%    because of problems in the \textsf{trivlist} environment.
%    \begin{macrocode}
  \def\@begintheorem##1##2{%
        \item[\hskip\labelsep \theorem@headerfont ##1\ ##2]}%
%    \end{macrocode}
%    The definition of |\@opargbegintheorem| is completely
%    analogous. The only difference is the fact that there exists a
%    third argument (which is the optional parameter of the
%    environment and contains additional information about the
%    theorem).  Customarily we enclose it in parentheses.
%    \begin{macrocode}
\def\@opargbegintheorem##1##2##3{%
   \item[\hskip\labelsep \theorem@headerfont ##1\ ##2\ (##3)]}}
%    \end{macrocode}
% \end{macro}
% \end{macro}
% \end{macro}
%
% \noindent
% We conclude with an |\endgroup| to restore the
% |\catcode| of |@|.
%    \begin{macrocode}
\endgroup
%</thp>
%    \end{macrocode}
%
%
% \subsubsection{The \texttt{break} style}
%
% This style option is stored in the file |thb.sty|.
% For the next two lines see the documentation for |\th@plain| on
% page~\pageref{dok:th@plain}.
%    \begin{macrocode}
%<*thb>
\begingroup \makeatletter
\@ifundefined{theorem@style}{\input{theorem.sty}}{}
\ProvidesFile{thb.sty}
             [\FMithmInfo]
%    \end{macrocode}
%
% \begin{macro}{\th@break}
%    |\theoremstyle{break}| produces a line break after the name
%    of the theorem.  The font is |\slshape|.  Hence, we define
%    |\th@break| as follows:
%    \begin{macrocode}
\gdef\th@break{\normalfont\slshape
  \def\@begintheorem##1##2{\item[%
%    \end{macrocode}
%    We run into the following problem: it is not possible to create
%    the header with |\item[|\meta{title}|]| and then start
%    a new line by, for example, |\mbox{}\\|. Such a definition
%    will fail whenever a list environment follows immediately.  With
%    the above construction, the |\mbox{}| causes the switch
%    |@inlabel| (cf.\ definition of |\list| and
%    |\trivlist| in \cite{latex:source}) to be set to \texttt{false}
%    and so the following list will insert additional vertical
%    space (|\topskip|). This is quite annoying. Therefore, we
%    create the line break within the |\item|. In order to ensure
%    that the text will begin at the proper position in the following
%    line, we simply pretend that the label does not take any
%    room.\footnote{This will lead to problems whenever very high
%    symbols occurring in the line tower into the
%    heading. So, something else has to be done here sometime.}
%    \begin{macrocode}
       \rlap{\vbox{\hbox{\hskip \labelsep\theorem@headerfont ##1\ ##2}%
                   \hbox{\strut}}}]}%
%    \end{macrocode}
%    Again, the definition of |\@opargbegintheorem| is
%    completely analogous.
%    \begin{macrocode}
\def\@opargbegintheorem##1##2##3{%
  \item[\rlap{\vbox{\hbox{\hskip \labelsep \theorem@headerfont
                     ##1\ ##2\ (##3)}%
                    \hbox{\strut}}}]}}
\endgroup
%</thb>
%    \end{macrocode}
% \end{macro}
%
%
% \subsubsection{The \texttt{changebreak} style}
%
%    \begin{macrocode}
% This style option is stored in the file |thcb.sty|.
%    \begin{macrocode}
%<*thcb>
\begingroup \makeatletter
\@ifundefined{theorem@style}{\input{theorem.sty}}{}
\ProvidesFile{thcb.sty}
             [\FMithmInfo]
%    \end{macrocode}
%
% \begin{macro}{\th@changebreak}
%    The \texttt{change-break} theorem style is like \texttt{break}
%    but with interchange of theorem name and theorem number.  Thus we
%    define |\th@changebreak| as follows:
%    \begin{macrocode}
\gdef\th@changebreak{\normalfont\slshape
  \def\@begintheorem##1##2{\item
       [\rlap{\vbox{\hbox{\hskip\labelsep \theorem@headerfont ##2\ ##1}%
                    \hbox{\strut}}}]}%
\def\@opargbegintheorem##1##2##3{%
  \item[\rlap{\vbox{\hbox{\hskip\labelsep \theorem@headerfont
                     ##2\ ##1\ (##3)}%
                    \hbox{\strut}}}]}}
\endgroup
%</thcb>
%    \end{macrocode}
% \end{macro}
%
%
% \subsubsection{The \texttt{change} style}
%
% This style option is stored in the file |thc.sty|.
%    \begin{macrocode}
%<*thc>
\begingroup \makeatletter
\@ifundefined{theorem@style}{\input{theorem.sty}}{}
\ProvidesFile{thc.sty}
             [\FMithmInfo]
%    \end{macrocode}
%
% \begin{macro}{\th@change}
%    The \texttt{change} theorem style corresponds to the
%    \texttt{change break} style without a linebreak after the header.
%    To say it in another way, it's the same as the \texttt{plain}
%    style but with number and name interchanged and |\slshape| as the
%    default font.
%    \begin{macrocode}
\gdef\th@change{\normalfont\slshape
  \def\@begintheorem##1##2{\item
       [\hskip\labelsep \theorem@headerfont ##2\ ##1]}%
\def\@opargbegintheorem##1##2##3{%
  \item[\hskip\labelsep \theorem@headerfont ##2\ ##1\ (##3)]}}
\endgroup
%</thc>
%    \end{macrocode}
% \end{macro}
%
%
% \subsubsection{The \texttt{marginbreak} style}
%
% This style option is the one used most often at Mainz.  It is saved
% in the file |thmb.sty|.
%    \begin{macrocode}
%<*thmb>
\begingroup \makeatletter
\@ifundefined{theorem@style}{\input{theorem.sty}}{}
\ProvidesFile{thmb.sty}
             [\FMithmInfo]
%    \end{macrocode}
%
% \begin{macro}{\th@marginbreak}
%    The \texttt{margin break} style is nearly the same as the
%    \texttt{change break} style. The only difference is the placement
%    of the theorem number.  We use |\llap| to place it in the left
%    margin.
%
%    In this style |\labelsep| denotes the separation between the
%    number and the text.
%    \begin{macrocode}
\gdef\th@marginbreak{\normalfont\slshape
  \def\@begintheorem##1##2{\item
       [\rlap{\vbox{\theorem@headerfont
                    \hbox{\llap{##2}\hskip\labelsep ##1}%
                    \hbox{\strut}}}]}%
\def\@opargbegintheorem##1##2##3{%
  \item[\rlap{\vbox{\theorem@headerfont
                    \hbox{\llap{##2}\hskip\labelsep ##1\ (##3)}%
                    \hbox{\strut}}}]}}
\endgroup
%</thmb>
%    \end{macrocode}
% \changes{v2.1c}{90/02/17}{`endgroup outside macrocode env.}
% \end{macro}
%
%
%
% \subsubsection{The \texttt{margin} style}
%
% This style option is stored in the file |thm.sty|.
%    \begin{macrocode}
%<*thm>
\begingroup \makeatletter
\@ifundefined{theorem@style}{\input{theorem.sty}}{}
\ProvidesFile{thm.sty}
             [\FMithmInfo]
%    \end{macrocode}
%
% \begin{macro}{\th@margin}
%    Again this is only a variant of the theorem styles described
%    above without any new ideas.
%    \begin{macrocode}
\gdef\th@margin{\normalfont\slshape
  \def\@begintheorem##1##2{\item
       [\theorem@headerfont \llap{##2}\hskip\labelsep ##1]}%
\def\@opargbegintheorem##1##2##3{%
  \item[\theorem@headerfont \llap{##2}\hskip\labelsep ##1\ (##3)]}}
\endgroup
%</thm>
%    \end{macrocode}
% \end{macro}
%
%
%
%
% \subsection{Final Definitions}
%
% \begin{macro}{\theorempreskipamount}
% \begin{macro}{\theorempostskipamount}
%    The \textsf{skip} parameters that regulate the vertical empty space
%    before and after the {\envfont theorem} environment have
%    to be allocated as well.
% \changes{v2.0f}{89/07/20}{Unnecessary `global' removed.}
%    \begin{macrocode}
%<*package>
\newskip\theorempreskipamount
\newskip\theorempostskipamount
%    \end{macrocode}
%    Since we have used the same values for all theorem sets, we now can
%    assign them.
%    \begin{macrocode}
\global\setlength\theorempreskipamount{12pt plus 5pt minus 3pt}
\global\setlength\theorempostskipamount{8pt plus 3pt minus 1.5pt}
%    \end{macrocode}
% \end{macro}
% \end{macro}
%
%
% \begin{macro}{\@endtheorem}
%    The same holds for the macro |\@endtheorem|, which ends a
%    {\envfont theorem} environment. Since it is the same for all
%    theorem sets, it is removed from the macros |\th@|\meta{style}.
%    It simply ends the {\envfont trivlist} environment, which was
%    begun in |\@thm|.
%    \begin{macrocode}
\global\let\@endtheorem=\endtrivlist
%    \end{macrocode}
% \end{macro}
%
%
%
% \begin{macro}{\@preamblecmds}
%    All macros defined above are to be used only in the preamble.
%    Therefore, we insert them in |\@preamblecmds| which will disable
%    them at begin document. This is done by the internal \LaTeXe{}
%    command |\@onlypreamble|.
% \changes{v2.0g}{89/09/19}{`theoremheaderfont and `theorembodyfont
%    added.}
% \changes{v2.2a}{94/02/03}{Use \LaTeXe{} interface}
%    \begin{macrocode}
\@onlypreamble\@xnthm
\@onlypreamble\@ynthm
\@onlypreamble\@othm
\@onlypreamble\newtheorem
\@onlypreamble\theoremstyle
\@onlypreamble\theorembodyfont
\@onlypreamble\theoremheaderfont
%    \end{macrocode}
% \end{macro}
%
% Finally we declare the \texttt{plain} theorem style to be the default.
% \changes{v2.1b}{90/06/04}{Fixed missing `item by declaring default
%    style.}
%    \begin{macrocode}
\theoremstyle{plain}
%</package>
%    \end{macrocode}
%
% \Finale

\endinput
}{}
\ProvidesFile{thc.sty}
             [\FMithmInfo]
%    \end{macrocode}
%
% \begin{macro}{\th@change}
%    The \texttt{change} theorem style corresponds to the
%    \texttt{change break} style without a linebreak after the header.
%    To say it in another way, it's the same as the \texttt{plain}
%    style but with number and name interchanged and |\slshape| as the
%    default font.
%    \begin{macrocode}
\gdef\th@change{\normalfont\slshape
  \def\@begintheorem##1##2{\item
       [\hskip\labelsep \theorem@headerfont ##2\ ##1]}%
\def\@opargbegintheorem##1##2##3{%
  \item[\hskip\labelsep \theorem@headerfont ##2\ ##1\ (##3)]}}
\endgroup
%</thc>
%    \end{macrocode}
% \end{macro}
%
%
% \subsubsection{The \texttt{marginbreak} style}
%
% This style option is the one used most often at Mainz.  It is saved
% in the file |thmb.sty|.
%    \begin{macrocode}
%<*thmb>
\begingroup \makeatletter
\@ifundefined{theorem@style}{% \iffalse meta-comment
%
% Copyright (C) 1993-2024
%
% The LaTeX Project and any individual authors listed elsewhere
% in this file.
%
% This file is part of the Standard LaTeX `Tools Bundle'.
% -------------------------------------------------------
%
% It may be distributed and/or modified under the
% conditions of the LaTeX Project Public License, either version 1.3c
% of this license or (at your option) any later version.
% The latest version of this license is in
%    https://www.latex-project.org/lppl.txt
% and version 1.3c or later is part of all distributions of LaTeX
% version 2005/12/01 or later.
%
% The list of all files belonging to the LaTeX `Tools Bundle' is
% given in the file `manifest.txt'.
%
% \fi
%
%%
%
%
\def\FMithmInfo{2023/07/05 v2.2c Theorem extension package (FMi)}
%
% \ProvidesFile{theorem.dtx}[\FMithmInfo]
%\iffalse   % this is a METACOMMENT !
%%
%% Package `theorem' to use with LaTeX2e
%% Copyright (C) 1989-1995 Frank Mittelbach, all rights reserved.
%<+package>\@ifundefined{theorem@style}{}{\endinput}
%<+package>\NeedsTeXFormat{LaTeX2e}[1995/06/01]
%<+package>\ProvidesPackage{theorem}
%<+package>         [\FMithmInfo]
%
% \fi
%
% \changes{v1.0}{87/05/07}{First draft.}
% \changes{v1.0a}{87/05/07}{`thistheoremstyle' added.}
% \changes{v1.1a}{87/07/29}{New concept: redefine the internal
%                          \protect\LaTeX{} commands `@xnthm' etc.}
% \changes{v1.1b}{87/07/31}{`theoremfont' used to specify font.}
% \changes{v1.2a}{88/01/08}{`@endtheorem' now globally defined.}
% \changes{v1.2b}{88/01/11}{`newtheorem' only allowed in the preamble.}
% \changes{v1.3a}{88/02/14}{`theoremfont' now a toks register.}
%
% \changes{v2.0a}{88/11/16}{New implementation and new documentation.}
% \changes{v2.0a}{88/11/16}{No layout changes but internal modifications
%                           old substyles won't work.}
% \changes{v2.0b}{89/01/30}{`global' used for all definitions.}
% \changes{v2.0c}{89/06/15}{Documentation translated to English.
%                            Thanks to LLL and Nina}
% \changes{v2.1a}{90/02/15}{Support for new font selection.}
% \changes{v2.2a}{94/02/03}{Cleanup for LaTeX2e}
%
% \DoNotIndex{\@endparenv,\@ifdefinable,\@ifnextchar,\@ifundefined}
% \DoNotIndex{\@namedef,\@nameuse,\@spaces,\@tempa,\@warning}
% \DoNotIndex{\ }
% \DoNotIndex{\begingroup,\bfseries}
% \DoNotIndex{\csname}
% \DoNotIndex{\def,\do,\docdate}
% \DoNotIndex{\endcsname,\endgroup,\endinput,\endtrivlist,\expandafter}
% \DoNotIndex{\filedate,\fileversion}
% \DoNotIndex{\gdef,\global}
% \DoNotIndex{\hbox,\hskip}
% \DoNotIndex{\ignorespaces,\input,\itshape,\item}
% \DoNotIndex{\labelsep,\let,\llap}
% \DoNotIndex{\makeatletter}
% \DoNotIndex{\noexpand}
% \DoNotIndex{\refstepcounter,\relax,\rlap}
% \DoNotIndex{\setlength,\slshape,\space,\string,\strut}
% \DoNotIndex{\the,\trivlist,\typeout}
% \DoNotIndex{\vbox}
% \DoNotIndex{\xdef}
%
% \def\deflabel#1{\ttfamily #1\hfill}
% \def\deflist#1{\begin{list}{}{\settowidth\labelwidth{\ttfamily #1}%
%                               \setlength\leftmargin\labelwidth
%                               \addtolength\leftmargin\labelsep
%                               \let\makelabel\deflabel}}
% \def\enddeflist{\end{list}}
% \def\envfont{\normalfont\ttfamily}
%
% \MakeShortVerb{\|}
%
% \newcommand{\TUB}{{\sl TUGboat\/}}
%
% \GetFileInfo{theorem.dtx}
% \title{An Extension of the
%        \LaTeX{} theorem environment\thanks{This file
%        has version number \fileversion, last
%        revised \filedate.}}
% \author{Frank Mittelbach}
%
% \MaintainedByLaTeXTeam{tools}
% \maketitle
%
% \begin{abstract}
%    The macros described in this paper yield an extension of the
%    \LaTeX{} theorem mechanism. It is designed to satisfy the
%    different requirements of various journals. Thus, the
%    layout of the ``theorems'' can be manipulated by determining a
%    ``style''. This article describes not only the use, but
%    also the definition, of the necessary macros.
% \end{abstract}
%
% \section*{Preface to version 2.2}
%
% For \LaTeXe{} this package did not need any fundamental changes. I
% only modified the messages generated so that theorem layout styles
% will show up with the |\listfiles| command and cleaned the section
% on the New Font Selection Scheme since this is now included in
% \LaTeX.
%
% \section*{Preface to version 2.1}
%
% This version is identical to 2.0g described in \TUB\ 10\#3 except
% for some internal defaults which are now set depending on the used
% font selection scheme.
%
% This was done to avoid unpleasant surprises if the new font selection
% scheme is in force. For further details see
% section~\ref{sec:newfonts} and \cite{bk:GMS94}.
%
% \section{Introduction}
%
% For our purposes here, ``theorems'' are labelled enunciations,
% often set off from the main text by extra space and a font change.
% Theorems, corollaries, conjectures, definitions, and remarks
% are all instances of ``theorems''.  The ``header'' of these structures
% is composed of a label (such as \textsc{Theorem} or \textsc{Remark})
% and a number which serializes an item in the sequence of items
% with the same label.
%
% Shortly after the introduction of \LaTeX{} at the Fachbereich
% Mathematik in Mainz, the desire to manipulate the layout of
% ``theorems'' arose.  In Mainz, the following two conventions came
% into general use:
% \begin{enumerate}
%   \item The number of the theorem is shown in the margin.
%   \item There is a line break at
%         the end of the theorem header.
% \end{enumerate}
%
% Additionally, some journals require different formats which depend on
% the ``sort of theorem'': e.g.\ often remarks and definitions are set
% in |\upshape|, while |\itshape| is employed for main theorems.
%
% Confronted with these requirements, a theorem environment was
% developed in Mainz which allows separate determination of the
% layout of the ``theorems sets'', comparable to |\pagestyle|.
%
%
% \section{The user interface}
% \subsection{Defining new theorem sets}
%
% \DescribeMacro\newtheorem
% As in the original \LaTeX{} version, the command |\newtheorem|
% defines a new ``theorem set'' or ``theorem-like structure''.
% Two required arguments name the new environment and give the text
% to be typeset with each instance of the new ``set'', while
% an optional argument determines how
% the ``set'' is enumerated:
% \begin{description}
%    \item[\ttfamily \bslash newtheorem\{foo\}\{bar\}]
%       The theorem set {\envfont foo} (whose name is \texttt{bar})
%       uses its own counter.
%    \item[\ttfamily \bslash newtheorem\{foo2\}{[foo]}\{bar2\}]
%       The theorem set {\envfont foo2} (printed name \texttt{bar2})
%       uses the same counter as the theorem set \texttt{foo}.
%    \item[\ttfamily \bslash newtheorem\{foo3\}\{bar3\}{[section]}]
%       The theorem set {\envfont foo3} (printed name \texttt{bar3}) is
%       enumerated within the counter \texttt{section}, i.e.\ with every
%       new |\section| the enumeration begins again with 1, and
%       the enumeration is composed from the section-number and the
%       theorem counter itself.
% \end{description}
%
% \DescribeMacro\theoremstyle
% Additionally, the command |\theoremstyle| can define the layout
% of various, or all, theorem sets.  It should be noted that any
% theorem set defined by |\newtheorem| is typeset in the
% |\theoremstyle| that is current at the time of the definition.
% Thus, the following
% \begin{verbatim}
%    \theoremstyle{break}        \newtheorem{Cor}{Corollary}
%    \theoremstyle{plain}        \newtheorem{Exa}{Example}[section]
%\end{verbatim}                             ^^A No blank at line start
% leads to the result that the set {\envfont Cor} is formatted in the
% style \texttt{break}, while the set {\envfont Exa} and all the
% following ones are formatted in the style \texttt{plain}, unless
% another |\theoremstyle| follows.  Since the definitions installed by
% |\newtheorem| are global, one also can limit |\theoremstyle| locally
% by grouping braces.
%
% \DescribeMacro\theorembodyfont
% The choice of the font for the theorem body is completely
% independent of the chosen |\theoremstyle|; this has proven to
% be very advantageous.  For example,
% \begin{verbatim}
%   {\theorembodyfont{\upshape}          \newtheorem{Rem}{Remark}}
%\end{verbatim}
% defines a theorem set {\envfont Rem}, which will be set in
% |\upshape| in the current layout (which in our example is
% \texttt{plain}).  As with |\theoremstyle|, the |\theorembodyfont|
% chosen is that current at the time of |\newtheorem|.  If
% |\theorembodyfont| is not specified or one defines
% |\theorembodyfont{}|, then the font used will be that defined by the
% |\theoremstyle|.
%
% \DescribeMacro\theoremheaderfont It is also possible to customize
% the font used for the theorem headers.  This is, however, a global
% declaration, and therefore there should be at most one
% |\theoremheaderfont| declaration in the preamble.\footnote{If it is
% actually necessary to have different header fonts, one has to define
% new theorem styles (substituting the desired font) or specify the
% information directly in the \texttt{\bslash newtheorem} declaration
% (the unclean variant).}
%
% \DescribeMacro\theorempreskipamount
% \DescribeMacro\theorempostskipamount Two additional parameters
% affect the vertical space around the theorem environments:
% |\theorempreskipamount| and |\theorempostskipamount| define,
% respectively, the spacing before and after such an environment.
% These parameters apply for all theorem sets and can be manipulated
% with the ordinary length macros.  They are rubber lengths
% (`\textsf{skips}'), and therefore can contain \texttt{plus} and
% \texttt{minus} parts.
%
% Since the definition of theorem sets should---most sensibly---be
% placed in the preamble, we only allow installation there.
% It is therefore possible to release the memory used here after
% |\begin{document}|, in order to make room for other
% applications.
%
%
% \subsection{Existing theorem styles}
%
% The following theorem styles exist to date:
% \begin{deflist}{marginbreak}
%    \item[plain]
%       This theorem style emulates the original \LaTeX{} definition,
%       except that additionally the parameters
%       |\theorem...skipamount| are used.
%    \item[break]
%       In this style, the theorem header is followed by a line
%       break.
%    \item[marginbreak]
%       The theorem number is set in the margin, and there is a line
%       break as in \texttt{break}.
%    \item[changebreak]
%       Like \texttt{break}, but with header number and text
%       interchanged.
%    \item[change]
%       Header number and text are interchanged, without a line break.
%    \item[margin]
%       The number is set in the left margin, without a line break.
% \end{deflist}
% All styles (except \texttt{plain}) select |\slshape| as the default
% |\theorembodyfont|.
%
%
% \subsection{Examples}
%
% Given the above theorem sets {\envfont Cor}, {\envfont Exa}
% and {\envfont Rem},
% suppose that the preamble also contains the declarations:
% \begin{verbatim}
%    \theoremstyle{marginbreak}   \newtheorem{Lem}[Cor]{Lemma}
%    \theoremstyle{change}
%    \theorembodyfont{\itshape}        \newtheorem{Def}[Cor]{Definition}
%
%    \theoremheaderfont{\scshape}
%\end{verbatim}
% Then the following are some typical examples of the typeset output
% resulting from their use.
%
% \begin{Cor}
%    This is a sentence typeset in the theorem environment {\envfont
%    Cor}.
% \end{Cor}
%
% \begin{Exa}
%    This is a sentence typeset in the theorem environment {\envfont
%    Exa}.
% \end{Exa}
%
% \begin{Rem}
%    This is a sentence typeset in the theorem environment {\envfont
%    Rem}.
% \end{Rem}
%
% \begin{Lem}[Ben User]
%    This is a sentence typeset in the theorem environment {\envfont
%    Lem}.
% \end{Lem}
%
% \begin{Def}[Very impressive Definition]
%    This is a sentence typeset in the theorem environment {\envfont
%    Def}.
% \end{Def}
%
% The last two examples show the effect of the optional argument to
% a theorem environment (it is the text typeset in parentheses).
%
%
% \section{Special Considerations}
% \label{sec:newfonts}
%
% Theoremheader and body are implemented as a unit. This means that
% the |\theoremheaderfont|  will inherit characteristics of the
% |\theorembodyfont| in \LaTeXe. Thus, if for example
% |\theorembodyfont| is |\itshape| and |\theoremheaderfont|
% is |\bfseries| the font selected for the header will have the
% characteristics `bold extended italic'. If this is not desired one
% should set the |\theoremheaderfont| to something like
% \begin{verbatim}
%      \theoremheaderfont{\normalfont\bfseries}
%\end{verbatim}
% i.e.\ supplying all necessary font information explicitly.
%
%
%
% \section{Acknowledgements}
%
% The publication of this set of macros was only possible with the
% help of Christina Busse (translating the manuscript into English),
% Joachim Pense (playing the r\^ole of typist), Chris Rowley
% (looking everything over) and many others providing useful
% suggestions.
%
% \MaybeStop{%
%    \begin{thebibliography}{9}
%       \bibitem{bk:GMS94} \textsc{M.~Goossens}, \textsc{F.~Mittelbach}
%         and \textsc{A.~Samarin}.
%         \newblock The \LaTeX{} Companion.
%         \newblock
%         Addison-Wesley, Reading, Massachusetts, 1994.
%       \bibitem{latex:source} \textsc{Lamport, Leslie.} \newblock
%                               \texttt{latex.tex}, version 2.09, date
%                               Feb. 1990.
%    \end{thebibliography}
%    \PrintIndex
%    \PrintChanges}
%
%
% \section{The documentation driver file}
%
% The next bit of code contains the documentation driver file for
% \TeX{}, i.e., the file that will produce the documentation you are
% currently reading. It will be extracted from this file by the
% \texttt{docstrip} program. Since it is the first code in the file
% one can alternatively process this file directly with \LaTeXe{} to
% obtain the documentation.
%
%    \begin{macrocode}
%<*driver>
\documentclass{ltxdoc}

\usepackage{theorem}

%<+driver>% The next few lines define theorem sets which are used
%<+driver>% in the example section of the documentation.

\theoremstyle{break}       \newtheorem{Cor}{Corollary}
\theoremstyle{plain}       \newtheorem{Exa}{Example}[section]
{\theorembodyfont{\upshape}\newtheorem{Rem}{Remark}}
\theoremstyle{marginbreak} \newtheorem{Lem}[Cor]{Lemma}
\theoremstyle{change}
\theorembodyfont{\itshape} \newtheorem{Def}[Cor]{Definition}

\theoremheaderfont{\scshape}

\RecordChanges

\begin{document}
   \DocInput{theorem.dtx}
\end{document}
%</driver>
%    \end{macrocode}
%
% \section{Definition of the Macros}
%
% If the file has been loaded before, we abort immediately. If not the
% package announces itself (this is actually done at the very top of
% the file---the way it is done isn't good style so don't copy it).
% \changes{v2.0e}{89/07/19}{Spaces removed from `typeout'.}
% \changes{v2.2b}{95/11/19}{Announce moved to top of file}
%    \begin{macrocode}
%<*package>
%\@ifundefined{theorem@style}{}{\endinput}
%\def\FMithmInfo{1995/11/19 v2.2b Theorem extension package (FMi)}
%\ProvidesPackage{theorem}[\FMithmInfo]
%    \end{macrocode}
%
%
% \subsection{Definition of theorem styles and fonts}
%
% All the definitions in this file are done globally to allow
% inputting this file inside a group.
%
% \begin{macro}{\theoremstyle}
%    Before a theorem style can be installed, the chosen style must be
%    known. For that reason, we must test to see that
%    |\th@|\meta{style} is known or, more precisely, that it is
%    different from |\relax|. If the style is not known then
%    |\th@plain| is used.
%    \begin{macrocode}
\gdef\theoremstyle#1{%
   \@ifundefined{th@#1}{\@warning
          {Unknown theoremstyle `#1'. Using `plain'}%
          \theorem@style{plain}}%
%    \end{macrocode}
%    We save the theorem style to be used in the token register
%    |\theorem@style|.
%    \begin{macrocode}
      {\theorem@style{#1}}%
%    \end{macrocode}
%    Now we ``evaluate'' the theorem style: this means, we call the
%    macro |\th@|\meta{style} which will activate the relevant
%    definitions which are contained in a separate file.  This is done
%    in a group to suppress changes to the current font.  This could
%    otherwise pose problems together with the new font selection
%    scheme\footnote{When I printed the original article using the new
%    font selection scheme I ended with a document with slanted
%    typefaces (text headings and all) simply because one of the
%    theorem styles used \texttt{\bslash sl} at toplevel.} if the
%    |\th@|\meta{style} is evaluated a second time.
% \changes{v2.1a}{90/02/15}{Group added around `th@..}
% \changes{v2.1a}{90/02/15}{Reseting of `..@begintheorem removed.}
%    \begin{macrocode}
      \begingroup
        \csname th@\the\theorem@style \endcsname
      \endgroup}
%    \end{macrocode}
% \end{macro}
%
%
% \begin{macro}{\@begintheorem}
% \changes{v2.1a}{90/02/15}{Relaxed at toplevel.}
% \begin{macro}{\@opargbegintheorem}
% \changes{v2.1a}{90/02/15}{Relaxed at toplevel.}
%    We reset |\@begintheorem| and |\@opargbegintheorem| to
%    |\relax| since these commands are no longer necessary at
%    toplevel. This will save a few tokens.
%    \begin{macrocode}
\global\let\@begintheorem\relax
\global\let\@opargbegintheorem\relax
%    \end{macrocode}
% \end{macro}
% \end{macro}
%
% \begin{macro}{\theorem@style}
%    Obviously the \textsf{token} register used above has to be
%    allocated. To assure the utmost compatibility with the
%    original \LaTeX{} definition, we set the default theorem style to
%    \texttt{plain}, which implements the usual \LaTeX{} convention.
% \changes{v2.0f}{89/07/20}{Unnecessary `global' removed.}
%    \begin{macrocode}
\newtoks\theorem@style
\global\theorem@style{plain}
%    \end{macrocode}
% \end{macro}
%
% \begin{macro}{\theorembodyfont}
% \begin{macro}{\theorem@bodyfont}
%    If the |\theorembodyfont| is set by the user then it should not
%    interact with the default font set in the theorem style. When the
%    new font selection is in force this may happen if, for example,
%    the default is |\itshape| and the new |\theorembodyfont| is
%    |\sffamily|. So we add a |\reset@font| command in front of the
%    user definition.
% \changes{v2.0f}{89/07/20}{Unnecessary `global' removed.}
% \changes{v2.0g}{89/09/19}{`theoremfont to `theorembodyfont.}
% \changes{v2.1c}{92/06/22}{`reset@font added and logic changed.}
% \changes{v2.1d}{92/10/28}{Reset to empty if arg is empty}
%    \begin{macrocode}
\gdef\theorembodyfont#1{%
%    \end{macrocode}
%    We check if the argument supplied is empty and if so put nothing
%    into the |\theorem@bodyfont| token register to allow for
%    |\theorembodyfont{}| as a mean of using the default of the
%    current |\theoremstyle|.
%    \begin{macrocode}
   \def\@tempa{#1}%
   \ifx\@tempa\@empty
    \theorem@bodyfont{}%
   \else
    \theorem@bodyfont{\reset@font#1}%
   \fi
   }
\newtoks\theorem@bodyfont
\global\theorem@bodyfont{}
%    \end{macrocode}
% \end{macro}
% \end{macro}
%
%
% \begin{macro}{\theoremheaderfont}
% \changes{v2.0g}{89/09/19}{Macro added.}
%    The font for the theorem headers is handled differently because
%    this definition applies to all theorem styles.
%    \begin{macrocode}
\gdef\theoremheaderfont#1{\gdef\theorem@headerfont{#1}%
%    \end{macrocode}
%    After using the macro once it is redefined to produce an error
%    message.
%    \begin{macrocode}
       \gdef\theoremheaderfont##1{%
        \typeout{\string\theoremheaderfont\space should be used
                 only once.}}}
%    \end{macrocode}
% \end{macro}
%
% \begin{macro}{\theorem@headerfont}
% \changes{v2.0g}{89/09/19}{Macro added.}
%    To set the |\theorem@headerfont| default we first test if
%    the new fontselection scheme is in force.
% \changes{v2.1a}{90/02/15}{Support for new font selection.}
%    \begin{macrocode}
\ifx\upshape\undefined
%    \end{macrocode}
%    If not we define it to expand into |\bfseries|. We don't use
%    |\let| just in case a following style option redefines this
%    macro.
%    \begin{macrocode}
\gdef\theorem@headerfont{\bfseries}
%    \end{macrocode}
%    Otherwise we reset the current shape before calling |\bfseries|.
%    \begin{macrocode}
\else \gdef\theorem@headerfont{\normalfont\bfseries}\fi
%    \end{macrocode}
% \end{macro}
%
% \begin{macro}{\th@plain}
% \begin{macro}{\th@break}
% \begin{macro}{\th@marginbreak}
% \begin{macro}{\th@changebreak}
% \begin{macro}{\th@change}
% \begin{macro}{\th@margin}
%    The different styles are defined in macros such as
%    |\th@plain|.  Since memory space is precious in
%    ``non-Big-versions'', we have to avoid offering too many unused
%    definitions.  Therefore we define these styles in separate files
%    that can be loaded on demand.  Thus the commands themselves only
%    load these files. We use |\@input@|, a \LaTeXe{} internal command
%    that ensures that the file will be listed with |\listfiles|.
% \changes{v2.2a}{94/02/02}{Use `@input@ to load theorem layout files}
%    \begin{macrocode}
\gdef\th@plain{\@input@{thp.sty}}
\gdef\th@break{\@input@{thb.sty}}
\gdef\th@marginbreak{\@input@{thmb.sty}}
\gdef\th@changebreak{\@input@{thcb.sty}}
\gdef\th@change{\@input@{thc.sty}}
\gdef\th@margin{\@input@{thm.sty}}
%    \end{macrocode}
%    This list will be expanded when new styles become
%    available.  For testing,  just append new theorem substyles
%    as document options.
% \end{macro}
% \end{macro}
% \end{macro}
% \end{macro}
% \end{macro}
% \end{macro}
%
%
%
% \subsection{Definition of a new theorem set}
%
% As already pointed out, a new theorem environment can be defined in
% three different ways:
% \begin{verbatim}
%    \newtheorem{Lem}{Lemma}
%    \newtheorem{Lem}{Lemma}[section]
%    \newtheorem{Lem}[Theorem]{Lemma}
%\end{verbatim}
% The function of the macro |\newtheorem| is to recognize these cases
% and then to branch into one of the three macros |\@ynthm|, |\@xnthm|
% or |\@othm|.  This mechanism is adopted unchanged from
% \cite{latex:source}; the essential point here is that, for example,
% in the second case, the arguments \texttt{Lem}, \texttt{Lemma} and
% \texttt{section} are passed over to the macro |\@xnthm|.
%
% We inspect this case first because the others present fewer problems,
% and thus are easily derived from this one.
%
% \begin{macro}{\@xnthm}
%    For our example arguments, the macro |\@xnthm| must fulfill
%    the following:
%    \begin{itemize}
%       \item
%          Define a new \LaTeX-counter `Lem'
%       \item
%          reset this counter within a |\section|
%       \item
%          define the macro |\theLem|
%       \item
%          define the environment macros |\Lem| and
%          |\endLem| using the current
%          |\theoremstyle| and |\theorem@bodyfont|.
%    \end{itemize}
%    Obviously, all this should happen only if the first argument
%    of |\@xnthm| (i.e.\ \texttt{Lem} in our example) is chosen so
%    as not to conflict with any previously defined commands or
%    environments.  This test is performed by the \LaTeX{} macro
%    |\@ifdefinable|.
%    \begin{macrocode}
\gdef\@xnthm#1#2[#3]{\expandafter\@ifdefinable\csname #1\endcsname
%    \end{macrocode}
%    Therefore, the first argument of |\@ifdefinable| is the
%    expansion (in the example, |\Lem|) of
%    |\csname||#1\endcsname|.  The second argument is
%    executed only if the test has been completed successfully.
%    \begin{macrocode}
   {%
%    \end{macrocode}
%    Now we define the new counter. The names of the \LaTeX{}
%    macros employed should speak for themselves:
% \changes{v2.2b}{1995/10/07}{Use \cs{@newctr}}
% \changes{v2.2c}{1995/11/23}{Use \cs{@newctr} properly.}
%    \begin{macrocode}
    \@definecounter{#1}\@newctr{#1}[#3]%
%    \end{macrocode}
%    Using |\@newctr| will give a proper error message if the counter
%    in |#3| is not defined.
%    In defining `|\theLem|' we must generate the desired macro
%    name by use of |\expandafter| and |\csname|.
%    \begin{macrocode}
    \expandafter\xdef\csname the#1\endcsname
%    \end{macrocode}
%    An |\xdef| is used in order to make the definition global,
%    and to ensure that it contains the replacement texts of
%    |\@thmcountersep| and |\@thmcounter|.\footnote{These
%    two macros can be defined by the document style.  Their default
%    values produce a `\texttt{.}' as separation and an arabic
%    representation of the number.} However, not everything should be
%    expanded. For example, it saves space to use |\thesection|
%    instead of its---at times---lengthy expansion.
%    \begin{macrocode}
      {\expandafter \noexpand \csname the#3\endcsname
       \@thmcountersep \@thmcounter{#1}}%
%    \end{macrocode}
%    Thus with the defaults of \LaTeX{}, |\theLem| would be
%    replaced by the command sequence |\thesection.\arabic{Lem}|.
%
%    We will now look at the definition of the macro which is executed
%    at the beginning of the actual environment (in our example this
%    macro is |\Lem|). It should be noted that we use an
%    ``|\expandafter| trick'' to expand only certain parts of the
%    replacement text at the time of the definition.
%    \begin{macrocode}
    \def\@tempa{\global\@namedef{#1}}%
    \expandafter \@tempa \expandafter{%
%    \end{macrocode}
%    First, the macro that contains the current definitions of
%    |\@begintheorem| and |\@opargtheorem| should be called
%    up.  The name of this macro---as is already known---has the form
%    |\th@|\meta{theorem\ style}; therefore, it must be called by
%    \begin{macrocode}
      \csname th@\the \theorem@style
%    \end{macrocode}
%    In addition the default theorem font should be changeable, i.e.\
%    we have to insert the contents of |\theorem@bodyfont|. For
%    that reason, we expand even further, beyond |\endcsname|,
%    and thus insert the contents of the \textsf{token} register
%    |\theorem@bodyfont| in the replacement text.
%    \begin{macrocode}
            \expandafter \endcsname \the \theorem@bodyfont
%    \end{macrocode}
%    Now it is time to call the macro |\@thm| which takes over
%    the further processing.  It has two arguments: the current
%    counter name (in our example, \texttt{Lem}), and the text of the
%    label (in our example, \texttt{Lemma}).
%    \begin{macrocode}
     \@thm{#1}{#2}}%
%    \end{macrocode}
%    With this, the `sub-definition' is complete.  The macro
%    |\@endtheorem| ends a theorem environment and is, so far, nothing
%    but an |\endtrivlist|.  (Hence it is defined globally, and not
%    within the theorem styles.\footnote{This has to be changed as
%    soon as theorem styles that change \texttt{\bslash @endtheorem}
%    exist.  In such a case, all existing styles must be changed as
%    well since they will have to reset the macro.}) Therefore, we can
%    set it equivalent to the macro that ends the theorem set (in our
%    example, |\endLem|).  However, if some day theorem styles exist
%    that do change |\@endtheorem|, we would have to use the
%    commented-out line instead.
%    \begin{macrocode}
    \global \expandafter \let \csname end#1\endcsname \@endtheorem
%   \global\@namedef{end#1}{\@endtheorem}%
%    \end{macrocode}
%    With these commands all the required definitions are employed,
%    unless the test |\@ifdefinable| has failed.  Therefore, we
%    end the second argument of this macro and with it the definition
%    of |\@xnthm|.
%    \begin{macrocode}
   }}
%    \end{macrocode}
% \end{macro}
%
% \begin{macro}{\@ynthm}
%    The definition of |\@ynthm| is completely analogous. In
%    this case the new counter that is defined is not reset within
%    another counter; thus the definition of |\the...|
%    is simplified:
%    \begin{macrocode}
\gdef\@ynthm#1#2{\expandafter\@ifdefinable\csname #1\endcsname
   {\@definecounter{#1}%
    \expandafter\xdef\csname the#1\endcsname{\@thmcounter{#1}}%
%    \end{macrocode}
%    The rest of the definition corresponds literally to that of
%    |\@xnthm|:
%    \begin{macrocode}
    \def\@tempa{\global\@namedef{#1}}\expandafter \@tempa
     \expandafter{\csname th@\the \theorem@style \expandafter
     \endcsname \the\theorem@bodyfont \@thm{#1}{#2}}%
    \global \expandafter \let \csname end#1\endcsname \@endtheorem}}
%    \end{macrocode}
% \end{macro}
%
% \begin{macro}{\@othm}
%    The definition of |\@othm| does not contain anything new.
%    \begin{macrocode}
\gdef\@othm#1[#2]#3{%
%    \end{macrocode}
% \changes{v2.2b}{95/10/07}{Build in check for counter defined}
%    We do not define a new counter but instead use one that has
%    already been defined.  Thus the only definition we need is that
%    of this pseudo-counter (i.e.\ |\the|\meta{env.\ name}).
%    First we check if |#2| corresponds to a known counter name.
%    \begin{macrocode}
  \expandafter\ifx\csname c@#2\endcsname\relax
   \@nocounterr{#2}%
  \else
   \expandafter\@ifdefinable\csname #1\endcsname
   {\expandafter \xdef \csname the#1\endcsname
     {\expandafter \noexpand \csname the#2\endcsname}%
%    \end{macrocode}
%    All other parts of the definition can be adopted from
%    |\@xnthm|. We have to remember, though, that in this case
%    the name of the current counter and the theorem label have moved
%    to the second and third arguments.
%    \begin{macrocode}
    \def\@tempa{\global\@namedef{#1}}\expandafter \@tempa
     \expandafter{\csname th@\the \theorem@style \expandafter
     \endcsname \the\theorem@bodyfont \@thm{#2}{#3}}%
    \global \expandafter \let \csname end#1\endcsname \@endtheorem}%
  \fi}
%    \end{macrocode}
% \end{macro}
%
% \subsection{Macros that are employed in a {\envfont theorem}
%    environment}
%
% \begin{macro}{\@thm}
%    The macro |\@thm| has to increase the current counter.  Then,
%    depending on whether the environment has (or does not have) an
%    optional argument, it has to branch into either |\@begintheorem|
%    or |\@opargtheorem|.
%    \begin{macrocode}
\gdef\@thm#1#2{\refstepcounter{#1}%
%    \end{macrocode}
%    Now we start a {\envfont trivlist} environment, and give
%    |\@topsep| and |\@topsepadd| the values of the \textsf{skip}
%    registers |\theorempreskipamount| and |\theorempostskipamount|.
%    The value in |\@topsep| is the vertical space that is inserted by
%    the first (and only) |\item| in our |\trivlist| whilst
%    |\@topsepadd| is inserted by |\@endparenv| at the end of that
%    {\envfont trivlist} environment.  By using these registers, we
%    obtain the desired space around a {\envfont theorem} environment.
%    \begin{macrocode}
   \trivlist
   \@topsep \theorempreskipamount               % used by first \item
   \@topsepadd \theorempostskipamount           % used by \@endparenv
%    \end{macrocode}
%    Now we have to test whether an optional argument has been given.
% \changes{v2.0d}{89/07/18}{Forgotten percent added.}
%    \begin{macrocode}
   \@ifnextchar [%
%    \end{macrocode}
%    If there is an optional argument, we will call |\@ythm|, and
%    move the arguments read back into the input stream.
%    \begin{macrocode}
   {\@ythm{#1}{#2}}%
%    \end{macrocode}
%    If not, we call |\@begintheorem|. Its first argument is the
%    name of the theorem set (hence the second argument of
%    |\@thm|). Its second argument is the macro that produces the
%    current number.
%    \begin{macrocode}
   {\@begintheorem{#2}{\csname the#1\endcsname}\ignorespaces}}
%    \end{macrocode}
% \end{macro}
%
% \begin{macro}{\@xthm}
% \begin{macro}{\@ythm}
%    Both these macros were originally called by |\@thm|.  We do
%    not need |\@xthm| anymore, hence we reset it to
%    |\relax|. The definition of |\@ythm| has not changed at
%    all from its definition in \LaTeX. In order to make the macros
%    easier to understand, we will
%    nevertheless present it (commented out).
%    \begin{macrocode}
\global\let\@xthm\relax
% \def\@ythm#1#2[#3]{\@opargbegintheorem{#2}{\csname
%           the#1\endcsname}{#3}\ignorespaces}
%</package>
%    \end{macrocode}
%    The primitive |\ignorespaces| in |\@ythm| and
%    |\@thm| is needed to remove the spaces between the
%    |\begin{...}| and the actual text.
% \end{macro}
% \end{macro}
%
%
% \subsection{Definition of the theorem substyles}
%
% As already pointed out, the theorem substyles, defined below, are
% only loaded when necessary.  Note that all these substyles, except
% \texttt{plain}, have |\slshape| as the default body font.
%
% \subsubsection{The \texttt{plain} style}
%
% \label{dok:th@plain} As the following macros use |@|, we have
% to locally set the |\catcode| of this symbol to ``letter''.
% This happens within a group, so that we do not have to worry about
% which |\catcode| that symbol had before.
%    \begin{macrocode}
%<*thp>
\begingroup \makeatletter
%    \end{macrocode}
% Since we are now within a group, we must make all definitions
% globally.  First we make sure that \texttt{theorem.sty} is loaded.
% This will allow us to use this file as a document style option
% without having to call \texttt{theorem} itself as an option.  At the
% same time, we assure that at least version 2 is loaded, since
% |\theorem@style| was not defined in earlier versions.
%    \begin{macrocode}
\@ifundefined{theorem@style}{\input{theorem.sty}}{}
\ProvidesFile{thp.sty}
             [\FMithmInfo]
%    \end{macrocode}
%
%
% \begin{macro}{\th@plain}
% \begin{macro}{\@begintheorem}
% \begin{macro}{\@opargbegintheorem}
%    |\theoremstyle{plain}| corresponds to the original
%    definition, except that the distances to the surrounding text are
%    determined by the parameters |\theorempreskipamount| and
%    |\theorempostskipamount|.
%    First we set the default body font.
%    \begin{macrocode}
\gdef\th@plain{\normalfont\itshape
%    \end{macrocode}
%    Then we define |\@begintheorem| and
%    |\@opargbegintheorem|. These two macros define how the
%    header of a theorem is typeset.  |\@opargbegintheorem| will
%    be called if a \textsf{theorem} environment with an optional
%    argument is encountered; otherwise, the header is constructed by
%    calling |\@begintheorem|.  If one of these macros is
%    executed, we are within a \textsf{trivlist} environment started by
%    |\@thm|.  So the theorem header is produced with an
%    |\item| command.
%
%    Instead of specifying the header font directly, all standard
%    theorem styles use the |\theorem@headerfont| macro to allow
%    customization.  The extra space (|\labelsep|) is necessary
%    because of problems in the \textsf{trivlist} environment.
%    \begin{macrocode}
  \def\@begintheorem##1##2{%
        \item[\hskip\labelsep \theorem@headerfont ##1\ ##2]}%
%    \end{macrocode}
%    The definition of |\@opargbegintheorem| is completely
%    analogous. The only difference is the fact that there exists a
%    third argument (which is the optional parameter of the
%    environment and contains additional information about the
%    theorem).  Customarily we enclose it in parentheses.
%    \begin{macrocode}
\def\@opargbegintheorem##1##2##3{%
   \item[\hskip\labelsep \theorem@headerfont ##1\ ##2\ (##3)]}}
%    \end{macrocode}
% \end{macro}
% \end{macro}
% \end{macro}
%
% \noindent
% We conclude with an |\endgroup| to restore the
% |\catcode| of |@|.
%    \begin{macrocode}
\endgroup
%</thp>
%    \end{macrocode}
%
%
% \subsubsection{The \texttt{break} style}
%
% This style option is stored in the file |thb.sty|.
% For the next two lines see the documentation for |\th@plain| on
% page~\pageref{dok:th@plain}.
%    \begin{macrocode}
%<*thb>
\begingroup \makeatletter
\@ifundefined{theorem@style}{\input{theorem.sty}}{}
\ProvidesFile{thb.sty}
             [\FMithmInfo]
%    \end{macrocode}
%
% \begin{macro}{\th@break}
%    |\theoremstyle{break}| produces a line break after the name
%    of the theorem.  The font is |\slshape|.  Hence, we define
%    |\th@break| as follows:
%    \begin{macrocode}
\gdef\th@break{\normalfont\slshape
  \def\@begintheorem##1##2{\item[%
%    \end{macrocode}
%    We run into the following problem: it is not possible to create
%    the header with |\item[|\meta{title}|]| and then start
%    a new line by, for example, |\mbox{}\\|. Such a definition
%    will fail whenever a list environment follows immediately.  With
%    the above construction, the |\mbox{}| causes the switch
%    |@inlabel| (cf.\ definition of |\list| and
%    |\trivlist| in \cite{latex:source}) to be set to \texttt{false}
%    and so the following list will insert additional vertical
%    space (|\topskip|). This is quite annoying. Therefore, we
%    create the line break within the |\item|. In order to ensure
%    that the text will begin at the proper position in the following
%    line, we simply pretend that the label does not take any
%    room.\footnote{This will lead to problems whenever very high
%    symbols occurring in the line tower into the
%    heading. So, something else has to be done here sometime.}
%    \begin{macrocode}
       \rlap{\vbox{\hbox{\hskip \labelsep\theorem@headerfont ##1\ ##2}%
                   \hbox{\strut}}}]}%
%    \end{macrocode}
%    Again, the definition of |\@opargbegintheorem| is
%    completely analogous.
%    \begin{macrocode}
\def\@opargbegintheorem##1##2##3{%
  \item[\rlap{\vbox{\hbox{\hskip \labelsep \theorem@headerfont
                     ##1\ ##2\ (##3)}%
                    \hbox{\strut}}}]}}
\endgroup
%</thb>
%    \end{macrocode}
% \end{macro}
%
%
% \subsubsection{The \texttt{changebreak} style}
%
%    \begin{macrocode}
% This style option is stored in the file |thcb.sty|.
%    \begin{macrocode}
%<*thcb>
\begingroup \makeatletter
\@ifundefined{theorem@style}{\input{theorem.sty}}{}
\ProvidesFile{thcb.sty}
             [\FMithmInfo]
%    \end{macrocode}
%
% \begin{macro}{\th@changebreak}
%    The \texttt{change-break} theorem style is like \texttt{break}
%    but with interchange of theorem name and theorem number.  Thus we
%    define |\th@changebreak| as follows:
%    \begin{macrocode}
\gdef\th@changebreak{\normalfont\slshape
  \def\@begintheorem##1##2{\item
       [\rlap{\vbox{\hbox{\hskip\labelsep \theorem@headerfont ##2\ ##1}%
                    \hbox{\strut}}}]}%
\def\@opargbegintheorem##1##2##3{%
  \item[\rlap{\vbox{\hbox{\hskip\labelsep \theorem@headerfont
                     ##2\ ##1\ (##3)}%
                    \hbox{\strut}}}]}}
\endgroup
%</thcb>
%    \end{macrocode}
% \end{macro}
%
%
% \subsubsection{The \texttt{change} style}
%
% This style option is stored in the file |thc.sty|.
%    \begin{macrocode}
%<*thc>
\begingroup \makeatletter
\@ifundefined{theorem@style}{\input{theorem.sty}}{}
\ProvidesFile{thc.sty}
             [\FMithmInfo]
%    \end{macrocode}
%
% \begin{macro}{\th@change}
%    The \texttt{change} theorem style corresponds to the
%    \texttt{change break} style without a linebreak after the header.
%    To say it in another way, it's the same as the \texttt{plain}
%    style but with number and name interchanged and |\slshape| as the
%    default font.
%    \begin{macrocode}
\gdef\th@change{\normalfont\slshape
  \def\@begintheorem##1##2{\item
       [\hskip\labelsep \theorem@headerfont ##2\ ##1]}%
\def\@opargbegintheorem##1##2##3{%
  \item[\hskip\labelsep \theorem@headerfont ##2\ ##1\ (##3)]}}
\endgroup
%</thc>
%    \end{macrocode}
% \end{macro}
%
%
% \subsubsection{The \texttt{marginbreak} style}
%
% This style option is the one used most often at Mainz.  It is saved
% in the file |thmb.sty|.
%    \begin{macrocode}
%<*thmb>
\begingroup \makeatletter
\@ifundefined{theorem@style}{\input{theorem.sty}}{}
\ProvidesFile{thmb.sty}
             [\FMithmInfo]
%    \end{macrocode}
%
% \begin{macro}{\th@marginbreak}
%    The \texttt{margin break} style is nearly the same as the
%    \texttt{change break} style. The only difference is the placement
%    of the theorem number.  We use |\llap| to place it in the left
%    margin.
%
%    In this style |\labelsep| denotes the separation between the
%    number and the text.
%    \begin{macrocode}
\gdef\th@marginbreak{\normalfont\slshape
  \def\@begintheorem##1##2{\item
       [\rlap{\vbox{\theorem@headerfont
                    \hbox{\llap{##2}\hskip\labelsep ##1}%
                    \hbox{\strut}}}]}%
\def\@opargbegintheorem##1##2##3{%
  \item[\rlap{\vbox{\theorem@headerfont
                    \hbox{\llap{##2}\hskip\labelsep ##1\ (##3)}%
                    \hbox{\strut}}}]}}
\endgroup
%</thmb>
%    \end{macrocode}
% \changes{v2.1c}{90/02/17}{`endgroup outside macrocode env.}
% \end{macro}
%
%
%
% \subsubsection{The \texttt{margin} style}
%
% This style option is stored in the file |thm.sty|.
%    \begin{macrocode}
%<*thm>
\begingroup \makeatletter
\@ifundefined{theorem@style}{\input{theorem.sty}}{}
\ProvidesFile{thm.sty}
             [\FMithmInfo]
%    \end{macrocode}
%
% \begin{macro}{\th@margin}
%    Again this is only a variant of the theorem styles described
%    above without any new ideas.
%    \begin{macrocode}
\gdef\th@margin{\normalfont\slshape
  \def\@begintheorem##1##2{\item
       [\theorem@headerfont \llap{##2}\hskip\labelsep ##1]}%
\def\@opargbegintheorem##1##2##3{%
  \item[\theorem@headerfont \llap{##2}\hskip\labelsep ##1\ (##3)]}}
\endgroup
%</thm>
%    \end{macrocode}
% \end{macro}
%
%
%
%
% \subsection{Final Definitions}
%
% \begin{macro}{\theorempreskipamount}
% \begin{macro}{\theorempostskipamount}
%    The \textsf{skip} parameters that regulate the vertical empty space
%    before and after the {\envfont theorem} environment have
%    to be allocated as well.
% \changes{v2.0f}{89/07/20}{Unnecessary `global' removed.}
%    \begin{macrocode}
%<*package>
\newskip\theorempreskipamount
\newskip\theorempostskipamount
%    \end{macrocode}
%    Since we have used the same values for all theorem sets, we now can
%    assign them.
%    \begin{macrocode}
\global\setlength\theorempreskipamount{12pt plus 5pt minus 3pt}
\global\setlength\theorempostskipamount{8pt plus 3pt minus 1.5pt}
%    \end{macrocode}
% \end{macro}
% \end{macro}
%
%
% \begin{macro}{\@endtheorem}
%    The same holds for the macro |\@endtheorem|, which ends a
%    {\envfont theorem} environment. Since it is the same for all
%    theorem sets, it is removed from the macros |\th@|\meta{style}.
%    It simply ends the {\envfont trivlist} environment, which was
%    begun in |\@thm|.
%    \begin{macrocode}
\global\let\@endtheorem=\endtrivlist
%    \end{macrocode}
% \end{macro}
%
%
%
% \begin{macro}{\@preamblecmds}
%    All macros defined above are to be used only in the preamble.
%    Therefore, we insert them in |\@preamblecmds| which will disable
%    them at begin document. This is done by the internal \LaTeXe{}
%    command |\@onlypreamble|.
% \changes{v2.0g}{89/09/19}{`theoremheaderfont and `theorembodyfont
%    added.}
% \changes{v2.2a}{94/02/03}{Use \LaTeXe{} interface}
%    \begin{macrocode}
\@onlypreamble\@xnthm
\@onlypreamble\@ynthm
\@onlypreamble\@othm
\@onlypreamble\newtheorem
\@onlypreamble\theoremstyle
\@onlypreamble\theorembodyfont
\@onlypreamble\theoremheaderfont
%    \end{macrocode}
% \end{macro}
%
% Finally we declare the \texttt{plain} theorem style to be the default.
% \changes{v2.1b}{90/06/04}{Fixed missing `item by declaring default
%    style.}
%    \begin{macrocode}
\theoremstyle{plain}
%</package>
%    \end{macrocode}
%
% \Finale

\endinput
}{}
\ProvidesFile{thmb.sty}
             [\FMithmInfo]
%    \end{macrocode}
%
% \begin{macro}{\th@marginbreak}
%    The \texttt{margin break} style is nearly the same as the
%    \texttt{change break} style. The only difference is the placement
%    of the theorem number.  We use |\llap| to place it in the left
%    margin.
%
%    In this style |\labelsep| denotes the separation between the
%    number and the text.
%    \begin{macrocode}
\gdef\th@marginbreak{\normalfont\slshape
  \def\@begintheorem##1##2{\item
       [\rlap{\vbox{\theorem@headerfont
                    \hbox{\llap{##2}\hskip\labelsep ##1}%
                    \hbox{\strut}}}]}%
\def\@opargbegintheorem##1##2##3{%
  \item[\rlap{\vbox{\theorem@headerfont
                    \hbox{\llap{##2}\hskip\labelsep ##1\ (##3)}%
                    \hbox{\strut}}}]}}
\endgroup
%</thmb>
%    \end{macrocode}
% \changes{v2.1c}{90/02/17}{`endgroup outside macrocode env.}
% \end{macro}
%
%
%
% \subsubsection{The \texttt{margin} style}
%
% This style option is stored in the file |thm.sty|.
%    \begin{macrocode}
%<*thm>
\begingroup \makeatletter
\@ifundefined{theorem@style}{% \iffalse meta-comment
%
% Copyright (C) 1993-2024
%
% The LaTeX Project and any individual authors listed elsewhere
% in this file.
%
% This file is part of the Standard LaTeX `Tools Bundle'.
% -------------------------------------------------------
%
% It may be distributed and/or modified under the
% conditions of the LaTeX Project Public License, either version 1.3c
% of this license or (at your option) any later version.
% The latest version of this license is in
%    https://www.latex-project.org/lppl.txt
% and version 1.3c or later is part of all distributions of LaTeX
% version 2005/12/01 or later.
%
% The list of all files belonging to the LaTeX `Tools Bundle' is
% given in the file `manifest.txt'.
%
% \fi
%
%%
%
%
\def\FMithmInfo{2023/07/05 v2.2c Theorem extension package (FMi)}
%
% \ProvidesFile{theorem.dtx}[\FMithmInfo]
%\iffalse   % this is a METACOMMENT !
%%
%% Package `theorem' to use with LaTeX2e
%% Copyright (C) 1989-1995 Frank Mittelbach, all rights reserved.
%<+package>\@ifundefined{theorem@style}{}{\endinput}
%<+package>\NeedsTeXFormat{LaTeX2e}[1995/06/01]
%<+package>\ProvidesPackage{theorem}
%<+package>         [\FMithmInfo]
%
% \fi
%
% \changes{v1.0}{87/05/07}{First draft.}
% \changes{v1.0a}{87/05/07}{`thistheoremstyle' added.}
% \changes{v1.1a}{87/07/29}{New concept: redefine the internal
%                          \protect\LaTeX{} commands `@xnthm' etc.}
% \changes{v1.1b}{87/07/31}{`theoremfont' used to specify font.}
% \changes{v1.2a}{88/01/08}{`@endtheorem' now globally defined.}
% \changes{v1.2b}{88/01/11}{`newtheorem' only allowed in the preamble.}
% \changes{v1.3a}{88/02/14}{`theoremfont' now a toks register.}
%
% \changes{v2.0a}{88/11/16}{New implementation and new documentation.}
% \changes{v2.0a}{88/11/16}{No layout changes but internal modifications
%                           old substyles won't work.}
% \changes{v2.0b}{89/01/30}{`global' used for all definitions.}
% \changes{v2.0c}{89/06/15}{Documentation translated to English.
%                            Thanks to LLL and Nina}
% \changes{v2.1a}{90/02/15}{Support for new font selection.}
% \changes{v2.2a}{94/02/03}{Cleanup for LaTeX2e}
%
% \DoNotIndex{\@endparenv,\@ifdefinable,\@ifnextchar,\@ifundefined}
% \DoNotIndex{\@namedef,\@nameuse,\@spaces,\@tempa,\@warning}
% \DoNotIndex{\ }
% \DoNotIndex{\begingroup,\bfseries}
% \DoNotIndex{\csname}
% \DoNotIndex{\def,\do,\docdate}
% \DoNotIndex{\endcsname,\endgroup,\endinput,\endtrivlist,\expandafter}
% \DoNotIndex{\filedate,\fileversion}
% \DoNotIndex{\gdef,\global}
% \DoNotIndex{\hbox,\hskip}
% \DoNotIndex{\ignorespaces,\input,\itshape,\item}
% \DoNotIndex{\labelsep,\let,\llap}
% \DoNotIndex{\makeatletter}
% \DoNotIndex{\noexpand}
% \DoNotIndex{\refstepcounter,\relax,\rlap}
% \DoNotIndex{\setlength,\slshape,\space,\string,\strut}
% \DoNotIndex{\the,\trivlist,\typeout}
% \DoNotIndex{\vbox}
% \DoNotIndex{\xdef}
%
% \def\deflabel#1{\ttfamily #1\hfill}
% \def\deflist#1{\begin{list}{}{\settowidth\labelwidth{\ttfamily #1}%
%                               \setlength\leftmargin\labelwidth
%                               \addtolength\leftmargin\labelsep
%                               \let\makelabel\deflabel}}
% \def\enddeflist{\end{list}}
% \def\envfont{\normalfont\ttfamily}
%
% \MakeShortVerb{\|}
%
% \newcommand{\TUB}{{\sl TUGboat\/}}
%
% \GetFileInfo{theorem.dtx}
% \title{An Extension of the
%        \LaTeX{} theorem environment\thanks{This file
%        has version number \fileversion, last
%        revised \filedate.}}
% \author{Frank Mittelbach}
%
% \MaintainedByLaTeXTeam{tools}
% \maketitle
%
% \begin{abstract}
%    The macros described in this paper yield an extension of the
%    \LaTeX{} theorem mechanism. It is designed to satisfy the
%    different requirements of various journals. Thus, the
%    layout of the ``theorems'' can be manipulated by determining a
%    ``style''. This article describes not only the use, but
%    also the definition, of the necessary macros.
% \end{abstract}
%
% \section*{Preface to version 2.2}
%
% For \LaTeXe{} this package did not need any fundamental changes. I
% only modified the messages generated so that theorem layout styles
% will show up with the |\listfiles| command and cleaned the section
% on the New Font Selection Scheme since this is now included in
% \LaTeX.
%
% \section*{Preface to version 2.1}
%
% This version is identical to 2.0g described in \TUB\ 10\#3 except
% for some internal defaults which are now set depending on the used
% font selection scheme.
%
% This was done to avoid unpleasant surprises if the new font selection
% scheme is in force. For further details see
% section~\ref{sec:newfonts} and \cite{bk:GMS94}.
%
% \section{Introduction}
%
% For our purposes here, ``theorems'' are labelled enunciations,
% often set off from the main text by extra space and a font change.
% Theorems, corollaries, conjectures, definitions, and remarks
% are all instances of ``theorems''.  The ``header'' of these structures
% is composed of a label (such as \textsc{Theorem} or \textsc{Remark})
% and a number which serializes an item in the sequence of items
% with the same label.
%
% Shortly after the introduction of \LaTeX{} at the Fachbereich
% Mathematik in Mainz, the desire to manipulate the layout of
% ``theorems'' arose.  In Mainz, the following two conventions came
% into general use:
% \begin{enumerate}
%   \item The number of the theorem is shown in the margin.
%   \item There is a line break at
%         the end of the theorem header.
% \end{enumerate}
%
% Additionally, some journals require different formats which depend on
% the ``sort of theorem'': e.g.\ often remarks and definitions are set
% in |\upshape|, while |\itshape| is employed for main theorems.
%
% Confronted with these requirements, a theorem environment was
% developed in Mainz which allows separate determination of the
% layout of the ``theorems sets'', comparable to |\pagestyle|.
%
%
% \section{The user interface}
% \subsection{Defining new theorem sets}
%
% \DescribeMacro\newtheorem
% As in the original \LaTeX{} version, the command |\newtheorem|
% defines a new ``theorem set'' or ``theorem-like structure''.
% Two required arguments name the new environment and give the text
% to be typeset with each instance of the new ``set'', while
% an optional argument determines how
% the ``set'' is enumerated:
% \begin{description}
%    \item[\ttfamily \bslash newtheorem\{foo\}\{bar\}]
%       The theorem set {\envfont foo} (whose name is \texttt{bar})
%       uses its own counter.
%    \item[\ttfamily \bslash newtheorem\{foo2\}{[foo]}\{bar2\}]
%       The theorem set {\envfont foo2} (printed name \texttt{bar2})
%       uses the same counter as the theorem set \texttt{foo}.
%    \item[\ttfamily \bslash newtheorem\{foo3\}\{bar3\}{[section]}]
%       The theorem set {\envfont foo3} (printed name \texttt{bar3}) is
%       enumerated within the counter \texttt{section}, i.e.\ with every
%       new |\section| the enumeration begins again with 1, and
%       the enumeration is composed from the section-number and the
%       theorem counter itself.
% \end{description}
%
% \DescribeMacro\theoremstyle
% Additionally, the command |\theoremstyle| can define the layout
% of various, or all, theorem sets.  It should be noted that any
% theorem set defined by |\newtheorem| is typeset in the
% |\theoremstyle| that is current at the time of the definition.
% Thus, the following
% \begin{verbatim}
%    \theoremstyle{break}        \newtheorem{Cor}{Corollary}
%    \theoremstyle{plain}        \newtheorem{Exa}{Example}[section]
%\end{verbatim}                             ^^A No blank at line start
% leads to the result that the set {\envfont Cor} is formatted in the
% style \texttt{break}, while the set {\envfont Exa} and all the
% following ones are formatted in the style \texttt{plain}, unless
% another |\theoremstyle| follows.  Since the definitions installed by
% |\newtheorem| are global, one also can limit |\theoremstyle| locally
% by grouping braces.
%
% \DescribeMacro\theorembodyfont
% The choice of the font for the theorem body is completely
% independent of the chosen |\theoremstyle|; this has proven to
% be very advantageous.  For example,
% \begin{verbatim}
%   {\theorembodyfont{\upshape}          \newtheorem{Rem}{Remark}}
%\end{verbatim}
% defines a theorem set {\envfont Rem}, which will be set in
% |\upshape| in the current layout (which in our example is
% \texttt{plain}).  As with |\theoremstyle|, the |\theorembodyfont|
% chosen is that current at the time of |\newtheorem|.  If
% |\theorembodyfont| is not specified or one defines
% |\theorembodyfont{}|, then the font used will be that defined by the
% |\theoremstyle|.
%
% \DescribeMacro\theoremheaderfont It is also possible to customize
% the font used for the theorem headers.  This is, however, a global
% declaration, and therefore there should be at most one
% |\theoremheaderfont| declaration in the preamble.\footnote{If it is
% actually necessary to have different header fonts, one has to define
% new theorem styles (substituting the desired font) or specify the
% information directly in the \texttt{\bslash newtheorem} declaration
% (the unclean variant).}
%
% \DescribeMacro\theorempreskipamount
% \DescribeMacro\theorempostskipamount Two additional parameters
% affect the vertical space around the theorem environments:
% |\theorempreskipamount| and |\theorempostskipamount| define,
% respectively, the spacing before and after such an environment.
% These parameters apply for all theorem sets and can be manipulated
% with the ordinary length macros.  They are rubber lengths
% (`\textsf{skips}'), and therefore can contain \texttt{plus} and
% \texttt{minus} parts.
%
% Since the definition of theorem sets should---most sensibly---be
% placed in the preamble, we only allow installation there.
% It is therefore possible to release the memory used here after
% |\begin{document}|, in order to make room for other
% applications.
%
%
% \subsection{Existing theorem styles}
%
% The following theorem styles exist to date:
% \begin{deflist}{marginbreak}
%    \item[plain]
%       This theorem style emulates the original \LaTeX{} definition,
%       except that additionally the parameters
%       |\theorem...skipamount| are used.
%    \item[break]
%       In this style, the theorem header is followed by a line
%       break.
%    \item[marginbreak]
%       The theorem number is set in the margin, and there is a line
%       break as in \texttt{break}.
%    \item[changebreak]
%       Like \texttt{break}, but with header number and text
%       interchanged.
%    \item[change]
%       Header number and text are interchanged, without a line break.
%    \item[margin]
%       The number is set in the left margin, without a line break.
% \end{deflist}
% All styles (except \texttt{plain}) select |\slshape| as the default
% |\theorembodyfont|.
%
%
% \subsection{Examples}
%
% Given the above theorem sets {\envfont Cor}, {\envfont Exa}
% and {\envfont Rem},
% suppose that the preamble also contains the declarations:
% \begin{verbatim}
%    \theoremstyle{marginbreak}   \newtheorem{Lem}[Cor]{Lemma}
%    \theoremstyle{change}
%    \theorembodyfont{\itshape}        \newtheorem{Def}[Cor]{Definition}
%
%    \theoremheaderfont{\scshape}
%\end{verbatim}
% Then the following are some typical examples of the typeset output
% resulting from their use.
%
% \begin{Cor}
%    This is a sentence typeset in the theorem environment {\envfont
%    Cor}.
% \end{Cor}
%
% \begin{Exa}
%    This is a sentence typeset in the theorem environment {\envfont
%    Exa}.
% \end{Exa}
%
% \begin{Rem}
%    This is a sentence typeset in the theorem environment {\envfont
%    Rem}.
% \end{Rem}
%
% \begin{Lem}[Ben User]
%    This is a sentence typeset in the theorem environment {\envfont
%    Lem}.
% \end{Lem}
%
% \begin{Def}[Very impressive Definition]
%    This is a sentence typeset in the theorem environment {\envfont
%    Def}.
% \end{Def}
%
% The last two examples show the effect of the optional argument to
% a theorem environment (it is the text typeset in parentheses).
%
%
% \section{Special Considerations}
% \label{sec:newfonts}
%
% Theoremheader and body are implemented as a unit. This means that
% the |\theoremheaderfont|  will inherit characteristics of the
% |\theorembodyfont| in \LaTeXe. Thus, if for example
% |\theorembodyfont| is |\itshape| and |\theoremheaderfont|
% is |\bfseries| the font selected for the header will have the
% characteristics `bold extended italic'. If this is not desired one
% should set the |\theoremheaderfont| to something like
% \begin{verbatim}
%      \theoremheaderfont{\normalfont\bfseries}
%\end{verbatim}
% i.e.\ supplying all necessary font information explicitly.
%
%
%
% \section{Acknowledgements}
%
% The publication of this set of macros was only possible with the
% help of Christina Busse (translating the manuscript into English),
% Joachim Pense (playing the r\^ole of typist), Chris Rowley
% (looking everything over) and many others providing useful
% suggestions.
%
% \MaybeStop{%
%    \begin{thebibliography}{9}
%       \bibitem{bk:GMS94} \textsc{M.~Goossens}, \textsc{F.~Mittelbach}
%         and \textsc{A.~Samarin}.
%         \newblock The \LaTeX{} Companion.
%         \newblock
%         Addison-Wesley, Reading, Massachusetts, 1994.
%       \bibitem{latex:source} \textsc{Lamport, Leslie.} \newblock
%                               \texttt{latex.tex}, version 2.09, date
%                               Feb. 1990.
%    \end{thebibliography}
%    \PrintIndex
%    \PrintChanges}
%
%
% \section{The documentation driver file}
%
% The next bit of code contains the documentation driver file for
% \TeX{}, i.e., the file that will produce the documentation you are
% currently reading. It will be extracted from this file by the
% \texttt{docstrip} program. Since it is the first code in the file
% one can alternatively process this file directly with \LaTeXe{} to
% obtain the documentation.
%
%    \begin{macrocode}
%<*driver>
\documentclass{ltxdoc}

\usepackage{theorem}

%<+driver>% The next few lines define theorem sets which are used
%<+driver>% in the example section of the documentation.

\theoremstyle{break}       \newtheorem{Cor}{Corollary}
\theoremstyle{plain}       \newtheorem{Exa}{Example}[section]
{\theorembodyfont{\upshape}\newtheorem{Rem}{Remark}}
\theoremstyle{marginbreak} \newtheorem{Lem}[Cor]{Lemma}
\theoremstyle{change}
\theorembodyfont{\itshape} \newtheorem{Def}[Cor]{Definition}

\theoremheaderfont{\scshape}

\RecordChanges

\begin{document}
   \DocInput{theorem.dtx}
\end{document}
%</driver>
%    \end{macrocode}
%
% \section{Definition of the Macros}
%
% If the file has been loaded before, we abort immediately. If not the
% package announces itself (this is actually done at the very top of
% the file---the way it is done isn't good style so don't copy it).
% \changes{v2.0e}{89/07/19}{Spaces removed from `typeout'.}
% \changes{v2.2b}{95/11/19}{Announce moved to top of file}
%    \begin{macrocode}
%<*package>
%\@ifundefined{theorem@style}{}{\endinput}
%\def\FMithmInfo{1995/11/19 v2.2b Theorem extension package (FMi)}
%\ProvidesPackage{theorem}[\FMithmInfo]
%    \end{macrocode}
%
%
% \subsection{Definition of theorem styles and fonts}
%
% All the definitions in this file are done globally to allow
% inputting this file inside a group.
%
% \begin{macro}{\theoremstyle}
%    Before a theorem style can be installed, the chosen style must be
%    known. For that reason, we must test to see that
%    |\th@|\meta{style} is known or, more precisely, that it is
%    different from |\relax|. If the style is not known then
%    |\th@plain| is used.
%    \begin{macrocode}
\gdef\theoremstyle#1{%
   \@ifundefined{th@#1}{\@warning
          {Unknown theoremstyle `#1'. Using `plain'}%
          \theorem@style{plain}}%
%    \end{macrocode}
%    We save the theorem style to be used in the token register
%    |\theorem@style|.
%    \begin{macrocode}
      {\theorem@style{#1}}%
%    \end{macrocode}
%    Now we ``evaluate'' the theorem style: this means, we call the
%    macro |\th@|\meta{style} which will activate the relevant
%    definitions which are contained in a separate file.  This is done
%    in a group to suppress changes to the current font.  This could
%    otherwise pose problems together with the new font selection
%    scheme\footnote{When I printed the original article using the new
%    font selection scheme I ended with a document with slanted
%    typefaces (text headings and all) simply because one of the
%    theorem styles used \texttt{\bslash sl} at toplevel.} if the
%    |\th@|\meta{style} is evaluated a second time.
% \changes{v2.1a}{90/02/15}{Group added around `th@..}
% \changes{v2.1a}{90/02/15}{Reseting of `..@begintheorem removed.}
%    \begin{macrocode}
      \begingroup
        \csname th@\the\theorem@style \endcsname
      \endgroup}
%    \end{macrocode}
% \end{macro}
%
%
% \begin{macro}{\@begintheorem}
% \changes{v2.1a}{90/02/15}{Relaxed at toplevel.}
% \begin{macro}{\@opargbegintheorem}
% \changes{v2.1a}{90/02/15}{Relaxed at toplevel.}
%    We reset |\@begintheorem| and |\@opargbegintheorem| to
%    |\relax| since these commands are no longer necessary at
%    toplevel. This will save a few tokens.
%    \begin{macrocode}
\global\let\@begintheorem\relax
\global\let\@opargbegintheorem\relax
%    \end{macrocode}
% \end{macro}
% \end{macro}
%
% \begin{macro}{\theorem@style}
%    Obviously the \textsf{token} register used above has to be
%    allocated. To assure the utmost compatibility with the
%    original \LaTeX{} definition, we set the default theorem style to
%    \texttt{plain}, which implements the usual \LaTeX{} convention.
% \changes{v2.0f}{89/07/20}{Unnecessary `global' removed.}
%    \begin{macrocode}
\newtoks\theorem@style
\global\theorem@style{plain}
%    \end{macrocode}
% \end{macro}
%
% \begin{macro}{\theorembodyfont}
% \begin{macro}{\theorem@bodyfont}
%    If the |\theorembodyfont| is set by the user then it should not
%    interact with the default font set in the theorem style. When the
%    new font selection is in force this may happen if, for example,
%    the default is |\itshape| and the new |\theorembodyfont| is
%    |\sffamily|. So we add a |\reset@font| command in front of the
%    user definition.
% \changes{v2.0f}{89/07/20}{Unnecessary `global' removed.}
% \changes{v2.0g}{89/09/19}{`theoremfont to `theorembodyfont.}
% \changes{v2.1c}{92/06/22}{`reset@font added and logic changed.}
% \changes{v2.1d}{92/10/28}{Reset to empty if arg is empty}
%    \begin{macrocode}
\gdef\theorembodyfont#1{%
%    \end{macrocode}
%    We check if the argument supplied is empty and if so put nothing
%    into the |\theorem@bodyfont| token register to allow for
%    |\theorembodyfont{}| as a mean of using the default of the
%    current |\theoremstyle|.
%    \begin{macrocode}
   \def\@tempa{#1}%
   \ifx\@tempa\@empty
    \theorem@bodyfont{}%
   \else
    \theorem@bodyfont{\reset@font#1}%
   \fi
   }
\newtoks\theorem@bodyfont
\global\theorem@bodyfont{}
%    \end{macrocode}
% \end{macro}
% \end{macro}
%
%
% \begin{macro}{\theoremheaderfont}
% \changes{v2.0g}{89/09/19}{Macro added.}
%    The font for the theorem headers is handled differently because
%    this definition applies to all theorem styles.
%    \begin{macrocode}
\gdef\theoremheaderfont#1{\gdef\theorem@headerfont{#1}%
%    \end{macrocode}
%    After using the macro once it is redefined to produce an error
%    message.
%    \begin{macrocode}
       \gdef\theoremheaderfont##1{%
        \typeout{\string\theoremheaderfont\space should be used
                 only once.}}}
%    \end{macrocode}
% \end{macro}
%
% \begin{macro}{\theorem@headerfont}
% \changes{v2.0g}{89/09/19}{Macro added.}
%    To set the |\theorem@headerfont| default we first test if
%    the new fontselection scheme is in force.
% \changes{v2.1a}{90/02/15}{Support for new font selection.}
%    \begin{macrocode}
\ifx\upshape\undefined
%    \end{macrocode}
%    If not we define it to expand into |\bfseries|. We don't use
%    |\let| just in case a following style option redefines this
%    macro.
%    \begin{macrocode}
\gdef\theorem@headerfont{\bfseries}
%    \end{macrocode}
%    Otherwise we reset the current shape before calling |\bfseries|.
%    \begin{macrocode}
\else \gdef\theorem@headerfont{\normalfont\bfseries}\fi
%    \end{macrocode}
% \end{macro}
%
% \begin{macro}{\th@plain}
% \begin{macro}{\th@break}
% \begin{macro}{\th@marginbreak}
% \begin{macro}{\th@changebreak}
% \begin{macro}{\th@change}
% \begin{macro}{\th@margin}
%    The different styles are defined in macros such as
%    |\th@plain|.  Since memory space is precious in
%    ``non-Big-versions'', we have to avoid offering too many unused
%    definitions.  Therefore we define these styles in separate files
%    that can be loaded on demand.  Thus the commands themselves only
%    load these files. We use |\@input@|, a \LaTeXe{} internal command
%    that ensures that the file will be listed with |\listfiles|.
% \changes{v2.2a}{94/02/02}{Use `@input@ to load theorem layout files}
%    \begin{macrocode}
\gdef\th@plain{\@input@{thp.sty}}
\gdef\th@break{\@input@{thb.sty}}
\gdef\th@marginbreak{\@input@{thmb.sty}}
\gdef\th@changebreak{\@input@{thcb.sty}}
\gdef\th@change{\@input@{thc.sty}}
\gdef\th@margin{\@input@{thm.sty}}
%    \end{macrocode}
%    This list will be expanded when new styles become
%    available.  For testing,  just append new theorem substyles
%    as document options.
% \end{macro}
% \end{macro}
% \end{macro}
% \end{macro}
% \end{macro}
% \end{macro}
%
%
%
% \subsection{Definition of a new theorem set}
%
% As already pointed out, a new theorem environment can be defined in
% three different ways:
% \begin{verbatim}
%    \newtheorem{Lem}{Lemma}
%    \newtheorem{Lem}{Lemma}[section]
%    \newtheorem{Lem}[Theorem]{Lemma}
%\end{verbatim}
% The function of the macro |\newtheorem| is to recognize these cases
% and then to branch into one of the three macros |\@ynthm|, |\@xnthm|
% or |\@othm|.  This mechanism is adopted unchanged from
% \cite{latex:source}; the essential point here is that, for example,
% in the second case, the arguments \texttt{Lem}, \texttt{Lemma} and
% \texttt{section} are passed over to the macro |\@xnthm|.
%
% We inspect this case first because the others present fewer problems,
% and thus are easily derived from this one.
%
% \begin{macro}{\@xnthm}
%    For our example arguments, the macro |\@xnthm| must fulfill
%    the following:
%    \begin{itemize}
%       \item
%          Define a new \LaTeX-counter `Lem'
%       \item
%          reset this counter within a |\section|
%       \item
%          define the macro |\theLem|
%       \item
%          define the environment macros |\Lem| and
%          |\endLem| using the current
%          |\theoremstyle| and |\theorem@bodyfont|.
%    \end{itemize}
%    Obviously, all this should happen only if the first argument
%    of |\@xnthm| (i.e.\ \texttt{Lem} in our example) is chosen so
%    as not to conflict with any previously defined commands or
%    environments.  This test is performed by the \LaTeX{} macro
%    |\@ifdefinable|.
%    \begin{macrocode}
\gdef\@xnthm#1#2[#3]{\expandafter\@ifdefinable\csname #1\endcsname
%    \end{macrocode}
%    Therefore, the first argument of |\@ifdefinable| is the
%    expansion (in the example, |\Lem|) of
%    |\csname||#1\endcsname|.  The second argument is
%    executed only if the test has been completed successfully.
%    \begin{macrocode}
   {%
%    \end{macrocode}
%    Now we define the new counter. The names of the \LaTeX{}
%    macros employed should speak for themselves:
% \changes{v2.2b}{1995/10/07}{Use \cs{@newctr}}
% \changes{v2.2c}{1995/11/23}{Use \cs{@newctr} properly.}
%    \begin{macrocode}
    \@definecounter{#1}\@newctr{#1}[#3]%
%    \end{macrocode}
%    Using |\@newctr| will give a proper error message if the counter
%    in |#3| is not defined.
%    In defining `|\theLem|' we must generate the desired macro
%    name by use of |\expandafter| and |\csname|.
%    \begin{macrocode}
    \expandafter\xdef\csname the#1\endcsname
%    \end{macrocode}
%    An |\xdef| is used in order to make the definition global,
%    and to ensure that it contains the replacement texts of
%    |\@thmcountersep| and |\@thmcounter|.\footnote{These
%    two macros can be defined by the document style.  Their default
%    values produce a `\texttt{.}' as separation and an arabic
%    representation of the number.} However, not everything should be
%    expanded. For example, it saves space to use |\thesection|
%    instead of its---at times---lengthy expansion.
%    \begin{macrocode}
      {\expandafter \noexpand \csname the#3\endcsname
       \@thmcountersep \@thmcounter{#1}}%
%    \end{macrocode}
%    Thus with the defaults of \LaTeX{}, |\theLem| would be
%    replaced by the command sequence |\thesection.\arabic{Lem}|.
%
%    We will now look at the definition of the macro which is executed
%    at the beginning of the actual environment (in our example this
%    macro is |\Lem|). It should be noted that we use an
%    ``|\expandafter| trick'' to expand only certain parts of the
%    replacement text at the time of the definition.
%    \begin{macrocode}
    \def\@tempa{\global\@namedef{#1}}%
    \expandafter \@tempa \expandafter{%
%    \end{macrocode}
%    First, the macro that contains the current definitions of
%    |\@begintheorem| and |\@opargtheorem| should be called
%    up.  The name of this macro---as is already known---has the form
%    |\th@|\meta{theorem\ style}; therefore, it must be called by
%    \begin{macrocode}
      \csname th@\the \theorem@style
%    \end{macrocode}
%    In addition the default theorem font should be changeable, i.e.\
%    we have to insert the contents of |\theorem@bodyfont|. For
%    that reason, we expand even further, beyond |\endcsname|,
%    and thus insert the contents of the \textsf{token} register
%    |\theorem@bodyfont| in the replacement text.
%    \begin{macrocode}
            \expandafter \endcsname \the \theorem@bodyfont
%    \end{macrocode}
%    Now it is time to call the macro |\@thm| which takes over
%    the further processing.  It has two arguments: the current
%    counter name (in our example, \texttt{Lem}), and the text of the
%    label (in our example, \texttt{Lemma}).
%    \begin{macrocode}
     \@thm{#1}{#2}}%
%    \end{macrocode}
%    With this, the `sub-definition' is complete.  The macro
%    |\@endtheorem| ends a theorem environment and is, so far, nothing
%    but an |\endtrivlist|.  (Hence it is defined globally, and not
%    within the theorem styles.\footnote{This has to be changed as
%    soon as theorem styles that change \texttt{\bslash @endtheorem}
%    exist.  In such a case, all existing styles must be changed as
%    well since they will have to reset the macro.}) Therefore, we can
%    set it equivalent to the macro that ends the theorem set (in our
%    example, |\endLem|).  However, if some day theorem styles exist
%    that do change |\@endtheorem|, we would have to use the
%    commented-out line instead.
%    \begin{macrocode}
    \global \expandafter \let \csname end#1\endcsname \@endtheorem
%   \global\@namedef{end#1}{\@endtheorem}%
%    \end{macrocode}
%    With these commands all the required definitions are employed,
%    unless the test |\@ifdefinable| has failed.  Therefore, we
%    end the second argument of this macro and with it the definition
%    of |\@xnthm|.
%    \begin{macrocode}
   }}
%    \end{macrocode}
% \end{macro}
%
% \begin{macro}{\@ynthm}
%    The definition of |\@ynthm| is completely analogous. In
%    this case the new counter that is defined is not reset within
%    another counter; thus the definition of |\the...|
%    is simplified:
%    \begin{macrocode}
\gdef\@ynthm#1#2{\expandafter\@ifdefinable\csname #1\endcsname
   {\@definecounter{#1}%
    \expandafter\xdef\csname the#1\endcsname{\@thmcounter{#1}}%
%    \end{macrocode}
%    The rest of the definition corresponds literally to that of
%    |\@xnthm|:
%    \begin{macrocode}
    \def\@tempa{\global\@namedef{#1}}\expandafter \@tempa
     \expandafter{\csname th@\the \theorem@style \expandafter
     \endcsname \the\theorem@bodyfont \@thm{#1}{#2}}%
    \global \expandafter \let \csname end#1\endcsname \@endtheorem}}
%    \end{macrocode}
% \end{macro}
%
% \begin{macro}{\@othm}
%    The definition of |\@othm| does not contain anything new.
%    \begin{macrocode}
\gdef\@othm#1[#2]#3{%
%    \end{macrocode}
% \changes{v2.2b}{95/10/07}{Build in check for counter defined}
%    We do not define a new counter but instead use one that has
%    already been defined.  Thus the only definition we need is that
%    of this pseudo-counter (i.e.\ |\the|\meta{env.\ name}).
%    First we check if |#2| corresponds to a known counter name.
%    \begin{macrocode}
  \expandafter\ifx\csname c@#2\endcsname\relax
   \@nocounterr{#2}%
  \else
   \expandafter\@ifdefinable\csname #1\endcsname
   {\expandafter \xdef \csname the#1\endcsname
     {\expandafter \noexpand \csname the#2\endcsname}%
%    \end{macrocode}
%    All other parts of the definition can be adopted from
%    |\@xnthm|. We have to remember, though, that in this case
%    the name of the current counter and the theorem label have moved
%    to the second and third arguments.
%    \begin{macrocode}
    \def\@tempa{\global\@namedef{#1}}\expandafter \@tempa
     \expandafter{\csname th@\the \theorem@style \expandafter
     \endcsname \the\theorem@bodyfont \@thm{#2}{#3}}%
    \global \expandafter \let \csname end#1\endcsname \@endtheorem}%
  \fi}
%    \end{macrocode}
% \end{macro}
%
% \subsection{Macros that are employed in a {\envfont theorem}
%    environment}
%
% \begin{macro}{\@thm}
%    The macro |\@thm| has to increase the current counter.  Then,
%    depending on whether the environment has (or does not have) an
%    optional argument, it has to branch into either |\@begintheorem|
%    or |\@opargtheorem|.
%    \begin{macrocode}
\gdef\@thm#1#2{\refstepcounter{#1}%
%    \end{macrocode}
%    Now we start a {\envfont trivlist} environment, and give
%    |\@topsep| and |\@topsepadd| the values of the \textsf{skip}
%    registers |\theorempreskipamount| and |\theorempostskipamount|.
%    The value in |\@topsep| is the vertical space that is inserted by
%    the first (and only) |\item| in our |\trivlist| whilst
%    |\@topsepadd| is inserted by |\@endparenv| at the end of that
%    {\envfont trivlist} environment.  By using these registers, we
%    obtain the desired space around a {\envfont theorem} environment.
%    \begin{macrocode}
   \trivlist
   \@topsep \theorempreskipamount               % used by first \item
   \@topsepadd \theorempostskipamount           % used by \@endparenv
%    \end{macrocode}
%    Now we have to test whether an optional argument has been given.
% \changes{v2.0d}{89/07/18}{Forgotten percent added.}
%    \begin{macrocode}
   \@ifnextchar [%
%    \end{macrocode}
%    If there is an optional argument, we will call |\@ythm|, and
%    move the arguments read back into the input stream.
%    \begin{macrocode}
   {\@ythm{#1}{#2}}%
%    \end{macrocode}
%    If not, we call |\@begintheorem|. Its first argument is the
%    name of the theorem set (hence the second argument of
%    |\@thm|). Its second argument is the macro that produces the
%    current number.
%    \begin{macrocode}
   {\@begintheorem{#2}{\csname the#1\endcsname}\ignorespaces}}
%    \end{macrocode}
% \end{macro}
%
% \begin{macro}{\@xthm}
% \begin{macro}{\@ythm}
%    Both these macros were originally called by |\@thm|.  We do
%    not need |\@xthm| anymore, hence we reset it to
%    |\relax|. The definition of |\@ythm| has not changed at
%    all from its definition in \LaTeX. In order to make the macros
%    easier to understand, we will
%    nevertheless present it (commented out).
%    \begin{macrocode}
\global\let\@xthm\relax
% \def\@ythm#1#2[#3]{\@opargbegintheorem{#2}{\csname
%           the#1\endcsname}{#3}\ignorespaces}
%</package>
%    \end{macrocode}
%    The primitive |\ignorespaces| in |\@ythm| and
%    |\@thm| is needed to remove the spaces between the
%    |\begin{...}| and the actual text.
% \end{macro}
% \end{macro}
%
%
% \subsection{Definition of the theorem substyles}
%
% As already pointed out, the theorem substyles, defined below, are
% only loaded when necessary.  Note that all these substyles, except
% \texttt{plain}, have |\slshape| as the default body font.
%
% \subsubsection{The \texttt{plain} style}
%
% \label{dok:th@plain} As the following macros use |@|, we have
% to locally set the |\catcode| of this symbol to ``letter''.
% This happens within a group, so that we do not have to worry about
% which |\catcode| that symbol had before.
%    \begin{macrocode}
%<*thp>
\begingroup \makeatletter
%    \end{macrocode}
% Since we are now within a group, we must make all definitions
% globally.  First we make sure that \texttt{theorem.sty} is loaded.
% This will allow us to use this file as a document style option
% without having to call \texttt{theorem} itself as an option.  At the
% same time, we assure that at least version 2 is loaded, since
% |\theorem@style| was not defined in earlier versions.
%    \begin{macrocode}
\@ifundefined{theorem@style}{\input{theorem.sty}}{}
\ProvidesFile{thp.sty}
             [\FMithmInfo]
%    \end{macrocode}
%
%
% \begin{macro}{\th@plain}
% \begin{macro}{\@begintheorem}
% \begin{macro}{\@opargbegintheorem}
%    |\theoremstyle{plain}| corresponds to the original
%    definition, except that the distances to the surrounding text are
%    determined by the parameters |\theorempreskipamount| and
%    |\theorempostskipamount|.
%    First we set the default body font.
%    \begin{macrocode}
\gdef\th@plain{\normalfont\itshape
%    \end{macrocode}
%    Then we define |\@begintheorem| and
%    |\@opargbegintheorem|. These two macros define how the
%    header of a theorem is typeset.  |\@opargbegintheorem| will
%    be called if a \textsf{theorem} environment with an optional
%    argument is encountered; otherwise, the header is constructed by
%    calling |\@begintheorem|.  If one of these macros is
%    executed, we are within a \textsf{trivlist} environment started by
%    |\@thm|.  So the theorem header is produced with an
%    |\item| command.
%
%    Instead of specifying the header font directly, all standard
%    theorem styles use the |\theorem@headerfont| macro to allow
%    customization.  The extra space (|\labelsep|) is necessary
%    because of problems in the \textsf{trivlist} environment.
%    \begin{macrocode}
  \def\@begintheorem##1##2{%
        \item[\hskip\labelsep \theorem@headerfont ##1\ ##2]}%
%    \end{macrocode}
%    The definition of |\@opargbegintheorem| is completely
%    analogous. The only difference is the fact that there exists a
%    third argument (which is the optional parameter of the
%    environment and contains additional information about the
%    theorem).  Customarily we enclose it in parentheses.
%    \begin{macrocode}
\def\@opargbegintheorem##1##2##3{%
   \item[\hskip\labelsep \theorem@headerfont ##1\ ##2\ (##3)]}}
%    \end{macrocode}
% \end{macro}
% \end{macro}
% \end{macro}
%
% \noindent
% We conclude with an |\endgroup| to restore the
% |\catcode| of |@|.
%    \begin{macrocode}
\endgroup
%</thp>
%    \end{macrocode}
%
%
% \subsubsection{The \texttt{break} style}
%
% This style option is stored in the file |thb.sty|.
% For the next two lines see the documentation for |\th@plain| on
% page~\pageref{dok:th@plain}.
%    \begin{macrocode}
%<*thb>
\begingroup \makeatletter
\@ifundefined{theorem@style}{\input{theorem.sty}}{}
\ProvidesFile{thb.sty}
             [\FMithmInfo]
%    \end{macrocode}
%
% \begin{macro}{\th@break}
%    |\theoremstyle{break}| produces a line break after the name
%    of the theorem.  The font is |\slshape|.  Hence, we define
%    |\th@break| as follows:
%    \begin{macrocode}
\gdef\th@break{\normalfont\slshape
  \def\@begintheorem##1##2{\item[%
%    \end{macrocode}
%    We run into the following problem: it is not possible to create
%    the header with |\item[|\meta{title}|]| and then start
%    a new line by, for example, |\mbox{}\\|. Such a definition
%    will fail whenever a list environment follows immediately.  With
%    the above construction, the |\mbox{}| causes the switch
%    |@inlabel| (cf.\ definition of |\list| and
%    |\trivlist| in \cite{latex:source}) to be set to \texttt{false}
%    and so the following list will insert additional vertical
%    space (|\topskip|). This is quite annoying. Therefore, we
%    create the line break within the |\item|. In order to ensure
%    that the text will begin at the proper position in the following
%    line, we simply pretend that the label does not take any
%    room.\footnote{This will lead to problems whenever very high
%    symbols occurring in the line tower into the
%    heading. So, something else has to be done here sometime.}
%    \begin{macrocode}
       \rlap{\vbox{\hbox{\hskip \labelsep\theorem@headerfont ##1\ ##2}%
                   \hbox{\strut}}}]}%
%    \end{macrocode}
%    Again, the definition of |\@opargbegintheorem| is
%    completely analogous.
%    \begin{macrocode}
\def\@opargbegintheorem##1##2##3{%
  \item[\rlap{\vbox{\hbox{\hskip \labelsep \theorem@headerfont
                     ##1\ ##2\ (##3)}%
                    \hbox{\strut}}}]}}
\endgroup
%</thb>
%    \end{macrocode}
% \end{macro}
%
%
% \subsubsection{The \texttt{changebreak} style}
%
%    \begin{macrocode}
% This style option is stored in the file |thcb.sty|.
%    \begin{macrocode}
%<*thcb>
\begingroup \makeatletter
\@ifundefined{theorem@style}{\input{theorem.sty}}{}
\ProvidesFile{thcb.sty}
             [\FMithmInfo]
%    \end{macrocode}
%
% \begin{macro}{\th@changebreak}
%    The \texttt{change-break} theorem style is like \texttt{break}
%    but with interchange of theorem name and theorem number.  Thus we
%    define |\th@changebreak| as follows:
%    \begin{macrocode}
\gdef\th@changebreak{\normalfont\slshape
  \def\@begintheorem##1##2{\item
       [\rlap{\vbox{\hbox{\hskip\labelsep \theorem@headerfont ##2\ ##1}%
                    \hbox{\strut}}}]}%
\def\@opargbegintheorem##1##2##3{%
  \item[\rlap{\vbox{\hbox{\hskip\labelsep \theorem@headerfont
                     ##2\ ##1\ (##3)}%
                    \hbox{\strut}}}]}}
\endgroup
%</thcb>
%    \end{macrocode}
% \end{macro}
%
%
% \subsubsection{The \texttt{change} style}
%
% This style option is stored in the file |thc.sty|.
%    \begin{macrocode}
%<*thc>
\begingroup \makeatletter
\@ifundefined{theorem@style}{\input{theorem.sty}}{}
\ProvidesFile{thc.sty}
             [\FMithmInfo]
%    \end{macrocode}
%
% \begin{macro}{\th@change}
%    The \texttt{change} theorem style corresponds to the
%    \texttt{change break} style without a linebreak after the header.
%    To say it in another way, it's the same as the \texttt{plain}
%    style but with number and name interchanged and |\slshape| as the
%    default font.
%    \begin{macrocode}
\gdef\th@change{\normalfont\slshape
  \def\@begintheorem##1##2{\item
       [\hskip\labelsep \theorem@headerfont ##2\ ##1]}%
\def\@opargbegintheorem##1##2##3{%
  \item[\hskip\labelsep \theorem@headerfont ##2\ ##1\ (##3)]}}
\endgroup
%</thc>
%    \end{macrocode}
% \end{macro}
%
%
% \subsubsection{The \texttt{marginbreak} style}
%
% This style option is the one used most often at Mainz.  It is saved
% in the file |thmb.sty|.
%    \begin{macrocode}
%<*thmb>
\begingroup \makeatletter
\@ifundefined{theorem@style}{\input{theorem.sty}}{}
\ProvidesFile{thmb.sty}
             [\FMithmInfo]
%    \end{macrocode}
%
% \begin{macro}{\th@marginbreak}
%    The \texttt{margin break} style is nearly the same as the
%    \texttt{change break} style. The only difference is the placement
%    of the theorem number.  We use |\llap| to place it in the left
%    margin.
%
%    In this style |\labelsep| denotes the separation between the
%    number and the text.
%    \begin{macrocode}
\gdef\th@marginbreak{\normalfont\slshape
  \def\@begintheorem##1##2{\item
       [\rlap{\vbox{\theorem@headerfont
                    \hbox{\llap{##2}\hskip\labelsep ##1}%
                    \hbox{\strut}}}]}%
\def\@opargbegintheorem##1##2##3{%
  \item[\rlap{\vbox{\theorem@headerfont
                    \hbox{\llap{##2}\hskip\labelsep ##1\ (##3)}%
                    \hbox{\strut}}}]}}
\endgroup
%</thmb>
%    \end{macrocode}
% \changes{v2.1c}{90/02/17}{`endgroup outside macrocode env.}
% \end{macro}
%
%
%
% \subsubsection{The \texttt{margin} style}
%
% This style option is stored in the file |thm.sty|.
%    \begin{macrocode}
%<*thm>
\begingroup \makeatletter
\@ifundefined{theorem@style}{\input{theorem.sty}}{}
\ProvidesFile{thm.sty}
             [\FMithmInfo]
%    \end{macrocode}
%
% \begin{macro}{\th@margin}
%    Again this is only a variant of the theorem styles described
%    above without any new ideas.
%    \begin{macrocode}
\gdef\th@margin{\normalfont\slshape
  \def\@begintheorem##1##2{\item
       [\theorem@headerfont \llap{##2}\hskip\labelsep ##1]}%
\def\@opargbegintheorem##1##2##3{%
  \item[\theorem@headerfont \llap{##2}\hskip\labelsep ##1\ (##3)]}}
\endgroup
%</thm>
%    \end{macrocode}
% \end{macro}
%
%
%
%
% \subsection{Final Definitions}
%
% \begin{macro}{\theorempreskipamount}
% \begin{macro}{\theorempostskipamount}
%    The \textsf{skip} parameters that regulate the vertical empty space
%    before and after the {\envfont theorem} environment have
%    to be allocated as well.
% \changes{v2.0f}{89/07/20}{Unnecessary `global' removed.}
%    \begin{macrocode}
%<*package>
\newskip\theorempreskipamount
\newskip\theorempostskipamount
%    \end{macrocode}
%    Since we have used the same values for all theorem sets, we now can
%    assign them.
%    \begin{macrocode}
\global\setlength\theorempreskipamount{12pt plus 5pt minus 3pt}
\global\setlength\theorempostskipamount{8pt plus 3pt minus 1.5pt}
%    \end{macrocode}
% \end{macro}
% \end{macro}
%
%
% \begin{macro}{\@endtheorem}
%    The same holds for the macro |\@endtheorem|, which ends a
%    {\envfont theorem} environment. Since it is the same for all
%    theorem sets, it is removed from the macros |\th@|\meta{style}.
%    It simply ends the {\envfont trivlist} environment, which was
%    begun in |\@thm|.
%    \begin{macrocode}
\global\let\@endtheorem=\endtrivlist
%    \end{macrocode}
% \end{macro}
%
%
%
% \begin{macro}{\@preamblecmds}
%    All macros defined above are to be used only in the preamble.
%    Therefore, we insert them in |\@preamblecmds| which will disable
%    them at begin document. This is done by the internal \LaTeXe{}
%    command |\@onlypreamble|.
% \changes{v2.0g}{89/09/19}{`theoremheaderfont and `theorembodyfont
%    added.}
% \changes{v2.2a}{94/02/03}{Use \LaTeXe{} interface}
%    \begin{macrocode}
\@onlypreamble\@xnthm
\@onlypreamble\@ynthm
\@onlypreamble\@othm
\@onlypreamble\newtheorem
\@onlypreamble\theoremstyle
\@onlypreamble\theorembodyfont
\@onlypreamble\theoremheaderfont
%    \end{macrocode}
% \end{macro}
%
% Finally we declare the \texttt{plain} theorem style to be the default.
% \changes{v2.1b}{90/06/04}{Fixed missing `item by declaring default
%    style.}
%    \begin{macrocode}
\theoremstyle{plain}
%</package>
%    \end{macrocode}
%
% \Finale

\endinput
}{}
\ProvidesFile{thm.sty}
             [\FMithmInfo]
%    \end{macrocode}
%
% \begin{macro}{\th@margin}
%    Again this is only a variant of the theorem styles described
%    above without any new ideas.
%    \begin{macrocode}
\gdef\th@margin{\normalfont\slshape
  \def\@begintheorem##1##2{\item
       [\theorem@headerfont \llap{##2}\hskip\labelsep ##1]}%
\def\@opargbegintheorem##1##2##3{%
  \item[\theorem@headerfont \llap{##2}\hskip\labelsep ##1\ (##3)]}}
\endgroup
%</thm>
%    \end{macrocode}
% \end{macro}
%
%
%
%
% \subsection{Final Definitions}
%
% \begin{macro}{\theorempreskipamount}
% \begin{macro}{\theorempostskipamount}
%    The \textsf{skip} parameters that regulate the vertical empty space
%    before and after the {\envfont theorem} environment have
%    to be allocated as well.
% \changes{v2.0f}{89/07/20}{Unnecessary `global' removed.}
%    \begin{macrocode}
%<*package>
\newskip\theorempreskipamount
\newskip\theorempostskipamount
%    \end{macrocode}
%    Since we have used the same values for all theorem sets, we now can
%    assign them.
%    \begin{macrocode}
\global\setlength\theorempreskipamount{12pt plus 5pt minus 3pt}
\global\setlength\theorempostskipamount{8pt plus 3pt minus 1.5pt}
%    \end{macrocode}
% \end{macro}
% \end{macro}
%
%
% \begin{macro}{\@endtheorem}
%    The same holds for the macro |\@endtheorem|, which ends a
%    {\envfont theorem} environment. Since it is the same for all
%    theorem sets, it is removed from the macros |\th@|\meta{style}.
%    It simply ends the {\envfont trivlist} environment, which was
%    begun in |\@thm|.
%    \begin{macrocode}
\global\let\@endtheorem=\endtrivlist
%    \end{macrocode}
% \end{macro}
%
%
%
% \begin{macro}{\@preamblecmds}
%    All macros defined above are to be used only in the preamble.
%    Therefore, we insert them in |\@preamblecmds| which will disable
%    them at begin document. This is done by the internal \LaTeXe{}
%    command |\@onlypreamble|.
% \changes{v2.0g}{89/09/19}{`theoremheaderfont and `theorembodyfont
%    added.}
% \changes{v2.2a}{94/02/03}{Use \LaTeXe{} interface}
%    \begin{macrocode}
\@onlypreamble\@xnthm
\@onlypreamble\@ynthm
\@onlypreamble\@othm
\@onlypreamble\newtheorem
\@onlypreamble\theoremstyle
\@onlypreamble\theorembodyfont
\@onlypreamble\theoremheaderfont
%    \end{macrocode}
% \end{macro}
%
% Finally we declare the \texttt{plain} theorem style to be the default.
% \changes{v2.1b}{90/06/04}{Fixed missing `item by declaring default
%    style.}
%    \begin{macrocode}
\theoremstyle{plain}
%</package>
%    \end{macrocode}
%
% \Finale

\endinput
}{}
\ProvidesFile{thp.sty}
             [\FMithmInfo]
%    \end{macrocode}
%
%
% \begin{macro}{\th@plain}
% \begin{macro}{\@begintheorem}
% \begin{macro}{\@opargbegintheorem}
%    |\theoremstyle{plain}| corresponds to the original
%    definition, except that the distances to the surrounding text are
%    determined by the parameters |\theorempreskipamount| and
%    |\theorempostskipamount|.
%    First we set the default body font.
%    \begin{macrocode}
\gdef\th@plain{\normalfont\itshape
%    \end{macrocode}
%    Then we define |\@begintheorem| and
%    |\@opargbegintheorem|. These two macros define how the
%    header of a theorem is typeset.  |\@opargbegintheorem| will
%    be called if a \textsf{theorem} environment with an optional
%    argument is encountered; otherwise, the header is constructed by
%    calling |\@begintheorem|.  If one of these macros is
%    executed, we are within a \textsf{trivlist} environment started by
%    |\@thm|.  So the theorem header is produced with an
%    |\item| command.
%
%    Instead of specifying the header font directly, all standard
%    theorem styles use the |\theorem@headerfont| macro to allow
%    customization.  The extra space (|\labelsep|) is necessary
%    because of problems in the \textsf{trivlist} environment.
%    \begin{macrocode}
  \def\@begintheorem##1##2{%
        \item[\hskip\labelsep \theorem@headerfont ##1\ ##2]}%
%    \end{macrocode}
%    The definition of |\@opargbegintheorem| is completely
%    analogous. The only difference is the fact that there exists a
%    third argument (which is the optional parameter of the
%    environment and contains additional information about the
%    theorem).  Customarily we enclose it in parentheses.
%    \begin{macrocode}
\def\@opargbegintheorem##1##2##3{%
   \item[\hskip\labelsep \theorem@headerfont ##1\ ##2\ (##3)]}}
%    \end{macrocode}
% \end{macro}
% \end{macro}
% \end{macro}
%
% \noindent
% We conclude with an |\endgroup| to restore the
% |\catcode| of |@|.
%    \begin{macrocode}
\endgroup
%</thp>
%    \end{macrocode}
%
%
% \subsubsection{The \texttt{break} style}
%
% This style option is stored in the file |thb.sty|.
% For the next two lines see the documentation for |\th@plain| on
% page~\pageref{dok:th@plain}.
%    \begin{macrocode}
%<*thb>
\begingroup \makeatletter
\@ifundefined{theorem@style}{% \iffalse meta-comment
%
% Copyright (C) 1993-2024
%
% The LaTeX Project and any individual authors listed elsewhere
% in this file.
%
% This file is part of the Standard LaTeX `Tools Bundle'.
% -------------------------------------------------------
%
% It may be distributed and/or modified under the
% conditions of the LaTeX Project Public License, either version 1.3c
% of this license or (at your option) any later version.
% The latest version of this license is in
%    https://www.latex-project.org/lppl.txt
% and version 1.3c or later is part of all distributions of LaTeX
% version 2005/12/01 or later.
%
% The list of all files belonging to the LaTeX `Tools Bundle' is
% given in the file `manifest.txt'.
%
% \fi
%
%%
%
%
\def\FMithmInfo{2023/07/05 v2.2c Theorem extension package (FMi)}
%
% \ProvidesFile{theorem.dtx}[\FMithmInfo]
%\iffalse   % this is a METACOMMENT !
%%
%% Package `theorem' to use with LaTeX2e
%% Copyright (C) 1989-1995 Frank Mittelbach, all rights reserved.
%<+package>\@ifundefined{theorem@style}{}{\endinput}
%<+package>\NeedsTeXFormat{LaTeX2e}[1995/06/01]
%<+package>\ProvidesPackage{theorem}
%<+package>         [\FMithmInfo]
%
% \fi
%
% \changes{v1.0}{87/05/07}{First draft.}
% \changes{v1.0a}{87/05/07}{`thistheoremstyle' added.}
% \changes{v1.1a}{87/07/29}{New concept: redefine the internal
%                          \protect\LaTeX{} commands `@xnthm' etc.}
% \changes{v1.1b}{87/07/31}{`theoremfont' used to specify font.}
% \changes{v1.2a}{88/01/08}{`@endtheorem' now globally defined.}
% \changes{v1.2b}{88/01/11}{`newtheorem' only allowed in the preamble.}
% \changes{v1.3a}{88/02/14}{`theoremfont' now a toks register.}
%
% \changes{v2.0a}{88/11/16}{New implementation and new documentation.}
% \changes{v2.0a}{88/11/16}{No layout changes but internal modifications
%                           old substyles won't work.}
% \changes{v2.0b}{89/01/30}{`global' used for all definitions.}
% \changes{v2.0c}{89/06/15}{Documentation translated to English.
%                            Thanks to LLL and Nina}
% \changes{v2.1a}{90/02/15}{Support for new font selection.}
% \changes{v2.2a}{94/02/03}{Cleanup for LaTeX2e}
%
% \DoNotIndex{\@endparenv,\@ifdefinable,\@ifnextchar,\@ifundefined}
% \DoNotIndex{\@namedef,\@nameuse,\@spaces,\@tempa,\@warning}
% \DoNotIndex{\ }
% \DoNotIndex{\begingroup,\bfseries}
% \DoNotIndex{\csname}
% \DoNotIndex{\def,\do,\docdate}
% \DoNotIndex{\endcsname,\endgroup,\endinput,\endtrivlist,\expandafter}
% \DoNotIndex{\filedate,\fileversion}
% \DoNotIndex{\gdef,\global}
% \DoNotIndex{\hbox,\hskip}
% \DoNotIndex{\ignorespaces,\input,\itshape,\item}
% \DoNotIndex{\labelsep,\let,\llap}
% \DoNotIndex{\makeatletter}
% \DoNotIndex{\noexpand}
% \DoNotIndex{\refstepcounter,\relax,\rlap}
% \DoNotIndex{\setlength,\slshape,\space,\string,\strut}
% \DoNotIndex{\the,\trivlist,\typeout}
% \DoNotIndex{\vbox}
% \DoNotIndex{\xdef}
%
% \def\deflabel#1{\ttfamily #1\hfill}
% \def\deflist#1{\begin{list}{}{\settowidth\labelwidth{\ttfamily #1}%
%                               \setlength\leftmargin\labelwidth
%                               \addtolength\leftmargin\labelsep
%                               \let\makelabel\deflabel}}
% \def\enddeflist{\end{list}}
% \def\envfont{\normalfont\ttfamily}
%
% \MakeShortVerb{\|}
%
% \newcommand{\TUB}{{\sl TUGboat\/}}
%
% \GetFileInfo{theorem.dtx}
% \title{An Extension of the
%        \LaTeX{} theorem environment\thanks{This file
%        has version number \fileversion, last
%        revised \filedate.}}
% \author{Frank Mittelbach}
%
% \MaintainedByLaTeXTeam{tools}
% \maketitle
%
% \begin{abstract}
%    The macros described in this paper yield an extension of the
%    \LaTeX{} theorem mechanism. It is designed to satisfy the
%    different requirements of various journals. Thus, the
%    layout of the ``theorems'' can be manipulated by determining a
%    ``style''. This article describes not only the use, but
%    also the definition, of the necessary macros.
% \end{abstract}
%
% \section*{Preface to version 2.2}
%
% For \LaTeXe{} this package did not need any fundamental changes. I
% only modified the messages generated so that theorem layout styles
% will show up with the |\listfiles| command and cleaned the section
% on the New Font Selection Scheme since this is now included in
% \LaTeX.
%
% \section*{Preface to version 2.1}
%
% This version is identical to 2.0g described in \TUB\ 10\#3 except
% for some internal defaults which are now set depending on the used
% font selection scheme.
%
% This was done to avoid unpleasant surprises if the new font selection
% scheme is in force. For further details see
% section~\ref{sec:newfonts} and \cite{bk:GMS94}.
%
% \section{Introduction}
%
% For our purposes here, ``theorems'' are labelled enunciations,
% often set off from the main text by extra space and a font change.
% Theorems, corollaries, conjectures, definitions, and remarks
% are all instances of ``theorems''.  The ``header'' of these structures
% is composed of a label (such as \textsc{Theorem} or \textsc{Remark})
% and a number which serializes an item in the sequence of items
% with the same label.
%
% Shortly after the introduction of \LaTeX{} at the Fachbereich
% Mathematik in Mainz, the desire to manipulate the layout of
% ``theorems'' arose.  In Mainz, the following two conventions came
% into general use:
% \begin{enumerate}
%   \item The number of the theorem is shown in the margin.
%   \item There is a line break at
%         the end of the theorem header.
% \end{enumerate}
%
% Additionally, some journals require different formats which depend on
% the ``sort of theorem'': e.g.\ often remarks and definitions are set
% in |\upshape|, while |\itshape| is employed for main theorems.
%
% Confronted with these requirements, a theorem environment was
% developed in Mainz which allows separate determination of the
% layout of the ``theorems sets'', comparable to |\pagestyle|.
%
%
% \section{The user interface}
% \subsection{Defining new theorem sets}
%
% \DescribeMacro\newtheorem
% As in the original \LaTeX{} version, the command |\newtheorem|
% defines a new ``theorem set'' or ``theorem-like structure''.
% Two required arguments name the new environment and give the text
% to be typeset with each instance of the new ``set'', while
% an optional argument determines how
% the ``set'' is enumerated:
% \begin{description}
%    \item[\ttfamily \bslash newtheorem\{foo\}\{bar\}]
%       The theorem set {\envfont foo} (whose name is \texttt{bar})
%       uses its own counter.
%    \item[\ttfamily \bslash newtheorem\{foo2\}{[foo]}\{bar2\}]
%       The theorem set {\envfont foo2} (printed name \texttt{bar2})
%       uses the same counter as the theorem set \texttt{foo}.
%    \item[\ttfamily \bslash newtheorem\{foo3\}\{bar3\}{[section]}]
%       The theorem set {\envfont foo3} (printed name \texttt{bar3}) is
%       enumerated within the counter \texttt{section}, i.e.\ with every
%       new |\section| the enumeration begins again with 1, and
%       the enumeration is composed from the section-number and the
%       theorem counter itself.
% \end{description}
%
% \DescribeMacro\theoremstyle
% Additionally, the command |\theoremstyle| can define the layout
% of various, or all, theorem sets.  It should be noted that any
% theorem set defined by |\newtheorem| is typeset in the
% |\theoremstyle| that is current at the time of the definition.
% Thus, the following
% \begin{verbatim}
%    \theoremstyle{break}        \newtheorem{Cor}{Corollary}
%    \theoremstyle{plain}        \newtheorem{Exa}{Example}[section]
%\end{verbatim}                             ^^A No blank at line start
% leads to the result that the set {\envfont Cor} is formatted in the
% style \texttt{break}, while the set {\envfont Exa} and all the
% following ones are formatted in the style \texttt{plain}, unless
% another |\theoremstyle| follows.  Since the definitions installed by
% |\newtheorem| are global, one also can limit |\theoremstyle| locally
% by grouping braces.
%
% \DescribeMacro\theorembodyfont
% The choice of the font for the theorem body is completely
% independent of the chosen |\theoremstyle|; this has proven to
% be very advantageous.  For example,
% \begin{verbatim}
%   {\theorembodyfont{\upshape}          \newtheorem{Rem}{Remark}}
%\end{verbatim}
% defines a theorem set {\envfont Rem}, which will be set in
% |\upshape| in the current layout (which in our example is
% \texttt{plain}).  As with |\theoremstyle|, the |\theorembodyfont|
% chosen is that current at the time of |\newtheorem|.  If
% |\theorembodyfont| is not specified or one defines
% |\theorembodyfont{}|, then the font used will be that defined by the
% |\theoremstyle|.
%
% \DescribeMacro\theoremheaderfont It is also possible to customize
% the font used for the theorem headers.  This is, however, a global
% declaration, and therefore there should be at most one
% |\theoremheaderfont| declaration in the preamble.\footnote{If it is
% actually necessary to have different header fonts, one has to define
% new theorem styles (substituting the desired font) or specify the
% information directly in the \texttt{\bslash newtheorem} declaration
% (the unclean variant).}
%
% \DescribeMacro\theorempreskipamount
% \DescribeMacro\theorempostskipamount Two additional parameters
% affect the vertical space around the theorem environments:
% |\theorempreskipamount| and |\theorempostskipamount| define,
% respectively, the spacing before and after such an environment.
% These parameters apply for all theorem sets and can be manipulated
% with the ordinary length macros.  They are rubber lengths
% (`\textsf{skips}'), and therefore can contain \texttt{plus} and
% \texttt{minus} parts.
%
% Since the definition of theorem sets should---most sensibly---be
% placed in the preamble, we only allow installation there.
% It is therefore possible to release the memory used here after
% |\begin{document}|, in order to make room for other
% applications.
%
%
% \subsection{Existing theorem styles}
%
% The following theorem styles exist to date:
% \begin{deflist}{marginbreak}
%    \item[plain]
%       This theorem style emulates the original \LaTeX{} definition,
%       except that additionally the parameters
%       |\theorem...skipamount| are used.
%    \item[break]
%       In this style, the theorem header is followed by a line
%       break.
%    \item[marginbreak]
%       The theorem number is set in the margin, and there is a line
%       break as in \texttt{break}.
%    \item[changebreak]
%       Like \texttt{break}, but with header number and text
%       interchanged.
%    \item[change]
%       Header number and text are interchanged, without a line break.
%    \item[margin]
%       The number is set in the left margin, without a line break.
% \end{deflist}
% All styles (except \texttt{plain}) select |\slshape| as the default
% |\theorembodyfont|.
%
%
% \subsection{Examples}
%
% Given the above theorem sets {\envfont Cor}, {\envfont Exa}
% and {\envfont Rem},
% suppose that the preamble also contains the declarations:
% \begin{verbatim}
%    \theoremstyle{marginbreak}   \newtheorem{Lem}[Cor]{Lemma}
%    \theoremstyle{change}
%    \theorembodyfont{\itshape}        \newtheorem{Def}[Cor]{Definition}
%
%    \theoremheaderfont{\scshape}
%\end{verbatim}
% Then the following are some typical examples of the typeset output
% resulting from their use.
%
% \begin{Cor}
%    This is a sentence typeset in the theorem environment {\envfont
%    Cor}.
% \end{Cor}
%
% \begin{Exa}
%    This is a sentence typeset in the theorem environment {\envfont
%    Exa}.
% \end{Exa}
%
% \begin{Rem}
%    This is a sentence typeset in the theorem environment {\envfont
%    Rem}.
% \end{Rem}
%
% \begin{Lem}[Ben User]
%    This is a sentence typeset in the theorem environment {\envfont
%    Lem}.
% \end{Lem}
%
% \begin{Def}[Very impressive Definition]
%    This is a sentence typeset in the theorem environment {\envfont
%    Def}.
% \end{Def}
%
% The last two examples show the effect of the optional argument to
% a theorem environment (it is the text typeset in parentheses).
%
%
% \section{Special Considerations}
% \label{sec:newfonts}
%
% Theoremheader and body are implemented as a unit. This means that
% the |\theoremheaderfont|  will inherit characteristics of the
% |\theorembodyfont| in \LaTeXe. Thus, if for example
% |\theorembodyfont| is |\itshape| and |\theoremheaderfont|
% is |\bfseries| the font selected for the header will have the
% characteristics `bold extended italic'. If this is not desired one
% should set the |\theoremheaderfont| to something like
% \begin{verbatim}
%      \theoremheaderfont{\normalfont\bfseries}
%\end{verbatim}
% i.e.\ supplying all necessary font information explicitly.
%
%
%
% \section{Acknowledgements}
%
% The publication of this set of macros was only possible with the
% help of Christina Busse (translating the manuscript into English),
% Joachim Pense (playing the r\^ole of typist), Chris Rowley
% (looking everything over) and many others providing useful
% suggestions.
%
% \MaybeStop{%
%    \begin{thebibliography}{9}
%       \bibitem{bk:GMS94} \textsc{M.~Goossens}, \textsc{F.~Mittelbach}
%         and \textsc{A.~Samarin}.
%         \newblock The \LaTeX{} Companion.
%         \newblock
%         Addison-Wesley, Reading, Massachusetts, 1994.
%       \bibitem{latex:source} \textsc{Lamport, Leslie.} \newblock
%                               \texttt{latex.tex}, version 2.09, date
%                               Feb. 1990.
%    \end{thebibliography}
%    \PrintIndex
%    \PrintChanges}
%
%
% \section{The documentation driver file}
%
% The next bit of code contains the documentation driver file for
% \TeX{}, i.e., the file that will produce the documentation you are
% currently reading. It will be extracted from this file by the
% \texttt{docstrip} program. Since it is the first code in the file
% one can alternatively process this file directly with \LaTeXe{} to
% obtain the documentation.
%
%    \begin{macrocode}
%<*driver>
\documentclass{ltxdoc}

\usepackage{theorem}

%<+driver>% The next few lines define theorem sets which are used
%<+driver>% in the example section of the documentation.

\theoremstyle{break}       \newtheorem{Cor}{Corollary}
\theoremstyle{plain}       \newtheorem{Exa}{Example}[section]
{\theorembodyfont{\upshape}\newtheorem{Rem}{Remark}}
\theoremstyle{marginbreak} \newtheorem{Lem}[Cor]{Lemma}
\theoremstyle{change}
\theorembodyfont{\itshape} \newtheorem{Def}[Cor]{Definition}

\theoremheaderfont{\scshape}

\RecordChanges

\begin{document}
   \DocInput{theorem.dtx}
\end{document}
%</driver>
%    \end{macrocode}
%
% \section{Definition of the Macros}
%
% If the file has been loaded before, we abort immediately. If not the
% package announces itself (this is actually done at the very top of
% the file---the way it is done isn't good style so don't copy it).
% \changes{v2.0e}{89/07/19}{Spaces removed from `typeout'.}
% \changes{v2.2b}{95/11/19}{Announce moved to top of file}
%    \begin{macrocode}
%<*package>
%\@ifundefined{theorem@style}{}{\endinput}
%\def\FMithmInfo{1995/11/19 v2.2b Theorem extension package (FMi)}
%\ProvidesPackage{theorem}[\FMithmInfo]
%    \end{macrocode}
%
%
% \subsection{Definition of theorem styles and fonts}
%
% All the definitions in this file are done globally to allow
% inputting this file inside a group.
%
% \begin{macro}{\theoremstyle}
%    Before a theorem style can be installed, the chosen style must be
%    known. For that reason, we must test to see that
%    |\th@|\meta{style} is known or, more precisely, that it is
%    different from |\relax|. If the style is not known then
%    |\th@plain| is used.
%    \begin{macrocode}
\gdef\theoremstyle#1{%
   \@ifundefined{th@#1}{\@warning
          {Unknown theoremstyle `#1'. Using `plain'}%
          \theorem@style{plain}}%
%    \end{macrocode}
%    We save the theorem style to be used in the token register
%    |\theorem@style|.
%    \begin{macrocode}
      {\theorem@style{#1}}%
%    \end{macrocode}
%    Now we ``evaluate'' the theorem style: this means, we call the
%    macro |\th@|\meta{style} which will activate the relevant
%    definitions which are contained in a separate file.  This is done
%    in a group to suppress changes to the current font.  This could
%    otherwise pose problems together with the new font selection
%    scheme\footnote{When I printed the original article using the new
%    font selection scheme I ended with a document with slanted
%    typefaces (text headings and all) simply because one of the
%    theorem styles used \texttt{\bslash sl} at toplevel.} if the
%    |\th@|\meta{style} is evaluated a second time.
% \changes{v2.1a}{90/02/15}{Group added around `th@..}
% \changes{v2.1a}{90/02/15}{Reseting of `..@begintheorem removed.}
%    \begin{macrocode}
      \begingroup
        \csname th@\the\theorem@style \endcsname
      \endgroup}
%    \end{macrocode}
% \end{macro}
%
%
% \begin{macro}{\@begintheorem}
% \changes{v2.1a}{90/02/15}{Relaxed at toplevel.}
% \begin{macro}{\@opargbegintheorem}
% \changes{v2.1a}{90/02/15}{Relaxed at toplevel.}
%    We reset |\@begintheorem| and |\@opargbegintheorem| to
%    |\relax| since these commands are no longer necessary at
%    toplevel. This will save a few tokens.
%    \begin{macrocode}
\global\let\@begintheorem\relax
\global\let\@opargbegintheorem\relax
%    \end{macrocode}
% \end{macro}
% \end{macro}
%
% \begin{macro}{\theorem@style}
%    Obviously the \textsf{token} register used above has to be
%    allocated. To assure the utmost compatibility with the
%    original \LaTeX{} definition, we set the default theorem style to
%    \texttt{plain}, which implements the usual \LaTeX{} convention.
% \changes{v2.0f}{89/07/20}{Unnecessary `global' removed.}
%    \begin{macrocode}
\newtoks\theorem@style
\global\theorem@style{plain}
%    \end{macrocode}
% \end{macro}
%
% \begin{macro}{\theorembodyfont}
% \begin{macro}{\theorem@bodyfont}
%    If the |\theorembodyfont| is set by the user then it should not
%    interact with the default font set in the theorem style. When the
%    new font selection is in force this may happen if, for example,
%    the default is |\itshape| and the new |\theorembodyfont| is
%    |\sffamily|. So we add a |\reset@font| command in front of the
%    user definition.
% \changes{v2.0f}{89/07/20}{Unnecessary `global' removed.}
% \changes{v2.0g}{89/09/19}{`theoremfont to `theorembodyfont.}
% \changes{v2.1c}{92/06/22}{`reset@font added and logic changed.}
% \changes{v2.1d}{92/10/28}{Reset to empty if arg is empty}
%    \begin{macrocode}
\gdef\theorembodyfont#1{%
%    \end{macrocode}
%    We check if the argument supplied is empty and if so put nothing
%    into the |\theorem@bodyfont| token register to allow for
%    |\theorembodyfont{}| as a mean of using the default of the
%    current |\theoremstyle|.
%    \begin{macrocode}
   \def\@tempa{#1}%
   \ifx\@tempa\@empty
    \theorem@bodyfont{}%
   \else
    \theorem@bodyfont{\reset@font#1}%
   \fi
   }
\newtoks\theorem@bodyfont
\global\theorem@bodyfont{}
%    \end{macrocode}
% \end{macro}
% \end{macro}
%
%
% \begin{macro}{\theoremheaderfont}
% \changes{v2.0g}{89/09/19}{Macro added.}
%    The font for the theorem headers is handled differently because
%    this definition applies to all theorem styles.
%    \begin{macrocode}
\gdef\theoremheaderfont#1{\gdef\theorem@headerfont{#1}%
%    \end{macrocode}
%    After using the macro once it is redefined to produce an error
%    message.
%    \begin{macrocode}
       \gdef\theoremheaderfont##1{%
        \typeout{\string\theoremheaderfont\space should be used
                 only once.}}}
%    \end{macrocode}
% \end{macro}
%
% \begin{macro}{\theorem@headerfont}
% \changes{v2.0g}{89/09/19}{Macro added.}
%    To set the |\theorem@headerfont| default we first test if
%    the new fontselection scheme is in force.
% \changes{v2.1a}{90/02/15}{Support for new font selection.}
%    \begin{macrocode}
\ifx\upshape\undefined
%    \end{macrocode}
%    If not we define it to expand into |\bfseries|. We don't use
%    |\let| just in case a following style option redefines this
%    macro.
%    \begin{macrocode}
\gdef\theorem@headerfont{\bfseries}
%    \end{macrocode}
%    Otherwise we reset the current shape before calling |\bfseries|.
%    \begin{macrocode}
\else \gdef\theorem@headerfont{\normalfont\bfseries}\fi
%    \end{macrocode}
% \end{macro}
%
% \begin{macro}{\th@plain}
% \begin{macro}{\th@break}
% \begin{macro}{\th@marginbreak}
% \begin{macro}{\th@changebreak}
% \begin{macro}{\th@change}
% \begin{macro}{\th@margin}
%    The different styles are defined in macros such as
%    |\th@plain|.  Since memory space is precious in
%    ``non-Big-versions'', we have to avoid offering too many unused
%    definitions.  Therefore we define these styles in separate files
%    that can be loaded on demand.  Thus the commands themselves only
%    load these files. We use |\@input@|, a \LaTeXe{} internal command
%    that ensures that the file will be listed with |\listfiles|.
% \changes{v2.2a}{94/02/02}{Use `@input@ to load theorem layout files}
%    \begin{macrocode}
\gdef\th@plain{\@input@{thp.sty}}
\gdef\th@break{\@input@{thb.sty}}
\gdef\th@marginbreak{\@input@{thmb.sty}}
\gdef\th@changebreak{\@input@{thcb.sty}}
\gdef\th@change{\@input@{thc.sty}}
\gdef\th@margin{\@input@{thm.sty}}
%    \end{macrocode}
%    This list will be expanded when new styles become
%    available.  For testing,  just append new theorem substyles
%    as document options.
% \end{macro}
% \end{macro}
% \end{macro}
% \end{macro}
% \end{macro}
% \end{macro}
%
%
%
% \subsection{Definition of a new theorem set}
%
% As already pointed out, a new theorem environment can be defined in
% three different ways:
% \begin{verbatim}
%    \newtheorem{Lem}{Lemma}
%    \newtheorem{Lem}{Lemma}[section]
%    \newtheorem{Lem}[Theorem]{Lemma}
%\end{verbatim}
% The function of the macro |\newtheorem| is to recognize these cases
% and then to branch into one of the three macros |\@ynthm|, |\@xnthm|
% or |\@othm|.  This mechanism is adopted unchanged from
% \cite{latex:source}; the essential point here is that, for example,
% in the second case, the arguments \texttt{Lem}, \texttt{Lemma} and
% \texttt{section} are passed over to the macro |\@xnthm|.
%
% We inspect this case first because the others present fewer problems,
% and thus are easily derived from this one.
%
% \begin{macro}{\@xnthm}
%    For our example arguments, the macro |\@xnthm| must fulfill
%    the following:
%    \begin{itemize}
%       \item
%          Define a new \LaTeX-counter `Lem'
%       \item
%          reset this counter within a |\section|
%       \item
%          define the macro |\theLem|
%       \item
%          define the environment macros |\Lem| and
%          |\endLem| using the current
%          |\theoremstyle| and |\theorem@bodyfont|.
%    \end{itemize}
%    Obviously, all this should happen only if the first argument
%    of |\@xnthm| (i.e.\ \texttt{Lem} in our example) is chosen so
%    as not to conflict with any previously defined commands or
%    environments.  This test is performed by the \LaTeX{} macro
%    |\@ifdefinable|.
%    \begin{macrocode}
\gdef\@xnthm#1#2[#3]{\expandafter\@ifdefinable\csname #1\endcsname
%    \end{macrocode}
%    Therefore, the first argument of |\@ifdefinable| is the
%    expansion (in the example, |\Lem|) of
%    |\csname||#1\endcsname|.  The second argument is
%    executed only if the test has been completed successfully.
%    \begin{macrocode}
   {%
%    \end{macrocode}
%    Now we define the new counter. The names of the \LaTeX{}
%    macros employed should speak for themselves:
% \changes{v2.2b}{1995/10/07}{Use \cs{@newctr}}
% \changes{v2.2c}{1995/11/23}{Use \cs{@newctr} properly.}
%    \begin{macrocode}
    \@definecounter{#1}\@newctr{#1}[#3]%
%    \end{macrocode}
%    Using |\@newctr| will give a proper error message if the counter
%    in |#3| is not defined.
%    In defining `|\theLem|' we must generate the desired macro
%    name by use of |\expandafter| and |\csname|.
%    \begin{macrocode}
    \expandafter\xdef\csname the#1\endcsname
%    \end{macrocode}
%    An |\xdef| is used in order to make the definition global,
%    and to ensure that it contains the replacement texts of
%    |\@thmcountersep| and |\@thmcounter|.\footnote{These
%    two macros can be defined by the document style.  Their default
%    values produce a `\texttt{.}' as separation and an arabic
%    representation of the number.} However, not everything should be
%    expanded. For example, it saves space to use |\thesection|
%    instead of its---at times---lengthy expansion.
%    \begin{macrocode}
      {\expandafter \noexpand \csname the#3\endcsname
       \@thmcountersep \@thmcounter{#1}}%
%    \end{macrocode}
%    Thus with the defaults of \LaTeX{}, |\theLem| would be
%    replaced by the command sequence |\thesection.\arabic{Lem}|.
%
%    We will now look at the definition of the macro which is executed
%    at the beginning of the actual environment (in our example this
%    macro is |\Lem|). It should be noted that we use an
%    ``|\expandafter| trick'' to expand only certain parts of the
%    replacement text at the time of the definition.
%    \begin{macrocode}
    \def\@tempa{\global\@namedef{#1}}%
    \expandafter \@tempa \expandafter{%
%    \end{macrocode}
%    First, the macro that contains the current definitions of
%    |\@begintheorem| and |\@opargtheorem| should be called
%    up.  The name of this macro---as is already known---has the form
%    |\th@|\meta{theorem\ style}; therefore, it must be called by
%    \begin{macrocode}
      \csname th@\the \theorem@style
%    \end{macrocode}
%    In addition the default theorem font should be changeable, i.e.\
%    we have to insert the contents of |\theorem@bodyfont|. For
%    that reason, we expand even further, beyond |\endcsname|,
%    and thus insert the contents of the \textsf{token} register
%    |\theorem@bodyfont| in the replacement text.
%    \begin{macrocode}
            \expandafter \endcsname \the \theorem@bodyfont
%    \end{macrocode}
%    Now it is time to call the macro |\@thm| which takes over
%    the further processing.  It has two arguments: the current
%    counter name (in our example, \texttt{Lem}), and the text of the
%    label (in our example, \texttt{Lemma}).
%    \begin{macrocode}
     \@thm{#1}{#2}}%
%    \end{macrocode}
%    With this, the `sub-definition' is complete.  The macro
%    |\@endtheorem| ends a theorem environment and is, so far, nothing
%    but an |\endtrivlist|.  (Hence it is defined globally, and not
%    within the theorem styles.\footnote{This has to be changed as
%    soon as theorem styles that change \texttt{\bslash @endtheorem}
%    exist.  In such a case, all existing styles must be changed as
%    well since they will have to reset the macro.}) Therefore, we can
%    set it equivalent to the macro that ends the theorem set (in our
%    example, |\endLem|).  However, if some day theorem styles exist
%    that do change |\@endtheorem|, we would have to use the
%    commented-out line instead.
%    \begin{macrocode}
    \global \expandafter \let \csname end#1\endcsname \@endtheorem
%   \global\@namedef{end#1}{\@endtheorem}%
%    \end{macrocode}
%    With these commands all the required definitions are employed,
%    unless the test |\@ifdefinable| has failed.  Therefore, we
%    end the second argument of this macro and with it the definition
%    of |\@xnthm|.
%    \begin{macrocode}
   }}
%    \end{macrocode}
% \end{macro}
%
% \begin{macro}{\@ynthm}
%    The definition of |\@ynthm| is completely analogous. In
%    this case the new counter that is defined is not reset within
%    another counter; thus the definition of |\the...|
%    is simplified:
%    \begin{macrocode}
\gdef\@ynthm#1#2{\expandafter\@ifdefinable\csname #1\endcsname
   {\@definecounter{#1}%
    \expandafter\xdef\csname the#1\endcsname{\@thmcounter{#1}}%
%    \end{macrocode}
%    The rest of the definition corresponds literally to that of
%    |\@xnthm|:
%    \begin{macrocode}
    \def\@tempa{\global\@namedef{#1}}\expandafter \@tempa
     \expandafter{\csname th@\the \theorem@style \expandafter
     \endcsname \the\theorem@bodyfont \@thm{#1}{#2}}%
    \global \expandafter \let \csname end#1\endcsname \@endtheorem}}
%    \end{macrocode}
% \end{macro}
%
% \begin{macro}{\@othm}
%    The definition of |\@othm| does not contain anything new.
%    \begin{macrocode}
\gdef\@othm#1[#2]#3{%
%    \end{macrocode}
% \changes{v2.2b}{95/10/07}{Build in check for counter defined}
%    We do not define a new counter but instead use one that has
%    already been defined.  Thus the only definition we need is that
%    of this pseudo-counter (i.e.\ |\the|\meta{env.\ name}).
%    First we check if |#2| corresponds to a known counter name.
%    \begin{macrocode}
  \expandafter\ifx\csname c@#2\endcsname\relax
   \@nocounterr{#2}%
  \else
   \expandafter\@ifdefinable\csname #1\endcsname
   {\expandafter \xdef \csname the#1\endcsname
     {\expandafter \noexpand \csname the#2\endcsname}%
%    \end{macrocode}
%    All other parts of the definition can be adopted from
%    |\@xnthm|. We have to remember, though, that in this case
%    the name of the current counter and the theorem label have moved
%    to the second and third arguments.
%    \begin{macrocode}
    \def\@tempa{\global\@namedef{#1}}\expandafter \@tempa
     \expandafter{\csname th@\the \theorem@style \expandafter
     \endcsname \the\theorem@bodyfont \@thm{#2}{#3}}%
    \global \expandafter \let \csname end#1\endcsname \@endtheorem}%
  \fi}
%    \end{macrocode}
% \end{macro}
%
% \subsection{Macros that are employed in a {\envfont theorem}
%    environment}
%
% \begin{macro}{\@thm}
%    The macro |\@thm| has to increase the current counter.  Then,
%    depending on whether the environment has (or does not have) an
%    optional argument, it has to branch into either |\@begintheorem|
%    or |\@opargtheorem|.
%    \begin{macrocode}
\gdef\@thm#1#2{\refstepcounter{#1}%
%    \end{macrocode}
%    Now we start a {\envfont trivlist} environment, and give
%    |\@topsep| and |\@topsepadd| the values of the \textsf{skip}
%    registers |\theorempreskipamount| and |\theorempostskipamount|.
%    The value in |\@topsep| is the vertical space that is inserted by
%    the first (and only) |\item| in our |\trivlist| whilst
%    |\@topsepadd| is inserted by |\@endparenv| at the end of that
%    {\envfont trivlist} environment.  By using these registers, we
%    obtain the desired space around a {\envfont theorem} environment.
%    \begin{macrocode}
   \trivlist
   \@topsep \theorempreskipamount               % used by first \item
   \@topsepadd \theorempostskipamount           % used by \@endparenv
%    \end{macrocode}
%    Now we have to test whether an optional argument has been given.
% \changes{v2.0d}{89/07/18}{Forgotten percent added.}
%    \begin{macrocode}
   \@ifnextchar [%
%    \end{macrocode}
%    If there is an optional argument, we will call |\@ythm|, and
%    move the arguments read back into the input stream.
%    \begin{macrocode}
   {\@ythm{#1}{#2}}%
%    \end{macrocode}
%    If not, we call |\@begintheorem|. Its first argument is the
%    name of the theorem set (hence the second argument of
%    |\@thm|). Its second argument is the macro that produces the
%    current number.
%    \begin{macrocode}
   {\@begintheorem{#2}{\csname the#1\endcsname}\ignorespaces}}
%    \end{macrocode}
% \end{macro}
%
% \begin{macro}{\@xthm}
% \begin{macro}{\@ythm}
%    Both these macros were originally called by |\@thm|.  We do
%    not need |\@xthm| anymore, hence we reset it to
%    |\relax|. The definition of |\@ythm| has not changed at
%    all from its definition in \LaTeX. In order to make the macros
%    easier to understand, we will
%    nevertheless present it (commented out).
%    \begin{macrocode}
\global\let\@xthm\relax
% \def\@ythm#1#2[#3]{\@opargbegintheorem{#2}{\csname
%           the#1\endcsname}{#3}\ignorespaces}
%</package>
%    \end{macrocode}
%    The primitive |\ignorespaces| in |\@ythm| and
%    |\@thm| is needed to remove the spaces between the
%    |\begin{...}| and the actual text.
% \end{macro}
% \end{macro}
%
%
% \subsection{Definition of the theorem substyles}
%
% As already pointed out, the theorem substyles, defined below, are
% only loaded when necessary.  Note that all these substyles, except
% \texttt{plain}, have |\slshape| as the default body font.
%
% \subsubsection{The \texttt{plain} style}
%
% \label{dok:th@plain} As the following macros use |@|, we have
% to locally set the |\catcode| of this symbol to ``letter''.
% This happens within a group, so that we do not have to worry about
% which |\catcode| that symbol had before.
%    \begin{macrocode}
%<*thp>
\begingroup \makeatletter
%    \end{macrocode}
% Since we are now within a group, we must make all definitions
% globally.  First we make sure that \texttt{theorem.sty} is loaded.
% This will allow us to use this file as a document style option
% without having to call \texttt{theorem} itself as an option.  At the
% same time, we assure that at least version 2 is loaded, since
% |\theorem@style| was not defined in earlier versions.
%    \begin{macrocode}
\@ifundefined{theorem@style}{% \iffalse meta-comment
%
% Copyright (C) 1993-2024
%
% The LaTeX Project and any individual authors listed elsewhere
% in this file.
%
% This file is part of the Standard LaTeX `Tools Bundle'.
% -------------------------------------------------------
%
% It may be distributed and/or modified under the
% conditions of the LaTeX Project Public License, either version 1.3c
% of this license or (at your option) any later version.
% The latest version of this license is in
%    https://www.latex-project.org/lppl.txt
% and version 1.3c or later is part of all distributions of LaTeX
% version 2005/12/01 or later.
%
% The list of all files belonging to the LaTeX `Tools Bundle' is
% given in the file `manifest.txt'.
%
% \fi
%
%%
%
%
\def\FMithmInfo{2023/07/05 v2.2c Theorem extension package (FMi)}
%
% \ProvidesFile{theorem.dtx}[\FMithmInfo]
%\iffalse   % this is a METACOMMENT !
%%
%% Package `theorem' to use with LaTeX2e
%% Copyright (C) 1989-1995 Frank Mittelbach, all rights reserved.
%<+package>\@ifundefined{theorem@style}{}{\endinput}
%<+package>\NeedsTeXFormat{LaTeX2e}[1995/06/01]
%<+package>\ProvidesPackage{theorem}
%<+package>         [\FMithmInfo]
%
% \fi
%
% \changes{v1.0}{87/05/07}{First draft.}
% \changes{v1.0a}{87/05/07}{`thistheoremstyle' added.}
% \changes{v1.1a}{87/07/29}{New concept: redefine the internal
%                          \protect\LaTeX{} commands `@xnthm' etc.}
% \changes{v1.1b}{87/07/31}{`theoremfont' used to specify font.}
% \changes{v1.2a}{88/01/08}{`@endtheorem' now globally defined.}
% \changes{v1.2b}{88/01/11}{`newtheorem' only allowed in the preamble.}
% \changes{v1.3a}{88/02/14}{`theoremfont' now a toks register.}
%
% \changes{v2.0a}{88/11/16}{New implementation and new documentation.}
% \changes{v2.0a}{88/11/16}{No layout changes but internal modifications
%                           old substyles won't work.}
% \changes{v2.0b}{89/01/30}{`global' used for all definitions.}
% \changes{v2.0c}{89/06/15}{Documentation translated to English.
%                            Thanks to LLL and Nina}
% \changes{v2.1a}{90/02/15}{Support for new font selection.}
% \changes{v2.2a}{94/02/03}{Cleanup for LaTeX2e}
%
% \DoNotIndex{\@endparenv,\@ifdefinable,\@ifnextchar,\@ifundefined}
% \DoNotIndex{\@namedef,\@nameuse,\@spaces,\@tempa,\@warning}
% \DoNotIndex{\ }
% \DoNotIndex{\begingroup,\bfseries}
% \DoNotIndex{\csname}
% \DoNotIndex{\def,\do,\docdate}
% \DoNotIndex{\endcsname,\endgroup,\endinput,\endtrivlist,\expandafter}
% \DoNotIndex{\filedate,\fileversion}
% \DoNotIndex{\gdef,\global}
% \DoNotIndex{\hbox,\hskip}
% \DoNotIndex{\ignorespaces,\input,\itshape,\item}
% \DoNotIndex{\labelsep,\let,\llap}
% \DoNotIndex{\makeatletter}
% \DoNotIndex{\noexpand}
% \DoNotIndex{\refstepcounter,\relax,\rlap}
% \DoNotIndex{\setlength,\slshape,\space,\string,\strut}
% \DoNotIndex{\the,\trivlist,\typeout}
% \DoNotIndex{\vbox}
% \DoNotIndex{\xdef}
%
% \def\deflabel#1{\ttfamily #1\hfill}
% \def\deflist#1{\begin{list}{}{\settowidth\labelwidth{\ttfamily #1}%
%                               \setlength\leftmargin\labelwidth
%                               \addtolength\leftmargin\labelsep
%                               \let\makelabel\deflabel}}
% \def\enddeflist{\end{list}}
% \def\envfont{\normalfont\ttfamily}
%
% \MakeShortVerb{\|}
%
% \newcommand{\TUB}{{\sl TUGboat\/}}
%
% \GetFileInfo{theorem.dtx}
% \title{An Extension of the
%        \LaTeX{} theorem environment\thanks{This file
%        has version number \fileversion, last
%        revised \filedate.}}
% \author{Frank Mittelbach}
%
% \MaintainedByLaTeXTeam{tools}
% \maketitle
%
% \begin{abstract}
%    The macros described in this paper yield an extension of the
%    \LaTeX{} theorem mechanism. It is designed to satisfy the
%    different requirements of various journals. Thus, the
%    layout of the ``theorems'' can be manipulated by determining a
%    ``style''. This article describes not only the use, but
%    also the definition, of the necessary macros.
% \end{abstract}
%
% \section*{Preface to version 2.2}
%
% For \LaTeXe{} this package did not need any fundamental changes. I
% only modified the messages generated so that theorem layout styles
% will show up with the |\listfiles| command and cleaned the section
% on the New Font Selection Scheme since this is now included in
% \LaTeX.
%
% \section*{Preface to version 2.1}
%
% This version is identical to 2.0g described in \TUB\ 10\#3 except
% for some internal defaults which are now set depending on the used
% font selection scheme.
%
% This was done to avoid unpleasant surprises if the new font selection
% scheme is in force. For further details see
% section~\ref{sec:newfonts} and \cite{bk:GMS94}.
%
% \section{Introduction}
%
% For our purposes here, ``theorems'' are labelled enunciations,
% often set off from the main text by extra space and a font change.
% Theorems, corollaries, conjectures, definitions, and remarks
% are all instances of ``theorems''.  The ``header'' of these structures
% is composed of a label (such as \textsc{Theorem} or \textsc{Remark})
% and a number which serializes an item in the sequence of items
% with the same label.
%
% Shortly after the introduction of \LaTeX{} at the Fachbereich
% Mathematik in Mainz, the desire to manipulate the layout of
% ``theorems'' arose.  In Mainz, the following two conventions came
% into general use:
% \begin{enumerate}
%   \item The number of the theorem is shown in the margin.
%   \item There is a line break at
%         the end of the theorem header.
% \end{enumerate}
%
% Additionally, some journals require different formats which depend on
% the ``sort of theorem'': e.g.\ often remarks and definitions are set
% in |\upshape|, while |\itshape| is employed for main theorems.
%
% Confronted with these requirements, a theorem environment was
% developed in Mainz which allows separate determination of the
% layout of the ``theorems sets'', comparable to |\pagestyle|.
%
%
% \section{The user interface}
% \subsection{Defining new theorem sets}
%
% \DescribeMacro\newtheorem
% As in the original \LaTeX{} version, the command |\newtheorem|
% defines a new ``theorem set'' or ``theorem-like structure''.
% Two required arguments name the new environment and give the text
% to be typeset with each instance of the new ``set'', while
% an optional argument determines how
% the ``set'' is enumerated:
% \begin{description}
%    \item[\ttfamily \bslash newtheorem\{foo\}\{bar\}]
%       The theorem set {\envfont foo} (whose name is \texttt{bar})
%       uses its own counter.
%    \item[\ttfamily \bslash newtheorem\{foo2\}{[foo]}\{bar2\}]
%       The theorem set {\envfont foo2} (printed name \texttt{bar2})
%       uses the same counter as the theorem set \texttt{foo}.
%    \item[\ttfamily \bslash newtheorem\{foo3\}\{bar3\}{[section]}]
%       The theorem set {\envfont foo3} (printed name \texttt{bar3}) is
%       enumerated within the counter \texttt{section}, i.e.\ with every
%       new |\section| the enumeration begins again with 1, and
%       the enumeration is composed from the section-number and the
%       theorem counter itself.
% \end{description}
%
% \DescribeMacro\theoremstyle
% Additionally, the command |\theoremstyle| can define the layout
% of various, or all, theorem sets.  It should be noted that any
% theorem set defined by |\newtheorem| is typeset in the
% |\theoremstyle| that is current at the time of the definition.
% Thus, the following
% \begin{verbatim}
%    \theoremstyle{break}        \newtheorem{Cor}{Corollary}
%    \theoremstyle{plain}        \newtheorem{Exa}{Example}[section]
%\end{verbatim}                             ^^A No blank at line start
% leads to the result that the set {\envfont Cor} is formatted in the
% style \texttt{break}, while the set {\envfont Exa} and all the
% following ones are formatted in the style \texttt{plain}, unless
% another |\theoremstyle| follows.  Since the definitions installed by
% |\newtheorem| are global, one also can limit |\theoremstyle| locally
% by grouping braces.
%
% \DescribeMacro\theorembodyfont
% The choice of the font for the theorem body is completely
% independent of the chosen |\theoremstyle|; this has proven to
% be very advantageous.  For example,
% \begin{verbatim}
%   {\theorembodyfont{\upshape}          \newtheorem{Rem}{Remark}}
%\end{verbatim}
% defines a theorem set {\envfont Rem}, which will be set in
% |\upshape| in the current layout (which in our example is
% \texttt{plain}).  As with |\theoremstyle|, the |\theorembodyfont|
% chosen is that current at the time of |\newtheorem|.  If
% |\theorembodyfont| is not specified or one defines
% |\theorembodyfont{}|, then the font used will be that defined by the
% |\theoremstyle|.
%
% \DescribeMacro\theoremheaderfont It is also possible to customize
% the font used for the theorem headers.  This is, however, a global
% declaration, and therefore there should be at most one
% |\theoremheaderfont| declaration in the preamble.\footnote{If it is
% actually necessary to have different header fonts, one has to define
% new theorem styles (substituting the desired font) or specify the
% information directly in the \texttt{\bslash newtheorem} declaration
% (the unclean variant).}
%
% \DescribeMacro\theorempreskipamount
% \DescribeMacro\theorempostskipamount Two additional parameters
% affect the vertical space around the theorem environments:
% |\theorempreskipamount| and |\theorempostskipamount| define,
% respectively, the spacing before and after such an environment.
% These parameters apply for all theorem sets and can be manipulated
% with the ordinary length macros.  They are rubber lengths
% (`\textsf{skips}'), and therefore can contain \texttt{plus} and
% \texttt{minus} parts.
%
% Since the definition of theorem sets should---most sensibly---be
% placed in the preamble, we only allow installation there.
% It is therefore possible to release the memory used here after
% |\begin{document}|, in order to make room for other
% applications.
%
%
% \subsection{Existing theorem styles}
%
% The following theorem styles exist to date:
% \begin{deflist}{marginbreak}
%    \item[plain]
%       This theorem style emulates the original \LaTeX{} definition,
%       except that additionally the parameters
%       |\theorem...skipamount| are used.
%    \item[break]
%       In this style, the theorem header is followed by a line
%       break.
%    \item[marginbreak]
%       The theorem number is set in the margin, and there is a line
%       break as in \texttt{break}.
%    \item[changebreak]
%       Like \texttt{break}, but with header number and text
%       interchanged.
%    \item[change]
%       Header number and text are interchanged, without a line break.
%    \item[margin]
%       The number is set in the left margin, without a line break.
% \end{deflist}
% All styles (except \texttt{plain}) select |\slshape| as the default
% |\theorembodyfont|.
%
%
% \subsection{Examples}
%
% Given the above theorem sets {\envfont Cor}, {\envfont Exa}
% and {\envfont Rem},
% suppose that the preamble also contains the declarations:
% \begin{verbatim}
%    \theoremstyle{marginbreak}   \newtheorem{Lem}[Cor]{Lemma}
%    \theoremstyle{change}
%    \theorembodyfont{\itshape}        \newtheorem{Def}[Cor]{Definition}
%
%    \theoremheaderfont{\scshape}
%\end{verbatim}
% Then the following are some typical examples of the typeset output
% resulting from their use.
%
% \begin{Cor}
%    This is a sentence typeset in the theorem environment {\envfont
%    Cor}.
% \end{Cor}
%
% \begin{Exa}
%    This is a sentence typeset in the theorem environment {\envfont
%    Exa}.
% \end{Exa}
%
% \begin{Rem}
%    This is a sentence typeset in the theorem environment {\envfont
%    Rem}.
% \end{Rem}
%
% \begin{Lem}[Ben User]
%    This is a sentence typeset in the theorem environment {\envfont
%    Lem}.
% \end{Lem}
%
% \begin{Def}[Very impressive Definition]
%    This is a sentence typeset in the theorem environment {\envfont
%    Def}.
% \end{Def}
%
% The last two examples show the effect of the optional argument to
% a theorem environment (it is the text typeset in parentheses).
%
%
% \section{Special Considerations}
% \label{sec:newfonts}
%
% Theoremheader and body are implemented as a unit. This means that
% the |\theoremheaderfont|  will inherit characteristics of the
% |\theorembodyfont| in \LaTeXe. Thus, if for example
% |\theorembodyfont| is |\itshape| and |\theoremheaderfont|
% is |\bfseries| the font selected for the header will have the
% characteristics `bold extended italic'. If this is not desired one
% should set the |\theoremheaderfont| to something like
% \begin{verbatim}
%      \theoremheaderfont{\normalfont\bfseries}
%\end{verbatim}
% i.e.\ supplying all necessary font information explicitly.
%
%
%
% \section{Acknowledgements}
%
% The publication of this set of macros was only possible with the
% help of Christina Busse (translating the manuscript into English),
% Joachim Pense (playing the r\^ole of typist), Chris Rowley
% (looking everything over) and many others providing useful
% suggestions.
%
% \MaybeStop{%
%    \begin{thebibliography}{9}
%       \bibitem{bk:GMS94} \textsc{M.~Goossens}, \textsc{F.~Mittelbach}
%         and \textsc{A.~Samarin}.
%         \newblock The \LaTeX{} Companion.
%         \newblock
%         Addison-Wesley, Reading, Massachusetts, 1994.
%       \bibitem{latex:source} \textsc{Lamport, Leslie.} \newblock
%                               \texttt{latex.tex}, version 2.09, date
%                               Feb. 1990.
%    \end{thebibliography}
%    \PrintIndex
%    \PrintChanges}
%
%
% \section{The documentation driver file}
%
% The next bit of code contains the documentation driver file for
% \TeX{}, i.e., the file that will produce the documentation you are
% currently reading. It will be extracted from this file by the
% \texttt{docstrip} program. Since it is the first code in the file
% one can alternatively process this file directly with \LaTeXe{} to
% obtain the documentation.
%
%    \begin{macrocode}
%<*driver>
\documentclass{ltxdoc}

\usepackage{theorem}

%<+driver>% The next few lines define theorem sets which are used
%<+driver>% in the example section of the documentation.

\theoremstyle{break}       \newtheorem{Cor}{Corollary}
\theoremstyle{plain}       \newtheorem{Exa}{Example}[section]
{\theorembodyfont{\upshape}\newtheorem{Rem}{Remark}}
\theoremstyle{marginbreak} \newtheorem{Lem}[Cor]{Lemma}
\theoremstyle{change}
\theorembodyfont{\itshape} \newtheorem{Def}[Cor]{Definition}

\theoremheaderfont{\scshape}

\RecordChanges

\begin{document}
   \DocInput{theorem.dtx}
\end{document}
%</driver>
%    \end{macrocode}
%
% \section{Definition of the Macros}
%
% If the file has been loaded before, we abort immediately. If not the
% package announces itself (this is actually done at the very top of
% the file---the way it is done isn't good style so don't copy it).
% \changes{v2.0e}{89/07/19}{Spaces removed from `typeout'.}
% \changes{v2.2b}{95/11/19}{Announce moved to top of file}
%    \begin{macrocode}
%<*package>
%\@ifundefined{theorem@style}{}{\endinput}
%\def\FMithmInfo{1995/11/19 v2.2b Theorem extension package (FMi)}
%\ProvidesPackage{theorem}[\FMithmInfo]
%    \end{macrocode}
%
%
% \subsection{Definition of theorem styles and fonts}
%
% All the definitions in this file are done globally to allow
% inputting this file inside a group.
%
% \begin{macro}{\theoremstyle}
%    Before a theorem style can be installed, the chosen style must be
%    known. For that reason, we must test to see that
%    |\th@|\meta{style} is known or, more precisely, that it is
%    different from |\relax|. If the style is not known then
%    |\th@plain| is used.
%    \begin{macrocode}
\gdef\theoremstyle#1{%
   \@ifundefined{th@#1}{\@warning
          {Unknown theoremstyle `#1'. Using `plain'}%
          \theorem@style{plain}}%
%    \end{macrocode}
%    We save the theorem style to be used in the token register
%    |\theorem@style|.
%    \begin{macrocode}
      {\theorem@style{#1}}%
%    \end{macrocode}
%    Now we ``evaluate'' the theorem style: this means, we call the
%    macro |\th@|\meta{style} which will activate the relevant
%    definitions which are contained in a separate file.  This is done
%    in a group to suppress changes to the current font.  This could
%    otherwise pose problems together with the new font selection
%    scheme\footnote{When I printed the original article using the new
%    font selection scheme I ended with a document with slanted
%    typefaces (text headings and all) simply because one of the
%    theorem styles used \texttt{\bslash sl} at toplevel.} if the
%    |\th@|\meta{style} is evaluated a second time.
% \changes{v2.1a}{90/02/15}{Group added around `th@..}
% \changes{v2.1a}{90/02/15}{Reseting of `..@begintheorem removed.}
%    \begin{macrocode}
      \begingroup
        \csname th@\the\theorem@style \endcsname
      \endgroup}
%    \end{macrocode}
% \end{macro}
%
%
% \begin{macro}{\@begintheorem}
% \changes{v2.1a}{90/02/15}{Relaxed at toplevel.}
% \begin{macro}{\@opargbegintheorem}
% \changes{v2.1a}{90/02/15}{Relaxed at toplevel.}
%    We reset |\@begintheorem| and |\@opargbegintheorem| to
%    |\relax| since these commands are no longer necessary at
%    toplevel. This will save a few tokens.
%    \begin{macrocode}
\global\let\@begintheorem\relax
\global\let\@opargbegintheorem\relax
%    \end{macrocode}
% \end{macro}
% \end{macro}
%
% \begin{macro}{\theorem@style}
%    Obviously the \textsf{token} register used above has to be
%    allocated. To assure the utmost compatibility with the
%    original \LaTeX{} definition, we set the default theorem style to
%    \texttt{plain}, which implements the usual \LaTeX{} convention.
% \changes{v2.0f}{89/07/20}{Unnecessary `global' removed.}
%    \begin{macrocode}
\newtoks\theorem@style
\global\theorem@style{plain}
%    \end{macrocode}
% \end{macro}
%
% \begin{macro}{\theorembodyfont}
% \begin{macro}{\theorem@bodyfont}
%    If the |\theorembodyfont| is set by the user then it should not
%    interact with the default font set in the theorem style. When the
%    new font selection is in force this may happen if, for example,
%    the default is |\itshape| and the new |\theorembodyfont| is
%    |\sffamily|. So we add a |\reset@font| command in front of the
%    user definition.
% \changes{v2.0f}{89/07/20}{Unnecessary `global' removed.}
% \changes{v2.0g}{89/09/19}{`theoremfont to `theorembodyfont.}
% \changes{v2.1c}{92/06/22}{`reset@font added and logic changed.}
% \changes{v2.1d}{92/10/28}{Reset to empty if arg is empty}
%    \begin{macrocode}
\gdef\theorembodyfont#1{%
%    \end{macrocode}
%    We check if the argument supplied is empty and if so put nothing
%    into the |\theorem@bodyfont| token register to allow for
%    |\theorembodyfont{}| as a mean of using the default of the
%    current |\theoremstyle|.
%    \begin{macrocode}
   \def\@tempa{#1}%
   \ifx\@tempa\@empty
    \theorem@bodyfont{}%
   \else
    \theorem@bodyfont{\reset@font#1}%
   \fi
   }
\newtoks\theorem@bodyfont
\global\theorem@bodyfont{}
%    \end{macrocode}
% \end{macro}
% \end{macro}
%
%
% \begin{macro}{\theoremheaderfont}
% \changes{v2.0g}{89/09/19}{Macro added.}
%    The font for the theorem headers is handled differently because
%    this definition applies to all theorem styles.
%    \begin{macrocode}
\gdef\theoremheaderfont#1{\gdef\theorem@headerfont{#1}%
%    \end{macrocode}
%    After using the macro once it is redefined to produce an error
%    message.
%    \begin{macrocode}
       \gdef\theoremheaderfont##1{%
        \typeout{\string\theoremheaderfont\space should be used
                 only once.}}}
%    \end{macrocode}
% \end{macro}
%
% \begin{macro}{\theorem@headerfont}
% \changes{v2.0g}{89/09/19}{Macro added.}
%    To set the |\theorem@headerfont| default we first test if
%    the new fontselection scheme is in force.
% \changes{v2.1a}{90/02/15}{Support for new font selection.}
%    \begin{macrocode}
\ifx\upshape\undefined
%    \end{macrocode}
%    If not we define it to expand into |\bfseries|. We don't use
%    |\let| just in case a following style option redefines this
%    macro.
%    \begin{macrocode}
\gdef\theorem@headerfont{\bfseries}
%    \end{macrocode}
%    Otherwise we reset the current shape before calling |\bfseries|.
%    \begin{macrocode}
\else \gdef\theorem@headerfont{\normalfont\bfseries}\fi
%    \end{macrocode}
% \end{macro}
%
% \begin{macro}{\th@plain}
% \begin{macro}{\th@break}
% \begin{macro}{\th@marginbreak}
% \begin{macro}{\th@changebreak}
% \begin{macro}{\th@change}
% \begin{macro}{\th@margin}
%    The different styles are defined in macros such as
%    |\th@plain|.  Since memory space is precious in
%    ``non-Big-versions'', we have to avoid offering too many unused
%    definitions.  Therefore we define these styles in separate files
%    that can be loaded on demand.  Thus the commands themselves only
%    load these files. We use |\@input@|, a \LaTeXe{} internal command
%    that ensures that the file will be listed with |\listfiles|.
% \changes{v2.2a}{94/02/02}{Use `@input@ to load theorem layout files}
%    \begin{macrocode}
\gdef\th@plain{\@input@{thp.sty}}
\gdef\th@break{\@input@{thb.sty}}
\gdef\th@marginbreak{\@input@{thmb.sty}}
\gdef\th@changebreak{\@input@{thcb.sty}}
\gdef\th@change{\@input@{thc.sty}}
\gdef\th@margin{\@input@{thm.sty}}
%    \end{macrocode}
%    This list will be expanded when new styles become
%    available.  For testing,  just append new theorem substyles
%    as document options.
% \end{macro}
% \end{macro}
% \end{macro}
% \end{macro}
% \end{macro}
% \end{macro}
%
%
%
% \subsection{Definition of a new theorem set}
%
% As already pointed out, a new theorem environment can be defined in
% three different ways:
% \begin{verbatim}
%    \newtheorem{Lem}{Lemma}
%    \newtheorem{Lem}{Lemma}[section]
%    \newtheorem{Lem}[Theorem]{Lemma}
%\end{verbatim}
% The function of the macro |\newtheorem| is to recognize these cases
% and then to branch into one of the three macros |\@ynthm|, |\@xnthm|
% or |\@othm|.  This mechanism is adopted unchanged from
% \cite{latex:source}; the essential point here is that, for example,
% in the second case, the arguments \texttt{Lem}, \texttt{Lemma} and
% \texttt{section} are passed over to the macro |\@xnthm|.
%
% We inspect this case first because the others present fewer problems,
% and thus are easily derived from this one.
%
% \begin{macro}{\@xnthm}
%    For our example arguments, the macro |\@xnthm| must fulfill
%    the following:
%    \begin{itemize}
%       \item
%          Define a new \LaTeX-counter `Lem'
%       \item
%          reset this counter within a |\section|
%       \item
%          define the macro |\theLem|
%       \item
%          define the environment macros |\Lem| and
%          |\endLem| using the current
%          |\theoremstyle| and |\theorem@bodyfont|.
%    \end{itemize}
%    Obviously, all this should happen only if the first argument
%    of |\@xnthm| (i.e.\ \texttt{Lem} in our example) is chosen so
%    as not to conflict with any previously defined commands or
%    environments.  This test is performed by the \LaTeX{} macro
%    |\@ifdefinable|.
%    \begin{macrocode}
\gdef\@xnthm#1#2[#3]{\expandafter\@ifdefinable\csname #1\endcsname
%    \end{macrocode}
%    Therefore, the first argument of |\@ifdefinable| is the
%    expansion (in the example, |\Lem|) of
%    |\csname||#1\endcsname|.  The second argument is
%    executed only if the test has been completed successfully.
%    \begin{macrocode}
   {%
%    \end{macrocode}
%    Now we define the new counter. The names of the \LaTeX{}
%    macros employed should speak for themselves:
% \changes{v2.2b}{1995/10/07}{Use \cs{@newctr}}
% \changes{v2.2c}{1995/11/23}{Use \cs{@newctr} properly.}
%    \begin{macrocode}
    \@definecounter{#1}\@newctr{#1}[#3]%
%    \end{macrocode}
%    Using |\@newctr| will give a proper error message if the counter
%    in |#3| is not defined.
%    In defining `|\theLem|' we must generate the desired macro
%    name by use of |\expandafter| and |\csname|.
%    \begin{macrocode}
    \expandafter\xdef\csname the#1\endcsname
%    \end{macrocode}
%    An |\xdef| is used in order to make the definition global,
%    and to ensure that it contains the replacement texts of
%    |\@thmcountersep| and |\@thmcounter|.\footnote{These
%    two macros can be defined by the document style.  Their default
%    values produce a `\texttt{.}' as separation and an arabic
%    representation of the number.} However, not everything should be
%    expanded. For example, it saves space to use |\thesection|
%    instead of its---at times---lengthy expansion.
%    \begin{macrocode}
      {\expandafter \noexpand \csname the#3\endcsname
       \@thmcountersep \@thmcounter{#1}}%
%    \end{macrocode}
%    Thus with the defaults of \LaTeX{}, |\theLem| would be
%    replaced by the command sequence |\thesection.\arabic{Lem}|.
%
%    We will now look at the definition of the macro which is executed
%    at the beginning of the actual environment (in our example this
%    macro is |\Lem|). It should be noted that we use an
%    ``|\expandafter| trick'' to expand only certain parts of the
%    replacement text at the time of the definition.
%    \begin{macrocode}
    \def\@tempa{\global\@namedef{#1}}%
    \expandafter \@tempa \expandafter{%
%    \end{macrocode}
%    First, the macro that contains the current definitions of
%    |\@begintheorem| and |\@opargtheorem| should be called
%    up.  The name of this macro---as is already known---has the form
%    |\th@|\meta{theorem\ style}; therefore, it must be called by
%    \begin{macrocode}
      \csname th@\the \theorem@style
%    \end{macrocode}
%    In addition the default theorem font should be changeable, i.e.\
%    we have to insert the contents of |\theorem@bodyfont|. For
%    that reason, we expand even further, beyond |\endcsname|,
%    and thus insert the contents of the \textsf{token} register
%    |\theorem@bodyfont| in the replacement text.
%    \begin{macrocode}
            \expandafter \endcsname \the \theorem@bodyfont
%    \end{macrocode}
%    Now it is time to call the macro |\@thm| which takes over
%    the further processing.  It has two arguments: the current
%    counter name (in our example, \texttt{Lem}), and the text of the
%    label (in our example, \texttt{Lemma}).
%    \begin{macrocode}
     \@thm{#1}{#2}}%
%    \end{macrocode}
%    With this, the `sub-definition' is complete.  The macro
%    |\@endtheorem| ends a theorem environment and is, so far, nothing
%    but an |\endtrivlist|.  (Hence it is defined globally, and not
%    within the theorem styles.\footnote{This has to be changed as
%    soon as theorem styles that change \texttt{\bslash @endtheorem}
%    exist.  In such a case, all existing styles must be changed as
%    well since they will have to reset the macro.}) Therefore, we can
%    set it equivalent to the macro that ends the theorem set (in our
%    example, |\endLem|).  However, if some day theorem styles exist
%    that do change |\@endtheorem|, we would have to use the
%    commented-out line instead.
%    \begin{macrocode}
    \global \expandafter \let \csname end#1\endcsname \@endtheorem
%   \global\@namedef{end#1}{\@endtheorem}%
%    \end{macrocode}
%    With these commands all the required definitions are employed,
%    unless the test |\@ifdefinable| has failed.  Therefore, we
%    end the second argument of this macro and with it the definition
%    of |\@xnthm|.
%    \begin{macrocode}
   }}
%    \end{macrocode}
% \end{macro}
%
% \begin{macro}{\@ynthm}
%    The definition of |\@ynthm| is completely analogous. In
%    this case the new counter that is defined is not reset within
%    another counter; thus the definition of |\the...|
%    is simplified:
%    \begin{macrocode}
\gdef\@ynthm#1#2{\expandafter\@ifdefinable\csname #1\endcsname
   {\@definecounter{#1}%
    \expandafter\xdef\csname the#1\endcsname{\@thmcounter{#1}}%
%    \end{macrocode}
%    The rest of the definition corresponds literally to that of
%    |\@xnthm|:
%    \begin{macrocode}
    \def\@tempa{\global\@namedef{#1}}\expandafter \@tempa
     \expandafter{\csname th@\the \theorem@style \expandafter
     \endcsname \the\theorem@bodyfont \@thm{#1}{#2}}%
    \global \expandafter \let \csname end#1\endcsname \@endtheorem}}
%    \end{macrocode}
% \end{macro}
%
% \begin{macro}{\@othm}
%    The definition of |\@othm| does not contain anything new.
%    \begin{macrocode}
\gdef\@othm#1[#2]#3{%
%    \end{macrocode}
% \changes{v2.2b}{95/10/07}{Build in check for counter defined}
%    We do not define a new counter but instead use one that has
%    already been defined.  Thus the only definition we need is that
%    of this pseudo-counter (i.e.\ |\the|\meta{env.\ name}).
%    First we check if |#2| corresponds to a known counter name.
%    \begin{macrocode}
  \expandafter\ifx\csname c@#2\endcsname\relax
   \@nocounterr{#2}%
  \else
   \expandafter\@ifdefinable\csname #1\endcsname
   {\expandafter \xdef \csname the#1\endcsname
     {\expandafter \noexpand \csname the#2\endcsname}%
%    \end{macrocode}
%    All other parts of the definition can be adopted from
%    |\@xnthm|. We have to remember, though, that in this case
%    the name of the current counter and the theorem label have moved
%    to the second and third arguments.
%    \begin{macrocode}
    \def\@tempa{\global\@namedef{#1}}\expandafter \@tempa
     \expandafter{\csname th@\the \theorem@style \expandafter
     \endcsname \the\theorem@bodyfont \@thm{#2}{#3}}%
    \global \expandafter \let \csname end#1\endcsname \@endtheorem}%
  \fi}
%    \end{macrocode}
% \end{macro}
%
% \subsection{Macros that are employed in a {\envfont theorem}
%    environment}
%
% \begin{macro}{\@thm}
%    The macro |\@thm| has to increase the current counter.  Then,
%    depending on whether the environment has (or does not have) an
%    optional argument, it has to branch into either |\@begintheorem|
%    or |\@opargtheorem|.
%    \begin{macrocode}
\gdef\@thm#1#2{\refstepcounter{#1}%
%    \end{macrocode}
%    Now we start a {\envfont trivlist} environment, and give
%    |\@topsep| and |\@topsepadd| the values of the \textsf{skip}
%    registers |\theorempreskipamount| and |\theorempostskipamount|.
%    The value in |\@topsep| is the vertical space that is inserted by
%    the first (and only) |\item| in our |\trivlist| whilst
%    |\@topsepadd| is inserted by |\@endparenv| at the end of that
%    {\envfont trivlist} environment.  By using these registers, we
%    obtain the desired space around a {\envfont theorem} environment.
%    \begin{macrocode}
   \trivlist
   \@topsep \theorempreskipamount               % used by first \item
   \@topsepadd \theorempostskipamount           % used by \@endparenv
%    \end{macrocode}
%    Now we have to test whether an optional argument has been given.
% \changes{v2.0d}{89/07/18}{Forgotten percent added.}
%    \begin{macrocode}
   \@ifnextchar [%
%    \end{macrocode}
%    If there is an optional argument, we will call |\@ythm|, and
%    move the arguments read back into the input stream.
%    \begin{macrocode}
   {\@ythm{#1}{#2}}%
%    \end{macrocode}
%    If not, we call |\@begintheorem|. Its first argument is the
%    name of the theorem set (hence the second argument of
%    |\@thm|). Its second argument is the macro that produces the
%    current number.
%    \begin{macrocode}
   {\@begintheorem{#2}{\csname the#1\endcsname}\ignorespaces}}
%    \end{macrocode}
% \end{macro}
%
% \begin{macro}{\@xthm}
% \begin{macro}{\@ythm}
%    Both these macros were originally called by |\@thm|.  We do
%    not need |\@xthm| anymore, hence we reset it to
%    |\relax|. The definition of |\@ythm| has not changed at
%    all from its definition in \LaTeX. In order to make the macros
%    easier to understand, we will
%    nevertheless present it (commented out).
%    \begin{macrocode}
\global\let\@xthm\relax
% \def\@ythm#1#2[#3]{\@opargbegintheorem{#2}{\csname
%           the#1\endcsname}{#3}\ignorespaces}
%</package>
%    \end{macrocode}
%    The primitive |\ignorespaces| in |\@ythm| and
%    |\@thm| is needed to remove the spaces between the
%    |\begin{...}| and the actual text.
% \end{macro}
% \end{macro}
%
%
% \subsection{Definition of the theorem substyles}
%
% As already pointed out, the theorem substyles, defined below, are
% only loaded when necessary.  Note that all these substyles, except
% \texttt{plain}, have |\slshape| as the default body font.
%
% \subsubsection{The \texttt{plain} style}
%
% \label{dok:th@plain} As the following macros use |@|, we have
% to locally set the |\catcode| of this symbol to ``letter''.
% This happens within a group, so that we do not have to worry about
% which |\catcode| that symbol had before.
%    \begin{macrocode}
%<*thp>
\begingroup \makeatletter
%    \end{macrocode}
% Since we are now within a group, we must make all definitions
% globally.  First we make sure that \texttt{theorem.sty} is loaded.
% This will allow us to use this file as a document style option
% without having to call \texttt{theorem} itself as an option.  At the
% same time, we assure that at least version 2 is loaded, since
% |\theorem@style| was not defined in earlier versions.
%    \begin{macrocode}
\@ifundefined{theorem@style}{\input{theorem.sty}}{}
\ProvidesFile{thp.sty}
             [\FMithmInfo]
%    \end{macrocode}
%
%
% \begin{macro}{\th@plain}
% \begin{macro}{\@begintheorem}
% \begin{macro}{\@opargbegintheorem}
%    |\theoremstyle{plain}| corresponds to the original
%    definition, except that the distances to the surrounding text are
%    determined by the parameters |\theorempreskipamount| and
%    |\theorempostskipamount|.
%    First we set the default body font.
%    \begin{macrocode}
\gdef\th@plain{\normalfont\itshape
%    \end{macrocode}
%    Then we define |\@begintheorem| and
%    |\@opargbegintheorem|. These two macros define how the
%    header of a theorem is typeset.  |\@opargbegintheorem| will
%    be called if a \textsf{theorem} environment with an optional
%    argument is encountered; otherwise, the header is constructed by
%    calling |\@begintheorem|.  If one of these macros is
%    executed, we are within a \textsf{trivlist} environment started by
%    |\@thm|.  So the theorem header is produced with an
%    |\item| command.
%
%    Instead of specifying the header font directly, all standard
%    theorem styles use the |\theorem@headerfont| macro to allow
%    customization.  The extra space (|\labelsep|) is necessary
%    because of problems in the \textsf{trivlist} environment.
%    \begin{macrocode}
  \def\@begintheorem##1##2{%
        \item[\hskip\labelsep \theorem@headerfont ##1\ ##2]}%
%    \end{macrocode}
%    The definition of |\@opargbegintheorem| is completely
%    analogous. The only difference is the fact that there exists a
%    third argument (which is the optional parameter of the
%    environment and contains additional information about the
%    theorem).  Customarily we enclose it in parentheses.
%    \begin{macrocode}
\def\@opargbegintheorem##1##2##3{%
   \item[\hskip\labelsep \theorem@headerfont ##1\ ##2\ (##3)]}}
%    \end{macrocode}
% \end{macro}
% \end{macro}
% \end{macro}
%
% \noindent
% We conclude with an |\endgroup| to restore the
% |\catcode| of |@|.
%    \begin{macrocode}
\endgroup
%</thp>
%    \end{macrocode}
%
%
% \subsubsection{The \texttt{break} style}
%
% This style option is stored in the file |thb.sty|.
% For the next two lines see the documentation for |\th@plain| on
% page~\pageref{dok:th@plain}.
%    \begin{macrocode}
%<*thb>
\begingroup \makeatletter
\@ifundefined{theorem@style}{\input{theorem.sty}}{}
\ProvidesFile{thb.sty}
             [\FMithmInfo]
%    \end{macrocode}
%
% \begin{macro}{\th@break}
%    |\theoremstyle{break}| produces a line break after the name
%    of the theorem.  The font is |\slshape|.  Hence, we define
%    |\th@break| as follows:
%    \begin{macrocode}
\gdef\th@break{\normalfont\slshape
  \def\@begintheorem##1##2{\item[%
%    \end{macrocode}
%    We run into the following problem: it is not possible to create
%    the header with |\item[|\meta{title}|]| and then start
%    a new line by, for example, |\mbox{}\\|. Such a definition
%    will fail whenever a list environment follows immediately.  With
%    the above construction, the |\mbox{}| causes the switch
%    |@inlabel| (cf.\ definition of |\list| and
%    |\trivlist| in \cite{latex:source}) to be set to \texttt{false}
%    and so the following list will insert additional vertical
%    space (|\topskip|). This is quite annoying. Therefore, we
%    create the line break within the |\item|. In order to ensure
%    that the text will begin at the proper position in the following
%    line, we simply pretend that the label does not take any
%    room.\footnote{This will lead to problems whenever very high
%    symbols occurring in the line tower into the
%    heading. So, something else has to be done here sometime.}
%    \begin{macrocode}
       \rlap{\vbox{\hbox{\hskip \labelsep\theorem@headerfont ##1\ ##2}%
                   \hbox{\strut}}}]}%
%    \end{macrocode}
%    Again, the definition of |\@opargbegintheorem| is
%    completely analogous.
%    \begin{macrocode}
\def\@opargbegintheorem##1##2##3{%
  \item[\rlap{\vbox{\hbox{\hskip \labelsep \theorem@headerfont
                     ##1\ ##2\ (##3)}%
                    \hbox{\strut}}}]}}
\endgroup
%</thb>
%    \end{macrocode}
% \end{macro}
%
%
% \subsubsection{The \texttt{changebreak} style}
%
%    \begin{macrocode}
% This style option is stored in the file |thcb.sty|.
%    \begin{macrocode}
%<*thcb>
\begingroup \makeatletter
\@ifundefined{theorem@style}{\input{theorem.sty}}{}
\ProvidesFile{thcb.sty}
             [\FMithmInfo]
%    \end{macrocode}
%
% \begin{macro}{\th@changebreak}
%    The \texttt{change-break} theorem style is like \texttt{break}
%    but with interchange of theorem name and theorem number.  Thus we
%    define |\th@changebreak| as follows:
%    \begin{macrocode}
\gdef\th@changebreak{\normalfont\slshape
  \def\@begintheorem##1##2{\item
       [\rlap{\vbox{\hbox{\hskip\labelsep \theorem@headerfont ##2\ ##1}%
                    \hbox{\strut}}}]}%
\def\@opargbegintheorem##1##2##3{%
  \item[\rlap{\vbox{\hbox{\hskip\labelsep \theorem@headerfont
                     ##2\ ##1\ (##3)}%
                    \hbox{\strut}}}]}}
\endgroup
%</thcb>
%    \end{macrocode}
% \end{macro}
%
%
% \subsubsection{The \texttt{change} style}
%
% This style option is stored in the file |thc.sty|.
%    \begin{macrocode}
%<*thc>
\begingroup \makeatletter
\@ifundefined{theorem@style}{\input{theorem.sty}}{}
\ProvidesFile{thc.sty}
             [\FMithmInfo]
%    \end{macrocode}
%
% \begin{macro}{\th@change}
%    The \texttt{change} theorem style corresponds to the
%    \texttt{change break} style without a linebreak after the header.
%    To say it in another way, it's the same as the \texttt{plain}
%    style but with number and name interchanged and |\slshape| as the
%    default font.
%    \begin{macrocode}
\gdef\th@change{\normalfont\slshape
  \def\@begintheorem##1##2{\item
       [\hskip\labelsep \theorem@headerfont ##2\ ##1]}%
\def\@opargbegintheorem##1##2##3{%
  \item[\hskip\labelsep \theorem@headerfont ##2\ ##1\ (##3)]}}
\endgroup
%</thc>
%    \end{macrocode}
% \end{macro}
%
%
% \subsubsection{The \texttt{marginbreak} style}
%
% This style option is the one used most often at Mainz.  It is saved
% in the file |thmb.sty|.
%    \begin{macrocode}
%<*thmb>
\begingroup \makeatletter
\@ifundefined{theorem@style}{\input{theorem.sty}}{}
\ProvidesFile{thmb.sty}
             [\FMithmInfo]
%    \end{macrocode}
%
% \begin{macro}{\th@marginbreak}
%    The \texttt{margin break} style is nearly the same as the
%    \texttt{change break} style. The only difference is the placement
%    of the theorem number.  We use |\llap| to place it in the left
%    margin.
%
%    In this style |\labelsep| denotes the separation between the
%    number and the text.
%    \begin{macrocode}
\gdef\th@marginbreak{\normalfont\slshape
  \def\@begintheorem##1##2{\item
       [\rlap{\vbox{\theorem@headerfont
                    \hbox{\llap{##2}\hskip\labelsep ##1}%
                    \hbox{\strut}}}]}%
\def\@opargbegintheorem##1##2##3{%
  \item[\rlap{\vbox{\theorem@headerfont
                    \hbox{\llap{##2}\hskip\labelsep ##1\ (##3)}%
                    \hbox{\strut}}}]}}
\endgroup
%</thmb>
%    \end{macrocode}
% \changes{v2.1c}{90/02/17}{`endgroup outside macrocode env.}
% \end{macro}
%
%
%
% \subsubsection{The \texttt{margin} style}
%
% This style option is stored in the file |thm.sty|.
%    \begin{macrocode}
%<*thm>
\begingroup \makeatletter
\@ifundefined{theorem@style}{\input{theorem.sty}}{}
\ProvidesFile{thm.sty}
             [\FMithmInfo]
%    \end{macrocode}
%
% \begin{macro}{\th@margin}
%    Again this is only a variant of the theorem styles described
%    above without any new ideas.
%    \begin{macrocode}
\gdef\th@margin{\normalfont\slshape
  \def\@begintheorem##1##2{\item
       [\theorem@headerfont \llap{##2}\hskip\labelsep ##1]}%
\def\@opargbegintheorem##1##2##3{%
  \item[\theorem@headerfont \llap{##2}\hskip\labelsep ##1\ (##3)]}}
\endgroup
%</thm>
%    \end{macrocode}
% \end{macro}
%
%
%
%
% \subsection{Final Definitions}
%
% \begin{macro}{\theorempreskipamount}
% \begin{macro}{\theorempostskipamount}
%    The \textsf{skip} parameters that regulate the vertical empty space
%    before and after the {\envfont theorem} environment have
%    to be allocated as well.
% \changes{v2.0f}{89/07/20}{Unnecessary `global' removed.}
%    \begin{macrocode}
%<*package>
\newskip\theorempreskipamount
\newskip\theorempostskipamount
%    \end{macrocode}
%    Since we have used the same values for all theorem sets, we now can
%    assign them.
%    \begin{macrocode}
\global\setlength\theorempreskipamount{12pt plus 5pt minus 3pt}
\global\setlength\theorempostskipamount{8pt plus 3pt minus 1.5pt}
%    \end{macrocode}
% \end{macro}
% \end{macro}
%
%
% \begin{macro}{\@endtheorem}
%    The same holds for the macro |\@endtheorem|, which ends a
%    {\envfont theorem} environment. Since it is the same for all
%    theorem sets, it is removed from the macros |\th@|\meta{style}.
%    It simply ends the {\envfont trivlist} environment, which was
%    begun in |\@thm|.
%    \begin{macrocode}
\global\let\@endtheorem=\endtrivlist
%    \end{macrocode}
% \end{macro}
%
%
%
% \begin{macro}{\@preamblecmds}
%    All macros defined above are to be used only in the preamble.
%    Therefore, we insert them in |\@preamblecmds| which will disable
%    them at begin document. This is done by the internal \LaTeXe{}
%    command |\@onlypreamble|.
% \changes{v2.0g}{89/09/19}{`theoremheaderfont and `theorembodyfont
%    added.}
% \changes{v2.2a}{94/02/03}{Use \LaTeXe{} interface}
%    \begin{macrocode}
\@onlypreamble\@xnthm
\@onlypreamble\@ynthm
\@onlypreamble\@othm
\@onlypreamble\newtheorem
\@onlypreamble\theoremstyle
\@onlypreamble\theorembodyfont
\@onlypreamble\theoremheaderfont
%    \end{macrocode}
% \end{macro}
%
% Finally we declare the \texttt{plain} theorem style to be the default.
% \changes{v2.1b}{90/06/04}{Fixed missing `item by declaring default
%    style.}
%    \begin{macrocode}
\theoremstyle{plain}
%</package>
%    \end{macrocode}
%
% \Finale

\endinput
}{}
\ProvidesFile{thp.sty}
             [\FMithmInfo]
%    \end{macrocode}
%
%
% \begin{macro}{\th@plain}
% \begin{macro}{\@begintheorem}
% \begin{macro}{\@opargbegintheorem}
%    |\theoremstyle{plain}| corresponds to the original
%    definition, except that the distances to the surrounding text are
%    determined by the parameters |\theorempreskipamount| and
%    |\theorempostskipamount|.
%    First we set the default body font.
%    \begin{macrocode}
\gdef\th@plain{\normalfont\itshape
%    \end{macrocode}
%    Then we define |\@begintheorem| and
%    |\@opargbegintheorem|. These two macros define how the
%    header of a theorem is typeset.  |\@opargbegintheorem| will
%    be called if a \textsf{theorem} environment with an optional
%    argument is encountered; otherwise, the header is constructed by
%    calling |\@begintheorem|.  If one of these macros is
%    executed, we are within a \textsf{trivlist} environment started by
%    |\@thm|.  So the theorem header is produced with an
%    |\item| command.
%
%    Instead of specifying the header font directly, all standard
%    theorem styles use the |\theorem@headerfont| macro to allow
%    customization.  The extra space (|\labelsep|) is necessary
%    because of problems in the \textsf{trivlist} environment.
%    \begin{macrocode}
  \def\@begintheorem##1##2{%
        \item[\hskip\labelsep \theorem@headerfont ##1\ ##2]}%
%    \end{macrocode}
%    The definition of |\@opargbegintheorem| is completely
%    analogous. The only difference is the fact that there exists a
%    third argument (which is the optional parameter of the
%    environment and contains additional information about the
%    theorem).  Customarily we enclose it in parentheses.
%    \begin{macrocode}
\def\@opargbegintheorem##1##2##3{%
   \item[\hskip\labelsep \theorem@headerfont ##1\ ##2\ (##3)]}}
%    \end{macrocode}
% \end{macro}
% \end{macro}
% \end{macro}
%
% \noindent
% We conclude with an |\endgroup| to restore the
% |\catcode| of |@|.
%    \begin{macrocode}
\endgroup
%</thp>
%    \end{macrocode}
%
%
% \subsubsection{The \texttt{break} style}
%
% This style option is stored in the file |thb.sty|.
% For the next two lines see the documentation for |\th@plain| on
% page~\pageref{dok:th@plain}.
%    \begin{macrocode}
%<*thb>
\begingroup \makeatletter
\@ifundefined{theorem@style}{% \iffalse meta-comment
%
% Copyright (C) 1993-2024
%
% The LaTeX Project and any individual authors listed elsewhere
% in this file.
%
% This file is part of the Standard LaTeX `Tools Bundle'.
% -------------------------------------------------------
%
% It may be distributed and/or modified under the
% conditions of the LaTeX Project Public License, either version 1.3c
% of this license or (at your option) any later version.
% The latest version of this license is in
%    https://www.latex-project.org/lppl.txt
% and version 1.3c or later is part of all distributions of LaTeX
% version 2005/12/01 or later.
%
% The list of all files belonging to the LaTeX `Tools Bundle' is
% given in the file `manifest.txt'.
%
% \fi
%
%%
%
%
\def\FMithmInfo{2023/07/05 v2.2c Theorem extension package (FMi)}
%
% \ProvidesFile{theorem.dtx}[\FMithmInfo]
%\iffalse   % this is a METACOMMENT !
%%
%% Package `theorem' to use with LaTeX2e
%% Copyright (C) 1989-1995 Frank Mittelbach, all rights reserved.
%<+package>\@ifundefined{theorem@style}{}{\endinput}
%<+package>\NeedsTeXFormat{LaTeX2e}[1995/06/01]
%<+package>\ProvidesPackage{theorem}
%<+package>         [\FMithmInfo]
%
% \fi
%
% \changes{v1.0}{87/05/07}{First draft.}
% \changes{v1.0a}{87/05/07}{`thistheoremstyle' added.}
% \changes{v1.1a}{87/07/29}{New concept: redefine the internal
%                          \protect\LaTeX{} commands `@xnthm' etc.}
% \changes{v1.1b}{87/07/31}{`theoremfont' used to specify font.}
% \changes{v1.2a}{88/01/08}{`@endtheorem' now globally defined.}
% \changes{v1.2b}{88/01/11}{`newtheorem' only allowed in the preamble.}
% \changes{v1.3a}{88/02/14}{`theoremfont' now a toks register.}
%
% \changes{v2.0a}{88/11/16}{New implementation and new documentation.}
% \changes{v2.0a}{88/11/16}{No layout changes but internal modifications
%                           old substyles won't work.}
% \changes{v2.0b}{89/01/30}{`global' used for all definitions.}
% \changes{v2.0c}{89/06/15}{Documentation translated to English.
%                            Thanks to LLL and Nina}
% \changes{v2.1a}{90/02/15}{Support for new font selection.}
% \changes{v2.2a}{94/02/03}{Cleanup for LaTeX2e}
%
% \DoNotIndex{\@endparenv,\@ifdefinable,\@ifnextchar,\@ifundefined}
% \DoNotIndex{\@namedef,\@nameuse,\@spaces,\@tempa,\@warning}
% \DoNotIndex{\ }
% \DoNotIndex{\begingroup,\bfseries}
% \DoNotIndex{\csname}
% \DoNotIndex{\def,\do,\docdate}
% \DoNotIndex{\endcsname,\endgroup,\endinput,\endtrivlist,\expandafter}
% \DoNotIndex{\filedate,\fileversion}
% \DoNotIndex{\gdef,\global}
% \DoNotIndex{\hbox,\hskip}
% \DoNotIndex{\ignorespaces,\input,\itshape,\item}
% \DoNotIndex{\labelsep,\let,\llap}
% \DoNotIndex{\makeatletter}
% \DoNotIndex{\noexpand}
% \DoNotIndex{\refstepcounter,\relax,\rlap}
% \DoNotIndex{\setlength,\slshape,\space,\string,\strut}
% \DoNotIndex{\the,\trivlist,\typeout}
% \DoNotIndex{\vbox}
% \DoNotIndex{\xdef}
%
% \def\deflabel#1{\ttfamily #1\hfill}
% \def\deflist#1{\begin{list}{}{\settowidth\labelwidth{\ttfamily #1}%
%                               \setlength\leftmargin\labelwidth
%                               \addtolength\leftmargin\labelsep
%                               \let\makelabel\deflabel}}
% \def\enddeflist{\end{list}}
% \def\envfont{\normalfont\ttfamily}
%
% \MakeShortVerb{\|}
%
% \newcommand{\TUB}{{\sl TUGboat\/}}
%
% \GetFileInfo{theorem.dtx}
% \title{An Extension of the
%        \LaTeX{} theorem environment\thanks{This file
%        has version number \fileversion, last
%        revised \filedate.}}
% \author{Frank Mittelbach}
%
% \MaintainedByLaTeXTeam{tools}
% \maketitle
%
% \begin{abstract}
%    The macros described in this paper yield an extension of the
%    \LaTeX{} theorem mechanism. It is designed to satisfy the
%    different requirements of various journals. Thus, the
%    layout of the ``theorems'' can be manipulated by determining a
%    ``style''. This article describes not only the use, but
%    also the definition, of the necessary macros.
% \end{abstract}
%
% \section*{Preface to version 2.2}
%
% For \LaTeXe{} this package did not need any fundamental changes. I
% only modified the messages generated so that theorem layout styles
% will show up with the |\listfiles| command and cleaned the section
% on the New Font Selection Scheme since this is now included in
% \LaTeX.
%
% \section*{Preface to version 2.1}
%
% This version is identical to 2.0g described in \TUB\ 10\#3 except
% for some internal defaults which are now set depending on the used
% font selection scheme.
%
% This was done to avoid unpleasant surprises if the new font selection
% scheme is in force. For further details see
% section~\ref{sec:newfonts} and \cite{bk:GMS94}.
%
% \section{Introduction}
%
% For our purposes here, ``theorems'' are labelled enunciations,
% often set off from the main text by extra space and a font change.
% Theorems, corollaries, conjectures, definitions, and remarks
% are all instances of ``theorems''.  The ``header'' of these structures
% is composed of a label (such as \textsc{Theorem} or \textsc{Remark})
% and a number which serializes an item in the sequence of items
% with the same label.
%
% Shortly after the introduction of \LaTeX{} at the Fachbereich
% Mathematik in Mainz, the desire to manipulate the layout of
% ``theorems'' arose.  In Mainz, the following two conventions came
% into general use:
% \begin{enumerate}
%   \item The number of the theorem is shown in the margin.
%   \item There is a line break at
%         the end of the theorem header.
% \end{enumerate}
%
% Additionally, some journals require different formats which depend on
% the ``sort of theorem'': e.g.\ often remarks and definitions are set
% in |\upshape|, while |\itshape| is employed for main theorems.
%
% Confronted with these requirements, a theorem environment was
% developed in Mainz which allows separate determination of the
% layout of the ``theorems sets'', comparable to |\pagestyle|.
%
%
% \section{The user interface}
% \subsection{Defining new theorem sets}
%
% \DescribeMacro\newtheorem
% As in the original \LaTeX{} version, the command |\newtheorem|
% defines a new ``theorem set'' or ``theorem-like structure''.
% Two required arguments name the new environment and give the text
% to be typeset with each instance of the new ``set'', while
% an optional argument determines how
% the ``set'' is enumerated:
% \begin{description}
%    \item[\ttfamily \bslash newtheorem\{foo\}\{bar\}]
%       The theorem set {\envfont foo} (whose name is \texttt{bar})
%       uses its own counter.
%    \item[\ttfamily \bslash newtheorem\{foo2\}{[foo]}\{bar2\}]
%       The theorem set {\envfont foo2} (printed name \texttt{bar2})
%       uses the same counter as the theorem set \texttt{foo}.
%    \item[\ttfamily \bslash newtheorem\{foo3\}\{bar3\}{[section]}]
%       The theorem set {\envfont foo3} (printed name \texttt{bar3}) is
%       enumerated within the counter \texttt{section}, i.e.\ with every
%       new |\section| the enumeration begins again with 1, and
%       the enumeration is composed from the section-number and the
%       theorem counter itself.
% \end{description}
%
% \DescribeMacro\theoremstyle
% Additionally, the command |\theoremstyle| can define the layout
% of various, or all, theorem sets.  It should be noted that any
% theorem set defined by |\newtheorem| is typeset in the
% |\theoremstyle| that is current at the time of the definition.
% Thus, the following
% \begin{verbatim}
%    \theoremstyle{break}        \newtheorem{Cor}{Corollary}
%    \theoremstyle{plain}        \newtheorem{Exa}{Example}[section]
%\end{verbatim}                             ^^A No blank at line start
% leads to the result that the set {\envfont Cor} is formatted in the
% style \texttt{break}, while the set {\envfont Exa} and all the
% following ones are formatted in the style \texttt{plain}, unless
% another |\theoremstyle| follows.  Since the definitions installed by
% |\newtheorem| are global, one also can limit |\theoremstyle| locally
% by grouping braces.
%
% \DescribeMacro\theorembodyfont
% The choice of the font for the theorem body is completely
% independent of the chosen |\theoremstyle|; this has proven to
% be very advantageous.  For example,
% \begin{verbatim}
%   {\theorembodyfont{\upshape}          \newtheorem{Rem}{Remark}}
%\end{verbatim}
% defines a theorem set {\envfont Rem}, which will be set in
% |\upshape| in the current layout (which in our example is
% \texttt{plain}).  As with |\theoremstyle|, the |\theorembodyfont|
% chosen is that current at the time of |\newtheorem|.  If
% |\theorembodyfont| is not specified or one defines
% |\theorembodyfont{}|, then the font used will be that defined by the
% |\theoremstyle|.
%
% \DescribeMacro\theoremheaderfont It is also possible to customize
% the font used for the theorem headers.  This is, however, a global
% declaration, and therefore there should be at most one
% |\theoremheaderfont| declaration in the preamble.\footnote{If it is
% actually necessary to have different header fonts, one has to define
% new theorem styles (substituting the desired font) or specify the
% information directly in the \texttt{\bslash newtheorem} declaration
% (the unclean variant).}
%
% \DescribeMacro\theorempreskipamount
% \DescribeMacro\theorempostskipamount Two additional parameters
% affect the vertical space around the theorem environments:
% |\theorempreskipamount| and |\theorempostskipamount| define,
% respectively, the spacing before and after such an environment.
% These parameters apply for all theorem sets and can be manipulated
% with the ordinary length macros.  They are rubber lengths
% (`\textsf{skips}'), and therefore can contain \texttt{plus} and
% \texttt{minus} parts.
%
% Since the definition of theorem sets should---most sensibly---be
% placed in the preamble, we only allow installation there.
% It is therefore possible to release the memory used here after
% |\begin{document}|, in order to make room for other
% applications.
%
%
% \subsection{Existing theorem styles}
%
% The following theorem styles exist to date:
% \begin{deflist}{marginbreak}
%    \item[plain]
%       This theorem style emulates the original \LaTeX{} definition,
%       except that additionally the parameters
%       |\theorem...skipamount| are used.
%    \item[break]
%       In this style, the theorem header is followed by a line
%       break.
%    \item[marginbreak]
%       The theorem number is set in the margin, and there is a line
%       break as in \texttt{break}.
%    \item[changebreak]
%       Like \texttt{break}, but with header number and text
%       interchanged.
%    \item[change]
%       Header number and text are interchanged, without a line break.
%    \item[margin]
%       The number is set in the left margin, without a line break.
% \end{deflist}
% All styles (except \texttt{plain}) select |\slshape| as the default
% |\theorembodyfont|.
%
%
% \subsection{Examples}
%
% Given the above theorem sets {\envfont Cor}, {\envfont Exa}
% and {\envfont Rem},
% suppose that the preamble also contains the declarations:
% \begin{verbatim}
%    \theoremstyle{marginbreak}   \newtheorem{Lem}[Cor]{Lemma}
%    \theoremstyle{change}
%    \theorembodyfont{\itshape}        \newtheorem{Def}[Cor]{Definition}
%
%    \theoremheaderfont{\scshape}
%\end{verbatim}
% Then the following are some typical examples of the typeset output
% resulting from their use.
%
% \begin{Cor}
%    This is a sentence typeset in the theorem environment {\envfont
%    Cor}.
% \end{Cor}
%
% \begin{Exa}
%    This is a sentence typeset in the theorem environment {\envfont
%    Exa}.
% \end{Exa}
%
% \begin{Rem}
%    This is a sentence typeset in the theorem environment {\envfont
%    Rem}.
% \end{Rem}
%
% \begin{Lem}[Ben User]
%    This is a sentence typeset in the theorem environment {\envfont
%    Lem}.
% \end{Lem}
%
% \begin{Def}[Very impressive Definition]
%    This is a sentence typeset in the theorem environment {\envfont
%    Def}.
% \end{Def}
%
% The last two examples show the effect of the optional argument to
% a theorem environment (it is the text typeset in parentheses).
%
%
% \section{Special Considerations}
% \label{sec:newfonts}
%
% Theoremheader and body are implemented as a unit. This means that
% the |\theoremheaderfont|  will inherit characteristics of the
% |\theorembodyfont| in \LaTeXe. Thus, if for example
% |\theorembodyfont| is |\itshape| and |\theoremheaderfont|
% is |\bfseries| the font selected for the header will have the
% characteristics `bold extended italic'. If this is not desired one
% should set the |\theoremheaderfont| to something like
% \begin{verbatim}
%      \theoremheaderfont{\normalfont\bfseries}
%\end{verbatim}
% i.e.\ supplying all necessary font information explicitly.
%
%
%
% \section{Acknowledgements}
%
% The publication of this set of macros was only possible with the
% help of Christina Busse (translating the manuscript into English),
% Joachim Pense (playing the r\^ole of typist), Chris Rowley
% (looking everything over) and many others providing useful
% suggestions.
%
% \MaybeStop{%
%    \begin{thebibliography}{9}
%       \bibitem{bk:GMS94} \textsc{M.~Goossens}, \textsc{F.~Mittelbach}
%         and \textsc{A.~Samarin}.
%         \newblock The \LaTeX{} Companion.
%         \newblock
%         Addison-Wesley, Reading, Massachusetts, 1994.
%       \bibitem{latex:source} \textsc{Lamport, Leslie.} \newblock
%                               \texttt{latex.tex}, version 2.09, date
%                               Feb. 1990.
%    \end{thebibliography}
%    \PrintIndex
%    \PrintChanges}
%
%
% \section{The documentation driver file}
%
% The next bit of code contains the documentation driver file for
% \TeX{}, i.e., the file that will produce the documentation you are
% currently reading. It will be extracted from this file by the
% \texttt{docstrip} program. Since it is the first code in the file
% one can alternatively process this file directly with \LaTeXe{} to
% obtain the documentation.
%
%    \begin{macrocode}
%<*driver>
\documentclass{ltxdoc}

\usepackage{theorem}

%<+driver>% The next few lines define theorem sets which are used
%<+driver>% in the example section of the documentation.

\theoremstyle{break}       \newtheorem{Cor}{Corollary}
\theoremstyle{plain}       \newtheorem{Exa}{Example}[section]
{\theorembodyfont{\upshape}\newtheorem{Rem}{Remark}}
\theoremstyle{marginbreak} \newtheorem{Lem}[Cor]{Lemma}
\theoremstyle{change}
\theorembodyfont{\itshape} \newtheorem{Def}[Cor]{Definition}

\theoremheaderfont{\scshape}

\RecordChanges

\begin{document}
   \DocInput{theorem.dtx}
\end{document}
%</driver>
%    \end{macrocode}
%
% \section{Definition of the Macros}
%
% If the file has been loaded before, we abort immediately. If not the
% package announces itself (this is actually done at the very top of
% the file---the way it is done isn't good style so don't copy it).
% \changes{v2.0e}{89/07/19}{Spaces removed from `typeout'.}
% \changes{v2.2b}{95/11/19}{Announce moved to top of file}
%    \begin{macrocode}
%<*package>
%\@ifundefined{theorem@style}{}{\endinput}
%\def\FMithmInfo{1995/11/19 v2.2b Theorem extension package (FMi)}
%\ProvidesPackage{theorem}[\FMithmInfo]
%    \end{macrocode}
%
%
% \subsection{Definition of theorem styles and fonts}
%
% All the definitions in this file are done globally to allow
% inputting this file inside a group.
%
% \begin{macro}{\theoremstyle}
%    Before a theorem style can be installed, the chosen style must be
%    known. For that reason, we must test to see that
%    |\th@|\meta{style} is known or, more precisely, that it is
%    different from |\relax|. If the style is not known then
%    |\th@plain| is used.
%    \begin{macrocode}
\gdef\theoremstyle#1{%
   \@ifundefined{th@#1}{\@warning
          {Unknown theoremstyle `#1'. Using `plain'}%
          \theorem@style{plain}}%
%    \end{macrocode}
%    We save the theorem style to be used in the token register
%    |\theorem@style|.
%    \begin{macrocode}
      {\theorem@style{#1}}%
%    \end{macrocode}
%    Now we ``evaluate'' the theorem style: this means, we call the
%    macro |\th@|\meta{style} which will activate the relevant
%    definitions which are contained in a separate file.  This is done
%    in a group to suppress changes to the current font.  This could
%    otherwise pose problems together with the new font selection
%    scheme\footnote{When I printed the original article using the new
%    font selection scheme I ended with a document with slanted
%    typefaces (text headings and all) simply because one of the
%    theorem styles used \texttt{\bslash sl} at toplevel.} if the
%    |\th@|\meta{style} is evaluated a second time.
% \changes{v2.1a}{90/02/15}{Group added around `th@..}
% \changes{v2.1a}{90/02/15}{Reseting of `..@begintheorem removed.}
%    \begin{macrocode}
      \begingroup
        \csname th@\the\theorem@style \endcsname
      \endgroup}
%    \end{macrocode}
% \end{macro}
%
%
% \begin{macro}{\@begintheorem}
% \changes{v2.1a}{90/02/15}{Relaxed at toplevel.}
% \begin{macro}{\@opargbegintheorem}
% \changes{v2.1a}{90/02/15}{Relaxed at toplevel.}
%    We reset |\@begintheorem| and |\@opargbegintheorem| to
%    |\relax| since these commands are no longer necessary at
%    toplevel. This will save a few tokens.
%    \begin{macrocode}
\global\let\@begintheorem\relax
\global\let\@opargbegintheorem\relax
%    \end{macrocode}
% \end{macro}
% \end{macro}
%
% \begin{macro}{\theorem@style}
%    Obviously the \textsf{token} register used above has to be
%    allocated. To assure the utmost compatibility with the
%    original \LaTeX{} definition, we set the default theorem style to
%    \texttt{plain}, which implements the usual \LaTeX{} convention.
% \changes{v2.0f}{89/07/20}{Unnecessary `global' removed.}
%    \begin{macrocode}
\newtoks\theorem@style
\global\theorem@style{plain}
%    \end{macrocode}
% \end{macro}
%
% \begin{macro}{\theorembodyfont}
% \begin{macro}{\theorem@bodyfont}
%    If the |\theorembodyfont| is set by the user then it should not
%    interact with the default font set in the theorem style. When the
%    new font selection is in force this may happen if, for example,
%    the default is |\itshape| and the new |\theorembodyfont| is
%    |\sffamily|. So we add a |\reset@font| command in front of the
%    user definition.
% \changes{v2.0f}{89/07/20}{Unnecessary `global' removed.}
% \changes{v2.0g}{89/09/19}{`theoremfont to `theorembodyfont.}
% \changes{v2.1c}{92/06/22}{`reset@font added and logic changed.}
% \changes{v2.1d}{92/10/28}{Reset to empty if arg is empty}
%    \begin{macrocode}
\gdef\theorembodyfont#1{%
%    \end{macrocode}
%    We check if the argument supplied is empty and if so put nothing
%    into the |\theorem@bodyfont| token register to allow for
%    |\theorembodyfont{}| as a mean of using the default of the
%    current |\theoremstyle|.
%    \begin{macrocode}
   \def\@tempa{#1}%
   \ifx\@tempa\@empty
    \theorem@bodyfont{}%
   \else
    \theorem@bodyfont{\reset@font#1}%
   \fi
   }
\newtoks\theorem@bodyfont
\global\theorem@bodyfont{}
%    \end{macrocode}
% \end{macro}
% \end{macro}
%
%
% \begin{macro}{\theoremheaderfont}
% \changes{v2.0g}{89/09/19}{Macro added.}
%    The font for the theorem headers is handled differently because
%    this definition applies to all theorem styles.
%    \begin{macrocode}
\gdef\theoremheaderfont#1{\gdef\theorem@headerfont{#1}%
%    \end{macrocode}
%    After using the macro once it is redefined to produce an error
%    message.
%    \begin{macrocode}
       \gdef\theoremheaderfont##1{%
        \typeout{\string\theoremheaderfont\space should be used
                 only once.}}}
%    \end{macrocode}
% \end{macro}
%
% \begin{macro}{\theorem@headerfont}
% \changes{v2.0g}{89/09/19}{Macro added.}
%    To set the |\theorem@headerfont| default we first test if
%    the new fontselection scheme is in force.
% \changes{v2.1a}{90/02/15}{Support for new font selection.}
%    \begin{macrocode}
\ifx\upshape\undefined
%    \end{macrocode}
%    If not we define it to expand into |\bfseries|. We don't use
%    |\let| just in case a following style option redefines this
%    macro.
%    \begin{macrocode}
\gdef\theorem@headerfont{\bfseries}
%    \end{macrocode}
%    Otherwise we reset the current shape before calling |\bfseries|.
%    \begin{macrocode}
\else \gdef\theorem@headerfont{\normalfont\bfseries}\fi
%    \end{macrocode}
% \end{macro}
%
% \begin{macro}{\th@plain}
% \begin{macro}{\th@break}
% \begin{macro}{\th@marginbreak}
% \begin{macro}{\th@changebreak}
% \begin{macro}{\th@change}
% \begin{macro}{\th@margin}
%    The different styles are defined in macros such as
%    |\th@plain|.  Since memory space is precious in
%    ``non-Big-versions'', we have to avoid offering too many unused
%    definitions.  Therefore we define these styles in separate files
%    that can be loaded on demand.  Thus the commands themselves only
%    load these files. We use |\@input@|, a \LaTeXe{} internal command
%    that ensures that the file will be listed with |\listfiles|.
% \changes{v2.2a}{94/02/02}{Use `@input@ to load theorem layout files}
%    \begin{macrocode}
\gdef\th@plain{\@input@{thp.sty}}
\gdef\th@break{\@input@{thb.sty}}
\gdef\th@marginbreak{\@input@{thmb.sty}}
\gdef\th@changebreak{\@input@{thcb.sty}}
\gdef\th@change{\@input@{thc.sty}}
\gdef\th@margin{\@input@{thm.sty}}
%    \end{macrocode}
%    This list will be expanded when new styles become
%    available.  For testing,  just append new theorem substyles
%    as document options.
% \end{macro}
% \end{macro}
% \end{macro}
% \end{macro}
% \end{macro}
% \end{macro}
%
%
%
% \subsection{Definition of a new theorem set}
%
% As already pointed out, a new theorem environment can be defined in
% three different ways:
% \begin{verbatim}
%    \newtheorem{Lem}{Lemma}
%    \newtheorem{Lem}{Lemma}[section]
%    \newtheorem{Lem}[Theorem]{Lemma}
%\end{verbatim}
% The function of the macro |\newtheorem| is to recognize these cases
% and then to branch into one of the three macros |\@ynthm|, |\@xnthm|
% or |\@othm|.  This mechanism is adopted unchanged from
% \cite{latex:source}; the essential point here is that, for example,
% in the second case, the arguments \texttt{Lem}, \texttt{Lemma} and
% \texttt{section} are passed over to the macro |\@xnthm|.
%
% We inspect this case first because the others present fewer problems,
% and thus are easily derived from this one.
%
% \begin{macro}{\@xnthm}
%    For our example arguments, the macro |\@xnthm| must fulfill
%    the following:
%    \begin{itemize}
%       \item
%          Define a new \LaTeX-counter `Lem'
%       \item
%          reset this counter within a |\section|
%       \item
%          define the macro |\theLem|
%       \item
%          define the environment macros |\Lem| and
%          |\endLem| using the current
%          |\theoremstyle| and |\theorem@bodyfont|.
%    \end{itemize}
%    Obviously, all this should happen only if the first argument
%    of |\@xnthm| (i.e.\ \texttt{Lem} in our example) is chosen so
%    as not to conflict with any previously defined commands or
%    environments.  This test is performed by the \LaTeX{} macro
%    |\@ifdefinable|.
%    \begin{macrocode}
\gdef\@xnthm#1#2[#3]{\expandafter\@ifdefinable\csname #1\endcsname
%    \end{macrocode}
%    Therefore, the first argument of |\@ifdefinable| is the
%    expansion (in the example, |\Lem|) of
%    |\csname||#1\endcsname|.  The second argument is
%    executed only if the test has been completed successfully.
%    \begin{macrocode}
   {%
%    \end{macrocode}
%    Now we define the new counter. The names of the \LaTeX{}
%    macros employed should speak for themselves:
% \changes{v2.2b}{1995/10/07}{Use \cs{@newctr}}
% \changes{v2.2c}{1995/11/23}{Use \cs{@newctr} properly.}
%    \begin{macrocode}
    \@definecounter{#1}\@newctr{#1}[#3]%
%    \end{macrocode}
%    Using |\@newctr| will give a proper error message if the counter
%    in |#3| is not defined.
%    In defining `|\theLem|' we must generate the desired macro
%    name by use of |\expandafter| and |\csname|.
%    \begin{macrocode}
    \expandafter\xdef\csname the#1\endcsname
%    \end{macrocode}
%    An |\xdef| is used in order to make the definition global,
%    and to ensure that it contains the replacement texts of
%    |\@thmcountersep| and |\@thmcounter|.\footnote{These
%    two macros can be defined by the document style.  Their default
%    values produce a `\texttt{.}' as separation and an arabic
%    representation of the number.} However, not everything should be
%    expanded. For example, it saves space to use |\thesection|
%    instead of its---at times---lengthy expansion.
%    \begin{macrocode}
      {\expandafter \noexpand \csname the#3\endcsname
       \@thmcountersep \@thmcounter{#1}}%
%    \end{macrocode}
%    Thus with the defaults of \LaTeX{}, |\theLem| would be
%    replaced by the command sequence |\thesection.\arabic{Lem}|.
%
%    We will now look at the definition of the macro which is executed
%    at the beginning of the actual environment (in our example this
%    macro is |\Lem|). It should be noted that we use an
%    ``|\expandafter| trick'' to expand only certain parts of the
%    replacement text at the time of the definition.
%    \begin{macrocode}
    \def\@tempa{\global\@namedef{#1}}%
    \expandafter \@tempa \expandafter{%
%    \end{macrocode}
%    First, the macro that contains the current definitions of
%    |\@begintheorem| and |\@opargtheorem| should be called
%    up.  The name of this macro---as is already known---has the form
%    |\th@|\meta{theorem\ style}; therefore, it must be called by
%    \begin{macrocode}
      \csname th@\the \theorem@style
%    \end{macrocode}
%    In addition the default theorem font should be changeable, i.e.\
%    we have to insert the contents of |\theorem@bodyfont|. For
%    that reason, we expand even further, beyond |\endcsname|,
%    and thus insert the contents of the \textsf{token} register
%    |\theorem@bodyfont| in the replacement text.
%    \begin{macrocode}
            \expandafter \endcsname \the \theorem@bodyfont
%    \end{macrocode}
%    Now it is time to call the macro |\@thm| which takes over
%    the further processing.  It has two arguments: the current
%    counter name (in our example, \texttt{Lem}), and the text of the
%    label (in our example, \texttt{Lemma}).
%    \begin{macrocode}
     \@thm{#1}{#2}}%
%    \end{macrocode}
%    With this, the `sub-definition' is complete.  The macro
%    |\@endtheorem| ends a theorem environment and is, so far, nothing
%    but an |\endtrivlist|.  (Hence it is defined globally, and not
%    within the theorem styles.\footnote{This has to be changed as
%    soon as theorem styles that change \texttt{\bslash @endtheorem}
%    exist.  In such a case, all existing styles must be changed as
%    well since they will have to reset the macro.}) Therefore, we can
%    set it equivalent to the macro that ends the theorem set (in our
%    example, |\endLem|).  However, if some day theorem styles exist
%    that do change |\@endtheorem|, we would have to use the
%    commented-out line instead.
%    \begin{macrocode}
    \global \expandafter \let \csname end#1\endcsname \@endtheorem
%   \global\@namedef{end#1}{\@endtheorem}%
%    \end{macrocode}
%    With these commands all the required definitions are employed,
%    unless the test |\@ifdefinable| has failed.  Therefore, we
%    end the second argument of this macro and with it the definition
%    of |\@xnthm|.
%    \begin{macrocode}
   }}
%    \end{macrocode}
% \end{macro}
%
% \begin{macro}{\@ynthm}
%    The definition of |\@ynthm| is completely analogous. In
%    this case the new counter that is defined is not reset within
%    another counter; thus the definition of |\the...|
%    is simplified:
%    \begin{macrocode}
\gdef\@ynthm#1#2{\expandafter\@ifdefinable\csname #1\endcsname
   {\@definecounter{#1}%
    \expandafter\xdef\csname the#1\endcsname{\@thmcounter{#1}}%
%    \end{macrocode}
%    The rest of the definition corresponds literally to that of
%    |\@xnthm|:
%    \begin{macrocode}
    \def\@tempa{\global\@namedef{#1}}\expandafter \@tempa
     \expandafter{\csname th@\the \theorem@style \expandafter
     \endcsname \the\theorem@bodyfont \@thm{#1}{#2}}%
    \global \expandafter \let \csname end#1\endcsname \@endtheorem}}
%    \end{macrocode}
% \end{macro}
%
% \begin{macro}{\@othm}
%    The definition of |\@othm| does not contain anything new.
%    \begin{macrocode}
\gdef\@othm#1[#2]#3{%
%    \end{macrocode}
% \changes{v2.2b}{95/10/07}{Build in check for counter defined}
%    We do not define a new counter but instead use one that has
%    already been defined.  Thus the only definition we need is that
%    of this pseudo-counter (i.e.\ |\the|\meta{env.\ name}).
%    First we check if |#2| corresponds to a known counter name.
%    \begin{macrocode}
  \expandafter\ifx\csname c@#2\endcsname\relax
   \@nocounterr{#2}%
  \else
   \expandafter\@ifdefinable\csname #1\endcsname
   {\expandafter \xdef \csname the#1\endcsname
     {\expandafter \noexpand \csname the#2\endcsname}%
%    \end{macrocode}
%    All other parts of the definition can be adopted from
%    |\@xnthm|. We have to remember, though, that in this case
%    the name of the current counter and the theorem label have moved
%    to the second and third arguments.
%    \begin{macrocode}
    \def\@tempa{\global\@namedef{#1}}\expandafter \@tempa
     \expandafter{\csname th@\the \theorem@style \expandafter
     \endcsname \the\theorem@bodyfont \@thm{#2}{#3}}%
    \global \expandafter \let \csname end#1\endcsname \@endtheorem}%
  \fi}
%    \end{macrocode}
% \end{macro}
%
% \subsection{Macros that are employed in a {\envfont theorem}
%    environment}
%
% \begin{macro}{\@thm}
%    The macro |\@thm| has to increase the current counter.  Then,
%    depending on whether the environment has (or does not have) an
%    optional argument, it has to branch into either |\@begintheorem|
%    or |\@opargtheorem|.
%    \begin{macrocode}
\gdef\@thm#1#2{\refstepcounter{#1}%
%    \end{macrocode}
%    Now we start a {\envfont trivlist} environment, and give
%    |\@topsep| and |\@topsepadd| the values of the \textsf{skip}
%    registers |\theorempreskipamount| and |\theorempostskipamount|.
%    The value in |\@topsep| is the vertical space that is inserted by
%    the first (and only) |\item| in our |\trivlist| whilst
%    |\@topsepadd| is inserted by |\@endparenv| at the end of that
%    {\envfont trivlist} environment.  By using these registers, we
%    obtain the desired space around a {\envfont theorem} environment.
%    \begin{macrocode}
   \trivlist
   \@topsep \theorempreskipamount               % used by first \item
   \@topsepadd \theorempostskipamount           % used by \@endparenv
%    \end{macrocode}
%    Now we have to test whether an optional argument has been given.
% \changes{v2.0d}{89/07/18}{Forgotten percent added.}
%    \begin{macrocode}
   \@ifnextchar [%
%    \end{macrocode}
%    If there is an optional argument, we will call |\@ythm|, and
%    move the arguments read back into the input stream.
%    \begin{macrocode}
   {\@ythm{#1}{#2}}%
%    \end{macrocode}
%    If not, we call |\@begintheorem|. Its first argument is the
%    name of the theorem set (hence the second argument of
%    |\@thm|). Its second argument is the macro that produces the
%    current number.
%    \begin{macrocode}
   {\@begintheorem{#2}{\csname the#1\endcsname}\ignorespaces}}
%    \end{macrocode}
% \end{macro}
%
% \begin{macro}{\@xthm}
% \begin{macro}{\@ythm}
%    Both these macros were originally called by |\@thm|.  We do
%    not need |\@xthm| anymore, hence we reset it to
%    |\relax|. The definition of |\@ythm| has not changed at
%    all from its definition in \LaTeX. In order to make the macros
%    easier to understand, we will
%    nevertheless present it (commented out).
%    \begin{macrocode}
\global\let\@xthm\relax
% \def\@ythm#1#2[#3]{\@opargbegintheorem{#2}{\csname
%           the#1\endcsname}{#3}\ignorespaces}
%</package>
%    \end{macrocode}
%    The primitive |\ignorespaces| in |\@ythm| and
%    |\@thm| is needed to remove the spaces between the
%    |\begin{...}| and the actual text.
% \end{macro}
% \end{macro}
%
%
% \subsection{Definition of the theorem substyles}
%
% As already pointed out, the theorem substyles, defined below, are
% only loaded when necessary.  Note that all these substyles, except
% \texttt{plain}, have |\slshape| as the default body font.
%
% \subsubsection{The \texttt{plain} style}
%
% \label{dok:th@plain} As the following macros use |@|, we have
% to locally set the |\catcode| of this symbol to ``letter''.
% This happens within a group, so that we do not have to worry about
% which |\catcode| that symbol had before.
%    \begin{macrocode}
%<*thp>
\begingroup \makeatletter
%    \end{macrocode}
% Since we are now within a group, we must make all definitions
% globally.  First we make sure that \texttt{theorem.sty} is loaded.
% This will allow us to use this file as a document style option
% without having to call \texttt{theorem} itself as an option.  At the
% same time, we assure that at least version 2 is loaded, since
% |\theorem@style| was not defined in earlier versions.
%    \begin{macrocode}
\@ifundefined{theorem@style}{\input{theorem.sty}}{}
\ProvidesFile{thp.sty}
             [\FMithmInfo]
%    \end{macrocode}
%
%
% \begin{macro}{\th@plain}
% \begin{macro}{\@begintheorem}
% \begin{macro}{\@opargbegintheorem}
%    |\theoremstyle{plain}| corresponds to the original
%    definition, except that the distances to the surrounding text are
%    determined by the parameters |\theorempreskipamount| and
%    |\theorempostskipamount|.
%    First we set the default body font.
%    \begin{macrocode}
\gdef\th@plain{\normalfont\itshape
%    \end{macrocode}
%    Then we define |\@begintheorem| and
%    |\@opargbegintheorem|. These two macros define how the
%    header of a theorem is typeset.  |\@opargbegintheorem| will
%    be called if a \textsf{theorem} environment with an optional
%    argument is encountered; otherwise, the header is constructed by
%    calling |\@begintheorem|.  If one of these macros is
%    executed, we are within a \textsf{trivlist} environment started by
%    |\@thm|.  So the theorem header is produced with an
%    |\item| command.
%
%    Instead of specifying the header font directly, all standard
%    theorem styles use the |\theorem@headerfont| macro to allow
%    customization.  The extra space (|\labelsep|) is necessary
%    because of problems in the \textsf{trivlist} environment.
%    \begin{macrocode}
  \def\@begintheorem##1##2{%
        \item[\hskip\labelsep \theorem@headerfont ##1\ ##2]}%
%    \end{macrocode}
%    The definition of |\@opargbegintheorem| is completely
%    analogous. The only difference is the fact that there exists a
%    third argument (which is the optional parameter of the
%    environment and contains additional information about the
%    theorem).  Customarily we enclose it in parentheses.
%    \begin{macrocode}
\def\@opargbegintheorem##1##2##3{%
   \item[\hskip\labelsep \theorem@headerfont ##1\ ##2\ (##3)]}}
%    \end{macrocode}
% \end{macro}
% \end{macro}
% \end{macro}
%
% \noindent
% We conclude with an |\endgroup| to restore the
% |\catcode| of |@|.
%    \begin{macrocode}
\endgroup
%</thp>
%    \end{macrocode}
%
%
% \subsubsection{The \texttt{break} style}
%
% This style option is stored in the file |thb.sty|.
% For the next two lines see the documentation for |\th@plain| on
% page~\pageref{dok:th@plain}.
%    \begin{macrocode}
%<*thb>
\begingroup \makeatletter
\@ifundefined{theorem@style}{\input{theorem.sty}}{}
\ProvidesFile{thb.sty}
             [\FMithmInfo]
%    \end{macrocode}
%
% \begin{macro}{\th@break}
%    |\theoremstyle{break}| produces a line break after the name
%    of the theorem.  The font is |\slshape|.  Hence, we define
%    |\th@break| as follows:
%    \begin{macrocode}
\gdef\th@break{\normalfont\slshape
  \def\@begintheorem##1##2{\item[%
%    \end{macrocode}
%    We run into the following problem: it is not possible to create
%    the header with |\item[|\meta{title}|]| and then start
%    a new line by, for example, |\mbox{}\\|. Such a definition
%    will fail whenever a list environment follows immediately.  With
%    the above construction, the |\mbox{}| causes the switch
%    |@inlabel| (cf.\ definition of |\list| and
%    |\trivlist| in \cite{latex:source}) to be set to \texttt{false}
%    and so the following list will insert additional vertical
%    space (|\topskip|). This is quite annoying. Therefore, we
%    create the line break within the |\item|. In order to ensure
%    that the text will begin at the proper position in the following
%    line, we simply pretend that the label does not take any
%    room.\footnote{This will lead to problems whenever very high
%    symbols occurring in the line tower into the
%    heading. So, something else has to be done here sometime.}
%    \begin{macrocode}
       \rlap{\vbox{\hbox{\hskip \labelsep\theorem@headerfont ##1\ ##2}%
                   \hbox{\strut}}}]}%
%    \end{macrocode}
%    Again, the definition of |\@opargbegintheorem| is
%    completely analogous.
%    \begin{macrocode}
\def\@opargbegintheorem##1##2##3{%
  \item[\rlap{\vbox{\hbox{\hskip \labelsep \theorem@headerfont
                     ##1\ ##2\ (##3)}%
                    \hbox{\strut}}}]}}
\endgroup
%</thb>
%    \end{macrocode}
% \end{macro}
%
%
% \subsubsection{The \texttt{changebreak} style}
%
%    \begin{macrocode}
% This style option is stored in the file |thcb.sty|.
%    \begin{macrocode}
%<*thcb>
\begingroup \makeatletter
\@ifundefined{theorem@style}{\input{theorem.sty}}{}
\ProvidesFile{thcb.sty}
             [\FMithmInfo]
%    \end{macrocode}
%
% \begin{macro}{\th@changebreak}
%    The \texttt{change-break} theorem style is like \texttt{break}
%    but with interchange of theorem name and theorem number.  Thus we
%    define |\th@changebreak| as follows:
%    \begin{macrocode}
\gdef\th@changebreak{\normalfont\slshape
  \def\@begintheorem##1##2{\item
       [\rlap{\vbox{\hbox{\hskip\labelsep \theorem@headerfont ##2\ ##1}%
                    \hbox{\strut}}}]}%
\def\@opargbegintheorem##1##2##3{%
  \item[\rlap{\vbox{\hbox{\hskip\labelsep \theorem@headerfont
                     ##2\ ##1\ (##3)}%
                    \hbox{\strut}}}]}}
\endgroup
%</thcb>
%    \end{macrocode}
% \end{macro}
%
%
% \subsubsection{The \texttt{change} style}
%
% This style option is stored in the file |thc.sty|.
%    \begin{macrocode}
%<*thc>
\begingroup \makeatletter
\@ifundefined{theorem@style}{\input{theorem.sty}}{}
\ProvidesFile{thc.sty}
             [\FMithmInfo]
%    \end{macrocode}
%
% \begin{macro}{\th@change}
%    The \texttt{change} theorem style corresponds to the
%    \texttt{change break} style without a linebreak after the header.
%    To say it in another way, it's the same as the \texttt{plain}
%    style but with number and name interchanged and |\slshape| as the
%    default font.
%    \begin{macrocode}
\gdef\th@change{\normalfont\slshape
  \def\@begintheorem##1##2{\item
       [\hskip\labelsep \theorem@headerfont ##2\ ##1]}%
\def\@opargbegintheorem##1##2##3{%
  \item[\hskip\labelsep \theorem@headerfont ##2\ ##1\ (##3)]}}
\endgroup
%</thc>
%    \end{macrocode}
% \end{macro}
%
%
% \subsubsection{The \texttt{marginbreak} style}
%
% This style option is the one used most often at Mainz.  It is saved
% in the file |thmb.sty|.
%    \begin{macrocode}
%<*thmb>
\begingroup \makeatletter
\@ifundefined{theorem@style}{\input{theorem.sty}}{}
\ProvidesFile{thmb.sty}
             [\FMithmInfo]
%    \end{macrocode}
%
% \begin{macro}{\th@marginbreak}
%    The \texttt{margin break} style is nearly the same as the
%    \texttt{change break} style. The only difference is the placement
%    of the theorem number.  We use |\llap| to place it in the left
%    margin.
%
%    In this style |\labelsep| denotes the separation between the
%    number and the text.
%    \begin{macrocode}
\gdef\th@marginbreak{\normalfont\slshape
  \def\@begintheorem##1##2{\item
       [\rlap{\vbox{\theorem@headerfont
                    \hbox{\llap{##2}\hskip\labelsep ##1}%
                    \hbox{\strut}}}]}%
\def\@opargbegintheorem##1##2##3{%
  \item[\rlap{\vbox{\theorem@headerfont
                    \hbox{\llap{##2}\hskip\labelsep ##1\ (##3)}%
                    \hbox{\strut}}}]}}
\endgroup
%</thmb>
%    \end{macrocode}
% \changes{v2.1c}{90/02/17}{`endgroup outside macrocode env.}
% \end{macro}
%
%
%
% \subsubsection{The \texttt{margin} style}
%
% This style option is stored in the file |thm.sty|.
%    \begin{macrocode}
%<*thm>
\begingroup \makeatletter
\@ifundefined{theorem@style}{\input{theorem.sty}}{}
\ProvidesFile{thm.sty}
             [\FMithmInfo]
%    \end{macrocode}
%
% \begin{macro}{\th@margin}
%    Again this is only a variant of the theorem styles described
%    above without any new ideas.
%    \begin{macrocode}
\gdef\th@margin{\normalfont\slshape
  \def\@begintheorem##1##2{\item
       [\theorem@headerfont \llap{##2}\hskip\labelsep ##1]}%
\def\@opargbegintheorem##1##2##3{%
  \item[\theorem@headerfont \llap{##2}\hskip\labelsep ##1\ (##3)]}}
\endgroup
%</thm>
%    \end{macrocode}
% \end{macro}
%
%
%
%
% \subsection{Final Definitions}
%
% \begin{macro}{\theorempreskipamount}
% \begin{macro}{\theorempostskipamount}
%    The \textsf{skip} parameters that regulate the vertical empty space
%    before and after the {\envfont theorem} environment have
%    to be allocated as well.
% \changes{v2.0f}{89/07/20}{Unnecessary `global' removed.}
%    \begin{macrocode}
%<*package>
\newskip\theorempreskipamount
\newskip\theorempostskipamount
%    \end{macrocode}
%    Since we have used the same values for all theorem sets, we now can
%    assign them.
%    \begin{macrocode}
\global\setlength\theorempreskipamount{12pt plus 5pt minus 3pt}
\global\setlength\theorempostskipamount{8pt plus 3pt minus 1.5pt}
%    \end{macrocode}
% \end{macro}
% \end{macro}
%
%
% \begin{macro}{\@endtheorem}
%    The same holds for the macro |\@endtheorem|, which ends a
%    {\envfont theorem} environment. Since it is the same for all
%    theorem sets, it is removed from the macros |\th@|\meta{style}.
%    It simply ends the {\envfont trivlist} environment, which was
%    begun in |\@thm|.
%    \begin{macrocode}
\global\let\@endtheorem=\endtrivlist
%    \end{macrocode}
% \end{macro}
%
%
%
% \begin{macro}{\@preamblecmds}
%    All macros defined above are to be used only in the preamble.
%    Therefore, we insert them in |\@preamblecmds| which will disable
%    them at begin document. This is done by the internal \LaTeXe{}
%    command |\@onlypreamble|.
% \changes{v2.0g}{89/09/19}{`theoremheaderfont and `theorembodyfont
%    added.}
% \changes{v2.2a}{94/02/03}{Use \LaTeXe{} interface}
%    \begin{macrocode}
\@onlypreamble\@xnthm
\@onlypreamble\@ynthm
\@onlypreamble\@othm
\@onlypreamble\newtheorem
\@onlypreamble\theoremstyle
\@onlypreamble\theorembodyfont
\@onlypreamble\theoremheaderfont
%    \end{macrocode}
% \end{macro}
%
% Finally we declare the \texttt{plain} theorem style to be the default.
% \changes{v2.1b}{90/06/04}{Fixed missing `item by declaring default
%    style.}
%    \begin{macrocode}
\theoremstyle{plain}
%</package>
%    \end{macrocode}
%
% \Finale

\endinput
}{}
\ProvidesFile{thb.sty}
             [\FMithmInfo]
%    \end{macrocode}
%
% \begin{macro}{\th@break}
%    |\theoremstyle{break}| produces a line break after the name
%    of the theorem.  The font is |\slshape|.  Hence, we define
%    |\th@break| as follows:
%    \begin{macrocode}
\gdef\th@break{\normalfont\slshape
  \def\@begintheorem##1##2{\item[%
%    \end{macrocode}
%    We run into the following problem: it is not possible to create
%    the header with |\item[|\meta{title}|]| and then start
%    a new line by, for example, |\mbox{}\\|. Such a definition
%    will fail whenever a list environment follows immediately.  With
%    the above construction, the |\mbox{}| causes the switch
%    |@inlabel| (cf.\ definition of |\list| and
%    |\trivlist| in \cite{latex:source}) to be set to \texttt{false}
%    and so the following list will insert additional vertical
%    space (|\topskip|). This is quite annoying. Therefore, we
%    create the line break within the |\item|. In order to ensure
%    that the text will begin at the proper position in the following
%    line, we simply pretend that the label does not take any
%    room.\footnote{This will lead to problems whenever very high
%    symbols occurring in the line tower into the
%    heading. So, something else has to be done here sometime.}
%    \begin{macrocode}
       \rlap{\vbox{\hbox{\hskip \labelsep\theorem@headerfont ##1\ ##2}%
                   \hbox{\strut}}}]}%
%    \end{macrocode}
%    Again, the definition of |\@opargbegintheorem| is
%    completely analogous.
%    \begin{macrocode}
\def\@opargbegintheorem##1##2##3{%
  \item[\rlap{\vbox{\hbox{\hskip \labelsep \theorem@headerfont
                     ##1\ ##2\ (##3)}%
                    \hbox{\strut}}}]}}
\endgroup
%</thb>
%    \end{macrocode}
% \end{macro}
%
%
% \subsubsection{The \texttt{changebreak} style}
%
%    \begin{macrocode}
% This style option is stored in the file |thcb.sty|.
%    \begin{macrocode}
%<*thcb>
\begingroup \makeatletter
\@ifundefined{theorem@style}{% \iffalse meta-comment
%
% Copyright (C) 1993-2024
%
% The LaTeX Project and any individual authors listed elsewhere
% in this file.
%
% This file is part of the Standard LaTeX `Tools Bundle'.
% -------------------------------------------------------
%
% It may be distributed and/or modified under the
% conditions of the LaTeX Project Public License, either version 1.3c
% of this license or (at your option) any later version.
% The latest version of this license is in
%    https://www.latex-project.org/lppl.txt
% and version 1.3c or later is part of all distributions of LaTeX
% version 2005/12/01 or later.
%
% The list of all files belonging to the LaTeX `Tools Bundle' is
% given in the file `manifest.txt'.
%
% \fi
%
%%
%
%
\def\FMithmInfo{2023/07/05 v2.2c Theorem extension package (FMi)}
%
% \ProvidesFile{theorem.dtx}[\FMithmInfo]
%\iffalse   % this is a METACOMMENT !
%%
%% Package `theorem' to use with LaTeX2e
%% Copyright (C) 1989-1995 Frank Mittelbach, all rights reserved.
%<+package>\@ifundefined{theorem@style}{}{\endinput}
%<+package>\NeedsTeXFormat{LaTeX2e}[1995/06/01]
%<+package>\ProvidesPackage{theorem}
%<+package>         [\FMithmInfo]
%
% \fi
%
% \changes{v1.0}{87/05/07}{First draft.}
% \changes{v1.0a}{87/05/07}{`thistheoremstyle' added.}
% \changes{v1.1a}{87/07/29}{New concept: redefine the internal
%                          \protect\LaTeX{} commands `@xnthm' etc.}
% \changes{v1.1b}{87/07/31}{`theoremfont' used to specify font.}
% \changes{v1.2a}{88/01/08}{`@endtheorem' now globally defined.}
% \changes{v1.2b}{88/01/11}{`newtheorem' only allowed in the preamble.}
% \changes{v1.3a}{88/02/14}{`theoremfont' now a toks register.}
%
% \changes{v2.0a}{88/11/16}{New implementation and new documentation.}
% \changes{v2.0a}{88/11/16}{No layout changes but internal modifications
%                           old substyles won't work.}
% \changes{v2.0b}{89/01/30}{`global' used for all definitions.}
% \changes{v2.0c}{89/06/15}{Documentation translated to English.
%                            Thanks to LLL and Nina}
% \changes{v2.1a}{90/02/15}{Support for new font selection.}
% \changes{v2.2a}{94/02/03}{Cleanup for LaTeX2e}
%
% \DoNotIndex{\@endparenv,\@ifdefinable,\@ifnextchar,\@ifundefined}
% \DoNotIndex{\@namedef,\@nameuse,\@spaces,\@tempa,\@warning}
% \DoNotIndex{\ }
% \DoNotIndex{\begingroup,\bfseries}
% \DoNotIndex{\csname}
% \DoNotIndex{\def,\do,\docdate}
% \DoNotIndex{\endcsname,\endgroup,\endinput,\endtrivlist,\expandafter}
% \DoNotIndex{\filedate,\fileversion}
% \DoNotIndex{\gdef,\global}
% \DoNotIndex{\hbox,\hskip}
% \DoNotIndex{\ignorespaces,\input,\itshape,\item}
% \DoNotIndex{\labelsep,\let,\llap}
% \DoNotIndex{\makeatletter}
% \DoNotIndex{\noexpand}
% \DoNotIndex{\refstepcounter,\relax,\rlap}
% \DoNotIndex{\setlength,\slshape,\space,\string,\strut}
% \DoNotIndex{\the,\trivlist,\typeout}
% \DoNotIndex{\vbox}
% \DoNotIndex{\xdef}
%
% \def\deflabel#1{\ttfamily #1\hfill}
% \def\deflist#1{\begin{list}{}{\settowidth\labelwidth{\ttfamily #1}%
%                               \setlength\leftmargin\labelwidth
%                               \addtolength\leftmargin\labelsep
%                               \let\makelabel\deflabel}}
% \def\enddeflist{\end{list}}
% \def\envfont{\normalfont\ttfamily}
%
% \MakeShortVerb{\|}
%
% \newcommand{\TUB}{{\sl TUGboat\/}}
%
% \GetFileInfo{theorem.dtx}
% \title{An Extension of the
%        \LaTeX{} theorem environment\thanks{This file
%        has version number \fileversion, last
%        revised \filedate.}}
% \author{Frank Mittelbach}
%
% \MaintainedByLaTeXTeam{tools}
% \maketitle
%
% \begin{abstract}
%    The macros described in this paper yield an extension of the
%    \LaTeX{} theorem mechanism. It is designed to satisfy the
%    different requirements of various journals. Thus, the
%    layout of the ``theorems'' can be manipulated by determining a
%    ``style''. This article describes not only the use, but
%    also the definition, of the necessary macros.
% \end{abstract}
%
% \section*{Preface to version 2.2}
%
% For \LaTeXe{} this package did not need any fundamental changes. I
% only modified the messages generated so that theorem layout styles
% will show up with the |\listfiles| command and cleaned the section
% on the New Font Selection Scheme since this is now included in
% \LaTeX.
%
% \section*{Preface to version 2.1}
%
% This version is identical to 2.0g described in \TUB\ 10\#3 except
% for some internal defaults which are now set depending on the used
% font selection scheme.
%
% This was done to avoid unpleasant surprises if the new font selection
% scheme is in force. For further details see
% section~\ref{sec:newfonts} and \cite{bk:GMS94}.
%
% \section{Introduction}
%
% For our purposes here, ``theorems'' are labelled enunciations,
% often set off from the main text by extra space and a font change.
% Theorems, corollaries, conjectures, definitions, and remarks
% are all instances of ``theorems''.  The ``header'' of these structures
% is composed of a label (such as \textsc{Theorem} or \textsc{Remark})
% and a number which serializes an item in the sequence of items
% with the same label.
%
% Shortly after the introduction of \LaTeX{} at the Fachbereich
% Mathematik in Mainz, the desire to manipulate the layout of
% ``theorems'' arose.  In Mainz, the following two conventions came
% into general use:
% \begin{enumerate}
%   \item The number of the theorem is shown in the margin.
%   \item There is a line break at
%         the end of the theorem header.
% \end{enumerate}
%
% Additionally, some journals require different formats which depend on
% the ``sort of theorem'': e.g.\ often remarks and definitions are set
% in |\upshape|, while |\itshape| is employed for main theorems.
%
% Confronted with these requirements, a theorem environment was
% developed in Mainz which allows separate determination of the
% layout of the ``theorems sets'', comparable to |\pagestyle|.
%
%
% \section{The user interface}
% \subsection{Defining new theorem sets}
%
% \DescribeMacro\newtheorem
% As in the original \LaTeX{} version, the command |\newtheorem|
% defines a new ``theorem set'' or ``theorem-like structure''.
% Two required arguments name the new environment and give the text
% to be typeset with each instance of the new ``set'', while
% an optional argument determines how
% the ``set'' is enumerated:
% \begin{description}
%    \item[\ttfamily \bslash newtheorem\{foo\}\{bar\}]
%       The theorem set {\envfont foo} (whose name is \texttt{bar})
%       uses its own counter.
%    \item[\ttfamily \bslash newtheorem\{foo2\}{[foo]}\{bar2\}]
%       The theorem set {\envfont foo2} (printed name \texttt{bar2})
%       uses the same counter as the theorem set \texttt{foo}.
%    \item[\ttfamily \bslash newtheorem\{foo3\}\{bar3\}{[section]}]
%       The theorem set {\envfont foo3} (printed name \texttt{bar3}) is
%       enumerated within the counter \texttt{section}, i.e.\ with every
%       new |\section| the enumeration begins again with 1, and
%       the enumeration is composed from the section-number and the
%       theorem counter itself.
% \end{description}
%
% \DescribeMacro\theoremstyle
% Additionally, the command |\theoremstyle| can define the layout
% of various, or all, theorem sets.  It should be noted that any
% theorem set defined by |\newtheorem| is typeset in the
% |\theoremstyle| that is current at the time of the definition.
% Thus, the following
% \begin{verbatim}
%    \theoremstyle{break}        \newtheorem{Cor}{Corollary}
%    \theoremstyle{plain}        \newtheorem{Exa}{Example}[section]
%\end{verbatim}                             ^^A No blank at line start
% leads to the result that the set {\envfont Cor} is formatted in the
% style \texttt{break}, while the set {\envfont Exa} and all the
% following ones are formatted in the style \texttt{plain}, unless
% another |\theoremstyle| follows.  Since the definitions installed by
% |\newtheorem| are global, one also can limit |\theoremstyle| locally
% by grouping braces.
%
% \DescribeMacro\theorembodyfont
% The choice of the font for the theorem body is completely
% independent of the chosen |\theoremstyle|; this has proven to
% be very advantageous.  For example,
% \begin{verbatim}
%   {\theorembodyfont{\upshape}          \newtheorem{Rem}{Remark}}
%\end{verbatim}
% defines a theorem set {\envfont Rem}, which will be set in
% |\upshape| in the current layout (which in our example is
% \texttt{plain}).  As with |\theoremstyle|, the |\theorembodyfont|
% chosen is that current at the time of |\newtheorem|.  If
% |\theorembodyfont| is not specified or one defines
% |\theorembodyfont{}|, then the font used will be that defined by the
% |\theoremstyle|.
%
% \DescribeMacro\theoremheaderfont It is also possible to customize
% the font used for the theorem headers.  This is, however, a global
% declaration, and therefore there should be at most one
% |\theoremheaderfont| declaration in the preamble.\footnote{If it is
% actually necessary to have different header fonts, one has to define
% new theorem styles (substituting the desired font) or specify the
% information directly in the \texttt{\bslash newtheorem} declaration
% (the unclean variant).}
%
% \DescribeMacro\theorempreskipamount
% \DescribeMacro\theorempostskipamount Two additional parameters
% affect the vertical space around the theorem environments:
% |\theorempreskipamount| and |\theorempostskipamount| define,
% respectively, the spacing before and after such an environment.
% These parameters apply for all theorem sets and can be manipulated
% with the ordinary length macros.  They are rubber lengths
% (`\textsf{skips}'), and therefore can contain \texttt{plus} and
% \texttt{minus} parts.
%
% Since the definition of theorem sets should---most sensibly---be
% placed in the preamble, we only allow installation there.
% It is therefore possible to release the memory used here after
% |\begin{document}|, in order to make room for other
% applications.
%
%
% \subsection{Existing theorem styles}
%
% The following theorem styles exist to date:
% \begin{deflist}{marginbreak}
%    \item[plain]
%       This theorem style emulates the original \LaTeX{} definition,
%       except that additionally the parameters
%       |\theorem...skipamount| are used.
%    \item[break]
%       In this style, the theorem header is followed by a line
%       break.
%    \item[marginbreak]
%       The theorem number is set in the margin, and there is a line
%       break as in \texttt{break}.
%    \item[changebreak]
%       Like \texttt{break}, but with header number and text
%       interchanged.
%    \item[change]
%       Header number and text are interchanged, without a line break.
%    \item[margin]
%       The number is set in the left margin, without a line break.
% \end{deflist}
% All styles (except \texttt{plain}) select |\slshape| as the default
% |\theorembodyfont|.
%
%
% \subsection{Examples}
%
% Given the above theorem sets {\envfont Cor}, {\envfont Exa}
% and {\envfont Rem},
% suppose that the preamble also contains the declarations:
% \begin{verbatim}
%    \theoremstyle{marginbreak}   \newtheorem{Lem}[Cor]{Lemma}
%    \theoremstyle{change}
%    \theorembodyfont{\itshape}        \newtheorem{Def}[Cor]{Definition}
%
%    \theoremheaderfont{\scshape}
%\end{verbatim}
% Then the following are some typical examples of the typeset output
% resulting from their use.
%
% \begin{Cor}
%    This is a sentence typeset in the theorem environment {\envfont
%    Cor}.
% \end{Cor}
%
% \begin{Exa}
%    This is a sentence typeset in the theorem environment {\envfont
%    Exa}.
% \end{Exa}
%
% \begin{Rem}
%    This is a sentence typeset in the theorem environment {\envfont
%    Rem}.
% \end{Rem}
%
% \begin{Lem}[Ben User]
%    This is a sentence typeset in the theorem environment {\envfont
%    Lem}.
% \end{Lem}
%
% \begin{Def}[Very impressive Definition]
%    This is a sentence typeset in the theorem environment {\envfont
%    Def}.
% \end{Def}
%
% The last two examples show the effect of the optional argument to
% a theorem environment (it is the text typeset in parentheses).
%
%
% \section{Special Considerations}
% \label{sec:newfonts}
%
% Theoremheader and body are implemented as a unit. This means that
% the |\theoremheaderfont|  will inherit characteristics of the
% |\theorembodyfont| in \LaTeXe. Thus, if for example
% |\theorembodyfont| is |\itshape| and |\theoremheaderfont|
% is |\bfseries| the font selected for the header will have the
% characteristics `bold extended italic'. If this is not desired one
% should set the |\theoremheaderfont| to something like
% \begin{verbatim}
%      \theoremheaderfont{\normalfont\bfseries}
%\end{verbatim}
% i.e.\ supplying all necessary font information explicitly.
%
%
%
% \section{Acknowledgements}
%
% The publication of this set of macros was only possible with the
% help of Christina Busse (translating the manuscript into English),
% Joachim Pense (playing the r\^ole of typist), Chris Rowley
% (looking everything over) and many others providing useful
% suggestions.
%
% \MaybeStop{%
%    \begin{thebibliography}{9}
%       \bibitem{bk:GMS94} \textsc{M.~Goossens}, \textsc{F.~Mittelbach}
%         and \textsc{A.~Samarin}.
%         \newblock The \LaTeX{} Companion.
%         \newblock
%         Addison-Wesley, Reading, Massachusetts, 1994.
%       \bibitem{latex:source} \textsc{Lamport, Leslie.} \newblock
%                               \texttt{latex.tex}, version 2.09, date
%                               Feb. 1990.
%    \end{thebibliography}
%    \PrintIndex
%    \PrintChanges}
%
%
% \section{The documentation driver file}
%
% The next bit of code contains the documentation driver file for
% \TeX{}, i.e., the file that will produce the documentation you are
% currently reading. It will be extracted from this file by the
% \texttt{docstrip} program. Since it is the first code in the file
% one can alternatively process this file directly with \LaTeXe{} to
% obtain the documentation.
%
%    \begin{macrocode}
%<*driver>
\documentclass{ltxdoc}

\usepackage{theorem}

%<+driver>% The next few lines define theorem sets which are used
%<+driver>% in the example section of the documentation.

\theoremstyle{break}       \newtheorem{Cor}{Corollary}
\theoremstyle{plain}       \newtheorem{Exa}{Example}[section]
{\theorembodyfont{\upshape}\newtheorem{Rem}{Remark}}
\theoremstyle{marginbreak} \newtheorem{Lem}[Cor]{Lemma}
\theoremstyle{change}
\theorembodyfont{\itshape} \newtheorem{Def}[Cor]{Definition}

\theoremheaderfont{\scshape}

\RecordChanges

\begin{document}
   \DocInput{theorem.dtx}
\end{document}
%</driver>
%    \end{macrocode}
%
% \section{Definition of the Macros}
%
% If the file has been loaded before, we abort immediately. If not the
% package announces itself (this is actually done at the very top of
% the file---the way it is done isn't good style so don't copy it).
% \changes{v2.0e}{89/07/19}{Spaces removed from `typeout'.}
% \changes{v2.2b}{95/11/19}{Announce moved to top of file}
%    \begin{macrocode}
%<*package>
%\@ifundefined{theorem@style}{}{\endinput}
%\def\FMithmInfo{1995/11/19 v2.2b Theorem extension package (FMi)}
%\ProvidesPackage{theorem}[\FMithmInfo]
%    \end{macrocode}
%
%
% \subsection{Definition of theorem styles and fonts}
%
% All the definitions in this file are done globally to allow
% inputting this file inside a group.
%
% \begin{macro}{\theoremstyle}
%    Before a theorem style can be installed, the chosen style must be
%    known. For that reason, we must test to see that
%    |\th@|\meta{style} is known or, more precisely, that it is
%    different from |\relax|. If the style is not known then
%    |\th@plain| is used.
%    \begin{macrocode}
\gdef\theoremstyle#1{%
   \@ifundefined{th@#1}{\@warning
          {Unknown theoremstyle `#1'. Using `plain'}%
          \theorem@style{plain}}%
%    \end{macrocode}
%    We save the theorem style to be used in the token register
%    |\theorem@style|.
%    \begin{macrocode}
      {\theorem@style{#1}}%
%    \end{macrocode}
%    Now we ``evaluate'' the theorem style: this means, we call the
%    macro |\th@|\meta{style} which will activate the relevant
%    definitions which are contained in a separate file.  This is done
%    in a group to suppress changes to the current font.  This could
%    otherwise pose problems together with the new font selection
%    scheme\footnote{When I printed the original article using the new
%    font selection scheme I ended with a document with slanted
%    typefaces (text headings and all) simply because one of the
%    theorem styles used \texttt{\bslash sl} at toplevel.} if the
%    |\th@|\meta{style} is evaluated a second time.
% \changes{v2.1a}{90/02/15}{Group added around `th@..}
% \changes{v2.1a}{90/02/15}{Reseting of `..@begintheorem removed.}
%    \begin{macrocode}
      \begingroup
        \csname th@\the\theorem@style \endcsname
      \endgroup}
%    \end{macrocode}
% \end{macro}
%
%
% \begin{macro}{\@begintheorem}
% \changes{v2.1a}{90/02/15}{Relaxed at toplevel.}
% \begin{macro}{\@opargbegintheorem}
% \changes{v2.1a}{90/02/15}{Relaxed at toplevel.}
%    We reset |\@begintheorem| and |\@opargbegintheorem| to
%    |\relax| since these commands are no longer necessary at
%    toplevel. This will save a few tokens.
%    \begin{macrocode}
\global\let\@begintheorem\relax
\global\let\@opargbegintheorem\relax
%    \end{macrocode}
% \end{macro}
% \end{macro}
%
% \begin{macro}{\theorem@style}
%    Obviously the \textsf{token} register used above has to be
%    allocated. To assure the utmost compatibility with the
%    original \LaTeX{} definition, we set the default theorem style to
%    \texttt{plain}, which implements the usual \LaTeX{} convention.
% \changes{v2.0f}{89/07/20}{Unnecessary `global' removed.}
%    \begin{macrocode}
\newtoks\theorem@style
\global\theorem@style{plain}
%    \end{macrocode}
% \end{macro}
%
% \begin{macro}{\theorembodyfont}
% \begin{macro}{\theorem@bodyfont}
%    If the |\theorembodyfont| is set by the user then it should not
%    interact with the default font set in the theorem style. When the
%    new font selection is in force this may happen if, for example,
%    the default is |\itshape| and the new |\theorembodyfont| is
%    |\sffamily|. So we add a |\reset@font| command in front of the
%    user definition.
% \changes{v2.0f}{89/07/20}{Unnecessary `global' removed.}
% \changes{v2.0g}{89/09/19}{`theoremfont to `theorembodyfont.}
% \changes{v2.1c}{92/06/22}{`reset@font added and logic changed.}
% \changes{v2.1d}{92/10/28}{Reset to empty if arg is empty}
%    \begin{macrocode}
\gdef\theorembodyfont#1{%
%    \end{macrocode}
%    We check if the argument supplied is empty and if so put nothing
%    into the |\theorem@bodyfont| token register to allow for
%    |\theorembodyfont{}| as a mean of using the default of the
%    current |\theoremstyle|.
%    \begin{macrocode}
   \def\@tempa{#1}%
   \ifx\@tempa\@empty
    \theorem@bodyfont{}%
   \else
    \theorem@bodyfont{\reset@font#1}%
   \fi
   }
\newtoks\theorem@bodyfont
\global\theorem@bodyfont{}
%    \end{macrocode}
% \end{macro}
% \end{macro}
%
%
% \begin{macro}{\theoremheaderfont}
% \changes{v2.0g}{89/09/19}{Macro added.}
%    The font for the theorem headers is handled differently because
%    this definition applies to all theorem styles.
%    \begin{macrocode}
\gdef\theoremheaderfont#1{\gdef\theorem@headerfont{#1}%
%    \end{macrocode}
%    After using the macro once it is redefined to produce an error
%    message.
%    \begin{macrocode}
       \gdef\theoremheaderfont##1{%
        \typeout{\string\theoremheaderfont\space should be used
                 only once.}}}
%    \end{macrocode}
% \end{macro}
%
% \begin{macro}{\theorem@headerfont}
% \changes{v2.0g}{89/09/19}{Macro added.}
%    To set the |\theorem@headerfont| default we first test if
%    the new fontselection scheme is in force.
% \changes{v2.1a}{90/02/15}{Support for new font selection.}
%    \begin{macrocode}
\ifx\upshape\undefined
%    \end{macrocode}
%    If not we define it to expand into |\bfseries|. We don't use
%    |\let| just in case a following style option redefines this
%    macro.
%    \begin{macrocode}
\gdef\theorem@headerfont{\bfseries}
%    \end{macrocode}
%    Otherwise we reset the current shape before calling |\bfseries|.
%    \begin{macrocode}
\else \gdef\theorem@headerfont{\normalfont\bfseries}\fi
%    \end{macrocode}
% \end{macro}
%
% \begin{macro}{\th@plain}
% \begin{macro}{\th@break}
% \begin{macro}{\th@marginbreak}
% \begin{macro}{\th@changebreak}
% \begin{macro}{\th@change}
% \begin{macro}{\th@margin}
%    The different styles are defined in macros such as
%    |\th@plain|.  Since memory space is precious in
%    ``non-Big-versions'', we have to avoid offering too many unused
%    definitions.  Therefore we define these styles in separate files
%    that can be loaded on demand.  Thus the commands themselves only
%    load these files. We use |\@input@|, a \LaTeXe{} internal command
%    that ensures that the file will be listed with |\listfiles|.
% \changes{v2.2a}{94/02/02}{Use `@input@ to load theorem layout files}
%    \begin{macrocode}
\gdef\th@plain{\@input@{thp.sty}}
\gdef\th@break{\@input@{thb.sty}}
\gdef\th@marginbreak{\@input@{thmb.sty}}
\gdef\th@changebreak{\@input@{thcb.sty}}
\gdef\th@change{\@input@{thc.sty}}
\gdef\th@margin{\@input@{thm.sty}}
%    \end{macrocode}
%    This list will be expanded when new styles become
%    available.  For testing,  just append new theorem substyles
%    as document options.
% \end{macro}
% \end{macro}
% \end{macro}
% \end{macro}
% \end{macro}
% \end{macro}
%
%
%
% \subsection{Definition of a new theorem set}
%
% As already pointed out, a new theorem environment can be defined in
% three different ways:
% \begin{verbatim}
%    \newtheorem{Lem}{Lemma}
%    \newtheorem{Lem}{Lemma}[section]
%    \newtheorem{Lem}[Theorem]{Lemma}
%\end{verbatim}
% The function of the macro |\newtheorem| is to recognize these cases
% and then to branch into one of the three macros |\@ynthm|, |\@xnthm|
% or |\@othm|.  This mechanism is adopted unchanged from
% \cite{latex:source}; the essential point here is that, for example,
% in the second case, the arguments \texttt{Lem}, \texttt{Lemma} and
% \texttt{section} are passed over to the macro |\@xnthm|.
%
% We inspect this case first because the others present fewer problems,
% and thus are easily derived from this one.
%
% \begin{macro}{\@xnthm}
%    For our example arguments, the macro |\@xnthm| must fulfill
%    the following:
%    \begin{itemize}
%       \item
%          Define a new \LaTeX-counter `Lem'
%       \item
%          reset this counter within a |\section|
%       \item
%          define the macro |\theLem|
%       \item
%          define the environment macros |\Lem| and
%          |\endLem| using the current
%          |\theoremstyle| and |\theorem@bodyfont|.
%    \end{itemize}
%    Obviously, all this should happen only if the first argument
%    of |\@xnthm| (i.e.\ \texttt{Lem} in our example) is chosen so
%    as not to conflict with any previously defined commands or
%    environments.  This test is performed by the \LaTeX{} macro
%    |\@ifdefinable|.
%    \begin{macrocode}
\gdef\@xnthm#1#2[#3]{\expandafter\@ifdefinable\csname #1\endcsname
%    \end{macrocode}
%    Therefore, the first argument of |\@ifdefinable| is the
%    expansion (in the example, |\Lem|) of
%    |\csname||#1\endcsname|.  The second argument is
%    executed only if the test has been completed successfully.
%    \begin{macrocode}
   {%
%    \end{macrocode}
%    Now we define the new counter. The names of the \LaTeX{}
%    macros employed should speak for themselves:
% \changes{v2.2b}{1995/10/07}{Use \cs{@newctr}}
% \changes{v2.2c}{1995/11/23}{Use \cs{@newctr} properly.}
%    \begin{macrocode}
    \@definecounter{#1}\@newctr{#1}[#3]%
%    \end{macrocode}
%    Using |\@newctr| will give a proper error message if the counter
%    in |#3| is not defined.
%    In defining `|\theLem|' we must generate the desired macro
%    name by use of |\expandafter| and |\csname|.
%    \begin{macrocode}
    \expandafter\xdef\csname the#1\endcsname
%    \end{macrocode}
%    An |\xdef| is used in order to make the definition global,
%    and to ensure that it contains the replacement texts of
%    |\@thmcountersep| and |\@thmcounter|.\footnote{These
%    two macros can be defined by the document style.  Their default
%    values produce a `\texttt{.}' as separation and an arabic
%    representation of the number.} However, not everything should be
%    expanded. For example, it saves space to use |\thesection|
%    instead of its---at times---lengthy expansion.
%    \begin{macrocode}
      {\expandafter \noexpand \csname the#3\endcsname
       \@thmcountersep \@thmcounter{#1}}%
%    \end{macrocode}
%    Thus with the defaults of \LaTeX{}, |\theLem| would be
%    replaced by the command sequence |\thesection.\arabic{Lem}|.
%
%    We will now look at the definition of the macro which is executed
%    at the beginning of the actual environment (in our example this
%    macro is |\Lem|). It should be noted that we use an
%    ``|\expandafter| trick'' to expand only certain parts of the
%    replacement text at the time of the definition.
%    \begin{macrocode}
    \def\@tempa{\global\@namedef{#1}}%
    \expandafter \@tempa \expandafter{%
%    \end{macrocode}
%    First, the macro that contains the current definitions of
%    |\@begintheorem| and |\@opargtheorem| should be called
%    up.  The name of this macro---as is already known---has the form
%    |\th@|\meta{theorem\ style}; therefore, it must be called by
%    \begin{macrocode}
      \csname th@\the \theorem@style
%    \end{macrocode}
%    In addition the default theorem font should be changeable, i.e.\
%    we have to insert the contents of |\theorem@bodyfont|. For
%    that reason, we expand even further, beyond |\endcsname|,
%    and thus insert the contents of the \textsf{token} register
%    |\theorem@bodyfont| in the replacement text.
%    \begin{macrocode}
            \expandafter \endcsname \the \theorem@bodyfont
%    \end{macrocode}
%    Now it is time to call the macro |\@thm| which takes over
%    the further processing.  It has two arguments: the current
%    counter name (in our example, \texttt{Lem}), and the text of the
%    label (in our example, \texttt{Lemma}).
%    \begin{macrocode}
     \@thm{#1}{#2}}%
%    \end{macrocode}
%    With this, the `sub-definition' is complete.  The macro
%    |\@endtheorem| ends a theorem environment and is, so far, nothing
%    but an |\endtrivlist|.  (Hence it is defined globally, and not
%    within the theorem styles.\footnote{This has to be changed as
%    soon as theorem styles that change \texttt{\bslash @endtheorem}
%    exist.  In such a case, all existing styles must be changed as
%    well since they will have to reset the macro.}) Therefore, we can
%    set it equivalent to the macro that ends the theorem set (in our
%    example, |\endLem|).  However, if some day theorem styles exist
%    that do change |\@endtheorem|, we would have to use the
%    commented-out line instead.
%    \begin{macrocode}
    \global \expandafter \let \csname end#1\endcsname \@endtheorem
%   \global\@namedef{end#1}{\@endtheorem}%
%    \end{macrocode}
%    With these commands all the required definitions are employed,
%    unless the test |\@ifdefinable| has failed.  Therefore, we
%    end the second argument of this macro and with it the definition
%    of |\@xnthm|.
%    \begin{macrocode}
   }}
%    \end{macrocode}
% \end{macro}
%
% \begin{macro}{\@ynthm}
%    The definition of |\@ynthm| is completely analogous. In
%    this case the new counter that is defined is not reset within
%    another counter; thus the definition of |\the...|
%    is simplified:
%    \begin{macrocode}
\gdef\@ynthm#1#2{\expandafter\@ifdefinable\csname #1\endcsname
   {\@definecounter{#1}%
    \expandafter\xdef\csname the#1\endcsname{\@thmcounter{#1}}%
%    \end{macrocode}
%    The rest of the definition corresponds literally to that of
%    |\@xnthm|:
%    \begin{macrocode}
    \def\@tempa{\global\@namedef{#1}}\expandafter \@tempa
     \expandafter{\csname th@\the \theorem@style \expandafter
     \endcsname \the\theorem@bodyfont \@thm{#1}{#2}}%
    \global \expandafter \let \csname end#1\endcsname \@endtheorem}}
%    \end{macrocode}
% \end{macro}
%
% \begin{macro}{\@othm}
%    The definition of |\@othm| does not contain anything new.
%    \begin{macrocode}
\gdef\@othm#1[#2]#3{%
%    \end{macrocode}
% \changes{v2.2b}{95/10/07}{Build in check for counter defined}
%    We do not define a new counter but instead use one that has
%    already been defined.  Thus the only definition we need is that
%    of this pseudo-counter (i.e.\ |\the|\meta{env.\ name}).
%    First we check if |#2| corresponds to a known counter name.
%    \begin{macrocode}
  \expandafter\ifx\csname c@#2\endcsname\relax
   \@nocounterr{#2}%
  \else
   \expandafter\@ifdefinable\csname #1\endcsname
   {\expandafter \xdef \csname the#1\endcsname
     {\expandafter \noexpand \csname the#2\endcsname}%
%    \end{macrocode}
%    All other parts of the definition can be adopted from
%    |\@xnthm|. We have to remember, though, that in this case
%    the name of the current counter and the theorem label have moved
%    to the second and third arguments.
%    \begin{macrocode}
    \def\@tempa{\global\@namedef{#1}}\expandafter \@tempa
     \expandafter{\csname th@\the \theorem@style \expandafter
     \endcsname \the\theorem@bodyfont \@thm{#2}{#3}}%
    \global \expandafter \let \csname end#1\endcsname \@endtheorem}%
  \fi}
%    \end{macrocode}
% \end{macro}
%
% \subsection{Macros that are employed in a {\envfont theorem}
%    environment}
%
% \begin{macro}{\@thm}
%    The macro |\@thm| has to increase the current counter.  Then,
%    depending on whether the environment has (or does not have) an
%    optional argument, it has to branch into either |\@begintheorem|
%    or |\@opargtheorem|.
%    \begin{macrocode}
\gdef\@thm#1#2{\refstepcounter{#1}%
%    \end{macrocode}
%    Now we start a {\envfont trivlist} environment, and give
%    |\@topsep| and |\@topsepadd| the values of the \textsf{skip}
%    registers |\theorempreskipamount| and |\theorempostskipamount|.
%    The value in |\@topsep| is the vertical space that is inserted by
%    the first (and only) |\item| in our |\trivlist| whilst
%    |\@topsepadd| is inserted by |\@endparenv| at the end of that
%    {\envfont trivlist} environment.  By using these registers, we
%    obtain the desired space around a {\envfont theorem} environment.
%    \begin{macrocode}
   \trivlist
   \@topsep \theorempreskipamount               % used by first \item
   \@topsepadd \theorempostskipamount           % used by \@endparenv
%    \end{macrocode}
%    Now we have to test whether an optional argument has been given.
% \changes{v2.0d}{89/07/18}{Forgotten percent added.}
%    \begin{macrocode}
   \@ifnextchar [%
%    \end{macrocode}
%    If there is an optional argument, we will call |\@ythm|, and
%    move the arguments read back into the input stream.
%    \begin{macrocode}
   {\@ythm{#1}{#2}}%
%    \end{macrocode}
%    If not, we call |\@begintheorem|. Its first argument is the
%    name of the theorem set (hence the second argument of
%    |\@thm|). Its second argument is the macro that produces the
%    current number.
%    \begin{macrocode}
   {\@begintheorem{#2}{\csname the#1\endcsname}\ignorespaces}}
%    \end{macrocode}
% \end{macro}
%
% \begin{macro}{\@xthm}
% \begin{macro}{\@ythm}
%    Both these macros were originally called by |\@thm|.  We do
%    not need |\@xthm| anymore, hence we reset it to
%    |\relax|. The definition of |\@ythm| has not changed at
%    all from its definition in \LaTeX. In order to make the macros
%    easier to understand, we will
%    nevertheless present it (commented out).
%    \begin{macrocode}
\global\let\@xthm\relax
% \def\@ythm#1#2[#3]{\@opargbegintheorem{#2}{\csname
%           the#1\endcsname}{#3}\ignorespaces}
%</package>
%    \end{macrocode}
%    The primitive |\ignorespaces| in |\@ythm| and
%    |\@thm| is needed to remove the spaces between the
%    |\begin{...}| and the actual text.
% \end{macro}
% \end{macro}
%
%
% \subsection{Definition of the theorem substyles}
%
% As already pointed out, the theorem substyles, defined below, are
% only loaded when necessary.  Note that all these substyles, except
% \texttt{plain}, have |\slshape| as the default body font.
%
% \subsubsection{The \texttt{plain} style}
%
% \label{dok:th@plain} As the following macros use |@|, we have
% to locally set the |\catcode| of this symbol to ``letter''.
% This happens within a group, so that we do not have to worry about
% which |\catcode| that symbol had before.
%    \begin{macrocode}
%<*thp>
\begingroup \makeatletter
%    \end{macrocode}
% Since we are now within a group, we must make all definitions
% globally.  First we make sure that \texttt{theorem.sty} is loaded.
% This will allow us to use this file as a document style option
% without having to call \texttt{theorem} itself as an option.  At the
% same time, we assure that at least version 2 is loaded, since
% |\theorem@style| was not defined in earlier versions.
%    \begin{macrocode}
\@ifundefined{theorem@style}{\input{theorem.sty}}{}
\ProvidesFile{thp.sty}
             [\FMithmInfo]
%    \end{macrocode}
%
%
% \begin{macro}{\th@plain}
% \begin{macro}{\@begintheorem}
% \begin{macro}{\@opargbegintheorem}
%    |\theoremstyle{plain}| corresponds to the original
%    definition, except that the distances to the surrounding text are
%    determined by the parameters |\theorempreskipamount| and
%    |\theorempostskipamount|.
%    First we set the default body font.
%    \begin{macrocode}
\gdef\th@plain{\normalfont\itshape
%    \end{macrocode}
%    Then we define |\@begintheorem| and
%    |\@opargbegintheorem|. These two macros define how the
%    header of a theorem is typeset.  |\@opargbegintheorem| will
%    be called if a \textsf{theorem} environment with an optional
%    argument is encountered; otherwise, the header is constructed by
%    calling |\@begintheorem|.  If one of these macros is
%    executed, we are within a \textsf{trivlist} environment started by
%    |\@thm|.  So the theorem header is produced with an
%    |\item| command.
%
%    Instead of specifying the header font directly, all standard
%    theorem styles use the |\theorem@headerfont| macro to allow
%    customization.  The extra space (|\labelsep|) is necessary
%    because of problems in the \textsf{trivlist} environment.
%    \begin{macrocode}
  \def\@begintheorem##1##2{%
        \item[\hskip\labelsep \theorem@headerfont ##1\ ##2]}%
%    \end{macrocode}
%    The definition of |\@opargbegintheorem| is completely
%    analogous. The only difference is the fact that there exists a
%    third argument (which is the optional parameter of the
%    environment and contains additional information about the
%    theorem).  Customarily we enclose it in parentheses.
%    \begin{macrocode}
\def\@opargbegintheorem##1##2##3{%
   \item[\hskip\labelsep \theorem@headerfont ##1\ ##2\ (##3)]}}
%    \end{macrocode}
% \end{macro}
% \end{macro}
% \end{macro}
%
% \noindent
% We conclude with an |\endgroup| to restore the
% |\catcode| of |@|.
%    \begin{macrocode}
\endgroup
%</thp>
%    \end{macrocode}
%
%
% \subsubsection{The \texttt{break} style}
%
% This style option is stored in the file |thb.sty|.
% For the next two lines see the documentation for |\th@plain| on
% page~\pageref{dok:th@plain}.
%    \begin{macrocode}
%<*thb>
\begingroup \makeatletter
\@ifundefined{theorem@style}{\input{theorem.sty}}{}
\ProvidesFile{thb.sty}
             [\FMithmInfo]
%    \end{macrocode}
%
% \begin{macro}{\th@break}
%    |\theoremstyle{break}| produces a line break after the name
%    of the theorem.  The font is |\slshape|.  Hence, we define
%    |\th@break| as follows:
%    \begin{macrocode}
\gdef\th@break{\normalfont\slshape
  \def\@begintheorem##1##2{\item[%
%    \end{macrocode}
%    We run into the following problem: it is not possible to create
%    the header with |\item[|\meta{title}|]| and then start
%    a new line by, for example, |\mbox{}\\|. Such a definition
%    will fail whenever a list environment follows immediately.  With
%    the above construction, the |\mbox{}| causes the switch
%    |@inlabel| (cf.\ definition of |\list| and
%    |\trivlist| in \cite{latex:source}) to be set to \texttt{false}
%    and so the following list will insert additional vertical
%    space (|\topskip|). This is quite annoying. Therefore, we
%    create the line break within the |\item|. In order to ensure
%    that the text will begin at the proper position in the following
%    line, we simply pretend that the label does not take any
%    room.\footnote{This will lead to problems whenever very high
%    symbols occurring in the line tower into the
%    heading. So, something else has to be done here sometime.}
%    \begin{macrocode}
       \rlap{\vbox{\hbox{\hskip \labelsep\theorem@headerfont ##1\ ##2}%
                   \hbox{\strut}}}]}%
%    \end{macrocode}
%    Again, the definition of |\@opargbegintheorem| is
%    completely analogous.
%    \begin{macrocode}
\def\@opargbegintheorem##1##2##3{%
  \item[\rlap{\vbox{\hbox{\hskip \labelsep \theorem@headerfont
                     ##1\ ##2\ (##3)}%
                    \hbox{\strut}}}]}}
\endgroup
%</thb>
%    \end{macrocode}
% \end{macro}
%
%
% \subsubsection{The \texttt{changebreak} style}
%
%    \begin{macrocode}
% This style option is stored in the file |thcb.sty|.
%    \begin{macrocode}
%<*thcb>
\begingroup \makeatletter
\@ifundefined{theorem@style}{\input{theorem.sty}}{}
\ProvidesFile{thcb.sty}
             [\FMithmInfo]
%    \end{macrocode}
%
% \begin{macro}{\th@changebreak}
%    The \texttt{change-break} theorem style is like \texttt{break}
%    but with interchange of theorem name and theorem number.  Thus we
%    define |\th@changebreak| as follows:
%    \begin{macrocode}
\gdef\th@changebreak{\normalfont\slshape
  \def\@begintheorem##1##2{\item
       [\rlap{\vbox{\hbox{\hskip\labelsep \theorem@headerfont ##2\ ##1}%
                    \hbox{\strut}}}]}%
\def\@opargbegintheorem##1##2##3{%
  \item[\rlap{\vbox{\hbox{\hskip\labelsep \theorem@headerfont
                     ##2\ ##1\ (##3)}%
                    \hbox{\strut}}}]}}
\endgroup
%</thcb>
%    \end{macrocode}
% \end{macro}
%
%
% \subsubsection{The \texttt{change} style}
%
% This style option is stored in the file |thc.sty|.
%    \begin{macrocode}
%<*thc>
\begingroup \makeatletter
\@ifundefined{theorem@style}{\input{theorem.sty}}{}
\ProvidesFile{thc.sty}
             [\FMithmInfo]
%    \end{macrocode}
%
% \begin{macro}{\th@change}
%    The \texttt{change} theorem style corresponds to the
%    \texttt{change break} style without a linebreak after the header.
%    To say it in another way, it's the same as the \texttt{plain}
%    style but with number and name interchanged and |\slshape| as the
%    default font.
%    \begin{macrocode}
\gdef\th@change{\normalfont\slshape
  \def\@begintheorem##1##2{\item
       [\hskip\labelsep \theorem@headerfont ##2\ ##1]}%
\def\@opargbegintheorem##1##2##3{%
  \item[\hskip\labelsep \theorem@headerfont ##2\ ##1\ (##3)]}}
\endgroup
%</thc>
%    \end{macrocode}
% \end{macro}
%
%
% \subsubsection{The \texttt{marginbreak} style}
%
% This style option is the one used most often at Mainz.  It is saved
% in the file |thmb.sty|.
%    \begin{macrocode}
%<*thmb>
\begingroup \makeatletter
\@ifundefined{theorem@style}{\input{theorem.sty}}{}
\ProvidesFile{thmb.sty}
             [\FMithmInfo]
%    \end{macrocode}
%
% \begin{macro}{\th@marginbreak}
%    The \texttt{margin break} style is nearly the same as the
%    \texttt{change break} style. The only difference is the placement
%    of the theorem number.  We use |\llap| to place it in the left
%    margin.
%
%    In this style |\labelsep| denotes the separation between the
%    number and the text.
%    \begin{macrocode}
\gdef\th@marginbreak{\normalfont\slshape
  \def\@begintheorem##1##2{\item
       [\rlap{\vbox{\theorem@headerfont
                    \hbox{\llap{##2}\hskip\labelsep ##1}%
                    \hbox{\strut}}}]}%
\def\@opargbegintheorem##1##2##3{%
  \item[\rlap{\vbox{\theorem@headerfont
                    \hbox{\llap{##2}\hskip\labelsep ##1\ (##3)}%
                    \hbox{\strut}}}]}}
\endgroup
%</thmb>
%    \end{macrocode}
% \changes{v2.1c}{90/02/17}{`endgroup outside macrocode env.}
% \end{macro}
%
%
%
% \subsubsection{The \texttt{margin} style}
%
% This style option is stored in the file |thm.sty|.
%    \begin{macrocode}
%<*thm>
\begingroup \makeatletter
\@ifundefined{theorem@style}{\input{theorem.sty}}{}
\ProvidesFile{thm.sty}
             [\FMithmInfo]
%    \end{macrocode}
%
% \begin{macro}{\th@margin}
%    Again this is only a variant of the theorem styles described
%    above without any new ideas.
%    \begin{macrocode}
\gdef\th@margin{\normalfont\slshape
  \def\@begintheorem##1##2{\item
       [\theorem@headerfont \llap{##2}\hskip\labelsep ##1]}%
\def\@opargbegintheorem##1##2##3{%
  \item[\theorem@headerfont \llap{##2}\hskip\labelsep ##1\ (##3)]}}
\endgroup
%</thm>
%    \end{macrocode}
% \end{macro}
%
%
%
%
% \subsection{Final Definitions}
%
% \begin{macro}{\theorempreskipamount}
% \begin{macro}{\theorempostskipamount}
%    The \textsf{skip} parameters that regulate the vertical empty space
%    before and after the {\envfont theorem} environment have
%    to be allocated as well.
% \changes{v2.0f}{89/07/20}{Unnecessary `global' removed.}
%    \begin{macrocode}
%<*package>
\newskip\theorempreskipamount
\newskip\theorempostskipamount
%    \end{macrocode}
%    Since we have used the same values for all theorem sets, we now can
%    assign them.
%    \begin{macrocode}
\global\setlength\theorempreskipamount{12pt plus 5pt minus 3pt}
\global\setlength\theorempostskipamount{8pt plus 3pt minus 1.5pt}
%    \end{macrocode}
% \end{macro}
% \end{macro}
%
%
% \begin{macro}{\@endtheorem}
%    The same holds for the macro |\@endtheorem|, which ends a
%    {\envfont theorem} environment. Since it is the same for all
%    theorem sets, it is removed from the macros |\th@|\meta{style}.
%    It simply ends the {\envfont trivlist} environment, which was
%    begun in |\@thm|.
%    \begin{macrocode}
\global\let\@endtheorem=\endtrivlist
%    \end{macrocode}
% \end{macro}
%
%
%
% \begin{macro}{\@preamblecmds}
%    All macros defined above are to be used only in the preamble.
%    Therefore, we insert them in |\@preamblecmds| which will disable
%    them at begin document. This is done by the internal \LaTeXe{}
%    command |\@onlypreamble|.
% \changes{v2.0g}{89/09/19}{`theoremheaderfont and `theorembodyfont
%    added.}
% \changes{v2.2a}{94/02/03}{Use \LaTeXe{} interface}
%    \begin{macrocode}
\@onlypreamble\@xnthm
\@onlypreamble\@ynthm
\@onlypreamble\@othm
\@onlypreamble\newtheorem
\@onlypreamble\theoremstyle
\@onlypreamble\theorembodyfont
\@onlypreamble\theoremheaderfont
%    \end{macrocode}
% \end{macro}
%
% Finally we declare the \texttt{plain} theorem style to be the default.
% \changes{v2.1b}{90/06/04}{Fixed missing `item by declaring default
%    style.}
%    \begin{macrocode}
\theoremstyle{plain}
%</package>
%    \end{macrocode}
%
% \Finale

\endinput
}{}
\ProvidesFile{thcb.sty}
             [\FMithmInfo]
%    \end{macrocode}
%
% \begin{macro}{\th@changebreak}
%    The \texttt{change-break} theorem style is like \texttt{break}
%    but with interchange of theorem name and theorem number.  Thus we
%    define |\th@changebreak| as follows:
%    \begin{macrocode}
\gdef\th@changebreak{\normalfont\slshape
  \def\@begintheorem##1##2{\item
       [\rlap{\vbox{\hbox{\hskip\labelsep \theorem@headerfont ##2\ ##1}%
                    \hbox{\strut}}}]}%
\def\@opargbegintheorem##1##2##3{%
  \item[\rlap{\vbox{\hbox{\hskip\labelsep \theorem@headerfont
                     ##2\ ##1\ (##3)}%
                    \hbox{\strut}}}]}}
\endgroup
%</thcb>
%    \end{macrocode}
% \end{macro}
%
%
% \subsubsection{The \texttt{change} style}
%
% This style option is stored in the file |thc.sty|.
%    \begin{macrocode}
%<*thc>
\begingroup \makeatletter
\@ifundefined{theorem@style}{% \iffalse meta-comment
%
% Copyright (C) 1993-2024
%
% The LaTeX Project and any individual authors listed elsewhere
% in this file.
%
% This file is part of the Standard LaTeX `Tools Bundle'.
% -------------------------------------------------------
%
% It may be distributed and/or modified under the
% conditions of the LaTeX Project Public License, either version 1.3c
% of this license or (at your option) any later version.
% The latest version of this license is in
%    https://www.latex-project.org/lppl.txt
% and version 1.3c or later is part of all distributions of LaTeX
% version 2005/12/01 or later.
%
% The list of all files belonging to the LaTeX `Tools Bundle' is
% given in the file `manifest.txt'.
%
% \fi
%
%%
%
%
\def\FMithmInfo{2023/07/05 v2.2c Theorem extension package (FMi)}
%
% \ProvidesFile{theorem.dtx}[\FMithmInfo]
%\iffalse   % this is a METACOMMENT !
%%
%% Package `theorem' to use with LaTeX2e
%% Copyright (C) 1989-1995 Frank Mittelbach, all rights reserved.
%<+package>\@ifundefined{theorem@style}{}{\endinput}
%<+package>\NeedsTeXFormat{LaTeX2e}[1995/06/01]
%<+package>\ProvidesPackage{theorem}
%<+package>         [\FMithmInfo]
%
% \fi
%
% \changes{v1.0}{87/05/07}{First draft.}
% \changes{v1.0a}{87/05/07}{`thistheoremstyle' added.}
% \changes{v1.1a}{87/07/29}{New concept: redefine the internal
%                          \protect\LaTeX{} commands `@xnthm' etc.}
% \changes{v1.1b}{87/07/31}{`theoremfont' used to specify font.}
% \changes{v1.2a}{88/01/08}{`@endtheorem' now globally defined.}
% \changes{v1.2b}{88/01/11}{`newtheorem' only allowed in the preamble.}
% \changes{v1.3a}{88/02/14}{`theoremfont' now a toks register.}
%
% \changes{v2.0a}{88/11/16}{New implementation and new documentation.}
% \changes{v2.0a}{88/11/16}{No layout changes but internal modifications
%                           old substyles won't work.}
% \changes{v2.0b}{89/01/30}{`global' used for all definitions.}
% \changes{v2.0c}{89/06/15}{Documentation translated to English.
%                            Thanks to LLL and Nina}
% \changes{v2.1a}{90/02/15}{Support for new font selection.}
% \changes{v2.2a}{94/02/03}{Cleanup for LaTeX2e}
%
% \DoNotIndex{\@endparenv,\@ifdefinable,\@ifnextchar,\@ifundefined}
% \DoNotIndex{\@namedef,\@nameuse,\@spaces,\@tempa,\@warning}
% \DoNotIndex{\ }
% \DoNotIndex{\begingroup,\bfseries}
% \DoNotIndex{\csname}
% \DoNotIndex{\def,\do,\docdate}
% \DoNotIndex{\endcsname,\endgroup,\endinput,\endtrivlist,\expandafter}
% \DoNotIndex{\filedate,\fileversion}
% \DoNotIndex{\gdef,\global}
% \DoNotIndex{\hbox,\hskip}
% \DoNotIndex{\ignorespaces,\input,\itshape,\item}
% \DoNotIndex{\labelsep,\let,\llap}
% \DoNotIndex{\makeatletter}
% \DoNotIndex{\noexpand}
% \DoNotIndex{\refstepcounter,\relax,\rlap}
% \DoNotIndex{\setlength,\slshape,\space,\string,\strut}
% \DoNotIndex{\the,\trivlist,\typeout}
% \DoNotIndex{\vbox}
% \DoNotIndex{\xdef}
%
% \def\deflabel#1{\ttfamily #1\hfill}
% \def\deflist#1{\begin{list}{}{\settowidth\labelwidth{\ttfamily #1}%
%                               \setlength\leftmargin\labelwidth
%                               \addtolength\leftmargin\labelsep
%                               \let\makelabel\deflabel}}
% \def\enddeflist{\end{list}}
% \def\envfont{\normalfont\ttfamily}
%
% \MakeShortVerb{\|}
%
% \newcommand{\TUB}{{\sl TUGboat\/}}
%
% \GetFileInfo{theorem.dtx}
% \title{An Extension of the
%        \LaTeX{} theorem environment\thanks{This file
%        has version number \fileversion, last
%        revised \filedate.}}
% \author{Frank Mittelbach}
%
% \MaintainedByLaTeXTeam{tools}
% \maketitle
%
% \begin{abstract}
%    The macros described in this paper yield an extension of the
%    \LaTeX{} theorem mechanism. It is designed to satisfy the
%    different requirements of various journals. Thus, the
%    layout of the ``theorems'' can be manipulated by determining a
%    ``style''. This article describes not only the use, but
%    also the definition, of the necessary macros.
% \end{abstract}
%
% \section*{Preface to version 2.2}
%
% For \LaTeXe{} this package did not need any fundamental changes. I
% only modified the messages generated so that theorem layout styles
% will show up with the |\listfiles| command and cleaned the section
% on the New Font Selection Scheme since this is now included in
% \LaTeX.
%
% \section*{Preface to version 2.1}
%
% This version is identical to 2.0g described in \TUB\ 10\#3 except
% for some internal defaults which are now set depending on the used
% font selection scheme.
%
% This was done to avoid unpleasant surprises if the new font selection
% scheme is in force. For further details see
% section~\ref{sec:newfonts} and \cite{bk:GMS94}.
%
% \section{Introduction}
%
% For our purposes here, ``theorems'' are labelled enunciations,
% often set off from the main text by extra space and a font change.
% Theorems, corollaries, conjectures, definitions, and remarks
% are all instances of ``theorems''.  The ``header'' of these structures
% is composed of a label (such as \textsc{Theorem} or \textsc{Remark})
% and a number which serializes an item in the sequence of items
% with the same label.
%
% Shortly after the introduction of \LaTeX{} at the Fachbereich
% Mathematik in Mainz, the desire to manipulate the layout of
% ``theorems'' arose.  In Mainz, the following two conventions came
% into general use:
% \begin{enumerate}
%   \item The number of the theorem is shown in the margin.
%   \item There is a line break at
%         the end of the theorem header.
% \end{enumerate}
%
% Additionally, some journals require different formats which depend on
% the ``sort of theorem'': e.g.\ often remarks and definitions are set
% in |\upshape|, while |\itshape| is employed for main theorems.
%
% Confronted with these requirements, a theorem environment was
% developed in Mainz which allows separate determination of the
% layout of the ``theorems sets'', comparable to |\pagestyle|.
%
%
% \section{The user interface}
% \subsection{Defining new theorem sets}
%
% \DescribeMacro\newtheorem
% As in the original \LaTeX{} version, the command |\newtheorem|
% defines a new ``theorem set'' or ``theorem-like structure''.
% Two required arguments name the new environment and give the text
% to be typeset with each instance of the new ``set'', while
% an optional argument determines how
% the ``set'' is enumerated:
% \begin{description}
%    \item[\ttfamily \bslash newtheorem\{foo\}\{bar\}]
%       The theorem set {\envfont foo} (whose name is \texttt{bar})
%       uses its own counter.
%    \item[\ttfamily \bslash newtheorem\{foo2\}{[foo]}\{bar2\}]
%       The theorem set {\envfont foo2} (printed name \texttt{bar2})
%       uses the same counter as the theorem set \texttt{foo}.
%    \item[\ttfamily \bslash newtheorem\{foo3\}\{bar3\}{[section]}]
%       The theorem set {\envfont foo3} (printed name \texttt{bar3}) is
%       enumerated within the counter \texttt{section}, i.e.\ with every
%       new |\section| the enumeration begins again with 1, and
%       the enumeration is composed from the section-number and the
%       theorem counter itself.
% \end{description}
%
% \DescribeMacro\theoremstyle
% Additionally, the command |\theoremstyle| can define the layout
% of various, or all, theorem sets.  It should be noted that any
% theorem set defined by |\newtheorem| is typeset in the
% |\theoremstyle| that is current at the time of the definition.
% Thus, the following
% \begin{verbatim}
%    \theoremstyle{break}        \newtheorem{Cor}{Corollary}
%    \theoremstyle{plain}        \newtheorem{Exa}{Example}[section]
%\end{verbatim}                             ^^A No blank at line start
% leads to the result that the set {\envfont Cor} is formatted in the
% style \texttt{break}, while the set {\envfont Exa} and all the
% following ones are formatted in the style \texttt{plain}, unless
% another |\theoremstyle| follows.  Since the definitions installed by
% |\newtheorem| are global, one also can limit |\theoremstyle| locally
% by grouping braces.
%
% \DescribeMacro\theorembodyfont
% The choice of the font for the theorem body is completely
% independent of the chosen |\theoremstyle|; this has proven to
% be very advantageous.  For example,
% \begin{verbatim}
%   {\theorembodyfont{\upshape}          \newtheorem{Rem}{Remark}}
%\end{verbatim}
% defines a theorem set {\envfont Rem}, which will be set in
% |\upshape| in the current layout (which in our example is
% \texttt{plain}).  As with |\theoremstyle|, the |\theorembodyfont|
% chosen is that current at the time of |\newtheorem|.  If
% |\theorembodyfont| is not specified or one defines
% |\theorembodyfont{}|, then the font used will be that defined by the
% |\theoremstyle|.
%
% \DescribeMacro\theoremheaderfont It is also possible to customize
% the font used for the theorem headers.  This is, however, a global
% declaration, and therefore there should be at most one
% |\theoremheaderfont| declaration in the preamble.\footnote{If it is
% actually necessary to have different header fonts, one has to define
% new theorem styles (substituting the desired font) or specify the
% information directly in the \texttt{\bslash newtheorem} declaration
% (the unclean variant).}
%
% \DescribeMacro\theorempreskipamount
% \DescribeMacro\theorempostskipamount Two additional parameters
% affect the vertical space around the theorem environments:
% |\theorempreskipamount| and |\theorempostskipamount| define,
% respectively, the spacing before and after such an environment.
% These parameters apply for all theorem sets and can be manipulated
% with the ordinary length macros.  They are rubber lengths
% (`\textsf{skips}'), and therefore can contain \texttt{plus} and
% \texttt{minus} parts.
%
% Since the definition of theorem sets should---most sensibly---be
% placed in the preamble, we only allow installation there.
% It is therefore possible to release the memory used here after
% |\begin{document}|, in order to make room for other
% applications.
%
%
% \subsection{Existing theorem styles}
%
% The following theorem styles exist to date:
% \begin{deflist}{marginbreak}
%    \item[plain]
%       This theorem style emulates the original \LaTeX{} definition,
%       except that additionally the parameters
%       |\theorem...skipamount| are used.
%    \item[break]
%       In this style, the theorem header is followed by a line
%       break.
%    \item[marginbreak]
%       The theorem number is set in the margin, and there is a line
%       break as in \texttt{break}.
%    \item[changebreak]
%       Like \texttt{break}, but with header number and text
%       interchanged.
%    \item[change]
%       Header number and text are interchanged, without a line break.
%    \item[margin]
%       The number is set in the left margin, without a line break.
% \end{deflist}
% All styles (except \texttt{plain}) select |\slshape| as the default
% |\theorembodyfont|.
%
%
% \subsection{Examples}
%
% Given the above theorem sets {\envfont Cor}, {\envfont Exa}
% and {\envfont Rem},
% suppose that the preamble also contains the declarations:
% \begin{verbatim}
%    \theoremstyle{marginbreak}   \newtheorem{Lem}[Cor]{Lemma}
%    \theoremstyle{change}
%    \theorembodyfont{\itshape}        \newtheorem{Def}[Cor]{Definition}
%
%    \theoremheaderfont{\scshape}
%\end{verbatim}
% Then the following are some typical examples of the typeset output
% resulting from their use.
%
% \begin{Cor}
%    This is a sentence typeset in the theorem environment {\envfont
%    Cor}.
% \end{Cor}
%
% \begin{Exa}
%    This is a sentence typeset in the theorem environment {\envfont
%    Exa}.
% \end{Exa}
%
% \begin{Rem}
%    This is a sentence typeset in the theorem environment {\envfont
%    Rem}.
% \end{Rem}
%
% \begin{Lem}[Ben User]
%    This is a sentence typeset in the theorem environment {\envfont
%    Lem}.
% \end{Lem}
%
% \begin{Def}[Very impressive Definition]
%    This is a sentence typeset in the theorem environment {\envfont
%    Def}.
% \end{Def}
%
% The last two examples show the effect of the optional argument to
% a theorem environment (it is the text typeset in parentheses).
%
%
% \section{Special Considerations}
% \label{sec:newfonts}
%
% Theoremheader and body are implemented as a unit. This means that
% the |\theoremheaderfont|  will inherit characteristics of the
% |\theorembodyfont| in \LaTeXe. Thus, if for example
% |\theorembodyfont| is |\itshape| and |\theoremheaderfont|
% is |\bfseries| the font selected for the header will have the
% characteristics `bold extended italic'. If this is not desired one
% should set the |\theoremheaderfont| to something like
% \begin{verbatim}
%      \theoremheaderfont{\normalfont\bfseries}
%\end{verbatim}
% i.e.\ supplying all necessary font information explicitly.
%
%
%
% \section{Acknowledgements}
%
% The publication of this set of macros was only possible with the
% help of Christina Busse (translating the manuscript into English),
% Joachim Pense (playing the r\^ole of typist), Chris Rowley
% (looking everything over) and many others providing useful
% suggestions.
%
% \MaybeStop{%
%    \begin{thebibliography}{9}
%       \bibitem{bk:GMS94} \textsc{M.~Goossens}, \textsc{F.~Mittelbach}
%         and \textsc{A.~Samarin}.
%         \newblock The \LaTeX{} Companion.
%         \newblock
%         Addison-Wesley, Reading, Massachusetts, 1994.
%       \bibitem{latex:source} \textsc{Lamport, Leslie.} \newblock
%                               \texttt{latex.tex}, version 2.09, date
%                               Feb. 1990.
%    \end{thebibliography}
%    \PrintIndex
%    \PrintChanges}
%
%
% \section{The documentation driver file}
%
% The next bit of code contains the documentation driver file for
% \TeX{}, i.e., the file that will produce the documentation you are
% currently reading. It will be extracted from this file by the
% \texttt{docstrip} program. Since it is the first code in the file
% one can alternatively process this file directly with \LaTeXe{} to
% obtain the documentation.
%
%    \begin{macrocode}
%<*driver>
\documentclass{ltxdoc}

\usepackage{theorem}

%<+driver>% The next few lines define theorem sets which are used
%<+driver>% in the example section of the documentation.

\theoremstyle{break}       \newtheorem{Cor}{Corollary}
\theoremstyle{plain}       \newtheorem{Exa}{Example}[section]
{\theorembodyfont{\upshape}\newtheorem{Rem}{Remark}}
\theoremstyle{marginbreak} \newtheorem{Lem}[Cor]{Lemma}
\theoremstyle{change}
\theorembodyfont{\itshape} \newtheorem{Def}[Cor]{Definition}

\theoremheaderfont{\scshape}

\RecordChanges

\begin{document}
   \DocInput{theorem.dtx}
\end{document}
%</driver>
%    \end{macrocode}
%
% \section{Definition of the Macros}
%
% If the file has been loaded before, we abort immediately. If not the
% package announces itself (this is actually done at the very top of
% the file---the way it is done isn't good style so don't copy it).
% \changes{v2.0e}{89/07/19}{Spaces removed from `typeout'.}
% \changes{v2.2b}{95/11/19}{Announce moved to top of file}
%    \begin{macrocode}
%<*package>
%\@ifundefined{theorem@style}{}{\endinput}
%\def\FMithmInfo{1995/11/19 v2.2b Theorem extension package (FMi)}
%\ProvidesPackage{theorem}[\FMithmInfo]
%    \end{macrocode}
%
%
% \subsection{Definition of theorem styles and fonts}
%
% All the definitions in this file are done globally to allow
% inputting this file inside a group.
%
% \begin{macro}{\theoremstyle}
%    Before a theorem style can be installed, the chosen style must be
%    known. For that reason, we must test to see that
%    |\th@|\meta{style} is known or, more precisely, that it is
%    different from |\relax|. If the style is not known then
%    |\th@plain| is used.
%    \begin{macrocode}
\gdef\theoremstyle#1{%
   \@ifundefined{th@#1}{\@warning
          {Unknown theoremstyle `#1'. Using `plain'}%
          \theorem@style{plain}}%
%    \end{macrocode}
%    We save the theorem style to be used in the token register
%    |\theorem@style|.
%    \begin{macrocode}
      {\theorem@style{#1}}%
%    \end{macrocode}
%    Now we ``evaluate'' the theorem style: this means, we call the
%    macro |\th@|\meta{style} which will activate the relevant
%    definitions which are contained in a separate file.  This is done
%    in a group to suppress changes to the current font.  This could
%    otherwise pose problems together with the new font selection
%    scheme\footnote{When I printed the original article using the new
%    font selection scheme I ended with a document with slanted
%    typefaces (text headings and all) simply because one of the
%    theorem styles used \texttt{\bslash sl} at toplevel.} if the
%    |\th@|\meta{style} is evaluated a second time.
% \changes{v2.1a}{90/02/15}{Group added around `th@..}
% \changes{v2.1a}{90/02/15}{Reseting of `..@begintheorem removed.}
%    \begin{macrocode}
      \begingroup
        \csname th@\the\theorem@style \endcsname
      \endgroup}
%    \end{macrocode}
% \end{macro}
%
%
% \begin{macro}{\@begintheorem}
% \changes{v2.1a}{90/02/15}{Relaxed at toplevel.}
% \begin{macro}{\@opargbegintheorem}
% \changes{v2.1a}{90/02/15}{Relaxed at toplevel.}
%    We reset |\@begintheorem| and |\@opargbegintheorem| to
%    |\relax| since these commands are no longer necessary at
%    toplevel. This will save a few tokens.
%    \begin{macrocode}
\global\let\@begintheorem\relax
\global\let\@opargbegintheorem\relax
%    \end{macrocode}
% \end{macro}
% \end{macro}
%
% \begin{macro}{\theorem@style}
%    Obviously the \textsf{token} register used above has to be
%    allocated. To assure the utmost compatibility with the
%    original \LaTeX{} definition, we set the default theorem style to
%    \texttt{plain}, which implements the usual \LaTeX{} convention.
% \changes{v2.0f}{89/07/20}{Unnecessary `global' removed.}
%    \begin{macrocode}
\newtoks\theorem@style
\global\theorem@style{plain}
%    \end{macrocode}
% \end{macro}
%
% \begin{macro}{\theorembodyfont}
% \begin{macro}{\theorem@bodyfont}
%    If the |\theorembodyfont| is set by the user then it should not
%    interact with the default font set in the theorem style. When the
%    new font selection is in force this may happen if, for example,
%    the default is |\itshape| and the new |\theorembodyfont| is
%    |\sffamily|. So we add a |\reset@font| command in front of the
%    user definition.
% \changes{v2.0f}{89/07/20}{Unnecessary `global' removed.}
% \changes{v2.0g}{89/09/19}{`theoremfont to `theorembodyfont.}
% \changes{v2.1c}{92/06/22}{`reset@font added and logic changed.}
% \changes{v2.1d}{92/10/28}{Reset to empty if arg is empty}
%    \begin{macrocode}
\gdef\theorembodyfont#1{%
%    \end{macrocode}
%    We check if the argument supplied is empty and if so put nothing
%    into the |\theorem@bodyfont| token register to allow for
%    |\theorembodyfont{}| as a mean of using the default of the
%    current |\theoremstyle|.
%    \begin{macrocode}
   \def\@tempa{#1}%
   \ifx\@tempa\@empty
    \theorem@bodyfont{}%
   \else
    \theorem@bodyfont{\reset@font#1}%
   \fi
   }
\newtoks\theorem@bodyfont
\global\theorem@bodyfont{}
%    \end{macrocode}
% \end{macro}
% \end{macro}
%
%
% \begin{macro}{\theoremheaderfont}
% \changes{v2.0g}{89/09/19}{Macro added.}
%    The font for the theorem headers is handled differently because
%    this definition applies to all theorem styles.
%    \begin{macrocode}
\gdef\theoremheaderfont#1{\gdef\theorem@headerfont{#1}%
%    \end{macrocode}
%    After using the macro once it is redefined to produce an error
%    message.
%    \begin{macrocode}
       \gdef\theoremheaderfont##1{%
        \typeout{\string\theoremheaderfont\space should be used
                 only once.}}}
%    \end{macrocode}
% \end{macro}
%
% \begin{macro}{\theorem@headerfont}
% \changes{v2.0g}{89/09/19}{Macro added.}
%    To set the |\theorem@headerfont| default we first test if
%    the new fontselection scheme is in force.
% \changes{v2.1a}{90/02/15}{Support for new font selection.}
%    \begin{macrocode}
\ifx\upshape\undefined
%    \end{macrocode}
%    If not we define it to expand into |\bfseries|. We don't use
%    |\let| just in case a following style option redefines this
%    macro.
%    \begin{macrocode}
\gdef\theorem@headerfont{\bfseries}
%    \end{macrocode}
%    Otherwise we reset the current shape before calling |\bfseries|.
%    \begin{macrocode}
\else \gdef\theorem@headerfont{\normalfont\bfseries}\fi
%    \end{macrocode}
% \end{macro}
%
% \begin{macro}{\th@plain}
% \begin{macro}{\th@break}
% \begin{macro}{\th@marginbreak}
% \begin{macro}{\th@changebreak}
% \begin{macro}{\th@change}
% \begin{macro}{\th@margin}
%    The different styles are defined in macros such as
%    |\th@plain|.  Since memory space is precious in
%    ``non-Big-versions'', we have to avoid offering too many unused
%    definitions.  Therefore we define these styles in separate files
%    that can be loaded on demand.  Thus the commands themselves only
%    load these files. We use |\@input@|, a \LaTeXe{} internal command
%    that ensures that the file will be listed with |\listfiles|.
% \changes{v2.2a}{94/02/02}{Use `@input@ to load theorem layout files}
%    \begin{macrocode}
\gdef\th@plain{\@input@{thp.sty}}
\gdef\th@break{\@input@{thb.sty}}
\gdef\th@marginbreak{\@input@{thmb.sty}}
\gdef\th@changebreak{\@input@{thcb.sty}}
\gdef\th@change{\@input@{thc.sty}}
\gdef\th@margin{\@input@{thm.sty}}
%    \end{macrocode}
%    This list will be expanded when new styles become
%    available.  For testing,  just append new theorem substyles
%    as document options.
% \end{macro}
% \end{macro}
% \end{macro}
% \end{macro}
% \end{macro}
% \end{macro}
%
%
%
% \subsection{Definition of a new theorem set}
%
% As already pointed out, a new theorem environment can be defined in
% three different ways:
% \begin{verbatim}
%    \newtheorem{Lem}{Lemma}
%    \newtheorem{Lem}{Lemma}[section]
%    \newtheorem{Lem}[Theorem]{Lemma}
%\end{verbatim}
% The function of the macro |\newtheorem| is to recognize these cases
% and then to branch into one of the three macros |\@ynthm|, |\@xnthm|
% or |\@othm|.  This mechanism is adopted unchanged from
% \cite{latex:source}; the essential point here is that, for example,
% in the second case, the arguments \texttt{Lem}, \texttt{Lemma} and
% \texttt{section} are passed over to the macro |\@xnthm|.
%
% We inspect this case first because the others present fewer problems,
% and thus are easily derived from this one.
%
% \begin{macro}{\@xnthm}
%    For our example arguments, the macro |\@xnthm| must fulfill
%    the following:
%    \begin{itemize}
%       \item
%          Define a new \LaTeX-counter `Lem'
%       \item
%          reset this counter within a |\section|
%       \item
%          define the macro |\theLem|
%       \item
%          define the environment macros |\Lem| and
%          |\endLem| using the current
%          |\theoremstyle| and |\theorem@bodyfont|.
%    \end{itemize}
%    Obviously, all this should happen only if the first argument
%    of |\@xnthm| (i.e.\ \texttt{Lem} in our example) is chosen so
%    as not to conflict with any previously defined commands or
%    environments.  This test is performed by the \LaTeX{} macro
%    |\@ifdefinable|.
%    \begin{macrocode}
\gdef\@xnthm#1#2[#3]{\expandafter\@ifdefinable\csname #1\endcsname
%    \end{macrocode}
%    Therefore, the first argument of |\@ifdefinable| is the
%    expansion (in the example, |\Lem|) of
%    |\csname||#1\endcsname|.  The second argument is
%    executed only if the test has been completed successfully.
%    \begin{macrocode}
   {%
%    \end{macrocode}
%    Now we define the new counter. The names of the \LaTeX{}
%    macros employed should speak for themselves:
% \changes{v2.2b}{1995/10/07}{Use \cs{@newctr}}
% \changes{v2.2c}{1995/11/23}{Use \cs{@newctr} properly.}
%    \begin{macrocode}
    \@definecounter{#1}\@newctr{#1}[#3]%
%    \end{macrocode}
%    Using |\@newctr| will give a proper error message if the counter
%    in |#3| is not defined.
%    In defining `|\theLem|' we must generate the desired macro
%    name by use of |\expandafter| and |\csname|.
%    \begin{macrocode}
    \expandafter\xdef\csname the#1\endcsname
%    \end{macrocode}
%    An |\xdef| is used in order to make the definition global,
%    and to ensure that it contains the replacement texts of
%    |\@thmcountersep| and |\@thmcounter|.\footnote{These
%    two macros can be defined by the document style.  Their default
%    values produce a `\texttt{.}' as separation and an arabic
%    representation of the number.} However, not everything should be
%    expanded. For example, it saves space to use |\thesection|
%    instead of its---at times---lengthy expansion.
%    \begin{macrocode}
      {\expandafter \noexpand \csname the#3\endcsname
       \@thmcountersep \@thmcounter{#1}}%
%    \end{macrocode}
%    Thus with the defaults of \LaTeX{}, |\theLem| would be
%    replaced by the command sequence |\thesection.\arabic{Lem}|.
%
%    We will now look at the definition of the macro which is executed
%    at the beginning of the actual environment (in our example this
%    macro is |\Lem|). It should be noted that we use an
%    ``|\expandafter| trick'' to expand only certain parts of the
%    replacement text at the time of the definition.
%    \begin{macrocode}
    \def\@tempa{\global\@namedef{#1}}%
    \expandafter \@tempa \expandafter{%
%    \end{macrocode}
%    First, the macro that contains the current definitions of
%    |\@begintheorem| and |\@opargtheorem| should be called
%    up.  The name of this macro---as is already known---has the form
%    |\th@|\meta{theorem\ style}; therefore, it must be called by
%    \begin{macrocode}
      \csname th@\the \theorem@style
%    \end{macrocode}
%    In addition the default theorem font should be changeable, i.e.\
%    we have to insert the contents of |\theorem@bodyfont|. For
%    that reason, we expand even further, beyond |\endcsname|,
%    and thus insert the contents of the \textsf{token} register
%    |\theorem@bodyfont| in the replacement text.
%    \begin{macrocode}
            \expandafter \endcsname \the \theorem@bodyfont
%    \end{macrocode}
%    Now it is time to call the macro |\@thm| which takes over
%    the further processing.  It has two arguments: the current
%    counter name (in our example, \texttt{Lem}), and the text of the
%    label (in our example, \texttt{Lemma}).
%    \begin{macrocode}
     \@thm{#1}{#2}}%
%    \end{macrocode}
%    With this, the `sub-definition' is complete.  The macro
%    |\@endtheorem| ends a theorem environment and is, so far, nothing
%    but an |\endtrivlist|.  (Hence it is defined globally, and not
%    within the theorem styles.\footnote{This has to be changed as
%    soon as theorem styles that change \texttt{\bslash @endtheorem}
%    exist.  In such a case, all existing styles must be changed as
%    well since they will have to reset the macro.}) Therefore, we can
%    set it equivalent to the macro that ends the theorem set (in our
%    example, |\endLem|).  However, if some day theorem styles exist
%    that do change |\@endtheorem|, we would have to use the
%    commented-out line instead.
%    \begin{macrocode}
    \global \expandafter \let \csname end#1\endcsname \@endtheorem
%   \global\@namedef{end#1}{\@endtheorem}%
%    \end{macrocode}
%    With these commands all the required definitions are employed,
%    unless the test |\@ifdefinable| has failed.  Therefore, we
%    end the second argument of this macro and with it the definition
%    of |\@xnthm|.
%    \begin{macrocode}
   }}
%    \end{macrocode}
% \end{macro}
%
% \begin{macro}{\@ynthm}
%    The definition of |\@ynthm| is completely analogous. In
%    this case the new counter that is defined is not reset within
%    another counter; thus the definition of |\the...|
%    is simplified:
%    \begin{macrocode}
\gdef\@ynthm#1#2{\expandafter\@ifdefinable\csname #1\endcsname
   {\@definecounter{#1}%
    \expandafter\xdef\csname the#1\endcsname{\@thmcounter{#1}}%
%    \end{macrocode}
%    The rest of the definition corresponds literally to that of
%    |\@xnthm|:
%    \begin{macrocode}
    \def\@tempa{\global\@namedef{#1}}\expandafter \@tempa
     \expandafter{\csname th@\the \theorem@style \expandafter
     \endcsname \the\theorem@bodyfont \@thm{#1}{#2}}%
    \global \expandafter \let \csname end#1\endcsname \@endtheorem}}
%    \end{macrocode}
% \end{macro}
%
% \begin{macro}{\@othm}
%    The definition of |\@othm| does not contain anything new.
%    \begin{macrocode}
\gdef\@othm#1[#2]#3{%
%    \end{macrocode}
% \changes{v2.2b}{95/10/07}{Build in check for counter defined}
%    We do not define a new counter but instead use one that has
%    already been defined.  Thus the only definition we need is that
%    of this pseudo-counter (i.e.\ |\the|\meta{env.\ name}).
%    First we check if |#2| corresponds to a known counter name.
%    \begin{macrocode}
  \expandafter\ifx\csname c@#2\endcsname\relax
   \@nocounterr{#2}%
  \else
   \expandafter\@ifdefinable\csname #1\endcsname
   {\expandafter \xdef \csname the#1\endcsname
     {\expandafter \noexpand \csname the#2\endcsname}%
%    \end{macrocode}
%    All other parts of the definition can be adopted from
%    |\@xnthm|. We have to remember, though, that in this case
%    the name of the current counter and the theorem label have moved
%    to the second and third arguments.
%    \begin{macrocode}
    \def\@tempa{\global\@namedef{#1}}\expandafter \@tempa
     \expandafter{\csname th@\the \theorem@style \expandafter
     \endcsname \the\theorem@bodyfont \@thm{#2}{#3}}%
    \global \expandafter \let \csname end#1\endcsname \@endtheorem}%
  \fi}
%    \end{macrocode}
% \end{macro}
%
% \subsection{Macros that are employed in a {\envfont theorem}
%    environment}
%
% \begin{macro}{\@thm}
%    The macro |\@thm| has to increase the current counter.  Then,
%    depending on whether the environment has (or does not have) an
%    optional argument, it has to branch into either |\@begintheorem|
%    or |\@opargtheorem|.
%    \begin{macrocode}
\gdef\@thm#1#2{\refstepcounter{#1}%
%    \end{macrocode}
%    Now we start a {\envfont trivlist} environment, and give
%    |\@topsep| and |\@topsepadd| the values of the \textsf{skip}
%    registers |\theorempreskipamount| and |\theorempostskipamount|.
%    The value in |\@topsep| is the vertical space that is inserted by
%    the first (and only) |\item| in our |\trivlist| whilst
%    |\@topsepadd| is inserted by |\@endparenv| at the end of that
%    {\envfont trivlist} environment.  By using these registers, we
%    obtain the desired space around a {\envfont theorem} environment.
%    \begin{macrocode}
   \trivlist
   \@topsep \theorempreskipamount               % used by first \item
   \@topsepadd \theorempostskipamount           % used by \@endparenv
%    \end{macrocode}
%    Now we have to test whether an optional argument has been given.
% \changes{v2.0d}{89/07/18}{Forgotten percent added.}
%    \begin{macrocode}
   \@ifnextchar [%
%    \end{macrocode}
%    If there is an optional argument, we will call |\@ythm|, and
%    move the arguments read back into the input stream.
%    \begin{macrocode}
   {\@ythm{#1}{#2}}%
%    \end{macrocode}
%    If not, we call |\@begintheorem|. Its first argument is the
%    name of the theorem set (hence the second argument of
%    |\@thm|). Its second argument is the macro that produces the
%    current number.
%    \begin{macrocode}
   {\@begintheorem{#2}{\csname the#1\endcsname}\ignorespaces}}
%    \end{macrocode}
% \end{macro}
%
% \begin{macro}{\@xthm}
% \begin{macro}{\@ythm}
%    Both these macros were originally called by |\@thm|.  We do
%    not need |\@xthm| anymore, hence we reset it to
%    |\relax|. The definition of |\@ythm| has not changed at
%    all from its definition in \LaTeX. In order to make the macros
%    easier to understand, we will
%    nevertheless present it (commented out).
%    \begin{macrocode}
\global\let\@xthm\relax
% \def\@ythm#1#2[#3]{\@opargbegintheorem{#2}{\csname
%           the#1\endcsname}{#3}\ignorespaces}
%</package>
%    \end{macrocode}
%    The primitive |\ignorespaces| in |\@ythm| and
%    |\@thm| is needed to remove the spaces between the
%    |\begin{...}| and the actual text.
% \end{macro}
% \end{macro}
%
%
% \subsection{Definition of the theorem substyles}
%
% As already pointed out, the theorem substyles, defined below, are
% only loaded when necessary.  Note that all these substyles, except
% \texttt{plain}, have |\slshape| as the default body font.
%
% \subsubsection{The \texttt{plain} style}
%
% \label{dok:th@plain} As the following macros use |@|, we have
% to locally set the |\catcode| of this symbol to ``letter''.
% This happens within a group, so that we do not have to worry about
% which |\catcode| that symbol had before.
%    \begin{macrocode}
%<*thp>
\begingroup \makeatletter
%    \end{macrocode}
% Since we are now within a group, we must make all definitions
% globally.  First we make sure that \texttt{theorem.sty} is loaded.
% This will allow us to use this file as a document style option
% without having to call \texttt{theorem} itself as an option.  At the
% same time, we assure that at least version 2 is loaded, since
% |\theorem@style| was not defined in earlier versions.
%    \begin{macrocode}
\@ifundefined{theorem@style}{\input{theorem.sty}}{}
\ProvidesFile{thp.sty}
             [\FMithmInfo]
%    \end{macrocode}
%
%
% \begin{macro}{\th@plain}
% \begin{macro}{\@begintheorem}
% \begin{macro}{\@opargbegintheorem}
%    |\theoremstyle{plain}| corresponds to the original
%    definition, except that the distances to the surrounding text are
%    determined by the parameters |\theorempreskipamount| and
%    |\theorempostskipamount|.
%    First we set the default body font.
%    \begin{macrocode}
\gdef\th@plain{\normalfont\itshape
%    \end{macrocode}
%    Then we define |\@begintheorem| and
%    |\@opargbegintheorem|. These two macros define how the
%    header of a theorem is typeset.  |\@opargbegintheorem| will
%    be called if a \textsf{theorem} environment with an optional
%    argument is encountered; otherwise, the header is constructed by
%    calling |\@begintheorem|.  If one of these macros is
%    executed, we are within a \textsf{trivlist} environment started by
%    |\@thm|.  So the theorem header is produced with an
%    |\item| command.
%
%    Instead of specifying the header font directly, all standard
%    theorem styles use the |\theorem@headerfont| macro to allow
%    customization.  The extra space (|\labelsep|) is necessary
%    because of problems in the \textsf{trivlist} environment.
%    \begin{macrocode}
  \def\@begintheorem##1##2{%
        \item[\hskip\labelsep \theorem@headerfont ##1\ ##2]}%
%    \end{macrocode}
%    The definition of |\@opargbegintheorem| is completely
%    analogous. The only difference is the fact that there exists a
%    third argument (which is the optional parameter of the
%    environment and contains additional information about the
%    theorem).  Customarily we enclose it in parentheses.
%    \begin{macrocode}
\def\@opargbegintheorem##1##2##3{%
   \item[\hskip\labelsep \theorem@headerfont ##1\ ##2\ (##3)]}}
%    \end{macrocode}
% \end{macro}
% \end{macro}
% \end{macro}
%
% \noindent
% We conclude with an |\endgroup| to restore the
% |\catcode| of |@|.
%    \begin{macrocode}
\endgroup
%</thp>
%    \end{macrocode}
%
%
% \subsubsection{The \texttt{break} style}
%
% This style option is stored in the file |thb.sty|.
% For the next two lines see the documentation for |\th@plain| on
% page~\pageref{dok:th@plain}.
%    \begin{macrocode}
%<*thb>
\begingroup \makeatletter
\@ifundefined{theorem@style}{\input{theorem.sty}}{}
\ProvidesFile{thb.sty}
             [\FMithmInfo]
%    \end{macrocode}
%
% \begin{macro}{\th@break}
%    |\theoremstyle{break}| produces a line break after the name
%    of the theorem.  The font is |\slshape|.  Hence, we define
%    |\th@break| as follows:
%    \begin{macrocode}
\gdef\th@break{\normalfont\slshape
  \def\@begintheorem##1##2{\item[%
%    \end{macrocode}
%    We run into the following problem: it is not possible to create
%    the header with |\item[|\meta{title}|]| and then start
%    a new line by, for example, |\mbox{}\\|. Such a definition
%    will fail whenever a list environment follows immediately.  With
%    the above construction, the |\mbox{}| causes the switch
%    |@inlabel| (cf.\ definition of |\list| and
%    |\trivlist| in \cite{latex:source}) to be set to \texttt{false}
%    and so the following list will insert additional vertical
%    space (|\topskip|). This is quite annoying. Therefore, we
%    create the line break within the |\item|. In order to ensure
%    that the text will begin at the proper position in the following
%    line, we simply pretend that the label does not take any
%    room.\footnote{This will lead to problems whenever very high
%    symbols occurring in the line tower into the
%    heading. So, something else has to be done here sometime.}
%    \begin{macrocode}
       \rlap{\vbox{\hbox{\hskip \labelsep\theorem@headerfont ##1\ ##2}%
                   \hbox{\strut}}}]}%
%    \end{macrocode}
%    Again, the definition of |\@opargbegintheorem| is
%    completely analogous.
%    \begin{macrocode}
\def\@opargbegintheorem##1##2##3{%
  \item[\rlap{\vbox{\hbox{\hskip \labelsep \theorem@headerfont
                     ##1\ ##2\ (##3)}%
                    \hbox{\strut}}}]}}
\endgroup
%</thb>
%    \end{macrocode}
% \end{macro}
%
%
% \subsubsection{The \texttt{changebreak} style}
%
%    \begin{macrocode}
% This style option is stored in the file |thcb.sty|.
%    \begin{macrocode}
%<*thcb>
\begingroup \makeatletter
\@ifundefined{theorem@style}{\input{theorem.sty}}{}
\ProvidesFile{thcb.sty}
             [\FMithmInfo]
%    \end{macrocode}
%
% \begin{macro}{\th@changebreak}
%    The \texttt{change-break} theorem style is like \texttt{break}
%    but with interchange of theorem name and theorem number.  Thus we
%    define |\th@changebreak| as follows:
%    \begin{macrocode}
\gdef\th@changebreak{\normalfont\slshape
  \def\@begintheorem##1##2{\item
       [\rlap{\vbox{\hbox{\hskip\labelsep \theorem@headerfont ##2\ ##1}%
                    \hbox{\strut}}}]}%
\def\@opargbegintheorem##1##2##3{%
  \item[\rlap{\vbox{\hbox{\hskip\labelsep \theorem@headerfont
                     ##2\ ##1\ (##3)}%
                    \hbox{\strut}}}]}}
\endgroup
%</thcb>
%    \end{macrocode}
% \end{macro}
%
%
% \subsubsection{The \texttt{change} style}
%
% This style option is stored in the file |thc.sty|.
%    \begin{macrocode}
%<*thc>
\begingroup \makeatletter
\@ifundefined{theorem@style}{\input{theorem.sty}}{}
\ProvidesFile{thc.sty}
             [\FMithmInfo]
%    \end{macrocode}
%
% \begin{macro}{\th@change}
%    The \texttt{change} theorem style corresponds to the
%    \texttt{change break} style without a linebreak after the header.
%    To say it in another way, it's the same as the \texttt{plain}
%    style but with number and name interchanged and |\slshape| as the
%    default font.
%    \begin{macrocode}
\gdef\th@change{\normalfont\slshape
  \def\@begintheorem##1##2{\item
       [\hskip\labelsep \theorem@headerfont ##2\ ##1]}%
\def\@opargbegintheorem##1##2##3{%
  \item[\hskip\labelsep \theorem@headerfont ##2\ ##1\ (##3)]}}
\endgroup
%</thc>
%    \end{macrocode}
% \end{macro}
%
%
% \subsubsection{The \texttt{marginbreak} style}
%
% This style option is the one used most often at Mainz.  It is saved
% in the file |thmb.sty|.
%    \begin{macrocode}
%<*thmb>
\begingroup \makeatletter
\@ifundefined{theorem@style}{\input{theorem.sty}}{}
\ProvidesFile{thmb.sty}
             [\FMithmInfo]
%    \end{macrocode}
%
% \begin{macro}{\th@marginbreak}
%    The \texttt{margin break} style is nearly the same as the
%    \texttt{change break} style. The only difference is the placement
%    of the theorem number.  We use |\llap| to place it in the left
%    margin.
%
%    In this style |\labelsep| denotes the separation between the
%    number and the text.
%    \begin{macrocode}
\gdef\th@marginbreak{\normalfont\slshape
  \def\@begintheorem##1##2{\item
       [\rlap{\vbox{\theorem@headerfont
                    \hbox{\llap{##2}\hskip\labelsep ##1}%
                    \hbox{\strut}}}]}%
\def\@opargbegintheorem##1##2##3{%
  \item[\rlap{\vbox{\theorem@headerfont
                    \hbox{\llap{##2}\hskip\labelsep ##1\ (##3)}%
                    \hbox{\strut}}}]}}
\endgroup
%</thmb>
%    \end{macrocode}
% \changes{v2.1c}{90/02/17}{`endgroup outside macrocode env.}
% \end{macro}
%
%
%
% \subsubsection{The \texttt{margin} style}
%
% This style option is stored in the file |thm.sty|.
%    \begin{macrocode}
%<*thm>
\begingroup \makeatletter
\@ifundefined{theorem@style}{\input{theorem.sty}}{}
\ProvidesFile{thm.sty}
             [\FMithmInfo]
%    \end{macrocode}
%
% \begin{macro}{\th@margin}
%    Again this is only a variant of the theorem styles described
%    above without any new ideas.
%    \begin{macrocode}
\gdef\th@margin{\normalfont\slshape
  \def\@begintheorem##1##2{\item
       [\theorem@headerfont \llap{##2}\hskip\labelsep ##1]}%
\def\@opargbegintheorem##1##2##3{%
  \item[\theorem@headerfont \llap{##2}\hskip\labelsep ##1\ (##3)]}}
\endgroup
%</thm>
%    \end{macrocode}
% \end{macro}
%
%
%
%
% \subsection{Final Definitions}
%
% \begin{macro}{\theorempreskipamount}
% \begin{macro}{\theorempostskipamount}
%    The \textsf{skip} parameters that regulate the vertical empty space
%    before and after the {\envfont theorem} environment have
%    to be allocated as well.
% \changes{v2.0f}{89/07/20}{Unnecessary `global' removed.}
%    \begin{macrocode}
%<*package>
\newskip\theorempreskipamount
\newskip\theorempostskipamount
%    \end{macrocode}
%    Since we have used the same values for all theorem sets, we now can
%    assign them.
%    \begin{macrocode}
\global\setlength\theorempreskipamount{12pt plus 5pt minus 3pt}
\global\setlength\theorempostskipamount{8pt plus 3pt minus 1.5pt}
%    \end{macrocode}
% \end{macro}
% \end{macro}
%
%
% \begin{macro}{\@endtheorem}
%    The same holds for the macro |\@endtheorem|, which ends a
%    {\envfont theorem} environment. Since it is the same for all
%    theorem sets, it is removed from the macros |\th@|\meta{style}.
%    It simply ends the {\envfont trivlist} environment, which was
%    begun in |\@thm|.
%    \begin{macrocode}
\global\let\@endtheorem=\endtrivlist
%    \end{macrocode}
% \end{macro}
%
%
%
% \begin{macro}{\@preamblecmds}
%    All macros defined above are to be used only in the preamble.
%    Therefore, we insert them in |\@preamblecmds| which will disable
%    them at begin document. This is done by the internal \LaTeXe{}
%    command |\@onlypreamble|.
% \changes{v2.0g}{89/09/19}{`theoremheaderfont and `theorembodyfont
%    added.}
% \changes{v2.2a}{94/02/03}{Use \LaTeXe{} interface}
%    \begin{macrocode}
\@onlypreamble\@xnthm
\@onlypreamble\@ynthm
\@onlypreamble\@othm
\@onlypreamble\newtheorem
\@onlypreamble\theoremstyle
\@onlypreamble\theorembodyfont
\@onlypreamble\theoremheaderfont
%    \end{macrocode}
% \end{macro}
%
% Finally we declare the \texttt{plain} theorem style to be the default.
% \changes{v2.1b}{90/06/04}{Fixed missing `item by declaring default
%    style.}
%    \begin{macrocode}
\theoremstyle{plain}
%</package>
%    \end{macrocode}
%
% \Finale

\endinput
}{}
\ProvidesFile{thc.sty}
             [\FMithmInfo]
%    \end{macrocode}
%
% \begin{macro}{\th@change}
%    The \texttt{change} theorem style corresponds to the
%    \texttt{change break} style without a linebreak after the header.
%    To say it in another way, it's the same as the \texttt{plain}
%    style but with number and name interchanged and |\slshape| as the
%    default font.
%    \begin{macrocode}
\gdef\th@change{\normalfont\slshape
  \def\@begintheorem##1##2{\item
       [\hskip\labelsep \theorem@headerfont ##2\ ##1]}%
\def\@opargbegintheorem##1##2##3{%
  \item[\hskip\labelsep \theorem@headerfont ##2\ ##1\ (##3)]}}
\endgroup
%</thc>
%    \end{macrocode}
% \end{macro}
%
%
% \subsubsection{The \texttt{marginbreak} style}
%
% This style option is the one used most often at Mainz.  It is saved
% in the file |thmb.sty|.
%    \begin{macrocode}
%<*thmb>
\begingroup \makeatletter
\@ifundefined{theorem@style}{% \iffalse meta-comment
%
% Copyright (C) 1993-2024
%
% The LaTeX Project and any individual authors listed elsewhere
% in this file.
%
% This file is part of the Standard LaTeX `Tools Bundle'.
% -------------------------------------------------------
%
% It may be distributed and/or modified under the
% conditions of the LaTeX Project Public License, either version 1.3c
% of this license or (at your option) any later version.
% The latest version of this license is in
%    https://www.latex-project.org/lppl.txt
% and version 1.3c or later is part of all distributions of LaTeX
% version 2005/12/01 or later.
%
% The list of all files belonging to the LaTeX `Tools Bundle' is
% given in the file `manifest.txt'.
%
% \fi
%
%%
%
%
\def\FMithmInfo{2023/07/05 v2.2c Theorem extension package (FMi)}
%
% \ProvidesFile{theorem.dtx}[\FMithmInfo]
%\iffalse   % this is a METACOMMENT !
%%
%% Package `theorem' to use with LaTeX2e
%% Copyright (C) 1989-1995 Frank Mittelbach, all rights reserved.
%<+package>\@ifundefined{theorem@style}{}{\endinput}
%<+package>\NeedsTeXFormat{LaTeX2e}[1995/06/01]
%<+package>\ProvidesPackage{theorem}
%<+package>         [\FMithmInfo]
%
% \fi
%
% \changes{v1.0}{87/05/07}{First draft.}
% \changes{v1.0a}{87/05/07}{`thistheoremstyle' added.}
% \changes{v1.1a}{87/07/29}{New concept: redefine the internal
%                          \protect\LaTeX{} commands `@xnthm' etc.}
% \changes{v1.1b}{87/07/31}{`theoremfont' used to specify font.}
% \changes{v1.2a}{88/01/08}{`@endtheorem' now globally defined.}
% \changes{v1.2b}{88/01/11}{`newtheorem' only allowed in the preamble.}
% \changes{v1.3a}{88/02/14}{`theoremfont' now a toks register.}
%
% \changes{v2.0a}{88/11/16}{New implementation and new documentation.}
% \changes{v2.0a}{88/11/16}{No layout changes but internal modifications
%                           old substyles won't work.}
% \changes{v2.0b}{89/01/30}{`global' used for all definitions.}
% \changes{v2.0c}{89/06/15}{Documentation translated to English.
%                            Thanks to LLL and Nina}
% \changes{v2.1a}{90/02/15}{Support for new font selection.}
% \changes{v2.2a}{94/02/03}{Cleanup for LaTeX2e}
%
% \DoNotIndex{\@endparenv,\@ifdefinable,\@ifnextchar,\@ifundefined}
% \DoNotIndex{\@namedef,\@nameuse,\@spaces,\@tempa,\@warning}
% \DoNotIndex{\ }
% \DoNotIndex{\begingroup,\bfseries}
% \DoNotIndex{\csname}
% \DoNotIndex{\def,\do,\docdate}
% \DoNotIndex{\endcsname,\endgroup,\endinput,\endtrivlist,\expandafter}
% \DoNotIndex{\filedate,\fileversion}
% \DoNotIndex{\gdef,\global}
% \DoNotIndex{\hbox,\hskip}
% \DoNotIndex{\ignorespaces,\input,\itshape,\item}
% \DoNotIndex{\labelsep,\let,\llap}
% \DoNotIndex{\makeatletter}
% \DoNotIndex{\noexpand}
% \DoNotIndex{\refstepcounter,\relax,\rlap}
% \DoNotIndex{\setlength,\slshape,\space,\string,\strut}
% \DoNotIndex{\the,\trivlist,\typeout}
% \DoNotIndex{\vbox}
% \DoNotIndex{\xdef}
%
% \def\deflabel#1{\ttfamily #1\hfill}
% \def\deflist#1{\begin{list}{}{\settowidth\labelwidth{\ttfamily #1}%
%                               \setlength\leftmargin\labelwidth
%                               \addtolength\leftmargin\labelsep
%                               \let\makelabel\deflabel}}
% \def\enddeflist{\end{list}}
% \def\envfont{\normalfont\ttfamily}
%
% \MakeShortVerb{\|}
%
% \newcommand{\TUB}{{\sl TUGboat\/}}
%
% \GetFileInfo{theorem.dtx}
% \title{An Extension of the
%        \LaTeX{} theorem environment\thanks{This file
%        has version number \fileversion, last
%        revised \filedate.}}
% \author{Frank Mittelbach}
%
% \MaintainedByLaTeXTeam{tools}
% \maketitle
%
% \begin{abstract}
%    The macros described in this paper yield an extension of the
%    \LaTeX{} theorem mechanism. It is designed to satisfy the
%    different requirements of various journals. Thus, the
%    layout of the ``theorems'' can be manipulated by determining a
%    ``style''. This article describes not only the use, but
%    also the definition, of the necessary macros.
% \end{abstract}
%
% \section*{Preface to version 2.2}
%
% For \LaTeXe{} this package did not need any fundamental changes. I
% only modified the messages generated so that theorem layout styles
% will show up with the |\listfiles| command and cleaned the section
% on the New Font Selection Scheme since this is now included in
% \LaTeX.
%
% \section*{Preface to version 2.1}
%
% This version is identical to 2.0g described in \TUB\ 10\#3 except
% for some internal defaults which are now set depending on the used
% font selection scheme.
%
% This was done to avoid unpleasant surprises if the new font selection
% scheme is in force. For further details see
% section~\ref{sec:newfonts} and \cite{bk:GMS94}.
%
% \section{Introduction}
%
% For our purposes here, ``theorems'' are labelled enunciations,
% often set off from the main text by extra space and a font change.
% Theorems, corollaries, conjectures, definitions, and remarks
% are all instances of ``theorems''.  The ``header'' of these structures
% is composed of a label (such as \textsc{Theorem} or \textsc{Remark})
% and a number which serializes an item in the sequence of items
% with the same label.
%
% Shortly after the introduction of \LaTeX{} at the Fachbereich
% Mathematik in Mainz, the desire to manipulate the layout of
% ``theorems'' arose.  In Mainz, the following two conventions came
% into general use:
% \begin{enumerate}
%   \item The number of the theorem is shown in the margin.
%   \item There is a line break at
%         the end of the theorem header.
% \end{enumerate}
%
% Additionally, some journals require different formats which depend on
% the ``sort of theorem'': e.g.\ often remarks and definitions are set
% in |\upshape|, while |\itshape| is employed for main theorems.
%
% Confronted with these requirements, a theorem environment was
% developed in Mainz which allows separate determination of the
% layout of the ``theorems sets'', comparable to |\pagestyle|.
%
%
% \section{The user interface}
% \subsection{Defining new theorem sets}
%
% \DescribeMacro\newtheorem
% As in the original \LaTeX{} version, the command |\newtheorem|
% defines a new ``theorem set'' or ``theorem-like structure''.
% Two required arguments name the new environment and give the text
% to be typeset with each instance of the new ``set'', while
% an optional argument determines how
% the ``set'' is enumerated:
% \begin{description}
%    \item[\ttfamily \bslash newtheorem\{foo\}\{bar\}]
%       The theorem set {\envfont foo} (whose name is \texttt{bar})
%       uses its own counter.
%    \item[\ttfamily \bslash newtheorem\{foo2\}{[foo]}\{bar2\}]
%       The theorem set {\envfont foo2} (printed name \texttt{bar2})
%       uses the same counter as the theorem set \texttt{foo}.
%    \item[\ttfamily \bslash newtheorem\{foo3\}\{bar3\}{[section]}]
%       The theorem set {\envfont foo3} (printed name \texttt{bar3}) is
%       enumerated within the counter \texttt{section}, i.e.\ with every
%       new |\section| the enumeration begins again with 1, and
%       the enumeration is composed from the section-number and the
%       theorem counter itself.
% \end{description}
%
% \DescribeMacro\theoremstyle
% Additionally, the command |\theoremstyle| can define the layout
% of various, or all, theorem sets.  It should be noted that any
% theorem set defined by |\newtheorem| is typeset in the
% |\theoremstyle| that is current at the time of the definition.
% Thus, the following
% \begin{verbatim}
%    \theoremstyle{break}        \newtheorem{Cor}{Corollary}
%    \theoremstyle{plain}        \newtheorem{Exa}{Example}[section]
%\end{verbatim}                             ^^A No blank at line start
% leads to the result that the set {\envfont Cor} is formatted in the
% style \texttt{break}, while the set {\envfont Exa} and all the
% following ones are formatted in the style \texttt{plain}, unless
% another |\theoremstyle| follows.  Since the definitions installed by
% |\newtheorem| are global, one also can limit |\theoremstyle| locally
% by grouping braces.
%
% \DescribeMacro\theorembodyfont
% The choice of the font for the theorem body is completely
% independent of the chosen |\theoremstyle|; this has proven to
% be very advantageous.  For example,
% \begin{verbatim}
%   {\theorembodyfont{\upshape}          \newtheorem{Rem}{Remark}}
%\end{verbatim}
% defines a theorem set {\envfont Rem}, which will be set in
% |\upshape| in the current layout (which in our example is
% \texttt{plain}).  As with |\theoremstyle|, the |\theorembodyfont|
% chosen is that current at the time of |\newtheorem|.  If
% |\theorembodyfont| is not specified or one defines
% |\theorembodyfont{}|, then the font used will be that defined by the
% |\theoremstyle|.
%
% \DescribeMacro\theoremheaderfont It is also possible to customize
% the font used for the theorem headers.  This is, however, a global
% declaration, and therefore there should be at most one
% |\theoremheaderfont| declaration in the preamble.\footnote{If it is
% actually necessary to have different header fonts, one has to define
% new theorem styles (substituting the desired font) or specify the
% information directly in the \texttt{\bslash newtheorem} declaration
% (the unclean variant).}
%
% \DescribeMacro\theorempreskipamount
% \DescribeMacro\theorempostskipamount Two additional parameters
% affect the vertical space around the theorem environments:
% |\theorempreskipamount| and |\theorempostskipamount| define,
% respectively, the spacing before and after such an environment.
% These parameters apply for all theorem sets and can be manipulated
% with the ordinary length macros.  They are rubber lengths
% (`\textsf{skips}'), and therefore can contain \texttt{plus} and
% \texttt{minus} parts.
%
% Since the definition of theorem sets should---most sensibly---be
% placed in the preamble, we only allow installation there.
% It is therefore possible to release the memory used here after
% |\begin{document}|, in order to make room for other
% applications.
%
%
% \subsection{Existing theorem styles}
%
% The following theorem styles exist to date:
% \begin{deflist}{marginbreak}
%    \item[plain]
%       This theorem style emulates the original \LaTeX{} definition,
%       except that additionally the parameters
%       |\theorem...skipamount| are used.
%    \item[break]
%       In this style, the theorem header is followed by a line
%       break.
%    \item[marginbreak]
%       The theorem number is set in the margin, and there is a line
%       break as in \texttt{break}.
%    \item[changebreak]
%       Like \texttt{break}, but with header number and text
%       interchanged.
%    \item[change]
%       Header number and text are interchanged, without a line break.
%    \item[margin]
%       The number is set in the left margin, without a line break.
% \end{deflist}
% All styles (except \texttt{plain}) select |\slshape| as the default
% |\theorembodyfont|.
%
%
% \subsection{Examples}
%
% Given the above theorem sets {\envfont Cor}, {\envfont Exa}
% and {\envfont Rem},
% suppose that the preamble also contains the declarations:
% \begin{verbatim}
%    \theoremstyle{marginbreak}   \newtheorem{Lem}[Cor]{Lemma}
%    \theoremstyle{change}
%    \theorembodyfont{\itshape}        \newtheorem{Def}[Cor]{Definition}
%
%    \theoremheaderfont{\scshape}
%\end{verbatim}
% Then the following are some typical examples of the typeset output
% resulting from their use.
%
% \begin{Cor}
%    This is a sentence typeset in the theorem environment {\envfont
%    Cor}.
% \end{Cor}
%
% \begin{Exa}
%    This is a sentence typeset in the theorem environment {\envfont
%    Exa}.
% \end{Exa}
%
% \begin{Rem}
%    This is a sentence typeset in the theorem environment {\envfont
%    Rem}.
% \end{Rem}
%
% \begin{Lem}[Ben User]
%    This is a sentence typeset in the theorem environment {\envfont
%    Lem}.
% \end{Lem}
%
% \begin{Def}[Very impressive Definition]
%    This is a sentence typeset in the theorem environment {\envfont
%    Def}.
% \end{Def}
%
% The last two examples show the effect of the optional argument to
% a theorem environment (it is the text typeset in parentheses).
%
%
% \section{Special Considerations}
% \label{sec:newfonts}
%
% Theoremheader and body are implemented as a unit. This means that
% the |\theoremheaderfont|  will inherit characteristics of the
% |\theorembodyfont| in \LaTeXe. Thus, if for example
% |\theorembodyfont| is |\itshape| and |\theoremheaderfont|
% is |\bfseries| the font selected for the header will have the
% characteristics `bold extended italic'. If this is not desired one
% should set the |\theoremheaderfont| to something like
% \begin{verbatim}
%      \theoremheaderfont{\normalfont\bfseries}
%\end{verbatim}
% i.e.\ supplying all necessary font information explicitly.
%
%
%
% \section{Acknowledgements}
%
% The publication of this set of macros was only possible with the
% help of Christina Busse (translating the manuscript into English),
% Joachim Pense (playing the r\^ole of typist), Chris Rowley
% (looking everything over) and many others providing useful
% suggestions.
%
% \MaybeStop{%
%    \begin{thebibliography}{9}
%       \bibitem{bk:GMS94} \textsc{M.~Goossens}, \textsc{F.~Mittelbach}
%         and \textsc{A.~Samarin}.
%         \newblock The \LaTeX{} Companion.
%         \newblock
%         Addison-Wesley, Reading, Massachusetts, 1994.
%       \bibitem{latex:source} \textsc{Lamport, Leslie.} \newblock
%                               \texttt{latex.tex}, version 2.09, date
%                               Feb. 1990.
%    \end{thebibliography}
%    \PrintIndex
%    \PrintChanges}
%
%
% \section{The documentation driver file}
%
% The next bit of code contains the documentation driver file for
% \TeX{}, i.e., the file that will produce the documentation you are
% currently reading. It will be extracted from this file by the
% \texttt{docstrip} program. Since it is the first code in the file
% one can alternatively process this file directly with \LaTeXe{} to
% obtain the documentation.
%
%    \begin{macrocode}
%<*driver>
\documentclass{ltxdoc}

\usepackage{theorem}

%<+driver>% The next few lines define theorem sets which are used
%<+driver>% in the example section of the documentation.

\theoremstyle{break}       \newtheorem{Cor}{Corollary}
\theoremstyle{plain}       \newtheorem{Exa}{Example}[section]
{\theorembodyfont{\upshape}\newtheorem{Rem}{Remark}}
\theoremstyle{marginbreak} \newtheorem{Lem}[Cor]{Lemma}
\theoremstyle{change}
\theorembodyfont{\itshape} \newtheorem{Def}[Cor]{Definition}

\theoremheaderfont{\scshape}

\RecordChanges

\begin{document}
   \DocInput{theorem.dtx}
\end{document}
%</driver>
%    \end{macrocode}
%
% \section{Definition of the Macros}
%
% If the file has been loaded before, we abort immediately. If not the
% package announces itself (this is actually done at the very top of
% the file---the way it is done isn't good style so don't copy it).
% \changes{v2.0e}{89/07/19}{Spaces removed from `typeout'.}
% \changes{v2.2b}{95/11/19}{Announce moved to top of file}
%    \begin{macrocode}
%<*package>
%\@ifundefined{theorem@style}{}{\endinput}
%\def\FMithmInfo{1995/11/19 v2.2b Theorem extension package (FMi)}
%\ProvidesPackage{theorem}[\FMithmInfo]
%    \end{macrocode}
%
%
% \subsection{Definition of theorem styles and fonts}
%
% All the definitions in this file are done globally to allow
% inputting this file inside a group.
%
% \begin{macro}{\theoremstyle}
%    Before a theorem style can be installed, the chosen style must be
%    known. For that reason, we must test to see that
%    |\th@|\meta{style} is known or, more precisely, that it is
%    different from |\relax|. If the style is not known then
%    |\th@plain| is used.
%    \begin{macrocode}
\gdef\theoremstyle#1{%
   \@ifundefined{th@#1}{\@warning
          {Unknown theoremstyle `#1'. Using `plain'}%
          \theorem@style{plain}}%
%    \end{macrocode}
%    We save the theorem style to be used in the token register
%    |\theorem@style|.
%    \begin{macrocode}
      {\theorem@style{#1}}%
%    \end{macrocode}
%    Now we ``evaluate'' the theorem style: this means, we call the
%    macro |\th@|\meta{style} which will activate the relevant
%    definitions which are contained in a separate file.  This is done
%    in a group to suppress changes to the current font.  This could
%    otherwise pose problems together with the new font selection
%    scheme\footnote{When I printed the original article using the new
%    font selection scheme I ended with a document with slanted
%    typefaces (text headings and all) simply because one of the
%    theorem styles used \texttt{\bslash sl} at toplevel.} if the
%    |\th@|\meta{style} is evaluated a second time.
% \changes{v2.1a}{90/02/15}{Group added around `th@..}
% \changes{v2.1a}{90/02/15}{Reseting of `..@begintheorem removed.}
%    \begin{macrocode}
      \begingroup
        \csname th@\the\theorem@style \endcsname
      \endgroup}
%    \end{macrocode}
% \end{macro}
%
%
% \begin{macro}{\@begintheorem}
% \changes{v2.1a}{90/02/15}{Relaxed at toplevel.}
% \begin{macro}{\@opargbegintheorem}
% \changes{v2.1a}{90/02/15}{Relaxed at toplevel.}
%    We reset |\@begintheorem| and |\@opargbegintheorem| to
%    |\relax| since these commands are no longer necessary at
%    toplevel. This will save a few tokens.
%    \begin{macrocode}
\global\let\@begintheorem\relax
\global\let\@opargbegintheorem\relax
%    \end{macrocode}
% \end{macro}
% \end{macro}
%
% \begin{macro}{\theorem@style}
%    Obviously the \textsf{token} register used above has to be
%    allocated. To assure the utmost compatibility with the
%    original \LaTeX{} definition, we set the default theorem style to
%    \texttt{plain}, which implements the usual \LaTeX{} convention.
% \changes{v2.0f}{89/07/20}{Unnecessary `global' removed.}
%    \begin{macrocode}
\newtoks\theorem@style
\global\theorem@style{plain}
%    \end{macrocode}
% \end{macro}
%
% \begin{macro}{\theorembodyfont}
% \begin{macro}{\theorem@bodyfont}
%    If the |\theorembodyfont| is set by the user then it should not
%    interact with the default font set in the theorem style. When the
%    new font selection is in force this may happen if, for example,
%    the default is |\itshape| and the new |\theorembodyfont| is
%    |\sffamily|. So we add a |\reset@font| command in front of the
%    user definition.
% \changes{v2.0f}{89/07/20}{Unnecessary `global' removed.}
% \changes{v2.0g}{89/09/19}{`theoremfont to `theorembodyfont.}
% \changes{v2.1c}{92/06/22}{`reset@font added and logic changed.}
% \changes{v2.1d}{92/10/28}{Reset to empty if arg is empty}
%    \begin{macrocode}
\gdef\theorembodyfont#1{%
%    \end{macrocode}
%    We check if the argument supplied is empty and if so put nothing
%    into the |\theorem@bodyfont| token register to allow for
%    |\theorembodyfont{}| as a mean of using the default of the
%    current |\theoremstyle|.
%    \begin{macrocode}
   \def\@tempa{#1}%
   \ifx\@tempa\@empty
    \theorem@bodyfont{}%
   \else
    \theorem@bodyfont{\reset@font#1}%
   \fi
   }
\newtoks\theorem@bodyfont
\global\theorem@bodyfont{}
%    \end{macrocode}
% \end{macro}
% \end{macro}
%
%
% \begin{macro}{\theoremheaderfont}
% \changes{v2.0g}{89/09/19}{Macro added.}
%    The font for the theorem headers is handled differently because
%    this definition applies to all theorem styles.
%    \begin{macrocode}
\gdef\theoremheaderfont#1{\gdef\theorem@headerfont{#1}%
%    \end{macrocode}
%    After using the macro once it is redefined to produce an error
%    message.
%    \begin{macrocode}
       \gdef\theoremheaderfont##1{%
        \typeout{\string\theoremheaderfont\space should be used
                 only once.}}}
%    \end{macrocode}
% \end{macro}
%
% \begin{macro}{\theorem@headerfont}
% \changes{v2.0g}{89/09/19}{Macro added.}
%    To set the |\theorem@headerfont| default we first test if
%    the new fontselection scheme is in force.
% \changes{v2.1a}{90/02/15}{Support for new font selection.}
%    \begin{macrocode}
\ifx\upshape\undefined
%    \end{macrocode}
%    If not we define it to expand into |\bfseries|. We don't use
%    |\let| just in case a following style option redefines this
%    macro.
%    \begin{macrocode}
\gdef\theorem@headerfont{\bfseries}
%    \end{macrocode}
%    Otherwise we reset the current shape before calling |\bfseries|.
%    \begin{macrocode}
\else \gdef\theorem@headerfont{\normalfont\bfseries}\fi
%    \end{macrocode}
% \end{macro}
%
% \begin{macro}{\th@plain}
% \begin{macro}{\th@break}
% \begin{macro}{\th@marginbreak}
% \begin{macro}{\th@changebreak}
% \begin{macro}{\th@change}
% \begin{macro}{\th@margin}
%    The different styles are defined in macros such as
%    |\th@plain|.  Since memory space is precious in
%    ``non-Big-versions'', we have to avoid offering too many unused
%    definitions.  Therefore we define these styles in separate files
%    that can be loaded on demand.  Thus the commands themselves only
%    load these files. We use |\@input@|, a \LaTeXe{} internal command
%    that ensures that the file will be listed with |\listfiles|.
% \changes{v2.2a}{94/02/02}{Use `@input@ to load theorem layout files}
%    \begin{macrocode}
\gdef\th@plain{\@input@{thp.sty}}
\gdef\th@break{\@input@{thb.sty}}
\gdef\th@marginbreak{\@input@{thmb.sty}}
\gdef\th@changebreak{\@input@{thcb.sty}}
\gdef\th@change{\@input@{thc.sty}}
\gdef\th@margin{\@input@{thm.sty}}
%    \end{macrocode}
%    This list will be expanded when new styles become
%    available.  For testing,  just append new theorem substyles
%    as document options.
% \end{macro}
% \end{macro}
% \end{macro}
% \end{macro}
% \end{macro}
% \end{macro}
%
%
%
% \subsection{Definition of a new theorem set}
%
% As already pointed out, a new theorem environment can be defined in
% three different ways:
% \begin{verbatim}
%    \newtheorem{Lem}{Lemma}
%    \newtheorem{Lem}{Lemma}[section]
%    \newtheorem{Lem}[Theorem]{Lemma}
%\end{verbatim}
% The function of the macro |\newtheorem| is to recognize these cases
% and then to branch into one of the three macros |\@ynthm|, |\@xnthm|
% or |\@othm|.  This mechanism is adopted unchanged from
% \cite{latex:source}; the essential point here is that, for example,
% in the second case, the arguments \texttt{Lem}, \texttt{Lemma} and
% \texttt{section} are passed over to the macro |\@xnthm|.
%
% We inspect this case first because the others present fewer problems,
% and thus are easily derived from this one.
%
% \begin{macro}{\@xnthm}
%    For our example arguments, the macro |\@xnthm| must fulfill
%    the following:
%    \begin{itemize}
%       \item
%          Define a new \LaTeX-counter `Lem'
%       \item
%          reset this counter within a |\section|
%       \item
%          define the macro |\theLem|
%       \item
%          define the environment macros |\Lem| and
%          |\endLem| using the current
%          |\theoremstyle| and |\theorem@bodyfont|.
%    \end{itemize}
%    Obviously, all this should happen only if the first argument
%    of |\@xnthm| (i.e.\ \texttt{Lem} in our example) is chosen so
%    as not to conflict with any previously defined commands or
%    environments.  This test is performed by the \LaTeX{} macro
%    |\@ifdefinable|.
%    \begin{macrocode}
\gdef\@xnthm#1#2[#3]{\expandafter\@ifdefinable\csname #1\endcsname
%    \end{macrocode}
%    Therefore, the first argument of |\@ifdefinable| is the
%    expansion (in the example, |\Lem|) of
%    |\csname||#1\endcsname|.  The second argument is
%    executed only if the test has been completed successfully.
%    \begin{macrocode}
   {%
%    \end{macrocode}
%    Now we define the new counter. The names of the \LaTeX{}
%    macros employed should speak for themselves:
% \changes{v2.2b}{1995/10/07}{Use \cs{@newctr}}
% \changes{v2.2c}{1995/11/23}{Use \cs{@newctr} properly.}
%    \begin{macrocode}
    \@definecounter{#1}\@newctr{#1}[#3]%
%    \end{macrocode}
%    Using |\@newctr| will give a proper error message if the counter
%    in |#3| is not defined.
%    In defining `|\theLem|' we must generate the desired macro
%    name by use of |\expandafter| and |\csname|.
%    \begin{macrocode}
    \expandafter\xdef\csname the#1\endcsname
%    \end{macrocode}
%    An |\xdef| is used in order to make the definition global,
%    and to ensure that it contains the replacement texts of
%    |\@thmcountersep| and |\@thmcounter|.\footnote{These
%    two macros can be defined by the document style.  Their default
%    values produce a `\texttt{.}' as separation and an arabic
%    representation of the number.} However, not everything should be
%    expanded. For example, it saves space to use |\thesection|
%    instead of its---at times---lengthy expansion.
%    \begin{macrocode}
      {\expandafter \noexpand \csname the#3\endcsname
       \@thmcountersep \@thmcounter{#1}}%
%    \end{macrocode}
%    Thus with the defaults of \LaTeX{}, |\theLem| would be
%    replaced by the command sequence |\thesection.\arabic{Lem}|.
%
%    We will now look at the definition of the macro which is executed
%    at the beginning of the actual environment (in our example this
%    macro is |\Lem|). It should be noted that we use an
%    ``|\expandafter| trick'' to expand only certain parts of the
%    replacement text at the time of the definition.
%    \begin{macrocode}
    \def\@tempa{\global\@namedef{#1}}%
    \expandafter \@tempa \expandafter{%
%    \end{macrocode}
%    First, the macro that contains the current definitions of
%    |\@begintheorem| and |\@opargtheorem| should be called
%    up.  The name of this macro---as is already known---has the form
%    |\th@|\meta{theorem\ style}; therefore, it must be called by
%    \begin{macrocode}
      \csname th@\the \theorem@style
%    \end{macrocode}
%    In addition the default theorem font should be changeable, i.e.\
%    we have to insert the contents of |\theorem@bodyfont|. For
%    that reason, we expand even further, beyond |\endcsname|,
%    and thus insert the contents of the \textsf{token} register
%    |\theorem@bodyfont| in the replacement text.
%    \begin{macrocode}
            \expandafter \endcsname \the \theorem@bodyfont
%    \end{macrocode}
%    Now it is time to call the macro |\@thm| which takes over
%    the further processing.  It has two arguments: the current
%    counter name (in our example, \texttt{Lem}), and the text of the
%    label (in our example, \texttt{Lemma}).
%    \begin{macrocode}
     \@thm{#1}{#2}}%
%    \end{macrocode}
%    With this, the `sub-definition' is complete.  The macro
%    |\@endtheorem| ends a theorem environment and is, so far, nothing
%    but an |\endtrivlist|.  (Hence it is defined globally, and not
%    within the theorem styles.\footnote{This has to be changed as
%    soon as theorem styles that change \texttt{\bslash @endtheorem}
%    exist.  In such a case, all existing styles must be changed as
%    well since they will have to reset the macro.}) Therefore, we can
%    set it equivalent to the macro that ends the theorem set (in our
%    example, |\endLem|).  However, if some day theorem styles exist
%    that do change |\@endtheorem|, we would have to use the
%    commented-out line instead.
%    \begin{macrocode}
    \global \expandafter \let \csname end#1\endcsname \@endtheorem
%   \global\@namedef{end#1}{\@endtheorem}%
%    \end{macrocode}
%    With these commands all the required definitions are employed,
%    unless the test |\@ifdefinable| has failed.  Therefore, we
%    end the second argument of this macro and with it the definition
%    of |\@xnthm|.
%    \begin{macrocode}
   }}
%    \end{macrocode}
% \end{macro}
%
% \begin{macro}{\@ynthm}
%    The definition of |\@ynthm| is completely analogous. In
%    this case the new counter that is defined is not reset within
%    another counter; thus the definition of |\the...|
%    is simplified:
%    \begin{macrocode}
\gdef\@ynthm#1#2{\expandafter\@ifdefinable\csname #1\endcsname
   {\@definecounter{#1}%
    \expandafter\xdef\csname the#1\endcsname{\@thmcounter{#1}}%
%    \end{macrocode}
%    The rest of the definition corresponds literally to that of
%    |\@xnthm|:
%    \begin{macrocode}
    \def\@tempa{\global\@namedef{#1}}\expandafter \@tempa
     \expandafter{\csname th@\the \theorem@style \expandafter
     \endcsname \the\theorem@bodyfont \@thm{#1}{#2}}%
    \global \expandafter \let \csname end#1\endcsname \@endtheorem}}
%    \end{macrocode}
% \end{macro}
%
% \begin{macro}{\@othm}
%    The definition of |\@othm| does not contain anything new.
%    \begin{macrocode}
\gdef\@othm#1[#2]#3{%
%    \end{macrocode}
% \changes{v2.2b}{95/10/07}{Build in check for counter defined}
%    We do not define a new counter but instead use one that has
%    already been defined.  Thus the only definition we need is that
%    of this pseudo-counter (i.e.\ |\the|\meta{env.\ name}).
%    First we check if |#2| corresponds to a known counter name.
%    \begin{macrocode}
  \expandafter\ifx\csname c@#2\endcsname\relax
   \@nocounterr{#2}%
  \else
   \expandafter\@ifdefinable\csname #1\endcsname
   {\expandafter \xdef \csname the#1\endcsname
     {\expandafter \noexpand \csname the#2\endcsname}%
%    \end{macrocode}
%    All other parts of the definition can be adopted from
%    |\@xnthm|. We have to remember, though, that in this case
%    the name of the current counter and the theorem label have moved
%    to the second and third arguments.
%    \begin{macrocode}
    \def\@tempa{\global\@namedef{#1}}\expandafter \@tempa
     \expandafter{\csname th@\the \theorem@style \expandafter
     \endcsname \the\theorem@bodyfont \@thm{#2}{#3}}%
    \global \expandafter \let \csname end#1\endcsname \@endtheorem}%
  \fi}
%    \end{macrocode}
% \end{macro}
%
% \subsection{Macros that are employed in a {\envfont theorem}
%    environment}
%
% \begin{macro}{\@thm}
%    The macro |\@thm| has to increase the current counter.  Then,
%    depending on whether the environment has (or does not have) an
%    optional argument, it has to branch into either |\@begintheorem|
%    or |\@opargtheorem|.
%    \begin{macrocode}
\gdef\@thm#1#2{\refstepcounter{#1}%
%    \end{macrocode}
%    Now we start a {\envfont trivlist} environment, and give
%    |\@topsep| and |\@topsepadd| the values of the \textsf{skip}
%    registers |\theorempreskipamount| and |\theorempostskipamount|.
%    The value in |\@topsep| is the vertical space that is inserted by
%    the first (and only) |\item| in our |\trivlist| whilst
%    |\@topsepadd| is inserted by |\@endparenv| at the end of that
%    {\envfont trivlist} environment.  By using these registers, we
%    obtain the desired space around a {\envfont theorem} environment.
%    \begin{macrocode}
   \trivlist
   \@topsep \theorempreskipamount               % used by first \item
   \@topsepadd \theorempostskipamount           % used by \@endparenv
%    \end{macrocode}
%    Now we have to test whether an optional argument has been given.
% \changes{v2.0d}{89/07/18}{Forgotten percent added.}
%    \begin{macrocode}
   \@ifnextchar [%
%    \end{macrocode}
%    If there is an optional argument, we will call |\@ythm|, and
%    move the arguments read back into the input stream.
%    \begin{macrocode}
   {\@ythm{#1}{#2}}%
%    \end{macrocode}
%    If not, we call |\@begintheorem|. Its first argument is the
%    name of the theorem set (hence the second argument of
%    |\@thm|). Its second argument is the macro that produces the
%    current number.
%    \begin{macrocode}
   {\@begintheorem{#2}{\csname the#1\endcsname}\ignorespaces}}
%    \end{macrocode}
% \end{macro}
%
% \begin{macro}{\@xthm}
% \begin{macro}{\@ythm}
%    Both these macros were originally called by |\@thm|.  We do
%    not need |\@xthm| anymore, hence we reset it to
%    |\relax|. The definition of |\@ythm| has not changed at
%    all from its definition in \LaTeX. In order to make the macros
%    easier to understand, we will
%    nevertheless present it (commented out).
%    \begin{macrocode}
\global\let\@xthm\relax
% \def\@ythm#1#2[#3]{\@opargbegintheorem{#2}{\csname
%           the#1\endcsname}{#3}\ignorespaces}
%</package>
%    \end{macrocode}
%    The primitive |\ignorespaces| in |\@ythm| and
%    |\@thm| is needed to remove the spaces between the
%    |\begin{...}| and the actual text.
% \end{macro}
% \end{macro}
%
%
% \subsection{Definition of the theorem substyles}
%
% As already pointed out, the theorem substyles, defined below, are
% only loaded when necessary.  Note that all these substyles, except
% \texttt{plain}, have |\slshape| as the default body font.
%
% \subsubsection{The \texttt{plain} style}
%
% \label{dok:th@plain} As the following macros use |@|, we have
% to locally set the |\catcode| of this symbol to ``letter''.
% This happens within a group, so that we do not have to worry about
% which |\catcode| that symbol had before.
%    \begin{macrocode}
%<*thp>
\begingroup \makeatletter
%    \end{macrocode}
% Since we are now within a group, we must make all definitions
% globally.  First we make sure that \texttt{theorem.sty} is loaded.
% This will allow us to use this file as a document style option
% without having to call \texttt{theorem} itself as an option.  At the
% same time, we assure that at least version 2 is loaded, since
% |\theorem@style| was not defined in earlier versions.
%    \begin{macrocode}
\@ifundefined{theorem@style}{\input{theorem.sty}}{}
\ProvidesFile{thp.sty}
             [\FMithmInfo]
%    \end{macrocode}
%
%
% \begin{macro}{\th@plain}
% \begin{macro}{\@begintheorem}
% \begin{macro}{\@opargbegintheorem}
%    |\theoremstyle{plain}| corresponds to the original
%    definition, except that the distances to the surrounding text are
%    determined by the parameters |\theorempreskipamount| and
%    |\theorempostskipamount|.
%    First we set the default body font.
%    \begin{macrocode}
\gdef\th@plain{\normalfont\itshape
%    \end{macrocode}
%    Then we define |\@begintheorem| and
%    |\@opargbegintheorem|. These two macros define how the
%    header of a theorem is typeset.  |\@opargbegintheorem| will
%    be called if a \textsf{theorem} environment with an optional
%    argument is encountered; otherwise, the header is constructed by
%    calling |\@begintheorem|.  If one of these macros is
%    executed, we are within a \textsf{trivlist} environment started by
%    |\@thm|.  So the theorem header is produced with an
%    |\item| command.
%
%    Instead of specifying the header font directly, all standard
%    theorem styles use the |\theorem@headerfont| macro to allow
%    customization.  The extra space (|\labelsep|) is necessary
%    because of problems in the \textsf{trivlist} environment.
%    \begin{macrocode}
  \def\@begintheorem##1##2{%
        \item[\hskip\labelsep \theorem@headerfont ##1\ ##2]}%
%    \end{macrocode}
%    The definition of |\@opargbegintheorem| is completely
%    analogous. The only difference is the fact that there exists a
%    third argument (which is the optional parameter of the
%    environment and contains additional information about the
%    theorem).  Customarily we enclose it in parentheses.
%    \begin{macrocode}
\def\@opargbegintheorem##1##2##3{%
   \item[\hskip\labelsep \theorem@headerfont ##1\ ##2\ (##3)]}}
%    \end{macrocode}
% \end{macro}
% \end{macro}
% \end{macro}
%
% \noindent
% We conclude with an |\endgroup| to restore the
% |\catcode| of |@|.
%    \begin{macrocode}
\endgroup
%</thp>
%    \end{macrocode}
%
%
% \subsubsection{The \texttt{break} style}
%
% This style option is stored in the file |thb.sty|.
% For the next two lines see the documentation for |\th@plain| on
% page~\pageref{dok:th@plain}.
%    \begin{macrocode}
%<*thb>
\begingroup \makeatletter
\@ifundefined{theorem@style}{\input{theorem.sty}}{}
\ProvidesFile{thb.sty}
             [\FMithmInfo]
%    \end{macrocode}
%
% \begin{macro}{\th@break}
%    |\theoremstyle{break}| produces a line break after the name
%    of the theorem.  The font is |\slshape|.  Hence, we define
%    |\th@break| as follows:
%    \begin{macrocode}
\gdef\th@break{\normalfont\slshape
  \def\@begintheorem##1##2{\item[%
%    \end{macrocode}
%    We run into the following problem: it is not possible to create
%    the header with |\item[|\meta{title}|]| and then start
%    a new line by, for example, |\mbox{}\\|. Such a definition
%    will fail whenever a list environment follows immediately.  With
%    the above construction, the |\mbox{}| causes the switch
%    |@inlabel| (cf.\ definition of |\list| and
%    |\trivlist| in \cite{latex:source}) to be set to \texttt{false}
%    and so the following list will insert additional vertical
%    space (|\topskip|). This is quite annoying. Therefore, we
%    create the line break within the |\item|. In order to ensure
%    that the text will begin at the proper position in the following
%    line, we simply pretend that the label does not take any
%    room.\footnote{This will lead to problems whenever very high
%    symbols occurring in the line tower into the
%    heading. So, something else has to be done here sometime.}
%    \begin{macrocode}
       \rlap{\vbox{\hbox{\hskip \labelsep\theorem@headerfont ##1\ ##2}%
                   \hbox{\strut}}}]}%
%    \end{macrocode}
%    Again, the definition of |\@opargbegintheorem| is
%    completely analogous.
%    \begin{macrocode}
\def\@opargbegintheorem##1##2##3{%
  \item[\rlap{\vbox{\hbox{\hskip \labelsep \theorem@headerfont
                     ##1\ ##2\ (##3)}%
                    \hbox{\strut}}}]}}
\endgroup
%</thb>
%    \end{macrocode}
% \end{macro}
%
%
% \subsubsection{The \texttt{changebreak} style}
%
%    \begin{macrocode}
% This style option is stored in the file |thcb.sty|.
%    \begin{macrocode}
%<*thcb>
\begingroup \makeatletter
\@ifundefined{theorem@style}{\input{theorem.sty}}{}
\ProvidesFile{thcb.sty}
             [\FMithmInfo]
%    \end{macrocode}
%
% \begin{macro}{\th@changebreak}
%    The \texttt{change-break} theorem style is like \texttt{break}
%    but with interchange of theorem name and theorem number.  Thus we
%    define |\th@changebreak| as follows:
%    \begin{macrocode}
\gdef\th@changebreak{\normalfont\slshape
  \def\@begintheorem##1##2{\item
       [\rlap{\vbox{\hbox{\hskip\labelsep \theorem@headerfont ##2\ ##1}%
                    \hbox{\strut}}}]}%
\def\@opargbegintheorem##1##2##3{%
  \item[\rlap{\vbox{\hbox{\hskip\labelsep \theorem@headerfont
                     ##2\ ##1\ (##3)}%
                    \hbox{\strut}}}]}}
\endgroup
%</thcb>
%    \end{macrocode}
% \end{macro}
%
%
% \subsubsection{The \texttt{change} style}
%
% This style option is stored in the file |thc.sty|.
%    \begin{macrocode}
%<*thc>
\begingroup \makeatletter
\@ifundefined{theorem@style}{\input{theorem.sty}}{}
\ProvidesFile{thc.sty}
             [\FMithmInfo]
%    \end{macrocode}
%
% \begin{macro}{\th@change}
%    The \texttt{change} theorem style corresponds to the
%    \texttt{change break} style without a linebreak after the header.
%    To say it in another way, it's the same as the \texttt{plain}
%    style but with number and name interchanged and |\slshape| as the
%    default font.
%    \begin{macrocode}
\gdef\th@change{\normalfont\slshape
  \def\@begintheorem##1##2{\item
       [\hskip\labelsep \theorem@headerfont ##2\ ##1]}%
\def\@opargbegintheorem##1##2##3{%
  \item[\hskip\labelsep \theorem@headerfont ##2\ ##1\ (##3)]}}
\endgroup
%</thc>
%    \end{macrocode}
% \end{macro}
%
%
% \subsubsection{The \texttt{marginbreak} style}
%
% This style option is the one used most often at Mainz.  It is saved
% in the file |thmb.sty|.
%    \begin{macrocode}
%<*thmb>
\begingroup \makeatletter
\@ifundefined{theorem@style}{\input{theorem.sty}}{}
\ProvidesFile{thmb.sty}
             [\FMithmInfo]
%    \end{macrocode}
%
% \begin{macro}{\th@marginbreak}
%    The \texttt{margin break} style is nearly the same as the
%    \texttt{change break} style. The only difference is the placement
%    of the theorem number.  We use |\llap| to place it in the left
%    margin.
%
%    In this style |\labelsep| denotes the separation between the
%    number and the text.
%    \begin{macrocode}
\gdef\th@marginbreak{\normalfont\slshape
  \def\@begintheorem##1##2{\item
       [\rlap{\vbox{\theorem@headerfont
                    \hbox{\llap{##2}\hskip\labelsep ##1}%
                    \hbox{\strut}}}]}%
\def\@opargbegintheorem##1##2##3{%
  \item[\rlap{\vbox{\theorem@headerfont
                    \hbox{\llap{##2}\hskip\labelsep ##1\ (##3)}%
                    \hbox{\strut}}}]}}
\endgroup
%</thmb>
%    \end{macrocode}
% \changes{v2.1c}{90/02/17}{`endgroup outside macrocode env.}
% \end{macro}
%
%
%
% \subsubsection{The \texttt{margin} style}
%
% This style option is stored in the file |thm.sty|.
%    \begin{macrocode}
%<*thm>
\begingroup \makeatletter
\@ifundefined{theorem@style}{\input{theorem.sty}}{}
\ProvidesFile{thm.sty}
             [\FMithmInfo]
%    \end{macrocode}
%
% \begin{macro}{\th@margin}
%    Again this is only a variant of the theorem styles described
%    above without any new ideas.
%    \begin{macrocode}
\gdef\th@margin{\normalfont\slshape
  \def\@begintheorem##1##2{\item
       [\theorem@headerfont \llap{##2}\hskip\labelsep ##1]}%
\def\@opargbegintheorem##1##2##3{%
  \item[\theorem@headerfont \llap{##2}\hskip\labelsep ##1\ (##3)]}}
\endgroup
%</thm>
%    \end{macrocode}
% \end{macro}
%
%
%
%
% \subsection{Final Definitions}
%
% \begin{macro}{\theorempreskipamount}
% \begin{macro}{\theorempostskipamount}
%    The \textsf{skip} parameters that regulate the vertical empty space
%    before and after the {\envfont theorem} environment have
%    to be allocated as well.
% \changes{v2.0f}{89/07/20}{Unnecessary `global' removed.}
%    \begin{macrocode}
%<*package>
\newskip\theorempreskipamount
\newskip\theorempostskipamount
%    \end{macrocode}
%    Since we have used the same values for all theorem sets, we now can
%    assign them.
%    \begin{macrocode}
\global\setlength\theorempreskipamount{12pt plus 5pt minus 3pt}
\global\setlength\theorempostskipamount{8pt plus 3pt minus 1.5pt}
%    \end{macrocode}
% \end{macro}
% \end{macro}
%
%
% \begin{macro}{\@endtheorem}
%    The same holds for the macro |\@endtheorem|, which ends a
%    {\envfont theorem} environment. Since it is the same for all
%    theorem sets, it is removed from the macros |\th@|\meta{style}.
%    It simply ends the {\envfont trivlist} environment, which was
%    begun in |\@thm|.
%    \begin{macrocode}
\global\let\@endtheorem=\endtrivlist
%    \end{macrocode}
% \end{macro}
%
%
%
% \begin{macro}{\@preamblecmds}
%    All macros defined above are to be used only in the preamble.
%    Therefore, we insert them in |\@preamblecmds| which will disable
%    them at begin document. This is done by the internal \LaTeXe{}
%    command |\@onlypreamble|.
% \changes{v2.0g}{89/09/19}{`theoremheaderfont and `theorembodyfont
%    added.}
% \changes{v2.2a}{94/02/03}{Use \LaTeXe{} interface}
%    \begin{macrocode}
\@onlypreamble\@xnthm
\@onlypreamble\@ynthm
\@onlypreamble\@othm
\@onlypreamble\newtheorem
\@onlypreamble\theoremstyle
\@onlypreamble\theorembodyfont
\@onlypreamble\theoremheaderfont
%    \end{macrocode}
% \end{macro}
%
% Finally we declare the \texttt{plain} theorem style to be the default.
% \changes{v2.1b}{90/06/04}{Fixed missing `item by declaring default
%    style.}
%    \begin{macrocode}
\theoremstyle{plain}
%</package>
%    \end{macrocode}
%
% \Finale

\endinput
}{}
\ProvidesFile{thmb.sty}
             [\FMithmInfo]
%    \end{macrocode}
%
% \begin{macro}{\th@marginbreak}
%    The \texttt{margin break} style is nearly the same as the
%    \texttt{change break} style. The only difference is the placement
%    of the theorem number.  We use |\llap| to place it in the left
%    margin.
%
%    In this style |\labelsep| denotes the separation between the
%    number and the text.
%    \begin{macrocode}
\gdef\th@marginbreak{\normalfont\slshape
  \def\@begintheorem##1##2{\item
       [\rlap{\vbox{\theorem@headerfont
                    \hbox{\llap{##2}\hskip\labelsep ##1}%
                    \hbox{\strut}}}]}%
\def\@opargbegintheorem##1##2##3{%
  \item[\rlap{\vbox{\theorem@headerfont
                    \hbox{\llap{##2}\hskip\labelsep ##1\ (##3)}%
                    \hbox{\strut}}}]}}
\endgroup
%</thmb>
%    \end{macrocode}
% \changes{v2.1c}{90/02/17}{`endgroup outside macrocode env.}
% \end{macro}
%
%
%
% \subsubsection{The \texttt{margin} style}
%
% This style option is stored in the file |thm.sty|.
%    \begin{macrocode}
%<*thm>
\begingroup \makeatletter
\@ifundefined{theorem@style}{% \iffalse meta-comment
%
% Copyright (C) 1993-2024
%
% The LaTeX Project and any individual authors listed elsewhere
% in this file.
%
% This file is part of the Standard LaTeX `Tools Bundle'.
% -------------------------------------------------------
%
% It may be distributed and/or modified under the
% conditions of the LaTeX Project Public License, either version 1.3c
% of this license or (at your option) any later version.
% The latest version of this license is in
%    https://www.latex-project.org/lppl.txt
% and version 1.3c or later is part of all distributions of LaTeX
% version 2005/12/01 or later.
%
% The list of all files belonging to the LaTeX `Tools Bundle' is
% given in the file `manifest.txt'.
%
% \fi
%
%%
%
%
\def\FMithmInfo{2023/07/05 v2.2c Theorem extension package (FMi)}
%
% \ProvidesFile{theorem.dtx}[\FMithmInfo]
%\iffalse   % this is a METACOMMENT !
%%
%% Package `theorem' to use with LaTeX2e
%% Copyright (C) 1989-1995 Frank Mittelbach, all rights reserved.
%<+package>\@ifundefined{theorem@style}{}{\endinput}
%<+package>\NeedsTeXFormat{LaTeX2e}[1995/06/01]
%<+package>\ProvidesPackage{theorem}
%<+package>         [\FMithmInfo]
%
% \fi
%
% \changes{v1.0}{87/05/07}{First draft.}
% \changes{v1.0a}{87/05/07}{`thistheoremstyle' added.}
% \changes{v1.1a}{87/07/29}{New concept: redefine the internal
%                          \protect\LaTeX{} commands `@xnthm' etc.}
% \changes{v1.1b}{87/07/31}{`theoremfont' used to specify font.}
% \changes{v1.2a}{88/01/08}{`@endtheorem' now globally defined.}
% \changes{v1.2b}{88/01/11}{`newtheorem' only allowed in the preamble.}
% \changes{v1.3a}{88/02/14}{`theoremfont' now a toks register.}
%
% \changes{v2.0a}{88/11/16}{New implementation and new documentation.}
% \changes{v2.0a}{88/11/16}{No layout changes but internal modifications
%                           old substyles won't work.}
% \changes{v2.0b}{89/01/30}{`global' used for all definitions.}
% \changes{v2.0c}{89/06/15}{Documentation translated to English.
%                            Thanks to LLL and Nina}
% \changes{v2.1a}{90/02/15}{Support for new font selection.}
% \changes{v2.2a}{94/02/03}{Cleanup for LaTeX2e}
%
% \DoNotIndex{\@endparenv,\@ifdefinable,\@ifnextchar,\@ifundefined}
% \DoNotIndex{\@namedef,\@nameuse,\@spaces,\@tempa,\@warning}
% \DoNotIndex{\ }
% \DoNotIndex{\begingroup,\bfseries}
% \DoNotIndex{\csname}
% \DoNotIndex{\def,\do,\docdate}
% \DoNotIndex{\endcsname,\endgroup,\endinput,\endtrivlist,\expandafter}
% \DoNotIndex{\filedate,\fileversion}
% \DoNotIndex{\gdef,\global}
% \DoNotIndex{\hbox,\hskip}
% \DoNotIndex{\ignorespaces,\input,\itshape,\item}
% \DoNotIndex{\labelsep,\let,\llap}
% \DoNotIndex{\makeatletter}
% \DoNotIndex{\noexpand}
% \DoNotIndex{\refstepcounter,\relax,\rlap}
% \DoNotIndex{\setlength,\slshape,\space,\string,\strut}
% \DoNotIndex{\the,\trivlist,\typeout}
% \DoNotIndex{\vbox}
% \DoNotIndex{\xdef}
%
% \def\deflabel#1{\ttfamily #1\hfill}
% \def\deflist#1{\begin{list}{}{\settowidth\labelwidth{\ttfamily #1}%
%                               \setlength\leftmargin\labelwidth
%                               \addtolength\leftmargin\labelsep
%                               \let\makelabel\deflabel}}
% \def\enddeflist{\end{list}}
% \def\envfont{\normalfont\ttfamily}
%
% \MakeShortVerb{\|}
%
% \newcommand{\TUB}{{\sl TUGboat\/}}
%
% \GetFileInfo{theorem.dtx}
% \title{An Extension of the
%        \LaTeX{} theorem environment\thanks{This file
%        has version number \fileversion, last
%        revised \filedate.}}
% \author{Frank Mittelbach}
%
% \MaintainedByLaTeXTeam{tools}
% \maketitle
%
% \begin{abstract}
%    The macros described in this paper yield an extension of the
%    \LaTeX{} theorem mechanism. It is designed to satisfy the
%    different requirements of various journals. Thus, the
%    layout of the ``theorems'' can be manipulated by determining a
%    ``style''. This article describes not only the use, but
%    also the definition, of the necessary macros.
% \end{abstract}
%
% \section*{Preface to version 2.2}
%
% For \LaTeXe{} this package did not need any fundamental changes. I
% only modified the messages generated so that theorem layout styles
% will show up with the |\listfiles| command and cleaned the section
% on the New Font Selection Scheme since this is now included in
% \LaTeX.
%
% \section*{Preface to version 2.1}
%
% This version is identical to 2.0g described in \TUB\ 10\#3 except
% for some internal defaults which are now set depending on the used
% font selection scheme.
%
% This was done to avoid unpleasant surprises if the new font selection
% scheme is in force. For further details see
% section~\ref{sec:newfonts} and \cite{bk:GMS94}.
%
% \section{Introduction}
%
% For our purposes here, ``theorems'' are labelled enunciations,
% often set off from the main text by extra space and a font change.
% Theorems, corollaries, conjectures, definitions, and remarks
% are all instances of ``theorems''.  The ``header'' of these structures
% is composed of a label (such as \textsc{Theorem} or \textsc{Remark})
% and a number which serializes an item in the sequence of items
% with the same label.
%
% Shortly after the introduction of \LaTeX{} at the Fachbereich
% Mathematik in Mainz, the desire to manipulate the layout of
% ``theorems'' arose.  In Mainz, the following two conventions came
% into general use:
% \begin{enumerate}
%   \item The number of the theorem is shown in the margin.
%   \item There is a line break at
%         the end of the theorem header.
% \end{enumerate}
%
% Additionally, some journals require different formats which depend on
% the ``sort of theorem'': e.g.\ often remarks and definitions are set
% in |\upshape|, while |\itshape| is employed for main theorems.
%
% Confronted with these requirements, a theorem environment was
% developed in Mainz which allows separate determination of the
% layout of the ``theorems sets'', comparable to |\pagestyle|.
%
%
% \section{The user interface}
% \subsection{Defining new theorem sets}
%
% \DescribeMacro\newtheorem
% As in the original \LaTeX{} version, the command |\newtheorem|
% defines a new ``theorem set'' or ``theorem-like structure''.
% Two required arguments name the new environment and give the text
% to be typeset with each instance of the new ``set'', while
% an optional argument determines how
% the ``set'' is enumerated:
% \begin{description}
%    \item[\ttfamily \bslash newtheorem\{foo\}\{bar\}]
%       The theorem set {\envfont foo} (whose name is \texttt{bar})
%       uses its own counter.
%    \item[\ttfamily \bslash newtheorem\{foo2\}{[foo]}\{bar2\}]
%       The theorem set {\envfont foo2} (printed name \texttt{bar2})
%       uses the same counter as the theorem set \texttt{foo}.
%    \item[\ttfamily \bslash newtheorem\{foo3\}\{bar3\}{[section]}]
%       The theorem set {\envfont foo3} (printed name \texttt{bar3}) is
%       enumerated within the counter \texttt{section}, i.e.\ with every
%       new |\section| the enumeration begins again with 1, and
%       the enumeration is composed from the section-number and the
%       theorem counter itself.
% \end{description}
%
% \DescribeMacro\theoremstyle
% Additionally, the command |\theoremstyle| can define the layout
% of various, or all, theorem sets.  It should be noted that any
% theorem set defined by |\newtheorem| is typeset in the
% |\theoremstyle| that is current at the time of the definition.
% Thus, the following
% \begin{verbatim}
%    \theoremstyle{break}        \newtheorem{Cor}{Corollary}
%    \theoremstyle{plain}        \newtheorem{Exa}{Example}[section]
%\end{verbatim}                             ^^A No blank at line start
% leads to the result that the set {\envfont Cor} is formatted in the
% style \texttt{break}, while the set {\envfont Exa} and all the
% following ones are formatted in the style \texttt{plain}, unless
% another |\theoremstyle| follows.  Since the definitions installed by
% |\newtheorem| are global, one also can limit |\theoremstyle| locally
% by grouping braces.
%
% \DescribeMacro\theorembodyfont
% The choice of the font for the theorem body is completely
% independent of the chosen |\theoremstyle|; this has proven to
% be very advantageous.  For example,
% \begin{verbatim}
%   {\theorembodyfont{\upshape}          \newtheorem{Rem}{Remark}}
%\end{verbatim}
% defines a theorem set {\envfont Rem}, which will be set in
% |\upshape| in the current layout (which in our example is
% \texttt{plain}).  As with |\theoremstyle|, the |\theorembodyfont|
% chosen is that current at the time of |\newtheorem|.  If
% |\theorembodyfont| is not specified or one defines
% |\theorembodyfont{}|, then the font used will be that defined by the
% |\theoremstyle|.
%
% \DescribeMacro\theoremheaderfont It is also possible to customize
% the font used for the theorem headers.  This is, however, a global
% declaration, and therefore there should be at most one
% |\theoremheaderfont| declaration in the preamble.\footnote{If it is
% actually necessary to have different header fonts, one has to define
% new theorem styles (substituting the desired font) or specify the
% information directly in the \texttt{\bslash newtheorem} declaration
% (the unclean variant).}
%
% \DescribeMacro\theorempreskipamount
% \DescribeMacro\theorempostskipamount Two additional parameters
% affect the vertical space around the theorem environments:
% |\theorempreskipamount| and |\theorempostskipamount| define,
% respectively, the spacing before and after such an environment.
% These parameters apply for all theorem sets and can be manipulated
% with the ordinary length macros.  They are rubber lengths
% (`\textsf{skips}'), and therefore can contain \texttt{plus} and
% \texttt{minus} parts.
%
% Since the definition of theorem sets should---most sensibly---be
% placed in the preamble, we only allow installation there.
% It is therefore possible to release the memory used here after
% |\begin{document}|, in order to make room for other
% applications.
%
%
% \subsection{Existing theorem styles}
%
% The following theorem styles exist to date:
% \begin{deflist}{marginbreak}
%    \item[plain]
%       This theorem style emulates the original \LaTeX{} definition,
%       except that additionally the parameters
%       |\theorem...skipamount| are used.
%    \item[break]
%       In this style, the theorem header is followed by a line
%       break.
%    \item[marginbreak]
%       The theorem number is set in the margin, and there is a line
%       break as in \texttt{break}.
%    \item[changebreak]
%       Like \texttt{break}, but with header number and text
%       interchanged.
%    \item[change]
%       Header number and text are interchanged, without a line break.
%    \item[margin]
%       The number is set in the left margin, without a line break.
% \end{deflist}
% All styles (except \texttt{plain}) select |\slshape| as the default
% |\theorembodyfont|.
%
%
% \subsection{Examples}
%
% Given the above theorem sets {\envfont Cor}, {\envfont Exa}
% and {\envfont Rem},
% suppose that the preamble also contains the declarations:
% \begin{verbatim}
%    \theoremstyle{marginbreak}   \newtheorem{Lem}[Cor]{Lemma}
%    \theoremstyle{change}
%    \theorembodyfont{\itshape}        \newtheorem{Def}[Cor]{Definition}
%
%    \theoremheaderfont{\scshape}
%\end{verbatim}
% Then the following are some typical examples of the typeset output
% resulting from their use.
%
% \begin{Cor}
%    This is a sentence typeset in the theorem environment {\envfont
%    Cor}.
% \end{Cor}
%
% \begin{Exa}
%    This is a sentence typeset in the theorem environment {\envfont
%    Exa}.
% \end{Exa}
%
% \begin{Rem}
%    This is a sentence typeset in the theorem environment {\envfont
%    Rem}.
% \end{Rem}
%
% \begin{Lem}[Ben User]
%    This is a sentence typeset in the theorem environment {\envfont
%    Lem}.
% \end{Lem}
%
% \begin{Def}[Very impressive Definition]
%    This is a sentence typeset in the theorem environment {\envfont
%    Def}.
% \end{Def}
%
% The last two examples show the effect of the optional argument to
% a theorem environment (it is the text typeset in parentheses).
%
%
% \section{Special Considerations}
% \label{sec:newfonts}
%
% Theoremheader and body are implemented as a unit. This means that
% the |\theoremheaderfont|  will inherit characteristics of the
% |\theorembodyfont| in \LaTeXe. Thus, if for example
% |\theorembodyfont| is |\itshape| and |\theoremheaderfont|
% is |\bfseries| the font selected for the header will have the
% characteristics `bold extended italic'. If this is not desired one
% should set the |\theoremheaderfont| to something like
% \begin{verbatim}
%      \theoremheaderfont{\normalfont\bfseries}
%\end{verbatim}
% i.e.\ supplying all necessary font information explicitly.
%
%
%
% \section{Acknowledgements}
%
% The publication of this set of macros was only possible with the
% help of Christina Busse (translating the manuscript into English),
% Joachim Pense (playing the r\^ole of typist), Chris Rowley
% (looking everything over) and many others providing useful
% suggestions.
%
% \MaybeStop{%
%    \begin{thebibliography}{9}
%       \bibitem{bk:GMS94} \textsc{M.~Goossens}, \textsc{F.~Mittelbach}
%         and \textsc{A.~Samarin}.
%         \newblock The \LaTeX{} Companion.
%         \newblock
%         Addison-Wesley, Reading, Massachusetts, 1994.
%       \bibitem{latex:source} \textsc{Lamport, Leslie.} \newblock
%                               \texttt{latex.tex}, version 2.09, date
%                               Feb. 1990.
%    \end{thebibliography}
%    \PrintIndex
%    \PrintChanges}
%
%
% \section{The documentation driver file}
%
% The next bit of code contains the documentation driver file for
% \TeX{}, i.e., the file that will produce the documentation you are
% currently reading. It will be extracted from this file by the
% \texttt{docstrip} program. Since it is the first code in the file
% one can alternatively process this file directly with \LaTeXe{} to
% obtain the documentation.
%
%    \begin{macrocode}
%<*driver>
\documentclass{ltxdoc}

\usepackage{theorem}

%<+driver>% The next few lines define theorem sets which are used
%<+driver>% in the example section of the documentation.

\theoremstyle{break}       \newtheorem{Cor}{Corollary}
\theoremstyle{plain}       \newtheorem{Exa}{Example}[section]
{\theorembodyfont{\upshape}\newtheorem{Rem}{Remark}}
\theoremstyle{marginbreak} \newtheorem{Lem}[Cor]{Lemma}
\theoremstyle{change}
\theorembodyfont{\itshape} \newtheorem{Def}[Cor]{Definition}

\theoremheaderfont{\scshape}

\RecordChanges

\begin{document}
   \DocInput{theorem.dtx}
\end{document}
%</driver>
%    \end{macrocode}
%
% \section{Definition of the Macros}
%
% If the file has been loaded before, we abort immediately. If not the
% package announces itself (this is actually done at the very top of
% the file---the way it is done isn't good style so don't copy it).
% \changes{v2.0e}{89/07/19}{Spaces removed from `typeout'.}
% \changes{v2.2b}{95/11/19}{Announce moved to top of file}
%    \begin{macrocode}
%<*package>
%\@ifundefined{theorem@style}{}{\endinput}
%\def\FMithmInfo{1995/11/19 v2.2b Theorem extension package (FMi)}
%\ProvidesPackage{theorem}[\FMithmInfo]
%    \end{macrocode}
%
%
% \subsection{Definition of theorem styles and fonts}
%
% All the definitions in this file are done globally to allow
% inputting this file inside a group.
%
% \begin{macro}{\theoremstyle}
%    Before a theorem style can be installed, the chosen style must be
%    known. For that reason, we must test to see that
%    |\th@|\meta{style} is known or, more precisely, that it is
%    different from |\relax|. If the style is not known then
%    |\th@plain| is used.
%    \begin{macrocode}
\gdef\theoremstyle#1{%
   \@ifundefined{th@#1}{\@warning
          {Unknown theoremstyle `#1'. Using `plain'}%
          \theorem@style{plain}}%
%    \end{macrocode}
%    We save the theorem style to be used in the token register
%    |\theorem@style|.
%    \begin{macrocode}
      {\theorem@style{#1}}%
%    \end{macrocode}
%    Now we ``evaluate'' the theorem style: this means, we call the
%    macro |\th@|\meta{style} which will activate the relevant
%    definitions which are contained in a separate file.  This is done
%    in a group to suppress changes to the current font.  This could
%    otherwise pose problems together with the new font selection
%    scheme\footnote{When I printed the original article using the new
%    font selection scheme I ended with a document with slanted
%    typefaces (text headings and all) simply because one of the
%    theorem styles used \texttt{\bslash sl} at toplevel.} if the
%    |\th@|\meta{style} is evaluated a second time.
% \changes{v2.1a}{90/02/15}{Group added around `th@..}
% \changes{v2.1a}{90/02/15}{Reseting of `..@begintheorem removed.}
%    \begin{macrocode}
      \begingroup
        \csname th@\the\theorem@style \endcsname
      \endgroup}
%    \end{macrocode}
% \end{macro}
%
%
% \begin{macro}{\@begintheorem}
% \changes{v2.1a}{90/02/15}{Relaxed at toplevel.}
% \begin{macro}{\@opargbegintheorem}
% \changes{v2.1a}{90/02/15}{Relaxed at toplevel.}
%    We reset |\@begintheorem| and |\@opargbegintheorem| to
%    |\relax| since these commands are no longer necessary at
%    toplevel. This will save a few tokens.
%    \begin{macrocode}
\global\let\@begintheorem\relax
\global\let\@opargbegintheorem\relax
%    \end{macrocode}
% \end{macro}
% \end{macro}
%
% \begin{macro}{\theorem@style}
%    Obviously the \textsf{token} register used above has to be
%    allocated. To assure the utmost compatibility with the
%    original \LaTeX{} definition, we set the default theorem style to
%    \texttt{plain}, which implements the usual \LaTeX{} convention.
% \changes{v2.0f}{89/07/20}{Unnecessary `global' removed.}
%    \begin{macrocode}
\newtoks\theorem@style
\global\theorem@style{plain}
%    \end{macrocode}
% \end{macro}
%
% \begin{macro}{\theorembodyfont}
% \begin{macro}{\theorem@bodyfont}
%    If the |\theorembodyfont| is set by the user then it should not
%    interact with the default font set in the theorem style. When the
%    new font selection is in force this may happen if, for example,
%    the default is |\itshape| and the new |\theorembodyfont| is
%    |\sffamily|. So we add a |\reset@font| command in front of the
%    user definition.
% \changes{v2.0f}{89/07/20}{Unnecessary `global' removed.}
% \changes{v2.0g}{89/09/19}{`theoremfont to `theorembodyfont.}
% \changes{v2.1c}{92/06/22}{`reset@font added and logic changed.}
% \changes{v2.1d}{92/10/28}{Reset to empty if arg is empty}
%    \begin{macrocode}
\gdef\theorembodyfont#1{%
%    \end{macrocode}
%    We check if the argument supplied is empty and if so put nothing
%    into the |\theorem@bodyfont| token register to allow for
%    |\theorembodyfont{}| as a mean of using the default of the
%    current |\theoremstyle|.
%    \begin{macrocode}
   \def\@tempa{#1}%
   \ifx\@tempa\@empty
    \theorem@bodyfont{}%
   \else
    \theorem@bodyfont{\reset@font#1}%
   \fi
   }
\newtoks\theorem@bodyfont
\global\theorem@bodyfont{}
%    \end{macrocode}
% \end{macro}
% \end{macro}
%
%
% \begin{macro}{\theoremheaderfont}
% \changes{v2.0g}{89/09/19}{Macro added.}
%    The font for the theorem headers is handled differently because
%    this definition applies to all theorem styles.
%    \begin{macrocode}
\gdef\theoremheaderfont#1{\gdef\theorem@headerfont{#1}%
%    \end{macrocode}
%    After using the macro once it is redefined to produce an error
%    message.
%    \begin{macrocode}
       \gdef\theoremheaderfont##1{%
        \typeout{\string\theoremheaderfont\space should be used
                 only once.}}}
%    \end{macrocode}
% \end{macro}
%
% \begin{macro}{\theorem@headerfont}
% \changes{v2.0g}{89/09/19}{Macro added.}
%    To set the |\theorem@headerfont| default we first test if
%    the new fontselection scheme is in force.
% \changes{v2.1a}{90/02/15}{Support for new font selection.}
%    \begin{macrocode}
\ifx\upshape\undefined
%    \end{macrocode}
%    If not we define it to expand into |\bfseries|. We don't use
%    |\let| just in case a following style option redefines this
%    macro.
%    \begin{macrocode}
\gdef\theorem@headerfont{\bfseries}
%    \end{macrocode}
%    Otherwise we reset the current shape before calling |\bfseries|.
%    \begin{macrocode}
\else \gdef\theorem@headerfont{\normalfont\bfseries}\fi
%    \end{macrocode}
% \end{macro}
%
% \begin{macro}{\th@plain}
% \begin{macro}{\th@break}
% \begin{macro}{\th@marginbreak}
% \begin{macro}{\th@changebreak}
% \begin{macro}{\th@change}
% \begin{macro}{\th@margin}
%    The different styles are defined in macros such as
%    |\th@plain|.  Since memory space is precious in
%    ``non-Big-versions'', we have to avoid offering too many unused
%    definitions.  Therefore we define these styles in separate files
%    that can be loaded on demand.  Thus the commands themselves only
%    load these files. We use |\@input@|, a \LaTeXe{} internal command
%    that ensures that the file will be listed with |\listfiles|.
% \changes{v2.2a}{94/02/02}{Use `@input@ to load theorem layout files}
%    \begin{macrocode}
\gdef\th@plain{\@input@{thp.sty}}
\gdef\th@break{\@input@{thb.sty}}
\gdef\th@marginbreak{\@input@{thmb.sty}}
\gdef\th@changebreak{\@input@{thcb.sty}}
\gdef\th@change{\@input@{thc.sty}}
\gdef\th@margin{\@input@{thm.sty}}
%    \end{macrocode}
%    This list will be expanded when new styles become
%    available.  For testing,  just append new theorem substyles
%    as document options.
% \end{macro}
% \end{macro}
% \end{macro}
% \end{macro}
% \end{macro}
% \end{macro}
%
%
%
% \subsection{Definition of a new theorem set}
%
% As already pointed out, a new theorem environment can be defined in
% three different ways:
% \begin{verbatim}
%    \newtheorem{Lem}{Lemma}
%    \newtheorem{Lem}{Lemma}[section]
%    \newtheorem{Lem}[Theorem]{Lemma}
%\end{verbatim}
% The function of the macro |\newtheorem| is to recognize these cases
% and then to branch into one of the three macros |\@ynthm|, |\@xnthm|
% or |\@othm|.  This mechanism is adopted unchanged from
% \cite{latex:source}; the essential point here is that, for example,
% in the second case, the arguments \texttt{Lem}, \texttt{Lemma} and
% \texttt{section} are passed over to the macro |\@xnthm|.
%
% We inspect this case first because the others present fewer problems,
% and thus are easily derived from this one.
%
% \begin{macro}{\@xnthm}
%    For our example arguments, the macro |\@xnthm| must fulfill
%    the following:
%    \begin{itemize}
%       \item
%          Define a new \LaTeX-counter `Lem'
%       \item
%          reset this counter within a |\section|
%       \item
%          define the macro |\theLem|
%       \item
%          define the environment macros |\Lem| and
%          |\endLem| using the current
%          |\theoremstyle| and |\theorem@bodyfont|.
%    \end{itemize}
%    Obviously, all this should happen only if the first argument
%    of |\@xnthm| (i.e.\ \texttt{Lem} in our example) is chosen so
%    as not to conflict with any previously defined commands or
%    environments.  This test is performed by the \LaTeX{} macro
%    |\@ifdefinable|.
%    \begin{macrocode}
\gdef\@xnthm#1#2[#3]{\expandafter\@ifdefinable\csname #1\endcsname
%    \end{macrocode}
%    Therefore, the first argument of |\@ifdefinable| is the
%    expansion (in the example, |\Lem|) of
%    |\csname||#1\endcsname|.  The second argument is
%    executed only if the test has been completed successfully.
%    \begin{macrocode}
   {%
%    \end{macrocode}
%    Now we define the new counter. The names of the \LaTeX{}
%    macros employed should speak for themselves:
% \changes{v2.2b}{1995/10/07}{Use \cs{@newctr}}
% \changes{v2.2c}{1995/11/23}{Use \cs{@newctr} properly.}
%    \begin{macrocode}
    \@definecounter{#1}\@newctr{#1}[#3]%
%    \end{macrocode}
%    Using |\@newctr| will give a proper error message if the counter
%    in |#3| is not defined.
%    In defining `|\theLem|' we must generate the desired macro
%    name by use of |\expandafter| and |\csname|.
%    \begin{macrocode}
    \expandafter\xdef\csname the#1\endcsname
%    \end{macrocode}
%    An |\xdef| is used in order to make the definition global,
%    and to ensure that it contains the replacement texts of
%    |\@thmcountersep| and |\@thmcounter|.\footnote{These
%    two macros can be defined by the document style.  Their default
%    values produce a `\texttt{.}' as separation and an arabic
%    representation of the number.} However, not everything should be
%    expanded. For example, it saves space to use |\thesection|
%    instead of its---at times---lengthy expansion.
%    \begin{macrocode}
      {\expandafter \noexpand \csname the#3\endcsname
       \@thmcountersep \@thmcounter{#1}}%
%    \end{macrocode}
%    Thus with the defaults of \LaTeX{}, |\theLem| would be
%    replaced by the command sequence |\thesection.\arabic{Lem}|.
%
%    We will now look at the definition of the macro which is executed
%    at the beginning of the actual environment (in our example this
%    macro is |\Lem|). It should be noted that we use an
%    ``|\expandafter| trick'' to expand only certain parts of the
%    replacement text at the time of the definition.
%    \begin{macrocode}
    \def\@tempa{\global\@namedef{#1}}%
    \expandafter \@tempa \expandafter{%
%    \end{macrocode}
%    First, the macro that contains the current definitions of
%    |\@begintheorem| and |\@opargtheorem| should be called
%    up.  The name of this macro---as is already known---has the form
%    |\th@|\meta{theorem\ style}; therefore, it must be called by
%    \begin{macrocode}
      \csname th@\the \theorem@style
%    \end{macrocode}
%    In addition the default theorem font should be changeable, i.e.\
%    we have to insert the contents of |\theorem@bodyfont|. For
%    that reason, we expand even further, beyond |\endcsname|,
%    and thus insert the contents of the \textsf{token} register
%    |\theorem@bodyfont| in the replacement text.
%    \begin{macrocode}
            \expandafter \endcsname \the \theorem@bodyfont
%    \end{macrocode}
%    Now it is time to call the macro |\@thm| which takes over
%    the further processing.  It has two arguments: the current
%    counter name (in our example, \texttt{Lem}), and the text of the
%    label (in our example, \texttt{Lemma}).
%    \begin{macrocode}
     \@thm{#1}{#2}}%
%    \end{macrocode}
%    With this, the `sub-definition' is complete.  The macro
%    |\@endtheorem| ends a theorem environment and is, so far, nothing
%    but an |\endtrivlist|.  (Hence it is defined globally, and not
%    within the theorem styles.\footnote{This has to be changed as
%    soon as theorem styles that change \texttt{\bslash @endtheorem}
%    exist.  In such a case, all existing styles must be changed as
%    well since they will have to reset the macro.}) Therefore, we can
%    set it equivalent to the macro that ends the theorem set (in our
%    example, |\endLem|).  However, if some day theorem styles exist
%    that do change |\@endtheorem|, we would have to use the
%    commented-out line instead.
%    \begin{macrocode}
    \global \expandafter \let \csname end#1\endcsname \@endtheorem
%   \global\@namedef{end#1}{\@endtheorem}%
%    \end{macrocode}
%    With these commands all the required definitions are employed,
%    unless the test |\@ifdefinable| has failed.  Therefore, we
%    end the second argument of this macro and with it the definition
%    of |\@xnthm|.
%    \begin{macrocode}
   }}
%    \end{macrocode}
% \end{macro}
%
% \begin{macro}{\@ynthm}
%    The definition of |\@ynthm| is completely analogous. In
%    this case the new counter that is defined is not reset within
%    another counter; thus the definition of |\the...|
%    is simplified:
%    \begin{macrocode}
\gdef\@ynthm#1#2{\expandafter\@ifdefinable\csname #1\endcsname
   {\@definecounter{#1}%
    \expandafter\xdef\csname the#1\endcsname{\@thmcounter{#1}}%
%    \end{macrocode}
%    The rest of the definition corresponds literally to that of
%    |\@xnthm|:
%    \begin{macrocode}
    \def\@tempa{\global\@namedef{#1}}\expandafter \@tempa
     \expandafter{\csname th@\the \theorem@style \expandafter
     \endcsname \the\theorem@bodyfont \@thm{#1}{#2}}%
    \global \expandafter \let \csname end#1\endcsname \@endtheorem}}
%    \end{macrocode}
% \end{macro}
%
% \begin{macro}{\@othm}
%    The definition of |\@othm| does not contain anything new.
%    \begin{macrocode}
\gdef\@othm#1[#2]#3{%
%    \end{macrocode}
% \changes{v2.2b}{95/10/07}{Build in check for counter defined}
%    We do not define a new counter but instead use one that has
%    already been defined.  Thus the only definition we need is that
%    of this pseudo-counter (i.e.\ |\the|\meta{env.\ name}).
%    First we check if |#2| corresponds to a known counter name.
%    \begin{macrocode}
  \expandafter\ifx\csname c@#2\endcsname\relax
   \@nocounterr{#2}%
  \else
   \expandafter\@ifdefinable\csname #1\endcsname
   {\expandafter \xdef \csname the#1\endcsname
     {\expandafter \noexpand \csname the#2\endcsname}%
%    \end{macrocode}
%    All other parts of the definition can be adopted from
%    |\@xnthm|. We have to remember, though, that in this case
%    the name of the current counter and the theorem label have moved
%    to the second and third arguments.
%    \begin{macrocode}
    \def\@tempa{\global\@namedef{#1}}\expandafter \@tempa
     \expandafter{\csname th@\the \theorem@style \expandafter
     \endcsname \the\theorem@bodyfont \@thm{#2}{#3}}%
    \global \expandafter \let \csname end#1\endcsname \@endtheorem}%
  \fi}
%    \end{macrocode}
% \end{macro}
%
% \subsection{Macros that are employed in a {\envfont theorem}
%    environment}
%
% \begin{macro}{\@thm}
%    The macro |\@thm| has to increase the current counter.  Then,
%    depending on whether the environment has (or does not have) an
%    optional argument, it has to branch into either |\@begintheorem|
%    or |\@opargtheorem|.
%    \begin{macrocode}
\gdef\@thm#1#2{\refstepcounter{#1}%
%    \end{macrocode}
%    Now we start a {\envfont trivlist} environment, and give
%    |\@topsep| and |\@topsepadd| the values of the \textsf{skip}
%    registers |\theorempreskipamount| and |\theorempostskipamount|.
%    The value in |\@topsep| is the vertical space that is inserted by
%    the first (and only) |\item| in our |\trivlist| whilst
%    |\@topsepadd| is inserted by |\@endparenv| at the end of that
%    {\envfont trivlist} environment.  By using these registers, we
%    obtain the desired space around a {\envfont theorem} environment.
%    \begin{macrocode}
   \trivlist
   \@topsep \theorempreskipamount               % used by first \item
   \@topsepadd \theorempostskipamount           % used by \@endparenv
%    \end{macrocode}
%    Now we have to test whether an optional argument has been given.
% \changes{v2.0d}{89/07/18}{Forgotten percent added.}
%    \begin{macrocode}
   \@ifnextchar [%
%    \end{macrocode}
%    If there is an optional argument, we will call |\@ythm|, and
%    move the arguments read back into the input stream.
%    \begin{macrocode}
   {\@ythm{#1}{#2}}%
%    \end{macrocode}
%    If not, we call |\@begintheorem|. Its first argument is the
%    name of the theorem set (hence the second argument of
%    |\@thm|). Its second argument is the macro that produces the
%    current number.
%    \begin{macrocode}
   {\@begintheorem{#2}{\csname the#1\endcsname}\ignorespaces}}
%    \end{macrocode}
% \end{macro}
%
% \begin{macro}{\@xthm}
% \begin{macro}{\@ythm}
%    Both these macros were originally called by |\@thm|.  We do
%    not need |\@xthm| anymore, hence we reset it to
%    |\relax|. The definition of |\@ythm| has not changed at
%    all from its definition in \LaTeX. In order to make the macros
%    easier to understand, we will
%    nevertheless present it (commented out).
%    \begin{macrocode}
\global\let\@xthm\relax
% \def\@ythm#1#2[#3]{\@opargbegintheorem{#2}{\csname
%           the#1\endcsname}{#3}\ignorespaces}
%</package>
%    \end{macrocode}
%    The primitive |\ignorespaces| in |\@ythm| and
%    |\@thm| is needed to remove the spaces between the
%    |\begin{...}| and the actual text.
% \end{macro}
% \end{macro}
%
%
% \subsection{Definition of the theorem substyles}
%
% As already pointed out, the theorem substyles, defined below, are
% only loaded when necessary.  Note that all these substyles, except
% \texttt{plain}, have |\slshape| as the default body font.
%
% \subsubsection{The \texttt{plain} style}
%
% \label{dok:th@plain} As the following macros use |@|, we have
% to locally set the |\catcode| of this symbol to ``letter''.
% This happens within a group, so that we do not have to worry about
% which |\catcode| that symbol had before.
%    \begin{macrocode}
%<*thp>
\begingroup \makeatletter
%    \end{macrocode}
% Since we are now within a group, we must make all definitions
% globally.  First we make sure that \texttt{theorem.sty} is loaded.
% This will allow us to use this file as a document style option
% without having to call \texttt{theorem} itself as an option.  At the
% same time, we assure that at least version 2 is loaded, since
% |\theorem@style| was not defined in earlier versions.
%    \begin{macrocode}
\@ifundefined{theorem@style}{\input{theorem.sty}}{}
\ProvidesFile{thp.sty}
             [\FMithmInfo]
%    \end{macrocode}
%
%
% \begin{macro}{\th@plain}
% \begin{macro}{\@begintheorem}
% \begin{macro}{\@opargbegintheorem}
%    |\theoremstyle{plain}| corresponds to the original
%    definition, except that the distances to the surrounding text are
%    determined by the parameters |\theorempreskipamount| and
%    |\theorempostskipamount|.
%    First we set the default body font.
%    \begin{macrocode}
\gdef\th@plain{\normalfont\itshape
%    \end{macrocode}
%    Then we define |\@begintheorem| and
%    |\@opargbegintheorem|. These two macros define how the
%    header of a theorem is typeset.  |\@opargbegintheorem| will
%    be called if a \textsf{theorem} environment with an optional
%    argument is encountered; otherwise, the header is constructed by
%    calling |\@begintheorem|.  If one of these macros is
%    executed, we are within a \textsf{trivlist} environment started by
%    |\@thm|.  So the theorem header is produced with an
%    |\item| command.
%
%    Instead of specifying the header font directly, all standard
%    theorem styles use the |\theorem@headerfont| macro to allow
%    customization.  The extra space (|\labelsep|) is necessary
%    because of problems in the \textsf{trivlist} environment.
%    \begin{macrocode}
  \def\@begintheorem##1##2{%
        \item[\hskip\labelsep \theorem@headerfont ##1\ ##2]}%
%    \end{macrocode}
%    The definition of |\@opargbegintheorem| is completely
%    analogous. The only difference is the fact that there exists a
%    third argument (which is the optional parameter of the
%    environment and contains additional information about the
%    theorem).  Customarily we enclose it in parentheses.
%    \begin{macrocode}
\def\@opargbegintheorem##1##2##3{%
   \item[\hskip\labelsep \theorem@headerfont ##1\ ##2\ (##3)]}}
%    \end{macrocode}
% \end{macro}
% \end{macro}
% \end{macro}
%
% \noindent
% We conclude with an |\endgroup| to restore the
% |\catcode| of |@|.
%    \begin{macrocode}
\endgroup
%</thp>
%    \end{macrocode}
%
%
% \subsubsection{The \texttt{break} style}
%
% This style option is stored in the file |thb.sty|.
% For the next two lines see the documentation for |\th@plain| on
% page~\pageref{dok:th@plain}.
%    \begin{macrocode}
%<*thb>
\begingroup \makeatletter
\@ifundefined{theorem@style}{\input{theorem.sty}}{}
\ProvidesFile{thb.sty}
             [\FMithmInfo]
%    \end{macrocode}
%
% \begin{macro}{\th@break}
%    |\theoremstyle{break}| produces a line break after the name
%    of the theorem.  The font is |\slshape|.  Hence, we define
%    |\th@break| as follows:
%    \begin{macrocode}
\gdef\th@break{\normalfont\slshape
  \def\@begintheorem##1##2{\item[%
%    \end{macrocode}
%    We run into the following problem: it is not possible to create
%    the header with |\item[|\meta{title}|]| and then start
%    a new line by, for example, |\mbox{}\\|. Such a definition
%    will fail whenever a list environment follows immediately.  With
%    the above construction, the |\mbox{}| causes the switch
%    |@inlabel| (cf.\ definition of |\list| and
%    |\trivlist| in \cite{latex:source}) to be set to \texttt{false}
%    and so the following list will insert additional vertical
%    space (|\topskip|). This is quite annoying. Therefore, we
%    create the line break within the |\item|. In order to ensure
%    that the text will begin at the proper position in the following
%    line, we simply pretend that the label does not take any
%    room.\footnote{This will lead to problems whenever very high
%    symbols occurring in the line tower into the
%    heading. So, something else has to be done here sometime.}
%    \begin{macrocode}
       \rlap{\vbox{\hbox{\hskip \labelsep\theorem@headerfont ##1\ ##2}%
                   \hbox{\strut}}}]}%
%    \end{macrocode}
%    Again, the definition of |\@opargbegintheorem| is
%    completely analogous.
%    \begin{macrocode}
\def\@opargbegintheorem##1##2##3{%
  \item[\rlap{\vbox{\hbox{\hskip \labelsep \theorem@headerfont
                     ##1\ ##2\ (##3)}%
                    \hbox{\strut}}}]}}
\endgroup
%</thb>
%    \end{macrocode}
% \end{macro}
%
%
% \subsubsection{The \texttt{changebreak} style}
%
%    \begin{macrocode}
% This style option is stored in the file |thcb.sty|.
%    \begin{macrocode}
%<*thcb>
\begingroup \makeatletter
\@ifundefined{theorem@style}{\input{theorem.sty}}{}
\ProvidesFile{thcb.sty}
             [\FMithmInfo]
%    \end{macrocode}
%
% \begin{macro}{\th@changebreak}
%    The \texttt{change-break} theorem style is like \texttt{break}
%    but with interchange of theorem name and theorem number.  Thus we
%    define |\th@changebreak| as follows:
%    \begin{macrocode}
\gdef\th@changebreak{\normalfont\slshape
  \def\@begintheorem##1##2{\item
       [\rlap{\vbox{\hbox{\hskip\labelsep \theorem@headerfont ##2\ ##1}%
                    \hbox{\strut}}}]}%
\def\@opargbegintheorem##1##2##3{%
  \item[\rlap{\vbox{\hbox{\hskip\labelsep \theorem@headerfont
                     ##2\ ##1\ (##3)}%
                    \hbox{\strut}}}]}}
\endgroup
%</thcb>
%    \end{macrocode}
% \end{macro}
%
%
% \subsubsection{The \texttt{change} style}
%
% This style option is stored in the file |thc.sty|.
%    \begin{macrocode}
%<*thc>
\begingroup \makeatletter
\@ifundefined{theorem@style}{\input{theorem.sty}}{}
\ProvidesFile{thc.sty}
             [\FMithmInfo]
%    \end{macrocode}
%
% \begin{macro}{\th@change}
%    The \texttt{change} theorem style corresponds to the
%    \texttt{change break} style without a linebreak after the header.
%    To say it in another way, it's the same as the \texttt{plain}
%    style but with number and name interchanged and |\slshape| as the
%    default font.
%    \begin{macrocode}
\gdef\th@change{\normalfont\slshape
  \def\@begintheorem##1##2{\item
       [\hskip\labelsep \theorem@headerfont ##2\ ##1]}%
\def\@opargbegintheorem##1##2##3{%
  \item[\hskip\labelsep \theorem@headerfont ##2\ ##1\ (##3)]}}
\endgroup
%</thc>
%    \end{macrocode}
% \end{macro}
%
%
% \subsubsection{The \texttt{marginbreak} style}
%
% This style option is the one used most often at Mainz.  It is saved
% in the file |thmb.sty|.
%    \begin{macrocode}
%<*thmb>
\begingroup \makeatletter
\@ifundefined{theorem@style}{\input{theorem.sty}}{}
\ProvidesFile{thmb.sty}
             [\FMithmInfo]
%    \end{macrocode}
%
% \begin{macro}{\th@marginbreak}
%    The \texttt{margin break} style is nearly the same as the
%    \texttt{change break} style. The only difference is the placement
%    of the theorem number.  We use |\llap| to place it in the left
%    margin.
%
%    In this style |\labelsep| denotes the separation between the
%    number and the text.
%    \begin{macrocode}
\gdef\th@marginbreak{\normalfont\slshape
  \def\@begintheorem##1##2{\item
       [\rlap{\vbox{\theorem@headerfont
                    \hbox{\llap{##2}\hskip\labelsep ##1}%
                    \hbox{\strut}}}]}%
\def\@opargbegintheorem##1##2##3{%
  \item[\rlap{\vbox{\theorem@headerfont
                    \hbox{\llap{##2}\hskip\labelsep ##1\ (##3)}%
                    \hbox{\strut}}}]}}
\endgroup
%</thmb>
%    \end{macrocode}
% \changes{v2.1c}{90/02/17}{`endgroup outside macrocode env.}
% \end{macro}
%
%
%
% \subsubsection{The \texttt{margin} style}
%
% This style option is stored in the file |thm.sty|.
%    \begin{macrocode}
%<*thm>
\begingroup \makeatletter
\@ifundefined{theorem@style}{\input{theorem.sty}}{}
\ProvidesFile{thm.sty}
             [\FMithmInfo]
%    \end{macrocode}
%
% \begin{macro}{\th@margin}
%    Again this is only a variant of the theorem styles described
%    above without any new ideas.
%    \begin{macrocode}
\gdef\th@margin{\normalfont\slshape
  \def\@begintheorem##1##2{\item
       [\theorem@headerfont \llap{##2}\hskip\labelsep ##1]}%
\def\@opargbegintheorem##1##2##3{%
  \item[\theorem@headerfont \llap{##2}\hskip\labelsep ##1\ (##3)]}}
\endgroup
%</thm>
%    \end{macrocode}
% \end{macro}
%
%
%
%
% \subsection{Final Definitions}
%
% \begin{macro}{\theorempreskipamount}
% \begin{macro}{\theorempostskipamount}
%    The \textsf{skip} parameters that regulate the vertical empty space
%    before and after the {\envfont theorem} environment have
%    to be allocated as well.
% \changes{v2.0f}{89/07/20}{Unnecessary `global' removed.}
%    \begin{macrocode}
%<*package>
\newskip\theorempreskipamount
\newskip\theorempostskipamount
%    \end{macrocode}
%    Since we have used the same values for all theorem sets, we now can
%    assign them.
%    \begin{macrocode}
\global\setlength\theorempreskipamount{12pt plus 5pt minus 3pt}
\global\setlength\theorempostskipamount{8pt plus 3pt minus 1.5pt}
%    \end{macrocode}
% \end{macro}
% \end{macro}
%
%
% \begin{macro}{\@endtheorem}
%    The same holds for the macro |\@endtheorem|, which ends a
%    {\envfont theorem} environment. Since it is the same for all
%    theorem sets, it is removed from the macros |\th@|\meta{style}.
%    It simply ends the {\envfont trivlist} environment, which was
%    begun in |\@thm|.
%    \begin{macrocode}
\global\let\@endtheorem=\endtrivlist
%    \end{macrocode}
% \end{macro}
%
%
%
% \begin{macro}{\@preamblecmds}
%    All macros defined above are to be used only in the preamble.
%    Therefore, we insert them in |\@preamblecmds| which will disable
%    them at begin document. This is done by the internal \LaTeXe{}
%    command |\@onlypreamble|.
% \changes{v2.0g}{89/09/19}{`theoremheaderfont and `theorembodyfont
%    added.}
% \changes{v2.2a}{94/02/03}{Use \LaTeXe{} interface}
%    \begin{macrocode}
\@onlypreamble\@xnthm
\@onlypreamble\@ynthm
\@onlypreamble\@othm
\@onlypreamble\newtheorem
\@onlypreamble\theoremstyle
\@onlypreamble\theorembodyfont
\@onlypreamble\theoremheaderfont
%    \end{macrocode}
% \end{macro}
%
% Finally we declare the \texttt{plain} theorem style to be the default.
% \changes{v2.1b}{90/06/04}{Fixed missing `item by declaring default
%    style.}
%    \begin{macrocode}
\theoremstyle{plain}
%</package>
%    \end{macrocode}
%
% \Finale

\endinput
}{}
\ProvidesFile{thm.sty}
             [\FMithmInfo]
%    \end{macrocode}
%
% \begin{macro}{\th@margin}
%    Again this is only a variant of the theorem styles described
%    above without any new ideas.
%    \begin{macrocode}
\gdef\th@margin{\normalfont\slshape
  \def\@begintheorem##1##2{\item
       [\theorem@headerfont \llap{##2}\hskip\labelsep ##1]}%
\def\@opargbegintheorem##1##2##3{%
  \item[\theorem@headerfont \llap{##2}\hskip\labelsep ##1\ (##3)]}}
\endgroup
%</thm>
%    \end{macrocode}
% \end{macro}
%
%
%
%
% \subsection{Final Definitions}
%
% \begin{macro}{\theorempreskipamount}
% \begin{macro}{\theorempostskipamount}
%    The \textsf{skip} parameters that regulate the vertical empty space
%    before and after the {\envfont theorem} environment have
%    to be allocated as well.
% \changes{v2.0f}{89/07/20}{Unnecessary `global' removed.}
%    \begin{macrocode}
%<*package>
\newskip\theorempreskipamount
\newskip\theorempostskipamount
%    \end{macrocode}
%    Since we have used the same values for all theorem sets, we now can
%    assign them.
%    \begin{macrocode}
\global\setlength\theorempreskipamount{12pt plus 5pt minus 3pt}
\global\setlength\theorempostskipamount{8pt plus 3pt minus 1.5pt}
%    \end{macrocode}
% \end{macro}
% \end{macro}
%
%
% \begin{macro}{\@endtheorem}
%    The same holds for the macro |\@endtheorem|, which ends a
%    {\envfont theorem} environment. Since it is the same for all
%    theorem sets, it is removed from the macros |\th@|\meta{style}.
%    It simply ends the {\envfont trivlist} environment, which was
%    begun in |\@thm|.
%    \begin{macrocode}
\global\let\@endtheorem=\endtrivlist
%    \end{macrocode}
% \end{macro}
%
%
%
% \begin{macro}{\@preamblecmds}
%    All macros defined above are to be used only in the preamble.
%    Therefore, we insert them in |\@preamblecmds| which will disable
%    them at begin document. This is done by the internal \LaTeXe{}
%    command |\@onlypreamble|.
% \changes{v2.0g}{89/09/19}{`theoremheaderfont and `theorembodyfont
%    added.}
% \changes{v2.2a}{94/02/03}{Use \LaTeXe{} interface}
%    \begin{macrocode}
\@onlypreamble\@xnthm
\@onlypreamble\@ynthm
\@onlypreamble\@othm
\@onlypreamble\newtheorem
\@onlypreamble\theoremstyle
\@onlypreamble\theorembodyfont
\@onlypreamble\theoremheaderfont
%    \end{macrocode}
% \end{macro}
%
% Finally we declare the \texttt{plain} theorem style to be the default.
% \changes{v2.1b}{90/06/04}{Fixed missing `item by declaring default
%    style.}
%    \begin{macrocode}
\theoremstyle{plain}
%</package>
%    \end{macrocode}
%
% \Finale

\endinput
}{}
\ProvidesFile{thb.sty}
             [\FMithmInfo]
%    \end{macrocode}
%
% \begin{macro}{\th@break}
%    |\theoremstyle{break}| produces a line break after the name
%    of the theorem.  The font is |\slshape|.  Hence, we define
%    |\th@break| as follows:
%    \begin{macrocode}
\gdef\th@break{\normalfont\slshape
  \def\@begintheorem##1##2{\item[%
%    \end{macrocode}
%    We run into the following problem: it is not possible to create
%    the header with |\item[|\meta{title}|]| and then start
%    a new line by, for example, |\mbox{}\\|. Such a definition
%    will fail whenever a list environment follows immediately.  With
%    the above construction, the |\mbox{}| causes the switch
%    |@inlabel| (cf.\ definition of |\list| and
%    |\trivlist| in \cite{latex:source}) to be set to \texttt{false}
%    and so the following list will insert additional vertical
%    space (|\topskip|). This is quite annoying. Therefore, we
%    create the line break within the |\item|. In order to ensure
%    that the text will begin at the proper position in the following
%    line, we simply pretend that the label does not take any
%    room.\footnote{This will lead to problems whenever very high
%    symbols occurring in the line tower into the
%    heading. So, something else has to be done here sometime.}
%    \begin{macrocode}
       \rlap{\vbox{\hbox{\hskip \labelsep\theorem@headerfont ##1\ ##2}%
                   \hbox{\strut}}}]}%
%    \end{macrocode}
%    Again, the definition of |\@opargbegintheorem| is
%    completely analogous.
%    \begin{macrocode}
\def\@opargbegintheorem##1##2##3{%
  \item[\rlap{\vbox{\hbox{\hskip \labelsep \theorem@headerfont
                     ##1\ ##2\ (##3)}%
                    \hbox{\strut}}}]}}
\endgroup
%</thb>
%    \end{macrocode}
% \end{macro}
%
%
% \subsubsection{The \texttt{changebreak} style}
%
%    \begin{macrocode}
% This style option is stored in the file |thcb.sty|.
%    \begin{macrocode}
%<*thcb>
\begingroup \makeatletter
\@ifundefined{theorem@style}{% \iffalse meta-comment
%
% Copyright (C) 1993-2024
%
% The LaTeX Project and any individual authors listed elsewhere
% in this file.
%
% This file is part of the Standard LaTeX `Tools Bundle'.
% -------------------------------------------------------
%
% It may be distributed and/or modified under the
% conditions of the LaTeX Project Public License, either version 1.3c
% of this license or (at your option) any later version.
% The latest version of this license is in
%    https://www.latex-project.org/lppl.txt
% and version 1.3c or later is part of all distributions of LaTeX
% version 2005/12/01 or later.
%
% The list of all files belonging to the LaTeX `Tools Bundle' is
% given in the file `manifest.txt'.
%
% \fi
%
%%
%
%
\def\FMithmInfo{2023/07/05 v2.2c Theorem extension package (FMi)}
%
% \ProvidesFile{theorem.dtx}[\FMithmInfo]
%\iffalse   % this is a METACOMMENT !
%%
%% Package `theorem' to use with LaTeX2e
%% Copyright (C) 1989-1995 Frank Mittelbach, all rights reserved.
%<+package>\@ifundefined{theorem@style}{}{\endinput}
%<+package>\NeedsTeXFormat{LaTeX2e}[1995/06/01]
%<+package>\ProvidesPackage{theorem}
%<+package>         [\FMithmInfo]
%
% \fi
%
% \changes{v1.0}{87/05/07}{First draft.}
% \changes{v1.0a}{87/05/07}{`thistheoremstyle' added.}
% \changes{v1.1a}{87/07/29}{New concept: redefine the internal
%                          \protect\LaTeX{} commands `@xnthm' etc.}
% \changes{v1.1b}{87/07/31}{`theoremfont' used to specify font.}
% \changes{v1.2a}{88/01/08}{`@endtheorem' now globally defined.}
% \changes{v1.2b}{88/01/11}{`newtheorem' only allowed in the preamble.}
% \changes{v1.3a}{88/02/14}{`theoremfont' now a toks register.}
%
% \changes{v2.0a}{88/11/16}{New implementation and new documentation.}
% \changes{v2.0a}{88/11/16}{No layout changes but internal modifications
%                           old substyles won't work.}
% \changes{v2.0b}{89/01/30}{`global' used for all definitions.}
% \changes{v2.0c}{89/06/15}{Documentation translated to English.
%                            Thanks to LLL and Nina}
% \changes{v2.1a}{90/02/15}{Support for new font selection.}
% \changes{v2.2a}{94/02/03}{Cleanup for LaTeX2e}
%
% \DoNotIndex{\@endparenv,\@ifdefinable,\@ifnextchar,\@ifundefined}
% \DoNotIndex{\@namedef,\@nameuse,\@spaces,\@tempa,\@warning}
% \DoNotIndex{\ }
% \DoNotIndex{\begingroup,\bfseries}
% \DoNotIndex{\csname}
% \DoNotIndex{\def,\do,\docdate}
% \DoNotIndex{\endcsname,\endgroup,\endinput,\endtrivlist,\expandafter}
% \DoNotIndex{\filedate,\fileversion}
% \DoNotIndex{\gdef,\global}
% \DoNotIndex{\hbox,\hskip}
% \DoNotIndex{\ignorespaces,\input,\itshape,\item}
% \DoNotIndex{\labelsep,\let,\llap}
% \DoNotIndex{\makeatletter}
% \DoNotIndex{\noexpand}
% \DoNotIndex{\refstepcounter,\relax,\rlap}
% \DoNotIndex{\setlength,\slshape,\space,\string,\strut}
% \DoNotIndex{\the,\trivlist,\typeout}
% \DoNotIndex{\vbox}
% \DoNotIndex{\xdef}
%
% \def\deflabel#1{\ttfamily #1\hfill}
% \def\deflist#1{\begin{list}{}{\settowidth\labelwidth{\ttfamily #1}%
%                               \setlength\leftmargin\labelwidth
%                               \addtolength\leftmargin\labelsep
%                               \let\makelabel\deflabel}}
% \def\enddeflist{\end{list}}
% \def\envfont{\normalfont\ttfamily}
%
% \MakeShortVerb{\|}
%
% \newcommand{\TUB}{{\sl TUGboat\/}}
%
% \GetFileInfo{theorem.dtx}
% \title{An Extension of the
%        \LaTeX{} theorem environment\thanks{This file
%        has version number \fileversion, last
%        revised \filedate.}}
% \author{Frank Mittelbach}
%
% \MaintainedByLaTeXTeam{tools}
% \maketitle
%
% \begin{abstract}
%    The macros described in this paper yield an extension of the
%    \LaTeX{} theorem mechanism. It is designed to satisfy the
%    different requirements of various journals. Thus, the
%    layout of the ``theorems'' can be manipulated by determining a
%    ``style''. This article describes not only the use, but
%    also the definition, of the necessary macros.
% \end{abstract}
%
% \section*{Preface to version 2.2}
%
% For \LaTeXe{} this package did not need any fundamental changes. I
% only modified the messages generated so that theorem layout styles
% will show up with the |\listfiles| command and cleaned the section
% on the New Font Selection Scheme since this is now included in
% \LaTeX.
%
% \section*{Preface to version 2.1}
%
% This version is identical to 2.0g described in \TUB\ 10\#3 except
% for some internal defaults which are now set depending on the used
% font selection scheme.
%
% This was done to avoid unpleasant surprises if the new font selection
% scheme is in force. For further details see
% section~\ref{sec:newfonts} and \cite{bk:GMS94}.
%
% \section{Introduction}
%
% For our purposes here, ``theorems'' are labelled enunciations,
% often set off from the main text by extra space and a font change.
% Theorems, corollaries, conjectures, definitions, and remarks
% are all instances of ``theorems''.  The ``header'' of these structures
% is composed of a label (such as \textsc{Theorem} or \textsc{Remark})
% and a number which serializes an item in the sequence of items
% with the same label.
%
% Shortly after the introduction of \LaTeX{} at the Fachbereich
% Mathematik in Mainz, the desire to manipulate the layout of
% ``theorems'' arose.  In Mainz, the following two conventions came
% into general use:
% \begin{enumerate}
%   \item The number of the theorem is shown in the margin.
%   \item There is a line break at
%         the end of the theorem header.
% \end{enumerate}
%
% Additionally, some journals require different formats which depend on
% the ``sort of theorem'': e.g.\ often remarks and definitions are set
% in |\upshape|, while |\itshape| is employed for main theorems.
%
% Confronted with these requirements, a theorem environment was
% developed in Mainz which allows separate determination of the
% layout of the ``theorems sets'', comparable to |\pagestyle|.
%
%
% \section{The user interface}
% \subsection{Defining new theorem sets}
%
% \DescribeMacro\newtheorem
% As in the original \LaTeX{} version, the command |\newtheorem|
% defines a new ``theorem set'' or ``theorem-like structure''.
% Two required arguments name the new environment and give the text
% to be typeset with each instance of the new ``set'', while
% an optional argument determines how
% the ``set'' is enumerated:
% \begin{description}
%    \item[\ttfamily \bslash newtheorem\{foo\}\{bar\}]
%       The theorem set {\envfont foo} (whose name is \texttt{bar})
%       uses its own counter.
%    \item[\ttfamily \bslash newtheorem\{foo2\}{[foo]}\{bar2\}]
%       The theorem set {\envfont foo2} (printed name \texttt{bar2})
%       uses the same counter as the theorem set \texttt{foo}.
%    \item[\ttfamily \bslash newtheorem\{foo3\}\{bar3\}{[section]}]
%       The theorem set {\envfont foo3} (printed name \texttt{bar3}) is
%       enumerated within the counter \texttt{section}, i.e.\ with every
%       new |\section| the enumeration begins again with 1, and
%       the enumeration is composed from the section-number and the
%       theorem counter itself.
% \end{description}
%
% \DescribeMacro\theoremstyle
% Additionally, the command |\theoremstyle| can define the layout
% of various, or all, theorem sets.  It should be noted that any
% theorem set defined by |\newtheorem| is typeset in the
% |\theoremstyle| that is current at the time of the definition.
% Thus, the following
% \begin{verbatim}
%    \theoremstyle{break}        \newtheorem{Cor}{Corollary}
%    \theoremstyle{plain}        \newtheorem{Exa}{Example}[section]
%\end{verbatim}                             ^^A No blank at line start
% leads to the result that the set {\envfont Cor} is formatted in the
% style \texttt{break}, while the set {\envfont Exa} and all the
% following ones are formatted in the style \texttt{plain}, unless
% another |\theoremstyle| follows.  Since the definitions installed by
% |\newtheorem| are global, one also can limit |\theoremstyle| locally
% by grouping braces.
%
% \DescribeMacro\theorembodyfont
% The choice of the font for the theorem body is completely
% independent of the chosen |\theoremstyle|; this has proven to
% be very advantageous.  For example,
% \begin{verbatim}
%   {\theorembodyfont{\upshape}          \newtheorem{Rem}{Remark}}
%\end{verbatim}
% defines a theorem set {\envfont Rem}, which will be set in
% |\upshape| in the current layout (which in our example is
% \texttt{plain}).  As with |\theoremstyle|, the |\theorembodyfont|
% chosen is that current at the time of |\newtheorem|.  If
% |\theorembodyfont| is not specified or one defines
% |\theorembodyfont{}|, then the font used will be that defined by the
% |\theoremstyle|.
%
% \DescribeMacro\theoremheaderfont It is also possible to customize
% the font used for the theorem headers.  This is, however, a global
% declaration, and therefore there should be at most one
% |\theoremheaderfont| declaration in the preamble.\footnote{If it is
% actually necessary to have different header fonts, one has to define
% new theorem styles (substituting the desired font) or specify the
% information directly in the \texttt{\bslash newtheorem} declaration
% (the unclean variant).}
%
% \DescribeMacro\theorempreskipamount
% \DescribeMacro\theorempostskipamount Two additional parameters
% affect the vertical space around the theorem environments:
% |\theorempreskipamount| and |\theorempostskipamount| define,
% respectively, the spacing before and after such an environment.
% These parameters apply for all theorem sets and can be manipulated
% with the ordinary length macros.  They are rubber lengths
% (`\textsf{skips}'), and therefore can contain \texttt{plus} and
% \texttt{minus} parts.
%
% Since the definition of theorem sets should---most sensibly---be
% placed in the preamble, we only allow installation there.
% It is therefore possible to release the memory used here after
% |\begin{document}|, in order to make room for other
% applications.
%
%
% \subsection{Existing theorem styles}
%
% The following theorem styles exist to date:
% \begin{deflist}{marginbreak}
%    \item[plain]
%       This theorem style emulates the original \LaTeX{} definition,
%       except that additionally the parameters
%       |\theorem...skipamount| are used.
%    \item[break]
%       In this style, the theorem header is followed by a line
%       break.
%    \item[marginbreak]
%       The theorem number is set in the margin, and there is a line
%       break as in \texttt{break}.
%    \item[changebreak]
%       Like \texttt{break}, but with header number and text
%       interchanged.
%    \item[change]
%       Header number and text are interchanged, without a line break.
%    \item[margin]
%       The number is set in the left margin, without a line break.
% \end{deflist}
% All styles (except \texttt{plain}) select |\slshape| as the default
% |\theorembodyfont|.
%
%
% \subsection{Examples}
%
% Given the above theorem sets {\envfont Cor}, {\envfont Exa}
% and {\envfont Rem},
% suppose that the preamble also contains the declarations:
% \begin{verbatim}
%    \theoremstyle{marginbreak}   \newtheorem{Lem}[Cor]{Lemma}
%    \theoremstyle{change}
%    \theorembodyfont{\itshape}        \newtheorem{Def}[Cor]{Definition}
%
%    \theoremheaderfont{\scshape}
%\end{verbatim}
% Then the following are some typical examples of the typeset output
% resulting from their use.
%
% \begin{Cor}
%    This is a sentence typeset in the theorem environment {\envfont
%    Cor}.
% \end{Cor}
%
% \begin{Exa}
%    This is a sentence typeset in the theorem environment {\envfont
%    Exa}.
% \end{Exa}
%
% \begin{Rem}
%    This is a sentence typeset in the theorem environment {\envfont
%    Rem}.
% \end{Rem}
%
% \begin{Lem}[Ben User]
%    This is a sentence typeset in the theorem environment {\envfont
%    Lem}.
% \end{Lem}
%
% \begin{Def}[Very impressive Definition]
%    This is a sentence typeset in the theorem environment {\envfont
%    Def}.
% \end{Def}
%
% The last two examples show the effect of the optional argument to
% a theorem environment (it is the text typeset in parentheses).
%
%
% \section{Special Considerations}
% \label{sec:newfonts}
%
% Theoremheader and body are implemented as a unit. This means that
% the |\theoremheaderfont|  will inherit characteristics of the
% |\theorembodyfont| in \LaTeXe. Thus, if for example
% |\theorembodyfont| is |\itshape| and |\theoremheaderfont|
% is |\bfseries| the font selected for the header will have the
% characteristics `bold extended italic'. If this is not desired one
% should set the |\theoremheaderfont| to something like
% \begin{verbatim}
%      \theoremheaderfont{\normalfont\bfseries}
%\end{verbatim}
% i.e.\ supplying all necessary font information explicitly.
%
%
%
% \section{Acknowledgements}
%
% The publication of this set of macros was only possible with the
% help of Christina Busse (translating the manuscript into English),
% Joachim Pense (playing the r\^ole of typist), Chris Rowley
% (looking everything over) and many others providing useful
% suggestions.
%
% \MaybeStop{%
%    \begin{thebibliography}{9}
%       \bibitem{bk:GMS94} \textsc{M.~Goossens}, \textsc{F.~Mittelbach}
%         and \textsc{A.~Samarin}.
%         \newblock The \LaTeX{} Companion.
%         \newblock
%         Addison-Wesley, Reading, Massachusetts, 1994.
%       \bibitem{latex:source} \textsc{Lamport, Leslie.} \newblock
%                               \texttt{latex.tex}, version 2.09, date
%                               Feb. 1990.
%    \end{thebibliography}
%    \PrintIndex
%    \PrintChanges}
%
%
% \section{The documentation driver file}
%
% The next bit of code contains the documentation driver file for
% \TeX{}, i.e., the file that will produce the documentation you are
% currently reading. It will be extracted from this file by the
% \texttt{docstrip} program. Since it is the first code in the file
% one can alternatively process this file directly with \LaTeXe{} to
% obtain the documentation.
%
%    \begin{macrocode}
%<*driver>
\documentclass{ltxdoc}

\usepackage{theorem}

%<+driver>% The next few lines define theorem sets which are used
%<+driver>% in the example section of the documentation.

\theoremstyle{break}       \newtheorem{Cor}{Corollary}
\theoremstyle{plain}       \newtheorem{Exa}{Example}[section]
{\theorembodyfont{\upshape}\newtheorem{Rem}{Remark}}
\theoremstyle{marginbreak} \newtheorem{Lem}[Cor]{Lemma}
\theoremstyle{change}
\theorembodyfont{\itshape} \newtheorem{Def}[Cor]{Definition}

\theoremheaderfont{\scshape}

\RecordChanges

\begin{document}
   \DocInput{theorem.dtx}
\end{document}
%</driver>
%    \end{macrocode}
%
% \section{Definition of the Macros}
%
% If the file has been loaded before, we abort immediately. If not the
% package announces itself (this is actually done at the very top of
% the file---the way it is done isn't good style so don't copy it).
% \changes{v2.0e}{89/07/19}{Spaces removed from `typeout'.}
% \changes{v2.2b}{95/11/19}{Announce moved to top of file}
%    \begin{macrocode}
%<*package>
%\@ifundefined{theorem@style}{}{\endinput}
%\def\FMithmInfo{1995/11/19 v2.2b Theorem extension package (FMi)}
%\ProvidesPackage{theorem}[\FMithmInfo]
%    \end{macrocode}
%
%
% \subsection{Definition of theorem styles and fonts}
%
% All the definitions in this file are done globally to allow
% inputting this file inside a group.
%
% \begin{macro}{\theoremstyle}
%    Before a theorem style can be installed, the chosen style must be
%    known. For that reason, we must test to see that
%    |\th@|\meta{style} is known or, more precisely, that it is
%    different from |\relax|. If the style is not known then
%    |\th@plain| is used.
%    \begin{macrocode}
\gdef\theoremstyle#1{%
   \@ifundefined{th@#1}{\@warning
          {Unknown theoremstyle `#1'. Using `plain'}%
          \theorem@style{plain}}%
%    \end{macrocode}
%    We save the theorem style to be used in the token register
%    |\theorem@style|.
%    \begin{macrocode}
      {\theorem@style{#1}}%
%    \end{macrocode}
%    Now we ``evaluate'' the theorem style: this means, we call the
%    macro |\th@|\meta{style} which will activate the relevant
%    definitions which are contained in a separate file.  This is done
%    in a group to suppress changes to the current font.  This could
%    otherwise pose problems together with the new font selection
%    scheme\footnote{When I printed the original article using the new
%    font selection scheme I ended with a document with slanted
%    typefaces (text headings and all) simply because one of the
%    theorem styles used \texttt{\bslash sl} at toplevel.} if the
%    |\th@|\meta{style} is evaluated a second time.
% \changes{v2.1a}{90/02/15}{Group added around `th@..}
% \changes{v2.1a}{90/02/15}{Reseting of `..@begintheorem removed.}
%    \begin{macrocode}
      \begingroup
        \csname th@\the\theorem@style \endcsname
      \endgroup}
%    \end{macrocode}
% \end{macro}
%
%
% \begin{macro}{\@begintheorem}
% \changes{v2.1a}{90/02/15}{Relaxed at toplevel.}
% \begin{macro}{\@opargbegintheorem}
% \changes{v2.1a}{90/02/15}{Relaxed at toplevel.}
%    We reset |\@begintheorem| and |\@opargbegintheorem| to
%    |\relax| since these commands are no longer necessary at
%    toplevel. This will save a few tokens.
%    \begin{macrocode}
\global\let\@begintheorem\relax
\global\let\@opargbegintheorem\relax
%    \end{macrocode}
% \end{macro}
% \end{macro}
%
% \begin{macro}{\theorem@style}
%    Obviously the \textsf{token} register used above has to be
%    allocated. To assure the utmost compatibility with the
%    original \LaTeX{} definition, we set the default theorem style to
%    \texttt{plain}, which implements the usual \LaTeX{} convention.
% \changes{v2.0f}{89/07/20}{Unnecessary `global' removed.}
%    \begin{macrocode}
\newtoks\theorem@style
\global\theorem@style{plain}
%    \end{macrocode}
% \end{macro}
%
% \begin{macro}{\theorembodyfont}
% \begin{macro}{\theorem@bodyfont}
%    If the |\theorembodyfont| is set by the user then it should not
%    interact with the default font set in the theorem style. When the
%    new font selection is in force this may happen if, for example,
%    the default is |\itshape| and the new |\theorembodyfont| is
%    |\sffamily|. So we add a |\reset@font| command in front of the
%    user definition.
% \changes{v2.0f}{89/07/20}{Unnecessary `global' removed.}
% \changes{v2.0g}{89/09/19}{`theoremfont to `theorembodyfont.}
% \changes{v2.1c}{92/06/22}{`reset@font added and logic changed.}
% \changes{v2.1d}{92/10/28}{Reset to empty if arg is empty}
%    \begin{macrocode}
\gdef\theorembodyfont#1{%
%    \end{macrocode}
%    We check if the argument supplied is empty and if so put nothing
%    into the |\theorem@bodyfont| token register to allow for
%    |\theorembodyfont{}| as a mean of using the default of the
%    current |\theoremstyle|.
%    \begin{macrocode}
   \def\@tempa{#1}%
   \ifx\@tempa\@empty
    \theorem@bodyfont{}%
   \else
    \theorem@bodyfont{\reset@font#1}%
   \fi
   }
\newtoks\theorem@bodyfont
\global\theorem@bodyfont{}
%    \end{macrocode}
% \end{macro}
% \end{macro}
%
%
% \begin{macro}{\theoremheaderfont}
% \changes{v2.0g}{89/09/19}{Macro added.}
%    The font for the theorem headers is handled differently because
%    this definition applies to all theorem styles.
%    \begin{macrocode}
\gdef\theoremheaderfont#1{\gdef\theorem@headerfont{#1}%
%    \end{macrocode}
%    After using the macro once it is redefined to produce an error
%    message.
%    \begin{macrocode}
       \gdef\theoremheaderfont##1{%
        \typeout{\string\theoremheaderfont\space should be used
                 only once.}}}
%    \end{macrocode}
% \end{macro}
%
% \begin{macro}{\theorem@headerfont}
% \changes{v2.0g}{89/09/19}{Macro added.}
%    To set the |\theorem@headerfont| default we first test if
%    the new fontselection scheme is in force.
% \changes{v2.1a}{90/02/15}{Support for new font selection.}
%    \begin{macrocode}
\ifx\upshape\undefined
%    \end{macrocode}
%    If not we define it to expand into |\bfseries|. We don't use
%    |\let| just in case a following style option redefines this
%    macro.
%    \begin{macrocode}
\gdef\theorem@headerfont{\bfseries}
%    \end{macrocode}
%    Otherwise we reset the current shape before calling |\bfseries|.
%    \begin{macrocode}
\else \gdef\theorem@headerfont{\normalfont\bfseries}\fi
%    \end{macrocode}
% \end{macro}
%
% \begin{macro}{\th@plain}
% \begin{macro}{\th@break}
% \begin{macro}{\th@marginbreak}
% \begin{macro}{\th@changebreak}
% \begin{macro}{\th@change}
% \begin{macro}{\th@margin}
%    The different styles are defined in macros such as
%    |\th@plain|.  Since memory space is precious in
%    ``non-Big-versions'', we have to avoid offering too many unused
%    definitions.  Therefore we define these styles in separate files
%    that can be loaded on demand.  Thus the commands themselves only
%    load these files. We use |\@input@|, a \LaTeXe{} internal command
%    that ensures that the file will be listed with |\listfiles|.
% \changes{v2.2a}{94/02/02}{Use `@input@ to load theorem layout files}
%    \begin{macrocode}
\gdef\th@plain{\@input@{thp.sty}}
\gdef\th@break{\@input@{thb.sty}}
\gdef\th@marginbreak{\@input@{thmb.sty}}
\gdef\th@changebreak{\@input@{thcb.sty}}
\gdef\th@change{\@input@{thc.sty}}
\gdef\th@margin{\@input@{thm.sty}}
%    \end{macrocode}
%    This list will be expanded when new styles become
%    available.  For testing,  just append new theorem substyles
%    as document options.
% \end{macro}
% \end{macro}
% \end{macro}
% \end{macro}
% \end{macro}
% \end{macro}
%
%
%
% \subsection{Definition of a new theorem set}
%
% As already pointed out, a new theorem environment can be defined in
% three different ways:
% \begin{verbatim}
%    \newtheorem{Lem}{Lemma}
%    \newtheorem{Lem}{Lemma}[section]
%    \newtheorem{Lem}[Theorem]{Lemma}
%\end{verbatim}
% The function of the macro |\newtheorem| is to recognize these cases
% and then to branch into one of the three macros |\@ynthm|, |\@xnthm|
% or |\@othm|.  This mechanism is adopted unchanged from
% \cite{latex:source}; the essential point here is that, for example,
% in the second case, the arguments \texttt{Lem}, \texttt{Lemma} and
% \texttt{section} are passed over to the macro |\@xnthm|.
%
% We inspect this case first because the others present fewer problems,
% and thus are easily derived from this one.
%
% \begin{macro}{\@xnthm}
%    For our example arguments, the macro |\@xnthm| must fulfill
%    the following:
%    \begin{itemize}
%       \item
%          Define a new \LaTeX-counter `Lem'
%       \item
%          reset this counter within a |\section|
%       \item
%          define the macro |\theLem|
%       \item
%          define the environment macros |\Lem| and
%          |\endLem| using the current
%          |\theoremstyle| and |\theorem@bodyfont|.
%    \end{itemize}
%    Obviously, all this should happen only if the first argument
%    of |\@xnthm| (i.e.\ \texttt{Lem} in our example) is chosen so
%    as not to conflict with any previously defined commands or
%    environments.  This test is performed by the \LaTeX{} macro
%    |\@ifdefinable|.
%    \begin{macrocode}
\gdef\@xnthm#1#2[#3]{\expandafter\@ifdefinable\csname #1\endcsname
%    \end{macrocode}
%    Therefore, the first argument of |\@ifdefinable| is the
%    expansion (in the example, |\Lem|) of
%    |\csname||#1\endcsname|.  The second argument is
%    executed only if the test has been completed successfully.
%    \begin{macrocode}
   {%
%    \end{macrocode}
%    Now we define the new counter. The names of the \LaTeX{}
%    macros employed should speak for themselves:
% \changes{v2.2b}{1995/10/07}{Use \cs{@newctr}}
% \changes{v2.2c}{1995/11/23}{Use \cs{@newctr} properly.}
%    \begin{macrocode}
    \@definecounter{#1}\@newctr{#1}[#3]%
%    \end{macrocode}
%    Using |\@newctr| will give a proper error message if the counter
%    in |#3| is not defined.
%    In defining `|\theLem|' we must generate the desired macro
%    name by use of |\expandafter| and |\csname|.
%    \begin{macrocode}
    \expandafter\xdef\csname the#1\endcsname
%    \end{macrocode}
%    An |\xdef| is used in order to make the definition global,
%    and to ensure that it contains the replacement texts of
%    |\@thmcountersep| and |\@thmcounter|.\footnote{These
%    two macros can be defined by the document style.  Their default
%    values produce a `\texttt{.}' as separation and an arabic
%    representation of the number.} However, not everything should be
%    expanded. For example, it saves space to use |\thesection|
%    instead of its---at times---lengthy expansion.
%    \begin{macrocode}
      {\expandafter \noexpand \csname the#3\endcsname
       \@thmcountersep \@thmcounter{#1}}%
%    \end{macrocode}
%    Thus with the defaults of \LaTeX{}, |\theLem| would be
%    replaced by the command sequence |\thesection.\arabic{Lem}|.
%
%    We will now look at the definition of the macro which is executed
%    at the beginning of the actual environment (in our example this
%    macro is |\Lem|). It should be noted that we use an
%    ``|\expandafter| trick'' to expand only certain parts of the
%    replacement text at the time of the definition.
%    \begin{macrocode}
    \def\@tempa{\global\@namedef{#1}}%
    \expandafter \@tempa \expandafter{%
%    \end{macrocode}
%    First, the macro that contains the current definitions of
%    |\@begintheorem| and |\@opargtheorem| should be called
%    up.  The name of this macro---as is already known---has the form
%    |\th@|\meta{theorem\ style}; therefore, it must be called by
%    \begin{macrocode}
      \csname th@\the \theorem@style
%    \end{macrocode}
%    In addition the default theorem font should be changeable, i.e.\
%    we have to insert the contents of |\theorem@bodyfont|. For
%    that reason, we expand even further, beyond |\endcsname|,
%    and thus insert the contents of the \textsf{token} register
%    |\theorem@bodyfont| in the replacement text.
%    \begin{macrocode}
            \expandafter \endcsname \the \theorem@bodyfont
%    \end{macrocode}
%    Now it is time to call the macro |\@thm| which takes over
%    the further processing.  It has two arguments: the current
%    counter name (in our example, \texttt{Lem}), and the text of the
%    label (in our example, \texttt{Lemma}).
%    \begin{macrocode}
     \@thm{#1}{#2}}%
%    \end{macrocode}
%    With this, the `sub-definition' is complete.  The macro
%    |\@endtheorem| ends a theorem environment and is, so far, nothing
%    but an |\endtrivlist|.  (Hence it is defined globally, and not
%    within the theorem styles.\footnote{This has to be changed as
%    soon as theorem styles that change \texttt{\bslash @endtheorem}
%    exist.  In such a case, all existing styles must be changed as
%    well since they will have to reset the macro.}) Therefore, we can
%    set it equivalent to the macro that ends the theorem set (in our
%    example, |\endLem|).  However, if some day theorem styles exist
%    that do change |\@endtheorem|, we would have to use the
%    commented-out line instead.
%    \begin{macrocode}
    \global \expandafter \let \csname end#1\endcsname \@endtheorem
%   \global\@namedef{end#1}{\@endtheorem}%
%    \end{macrocode}
%    With these commands all the required definitions are employed,
%    unless the test |\@ifdefinable| has failed.  Therefore, we
%    end the second argument of this macro and with it the definition
%    of |\@xnthm|.
%    \begin{macrocode}
   }}
%    \end{macrocode}
% \end{macro}
%
% \begin{macro}{\@ynthm}
%    The definition of |\@ynthm| is completely analogous. In
%    this case the new counter that is defined is not reset within
%    another counter; thus the definition of |\the...|
%    is simplified:
%    \begin{macrocode}
\gdef\@ynthm#1#2{\expandafter\@ifdefinable\csname #1\endcsname
   {\@definecounter{#1}%
    \expandafter\xdef\csname the#1\endcsname{\@thmcounter{#1}}%
%    \end{macrocode}
%    The rest of the definition corresponds literally to that of
%    |\@xnthm|:
%    \begin{macrocode}
    \def\@tempa{\global\@namedef{#1}}\expandafter \@tempa
     \expandafter{\csname th@\the \theorem@style \expandafter
     \endcsname \the\theorem@bodyfont \@thm{#1}{#2}}%
    \global \expandafter \let \csname end#1\endcsname \@endtheorem}}
%    \end{macrocode}
% \end{macro}
%
% \begin{macro}{\@othm}
%    The definition of |\@othm| does not contain anything new.
%    \begin{macrocode}
\gdef\@othm#1[#2]#3{%
%    \end{macrocode}
% \changes{v2.2b}{95/10/07}{Build in check for counter defined}
%    We do not define a new counter but instead use one that has
%    already been defined.  Thus the only definition we need is that
%    of this pseudo-counter (i.e.\ |\the|\meta{env.\ name}).
%    First we check if |#2| corresponds to a known counter name.
%    \begin{macrocode}
  \expandafter\ifx\csname c@#2\endcsname\relax
   \@nocounterr{#2}%
  \else
   \expandafter\@ifdefinable\csname #1\endcsname
   {\expandafter \xdef \csname the#1\endcsname
     {\expandafter \noexpand \csname the#2\endcsname}%
%    \end{macrocode}
%    All other parts of the definition can be adopted from
%    |\@xnthm|. We have to remember, though, that in this case
%    the name of the current counter and the theorem label have moved
%    to the second and third arguments.
%    \begin{macrocode}
    \def\@tempa{\global\@namedef{#1}}\expandafter \@tempa
     \expandafter{\csname th@\the \theorem@style \expandafter
     \endcsname \the\theorem@bodyfont \@thm{#2}{#3}}%
    \global \expandafter \let \csname end#1\endcsname \@endtheorem}%
  \fi}
%    \end{macrocode}
% \end{macro}
%
% \subsection{Macros that are employed in a {\envfont theorem}
%    environment}
%
% \begin{macro}{\@thm}
%    The macro |\@thm| has to increase the current counter.  Then,
%    depending on whether the environment has (or does not have) an
%    optional argument, it has to branch into either |\@begintheorem|
%    or |\@opargtheorem|.
%    \begin{macrocode}
\gdef\@thm#1#2{\refstepcounter{#1}%
%    \end{macrocode}
%    Now we start a {\envfont trivlist} environment, and give
%    |\@topsep| and |\@topsepadd| the values of the \textsf{skip}
%    registers |\theorempreskipamount| and |\theorempostskipamount|.
%    The value in |\@topsep| is the vertical space that is inserted by
%    the first (and only) |\item| in our |\trivlist| whilst
%    |\@topsepadd| is inserted by |\@endparenv| at the end of that
%    {\envfont trivlist} environment.  By using these registers, we
%    obtain the desired space around a {\envfont theorem} environment.
%    \begin{macrocode}
   \trivlist
   \@topsep \theorempreskipamount               % used by first \item
   \@topsepadd \theorempostskipamount           % used by \@endparenv
%    \end{macrocode}
%    Now we have to test whether an optional argument has been given.
% \changes{v2.0d}{89/07/18}{Forgotten percent added.}
%    \begin{macrocode}
   \@ifnextchar [%
%    \end{macrocode}
%    If there is an optional argument, we will call |\@ythm|, and
%    move the arguments read back into the input stream.
%    \begin{macrocode}
   {\@ythm{#1}{#2}}%
%    \end{macrocode}
%    If not, we call |\@begintheorem|. Its first argument is the
%    name of the theorem set (hence the second argument of
%    |\@thm|). Its second argument is the macro that produces the
%    current number.
%    \begin{macrocode}
   {\@begintheorem{#2}{\csname the#1\endcsname}\ignorespaces}}
%    \end{macrocode}
% \end{macro}
%
% \begin{macro}{\@xthm}
% \begin{macro}{\@ythm}
%    Both these macros were originally called by |\@thm|.  We do
%    not need |\@xthm| anymore, hence we reset it to
%    |\relax|. The definition of |\@ythm| has not changed at
%    all from its definition in \LaTeX. In order to make the macros
%    easier to understand, we will
%    nevertheless present it (commented out).
%    \begin{macrocode}
\global\let\@xthm\relax
% \def\@ythm#1#2[#3]{\@opargbegintheorem{#2}{\csname
%           the#1\endcsname}{#3}\ignorespaces}
%</package>
%    \end{macrocode}
%    The primitive |\ignorespaces| in |\@ythm| and
%    |\@thm| is needed to remove the spaces between the
%    |\begin{...}| and the actual text.
% \end{macro}
% \end{macro}
%
%
% \subsection{Definition of the theorem substyles}
%
% As already pointed out, the theorem substyles, defined below, are
% only loaded when necessary.  Note that all these substyles, except
% \texttt{plain}, have |\slshape| as the default body font.
%
% \subsubsection{The \texttt{plain} style}
%
% \label{dok:th@plain} As the following macros use |@|, we have
% to locally set the |\catcode| of this symbol to ``letter''.
% This happens within a group, so that we do not have to worry about
% which |\catcode| that symbol had before.
%    \begin{macrocode}
%<*thp>
\begingroup \makeatletter
%    \end{macrocode}
% Since we are now within a group, we must make all definitions
% globally.  First we make sure that \texttt{theorem.sty} is loaded.
% This will allow us to use this file as a document style option
% without having to call \texttt{theorem} itself as an option.  At the
% same time, we assure that at least version 2 is loaded, since
% |\theorem@style| was not defined in earlier versions.
%    \begin{macrocode}
\@ifundefined{theorem@style}{% \iffalse meta-comment
%
% Copyright (C) 1993-2024
%
% The LaTeX Project and any individual authors listed elsewhere
% in this file.
%
% This file is part of the Standard LaTeX `Tools Bundle'.
% -------------------------------------------------------
%
% It may be distributed and/or modified under the
% conditions of the LaTeX Project Public License, either version 1.3c
% of this license or (at your option) any later version.
% The latest version of this license is in
%    https://www.latex-project.org/lppl.txt
% and version 1.3c or later is part of all distributions of LaTeX
% version 2005/12/01 or later.
%
% The list of all files belonging to the LaTeX `Tools Bundle' is
% given in the file `manifest.txt'.
%
% \fi
%
%%
%
%
\def\FMithmInfo{2023/07/05 v2.2c Theorem extension package (FMi)}
%
% \ProvidesFile{theorem.dtx}[\FMithmInfo]
%\iffalse   % this is a METACOMMENT !
%%
%% Package `theorem' to use with LaTeX2e
%% Copyright (C) 1989-1995 Frank Mittelbach, all rights reserved.
%<+package>\@ifundefined{theorem@style}{}{\endinput}
%<+package>\NeedsTeXFormat{LaTeX2e}[1995/06/01]
%<+package>\ProvidesPackage{theorem}
%<+package>         [\FMithmInfo]
%
% \fi
%
% \changes{v1.0}{87/05/07}{First draft.}
% \changes{v1.0a}{87/05/07}{`thistheoremstyle' added.}
% \changes{v1.1a}{87/07/29}{New concept: redefine the internal
%                          \protect\LaTeX{} commands `@xnthm' etc.}
% \changes{v1.1b}{87/07/31}{`theoremfont' used to specify font.}
% \changes{v1.2a}{88/01/08}{`@endtheorem' now globally defined.}
% \changes{v1.2b}{88/01/11}{`newtheorem' only allowed in the preamble.}
% \changes{v1.3a}{88/02/14}{`theoremfont' now a toks register.}
%
% \changes{v2.0a}{88/11/16}{New implementation and new documentation.}
% \changes{v2.0a}{88/11/16}{No layout changes but internal modifications
%                           old substyles won't work.}
% \changes{v2.0b}{89/01/30}{`global' used for all definitions.}
% \changes{v2.0c}{89/06/15}{Documentation translated to English.
%                            Thanks to LLL and Nina}
% \changes{v2.1a}{90/02/15}{Support for new font selection.}
% \changes{v2.2a}{94/02/03}{Cleanup for LaTeX2e}
%
% \DoNotIndex{\@endparenv,\@ifdefinable,\@ifnextchar,\@ifundefined}
% \DoNotIndex{\@namedef,\@nameuse,\@spaces,\@tempa,\@warning}
% \DoNotIndex{\ }
% \DoNotIndex{\begingroup,\bfseries}
% \DoNotIndex{\csname}
% \DoNotIndex{\def,\do,\docdate}
% \DoNotIndex{\endcsname,\endgroup,\endinput,\endtrivlist,\expandafter}
% \DoNotIndex{\filedate,\fileversion}
% \DoNotIndex{\gdef,\global}
% \DoNotIndex{\hbox,\hskip}
% \DoNotIndex{\ignorespaces,\input,\itshape,\item}
% \DoNotIndex{\labelsep,\let,\llap}
% \DoNotIndex{\makeatletter}
% \DoNotIndex{\noexpand}
% \DoNotIndex{\refstepcounter,\relax,\rlap}
% \DoNotIndex{\setlength,\slshape,\space,\string,\strut}
% \DoNotIndex{\the,\trivlist,\typeout}
% \DoNotIndex{\vbox}
% \DoNotIndex{\xdef}
%
% \def\deflabel#1{\ttfamily #1\hfill}
% \def\deflist#1{\begin{list}{}{\settowidth\labelwidth{\ttfamily #1}%
%                               \setlength\leftmargin\labelwidth
%                               \addtolength\leftmargin\labelsep
%                               \let\makelabel\deflabel}}
% \def\enddeflist{\end{list}}
% \def\envfont{\normalfont\ttfamily}
%
% \MakeShortVerb{\|}
%
% \newcommand{\TUB}{{\sl TUGboat\/}}
%
% \GetFileInfo{theorem.dtx}
% \title{An Extension of the
%        \LaTeX{} theorem environment\thanks{This file
%        has version number \fileversion, last
%        revised \filedate.}}
% \author{Frank Mittelbach}
%
% \MaintainedByLaTeXTeam{tools}
% \maketitle
%
% \begin{abstract}
%    The macros described in this paper yield an extension of the
%    \LaTeX{} theorem mechanism. It is designed to satisfy the
%    different requirements of various journals. Thus, the
%    layout of the ``theorems'' can be manipulated by determining a
%    ``style''. This article describes not only the use, but
%    also the definition, of the necessary macros.
% \end{abstract}
%
% \section*{Preface to version 2.2}
%
% For \LaTeXe{} this package did not need any fundamental changes. I
% only modified the messages generated so that theorem layout styles
% will show up with the |\listfiles| command and cleaned the section
% on the New Font Selection Scheme since this is now included in
% \LaTeX.
%
% \section*{Preface to version 2.1}
%
% This version is identical to 2.0g described in \TUB\ 10\#3 except
% for some internal defaults which are now set depending on the used
% font selection scheme.
%
% This was done to avoid unpleasant surprises if the new font selection
% scheme is in force. For further details see
% section~\ref{sec:newfonts} and \cite{bk:GMS94}.
%
% \section{Introduction}
%
% For our purposes here, ``theorems'' are labelled enunciations,
% often set off from the main text by extra space and a font change.
% Theorems, corollaries, conjectures, definitions, and remarks
% are all instances of ``theorems''.  The ``header'' of these structures
% is composed of a label (such as \textsc{Theorem} or \textsc{Remark})
% and a number which serializes an item in the sequence of items
% with the same label.
%
% Shortly after the introduction of \LaTeX{} at the Fachbereich
% Mathematik in Mainz, the desire to manipulate the layout of
% ``theorems'' arose.  In Mainz, the following two conventions came
% into general use:
% \begin{enumerate}
%   \item The number of the theorem is shown in the margin.
%   \item There is a line break at
%         the end of the theorem header.
% \end{enumerate}
%
% Additionally, some journals require different formats which depend on
% the ``sort of theorem'': e.g.\ often remarks and definitions are set
% in |\upshape|, while |\itshape| is employed for main theorems.
%
% Confronted with these requirements, a theorem environment was
% developed in Mainz which allows separate determination of the
% layout of the ``theorems sets'', comparable to |\pagestyle|.
%
%
% \section{The user interface}
% \subsection{Defining new theorem sets}
%
% \DescribeMacro\newtheorem
% As in the original \LaTeX{} version, the command |\newtheorem|
% defines a new ``theorem set'' or ``theorem-like structure''.
% Two required arguments name the new environment and give the text
% to be typeset with each instance of the new ``set'', while
% an optional argument determines how
% the ``set'' is enumerated:
% \begin{description}
%    \item[\ttfamily \bslash newtheorem\{foo\}\{bar\}]
%       The theorem set {\envfont foo} (whose name is \texttt{bar})
%       uses its own counter.
%    \item[\ttfamily \bslash newtheorem\{foo2\}{[foo]}\{bar2\}]
%       The theorem set {\envfont foo2} (printed name \texttt{bar2})
%       uses the same counter as the theorem set \texttt{foo}.
%    \item[\ttfamily \bslash newtheorem\{foo3\}\{bar3\}{[section]}]
%       The theorem set {\envfont foo3} (printed name \texttt{bar3}) is
%       enumerated within the counter \texttt{section}, i.e.\ with every
%       new |\section| the enumeration begins again with 1, and
%       the enumeration is composed from the section-number and the
%       theorem counter itself.
% \end{description}
%
% \DescribeMacro\theoremstyle
% Additionally, the command |\theoremstyle| can define the layout
% of various, or all, theorem sets.  It should be noted that any
% theorem set defined by |\newtheorem| is typeset in the
% |\theoremstyle| that is current at the time of the definition.
% Thus, the following
% \begin{verbatim}
%    \theoremstyle{break}        \newtheorem{Cor}{Corollary}
%    \theoremstyle{plain}        \newtheorem{Exa}{Example}[section]
%\end{verbatim}                             ^^A No blank at line start
% leads to the result that the set {\envfont Cor} is formatted in the
% style \texttt{break}, while the set {\envfont Exa} and all the
% following ones are formatted in the style \texttt{plain}, unless
% another |\theoremstyle| follows.  Since the definitions installed by
% |\newtheorem| are global, one also can limit |\theoremstyle| locally
% by grouping braces.
%
% \DescribeMacro\theorembodyfont
% The choice of the font for the theorem body is completely
% independent of the chosen |\theoremstyle|; this has proven to
% be very advantageous.  For example,
% \begin{verbatim}
%   {\theorembodyfont{\upshape}          \newtheorem{Rem}{Remark}}
%\end{verbatim}
% defines a theorem set {\envfont Rem}, which will be set in
% |\upshape| in the current layout (which in our example is
% \texttt{plain}).  As with |\theoremstyle|, the |\theorembodyfont|
% chosen is that current at the time of |\newtheorem|.  If
% |\theorembodyfont| is not specified or one defines
% |\theorembodyfont{}|, then the font used will be that defined by the
% |\theoremstyle|.
%
% \DescribeMacro\theoremheaderfont It is also possible to customize
% the font used for the theorem headers.  This is, however, a global
% declaration, and therefore there should be at most one
% |\theoremheaderfont| declaration in the preamble.\footnote{If it is
% actually necessary to have different header fonts, one has to define
% new theorem styles (substituting the desired font) or specify the
% information directly in the \texttt{\bslash newtheorem} declaration
% (the unclean variant).}
%
% \DescribeMacro\theorempreskipamount
% \DescribeMacro\theorempostskipamount Two additional parameters
% affect the vertical space around the theorem environments:
% |\theorempreskipamount| and |\theorempostskipamount| define,
% respectively, the spacing before and after such an environment.
% These parameters apply for all theorem sets and can be manipulated
% with the ordinary length macros.  They are rubber lengths
% (`\textsf{skips}'), and therefore can contain \texttt{plus} and
% \texttt{minus} parts.
%
% Since the definition of theorem sets should---most sensibly---be
% placed in the preamble, we only allow installation there.
% It is therefore possible to release the memory used here after
% |\begin{document}|, in order to make room for other
% applications.
%
%
% \subsection{Existing theorem styles}
%
% The following theorem styles exist to date:
% \begin{deflist}{marginbreak}
%    \item[plain]
%       This theorem style emulates the original \LaTeX{} definition,
%       except that additionally the parameters
%       |\theorem...skipamount| are used.
%    \item[break]
%       In this style, the theorem header is followed by a line
%       break.
%    \item[marginbreak]
%       The theorem number is set in the margin, and there is a line
%       break as in \texttt{break}.
%    \item[changebreak]
%       Like \texttt{break}, but with header number and text
%       interchanged.
%    \item[change]
%       Header number and text are interchanged, without a line break.
%    \item[margin]
%       The number is set in the left margin, without a line break.
% \end{deflist}
% All styles (except \texttt{plain}) select |\slshape| as the default
% |\theorembodyfont|.
%
%
% \subsection{Examples}
%
% Given the above theorem sets {\envfont Cor}, {\envfont Exa}
% and {\envfont Rem},
% suppose that the preamble also contains the declarations:
% \begin{verbatim}
%    \theoremstyle{marginbreak}   \newtheorem{Lem}[Cor]{Lemma}
%    \theoremstyle{change}
%    \theorembodyfont{\itshape}        \newtheorem{Def}[Cor]{Definition}
%
%    \theoremheaderfont{\scshape}
%\end{verbatim}
% Then the following are some typical examples of the typeset output
% resulting from their use.
%
% \begin{Cor}
%    This is a sentence typeset in the theorem environment {\envfont
%    Cor}.
% \end{Cor}
%
% \begin{Exa}
%    This is a sentence typeset in the theorem environment {\envfont
%    Exa}.
% \end{Exa}
%
% \begin{Rem}
%    This is a sentence typeset in the theorem environment {\envfont
%    Rem}.
% \end{Rem}
%
% \begin{Lem}[Ben User]
%    This is a sentence typeset in the theorem environment {\envfont
%    Lem}.
% \end{Lem}
%
% \begin{Def}[Very impressive Definition]
%    This is a sentence typeset in the theorem environment {\envfont
%    Def}.
% \end{Def}
%
% The last two examples show the effect of the optional argument to
% a theorem environment (it is the text typeset in parentheses).
%
%
% \section{Special Considerations}
% \label{sec:newfonts}
%
% Theoremheader and body are implemented as a unit. This means that
% the |\theoremheaderfont|  will inherit characteristics of the
% |\theorembodyfont| in \LaTeXe. Thus, if for example
% |\theorembodyfont| is |\itshape| and |\theoremheaderfont|
% is |\bfseries| the font selected for the header will have the
% characteristics `bold extended italic'. If this is not desired one
% should set the |\theoremheaderfont| to something like
% \begin{verbatim}
%      \theoremheaderfont{\normalfont\bfseries}
%\end{verbatim}
% i.e.\ supplying all necessary font information explicitly.
%
%
%
% \section{Acknowledgements}
%
% The publication of this set of macros was only possible with the
% help of Christina Busse (translating the manuscript into English),
% Joachim Pense (playing the r\^ole of typist), Chris Rowley
% (looking everything over) and many others providing useful
% suggestions.
%
% \MaybeStop{%
%    \begin{thebibliography}{9}
%       \bibitem{bk:GMS94} \textsc{M.~Goossens}, \textsc{F.~Mittelbach}
%         and \textsc{A.~Samarin}.
%         \newblock The \LaTeX{} Companion.
%         \newblock
%         Addison-Wesley, Reading, Massachusetts, 1994.
%       \bibitem{latex:source} \textsc{Lamport, Leslie.} \newblock
%                               \texttt{latex.tex}, version 2.09, date
%                               Feb. 1990.
%    \end{thebibliography}
%    \PrintIndex
%    \PrintChanges}
%
%
% \section{The documentation driver file}
%
% The next bit of code contains the documentation driver file for
% \TeX{}, i.e., the file that will produce the documentation you are
% currently reading. It will be extracted from this file by the
% \texttt{docstrip} program. Since it is the first code in the file
% one can alternatively process this file directly with \LaTeXe{} to
% obtain the documentation.
%
%    \begin{macrocode}
%<*driver>
\documentclass{ltxdoc}

\usepackage{theorem}

%<+driver>% The next few lines define theorem sets which are used
%<+driver>% in the example section of the documentation.

\theoremstyle{break}       \newtheorem{Cor}{Corollary}
\theoremstyle{plain}       \newtheorem{Exa}{Example}[section]
{\theorembodyfont{\upshape}\newtheorem{Rem}{Remark}}
\theoremstyle{marginbreak} \newtheorem{Lem}[Cor]{Lemma}
\theoremstyle{change}
\theorembodyfont{\itshape} \newtheorem{Def}[Cor]{Definition}

\theoremheaderfont{\scshape}

\RecordChanges

\begin{document}
   \DocInput{theorem.dtx}
\end{document}
%</driver>
%    \end{macrocode}
%
% \section{Definition of the Macros}
%
% If the file has been loaded before, we abort immediately. If not the
% package announces itself (this is actually done at the very top of
% the file---the way it is done isn't good style so don't copy it).
% \changes{v2.0e}{89/07/19}{Spaces removed from `typeout'.}
% \changes{v2.2b}{95/11/19}{Announce moved to top of file}
%    \begin{macrocode}
%<*package>
%\@ifundefined{theorem@style}{}{\endinput}
%\def\FMithmInfo{1995/11/19 v2.2b Theorem extension package (FMi)}
%\ProvidesPackage{theorem}[\FMithmInfo]
%    \end{macrocode}
%
%
% \subsection{Definition of theorem styles and fonts}
%
% All the definitions in this file are done globally to allow
% inputting this file inside a group.
%
% \begin{macro}{\theoremstyle}
%    Before a theorem style can be installed, the chosen style must be
%    known. For that reason, we must test to see that
%    |\th@|\meta{style} is known or, more precisely, that it is
%    different from |\relax|. If the style is not known then
%    |\th@plain| is used.
%    \begin{macrocode}
\gdef\theoremstyle#1{%
   \@ifundefined{th@#1}{\@warning
          {Unknown theoremstyle `#1'. Using `plain'}%
          \theorem@style{plain}}%
%    \end{macrocode}
%    We save the theorem style to be used in the token register
%    |\theorem@style|.
%    \begin{macrocode}
      {\theorem@style{#1}}%
%    \end{macrocode}
%    Now we ``evaluate'' the theorem style: this means, we call the
%    macro |\th@|\meta{style} which will activate the relevant
%    definitions which are contained in a separate file.  This is done
%    in a group to suppress changes to the current font.  This could
%    otherwise pose problems together with the new font selection
%    scheme\footnote{When I printed the original article using the new
%    font selection scheme I ended with a document with slanted
%    typefaces (text headings and all) simply because one of the
%    theorem styles used \texttt{\bslash sl} at toplevel.} if the
%    |\th@|\meta{style} is evaluated a second time.
% \changes{v2.1a}{90/02/15}{Group added around `th@..}
% \changes{v2.1a}{90/02/15}{Reseting of `..@begintheorem removed.}
%    \begin{macrocode}
      \begingroup
        \csname th@\the\theorem@style \endcsname
      \endgroup}
%    \end{macrocode}
% \end{macro}
%
%
% \begin{macro}{\@begintheorem}
% \changes{v2.1a}{90/02/15}{Relaxed at toplevel.}
% \begin{macro}{\@opargbegintheorem}
% \changes{v2.1a}{90/02/15}{Relaxed at toplevel.}
%    We reset |\@begintheorem| and |\@opargbegintheorem| to
%    |\relax| since these commands are no longer necessary at
%    toplevel. This will save a few tokens.
%    \begin{macrocode}
\global\let\@begintheorem\relax
\global\let\@opargbegintheorem\relax
%    \end{macrocode}
% \end{macro}
% \end{macro}
%
% \begin{macro}{\theorem@style}
%    Obviously the \textsf{token} register used above has to be
%    allocated. To assure the utmost compatibility with the
%    original \LaTeX{} definition, we set the default theorem style to
%    \texttt{plain}, which implements the usual \LaTeX{} convention.
% \changes{v2.0f}{89/07/20}{Unnecessary `global' removed.}
%    \begin{macrocode}
\newtoks\theorem@style
\global\theorem@style{plain}
%    \end{macrocode}
% \end{macro}
%
% \begin{macro}{\theorembodyfont}
% \begin{macro}{\theorem@bodyfont}
%    If the |\theorembodyfont| is set by the user then it should not
%    interact with the default font set in the theorem style. When the
%    new font selection is in force this may happen if, for example,
%    the default is |\itshape| and the new |\theorembodyfont| is
%    |\sffamily|. So we add a |\reset@font| command in front of the
%    user definition.
% \changes{v2.0f}{89/07/20}{Unnecessary `global' removed.}
% \changes{v2.0g}{89/09/19}{`theoremfont to `theorembodyfont.}
% \changes{v2.1c}{92/06/22}{`reset@font added and logic changed.}
% \changes{v2.1d}{92/10/28}{Reset to empty if arg is empty}
%    \begin{macrocode}
\gdef\theorembodyfont#1{%
%    \end{macrocode}
%    We check if the argument supplied is empty and if so put nothing
%    into the |\theorem@bodyfont| token register to allow for
%    |\theorembodyfont{}| as a mean of using the default of the
%    current |\theoremstyle|.
%    \begin{macrocode}
   \def\@tempa{#1}%
   \ifx\@tempa\@empty
    \theorem@bodyfont{}%
   \else
    \theorem@bodyfont{\reset@font#1}%
   \fi
   }
\newtoks\theorem@bodyfont
\global\theorem@bodyfont{}
%    \end{macrocode}
% \end{macro}
% \end{macro}
%
%
% \begin{macro}{\theoremheaderfont}
% \changes{v2.0g}{89/09/19}{Macro added.}
%    The font for the theorem headers is handled differently because
%    this definition applies to all theorem styles.
%    \begin{macrocode}
\gdef\theoremheaderfont#1{\gdef\theorem@headerfont{#1}%
%    \end{macrocode}
%    After using the macro once it is redefined to produce an error
%    message.
%    \begin{macrocode}
       \gdef\theoremheaderfont##1{%
        \typeout{\string\theoremheaderfont\space should be used
                 only once.}}}
%    \end{macrocode}
% \end{macro}
%
% \begin{macro}{\theorem@headerfont}
% \changes{v2.0g}{89/09/19}{Macro added.}
%    To set the |\theorem@headerfont| default we first test if
%    the new fontselection scheme is in force.
% \changes{v2.1a}{90/02/15}{Support for new font selection.}
%    \begin{macrocode}
\ifx\upshape\undefined
%    \end{macrocode}
%    If not we define it to expand into |\bfseries|. We don't use
%    |\let| just in case a following style option redefines this
%    macro.
%    \begin{macrocode}
\gdef\theorem@headerfont{\bfseries}
%    \end{macrocode}
%    Otherwise we reset the current shape before calling |\bfseries|.
%    \begin{macrocode}
\else \gdef\theorem@headerfont{\normalfont\bfseries}\fi
%    \end{macrocode}
% \end{macro}
%
% \begin{macro}{\th@plain}
% \begin{macro}{\th@break}
% \begin{macro}{\th@marginbreak}
% \begin{macro}{\th@changebreak}
% \begin{macro}{\th@change}
% \begin{macro}{\th@margin}
%    The different styles are defined in macros such as
%    |\th@plain|.  Since memory space is precious in
%    ``non-Big-versions'', we have to avoid offering too many unused
%    definitions.  Therefore we define these styles in separate files
%    that can be loaded on demand.  Thus the commands themselves only
%    load these files. We use |\@input@|, a \LaTeXe{} internal command
%    that ensures that the file will be listed with |\listfiles|.
% \changes{v2.2a}{94/02/02}{Use `@input@ to load theorem layout files}
%    \begin{macrocode}
\gdef\th@plain{\@input@{thp.sty}}
\gdef\th@break{\@input@{thb.sty}}
\gdef\th@marginbreak{\@input@{thmb.sty}}
\gdef\th@changebreak{\@input@{thcb.sty}}
\gdef\th@change{\@input@{thc.sty}}
\gdef\th@margin{\@input@{thm.sty}}
%    \end{macrocode}
%    This list will be expanded when new styles become
%    available.  For testing,  just append new theorem substyles
%    as document options.
% \end{macro}
% \end{macro}
% \end{macro}
% \end{macro}
% \end{macro}
% \end{macro}
%
%
%
% \subsection{Definition of a new theorem set}
%
% As already pointed out, a new theorem environment can be defined in
% three different ways:
% \begin{verbatim}
%    \newtheorem{Lem}{Lemma}
%    \newtheorem{Lem}{Lemma}[section]
%    \newtheorem{Lem}[Theorem]{Lemma}
%\end{verbatim}
% The function of the macro |\newtheorem| is to recognize these cases
% and then to branch into one of the three macros |\@ynthm|, |\@xnthm|
% or |\@othm|.  This mechanism is adopted unchanged from
% \cite{latex:source}; the essential point here is that, for example,
% in the second case, the arguments \texttt{Lem}, \texttt{Lemma} and
% \texttt{section} are passed over to the macro |\@xnthm|.
%
% We inspect this case first because the others present fewer problems,
% and thus are easily derived from this one.
%
% \begin{macro}{\@xnthm}
%    For our example arguments, the macro |\@xnthm| must fulfill
%    the following:
%    \begin{itemize}
%       \item
%          Define a new \LaTeX-counter `Lem'
%       \item
%          reset this counter within a |\section|
%       \item
%          define the macro |\theLem|
%       \item
%          define the environment macros |\Lem| and
%          |\endLem| using the current
%          |\theoremstyle| and |\theorem@bodyfont|.
%    \end{itemize}
%    Obviously, all this should happen only if the first argument
%    of |\@xnthm| (i.e.\ \texttt{Lem} in our example) is chosen so
%    as not to conflict with any previously defined commands or
%    environments.  This test is performed by the \LaTeX{} macro
%    |\@ifdefinable|.
%    \begin{macrocode}
\gdef\@xnthm#1#2[#3]{\expandafter\@ifdefinable\csname #1\endcsname
%    \end{macrocode}
%    Therefore, the first argument of |\@ifdefinable| is the
%    expansion (in the example, |\Lem|) of
%    |\csname||#1\endcsname|.  The second argument is
%    executed only if the test has been completed successfully.
%    \begin{macrocode}
   {%
%    \end{macrocode}
%    Now we define the new counter. The names of the \LaTeX{}
%    macros employed should speak for themselves:
% \changes{v2.2b}{1995/10/07}{Use \cs{@newctr}}
% \changes{v2.2c}{1995/11/23}{Use \cs{@newctr} properly.}
%    \begin{macrocode}
    \@definecounter{#1}\@newctr{#1}[#3]%
%    \end{macrocode}
%    Using |\@newctr| will give a proper error message if the counter
%    in |#3| is not defined.
%    In defining `|\theLem|' we must generate the desired macro
%    name by use of |\expandafter| and |\csname|.
%    \begin{macrocode}
    \expandafter\xdef\csname the#1\endcsname
%    \end{macrocode}
%    An |\xdef| is used in order to make the definition global,
%    and to ensure that it contains the replacement texts of
%    |\@thmcountersep| and |\@thmcounter|.\footnote{These
%    two macros can be defined by the document style.  Their default
%    values produce a `\texttt{.}' as separation and an arabic
%    representation of the number.} However, not everything should be
%    expanded. For example, it saves space to use |\thesection|
%    instead of its---at times---lengthy expansion.
%    \begin{macrocode}
      {\expandafter \noexpand \csname the#3\endcsname
       \@thmcountersep \@thmcounter{#1}}%
%    \end{macrocode}
%    Thus with the defaults of \LaTeX{}, |\theLem| would be
%    replaced by the command sequence |\thesection.\arabic{Lem}|.
%
%    We will now look at the definition of the macro which is executed
%    at the beginning of the actual environment (in our example this
%    macro is |\Lem|). It should be noted that we use an
%    ``|\expandafter| trick'' to expand only certain parts of the
%    replacement text at the time of the definition.
%    \begin{macrocode}
    \def\@tempa{\global\@namedef{#1}}%
    \expandafter \@tempa \expandafter{%
%    \end{macrocode}
%    First, the macro that contains the current definitions of
%    |\@begintheorem| and |\@opargtheorem| should be called
%    up.  The name of this macro---as is already known---has the form
%    |\th@|\meta{theorem\ style}; therefore, it must be called by
%    \begin{macrocode}
      \csname th@\the \theorem@style
%    \end{macrocode}
%    In addition the default theorem font should be changeable, i.e.\
%    we have to insert the contents of |\theorem@bodyfont|. For
%    that reason, we expand even further, beyond |\endcsname|,
%    and thus insert the contents of the \textsf{token} register
%    |\theorem@bodyfont| in the replacement text.
%    \begin{macrocode}
            \expandafter \endcsname \the \theorem@bodyfont
%    \end{macrocode}
%    Now it is time to call the macro |\@thm| which takes over
%    the further processing.  It has two arguments: the current
%    counter name (in our example, \texttt{Lem}), and the text of the
%    label (in our example, \texttt{Lemma}).
%    \begin{macrocode}
     \@thm{#1}{#2}}%
%    \end{macrocode}
%    With this, the `sub-definition' is complete.  The macro
%    |\@endtheorem| ends a theorem environment and is, so far, nothing
%    but an |\endtrivlist|.  (Hence it is defined globally, and not
%    within the theorem styles.\footnote{This has to be changed as
%    soon as theorem styles that change \texttt{\bslash @endtheorem}
%    exist.  In such a case, all existing styles must be changed as
%    well since they will have to reset the macro.}) Therefore, we can
%    set it equivalent to the macro that ends the theorem set (in our
%    example, |\endLem|).  However, if some day theorem styles exist
%    that do change |\@endtheorem|, we would have to use the
%    commented-out line instead.
%    \begin{macrocode}
    \global \expandafter \let \csname end#1\endcsname \@endtheorem
%   \global\@namedef{end#1}{\@endtheorem}%
%    \end{macrocode}
%    With these commands all the required definitions are employed,
%    unless the test |\@ifdefinable| has failed.  Therefore, we
%    end the second argument of this macro and with it the definition
%    of |\@xnthm|.
%    \begin{macrocode}
   }}
%    \end{macrocode}
% \end{macro}
%
% \begin{macro}{\@ynthm}
%    The definition of |\@ynthm| is completely analogous. In
%    this case the new counter that is defined is not reset within
%    another counter; thus the definition of |\the...|
%    is simplified:
%    \begin{macrocode}
\gdef\@ynthm#1#2{\expandafter\@ifdefinable\csname #1\endcsname
   {\@definecounter{#1}%
    \expandafter\xdef\csname the#1\endcsname{\@thmcounter{#1}}%
%    \end{macrocode}
%    The rest of the definition corresponds literally to that of
%    |\@xnthm|:
%    \begin{macrocode}
    \def\@tempa{\global\@namedef{#1}}\expandafter \@tempa
     \expandafter{\csname th@\the \theorem@style \expandafter
     \endcsname \the\theorem@bodyfont \@thm{#1}{#2}}%
    \global \expandafter \let \csname end#1\endcsname \@endtheorem}}
%    \end{macrocode}
% \end{macro}
%
% \begin{macro}{\@othm}
%    The definition of |\@othm| does not contain anything new.
%    \begin{macrocode}
\gdef\@othm#1[#2]#3{%
%    \end{macrocode}
% \changes{v2.2b}{95/10/07}{Build in check for counter defined}
%    We do not define a new counter but instead use one that has
%    already been defined.  Thus the only definition we need is that
%    of this pseudo-counter (i.e.\ |\the|\meta{env.\ name}).
%    First we check if |#2| corresponds to a known counter name.
%    \begin{macrocode}
  \expandafter\ifx\csname c@#2\endcsname\relax
   \@nocounterr{#2}%
  \else
   \expandafter\@ifdefinable\csname #1\endcsname
   {\expandafter \xdef \csname the#1\endcsname
     {\expandafter \noexpand \csname the#2\endcsname}%
%    \end{macrocode}
%    All other parts of the definition can be adopted from
%    |\@xnthm|. We have to remember, though, that in this case
%    the name of the current counter and the theorem label have moved
%    to the second and third arguments.
%    \begin{macrocode}
    \def\@tempa{\global\@namedef{#1}}\expandafter \@tempa
     \expandafter{\csname th@\the \theorem@style \expandafter
     \endcsname \the\theorem@bodyfont \@thm{#2}{#3}}%
    \global \expandafter \let \csname end#1\endcsname \@endtheorem}%
  \fi}
%    \end{macrocode}
% \end{macro}
%
% \subsection{Macros that are employed in a {\envfont theorem}
%    environment}
%
% \begin{macro}{\@thm}
%    The macro |\@thm| has to increase the current counter.  Then,
%    depending on whether the environment has (or does not have) an
%    optional argument, it has to branch into either |\@begintheorem|
%    or |\@opargtheorem|.
%    \begin{macrocode}
\gdef\@thm#1#2{\refstepcounter{#1}%
%    \end{macrocode}
%    Now we start a {\envfont trivlist} environment, and give
%    |\@topsep| and |\@topsepadd| the values of the \textsf{skip}
%    registers |\theorempreskipamount| and |\theorempostskipamount|.
%    The value in |\@topsep| is the vertical space that is inserted by
%    the first (and only) |\item| in our |\trivlist| whilst
%    |\@topsepadd| is inserted by |\@endparenv| at the end of that
%    {\envfont trivlist} environment.  By using these registers, we
%    obtain the desired space around a {\envfont theorem} environment.
%    \begin{macrocode}
   \trivlist
   \@topsep \theorempreskipamount               % used by first \item
   \@topsepadd \theorempostskipamount           % used by \@endparenv
%    \end{macrocode}
%    Now we have to test whether an optional argument has been given.
% \changes{v2.0d}{89/07/18}{Forgotten percent added.}
%    \begin{macrocode}
   \@ifnextchar [%
%    \end{macrocode}
%    If there is an optional argument, we will call |\@ythm|, and
%    move the arguments read back into the input stream.
%    \begin{macrocode}
   {\@ythm{#1}{#2}}%
%    \end{macrocode}
%    If not, we call |\@begintheorem|. Its first argument is the
%    name of the theorem set (hence the second argument of
%    |\@thm|). Its second argument is the macro that produces the
%    current number.
%    \begin{macrocode}
   {\@begintheorem{#2}{\csname the#1\endcsname}\ignorespaces}}
%    \end{macrocode}
% \end{macro}
%
% \begin{macro}{\@xthm}
% \begin{macro}{\@ythm}
%    Both these macros were originally called by |\@thm|.  We do
%    not need |\@xthm| anymore, hence we reset it to
%    |\relax|. The definition of |\@ythm| has not changed at
%    all from its definition in \LaTeX. In order to make the macros
%    easier to understand, we will
%    nevertheless present it (commented out).
%    \begin{macrocode}
\global\let\@xthm\relax
% \def\@ythm#1#2[#3]{\@opargbegintheorem{#2}{\csname
%           the#1\endcsname}{#3}\ignorespaces}
%</package>
%    \end{macrocode}
%    The primitive |\ignorespaces| in |\@ythm| and
%    |\@thm| is needed to remove the spaces between the
%    |\begin{...}| and the actual text.
% \end{macro}
% \end{macro}
%
%
% \subsection{Definition of the theorem substyles}
%
% As already pointed out, the theorem substyles, defined below, are
% only loaded when necessary.  Note that all these substyles, except
% \texttt{plain}, have |\slshape| as the default body font.
%
% \subsubsection{The \texttt{plain} style}
%
% \label{dok:th@plain} As the following macros use |@|, we have
% to locally set the |\catcode| of this symbol to ``letter''.
% This happens within a group, so that we do not have to worry about
% which |\catcode| that symbol had before.
%    \begin{macrocode}
%<*thp>
\begingroup \makeatletter
%    \end{macrocode}
% Since we are now within a group, we must make all definitions
% globally.  First we make sure that \texttt{theorem.sty} is loaded.
% This will allow us to use this file as a document style option
% without having to call \texttt{theorem} itself as an option.  At the
% same time, we assure that at least version 2 is loaded, since
% |\theorem@style| was not defined in earlier versions.
%    \begin{macrocode}
\@ifundefined{theorem@style}{\input{theorem.sty}}{}
\ProvidesFile{thp.sty}
             [\FMithmInfo]
%    \end{macrocode}
%
%
% \begin{macro}{\th@plain}
% \begin{macro}{\@begintheorem}
% \begin{macro}{\@opargbegintheorem}
%    |\theoremstyle{plain}| corresponds to the original
%    definition, except that the distances to the surrounding text are
%    determined by the parameters |\theorempreskipamount| and
%    |\theorempostskipamount|.
%    First we set the default body font.
%    \begin{macrocode}
\gdef\th@plain{\normalfont\itshape
%    \end{macrocode}
%    Then we define |\@begintheorem| and
%    |\@opargbegintheorem|. These two macros define how the
%    header of a theorem is typeset.  |\@opargbegintheorem| will
%    be called if a \textsf{theorem} environment with an optional
%    argument is encountered; otherwise, the header is constructed by
%    calling |\@begintheorem|.  If one of these macros is
%    executed, we are within a \textsf{trivlist} environment started by
%    |\@thm|.  So the theorem header is produced with an
%    |\item| command.
%
%    Instead of specifying the header font directly, all standard
%    theorem styles use the |\theorem@headerfont| macro to allow
%    customization.  The extra space (|\labelsep|) is necessary
%    because of problems in the \textsf{trivlist} environment.
%    \begin{macrocode}
  \def\@begintheorem##1##2{%
        \item[\hskip\labelsep \theorem@headerfont ##1\ ##2]}%
%    \end{macrocode}
%    The definition of |\@opargbegintheorem| is completely
%    analogous. The only difference is the fact that there exists a
%    third argument (which is the optional parameter of the
%    environment and contains additional information about the
%    theorem).  Customarily we enclose it in parentheses.
%    \begin{macrocode}
\def\@opargbegintheorem##1##2##3{%
   \item[\hskip\labelsep \theorem@headerfont ##1\ ##2\ (##3)]}}
%    \end{macrocode}
% \end{macro}
% \end{macro}
% \end{macro}
%
% \noindent
% We conclude with an |\endgroup| to restore the
% |\catcode| of |@|.
%    \begin{macrocode}
\endgroup
%</thp>
%    \end{macrocode}
%
%
% \subsubsection{The \texttt{break} style}
%
% This style option is stored in the file |thb.sty|.
% For the next two lines see the documentation for |\th@plain| on
% page~\pageref{dok:th@plain}.
%    \begin{macrocode}
%<*thb>
\begingroup \makeatletter
\@ifundefined{theorem@style}{\input{theorem.sty}}{}
\ProvidesFile{thb.sty}
             [\FMithmInfo]
%    \end{macrocode}
%
% \begin{macro}{\th@break}
%    |\theoremstyle{break}| produces a line break after the name
%    of the theorem.  The font is |\slshape|.  Hence, we define
%    |\th@break| as follows:
%    \begin{macrocode}
\gdef\th@break{\normalfont\slshape
  \def\@begintheorem##1##2{\item[%
%    \end{macrocode}
%    We run into the following problem: it is not possible to create
%    the header with |\item[|\meta{title}|]| and then start
%    a new line by, for example, |\mbox{}\\|. Such a definition
%    will fail whenever a list environment follows immediately.  With
%    the above construction, the |\mbox{}| causes the switch
%    |@inlabel| (cf.\ definition of |\list| and
%    |\trivlist| in \cite{latex:source}) to be set to \texttt{false}
%    and so the following list will insert additional vertical
%    space (|\topskip|). This is quite annoying. Therefore, we
%    create the line break within the |\item|. In order to ensure
%    that the text will begin at the proper position in the following
%    line, we simply pretend that the label does not take any
%    room.\footnote{This will lead to problems whenever very high
%    symbols occurring in the line tower into the
%    heading. So, something else has to be done here sometime.}
%    \begin{macrocode}
       \rlap{\vbox{\hbox{\hskip \labelsep\theorem@headerfont ##1\ ##2}%
                   \hbox{\strut}}}]}%
%    \end{macrocode}
%    Again, the definition of |\@opargbegintheorem| is
%    completely analogous.
%    \begin{macrocode}
\def\@opargbegintheorem##1##2##3{%
  \item[\rlap{\vbox{\hbox{\hskip \labelsep \theorem@headerfont
                     ##1\ ##2\ (##3)}%
                    \hbox{\strut}}}]}}
\endgroup
%</thb>
%    \end{macrocode}
% \end{macro}
%
%
% \subsubsection{The \texttt{changebreak} style}
%
%    \begin{macrocode}
% This style option is stored in the file |thcb.sty|.
%    \begin{macrocode}
%<*thcb>
\begingroup \makeatletter
\@ifundefined{theorem@style}{\input{theorem.sty}}{}
\ProvidesFile{thcb.sty}
             [\FMithmInfo]
%    \end{macrocode}
%
% \begin{macro}{\th@changebreak}
%    The \texttt{change-break} theorem style is like \texttt{break}
%    but with interchange of theorem name and theorem number.  Thus we
%    define |\th@changebreak| as follows:
%    \begin{macrocode}
\gdef\th@changebreak{\normalfont\slshape
  \def\@begintheorem##1##2{\item
       [\rlap{\vbox{\hbox{\hskip\labelsep \theorem@headerfont ##2\ ##1}%
                    \hbox{\strut}}}]}%
\def\@opargbegintheorem##1##2##3{%
  \item[\rlap{\vbox{\hbox{\hskip\labelsep \theorem@headerfont
                     ##2\ ##1\ (##3)}%
                    \hbox{\strut}}}]}}
\endgroup
%</thcb>
%    \end{macrocode}
% \end{macro}
%
%
% \subsubsection{The \texttt{change} style}
%
% This style option is stored in the file |thc.sty|.
%    \begin{macrocode}
%<*thc>
\begingroup \makeatletter
\@ifundefined{theorem@style}{\input{theorem.sty}}{}
\ProvidesFile{thc.sty}
             [\FMithmInfo]
%    \end{macrocode}
%
% \begin{macro}{\th@change}
%    The \texttt{change} theorem style corresponds to the
%    \texttt{change break} style without a linebreak after the header.
%    To say it in another way, it's the same as the \texttt{plain}
%    style but with number and name interchanged and |\slshape| as the
%    default font.
%    \begin{macrocode}
\gdef\th@change{\normalfont\slshape
  \def\@begintheorem##1##2{\item
       [\hskip\labelsep \theorem@headerfont ##2\ ##1]}%
\def\@opargbegintheorem##1##2##3{%
  \item[\hskip\labelsep \theorem@headerfont ##2\ ##1\ (##3)]}}
\endgroup
%</thc>
%    \end{macrocode}
% \end{macro}
%
%
% \subsubsection{The \texttt{marginbreak} style}
%
% This style option is the one used most often at Mainz.  It is saved
% in the file |thmb.sty|.
%    \begin{macrocode}
%<*thmb>
\begingroup \makeatletter
\@ifundefined{theorem@style}{\input{theorem.sty}}{}
\ProvidesFile{thmb.sty}
             [\FMithmInfo]
%    \end{macrocode}
%
% \begin{macro}{\th@marginbreak}
%    The \texttt{margin break} style is nearly the same as the
%    \texttt{change break} style. The only difference is the placement
%    of the theorem number.  We use |\llap| to place it in the left
%    margin.
%
%    In this style |\labelsep| denotes the separation between the
%    number and the text.
%    \begin{macrocode}
\gdef\th@marginbreak{\normalfont\slshape
  \def\@begintheorem##1##2{\item
       [\rlap{\vbox{\theorem@headerfont
                    \hbox{\llap{##2}\hskip\labelsep ##1}%
                    \hbox{\strut}}}]}%
\def\@opargbegintheorem##1##2##3{%
  \item[\rlap{\vbox{\theorem@headerfont
                    \hbox{\llap{##2}\hskip\labelsep ##1\ (##3)}%
                    \hbox{\strut}}}]}}
\endgroup
%</thmb>
%    \end{macrocode}
% \changes{v2.1c}{90/02/17}{`endgroup outside macrocode env.}
% \end{macro}
%
%
%
% \subsubsection{The \texttt{margin} style}
%
% This style option is stored in the file |thm.sty|.
%    \begin{macrocode}
%<*thm>
\begingroup \makeatletter
\@ifundefined{theorem@style}{\input{theorem.sty}}{}
\ProvidesFile{thm.sty}
             [\FMithmInfo]
%    \end{macrocode}
%
% \begin{macro}{\th@margin}
%    Again this is only a variant of the theorem styles described
%    above without any new ideas.
%    \begin{macrocode}
\gdef\th@margin{\normalfont\slshape
  \def\@begintheorem##1##2{\item
       [\theorem@headerfont \llap{##2}\hskip\labelsep ##1]}%
\def\@opargbegintheorem##1##2##3{%
  \item[\theorem@headerfont \llap{##2}\hskip\labelsep ##1\ (##3)]}}
\endgroup
%</thm>
%    \end{macrocode}
% \end{macro}
%
%
%
%
% \subsection{Final Definitions}
%
% \begin{macro}{\theorempreskipamount}
% \begin{macro}{\theorempostskipamount}
%    The \textsf{skip} parameters that regulate the vertical empty space
%    before and after the {\envfont theorem} environment have
%    to be allocated as well.
% \changes{v2.0f}{89/07/20}{Unnecessary `global' removed.}
%    \begin{macrocode}
%<*package>
\newskip\theorempreskipamount
\newskip\theorempostskipamount
%    \end{macrocode}
%    Since we have used the same values for all theorem sets, we now can
%    assign them.
%    \begin{macrocode}
\global\setlength\theorempreskipamount{12pt plus 5pt minus 3pt}
\global\setlength\theorempostskipamount{8pt plus 3pt minus 1.5pt}
%    \end{macrocode}
% \end{macro}
% \end{macro}
%
%
% \begin{macro}{\@endtheorem}
%    The same holds for the macro |\@endtheorem|, which ends a
%    {\envfont theorem} environment. Since it is the same for all
%    theorem sets, it is removed from the macros |\th@|\meta{style}.
%    It simply ends the {\envfont trivlist} environment, which was
%    begun in |\@thm|.
%    \begin{macrocode}
\global\let\@endtheorem=\endtrivlist
%    \end{macrocode}
% \end{macro}
%
%
%
% \begin{macro}{\@preamblecmds}
%    All macros defined above are to be used only in the preamble.
%    Therefore, we insert them in |\@preamblecmds| which will disable
%    them at begin document. This is done by the internal \LaTeXe{}
%    command |\@onlypreamble|.
% \changes{v2.0g}{89/09/19}{`theoremheaderfont and `theorembodyfont
%    added.}
% \changes{v2.2a}{94/02/03}{Use \LaTeXe{} interface}
%    \begin{macrocode}
\@onlypreamble\@xnthm
\@onlypreamble\@ynthm
\@onlypreamble\@othm
\@onlypreamble\newtheorem
\@onlypreamble\theoremstyle
\@onlypreamble\theorembodyfont
\@onlypreamble\theoremheaderfont
%    \end{macrocode}
% \end{macro}
%
% Finally we declare the \texttt{plain} theorem style to be the default.
% \changes{v2.1b}{90/06/04}{Fixed missing `item by declaring default
%    style.}
%    \begin{macrocode}
\theoremstyle{plain}
%</package>
%    \end{macrocode}
%
% \Finale

\endinput
}{}
\ProvidesFile{thp.sty}
             [\FMithmInfo]
%    \end{macrocode}
%
%
% \begin{macro}{\th@plain}
% \begin{macro}{\@begintheorem}
% \begin{macro}{\@opargbegintheorem}
%    |\theoremstyle{plain}| corresponds to the original
%    definition, except that the distances to the surrounding text are
%    determined by the parameters |\theorempreskipamount| and
%    |\theorempostskipamount|.
%    First we set the default body font.
%    \begin{macrocode}
\gdef\th@plain{\normalfont\itshape
%    \end{macrocode}
%    Then we define |\@begintheorem| and
%    |\@opargbegintheorem|. These two macros define how the
%    header of a theorem is typeset.  |\@opargbegintheorem| will
%    be called if a \textsf{theorem} environment with an optional
%    argument is encountered; otherwise, the header is constructed by
%    calling |\@begintheorem|.  If one of these macros is
%    executed, we are within a \textsf{trivlist} environment started by
%    |\@thm|.  So the theorem header is produced with an
%    |\item| command.
%
%    Instead of specifying the header font directly, all standard
%    theorem styles use the |\theorem@headerfont| macro to allow
%    customization.  The extra space (|\labelsep|) is necessary
%    because of problems in the \textsf{trivlist} environment.
%    \begin{macrocode}
  \def\@begintheorem##1##2{%
        \item[\hskip\labelsep \theorem@headerfont ##1\ ##2]}%
%    \end{macrocode}
%    The definition of |\@opargbegintheorem| is completely
%    analogous. The only difference is the fact that there exists a
%    third argument (which is the optional parameter of the
%    environment and contains additional information about the
%    theorem).  Customarily we enclose it in parentheses.
%    \begin{macrocode}
\def\@opargbegintheorem##1##2##3{%
   \item[\hskip\labelsep \theorem@headerfont ##1\ ##2\ (##3)]}}
%    \end{macrocode}
% \end{macro}
% \end{macro}
% \end{macro}
%
% \noindent
% We conclude with an |\endgroup| to restore the
% |\catcode| of |@|.
%    \begin{macrocode}
\endgroup
%</thp>
%    \end{macrocode}
%
%
% \subsubsection{The \texttt{break} style}
%
% This style option is stored in the file |thb.sty|.
% For the next two lines see the documentation for |\th@plain| on
% page~\pageref{dok:th@plain}.
%    \begin{macrocode}
%<*thb>
\begingroup \makeatletter
\@ifundefined{theorem@style}{% \iffalse meta-comment
%
% Copyright (C) 1993-2024
%
% The LaTeX Project and any individual authors listed elsewhere
% in this file.
%
% This file is part of the Standard LaTeX `Tools Bundle'.
% -------------------------------------------------------
%
% It may be distributed and/or modified under the
% conditions of the LaTeX Project Public License, either version 1.3c
% of this license or (at your option) any later version.
% The latest version of this license is in
%    https://www.latex-project.org/lppl.txt
% and version 1.3c or later is part of all distributions of LaTeX
% version 2005/12/01 or later.
%
% The list of all files belonging to the LaTeX `Tools Bundle' is
% given in the file `manifest.txt'.
%
% \fi
%
%%
%
%
\def\FMithmInfo{2023/07/05 v2.2c Theorem extension package (FMi)}
%
% \ProvidesFile{theorem.dtx}[\FMithmInfo]
%\iffalse   % this is a METACOMMENT !
%%
%% Package `theorem' to use with LaTeX2e
%% Copyright (C) 1989-1995 Frank Mittelbach, all rights reserved.
%<+package>\@ifundefined{theorem@style}{}{\endinput}
%<+package>\NeedsTeXFormat{LaTeX2e}[1995/06/01]
%<+package>\ProvidesPackage{theorem}
%<+package>         [\FMithmInfo]
%
% \fi
%
% \changes{v1.0}{87/05/07}{First draft.}
% \changes{v1.0a}{87/05/07}{`thistheoremstyle' added.}
% \changes{v1.1a}{87/07/29}{New concept: redefine the internal
%                          \protect\LaTeX{} commands `@xnthm' etc.}
% \changes{v1.1b}{87/07/31}{`theoremfont' used to specify font.}
% \changes{v1.2a}{88/01/08}{`@endtheorem' now globally defined.}
% \changes{v1.2b}{88/01/11}{`newtheorem' only allowed in the preamble.}
% \changes{v1.3a}{88/02/14}{`theoremfont' now a toks register.}
%
% \changes{v2.0a}{88/11/16}{New implementation and new documentation.}
% \changes{v2.0a}{88/11/16}{No layout changes but internal modifications
%                           old substyles won't work.}
% \changes{v2.0b}{89/01/30}{`global' used for all definitions.}
% \changes{v2.0c}{89/06/15}{Documentation translated to English.
%                            Thanks to LLL and Nina}
% \changes{v2.1a}{90/02/15}{Support for new font selection.}
% \changes{v2.2a}{94/02/03}{Cleanup for LaTeX2e}
%
% \DoNotIndex{\@endparenv,\@ifdefinable,\@ifnextchar,\@ifundefined}
% \DoNotIndex{\@namedef,\@nameuse,\@spaces,\@tempa,\@warning}
% \DoNotIndex{\ }
% \DoNotIndex{\begingroup,\bfseries}
% \DoNotIndex{\csname}
% \DoNotIndex{\def,\do,\docdate}
% \DoNotIndex{\endcsname,\endgroup,\endinput,\endtrivlist,\expandafter}
% \DoNotIndex{\filedate,\fileversion}
% \DoNotIndex{\gdef,\global}
% \DoNotIndex{\hbox,\hskip}
% \DoNotIndex{\ignorespaces,\input,\itshape,\item}
% \DoNotIndex{\labelsep,\let,\llap}
% \DoNotIndex{\makeatletter}
% \DoNotIndex{\noexpand}
% \DoNotIndex{\refstepcounter,\relax,\rlap}
% \DoNotIndex{\setlength,\slshape,\space,\string,\strut}
% \DoNotIndex{\the,\trivlist,\typeout}
% \DoNotIndex{\vbox}
% \DoNotIndex{\xdef}
%
% \def\deflabel#1{\ttfamily #1\hfill}
% \def\deflist#1{\begin{list}{}{\settowidth\labelwidth{\ttfamily #1}%
%                               \setlength\leftmargin\labelwidth
%                               \addtolength\leftmargin\labelsep
%                               \let\makelabel\deflabel}}
% \def\enddeflist{\end{list}}
% \def\envfont{\normalfont\ttfamily}
%
% \MakeShortVerb{\|}
%
% \newcommand{\TUB}{{\sl TUGboat\/}}
%
% \GetFileInfo{theorem.dtx}
% \title{An Extension of the
%        \LaTeX{} theorem environment\thanks{This file
%        has version number \fileversion, last
%        revised \filedate.}}
% \author{Frank Mittelbach}
%
% \MaintainedByLaTeXTeam{tools}
% \maketitle
%
% \begin{abstract}
%    The macros described in this paper yield an extension of the
%    \LaTeX{} theorem mechanism. It is designed to satisfy the
%    different requirements of various journals. Thus, the
%    layout of the ``theorems'' can be manipulated by determining a
%    ``style''. This article describes not only the use, but
%    also the definition, of the necessary macros.
% \end{abstract}
%
% \section*{Preface to version 2.2}
%
% For \LaTeXe{} this package did not need any fundamental changes. I
% only modified the messages generated so that theorem layout styles
% will show up with the |\listfiles| command and cleaned the section
% on the New Font Selection Scheme since this is now included in
% \LaTeX.
%
% \section*{Preface to version 2.1}
%
% This version is identical to 2.0g described in \TUB\ 10\#3 except
% for some internal defaults which are now set depending on the used
% font selection scheme.
%
% This was done to avoid unpleasant surprises if the new font selection
% scheme is in force. For further details see
% section~\ref{sec:newfonts} and \cite{bk:GMS94}.
%
% \section{Introduction}
%
% For our purposes here, ``theorems'' are labelled enunciations,
% often set off from the main text by extra space and a font change.
% Theorems, corollaries, conjectures, definitions, and remarks
% are all instances of ``theorems''.  The ``header'' of these structures
% is composed of a label (such as \textsc{Theorem} or \textsc{Remark})
% and a number which serializes an item in the sequence of items
% with the same label.
%
% Shortly after the introduction of \LaTeX{} at the Fachbereich
% Mathematik in Mainz, the desire to manipulate the layout of
% ``theorems'' arose.  In Mainz, the following two conventions came
% into general use:
% \begin{enumerate}
%   \item The number of the theorem is shown in the margin.
%   \item There is a line break at
%         the end of the theorem header.
% \end{enumerate}
%
% Additionally, some journals require different formats which depend on
% the ``sort of theorem'': e.g.\ often remarks and definitions are set
% in |\upshape|, while |\itshape| is employed for main theorems.
%
% Confronted with these requirements, a theorem environment was
% developed in Mainz which allows separate determination of the
% layout of the ``theorems sets'', comparable to |\pagestyle|.
%
%
% \section{The user interface}
% \subsection{Defining new theorem sets}
%
% \DescribeMacro\newtheorem
% As in the original \LaTeX{} version, the command |\newtheorem|
% defines a new ``theorem set'' or ``theorem-like structure''.
% Two required arguments name the new environment and give the text
% to be typeset with each instance of the new ``set'', while
% an optional argument determines how
% the ``set'' is enumerated:
% \begin{description}
%    \item[\ttfamily \bslash newtheorem\{foo\}\{bar\}]
%       The theorem set {\envfont foo} (whose name is \texttt{bar})
%       uses its own counter.
%    \item[\ttfamily \bslash newtheorem\{foo2\}{[foo]}\{bar2\}]
%       The theorem set {\envfont foo2} (printed name \texttt{bar2})
%       uses the same counter as the theorem set \texttt{foo}.
%    \item[\ttfamily \bslash newtheorem\{foo3\}\{bar3\}{[section]}]
%       The theorem set {\envfont foo3} (printed name \texttt{bar3}) is
%       enumerated within the counter \texttt{section}, i.e.\ with every
%       new |\section| the enumeration begins again with 1, and
%       the enumeration is composed from the section-number and the
%       theorem counter itself.
% \end{description}
%
% \DescribeMacro\theoremstyle
% Additionally, the command |\theoremstyle| can define the layout
% of various, or all, theorem sets.  It should be noted that any
% theorem set defined by |\newtheorem| is typeset in the
% |\theoremstyle| that is current at the time of the definition.
% Thus, the following
% \begin{verbatim}
%    \theoremstyle{break}        \newtheorem{Cor}{Corollary}
%    \theoremstyle{plain}        \newtheorem{Exa}{Example}[section]
%\end{verbatim}                             ^^A No blank at line start
% leads to the result that the set {\envfont Cor} is formatted in the
% style \texttt{break}, while the set {\envfont Exa} and all the
% following ones are formatted in the style \texttt{plain}, unless
% another |\theoremstyle| follows.  Since the definitions installed by
% |\newtheorem| are global, one also can limit |\theoremstyle| locally
% by grouping braces.
%
% \DescribeMacro\theorembodyfont
% The choice of the font for the theorem body is completely
% independent of the chosen |\theoremstyle|; this has proven to
% be very advantageous.  For example,
% \begin{verbatim}
%   {\theorembodyfont{\upshape}          \newtheorem{Rem}{Remark}}
%\end{verbatim}
% defines a theorem set {\envfont Rem}, which will be set in
% |\upshape| in the current layout (which in our example is
% \texttt{plain}).  As with |\theoremstyle|, the |\theorembodyfont|
% chosen is that current at the time of |\newtheorem|.  If
% |\theorembodyfont| is not specified or one defines
% |\theorembodyfont{}|, then the font used will be that defined by the
% |\theoremstyle|.
%
% \DescribeMacro\theoremheaderfont It is also possible to customize
% the font used for the theorem headers.  This is, however, a global
% declaration, and therefore there should be at most one
% |\theoremheaderfont| declaration in the preamble.\footnote{If it is
% actually necessary to have different header fonts, one has to define
% new theorem styles (substituting the desired font) or specify the
% information directly in the \texttt{\bslash newtheorem} declaration
% (the unclean variant).}
%
% \DescribeMacro\theorempreskipamount
% \DescribeMacro\theorempostskipamount Two additional parameters
% affect the vertical space around the theorem environments:
% |\theorempreskipamount| and |\theorempostskipamount| define,
% respectively, the spacing before and after such an environment.
% These parameters apply for all theorem sets and can be manipulated
% with the ordinary length macros.  They are rubber lengths
% (`\textsf{skips}'), and therefore can contain \texttt{plus} and
% \texttt{minus} parts.
%
% Since the definition of theorem sets should---most sensibly---be
% placed in the preamble, we only allow installation there.
% It is therefore possible to release the memory used here after
% |\begin{document}|, in order to make room for other
% applications.
%
%
% \subsection{Existing theorem styles}
%
% The following theorem styles exist to date:
% \begin{deflist}{marginbreak}
%    \item[plain]
%       This theorem style emulates the original \LaTeX{} definition,
%       except that additionally the parameters
%       |\theorem...skipamount| are used.
%    \item[break]
%       In this style, the theorem header is followed by a line
%       break.
%    \item[marginbreak]
%       The theorem number is set in the margin, and there is a line
%       break as in \texttt{break}.
%    \item[changebreak]
%       Like \texttt{break}, but with header number and text
%       interchanged.
%    \item[change]
%       Header number and text are interchanged, without a line break.
%    \item[margin]
%       The number is set in the left margin, without a line break.
% \end{deflist}
% All styles (except \texttt{plain}) select |\slshape| as the default
% |\theorembodyfont|.
%
%
% \subsection{Examples}
%
% Given the above theorem sets {\envfont Cor}, {\envfont Exa}
% and {\envfont Rem},
% suppose that the preamble also contains the declarations:
% \begin{verbatim}
%    \theoremstyle{marginbreak}   \newtheorem{Lem}[Cor]{Lemma}
%    \theoremstyle{change}
%    \theorembodyfont{\itshape}        \newtheorem{Def}[Cor]{Definition}
%
%    \theoremheaderfont{\scshape}
%\end{verbatim}
% Then the following are some typical examples of the typeset output
% resulting from their use.
%
% \begin{Cor}
%    This is a sentence typeset in the theorem environment {\envfont
%    Cor}.
% \end{Cor}
%
% \begin{Exa}
%    This is a sentence typeset in the theorem environment {\envfont
%    Exa}.
% \end{Exa}
%
% \begin{Rem}
%    This is a sentence typeset in the theorem environment {\envfont
%    Rem}.
% \end{Rem}
%
% \begin{Lem}[Ben User]
%    This is a sentence typeset in the theorem environment {\envfont
%    Lem}.
% \end{Lem}
%
% \begin{Def}[Very impressive Definition]
%    This is a sentence typeset in the theorem environment {\envfont
%    Def}.
% \end{Def}
%
% The last two examples show the effect of the optional argument to
% a theorem environment (it is the text typeset in parentheses).
%
%
% \section{Special Considerations}
% \label{sec:newfonts}
%
% Theoremheader and body are implemented as a unit. This means that
% the |\theoremheaderfont|  will inherit characteristics of the
% |\theorembodyfont| in \LaTeXe. Thus, if for example
% |\theorembodyfont| is |\itshape| and |\theoremheaderfont|
% is |\bfseries| the font selected for the header will have the
% characteristics `bold extended italic'. If this is not desired one
% should set the |\theoremheaderfont| to something like
% \begin{verbatim}
%      \theoremheaderfont{\normalfont\bfseries}
%\end{verbatim}
% i.e.\ supplying all necessary font information explicitly.
%
%
%
% \section{Acknowledgements}
%
% The publication of this set of macros was only possible with the
% help of Christina Busse (translating the manuscript into English),
% Joachim Pense (playing the r\^ole of typist), Chris Rowley
% (looking everything over) and many others providing useful
% suggestions.
%
% \MaybeStop{%
%    \begin{thebibliography}{9}
%       \bibitem{bk:GMS94} \textsc{M.~Goossens}, \textsc{F.~Mittelbach}
%         and \textsc{A.~Samarin}.
%         \newblock The \LaTeX{} Companion.
%         \newblock
%         Addison-Wesley, Reading, Massachusetts, 1994.
%       \bibitem{latex:source} \textsc{Lamport, Leslie.} \newblock
%                               \texttt{latex.tex}, version 2.09, date
%                               Feb. 1990.
%    \end{thebibliography}
%    \PrintIndex
%    \PrintChanges}
%
%
% \section{The documentation driver file}
%
% The next bit of code contains the documentation driver file for
% \TeX{}, i.e., the file that will produce the documentation you are
% currently reading. It will be extracted from this file by the
% \texttt{docstrip} program. Since it is the first code in the file
% one can alternatively process this file directly with \LaTeXe{} to
% obtain the documentation.
%
%    \begin{macrocode}
%<*driver>
\documentclass{ltxdoc}

\usepackage{theorem}

%<+driver>% The next few lines define theorem sets which are used
%<+driver>% in the example section of the documentation.

\theoremstyle{break}       \newtheorem{Cor}{Corollary}
\theoremstyle{plain}       \newtheorem{Exa}{Example}[section]
{\theorembodyfont{\upshape}\newtheorem{Rem}{Remark}}
\theoremstyle{marginbreak} \newtheorem{Lem}[Cor]{Lemma}
\theoremstyle{change}
\theorembodyfont{\itshape} \newtheorem{Def}[Cor]{Definition}

\theoremheaderfont{\scshape}

\RecordChanges

\begin{document}
   \DocInput{theorem.dtx}
\end{document}
%</driver>
%    \end{macrocode}
%
% \section{Definition of the Macros}
%
% If the file has been loaded before, we abort immediately. If not the
% package announces itself (this is actually done at the very top of
% the file---the way it is done isn't good style so don't copy it).
% \changes{v2.0e}{89/07/19}{Spaces removed from `typeout'.}
% \changes{v2.2b}{95/11/19}{Announce moved to top of file}
%    \begin{macrocode}
%<*package>
%\@ifundefined{theorem@style}{}{\endinput}
%\def\FMithmInfo{1995/11/19 v2.2b Theorem extension package (FMi)}
%\ProvidesPackage{theorem}[\FMithmInfo]
%    \end{macrocode}
%
%
% \subsection{Definition of theorem styles and fonts}
%
% All the definitions in this file are done globally to allow
% inputting this file inside a group.
%
% \begin{macro}{\theoremstyle}
%    Before a theorem style can be installed, the chosen style must be
%    known. For that reason, we must test to see that
%    |\th@|\meta{style} is known or, more precisely, that it is
%    different from |\relax|. If the style is not known then
%    |\th@plain| is used.
%    \begin{macrocode}
\gdef\theoremstyle#1{%
   \@ifundefined{th@#1}{\@warning
          {Unknown theoremstyle `#1'. Using `plain'}%
          \theorem@style{plain}}%
%    \end{macrocode}
%    We save the theorem style to be used in the token register
%    |\theorem@style|.
%    \begin{macrocode}
      {\theorem@style{#1}}%
%    \end{macrocode}
%    Now we ``evaluate'' the theorem style: this means, we call the
%    macro |\th@|\meta{style} which will activate the relevant
%    definitions which are contained in a separate file.  This is done
%    in a group to suppress changes to the current font.  This could
%    otherwise pose problems together with the new font selection
%    scheme\footnote{When I printed the original article using the new
%    font selection scheme I ended with a document with slanted
%    typefaces (text headings and all) simply because one of the
%    theorem styles used \texttt{\bslash sl} at toplevel.} if the
%    |\th@|\meta{style} is evaluated a second time.
% \changes{v2.1a}{90/02/15}{Group added around `th@..}
% \changes{v2.1a}{90/02/15}{Reseting of `..@begintheorem removed.}
%    \begin{macrocode}
      \begingroup
        \csname th@\the\theorem@style \endcsname
      \endgroup}
%    \end{macrocode}
% \end{macro}
%
%
% \begin{macro}{\@begintheorem}
% \changes{v2.1a}{90/02/15}{Relaxed at toplevel.}
% \begin{macro}{\@opargbegintheorem}
% \changes{v2.1a}{90/02/15}{Relaxed at toplevel.}
%    We reset |\@begintheorem| and |\@opargbegintheorem| to
%    |\relax| since these commands are no longer necessary at
%    toplevel. This will save a few tokens.
%    \begin{macrocode}
\global\let\@begintheorem\relax
\global\let\@opargbegintheorem\relax
%    \end{macrocode}
% \end{macro}
% \end{macro}
%
% \begin{macro}{\theorem@style}
%    Obviously the \textsf{token} register used above has to be
%    allocated. To assure the utmost compatibility with the
%    original \LaTeX{} definition, we set the default theorem style to
%    \texttt{plain}, which implements the usual \LaTeX{} convention.
% \changes{v2.0f}{89/07/20}{Unnecessary `global' removed.}
%    \begin{macrocode}
\newtoks\theorem@style
\global\theorem@style{plain}
%    \end{macrocode}
% \end{macro}
%
% \begin{macro}{\theorembodyfont}
% \begin{macro}{\theorem@bodyfont}
%    If the |\theorembodyfont| is set by the user then it should not
%    interact with the default font set in the theorem style. When the
%    new font selection is in force this may happen if, for example,
%    the default is |\itshape| and the new |\theorembodyfont| is
%    |\sffamily|. So we add a |\reset@font| command in front of the
%    user definition.
% \changes{v2.0f}{89/07/20}{Unnecessary `global' removed.}
% \changes{v2.0g}{89/09/19}{`theoremfont to `theorembodyfont.}
% \changes{v2.1c}{92/06/22}{`reset@font added and logic changed.}
% \changes{v2.1d}{92/10/28}{Reset to empty if arg is empty}
%    \begin{macrocode}
\gdef\theorembodyfont#1{%
%    \end{macrocode}
%    We check if the argument supplied is empty and if so put nothing
%    into the |\theorem@bodyfont| token register to allow for
%    |\theorembodyfont{}| as a mean of using the default of the
%    current |\theoremstyle|.
%    \begin{macrocode}
   \def\@tempa{#1}%
   \ifx\@tempa\@empty
    \theorem@bodyfont{}%
   \else
    \theorem@bodyfont{\reset@font#1}%
   \fi
   }
\newtoks\theorem@bodyfont
\global\theorem@bodyfont{}
%    \end{macrocode}
% \end{macro}
% \end{macro}
%
%
% \begin{macro}{\theoremheaderfont}
% \changes{v2.0g}{89/09/19}{Macro added.}
%    The font for the theorem headers is handled differently because
%    this definition applies to all theorem styles.
%    \begin{macrocode}
\gdef\theoremheaderfont#1{\gdef\theorem@headerfont{#1}%
%    \end{macrocode}
%    After using the macro once it is redefined to produce an error
%    message.
%    \begin{macrocode}
       \gdef\theoremheaderfont##1{%
        \typeout{\string\theoremheaderfont\space should be used
                 only once.}}}
%    \end{macrocode}
% \end{macro}
%
% \begin{macro}{\theorem@headerfont}
% \changes{v2.0g}{89/09/19}{Macro added.}
%    To set the |\theorem@headerfont| default we first test if
%    the new fontselection scheme is in force.
% \changes{v2.1a}{90/02/15}{Support for new font selection.}
%    \begin{macrocode}
\ifx\upshape\undefined
%    \end{macrocode}
%    If not we define it to expand into |\bfseries|. We don't use
%    |\let| just in case a following style option redefines this
%    macro.
%    \begin{macrocode}
\gdef\theorem@headerfont{\bfseries}
%    \end{macrocode}
%    Otherwise we reset the current shape before calling |\bfseries|.
%    \begin{macrocode}
\else \gdef\theorem@headerfont{\normalfont\bfseries}\fi
%    \end{macrocode}
% \end{macro}
%
% \begin{macro}{\th@plain}
% \begin{macro}{\th@break}
% \begin{macro}{\th@marginbreak}
% \begin{macro}{\th@changebreak}
% \begin{macro}{\th@change}
% \begin{macro}{\th@margin}
%    The different styles are defined in macros such as
%    |\th@plain|.  Since memory space is precious in
%    ``non-Big-versions'', we have to avoid offering too many unused
%    definitions.  Therefore we define these styles in separate files
%    that can be loaded on demand.  Thus the commands themselves only
%    load these files. We use |\@input@|, a \LaTeXe{} internal command
%    that ensures that the file will be listed with |\listfiles|.
% \changes{v2.2a}{94/02/02}{Use `@input@ to load theorem layout files}
%    \begin{macrocode}
\gdef\th@plain{\@input@{thp.sty}}
\gdef\th@break{\@input@{thb.sty}}
\gdef\th@marginbreak{\@input@{thmb.sty}}
\gdef\th@changebreak{\@input@{thcb.sty}}
\gdef\th@change{\@input@{thc.sty}}
\gdef\th@margin{\@input@{thm.sty}}
%    \end{macrocode}
%    This list will be expanded when new styles become
%    available.  For testing,  just append new theorem substyles
%    as document options.
% \end{macro}
% \end{macro}
% \end{macro}
% \end{macro}
% \end{macro}
% \end{macro}
%
%
%
% \subsection{Definition of a new theorem set}
%
% As already pointed out, a new theorem environment can be defined in
% three different ways:
% \begin{verbatim}
%    \newtheorem{Lem}{Lemma}
%    \newtheorem{Lem}{Lemma}[section]
%    \newtheorem{Lem}[Theorem]{Lemma}
%\end{verbatim}
% The function of the macro |\newtheorem| is to recognize these cases
% and then to branch into one of the three macros |\@ynthm|, |\@xnthm|
% or |\@othm|.  This mechanism is adopted unchanged from
% \cite{latex:source}; the essential point here is that, for example,
% in the second case, the arguments \texttt{Lem}, \texttt{Lemma} and
% \texttt{section} are passed over to the macro |\@xnthm|.
%
% We inspect this case first because the others present fewer problems,
% and thus are easily derived from this one.
%
% \begin{macro}{\@xnthm}
%    For our example arguments, the macro |\@xnthm| must fulfill
%    the following:
%    \begin{itemize}
%       \item
%          Define a new \LaTeX-counter `Lem'
%       \item
%          reset this counter within a |\section|
%       \item
%          define the macro |\theLem|
%       \item
%          define the environment macros |\Lem| and
%          |\endLem| using the current
%          |\theoremstyle| and |\theorem@bodyfont|.
%    \end{itemize}
%    Obviously, all this should happen only if the first argument
%    of |\@xnthm| (i.e.\ \texttt{Lem} in our example) is chosen so
%    as not to conflict with any previously defined commands or
%    environments.  This test is performed by the \LaTeX{} macro
%    |\@ifdefinable|.
%    \begin{macrocode}
\gdef\@xnthm#1#2[#3]{\expandafter\@ifdefinable\csname #1\endcsname
%    \end{macrocode}
%    Therefore, the first argument of |\@ifdefinable| is the
%    expansion (in the example, |\Lem|) of
%    |\csname||#1\endcsname|.  The second argument is
%    executed only if the test has been completed successfully.
%    \begin{macrocode}
   {%
%    \end{macrocode}
%    Now we define the new counter. The names of the \LaTeX{}
%    macros employed should speak for themselves:
% \changes{v2.2b}{1995/10/07}{Use \cs{@newctr}}
% \changes{v2.2c}{1995/11/23}{Use \cs{@newctr} properly.}
%    \begin{macrocode}
    \@definecounter{#1}\@newctr{#1}[#3]%
%    \end{macrocode}
%    Using |\@newctr| will give a proper error message if the counter
%    in |#3| is not defined.
%    In defining `|\theLem|' we must generate the desired macro
%    name by use of |\expandafter| and |\csname|.
%    \begin{macrocode}
    \expandafter\xdef\csname the#1\endcsname
%    \end{macrocode}
%    An |\xdef| is used in order to make the definition global,
%    and to ensure that it contains the replacement texts of
%    |\@thmcountersep| and |\@thmcounter|.\footnote{These
%    two macros can be defined by the document style.  Their default
%    values produce a `\texttt{.}' as separation and an arabic
%    representation of the number.} However, not everything should be
%    expanded. For example, it saves space to use |\thesection|
%    instead of its---at times---lengthy expansion.
%    \begin{macrocode}
      {\expandafter \noexpand \csname the#3\endcsname
       \@thmcountersep \@thmcounter{#1}}%
%    \end{macrocode}
%    Thus with the defaults of \LaTeX{}, |\theLem| would be
%    replaced by the command sequence |\thesection.\arabic{Lem}|.
%
%    We will now look at the definition of the macro which is executed
%    at the beginning of the actual environment (in our example this
%    macro is |\Lem|). It should be noted that we use an
%    ``|\expandafter| trick'' to expand only certain parts of the
%    replacement text at the time of the definition.
%    \begin{macrocode}
    \def\@tempa{\global\@namedef{#1}}%
    \expandafter \@tempa \expandafter{%
%    \end{macrocode}
%    First, the macro that contains the current definitions of
%    |\@begintheorem| and |\@opargtheorem| should be called
%    up.  The name of this macro---as is already known---has the form
%    |\th@|\meta{theorem\ style}; therefore, it must be called by
%    \begin{macrocode}
      \csname th@\the \theorem@style
%    \end{macrocode}
%    In addition the default theorem font should be changeable, i.e.\
%    we have to insert the contents of |\theorem@bodyfont|. For
%    that reason, we expand even further, beyond |\endcsname|,
%    and thus insert the contents of the \textsf{token} register
%    |\theorem@bodyfont| in the replacement text.
%    \begin{macrocode}
            \expandafter \endcsname \the \theorem@bodyfont
%    \end{macrocode}
%    Now it is time to call the macro |\@thm| which takes over
%    the further processing.  It has two arguments: the current
%    counter name (in our example, \texttt{Lem}), and the text of the
%    label (in our example, \texttt{Lemma}).
%    \begin{macrocode}
     \@thm{#1}{#2}}%
%    \end{macrocode}
%    With this, the `sub-definition' is complete.  The macro
%    |\@endtheorem| ends a theorem environment and is, so far, nothing
%    but an |\endtrivlist|.  (Hence it is defined globally, and not
%    within the theorem styles.\footnote{This has to be changed as
%    soon as theorem styles that change \texttt{\bslash @endtheorem}
%    exist.  In such a case, all existing styles must be changed as
%    well since they will have to reset the macro.}) Therefore, we can
%    set it equivalent to the macro that ends the theorem set (in our
%    example, |\endLem|).  However, if some day theorem styles exist
%    that do change |\@endtheorem|, we would have to use the
%    commented-out line instead.
%    \begin{macrocode}
    \global \expandafter \let \csname end#1\endcsname \@endtheorem
%   \global\@namedef{end#1}{\@endtheorem}%
%    \end{macrocode}
%    With these commands all the required definitions are employed,
%    unless the test |\@ifdefinable| has failed.  Therefore, we
%    end the second argument of this macro and with it the definition
%    of |\@xnthm|.
%    \begin{macrocode}
   }}
%    \end{macrocode}
% \end{macro}
%
% \begin{macro}{\@ynthm}
%    The definition of |\@ynthm| is completely analogous. In
%    this case the new counter that is defined is not reset within
%    another counter; thus the definition of |\the...|
%    is simplified:
%    \begin{macrocode}
\gdef\@ynthm#1#2{\expandafter\@ifdefinable\csname #1\endcsname
   {\@definecounter{#1}%
    \expandafter\xdef\csname the#1\endcsname{\@thmcounter{#1}}%
%    \end{macrocode}
%    The rest of the definition corresponds literally to that of
%    |\@xnthm|:
%    \begin{macrocode}
    \def\@tempa{\global\@namedef{#1}}\expandafter \@tempa
     \expandafter{\csname th@\the \theorem@style \expandafter
     \endcsname \the\theorem@bodyfont \@thm{#1}{#2}}%
    \global \expandafter \let \csname end#1\endcsname \@endtheorem}}
%    \end{macrocode}
% \end{macro}
%
% \begin{macro}{\@othm}
%    The definition of |\@othm| does not contain anything new.
%    \begin{macrocode}
\gdef\@othm#1[#2]#3{%
%    \end{macrocode}
% \changes{v2.2b}{95/10/07}{Build in check for counter defined}
%    We do not define a new counter but instead use one that has
%    already been defined.  Thus the only definition we need is that
%    of this pseudo-counter (i.e.\ |\the|\meta{env.\ name}).
%    First we check if |#2| corresponds to a known counter name.
%    \begin{macrocode}
  \expandafter\ifx\csname c@#2\endcsname\relax
   \@nocounterr{#2}%
  \else
   \expandafter\@ifdefinable\csname #1\endcsname
   {\expandafter \xdef \csname the#1\endcsname
     {\expandafter \noexpand \csname the#2\endcsname}%
%    \end{macrocode}
%    All other parts of the definition can be adopted from
%    |\@xnthm|. We have to remember, though, that in this case
%    the name of the current counter and the theorem label have moved
%    to the second and third arguments.
%    \begin{macrocode}
    \def\@tempa{\global\@namedef{#1}}\expandafter \@tempa
     \expandafter{\csname th@\the \theorem@style \expandafter
     \endcsname \the\theorem@bodyfont \@thm{#2}{#3}}%
    \global \expandafter \let \csname end#1\endcsname \@endtheorem}%
  \fi}
%    \end{macrocode}
% \end{macro}
%
% \subsection{Macros that are employed in a {\envfont theorem}
%    environment}
%
% \begin{macro}{\@thm}
%    The macro |\@thm| has to increase the current counter.  Then,
%    depending on whether the environment has (or does not have) an
%    optional argument, it has to branch into either |\@begintheorem|
%    or |\@opargtheorem|.
%    \begin{macrocode}
\gdef\@thm#1#2{\refstepcounter{#1}%
%    \end{macrocode}
%    Now we start a {\envfont trivlist} environment, and give
%    |\@topsep| and |\@topsepadd| the values of the \textsf{skip}
%    registers |\theorempreskipamount| and |\theorempostskipamount|.
%    The value in |\@topsep| is the vertical space that is inserted by
%    the first (and only) |\item| in our |\trivlist| whilst
%    |\@topsepadd| is inserted by |\@endparenv| at the end of that
%    {\envfont trivlist} environment.  By using these registers, we
%    obtain the desired space around a {\envfont theorem} environment.
%    \begin{macrocode}
   \trivlist
   \@topsep \theorempreskipamount               % used by first \item
   \@topsepadd \theorempostskipamount           % used by \@endparenv
%    \end{macrocode}
%    Now we have to test whether an optional argument has been given.
% \changes{v2.0d}{89/07/18}{Forgotten percent added.}
%    \begin{macrocode}
   \@ifnextchar [%
%    \end{macrocode}
%    If there is an optional argument, we will call |\@ythm|, and
%    move the arguments read back into the input stream.
%    \begin{macrocode}
   {\@ythm{#1}{#2}}%
%    \end{macrocode}
%    If not, we call |\@begintheorem|. Its first argument is the
%    name of the theorem set (hence the second argument of
%    |\@thm|). Its second argument is the macro that produces the
%    current number.
%    \begin{macrocode}
   {\@begintheorem{#2}{\csname the#1\endcsname}\ignorespaces}}
%    \end{macrocode}
% \end{macro}
%
% \begin{macro}{\@xthm}
% \begin{macro}{\@ythm}
%    Both these macros were originally called by |\@thm|.  We do
%    not need |\@xthm| anymore, hence we reset it to
%    |\relax|. The definition of |\@ythm| has not changed at
%    all from its definition in \LaTeX. In order to make the macros
%    easier to understand, we will
%    nevertheless present it (commented out).
%    \begin{macrocode}
\global\let\@xthm\relax
% \def\@ythm#1#2[#3]{\@opargbegintheorem{#2}{\csname
%           the#1\endcsname}{#3}\ignorespaces}
%</package>
%    \end{macrocode}
%    The primitive |\ignorespaces| in |\@ythm| and
%    |\@thm| is needed to remove the spaces between the
%    |\begin{...}| and the actual text.
% \end{macro}
% \end{macro}
%
%
% \subsection{Definition of the theorem substyles}
%
% As already pointed out, the theorem substyles, defined below, are
% only loaded when necessary.  Note that all these substyles, except
% \texttt{plain}, have |\slshape| as the default body font.
%
% \subsubsection{The \texttt{plain} style}
%
% \label{dok:th@plain} As the following macros use |@|, we have
% to locally set the |\catcode| of this symbol to ``letter''.
% This happens within a group, so that we do not have to worry about
% which |\catcode| that symbol had before.
%    \begin{macrocode}
%<*thp>
\begingroup \makeatletter
%    \end{macrocode}
% Since we are now within a group, we must make all definitions
% globally.  First we make sure that \texttt{theorem.sty} is loaded.
% This will allow us to use this file as a document style option
% without having to call \texttt{theorem} itself as an option.  At the
% same time, we assure that at least version 2 is loaded, since
% |\theorem@style| was not defined in earlier versions.
%    \begin{macrocode}
\@ifundefined{theorem@style}{\input{theorem.sty}}{}
\ProvidesFile{thp.sty}
             [\FMithmInfo]
%    \end{macrocode}
%
%
% \begin{macro}{\th@plain}
% \begin{macro}{\@begintheorem}
% \begin{macro}{\@opargbegintheorem}
%    |\theoremstyle{plain}| corresponds to the original
%    definition, except that the distances to the surrounding text are
%    determined by the parameters |\theorempreskipamount| and
%    |\theorempostskipamount|.
%    First we set the default body font.
%    \begin{macrocode}
\gdef\th@plain{\normalfont\itshape
%    \end{macrocode}
%    Then we define |\@begintheorem| and
%    |\@opargbegintheorem|. These two macros define how the
%    header of a theorem is typeset.  |\@opargbegintheorem| will
%    be called if a \textsf{theorem} environment with an optional
%    argument is encountered; otherwise, the header is constructed by
%    calling |\@begintheorem|.  If one of these macros is
%    executed, we are within a \textsf{trivlist} environment started by
%    |\@thm|.  So the theorem header is produced with an
%    |\item| command.
%
%    Instead of specifying the header font directly, all standard
%    theorem styles use the |\theorem@headerfont| macro to allow
%    customization.  The extra space (|\labelsep|) is necessary
%    because of problems in the \textsf{trivlist} environment.
%    \begin{macrocode}
  \def\@begintheorem##1##2{%
        \item[\hskip\labelsep \theorem@headerfont ##1\ ##2]}%
%    \end{macrocode}
%    The definition of |\@opargbegintheorem| is completely
%    analogous. The only difference is the fact that there exists a
%    third argument (which is the optional parameter of the
%    environment and contains additional information about the
%    theorem).  Customarily we enclose it in parentheses.
%    \begin{macrocode}
\def\@opargbegintheorem##1##2##3{%
   \item[\hskip\labelsep \theorem@headerfont ##1\ ##2\ (##3)]}}
%    \end{macrocode}
% \end{macro}
% \end{macro}
% \end{macro}
%
% \noindent
% We conclude with an |\endgroup| to restore the
% |\catcode| of |@|.
%    \begin{macrocode}
\endgroup
%</thp>
%    \end{macrocode}
%
%
% \subsubsection{The \texttt{break} style}
%
% This style option is stored in the file |thb.sty|.
% For the next two lines see the documentation for |\th@plain| on
% page~\pageref{dok:th@plain}.
%    \begin{macrocode}
%<*thb>
\begingroup \makeatletter
\@ifundefined{theorem@style}{\input{theorem.sty}}{}
\ProvidesFile{thb.sty}
             [\FMithmInfo]
%    \end{macrocode}
%
% \begin{macro}{\th@break}
%    |\theoremstyle{break}| produces a line break after the name
%    of the theorem.  The font is |\slshape|.  Hence, we define
%    |\th@break| as follows:
%    \begin{macrocode}
\gdef\th@break{\normalfont\slshape
  \def\@begintheorem##1##2{\item[%
%    \end{macrocode}
%    We run into the following problem: it is not possible to create
%    the header with |\item[|\meta{title}|]| and then start
%    a new line by, for example, |\mbox{}\\|. Such a definition
%    will fail whenever a list environment follows immediately.  With
%    the above construction, the |\mbox{}| causes the switch
%    |@inlabel| (cf.\ definition of |\list| and
%    |\trivlist| in \cite{latex:source}) to be set to \texttt{false}
%    and so the following list will insert additional vertical
%    space (|\topskip|). This is quite annoying. Therefore, we
%    create the line break within the |\item|. In order to ensure
%    that the text will begin at the proper position in the following
%    line, we simply pretend that the label does not take any
%    room.\footnote{This will lead to problems whenever very high
%    symbols occurring in the line tower into the
%    heading. So, something else has to be done here sometime.}
%    \begin{macrocode}
       \rlap{\vbox{\hbox{\hskip \labelsep\theorem@headerfont ##1\ ##2}%
                   \hbox{\strut}}}]}%
%    \end{macrocode}
%    Again, the definition of |\@opargbegintheorem| is
%    completely analogous.
%    \begin{macrocode}
\def\@opargbegintheorem##1##2##3{%
  \item[\rlap{\vbox{\hbox{\hskip \labelsep \theorem@headerfont
                     ##1\ ##2\ (##3)}%
                    \hbox{\strut}}}]}}
\endgroup
%</thb>
%    \end{macrocode}
% \end{macro}
%
%
% \subsubsection{The \texttt{changebreak} style}
%
%    \begin{macrocode}
% This style option is stored in the file |thcb.sty|.
%    \begin{macrocode}
%<*thcb>
\begingroup \makeatletter
\@ifundefined{theorem@style}{\input{theorem.sty}}{}
\ProvidesFile{thcb.sty}
             [\FMithmInfo]
%    \end{macrocode}
%
% \begin{macro}{\th@changebreak}
%    The \texttt{change-break} theorem style is like \texttt{break}
%    but with interchange of theorem name and theorem number.  Thus we
%    define |\th@changebreak| as follows:
%    \begin{macrocode}
\gdef\th@changebreak{\normalfont\slshape
  \def\@begintheorem##1##2{\item
       [\rlap{\vbox{\hbox{\hskip\labelsep \theorem@headerfont ##2\ ##1}%
                    \hbox{\strut}}}]}%
\def\@opargbegintheorem##1##2##3{%
  \item[\rlap{\vbox{\hbox{\hskip\labelsep \theorem@headerfont
                     ##2\ ##1\ (##3)}%
                    \hbox{\strut}}}]}}
\endgroup
%</thcb>
%    \end{macrocode}
% \end{macro}
%
%
% \subsubsection{The \texttt{change} style}
%
% This style option is stored in the file |thc.sty|.
%    \begin{macrocode}
%<*thc>
\begingroup \makeatletter
\@ifundefined{theorem@style}{\input{theorem.sty}}{}
\ProvidesFile{thc.sty}
             [\FMithmInfo]
%    \end{macrocode}
%
% \begin{macro}{\th@change}
%    The \texttt{change} theorem style corresponds to the
%    \texttt{change break} style without a linebreak after the header.
%    To say it in another way, it's the same as the \texttt{plain}
%    style but with number and name interchanged and |\slshape| as the
%    default font.
%    \begin{macrocode}
\gdef\th@change{\normalfont\slshape
  \def\@begintheorem##1##2{\item
       [\hskip\labelsep \theorem@headerfont ##2\ ##1]}%
\def\@opargbegintheorem##1##2##3{%
  \item[\hskip\labelsep \theorem@headerfont ##2\ ##1\ (##3)]}}
\endgroup
%</thc>
%    \end{macrocode}
% \end{macro}
%
%
% \subsubsection{The \texttt{marginbreak} style}
%
% This style option is the one used most often at Mainz.  It is saved
% in the file |thmb.sty|.
%    \begin{macrocode}
%<*thmb>
\begingroup \makeatletter
\@ifundefined{theorem@style}{\input{theorem.sty}}{}
\ProvidesFile{thmb.sty}
             [\FMithmInfo]
%    \end{macrocode}
%
% \begin{macro}{\th@marginbreak}
%    The \texttt{margin break} style is nearly the same as the
%    \texttt{change break} style. The only difference is the placement
%    of the theorem number.  We use |\llap| to place it in the left
%    margin.
%
%    In this style |\labelsep| denotes the separation between the
%    number and the text.
%    \begin{macrocode}
\gdef\th@marginbreak{\normalfont\slshape
  \def\@begintheorem##1##2{\item
       [\rlap{\vbox{\theorem@headerfont
                    \hbox{\llap{##2}\hskip\labelsep ##1}%
                    \hbox{\strut}}}]}%
\def\@opargbegintheorem##1##2##3{%
  \item[\rlap{\vbox{\theorem@headerfont
                    \hbox{\llap{##2}\hskip\labelsep ##1\ (##3)}%
                    \hbox{\strut}}}]}}
\endgroup
%</thmb>
%    \end{macrocode}
% \changes{v2.1c}{90/02/17}{`endgroup outside macrocode env.}
% \end{macro}
%
%
%
% \subsubsection{The \texttt{margin} style}
%
% This style option is stored in the file |thm.sty|.
%    \begin{macrocode}
%<*thm>
\begingroup \makeatletter
\@ifundefined{theorem@style}{\input{theorem.sty}}{}
\ProvidesFile{thm.sty}
             [\FMithmInfo]
%    \end{macrocode}
%
% \begin{macro}{\th@margin}
%    Again this is only a variant of the theorem styles described
%    above without any new ideas.
%    \begin{macrocode}
\gdef\th@margin{\normalfont\slshape
  \def\@begintheorem##1##2{\item
       [\theorem@headerfont \llap{##2}\hskip\labelsep ##1]}%
\def\@opargbegintheorem##1##2##3{%
  \item[\theorem@headerfont \llap{##2}\hskip\labelsep ##1\ (##3)]}}
\endgroup
%</thm>
%    \end{macrocode}
% \end{macro}
%
%
%
%
% \subsection{Final Definitions}
%
% \begin{macro}{\theorempreskipamount}
% \begin{macro}{\theorempostskipamount}
%    The \textsf{skip} parameters that regulate the vertical empty space
%    before and after the {\envfont theorem} environment have
%    to be allocated as well.
% \changes{v2.0f}{89/07/20}{Unnecessary `global' removed.}
%    \begin{macrocode}
%<*package>
\newskip\theorempreskipamount
\newskip\theorempostskipamount
%    \end{macrocode}
%    Since we have used the same values for all theorem sets, we now can
%    assign them.
%    \begin{macrocode}
\global\setlength\theorempreskipamount{12pt plus 5pt minus 3pt}
\global\setlength\theorempostskipamount{8pt plus 3pt minus 1.5pt}
%    \end{macrocode}
% \end{macro}
% \end{macro}
%
%
% \begin{macro}{\@endtheorem}
%    The same holds for the macro |\@endtheorem|, which ends a
%    {\envfont theorem} environment. Since it is the same for all
%    theorem sets, it is removed from the macros |\th@|\meta{style}.
%    It simply ends the {\envfont trivlist} environment, which was
%    begun in |\@thm|.
%    \begin{macrocode}
\global\let\@endtheorem=\endtrivlist
%    \end{macrocode}
% \end{macro}
%
%
%
% \begin{macro}{\@preamblecmds}
%    All macros defined above are to be used only in the preamble.
%    Therefore, we insert them in |\@preamblecmds| which will disable
%    them at begin document. This is done by the internal \LaTeXe{}
%    command |\@onlypreamble|.
% \changes{v2.0g}{89/09/19}{`theoremheaderfont and `theorembodyfont
%    added.}
% \changes{v2.2a}{94/02/03}{Use \LaTeXe{} interface}
%    \begin{macrocode}
\@onlypreamble\@xnthm
\@onlypreamble\@ynthm
\@onlypreamble\@othm
\@onlypreamble\newtheorem
\@onlypreamble\theoremstyle
\@onlypreamble\theorembodyfont
\@onlypreamble\theoremheaderfont
%    \end{macrocode}
% \end{macro}
%
% Finally we declare the \texttt{plain} theorem style to be the default.
% \changes{v2.1b}{90/06/04}{Fixed missing `item by declaring default
%    style.}
%    \begin{macrocode}
\theoremstyle{plain}
%</package>
%    \end{macrocode}
%
% \Finale

\endinput
}{}
\ProvidesFile{thb.sty}
             [\FMithmInfo]
%    \end{macrocode}
%
% \begin{macro}{\th@break}
%    |\theoremstyle{break}| produces a line break after the name
%    of the theorem.  The font is |\slshape|.  Hence, we define
%    |\th@break| as follows:
%    \begin{macrocode}
\gdef\th@break{\normalfont\slshape
  \def\@begintheorem##1##2{\item[%
%    \end{macrocode}
%    We run into the following problem: it is not possible to create
%    the header with |\item[|\meta{title}|]| and then start
%    a new line by, for example, |\mbox{}\\|. Such a definition
%    will fail whenever a list environment follows immediately.  With
%    the above construction, the |\mbox{}| causes the switch
%    |@inlabel| (cf.\ definition of |\list| and
%    |\trivlist| in \cite{latex:source}) to be set to \texttt{false}
%    and so the following list will insert additional vertical
%    space (|\topskip|). This is quite annoying. Therefore, we
%    create the line break within the |\item|. In order to ensure
%    that the text will begin at the proper position in the following
%    line, we simply pretend that the label does not take any
%    room.\footnote{This will lead to problems whenever very high
%    symbols occurring in the line tower into the
%    heading. So, something else has to be done here sometime.}
%    \begin{macrocode}
       \rlap{\vbox{\hbox{\hskip \labelsep\theorem@headerfont ##1\ ##2}%
                   \hbox{\strut}}}]}%
%    \end{macrocode}
%    Again, the definition of |\@opargbegintheorem| is
%    completely analogous.
%    \begin{macrocode}
\def\@opargbegintheorem##1##2##3{%
  \item[\rlap{\vbox{\hbox{\hskip \labelsep \theorem@headerfont
                     ##1\ ##2\ (##3)}%
                    \hbox{\strut}}}]}}
\endgroup
%</thb>
%    \end{macrocode}
% \end{macro}
%
%
% \subsubsection{The \texttt{changebreak} style}
%
%    \begin{macrocode}
% This style option is stored in the file |thcb.sty|.
%    \begin{macrocode}
%<*thcb>
\begingroup \makeatletter
\@ifundefined{theorem@style}{% \iffalse meta-comment
%
% Copyright (C) 1993-2024
%
% The LaTeX Project and any individual authors listed elsewhere
% in this file.
%
% This file is part of the Standard LaTeX `Tools Bundle'.
% -------------------------------------------------------
%
% It may be distributed and/or modified under the
% conditions of the LaTeX Project Public License, either version 1.3c
% of this license or (at your option) any later version.
% The latest version of this license is in
%    https://www.latex-project.org/lppl.txt
% and version 1.3c or later is part of all distributions of LaTeX
% version 2005/12/01 or later.
%
% The list of all files belonging to the LaTeX `Tools Bundle' is
% given in the file `manifest.txt'.
%
% \fi
%
%%
%
%
\def\FMithmInfo{2023/07/05 v2.2c Theorem extension package (FMi)}
%
% \ProvidesFile{theorem.dtx}[\FMithmInfo]
%\iffalse   % this is a METACOMMENT !
%%
%% Package `theorem' to use with LaTeX2e
%% Copyright (C) 1989-1995 Frank Mittelbach, all rights reserved.
%<+package>\@ifundefined{theorem@style}{}{\endinput}
%<+package>\NeedsTeXFormat{LaTeX2e}[1995/06/01]
%<+package>\ProvidesPackage{theorem}
%<+package>         [\FMithmInfo]
%
% \fi
%
% \changes{v1.0}{87/05/07}{First draft.}
% \changes{v1.0a}{87/05/07}{`thistheoremstyle' added.}
% \changes{v1.1a}{87/07/29}{New concept: redefine the internal
%                          \protect\LaTeX{} commands `@xnthm' etc.}
% \changes{v1.1b}{87/07/31}{`theoremfont' used to specify font.}
% \changes{v1.2a}{88/01/08}{`@endtheorem' now globally defined.}
% \changes{v1.2b}{88/01/11}{`newtheorem' only allowed in the preamble.}
% \changes{v1.3a}{88/02/14}{`theoremfont' now a toks register.}
%
% \changes{v2.0a}{88/11/16}{New implementation and new documentation.}
% \changes{v2.0a}{88/11/16}{No layout changes but internal modifications
%                           old substyles won't work.}
% \changes{v2.0b}{89/01/30}{`global' used for all definitions.}
% \changes{v2.0c}{89/06/15}{Documentation translated to English.
%                            Thanks to LLL and Nina}
% \changes{v2.1a}{90/02/15}{Support for new font selection.}
% \changes{v2.2a}{94/02/03}{Cleanup for LaTeX2e}
%
% \DoNotIndex{\@endparenv,\@ifdefinable,\@ifnextchar,\@ifundefined}
% \DoNotIndex{\@namedef,\@nameuse,\@spaces,\@tempa,\@warning}
% \DoNotIndex{\ }
% \DoNotIndex{\begingroup,\bfseries}
% \DoNotIndex{\csname}
% \DoNotIndex{\def,\do,\docdate}
% \DoNotIndex{\endcsname,\endgroup,\endinput,\endtrivlist,\expandafter}
% \DoNotIndex{\filedate,\fileversion}
% \DoNotIndex{\gdef,\global}
% \DoNotIndex{\hbox,\hskip}
% \DoNotIndex{\ignorespaces,\input,\itshape,\item}
% \DoNotIndex{\labelsep,\let,\llap}
% \DoNotIndex{\makeatletter}
% \DoNotIndex{\noexpand}
% \DoNotIndex{\refstepcounter,\relax,\rlap}
% \DoNotIndex{\setlength,\slshape,\space,\string,\strut}
% \DoNotIndex{\the,\trivlist,\typeout}
% \DoNotIndex{\vbox}
% \DoNotIndex{\xdef}
%
% \def\deflabel#1{\ttfamily #1\hfill}
% \def\deflist#1{\begin{list}{}{\settowidth\labelwidth{\ttfamily #1}%
%                               \setlength\leftmargin\labelwidth
%                               \addtolength\leftmargin\labelsep
%                               \let\makelabel\deflabel}}
% \def\enddeflist{\end{list}}
% \def\envfont{\normalfont\ttfamily}
%
% \MakeShortVerb{\|}
%
% \newcommand{\TUB}{{\sl TUGboat\/}}
%
% \GetFileInfo{theorem.dtx}
% \title{An Extension of the
%        \LaTeX{} theorem environment\thanks{This file
%        has version number \fileversion, last
%        revised \filedate.}}
% \author{Frank Mittelbach}
%
% \MaintainedByLaTeXTeam{tools}
% \maketitle
%
% \begin{abstract}
%    The macros described in this paper yield an extension of the
%    \LaTeX{} theorem mechanism. It is designed to satisfy the
%    different requirements of various journals. Thus, the
%    layout of the ``theorems'' can be manipulated by determining a
%    ``style''. This article describes not only the use, but
%    also the definition, of the necessary macros.
% \end{abstract}
%
% \section*{Preface to version 2.2}
%
% For \LaTeXe{} this package did not need any fundamental changes. I
% only modified the messages generated so that theorem layout styles
% will show up with the |\listfiles| command and cleaned the section
% on the New Font Selection Scheme since this is now included in
% \LaTeX.
%
% \section*{Preface to version 2.1}
%
% This version is identical to 2.0g described in \TUB\ 10\#3 except
% for some internal defaults which are now set depending on the used
% font selection scheme.
%
% This was done to avoid unpleasant surprises if the new font selection
% scheme is in force. For further details see
% section~\ref{sec:newfonts} and \cite{bk:GMS94}.
%
% \section{Introduction}
%
% For our purposes here, ``theorems'' are labelled enunciations,
% often set off from the main text by extra space and a font change.
% Theorems, corollaries, conjectures, definitions, and remarks
% are all instances of ``theorems''.  The ``header'' of these structures
% is composed of a label (such as \textsc{Theorem} or \textsc{Remark})
% and a number which serializes an item in the sequence of items
% with the same label.
%
% Shortly after the introduction of \LaTeX{} at the Fachbereich
% Mathematik in Mainz, the desire to manipulate the layout of
% ``theorems'' arose.  In Mainz, the following two conventions came
% into general use:
% \begin{enumerate}
%   \item The number of the theorem is shown in the margin.
%   \item There is a line break at
%         the end of the theorem header.
% \end{enumerate}
%
% Additionally, some journals require different formats which depend on
% the ``sort of theorem'': e.g.\ often remarks and definitions are set
% in |\upshape|, while |\itshape| is employed for main theorems.
%
% Confronted with these requirements, a theorem environment was
% developed in Mainz which allows separate determination of the
% layout of the ``theorems sets'', comparable to |\pagestyle|.
%
%
% \section{The user interface}
% \subsection{Defining new theorem sets}
%
% \DescribeMacro\newtheorem
% As in the original \LaTeX{} version, the command |\newtheorem|
% defines a new ``theorem set'' or ``theorem-like structure''.
% Two required arguments name the new environment and give the text
% to be typeset with each instance of the new ``set'', while
% an optional argument determines how
% the ``set'' is enumerated:
% \begin{description}
%    \item[\ttfamily \bslash newtheorem\{foo\}\{bar\}]
%       The theorem set {\envfont foo} (whose name is \texttt{bar})
%       uses its own counter.
%    \item[\ttfamily \bslash newtheorem\{foo2\}{[foo]}\{bar2\}]
%       The theorem set {\envfont foo2} (printed name \texttt{bar2})
%       uses the same counter as the theorem set \texttt{foo}.
%    \item[\ttfamily \bslash newtheorem\{foo3\}\{bar3\}{[section]}]
%       The theorem set {\envfont foo3} (printed name \texttt{bar3}) is
%       enumerated within the counter \texttt{section}, i.e.\ with every
%       new |\section| the enumeration begins again with 1, and
%       the enumeration is composed from the section-number and the
%       theorem counter itself.
% \end{description}
%
% \DescribeMacro\theoremstyle
% Additionally, the command |\theoremstyle| can define the layout
% of various, or all, theorem sets.  It should be noted that any
% theorem set defined by |\newtheorem| is typeset in the
% |\theoremstyle| that is current at the time of the definition.
% Thus, the following
% \begin{verbatim}
%    \theoremstyle{break}        \newtheorem{Cor}{Corollary}
%    \theoremstyle{plain}        \newtheorem{Exa}{Example}[section]
%\end{verbatim}                             ^^A No blank at line start
% leads to the result that the set {\envfont Cor} is formatted in the
% style \texttt{break}, while the set {\envfont Exa} and all the
% following ones are formatted in the style \texttt{plain}, unless
% another |\theoremstyle| follows.  Since the definitions installed by
% |\newtheorem| are global, one also can limit |\theoremstyle| locally
% by grouping braces.
%
% \DescribeMacro\theorembodyfont
% The choice of the font for the theorem body is completely
% independent of the chosen |\theoremstyle|; this has proven to
% be very advantageous.  For example,
% \begin{verbatim}
%   {\theorembodyfont{\upshape}          \newtheorem{Rem}{Remark}}
%\end{verbatim}
% defines a theorem set {\envfont Rem}, which will be set in
% |\upshape| in the current layout (which in our example is
% \texttt{plain}).  As with |\theoremstyle|, the |\theorembodyfont|
% chosen is that current at the time of |\newtheorem|.  If
% |\theorembodyfont| is not specified or one defines
% |\theorembodyfont{}|, then the font used will be that defined by the
% |\theoremstyle|.
%
% \DescribeMacro\theoremheaderfont It is also possible to customize
% the font used for the theorem headers.  This is, however, a global
% declaration, and therefore there should be at most one
% |\theoremheaderfont| declaration in the preamble.\footnote{If it is
% actually necessary to have different header fonts, one has to define
% new theorem styles (substituting the desired font) or specify the
% information directly in the \texttt{\bslash newtheorem} declaration
% (the unclean variant).}
%
% \DescribeMacro\theorempreskipamount
% \DescribeMacro\theorempostskipamount Two additional parameters
% affect the vertical space around the theorem environments:
% |\theorempreskipamount| and |\theorempostskipamount| define,
% respectively, the spacing before and after such an environment.
% These parameters apply for all theorem sets and can be manipulated
% with the ordinary length macros.  They are rubber lengths
% (`\textsf{skips}'), and therefore can contain \texttt{plus} and
% \texttt{minus} parts.
%
% Since the definition of theorem sets should---most sensibly---be
% placed in the preamble, we only allow installation there.
% It is therefore possible to release the memory used here after
% |\begin{document}|, in order to make room for other
% applications.
%
%
% \subsection{Existing theorem styles}
%
% The following theorem styles exist to date:
% \begin{deflist}{marginbreak}
%    \item[plain]
%       This theorem style emulates the original \LaTeX{} definition,
%       except that additionally the parameters
%       |\theorem...skipamount| are used.
%    \item[break]
%       In this style, the theorem header is followed by a line
%       break.
%    \item[marginbreak]
%       The theorem number is set in the margin, and there is a line
%       break as in \texttt{break}.
%    \item[changebreak]
%       Like \texttt{break}, but with header number and text
%       interchanged.
%    \item[change]
%       Header number and text are interchanged, without a line break.
%    \item[margin]
%       The number is set in the left margin, without a line break.
% \end{deflist}
% All styles (except \texttt{plain}) select |\slshape| as the default
% |\theorembodyfont|.
%
%
% \subsection{Examples}
%
% Given the above theorem sets {\envfont Cor}, {\envfont Exa}
% and {\envfont Rem},
% suppose that the preamble also contains the declarations:
% \begin{verbatim}
%    \theoremstyle{marginbreak}   \newtheorem{Lem}[Cor]{Lemma}
%    \theoremstyle{change}
%    \theorembodyfont{\itshape}        \newtheorem{Def}[Cor]{Definition}
%
%    \theoremheaderfont{\scshape}
%\end{verbatim}
% Then the following are some typical examples of the typeset output
% resulting from their use.
%
% \begin{Cor}
%    This is a sentence typeset in the theorem environment {\envfont
%    Cor}.
% \end{Cor}
%
% \begin{Exa}
%    This is a sentence typeset in the theorem environment {\envfont
%    Exa}.
% \end{Exa}
%
% \begin{Rem}
%    This is a sentence typeset in the theorem environment {\envfont
%    Rem}.
% \end{Rem}
%
% \begin{Lem}[Ben User]
%    This is a sentence typeset in the theorem environment {\envfont
%    Lem}.
% \end{Lem}
%
% \begin{Def}[Very impressive Definition]
%    This is a sentence typeset in the theorem environment {\envfont
%    Def}.
% \end{Def}
%
% The last two examples show the effect of the optional argument to
% a theorem environment (it is the text typeset in parentheses).
%
%
% \section{Special Considerations}
% \label{sec:newfonts}
%
% Theoremheader and body are implemented as a unit. This means that
% the |\theoremheaderfont|  will inherit characteristics of the
% |\theorembodyfont| in \LaTeXe. Thus, if for example
% |\theorembodyfont| is |\itshape| and |\theoremheaderfont|
% is |\bfseries| the font selected for the header will have the
% characteristics `bold extended italic'. If this is not desired one
% should set the |\theoremheaderfont| to something like
% \begin{verbatim}
%      \theoremheaderfont{\normalfont\bfseries}
%\end{verbatim}
% i.e.\ supplying all necessary font information explicitly.
%
%
%
% \section{Acknowledgements}
%
% The publication of this set of macros was only possible with the
% help of Christina Busse (translating the manuscript into English),
% Joachim Pense (playing the r\^ole of typist), Chris Rowley
% (looking everything over) and many others providing useful
% suggestions.
%
% \MaybeStop{%
%    \begin{thebibliography}{9}
%       \bibitem{bk:GMS94} \textsc{M.~Goossens}, \textsc{F.~Mittelbach}
%         and \textsc{A.~Samarin}.
%         \newblock The \LaTeX{} Companion.
%         \newblock
%         Addison-Wesley, Reading, Massachusetts, 1994.
%       \bibitem{latex:source} \textsc{Lamport, Leslie.} \newblock
%                               \texttt{latex.tex}, version 2.09, date
%                               Feb. 1990.
%    \end{thebibliography}
%    \PrintIndex
%    \PrintChanges}
%
%
% \section{The documentation driver file}
%
% The next bit of code contains the documentation driver file for
% \TeX{}, i.e., the file that will produce the documentation you are
% currently reading. It will be extracted from this file by the
% \texttt{docstrip} program. Since it is the first code in the file
% one can alternatively process this file directly with \LaTeXe{} to
% obtain the documentation.
%
%    \begin{macrocode}
%<*driver>
\documentclass{ltxdoc}

\usepackage{theorem}

%<+driver>% The next few lines define theorem sets which are used
%<+driver>% in the example section of the documentation.

\theoremstyle{break}       \newtheorem{Cor}{Corollary}
\theoremstyle{plain}       \newtheorem{Exa}{Example}[section]
{\theorembodyfont{\upshape}\newtheorem{Rem}{Remark}}
\theoremstyle{marginbreak} \newtheorem{Lem}[Cor]{Lemma}
\theoremstyle{change}
\theorembodyfont{\itshape} \newtheorem{Def}[Cor]{Definition}

\theoremheaderfont{\scshape}

\RecordChanges

\begin{document}
   \DocInput{theorem.dtx}
\end{document}
%</driver>
%    \end{macrocode}
%
% \section{Definition of the Macros}
%
% If the file has been loaded before, we abort immediately. If not the
% package announces itself (this is actually done at the very top of
% the file---the way it is done isn't good style so don't copy it).
% \changes{v2.0e}{89/07/19}{Spaces removed from `typeout'.}
% \changes{v2.2b}{95/11/19}{Announce moved to top of file}
%    \begin{macrocode}
%<*package>
%\@ifundefined{theorem@style}{}{\endinput}
%\def\FMithmInfo{1995/11/19 v2.2b Theorem extension package (FMi)}
%\ProvidesPackage{theorem}[\FMithmInfo]
%    \end{macrocode}
%
%
% \subsection{Definition of theorem styles and fonts}
%
% All the definitions in this file are done globally to allow
% inputting this file inside a group.
%
% \begin{macro}{\theoremstyle}
%    Before a theorem style can be installed, the chosen style must be
%    known. For that reason, we must test to see that
%    |\th@|\meta{style} is known or, more precisely, that it is
%    different from |\relax|. If the style is not known then
%    |\th@plain| is used.
%    \begin{macrocode}
\gdef\theoremstyle#1{%
   \@ifundefined{th@#1}{\@warning
          {Unknown theoremstyle `#1'. Using `plain'}%
          \theorem@style{plain}}%
%    \end{macrocode}
%    We save the theorem style to be used in the token register
%    |\theorem@style|.
%    \begin{macrocode}
      {\theorem@style{#1}}%
%    \end{macrocode}
%    Now we ``evaluate'' the theorem style: this means, we call the
%    macro |\th@|\meta{style} which will activate the relevant
%    definitions which are contained in a separate file.  This is done
%    in a group to suppress changes to the current font.  This could
%    otherwise pose problems together with the new font selection
%    scheme\footnote{When I printed the original article using the new
%    font selection scheme I ended with a document with slanted
%    typefaces (text headings and all) simply because one of the
%    theorem styles used \texttt{\bslash sl} at toplevel.} if the
%    |\th@|\meta{style} is evaluated a second time.
% \changes{v2.1a}{90/02/15}{Group added around `th@..}
% \changes{v2.1a}{90/02/15}{Reseting of `..@begintheorem removed.}
%    \begin{macrocode}
      \begingroup
        \csname th@\the\theorem@style \endcsname
      \endgroup}
%    \end{macrocode}
% \end{macro}
%
%
% \begin{macro}{\@begintheorem}
% \changes{v2.1a}{90/02/15}{Relaxed at toplevel.}
% \begin{macro}{\@opargbegintheorem}
% \changes{v2.1a}{90/02/15}{Relaxed at toplevel.}
%    We reset |\@begintheorem| and |\@opargbegintheorem| to
%    |\relax| since these commands are no longer necessary at
%    toplevel. This will save a few tokens.
%    \begin{macrocode}
\global\let\@begintheorem\relax
\global\let\@opargbegintheorem\relax
%    \end{macrocode}
% \end{macro}
% \end{macro}
%
% \begin{macro}{\theorem@style}
%    Obviously the \textsf{token} register used above has to be
%    allocated. To assure the utmost compatibility with the
%    original \LaTeX{} definition, we set the default theorem style to
%    \texttt{plain}, which implements the usual \LaTeX{} convention.
% \changes{v2.0f}{89/07/20}{Unnecessary `global' removed.}
%    \begin{macrocode}
\newtoks\theorem@style
\global\theorem@style{plain}
%    \end{macrocode}
% \end{macro}
%
% \begin{macro}{\theorembodyfont}
% \begin{macro}{\theorem@bodyfont}
%    If the |\theorembodyfont| is set by the user then it should not
%    interact with the default font set in the theorem style. When the
%    new font selection is in force this may happen if, for example,
%    the default is |\itshape| and the new |\theorembodyfont| is
%    |\sffamily|. So we add a |\reset@font| command in front of the
%    user definition.
% \changes{v2.0f}{89/07/20}{Unnecessary `global' removed.}
% \changes{v2.0g}{89/09/19}{`theoremfont to `theorembodyfont.}
% \changes{v2.1c}{92/06/22}{`reset@font added and logic changed.}
% \changes{v2.1d}{92/10/28}{Reset to empty if arg is empty}
%    \begin{macrocode}
\gdef\theorembodyfont#1{%
%    \end{macrocode}
%    We check if the argument supplied is empty and if so put nothing
%    into the |\theorem@bodyfont| token register to allow for
%    |\theorembodyfont{}| as a mean of using the default of the
%    current |\theoremstyle|.
%    \begin{macrocode}
   \def\@tempa{#1}%
   \ifx\@tempa\@empty
    \theorem@bodyfont{}%
   \else
    \theorem@bodyfont{\reset@font#1}%
   \fi
   }
\newtoks\theorem@bodyfont
\global\theorem@bodyfont{}
%    \end{macrocode}
% \end{macro}
% \end{macro}
%
%
% \begin{macro}{\theoremheaderfont}
% \changes{v2.0g}{89/09/19}{Macro added.}
%    The font for the theorem headers is handled differently because
%    this definition applies to all theorem styles.
%    \begin{macrocode}
\gdef\theoremheaderfont#1{\gdef\theorem@headerfont{#1}%
%    \end{macrocode}
%    After using the macro once it is redefined to produce an error
%    message.
%    \begin{macrocode}
       \gdef\theoremheaderfont##1{%
        \typeout{\string\theoremheaderfont\space should be used
                 only once.}}}
%    \end{macrocode}
% \end{macro}
%
% \begin{macro}{\theorem@headerfont}
% \changes{v2.0g}{89/09/19}{Macro added.}
%    To set the |\theorem@headerfont| default we first test if
%    the new fontselection scheme is in force.
% \changes{v2.1a}{90/02/15}{Support for new font selection.}
%    \begin{macrocode}
\ifx\upshape\undefined
%    \end{macrocode}
%    If not we define it to expand into |\bfseries|. We don't use
%    |\let| just in case a following style option redefines this
%    macro.
%    \begin{macrocode}
\gdef\theorem@headerfont{\bfseries}
%    \end{macrocode}
%    Otherwise we reset the current shape before calling |\bfseries|.
%    \begin{macrocode}
\else \gdef\theorem@headerfont{\normalfont\bfseries}\fi
%    \end{macrocode}
% \end{macro}
%
% \begin{macro}{\th@plain}
% \begin{macro}{\th@break}
% \begin{macro}{\th@marginbreak}
% \begin{macro}{\th@changebreak}
% \begin{macro}{\th@change}
% \begin{macro}{\th@margin}
%    The different styles are defined in macros such as
%    |\th@plain|.  Since memory space is precious in
%    ``non-Big-versions'', we have to avoid offering too many unused
%    definitions.  Therefore we define these styles in separate files
%    that can be loaded on demand.  Thus the commands themselves only
%    load these files. We use |\@input@|, a \LaTeXe{} internal command
%    that ensures that the file will be listed with |\listfiles|.
% \changes{v2.2a}{94/02/02}{Use `@input@ to load theorem layout files}
%    \begin{macrocode}
\gdef\th@plain{\@input@{thp.sty}}
\gdef\th@break{\@input@{thb.sty}}
\gdef\th@marginbreak{\@input@{thmb.sty}}
\gdef\th@changebreak{\@input@{thcb.sty}}
\gdef\th@change{\@input@{thc.sty}}
\gdef\th@margin{\@input@{thm.sty}}
%    \end{macrocode}
%    This list will be expanded when new styles become
%    available.  For testing,  just append new theorem substyles
%    as document options.
% \end{macro}
% \end{macro}
% \end{macro}
% \end{macro}
% \end{macro}
% \end{macro}
%
%
%
% \subsection{Definition of a new theorem set}
%
% As already pointed out, a new theorem environment can be defined in
% three different ways:
% \begin{verbatim}
%    \newtheorem{Lem}{Lemma}
%    \newtheorem{Lem}{Lemma}[section]
%    \newtheorem{Lem}[Theorem]{Lemma}
%\end{verbatim}
% The function of the macro |\newtheorem| is to recognize these cases
% and then to branch into one of the three macros |\@ynthm|, |\@xnthm|
% or |\@othm|.  This mechanism is adopted unchanged from
% \cite{latex:source}; the essential point here is that, for example,
% in the second case, the arguments \texttt{Lem}, \texttt{Lemma} and
% \texttt{section} are passed over to the macro |\@xnthm|.
%
% We inspect this case first because the others present fewer problems,
% and thus are easily derived from this one.
%
% \begin{macro}{\@xnthm}
%    For our example arguments, the macro |\@xnthm| must fulfill
%    the following:
%    \begin{itemize}
%       \item
%          Define a new \LaTeX-counter `Lem'
%       \item
%          reset this counter within a |\section|
%       \item
%          define the macro |\theLem|
%       \item
%          define the environment macros |\Lem| and
%          |\endLem| using the current
%          |\theoremstyle| and |\theorem@bodyfont|.
%    \end{itemize}
%    Obviously, all this should happen only if the first argument
%    of |\@xnthm| (i.e.\ \texttt{Lem} in our example) is chosen so
%    as not to conflict with any previously defined commands or
%    environments.  This test is performed by the \LaTeX{} macro
%    |\@ifdefinable|.
%    \begin{macrocode}
\gdef\@xnthm#1#2[#3]{\expandafter\@ifdefinable\csname #1\endcsname
%    \end{macrocode}
%    Therefore, the first argument of |\@ifdefinable| is the
%    expansion (in the example, |\Lem|) of
%    |\csname||#1\endcsname|.  The second argument is
%    executed only if the test has been completed successfully.
%    \begin{macrocode}
   {%
%    \end{macrocode}
%    Now we define the new counter. The names of the \LaTeX{}
%    macros employed should speak for themselves:
% \changes{v2.2b}{1995/10/07}{Use \cs{@newctr}}
% \changes{v2.2c}{1995/11/23}{Use \cs{@newctr} properly.}
%    \begin{macrocode}
    \@definecounter{#1}\@newctr{#1}[#3]%
%    \end{macrocode}
%    Using |\@newctr| will give a proper error message if the counter
%    in |#3| is not defined.
%    In defining `|\theLem|' we must generate the desired macro
%    name by use of |\expandafter| and |\csname|.
%    \begin{macrocode}
    \expandafter\xdef\csname the#1\endcsname
%    \end{macrocode}
%    An |\xdef| is used in order to make the definition global,
%    and to ensure that it contains the replacement texts of
%    |\@thmcountersep| and |\@thmcounter|.\footnote{These
%    two macros can be defined by the document style.  Their default
%    values produce a `\texttt{.}' as separation and an arabic
%    representation of the number.} However, not everything should be
%    expanded. For example, it saves space to use |\thesection|
%    instead of its---at times---lengthy expansion.
%    \begin{macrocode}
      {\expandafter \noexpand \csname the#3\endcsname
       \@thmcountersep \@thmcounter{#1}}%
%    \end{macrocode}
%    Thus with the defaults of \LaTeX{}, |\theLem| would be
%    replaced by the command sequence |\thesection.\arabic{Lem}|.
%
%    We will now look at the definition of the macro which is executed
%    at the beginning of the actual environment (in our example this
%    macro is |\Lem|). It should be noted that we use an
%    ``|\expandafter| trick'' to expand only certain parts of the
%    replacement text at the time of the definition.
%    \begin{macrocode}
    \def\@tempa{\global\@namedef{#1}}%
    \expandafter \@tempa \expandafter{%
%    \end{macrocode}
%    First, the macro that contains the current definitions of
%    |\@begintheorem| and |\@opargtheorem| should be called
%    up.  The name of this macro---as is already known---has the form
%    |\th@|\meta{theorem\ style}; therefore, it must be called by
%    \begin{macrocode}
      \csname th@\the \theorem@style
%    \end{macrocode}
%    In addition the default theorem font should be changeable, i.e.\
%    we have to insert the contents of |\theorem@bodyfont|. For
%    that reason, we expand even further, beyond |\endcsname|,
%    and thus insert the contents of the \textsf{token} register
%    |\theorem@bodyfont| in the replacement text.
%    \begin{macrocode}
            \expandafter \endcsname \the \theorem@bodyfont
%    \end{macrocode}
%    Now it is time to call the macro |\@thm| which takes over
%    the further processing.  It has two arguments: the current
%    counter name (in our example, \texttt{Lem}), and the text of the
%    label (in our example, \texttt{Lemma}).
%    \begin{macrocode}
     \@thm{#1}{#2}}%
%    \end{macrocode}
%    With this, the `sub-definition' is complete.  The macro
%    |\@endtheorem| ends a theorem environment and is, so far, nothing
%    but an |\endtrivlist|.  (Hence it is defined globally, and not
%    within the theorem styles.\footnote{This has to be changed as
%    soon as theorem styles that change \texttt{\bslash @endtheorem}
%    exist.  In such a case, all existing styles must be changed as
%    well since they will have to reset the macro.}) Therefore, we can
%    set it equivalent to the macro that ends the theorem set (in our
%    example, |\endLem|).  However, if some day theorem styles exist
%    that do change |\@endtheorem|, we would have to use the
%    commented-out line instead.
%    \begin{macrocode}
    \global \expandafter \let \csname end#1\endcsname \@endtheorem
%   \global\@namedef{end#1}{\@endtheorem}%
%    \end{macrocode}
%    With these commands all the required definitions are employed,
%    unless the test |\@ifdefinable| has failed.  Therefore, we
%    end the second argument of this macro and with it the definition
%    of |\@xnthm|.
%    \begin{macrocode}
   }}
%    \end{macrocode}
% \end{macro}
%
% \begin{macro}{\@ynthm}
%    The definition of |\@ynthm| is completely analogous. In
%    this case the new counter that is defined is not reset within
%    another counter; thus the definition of |\the...|
%    is simplified:
%    \begin{macrocode}
\gdef\@ynthm#1#2{\expandafter\@ifdefinable\csname #1\endcsname
   {\@definecounter{#1}%
    \expandafter\xdef\csname the#1\endcsname{\@thmcounter{#1}}%
%    \end{macrocode}
%    The rest of the definition corresponds literally to that of
%    |\@xnthm|:
%    \begin{macrocode}
    \def\@tempa{\global\@namedef{#1}}\expandafter \@tempa
     \expandafter{\csname th@\the \theorem@style \expandafter
     \endcsname \the\theorem@bodyfont \@thm{#1}{#2}}%
    \global \expandafter \let \csname end#1\endcsname \@endtheorem}}
%    \end{macrocode}
% \end{macro}
%
% \begin{macro}{\@othm}
%    The definition of |\@othm| does not contain anything new.
%    \begin{macrocode}
\gdef\@othm#1[#2]#3{%
%    \end{macrocode}
% \changes{v2.2b}{95/10/07}{Build in check for counter defined}
%    We do not define a new counter but instead use one that has
%    already been defined.  Thus the only definition we need is that
%    of this pseudo-counter (i.e.\ |\the|\meta{env.\ name}).
%    First we check if |#2| corresponds to a known counter name.
%    \begin{macrocode}
  \expandafter\ifx\csname c@#2\endcsname\relax
   \@nocounterr{#2}%
  \else
   \expandafter\@ifdefinable\csname #1\endcsname
   {\expandafter \xdef \csname the#1\endcsname
     {\expandafter \noexpand \csname the#2\endcsname}%
%    \end{macrocode}
%    All other parts of the definition can be adopted from
%    |\@xnthm|. We have to remember, though, that in this case
%    the name of the current counter and the theorem label have moved
%    to the second and third arguments.
%    \begin{macrocode}
    \def\@tempa{\global\@namedef{#1}}\expandafter \@tempa
     \expandafter{\csname th@\the \theorem@style \expandafter
     \endcsname \the\theorem@bodyfont \@thm{#2}{#3}}%
    \global \expandafter \let \csname end#1\endcsname \@endtheorem}%
  \fi}
%    \end{macrocode}
% \end{macro}
%
% \subsection{Macros that are employed in a {\envfont theorem}
%    environment}
%
% \begin{macro}{\@thm}
%    The macro |\@thm| has to increase the current counter.  Then,
%    depending on whether the environment has (or does not have) an
%    optional argument, it has to branch into either |\@begintheorem|
%    or |\@opargtheorem|.
%    \begin{macrocode}
\gdef\@thm#1#2{\refstepcounter{#1}%
%    \end{macrocode}
%    Now we start a {\envfont trivlist} environment, and give
%    |\@topsep| and |\@topsepadd| the values of the \textsf{skip}
%    registers |\theorempreskipamount| and |\theorempostskipamount|.
%    The value in |\@topsep| is the vertical space that is inserted by
%    the first (and only) |\item| in our |\trivlist| whilst
%    |\@topsepadd| is inserted by |\@endparenv| at the end of that
%    {\envfont trivlist} environment.  By using these registers, we
%    obtain the desired space around a {\envfont theorem} environment.
%    \begin{macrocode}
   \trivlist
   \@topsep \theorempreskipamount               % used by first \item
   \@topsepadd \theorempostskipamount           % used by \@endparenv
%    \end{macrocode}
%    Now we have to test whether an optional argument has been given.
% \changes{v2.0d}{89/07/18}{Forgotten percent added.}
%    \begin{macrocode}
   \@ifnextchar [%
%    \end{macrocode}
%    If there is an optional argument, we will call |\@ythm|, and
%    move the arguments read back into the input stream.
%    \begin{macrocode}
   {\@ythm{#1}{#2}}%
%    \end{macrocode}
%    If not, we call |\@begintheorem|. Its first argument is the
%    name of the theorem set (hence the second argument of
%    |\@thm|). Its second argument is the macro that produces the
%    current number.
%    \begin{macrocode}
   {\@begintheorem{#2}{\csname the#1\endcsname}\ignorespaces}}
%    \end{macrocode}
% \end{macro}
%
% \begin{macro}{\@xthm}
% \begin{macro}{\@ythm}
%    Both these macros were originally called by |\@thm|.  We do
%    not need |\@xthm| anymore, hence we reset it to
%    |\relax|. The definition of |\@ythm| has not changed at
%    all from its definition in \LaTeX. In order to make the macros
%    easier to understand, we will
%    nevertheless present it (commented out).
%    \begin{macrocode}
\global\let\@xthm\relax
% \def\@ythm#1#2[#3]{\@opargbegintheorem{#2}{\csname
%           the#1\endcsname}{#3}\ignorespaces}
%</package>
%    \end{macrocode}
%    The primitive |\ignorespaces| in |\@ythm| and
%    |\@thm| is needed to remove the spaces between the
%    |\begin{...}| and the actual text.
% \end{macro}
% \end{macro}
%
%
% \subsection{Definition of the theorem substyles}
%
% As already pointed out, the theorem substyles, defined below, are
% only loaded when necessary.  Note that all these substyles, except
% \texttt{plain}, have |\slshape| as the default body font.
%
% \subsubsection{The \texttt{plain} style}
%
% \label{dok:th@plain} As the following macros use |@|, we have
% to locally set the |\catcode| of this symbol to ``letter''.
% This happens within a group, so that we do not have to worry about
% which |\catcode| that symbol had before.
%    \begin{macrocode}
%<*thp>
\begingroup \makeatletter
%    \end{macrocode}
% Since we are now within a group, we must make all definitions
% globally.  First we make sure that \texttt{theorem.sty} is loaded.
% This will allow us to use this file as a document style option
% without having to call \texttt{theorem} itself as an option.  At the
% same time, we assure that at least version 2 is loaded, since
% |\theorem@style| was not defined in earlier versions.
%    \begin{macrocode}
\@ifundefined{theorem@style}{\input{theorem.sty}}{}
\ProvidesFile{thp.sty}
             [\FMithmInfo]
%    \end{macrocode}
%
%
% \begin{macro}{\th@plain}
% \begin{macro}{\@begintheorem}
% \begin{macro}{\@opargbegintheorem}
%    |\theoremstyle{plain}| corresponds to the original
%    definition, except that the distances to the surrounding text are
%    determined by the parameters |\theorempreskipamount| and
%    |\theorempostskipamount|.
%    First we set the default body font.
%    \begin{macrocode}
\gdef\th@plain{\normalfont\itshape
%    \end{macrocode}
%    Then we define |\@begintheorem| and
%    |\@opargbegintheorem|. These two macros define how the
%    header of a theorem is typeset.  |\@opargbegintheorem| will
%    be called if a \textsf{theorem} environment with an optional
%    argument is encountered; otherwise, the header is constructed by
%    calling |\@begintheorem|.  If one of these macros is
%    executed, we are within a \textsf{trivlist} environment started by
%    |\@thm|.  So the theorem header is produced with an
%    |\item| command.
%
%    Instead of specifying the header font directly, all standard
%    theorem styles use the |\theorem@headerfont| macro to allow
%    customization.  The extra space (|\labelsep|) is necessary
%    because of problems in the \textsf{trivlist} environment.
%    \begin{macrocode}
  \def\@begintheorem##1##2{%
        \item[\hskip\labelsep \theorem@headerfont ##1\ ##2]}%
%    \end{macrocode}
%    The definition of |\@opargbegintheorem| is completely
%    analogous. The only difference is the fact that there exists a
%    third argument (which is the optional parameter of the
%    environment and contains additional information about the
%    theorem).  Customarily we enclose it in parentheses.
%    \begin{macrocode}
\def\@opargbegintheorem##1##2##3{%
   \item[\hskip\labelsep \theorem@headerfont ##1\ ##2\ (##3)]}}
%    \end{macrocode}
% \end{macro}
% \end{macro}
% \end{macro}
%
% \noindent
% We conclude with an |\endgroup| to restore the
% |\catcode| of |@|.
%    \begin{macrocode}
\endgroup
%</thp>
%    \end{macrocode}
%
%
% \subsubsection{The \texttt{break} style}
%
% This style option is stored in the file |thb.sty|.
% For the next two lines see the documentation for |\th@plain| on
% page~\pageref{dok:th@plain}.
%    \begin{macrocode}
%<*thb>
\begingroup \makeatletter
\@ifundefined{theorem@style}{\input{theorem.sty}}{}
\ProvidesFile{thb.sty}
             [\FMithmInfo]
%    \end{macrocode}
%
% \begin{macro}{\th@break}
%    |\theoremstyle{break}| produces a line break after the name
%    of the theorem.  The font is |\slshape|.  Hence, we define
%    |\th@break| as follows:
%    \begin{macrocode}
\gdef\th@break{\normalfont\slshape
  \def\@begintheorem##1##2{\item[%
%    \end{macrocode}
%    We run into the following problem: it is not possible to create
%    the header with |\item[|\meta{title}|]| and then start
%    a new line by, for example, |\mbox{}\\|. Such a definition
%    will fail whenever a list environment follows immediately.  With
%    the above construction, the |\mbox{}| causes the switch
%    |@inlabel| (cf.\ definition of |\list| and
%    |\trivlist| in \cite{latex:source}) to be set to \texttt{false}
%    and so the following list will insert additional vertical
%    space (|\topskip|). This is quite annoying. Therefore, we
%    create the line break within the |\item|. In order to ensure
%    that the text will begin at the proper position in the following
%    line, we simply pretend that the label does not take any
%    room.\footnote{This will lead to problems whenever very high
%    symbols occurring in the line tower into the
%    heading. So, something else has to be done here sometime.}
%    \begin{macrocode}
       \rlap{\vbox{\hbox{\hskip \labelsep\theorem@headerfont ##1\ ##2}%
                   \hbox{\strut}}}]}%
%    \end{macrocode}
%    Again, the definition of |\@opargbegintheorem| is
%    completely analogous.
%    \begin{macrocode}
\def\@opargbegintheorem##1##2##3{%
  \item[\rlap{\vbox{\hbox{\hskip \labelsep \theorem@headerfont
                     ##1\ ##2\ (##3)}%
                    \hbox{\strut}}}]}}
\endgroup
%</thb>
%    \end{macrocode}
% \end{macro}
%
%
% \subsubsection{The \texttt{changebreak} style}
%
%    \begin{macrocode}
% This style option is stored in the file |thcb.sty|.
%    \begin{macrocode}
%<*thcb>
\begingroup \makeatletter
\@ifundefined{theorem@style}{\input{theorem.sty}}{}
\ProvidesFile{thcb.sty}
             [\FMithmInfo]
%    \end{macrocode}
%
% \begin{macro}{\th@changebreak}
%    The \texttt{change-break} theorem style is like \texttt{break}
%    but with interchange of theorem name and theorem number.  Thus we
%    define |\th@changebreak| as follows:
%    \begin{macrocode}
\gdef\th@changebreak{\normalfont\slshape
  \def\@begintheorem##1##2{\item
       [\rlap{\vbox{\hbox{\hskip\labelsep \theorem@headerfont ##2\ ##1}%
                    \hbox{\strut}}}]}%
\def\@opargbegintheorem##1##2##3{%
  \item[\rlap{\vbox{\hbox{\hskip\labelsep \theorem@headerfont
                     ##2\ ##1\ (##3)}%
                    \hbox{\strut}}}]}}
\endgroup
%</thcb>
%    \end{macrocode}
% \end{macro}
%
%
% \subsubsection{The \texttt{change} style}
%
% This style option is stored in the file |thc.sty|.
%    \begin{macrocode}
%<*thc>
\begingroup \makeatletter
\@ifundefined{theorem@style}{\input{theorem.sty}}{}
\ProvidesFile{thc.sty}
             [\FMithmInfo]
%    \end{macrocode}
%
% \begin{macro}{\th@change}
%    The \texttt{change} theorem style corresponds to the
%    \texttt{change break} style without a linebreak after the header.
%    To say it in another way, it's the same as the \texttt{plain}
%    style but with number and name interchanged and |\slshape| as the
%    default font.
%    \begin{macrocode}
\gdef\th@change{\normalfont\slshape
  \def\@begintheorem##1##2{\item
       [\hskip\labelsep \theorem@headerfont ##2\ ##1]}%
\def\@opargbegintheorem##1##2##3{%
  \item[\hskip\labelsep \theorem@headerfont ##2\ ##1\ (##3)]}}
\endgroup
%</thc>
%    \end{macrocode}
% \end{macro}
%
%
% \subsubsection{The \texttt{marginbreak} style}
%
% This style option is the one used most often at Mainz.  It is saved
% in the file |thmb.sty|.
%    \begin{macrocode}
%<*thmb>
\begingroup \makeatletter
\@ifundefined{theorem@style}{\input{theorem.sty}}{}
\ProvidesFile{thmb.sty}
             [\FMithmInfo]
%    \end{macrocode}
%
% \begin{macro}{\th@marginbreak}
%    The \texttt{margin break} style is nearly the same as the
%    \texttt{change break} style. The only difference is the placement
%    of the theorem number.  We use |\llap| to place it in the left
%    margin.
%
%    In this style |\labelsep| denotes the separation between the
%    number and the text.
%    \begin{macrocode}
\gdef\th@marginbreak{\normalfont\slshape
  \def\@begintheorem##1##2{\item
       [\rlap{\vbox{\theorem@headerfont
                    \hbox{\llap{##2}\hskip\labelsep ##1}%
                    \hbox{\strut}}}]}%
\def\@opargbegintheorem##1##2##3{%
  \item[\rlap{\vbox{\theorem@headerfont
                    \hbox{\llap{##2}\hskip\labelsep ##1\ (##3)}%
                    \hbox{\strut}}}]}}
\endgroup
%</thmb>
%    \end{macrocode}
% \changes{v2.1c}{90/02/17}{`endgroup outside macrocode env.}
% \end{macro}
%
%
%
% \subsubsection{The \texttt{margin} style}
%
% This style option is stored in the file |thm.sty|.
%    \begin{macrocode}
%<*thm>
\begingroup \makeatletter
\@ifundefined{theorem@style}{\input{theorem.sty}}{}
\ProvidesFile{thm.sty}
             [\FMithmInfo]
%    \end{macrocode}
%
% \begin{macro}{\th@margin}
%    Again this is only a variant of the theorem styles described
%    above without any new ideas.
%    \begin{macrocode}
\gdef\th@margin{\normalfont\slshape
  \def\@begintheorem##1##2{\item
       [\theorem@headerfont \llap{##2}\hskip\labelsep ##1]}%
\def\@opargbegintheorem##1##2##3{%
  \item[\theorem@headerfont \llap{##2}\hskip\labelsep ##1\ (##3)]}}
\endgroup
%</thm>
%    \end{macrocode}
% \end{macro}
%
%
%
%
% \subsection{Final Definitions}
%
% \begin{macro}{\theorempreskipamount}
% \begin{macro}{\theorempostskipamount}
%    The \textsf{skip} parameters that regulate the vertical empty space
%    before and after the {\envfont theorem} environment have
%    to be allocated as well.
% \changes{v2.0f}{89/07/20}{Unnecessary `global' removed.}
%    \begin{macrocode}
%<*package>
\newskip\theorempreskipamount
\newskip\theorempostskipamount
%    \end{macrocode}
%    Since we have used the same values for all theorem sets, we now can
%    assign them.
%    \begin{macrocode}
\global\setlength\theorempreskipamount{12pt plus 5pt minus 3pt}
\global\setlength\theorempostskipamount{8pt plus 3pt minus 1.5pt}
%    \end{macrocode}
% \end{macro}
% \end{macro}
%
%
% \begin{macro}{\@endtheorem}
%    The same holds for the macro |\@endtheorem|, which ends a
%    {\envfont theorem} environment. Since it is the same for all
%    theorem sets, it is removed from the macros |\th@|\meta{style}.
%    It simply ends the {\envfont trivlist} environment, which was
%    begun in |\@thm|.
%    \begin{macrocode}
\global\let\@endtheorem=\endtrivlist
%    \end{macrocode}
% \end{macro}
%
%
%
% \begin{macro}{\@preamblecmds}
%    All macros defined above are to be used only in the preamble.
%    Therefore, we insert them in |\@preamblecmds| which will disable
%    them at begin document. This is done by the internal \LaTeXe{}
%    command |\@onlypreamble|.
% \changes{v2.0g}{89/09/19}{`theoremheaderfont and `theorembodyfont
%    added.}
% \changes{v2.2a}{94/02/03}{Use \LaTeXe{} interface}
%    \begin{macrocode}
\@onlypreamble\@xnthm
\@onlypreamble\@ynthm
\@onlypreamble\@othm
\@onlypreamble\newtheorem
\@onlypreamble\theoremstyle
\@onlypreamble\theorembodyfont
\@onlypreamble\theoremheaderfont
%    \end{macrocode}
% \end{macro}
%
% Finally we declare the \texttt{plain} theorem style to be the default.
% \changes{v2.1b}{90/06/04}{Fixed missing `item by declaring default
%    style.}
%    \begin{macrocode}
\theoremstyle{plain}
%</package>
%    \end{macrocode}
%
% \Finale

\endinput
}{}
\ProvidesFile{thcb.sty}
             [\FMithmInfo]
%    \end{macrocode}
%
% \begin{macro}{\th@changebreak}
%    The \texttt{change-break} theorem style is like \texttt{break}
%    but with interchange of theorem name and theorem number.  Thus we
%    define |\th@changebreak| as follows:
%    \begin{macrocode}
\gdef\th@changebreak{\normalfont\slshape
  \def\@begintheorem##1##2{\item
       [\rlap{\vbox{\hbox{\hskip\labelsep \theorem@headerfont ##2\ ##1}%
                    \hbox{\strut}}}]}%
\def\@opargbegintheorem##1##2##3{%
  \item[\rlap{\vbox{\hbox{\hskip\labelsep \theorem@headerfont
                     ##2\ ##1\ (##3)}%
                    \hbox{\strut}}}]}}
\endgroup
%</thcb>
%    \end{macrocode}
% \end{macro}
%
%
% \subsubsection{The \texttt{change} style}
%
% This style option is stored in the file |thc.sty|.
%    \begin{macrocode}
%<*thc>
\begingroup \makeatletter
\@ifundefined{theorem@style}{% \iffalse meta-comment
%
% Copyright (C) 1993-2024
%
% The LaTeX Project and any individual authors listed elsewhere
% in this file.
%
% This file is part of the Standard LaTeX `Tools Bundle'.
% -------------------------------------------------------
%
% It may be distributed and/or modified under the
% conditions of the LaTeX Project Public License, either version 1.3c
% of this license or (at your option) any later version.
% The latest version of this license is in
%    https://www.latex-project.org/lppl.txt
% and version 1.3c or later is part of all distributions of LaTeX
% version 2005/12/01 or later.
%
% The list of all files belonging to the LaTeX `Tools Bundle' is
% given in the file `manifest.txt'.
%
% \fi
%
%%
%
%
\def\FMithmInfo{2023/07/05 v2.2c Theorem extension package (FMi)}
%
% \ProvidesFile{theorem.dtx}[\FMithmInfo]
%\iffalse   % this is a METACOMMENT !
%%
%% Package `theorem' to use with LaTeX2e
%% Copyright (C) 1989-1995 Frank Mittelbach, all rights reserved.
%<+package>\@ifundefined{theorem@style}{}{\endinput}
%<+package>\NeedsTeXFormat{LaTeX2e}[1995/06/01]
%<+package>\ProvidesPackage{theorem}
%<+package>         [\FMithmInfo]
%
% \fi
%
% \changes{v1.0}{87/05/07}{First draft.}
% \changes{v1.0a}{87/05/07}{`thistheoremstyle' added.}
% \changes{v1.1a}{87/07/29}{New concept: redefine the internal
%                          \protect\LaTeX{} commands `@xnthm' etc.}
% \changes{v1.1b}{87/07/31}{`theoremfont' used to specify font.}
% \changes{v1.2a}{88/01/08}{`@endtheorem' now globally defined.}
% \changes{v1.2b}{88/01/11}{`newtheorem' only allowed in the preamble.}
% \changes{v1.3a}{88/02/14}{`theoremfont' now a toks register.}
%
% \changes{v2.0a}{88/11/16}{New implementation and new documentation.}
% \changes{v2.0a}{88/11/16}{No layout changes but internal modifications
%                           old substyles won't work.}
% \changes{v2.0b}{89/01/30}{`global' used for all definitions.}
% \changes{v2.0c}{89/06/15}{Documentation translated to English.
%                            Thanks to LLL and Nina}
% \changes{v2.1a}{90/02/15}{Support for new font selection.}
% \changes{v2.2a}{94/02/03}{Cleanup for LaTeX2e}
%
% \DoNotIndex{\@endparenv,\@ifdefinable,\@ifnextchar,\@ifundefined}
% \DoNotIndex{\@namedef,\@nameuse,\@spaces,\@tempa,\@warning}
% \DoNotIndex{\ }
% \DoNotIndex{\begingroup,\bfseries}
% \DoNotIndex{\csname}
% \DoNotIndex{\def,\do,\docdate}
% \DoNotIndex{\endcsname,\endgroup,\endinput,\endtrivlist,\expandafter}
% \DoNotIndex{\filedate,\fileversion}
% \DoNotIndex{\gdef,\global}
% \DoNotIndex{\hbox,\hskip}
% \DoNotIndex{\ignorespaces,\input,\itshape,\item}
% \DoNotIndex{\labelsep,\let,\llap}
% \DoNotIndex{\makeatletter}
% \DoNotIndex{\noexpand}
% \DoNotIndex{\refstepcounter,\relax,\rlap}
% \DoNotIndex{\setlength,\slshape,\space,\string,\strut}
% \DoNotIndex{\the,\trivlist,\typeout}
% \DoNotIndex{\vbox}
% \DoNotIndex{\xdef}
%
% \def\deflabel#1{\ttfamily #1\hfill}
% \def\deflist#1{\begin{list}{}{\settowidth\labelwidth{\ttfamily #1}%
%                               \setlength\leftmargin\labelwidth
%                               \addtolength\leftmargin\labelsep
%                               \let\makelabel\deflabel}}
% \def\enddeflist{\end{list}}
% \def\envfont{\normalfont\ttfamily}
%
% \MakeShortVerb{\|}
%
% \newcommand{\TUB}{{\sl TUGboat\/}}
%
% \GetFileInfo{theorem.dtx}
% \title{An Extension of the
%        \LaTeX{} theorem environment\thanks{This file
%        has version number \fileversion, last
%        revised \filedate.}}
% \author{Frank Mittelbach}
%
% \MaintainedByLaTeXTeam{tools}
% \maketitle
%
% \begin{abstract}
%    The macros described in this paper yield an extension of the
%    \LaTeX{} theorem mechanism. It is designed to satisfy the
%    different requirements of various journals. Thus, the
%    layout of the ``theorems'' can be manipulated by determining a
%    ``style''. This article describes not only the use, but
%    also the definition, of the necessary macros.
% \end{abstract}
%
% \section*{Preface to version 2.2}
%
% For \LaTeXe{} this package did not need any fundamental changes. I
% only modified the messages generated so that theorem layout styles
% will show up with the |\listfiles| command and cleaned the section
% on the New Font Selection Scheme since this is now included in
% \LaTeX.
%
% \section*{Preface to version 2.1}
%
% This version is identical to 2.0g described in \TUB\ 10\#3 except
% for some internal defaults which are now set depending on the used
% font selection scheme.
%
% This was done to avoid unpleasant surprises if the new font selection
% scheme is in force. For further details see
% section~\ref{sec:newfonts} and \cite{bk:GMS94}.
%
% \section{Introduction}
%
% For our purposes here, ``theorems'' are labelled enunciations,
% often set off from the main text by extra space and a font change.
% Theorems, corollaries, conjectures, definitions, and remarks
% are all instances of ``theorems''.  The ``header'' of these structures
% is composed of a label (such as \textsc{Theorem} or \textsc{Remark})
% and a number which serializes an item in the sequence of items
% with the same label.
%
% Shortly after the introduction of \LaTeX{} at the Fachbereich
% Mathematik in Mainz, the desire to manipulate the layout of
% ``theorems'' arose.  In Mainz, the following two conventions came
% into general use:
% \begin{enumerate}
%   \item The number of the theorem is shown in the margin.
%   \item There is a line break at
%         the end of the theorem header.
% \end{enumerate}
%
% Additionally, some journals require different formats which depend on
% the ``sort of theorem'': e.g.\ often remarks and definitions are set
% in |\upshape|, while |\itshape| is employed for main theorems.
%
% Confronted with these requirements, a theorem environment was
% developed in Mainz which allows separate determination of the
% layout of the ``theorems sets'', comparable to |\pagestyle|.
%
%
% \section{The user interface}
% \subsection{Defining new theorem sets}
%
% \DescribeMacro\newtheorem
% As in the original \LaTeX{} version, the command |\newtheorem|
% defines a new ``theorem set'' or ``theorem-like structure''.
% Two required arguments name the new environment and give the text
% to be typeset with each instance of the new ``set'', while
% an optional argument determines how
% the ``set'' is enumerated:
% \begin{description}
%    \item[\ttfamily \bslash newtheorem\{foo\}\{bar\}]
%       The theorem set {\envfont foo} (whose name is \texttt{bar})
%       uses its own counter.
%    \item[\ttfamily \bslash newtheorem\{foo2\}{[foo]}\{bar2\}]
%       The theorem set {\envfont foo2} (printed name \texttt{bar2})
%       uses the same counter as the theorem set \texttt{foo}.
%    \item[\ttfamily \bslash newtheorem\{foo3\}\{bar3\}{[section]}]
%       The theorem set {\envfont foo3} (printed name \texttt{bar3}) is
%       enumerated within the counter \texttt{section}, i.e.\ with every
%       new |\section| the enumeration begins again with 1, and
%       the enumeration is composed from the section-number and the
%       theorem counter itself.
% \end{description}
%
% \DescribeMacro\theoremstyle
% Additionally, the command |\theoremstyle| can define the layout
% of various, or all, theorem sets.  It should be noted that any
% theorem set defined by |\newtheorem| is typeset in the
% |\theoremstyle| that is current at the time of the definition.
% Thus, the following
% \begin{verbatim}
%    \theoremstyle{break}        \newtheorem{Cor}{Corollary}
%    \theoremstyle{plain}        \newtheorem{Exa}{Example}[section]
%\end{verbatim}                             ^^A No blank at line start
% leads to the result that the set {\envfont Cor} is formatted in the
% style \texttt{break}, while the set {\envfont Exa} and all the
% following ones are formatted in the style \texttt{plain}, unless
% another |\theoremstyle| follows.  Since the definitions installed by
% |\newtheorem| are global, one also can limit |\theoremstyle| locally
% by grouping braces.
%
% \DescribeMacro\theorembodyfont
% The choice of the font for the theorem body is completely
% independent of the chosen |\theoremstyle|; this has proven to
% be very advantageous.  For example,
% \begin{verbatim}
%   {\theorembodyfont{\upshape}          \newtheorem{Rem}{Remark}}
%\end{verbatim}
% defines a theorem set {\envfont Rem}, which will be set in
% |\upshape| in the current layout (which in our example is
% \texttt{plain}).  As with |\theoremstyle|, the |\theorembodyfont|
% chosen is that current at the time of |\newtheorem|.  If
% |\theorembodyfont| is not specified or one defines
% |\theorembodyfont{}|, then the font used will be that defined by the
% |\theoremstyle|.
%
% \DescribeMacro\theoremheaderfont It is also possible to customize
% the font used for the theorem headers.  This is, however, a global
% declaration, and therefore there should be at most one
% |\theoremheaderfont| declaration in the preamble.\footnote{If it is
% actually necessary to have different header fonts, one has to define
% new theorem styles (substituting the desired font) or specify the
% information directly in the \texttt{\bslash newtheorem} declaration
% (the unclean variant).}
%
% \DescribeMacro\theorempreskipamount
% \DescribeMacro\theorempostskipamount Two additional parameters
% affect the vertical space around the theorem environments:
% |\theorempreskipamount| and |\theorempostskipamount| define,
% respectively, the spacing before and after such an environment.
% These parameters apply for all theorem sets and can be manipulated
% with the ordinary length macros.  They are rubber lengths
% (`\textsf{skips}'), and therefore can contain \texttt{plus} and
% \texttt{minus} parts.
%
% Since the definition of theorem sets should---most sensibly---be
% placed in the preamble, we only allow installation there.
% It is therefore possible to release the memory used here after
% |\begin{document}|, in order to make room for other
% applications.
%
%
% \subsection{Existing theorem styles}
%
% The following theorem styles exist to date:
% \begin{deflist}{marginbreak}
%    \item[plain]
%       This theorem style emulates the original \LaTeX{} definition,
%       except that additionally the parameters
%       |\theorem...skipamount| are used.
%    \item[break]
%       In this style, the theorem header is followed by a line
%       break.
%    \item[marginbreak]
%       The theorem number is set in the margin, and there is a line
%       break as in \texttt{break}.
%    \item[changebreak]
%       Like \texttt{break}, but with header number and text
%       interchanged.
%    \item[change]
%       Header number and text are interchanged, without a line break.
%    \item[margin]
%       The number is set in the left margin, without a line break.
% \end{deflist}
% All styles (except \texttt{plain}) select |\slshape| as the default
% |\theorembodyfont|.
%
%
% \subsection{Examples}
%
% Given the above theorem sets {\envfont Cor}, {\envfont Exa}
% and {\envfont Rem},
% suppose that the preamble also contains the declarations:
% \begin{verbatim}
%    \theoremstyle{marginbreak}   \newtheorem{Lem}[Cor]{Lemma}
%    \theoremstyle{change}
%    \theorembodyfont{\itshape}        \newtheorem{Def}[Cor]{Definition}
%
%    \theoremheaderfont{\scshape}
%\end{verbatim}
% Then the following are some typical examples of the typeset output
% resulting from their use.
%
% \begin{Cor}
%    This is a sentence typeset in the theorem environment {\envfont
%    Cor}.
% \end{Cor}
%
% \begin{Exa}
%    This is a sentence typeset in the theorem environment {\envfont
%    Exa}.
% \end{Exa}
%
% \begin{Rem}
%    This is a sentence typeset in the theorem environment {\envfont
%    Rem}.
% \end{Rem}
%
% \begin{Lem}[Ben User]
%    This is a sentence typeset in the theorem environment {\envfont
%    Lem}.
% \end{Lem}
%
% \begin{Def}[Very impressive Definition]
%    This is a sentence typeset in the theorem environment {\envfont
%    Def}.
% \end{Def}
%
% The last two examples show the effect of the optional argument to
% a theorem environment (it is the text typeset in parentheses).
%
%
% \section{Special Considerations}
% \label{sec:newfonts}
%
% Theoremheader and body are implemented as a unit. This means that
% the |\theoremheaderfont|  will inherit characteristics of the
% |\theorembodyfont| in \LaTeXe. Thus, if for example
% |\theorembodyfont| is |\itshape| and |\theoremheaderfont|
% is |\bfseries| the font selected for the header will have the
% characteristics `bold extended italic'. If this is not desired one
% should set the |\theoremheaderfont| to something like
% \begin{verbatim}
%      \theoremheaderfont{\normalfont\bfseries}
%\end{verbatim}
% i.e.\ supplying all necessary font information explicitly.
%
%
%
% \section{Acknowledgements}
%
% The publication of this set of macros was only possible with the
% help of Christina Busse (translating the manuscript into English),
% Joachim Pense (playing the r\^ole of typist), Chris Rowley
% (looking everything over) and many others providing useful
% suggestions.
%
% \MaybeStop{%
%    \begin{thebibliography}{9}
%       \bibitem{bk:GMS94} \textsc{M.~Goossens}, \textsc{F.~Mittelbach}
%         and \textsc{A.~Samarin}.
%         \newblock The \LaTeX{} Companion.
%         \newblock
%         Addison-Wesley, Reading, Massachusetts, 1994.
%       \bibitem{latex:source} \textsc{Lamport, Leslie.} \newblock
%                               \texttt{latex.tex}, version 2.09, date
%                               Feb. 1990.
%    \end{thebibliography}
%    \PrintIndex
%    \PrintChanges}
%
%
% \section{The documentation driver file}
%
% The next bit of code contains the documentation driver file for
% \TeX{}, i.e., the file that will produce the documentation you are
% currently reading. It will be extracted from this file by the
% \texttt{docstrip} program. Since it is the first code in the file
% one can alternatively process this file directly with \LaTeXe{} to
% obtain the documentation.
%
%    \begin{macrocode}
%<*driver>
\documentclass{ltxdoc}

\usepackage{theorem}

%<+driver>% The next few lines define theorem sets which are used
%<+driver>% in the example section of the documentation.

\theoremstyle{break}       \newtheorem{Cor}{Corollary}
\theoremstyle{plain}       \newtheorem{Exa}{Example}[section]
{\theorembodyfont{\upshape}\newtheorem{Rem}{Remark}}
\theoremstyle{marginbreak} \newtheorem{Lem}[Cor]{Lemma}
\theoremstyle{change}
\theorembodyfont{\itshape} \newtheorem{Def}[Cor]{Definition}

\theoremheaderfont{\scshape}

\RecordChanges

\begin{document}
   \DocInput{theorem.dtx}
\end{document}
%</driver>
%    \end{macrocode}
%
% \section{Definition of the Macros}
%
% If the file has been loaded before, we abort immediately. If not the
% package announces itself (this is actually done at the very top of
% the file---the way it is done isn't good style so don't copy it).
% \changes{v2.0e}{89/07/19}{Spaces removed from `typeout'.}
% \changes{v2.2b}{95/11/19}{Announce moved to top of file}
%    \begin{macrocode}
%<*package>
%\@ifundefined{theorem@style}{}{\endinput}
%\def\FMithmInfo{1995/11/19 v2.2b Theorem extension package (FMi)}
%\ProvidesPackage{theorem}[\FMithmInfo]
%    \end{macrocode}
%
%
% \subsection{Definition of theorem styles and fonts}
%
% All the definitions in this file are done globally to allow
% inputting this file inside a group.
%
% \begin{macro}{\theoremstyle}
%    Before a theorem style can be installed, the chosen style must be
%    known. For that reason, we must test to see that
%    |\th@|\meta{style} is known or, more precisely, that it is
%    different from |\relax|. If the style is not known then
%    |\th@plain| is used.
%    \begin{macrocode}
\gdef\theoremstyle#1{%
   \@ifundefined{th@#1}{\@warning
          {Unknown theoremstyle `#1'. Using `plain'}%
          \theorem@style{plain}}%
%    \end{macrocode}
%    We save the theorem style to be used in the token register
%    |\theorem@style|.
%    \begin{macrocode}
      {\theorem@style{#1}}%
%    \end{macrocode}
%    Now we ``evaluate'' the theorem style: this means, we call the
%    macro |\th@|\meta{style} which will activate the relevant
%    definitions which are contained in a separate file.  This is done
%    in a group to suppress changes to the current font.  This could
%    otherwise pose problems together with the new font selection
%    scheme\footnote{When I printed the original article using the new
%    font selection scheme I ended with a document with slanted
%    typefaces (text headings and all) simply because one of the
%    theorem styles used \texttt{\bslash sl} at toplevel.} if the
%    |\th@|\meta{style} is evaluated a second time.
% \changes{v2.1a}{90/02/15}{Group added around `th@..}
% \changes{v2.1a}{90/02/15}{Reseting of `..@begintheorem removed.}
%    \begin{macrocode}
      \begingroup
        \csname th@\the\theorem@style \endcsname
      \endgroup}
%    \end{macrocode}
% \end{macro}
%
%
% \begin{macro}{\@begintheorem}
% \changes{v2.1a}{90/02/15}{Relaxed at toplevel.}
% \begin{macro}{\@opargbegintheorem}
% \changes{v2.1a}{90/02/15}{Relaxed at toplevel.}
%    We reset |\@begintheorem| and |\@opargbegintheorem| to
%    |\relax| since these commands are no longer necessary at
%    toplevel. This will save a few tokens.
%    \begin{macrocode}
\global\let\@begintheorem\relax
\global\let\@opargbegintheorem\relax
%    \end{macrocode}
% \end{macro}
% \end{macro}
%
% \begin{macro}{\theorem@style}
%    Obviously the \textsf{token} register used above has to be
%    allocated. To assure the utmost compatibility with the
%    original \LaTeX{} definition, we set the default theorem style to
%    \texttt{plain}, which implements the usual \LaTeX{} convention.
% \changes{v2.0f}{89/07/20}{Unnecessary `global' removed.}
%    \begin{macrocode}
\newtoks\theorem@style
\global\theorem@style{plain}
%    \end{macrocode}
% \end{macro}
%
% \begin{macro}{\theorembodyfont}
% \begin{macro}{\theorem@bodyfont}
%    If the |\theorembodyfont| is set by the user then it should not
%    interact with the default font set in the theorem style. When the
%    new font selection is in force this may happen if, for example,
%    the default is |\itshape| and the new |\theorembodyfont| is
%    |\sffamily|. So we add a |\reset@font| command in front of the
%    user definition.
% \changes{v2.0f}{89/07/20}{Unnecessary `global' removed.}
% \changes{v2.0g}{89/09/19}{`theoremfont to `theorembodyfont.}
% \changes{v2.1c}{92/06/22}{`reset@font added and logic changed.}
% \changes{v2.1d}{92/10/28}{Reset to empty if arg is empty}
%    \begin{macrocode}
\gdef\theorembodyfont#1{%
%    \end{macrocode}
%    We check if the argument supplied is empty and if so put nothing
%    into the |\theorem@bodyfont| token register to allow for
%    |\theorembodyfont{}| as a mean of using the default of the
%    current |\theoremstyle|.
%    \begin{macrocode}
   \def\@tempa{#1}%
   \ifx\@tempa\@empty
    \theorem@bodyfont{}%
   \else
    \theorem@bodyfont{\reset@font#1}%
   \fi
   }
\newtoks\theorem@bodyfont
\global\theorem@bodyfont{}
%    \end{macrocode}
% \end{macro}
% \end{macro}
%
%
% \begin{macro}{\theoremheaderfont}
% \changes{v2.0g}{89/09/19}{Macro added.}
%    The font for the theorem headers is handled differently because
%    this definition applies to all theorem styles.
%    \begin{macrocode}
\gdef\theoremheaderfont#1{\gdef\theorem@headerfont{#1}%
%    \end{macrocode}
%    After using the macro once it is redefined to produce an error
%    message.
%    \begin{macrocode}
       \gdef\theoremheaderfont##1{%
        \typeout{\string\theoremheaderfont\space should be used
                 only once.}}}
%    \end{macrocode}
% \end{macro}
%
% \begin{macro}{\theorem@headerfont}
% \changes{v2.0g}{89/09/19}{Macro added.}
%    To set the |\theorem@headerfont| default we first test if
%    the new fontselection scheme is in force.
% \changes{v2.1a}{90/02/15}{Support for new font selection.}
%    \begin{macrocode}
\ifx\upshape\undefined
%    \end{macrocode}
%    If not we define it to expand into |\bfseries|. We don't use
%    |\let| just in case a following style option redefines this
%    macro.
%    \begin{macrocode}
\gdef\theorem@headerfont{\bfseries}
%    \end{macrocode}
%    Otherwise we reset the current shape before calling |\bfseries|.
%    \begin{macrocode}
\else \gdef\theorem@headerfont{\normalfont\bfseries}\fi
%    \end{macrocode}
% \end{macro}
%
% \begin{macro}{\th@plain}
% \begin{macro}{\th@break}
% \begin{macro}{\th@marginbreak}
% \begin{macro}{\th@changebreak}
% \begin{macro}{\th@change}
% \begin{macro}{\th@margin}
%    The different styles are defined in macros such as
%    |\th@plain|.  Since memory space is precious in
%    ``non-Big-versions'', we have to avoid offering too many unused
%    definitions.  Therefore we define these styles in separate files
%    that can be loaded on demand.  Thus the commands themselves only
%    load these files. We use |\@input@|, a \LaTeXe{} internal command
%    that ensures that the file will be listed with |\listfiles|.
% \changes{v2.2a}{94/02/02}{Use `@input@ to load theorem layout files}
%    \begin{macrocode}
\gdef\th@plain{\@input@{thp.sty}}
\gdef\th@break{\@input@{thb.sty}}
\gdef\th@marginbreak{\@input@{thmb.sty}}
\gdef\th@changebreak{\@input@{thcb.sty}}
\gdef\th@change{\@input@{thc.sty}}
\gdef\th@margin{\@input@{thm.sty}}
%    \end{macrocode}
%    This list will be expanded when new styles become
%    available.  For testing,  just append new theorem substyles
%    as document options.
% \end{macro}
% \end{macro}
% \end{macro}
% \end{macro}
% \end{macro}
% \end{macro}
%
%
%
% \subsection{Definition of a new theorem set}
%
% As already pointed out, a new theorem environment can be defined in
% three different ways:
% \begin{verbatim}
%    \newtheorem{Lem}{Lemma}
%    \newtheorem{Lem}{Lemma}[section]
%    \newtheorem{Lem}[Theorem]{Lemma}
%\end{verbatim}
% The function of the macro |\newtheorem| is to recognize these cases
% and then to branch into one of the three macros |\@ynthm|, |\@xnthm|
% or |\@othm|.  This mechanism is adopted unchanged from
% \cite{latex:source}; the essential point here is that, for example,
% in the second case, the arguments \texttt{Lem}, \texttt{Lemma} and
% \texttt{section} are passed over to the macro |\@xnthm|.
%
% We inspect this case first because the others present fewer problems,
% and thus are easily derived from this one.
%
% \begin{macro}{\@xnthm}
%    For our example arguments, the macro |\@xnthm| must fulfill
%    the following:
%    \begin{itemize}
%       \item
%          Define a new \LaTeX-counter `Lem'
%       \item
%          reset this counter within a |\section|
%       \item
%          define the macro |\theLem|
%       \item
%          define the environment macros |\Lem| and
%          |\endLem| using the current
%          |\theoremstyle| and |\theorem@bodyfont|.
%    \end{itemize}
%    Obviously, all this should happen only if the first argument
%    of |\@xnthm| (i.e.\ \texttt{Lem} in our example) is chosen so
%    as not to conflict with any previously defined commands or
%    environments.  This test is performed by the \LaTeX{} macro
%    |\@ifdefinable|.
%    \begin{macrocode}
\gdef\@xnthm#1#2[#3]{\expandafter\@ifdefinable\csname #1\endcsname
%    \end{macrocode}
%    Therefore, the first argument of |\@ifdefinable| is the
%    expansion (in the example, |\Lem|) of
%    |\csname||#1\endcsname|.  The second argument is
%    executed only if the test has been completed successfully.
%    \begin{macrocode}
   {%
%    \end{macrocode}
%    Now we define the new counter. The names of the \LaTeX{}
%    macros employed should speak for themselves:
% \changes{v2.2b}{1995/10/07}{Use \cs{@newctr}}
% \changes{v2.2c}{1995/11/23}{Use \cs{@newctr} properly.}
%    \begin{macrocode}
    \@definecounter{#1}\@newctr{#1}[#3]%
%    \end{macrocode}
%    Using |\@newctr| will give a proper error message if the counter
%    in |#3| is not defined.
%    In defining `|\theLem|' we must generate the desired macro
%    name by use of |\expandafter| and |\csname|.
%    \begin{macrocode}
    \expandafter\xdef\csname the#1\endcsname
%    \end{macrocode}
%    An |\xdef| is used in order to make the definition global,
%    and to ensure that it contains the replacement texts of
%    |\@thmcountersep| and |\@thmcounter|.\footnote{These
%    two macros can be defined by the document style.  Their default
%    values produce a `\texttt{.}' as separation and an arabic
%    representation of the number.} However, not everything should be
%    expanded. For example, it saves space to use |\thesection|
%    instead of its---at times---lengthy expansion.
%    \begin{macrocode}
      {\expandafter \noexpand \csname the#3\endcsname
       \@thmcountersep \@thmcounter{#1}}%
%    \end{macrocode}
%    Thus with the defaults of \LaTeX{}, |\theLem| would be
%    replaced by the command sequence |\thesection.\arabic{Lem}|.
%
%    We will now look at the definition of the macro which is executed
%    at the beginning of the actual environment (in our example this
%    macro is |\Lem|). It should be noted that we use an
%    ``|\expandafter| trick'' to expand only certain parts of the
%    replacement text at the time of the definition.
%    \begin{macrocode}
    \def\@tempa{\global\@namedef{#1}}%
    \expandafter \@tempa \expandafter{%
%    \end{macrocode}
%    First, the macro that contains the current definitions of
%    |\@begintheorem| and |\@opargtheorem| should be called
%    up.  The name of this macro---as is already known---has the form
%    |\th@|\meta{theorem\ style}; therefore, it must be called by
%    \begin{macrocode}
      \csname th@\the \theorem@style
%    \end{macrocode}
%    In addition the default theorem font should be changeable, i.e.\
%    we have to insert the contents of |\theorem@bodyfont|. For
%    that reason, we expand even further, beyond |\endcsname|,
%    and thus insert the contents of the \textsf{token} register
%    |\theorem@bodyfont| in the replacement text.
%    \begin{macrocode}
            \expandafter \endcsname \the \theorem@bodyfont
%    \end{macrocode}
%    Now it is time to call the macro |\@thm| which takes over
%    the further processing.  It has two arguments: the current
%    counter name (in our example, \texttt{Lem}), and the text of the
%    label (in our example, \texttt{Lemma}).
%    \begin{macrocode}
     \@thm{#1}{#2}}%
%    \end{macrocode}
%    With this, the `sub-definition' is complete.  The macro
%    |\@endtheorem| ends a theorem environment and is, so far, nothing
%    but an |\endtrivlist|.  (Hence it is defined globally, and not
%    within the theorem styles.\footnote{This has to be changed as
%    soon as theorem styles that change \texttt{\bslash @endtheorem}
%    exist.  In such a case, all existing styles must be changed as
%    well since they will have to reset the macro.}) Therefore, we can
%    set it equivalent to the macro that ends the theorem set (in our
%    example, |\endLem|).  However, if some day theorem styles exist
%    that do change |\@endtheorem|, we would have to use the
%    commented-out line instead.
%    \begin{macrocode}
    \global \expandafter \let \csname end#1\endcsname \@endtheorem
%   \global\@namedef{end#1}{\@endtheorem}%
%    \end{macrocode}
%    With these commands all the required definitions are employed,
%    unless the test |\@ifdefinable| has failed.  Therefore, we
%    end the second argument of this macro and with it the definition
%    of |\@xnthm|.
%    \begin{macrocode}
   }}
%    \end{macrocode}
% \end{macro}
%
% \begin{macro}{\@ynthm}
%    The definition of |\@ynthm| is completely analogous. In
%    this case the new counter that is defined is not reset within
%    another counter; thus the definition of |\the...|
%    is simplified:
%    \begin{macrocode}
\gdef\@ynthm#1#2{\expandafter\@ifdefinable\csname #1\endcsname
   {\@definecounter{#1}%
    \expandafter\xdef\csname the#1\endcsname{\@thmcounter{#1}}%
%    \end{macrocode}
%    The rest of the definition corresponds literally to that of
%    |\@xnthm|:
%    \begin{macrocode}
    \def\@tempa{\global\@namedef{#1}}\expandafter \@tempa
     \expandafter{\csname th@\the \theorem@style \expandafter
     \endcsname \the\theorem@bodyfont \@thm{#1}{#2}}%
    \global \expandafter \let \csname end#1\endcsname \@endtheorem}}
%    \end{macrocode}
% \end{macro}
%
% \begin{macro}{\@othm}
%    The definition of |\@othm| does not contain anything new.
%    \begin{macrocode}
\gdef\@othm#1[#2]#3{%
%    \end{macrocode}
% \changes{v2.2b}{95/10/07}{Build in check for counter defined}
%    We do not define a new counter but instead use one that has
%    already been defined.  Thus the only definition we need is that
%    of this pseudo-counter (i.e.\ |\the|\meta{env.\ name}).
%    First we check if |#2| corresponds to a known counter name.
%    \begin{macrocode}
  \expandafter\ifx\csname c@#2\endcsname\relax
   \@nocounterr{#2}%
  \else
   \expandafter\@ifdefinable\csname #1\endcsname
   {\expandafter \xdef \csname the#1\endcsname
     {\expandafter \noexpand \csname the#2\endcsname}%
%    \end{macrocode}
%    All other parts of the definition can be adopted from
%    |\@xnthm|. We have to remember, though, that in this case
%    the name of the current counter and the theorem label have moved
%    to the second and third arguments.
%    \begin{macrocode}
    \def\@tempa{\global\@namedef{#1}}\expandafter \@tempa
     \expandafter{\csname th@\the \theorem@style \expandafter
     \endcsname \the\theorem@bodyfont \@thm{#2}{#3}}%
    \global \expandafter \let \csname end#1\endcsname \@endtheorem}%
  \fi}
%    \end{macrocode}
% \end{macro}
%
% \subsection{Macros that are employed in a {\envfont theorem}
%    environment}
%
% \begin{macro}{\@thm}
%    The macro |\@thm| has to increase the current counter.  Then,
%    depending on whether the environment has (or does not have) an
%    optional argument, it has to branch into either |\@begintheorem|
%    or |\@opargtheorem|.
%    \begin{macrocode}
\gdef\@thm#1#2{\refstepcounter{#1}%
%    \end{macrocode}
%    Now we start a {\envfont trivlist} environment, and give
%    |\@topsep| and |\@topsepadd| the values of the \textsf{skip}
%    registers |\theorempreskipamount| and |\theorempostskipamount|.
%    The value in |\@topsep| is the vertical space that is inserted by
%    the first (and only) |\item| in our |\trivlist| whilst
%    |\@topsepadd| is inserted by |\@endparenv| at the end of that
%    {\envfont trivlist} environment.  By using these registers, we
%    obtain the desired space around a {\envfont theorem} environment.
%    \begin{macrocode}
   \trivlist
   \@topsep \theorempreskipamount               % used by first \item
   \@topsepadd \theorempostskipamount           % used by \@endparenv
%    \end{macrocode}
%    Now we have to test whether an optional argument has been given.
% \changes{v2.0d}{89/07/18}{Forgotten percent added.}
%    \begin{macrocode}
   \@ifnextchar [%
%    \end{macrocode}
%    If there is an optional argument, we will call |\@ythm|, and
%    move the arguments read back into the input stream.
%    \begin{macrocode}
   {\@ythm{#1}{#2}}%
%    \end{macrocode}
%    If not, we call |\@begintheorem|. Its first argument is the
%    name of the theorem set (hence the second argument of
%    |\@thm|). Its second argument is the macro that produces the
%    current number.
%    \begin{macrocode}
   {\@begintheorem{#2}{\csname the#1\endcsname}\ignorespaces}}
%    \end{macrocode}
% \end{macro}
%
% \begin{macro}{\@xthm}
% \begin{macro}{\@ythm}
%    Both these macros were originally called by |\@thm|.  We do
%    not need |\@xthm| anymore, hence we reset it to
%    |\relax|. The definition of |\@ythm| has not changed at
%    all from its definition in \LaTeX. In order to make the macros
%    easier to understand, we will
%    nevertheless present it (commented out).
%    \begin{macrocode}
\global\let\@xthm\relax
% \def\@ythm#1#2[#3]{\@opargbegintheorem{#2}{\csname
%           the#1\endcsname}{#3}\ignorespaces}
%</package>
%    \end{macrocode}
%    The primitive |\ignorespaces| in |\@ythm| and
%    |\@thm| is needed to remove the spaces between the
%    |\begin{...}| and the actual text.
% \end{macro}
% \end{macro}
%
%
% \subsection{Definition of the theorem substyles}
%
% As already pointed out, the theorem substyles, defined below, are
% only loaded when necessary.  Note that all these substyles, except
% \texttt{plain}, have |\slshape| as the default body font.
%
% \subsubsection{The \texttt{plain} style}
%
% \label{dok:th@plain} As the following macros use |@|, we have
% to locally set the |\catcode| of this symbol to ``letter''.
% This happens within a group, so that we do not have to worry about
% which |\catcode| that symbol had before.
%    \begin{macrocode}
%<*thp>
\begingroup \makeatletter
%    \end{macrocode}
% Since we are now within a group, we must make all definitions
% globally.  First we make sure that \texttt{theorem.sty} is loaded.
% This will allow us to use this file as a document style option
% without having to call \texttt{theorem} itself as an option.  At the
% same time, we assure that at least version 2 is loaded, since
% |\theorem@style| was not defined in earlier versions.
%    \begin{macrocode}
\@ifundefined{theorem@style}{\input{theorem.sty}}{}
\ProvidesFile{thp.sty}
             [\FMithmInfo]
%    \end{macrocode}
%
%
% \begin{macro}{\th@plain}
% \begin{macro}{\@begintheorem}
% \begin{macro}{\@opargbegintheorem}
%    |\theoremstyle{plain}| corresponds to the original
%    definition, except that the distances to the surrounding text are
%    determined by the parameters |\theorempreskipamount| and
%    |\theorempostskipamount|.
%    First we set the default body font.
%    \begin{macrocode}
\gdef\th@plain{\normalfont\itshape
%    \end{macrocode}
%    Then we define |\@begintheorem| and
%    |\@opargbegintheorem|. These two macros define how the
%    header of a theorem is typeset.  |\@opargbegintheorem| will
%    be called if a \textsf{theorem} environment with an optional
%    argument is encountered; otherwise, the header is constructed by
%    calling |\@begintheorem|.  If one of these macros is
%    executed, we are within a \textsf{trivlist} environment started by
%    |\@thm|.  So the theorem header is produced with an
%    |\item| command.
%
%    Instead of specifying the header font directly, all standard
%    theorem styles use the |\theorem@headerfont| macro to allow
%    customization.  The extra space (|\labelsep|) is necessary
%    because of problems in the \textsf{trivlist} environment.
%    \begin{macrocode}
  \def\@begintheorem##1##2{%
        \item[\hskip\labelsep \theorem@headerfont ##1\ ##2]}%
%    \end{macrocode}
%    The definition of |\@opargbegintheorem| is completely
%    analogous. The only difference is the fact that there exists a
%    third argument (which is the optional parameter of the
%    environment and contains additional information about the
%    theorem).  Customarily we enclose it in parentheses.
%    \begin{macrocode}
\def\@opargbegintheorem##1##2##3{%
   \item[\hskip\labelsep \theorem@headerfont ##1\ ##2\ (##3)]}}
%    \end{macrocode}
% \end{macro}
% \end{macro}
% \end{macro}
%
% \noindent
% We conclude with an |\endgroup| to restore the
% |\catcode| of |@|.
%    \begin{macrocode}
\endgroup
%</thp>
%    \end{macrocode}
%
%
% \subsubsection{The \texttt{break} style}
%
% This style option is stored in the file |thb.sty|.
% For the next two lines see the documentation for |\th@plain| on
% page~\pageref{dok:th@plain}.
%    \begin{macrocode}
%<*thb>
\begingroup \makeatletter
\@ifundefined{theorem@style}{\input{theorem.sty}}{}
\ProvidesFile{thb.sty}
             [\FMithmInfo]
%    \end{macrocode}
%
% \begin{macro}{\th@break}
%    |\theoremstyle{break}| produces a line break after the name
%    of the theorem.  The font is |\slshape|.  Hence, we define
%    |\th@break| as follows:
%    \begin{macrocode}
\gdef\th@break{\normalfont\slshape
  \def\@begintheorem##1##2{\item[%
%    \end{macrocode}
%    We run into the following problem: it is not possible to create
%    the header with |\item[|\meta{title}|]| and then start
%    a new line by, for example, |\mbox{}\\|. Such a definition
%    will fail whenever a list environment follows immediately.  With
%    the above construction, the |\mbox{}| causes the switch
%    |@inlabel| (cf.\ definition of |\list| and
%    |\trivlist| in \cite{latex:source}) to be set to \texttt{false}
%    and so the following list will insert additional vertical
%    space (|\topskip|). This is quite annoying. Therefore, we
%    create the line break within the |\item|. In order to ensure
%    that the text will begin at the proper position in the following
%    line, we simply pretend that the label does not take any
%    room.\footnote{This will lead to problems whenever very high
%    symbols occurring in the line tower into the
%    heading. So, something else has to be done here sometime.}
%    \begin{macrocode}
       \rlap{\vbox{\hbox{\hskip \labelsep\theorem@headerfont ##1\ ##2}%
                   \hbox{\strut}}}]}%
%    \end{macrocode}
%    Again, the definition of |\@opargbegintheorem| is
%    completely analogous.
%    \begin{macrocode}
\def\@opargbegintheorem##1##2##3{%
  \item[\rlap{\vbox{\hbox{\hskip \labelsep \theorem@headerfont
                     ##1\ ##2\ (##3)}%
                    \hbox{\strut}}}]}}
\endgroup
%</thb>
%    \end{macrocode}
% \end{macro}
%
%
% \subsubsection{The \texttt{changebreak} style}
%
%    \begin{macrocode}
% This style option is stored in the file |thcb.sty|.
%    \begin{macrocode}
%<*thcb>
\begingroup \makeatletter
\@ifundefined{theorem@style}{\input{theorem.sty}}{}
\ProvidesFile{thcb.sty}
             [\FMithmInfo]
%    \end{macrocode}
%
% \begin{macro}{\th@changebreak}
%    The \texttt{change-break} theorem style is like \texttt{break}
%    but with interchange of theorem name and theorem number.  Thus we
%    define |\th@changebreak| as follows:
%    \begin{macrocode}
\gdef\th@changebreak{\normalfont\slshape
  \def\@begintheorem##1##2{\item
       [\rlap{\vbox{\hbox{\hskip\labelsep \theorem@headerfont ##2\ ##1}%
                    \hbox{\strut}}}]}%
\def\@opargbegintheorem##1##2##3{%
  \item[\rlap{\vbox{\hbox{\hskip\labelsep \theorem@headerfont
                     ##2\ ##1\ (##3)}%
                    \hbox{\strut}}}]}}
\endgroup
%</thcb>
%    \end{macrocode}
% \end{macro}
%
%
% \subsubsection{The \texttt{change} style}
%
% This style option is stored in the file |thc.sty|.
%    \begin{macrocode}
%<*thc>
\begingroup \makeatletter
\@ifundefined{theorem@style}{\input{theorem.sty}}{}
\ProvidesFile{thc.sty}
             [\FMithmInfo]
%    \end{macrocode}
%
% \begin{macro}{\th@change}
%    The \texttt{change} theorem style corresponds to the
%    \texttt{change break} style without a linebreak after the header.
%    To say it in another way, it's the same as the \texttt{plain}
%    style but with number and name interchanged and |\slshape| as the
%    default font.
%    \begin{macrocode}
\gdef\th@change{\normalfont\slshape
  \def\@begintheorem##1##2{\item
       [\hskip\labelsep \theorem@headerfont ##2\ ##1]}%
\def\@opargbegintheorem##1##2##3{%
  \item[\hskip\labelsep \theorem@headerfont ##2\ ##1\ (##3)]}}
\endgroup
%</thc>
%    \end{macrocode}
% \end{macro}
%
%
% \subsubsection{The \texttt{marginbreak} style}
%
% This style option is the one used most often at Mainz.  It is saved
% in the file |thmb.sty|.
%    \begin{macrocode}
%<*thmb>
\begingroup \makeatletter
\@ifundefined{theorem@style}{\input{theorem.sty}}{}
\ProvidesFile{thmb.sty}
             [\FMithmInfo]
%    \end{macrocode}
%
% \begin{macro}{\th@marginbreak}
%    The \texttt{margin break} style is nearly the same as the
%    \texttt{change break} style. The only difference is the placement
%    of the theorem number.  We use |\llap| to place it in the left
%    margin.
%
%    In this style |\labelsep| denotes the separation between the
%    number and the text.
%    \begin{macrocode}
\gdef\th@marginbreak{\normalfont\slshape
  \def\@begintheorem##1##2{\item
       [\rlap{\vbox{\theorem@headerfont
                    \hbox{\llap{##2}\hskip\labelsep ##1}%
                    \hbox{\strut}}}]}%
\def\@opargbegintheorem##1##2##3{%
  \item[\rlap{\vbox{\theorem@headerfont
                    \hbox{\llap{##2}\hskip\labelsep ##1\ (##3)}%
                    \hbox{\strut}}}]}}
\endgroup
%</thmb>
%    \end{macrocode}
% \changes{v2.1c}{90/02/17}{`endgroup outside macrocode env.}
% \end{macro}
%
%
%
% \subsubsection{The \texttt{margin} style}
%
% This style option is stored in the file |thm.sty|.
%    \begin{macrocode}
%<*thm>
\begingroup \makeatletter
\@ifundefined{theorem@style}{\input{theorem.sty}}{}
\ProvidesFile{thm.sty}
             [\FMithmInfo]
%    \end{macrocode}
%
% \begin{macro}{\th@margin}
%    Again this is only a variant of the theorem styles described
%    above without any new ideas.
%    \begin{macrocode}
\gdef\th@margin{\normalfont\slshape
  \def\@begintheorem##1##2{\item
       [\theorem@headerfont \llap{##2}\hskip\labelsep ##1]}%
\def\@opargbegintheorem##1##2##3{%
  \item[\theorem@headerfont \llap{##2}\hskip\labelsep ##1\ (##3)]}}
\endgroup
%</thm>
%    \end{macrocode}
% \end{macro}
%
%
%
%
% \subsection{Final Definitions}
%
% \begin{macro}{\theorempreskipamount}
% \begin{macro}{\theorempostskipamount}
%    The \textsf{skip} parameters that regulate the vertical empty space
%    before and after the {\envfont theorem} environment have
%    to be allocated as well.
% \changes{v2.0f}{89/07/20}{Unnecessary `global' removed.}
%    \begin{macrocode}
%<*package>
\newskip\theorempreskipamount
\newskip\theorempostskipamount
%    \end{macrocode}
%    Since we have used the same values for all theorem sets, we now can
%    assign them.
%    \begin{macrocode}
\global\setlength\theorempreskipamount{12pt plus 5pt minus 3pt}
\global\setlength\theorempostskipamount{8pt plus 3pt minus 1.5pt}
%    \end{macrocode}
% \end{macro}
% \end{macro}
%
%
% \begin{macro}{\@endtheorem}
%    The same holds for the macro |\@endtheorem|, which ends a
%    {\envfont theorem} environment. Since it is the same for all
%    theorem sets, it is removed from the macros |\th@|\meta{style}.
%    It simply ends the {\envfont trivlist} environment, which was
%    begun in |\@thm|.
%    \begin{macrocode}
\global\let\@endtheorem=\endtrivlist
%    \end{macrocode}
% \end{macro}
%
%
%
% \begin{macro}{\@preamblecmds}
%    All macros defined above are to be used only in the preamble.
%    Therefore, we insert them in |\@preamblecmds| which will disable
%    them at begin document. This is done by the internal \LaTeXe{}
%    command |\@onlypreamble|.
% \changes{v2.0g}{89/09/19}{`theoremheaderfont and `theorembodyfont
%    added.}
% \changes{v2.2a}{94/02/03}{Use \LaTeXe{} interface}
%    \begin{macrocode}
\@onlypreamble\@xnthm
\@onlypreamble\@ynthm
\@onlypreamble\@othm
\@onlypreamble\newtheorem
\@onlypreamble\theoremstyle
\@onlypreamble\theorembodyfont
\@onlypreamble\theoremheaderfont
%    \end{macrocode}
% \end{macro}
%
% Finally we declare the \texttt{plain} theorem style to be the default.
% \changes{v2.1b}{90/06/04}{Fixed missing `item by declaring default
%    style.}
%    \begin{macrocode}
\theoremstyle{plain}
%</package>
%    \end{macrocode}
%
% \Finale

\endinput
}{}
\ProvidesFile{thc.sty}
             [\FMithmInfo]
%    \end{macrocode}
%
% \begin{macro}{\th@change}
%    The \texttt{change} theorem style corresponds to the
%    \texttt{change break} style without a linebreak after the header.
%    To say it in another way, it's the same as the \texttt{plain}
%    style but with number and name interchanged and |\slshape| as the
%    default font.
%    \begin{macrocode}
\gdef\th@change{\normalfont\slshape
  \def\@begintheorem##1##2{\item
       [\hskip\labelsep \theorem@headerfont ##2\ ##1]}%
\def\@opargbegintheorem##1##2##3{%
  \item[\hskip\labelsep \theorem@headerfont ##2\ ##1\ (##3)]}}
\endgroup
%</thc>
%    \end{macrocode}
% \end{macro}
%
%
% \subsubsection{The \texttt{marginbreak} style}
%
% This style option is the one used most often at Mainz.  It is saved
% in the file |thmb.sty|.
%    \begin{macrocode}
%<*thmb>
\begingroup \makeatletter
\@ifundefined{theorem@style}{% \iffalse meta-comment
%
% Copyright (C) 1993-2024
%
% The LaTeX Project and any individual authors listed elsewhere
% in this file.
%
% This file is part of the Standard LaTeX `Tools Bundle'.
% -------------------------------------------------------
%
% It may be distributed and/or modified under the
% conditions of the LaTeX Project Public License, either version 1.3c
% of this license or (at your option) any later version.
% The latest version of this license is in
%    https://www.latex-project.org/lppl.txt
% and version 1.3c or later is part of all distributions of LaTeX
% version 2005/12/01 or later.
%
% The list of all files belonging to the LaTeX `Tools Bundle' is
% given in the file `manifest.txt'.
%
% \fi
%
%%
%
%
\def\FMithmInfo{2023/07/05 v2.2c Theorem extension package (FMi)}
%
% \ProvidesFile{theorem.dtx}[\FMithmInfo]
%\iffalse   % this is a METACOMMENT !
%%
%% Package `theorem' to use with LaTeX2e
%% Copyright (C) 1989-1995 Frank Mittelbach, all rights reserved.
%<+package>\@ifundefined{theorem@style}{}{\endinput}
%<+package>\NeedsTeXFormat{LaTeX2e}[1995/06/01]
%<+package>\ProvidesPackage{theorem}
%<+package>         [\FMithmInfo]
%
% \fi
%
% \changes{v1.0}{87/05/07}{First draft.}
% \changes{v1.0a}{87/05/07}{`thistheoremstyle' added.}
% \changes{v1.1a}{87/07/29}{New concept: redefine the internal
%                          \protect\LaTeX{} commands `@xnthm' etc.}
% \changes{v1.1b}{87/07/31}{`theoremfont' used to specify font.}
% \changes{v1.2a}{88/01/08}{`@endtheorem' now globally defined.}
% \changes{v1.2b}{88/01/11}{`newtheorem' only allowed in the preamble.}
% \changes{v1.3a}{88/02/14}{`theoremfont' now a toks register.}
%
% \changes{v2.0a}{88/11/16}{New implementation and new documentation.}
% \changes{v2.0a}{88/11/16}{No layout changes but internal modifications
%                           old substyles won't work.}
% \changes{v2.0b}{89/01/30}{`global' used for all definitions.}
% \changes{v2.0c}{89/06/15}{Documentation translated to English.
%                            Thanks to LLL and Nina}
% \changes{v2.1a}{90/02/15}{Support for new font selection.}
% \changes{v2.2a}{94/02/03}{Cleanup for LaTeX2e}
%
% \DoNotIndex{\@endparenv,\@ifdefinable,\@ifnextchar,\@ifundefined}
% \DoNotIndex{\@namedef,\@nameuse,\@spaces,\@tempa,\@warning}
% \DoNotIndex{\ }
% \DoNotIndex{\begingroup,\bfseries}
% \DoNotIndex{\csname}
% \DoNotIndex{\def,\do,\docdate}
% \DoNotIndex{\endcsname,\endgroup,\endinput,\endtrivlist,\expandafter}
% \DoNotIndex{\filedate,\fileversion}
% \DoNotIndex{\gdef,\global}
% \DoNotIndex{\hbox,\hskip}
% \DoNotIndex{\ignorespaces,\input,\itshape,\item}
% \DoNotIndex{\labelsep,\let,\llap}
% \DoNotIndex{\makeatletter}
% \DoNotIndex{\noexpand}
% \DoNotIndex{\refstepcounter,\relax,\rlap}
% \DoNotIndex{\setlength,\slshape,\space,\string,\strut}
% \DoNotIndex{\the,\trivlist,\typeout}
% \DoNotIndex{\vbox}
% \DoNotIndex{\xdef}
%
% \def\deflabel#1{\ttfamily #1\hfill}
% \def\deflist#1{\begin{list}{}{\settowidth\labelwidth{\ttfamily #1}%
%                               \setlength\leftmargin\labelwidth
%                               \addtolength\leftmargin\labelsep
%                               \let\makelabel\deflabel}}
% \def\enddeflist{\end{list}}
% \def\envfont{\normalfont\ttfamily}
%
% \MakeShortVerb{\|}
%
% \newcommand{\TUB}{{\sl TUGboat\/}}
%
% \GetFileInfo{theorem.dtx}
% \title{An Extension of the
%        \LaTeX{} theorem environment\thanks{This file
%        has version number \fileversion, last
%        revised \filedate.}}
% \author{Frank Mittelbach}
%
% \MaintainedByLaTeXTeam{tools}
% \maketitle
%
% \begin{abstract}
%    The macros described in this paper yield an extension of the
%    \LaTeX{} theorem mechanism. It is designed to satisfy the
%    different requirements of various journals. Thus, the
%    layout of the ``theorems'' can be manipulated by determining a
%    ``style''. This article describes not only the use, but
%    also the definition, of the necessary macros.
% \end{abstract}
%
% \section*{Preface to version 2.2}
%
% For \LaTeXe{} this package did not need any fundamental changes. I
% only modified the messages generated so that theorem layout styles
% will show up with the |\listfiles| command and cleaned the section
% on the New Font Selection Scheme since this is now included in
% \LaTeX.
%
% \section*{Preface to version 2.1}
%
% This version is identical to 2.0g described in \TUB\ 10\#3 except
% for some internal defaults which are now set depending on the used
% font selection scheme.
%
% This was done to avoid unpleasant surprises if the new font selection
% scheme is in force. For further details see
% section~\ref{sec:newfonts} and \cite{bk:GMS94}.
%
% \section{Introduction}
%
% For our purposes here, ``theorems'' are labelled enunciations,
% often set off from the main text by extra space and a font change.
% Theorems, corollaries, conjectures, definitions, and remarks
% are all instances of ``theorems''.  The ``header'' of these structures
% is composed of a label (such as \textsc{Theorem} or \textsc{Remark})
% and a number which serializes an item in the sequence of items
% with the same label.
%
% Shortly after the introduction of \LaTeX{} at the Fachbereich
% Mathematik in Mainz, the desire to manipulate the layout of
% ``theorems'' arose.  In Mainz, the following two conventions came
% into general use:
% \begin{enumerate}
%   \item The number of the theorem is shown in the margin.
%   \item There is a line break at
%         the end of the theorem header.
% \end{enumerate}
%
% Additionally, some journals require different formats which depend on
% the ``sort of theorem'': e.g.\ often remarks and definitions are set
% in |\upshape|, while |\itshape| is employed for main theorems.
%
% Confronted with these requirements, a theorem environment was
% developed in Mainz which allows separate determination of the
% layout of the ``theorems sets'', comparable to |\pagestyle|.
%
%
% \section{The user interface}
% \subsection{Defining new theorem sets}
%
% \DescribeMacro\newtheorem
% As in the original \LaTeX{} version, the command |\newtheorem|
% defines a new ``theorem set'' or ``theorem-like structure''.
% Two required arguments name the new environment and give the text
% to be typeset with each instance of the new ``set'', while
% an optional argument determines how
% the ``set'' is enumerated:
% \begin{description}
%    \item[\ttfamily \bslash newtheorem\{foo\}\{bar\}]
%       The theorem set {\envfont foo} (whose name is \texttt{bar})
%       uses its own counter.
%    \item[\ttfamily \bslash newtheorem\{foo2\}{[foo]}\{bar2\}]
%       The theorem set {\envfont foo2} (printed name \texttt{bar2})
%       uses the same counter as the theorem set \texttt{foo}.
%    \item[\ttfamily \bslash newtheorem\{foo3\}\{bar3\}{[section]}]
%       The theorem set {\envfont foo3} (printed name \texttt{bar3}) is
%       enumerated within the counter \texttt{section}, i.e.\ with every
%       new |\section| the enumeration begins again with 1, and
%       the enumeration is composed from the section-number and the
%       theorem counter itself.
% \end{description}
%
% \DescribeMacro\theoremstyle
% Additionally, the command |\theoremstyle| can define the layout
% of various, or all, theorem sets.  It should be noted that any
% theorem set defined by |\newtheorem| is typeset in the
% |\theoremstyle| that is current at the time of the definition.
% Thus, the following
% \begin{verbatim}
%    \theoremstyle{break}        \newtheorem{Cor}{Corollary}
%    \theoremstyle{plain}        \newtheorem{Exa}{Example}[section]
%\end{verbatim}                             ^^A No blank at line start
% leads to the result that the set {\envfont Cor} is formatted in the
% style \texttt{break}, while the set {\envfont Exa} and all the
% following ones are formatted in the style \texttt{plain}, unless
% another |\theoremstyle| follows.  Since the definitions installed by
% |\newtheorem| are global, one also can limit |\theoremstyle| locally
% by grouping braces.
%
% \DescribeMacro\theorembodyfont
% The choice of the font for the theorem body is completely
% independent of the chosen |\theoremstyle|; this has proven to
% be very advantageous.  For example,
% \begin{verbatim}
%   {\theorembodyfont{\upshape}          \newtheorem{Rem}{Remark}}
%\end{verbatim}
% defines a theorem set {\envfont Rem}, which will be set in
% |\upshape| in the current layout (which in our example is
% \texttt{plain}).  As with |\theoremstyle|, the |\theorembodyfont|
% chosen is that current at the time of |\newtheorem|.  If
% |\theorembodyfont| is not specified or one defines
% |\theorembodyfont{}|, then the font used will be that defined by the
% |\theoremstyle|.
%
% \DescribeMacro\theoremheaderfont It is also possible to customize
% the font used for the theorem headers.  This is, however, a global
% declaration, and therefore there should be at most one
% |\theoremheaderfont| declaration in the preamble.\footnote{If it is
% actually necessary to have different header fonts, one has to define
% new theorem styles (substituting the desired font) or specify the
% information directly in the \texttt{\bslash newtheorem} declaration
% (the unclean variant).}
%
% \DescribeMacro\theorempreskipamount
% \DescribeMacro\theorempostskipamount Two additional parameters
% affect the vertical space around the theorem environments:
% |\theorempreskipamount| and |\theorempostskipamount| define,
% respectively, the spacing before and after such an environment.
% These parameters apply for all theorem sets and can be manipulated
% with the ordinary length macros.  They are rubber lengths
% (`\textsf{skips}'), and therefore can contain \texttt{plus} and
% \texttt{minus} parts.
%
% Since the definition of theorem sets should---most sensibly---be
% placed in the preamble, we only allow installation there.
% It is therefore possible to release the memory used here after
% |\begin{document}|, in order to make room for other
% applications.
%
%
% \subsection{Existing theorem styles}
%
% The following theorem styles exist to date:
% \begin{deflist}{marginbreak}
%    \item[plain]
%       This theorem style emulates the original \LaTeX{} definition,
%       except that additionally the parameters
%       |\theorem...skipamount| are used.
%    \item[break]
%       In this style, the theorem header is followed by a line
%       break.
%    \item[marginbreak]
%       The theorem number is set in the margin, and there is a line
%       break as in \texttt{break}.
%    \item[changebreak]
%       Like \texttt{break}, but with header number and text
%       interchanged.
%    \item[change]
%       Header number and text are interchanged, without a line break.
%    \item[margin]
%       The number is set in the left margin, without a line break.
% \end{deflist}
% All styles (except \texttt{plain}) select |\slshape| as the default
% |\theorembodyfont|.
%
%
% \subsection{Examples}
%
% Given the above theorem sets {\envfont Cor}, {\envfont Exa}
% and {\envfont Rem},
% suppose that the preamble also contains the declarations:
% \begin{verbatim}
%    \theoremstyle{marginbreak}   \newtheorem{Lem}[Cor]{Lemma}
%    \theoremstyle{change}
%    \theorembodyfont{\itshape}        \newtheorem{Def}[Cor]{Definition}
%
%    \theoremheaderfont{\scshape}
%\end{verbatim}
% Then the following are some typical examples of the typeset output
% resulting from their use.
%
% \begin{Cor}
%    This is a sentence typeset in the theorem environment {\envfont
%    Cor}.
% \end{Cor}
%
% \begin{Exa}
%    This is a sentence typeset in the theorem environment {\envfont
%    Exa}.
% \end{Exa}
%
% \begin{Rem}
%    This is a sentence typeset in the theorem environment {\envfont
%    Rem}.
% \end{Rem}
%
% \begin{Lem}[Ben User]
%    This is a sentence typeset in the theorem environment {\envfont
%    Lem}.
% \end{Lem}
%
% \begin{Def}[Very impressive Definition]
%    This is a sentence typeset in the theorem environment {\envfont
%    Def}.
% \end{Def}
%
% The last two examples show the effect of the optional argument to
% a theorem environment (it is the text typeset in parentheses).
%
%
% \section{Special Considerations}
% \label{sec:newfonts}
%
% Theoremheader and body are implemented as a unit. This means that
% the |\theoremheaderfont|  will inherit characteristics of the
% |\theorembodyfont| in \LaTeXe. Thus, if for example
% |\theorembodyfont| is |\itshape| and |\theoremheaderfont|
% is |\bfseries| the font selected for the header will have the
% characteristics `bold extended italic'. If this is not desired one
% should set the |\theoremheaderfont| to something like
% \begin{verbatim}
%      \theoremheaderfont{\normalfont\bfseries}
%\end{verbatim}
% i.e.\ supplying all necessary font information explicitly.
%
%
%
% \section{Acknowledgements}
%
% The publication of this set of macros was only possible with the
% help of Christina Busse (translating the manuscript into English),
% Joachim Pense (playing the r\^ole of typist), Chris Rowley
% (looking everything over) and many others providing useful
% suggestions.
%
% \MaybeStop{%
%    \begin{thebibliography}{9}
%       \bibitem{bk:GMS94} \textsc{M.~Goossens}, \textsc{F.~Mittelbach}
%         and \textsc{A.~Samarin}.
%         \newblock The \LaTeX{} Companion.
%         \newblock
%         Addison-Wesley, Reading, Massachusetts, 1994.
%       \bibitem{latex:source} \textsc{Lamport, Leslie.} \newblock
%                               \texttt{latex.tex}, version 2.09, date
%                               Feb. 1990.
%    \end{thebibliography}
%    \PrintIndex
%    \PrintChanges}
%
%
% \section{The documentation driver file}
%
% The next bit of code contains the documentation driver file for
% \TeX{}, i.e., the file that will produce the documentation you are
% currently reading. It will be extracted from this file by the
% \texttt{docstrip} program. Since it is the first code in the file
% one can alternatively process this file directly with \LaTeXe{} to
% obtain the documentation.
%
%    \begin{macrocode}
%<*driver>
\documentclass{ltxdoc}

\usepackage{theorem}

%<+driver>% The next few lines define theorem sets which are used
%<+driver>% in the example section of the documentation.

\theoremstyle{break}       \newtheorem{Cor}{Corollary}
\theoremstyle{plain}       \newtheorem{Exa}{Example}[section]
{\theorembodyfont{\upshape}\newtheorem{Rem}{Remark}}
\theoremstyle{marginbreak} \newtheorem{Lem}[Cor]{Lemma}
\theoremstyle{change}
\theorembodyfont{\itshape} \newtheorem{Def}[Cor]{Definition}

\theoremheaderfont{\scshape}

\RecordChanges

\begin{document}
   \DocInput{theorem.dtx}
\end{document}
%</driver>
%    \end{macrocode}
%
% \section{Definition of the Macros}
%
% If the file has been loaded before, we abort immediately. If not the
% package announces itself (this is actually done at the very top of
% the file---the way it is done isn't good style so don't copy it).
% \changes{v2.0e}{89/07/19}{Spaces removed from `typeout'.}
% \changes{v2.2b}{95/11/19}{Announce moved to top of file}
%    \begin{macrocode}
%<*package>
%\@ifundefined{theorem@style}{}{\endinput}
%\def\FMithmInfo{1995/11/19 v2.2b Theorem extension package (FMi)}
%\ProvidesPackage{theorem}[\FMithmInfo]
%    \end{macrocode}
%
%
% \subsection{Definition of theorem styles and fonts}
%
% All the definitions in this file are done globally to allow
% inputting this file inside a group.
%
% \begin{macro}{\theoremstyle}
%    Before a theorem style can be installed, the chosen style must be
%    known. For that reason, we must test to see that
%    |\th@|\meta{style} is known or, more precisely, that it is
%    different from |\relax|. If the style is not known then
%    |\th@plain| is used.
%    \begin{macrocode}
\gdef\theoremstyle#1{%
   \@ifundefined{th@#1}{\@warning
          {Unknown theoremstyle `#1'. Using `plain'}%
          \theorem@style{plain}}%
%    \end{macrocode}
%    We save the theorem style to be used in the token register
%    |\theorem@style|.
%    \begin{macrocode}
      {\theorem@style{#1}}%
%    \end{macrocode}
%    Now we ``evaluate'' the theorem style: this means, we call the
%    macro |\th@|\meta{style} which will activate the relevant
%    definitions which are contained in a separate file.  This is done
%    in a group to suppress changes to the current font.  This could
%    otherwise pose problems together with the new font selection
%    scheme\footnote{When I printed the original article using the new
%    font selection scheme I ended with a document with slanted
%    typefaces (text headings and all) simply because one of the
%    theorem styles used \texttt{\bslash sl} at toplevel.} if the
%    |\th@|\meta{style} is evaluated a second time.
% \changes{v2.1a}{90/02/15}{Group added around `th@..}
% \changes{v2.1a}{90/02/15}{Reseting of `..@begintheorem removed.}
%    \begin{macrocode}
      \begingroup
        \csname th@\the\theorem@style \endcsname
      \endgroup}
%    \end{macrocode}
% \end{macro}
%
%
% \begin{macro}{\@begintheorem}
% \changes{v2.1a}{90/02/15}{Relaxed at toplevel.}
% \begin{macro}{\@opargbegintheorem}
% \changes{v2.1a}{90/02/15}{Relaxed at toplevel.}
%    We reset |\@begintheorem| and |\@opargbegintheorem| to
%    |\relax| since these commands are no longer necessary at
%    toplevel. This will save a few tokens.
%    \begin{macrocode}
\global\let\@begintheorem\relax
\global\let\@opargbegintheorem\relax
%    \end{macrocode}
% \end{macro}
% \end{macro}
%
% \begin{macro}{\theorem@style}
%    Obviously the \textsf{token} register used above has to be
%    allocated. To assure the utmost compatibility with the
%    original \LaTeX{} definition, we set the default theorem style to
%    \texttt{plain}, which implements the usual \LaTeX{} convention.
% \changes{v2.0f}{89/07/20}{Unnecessary `global' removed.}
%    \begin{macrocode}
\newtoks\theorem@style
\global\theorem@style{plain}
%    \end{macrocode}
% \end{macro}
%
% \begin{macro}{\theorembodyfont}
% \begin{macro}{\theorem@bodyfont}
%    If the |\theorembodyfont| is set by the user then it should not
%    interact with the default font set in the theorem style. When the
%    new font selection is in force this may happen if, for example,
%    the default is |\itshape| and the new |\theorembodyfont| is
%    |\sffamily|. So we add a |\reset@font| command in front of the
%    user definition.
% \changes{v2.0f}{89/07/20}{Unnecessary `global' removed.}
% \changes{v2.0g}{89/09/19}{`theoremfont to `theorembodyfont.}
% \changes{v2.1c}{92/06/22}{`reset@font added and logic changed.}
% \changes{v2.1d}{92/10/28}{Reset to empty if arg is empty}
%    \begin{macrocode}
\gdef\theorembodyfont#1{%
%    \end{macrocode}
%    We check if the argument supplied is empty and if so put nothing
%    into the |\theorem@bodyfont| token register to allow for
%    |\theorembodyfont{}| as a mean of using the default of the
%    current |\theoremstyle|.
%    \begin{macrocode}
   \def\@tempa{#1}%
   \ifx\@tempa\@empty
    \theorem@bodyfont{}%
   \else
    \theorem@bodyfont{\reset@font#1}%
   \fi
   }
\newtoks\theorem@bodyfont
\global\theorem@bodyfont{}
%    \end{macrocode}
% \end{macro}
% \end{macro}
%
%
% \begin{macro}{\theoremheaderfont}
% \changes{v2.0g}{89/09/19}{Macro added.}
%    The font for the theorem headers is handled differently because
%    this definition applies to all theorem styles.
%    \begin{macrocode}
\gdef\theoremheaderfont#1{\gdef\theorem@headerfont{#1}%
%    \end{macrocode}
%    After using the macro once it is redefined to produce an error
%    message.
%    \begin{macrocode}
       \gdef\theoremheaderfont##1{%
        \typeout{\string\theoremheaderfont\space should be used
                 only once.}}}
%    \end{macrocode}
% \end{macro}
%
% \begin{macro}{\theorem@headerfont}
% \changes{v2.0g}{89/09/19}{Macro added.}
%    To set the |\theorem@headerfont| default we first test if
%    the new fontselection scheme is in force.
% \changes{v2.1a}{90/02/15}{Support for new font selection.}
%    \begin{macrocode}
\ifx\upshape\undefined
%    \end{macrocode}
%    If not we define it to expand into |\bfseries|. We don't use
%    |\let| just in case a following style option redefines this
%    macro.
%    \begin{macrocode}
\gdef\theorem@headerfont{\bfseries}
%    \end{macrocode}
%    Otherwise we reset the current shape before calling |\bfseries|.
%    \begin{macrocode}
\else \gdef\theorem@headerfont{\normalfont\bfseries}\fi
%    \end{macrocode}
% \end{macro}
%
% \begin{macro}{\th@plain}
% \begin{macro}{\th@break}
% \begin{macro}{\th@marginbreak}
% \begin{macro}{\th@changebreak}
% \begin{macro}{\th@change}
% \begin{macro}{\th@margin}
%    The different styles are defined in macros such as
%    |\th@plain|.  Since memory space is precious in
%    ``non-Big-versions'', we have to avoid offering too many unused
%    definitions.  Therefore we define these styles in separate files
%    that can be loaded on demand.  Thus the commands themselves only
%    load these files. We use |\@input@|, a \LaTeXe{} internal command
%    that ensures that the file will be listed with |\listfiles|.
% \changes{v2.2a}{94/02/02}{Use `@input@ to load theorem layout files}
%    \begin{macrocode}
\gdef\th@plain{\@input@{thp.sty}}
\gdef\th@break{\@input@{thb.sty}}
\gdef\th@marginbreak{\@input@{thmb.sty}}
\gdef\th@changebreak{\@input@{thcb.sty}}
\gdef\th@change{\@input@{thc.sty}}
\gdef\th@margin{\@input@{thm.sty}}
%    \end{macrocode}
%    This list will be expanded when new styles become
%    available.  For testing,  just append new theorem substyles
%    as document options.
% \end{macro}
% \end{macro}
% \end{macro}
% \end{macro}
% \end{macro}
% \end{macro}
%
%
%
% \subsection{Definition of a new theorem set}
%
% As already pointed out, a new theorem environment can be defined in
% three different ways:
% \begin{verbatim}
%    \newtheorem{Lem}{Lemma}
%    \newtheorem{Lem}{Lemma}[section]
%    \newtheorem{Lem}[Theorem]{Lemma}
%\end{verbatim}
% The function of the macro |\newtheorem| is to recognize these cases
% and then to branch into one of the three macros |\@ynthm|, |\@xnthm|
% or |\@othm|.  This mechanism is adopted unchanged from
% \cite{latex:source}; the essential point here is that, for example,
% in the second case, the arguments \texttt{Lem}, \texttt{Lemma} and
% \texttt{section} are passed over to the macro |\@xnthm|.
%
% We inspect this case first because the others present fewer problems,
% and thus are easily derived from this one.
%
% \begin{macro}{\@xnthm}
%    For our example arguments, the macro |\@xnthm| must fulfill
%    the following:
%    \begin{itemize}
%       \item
%          Define a new \LaTeX-counter `Lem'
%       \item
%          reset this counter within a |\section|
%       \item
%          define the macro |\theLem|
%       \item
%          define the environment macros |\Lem| and
%          |\endLem| using the current
%          |\theoremstyle| and |\theorem@bodyfont|.
%    \end{itemize}
%    Obviously, all this should happen only if the first argument
%    of |\@xnthm| (i.e.\ \texttt{Lem} in our example) is chosen so
%    as not to conflict with any previously defined commands or
%    environments.  This test is performed by the \LaTeX{} macro
%    |\@ifdefinable|.
%    \begin{macrocode}
\gdef\@xnthm#1#2[#3]{\expandafter\@ifdefinable\csname #1\endcsname
%    \end{macrocode}
%    Therefore, the first argument of |\@ifdefinable| is the
%    expansion (in the example, |\Lem|) of
%    |\csname||#1\endcsname|.  The second argument is
%    executed only if the test has been completed successfully.
%    \begin{macrocode}
   {%
%    \end{macrocode}
%    Now we define the new counter. The names of the \LaTeX{}
%    macros employed should speak for themselves:
% \changes{v2.2b}{1995/10/07}{Use \cs{@newctr}}
% \changes{v2.2c}{1995/11/23}{Use \cs{@newctr} properly.}
%    \begin{macrocode}
    \@definecounter{#1}\@newctr{#1}[#3]%
%    \end{macrocode}
%    Using |\@newctr| will give a proper error message if the counter
%    in |#3| is not defined.
%    In defining `|\theLem|' we must generate the desired macro
%    name by use of |\expandafter| and |\csname|.
%    \begin{macrocode}
    \expandafter\xdef\csname the#1\endcsname
%    \end{macrocode}
%    An |\xdef| is used in order to make the definition global,
%    and to ensure that it contains the replacement texts of
%    |\@thmcountersep| and |\@thmcounter|.\footnote{These
%    two macros can be defined by the document style.  Their default
%    values produce a `\texttt{.}' as separation and an arabic
%    representation of the number.} However, not everything should be
%    expanded. For example, it saves space to use |\thesection|
%    instead of its---at times---lengthy expansion.
%    \begin{macrocode}
      {\expandafter \noexpand \csname the#3\endcsname
       \@thmcountersep \@thmcounter{#1}}%
%    \end{macrocode}
%    Thus with the defaults of \LaTeX{}, |\theLem| would be
%    replaced by the command sequence |\thesection.\arabic{Lem}|.
%
%    We will now look at the definition of the macro which is executed
%    at the beginning of the actual environment (in our example this
%    macro is |\Lem|). It should be noted that we use an
%    ``|\expandafter| trick'' to expand only certain parts of the
%    replacement text at the time of the definition.
%    \begin{macrocode}
    \def\@tempa{\global\@namedef{#1}}%
    \expandafter \@tempa \expandafter{%
%    \end{macrocode}
%    First, the macro that contains the current definitions of
%    |\@begintheorem| and |\@opargtheorem| should be called
%    up.  The name of this macro---as is already known---has the form
%    |\th@|\meta{theorem\ style}; therefore, it must be called by
%    \begin{macrocode}
      \csname th@\the \theorem@style
%    \end{macrocode}
%    In addition the default theorem font should be changeable, i.e.\
%    we have to insert the contents of |\theorem@bodyfont|. For
%    that reason, we expand even further, beyond |\endcsname|,
%    and thus insert the contents of the \textsf{token} register
%    |\theorem@bodyfont| in the replacement text.
%    \begin{macrocode}
            \expandafter \endcsname \the \theorem@bodyfont
%    \end{macrocode}
%    Now it is time to call the macro |\@thm| which takes over
%    the further processing.  It has two arguments: the current
%    counter name (in our example, \texttt{Lem}), and the text of the
%    label (in our example, \texttt{Lemma}).
%    \begin{macrocode}
     \@thm{#1}{#2}}%
%    \end{macrocode}
%    With this, the `sub-definition' is complete.  The macro
%    |\@endtheorem| ends a theorem environment and is, so far, nothing
%    but an |\endtrivlist|.  (Hence it is defined globally, and not
%    within the theorem styles.\footnote{This has to be changed as
%    soon as theorem styles that change \texttt{\bslash @endtheorem}
%    exist.  In such a case, all existing styles must be changed as
%    well since they will have to reset the macro.}) Therefore, we can
%    set it equivalent to the macro that ends the theorem set (in our
%    example, |\endLem|).  However, if some day theorem styles exist
%    that do change |\@endtheorem|, we would have to use the
%    commented-out line instead.
%    \begin{macrocode}
    \global \expandafter \let \csname end#1\endcsname \@endtheorem
%   \global\@namedef{end#1}{\@endtheorem}%
%    \end{macrocode}
%    With these commands all the required definitions are employed,
%    unless the test |\@ifdefinable| has failed.  Therefore, we
%    end the second argument of this macro and with it the definition
%    of |\@xnthm|.
%    \begin{macrocode}
   }}
%    \end{macrocode}
% \end{macro}
%
% \begin{macro}{\@ynthm}
%    The definition of |\@ynthm| is completely analogous. In
%    this case the new counter that is defined is not reset within
%    another counter; thus the definition of |\the...|
%    is simplified:
%    \begin{macrocode}
\gdef\@ynthm#1#2{\expandafter\@ifdefinable\csname #1\endcsname
   {\@definecounter{#1}%
    \expandafter\xdef\csname the#1\endcsname{\@thmcounter{#1}}%
%    \end{macrocode}
%    The rest of the definition corresponds literally to that of
%    |\@xnthm|:
%    \begin{macrocode}
    \def\@tempa{\global\@namedef{#1}}\expandafter \@tempa
     \expandafter{\csname th@\the \theorem@style \expandafter
     \endcsname \the\theorem@bodyfont \@thm{#1}{#2}}%
    \global \expandafter \let \csname end#1\endcsname \@endtheorem}}
%    \end{macrocode}
% \end{macro}
%
% \begin{macro}{\@othm}
%    The definition of |\@othm| does not contain anything new.
%    \begin{macrocode}
\gdef\@othm#1[#2]#3{%
%    \end{macrocode}
% \changes{v2.2b}{95/10/07}{Build in check for counter defined}
%    We do not define a new counter but instead use one that has
%    already been defined.  Thus the only definition we need is that
%    of this pseudo-counter (i.e.\ |\the|\meta{env.\ name}).
%    First we check if |#2| corresponds to a known counter name.
%    \begin{macrocode}
  \expandafter\ifx\csname c@#2\endcsname\relax
   \@nocounterr{#2}%
  \else
   \expandafter\@ifdefinable\csname #1\endcsname
   {\expandafter \xdef \csname the#1\endcsname
     {\expandafter \noexpand \csname the#2\endcsname}%
%    \end{macrocode}
%    All other parts of the definition can be adopted from
%    |\@xnthm|. We have to remember, though, that in this case
%    the name of the current counter and the theorem label have moved
%    to the second and third arguments.
%    \begin{macrocode}
    \def\@tempa{\global\@namedef{#1}}\expandafter \@tempa
     \expandafter{\csname th@\the \theorem@style \expandafter
     \endcsname \the\theorem@bodyfont \@thm{#2}{#3}}%
    \global \expandafter \let \csname end#1\endcsname \@endtheorem}%
  \fi}
%    \end{macrocode}
% \end{macro}
%
% \subsection{Macros that are employed in a {\envfont theorem}
%    environment}
%
% \begin{macro}{\@thm}
%    The macro |\@thm| has to increase the current counter.  Then,
%    depending on whether the environment has (or does not have) an
%    optional argument, it has to branch into either |\@begintheorem|
%    or |\@opargtheorem|.
%    \begin{macrocode}
\gdef\@thm#1#2{\refstepcounter{#1}%
%    \end{macrocode}
%    Now we start a {\envfont trivlist} environment, and give
%    |\@topsep| and |\@topsepadd| the values of the \textsf{skip}
%    registers |\theorempreskipamount| and |\theorempostskipamount|.
%    The value in |\@topsep| is the vertical space that is inserted by
%    the first (and only) |\item| in our |\trivlist| whilst
%    |\@topsepadd| is inserted by |\@endparenv| at the end of that
%    {\envfont trivlist} environment.  By using these registers, we
%    obtain the desired space around a {\envfont theorem} environment.
%    \begin{macrocode}
   \trivlist
   \@topsep \theorempreskipamount               % used by first \item
   \@topsepadd \theorempostskipamount           % used by \@endparenv
%    \end{macrocode}
%    Now we have to test whether an optional argument has been given.
% \changes{v2.0d}{89/07/18}{Forgotten percent added.}
%    \begin{macrocode}
   \@ifnextchar [%
%    \end{macrocode}
%    If there is an optional argument, we will call |\@ythm|, and
%    move the arguments read back into the input stream.
%    \begin{macrocode}
   {\@ythm{#1}{#2}}%
%    \end{macrocode}
%    If not, we call |\@begintheorem|. Its first argument is the
%    name of the theorem set (hence the second argument of
%    |\@thm|). Its second argument is the macro that produces the
%    current number.
%    \begin{macrocode}
   {\@begintheorem{#2}{\csname the#1\endcsname}\ignorespaces}}
%    \end{macrocode}
% \end{macro}
%
% \begin{macro}{\@xthm}
% \begin{macro}{\@ythm}
%    Both these macros were originally called by |\@thm|.  We do
%    not need |\@xthm| anymore, hence we reset it to
%    |\relax|. The definition of |\@ythm| has not changed at
%    all from its definition in \LaTeX. In order to make the macros
%    easier to understand, we will
%    nevertheless present it (commented out).
%    \begin{macrocode}
\global\let\@xthm\relax
% \def\@ythm#1#2[#3]{\@opargbegintheorem{#2}{\csname
%           the#1\endcsname}{#3}\ignorespaces}
%</package>
%    \end{macrocode}
%    The primitive |\ignorespaces| in |\@ythm| and
%    |\@thm| is needed to remove the spaces between the
%    |\begin{...}| and the actual text.
% \end{macro}
% \end{macro}
%
%
% \subsection{Definition of the theorem substyles}
%
% As already pointed out, the theorem substyles, defined below, are
% only loaded when necessary.  Note that all these substyles, except
% \texttt{plain}, have |\slshape| as the default body font.
%
% \subsubsection{The \texttt{plain} style}
%
% \label{dok:th@plain} As the following macros use |@|, we have
% to locally set the |\catcode| of this symbol to ``letter''.
% This happens within a group, so that we do not have to worry about
% which |\catcode| that symbol had before.
%    \begin{macrocode}
%<*thp>
\begingroup \makeatletter
%    \end{macrocode}
% Since we are now within a group, we must make all definitions
% globally.  First we make sure that \texttt{theorem.sty} is loaded.
% This will allow us to use this file as a document style option
% without having to call \texttt{theorem} itself as an option.  At the
% same time, we assure that at least version 2 is loaded, since
% |\theorem@style| was not defined in earlier versions.
%    \begin{macrocode}
\@ifundefined{theorem@style}{\input{theorem.sty}}{}
\ProvidesFile{thp.sty}
             [\FMithmInfo]
%    \end{macrocode}
%
%
% \begin{macro}{\th@plain}
% \begin{macro}{\@begintheorem}
% \begin{macro}{\@opargbegintheorem}
%    |\theoremstyle{plain}| corresponds to the original
%    definition, except that the distances to the surrounding text are
%    determined by the parameters |\theorempreskipamount| and
%    |\theorempostskipamount|.
%    First we set the default body font.
%    \begin{macrocode}
\gdef\th@plain{\normalfont\itshape
%    \end{macrocode}
%    Then we define |\@begintheorem| and
%    |\@opargbegintheorem|. These two macros define how the
%    header of a theorem is typeset.  |\@opargbegintheorem| will
%    be called if a \textsf{theorem} environment with an optional
%    argument is encountered; otherwise, the header is constructed by
%    calling |\@begintheorem|.  If one of these macros is
%    executed, we are within a \textsf{trivlist} environment started by
%    |\@thm|.  So the theorem header is produced with an
%    |\item| command.
%
%    Instead of specifying the header font directly, all standard
%    theorem styles use the |\theorem@headerfont| macro to allow
%    customization.  The extra space (|\labelsep|) is necessary
%    because of problems in the \textsf{trivlist} environment.
%    \begin{macrocode}
  \def\@begintheorem##1##2{%
        \item[\hskip\labelsep \theorem@headerfont ##1\ ##2]}%
%    \end{macrocode}
%    The definition of |\@opargbegintheorem| is completely
%    analogous. The only difference is the fact that there exists a
%    third argument (which is the optional parameter of the
%    environment and contains additional information about the
%    theorem).  Customarily we enclose it in parentheses.
%    \begin{macrocode}
\def\@opargbegintheorem##1##2##3{%
   \item[\hskip\labelsep \theorem@headerfont ##1\ ##2\ (##3)]}}
%    \end{macrocode}
% \end{macro}
% \end{macro}
% \end{macro}
%
% \noindent
% We conclude with an |\endgroup| to restore the
% |\catcode| of |@|.
%    \begin{macrocode}
\endgroup
%</thp>
%    \end{macrocode}
%
%
% \subsubsection{The \texttt{break} style}
%
% This style option is stored in the file |thb.sty|.
% For the next two lines see the documentation for |\th@plain| on
% page~\pageref{dok:th@plain}.
%    \begin{macrocode}
%<*thb>
\begingroup \makeatletter
\@ifundefined{theorem@style}{\input{theorem.sty}}{}
\ProvidesFile{thb.sty}
             [\FMithmInfo]
%    \end{macrocode}
%
% \begin{macro}{\th@break}
%    |\theoremstyle{break}| produces a line break after the name
%    of the theorem.  The font is |\slshape|.  Hence, we define
%    |\th@break| as follows:
%    \begin{macrocode}
\gdef\th@break{\normalfont\slshape
  \def\@begintheorem##1##2{\item[%
%    \end{macrocode}
%    We run into the following problem: it is not possible to create
%    the header with |\item[|\meta{title}|]| and then start
%    a new line by, for example, |\mbox{}\\|. Such a definition
%    will fail whenever a list environment follows immediately.  With
%    the above construction, the |\mbox{}| causes the switch
%    |@inlabel| (cf.\ definition of |\list| and
%    |\trivlist| in \cite{latex:source}) to be set to \texttt{false}
%    and so the following list will insert additional vertical
%    space (|\topskip|). This is quite annoying. Therefore, we
%    create the line break within the |\item|. In order to ensure
%    that the text will begin at the proper position in the following
%    line, we simply pretend that the label does not take any
%    room.\footnote{This will lead to problems whenever very high
%    symbols occurring in the line tower into the
%    heading. So, something else has to be done here sometime.}
%    \begin{macrocode}
       \rlap{\vbox{\hbox{\hskip \labelsep\theorem@headerfont ##1\ ##2}%
                   \hbox{\strut}}}]}%
%    \end{macrocode}
%    Again, the definition of |\@opargbegintheorem| is
%    completely analogous.
%    \begin{macrocode}
\def\@opargbegintheorem##1##2##3{%
  \item[\rlap{\vbox{\hbox{\hskip \labelsep \theorem@headerfont
                     ##1\ ##2\ (##3)}%
                    \hbox{\strut}}}]}}
\endgroup
%</thb>
%    \end{macrocode}
% \end{macro}
%
%
% \subsubsection{The \texttt{changebreak} style}
%
%    \begin{macrocode}
% This style option is stored in the file |thcb.sty|.
%    \begin{macrocode}
%<*thcb>
\begingroup \makeatletter
\@ifundefined{theorem@style}{\input{theorem.sty}}{}
\ProvidesFile{thcb.sty}
             [\FMithmInfo]
%    \end{macrocode}
%
% \begin{macro}{\th@changebreak}
%    The \texttt{change-break} theorem style is like \texttt{break}
%    but with interchange of theorem name and theorem number.  Thus we
%    define |\th@changebreak| as follows:
%    \begin{macrocode}
\gdef\th@changebreak{\normalfont\slshape
  \def\@begintheorem##1##2{\item
       [\rlap{\vbox{\hbox{\hskip\labelsep \theorem@headerfont ##2\ ##1}%
                    \hbox{\strut}}}]}%
\def\@opargbegintheorem##1##2##3{%
  \item[\rlap{\vbox{\hbox{\hskip\labelsep \theorem@headerfont
                     ##2\ ##1\ (##3)}%
                    \hbox{\strut}}}]}}
\endgroup
%</thcb>
%    \end{macrocode}
% \end{macro}
%
%
% \subsubsection{The \texttt{change} style}
%
% This style option is stored in the file |thc.sty|.
%    \begin{macrocode}
%<*thc>
\begingroup \makeatletter
\@ifundefined{theorem@style}{\input{theorem.sty}}{}
\ProvidesFile{thc.sty}
             [\FMithmInfo]
%    \end{macrocode}
%
% \begin{macro}{\th@change}
%    The \texttt{change} theorem style corresponds to the
%    \texttt{change break} style without a linebreak after the header.
%    To say it in another way, it's the same as the \texttt{plain}
%    style but with number and name interchanged and |\slshape| as the
%    default font.
%    \begin{macrocode}
\gdef\th@change{\normalfont\slshape
  \def\@begintheorem##1##2{\item
       [\hskip\labelsep \theorem@headerfont ##2\ ##1]}%
\def\@opargbegintheorem##1##2##3{%
  \item[\hskip\labelsep \theorem@headerfont ##2\ ##1\ (##3)]}}
\endgroup
%</thc>
%    \end{macrocode}
% \end{macro}
%
%
% \subsubsection{The \texttt{marginbreak} style}
%
% This style option is the one used most often at Mainz.  It is saved
% in the file |thmb.sty|.
%    \begin{macrocode}
%<*thmb>
\begingroup \makeatletter
\@ifundefined{theorem@style}{\input{theorem.sty}}{}
\ProvidesFile{thmb.sty}
             [\FMithmInfo]
%    \end{macrocode}
%
% \begin{macro}{\th@marginbreak}
%    The \texttt{margin break} style is nearly the same as the
%    \texttt{change break} style. The only difference is the placement
%    of the theorem number.  We use |\llap| to place it in the left
%    margin.
%
%    In this style |\labelsep| denotes the separation between the
%    number and the text.
%    \begin{macrocode}
\gdef\th@marginbreak{\normalfont\slshape
  \def\@begintheorem##1##2{\item
       [\rlap{\vbox{\theorem@headerfont
                    \hbox{\llap{##2}\hskip\labelsep ##1}%
                    \hbox{\strut}}}]}%
\def\@opargbegintheorem##1##2##3{%
  \item[\rlap{\vbox{\theorem@headerfont
                    \hbox{\llap{##2}\hskip\labelsep ##1\ (##3)}%
                    \hbox{\strut}}}]}}
\endgroup
%</thmb>
%    \end{macrocode}
% \changes{v2.1c}{90/02/17}{`endgroup outside macrocode env.}
% \end{macro}
%
%
%
% \subsubsection{The \texttt{margin} style}
%
% This style option is stored in the file |thm.sty|.
%    \begin{macrocode}
%<*thm>
\begingroup \makeatletter
\@ifundefined{theorem@style}{\input{theorem.sty}}{}
\ProvidesFile{thm.sty}
             [\FMithmInfo]
%    \end{macrocode}
%
% \begin{macro}{\th@margin}
%    Again this is only a variant of the theorem styles described
%    above without any new ideas.
%    \begin{macrocode}
\gdef\th@margin{\normalfont\slshape
  \def\@begintheorem##1##2{\item
       [\theorem@headerfont \llap{##2}\hskip\labelsep ##1]}%
\def\@opargbegintheorem##1##2##3{%
  \item[\theorem@headerfont \llap{##2}\hskip\labelsep ##1\ (##3)]}}
\endgroup
%</thm>
%    \end{macrocode}
% \end{macro}
%
%
%
%
% \subsection{Final Definitions}
%
% \begin{macro}{\theorempreskipamount}
% \begin{macro}{\theorempostskipamount}
%    The \textsf{skip} parameters that regulate the vertical empty space
%    before and after the {\envfont theorem} environment have
%    to be allocated as well.
% \changes{v2.0f}{89/07/20}{Unnecessary `global' removed.}
%    \begin{macrocode}
%<*package>
\newskip\theorempreskipamount
\newskip\theorempostskipamount
%    \end{macrocode}
%    Since we have used the same values for all theorem sets, we now can
%    assign them.
%    \begin{macrocode}
\global\setlength\theorempreskipamount{12pt plus 5pt minus 3pt}
\global\setlength\theorempostskipamount{8pt plus 3pt minus 1.5pt}
%    \end{macrocode}
% \end{macro}
% \end{macro}
%
%
% \begin{macro}{\@endtheorem}
%    The same holds for the macro |\@endtheorem|, which ends a
%    {\envfont theorem} environment. Since it is the same for all
%    theorem sets, it is removed from the macros |\th@|\meta{style}.
%    It simply ends the {\envfont trivlist} environment, which was
%    begun in |\@thm|.
%    \begin{macrocode}
\global\let\@endtheorem=\endtrivlist
%    \end{macrocode}
% \end{macro}
%
%
%
% \begin{macro}{\@preamblecmds}
%    All macros defined above are to be used only in the preamble.
%    Therefore, we insert them in |\@preamblecmds| which will disable
%    them at begin document. This is done by the internal \LaTeXe{}
%    command |\@onlypreamble|.
% \changes{v2.0g}{89/09/19}{`theoremheaderfont and `theorembodyfont
%    added.}
% \changes{v2.2a}{94/02/03}{Use \LaTeXe{} interface}
%    \begin{macrocode}
\@onlypreamble\@xnthm
\@onlypreamble\@ynthm
\@onlypreamble\@othm
\@onlypreamble\newtheorem
\@onlypreamble\theoremstyle
\@onlypreamble\theorembodyfont
\@onlypreamble\theoremheaderfont
%    \end{macrocode}
% \end{macro}
%
% Finally we declare the \texttt{plain} theorem style to be the default.
% \changes{v2.1b}{90/06/04}{Fixed missing `item by declaring default
%    style.}
%    \begin{macrocode}
\theoremstyle{plain}
%</package>
%    \end{macrocode}
%
% \Finale

\endinput
}{}
\ProvidesFile{thmb.sty}
             [\FMithmInfo]
%    \end{macrocode}
%
% \begin{macro}{\th@marginbreak}
%    The \texttt{margin break} style is nearly the same as the
%    \texttt{change break} style. The only difference is the placement
%    of the theorem number.  We use |\llap| to place it in the left
%    margin.
%
%    In this style |\labelsep| denotes the separation between the
%    number and the text.
%    \begin{macrocode}
\gdef\th@marginbreak{\normalfont\slshape
  \def\@begintheorem##1##2{\item
       [\rlap{\vbox{\theorem@headerfont
                    \hbox{\llap{##2}\hskip\labelsep ##1}%
                    \hbox{\strut}}}]}%
\def\@opargbegintheorem##1##2##3{%
  \item[\rlap{\vbox{\theorem@headerfont
                    \hbox{\llap{##2}\hskip\labelsep ##1\ (##3)}%
                    \hbox{\strut}}}]}}
\endgroup
%</thmb>
%    \end{macrocode}
% \changes{v2.1c}{90/02/17}{`endgroup outside macrocode env.}
% \end{macro}
%
%
%
% \subsubsection{The \texttt{margin} style}
%
% This style option is stored in the file |thm.sty|.
%    \begin{macrocode}
%<*thm>
\begingroup \makeatletter
\@ifundefined{theorem@style}{% \iffalse meta-comment
%
% Copyright (C) 1993-2024
%
% The LaTeX Project and any individual authors listed elsewhere
% in this file.
%
% This file is part of the Standard LaTeX `Tools Bundle'.
% -------------------------------------------------------
%
% It may be distributed and/or modified under the
% conditions of the LaTeX Project Public License, either version 1.3c
% of this license or (at your option) any later version.
% The latest version of this license is in
%    https://www.latex-project.org/lppl.txt
% and version 1.3c or later is part of all distributions of LaTeX
% version 2005/12/01 or later.
%
% The list of all files belonging to the LaTeX `Tools Bundle' is
% given in the file `manifest.txt'.
%
% \fi
%
%%
%
%
\def\FMithmInfo{2023/07/05 v2.2c Theorem extension package (FMi)}
%
% \ProvidesFile{theorem.dtx}[\FMithmInfo]
%\iffalse   % this is a METACOMMENT !
%%
%% Package `theorem' to use with LaTeX2e
%% Copyright (C) 1989-1995 Frank Mittelbach, all rights reserved.
%<+package>\@ifundefined{theorem@style}{}{\endinput}
%<+package>\NeedsTeXFormat{LaTeX2e}[1995/06/01]
%<+package>\ProvidesPackage{theorem}
%<+package>         [\FMithmInfo]
%
% \fi
%
% \changes{v1.0}{87/05/07}{First draft.}
% \changes{v1.0a}{87/05/07}{`thistheoremstyle' added.}
% \changes{v1.1a}{87/07/29}{New concept: redefine the internal
%                          \protect\LaTeX{} commands `@xnthm' etc.}
% \changes{v1.1b}{87/07/31}{`theoremfont' used to specify font.}
% \changes{v1.2a}{88/01/08}{`@endtheorem' now globally defined.}
% \changes{v1.2b}{88/01/11}{`newtheorem' only allowed in the preamble.}
% \changes{v1.3a}{88/02/14}{`theoremfont' now a toks register.}
%
% \changes{v2.0a}{88/11/16}{New implementation and new documentation.}
% \changes{v2.0a}{88/11/16}{No layout changes but internal modifications
%                           old substyles won't work.}
% \changes{v2.0b}{89/01/30}{`global' used for all definitions.}
% \changes{v2.0c}{89/06/15}{Documentation translated to English.
%                            Thanks to LLL and Nina}
% \changes{v2.1a}{90/02/15}{Support for new font selection.}
% \changes{v2.2a}{94/02/03}{Cleanup for LaTeX2e}
%
% \DoNotIndex{\@endparenv,\@ifdefinable,\@ifnextchar,\@ifundefined}
% \DoNotIndex{\@namedef,\@nameuse,\@spaces,\@tempa,\@warning}
% \DoNotIndex{\ }
% \DoNotIndex{\begingroup,\bfseries}
% \DoNotIndex{\csname}
% \DoNotIndex{\def,\do,\docdate}
% \DoNotIndex{\endcsname,\endgroup,\endinput,\endtrivlist,\expandafter}
% \DoNotIndex{\filedate,\fileversion}
% \DoNotIndex{\gdef,\global}
% \DoNotIndex{\hbox,\hskip}
% \DoNotIndex{\ignorespaces,\input,\itshape,\item}
% \DoNotIndex{\labelsep,\let,\llap}
% \DoNotIndex{\makeatletter}
% \DoNotIndex{\noexpand}
% \DoNotIndex{\refstepcounter,\relax,\rlap}
% \DoNotIndex{\setlength,\slshape,\space,\string,\strut}
% \DoNotIndex{\the,\trivlist,\typeout}
% \DoNotIndex{\vbox}
% \DoNotIndex{\xdef}
%
% \def\deflabel#1{\ttfamily #1\hfill}
% \def\deflist#1{\begin{list}{}{\settowidth\labelwidth{\ttfamily #1}%
%                               \setlength\leftmargin\labelwidth
%                               \addtolength\leftmargin\labelsep
%                               \let\makelabel\deflabel}}
% \def\enddeflist{\end{list}}
% \def\envfont{\normalfont\ttfamily}
%
% \MakeShortVerb{\|}
%
% \newcommand{\TUB}{{\sl TUGboat\/}}
%
% \GetFileInfo{theorem.dtx}
% \title{An Extension of the
%        \LaTeX{} theorem environment\thanks{This file
%        has version number \fileversion, last
%        revised \filedate.}}
% \author{Frank Mittelbach}
%
% \MaintainedByLaTeXTeam{tools}
% \maketitle
%
% \begin{abstract}
%    The macros described in this paper yield an extension of the
%    \LaTeX{} theorem mechanism. It is designed to satisfy the
%    different requirements of various journals. Thus, the
%    layout of the ``theorems'' can be manipulated by determining a
%    ``style''. This article describes not only the use, but
%    also the definition, of the necessary macros.
% \end{abstract}
%
% \section*{Preface to version 2.2}
%
% For \LaTeXe{} this package did not need any fundamental changes. I
% only modified the messages generated so that theorem layout styles
% will show up with the |\listfiles| command and cleaned the section
% on the New Font Selection Scheme since this is now included in
% \LaTeX.
%
% \section*{Preface to version 2.1}
%
% This version is identical to 2.0g described in \TUB\ 10\#3 except
% for some internal defaults which are now set depending on the used
% font selection scheme.
%
% This was done to avoid unpleasant surprises if the new font selection
% scheme is in force. For further details see
% section~\ref{sec:newfonts} and \cite{bk:GMS94}.
%
% \section{Introduction}
%
% For our purposes here, ``theorems'' are labelled enunciations,
% often set off from the main text by extra space and a font change.
% Theorems, corollaries, conjectures, definitions, and remarks
% are all instances of ``theorems''.  The ``header'' of these structures
% is composed of a label (such as \textsc{Theorem} or \textsc{Remark})
% and a number which serializes an item in the sequence of items
% with the same label.
%
% Shortly after the introduction of \LaTeX{} at the Fachbereich
% Mathematik in Mainz, the desire to manipulate the layout of
% ``theorems'' arose.  In Mainz, the following two conventions came
% into general use:
% \begin{enumerate}
%   \item The number of the theorem is shown in the margin.
%   \item There is a line break at
%         the end of the theorem header.
% \end{enumerate}
%
% Additionally, some journals require different formats which depend on
% the ``sort of theorem'': e.g.\ often remarks and definitions are set
% in |\upshape|, while |\itshape| is employed for main theorems.
%
% Confronted with these requirements, a theorem environment was
% developed in Mainz which allows separate determination of the
% layout of the ``theorems sets'', comparable to |\pagestyle|.
%
%
% \section{The user interface}
% \subsection{Defining new theorem sets}
%
% \DescribeMacro\newtheorem
% As in the original \LaTeX{} version, the command |\newtheorem|
% defines a new ``theorem set'' or ``theorem-like structure''.
% Two required arguments name the new environment and give the text
% to be typeset with each instance of the new ``set'', while
% an optional argument determines how
% the ``set'' is enumerated:
% \begin{description}
%    \item[\ttfamily \bslash newtheorem\{foo\}\{bar\}]
%       The theorem set {\envfont foo} (whose name is \texttt{bar})
%       uses its own counter.
%    \item[\ttfamily \bslash newtheorem\{foo2\}{[foo]}\{bar2\}]
%       The theorem set {\envfont foo2} (printed name \texttt{bar2})
%       uses the same counter as the theorem set \texttt{foo}.
%    \item[\ttfamily \bslash newtheorem\{foo3\}\{bar3\}{[section]}]
%       The theorem set {\envfont foo3} (printed name \texttt{bar3}) is
%       enumerated within the counter \texttt{section}, i.e.\ with every
%       new |\section| the enumeration begins again with 1, and
%       the enumeration is composed from the section-number and the
%       theorem counter itself.
% \end{description}
%
% \DescribeMacro\theoremstyle
% Additionally, the command |\theoremstyle| can define the layout
% of various, or all, theorem sets.  It should be noted that any
% theorem set defined by |\newtheorem| is typeset in the
% |\theoremstyle| that is current at the time of the definition.
% Thus, the following
% \begin{verbatim}
%    \theoremstyle{break}        \newtheorem{Cor}{Corollary}
%    \theoremstyle{plain}        \newtheorem{Exa}{Example}[section]
%\end{verbatim}                             ^^A No blank at line start
% leads to the result that the set {\envfont Cor} is formatted in the
% style \texttt{break}, while the set {\envfont Exa} and all the
% following ones are formatted in the style \texttt{plain}, unless
% another |\theoremstyle| follows.  Since the definitions installed by
% |\newtheorem| are global, one also can limit |\theoremstyle| locally
% by grouping braces.
%
% \DescribeMacro\theorembodyfont
% The choice of the font for the theorem body is completely
% independent of the chosen |\theoremstyle|; this has proven to
% be very advantageous.  For example,
% \begin{verbatim}
%   {\theorembodyfont{\upshape}          \newtheorem{Rem}{Remark}}
%\end{verbatim}
% defines a theorem set {\envfont Rem}, which will be set in
% |\upshape| in the current layout (which in our example is
% \texttt{plain}).  As with |\theoremstyle|, the |\theorembodyfont|
% chosen is that current at the time of |\newtheorem|.  If
% |\theorembodyfont| is not specified or one defines
% |\theorembodyfont{}|, then the font used will be that defined by the
% |\theoremstyle|.
%
% \DescribeMacro\theoremheaderfont It is also possible to customize
% the font used for the theorem headers.  This is, however, a global
% declaration, and therefore there should be at most one
% |\theoremheaderfont| declaration in the preamble.\footnote{If it is
% actually necessary to have different header fonts, one has to define
% new theorem styles (substituting the desired font) or specify the
% information directly in the \texttt{\bslash newtheorem} declaration
% (the unclean variant).}
%
% \DescribeMacro\theorempreskipamount
% \DescribeMacro\theorempostskipamount Two additional parameters
% affect the vertical space around the theorem environments:
% |\theorempreskipamount| and |\theorempostskipamount| define,
% respectively, the spacing before and after such an environment.
% These parameters apply for all theorem sets and can be manipulated
% with the ordinary length macros.  They are rubber lengths
% (`\textsf{skips}'), and therefore can contain \texttt{plus} and
% \texttt{minus} parts.
%
% Since the definition of theorem sets should---most sensibly---be
% placed in the preamble, we only allow installation there.
% It is therefore possible to release the memory used here after
% |\begin{document}|, in order to make room for other
% applications.
%
%
% \subsection{Existing theorem styles}
%
% The following theorem styles exist to date:
% \begin{deflist}{marginbreak}
%    \item[plain]
%       This theorem style emulates the original \LaTeX{} definition,
%       except that additionally the parameters
%       |\theorem...skipamount| are used.
%    \item[break]
%       In this style, the theorem header is followed by a line
%       break.
%    \item[marginbreak]
%       The theorem number is set in the margin, and there is a line
%       break as in \texttt{break}.
%    \item[changebreak]
%       Like \texttt{break}, but with header number and text
%       interchanged.
%    \item[change]
%       Header number and text are interchanged, without a line break.
%    \item[margin]
%       The number is set in the left margin, without a line break.
% \end{deflist}
% All styles (except \texttt{plain}) select |\slshape| as the default
% |\theorembodyfont|.
%
%
% \subsection{Examples}
%
% Given the above theorem sets {\envfont Cor}, {\envfont Exa}
% and {\envfont Rem},
% suppose that the preamble also contains the declarations:
% \begin{verbatim}
%    \theoremstyle{marginbreak}   \newtheorem{Lem}[Cor]{Lemma}
%    \theoremstyle{change}
%    \theorembodyfont{\itshape}        \newtheorem{Def}[Cor]{Definition}
%
%    \theoremheaderfont{\scshape}
%\end{verbatim}
% Then the following are some typical examples of the typeset output
% resulting from their use.
%
% \begin{Cor}
%    This is a sentence typeset in the theorem environment {\envfont
%    Cor}.
% \end{Cor}
%
% \begin{Exa}
%    This is a sentence typeset in the theorem environment {\envfont
%    Exa}.
% \end{Exa}
%
% \begin{Rem}
%    This is a sentence typeset in the theorem environment {\envfont
%    Rem}.
% \end{Rem}
%
% \begin{Lem}[Ben User]
%    This is a sentence typeset in the theorem environment {\envfont
%    Lem}.
% \end{Lem}
%
% \begin{Def}[Very impressive Definition]
%    This is a sentence typeset in the theorem environment {\envfont
%    Def}.
% \end{Def}
%
% The last two examples show the effect of the optional argument to
% a theorem environment (it is the text typeset in parentheses).
%
%
% \section{Special Considerations}
% \label{sec:newfonts}
%
% Theoremheader and body are implemented as a unit. This means that
% the |\theoremheaderfont|  will inherit characteristics of the
% |\theorembodyfont| in \LaTeXe. Thus, if for example
% |\theorembodyfont| is |\itshape| and |\theoremheaderfont|
% is |\bfseries| the font selected for the header will have the
% characteristics `bold extended italic'. If this is not desired one
% should set the |\theoremheaderfont| to something like
% \begin{verbatim}
%      \theoremheaderfont{\normalfont\bfseries}
%\end{verbatim}
% i.e.\ supplying all necessary font information explicitly.
%
%
%
% \section{Acknowledgements}
%
% The publication of this set of macros was only possible with the
% help of Christina Busse (translating the manuscript into English),
% Joachim Pense (playing the r\^ole of typist), Chris Rowley
% (looking everything over) and many others providing useful
% suggestions.
%
% \MaybeStop{%
%    \begin{thebibliography}{9}
%       \bibitem{bk:GMS94} \textsc{M.~Goossens}, \textsc{F.~Mittelbach}
%         and \textsc{A.~Samarin}.
%         \newblock The \LaTeX{} Companion.
%         \newblock
%         Addison-Wesley, Reading, Massachusetts, 1994.
%       \bibitem{latex:source} \textsc{Lamport, Leslie.} \newblock
%                               \texttt{latex.tex}, version 2.09, date
%                               Feb. 1990.
%    \end{thebibliography}
%    \PrintIndex
%    \PrintChanges}
%
%
% \section{The documentation driver file}
%
% The next bit of code contains the documentation driver file for
% \TeX{}, i.e., the file that will produce the documentation you are
% currently reading. It will be extracted from this file by the
% \texttt{docstrip} program. Since it is the first code in the file
% one can alternatively process this file directly with \LaTeXe{} to
% obtain the documentation.
%
%    \begin{macrocode}
%<*driver>
\documentclass{ltxdoc}

\usepackage{theorem}

%<+driver>% The next few lines define theorem sets which are used
%<+driver>% in the example section of the documentation.

\theoremstyle{break}       \newtheorem{Cor}{Corollary}
\theoremstyle{plain}       \newtheorem{Exa}{Example}[section]
{\theorembodyfont{\upshape}\newtheorem{Rem}{Remark}}
\theoremstyle{marginbreak} \newtheorem{Lem}[Cor]{Lemma}
\theoremstyle{change}
\theorembodyfont{\itshape} \newtheorem{Def}[Cor]{Definition}

\theoremheaderfont{\scshape}

\RecordChanges

\begin{document}
   \DocInput{theorem.dtx}
\end{document}
%</driver>
%    \end{macrocode}
%
% \section{Definition of the Macros}
%
% If the file has been loaded before, we abort immediately. If not the
% package announces itself (this is actually done at the very top of
% the file---the way it is done isn't good style so don't copy it).
% \changes{v2.0e}{89/07/19}{Spaces removed from `typeout'.}
% \changes{v2.2b}{95/11/19}{Announce moved to top of file}
%    \begin{macrocode}
%<*package>
%\@ifundefined{theorem@style}{}{\endinput}
%\def\FMithmInfo{1995/11/19 v2.2b Theorem extension package (FMi)}
%\ProvidesPackage{theorem}[\FMithmInfo]
%    \end{macrocode}
%
%
% \subsection{Definition of theorem styles and fonts}
%
% All the definitions in this file are done globally to allow
% inputting this file inside a group.
%
% \begin{macro}{\theoremstyle}
%    Before a theorem style can be installed, the chosen style must be
%    known. For that reason, we must test to see that
%    |\th@|\meta{style} is known or, more precisely, that it is
%    different from |\relax|. If the style is not known then
%    |\th@plain| is used.
%    \begin{macrocode}
\gdef\theoremstyle#1{%
   \@ifundefined{th@#1}{\@warning
          {Unknown theoremstyle `#1'. Using `plain'}%
          \theorem@style{plain}}%
%    \end{macrocode}
%    We save the theorem style to be used in the token register
%    |\theorem@style|.
%    \begin{macrocode}
      {\theorem@style{#1}}%
%    \end{macrocode}
%    Now we ``evaluate'' the theorem style: this means, we call the
%    macro |\th@|\meta{style} which will activate the relevant
%    definitions which are contained in a separate file.  This is done
%    in a group to suppress changes to the current font.  This could
%    otherwise pose problems together with the new font selection
%    scheme\footnote{When I printed the original article using the new
%    font selection scheme I ended with a document with slanted
%    typefaces (text headings and all) simply because one of the
%    theorem styles used \texttt{\bslash sl} at toplevel.} if the
%    |\th@|\meta{style} is evaluated a second time.
% \changes{v2.1a}{90/02/15}{Group added around `th@..}
% \changes{v2.1a}{90/02/15}{Reseting of `..@begintheorem removed.}
%    \begin{macrocode}
      \begingroup
        \csname th@\the\theorem@style \endcsname
      \endgroup}
%    \end{macrocode}
% \end{macro}
%
%
% \begin{macro}{\@begintheorem}
% \changes{v2.1a}{90/02/15}{Relaxed at toplevel.}
% \begin{macro}{\@opargbegintheorem}
% \changes{v2.1a}{90/02/15}{Relaxed at toplevel.}
%    We reset |\@begintheorem| and |\@opargbegintheorem| to
%    |\relax| since these commands are no longer necessary at
%    toplevel. This will save a few tokens.
%    \begin{macrocode}
\global\let\@begintheorem\relax
\global\let\@opargbegintheorem\relax
%    \end{macrocode}
% \end{macro}
% \end{macro}
%
% \begin{macro}{\theorem@style}
%    Obviously the \textsf{token} register used above has to be
%    allocated. To assure the utmost compatibility with the
%    original \LaTeX{} definition, we set the default theorem style to
%    \texttt{plain}, which implements the usual \LaTeX{} convention.
% \changes{v2.0f}{89/07/20}{Unnecessary `global' removed.}
%    \begin{macrocode}
\newtoks\theorem@style
\global\theorem@style{plain}
%    \end{macrocode}
% \end{macro}
%
% \begin{macro}{\theorembodyfont}
% \begin{macro}{\theorem@bodyfont}
%    If the |\theorembodyfont| is set by the user then it should not
%    interact with the default font set in the theorem style. When the
%    new font selection is in force this may happen if, for example,
%    the default is |\itshape| and the new |\theorembodyfont| is
%    |\sffamily|. So we add a |\reset@font| command in front of the
%    user definition.
% \changes{v2.0f}{89/07/20}{Unnecessary `global' removed.}
% \changes{v2.0g}{89/09/19}{`theoremfont to `theorembodyfont.}
% \changes{v2.1c}{92/06/22}{`reset@font added and logic changed.}
% \changes{v2.1d}{92/10/28}{Reset to empty if arg is empty}
%    \begin{macrocode}
\gdef\theorembodyfont#1{%
%    \end{macrocode}
%    We check if the argument supplied is empty and if so put nothing
%    into the |\theorem@bodyfont| token register to allow for
%    |\theorembodyfont{}| as a mean of using the default of the
%    current |\theoremstyle|.
%    \begin{macrocode}
   \def\@tempa{#1}%
   \ifx\@tempa\@empty
    \theorem@bodyfont{}%
   \else
    \theorem@bodyfont{\reset@font#1}%
   \fi
   }
\newtoks\theorem@bodyfont
\global\theorem@bodyfont{}
%    \end{macrocode}
% \end{macro}
% \end{macro}
%
%
% \begin{macro}{\theoremheaderfont}
% \changes{v2.0g}{89/09/19}{Macro added.}
%    The font for the theorem headers is handled differently because
%    this definition applies to all theorem styles.
%    \begin{macrocode}
\gdef\theoremheaderfont#1{\gdef\theorem@headerfont{#1}%
%    \end{macrocode}
%    After using the macro once it is redefined to produce an error
%    message.
%    \begin{macrocode}
       \gdef\theoremheaderfont##1{%
        \typeout{\string\theoremheaderfont\space should be used
                 only once.}}}
%    \end{macrocode}
% \end{macro}
%
% \begin{macro}{\theorem@headerfont}
% \changes{v2.0g}{89/09/19}{Macro added.}
%    To set the |\theorem@headerfont| default we first test if
%    the new fontselection scheme is in force.
% \changes{v2.1a}{90/02/15}{Support for new font selection.}
%    \begin{macrocode}
\ifx\upshape\undefined
%    \end{macrocode}
%    If not we define it to expand into |\bfseries|. We don't use
%    |\let| just in case a following style option redefines this
%    macro.
%    \begin{macrocode}
\gdef\theorem@headerfont{\bfseries}
%    \end{macrocode}
%    Otherwise we reset the current shape before calling |\bfseries|.
%    \begin{macrocode}
\else \gdef\theorem@headerfont{\normalfont\bfseries}\fi
%    \end{macrocode}
% \end{macro}
%
% \begin{macro}{\th@plain}
% \begin{macro}{\th@break}
% \begin{macro}{\th@marginbreak}
% \begin{macro}{\th@changebreak}
% \begin{macro}{\th@change}
% \begin{macro}{\th@margin}
%    The different styles are defined in macros such as
%    |\th@plain|.  Since memory space is precious in
%    ``non-Big-versions'', we have to avoid offering too many unused
%    definitions.  Therefore we define these styles in separate files
%    that can be loaded on demand.  Thus the commands themselves only
%    load these files. We use |\@input@|, a \LaTeXe{} internal command
%    that ensures that the file will be listed with |\listfiles|.
% \changes{v2.2a}{94/02/02}{Use `@input@ to load theorem layout files}
%    \begin{macrocode}
\gdef\th@plain{\@input@{thp.sty}}
\gdef\th@break{\@input@{thb.sty}}
\gdef\th@marginbreak{\@input@{thmb.sty}}
\gdef\th@changebreak{\@input@{thcb.sty}}
\gdef\th@change{\@input@{thc.sty}}
\gdef\th@margin{\@input@{thm.sty}}
%    \end{macrocode}
%    This list will be expanded when new styles become
%    available.  For testing,  just append new theorem substyles
%    as document options.
% \end{macro}
% \end{macro}
% \end{macro}
% \end{macro}
% \end{macro}
% \end{macro}
%
%
%
% \subsection{Definition of a new theorem set}
%
% As already pointed out, a new theorem environment can be defined in
% three different ways:
% \begin{verbatim}
%    \newtheorem{Lem}{Lemma}
%    \newtheorem{Lem}{Lemma}[section]
%    \newtheorem{Lem}[Theorem]{Lemma}
%\end{verbatim}
% The function of the macro |\newtheorem| is to recognize these cases
% and then to branch into one of the three macros |\@ynthm|, |\@xnthm|
% or |\@othm|.  This mechanism is adopted unchanged from
% \cite{latex:source}; the essential point here is that, for example,
% in the second case, the arguments \texttt{Lem}, \texttt{Lemma} and
% \texttt{section} are passed over to the macro |\@xnthm|.
%
% We inspect this case first because the others present fewer problems,
% and thus are easily derived from this one.
%
% \begin{macro}{\@xnthm}
%    For our example arguments, the macro |\@xnthm| must fulfill
%    the following:
%    \begin{itemize}
%       \item
%          Define a new \LaTeX-counter `Lem'
%       \item
%          reset this counter within a |\section|
%       \item
%          define the macro |\theLem|
%       \item
%          define the environment macros |\Lem| and
%          |\endLem| using the current
%          |\theoremstyle| and |\theorem@bodyfont|.
%    \end{itemize}
%    Obviously, all this should happen only if the first argument
%    of |\@xnthm| (i.e.\ \texttt{Lem} in our example) is chosen so
%    as not to conflict with any previously defined commands or
%    environments.  This test is performed by the \LaTeX{} macro
%    |\@ifdefinable|.
%    \begin{macrocode}
\gdef\@xnthm#1#2[#3]{\expandafter\@ifdefinable\csname #1\endcsname
%    \end{macrocode}
%    Therefore, the first argument of |\@ifdefinable| is the
%    expansion (in the example, |\Lem|) of
%    |\csname||#1\endcsname|.  The second argument is
%    executed only if the test has been completed successfully.
%    \begin{macrocode}
   {%
%    \end{macrocode}
%    Now we define the new counter. The names of the \LaTeX{}
%    macros employed should speak for themselves:
% \changes{v2.2b}{1995/10/07}{Use \cs{@newctr}}
% \changes{v2.2c}{1995/11/23}{Use \cs{@newctr} properly.}
%    \begin{macrocode}
    \@definecounter{#1}\@newctr{#1}[#3]%
%    \end{macrocode}
%    Using |\@newctr| will give a proper error message if the counter
%    in |#3| is not defined.
%    In defining `|\theLem|' we must generate the desired macro
%    name by use of |\expandafter| and |\csname|.
%    \begin{macrocode}
    \expandafter\xdef\csname the#1\endcsname
%    \end{macrocode}
%    An |\xdef| is used in order to make the definition global,
%    and to ensure that it contains the replacement texts of
%    |\@thmcountersep| and |\@thmcounter|.\footnote{These
%    two macros can be defined by the document style.  Their default
%    values produce a `\texttt{.}' as separation and an arabic
%    representation of the number.} However, not everything should be
%    expanded. For example, it saves space to use |\thesection|
%    instead of its---at times---lengthy expansion.
%    \begin{macrocode}
      {\expandafter \noexpand \csname the#3\endcsname
       \@thmcountersep \@thmcounter{#1}}%
%    \end{macrocode}
%    Thus with the defaults of \LaTeX{}, |\theLem| would be
%    replaced by the command sequence |\thesection.\arabic{Lem}|.
%
%    We will now look at the definition of the macro which is executed
%    at the beginning of the actual environment (in our example this
%    macro is |\Lem|). It should be noted that we use an
%    ``|\expandafter| trick'' to expand only certain parts of the
%    replacement text at the time of the definition.
%    \begin{macrocode}
    \def\@tempa{\global\@namedef{#1}}%
    \expandafter \@tempa \expandafter{%
%    \end{macrocode}
%    First, the macro that contains the current definitions of
%    |\@begintheorem| and |\@opargtheorem| should be called
%    up.  The name of this macro---as is already known---has the form
%    |\th@|\meta{theorem\ style}; therefore, it must be called by
%    \begin{macrocode}
      \csname th@\the \theorem@style
%    \end{macrocode}
%    In addition the default theorem font should be changeable, i.e.\
%    we have to insert the contents of |\theorem@bodyfont|. For
%    that reason, we expand even further, beyond |\endcsname|,
%    and thus insert the contents of the \textsf{token} register
%    |\theorem@bodyfont| in the replacement text.
%    \begin{macrocode}
            \expandafter \endcsname \the \theorem@bodyfont
%    \end{macrocode}
%    Now it is time to call the macro |\@thm| which takes over
%    the further processing.  It has two arguments: the current
%    counter name (in our example, \texttt{Lem}), and the text of the
%    label (in our example, \texttt{Lemma}).
%    \begin{macrocode}
     \@thm{#1}{#2}}%
%    \end{macrocode}
%    With this, the `sub-definition' is complete.  The macro
%    |\@endtheorem| ends a theorem environment and is, so far, nothing
%    but an |\endtrivlist|.  (Hence it is defined globally, and not
%    within the theorem styles.\footnote{This has to be changed as
%    soon as theorem styles that change \texttt{\bslash @endtheorem}
%    exist.  In such a case, all existing styles must be changed as
%    well since they will have to reset the macro.}) Therefore, we can
%    set it equivalent to the macro that ends the theorem set (in our
%    example, |\endLem|).  However, if some day theorem styles exist
%    that do change |\@endtheorem|, we would have to use the
%    commented-out line instead.
%    \begin{macrocode}
    \global \expandafter \let \csname end#1\endcsname \@endtheorem
%   \global\@namedef{end#1}{\@endtheorem}%
%    \end{macrocode}
%    With these commands all the required definitions are employed,
%    unless the test |\@ifdefinable| has failed.  Therefore, we
%    end the second argument of this macro and with it the definition
%    of |\@xnthm|.
%    \begin{macrocode}
   }}
%    \end{macrocode}
% \end{macro}
%
% \begin{macro}{\@ynthm}
%    The definition of |\@ynthm| is completely analogous. In
%    this case the new counter that is defined is not reset within
%    another counter; thus the definition of |\the...|
%    is simplified:
%    \begin{macrocode}
\gdef\@ynthm#1#2{\expandafter\@ifdefinable\csname #1\endcsname
   {\@definecounter{#1}%
    \expandafter\xdef\csname the#1\endcsname{\@thmcounter{#1}}%
%    \end{macrocode}
%    The rest of the definition corresponds literally to that of
%    |\@xnthm|:
%    \begin{macrocode}
    \def\@tempa{\global\@namedef{#1}}\expandafter \@tempa
     \expandafter{\csname th@\the \theorem@style \expandafter
     \endcsname \the\theorem@bodyfont \@thm{#1}{#2}}%
    \global \expandafter \let \csname end#1\endcsname \@endtheorem}}
%    \end{macrocode}
% \end{macro}
%
% \begin{macro}{\@othm}
%    The definition of |\@othm| does not contain anything new.
%    \begin{macrocode}
\gdef\@othm#1[#2]#3{%
%    \end{macrocode}
% \changes{v2.2b}{95/10/07}{Build in check for counter defined}
%    We do not define a new counter but instead use one that has
%    already been defined.  Thus the only definition we need is that
%    of this pseudo-counter (i.e.\ |\the|\meta{env.\ name}).
%    First we check if |#2| corresponds to a known counter name.
%    \begin{macrocode}
  \expandafter\ifx\csname c@#2\endcsname\relax
   \@nocounterr{#2}%
  \else
   \expandafter\@ifdefinable\csname #1\endcsname
   {\expandafter \xdef \csname the#1\endcsname
     {\expandafter \noexpand \csname the#2\endcsname}%
%    \end{macrocode}
%    All other parts of the definition can be adopted from
%    |\@xnthm|. We have to remember, though, that in this case
%    the name of the current counter and the theorem label have moved
%    to the second and third arguments.
%    \begin{macrocode}
    \def\@tempa{\global\@namedef{#1}}\expandafter \@tempa
     \expandafter{\csname th@\the \theorem@style \expandafter
     \endcsname \the\theorem@bodyfont \@thm{#2}{#3}}%
    \global \expandafter \let \csname end#1\endcsname \@endtheorem}%
  \fi}
%    \end{macrocode}
% \end{macro}
%
% \subsection{Macros that are employed in a {\envfont theorem}
%    environment}
%
% \begin{macro}{\@thm}
%    The macro |\@thm| has to increase the current counter.  Then,
%    depending on whether the environment has (or does not have) an
%    optional argument, it has to branch into either |\@begintheorem|
%    or |\@opargtheorem|.
%    \begin{macrocode}
\gdef\@thm#1#2{\refstepcounter{#1}%
%    \end{macrocode}
%    Now we start a {\envfont trivlist} environment, and give
%    |\@topsep| and |\@topsepadd| the values of the \textsf{skip}
%    registers |\theorempreskipamount| and |\theorempostskipamount|.
%    The value in |\@topsep| is the vertical space that is inserted by
%    the first (and only) |\item| in our |\trivlist| whilst
%    |\@topsepadd| is inserted by |\@endparenv| at the end of that
%    {\envfont trivlist} environment.  By using these registers, we
%    obtain the desired space around a {\envfont theorem} environment.
%    \begin{macrocode}
   \trivlist
   \@topsep \theorempreskipamount               % used by first \item
   \@topsepadd \theorempostskipamount           % used by \@endparenv
%    \end{macrocode}
%    Now we have to test whether an optional argument has been given.
% \changes{v2.0d}{89/07/18}{Forgotten percent added.}
%    \begin{macrocode}
   \@ifnextchar [%
%    \end{macrocode}
%    If there is an optional argument, we will call |\@ythm|, and
%    move the arguments read back into the input stream.
%    \begin{macrocode}
   {\@ythm{#1}{#2}}%
%    \end{macrocode}
%    If not, we call |\@begintheorem|. Its first argument is the
%    name of the theorem set (hence the second argument of
%    |\@thm|). Its second argument is the macro that produces the
%    current number.
%    \begin{macrocode}
   {\@begintheorem{#2}{\csname the#1\endcsname}\ignorespaces}}
%    \end{macrocode}
% \end{macro}
%
% \begin{macro}{\@xthm}
% \begin{macro}{\@ythm}
%    Both these macros were originally called by |\@thm|.  We do
%    not need |\@xthm| anymore, hence we reset it to
%    |\relax|. The definition of |\@ythm| has not changed at
%    all from its definition in \LaTeX. In order to make the macros
%    easier to understand, we will
%    nevertheless present it (commented out).
%    \begin{macrocode}
\global\let\@xthm\relax
% \def\@ythm#1#2[#3]{\@opargbegintheorem{#2}{\csname
%           the#1\endcsname}{#3}\ignorespaces}
%</package>
%    \end{macrocode}
%    The primitive |\ignorespaces| in |\@ythm| and
%    |\@thm| is needed to remove the spaces between the
%    |\begin{...}| and the actual text.
% \end{macro}
% \end{macro}
%
%
% \subsection{Definition of the theorem substyles}
%
% As already pointed out, the theorem substyles, defined below, are
% only loaded when necessary.  Note that all these substyles, except
% \texttt{plain}, have |\slshape| as the default body font.
%
% \subsubsection{The \texttt{plain} style}
%
% \label{dok:th@plain} As the following macros use |@|, we have
% to locally set the |\catcode| of this symbol to ``letter''.
% This happens within a group, so that we do not have to worry about
% which |\catcode| that symbol had before.
%    \begin{macrocode}
%<*thp>
\begingroup \makeatletter
%    \end{macrocode}
% Since we are now within a group, we must make all definitions
% globally.  First we make sure that \texttt{theorem.sty} is loaded.
% This will allow us to use this file as a document style option
% without having to call \texttt{theorem} itself as an option.  At the
% same time, we assure that at least version 2 is loaded, since
% |\theorem@style| was not defined in earlier versions.
%    \begin{macrocode}
\@ifundefined{theorem@style}{\input{theorem.sty}}{}
\ProvidesFile{thp.sty}
             [\FMithmInfo]
%    \end{macrocode}
%
%
% \begin{macro}{\th@plain}
% \begin{macro}{\@begintheorem}
% \begin{macro}{\@opargbegintheorem}
%    |\theoremstyle{plain}| corresponds to the original
%    definition, except that the distances to the surrounding text are
%    determined by the parameters |\theorempreskipamount| and
%    |\theorempostskipamount|.
%    First we set the default body font.
%    \begin{macrocode}
\gdef\th@plain{\normalfont\itshape
%    \end{macrocode}
%    Then we define |\@begintheorem| and
%    |\@opargbegintheorem|. These two macros define how the
%    header of a theorem is typeset.  |\@opargbegintheorem| will
%    be called if a \textsf{theorem} environment with an optional
%    argument is encountered; otherwise, the header is constructed by
%    calling |\@begintheorem|.  If one of these macros is
%    executed, we are within a \textsf{trivlist} environment started by
%    |\@thm|.  So the theorem header is produced with an
%    |\item| command.
%
%    Instead of specifying the header font directly, all standard
%    theorem styles use the |\theorem@headerfont| macro to allow
%    customization.  The extra space (|\labelsep|) is necessary
%    because of problems in the \textsf{trivlist} environment.
%    \begin{macrocode}
  \def\@begintheorem##1##2{%
        \item[\hskip\labelsep \theorem@headerfont ##1\ ##2]}%
%    \end{macrocode}
%    The definition of |\@opargbegintheorem| is completely
%    analogous. The only difference is the fact that there exists a
%    third argument (which is the optional parameter of the
%    environment and contains additional information about the
%    theorem).  Customarily we enclose it in parentheses.
%    \begin{macrocode}
\def\@opargbegintheorem##1##2##3{%
   \item[\hskip\labelsep \theorem@headerfont ##1\ ##2\ (##3)]}}
%    \end{macrocode}
% \end{macro}
% \end{macro}
% \end{macro}
%
% \noindent
% We conclude with an |\endgroup| to restore the
% |\catcode| of |@|.
%    \begin{macrocode}
\endgroup
%</thp>
%    \end{macrocode}
%
%
% \subsubsection{The \texttt{break} style}
%
% This style option is stored in the file |thb.sty|.
% For the next two lines see the documentation for |\th@plain| on
% page~\pageref{dok:th@plain}.
%    \begin{macrocode}
%<*thb>
\begingroup \makeatletter
\@ifundefined{theorem@style}{\input{theorem.sty}}{}
\ProvidesFile{thb.sty}
             [\FMithmInfo]
%    \end{macrocode}
%
% \begin{macro}{\th@break}
%    |\theoremstyle{break}| produces a line break after the name
%    of the theorem.  The font is |\slshape|.  Hence, we define
%    |\th@break| as follows:
%    \begin{macrocode}
\gdef\th@break{\normalfont\slshape
  \def\@begintheorem##1##2{\item[%
%    \end{macrocode}
%    We run into the following problem: it is not possible to create
%    the header with |\item[|\meta{title}|]| and then start
%    a new line by, for example, |\mbox{}\\|. Such a definition
%    will fail whenever a list environment follows immediately.  With
%    the above construction, the |\mbox{}| causes the switch
%    |@inlabel| (cf.\ definition of |\list| and
%    |\trivlist| in \cite{latex:source}) to be set to \texttt{false}
%    and so the following list will insert additional vertical
%    space (|\topskip|). This is quite annoying. Therefore, we
%    create the line break within the |\item|. In order to ensure
%    that the text will begin at the proper position in the following
%    line, we simply pretend that the label does not take any
%    room.\footnote{This will lead to problems whenever very high
%    symbols occurring in the line tower into the
%    heading. So, something else has to be done here sometime.}
%    \begin{macrocode}
       \rlap{\vbox{\hbox{\hskip \labelsep\theorem@headerfont ##1\ ##2}%
                   \hbox{\strut}}}]}%
%    \end{macrocode}
%    Again, the definition of |\@opargbegintheorem| is
%    completely analogous.
%    \begin{macrocode}
\def\@opargbegintheorem##1##2##3{%
  \item[\rlap{\vbox{\hbox{\hskip \labelsep \theorem@headerfont
                     ##1\ ##2\ (##3)}%
                    \hbox{\strut}}}]}}
\endgroup
%</thb>
%    \end{macrocode}
% \end{macro}
%
%
% \subsubsection{The \texttt{changebreak} style}
%
%    \begin{macrocode}
% This style option is stored in the file |thcb.sty|.
%    \begin{macrocode}
%<*thcb>
\begingroup \makeatletter
\@ifundefined{theorem@style}{\input{theorem.sty}}{}
\ProvidesFile{thcb.sty}
             [\FMithmInfo]
%    \end{macrocode}
%
% \begin{macro}{\th@changebreak}
%    The \texttt{change-break} theorem style is like \texttt{break}
%    but with interchange of theorem name and theorem number.  Thus we
%    define |\th@changebreak| as follows:
%    \begin{macrocode}
\gdef\th@changebreak{\normalfont\slshape
  \def\@begintheorem##1##2{\item
       [\rlap{\vbox{\hbox{\hskip\labelsep \theorem@headerfont ##2\ ##1}%
                    \hbox{\strut}}}]}%
\def\@opargbegintheorem##1##2##3{%
  \item[\rlap{\vbox{\hbox{\hskip\labelsep \theorem@headerfont
                     ##2\ ##1\ (##3)}%
                    \hbox{\strut}}}]}}
\endgroup
%</thcb>
%    \end{macrocode}
% \end{macro}
%
%
% \subsubsection{The \texttt{change} style}
%
% This style option is stored in the file |thc.sty|.
%    \begin{macrocode}
%<*thc>
\begingroup \makeatletter
\@ifundefined{theorem@style}{\input{theorem.sty}}{}
\ProvidesFile{thc.sty}
             [\FMithmInfo]
%    \end{macrocode}
%
% \begin{macro}{\th@change}
%    The \texttt{change} theorem style corresponds to the
%    \texttt{change break} style without a linebreak after the header.
%    To say it in another way, it's the same as the \texttt{plain}
%    style but with number and name interchanged and |\slshape| as the
%    default font.
%    \begin{macrocode}
\gdef\th@change{\normalfont\slshape
  \def\@begintheorem##1##2{\item
       [\hskip\labelsep \theorem@headerfont ##2\ ##1]}%
\def\@opargbegintheorem##1##2##3{%
  \item[\hskip\labelsep \theorem@headerfont ##2\ ##1\ (##3)]}}
\endgroup
%</thc>
%    \end{macrocode}
% \end{macro}
%
%
% \subsubsection{The \texttt{marginbreak} style}
%
% This style option is the one used most often at Mainz.  It is saved
% in the file |thmb.sty|.
%    \begin{macrocode}
%<*thmb>
\begingroup \makeatletter
\@ifundefined{theorem@style}{\input{theorem.sty}}{}
\ProvidesFile{thmb.sty}
             [\FMithmInfo]
%    \end{macrocode}
%
% \begin{macro}{\th@marginbreak}
%    The \texttt{margin break} style is nearly the same as the
%    \texttt{change break} style. The only difference is the placement
%    of the theorem number.  We use |\llap| to place it in the left
%    margin.
%
%    In this style |\labelsep| denotes the separation between the
%    number and the text.
%    \begin{macrocode}
\gdef\th@marginbreak{\normalfont\slshape
  \def\@begintheorem##1##2{\item
       [\rlap{\vbox{\theorem@headerfont
                    \hbox{\llap{##2}\hskip\labelsep ##1}%
                    \hbox{\strut}}}]}%
\def\@opargbegintheorem##1##2##3{%
  \item[\rlap{\vbox{\theorem@headerfont
                    \hbox{\llap{##2}\hskip\labelsep ##1\ (##3)}%
                    \hbox{\strut}}}]}}
\endgroup
%</thmb>
%    \end{macrocode}
% \changes{v2.1c}{90/02/17}{`endgroup outside macrocode env.}
% \end{macro}
%
%
%
% \subsubsection{The \texttt{margin} style}
%
% This style option is stored in the file |thm.sty|.
%    \begin{macrocode}
%<*thm>
\begingroup \makeatletter
\@ifundefined{theorem@style}{\input{theorem.sty}}{}
\ProvidesFile{thm.sty}
             [\FMithmInfo]
%    \end{macrocode}
%
% \begin{macro}{\th@margin}
%    Again this is only a variant of the theorem styles described
%    above without any new ideas.
%    \begin{macrocode}
\gdef\th@margin{\normalfont\slshape
  \def\@begintheorem##1##2{\item
       [\theorem@headerfont \llap{##2}\hskip\labelsep ##1]}%
\def\@opargbegintheorem##1##2##3{%
  \item[\theorem@headerfont \llap{##2}\hskip\labelsep ##1\ (##3)]}}
\endgroup
%</thm>
%    \end{macrocode}
% \end{macro}
%
%
%
%
% \subsection{Final Definitions}
%
% \begin{macro}{\theorempreskipamount}
% \begin{macro}{\theorempostskipamount}
%    The \textsf{skip} parameters that regulate the vertical empty space
%    before and after the {\envfont theorem} environment have
%    to be allocated as well.
% \changes{v2.0f}{89/07/20}{Unnecessary `global' removed.}
%    \begin{macrocode}
%<*package>
\newskip\theorempreskipamount
\newskip\theorempostskipamount
%    \end{macrocode}
%    Since we have used the same values for all theorem sets, we now can
%    assign them.
%    \begin{macrocode}
\global\setlength\theorempreskipamount{12pt plus 5pt minus 3pt}
\global\setlength\theorempostskipamount{8pt plus 3pt minus 1.5pt}
%    \end{macrocode}
% \end{macro}
% \end{macro}
%
%
% \begin{macro}{\@endtheorem}
%    The same holds for the macro |\@endtheorem|, which ends a
%    {\envfont theorem} environment. Since it is the same for all
%    theorem sets, it is removed from the macros |\th@|\meta{style}.
%    It simply ends the {\envfont trivlist} environment, which was
%    begun in |\@thm|.
%    \begin{macrocode}
\global\let\@endtheorem=\endtrivlist
%    \end{macrocode}
% \end{macro}
%
%
%
% \begin{macro}{\@preamblecmds}
%    All macros defined above are to be used only in the preamble.
%    Therefore, we insert them in |\@preamblecmds| which will disable
%    them at begin document. This is done by the internal \LaTeXe{}
%    command |\@onlypreamble|.
% \changes{v2.0g}{89/09/19}{`theoremheaderfont and `theorembodyfont
%    added.}
% \changes{v2.2a}{94/02/03}{Use \LaTeXe{} interface}
%    \begin{macrocode}
\@onlypreamble\@xnthm
\@onlypreamble\@ynthm
\@onlypreamble\@othm
\@onlypreamble\newtheorem
\@onlypreamble\theoremstyle
\@onlypreamble\theorembodyfont
\@onlypreamble\theoremheaderfont
%    \end{macrocode}
% \end{macro}
%
% Finally we declare the \texttt{plain} theorem style to be the default.
% \changes{v2.1b}{90/06/04}{Fixed missing `item by declaring default
%    style.}
%    \begin{macrocode}
\theoremstyle{plain}
%</package>
%    \end{macrocode}
%
% \Finale

\endinput
}{}
\ProvidesFile{thm.sty}
             [\FMithmInfo]
%    \end{macrocode}
%
% \begin{macro}{\th@margin}
%    Again this is only a variant of the theorem styles described
%    above without any new ideas.
%    \begin{macrocode}
\gdef\th@margin{\normalfont\slshape
  \def\@begintheorem##1##2{\item
       [\theorem@headerfont \llap{##2}\hskip\labelsep ##1]}%
\def\@opargbegintheorem##1##2##3{%
  \item[\theorem@headerfont \llap{##2}\hskip\labelsep ##1\ (##3)]}}
\endgroup
%</thm>
%    \end{macrocode}
% \end{macro}
%
%
%
%
% \subsection{Final Definitions}
%
% \begin{macro}{\theorempreskipamount}
% \begin{macro}{\theorempostskipamount}
%    The \textsf{skip} parameters that regulate the vertical empty space
%    before and after the {\envfont theorem} environment have
%    to be allocated as well.
% \changes{v2.0f}{89/07/20}{Unnecessary `global' removed.}
%    \begin{macrocode}
%<*package>
\newskip\theorempreskipamount
\newskip\theorempostskipamount
%    \end{macrocode}
%    Since we have used the same values for all theorem sets, we now can
%    assign them.
%    \begin{macrocode}
\global\setlength\theorempreskipamount{12pt plus 5pt minus 3pt}
\global\setlength\theorempostskipamount{8pt plus 3pt minus 1.5pt}
%    \end{macrocode}
% \end{macro}
% \end{macro}
%
%
% \begin{macro}{\@endtheorem}
%    The same holds for the macro |\@endtheorem|, which ends a
%    {\envfont theorem} environment. Since it is the same for all
%    theorem sets, it is removed from the macros |\th@|\meta{style}.
%    It simply ends the {\envfont trivlist} environment, which was
%    begun in |\@thm|.
%    \begin{macrocode}
\global\let\@endtheorem=\endtrivlist
%    \end{macrocode}
% \end{macro}
%
%
%
% \begin{macro}{\@preamblecmds}
%    All macros defined above are to be used only in the preamble.
%    Therefore, we insert them in |\@preamblecmds| which will disable
%    them at begin document. This is done by the internal \LaTeXe{}
%    command |\@onlypreamble|.
% \changes{v2.0g}{89/09/19}{`theoremheaderfont and `theorembodyfont
%    added.}
% \changes{v2.2a}{94/02/03}{Use \LaTeXe{} interface}
%    \begin{macrocode}
\@onlypreamble\@xnthm
\@onlypreamble\@ynthm
\@onlypreamble\@othm
\@onlypreamble\newtheorem
\@onlypreamble\theoremstyle
\@onlypreamble\theorembodyfont
\@onlypreamble\theoremheaderfont
%    \end{macrocode}
% \end{macro}
%
% Finally we declare the \texttt{plain} theorem style to be the default.
% \changes{v2.1b}{90/06/04}{Fixed missing `item by declaring default
%    style.}
%    \begin{macrocode}
\theoremstyle{plain}
%</package>
%    \end{macrocode}
%
% \Finale

\endinput
}{}
\ProvidesFile{thcb.sty}
             [\FMithmInfo]
%    \end{macrocode}
%
% \begin{macro}{\th@changebreak}
%    The \texttt{change-break} theorem style is like \texttt{break}
%    but with interchange of theorem name and theorem number.  Thus we
%    define |\th@changebreak| as follows:
%    \begin{macrocode}
\gdef\th@changebreak{\normalfont\slshape
  \def\@begintheorem##1##2{\item
       [\rlap{\vbox{\hbox{\hskip\labelsep \theorem@headerfont ##2\ ##1}%
                    \hbox{\strut}}}]}%
\def\@opargbegintheorem##1##2##3{%
  \item[\rlap{\vbox{\hbox{\hskip\labelsep \theorem@headerfont
                     ##2\ ##1\ (##3)}%
                    \hbox{\strut}}}]}}
\endgroup
%</thcb>
%    \end{macrocode}
% \end{macro}
%
%
% \subsubsection{The \texttt{change} style}
%
% This style option is stored in the file |thc.sty|.
%    \begin{macrocode}
%<*thc>
\begingroup \makeatletter
\@ifundefined{theorem@style}{% \iffalse meta-comment
%
% Copyright (C) 1993-2024
%
% The LaTeX Project and any individual authors listed elsewhere
% in this file.
%
% This file is part of the Standard LaTeX `Tools Bundle'.
% -------------------------------------------------------
%
% It may be distributed and/or modified under the
% conditions of the LaTeX Project Public License, either version 1.3c
% of this license or (at your option) any later version.
% The latest version of this license is in
%    https://www.latex-project.org/lppl.txt
% and version 1.3c or later is part of all distributions of LaTeX
% version 2005/12/01 or later.
%
% The list of all files belonging to the LaTeX `Tools Bundle' is
% given in the file `manifest.txt'.
%
% \fi
%
%%
%
%
\def\FMithmInfo{2023/07/05 v2.2c Theorem extension package (FMi)}
%
% \ProvidesFile{theorem.dtx}[\FMithmInfo]
%\iffalse   % this is a METACOMMENT !
%%
%% Package `theorem' to use with LaTeX2e
%% Copyright (C) 1989-1995 Frank Mittelbach, all rights reserved.
%<+package>\@ifundefined{theorem@style}{}{\endinput}
%<+package>\NeedsTeXFormat{LaTeX2e}[1995/06/01]
%<+package>\ProvidesPackage{theorem}
%<+package>         [\FMithmInfo]
%
% \fi
%
% \changes{v1.0}{87/05/07}{First draft.}
% \changes{v1.0a}{87/05/07}{`thistheoremstyle' added.}
% \changes{v1.1a}{87/07/29}{New concept: redefine the internal
%                          \protect\LaTeX{} commands `@xnthm' etc.}
% \changes{v1.1b}{87/07/31}{`theoremfont' used to specify font.}
% \changes{v1.2a}{88/01/08}{`@endtheorem' now globally defined.}
% \changes{v1.2b}{88/01/11}{`newtheorem' only allowed in the preamble.}
% \changes{v1.3a}{88/02/14}{`theoremfont' now a toks register.}
%
% \changes{v2.0a}{88/11/16}{New implementation and new documentation.}
% \changes{v2.0a}{88/11/16}{No layout changes but internal modifications
%                           old substyles won't work.}
% \changes{v2.0b}{89/01/30}{`global' used for all definitions.}
% \changes{v2.0c}{89/06/15}{Documentation translated to English.
%                            Thanks to LLL and Nina}
% \changes{v2.1a}{90/02/15}{Support for new font selection.}
% \changes{v2.2a}{94/02/03}{Cleanup for LaTeX2e}
%
% \DoNotIndex{\@endparenv,\@ifdefinable,\@ifnextchar,\@ifundefined}
% \DoNotIndex{\@namedef,\@nameuse,\@spaces,\@tempa,\@warning}
% \DoNotIndex{\ }
% \DoNotIndex{\begingroup,\bfseries}
% \DoNotIndex{\csname}
% \DoNotIndex{\def,\do,\docdate}
% \DoNotIndex{\endcsname,\endgroup,\endinput,\endtrivlist,\expandafter}
% \DoNotIndex{\filedate,\fileversion}
% \DoNotIndex{\gdef,\global}
% \DoNotIndex{\hbox,\hskip}
% \DoNotIndex{\ignorespaces,\input,\itshape,\item}
% \DoNotIndex{\labelsep,\let,\llap}
% \DoNotIndex{\makeatletter}
% \DoNotIndex{\noexpand}
% \DoNotIndex{\refstepcounter,\relax,\rlap}
% \DoNotIndex{\setlength,\slshape,\space,\string,\strut}
% \DoNotIndex{\the,\trivlist,\typeout}
% \DoNotIndex{\vbox}
% \DoNotIndex{\xdef}
%
% \def\deflabel#1{\ttfamily #1\hfill}
% \def\deflist#1{\begin{list}{}{\settowidth\labelwidth{\ttfamily #1}%
%                               \setlength\leftmargin\labelwidth
%                               \addtolength\leftmargin\labelsep
%                               \let\makelabel\deflabel}}
% \def\enddeflist{\end{list}}
% \def\envfont{\normalfont\ttfamily}
%
% \MakeShortVerb{\|}
%
% \newcommand{\TUB}{{\sl TUGboat\/}}
%
% \GetFileInfo{theorem.dtx}
% \title{An Extension of the
%        \LaTeX{} theorem environment\thanks{This file
%        has version number \fileversion, last
%        revised \filedate.}}
% \author{Frank Mittelbach}
%
% \MaintainedByLaTeXTeam{tools}
% \maketitle
%
% \begin{abstract}
%    The macros described in this paper yield an extension of the
%    \LaTeX{} theorem mechanism. It is designed to satisfy the
%    different requirements of various journals. Thus, the
%    layout of the ``theorems'' can be manipulated by determining a
%    ``style''. This article describes not only the use, but
%    also the definition, of the necessary macros.
% \end{abstract}
%
% \section*{Preface to version 2.2}
%
% For \LaTeXe{} this package did not need any fundamental changes. I
% only modified the messages generated so that theorem layout styles
% will show up with the |\listfiles| command and cleaned the section
% on the New Font Selection Scheme since this is now included in
% \LaTeX.
%
% \section*{Preface to version 2.1}
%
% This version is identical to 2.0g described in \TUB\ 10\#3 except
% for some internal defaults which are now set depending on the used
% font selection scheme.
%
% This was done to avoid unpleasant surprises if the new font selection
% scheme is in force. For further details see
% section~\ref{sec:newfonts} and \cite{bk:GMS94}.
%
% \section{Introduction}
%
% For our purposes here, ``theorems'' are labelled enunciations,
% often set off from the main text by extra space and a font change.
% Theorems, corollaries, conjectures, definitions, and remarks
% are all instances of ``theorems''.  The ``header'' of these structures
% is composed of a label (such as \textsc{Theorem} or \textsc{Remark})
% and a number which serializes an item in the sequence of items
% with the same label.
%
% Shortly after the introduction of \LaTeX{} at the Fachbereich
% Mathematik in Mainz, the desire to manipulate the layout of
% ``theorems'' arose.  In Mainz, the following two conventions came
% into general use:
% \begin{enumerate}
%   \item The number of the theorem is shown in the margin.
%   \item There is a line break at
%         the end of the theorem header.
% \end{enumerate}
%
% Additionally, some journals require different formats which depend on
% the ``sort of theorem'': e.g.\ often remarks and definitions are set
% in |\upshape|, while |\itshape| is employed for main theorems.
%
% Confronted with these requirements, a theorem environment was
% developed in Mainz which allows separate determination of the
% layout of the ``theorems sets'', comparable to |\pagestyle|.
%
%
% \section{The user interface}
% \subsection{Defining new theorem sets}
%
% \DescribeMacro\newtheorem
% As in the original \LaTeX{} version, the command |\newtheorem|
% defines a new ``theorem set'' or ``theorem-like structure''.
% Two required arguments name the new environment and give the text
% to be typeset with each instance of the new ``set'', while
% an optional argument determines how
% the ``set'' is enumerated:
% \begin{description}
%    \item[\ttfamily \bslash newtheorem\{foo\}\{bar\}]
%       The theorem set {\envfont foo} (whose name is \texttt{bar})
%       uses its own counter.
%    \item[\ttfamily \bslash newtheorem\{foo2\}{[foo]}\{bar2\}]
%       The theorem set {\envfont foo2} (printed name \texttt{bar2})
%       uses the same counter as the theorem set \texttt{foo}.
%    \item[\ttfamily \bslash newtheorem\{foo3\}\{bar3\}{[section]}]
%       The theorem set {\envfont foo3} (printed name \texttt{bar3}) is
%       enumerated within the counter \texttt{section}, i.e.\ with every
%       new |\section| the enumeration begins again with 1, and
%       the enumeration is composed from the section-number and the
%       theorem counter itself.
% \end{description}
%
% \DescribeMacro\theoremstyle
% Additionally, the command |\theoremstyle| can define the layout
% of various, or all, theorem sets.  It should be noted that any
% theorem set defined by |\newtheorem| is typeset in the
% |\theoremstyle| that is current at the time of the definition.
% Thus, the following
% \begin{verbatim}
%    \theoremstyle{break}        \newtheorem{Cor}{Corollary}
%    \theoremstyle{plain}        \newtheorem{Exa}{Example}[section]
%\end{verbatim}                             ^^A No blank at line start
% leads to the result that the set {\envfont Cor} is formatted in the
% style \texttt{break}, while the set {\envfont Exa} and all the
% following ones are formatted in the style \texttt{plain}, unless
% another |\theoremstyle| follows.  Since the definitions installed by
% |\newtheorem| are global, one also can limit |\theoremstyle| locally
% by grouping braces.
%
% \DescribeMacro\theorembodyfont
% The choice of the font for the theorem body is completely
% independent of the chosen |\theoremstyle|; this has proven to
% be very advantageous.  For example,
% \begin{verbatim}
%   {\theorembodyfont{\upshape}          \newtheorem{Rem}{Remark}}
%\end{verbatim}
% defines a theorem set {\envfont Rem}, which will be set in
% |\upshape| in the current layout (which in our example is
% \texttt{plain}).  As with |\theoremstyle|, the |\theorembodyfont|
% chosen is that current at the time of |\newtheorem|.  If
% |\theorembodyfont| is not specified or one defines
% |\theorembodyfont{}|, then the font used will be that defined by the
% |\theoremstyle|.
%
% \DescribeMacro\theoremheaderfont It is also possible to customize
% the font used for the theorem headers.  This is, however, a global
% declaration, and therefore there should be at most one
% |\theoremheaderfont| declaration in the preamble.\footnote{If it is
% actually necessary to have different header fonts, one has to define
% new theorem styles (substituting the desired font) or specify the
% information directly in the \texttt{\bslash newtheorem} declaration
% (the unclean variant).}
%
% \DescribeMacro\theorempreskipamount
% \DescribeMacro\theorempostskipamount Two additional parameters
% affect the vertical space around the theorem environments:
% |\theorempreskipamount| and |\theorempostskipamount| define,
% respectively, the spacing before and after such an environment.
% These parameters apply for all theorem sets and can be manipulated
% with the ordinary length macros.  They are rubber lengths
% (`\textsf{skips}'), and therefore can contain \texttt{plus} and
% \texttt{minus} parts.
%
% Since the definition of theorem sets should---most sensibly---be
% placed in the preamble, we only allow installation there.
% It is therefore possible to release the memory used here after
% |\begin{document}|, in order to make room for other
% applications.
%
%
% \subsection{Existing theorem styles}
%
% The following theorem styles exist to date:
% \begin{deflist}{marginbreak}
%    \item[plain]
%       This theorem style emulates the original \LaTeX{} definition,
%       except that additionally the parameters
%       |\theorem...skipamount| are used.
%    \item[break]
%       In this style, the theorem header is followed by a line
%       break.
%    \item[marginbreak]
%       The theorem number is set in the margin, and there is a line
%       break as in \texttt{break}.
%    \item[changebreak]
%       Like \texttt{break}, but with header number and text
%       interchanged.
%    \item[change]
%       Header number and text are interchanged, without a line break.
%    \item[margin]
%       The number is set in the left margin, without a line break.
% \end{deflist}
% All styles (except \texttt{plain}) select |\slshape| as the default
% |\theorembodyfont|.
%
%
% \subsection{Examples}
%
% Given the above theorem sets {\envfont Cor}, {\envfont Exa}
% and {\envfont Rem},
% suppose that the preamble also contains the declarations:
% \begin{verbatim}
%    \theoremstyle{marginbreak}   \newtheorem{Lem}[Cor]{Lemma}
%    \theoremstyle{change}
%    \theorembodyfont{\itshape}        \newtheorem{Def}[Cor]{Definition}
%
%    \theoremheaderfont{\scshape}
%\end{verbatim}
% Then the following are some typical examples of the typeset output
% resulting from their use.
%
% \begin{Cor}
%    This is a sentence typeset in the theorem environment {\envfont
%    Cor}.
% \end{Cor}
%
% \begin{Exa}
%    This is a sentence typeset in the theorem environment {\envfont
%    Exa}.
% \end{Exa}
%
% \begin{Rem}
%    This is a sentence typeset in the theorem environment {\envfont
%    Rem}.
% \end{Rem}
%
% \begin{Lem}[Ben User]
%    This is a sentence typeset in the theorem environment {\envfont
%    Lem}.
% \end{Lem}
%
% \begin{Def}[Very impressive Definition]
%    This is a sentence typeset in the theorem environment {\envfont
%    Def}.
% \end{Def}
%
% The last two examples show the effect of the optional argument to
% a theorem environment (it is the text typeset in parentheses).
%
%
% \section{Special Considerations}
% \label{sec:newfonts}
%
% Theoremheader and body are implemented as a unit. This means that
% the |\theoremheaderfont|  will inherit characteristics of the
% |\theorembodyfont| in \LaTeXe. Thus, if for example
% |\theorembodyfont| is |\itshape| and |\theoremheaderfont|
% is |\bfseries| the font selected for the header will have the
% characteristics `bold extended italic'. If this is not desired one
% should set the |\theoremheaderfont| to something like
% \begin{verbatim}
%      \theoremheaderfont{\normalfont\bfseries}
%\end{verbatim}
% i.e.\ supplying all necessary font information explicitly.
%
%
%
% \section{Acknowledgements}
%
% The publication of this set of macros was only possible with the
% help of Christina Busse (translating the manuscript into English),
% Joachim Pense (playing the r\^ole of typist), Chris Rowley
% (looking everything over) and many others providing useful
% suggestions.
%
% \MaybeStop{%
%    \begin{thebibliography}{9}
%       \bibitem{bk:GMS94} \textsc{M.~Goossens}, \textsc{F.~Mittelbach}
%         and \textsc{A.~Samarin}.
%         \newblock The \LaTeX{} Companion.
%         \newblock
%         Addison-Wesley, Reading, Massachusetts, 1994.
%       \bibitem{latex:source} \textsc{Lamport, Leslie.} \newblock
%                               \texttt{latex.tex}, version 2.09, date
%                               Feb. 1990.
%    \end{thebibliography}
%    \PrintIndex
%    \PrintChanges}
%
%
% \section{The documentation driver file}
%
% The next bit of code contains the documentation driver file for
% \TeX{}, i.e., the file that will produce the documentation you are
% currently reading. It will be extracted from this file by the
% \texttt{docstrip} program. Since it is the first code in the file
% one can alternatively process this file directly with \LaTeXe{} to
% obtain the documentation.
%
%    \begin{macrocode}
%<*driver>
\documentclass{ltxdoc}

\usepackage{theorem}

%<+driver>% The next few lines define theorem sets which are used
%<+driver>% in the example section of the documentation.

\theoremstyle{break}       \newtheorem{Cor}{Corollary}
\theoremstyle{plain}       \newtheorem{Exa}{Example}[section]
{\theorembodyfont{\upshape}\newtheorem{Rem}{Remark}}
\theoremstyle{marginbreak} \newtheorem{Lem}[Cor]{Lemma}
\theoremstyle{change}
\theorembodyfont{\itshape} \newtheorem{Def}[Cor]{Definition}

\theoremheaderfont{\scshape}

\RecordChanges

\begin{document}
   \DocInput{theorem.dtx}
\end{document}
%</driver>
%    \end{macrocode}
%
% \section{Definition of the Macros}
%
% If the file has been loaded before, we abort immediately. If not the
% package announces itself (this is actually done at the very top of
% the file---the way it is done isn't good style so don't copy it).
% \changes{v2.0e}{89/07/19}{Spaces removed from `typeout'.}
% \changes{v2.2b}{95/11/19}{Announce moved to top of file}
%    \begin{macrocode}
%<*package>
%\@ifundefined{theorem@style}{}{\endinput}
%\def\FMithmInfo{1995/11/19 v2.2b Theorem extension package (FMi)}
%\ProvidesPackage{theorem}[\FMithmInfo]
%    \end{macrocode}
%
%
% \subsection{Definition of theorem styles and fonts}
%
% All the definitions in this file are done globally to allow
% inputting this file inside a group.
%
% \begin{macro}{\theoremstyle}
%    Before a theorem style can be installed, the chosen style must be
%    known. For that reason, we must test to see that
%    |\th@|\meta{style} is known or, more precisely, that it is
%    different from |\relax|. If the style is not known then
%    |\th@plain| is used.
%    \begin{macrocode}
\gdef\theoremstyle#1{%
   \@ifundefined{th@#1}{\@warning
          {Unknown theoremstyle `#1'. Using `plain'}%
          \theorem@style{plain}}%
%    \end{macrocode}
%    We save the theorem style to be used in the token register
%    |\theorem@style|.
%    \begin{macrocode}
      {\theorem@style{#1}}%
%    \end{macrocode}
%    Now we ``evaluate'' the theorem style: this means, we call the
%    macro |\th@|\meta{style} which will activate the relevant
%    definitions which are contained in a separate file.  This is done
%    in a group to suppress changes to the current font.  This could
%    otherwise pose problems together with the new font selection
%    scheme\footnote{When I printed the original article using the new
%    font selection scheme I ended with a document with slanted
%    typefaces (text headings and all) simply because one of the
%    theorem styles used \texttt{\bslash sl} at toplevel.} if the
%    |\th@|\meta{style} is evaluated a second time.
% \changes{v2.1a}{90/02/15}{Group added around `th@..}
% \changes{v2.1a}{90/02/15}{Reseting of `..@begintheorem removed.}
%    \begin{macrocode}
      \begingroup
        \csname th@\the\theorem@style \endcsname
      \endgroup}
%    \end{macrocode}
% \end{macro}
%
%
% \begin{macro}{\@begintheorem}
% \changes{v2.1a}{90/02/15}{Relaxed at toplevel.}
% \begin{macro}{\@opargbegintheorem}
% \changes{v2.1a}{90/02/15}{Relaxed at toplevel.}
%    We reset |\@begintheorem| and |\@opargbegintheorem| to
%    |\relax| since these commands are no longer necessary at
%    toplevel. This will save a few tokens.
%    \begin{macrocode}
\global\let\@begintheorem\relax
\global\let\@opargbegintheorem\relax
%    \end{macrocode}
% \end{macro}
% \end{macro}
%
% \begin{macro}{\theorem@style}
%    Obviously the \textsf{token} register used above has to be
%    allocated. To assure the utmost compatibility with the
%    original \LaTeX{} definition, we set the default theorem style to
%    \texttt{plain}, which implements the usual \LaTeX{} convention.
% \changes{v2.0f}{89/07/20}{Unnecessary `global' removed.}
%    \begin{macrocode}
\newtoks\theorem@style
\global\theorem@style{plain}
%    \end{macrocode}
% \end{macro}
%
% \begin{macro}{\theorembodyfont}
% \begin{macro}{\theorem@bodyfont}
%    If the |\theorembodyfont| is set by the user then it should not
%    interact with the default font set in the theorem style. When the
%    new font selection is in force this may happen if, for example,
%    the default is |\itshape| and the new |\theorembodyfont| is
%    |\sffamily|. So we add a |\reset@font| command in front of the
%    user definition.
% \changes{v2.0f}{89/07/20}{Unnecessary `global' removed.}
% \changes{v2.0g}{89/09/19}{`theoremfont to `theorembodyfont.}
% \changes{v2.1c}{92/06/22}{`reset@font added and logic changed.}
% \changes{v2.1d}{92/10/28}{Reset to empty if arg is empty}
%    \begin{macrocode}
\gdef\theorembodyfont#1{%
%    \end{macrocode}
%    We check if the argument supplied is empty and if so put nothing
%    into the |\theorem@bodyfont| token register to allow for
%    |\theorembodyfont{}| as a mean of using the default of the
%    current |\theoremstyle|.
%    \begin{macrocode}
   \def\@tempa{#1}%
   \ifx\@tempa\@empty
    \theorem@bodyfont{}%
   \else
    \theorem@bodyfont{\reset@font#1}%
   \fi
   }
\newtoks\theorem@bodyfont
\global\theorem@bodyfont{}
%    \end{macrocode}
% \end{macro}
% \end{macro}
%
%
% \begin{macro}{\theoremheaderfont}
% \changes{v2.0g}{89/09/19}{Macro added.}
%    The font for the theorem headers is handled differently because
%    this definition applies to all theorem styles.
%    \begin{macrocode}
\gdef\theoremheaderfont#1{\gdef\theorem@headerfont{#1}%
%    \end{macrocode}
%    After using the macro once it is redefined to produce an error
%    message.
%    \begin{macrocode}
       \gdef\theoremheaderfont##1{%
        \typeout{\string\theoremheaderfont\space should be used
                 only once.}}}
%    \end{macrocode}
% \end{macro}
%
% \begin{macro}{\theorem@headerfont}
% \changes{v2.0g}{89/09/19}{Macro added.}
%    To set the |\theorem@headerfont| default we first test if
%    the new fontselection scheme is in force.
% \changes{v2.1a}{90/02/15}{Support for new font selection.}
%    \begin{macrocode}
\ifx\upshape\undefined
%    \end{macrocode}
%    If not we define it to expand into |\bfseries|. We don't use
%    |\let| just in case a following style option redefines this
%    macro.
%    \begin{macrocode}
\gdef\theorem@headerfont{\bfseries}
%    \end{macrocode}
%    Otherwise we reset the current shape before calling |\bfseries|.
%    \begin{macrocode}
\else \gdef\theorem@headerfont{\normalfont\bfseries}\fi
%    \end{macrocode}
% \end{macro}
%
% \begin{macro}{\th@plain}
% \begin{macro}{\th@break}
% \begin{macro}{\th@marginbreak}
% \begin{macro}{\th@changebreak}
% \begin{macro}{\th@change}
% \begin{macro}{\th@margin}
%    The different styles are defined in macros such as
%    |\th@plain|.  Since memory space is precious in
%    ``non-Big-versions'', we have to avoid offering too many unused
%    definitions.  Therefore we define these styles in separate files
%    that can be loaded on demand.  Thus the commands themselves only
%    load these files. We use |\@input@|, a \LaTeXe{} internal command
%    that ensures that the file will be listed with |\listfiles|.
% \changes{v2.2a}{94/02/02}{Use `@input@ to load theorem layout files}
%    \begin{macrocode}
\gdef\th@plain{\@input@{thp.sty}}
\gdef\th@break{\@input@{thb.sty}}
\gdef\th@marginbreak{\@input@{thmb.sty}}
\gdef\th@changebreak{\@input@{thcb.sty}}
\gdef\th@change{\@input@{thc.sty}}
\gdef\th@margin{\@input@{thm.sty}}
%    \end{macrocode}
%    This list will be expanded when new styles become
%    available.  For testing,  just append new theorem substyles
%    as document options.
% \end{macro}
% \end{macro}
% \end{macro}
% \end{macro}
% \end{macro}
% \end{macro}
%
%
%
% \subsection{Definition of a new theorem set}
%
% As already pointed out, a new theorem environment can be defined in
% three different ways:
% \begin{verbatim}
%    \newtheorem{Lem}{Lemma}
%    \newtheorem{Lem}{Lemma}[section]
%    \newtheorem{Lem}[Theorem]{Lemma}
%\end{verbatim}
% The function of the macro |\newtheorem| is to recognize these cases
% and then to branch into one of the three macros |\@ynthm|, |\@xnthm|
% or |\@othm|.  This mechanism is adopted unchanged from
% \cite{latex:source}; the essential point here is that, for example,
% in the second case, the arguments \texttt{Lem}, \texttt{Lemma} and
% \texttt{section} are passed over to the macro |\@xnthm|.
%
% We inspect this case first because the others present fewer problems,
% and thus are easily derived from this one.
%
% \begin{macro}{\@xnthm}
%    For our example arguments, the macro |\@xnthm| must fulfill
%    the following:
%    \begin{itemize}
%       \item
%          Define a new \LaTeX-counter `Lem'
%       \item
%          reset this counter within a |\section|
%       \item
%          define the macro |\theLem|
%       \item
%          define the environment macros |\Lem| and
%          |\endLem| using the current
%          |\theoremstyle| and |\theorem@bodyfont|.
%    \end{itemize}
%    Obviously, all this should happen only if the first argument
%    of |\@xnthm| (i.e.\ \texttt{Lem} in our example) is chosen so
%    as not to conflict with any previously defined commands or
%    environments.  This test is performed by the \LaTeX{} macro
%    |\@ifdefinable|.
%    \begin{macrocode}
\gdef\@xnthm#1#2[#3]{\expandafter\@ifdefinable\csname #1\endcsname
%    \end{macrocode}
%    Therefore, the first argument of |\@ifdefinable| is the
%    expansion (in the example, |\Lem|) of
%    |\csname||#1\endcsname|.  The second argument is
%    executed only if the test has been completed successfully.
%    \begin{macrocode}
   {%
%    \end{macrocode}
%    Now we define the new counter. The names of the \LaTeX{}
%    macros employed should speak for themselves:
% \changes{v2.2b}{1995/10/07}{Use \cs{@newctr}}
% \changes{v2.2c}{1995/11/23}{Use \cs{@newctr} properly.}
%    \begin{macrocode}
    \@definecounter{#1}\@newctr{#1}[#3]%
%    \end{macrocode}
%    Using |\@newctr| will give a proper error message if the counter
%    in |#3| is not defined.
%    In defining `|\theLem|' we must generate the desired macro
%    name by use of |\expandafter| and |\csname|.
%    \begin{macrocode}
    \expandafter\xdef\csname the#1\endcsname
%    \end{macrocode}
%    An |\xdef| is used in order to make the definition global,
%    and to ensure that it contains the replacement texts of
%    |\@thmcountersep| and |\@thmcounter|.\footnote{These
%    two macros can be defined by the document style.  Their default
%    values produce a `\texttt{.}' as separation and an arabic
%    representation of the number.} However, not everything should be
%    expanded. For example, it saves space to use |\thesection|
%    instead of its---at times---lengthy expansion.
%    \begin{macrocode}
      {\expandafter \noexpand \csname the#3\endcsname
       \@thmcountersep \@thmcounter{#1}}%
%    \end{macrocode}
%    Thus with the defaults of \LaTeX{}, |\theLem| would be
%    replaced by the command sequence |\thesection.\arabic{Lem}|.
%
%    We will now look at the definition of the macro which is executed
%    at the beginning of the actual environment (in our example this
%    macro is |\Lem|). It should be noted that we use an
%    ``|\expandafter| trick'' to expand only certain parts of the
%    replacement text at the time of the definition.
%    \begin{macrocode}
    \def\@tempa{\global\@namedef{#1}}%
    \expandafter \@tempa \expandafter{%
%    \end{macrocode}
%    First, the macro that contains the current definitions of
%    |\@begintheorem| and |\@opargtheorem| should be called
%    up.  The name of this macro---as is already known---has the form
%    |\th@|\meta{theorem\ style}; therefore, it must be called by
%    \begin{macrocode}
      \csname th@\the \theorem@style
%    \end{macrocode}
%    In addition the default theorem font should be changeable, i.e.\
%    we have to insert the contents of |\theorem@bodyfont|. For
%    that reason, we expand even further, beyond |\endcsname|,
%    and thus insert the contents of the \textsf{token} register
%    |\theorem@bodyfont| in the replacement text.
%    \begin{macrocode}
            \expandafter \endcsname \the \theorem@bodyfont
%    \end{macrocode}
%    Now it is time to call the macro |\@thm| which takes over
%    the further processing.  It has two arguments: the current
%    counter name (in our example, \texttt{Lem}), and the text of the
%    label (in our example, \texttt{Lemma}).
%    \begin{macrocode}
     \@thm{#1}{#2}}%
%    \end{macrocode}
%    With this, the `sub-definition' is complete.  The macro
%    |\@endtheorem| ends a theorem environment and is, so far, nothing
%    but an |\endtrivlist|.  (Hence it is defined globally, and not
%    within the theorem styles.\footnote{This has to be changed as
%    soon as theorem styles that change \texttt{\bslash @endtheorem}
%    exist.  In such a case, all existing styles must be changed as
%    well since they will have to reset the macro.}) Therefore, we can
%    set it equivalent to the macro that ends the theorem set (in our
%    example, |\endLem|).  However, if some day theorem styles exist
%    that do change |\@endtheorem|, we would have to use the
%    commented-out line instead.
%    \begin{macrocode}
    \global \expandafter \let \csname end#1\endcsname \@endtheorem
%   \global\@namedef{end#1}{\@endtheorem}%
%    \end{macrocode}
%    With these commands all the required definitions are employed,
%    unless the test |\@ifdefinable| has failed.  Therefore, we
%    end the second argument of this macro and with it the definition
%    of |\@xnthm|.
%    \begin{macrocode}
   }}
%    \end{macrocode}
% \end{macro}
%
% \begin{macro}{\@ynthm}
%    The definition of |\@ynthm| is completely analogous. In
%    this case the new counter that is defined is not reset within
%    another counter; thus the definition of |\the...|
%    is simplified:
%    \begin{macrocode}
\gdef\@ynthm#1#2{\expandafter\@ifdefinable\csname #1\endcsname
   {\@definecounter{#1}%
    \expandafter\xdef\csname the#1\endcsname{\@thmcounter{#1}}%
%    \end{macrocode}
%    The rest of the definition corresponds literally to that of
%    |\@xnthm|:
%    \begin{macrocode}
    \def\@tempa{\global\@namedef{#1}}\expandafter \@tempa
     \expandafter{\csname th@\the \theorem@style \expandafter
     \endcsname \the\theorem@bodyfont \@thm{#1}{#2}}%
    \global \expandafter \let \csname end#1\endcsname \@endtheorem}}
%    \end{macrocode}
% \end{macro}
%
% \begin{macro}{\@othm}
%    The definition of |\@othm| does not contain anything new.
%    \begin{macrocode}
\gdef\@othm#1[#2]#3{%
%    \end{macrocode}
% \changes{v2.2b}{95/10/07}{Build in check for counter defined}
%    We do not define a new counter but instead use one that has
%    already been defined.  Thus the only definition we need is that
%    of this pseudo-counter (i.e.\ |\the|\meta{env.\ name}).
%    First we check if |#2| corresponds to a known counter name.
%    \begin{macrocode}
  \expandafter\ifx\csname c@#2\endcsname\relax
   \@nocounterr{#2}%
  \else
   \expandafter\@ifdefinable\csname #1\endcsname
   {\expandafter \xdef \csname the#1\endcsname
     {\expandafter \noexpand \csname the#2\endcsname}%
%    \end{macrocode}
%    All other parts of the definition can be adopted from
%    |\@xnthm|. We have to remember, though, that in this case
%    the name of the current counter and the theorem label have moved
%    to the second and third arguments.
%    \begin{macrocode}
    \def\@tempa{\global\@namedef{#1}}\expandafter \@tempa
     \expandafter{\csname th@\the \theorem@style \expandafter
     \endcsname \the\theorem@bodyfont \@thm{#2}{#3}}%
    \global \expandafter \let \csname end#1\endcsname \@endtheorem}%
  \fi}
%    \end{macrocode}
% \end{macro}
%
% \subsection{Macros that are employed in a {\envfont theorem}
%    environment}
%
% \begin{macro}{\@thm}
%    The macro |\@thm| has to increase the current counter.  Then,
%    depending on whether the environment has (or does not have) an
%    optional argument, it has to branch into either |\@begintheorem|
%    or |\@opargtheorem|.
%    \begin{macrocode}
\gdef\@thm#1#2{\refstepcounter{#1}%
%    \end{macrocode}
%    Now we start a {\envfont trivlist} environment, and give
%    |\@topsep| and |\@topsepadd| the values of the \textsf{skip}
%    registers |\theorempreskipamount| and |\theorempostskipamount|.
%    The value in |\@topsep| is the vertical space that is inserted by
%    the first (and only) |\item| in our |\trivlist| whilst
%    |\@topsepadd| is inserted by |\@endparenv| at the end of that
%    {\envfont trivlist} environment.  By using these registers, we
%    obtain the desired space around a {\envfont theorem} environment.
%    \begin{macrocode}
   \trivlist
   \@topsep \theorempreskipamount               % used by first \item
   \@topsepadd \theorempostskipamount           % used by \@endparenv
%    \end{macrocode}
%    Now we have to test whether an optional argument has been given.
% \changes{v2.0d}{89/07/18}{Forgotten percent added.}
%    \begin{macrocode}
   \@ifnextchar [%
%    \end{macrocode}
%    If there is an optional argument, we will call |\@ythm|, and
%    move the arguments read back into the input stream.
%    \begin{macrocode}
   {\@ythm{#1}{#2}}%
%    \end{macrocode}
%    If not, we call |\@begintheorem|. Its first argument is the
%    name of the theorem set (hence the second argument of
%    |\@thm|). Its second argument is the macro that produces the
%    current number.
%    \begin{macrocode}
   {\@begintheorem{#2}{\csname the#1\endcsname}\ignorespaces}}
%    \end{macrocode}
% \end{macro}
%
% \begin{macro}{\@xthm}
% \begin{macro}{\@ythm}
%    Both these macros were originally called by |\@thm|.  We do
%    not need |\@xthm| anymore, hence we reset it to
%    |\relax|. The definition of |\@ythm| has not changed at
%    all from its definition in \LaTeX. In order to make the macros
%    easier to understand, we will
%    nevertheless present it (commented out).
%    \begin{macrocode}
\global\let\@xthm\relax
% \def\@ythm#1#2[#3]{\@opargbegintheorem{#2}{\csname
%           the#1\endcsname}{#3}\ignorespaces}
%</package>
%    \end{macrocode}
%    The primitive |\ignorespaces| in |\@ythm| and
%    |\@thm| is needed to remove the spaces between the
%    |\begin{...}| and the actual text.
% \end{macro}
% \end{macro}
%
%
% \subsection{Definition of the theorem substyles}
%
% As already pointed out, the theorem substyles, defined below, are
% only loaded when necessary.  Note that all these substyles, except
% \texttt{plain}, have |\slshape| as the default body font.
%
% \subsubsection{The \texttt{plain} style}
%
% \label{dok:th@plain} As the following macros use |@|, we have
% to locally set the |\catcode| of this symbol to ``letter''.
% This happens within a group, so that we do not have to worry about
% which |\catcode| that symbol had before.
%    \begin{macrocode}
%<*thp>
\begingroup \makeatletter
%    \end{macrocode}
% Since we are now within a group, we must make all definitions
% globally.  First we make sure that \texttt{theorem.sty} is loaded.
% This will allow us to use this file as a document style option
% without having to call \texttt{theorem} itself as an option.  At the
% same time, we assure that at least version 2 is loaded, since
% |\theorem@style| was not defined in earlier versions.
%    \begin{macrocode}
\@ifundefined{theorem@style}{% \iffalse meta-comment
%
% Copyright (C) 1993-2024
%
% The LaTeX Project and any individual authors listed elsewhere
% in this file.
%
% This file is part of the Standard LaTeX `Tools Bundle'.
% -------------------------------------------------------
%
% It may be distributed and/or modified under the
% conditions of the LaTeX Project Public License, either version 1.3c
% of this license or (at your option) any later version.
% The latest version of this license is in
%    https://www.latex-project.org/lppl.txt
% and version 1.3c or later is part of all distributions of LaTeX
% version 2005/12/01 or later.
%
% The list of all files belonging to the LaTeX `Tools Bundle' is
% given in the file `manifest.txt'.
%
% \fi
%
%%
%
%
\def\FMithmInfo{2023/07/05 v2.2c Theorem extension package (FMi)}
%
% \ProvidesFile{theorem.dtx}[\FMithmInfo]
%\iffalse   % this is a METACOMMENT !
%%
%% Package `theorem' to use with LaTeX2e
%% Copyright (C) 1989-1995 Frank Mittelbach, all rights reserved.
%<+package>\@ifundefined{theorem@style}{}{\endinput}
%<+package>\NeedsTeXFormat{LaTeX2e}[1995/06/01]
%<+package>\ProvidesPackage{theorem}
%<+package>         [\FMithmInfo]
%
% \fi
%
% \changes{v1.0}{87/05/07}{First draft.}
% \changes{v1.0a}{87/05/07}{`thistheoremstyle' added.}
% \changes{v1.1a}{87/07/29}{New concept: redefine the internal
%                          \protect\LaTeX{} commands `@xnthm' etc.}
% \changes{v1.1b}{87/07/31}{`theoremfont' used to specify font.}
% \changes{v1.2a}{88/01/08}{`@endtheorem' now globally defined.}
% \changes{v1.2b}{88/01/11}{`newtheorem' only allowed in the preamble.}
% \changes{v1.3a}{88/02/14}{`theoremfont' now a toks register.}
%
% \changes{v2.0a}{88/11/16}{New implementation and new documentation.}
% \changes{v2.0a}{88/11/16}{No layout changes but internal modifications
%                           old substyles won't work.}
% \changes{v2.0b}{89/01/30}{`global' used for all definitions.}
% \changes{v2.0c}{89/06/15}{Documentation translated to English.
%                            Thanks to LLL and Nina}
% \changes{v2.1a}{90/02/15}{Support for new font selection.}
% \changes{v2.2a}{94/02/03}{Cleanup for LaTeX2e}
%
% \DoNotIndex{\@endparenv,\@ifdefinable,\@ifnextchar,\@ifundefined}
% \DoNotIndex{\@namedef,\@nameuse,\@spaces,\@tempa,\@warning}
% \DoNotIndex{\ }
% \DoNotIndex{\begingroup,\bfseries}
% \DoNotIndex{\csname}
% \DoNotIndex{\def,\do,\docdate}
% \DoNotIndex{\endcsname,\endgroup,\endinput,\endtrivlist,\expandafter}
% \DoNotIndex{\filedate,\fileversion}
% \DoNotIndex{\gdef,\global}
% \DoNotIndex{\hbox,\hskip}
% \DoNotIndex{\ignorespaces,\input,\itshape,\item}
% \DoNotIndex{\labelsep,\let,\llap}
% \DoNotIndex{\makeatletter}
% \DoNotIndex{\noexpand}
% \DoNotIndex{\refstepcounter,\relax,\rlap}
% \DoNotIndex{\setlength,\slshape,\space,\string,\strut}
% \DoNotIndex{\the,\trivlist,\typeout}
% \DoNotIndex{\vbox}
% \DoNotIndex{\xdef}
%
% \def\deflabel#1{\ttfamily #1\hfill}
% \def\deflist#1{\begin{list}{}{\settowidth\labelwidth{\ttfamily #1}%
%                               \setlength\leftmargin\labelwidth
%                               \addtolength\leftmargin\labelsep
%                               \let\makelabel\deflabel}}
% \def\enddeflist{\end{list}}
% \def\envfont{\normalfont\ttfamily}
%
% \MakeShortVerb{\|}
%
% \newcommand{\TUB}{{\sl TUGboat\/}}
%
% \GetFileInfo{theorem.dtx}
% \title{An Extension of the
%        \LaTeX{} theorem environment\thanks{This file
%        has version number \fileversion, last
%        revised \filedate.}}
% \author{Frank Mittelbach}
%
% \MaintainedByLaTeXTeam{tools}
% \maketitle
%
% \begin{abstract}
%    The macros described in this paper yield an extension of the
%    \LaTeX{} theorem mechanism. It is designed to satisfy the
%    different requirements of various journals. Thus, the
%    layout of the ``theorems'' can be manipulated by determining a
%    ``style''. This article describes not only the use, but
%    also the definition, of the necessary macros.
% \end{abstract}
%
% \section*{Preface to version 2.2}
%
% For \LaTeXe{} this package did not need any fundamental changes. I
% only modified the messages generated so that theorem layout styles
% will show up with the |\listfiles| command and cleaned the section
% on the New Font Selection Scheme since this is now included in
% \LaTeX.
%
% \section*{Preface to version 2.1}
%
% This version is identical to 2.0g described in \TUB\ 10\#3 except
% for some internal defaults which are now set depending on the used
% font selection scheme.
%
% This was done to avoid unpleasant surprises if the new font selection
% scheme is in force. For further details see
% section~\ref{sec:newfonts} and \cite{bk:GMS94}.
%
% \section{Introduction}
%
% For our purposes here, ``theorems'' are labelled enunciations,
% often set off from the main text by extra space and a font change.
% Theorems, corollaries, conjectures, definitions, and remarks
% are all instances of ``theorems''.  The ``header'' of these structures
% is composed of a label (such as \textsc{Theorem} or \textsc{Remark})
% and a number which serializes an item in the sequence of items
% with the same label.
%
% Shortly after the introduction of \LaTeX{} at the Fachbereich
% Mathematik in Mainz, the desire to manipulate the layout of
% ``theorems'' arose.  In Mainz, the following two conventions came
% into general use:
% \begin{enumerate}
%   \item The number of the theorem is shown in the margin.
%   \item There is a line break at
%         the end of the theorem header.
% \end{enumerate}
%
% Additionally, some journals require different formats which depend on
% the ``sort of theorem'': e.g.\ often remarks and definitions are set
% in |\upshape|, while |\itshape| is employed for main theorems.
%
% Confronted with these requirements, a theorem environment was
% developed in Mainz which allows separate determination of the
% layout of the ``theorems sets'', comparable to |\pagestyle|.
%
%
% \section{The user interface}
% \subsection{Defining new theorem sets}
%
% \DescribeMacro\newtheorem
% As in the original \LaTeX{} version, the command |\newtheorem|
% defines a new ``theorem set'' or ``theorem-like structure''.
% Two required arguments name the new environment and give the text
% to be typeset with each instance of the new ``set'', while
% an optional argument determines how
% the ``set'' is enumerated:
% \begin{description}
%    \item[\ttfamily \bslash newtheorem\{foo\}\{bar\}]
%       The theorem set {\envfont foo} (whose name is \texttt{bar})
%       uses its own counter.
%    \item[\ttfamily \bslash newtheorem\{foo2\}{[foo]}\{bar2\}]
%       The theorem set {\envfont foo2} (printed name \texttt{bar2})
%       uses the same counter as the theorem set \texttt{foo}.
%    \item[\ttfamily \bslash newtheorem\{foo3\}\{bar3\}{[section]}]
%       The theorem set {\envfont foo3} (printed name \texttt{bar3}) is
%       enumerated within the counter \texttt{section}, i.e.\ with every
%       new |\section| the enumeration begins again with 1, and
%       the enumeration is composed from the section-number and the
%       theorem counter itself.
% \end{description}
%
% \DescribeMacro\theoremstyle
% Additionally, the command |\theoremstyle| can define the layout
% of various, or all, theorem sets.  It should be noted that any
% theorem set defined by |\newtheorem| is typeset in the
% |\theoremstyle| that is current at the time of the definition.
% Thus, the following
% \begin{verbatim}
%    \theoremstyle{break}        \newtheorem{Cor}{Corollary}
%    \theoremstyle{plain}        \newtheorem{Exa}{Example}[section]
%\end{verbatim}                             ^^A No blank at line start
% leads to the result that the set {\envfont Cor} is formatted in the
% style \texttt{break}, while the set {\envfont Exa} and all the
% following ones are formatted in the style \texttt{plain}, unless
% another |\theoremstyle| follows.  Since the definitions installed by
% |\newtheorem| are global, one also can limit |\theoremstyle| locally
% by grouping braces.
%
% \DescribeMacro\theorembodyfont
% The choice of the font for the theorem body is completely
% independent of the chosen |\theoremstyle|; this has proven to
% be very advantageous.  For example,
% \begin{verbatim}
%   {\theorembodyfont{\upshape}          \newtheorem{Rem}{Remark}}
%\end{verbatim}
% defines a theorem set {\envfont Rem}, which will be set in
% |\upshape| in the current layout (which in our example is
% \texttt{plain}).  As with |\theoremstyle|, the |\theorembodyfont|
% chosen is that current at the time of |\newtheorem|.  If
% |\theorembodyfont| is not specified or one defines
% |\theorembodyfont{}|, then the font used will be that defined by the
% |\theoremstyle|.
%
% \DescribeMacro\theoremheaderfont It is also possible to customize
% the font used for the theorem headers.  This is, however, a global
% declaration, and therefore there should be at most one
% |\theoremheaderfont| declaration in the preamble.\footnote{If it is
% actually necessary to have different header fonts, one has to define
% new theorem styles (substituting the desired font) or specify the
% information directly in the \texttt{\bslash newtheorem} declaration
% (the unclean variant).}
%
% \DescribeMacro\theorempreskipamount
% \DescribeMacro\theorempostskipamount Two additional parameters
% affect the vertical space around the theorem environments:
% |\theorempreskipamount| and |\theorempostskipamount| define,
% respectively, the spacing before and after such an environment.
% These parameters apply for all theorem sets and can be manipulated
% with the ordinary length macros.  They are rubber lengths
% (`\textsf{skips}'), and therefore can contain \texttt{plus} and
% \texttt{minus} parts.
%
% Since the definition of theorem sets should---most sensibly---be
% placed in the preamble, we only allow installation there.
% It is therefore possible to release the memory used here after
% |\begin{document}|, in order to make room for other
% applications.
%
%
% \subsection{Existing theorem styles}
%
% The following theorem styles exist to date:
% \begin{deflist}{marginbreak}
%    \item[plain]
%       This theorem style emulates the original \LaTeX{} definition,
%       except that additionally the parameters
%       |\theorem...skipamount| are used.
%    \item[break]
%       In this style, the theorem header is followed by a line
%       break.
%    \item[marginbreak]
%       The theorem number is set in the margin, and there is a line
%       break as in \texttt{break}.
%    \item[changebreak]
%       Like \texttt{break}, but with header number and text
%       interchanged.
%    \item[change]
%       Header number and text are interchanged, without a line break.
%    \item[margin]
%       The number is set in the left margin, without a line break.
% \end{deflist}
% All styles (except \texttt{plain}) select |\slshape| as the default
% |\theorembodyfont|.
%
%
% \subsection{Examples}
%
% Given the above theorem sets {\envfont Cor}, {\envfont Exa}
% and {\envfont Rem},
% suppose that the preamble also contains the declarations:
% \begin{verbatim}
%    \theoremstyle{marginbreak}   \newtheorem{Lem}[Cor]{Lemma}
%    \theoremstyle{change}
%    \theorembodyfont{\itshape}        \newtheorem{Def}[Cor]{Definition}
%
%    \theoremheaderfont{\scshape}
%\end{verbatim}
% Then the following are some typical examples of the typeset output
% resulting from their use.
%
% \begin{Cor}
%    This is a sentence typeset in the theorem environment {\envfont
%    Cor}.
% \end{Cor}
%
% \begin{Exa}
%    This is a sentence typeset in the theorem environment {\envfont
%    Exa}.
% \end{Exa}
%
% \begin{Rem}
%    This is a sentence typeset in the theorem environment {\envfont
%    Rem}.
% \end{Rem}
%
% \begin{Lem}[Ben User]
%    This is a sentence typeset in the theorem environment {\envfont
%    Lem}.
% \end{Lem}
%
% \begin{Def}[Very impressive Definition]
%    This is a sentence typeset in the theorem environment {\envfont
%    Def}.
% \end{Def}
%
% The last two examples show the effect of the optional argument to
% a theorem environment (it is the text typeset in parentheses).
%
%
% \section{Special Considerations}
% \label{sec:newfonts}
%
% Theoremheader and body are implemented as a unit. This means that
% the |\theoremheaderfont|  will inherit characteristics of the
% |\theorembodyfont| in \LaTeXe. Thus, if for example
% |\theorembodyfont| is |\itshape| and |\theoremheaderfont|
% is |\bfseries| the font selected for the header will have the
% characteristics `bold extended italic'. If this is not desired one
% should set the |\theoremheaderfont| to something like
% \begin{verbatim}
%      \theoremheaderfont{\normalfont\bfseries}
%\end{verbatim}
% i.e.\ supplying all necessary font information explicitly.
%
%
%
% \section{Acknowledgements}
%
% The publication of this set of macros was only possible with the
% help of Christina Busse (translating the manuscript into English),
% Joachim Pense (playing the r\^ole of typist), Chris Rowley
% (looking everything over) and many others providing useful
% suggestions.
%
% \MaybeStop{%
%    \begin{thebibliography}{9}
%       \bibitem{bk:GMS94} \textsc{M.~Goossens}, \textsc{F.~Mittelbach}
%         and \textsc{A.~Samarin}.
%         \newblock The \LaTeX{} Companion.
%         \newblock
%         Addison-Wesley, Reading, Massachusetts, 1994.
%       \bibitem{latex:source} \textsc{Lamport, Leslie.} \newblock
%                               \texttt{latex.tex}, version 2.09, date
%                               Feb. 1990.
%    \end{thebibliography}
%    \PrintIndex
%    \PrintChanges}
%
%
% \section{The documentation driver file}
%
% The next bit of code contains the documentation driver file for
% \TeX{}, i.e., the file that will produce the documentation you are
% currently reading. It will be extracted from this file by the
% \texttt{docstrip} program. Since it is the first code in the file
% one can alternatively process this file directly with \LaTeXe{} to
% obtain the documentation.
%
%    \begin{macrocode}
%<*driver>
\documentclass{ltxdoc}

\usepackage{theorem}

%<+driver>% The next few lines define theorem sets which are used
%<+driver>% in the example section of the documentation.

\theoremstyle{break}       \newtheorem{Cor}{Corollary}
\theoremstyle{plain}       \newtheorem{Exa}{Example}[section]
{\theorembodyfont{\upshape}\newtheorem{Rem}{Remark}}
\theoremstyle{marginbreak} \newtheorem{Lem}[Cor]{Lemma}
\theoremstyle{change}
\theorembodyfont{\itshape} \newtheorem{Def}[Cor]{Definition}

\theoremheaderfont{\scshape}

\RecordChanges

\begin{document}
   \DocInput{theorem.dtx}
\end{document}
%</driver>
%    \end{macrocode}
%
% \section{Definition of the Macros}
%
% If the file has been loaded before, we abort immediately. If not the
% package announces itself (this is actually done at the very top of
% the file---the way it is done isn't good style so don't copy it).
% \changes{v2.0e}{89/07/19}{Spaces removed from `typeout'.}
% \changes{v2.2b}{95/11/19}{Announce moved to top of file}
%    \begin{macrocode}
%<*package>
%\@ifundefined{theorem@style}{}{\endinput}
%\def\FMithmInfo{1995/11/19 v2.2b Theorem extension package (FMi)}
%\ProvidesPackage{theorem}[\FMithmInfo]
%    \end{macrocode}
%
%
% \subsection{Definition of theorem styles and fonts}
%
% All the definitions in this file are done globally to allow
% inputting this file inside a group.
%
% \begin{macro}{\theoremstyle}
%    Before a theorem style can be installed, the chosen style must be
%    known. For that reason, we must test to see that
%    |\th@|\meta{style} is known or, more precisely, that it is
%    different from |\relax|. If the style is not known then
%    |\th@plain| is used.
%    \begin{macrocode}
\gdef\theoremstyle#1{%
   \@ifundefined{th@#1}{\@warning
          {Unknown theoremstyle `#1'. Using `plain'}%
          \theorem@style{plain}}%
%    \end{macrocode}
%    We save the theorem style to be used in the token register
%    |\theorem@style|.
%    \begin{macrocode}
      {\theorem@style{#1}}%
%    \end{macrocode}
%    Now we ``evaluate'' the theorem style: this means, we call the
%    macro |\th@|\meta{style} which will activate the relevant
%    definitions which are contained in a separate file.  This is done
%    in a group to suppress changes to the current font.  This could
%    otherwise pose problems together with the new font selection
%    scheme\footnote{When I printed the original article using the new
%    font selection scheme I ended with a document with slanted
%    typefaces (text headings and all) simply because one of the
%    theorem styles used \texttt{\bslash sl} at toplevel.} if the
%    |\th@|\meta{style} is evaluated a second time.
% \changes{v2.1a}{90/02/15}{Group added around `th@..}
% \changes{v2.1a}{90/02/15}{Reseting of `..@begintheorem removed.}
%    \begin{macrocode}
      \begingroup
        \csname th@\the\theorem@style \endcsname
      \endgroup}
%    \end{macrocode}
% \end{macro}
%
%
% \begin{macro}{\@begintheorem}
% \changes{v2.1a}{90/02/15}{Relaxed at toplevel.}
% \begin{macro}{\@opargbegintheorem}
% \changes{v2.1a}{90/02/15}{Relaxed at toplevel.}
%    We reset |\@begintheorem| and |\@opargbegintheorem| to
%    |\relax| since these commands are no longer necessary at
%    toplevel. This will save a few tokens.
%    \begin{macrocode}
\global\let\@begintheorem\relax
\global\let\@opargbegintheorem\relax
%    \end{macrocode}
% \end{macro}
% \end{macro}
%
% \begin{macro}{\theorem@style}
%    Obviously the \textsf{token} register used above has to be
%    allocated. To assure the utmost compatibility with the
%    original \LaTeX{} definition, we set the default theorem style to
%    \texttt{plain}, which implements the usual \LaTeX{} convention.
% \changes{v2.0f}{89/07/20}{Unnecessary `global' removed.}
%    \begin{macrocode}
\newtoks\theorem@style
\global\theorem@style{plain}
%    \end{macrocode}
% \end{macro}
%
% \begin{macro}{\theorembodyfont}
% \begin{macro}{\theorem@bodyfont}
%    If the |\theorembodyfont| is set by the user then it should not
%    interact with the default font set in the theorem style. When the
%    new font selection is in force this may happen if, for example,
%    the default is |\itshape| and the new |\theorembodyfont| is
%    |\sffamily|. So we add a |\reset@font| command in front of the
%    user definition.
% \changes{v2.0f}{89/07/20}{Unnecessary `global' removed.}
% \changes{v2.0g}{89/09/19}{`theoremfont to `theorembodyfont.}
% \changes{v2.1c}{92/06/22}{`reset@font added and logic changed.}
% \changes{v2.1d}{92/10/28}{Reset to empty if arg is empty}
%    \begin{macrocode}
\gdef\theorembodyfont#1{%
%    \end{macrocode}
%    We check if the argument supplied is empty and if so put nothing
%    into the |\theorem@bodyfont| token register to allow for
%    |\theorembodyfont{}| as a mean of using the default of the
%    current |\theoremstyle|.
%    \begin{macrocode}
   \def\@tempa{#1}%
   \ifx\@tempa\@empty
    \theorem@bodyfont{}%
   \else
    \theorem@bodyfont{\reset@font#1}%
   \fi
   }
\newtoks\theorem@bodyfont
\global\theorem@bodyfont{}
%    \end{macrocode}
% \end{macro}
% \end{macro}
%
%
% \begin{macro}{\theoremheaderfont}
% \changes{v2.0g}{89/09/19}{Macro added.}
%    The font for the theorem headers is handled differently because
%    this definition applies to all theorem styles.
%    \begin{macrocode}
\gdef\theoremheaderfont#1{\gdef\theorem@headerfont{#1}%
%    \end{macrocode}
%    After using the macro once it is redefined to produce an error
%    message.
%    \begin{macrocode}
       \gdef\theoremheaderfont##1{%
        \typeout{\string\theoremheaderfont\space should be used
                 only once.}}}
%    \end{macrocode}
% \end{macro}
%
% \begin{macro}{\theorem@headerfont}
% \changes{v2.0g}{89/09/19}{Macro added.}
%    To set the |\theorem@headerfont| default we first test if
%    the new fontselection scheme is in force.
% \changes{v2.1a}{90/02/15}{Support for new font selection.}
%    \begin{macrocode}
\ifx\upshape\undefined
%    \end{macrocode}
%    If not we define it to expand into |\bfseries|. We don't use
%    |\let| just in case a following style option redefines this
%    macro.
%    \begin{macrocode}
\gdef\theorem@headerfont{\bfseries}
%    \end{macrocode}
%    Otherwise we reset the current shape before calling |\bfseries|.
%    \begin{macrocode}
\else \gdef\theorem@headerfont{\normalfont\bfseries}\fi
%    \end{macrocode}
% \end{macro}
%
% \begin{macro}{\th@plain}
% \begin{macro}{\th@break}
% \begin{macro}{\th@marginbreak}
% \begin{macro}{\th@changebreak}
% \begin{macro}{\th@change}
% \begin{macro}{\th@margin}
%    The different styles are defined in macros such as
%    |\th@plain|.  Since memory space is precious in
%    ``non-Big-versions'', we have to avoid offering too many unused
%    definitions.  Therefore we define these styles in separate files
%    that can be loaded on demand.  Thus the commands themselves only
%    load these files. We use |\@input@|, a \LaTeXe{} internal command
%    that ensures that the file will be listed with |\listfiles|.
% \changes{v2.2a}{94/02/02}{Use `@input@ to load theorem layout files}
%    \begin{macrocode}
\gdef\th@plain{\@input@{thp.sty}}
\gdef\th@break{\@input@{thb.sty}}
\gdef\th@marginbreak{\@input@{thmb.sty}}
\gdef\th@changebreak{\@input@{thcb.sty}}
\gdef\th@change{\@input@{thc.sty}}
\gdef\th@margin{\@input@{thm.sty}}
%    \end{macrocode}
%    This list will be expanded when new styles become
%    available.  For testing,  just append new theorem substyles
%    as document options.
% \end{macro}
% \end{macro}
% \end{macro}
% \end{macro}
% \end{macro}
% \end{macro}
%
%
%
% \subsection{Definition of a new theorem set}
%
% As already pointed out, a new theorem environment can be defined in
% three different ways:
% \begin{verbatim}
%    \newtheorem{Lem}{Lemma}
%    \newtheorem{Lem}{Lemma}[section]
%    \newtheorem{Lem}[Theorem]{Lemma}
%\end{verbatim}
% The function of the macro |\newtheorem| is to recognize these cases
% and then to branch into one of the three macros |\@ynthm|, |\@xnthm|
% or |\@othm|.  This mechanism is adopted unchanged from
% \cite{latex:source}; the essential point here is that, for example,
% in the second case, the arguments \texttt{Lem}, \texttt{Lemma} and
% \texttt{section} are passed over to the macro |\@xnthm|.
%
% We inspect this case first because the others present fewer problems,
% and thus are easily derived from this one.
%
% \begin{macro}{\@xnthm}
%    For our example arguments, the macro |\@xnthm| must fulfill
%    the following:
%    \begin{itemize}
%       \item
%          Define a new \LaTeX-counter `Lem'
%       \item
%          reset this counter within a |\section|
%       \item
%          define the macro |\theLem|
%       \item
%          define the environment macros |\Lem| and
%          |\endLem| using the current
%          |\theoremstyle| and |\theorem@bodyfont|.
%    \end{itemize}
%    Obviously, all this should happen only if the first argument
%    of |\@xnthm| (i.e.\ \texttt{Lem} in our example) is chosen so
%    as not to conflict with any previously defined commands or
%    environments.  This test is performed by the \LaTeX{} macro
%    |\@ifdefinable|.
%    \begin{macrocode}
\gdef\@xnthm#1#2[#3]{\expandafter\@ifdefinable\csname #1\endcsname
%    \end{macrocode}
%    Therefore, the first argument of |\@ifdefinable| is the
%    expansion (in the example, |\Lem|) of
%    |\csname||#1\endcsname|.  The second argument is
%    executed only if the test has been completed successfully.
%    \begin{macrocode}
   {%
%    \end{macrocode}
%    Now we define the new counter. The names of the \LaTeX{}
%    macros employed should speak for themselves:
% \changes{v2.2b}{1995/10/07}{Use \cs{@newctr}}
% \changes{v2.2c}{1995/11/23}{Use \cs{@newctr} properly.}
%    \begin{macrocode}
    \@definecounter{#1}\@newctr{#1}[#3]%
%    \end{macrocode}
%    Using |\@newctr| will give a proper error message if the counter
%    in |#3| is not defined.
%    In defining `|\theLem|' we must generate the desired macro
%    name by use of |\expandafter| and |\csname|.
%    \begin{macrocode}
    \expandafter\xdef\csname the#1\endcsname
%    \end{macrocode}
%    An |\xdef| is used in order to make the definition global,
%    and to ensure that it contains the replacement texts of
%    |\@thmcountersep| and |\@thmcounter|.\footnote{These
%    two macros can be defined by the document style.  Their default
%    values produce a `\texttt{.}' as separation and an arabic
%    representation of the number.} However, not everything should be
%    expanded. For example, it saves space to use |\thesection|
%    instead of its---at times---lengthy expansion.
%    \begin{macrocode}
      {\expandafter \noexpand \csname the#3\endcsname
       \@thmcountersep \@thmcounter{#1}}%
%    \end{macrocode}
%    Thus with the defaults of \LaTeX{}, |\theLem| would be
%    replaced by the command sequence |\thesection.\arabic{Lem}|.
%
%    We will now look at the definition of the macro which is executed
%    at the beginning of the actual environment (in our example this
%    macro is |\Lem|). It should be noted that we use an
%    ``|\expandafter| trick'' to expand only certain parts of the
%    replacement text at the time of the definition.
%    \begin{macrocode}
    \def\@tempa{\global\@namedef{#1}}%
    \expandafter \@tempa \expandafter{%
%    \end{macrocode}
%    First, the macro that contains the current definitions of
%    |\@begintheorem| and |\@opargtheorem| should be called
%    up.  The name of this macro---as is already known---has the form
%    |\th@|\meta{theorem\ style}; therefore, it must be called by
%    \begin{macrocode}
      \csname th@\the \theorem@style
%    \end{macrocode}
%    In addition the default theorem font should be changeable, i.e.\
%    we have to insert the contents of |\theorem@bodyfont|. For
%    that reason, we expand even further, beyond |\endcsname|,
%    and thus insert the contents of the \textsf{token} register
%    |\theorem@bodyfont| in the replacement text.
%    \begin{macrocode}
            \expandafter \endcsname \the \theorem@bodyfont
%    \end{macrocode}
%    Now it is time to call the macro |\@thm| which takes over
%    the further processing.  It has two arguments: the current
%    counter name (in our example, \texttt{Lem}), and the text of the
%    label (in our example, \texttt{Lemma}).
%    \begin{macrocode}
     \@thm{#1}{#2}}%
%    \end{macrocode}
%    With this, the `sub-definition' is complete.  The macro
%    |\@endtheorem| ends a theorem environment and is, so far, nothing
%    but an |\endtrivlist|.  (Hence it is defined globally, and not
%    within the theorem styles.\footnote{This has to be changed as
%    soon as theorem styles that change \texttt{\bslash @endtheorem}
%    exist.  In such a case, all existing styles must be changed as
%    well since they will have to reset the macro.}) Therefore, we can
%    set it equivalent to the macro that ends the theorem set (in our
%    example, |\endLem|).  However, if some day theorem styles exist
%    that do change |\@endtheorem|, we would have to use the
%    commented-out line instead.
%    \begin{macrocode}
    \global \expandafter \let \csname end#1\endcsname \@endtheorem
%   \global\@namedef{end#1}{\@endtheorem}%
%    \end{macrocode}
%    With these commands all the required definitions are employed,
%    unless the test |\@ifdefinable| has failed.  Therefore, we
%    end the second argument of this macro and with it the definition
%    of |\@xnthm|.
%    \begin{macrocode}
   }}
%    \end{macrocode}
% \end{macro}
%
% \begin{macro}{\@ynthm}
%    The definition of |\@ynthm| is completely analogous. In
%    this case the new counter that is defined is not reset within
%    another counter; thus the definition of |\the...|
%    is simplified:
%    \begin{macrocode}
\gdef\@ynthm#1#2{\expandafter\@ifdefinable\csname #1\endcsname
   {\@definecounter{#1}%
    \expandafter\xdef\csname the#1\endcsname{\@thmcounter{#1}}%
%    \end{macrocode}
%    The rest of the definition corresponds literally to that of
%    |\@xnthm|:
%    \begin{macrocode}
    \def\@tempa{\global\@namedef{#1}}\expandafter \@tempa
     \expandafter{\csname th@\the \theorem@style \expandafter
     \endcsname \the\theorem@bodyfont \@thm{#1}{#2}}%
    \global \expandafter \let \csname end#1\endcsname \@endtheorem}}
%    \end{macrocode}
% \end{macro}
%
% \begin{macro}{\@othm}
%    The definition of |\@othm| does not contain anything new.
%    \begin{macrocode}
\gdef\@othm#1[#2]#3{%
%    \end{macrocode}
% \changes{v2.2b}{95/10/07}{Build in check for counter defined}
%    We do not define a new counter but instead use one that has
%    already been defined.  Thus the only definition we need is that
%    of this pseudo-counter (i.e.\ |\the|\meta{env.\ name}).
%    First we check if |#2| corresponds to a known counter name.
%    \begin{macrocode}
  \expandafter\ifx\csname c@#2\endcsname\relax
   \@nocounterr{#2}%
  \else
   \expandafter\@ifdefinable\csname #1\endcsname
   {\expandafter \xdef \csname the#1\endcsname
     {\expandafter \noexpand \csname the#2\endcsname}%
%    \end{macrocode}
%    All other parts of the definition can be adopted from
%    |\@xnthm|. We have to remember, though, that in this case
%    the name of the current counter and the theorem label have moved
%    to the second and third arguments.
%    \begin{macrocode}
    \def\@tempa{\global\@namedef{#1}}\expandafter \@tempa
     \expandafter{\csname th@\the \theorem@style \expandafter
     \endcsname \the\theorem@bodyfont \@thm{#2}{#3}}%
    \global \expandafter \let \csname end#1\endcsname \@endtheorem}%
  \fi}
%    \end{macrocode}
% \end{macro}
%
% \subsection{Macros that are employed in a {\envfont theorem}
%    environment}
%
% \begin{macro}{\@thm}
%    The macro |\@thm| has to increase the current counter.  Then,
%    depending on whether the environment has (or does not have) an
%    optional argument, it has to branch into either |\@begintheorem|
%    or |\@opargtheorem|.
%    \begin{macrocode}
\gdef\@thm#1#2{\refstepcounter{#1}%
%    \end{macrocode}
%    Now we start a {\envfont trivlist} environment, and give
%    |\@topsep| and |\@topsepadd| the values of the \textsf{skip}
%    registers |\theorempreskipamount| and |\theorempostskipamount|.
%    The value in |\@topsep| is the vertical space that is inserted by
%    the first (and only) |\item| in our |\trivlist| whilst
%    |\@topsepadd| is inserted by |\@endparenv| at the end of that
%    {\envfont trivlist} environment.  By using these registers, we
%    obtain the desired space around a {\envfont theorem} environment.
%    \begin{macrocode}
   \trivlist
   \@topsep \theorempreskipamount               % used by first \item
   \@topsepadd \theorempostskipamount           % used by \@endparenv
%    \end{macrocode}
%    Now we have to test whether an optional argument has been given.
% \changes{v2.0d}{89/07/18}{Forgotten percent added.}
%    \begin{macrocode}
   \@ifnextchar [%
%    \end{macrocode}
%    If there is an optional argument, we will call |\@ythm|, and
%    move the arguments read back into the input stream.
%    \begin{macrocode}
   {\@ythm{#1}{#2}}%
%    \end{macrocode}
%    If not, we call |\@begintheorem|. Its first argument is the
%    name of the theorem set (hence the second argument of
%    |\@thm|). Its second argument is the macro that produces the
%    current number.
%    \begin{macrocode}
   {\@begintheorem{#2}{\csname the#1\endcsname}\ignorespaces}}
%    \end{macrocode}
% \end{macro}
%
% \begin{macro}{\@xthm}
% \begin{macro}{\@ythm}
%    Both these macros were originally called by |\@thm|.  We do
%    not need |\@xthm| anymore, hence we reset it to
%    |\relax|. The definition of |\@ythm| has not changed at
%    all from its definition in \LaTeX. In order to make the macros
%    easier to understand, we will
%    nevertheless present it (commented out).
%    \begin{macrocode}
\global\let\@xthm\relax
% \def\@ythm#1#2[#3]{\@opargbegintheorem{#2}{\csname
%           the#1\endcsname}{#3}\ignorespaces}
%</package>
%    \end{macrocode}
%    The primitive |\ignorespaces| in |\@ythm| and
%    |\@thm| is needed to remove the spaces between the
%    |\begin{...}| and the actual text.
% \end{macro}
% \end{macro}
%
%
% \subsection{Definition of the theorem substyles}
%
% As already pointed out, the theorem substyles, defined below, are
% only loaded when necessary.  Note that all these substyles, except
% \texttt{plain}, have |\slshape| as the default body font.
%
% \subsubsection{The \texttt{plain} style}
%
% \label{dok:th@plain} As the following macros use |@|, we have
% to locally set the |\catcode| of this symbol to ``letter''.
% This happens within a group, so that we do not have to worry about
% which |\catcode| that symbol had before.
%    \begin{macrocode}
%<*thp>
\begingroup \makeatletter
%    \end{macrocode}
% Since we are now within a group, we must make all definitions
% globally.  First we make sure that \texttt{theorem.sty} is loaded.
% This will allow us to use this file as a document style option
% without having to call \texttt{theorem} itself as an option.  At the
% same time, we assure that at least version 2 is loaded, since
% |\theorem@style| was not defined in earlier versions.
%    \begin{macrocode}
\@ifundefined{theorem@style}{\input{theorem.sty}}{}
\ProvidesFile{thp.sty}
             [\FMithmInfo]
%    \end{macrocode}
%
%
% \begin{macro}{\th@plain}
% \begin{macro}{\@begintheorem}
% \begin{macro}{\@opargbegintheorem}
%    |\theoremstyle{plain}| corresponds to the original
%    definition, except that the distances to the surrounding text are
%    determined by the parameters |\theorempreskipamount| and
%    |\theorempostskipamount|.
%    First we set the default body font.
%    \begin{macrocode}
\gdef\th@plain{\normalfont\itshape
%    \end{macrocode}
%    Then we define |\@begintheorem| and
%    |\@opargbegintheorem|. These two macros define how the
%    header of a theorem is typeset.  |\@opargbegintheorem| will
%    be called if a \textsf{theorem} environment with an optional
%    argument is encountered; otherwise, the header is constructed by
%    calling |\@begintheorem|.  If one of these macros is
%    executed, we are within a \textsf{trivlist} environment started by
%    |\@thm|.  So the theorem header is produced with an
%    |\item| command.
%
%    Instead of specifying the header font directly, all standard
%    theorem styles use the |\theorem@headerfont| macro to allow
%    customization.  The extra space (|\labelsep|) is necessary
%    because of problems in the \textsf{trivlist} environment.
%    \begin{macrocode}
  \def\@begintheorem##1##2{%
        \item[\hskip\labelsep \theorem@headerfont ##1\ ##2]}%
%    \end{macrocode}
%    The definition of |\@opargbegintheorem| is completely
%    analogous. The only difference is the fact that there exists a
%    third argument (which is the optional parameter of the
%    environment and contains additional information about the
%    theorem).  Customarily we enclose it in parentheses.
%    \begin{macrocode}
\def\@opargbegintheorem##1##2##3{%
   \item[\hskip\labelsep \theorem@headerfont ##1\ ##2\ (##3)]}}
%    \end{macrocode}
% \end{macro}
% \end{macro}
% \end{macro}
%
% \noindent
% We conclude with an |\endgroup| to restore the
% |\catcode| of |@|.
%    \begin{macrocode}
\endgroup
%</thp>
%    \end{macrocode}
%
%
% \subsubsection{The \texttt{break} style}
%
% This style option is stored in the file |thb.sty|.
% For the next two lines see the documentation for |\th@plain| on
% page~\pageref{dok:th@plain}.
%    \begin{macrocode}
%<*thb>
\begingroup \makeatletter
\@ifundefined{theorem@style}{\input{theorem.sty}}{}
\ProvidesFile{thb.sty}
             [\FMithmInfo]
%    \end{macrocode}
%
% \begin{macro}{\th@break}
%    |\theoremstyle{break}| produces a line break after the name
%    of the theorem.  The font is |\slshape|.  Hence, we define
%    |\th@break| as follows:
%    \begin{macrocode}
\gdef\th@break{\normalfont\slshape
  \def\@begintheorem##1##2{\item[%
%    \end{macrocode}
%    We run into the following problem: it is not possible to create
%    the header with |\item[|\meta{title}|]| and then start
%    a new line by, for example, |\mbox{}\\|. Such a definition
%    will fail whenever a list environment follows immediately.  With
%    the above construction, the |\mbox{}| causes the switch
%    |@inlabel| (cf.\ definition of |\list| and
%    |\trivlist| in \cite{latex:source}) to be set to \texttt{false}
%    and so the following list will insert additional vertical
%    space (|\topskip|). This is quite annoying. Therefore, we
%    create the line break within the |\item|. In order to ensure
%    that the text will begin at the proper position in the following
%    line, we simply pretend that the label does not take any
%    room.\footnote{This will lead to problems whenever very high
%    symbols occurring in the line tower into the
%    heading. So, something else has to be done here sometime.}
%    \begin{macrocode}
       \rlap{\vbox{\hbox{\hskip \labelsep\theorem@headerfont ##1\ ##2}%
                   \hbox{\strut}}}]}%
%    \end{macrocode}
%    Again, the definition of |\@opargbegintheorem| is
%    completely analogous.
%    \begin{macrocode}
\def\@opargbegintheorem##1##2##3{%
  \item[\rlap{\vbox{\hbox{\hskip \labelsep \theorem@headerfont
                     ##1\ ##2\ (##3)}%
                    \hbox{\strut}}}]}}
\endgroup
%</thb>
%    \end{macrocode}
% \end{macro}
%
%
% \subsubsection{The \texttt{changebreak} style}
%
%    \begin{macrocode}
% This style option is stored in the file |thcb.sty|.
%    \begin{macrocode}
%<*thcb>
\begingroup \makeatletter
\@ifundefined{theorem@style}{\input{theorem.sty}}{}
\ProvidesFile{thcb.sty}
             [\FMithmInfo]
%    \end{macrocode}
%
% \begin{macro}{\th@changebreak}
%    The \texttt{change-break} theorem style is like \texttt{break}
%    but with interchange of theorem name and theorem number.  Thus we
%    define |\th@changebreak| as follows:
%    \begin{macrocode}
\gdef\th@changebreak{\normalfont\slshape
  \def\@begintheorem##1##2{\item
       [\rlap{\vbox{\hbox{\hskip\labelsep \theorem@headerfont ##2\ ##1}%
                    \hbox{\strut}}}]}%
\def\@opargbegintheorem##1##2##3{%
  \item[\rlap{\vbox{\hbox{\hskip\labelsep \theorem@headerfont
                     ##2\ ##1\ (##3)}%
                    \hbox{\strut}}}]}}
\endgroup
%</thcb>
%    \end{macrocode}
% \end{macro}
%
%
% \subsubsection{The \texttt{change} style}
%
% This style option is stored in the file |thc.sty|.
%    \begin{macrocode}
%<*thc>
\begingroup \makeatletter
\@ifundefined{theorem@style}{\input{theorem.sty}}{}
\ProvidesFile{thc.sty}
             [\FMithmInfo]
%    \end{macrocode}
%
% \begin{macro}{\th@change}
%    The \texttt{change} theorem style corresponds to the
%    \texttt{change break} style without a linebreak after the header.
%    To say it in another way, it's the same as the \texttt{plain}
%    style but with number and name interchanged and |\slshape| as the
%    default font.
%    \begin{macrocode}
\gdef\th@change{\normalfont\slshape
  \def\@begintheorem##1##2{\item
       [\hskip\labelsep \theorem@headerfont ##2\ ##1]}%
\def\@opargbegintheorem##1##2##3{%
  \item[\hskip\labelsep \theorem@headerfont ##2\ ##1\ (##3)]}}
\endgroup
%</thc>
%    \end{macrocode}
% \end{macro}
%
%
% \subsubsection{The \texttt{marginbreak} style}
%
% This style option is the one used most often at Mainz.  It is saved
% in the file |thmb.sty|.
%    \begin{macrocode}
%<*thmb>
\begingroup \makeatletter
\@ifundefined{theorem@style}{\input{theorem.sty}}{}
\ProvidesFile{thmb.sty}
             [\FMithmInfo]
%    \end{macrocode}
%
% \begin{macro}{\th@marginbreak}
%    The \texttt{margin break} style is nearly the same as the
%    \texttt{change break} style. The only difference is the placement
%    of the theorem number.  We use |\llap| to place it in the left
%    margin.
%
%    In this style |\labelsep| denotes the separation between the
%    number and the text.
%    \begin{macrocode}
\gdef\th@marginbreak{\normalfont\slshape
  \def\@begintheorem##1##2{\item
       [\rlap{\vbox{\theorem@headerfont
                    \hbox{\llap{##2}\hskip\labelsep ##1}%
                    \hbox{\strut}}}]}%
\def\@opargbegintheorem##1##2##3{%
  \item[\rlap{\vbox{\theorem@headerfont
                    \hbox{\llap{##2}\hskip\labelsep ##1\ (##3)}%
                    \hbox{\strut}}}]}}
\endgroup
%</thmb>
%    \end{macrocode}
% \changes{v2.1c}{90/02/17}{`endgroup outside macrocode env.}
% \end{macro}
%
%
%
% \subsubsection{The \texttt{margin} style}
%
% This style option is stored in the file |thm.sty|.
%    \begin{macrocode}
%<*thm>
\begingroup \makeatletter
\@ifundefined{theorem@style}{\input{theorem.sty}}{}
\ProvidesFile{thm.sty}
             [\FMithmInfo]
%    \end{macrocode}
%
% \begin{macro}{\th@margin}
%    Again this is only a variant of the theorem styles described
%    above without any new ideas.
%    \begin{macrocode}
\gdef\th@margin{\normalfont\slshape
  \def\@begintheorem##1##2{\item
       [\theorem@headerfont \llap{##2}\hskip\labelsep ##1]}%
\def\@opargbegintheorem##1##2##3{%
  \item[\theorem@headerfont \llap{##2}\hskip\labelsep ##1\ (##3)]}}
\endgroup
%</thm>
%    \end{macrocode}
% \end{macro}
%
%
%
%
% \subsection{Final Definitions}
%
% \begin{macro}{\theorempreskipamount}
% \begin{macro}{\theorempostskipamount}
%    The \textsf{skip} parameters that regulate the vertical empty space
%    before and after the {\envfont theorem} environment have
%    to be allocated as well.
% \changes{v2.0f}{89/07/20}{Unnecessary `global' removed.}
%    \begin{macrocode}
%<*package>
\newskip\theorempreskipamount
\newskip\theorempostskipamount
%    \end{macrocode}
%    Since we have used the same values for all theorem sets, we now can
%    assign them.
%    \begin{macrocode}
\global\setlength\theorempreskipamount{12pt plus 5pt minus 3pt}
\global\setlength\theorempostskipamount{8pt plus 3pt minus 1.5pt}
%    \end{macrocode}
% \end{macro}
% \end{macro}
%
%
% \begin{macro}{\@endtheorem}
%    The same holds for the macro |\@endtheorem|, which ends a
%    {\envfont theorem} environment. Since it is the same for all
%    theorem sets, it is removed from the macros |\th@|\meta{style}.
%    It simply ends the {\envfont trivlist} environment, which was
%    begun in |\@thm|.
%    \begin{macrocode}
\global\let\@endtheorem=\endtrivlist
%    \end{macrocode}
% \end{macro}
%
%
%
% \begin{macro}{\@preamblecmds}
%    All macros defined above are to be used only in the preamble.
%    Therefore, we insert them in |\@preamblecmds| which will disable
%    them at begin document. This is done by the internal \LaTeXe{}
%    command |\@onlypreamble|.
% \changes{v2.0g}{89/09/19}{`theoremheaderfont and `theorembodyfont
%    added.}
% \changes{v2.2a}{94/02/03}{Use \LaTeXe{} interface}
%    \begin{macrocode}
\@onlypreamble\@xnthm
\@onlypreamble\@ynthm
\@onlypreamble\@othm
\@onlypreamble\newtheorem
\@onlypreamble\theoremstyle
\@onlypreamble\theorembodyfont
\@onlypreamble\theoremheaderfont
%    \end{macrocode}
% \end{macro}
%
% Finally we declare the \texttt{plain} theorem style to be the default.
% \changes{v2.1b}{90/06/04}{Fixed missing `item by declaring default
%    style.}
%    \begin{macrocode}
\theoremstyle{plain}
%</package>
%    \end{macrocode}
%
% \Finale

\endinput
}{}
\ProvidesFile{thp.sty}
             [\FMithmInfo]
%    \end{macrocode}
%
%
% \begin{macro}{\th@plain}
% \begin{macro}{\@begintheorem}
% \begin{macro}{\@opargbegintheorem}
%    |\theoremstyle{plain}| corresponds to the original
%    definition, except that the distances to the surrounding text are
%    determined by the parameters |\theorempreskipamount| and
%    |\theorempostskipamount|.
%    First we set the default body font.
%    \begin{macrocode}
\gdef\th@plain{\normalfont\itshape
%    \end{macrocode}
%    Then we define |\@begintheorem| and
%    |\@opargbegintheorem|. These two macros define how the
%    header of a theorem is typeset.  |\@opargbegintheorem| will
%    be called if a \textsf{theorem} environment with an optional
%    argument is encountered; otherwise, the header is constructed by
%    calling |\@begintheorem|.  If one of these macros is
%    executed, we are within a \textsf{trivlist} environment started by
%    |\@thm|.  So the theorem header is produced with an
%    |\item| command.
%
%    Instead of specifying the header font directly, all standard
%    theorem styles use the |\theorem@headerfont| macro to allow
%    customization.  The extra space (|\labelsep|) is necessary
%    because of problems in the \textsf{trivlist} environment.
%    \begin{macrocode}
  \def\@begintheorem##1##2{%
        \item[\hskip\labelsep \theorem@headerfont ##1\ ##2]}%
%    \end{macrocode}
%    The definition of |\@opargbegintheorem| is completely
%    analogous. The only difference is the fact that there exists a
%    third argument (which is the optional parameter of the
%    environment and contains additional information about the
%    theorem).  Customarily we enclose it in parentheses.
%    \begin{macrocode}
\def\@opargbegintheorem##1##2##3{%
   \item[\hskip\labelsep \theorem@headerfont ##1\ ##2\ (##3)]}}
%    \end{macrocode}
% \end{macro}
% \end{macro}
% \end{macro}
%
% \noindent
% We conclude with an |\endgroup| to restore the
% |\catcode| of |@|.
%    \begin{macrocode}
\endgroup
%</thp>
%    \end{macrocode}
%
%
% \subsubsection{The \texttt{break} style}
%
% This style option is stored in the file |thb.sty|.
% For the next two lines see the documentation for |\th@plain| on
% page~\pageref{dok:th@plain}.
%    \begin{macrocode}
%<*thb>
\begingroup \makeatletter
\@ifundefined{theorem@style}{% \iffalse meta-comment
%
% Copyright (C) 1993-2024
%
% The LaTeX Project and any individual authors listed elsewhere
% in this file.
%
% This file is part of the Standard LaTeX `Tools Bundle'.
% -------------------------------------------------------
%
% It may be distributed and/or modified under the
% conditions of the LaTeX Project Public License, either version 1.3c
% of this license or (at your option) any later version.
% The latest version of this license is in
%    https://www.latex-project.org/lppl.txt
% and version 1.3c or later is part of all distributions of LaTeX
% version 2005/12/01 or later.
%
% The list of all files belonging to the LaTeX `Tools Bundle' is
% given in the file `manifest.txt'.
%
% \fi
%
%%
%
%
\def\FMithmInfo{2023/07/05 v2.2c Theorem extension package (FMi)}
%
% \ProvidesFile{theorem.dtx}[\FMithmInfo]
%\iffalse   % this is a METACOMMENT !
%%
%% Package `theorem' to use with LaTeX2e
%% Copyright (C) 1989-1995 Frank Mittelbach, all rights reserved.
%<+package>\@ifundefined{theorem@style}{}{\endinput}
%<+package>\NeedsTeXFormat{LaTeX2e}[1995/06/01]
%<+package>\ProvidesPackage{theorem}
%<+package>         [\FMithmInfo]
%
% \fi
%
% \changes{v1.0}{87/05/07}{First draft.}
% \changes{v1.0a}{87/05/07}{`thistheoremstyle' added.}
% \changes{v1.1a}{87/07/29}{New concept: redefine the internal
%                          \protect\LaTeX{} commands `@xnthm' etc.}
% \changes{v1.1b}{87/07/31}{`theoremfont' used to specify font.}
% \changes{v1.2a}{88/01/08}{`@endtheorem' now globally defined.}
% \changes{v1.2b}{88/01/11}{`newtheorem' only allowed in the preamble.}
% \changes{v1.3a}{88/02/14}{`theoremfont' now a toks register.}
%
% \changes{v2.0a}{88/11/16}{New implementation and new documentation.}
% \changes{v2.0a}{88/11/16}{No layout changes but internal modifications
%                           old substyles won't work.}
% \changes{v2.0b}{89/01/30}{`global' used for all definitions.}
% \changes{v2.0c}{89/06/15}{Documentation translated to English.
%                            Thanks to LLL and Nina}
% \changes{v2.1a}{90/02/15}{Support for new font selection.}
% \changes{v2.2a}{94/02/03}{Cleanup for LaTeX2e}
%
% \DoNotIndex{\@endparenv,\@ifdefinable,\@ifnextchar,\@ifundefined}
% \DoNotIndex{\@namedef,\@nameuse,\@spaces,\@tempa,\@warning}
% \DoNotIndex{\ }
% \DoNotIndex{\begingroup,\bfseries}
% \DoNotIndex{\csname}
% \DoNotIndex{\def,\do,\docdate}
% \DoNotIndex{\endcsname,\endgroup,\endinput,\endtrivlist,\expandafter}
% \DoNotIndex{\filedate,\fileversion}
% \DoNotIndex{\gdef,\global}
% \DoNotIndex{\hbox,\hskip}
% \DoNotIndex{\ignorespaces,\input,\itshape,\item}
% \DoNotIndex{\labelsep,\let,\llap}
% \DoNotIndex{\makeatletter}
% \DoNotIndex{\noexpand}
% \DoNotIndex{\refstepcounter,\relax,\rlap}
% \DoNotIndex{\setlength,\slshape,\space,\string,\strut}
% \DoNotIndex{\the,\trivlist,\typeout}
% \DoNotIndex{\vbox}
% \DoNotIndex{\xdef}
%
% \def\deflabel#1{\ttfamily #1\hfill}
% \def\deflist#1{\begin{list}{}{\settowidth\labelwidth{\ttfamily #1}%
%                               \setlength\leftmargin\labelwidth
%                               \addtolength\leftmargin\labelsep
%                               \let\makelabel\deflabel}}
% \def\enddeflist{\end{list}}
% \def\envfont{\normalfont\ttfamily}
%
% \MakeShortVerb{\|}
%
% \newcommand{\TUB}{{\sl TUGboat\/}}
%
% \GetFileInfo{theorem.dtx}
% \title{An Extension of the
%        \LaTeX{} theorem environment\thanks{This file
%        has version number \fileversion, last
%        revised \filedate.}}
% \author{Frank Mittelbach}
%
% \MaintainedByLaTeXTeam{tools}
% \maketitle
%
% \begin{abstract}
%    The macros described in this paper yield an extension of the
%    \LaTeX{} theorem mechanism. It is designed to satisfy the
%    different requirements of various journals. Thus, the
%    layout of the ``theorems'' can be manipulated by determining a
%    ``style''. This article describes not only the use, but
%    also the definition, of the necessary macros.
% \end{abstract}
%
% \section*{Preface to version 2.2}
%
% For \LaTeXe{} this package did not need any fundamental changes. I
% only modified the messages generated so that theorem layout styles
% will show up with the |\listfiles| command and cleaned the section
% on the New Font Selection Scheme since this is now included in
% \LaTeX.
%
% \section*{Preface to version 2.1}
%
% This version is identical to 2.0g described in \TUB\ 10\#3 except
% for some internal defaults which are now set depending on the used
% font selection scheme.
%
% This was done to avoid unpleasant surprises if the new font selection
% scheme is in force. For further details see
% section~\ref{sec:newfonts} and \cite{bk:GMS94}.
%
% \section{Introduction}
%
% For our purposes here, ``theorems'' are labelled enunciations,
% often set off from the main text by extra space and a font change.
% Theorems, corollaries, conjectures, definitions, and remarks
% are all instances of ``theorems''.  The ``header'' of these structures
% is composed of a label (such as \textsc{Theorem} or \textsc{Remark})
% and a number which serializes an item in the sequence of items
% with the same label.
%
% Shortly after the introduction of \LaTeX{} at the Fachbereich
% Mathematik in Mainz, the desire to manipulate the layout of
% ``theorems'' arose.  In Mainz, the following two conventions came
% into general use:
% \begin{enumerate}
%   \item The number of the theorem is shown in the margin.
%   \item There is a line break at
%         the end of the theorem header.
% \end{enumerate}
%
% Additionally, some journals require different formats which depend on
% the ``sort of theorem'': e.g.\ often remarks and definitions are set
% in |\upshape|, while |\itshape| is employed for main theorems.
%
% Confronted with these requirements, a theorem environment was
% developed in Mainz which allows separate determination of the
% layout of the ``theorems sets'', comparable to |\pagestyle|.
%
%
% \section{The user interface}
% \subsection{Defining new theorem sets}
%
% \DescribeMacro\newtheorem
% As in the original \LaTeX{} version, the command |\newtheorem|
% defines a new ``theorem set'' or ``theorem-like structure''.
% Two required arguments name the new environment and give the text
% to be typeset with each instance of the new ``set'', while
% an optional argument determines how
% the ``set'' is enumerated:
% \begin{description}
%    \item[\ttfamily \bslash newtheorem\{foo\}\{bar\}]
%       The theorem set {\envfont foo} (whose name is \texttt{bar})
%       uses its own counter.
%    \item[\ttfamily \bslash newtheorem\{foo2\}{[foo]}\{bar2\}]
%       The theorem set {\envfont foo2} (printed name \texttt{bar2})
%       uses the same counter as the theorem set \texttt{foo}.
%    \item[\ttfamily \bslash newtheorem\{foo3\}\{bar3\}{[section]}]
%       The theorem set {\envfont foo3} (printed name \texttt{bar3}) is
%       enumerated within the counter \texttt{section}, i.e.\ with every
%       new |\section| the enumeration begins again with 1, and
%       the enumeration is composed from the section-number and the
%       theorem counter itself.
% \end{description}
%
% \DescribeMacro\theoremstyle
% Additionally, the command |\theoremstyle| can define the layout
% of various, or all, theorem sets.  It should be noted that any
% theorem set defined by |\newtheorem| is typeset in the
% |\theoremstyle| that is current at the time of the definition.
% Thus, the following
% \begin{verbatim}
%    \theoremstyle{break}        \newtheorem{Cor}{Corollary}
%    \theoremstyle{plain}        \newtheorem{Exa}{Example}[section]
%\end{verbatim}                             ^^A No blank at line start
% leads to the result that the set {\envfont Cor} is formatted in the
% style \texttt{break}, while the set {\envfont Exa} and all the
% following ones are formatted in the style \texttt{plain}, unless
% another |\theoremstyle| follows.  Since the definitions installed by
% |\newtheorem| are global, one also can limit |\theoremstyle| locally
% by grouping braces.
%
% \DescribeMacro\theorembodyfont
% The choice of the font for the theorem body is completely
% independent of the chosen |\theoremstyle|; this has proven to
% be very advantageous.  For example,
% \begin{verbatim}
%   {\theorembodyfont{\upshape}          \newtheorem{Rem}{Remark}}
%\end{verbatim}
% defines a theorem set {\envfont Rem}, which will be set in
% |\upshape| in the current layout (which in our example is
% \texttt{plain}).  As with |\theoremstyle|, the |\theorembodyfont|
% chosen is that current at the time of |\newtheorem|.  If
% |\theorembodyfont| is not specified or one defines
% |\theorembodyfont{}|, then the font used will be that defined by the
% |\theoremstyle|.
%
% \DescribeMacro\theoremheaderfont It is also possible to customize
% the font used for the theorem headers.  This is, however, a global
% declaration, and therefore there should be at most one
% |\theoremheaderfont| declaration in the preamble.\footnote{If it is
% actually necessary to have different header fonts, one has to define
% new theorem styles (substituting the desired font) or specify the
% information directly in the \texttt{\bslash newtheorem} declaration
% (the unclean variant).}
%
% \DescribeMacro\theorempreskipamount
% \DescribeMacro\theorempostskipamount Two additional parameters
% affect the vertical space around the theorem environments:
% |\theorempreskipamount| and |\theorempostskipamount| define,
% respectively, the spacing before and after such an environment.
% These parameters apply for all theorem sets and can be manipulated
% with the ordinary length macros.  They are rubber lengths
% (`\textsf{skips}'), and therefore can contain \texttt{plus} and
% \texttt{minus} parts.
%
% Since the definition of theorem sets should---most sensibly---be
% placed in the preamble, we only allow installation there.
% It is therefore possible to release the memory used here after
% |\begin{document}|, in order to make room for other
% applications.
%
%
% \subsection{Existing theorem styles}
%
% The following theorem styles exist to date:
% \begin{deflist}{marginbreak}
%    \item[plain]
%       This theorem style emulates the original \LaTeX{} definition,
%       except that additionally the parameters
%       |\theorem...skipamount| are used.
%    \item[break]
%       In this style, the theorem header is followed by a line
%       break.
%    \item[marginbreak]
%       The theorem number is set in the margin, and there is a line
%       break as in \texttt{break}.
%    \item[changebreak]
%       Like \texttt{break}, but with header number and text
%       interchanged.
%    \item[change]
%       Header number and text are interchanged, without a line break.
%    \item[margin]
%       The number is set in the left margin, without a line break.
% \end{deflist}
% All styles (except \texttt{plain}) select |\slshape| as the default
% |\theorembodyfont|.
%
%
% \subsection{Examples}
%
% Given the above theorem sets {\envfont Cor}, {\envfont Exa}
% and {\envfont Rem},
% suppose that the preamble also contains the declarations:
% \begin{verbatim}
%    \theoremstyle{marginbreak}   \newtheorem{Lem}[Cor]{Lemma}
%    \theoremstyle{change}
%    \theorembodyfont{\itshape}        \newtheorem{Def}[Cor]{Definition}
%
%    \theoremheaderfont{\scshape}
%\end{verbatim}
% Then the following are some typical examples of the typeset output
% resulting from their use.
%
% \begin{Cor}
%    This is a sentence typeset in the theorem environment {\envfont
%    Cor}.
% \end{Cor}
%
% \begin{Exa}
%    This is a sentence typeset in the theorem environment {\envfont
%    Exa}.
% \end{Exa}
%
% \begin{Rem}
%    This is a sentence typeset in the theorem environment {\envfont
%    Rem}.
% \end{Rem}
%
% \begin{Lem}[Ben User]
%    This is a sentence typeset in the theorem environment {\envfont
%    Lem}.
% \end{Lem}
%
% \begin{Def}[Very impressive Definition]
%    This is a sentence typeset in the theorem environment {\envfont
%    Def}.
% \end{Def}
%
% The last two examples show the effect of the optional argument to
% a theorem environment (it is the text typeset in parentheses).
%
%
% \section{Special Considerations}
% \label{sec:newfonts}
%
% Theoremheader and body are implemented as a unit. This means that
% the |\theoremheaderfont|  will inherit characteristics of the
% |\theorembodyfont| in \LaTeXe. Thus, if for example
% |\theorembodyfont| is |\itshape| and |\theoremheaderfont|
% is |\bfseries| the font selected for the header will have the
% characteristics `bold extended italic'. If this is not desired one
% should set the |\theoremheaderfont| to something like
% \begin{verbatim}
%      \theoremheaderfont{\normalfont\bfseries}
%\end{verbatim}
% i.e.\ supplying all necessary font information explicitly.
%
%
%
% \section{Acknowledgements}
%
% The publication of this set of macros was only possible with the
% help of Christina Busse (translating the manuscript into English),
% Joachim Pense (playing the r\^ole of typist), Chris Rowley
% (looking everything over) and many others providing useful
% suggestions.
%
% \MaybeStop{%
%    \begin{thebibliography}{9}
%       \bibitem{bk:GMS94} \textsc{M.~Goossens}, \textsc{F.~Mittelbach}
%         and \textsc{A.~Samarin}.
%         \newblock The \LaTeX{} Companion.
%         \newblock
%         Addison-Wesley, Reading, Massachusetts, 1994.
%       \bibitem{latex:source} \textsc{Lamport, Leslie.} \newblock
%                               \texttt{latex.tex}, version 2.09, date
%                               Feb. 1990.
%    \end{thebibliography}
%    \PrintIndex
%    \PrintChanges}
%
%
% \section{The documentation driver file}
%
% The next bit of code contains the documentation driver file for
% \TeX{}, i.e., the file that will produce the documentation you are
% currently reading. It will be extracted from this file by the
% \texttt{docstrip} program. Since it is the first code in the file
% one can alternatively process this file directly with \LaTeXe{} to
% obtain the documentation.
%
%    \begin{macrocode}
%<*driver>
\documentclass{ltxdoc}

\usepackage{theorem}

%<+driver>% The next few lines define theorem sets which are used
%<+driver>% in the example section of the documentation.

\theoremstyle{break}       \newtheorem{Cor}{Corollary}
\theoremstyle{plain}       \newtheorem{Exa}{Example}[section]
{\theorembodyfont{\upshape}\newtheorem{Rem}{Remark}}
\theoremstyle{marginbreak} \newtheorem{Lem}[Cor]{Lemma}
\theoremstyle{change}
\theorembodyfont{\itshape} \newtheorem{Def}[Cor]{Definition}

\theoremheaderfont{\scshape}

\RecordChanges

\begin{document}
   \DocInput{theorem.dtx}
\end{document}
%</driver>
%    \end{macrocode}
%
% \section{Definition of the Macros}
%
% If the file has been loaded before, we abort immediately. If not the
% package announces itself (this is actually done at the very top of
% the file---the way it is done isn't good style so don't copy it).
% \changes{v2.0e}{89/07/19}{Spaces removed from `typeout'.}
% \changes{v2.2b}{95/11/19}{Announce moved to top of file}
%    \begin{macrocode}
%<*package>
%\@ifundefined{theorem@style}{}{\endinput}
%\def\FMithmInfo{1995/11/19 v2.2b Theorem extension package (FMi)}
%\ProvidesPackage{theorem}[\FMithmInfo]
%    \end{macrocode}
%
%
% \subsection{Definition of theorem styles and fonts}
%
% All the definitions in this file are done globally to allow
% inputting this file inside a group.
%
% \begin{macro}{\theoremstyle}
%    Before a theorem style can be installed, the chosen style must be
%    known. For that reason, we must test to see that
%    |\th@|\meta{style} is known or, more precisely, that it is
%    different from |\relax|. If the style is not known then
%    |\th@plain| is used.
%    \begin{macrocode}
\gdef\theoremstyle#1{%
   \@ifundefined{th@#1}{\@warning
          {Unknown theoremstyle `#1'. Using `plain'}%
          \theorem@style{plain}}%
%    \end{macrocode}
%    We save the theorem style to be used in the token register
%    |\theorem@style|.
%    \begin{macrocode}
      {\theorem@style{#1}}%
%    \end{macrocode}
%    Now we ``evaluate'' the theorem style: this means, we call the
%    macro |\th@|\meta{style} which will activate the relevant
%    definitions which are contained in a separate file.  This is done
%    in a group to suppress changes to the current font.  This could
%    otherwise pose problems together with the new font selection
%    scheme\footnote{When I printed the original article using the new
%    font selection scheme I ended with a document with slanted
%    typefaces (text headings and all) simply because one of the
%    theorem styles used \texttt{\bslash sl} at toplevel.} if the
%    |\th@|\meta{style} is evaluated a second time.
% \changes{v2.1a}{90/02/15}{Group added around `th@..}
% \changes{v2.1a}{90/02/15}{Reseting of `..@begintheorem removed.}
%    \begin{macrocode}
      \begingroup
        \csname th@\the\theorem@style \endcsname
      \endgroup}
%    \end{macrocode}
% \end{macro}
%
%
% \begin{macro}{\@begintheorem}
% \changes{v2.1a}{90/02/15}{Relaxed at toplevel.}
% \begin{macro}{\@opargbegintheorem}
% \changes{v2.1a}{90/02/15}{Relaxed at toplevel.}
%    We reset |\@begintheorem| and |\@opargbegintheorem| to
%    |\relax| since these commands are no longer necessary at
%    toplevel. This will save a few tokens.
%    \begin{macrocode}
\global\let\@begintheorem\relax
\global\let\@opargbegintheorem\relax
%    \end{macrocode}
% \end{macro}
% \end{macro}
%
% \begin{macro}{\theorem@style}
%    Obviously the \textsf{token} register used above has to be
%    allocated. To assure the utmost compatibility with the
%    original \LaTeX{} definition, we set the default theorem style to
%    \texttt{plain}, which implements the usual \LaTeX{} convention.
% \changes{v2.0f}{89/07/20}{Unnecessary `global' removed.}
%    \begin{macrocode}
\newtoks\theorem@style
\global\theorem@style{plain}
%    \end{macrocode}
% \end{macro}
%
% \begin{macro}{\theorembodyfont}
% \begin{macro}{\theorem@bodyfont}
%    If the |\theorembodyfont| is set by the user then it should not
%    interact with the default font set in the theorem style. When the
%    new font selection is in force this may happen if, for example,
%    the default is |\itshape| and the new |\theorembodyfont| is
%    |\sffamily|. So we add a |\reset@font| command in front of the
%    user definition.
% \changes{v2.0f}{89/07/20}{Unnecessary `global' removed.}
% \changes{v2.0g}{89/09/19}{`theoremfont to `theorembodyfont.}
% \changes{v2.1c}{92/06/22}{`reset@font added and logic changed.}
% \changes{v2.1d}{92/10/28}{Reset to empty if arg is empty}
%    \begin{macrocode}
\gdef\theorembodyfont#1{%
%    \end{macrocode}
%    We check if the argument supplied is empty and if so put nothing
%    into the |\theorem@bodyfont| token register to allow for
%    |\theorembodyfont{}| as a mean of using the default of the
%    current |\theoremstyle|.
%    \begin{macrocode}
   \def\@tempa{#1}%
   \ifx\@tempa\@empty
    \theorem@bodyfont{}%
   \else
    \theorem@bodyfont{\reset@font#1}%
   \fi
   }
\newtoks\theorem@bodyfont
\global\theorem@bodyfont{}
%    \end{macrocode}
% \end{macro}
% \end{macro}
%
%
% \begin{macro}{\theoremheaderfont}
% \changes{v2.0g}{89/09/19}{Macro added.}
%    The font for the theorem headers is handled differently because
%    this definition applies to all theorem styles.
%    \begin{macrocode}
\gdef\theoremheaderfont#1{\gdef\theorem@headerfont{#1}%
%    \end{macrocode}
%    After using the macro once it is redefined to produce an error
%    message.
%    \begin{macrocode}
       \gdef\theoremheaderfont##1{%
        \typeout{\string\theoremheaderfont\space should be used
                 only once.}}}
%    \end{macrocode}
% \end{macro}
%
% \begin{macro}{\theorem@headerfont}
% \changes{v2.0g}{89/09/19}{Macro added.}
%    To set the |\theorem@headerfont| default we first test if
%    the new fontselection scheme is in force.
% \changes{v2.1a}{90/02/15}{Support for new font selection.}
%    \begin{macrocode}
\ifx\upshape\undefined
%    \end{macrocode}
%    If not we define it to expand into |\bfseries|. We don't use
%    |\let| just in case a following style option redefines this
%    macro.
%    \begin{macrocode}
\gdef\theorem@headerfont{\bfseries}
%    \end{macrocode}
%    Otherwise we reset the current shape before calling |\bfseries|.
%    \begin{macrocode}
\else \gdef\theorem@headerfont{\normalfont\bfseries}\fi
%    \end{macrocode}
% \end{macro}
%
% \begin{macro}{\th@plain}
% \begin{macro}{\th@break}
% \begin{macro}{\th@marginbreak}
% \begin{macro}{\th@changebreak}
% \begin{macro}{\th@change}
% \begin{macro}{\th@margin}
%    The different styles are defined in macros such as
%    |\th@plain|.  Since memory space is precious in
%    ``non-Big-versions'', we have to avoid offering too many unused
%    definitions.  Therefore we define these styles in separate files
%    that can be loaded on demand.  Thus the commands themselves only
%    load these files. We use |\@input@|, a \LaTeXe{} internal command
%    that ensures that the file will be listed with |\listfiles|.
% \changes{v2.2a}{94/02/02}{Use `@input@ to load theorem layout files}
%    \begin{macrocode}
\gdef\th@plain{\@input@{thp.sty}}
\gdef\th@break{\@input@{thb.sty}}
\gdef\th@marginbreak{\@input@{thmb.sty}}
\gdef\th@changebreak{\@input@{thcb.sty}}
\gdef\th@change{\@input@{thc.sty}}
\gdef\th@margin{\@input@{thm.sty}}
%    \end{macrocode}
%    This list will be expanded when new styles become
%    available.  For testing,  just append new theorem substyles
%    as document options.
% \end{macro}
% \end{macro}
% \end{macro}
% \end{macro}
% \end{macro}
% \end{macro}
%
%
%
% \subsection{Definition of a new theorem set}
%
% As already pointed out, a new theorem environment can be defined in
% three different ways:
% \begin{verbatim}
%    \newtheorem{Lem}{Lemma}
%    \newtheorem{Lem}{Lemma}[section]
%    \newtheorem{Lem}[Theorem]{Lemma}
%\end{verbatim}
% The function of the macro |\newtheorem| is to recognize these cases
% and then to branch into one of the three macros |\@ynthm|, |\@xnthm|
% or |\@othm|.  This mechanism is adopted unchanged from
% \cite{latex:source}; the essential point here is that, for example,
% in the second case, the arguments \texttt{Lem}, \texttt{Lemma} and
% \texttt{section} are passed over to the macro |\@xnthm|.
%
% We inspect this case first because the others present fewer problems,
% and thus are easily derived from this one.
%
% \begin{macro}{\@xnthm}
%    For our example arguments, the macro |\@xnthm| must fulfill
%    the following:
%    \begin{itemize}
%       \item
%          Define a new \LaTeX-counter `Lem'
%       \item
%          reset this counter within a |\section|
%       \item
%          define the macro |\theLem|
%       \item
%          define the environment macros |\Lem| and
%          |\endLem| using the current
%          |\theoremstyle| and |\theorem@bodyfont|.
%    \end{itemize}
%    Obviously, all this should happen only if the first argument
%    of |\@xnthm| (i.e.\ \texttt{Lem} in our example) is chosen so
%    as not to conflict with any previously defined commands or
%    environments.  This test is performed by the \LaTeX{} macro
%    |\@ifdefinable|.
%    \begin{macrocode}
\gdef\@xnthm#1#2[#3]{\expandafter\@ifdefinable\csname #1\endcsname
%    \end{macrocode}
%    Therefore, the first argument of |\@ifdefinable| is the
%    expansion (in the example, |\Lem|) of
%    |\csname||#1\endcsname|.  The second argument is
%    executed only if the test has been completed successfully.
%    \begin{macrocode}
   {%
%    \end{macrocode}
%    Now we define the new counter. The names of the \LaTeX{}
%    macros employed should speak for themselves:
% \changes{v2.2b}{1995/10/07}{Use \cs{@newctr}}
% \changes{v2.2c}{1995/11/23}{Use \cs{@newctr} properly.}
%    \begin{macrocode}
    \@definecounter{#1}\@newctr{#1}[#3]%
%    \end{macrocode}
%    Using |\@newctr| will give a proper error message if the counter
%    in |#3| is not defined.
%    In defining `|\theLem|' we must generate the desired macro
%    name by use of |\expandafter| and |\csname|.
%    \begin{macrocode}
    \expandafter\xdef\csname the#1\endcsname
%    \end{macrocode}
%    An |\xdef| is used in order to make the definition global,
%    and to ensure that it contains the replacement texts of
%    |\@thmcountersep| and |\@thmcounter|.\footnote{These
%    two macros can be defined by the document style.  Their default
%    values produce a `\texttt{.}' as separation and an arabic
%    representation of the number.} However, not everything should be
%    expanded. For example, it saves space to use |\thesection|
%    instead of its---at times---lengthy expansion.
%    \begin{macrocode}
      {\expandafter \noexpand \csname the#3\endcsname
       \@thmcountersep \@thmcounter{#1}}%
%    \end{macrocode}
%    Thus with the defaults of \LaTeX{}, |\theLem| would be
%    replaced by the command sequence |\thesection.\arabic{Lem}|.
%
%    We will now look at the definition of the macro which is executed
%    at the beginning of the actual environment (in our example this
%    macro is |\Lem|). It should be noted that we use an
%    ``|\expandafter| trick'' to expand only certain parts of the
%    replacement text at the time of the definition.
%    \begin{macrocode}
    \def\@tempa{\global\@namedef{#1}}%
    \expandafter \@tempa \expandafter{%
%    \end{macrocode}
%    First, the macro that contains the current definitions of
%    |\@begintheorem| and |\@opargtheorem| should be called
%    up.  The name of this macro---as is already known---has the form
%    |\th@|\meta{theorem\ style}; therefore, it must be called by
%    \begin{macrocode}
      \csname th@\the \theorem@style
%    \end{macrocode}
%    In addition the default theorem font should be changeable, i.e.\
%    we have to insert the contents of |\theorem@bodyfont|. For
%    that reason, we expand even further, beyond |\endcsname|,
%    and thus insert the contents of the \textsf{token} register
%    |\theorem@bodyfont| in the replacement text.
%    \begin{macrocode}
            \expandafter \endcsname \the \theorem@bodyfont
%    \end{macrocode}
%    Now it is time to call the macro |\@thm| which takes over
%    the further processing.  It has two arguments: the current
%    counter name (in our example, \texttt{Lem}), and the text of the
%    label (in our example, \texttt{Lemma}).
%    \begin{macrocode}
     \@thm{#1}{#2}}%
%    \end{macrocode}
%    With this, the `sub-definition' is complete.  The macro
%    |\@endtheorem| ends a theorem environment and is, so far, nothing
%    but an |\endtrivlist|.  (Hence it is defined globally, and not
%    within the theorem styles.\footnote{This has to be changed as
%    soon as theorem styles that change \texttt{\bslash @endtheorem}
%    exist.  In such a case, all existing styles must be changed as
%    well since they will have to reset the macro.}) Therefore, we can
%    set it equivalent to the macro that ends the theorem set (in our
%    example, |\endLem|).  However, if some day theorem styles exist
%    that do change |\@endtheorem|, we would have to use the
%    commented-out line instead.
%    \begin{macrocode}
    \global \expandafter \let \csname end#1\endcsname \@endtheorem
%   \global\@namedef{end#1}{\@endtheorem}%
%    \end{macrocode}
%    With these commands all the required definitions are employed,
%    unless the test |\@ifdefinable| has failed.  Therefore, we
%    end the second argument of this macro and with it the definition
%    of |\@xnthm|.
%    \begin{macrocode}
   }}
%    \end{macrocode}
% \end{macro}
%
% \begin{macro}{\@ynthm}
%    The definition of |\@ynthm| is completely analogous. In
%    this case the new counter that is defined is not reset within
%    another counter; thus the definition of |\the...|
%    is simplified:
%    \begin{macrocode}
\gdef\@ynthm#1#2{\expandafter\@ifdefinable\csname #1\endcsname
   {\@definecounter{#1}%
    \expandafter\xdef\csname the#1\endcsname{\@thmcounter{#1}}%
%    \end{macrocode}
%    The rest of the definition corresponds literally to that of
%    |\@xnthm|:
%    \begin{macrocode}
    \def\@tempa{\global\@namedef{#1}}\expandafter \@tempa
     \expandafter{\csname th@\the \theorem@style \expandafter
     \endcsname \the\theorem@bodyfont \@thm{#1}{#2}}%
    \global \expandafter \let \csname end#1\endcsname \@endtheorem}}
%    \end{macrocode}
% \end{macro}
%
% \begin{macro}{\@othm}
%    The definition of |\@othm| does not contain anything new.
%    \begin{macrocode}
\gdef\@othm#1[#2]#3{%
%    \end{macrocode}
% \changes{v2.2b}{95/10/07}{Build in check for counter defined}
%    We do not define a new counter but instead use one that has
%    already been defined.  Thus the only definition we need is that
%    of this pseudo-counter (i.e.\ |\the|\meta{env.\ name}).
%    First we check if |#2| corresponds to a known counter name.
%    \begin{macrocode}
  \expandafter\ifx\csname c@#2\endcsname\relax
   \@nocounterr{#2}%
  \else
   \expandafter\@ifdefinable\csname #1\endcsname
   {\expandafter \xdef \csname the#1\endcsname
     {\expandafter \noexpand \csname the#2\endcsname}%
%    \end{macrocode}
%    All other parts of the definition can be adopted from
%    |\@xnthm|. We have to remember, though, that in this case
%    the name of the current counter and the theorem label have moved
%    to the second and third arguments.
%    \begin{macrocode}
    \def\@tempa{\global\@namedef{#1}}\expandafter \@tempa
     \expandafter{\csname th@\the \theorem@style \expandafter
     \endcsname \the\theorem@bodyfont \@thm{#2}{#3}}%
    \global \expandafter \let \csname end#1\endcsname \@endtheorem}%
  \fi}
%    \end{macrocode}
% \end{macro}
%
% \subsection{Macros that are employed in a {\envfont theorem}
%    environment}
%
% \begin{macro}{\@thm}
%    The macro |\@thm| has to increase the current counter.  Then,
%    depending on whether the environment has (or does not have) an
%    optional argument, it has to branch into either |\@begintheorem|
%    or |\@opargtheorem|.
%    \begin{macrocode}
\gdef\@thm#1#2{\refstepcounter{#1}%
%    \end{macrocode}
%    Now we start a {\envfont trivlist} environment, and give
%    |\@topsep| and |\@topsepadd| the values of the \textsf{skip}
%    registers |\theorempreskipamount| and |\theorempostskipamount|.
%    The value in |\@topsep| is the vertical space that is inserted by
%    the first (and only) |\item| in our |\trivlist| whilst
%    |\@topsepadd| is inserted by |\@endparenv| at the end of that
%    {\envfont trivlist} environment.  By using these registers, we
%    obtain the desired space around a {\envfont theorem} environment.
%    \begin{macrocode}
   \trivlist
   \@topsep \theorempreskipamount               % used by first \item
   \@topsepadd \theorempostskipamount           % used by \@endparenv
%    \end{macrocode}
%    Now we have to test whether an optional argument has been given.
% \changes{v2.0d}{89/07/18}{Forgotten percent added.}
%    \begin{macrocode}
   \@ifnextchar [%
%    \end{macrocode}
%    If there is an optional argument, we will call |\@ythm|, and
%    move the arguments read back into the input stream.
%    \begin{macrocode}
   {\@ythm{#1}{#2}}%
%    \end{macrocode}
%    If not, we call |\@begintheorem|. Its first argument is the
%    name of the theorem set (hence the second argument of
%    |\@thm|). Its second argument is the macro that produces the
%    current number.
%    \begin{macrocode}
   {\@begintheorem{#2}{\csname the#1\endcsname}\ignorespaces}}
%    \end{macrocode}
% \end{macro}
%
% \begin{macro}{\@xthm}
% \begin{macro}{\@ythm}
%    Both these macros were originally called by |\@thm|.  We do
%    not need |\@xthm| anymore, hence we reset it to
%    |\relax|. The definition of |\@ythm| has not changed at
%    all from its definition in \LaTeX. In order to make the macros
%    easier to understand, we will
%    nevertheless present it (commented out).
%    \begin{macrocode}
\global\let\@xthm\relax
% \def\@ythm#1#2[#3]{\@opargbegintheorem{#2}{\csname
%           the#1\endcsname}{#3}\ignorespaces}
%</package>
%    \end{macrocode}
%    The primitive |\ignorespaces| in |\@ythm| and
%    |\@thm| is needed to remove the spaces between the
%    |\begin{...}| and the actual text.
% \end{macro}
% \end{macro}
%
%
% \subsection{Definition of the theorem substyles}
%
% As already pointed out, the theorem substyles, defined below, are
% only loaded when necessary.  Note that all these substyles, except
% \texttt{plain}, have |\slshape| as the default body font.
%
% \subsubsection{The \texttt{plain} style}
%
% \label{dok:th@plain} As the following macros use |@|, we have
% to locally set the |\catcode| of this symbol to ``letter''.
% This happens within a group, so that we do not have to worry about
% which |\catcode| that symbol had before.
%    \begin{macrocode}
%<*thp>
\begingroup \makeatletter
%    \end{macrocode}
% Since we are now within a group, we must make all definitions
% globally.  First we make sure that \texttt{theorem.sty} is loaded.
% This will allow us to use this file as a document style option
% without having to call \texttt{theorem} itself as an option.  At the
% same time, we assure that at least version 2 is loaded, since
% |\theorem@style| was not defined in earlier versions.
%    \begin{macrocode}
\@ifundefined{theorem@style}{\input{theorem.sty}}{}
\ProvidesFile{thp.sty}
             [\FMithmInfo]
%    \end{macrocode}
%
%
% \begin{macro}{\th@plain}
% \begin{macro}{\@begintheorem}
% \begin{macro}{\@opargbegintheorem}
%    |\theoremstyle{plain}| corresponds to the original
%    definition, except that the distances to the surrounding text are
%    determined by the parameters |\theorempreskipamount| and
%    |\theorempostskipamount|.
%    First we set the default body font.
%    \begin{macrocode}
\gdef\th@plain{\normalfont\itshape
%    \end{macrocode}
%    Then we define |\@begintheorem| and
%    |\@opargbegintheorem|. These two macros define how the
%    header of a theorem is typeset.  |\@opargbegintheorem| will
%    be called if a \textsf{theorem} environment with an optional
%    argument is encountered; otherwise, the header is constructed by
%    calling |\@begintheorem|.  If one of these macros is
%    executed, we are within a \textsf{trivlist} environment started by
%    |\@thm|.  So the theorem header is produced with an
%    |\item| command.
%
%    Instead of specifying the header font directly, all standard
%    theorem styles use the |\theorem@headerfont| macro to allow
%    customization.  The extra space (|\labelsep|) is necessary
%    because of problems in the \textsf{trivlist} environment.
%    \begin{macrocode}
  \def\@begintheorem##1##2{%
        \item[\hskip\labelsep \theorem@headerfont ##1\ ##2]}%
%    \end{macrocode}
%    The definition of |\@opargbegintheorem| is completely
%    analogous. The only difference is the fact that there exists a
%    third argument (which is the optional parameter of the
%    environment and contains additional information about the
%    theorem).  Customarily we enclose it in parentheses.
%    \begin{macrocode}
\def\@opargbegintheorem##1##2##3{%
   \item[\hskip\labelsep \theorem@headerfont ##1\ ##2\ (##3)]}}
%    \end{macrocode}
% \end{macro}
% \end{macro}
% \end{macro}
%
% \noindent
% We conclude with an |\endgroup| to restore the
% |\catcode| of |@|.
%    \begin{macrocode}
\endgroup
%</thp>
%    \end{macrocode}
%
%
% \subsubsection{The \texttt{break} style}
%
% This style option is stored in the file |thb.sty|.
% For the next two lines see the documentation for |\th@plain| on
% page~\pageref{dok:th@plain}.
%    \begin{macrocode}
%<*thb>
\begingroup \makeatletter
\@ifundefined{theorem@style}{\input{theorem.sty}}{}
\ProvidesFile{thb.sty}
             [\FMithmInfo]
%    \end{macrocode}
%
% \begin{macro}{\th@break}
%    |\theoremstyle{break}| produces a line break after the name
%    of the theorem.  The font is |\slshape|.  Hence, we define
%    |\th@break| as follows:
%    \begin{macrocode}
\gdef\th@break{\normalfont\slshape
  \def\@begintheorem##1##2{\item[%
%    \end{macrocode}
%    We run into the following problem: it is not possible to create
%    the header with |\item[|\meta{title}|]| and then start
%    a new line by, for example, |\mbox{}\\|. Such a definition
%    will fail whenever a list environment follows immediately.  With
%    the above construction, the |\mbox{}| causes the switch
%    |@inlabel| (cf.\ definition of |\list| and
%    |\trivlist| in \cite{latex:source}) to be set to \texttt{false}
%    and so the following list will insert additional vertical
%    space (|\topskip|). This is quite annoying. Therefore, we
%    create the line break within the |\item|. In order to ensure
%    that the text will begin at the proper position in the following
%    line, we simply pretend that the label does not take any
%    room.\footnote{This will lead to problems whenever very high
%    symbols occurring in the line tower into the
%    heading. So, something else has to be done here sometime.}
%    \begin{macrocode}
       \rlap{\vbox{\hbox{\hskip \labelsep\theorem@headerfont ##1\ ##2}%
                   \hbox{\strut}}}]}%
%    \end{macrocode}
%    Again, the definition of |\@opargbegintheorem| is
%    completely analogous.
%    \begin{macrocode}
\def\@opargbegintheorem##1##2##3{%
  \item[\rlap{\vbox{\hbox{\hskip \labelsep \theorem@headerfont
                     ##1\ ##2\ (##3)}%
                    \hbox{\strut}}}]}}
\endgroup
%</thb>
%    \end{macrocode}
% \end{macro}
%
%
% \subsubsection{The \texttt{changebreak} style}
%
%    \begin{macrocode}
% This style option is stored in the file |thcb.sty|.
%    \begin{macrocode}
%<*thcb>
\begingroup \makeatletter
\@ifundefined{theorem@style}{\input{theorem.sty}}{}
\ProvidesFile{thcb.sty}
             [\FMithmInfo]
%    \end{macrocode}
%
% \begin{macro}{\th@changebreak}
%    The \texttt{change-break} theorem style is like \texttt{break}
%    but with interchange of theorem name and theorem number.  Thus we
%    define |\th@changebreak| as follows:
%    \begin{macrocode}
\gdef\th@changebreak{\normalfont\slshape
  \def\@begintheorem##1##2{\item
       [\rlap{\vbox{\hbox{\hskip\labelsep \theorem@headerfont ##2\ ##1}%
                    \hbox{\strut}}}]}%
\def\@opargbegintheorem##1##2##3{%
  \item[\rlap{\vbox{\hbox{\hskip\labelsep \theorem@headerfont
                     ##2\ ##1\ (##3)}%
                    \hbox{\strut}}}]}}
\endgroup
%</thcb>
%    \end{macrocode}
% \end{macro}
%
%
% \subsubsection{The \texttt{change} style}
%
% This style option is stored in the file |thc.sty|.
%    \begin{macrocode}
%<*thc>
\begingroup \makeatletter
\@ifundefined{theorem@style}{\input{theorem.sty}}{}
\ProvidesFile{thc.sty}
             [\FMithmInfo]
%    \end{macrocode}
%
% \begin{macro}{\th@change}
%    The \texttt{change} theorem style corresponds to the
%    \texttt{change break} style without a linebreak after the header.
%    To say it in another way, it's the same as the \texttt{plain}
%    style but with number and name interchanged and |\slshape| as the
%    default font.
%    \begin{macrocode}
\gdef\th@change{\normalfont\slshape
  \def\@begintheorem##1##2{\item
       [\hskip\labelsep \theorem@headerfont ##2\ ##1]}%
\def\@opargbegintheorem##1##2##3{%
  \item[\hskip\labelsep \theorem@headerfont ##2\ ##1\ (##3)]}}
\endgroup
%</thc>
%    \end{macrocode}
% \end{macro}
%
%
% \subsubsection{The \texttt{marginbreak} style}
%
% This style option is the one used most often at Mainz.  It is saved
% in the file |thmb.sty|.
%    \begin{macrocode}
%<*thmb>
\begingroup \makeatletter
\@ifundefined{theorem@style}{\input{theorem.sty}}{}
\ProvidesFile{thmb.sty}
             [\FMithmInfo]
%    \end{macrocode}
%
% \begin{macro}{\th@marginbreak}
%    The \texttt{margin break} style is nearly the same as the
%    \texttt{change break} style. The only difference is the placement
%    of the theorem number.  We use |\llap| to place it in the left
%    margin.
%
%    In this style |\labelsep| denotes the separation between the
%    number and the text.
%    \begin{macrocode}
\gdef\th@marginbreak{\normalfont\slshape
  \def\@begintheorem##1##2{\item
       [\rlap{\vbox{\theorem@headerfont
                    \hbox{\llap{##2}\hskip\labelsep ##1}%
                    \hbox{\strut}}}]}%
\def\@opargbegintheorem##1##2##3{%
  \item[\rlap{\vbox{\theorem@headerfont
                    \hbox{\llap{##2}\hskip\labelsep ##1\ (##3)}%
                    \hbox{\strut}}}]}}
\endgroup
%</thmb>
%    \end{macrocode}
% \changes{v2.1c}{90/02/17}{`endgroup outside macrocode env.}
% \end{macro}
%
%
%
% \subsubsection{The \texttt{margin} style}
%
% This style option is stored in the file |thm.sty|.
%    \begin{macrocode}
%<*thm>
\begingroup \makeatletter
\@ifundefined{theorem@style}{\input{theorem.sty}}{}
\ProvidesFile{thm.sty}
             [\FMithmInfo]
%    \end{macrocode}
%
% \begin{macro}{\th@margin}
%    Again this is only a variant of the theorem styles described
%    above without any new ideas.
%    \begin{macrocode}
\gdef\th@margin{\normalfont\slshape
  \def\@begintheorem##1##2{\item
       [\theorem@headerfont \llap{##2}\hskip\labelsep ##1]}%
\def\@opargbegintheorem##1##2##3{%
  \item[\theorem@headerfont \llap{##2}\hskip\labelsep ##1\ (##3)]}}
\endgroup
%</thm>
%    \end{macrocode}
% \end{macro}
%
%
%
%
% \subsection{Final Definitions}
%
% \begin{macro}{\theorempreskipamount}
% \begin{macro}{\theorempostskipamount}
%    The \textsf{skip} parameters that regulate the vertical empty space
%    before and after the {\envfont theorem} environment have
%    to be allocated as well.
% \changes{v2.0f}{89/07/20}{Unnecessary `global' removed.}
%    \begin{macrocode}
%<*package>
\newskip\theorempreskipamount
\newskip\theorempostskipamount
%    \end{macrocode}
%    Since we have used the same values for all theorem sets, we now can
%    assign them.
%    \begin{macrocode}
\global\setlength\theorempreskipamount{12pt plus 5pt minus 3pt}
\global\setlength\theorempostskipamount{8pt plus 3pt minus 1.5pt}
%    \end{macrocode}
% \end{macro}
% \end{macro}
%
%
% \begin{macro}{\@endtheorem}
%    The same holds for the macro |\@endtheorem|, which ends a
%    {\envfont theorem} environment. Since it is the same for all
%    theorem sets, it is removed from the macros |\th@|\meta{style}.
%    It simply ends the {\envfont trivlist} environment, which was
%    begun in |\@thm|.
%    \begin{macrocode}
\global\let\@endtheorem=\endtrivlist
%    \end{macrocode}
% \end{macro}
%
%
%
% \begin{macro}{\@preamblecmds}
%    All macros defined above are to be used only in the preamble.
%    Therefore, we insert them in |\@preamblecmds| which will disable
%    them at begin document. This is done by the internal \LaTeXe{}
%    command |\@onlypreamble|.
% \changes{v2.0g}{89/09/19}{`theoremheaderfont and `theorembodyfont
%    added.}
% \changes{v2.2a}{94/02/03}{Use \LaTeXe{} interface}
%    \begin{macrocode}
\@onlypreamble\@xnthm
\@onlypreamble\@ynthm
\@onlypreamble\@othm
\@onlypreamble\newtheorem
\@onlypreamble\theoremstyle
\@onlypreamble\theorembodyfont
\@onlypreamble\theoremheaderfont
%    \end{macrocode}
% \end{macro}
%
% Finally we declare the \texttt{plain} theorem style to be the default.
% \changes{v2.1b}{90/06/04}{Fixed missing `item by declaring default
%    style.}
%    \begin{macrocode}
\theoremstyle{plain}
%</package>
%    \end{macrocode}
%
% \Finale

\endinput
}{}
\ProvidesFile{thb.sty}
             [\FMithmInfo]
%    \end{macrocode}
%
% \begin{macro}{\th@break}
%    |\theoremstyle{break}| produces a line break after the name
%    of the theorem.  The font is |\slshape|.  Hence, we define
%    |\th@break| as follows:
%    \begin{macrocode}
\gdef\th@break{\normalfont\slshape
  \def\@begintheorem##1##2{\item[%
%    \end{macrocode}
%    We run into the following problem: it is not possible to create
%    the header with |\item[|\meta{title}|]| and then start
%    a new line by, for example, |\mbox{}\\|. Such a definition
%    will fail whenever a list environment follows immediately.  With
%    the above construction, the |\mbox{}| causes the switch
%    |@inlabel| (cf.\ definition of |\list| and
%    |\trivlist| in \cite{latex:source}) to be set to \texttt{false}
%    and so the following list will insert additional vertical
%    space (|\topskip|). This is quite annoying. Therefore, we
%    create the line break within the |\item|. In order to ensure
%    that the text will begin at the proper position in the following
%    line, we simply pretend that the label does not take any
%    room.\footnote{This will lead to problems whenever very high
%    symbols occurring in the line tower into the
%    heading. So, something else has to be done here sometime.}
%    \begin{macrocode}
       \rlap{\vbox{\hbox{\hskip \labelsep\theorem@headerfont ##1\ ##2}%
                   \hbox{\strut}}}]}%
%    \end{macrocode}
%    Again, the definition of |\@opargbegintheorem| is
%    completely analogous.
%    \begin{macrocode}
\def\@opargbegintheorem##1##2##3{%
  \item[\rlap{\vbox{\hbox{\hskip \labelsep \theorem@headerfont
                     ##1\ ##2\ (##3)}%
                    \hbox{\strut}}}]}}
\endgroup
%</thb>
%    \end{macrocode}
% \end{macro}
%
%
% \subsubsection{The \texttt{changebreak} style}
%
%    \begin{macrocode}
% This style option is stored in the file |thcb.sty|.
%    \begin{macrocode}
%<*thcb>
\begingroup \makeatletter
\@ifundefined{theorem@style}{% \iffalse meta-comment
%
% Copyright (C) 1993-2024
%
% The LaTeX Project and any individual authors listed elsewhere
% in this file.
%
% This file is part of the Standard LaTeX `Tools Bundle'.
% -------------------------------------------------------
%
% It may be distributed and/or modified under the
% conditions of the LaTeX Project Public License, either version 1.3c
% of this license or (at your option) any later version.
% The latest version of this license is in
%    https://www.latex-project.org/lppl.txt
% and version 1.3c or later is part of all distributions of LaTeX
% version 2005/12/01 or later.
%
% The list of all files belonging to the LaTeX `Tools Bundle' is
% given in the file `manifest.txt'.
%
% \fi
%
%%
%
%
\def\FMithmInfo{2023/07/05 v2.2c Theorem extension package (FMi)}
%
% \ProvidesFile{theorem.dtx}[\FMithmInfo]
%\iffalse   % this is a METACOMMENT !
%%
%% Package `theorem' to use with LaTeX2e
%% Copyright (C) 1989-1995 Frank Mittelbach, all rights reserved.
%<+package>\@ifundefined{theorem@style}{}{\endinput}
%<+package>\NeedsTeXFormat{LaTeX2e}[1995/06/01]
%<+package>\ProvidesPackage{theorem}
%<+package>         [\FMithmInfo]
%
% \fi
%
% \changes{v1.0}{87/05/07}{First draft.}
% \changes{v1.0a}{87/05/07}{`thistheoremstyle' added.}
% \changes{v1.1a}{87/07/29}{New concept: redefine the internal
%                          \protect\LaTeX{} commands `@xnthm' etc.}
% \changes{v1.1b}{87/07/31}{`theoremfont' used to specify font.}
% \changes{v1.2a}{88/01/08}{`@endtheorem' now globally defined.}
% \changes{v1.2b}{88/01/11}{`newtheorem' only allowed in the preamble.}
% \changes{v1.3a}{88/02/14}{`theoremfont' now a toks register.}
%
% \changes{v2.0a}{88/11/16}{New implementation and new documentation.}
% \changes{v2.0a}{88/11/16}{No layout changes but internal modifications
%                           old substyles won't work.}
% \changes{v2.0b}{89/01/30}{`global' used for all definitions.}
% \changes{v2.0c}{89/06/15}{Documentation translated to English.
%                            Thanks to LLL and Nina}
% \changes{v2.1a}{90/02/15}{Support for new font selection.}
% \changes{v2.2a}{94/02/03}{Cleanup for LaTeX2e}
%
% \DoNotIndex{\@endparenv,\@ifdefinable,\@ifnextchar,\@ifundefined}
% \DoNotIndex{\@namedef,\@nameuse,\@spaces,\@tempa,\@warning}
% \DoNotIndex{\ }
% \DoNotIndex{\begingroup,\bfseries}
% \DoNotIndex{\csname}
% \DoNotIndex{\def,\do,\docdate}
% \DoNotIndex{\endcsname,\endgroup,\endinput,\endtrivlist,\expandafter}
% \DoNotIndex{\filedate,\fileversion}
% \DoNotIndex{\gdef,\global}
% \DoNotIndex{\hbox,\hskip}
% \DoNotIndex{\ignorespaces,\input,\itshape,\item}
% \DoNotIndex{\labelsep,\let,\llap}
% \DoNotIndex{\makeatletter}
% \DoNotIndex{\noexpand}
% \DoNotIndex{\refstepcounter,\relax,\rlap}
% \DoNotIndex{\setlength,\slshape,\space,\string,\strut}
% \DoNotIndex{\the,\trivlist,\typeout}
% \DoNotIndex{\vbox}
% \DoNotIndex{\xdef}
%
% \def\deflabel#1{\ttfamily #1\hfill}
% \def\deflist#1{\begin{list}{}{\settowidth\labelwidth{\ttfamily #1}%
%                               \setlength\leftmargin\labelwidth
%                               \addtolength\leftmargin\labelsep
%                               \let\makelabel\deflabel}}
% \def\enddeflist{\end{list}}
% \def\envfont{\normalfont\ttfamily}
%
% \MakeShortVerb{\|}
%
% \newcommand{\TUB}{{\sl TUGboat\/}}
%
% \GetFileInfo{theorem.dtx}
% \title{An Extension of the
%        \LaTeX{} theorem environment\thanks{This file
%        has version number \fileversion, last
%        revised \filedate.}}
% \author{Frank Mittelbach}
%
% \MaintainedByLaTeXTeam{tools}
% \maketitle
%
% \begin{abstract}
%    The macros described in this paper yield an extension of the
%    \LaTeX{} theorem mechanism. It is designed to satisfy the
%    different requirements of various journals. Thus, the
%    layout of the ``theorems'' can be manipulated by determining a
%    ``style''. This article describes not only the use, but
%    also the definition, of the necessary macros.
% \end{abstract}
%
% \section*{Preface to version 2.2}
%
% For \LaTeXe{} this package did not need any fundamental changes. I
% only modified the messages generated so that theorem layout styles
% will show up with the |\listfiles| command and cleaned the section
% on the New Font Selection Scheme since this is now included in
% \LaTeX.
%
% \section*{Preface to version 2.1}
%
% This version is identical to 2.0g described in \TUB\ 10\#3 except
% for some internal defaults which are now set depending on the used
% font selection scheme.
%
% This was done to avoid unpleasant surprises if the new font selection
% scheme is in force. For further details see
% section~\ref{sec:newfonts} and \cite{bk:GMS94}.
%
% \section{Introduction}
%
% For our purposes here, ``theorems'' are labelled enunciations,
% often set off from the main text by extra space and a font change.
% Theorems, corollaries, conjectures, definitions, and remarks
% are all instances of ``theorems''.  The ``header'' of these structures
% is composed of a label (such as \textsc{Theorem} or \textsc{Remark})
% and a number which serializes an item in the sequence of items
% with the same label.
%
% Shortly after the introduction of \LaTeX{} at the Fachbereich
% Mathematik in Mainz, the desire to manipulate the layout of
% ``theorems'' arose.  In Mainz, the following two conventions came
% into general use:
% \begin{enumerate}
%   \item The number of the theorem is shown in the margin.
%   \item There is a line break at
%         the end of the theorem header.
% \end{enumerate}
%
% Additionally, some journals require different formats which depend on
% the ``sort of theorem'': e.g.\ often remarks and definitions are set
% in |\upshape|, while |\itshape| is employed for main theorems.
%
% Confronted with these requirements, a theorem environment was
% developed in Mainz which allows separate determination of the
% layout of the ``theorems sets'', comparable to |\pagestyle|.
%
%
% \section{The user interface}
% \subsection{Defining new theorem sets}
%
% \DescribeMacro\newtheorem
% As in the original \LaTeX{} version, the command |\newtheorem|
% defines a new ``theorem set'' or ``theorem-like structure''.
% Two required arguments name the new environment and give the text
% to be typeset with each instance of the new ``set'', while
% an optional argument determines how
% the ``set'' is enumerated:
% \begin{description}
%    \item[\ttfamily \bslash newtheorem\{foo\}\{bar\}]
%       The theorem set {\envfont foo} (whose name is \texttt{bar})
%       uses its own counter.
%    \item[\ttfamily \bslash newtheorem\{foo2\}{[foo]}\{bar2\}]
%       The theorem set {\envfont foo2} (printed name \texttt{bar2})
%       uses the same counter as the theorem set \texttt{foo}.
%    \item[\ttfamily \bslash newtheorem\{foo3\}\{bar3\}{[section]}]
%       The theorem set {\envfont foo3} (printed name \texttt{bar3}) is
%       enumerated within the counter \texttt{section}, i.e.\ with every
%       new |\section| the enumeration begins again with 1, and
%       the enumeration is composed from the section-number and the
%       theorem counter itself.
% \end{description}
%
% \DescribeMacro\theoremstyle
% Additionally, the command |\theoremstyle| can define the layout
% of various, or all, theorem sets.  It should be noted that any
% theorem set defined by |\newtheorem| is typeset in the
% |\theoremstyle| that is current at the time of the definition.
% Thus, the following
% \begin{verbatim}
%    \theoremstyle{break}        \newtheorem{Cor}{Corollary}
%    \theoremstyle{plain}        \newtheorem{Exa}{Example}[section]
%\end{verbatim}                             ^^A No blank at line start
% leads to the result that the set {\envfont Cor} is formatted in the
% style \texttt{break}, while the set {\envfont Exa} and all the
% following ones are formatted in the style \texttt{plain}, unless
% another |\theoremstyle| follows.  Since the definitions installed by
% |\newtheorem| are global, one also can limit |\theoremstyle| locally
% by grouping braces.
%
% \DescribeMacro\theorembodyfont
% The choice of the font for the theorem body is completely
% independent of the chosen |\theoremstyle|; this has proven to
% be very advantageous.  For example,
% \begin{verbatim}
%   {\theorembodyfont{\upshape}          \newtheorem{Rem}{Remark}}
%\end{verbatim}
% defines a theorem set {\envfont Rem}, which will be set in
% |\upshape| in the current layout (which in our example is
% \texttt{plain}).  As with |\theoremstyle|, the |\theorembodyfont|
% chosen is that current at the time of |\newtheorem|.  If
% |\theorembodyfont| is not specified or one defines
% |\theorembodyfont{}|, then the font used will be that defined by the
% |\theoremstyle|.
%
% \DescribeMacro\theoremheaderfont It is also possible to customize
% the font used for the theorem headers.  This is, however, a global
% declaration, and therefore there should be at most one
% |\theoremheaderfont| declaration in the preamble.\footnote{If it is
% actually necessary to have different header fonts, one has to define
% new theorem styles (substituting the desired font) or specify the
% information directly in the \texttt{\bslash newtheorem} declaration
% (the unclean variant).}
%
% \DescribeMacro\theorempreskipamount
% \DescribeMacro\theorempostskipamount Two additional parameters
% affect the vertical space around the theorem environments:
% |\theorempreskipamount| and |\theorempostskipamount| define,
% respectively, the spacing before and after such an environment.
% These parameters apply for all theorem sets and can be manipulated
% with the ordinary length macros.  They are rubber lengths
% (`\textsf{skips}'), and therefore can contain \texttt{plus} and
% \texttt{minus} parts.
%
% Since the definition of theorem sets should---most sensibly---be
% placed in the preamble, we only allow installation there.
% It is therefore possible to release the memory used here after
% |\begin{document}|, in order to make room for other
% applications.
%
%
% \subsection{Existing theorem styles}
%
% The following theorem styles exist to date:
% \begin{deflist}{marginbreak}
%    \item[plain]
%       This theorem style emulates the original \LaTeX{} definition,
%       except that additionally the parameters
%       |\theorem...skipamount| are used.
%    \item[break]
%       In this style, the theorem header is followed by a line
%       break.
%    \item[marginbreak]
%       The theorem number is set in the margin, and there is a line
%       break as in \texttt{break}.
%    \item[changebreak]
%       Like \texttt{break}, but with header number and text
%       interchanged.
%    \item[change]
%       Header number and text are interchanged, without a line break.
%    \item[margin]
%       The number is set in the left margin, without a line break.
% \end{deflist}
% All styles (except \texttt{plain}) select |\slshape| as the default
% |\theorembodyfont|.
%
%
% \subsection{Examples}
%
% Given the above theorem sets {\envfont Cor}, {\envfont Exa}
% and {\envfont Rem},
% suppose that the preamble also contains the declarations:
% \begin{verbatim}
%    \theoremstyle{marginbreak}   \newtheorem{Lem}[Cor]{Lemma}
%    \theoremstyle{change}
%    \theorembodyfont{\itshape}        \newtheorem{Def}[Cor]{Definition}
%
%    \theoremheaderfont{\scshape}
%\end{verbatim}
% Then the following are some typical examples of the typeset output
% resulting from their use.
%
% \begin{Cor}
%    This is a sentence typeset in the theorem environment {\envfont
%    Cor}.
% \end{Cor}
%
% \begin{Exa}
%    This is a sentence typeset in the theorem environment {\envfont
%    Exa}.
% \end{Exa}
%
% \begin{Rem}
%    This is a sentence typeset in the theorem environment {\envfont
%    Rem}.
% \end{Rem}
%
% \begin{Lem}[Ben User]
%    This is a sentence typeset in the theorem environment {\envfont
%    Lem}.
% \end{Lem}
%
% \begin{Def}[Very impressive Definition]
%    This is a sentence typeset in the theorem environment {\envfont
%    Def}.
% \end{Def}
%
% The last two examples show the effect of the optional argument to
% a theorem environment (it is the text typeset in parentheses).
%
%
% \section{Special Considerations}
% \label{sec:newfonts}
%
% Theoremheader and body are implemented as a unit. This means that
% the |\theoremheaderfont|  will inherit characteristics of the
% |\theorembodyfont| in \LaTeXe. Thus, if for example
% |\theorembodyfont| is |\itshape| and |\theoremheaderfont|
% is |\bfseries| the font selected for the header will have the
% characteristics `bold extended italic'. If this is not desired one
% should set the |\theoremheaderfont| to something like
% \begin{verbatim}
%      \theoremheaderfont{\normalfont\bfseries}
%\end{verbatim}
% i.e.\ supplying all necessary font information explicitly.
%
%
%
% \section{Acknowledgements}
%
% The publication of this set of macros was only possible with the
% help of Christina Busse (translating the manuscript into English),
% Joachim Pense (playing the r\^ole of typist), Chris Rowley
% (looking everything over) and many others providing useful
% suggestions.
%
% \MaybeStop{%
%    \begin{thebibliography}{9}
%       \bibitem{bk:GMS94} \textsc{M.~Goossens}, \textsc{F.~Mittelbach}
%         and \textsc{A.~Samarin}.
%         \newblock The \LaTeX{} Companion.
%         \newblock
%         Addison-Wesley, Reading, Massachusetts, 1994.
%       \bibitem{latex:source} \textsc{Lamport, Leslie.} \newblock
%                               \texttt{latex.tex}, version 2.09, date
%                               Feb. 1990.
%    \end{thebibliography}
%    \PrintIndex
%    \PrintChanges}
%
%
% \section{The documentation driver file}
%
% The next bit of code contains the documentation driver file for
% \TeX{}, i.e., the file that will produce the documentation you are
% currently reading. It will be extracted from this file by the
% \texttt{docstrip} program. Since it is the first code in the file
% one can alternatively process this file directly with \LaTeXe{} to
% obtain the documentation.
%
%    \begin{macrocode}
%<*driver>
\documentclass{ltxdoc}

\usepackage{theorem}

%<+driver>% The next few lines define theorem sets which are used
%<+driver>% in the example section of the documentation.

\theoremstyle{break}       \newtheorem{Cor}{Corollary}
\theoremstyle{plain}       \newtheorem{Exa}{Example}[section]
{\theorembodyfont{\upshape}\newtheorem{Rem}{Remark}}
\theoremstyle{marginbreak} \newtheorem{Lem}[Cor]{Lemma}
\theoremstyle{change}
\theorembodyfont{\itshape} \newtheorem{Def}[Cor]{Definition}

\theoremheaderfont{\scshape}

\RecordChanges

\begin{document}
   \DocInput{theorem.dtx}
\end{document}
%</driver>
%    \end{macrocode}
%
% \section{Definition of the Macros}
%
% If the file has been loaded before, we abort immediately. If not the
% package announces itself (this is actually done at the very top of
% the file---the way it is done isn't good style so don't copy it).
% \changes{v2.0e}{89/07/19}{Spaces removed from `typeout'.}
% \changes{v2.2b}{95/11/19}{Announce moved to top of file}
%    \begin{macrocode}
%<*package>
%\@ifundefined{theorem@style}{}{\endinput}
%\def\FMithmInfo{1995/11/19 v2.2b Theorem extension package (FMi)}
%\ProvidesPackage{theorem}[\FMithmInfo]
%    \end{macrocode}
%
%
% \subsection{Definition of theorem styles and fonts}
%
% All the definitions in this file are done globally to allow
% inputting this file inside a group.
%
% \begin{macro}{\theoremstyle}
%    Before a theorem style can be installed, the chosen style must be
%    known. For that reason, we must test to see that
%    |\th@|\meta{style} is known or, more precisely, that it is
%    different from |\relax|. If the style is not known then
%    |\th@plain| is used.
%    \begin{macrocode}
\gdef\theoremstyle#1{%
   \@ifundefined{th@#1}{\@warning
          {Unknown theoremstyle `#1'. Using `plain'}%
          \theorem@style{plain}}%
%    \end{macrocode}
%    We save the theorem style to be used in the token register
%    |\theorem@style|.
%    \begin{macrocode}
      {\theorem@style{#1}}%
%    \end{macrocode}
%    Now we ``evaluate'' the theorem style: this means, we call the
%    macro |\th@|\meta{style} which will activate the relevant
%    definitions which are contained in a separate file.  This is done
%    in a group to suppress changes to the current font.  This could
%    otherwise pose problems together with the new font selection
%    scheme\footnote{When I printed the original article using the new
%    font selection scheme I ended with a document with slanted
%    typefaces (text headings and all) simply because one of the
%    theorem styles used \texttt{\bslash sl} at toplevel.} if the
%    |\th@|\meta{style} is evaluated a second time.
% \changes{v2.1a}{90/02/15}{Group added around `th@..}
% \changes{v2.1a}{90/02/15}{Reseting of `..@begintheorem removed.}
%    \begin{macrocode}
      \begingroup
        \csname th@\the\theorem@style \endcsname
      \endgroup}
%    \end{macrocode}
% \end{macro}
%
%
% \begin{macro}{\@begintheorem}
% \changes{v2.1a}{90/02/15}{Relaxed at toplevel.}
% \begin{macro}{\@opargbegintheorem}
% \changes{v2.1a}{90/02/15}{Relaxed at toplevel.}
%    We reset |\@begintheorem| and |\@opargbegintheorem| to
%    |\relax| since these commands are no longer necessary at
%    toplevel. This will save a few tokens.
%    \begin{macrocode}
\global\let\@begintheorem\relax
\global\let\@opargbegintheorem\relax
%    \end{macrocode}
% \end{macro}
% \end{macro}
%
% \begin{macro}{\theorem@style}
%    Obviously the \textsf{token} register used above has to be
%    allocated. To assure the utmost compatibility with the
%    original \LaTeX{} definition, we set the default theorem style to
%    \texttt{plain}, which implements the usual \LaTeX{} convention.
% \changes{v2.0f}{89/07/20}{Unnecessary `global' removed.}
%    \begin{macrocode}
\newtoks\theorem@style
\global\theorem@style{plain}
%    \end{macrocode}
% \end{macro}
%
% \begin{macro}{\theorembodyfont}
% \begin{macro}{\theorem@bodyfont}
%    If the |\theorembodyfont| is set by the user then it should not
%    interact with the default font set in the theorem style. When the
%    new font selection is in force this may happen if, for example,
%    the default is |\itshape| and the new |\theorembodyfont| is
%    |\sffamily|. So we add a |\reset@font| command in front of the
%    user definition.
% \changes{v2.0f}{89/07/20}{Unnecessary `global' removed.}
% \changes{v2.0g}{89/09/19}{`theoremfont to `theorembodyfont.}
% \changes{v2.1c}{92/06/22}{`reset@font added and logic changed.}
% \changes{v2.1d}{92/10/28}{Reset to empty if arg is empty}
%    \begin{macrocode}
\gdef\theorembodyfont#1{%
%    \end{macrocode}
%    We check if the argument supplied is empty and if so put nothing
%    into the |\theorem@bodyfont| token register to allow for
%    |\theorembodyfont{}| as a mean of using the default of the
%    current |\theoremstyle|.
%    \begin{macrocode}
   \def\@tempa{#1}%
   \ifx\@tempa\@empty
    \theorem@bodyfont{}%
   \else
    \theorem@bodyfont{\reset@font#1}%
   \fi
   }
\newtoks\theorem@bodyfont
\global\theorem@bodyfont{}
%    \end{macrocode}
% \end{macro}
% \end{macro}
%
%
% \begin{macro}{\theoremheaderfont}
% \changes{v2.0g}{89/09/19}{Macro added.}
%    The font for the theorem headers is handled differently because
%    this definition applies to all theorem styles.
%    \begin{macrocode}
\gdef\theoremheaderfont#1{\gdef\theorem@headerfont{#1}%
%    \end{macrocode}
%    After using the macro once it is redefined to produce an error
%    message.
%    \begin{macrocode}
       \gdef\theoremheaderfont##1{%
        \typeout{\string\theoremheaderfont\space should be used
                 only once.}}}
%    \end{macrocode}
% \end{macro}
%
% \begin{macro}{\theorem@headerfont}
% \changes{v2.0g}{89/09/19}{Macro added.}
%    To set the |\theorem@headerfont| default we first test if
%    the new fontselection scheme is in force.
% \changes{v2.1a}{90/02/15}{Support for new font selection.}
%    \begin{macrocode}
\ifx\upshape\undefined
%    \end{macrocode}
%    If not we define it to expand into |\bfseries|. We don't use
%    |\let| just in case a following style option redefines this
%    macro.
%    \begin{macrocode}
\gdef\theorem@headerfont{\bfseries}
%    \end{macrocode}
%    Otherwise we reset the current shape before calling |\bfseries|.
%    \begin{macrocode}
\else \gdef\theorem@headerfont{\normalfont\bfseries}\fi
%    \end{macrocode}
% \end{macro}
%
% \begin{macro}{\th@plain}
% \begin{macro}{\th@break}
% \begin{macro}{\th@marginbreak}
% \begin{macro}{\th@changebreak}
% \begin{macro}{\th@change}
% \begin{macro}{\th@margin}
%    The different styles are defined in macros such as
%    |\th@plain|.  Since memory space is precious in
%    ``non-Big-versions'', we have to avoid offering too many unused
%    definitions.  Therefore we define these styles in separate files
%    that can be loaded on demand.  Thus the commands themselves only
%    load these files. We use |\@input@|, a \LaTeXe{} internal command
%    that ensures that the file will be listed with |\listfiles|.
% \changes{v2.2a}{94/02/02}{Use `@input@ to load theorem layout files}
%    \begin{macrocode}
\gdef\th@plain{\@input@{thp.sty}}
\gdef\th@break{\@input@{thb.sty}}
\gdef\th@marginbreak{\@input@{thmb.sty}}
\gdef\th@changebreak{\@input@{thcb.sty}}
\gdef\th@change{\@input@{thc.sty}}
\gdef\th@margin{\@input@{thm.sty}}
%    \end{macrocode}
%    This list will be expanded when new styles become
%    available.  For testing,  just append new theorem substyles
%    as document options.
% \end{macro}
% \end{macro}
% \end{macro}
% \end{macro}
% \end{macro}
% \end{macro}
%
%
%
% \subsection{Definition of a new theorem set}
%
% As already pointed out, a new theorem environment can be defined in
% three different ways:
% \begin{verbatim}
%    \newtheorem{Lem}{Lemma}
%    \newtheorem{Lem}{Lemma}[section]
%    \newtheorem{Lem}[Theorem]{Lemma}
%\end{verbatim}
% The function of the macro |\newtheorem| is to recognize these cases
% and then to branch into one of the three macros |\@ynthm|, |\@xnthm|
% or |\@othm|.  This mechanism is adopted unchanged from
% \cite{latex:source}; the essential point here is that, for example,
% in the second case, the arguments \texttt{Lem}, \texttt{Lemma} and
% \texttt{section} are passed over to the macro |\@xnthm|.
%
% We inspect this case first because the others present fewer problems,
% and thus are easily derived from this one.
%
% \begin{macro}{\@xnthm}
%    For our example arguments, the macro |\@xnthm| must fulfill
%    the following:
%    \begin{itemize}
%       \item
%          Define a new \LaTeX-counter `Lem'
%       \item
%          reset this counter within a |\section|
%       \item
%          define the macro |\theLem|
%       \item
%          define the environment macros |\Lem| and
%          |\endLem| using the current
%          |\theoremstyle| and |\theorem@bodyfont|.
%    \end{itemize}
%    Obviously, all this should happen only if the first argument
%    of |\@xnthm| (i.e.\ \texttt{Lem} in our example) is chosen so
%    as not to conflict with any previously defined commands or
%    environments.  This test is performed by the \LaTeX{} macro
%    |\@ifdefinable|.
%    \begin{macrocode}
\gdef\@xnthm#1#2[#3]{\expandafter\@ifdefinable\csname #1\endcsname
%    \end{macrocode}
%    Therefore, the first argument of |\@ifdefinable| is the
%    expansion (in the example, |\Lem|) of
%    |\csname||#1\endcsname|.  The second argument is
%    executed only if the test has been completed successfully.
%    \begin{macrocode}
   {%
%    \end{macrocode}
%    Now we define the new counter. The names of the \LaTeX{}
%    macros employed should speak for themselves:
% \changes{v2.2b}{1995/10/07}{Use \cs{@newctr}}
% \changes{v2.2c}{1995/11/23}{Use \cs{@newctr} properly.}
%    \begin{macrocode}
    \@definecounter{#1}\@newctr{#1}[#3]%
%    \end{macrocode}
%    Using |\@newctr| will give a proper error message if the counter
%    in |#3| is not defined.
%    In defining `|\theLem|' we must generate the desired macro
%    name by use of |\expandafter| and |\csname|.
%    \begin{macrocode}
    \expandafter\xdef\csname the#1\endcsname
%    \end{macrocode}
%    An |\xdef| is used in order to make the definition global,
%    and to ensure that it contains the replacement texts of
%    |\@thmcountersep| and |\@thmcounter|.\footnote{These
%    two macros can be defined by the document style.  Their default
%    values produce a `\texttt{.}' as separation and an arabic
%    representation of the number.} However, not everything should be
%    expanded. For example, it saves space to use |\thesection|
%    instead of its---at times---lengthy expansion.
%    \begin{macrocode}
      {\expandafter \noexpand \csname the#3\endcsname
       \@thmcountersep \@thmcounter{#1}}%
%    \end{macrocode}
%    Thus with the defaults of \LaTeX{}, |\theLem| would be
%    replaced by the command sequence |\thesection.\arabic{Lem}|.
%
%    We will now look at the definition of the macro which is executed
%    at the beginning of the actual environment (in our example this
%    macro is |\Lem|). It should be noted that we use an
%    ``|\expandafter| trick'' to expand only certain parts of the
%    replacement text at the time of the definition.
%    \begin{macrocode}
    \def\@tempa{\global\@namedef{#1}}%
    \expandafter \@tempa \expandafter{%
%    \end{macrocode}
%    First, the macro that contains the current definitions of
%    |\@begintheorem| and |\@opargtheorem| should be called
%    up.  The name of this macro---as is already known---has the form
%    |\th@|\meta{theorem\ style}; therefore, it must be called by
%    \begin{macrocode}
      \csname th@\the \theorem@style
%    \end{macrocode}
%    In addition the default theorem font should be changeable, i.e.\
%    we have to insert the contents of |\theorem@bodyfont|. For
%    that reason, we expand even further, beyond |\endcsname|,
%    and thus insert the contents of the \textsf{token} register
%    |\theorem@bodyfont| in the replacement text.
%    \begin{macrocode}
            \expandafter \endcsname \the \theorem@bodyfont
%    \end{macrocode}
%    Now it is time to call the macro |\@thm| which takes over
%    the further processing.  It has two arguments: the current
%    counter name (in our example, \texttt{Lem}), and the text of the
%    label (in our example, \texttt{Lemma}).
%    \begin{macrocode}
     \@thm{#1}{#2}}%
%    \end{macrocode}
%    With this, the `sub-definition' is complete.  The macro
%    |\@endtheorem| ends a theorem environment and is, so far, nothing
%    but an |\endtrivlist|.  (Hence it is defined globally, and not
%    within the theorem styles.\footnote{This has to be changed as
%    soon as theorem styles that change \texttt{\bslash @endtheorem}
%    exist.  In such a case, all existing styles must be changed as
%    well since they will have to reset the macro.}) Therefore, we can
%    set it equivalent to the macro that ends the theorem set (in our
%    example, |\endLem|).  However, if some day theorem styles exist
%    that do change |\@endtheorem|, we would have to use the
%    commented-out line instead.
%    \begin{macrocode}
    \global \expandafter \let \csname end#1\endcsname \@endtheorem
%   \global\@namedef{end#1}{\@endtheorem}%
%    \end{macrocode}
%    With these commands all the required definitions are employed,
%    unless the test |\@ifdefinable| has failed.  Therefore, we
%    end the second argument of this macro and with it the definition
%    of |\@xnthm|.
%    \begin{macrocode}
   }}
%    \end{macrocode}
% \end{macro}
%
% \begin{macro}{\@ynthm}
%    The definition of |\@ynthm| is completely analogous. In
%    this case the new counter that is defined is not reset within
%    another counter; thus the definition of |\the...|
%    is simplified:
%    \begin{macrocode}
\gdef\@ynthm#1#2{\expandafter\@ifdefinable\csname #1\endcsname
   {\@definecounter{#1}%
    \expandafter\xdef\csname the#1\endcsname{\@thmcounter{#1}}%
%    \end{macrocode}
%    The rest of the definition corresponds literally to that of
%    |\@xnthm|:
%    \begin{macrocode}
    \def\@tempa{\global\@namedef{#1}}\expandafter \@tempa
     \expandafter{\csname th@\the \theorem@style \expandafter
     \endcsname \the\theorem@bodyfont \@thm{#1}{#2}}%
    \global \expandafter \let \csname end#1\endcsname \@endtheorem}}
%    \end{macrocode}
% \end{macro}
%
% \begin{macro}{\@othm}
%    The definition of |\@othm| does not contain anything new.
%    \begin{macrocode}
\gdef\@othm#1[#2]#3{%
%    \end{macrocode}
% \changes{v2.2b}{95/10/07}{Build in check for counter defined}
%    We do not define a new counter but instead use one that has
%    already been defined.  Thus the only definition we need is that
%    of this pseudo-counter (i.e.\ |\the|\meta{env.\ name}).
%    First we check if |#2| corresponds to a known counter name.
%    \begin{macrocode}
  \expandafter\ifx\csname c@#2\endcsname\relax
   \@nocounterr{#2}%
  \else
   \expandafter\@ifdefinable\csname #1\endcsname
   {\expandafter \xdef \csname the#1\endcsname
     {\expandafter \noexpand \csname the#2\endcsname}%
%    \end{macrocode}
%    All other parts of the definition can be adopted from
%    |\@xnthm|. We have to remember, though, that in this case
%    the name of the current counter and the theorem label have moved
%    to the second and third arguments.
%    \begin{macrocode}
    \def\@tempa{\global\@namedef{#1}}\expandafter \@tempa
     \expandafter{\csname th@\the \theorem@style \expandafter
     \endcsname \the\theorem@bodyfont \@thm{#2}{#3}}%
    \global \expandafter \let \csname end#1\endcsname \@endtheorem}%
  \fi}
%    \end{macrocode}
% \end{macro}
%
% \subsection{Macros that are employed in a {\envfont theorem}
%    environment}
%
% \begin{macro}{\@thm}
%    The macro |\@thm| has to increase the current counter.  Then,
%    depending on whether the environment has (or does not have) an
%    optional argument, it has to branch into either |\@begintheorem|
%    or |\@opargtheorem|.
%    \begin{macrocode}
\gdef\@thm#1#2{\refstepcounter{#1}%
%    \end{macrocode}
%    Now we start a {\envfont trivlist} environment, and give
%    |\@topsep| and |\@topsepadd| the values of the \textsf{skip}
%    registers |\theorempreskipamount| and |\theorempostskipamount|.
%    The value in |\@topsep| is the vertical space that is inserted by
%    the first (and only) |\item| in our |\trivlist| whilst
%    |\@topsepadd| is inserted by |\@endparenv| at the end of that
%    {\envfont trivlist} environment.  By using these registers, we
%    obtain the desired space around a {\envfont theorem} environment.
%    \begin{macrocode}
   \trivlist
   \@topsep \theorempreskipamount               % used by first \item
   \@topsepadd \theorempostskipamount           % used by \@endparenv
%    \end{macrocode}
%    Now we have to test whether an optional argument has been given.
% \changes{v2.0d}{89/07/18}{Forgotten percent added.}
%    \begin{macrocode}
   \@ifnextchar [%
%    \end{macrocode}
%    If there is an optional argument, we will call |\@ythm|, and
%    move the arguments read back into the input stream.
%    \begin{macrocode}
   {\@ythm{#1}{#2}}%
%    \end{macrocode}
%    If not, we call |\@begintheorem|. Its first argument is the
%    name of the theorem set (hence the second argument of
%    |\@thm|). Its second argument is the macro that produces the
%    current number.
%    \begin{macrocode}
   {\@begintheorem{#2}{\csname the#1\endcsname}\ignorespaces}}
%    \end{macrocode}
% \end{macro}
%
% \begin{macro}{\@xthm}
% \begin{macro}{\@ythm}
%    Both these macros were originally called by |\@thm|.  We do
%    not need |\@xthm| anymore, hence we reset it to
%    |\relax|. The definition of |\@ythm| has not changed at
%    all from its definition in \LaTeX. In order to make the macros
%    easier to understand, we will
%    nevertheless present it (commented out).
%    \begin{macrocode}
\global\let\@xthm\relax
% \def\@ythm#1#2[#3]{\@opargbegintheorem{#2}{\csname
%           the#1\endcsname}{#3}\ignorespaces}
%</package>
%    \end{macrocode}
%    The primitive |\ignorespaces| in |\@ythm| and
%    |\@thm| is needed to remove the spaces between the
%    |\begin{...}| and the actual text.
% \end{macro}
% \end{macro}
%
%
% \subsection{Definition of the theorem substyles}
%
% As already pointed out, the theorem substyles, defined below, are
% only loaded when necessary.  Note that all these substyles, except
% \texttt{plain}, have |\slshape| as the default body font.
%
% \subsubsection{The \texttt{plain} style}
%
% \label{dok:th@plain} As the following macros use |@|, we have
% to locally set the |\catcode| of this symbol to ``letter''.
% This happens within a group, so that we do not have to worry about
% which |\catcode| that symbol had before.
%    \begin{macrocode}
%<*thp>
\begingroup \makeatletter
%    \end{macrocode}
% Since we are now within a group, we must make all definitions
% globally.  First we make sure that \texttt{theorem.sty} is loaded.
% This will allow us to use this file as a document style option
% without having to call \texttt{theorem} itself as an option.  At the
% same time, we assure that at least version 2 is loaded, since
% |\theorem@style| was not defined in earlier versions.
%    \begin{macrocode}
\@ifundefined{theorem@style}{\input{theorem.sty}}{}
\ProvidesFile{thp.sty}
             [\FMithmInfo]
%    \end{macrocode}
%
%
% \begin{macro}{\th@plain}
% \begin{macro}{\@begintheorem}
% \begin{macro}{\@opargbegintheorem}
%    |\theoremstyle{plain}| corresponds to the original
%    definition, except that the distances to the surrounding text are
%    determined by the parameters |\theorempreskipamount| and
%    |\theorempostskipamount|.
%    First we set the default body font.
%    \begin{macrocode}
\gdef\th@plain{\normalfont\itshape
%    \end{macrocode}
%    Then we define |\@begintheorem| and
%    |\@opargbegintheorem|. These two macros define how the
%    header of a theorem is typeset.  |\@opargbegintheorem| will
%    be called if a \textsf{theorem} environment with an optional
%    argument is encountered; otherwise, the header is constructed by
%    calling |\@begintheorem|.  If one of these macros is
%    executed, we are within a \textsf{trivlist} environment started by
%    |\@thm|.  So the theorem header is produced with an
%    |\item| command.
%
%    Instead of specifying the header font directly, all standard
%    theorem styles use the |\theorem@headerfont| macro to allow
%    customization.  The extra space (|\labelsep|) is necessary
%    because of problems in the \textsf{trivlist} environment.
%    \begin{macrocode}
  \def\@begintheorem##1##2{%
        \item[\hskip\labelsep \theorem@headerfont ##1\ ##2]}%
%    \end{macrocode}
%    The definition of |\@opargbegintheorem| is completely
%    analogous. The only difference is the fact that there exists a
%    third argument (which is the optional parameter of the
%    environment and contains additional information about the
%    theorem).  Customarily we enclose it in parentheses.
%    \begin{macrocode}
\def\@opargbegintheorem##1##2##3{%
   \item[\hskip\labelsep \theorem@headerfont ##1\ ##2\ (##3)]}}
%    \end{macrocode}
% \end{macro}
% \end{macro}
% \end{macro}
%
% \noindent
% We conclude with an |\endgroup| to restore the
% |\catcode| of |@|.
%    \begin{macrocode}
\endgroup
%</thp>
%    \end{macrocode}
%
%
% \subsubsection{The \texttt{break} style}
%
% This style option is stored in the file |thb.sty|.
% For the next two lines see the documentation for |\th@plain| on
% page~\pageref{dok:th@plain}.
%    \begin{macrocode}
%<*thb>
\begingroup \makeatletter
\@ifundefined{theorem@style}{\input{theorem.sty}}{}
\ProvidesFile{thb.sty}
             [\FMithmInfo]
%    \end{macrocode}
%
% \begin{macro}{\th@break}
%    |\theoremstyle{break}| produces a line break after the name
%    of the theorem.  The font is |\slshape|.  Hence, we define
%    |\th@break| as follows:
%    \begin{macrocode}
\gdef\th@break{\normalfont\slshape
  \def\@begintheorem##1##2{\item[%
%    \end{macrocode}
%    We run into the following problem: it is not possible to create
%    the header with |\item[|\meta{title}|]| and then start
%    a new line by, for example, |\mbox{}\\|. Such a definition
%    will fail whenever a list environment follows immediately.  With
%    the above construction, the |\mbox{}| causes the switch
%    |@inlabel| (cf.\ definition of |\list| and
%    |\trivlist| in \cite{latex:source}) to be set to \texttt{false}
%    and so the following list will insert additional vertical
%    space (|\topskip|). This is quite annoying. Therefore, we
%    create the line break within the |\item|. In order to ensure
%    that the text will begin at the proper position in the following
%    line, we simply pretend that the label does not take any
%    room.\footnote{This will lead to problems whenever very high
%    symbols occurring in the line tower into the
%    heading. So, something else has to be done here sometime.}
%    \begin{macrocode}
       \rlap{\vbox{\hbox{\hskip \labelsep\theorem@headerfont ##1\ ##2}%
                   \hbox{\strut}}}]}%
%    \end{macrocode}
%    Again, the definition of |\@opargbegintheorem| is
%    completely analogous.
%    \begin{macrocode}
\def\@opargbegintheorem##1##2##3{%
  \item[\rlap{\vbox{\hbox{\hskip \labelsep \theorem@headerfont
                     ##1\ ##2\ (##3)}%
                    \hbox{\strut}}}]}}
\endgroup
%</thb>
%    \end{macrocode}
% \end{macro}
%
%
% \subsubsection{The \texttt{changebreak} style}
%
%    \begin{macrocode}
% This style option is stored in the file |thcb.sty|.
%    \begin{macrocode}
%<*thcb>
\begingroup \makeatletter
\@ifundefined{theorem@style}{\input{theorem.sty}}{}
\ProvidesFile{thcb.sty}
             [\FMithmInfo]
%    \end{macrocode}
%
% \begin{macro}{\th@changebreak}
%    The \texttt{change-break} theorem style is like \texttt{break}
%    but with interchange of theorem name and theorem number.  Thus we
%    define |\th@changebreak| as follows:
%    \begin{macrocode}
\gdef\th@changebreak{\normalfont\slshape
  \def\@begintheorem##1##2{\item
       [\rlap{\vbox{\hbox{\hskip\labelsep \theorem@headerfont ##2\ ##1}%
                    \hbox{\strut}}}]}%
\def\@opargbegintheorem##1##2##3{%
  \item[\rlap{\vbox{\hbox{\hskip\labelsep \theorem@headerfont
                     ##2\ ##1\ (##3)}%
                    \hbox{\strut}}}]}}
\endgroup
%</thcb>
%    \end{macrocode}
% \end{macro}
%
%
% \subsubsection{The \texttt{change} style}
%
% This style option is stored in the file |thc.sty|.
%    \begin{macrocode}
%<*thc>
\begingroup \makeatletter
\@ifundefined{theorem@style}{\input{theorem.sty}}{}
\ProvidesFile{thc.sty}
             [\FMithmInfo]
%    \end{macrocode}
%
% \begin{macro}{\th@change}
%    The \texttt{change} theorem style corresponds to the
%    \texttt{change break} style without a linebreak after the header.
%    To say it in another way, it's the same as the \texttt{plain}
%    style but with number and name interchanged and |\slshape| as the
%    default font.
%    \begin{macrocode}
\gdef\th@change{\normalfont\slshape
  \def\@begintheorem##1##2{\item
       [\hskip\labelsep \theorem@headerfont ##2\ ##1]}%
\def\@opargbegintheorem##1##2##3{%
  \item[\hskip\labelsep \theorem@headerfont ##2\ ##1\ (##3)]}}
\endgroup
%</thc>
%    \end{macrocode}
% \end{macro}
%
%
% \subsubsection{The \texttt{marginbreak} style}
%
% This style option is the one used most often at Mainz.  It is saved
% in the file |thmb.sty|.
%    \begin{macrocode}
%<*thmb>
\begingroup \makeatletter
\@ifundefined{theorem@style}{\input{theorem.sty}}{}
\ProvidesFile{thmb.sty}
             [\FMithmInfo]
%    \end{macrocode}
%
% \begin{macro}{\th@marginbreak}
%    The \texttt{margin break} style is nearly the same as the
%    \texttt{change break} style. The only difference is the placement
%    of the theorem number.  We use |\llap| to place it in the left
%    margin.
%
%    In this style |\labelsep| denotes the separation between the
%    number and the text.
%    \begin{macrocode}
\gdef\th@marginbreak{\normalfont\slshape
  \def\@begintheorem##1##2{\item
       [\rlap{\vbox{\theorem@headerfont
                    \hbox{\llap{##2}\hskip\labelsep ##1}%
                    \hbox{\strut}}}]}%
\def\@opargbegintheorem##1##2##3{%
  \item[\rlap{\vbox{\theorem@headerfont
                    \hbox{\llap{##2}\hskip\labelsep ##1\ (##3)}%
                    \hbox{\strut}}}]}}
\endgroup
%</thmb>
%    \end{macrocode}
% \changes{v2.1c}{90/02/17}{`endgroup outside macrocode env.}
% \end{macro}
%
%
%
% \subsubsection{The \texttt{margin} style}
%
% This style option is stored in the file |thm.sty|.
%    \begin{macrocode}
%<*thm>
\begingroup \makeatletter
\@ifundefined{theorem@style}{\input{theorem.sty}}{}
\ProvidesFile{thm.sty}
             [\FMithmInfo]
%    \end{macrocode}
%
% \begin{macro}{\th@margin}
%    Again this is only a variant of the theorem styles described
%    above without any new ideas.
%    \begin{macrocode}
\gdef\th@margin{\normalfont\slshape
  \def\@begintheorem##1##2{\item
       [\theorem@headerfont \llap{##2}\hskip\labelsep ##1]}%
\def\@opargbegintheorem##1##2##3{%
  \item[\theorem@headerfont \llap{##2}\hskip\labelsep ##1\ (##3)]}}
\endgroup
%</thm>
%    \end{macrocode}
% \end{macro}
%
%
%
%
% \subsection{Final Definitions}
%
% \begin{macro}{\theorempreskipamount}
% \begin{macro}{\theorempostskipamount}
%    The \textsf{skip} parameters that regulate the vertical empty space
%    before and after the {\envfont theorem} environment have
%    to be allocated as well.
% \changes{v2.0f}{89/07/20}{Unnecessary `global' removed.}
%    \begin{macrocode}
%<*package>
\newskip\theorempreskipamount
\newskip\theorempostskipamount
%    \end{macrocode}
%    Since we have used the same values for all theorem sets, we now can
%    assign them.
%    \begin{macrocode}
\global\setlength\theorempreskipamount{12pt plus 5pt minus 3pt}
\global\setlength\theorempostskipamount{8pt plus 3pt minus 1.5pt}
%    \end{macrocode}
% \end{macro}
% \end{macro}
%
%
% \begin{macro}{\@endtheorem}
%    The same holds for the macro |\@endtheorem|, which ends a
%    {\envfont theorem} environment. Since it is the same for all
%    theorem sets, it is removed from the macros |\th@|\meta{style}.
%    It simply ends the {\envfont trivlist} environment, which was
%    begun in |\@thm|.
%    \begin{macrocode}
\global\let\@endtheorem=\endtrivlist
%    \end{macrocode}
% \end{macro}
%
%
%
% \begin{macro}{\@preamblecmds}
%    All macros defined above are to be used only in the preamble.
%    Therefore, we insert them in |\@preamblecmds| which will disable
%    them at begin document. This is done by the internal \LaTeXe{}
%    command |\@onlypreamble|.
% \changes{v2.0g}{89/09/19}{`theoremheaderfont and `theorembodyfont
%    added.}
% \changes{v2.2a}{94/02/03}{Use \LaTeXe{} interface}
%    \begin{macrocode}
\@onlypreamble\@xnthm
\@onlypreamble\@ynthm
\@onlypreamble\@othm
\@onlypreamble\newtheorem
\@onlypreamble\theoremstyle
\@onlypreamble\theorembodyfont
\@onlypreamble\theoremheaderfont
%    \end{macrocode}
% \end{macro}
%
% Finally we declare the \texttt{plain} theorem style to be the default.
% \changes{v2.1b}{90/06/04}{Fixed missing `item by declaring default
%    style.}
%    \begin{macrocode}
\theoremstyle{plain}
%</package>
%    \end{macrocode}
%
% \Finale

\endinput
}{}
\ProvidesFile{thcb.sty}
             [\FMithmInfo]
%    \end{macrocode}
%
% \begin{macro}{\th@changebreak}
%    The \texttt{change-break} theorem style is like \texttt{break}
%    but with interchange of theorem name and theorem number.  Thus we
%    define |\th@changebreak| as follows:
%    \begin{macrocode}
\gdef\th@changebreak{\normalfont\slshape
  \def\@begintheorem##1##2{\item
       [\rlap{\vbox{\hbox{\hskip\labelsep \theorem@headerfont ##2\ ##1}%
                    \hbox{\strut}}}]}%
\def\@opargbegintheorem##1##2##3{%
  \item[\rlap{\vbox{\hbox{\hskip\labelsep \theorem@headerfont
                     ##2\ ##1\ (##3)}%
                    \hbox{\strut}}}]}}
\endgroup
%</thcb>
%    \end{macrocode}
% \end{macro}
%
%
% \subsubsection{The \texttt{change} style}
%
% This style option is stored in the file |thc.sty|.
%    \begin{macrocode}
%<*thc>
\begingroup \makeatletter
\@ifundefined{theorem@style}{% \iffalse meta-comment
%
% Copyright (C) 1993-2024
%
% The LaTeX Project and any individual authors listed elsewhere
% in this file.
%
% This file is part of the Standard LaTeX `Tools Bundle'.
% -------------------------------------------------------
%
% It may be distributed and/or modified under the
% conditions of the LaTeX Project Public License, either version 1.3c
% of this license or (at your option) any later version.
% The latest version of this license is in
%    https://www.latex-project.org/lppl.txt
% and version 1.3c or later is part of all distributions of LaTeX
% version 2005/12/01 or later.
%
% The list of all files belonging to the LaTeX `Tools Bundle' is
% given in the file `manifest.txt'.
%
% \fi
%
%%
%
%
\def\FMithmInfo{2023/07/05 v2.2c Theorem extension package (FMi)}
%
% \ProvidesFile{theorem.dtx}[\FMithmInfo]
%\iffalse   % this is a METACOMMENT !
%%
%% Package `theorem' to use with LaTeX2e
%% Copyright (C) 1989-1995 Frank Mittelbach, all rights reserved.
%<+package>\@ifundefined{theorem@style}{}{\endinput}
%<+package>\NeedsTeXFormat{LaTeX2e}[1995/06/01]
%<+package>\ProvidesPackage{theorem}
%<+package>         [\FMithmInfo]
%
% \fi
%
% \changes{v1.0}{87/05/07}{First draft.}
% \changes{v1.0a}{87/05/07}{`thistheoremstyle' added.}
% \changes{v1.1a}{87/07/29}{New concept: redefine the internal
%                          \protect\LaTeX{} commands `@xnthm' etc.}
% \changes{v1.1b}{87/07/31}{`theoremfont' used to specify font.}
% \changes{v1.2a}{88/01/08}{`@endtheorem' now globally defined.}
% \changes{v1.2b}{88/01/11}{`newtheorem' only allowed in the preamble.}
% \changes{v1.3a}{88/02/14}{`theoremfont' now a toks register.}
%
% \changes{v2.0a}{88/11/16}{New implementation and new documentation.}
% \changes{v2.0a}{88/11/16}{No layout changes but internal modifications
%                           old substyles won't work.}
% \changes{v2.0b}{89/01/30}{`global' used for all definitions.}
% \changes{v2.0c}{89/06/15}{Documentation translated to English.
%                            Thanks to LLL and Nina}
% \changes{v2.1a}{90/02/15}{Support for new font selection.}
% \changes{v2.2a}{94/02/03}{Cleanup for LaTeX2e}
%
% \DoNotIndex{\@endparenv,\@ifdefinable,\@ifnextchar,\@ifundefined}
% \DoNotIndex{\@namedef,\@nameuse,\@spaces,\@tempa,\@warning}
% \DoNotIndex{\ }
% \DoNotIndex{\begingroup,\bfseries}
% \DoNotIndex{\csname}
% \DoNotIndex{\def,\do,\docdate}
% \DoNotIndex{\endcsname,\endgroup,\endinput,\endtrivlist,\expandafter}
% \DoNotIndex{\filedate,\fileversion}
% \DoNotIndex{\gdef,\global}
% \DoNotIndex{\hbox,\hskip}
% \DoNotIndex{\ignorespaces,\input,\itshape,\item}
% \DoNotIndex{\labelsep,\let,\llap}
% \DoNotIndex{\makeatletter}
% \DoNotIndex{\noexpand}
% \DoNotIndex{\refstepcounter,\relax,\rlap}
% \DoNotIndex{\setlength,\slshape,\space,\string,\strut}
% \DoNotIndex{\the,\trivlist,\typeout}
% \DoNotIndex{\vbox}
% \DoNotIndex{\xdef}
%
% \def\deflabel#1{\ttfamily #1\hfill}
% \def\deflist#1{\begin{list}{}{\settowidth\labelwidth{\ttfamily #1}%
%                               \setlength\leftmargin\labelwidth
%                               \addtolength\leftmargin\labelsep
%                               \let\makelabel\deflabel}}
% \def\enddeflist{\end{list}}
% \def\envfont{\normalfont\ttfamily}
%
% \MakeShortVerb{\|}
%
% \newcommand{\TUB}{{\sl TUGboat\/}}
%
% \GetFileInfo{theorem.dtx}
% \title{An Extension of the
%        \LaTeX{} theorem environment\thanks{This file
%        has version number \fileversion, last
%        revised \filedate.}}
% \author{Frank Mittelbach}
%
% \MaintainedByLaTeXTeam{tools}
% \maketitle
%
% \begin{abstract}
%    The macros described in this paper yield an extension of the
%    \LaTeX{} theorem mechanism. It is designed to satisfy the
%    different requirements of various journals. Thus, the
%    layout of the ``theorems'' can be manipulated by determining a
%    ``style''. This article describes not only the use, but
%    also the definition, of the necessary macros.
% \end{abstract}
%
% \section*{Preface to version 2.2}
%
% For \LaTeXe{} this package did not need any fundamental changes. I
% only modified the messages generated so that theorem layout styles
% will show up with the |\listfiles| command and cleaned the section
% on the New Font Selection Scheme since this is now included in
% \LaTeX.
%
% \section*{Preface to version 2.1}
%
% This version is identical to 2.0g described in \TUB\ 10\#3 except
% for some internal defaults which are now set depending on the used
% font selection scheme.
%
% This was done to avoid unpleasant surprises if the new font selection
% scheme is in force. For further details see
% section~\ref{sec:newfonts} and \cite{bk:GMS94}.
%
% \section{Introduction}
%
% For our purposes here, ``theorems'' are labelled enunciations,
% often set off from the main text by extra space and a font change.
% Theorems, corollaries, conjectures, definitions, and remarks
% are all instances of ``theorems''.  The ``header'' of these structures
% is composed of a label (such as \textsc{Theorem} or \textsc{Remark})
% and a number which serializes an item in the sequence of items
% with the same label.
%
% Shortly after the introduction of \LaTeX{} at the Fachbereich
% Mathematik in Mainz, the desire to manipulate the layout of
% ``theorems'' arose.  In Mainz, the following two conventions came
% into general use:
% \begin{enumerate}
%   \item The number of the theorem is shown in the margin.
%   \item There is a line break at
%         the end of the theorem header.
% \end{enumerate}
%
% Additionally, some journals require different formats which depend on
% the ``sort of theorem'': e.g.\ often remarks and definitions are set
% in |\upshape|, while |\itshape| is employed for main theorems.
%
% Confronted with these requirements, a theorem environment was
% developed in Mainz which allows separate determination of the
% layout of the ``theorems sets'', comparable to |\pagestyle|.
%
%
% \section{The user interface}
% \subsection{Defining new theorem sets}
%
% \DescribeMacro\newtheorem
% As in the original \LaTeX{} version, the command |\newtheorem|
% defines a new ``theorem set'' or ``theorem-like structure''.
% Two required arguments name the new environment and give the text
% to be typeset with each instance of the new ``set'', while
% an optional argument determines how
% the ``set'' is enumerated:
% \begin{description}
%    \item[\ttfamily \bslash newtheorem\{foo\}\{bar\}]
%       The theorem set {\envfont foo} (whose name is \texttt{bar})
%       uses its own counter.
%    \item[\ttfamily \bslash newtheorem\{foo2\}{[foo]}\{bar2\}]
%       The theorem set {\envfont foo2} (printed name \texttt{bar2})
%       uses the same counter as the theorem set \texttt{foo}.
%    \item[\ttfamily \bslash newtheorem\{foo3\}\{bar3\}{[section]}]
%       The theorem set {\envfont foo3} (printed name \texttt{bar3}) is
%       enumerated within the counter \texttt{section}, i.e.\ with every
%       new |\section| the enumeration begins again with 1, and
%       the enumeration is composed from the section-number and the
%       theorem counter itself.
% \end{description}
%
% \DescribeMacro\theoremstyle
% Additionally, the command |\theoremstyle| can define the layout
% of various, or all, theorem sets.  It should be noted that any
% theorem set defined by |\newtheorem| is typeset in the
% |\theoremstyle| that is current at the time of the definition.
% Thus, the following
% \begin{verbatim}
%    \theoremstyle{break}        \newtheorem{Cor}{Corollary}
%    \theoremstyle{plain}        \newtheorem{Exa}{Example}[section]
%\end{verbatim}                             ^^A No blank at line start
% leads to the result that the set {\envfont Cor} is formatted in the
% style \texttt{break}, while the set {\envfont Exa} and all the
% following ones are formatted in the style \texttt{plain}, unless
% another |\theoremstyle| follows.  Since the definitions installed by
% |\newtheorem| are global, one also can limit |\theoremstyle| locally
% by grouping braces.
%
% \DescribeMacro\theorembodyfont
% The choice of the font for the theorem body is completely
% independent of the chosen |\theoremstyle|; this has proven to
% be very advantageous.  For example,
% \begin{verbatim}
%   {\theorembodyfont{\upshape}          \newtheorem{Rem}{Remark}}
%\end{verbatim}
% defines a theorem set {\envfont Rem}, which will be set in
% |\upshape| in the current layout (which in our example is
% \texttt{plain}).  As with |\theoremstyle|, the |\theorembodyfont|
% chosen is that current at the time of |\newtheorem|.  If
% |\theorembodyfont| is not specified or one defines
% |\theorembodyfont{}|, then the font used will be that defined by the
% |\theoremstyle|.
%
% \DescribeMacro\theoremheaderfont It is also possible to customize
% the font used for the theorem headers.  This is, however, a global
% declaration, and therefore there should be at most one
% |\theoremheaderfont| declaration in the preamble.\footnote{If it is
% actually necessary to have different header fonts, one has to define
% new theorem styles (substituting the desired font) or specify the
% information directly in the \texttt{\bslash newtheorem} declaration
% (the unclean variant).}
%
% \DescribeMacro\theorempreskipamount
% \DescribeMacro\theorempostskipamount Two additional parameters
% affect the vertical space around the theorem environments:
% |\theorempreskipamount| and |\theorempostskipamount| define,
% respectively, the spacing before and after such an environment.
% These parameters apply for all theorem sets and can be manipulated
% with the ordinary length macros.  They are rubber lengths
% (`\textsf{skips}'), and therefore can contain \texttt{plus} and
% \texttt{minus} parts.
%
% Since the definition of theorem sets should---most sensibly---be
% placed in the preamble, we only allow installation there.
% It is therefore possible to release the memory used here after
% |\begin{document}|, in order to make room for other
% applications.
%
%
% \subsection{Existing theorem styles}
%
% The following theorem styles exist to date:
% \begin{deflist}{marginbreak}
%    \item[plain]
%       This theorem style emulates the original \LaTeX{} definition,
%       except that additionally the parameters
%       |\theorem...skipamount| are used.
%    \item[break]
%       In this style, the theorem header is followed by a line
%       break.
%    \item[marginbreak]
%       The theorem number is set in the margin, and there is a line
%       break as in \texttt{break}.
%    \item[changebreak]
%       Like \texttt{break}, but with header number and text
%       interchanged.
%    \item[change]
%       Header number and text are interchanged, without a line break.
%    \item[margin]
%       The number is set in the left margin, without a line break.
% \end{deflist}
% All styles (except \texttt{plain}) select |\slshape| as the default
% |\theorembodyfont|.
%
%
% \subsection{Examples}
%
% Given the above theorem sets {\envfont Cor}, {\envfont Exa}
% and {\envfont Rem},
% suppose that the preamble also contains the declarations:
% \begin{verbatim}
%    \theoremstyle{marginbreak}   \newtheorem{Lem}[Cor]{Lemma}
%    \theoremstyle{change}
%    \theorembodyfont{\itshape}        \newtheorem{Def}[Cor]{Definition}
%
%    \theoremheaderfont{\scshape}
%\end{verbatim}
% Then the following are some typical examples of the typeset output
% resulting from their use.
%
% \begin{Cor}
%    This is a sentence typeset in the theorem environment {\envfont
%    Cor}.
% \end{Cor}
%
% \begin{Exa}
%    This is a sentence typeset in the theorem environment {\envfont
%    Exa}.
% \end{Exa}
%
% \begin{Rem}
%    This is a sentence typeset in the theorem environment {\envfont
%    Rem}.
% \end{Rem}
%
% \begin{Lem}[Ben User]
%    This is a sentence typeset in the theorem environment {\envfont
%    Lem}.
% \end{Lem}
%
% \begin{Def}[Very impressive Definition]
%    This is a sentence typeset in the theorem environment {\envfont
%    Def}.
% \end{Def}
%
% The last two examples show the effect of the optional argument to
% a theorem environment (it is the text typeset in parentheses).
%
%
% \section{Special Considerations}
% \label{sec:newfonts}
%
% Theoremheader and body are implemented as a unit. This means that
% the |\theoremheaderfont|  will inherit characteristics of the
% |\theorembodyfont| in \LaTeXe. Thus, if for example
% |\theorembodyfont| is |\itshape| and |\theoremheaderfont|
% is |\bfseries| the font selected for the header will have the
% characteristics `bold extended italic'. If this is not desired one
% should set the |\theoremheaderfont| to something like
% \begin{verbatim}
%      \theoremheaderfont{\normalfont\bfseries}
%\end{verbatim}
% i.e.\ supplying all necessary font information explicitly.
%
%
%
% \section{Acknowledgements}
%
% The publication of this set of macros was only possible with the
% help of Christina Busse (translating the manuscript into English),
% Joachim Pense (playing the r\^ole of typist), Chris Rowley
% (looking everything over) and many others providing useful
% suggestions.
%
% \MaybeStop{%
%    \begin{thebibliography}{9}
%       \bibitem{bk:GMS94} \textsc{M.~Goossens}, \textsc{F.~Mittelbach}
%         and \textsc{A.~Samarin}.
%         \newblock The \LaTeX{} Companion.
%         \newblock
%         Addison-Wesley, Reading, Massachusetts, 1994.
%       \bibitem{latex:source} \textsc{Lamport, Leslie.} \newblock
%                               \texttt{latex.tex}, version 2.09, date
%                               Feb. 1990.
%    \end{thebibliography}
%    \PrintIndex
%    \PrintChanges}
%
%
% \section{The documentation driver file}
%
% The next bit of code contains the documentation driver file for
% \TeX{}, i.e., the file that will produce the documentation you are
% currently reading. It will be extracted from this file by the
% \texttt{docstrip} program. Since it is the first code in the file
% one can alternatively process this file directly with \LaTeXe{} to
% obtain the documentation.
%
%    \begin{macrocode}
%<*driver>
\documentclass{ltxdoc}

\usepackage{theorem}

%<+driver>% The next few lines define theorem sets which are used
%<+driver>% in the example section of the documentation.

\theoremstyle{break}       \newtheorem{Cor}{Corollary}
\theoremstyle{plain}       \newtheorem{Exa}{Example}[section]
{\theorembodyfont{\upshape}\newtheorem{Rem}{Remark}}
\theoremstyle{marginbreak} \newtheorem{Lem}[Cor]{Lemma}
\theoremstyle{change}
\theorembodyfont{\itshape} \newtheorem{Def}[Cor]{Definition}

\theoremheaderfont{\scshape}

\RecordChanges

\begin{document}
   \DocInput{theorem.dtx}
\end{document}
%</driver>
%    \end{macrocode}
%
% \section{Definition of the Macros}
%
% If the file has been loaded before, we abort immediately. If not the
% package announces itself (this is actually done at the very top of
% the file---the way it is done isn't good style so don't copy it).
% \changes{v2.0e}{89/07/19}{Spaces removed from `typeout'.}
% \changes{v2.2b}{95/11/19}{Announce moved to top of file}
%    \begin{macrocode}
%<*package>
%\@ifundefined{theorem@style}{}{\endinput}
%\def\FMithmInfo{1995/11/19 v2.2b Theorem extension package (FMi)}
%\ProvidesPackage{theorem}[\FMithmInfo]
%    \end{macrocode}
%
%
% \subsection{Definition of theorem styles and fonts}
%
% All the definitions in this file are done globally to allow
% inputting this file inside a group.
%
% \begin{macro}{\theoremstyle}
%    Before a theorem style can be installed, the chosen style must be
%    known. For that reason, we must test to see that
%    |\th@|\meta{style} is known or, more precisely, that it is
%    different from |\relax|. If the style is not known then
%    |\th@plain| is used.
%    \begin{macrocode}
\gdef\theoremstyle#1{%
   \@ifundefined{th@#1}{\@warning
          {Unknown theoremstyle `#1'. Using `plain'}%
          \theorem@style{plain}}%
%    \end{macrocode}
%    We save the theorem style to be used in the token register
%    |\theorem@style|.
%    \begin{macrocode}
      {\theorem@style{#1}}%
%    \end{macrocode}
%    Now we ``evaluate'' the theorem style: this means, we call the
%    macro |\th@|\meta{style} which will activate the relevant
%    definitions which are contained in a separate file.  This is done
%    in a group to suppress changes to the current font.  This could
%    otherwise pose problems together with the new font selection
%    scheme\footnote{When I printed the original article using the new
%    font selection scheme I ended with a document with slanted
%    typefaces (text headings and all) simply because one of the
%    theorem styles used \texttt{\bslash sl} at toplevel.} if the
%    |\th@|\meta{style} is evaluated a second time.
% \changes{v2.1a}{90/02/15}{Group added around `th@..}
% \changes{v2.1a}{90/02/15}{Reseting of `..@begintheorem removed.}
%    \begin{macrocode}
      \begingroup
        \csname th@\the\theorem@style \endcsname
      \endgroup}
%    \end{macrocode}
% \end{macro}
%
%
% \begin{macro}{\@begintheorem}
% \changes{v2.1a}{90/02/15}{Relaxed at toplevel.}
% \begin{macro}{\@opargbegintheorem}
% \changes{v2.1a}{90/02/15}{Relaxed at toplevel.}
%    We reset |\@begintheorem| and |\@opargbegintheorem| to
%    |\relax| since these commands are no longer necessary at
%    toplevel. This will save a few tokens.
%    \begin{macrocode}
\global\let\@begintheorem\relax
\global\let\@opargbegintheorem\relax
%    \end{macrocode}
% \end{macro}
% \end{macro}
%
% \begin{macro}{\theorem@style}
%    Obviously the \textsf{token} register used above has to be
%    allocated. To assure the utmost compatibility with the
%    original \LaTeX{} definition, we set the default theorem style to
%    \texttt{plain}, which implements the usual \LaTeX{} convention.
% \changes{v2.0f}{89/07/20}{Unnecessary `global' removed.}
%    \begin{macrocode}
\newtoks\theorem@style
\global\theorem@style{plain}
%    \end{macrocode}
% \end{macro}
%
% \begin{macro}{\theorembodyfont}
% \begin{macro}{\theorem@bodyfont}
%    If the |\theorembodyfont| is set by the user then it should not
%    interact with the default font set in the theorem style. When the
%    new font selection is in force this may happen if, for example,
%    the default is |\itshape| and the new |\theorembodyfont| is
%    |\sffamily|. So we add a |\reset@font| command in front of the
%    user definition.
% \changes{v2.0f}{89/07/20}{Unnecessary `global' removed.}
% \changes{v2.0g}{89/09/19}{`theoremfont to `theorembodyfont.}
% \changes{v2.1c}{92/06/22}{`reset@font added and logic changed.}
% \changes{v2.1d}{92/10/28}{Reset to empty if arg is empty}
%    \begin{macrocode}
\gdef\theorembodyfont#1{%
%    \end{macrocode}
%    We check if the argument supplied is empty and if so put nothing
%    into the |\theorem@bodyfont| token register to allow for
%    |\theorembodyfont{}| as a mean of using the default of the
%    current |\theoremstyle|.
%    \begin{macrocode}
   \def\@tempa{#1}%
   \ifx\@tempa\@empty
    \theorem@bodyfont{}%
   \else
    \theorem@bodyfont{\reset@font#1}%
   \fi
   }
\newtoks\theorem@bodyfont
\global\theorem@bodyfont{}
%    \end{macrocode}
% \end{macro}
% \end{macro}
%
%
% \begin{macro}{\theoremheaderfont}
% \changes{v2.0g}{89/09/19}{Macro added.}
%    The font for the theorem headers is handled differently because
%    this definition applies to all theorem styles.
%    \begin{macrocode}
\gdef\theoremheaderfont#1{\gdef\theorem@headerfont{#1}%
%    \end{macrocode}
%    After using the macro once it is redefined to produce an error
%    message.
%    \begin{macrocode}
       \gdef\theoremheaderfont##1{%
        \typeout{\string\theoremheaderfont\space should be used
                 only once.}}}
%    \end{macrocode}
% \end{macro}
%
% \begin{macro}{\theorem@headerfont}
% \changes{v2.0g}{89/09/19}{Macro added.}
%    To set the |\theorem@headerfont| default we first test if
%    the new fontselection scheme is in force.
% \changes{v2.1a}{90/02/15}{Support for new font selection.}
%    \begin{macrocode}
\ifx\upshape\undefined
%    \end{macrocode}
%    If not we define it to expand into |\bfseries|. We don't use
%    |\let| just in case a following style option redefines this
%    macro.
%    \begin{macrocode}
\gdef\theorem@headerfont{\bfseries}
%    \end{macrocode}
%    Otherwise we reset the current shape before calling |\bfseries|.
%    \begin{macrocode}
\else \gdef\theorem@headerfont{\normalfont\bfseries}\fi
%    \end{macrocode}
% \end{macro}
%
% \begin{macro}{\th@plain}
% \begin{macro}{\th@break}
% \begin{macro}{\th@marginbreak}
% \begin{macro}{\th@changebreak}
% \begin{macro}{\th@change}
% \begin{macro}{\th@margin}
%    The different styles are defined in macros such as
%    |\th@plain|.  Since memory space is precious in
%    ``non-Big-versions'', we have to avoid offering too many unused
%    definitions.  Therefore we define these styles in separate files
%    that can be loaded on demand.  Thus the commands themselves only
%    load these files. We use |\@input@|, a \LaTeXe{} internal command
%    that ensures that the file will be listed with |\listfiles|.
% \changes{v2.2a}{94/02/02}{Use `@input@ to load theorem layout files}
%    \begin{macrocode}
\gdef\th@plain{\@input@{thp.sty}}
\gdef\th@break{\@input@{thb.sty}}
\gdef\th@marginbreak{\@input@{thmb.sty}}
\gdef\th@changebreak{\@input@{thcb.sty}}
\gdef\th@change{\@input@{thc.sty}}
\gdef\th@margin{\@input@{thm.sty}}
%    \end{macrocode}
%    This list will be expanded when new styles become
%    available.  For testing,  just append new theorem substyles
%    as document options.
% \end{macro}
% \end{macro}
% \end{macro}
% \end{macro}
% \end{macro}
% \end{macro}
%
%
%
% \subsection{Definition of a new theorem set}
%
% As already pointed out, a new theorem environment can be defined in
% three different ways:
% \begin{verbatim}
%    \newtheorem{Lem}{Lemma}
%    \newtheorem{Lem}{Lemma}[section]
%    \newtheorem{Lem}[Theorem]{Lemma}
%\end{verbatim}
% The function of the macro |\newtheorem| is to recognize these cases
% and then to branch into one of the three macros |\@ynthm|, |\@xnthm|
% or |\@othm|.  This mechanism is adopted unchanged from
% \cite{latex:source}; the essential point here is that, for example,
% in the second case, the arguments \texttt{Lem}, \texttt{Lemma} and
% \texttt{section} are passed over to the macro |\@xnthm|.
%
% We inspect this case first because the others present fewer problems,
% and thus are easily derived from this one.
%
% \begin{macro}{\@xnthm}
%    For our example arguments, the macro |\@xnthm| must fulfill
%    the following:
%    \begin{itemize}
%       \item
%          Define a new \LaTeX-counter `Lem'
%       \item
%          reset this counter within a |\section|
%       \item
%          define the macro |\theLem|
%       \item
%          define the environment macros |\Lem| and
%          |\endLem| using the current
%          |\theoremstyle| and |\theorem@bodyfont|.
%    \end{itemize}
%    Obviously, all this should happen only if the first argument
%    of |\@xnthm| (i.e.\ \texttt{Lem} in our example) is chosen so
%    as not to conflict with any previously defined commands or
%    environments.  This test is performed by the \LaTeX{} macro
%    |\@ifdefinable|.
%    \begin{macrocode}
\gdef\@xnthm#1#2[#3]{\expandafter\@ifdefinable\csname #1\endcsname
%    \end{macrocode}
%    Therefore, the first argument of |\@ifdefinable| is the
%    expansion (in the example, |\Lem|) of
%    |\csname||#1\endcsname|.  The second argument is
%    executed only if the test has been completed successfully.
%    \begin{macrocode}
   {%
%    \end{macrocode}
%    Now we define the new counter. The names of the \LaTeX{}
%    macros employed should speak for themselves:
% \changes{v2.2b}{1995/10/07}{Use \cs{@newctr}}
% \changes{v2.2c}{1995/11/23}{Use \cs{@newctr} properly.}
%    \begin{macrocode}
    \@definecounter{#1}\@newctr{#1}[#3]%
%    \end{macrocode}
%    Using |\@newctr| will give a proper error message if the counter
%    in |#3| is not defined.
%    In defining `|\theLem|' we must generate the desired macro
%    name by use of |\expandafter| and |\csname|.
%    \begin{macrocode}
    \expandafter\xdef\csname the#1\endcsname
%    \end{macrocode}
%    An |\xdef| is used in order to make the definition global,
%    and to ensure that it contains the replacement texts of
%    |\@thmcountersep| and |\@thmcounter|.\footnote{These
%    two macros can be defined by the document style.  Their default
%    values produce a `\texttt{.}' as separation and an arabic
%    representation of the number.} However, not everything should be
%    expanded. For example, it saves space to use |\thesection|
%    instead of its---at times---lengthy expansion.
%    \begin{macrocode}
      {\expandafter \noexpand \csname the#3\endcsname
       \@thmcountersep \@thmcounter{#1}}%
%    \end{macrocode}
%    Thus with the defaults of \LaTeX{}, |\theLem| would be
%    replaced by the command sequence |\thesection.\arabic{Lem}|.
%
%    We will now look at the definition of the macro which is executed
%    at the beginning of the actual environment (in our example this
%    macro is |\Lem|). It should be noted that we use an
%    ``|\expandafter| trick'' to expand only certain parts of the
%    replacement text at the time of the definition.
%    \begin{macrocode}
    \def\@tempa{\global\@namedef{#1}}%
    \expandafter \@tempa \expandafter{%
%    \end{macrocode}
%    First, the macro that contains the current definitions of
%    |\@begintheorem| and |\@opargtheorem| should be called
%    up.  The name of this macro---as is already known---has the form
%    |\th@|\meta{theorem\ style}; therefore, it must be called by
%    \begin{macrocode}
      \csname th@\the \theorem@style
%    \end{macrocode}
%    In addition the default theorem font should be changeable, i.e.\
%    we have to insert the contents of |\theorem@bodyfont|. For
%    that reason, we expand even further, beyond |\endcsname|,
%    and thus insert the contents of the \textsf{token} register
%    |\theorem@bodyfont| in the replacement text.
%    \begin{macrocode}
            \expandafter \endcsname \the \theorem@bodyfont
%    \end{macrocode}
%    Now it is time to call the macro |\@thm| which takes over
%    the further processing.  It has two arguments: the current
%    counter name (in our example, \texttt{Lem}), and the text of the
%    label (in our example, \texttt{Lemma}).
%    \begin{macrocode}
     \@thm{#1}{#2}}%
%    \end{macrocode}
%    With this, the `sub-definition' is complete.  The macro
%    |\@endtheorem| ends a theorem environment and is, so far, nothing
%    but an |\endtrivlist|.  (Hence it is defined globally, and not
%    within the theorem styles.\footnote{This has to be changed as
%    soon as theorem styles that change \texttt{\bslash @endtheorem}
%    exist.  In such a case, all existing styles must be changed as
%    well since they will have to reset the macro.}) Therefore, we can
%    set it equivalent to the macro that ends the theorem set (in our
%    example, |\endLem|).  However, if some day theorem styles exist
%    that do change |\@endtheorem|, we would have to use the
%    commented-out line instead.
%    \begin{macrocode}
    \global \expandafter \let \csname end#1\endcsname \@endtheorem
%   \global\@namedef{end#1}{\@endtheorem}%
%    \end{macrocode}
%    With these commands all the required definitions are employed,
%    unless the test |\@ifdefinable| has failed.  Therefore, we
%    end the second argument of this macro and with it the definition
%    of |\@xnthm|.
%    \begin{macrocode}
   }}
%    \end{macrocode}
% \end{macro}
%
% \begin{macro}{\@ynthm}
%    The definition of |\@ynthm| is completely analogous. In
%    this case the new counter that is defined is not reset within
%    another counter; thus the definition of |\the...|
%    is simplified:
%    \begin{macrocode}
\gdef\@ynthm#1#2{\expandafter\@ifdefinable\csname #1\endcsname
   {\@definecounter{#1}%
    \expandafter\xdef\csname the#1\endcsname{\@thmcounter{#1}}%
%    \end{macrocode}
%    The rest of the definition corresponds literally to that of
%    |\@xnthm|:
%    \begin{macrocode}
    \def\@tempa{\global\@namedef{#1}}\expandafter \@tempa
     \expandafter{\csname th@\the \theorem@style \expandafter
     \endcsname \the\theorem@bodyfont \@thm{#1}{#2}}%
    \global \expandafter \let \csname end#1\endcsname \@endtheorem}}
%    \end{macrocode}
% \end{macro}
%
% \begin{macro}{\@othm}
%    The definition of |\@othm| does not contain anything new.
%    \begin{macrocode}
\gdef\@othm#1[#2]#3{%
%    \end{macrocode}
% \changes{v2.2b}{95/10/07}{Build in check for counter defined}
%    We do not define a new counter but instead use one that has
%    already been defined.  Thus the only definition we need is that
%    of this pseudo-counter (i.e.\ |\the|\meta{env.\ name}).
%    First we check if |#2| corresponds to a known counter name.
%    \begin{macrocode}
  \expandafter\ifx\csname c@#2\endcsname\relax
   \@nocounterr{#2}%
  \else
   \expandafter\@ifdefinable\csname #1\endcsname
   {\expandafter \xdef \csname the#1\endcsname
     {\expandafter \noexpand \csname the#2\endcsname}%
%    \end{macrocode}
%    All other parts of the definition can be adopted from
%    |\@xnthm|. We have to remember, though, that in this case
%    the name of the current counter and the theorem label have moved
%    to the second and third arguments.
%    \begin{macrocode}
    \def\@tempa{\global\@namedef{#1}}\expandafter \@tempa
     \expandafter{\csname th@\the \theorem@style \expandafter
     \endcsname \the\theorem@bodyfont \@thm{#2}{#3}}%
    \global \expandafter \let \csname end#1\endcsname \@endtheorem}%
  \fi}
%    \end{macrocode}
% \end{macro}
%
% \subsection{Macros that are employed in a {\envfont theorem}
%    environment}
%
% \begin{macro}{\@thm}
%    The macro |\@thm| has to increase the current counter.  Then,
%    depending on whether the environment has (or does not have) an
%    optional argument, it has to branch into either |\@begintheorem|
%    or |\@opargtheorem|.
%    \begin{macrocode}
\gdef\@thm#1#2{\refstepcounter{#1}%
%    \end{macrocode}
%    Now we start a {\envfont trivlist} environment, and give
%    |\@topsep| and |\@topsepadd| the values of the \textsf{skip}
%    registers |\theorempreskipamount| and |\theorempostskipamount|.
%    The value in |\@topsep| is the vertical space that is inserted by
%    the first (and only) |\item| in our |\trivlist| whilst
%    |\@topsepadd| is inserted by |\@endparenv| at the end of that
%    {\envfont trivlist} environment.  By using these registers, we
%    obtain the desired space around a {\envfont theorem} environment.
%    \begin{macrocode}
   \trivlist
   \@topsep \theorempreskipamount               % used by first \item
   \@topsepadd \theorempostskipamount           % used by \@endparenv
%    \end{macrocode}
%    Now we have to test whether an optional argument has been given.
% \changes{v2.0d}{89/07/18}{Forgotten percent added.}
%    \begin{macrocode}
   \@ifnextchar [%
%    \end{macrocode}
%    If there is an optional argument, we will call |\@ythm|, and
%    move the arguments read back into the input stream.
%    \begin{macrocode}
   {\@ythm{#1}{#2}}%
%    \end{macrocode}
%    If not, we call |\@begintheorem|. Its first argument is the
%    name of the theorem set (hence the second argument of
%    |\@thm|). Its second argument is the macro that produces the
%    current number.
%    \begin{macrocode}
   {\@begintheorem{#2}{\csname the#1\endcsname}\ignorespaces}}
%    \end{macrocode}
% \end{macro}
%
% \begin{macro}{\@xthm}
% \begin{macro}{\@ythm}
%    Both these macros were originally called by |\@thm|.  We do
%    not need |\@xthm| anymore, hence we reset it to
%    |\relax|. The definition of |\@ythm| has not changed at
%    all from its definition in \LaTeX. In order to make the macros
%    easier to understand, we will
%    nevertheless present it (commented out).
%    \begin{macrocode}
\global\let\@xthm\relax
% \def\@ythm#1#2[#3]{\@opargbegintheorem{#2}{\csname
%           the#1\endcsname}{#3}\ignorespaces}
%</package>
%    \end{macrocode}
%    The primitive |\ignorespaces| in |\@ythm| and
%    |\@thm| is needed to remove the spaces between the
%    |\begin{...}| and the actual text.
% \end{macro}
% \end{macro}
%
%
% \subsection{Definition of the theorem substyles}
%
% As already pointed out, the theorem substyles, defined below, are
% only loaded when necessary.  Note that all these substyles, except
% \texttt{plain}, have |\slshape| as the default body font.
%
% \subsubsection{The \texttt{plain} style}
%
% \label{dok:th@plain} As the following macros use |@|, we have
% to locally set the |\catcode| of this symbol to ``letter''.
% This happens within a group, so that we do not have to worry about
% which |\catcode| that symbol had before.
%    \begin{macrocode}
%<*thp>
\begingroup \makeatletter
%    \end{macrocode}
% Since we are now within a group, we must make all definitions
% globally.  First we make sure that \texttt{theorem.sty} is loaded.
% This will allow us to use this file as a document style option
% without having to call \texttt{theorem} itself as an option.  At the
% same time, we assure that at least version 2 is loaded, since
% |\theorem@style| was not defined in earlier versions.
%    \begin{macrocode}
\@ifundefined{theorem@style}{\input{theorem.sty}}{}
\ProvidesFile{thp.sty}
             [\FMithmInfo]
%    \end{macrocode}
%
%
% \begin{macro}{\th@plain}
% \begin{macro}{\@begintheorem}
% \begin{macro}{\@opargbegintheorem}
%    |\theoremstyle{plain}| corresponds to the original
%    definition, except that the distances to the surrounding text are
%    determined by the parameters |\theorempreskipamount| and
%    |\theorempostskipamount|.
%    First we set the default body font.
%    \begin{macrocode}
\gdef\th@plain{\normalfont\itshape
%    \end{macrocode}
%    Then we define |\@begintheorem| and
%    |\@opargbegintheorem|. These two macros define how the
%    header of a theorem is typeset.  |\@opargbegintheorem| will
%    be called if a \textsf{theorem} environment with an optional
%    argument is encountered; otherwise, the header is constructed by
%    calling |\@begintheorem|.  If one of these macros is
%    executed, we are within a \textsf{trivlist} environment started by
%    |\@thm|.  So the theorem header is produced with an
%    |\item| command.
%
%    Instead of specifying the header font directly, all standard
%    theorem styles use the |\theorem@headerfont| macro to allow
%    customization.  The extra space (|\labelsep|) is necessary
%    because of problems in the \textsf{trivlist} environment.
%    \begin{macrocode}
  \def\@begintheorem##1##2{%
        \item[\hskip\labelsep \theorem@headerfont ##1\ ##2]}%
%    \end{macrocode}
%    The definition of |\@opargbegintheorem| is completely
%    analogous. The only difference is the fact that there exists a
%    third argument (which is the optional parameter of the
%    environment and contains additional information about the
%    theorem).  Customarily we enclose it in parentheses.
%    \begin{macrocode}
\def\@opargbegintheorem##1##2##3{%
   \item[\hskip\labelsep \theorem@headerfont ##1\ ##2\ (##3)]}}
%    \end{macrocode}
% \end{macro}
% \end{macro}
% \end{macro}
%
% \noindent
% We conclude with an |\endgroup| to restore the
% |\catcode| of |@|.
%    \begin{macrocode}
\endgroup
%</thp>
%    \end{macrocode}
%
%
% \subsubsection{The \texttt{break} style}
%
% This style option is stored in the file |thb.sty|.
% For the next two lines see the documentation for |\th@plain| on
% page~\pageref{dok:th@plain}.
%    \begin{macrocode}
%<*thb>
\begingroup \makeatletter
\@ifundefined{theorem@style}{\input{theorem.sty}}{}
\ProvidesFile{thb.sty}
             [\FMithmInfo]
%    \end{macrocode}
%
% \begin{macro}{\th@break}
%    |\theoremstyle{break}| produces a line break after the name
%    of the theorem.  The font is |\slshape|.  Hence, we define
%    |\th@break| as follows:
%    \begin{macrocode}
\gdef\th@break{\normalfont\slshape
  \def\@begintheorem##1##2{\item[%
%    \end{macrocode}
%    We run into the following problem: it is not possible to create
%    the header with |\item[|\meta{title}|]| and then start
%    a new line by, for example, |\mbox{}\\|. Such a definition
%    will fail whenever a list environment follows immediately.  With
%    the above construction, the |\mbox{}| causes the switch
%    |@inlabel| (cf.\ definition of |\list| and
%    |\trivlist| in \cite{latex:source}) to be set to \texttt{false}
%    and so the following list will insert additional vertical
%    space (|\topskip|). This is quite annoying. Therefore, we
%    create the line break within the |\item|. In order to ensure
%    that the text will begin at the proper position in the following
%    line, we simply pretend that the label does not take any
%    room.\footnote{This will lead to problems whenever very high
%    symbols occurring in the line tower into the
%    heading. So, something else has to be done here sometime.}
%    \begin{macrocode}
       \rlap{\vbox{\hbox{\hskip \labelsep\theorem@headerfont ##1\ ##2}%
                   \hbox{\strut}}}]}%
%    \end{macrocode}
%    Again, the definition of |\@opargbegintheorem| is
%    completely analogous.
%    \begin{macrocode}
\def\@opargbegintheorem##1##2##3{%
  \item[\rlap{\vbox{\hbox{\hskip \labelsep \theorem@headerfont
                     ##1\ ##2\ (##3)}%
                    \hbox{\strut}}}]}}
\endgroup
%</thb>
%    \end{macrocode}
% \end{macro}
%
%
% \subsubsection{The \texttt{changebreak} style}
%
%    \begin{macrocode}
% This style option is stored in the file |thcb.sty|.
%    \begin{macrocode}
%<*thcb>
\begingroup \makeatletter
\@ifundefined{theorem@style}{\input{theorem.sty}}{}
\ProvidesFile{thcb.sty}
             [\FMithmInfo]
%    \end{macrocode}
%
% \begin{macro}{\th@changebreak}
%    The \texttt{change-break} theorem style is like \texttt{break}
%    but with interchange of theorem name and theorem number.  Thus we
%    define |\th@changebreak| as follows:
%    \begin{macrocode}
\gdef\th@changebreak{\normalfont\slshape
  \def\@begintheorem##1##2{\item
       [\rlap{\vbox{\hbox{\hskip\labelsep \theorem@headerfont ##2\ ##1}%
                    \hbox{\strut}}}]}%
\def\@opargbegintheorem##1##2##3{%
  \item[\rlap{\vbox{\hbox{\hskip\labelsep \theorem@headerfont
                     ##2\ ##1\ (##3)}%
                    \hbox{\strut}}}]}}
\endgroup
%</thcb>
%    \end{macrocode}
% \end{macro}
%
%
% \subsubsection{The \texttt{change} style}
%
% This style option is stored in the file |thc.sty|.
%    \begin{macrocode}
%<*thc>
\begingroup \makeatletter
\@ifundefined{theorem@style}{\input{theorem.sty}}{}
\ProvidesFile{thc.sty}
             [\FMithmInfo]
%    \end{macrocode}
%
% \begin{macro}{\th@change}
%    The \texttt{change} theorem style corresponds to the
%    \texttt{change break} style without a linebreak after the header.
%    To say it in another way, it's the same as the \texttt{plain}
%    style but with number and name interchanged and |\slshape| as the
%    default font.
%    \begin{macrocode}
\gdef\th@change{\normalfont\slshape
  \def\@begintheorem##1##2{\item
       [\hskip\labelsep \theorem@headerfont ##2\ ##1]}%
\def\@opargbegintheorem##1##2##3{%
  \item[\hskip\labelsep \theorem@headerfont ##2\ ##1\ (##3)]}}
\endgroup
%</thc>
%    \end{macrocode}
% \end{macro}
%
%
% \subsubsection{The \texttt{marginbreak} style}
%
% This style option is the one used most often at Mainz.  It is saved
% in the file |thmb.sty|.
%    \begin{macrocode}
%<*thmb>
\begingroup \makeatletter
\@ifundefined{theorem@style}{\input{theorem.sty}}{}
\ProvidesFile{thmb.sty}
             [\FMithmInfo]
%    \end{macrocode}
%
% \begin{macro}{\th@marginbreak}
%    The \texttt{margin break} style is nearly the same as the
%    \texttt{change break} style. The only difference is the placement
%    of the theorem number.  We use |\llap| to place it in the left
%    margin.
%
%    In this style |\labelsep| denotes the separation between the
%    number and the text.
%    \begin{macrocode}
\gdef\th@marginbreak{\normalfont\slshape
  \def\@begintheorem##1##2{\item
       [\rlap{\vbox{\theorem@headerfont
                    \hbox{\llap{##2}\hskip\labelsep ##1}%
                    \hbox{\strut}}}]}%
\def\@opargbegintheorem##1##2##3{%
  \item[\rlap{\vbox{\theorem@headerfont
                    \hbox{\llap{##2}\hskip\labelsep ##1\ (##3)}%
                    \hbox{\strut}}}]}}
\endgroup
%</thmb>
%    \end{macrocode}
% \changes{v2.1c}{90/02/17}{`endgroup outside macrocode env.}
% \end{macro}
%
%
%
% \subsubsection{The \texttt{margin} style}
%
% This style option is stored in the file |thm.sty|.
%    \begin{macrocode}
%<*thm>
\begingroup \makeatletter
\@ifundefined{theorem@style}{\input{theorem.sty}}{}
\ProvidesFile{thm.sty}
             [\FMithmInfo]
%    \end{macrocode}
%
% \begin{macro}{\th@margin}
%    Again this is only a variant of the theorem styles described
%    above without any new ideas.
%    \begin{macrocode}
\gdef\th@margin{\normalfont\slshape
  \def\@begintheorem##1##2{\item
       [\theorem@headerfont \llap{##2}\hskip\labelsep ##1]}%
\def\@opargbegintheorem##1##2##3{%
  \item[\theorem@headerfont \llap{##2}\hskip\labelsep ##1\ (##3)]}}
\endgroup
%</thm>
%    \end{macrocode}
% \end{macro}
%
%
%
%
% \subsection{Final Definitions}
%
% \begin{macro}{\theorempreskipamount}
% \begin{macro}{\theorempostskipamount}
%    The \textsf{skip} parameters that regulate the vertical empty space
%    before and after the {\envfont theorem} environment have
%    to be allocated as well.
% \changes{v2.0f}{89/07/20}{Unnecessary `global' removed.}
%    \begin{macrocode}
%<*package>
\newskip\theorempreskipamount
\newskip\theorempostskipamount
%    \end{macrocode}
%    Since we have used the same values for all theorem sets, we now can
%    assign them.
%    \begin{macrocode}
\global\setlength\theorempreskipamount{12pt plus 5pt minus 3pt}
\global\setlength\theorempostskipamount{8pt plus 3pt minus 1.5pt}
%    \end{macrocode}
% \end{macro}
% \end{macro}
%
%
% \begin{macro}{\@endtheorem}
%    The same holds for the macro |\@endtheorem|, which ends a
%    {\envfont theorem} environment. Since it is the same for all
%    theorem sets, it is removed from the macros |\th@|\meta{style}.
%    It simply ends the {\envfont trivlist} environment, which was
%    begun in |\@thm|.
%    \begin{macrocode}
\global\let\@endtheorem=\endtrivlist
%    \end{macrocode}
% \end{macro}
%
%
%
% \begin{macro}{\@preamblecmds}
%    All macros defined above are to be used only in the preamble.
%    Therefore, we insert them in |\@preamblecmds| which will disable
%    them at begin document. This is done by the internal \LaTeXe{}
%    command |\@onlypreamble|.
% \changes{v2.0g}{89/09/19}{`theoremheaderfont and `theorembodyfont
%    added.}
% \changes{v2.2a}{94/02/03}{Use \LaTeXe{} interface}
%    \begin{macrocode}
\@onlypreamble\@xnthm
\@onlypreamble\@ynthm
\@onlypreamble\@othm
\@onlypreamble\newtheorem
\@onlypreamble\theoremstyle
\@onlypreamble\theorembodyfont
\@onlypreamble\theoremheaderfont
%    \end{macrocode}
% \end{macro}
%
% Finally we declare the \texttt{plain} theorem style to be the default.
% \changes{v2.1b}{90/06/04}{Fixed missing `item by declaring default
%    style.}
%    \begin{macrocode}
\theoremstyle{plain}
%</package>
%    \end{macrocode}
%
% \Finale

\endinput
}{}
\ProvidesFile{thc.sty}
             [\FMithmInfo]
%    \end{macrocode}
%
% \begin{macro}{\th@change}
%    The \texttt{change} theorem style corresponds to the
%    \texttt{change break} style without a linebreak after the header.
%    To say it in another way, it's the same as the \texttt{plain}
%    style but with number and name interchanged and |\slshape| as the
%    default font.
%    \begin{macrocode}
\gdef\th@change{\normalfont\slshape
  \def\@begintheorem##1##2{\item
       [\hskip\labelsep \theorem@headerfont ##2\ ##1]}%
\def\@opargbegintheorem##1##2##3{%
  \item[\hskip\labelsep \theorem@headerfont ##2\ ##1\ (##3)]}}
\endgroup
%</thc>
%    \end{macrocode}
% \end{macro}
%
%
% \subsubsection{The \texttt{marginbreak} style}
%
% This style option is the one used most often at Mainz.  It is saved
% in the file |thmb.sty|.
%    \begin{macrocode}
%<*thmb>
\begingroup \makeatletter
\@ifundefined{theorem@style}{% \iffalse meta-comment
%
% Copyright (C) 1993-2024
%
% The LaTeX Project and any individual authors listed elsewhere
% in this file.
%
% This file is part of the Standard LaTeX `Tools Bundle'.
% -------------------------------------------------------
%
% It may be distributed and/or modified under the
% conditions of the LaTeX Project Public License, either version 1.3c
% of this license or (at your option) any later version.
% The latest version of this license is in
%    https://www.latex-project.org/lppl.txt
% and version 1.3c or later is part of all distributions of LaTeX
% version 2005/12/01 or later.
%
% The list of all files belonging to the LaTeX `Tools Bundle' is
% given in the file `manifest.txt'.
%
% \fi
%
%%
%
%
\def\FMithmInfo{2023/07/05 v2.2c Theorem extension package (FMi)}
%
% \ProvidesFile{theorem.dtx}[\FMithmInfo]
%\iffalse   % this is a METACOMMENT !
%%
%% Package `theorem' to use with LaTeX2e
%% Copyright (C) 1989-1995 Frank Mittelbach, all rights reserved.
%<+package>\@ifundefined{theorem@style}{}{\endinput}
%<+package>\NeedsTeXFormat{LaTeX2e}[1995/06/01]
%<+package>\ProvidesPackage{theorem}
%<+package>         [\FMithmInfo]
%
% \fi
%
% \changes{v1.0}{87/05/07}{First draft.}
% \changes{v1.0a}{87/05/07}{`thistheoremstyle' added.}
% \changes{v1.1a}{87/07/29}{New concept: redefine the internal
%                          \protect\LaTeX{} commands `@xnthm' etc.}
% \changes{v1.1b}{87/07/31}{`theoremfont' used to specify font.}
% \changes{v1.2a}{88/01/08}{`@endtheorem' now globally defined.}
% \changes{v1.2b}{88/01/11}{`newtheorem' only allowed in the preamble.}
% \changes{v1.3a}{88/02/14}{`theoremfont' now a toks register.}
%
% \changes{v2.0a}{88/11/16}{New implementation and new documentation.}
% \changes{v2.0a}{88/11/16}{No layout changes but internal modifications
%                           old substyles won't work.}
% \changes{v2.0b}{89/01/30}{`global' used for all definitions.}
% \changes{v2.0c}{89/06/15}{Documentation translated to English.
%                            Thanks to LLL and Nina}
% \changes{v2.1a}{90/02/15}{Support for new font selection.}
% \changes{v2.2a}{94/02/03}{Cleanup for LaTeX2e}
%
% \DoNotIndex{\@endparenv,\@ifdefinable,\@ifnextchar,\@ifundefined}
% \DoNotIndex{\@namedef,\@nameuse,\@spaces,\@tempa,\@warning}
% \DoNotIndex{\ }
% \DoNotIndex{\begingroup,\bfseries}
% \DoNotIndex{\csname}
% \DoNotIndex{\def,\do,\docdate}
% \DoNotIndex{\endcsname,\endgroup,\endinput,\endtrivlist,\expandafter}
% \DoNotIndex{\filedate,\fileversion}
% \DoNotIndex{\gdef,\global}
% \DoNotIndex{\hbox,\hskip}
% \DoNotIndex{\ignorespaces,\input,\itshape,\item}
% \DoNotIndex{\labelsep,\let,\llap}
% \DoNotIndex{\makeatletter}
% \DoNotIndex{\noexpand}
% \DoNotIndex{\refstepcounter,\relax,\rlap}
% \DoNotIndex{\setlength,\slshape,\space,\string,\strut}
% \DoNotIndex{\the,\trivlist,\typeout}
% \DoNotIndex{\vbox}
% \DoNotIndex{\xdef}
%
% \def\deflabel#1{\ttfamily #1\hfill}
% \def\deflist#1{\begin{list}{}{\settowidth\labelwidth{\ttfamily #1}%
%                               \setlength\leftmargin\labelwidth
%                               \addtolength\leftmargin\labelsep
%                               \let\makelabel\deflabel}}
% \def\enddeflist{\end{list}}
% \def\envfont{\normalfont\ttfamily}
%
% \MakeShortVerb{\|}
%
% \newcommand{\TUB}{{\sl TUGboat\/}}
%
% \GetFileInfo{theorem.dtx}
% \title{An Extension of the
%        \LaTeX{} theorem environment\thanks{This file
%        has version number \fileversion, last
%        revised \filedate.}}
% \author{Frank Mittelbach}
%
% \MaintainedByLaTeXTeam{tools}
% \maketitle
%
% \begin{abstract}
%    The macros described in this paper yield an extension of the
%    \LaTeX{} theorem mechanism. It is designed to satisfy the
%    different requirements of various journals. Thus, the
%    layout of the ``theorems'' can be manipulated by determining a
%    ``style''. This article describes not only the use, but
%    also the definition, of the necessary macros.
% \end{abstract}
%
% \section*{Preface to version 2.2}
%
% For \LaTeXe{} this package did not need any fundamental changes. I
% only modified the messages generated so that theorem layout styles
% will show up with the |\listfiles| command and cleaned the section
% on the New Font Selection Scheme since this is now included in
% \LaTeX.
%
% \section*{Preface to version 2.1}
%
% This version is identical to 2.0g described in \TUB\ 10\#3 except
% for some internal defaults which are now set depending on the used
% font selection scheme.
%
% This was done to avoid unpleasant surprises if the new font selection
% scheme is in force. For further details see
% section~\ref{sec:newfonts} and \cite{bk:GMS94}.
%
% \section{Introduction}
%
% For our purposes here, ``theorems'' are labelled enunciations,
% often set off from the main text by extra space and a font change.
% Theorems, corollaries, conjectures, definitions, and remarks
% are all instances of ``theorems''.  The ``header'' of these structures
% is composed of a label (such as \textsc{Theorem} or \textsc{Remark})
% and a number which serializes an item in the sequence of items
% with the same label.
%
% Shortly after the introduction of \LaTeX{} at the Fachbereich
% Mathematik in Mainz, the desire to manipulate the layout of
% ``theorems'' arose.  In Mainz, the following two conventions came
% into general use:
% \begin{enumerate}
%   \item The number of the theorem is shown in the margin.
%   \item There is a line break at
%         the end of the theorem header.
% \end{enumerate}
%
% Additionally, some journals require different formats which depend on
% the ``sort of theorem'': e.g.\ often remarks and definitions are set
% in |\upshape|, while |\itshape| is employed for main theorems.
%
% Confronted with these requirements, a theorem environment was
% developed in Mainz which allows separate determination of the
% layout of the ``theorems sets'', comparable to |\pagestyle|.
%
%
% \section{The user interface}
% \subsection{Defining new theorem sets}
%
% \DescribeMacro\newtheorem
% As in the original \LaTeX{} version, the command |\newtheorem|
% defines a new ``theorem set'' or ``theorem-like structure''.
% Two required arguments name the new environment and give the text
% to be typeset with each instance of the new ``set'', while
% an optional argument determines how
% the ``set'' is enumerated:
% \begin{description}
%    \item[\ttfamily \bslash newtheorem\{foo\}\{bar\}]
%       The theorem set {\envfont foo} (whose name is \texttt{bar})
%       uses its own counter.
%    \item[\ttfamily \bslash newtheorem\{foo2\}{[foo]}\{bar2\}]
%       The theorem set {\envfont foo2} (printed name \texttt{bar2})
%       uses the same counter as the theorem set \texttt{foo}.
%    \item[\ttfamily \bslash newtheorem\{foo3\}\{bar3\}{[section]}]
%       The theorem set {\envfont foo3} (printed name \texttt{bar3}) is
%       enumerated within the counter \texttt{section}, i.e.\ with every
%       new |\section| the enumeration begins again with 1, and
%       the enumeration is composed from the section-number and the
%       theorem counter itself.
% \end{description}
%
% \DescribeMacro\theoremstyle
% Additionally, the command |\theoremstyle| can define the layout
% of various, or all, theorem sets.  It should be noted that any
% theorem set defined by |\newtheorem| is typeset in the
% |\theoremstyle| that is current at the time of the definition.
% Thus, the following
% \begin{verbatim}
%    \theoremstyle{break}        \newtheorem{Cor}{Corollary}
%    \theoremstyle{plain}        \newtheorem{Exa}{Example}[section]
%\end{verbatim}                             ^^A No blank at line start
% leads to the result that the set {\envfont Cor} is formatted in the
% style \texttt{break}, while the set {\envfont Exa} and all the
% following ones are formatted in the style \texttt{plain}, unless
% another |\theoremstyle| follows.  Since the definitions installed by
% |\newtheorem| are global, one also can limit |\theoremstyle| locally
% by grouping braces.
%
% \DescribeMacro\theorembodyfont
% The choice of the font for the theorem body is completely
% independent of the chosen |\theoremstyle|; this has proven to
% be very advantageous.  For example,
% \begin{verbatim}
%   {\theorembodyfont{\upshape}          \newtheorem{Rem}{Remark}}
%\end{verbatim}
% defines a theorem set {\envfont Rem}, which will be set in
% |\upshape| in the current layout (which in our example is
% \texttt{plain}).  As with |\theoremstyle|, the |\theorembodyfont|
% chosen is that current at the time of |\newtheorem|.  If
% |\theorembodyfont| is not specified or one defines
% |\theorembodyfont{}|, then the font used will be that defined by the
% |\theoremstyle|.
%
% \DescribeMacro\theoremheaderfont It is also possible to customize
% the font used for the theorem headers.  This is, however, a global
% declaration, and therefore there should be at most one
% |\theoremheaderfont| declaration in the preamble.\footnote{If it is
% actually necessary to have different header fonts, one has to define
% new theorem styles (substituting the desired font) or specify the
% information directly in the \texttt{\bslash newtheorem} declaration
% (the unclean variant).}
%
% \DescribeMacro\theorempreskipamount
% \DescribeMacro\theorempostskipamount Two additional parameters
% affect the vertical space around the theorem environments:
% |\theorempreskipamount| and |\theorempostskipamount| define,
% respectively, the spacing before and after such an environment.
% These parameters apply for all theorem sets and can be manipulated
% with the ordinary length macros.  They are rubber lengths
% (`\textsf{skips}'), and therefore can contain \texttt{plus} and
% \texttt{minus} parts.
%
% Since the definition of theorem sets should---most sensibly---be
% placed in the preamble, we only allow installation there.
% It is therefore possible to release the memory used here after
% |\begin{document}|, in order to make room for other
% applications.
%
%
% \subsection{Existing theorem styles}
%
% The following theorem styles exist to date:
% \begin{deflist}{marginbreak}
%    \item[plain]
%       This theorem style emulates the original \LaTeX{} definition,
%       except that additionally the parameters
%       |\theorem...skipamount| are used.
%    \item[break]
%       In this style, the theorem header is followed by a line
%       break.
%    \item[marginbreak]
%       The theorem number is set in the margin, and there is a line
%       break as in \texttt{break}.
%    \item[changebreak]
%       Like \texttt{break}, but with header number and text
%       interchanged.
%    \item[change]
%       Header number and text are interchanged, without a line break.
%    \item[margin]
%       The number is set in the left margin, without a line break.
% \end{deflist}
% All styles (except \texttt{plain}) select |\slshape| as the default
% |\theorembodyfont|.
%
%
% \subsection{Examples}
%
% Given the above theorem sets {\envfont Cor}, {\envfont Exa}
% and {\envfont Rem},
% suppose that the preamble also contains the declarations:
% \begin{verbatim}
%    \theoremstyle{marginbreak}   \newtheorem{Lem}[Cor]{Lemma}
%    \theoremstyle{change}
%    \theorembodyfont{\itshape}        \newtheorem{Def}[Cor]{Definition}
%
%    \theoremheaderfont{\scshape}
%\end{verbatim}
% Then the following are some typical examples of the typeset output
% resulting from their use.
%
% \begin{Cor}
%    This is a sentence typeset in the theorem environment {\envfont
%    Cor}.
% \end{Cor}
%
% \begin{Exa}
%    This is a sentence typeset in the theorem environment {\envfont
%    Exa}.
% \end{Exa}
%
% \begin{Rem}
%    This is a sentence typeset in the theorem environment {\envfont
%    Rem}.
% \end{Rem}
%
% \begin{Lem}[Ben User]
%    This is a sentence typeset in the theorem environment {\envfont
%    Lem}.
% \end{Lem}
%
% \begin{Def}[Very impressive Definition]
%    This is a sentence typeset in the theorem environment {\envfont
%    Def}.
% \end{Def}
%
% The last two examples show the effect of the optional argument to
% a theorem environment (it is the text typeset in parentheses).
%
%
% \section{Special Considerations}
% \label{sec:newfonts}
%
% Theoremheader and body are implemented as a unit. This means that
% the |\theoremheaderfont|  will inherit characteristics of the
% |\theorembodyfont| in \LaTeXe. Thus, if for example
% |\theorembodyfont| is |\itshape| and |\theoremheaderfont|
% is |\bfseries| the font selected for the header will have the
% characteristics `bold extended italic'. If this is not desired one
% should set the |\theoremheaderfont| to something like
% \begin{verbatim}
%      \theoremheaderfont{\normalfont\bfseries}
%\end{verbatim}
% i.e.\ supplying all necessary font information explicitly.
%
%
%
% \section{Acknowledgements}
%
% The publication of this set of macros was only possible with the
% help of Christina Busse (translating the manuscript into English),
% Joachim Pense (playing the r\^ole of typist), Chris Rowley
% (looking everything over) and many others providing useful
% suggestions.
%
% \MaybeStop{%
%    \begin{thebibliography}{9}
%       \bibitem{bk:GMS94} \textsc{M.~Goossens}, \textsc{F.~Mittelbach}
%         and \textsc{A.~Samarin}.
%         \newblock The \LaTeX{} Companion.
%         \newblock
%         Addison-Wesley, Reading, Massachusetts, 1994.
%       \bibitem{latex:source} \textsc{Lamport, Leslie.} \newblock
%                               \texttt{latex.tex}, version 2.09, date
%                               Feb. 1990.
%    \end{thebibliography}
%    \PrintIndex
%    \PrintChanges}
%
%
% \section{The documentation driver file}
%
% The next bit of code contains the documentation driver file for
% \TeX{}, i.e., the file that will produce the documentation you are
% currently reading. It will be extracted from this file by the
% \texttt{docstrip} program. Since it is the first code in the file
% one can alternatively process this file directly with \LaTeXe{} to
% obtain the documentation.
%
%    \begin{macrocode}
%<*driver>
\documentclass{ltxdoc}

\usepackage{theorem}

%<+driver>% The next few lines define theorem sets which are used
%<+driver>% in the example section of the documentation.

\theoremstyle{break}       \newtheorem{Cor}{Corollary}
\theoremstyle{plain}       \newtheorem{Exa}{Example}[section]
{\theorembodyfont{\upshape}\newtheorem{Rem}{Remark}}
\theoremstyle{marginbreak} \newtheorem{Lem}[Cor]{Lemma}
\theoremstyle{change}
\theorembodyfont{\itshape} \newtheorem{Def}[Cor]{Definition}

\theoremheaderfont{\scshape}

\RecordChanges

\begin{document}
   \DocInput{theorem.dtx}
\end{document}
%</driver>
%    \end{macrocode}
%
% \section{Definition of the Macros}
%
% If the file has been loaded before, we abort immediately. If not the
% package announces itself (this is actually done at the very top of
% the file---the way it is done isn't good style so don't copy it).
% \changes{v2.0e}{89/07/19}{Spaces removed from `typeout'.}
% \changes{v2.2b}{95/11/19}{Announce moved to top of file}
%    \begin{macrocode}
%<*package>
%\@ifundefined{theorem@style}{}{\endinput}
%\def\FMithmInfo{1995/11/19 v2.2b Theorem extension package (FMi)}
%\ProvidesPackage{theorem}[\FMithmInfo]
%    \end{macrocode}
%
%
% \subsection{Definition of theorem styles and fonts}
%
% All the definitions in this file are done globally to allow
% inputting this file inside a group.
%
% \begin{macro}{\theoremstyle}
%    Before a theorem style can be installed, the chosen style must be
%    known. For that reason, we must test to see that
%    |\th@|\meta{style} is known or, more precisely, that it is
%    different from |\relax|. If the style is not known then
%    |\th@plain| is used.
%    \begin{macrocode}
\gdef\theoremstyle#1{%
   \@ifundefined{th@#1}{\@warning
          {Unknown theoremstyle `#1'. Using `plain'}%
          \theorem@style{plain}}%
%    \end{macrocode}
%    We save the theorem style to be used in the token register
%    |\theorem@style|.
%    \begin{macrocode}
      {\theorem@style{#1}}%
%    \end{macrocode}
%    Now we ``evaluate'' the theorem style: this means, we call the
%    macro |\th@|\meta{style} which will activate the relevant
%    definitions which are contained in a separate file.  This is done
%    in a group to suppress changes to the current font.  This could
%    otherwise pose problems together with the new font selection
%    scheme\footnote{When I printed the original article using the new
%    font selection scheme I ended with a document with slanted
%    typefaces (text headings and all) simply because one of the
%    theorem styles used \texttt{\bslash sl} at toplevel.} if the
%    |\th@|\meta{style} is evaluated a second time.
% \changes{v2.1a}{90/02/15}{Group added around `th@..}
% \changes{v2.1a}{90/02/15}{Reseting of `..@begintheorem removed.}
%    \begin{macrocode}
      \begingroup
        \csname th@\the\theorem@style \endcsname
      \endgroup}
%    \end{macrocode}
% \end{macro}
%
%
% \begin{macro}{\@begintheorem}
% \changes{v2.1a}{90/02/15}{Relaxed at toplevel.}
% \begin{macro}{\@opargbegintheorem}
% \changes{v2.1a}{90/02/15}{Relaxed at toplevel.}
%    We reset |\@begintheorem| and |\@opargbegintheorem| to
%    |\relax| since these commands are no longer necessary at
%    toplevel. This will save a few tokens.
%    \begin{macrocode}
\global\let\@begintheorem\relax
\global\let\@opargbegintheorem\relax
%    \end{macrocode}
% \end{macro}
% \end{macro}
%
% \begin{macro}{\theorem@style}
%    Obviously the \textsf{token} register used above has to be
%    allocated. To assure the utmost compatibility with the
%    original \LaTeX{} definition, we set the default theorem style to
%    \texttt{plain}, which implements the usual \LaTeX{} convention.
% \changes{v2.0f}{89/07/20}{Unnecessary `global' removed.}
%    \begin{macrocode}
\newtoks\theorem@style
\global\theorem@style{plain}
%    \end{macrocode}
% \end{macro}
%
% \begin{macro}{\theorembodyfont}
% \begin{macro}{\theorem@bodyfont}
%    If the |\theorembodyfont| is set by the user then it should not
%    interact with the default font set in the theorem style. When the
%    new font selection is in force this may happen if, for example,
%    the default is |\itshape| and the new |\theorembodyfont| is
%    |\sffamily|. So we add a |\reset@font| command in front of the
%    user definition.
% \changes{v2.0f}{89/07/20}{Unnecessary `global' removed.}
% \changes{v2.0g}{89/09/19}{`theoremfont to `theorembodyfont.}
% \changes{v2.1c}{92/06/22}{`reset@font added and logic changed.}
% \changes{v2.1d}{92/10/28}{Reset to empty if arg is empty}
%    \begin{macrocode}
\gdef\theorembodyfont#1{%
%    \end{macrocode}
%    We check if the argument supplied is empty and if so put nothing
%    into the |\theorem@bodyfont| token register to allow for
%    |\theorembodyfont{}| as a mean of using the default of the
%    current |\theoremstyle|.
%    \begin{macrocode}
   \def\@tempa{#1}%
   \ifx\@tempa\@empty
    \theorem@bodyfont{}%
   \else
    \theorem@bodyfont{\reset@font#1}%
   \fi
   }
\newtoks\theorem@bodyfont
\global\theorem@bodyfont{}
%    \end{macrocode}
% \end{macro}
% \end{macro}
%
%
% \begin{macro}{\theoremheaderfont}
% \changes{v2.0g}{89/09/19}{Macro added.}
%    The font for the theorem headers is handled differently because
%    this definition applies to all theorem styles.
%    \begin{macrocode}
\gdef\theoremheaderfont#1{\gdef\theorem@headerfont{#1}%
%    \end{macrocode}
%    After using the macro once it is redefined to produce an error
%    message.
%    \begin{macrocode}
       \gdef\theoremheaderfont##1{%
        \typeout{\string\theoremheaderfont\space should be used
                 only once.}}}
%    \end{macrocode}
% \end{macro}
%
% \begin{macro}{\theorem@headerfont}
% \changes{v2.0g}{89/09/19}{Macro added.}
%    To set the |\theorem@headerfont| default we first test if
%    the new fontselection scheme is in force.
% \changes{v2.1a}{90/02/15}{Support for new font selection.}
%    \begin{macrocode}
\ifx\upshape\undefined
%    \end{macrocode}
%    If not we define it to expand into |\bfseries|. We don't use
%    |\let| just in case a following style option redefines this
%    macro.
%    \begin{macrocode}
\gdef\theorem@headerfont{\bfseries}
%    \end{macrocode}
%    Otherwise we reset the current shape before calling |\bfseries|.
%    \begin{macrocode}
\else \gdef\theorem@headerfont{\normalfont\bfseries}\fi
%    \end{macrocode}
% \end{macro}
%
% \begin{macro}{\th@plain}
% \begin{macro}{\th@break}
% \begin{macro}{\th@marginbreak}
% \begin{macro}{\th@changebreak}
% \begin{macro}{\th@change}
% \begin{macro}{\th@margin}
%    The different styles are defined in macros such as
%    |\th@plain|.  Since memory space is precious in
%    ``non-Big-versions'', we have to avoid offering too many unused
%    definitions.  Therefore we define these styles in separate files
%    that can be loaded on demand.  Thus the commands themselves only
%    load these files. We use |\@input@|, a \LaTeXe{} internal command
%    that ensures that the file will be listed with |\listfiles|.
% \changes{v2.2a}{94/02/02}{Use `@input@ to load theorem layout files}
%    \begin{macrocode}
\gdef\th@plain{\@input@{thp.sty}}
\gdef\th@break{\@input@{thb.sty}}
\gdef\th@marginbreak{\@input@{thmb.sty}}
\gdef\th@changebreak{\@input@{thcb.sty}}
\gdef\th@change{\@input@{thc.sty}}
\gdef\th@margin{\@input@{thm.sty}}
%    \end{macrocode}
%    This list will be expanded when new styles become
%    available.  For testing,  just append new theorem substyles
%    as document options.
% \end{macro}
% \end{macro}
% \end{macro}
% \end{macro}
% \end{macro}
% \end{macro}
%
%
%
% \subsection{Definition of a new theorem set}
%
% As already pointed out, a new theorem environment can be defined in
% three different ways:
% \begin{verbatim}
%    \newtheorem{Lem}{Lemma}
%    \newtheorem{Lem}{Lemma}[section]
%    \newtheorem{Lem}[Theorem]{Lemma}
%\end{verbatim}
% The function of the macro |\newtheorem| is to recognize these cases
% and then to branch into one of the three macros |\@ynthm|, |\@xnthm|
% or |\@othm|.  This mechanism is adopted unchanged from
% \cite{latex:source}; the essential point here is that, for example,
% in the second case, the arguments \texttt{Lem}, \texttt{Lemma} and
% \texttt{section} are passed over to the macro |\@xnthm|.
%
% We inspect this case first because the others present fewer problems,
% and thus are easily derived from this one.
%
% \begin{macro}{\@xnthm}
%    For our example arguments, the macro |\@xnthm| must fulfill
%    the following:
%    \begin{itemize}
%       \item
%          Define a new \LaTeX-counter `Lem'
%       \item
%          reset this counter within a |\section|
%       \item
%          define the macro |\theLem|
%       \item
%          define the environment macros |\Lem| and
%          |\endLem| using the current
%          |\theoremstyle| and |\theorem@bodyfont|.
%    \end{itemize}
%    Obviously, all this should happen only if the first argument
%    of |\@xnthm| (i.e.\ \texttt{Lem} in our example) is chosen so
%    as not to conflict with any previously defined commands or
%    environments.  This test is performed by the \LaTeX{} macro
%    |\@ifdefinable|.
%    \begin{macrocode}
\gdef\@xnthm#1#2[#3]{\expandafter\@ifdefinable\csname #1\endcsname
%    \end{macrocode}
%    Therefore, the first argument of |\@ifdefinable| is the
%    expansion (in the example, |\Lem|) of
%    |\csname||#1\endcsname|.  The second argument is
%    executed only if the test has been completed successfully.
%    \begin{macrocode}
   {%
%    \end{macrocode}
%    Now we define the new counter. The names of the \LaTeX{}
%    macros employed should speak for themselves:
% \changes{v2.2b}{1995/10/07}{Use \cs{@newctr}}
% \changes{v2.2c}{1995/11/23}{Use \cs{@newctr} properly.}
%    \begin{macrocode}
    \@definecounter{#1}\@newctr{#1}[#3]%
%    \end{macrocode}
%    Using |\@newctr| will give a proper error message if the counter
%    in |#3| is not defined.
%    In defining `|\theLem|' we must generate the desired macro
%    name by use of |\expandafter| and |\csname|.
%    \begin{macrocode}
    \expandafter\xdef\csname the#1\endcsname
%    \end{macrocode}
%    An |\xdef| is used in order to make the definition global,
%    and to ensure that it contains the replacement texts of
%    |\@thmcountersep| and |\@thmcounter|.\footnote{These
%    two macros can be defined by the document style.  Their default
%    values produce a `\texttt{.}' as separation and an arabic
%    representation of the number.} However, not everything should be
%    expanded. For example, it saves space to use |\thesection|
%    instead of its---at times---lengthy expansion.
%    \begin{macrocode}
      {\expandafter \noexpand \csname the#3\endcsname
       \@thmcountersep \@thmcounter{#1}}%
%    \end{macrocode}
%    Thus with the defaults of \LaTeX{}, |\theLem| would be
%    replaced by the command sequence |\thesection.\arabic{Lem}|.
%
%    We will now look at the definition of the macro which is executed
%    at the beginning of the actual environment (in our example this
%    macro is |\Lem|). It should be noted that we use an
%    ``|\expandafter| trick'' to expand only certain parts of the
%    replacement text at the time of the definition.
%    \begin{macrocode}
    \def\@tempa{\global\@namedef{#1}}%
    \expandafter \@tempa \expandafter{%
%    \end{macrocode}
%    First, the macro that contains the current definitions of
%    |\@begintheorem| and |\@opargtheorem| should be called
%    up.  The name of this macro---as is already known---has the form
%    |\th@|\meta{theorem\ style}; therefore, it must be called by
%    \begin{macrocode}
      \csname th@\the \theorem@style
%    \end{macrocode}
%    In addition the default theorem font should be changeable, i.e.\
%    we have to insert the contents of |\theorem@bodyfont|. For
%    that reason, we expand even further, beyond |\endcsname|,
%    and thus insert the contents of the \textsf{token} register
%    |\theorem@bodyfont| in the replacement text.
%    \begin{macrocode}
            \expandafter \endcsname \the \theorem@bodyfont
%    \end{macrocode}
%    Now it is time to call the macro |\@thm| which takes over
%    the further processing.  It has two arguments: the current
%    counter name (in our example, \texttt{Lem}), and the text of the
%    label (in our example, \texttt{Lemma}).
%    \begin{macrocode}
     \@thm{#1}{#2}}%
%    \end{macrocode}
%    With this, the `sub-definition' is complete.  The macro
%    |\@endtheorem| ends a theorem environment and is, so far, nothing
%    but an |\endtrivlist|.  (Hence it is defined globally, and not
%    within the theorem styles.\footnote{This has to be changed as
%    soon as theorem styles that change \texttt{\bslash @endtheorem}
%    exist.  In such a case, all existing styles must be changed as
%    well since they will have to reset the macro.}) Therefore, we can
%    set it equivalent to the macro that ends the theorem set (in our
%    example, |\endLem|).  However, if some day theorem styles exist
%    that do change |\@endtheorem|, we would have to use the
%    commented-out line instead.
%    \begin{macrocode}
    \global \expandafter \let \csname end#1\endcsname \@endtheorem
%   \global\@namedef{end#1}{\@endtheorem}%
%    \end{macrocode}
%    With these commands all the required definitions are employed,
%    unless the test |\@ifdefinable| has failed.  Therefore, we
%    end the second argument of this macro and with it the definition
%    of |\@xnthm|.
%    \begin{macrocode}
   }}
%    \end{macrocode}
% \end{macro}
%
% \begin{macro}{\@ynthm}
%    The definition of |\@ynthm| is completely analogous. In
%    this case the new counter that is defined is not reset within
%    another counter; thus the definition of |\the...|
%    is simplified:
%    \begin{macrocode}
\gdef\@ynthm#1#2{\expandafter\@ifdefinable\csname #1\endcsname
   {\@definecounter{#1}%
    \expandafter\xdef\csname the#1\endcsname{\@thmcounter{#1}}%
%    \end{macrocode}
%    The rest of the definition corresponds literally to that of
%    |\@xnthm|:
%    \begin{macrocode}
    \def\@tempa{\global\@namedef{#1}}\expandafter \@tempa
     \expandafter{\csname th@\the \theorem@style \expandafter
     \endcsname \the\theorem@bodyfont \@thm{#1}{#2}}%
    \global \expandafter \let \csname end#1\endcsname \@endtheorem}}
%    \end{macrocode}
% \end{macro}
%
% \begin{macro}{\@othm}
%    The definition of |\@othm| does not contain anything new.
%    \begin{macrocode}
\gdef\@othm#1[#2]#3{%
%    \end{macrocode}
% \changes{v2.2b}{95/10/07}{Build in check for counter defined}
%    We do not define a new counter but instead use one that has
%    already been defined.  Thus the only definition we need is that
%    of this pseudo-counter (i.e.\ |\the|\meta{env.\ name}).
%    First we check if |#2| corresponds to a known counter name.
%    \begin{macrocode}
  \expandafter\ifx\csname c@#2\endcsname\relax
   \@nocounterr{#2}%
  \else
   \expandafter\@ifdefinable\csname #1\endcsname
   {\expandafter \xdef \csname the#1\endcsname
     {\expandafter \noexpand \csname the#2\endcsname}%
%    \end{macrocode}
%    All other parts of the definition can be adopted from
%    |\@xnthm|. We have to remember, though, that in this case
%    the name of the current counter and the theorem label have moved
%    to the second and third arguments.
%    \begin{macrocode}
    \def\@tempa{\global\@namedef{#1}}\expandafter \@tempa
     \expandafter{\csname th@\the \theorem@style \expandafter
     \endcsname \the\theorem@bodyfont \@thm{#2}{#3}}%
    \global \expandafter \let \csname end#1\endcsname \@endtheorem}%
  \fi}
%    \end{macrocode}
% \end{macro}
%
% \subsection{Macros that are employed in a {\envfont theorem}
%    environment}
%
% \begin{macro}{\@thm}
%    The macro |\@thm| has to increase the current counter.  Then,
%    depending on whether the environment has (or does not have) an
%    optional argument, it has to branch into either |\@begintheorem|
%    or |\@opargtheorem|.
%    \begin{macrocode}
\gdef\@thm#1#2{\refstepcounter{#1}%
%    \end{macrocode}
%    Now we start a {\envfont trivlist} environment, and give
%    |\@topsep| and |\@topsepadd| the values of the \textsf{skip}
%    registers |\theorempreskipamount| and |\theorempostskipamount|.
%    The value in |\@topsep| is the vertical space that is inserted by
%    the first (and only) |\item| in our |\trivlist| whilst
%    |\@topsepadd| is inserted by |\@endparenv| at the end of that
%    {\envfont trivlist} environment.  By using these registers, we
%    obtain the desired space around a {\envfont theorem} environment.
%    \begin{macrocode}
   \trivlist
   \@topsep \theorempreskipamount               % used by first \item
   \@topsepadd \theorempostskipamount           % used by \@endparenv
%    \end{macrocode}
%    Now we have to test whether an optional argument has been given.
% \changes{v2.0d}{89/07/18}{Forgotten percent added.}
%    \begin{macrocode}
   \@ifnextchar [%
%    \end{macrocode}
%    If there is an optional argument, we will call |\@ythm|, and
%    move the arguments read back into the input stream.
%    \begin{macrocode}
   {\@ythm{#1}{#2}}%
%    \end{macrocode}
%    If not, we call |\@begintheorem|. Its first argument is the
%    name of the theorem set (hence the second argument of
%    |\@thm|). Its second argument is the macro that produces the
%    current number.
%    \begin{macrocode}
   {\@begintheorem{#2}{\csname the#1\endcsname}\ignorespaces}}
%    \end{macrocode}
% \end{macro}
%
% \begin{macro}{\@xthm}
% \begin{macro}{\@ythm}
%    Both these macros were originally called by |\@thm|.  We do
%    not need |\@xthm| anymore, hence we reset it to
%    |\relax|. The definition of |\@ythm| has not changed at
%    all from its definition in \LaTeX. In order to make the macros
%    easier to understand, we will
%    nevertheless present it (commented out).
%    \begin{macrocode}
\global\let\@xthm\relax
% \def\@ythm#1#2[#3]{\@opargbegintheorem{#2}{\csname
%           the#1\endcsname}{#3}\ignorespaces}
%</package>
%    \end{macrocode}
%    The primitive |\ignorespaces| in |\@ythm| and
%    |\@thm| is needed to remove the spaces between the
%    |\begin{...}| and the actual text.
% \end{macro}
% \end{macro}
%
%
% \subsection{Definition of the theorem substyles}
%
% As already pointed out, the theorem substyles, defined below, are
% only loaded when necessary.  Note that all these substyles, except
% \texttt{plain}, have |\slshape| as the default body font.
%
% \subsubsection{The \texttt{plain} style}
%
% \label{dok:th@plain} As the following macros use |@|, we have
% to locally set the |\catcode| of this symbol to ``letter''.
% This happens within a group, so that we do not have to worry about
% which |\catcode| that symbol had before.
%    \begin{macrocode}
%<*thp>
\begingroup \makeatletter
%    \end{macrocode}
% Since we are now within a group, we must make all definitions
% globally.  First we make sure that \texttt{theorem.sty} is loaded.
% This will allow us to use this file as a document style option
% without having to call \texttt{theorem} itself as an option.  At the
% same time, we assure that at least version 2 is loaded, since
% |\theorem@style| was not defined in earlier versions.
%    \begin{macrocode}
\@ifundefined{theorem@style}{\input{theorem.sty}}{}
\ProvidesFile{thp.sty}
             [\FMithmInfo]
%    \end{macrocode}
%
%
% \begin{macro}{\th@plain}
% \begin{macro}{\@begintheorem}
% \begin{macro}{\@opargbegintheorem}
%    |\theoremstyle{plain}| corresponds to the original
%    definition, except that the distances to the surrounding text are
%    determined by the parameters |\theorempreskipamount| and
%    |\theorempostskipamount|.
%    First we set the default body font.
%    \begin{macrocode}
\gdef\th@plain{\normalfont\itshape
%    \end{macrocode}
%    Then we define |\@begintheorem| and
%    |\@opargbegintheorem|. These two macros define how the
%    header of a theorem is typeset.  |\@opargbegintheorem| will
%    be called if a \textsf{theorem} environment with an optional
%    argument is encountered; otherwise, the header is constructed by
%    calling |\@begintheorem|.  If one of these macros is
%    executed, we are within a \textsf{trivlist} environment started by
%    |\@thm|.  So the theorem header is produced with an
%    |\item| command.
%
%    Instead of specifying the header font directly, all standard
%    theorem styles use the |\theorem@headerfont| macro to allow
%    customization.  The extra space (|\labelsep|) is necessary
%    because of problems in the \textsf{trivlist} environment.
%    \begin{macrocode}
  \def\@begintheorem##1##2{%
        \item[\hskip\labelsep \theorem@headerfont ##1\ ##2]}%
%    \end{macrocode}
%    The definition of |\@opargbegintheorem| is completely
%    analogous. The only difference is the fact that there exists a
%    third argument (which is the optional parameter of the
%    environment and contains additional information about the
%    theorem).  Customarily we enclose it in parentheses.
%    \begin{macrocode}
\def\@opargbegintheorem##1##2##3{%
   \item[\hskip\labelsep \theorem@headerfont ##1\ ##2\ (##3)]}}
%    \end{macrocode}
% \end{macro}
% \end{macro}
% \end{macro}
%
% \noindent
% We conclude with an |\endgroup| to restore the
% |\catcode| of |@|.
%    \begin{macrocode}
\endgroup
%</thp>
%    \end{macrocode}
%
%
% \subsubsection{The \texttt{break} style}
%
% This style option is stored in the file |thb.sty|.
% For the next two lines see the documentation for |\th@plain| on
% page~\pageref{dok:th@plain}.
%    \begin{macrocode}
%<*thb>
\begingroup \makeatletter
\@ifundefined{theorem@style}{\input{theorem.sty}}{}
\ProvidesFile{thb.sty}
             [\FMithmInfo]
%    \end{macrocode}
%
% \begin{macro}{\th@break}
%    |\theoremstyle{break}| produces a line break after the name
%    of the theorem.  The font is |\slshape|.  Hence, we define
%    |\th@break| as follows:
%    \begin{macrocode}
\gdef\th@break{\normalfont\slshape
  \def\@begintheorem##1##2{\item[%
%    \end{macrocode}
%    We run into the following problem: it is not possible to create
%    the header with |\item[|\meta{title}|]| and then start
%    a new line by, for example, |\mbox{}\\|. Such a definition
%    will fail whenever a list environment follows immediately.  With
%    the above construction, the |\mbox{}| causes the switch
%    |@inlabel| (cf.\ definition of |\list| and
%    |\trivlist| in \cite{latex:source}) to be set to \texttt{false}
%    and so the following list will insert additional vertical
%    space (|\topskip|). This is quite annoying. Therefore, we
%    create the line break within the |\item|. In order to ensure
%    that the text will begin at the proper position in the following
%    line, we simply pretend that the label does not take any
%    room.\footnote{This will lead to problems whenever very high
%    symbols occurring in the line tower into the
%    heading. So, something else has to be done here sometime.}
%    \begin{macrocode}
       \rlap{\vbox{\hbox{\hskip \labelsep\theorem@headerfont ##1\ ##2}%
                   \hbox{\strut}}}]}%
%    \end{macrocode}
%    Again, the definition of |\@opargbegintheorem| is
%    completely analogous.
%    \begin{macrocode}
\def\@opargbegintheorem##1##2##3{%
  \item[\rlap{\vbox{\hbox{\hskip \labelsep \theorem@headerfont
                     ##1\ ##2\ (##3)}%
                    \hbox{\strut}}}]}}
\endgroup
%</thb>
%    \end{macrocode}
% \end{macro}
%
%
% \subsubsection{The \texttt{changebreak} style}
%
%    \begin{macrocode}
% This style option is stored in the file |thcb.sty|.
%    \begin{macrocode}
%<*thcb>
\begingroup \makeatletter
\@ifundefined{theorem@style}{\input{theorem.sty}}{}
\ProvidesFile{thcb.sty}
             [\FMithmInfo]
%    \end{macrocode}
%
% \begin{macro}{\th@changebreak}
%    The \texttt{change-break} theorem style is like \texttt{break}
%    but with interchange of theorem name and theorem number.  Thus we
%    define |\th@changebreak| as follows:
%    \begin{macrocode}
\gdef\th@changebreak{\normalfont\slshape
  \def\@begintheorem##1##2{\item
       [\rlap{\vbox{\hbox{\hskip\labelsep \theorem@headerfont ##2\ ##1}%
                    \hbox{\strut}}}]}%
\def\@opargbegintheorem##1##2##3{%
  \item[\rlap{\vbox{\hbox{\hskip\labelsep \theorem@headerfont
                     ##2\ ##1\ (##3)}%
                    \hbox{\strut}}}]}}
\endgroup
%</thcb>
%    \end{macrocode}
% \end{macro}
%
%
% \subsubsection{The \texttt{change} style}
%
% This style option is stored in the file |thc.sty|.
%    \begin{macrocode}
%<*thc>
\begingroup \makeatletter
\@ifundefined{theorem@style}{\input{theorem.sty}}{}
\ProvidesFile{thc.sty}
             [\FMithmInfo]
%    \end{macrocode}
%
% \begin{macro}{\th@change}
%    The \texttt{change} theorem style corresponds to the
%    \texttt{change break} style without a linebreak after the header.
%    To say it in another way, it's the same as the \texttt{plain}
%    style but with number and name interchanged and |\slshape| as the
%    default font.
%    \begin{macrocode}
\gdef\th@change{\normalfont\slshape
  \def\@begintheorem##1##2{\item
       [\hskip\labelsep \theorem@headerfont ##2\ ##1]}%
\def\@opargbegintheorem##1##2##3{%
  \item[\hskip\labelsep \theorem@headerfont ##2\ ##1\ (##3)]}}
\endgroup
%</thc>
%    \end{macrocode}
% \end{macro}
%
%
% \subsubsection{The \texttt{marginbreak} style}
%
% This style option is the one used most often at Mainz.  It is saved
% in the file |thmb.sty|.
%    \begin{macrocode}
%<*thmb>
\begingroup \makeatletter
\@ifundefined{theorem@style}{\input{theorem.sty}}{}
\ProvidesFile{thmb.sty}
             [\FMithmInfo]
%    \end{macrocode}
%
% \begin{macro}{\th@marginbreak}
%    The \texttt{margin break} style is nearly the same as the
%    \texttt{change break} style. The only difference is the placement
%    of the theorem number.  We use |\llap| to place it in the left
%    margin.
%
%    In this style |\labelsep| denotes the separation between the
%    number and the text.
%    \begin{macrocode}
\gdef\th@marginbreak{\normalfont\slshape
  \def\@begintheorem##1##2{\item
       [\rlap{\vbox{\theorem@headerfont
                    \hbox{\llap{##2}\hskip\labelsep ##1}%
                    \hbox{\strut}}}]}%
\def\@opargbegintheorem##1##2##3{%
  \item[\rlap{\vbox{\theorem@headerfont
                    \hbox{\llap{##2}\hskip\labelsep ##1\ (##3)}%
                    \hbox{\strut}}}]}}
\endgroup
%</thmb>
%    \end{macrocode}
% \changes{v2.1c}{90/02/17}{`endgroup outside macrocode env.}
% \end{macro}
%
%
%
% \subsubsection{The \texttt{margin} style}
%
% This style option is stored in the file |thm.sty|.
%    \begin{macrocode}
%<*thm>
\begingroup \makeatletter
\@ifundefined{theorem@style}{\input{theorem.sty}}{}
\ProvidesFile{thm.sty}
             [\FMithmInfo]
%    \end{macrocode}
%
% \begin{macro}{\th@margin}
%    Again this is only a variant of the theorem styles described
%    above without any new ideas.
%    \begin{macrocode}
\gdef\th@margin{\normalfont\slshape
  \def\@begintheorem##1##2{\item
       [\theorem@headerfont \llap{##2}\hskip\labelsep ##1]}%
\def\@opargbegintheorem##1##2##3{%
  \item[\theorem@headerfont \llap{##2}\hskip\labelsep ##1\ (##3)]}}
\endgroup
%</thm>
%    \end{macrocode}
% \end{macro}
%
%
%
%
% \subsection{Final Definitions}
%
% \begin{macro}{\theorempreskipamount}
% \begin{macro}{\theorempostskipamount}
%    The \textsf{skip} parameters that regulate the vertical empty space
%    before and after the {\envfont theorem} environment have
%    to be allocated as well.
% \changes{v2.0f}{89/07/20}{Unnecessary `global' removed.}
%    \begin{macrocode}
%<*package>
\newskip\theorempreskipamount
\newskip\theorempostskipamount
%    \end{macrocode}
%    Since we have used the same values for all theorem sets, we now can
%    assign them.
%    \begin{macrocode}
\global\setlength\theorempreskipamount{12pt plus 5pt minus 3pt}
\global\setlength\theorempostskipamount{8pt plus 3pt minus 1.5pt}
%    \end{macrocode}
% \end{macro}
% \end{macro}
%
%
% \begin{macro}{\@endtheorem}
%    The same holds for the macro |\@endtheorem|, which ends a
%    {\envfont theorem} environment. Since it is the same for all
%    theorem sets, it is removed from the macros |\th@|\meta{style}.
%    It simply ends the {\envfont trivlist} environment, which was
%    begun in |\@thm|.
%    \begin{macrocode}
\global\let\@endtheorem=\endtrivlist
%    \end{macrocode}
% \end{macro}
%
%
%
% \begin{macro}{\@preamblecmds}
%    All macros defined above are to be used only in the preamble.
%    Therefore, we insert them in |\@preamblecmds| which will disable
%    them at begin document. This is done by the internal \LaTeXe{}
%    command |\@onlypreamble|.
% \changes{v2.0g}{89/09/19}{`theoremheaderfont and `theorembodyfont
%    added.}
% \changes{v2.2a}{94/02/03}{Use \LaTeXe{} interface}
%    \begin{macrocode}
\@onlypreamble\@xnthm
\@onlypreamble\@ynthm
\@onlypreamble\@othm
\@onlypreamble\newtheorem
\@onlypreamble\theoremstyle
\@onlypreamble\theorembodyfont
\@onlypreamble\theoremheaderfont
%    \end{macrocode}
% \end{macro}
%
% Finally we declare the \texttt{plain} theorem style to be the default.
% \changes{v2.1b}{90/06/04}{Fixed missing `item by declaring default
%    style.}
%    \begin{macrocode}
\theoremstyle{plain}
%</package>
%    \end{macrocode}
%
% \Finale

\endinput
}{}
\ProvidesFile{thmb.sty}
             [\FMithmInfo]
%    \end{macrocode}
%
% \begin{macro}{\th@marginbreak}
%    The \texttt{margin break} style is nearly the same as the
%    \texttt{change break} style. The only difference is the placement
%    of the theorem number.  We use |\llap| to place it in the left
%    margin.
%
%    In this style |\labelsep| denotes the separation between the
%    number and the text.
%    \begin{macrocode}
\gdef\th@marginbreak{\normalfont\slshape
  \def\@begintheorem##1##2{\item
       [\rlap{\vbox{\theorem@headerfont
                    \hbox{\llap{##2}\hskip\labelsep ##1}%
                    \hbox{\strut}}}]}%
\def\@opargbegintheorem##1##2##3{%
  \item[\rlap{\vbox{\theorem@headerfont
                    \hbox{\llap{##2}\hskip\labelsep ##1\ (##3)}%
                    \hbox{\strut}}}]}}
\endgroup
%</thmb>
%    \end{macrocode}
% \changes{v2.1c}{90/02/17}{`endgroup outside macrocode env.}
% \end{macro}
%
%
%
% \subsubsection{The \texttt{margin} style}
%
% This style option is stored in the file |thm.sty|.
%    \begin{macrocode}
%<*thm>
\begingroup \makeatletter
\@ifundefined{theorem@style}{% \iffalse meta-comment
%
% Copyright (C) 1993-2024
%
% The LaTeX Project and any individual authors listed elsewhere
% in this file.
%
% This file is part of the Standard LaTeX `Tools Bundle'.
% -------------------------------------------------------
%
% It may be distributed and/or modified under the
% conditions of the LaTeX Project Public License, either version 1.3c
% of this license or (at your option) any later version.
% The latest version of this license is in
%    https://www.latex-project.org/lppl.txt
% and version 1.3c or later is part of all distributions of LaTeX
% version 2005/12/01 or later.
%
% The list of all files belonging to the LaTeX `Tools Bundle' is
% given in the file `manifest.txt'.
%
% \fi
%
%%
%
%
\def\FMithmInfo{2023/07/05 v2.2c Theorem extension package (FMi)}
%
% \ProvidesFile{theorem.dtx}[\FMithmInfo]
%\iffalse   % this is a METACOMMENT !
%%
%% Package `theorem' to use with LaTeX2e
%% Copyright (C) 1989-1995 Frank Mittelbach, all rights reserved.
%<+package>\@ifundefined{theorem@style}{}{\endinput}
%<+package>\NeedsTeXFormat{LaTeX2e}[1995/06/01]
%<+package>\ProvidesPackage{theorem}
%<+package>         [\FMithmInfo]
%
% \fi
%
% \changes{v1.0}{87/05/07}{First draft.}
% \changes{v1.0a}{87/05/07}{`thistheoremstyle' added.}
% \changes{v1.1a}{87/07/29}{New concept: redefine the internal
%                          \protect\LaTeX{} commands `@xnthm' etc.}
% \changes{v1.1b}{87/07/31}{`theoremfont' used to specify font.}
% \changes{v1.2a}{88/01/08}{`@endtheorem' now globally defined.}
% \changes{v1.2b}{88/01/11}{`newtheorem' only allowed in the preamble.}
% \changes{v1.3a}{88/02/14}{`theoremfont' now a toks register.}
%
% \changes{v2.0a}{88/11/16}{New implementation and new documentation.}
% \changes{v2.0a}{88/11/16}{No layout changes but internal modifications
%                           old substyles won't work.}
% \changes{v2.0b}{89/01/30}{`global' used for all definitions.}
% \changes{v2.0c}{89/06/15}{Documentation translated to English.
%                            Thanks to LLL and Nina}
% \changes{v2.1a}{90/02/15}{Support for new font selection.}
% \changes{v2.2a}{94/02/03}{Cleanup for LaTeX2e}
%
% \DoNotIndex{\@endparenv,\@ifdefinable,\@ifnextchar,\@ifundefined}
% \DoNotIndex{\@namedef,\@nameuse,\@spaces,\@tempa,\@warning}
% \DoNotIndex{\ }
% \DoNotIndex{\begingroup,\bfseries}
% \DoNotIndex{\csname}
% \DoNotIndex{\def,\do,\docdate}
% \DoNotIndex{\endcsname,\endgroup,\endinput,\endtrivlist,\expandafter}
% \DoNotIndex{\filedate,\fileversion}
% \DoNotIndex{\gdef,\global}
% \DoNotIndex{\hbox,\hskip}
% \DoNotIndex{\ignorespaces,\input,\itshape,\item}
% \DoNotIndex{\labelsep,\let,\llap}
% \DoNotIndex{\makeatletter}
% \DoNotIndex{\noexpand}
% \DoNotIndex{\refstepcounter,\relax,\rlap}
% \DoNotIndex{\setlength,\slshape,\space,\string,\strut}
% \DoNotIndex{\the,\trivlist,\typeout}
% \DoNotIndex{\vbox}
% \DoNotIndex{\xdef}
%
% \def\deflabel#1{\ttfamily #1\hfill}
% \def\deflist#1{\begin{list}{}{\settowidth\labelwidth{\ttfamily #1}%
%                               \setlength\leftmargin\labelwidth
%                               \addtolength\leftmargin\labelsep
%                               \let\makelabel\deflabel}}
% \def\enddeflist{\end{list}}
% \def\envfont{\normalfont\ttfamily}
%
% \MakeShortVerb{\|}
%
% \newcommand{\TUB}{{\sl TUGboat\/}}
%
% \GetFileInfo{theorem.dtx}
% \title{An Extension of the
%        \LaTeX{} theorem environment\thanks{This file
%        has version number \fileversion, last
%        revised \filedate.}}
% \author{Frank Mittelbach}
%
% \MaintainedByLaTeXTeam{tools}
% \maketitle
%
% \begin{abstract}
%    The macros described in this paper yield an extension of the
%    \LaTeX{} theorem mechanism. It is designed to satisfy the
%    different requirements of various journals. Thus, the
%    layout of the ``theorems'' can be manipulated by determining a
%    ``style''. This article describes not only the use, but
%    also the definition, of the necessary macros.
% \end{abstract}
%
% \section*{Preface to version 2.2}
%
% For \LaTeXe{} this package did not need any fundamental changes. I
% only modified the messages generated so that theorem layout styles
% will show up with the |\listfiles| command and cleaned the section
% on the New Font Selection Scheme since this is now included in
% \LaTeX.
%
% \section*{Preface to version 2.1}
%
% This version is identical to 2.0g described in \TUB\ 10\#3 except
% for some internal defaults which are now set depending on the used
% font selection scheme.
%
% This was done to avoid unpleasant surprises if the new font selection
% scheme is in force. For further details see
% section~\ref{sec:newfonts} and \cite{bk:GMS94}.
%
% \section{Introduction}
%
% For our purposes here, ``theorems'' are labelled enunciations,
% often set off from the main text by extra space and a font change.
% Theorems, corollaries, conjectures, definitions, and remarks
% are all instances of ``theorems''.  The ``header'' of these structures
% is composed of a label (such as \textsc{Theorem} or \textsc{Remark})
% and a number which serializes an item in the sequence of items
% with the same label.
%
% Shortly after the introduction of \LaTeX{} at the Fachbereich
% Mathematik in Mainz, the desire to manipulate the layout of
% ``theorems'' arose.  In Mainz, the following two conventions came
% into general use:
% \begin{enumerate}
%   \item The number of the theorem is shown in the margin.
%   \item There is a line break at
%         the end of the theorem header.
% \end{enumerate}
%
% Additionally, some journals require different formats which depend on
% the ``sort of theorem'': e.g.\ often remarks and definitions are set
% in |\upshape|, while |\itshape| is employed for main theorems.
%
% Confronted with these requirements, a theorem environment was
% developed in Mainz which allows separate determination of the
% layout of the ``theorems sets'', comparable to |\pagestyle|.
%
%
% \section{The user interface}
% \subsection{Defining new theorem sets}
%
% \DescribeMacro\newtheorem
% As in the original \LaTeX{} version, the command |\newtheorem|
% defines a new ``theorem set'' or ``theorem-like structure''.
% Two required arguments name the new environment and give the text
% to be typeset with each instance of the new ``set'', while
% an optional argument determines how
% the ``set'' is enumerated:
% \begin{description}
%    \item[\ttfamily \bslash newtheorem\{foo\}\{bar\}]
%       The theorem set {\envfont foo} (whose name is \texttt{bar})
%       uses its own counter.
%    \item[\ttfamily \bslash newtheorem\{foo2\}{[foo]}\{bar2\}]
%       The theorem set {\envfont foo2} (printed name \texttt{bar2})
%       uses the same counter as the theorem set \texttt{foo}.
%    \item[\ttfamily \bslash newtheorem\{foo3\}\{bar3\}{[section]}]
%       The theorem set {\envfont foo3} (printed name \texttt{bar3}) is
%       enumerated within the counter \texttt{section}, i.e.\ with every
%       new |\section| the enumeration begins again with 1, and
%       the enumeration is composed from the section-number and the
%       theorem counter itself.
% \end{description}
%
% \DescribeMacro\theoremstyle
% Additionally, the command |\theoremstyle| can define the layout
% of various, or all, theorem sets.  It should be noted that any
% theorem set defined by |\newtheorem| is typeset in the
% |\theoremstyle| that is current at the time of the definition.
% Thus, the following
% \begin{verbatim}
%    \theoremstyle{break}        \newtheorem{Cor}{Corollary}
%    \theoremstyle{plain}        \newtheorem{Exa}{Example}[section]
%\end{verbatim}                             ^^A No blank at line start
% leads to the result that the set {\envfont Cor} is formatted in the
% style \texttt{break}, while the set {\envfont Exa} and all the
% following ones are formatted in the style \texttt{plain}, unless
% another |\theoremstyle| follows.  Since the definitions installed by
% |\newtheorem| are global, one also can limit |\theoremstyle| locally
% by grouping braces.
%
% \DescribeMacro\theorembodyfont
% The choice of the font for the theorem body is completely
% independent of the chosen |\theoremstyle|; this has proven to
% be very advantageous.  For example,
% \begin{verbatim}
%   {\theorembodyfont{\upshape}          \newtheorem{Rem}{Remark}}
%\end{verbatim}
% defines a theorem set {\envfont Rem}, which will be set in
% |\upshape| in the current layout (which in our example is
% \texttt{plain}).  As with |\theoremstyle|, the |\theorembodyfont|
% chosen is that current at the time of |\newtheorem|.  If
% |\theorembodyfont| is not specified or one defines
% |\theorembodyfont{}|, then the font used will be that defined by the
% |\theoremstyle|.
%
% \DescribeMacro\theoremheaderfont It is also possible to customize
% the font used for the theorem headers.  This is, however, a global
% declaration, and therefore there should be at most one
% |\theoremheaderfont| declaration in the preamble.\footnote{If it is
% actually necessary to have different header fonts, one has to define
% new theorem styles (substituting the desired font) or specify the
% information directly in the \texttt{\bslash newtheorem} declaration
% (the unclean variant).}
%
% \DescribeMacro\theorempreskipamount
% \DescribeMacro\theorempostskipamount Two additional parameters
% affect the vertical space around the theorem environments:
% |\theorempreskipamount| and |\theorempostskipamount| define,
% respectively, the spacing before and after such an environment.
% These parameters apply for all theorem sets and can be manipulated
% with the ordinary length macros.  They are rubber lengths
% (`\textsf{skips}'), and therefore can contain \texttt{plus} and
% \texttt{minus} parts.
%
% Since the definition of theorem sets should---most sensibly---be
% placed in the preamble, we only allow installation there.
% It is therefore possible to release the memory used here after
% |\begin{document}|, in order to make room for other
% applications.
%
%
% \subsection{Existing theorem styles}
%
% The following theorem styles exist to date:
% \begin{deflist}{marginbreak}
%    \item[plain]
%       This theorem style emulates the original \LaTeX{} definition,
%       except that additionally the parameters
%       |\theorem...skipamount| are used.
%    \item[break]
%       In this style, the theorem header is followed by a line
%       break.
%    \item[marginbreak]
%       The theorem number is set in the margin, and there is a line
%       break as in \texttt{break}.
%    \item[changebreak]
%       Like \texttt{break}, but with header number and text
%       interchanged.
%    \item[change]
%       Header number and text are interchanged, without a line break.
%    \item[margin]
%       The number is set in the left margin, without a line break.
% \end{deflist}
% All styles (except \texttt{plain}) select |\slshape| as the default
% |\theorembodyfont|.
%
%
% \subsection{Examples}
%
% Given the above theorem sets {\envfont Cor}, {\envfont Exa}
% and {\envfont Rem},
% suppose that the preamble also contains the declarations:
% \begin{verbatim}
%    \theoremstyle{marginbreak}   \newtheorem{Lem}[Cor]{Lemma}
%    \theoremstyle{change}
%    \theorembodyfont{\itshape}        \newtheorem{Def}[Cor]{Definition}
%
%    \theoremheaderfont{\scshape}
%\end{verbatim}
% Then the following are some typical examples of the typeset output
% resulting from their use.
%
% \begin{Cor}
%    This is a sentence typeset in the theorem environment {\envfont
%    Cor}.
% \end{Cor}
%
% \begin{Exa}
%    This is a sentence typeset in the theorem environment {\envfont
%    Exa}.
% \end{Exa}
%
% \begin{Rem}
%    This is a sentence typeset in the theorem environment {\envfont
%    Rem}.
% \end{Rem}
%
% \begin{Lem}[Ben User]
%    This is a sentence typeset in the theorem environment {\envfont
%    Lem}.
% \end{Lem}
%
% \begin{Def}[Very impressive Definition]
%    This is a sentence typeset in the theorem environment {\envfont
%    Def}.
% \end{Def}
%
% The last two examples show the effect of the optional argument to
% a theorem environment (it is the text typeset in parentheses).
%
%
% \section{Special Considerations}
% \label{sec:newfonts}
%
% Theoremheader and body are implemented as a unit. This means that
% the |\theoremheaderfont|  will inherit characteristics of the
% |\theorembodyfont| in \LaTeXe. Thus, if for example
% |\theorembodyfont| is |\itshape| and |\theoremheaderfont|
% is |\bfseries| the font selected for the header will have the
% characteristics `bold extended italic'. If this is not desired one
% should set the |\theoremheaderfont| to something like
% \begin{verbatim}
%      \theoremheaderfont{\normalfont\bfseries}
%\end{verbatim}
% i.e.\ supplying all necessary font information explicitly.
%
%
%
% \section{Acknowledgements}
%
% The publication of this set of macros was only possible with the
% help of Christina Busse (translating the manuscript into English),
% Joachim Pense (playing the r\^ole of typist), Chris Rowley
% (looking everything over) and many others providing useful
% suggestions.
%
% \MaybeStop{%
%    \begin{thebibliography}{9}
%       \bibitem{bk:GMS94} \textsc{M.~Goossens}, \textsc{F.~Mittelbach}
%         and \textsc{A.~Samarin}.
%         \newblock The \LaTeX{} Companion.
%         \newblock
%         Addison-Wesley, Reading, Massachusetts, 1994.
%       \bibitem{latex:source} \textsc{Lamport, Leslie.} \newblock
%                               \texttt{latex.tex}, version 2.09, date
%                               Feb. 1990.
%    \end{thebibliography}
%    \PrintIndex
%    \PrintChanges}
%
%
% \section{The documentation driver file}
%
% The next bit of code contains the documentation driver file for
% \TeX{}, i.e., the file that will produce the documentation you are
% currently reading. It will be extracted from this file by the
% \texttt{docstrip} program. Since it is the first code in the file
% one can alternatively process this file directly with \LaTeXe{} to
% obtain the documentation.
%
%    \begin{macrocode}
%<*driver>
\documentclass{ltxdoc}

\usepackage{theorem}

%<+driver>% The next few lines define theorem sets which are used
%<+driver>% in the example section of the documentation.

\theoremstyle{break}       \newtheorem{Cor}{Corollary}
\theoremstyle{plain}       \newtheorem{Exa}{Example}[section]
{\theorembodyfont{\upshape}\newtheorem{Rem}{Remark}}
\theoremstyle{marginbreak} \newtheorem{Lem}[Cor]{Lemma}
\theoremstyle{change}
\theorembodyfont{\itshape} \newtheorem{Def}[Cor]{Definition}

\theoremheaderfont{\scshape}

\RecordChanges

\begin{document}
   \DocInput{theorem.dtx}
\end{document}
%</driver>
%    \end{macrocode}
%
% \section{Definition of the Macros}
%
% If the file has been loaded before, we abort immediately. If not the
% package announces itself (this is actually done at the very top of
% the file---the way it is done isn't good style so don't copy it).
% \changes{v2.0e}{89/07/19}{Spaces removed from `typeout'.}
% \changes{v2.2b}{95/11/19}{Announce moved to top of file}
%    \begin{macrocode}
%<*package>
%\@ifundefined{theorem@style}{}{\endinput}
%\def\FMithmInfo{1995/11/19 v2.2b Theorem extension package (FMi)}
%\ProvidesPackage{theorem}[\FMithmInfo]
%    \end{macrocode}
%
%
% \subsection{Definition of theorem styles and fonts}
%
% All the definitions in this file are done globally to allow
% inputting this file inside a group.
%
% \begin{macro}{\theoremstyle}
%    Before a theorem style can be installed, the chosen style must be
%    known. For that reason, we must test to see that
%    |\th@|\meta{style} is known or, more precisely, that it is
%    different from |\relax|. If the style is not known then
%    |\th@plain| is used.
%    \begin{macrocode}
\gdef\theoremstyle#1{%
   \@ifundefined{th@#1}{\@warning
          {Unknown theoremstyle `#1'. Using `plain'}%
          \theorem@style{plain}}%
%    \end{macrocode}
%    We save the theorem style to be used in the token register
%    |\theorem@style|.
%    \begin{macrocode}
      {\theorem@style{#1}}%
%    \end{macrocode}
%    Now we ``evaluate'' the theorem style: this means, we call the
%    macro |\th@|\meta{style} which will activate the relevant
%    definitions which are contained in a separate file.  This is done
%    in a group to suppress changes to the current font.  This could
%    otherwise pose problems together with the new font selection
%    scheme\footnote{When I printed the original article using the new
%    font selection scheme I ended with a document with slanted
%    typefaces (text headings and all) simply because one of the
%    theorem styles used \texttt{\bslash sl} at toplevel.} if the
%    |\th@|\meta{style} is evaluated a second time.
% \changes{v2.1a}{90/02/15}{Group added around `th@..}
% \changes{v2.1a}{90/02/15}{Reseting of `..@begintheorem removed.}
%    \begin{macrocode}
      \begingroup
        \csname th@\the\theorem@style \endcsname
      \endgroup}
%    \end{macrocode}
% \end{macro}
%
%
% \begin{macro}{\@begintheorem}
% \changes{v2.1a}{90/02/15}{Relaxed at toplevel.}
% \begin{macro}{\@opargbegintheorem}
% \changes{v2.1a}{90/02/15}{Relaxed at toplevel.}
%    We reset |\@begintheorem| and |\@opargbegintheorem| to
%    |\relax| since these commands are no longer necessary at
%    toplevel. This will save a few tokens.
%    \begin{macrocode}
\global\let\@begintheorem\relax
\global\let\@opargbegintheorem\relax
%    \end{macrocode}
% \end{macro}
% \end{macro}
%
% \begin{macro}{\theorem@style}
%    Obviously the \textsf{token} register used above has to be
%    allocated. To assure the utmost compatibility with the
%    original \LaTeX{} definition, we set the default theorem style to
%    \texttt{plain}, which implements the usual \LaTeX{} convention.
% \changes{v2.0f}{89/07/20}{Unnecessary `global' removed.}
%    \begin{macrocode}
\newtoks\theorem@style
\global\theorem@style{plain}
%    \end{macrocode}
% \end{macro}
%
% \begin{macro}{\theorembodyfont}
% \begin{macro}{\theorem@bodyfont}
%    If the |\theorembodyfont| is set by the user then it should not
%    interact with the default font set in the theorem style. When the
%    new font selection is in force this may happen if, for example,
%    the default is |\itshape| and the new |\theorembodyfont| is
%    |\sffamily|. So we add a |\reset@font| command in front of the
%    user definition.
% \changes{v2.0f}{89/07/20}{Unnecessary `global' removed.}
% \changes{v2.0g}{89/09/19}{`theoremfont to `theorembodyfont.}
% \changes{v2.1c}{92/06/22}{`reset@font added and logic changed.}
% \changes{v2.1d}{92/10/28}{Reset to empty if arg is empty}
%    \begin{macrocode}
\gdef\theorembodyfont#1{%
%    \end{macrocode}
%    We check if the argument supplied is empty and if so put nothing
%    into the |\theorem@bodyfont| token register to allow for
%    |\theorembodyfont{}| as a mean of using the default of the
%    current |\theoremstyle|.
%    \begin{macrocode}
   \def\@tempa{#1}%
   \ifx\@tempa\@empty
    \theorem@bodyfont{}%
   \else
    \theorem@bodyfont{\reset@font#1}%
   \fi
   }
\newtoks\theorem@bodyfont
\global\theorem@bodyfont{}
%    \end{macrocode}
% \end{macro}
% \end{macro}
%
%
% \begin{macro}{\theoremheaderfont}
% \changes{v2.0g}{89/09/19}{Macro added.}
%    The font for the theorem headers is handled differently because
%    this definition applies to all theorem styles.
%    \begin{macrocode}
\gdef\theoremheaderfont#1{\gdef\theorem@headerfont{#1}%
%    \end{macrocode}
%    After using the macro once it is redefined to produce an error
%    message.
%    \begin{macrocode}
       \gdef\theoremheaderfont##1{%
        \typeout{\string\theoremheaderfont\space should be used
                 only once.}}}
%    \end{macrocode}
% \end{macro}
%
% \begin{macro}{\theorem@headerfont}
% \changes{v2.0g}{89/09/19}{Macro added.}
%    To set the |\theorem@headerfont| default we first test if
%    the new fontselection scheme is in force.
% \changes{v2.1a}{90/02/15}{Support for new font selection.}
%    \begin{macrocode}
\ifx\upshape\undefined
%    \end{macrocode}
%    If not we define it to expand into |\bfseries|. We don't use
%    |\let| just in case a following style option redefines this
%    macro.
%    \begin{macrocode}
\gdef\theorem@headerfont{\bfseries}
%    \end{macrocode}
%    Otherwise we reset the current shape before calling |\bfseries|.
%    \begin{macrocode}
\else \gdef\theorem@headerfont{\normalfont\bfseries}\fi
%    \end{macrocode}
% \end{macro}
%
% \begin{macro}{\th@plain}
% \begin{macro}{\th@break}
% \begin{macro}{\th@marginbreak}
% \begin{macro}{\th@changebreak}
% \begin{macro}{\th@change}
% \begin{macro}{\th@margin}
%    The different styles are defined in macros such as
%    |\th@plain|.  Since memory space is precious in
%    ``non-Big-versions'', we have to avoid offering too many unused
%    definitions.  Therefore we define these styles in separate files
%    that can be loaded on demand.  Thus the commands themselves only
%    load these files. We use |\@input@|, a \LaTeXe{} internal command
%    that ensures that the file will be listed with |\listfiles|.
% \changes{v2.2a}{94/02/02}{Use `@input@ to load theorem layout files}
%    \begin{macrocode}
\gdef\th@plain{\@input@{thp.sty}}
\gdef\th@break{\@input@{thb.sty}}
\gdef\th@marginbreak{\@input@{thmb.sty}}
\gdef\th@changebreak{\@input@{thcb.sty}}
\gdef\th@change{\@input@{thc.sty}}
\gdef\th@margin{\@input@{thm.sty}}
%    \end{macrocode}
%    This list will be expanded when new styles become
%    available.  For testing,  just append new theorem substyles
%    as document options.
% \end{macro}
% \end{macro}
% \end{macro}
% \end{macro}
% \end{macro}
% \end{macro}
%
%
%
% \subsection{Definition of a new theorem set}
%
% As already pointed out, a new theorem environment can be defined in
% three different ways:
% \begin{verbatim}
%    \newtheorem{Lem}{Lemma}
%    \newtheorem{Lem}{Lemma}[section]
%    \newtheorem{Lem}[Theorem]{Lemma}
%\end{verbatim}
% The function of the macro |\newtheorem| is to recognize these cases
% and then to branch into one of the three macros |\@ynthm|, |\@xnthm|
% or |\@othm|.  This mechanism is adopted unchanged from
% \cite{latex:source}; the essential point here is that, for example,
% in the second case, the arguments \texttt{Lem}, \texttt{Lemma} and
% \texttt{section} are passed over to the macro |\@xnthm|.
%
% We inspect this case first because the others present fewer problems,
% and thus are easily derived from this one.
%
% \begin{macro}{\@xnthm}
%    For our example arguments, the macro |\@xnthm| must fulfill
%    the following:
%    \begin{itemize}
%       \item
%          Define a new \LaTeX-counter `Lem'
%       \item
%          reset this counter within a |\section|
%       \item
%          define the macro |\theLem|
%       \item
%          define the environment macros |\Lem| and
%          |\endLem| using the current
%          |\theoremstyle| and |\theorem@bodyfont|.
%    \end{itemize}
%    Obviously, all this should happen only if the first argument
%    of |\@xnthm| (i.e.\ \texttt{Lem} in our example) is chosen so
%    as not to conflict with any previously defined commands or
%    environments.  This test is performed by the \LaTeX{} macro
%    |\@ifdefinable|.
%    \begin{macrocode}
\gdef\@xnthm#1#2[#3]{\expandafter\@ifdefinable\csname #1\endcsname
%    \end{macrocode}
%    Therefore, the first argument of |\@ifdefinable| is the
%    expansion (in the example, |\Lem|) of
%    |\csname||#1\endcsname|.  The second argument is
%    executed only if the test has been completed successfully.
%    \begin{macrocode}
   {%
%    \end{macrocode}
%    Now we define the new counter. The names of the \LaTeX{}
%    macros employed should speak for themselves:
% \changes{v2.2b}{1995/10/07}{Use \cs{@newctr}}
% \changes{v2.2c}{1995/11/23}{Use \cs{@newctr} properly.}
%    \begin{macrocode}
    \@definecounter{#1}\@newctr{#1}[#3]%
%    \end{macrocode}
%    Using |\@newctr| will give a proper error message if the counter
%    in |#3| is not defined.
%    In defining `|\theLem|' we must generate the desired macro
%    name by use of |\expandafter| and |\csname|.
%    \begin{macrocode}
    \expandafter\xdef\csname the#1\endcsname
%    \end{macrocode}
%    An |\xdef| is used in order to make the definition global,
%    and to ensure that it contains the replacement texts of
%    |\@thmcountersep| and |\@thmcounter|.\footnote{These
%    two macros can be defined by the document style.  Their default
%    values produce a `\texttt{.}' as separation and an arabic
%    representation of the number.} However, not everything should be
%    expanded. For example, it saves space to use |\thesection|
%    instead of its---at times---lengthy expansion.
%    \begin{macrocode}
      {\expandafter \noexpand \csname the#3\endcsname
       \@thmcountersep \@thmcounter{#1}}%
%    \end{macrocode}
%    Thus with the defaults of \LaTeX{}, |\theLem| would be
%    replaced by the command sequence |\thesection.\arabic{Lem}|.
%
%    We will now look at the definition of the macro which is executed
%    at the beginning of the actual environment (in our example this
%    macro is |\Lem|). It should be noted that we use an
%    ``|\expandafter| trick'' to expand only certain parts of the
%    replacement text at the time of the definition.
%    \begin{macrocode}
    \def\@tempa{\global\@namedef{#1}}%
    \expandafter \@tempa \expandafter{%
%    \end{macrocode}
%    First, the macro that contains the current definitions of
%    |\@begintheorem| and |\@opargtheorem| should be called
%    up.  The name of this macro---as is already known---has the form
%    |\th@|\meta{theorem\ style}; therefore, it must be called by
%    \begin{macrocode}
      \csname th@\the \theorem@style
%    \end{macrocode}
%    In addition the default theorem font should be changeable, i.e.\
%    we have to insert the contents of |\theorem@bodyfont|. For
%    that reason, we expand even further, beyond |\endcsname|,
%    and thus insert the contents of the \textsf{token} register
%    |\theorem@bodyfont| in the replacement text.
%    \begin{macrocode}
            \expandafter \endcsname \the \theorem@bodyfont
%    \end{macrocode}
%    Now it is time to call the macro |\@thm| which takes over
%    the further processing.  It has two arguments: the current
%    counter name (in our example, \texttt{Lem}), and the text of the
%    label (in our example, \texttt{Lemma}).
%    \begin{macrocode}
     \@thm{#1}{#2}}%
%    \end{macrocode}
%    With this, the `sub-definition' is complete.  The macro
%    |\@endtheorem| ends a theorem environment and is, so far, nothing
%    but an |\endtrivlist|.  (Hence it is defined globally, and not
%    within the theorem styles.\footnote{This has to be changed as
%    soon as theorem styles that change \texttt{\bslash @endtheorem}
%    exist.  In such a case, all existing styles must be changed as
%    well since they will have to reset the macro.}) Therefore, we can
%    set it equivalent to the macro that ends the theorem set (in our
%    example, |\endLem|).  However, if some day theorem styles exist
%    that do change |\@endtheorem|, we would have to use the
%    commented-out line instead.
%    \begin{macrocode}
    \global \expandafter \let \csname end#1\endcsname \@endtheorem
%   \global\@namedef{end#1}{\@endtheorem}%
%    \end{macrocode}
%    With these commands all the required definitions are employed,
%    unless the test |\@ifdefinable| has failed.  Therefore, we
%    end the second argument of this macro and with it the definition
%    of |\@xnthm|.
%    \begin{macrocode}
   }}
%    \end{macrocode}
% \end{macro}
%
% \begin{macro}{\@ynthm}
%    The definition of |\@ynthm| is completely analogous. In
%    this case the new counter that is defined is not reset within
%    another counter; thus the definition of |\the...|
%    is simplified:
%    \begin{macrocode}
\gdef\@ynthm#1#2{\expandafter\@ifdefinable\csname #1\endcsname
   {\@definecounter{#1}%
    \expandafter\xdef\csname the#1\endcsname{\@thmcounter{#1}}%
%    \end{macrocode}
%    The rest of the definition corresponds literally to that of
%    |\@xnthm|:
%    \begin{macrocode}
    \def\@tempa{\global\@namedef{#1}}\expandafter \@tempa
     \expandafter{\csname th@\the \theorem@style \expandafter
     \endcsname \the\theorem@bodyfont \@thm{#1}{#2}}%
    \global \expandafter \let \csname end#1\endcsname \@endtheorem}}
%    \end{macrocode}
% \end{macro}
%
% \begin{macro}{\@othm}
%    The definition of |\@othm| does not contain anything new.
%    \begin{macrocode}
\gdef\@othm#1[#2]#3{%
%    \end{macrocode}
% \changes{v2.2b}{95/10/07}{Build in check for counter defined}
%    We do not define a new counter but instead use one that has
%    already been defined.  Thus the only definition we need is that
%    of this pseudo-counter (i.e.\ |\the|\meta{env.\ name}).
%    First we check if |#2| corresponds to a known counter name.
%    \begin{macrocode}
  \expandafter\ifx\csname c@#2\endcsname\relax
   \@nocounterr{#2}%
  \else
   \expandafter\@ifdefinable\csname #1\endcsname
   {\expandafter \xdef \csname the#1\endcsname
     {\expandafter \noexpand \csname the#2\endcsname}%
%    \end{macrocode}
%    All other parts of the definition can be adopted from
%    |\@xnthm|. We have to remember, though, that in this case
%    the name of the current counter and the theorem label have moved
%    to the second and third arguments.
%    \begin{macrocode}
    \def\@tempa{\global\@namedef{#1}}\expandafter \@tempa
     \expandafter{\csname th@\the \theorem@style \expandafter
     \endcsname \the\theorem@bodyfont \@thm{#2}{#3}}%
    \global \expandafter \let \csname end#1\endcsname \@endtheorem}%
  \fi}
%    \end{macrocode}
% \end{macro}
%
% \subsection{Macros that are employed in a {\envfont theorem}
%    environment}
%
% \begin{macro}{\@thm}
%    The macro |\@thm| has to increase the current counter.  Then,
%    depending on whether the environment has (or does not have) an
%    optional argument, it has to branch into either |\@begintheorem|
%    or |\@opargtheorem|.
%    \begin{macrocode}
\gdef\@thm#1#2{\refstepcounter{#1}%
%    \end{macrocode}
%    Now we start a {\envfont trivlist} environment, and give
%    |\@topsep| and |\@topsepadd| the values of the \textsf{skip}
%    registers |\theorempreskipamount| and |\theorempostskipamount|.
%    The value in |\@topsep| is the vertical space that is inserted by
%    the first (and only) |\item| in our |\trivlist| whilst
%    |\@topsepadd| is inserted by |\@endparenv| at the end of that
%    {\envfont trivlist} environment.  By using these registers, we
%    obtain the desired space around a {\envfont theorem} environment.
%    \begin{macrocode}
   \trivlist
   \@topsep \theorempreskipamount               % used by first \item
   \@topsepadd \theorempostskipamount           % used by \@endparenv
%    \end{macrocode}
%    Now we have to test whether an optional argument has been given.
% \changes{v2.0d}{89/07/18}{Forgotten percent added.}
%    \begin{macrocode}
   \@ifnextchar [%
%    \end{macrocode}
%    If there is an optional argument, we will call |\@ythm|, and
%    move the arguments read back into the input stream.
%    \begin{macrocode}
   {\@ythm{#1}{#2}}%
%    \end{macrocode}
%    If not, we call |\@begintheorem|. Its first argument is the
%    name of the theorem set (hence the second argument of
%    |\@thm|). Its second argument is the macro that produces the
%    current number.
%    \begin{macrocode}
   {\@begintheorem{#2}{\csname the#1\endcsname}\ignorespaces}}
%    \end{macrocode}
% \end{macro}
%
% \begin{macro}{\@xthm}
% \begin{macro}{\@ythm}
%    Both these macros were originally called by |\@thm|.  We do
%    not need |\@xthm| anymore, hence we reset it to
%    |\relax|. The definition of |\@ythm| has not changed at
%    all from its definition in \LaTeX. In order to make the macros
%    easier to understand, we will
%    nevertheless present it (commented out).
%    \begin{macrocode}
\global\let\@xthm\relax
% \def\@ythm#1#2[#3]{\@opargbegintheorem{#2}{\csname
%           the#1\endcsname}{#3}\ignorespaces}
%</package>
%    \end{macrocode}
%    The primitive |\ignorespaces| in |\@ythm| and
%    |\@thm| is needed to remove the spaces between the
%    |\begin{...}| and the actual text.
% \end{macro}
% \end{macro}
%
%
% \subsection{Definition of the theorem substyles}
%
% As already pointed out, the theorem substyles, defined below, are
% only loaded when necessary.  Note that all these substyles, except
% \texttt{plain}, have |\slshape| as the default body font.
%
% \subsubsection{The \texttt{plain} style}
%
% \label{dok:th@plain} As the following macros use |@|, we have
% to locally set the |\catcode| of this symbol to ``letter''.
% This happens within a group, so that we do not have to worry about
% which |\catcode| that symbol had before.
%    \begin{macrocode}
%<*thp>
\begingroup \makeatletter
%    \end{macrocode}
% Since we are now within a group, we must make all definitions
% globally.  First we make sure that \texttt{theorem.sty} is loaded.
% This will allow us to use this file as a document style option
% without having to call \texttt{theorem} itself as an option.  At the
% same time, we assure that at least version 2 is loaded, since
% |\theorem@style| was not defined in earlier versions.
%    \begin{macrocode}
\@ifundefined{theorem@style}{\input{theorem.sty}}{}
\ProvidesFile{thp.sty}
             [\FMithmInfo]
%    \end{macrocode}
%
%
% \begin{macro}{\th@plain}
% \begin{macro}{\@begintheorem}
% \begin{macro}{\@opargbegintheorem}
%    |\theoremstyle{plain}| corresponds to the original
%    definition, except that the distances to the surrounding text are
%    determined by the parameters |\theorempreskipamount| and
%    |\theorempostskipamount|.
%    First we set the default body font.
%    \begin{macrocode}
\gdef\th@plain{\normalfont\itshape
%    \end{macrocode}
%    Then we define |\@begintheorem| and
%    |\@opargbegintheorem|. These two macros define how the
%    header of a theorem is typeset.  |\@opargbegintheorem| will
%    be called if a \textsf{theorem} environment with an optional
%    argument is encountered; otherwise, the header is constructed by
%    calling |\@begintheorem|.  If one of these macros is
%    executed, we are within a \textsf{trivlist} environment started by
%    |\@thm|.  So the theorem header is produced with an
%    |\item| command.
%
%    Instead of specifying the header font directly, all standard
%    theorem styles use the |\theorem@headerfont| macro to allow
%    customization.  The extra space (|\labelsep|) is necessary
%    because of problems in the \textsf{trivlist} environment.
%    \begin{macrocode}
  \def\@begintheorem##1##2{%
        \item[\hskip\labelsep \theorem@headerfont ##1\ ##2]}%
%    \end{macrocode}
%    The definition of |\@opargbegintheorem| is completely
%    analogous. The only difference is the fact that there exists a
%    third argument (which is the optional parameter of the
%    environment and contains additional information about the
%    theorem).  Customarily we enclose it in parentheses.
%    \begin{macrocode}
\def\@opargbegintheorem##1##2##3{%
   \item[\hskip\labelsep \theorem@headerfont ##1\ ##2\ (##3)]}}
%    \end{macrocode}
% \end{macro}
% \end{macro}
% \end{macro}
%
% \noindent
% We conclude with an |\endgroup| to restore the
% |\catcode| of |@|.
%    \begin{macrocode}
\endgroup
%</thp>
%    \end{macrocode}
%
%
% \subsubsection{The \texttt{break} style}
%
% This style option is stored in the file |thb.sty|.
% For the next two lines see the documentation for |\th@plain| on
% page~\pageref{dok:th@plain}.
%    \begin{macrocode}
%<*thb>
\begingroup \makeatletter
\@ifundefined{theorem@style}{\input{theorem.sty}}{}
\ProvidesFile{thb.sty}
             [\FMithmInfo]
%    \end{macrocode}
%
% \begin{macro}{\th@break}
%    |\theoremstyle{break}| produces a line break after the name
%    of the theorem.  The font is |\slshape|.  Hence, we define
%    |\th@break| as follows:
%    \begin{macrocode}
\gdef\th@break{\normalfont\slshape
  \def\@begintheorem##1##2{\item[%
%    \end{macrocode}
%    We run into the following problem: it is not possible to create
%    the header with |\item[|\meta{title}|]| and then start
%    a new line by, for example, |\mbox{}\\|. Such a definition
%    will fail whenever a list environment follows immediately.  With
%    the above construction, the |\mbox{}| causes the switch
%    |@inlabel| (cf.\ definition of |\list| and
%    |\trivlist| in \cite{latex:source}) to be set to \texttt{false}
%    and so the following list will insert additional vertical
%    space (|\topskip|). This is quite annoying. Therefore, we
%    create the line break within the |\item|. In order to ensure
%    that the text will begin at the proper position in the following
%    line, we simply pretend that the label does not take any
%    room.\footnote{This will lead to problems whenever very high
%    symbols occurring in the line tower into the
%    heading. So, something else has to be done here sometime.}
%    \begin{macrocode}
       \rlap{\vbox{\hbox{\hskip \labelsep\theorem@headerfont ##1\ ##2}%
                   \hbox{\strut}}}]}%
%    \end{macrocode}
%    Again, the definition of |\@opargbegintheorem| is
%    completely analogous.
%    \begin{macrocode}
\def\@opargbegintheorem##1##2##3{%
  \item[\rlap{\vbox{\hbox{\hskip \labelsep \theorem@headerfont
                     ##1\ ##2\ (##3)}%
                    \hbox{\strut}}}]}}
\endgroup
%</thb>
%    \end{macrocode}
% \end{macro}
%
%
% \subsubsection{The \texttt{changebreak} style}
%
%    \begin{macrocode}
% This style option is stored in the file |thcb.sty|.
%    \begin{macrocode}
%<*thcb>
\begingroup \makeatletter
\@ifundefined{theorem@style}{\input{theorem.sty}}{}
\ProvidesFile{thcb.sty}
             [\FMithmInfo]
%    \end{macrocode}
%
% \begin{macro}{\th@changebreak}
%    The \texttt{change-break} theorem style is like \texttt{break}
%    but with interchange of theorem name and theorem number.  Thus we
%    define |\th@changebreak| as follows:
%    \begin{macrocode}
\gdef\th@changebreak{\normalfont\slshape
  \def\@begintheorem##1##2{\item
       [\rlap{\vbox{\hbox{\hskip\labelsep \theorem@headerfont ##2\ ##1}%
                    \hbox{\strut}}}]}%
\def\@opargbegintheorem##1##2##3{%
  \item[\rlap{\vbox{\hbox{\hskip\labelsep \theorem@headerfont
                     ##2\ ##1\ (##3)}%
                    \hbox{\strut}}}]}}
\endgroup
%</thcb>
%    \end{macrocode}
% \end{macro}
%
%
% \subsubsection{The \texttt{change} style}
%
% This style option is stored in the file |thc.sty|.
%    \begin{macrocode}
%<*thc>
\begingroup \makeatletter
\@ifundefined{theorem@style}{\input{theorem.sty}}{}
\ProvidesFile{thc.sty}
             [\FMithmInfo]
%    \end{macrocode}
%
% \begin{macro}{\th@change}
%    The \texttt{change} theorem style corresponds to the
%    \texttt{change break} style without a linebreak after the header.
%    To say it in another way, it's the same as the \texttt{plain}
%    style but with number and name interchanged and |\slshape| as the
%    default font.
%    \begin{macrocode}
\gdef\th@change{\normalfont\slshape
  \def\@begintheorem##1##2{\item
       [\hskip\labelsep \theorem@headerfont ##2\ ##1]}%
\def\@opargbegintheorem##1##2##3{%
  \item[\hskip\labelsep \theorem@headerfont ##2\ ##1\ (##3)]}}
\endgroup
%</thc>
%    \end{macrocode}
% \end{macro}
%
%
% \subsubsection{The \texttt{marginbreak} style}
%
% This style option is the one used most often at Mainz.  It is saved
% in the file |thmb.sty|.
%    \begin{macrocode}
%<*thmb>
\begingroup \makeatletter
\@ifundefined{theorem@style}{\input{theorem.sty}}{}
\ProvidesFile{thmb.sty}
             [\FMithmInfo]
%    \end{macrocode}
%
% \begin{macro}{\th@marginbreak}
%    The \texttt{margin break} style is nearly the same as the
%    \texttt{change break} style. The only difference is the placement
%    of the theorem number.  We use |\llap| to place it in the left
%    margin.
%
%    In this style |\labelsep| denotes the separation between the
%    number and the text.
%    \begin{macrocode}
\gdef\th@marginbreak{\normalfont\slshape
  \def\@begintheorem##1##2{\item
       [\rlap{\vbox{\theorem@headerfont
                    \hbox{\llap{##2}\hskip\labelsep ##1}%
                    \hbox{\strut}}}]}%
\def\@opargbegintheorem##1##2##3{%
  \item[\rlap{\vbox{\theorem@headerfont
                    \hbox{\llap{##2}\hskip\labelsep ##1\ (##3)}%
                    \hbox{\strut}}}]}}
\endgroup
%</thmb>
%    \end{macrocode}
% \changes{v2.1c}{90/02/17}{`endgroup outside macrocode env.}
% \end{macro}
%
%
%
% \subsubsection{The \texttt{margin} style}
%
% This style option is stored in the file |thm.sty|.
%    \begin{macrocode}
%<*thm>
\begingroup \makeatletter
\@ifundefined{theorem@style}{\input{theorem.sty}}{}
\ProvidesFile{thm.sty}
             [\FMithmInfo]
%    \end{macrocode}
%
% \begin{macro}{\th@margin}
%    Again this is only a variant of the theorem styles described
%    above without any new ideas.
%    \begin{macrocode}
\gdef\th@margin{\normalfont\slshape
  \def\@begintheorem##1##2{\item
       [\theorem@headerfont \llap{##2}\hskip\labelsep ##1]}%
\def\@opargbegintheorem##1##2##3{%
  \item[\theorem@headerfont \llap{##2}\hskip\labelsep ##1\ (##3)]}}
\endgroup
%</thm>
%    \end{macrocode}
% \end{macro}
%
%
%
%
% \subsection{Final Definitions}
%
% \begin{macro}{\theorempreskipamount}
% \begin{macro}{\theorempostskipamount}
%    The \textsf{skip} parameters that regulate the vertical empty space
%    before and after the {\envfont theorem} environment have
%    to be allocated as well.
% \changes{v2.0f}{89/07/20}{Unnecessary `global' removed.}
%    \begin{macrocode}
%<*package>
\newskip\theorempreskipamount
\newskip\theorempostskipamount
%    \end{macrocode}
%    Since we have used the same values for all theorem sets, we now can
%    assign them.
%    \begin{macrocode}
\global\setlength\theorempreskipamount{12pt plus 5pt minus 3pt}
\global\setlength\theorempostskipamount{8pt plus 3pt minus 1.5pt}
%    \end{macrocode}
% \end{macro}
% \end{macro}
%
%
% \begin{macro}{\@endtheorem}
%    The same holds for the macro |\@endtheorem|, which ends a
%    {\envfont theorem} environment. Since it is the same for all
%    theorem sets, it is removed from the macros |\th@|\meta{style}.
%    It simply ends the {\envfont trivlist} environment, which was
%    begun in |\@thm|.
%    \begin{macrocode}
\global\let\@endtheorem=\endtrivlist
%    \end{macrocode}
% \end{macro}
%
%
%
% \begin{macro}{\@preamblecmds}
%    All macros defined above are to be used only in the preamble.
%    Therefore, we insert them in |\@preamblecmds| which will disable
%    them at begin document. This is done by the internal \LaTeXe{}
%    command |\@onlypreamble|.
% \changes{v2.0g}{89/09/19}{`theoremheaderfont and `theorembodyfont
%    added.}
% \changes{v2.2a}{94/02/03}{Use \LaTeXe{} interface}
%    \begin{macrocode}
\@onlypreamble\@xnthm
\@onlypreamble\@ynthm
\@onlypreamble\@othm
\@onlypreamble\newtheorem
\@onlypreamble\theoremstyle
\@onlypreamble\theorembodyfont
\@onlypreamble\theoremheaderfont
%    \end{macrocode}
% \end{macro}
%
% Finally we declare the \texttt{plain} theorem style to be the default.
% \changes{v2.1b}{90/06/04}{Fixed missing `item by declaring default
%    style.}
%    \begin{macrocode}
\theoremstyle{plain}
%</package>
%    \end{macrocode}
%
% \Finale

\endinput
}{}
\ProvidesFile{thm.sty}
             [\FMithmInfo]
%    \end{macrocode}
%
% \begin{macro}{\th@margin}
%    Again this is only a variant of the theorem styles described
%    above without any new ideas.
%    \begin{macrocode}
\gdef\th@margin{\normalfont\slshape
  \def\@begintheorem##1##2{\item
       [\theorem@headerfont \llap{##2}\hskip\labelsep ##1]}%
\def\@opargbegintheorem##1##2##3{%
  \item[\theorem@headerfont \llap{##2}\hskip\labelsep ##1\ (##3)]}}
\endgroup
%</thm>
%    \end{macrocode}
% \end{macro}
%
%
%
%
% \subsection{Final Definitions}
%
% \begin{macro}{\theorempreskipamount}
% \begin{macro}{\theorempostskipamount}
%    The \textsf{skip} parameters that regulate the vertical empty space
%    before and after the {\envfont theorem} environment have
%    to be allocated as well.
% \changes{v2.0f}{89/07/20}{Unnecessary `global' removed.}
%    \begin{macrocode}
%<*package>
\newskip\theorempreskipamount
\newskip\theorempostskipamount
%    \end{macrocode}
%    Since we have used the same values for all theorem sets, we now can
%    assign them.
%    \begin{macrocode}
\global\setlength\theorempreskipamount{12pt plus 5pt minus 3pt}
\global\setlength\theorempostskipamount{8pt plus 3pt minus 1.5pt}
%    \end{macrocode}
% \end{macro}
% \end{macro}
%
%
% \begin{macro}{\@endtheorem}
%    The same holds for the macro |\@endtheorem|, which ends a
%    {\envfont theorem} environment. Since it is the same for all
%    theorem sets, it is removed from the macros |\th@|\meta{style}.
%    It simply ends the {\envfont trivlist} environment, which was
%    begun in |\@thm|.
%    \begin{macrocode}
\global\let\@endtheorem=\endtrivlist
%    \end{macrocode}
% \end{macro}
%
%
%
% \begin{macro}{\@preamblecmds}
%    All macros defined above are to be used only in the preamble.
%    Therefore, we insert them in |\@preamblecmds| which will disable
%    them at begin document. This is done by the internal \LaTeXe{}
%    command |\@onlypreamble|.
% \changes{v2.0g}{89/09/19}{`theoremheaderfont and `theorembodyfont
%    added.}
% \changes{v2.2a}{94/02/03}{Use \LaTeXe{} interface}
%    \begin{macrocode}
\@onlypreamble\@xnthm
\@onlypreamble\@ynthm
\@onlypreamble\@othm
\@onlypreamble\newtheorem
\@onlypreamble\theoremstyle
\@onlypreamble\theorembodyfont
\@onlypreamble\theoremheaderfont
%    \end{macrocode}
% \end{macro}
%
% Finally we declare the \texttt{plain} theorem style to be the default.
% \changes{v2.1b}{90/06/04}{Fixed missing `item by declaring default
%    style.}
%    \begin{macrocode}
\theoremstyle{plain}
%</package>
%    \end{macrocode}
%
% \Finale

\endinput
}{}
\ProvidesFile{thc.sty}
             [\FMithmInfo]
%    \end{macrocode}
%
% \begin{macro}{\th@change}
%    The \texttt{change} theorem style corresponds to the
%    \texttt{change break} style without a linebreak after the header.
%    To say it in another way, it's the same as the \texttt{plain}
%    style but with number and name interchanged and |\slshape| as the
%    default font.
%    \begin{macrocode}
\gdef\th@change{\normalfont\slshape
  \def\@begintheorem##1##2{\item
       [\hskip\labelsep \theorem@headerfont ##2\ ##1]}%
\def\@opargbegintheorem##1##2##3{%
  \item[\hskip\labelsep \theorem@headerfont ##2\ ##1\ (##3)]}}
\endgroup
%</thc>
%    \end{macrocode}
% \end{macro}
%
%
% \subsubsection{The \texttt{marginbreak} style}
%
% This style option is the one used most often at Mainz.  It is saved
% in the file |thmb.sty|.
%    \begin{macrocode}
%<*thmb>
\begingroup \makeatletter
\@ifundefined{theorem@style}{% \iffalse meta-comment
%
% Copyright (C) 1993-2024
%
% The LaTeX Project and any individual authors listed elsewhere
% in this file.
%
% This file is part of the Standard LaTeX `Tools Bundle'.
% -------------------------------------------------------
%
% It may be distributed and/or modified under the
% conditions of the LaTeX Project Public License, either version 1.3c
% of this license or (at your option) any later version.
% The latest version of this license is in
%    https://www.latex-project.org/lppl.txt
% and version 1.3c or later is part of all distributions of LaTeX
% version 2005/12/01 or later.
%
% The list of all files belonging to the LaTeX `Tools Bundle' is
% given in the file `manifest.txt'.
%
% \fi
%
%%
%
%
\def\FMithmInfo{2023/07/05 v2.2c Theorem extension package (FMi)}
%
% \ProvidesFile{theorem.dtx}[\FMithmInfo]
%\iffalse   % this is a METACOMMENT !
%%
%% Package `theorem' to use with LaTeX2e
%% Copyright (C) 1989-1995 Frank Mittelbach, all rights reserved.
%<+package>\@ifundefined{theorem@style}{}{\endinput}
%<+package>\NeedsTeXFormat{LaTeX2e}[1995/06/01]
%<+package>\ProvidesPackage{theorem}
%<+package>         [\FMithmInfo]
%
% \fi
%
% \changes{v1.0}{87/05/07}{First draft.}
% \changes{v1.0a}{87/05/07}{`thistheoremstyle' added.}
% \changes{v1.1a}{87/07/29}{New concept: redefine the internal
%                          \protect\LaTeX{} commands `@xnthm' etc.}
% \changes{v1.1b}{87/07/31}{`theoremfont' used to specify font.}
% \changes{v1.2a}{88/01/08}{`@endtheorem' now globally defined.}
% \changes{v1.2b}{88/01/11}{`newtheorem' only allowed in the preamble.}
% \changes{v1.3a}{88/02/14}{`theoremfont' now a toks register.}
%
% \changes{v2.0a}{88/11/16}{New implementation and new documentation.}
% \changes{v2.0a}{88/11/16}{No layout changes but internal modifications
%                           old substyles won't work.}
% \changes{v2.0b}{89/01/30}{`global' used for all definitions.}
% \changes{v2.0c}{89/06/15}{Documentation translated to English.
%                            Thanks to LLL and Nina}
% \changes{v2.1a}{90/02/15}{Support for new font selection.}
% \changes{v2.2a}{94/02/03}{Cleanup for LaTeX2e}
%
% \DoNotIndex{\@endparenv,\@ifdefinable,\@ifnextchar,\@ifundefined}
% \DoNotIndex{\@namedef,\@nameuse,\@spaces,\@tempa,\@warning}
% \DoNotIndex{\ }
% \DoNotIndex{\begingroup,\bfseries}
% \DoNotIndex{\csname}
% \DoNotIndex{\def,\do,\docdate}
% \DoNotIndex{\endcsname,\endgroup,\endinput,\endtrivlist,\expandafter}
% \DoNotIndex{\filedate,\fileversion}
% \DoNotIndex{\gdef,\global}
% \DoNotIndex{\hbox,\hskip}
% \DoNotIndex{\ignorespaces,\input,\itshape,\item}
% \DoNotIndex{\labelsep,\let,\llap}
% \DoNotIndex{\makeatletter}
% \DoNotIndex{\noexpand}
% \DoNotIndex{\refstepcounter,\relax,\rlap}
% \DoNotIndex{\setlength,\slshape,\space,\string,\strut}
% \DoNotIndex{\the,\trivlist,\typeout}
% \DoNotIndex{\vbox}
% \DoNotIndex{\xdef}
%
% \def\deflabel#1{\ttfamily #1\hfill}
% \def\deflist#1{\begin{list}{}{\settowidth\labelwidth{\ttfamily #1}%
%                               \setlength\leftmargin\labelwidth
%                               \addtolength\leftmargin\labelsep
%                               \let\makelabel\deflabel}}
% \def\enddeflist{\end{list}}
% \def\envfont{\normalfont\ttfamily}
%
% \MakeShortVerb{\|}
%
% \newcommand{\TUB}{{\sl TUGboat\/}}
%
% \GetFileInfo{theorem.dtx}
% \title{An Extension of the
%        \LaTeX{} theorem environment\thanks{This file
%        has version number \fileversion, last
%        revised \filedate.}}
% \author{Frank Mittelbach}
%
% \MaintainedByLaTeXTeam{tools}
% \maketitle
%
% \begin{abstract}
%    The macros described in this paper yield an extension of the
%    \LaTeX{} theorem mechanism. It is designed to satisfy the
%    different requirements of various journals. Thus, the
%    layout of the ``theorems'' can be manipulated by determining a
%    ``style''. This article describes not only the use, but
%    also the definition, of the necessary macros.
% \end{abstract}
%
% \section*{Preface to version 2.2}
%
% For \LaTeXe{} this package did not need any fundamental changes. I
% only modified the messages generated so that theorem layout styles
% will show up with the |\listfiles| command and cleaned the section
% on the New Font Selection Scheme since this is now included in
% \LaTeX.
%
% \section*{Preface to version 2.1}
%
% This version is identical to 2.0g described in \TUB\ 10\#3 except
% for some internal defaults which are now set depending on the used
% font selection scheme.
%
% This was done to avoid unpleasant surprises if the new font selection
% scheme is in force. For further details see
% section~\ref{sec:newfonts} and \cite{bk:GMS94}.
%
% \section{Introduction}
%
% For our purposes here, ``theorems'' are labelled enunciations,
% often set off from the main text by extra space and a font change.
% Theorems, corollaries, conjectures, definitions, and remarks
% are all instances of ``theorems''.  The ``header'' of these structures
% is composed of a label (such as \textsc{Theorem} or \textsc{Remark})
% and a number which serializes an item in the sequence of items
% with the same label.
%
% Shortly after the introduction of \LaTeX{} at the Fachbereich
% Mathematik in Mainz, the desire to manipulate the layout of
% ``theorems'' arose.  In Mainz, the following two conventions came
% into general use:
% \begin{enumerate}
%   \item The number of the theorem is shown in the margin.
%   \item There is a line break at
%         the end of the theorem header.
% \end{enumerate}
%
% Additionally, some journals require different formats which depend on
% the ``sort of theorem'': e.g.\ often remarks and definitions are set
% in |\upshape|, while |\itshape| is employed for main theorems.
%
% Confronted with these requirements, a theorem environment was
% developed in Mainz which allows separate determination of the
% layout of the ``theorems sets'', comparable to |\pagestyle|.
%
%
% \section{The user interface}
% \subsection{Defining new theorem sets}
%
% \DescribeMacro\newtheorem
% As in the original \LaTeX{} version, the command |\newtheorem|
% defines a new ``theorem set'' or ``theorem-like structure''.
% Two required arguments name the new environment and give the text
% to be typeset with each instance of the new ``set'', while
% an optional argument determines how
% the ``set'' is enumerated:
% \begin{description}
%    \item[\ttfamily \bslash newtheorem\{foo\}\{bar\}]
%       The theorem set {\envfont foo} (whose name is \texttt{bar})
%       uses its own counter.
%    \item[\ttfamily \bslash newtheorem\{foo2\}{[foo]}\{bar2\}]
%       The theorem set {\envfont foo2} (printed name \texttt{bar2})
%       uses the same counter as the theorem set \texttt{foo}.
%    \item[\ttfamily \bslash newtheorem\{foo3\}\{bar3\}{[section]}]
%       The theorem set {\envfont foo3} (printed name \texttt{bar3}) is
%       enumerated within the counter \texttt{section}, i.e.\ with every
%       new |\section| the enumeration begins again with 1, and
%       the enumeration is composed from the section-number and the
%       theorem counter itself.
% \end{description}
%
% \DescribeMacro\theoremstyle
% Additionally, the command |\theoremstyle| can define the layout
% of various, or all, theorem sets.  It should be noted that any
% theorem set defined by |\newtheorem| is typeset in the
% |\theoremstyle| that is current at the time of the definition.
% Thus, the following
% \begin{verbatim}
%    \theoremstyle{break}        \newtheorem{Cor}{Corollary}
%    \theoremstyle{plain}        \newtheorem{Exa}{Example}[section]
%\end{verbatim}                             ^^A No blank at line start
% leads to the result that the set {\envfont Cor} is formatted in the
% style \texttt{break}, while the set {\envfont Exa} and all the
% following ones are formatted in the style \texttt{plain}, unless
% another |\theoremstyle| follows.  Since the definitions installed by
% |\newtheorem| are global, one also can limit |\theoremstyle| locally
% by grouping braces.
%
% \DescribeMacro\theorembodyfont
% The choice of the font for the theorem body is completely
% independent of the chosen |\theoremstyle|; this has proven to
% be very advantageous.  For example,
% \begin{verbatim}
%   {\theorembodyfont{\upshape}          \newtheorem{Rem}{Remark}}
%\end{verbatim}
% defines a theorem set {\envfont Rem}, which will be set in
% |\upshape| in the current layout (which in our example is
% \texttt{plain}).  As with |\theoremstyle|, the |\theorembodyfont|
% chosen is that current at the time of |\newtheorem|.  If
% |\theorembodyfont| is not specified or one defines
% |\theorembodyfont{}|, then the font used will be that defined by the
% |\theoremstyle|.
%
% \DescribeMacro\theoremheaderfont It is also possible to customize
% the font used for the theorem headers.  This is, however, a global
% declaration, and therefore there should be at most one
% |\theoremheaderfont| declaration in the preamble.\footnote{If it is
% actually necessary to have different header fonts, one has to define
% new theorem styles (substituting the desired font) or specify the
% information directly in the \texttt{\bslash newtheorem} declaration
% (the unclean variant).}
%
% \DescribeMacro\theorempreskipamount
% \DescribeMacro\theorempostskipamount Two additional parameters
% affect the vertical space around the theorem environments:
% |\theorempreskipamount| and |\theorempostskipamount| define,
% respectively, the spacing before and after such an environment.
% These parameters apply for all theorem sets and can be manipulated
% with the ordinary length macros.  They are rubber lengths
% (`\textsf{skips}'), and therefore can contain \texttt{plus} and
% \texttt{minus} parts.
%
% Since the definition of theorem sets should---most sensibly---be
% placed in the preamble, we only allow installation there.
% It is therefore possible to release the memory used here after
% |\begin{document}|, in order to make room for other
% applications.
%
%
% \subsection{Existing theorem styles}
%
% The following theorem styles exist to date:
% \begin{deflist}{marginbreak}
%    \item[plain]
%       This theorem style emulates the original \LaTeX{} definition,
%       except that additionally the parameters
%       |\theorem...skipamount| are used.
%    \item[break]
%       In this style, the theorem header is followed by a line
%       break.
%    \item[marginbreak]
%       The theorem number is set in the margin, and there is a line
%       break as in \texttt{break}.
%    \item[changebreak]
%       Like \texttt{break}, but with header number and text
%       interchanged.
%    \item[change]
%       Header number and text are interchanged, without a line break.
%    \item[margin]
%       The number is set in the left margin, without a line break.
% \end{deflist}
% All styles (except \texttt{plain}) select |\slshape| as the default
% |\theorembodyfont|.
%
%
% \subsection{Examples}
%
% Given the above theorem sets {\envfont Cor}, {\envfont Exa}
% and {\envfont Rem},
% suppose that the preamble also contains the declarations:
% \begin{verbatim}
%    \theoremstyle{marginbreak}   \newtheorem{Lem}[Cor]{Lemma}
%    \theoremstyle{change}
%    \theorembodyfont{\itshape}        \newtheorem{Def}[Cor]{Definition}
%
%    \theoremheaderfont{\scshape}
%\end{verbatim}
% Then the following are some typical examples of the typeset output
% resulting from their use.
%
% \begin{Cor}
%    This is a sentence typeset in the theorem environment {\envfont
%    Cor}.
% \end{Cor}
%
% \begin{Exa}
%    This is a sentence typeset in the theorem environment {\envfont
%    Exa}.
% \end{Exa}
%
% \begin{Rem}
%    This is a sentence typeset in the theorem environment {\envfont
%    Rem}.
% \end{Rem}
%
% \begin{Lem}[Ben User]
%    This is a sentence typeset in the theorem environment {\envfont
%    Lem}.
% \end{Lem}
%
% \begin{Def}[Very impressive Definition]
%    This is a sentence typeset in the theorem environment {\envfont
%    Def}.
% \end{Def}
%
% The last two examples show the effect of the optional argument to
% a theorem environment (it is the text typeset in parentheses).
%
%
% \section{Special Considerations}
% \label{sec:newfonts}
%
% Theoremheader and body are implemented as a unit. This means that
% the |\theoremheaderfont|  will inherit characteristics of the
% |\theorembodyfont| in \LaTeXe. Thus, if for example
% |\theorembodyfont| is |\itshape| and |\theoremheaderfont|
% is |\bfseries| the font selected for the header will have the
% characteristics `bold extended italic'. If this is not desired one
% should set the |\theoremheaderfont| to something like
% \begin{verbatim}
%      \theoremheaderfont{\normalfont\bfseries}
%\end{verbatim}
% i.e.\ supplying all necessary font information explicitly.
%
%
%
% \section{Acknowledgements}
%
% The publication of this set of macros was only possible with the
% help of Christina Busse (translating the manuscript into English),
% Joachim Pense (playing the r\^ole of typist), Chris Rowley
% (looking everything over) and many others providing useful
% suggestions.
%
% \MaybeStop{%
%    \begin{thebibliography}{9}
%       \bibitem{bk:GMS94} \textsc{M.~Goossens}, \textsc{F.~Mittelbach}
%         and \textsc{A.~Samarin}.
%         \newblock The \LaTeX{} Companion.
%         \newblock
%         Addison-Wesley, Reading, Massachusetts, 1994.
%       \bibitem{latex:source} \textsc{Lamport, Leslie.} \newblock
%                               \texttt{latex.tex}, version 2.09, date
%                               Feb. 1990.
%    \end{thebibliography}
%    \PrintIndex
%    \PrintChanges}
%
%
% \section{The documentation driver file}
%
% The next bit of code contains the documentation driver file for
% \TeX{}, i.e., the file that will produce the documentation you are
% currently reading. It will be extracted from this file by the
% \texttt{docstrip} program. Since it is the first code in the file
% one can alternatively process this file directly with \LaTeXe{} to
% obtain the documentation.
%
%    \begin{macrocode}
%<*driver>
\documentclass{ltxdoc}

\usepackage{theorem}

%<+driver>% The next few lines define theorem sets which are used
%<+driver>% in the example section of the documentation.

\theoremstyle{break}       \newtheorem{Cor}{Corollary}
\theoremstyle{plain}       \newtheorem{Exa}{Example}[section]
{\theorembodyfont{\upshape}\newtheorem{Rem}{Remark}}
\theoremstyle{marginbreak} \newtheorem{Lem}[Cor]{Lemma}
\theoremstyle{change}
\theorembodyfont{\itshape} \newtheorem{Def}[Cor]{Definition}

\theoremheaderfont{\scshape}

\RecordChanges

\begin{document}
   \DocInput{theorem.dtx}
\end{document}
%</driver>
%    \end{macrocode}
%
% \section{Definition of the Macros}
%
% If the file has been loaded before, we abort immediately. If not the
% package announces itself (this is actually done at the very top of
% the file---the way it is done isn't good style so don't copy it).
% \changes{v2.0e}{89/07/19}{Spaces removed from `typeout'.}
% \changes{v2.2b}{95/11/19}{Announce moved to top of file}
%    \begin{macrocode}
%<*package>
%\@ifundefined{theorem@style}{}{\endinput}
%\def\FMithmInfo{1995/11/19 v2.2b Theorem extension package (FMi)}
%\ProvidesPackage{theorem}[\FMithmInfo]
%    \end{macrocode}
%
%
% \subsection{Definition of theorem styles and fonts}
%
% All the definitions in this file are done globally to allow
% inputting this file inside a group.
%
% \begin{macro}{\theoremstyle}
%    Before a theorem style can be installed, the chosen style must be
%    known. For that reason, we must test to see that
%    |\th@|\meta{style} is known or, more precisely, that it is
%    different from |\relax|. If the style is not known then
%    |\th@plain| is used.
%    \begin{macrocode}
\gdef\theoremstyle#1{%
   \@ifundefined{th@#1}{\@warning
          {Unknown theoremstyle `#1'. Using `plain'}%
          \theorem@style{plain}}%
%    \end{macrocode}
%    We save the theorem style to be used in the token register
%    |\theorem@style|.
%    \begin{macrocode}
      {\theorem@style{#1}}%
%    \end{macrocode}
%    Now we ``evaluate'' the theorem style: this means, we call the
%    macro |\th@|\meta{style} which will activate the relevant
%    definitions which are contained in a separate file.  This is done
%    in a group to suppress changes to the current font.  This could
%    otherwise pose problems together with the new font selection
%    scheme\footnote{When I printed the original article using the new
%    font selection scheme I ended with a document with slanted
%    typefaces (text headings and all) simply because one of the
%    theorem styles used \texttt{\bslash sl} at toplevel.} if the
%    |\th@|\meta{style} is evaluated a second time.
% \changes{v2.1a}{90/02/15}{Group added around `th@..}
% \changes{v2.1a}{90/02/15}{Reseting of `..@begintheorem removed.}
%    \begin{macrocode}
      \begingroup
        \csname th@\the\theorem@style \endcsname
      \endgroup}
%    \end{macrocode}
% \end{macro}
%
%
% \begin{macro}{\@begintheorem}
% \changes{v2.1a}{90/02/15}{Relaxed at toplevel.}
% \begin{macro}{\@opargbegintheorem}
% \changes{v2.1a}{90/02/15}{Relaxed at toplevel.}
%    We reset |\@begintheorem| and |\@opargbegintheorem| to
%    |\relax| since these commands are no longer necessary at
%    toplevel. This will save a few tokens.
%    \begin{macrocode}
\global\let\@begintheorem\relax
\global\let\@opargbegintheorem\relax
%    \end{macrocode}
% \end{macro}
% \end{macro}
%
% \begin{macro}{\theorem@style}
%    Obviously the \textsf{token} register used above has to be
%    allocated. To assure the utmost compatibility with the
%    original \LaTeX{} definition, we set the default theorem style to
%    \texttt{plain}, which implements the usual \LaTeX{} convention.
% \changes{v2.0f}{89/07/20}{Unnecessary `global' removed.}
%    \begin{macrocode}
\newtoks\theorem@style
\global\theorem@style{plain}
%    \end{macrocode}
% \end{macro}
%
% \begin{macro}{\theorembodyfont}
% \begin{macro}{\theorem@bodyfont}
%    If the |\theorembodyfont| is set by the user then it should not
%    interact with the default font set in the theorem style. When the
%    new font selection is in force this may happen if, for example,
%    the default is |\itshape| and the new |\theorembodyfont| is
%    |\sffamily|. So we add a |\reset@font| command in front of the
%    user definition.
% \changes{v2.0f}{89/07/20}{Unnecessary `global' removed.}
% \changes{v2.0g}{89/09/19}{`theoremfont to `theorembodyfont.}
% \changes{v2.1c}{92/06/22}{`reset@font added and logic changed.}
% \changes{v2.1d}{92/10/28}{Reset to empty if arg is empty}
%    \begin{macrocode}
\gdef\theorembodyfont#1{%
%    \end{macrocode}
%    We check if the argument supplied is empty and if so put nothing
%    into the |\theorem@bodyfont| token register to allow for
%    |\theorembodyfont{}| as a mean of using the default of the
%    current |\theoremstyle|.
%    \begin{macrocode}
   \def\@tempa{#1}%
   \ifx\@tempa\@empty
    \theorem@bodyfont{}%
   \else
    \theorem@bodyfont{\reset@font#1}%
   \fi
   }
\newtoks\theorem@bodyfont
\global\theorem@bodyfont{}
%    \end{macrocode}
% \end{macro}
% \end{macro}
%
%
% \begin{macro}{\theoremheaderfont}
% \changes{v2.0g}{89/09/19}{Macro added.}
%    The font for the theorem headers is handled differently because
%    this definition applies to all theorem styles.
%    \begin{macrocode}
\gdef\theoremheaderfont#1{\gdef\theorem@headerfont{#1}%
%    \end{macrocode}
%    After using the macro once it is redefined to produce an error
%    message.
%    \begin{macrocode}
       \gdef\theoremheaderfont##1{%
        \typeout{\string\theoremheaderfont\space should be used
                 only once.}}}
%    \end{macrocode}
% \end{macro}
%
% \begin{macro}{\theorem@headerfont}
% \changes{v2.0g}{89/09/19}{Macro added.}
%    To set the |\theorem@headerfont| default we first test if
%    the new fontselection scheme is in force.
% \changes{v2.1a}{90/02/15}{Support for new font selection.}
%    \begin{macrocode}
\ifx\upshape\undefined
%    \end{macrocode}
%    If not we define it to expand into |\bfseries|. We don't use
%    |\let| just in case a following style option redefines this
%    macro.
%    \begin{macrocode}
\gdef\theorem@headerfont{\bfseries}
%    \end{macrocode}
%    Otherwise we reset the current shape before calling |\bfseries|.
%    \begin{macrocode}
\else \gdef\theorem@headerfont{\normalfont\bfseries}\fi
%    \end{macrocode}
% \end{macro}
%
% \begin{macro}{\th@plain}
% \begin{macro}{\th@break}
% \begin{macro}{\th@marginbreak}
% \begin{macro}{\th@changebreak}
% \begin{macro}{\th@change}
% \begin{macro}{\th@margin}
%    The different styles are defined in macros such as
%    |\th@plain|.  Since memory space is precious in
%    ``non-Big-versions'', we have to avoid offering too many unused
%    definitions.  Therefore we define these styles in separate files
%    that can be loaded on demand.  Thus the commands themselves only
%    load these files. We use |\@input@|, a \LaTeXe{} internal command
%    that ensures that the file will be listed with |\listfiles|.
% \changes{v2.2a}{94/02/02}{Use `@input@ to load theorem layout files}
%    \begin{macrocode}
\gdef\th@plain{\@input@{thp.sty}}
\gdef\th@break{\@input@{thb.sty}}
\gdef\th@marginbreak{\@input@{thmb.sty}}
\gdef\th@changebreak{\@input@{thcb.sty}}
\gdef\th@change{\@input@{thc.sty}}
\gdef\th@margin{\@input@{thm.sty}}
%    \end{macrocode}
%    This list will be expanded when new styles become
%    available.  For testing,  just append new theorem substyles
%    as document options.
% \end{macro}
% \end{macro}
% \end{macro}
% \end{macro}
% \end{macro}
% \end{macro}
%
%
%
% \subsection{Definition of a new theorem set}
%
% As already pointed out, a new theorem environment can be defined in
% three different ways:
% \begin{verbatim}
%    \newtheorem{Lem}{Lemma}
%    \newtheorem{Lem}{Lemma}[section]
%    \newtheorem{Lem}[Theorem]{Lemma}
%\end{verbatim}
% The function of the macro |\newtheorem| is to recognize these cases
% and then to branch into one of the three macros |\@ynthm|, |\@xnthm|
% or |\@othm|.  This mechanism is adopted unchanged from
% \cite{latex:source}; the essential point here is that, for example,
% in the second case, the arguments \texttt{Lem}, \texttt{Lemma} and
% \texttt{section} are passed over to the macro |\@xnthm|.
%
% We inspect this case first because the others present fewer problems,
% and thus are easily derived from this one.
%
% \begin{macro}{\@xnthm}
%    For our example arguments, the macro |\@xnthm| must fulfill
%    the following:
%    \begin{itemize}
%       \item
%          Define a new \LaTeX-counter `Lem'
%       \item
%          reset this counter within a |\section|
%       \item
%          define the macro |\theLem|
%       \item
%          define the environment macros |\Lem| and
%          |\endLem| using the current
%          |\theoremstyle| and |\theorem@bodyfont|.
%    \end{itemize}
%    Obviously, all this should happen only if the first argument
%    of |\@xnthm| (i.e.\ \texttt{Lem} in our example) is chosen so
%    as not to conflict with any previously defined commands or
%    environments.  This test is performed by the \LaTeX{} macro
%    |\@ifdefinable|.
%    \begin{macrocode}
\gdef\@xnthm#1#2[#3]{\expandafter\@ifdefinable\csname #1\endcsname
%    \end{macrocode}
%    Therefore, the first argument of |\@ifdefinable| is the
%    expansion (in the example, |\Lem|) of
%    |\csname||#1\endcsname|.  The second argument is
%    executed only if the test has been completed successfully.
%    \begin{macrocode}
   {%
%    \end{macrocode}
%    Now we define the new counter. The names of the \LaTeX{}
%    macros employed should speak for themselves:
% \changes{v2.2b}{1995/10/07}{Use \cs{@newctr}}
% \changes{v2.2c}{1995/11/23}{Use \cs{@newctr} properly.}
%    \begin{macrocode}
    \@definecounter{#1}\@newctr{#1}[#3]%
%    \end{macrocode}
%    Using |\@newctr| will give a proper error message if the counter
%    in |#3| is not defined.
%    In defining `|\theLem|' we must generate the desired macro
%    name by use of |\expandafter| and |\csname|.
%    \begin{macrocode}
    \expandafter\xdef\csname the#1\endcsname
%    \end{macrocode}
%    An |\xdef| is used in order to make the definition global,
%    and to ensure that it contains the replacement texts of
%    |\@thmcountersep| and |\@thmcounter|.\footnote{These
%    two macros can be defined by the document style.  Their default
%    values produce a `\texttt{.}' as separation and an arabic
%    representation of the number.} However, not everything should be
%    expanded. For example, it saves space to use |\thesection|
%    instead of its---at times---lengthy expansion.
%    \begin{macrocode}
      {\expandafter \noexpand \csname the#3\endcsname
       \@thmcountersep \@thmcounter{#1}}%
%    \end{macrocode}
%    Thus with the defaults of \LaTeX{}, |\theLem| would be
%    replaced by the command sequence |\thesection.\arabic{Lem}|.
%
%    We will now look at the definition of the macro which is executed
%    at the beginning of the actual environment (in our example this
%    macro is |\Lem|). It should be noted that we use an
%    ``|\expandafter| trick'' to expand only certain parts of the
%    replacement text at the time of the definition.
%    \begin{macrocode}
    \def\@tempa{\global\@namedef{#1}}%
    \expandafter \@tempa \expandafter{%
%    \end{macrocode}
%    First, the macro that contains the current definitions of
%    |\@begintheorem| and |\@opargtheorem| should be called
%    up.  The name of this macro---as is already known---has the form
%    |\th@|\meta{theorem\ style}; therefore, it must be called by
%    \begin{macrocode}
      \csname th@\the \theorem@style
%    \end{macrocode}
%    In addition the default theorem font should be changeable, i.e.\
%    we have to insert the contents of |\theorem@bodyfont|. For
%    that reason, we expand even further, beyond |\endcsname|,
%    and thus insert the contents of the \textsf{token} register
%    |\theorem@bodyfont| in the replacement text.
%    \begin{macrocode}
            \expandafter \endcsname \the \theorem@bodyfont
%    \end{macrocode}
%    Now it is time to call the macro |\@thm| which takes over
%    the further processing.  It has two arguments: the current
%    counter name (in our example, \texttt{Lem}), and the text of the
%    label (in our example, \texttt{Lemma}).
%    \begin{macrocode}
     \@thm{#1}{#2}}%
%    \end{macrocode}
%    With this, the `sub-definition' is complete.  The macro
%    |\@endtheorem| ends a theorem environment and is, so far, nothing
%    but an |\endtrivlist|.  (Hence it is defined globally, and not
%    within the theorem styles.\footnote{This has to be changed as
%    soon as theorem styles that change \texttt{\bslash @endtheorem}
%    exist.  In such a case, all existing styles must be changed as
%    well since they will have to reset the macro.}) Therefore, we can
%    set it equivalent to the macro that ends the theorem set (in our
%    example, |\endLem|).  However, if some day theorem styles exist
%    that do change |\@endtheorem|, we would have to use the
%    commented-out line instead.
%    \begin{macrocode}
    \global \expandafter \let \csname end#1\endcsname \@endtheorem
%   \global\@namedef{end#1}{\@endtheorem}%
%    \end{macrocode}
%    With these commands all the required definitions are employed,
%    unless the test |\@ifdefinable| has failed.  Therefore, we
%    end the second argument of this macro and with it the definition
%    of |\@xnthm|.
%    \begin{macrocode}
   }}
%    \end{macrocode}
% \end{macro}
%
% \begin{macro}{\@ynthm}
%    The definition of |\@ynthm| is completely analogous. In
%    this case the new counter that is defined is not reset within
%    another counter; thus the definition of |\the...|
%    is simplified:
%    \begin{macrocode}
\gdef\@ynthm#1#2{\expandafter\@ifdefinable\csname #1\endcsname
   {\@definecounter{#1}%
    \expandafter\xdef\csname the#1\endcsname{\@thmcounter{#1}}%
%    \end{macrocode}
%    The rest of the definition corresponds literally to that of
%    |\@xnthm|:
%    \begin{macrocode}
    \def\@tempa{\global\@namedef{#1}}\expandafter \@tempa
     \expandafter{\csname th@\the \theorem@style \expandafter
     \endcsname \the\theorem@bodyfont \@thm{#1}{#2}}%
    \global \expandafter \let \csname end#1\endcsname \@endtheorem}}
%    \end{macrocode}
% \end{macro}
%
% \begin{macro}{\@othm}
%    The definition of |\@othm| does not contain anything new.
%    \begin{macrocode}
\gdef\@othm#1[#2]#3{%
%    \end{macrocode}
% \changes{v2.2b}{95/10/07}{Build in check for counter defined}
%    We do not define a new counter but instead use one that has
%    already been defined.  Thus the only definition we need is that
%    of this pseudo-counter (i.e.\ |\the|\meta{env.\ name}).
%    First we check if |#2| corresponds to a known counter name.
%    \begin{macrocode}
  \expandafter\ifx\csname c@#2\endcsname\relax
   \@nocounterr{#2}%
  \else
   \expandafter\@ifdefinable\csname #1\endcsname
   {\expandafter \xdef \csname the#1\endcsname
     {\expandafter \noexpand \csname the#2\endcsname}%
%    \end{macrocode}
%    All other parts of the definition can be adopted from
%    |\@xnthm|. We have to remember, though, that in this case
%    the name of the current counter and the theorem label have moved
%    to the second and third arguments.
%    \begin{macrocode}
    \def\@tempa{\global\@namedef{#1}}\expandafter \@tempa
     \expandafter{\csname th@\the \theorem@style \expandafter
     \endcsname \the\theorem@bodyfont \@thm{#2}{#3}}%
    \global \expandafter \let \csname end#1\endcsname \@endtheorem}%
  \fi}
%    \end{macrocode}
% \end{macro}
%
% \subsection{Macros that are employed in a {\envfont theorem}
%    environment}
%
% \begin{macro}{\@thm}
%    The macro |\@thm| has to increase the current counter.  Then,
%    depending on whether the environment has (or does not have) an
%    optional argument, it has to branch into either |\@begintheorem|
%    or |\@opargtheorem|.
%    \begin{macrocode}
\gdef\@thm#1#2{\refstepcounter{#1}%
%    \end{macrocode}
%    Now we start a {\envfont trivlist} environment, and give
%    |\@topsep| and |\@topsepadd| the values of the \textsf{skip}
%    registers |\theorempreskipamount| and |\theorempostskipamount|.
%    The value in |\@topsep| is the vertical space that is inserted by
%    the first (and only) |\item| in our |\trivlist| whilst
%    |\@topsepadd| is inserted by |\@endparenv| at the end of that
%    {\envfont trivlist} environment.  By using these registers, we
%    obtain the desired space around a {\envfont theorem} environment.
%    \begin{macrocode}
   \trivlist
   \@topsep \theorempreskipamount               % used by first \item
   \@topsepadd \theorempostskipamount           % used by \@endparenv
%    \end{macrocode}
%    Now we have to test whether an optional argument has been given.
% \changes{v2.0d}{89/07/18}{Forgotten percent added.}
%    \begin{macrocode}
   \@ifnextchar [%
%    \end{macrocode}
%    If there is an optional argument, we will call |\@ythm|, and
%    move the arguments read back into the input stream.
%    \begin{macrocode}
   {\@ythm{#1}{#2}}%
%    \end{macrocode}
%    If not, we call |\@begintheorem|. Its first argument is the
%    name of the theorem set (hence the second argument of
%    |\@thm|). Its second argument is the macro that produces the
%    current number.
%    \begin{macrocode}
   {\@begintheorem{#2}{\csname the#1\endcsname}\ignorespaces}}
%    \end{macrocode}
% \end{macro}
%
% \begin{macro}{\@xthm}
% \begin{macro}{\@ythm}
%    Both these macros were originally called by |\@thm|.  We do
%    not need |\@xthm| anymore, hence we reset it to
%    |\relax|. The definition of |\@ythm| has not changed at
%    all from its definition in \LaTeX. In order to make the macros
%    easier to understand, we will
%    nevertheless present it (commented out).
%    \begin{macrocode}
\global\let\@xthm\relax
% \def\@ythm#1#2[#3]{\@opargbegintheorem{#2}{\csname
%           the#1\endcsname}{#3}\ignorespaces}
%</package>
%    \end{macrocode}
%    The primitive |\ignorespaces| in |\@ythm| and
%    |\@thm| is needed to remove the spaces between the
%    |\begin{...}| and the actual text.
% \end{macro}
% \end{macro}
%
%
% \subsection{Definition of the theorem substyles}
%
% As already pointed out, the theorem substyles, defined below, are
% only loaded when necessary.  Note that all these substyles, except
% \texttt{plain}, have |\slshape| as the default body font.
%
% \subsubsection{The \texttt{plain} style}
%
% \label{dok:th@plain} As the following macros use |@|, we have
% to locally set the |\catcode| of this symbol to ``letter''.
% This happens within a group, so that we do not have to worry about
% which |\catcode| that symbol had before.
%    \begin{macrocode}
%<*thp>
\begingroup \makeatletter
%    \end{macrocode}
% Since we are now within a group, we must make all definitions
% globally.  First we make sure that \texttt{theorem.sty} is loaded.
% This will allow us to use this file as a document style option
% without having to call \texttt{theorem} itself as an option.  At the
% same time, we assure that at least version 2 is loaded, since
% |\theorem@style| was not defined in earlier versions.
%    \begin{macrocode}
\@ifundefined{theorem@style}{% \iffalse meta-comment
%
% Copyright (C) 1993-2024
%
% The LaTeX Project and any individual authors listed elsewhere
% in this file.
%
% This file is part of the Standard LaTeX `Tools Bundle'.
% -------------------------------------------------------
%
% It may be distributed and/or modified under the
% conditions of the LaTeX Project Public License, either version 1.3c
% of this license or (at your option) any later version.
% The latest version of this license is in
%    https://www.latex-project.org/lppl.txt
% and version 1.3c or later is part of all distributions of LaTeX
% version 2005/12/01 or later.
%
% The list of all files belonging to the LaTeX `Tools Bundle' is
% given in the file `manifest.txt'.
%
% \fi
%
%%
%
%
\def\FMithmInfo{2023/07/05 v2.2c Theorem extension package (FMi)}
%
% \ProvidesFile{theorem.dtx}[\FMithmInfo]
%\iffalse   % this is a METACOMMENT !
%%
%% Package `theorem' to use with LaTeX2e
%% Copyright (C) 1989-1995 Frank Mittelbach, all rights reserved.
%<+package>\@ifundefined{theorem@style}{}{\endinput}
%<+package>\NeedsTeXFormat{LaTeX2e}[1995/06/01]
%<+package>\ProvidesPackage{theorem}
%<+package>         [\FMithmInfo]
%
% \fi
%
% \changes{v1.0}{87/05/07}{First draft.}
% \changes{v1.0a}{87/05/07}{`thistheoremstyle' added.}
% \changes{v1.1a}{87/07/29}{New concept: redefine the internal
%                          \protect\LaTeX{} commands `@xnthm' etc.}
% \changes{v1.1b}{87/07/31}{`theoremfont' used to specify font.}
% \changes{v1.2a}{88/01/08}{`@endtheorem' now globally defined.}
% \changes{v1.2b}{88/01/11}{`newtheorem' only allowed in the preamble.}
% \changes{v1.3a}{88/02/14}{`theoremfont' now a toks register.}
%
% \changes{v2.0a}{88/11/16}{New implementation and new documentation.}
% \changes{v2.0a}{88/11/16}{No layout changes but internal modifications
%                           old substyles won't work.}
% \changes{v2.0b}{89/01/30}{`global' used for all definitions.}
% \changes{v2.0c}{89/06/15}{Documentation translated to English.
%                            Thanks to LLL and Nina}
% \changes{v2.1a}{90/02/15}{Support for new font selection.}
% \changes{v2.2a}{94/02/03}{Cleanup for LaTeX2e}
%
% \DoNotIndex{\@endparenv,\@ifdefinable,\@ifnextchar,\@ifundefined}
% \DoNotIndex{\@namedef,\@nameuse,\@spaces,\@tempa,\@warning}
% \DoNotIndex{\ }
% \DoNotIndex{\begingroup,\bfseries}
% \DoNotIndex{\csname}
% \DoNotIndex{\def,\do,\docdate}
% \DoNotIndex{\endcsname,\endgroup,\endinput,\endtrivlist,\expandafter}
% \DoNotIndex{\filedate,\fileversion}
% \DoNotIndex{\gdef,\global}
% \DoNotIndex{\hbox,\hskip}
% \DoNotIndex{\ignorespaces,\input,\itshape,\item}
% \DoNotIndex{\labelsep,\let,\llap}
% \DoNotIndex{\makeatletter}
% \DoNotIndex{\noexpand}
% \DoNotIndex{\refstepcounter,\relax,\rlap}
% \DoNotIndex{\setlength,\slshape,\space,\string,\strut}
% \DoNotIndex{\the,\trivlist,\typeout}
% \DoNotIndex{\vbox}
% \DoNotIndex{\xdef}
%
% \def\deflabel#1{\ttfamily #1\hfill}
% \def\deflist#1{\begin{list}{}{\settowidth\labelwidth{\ttfamily #1}%
%                               \setlength\leftmargin\labelwidth
%                               \addtolength\leftmargin\labelsep
%                               \let\makelabel\deflabel}}
% \def\enddeflist{\end{list}}
% \def\envfont{\normalfont\ttfamily}
%
% \MakeShortVerb{\|}
%
% \newcommand{\TUB}{{\sl TUGboat\/}}
%
% \GetFileInfo{theorem.dtx}
% \title{An Extension of the
%        \LaTeX{} theorem environment\thanks{This file
%        has version number \fileversion, last
%        revised \filedate.}}
% \author{Frank Mittelbach}
%
% \MaintainedByLaTeXTeam{tools}
% \maketitle
%
% \begin{abstract}
%    The macros described in this paper yield an extension of the
%    \LaTeX{} theorem mechanism. It is designed to satisfy the
%    different requirements of various journals. Thus, the
%    layout of the ``theorems'' can be manipulated by determining a
%    ``style''. This article describes not only the use, but
%    also the definition, of the necessary macros.
% \end{abstract}
%
% \section*{Preface to version 2.2}
%
% For \LaTeXe{} this package did not need any fundamental changes. I
% only modified the messages generated so that theorem layout styles
% will show up with the |\listfiles| command and cleaned the section
% on the New Font Selection Scheme since this is now included in
% \LaTeX.
%
% \section*{Preface to version 2.1}
%
% This version is identical to 2.0g described in \TUB\ 10\#3 except
% for some internal defaults which are now set depending on the used
% font selection scheme.
%
% This was done to avoid unpleasant surprises if the new font selection
% scheme is in force. For further details see
% section~\ref{sec:newfonts} and \cite{bk:GMS94}.
%
% \section{Introduction}
%
% For our purposes here, ``theorems'' are labelled enunciations,
% often set off from the main text by extra space and a font change.
% Theorems, corollaries, conjectures, definitions, and remarks
% are all instances of ``theorems''.  The ``header'' of these structures
% is composed of a label (such as \textsc{Theorem} or \textsc{Remark})
% and a number which serializes an item in the sequence of items
% with the same label.
%
% Shortly after the introduction of \LaTeX{} at the Fachbereich
% Mathematik in Mainz, the desire to manipulate the layout of
% ``theorems'' arose.  In Mainz, the following two conventions came
% into general use:
% \begin{enumerate}
%   \item The number of the theorem is shown in the margin.
%   \item There is a line break at
%         the end of the theorem header.
% \end{enumerate}
%
% Additionally, some journals require different formats which depend on
% the ``sort of theorem'': e.g.\ often remarks and definitions are set
% in |\upshape|, while |\itshape| is employed for main theorems.
%
% Confronted with these requirements, a theorem environment was
% developed in Mainz which allows separate determination of the
% layout of the ``theorems sets'', comparable to |\pagestyle|.
%
%
% \section{The user interface}
% \subsection{Defining new theorem sets}
%
% \DescribeMacro\newtheorem
% As in the original \LaTeX{} version, the command |\newtheorem|
% defines a new ``theorem set'' or ``theorem-like structure''.
% Two required arguments name the new environment and give the text
% to be typeset with each instance of the new ``set'', while
% an optional argument determines how
% the ``set'' is enumerated:
% \begin{description}
%    \item[\ttfamily \bslash newtheorem\{foo\}\{bar\}]
%       The theorem set {\envfont foo} (whose name is \texttt{bar})
%       uses its own counter.
%    \item[\ttfamily \bslash newtheorem\{foo2\}{[foo]}\{bar2\}]
%       The theorem set {\envfont foo2} (printed name \texttt{bar2})
%       uses the same counter as the theorem set \texttt{foo}.
%    \item[\ttfamily \bslash newtheorem\{foo3\}\{bar3\}{[section]}]
%       The theorem set {\envfont foo3} (printed name \texttt{bar3}) is
%       enumerated within the counter \texttt{section}, i.e.\ with every
%       new |\section| the enumeration begins again with 1, and
%       the enumeration is composed from the section-number and the
%       theorem counter itself.
% \end{description}
%
% \DescribeMacro\theoremstyle
% Additionally, the command |\theoremstyle| can define the layout
% of various, or all, theorem sets.  It should be noted that any
% theorem set defined by |\newtheorem| is typeset in the
% |\theoremstyle| that is current at the time of the definition.
% Thus, the following
% \begin{verbatim}
%    \theoremstyle{break}        \newtheorem{Cor}{Corollary}
%    \theoremstyle{plain}        \newtheorem{Exa}{Example}[section]
%\end{verbatim}                             ^^A No blank at line start
% leads to the result that the set {\envfont Cor} is formatted in the
% style \texttt{break}, while the set {\envfont Exa} and all the
% following ones are formatted in the style \texttt{plain}, unless
% another |\theoremstyle| follows.  Since the definitions installed by
% |\newtheorem| are global, one also can limit |\theoremstyle| locally
% by grouping braces.
%
% \DescribeMacro\theorembodyfont
% The choice of the font for the theorem body is completely
% independent of the chosen |\theoremstyle|; this has proven to
% be very advantageous.  For example,
% \begin{verbatim}
%   {\theorembodyfont{\upshape}          \newtheorem{Rem}{Remark}}
%\end{verbatim}
% defines a theorem set {\envfont Rem}, which will be set in
% |\upshape| in the current layout (which in our example is
% \texttt{plain}).  As with |\theoremstyle|, the |\theorembodyfont|
% chosen is that current at the time of |\newtheorem|.  If
% |\theorembodyfont| is not specified or one defines
% |\theorembodyfont{}|, then the font used will be that defined by the
% |\theoremstyle|.
%
% \DescribeMacro\theoremheaderfont It is also possible to customize
% the font used for the theorem headers.  This is, however, a global
% declaration, and therefore there should be at most one
% |\theoremheaderfont| declaration in the preamble.\footnote{If it is
% actually necessary to have different header fonts, one has to define
% new theorem styles (substituting the desired font) or specify the
% information directly in the \texttt{\bslash newtheorem} declaration
% (the unclean variant).}
%
% \DescribeMacro\theorempreskipamount
% \DescribeMacro\theorempostskipamount Two additional parameters
% affect the vertical space around the theorem environments:
% |\theorempreskipamount| and |\theorempostskipamount| define,
% respectively, the spacing before and after such an environment.
% These parameters apply for all theorem sets and can be manipulated
% with the ordinary length macros.  They are rubber lengths
% (`\textsf{skips}'), and therefore can contain \texttt{plus} and
% \texttt{minus} parts.
%
% Since the definition of theorem sets should---most sensibly---be
% placed in the preamble, we only allow installation there.
% It is therefore possible to release the memory used here after
% |\begin{document}|, in order to make room for other
% applications.
%
%
% \subsection{Existing theorem styles}
%
% The following theorem styles exist to date:
% \begin{deflist}{marginbreak}
%    \item[plain]
%       This theorem style emulates the original \LaTeX{} definition,
%       except that additionally the parameters
%       |\theorem...skipamount| are used.
%    \item[break]
%       In this style, the theorem header is followed by a line
%       break.
%    \item[marginbreak]
%       The theorem number is set in the margin, and there is a line
%       break as in \texttt{break}.
%    \item[changebreak]
%       Like \texttt{break}, but with header number and text
%       interchanged.
%    \item[change]
%       Header number and text are interchanged, without a line break.
%    \item[margin]
%       The number is set in the left margin, without a line break.
% \end{deflist}
% All styles (except \texttt{plain}) select |\slshape| as the default
% |\theorembodyfont|.
%
%
% \subsection{Examples}
%
% Given the above theorem sets {\envfont Cor}, {\envfont Exa}
% and {\envfont Rem},
% suppose that the preamble also contains the declarations:
% \begin{verbatim}
%    \theoremstyle{marginbreak}   \newtheorem{Lem}[Cor]{Lemma}
%    \theoremstyle{change}
%    \theorembodyfont{\itshape}        \newtheorem{Def}[Cor]{Definition}
%
%    \theoremheaderfont{\scshape}
%\end{verbatim}
% Then the following are some typical examples of the typeset output
% resulting from their use.
%
% \begin{Cor}
%    This is a sentence typeset in the theorem environment {\envfont
%    Cor}.
% \end{Cor}
%
% \begin{Exa}
%    This is a sentence typeset in the theorem environment {\envfont
%    Exa}.
% \end{Exa}
%
% \begin{Rem}
%    This is a sentence typeset in the theorem environment {\envfont
%    Rem}.
% \end{Rem}
%
% \begin{Lem}[Ben User]
%    This is a sentence typeset in the theorem environment {\envfont
%    Lem}.
% \end{Lem}
%
% \begin{Def}[Very impressive Definition]
%    This is a sentence typeset in the theorem environment {\envfont
%    Def}.
% \end{Def}
%
% The last two examples show the effect of the optional argument to
% a theorem environment (it is the text typeset in parentheses).
%
%
% \section{Special Considerations}
% \label{sec:newfonts}
%
% Theoremheader and body are implemented as a unit. This means that
% the |\theoremheaderfont|  will inherit characteristics of the
% |\theorembodyfont| in \LaTeXe. Thus, if for example
% |\theorembodyfont| is |\itshape| and |\theoremheaderfont|
% is |\bfseries| the font selected for the header will have the
% characteristics `bold extended italic'. If this is not desired one
% should set the |\theoremheaderfont| to something like
% \begin{verbatim}
%      \theoremheaderfont{\normalfont\bfseries}
%\end{verbatim}
% i.e.\ supplying all necessary font information explicitly.
%
%
%
% \section{Acknowledgements}
%
% The publication of this set of macros was only possible with the
% help of Christina Busse (translating the manuscript into English),
% Joachim Pense (playing the r\^ole of typist), Chris Rowley
% (looking everything over) and many others providing useful
% suggestions.
%
% \MaybeStop{%
%    \begin{thebibliography}{9}
%       \bibitem{bk:GMS94} \textsc{M.~Goossens}, \textsc{F.~Mittelbach}
%         and \textsc{A.~Samarin}.
%         \newblock The \LaTeX{} Companion.
%         \newblock
%         Addison-Wesley, Reading, Massachusetts, 1994.
%       \bibitem{latex:source} \textsc{Lamport, Leslie.} \newblock
%                               \texttt{latex.tex}, version 2.09, date
%                               Feb. 1990.
%    \end{thebibliography}
%    \PrintIndex
%    \PrintChanges}
%
%
% \section{The documentation driver file}
%
% The next bit of code contains the documentation driver file for
% \TeX{}, i.e., the file that will produce the documentation you are
% currently reading. It will be extracted from this file by the
% \texttt{docstrip} program. Since it is the first code in the file
% one can alternatively process this file directly with \LaTeXe{} to
% obtain the documentation.
%
%    \begin{macrocode}
%<*driver>
\documentclass{ltxdoc}

\usepackage{theorem}

%<+driver>% The next few lines define theorem sets which are used
%<+driver>% in the example section of the documentation.

\theoremstyle{break}       \newtheorem{Cor}{Corollary}
\theoremstyle{plain}       \newtheorem{Exa}{Example}[section]
{\theorembodyfont{\upshape}\newtheorem{Rem}{Remark}}
\theoremstyle{marginbreak} \newtheorem{Lem}[Cor]{Lemma}
\theoremstyle{change}
\theorembodyfont{\itshape} \newtheorem{Def}[Cor]{Definition}

\theoremheaderfont{\scshape}

\RecordChanges

\begin{document}
   \DocInput{theorem.dtx}
\end{document}
%</driver>
%    \end{macrocode}
%
% \section{Definition of the Macros}
%
% If the file has been loaded before, we abort immediately. If not the
% package announces itself (this is actually done at the very top of
% the file---the way it is done isn't good style so don't copy it).
% \changes{v2.0e}{89/07/19}{Spaces removed from `typeout'.}
% \changes{v2.2b}{95/11/19}{Announce moved to top of file}
%    \begin{macrocode}
%<*package>
%\@ifundefined{theorem@style}{}{\endinput}
%\def\FMithmInfo{1995/11/19 v2.2b Theorem extension package (FMi)}
%\ProvidesPackage{theorem}[\FMithmInfo]
%    \end{macrocode}
%
%
% \subsection{Definition of theorem styles and fonts}
%
% All the definitions in this file are done globally to allow
% inputting this file inside a group.
%
% \begin{macro}{\theoremstyle}
%    Before a theorem style can be installed, the chosen style must be
%    known. For that reason, we must test to see that
%    |\th@|\meta{style} is known or, more precisely, that it is
%    different from |\relax|. If the style is not known then
%    |\th@plain| is used.
%    \begin{macrocode}
\gdef\theoremstyle#1{%
   \@ifundefined{th@#1}{\@warning
          {Unknown theoremstyle `#1'. Using `plain'}%
          \theorem@style{plain}}%
%    \end{macrocode}
%    We save the theorem style to be used in the token register
%    |\theorem@style|.
%    \begin{macrocode}
      {\theorem@style{#1}}%
%    \end{macrocode}
%    Now we ``evaluate'' the theorem style: this means, we call the
%    macro |\th@|\meta{style} which will activate the relevant
%    definitions which are contained in a separate file.  This is done
%    in a group to suppress changes to the current font.  This could
%    otherwise pose problems together with the new font selection
%    scheme\footnote{When I printed the original article using the new
%    font selection scheme I ended with a document with slanted
%    typefaces (text headings and all) simply because one of the
%    theorem styles used \texttt{\bslash sl} at toplevel.} if the
%    |\th@|\meta{style} is evaluated a second time.
% \changes{v2.1a}{90/02/15}{Group added around `th@..}
% \changes{v2.1a}{90/02/15}{Reseting of `..@begintheorem removed.}
%    \begin{macrocode}
      \begingroup
        \csname th@\the\theorem@style \endcsname
      \endgroup}
%    \end{macrocode}
% \end{macro}
%
%
% \begin{macro}{\@begintheorem}
% \changes{v2.1a}{90/02/15}{Relaxed at toplevel.}
% \begin{macro}{\@opargbegintheorem}
% \changes{v2.1a}{90/02/15}{Relaxed at toplevel.}
%    We reset |\@begintheorem| and |\@opargbegintheorem| to
%    |\relax| since these commands are no longer necessary at
%    toplevel. This will save a few tokens.
%    \begin{macrocode}
\global\let\@begintheorem\relax
\global\let\@opargbegintheorem\relax
%    \end{macrocode}
% \end{macro}
% \end{macro}
%
% \begin{macro}{\theorem@style}
%    Obviously the \textsf{token} register used above has to be
%    allocated. To assure the utmost compatibility with the
%    original \LaTeX{} definition, we set the default theorem style to
%    \texttt{plain}, which implements the usual \LaTeX{} convention.
% \changes{v2.0f}{89/07/20}{Unnecessary `global' removed.}
%    \begin{macrocode}
\newtoks\theorem@style
\global\theorem@style{plain}
%    \end{macrocode}
% \end{macro}
%
% \begin{macro}{\theorembodyfont}
% \begin{macro}{\theorem@bodyfont}
%    If the |\theorembodyfont| is set by the user then it should not
%    interact with the default font set in the theorem style. When the
%    new font selection is in force this may happen if, for example,
%    the default is |\itshape| and the new |\theorembodyfont| is
%    |\sffamily|. So we add a |\reset@font| command in front of the
%    user definition.
% \changes{v2.0f}{89/07/20}{Unnecessary `global' removed.}
% \changes{v2.0g}{89/09/19}{`theoremfont to `theorembodyfont.}
% \changes{v2.1c}{92/06/22}{`reset@font added and logic changed.}
% \changes{v2.1d}{92/10/28}{Reset to empty if arg is empty}
%    \begin{macrocode}
\gdef\theorembodyfont#1{%
%    \end{macrocode}
%    We check if the argument supplied is empty and if so put nothing
%    into the |\theorem@bodyfont| token register to allow for
%    |\theorembodyfont{}| as a mean of using the default of the
%    current |\theoremstyle|.
%    \begin{macrocode}
   \def\@tempa{#1}%
   \ifx\@tempa\@empty
    \theorem@bodyfont{}%
   \else
    \theorem@bodyfont{\reset@font#1}%
   \fi
   }
\newtoks\theorem@bodyfont
\global\theorem@bodyfont{}
%    \end{macrocode}
% \end{macro}
% \end{macro}
%
%
% \begin{macro}{\theoremheaderfont}
% \changes{v2.0g}{89/09/19}{Macro added.}
%    The font for the theorem headers is handled differently because
%    this definition applies to all theorem styles.
%    \begin{macrocode}
\gdef\theoremheaderfont#1{\gdef\theorem@headerfont{#1}%
%    \end{macrocode}
%    After using the macro once it is redefined to produce an error
%    message.
%    \begin{macrocode}
       \gdef\theoremheaderfont##1{%
        \typeout{\string\theoremheaderfont\space should be used
                 only once.}}}
%    \end{macrocode}
% \end{macro}
%
% \begin{macro}{\theorem@headerfont}
% \changes{v2.0g}{89/09/19}{Macro added.}
%    To set the |\theorem@headerfont| default we first test if
%    the new fontselection scheme is in force.
% \changes{v2.1a}{90/02/15}{Support for new font selection.}
%    \begin{macrocode}
\ifx\upshape\undefined
%    \end{macrocode}
%    If not we define it to expand into |\bfseries|. We don't use
%    |\let| just in case a following style option redefines this
%    macro.
%    \begin{macrocode}
\gdef\theorem@headerfont{\bfseries}
%    \end{macrocode}
%    Otherwise we reset the current shape before calling |\bfseries|.
%    \begin{macrocode}
\else \gdef\theorem@headerfont{\normalfont\bfseries}\fi
%    \end{macrocode}
% \end{macro}
%
% \begin{macro}{\th@plain}
% \begin{macro}{\th@break}
% \begin{macro}{\th@marginbreak}
% \begin{macro}{\th@changebreak}
% \begin{macro}{\th@change}
% \begin{macro}{\th@margin}
%    The different styles are defined in macros such as
%    |\th@plain|.  Since memory space is precious in
%    ``non-Big-versions'', we have to avoid offering too many unused
%    definitions.  Therefore we define these styles in separate files
%    that can be loaded on demand.  Thus the commands themselves only
%    load these files. We use |\@input@|, a \LaTeXe{} internal command
%    that ensures that the file will be listed with |\listfiles|.
% \changes{v2.2a}{94/02/02}{Use `@input@ to load theorem layout files}
%    \begin{macrocode}
\gdef\th@plain{\@input@{thp.sty}}
\gdef\th@break{\@input@{thb.sty}}
\gdef\th@marginbreak{\@input@{thmb.sty}}
\gdef\th@changebreak{\@input@{thcb.sty}}
\gdef\th@change{\@input@{thc.sty}}
\gdef\th@margin{\@input@{thm.sty}}
%    \end{macrocode}
%    This list will be expanded when new styles become
%    available.  For testing,  just append new theorem substyles
%    as document options.
% \end{macro}
% \end{macro}
% \end{macro}
% \end{macro}
% \end{macro}
% \end{macro}
%
%
%
% \subsection{Definition of a new theorem set}
%
% As already pointed out, a new theorem environment can be defined in
% three different ways:
% \begin{verbatim}
%    \newtheorem{Lem}{Lemma}
%    \newtheorem{Lem}{Lemma}[section]
%    \newtheorem{Lem}[Theorem]{Lemma}
%\end{verbatim}
% The function of the macro |\newtheorem| is to recognize these cases
% and then to branch into one of the three macros |\@ynthm|, |\@xnthm|
% or |\@othm|.  This mechanism is adopted unchanged from
% \cite{latex:source}; the essential point here is that, for example,
% in the second case, the arguments \texttt{Lem}, \texttt{Lemma} and
% \texttt{section} are passed over to the macro |\@xnthm|.
%
% We inspect this case first because the others present fewer problems,
% and thus are easily derived from this one.
%
% \begin{macro}{\@xnthm}
%    For our example arguments, the macro |\@xnthm| must fulfill
%    the following:
%    \begin{itemize}
%       \item
%          Define a new \LaTeX-counter `Lem'
%       \item
%          reset this counter within a |\section|
%       \item
%          define the macro |\theLem|
%       \item
%          define the environment macros |\Lem| and
%          |\endLem| using the current
%          |\theoremstyle| and |\theorem@bodyfont|.
%    \end{itemize}
%    Obviously, all this should happen only if the first argument
%    of |\@xnthm| (i.e.\ \texttt{Lem} in our example) is chosen so
%    as not to conflict with any previously defined commands or
%    environments.  This test is performed by the \LaTeX{} macro
%    |\@ifdefinable|.
%    \begin{macrocode}
\gdef\@xnthm#1#2[#3]{\expandafter\@ifdefinable\csname #1\endcsname
%    \end{macrocode}
%    Therefore, the first argument of |\@ifdefinable| is the
%    expansion (in the example, |\Lem|) of
%    |\csname||#1\endcsname|.  The second argument is
%    executed only if the test has been completed successfully.
%    \begin{macrocode}
   {%
%    \end{macrocode}
%    Now we define the new counter. The names of the \LaTeX{}
%    macros employed should speak for themselves:
% \changes{v2.2b}{1995/10/07}{Use \cs{@newctr}}
% \changes{v2.2c}{1995/11/23}{Use \cs{@newctr} properly.}
%    \begin{macrocode}
    \@definecounter{#1}\@newctr{#1}[#3]%
%    \end{macrocode}
%    Using |\@newctr| will give a proper error message if the counter
%    in |#3| is not defined.
%    In defining `|\theLem|' we must generate the desired macro
%    name by use of |\expandafter| and |\csname|.
%    \begin{macrocode}
    \expandafter\xdef\csname the#1\endcsname
%    \end{macrocode}
%    An |\xdef| is used in order to make the definition global,
%    and to ensure that it contains the replacement texts of
%    |\@thmcountersep| and |\@thmcounter|.\footnote{These
%    two macros can be defined by the document style.  Their default
%    values produce a `\texttt{.}' as separation and an arabic
%    representation of the number.} However, not everything should be
%    expanded. For example, it saves space to use |\thesection|
%    instead of its---at times---lengthy expansion.
%    \begin{macrocode}
      {\expandafter \noexpand \csname the#3\endcsname
       \@thmcountersep \@thmcounter{#1}}%
%    \end{macrocode}
%    Thus with the defaults of \LaTeX{}, |\theLem| would be
%    replaced by the command sequence |\thesection.\arabic{Lem}|.
%
%    We will now look at the definition of the macro which is executed
%    at the beginning of the actual environment (in our example this
%    macro is |\Lem|). It should be noted that we use an
%    ``|\expandafter| trick'' to expand only certain parts of the
%    replacement text at the time of the definition.
%    \begin{macrocode}
    \def\@tempa{\global\@namedef{#1}}%
    \expandafter \@tempa \expandafter{%
%    \end{macrocode}
%    First, the macro that contains the current definitions of
%    |\@begintheorem| and |\@opargtheorem| should be called
%    up.  The name of this macro---as is already known---has the form
%    |\th@|\meta{theorem\ style}; therefore, it must be called by
%    \begin{macrocode}
      \csname th@\the \theorem@style
%    \end{macrocode}
%    In addition the default theorem font should be changeable, i.e.\
%    we have to insert the contents of |\theorem@bodyfont|. For
%    that reason, we expand even further, beyond |\endcsname|,
%    and thus insert the contents of the \textsf{token} register
%    |\theorem@bodyfont| in the replacement text.
%    \begin{macrocode}
            \expandafter \endcsname \the \theorem@bodyfont
%    \end{macrocode}
%    Now it is time to call the macro |\@thm| which takes over
%    the further processing.  It has two arguments: the current
%    counter name (in our example, \texttt{Lem}), and the text of the
%    label (in our example, \texttt{Lemma}).
%    \begin{macrocode}
     \@thm{#1}{#2}}%
%    \end{macrocode}
%    With this, the `sub-definition' is complete.  The macro
%    |\@endtheorem| ends a theorem environment and is, so far, nothing
%    but an |\endtrivlist|.  (Hence it is defined globally, and not
%    within the theorem styles.\footnote{This has to be changed as
%    soon as theorem styles that change \texttt{\bslash @endtheorem}
%    exist.  In such a case, all existing styles must be changed as
%    well since they will have to reset the macro.}) Therefore, we can
%    set it equivalent to the macro that ends the theorem set (in our
%    example, |\endLem|).  However, if some day theorem styles exist
%    that do change |\@endtheorem|, we would have to use the
%    commented-out line instead.
%    \begin{macrocode}
    \global \expandafter \let \csname end#1\endcsname \@endtheorem
%   \global\@namedef{end#1}{\@endtheorem}%
%    \end{macrocode}
%    With these commands all the required definitions are employed,
%    unless the test |\@ifdefinable| has failed.  Therefore, we
%    end the second argument of this macro and with it the definition
%    of |\@xnthm|.
%    \begin{macrocode}
   }}
%    \end{macrocode}
% \end{macro}
%
% \begin{macro}{\@ynthm}
%    The definition of |\@ynthm| is completely analogous. In
%    this case the new counter that is defined is not reset within
%    another counter; thus the definition of |\the...|
%    is simplified:
%    \begin{macrocode}
\gdef\@ynthm#1#2{\expandafter\@ifdefinable\csname #1\endcsname
   {\@definecounter{#1}%
    \expandafter\xdef\csname the#1\endcsname{\@thmcounter{#1}}%
%    \end{macrocode}
%    The rest of the definition corresponds literally to that of
%    |\@xnthm|:
%    \begin{macrocode}
    \def\@tempa{\global\@namedef{#1}}\expandafter \@tempa
     \expandafter{\csname th@\the \theorem@style \expandafter
     \endcsname \the\theorem@bodyfont \@thm{#1}{#2}}%
    \global \expandafter \let \csname end#1\endcsname \@endtheorem}}
%    \end{macrocode}
% \end{macro}
%
% \begin{macro}{\@othm}
%    The definition of |\@othm| does not contain anything new.
%    \begin{macrocode}
\gdef\@othm#1[#2]#3{%
%    \end{macrocode}
% \changes{v2.2b}{95/10/07}{Build in check for counter defined}
%    We do not define a new counter but instead use one that has
%    already been defined.  Thus the only definition we need is that
%    of this pseudo-counter (i.e.\ |\the|\meta{env.\ name}).
%    First we check if |#2| corresponds to a known counter name.
%    \begin{macrocode}
  \expandafter\ifx\csname c@#2\endcsname\relax
   \@nocounterr{#2}%
  \else
   \expandafter\@ifdefinable\csname #1\endcsname
   {\expandafter \xdef \csname the#1\endcsname
     {\expandafter \noexpand \csname the#2\endcsname}%
%    \end{macrocode}
%    All other parts of the definition can be adopted from
%    |\@xnthm|. We have to remember, though, that in this case
%    the name of the current counter and the theorem label have moved
%    to the second and third arguments.
%    \begin{macrocode}
    \def\@tempa{\global\@namedef{#1}}\expandafter \@tempa
     \expandafter{\csname th@\the \theorem@style \expandafter
     \endcsname \the\theorem@bodyfont \@thm{#2}{#3}}%
    \global \expandafter \let \csname end#1\endcsname \@endtheorem}%
  \fi}
%    \end{macrocode}
% \end{macro}
%
% \subsection{Macros that are employed in a {\envfont theorem}
%    environment}
%
% \begin{macro}{\@thm}
%    The macro |\@thm| has to increase the current counter.  Then,
%    depending on whether the environment has (or does not have) an
%    optional argument, it has to branch into either |\@begintheorem|
%    or |\@opargtheorem|.
%    \begin{macrocode}
\gdef\@thm#1#2{\refstepcounter{#1}%
%    \end{macrocode}
%    Now we start a {\envfont trivlist} environment, and give
%    |\@topsep| and |\@topsepadd| the values of the \textsf{skip}
%    registers |\theorempreskipamount| and |\theorempostskipamount|.
%    The value in |\@topsep| is the vertical space that is inserted by
%    the first (and only) |\item| in our |\trivlist| whilst
%    |\@topsepadd| is inserted by |\@endparenv| at the end of that
%    {\envfont trivlist} environment.  By using these registers, we
%    obtain the desired space around a {\envfont theorem} environment.
%    \begin{macrocode}
   \trivlist
   \@topsep \theorempreskipamount               % used by first \item
   \@topsepadd \theorempostskipamount           % used by \@endparenv
%    \end{macrocode}
%    Now we have to test whether an optional argument has been given.
% \changes{v2.0d}{89/07/18}{Forgotten percent added.}
%    \begin{macrocode}
   \@ifnextchar [%
%    \end{macrocode}
%    If there is an optional argument, we will call |\@ythm|, and
%    move the arguments read back into the input stream.
%    \begin{macrocode}
   {\@ythm{#1}{#2}}%
%    \end{macrocode}
%    If not, we call |\@begintheorem|. Its first argument is the
%    name of the theorem set (hence the second argument of
%    |\@thm|). Its second argument is the macro that produces the
%    current number.
%    \begin{macrocode}
   {\@begintheorem{#2}{\csname the#1\endcsname}\ignorespaces}}
%    \end{macrocode}
% \end{macro}
%
% \begin{macro}{\@xthm}
% \begin{macro}{\@ythm}
%    Both these macros were originally called by |\@thm|.  We do
%    not need |\@xthm| anymore, hence we reset it to
%    |\relax|. The definition of |\@ythm| has not changed at
%    all from its definition in \LaTeX. In order to make the macros
%    easier to understand, we will
%    nevertheless present it (commented out).
%    \begin{macrocode}
\global\let\@xthm\relax
% \def\@ythm#1#2[#3]{\@opargbegintheorem{#2}{\csname
%           the#1\endcsname}{#3}\ignorespaces}
%</package>
%    \end{macrocode}
%    The primitive |\ignorespaces| in |\@ythm| and
%    |\@thm| is needed to remove the spaces between the
%    |\begin{...}| and the actual text.
% \end{macro}
% \end{macro}
%
%
% \subsection{Definition of the theorem substyles}
%
% As already pointed out, the theorem substyles, defined below, are
% only loaded when necessary.  Note that all these substyles, except
% \texttt{plain}, have |\slshape| as the default body font.
%
% \subsubsection{The \texttt{plain} style}
%
% \label{dok:th@plain} As the following macros use |@|, we have
% to locally set the |\catcode| of this symbol to ``letter''.
% This happens within a group, so that we do not have to worry about
% which |\catcode| that symbol had before.
%    \begin{macrocode}
%<*thp>
\begingroup \makeatletter
%    \end{macrocode}
% Since we are now within a group, we must make all definitions
% globally.  First we make sure that \texttt{theorem.sty} is loaded.
% This will allow us to use this file as a document style option
% without having to call \texttt{theorem} itself as an option.  At the
% same time, we assure that at least version 2 is loaded, since
% |\theorem@style| was not defined in earlier versions.
%    \begin{macrocode}
\@ifundefined{theorem@style}{\input{theorem.sty}}{}
\ProvidesFile{thp.sty}
             [\FMithmInfo]
%    \end{macrocode}
%
%
% \begin{macro}{\th@plain}
% \begin{macro}{\@begintheorem}
% \begin{macro}{\@opargbegintheorem}
%    |\theoremstyle{plain}| corresponds to the original
%    definition, except that the distances to the surrounding text are
%    determined by the parameters |\theorempreskipamount| and
%    |\theorempostskipamount|.
%    First we set the default body font.
%    \begin{macrocode}
\gdef\th@plain{\normalfont\itshape
%    \end{macrocode}
%    Then we define |\@begintheorem| and
%    |\@opargbegintheorem|. These two macros define how the
%    header of a theorem is typeset.  |\@opargbegintheorem| will
%    be called if a \textsf{theorem} environment with an optional
%    argument is encountered; otherwise, the header is constructed by
%    calling |\@begintheorem|.  If one of these macros is
%    executed, we are within a \textsf{trivlist} environment started by
%    |\@thm|.  So the theorem header is produced with an
%    |\item| command.
%
%    Instead of specifying the header font directly, all standard
%    theorem styles use the |\theorem@headerfont| macro to allow
%    customization.  The extra space (|\labelsep|) is necessary
%    because of problems in the \textsf{trivlist} environment.
%    \begin{macrocode}
  \def\@begintheorem##1##2{%
        \item[\hskip\labelsep \theorem@headerfont ##1\ ##2]}%
%    \end{macrocode}
%    The definition of |\@opargbegintheorem| is completely
%    analogous. The only difference is the fact that there exists a
%    third argument (which is the optional parameter of the
%    environment and contains additional information about the
%    theorem).  Customarily we enclose it in parentheses.
%    \begin{macrocode}
\def\@opargbegintheorem##1##2##3{%
   \item[\hskip\labelsep \theorem@headerfont ##1\ ##2\ (##3)]}}
%    \end{macrocode}
% \end{macro}
% \end{macro}
% \end{macro}
%
% \noindent
% We conclude with an |\endgroup| to restore the
% |\catcode| of |@|.
%    \begin{macrocode}
\endgroup
%</thp>
%    \end{macrocode}
%
%
% \subsubsection{The \texttt{break} style}
%
% This style option is stored in the file |thb.sty|.
% For the next two lines see the documentation for |\th@plain| on
% page~\pageref{dok:th@plain}.
%    \begin{macrocode}
%<*thb>
\begingroup \makeatletter
\@ifundefined{theorem@style}{\input{theorem.sty}}{}
\ProvidesFile{thb.sty}
             [\FMithmInfo]
%    \end{macrocode}
%
% \begin{macro}{\th@break}
%    |\theoremstyle{break}| produces a line break after the name
%    of the theorem.  The font is |\slshape|.  Hence, we define
%    |\th@break| as follows:
%    \begin{macrocode}
\gdef\th@break{\normalfont\slshape
  \def\@begintheorem##1##2{\item[%
%    \end{macrocode}
%    We run into the following problem: it is not possible to create
%    the header with |\item[|\meta{title}|]| and then start
%    a new line by, for example, |\mbox{}\\|. Such a definition
%    will fail whenever a list environment follows immediately.  With
%    the above construction, the |\mbox{}| causes the switch
%    |@inlabel| (cf.\ definition of |\list| and
%    |\trivlist| in \cite{latex:source}) to be set to \texttt{false}
%    and so the following list will insert additional vertical
%    space (|\topskip|). This is quite annoying. Therefore, we
%    create the line break within the |\item|. In order to ensure
%    that the text will begin at the proper position in the following
%    line, we simply pretend that the label does not take any
%    room.\footnote{This will lead to problems whenever very high
%    symbols occurring in the line tower into the
%    heading. So, something else has to be done here sometime.}
%    \begin{macrocode}
       \rlap{\vbox{\hbox{\hskip \labelsep\theorem@headerfont ##1\ ##2}%
                   \hbox{\strut}}}]}%
%    \end{macrocode}
%    Again, the definition of |\@opargbegintheorem| is
%    completely analogous.
%    \begin{macrocode}
\def\@opargbegintheorem##1##2##3{%
  \item[\rlap{\vbox{\hbox{\hskip \labelsep \theorem@headerfont
                     ##1\ ##2\ (##3)}%
                    \hbox{\strut}}}]}}
\endgroup
%</thb>
%    \end{macrocode}
% \end{macro}
%
%
% \subsubsection{The \texttt{changebreak} style}
%
%    \begin{macrocode}
% This style option is stored in the file |thcb.sty|.
%    \begin{macrocode}
%<*thcb>
\begingroup \makeatletter
\@ifundefined{theorem@style}{\input{theorem.sty}}{}
\ProvidesFile{thcb.sty}
             [\FMithmInfo]
%    \end{macrocode}
%
% \begin{macro}{\th@changebreak}
%    The \texttt{change-break} theorem style is like \texttt{break}
%    but with interchange of theorem name and theorem number.  Thus we
%    define |\th@changebreak| as follows:
%    \begin{macrocode}
\gdef\th@changebreak{\normalfont\slshape
  \def\@begintheorem##1##2{\item
       [\rlap{\vbox{\hbox{\hskip\labelsep \theorem@headerfont ##2\ ##1}%
                    \hbox{\strut}}}]}%
\def\@opargbegintheorem##1##2##3{%
  \item[\rlap{\vbox{\hbox{\hskip\labelsep \theorem@headerfont
                     ##2\ ##1\ (##3)}%
                    \hbox{\strut}}}]}}
\endgroup
%</thcb>
%    \end{macrocode}
% \end{macro}
%
%
% \subsubsection{The \texttt{change} style}
%
% This style option is stored in the file |thc.sty|.
%    \begin{macrocode}
%<*thc>
\begingroup \makeatletter
\@ifundefined{theorem@style}{\input{theorem.sty}}{}
\ProvidesFile{thc.sty}
             [\FMithmInfo]
%    \end{macrocode}
%
% \begin{macro}{\th@change}
%    The \texttt{change} theorem style corresponds to the
%    \texttt{change break} style without a linebreak after the header.
%    To say it in another way, it's the same as the \texttt{plain}
%    style but with number and name interchanged and |\slshape| as the
%    default font.
%    \begin{macrocode}
\gdef\th@change{\normalfont\slshape
  \def\@begintheorem##1##2{\item
       [\hskip\labelsep \theorem@headerfont ##2\ ##1]}%
\def\@opargbegintheorem##1##2##3{%
  \item[\hskip\labelsep \theorem@headerfont ##2\ ##1\ (##3)]}}
\endgroup
%</thc>
%    \end{macrocode}
% \end{macro}
%
%
% \subsubsection{The \texttt{marginbreak} style}
%
% This style option is the one used most often at Mainz.  It is saved
% in the file |thmb.sty|.
%    \begin{macrocode}
%<*thmb>
\begingroup \makeatletter
\@ifundefined{theorem@style}{\input{theorem.sty}}{}
\ProvidesFile{thmb.sty}
             [\FMithmInfo]
%    \end{macrocode}
%
% \begin{macro}{\th@marginbreak}
%    The \texttt{margin break} style is nearly the same as the
%    \texttt{change break} style. The only difference is the placement
%    of the theorem number.  We use |\llap| to place it in the left
%    margin.
%
%    In this style |\labelsep| denotes the separation between the
%    number and the text.
%    \begin{macrocode}
\gdef\th@marginbreak{\normalfont\slshape
  \def\@begintheorem##1##2{\item
       [\rlap{\vbox{\theorem@headerfont
                    \hbox{\llap{##2}\hskip\labelsep ##1}%
                    \hbox{\strut}}}]}%
\def\@opargbegintheorem##1##2##3{%
  \item[\rlap{\vbox{\theorem@headerfont
                    \hbox{\llap{##2}\hskip\labelsep ##1\ (##3)}%
                    \hbox{\strut}}}]}}
\endgroup
%</thmb>
%    \end{macrocode}
% \changes{v2.1c}{90/02/17}{`endgroup outside macrocode env.}
% \end{macro}
%
%
%
% \subsubsection{The \texttt{margin} style}
%
% This style option is stored in the file |thm.sty|.
%    \begin{macrocode}
%<*thm>
\begingroup \makeatletter
\@ifundefined{theorem@style}{\input{theorem.sty}}{}
\ProvidesFile{thm.sty}
             [\FMithmInfo]
%    \end{macrocode}
%
% \begin{macro}{\th@margin}
%    Again this is only a variant of the theorem styles described
%    above without any new ideas.
%    \begin{macrocode}
\gdef\th@margin{\normalfont\slshape
  \def\@begintheorem##1##2{\item
       [\theorem@headerfont \llap{##2}\hskip\labelsep ##1]}%
\def\@opargbegintheorem##1##2##3{%
  \item[\theorem@headerfont \llap{##2}\hskip\labelsep ##1\ (##3)]}}
\endgroup
%</thm>
%    \end{macrocode}
% \end{macro}
%
%
%
%
% \subsection{Final Definitions}
%
% \begin{macro}{\theorempreskipamount}
% \begin{macro}{\theorempostskipamount}
%    The \textsf{skip} parameters that regulate the vertical empty space
%    before and after the {\envfont theorem} environment have
%    to be allocated as well.
% \changes{v2.0f}{89/07/20}{Unnecessary `global' removed.}
%    \begin{macrocode}
%<*package>
\newskip\theorempreskipamount
\newskip\theorempostskipamount
%    \end{macrocode}
%    Since we have used the same values for all theorem sets, we now can
%    assign them.
%    \begin{macrocode}
\global\setlength\theorempreskipamount{12pt plus 5pt minus 3pt}
\global\setlength\theorempostskipamount{8pt plus 3pt minus 1.5pt}
%    \end{macrocode}
% \end{macro}
% \end{macro}
%
%
% \begin{macro}{\@endtheorem}
%    The same holds for the macro |\@endtheorem|, which ends a
%    {\envfont theorem} environment. Since it is the same for all
%    theorem sets, it is removed from the macros |\th@|\meta{style}.
%    It simply ends the {\envfont trivlist} environment, which was
%    begun in |\@thm|.
%    \begin{macrocode}
\global\let\@endtheorem=\endtrivlist
%    \end{macrocode}
% \end{macro}
%
%
%
% \begin{macro}{\@preamblecmds}
%    All macros defined above are to be used only in the preamble.
%    Therefore, we insert them in |\@preamblecmds| which will disable
%    them at begin document. This is done by the internal \LaTeXe{}
%    command |\@onlypreamble|.
% \changes{v2.0g}{89/09/19}{`theoremheaderfont and `theorembodyfont
%    added.}
% \changes{v2.2a}{94/02/03}{Use \LaTeXe{} interface}
%    \begin{macrocode}
\@onlypreamble\@xnthm
\@onlypreamble\@ynthm
\@onlypreamble\@othm
\@onlypreamble\newtheorem
\@onlypreamble\theoremstyle
\@onlypreamble\theorembodyfont
\@onlypreamble\theoremheaderfont
%    \end{macrocode}
% \end{macro}
%
% Finally we declare the \texttt{plain} theorem style to be the default.
% \changes{v2.1b}{90/06/04}{Fixed missing `item by declaring default
%    style.}
%    \begin{macrocode}
\theoremstyle{plain}
%</package>
%    \end{macrocode}
%
% \Finale

\endinput
}{}
\ProvidesFile{thp.sty}
             [\FMithmInfo]
%    \end{macrocode}
%
%
% \begin{macro}{\th@plain}
% \begin{macro}{\@begintheorem}
% \begin{macro}{\@opargbegintheorem}
%    |\theoremstyle{plain}| corresponds to the original
%    definition, except that the distances to the surrounding text are
%    determined by the parameters |\theorempreskipamount| and
%    |\theorempostskipamount|.
%    First we set the default body font.
%    \begin{macrocode}
\gdef\th@plain{\normalfont\itshape
%    \end{macrocode}
%    Then we define |\@begintheorem| and
%    |\@opargbegintheorem|. These two macros define how the
%    header of a theorem is typeset.  |\@opargbegintheorem| will
%    be called if a \textsf{theorem} environment with an optional
%    argument is encountered; otherwise, the header is constructed by
%    calling |\@begintheorem|.  If one of these macros is
%    executed, we are within a \textsf{trivlist} environment started by
%    |\@thm|.  So the theorem header is produced with an
%    |\item| command.
%
%    Instead of specifying the header font directly, all standard
%    theorem styles use the |\theorem@headerfont| macro to allow
%    customization.  The extra space (|\labelsep|) is necessary
%    because of problems in the \textsf{trivlist} environment.
%    \begin{macrocode}
  \def\@begintheorem##1##2{%
        \item[\hskip\labelsep \theorem@headerfont ##1\ ##2]}%
%    \end{macrocode}
%    The definition of |\@opargbegintheorem| is completely
%    analogous. The only difference is the fact that there exists a
%    third argument (which is the optional parameter of the
%    environment and contains additional information about the
%    theorem).  Customarily we enclose it in parentheses.
%    \begin{macrocode}
\def\@opargbegintheorem##1##2##3{%
   \item[\hskip\labelsep \theorem@headerfont ##1\ ##2\ (##3)]}}
%    \end{macrocode}
% \end{macro}
% \end{macro}
% \end{macro}
%
% \noindent
% We conclude with an |\endgroup| to restore the
% |\catcode| of |@|.
%    \begin{macrocode}
\endgroup
%</thp>
%    \end{macrocode}
%
%
% \subsubsection{The \texttt{break} style}
%
% This style option is stored in the file |thb.sty|.
% For the next two lines see the documentation for |\th@plain| on
% page~\pageref{dok:th@plain}.
%    \begin{macrocode}
%<*thb>
\begingroup \makeatletter
\@ifundefined{theorem@style}{% \iffalse meta-comment
%
% Copyright (C) 1993-2024
%
% The LaTeX Project and any individual authors listed elsewhere
% in this file.
%
% This file is part of the Standard LaTeX `Tools Bundle'.
% -------------------------------------------------------
%
% It may be distributed and/or modified under the
% conditions of the LaTeX Project Public License, either version 1.3c
% of this license or (at your option) any later version.
% The latest version of this license is in
%    https://www.latex-project.org/lppl.txt
% and version 1.3c or later is part of all distributions of LaTeX
% version 2005/12/01 or later.
%
% The list of all files belonging to the LaTeX `Tools Bundle' is
% given in the file `manifest.txt'.
%
% \fi
%
%%
%
%
\def\FMithmInfo{2023/07/05 v2.2c Theorem extension package (FMi)}
%
% \ProvidesFile{theorem.dtx}[\FMithmInfo]
%\iffalse   % this is a METACOMMENT !
%%
%% Package `theorem' to use with LaTeX2e
%% Copyright (C) 1989-1995 Frank Mittelbach, all rights reserved.
%<+package>\@ifundefined{theorem@style}{}{\endinput}
%<+package>\NeedsTeXFormat{LaTeX2e}[1995/06/01]
%<+package>\ProvidesPackage{theorem}
%<+package>         [\FMithmInfo]
%
% \fi
%
% \changes{v1.0}{87/05/07}{First draft.}
% \changes{v1.0a}{87/05/07}{`thistheoremstyle' added.}
% \changes{v1.1a}{87/07/29}{New concept: redefine the internal
%                          \protect\LaTeX{} commands `@xnthm' etc.}
% \changes{v1.1b}{87/07/31}{`theoremfont' used to specify font.}
% \changes{v1.2a}{88/01/08}{`@endtheorem' now globally defined.}
% \changes{v1.2b}{88/01/11}{`newtheorem' only allowed in the preamble.}
% \changes{v1.3a}{88/02/14}{`theoremfont' now a toks register.}
%
% \changes{v2.0a}{88/11/16}{New implementation and new documentation.}
% \changes{v2.0a}{88/11/16}{No layout changes but internal modifications
%                           old substyles won't work.}
% \changes{v2.0b}{89/01/30}{`global' used for all definitions.}
% \changes{v2.0c}{89/06/15}{Documentation translated to English.
%                            Thanks to LLL and Nina}
% \changes{v2.1a}{90/02/15}{Support for new font selection.}
% \changes{v2.2a}{94/02/03}{Cleanup for LaTeX2e}
%
% \DoNotIndex{\@endparenv,\@ifdefinable,\@ifnextchar,\@ifundefined}
% \DoNotIndex{\@namedef,\@nameuse,\@spaces,\@tempa,\@warning}
% \DoNotIndex{\ }
% \DoNotIndex{\begingroup,\bfseries}
% \DoNotIndex{\csname}
% \DoNotIndex{\def,\do,\docdate}
% \DoNotIndex{\endcsname,\endgroup,\endinput,\endtrivlist,\expandafter}
% \DoNotIndex{\filedate,\fileversion}
% \DoNotIndex{\gdef,\global}
% \DoNotIndex{\hbox,\hskip}
% \DoNotIndex{\ignorespaces,\input,\itshape,\item}
% \DoNotIndex{\labelsep,\let,\llap}
% \DoNotIndex{\makeatletter}
% \DoNotIndex{\noexpand}
% \DoNotIndex{\refstepcounter,\relax,\rlap}
% \DoNotIndex{\setlength,\slshape,\space,\string,\strut}
% \DoNotIndex{\the,\trivlist,\typeout}
% \DoNotIndex{\vbox}
% \DoNotIndex{\xdef}
%
% \def\deflabel#1{\ttfamily #1\hfill}
% \def\deflist#1{\begin{list}{}{\settowidth\labelwidth{\ttfamily #1}%
%                               \setlength\leftmargin\labelwidth
%                               \addtolength\leftmargin\labelsep
%                               \let\makelabel\deflabel}}
% \def\enddeflist{\end{list}}
% \def\envfont{\normalfont\ttfamily}
%
% \MakeShortVerb{\|}
%
% \newcommand{\TUB}{{\sl TUGboat\/}}
%
% \GetFileInfo{theorem.dtx}
% \title{An Extension of the
%        \LaTeX{} theorem environment\thanks{This file
%        has version number \fileversion, last
%        revised \filedate.}}
% \author{Frank Mittelbach}
%
% \MaintainedByLaTeXTeam{tools}
% \maketitle
%
% \begin{abstract}
%    The macros described in this paper yield an extension of the
%    \LaTeX{} theorem mechanism. It is designed to satisfy the
%    different requirements of various journals. Thus, the
%    layout of the ``theorems'' can be manipulated by determining a
%    ``style''. This article describes not only the use, but
%    also the definition, of the necessary macros.
% \end{abstract}
%
% \section*{Preface to version 2.2}
%
% For \LaTeXe{} this package did not need any fundamental changes. I
% only modified the messages generated so that theorem layout styles
% will show up with the |\listfiles| command and cleaned the section
% on the New Font Selection Scheme since this is now included in
% \LaTeX.
%
% \section*{Preface to version 2.1}
%
% This version is identical to 2.0g described in \TUB\ 10\#3 except
% for some internal defaults which are now set depending on the used
% font selection scheme.
%
% This was done to avoid unpleasant surprises if the new font selection
% scheme is in force. For further details see
% section~\ref{sec:newfonts} and \cite{bk:GMS94}.
%
% \section{Introduction}
%
% For our purposes here, ``theorems'' are labelled enunciations,
% often set off from the main text by extra space and a font change.
% Theorems, corollaries, conjectures, definitions, and remarks
% are all instances of ``theorems''.  The ``header'' of these structures
% is composed of a label (such as \textsc{Theorem} or \textsc{Remark})
% and a number which serializes an item in the sequence of items
% with the same label.
%
% Shortly after the introduction of \LaTeX{} at the Fachbereich
% Mathematik in Mainz, the desire to manipulate the layout of
% ``theorems'' arose.  In Mainz, the following two conventions came
% into general use:
% \begin{enumerate}
%   \item The number of the theorem is shown in the margin.
%   \item There is a line break at
%         the end of the theorem header.
% \end{enumerate}
%
% Additionally, some journals require different formats which depend on
% the ``sort of theorem'': e.g.\ often remarks and definitions are set
% in |\upshape|, while |\itshape| is employed for main theorems.
%
% Confronted with these requirements, a theorem environment was
% developed in Mainz which allows separate determination of the
% layout of the ``theorems sets'', comparable to |\pagestyle|.
%
%
% \section{The user interface}
% \subsection{Defining new theorem sets}
%
% \DescribeMacro\newtheorem
% As in the original \LaTeX{} version, the command |\newtheorem|
% defines a new ``theorem set'' or ``theorem-like structure''.
% Two required arguments name the new environment and give the text
% to be typeset with each instance of the new ``set'', while
% an optional argument determines how
% the ``set'' is enumerated:
% \begin{description}
%    \item[\ttfamily \bslash newtheorem\{foo\}\{bar\}]
%       The theorem set {\envfont foo} (whose name is \texttt{bar})
%       uses its own counter.
%    \item[\ttfamily \bslash newtheorem\{foo2\}{[foo]}\{bar2\}]
%       The theorem set {\envfont foo2} (printed name \texttt{bar2})
%       uses the same counter as the theorem set \texttt{foo}.
%    \item[\ttfamily \bslash newtheorem\{foo3\}\{bar3\}{[section]}]
%       The theorem set {\envfont foo3} (printed name \texttt{bar3}) is
%       enumerated within the counter \texttt{section}, i.e.\ with every
%       new |\section| the enumeration begins again with 1, and
%       the enumeration is composed from the section-number and the
%       theorem counter itself.
% \end{description}
%
% \DescribeMacro\theoremstyle
% Additionally, the command |\theoremstyle| can define the layout
% of various, or all, theorem sets.  It should be noted that any
% theorem set defined by |\newtheorem| is typeset in the
% |\theoremstyle| that is current at the time of the definition.
% Thus, the following
% \begin{verbatim}
%    \theoremstyle{break}        \newtheorem{Cor}{Corollary}
%    \theoremstyle{plain}        \newtheorem{Exa}{Example}[section]
%\end{verbatim}                             ^^A No blank at line start
% leads to the result that the set {\envfont Cor} is formatted in the
% style \texttt{break}, while the set {\envfont Exa} and all the
% following ones are formatted in the style \texttt{plain}, unless
% another |\theoremstyle| follows.  Since the definitions installed by
% |\newtheorem| are global, one also can limit |\theoremstyle| locally
% by grouping braces.
%
% \DescribeMacro\theorembodyfont
% The choice of the font for the theorem body is completely
% independent of the chosen |\theoremstyle|; this has proven to
% be very advantageous.  For example,
% \begin{verbatim}
%   {\theorembodyfont{\upshape}          \newtheorem{Rem}{Remark}}
%\end{verbatim}
% defines a theorem set {\envfont Rem}, which will be set in
% |\upshape| in the current layout (which in our example is
% \texttt{plain}).  As with |\theoremstyle|, the |\theorembodyfont|
% chosen is that current at the time of |\newtheorem|.  If
% |\theorembodyfont| is not specified or one defines
% |\theorembodyfont{}|, then the font used will be that defined by the
% |\theoremstyle|.
%
% \DescribeMacro\theoremheaderfont It is also possible to customize
% the font used for the theorem headers.  This is, however, a global
% declaration, and therefore there should be at most one
% |\theoremheaderfont| declaration in the preamble.\footnote{If it is
% actually necessary to have different header fonts, one has to define
% new theorem styles (substituting the desired font) or specify the
% information directly in the \texttt{\bslash newtheorem} declaration
% (the unclean variant).}
%
% \DescribeMacro\theorempreskipamount
% \DescribeMacro\theorempostskipamount Two additional parameters
% affect the vertical space around the theorem environments:
% |\theorempreskipamount| and |\theorempostskipamount| define,
% respectively, the spacing before and after such an environment.
% These parameters apply for all theorem sets and can be manipulated
% with the ordinary length macros.  They are rubber lengths
% (`\textsf{skips}'), and therefore can contain \texttt{plus} and
% \texttt{minus} parts.
%
% Since the definition of theorem sets should---most sensibly---be
% placed in the preamble, we only allow installation there.
% It is therefore possible to release the memory used here after
% |\begin{document}|, in order to make room for other
% applications.
%
%
% \subsection{Existing theorem styles}
%
% The following theorem styles exist to date:
% \begin{deflist}{marginbreak}
%    \item[plain]
%       This theorem style emulates the original \LaTeX{} definition,
%       except that additionally the parameters
%       |\theorem...skipamount| are used.
%    \item[break]
%       In this style, the theorem header is followed by a line
%       break.
%    \item[marginbreak]
%       The theorem number is set in the margin, and there is a line
%       break as in \texttt{break}.
%    \item[changebreak]
%       Like \texttt{break}, but with header number and text
%       interchanged.
%    \item[change]
%       Header number and text are interchanged, without a line break.
%    \item[margin]
%       The number is set in the left margin, without a line break.
% \end{deflist}
% All styles (except \texttt{plain}) select |\slshape| as the default
% |\theorembodyfont|.
%
%
% \subsection{Examples}
%
% Given the above theorem sets {\envfont Cor}, {\envfont Exa}
% and {\envfont Rem},
% suppose that the preamble also contains the declarations:
% \begin{verbatim}
%    \theoremstyle{marginbreak}   \newtheorem{Lem}[Cor]{Lemma}
%    \theoremstyle{change}
%    \theorembodyfont{\itshape}        \newtheorem{Def}[Cor]{Definition}
%
%    \theoremheaderfont{\scshape}
%\end{verbatim}
% Then the following are some typical examples of the typeset output
% resulting from their use.
%
% \begin{Cor}
%    This is a sentence typeset in the theorem environment {\envfont
%    Cor}.
% \end{Cor}
%
% \begin{Exa}
%    This is a sentence typeset in the theorem environment {\envfont
%    Exa}.
% \end{Exa}
%
% \begin{Rem}
%    This is a sentence typeset in the theorem environment {\envfont
%    Rem}.
% \end{Rem}
%
% \begin{Lem}[Ben User]
%    This is a sentence typeset in the theorem environment {\envfont
%    Lem}.
% \end{Lem}
%
% \begin{Def}[Very impressive Definition]
%    This is a sentence typeset in the theorem environment {\envfont
%    Def}.
% \end{Def}
%
% The last two examples show the effect of the optional argument to
% a theorem environment (it is the text typeset in parentheses).
%
%
% \section{Special Considerations}
% \label{sec:newfonts}
%
% Theoremheader and body are implemented as a unit. This means that
% the |\theoremheaderfont|  will inherit characteristics of the
% |\theorembodyfont| in \LaTeXe. Thus, if for example
% |\theorembodyfont| is |\itshape| and |\theoremheaderfont|
% is |\bfseries| the font selected for the header will have the
% characteristics `bold extended italic'. If this is not desired one
% should set the |\theoremheaderfont| to something like
% \begin{verbatim}
%      \theoremheaderfont{\normalfont\bfseries}
%\end{verbatim}
% i.e.\ supplying all necessary font information explicitly.
%
%
%
% \section{Acknowledgements}
%
% The publication of this set of macros was only possible with the
% help of Christina Busse (translating the manuscript into English),
% Joachim Pense (playing the r\^ole of typist), Chris Rowley
% (looking everything over) and many others providing useful
% suggestions.
%
% \MaybeStop{%
%    \begin{thebibliography}{9}
%       \bibitem{bk:GMS94} \textsc{M.~Goossens}, \textsc{F.~Mittelbach}
%         and \textsc{A.~Samarin}.
%         \newblock The \LaTeX{} Companion.
%         \newblock
%         Addison-Wesley, Reading, Massachusetts, 1994.
%       \bibitem{latex:source} \textsc{Lamport, Leslie.} \newblock
%                               \texttt{latex.tex}, version 2.09, date
%                               Feb. 1990.
%    \end{thebibliography}
%    \PrintIndex
%    \PrintChanges}
%
%
% \section{The documentation driver file}
%
% The next bit of code contains the documentation driver file for
% \TeX{}, i.e., the file that will produce the documentation you are
% currently reading. It will be extracted from this file by the
% \texttt{docstrip} program. Since it is the first code in the file
% one can alternatively process this file directly with \LaTeXe{} to
% obtain the documentation.
%
%    \begin{macrocode}
%<*driver>
\documentclass{ltxdoc}

\usepackage{theorem}

%<+driver>% The next few lines define theorem sets which are used
%<+driver>% in the example section of the documentation.

\theoremstyle{break}       \newtheorem{Cor}{Corollary}
\theoremstyle{plain}       \newtheorem{Exa}{Example}[section]
{\theorembodyfont{\upshape}\newtheorem{Rem}{Remark}}
\theoremstyle{marginbreak} \newtheorem{Lem}[Cor]{Lemma}
\theoremstyle{change}
\theorembodyfont{\itshape} \newtheorem{Def}[Cor]{Definition}

\theoremheaderfont{\scshape}

\RecordChanges

\begin{document}
   \DocInput{theorem.dtx}
\end{document}
%</driver>
%    \end{macrocode}
%
% \section{Definition of the Macros}
%
% If the file has been loaded before, we abort immediately. If not the
% package announces itself (this is actually done at the very top of
% the file---the way it is done isn't good style so don't copy it).
% \changes{v2.0e}{89/07/19}{Spaces removed from `typeout'.}
% \changes{v2.2b}{95/11/19}{Announce moved to top of file}
%    \begin{macrocode}
%<*package>
%\@ifundefined{theorem@style}{}{\endinput}
%\def\FMithmInfo{1995/11/19 v2.2b Theorem extension package (FMi)}
%\ProvidesPackage{theorem}[\FMithmInfo]
%    \end{macrocode}
%
%
% \subsection{Definition of theorem styles and fonts}
%
% All the definitions in this file are done globally to allow
% inputting this file inside a group.
%
% \begin{macro}{\theoremstyle}
%    Before a theorem style can be installed, the chosen style must be
%    known. For that reason, we must test to see that
%    |\th@|\meta{style} is known or, more precisely, that it is
%    different from |\relax|. If the style is not known then
%    |\th@plain| is used.
%    \begin{macrocode}
\gdef\theoremstyle#1{%
   \@ifundefined{th@#1}{\@warning
          {Unknown theoremstyle `#1'. Using `plain'}%
          \theorem@style{plain}}%
%    \end{macrocode}
%    We save the theorem style to be used in the token register
%    |\theorem@style|.
%    \begin{macrocode}
      {\theorem@style{#1}}%
%    \end{macrocode}
%    Now we ``evaluate'' the theorem style: this means, we call the
%    macro |\th@|\meta{style} which will activate the relevant
%    definitions which are contained in a separate file.  This is done
%    in a group to suppress changes to the current font.  This could
%    otherwise pose problems together with the new font selection
%    scheme\footnote{When I printed the original article using the new
%    font selection scheme I ended with a document with slanted
%    typefaces (text headings and all) simply because one of the
%    theorem styles used \texttt{\bslash sl} at toplevel.} if the
%    |\th@|\meta{style} is evaluated a second time.
% \changes{v2.1a}{90/02/15}{Group added around `th@..}
% \changes{v2.1a}{90/02/15}{Reseting of `..@begintheorem removed.}
%    \begin{macrocode}
      \begingroup
        \csname th@\the\theorem@style \endcsname
      \endgroup}
%    \end{macrocode}
% \end{macro}
%
%
% \begin{macro}{\@begintheorem}
% \changes{v2.1a}{90/02/15}{Relaxed at toplevel.}
% \begin{macro}{\@opargbegintheorem}
% \changes{v2.1a}{90/02/15}{Relaxed at toplevel.}
%    We reset |\@begintheorem| and |\@opargbegintheorem| to
%    |\relax| since these commands are no longer necessary at
%    toplevel. This will save a few tokens.
%    \begin{macrocode}
\global\let\@begintheorem\relax
\global\let\@opargbegintheorem\relax
%    \end{macrocode}
% \end{macro}
% \end{macro}
%
% \begin{macro}{\theorem@style}
%    Obviously the \textsf{token} register used above has to be
%    allocated. To assure the utmost compatibility with the
%    original \LaTeX{} definition, we set the default theorem style to
%    \texttt{plain}, which implements the usual \LaTeX{} convention.
% \changes{v2.0f}{89/07/20}{Unnecessary `global' removed.}
%    \begin{macrocode}
\newtoks\theorem@style
\global\theorem@style{plain}
%    \end{macrocode}
% \end{macro}
%
% \begin{macro}{\theorembodyfont}
% \begin{macro}{\theorem@bodyfont}
%    If the |\theorembodyfont| is set by the user then it should not
%    interact with the default font set in the theorem style. When the
%    new font selection is in force this may happen if, for example,
%    the default is |\itshape| and the new |\theorembodyfont| is
%    |\sffamily|. So we add a |\reset@font| command in front of the
%    user definition.
% \changes{v2.0f}{89/07/20}{Unnecessary `global' removed.}
% \changes{v2.0g}{89/09/19}{`theoremfont to `theorembodyfont.}
% \changes{v2.1c}{92/06/22}{`reset@font added and logic changed.}
% \changes{v2.1d}{92/10/28}{Reset to empty if arg is empty}
%    \begin{macrocode}
\gdef\theorembodyfont#1{%
%    \end{macrocode}
%    We check if the argument supplied is empty and if so put nothing
%    into the |\theorem@bodyfont| token register to allow for
%    |\theorembodyfont{}| as a mean of using the default of the
%    current |\theoremstyle|.
%    \begin{macrocode}
   \def\@tempa{#1}%
   \ifx\@tempa\@empty
    \theorem@bodyfont{}%
   \else
    \theorem@bodyfont{\reset@font#1}%
   \fi
   }
\newtoks\theorem@bodyfont
\global\theorem@bodyfont{}
%    \end{macrocode}
% \end{macro}
% \end{macro}
%
%
% \begin{macro}{\theoremheaderfont}
% \changes{v2.0g}{89/09/19}{Macro added.}
%    The font for the theorem headers is handled differently because
%    this definition applies to all theorem styles.
%    \begin{macrocode}
\gdef\theoremheaderfont#1{\gdef\theorem@headerfont{#1}%
%    \end{macrocode}
%    After using the macro once it is redefined to produce an error
%    message.
%    \begin{macrocode}
       \gdef\theoremheaderfont##1{%
        \typeout{\string\theoremheaderfont\space should be used
                 only once.}}}
%    \end{macrocode}
% \end{macro}
%
% \begin{macro}{\theorem@headerfont}
% \changes{v2.0g}{89/09/19}{Macro added.}
%    To set the |\theorem@headerfont| default we first test if
%    the new fontselection scheme is in force.
% \changes{v2.1a}{90/02/15}{Support for new font selection.}
%    \begin{macrocode}
\ifx\upshape\undefined
%    \end{macrocode}
%    If not we define it to expand into |\bfseries|. We don't use
%    |\let| just in case a following style option redefines this
%    macro.
%    \begin{macrocode}
\gdef\theorem@headerfont{\bfseries}
%    \end{macrocode}
%    Otherwise we reset the current shape before calling |\bfseries|.
%    \begin{macrocode}
\else \gdef\theorem@headerfont{\normalfont\bfseries}\fi
%    \end{macrocode}
% \end{macro}
%
% \begin{macro}{\th@plain}
% \begin{macro}{\th@break}
% \begin{macro}{\th@marginbreak}
% \begin{macro}{\th@changebreak}
% \begin{macro}{\th@change}
% \begin{macro}{\th@margin}
%    The different styles are defined in macros such as
%    |\th@plain|.  Since memory space is precious in
%    ``non-Big-versions'', we have to avoid offering too many unused
%    definitions.  Therefore we define these styles in separate files
%    that can be loaded on demand.  Thus the commands themselves only
%    load these files. We use |\@input@|, a \LaTeXe{} internal command
%    that ensures that the file will be listed with |\listfiles|.
% \changes{v2.2a}{94/02/02}{Use `@input@ to load theorem layout files}
%    \begin{macrocode}
\gdef\th@plain{\@input@{thp.sty}}
\gdef\th@break{\@input@{thb.sty}}
\gdef\th@marginbreak{\@input@{thmb.sty}}
\gdef\th@changebreak{\@input@{thcb.sty}}
\gdef\th@change{\@input@{thc.sty}}
\gdef\th@margin{\@input@{thm.sty}}
%    \end{macrocode}
%    This list will be expanded when new styles become
%    available.  For testing,  just append new theorem substyles
%    as document options.
% \end{macro}
% \end{macro}
% \end{macro}
% \end{macro}
% \end{macro}
% \end{macro}
%
%
%
% \subsection{Definition of a new theorem set}
%
% As already pointed out, a new theorem environment can be defined in
% three different ways:
% \begin{verbatim}
%    \newtheorem{Lem}{Lemma}
%    \newtheorem{Lem}{Lemma}[section]
%    \newtheorem{Lem}[Theorem]{Lemma}
%\end{verbatim}
% The function of the macro |\newtheorem| is to recognize these cases
% and then to branch into one of the three macros |\@ynthm|, |\@xnthm|
% or |\@othm|.  This mechanism is adopted unchanged from
% \cite{latex:source}; the essential point here is that, for example,
% in the second case, the arguments \texttt{Lem}, \texttt{Lemma} and
% \texttt{section} are passed over to the macro |\@xnthm|.
%
% We inspect this case first because the others present fewer problems,
% and thus are easily derived from this one.
%
% \begin{macro}{\@xnthm}
%    For our example arguments, the macro |\@xnthm| must fulfill
%    the following:
%    \begin{itemize}
%       \item
%          Define a new \LaTeX-counter `Lem'
%       \item
%          reset this counter within a |\section|
%       \item
%          define the macro |\theLem|
%       \item
%          define the environment macros |\Lem| and
%          |\endLem| using the current
%          |\theoremstyle| and |\theorem@bodyfont|.
%    \end{itemize}
%    Obviously, all this should happen only if the first argument
%    of |\@xnthm| (i.e.\ \texttt{Lem} in our example) is chosen so
%    as not to conflict with any previously defined commands or
%    environments.  This test is performed by the \LaTeX{} macro
%    |\@ifdefinable|.
%    \begin{macrocode}
\gdef\@xnthm#1#2[#3]{\expandafter\@ifdefinable\csname #1\endcsname
%    \end{macrocode}
%    Therefore, the first argument of |\@ifdefinable| is the
%    expansion (in the example, |\Lem|) of
%    |\csname||#1\endcsname|.  The second argument is
%    executed only if the test has been completed successfully.
%    \begin{macrocode}
   {%
%    \end{macrocode}
%    Now we define the new counter. The names of the \LaTeX{}
%    macros employed should speak for themselves:
% \changes{v2.2b}{1995/10/07}{Use \cs{@newctr}}
% \changes{v2.2c}{1995/11/23}{Use \cs{@newctr} properly.}
%    \begin{macrocode}
    \@definecounter{#1}\@newctr{#1}[#3]%
%    \end{macrocode}
%    Using |\@newctr| will give a proper error message if the counter
%    in |#3| is not defined.
%    In defining `|\theLem|' we must generate the desired macro
%    name by use of |\expandafter| and |\csname|.
%    \begin{macrocode}
    \expandafter\xdef\csname the#1\endcsname
%    \end{macrocode}
%    An |\xdef| is used in order to make the definition global,
%    and to ensure that it contains the replacement texts of
%    |\@thmcountersep| and |\@thmcounter|.\footnote{These
%    two macros can be defined by the document style.  Their default
%    values produce a `\texttt{.}' as separation and an arabic
%    representation of the number.} However, not everything should be
%    expanded. For example, it saves space to use |\thesection|
%    instead of its---at times---lengthy expansion.
%    \begin{macrocode}
      {\expandafter \noexpand \csname the#3\endcsname
       \@thmcountersep \@thmcounter{#1}}%
%    \end{macrocode}
%    Thus with the defaults of \LaTeX{}, |\theLem| would be
%    replaced by the command sequence |\thesection.\arabic{Lem}|.
%
%    We will now look at the definition of the macro which is executed
%    at the beginning of the actual environment (in our example this
%    macro is |\Lem|). It should be noted that we use an
%    ``|\expandafter| trick'' to expand only certain parts of the
%    replacement text at the time of the definition.
%    \begin{macrocode}
    \def\@tempa{\global\@namedef{#1}}%
    \expandafter \@tempa \expandafter{%
%    \end{macrocode}
%    First, the macro that contains the current definitions of
%    |\@begintheorem| and |\@opargtheorem| should be called
%    up.  The name of this macro---as is already known---has the form
%    |\th@|\meta{theorem\ style}; therefore, it must be called by
%    \begin{macrocode}
      \csname th@\the \theorem@style
%    \end{macrocode}
%    In addition the default theorem font should be changeable, i.e.\
%    we have to insert the contents of |\theorem@bodyfont|. For
%    that reason, we expand even further, beyond |\endcsname|,
%    and thus insert the contents of the \textsf{token} register
%    |\theorem@bodyfont| in the replacement text.
%    \begin{macrocode}
            \expandafter \endcsname \the \theorem@bodyfont
%    \end{macrocode}
%    Now it is time to call the macro |\@thm| which takes over
%    the further processing.  It has two arguments: the current
%    counter name (in our example, \texttt{Lem}), and the text of the
%    label (in our example, \texttt{Lemma}).
%    \begin{macrocode}
     \@thm{#1}{#2}}%
%    \end{macrocode}
%    With this, the `sub-definition' is complete.  The macro
%    |\@endtheorem| ends a theorem environment and is, so far, nothing
%    but an |\endtrivlist|.  (Hence it is defined globally, and not
%    within the theorem styles.\footnote{This has to be changed as
%    soon as theorem styles that change \texttt{\bslash @endtheorem}
%    exist.  In such a case, all existing styles must be changed as
%    well since they will have to reset the macro.}) Therefore, we can
%    set it equivalent to the macro that ends the theorem set (in our
%    example, |\endLem|).  However, if some day theorem styles exist
%    that do change |\@endtheorem|, we would have to use the
%    commented-out line instead.
%    \begin{macrocode}
    \global \expandafter \let \csname end#1\endcsname \@endtheorem
%   \global\@namedef{end#1}{\@endtheorem}%
%    \end{macrocode}
%    With these commands all the required definitions are employed,
%    unless the test |\@ifdefinable| has failed.  Therefore, we
%    end the second argument of this macro and with it the definition
%    of |\@xnthm|.
%    \begin{macrocode}
   }}
%    \end{macrocode}
% \end{macro}
%
% \begin{macro}{\@ynthm}
%    The definition of |\@ynthm| is completely analogous. In
%    this case the new counter that is defined is not reset within
%    another counter; thus the definition of |\the...|
%    is simplified:
%    \begin{macrocode}
\gdef\@ynthm#1#2{\expandafter\@ifdefinable\csname #1\endcsname
   {\@definecounter{#1}%
    \expandafter\xdef\csname the#1\endcsname{\@thmcounter{#1}}%
%    \end{macrocode}
%    The rest of the definition corresponds literally to that of
%    |\@xnthm|:
%    \begin{macrocode}
    \def\@tempa{\global\@namedef{#1}}\expandafter \@tempa
     \expandafter{\csname th@\the \theorem@style \expandafter
     \endcsname \the\theorem@bodyfont \@thm{#1}{#2}}%
    \global \expandafter \let \csname end#1\endcsname \@endtheorem}}
%    \end{macrocode}
% \end{macro}
%
% \begin{macro}{\@othm}
%    The definition of |\@othm| does not contain anything new.
%    \begin{macrocode}
\gdef\@othm#1[#2]#3{%
%    \end{macrocode}
% \changes{v2.2b}{95/10/07}{Build in check for counter defined}
%    We do not define a new counter but instead use one that has
%    already been defined.  Thus the only definition we need is that
%    of this pseudo-counter (i.e.\ |\the|\meta{env.\ name}).
%    First we check if |#2| corresponds to a known counter name.
%    \begin{macrocode}
  \expandafter\ifx\csname c@#2\endcsname\relax
   \@nocounterr{#2}%
  \else
   \expandafter\@ifdefinable\csname #1\endcsname
   {\expandafter \xdef \csname the#1\endcsname
     {\expandafter \noexpand \csname the#2\endcsname}%
%    \end{macrocode}
%    All other parts of the definition can be adopted from
%    |\@xnthm|. We have to remember, though, that in this case
%    the name of the current counter and the theorem label have moved
%    to the second and third arguments.
%    \begin{macrocode}
    \def\@tempa{\global\@namedef{#1}}\expandafter \@tempa
     \expandafter{\csname th@\the \theorem@style \expandafter
     \endcsname \the\theorem@bodyfont \@thm{#2}{#3}}%
    \global \expandafter \let \csname end#1\endcsname \@endtheorem}%
  \fi}
%    \end{macrocode}
% \end{macro}
%
% \subsection{Macros that are employed in a {\envfont theorem}
%    environment}
%
% \begin{macro}{\@thm}
%    The macro |\@thm| has to increase the current counter.  Then,
%    depending on whether the environment has (or does not have) an
%    optional argument, it has to branch into either |\@begintheorem|
%    or |\@opargtheorem|.
%    \begin{macrocode}
\gdef\@thm#1#2{\refstepcounter{#1}%
%    \end{macrocode}
%    Now we start a {\envfont trivlist} environment, and give
%    |\@topsep| and |\@topsepadd| the values of the \textsf{skip}
%    registers |\theorempreskipamount| and |\theorempostskipamount|.
%    The value in |\@topsep| is the vertical space that is inserted by
%    the first (and only) |\item| in our |\trivlist| whilst
%    |\@topsepadd| is inserted by |\@endparenv| at the end of that
%    {\envfont trivlist} environment.  By using these registers, we
%    obtain the desired space around a {\envfont theorem} environment.
%    \begin{macrocode}
   \trivlist
   \@topsep \theorempreskipamount               % used by first \item
   \@topsepadd \theorempostskipamount           % used by \@endparenv
%    \end{macrocode}
%    Now we have to test whether an optional argument has been given.
% \changes{v2.0d}{89/07/18}{Forgotten percent added.}
%    \begin{macrocode}
   \@ifnextchar [%
%    \end{macrocode}
%    If there is an optional argument, we will call |\@ythm|, and
%    move the arguments read back into the input stream.
%    \begin{macrocode}
   {\@ythm{#1}{#2}}%
%    \end{macrocode}
%    If not, we call |\@begintheorem|. Its first argument is the
%    name of the theorem set (hence the second argument of
%    |\@thm|). Its second argument is the macro that produces the
%    current number.
%    \begin{macrocode}
   {\@begintheorem{#2}{\csname the#1\endcsname}\ignorespaces}}
%    \end{macrocode}
% \end{macro}
%
% \begin{macro}{\@xthm}
% \begin{macro}{\@ythm}
%    Both these macros were originally called by |\@thm|.  We do
%    not need |\@xthm| anymore, hence we reset it to
%    |\relax|. The definition of |\@ythm| has not changed at
%    all from its definition in \LaTeX. In order to make the macros
%    easier to understand, we will
%    nevertheless present it (commented out).
%    \begin{macrocode}
\global\let\@xthm\relax
% \def\@ythm#1#2[#3]{\@opargbegintheorem{#2}{\csname
%           the#1\endcsname}{#3}\ignorespaces}
%</package>
%    \end{macrocode}
%    The primitive |\ignorespaces| in |\@ythm| and
%    |\@thm| is needed to remove the spaces between the
%    |\begin{...}| and the actual text.
% \end{macro}
% \end{macro}
%
%
% \subsection{Definition of the theorem substyles}
%
% As already pointed out, the theorem substyles, defined below, are
% only loaded when necessary.  Note that all these substyles, except
% \texttt{plain}, have |\slshape| as the default body font.
%
% \subsubsection{The \texttt{plain} style}
%
% \label{dok:th@plain} As the following macros use |@|, we have
% to locally set the |\catcode| of this symbol to ``letter''.
% This happens within a group, so that we do not have to worry about
% which |\catcode| that symbol had before.
%    \begin{macrocode}
%<*thp>
\begingroup \makeatletter
%    \end{macrocode}
% Since we are now within a group, we must make all definitions
% globally.  First we make sure that \texttt{theorem.sty} is loaded.
% This will allow us to use this file as a document style option
% without having to call \texttt{theorem} itself as an option.  At the
% same time, we assure that at least version 2 is loaded, since
% |\theorem@style| was not defined in earlier versions.
%    \begin{macrocode}
\@ifundefined{theorem@style}{\input{theorem.sty}}{}
\ProvidesFile{thp.sty}
             [\FMithmInfo]
%    \end{macrocode}
%
%
% \begin{macro}{\th@plain}
% \begin{macro}{\@begintheorem}
% \begin{macro}{\@opargbegintheorem}
%    |\theoremstyle{plain}| corresponds to the original
%    definition, except that the distances to the surrounding text are
%    determined by the parameters |\theorempreskipamount| and
%    |\theorempostskipamount|.
%    First we set the default body font.
%    \begin{macrocode}
\gdef\th@plain{\normalfont\itshape
%    \end{macrocode}
%    Then we define |\@begintheorem| and
%    |\@opargbegintheorem|. These two macros define how the
%    header of a theorem is typeset.  |\@opargbegintheorem| will
%    be called if a \textsf{theorem} environment with an optional
%    argument is encountered; otherwise, the header is constructed by
%    calling |\@begintheorem|.  If one of these macros is
%    executed, we are within a \textsf{trivlist} environment started by
%    |\@thm|.  So the theorem header is produced with an
%    |\item| command.
%
%    Instead of specifying the header font directly, all standard
%    theorem styles use the |\theorem@headerfont| macro to allow
%    customization.  The extra space (|\labelsep|) is necessary
%    because of problems in the \textsf{trivlist} environment.
%    \begin{macrocode}
  \def\@begintheorem##1##2{%
        \item[\hskip\labelsep \theorem@headerfont ##1\ ##2]}%
%    \end{macrocode}
%    The definition of |\@opargbegintheorem| is completely
%    analogous. The only difference is the fact that there exists a
%    third argument (which is the optional parameter of the
%    environment and contains additional information about the
%    theorem).  Customarily we enclose it in parentheses.
%    \begin{macrocode}
\def\@opargbegintheorem##1##2##3{%
   \item[\hskip\labelsep \theorem@headerfont ##1\ ##2\ (##3)]}}
%    \end{macrocode}
% \end{macro}
% \end{macro}
% \end{macro}
%
% \noindent
% We conclude with an |\endgroup| to restore the
% |\catcode| of |@|.
%    \begin{macrocode}
\endgroup
%</thp>
%    \end{macrocode}
%
%
% \subsubsection{The \texttt{break} style}
%
% This style option is stored in the file |thb.sty|.
% For the next two lines see the documentation for |\th@plain| on
% page~\pageref{dok:th@plain}.
%    \begin{macrocode}
%<*thb>
\begingroup \makeatletter
\@ifundefined{theorem@style}{\input{theorem.sty}}{}
\ProvidesFile{thb.sty}
             [\FMithmInfo]
%    \end{macrocode}
%
% \begin{macro}{\th@break}
%    |\theoremstyle{break}| produces a line break after the name
%    of the theorem.  The font is |\slshape|.  Hence, we define
%    |\th@break| as follows:
%    \begin{macrocode}
\gdef\th@break{\normalfont\slshape
  \def\@begintheorem##1##2{\item[%
%    \end{macrocode}
%    We run into the following problem: it is not possible to create
%    the header with |\item[|\meta{title}|]| and then start
%    a new line by, for example, |\mbox{}\\|. Such a definition
%    will fail whenever a list environment follows immediately.  With
%    the above construction, the |\mbox{}| causes the switch
%    |@inlabel| (cf.\ definition of |\list| and
%    |\trivlist| in \cite{latex:source}) to be set to \texttt{false}
%    and so the following list will insert additional vertical
%    space (|\topskip|). This is quite annoying. Therefore, we
%    create the line break within the |\item|. In order to ensure
%    that the text will begin at the proper position in the following
%    line, we simply pretend that the label does not take any
%    room.\footnote{This will lead to problems whenever very high
%    symbols occurring in the line tower into the
%    heading. So, something else has to be done here sometime.}
%    \begin{macrocode}
       \rlap{\vbox{\hbox{\hskip \labelsep\theorem@headerfont ##1\ ##2}%
                   \hbox{\strut}}}]}%
%    \end{macrocode}
%    Again, the definition of |\@opargbegintheorem| is
%    completely analogous.
%    \begin{macrocode}
\def\@opargbegintheorem##1##2##3{%
  \item[\rlap{\vbox{\hbox{\hskip \labelsep \theorem@headerfont
                     ##1\ ##2\ (##3)}%
                    \hbox{\strut}}}]}}
\endgroup
%</thb>
%    \end{macrocode}
% \end{macro}
%
%
% \subsubsection{The \texttt{changebreak} style}
%
%    \begin{macrocode}
% This style option is stored in the file |thcb.sty|.
%    \begin{macrocode}
%<*thcb>
\begingroup \makeatletter
\@ifundefined{theorem@style}{\input{theorem.sty}}{}
\ProvidesFile{thcb.sty}
             [\FMithmInfo]
%    \end{macrocode}
%
% \begin{macro}{\th@changebreak}
%    The \texttt{change-break} theorem style is like \texttt{break}
%    but with interchange of theorem name and theorem number.  Thus we
%    define |\th@changebreak| as follows:
%    \begin{macrocode}
\gdef\th@changebreak{\normalfont\slshape
  \def\@begintheorem##1##2{\item
       [\rlap{\vbox{\hbox{\hskip\labelsep \theorem@headerfont ##2\ ##1}%
                    \hbox{\strut}}}]}%
\def\@opargbegintheorem##1##2##3{%
  \item[\rlap{\vbox{\hbox{\hskip\labelsep \theorem@headerfont
                     ##2\ ##1\ (##3)}%
                    \hbox{\strut}}}]}}
\endgroup
%</thcb>
%    \end{macrocode}
% \end{macro}
%
%
% \subsubsection{The \texttt{change} style}
%
% This style option is stored in the file |thc.sty|.
%    \begin{macrocode}
%<*thc>
\begingroup \makeatletter
\@ifundefined{theorem@style}{\input{theorem.sty}}{}
\ProvidesFile{thc.sty}
             [\FMithmInfo]
%    \end{macrocode}
%
% \begin{macro}{\th@change}
%    The \texttt{change} theorem style corresponds to the
%    \texttt{change break} style without a linebreak after the header.
%    To say it in another way, it's the same as the \texttt{plain}
%    style but with number and name interchanged and |\slshape| as the
%    default font.
%    \begin{macrocode}
\gdef\th@change{\normalfont\slshape
  \def\@begintheorem##1##2{\item
       [\hskip\labelsep \theorem@headerfont ##2\ ##1]}%
\def\@opargbegintheorem##1##2##3{%
  \item[\hskip\labelsep \theorem@headerfont ##2\ ##1\ (##3)]}}
\endgroup
%</thc>
%    \end{macrocode}
% \end{macro}
%
%
% \subsubsection{The \texttt{marginbreak} style}
%
% This style option is the one used most often at Mainz.  It is saved
% in the file |thmb.sty|.
%    \begin{macrocode}
%<*thmb>
\begingroup \makeatletter
\@ifundefined{theorem@style}{\input{theorem.sty}}{}
\ProvidesFile{thmb.sty}
             [\FMithmInfo]
%    \end{macrocode}
%
% \begin{macro}{\th@marginbreak}
%    The \texttt{margin break} style is nearly the same as the
%    \texttt{change break} style. The only difference is the placement
%    of the theorem number.  We use |\llap| to place it in the left
%    margin.
%
%    In this style |\labelsep| denotes the separation between the
%    number and the text.
%    \begin{macrocode}
\gdef\th@marginbreak{\normalfont\slshape
  \def\@begintheorem##1##2{\item
       [\rlap{\vbox{\theorem@headerfont
                    \hbox{\llap{##2}\hskip\labelsep ##1}%
                    \hbox{\strut}}}]}%
\def\@opargbegintheorem##1##2##3{%
  \item[\rlap{\vbox{\theorem@headerfont
                    \hbox{\llap{##2}\hskip\labelsep ##1\ (##3)}%
                    \hbox{\strut}}}]}}
\endgroup
%</thmb>
%    \end{macrocode}
% \changes{v2.1c}{90/02/17}{`endgroup outside macrocode env.}
% \end{macro}
%
%
%
% \subsubsection{The \texttt{margin} style}
%
% This style option is stored in the file |thm.sty|.
%    \begin{macrocode}
%<*thm>
\begingroup \makeatletter
\@ifundefined{theorem@style}{\input{theorem.sty}}{}
\ProvidesFile{thm.sty}
             [\FMithmInfo]
%    \end{macrocode}
%
% \begin{macro}{\th@margin}
%    Again this is only a variant of the theorem styles described
%    above without any new ideas.
%    \begin{macrocode}
\gdef\th@margin{\normalfont\slshape
  \def\@begintheorem##1##2{\item
       [\theorem@headerfont \llap{##2}\hskip\labelsep ##1]}%
\def\@opargbegintheorem##1##2##3{%
  \item[\theorem@headerfont \llap{##2}\hskip\labelsep ##1\ (##3)]}}
\endgroup
%</thm>
%    \end{macrocode}
% \end{macro}
%
%
%
%
% \subsection{Final Definitions}
%
% \begin{macro}{\theorempreskipamount}
% \begin{macro}{\theorempostskipamount}
%    The \textsf{skip} parameters that regulate the vertical empty space
%    before and after the {\envfont theorem} environment have
%    to be allocated as well.
% \changes{v2.0f}{89/07/20}{Unnecessary `global' removed.}
%    \begin{macrocode}
%<*package>
\newskip\theorempreskipamount
\newskip\theorempostskipamount
%    \end{macrocode}
%    Since we have used the same values for all theorem sets, we now can
%    assign them.
%    \begin{macrocode}
\global\setlength\theorempreskipamount{12pt plus 5pt minus 3pt}
\global\setlength\theorempostskipamount{8pt plus 3pt minus 1.5pt}
%    \end{macrocode}
% \end{macro}
% \end{macro}
%
%
% \begin{macro}{\@endtheorem}
%    The same holds for the macro |\@endtheorem|, which ends a
%    {\envfont theorem} environment. Since it is the same for all
%    theorem sets, it is removed from the macros |\th@|\meta{style}.
%    It simply ends the {\envfont trivlist} environment, which was
%    begun in |\@thm|.
%    \begin{macrocode}
\global\let\@endtheorem=\endtrivlist
%    \end{macrocode}
% \end{macro}
%
%
%
% \begin{macro}{\@preamblecmds}
%    All macros defined above are to be used only in the preamble.
%    Therefore, we insert them in |\@preamblecmds| which will disable
%    them at begin document. This is done by the internal \LaTeXe{}
%    command |\@onlypreamble|.
% \changes{v2.0g}{89/09/19}{`theoremheaderfont and `theorembodyfont
%    added.}
% \changes{v2.2a}{94/02/03}{Use \LaTeXe{} interface}
%    \begin{macrocode}
\@onlypreamble\@xnthm
\@onlypreamble\@ynthm
\@onlypreamble\@othm
\@onlypreamble\newtheorem
\@onlypreamble\theoremstyle
\@onlypreamble\theorembodyfont
\@onlypreamble\theoremheaderfont
%    \end{macrocode}
% \end{macro}
%
% Finally we declare the \texttt{plain} theorem style to be the default.
% \changes{v2.1b}{90/06/04}{Fixed missing `item by declaring default
%    style.}
%    \begin{macrocode}
\theoremstyle{plain}
%</package>
%    \end{macrocode}
%
% \Finale

\endinput
}{}
\ProvidesFile{thb.sty}
             [\FMithmInfo]
%    \end{macrocode}
%
% \begin{macro}{\th@break}
%    |\theoremstyle{break}| produces a line break after the name
%    of the theorem.  The font is |\slshape|.  Hence, we define
%    |\th@break| as follows:
%    \begin{macrocode}
\gdef\th@break{\normalfont\slshape
  \def\@begintheorem##1##2{\item[%
%    \end{macrocode}
%    We run into the following problem: it is not possible to create
%    the header with |\item[|\meta{title}|]| and then start
%    a new line by, for example, |\mbox{}\\|. Such a definition
%    will fail whenever a list environment follows immediately.  With
%    the above construction, the |\mbox{}| causes the switch
%    |@inlabel| (cf.\ definition of |\list| and
%    |\trivlist| in \cite{latex:source}) to be set to \texttt{false}
%    and so the following list will insert additional vertical
%    space (|\topskip|). This is quite annoying. Therefore, we
%    create the line break within the |\item|. In order to ensure
%    that the text will begin at the proper position in the following
%    line, we simply pretend that the label does not take any
%    room.\footnote{This will lead to problems whenever very high
%    symbols occurring in the line tower into the
%    heading. So, something else has to be done here sometime.}
%    \begin{macrocode}
       \rlap{\vbox{\hbox{\hskip \labelsep\theorem@headerfont ##1\ ##2}%
                   \hbox{\strut}}}]}%
%    \end{macrocode}
%    Again, the definition of |\@opargbegintheorem| is
%    completely analogous.
%    \begin{macrocode}
\def\@opargbegintheorem##1##2##3{%
  \item[\rlap{\vbox{\hbox{\hskip \labelsep \theorem@headerfont
                     ##1\ ##2\ (##3)}%
                    \hbox{\strut}}}]}}
\endgroup
%</thb>
%    \end{macrocode}
% \end{macro}
%
%
% \subsubsection{The \texttt{changebreak} style}
%
%    \begin{macrocode}
% This style option is stored in the file |thcb.sty|.
%    \begin{macrocode}
%<*thcb>
\begingroup \makeatletter
\@ifundefined{theorem@style}{% \iffalse meta-comment
%
% Copyright (C) 1993-2024
%
% The LaTeX Project and any individual authors listed elsewhere
% in this file.
%
% This file is part of the Standard LaTeX `Tools Bundle'.
% -------------------------------------------------------
%
% It may be distributed and/or modified under the
% conditions of the LaTeX Project Public License, either version 1.3c
% of this license or (at your option) any later version.
% The latest version of this license is in
%    https://www.latex-project.org/lppl.txt
% and version 1.3c or later is part of all distributions of LaTeX
% version 2005/12/01 or later.
%
% The list of all files belonging to the LaTeX `Tools Bundle' is
% given in the file `manifest.txt'.
%
% \fi
%
%%
%
%
\def\FMithmInfo{2023/07/05 v2.2c Theorem extension package (FMi)}
%
% \ProvidesFile{theorem.dtx}[\FMithmInfo]
%\iffalse   % this is a METACOMMENT !
%%
%% Package `theorem' to use with LaTeX2e
%% Copyright (C) 1989-1995 Frank Mittelbach, all rights reserved.
%<+package>\@ifundefined{theorem@style}{}{\endinput}
%<+package>\NeedsTeXFormat{LaTeX2e}[1995/06/01]
%<+package>\ProvidesPackage{theorem}
%<+package>         [\FMithmInfo]
%
% \fi
%
% \changes{v1.0}{87/05/07}{First draft.}
% \changes{v1.0a}{87/05/07}{`thistheoremstyle' added.}
% \changes{v1.1a}{87/07/29}{New concept: redefine the internal
%                          \protect\LaTeX{} commands `@xnthm' etc.}
% \changes{v1.1b}{87/07/31}{`theoremfont' used to specify font.}
% \changes{v1.2a}{88/01/08}{`@endtheorem' now globally defined.}
% \changes{v1.2b}{88/01/11}{`newtheorem' only allowed in the preamble.}
% \changes{v1.3a}{88/02/14}{`theoremfont' now a toks register.}
%
% \changes{v2.0a}{88/11/16}{New implementation and new documentation.}
% \changes{v2.0a}{88/11/16}{No layout changes but internal modifications
%                           old substyles won't work.}
% \changes{v2.0b}{89/01/30}{`global' used for all definitions.}
% \changes{v2.0c}{89/06/15}{Documentation translated to English.
%                            Thanks to LLL and Nina}
% \changes{v2.1a}{90/02/15}{Support for new font selection.}
% \changes{v2.2a}{94/02/03}{Cleanup for LaTeX2e}
%
% \DoNotIndex{\@endparenv,\@ifdefinable,\@ifnextchar,\@ifundefined}
% \DoNotIndex{\@namedef,\@nameuse,\@spaces,\@tempa,\@warning}
% \DoNotIndex{\ }
% \DoNotIndex{\begingroup,\bfseries}
% \DoNotIndex{\csname}
% \DoNotIndex{\def,\do,\docdate}
% \DoNotIndex{\endcsname,\endgroup,\endinput,\endtrivlist,\expandafter}
% \DoNotIndex{\filedate,\fileversion}
% \DoNotIndex{\gdef,\global}
% \DoNotIndex{\hbox,\hskip}
% \DoNotIndex{\ignorespaces,\input,\itshape,\item}
% \DoNotIndex{\labelsep,\let,\llap}
% \DoNotIndex{\makeatletter}
% \DoNotIndex{\noexpand}
% \DoNotIndex{\refstepcounter,\relax,\rlap}
% \DoNotIndex{\setlength,\slshape,\space,\string,\strut}
% \DoNotIndex{\the,\trivlist,\typeout}
% \DoNotIndex{\vbox}
% \DoNotIndex{\xdef}
%
% \def\deflabel#1{\ttfamily #1\hfill}
% \def\deflist#1{\begin{list}{}{\settowidth\labelwidth{\ttfamily #1}%
%                               \setlength\leftmargin\labelwidth
%                               \addtolength\leftmargin\labelsep
%                               \let\makelabel\deflabel}}
% \def\enddeflist{\end{list}}
% \def\envfont{\normalfont\ttfamily}
%
% \MakeShortVerb{\|}
%
% \newcommand{\TUB}{{\sl TUGboat\/}}
%
% \GetFileInfo{theorem.dtx}
% \title{An Extension of the
%        \LaTeX{} theorem environment\thanks{This file
%        has version number \fileversion, last
%        revised \filedate.}}
% \author{Frank Mittelbach}
%
% \MaintainedByLaTeXTeam{tools}
% \maketitle
%
% \begin{abstract}
%    The macros described in this paper yield an extension of the
%    \LaTeX{} theorem mechanism. It is designed to satisfy the
%    different requirements of various journals. Thus, the
%    layout of the ``theorems'' can be manipulated by determining a
%    ``style''. This article describes not only the use, but
%    also the definition, of the necessary macros.
% \end{abstract}
%
% \section*{Preface to version 2.2}
%
% For \LaTeXe{} this package did not need any fundamental changes. I
% only modified the messages generated so that theorem layout styles
% will show up with the |\listfiles| command and cleaned the section
% on the New Font Selection Scheme since this is now included in
% \LaTeX.
%
% \section*{Preface to version 2.1}
%
% This version is identical to 2.0g described in \TUB\ 10\#3 except
% for some internal defaults which are now set depending on the used
% font selection scheme.
%
% This was done to avoid unpleasant surprises if the new font selection
% scheme is in force. For further details see
% section~\ref{sec:newfonts} and \cite{bk:GMS94}.
%
% \section{Introduction}
%
% For our purposes here, ``theorems'' are labelled enunciations,
% often set off from the main text by extra space and a font change.
% Theorems, corollaries, conjectures, definitions, and remarks
% are all instances of ``theorems''.  The ``header'' of these structures
% is composed of a label (such as \textsc{Theorem} or \textsc{Remark})
% and a number which serializes an item in the sequence of items
% with the same label.
%
% Shortly after the introduction of \LaTeX{} at the Fachbereich
% Mathematik in Mainz, the desire to manipulate the layout of
% ``theorems'' arose.  In Mainz, the following two conventions came
% into general use:
% \begin{enumerate}
%   \item The number of the theorem is shown in the margin.
%   \item There is a line break at
%         the end of the theorem header.
% \end{enumerate}
%
% Additionally, some journals require different formats which depend on
% the ``sort of theorem'': e.g.\ often remarks and definitions are set
% in |\upshape|, while |\itshape| is employed for main theorems.
%
% Confronted with these requirements, a theorem environment was
% developed in Mainz which allows separate determination of the
% layout of the ``theorems sets'', comparable to |\pagestyle|.
%
%
% \section{The user interface}
% \subsection{Defining new theorem sets}
%
% \DescribeMacro\newtheorem
% As in the original \LaTeX{} version, the command |\newtheorem|
% defines a new ``theorem set'' or ``theorem-like structure''.
% Two required arguments name the new environment and give the text
% to be typeset with each instance of the new ``set'', while
% an optional argument determines how
% the ``set'' is enumerated:
% \begin{description}
%    \item[\ttfamily \bslash newtheorem\{foo\}\{bar\}]
%       The theorem set {\envfont foo} (whose name is \texttt{bar})
%       uses its own counter.
%    \item[\ttfamily \bslash newtheorem\{foo2\}{[foo]}\{bar2\}]
%       The theorem set {\envfont foo2} (printed name \texttt{bar2})
%       uses the same counter as the theorem set \texttt{foo}.
%    \item[\ttfamily \bslash newtheorem\{foo3\}\{bar3\}{[section]}]
%       The theorem set {\envfont foo3} (printed name \texttt{bar3}) is
%       enumerated within the counter \texttt{section}, i.e.\ with every
%       new |\section| the enumeration begins again with 1, and
%       the enumeration is composed from the section-number and the
%       theorem counter itself.
% \end{description}
%
% \DescribeMacro\theoremstyle
% Additionally, the command |\theoremstyle| can define the layout
% of various, or all, theorem sets.  It should be noted that any
% theorem set defined by |\newtheorem| is typeset in the
% |\theoremstyle| that is current at the time of the definition.
% Thus, the following
% \begin{verbatim}
%    \theoremstyle{break}        \newtheorem{Cor}{Corollary}
%    \theoremstyle{plain}        \newtheorem{Exa}{Example}[section]
%\end{verbatim}                             ^^A No blank at line start
% leads to the result that the set {\envfont Cor} is formatted in the
% style \texttt{break}, while the set {\envfont Exa} and all the
% following ones are formatted in the style \texttt{plain}, unless
% another |\theoremstyle| follows.  Since the definitions installed by
% |\newtheorem| are global, one also can limit |\theoremstyle| locally
% by grouping braces.
%
% \DescribeMacro\theorembodyfont
% The choice of the font for the theorem body is completely
% independent of the chosen |\theoremstyle|; this has proven to
% be very advantageous.  For example,
% \begin{verbatim}
%   {\theorembodyfont{\upshape}          \newtheorem{Rem}{Remark}}
%\end{verbatim}
% defines a theorem set {\envfont Rem}, which will be set in
% |\upshape| in the current layout (which in our example is
% \texttt{plain}).  As with |\theoremstyle|, the |\theorembodyfont|
% chosen is that current at the time of |\newtheorem|.  If
% |\theorembodyfont| is not specified or one defines
% |\theorembodyfont{}|, then the font used will be that defined by the
% |\theoremstyle|.
%
% \DescribeMacro\theoremheaderfont It is also possible to customize
% the font used for the theorem headers.  This is, however, a global
% declaration, and therefore there should be at most one
% |\theoremheaderfont| declaration in the preamble.\footnote{If it is
% actually necessary to have different header fonts, one has to define
% new theorem styles (substituting the desired font) or specify the
% information directly in the \texttt{\bslash newtheorem} declaration
% (the unclean variant).}
%
% \DescribeMacro\theorempreskipamount
% \DescribeMacro\theorempostskipamount Two additional parameters
% affect the vertical space around the theorem environments:
% |\theorempreskipamount| and |\theorempostskipamount| define,
% respectively, the spacing before and after such an environment.
% These parameters apply for all theorem sets and can be manipulated
% with the ordinary length macros.  They are rubber lengths
% (`\textsf{skips}'), and therefore can contain \texttt{plus} and
% \texttt{minus} parts.
%
% Since the definition of theorem sets should---most sensibly---be
% placed in the preamble, we only allow installation there.
% It is therefore possible to release the memory used here after
% |\begin{document}|, in order to make room for other
% applications.
%
%
% \subsection{Existing theorem styles}
%
% The following theorem styles exist to date:
% \begin{deflist}{marginbreak}
%    \item[plain]
%       This theorem style emulates the original \LaTeX{} definition,
%       except that additionally the parameters
%       |\theorem...skipamount| are used.
%    \item[break]
%       In this style, the theorem header is followed by a line
%       break.
%    \item[marginbreak]
%       The theorem number is set in the margin, and there is a line
%       break as in \texttt{break}.
%    \item[changebreak]
%       Like \texttt{break}, but with header number and text
%       interchanged.
%    \item[change]
%       Header number and text are interchanged, without a line break.
%    \item[margin]
%       The number is set in the left margin, without a line break.
% \end{deflist}
% All styles (except \texttt{plain}) select |\slshape| as the default
% |\theorembodyfont|.
%
%
% \subsection{Examples}
%
% Given the above theorem sets {\envfont Cor}, {\envfont Exa}
% and {\envfont Rem},
% suppose that the preamble also contains the declarations:
% \begin{verbatim}
%    \theoremstyle{marginbreak}   \newtheorem{Lem}[Cor]{Lemma}
%    \theoremstyle{change}
%    \theorembodyfont{\itshape}        \newtheorem{Def}[Cor]{Definition}
%
%    \theoremheaderfont{\scshape}
%\end{verbatim}
% Then the following are some typical examples of the typeset output
% resulting from their use.
%
% \begin{Cor}
%    This is a sentence typeset in the theorem environment {\envfont
%    Cor}.
% \end{Cor}
%
% \begin{Exa}
%    This is a sentence typeset in the theorem environment {\envfont
%    Exa}.
% \end{Exa}
%
% \begin{Rem}
%    This is a sentence typeset in the theorem environment {\envfont
%    Rem}.
% \end{Rem}
%
% \begin{Lem}[Ben User]
%    This is a sentence typeset in the theorem environment {\envfont
%    Lem}.
% \end{Lem}
%
% \begin{Def}[Very impressive Definition]
%    This is a sentence typeset in the theorem environment {\envfont
%    Def}.
% \end{Def}
%
% The last two examples show the effect of the optional argument to
% a theorem environment (it is the text typeset in parentheses).
%
%
% \section{Special Considerations}
% \label{sec:newfonts}
%
% Theoremheader and body are implemented as a unit. This means that
% the |\theoremheaderfont|  will inherit characteristics of the
% |\theorembodyfont| in \LaTeXe. Thus, if for example
% |\theorembodyfont| is |\itshape| and |\theoremheaderfont|
% is |\bfseries| the font selected for the header will have the
% characteristics `bold extended italic'. If this is not desired one
% should set the |\theoremheaderfont| to something like
% \begin{verbatim}
%      \theoremheaderfont{\normalfont\bfseries}
%\end{verbatim}
% i.e.\ supplying all necessary font information explicitly.
%
%
%
% \section{Acknowledgements}
%
% The publication of this set of macros was only possible with the
% help of Christina Busse (translating the manuscript into English),
% Joachim Pense (playing the r\^ole of typist), Chris Rowley
% (looking everything over) and many others providing useful
% suggestions.
%
% \MaybeStop{%
%    \begin{thebibliography}{9}
%       \bibitem{bk:GMS94} \textsc{M.~Goossens}, \textsc{F.~Mittelbach}
%         and \textsc{A.~Samarin}.
%         \newblock The \LaTeX{} Companion.
%         \newblock
%         Addison-Wesley, Reading, Massachusetts, 1994.
%       \bibitem{latex:source} \textsc{Lamport, Leslie.} \newblock
%                               \texttt{latex.tex}, version 2.09, date
%                               Feb. 1990.
%    \end{thebibliography}
%    \PrintIndex
%    \PrintChanges}
%
%
% \section{The documentation driver file}
%
% The next bit of code contains the documentation driver file for
% \TeX{}, i.e., the file that will produce the documentation you are
% currently reading. It will be extracted from this file by the
% \texttt{docstrip} program. Since it is the first code in the file
% one can alternatively process this file directly with \LaTeXe{} to
% obtain the documentation.
%
%    \begin{macrocode}
%<*driver>
\documentclass{ltxdoc}

\usepackage{theorem}

%<+driver>% The next few lines define theorem sets which are used
%<+driver>% in the example section of the documentation.

\theoremstyle{break}       \newtheorem{Cor}{Corollary}
\theoremstyle{plain}       \newtheorem{Exa}{Example}[section]
{\theorembodyfont{\upshape}\newtheorem{Rem}{Remark}}
\theoremstyle{marginbreak} \newtheorem{Lem}[Cor]{Lemma}
\theoremstyle{change}
\theorembodyfont{\itshape} \newtheorem{Def}[Cor]{Definition}

\theoremheaderfont{\scshape}

\RecordChanges

\begin{document}
   \DocInput{theorem.dtx}
\end{document}
%</driver>
%    \end{macrocode}
%
% \section{Definition of the Macros}
%
% If the file has been loaded before, we abort immediately. If not the
% package announces itself (this is actually done at the very top of
% the file---the way it is done isn't good style so don't copy it).
% \changes{v2.0e}{89/07/19}{Spaces removed from `typeout'.}
% \changes{v2.2b}{95/11/19}{Announce moved to top of file}
%    \begin{macrocode}
%<*package>
%\@ifundefined{theorem@style}{}{\endinput}
%\def\FMithmInfo{1995/11/19 v2.2b Theorem extension package (FMi)}
%\ProvidesPackage{theorem}[\FMithmInfo]
%    \end{macrocode}
%
%
% \subsection{Definition of theorem styles and fonts}
%
% All the definitions in this file are done globally to allow
% inputting this file inside a group.
%
% \begin{macro}{\theoremstyle}
%    Before a theorem style can be installed, the chosen style must be
%    known. For that reason, we must test to see that
%    |\th@|\meta{style} is known or, more precisely, that it is
%    different from |\relax|. If the style is not known then
%    |\th@plain| is used.
%    \begin{macrocode}
\gdef\theoremstyle#1{%
   \@ifundefined{th@#1}{\@warning
          {Unknown theoremstyle `#1'. Using `plain'}%
          \theorem@style{plain}}%
%    \end{macrocode}
%    We save the theorem style to be used in the token register
%    |\theorem@style|.
%    \begin{macrocode}
      {\theorem@style{#1}}%
%    \end{macrocode}
%    Now we ``evaluate'' the theorem style: this means, we call the
%    macro |\th@|\meta{style} which will activate the relevant
%    definitions which are contained in a separate file.  This is done
%    in a group to suppress changes to the current font.  This could
%    otherwise pose problems together with the new font selection
%    scheme\footnote{When I printed the original article using the new
%    font selection scheme I ended with a document with slanted
%    typefaces (text headings and all) simply because one of the
%    theorem styles used \texttt{\bslash sl} at toplevel.} if the
%    |\th@|\meta{style} is evaluated a second time.
% \changes{v2.1a}{90/02/15}{Group added around `th@..}
% \changes{v2.1a}{90/02/15}{Reseting of `..@begintheorem removed.}
%    \begin{macrocode}
      \begingroup
        \csname th@\the\theorem@style \endcsname
      \endgroup}
%    \end{macrocode}
% \end{macro}
%
%
% \begin{macro}{\@begintheorem}
% \changes{v2.1a}{90/02/15}{Relaxed at toplevel.}
% \begin{macro}{\@opargbegintheorem}
% \changes{v2.1a}{90/02/15}{Relaxed at toplevel.}
%    We reset |\@begintheorem| and |\@opargbegintheorem| to
%    |\relax| since these commands are no longer necessary at
%    toplevel. This will save a few tokens.
%    \begin{macrocode}
\global\let\@begintheorem\relax
\global\let\@opargbegintheorem\relax
%    \end{macrocode}
% \end{macro}
% \end{macro}
%
% \begin{macro}{\theorem@style}
%    Obviously the \textsf{token} register used above has to be
%    allocated. To assure the utmost compatibility with the
%    original \LaTeX{} definition, we set the default theorem style to
%    \texttt{plain}, which implements the usual \LaTeX{} convention.
% \changes{v2.0f}{89/07/20}{Unnecessary `global' removed.}
%    \begin{macrocode}
\newtoks\theorem@style
\global\theorem@style{plain}
%    \end{macrocode}
% \end{macro}
%
% \begin{macro}{\theorembodyfont}
% \begin{macro}{\theorem@bodyfont}
%    If the |\theorembodyfont| is set by the user then it should not
%    interact with the default font set in the theorem style. When the
%    new font selection is in force this may happen if, for example,
%    the default is |\itshape| and the new |\theorembodyfont| is
%    |\sffamily|. So we add a |\reset@font| command in front of the
%    user definition.
% \changes{v2.0f}{89/07/20}{Unnecessary `global' removed.}
% \changes{v2.0g}{89/09/19}{`theoremfont to `theorembodyfont.}
% \changes{v2.1c}{92/06/22}{`reset@font added and logic changed.}
% \changes{v2.1d}{92/10/28}{Reset to empty if arg is empty}
%    \begin{macrocode}
\gdef\theorembodyfont#1{%
%    \end{macrocode}
%    We check if the argument supplied is empty and if so put nothing
%    into the |\theorem@bodyfont| token register to allow for
%    |\theorembodyfont{}| as a mean of using the default of the
%    current |\theoremstyle|.
%    \begin{macrocode}
   \def\@tempa{#1}%
   \ifx\@tempa\@empty
    \theorem@bodyfont{}%
   \else
    \theorem@bodyfont{\reset@font#1}%
   \fi
   }
\newtoks\theorem@bodyfont
\global\theorem@bodyfont{}
%    \end{macrocode}
% \end{macro}
% \end{macro}
%
%
% \begin{macro}{\theoremheaderfont}
% \changes{v2.0g}{89/09/19}{Macro added.}
%    The font for the theorem headers is handled differently because
%    this definition applies to all theorem styles.
%    \begin{macrocode}
\gdef\theoremheaderfont#1{\gdef\theorem@headerfont{#1}%
%    \end{macrocode}
%    After using the macro once it is redefined to produce an error
%    message.
%    \begin{macrocode}
       \gdef\theoremheaderfont##1{%
        \typeout{\string\theoremheaderfont\space should be used
                 only once.}}}
%    \end{macrocode}
% \end{macro}
%
% \begin{macro}{\theorem@headerfont}
% \changes{v2.0g}{89/09/19}{Macro added.}
%    To set the |\theorem@headerfont| default we first test if
%    the new fontselection scheme is in force.
% \changes{v2.1a}{90/02/15}{Support for new font selection.}
%    \begin{macrocode}
\ifx\upshape\undefined
%    \end{macrocode}
%    If not we define it to expand into |\bfseries|. We don't use
%    |\let| just in case a following style option redefines this
%    macro.
%    \begin{macrocode}
\gdef\theorem@headerfont{\bfseries}
%    \end{macrocode}
%    Otherwise we reset the current shape before calling |\bfseries|.
%    \begin{macrocode}
\else \gdef\theorem@headerfont{\normalfont\bfseries}\fi
%    \end{macrocode}
% \end{macro}
%
% \begin{macro}{\th@plain}
% \begin{macro}{\th@break}
% \begin{macro}{\th@marginbreak}
% \begin{macro}{\th@changebreak}
% \begin{macro}{\th@change}
% \begin{macro}{\th@margin}
%    The different styles are defined in macros such as
%    |\th@plain|.  Since memory space is precious in
%    ``non-Big-versions'', we have to avoid offering too many unused
%    definitions.  Therefore we define these styles in separate files
%    that can be loaded on demand.  Thus the commands themselves only
%    load these files. We use |\@input@|, a \LaTeXe{} internal command
%    that ensures that the file will be listed with |\listfiles|.
% \changes{v2.2a}{94/02/02}{Use `@input@ to load theorem layout files}
%    \begin{macrocode}
\gdef\th@plain{\@input@{thp.sty}}
\gdef\th@break{\@input@{thb.sty}}
\gdef\th@marginbreak{\@input@{thmb.sty}}
\gdef\th@changebreak{\@input@{thcb.sty}}
\gdef\th@change{\@input@{thc.sty}}
\gdef\th@margin{\@input@{thm.sty}}
%    \end{macrocode}
%    This list will be expanded when new styles become
%    available.  For testing,  just append new theorem substyles
%    as document options.
% \end{macro}
% \end{macro}
% \end{macro}
% \end{macro}
% \end{macro}
% \end{macro}
%
%
%
% \subsection{Definition of a new theorem set}
%
% As already pointed out, a new theorem environment can be defined in
% three different ways:
% \begin{verbatim}
%    \newtheorem{Lem}{Lemma}
%    \newtheorem{Lem}{Lemma}[section]
%    \newtheorem{Lem}[Theorem]{Lemma}
%\end{verbatim}
% The function of the macro |\newtheorem| is to recognize these cases
% and then to branch into one of the three macros |\@ynthm|, |\@xnthm|
% or |\@othm|.  This mechanism is adopted unchanged from
% \cite{latex:source}; the essential point here is that, for example,
% in the second case, the arguments \texttt{Lem}, \texttt{Lemma} and
% \texttt{section} are passed over to the macro |\@xnthm|.
%
% We inspect this case first because the others present fewer problems,
% and thus are easily derived from this one.
%
% \begin{macro}{\@xnthm}
%    For our example arguments, the macro |\@xnthm| must fulfill
%    the following:
%    \begin{itemize}
%       \item
%          Define a new \LaTeX-counter `Lem'
%       \item
%          reset this counter within a |\section|
%       \item
%          define the macro |\theLem|
%       \item
%          define the environment macros |\Lem| and
%          |\endLem| using the current
%          |\theoremstyle| and |\theorem@bodyfont|.
%    \end{itemize}
%    Obviously, all this should happen only if the first argument
%    of |\@xnthm| (i.e.\ \texttt{Lem} in our example) is chosen so
%    as not to conflict with any previously defined commands or
%    environments.  This test is performed by the \LaTeX{} macro
%    |\@ifdefinable|.
%    \begin{macrocode}
\gdef\@xnthm#1#2[#3]{\expandafter\@ifdefinable\csname #1\endcsname
%    \end{macrocode}
%    Therefore, the first argument of |\@ifdefinable| is the
%    expansion (in the example, |\Lem|) of
%    |\csname||#1\endcsname|.  The second argument is
%    executed only if the test has been completed successfully.
%    \begin{macrocode}
   {%
%    \end{macrocode}
%    Now we define the new counter. The names of the \LaTeX{}
%    macros employed should speak for themselves:
% \changes{v2.2b}{1995/10/07}{Use \cs{@newctr}}
% \changes{v2.2c}{1995/11/23}{Use \cs{@newctr} properly.}
%    \begin{macrocode}
    \@definecounter{#1}\@newctr{#1}[#3]%
%    \end{macrocode}
%    Using |\@newctr| will give a proper error message if the counter
%    in |#3| is not defined.
%    In defining `|\theLem|' we must generate the desired macro
%    name by use of |\expandafter| and |\csname|.
%    \begin{macrocode}
    \expandafter\xdef\csname the#1\endcsname
%    \end{macrocode}
%    An |\xdef| is used in order to make the definition global,
%    and to ensure that it contains the replacement texts of
%    |\@thmcountersep| and |\@thmcounter|.\footnote{These
%    two macros can be defined by the document style.  Their default
%    values produce a `\texttt{.}' as separation and an arabic
%    representation of the number.} However, not everything should be
%    expanded. For example, it saves space to use |\thesection|
%    instead of its---at times---lengthy expansion.
%    \begin{macrocode}
      {\expandafter \noexpand \csname the#3\endcsname
       \@thmcountersep \@thmcounter{#1}}%
%    \end{macrocode}
%    Thus with the defaults of \LaTeX{}, |\theLem| would be
%    replaced by the command sequence |\thesection.\arabic{Lem}|.
%
%    We will now look at the definition of the macro which is executed
%    at the beginning of the actual environment (in our example this
%    macro is |\Lem|). It should be noted that we use an
%    ``|\expandafter| trick'' to expand only certain parts of the
%    replacement text at the time of the definition.
%    \begin{macrocode}
    \def\@tempa{\global\@namedef{#1}}%
    \expandafter \@tempa \expandafter{%
%    \end{macrocode}
%    First, the macro that contains the current definitions of
%    |\@begintheorem| and |\@opargtheorem| should be called
%    up.  The name of this macro---as is already known---has the form
%    |\th@|\meta{theorem\ style}; therefore, it must be called by
%    \begin{macrocode}
      \csname th@\the \theorem@style
%    \end{macrocode}
%    In addition the default theorem font should be changeable, i.e.\
%    we have to insert the contents of |\theorem@bodyfont|. For
%    that reason, we expand even further, beyond |\endcsname|,
%    and thus insert the contents of the \textsf{token} register
%    |\theorem@bodyfont| in the replacement text.
%    \begin{macrocode}
            \expandafter \endcsname \the \theorem@bodyfont
%    \end{macrocode}
%    Now it is time to call the macro |\@thm| which takes over
%    the further processing.  It has two arguments: the current
%    counter name (in our example, \texttt{Lem}), and the text of the
%    label (in our example, \texttt{Lemma}).
%    \begin{macrocode}
     \@thm{#1}{#2}}%
%    \end{macrocode}
%    With this, the `sub-definition' is complete.  The macro
%    |\@endtheorem| ends a theorem environment and is, so far, nothing
%    but an |\endtrivlist|.  (Hence it is defined globally, and not
%    within the theorem styles.\footnote{This has to be changed as
%    soon as theorem styles that change \texttt{\bslash @endtheorem}
%    exist.  In such a case, all existing styles must be changed as
%    well since they will have to reset the macro.}) Therefore, we can
%    set it equivalent to the macro that ends the theorem set (in our
%    example, |\endLem|).  However, if some day theorem styles exist
%    that do change |\@endtheorem|, we would have to use the
%    commented-out line instead.
%    \begin{macrocode}
    \global \expandafter \let \csname end#1\endcsname \@endtheorem
%   \global\@namedef{end#1}{\@endtheorem}%
%    \end{macrocode}
%    With these commands all the required definitions are employed,
%    unless the test |\@ifdefinable| has failed.  Therefore, we
%    end the second argument of this macro and with it the definition
%    of |\@xnthm|.
%    \begin{macrocode}
   }}
%    \end{macrocode}
% \end{macro}
%
% \begin{macro}{\@ynthm}
%    The definition of |\@ynthm| is completely analogous. In
%    this case the new counter that is defined is not reset within
%    another counter; thus the definition of |\the...|
%    is simplified:
%    \begin{macrocode}
\gdef\@ynthm#1#2{\expandafter\@ifdefinable\csname #1\endcsname
   {\@definecounter{#1}%
    \expandafter\xdef\csname the#1\endcsname{\@thmcounter{#1}}%
%    \end{macrocode}
%    The rest of the definition corresponds literally to that of
%    |\@xnthm|:
%    \begin{macrocode}
    \def\@tempa{\global\@namedef{#1}}\expandafter \@tempa
     \expandafter{\csname th@\the \theorem@style \expandafter
     \endcsname \the\theorem@bodyfont \@thm{#1}{#2}}%
    \global \expandafter \let \csname end#1\endcsname \@endtheorem}}
%    \end{macrocode}
% \end{macro}
%
% \begin{macro}{\@othm}
%    The definition of |\@othm| does not contain anything new.
%    \begin{macrocode}
\gdef\@othm#1[#2]#3{%
%    \end{macrocode}
% \changes{v2.2b}{95/10/07}{Build in check for counter defined}
%    We do not define a new counter but instead use one that has
%    already been defined.  Thus the only definition we need is that
%    of this pseudo-counter (i.e.\ |\the|\meta{env.\ name}).
%    First we check if |#2| corresponds to a known counter name.
%    \begin{macrocode}
  \expandafter\ifx\csname c@#2\endcsname\relax
   \@nocounterr{#2}%
  \else
   \expandafter\@ifdefinable\csname #1\endcsname
   {\expandafter \xdef \csname the#1\endcsname
     {\expandafter \noexpand \csname the#2\endcsname}%
%    \end{macrocode}
%    All other parts of the definition can be adopted from
%    |\@xnthm|. We have to remember, though, that in this case
%    the name of the current counter and the theorem label have moved
%    to the second and third arguments.
%    \begin{macrocode}
    \def\@tempa{\global\@namedef{#1}}\expandafter \@tempa
     \expandafter{\csname th@\the \theorem@style \expandafter
     \endcsname \the\theorem@bodyfont \@thm{#2}{#3}}%
    \global \expandafter \let \csname end#1\endcsname \@endtheorem}%
  \fi}
%    \end{macrocode}
% \end{macro}
%
% \subsection{Macros that are employed in a {\envfont theorem}
%    environment}
%
% \begin{macro}{\@thm}
%    The macro |\@thm| has to increase the current counter.  Then,
%    depending on whether the environment has (or does not have) an
%    optional argument, it has to branch into either |\@begintheorem|
%    or |\@opargtheorem|.
%    \begin{macrocode}
\gdef\@thm#1#2{\refstepcounter{#1}%
%    \end{macrocode}
%    Now we start a {\envfont trivlist} environment, and give
%    |\@topsep| and |\@topsepadd| the values of the \textsf{skip}
%    registers |\theorempreskipamount| and |\theorempostskipamount|.
%    The value in |\@topsep| is the vertical space that is inserted by
%    the first (and only) |\item| in our |\trivlist| whilst
%    |\@topsepadd| is inserted by |\@endparenv| at the end of that
%    {\envfont trivlist} environment.  By using these registers, we
%    obtain the desired space around a {\envfont theorem} environment.
%    \begin{macrocode}
   \trivlist
   \@topsep \theorempreskipamount               % used by first \item
   \@topsepadd \theorempostskipamount           % used by \@endparenv
%    \end{macrocode}
%    Now we have to test whether an optional argument has been given.
% \changes{v2.0d}{89/07/18}{Forgotten percent added.}
%    \begin{macrocode}
   \@ifnextchar [%
%    \end{macrocode}
%    If there is an optional argument, we will call |\@ythm|, and
%    move the arguments read back into the input stream.
%    \begin{macrocode}
   {\@ythm{#1}{#2}}%
%    \end{macrocode}
%    If not, we call |\@begintheorem|. Its first argument is the
%    name of the theorem set (hence the second argument of
%    |\@thm|). Its second argument is the macro that produces the
%    current number.
%    \begin{macrocode}
   {\@begintheorem{#2}{\csname the#1\endcsname}\ignorespaces}}
%    \end{macrocode}
% \end{macro}
%
% \begin{macro}{\@xthm}
% \begin{macro}{\@ythm}
%    Both these macros were originally called by |\@thm|.  We do
%    not need |\@xthm| anymore, hence we reset it to
%    |\relax|. The definition of |\@ythm| has not changed at
%    all from its definition in \LaTeX. In order to make the macros
%    easier to understand, we will
%    nevertheless present it (commented out).
%    \begin{macrocode}
\global\let\@xthm\relax
% \def\@ythm#1#2[#3]{\@opargbegintheorem{#2}{\csname
%           the#1\endcsname}{#3}\ignorespaces}
%</package>
%    \end{macrocode}
%    The primitive |\ignorespaces| in |\@ythm| and
%    |\@thm| is needed to remove the spaces between the
%    |\begin{...}| and the actual text.
% \end{macro}
% \end{macro}
%
%
% \subsection{Definition of the theorem substyles}
%
% As already pointed out, the theorem substyles, defined below, are
% only loaded when necessary.  Note that all these substyles, except
% \texttt{plain}, have |\slshape| as the default body font.
%
% \subsubsection{The \texttt{plain} style}
%
% \label{dok:th@plain} As the following macros use |@|, we have
% to locally set the |\catcode| of this symbol to ``letter''.
% This happens within a group, so that we do not have to worry about
% which |\catcode| that symbol had before.
%    \begin{macrocode}
%<*thp>
\begingroup \makeatletter
%    \end{macrocode}
% Since we are now within a group, we must make all definitions
% globally.  First we make sure that \texttt{theorem.sty} is loaded.
% This will allow us to use this file as a document style option
% without having to call \texttt{theorem} itself as an option.  At the
% same time, we assure that at least version 2 is loaded, since
% |\theorem@style| was not defined in earlier versions.
%    \begin{macrocode}
\@ifundefined{theorem@style}{\input{theorem.sty}}{}
\ProvidesFile{thp.sty}
             [\FMithmInfo]
%    \end{macrocode}
%
%
% \begin{macro}{\th@plain}
% \begin{macro}{\@begintheorem}
% \begin{macro}{\@opargbegintheorem}
%    |\theoremstyle{plain}| corresponds to the original
%    definition, except that the distances to the surrounding text are
%    determined by the parameters |\theorempreskipamount| and
%    |\theorempostskipamount|.
%    First we set the default body font.
%    \begin{macrocode}
\gdef\th@plain{\normalfont\itshape
%    \end{macrocode}
%    Then we define |\@begintheorem| and
%    |\@opargbegintheorem|. These two macros define how the
%    header of a theorem is typeset.  |\@opargbegintheorem| will
%    be called if a \textsf{theorem} environment with an optional
%    argument is encountered; otherwise, the header is constructed by
%    calling |\@begintheorem|.  If one of these macros is
%    executed, we are within a \textsf{trivlist} environment started by
%    |\@thm|.  So the theorem header is produced with an
%    |\item| command.
%
%    Instead of specifying the header font directly, all standard
%    theorem styles use the |\theorem@headerfont| macro to allow
%    customization.  The extra space (|\labelsep|) is necessary
%    because of problems in the \textsf{trivlist} environment.
%    \begin{macrocode}
  \def\@begintheorem##1##2{%
        \item[\hskip\labelsep \theorem@headerfont ##1\ ##2]}%
%    \end{macrocode}
%    The definition of |\@opargbegintheorem| is completely
%    analogous. The only difference is the fact that there exists a
%    third argument (which is the optional parameter of the
%    environment and contains additional information about the
%    theorem).  Customarily we enclose it in parentheses.
%    \begin{macrocode}
\def\@opargbegintheorem##1##2##3{%
   \item[\hskip\labelsep \theorem@headerfont ##1\ ##2\ (##3)]}}
%    \end{macrocode}
% \end{macro}
% \end{macro}
% \end{macro}
%
% \noindent
% We conclude with an |\endgroup| to restore the
% |\catcode| of |@|.
%    \begin{macrocode}
\endgroup
%</thp>
%    \end{macrocode}
%
%
% \subsubsection{The \texttt{break} style}
%
% This style option is stored in the file |thb.sty|.
% For the next two lines see the documentation for |\th@plain| on
% page~\pageref{dok:th@plain}.
%    \begin{macrocode}
%<*thb>
\begingroup \makeatletter
\@ifundefined{theorem@style}{\input{theorem.sty}}{}
\ProvidesFile{thb.sty}
             [\FMithmInfo]
%    \end{macrocode}
%
% \begin{macro}{\th@break}
%    |\theoremstyle{break}| produces a line break after the name
%    of the theorem.  The font is |\slshape|.  Hence, we define
%    |\th@break| as follows:
%    \begin{macrocode}
\gdef\th@break{\normalfont\slshape
  \def\@begintheorem##1##2{\item[%
%    \end{macrocode}
%    We run into the following problem: it is not possible to create
%    the header with |\item[|\meta{title}|]| and then start
%    a new line by, for example, |\mbox{}\\|. Such a definition
%    will fail whenever a list environment follows immediately.  With
%    the above construction, the |\mbox{}| causes the switch
%    |@inlabel| (cf.\ definition of |\list| and
%    |\trivlist| in \cite{latex:source}) to be set to \texttt{false}
%    and so the following list will insert additional vertical
%    space (|\topskip|). This is quite annoying. Therefore, we
%    create the line break within the |\item|. In order to ensure
%    that the text will begin at the proper position in the following
%    line, we simply pretend that the label does not take any
%    room.\footnote{This will lead to problems whenever very high
%    symbols occurring in the line tower into the
%    heading. So, something else has to be done here sometime.}
%    \begin{macrocode}
       \rlap{\vbox{\hbox{\hskip \labelsep\theorem@headerfont ##1\ ##2}%
                   \hbox{\strut}}}]}%
%    \end{macrocode}
%    Again, the definition of |\@opargbegintheorem| is
%    completely analogous.
%    \begin{macrocode}
\def\@opargbegintheorem##1##2##3{%
  \item[\rlap{\vbox{\hbox{\hskip \labelsep \theorem@headerfont
                     ##1\ ##2\ (##3)}%
                    \hbox{\strut}}}]}}
\endgroup
%</thb>
%    \end{macrocode}
% \end{macro}
%
%
% \subsubsection{The \texttt{changebreak} style}
%
%    \begin{macrocode}
% This style option is stored in the file |thcb.sty|.
%    \begin{macrocode}
%<*thcb>
\begingroup \makeatletter
\@ifundefined{theorem@style}{\input{theorem.sty}}{}
\ProvidesFile{thcb.sty}
             [\FMithmInfo]
%    \end{macrocode}
%
% \begin{macro}{\th@changebreak}
%    The \texttt{change-break} theorem style is like \texttt{break}
%    but with interchange of theorem name and theorem number.  Thus we
%    define |\th@changebreak| as follows:
%    \begin{macrocode}
\gdef\th@changebreak{\normalfont\slshape
  \def\@begintheorem##1##2{\item
       [\rlap{\vbox{\hbox{\hskip\labelsep \theorem@headerfont ##2\ ##1}%
                    \hbox{\strut}}}]}%
\def\@opargbegintheorem##1##2##3{%
  \item[\rlap{\vbox{\hbox{\hskip\labelsep \theorem@headerfont
                     ##2\ ##1\ (##3)}%
                    \hbox{\strut}}}]}}
\endgroup
%</thcb>
%    \end{macrocode}
% \end{macro}
%
%
% \subsubsection{The \texttt{change} style}
%
% This style option is stored in the file |thc.sty|.
%    \begin{macrocode}
%<*thc>
\begingroup \makeatletter
\@ifundefined{theorem@style}{\input{theorem.sty}}{}
\ProvidesFile{thc.sty}
             [\FMithmInfo]
%    \end{macrocode}
%
% \begin{macro}{\th@change}
%    The \texttt{change} theorem style corresponds to the
%    \texttt{change break} style without a linebreak after the header.
%    To say it in another way, it's the same as the \texttt{plain}
%    style but with number and name interchanged and |\slshape| as the
%    default font.
%    \begin{macrocode}
\gdef\th@change{\normalfont\slshape
  \def\@begintheorem##1##2{\item
       [\hskip\labelsep \theorem@headerfont ##2\ ##1]}%
\def\@opargbegintheorem##1##2##3{%
  \item[\hskip\labelsep \theorem@headerfont ##2\ ##1\ (##3)]}}
\endgroup
%</thc>
%    \end{macrocode}
% \end{macro}
%
%
% \subsubsection{The \texttt{marginbreak} style}
%
% This style option is the one used most often at Mainz.  It is saved
% in the file |thmb.sty|.
%    \begin{macrocode}
%<*thmb>
\begingroup \makeatletter
\@ifundefined{theorem@style}{\input{theorem.sty}}{}
\ProvidesFile{thmb.sty}
             [\FMithmInfo]
%    \end{macrocode}
%
% \begin{macro}{\th@marginbreak}
%    The \texttt{margin break} style is nearly the same as the
%    \texttt{change break} style. The only difference is the placement
%    of the theorem number.  We use |\llap| to place it in the left
%    margin.
%
%    In this style |\labelsep| denotes the separation between the
%    number and the text.
%    \begin{macrocode}
\gdef\th@marginbreak{\normalfont\slshape
  \def\@begintheorem##1##2{\item
       [\rlap{\vbox{\theorem@headerfont
                    \hbox{\llap{##2}\hskip\labelsep ##1}%
                    \hbox{\strut}}}]}%
\def\@opargbegintheorem##1##2##3{%
  \item[\rlap{\vbox{\theorem@headerfont
                    \hbox{\llap{##2}\hskip\labelsep ##1\ (##3)}%
                    \hbox{\strut}}}]}}
\endgroup
%</thmb>
%    \end{macrocode}
% \changes{v2.1c}{90/02/17}{`endgroup outside macrocode env.}
% \end{macro}
%
%
%
% \subsubsection{The \texttt{margin} style}
%
% This style option is stored in the file |thm.sty|.
%    \begin{macrocode}
%<*thm>
\begingroup \makeatletter
\@ifundefined{theorem@style}{\input{theorem.sty}}{}
\ProvidesFile{thm.sty}
             [\FMithmInfo]
%    \end{macrocode}
%
% \begin{macro}{\th@margin}
%    Again this is only a variant of the theorem styles described
%    above without any new ideas.
%    \begin{macrocode}
\gdef\th@margin{\normalfont\slshape
  \def\@begintheorem##1##2{\item
       [\theorem@headerfont \llap{##2}\hskip\labelsep ##1]}%
\def\@opargbegintheorem##1##2##3{%
  \item[\theorem@headerfont \llap{##2}\hskip\labelsep ##1\ (##3)]}}
\endgroup
%</thm>
%    \end{macrocode}
% \end{macro}
%
%
%
%
% \subsection{Final Definitions}
%
% \begin{macro}{\theorempreskipamount}
% \begin{macro}{\theorempostskipamount}
%    The \textsf{skip} parameters that regulate the vertical empty space
%    before and after the {\envfont theorem} environment have
%    to be allocated as well.
% \changes{v2.0f}{89/07/20}{Unnecessary `global' removed.}
%    \begin{macrocode}
%<*package>
\newskip\theorempreskipamount
\newskip\theorempostskipamount
%    \end{macrocode}
%    Since we have used the same values for all theorem sets, we now can
%    assign them.
%    \begin{macrocode}
\global\setlength\theorempreskipamount{12pt plus 5pt minus 3pt}
\global\setlength\theorempostskipamount{8pt plus 3pt minus 1.5pt}
%    \end{macrocode}
% \end{macro}
% \end{macro}
%
%
% \begin{macro}{\@endtheorem}
%    The same holds for the macro |\@endtheorem|, which ends a
%    {\envfont theorem} environment. Since it is the same for all
%    theorem sets, it is removed from the macros |\th@|\meta{style}.
%    It simply ends the {\envfont trivlist} environment, which was
%    begun in |\@thm|.
%    \begin{macrocode}
\global\let\@endtheorem=\endtrivlist
%    \end{macrocode}
% \end{macro}
%
%
%
% \begin{macro}{\@preamblecmds}
%    All macros defined above are to be used only in the preamble.
%    Therefore, we insert them in |\@preamblecmds| which will disable
%    them at begin document. This is done by the internal \LaTeXe{}
%    command |\@onlypreamble|.
% \changes{v2.0g}{89/09/19}{`theoremheaderfont and `theorembodyfont
%    added.}
% \changes{v2.2a}{94/02/03}{Use \LaTeXe{} interface}
%    \begin{macrocode}
\@onlypreamble\@xnthm
\@onlypreamble\@ynthm
\@onlypreamble\@othm
\@onlypreamble\newtheorem
\@onlypreamble\theoremstyle
\@onlypreamble\theorembodyfont
\@onlypreamble\theoremheaderfont
%    \end{macrocode}
% \end{macro}
%
% Finally we declare the \texttt{plain} theorem style to be the default.
% \changes{v2.1b}{90/06/04}{Fixed missing `item by declaring default
%    style.}
%    \begin{macrocode}
\theoremstyle{plain}
%</package>
%    \end{macrocode}
%
% \Finale

\endinput
}{}
\ProvidesFile{thcb.sty}
             [\FMithmInfo]
%    \end{macrocode}
%
% \begin{macro}{\th@changebreak}
%    The \texttt{change-break} theorem style is like \texttt{break}
%    but with interchange of theorem name and theorem number.  Thus we
%    define |\th@changebreak| as follows:
%    \begin{macrocode}
\gdef\th@changebreak{\normalfont\slshape
  \def\@begintheorem##1##2{\item
       [\rlap{\vbox{\hbox{\hskip\labelsep \theorem@headerfont ##2\ ##1}%
                    \hbox{\strut}}}]}%
\def\@opargbegintheorem##1##2##3{%
  \item[\rlap{\vbox{\hbox{\hskip\labelsep \theorem@headerfont
                     ##2\ ##1\ (##3)}%
                    \hbox{\strut}}}]}}
\endgroup
%</thcb>
%    \end{macrocode}
% \end{macro}
%
%
% \subsubsection{The \texttt{change} style}
%
% This style option is stored in the file |thc.sty|.
%    \begin{macrocode}
%<*thc>
\begingroup \makeatletter
\@ifundefined{theorem@style}{% \iffalse meta-comment
%
% Copyright (C) 1993-2024
%
% The LaTeX Project and any individual authors listed elsewhere
% in this file.
%
% This file is part of the Standard LaTeX `Tools Bundle'.
% -------------------------------------------------------
%
% It may be distributed and/or modified under the
% conditions of the LaTeX Project Public License, either version 1.3c
% of this license or (at your option) any later version.
% The latest version of this license is in
%    https://www.latex-project.org/lppl.txt
% and version 1.3c or later is part of all distributions of LaTeX
% version 2005/12/01 or later.
%
% The list of all files belonging to the LaTeX `Tools Bundle' is
% given in the file `manifest.txt'.
%
% \fi
%
%%
%
%
\def\FMithmInfo{2023/07/05 v2.2c Theorem extension package (FMi)}
%
% \ProvidesFile{theorem.dtx}[\FMithmInfo]
%\iffalse   % this is a METACOMMENT !
%%
%% Package `theorem' to use with LaTeX2e
%% Copyright (C) 1989-1995 Frank Mittelbach, all rights reserved.
%<+package>\@ifundefined{theorem@style}{}{\endinput}
%<+package>\NeedsTeXFormat{LaTeX2e}[1995/06/01]
%<+package>\ProvidesPackage{theorem}
%<+package>         [\FMithmInfo]
%
% \fi
%
% \changes{v1.0}{87/05/07}{First draft.}
% \changes{v1.0a}{87/05/07}{`thistheoremstyle' added.}
% \changes{v1.1a}{87/07/29}{New concept: redefine the internal
%                          \protect\LaTeX{} commands `@xnthm' etc.}
% \changes{v1.1b}{87/07/31}{`theoremfont' used to specify font.}
% \changes{v1.2a}{88/01/08}{`@endtheorem' now globally defined.}
% \changes{v1.2b}{88/01/11}{`newtheorem' only allowed in the preamble.}
% \changes{v1.3a}{88/02/14}{`theoremfont' now a toks register.}
%
% \changes{v2.0a}{88/11/16}{New implementation and new documentation.}
% \changes{v2.0a}{88/11/16}{No layout changes but internal modifications
%                           old substyles won't work.}
% \changes{v2.0b}{89/01/30}{`global' used for all definitions.}
% \changes{v2.0c}{89/06/15}{Documentation translated to English.
%                            Thanks to LLL and Nina}
% \changes{v2.1a}{90/02/15}{Support for new font selection.}
% \changes{v2.2a}{94/02/03}{Cleanup for LaTeX2e}
%
% \DoNotIndex{\@endparenv,\@ifdefinable,\@ifnextchar,\@ifundefined}
% \DoNotIndex{\@namedef,\@nameuse,\@spaces,\@tempa,\@warning}
% \DoNotIndex{\ }
% \DoNotIndex{\begingroup,\bfseries}
% \DoNotIndex{\csname}
% \DoNotIndex{\def,\do,\docdate}
% \DoNotIndex{\endcsname,\endgroup,\endinput,\endtrivlist,\expandafter}
% \DoNotIndex{\filedate,\fileversion}
% \DoNotIndex{\gdef,\global}
% \DoNotIndex{\hbox,\hskip}
% \DoNotIndex{\ignorespaces,\input,\itshape,\item}
% \DoNotIndex{\labelsep,\let,\llap}
% \DoNotIndex{\makeatletter}
% \DoNotIndex{\noexpand}
% \DoNotIndex{\refstepcounter,\relax,\rlap}
% \DoNotIndex{\setlength,\slshape,\space,\string,\strut}
% \DoNotIndex{\the,\trivlist,\typeout}
% \DoNotIndex{\vbox}
% \DoNotIndex{\xdef}
%
% \def\deflabel#1{\ttfamily #1\hfill}
% \def\deflist#1{\begin{list}{}{\settowidth\labelwidth{\ttfamily #1}%
%                               \setlength\leftmargin\labelwidth
%                               \addtolength\leftmargin\labelsep
%                               \let\makelabel\deflabel}}
% \def\enddeflist{\end{list}}
% \def\envfont{\normalfont\ttfamily}
%
% \MakeShortVerb{\|}
%
% \newcommand{\TUB}{{\sl TUGboat\/}}
%
% \GetFileInfo{theorem.dtx}
% \title{An Extension of the
%        \LaTeX{} theorem environment\thanks{This file
%        has version number \fileversion, last
%        revised \filedate.}}
% \author{Frank Mittelbach}
%
% \MaintainedByLaTeXTeam{tools}
% \maketitle
%
% \begin{abstract}
%    The macros described in this paper yield an extension of the
%    \LaTeX{} theorem mechanism. It is designed to satisfy the
%    different requirements of various journals. Thus, the
%    layout of the ``theorems'' can be manipulated by determining a
%    ``style''. This article describes not only the use, but
%    also the definition, of the necessary macros.
% \end{abstract}
%
% \section*{Preface to version 2.2}
%
% For \LaTeXe{} this package did not need any fundamental changes. I
% only modified the messages generated so that theorem layout styles
% will show up with the |\listfiles| command and cleaned the section
% on the New Font Selection Scheme since this is now included in
% \LaTeX.
%
% \section*{Preface to version 2.1}
%
% This version is identical to 2.0g described in \TUB\ 10\#3 except
% for some internal defaults which are now set depending on the used
% font selection scheme.
%
% This was done to avoid unpleasant surprises if the new font selection
% scheme is in force. For further details see
% section~\ref{sec:newfonts} and \cite{bk:GMS94}.
%
% \section{Introduction}
%
% For our purposes here, ``theorems'' are labelled enunciations,
% often set off from the main text by extra space and a font change.
% Theorems, corollaries, conjectures, definitions, and remarks
% are all instances of ``theorems''.  The ``header'' of these structures
% is composed of a label (such as \textsc{Theorem} or \textsc{Remark})
% and a number which serializes an item in the sequence of items
% with the same label.
%
% Shortly after the introduction of \LaTeX{} at the Fachbereich
% Mathematik in Mainz, the desire to manipulate the layout of
% ``theorems'' arose.  In Mainz, the following two conventions came
% into general use:
% \begin{enumerate}
%   \item The number of the theorem is shown in the margin.
%   \item There is a line break at
%         the end of the theorem header.
% \end{enumerate}
%
% Additionally, some journals require different formats which depend on
% the ``sort of theorem'': e.g.\ often remarks and definitions are set
% in |\upshape|, while |\itshape| is employed for main theorems.
%
% Confronted with these requirements, a theorem environment was
% developed in Mainz which allows separate determination of the
% layout of the ``theorems sets'', comparable to |\pagestyle|.
%
%
% \section{The user interface}
% \subsection{Defining new theorem sets}
%
% \DescribeMacro\newtheorem
% As in the original \LaTeX{} version, the command |\newtheorem|
% defines a new ``theorem set'' or ``theorem-like structure''.
% Two required arguments name the new environment and give the text
% to be typeset with each instance of the new ``set'', while
% an optional argument determines how
% the ``set'' is enumerated:
% \begin{description}
%    \item[\ttfamily \bslash newtheorem\{foo\}\{bar\}]
%       The theorem set {\envfont foo} (whose name is \texttt{bar})
%       uses its own counter.
%    \item[\ttfamily \bslash newtheorem\{foo2\}{[foo]}\{bar2\}]
%       The theorem set {\envfont foo2} (printed name \texttt{bar2})
%       uses the same counter as the theorem set \texttt{foo}.
%    \item[\ttfamily \bslash newtheorem\{foo3\}\{bar3\}{[section]}]
%       The theorem set {\envfont foo3} (printed name \texttt{bar3}) is
%       enumerated within the counter \texttt{section}, i.e.\ with every
%       new |\section| the enumeration begins again with 1, and
%       the enumeration is composed from the section-number and the
%       theorem counter itself.
% \end{description}
%
% \DescribeMacro\theoremstyle
% Additionally, the command |\theoremstyle| can define the layout
% of various, or all, theorem sets.  It should be noted that any
% theorem set defined by |\newtheorem| is typeset in the
% |\theoremstyle| that is current at the time of the definition.
% Thus, the following
% \begin{verbatim}
%    \theoremstyle{break}        \newtheorem{Cor}{Corollary}
%    \theoremstyle{plain}        \newtheorem{Exa}{Example}[section]
%\end{verbatim}                             ^^A No blank at line start
% leads to the result that the set {\envfont Cor} is formatted in the
% style \texttt{break}, while the set {\envfont Exa} and all the
% following ones are formatted in the style \texttt{plain}, unless
% another |\theoremstyle| follows.  Since the definitions installed by
% |\newtheorem| are global, one also can limit |\theoremstyle| locally
% by grouping braces.
%
% \DescribeMacro\theorembodyfont
% The choice of the font for the theorem body is completely
% independent of the chosen |\theoremstyle|; this has proven to
% be very advantageous.  For example,
% \begin{verbatim}
%   {\theorembodyfont{\upshape}          \newtheorem{Rem}{Remark}}
%\end{verbatim}
% defines a theorem set {\envfont Rem}, which will be set in
% |\upshape| in the current layout (which in our example is
% \texttt{plain}).  As with |\theoremstyle|, the |\theorembodyfont|
% chosen is that current at the time of |\newtheorem|.  If
% |\theorembodyfont| is not specified or one defines
% |\theorembodyfont{}|, then the font used will be that defined by the
% |\theoremstyle|.
%
% \DescribeMacro\theoremheaderfont It is also possible to customize
% the font used for the theorem headers.  This is, however, a global
% declaration, and therefore there should be at most one
% |\theoremheaderfont| declaration in the preamble.\footnote{If it is
% actually necessary to have different header fonts, one has to define
% new theorem styles (substituting the desired font) or specify the
% information directly in the \texttt{\bslash newtheorem} declaration
% (the unclean variant).}
%
% \DescribeMacro\theorempreskipamount
% \DescribeMacro\theorempostskipamount Two additional parameters
% affect the vertical space around the theorem environments:
% |\theorempreskipamount| and |\theorempostskipamount| define,
% respectively, the spacing before and after such an environment.
% These parameters apply for all theorem sets and can be manipulated
% with the ordinary length macros.  They are rubber lengths
% (`\textsf{skips}'), and therefore can contain \texttt{plus} and
% \texttt{minus} parts.
%
% Since the definition of theorem sets should---most sensibly---be
% placed in the preamble, we only allow installation there.
% It is therefore possible to release the memory used here after
% |\begin{document}|, in order to make room for other
% applications.
%
%
% \subsection{Existing theorem styles}
%
% The following theorem styles exist to date:
% \begin{deflist}{marginbreak}
%    \item[plain]
%       This theorem style emulates the original \LaTeX{} definition,
%       except that additionally the parameters
%       |\theorem...skipamount| are used.
%    \item[break]
%       In this style, the theorem header is followed by a line
%       break.
%    \item[marginbreak]
%       The theorem number is set in the margin, and there is a line
%       break as in \texttt{break}.
%    \item[changebreak]
%       Like \texttt{break}, but with header number and text
%       interchanged.
%    \item[change]
%       Header number and text are interchanged, without a line break.
%    \item[margin]
%       The number is set in the left margin, without a line break.
% \end{deflist}
% All styles (except \texttt{plain}) select |\slshape| as the default
% |\theorembodyfont|.
%
%
% \subsection{Examples}
%
% Given the above theorem sets {\envfont Cor}, {\envfont Exa}
% and {\envfont Rem},
% suppose that the preamble also contains the declarations:
% \begin{verbatim}
%    \theoremstyle{marginbreak}   \newtheorem{Lem}[Cor]{Lemma}
%    \theoremstyle{change}
%    \theorembodyfont{\itshape}        \newtheorem{Def}[Cor]{Definition}
%
%    \theoremheaderfont{\scshape}
%\end{verbatim}
% Then the following are some typical examples of the typeset output
% resulting from their use.
%
% \begin{Cor}
%    This is a sentence typeset in the theorem environment {\envfont
%    Cor}.
% \end{Cor}
%
% \begin{Exa}
%    This is a sentence typeset in the theorem environment {\envfont
%    Exa}.
% \end{Exa}
%
% \begin{Rem}
%    This is a sentence typeset in the theorem environment {\envfont
%    Rem}.
% \end{Rem}
%
% \begin{Lem}[Ben User]
%    This is a sentence typeset in the theorem environment {\envfont
%    Lem}.
% \end{Lem}
%
% \begin{Def}[Very impressive Definition]
%    This is a sentence typeset in the theorem environment {\envfont
%    Def}.
% \end{Def}
%
% The last two examples show the effect of the optional argument to
% a theorem environment (it is the text typeset in parentheses).
%
%
% \section{Special Considerations}
% \label{sec:newfonts}
%
% Theoremheader and body are implemented as a unit. This means that
% the |\theoremheaderfont|  will inherit characteristics of the
% |\theorembodyfont| in \LaTeXe. Thus, if for example
% |\theorembodyfont| is |\itshape| and |\theoremheaderfont|
% is |\bfseries| the font selected for the header will have the
% characteristics `bold extended italic'. If this is not desired one
% should set the |\theoremheaderfont| to something like
% \begin{verbatim}
%      \theoremheaderfont{\normalfont\bfseries}
%\end{verbatim}
% i.e.\ supplying all necessary font information explicitly.
%
%
%
% \section{Acknowledgements}
%
% The publication of this set of macros was only possible with the
% help of Christina Busse (translating the manuscript into English),
% Joachim Pense (playing the r\^ole of typist), Chris Rowley
% (looking everything over) and many others providing useful
% suggestions.
%
% \MaybeStop{%
%    \begin{thebibliography}{9}
%       \bibitem{bk:GMS94} \textsc{M.~Goossens}, \textsc{F.~Mittelbach}
%         and \textsc{A.~Samarin}.
%         \newblock The \LaTeX{} Companion.
%         \newblock
%         Addison-Wesley, Reading, Massachusetts, 1994.
%       \bibitem{latex:source} \textsc{Lamport, Leslie.} \newblock
%                               \texttt{latex.tex}, version 2.09, date
%                               Feb. 1990.
%    \end{thebibliography}
%    \PrintIndex
%    \PrintChanges}
%
%
% \section{The documentation driver file}
%
% The next bit of code contains the documentation driver file for
% \TeX{}, i.e., the file that will produce the documentation you are
% currently reading. It will be extracted from this file by the
% \texttt{docstrip} program. Since it is the first code in the file
% one can alternatively process this file directly with \LaTeXe{} to
% obtain the documentation.
%
%    \begin{macrocode}
%<*driver>
\documentclass{ltxdoc}

\usepackage{theorem}

%<+driver>% The next few lines define theorem sets which are used
%<+driver>% in the example section of the documentation.

\theoremstyle{break}       \newtheorem{Cor}{Corollary}
\theoremstyle{plain}       \newtheorem{Exa}{Example}[section]
{\theorembodyfont{\upshape}\newtheorem{Rem}{Remark}}
\theoremstyle{marginbreak} \newtheorem{Lem}[Cor]{Lemma}
\theoremstyle{change}
\theorembodyfont{\itshape} \newtheorem{Def}[Cor]{Definition}

\theoremheaderfont{\scshape}

\RecordChanges

\begin{document}
   \DocInput{theorem.dtx}
\end{document}
%</driver>
%    \end{macrocode}
%
% \section{Definition of the Macros}
%
% If the file has been loaded before, we abort immediately. If not the
% package announces itself (this is actually done at the very top of
% the file---the way it is done isn't good style so don't copy it).
% \changes{v2.0e}{89/07/19}{Spaces removed from `typeout'.}
% \changes{v2.2b}{95/11/19}{Announce moved to top of file}
%    \begin{macrocode}
%<*package>
%\@ifundefined{theorem@style}{}{\endinput}
%\def\FMithmInfo{1995/11/19 v2.2b Theorem extension package (FMi)}
%\ProvidesPackage{theorem}[\FMithmInfo]
%    \end{macrocode}
%
%
% \subsection{Definition of theorem styles and fonts}
%
% All the definitions in this file are done globally to allow
% inputting this file inside a group.
%
% \begin{macro}{\theoremstyle}
%    Before a theorem style can be installed, the chosen style must be
%    known. For that reason, we must test to see that
%    |\th@|\meta{style} is known or, more precisely, that it is
%    different from |\relax|. If the style is not known then
%    |\th@plain| is used.
%    \begin{macrocode}
\gdef\theoremstyle#1{%
   \@ifundefined{th@#1}{\@warning
          {Unknown theoremstyle `#1'. Using `plain'}%
          \theorem@style{plain}}%
%    \end{macrocode}
%    We save the theorem style to be used in the token register
%    |\theorem@style|.
%    \begin{macrocode}
      {\theorem@style{#1}}%
%    \end{macrocode}
%    Now we ``evaluate'' the theorem style: this means, we call the
%    macro |\th@|\meta{style} which will activate the relevant
%    definitions which are contained in a separate file.  This is done
%    in a group to suppress changes to the current font.  This could
%    otherwise pose problems together with the new font selection
%    scheme\footnote{When I printed the original article using the new
%    font selection scheme I ended with a document with slanted
%    typefaces (text headings and all) simply because one of the
%    theorem styles used \texttt{\bslash sl} at toplevel.} if the
%    |\th@|\meta{style} is evaluated a second time.
% \changes{v2.1a}{90/02/15}{Group added around `th@..}
% \changes{v2.1a}{90/02/15}{Reseting of `..@begintheorem removed.}
%    \begin{macrocode}
      \begingroup
        \csname th@\the\theorem@style \endcsname
      \endgroup}
%    \end{macrocode}
% \end{macro}
%
%
% \begin{macro}{\@begintheorem}
% \changes{v2.1a}{90/02/15}{Relaxed at toplevel.}
% \begin{macro}{\@opargbegintheorem}
% \changes{v2.1a}{90/02/15}{Relaxed at toplevel.}
%    We reset |\@begintheorem| and |\@opargbegintheorem| to
%    |\relax| since these commands are no longer necessary at
%    toplevel. This will save a few tokens.
%    \begin{macrocode}
\global\let\@begintheorem\relax
\global\let\@opargbegintheorem\relax
%    \end{macrocode}
% \end{macro}
% \end{macro}
%
% \begin{macro}{\theorem@style}
%    Obviously the \textsf{token} register used above has to be
%    allocated. To assure the utmost compatibility with the
%    original \LaTeX{} definition, we set the default theorem style to
%    \texttt{plain}, which implements the usual \LaTeX{} convention.
% \changes{v2.0f}{89/07/20}{Unnecessary `global' removed.}
%    \begin{macrocode}
\newtoks\theorem@style
\global\theorem@style{plain}
%    \end{macrocode}
% \end{macro}
%
% \begin{macro}{\theorembodyfont}
% \begin{macro}{\theorem@bodyfont}
%    If the |\theorembodyfont| is set by the user then it should not
%    interact with the default font set in the theorem style. When the
%    new font selection is in force this may happen if, for example,
%    the default is |\itshape| and the new |\theorembodyfont| is
%    |\sffamily|. So we add a |\reset@font| command in front of the
%    user definition.
% \changes{v2.0f}{89/07/20}{Unnecessary `global' removed.}
% \changes{v2.0g}{89/09/19}{`theoremfont to `theorembodyfont.}
% \changes{v2.1c}{92/06/22}{`reset@font added and logic changed.}
% \changes{v2.1d}{92/10/28}{Reset to empty if arg is empty}
%    \begin{macrocode}
\gdef\theorembodyfont#1{%
%    \end{macrocode}
%    We check if the argument supplied is empty and if so put nothing
%    into the |\theorem@bodyfont| token register to allow for
%    |\theorembodyfont{}| as a mean of using the default of the
%    current |\theoremstyle|.
%    \begin{macrocode}
   \def\@tempa{#1}%
   \ifx\@tempa\@empty
    \theorem@bodyfont{}%
   \else
    \theorem@bodyfont{\reset@font#1}%
   \fi
   }
\newtoks\theorem@bodyfont
\global\theorem@bodyfont{}
%    \end{macrocode}
% \end{macro}
% \end{macro}
%
%
% \begin{macro}{\theoremheaderfont}
% \changes{v2.0g}{89/09/19}{Macro added.}
%    The font for the theorem headers is handled differently because
%    this definition applies to all theorem styles.
%    \begin{macrocode}
\gdef\theoremheaderfont#1{\gdef\theorem@headerfont{#1}%
%    \end{macrocode}
%    After using the macro once it is redefined to produce an error
%    message.
%    \begin{macrocode}
       \gdef\theoremheaderfont##1{%
        \typeout{\string\theoremheaderfont\space should be used
                 only once.}}}
%    \end{macrocode}
% \end{macro}
%
% \begin{macro}{\theorem@headerfont}
% \changes{v2.0g}{89/09/19}{Macro added.}
%    To set the |\theorem@headerfont| default we first test if
%    the new fontselection scheme is in force.
% \changes{v2.1a}{90/02/15}{Support for new font selection.}
%    \begin{macrocode}
\ifx\upshape\undefined
%    \end{macrocode}
%    If not we define it to expand into |\bfseries|. We don't use
%    |\let| just in case a following style option redefines this
%    macro.
%    \begin{macrocode}
\gdef\theorem@headerfont{\bfseries}
%    \end{macrocode}
%    Otherwise we reset the current shape before calling |\bfseries|.
%    \begin{macrocode}
\else \gdef\theorem@headerfont{\normalfont\bfseries}\fi
%    \end{macrocode}
% \end{macro}
%
% \begin{macro}{\th@plain}
% \begin{macro}{\th@break}
% \begin{macro}{\th@marginbreak}
% \begin{macro}{\th@changebreak}
% \begin{macro}{\th@change}
% \begin{macro}{\th@margin}
%    The different styles are defined in macros such as
%    |\th@plain|.  Since memory space is precious in
%    ``non-Big-versions'', we have to avoid offering too many unused
%    definitions.  Therefore we define these styles in separate files
%    that can be loaded on demand.  Thus the commands themselves only
%    load these files. We use |\@input@|, a \LaTeXe{} internal command
%    that ensures that the file will be listed with |\listfiles|.
% \changes{v2.2a}{94/02/02}{Use `@input@ to load theorem layout files}
%    \begin{macrocode}
\gdef\th@plain{\@input@{thp.sty}}
\gdef\th@break{\@input@{thb.sty}}
\gdef\th@marginbreak{\@input@{thmb.sty}}
\gdef\th@changebreak{\@input@{thcb.sty}}
\gdef\th@change{\@input@{thc.sty}}
\gdef\th@margin{\@input@{thm.sty}}
%    \end{macrocode}
%    This list will be expanded when new styles become
%    available.  For testing,  just append new theorem substyles
%    as document options.
% \end{macro}
% \end{macro}
% \end{macro}
% \end{macro}
% \end{macro}
% \end{macro}
%
%
%
% \subsection{Definition of a new theorem set}
%
% As already pointed out, a new theorem environment can be defined in
% three different ways:
% \begin{verbatim}
%    \newtheorem{Lem}{Lemma}
%    \newtheorem{Lem}{Lemma}[section]
%    \newtheorem{Lem}[Theorem]{Lemma}
%\end{verbatim}
% The function of the macro |\newtheorem| is to recognize these cases
% and then to branch into one of the three macros |\@ynthm|, |\@xnthm|
% or |\@othm|.  This mechanism is adopted unchanged from
% \cite{latex:source}; the essential point here is that, for example,
% in the second case, the arguments \texttt{Lem}, \texttt{Lemma} and
% \texttt{section} are passed over to the macro |\@xnthm|.
%
% We inspect this case first because the others present fewer problems,
% and thus are easily derived from this one.
%
% \begin{macro}{\@xnthm}
%    For our example arguments, the macro |\@xnthm| must fulfill
%    the following:
%    \begin{itemize}
%       \item
%          Define a new \LaTeX-counter `Lem'
%       \item
%          reset this counter within a |\section|
%       \item
%          define the macro |\theLem|
%       \item
%          define the environment macros |\Lem| and
%          |\endLem| using the current
%          |\theoremstyle| and |\theorem@bodyfont|.
%    \end{itemize}
%    Obviously, all this should happen only if the first argument
%    of |\@xnthm| (i.e.\ \texttt{Lem} in our example) is chosen so
%    as not to conflict with any previously defined commands or
%    environments.  This test is performed by the \LaTeX{} macro
%    |\@ifdefinable|.
%    \begin{macrocode}
\gdef\@xnthm#1#2[#3]{\expandafter\@ifdefinable\csname #1\endcsname
%    \end{macrocode}
%    Therefore, the first argument of |\@ifdefinable| is the
%    expansion (in the example, |\Lem|) of
%    |\csname||#1\endcsname|.  The second argument is
%    executed only if the test has been completed successfully.
%    \begin{macrocode}
   {%
%    \end{macrocode}
%    Now we define the new counter. The names of the \LaTeX{}
%    macros employed should speak for themselves:
% \changes{v2.2b}{1995/10/07}{Use \cs{@newctr}}
% \changes{v2.2c}{1995/11/23}{Use \cs{@newctr} properly.}
%    \begin{macrocode}
    \@definecounter{#1}\@newctr{#1}[#3]%
%    \end{macrocode}
%    Using |\@newctr| will give a proper error message if the counter
%    in |#3| is not defined.
%    In defining `|\theLem|' we must generate the desired macro
%    name by use of |\expandafter| and |\csname|.
%    \begin{macrocode}
    \expandafter\xdef\csname the#1\endcsname
%    \end{macrocode}
%    An |\xdef| is used in order to make the definition global,
%    and to ensure that it contains the replacement texts of
%    |\@thmcountersep| and |\@thmcounter|.\footnote{These
%    two macros can be defined by the document style.  Their default
%    values produce a `\texttt{.}' as separation and an arabic
%    representation of the number.} However, not everything should be
%    expanded. For example, it saves space to use |\thesection|
%    instead of its---at times---lengthy expansion.
%    \begin{macrocode}
      {\expandafter \noexpand \csname the#3\endcsname
       \@thmcountersep \@thmcounter{#1}}%
%    \end{macrocode}
%    Thus with the defaults of \LaTeX{}, |\theLem| would be
%    replaced by the command sequence |\thesection.\arabic{Lem}|.
%
%    We will now look at the definition of the macro which is executed
%    at the beginning of the actual environment (in our example this
%    macro is |\Lem|). It should be noted that we use an
%    ``|\expandafter| trick'' to expand only certain parts of the
%    replacement text at the time of the definition.
%    \begin{macrocode}
    \def\@tempa{\global\@namedef{#1}}%
    \expandafter \@tempa \expandafter{%
%    \end{macrocode}
%    First, the macro that contains the current definitions of
%    |\@begintheorem| and |\@opargtheorem| should be called
%    up.  The name of this macro---as is already known---has the form
%    |\th@|\meta{theorem\ style}; therefore, it must be called by
%    \begin{macrocode}
      \csname th@\the \theorem@style
%    \end{macrocode}
%    In addition the default theorem font should be changeable, i.e.\
%    we have to insert the contents of |\theorem@bodyfont|. For
%    that reason, we expand even further, beyond |\endcsname|,
%    and thus insert the contents of the \textsf{token} register
%    |\theorem@bodyfont| in the replacement text.
%    \begin{macrocode}
            \expandafter \endcsname \the \theorem@bodyfont
%    \end{macrocode}
%    Now it is time to call the macro |\@thm| which takes over
%    the further processing.  It has two arguments: the current
%    counter name (in our example, \texttt{Lem}), and the text of the
%    label (in our example, \texttt{Lemma}).
%    \begin{macrocode}
     \@thm{#1}{#2}}%
%    \end{macrocode}
%    With this, the `sub-definition' is complete.  The macro
%    |\@endtheorem| ends a theorem environment and is, so far, nothing
%    but an |\endtrivlist|.  (Hence it is defined globally, and not
%    within the theorem styles.\footnote{This has to be changed as
%    soon as theorem styles that change \texttt{\bslash @endtheorem}
%    exist.  In such a case, all existing styles must be changed as
%    well since they will have to reset the macro.}) Therefore, we can
%    set it equivalent to the macro that ends the theorem set (in our
%    example, |\endLem|).  However, if some day theorem styles exist
%    that do change |\@endtheorem|, we would have to use the
%    commented-out line instead.
%    \begin{macrocode}
    \global \expandafter \let \csname end#1\endcsname \@endtheorem
%   \global\@namedef{end#1}{\@endtheorem}%
%    \end{macrocode}
%    With these commands all the required definitions are employed,
%    unless the test |\@ifdefinable| has failed.  Therefore, we
%    end the second argument of this macro and with it the definition
%    of |\@xnthm|.
%    \begin{macrocode}
   }}
%    \end{macrocode}
% \end{macro}
%
% \begin{macro}{\@ynthm}
%    The definition of |\@ynthm| is completely analogous. In
%    this case the new counter that is defined is not reset within
%    another counter; thus the definition of |\the...|
%    is simplified:
%    \begin{macrocode}
\gdef\@ynthm#1#2{\expandafter\@ifdefinable\csname #1\endcsname
   {\@definecounter{#1}%
    \expandafter\xdef\csname the#1\endcsname{\@thmcounter{#1}}%
%    \end{macrocode}
%    The rest of the definition corresponds literally to that of
%    |\@xnthm|:
%    \begin{macrocode}
    \def\@tempa{\global\@namedef{#1}}\expandafter \@tempa
     \expandafter{\csname th@\the \theorem@style \expandafter
     \endcsname \the\theorem@bodyfont \@thm{#1}{#2}}%
    \global \expandafter \let \csname end#1\endcsname \@endtheorem}}
%    \end{macrocode}
% \end{macro}
%
% \begin{macro}{\@othm}
%    The definition of |\@othm| does not contain anything new.
%    \begin{macrocode}
\gdef\@othm#1[#2]#3{%
%    \end{macrocode}
% \changes{v2.2b}{95/10/07}{Build in check for counter defined}
%    We do not define a new counter but instead use one that has
%    already been defined.  Thus the only definition we need is that
%    of this pseudo-counter (i.e.\ |\the|\meta{env.\ name}).
%    First we check if |#2| corresponds to a known counter name.
%    \begin{macrocode}
  \expandafter\ifx\csname c@#2\endcsname\relax
   \@nocounterr{#2}%
  \else
   \expandafter\@ifdefinable\csname #1\endcsname
   {\expandafter \xdef \csname the#1\endcsname
     {\expandafter \noexpand \csname the#2\endcsname}%
%    \end{macrocode}
%    All other parts of the definition can be adopted from
%    |\@xnthm|. We have to remember, though, that in this case
%    the name of the current counter and the theorem label have moved
%    to the second and third arguments.
%    \begin{macrocode}
    \def\@tempa{\global\@namedef{#1}}\expandafter \@tempa
     \expandafter{\csname th@\the \theorem@style \expandafter
     \endcsname \the\theorem@bodyfont \@thm{#2}{#3}}%
    \global \expandafter \let \csname end#1\endcsname \@endtheorem}%
  \fi}
%    \end{macrocode}
% \end{macro}
%
% \subsection{Macros that are employed in a {\envfont theorem}
%    environment}
%
% \begin{macro}{\@thm}
%    The macro |\@thm| has to increase the current counter.  Then,
%    depending on whether the environment has (or does not have) an
%    optional argument, it has to branch into either |\@begintheorem|
%    or |\@opargtheorem|.
%    \begin{macrocode}
\gdef\@thm#1#2{\refstepcounter{#1}%
%    \end{macrocode}
%    Now we start a {\envfont trivlist} environment, and give
%    |\@topsep| and |\@topsepadd| the values of the \textsf{skip}
%    registers |\theorempreskipamount| and |\theorempostskipamount|.
%    The value in |\@topsep| is the vertical space that is inserted by
%    the first (and only) |\item| in our |\trivlist| whilst
%    |\@topsepadd| is inserted by |\@endparenv| at the end of that
%    {\envfont trivlist} environment.  By using these registers, we
%    obtain the desired space around a {\envfont theorem} environment.
%    \begin{macrocode}
   \trivlist
   \@topsep \theorempreskipamount               % used by first \item
   \@topsepadd \theorempostskipamount           % used by \@endparenv
%    \end{macrocode}
%    Now we have to test whether an optional argument has been given.
% \changes{v2.0d}{89/07/18}{Forgotten percent added.}
%    \begin{macrocode}
   \@ifnextchar [%
%    \end{macrocode}
%    If there is an optional argument, we will call |\@ythm|, and
%    move the arguments read back into the input stream.
%    \begin{macrocode}
   {\@ythm{#1}{#2}}%
%    \end{macrocode}
%    If not, we call |\@begintheorem|. Its first argument is the
%    name of the theorem set (hence the second argument of
%    |\@thm|). Its second argument is the macro that produces the
%    current number.
%    \begin{macrocode}
   {\@begintheorem{#2}{\csname the#1\endcsname}\ignorespaces}}
%    \end{macrocode}
% \end{macro}
%
% \begin{macro}{\@xthm}
% \begin{macro}{\@ythm}
%    Both these macros were originally called by |\@thm|.  We do
%    not need |\@xthm| anymore, hence we reset it to
%    |\relax|. The definition of |\@ythm| has not changed at
%    all from its definition in \LaTeX. In order to make the macros
%    easier to understand, we will
%    nevertheless present it (commented out).
%    \begin{macrocode}
\global\let\@xthm\relax
% \def\@ythm#1#2[#3]{\@opargbegintheorem{#2}{\csname
%           the#1\endcsname}{#3}\ignorespaces}
%</package>
%    \end{macrocode}
%    The primitive |\ignorespaces| in |\@ythm| and
%    |\@thm| is needed to remove the spaces between the
%    |\begin{...}| and the actual text.
% \end{macro}
% \end{macro}
%
%
% \subsection{Definition of the theorem substyles}
%
% As already pointed out, the theorem substyles, defined below, are
% only loaded when necessary.  Note that all these substyles, except
% \texttt{plain}, have |\slshape| as the default body font.
%
% \subsubsection{The \texttt{plain} style}
%
% \label{dok:th@plain} As the following macros use |@|, we have
% to locally set the |\catcode| of this symbol to ``letter''.
% This happens within a group, so that we do not have to worry about
% which |\catcode| that symbol had before.
%    \begin{macrocode}
%<*thp>
\begingroup \makeatletter
%    \end{macrocode}
% Since we are now within a group, we must make all definitions
% globally.  First we make sure that \texttt{theorem.sty} is loaded.
% This will allow us to use this file as a document style option
% without having to call \texttt{theorem} itself as an option.  At the
% same time, we assure that at least version 2 is loaded, since
% |\theorem@style| was not defined in earlier versions.
%    \begin{macrocode}
\@ifundefined{theorem@style}{\input{theorem.sty}}{}
\ProvidesFile{thp.sty}
             [\FMithmInfo]
%    \end{macrocode}
%
%
% \begin{macro}{\th@plain}
% \begin{macro}{\@begintheorem}
% \begin{macro}{\@opargbegintheorem}
%    |\theoremstyle{plain}| corresponds to the original
%    definition, except that the distances to the surrounding text are
%    determined by the parameters |\theorempreskipamount| and
%    |\theorempostskipamount|.
%    First we set the default body font.
%    \begin{macrocode}
\gdef\th@plain{\normalfont\itshape
%    \end{macrocode}
%    Then we define |\@begintheorem| and
%    |\@opargbegintheorem|. These two macros define how the
%    header of a theorem is typeset.  |\@opargbegintheorem| will
%    be called if a \textsf{theorem} environment with an optional
%    argument is encountered; otherwise, the header is constructed by
%    calling |\@begintheorem|.  If one of these macros is
%    executed, we are within a \textsf{trivlist} environment started by
%    |\@thm|.  So the theorem header is produced with an
%    |\item| command.
%
%    Instead of specifying the header font directly, all standard
%    theorem styles use the |\theorem@headerfont| macro to allow
%    customization.  The extra space (|\labelsep|) is necessary
%    because of problems in the \textsf{trivlist} environment.
%    \begin{macrocode}
  \def\@begintheorem##1##2{%
        \item[\hskip\labelsep \theorem@headerfont ##1\ ##2]}%
%    \end{macrocode}
%    The definition of |\@opargbegintheorem| is completely
%    analogous. The only difference is the fact that there exists a
%    third argument (which is the optional parameter of the
%    environment and contains additional information about the
%    theorem).  Customarily we enclose it in parentheses.
%    \begin{macrocode}
\def\@opargbegintheorem##1##2##3{%
   \item[\hskip\labelsep \theorem@headerfont ##1\ ##2\ (##3)]}}
%    \end{macrocode}
% \end{macro}
% \end{macro}
% \end{macro}
%
% \noindent
% We conclude with an |\endgroup| to restore the
% |\catcode| of |@|.
%    \begin{macrocode}
\endgroup
%</thp>
%    \end{macrocode}
%
%
% \subsubsection{The \texttt{break} style}
%
% This style option is stored in the file |thb.sty|.
% For the next two lines see the documentation for |\th@plain| on
% page~\pageref{dok:th@plain}.
%    \begin{macrocode}
%<*thb>
\begingroup \makeatletter
\@ifundefined{theorem@style}{\input{theorem.sty}}{}
\ProvidesFile{thb.sty}
             [\FMithmInfo]
%    \end{macrocode}
%
% \begin{macro}{\th@break}
%    |\theoremstyle{break}| produces a line break after the name
%    of the theorem.  The font is |\slshape|.  Hence, we define
%    |\th@break| as follows:
%    \begin{macrocode}
\gdef\th@break{\normalfont\slshape
  \def\@begintheorem##1##2{\item[%
%    \end{macrocode}
%    We run into the following problem: it is not possible to create
%    the header with |\item[|\meta{title}|]| and then start
%    a new line by, for example, |\mbox{}\\|. Such a definition
%    will fail whenever a list environment follows immediately.  With
%    the above construction, the |\mbox{}| causes the switch
%    |@inlabel| (cf.\ definition of |\list| and
%    |\trivlist| in \cite{latex:source}) to be set to \texttt{false}
%    and so the following list will insert additional vertical
%    space (|\topskip|). This is quite annoying. Therefore, we
%    create the line break within the |\item|. In order to ensure
%    that the text will begin at the proper position in the following
%    line, we simply pretend that the label does not take any
%    room.\footnote{This will lead to problems whenever very high
%    symbols occurring in the line tower into the
%    heading. So, something else has to be done here sometime.}
%    \begin{macrocode}
       \rlap{\vbox{\hbox{\hskip \labelsep\theorem@headerfont ##1\ ##2}%
                   \hbox{\strut}}}]}%
%    \end{macrocode}
%    Again, the definition of |\@opargbegintheorem| is
%    completely analogous.
%    \begin{macrocode}
\def\@opargbegintheorem##1##2##3{%
  \item[\rlap{\vbox{\hbox{\hskip \labelsep \theorem@headerfont
                     ##1\ ##2\ (##3)}%
                    \hbox{\strut}}}]}}
\endgroup
%</thb>
%    \end{macrocode}
% \end{macro}
%
%
% \subsubsection{The \texttt{changebreak} style}
%
%    \begin{macrocode}
% This style option is stored in the file |thcb.sty|.
%    \begin{macrocode}
%<*thcb>
\begingroup \makeatletter
\@ifundefined{theorem@style}{\input{theorem.sty}}{}
\ProvidesFile{thcb.sty}
             [\FMithmInfo]
%    \end{macrocode}
%
% \begin{macro}{\th@changebreak}
%    The \texttt{change-break} theorem style is like \texttt{break}
%    but with interchange of theorem name and theorem number.  Thus we
%    define |\th@changebreak| as follows:
%    \begin{macrocode}
\gdef\th@changebreak{\normalfont\slshape
  \def\@begintheorem##1##2{\item
       [\rlap{\vbox{\hbox{\hskip\labelsep \theorem@headerfont ##2\ ##1}%
                    \hbox{\strut}}}]}%
\def\@opargbegintheorem##1##2##3{%
  \item[\rlap{\vbox{\hbox{\hskip\labelsep \theorem@headerfont
                     ##2\ ##1\ (##3)}%
                    \hbox{\strut}}}]}}
\endgroup
%</thcb>
%    \end{macrocode}
% \end{macro}
%
%
% \subsubsection{The \texttt{change} style}
%
% This style option is stored in the file |thc.sty|.
%    \begin{macrocode}
%<*thc>
\begingroup \makeatletter
\@ifundefined{theorem@style}{\input{theorem.sty}}{}
\ProvidesFile{thc.sty}
             [\FMithmInfo]
%    \end{macrocode}
%
% \begin{macro}{\th@change}
%    The \texttt{change} theorem style corresponds to the
%    \texttt{change break} style without a linebreak after the header.
%    To say it in another way, it's the same as the \texttt{plain}
%    style but with number and name interchanged and |\slshape| as the
%    default font.
%    \begin{macrocode}
\gdef\th@change{\normalfont\slshape
  \def\@begintheorem##1##2{\item
       [\hskip\labelsep \theorem@headerfont ##2\ ##1]}%
\def\@opargbegintheorem##1##2##3{%
  \item[\hskip\labelsep \theorem@headerfont ##2\ ##1\ (##3)]}}
\endgroup
%</thc>
%    \end{macrocode}
% \end{macro}
%
%
% \subsubsection{The \texttt{marginbreak} style}
%
% This style option is the one used most often at Mainz.  It is saved
% in the file |thmb.sty|.
%    \begin{macrocode}
%<*thmb>
\begingroup \makeatletter
\@ifundefined{theorem@style}{\input{theorem.sty}}{}
\ProvidesFile{thmb.sty}
             [\FMithmInfo]
%    \end{macrocode}
%
% \begin{macro}{\th@marginbreak}
%    The \texttt{margin break} style is nearly the same as the
%    \texttt{change break} style. The only difference is the placement
%    of the theorem number.  We use |\llap| to place it in the left
%    margin.
%
%    In this style |\labelsep| denotes the separation between the
%    number and the text.
%    \begin{macrocode}
\gdef\th@marginbreak{\normalfont\slshape
  \def\@begintheorem##1##2{\item
       [\rlap{\vbox{\theorem@headerfont
                    \hbox{\llap{##2}\hskip\labelsep ##1}%
                    \hbox{\strut}}}]}%
\def\@opargbegintheorem##1##2##3{%
  \item[\rlap{\vbox{\theorem@headerfont
                    \hbox{\llap{##2}\hskip\labelsep ##1\ (##3)}%
                    \hbox{\strut}}}]}}
\endgroup
%</thmb>
%    \end{macrocode}
% \changes{v2.1c}{90/02/17}{`endgroup outside macrocode env.}
% \end{macro}
%
%
%
% \subsubsection{The \texttt{margin} style}
%
% This style option is stored in the file |thm.sty|.
%    \begin{macrocode}
%<*thm>
\begingroup \makeatletter
\@ifundefined{theorem@style}{\input{theorem.sty}}{}
\ProvidesFile{thm.sty}
             [\FMithmInfo]
%    \end{macrocode}
%
% \begin{macro}{\th@margin}
%    Again this is only a variant of the theorem styles described
%    above without any new ideas.
%    \begin{macrocode}
\gdef\th@margin{\normalfont\slshape
  \def\@begintheorem##1##2{\item
       [\theorem@headerfont \llap{##2}\hskip\labelsep ##1]}%
\def\@opargbegintheorem##1##2##3{%
  \item[\theorem@headerfont \llap{##2}\hskip\labelsep ##1\ (##3)]}}
\endgroup
%</thm>
%    \end{macrocode}
% \end{macro}
%
%
%
%
% \subsection{Final Definitions}
%
% \begin{macro}{\theorempreskipamount}
% \begin{macro}{\theorempostskipamount}
%    The \textsf{skip} parameters that regulate the vertical empty space
%    before and after the {\envfont theorem} environment have
%    to be allocated as well.
% \changes{v2.0f}{89/07/20}{Unnecessary `global' removed.}
%    \begin{macrocode}
%<*package>
\newskip\theorempreskipamount
\newskip\theorempostskipamount
%    \end{macrocode}
%    Since we have used the same values for all theorem sets, we now can
%    assign them.
%    \begin{macrocode}
\global\setlength\theorempreskipamount{12pt plus 5pt minus 3pt}
\global\setlength\theorempostskipamount{8pt plus 3pt minus 1.5pt}
%    \end{macrocode}
% \end{macro}
% \end{macro}
%
%
% \begin{macro}{\@endtheorem}
%    The same holds for the macro |\@endtheorem|, which ends a
%    {\envfont theorem} environment. Since it is the same for all
%    theorem sets, it is removed from the macros |\th@|\meta{style}.
%    It simply ends the {\envfont trivlist} environment, which was
%    begun in |\@thm|.
%    \begin{macrocode}
\global\let\@endtheorem=\endtrivlist
%    \end{macrocode}
% \end{macro}
%
%
%
% \begin{macro}{\@preamblecmds}
%    All macros defined above are to be used only in the preamble.
%    Therefore, we insert them in |\@preamblecmds| which will disable
%    them at begin document. This is done by the internal \LaTeXe{}
%    command |\@onlypreamble|.
% \changes{v2.0g}{89/09/19}{`theoremheaderfont and `theorembodyfont
%    added.}
% \changes{v2.2a}{94/02/03}{Use \LaTeXe{} interface}
%    \begin{macrocode}
\@onlypreamble\@xnthm
\@onlypreamble\@ynthm
\@onlypreamble\@othm
\@onlypreamble\newtheorem
\@onlypreamble\theoremstyle
\@onlypreamble\theorembodyfont
\@onlypreamble\theoremheaderfont
%    \end{macrocode}
% \end{macro}
%
% Finally we declare the \texttt{plain} theorem style to be the default.
% \changes{v2.1b}{90/06/04}{Fixed missing `item by declaring default
%    style.}
%    \begin{macrocode}
\theoremstyle{plain}
%</package>
%    \end{macrocode}
%
% \Finale

\endinput
}{}
\ProvidesFile{thc.sty}
             [\FMithmInfo]
%    \end{macrocode}
%
% \begin{macro}{\th@change}
%    The \texttt{change} theorem style corresponds to the
%    \texttt{change break} style without a linebreak after the header.
%    To say it in another way, it's the same as the \texttt{plain}
%    style but with number and name interchanged and |\slshape| as the
%    default font.
%    \begin{macrocode}
\gdef\th@change{\normalfont\slshape
  \def\@begintheorem##1##2{\item
       [\hskip\labelsep \theorem@headerfont ##2\ ##1]}%
\def\@opargbegintheorem##1##2##3{%
  \item[\hskip\labelsep \theorem@headerfont ##2\ ##1\ (##3)]}}
\endgroup
%</thc>
%    \end{macrocode}
% \end{macro}
%
%
% \subsubsection{The \texttt{marginbreak} style}
%
% This style option is the one used most often at Mainz.  It is saved
% in the file |thmb.sty|.
%    \begin{macrocode}
%<*thmb>
\begingroup \makeatletter
\@ifundefined{theorem@style}{% \iffalse meta-comment
%
% Copyright (C) 1993-2024
%
% The LaTeX Project and any individual authors listed elsewhere
% in this file.
%
% This file is part of the Standard LaTeX `Tools Bundle'.
% -------------------------------------------------------
%
% It may be distributed and/or modified under the
% conditions of the LaTeX Project Public License, either version 1.3c
% of this license or (at your option) any later version.
% The latest version of this license is in
%    https://www.latex-project.org/lppl.txt
% and version 1.3c or later is part of all distributions of LaTeX
% version 2005/12/01 or later.
%
% The list of all files belonging to the LaTeX `Tools Bundle' is
% given in the file `manifest.txt'.
%
% \fi
%
%%
%
%
\def\FMithmInfo{2023/07/05 v2.2c Theorem extension package (FMi)}
%
% \ProvidesFile{theorem.dtx}[\FMithmInfo]
%\iffalse   % this is a METACOMMENT !
%%
%% Package `theorem' to use with LaTeX2e
%% Copyright (C) 1989-1995 Frank Mittelbach, all rights reserved.
%<+package>\@ifundefined{theorem@style}{}{\endinput}
%<+package>\NeedsTeXFormat{LaTeX2e}[1995/06/01]
%<+package>\ProvidesPackage{theorem}
%<+package>         [\FMithmInfo]
%
% \fi
%
% \changes{v1.0}{87/05/07}{First draft.}
% \changes{v1.0a}{87/05/07}{`thistheoremstyle' added.}
% \changes{v1.1a}{87/07/29}{New concept: redefine the internal
%                          \protect\LaTeX{} commands `@xnthm' etc.}
% \changes{v1.1b}{87/07/31}{`theoremfont' used to specify font.}
% \changes{v1.2a}{88/01/08}{`@endtheorem' now globally defined.}
% \changes{v1.2b}{88/01/11}{`newtheorem' only allowed in the preamble.}
% \changes{v1.3a}{88/02/14}{`theoremfont' now a toks register.}
%
% \changes{v2.0a}{88/11/16}{New implementation and new documentation.}
% \changes{v2.0a}{88/11/16}{No layout changes but internal modifications
%                           old substyles won't work.}
% \changes{v2.0b}{89/01/30}{`global' used for all definitions.}
% \changes{v2.0c}{89/06/15}{Documentation translated to English.
%                            Thanks to LLL and Nina}
% \changes{v2.1a}{90/02/15}{Support for new font selection.}
% \changes{v2.2a}{94/02/03}{Cleanup for LaTeX2e}
%
% \DoNotIndex{\@endparenv,\@ifdefinable,\@ifnextchar,\@ifundefined}
% \DoNotIndex{\@namedef,\@nameuse,\@spaces,\@tempa,\@warning}
% \DoNotIndex{\ }
% \DoNotIndex{\begingroup,\bfseries}
% \DoNotIndex{\csname}
% \DoNotIndex{\def,\do,\docdate}
% \DoNotIndex{\endcsname,\endgroup,\endinput,\endtrivlist,\expandafter}
% \DoNotIndex{\filedate,\fileversion}
% \DoNotIndex{\gdef,\global}
% \DoNotIndex{\hbox,\hskip}
% \DoNotIndex{\ignorespaces,\input,\itshape,\item}
% \DoNotIndex{\labelsep,\let,\llap}
% \DoNotIndex{\makeatletter}
% \DoNotIndex{\noexpand}
% \DoNotIndex{\refstepcounter,\relax,\rlap}
% \DoNotIndex{\setlength,\slshape,\space,\string,\strut}
% \DoNotIndex{\the,\trivlist,\typeout}
% \DoNotIndex{\vbox}
% \DoNotIndex{\xdef}
%
% \def\deflabel#1{\ttfamily #1\hfill}
% \def\deflist#1{\begin{list}{}{\settowidth\labelwidth{\ttfamily #1}%
%                               \setlength\leftmargin\labelwidth
%                               \addtolength\leftmargin\labelsep
%                               \let\makelabel\deflabel}}
% \def\enddeflist{\end{list}}
% \def\envfont{\normalfont\ttfamily}
%
% \MakeShortVerb{\|}
%
% \newcommand{\TUB}{{\sl TUGboat\/}}
%
% \GetFileInfo{theorem.dtx}
% \title{An Extension of the
%        \LaTeX{} theorem environment\thanks{This file
%        has version number \fileversion, last
%        revised \filedate.}}
% \author{Frank Mittelbach}
%
% \MaintainedByLaTeXTeam{tools}
% \maketitle
%
% \begin{abstract}
%    The macros described in this paper yield an extension of the
%    \LaTeX{} theorem mechanism. It is designed to satisfy the
%    different requirements of various journals. Thus, the
%    layout of the ``theorems'' can be manipulated by determining a
%    ``style''. This article describes not only the use, but
%    also the definition, of the necessary macros.
% \end{abstract}
%
% \section*{Preface to version 2.2}
%
% For \LaTeXe{} this package did not need any fundamental changes. I
% only modified the messages generated so that theorem layout styles
% will show up with the |\listfiles| command and cleaned the section
% on the New Font Selection Scheme since this is now included in
% \LaTeX.
%
% \section*{Preface to version 2.1}
%
% This version is identical to 2.0g described in \TUB\ 10\#3 except
% for some internal defaults which are now set depending on the used
% font selection scheme.
%
% This was done to avoid unpleasant surprises if the new font selection
% scheme is in force. For further details see
% section~\ref{sec:newfonts} and \cite{bk:GMS94}.
%
% \section{Introduction}
%
% For our purposes here, ``theorems'' are labelled enunciations,
% often set off from the main text by extra space and a font change.
% Theorems, corollaries, conjectures, definitions, and remarks
% are all instances of ``theorems''.  The ``header'' of these structures
% is composed of a label (such as \textsc{Theorem} or \textsc{Remark})
% and a number which serializes an item in the sequence of items
% with the same label.
%
% Shortly after the introduction of \LaTeX{} at the Fachbereich
% Mathematik in Mainz, the desire to manipulate the layout of
% ``theorems'' arose.  In Mainz, the following two conventions came
% into general use:
% \begin{enumerate}
%   \item The number of the theorem is shown in the margin.
%   \item There is a line break at
%         the end of the theorem header.
% \end{enumerate}
%
% Additionally, some journals require different formats which depend on
% the ``sort of theorem'': e.g.\ often remarks and definitions are set
% in |\upshape|, while |\itshape| is employed for main theorems.
%
% Confronted with these requirements, a theorem environment was
% developed in Mainz which allows separate determination of the
% layout of the ``theorems sets'', comparable to |\pagestyle|.
%
%
% \section{The user interface}
% \subsection{Defining new theorem sets}
%
% \DescribeMacro\newtheorem
% As in the original \LaTeX{} version, the command |\newtheorem|
% defines a new ``theorem set'' or ``theorem-like structure''.
% Two required arguments name the new environment and give the text
% to be typeset with each instance of the new ``set'', while
% an optional argument determines how
% the ``set'' is enumerated:
% \begin{description}
%    \item[\ttfamily \bslash newtheorem\{foo\}\{bar\}]
%       The theorem set {\envfont foo} (whose name is \texttt{bar})
%       uses its own counter.
%    \item[\ttfamily \bslash newtheorem\{foo2\}{[foo]}\{bar2\}]
%       The theorem set {\envfont foo2} (printed name \texttt{bar2})
%       uses the same counter as the theorem set \texttt{foo}.
%    \item[\ttfamily \bslash newtheorem\{foo3\}\{bar3\}{[section]}]
%       The theorem set {\envfont foo3} (printed name \texttt{bar3}) is
%       enumerated within the counter \texttt{section}, i.e.\ with every
%       new |\section| the enumeration begins again with 1, and
%       the enumeration is composed from the section-number and the
%       theorem counter itself.
% \end{description}
%
% \DescribeMacro\theoremstyle
% Additionally, the command |\theoremstyle| can define the layout
% of various, or all, theorem sets.  It should be noted that any
% theorem set defined by |\newtheorem| is typeset in the
% |\theoremstyle| that is current at the time of the definition.
% Thus, the following
% \begin{verbatim}
%    \theoremstyle{break}        \newtheorem{Cor}{Corollary}
%    \theoremstyle{plain}        \newtheorem{Exa}{Example}[section]
%\end{verbatim}                             ^^A No blank at line start
% leads to the result that the set {\envfont Cor} is formatted in the
% style \texttt{break}, while the set {\envfont Exa} and all the
% following ones are formatted in the style \texttt{plain}, unless
% another |\theoremstyle| follows.  Since the definitions installed by
% |\newtheorem| are global, one also can limit |\theoremstyle| locally
% by grouping braces.
%
% \DescribeMacro\theorembodyfont
% The choice of the font for the theorem body is completely
% independent of the chosen |\theoremstyle|; this has proven to
% be very advantageous.  For example,
% \begin{verbatim}
%   {\theorembodyfont{\upshape}          \newtheorem{Rem}{Remark}}
%\end{verbatim}
% defines a theorem set {\envfont Rem}, which will be set in
% |\upshape| in the current layout (which in our example is
% \texttt{plain}).  As with |\theoremstyle|, the |\theorembodyfont|
% chosen is that current at the time of |\newtheorem|.  If
% |\theorembodyfont| is not specified or one defines
% |\theorembodyfont{}|, then the font used will be that defined by the
% |\theoremstyle|.
%
% \DescribeMacro\theoremheaderfont It is also possible to customize
% the font used for the theorem headers.  This is, however, a global
% declaration, and therefore there should be at most one
% |\theoremheaderfont| declaration in the preamble.\footnote{If it is
% actually necessary to have different header fonts, one has to define
% new theorem styles (substituting the desired font) or specify the
% information directly in the \texttt{\bslash newtheorem} declaration
% (the unclean variant).}
%
% \DescribeMacro\theorempreskipamount
% \DescribeMacro\theorempostskipamount Two additional parameters
% affect the vertical space around the theorem environments:
% |\theorempreskipamount| and |\theorempostskipamount| define,
% respectively, the spacing before and after such an environment.
% These parameters apply for all theorem sets and can be manipulated
% with the ordinary length macros.  They are rubber lengths
% (`\textsf{skips}'), and therefore can contain \texttt{plus} and
% \texttt{minus} parts.
%
% Since the definition of theorem sets should---most sensibly---be
% placed in the preamble, we only allow installation there.
% It is therefore possible to release the memory used here after
% |\begin{document}|, in order to make room for other
% applications.
%
%
% \subsection{Existing theorem styles}
%
% The following theorem styles exist to date:
% \begin{deflist}{marginbreak}
%    \item[plain]
%       This theorem style emulates the original \LaTeX{} definition,
%       except that additionally the parameters
%       |\theorem...skipamount| are used.
%    \item[break]
%       In this style, the theorem header is followed by a line
%       break.
%    \item[marginbreak]
%       The theorem number is set in the margin, and there is a line
%       break as in \texttt{break}.
%    \item[changebreak]
%       Like \texttt{break}, but with header number and text
%       interchanged.
%    \item[change]
%       Header number and text are interchanged, without a line break.
%    \item[margin]
%       The number is set in the left margin, without a line break.
% \end{deflist}
% All styles (except \texttt{plain}) select |\slshape| as the default
% |\theorembodyfont|.
%
%
% \subsection{Examples}
%
% Given the above theorem sets {\envfont Cor}, {\envfont Exa}
% and {\envfont Rem},
% suppose that the preamble also contains the declarations:
% \begin{verbatim}
%    \theoremstyle{marginbreak}   \newtheorem{Lem}[Cor]{Lemma}
%    \theoremstyle{change}
%    \theorembodyfont{\itshape}        \newtheorem{Def}[Cor]{Definition}
%
%    \theoremheaderfont{\scshape}
%\end{verbatim}
% Then the following are some typical examples of the typeset output
% resulting from their use.
%
% \begin{Cor}
%    This is a sentence typeset in the theorem environment {\envfont
%    Cor}.
% \end{Cor}
%
% \begin{Exa}
%    This is a sentence typeset in the theorem environment {\envfont
%    Exa}.
% \end{Exa}
%
% \begin{Rem}
%    This is a sentence typeset in the theorem environment {\envfont
%    Rem}.
% \end{Rem}
%
% \begin{Lem}[Ben User]
%    This is a sentence typeset in the theorem environment {\envfont
%    Lem}.
% \end{Lem}
%
% \begin{Def}[Very impressive Definition]
%    This is a sentence typeset in the theorem environment {\envfont
%    Def}.
% \end{Def}
%
% The last two examples show the effect of the optional argument to
% a theorem environment (it is the text typeset in parentheses).
%
%
% \section{Special Considerations}
% \label{sec:newfonts}
%
% Theoremheader and body are implemented as a unit. This means that
% the |\theoremheaderfont|  will inherit characteristics of the
% |\theorembodyfont| in \LaTeXe. Thus, if for example
% |\theorembodyfont| is |\itshape| and |\theoremheaderfont|
% is |\bfseries| the font selected for the header will have the
% characteristics `bold extended italic'. If this is not desired one
% should set the |\theoremheaderfont| to something like
% \begin{verbatim}
%      \theoremheaderfont{\normalfont\bfseries}
%\end{verbatim}
% i.e.\ supplying all necessary font information explicitly.
%
%
%
% \section{Acknowledgements}
%
% The publication of this set of macros was only possible with the
% help of Christina Busse (translating the manuscript into English),
% Joachim Pense (playing the r\^ole of typist), Chris Rowley
% (looking everything over) and many others providing useful
% suggestions.
%
% \MaybeStop{%
%    \begin{thebibliography}{9}
%       \bibitem{bk:GMS94} \textsc{M.~Goossens}, \textsc{F.~Mittelbach}
%         and \textsc{A.~Samarin}.
%         \newblock The \LaTeX{} Companion.
%         \newblock
%         Addison-Wesley, Reading, Massachusetts, 1994.
%       \bibitem{latex:source} \textsc{Lamport, Leslie.} \newblock
%                               \texttt{latex.tex}, version 2.09, date
%                               Feb. 1990.
%    \end{thebibliography}
%    \PrintIndex
%    \PrintChanges}
%
%
% \section{The documentation driver file}
%
% The next bit of code contains the documentation driver file for
% \TeX{}, i.e., the file that will produce the documentation you are
% currently reading. It will be extracted from this file by the
% \texttt{docstrip} program. Since it is the first code in the file
% one can alternatively process this file directly with \LaTeXe{} to
% obtain the documentation.
%
%    \begin{macrocode}
%<*driver>
\documentclass{ltxdoc}

\usepackage{theorem}

%<+driver>% The next few lines define theorem sets which are used
%<+driver>% in the example section of the documentation.

\theoremstyle{break}       \newtheorem{Cor}{Corollary}
\theoremstyle{plain}       \newtheorem{Exa}{Example}[section]
{\theorembodyfont{\upshape}\newtheorem{Rem}{Remark}}
\theoremstyle{marginbreak} \newtheorem{Lem}[Cor]{Lemma}
\theoremstyle{change}
\theorembodyfont{\itshape} \newtheorem{Def}[Cor]{Definition}

\theoremheaderfont{\scshape}

\RecordChanges

\begin{document}
   \DocInput{theorem.dtx}
\end{document}
%</driver>
%    \end{macrocode}
%
% \section{Definition of the Macros}
%
% If the file has been loaded before, we abort immediately. If not the
% package announces itself (this is actually done at the very top of
% the file---the way it is done isn't good style so don't copy it).
% \changes{v2.0e}{89/07/19}{Spaces removed from `typeout'.}
% \changes{v2.2b}{95/11/19}{Announce moved to top of file}
%    \begin{macrocode}
%<*package>
%\@ifundefined{theorem@style}{}{\endinput}
%\def\FMithmInfo{1995/11/19 v2.2b Theorem extension package (FMi)}
%\ProvidesPackage{theorem}[\FMithmInfo]
%    \end{macrocode}
%
%
% \subsection{Definition of theorem styles and fonts}
%
% All the definitions in this file are done globally to allow
% inputting this file inside a group.
%
% \begin{macro}{\theoremstyle}
%    Before a theorem style can be installed, the chosen style must be
%    known. For that reason, we must test to see that
%    |\th@|\meta{style} is known or, more precisely, that it is
%    different from |\relax|. If the style is not known then
%    |\th@plain| is used.
%    \begin{macrocode}
\gdef\theoremstyle#1{%
   \@ifundefined{th@#1}{\@warning
          {Unknown theoremstyle `#1'. Using `plain'}%
          \theorem@style{plain}}%
%    \end{macrocode}
%    We save the theorem style to be used in the token register
%    |\theorem@style|.
%    \begin{macrocode}
      {\theorem@style{#1}}%
%    \end{macrocode}
%    Now we ``evaluate'' the theorem style: this means, we call the
%    macro |\th@|\meta{style} which will activate the relevant
%    definitions which are contained in a separate file.  This is done
%    in a group to suppress changes to the current font.  This could
%    otherwise pose problems together with the new font selection
%    scheme\footnote{When I printed the original article using the new
%    font selection scheme I ended with a document with slanted
%    typefaces (text headings and all) simply because one of the
%    theorem styles used \texttt{\bslash sl} at toplevel.} if the
%    |\th@|\meta{style} is evaluated a second time.
% \changes{v2.1a}{90/02/15}{Group added around `th@..}
% \changes{v2.1a}{90/02/15}{Reseting of `..@begintheorem removed.}
%    \begin{macrocode}
      \begingroup
        \csname th@\the\theorem@style \endcsname
      \endgroup}
%    \end{macrocode}
% \end{macro}
%
%
% \begin{macro}{\@begintheorem}
% \changes{v2.1a}{90/02/15}{Relaxed at toplevel.}
% \begin{macro}{\@opargbegintheorem}
% \changes{v2.1a}{90/02/15}{Relaxed at toplevel.}
%    We reset |\@begintheorem| and |\@opargbegintheorem| to
%    |\relax| since these commands are no longer necessary at
%    toplevel. This will save a few tokens.
%    \begin{macrocode}
\global\let\@begintheorem\relax
\global\let\@opargbegintheorem\relax
%    \end{macrocode}
% \end{macro}
% \end{macro}
%
% \begin{macro}{\theorem@style}
%    Obviously the \textsf{token} register used above has to be
%    allocated. To assure the utmost compatibility with the
%    original \LaTeX{} definition, we set the default theorem style to
%    \texttt{plain}, which implements the usual \LaTeX{} convention.
% \changes{v2.0f}{89/07/20}{Unnecessary `global' removed.}
%    \begin{macrocode}
\newtoks\theorem@style
\global\theorem@style{plain}
%    \end{macrocode}
% \end{macro}
%
% \begin{macro}{\theorembodyfont}
% \begin{macro}{\theorem@bodyfont}
%    If the |\theorembodyfont| is set by the user then it should not
%    interact with the default font set in the theorem style. When the
%    new font selection is in force this may happen if, for example,
%    the default is |\itshape| and the new |\theorembodyfont| is
%    |\sffamily|. So we add a |\reset@font| command in front of the
%    user definition.
% \changes{v2.0f}{89/07/20}{Unnecessary `global' removed.}
% \changes{v2.0g}{89/09/19}{`theoremfont to `theorembodyfont.}
% \changes{v2.1c}{92/06/22}{`reset@font added and logic changed.}
% \changes{v2.1d}{92/10/28}{Reset to empty if arg is empty}
%    \begin{macrocode}
\gdef\theorembodyfont#1{%
%    \end{macrocode}
%    We check if the argument supplied is empty and if so put nothing
%    into the |\theorem@bodyfont| token register to allow for
%    |\theorembodyfont{}| as a mean of using the default of the
%    current |\theoremstyle|.
%    \begin{macrocode}
   \def\@tempa{#1}%
   \ifx\@tempa\@empty
    \theorem@bodyfont{}%
   \else
    \theorem@bodyfont{\reset@font#1}%
   \fi
   }
\newtoks\theorem@bodyfont
\global\theorem@bodyfont{}
%    \end{macrocode}
% \end{macro}
% \end{macro}
%
%
% \begin{macro}{\theoremheaderfont}
% \changes{v2.0g}{89/09/19}{Macro added.}
%    The font for the theorem headers is handled differently because
%    this definition applies to all theorem styles.
%    \begin{macrocode}
\gdef\theoremheaderfont#1{\gdef\theorem@headerfont{#1}%
%    \end{macrocode}
%    After using the macro once it is redefined to produce an error
%    message.
%    \begin{macrocode}
       \gdef\theoremheaderfont##1{%
        \typeout{\string\theoremheaderfont\space should be used
                 only once.}}}
%    \end{macrocode}
% \end{macro}
%
% \begin{macro}{\theorem@headerfont}
% \changes{v2.0g}{89/09/19}{Macro added.}
%    To set the |\theorem@headerfont| default we first test if
%    the new fontselection scheme is in force.
% \changes{v2.1a}{90/02/15}{Support for new font selection.}
%    \begin{macrocode}
\ifx\upshape\undefined
%    \end{macrocode}
%    If not we define it to expand into |\bfseries|. We don't use
%    |\let| just in case a following style option redefines this
%    macro.
%    \begin{macrocode}
\gdef\theorem@headerfont{\bfseries}
%    \end{macrocode}
%    Otherwise we reset the current shape before calling |\bfseries|.
%    \begin{macrocode}
\else \gdef\theorem@headerfont{\normalfont\bfseries}\fi
%    \end{macrocode}
% \end{macro}
%
% \begin{macro}{\th@plain}
% \begin{macro}{\th@break}
% \begin{macro}{\th@marginbreak}
% \begin{macro}{\th@changebreak}
% \begin{macro}{\th@change}
% \begin{macro}{\th@margin}
%    The different styles are defined in macros such as
%    |\th@plain|.  Since memory space is precious in
%    ``non-Big-versions'', we have to avoid offering too many unused
%    definitions.  Therefore we define these styles in separate files
%    that can be loaded on demand.  Thus the commands themselves only
%    load these files. We use |\@input@|, a \LaTeXe{} internal command
%    that ensures that the file will be listed with |\listfiles|.
% \changes{v2.2a}{94/02/02}{Use `@input@ to load theorem layout files}
%    \begin{macrocode}
\gdef\th@plain{\@input@{thp.sty}}
\gdef\th@break{\@input@{thb.sty}}
\gdef\th@marginbreak{\@input@{thmb.sty}}
\gdef\th@changebreak{\@input@{thcb.sty}}
\gdef\th@change{\@input@{thc.sty}}
\gdef\th@margin{\@input@{thm.sty}}
%    \end{macrocode}
%    This list will be expanded when new styles become
%    available.  For testing,  just append new theorem substyles
%    as document options.
% \end{macro}
% \end{macro}
% \end{macro}
% \end{macro}
% \end{macro}
% \end{macro}
%
%
%
% \subsection{Definition of a new theorem set}
%
% As already pointed out, a new theorem environment can be defined in
% three different ways:
% \begin{verbatim}
%    \newtheorem{Lem}{Lemma}
%    \newtheorem{Lem}{Lemma}[section]
%    \newtheorem{Lem}[Theorem]{Lemma}
%\end{verbatim}
% The function of the macro |\newtheorem| is to recognize these cases
% and then to branch into one of the three macros |\@ynthm|, |\@xnthm|
% or |\@othm|.  This mechanism is adopted unchanged from
% \cite{latex:source}; the essential point here is that, for example,
% in the second case, the arguments \texttt{Lem}, \texttt{Lemma} and
% \texttt{section} are passed over to the macro |\@xnthm|.
%
% We inspect this case first because the others present fewer problems,
% and thus are easily derived from this one.
%
% \begin{macro}{\@xnthm}
%    For our example arguments, the macro |\@xnthm| must fulfill
%    the following:
%    \begin{itemize}
%       \item
%          Define a new \LaTeX-counter `Lem'
%       \item
%          reset this counter within a |\section|
%       \item
%          define the macro |\theLem|
%       \item
%          define the environment macros |\Lem| and
%          |\endLem| using the current
%          |\theoremstyle| and |\theorem@bodyfont|.
%    \end{itemize}
%    Obviously, all this should happen only if the first argument
%    of |\@xnthm| (i.e.\ \texttt{Lem} in our example) is chosen so
%    as not to conflict with any previously defined commands or
%    environments.  This test is performed by the \LaTeX{} macro
%    |\@ifdefinable|.
%    \begin{macrocode}
\gdef\@xnthm#1#2[#3]{\expandafter\@ifdefinable\csname #1\endcsname
%    \end{macrocode}
%    Therefore, the first argument of |\@ifdefinable| is the
%    expansion (in the example, |\Lem|) of
%    |\csname||#1\endcsname|.  The second argument is
%    executed only if the test has been completed successfully.
%    \begin{macrocode}
   {%
%    \end{macrocode}
%    Now we define the new counter. The names of the \LaTeX{}
%    macros employed should speak for themselves:
% \changes{v2.2b}{1995/10/07}{Use \cs{@newctr}}
% \changes{v2.2c}{1995/11/23}{Use \cs{@newctr} properly.}
%    \begin{macrocode}
    \@definecounter{#1}\@newctr{#1}[#3]%
%    \end{macrocode}
%    Using |\@newctr| will give a proper error message if the counter
%    in |#3| is not defined.
%    In defining `|\theLem|' we must generate the desired macro
%    name by use of |\expandafter| and |\csname|.
%    \begin{macrocode}
    \expandafter\xdef\csname the#1\endcsname
%    \end{macrocode}
%    An |\xdef| is used in order to make the definition global,
%    and to ensure that it contains the replacement texts of
%    |\@thmcountersep| and |\@thmcounter|.\footnote{These
%    two macros can be defined by the document style.  Their default
%    values produce a `\texttt{.}' as separation and an arabic
%    representation of the number.} However, not everything should be
%    expanded. For example, it saves space to use |\thesection|
%    instead of its---at times---lengthy expansion.
%    \begin{macrocode}
      {\expandafter \noexpand \csname the#3\endcsname
       \@thmcountersep \@thmcounter{#1}}%
%    \end{macrocode}
%    Thus with the defaults of \LaTeX{}, |\theLem| would be
%    replaced by the command sequence |\thesection.\arabic{Lem}|.
%
%    We will now look at the definition of the macro which is executed
%    at the beginning of the actual environment (in our example this
%    macro is |\Lem|). It should be noted that we use an
%    ``|\expandafter| trick'' to expand only certain parts of the
%    replacement text at the time of the definition.
%    \begin{macrocode}
    \def\@tempa{\global\@namedef{#1}}%
    \expandafter \@tempa \expandafter{%
%    \end{macrocode}
%    First, the macro that contains the current definitions of
%    |\@begintheorem| and |\@opargtheorem| should be called
%    up.  The name of this macro---as is already known---has the form
%    |\th@|\meta{theorem\ style}; therefore, it must be called by
%    \begin{macrocode}
      \csname th@\the \theorem@style
%    \end{macrocode}
%    In addition the default theorem font should be changeable, i.e.\
%    we have to insert the contents of |\theorem@bodyfont|. For
%    that reason, we expand even further, beyond |\endcsname|,
%    and thus insert the contents of the \textsf{token} register
%    |\theorem@bodyfont| in the replacement text.
%    \begin{macrocode}
            \expandafter \endcsname \the \theorem@bodyfont
%    \end{macrocode}
%    Now it is time to call the macro |\@thm| which takes over
%    the further processing.  It has two arguments: the current
%    counter name (in our example, \texttt{Lem}), and the text of the
%    label (in our example, \texttt{Lemma}).
%    \begin{macrocode}
     \@thm{#1}{#2}}%
%    \end{macrocode}
%    With this, the `sub-definition' is complete.  The macro
%    |\@endtheorem| ends a theorem environment and is, so far, nothing
%    but an |\endtrivlist|.  (Hence it is defined globally, and not
%    within the theorem styles.\footnote{This has to be changed as
%    soon as theorem styles that change \texttt{\bslash @endtheorem}
%    exist.  In such a case, all existing styles must be changed as
%    well since they will have to reset the macro.}) Therefore, we can
%    set it equivalent to the macro that ends the theorem set (in our
%    example, |\endLem|).  However, if some day theorem styles exist
%    that do change |\@endtheorem|, we would have to use the
%    commented-out line instead.
%    \begin{macrocode}
    \global \expandafter \let \csname end#1\endcsname \@endtheorem
%   \global\@namedef{end#1}{\@endtheorem}%
%    \end{macrocode}
%    With these commands all the required definitions are employed,
%    unless the test |\@ifdefinable| has failed.  Therefore, we
%    end the second argument of this macro and with it the definition
%    of |\@xnthm|.
%    \begin{macrocode}
   }}
%    \end{macrocode}
% \end{macro}
%
% \begin{macro}{\@ynthm}
%    The definition of |\@ynthm| is completely analogous. In
%    this case the new counter that is defined is not reset within
%    another counter; thus the definition of |\the...|
%    is simplified:
%    \begin{macrocode}
\gdef\@ynthm#1#2{\expandafter\@ifdefinable\csname #1\endcsname
   {\@definecounter{#1}%
    \expandafter\xdef\csname the#1\endcsname{\@thmcounter{#1}}%
%    \end{macrocode}
%    The rest of the definition corresponds literally to that of
%    |\@xnthm|:
%    \begin{macrocode}
    \def\@tempa{\global\@namedef{#1}}\expandafter \@tempa
     \expandafter{\csname th@\the \theorem@style \expandafter
     \endcsname \the\theorem@bodyfont \@thm{#1}{#2}}%
    \global \expandafter \let \csname end#1\endcsname \@endtheorem}}
%    \end{macrocode}
% \end{macro}
%
% \begin{macro}{\@othm}
%    The definition of |\@othm| does not contain anything new.
%    \begin{macrocode}
\gdef\@othm#1[#2]#3{%
%    \end{macrocode}
% \changes{v2.2b}{95/10/07}{Build in check for counter defined}
%    We do not define a new counter but instead use one that has
%    already been defined.  Thus the only definition we need is that
%    of this pseudo-counter (i.e.\ |\the|\meta{env.\ name}).
%    First we check if |#2| corresponds to a known counter name.
%    \begin{macrocode}
  \expandafter\ifx\csname c@#2\endcsname\relax
   \@nocounterr{#2}%
  \else
   \expandafter\@ifdefinable\csname #1\endcsname
   {\expandafter \xdef \csname the#1\endcsname
     {\expandafter \noexpand \csname the#2\endcsname}%
%    \end{macrocode}
%    All other parts of the definition can be adopted from
%    |\@xnthm|. We have to remember, though, that in this case
%    the name of the current counter and the theorem label have moved
%    to the second and third arguments.
%    \begin{macrocode}
    \def\@tempa{\global\@namedef{#1}}\expandafter \@tempa
     \expandafter{\csname th@\the \theorem@style \expandafter
     \endcsname \the\theorem@bodyfont \@thm{#2}{#3}}%
    \global \expandafter \let \csname end#1\endcsname \@endtheorem}%
  \fi}
%    \end{macrocode}
% \end{macro}
%
% \subsection{Macros that are employed in a {\envfont theorem}
%    environment}
%
% \begin{macro}{\@thm}
%    The macro |\@thm| has to increase the current counter.  Then,
%    depending on whether the environment has (or does not have) an
%    optional argument, it has to branch into either |\@begintheorem|
%    or |\@opargtheorem|.
%    \begin{macrocode}
\gdef\@thm#1#2{\refstepcounter{#1}%
%    \end{macrocode}
%    Now we start a {\envfont trivlist} environment, and give
%    |\@topsep| and |\@topsepadd| the values of the \textsf{skip}
%    registers |\theorempreskipamount| and |\theorempostskipamount|.
%    The value in |\@topsep| is the vertical space that is inserted by
%    the first (and only) |\item| in our |\trivlist| whilst
%    |\@topsepadd| is inserted by |\@endparenv| at the end of that
%    {\envfont trivlist} environment.  By using these registers, we
%    obtain the desired space around a {\envfont theorem} environment.
%    \begin{macrocode}
   \trivlist
   \@topsep \theorempreskipamount               % used by first \item
   \@topsepadd \theorempostskipamount           % used by \@endparenv
%    \end{macrocode}
%    Now we have to test whether an optional argument has been given.
% \changes{v2.0d}{89/07/18}{Forgotten percent added.}
%    \begin{macrocode}
   \@ifnextchar [%
%    \end{macrocode}
%    If there is an optional argument, we will call |\@ythm|, and
%    move the arguments read back into the input stream.
%    \begin{macrocode}
   {\@ythm{#1}{#2}}%
%    \end{macrocode}
%    If not, we call |\@begintheorem|. Its first argument is the
%    name of the theorem set (hence the second argument of
%    |\@thm|). Its second argument is the macro that produces the
%    current number.
%    \begin{macrocode}
   {\@begintheorem{#2}{\csname the#1\endcsname}\ignorespaces}}
%    \end{macrocode}
% \end{macro}
%
% \begin{macro}{\@xthm}
% \begin{macro}{\@ythm}
%    Both these macros were originally called by |\@thm|.  We do
%    not need |\@xthm| anymore, hence we reset it to
%    |\relax|. The definition of |\@ythm| has not changed at
%    all from its definition in \LaTeX. In order to make the macros
%    easier to understand, we will
%    nevertheless present it (commented out).
%    \begin{macrocode}
\global\let\@xthm\relax
% \def\@ythm#1#2[#3]{\@opargbegintheorem{#2}{\csname
%           the#1\endcsname}{#3}\ignorespaces}
%</package>
%    \end{macrocode}
%    The primitive |\ignorespaces| in |\@ythm| and
%    |\@thm| is needed to remove the spaces between the
%    |\begin{...}| and the actual text.
% \end{macro}
% \end{macro}
%
%
% \subsection{Definition of the theorem substyles}
%
% As already pointed out, the theorem substyles, defined below, are
% only loaded when necessary.  Note that all these substyles, except
% \texttt{plain}, have |\slshape| as the default body font.
%
% \subsubsection{The \texttt{plain} style}
%
% \label{dok:th@plain} As the following macros use |@|, we have
% to locally set the |\catcode| of this symbol to ``letter''.
% This happens within a group, so that we do not have to worry about
% which |\catcode| that symbol had before.
%    \begin{macrocode}
%<*thp>
\begingroup \makeatletter
%    \end{macrocode}
% Since we are now within a group, we must make all definitions
% globally.  First we make sure that \texttt{theorem.sty} is loaded.
% This will allow us to use this file as a document style option
% without having to call \texttt{theorem} itself as an option.  At the
% same time, we assure that at least version 2 is loaded, since
% |\theorem@style| was not defined in earlier versions.
%    \begin{macrocode}
\@ifundefined{theorem@style}{\input{theorem.sty}}{}
\ProvidesFile{thp.sty}
             [\FMithmInfo]
%    \end{macrocode}
%
%
% \begin{macro}{\th@plain}
% \begin{macro}{\@begintheorem}
% \begin{macro}{\@opargbegintheorem}
%    |\theoremstyle{plain}| corresponds to the original
%    definition, except that the distances to the surrounding text are
%    determined by the parameters |\theorempreskipamount| and
%    |\theorempostskipamount|.
%    First we set the default body font.
%    \begin{macrocode}
\gdef\th@plain{\normalfont\itshape
%    \end{macrocode}
%    Then we define |\@begintheorem| and
%    |\@opargbegintheorem|. These two macros define how the
%    header of a theorem is typeset.  |\@opargbegintheorem| will
%    be called if a \textsf{theorem} environment with an optional
%    argument is encountered; otherwise, the header is constructed by
%    calling |\@begintheorem|.  If one of these macros is
%    executed, we are within a \textsf{trivlist} environment started by
%    |\@thm|.  So the theorem header is produced with an
%    |\item| command.
%
%    Instead of specifying the header font directly, all standard
%    theorem styles use the |\theorem@headerfont| macro to allow
%    customization.  The extra space (|\labelsep|) is necessary
%    because of problems in the \textsf{trivlist} environment.
%    \begin{macrocode}
  \def\@begintheorem##1##2{%
        \item[\hskip\labelsep \theorem@headerfont ##1\ ##2]}%
%    \end{macrocode}
%    The definition of |\@opargbegintheorem| is completely
%    analogous. The only difference is the fact that there exists a
%    third argument (which is the optional parameter of the
%    environment and contains additional information about the
%    theorem).  Customarily we enclose it in parentheses.
%    \begin{macrocode}
\def\@opargbegintheorem##1##2##3{%
   \item[\hskip\labelsep \theorem@headerfont ##1\ ##2\ (##3)]}}
%    \end{macrocode}
% \end{macro}
% \end{macro}
% \end{macro}
%
% \noindent
% We conclude with an |\endgroup| to restore the
% |\catcode| of |@|.
%    \begin{macrocode}
\endgroup
%</thp>
%    \end{macrocode}
%
%
% \subsubsection{The \texttt{break} style}
%
% This style option is stored in the file |thb.sty|.
% For the next two lines see the documentation for |\th@plain| on
% page~\pageref{dok:th@plain}.
%    \begin{macrocode}
%<*thb>
\begingroup \makeatletter
\@ifundefined{theorem@style}{\input{theorem.sty}}{}
\ProvidesFile{thb.sty}
             [\FMithmInfo]
%    \end{macrocode}
%
% \begin{macro}{\th@break}
%    |\theoremstyle{break}| produces a line break after the name
%    of the theorem.  The font is |\slshape|.  Hence, we define
%    |\th@break| as follows:
%    \begin{macrocode}
\gdef\th@break{\normalfont\slshape
  \def\@begintheorem##1##2{\item[%
%    \end{macrocode}
%    We run into the following problem: it is not possible to create
%    the header with |\item[|\meta{title}|]| and then start
%    a new line by, for example, |\mbox{}\\|. Such a definition
%    will fail whenever a list environment follows immediately.  With
%    the above construction, the |\mbox{}| causes the switch
%    |@inlabel| (cf.\ definition of |\list| and
%    |\trivlist| in \cite{latex:source}) to be set to \texttt{false}
%    and so the following list will insert additional vertical
%    space (|\topskip|). This is quite annoying. Therefore, we
%    create the line break within the |\item|. In order to ensure
%    that the text will begin at the proper position in the following
%    line, we simply pretend that the label does not take any
%    room.\footnote{This will lead to problems whenever very high
%    symbols occurring in the line tower into the
%    heading. So, something else has to be done here sometime.}
%    \begin{macrocode}
       \rlap{\vbox{\hbox{\hskip \labelsep\theorem@headerfont ##1\ ##2}%
                   \hbox{\strut}}}]}%
%    \end{macrocode}
%    Again, the definition of |\@opargbegintheorem| is
%    completely analogous.
%    \begin{macrocode}
\def\@opargbegintheorem##1##2##3{%
  \item[\rlap{\vbox{\hbox{\hskip \labelsep \theorem@headerfont
                     ##1\ ##2\ (##3)}%
                    \hbox{\strut}}}]}}
\endgroup
%</thb>
%    \end{macrocode}
% \end{macro}
%
%
% \subsubsection{The \texttt{changebreak} style}
%
%    \begin{macrocode}
% This style option is stored in the file |thcb.sty|.
%    \begin{macrocode}
%<*thcb>
\begingroup \makeatletter
\@ifundefined{theorem@style}{\input{theorem.sty}}{}
\ProvidesFile{thcb.sty}
             [\FMithmInfo]
%    \end{macrocode}
%
% \begin{macro}{\th@changebreak}
%    The \texttt{change-break} theorem style is like \texttt{break}
%    but with interchange of theorem name and theorem number.  Thus we
%    define |\th@changebreak| as follows:
%    \begin{macrocode}
\gdef\th@changebreak{\normalfont\slshape
  \def\@begintheorem##1##2{\item
       [\rlap{\vbox{\hbox{\hskip\labelsep \theorem@headerfont ##2\ ##1}%
                    \hbox{\strut}}}]}%
\def\@opargbegintheorem##1##2##3{%
  \item[\rlap{\vbox{\hbox{\hskip\labelsep \theorem@headerfont
                     ##2\ ##1\ (##3)}%
                    \hbox{\strut}}}]}}
\endgroup
%</thcb>
%    \end{macrocode}
% \end{macro}
%
%
% \subsubsection{The \texttt{change} style}
%
% This style option is stored in the file |thc.sty|.
%    \begin{macrocode}
%<*thc>
\begingroup \makeatletter
\@ifundefined{theorem@style}{\input{theorem.sty}}{}
\ProvidesFile{thc.sty}
             [\FMithmInfo]
%    \end{macrocode}
%
% \begin{macro}{\th@change}
%    The \texttt{change} theorem style corresponds to the
%    \texttt{change break} style without a linebreak after the header.
%    To say it in another way, it's the same as the \texttt{plain}
%    style but with number and name interchanged and |\slshape| as the
%    default font.
%    \begin{macrocode}
\gdef\th@change{\normalfont\slshape
  \def\@begintheorem##1##2{\item
       [\hskip\labelsep \theorem@headerfont ##2\ ##1]}%
\def\@opargbegintheorem##1##2##3{%
  \item[\hskip\labelsep \theorem@headerfont ##2\ ##1\ (##3)]}}
\endgroup
%</thc>
%    \end{macrocode}
% \end{macro}
%
%
% \subsubsection{The \texttt{marginbreak} style}
%
% This style option is the one used most often at Mainz.  It is saved
% in the file |thmb.sty|.
%    \begin{macrocode}
%<*thmb>
\begingroup \makeatletter
\@ifundefined{theorem@style}{\input{theorem.sty}}{}
\ProvidesFile{thmb.sty}
             [\FMithmInfo]
%    \end{macrocode}
%
% \begin{macro}{\th@marginbreak}
%    The \texttt{margin break} style is nearly the same as the
%    \texttt{change break} style. The only difference is the placement
%    of the theorem number.  We use |\llap| to place it in the left
%    margin.
%
%    In this style |\labelsep| denotes the separation between the
%    number and the text.
%    \begin{macrocode}
\gdef\th@marginbreak{\normalfont\slshape
  \def\@begintheorem##1##2{\item
       [\rlap{\vbox{\theorem@headerfont
                    \hbox{\llap{##2}\hskip\labelsep ##1}%
                    \hbox{\strut}}}]}%
\def\@opargbegintheorem##1##2##3{%
  \item[\rlap{\vbox{\theorem@headerfont
                    \hbox{\llap{##2}\hskip\labelsep ##1\ (##3)}%
                    \hbox{\strut}}}]}}
\endgroup
%</thmb>
%    \end{macrocode}
% \changes{v2.1c}{90/02/17}{`endgroup outside macrocode env.}
% \end{macro}
%
%
%
% \subsubsection{The \texttt{margin} style}
%
% This style option is stored in the file |thm.sty|.
%    \begin{macrocode}
%<*thm>
\begingroup \makeatletter
\@ifundefined{theorem@style}{\input{theorem.sty}}{}
\ProvidesFile{thm.sty}
             [\FMithmInfo]
%    \end{macrocode}
%
% \begin{macro}{\th@margin}
%    Again this is only a variant of the theorem styles described
%    above without any new ideas.
%    \begin{macrocode}
\gdef\th@margin{\normalfont\slshape
  \def\@begintheorem##1##2{\item
       [\theorem@headerfont \llap{##2}\hskip\labelsep ##1]}%
\def\@opargbegintheorem##1##2##3{%
  \item[\theorem@headerfont \llap{##2}\hskip\labelsep ##1\ (##3)]}}
\endgroup
%</thm>
%    \end{macrocode}
% \end{macro}
%
%
%
%
% \subsection{Final Definitions}
%
% \begin{macro}{\theorempreskipamount}
% \begin{macro}{\theorempostskipamount}
%    The \textsf{skip} parameters that regulate the vertical empty space
%    before and after the {\envfont theorem} environment have
%    to be allocated as well.
% \changes{v2.0f}{89/07/20}{Unnecessary `global' removed.}
%    \begin{macrocode}
%<*package>
\newskip\theorempreskipamount
\newskip\theorempostskipamount
%    \end{macrocode}
%    Since we have used the same values for all theorem sets, we now can
%    assign them.
%    \begin{macrocode}
\global\setlength\theorempreskipamount{12pt plus 5pt minus 3pt}
\global\setlength\theorempostskipamount{8pt plus 3pt minus 1.5pt}
%    \end{macrocode}
% \end{macro}
% \end{macro}
%
%
% \begin{macro}{\@endtheorem}
%    The same holds for the macro |\@endtheorem|, which ends a
%    {\envfont theorem} environment. Since it is the same for all
%    theorem sets, it is removed from the macros |\th@|\meta{style}.
%    It simply ends the {\envfont trivlist} environment, which was
%    begun in |\@thm|.
%    \begin{macrocode}
\global\let\@endtheorem=\endtrivlist
%    \end{macrocode}
% \end{macro}
%
%
%
% \begin{macro}{\@preamblecmds}
%    All macros defined above are to be used only in the preamble.
%    Therefore, we insert them in |\@preamblecmds| which will disable
%    them at begin document. This is done by the internal \LaTeXe{}
%    command |\@onlypreamble|.
% \changes{v2.0g}{89/09/19}{`theoremheaderfont and `theorembodyfont
%    added.}
% \changes{v2.2a}{94/02/03}{Use \LaTeXe{} interface}
%    \begin{macrocode}
\@onlypreamble\@xnthm
\@onlypreamble\@ynthm
\@onlypreamble\@othm
\@onlypreamble\newtheorem
\@onlypreamble\theoremstyle
\@onlypreamble\theorembodyfont
\@onlypreamble\theoremheaderfont
%    \end{macrocode}
% \end{macro}
%
% Finally we declare the \texttt{plain} theorem style to be the default.
% \changes{v2.1b}{90/06/04}{Fixed missing `item by declaring default
%    style.}
%    \begin{macrocode}
\theoremstyle{plain}
%</package>
%    \end{macrocode}
%
% \Finale

\endinput
}{}
\ProvidesFile{thmb.sty}
             [\FMithmInfo]
%    \end{macrocode}
%
% \begin{macro}{\th@marginbreak}
%    The \texttt{margin break} style is nearly the same as the
%    \texttt{change break} style. The only difference is the placement
%    of the theorem number.  We use |\llap| to place it in the left
%    margin.
%
%    In this style |\labelsep| denotes the separation between the
%    number and the text.
%    \begin{macrocode}
\gdef\th@marginbreak{\normalfont\slshape
  \def\@begintheorem##1##2{\item
       [\rlap{\vbox{\theorem@headerfont
                    \hbox{\llap{##2}\hskip\labelsep ##1}%
                    \hbox{\strut}}}]}%
\def\@opargbegintheorem##1##2##3{%
  \item[\rlap{\vbox{\theorem@headerfont
                    \hbox{\llap{##2}\hskip\labelsep ##1\ (##3)}%
                    \hbox{\strut}}}]}}
\endgroup
%</thmb>
%    \end{macrocode}
% \changes{v2.1c}{90/02/17}{`endgroup outside macrocode env.}
% \end{macro}
%
%
%
% \subsubsection{The \texttt{margin} style}
%
% This style option is stored in the file |thm.sty|.
%    \begin{macrocode}
%<*thm>
\begingroup \makeatletter
\@ifundefined{theorem@style}{% \iffalse meta-comment
%
% Copyright (C) 1993-2024
%
% The LaTeX Project and any individual authors listed elsewhere
% in this file.
%
% This file is part of the Standard LaTeX `Tools Bundle'.
% -------------------------------------------------------
%
% It may be distributed and/or modified under the
% conditions of the LaTeX Project Public License, either version 1.3c
% of this license or (at your option) any later version.
% The latest version of this license is in
%    https://www.latex-project.org/lppl.txt
% and version 1.3c or later is part of all distributions of LaTeX
% version 2005/12/01 or later.
%
% The list of all files belonging to the LaTeX `Tools Bundle' is
% given in the file `manifest.txt'.
%
% \fi
%
%%
%
%
\def\FMithmInfo{2023/07/05 v2.2c Theorem extension package (FMi)}
%
% \ProvidesFile{theorem.dtx}[\FMithmInfo]
%\iffalse   % this is a METACOMMENT !
%%
%% Package `theorem' to use with LaTeX2e
%% Copyright (C) 1989-1995 Frank Mittelbach, all rights reserved.
%<+package>\@ifundefined{theorem@style}{}{\endinput}
%<+package>\NeedsTeXFormat{LaTeX2e}[1995/06/01]
%<+package>\ProvidesPackage{theorem}
%<+package>         [\FMithmInfo]
%
% \fi
%
% \changes{v1.0}{87/05/07}{First draft.}
% \changes{v1.0a}{87/05/07}{`thistheoremstyle' added.}
% \changes{v1.1a}{87/07/29}{New concept: redefine the internal
%                          \protect\LaTeX{} commands `@xnthm' etc.}
% \changes{v1.1b}{87/07/31}{`theoremfont' used to specify font.}
% \changes{v1.2a}{88/01/08}{`@endtheorem' now globally defined.}
% \changes{v1.2b}{88/01/11}{`newtheorem' only allowed in the preamble.}
% \changes{v1.3a}{88/02/14}{`theoremfont' now a toks register.}
%
% \changes{v2.0a}{88/11/16}{New implementation and new documentation.}
% \changes{v2.0a}{88/11/16}{No layout changes but internal modifications
%                           old substyles won't work.}
% \changes{v2.0b}{89/01/30}{`global' used for all definitions.}
% \changes{v2.0c}{89/06/15}{Documentation translated to English.
%                            Thanks to LLL and Nina}
% \changes{v2.1a}{90/02/15}{Support for new font selection.}
% \changes{v2.2a}{94/02/03}{Cleanup for LaTeX2e}
%
% \DoNotIndex{\@endparenv,\@ifdefinable,\@ifnextchar,\@ifundefined}
% \DoNotIndex{\@namedef,\@nameuse,\@spaces,\@tempa,\@warning}
% \DoNotIndex{\ }
% \DoNotIndex{\begingroup,\bfseries}
% \DoNotIndex{\csname}
% \DoNotIndex{\def,\do,\docdate}
% \DoNotIndex{\endcsname,\endgroup,\endinput,\endtrivlist,\expandafter}
% \DoNotIndex{\filedate,\fileversion}
% \DoNotIndex{\gdef,\global}
% \DoNotIndex{\hbox,\hskip}
% \DoNotIndex{\ignorespaces,\input,\itshape,\item}
% \DoNotIndex{\labelsep,\let,\llap}
% \DoNotIndex{\makeatletter}
% \DoNotIndex{\noexpand}
% \DoNotIndex{\refstepcounter,\relax,\rlap}
% \DoNotIndex{\setlength,\slshape,\space,\string,\strut}
% \DoNotIndex{\the,\trivlist,\typeout}
% \DoNotIndex{\vbox}
% \DoNotIndex{\xdef}
%
% \def\deflabel#1{\ttfamily #1\hfill}
% \def\deflist#1{\begin{list}{}{\settowidth\labelwidth{\ttfamily #1}%
%                               \setlength\leftmargin\labelwidth
%                               \addtolength\leftmargin\labelsep
%                               \let\makelabel\deflabel}}
% \def\enddeflist{\end{list}}
% \def\envfont{\normalfont\ttfamily}
%
% \MakeShortVerb{\|}
%
% \newcommand{\TUB}{{\sl TUGboat\/}}
%
% \GetFileInfo{theorem.dtx}
% \title{An Extension of the
%        \LaTeX{} theorem environment\thanks{This file
%        has version number \fileversion, last
%        revised \filedate.}}
% \author{Frank Mittelbach}
%
% \MaintainedByLaTeXTeam{tools}
% \maketitle
%
% \begin{abstract}
%    The macros described in this paper yield an extension of the
%    \LaTeX{} theorem mechanism. It is designed to satisfy the
%    different requirements of various journals. Thus, the
%    layout of the ``theorems'' can be manipulated by determining a
%    ``style''. This article describes not only the use, but
%    also the definition, of the necessary macros.
% \end{abstract}
%
% \section*{Preface to version 2.2}
%
% For \LaTeXe{} this package did not need any fundamental changes. I
% only modified the messages generated so that theorem layout styles
% will show up with the |\listfiles| command and cleaned the section
% on the New Font Selection Scheme since this is now included in
% \LaTeX.
%
% \section*{Preface to version 2.1}
%
% This version is identical to 2.0g described in \TUB\ 10\#3 except
% for some internal defaults which are now set depending on the used
% font selection scheme.
%
% This was done to avoid unpleasant surprises if the new font selection
% scheme is in force. For further details see
% section~\ref{sec:newfonts} and \cite{bk:GMS94}.
%
% \section{Introduction}
%
% For our purposes here, ``theorems'' are labelled enunciations,
% often set off from the main text by extra space and a font change.
% Theorems, corollaries, conjectures, definitions, and remarks
% are all instances of ``theorems''.  The ``header'' of these structures
% is composed of a label (such as \textsc{Theorem} or \textsc{Remark})
% and a number which serializes an item in the sequence of items
% with the same label.
%
% Shortly after the introduction of \LaTeX{} at the Fachbereich
% Mathematik in Mainz, the desire to manipulate the layout of
% ``theorems'' arose.  In Mainz, the following two conventions came
% into general use:
% \begin{enumerate}
%   \item The number of the theorem is shown in the margin.
%   \item There is a line break at
%         the end of the theorem header.
% \end{enumerate}
%
% Additionally, some journals require different formats which depend on
% the ``sort of theorem'': e.g.\ often remarks and definitions are set
% in |\upshape|, while |\itshape| is employed for main theorems.
%
% Confronted with these requirements, a theorem environment was
% developed in Mainz which allows separate determination of the
% layout of the ``theorems sets'', comparable to |\pagestyle|.
%
%
% \section{The user interface}
% \subsection{Defining new theorem sets}
%
% \DescribeMacro\newtheorem
% As in the original \LaTeX{} version, the command |\newtheorem|
% defines a new ``theorem set'' or ``theorem-like structure''.
% Two required arguments name the new environment and give the text
% to be typeset with each instance of the new ``set'', while
% an optional argument determines how
% the ``set'' is enumerated:
% \begin{description}
%    \item[\ttfamily \bslash newtheorem\{foo\}\{bar\}]
%       The theorem set {\envfont foo} (whose name is \texttt{bar})
%       uses its own counter.
%    \item[\ttfamily \bslash newtheorem\{foo2\}{[foo]}\{bar2\}]
%       The theorem set {\envfont foo2} (printed name \texttt{bar2})
%       uses the same counter as the theorem set \texttt{foo}.
%    \item[\ttfamily \bslash newtheorem\{foo3\}\{bar3\}{[section]}]
%       The theorem set {\envfont foo3} (printed name \texttt{bar3}) is
%       enumerated within the counter \texttt{section}, i.e.\ with every
%       new |\section| the enumeration begins again with 1, and
%       the enumeration is composed from the section-number and the
%       theorem counter itself.
% \end{description}
%
% \DescribeMacro\theoremstyle
% Additionally, the command |\theoremstyle| can define the layout
% of various, or all, theorem sets.  It should be noted that any
% theorem set defined by |\newtheorem| is typeset in the
% |\theoremstyle| that is current at the time of the definition.
% Thus, the following
% \begin{verbatim}
%    \theoremstyle{break}        \newtheorem{Cor}{Corollary}
%    \theoremstyle{plain}        \newtheorem{Exa}{Example}[section]
%\end{verbatim}                             ^^A No blank at line start
% leads to the result that the set {\envfont Cor} is formatted in the
% style \texttt{break}, while the set {\envfont Exa} and all the
% following ones are formatted in the style \texttt{plain}, unless
% another |\theoremstyle| follows.  Since the definitions installed by
% |\newtheorem| are global, one also can limit |\theoremstyle| locally
% by grouping braces.
%
% \DescribeMacro\theorembodyfont
% The choice of the font for the theorem body is completely
% independent of the chosen |\theoremstyle|; this has proven to
% be very advantageous.  For example,
% \begin{verbatim}
%   {\theorembodyfont{\upshape}          \newtheorem{Rem}{Remark}}
%\end{verbatim}
% defines a theorem set {\envfont Rem}, which will be set in
% |\upshape| in the current layout (which in our example is
% \texttt{plain}).  As with |\theoremstyle|, the |\theorembodyfont|
% chosen is that current at the time of |\newtheorem|.  If
% |\theorembodyfont| is not specified or one defines
% |\theorembodyfont{}|, then the font used will be that defined by the
% |\theoremstyle|.
%
% \DescribeMacro\theoremheaderfont It is also possible to customize
% the font used for the theorem headers.  This is, however, a global
% declaration, and therefore there should be at most one
% |\theoremheaderfont| declaration in the preamble.\footnote{If it is
% actually necessary to have different header fonts, one has to define
% new theorem styles (substituting the desired font) or specify the
% information directly in the \texttt{\bslash newtheorem} declaration
% (the unclean variant).}
%
% \DescribeMacro\theorempreskipamount
% \DescribeMacro\theorempostskipamount Two additional parameters
% affect the vertical space around the theorem environments:
% |\theorempreskipamount| and |\theorempostskipamount| define,
% respectively, the spacing before and after such an environment.
% These parameters apply for all theorem sets and can be manipulated
% with the ordinary length macros.  They are rubber lengths
% (`\textsf{skips}'), and therefore can contain \texttt{plus} and
% \texttt{minus} parts.
%
% Since the definition of theorem sets should---most sensibly---be
% placed in the preamble, we only allow installation there.
% It is therefore possible to release the memory used here after
% |\begin{document}|, in order to make room for other
% applications.
%
%
% \subsection{Existing theorem styles}
%
% The following theorem styles exist to date:
% \begin{deflist}{marginbreak}
%    \item[plain]
%       This theorem style emulates the original \LaTeX{} definition,
%       except that additionally the parameters
%       |\theorem...skipamount| are used.
%    \item[break]
%       In this style, the theorem header is followed by a line
%       break.
%    \item[marginbreak]
%       The theorem number is set in the margin, and there is a line
%       break as in \texttt{break}.
%    \item[changebreak]
%       Like \texttt{break}, but with header number and text
%       interchanged.
%    \item[change]
%       Header number and text are interchanged, without a line break.
%    \item[margin]
%       The number is set in the left margin, without a line break.
% \end{deflist}
% All styles (except \texttt{plain}) select |\slshape| as the default
% |\theorembodyfont|.
%
%
% \subsection{Examples}
%
% Given the above theorem sets {\envfont Cor}, {\envfont Exa}
% and {\envfont Rem},
% suppose that the preamble also contains the declarations:
% \begin{verbatim}
%    \theoremstyle{marginbreak}   \newtheorem{Lem}[Cor]{Lemma}
%    \theoremstyle{change}
%    \theorembodyfont{\itshape}        \newtheorem{Def}[Cor]{Definition}
%
%    \theoremheaderfont{\scshape}
%\end{verbatim}
% Then the following are some typical examples of the typeset output
% resulting from their use.
%
% \begin{Cor}
%    This is a sentence typeset in the theorem environment {\envfont
%    Cor}.
% \end{Cor}
%
% \begin{Exa}
%    This is a sentence typeset in the theorem environment {\envfont
%    Exa}.
% \end{Exa}
%
% \begin{Rem}
%    This is a sentence typeset in the theorem environment {\envfont
%    Rem}.
% \end{Rem}
%
% \begin{Lem}[Ben User]
%    This is a sentence typeset in the theorem environment {\envfont
%    Lem}.
% \end{Lem}
%
% \begin{Def}[Very impressive Definition]
%    This is a sentence typeset in the theorem environment {\envfont
%    Def}.
% \end{Def}
%
% The last two examples show the effect of the optional argument to
% a theorem environment (it is the text typeset in parentheses).
%
%
% \section{Special Considerations}
% \label{sec:newfonts}
%
% Theoremheader and body are implemented as a unit. This means that
% the |\theoremheaderfont|  will inherit characteristics of the
% |\theorembodyfont| in \LaTeXe. Thus, if for example
% |\theorembodyfont| is |\itshape| and |\theoremheaderfont|
% is |\bfseries| the font selected for the header will have the
% characteristics `bold extended italic'. If this is not desired one
% should set the |\theoremheaderfont| to something like
% \begin{verbatim}
%      \theoremheaderfont{\normalfont\bfseries}
%\end{verbatim}
% i.e.\ supplying all necessary font information explicitly.
%
%
%
% \section{Acknowledgements}
%
% The publication of this set of macros was only possible with the
% help of Christina Busse (translating the manuscript into English),
% Joachim Pense (playing the r\^ole of typist), Chris Rowley
% (looking everything over) and many others providing useful
% suggestions.
%
% \MaybeStop{%
%    \begin{thebibliography}{9}
%       \bibitem{bk:GMS94} \textsc{M.~Goossens}, \textsc{F.~Mittelbach}
%         and \textsc{A.~Samarin}.
%         \newblock The \LaTeX{} Companion.
%         \newblock
%         Addison-Wesley, Reading, Massachusetts, 1994.
%       \bibitem{latex:source} \textsc{Lamport, Leslie.} \newblock
%                               \texttt{latex.tex}, version 2.09, date
%                               Feb. 1990.
%    \end{thebibliography}
%    \PrintIndex
%    \PrintChanges}
%
%
% \section{The documentation driver file}
%
% The next bit of code contains the documentation driver file for
% \TeX{}, i.e., the file that will produce the documentation you are
% currently reading. It will be extracted from this file by the
% \texttt{docstrip} program. Since it is the first code in the file
% one can alternatively process this file directly with \LaTeXe{} to
% obtain the documentation.
%
%    \begin{macrocode}
%<*driver>
\documentclass{ltxdoc}

\usepackage{theorem}

%<+driver>% The next few lines define theorem sets which are used
%<+driver>% in the example section of the documentation.

\theoremstyle{break}       \newtheorem{Cor}{Corollary}
\theoremstyle{plain}       \newtheorem{Exa}{Example}[section]
{\theorembodyfont{\upshape}\newtheorem{Rem}{Remark}}
\theoremstyle{marginbreak} \newtheorem{Lem}[Cor]{Lemma}
\theoremstyle{change}
\theorembodyfont{\itshape} \newtheorem{Def}[Cor]{Definition}

\theoremheaderfont{\scshape}

\RecordChanges

\begin{document}
   \DocInput{theorem.dtx}
\end{document}
%</driver>
%    \end{macrocode}
%
% \section{Definition of the Macros}
%
% If the file has been loaded before, we abort immediately. If not the
% package announces itself (this is actually done at the very top of
% the file---the way it is done isn't good style so don't copy it).
% \changes{v2.0e}{89/07/19}{Spaces removed from `typeout'.}
% \changes{v2.2b}{95/11/19}{Announce moved to top of file}
%    \begin{macrocode}
%<*package>
%\@ifundefined{theorem@style}{}{\endinput}
%\def\FMithmInfo{1995/11/19 v2.2b Theorem extension package (FMi)}
%\ProvidesPackage{theorem}[\FMithmInfo]
%    \end{macrocode}
%
%
% \subsection{Definition of theorem styles and fonts}
%
% All the definitions in this file are done globally to allow
% inputting this file inside a group.
%
% \begin{macro}{\theoremstyle}
%    Before a theorem style can be installed, the chosen style must be
%    known. For that reason, we must test to see that
%    |\th@|\meta{style} is known or, more precisely, that it is
%    different from |\relax|. If the style is not known then
%    |\th@plain| is used.
%    \begin{macrocode}
\gdef\theoremstyle#1{%
   \@ifundefined{th@#1}{\@warning
          {Unknown theoremstyle `#1'. Using `plain'}%
          \theorem@style{plain}}%
%    \end{macrocode}
%    We save the theorem style to be used in the token register
%    |\theorem@style|.
%    \begin{macrocode}
      {\theorem@style{#1}}%
%    \end{macrocode}
%    Now we ``evaluate'' the theorem style: this means, we call the
%    macro |\th@|\meta{style} which will activate the relevant
%    definitions which are contained in a separate file.  This is done
%    in a group to suppress changes to the current font.  This could
%    otherwise pose problems together with the new font selection
%    scheme\footnote{When I printed the original article using the new
%    font selection scheme I ended with a document with slanted
%    typefaces (text headings and all) simply because one of the
%    theorem styles used \texttt{\bslash sl} at toplevel.} if the
%    |\th@|\meta{style} is evaluated a second time.
% \changes{v2.1a}{90/02/15}{Group added around `th@..}
% \changes{v2.1a}{90/02/15}{Reseting of `..@begintheorem removed.}
%    \begin{macrocode}
      \begingroup
        \csname th@\the\theorem@style \endcsname
      \endgroup}
%    \end{macrocode}
% \end{macro}
%
%
% \begin{macro}{\@begintheorem}
% \changes{v2.1a}{90/02/15}{Relaxed at toplevel.}
% \begin{macro}{\@opargbegintheorem}
% \changes{v2.1a}{90/02/15}{Relaxed at toplevel.}
%    We reset |\@begintheorem| and |\@opargbegintheorem| to
%    |\relax| since these commands are no longer necessary at
%    toplevel. This will save a few tokens.
%    \begin{macrocode}
\global\let\@begintheorem\relax
\global\let\@opargbegintheorem\relax
%    \end{macrocode}
% \end{macro}
% \end{macro}
%
% \begin{macro}{\theorem@style}
%    Obviously the \textsf{token} register used above has to be
%    allocated. To assure the utmost compatibility with the
%    original \LaTeX{} definition, we set the default theorem style to
%    \texttt{plain}, which implements the usual \LaTeX{} convention.
% \changes{v2.0f}{89/07/20}{Unnecessary `global' removed.}
%    \begin{macrocode}
\newtoks\theorem@style
\global\theorem@style{plain}
%    \end{macrocode}
% \end{macro}
%
% \begin{macro}{\theorembodyfont}
% \begin{macro}{\theorem@bodyfont}
%    If the |\theorembodyfont| is set by the user then it should not
%    interact with the default font set in the theorem style. When the
%    new font selection is in force this may happen if, for example,
%    the default is |\itshape| and the new |\theorembodyfont| is
%    |\sffamily|. So we add a |\reset@font| command in front of the
%    user definition.
% \changes{v2.0f}{89/07/20}{Unnecessary `global' removed.}
% \changes{v2.0g}{89/09/19}{`theoremfont to `theorembodyfont.}
% \changes{v2.1c}{92/06/22}{`reset@font added and logic changed.}
% \changes{v2.1d}{92/10/28}{Reset to empty if arg is empty}
%    \begin{macrocode}
\gdef\theorembodyfont#1{%
%    \end{macrocode}
%    We check if the argument supplied is empty and if so put nothing
%    into the |\theorem@bodyfont| token register to allow for
%    |\theorembodyfont{}| as a mean of using the default of the
%    current |\theoremstyle|.
%    \begin{macrocode}
   \def\@tempa{#1}%
   \ifx\@tempa\@empty
    \theorem@bodyfont{}%
   \else
    \theorem@bodyfont{\reset@font#1}%
   \fi
   }
\newtoks\theorem@bodyfont
\global\theorem@bodyfont{}
%    \end{macrocode}
% \end{macro}
% \end{macro}
%
%
% \begin{macro}{\theoremheaderfont}
% \changes{v2.0g}{89/09/19}{Macro added.}
%    The font for the theorem headers is handled differently because
%    this definition applies to all theorem styles.
%    \begin{macrocode}
\gdef\theoremheaderfont#1{\gdef\theorem@headerfont{#1}%
%    \end{macrocode}
%    After using the macro once it is redefined to produce an error
%    message.
%    \begin{macrocode}
       \gdef\theoremheaderfont##1{%
        \typeout{\string\theoremheaderfont\space should be used
                 only once.}}}
%    \end{macrocode}
% \end{macro}
%
% \begin{macro}{\theorem@headerfont}
% \changes{v2.0g}{89/09/19}{Macro added.}
%    To set the |\theorem@headerfont| default we first test if
%    the new fontselection scheme is in force.
% \changes{v2.1a}{90/02/15}{Support for new font selection.}
%    \begin{macrocode}
\ifx\upshape\undefined
%    \end{macrocode}
%    If not we define it to expand into |\bfseries|. We don't use
%    |\let| just in case a following style option redefines this
%    macro.
%    \begin{macrocode}
\gdef\theorem@headerfont{\bfseries}
%    \end{macrocode}
%    Otherwise we reset the current shape before calling |\bfseries|.
%    \begin{macrocode}
\else \gdef\theorem@headerfont{\normalfont\bfseries}\fi
%    \end{macrocode}
% \end{macro}
%
% \begin{macro}{\th@plain}
% \begin{macro}{\th@break}
% \begin{macro}{\th@marginbreak}
% \begin{macro}{\th@changebreak}
% \begin{macro}{\th@change}
% \begin{macro}{\th@margin}
%    The different styles are defined in macros such as
%    |\th@plain|.  Since memory space is precious in
%    ``non-Big-versions'', we have to avoid offering too many unused
%    definitions.  Therefore we define these styles in separate files
%    that can be loaded on demand.  Thus the commands themselves only
%    load these files. We use |\@input@|, a \LaTeXe{} internal command
%    that ensures that the file will be listed with |\listfiles|.
% \changes{v2.2a}{94/02/02}{Use `@input@ to load theorem layout files}
%    \begin{macrocode}
\gdef\th@plain{\@input@{thp.sty}}
\gdef\th@break{\@input@{thb.sty}}
\gdef\th@marginbreak{\@input@{thmb.sty}}
\gdef\th@changebreak{\@input@{thcb.sty}}
\gdef\th@change{\@input@{thc.sty}}
\gdef\th@margin{\@input@{thm.sty}}
%    \end{macrocode}
%    This list will be expanded when new styles become
%    available.  For testing,  just append new theorem substyles
%    as document options.
% \end{macro}
% \end{macro}
% \end{macro}
% \end{macro}
% \end{macro}
% \end{macro}
%
%
%
% \subsection{Definition of a new theorem set}
%
% As already pointed out, a new theorem environment can be defined in
% three different ways:
% \begin{verbatim}
%    \newtheorem{Lem}{Lemma}
%    \newtheorem{Lem}{Lemma}[section]
%    \newtheorem{Lem}[Theorem]{Lemma}
%\end{verbatim}
% The function of the macro |\newtheorem| is to recognize these cases
% and then to branch into one of the three macros |\@ynthm|, |\@xnthm|
% or |\@othm|.  This mechanism is adopted unchanged from
% \cite{latex:source}; the essential point here is that, for example,
% in the second case, the arguments \texttt{Lem}, \texttt{Lemma} and
% \texttt{section} are passed over to the macro |\@xnthm|.
%
% We inspect this case first because the others present fewer problems,
% and thus are easily derived from this one.
%
% \begin{macro}{\@xnthm}
%    For our example arguments, the macro |\@xnthm| must fulfill
%    the following:
%    \begin{itemize}
%       \item
%          Define a new \LaTeX-counter `Lem'
%       \item
%          reset this counter within a |\section|
%       \item
%          define the macro |\theLem|
%       \item
%          define the environment macros |\Lem| and
%          |\endLem| using the current
%          |\theoremstyle| and |\theorem@bodyfont|.
%    \end{itemize}
%    Obviously, all this should happen only if the first argument
%    of |\@xnthm| (i.e.\ \texttt{Lem} in our example) is chosen so
%    as not to conflict with any previously defined commands or
%    environments.  This test is performed by the \LaTeX{} macro
%    |\@ifdefinable|.
%    \begin{macrocode}
\gdef\@xnthm#1#2[#3]{\expandafter\@ifdefinable\csname #1\endcsname
%    \end{macrocode}
%    Therefore, the first argument of |\@ifdefinable| is the
%    expansion (in the example, |\Lem|) of
%    |\csname||#1\endcsname|.  The second argument is
%    executed only if the test has been completed successfully.
%    \begin{macrocode}
   {%
%    \end{macrocode}
%    Now we define the new counter. The names of the \LaTeX{}
%    macros employed should speak for themselves:
% \changes{v2.2b}{1995/10/07}{Use \cs{@newctr}}
% \changes{v2.2c}{1995/11/23}{Use \cs{@newctr} properly.}
%    \begin{macrocode}
    \@definecounter{#1}\@newctr{#1}[#3]%
%    \end{macrocode}
%    Using |\@newctr| will give a proper error message if the counter
%    in |#3| is not defined.
%    In defining `|\theLem|' we must generate the desired macro
%    name by use of |\expandafter| and |\csname|.
%    \begin{macrocode}
    \expandafter\xdef\csname the#1\endcsname
%    \end{macrocode}
%    An |\xdef| is used in order to make the definition global,
%    and to ensure that it contains the replacement texts of
%    |\@thmcountersep| and |\@thmcounter|.\footnote{These
%    two macros can be defined by the document style.  Their default
%    values produce a `\texttt{.}' as separation and an arabic
%    representation of the number.} However, not everything should be
%    expanded. For example, it saves space to use |\thesection|
%    instead of its---at times---lengthy expansion.
%    \begin{macrocode}
      {\expandafter \noexpand \csname the#3\endcsname
       \@thmcountersep \@thmcounter{#1}}%
%    \end{macrocode}
%    Thus with the defaults of \LaTeX{}, |\theLem| would be
%    replaced by the command sequence |\thesection.\arabic{Lem}|.
%
%    We will now look at the definition of the macro which is executed
%    at the beginning of the actual environment (in our example this
%    macro is |\Lem|). It should be noted that we use an
%    ``|\expandafter| trick'' to expand only certain parts of the
%    replacement text at the time of the definition.
%    \begin{macrocode}
    \def\@tempa{\global\@namedef{#1}}%
    \expandafter \@tempa \expandafter{%
%    \end{macrocode}
%    First, the macro that contains the current definitions of
%    |\@begintheorem| and |\@opargtheorem| should be called
%    up.  The name of this macro---as is already known---has the form
%    |\th@|\meta{theorem\ style}; therefore, it must be called by
%    \begin{macrocode}
      \csname th@\the \theorem@style
%    \end{macrocode}
%    In addition the default theorem font should be changeable, i.e.\
%    we have to insert the contents of |\theorem@bodyfont|. For
%    that reason, we expand even further, beyond |\endcsname|,
%    and thus insert the contents of the \textsf{token} register
%    |\theorem@bodyfont| in the replacement text.
%    \begin{macrocode}
            \expandafter \endcsname \the \theorem@bodyfont
%    \end{macrocode}
%    Now it is time to call the macro |\@thm| which takes over
%    the further processing.  It has two arguments: the current
%    counter name (in our example, \texttt{Lem}), and the text of the
%    label (in our example, \texttt{Lemma}).
%    \begin{macrocode}
     \@thm{#1}{#2}}%
%    \end{macrocode}
%    With this, the `sub-definition' is complete.  The macro
%    |\@endtheorem| ends a theorem environment and is, so far, nothing
%    but an |\endtrivlist|.  (Hence it is defined globally, and not
%    within the theorem styles.\footnote{This has to be changed as
%    soon as theorem styles that change \texttt{\bslash @endtheorem}
%    exist.  In such a case, all existing styles must be changed as
%    well since they will have to reset the macro.}) Therefore, we can
%    set it equivalent to the macro that ends the theorem set (in our
%    example, |\endLem|).  However, if some day theorem styles exist
%    that do change |\@endtheorem|, we would have to use the
%    commented-out line instead.
%    \begin{macrocode}
    \global \expandafter \let \csname end#1\endcsname \@endtheorem
%   \global\@namedef{end#1}{\@endtheorem}%
%    \end{macrocode}
%    With these commands all the required definitions are employed,
%    unless the test |\@ifdefinable| has failed.  Therefore, we
%    end the second argument of this macro and with it the definition
%    of |\@xnthm|.
%    \begin{macrocode}
   }}
%    \end{macrocode}
% \end{macro}
%
% \begin{macro}{\@ynthm}
%    The definition of |\@ynthm| is completely analogous. In
%    this case the new counter that is defined is not reset within
%    another counter; thus the definition of |\the...|
%    is simplified:
%    \begin{macrocode}
\gdef\@ynthm#1#2{\expandafter\@ifdefinable\csname #1\endcsname
   {\@definecounter{#1}%
    \expandafter\xdef\csname the#1\endcsname{\@thmcounter{#1}}%
%    \end{macrocode}
%    The rest of the definition corresponds literally to that of
%    |\@xnthm|:
%    \begin{macrocode}
    \def\@tempa{\global\@namedef{#1}}\expandafter \@tempa
     \expandafter{\csname th@\the \theorem@style \expandafter
     \endcsname \the\theorem@bodyfont \@thm{#1}{#2}}%
    \global \expandafter \let \csname end#1\endcsname \@endtheorem}}
%    \end{macrocode}
% \end{macro}
%
% \begin{macro}{\@othm}
%    The definition of |\@othm| does not contain anything new.
%    \begin{macrocode}
\gdef\@othm#1[#2]#3{%
%    \end{macrocode}
% \changes{v2.2b}{95/10/07}{Build in check for counter defined}
%    We do not define a new counter but instead use one that has
%    already been defined.  Thus the only definition we need is that
%    of this pseudo-counter (i.e.\ |\the|\meta{env.\ name}).
%    First we check if |#2| corresponds to a known counter name.
%    \begin{macrocode}
  \expandafter\ifx\csname c@#2\endcsname\relax
   \@nocounterr{#2}%
  \else
   \expandafter\@ifdefinable\csname #1\endcsname
   {\expandafter \xdef \csname the#1\endcsname
     {\expandafter \noexpand \csname the#2\endcsname}%
%    \end{macrocode}
%    All other parts of the definition can be adopted from
%    |\@xnthm|. We have to remember, though, that in this case
%    the name of the current counter and the theorem label have moved
%    to the second and third arguments.
%    \begin{macrocode}
    \def\@tempa{\global\@namedef{#1}}\expandafter \@tempa
     \expandafter{\csname th@\the \theorem@style \expandafter
     \endcsname \the\theorem@bodyfont \@thm{#2}{#3}}%
    \global \expandafter \let \csname end#1\endcsname \@endtheorem}%
  \fi}
%    \end{macrocode}
% \end{macro}
%
% \subsection{Macros that are employed in a {\envfont theorem}
%    environment}
%
% \begin{macro}{\@thm}
%    The macro |\@thm| has to increase the current counter.  Then,
%    depending on whether the environment has (or does not have) an
%    optional argument, it has to branch into either |\@begintheorem|
%    or |\@opargtheorem|.
%    \begin{macrocode}
\gdef\@thm#1#2{\refstepcounter{#1}%
%    \end{macrocode}
%    Now we start a {\envfont trivlist} environment, and give
%    |\@topsep| and |\@topsepadd| the values of the \textsf{skip}
%    registers |\theorempreskipamount| and |\theorempostskipamount|.
%    The value in |\@topsep| is the vertical space that is inserted by
%    the first (and only) |\item| in our |\trivlist| whilst
%    |\@topsepadd| is inserted by |\@endparenv| at the end of that
%    {\envfont trivlist} environment.  By using these registers, we
%    obtain the desired space around a {\envfont theorem} environment.
%    \begin{macrocode}
   \trivlist
   \@topsep \theorempreskipamount               % used by first \item
   \@topsepadd \theorempostskipamount           % used by \@endparenv
%    \end{macrocode}
%    Now we have to test whether an optional argument has been given.
% \changes{v2.0d}{89/07/18}{Forgotten percent added.}
%    \begin{macrocode}
   \@ifnextchar [%
%    \end{macrocode}
%    If there is an optional argument, we will call |\@ythm|, and
%    move the arguments read back into the input stream.
%    \begin{macrocode}
   {\@ythm{#1}{#2}}%
%    \end{macrocode}
%    If not, we call |\@begintheorem|. Its first argument is the
%    name of the theorem set (hence the second argument of
%    |\@thm|). Its second argument is the macro that produces the
%    current number.
%    \begin{macrocode}
   {\@begintheorem{#2}{\csname the#1\endcsname}\ignorespaces}}
%    \end{macrocode}
% \end{macro}
%
% \begin{macro}{\@xthm}
% \begin{macro}{\@ythm}
%    Both these macros were originally called by |\@thm|.  We do
%    not need |\@xthm| anymore, hence we reset it to
%    |\relax|. The definition of |\@ythm| has not changed at
%    all from its definition in \LaTeX. In order to make the macros
%    easier to understand, we will
%    nevertheless present it (commented out).
%    \begin{macrocode}
\global\let\@xthm\relax
% \def\@ythm#1#2[#3]{\@opargbegintheorem{#2}{\csname
%           the#1\endcsname}{#3}\ignorespaces}
%</package>
%    \end{macrocode}
%    The primitive |\ignorespaces| in |\@ythm| and
%    |\@thm| is needed to remove the spaces between the
%    |\begin{...}| and the actual text.
% \end{macro}
% \end{macro}
%
%
% \subsection{Definition of the theorem substyles}
%
% As already pointed out, the theorem substyles, defined below, are
% only loaded when necessary.  Note that all these substyles, except
% \texttt{plain}, have |\slshape| as the default body font.
%
% \subsubsection{The \texttt{plain} style}
%
% \label{dok:th@plain} As the following macros use |@|, we have
% to locally set the |\catcode| of this symbol to ``letter''.
% This happens within a group, so that we do not have to worry about
% which |\catcode| that symbol had before.
%    \begin{macrocode}
%<*thp>
\begingroup \makeatletter
%    \end{macrocode}
% Since we are now within a group, we must make all definitions
% globally.  First we make sure that \texttt{theorem.sty} is loaded.
% This will allow us to use this file as a document style option
% without having to call \texttt{theorem} itself as an option.  At the
% same time, we assure that at least version 2 is loaded, since
% |\theorem@style| was not defined in earlier versions.
%    \begin{macrocode}
\@ifundefined{theorem@style}{\input{theorem.sty}}{}
\ProvidesFile{thp.sty}
             [\FMithmInfo]
%    \end{macrocode}
%
%
% \begin{macro}{\th@plain}
% \begin{macro}{\@begintheorem}
% \begin{macro}{\@opargbegintheorem}
%    |\theoremstyle{plain}| corresponds to the original
%    definition, except that the distances to the surrounding text are
%    determined by the parameters |\theorempreskipamount| and
%    |\theorempostskipamount|.
%    First we set the default body font.
%    \begin{macrocode}
\gdef\th@plain{\normalfont\itshape
%    \end{macrocode}
%    Then we define |\@begintheorem| and
%    |\@opargbegintheorem|. These two macros define how the
%    header of a theorem is typeset.  |\@opargbegintheorem| will
%    be called if a \textsf{theorem} environment with an optional
%    argument is encountered; otherwise, the header is constructed by
%    calling |\@begintheorem|.  If one of these macros is
%    executed, we are within a \textsf{trivlist} environment started by
%    |\@thm|.  So the theorem header is produced with an
%    |\item| command.
%
%    Instead of specifying the header font directly, all standard
%    theorem styles use the |\theorem@headerfont| macro to allow
%    customization.  The extra space (|\labelsep|) is necessary
%    because of problems in the \textsf{trivlist} environment.
%    \begin{macrocode}
  \def\@begintheorem##1##2{%
        \item[\hskip\labelsep \theorem@headerfont ##1\ ##2]}%
%    \end{macrocode}
%    The definition of |\@opargbegintheorem| is completely
%    analogous. The only difference is the fact that there exists a
%    third argument (which is the optional parameter of the
%    environment and contains additional information about the
%    theorem).  Customarily we enclose it in parentheses.
%    \begin{macrocode}
\def\@opargbegintheorem##1##2##3{%
   \item[\hskip\labelsep \theorem@headerfont ##1\ ##2\ (##3)]}}
%    \end{macrocode}
% \end{macro}
% \end{macro}
% \end{macro}
%
% \noindent
% We conclude with an |\endgroup| to restore the
% |\catcode| of |@|.
%    \begin{macrocode}
\endgroup
%</thp>
%    \end{macrocode}
%
%
% \subsubsection{The \texttt{break} style}
%
% This style option is stored in the file |thb.sty|.
% For the next two lines see the documentation for |\th@plain| on
% page~\pageref{dok:th@plain}.
%    \begin{macrocode}
%<*thb>
\begingroup \makeatletter
\@ifundefined{theorem@style}{\input{theorem.sty}}{}
\ProvidesFile{thb.sty}
             [\FMithmInfo]
%    \end{macrocode}
%
% \begin{macro}{\th@break}
%    |\theoremstyle{break}| produces a line break after the name
%    of the theorem.  The font is |\slshape|.  Hence, we define
%    |\th@break| as follows:
%    \begin{macrocode}
\gdef\th@break{\normalfont\slshape
  \def\@begintheorem##1##2{\item[%
%    \end{macrocode}
%    We run into the following problem: it is not possible to create
%    the header with |\item[|\meta{title}|]| and then start
%    a new line by, for example, |\mbox{}\\|. Such a definition
%    will fail whenever a list environment follows immediately.  With
%    the above construction, the |\mbox{}| causes the switch
%    |@inlabel| (cf.\ definition of |\list| and
%    |\trivlist| in \cite{latex:source}) to be set to \texttt{false}
%    and so the following list will insert additional vertical
%    space (|\topskip|). This is quite annoying. Therefore, we
%    create the line break within the |\item|. In order to ensure
%    that the text will begin at the proper position in the following
%    line, we simply pretend that the label does not take any
%    room.\footnote{This will lead to problems whenever very high
%    symbols occurring in the line tower into the
%    heading. So, something else has to be done here sometime.}
%    \begin{macrocode}
       \rlap{\vbox{\hbox{\hskip \labelsep\theorem@headerfont ##1\ ##2}%
                   \hbox{\strut}}}]}%
%    \end{macrocode}
%    Again, the definition of |\@opargbegintheorem| is
%    completely analogous.
%    \begin{macrocode}
\def\@opargbegintheorem##1##2##3{%
  \item[\rlap{\vbox{\hbox{\hskip \labelsep \theorem@headerfont
                     ##1\ ##2\ (##3)}%
                    \hbox{\strut}}}]}}
\endgroup
%</thb>
%    \end{macrocode}
% \end{macro}
%
%
% \subsubsection{The \texttt{changebreak} style}
%
%    \begin{macrocode}
% This style option is stored in the file |thcb.sty|.
%    \begin{macrocode}
%<*thcb>
\begingroup \makeatletter
\@ifundefined{theorem@style}{\input{theorem.sty}}{}
\ProvidesFile{thcb.sty}
             [\FMithmInfo]
%    \end{macrocode}
%
% \begin{macro}{\th@changebreak}
%    The \texttt{change-break} theorem style is like \texttt{break}
%    but with interchange of theorem name and theorem number.  Thus we
%    define |\th@changebreak| as follows:
%    \begin{macrocode}
\gdef\th@changebreak{\normalfont\slshape
  \def\@begintheorem##1##2{\item
       [\rlap{\vbox{\hbox{\hskip\labelsep \theorem@headerfont ##2\ ##1}%
                    \hbox{\strut}}}]}%
\def\@opargbegintheorem##1##2##3{%
  \item[\rlap{\vbox{\hbox{\hskip\labelsep \theorem@headerfont
                     ##2\ ##1\ (##3)}%
                    \hbox{\strut}}}]}}
\endgroup
%</thcb>
%    \end{macrocode}
% \end{macro}
%
%
% \subsubsection{The \texttt{change} style}
%
% This style option is stored in the file |thc.sty|.
%    \begin{macrocode}
%<*thc>
\begingroup \makeatletter
\@ifundefined{theorem@style}{\input{theorem.sty}}{}
\ProvidesFile{thc.sty}
             [\FMithmInfo]
%    \end{macrocode}
%
% \begin{macro}{\th@change}
%    The \texttt{change} theorem style corresponds to the
%    \texttt{change break} style without a linebreak after the header.
%    To say it in another way, it's the same as the \texttt{plain}
%    style but with number and name interchanged and |\slshape| as the
%    default font.
%    \begin{macrocode}
\gdef\th@change{\normalfont\slshape
  \def\@begintheorem##1##2{\item
       [\hskip\labelsep \theorem@headerfont ##2\ ##1]}%
\def\@opargbegintheorem##1##2##3{%
  \item[\hskip\labelsep \theorem@headerfont ##2\ ##1\ (##3)]}}
\endgroup
%</thc>
%    \end{macrocode}
% \end{macro}
%
%
% \subsubsection{The \texttt{marginbreak} style}
%
% This style option is the one used most often at Mainz.  It is saved
% in the file |thmb.sty|.
%    \begin{macrocode}
%<*thmb>
\begingroup \makeatletter
\@ifundefined{theorem@style}{\input{theorem.sty}}{}
\ProvidesFile{thmb.sty}
             [\FMithmInfo]
%    \end{macrocode}
%
% \begin{macro}{\th@marginbreak}
%    The \texttt{margin break} style is nearly the same as the
%    \texttt{change break} style. The only difference is the placement
%    of the theorem number.  We use |\llap| to place it in the left
%    margin.
%
%    In this style |\labelsep| denotes the separation between the
%    number and the text.
%    \begin{macrocode}
\gdef\th@marginbreak{\normalfont\slshape
  \def\@begintheorem##1##2{\item
       [\rlap{\vbox{\theorem@headerfont
                    \hbox{\llap{##2}\hskip\labelsep ##1}%
                    \hbox{\strut}}}]}%
\def\@opargbegintheorem##1##2##3{%
  \item[\rlap{\vbox{\theorem@headerfont
                    \hbox{\llap{##2}\hskip\labelsep ##1\ (##3)}%
                    \hbox{\strut}}}]}}
\endgroup
%</thmb>
%    \end{macrocode}
% \changes{v2.1c}{90/02/17}{`endgroup outside macrocode env.}
% \end{macro}
%
%
%
% \subsubsection{The \texttt{margin} style}
%
% This style option is stored in the file |thm.sty|.
%    \begin{macrocode}
%<*thm>
\begingroup \makeatletter
\@ifundefined{theorem@style}{\input{theorem.sty}}{}
\ProvidesFile{thm.sty}
             [\FMithmInfo]
%    \end{macrocode}
%
% \begin{macro}{\th@margin}
%    Again this is only a variant of the theorem styles described
%    above without any new ideas.
%    \begin{macrocode}
\gdef\th@margin{\normalfont\slshape
  \def\@begintheorem##1##2{\item
       [\theorem@headerfont \llap{##2}\hskip\labelsep ##1]}%
\def\@opargbegintheorem##1##2##3{%
  \item[\theorem@headerfont \llap{##2}\hskip\labelsep ##1\ (##3)]}}
\endgroup
%</thm>
%    \end{macrocode}
% \end{macro}
%
%
%
%
% \subsection{Final Definitions}
%
% \begin{macro}{\theorempreskipamount}
% \begin{macro}{\theorempostskipamount}
%    The \textsf{skip} parameters that regulate the vertical empty space
%    before and after the {\envfont theorem} environment have
%    to be allocated as well.
% \changes{v2.0f}{89/07/20}{Unnecessary `global' removed.}
%    \begin{macrocode}
%<*package>
\newskip\theorempreskipamount
\newskip\theorempostskipamount
%    \end{macrocode}
%    Since we have used the same values for all theorem sets, we now can
%    assign them.
%    \begin{macrocode}
\global\setlength\theorempreskipamount{12pt plus 5pt minus 3pt}
\global\setlength\theorempostskipamount{8pt plus 3pt minus 1.5pt}
%    \end{macrocode}
% \end{macro}
% \end{macro}
%
%
% \begin{macro}{\@endtheorem}
%    The same holds for the macro |\@endtheorem|, which ends a
%    {\envfont theorem} environment. Since it is the same for all
%    theorem sets, it is removed from the macros |\th@|\meta{style}.
%    It simply ends the {\envfont trivlist} environment, which was
%    begun in |\@thm|.
%    \begin{macrocode}
\global\let\@endtheorem=\endtrivlist
%    \end{macrocode}
% \end{macro}
%
%
%
% \begin{macro}{\@preamblecmds}
%    All macros defined above are to be used only in the preamble.
%    Therefore, we insert them in |\@preamblecmds| which will disable
%    them at begin document. This is done by the internal \LaTeXe{}
%    command |\@onlypreamble|.
% \changes{v2.0g}{89/09/19}{`theoremheaderfont and `theorembodyfont
%    added.}
% \changes{v2.2a}{94/02/03}{Use \LaTeXe{} interface}
%    \begin{macrocode}
\@onlypreamble\@xnthm
\@onlypreamble\@ynthm
\@onlypreamble\@othm
\@onlypreamble\newtheorem
\@onlypreamble\theoremstyle
\@onlypreamble\theorembodyfont
\@onlypreamble\theoremheaderfont
%    \end{macrocode}
% \end{macro}
%
% Finally we declare the \texttt{plain} theorem style to be the default.
% \changes{v2.1b}{90/06/04}{Fixed missing `item by declaring default
%    style.}
%    \begin{macrocode}
\theoremstyle{plain}
%</package>
%    \end{macrocode}
%
% \Finale

\endinput
}{}
\ProvidesFile{thm.sty}
             [\FMithmInfo]
%    \end{macrocode}
%
% \begin{macro}{\th@margin}
%    Again this is only a variant of the theorem styles described
%    above without any new ideas.
%    \begin{macrocode}
\gdef\th@margin{\normalfont\slshape
  \def\@begintheorem##1##2{\item
       [\theorem@headerfont \llap{##2}\hskip\labelsep ##1]}%
\def\@opargbegintheorem##1##2##3{%
  \item[\theorem@headerfont \llap{##2}\hskip\labelsep ##1\ (##3)]}}
\endgroup
%</thm>
%    \end{macrocode}
% \end{macro}
%
%
%
%
% \subsection{Final Definitions}
%
% \begin{macro}{\theorempreskipamount}
% \begin{macro}{\theorempostskipamount}
%    The \textsf{skip} parameters that regulate the vertical empty space
%    before and after the {\envfont theorem} environment have
%    to be allocated as well.
% \changes{v2.0f}{89/07/20}{Unnecessary `global' removed.}
%    \begin{macrocode}
%<*package>
\newskip\theorempreskipamount
\newskip\theorempostskipamount
%    \end{macrocode}
%    Since we have used the same values for all theorem sets, we now can
%    assign them.
%    \begin{macrocode}
\global\setlength\theorempreskipamount{12pt plus 5pt minus 3pt}
\global\setlength\theorempostskipamount{8pt plus 3pt minus 1.5pt}
%    \end{macrocode}
% \end{macro}
% \end{macro}
%
%
% \begin{macro}{\@endtheorem}
%    The same holds for the macro |\@endtheorem|, which ends a
%    {\envfont theorem} environment. Since it is the same for all
%    theorem sets, it is removed from the macros |\th@|\meta{style}.
%    It simply ends the {\envfont trivlist} environment, which was
%    begun in |\@thm|.
%    \begin{macrocode}
\global\let\@endtheorem=\endtrivlist
%    \end{macrocode}
% \end{macro}
%
%
%
% \begin{macro}{\@preamblecmds}
%    All macros defined above are to be used only in the preamble.
%    Therefore, we insert them in |\@preamblecmds| which will disable
%    them at begin document. This is done by the internal \LaTeXe{}
%    command |\@onlypreamble|.
% \changes{v2.0g}{89/09/19}{`theoremheaderfont and `theorembodyfont
%    added.}
% \changes{v2.2a}{94/02/03}{Use \LaTeXe{} interface}
%    \begin{macrocode}
\@onlypreamble\@xnthm
\@onlypreamble\@ynthm
\@onlypreamble\@othm
\@onlypreamble\newtheorem
\@onlypreamble\theoremstyle
\@onlypreamble\theorembodyfont
\@onlypreamble\theoremheaderfont
%    \end{macrocode}
% \end{macro}
%
% Finally we declare the \texttt{plain} theorem style to be the default.
% \changes{v2.1b}{90/06/04}{Fixed missing `item by declaring default
%    style.}
%    \begin{macrocode}
\theoremstyle{plain}
%</package>
%    \end{macrocode}
%
% \Finale

\endinput
}{}
\ProvidesFile{thmb.sty}
             [\FMithmInfo]
%    \end{macrocode}
%
% \begin{macro}{\th@marginbreak}
%    The \texttt{margin break} style is nearly the same as the
%    \texttt{change break} style. The only difference is the placement
%    of the theorem number.  We use |\llap| to place it in the left
%    margin.
%
%    In this style |\labelsep| denotes the separation between the
%    number and the text.
%    \begin{macrocode}
\gdef\th@marginbreak{\normalfont\slshape
  \def\@begintheorem##1##2{\item
       [\rlap{\vbox{\theorem@headerfont
                    \hbox{\llap{##2}\hskip\labelsep ##1}%
                    \hbox{\strut}}}]}%
\def\@opargbegintheorem##1##2##3{%
  \item[\rlap{\vbox{\theorem@headerfont
                    \hbox{\llap{##2}\hskip\labelsep ##1\ (##3)}%
                    \hbox{\strut}}}]}}
\endgroup
%</thmb>
%    \end{macrocode}
% \changes{v2.1c}{90/02/17}{`endgroup outside macrocode env.}
% \end{macro}
%
%
%
% \subsubsection{The \texttt{margin} style}
%
% This style option is stored in the file |thm.sty|.
%    \begin{macrocode}
%<*thm>
\begingroup \makeatletter
\@ifundefined{theorem@style}{% \iffalse meta-comment
%
% Copyright (C) 1993-2024
%
% The LaTeX Project and any individual authors listed elsewhere
% in this file.
%
% This file is part of the Standard LaTeX `Tools Bundle'.
% -------------------------------------------------------
%
% It may be distributed and/or modified under the
% conditions of the LaTeX Project Public License, either version 1.3c
% of this license or (at your option) any later version.
% The latest version of this license is in
%    https://www.latex-project.org/lppl.txt
% and version 1.3c or later is part of all distributions of LaTeX
% version 2005/12/01 or later.
%
% The list of all files belonging to the LaTeX `Tools Bundle' is
% given in the file `manifest.txt'.
%
% \fi
%
%%
%
%
\def\FMithmInfo{2023/07/05 v2.2c Theorem extension package (FMi)}
%
% \ProvidesFile{theorem.dtx}[\FMithmInfo]
%\iffalse   % this is a METACOMMENT !
%%
%% Package `theorem' to use with LaTeX2e
%% Copyright (C) 1989-1995 Frank Mittelbach, all rights reserved.
%<+package>\@ifundefined{theorem@style}{}{\endinput}
%<+package>\NeedsTeXFormat{LaTeX2e}[1995/06/01]
%<+package>\ProvidesPackage{theorem}
%<+package>         [\FMithmInfo]
%
% \fi
%
% \changes{v1.0}{87/05/07}{First draft.}
% \changes{v1.0a}{87/05/07}{`thistheoremstyle' added.}
% \changes{v1.1a}{87/07/29}{New concept: redefine the internal
%                          \protect\LaTeX{} commands `@xnthm' etc.}
% \changes{v1.1b}{87/07/31}{`theoremfont' used to specify font.}
% \changes{v1.2a}{88/01/08}{`@endtheorem' now globally defined.}
% \changes{v1.2b}{88/01/11}{`newtheorem' only allowed in the preamble.}
% \changes{v1.3a}{88/02/14}{`theoremfont' now a toks register.}
%
% \changes{v2.0a}{88/11/16}{New implementation and new documentation.}
% \changes{v2.0a}{88/11/16}{No layout changes but internal modifications
%                           old substyles won't work.}
% \changes{v2.0b}{89/01/30}{`global' used for all definitions.}
% \changes{v2.0c}{89/06/15}{Documentation translated to English.
%                            Thanks to LLL and Nina}
% \changes{v2.1a}{90/02/15}{Support for new font selection.}
% \changes{v2.2a}{94/02/03}{Cleanup for LaTeX2e}
%
% \DoNotIndex{\@endparenv,\@ifdefinable,\@ifnextchar,\@ifundefined}
% \DoNotIndex{\@namedef,\@nameuse,\@spaces,\@tempa,\@warning}
% \DoNotIndex{\ }
% \DoNotIndex{\begingroup,\bfseries}
% \DoNotIndex{\csname}
% \DoNotIndex{\def,\do,\docdate}
% \DoNotIndex{\endcsname,\endgroup,\endinput,\endtrivlist,\expandafter}
% \DoNotIndex{\filedate,\fileversion}
% \DoNotIndex{\gdef,\global}
% \DoNotIndex{\hbox,\hskip}
% \DoNotIndex{\ignorespaces,\input,\itshape,\item}
% \DoNotIndex{\labelsep,\let,\llap}
% \DoNotIndex{\makeatletter}
% \DoNotIndex{\noexpand}
% \DoNotIndex{\refstepcounter,\relax,\rlap}
% \DoNotIndex{\setlength,\slshape,\space,\string,\strut}
% \DoNotIndex{\the,\trivlist,\typeout}
% \DoNotIndex{\vbox}
% \DoNotIndex{\xdef}
%
% \def\deflabel#1{\ttfamily #1\hfill}
% \def\deflist#1{\begin{list}{}{\settowidth\labelwidth{\ttfamily #1}%
%                               \setlength\leftmargin\labelwidth
%                               \addtolength\leftmargin\labelsep
%                               \let\makelabel\deflabel}}
% \def\enddeflist{\end{list}}
% \def\envfont{\normalfont\ttfamily}
%
% \MakeShortVerb{\|}
%
% \newcommand{\TUB}{{\sl TUGboat\/}}
%
% \GetFileInfo{theorem.dtx}
% \title{An Extension of the
%        \LaTeX{} theorem environment\thanks{This file
%        has version number \fileversion, last
%        revised \filedate.}}
% \author{Frank Mittelbach}
%
% \MaintainedByLaTeXTeam{tools}
% \maketitle
%
% \begin{abstract}
%    The macros described in this paper yield an extension of the
%    \LaTeX{} theorem mechanism. It is designed to satisfy the
%    different requirements of various journals. Thus, the
%    layout of the ``theorems'' can be manipulated by determining a
%    ``style''. This article describes not only the use, but
%    also the definition, of the necessary macros.
% \end{abstract}
%
% \section*{Preface to version 2.2}
%
% For \LaTeXe{} this package did not need any fundamental changes. I
% only modified the messages generated so that theorem layout styles
% will show up with the |\listfiles| command and cleaned the section
% on the New Font Selection Scheme since this is now included in
% \LaTeX.
%
% \section*{Preface to version 2.1}
%
% This version is identical to 2.0g described in \TUB\ 10\#3 except
% for some internal defaults which are now set depending on the used
% font selection scheme.
%
% This was done to avoid unpleasant surprises if the new font selection
% scheme is in force. For further details see
% section~\ref{sec:newfonts} and \cite{bk:GMS94}.
%
% \section{Introduction}
%
% For our purposes here, ``theorems'' are labelled enunciations,
% often set off from the main text by extra space and a font change.
% Theorems, corollaries, conjectures, definitions, and remarks
% are all instances of ``theorems''.  The ``header'' of these structures
% is composed of a label (such as \textsc{Theorem} or \textsc{Remark})
% and a number which serializes an item in the sequence of items
% with the same label.
%
% Shortly after the introduction of \LaTeX{} at the Fachbereich
% Mathematik in Mainz, the desire to manipulate the layout of
% ``theorems'' arose.  In Mainz, the following two conventions came
% into general use:
% \begin{enumerate}
%   \item The number of the theorem is shown in the margin.
%   \item There is a line break at
%         the end of the theorem header.
% \end{enumerate}
%
% Additionally, some journals require different formats which depend on
% the ``sort of theorem'': e.g.\ often remarks and definitions are set
% in |\upshape|, while |\itshape| is employed for main theorems.
%
% Confronted with these requirements, a theorem environment was
% developed in Mainz which allows separate determination of the
% layout of the ``theorems sets'', comparable to |\pagestyle|.
%
%
% \section{The user interface}
% \subsection{Defining new theorem sets}
%
% \DescribeMacro\newtheorem
% As in the original \LaTeX{} version, the command |\newtheorem|
% defines a new ``theorem set'' or ``theorem-like structure''.
% Two required arguments name the new environment and give the text
% to be typeset with each instance of the new ``set'', while
% an optional argument determines how
% the ``set'' is enumerated:
% \begin{description}
%    \item[\ttfamily \bslash newtheorem\{foo\}\{bar\}]
%       The theorem set {\envfont foo} (whose name is \texttt{bar})
%       uses its own counter.
%    \item[\ttfamily \bslash newtheorem\{foo2\}{[foo]}\{bar2\}]
%       The theorem set {\envfont foo2} (printed name \texttt{bar2})
%       uses the same counter as the theorem set \texttt{foo}.
%    \item[\ttfamily \bslash newtheorem\{foo3\}\{bar3\}{[section]}]
%       The theorem set {\envfont foo3} (printed name \texttt{bar3}) is
%       enumerated within the counter \texttt{section}, i.e.\ with every
%       new |\section| the enumeration begins again with 1, and
%       the enumeration is composed from the section-number and the
%       theorem counter itself.
% \end{description}
%
% \DescribeMacro\theoremstyle
% Additionally, the command |\theoremstyle| can define the layout
% of various, or all, theorem sets.  It should be noted that any
% theorem set defined by |\newtheorem| is typeset in the
% |\theoremstyle| that is current at the time of the definition.
% Thus, the following
% \begin{verbatim}
%    \theoremstyle{break}        \newtheorem{Cor}{Corollary}
%    \theoremstyle{plain}        \newtheorem{Exa}{Example}[section]
%\end{verbatim}                             ^^A No blank at line start
% leads to the result that the set {\envfont Cor} is formatted in the
% style \texttt{break}, while the set {\envfont Exa} and all the
% following ones are formatted in the style \texttt{plain}, unless
% another |\theoremstyle| follows.  Since the definitions installed by
% |\newtheorem| are global, one also can limit |\theoremstyle| locally
% by grouping braces.
%
% \DescribeMacro\theorembodyfont
% The choice of the font for the theorem body is completely
% independent of the chosen |\theoremstyle|; this has proven to
% be very advantageous.  For example,
% \begin{verbatim}
%   {\theorembodyfont{\upshape}          \newtheorem{Rem}{Remark}}
%\end{verbatim}
% defines a theorem set {\envfont Rem}, which will be set in
% |\upshape| in the current layout (which in our example is
% \texttt{plain}).  As with |\theoremstyle|, the |\theorembodyfont|
% chosen is that current at the time of |\newtheorem|.  If
% |\theorembodyfont| is not specified or one defines
% |\theorembodyfont{}|, then the font used will be that defined by the
% |\theoremstyle|.
%
% \DescribeMacro\theoremheaderfont It is also possible to customize
% the font used for the theorem headers.  This is, however, a global
% declaration, and therefore there should be at most one
% |\theoremheaderfont| declaration in the preamble.\footnote{If it is
% actually necessary to have different header fonts, one has to define
% new theorem styles (substituting the desired font) or specify the
% information directly in the \texttt{\bslash newtheorem} declaration
% (the unclean variant).}
%
% \DescribeMacro\theorempreskipamount
% \DescribeMacro\theorempostskipamount Two additional parameters
% affect the vertical space around the theorem environments:
% |\theorempreskipamount| and |\theorempostskipamount| define,
% respectively, the spacing before and after such an environment.
% These parameters apply for all theorem sets and can be manipulated
% with the ordinary length macros.  They are rubber lengths
% (`\textsf{skips}'), and therefore can contain \texttt{plus} and
% \texttt{minus} parts.
%
% Since the definition of theorem sets should---most sensibly---be
% placed in the preamble, we only allow installation there.
% It is therefore possible to release the memory used here after
% |\begin{document}|, in order to make room for other
% applications.
%
%
% \subsection{Existing theorem styles}
%
% The following theorem styles exist to date:
% \begin{deflist}{marginbreak}
%    \item[plain]
%       This theorem style emulates the original \LaTeX{} definition,
%       except that additionally the parameters
%       |\theorem...skipamount| are used.
%    \item[break]
%       In this style, the theorem header is followed by a line
%       break.
%    \item[marginbreak]
%       The theorem number is set in the margin, and there is a line
%       break as in \texttt{break}.
%    \item[changebreak]
%       Like \texttt{break}, but with header number and text
%       interchanged.
%    \item[change]
%       Header number and text are interchanged, without a line break.
%    \item[margin]
%       The number is set in the left margin, without a line break.
% \end{deflist}
% All styles (except \texttt{plain}) select |\slshape| as the default
% |\theorembodyfont|.
%
%
% \subsection{Examples}
%
% Given the above theorem sets {\envfont Cor}, {\envfont Exa}
% and {\envfont Rem},
% suppose that the preamble also contains the declarations:
% \begin{verbatim}
%    \theoremstyle{marginbreak}   \newtheorem{Lem}[Cor]{Lemma}
%    \theoremstyle{change}
%    \theorembodyfont{\itshape}        \newtheorem{Def}[Cor]{Definition}
%
%    \theoremheaderfont{\scshape}
%\end{verbatim}
% Then the following are some typical examples of the typeset output
% resulting from their use.
%
% \begin{Cor}
%    This is a sentence typeset in the theorem environment {\envfont
%    Cor}.
% \end{Cor}
%
% \begin{Exa}
%    This is a sentence typeset in the theorem environment {\envfont
%    Exa}.
% \end{Exa}
%
% \begin{Rem}
%    This is a sentence typeset in the theorem environment {\envfont
%    Rem}.
% \end{Rem}
%
% \begin{Lem}[Ben User]
%    This is a sentence typeset in the theorem environment {\envfont
%    Lem}.
% \end{Lem}
%
% \begin{Def}[Very impressive Definition]
%    This is a sentence typeset in the theorem environment {\envfont
%    Def}.
% \end{Def}
%
% The last two examples show the effect of the optional argument to
% a theorem environment (it is the text typeset in parentheses).
%
%
% \section{Special Considerations}
% \label{sec:newfonts}
%
% Theoremheader and body are implemented as a unit. This means that
% the |\theoremheaderfont|  will inherit characteristics of the
% |\theorembodyfont| in \LaTeXe. Thus, if for example
% |\theorembodyfont| is |\itshape| and |\theoremheaderfont|
% is |\bfseries| the font selected for the header will have the
% characteristics `bold extended italic'. If this is not desired one
% should set the |\theoremheaderfont| to something like
% \begin{verbatim}
%      \theoremheaderfont{\normalfont\bfseries}
%\end{verbatim}
% i.e.\ supplying all necessary font information explicitly.
%
%
%
% \section{Acknowledgements}
%
% The publication of this set of macros was only possible with the
% help of Christina Busse (translating the manuscript into English),
% Joachim Pense (playing the r\^ole of typist), Chris Rowley
% (looking everything over) and many others providing useful
% suggestions.
%
% \MaybeStop{%
%    \begin{thebibliography}{9}
%       \bibitem{bk:GMS94} \textsc{M.~Goossens}, \textsc{F.~Mittelbach}
%         and \textsc{A.~Samarin}.
%         \newblock The \LaTeX{} Companion.
%         \newblock
%         Addison-Wesley, Reading, Massachusetts, 1994.
%       \bibitem{latex:source} \textsc{Lamport, Leslie.} \newblock
%                               \texttt{latex.tex}, version 2.09, date
%                               Feb. 1990.
%    \end{thebibliography}
%    \PrintIndex
%    \PrintChanges}
%
%
% \section{The documentation driver file}
%
% The next bit of code contains the documentation driver file for
% \TeX{}, i.e., the file that will produce the documentation you are
% currently reading. It will be extracted from this file by the
% \texttt{docstrip} program. Since it is the first code in the file
% one can alternatively process this file directly with \LaTeXe{} to
% obtain the documentation.
%
%    \begin{macrocode}
%<*driver>
\documentclass{ltxdoc}

\usepackage{theorem}

%<+driver>% The next few lines define theorem sets which are used
%<+driver>% in the example section of the documentation.

\theoremstyle{break}       \newtheorem{Cor}{Corollary}
\theoremstyle{plain}       \newtheorem{Exa}{Example}[section]
{\theorembodyfont{\upshape}\newtheorem{Rem}{Remark}}
\theoremstyle{marginbreak} \newtheorem{Lem}[Cor]{Lemma}
\theoremstyle{change}
\theorembodyfont{\itshape} \newtheorem{Def}[Cor]{Definition}

\theoremheaderfont{\scshape}

\RecordChanges

\begin{document}
   \DocInput{theorem.dtx}
\end{document}
%</driver>
%    \end{macrocode}
%
% \section{Definition of the Macros}
%
% If the file has been loaded before, we abort immediately. If not the
% package announces itself (this is actually done at the very top of
% the file---the way it is done isn't good style so don't copy it).
% \changes{v2.0e}{89/07/19}{Spaces removed from `typeout'.}
% \changes{v2.2b}{95/11/19}{Announce moved to top of file}
%    \begin{macrocode}
%<*package>
%\@ifundefined{theorem@style}{}{\endinput}
%\def\FMithmInfo{1995/11/19 v2.2b Theorem extension package (FMi)}
%\ProvidesPackage{theorem}[\FMithmInfo]
%    \end{macrocode}
%
%
% \subsection{Definition of theorem styles and fonts}
%
% All the definitions in this file are done globally to allow
% inputting this file inside a group.
%
% \begin{macro}{\theoremstyle}
%    Before a theorem style can be installed, the chosen style must be
%    known. For that reason, we must test to see that
%    |\th@|\meta{style} is known or, more precisely, that it is
%    different from |\relax|. If the style is not known then
%    |\th@plain| is used.
%    \begin{macrocode}
\gdef\theoremstyle#1{%
   \@ifundefined{th@#1}{\@warning
          {Unknown theoremstyle `#1'. Using `plain'}%
          \theorem@style{plain}}%
%    \end{macrocode}
%    We save the theorem style to be used in the token register
%    |\theorem@style|.
%    \begin{macrocode}
      {\theorem@style{#1}}%
%    \end{macrocode}
%    Now we ``evaluate'' the theorem style: this means, we call the
%    macro |\th@|\meta{style} which will activate the relevant
%    definitions which are contained in a separate file.  This is done
%    in a group to suppress changes to the current font.  This could
%    otherwise pose problems together with the new font selection
%    scheme\footnote{When I printed the original article using the new
%    font selection scheme I ended with a document with slanted
%    typefaces (text headings and all) simply because one of the
%    theorem styles used \texttt{\bslash sl} at toplevel.} if the
%    |\th@|\meta{style} is evaluated a second time.
% \changes{v2.1a}{90/02/15}{Group added around `th@..}
% \changes{v2.1a}{90/02/15}{Reseting of `..@begintheorem removed.}
%    \begin{macrocode}
      \begingroup
        \csname th@\the\theorem@style \endcsname
      \endgroup}
%    \end{macrocode}
% \end{macro}
%
%
% \begin{macro}{\@begintheorem}
% \changes{v2.1a}{90/02/15}{Relaxed at toplevel.}
% \begin{macro}{\@opargbegintheorem}
% \changes{v2.1a}{90/02/15}{Relaxed at toplevel.}
%    We reset |\@begintheorem| and |\@opargbegintheorem| to
%    |\relax| since these commands are no longer necessary at
%    toplevel. This will save a few tokens.
%    \begin{macrocode}
\global\let\@begintheorem\relax
\global\let\@opargbegintheorem\relax
%    \end{macrocode}
% \end{macro}
% \end{macro}
%
% \begin{macro}{\theorem@style}
%    Obviously the \textsf{token} register used above has to be
%    allocated. To assure the utmost compatibility with the
%    original \LaTeX{} definition, we set the default theorem style to
%    \texttt{plain}, which implements the usual \LaTeX{} convention.
% \changes{v2.0f}{89/07/20}{Unnecessary `global' removed.}
%    \begin{macrocode}
\newtoks\theorem@style
\global\theorem@style{plain}
%    \end{macrocode}
% \end{macro}
%
% \begin{macro}{\theorembodyfont}
% \begin{macro}{\theorem@bodyfont}
%    If the |\theorembodyfont| is set by the user then it should not
%    interact with the default font set in the theorem style. When the
%    new font selection is in force this may happen if, for example,
%    the default is |\itshape| and the new |\theorembodyfont| is
%    |\sffamily|. So we add a |\reset@font| command in front of the
%    user definition.
% \changes{v2.0f}{89/07/20}{Unnecessary `global' removed.}
% \changes{v2.0g}{89/09/19}{`theoremfont to `theorembodyfont.}
% \changes{v2.1c}{92/06/22}{`reset@font added and logic changed.}
% \changes{v2.1d}{92/10/28}{Reset to empty if arg is empty}
%    \begin{macrocode}
\gdef\theorembodyfont#1{%
%    \end{macrocode}
%    We check if the argument supplied is empty and if so put nothing
%    into the |\theorem@bodyfont| token register to allow for
%    |\theorembodyfont{}| as a mean of using the default of the
%    current |\theoremstyle|.
%    \begin{macrocode}
   \def\@tempa{#1}%
   \ifx\@tempa\@empty
    \theorem@bodyfont{}%
   \else
    \theorem@bodyfont{\reset@font#1}%
   \fi
   }
\newtoks\theorem@bodyfont
\global\theorem@bodyfont{}
%    \end{macrocode}
% \end{macro}
% \end{macro}
%
%
% \begin{macro}{\theoremheaderfont}
% \changes{v2.0g}{89/09/19}{Macro added.}
%    The font for the theorem headers is handled differently because
%    this definition applies to all theorem styles.
%    \begin{macrocode}
\gdef\theoremheaderfont#1{\gdef\theorem@headerfont{#1}%
%    \end{macrocode}
%    After using the macro once it is redefined to produce an error
%    message.
%    \begin{macrocode}
       \gdef\theoremheaderfont##1{%
        \typeout{\string\theoremheaderfont\space should be used
                 only once.}}}
%    \end{macrocode}
% \end{macro}
%
% \begin{macro}{\theorem@headerfont}
% \changes{v2.0g}{89/09/19}{Macro added.}
%    To set the |\theorem@headerfont| default we first test if
%    the new fontselection scheme is in force.
% \changes{v2.1a}{90/02/15}{Support for new font selection.}
%    \begin{macrocode}
\ifx\upshape\undefined
%    \end{macrocode}
%    If not we define it to expand into |\bfseries|. We don't use
%    |\let| just in case a following style option redefines this
%    macro.
%    \begin{macrocode}
\gdef\theorem@headerfont{\bfseries}
%    \end{macrocode}
%    Otherwise we reset the current shape before calling |\bfseries|.
%    \begin{macrocode}
\else \gdef\theorem@headerfont{\normalfont\bfseries}\fi
%    \end{macrocode}
% \end{macro}
%
% \begin{macro}{\th@plain}
% \begin{macro}{\th@break}
% \begin{macro}{\th@marginbreak}
% \begin{macro}{\th@changebreak}
% \begin{macro}{\th@change}
% \begin{macro}{\th@margin}
%    The different styles are defined in macros such as
%    |\th@plain|.  Since memory space is precious in
%    ``non-Big-versions'', we have to avoid offering too many unused
%    definitions.  Therefore we define these styles in separate files
%    that can be loaded on demand.  Thus the commands themselves only
%    load these files. We use |\@input@|, a \LaTeXe{} internal command
%    that ensures that the file will be listed with |\listfiles|.
% \changes{v2.2a}{94/02/02}{Use `@input@ to load theorem layout files}
%    \begin{macrocode}
\gdef\th@plain{\@input@{thp.sty}}
\gdef\th@break{\@input@{thb.sty}}
\gdef\th@marginbreak{\@input@{thmb.sty}}
\gdef\th@changebreak{\@input@{thcb.sty}}
\gdef\th@change{\@input@{thc.sty}}
\gdef\th@margin{\@input@{thm.sty}}
%    \end{macrocode}
%    This list will be expanded when new styles become
%    available.  For testing,  just append new theorem substyles
%    as document options.
% \end{macro}
% \end{macro}
% \end{macro}
% \end{macro}
% \end{macro}
% \end{macro}
%
%
%
% \subsection{Definition of a new theorem set}
%
% As already pointed out, a new theorem environment can be defined in
% three different ways:
% \begin{verbatim}
%    \newtheorem{Lem}{Lemma}
%    \newtheorem{Lem}{Lemma}[section]
%    \newtheorem{Lem}[Theorem]{Lemma}
%\end{verbatim}
% The function of the macro |\newtheorem| is to recognize these cases
% and then to branch into one of the three macros |\@ynthm|, |\@xnthm|
% or |\@othm|.  This mechanism is adopted unchanged from
% \cite{latex:source}; the essential point here is that, for example,
% in the second case, the arguments \texttt{Lem}, \texttt{Lemma} and
% \texttt{section} are passed over to the macro |\@xnthm|.
%
% We inspect this case first because the others present fewer problems,
% and thus are easily derived from this one.
%
% \begin{macro}{\@xnthm}
%    For our example arguments, the macro |\@xnthm| must fulfill
%    the following:
%    \begin{itemize}
%       \item
%          Define a new \LaTeX-counter `Lem'
%       \item
%          reset this counter within a |\section|
%       \item
%          define the macro |\theLem|
%       \item
%          define the environment macros |\Lem| and
%          |\endLem| using the current
%          |\theoremstyle| and |\theorem@bodyfont|.
%    \end{itemize}
%    Obviously, all this should happen only if the first argument
%    of |\@xnthm| (i.e.\ \texttt{Lem} in our example) is chosen so
%    as not to conflict with any previously defined commands or
%    environments.  This test is performed by the \LaTeX{} macro
%    |\@ifdefinable|.
%    \begin{macrocode}
\gdef\@xnthm#1#2[#3]{\expandafter\@ifdefinable\csname #1\endcsname
%    \end{macrocode}
%    Therefore, the first argument of |\@ifdefinable| is the
%    expansion (in the example, |\Lem|) of
%    |\csname||#1\endcsname|.  The second argument is
%    executed only if the test has been completed successfully.
%    \begin{macrocode}
   {%
%    \end{macrocode}
%    Now we define the new counter. The names of the \LaTeX{}
%    macros employed should speak for themselves:
% \changes{v2.2b}{1995/10/07}{Use \cs{@newctr}}
% \changes{v2.2c}{1995/11/23}{Use \cs{@newctr} properly.}
%    \begin{macrocode}
    \@definecounter{#1}\@newctr{#1}[#3]%
%    \end{macrocode}
%    Using |\@newctr| will give a proper error message if the counter
%    in |#3| is not defined.
%    In defining `|\theLem|' we must generate the desired macro
%    name by use of |\expandafter| and |\csname|.
%    \begin{macrocode}
    \expandafter\xdef\csname the#1\endcsname
%    \end{macrocode}
%    An |\xdef| is used in order to make the definition global,
%    and to ensure that it contains the replacement texts of
%    |\@thmcountersep| and |\@thmcounter|.\footnote{These
%    two macros can be defined by the document style.  Their default
%    values produce a `\texttt{.}' as separation and an arabic
%    representation of the number.} However, not everything should be
%    expanded. For example, it saves space to use |\thesection|
%    instead of its---at times---lengthy expansion.
%    \begin{macrocode}
      {\expandafter \noexpand \csname the#3\endcsname
       \@thmcountersep \@thmcounter{#1}}%
%    \end{macrocode}
%    Thus with the defaults of \LaTeX{}, |\theLem| would be
%    replaced by the command sequence |\thesection.\arabic{Lem}|.
%
%    We will now look at the definition of the macro which is executed
%    at the beginning of the actual environment (in our example this
%    macro is |\Lem|). It should be noted that we use an
%    ``|\expandafter| trick'' to expand only certain parts of the
%    replacement text at the time of the definition.
%    \begin{macrocode}
    \def\@tempa{\global\@namedef{#1}}%
    \expandafter \@tempa \expandafter{%
%    \end{macrocode}
%    First, the macro that contains the current definitions of
%    |\@begintheorem| and |\@opargtheorem| should be called
%    up.  The name of this macro---as is already known---has the form
%    |\th@|\meta{theorem\ style}; therefore, it must be called by
%    \begin{macrocode}
      \csname th@\the \theorem@style
%    \end{macrocode}
%    In addition the default theorem font should be changeable, i.e.\
%    we have to insert the contents of |\theorem@bodyfont|. For
%    that reason, we expand even further, beyond |\endcsname|,
%    and thus insert the contents of the \textsf{token} register
%    |\theorem@bodyfont| in the replacement text.
%    \begin{macrocode}
            \expandafter \endcsname \the \theorem@bodyfont
%    \end{macrocode}
%    Now it is time to call the macro |\@thm| which takes over
%    the further processing.  It has two arguments: the current
%    counter name (in our example, \texttt{Lem}), and the text of the
%    label (in our example, \texttt{Lemma}).
%    \begin{macrocode}
     \@thm{#1}{#2}}%
%    \end{macrocode}
%    With this, the `sub-definition' is complete.  The macro
%    |\@endtheorem| ends a theorem environment and is, so far, nothing
%    but an |\endtrivlist|.  (Hence it is defined globally, and not
%    within the theorem styles.\footnote{This has to be changed as
%    soon as theorem styles that change \texttt{\bslash @endtheorem}
%    exist.  In such a case, all existing styles must be changed as
%    well since they will have to reset the macro.}) Therefore, we can
%    set it equivalent to the macro that ends the theorem set (in our
%    example, |\endLem|).  However, if some day theorem styles exist
%    that do change |\@endtheorem|, we would have to use the
%    commented-out line instead.
%    \begin{macrocode}
    \global \expandafter \let \csname end#1\endcsname \@endtheorem
%   \global\@namedef{end#1}{\@endtheorem}%
%    \end{macrocode}
%    With these commands all the required definitions are employed,
%    unless the test |\@ifdefinable| has failed.  Therefore, we
%    end the second argument of this macro and with it the definition
%    of |\@xnthm|.
%    \begin{macrocode}
   }}
%    \end{macrocode}
% \end{macro}
%
% \begin{macro}{\@ynthm}
%    The definition of |\@ynthm| is completely analogous. In
%    this case the new counter that is defined is not reset within
%    another counter; thus the definition of |\the...|
%    is simplified:
%    \begin{macrocode}
\gdef\@ynthm#1#2{\expandafter\@ifdefinable\csname #1\endcsname
   {\@definecounter{#1}%
    \expandafter\xdef\csname the#1\endcsname{\@thmcounter{#1}}%
%    \end{macrocode}
%    The rest of the definition corresponds literally to that of
%    |\@xnthm|:
%    \begin{macrocode}
    \def\@tempa{\global\@namedef{#1}}\expandafter \@tempa
     \expandafter{\csname th@\the \theorem@style \expandafter
     \endcsname \the\theorem@bodyfont \@thm{#1}{#2}}%
    \global \expandafter \let \csname end#1\endcsname \@endtheorem}}
%    \end{macrocode}
% \end{macro}
%
% \begin{macro}{\@othm}
%    The definition of |\@othm| does not contain anything new.
%    \begin{macrocode}
\gdef\@othm#1[#2]#3{%
%    \end{macrocode}
% \changes{v2.2b}{95/10/07}{Build in check for counter defined}
%    We do not define a new counter but instead use one that has
%    already been defined.  Thus the only definition we need is that
%    of this pseudo-counter (i.e.\ |\the|\meta{env.\ name}).
%    First we check if |#2| corresponds to a known counter name.
%    \begin{macrocode}
  \expandafter\ifx\csname c@#2\endcsname\relax
   \@nocounterr{#2}%
  \else
   \expandafter\@ifdefinable\csname #1\endcsname
   {\expandafter \xdef \csname the#1\endcsname
     {\expandafter \noexpand \csname the#2\endcsname}%
%    \end{macrocode}
%    All other parts of the definition can be adopted from
%    |\@xnthm|. We have to remember, though, that in this case
%    the name of the current counter and the theorem label have moved
%    to the second and third arguments.
%    \begin{macrocode}
    \def\@tempa{\global\@namedef{#1}}\expandafter \@tempa
     \expandafter{\csname th@\the \theorem@style \expandafter
     \endcsname \the\theorem@bodyfont \@thm{#2}{#3}}%
    \global \expandafter \let \csname end#1\endcsname \@endtheorem}%
  \fi}
%    \end{macrocode}
% \end{macro}
%
% \subsection{Macros that are employed in a {\envfont theorem}
%    environment}
%
% \begin{macro}{\@thm}
%    The macro |\@thm| has to increase the current counter.  Then,
%    depending on whether the environment has (or does not have) an
%    optional argument, it has to branch into either |\@begintheorem|
%    or |\@opargtheorem|.
%    \begin{macrocode}
\gdef\@thm#1#2{\refstepcounter{#1}%
%    \end{macrocode}
%    Now we start a {\envfont trivlist} environment, and give
%    |\@topsep| and |\@topsepadd| the values of the \textsf{skip}
%    registers |\theorempreskipamount| and |\theorempostskipamount|.
%    The value in |\@topsep| is the vertical space that is inserted by
%    the first (and only) |\item| in our |\trivlist| whilst
%    |\@topsepadd| is inserted by |\@endparenv| at the end of that
%    {\envfont trivlist} environment.  By using these registers, we
%    obtain the desired space around a {\envfont theorem} environment.
%    \begin{macrocode}
   \trivlist
   \@topsep \theorempreskipamount               % used by first \item
   \@topsepadd \theorempostskipamount           % used by \@endparenv
%    \end{macrocode}
%    Now we have to test whether an optional argument has been given.
% \changes{v2.0d}{89/07/18}{Forgotten percent added.}
%    \begin{macrocode}
   \@ifnextchar [%
%    \end{macrocode}
%    If there is an optional argument, we will call |\@ythm|, and
%    move the arguments read back into the input stream.
%    \begin{macrocode}
   {\@ythm{#1}{#2}}%
%    \end{macrocode}
%    If not, we call |\@begintheorem|. Its first argument is the
%    name of the theorem set (hence the second argument of
%    |\@thm|). Its second argument is the macro that produces the
%    current number.
%    \begin{macrocode}
   {\@begintheorem{#2}{\csname the#1\endcsname}\ignorespaces}}
%    \end{macrocode}
% \end{macro}
%
% \begin{macro}{\@xthm}
% \begin{macro}{\@ythm}
%    Both these macros were originally called by |\@thm|.  We do
%    not need |\@xthm| anymore, hence we reset it to
%    |\relax|. The definition of |\@ythm| has not changed at
%    all from its definition in \LaTeX. In order to make the macros
%    easier to understand, we will
%    nevertheless present it (commented out).
%    \begin{macrocode}
\global\let\@xthm\relax
% \def\@ythm#1#2[#3]{\@opargbegintheorem{#2}{\csname
%           the#1\endcsname}{#3}\ignorespaces}
%</package>
%    \end{macrocode}
%    The primitive |\ignorespaces| in |\@ythm| and
%    |\@thm| is needed to remove the spaces between the
%    |\begin{...}| and the actual text.
% \end{macro}
% \end{macro}
%
%
% \subsection{Definition of the theorem substyles}
%
% As already pointed out, the theorem substyles, defined below, are
% only loaded when necessary.  Note that all these substyles, except
% \texttt{plain}, have |\slshape| as the default body font.
%
% \subsubsection{The \texttt{plain} style}
%
% \label{dok:th@plain} As the following macros use |@|, we have
% to locally set the |\catcode| of this symbol to ``letter''.
% This happens within a group, so that we do not have to worry about
% which |\catcode| that symbol had before.
%    \begin{macrocode}
%<*thp>
\begingroup \makeatletter
%    \end{macrocode}
% Since we are now within a group, we must make all definitions
% globally.  First we make sure that \texttt{theorem.sty} is loaded.
% This will allow us to use this file as a document style option
% without having to call \texttt{theorem} itself as an option.  At the
% same time, we assure that at least version 2 is loaded, since
% |\theorem@style| was not defined in earlier versions.
%    \begin{macrocode}
\@ifundefined{theorem@style}{% \iffalse meta-comment
%
% Copyright (C) 1993-2024
%
% The LaTeX Project and any individual authors listed elsewhere
% in this file.
%
% This file is part of the Standard LaTeX `Tools Bundle'.
% -------------------------------------------------------
%
% It may be distributed and/or modified under the
% conditions of the LaTeX Project Public License, either version 1.3c
% of this license or (at your option) any later version.
% The latest version of this license is in
%    https://www.latex-project.org/lppl.txt
% and version 1.3c or later is part of all distributions of LaTeX
% version 2005/12/01 or later.
%
% The list of all files belonging to the LaTeX `Tools Bundle' is
% given in the file `manifest.txt'.
%
% \fi
%
%%
%
%
\def\FMithmInfo{2023/07/05 v2.2c Theorem extension package (FMi)}
%
% \ProvidesFile{theorem.dtx}[\FMithmInfo]
%\iffalse   % this is a METACOMMENT !
%%
%% Package `theorem' to use with LaTeX2e
%% Copyright (C) 1989-1995 Frank Mittelbach, all rights reserved.
%<+package>\@ifundefined{theorem@style}{}{\endinput}
%<+package>\NeedsTeXFormat{LaTeX2e}[1995/06/01]
%<+package>\ProvidesPackage{theorem}
%<+package>         [\FMithmInfo]
%
% \fi
%
% \changes{v1.0}{87/05/07}{First draft.}
% \changes{v1.0a}{87/05/07}{`thistheoremstyle' added.}
% \changes{v1.1a}{87/07/29}{New concept: redefine the internal
%                          \protect\LaTeX{} commands `@xnthm' etc.}
% \changes{v1.1b}{87/07/31}{`theoremfont' used to specify font.}
% \changes{v1.2a}{88/01/08}{`@endtheorem' now globally defined.}
% \changes{v1.2b}{88/01/11}{`newtheorem' only allowed in the preamble.}
% \changes{v1.3a}{88/02/14}{`theoremfont' now a toks register.}
%
% \changes{v2.0a}{88/11/16}{New implementation and new documentation.}
% \changes{v2.0a}{88/11/16}{No layout changes but internal modifications
%                           old substyles won't work.}
% \changes{v2.0b}{89/01/30}{`global' used for all definitions.}
% \changes{v2.0c}{89/06/15}{Documentation translated to English.
%                            Thanks to LLL and Nina}
% \changes{v2.1a}{90/02/15}{Support for new font selection.}
% \changes{v2.2a}{94/02/03}{Cleanup for LaTeX2e}
%
% \DoNotIndex{\@endparenv,\@ifdefinable,\@ifnextchar,\@ifundefined}
% \DoNotIndex{\@namedef,\@nameuse,\@spaces,\@tempa,\@warning}
% \DoNotIndex{\ }
% \DoNotIndex{\begingroup,\bfseries}
% \DoNotIndex{\csname}
% \DoNotIndex{\def,\do,\docdate}
% \DoNotIndex{\endcsname,\endgroup,\endinput,\endtrivlist,\expandafter}
% \DoNotIndex{\filedate,\fileversion}
% \DoNotIndex{\gdef,\global}
% \DoNotIndex{\hbox,\hskip}
% \DoNotIndex{\ignorespaces,\input,\itshape,\item}
% \DoNotIndex{\labelsep,\let,\llap}
% \DoNotIndex{\makeatletter}
% \DoNotIndex{\noexpand}
% \DoNotIndex{\refstepcounter,\relax,\rlap}
% \DoNotIndex{\setlength,\slshape,\space,\string,\strut}
% \DoNotIndex{\the,\trivlist,\typeout}
% \DoNotIndex{\vbox}
% \DoNotIndex{\xdef}
%
% \def\deflabel#1{\ttfamily #1\hfill}
% \def\deflist#1{\begin{list}{}{\settowidth\labelwidth{\ttfamily #1}%
%                               \setlength\leftmargin\labelwidth
%                               \addtolength\leftmargin\labelsep
%                               \let\makelabel\deflabel}}
% \def\enddeflist{\end{list}}
% \def\envfont{\normalfont\ttfamily}
%
% \MakeShortVerb{\|}
%
% \newcommand{\TUB}{{\sl TUGboat\/}}
%
% \GetFileInfo{theorem.dtx}
% \title{An Extension of the
%        \LaTeX{} theorem environment\thanks{This file
%        has version number \fileversion, last
%        revised \filedate.}}
% \author{Frank Mittelbach}
%
% \MaintainedByLaTeXTeam{tools}
% \maketitle
%
% \begin{abstract}
%    The macros described in this paper yield an extension of the
%    \LaTeX{} theorem mechanism. It is designed to satisfy the
%    different requirements of various journals. Thus, the
%    layout of the ``theorems'' can be manipulated by determining a
%    ``style''. This article describes not only the use, but
%    also the definition, of the necessary macros.
% \end{abstract}
%
% \section*{Preface to version 2.2}
%
% For \LaTeXe{} this package did not need any fundamental changes. I
% only modified the messages generated so that theorem layout styles
% will show up with the |\listfiles| command and cleaned the section
% on the New Font Selection Scheme since this is now included in
% \LaTeX.
%
% \section*{Preface to version 2.1}
%
% This version is identical to 2.0g described in \TUB\ 10\#3 except
% for some internal defaults which are now set depending on the used
% font selection scheme.
%
% This was done to avoid unpleasant surprises if the new font selection
% scheme is in force. For further details see
% section~\ref{sec:newfonts} and \cite{bk:GMS94}.
%
% \section{Introduction}
%
% For our purposes here, ``theorems'' are labelled enunciations,
% often set off from the main text by extra space and a font change.
% Theorems, corollaries, conjectures, definitions, and remarks
% are all instances of ``theorems''.  The ``header'' of these structures
% is composed of a label (such as \textsc{Theorem} or \textsc{Remark})
% and a number which serializes an item in the sequence of items
% with the same label.
%
% Shortly after the introduction of \LaTeX{} at the Fachbereich
% Mathematik in Mainz, the desire to manipulate the layout of
% ``theorems'' arose.  In Mainz, the following two conventions came
% into general use:
% \begin{enumerate}
%   \item The number of the theorem is shown in the margin.
%   \item There is a line break at
%         the end of the theorem header.
% \end{enumerate}
%
% Additionally, some journals require different formats which depend on
% the ``sort of theorem'': e.g.\ often remarks and definitions are set
% in |\upshape|, while |\itshape| is employed for main theorems.
%
% Confronted with these requirements, a theorem environment was
% developed in Mainz which allows separate determination of the
% layout of the ``theorems sets'', comparable to |\pagestyle|.
%
%
% \section{The user interface}
% \subsection{Defining new theorem sets}
%
% \DescribeMacro\newtheorem
% As in the original \LaTeX{} version, the command |\newtheorem|
% defines a new ``theorem set'' or ``theorem-like structure''.
% Two required arguments name the new environment and give the text
% to be typeset with each instance of the new ``set'', while
% an optional argument determines how
% the ``set'' is enumerated:
% \begin{description}
%    \item[\ttfamily \bslash newtheorem\{foo\}\{bar\}]
%       The theorem set {\envfont foo} (whose name is \texttt{bar})
%       uses its own counter.
%    \item[\ttfamily \bslash newtheorem\{foo2\}{[foo]}\{bar2\}]
%       The theorem set {\envfont foo2} (printed name \texttt{bar2})
%       uses the same counter as the theorem set \texttt{foo}.
%    \item[\ttfamily \bslash newtheorem\{foo3\}\{bar3\}{[section]}]
%       The theorem set {\envfont foo3} (printed name \texttt{bar3}) is
%       enumerated within the counter \texttt{section}, i.e.\ with every
%       new |\section| the enumeration begins again with 1, and
%       the enumeration is composed from the section-number and the
%       theorem counter itself.
% \end{description}
%
% \DescribeMacro\theoremstyle
% Additionally, the command |\theoremstyle| can define the layout
% of various, or all, theorem sets.  It should be noted that any
% theorem set defined by |\newtheorem| is typeset in the
% |\theoremstyle| that is current at the time of the definition.
% Thus, the following
% \begin{verbatim}
%    \theoremstyle{break}        \newtheorem{Cor}{Corollary}
%    \theoremstyle{plain}        \newtheorem{Exa}{Example}[section]
%\end{verbatim}                             ^^A No blank at line start
% leads to the result that the set {\envfont Cor} is formatted in the
% style \texttt{break}, while the set {\envfont Exa} and all the
% following ones are formatted in the style \texttt{plain}, unless
% another |\theoremstyle| follows.  Since the definitions installed by
% |\newtheorem| are global, one also can limit |\theoremstyle| locally
% by grouping braces.
%
% \DescribeMacro\theorembodyfont
% The choice of the font for the theorem body is completely
% independent of the chosen |\theoremstyle|; this has proven to
% be very advantageous.  For example,
% \begin{verbatim}
%   {\theorembodyfont{\upshape}          \newtheorem{Rem}{Remark}}
%\end{verbatim}
% defines a theorem set {\envfont Rem}, which will be set in
% |\upshape| in the current layout (which in our example is
% \texttt{plain}).  As with |\theoremstyle|, the |\theorembodyfont|
% chosen is that current at the time of |\newtheorem|.  If
% |\theorembodyfont| is not specified or one defines
% |\theorembodyfont{}|, then the font used will be that defined by the
% |\theoremstyle|.
%
% \DescribeMacro\theoremheaderfont It is also possible to customize
% the font used for the theorem headers.  This is, however, a global
% declaration, and therefore there should be at most one
% |\theoremheaderfont| declaration in the preamble.\footnote{If it is
% actually necessary to have different header fonts, one has to define
% new theorem styles (substituting the desired font) or specify the
% information directly in the \texttt{\bslash newtheorem} declaration
% (the unclean variant).}
%
% \DescribeMacro\theorempreskipamount
% \DescribeMacro\theorempostskipamount Two additional parameters
% affect the vertical space around the theorem environments:
% |\theorempreskipamount| and |\theorempostskipamount| define,
% respectively, the spacing before and after such an environment.
% These parameters apply for all theorem sets and can be manipulated
% with the ordinary length macros.  They are rubber lengths
% (`\textsf{skips}'), and therefore can contain \texttt{plus} and
% \texttt{minus} parts.
%
% Since the definition of theorem sets should---most sensibly---be
% placed in the preamble, we only allow installation there.
% It is therefore possible to release the memory used here after
% |\begin{document}|, in order to make room for other
% applications.
%
%
% \subsection{Existing theorem styles}
%
% The following theorem styles exist to date:
% \begin{deflist}{marginbreak}
%    \item[plain]
%       This theorem style emulates the original \LaTeX{} definition,
%       except that additionally the parameters
%       |\theorem...skipamount| are used.
%    \item[break]
%       In this style, the theorem header is followed by a line
%       break.
%    \item[marginbreak]
%       The theorem number is set in the margin, and there is a line
%       break as in \texttt{break}.
%    \item[changebreak]
%       Like \texttt{break}, but with header number and text
%       interchanged.
%    \item[change]
%       Header number and text are interchanged, without a line break.
%    \item[margin]
%       The number is set in the left margin, without a line break.
% \end{deflist}
% All styles (except \texttt{plain}) select |\slshape| as the default
% |\theorembodyfont|.
%
%
% \subsection{Examples}
%
% Given the above theorem sets {\envfont Cor}, {\envfont Exa}
% and {\envfont Rem},
% suppose that the preamble also contains the declarations:
% \begin{verbatim}
%    \theoremstyle{marginbreak}   \newtheorem{Lem}[Cor]{Lemma}
%    \theoremstyle{change}
%    \theorembodyfont{\itshape}        \newtheorem{Def}[Cor]{Definition}
%
%    \theoremheaderfont{\scshape}
%\end{verbatim}
% Then the following are some typical examples of the typeset output
% resulting from their use.
%
% \begin{Cor}
%    This is a sentence typeset in the theorem environment {\envfont
%    Cor}.
% \end{Cor}
%
% \begin{Exa}
%    This is a sentence typeset in the theorem environment {\envfont
%    Exa}.
% \end{Exa}
%
% \begin{Rem}
%    This is a sentence typeset in the theorem environment {\envfont
%    Rem}.
% \end{Rem}
%
% \begin{Lem}[Ben User]
%    This is a sentence typeset in the theorem environment {\envfont
%    Lem}.
% \end{Lem}
%
% \begin{Def}[Very impressive Definition]
%    This is a sentence typeset in the theorem environment {\envfont
%    Def}.
% \end{Def}
%
% The last two examples show the effect of the optional argument to
% a theorem environment (it is the text typeset in parentheses).
%
%
% \section{Special Considerations}
% \label{sec:newfonts}
%
% Theoremheader and body are implemented as a unit. This means that
% the |\theoremheaderfont|  will inherit characteristics of the
% |\theorembodyfont| in \LaTeXe. Thus, if for example
% |\theorembodyfont| is |\itshape| and |\theoremheaderfont|
% is |\bfseries| the font selected for the header will have the
% characteristics `bold extended italic'. If this is not desired one
% should set the |\theoremheaderfont| to something like
% \begin{verbatim}
%      \theoremheaderfont{\normalfont\bfseries}
%\end{verbatim}
% i.e.\ supplying all necessary font information explicitly.
%
%
%
% \section{Acknowledgements}
%
% The publication of this set of macros was only possible with the
% help of Christina Busse (translating the manuscript into English),
% Joachim Pense (playing the r\^ole of typist), Chris Rowley
% (looking everything over) and many others providing useful
% suggestions.
%
% \MaybeStop{%
%    \begin{thebibliography}{9}
%       \bibitem{bk:GMS94} \textsc{M.~Goossens}, \textsc{F.~Mittelbach}
%         and \textsc{A.~Samarin}.
%         \newblock The \LaTeX{} Companion.
%         \newblock
%         Addison-Wesley, Reading, Massachusetts, 1994.
%       \bibitem{latex:source} \textsc{Lamport, Leslie.} \newblock
%                               \texttt{latex.tex}, version 2.09, date
%                               Feb. 1990.
%    \end{thebibliography}
%    \PrintIndex
%    \PrintChanges}
%
%
% \section{The documentation driver file}
%
% The next bit of code contains the documentation driver file for
% \TeX{}, i.e., the file that will produce the documentation you are
% currently reading. It will be extracted from this file by the
% \texttt{docstrip} program. Since it is the first code in the file
% one can alternatively process this file directly with \LaTeXe{} to
% obtain the documentation.
%
%    \begin{macrocode}
%<*driver>
\documentclass{ltxdoc}

\usepackage{theorem}

%<+driver>% The next few lines define theorem sets which are used
%<+driver>% in the example section of the documentation.

\theoremstyle{break}       \newtheorem{Cor}{Corollary}
\theoremstyle{plain}       \newtheorem{Exa}{Example}[section]
{\theorembodyfont{\upshape}\newtheorem{Rem}{Remark}}
\theoremstyle{marginbreak} \newtheorem{Lem}[Cor]{Lemma}
\theoremstyle{change}
\theorembodyfont{\itshape} \newtheorem{Def}[Cor]{Definition}

\theoremheaderfont{\scshape}

\RecordChanges

\begin{document}
   \DocInput{theorem.dtx}
\end{document}
%</driver>
%    \end{macrocode}
%
% \section{Definition of the Macros}
%
% If the file has been loaded before, we abort immediately. If not the
% package announces itself (this is actually done at the very top of
% the file---the way it is done isn't good style so don't copy it).
% \changes{v2.0e}{89/07/19}{Spaces removed from `typeout'.}
% \changes{v2.2b}{95/11/19}{Announce moved to top of file}
%    \begin{macrocode}
%<*package>
%\@ifundefined{theorem@style}{}{\endinput}
%\def\FMithmInfo{1995/11/19 v2.2b Theorem extension package (FMi)}
%\ProvidesPackage{theorem}[\FMithmInfo]
%    \end{macrocode}
%
%
% \subsection{Definition of theorem styles and fonts}
%
% All the definitions in this file are done globally to allow
% inputting this file inside a group.
%
% \begin{macro}{\theoremstyle}
%    Before a theorem style can be installed, the chosen style must be
%    known. For that reason, we must test to see that
%    |\th@|\meta{style} is known or, more precisely, that it is
%    different from |\relax|. If the style is not known then
%    |\th@plain| is used.
%    \begin{macrocode}
\gdef\theoremstyle#1{%
   \@ifundefined{th@#1}{\@warning
          {Unknown theoremstyle `#1'. Using `plain'}%
          \theorem@style{plain}}%
%    \end{macrocode}
%    We save the theorem style to be used in the token register
%    |\theorem@style|.
%    \begin{macrocode}
      {\theorem@style{#1}}%
%    \end{macrocode}
%    Now we ``evaluate'' the theorem style: this means, we call the
%    macro |\th@|\meta{style} which will activate the relevant
%    definitions which are contained in a separate file.  This is done
%    in a group to suppress changes to the current font.  This could
%    otherwise pose problems together with the new font selection
%    scheme\footnote{When I printed the original article using the new
%    font selection scheme I ended with a document with slanted
%    typefaces (text headings and all) simply because one of the
%    theorem styles used \texttt{\bslash sl} at toplevel.} if the
%    |\th@|\meta{style} is evaluated a second time.
% \changes{v2.1a}{90/02/15}{Group added around `th@..}
% \changes{v2.1a}{90/02/15}{Reseting of `..@begintheorem removed.}
%    \begin{macrocode}
      \begingroup
        \csname th@\the\theorem@style \endcsname
      \endgroup}
%    \end{macrocode}
% \end{macro}
%
%
% \begin{macro}{\@begintheorem}
% \changes{v2.1a}{90/02/15}{Relaxed at toplevel.}
% \begin{macro}{\@opargbegintheorem}
% \changes{v2.1a}{90/02/15}{Relaxed at toplevel.}
%    We reset |\@begintheorem| and |\@opargbegintheorem| to
%    |\relax| since these commands are no longer necessary at
%    toplevel. This will save a few tokens.
%    \begin{macrocode}
\global\let\@begintheorem\relax
\global\let\@opargbegintheorem\relax
%    \end{macrocode}
% \end{macro}
% \end{macro}
%
% \begin{macro}{\theorem@style}
%    Obviously the \textsf{token} register used above has to be
%    allocated. To assure the utmost compatibility with the
%    original \LaTeX{} definition, we set the default theorem style to
%    \texttt{plain}, which implements the usual \LaTeX{} convention.
% \changes{v2.0f}{89/07/20}{Unnecessary `global' removed.}
%    \begin{macrocode}
\newtoks\theorem@style
\global\theorem@style{plain}
%    \end{macrocode}
% \end{macro}
%
% \begin{macro}{\theorembodyfont}
% \begin{macro}{\theorem@bodyfont}
%    If the |\theorembodyfont| is set by the user then it should not
%    interact with the default font set in the theorem style. When the
%    new font selection is in force this may happen if, for example,
%    the default is |\itshape| and the new |\theorembodyfont| is
%    |\sffamily|. So we add a |\reset@font| command in front of the
%    user definition.
% \changes{v2.0f}{89/07/20}{Unnecessary `global' removed.}
% \changes{v2.0g}{89/09/19}{`theoremfont to `theorembodyfont.}
% \changes{v2.1c}{92/06/22}{`reset@font added and logic changed.}
% \changes{v2.1d}{92/10/28}{Reset to empty if arg is empty}
%    \begin{macrocode}
\gdef\theorembodyfont#1{%
%    \end{macrocode}
%    We check if the argument supplied is empty and if so put nothing
%    into the |\theorem@bodyfont| token register to allow for
%    |\theorembodyfont{}| as a mean of using the default of the
%    current |\theoremstyle|.
%    \begin{macrocode}
   \def\@tempa{#1}%
   \ifx\@tempa\@empty
    \theorem@bodyfont{}%
   \else
    \theorem@bodyfont{\reset@font#1}%
   \fi
   }
\newtoks\theorem@bodyfont
\global\theorem@bodyfont{}
%    \end{macrocode}
% \end{macro}
% \end{macro}
%
%
% \begin{macro}{\theoremheaderfont}
% \changes{v2.0g}{89/09/19}{Macro added.}
%    The font for the theorem headers is handled differently because
%    this definition applies to all theorem styles.
%    \begin{macrocode}
\gdef\theoremheaderfont#1{\gdef\theorem@headerfont{#1}%
%    \end{macrocode}
%    After using the macro once it is redefined to produce an error
%    message.
%    \begin{macrocode}
       \gdef\theoremheaderfont##1{%
        \typeout{\string\theoremheaderfont\space should be used
                 only once.}}}
%    \end{macrocode}
% \end{macro}
%
% \begin{macro}{\theorem@headerfont}
% \changes{v2.0g}{89/09/19}{Macro added.}
%    To set the |\theorem@headerfont| default we first test if
%    the new fontselection scheme is in force.
% \changes{v2.1a}{90/02/15}{Support for new font selection.}
%    \begin{macrocode}
\ifx\upshape\undefined
%    \end{macrocode}
%    If not we define it to expand into |\bfseries|. We don't use
%    |\let| just in case a following style option redefines this
%    macro.
%    \begin{macrocode}
\gdef\theorem@headerfont{\bfseries}
%    \end{macrocode}
%    Otherwise we reset the current shape before calling |\bfseries|.
%    \begin{macrocode}
\else \gdef\theorem@headerfont{\normalfont\bfseries}\fi
%    \end{macrocode}
% \end{macro}
%
% \begin{macro}{\th@plain}
% \begin{macro}{\th@break}
% \begin{macro}{\th@marginbreak}
% \begin{macro}{\th@changebreak}
% \begin{macro}{\th@change}
% \begin{macro}{\th@margin}
%    The different styles are defined in macros such as
%    |\th@plain|.  Since memory space is precious in
%    ``non-Big-versions'', we have to avoid offering too many unused
%    definitions.  Therefore we define these styles in separate files
%    that can be loaded on demand.  Thus the commands themselves only
%    load these files. We use |\@input@|, a \LaTeXe{} internal command
%    that ensures that the file will be listed with |\listfiles|.
% \changes{v2.2a}{94/02/02}{Use `@input@ to load theorem layout files}
%    \begin{macrocode}
\gdef\th@plain{\@input@{thp.sty}}
\gdef\th@break{\@input@{thb.sty}}
\gdef\th@marginbreak{\@input@{thmb.sty}}
\gdef\th@changebreak{\@input@{thcb.sty}}
\gdef\th@change{\@input@{thc.sty}}
\gdef\th@margin{\@input@{thm.sty}}
%    \end{macrocode}
%    This list will be expanded when new styles become
%    available.  For testing,  just append new theorem substyles
%    as document options.
% \end{macro}
% \end{macro}
% \end{macro}
% \end{macro}
% \end{macro}
% \end{macro}
%
%
%
% \subsection{Definition of a new theorem set}
%
% As already pointed out, a new theorem environment can be defined in
% three different ways:
% \begin{verbatim}
%    \newtheorem{Lem}{Lemma}
%    \newtheorem{Lem}{Lemma}[section]
%    \newtheorem{Lem}[Theorem]{Lemma}
%\end{verbatim}
% The function of the macro |\newtheorem| is to recognize these cases
% and then to branch into one of the three macros |\@ynthm|, |\@xnthm|
% or |\@othm|.  This mechanism is adopted unchanged from
% \cite{latex:source}; the essential point here is that, for example,
% in the second case, the arguments \texttt{Lem}, \texttt{Lemma} and
% \texttt{section} are passed over to the macro |\@xnthm|.
%
% We inspect this case first because the others present fewer problems,
% and thus are easily derived from this one.
%
% \begin{macro}{\@xnthm}
%    For our example arguments, the macro |\@xnthm| must fulfill
%    the following:
%    \begin{itemize}
%       \item
%          Define a new \LaTeX-counter `Lem'
%       \item
%          reset this counter within a |\section|
%       \item
%          define the macro |\theLem|
%       \item
%          define the environment macros |\Lem| and
%          |\endLem| using the current
%          |\theoremstyle| and |\theorem@bodyfont|.
%    \end{itemize}
%    Obviously, all this should happen only if the first argument
%    of |\@xnthm| (i.e.\ \texttt{Lem} in our example) is chosen so
%    as not to conflict with any previously defined commands or
%    environments.  This test is performed by the \LaTeX{} macro
%    |\@ifdefinable|.
%    \begin{macrocode}
\gdef\@xnthm#1#2[#3]{\expandafter\@ifdefinable\csname #1\endcsname
%    \end{macrocode}
%    Therefore, the first argument of |\@ifdefinable| is the
%    expansion (in the example, |\Lem|) of
%    |\csname||#1\endcsname|.  The second argument is
%    executed only if the test has been completed successfully.
%    \begin{macrocode}
   {%
%    \end{macrocode}
%    Now we define the new counter. The names of the \LaTeX{}
%    macros employed should speak for themselves:
% \changes{v2.2b}{1995/10/07}{Use \cs{@newctr}}
% \changes{v2.2c}{1995/11/23}{Use \cs{@newctr} properly.}
%    \begin{macrocode}
    \@definecounter{#1}\@newctr{#1}[#3]%
%    \end{macrocode}
%    Using |\@newctr| will give a proper error message if the counter
%    in |#3| is not defined.
%    In defining `|\theLem|' we must generate the desired macro
%    name by use of |\expandafter| and |\csname|.
%    \begin{macrocode}
    \expandafter\xdef\csname the#1\endcsname
%    \end{macrocode}
%    An |\xdef| is used in order to make the definition global,
%    and to ensure that it contains the replacement texts of
%    |\@thmcountersep| and |\@thmcounter|.\footnote{These
%    two macros can be defined by the document style.  Their default
%    values produce a `\texttt{.}' as separation and an arabic
%    representation of the number.} However, not everything should be
%    expanded. For example, it saves space to use |\thesection|
%    instead of its---at times---lengthy expansion.
%    \begin{macrocode}
      {\expandafter \noexpand \csname the#3\endcsname
       \@thmcountersep \@thmcounter{#1}}%
%    \end{macrocode}
%    Thus with the defaults of \LaTeX{}, |\theLem| would be
%    replaced by the command sequence |\thesection.\arabic{Lem}|.
%
%    We will now look at the definition of the macro which is executed
%    at the beginning of the actual environment (in our example this
%    macro is |\Lem|). It should be noted that we use an
%    ``|\expandafter| trick'' to expand only certain parts of the
%    replacement text at the time of the definition.
%    \begin{macrocode}
    \def\@tempa{\global\@namedef{#1}}%
    \expandafter \@tempa \expandafter{%
%    \end{macrocode}
%    First, the macro that contains the current definitions of
%    |\@begintheorem| and |\@opargtheorem| should be called
%    up.  The name of this macro---as is already known---has the form
%    |\th@|\meta{theorem\ style}; therefore, it must be called by
%    \begin{macrocode}
      \csname th@\the \theorem@style
%    \end{macrocode}
%    In addition the default theorem font should be changeable, i.e.\
%    we have to insert the contents of |\theorem@bodyfont|. For
%    that reason, we expand even further, beyond |\endcsname|,
%    and thus insert the contents of the \textsf{token} register
%    |\theorem@bodyfont| in the replacement text.
%    \begin{macrocode}
            \expandafter \endcsname \the \theorem@bodyfont
%    \end{macrocode}
%    Now it is time to call the macro |\@thm| which takes over
%    the further processing.  It has two arguments: the current
%    counter name (in our example, \texttt{Lem}), and the text of the
%    label (in our example, \texttt{Lemma}).
%    \begin{macrocode}
     \@thm{#1}{#2}}%
%    \end{macrocode}
%    With this, the `sub-definition' is complete.  The macro
%    |\@endtheorem| ends a theorem environment and is, so far, nothing
%    but an |\endtrivlist|.  (Hence it is defined globally, and not
%    within the theorem styles.\footnote{This has to be changed as
%    soon as theorem styles that change \texttt{\bslash @endtheorem}
%    exist.  In such a case, all existing styles must be changed as
%    well since they will have to reset the macro.}) Therefore, we can
%    set it equivalent to the macro that ends the theorem set (in our
%    example, |\endLem|).  However, if some day theorem styles exist
%    that do change |\@endtheorem|, we would have to use the
%    commented-out line instead.
%    \begin{macrocode}
    \global \expandafter \let \csname end#1\endcsname \@endtheorem
%   \global\@namedef{end#1}{\@endtheorem}%
%    \end{macrocode}
%    With these commands all the required definitions are employed,
%    unless the test |\@ifdefinable| has failed.  Therefore, we
%    end the second argument of this macro and with it the definition
%    of |\@xnthm|.
%    \begin{macrocode}
   }}
%    \end{macrocode}
% \end{macro}
%
% \begin{macro}{\@ynthm}
%    The definition of |\@ynthm| is completely analogous. In
%    this case the new counter that is defined is not reset within
%    another counter; thus the definition of |\the...|
%    is simplified:
%    \begin{macrocode}
\gdef\@ynthm#1#2{\expandafter\@ifdefinable\csname #1\endcsname
   {\@definecounter{#1}%
    \expandafter\xdef\csname the#1\endcsname{\@thmcounter{#1}}%
%    \end{macrocode}
%    The rest of the definition corresponds literally to that of
%    |\@xnthm|:
%    \begin{macrocode}
    \def\@tempa{\global\@namedef{#1}}\expandafter \@tempa
     \expandafter{\csname th@\the \theorem@style \expandafter
     \endcsname \the\theorem@bodyfont \@thm{#1}{#2}}%
    \global \expandafter \let \csname end#1\endcsname \@endtheorem}}
%    \end{macrocode}
% \end{macro}
%
% \begin{macro}{\@othm}
%    The definition of |\@othm| does not contain anything new.
%    \begin{macrocode}
\gdef\@othm#1[#2]#3{%
%    \end{macrocode}
% \changes{v2.2b}{95/10/07}{Build in check for counter defined}
%    We do not define a new counter but instead use one that has
%    already been defined.  Thus the only definition we need is that
%    of this pseudo-counter (i.e.\ |\the|\meta{env.\ name}).
%    First we check if |#2| corresponds to a known counter name.
%    \begin{macrocode}
  \expandafter\ifx\csname c@#2\endcsname\relax
   \@nocounterr{#2}%
  \else
   \expandafter\@ifdefinable\csname #1\endcsname
   {\expandafter \xdef \csname the#1\endcsname
     {\expandafter \noexpand \csname the#2\endcsname}%
%    \end{macrocode}
%    All other parts of the definition can be adopted from
%    |\@xnthm|. We have to remember, though, that in this case
%    the name of the current counter and the theorem label have moved
%    to the second and third arguments.
%    \begin{macrocode}
    \def\@tempa{\global\@namedef{#1}}\expandafter \@tempa
     \expandafter{\csname th@\the \theorem@style \expandafter
     \endcsname \the\theorem@bodyfont \@thm{#2}{#3}}%
    \global \expandafter \let \csname end#1\endcsname \@endtheorem}%
  \fi}
%    \end{macrocode}
% \end{macro}
%
% \subsection{Macros that are employed in a {\envfont theorem}
%    environment}
%
% \begin{macro}{\@thm}
%    The macro |\@thm| has to increase the current counter.  Then,
%    depending on whether the environment has (or does not have) an
%    optional argument, it has to branch into either |\@begintheorem|
%    or |\@opargtheorem|.
%    \begin{macrocode}
\gdef\@thm#1#2{\refstepcounter{#1}%
%    \end{macrocode}
%    Now we start a {\envfont trivlist} environment, and give
%    |\@topsep| and |\@topsepadd| the values of the \textsf{skip}
%    registers |\theorempreskipamount| and |\theorempostskipamount|.
%    The value in |\@topsep| is the vertical space that is inserted by
%    the first (and only) |\item| in our |\trivlist| whilst
%    |\@topsepadd| is inserted by |\@endparenv| at the end of that
%    {\envfont trivlist} environment.  By using these registers, we
%    obtain the desired space around a {\envfont theorem} environment.
%    \begin{macrocode}
   \trivlist
   \@topsep \theorempreskipamount               % used by first \item
   \@topsepadd \theorempostskipamount           % used by \@endparenv
%    \end{macrocode}
%    Now we have to test whether an optional argument has been given.
% \changes{v2.0d}{89/07/18}{Forgotten percent added.}
%    \begin{macrocode}
   \@ifnextchar [%
%    \end{macrocode}
%    If there is an optional argument, we will call |\@ythm|, and
%    move the arguments read back into the input stream.
%    \begin{macrocode}
   {\@ythm{#1}{#2}}%
%    \end{macrocode}
%    If not, we call |\@begintheorem|. Its first argument is the
%    name of the theorem set (hence the second argument of
%    |\@thm|). Its second argument is the macro that produces the
%    current number.
%    \begin{macrocode}
   {\@begintheorem{#2}{\csname the#1\endcsname}\ignorespaces}}
%    \end{macrocode}
% \end{macro}
%
% \begin{macro}{\@xthm}
% \begin{macro}{\@ythm}
%    Both these macros were originally called by |\@thm|.  We do
%    not need |\@xthm| anymore, hence we reset it to
%    |\relax|. The definition of |\@ythm| has not changed at
%    all from its definition in \LaTeX. In order to make the macros
%    easier to understand, we will
%    nevertheless present it (commented out).
%    \begin{macrocode}
\global\let\@xthm\relax
% \def\@ythm#1#2[#3]{\@opargbegintheorem{#2}{\csname
%           the#1\endcsname}{#3}\ignorespaces}
%</package>
%    \end{macrocode}
%    The primitive |\ignorespaces| in |\@ythm| and
%    |\@thm| is needed to remove the spaces between the
%    |\begin{...}| and the actual text.
% \end{macro}
% \end{macro}
%
%
% \subsection{Definition of the theorem substyles}
%
% As already pointed out, the theorem substyles, defined below, are
% only loaded when necessary.  Note that all these substyles, except
% \texttt{plain}, have |\slshape| as the default body font.
%
% \subsubsection{The \texttt{plain} style}
%
% \label{dok:th@plain} As the following macros use |@|, we have
% to locally set the |\catcode| of this symbol to ``letter''.
% This happens within a group, so that we do not have to worry about
% which |\catcode| that symbol had before.
%    \begin{macrocode}
%<*thp>
\begingroup \makeatletter
%    \end{macrocode}
% Since we are now within a group, we must make all definitions
% globally.  First we make sure that \texttt{theorem.sty} is loaded.
% This will allow us to use this file as a document style option
% without having to call \texttt{theorem} itself as an option.  At the
% same time, we assure that at least version 2 is loaded, since
% |\theorem@style| was not defined in earlier versions.
%    \begin{macrocode}
\@ifundefined{theorem@style}{\input{theorem.sty}}{}
\ProvidesFile{thp.sty}
             [\FMithmInfo]
%    \end{macrocode}
%
%
% \begin{macro}{\th@plain}
% \begin{macro}{\@begintheorem}
% \begin{macro}{\@opargbegintheorem}
%    |\theoremstyle{plain}| corresponds to the original
%    definition, except that the distances to the surrounding text are
%    determined by the parameters |\theorempreskipamount| and
%    |\theorempostskipamount|.
%    First we set the default body font.
%    \begin{macrocode}
\gdef\th@plain{\normalfont\itshape
%    \end{macrocode}
%    Then we define |\@begintheorem| and
%    |\@opargbegintheorem|. These two macros define how the
%    header of a theorem is typeset.  |\@opargbegintheorem| will
%    be called if a \textsf{theorem} environment with an optional
%    argument is encountered; otherwise, the header is constructed by
%    calling |\@begintheorem|.  If one of these macros is
%    executed, we are within a \textsf{trivlist} environment started by
%    |\@thm|.  So the theorem header is produced with an
%    |\item| command.
%
%    Instead of specifying the header font directly, all standard
%    theorem styles use the |\theorem@headerfont| macro to allow
%    customization.  The extra space (|\labelsep|) is necessary
%    because of problems in the \textsf{trivlist} environment.
%    \begin{macrocode}
  \def\@begintheorem##1##2{%
        \item[\hskip\labelsep \theorem@headerfont ##1\ ##2]}%
%    \end{macrocode}
%    The definition of |\@opargbegintheorem| is completely
%    analogous. The only difference is the fact that there exists a
%    third argument (which is the optional parameter of the
%    environment and contains additional information about the
%    theorem).  Customarily we enclose it in parentheses.
%    \begin{macrocode}
\def\@opargbegintheorem##1##2##3{%
   \item[\hskip\labelsep \theorem@headerfont ##1\ ##2\ (##3)]}}
%    \end{macrocode}
% \end{macro}
% \end{macro}
% \end{macro}
%
% \noindent
% We conclude with an |\endgroup| to restore the
% |\catcode| of |@|.
%    \begin{macrocode}
\endgroup
%</thp>
%    \end{macrocode}
%
%
% \subsubsection{The \texttt{break} style}
%
% This style option is stored in the file |thb.sty|.
% For the next two lines see the documentation for |\th@plain| on
% page~\pageref{dok:th@plain}.
%    \begin{macrocode}
%<*thb>
\begingroup \makeatletter
\@ifundefined{theorem@style}{\input{theorem.sty}}{}
\ProvidesFile{thb.sty}
             [\FMithmInfo]
%    \end{macrocode}
%
% \begin{macro}{\th@break}
%    |\theoremstyle{break}| produces a line break after the name
%    of the theorem.  The font is |\slshape|.  Hence, we define
%    |\th@break| as follows:
%    \begin{macrocode}
\gdef\th@break{\normalfont\slshape
  \def\@begintheorem##1##2{\item[%
%    \end{macrocode}
%    We run into the following problem: it is not possible to create
%    the header with |\item[|\meta{title}|]| and then start
%    a new line by, for example, |\mbox{}\\|. Such a definition
%    will fail whenever a list environment follows immediately.  With
%    the above construction, the |\mbox{}| causes the switch
%    |@inlabel| (cf.\ definition of |\list| and
%    |\trivlist| in \cite{latex:source}) to be set to \texttt{false}
%    and so the following list will insert additional vertical
%    space (|\topskip|). This is quite annoying. Therefore, we
%    create the line break within the |\item|. In order to ensure
%    that the text will begin at the proper position in the following
%    line, we simply pretend that the label does not take any
%    room.\footnote{This will lead to problems whenever very high
%    symbols occurring in the line tower into the
%    heading. So, something else has to be done here sometime.}
%    \begin{macrocode}
       \rlap{\vbox{\hbox{\hskip \labelsep\theorem@headerfont ##1\ ##2}%
                   \hbox{\strut}}}]}%
%    \end{macrocode}
%    Again, the definition of |\@opargbegintheorem| is
%    completely analogous.
%    \begin{macrocode}
\def\@opargbegintheorem##1##2##3{%
  \item[\rlap{\vbox{\hbox{\hskip \labelsep \theorem@headerfont
                     ##1\ ##2\ (##3)}%
                    \hbox{\strut}}}]}}
\endgroup
%</thb>
%    \end{macrocode}
% \end{macro}
%
%
% \subsubsection{The \texttt{changebreak} style}
%
%    \begin{macrocode}
% This style option is stored in the file |thcb.sty|.
%    \begin{macrocode}
%<*thcb>
\begingroup \makeatletter
\@ifundefined{theorem@style}{\input{theorem.sty}}{}
\ProvidesFile{thcb.sty}
             [\FMithmInfo]
%    \end{macrocode}
%
% \begin{macro}{\th@changebreak}
%    The \texttt{change-break} theorem style is like \texttt{break}
%    but with interchange of theorem name and theorem number.  Thus we
%    define |\th@changebreak| as follows:
%    \begin{macrocode}
\gdef\th@changebreak{\normalfont\slshape
  \def\@begintheorem##1##2{\item
       [\rlap{\vbox{\hbox{\hskip\labelsep \theorem@headerfont ##2\ ##1}%
                    \hbox{\strut}}}]}%
\def\@opargbegintheorem##1##2##3{%
  \item[\rlap{\vbox{\hbox{\hskip\labelsep \theorem@headerfont
                     ##2\ ##1\ (##3)}%
                    \hbox{\strut}}}]}}
\endgroup
%</thcb>
%    \end{macrocode}
% \end{macro}
%
%
% \subsubsection{The \texttt{change} style}
%
% This style option is stored in the file |thc.sty|.
%    \begin{macrocode}
%<*thc>
\begingroup \makeatletter
\@ifundefined{theorem@style}{\input{theorem.sty}}{}
\ProvidesFile{thc.sty}
             [\FMithmInfo]
%    \end{macrocode}
%
% \begin{macro}{\th@change}
%    The \texttt{change} theorem style corresponds to the
%    \texttt{change break} style without a linebreak after the header.
%    To say it in another way, it's the same as the \texttt{plain}
%    style but with number and name interchanged and |\slshape| as the
%    default font.
%    \begin{macrocode}
\gdef\th@change{\normalfont\slshape
  \def\@begintheorem##1##2{\item
       [\hskip\labelsep \theorem@headerfont ##2\ ##1]}%
\def\@opargbegintheorem##1##2##3{%
  \item[\hskip\labelsep \theorem@headerfont ##2\ ##1\ (##3)]}}
\endgroup
%</thc>
%    \end{macrocode}
% \end{macro}
%
%
% \subsubsection{The \texttt{marginbreak} style}
%
% This style option is the one used most often at Mainz.  It is saved
% in the file |thmb.sty|.
%    \begin{macrocode}
%<*thmb>
\begingroup \makeatletter
\@ifundefined{theorem@style}{\input{theorem.sty}}{}
\ProvidesFile{thmb.sty}
             [\FMithmInfo]
%    \end{macrocode}
%
% \begin{macro}{\th@marginbreak}
%    The \texttt{margin break} style is nearly the same as the
%    \texttt{change break} style. The only difference is the placement
%    of the theorem number.  We use |\llap| to place it in the left
%    margin.
%
%    In this style |\labelsep| denotes the separation between the
%    number and the text.
%    \begin{macrocode}
\gdef\th@marginbreak{\normalfont\slshape
  \def\@begintheorem##1##2{\item
       [\rlap{\vbox{\theorem@headerfont
                    \hbox{\llap{##2}\hskip\labelsep ##1}%
                    \hbox{\strut}}}]}%
\def\@opargbegintheorem##1##2##3{%
  \item[\rlap{\vbox{\theorem@headerfont
                    \hbox{\llap{##2}\hskip\labelsep ##1\ (##3)}%
                    \hbox{\strut}}}]}}
\endgroup
%</thmb>
%    \end{macrocode}
% \changes{v2.1c}{90/02/17}{`endgroup outside macrocode env.}
% \end{macro}
%
%
%
% \subsubsection{The \texttt{margin} style}
%
% This style option is stored in the file |thm.sty|.
%    \begin{macrocode}
%<*thm>
\begingroup \makeatletter
\@ifundefined{theorem@style}{\input{theorem.sty}}{}
\ProvidesFile{thm.sty}
             [\FMithmInfo]
%    \end{macrocode}
%
% \begin{macro}{\th@margin}
%    Again this is only a variant of the theorem styles described
%    above without any new ideas.
%    \begin{macrocode}
\gdef\th@margin{\normalfont\slshape
  \def\@begintheorem##1##2{\item
       [\theorem@headerfont \llap{##2}\hskip\labelsep ##1]}%
\def\@opargbegintheorem##1##2##3{%
  \item[\theorem@headerfont \llap{##2}\hskip\labelsep ##1\ (##3)]}}
\endgroup
%</thm>
%    \end{macrocode}
% \end{macro}
%
%
%
%
% \subsection{Final Definitions}
%
% \begin{macro}{\theorempreskipamount}
% \begin{macro}{\theorempostskipamount}
%    The \textsf{skip} parameters that regulate the vertical empty space
%    before and after the {\envfont theorem} environment have
%    to be allocated as well.
% \changes{v2.0f}{89/07/20}{Unnecessary `global' removed.}
%    \begin{macrocode}
%<*package>
\newskip\theorempreskipamount
\newskip\theorempostskipamount
%    \end{macrocode}
%    Since we have used the same values for all theorem sets, we now can
%    assign them.
%    \begin{macrocode}
\global\setlength\theorempreskipamount{12pt plus 5pt minus 3pt}
\global\setlength\theorempostskipamount{8pt plus 3pt minus 1.5pt}
%    \end{macrocode}
% \end{macro}
% \end{macro}
%
%
% \begin{macro}{\@endtheorem}
%    The same holds for the macro |\@endtheorem|, which ends a
%    {\envfont theorem} environment. Since it is the same for all
%    theorem sets, it is removed from the macros |\th@|\meta{style}.
%    It simply ends the {\envfont trivlist} environment, which was
%    begun in |\@thm|.
%    \begin{macrocode}
\global\let\@endtheorem=\endtrivlist
%    \end{macrocode}
% \end{macro}
%
%
%
% \begin{macro}{\@preamblecmds}
%    All macros defined above are to be used only in the preamble.
%    Therefore, we insert them in |\@preamblecmds| which will disable
%    them at begin document. This is done by the internal \LaTeXe{}
%    command |\@onlypreamble|.
% \changes{v2.0g}{89/09/19}{`theoremheaderfont and `theorembodyfont
%    added.}
% \changes{v2.2a}{94/02/03}{Use \LaTeXe{} interface}
%    \begin{macrocode}
\@onlypreamble\@xnthm
\@onlypreamble\@ynthm
\@onlypreamble\@othm
\@onlypreamble\newtheorem
\@onlypreamble\theoremstyle
\@onlypreamble\theorembodyfont
\@onlypreamble\theoremheaderfont
%    \end{macrocode}
% \end{macro}
%
% Finally we declare the \texttt{plain} theorem style to be the default.
% \changes{v2.1b}{90/06/04}{Fixed missing `item by declaring default
%    style.}
%    \begin{macrocode}
\theoremstyle{plain}
%</package>
%    \end{macrocode}
%
% \Finale

\endinput
}{}
\ProvidesFile{thp.sty}
             [\FMithmInfo]
%    \end{macrocode}
%
%
% \begin{macro}{\th@plain}
% \begin{macro}{\@begintheorem}
% \begin{macro}{\@opargbegintheorem}
%    |\theoremstyle{plain}| corresponds to the original
%    definition, except that the distances to the surrounding text are
%    determined by the parameters |\theorempreskipamount| and
%    |\theorempostskipamount|.
%    First we set the default body font.
%    \begin{macrocode}
\gdef\th@plain{\normalfont\itshape
%    \end{macrocode}
%    Then we define |\@begintheorem| and
%    |\@opargbegintheorem|. These two macros define how the
%    header of a theorem is typeset.  |\@opargbegintheorem| will
%    be called if a \textsf{theorem} environment with an optional
%    argument is encountered; otherwise, the header is constructed by
%    calling |\@begintheorem|.  If one of these macros is
%    executed, we are within a \textsf{trivlist} environment started by
%    |\@thm|.  So the theorem header is produced with an
%    |\item| command.
%
%    Instead of specifying the header font directly, all standard
%    theorem styles use the |\theorem@headerfont| macro to allow
%    customization.  The extra space (|\labelsep|) is necessary
%    because of problems in the \textsf{trivlist} environment.
%    \begin{macrocode}
  \def\@begintheorem##1##2{%
        \item[\hskip\labelsep \theorem@headerfont ##1\ ##2]}%
%    \end{macrocode}
%    The definition of |\@opargbegintheorem| is completely
%    analogous. The only difference is the fact that there exists a
%    third argument (which is the optional parameter of the
%    environment and contains additional information about the
%    theorem).  Customarily we enclose it in parentheses.
%    \begin{macrocode}
\def\@opargbegintheorem##1##2##3{%
   \item[\hskip\labelsep \theorem@headerfont ##1\ ##2\ (##3)]}}
%    \end{macrocode}
% \end{macro}
% \end{macro}
% \end{macro}
%
% \noindent
% We conclude with an |\endgroup| to restore the
% |\catcode| of |@|.
%    \begin{macrocode}
\endgroup
%</thp>
%    \end{macrocode}
%
%
% \subsubsection{The \texttt{break} style}
%
% This style option is stored in the file |thb.sty|.
% For the next two lines see the documentation for |\th@plain| on
% page~\pageref{dok:th@plain}.
%    \begin{macrocode}
%<*thb>
\begingroup \makeatletter
\@ifundefined{theorem@style}{% \iffalse meta-comment
%
% Copyright (C) 1993-2024
%
% The LaTeX Project and any individual authors listed elsewhere
% in this file.
%
% This file is part of the Standard LaTeX `Tools Bundle'.
% -------------------------------------------------------
%
% It may be distributed and/or modified under the
% conditions of the LaTeX Project Public License, either version 1.3c
% of this license or (at your option) any later version.
% The latest version of this license is in
%    https://www.latex-project.org/lppl.txt
% and version 1.3c or later is part of all distributions of LaTeX
% version 2005/12/01 or later.
%
% The list of all files belonging to the LaTeX `Tools Bundle' is
% given in the file `manifest.txt'.
%
% \fi
%
%%
%
%
\def\FMithmInfo{2023/07/05 v2.2c Theorem extension package (FMi)}
%
% \ProvidesFile{theorem.dtx}[\FMithmInfo]
%\iffalse   % this is a METACOMMENT !
%%
%% Package `theorem' to use with LaTeX2e
%% Copyright (C) 1989-1995 Frank Mittelbach, all rights reserved.
%<+package>\@ifundefined{theorem@style}{}{\endinput}
%<+package>\NeedsTeXFormat{LaTeX2e}[1995/06/01]
%<+package>\ProvidesPackage{theorem}
%<+package>         [\FMithmInfo]
%
% \fi
%
% \changes{v1.0}{87/05/07}{First draft.}
% \changes{v1.0a}{87/05/07}{`thistheoremstyle' added.}
% \changes{v1.1a}{87/07/29}{New concept: redefine the internal
%                          \protect\LaTeX{} commands `@xnthm' etc.}
% \changes{v1.1b}{87/07/31}{`theoremfont' used to specify font.}
% \changes{v1.2a}{88/01/08}{`@endtheorem' now globally defined.}
% \changes{v1.2b}{88/01/11}{`newtheorem' only allowed in the preamble.}
% \changes{v1.3a}{88/02/14}{`theoremfont' now a toks register.}
%
% \changes{v2.0a}{88/11/16}{New implementation and new documentation.}
% \changes{v2.0a}{88/11/16}{No layout changes but internal modifications
%                           old substyles won't work.}
% \changes{v2.0b}{89/01/30}{`global' used for all definitions.}
% \changes{v2.0c}{89/06/15}{Documentation translated to English.
%                            Thanks to LLL and Nina}
% \changes{v2.1a}{90/02/15}{Support for new font selection.}
% \changes{v2.2a}{94/02/03}{Cleanup for LaTeX2e}
%
% \DoNotIndex{\@endparenv,\@ifdefinable,\@ifnextchar,\@ifundefined}
% \DoNotIndex{\@namedef,\@nameuse,\@spaces,\@tempa,\@warning}
% \DoNotIndex{\ }
% \DoNotIndex{\begingroup,\bfseries}
% \DoNotIndex{\csname}
% \DoNotIndex{\def,\do,\docdate}
% \DoNotIndex{\endcsname,\endgroup,\endinput,\endtrivlist,\expandafter}
% \DoNotIndex{\filedate,\fileversion}
% \DoNotIndex{\gdef,\global}
% \DoNotIndex{\hbox,\hskip}
% \DoNotIndex{\ignorespaces,\input,\itshape,\item}
% \DoNotIndex{\labelsep,\let,\llap}
% \DoNotIndex{\makeatletter}
% \DoNotIndex{\noexpand}
% \DoNotIndex{\refstepcounter,\relax,\rlap}
% \DoNotIndex{\setlength,\slshape,\space,\string,\strut}
% \DoNotIndex{\the,\trivlist,\typeout}
% \DoNotIndex{\vbox}
% \DoNotIndex{\xdef}
%
% \def\deflabel#1{\ttfamily #1\hfill}
% \def\deflist#1{\begin{list}{}{\settowidth\labelwidth{\ttfamily #1}%
%                               \setlength\leftmargin\labelwidth
%                               \addtolength\leftmargin\labelsep
%                               \let\makelabel\deflabel}}
% \def\enddeflist{\end{list}}
% \def\envfont{\normalfont\ttfamily}
%
% \MakeShortVerb{\|}
%
% \newcommand{\TUB}{{\sl TUGboat\/}}
%
% \GetFileInfo{theorem.dtx}
% \title{An Extension of the
%        \LaTeX{} theorem environment\thanks{This file
%        has version number \fileversion, last
%        revised \filedate.}}
% \author{Frank Mittelbach}
%
% \MaintainedByLaTeXTeam{tools}
% \maketitle
%
% \begin{abstract}
%    The macros described in this paper yield an extension of the
%    \LaTeX{} theorem mechanism. It is designed to satisfy the
%    different requirements of various journals. Thus, the
%    layout of the ``theorems'' can be manipulated by determining a
%    ``style''. This article describes not only the use, but
%    also the definition, of the necessary macros.
% \end{abstract}
%
% \section*{Preface to version 2.2}
%
% For \LaTeXe{} this package did not need any fundamental changes. I
% only modified the messages generated so that theorem layout styles
% will show up with the |\listfiles| command and cleaned the section
% on the New Font Selection Scheme since this is now included in
% \LaTeX.
%
% \section*{Preface to version 2.1}
%
% This version is identical to 2.0g described in \TUB\ 10\#3 except
% for some internal defaults which are now set depending on the used
% font selection scheme.
%
% This was done to avoid unpleasant surprises if the new font selection
% scheme is in force. For further details see
% section~\ref{sec:newfonts} and \cite{bk:GMS94}.
%
% \section{Introduction}
%
% For our purposes here, ``theorems'' are labelled enunciations,
% often set off from the main text by extra space and a font change.
% Theorems, corollaries, conjectures, definitions, and remarks
% are all instances of ``theorems''.  The ``header'' of these structures
% is composed of a label (such as \textsc{Theorem} or \textsc{Remark})
% and a number which serializes an item in the sequence of items
% with the same label.
%
% Shortly after the introduction of \LaTeX{} at the Fachbereich
% Mathematik in Mainz, the desire to manipulate the layout of
% ``theorems'' arose.  In Mainz, the following two conventions came
% into general use:
% \begin{enumerate}
%   \item The number of the theorem is shown in the margin.
%   \item There is a line break at
%         the end of the theorem header.
% \end{enumerate}
%
% Additionally, some journals require different formats which depend on
% the ``sort of theorem'': e.g.\ often remarks and definitions are set
% in |\upshape|, while |\itshape| is employed for main theorems.
%
% Confronted with these requirements, a theorem environment was
% developed in Mainz which allows separate determination of the
% layout of the ``theorems sets'', comparable to |\pagestyle|.
%
%
% \section{The user interface}
% \subsection{Defining new theorem sets}
%
% \DescribeMacro\newtheorem
% As in the original \LaTeX{} version, the command |\newtheorem|
% defines a new ``theorem set'' or ``theorem-like structure''.
% Two required arguments name the new environment and give the text
% to be typeset with each instance of the new ``set'', while
% an optional argument determines how
% the ``set'' is enumerated:
% \begin{description}
%    \item[\ttfamily \bslash newtheorem\{foo\}\{bar\}]
%       The theorem set {\envfont foo} (whose name is \texttt{bar})
%       uses its own counter.
%    \item[\ttfamily \bslash newtheorem\{foo2\}{[foo]}\{bar2\}]
%       The theorem set {\envfont foo2} (printed name \texttt{bar2})
%       uses the same counter as the theorem set \texttt{foo}.
%    \item[\ttfamily \bslash newtheorem\{foo3\}\{bar3\}{[section]}]
%       The theorem set {\envfont foo3} (printed name \texttt{bar3}) is
%       enumerated within the counter \texttt{section}, i.e.\ with every
%       new |\section| the enumeration begins again with 1, and
%       the enumeration is composed from the section-number and the
%       theorem counter itself.
% \end{description}
%
% \DescribeMacro\theoremstyle
% Additionally, the command |\theoremstyle| can define the layout
% of various, or all, theorem sets.  It should be noted that any
% theorem set defined by |\newtheorem| is typeset in the
% |\theoremstyle| that is current at the time of the definition.
% Thus, the following
% \begin{verbatim}
%    \theoremstyle{break}        \newtheorem{Cor}{Corollary}
%    \theoremstyle{plain}        \newtheorem{Exa}{Example}[section]
%\end{verbatim}                             ^^A No blank at line start
% leads to the result that the set {\envfont Cor} is formatted in the
% style \texttt{break}, while the set {\envfont Exa} and all the
% following ones are formatted in the style \texttt{plain}, unless
% another |\theoremstyle| follows.  Since the definitions installed by
% |\newtheorem| are global, one also can limit |\theoremstyle| locally
% by grouping braces.
%
% \DescribeMacro\theorembodyfont
% The choice of the font for the theorem body is completely
% independent of the chosen |\theoremstyle|; this has proven to
% be very advantageous.  For example,
% \begin{verbatim}
%   {\theorembodyfont{\upshape}          \newtheorem{Rem}{Remark}}
%\end{verbatim}
% defines a theorem set {\envfont Rem}, which will be set in
% |\upshape| in the current layout (which in our example is
% \texttt{plain}).  As with |\theoremstyle|, the |\theorembodyfont|
% chosen is that current at the time of |\newtheorem|.  If
% |\theorembodyfont| is not specified or one defines
% |\theorembodyfont{}|, then the font used will be that defined by the
% |\theoremstyle|.
%
% \DescribeMacro\theoremheaderfont It is also possible to customize
% the font used for the theorem headers.  This is, however, a global
% declaration, and therefore there should be at most one
% |\theoremheaderfont| declaration in the preamble.\footnote{If it is
% actually necessary to have different header fonts, one has to define
% new theorem styles (substituting the desired font) or specify the
% information directly in the \texttt{\bslash newtheorem} declaration
% (the unclean variant).}
%
% \DescribeMacro\theorempreskipamount
% \DescribeMacro\theorempostskipamount Two additional parameters
% affect the vertical space around the theorem environments:
% |\theorempreskipamount| and |\theorempostskipamount| define,
% respectively, the spacing before and after such an environment.
% These parameters apply for all theorem sets and can be manipulated
% with the ordinary length macros.  They are rubber lengths
% (`\textsf{skips}'), and therefore can contain \texttt{plus} and
% \texttt{minus} parts.
%
% Since the definition of theorem sets should---most sensibly---be
% placed in the preamble, we only allow installation there.
% It is therefore possible to release the memory used here after
% |\begin{document}|, in order to make room for other
% applications.
%
%
% \subsection{Existing theorem styles}
%
% The following theorem styles exist to date:
% \begin{deflist}{marginbreak}
%    \item[plain]
%       This theorem style emulates the original \LaTeX{} definition,
%       except that additionally the parameters
%       |\theorem...skipamount| are used.
%    \item[break]
%       In this style, the theorem header is followed by a line
%       break.
%    \item[marginbreak]
%       The theorem number is set in the margin, and there is a line
%       break as in \texttt{break}.
%    \item[changebreak]
%       Like \texttt{break}, but with header number and text
%       interchanged.
%    \item[change]
%       Header number and text are interchanged, without a line break.
%    \item[margin]
%       The number is set in the left margin, without a line break.
% \end{deflist}
% All styles (except \texttt{plain}) select |\slshape| as the default
% |\theorembodyfont|.
%
%
% \subsection{Examples}
%
% Given the above theorem sets {\envfont Cor}, {\envfont Exa}
% and {\envfont Rem},
% suppose that the preamble also contains the declarations:
% \begin{verbatim}
%    \theoremstyle{marginbreak}   \newtheorem{Lem}[Cor]{Lemma}
%    \theoremstyle{change}
%    \theorembodyfont{\itshape}        \newtheorem{Def}[Cor]{Definition}
%
%    \theoremheaderfont{\scshape}
%\end{verbatim}
% Then the following are some typical examples of the typeset output
% resulting from their use.
%
% \begin{Cor}
%    This is a sentence typeset in the theorem environment {\envfont
%    Cor}.
% \end{Cor}
%
% \begin{Exa}
%    This is a sentence typeset in the theorem environment {\envfont
%    Exa}.
% \end{Exa}
%
% \begin{Rem}
%    This is a sentence typeset in the theorem environment {\envfont
%    Rem}.
% \end{Rem}
%
% \begin{Lem}[Ben User]
%    This is a sentence typeset in the theorem environment {\envfont
%    Lem}.
% \end{Lem}
%
% \begin{Def}[Very impressive Definition]
%    This is a sentence typeset in the theorem environment {\envfont
%    Def}.
% \end{Def}
%
% The last two examples show the effect of the optional argument to
% a theorem environment (it is the text typeset in parentheses).
%
%
% \section{Special Considerations}
% \label{sec:newfonts}
%
% Theoremheader and body are implemented as a unit. This means that
% the |\theoremheaderfont|  will inherit characteristics of the
% |\theorembodyfont| in \LaTeXe. Thus, if for example
% |\theorembodyfont| is |\itshape| and |\theoremheaderfont|
% is |\bfseries| the font selected for the header will have the
% characteristics `bold extended italic'. If this is not desired one
% should set the |\theoremheaderfont| to something like
% \begin{verbatim}
%      \theoremheaderfont{\normalfont\bfseries}
%\end{verbatim}
% i.e.\ supplying all necessary font information explicitly.
%
%
%
% \section{Acknowledgements}
%
% The publication of this set of macros was only possible with the
% help of Christina Busse (translating the manuscript into English),
% Joachim Pense (playing the r\^ole of typist), Chris Rowley
% (looking everything over) and many others providing useful
% suggestions.
%
% \MaybeStop{%
%    \begin{thebibliography}{9}
%       \bibitem{bk:GMS94} \textsc{M.~Goossens}, \textsc{F.~Mittelbach}
%         and \textsc{A.~Samarin}.
%         \newblock The \LaTeX{} Companion.
%         \newblock
%         Addison-Wesley, Reading, Massachusetts, 1994.
%       \bibitem{latex:source} \textsc{Lamport, Leslie.} \newblock
%                               \texttt{latex.tex}, version 2.09, date
%                               Feb. 1990.
%    \end{thebibliography}
%    \PrintIndex
%    \PrintChanges}
%
%
% \section{The documentation driver file}
%
% The next bit of code contains the documentation driver file for
% \TeX{}, i.e., the file that will produce the documentation you are
% currently reading. It will be extracted from this file by the
% \texttt{docstrip} program. Since it is the first code in the file
% one can alternatively process this file directly with \LaTeXe{} to
% obtain the documentation.
%
%    \begin{macrocode}
%<*driver>
\documentclass{ltxdoc}

\usepackage{theorem}

%<+driver>% The next few lines define theorem sets which are used
%<+driver>% in the example section of the documentation.

\theoremstyle{break}       \newtheorem{Cor}{Corollary}
\theoremstyle{plain}       \newtheorem{Exa}{Example}[section]
{\theorembodyfont{\upshape}\newtheorem{Rem}{Remark}}
\theoremstyle{marginbreak} \newtheorem{Lem}[Cor]{Lemma}
\theoremstyle{change}
\theorembodyfont{\itshape} \newtheorem{Def}[Cor]{Definition}

\theoremheaderfont{\scshape}

\RecordChanges

\begin{document}
   \DocInput{theorem.dtx}
\end{document}
%</driver>
%    \end{macrocode}
%
% \section{Definition of the Macros}
%
% If the file has been loaded before, we abort immediately. If not the
% package announces itself (this is actually done at the very top of
% the file---the way it is done isn't good style so don't copy it).
% \changes{v2.0e}{89/07/19}{Spaces removed from `typeout'.}
% \changes{v2.2b}{95/11/19}{Announce moved to top of file}
%    \begin{macrocode}
%<*package>
%\@ifundefined{theorem@style}{}{\endinput}
%\def\FMithmInfo{1995/11/19 v2.2b Theorem extension package (FMi)}
%\ProvidesPackage{theorem}[\FMithmInfo]
%    \end{macrocode}
%
%
% \subsection{Definition of theorem styles and fonts}
%
% All the definitions in this file are done globally to allow
% inputting this file inside a group.
%
% \begin{macro}{\theoremstyle}
%    Before a theorem style can be installed, the chosen style must be
%    known. For that reason, we must test to see that
%    |\th@|\meta{style} is known or, more precisely, that it is
%    different from |\relax|. If the style is not known then
%    |\th@plain| is used.
%    \begin{macrocode}
\gdef\theoremstyle#1{%
   \@ifundefined{th@#1}{\@warning
          {Unknown theoremstyle `#1'. Using `plain'}%
          \theorem@style{plain}}%
%    \end{macrocode}
%    We save the theorem style to be used in the token register
%    |\theorem@style|.
%    \begin{macrocode}
      {\theorem@style{#1}}%
%    \end{macrocode}
%    Now we ``evaluate'' the theorem style: this means, we call the
%    macro |\th@|\meta{style} which will activate the relevant
%    definitions which are contained in a separate file.  This is done
%    in a group to suppress changes to the current font.  This could
%    otherwise pose problems together with the new font selection
%    scheme\footnote{When I printed the original article using the new
%    font selection scheme I ended with a document with slanted
%    typefaces (text headings and all) simply because one of the
%    theorem styles used \texttt{\bslash sl} at toplevel.} if the
%    |\th@|\meta{style} is evaluated a second time.
% \changes{v2.1a}{90/02/15}{Group added around `th@..}
% \changes{v2.1a}{90/02/15}{Reseting of `..@begintheorem removed.}
%    \begin{macrocode}
      \begingroup
        \csname th@\the\theorem@style \endcsname
      \endgroup}
%    \end{macrocode}
% \end{macro}
%
%
% \begin{macro}{\@begintheorem}
% \changes{v2.1a}{90/02/15}{Relaxed at toplevel.}
% \begin{macro}{\@opargbegintheorem}
% \changes{v2.1a}{90/02/15}{Relaxed at toplevel.}
%    We reset |\@begintheorem| and |\@opargbegintheorem| to
%    |\relax| since these commands are no longer necessary at
%    toplevel. This will save a few tokens.
%    \begin{macrocode}
\global\let\@begintheorem\relax
\global\let\@opargbegintheorem\relax
%    \end{macrocode}
% \end{macro}
% \end{macro}
%
% \begin{macro}{\theorem@style}
%    Obviously the \textsf{token} register used above has to be
%    allocated. To assure the utmost compatibility with the
%    original \LaTeX{} definition, we set the default theorem style to
%    \texttt{plain}, which implements the usual \LaTeX{} convention.
% \changes{v2.0f}{89/07/20}{Unnecessary `global' removed.}
%    \begin{macrocode}
\newtoks\theorem@style
\global\theorem@style{plain}
%    \end{macrocode}
% \end{macro}
%
% \begin{macro}{\theorembodyfont}
% \begin{macro}{\theorem@bodyfont}
%    If the |\theorembodyfont| is set by the user then it should not
%    interact with the default font set in the theorem style. When the
%    new font selection is in force this may happen if, for example,
%    the default is |\itshape| and the new |\theorembodyfont| is
%    |\sffamily|. So we add a |\reset@font| command in front of the
%    user definition.
% \changes{v2.0f}{89/07/20}{Unnecessary `global' removed.}
% \changes{v2.0g}{89/09/19}{`theoremfont to `theorembodyfont.}
% \changes{v2.1c}{92/06/22}{`reset@font added and logic changed.}
% \changes{v2.1d}{92/10/28}{Reset to empty if arg is empty}
%    \begin{macrocode}
\gdef\theorembodyfont#1{%
%    \end{macrocode}
%    We check if the argument supplied is empty and if so put nothing
%    into the |\theorem@bodyfont| token register to allow for
%    |\theorembodyfont{}| as a mean of using the default of the
%    current |\theoremstyle|.
%    \begin{macrocode}
   \def\@tempa{#1}%
   \ifx\@tempa\@empty
    \theorem@bodyfont{}%
   \else
    \theorem@bodyfont{\reset@font#1}%
   \fi
   }
\newtoks\theorem@bodyfont
\global\theorem@bodyfont{}
%    \end{macrocode}
% \end{macro}
% \end{macro}
%
%
% \begin{macro}{\theoremheaderfont}
% \changes{v2.0g}{89/09/19}{Macro added.}
%    The font for the theorem headers is handled differently because
%    this definition applies to all theorem styles.
%    \begin{macrocode}
\gdef\theoremheaderfont#1{\gdef\theorem@headerfont{#1}%
%    \end{macrocode}
%    After using the macro once it is redefined to produce an error
%    message.
%    \begin{macrocode}
       \gdef\theoremheaderfont##1{%
        \typeout{\string\theoremheaderfont\space should be used
                 only once.}}}
%    \end{macrocode}
% \end{macro}
%
% \begin{macro}{\theorem@headerfont}
% \changes{v2.0g}{89/09/19}{Macro added.}
%    To set the |\theorem@headerfont| default we first test if
%    the new fontselection scheme is in force.
% \changes{v2.1a}{90/02/15}{Support for new font selection.}
%    \begin{macrocode}
\ifx\upshape\undefined
%    \end{macrocode}
%    If not we define it to expand into |\bfseries|. We don't use
%    |\let| just in case a following style option redefines this
%    macro.
%    \begin{macrocode}
\gdef\theorem@headerfont{\bfseries}
%    \end{macrocode}
%    Otherwise we reset the current shape before calling |\bfseries|.
%    \begin{macrocode}
\else \gdef\theorem@headerfont{\normalfont\bfseries}\fi
%    \end{macrocode}
% \end{macro}
%
% \begin{macro}{\th@plain}
% \begin{macro}{\th@break}
% \begin{macro}{\th@marginbreak}
% \begin{macro}{\th@changebreak}
% \begin{macro}{\th@change}
% \begin{macro}{\th@margin}
%    The different styles are defined in macros such as
%    |\th@plain|.  Since memory space is precious in
%    ``non-Big-versions'', we have to avoid offering too many unused
%    definitions.  Therefore we define these styles in separate files
%    that can be loaded on demand.  Thus the commands themselves only
%    load these files. We use |\@input@|, a \LaTeXe{} internal command
%    that ensures that the file will be listed with |\listfiles|.
% \changes{v2.2a}{94/02/02}{Use `@input@ to load theorem layout files}
%    \begin{macrocode}
\gdef\th@plain{\@input@{thp.sty}}
\gdef\th@break{\@input@{thb.sty}}
\gdef\th@marginbreak{\@input@{thmb.sty}}
\gdef\th@changebreak{\@input@{thcb.sty}}
\gdef\th@change{\@input@{thc.sty}}
\gdef\th@margin{\@input@{thm.sty}}
%    \end{macrocode}
%    This list will be expanded when new styles become
%    available.  For testing,  just append new theorem substyles
%    as document options.
% \end{macro}
% \end{macro}
% \end{macro}
% \end{macro}
% \end{macro}
% \end{macro}
%
%
%
% \subsection{Definition of a new theorem set}
%
% As already pointed out, a new theorem environment can be defined in
% three different ways:
% \begin{verbatim}
%    \newtheorem{Lem}{Lemma}
%    \newtheorem{Lem}{Lemma}[section]
%    \newtheorem{Lem}[Theorem]{Lemma}
%\end{verbatim}
% The function of the macro |\newtheorem| is to recognize these cases
% and then to branch into one of the three macros |\@ynthm|, |\@xnthm|
% or |\@othm|.  This mechanism is adopted unchanged from
% \cite{latex:source}; the essential point here is that, for example,
% in the second case, the arguments \texttt{Lem}, \texttt{Lemma} and
% \texttt{section} are passed over to the macro |\@xnthm|.
%
% We inspect this case first because the others present fewer problems,
% and thus are easily derived from this one.
%
% \begin{macro}{\@xnthm}
%    For our example arguments, the macro |\@xnthm| must fulfill
%    the following:
%    \begin{itemize}
%       \item
%          Define a new \LaTeX-counter `Lem'
%       \item
%          reset this counter within a |\section|
%       \item
%          define the macro |\theLem|
%       \item
%          define the environment macros |\Lem| and
%          |\endLem| using the current
%          |\theoremstyle| and |\theorem@bodyfont|.
%    \end{itemize}
%    Obviously, all this should happen only if the first argument
%    of |\@xnthm| (i.e.\ \texttt{Lem} in our example) is chosen so
%    as not to conflict with any previously defined commands or
%    environments.  This test is performed by the \LaTeX{} macro
%    |\@ifdefinable|.
%    \begin{macrocode}
\gdef\@xnthm#1#2[#3]{\expandafter\@ifdefinable\csname #1\endcsname
%    \end{macrocode}
%    Therefore, the first argument of |\@ifdefinable| is the
%    expansion (in the example, |\Lem|) of
%    |\csname||#1\endcsname|.  The second argument is
%    executed only if the test has been completed successfully.
%    \begin{macrocode}
   {%
%    \end{macrocode}
%    Now we define the new counter. The names of the \LaTeX{}
%    macros employed should speak for themselves:
% \changes{v2.2b}{1995/10/07}{Use \cs{@newctr}}
% \changes{v2.2c}{1995/11/23}{Use \cs{@newctr} properly.}
%    \begin{macrocode}
    \@definecounter{#1}\@newctr{#1}[#3]%
%    \end{macrocode}
%    Using |\@newctr| will give a proper error message if the counter
%    in |#3| is not defined.
%    In defining `|\theLem|' we must generate the desired macro
%    name by use of |\expandafter| and |\csname|.
%    \begin{macrocode}
    \expandafter\xdef\csname the#1\endcsname
%    \end{macrocode}
%    An |\xdef| is used in order to make the definition global,
%    and to ensure that it contains the replacement texts of
%    |\@thmcountersep| and |\@thmcounter|.\footnote{These
%    two macros can be defined by the document style.  Their default
%    values produce a `\texttt{.}' as separation and an arabic
%    representation of the number.} However, not everything should be
%    expanded. For example, it saves space to use |\thesection|
%    instead of its---at times---lengthy expansion.
%    \begin{macrocode}
      {\expandafter \noexpand \csname the#3\endcsname
       \@thmcountersep \@thmcounter{#1}}%
%    \end{macrocode}
%    Thus with the defaults of \LaTeX{}, |\theLem| would be
%    replaced by the command sequence |\thesection.\arabic{Lem}|.
%
%    We will now look at the definition of the macro which is executed
%    at the beginning of the actual environment (in our example this
%    macro is |\Lem|). It should be noted that we use an
%    ``|\expandafter| trick'' to expand only certain parts of the
%    replacement text at the time of the definition.
%    \begin{macrocode}
    \def\@tempa{\global\@namedef{#1}}%
    \expandafter \@tempa \expandafter{%
%    \end{macrocode}
%    First, the macro that contains the current definitions of
%    |\@begintheorem| and |\@opargtheorem| should be called
%    up.  The name of this macro---as is already known---has the form
%    |\th@|\meta{theorem\ style}; therefore, it must be called by
%    \begin{macrocode}
      \csname th@\the \theorem@style
%    \end{macrocode}
%    In addition the default theorem font should be changeable, i.e.\
%    we have to insert the contents of |\theorem@bodyfont|. For
%    that reason, we expand even further, beyond |\endcsname|,
%    and thus insert the contents of the \textsf{token} register
%    |\theorem@bodyfont| in the replacement text.
%    \begin{macrocode}
            \expandafter \endcsname \the \theorem@bodyfont
%    \end{macrocode}
%    Now it is time to call the macro |\@thm| which takes over
%    the further processing.  It has two arguments: the current
%    counter name (in our example, \texttt{Lem}), and the text of the
%    label (in our example, \texttt{Lemma}).
%    \begin{macrocode}
     \@thm{#1}{#2}}%
%    \end{macrocode}
%    With this, the `sub-definition' is complete.  The macro
%    |\@endtheorem| ends a theorem environment and is, so far, nothing
%    but an |\endtrivlist|.  (Hence it is defined globally, and not
%    within the theorem styles.\footnote{This has to be changed as
%    soon as theorem styles that change \texttt{\bslash @endtheorem}
%    exist.  In such a case, all existing styles must be changed as
%    well since they will have to reset the macro.}) Therefore, we can
%    set it equivalent to the macro that ends the theorem set (in our
%    example, |\endLem|).  However, if some day theorem styles exist
%    that do change |\@endtheorem|, we would have to use the
%    commented-out line instead.
%    \begin{macrocode}
    \global \expandafter \let \csname end#1\endcsname \@endtheorem
%   \global\@namedef{end#1}{\@endtheorem}%
%    \end{macrocode}
%    With these commands all the required definitions are employed,
%    unless the test |\@ifdefinable| has failed.  Therefore, we
%    end the second argument of this macro and with it the definition
%    of |\@xnthm|.
%    \begin{macrocode}
   }}
%    \end{macrocode}
% \end{macro}
%
% \begin{macro}{\@ynthm}
%    The definition of |\@ynthm| is completely analogous. In
%    this case the new counter that is defined is not reset within
%    another counter; thus the definition of |\the...|
%    is simplified:
%    \begin{macrocode}
\gdef\@ynthm#1#2{\expandafter\@ifdefinable\csname #1\endcsname
   {\@definecounter{#1}%
    \expandafter\xdef\csname the#1\endcsname{\@thmcounter{#1}}%
%    \end{macrocode}
%    The rest of the definition corresponds literally to that of
%    |\@xnthm|:
%    \begin{macrocode}
    \def\@tempa{\global\@namedef{#1}}\expandafter \@tempa
     \expandafter{\csname th@\the \theorem@style \expandafter
     \endcsname \the\theorem@bodyfont \@thm{#1}{#2}}%
    \global \expandafter \let \csname end#1\endcsname \@endtheorem}}
%    \end{macrocode}
% \end{macro}
%
% \begin{macro}{\@othm}
%    The definition of |\@othm| does not contain anything new.
%    \begin{macrocode}
\gdef\@othm#1[#2]#3{%
%    \end{macrocode}
% \changes{v2.2b}{95/10/07}{Build in check for counter defined}
%    We do not define a new counter but instead use one that has
%    already been defined.  Thus the only definition we need is that
%    of this pseudo-counter (i.e.\ |\the|\meta{env.\ name}).
%    First we check if |#2| corresponds to a known counter name.
%    \begin{macrocode}
  \expandafter\ifx\csname c@#2\endcsname\relax
   \@nocounterr{#2}%
  \else
   \expandafter\@ifdefinable\csname #1\endcsname
   {\expandafter \xdef \csname the#1\endcsname
     {\expandafter \noexpand \csname the#2\endcsname}%
%    \end{macrocode}
%    All other parts of the definition can be adopted from
%    |\@xnthm|. We have to remember, though, that in this case
%    the name of the current counter and the theorem label have moved
%    to the second and third arguments.
%    \begin{macrocode}
    \def\@tempa{\global\@namedef{#1}}\expandafter \@tempa
     \expandafter{\csname th@\the \theorem@style \expandafter
     \endcsname \the\theorem@bodyfont \@thm{#2}{#3}}%
    \global \expandafter \let \csname end#1\endcsname \@endtheorem}%
  \fi}
%    \end{macrocode}
% \end{macro}
%
% \subsection{Macros that are employed in a {\envfont theorem}
%    environment}
%
% \begin{macro}{\@thm}
%    The macro |\@thm| has to increase the current counter.  Then,
%    depending on whether the environment has (or does not have) an
%    optional argument, it has to branch into either |\@begintheorem|
%    or |\@opargtheorem|.
%    \begin{macrocode}
\gdef\@thm#1#2{\refstepcounter{#1}%
%    \end{macrocode}
%    Now we start a {\envfont trivlist} environment, and give
%    |\@topsep| and |\@topsepadd| the values of the \textsf{skip}
%    registers |\theorempreskipamount| and |\theorempostskipamount|.
%    The value in |\@topsep| is the vertical space that is inserted by
%    the first (and only) |\item| in our |\trivlist| whilst
%    |\@topsepadd| is inserted by |\@endparenv| at the end of that
%    {\envfont trivlist} environment.  By using these registers, we
%    obtain the desired space around a {\envfont theorem} environment.
%    \begin{macrocode}
   \trivlist
   \@topsep \theorempreskipamount               % used by first \item
   \@topsepadd \theorempostskipamount           % used by \@endparenv
%    \end{macrocode}
%    Now we have to test whether an optional argument has been given.
% \changes{v2.0d}{89/07/18}{Forgotten percent added.}
%    \begin{macrocode}
   \@ifnextchar [%
%    \end{macrocode}
%    If there is an optional argument, we will call |\@ythm|, and
%    move the arguments read back into the input stream.
%    \begin{macrocode}
   {\@ythm{#1}{#2}}%
%    \end{macrocode}
%    If not, we call |\@begintheorem|. Its first argument is the
%    name of the theorem set (hence the second argument of
%    |\@thm|). Its second argument is the macro that produces the
%    current number.
%    \begin{macrocode}
   {\@begintheorem{#2}{\csname the#1\endcsname}\ignorespaces}}
%    \end{macrocode}
% \end{macro}
%
% \begin{macro}{\@xthm}
% \begin{macro}{\@ythm}
%    Both these macros were originally called by |\@thm|.  We do
%    not need |\@xthm| anymore, hence we reset it to
%    |\relax|. The definition of |\@ythm| has not changed at
%    all from its definition in \LaTeX. In order to make the macros
%    easier to understand, we will
%    nevertheless present it (commented out).
%    \begin{macrocode}
\global\let\@xthm\relax
% \def\@ythm#1#2[#3]{\@opargbegintheorem{#2}{\csname
%           the#1\endcsname}{#3}\ignorespaces}
%</package>
%    \end{macrocode}
%    The primitive |\ignorespaces| in |\@ythm| and
%    |\@thm| is needed to remove the spaces between the
%    |\begin{...}| and the actual text.
% \end{macro}
% \end{macro}
%
%
% \subsection{Definition of the theorem substyles}
%
% As already pointed out, the theorem substyles, defined below, are
% only loaded when necessary.  Note that all these substyles, except
% \texttt{plain}, have |\slshape| as the default body font.
%
% \subsubsection{The \texttt{plain} style}
%
% \label{dok:th@plain} As the following macros use |@|, we have
% to locally set the |\catcode| of this symbol to ``letter''.
% This happens within a group, so that we do not have to worry about
% which |\catcode| that symbol had before.
%    \begin{macrocode}
%<*thp>
\begingroup \makeatletter
%    \end{macrocode}
% Since we are now within a group, we must make all definitions
% globally.  First we make sure that \texttt{theorem.sty} is loaded.
% This will allow us to use this file as a document style option
% without having to call \texttt{theorem} itself as an option.  At the
% same time, we assure that at least version 2 is loaded, since
% |\theorem@style| was not defined in earlier versions.
%    \begin{macrocode}
\@ifundefined{theorem@style}{\input{theorem.sty}}{}
\ProvidesFile{thp.sty}
             [\FMithmInfo]
%    \end{macrocode}
%
%
% \begin{macro}{\th@plain}
% \begin{macro}{\@begintheorem}
% \begin{macro}{\@opargbegintheorem}
%    |\theoremstyle{plain}| corresponds to the original
%    definition, except that the distances to the surrounding text are
%    determined by the parameters |\theorempreskipamount| and
%    |\theorempostskipamount|.
%    First we set the default body font.
%    \begin{macrocode}
\gdef\th@plain{\normalfont\itshape
%    \end{macrocode}
%    Then we define |\@begintheorem| and
%    |\@opargbegintheorem|. These two macros define how the
%    header of a theorem is typeset.  |\@opargbegintheorem| will
%    be called if a \textsf{theorem} environment with an optional
%    argument is encountered; otherwise, the header is constructed by
%    calling |\@begintheorem|.  If one of these macros is
%    executed, we are within a \textsf{trivlist} environment started by
%    |\@thm|.  So the theorem header is produced with an
%    |\item| command.
%
%    Instead of specifying the header font directly, all standard
%    theorem styles use the |\theorem@headerfont| macro to allow
%    customization.  The extra space (|\labelsep|) is necessary
%    because of problems in the \textsf{trivlist} environment.
%    \begin{macrocode}
  \def\@begintheorem##1##2{%
        \item[\hskip\labelsep \theorem@headerfont ##1\ ##2]}%
%    \end{macrocode}
%    The definition of |\@opargbegintheorem| is completely
%    analogous. The only difference is the fact that there exists a
%    third argument (which is the optional parameter of the
%    environment and contains additional information about the
%    theorem).  Customarily we enclose it in parentheses.
%    \begin{macrocode}
\def\@opargbegintheorem##1##2##3{%
   \item[\hskip\labelsep \theorem@headerfont ##1\ ##2\ (##3)]}}
%    \end{macrocode}
% \end{macro}
% \end{macro}
% \end{macro}
%
% \noindent
% We conclude with an |\endgroup| to restore the
% |\catcode| of |@|.
%    \begin{macrocode}
\endgroup
%</thp>
%    \end{macrocode}
%
%
% \subsubsection{The \texttt{break} style}
%
% This style option is stored in the file |thb.sty|.
% For the next two lines see the documentation for |\th@plain| on
% page~\pageref{dok:th@plain}.
%    \begin{macrocode}
%<*thb>
\begingroup \makeatletter
\@ifundefined{theorem@style}{\input{theorem.sty}}{}
\ProvidesFile{thb.sty}
             [\FMithmInfo]
%    \end{macrocode}
%
% \begin{macro}{\th@break}
%    |\theoremstyle{break}| produces a line break after the name
%    of the theorem.  The font is |\slshape|.  Hence, we define
%    |\th@break| as follows:
%    \begin{macrocode}
\gdef\th@break{\normalfont\slshape
  \def\@begintheorem##1##2{\item[%
%    \end{macrocode}
%    We run into the following problem: it is not possible to create
%    the header with |\item[|\meta{title}|]| and then start
%    a new line by, for example, |\mbox{}\\|. Such a definition
%    will fail whenever a list environment follows immediately.  With
%    the above construction, the |\mbox{}| causes the switch
%    |@inlabel| (cf.\ definition of |\list| and
%    |\trivlist| in \cite{latex:source}) to be set to \texttt{false}
%    and so the following list will insert additional vertical
%    space (|\topskip|). This is quite annoying. Therefore, we
%    create the line break within the |\item|. In order to ensure
%    that the text will begin at the proper position in the following
%    line, we simply pretend that the label does not take any
%    room.\footnote{This will lead to problems whenever very high
%    symbols occurring in the line tower into the
%    heading. So, something else has to be done here sometime.}
%    \begin{macrocode}
       \rlap{\vbox{\hbox{\hskip \labelsep\theorem@headerfont ##1\ ##2}%
                   \hbox{\strut}}}]}%
%    \end{macrocode}
%    Again, the definition of |\@opargbegintheorem| is
%    completely analogous.
%    \begin{macrocode}
\def\@opargbegintheorem##1##2##3{%
  \item[\rlap{\vbox{\hbox{\hskip \labelsep \theorem@headerfont
                     ##1\ ##2\ (##3)}%
                    \hbox{\strut}}}]}}
\endgroup
%</thb>
%    \end{macrocode}
% \end{macro}
%
%
% \subsubsection{The \texttt{changebreak} style}
%
%    \begin{macrocode}
% This style option is stored in the file |thcb.sty|.
%    \begin{macrocode}
%<*thcb>
\begingroup \makeatletter
\@ifundefined{theorem@style}{\input{theorem.sty}}{}
\ProvidesFile{thcb.sty}
             [\FMithmInfo]
%    \end{macrocode}
%
% \begin{macro}{\th@changebreak}
%    The \texttt{change-break} theorem style is like \texttt{break}
%    but with interchange of theorem name and theorem number.  Thus we
%    define |\th@changebreak| as follows:
%    \begin{macrocode}
\gdef\th@changebreak{\normalfont\slshape
  \def\@begintheorem##1##2{\item
       [\rlap{\vbox{\hbox{\hskip\labelsep \theorem@headerfont ##2\ ##1}%
                    \hbox{\strut}}}]}%
\def\@opargbegintheorem##1##2##3{%
  \item[\rlap{\vbox{\hbox{\hskip\labelsep \theorem@headerfont
                     ##2\ ##1\ (##3)}%
                    \hbox{\strut}}}]}}
\endgroup
%</thcb>
%    \end{macrocode}
% \end{macro}
%
%
% \subsubsection{The \texttt{change} style}
%
% This style option is stored in the file |thc.sty|.
%    \begin{macrocode}
%<*thc>
\begingroup \makeatletter
\@ifundefined{theorem@style}{\input{theorem.sty}}{}
\ProvidesFile{thc.sty}
             [\FMithmInfo]
%    \end{macrocode}
%
% \begin{macro}{\th@change}
%    The \texttt{change} theorem style corresponds to the
%    \texttt{change break} style without a linebreak after the header.
%    To say it in another way, it's the same as the \texttt{plain}
%    style but with number and name interchanged and |\slshape| as the
%    default font.
%    \begin{macrocode}
\gdef\th@change{\normalfont\slshape
  \def\@begintheorem##1##2{\item
       [\hskip\labelsep \theorem@headerfont ##2\ ##1]}%
\def\@opargbegintheorem##1##2##3{%
  \item[\hskip\labelsep \theorem@headerfont ##2\ ##1\ (##3)]}}
\endgroup
%</thc>
%    \end{macrocode}
% \end{macro}
%
%
% \subsubsection{The \texttt{marginbreak} style}
%
% This style option is the one used most often at Mainz.  It is saved
% in the file |thmb.sty|.
%    \begin{macrocode}
%<*thmb>
\begingroup \makeatletter
\@ifundefined{theorem@style}{\input{theorem.sty}}{}
\ProvidesFile{thmb.sty}
             [\FMithmInfo]
%    \end{macrocode}
%
% \begin{macro}{\th@marginbreak}
%    The \texttt{margin break} style is nearly the same as the
%    \texttt{change break} style. The only difference is the placement
%    of the theorem number.  We use |\llap| to place it in the left
%    margin.
%
%    In this style |\labelsep| denotes the separation between the
%    number and the text.
%    \begin{macrocode}
\gdef\th@marginbreak{\normalfont\slshape
  \def\@begintheorem##1##2{\item
       [\rlap{\vbox{\theorem@headerfont
                    \hbox{\llap{##2}\hskip\labelsep ##1}%
                    \hbox{\strut}}}]}%
\def\@opargbegintheorem##1##2##3{%
  \item[\rlap{\vbox{\theorem@headerfont
                    \hbox{\llap{##2}\hskip\labelsep ##1\ (##3)}%
                    \hbox{\strut}}}]}}
\endgroup
%</thmb>
%    \end{macrocode}
% \changes{v2.1c}{90/02/17}{`endgroup outside macrocode env.}
% \end{macro}
%
%
%
% \subsubsection{The \texttt{margin} style}
%
% This style option is stored in the file |thm.sty|.
%    \begin{macrocode}
%<*thm>
\begingroup \makeatletter
\@ifundefined{theorem@style}{\input{theorem.sty}}{}
\ProvidesFile{thm.sty}
             [\FMithmInfo]
%    \end{macrocode}
%
% \begin{macro}{\th@margin}
%    Again this is only a variant of the theorem styles described
%    above without any new ideas.
%    \begin{macrocode}
\gdef\th@margin{\normalfont\slshape
  \def\@begintheorem##1##2{\item
       [\theorem@headerfont \llap{##2}\hskip\labelsep ##1]}%
\def\@opargbegintheorem##1##2##3{%
  \item[\theorem@headerfont \llap{##2}\hskip\labelsep ##1\ (##3)]}}
\endgroup
%</thm>
%    \end{macrocode}
% \end{macro}
%
%
%
%
% \subsection{Final Definitions}
%
% \begin{macro}{\theorempreskipamount}
% \begin{macro}{\theorempostskipamount}
%    The \textsf{skip} parameters that regulate the vertical empty space
%    before and after the {\envfont theorem} environment have
%    to be allocated as well.
% \changes{v2.0f}{89/07/20}{Unnecessary `global' removed.}
%    \begin{macrocode}
%<*package>
\newskip\theorempreskipamount
\newskip\theorempostskipamount
%    \end{macrocode}
%    Since we have used the same values for all theorem sets, we now can
%    assign them.
%    \begin{macrocode}
\global\setlength\theorempreskipamount{12pt plus 5pt minus 3pt}
\global\setlength\theorempostskipamount{8pt plus 3pt minus 1.5pt}
%    \end{macrocode}
% \end{macro}
% \end{macro}
%
%
% \begin{macro}{\@endtheorem}
%    The same holds for the macro |\@endtheorem|, which ends a
%    {\envfont theorem} environment. Since it is the same for all
%    theorem sets, it is removed from the macros |\th@|\meta{style}.
%    It simply ends the {\envfont trivlist} environment, which was
%    begun in |\@thm|.
%    \begin{macrocode}
\global\let\@endtheorem=\endtrivlist
%    \end{macrocode}
% \end{macro}
%
%
%
% \begin{macro}{\@preamblecmds}
%    All macros defined above are to be used only in the preamble.
%    Therefore, we insert them in |\@preamblecmds| which will disable
%    them at begin document. This is done by the internal \LaTeXe{}
%    command |\@onlypreamble|.
% \changes{v2.0g}{89/09/19}{`theoremheaderfont and `theorembodyfont
%    added.}
% \changes{v2.2a}{94/02/03}{Use \LaTeXe{} interface}
%    \begin{macrocode}
\@onlypreamble\@xnthm
\@onlypreamble\@ynthm
\@onlypreamble\@othm
\@onlypreamble\newtheorem
\@onlypreamble\theoremstyle
\@onlypreamble\theorembodyfont
\@onlypreamble\theoremheaderfont
%    \end{macrocode}
% \end{macro}
%
% Finally we declare the \texttt{plain} theorem style to be the default.
% \changes{v2.1b}{90/06/04}{Fixed missing `item by declaring default
%    style.}
%    \begin{macrocode}
\theoremstyle{plain}
%</package>
%    \end{macrocode}
%
% \Finale

\endinput
}{}
\ProvidesFile{thb.sty}
             [\FMithmInfo]
%    \end{macrocode}
%
% \begin{macro}{\th@break}
%    |\theoremstyle{break}| produces a line break after the name
%    of the theorem.  The font is |\slshape|.  Hence, we define
%    |\th@break| as follows:
%    \begin{macrocode}
\gdef\th@break{\normalfont\slshape
  \def\@begintheorem##1##2{\item[%
%    \end{macrocode}
%    We run into the following problem: it is not possible to create
%    the header with |\item[|\meta{title}|]| and then start
%    a new line by, for example, |\mbox{}\\|. Such a definition
%    will fail whenever a list environment follows immediately.  With
%    the above construction, the |\mbox{}| causes the switch
%    |@inlabel| (cf.\ definition of |\list| and
%    |\trivlist| in \cite{latex:source}) to be set to \texttt{false}
%    and so the following list will insert additional vertical
%    space (|\topskip|). This is quite annoying. Therefore, we
%    create the line break within the |\item|. In order to ensure
%    that the text will begin at the proper position in the following
%    line, we simply pretend that the label does not take any
%    room.\footnote{This will lead to problems whenever very high
%    symbols occurring in the line tower into the
%    heading. So, something else has to be done here sometime.}
%    \begin{macrocode}
       \rlap{\vbox{\hbox{\hskip \labelsep\theorem@headerfont ##1\ ##2}%
                   \hbox{\strut}}}]}%
%    \end{macrocode}
%    Again, the definition of |\@opargbegintheorem| is
%    completely analogous.
%    \begin{macrocode}
\def\@opargbegintheorem##1##2##3{%
  \item[\rlap{\vbox{\hbox{\hskip \labelsep \theorem@headerfont
                     ##1\ ##2\ (##3)}%
                    \hbox{\strut}}}]}}
\endgroup
%</thb>
%    \end{macrocode}
% \end{macro}
%
%
% \subsubsection{The \texttt{changebreak} style}
%
%    \begin{macrocode}
% This style option is stored in the file |thcb.sty|.
%    \begin{macrocode}
%<*thcb>
\begingroup \makeatletter
\@ifundefined{theorem@style}{% \iffalse meta-comment
%
% Copyright (C) 1993-2024
%
% The LaTeX Project and any individual authors listed elsewhere
% in this file.
%
% This file is part of the Standard LaTeX `Tools Bundle'.
% -------------------------------------------------------
%
% It may be distributed and/or modified under the
% conditions of the LaTeX Project Public License, either version 1.3c
% of this license or (at your option) any later version.
% The latest version of this license is in
%    https://www.latex-project.org/lppl.txt
% and version 1.3c or later is part of all distributions of LaTeX
% version 2005/12/01 or later.
%
% The list of all files belonging to the LaTeX `Tools Bundle' is
% given in the file `manifest.txt'.
%
% \fi
%
%%
%
%
\def\FMithmInfo{2023/07/05 v2.2c Theorem extension package (FMi)}
%
% \ProvidesFile{theorem.dtx}[\FMithmInfo]
%\iffalse   % this is a METACOMMENT !
%%
%% Package `theorem' to use with LaTeX2e
%% Copyright (C) 1989-1995 Frank Mittelbach, all rights reserved.
%<+package>\@ifundefined{theorem@style}{}{\endinput}
%<+package>\NeedsTeXFormat{LaTeX2e}[1995/06/01]
%<+package>\ProvidesPackage{theorem}
%<+package>         [\FMithmInfo]
%
% \fi
%
% \changes{v1.0}{87/05/07}{First draft.}
% \changes{v1.0a}{87/05/07}{`thistheoremstyle' added.}
% \changes{v1.1a}{87/07/29}{New concept: redefine the internal
%                          \protect\LaTeX{} commands `@xnthm' etc.}
% \changes{v1.1b}{87/07/31}{`theoremfont' used to specify font.}
% \changes{v1.2a}{88/01/08}{`@endtheorem' now globally defined.}
% \changes{v1.2b}{88/01/11}{`newtheorem' only allowed in the preamble.}
% \changes{v1.3a}{88/02/14}{`theoremfont' now a toks register.}
%
% \changes{v2.0a}{88/11/16}{New implementation and new documentation.}
% \changes{v2.0a}{88/11/16}{No layout changes but internal modifications
%                           old substyles won't work.}
% \changes{v2.0b}{89/01/30}{`global' used for all definitions.}
% \changes{v2.0c}{89/06/15}{Documentation translated to English.
%                            Thanks to LLL and Nina}
% \changes{v2.1a}{90/02/15}{Support for new font selection.}
% \changes{v2.2a}{94/02/03}{Cleanup for LaTeX2e}
%
% \DoNotIndex{\@endparenv,\@ifdefinable,\@ifnextchar,\@ifundefined}
% \DoNotIndex{\@namedef,\@nameuse,\@spaces,\@tempa,\@warning}
% \DoNotIndex{\ }
% \DoNotIndex{\begingroup,\bfseries}
% \DoNotIndex{\csname}
% \DoNotIndex{\def,\do,\docdate}
% \DoNotIndex{\endcsname,\endgroup,\endinput,\endtrivlist,\expandafter}
% \DoNotIndex{\filedate,\fileversion}
% \DoNotIndex{\gdef,\global}
% \DoNotIndex{\hbox,\hskip}
% \DoNotIndex{\ignorespaces,\input,\itshape,\item}
% \DoNotIndex{\labelsep,\let,\llap}
% \DoNotIndex{\makeatletter}
% \DoNotIndex{\noexpand}
% \DoNotIndex{\refstepcounter,\relax,\rlap}
% \DoNotIndex{\setlength,\slshape,\space,\string,\strut}
% \DoNotIndex{\the,\trivlist,\typeout}
% \DoNotIndex{\vbox}
% \DoNotIndex{\xdef}
%
% \def\deflabel#1{\ttfamily #1\hfill}
% \def\deflist#1{\begin{list}{}{\settowidth\labelwidth{\ttfamily #1}%
%                               \setlength\leftmargin\labelwidth
%                               \addtolength\leftmargin\labelsep
%                               \let\makelabel\deflabel}}
% \def\enddeflist{\end{list}}
% \def\envfont{\normalfont\ttfamily}
%
% \MakeShortVerb{\|}
%
% \newcommand{\TUB}{{\sl TUGboat\/}}
%
% \GetFileInfo{theorem.dtx}
% \title{An Extension of the
%        \LaTeX{} theorem environment\thanks{This file
%        has version number \fileversion, last
%        revised \filedate.}}
% \author{Frank Mittelbach}
%
% \MaintainedByLaTeXTeam{tools}
% \maketitle
%
% \begin{abstract}
%    The macros described in this paper yield an extension of the
%    \LaTeX{} theorem mechanism. It is designed to satisfy the
%    different requirements of various journals. Thus, the
%    layout of the ``theorems'' can be manipulated by determining a
%    ``style''. This article describes not only the use, but
%    also the definition, of the necessary macros.
% \end{abstract}
%
% \section*{Preface to version 2.2}
%
% For \LaTeXe{} this package did not need any fundamental changes. I
% only modified the messages generated so that theorem layout styles
% will show up with the |\listfiles| command and cleaned the section
% on the New Font Selection Scheme since this is now included in
% \LaTeX.
%
% \section*{Preface to version 2.1}
%
% This version is identical to 2.0g described in \TUB\ 10\#3 except
% for some internal defaults which are now set depending on the used
% font selection scheme.
%
% This was done to avoid unpleasant surprises if the new font selection
% scheme is in force. For further details see
% section~\ref{sec:newfonts} and \cite{bk:GMS94}.
%
% \section{Introduction}
%
% For our purposes here, ``theorems'' are labelled enunciations,
% often set off from the main text by extra space and a font change.
% Theorems, corollaries, conjectures, definitions, and remarks
% are all instances of ``theorems''.  The ``header'' of these structures
% is composed of a label (such as \textsc{Theorem} or \textsc{Remark})
% and a number which serializes an item in the sequence of items
% with the same label.
%
% Shortly after the introduction of \LaTeX{} at the Fachbereich
% Mathematik in Mainz, the desire to manipulate the layout of
% ``theorems'' arose.  In Mainz, the following two conventions came
% into general use:
% \begin{enumerate}
%   \item The number of the theorem is shown in the margin.
%   \item There is a line break at
%         the end of the theorem header.
% \end{enumerate}
%
% Additionally, some journals require different formats which depend on
% the ``sort of theorem'': e.g.\ often remarks and definitions are set
% in |\upshape|, while |\itshape| is employed for main theorems.
%
% Confronted with these requirements, a theorem environment was
% developed in Mainz which allows separate determination of the
% layout of the ``theorems sets'', comparable to |\pagestyle|.
%
%
% \section{The user interface}
% \subsection{Defining new theorem sets}
%
% \DescribeMacro\newtheorem
% As in the original \LaTeX{} version, the command |\newtheorem|
% defines a new ``theorem set'' or ``theorem-like structure''.
% Two required arguments name the new environment and give the text
% to be typeset with each instance of the new ``set'', while
% an optional argument determines how
% the ``set'' is enumerated:
% \begin{description}
%    \item[\ttfamily \bslash newtheorem\{foo\}\{bar\}]
%       The theorem set {\envfont foo} (whose name is \texttt{bar})
%       uses its own counter.
%    \item[\ttfamily \bslash newtheorem\{foo2\}{[foo]}\{bar2\}]
%       The theorem set {\envfont foo2} (printed name \texttt{bar2})
%       uses the same counter as the theorem set \texttt{foo}.
%    \item[\ttfamily \bslash newtheorem\{foo3\}\{bar3\}{[section]}]
%       The theorem set {\envfont foo3} (printed name \texttt{bar3}) is
%       enumerated within the counter \texttt{section}, i.e.\ with every
%       new |\section| the enumeration begins again with 1, and
%       the enumeration is composed from the section-number and the
%       theorem counter itself.
% \end{description}
%
% \DescribeMacro\theoremstyle
% Additionally, the command |\theoremstyle| can define the layout
% of various, or all, theorem sets.  It should be noted that any
% theorem set defined by |\newtheorem| is typeset in the
% |\theoremstyle| that is current at the time of the definition.
% Thus, the following
% \begin{verbatim}
%    \theoremstyle{break}        \newtheorem{Cor}{Corollary}
%    \theoremstyle{plain}        \newtheorem{Exa}{Example}[section]
%\end{verbatim}                             ^^A No blank at line start
% leads to the result that the set {\envfont Cor} is formatted in the
% style \texttt{break}, while the set {\envfont Exa} and all the
% following ones are formatted in the style \texttt{plain}, unless
% another |\theoremstyle| follows.  Since the definitions installed by
% |\newtheorem| are global, one also can limit |\theoremstyle| locally
% by grouping braces.
%
% \DescribeMacro\theorembodyfont
% The choice of the font for the theorem body is completely
% independent of the chosen |\theoremstyle|; this has proven to
% be very advantageous.  For example,
% \begin{verbatim}
%   {\theorembodyfont{\upshape}          \newtheorem{Rem}{Remark}}
%\end{verbatim}
% defines a theorem set {\envfont Rem}, which will be set in
% |\upshape| in the current layout (which in our example is
% \texttt{plain}).  As with |\theoremstyle|, the |\theorembodyfont|
% chosen is that current at the time of |\newtheorem|.  If
% |\theorembodyfont| is not specified or one defines
% |\theorembodyfont{}|, then the font used will be that defined by the
% |\theoremstyle|.
%
% \DescribeMacro\theoremheaderfont It is also possible to customize
% the font used for the theorem headers.  This is, however, a global
% declaration, and therefore there should be at most one
% |\theoremheaderfont| declaration in the preamble.\footnote{If it is
% actually necessary to have different header fonts, one has to define
% new theorem styles (substituting the desired font) or specify the
% information directly in the \texttt{\bslash newtheorem} declaration
% (the unclean variant).}
%
% \DescribeMacro\theorempreskipamount
% \DescribeMacro\theorempostskipamount Two additional parameters
% affect the vertical space around the theorem environments:
% |\theorempreskipamount| and |\theorempostskipamount| define,
% respectively, the spacing before and after such an environment.
% These parameters apply for all theorem sets and can be manipulated
% with the ordinary length macros.  They are rubber lengths
% (`\textsf{skips}'), and therefore can contain \texttt{plus} and
% \texttt{minus} parts.
%
% Since the definition of theorem sets should---most sensibly---be
% placed in the preamble, we only allow installation there.
% It is therefore possible to release the memory used here after
% |\begin{document}|, in order to make room for other
% applications.
%
%
% \subsection{Existing theorem styles}
%
% The following theorem styles exist to date:
% \begin{deflist}{marginbreak}
%    \item[plain]
%       This theorem style emulates the original \LaTeX{} definition,
%       except that additionally the parameters
%       |\theorem...skipamount| are used.
%    \item[break]
%       In this style, the theorem header is followed by a line
%       break.
%    \item[marginbreak]
%       The theorem number is set in the margin, and there is a line
%       break as in \texttt{break}.
%    \item[changebreak]
%       Like \texttt{break}, but with header number and text
%       interchanged.
%    \item[change]
%       Header number and text are interchanged, without a line break.
%    \item[margin]
%       The number is set in the left margin, without a line break.
% \end{deflist}
% All styles (except \texttt{plain}) select |\slshape| as the default
% |\theorembodyfont|.
%
%
% \subsection{Examples}
%
% Given the above theorem sets {\envfont Cor}, {\envfont Exa}
% and {\envfont Rem},
% suppose that the preamble also contains the declarations:
% \begin{verbatim}
%    \theoremstyle{marginbreak}   \newtheorem{Lem}[Cor]{Lemma}
%    \theoremstyle{change}
%    \theorembodyfont{\itshape}        \newtheorem{Def}[Cor]{Definition}
%
%    \theoremheaderfont{\scshape}
%\end{verbatim}
% Then the following are some typical examples of the typeset output
% resulting from their use.
%
% \begin{Cor}
%    This is a sentence typeset in the theorem environment {\envfont
%    Cor}.
% \end{Cor}
%
% \begin{Exa}
%    This is a sentence typeset in the theorem environment {\envfont
%    Exa}.
% \end{Exa}
%
% \begin{Rem}
%    This is a sentence typeset in the theorem environment {\envfont
%    Rem}.
% \end{Rem}
%
% \begin{Lem}[Ben User]
%    This is a sentence typeset in the theorem environment {\envfont
%    Lem}.
% \end{Lem}
%
% \begin{Def}[Very impressive Definition]
%    This is a sentence typeset in the theorem environment {\envfont
%    Def}.
% \end{Def}
%
% The last two examples show the effect of the optional argument to
% a theorem environment (it is the text typeset in parentheses).
%
%
% \section{Special Considerations}
% \label{sec:newfonts}
%
% Theoremheader and body are implemented as a unit. This means that
% the |\theoremheaderfont|  will inherit characteristics of the
% |\theorembodyfont| in \LaTeXe. Thus, if for example
% |\theorembodyfont| is |\itshape| and |\theoremheaderfont|
% is |\bfseries| the font selected for the header will have the
% characteristics `bold extended italic'. If this is not desired one
% should set the |\theoremheaderfont| to something like
% \begin{verbatim}
%      \theoremheaderfont{\normalfont\bfseries}
%\end{verbatim}
% i.e.\ supplying all necessary font information explicitly.
%
%
%
% \section{Acknowledgements}
%
% The publication of this set of macros was only possible with the
% help of Christina Busse (translating the manuscript into English),
% Joachim Pense (playing the r\^ole of typist), Chris Rowley
% (looking everything over) and many others providing useful
% suggestions.
%
% \MaybeStop{%
%    \begin{thebibliography}{9}
%       \bibitem{bk:GMS94} \textsc{M.~Goossens}, \textsc{F.~Mittelbach}
%         and \textsc{A.~Samarin}.
%         \newblock The \LaTeX{} Companion.
%         \newblock
%         Addison-Wesley, Reading, Massachusetts, 1994.
%       \bibitem{latex:source} \textsc{Lamport, Leslie.} \newblock
%                               \texttt{latex.tex}, version 2.09, date
%                               Feb. 1990.
%    \end{thebibliography}
%    \PrintIndex
%    \PrintChanges}
%
%
% \section{The documentation driver file}
%
% The next bit of code contains the documentation driver file for
% \TeX{}, i.e., the file that will produce the documentation you are
% currently reading. It will be extracted from this file by the
% \texttt{docstrip} program. Since it is the first code in the file
% one can alternatively process this file directly with \LaTeXe{} to
% obtain the documentation.
%
%    \begin{macrocode}
%<*driver>
\documentclass{ltxdoc}

\usepackage{theorem}

%<+driver>% The next few lines define theorem sets which are used
%<+driver>% in the example section of the documentation.

\theoremstyle{break}       \newtheorem{Cor}{Corollary}
\theoremstyle{plain}       \newtheorem{Exa}{Example}[section]
{\theorembodyfont{\upshape}\newtheorem{Rem}{Remark}}
\theoremstyle{marginbreak} \newtheorem{Lem}[Cor]{Lemma}
\theoremstyle{change}
\theorembodyfont{\itshape} \newtheorem{Def}[Cor]{Definition}

\theoremheaderfont{\scshape}

\RecordChanges

\begin{document}
   \DocInput{theorem.dtx}
\end{document}
%</driver>
%    \end{macrocode}
%
% \section{Definition of the Macros}
%
% If the file has been loaded before, we abort immediately. If not the
% package announces itself (this is actually done at the very top of
% the file---the way it is done isn't good style so don't copy it).
% \changes{v2.0e}{89/07/19}{Spaces removed from `typeout'.}
% \changes{v2.2b}{95/11/19}{Announce moved to top of file}
%    \begin{macrocode}
%<*package>
%\@ifundefined{theorem@style}{}{\endinput}
%\def\FMithmInfo{1995/11/19 v2.2b Theorem extension package (FMi)}
%\ProvidesPackage{theorem}[\FMithmInfo]
%    \end{macrocode}
%
%
% \subsection{Definition of theorem styles and fonts}
%
% All the definitions in this file are done globally to allow
% inputting this file inside a group.
%
% \begin{macro}{\theoremstyle}
%    Before a theorem style can be installed, the chosen style must be
%    known. For that reason, we must test to see that
%    |\th@|\meta{style} is known or, more precisely, that it is
%    different from |\relax|. If the style is not known then
%    |\th@plain| is used.
%    \begin{macrocode}
\gdef\theoremstyle#1{%
   \@ifundefined{th@#1}{\@warning
          {Unknown theoremstyle `#1'. Using `plain'}%
          \theorem@style{plain}}%
%    \end{macrocode}
%    We save the theorem style to be used in the token register
%    |\theorem@style|.
%    \begin{macrocode}
      {\theorem@style{#1}}%
%    \end{macrocode}
%    Now we ``evaluate'' the theorem style: this means, we call the
%    macro |\th@|\meta{style} which will activate the relevant
%    definitions which are contained in a separate file.  This is done
%    in a group to suppress changes to the current font.  This could
%    otherwise pose problems together with the new font selection
%    scheme\footnote{When I printed the original article using the new
%    font selection scheme I ended with a document with slanted
%    typefaces (text headings and all) simply because one of the
%    theorem styles used \texttt{\bslash sl} at toplevel.} if the
%    |\th@|\meta{style} is evaluated a second time.
% \changes{v2.1a}{90/02/15}{Group added around `th@..}
% \changes{v2.1a}{90/02/15}{Reseting of `..@begintheorem removed.}
%    \begin{macrocode}
      \begingroup
        \csname th@\the\theorem@style \endcsname
      \endgroup}
%    \end{macrocode}
% \end{macro}
%
%
% \begin{macro}{\@begintheorem}
% \changes{v2.1a}{90/02/15}{Relaxed at toplevel.}
% \begin{macro}{\@opargbegintheorem}
% \changes{v2.1a}{90/02/15}{Relaxed at toplevel.}
%    We reset |\@begintheorem| and |\@opargbegintheorem| to
%    |\relax| since these commands are no longer necessary at
%    toplevel. This will save a few tokens.
%    \begin{macrocode}
\global\let\@begintheorem\relax
\global\let\@opargbegintheorem\relax
%    \end{macrocode}
% \end{macro}
% \end{macro}
%
% \begin{macro}{\theorem@style}
%    Obviously the \textsf{token} register used above has to be
%    allocated. To assure the utmost compatibility with the
%    original \LaTeX{} definition, we set the default theorem style to
%    \texttt{plain}, which implements the usual \LaTeX{} convention.
% \changes{v2.0f}{89/07/20}{Unnecessary `global' removed.}
%    \begin{macrocode}
\newtoks\theorem@style
\global\theorem@style{plain}
%    \end{macrocode}
% \end{macro}
%
% \begin{macro}{\theorembodyfont}
% \begin{macro}{\theorem@bodyfont}
%    If the |\theorembodyfont| is set by the user then it should not
%    interact with the default font set in the theorem style. When the
%    new font selection is in force this may happen if, for example,
%    the default is |\itshape| and the new |\theorembodyfont| is
%    |\sffamily|. So we add a |\reset@font| command in front of the
%    user definition.
% \changes{v2.0f}{89/07/20}{Unnecessary `global' removed.}
% \changes{v2.0g}{89/09/19}{`theoremfont to `theorembodyfont.}
% \changes{v2.1c}{92/06/22}{`reset@font added and logic changed.}
% \changes{v2.1d}{92/10/28}{Reset to empty if arg is empty}
%    \begin{macrocode}
\gdef\theorembodyfont#1{%
%    \end{macrocode}
%    We check if the argument supplied is empty and if so put nothing
%    into the |\theorem@bodyfont| token register to allow for
%    |\theorembodyfont{}| as a mean of using the default of the
%    current |\theoremstyle|.
%    \begin{macrocode}
   \def\@tempa{#1}%
   \ifx\@tempa\@empty
    \theorem@bodyfont{}%
   \else
    \theorem@bodyfont{\reset@font#1}%
   \fi
   }
\newtoks\theorem@bodyfont
\global\theorem@bodyfont{}
%    \end{macrocode}
% \end{macro}
% \end{macro}
%
%
% \begin{macro}{\theoremheaderfont}
% \changes{v2.0g}{89/09/19}{Macro added.}
%    The font for the theorem headers is handled differently because
%    this definition applies to all theorem styles.
%    \begin{macrocode}
\gdef\theoremheaderfont#1{\gdef\theorem@headerfont{#1}%
%    \end{macrocode}
%    After using the macro once it is redefined to produce an error
%    message.
%    \begin{macrocode}
       \gdef\theoremheaderfont##1{%
        \typeout{\string\theoremheaderfont\space should be used
                 only once.}}}
%    \end{macrocode}
% \end{macro}
%
% \begin{macro}{\theorem@headerfont}
% \changes{v2.0g}{89/09/19}{Macro added.}
%    To set the |\theorem@headerfont| default we first test if
%    the new fontselection scheme is in force.
% \changes{v2.1a}{90/02/15}{Support for new font selection.}
%    \begin{macrocode}
\ifx\upshape\undefined
%    \end{macrocode}
%    If not we define it to expand into |\bfseries|. We don't use
%    |\let| just in case a following style option redefines this
%    macro.
%    \begin{macrocode}
\gdef\theorem@headerfont{\bfseries}
%    \end{macrocode}
%    Otherwise we reset the current shape before calling |\bfseries|.
%    \begin{macrocode}
\else \gdef\theorem@headerfont{\normalfont\bfseries}\fi
%    \end{macrocode}
% \end{macro}
%
% \begin{macro}{\th@plain}
% \begin{macro}{\th@break}
% \begin{macro}{\th@marginbreak}
% \begin{macro}{\th@changebreak}
% \begin{macro}{\th@change}
% \begin{macro}{\th@margin}
%    The different styles are defined in macros such as
%    |\th@plain|.  Since memory space is precious in
%    ``non-Big-versions'', we have to avoid offering too many unused
%    definitions.  Therefore we define these styles in separate files
%    that can be loaded on demand.  Thus the commands themselves only
%    load these files. We use |\@input@|, a \LaTeXe{} internal command
%    that ensures that the file will be listed with |\listfiles|.
% \changes{v2.2a}{94/02/02}{Use `@input@ to load theorem layout files}
%    \begin{macrocode}
\gdef\th@plain{\@input@{thp.sty}}
\gdef\th@break{\@input@{thb.sty}}
\gdef\th@marginbreak{\@input@{thmb.sty}}
\gdef\th@changebreak{\@input@{thcb.sty}}
\gdef\th@change{\@input@{thc.sty}}
\gdef\th@margin{\@input@{thm.sty}}
%    \end{macrocode}
%    This list will be expanded when new styles become
%    available.  For testing,  just append new theorem substyles
%    as document options.
% \end{macro}
% \end{macro}
% \end{macro}
% \end{macro}
% \end{macro}
% \end{macro}
%
%
%
% \subsection{Definition of a new theorem set}
%
% As already pointed out, a new theorem environment can be defined in
% three different ways:
% \begin{verbatim}
%    \newtheorem{Lem}{Lemma}
%    \newtheorem{Lem}{Lemma}[section]
%    \newtheorem{Lem}[Theorem]{Lemma}
%\end{verbatim}
% The function of the macro |\newtheorem| is to recognize these cases
% and then to branch into one of the three macros |\@ynthm|, |\@xnthm|
% or |\@othm|.  This mechanism is adopted unchanged from
% \cite{latex:source}; the essential point here is that, for example,
% in the second case, the arguments \texttt{Lem}, \texttt{Lemma} and
% \texttt{section} are passed over to the macro |\@xnthm|.
%
% We inspect this case first because the others present fewer problems,
% and thus are easily derived from this one.
%
% \begin{macro}{\@xnthm}
%    For our example arguments, the macro |\@xnthm| must fulfill
%    the following:
%    \begin{itemize}
%       \item
%          Define a new \LaTeX-counter `Lem'
%       \item
%          reset this counter within a |\section|
%       \item
%          define the macro |\theLem|
%       \item
%          define the environment macros |\Lem| and
%          |\endLem| using the current
%          |\theoremstyle| and |\theorem@bodyfont|.
%    \end{itemize}
%    Obviously, all this should happen only if the first argument
%    of |\@xnthm| (i.e.\ \texttt{Lem} in our example) is chosen so
%    as not to conflict with any previously defined commands or
%    environments.  This test is performed by the \LaTeX{} macro
%    |\@ifdefinable|.
%    \begin{macrocode}
\gdef\@xnthm#1#2[#3]{\expandafter\@ifdefinable\csname #1\endcsname
%    \end{macrocode}
%    Therefore, the first argument of |\@ifdefinable| is the
%    expansion (in the example, |\Lem|) of
%    |\csname||#1\endcsname|.  The second argument is
%    executed only if the test has been completed successfully.
%    \begin{macrocode}
   {%
%    \end{macrocode}
%    Now we define the new counter. The names of the \LaTeX{}
%    macros employed should speak for themselves:
% \changes{v2.2b}{1995/10/07}{Use \cs{@newctr}}
% \changes{v2.2c}{1995/11/23}{Use \cs{@newctr} properly.}
%    \begin{macrocode}
    \@definecounter{#1}\@newctr{#1}[#3]%
%    \end{macrocode}
%    Using |\@newctr| will give a proper error message if the counter
%    in |#3| is not defined.
%    In defining `|\theLem|' we must generate the desired macro
%    name by use of |\expandafter| and |\csname|.
%    \begin{macrocode}
    \expandafter\xdef\csname the#1\endcsname
%    \end{macrocode}
%    An |\xdef| is used in order to make the definition global,
%    and to ensure that it contains the replacement texts of
%    |\@thmcountersep| and |\@thmcounter|.\footnote{These
%    two macros can be defined by the document style.  Their default
%    values produce a `\texttt{.}' as separation and an arabic
%    representation of the number.} However, not everything should be
%    expanded. For example, it saves space to use |\thesection|
%    instead of its---at times---lengthy expansion.
%    \begin{macrocode}
      {\expandafter \noexpand \csname the#3\endcsname
       \@thmcountersep \@thmcounter{#1}}%
%    \end{macrocode}
%    Thus with the defaults of \LaTeX{}, |\theLem| would be
%    replaced by the command sequence |\thesection.\arabic{Lem}|.
%
%    We will now look at the definition of the macro which is executed
%    at the beginning of the actual environment (in our example this
%    macro is |\Lem|). It should be noted that we use an
%    ``|\expandafter| trick'' to expand only certain parts of the
%    replacement text at the time of the definition.
%    \begin{macrocode}
    \def\@tempa{\global\@namedef{#1}}%
    \expandafter \@tempa \expandafter{%
%    \end{macrocode}
%    First, the macro that contains the current definitions of
%    |\@begintheorem| and |\@opargtheorem| should be called
%    up.  The name of this macro---as is already known---has the form
%    |\th@|\meta{theorem\ style}; therefore, it must be called by
%    \begin{macrocode}
      \csname th@\the \theorem@style
%    \end{macrocode}
%    In addition the default theorem font should be changeable, i.e.\
%    we have to insert the contents of |\theorem@bodyfont|. For
%    that reason, we expand even further, beyond |\endcsname|,
%    and thus insert the contents of the \textsf{token} register
%    |\theorem@bodyfont| in the replacement text.
%    \begin{macrocode}
            \expandafter \endcsname \the \theorem@bodyfont
%    \end{macrocode}
%    Now it is time to call the macro |\@thm| which takes over
%    the further processing.  It has two arguments: the current
%    counter name (in our example, \texttt{Lem}), and the text of the
%    label (in our example, \texttt{Lemma}).
%    \begin{macrocode}
     \@thm{#1}{#2}}%
%    \end{macrocode}
%    With this, the `sub-definition' is complete.  The macro
%    |\@endtheorem| ends a theorem environment and is, so far, nothing
%    but an |\endtrivlist|.  (Hence it is defined globally, and not
%    within the theorem styles.\footnote{This has to be changed as
%    soon as theorem styles that change \texttt{\bslash @endtheorem}
%    exist.  In such a case, all existing styles must be changed as
%    well since they will have to reset the macro.}) Therefore, we can
%    set it equivalent to the macro that ends the theorem set (in our
%    example, |\endLem|).  However, if some day theorem styles exist
%    that do change |\@endtheorem|, we would have to use the
%    commented-out line instead.
%    \begin{macrocode}
    \global \expandafter \let \csname end#1\endcsname \@endtheorem
%   \global\@namedef{end#1}{\@endtheorem}%
%    \end{macrocode}
%    With these commands all the required definitions are employed,
%    unless the test |\@ifdefinable| has failed.  Therefore, we
%    end the second argument of this macro and with it the definition
%    of |\@xnthm|.
%    \begin{macrocode}
   }}
%    \end{macrocode}
% \end{macro}
%
% \begin{macro}{\@ynthm}
%    The definition of |\@ynthm| is completely analogous. In
%    this case the new counter that is defined is not reset within
%    another counter; thus the definition of |\the...|
%    is simplified:
%    \begin{macrocode}
\gdef\@ynthm#1#2{\expandafter\@ifdefinable\csname #1\endcsname
   {\@definecounter{#1}%
    \expandafter\xdef\csname the#1\endcsname{\@thmcounter{#1}}%
%    \end{macrocode}
%    The rest of the definition corresponds literally to that of
%    |\@xnthm|:
%    \begin{macrocode}
    \def\@tempa{\global\@namedef{#1}}\expandafter \@tempa
     \expandafter{\csname th@\the \theorem@style \expandafter
     \endcsname \the\theorem@bodyfont \@thm{#1}{#2}}%
    \global \expandafter \let \csname end#1\endcsname \@endtheorem}}
%    \end{macrocode}
% \end{macro}
%
% \begin{macro}{\@othm}
%    The definition of |\@othm| does not contain anything new.
%    \begin{macrocode}
\gdef\@othm#1[#2]#3{%
%    \end{macrocode}
% \changes{v2.2b}{95/10/07}{Build in check for counter defined}
%    We do not define a new counter but instead use one that has
%    already been defined.  Thus the only definition we need is that
%    of this pseudo-counter (i.e.\ |\the|\meta{env.\ name}).
%    First we check if |#2| corresponds to a known counter name.
%    \begin{macrocode}
  \expandafter\ifx\csname c@#2\endcsname\relax
   \@nocounterr{#2}%
  \else
   \expandafter\@ifdefinable\csname #1\endcsname
   {\expandafter \xdef \csname the#1\endcsname
     {\expandafter \noexpand \csname the#2\endcsname}%
%    \end{macrocode}
%    All other parts of the definition can be adopted from
%    |\@xnthm|. We have to remember, though, that in this case
%    the name of the current counter and the theorem label have moved
%    to the second and third arguments.
%    \begin{macrocode}
    \def\@tempa{\global\@namedef{#1}}\expandafter \@tempa
     \expandafter{\csname th@\the \theorem@style \expandafter
     \endcsname \the\theorem@bodyfont \@thm{#2}{#3}}%
    \global \expandafter \let \csname end#1\endcsname \@endtheorem}%
  \fi}
%    \end{macrocode}
% \end{macro}
%
% \subsection{Macros that are employed in a {\envfont theorem}
%    environment}
%
% \begin{macro}{\@thm}
%    The macro |\@thm| has to increase the current counter.  Then,
%    depending on whether the environment has (or does not have) an
%    optional argument, it has to branch into either |\@begintheorem|
%    or |\@opargtheorem|.
%    \begin{macrocode}
\gdef\@thm#1#2{\refstepcounter{#1}%
%    \end{macrocode}
%    Now we start a {\envfont trivlist} environment, and give
%    |\@topsep| and |\@topsepadd| the values of the \textsf{skip}
%    registers |\theorempreskipamount| and |\theorempostskipamount|.
%    The value in |\@topsep| is the vertical space that is inserted by
%    the first (and only) |\item| in our |\trivlist| whilst
%    |\@topsepadd| is inserted by |\@endparenv| at the end of that
%    {\envfont trivlist} environment.  By using these registers, we
%    obtain the desired space around a {\envfont theorem} environment.
%    \begin{macrocode}
   \trivlist
   \@topsep \theorempreskipamount               % used by first \item
   \@topsepadd \theorempostskipamount           % used by \@endparenv
%    \end{macrocode}
%    Now we have to test whether an optional argument has been given.
% \changes{v2.0d}{89/07/18}{Forgotten percent added.}
%    \begin{macrocode}
   \@ifnextchar [%
%    \end{macrocode}
%    If there is an optional argument, we will call |\@ythm|, and
%    move the arguments read back into the input stream.
%    \begin{macrocode}
   {\@ythm{#1}{#2}}%
%    \end{macrocode}
%    If not, we call |\@begintheorem|. Its first argument is the
%    name of the theorem set (hence the second argument of
%    |\@thm|). Its second argument is the macro that produces the
%    current number.
%    \begin{macrocode}
   {\@begintheorem{#2}{\csname the#1\endcsname}\ignorespaces}}
%    \end{macrocode}
% \end{macro}
%
% \begin{macro}{\@xthm}
% \begin{macro}{\@ythm}
%    Both these macros were originally called by |\@thm|.  We do
%    not need |\@xthm| anymore, hence we reset it to
%    |\relax|. The definition of |\@ythm| has not changed at
%    all from its definition in \LaTeX. In order to make the macros
%    easier to understand, we will
%    nevertheless present it (commented out).
%    \begin{macrocode}
\global\let\@xthm\relax
% \def\@ythm#1#2[#3]{\@opargbegintheorem{#2}{\csname
%           the#1\endcsname}{#3}\ignorespaces}
%</package>
%    \end{macrocode}
%    The primitive |\ignorespaces| in |\@ythm| and
%    |\@thm| is needed to remove the spaces between the
%    |\begin{...}| and the actual text.
% \end{macro}
% \end{macro}
%
%
% \subsection{Definition of the theorem substyles}
%
% As already pointed out, the theorem substyles, defined below, are
% only loaded when necessary.  Note that all these substyles, except
% \texttt{plain}, have |\slshape| as the default body font.
%
% \subsubsection{The \texttt{plain} style}
%
% \label{dok:th@plain} As the following macros use |@|, we have
% to locally set the |\catcode| of this symbol to ``letter''.
% This happens within a group, so that we do not have to worry about
% which |\catcode| that symbol had before.
%    \begin{macrocode}
%<*thp>
\begingroup \makeatletter
%    \end{macrocode}
% Since we are now within a group, we must make all definitions
% globally.  First we make sure that \texttt{theorem.sty} is loaded.
% This will allow us to use this file as a document style option
% without having to call \texttt{theorem} itself as an option.  At the
% same time, we assure that at least version 2 is loaded, since
% |\theorem@style| was not defined in earlier versions.
%    \begin{macrocode}
\@ifundefined{theorem@style}{\input{theorem.sty}}{}
\ProvidesFile{thp.sty}
             [\FMithmInfo]
%    \end{macrocode}
%
%
% \begin{macro}{\th@plain}
% \begin{macro}{\@begintheorem}
% \begin{macro}{\@opargbegintheorem}
%    |\theoremstyle{plain}| corresponds to the original
%    definition, except that the distances to the surrounding text are
%    determined by the parameters |\theorempreskipamount| and
%    |\theorempostskipamount|.
%    First we set the default body font.
%    \begin{macrocode}
\gdef\th@plain{\normalfont\itshape
%    \end{macrocode}
%    Then we define |\@begintheorem| and
%    |\@opargbegintheorem|. These two macros define how the
%    header of a theorem is typeset.  |\@opargbegintheorem| will
%    be called if a \textsf{theorem} environment with an optional
%    argument is encountered; otherwise, the header is constructed by
%    calling |\@begintheorem|.  If one of these macros is
%    executed, we are within a \textsf{trivlist} environment started by
%    |\@thm|.  So the theorem header is produced with an
%    |\item| command.
%
%    Instead of specifying the header font directly, all standard
%    theorem styles use the |\theorem@headerfont| macro to allow
%    customization.  The extra space (|\labelsep|) is necessary
%    because of problems in the \textsf{trivlist} environment.
%    \begin{macrocode}
  \def\@begintheorem##1##2{%
        \item[\hskip\labelsep \theorem@headerfont ##1\ ##2]}%
%    \end{macrocode}
%    The definition of |\@opargbegintheorem| is completely
%    analogous. The only difference is the fact that there exists a
%    third argument (which is the optional parameter of the
%    environment and contains additional information about the
%    theorem).  Customarily we enclose it in parentheses.
%    \begin{macrocode}
\def\@opargbegintheorem##1##2##3{%
   \item[\hskip\labelsep \theorem@headerfont ##1\ ##2\ (##3)]}}
%    \end{macrocode}
% \end{macro}
% \end{macro}
% \end{macro}
%
% \noindent
% We conclude with an |\endgroup| to restore the
% |\catcode| of |@|.
%    \begin{macrocode}
\endgroup
%</thp>
%    \end{macrocode}
%
%
% \subsubsection{The \texttt{break} style}
%
% This style option is stored in the file |thb.sty|.
% For the next two lines see the documentation for |\th@plain| on
% page~\pageref{dok:th@plain}.
%    \begin{macrocode}
%<*thb>
\begingroup \makeatletter
\@ifundefined{theorem@style}{\input{theorem.sty}}{}
\ProvidesFile{thb.sty}
             [\FMithmInfo]
%    \end{macrocode}
%
% \begin{macro}{\th@break}
%    |\theoremstyle{break}| produces a line break after the name
%    of the theorem.  The font is |\slshape|.  Hence, we define
%    |\th@break| as follows:
%    \begin{macrocode}
\gdef\th@break{\normalfont\slshape
  \def\@begintheorem##1##2{\item[%
%    \end{macrocode}
%    We run into the following problem: it is not possible to create
%    the header with |\item[|\meta{title}|]| and then start
%    a new line by, for example, |\mbox{}\\|. Such a definition
%    will fail whenever a list environment follows immediately.  With
%    the above construction, the |\mbox{}| causes the switch
%    |@inlabel| (cf.\ definition of |\list| and
%    |\trivlist| in \cite{latex:source}) to be set to \texttt{false}
%    and so the following list will insert additional vertical
%    space (|\topskip|). This is quite annoying. Therefore, we
%    create the line break within the |\item|. In order to ensure
%    that the text will begin at the proper position in the following
%    line, we simply pretend that the label does not take any
%    room.\footnote{This will lead to problems whenever very high
%    symbols occurring in the line tower into the
%    heading. So, something else has to be done here sometime.}
%    \begin{macrocode}
       \rlap{\vbox{\hbox{\hskip \labelsep\theorem@headerfont ##1\ ##2}%
                   \hbox{\strut}}}]}%
%    \end{macrocode}
%    Again, the definition of |\@opargbegintheorem| is
%    completely analogous.
%    \begin{macrocode}
\def\@opargbegintheorem##1##2##3{%
  \item[\rlap{\vbox{\hbox{\hskip \labelsep \theorem@headerfont
                     ##1\ ##2\ (##3)}%
                    \hbox{\strut}}}]}}
\endgroup
%</thb>
%    \end{macrocode}
% \end{macro}
%
%
% \subsubsection{The \texttt{changebreak} style}
%
%    \begin{macrocode}
% This style option is stored in the file |thcb.sty|.
%    \begin{macrocode}
%<*thcb>
\begingroup \makeatletter
\@ifundefined{theorem@style}{\input{theorem.sty}}{}
\ProvidesFile{thcb.sty}
             [\FMithmInfo]
%    \end{macrocode}
%
% \begin{macro}{\th@changebreak}
%    The \texttt{change-break} theorem style is like \texttt{break}
%    but with interchange of theorem name and theorem number.  Thus we
%    define |\th@changebreak| as follows:
%    \begin{macrocode}
\gdef\th@changebreak{\normalfont\slshape
  \def\@begintheorem##1##2{\item
       [\rlap{\vbox{\hbox{\hskip\labelsep \theorem@headerfont ##2\ ##1}%
                    \hbox{\strut}}}]}%
\def\@opargbegintheorem##1##2##3{%
  \item[\rlap{\vbox{\hbox{\hskip\labelsep \theorem@headerfont
                     ##2\ ##1\ (##3)}%
                    \hbox{\strut}}}]}}
\endgroup
%</thcb>
%    \end{macrocode}
% \end{macro}
%
%
% \subsubsection{The \texttt{change} style}
%
% This style option is stored in the file |thc.sty|.
%    \begin{macrocode}
%<*thc>
\begingroup \makeatletter
\@ifundefined{theorem@style}{\input{theorem.sty}}{}
\ProvidesFile{thc.sty}
             [\FMithmInfo]
%    \end{macrocode}
%
% \begin{macro}{\th@change}
%    The \texttt{change} theorem style corresponds to the
%    \texttt{change break} style without a linebreak after the header.
%    To say it in another way, it's the same as the \texttt{plain}
%    style but with number and name interchanged and |\slshape| as the
%    default font.
%    \begin{macrocode}
\gdef\th@change{\normalfont\slshape
  \def\@begintheorem##1##2{\item
       [\hskip\labelsep \theorem@headerfont ##2\ ##1]}%
\def\@opargbegintheorem##1##2##3{%
  \item[\hskip\labelsep \theorem@headerfont ##2\ ##1\ (##3)]}}
\endgroup
%</thc>
%    \end{macrocode}
% \end{macro}
%
%
% \subsubsection{The \texttt{marginbreak} style}
%
% This style option is the one used most often at Mainz.  It is saved
% in the file |thmb.sty|.
%    \begin{macrocode}
%<*thmb>
\begingroup \makeatletter
\@ifundefined{theorem@style}{\input{theorem.sty}}{}
\ProvidesFile{thmb.sty}
             [\FMithmInfo]
%    \end{macrocode}
%
% \begin{macro}{\th@marginbreak}
%    The \texttt{margin break} style is nearly the same as the
%    \texttt{change break} style. The only difference is the placement
%    of the theorem number.  We use |\llap| to place it in the left
%    margin.
%
%    In this style |\labelsep| denotes the separation between the
%    number and the text.
%    \begin{macrocode}
\gdef\th@marginbreak{\normalfont\slshape
  \def\@begintheorem##1##2{\item
       [\rlap{\vbox{\theorem@headerfont
                    \hbox{\llap{##2}\hskip\labelsep ##1}%
                    \hbox{\strut}}}]}%
\def\@opargbegintheorem##1##2##3{%
  \item[\rlap{\vbox{\theorem@headerfont
                    \hbox{\llap{##2}\hskip\labelsep ##1\ (##3)}%
                    \hbox{\strut}}}]}}
\endgroup
%</thmb>
%    \end{macrocode}
% \changes{v2.1c}{90/02/17}{`endgroup outside macrocode env.}
% \end{macro}
%
%
%
% \subsubsection{The \texttt{margin} style}
%
% This style option is stored in the file |thm.sty|.
%    \begin{macrocode}
%<*thm>
\begingroup \makeatletter
\@ifundefined{theorem@style}{\input{theorem.sty}}{}
\ProvidesFile{thm.sty}
             [\FMithmInfo]
%    \end{macrocode}
%
% \begin{macro}{\th@margin}
%    Again this is only a variant of the theorem styles described
%    above without any new ideas.
%    \begin{macrocode}
\gdef\th@margin{\normalfont\slshape
  \def\@begintheorem##1##2{\item
       [\theorem@headerfont \llap{##2}\hskip\labelsep ##1]}%
\def\@opargbegintheorem##1##2##3{%
  \item[\theorem@headerfont \llap{##2}\hskip\labelsep ##1\ (##3)]}}
\endgroup
%</thm>
%    \end{macrocode}
% \end{macro}
%
%
%
%
% \subsection{Final Definitions}
%
% \begin{macro}{\theorempreskipamount}
% \begin{macro}{\theorempostskipamount}
%    The \textsf{skip} parameters that regulate the vertical empty space
%    before and after the {\envfont theorem} environment have
%    to be allocated as well.
% \changes{v2.0f}{89/07/20}{Unnecessary `global' removed.}
%    \begin{macrocode}
%<*package>
\newskip\theorempreskipamount
\newskip\theorempostskipamount
%    \end{macrocode}
%    Since we have used the same values for all theorem sets, we now can
%    assign them.
%    \begin{macrocode}
\global\setlength\theorempreskipamount{12pt plus 5pt minus 3pt}
\global\setlength\theorempostskipamount{8pt plus 3pt minus 1.5pt}
%    \end{macrocode}
% \end{macro}
% \end{macro}
%
%
% \begin{macro}{\@endtheorem}
%    The same holds for the macro |\@endtheorem|, which ends a
%    {\envfont theorem} environment. Since it is the same for all
%    theorem sets, it is removed from the macros |\th@|\meta{style}.
%    It simply ends the {\envfont trivlist} environment, which was
%    begun in |\@thm|.
%    \begin{macrocode}
\global\let\@endtheorem=\endtrivlist
%    \end{macrocode}
% \end{macro}
%
%
%
% \begin{macro}{\@preamblecmds}
%    All macros defined above are to be used only in the preamble.
%    Therefore, we insert them in |\@preamblecmds| which will disable
%    them at begin document. This is done by the internal \LaTeXe{}
%    command |\@onlypreamble|.
% \changes{v2.0g}{89/09/19}{`theoremheaderfont and `theorembodyfont
%    added.}
% \changes{v2.2a}{94/02/03}{Use \LaTeXe{} interface}
%    \begin{macrocode}
\@onlypreamble\@xnthm
\@onlypreamble\@ynthm
\@onlypreamble\@othm
\@onlypreamble\newtheorem
\@onlypreamble\theoremstyle
\@onlypreamble\theorembodyfont
\@onlypreamble\theoremheaderfont
%    \end{macrocode}
% \end{macro}
%
% Finally we declare the \texttt{plain} theorem style to be the default.
% \changes{v2.1b}{90/06/04}{Fixed missing `item by declaring default
%    style.}
%    \begin{macrocode}
\theoremstyle{plain}
%</package>
%    \end{macrocode}
%
% \Finale

\endinput
}{}
\ProvidesFile{thcb.sty}
             [\FMithmInfo]
%    \end{macrocode}
%
% \begin{macro}{\th@changebreak}
%    The \texttt{change-break} theorem style is like \texttt{break}
%    but with interchange of theorem name and theorem number.  Thus we
%    define |\th@changebreak| as follows:
%    \begin{macrocode}
\gdef\th@changebreak{\normalfont\slshape
  \def\@begintheorem##1##2{\item
       [\rlap{\vbox{\hbox{\hskip\labelsep \theorem@headerfont ##2\ ##1}%
                    \hbox{\strut}}}]}%
\def\@opargbegintheorem##1##2##3{%
  \item[\rlap{\vbox{\hbox{\hskip\labelsep \theorem@headerfont
                     ##2\ ##1\ (##3)}%
                    \hbox{\strut}}}]}}
\endgroup
%</thcb>
%    \end{macrocode}
% \end{macro}
%
%
% \subsubsection{The \texttt{change} style}
%
% This style option is stored in the file |thc.sty|.
%    \begin{macrocode}
%<*thc>
\begingroup \makeatletter
\@ifundefined{theorem@style}{% \iffalse meta-comment
%
% Copyright (C) 1993-2024
%
% The LaTeX Project and any individual authors listed elsewhere
% in this file.
%
% This file is part of the Standard LaTeX `Tools Bundle'.
% -------------------------------------------------------
%
% It may be distributed and/or modified under the
% conditions of the LaTeX Project Public License, either version 1.3c
% of this license or (at your option) any later version.
% The latest version of this license is in
%    https://www.latex-project.org/lppl.txt
% and version 1.3c or later is part of all distributions of LaTeX
% version 2005/12/01 or later.
%
% The list of all files belonging to the LaTeX `Tools Bundle' is
% given in the file `manifest.txt'.
%
% \fi
%
%%
%
%
\def\FMithmInfo{2023/07/05 v2.2c Theorem extension package (FMi)}
%
% \ProvidesFile{theorem.dtx}[\FMithmInfo]
%\iffalse   % this is a METACOMMENT !
%%
%% Package `theorem' to use with LaTeX2e
%% Copyright (C) 1989-1995 Frank Mittelbach, all rights reserved.
%<+package>\@ifundefined{theorem@style}{}{\endinput}
%<+package>\NeedsTeXFormat{LaTeX2e}[1995/06/01]
%<+package>\ProvidesPackage{theorem}
%<+package>         [\FMithmInfo]
%
% \fi
%
% \changes{v1.0}{87/05/07}{First draft.}
% \changes{v1.0a}{87/05/07}{`thistheoremstyle' added.}
% \changes{v1.1a}{87/07/29}{New concept: redefine the internal
%                          \protect\LaTeX{} commands `@xnthm' etc.}
% \changes{v1.1b}{87/07/31}{`theoremfont' used to specify font.}
% \changes{v1.2a}{88/01/08}{`@endtheorem' now globally defined.}
% \changes{v1.2b}{88/01/11}{`newtheorem' only allowed in the preamble.}
% \changes{v1.3a}{88/02/14}{`theoremfont' now a toks register.}
%
% \changes{v2.0a}{88/11/16}{New implementation and new documentation.}
% \changes{v2.0a}{88/11/16}{No layout changes but internal modifications
%                           old substyles won't work.}
% \changes{v2.0b}{89/01/30}{`global' used for all definitions.}
% \changes{v2.0c}{89/06/15}{Documentation translated to English.
%                            Thanks to LLL and Nina}
% \changes{v2.1a}{90/02/15}{Support for new font selection.}
% \changes{v2.2a}{94/02/03}{Cleanup for LaTeX2e}
%
% \DoNotIndex{\@endparenv,\@ifdefinable,\@ifnextchar,\@ifundefined}
% \DoNotIndex{\@namedef,\@nameuse,\@spaces,\@tempa,\@warning}
% \DoNotIndex{\ }
% \DoNotIndex{\begingroup,\bfseries}
% \DoNotIndex{\csname}
% \DoNotIndex{\def,\do,\docdate}
% \DoNotIndex{\endcsname,\endgroup,\endinput,\endtrivlist,\expandafter}
% \DoNotIndex{\filedate,\fileversion}
% \DoNotIndex{\gdef,\global}
% \DoNotIndex{\hbox,\hskip}
% \DoNotIndex{\ignorespaces,\input,\itshape,\item}
% \DoNotIndex{\labelsep,\let,\llap}
% \DoNotIndex{\makeatletter}
% \DoNotIndex{\noexpand}
% \DoNotIndex{\refstepcounter,\relax,\rlap}
% \DoNotIndex{\setlength,\slshape,\space,\string,\strut}
% \DoNotIndex{\the,\trivlist,\typeout}
% \DoNotIndex{\vbox}
% \DoNotIndex{\xdef}
%
% \def\deflabel#1{\ttfamily #1\hfill}
% \def\deflist#1{\begin{list}{}{\settowidth\labelwidth{\ttfamily #1}%
%                               \setlength\leftmargin\labelwidth
%                               \addtolength\leftmargin\labelsep
%                               \let\makelabel\deflabel}}
% \def\enddeflist{\end{list}}
% \def\envfont{\normalfont\ttfamily}
%
% \MakeShortVerb{\|}
%
% \newcommand{\TUB}{{\sl TUGboat\/}}
%
% \GetFileInfo{theorem.dtx}
% \title{An Extension of the
%        \LaTeX{} theorem environment\thanks{This file
%        has version number \fileversion, last
%        revised \filedate.}}
% \author{Frank Mittelbach}
%
% \MaintainedByLaTeXTeam{tools}
% \maketitle
%
% \begin{abstract}
%    The macros described in this paper yield an extension of the
%    \LaTeX{} theorem mechanism. It is designed to satisfy the
%    different requirements of various journals. Thus, the
%    layout of the ``theorems'' can be manipulated by determining a
%    ``style''. This article describes not only the use, but
%    also the definition, of the necessary macros.
% \end{abstract}
%
% \section*{Preface to version 2.2}
%
% For \LaTeXe{} this package did not need any fundamental changes. I
% only modified the messages generated so that theorem layout styles
% will show up with the |\listfiles| command and cleaned the section
% on the New Font Selection Scheme since this is now included in
% \LaTeX.
%
% \section*{Preface to version 2.1}
%
% This version is identical to 2.0g described in \TUB\ 10\#3 except
% for some internal defaults which are now set depending on the used
% font selection scheme.
%
% This was done to avoid unpleasant surprises if the new font selection
% scheme is in force. For further details see
% section~\ref{sec:newfonts} and \cite{bk:GMS94}.
%
% \section{Introduction}
%
% For our purposes here, ``theorems'' are labelled enunciations,
% often set off from the main text by extra space and a font change.
% Theorems, corollaries, conjectures, definitions, and remarks
% are all instances of ``theorems''.  The ``header'' of these structures
% is composed of a label (such as \textsc{Theorem} or \textsc{Remark})
% and a number which serializes an item in the sequence of items
% with the same label.
%
% Shortly after the introduction of \LaTeX{} at the Fachbereich
% Mathematik in Mainz, the desire to manipulate the layout of
% ``theorems'' arose.  In Mainz, the following two conventions came
% into general use:
% \begin{enumerate}
%   \item The number of the theorem is shown in the margin.
%   \item There is a line break at
%         the end of the theorem header.
% \end{enumerate}
%
% Additionally, some journals require different formats which depend on
% the ``sort of theorem'': e.g.\ often remarks and definitions are set
% in |\upshape|, while |\itshape| is employed for main theorems.
%
% Confronted with these requirements, a theorem environment was
% developed in Mainz which allows separate determination of the
% layout of the ``theorems sets'', comparable to |\pagestyle|.
%
%
% \section{The user interface}
% \subsection{Defining new theorem sets}
%
% \DescribeMacro\newtheorem
% As in the original \LaTeX{} version, the command |\newtheorem|
% defines a new ``theorem set'' or ``theorem-like structure''.
% Two required arguments name the new environment and give the text
% to be typeset with each instance of the new ``set'', while
% an optional argument determines how
% the ``set'' is enumerated:
% \begin{description}
%    \item[\ttfamily \bslash newtheorem\{foo\}\{bar\}]
%       The theorem set {\envfont foo} (whose name is \texttt{bar})
%       uses its own counter.
%    \item[\ttfamily \bslash newtheorem\{foo2\}{[foo]}\{bar2\}]
%       The theorem set {\envfont foo2} (printed name \texttt{bar2})
%       uses the same counter as the theorem set \texttt{foo}.
%    \item[\ttfamily \bslash newtheorem\{foo3\}\{bar3\}{[section]}]
%       The theorem set {\envfont foo3} (printed name \texttt{bar3}) is
%       enumerated within the counter \texttt{section}, i.e.\ with every
%       new |\section| the enumeration begins again with 1, and
%       the enumeration is composed from the section-number and the
%       theorem counter itself.
% \end{description}
%
% \DescribeMacro\theoremstyle
% Additionally, the command |\theoremstyle| can define the layout
% of various, or all, theorem sets.  It should be noted that any
% theorem set defined by |\newtheorem| is typeset in the
% |\theoremstyle| that is current at the time of the definition.
% Thus, the following
% \begin{verbatim}
%    \theoremstyle{break}        \newtheorem{Cor}{Corollary}
%    \theoremstyle{plain}        \newtheorem{Exa}{Example}[section]
%\end{verbatim}                             ^^A No blank at line start
% leads to the result that the set {\envfont Cor} is formatted in the
% style \texttt{break}, while the set {\envfont Exa} and all the
% following ones are formatted in the style \texttt{plain}, unless
% another |\theoremstyle| follows.  Since the definitions installed by
% |\newtheorem| are global, one also can limit |\theoremstyle| locally
% by grouping braces.
%
% \DescribeMacro\theorembodyfont
% The choice of the font for the theorem body is completely
% independent of the chosen |\theoremstyle|; this has proven to
% be very advantageous.  For example,
% \begin{verbatim}
%   {\theorembodyfont{\upshape}          \newtheorem{Rem}{Remark}}
%\end{verbatim}
% defines a theorem set {\envfont Rem}, which will be set in
% |\upshape| in the current layout (which in our example is
% \texttt{plain}).  As with |\theoremstyle|, the |\theorembodyfont|
% chosen is that current at the time of |\newtheorem|.  If
% |\theorembodyfont| is not specified or one defines
% |\theorembodyfont{}|, then the font used will be that defined by the
% |\theoremstyle|.
%
% \DescribeMacro\theoremheaderfont It is also possible to customize
% the font used for the theorem headers.  This is, however, a global
% declaration, and therefore there should be at most one
% |\theoremheaderfont| declaration in the preamble.\footnote{If it is
% actually necessary to have different header fonts, one has to define
% new theorem styles (substituting the desired font) or specify the
% information directly in the \texttt{\bslash newtheorem} declaration
% (the unclean variant).}
%
% \DescribeMacro\theorempreskipamount
% \DescribeMacro\theorempostskipamount Two additional parameters
% affect the vertical space around the theorem environments:
% |\theorempreskipamount| and |\theorempostskipamount| define,
% respectively, the spacing before and after such an environment.
% These parameters apply for all theorem sets and can be manipulated
% with the ordinary length macros.  They are rubber lengths
% (`\textsf{skips}'), and therefore can contain \texttt{plus} and
% \texttt{minus} parts.
%
% Since the definition of theorem sets should---most sensibly---be
% placed in the preamble, we only allow installation there.
% It is therefore possible to release the memory used here after
% |\begin{document}|, in order to make room for other
% applications.
%
%
% \subsection{Existing theorem styles}
%
% The following theorem styles exist to date:
% \begin{deflist}{marginbreak}
%    \item[plain]
%       This theorem style emulates the original \LaTeX{} definition,
%       except that additionally the parameters
%       |\theorem...skipamount| are used.
%    \item[break]
%       In this style, the theorem header is followed by a line
%       break.
%    \item[marginbreak]
%       The theorem number is set in the margin, and there is a line
%       break as in \texttt{break}.
%    \item[changebreak]
%       Like \texttt{break}, but with header number and text
%       interchanged.
%    \item[change]
%       Header number and text are interchanged, without a line break.
%    \item[margin]
%       The number is set in the left margin, without a line break.
% \end{deflist}
% All styles (except \texttt{plain}) select |\slshape| as the default
% |\theorembodyfont|.
%
%
% \subsection{Examples}
%
% Given the above theorem sets {\envfont Cor}, {\envfont Exa}
% and {\envfont Rem},
% suppose that the preamble also contains the declarations:
% \begin{verbatim}
%    \theoremstyle{marginbreak}   \newtheorem{Lem}[Cor]{Lemma}
%    \theoremstyle{change}
%    \theorembodyfont{\itshape}        \newtheorem{Def}[Cor]{Definition}
%
%    \theoremheaderfont{\scshape}
%\end{verbatim}
% Then the following are some typical examples of the typeset output
% resulting from their use.
%
% \begin{Cor}
%    This is a sentence typeset in the theorem environment {\envfont
%    Cor}.
% \end{Cor}
%
% \begin{Exa}
%    This is a sentence typeset in the theorem environment {\envfont
%    Exa}.
% \end{Exa}
%
% \begin{Rem}
%    This is a sentence typeset in the theorem environment {\envfont
%    Rem}.
% \end{Rem}
%
% \begin{Lem}[Ben User]
%    This is a sentence typeset in the theorem environment {\envfont
%    Lem}.
% \end{Lem}
%
% \begin{Def}[Very impressive Definition]
%    This is a sentence typeset in the theorem environment {\envfont
%    Def}.
% \end{Def}
%
% The last two examples show the effect of the optional argument to
% a theorem environment (it is the text typeset in parentheses).
%
%
% \section{Special Considerations}
% \label{sec:newfonts}
%
% Theoremheader and body are implemented as a unit. This means that
% the |\theoremheaderfont|  will inherit characteristics of the
% |\theorembodyfont| in \LaTeXe. Thus, if for example
% |\theorembodyfont| is |\itshape| and |\theoremheaderfont|
% is |\bfseries| the font selected for the header will have the
% characteristics `bold extended italic'. If this is not desired one
% should set the |\theoremheaderfont| to something like
% \begin{verbatim}
%      \theoremheaderfont{\normalfont\bfseries}
%\end{verbatim}
% i.e.\ supplying all necessary font information explicitly.
%
%
%
% \section{Acknowledgements}
%
% The publication of this set of macros was only possible with the
% help of Christina Busse (translating the manuscript into English),
% Joachim Pense (playing the r\^ole of typist), Chris Rowley
% (looking everything over) and many others providing useful
% suggestions.
%
% \MaybeStop{%
%    \begin{thebibliography}{9}
%       \bibitem{bk:GMS94} \textsc{M.~Goossens}, \textsc{F.~Mittelbach}
%         and \textsc{A.~Samarin}.
%         \newblock The \LaTeX{} Companion.
%         \newblock
%         Addison-Wesley, Reading, Massachusetts, 1994.
%       \bibitem{latex:source} \textsc{Lamport, Leslie.} \newblock
%                               \texttt{latex.tex}, version 2.09, date
%                               Feb. 1990.
%    \end{thebibliography}
%    \PrintIndex
%    \PrintChanges}
%
%
% \section{The documentation driver file}
%
% The next bit of code contains the documentation driver file for
% \TeX{}, i.e., the file that will produce the documentation you are
% currently reading. It will be extracted from this file by the
% \texttt{docstrip} program. Since it is the first code in the file
% one can alternatively process this file directly with \LaTeXe{} to
% obtain the documentation.
%
%    \begin{macrocode}
%<*driver>
\documentclass{ltxdoc}

\usepackage{theorem}

%<+driver>% The next few lines define theorem sets which are used
%<+driver>% in the example section of the documentation.

\theoremstyle{break}       \newtheorem{Cor}{Corollary}
\theoremstyle{plain}       \newtheorem{Exa}{Example}[section]
{\theorembodyfont{\upshape}\newtheorem{Rem}{Remark}}
\theoremstyle{marginbreak} \newtheorem{Lem}[Cor]{Lemma}
\theoremstyle{change}
\theorembodyfont{\itshape} \newtheorem{Def}[Cor]{Definition}

\theoremheaderfont{\scshape}

\RecordChanges

\begin{document}
   \DocInput{theorem.dtx}
\end{document}
%</driver>
%    \end{macrocode}
%
% \section{Definition of the Macros}
%
% If the file has been loaded before, we abort immediately. If not the
% package announces itself (this is actually done at the very top of
% the file---the way it is done isn't good style so don't copy it).
% \changes{v2.0e}{89/07/19}{Spaces removed from `typeout'.}
% \changes{v2.2b}{95/11/19}{Announce moved to top of file}
%    \begin{macrocode}
%<*package>
%\@ifundefined{theorem@style}{}{\endinput}
%\def\FMithmInfo{1995/11/19 v2.2b Theorem extension package (FMi)}
%\ProvidesPackage{theorem}[\FMithmInfo]
%    \end{macrocode}
%
%
% \subsection{Definition of theorem styles and fonts}
%
% All the definitions in this file are done globally to allow
% inputting this file inside a group.
%
% \begin{macro}{\theoremstyle}
%    Before a theorem style can be installed, the chosen style must be
%    known. For that reason, we must test to see that
%    |\th@|\meta{style} is known or, more precisely, that it is
%    different from |\relax|. If the style is not known then
%    |\th@plain| is used.
%    \begin{macrocode}
\gdef\theoremstyle#1{%
   \@ifundefined{th@#1}{\@warning
          {Unknown theoremstyle `#1'. Using `plain'}%
          \theorem@style{plain}}%
%    \end{macrocode}
%    We save the theorem style to be used in the token register
%    |\theorem@style|.
%    \begin{macrocode}
      {\theorem@style{#1}}%
%    \end{macrocode}
%    Now we ``evaluate'' the theorem style: this means, we call the
%    macro |\th@|\meta{style} which will activate the relevant
%    definitions which are contained in a separate file.  This is done
%    in a group to suppress changes to the current font.  This could
%    otherwise pose problems together with the new font selection
%    scheme\footnote{When I printed the original article using the new
%    font selection scheme I ended with a document with slanted
%    typefaces (text headings and all) simply because one of the
%    theorem styles used \texttt{\bslash sl} at toplevel.} if the
%    |\th@|\meta{style} is evaluated a second time.
% \changes{v2.1a}{90/02/15}{Group added around `th@..}
% \changes{v2.1a}{90/02/15}{Reseting of `..@begintheorem removed.}
%    \begin{macrocode}
      \begingroup
        \csname th@\the\theorem@style \endcsname
      \endgroup}
%    \end{macrocode}
% \end{macro}
%
%
% \begin{macro}{\@begintheorem}
% \changes{v2.1a}{90/02/15}{Relaxed at toplevel.}
% \begin{macro}{\@opargbegintheorem}
% \changes{v2.1a}{90/02/15}{Relaxed at toplevel.}
%    We reset |\@begintheorem| and |\@opargbegintheorem| to
%    |\relax| since these commands are no longer necessary at
%    toplevel. This will save a few tokens.
%    \begin{macrocode}
\global\let\@begintheorem\relax
\global\let\@opargbegintheorem\relax
%    \end{macrocode}
% \end{macro}
% \end{macro}
%
% \begin{macro}{\theorem@style}
%    Obviously the \textsf{token} register used above has to be
%    allocated. To assure the utmost compatibility with the
%    original \LaTeX{} definition, we set the default theorem style to
%    \texttt{plain}, which implements the usual \LaTeX{} convention.
% \changes{v2.0f}{89/07/20}{Unnecessary `global' removed.}
%    \begin{macrocode}
\newtoks\theorem@style
\global\theorem@style{plain}
%    \end{macrocode}
% \end{macro}
%
% \begin{macro}{\theorembodyfont}
% \begin{macro}{\theorem@bodyfont}
%    If the |\theorembodyfont| is set by the user then it should not
%    interact with the default font set in the theorem style. When the
%    new font selection is in force this may happen if, for example,
%    the default is |\itshape| and the new |\theorembodyfont| is
%    |\sffamily|. So we add a |\reset@font| command in front of the
%    user definition.
% \changes{v2.0f}{89/07/20}{Unnecessary `global' removed.}
% \changes{v2.0g}{89/09/19}{`theoremfont to `theorembodyfont.}
% \changes{v2.1c}{92/06/22}{`reset@font added and logic changed.}
% \changes{v2.1d}{92/10/28}{Reset to empty if arg is empty}
%    \begin{macrocode}
\gdef\theorembodyfont#1{%
%    \end{macrocode}
%    We check if the argument supplied is empty and if so put nothing
%    into the |\theorem@bodyfont| token register to allow for
%    |\theorembodyfont{}| as a mean of using the default of the
%    current |\theoremstyle|.
%    \begin{macrocode}
   \def\@tempa{#1}%
   \ifx\@tempa\@empty
    \theorem@bodyfont{}%
   \else
    \theorem@bodyfont{\reset@font#1}%
   \fi
   }
\newtoks\theorem@bodyfont
\global\theorem@bodyfont{}
%    \end{macrocode}
% \end{macro}
% \end{macro}
%
%
% \begin{macro}{\theoremheaderfont}
% \changes{v2.0g}{89/09/19}{Macro added.}
%    The font for the theorem headers is handled differently because
%    this definition applies to all theorem styles.
%    \begin{macrocode}
\gdef\theoremheaderfont#1{\gdef\theorem@headerfont{#1}%
%    \end{macrocode}
%    After using the macro once it is redefined to produce an error
%    message.
%    \begin{macrocode}
       \gdef\theoremheaderfont##1{%
        \typeout{\string\theoremheaderfont\space should be used
                 only once.}}}
%    \end{macrocode}
% \end{macro}
%
% \begin{macro}{\theorem@headerfont}
% \changes{v2.0g}{89/09/19}{Macro added.}
%    To set the |\theorem@headerfont| default we first test if
%    the new fontselection scheme is in force.
% \changes{v2.1a}{90/02/15}{Support for new font selection.}
%    \begin{macrocode}
\ifx\upshape\undefined
%    \end{macrocode}
%    If not we define it to expand into |\bfseries|. We don't use
%    |\let| just in case a following style option redefines this
%    macro.
%    \begin{macrocode}
\gdef\theorem@headerfont{\bfseries}
%    \end{macrocode}
%    Otherwise we reset the current shape before calling |\bfseries|.
%    \begin{macrocode}
\else \gdef\theorem@headerfont{\normalfont\bfseries}\fi
%    \end{macrocode}
% \end{macro}
%
% \begin{macro}{\th@plain}
% \begin{macro}{\th@break}
% \begin{macro}{\th@marginbreak}
% \begin{macro}{\th@changebreak}
% \begin{macro}{\th@change}
% \begin{macro}{\th@margin}
%    The different styles are defined in macros such as
%    |\th@plain|.  Since memory space is precious in
%    ``non-Big-versions'', we have to avoid offering too many unused
%    definitions.  Therefore we define these styles in separate files
%    that can be loaded on demand.  Thus the commands themselves only
%    load these files. We use |\@input@|, a \LaTeXe{} internal command
%    that ensures that the file will be listed with |\listfiles|.
% \changes{v2.2a}{94/02/02}{Use `@input@ to load theorem layout files}
%    \begin{macrocode}
\gdef\th@plain{\@input@{thp.sty}}
\gdef\th@break{\@input@{thb.sty}}
\gdef\th@marginbreak{\@input@{thmb.sty}}
\gdef\th@changebreak{\@input@{thcb.sty}}
\gdef\th@change{\@input@{thc.sty}}
\gdef\th@margin{\@input@{thm.sty}}
%    \end{macrocode}
%    This list will be expanded when new styles become
%    available.  For testing,  just append new theorem substyles
%    as document options.
% \end{macro}
% \end{macro}
% \end{macro}
% \end{macro}
% \end{macro}
% \end{macro}
%
%
%
% \subsection{Definition of a new theorem set}
%
% As already pointed out, a new theorem environment can be defined in
% three different ways:
% \begin{verbatim}
%    \newtheorem{Lem}{Lemma}
%    \newtheorem{Lem}{Lemma}[section]
%    \newtheorem{Lem}[Theorem]{Lemma}
%\end{verbatim}
% The function of the macro |\newtheorem| is to recognize these cases
% and then to branch into one of the three macros |\@ynthm|, |\@xnthm|
% or |\@othm|.  This mechanism is adopted unchanged from
% \cite{latex:source}; the essential point here is that, for example,
% in the second case, the arguments \texttt{Lem}, \texttt{Lemma} and
% \texttt{section} are passed over to the macro |\@xnthm|.
%
% We inspect this case first because the others present fewer problems,
% and thus are easily derived from this one.
%
% \begin{macro}{\@xnthm}
%    For our example arguments, the macro |\@xnthm| must fulfill
%    the following:
%    \begin{itemize}
%       \item
%          Define a new \LaTeX-counter `Lem'
%       \item
%          reset this counter within a |\section|
%       \item
%          define the macro |\theLem|
%       \item
%          define the environment macros |\Lem| and
%          |\endLem| using the current
%          |\theoremstyle| and |\theorem@bodyfont|.
%    \end{itemize}
%    Obviously, all this should happen only if the first argument
%    of |\@xnthm| (i.e.\ \texttt{Lem} in our example) is chosen so
%    as not to conflict with any previously defined commands or
%    environments.  This test is performed by the \LaTeX{} macro
%    |\@ifdefinable|.
%    \begin{macrocode}
\gdef\@xnthm#1#2[#3]{\expandafter\@ifdefinable\csname #1\endcsname
%    \end{macrocode}
%    Therefore, the first argument of |\@ifdefinable| is the
%    expansion (in the example, |\Lem|) of
%    |\csname||#1\endcsname|.  The second argument is
%    executed only if the test has been completed successfully.
%    \begin{macrocode}
   {%
%    \end{macrocode}
%    Now we define the new counter. The names of the \LaTeX{}
%    macros employed should speak for themselves:
% \changes{v2.2b}{1995/10/07}{Use \cs{@newctr}}
% \changes{v2.2c}{1995/11/23}{Use \cs{@newctr} properly.}
%    \begin{macrocode}
    \@definecounter{#1}\@newctr{#1}[#3]%
%    \end{macrocode}
%    Using |\@newctr| will give a proper error message if the counter
%    in |#3| is not defined.
%    In defining `|\theLem|' we must generate the desired macro
%    name by use of |\expandafter| and |\csname|.
%    \begin{macrocode}
    \expandafter\xdef\csname the#1\endcsname
%    \end{macrocode}
%    An |\xdef| is used in order to make the definition global,
%    and to ensure that it contains the replacement texts of
%    |\@thmcountersep| and |\@thmcounter|.\footnote{These
%    two macros can be defined by the document style.  Their default
%    values produce a `\texttt{.}' as separation and an arabic
%    representation of the number.} However, not everything should be
%    expanded. For example, it saves space to use |\thesection|
%    instead of its---at times---lengthy expansion.
%    \begin{macrocode}
      {\expandafter \noexpand \csname the#3\endcsname
       \@thmcountersep \@thmcounter{#1}}%
%    \end{macrocode}
%    Thus with the defaults of \LaTeX{}, |\theLem| would be
%    replaced by the command sequence |\thesection.\arabic{Lem}|.
%
%    We will now look at the definition of the macro which is executed
%    at the beginning of the actual environment (in our example this
%    macro is |\Lem|). It should be noted that we use an
%    ``|\expandafter| trick'' to expand only certain parts of the
%    replacement text at the time of the definition.
%    \begin{macrocode}
    \def\@tempa{\global\@namedef{#1}}%
    \expandafter \@tempa \expandafter{%
%    \end{macrocode}
%    First, the macro that contains the current definitions of
%    |\@begintheorem| and |\@opargtheorem| should be called
%    up.  The name of this macro---as is already known---has the form
%    |\th@|\meta{theorem\ style}; therefore, it must be called by
%    \begin{macrocode}
      \csname th@\the \theorem@style
%    \end{macrocode}
%    In addition the default theorem font should be changeable, i.e.\
%    we have to insert the contents of |\theorem@bodyfont|. For
%    that reason, we expand even further, beyond |\endcsname|,
%    and thus insert the contents of the \textsf{token} register
%    |\theorem@bodyfont| in the replacement text.
%    \begin{macrocode}
            \expandafter \endcsname \the \theorem@bodyfont
%    \end{macrocode}
%    Now it is time to call the macro |\@thm| which takes over
%    the further processing.  It has two arguments: the current
%    counter name (in our example, \texttt{Lem}), and the text of the
%    label (in our example, \texttt{Lemma}).
%    \begin{macrocode}
     \@thm{#1}{#2}}%
%    \end{macrocode}
%    With this, the `sub-definition' is complete.  The macro
%    |\@endtheorem| ends a theorem environment and is, so far, nothing
%    but an |\endtrivlist|.  (Hence it is defined globally, and not
%    within the theorem styles.\footnote{This has to be changed as
%    soon as theorem styles that change \texttt{\bslash @endtheorem}
%    exist.  In such a case, all existing styles must be changed as
%    well since they will have to reset the macro.}) Therefore, we can
%    set it equivalent to the macro that ends the theorem set (in our
%    example, |\endLem|).  However, if some day theorem styles exist
%    that do change |\@endtheorem|, we would have to use the
%    commented-out line instead.
%    \begin{macrocode}
    \global \expandafter \let \csname end#1\endcsname \@endtheorem
%   \global\@namedef{end#1}{\@endtheorem}%
%    \end{macrocode}
%    With these commands all the required definitions are employed,
%    unless the test |\@ifdefinable| has failed.  Therefore, we
%    end the second argument of this macro and with it the definition
%    of |\@xnthm|.
%    \begin{macrocode}
   }}
%    \end{macrocode}
% \end{macro}
%
% \begin{macro}{\@ynthm}
%    The definition of |\@ynthm| is completely analogous. In
%    this case the new counter that is defined is not reset within
%    another counter; thus the definition of |\the...|
%    is simplified:
%    \begin{macrocode}
\gdef\@ynthm#1#2{\expandafter\@ifdefinable\csname #1\endcsname
   {\@definecounter{#1}%
    \expandafter\xdef\csname the#1\endcsname{\@thmcounter{#1}}%
%    \end{macrocode}
%    The rest of the definition corresponds literally to that of
%    |\@xnthm|:
%    \begin{macrocode}
    \def\@tempa{\global\@namedef{#1}}\expandafter \@tempa
     \expandafter{\csname th@\the \theorem@style \expandafter
     \endcsname \the\theorem@bodyfont \@thm{#1}{#2}}%
    \global \expandafter \let \csname end#1\endcsname \@endtheorem}}
%    \end{macrocode}
% \end{macro}
%
% \begin{macro}{\@othm}
%    The definition of |\@othm| does not contain anything new.
%    \begin{macrocode}
\gdef\@othm#1[#2]#3{%
%    \end{macrocode}
% \changes{v2.2b}{95/10/07}{Build in check for counter defined}
%    We do not define a new counter but instead use one that has
%    already been defined.  Thus the only definition we need is that
%    of this pseudo-counter (i.e.\ |\the|\meta{env.\ name}).
%    First we check if |#2| corresponds to a known counter name.
%    \begin{macrocode}
  \expandafter\ifx\csname c@#2\endcsname\relax
   \@nocounterr{#2}%
  \else
   \expandafter\@ifdefinable\csname #1\endcsname
   {\expandafter \xdef \csname the#1\endcsname
     {\expandafter \noexpand \csname the#2\endcsname}%
%    \end{macrocode}
%    All other parts of the definition can be adopted from
%    |\@xnthm|. We have to remember, though, that in this case
%    the name of the current counter and the theorem label have moved
%    to the second and third arguments.
%    \begin{macrocode}
    \def\@tempa{\global\@namedef{#1}}\expandafter \@tempa
     \expandafter{\csname th@\the \theorem@style \expandafter
     \endcsname \the\theorem@bodyfont \@thm{#2}{#3}}%
    \global \expandafter \let \csname end#1\endcsname \@endtheorem}%
  \fi}
%    \end{macrocode}
% \end{macro}
%
% \subsection{Macros that are employed in a {\envfont theorem}
%    environment}
%
% \begin{macro}{\@thm}
%    The macro |\@thm| has to increase the current counter.  Then,
%    depending on whether the environment has (or does not have) an
%    optional argument, it has to branch into either |\@begintheorem|
%    or |\@opargtheorem|.
%    \begin{macrocode}
\gdef\@thm#1#2{\refstepcounter{#1}%
%    \end{macrocode}
%    Now we start a {\envfont trivlist} environment, and give
%    |\@topsep| and |\@topsepadd| the values of the \textsf{skip}
%    registers |\theorempreskipamount| and |\theorempostskipamount|.
%    The value in |\@topsep| is the vertical space that is inserted by
%    the first (and only) |\item| in our |\trivlist| whilst
%    |\@topsepadd| is inserted by |\@endparenv| at the end of that
%    {\envfont trivlist} environment.  By using these registers, we
%    obtain the desired space around a {\envfont theorem} environment.
%    \begin{macrocode}
   \trivlist
   \@topsep \theorempreskipamount               % used by first \item
   \@topsepadd \theorempostskipamount           % used by \@endparenv
%    \end{macrocode}
%    Now we have to test whether an optional argument has been given.
% \changes{v2.0d}{89/07/18}{Forgotten percent added.}
%    \begin{macrocode}
   \@ifnextchar [%
%    \end{macrocode}
%    If there is an optional argument, we will call |\@ythm|, and
%    move the arguments read back into the input stream.
%    \begin{macrocode}
   {\@ythm{#1}{#2}}%
%    \end{macrocode}
%    If not, we call |\@begintheorem|. Its first argument is the
%    name of the theorem set (hence the second argument of
%    |\@thm|). Its second argument is the macro that produces the
%    current number.
%    \begin{macrocode}
   {\@begintheorem{#2}{\csname the#1\endcsname}\ignorespaces}}
%    \end{macrocode}
% \end{macro}
%
% \begin{macro}{\@xthm}
% \begin{macro}{\@ythm}
%    Both these macros were originally called by |\@thm|.  We do
%    not need |\@xthm| anymore, hence we reset it to
%    |\relax|. The definition of |\@ythm| has not changed at
%    all from its definition in \LaTeX. In order to make the macros
%    easier to understand, we will
%    nevertheless present it (commented out).
%    \begin{macrocode}
\global\let\@xthm\relax
% \def\@ythm#1#2[#3]{\@opargbegintheorem{#2}{\csname
%           the#1\endcsname}{#3}\ignorespaces}
%</package>
%    \end{macrocode}
%    The primitive |\ignorespaces| in |\@ythm| and
%    |\@thm| is needed to remove the spaces between the
%    |\begin{...}| and the actual text.
% \end{macro}
% \end{macro}
%
%
% \subsection{Definition of the theorem substyles}
%
% As already pointed out, the theorem substyles, defined below, are
% only loaded when necessary.  Note that all these substyles, except
% \texttt{plain}, have |\slshape| as the default body font.
%
% \subsubsection{The \texttt{plain} style}
%
% \label{dok:th@plain} As the following macros use |@|, we have
% to locally set the |\catcode| of this symbol to ``letter''.
% This happens within a group, so that we do not have to worry about
% which |\catcode| that symbol had before.
%    \begin{macrocode}
%<*thp>
\begingroup \makeatletter
%    \end{macrocode}
% Since we are now within a group, we must make all definitions
% globally.  First we make sure that \texttt{theorem.sty} is loaded.
% This will allow us to use this file as a document style option
% without having to call \texttt{theorem} itself as an option.  At the
% same time, we assure that at least version 2 is loaded, since
% |\theorem@style| was not defined in earlier versions.
%    \begin{macrocode}
\@ifundefined{theorem@style}{\input{theorem.sty}}{}
\ProvidesFile{thp.sty}
             [\FMithmInfo]
%    \end{macrocode}
%
%
% \begin{macro}{\th@plain}
% \begin{macro}{\@begintheorem}
% \begin{macro}{\@opargbegintheorem}
%    |\theoremstyle{plain}| corresponds to the original
%    definition, except that the distances to the surrounding text are
%    determined by the parameters |\theorempreskipamount| and
%    |\theorempostskipamount|.
%    First we set the default body font.
%    \begin{macrocode}
\gdef\th@plain{\normalfont\itshape
%    \end{macrocode}
%    Then we define |\@begintheorem| and
%    |\@opargbegintheorem|. These two macros define how the
%    header of a theorem is typeset.  |\@opargbegintheorem| will
%    be called if a \textsf{theorem} environment with an optional
%    argument is encountered; otherwise, the header is constructed by
%    calling |\@begintheorem|.  If one of these macros is
%    executed, we are within a \textsf{trivlist} environment started by
%    |\@thm|.  So the theorem header is produced with an
%    |\item| command.
%
%    Instead of specifying the header font directly, all standard
%    theorem styles use the |\theorem@headerfont| macro to allow
%    customization.  The extra space (|\labelsep|) is necessary
%    because of problems in the \textsf{trivlist} environment.
%    \begin{macrocode}
  \def\@begintheorem##1##2{%
        \item[\hskip\labelsep \theorem@headerfont ##1\ ##2]}%
%    \end{macrocode}
%    The definition of |\@opargbegintheorem| is completely
%    analogous. The only difference is the fact that there exists a
%    third argument (which is the optional parameter of the
%    environment and contains additional information about the
%    theorem).  Customarily we enclose it in parentheses.
%    \begin{macrocode}
\def\@opargbegintheorem##1##2##3{%
   \item[\hskip\labelsep \theorem@headerfont ##1\ ##2\ (##3)]}}
%    \end{macrocode}
% \end{macro}
% \end{macro}
% \end{macro}
%
% \noindent
% We conclude with an |\endgroup| to restore the
% |\catcode| of |@|.
%    \begin{macrocode}
\endgroup
%</thp>
%    \end{macrocode}
%
%
% \subsubsection{The \texttt{break} style}
%
% This style option is stored in the file |thb.sty|.
% For the next two lines see the documentation for |\th@plain| on
% page~\pageref{dok:th@plain}.
%    \begin{macrocode}
%<*thb>
\begingroup \makeatletter
\@ifundefined{theorem@style}{\input{theorem.sty}}{}
\ProvidesFile{thb.sty}
             [\FMithmInfo]
%    \end{macrocode}
%
% \begin{macro}{\th@break}
%    |\theoremstyle{break}| produces a line break after the name
%    of the theorem.  The font is |\slshape|.  Hence, we define
%    |\th@break| as follows:
%    \begin{macrocode}
\gdef\th@break{\normalfont\slshape
  \def\@begintheorem##1##2{\item[%
%    \end{macrocode}
%    We run into the following problem: it is not possible to create
%    the header with |\item[|\meta{title}|]| and then start
%    a new line by, for example, |\mbox{}\\|. Such a definition
%    will fail whenever a list environment follows immediately.  With
%    the above construction, the |\mbox{}| causes the switch
%    |@inlabel| (cf.\ definition of |\list| and
%    |\trivlist| in \cite{latex:source}) to be set to \texttt{false}
%    and so the following list will insert additional vertical
%    space (|\topskip|). This is quite annoying. Therefore, we
%    create the line break within the |\item|. In order to ensure
%    that the text will begin at the proper position in the following
%    line, we simply pretend that the label does not take any
%    room.\footnote{This will lead to problems whenever very high
%    symbols occurring in the line tower into the
%    heading. So, something else has to be done here sometime.}
%    \begin{macrocode}
       \rlap{\vbox{\hbox{\hskip \labelsep\theorem@headerfont ##1\ ##2}%
                   \hbox{\strut}}}]}%
%    \end{macrocode}
%    Again, the definition of |\@opargbegintheorem| is
%    completely analogous.
%    \begin{macrocode}
\def\@opargbegintheorem##1##2##3{%
  \item[\rlap{\vbox{\hbox{\hskip \labelsep \theorem@headerfont
                     ##1\ ##2\ (##3)}%
                    \hbox{\strut}}}]}}
\endgroup
%</thb>
%    \end{macrocode}
% \end{macro}
%
%
% \subsubsection{The \texttt{changebreak} style}
%
%    \begin{macrocode}
% This style option is stored in the file |thcb.sty|.
%    \begin{macrocode}
%<*thcb>
\begingroup \makeatletter
\@ifundefined{theorem@style}{\input{theorem.sty}}{}
\ProvidesFile{thcb.sty}
             [\FMithmInfo]
%    \end{macrocode}
%
% \begin{macro}{\th@changebreak}
%    The \texttt{change-break} theorem style is like \texttt{break}
%    but with interchange of theorem name and theorem number.  Thus we
%    define |\th@changebreak| as follows:
%    \begin{macrocode}
\gdef\th@changebreak{\normalfont\slshape
  \def\@begintheorem##1##2{\item
       [\rlap{\vbox{\hbox{\hskip\labelsep \theorem@headerfont ##2\ ##1}%
                    \hbox{\strut}}}]}%
\def\@opargbegintheorem##1##2##3{%
  \item[\rlap{\vbox{\hbox{\hskip\labelsep \theorem@headerfont
                     ##2\ ##1\ (##3)}%
                    \hbox{\strut}}}]}}
\endgroup
%</thcb>
%    \end{macrocode}
% \end{macro}
%
%
% \subsubsection{The \texttt{change} style}
%
% This style option is stored in the file |thc.sty|.
%    \begin{macrocode}
%<*thc>
\begingroup \makeatletter
\@ifundefined{theorem@style}{\input{theorem.sty}}{}
\ProvidesFile{thc.sty}
             [\FMithmInfo]
%    \end{macrocode}
%
% \begin{macro}{\th@change}
%    The \texttt{change} theorem style corresponds to the
%    \texttt{change break} style without a linebreak after the header.
%    To say it in another way, it's the same as the \texttt{plain}
%    style but with number and name interchanged and |\slshape| as the
%    default font.
%    \begin{macrocode}
\gdef\th@change{\normalfont\slshape
  \def\@begintheorem##1##2{\item
       [\hskip\labelsep \theorem@headerfont ##2\ ##1]}%
\def\@opargbegintheorem##1##2##3{%
  \item[\hskip\labelsep \theorem@headerfont ##2\ ##1\ (##3)]}}
\endgroup
%</thc>
%    \end{macrocode}
% \end{macro}
%
%
% \subsubsection{The \texttt{marginbreak} style}
%
% This style option is the one used most often at Mainz.  It is saved
% in the file |thmb.sty|.
%    \begin{macrocode}
%<*thmb>
\begingroup \makeatletter
\@ifundefined{theorem@style}{\input{theorem.sty}}{}
\ProvidesFile{thmb.sty}
             [\FMithmInfo]
%    \end{macrocode}
%
% \begin{macro}{\th@marginbreak}
%    The \texttt{margin break} style is nearly the same as the
%    \texttt{change break} style. The only difference is the placement
%    of the theorem number.  We use |\llap| to place it in the left
%    margin.
%
%    In this style |\labelsep| denotes the separation between the
%    number and the text.
%    \begin{macrocode}
\gdef\th@marginbreak{\normalfont\slshape
  \def\@begintheorem##1##2{\item
       [\rlap{\vbox{\theorem@headerfont
                    \hbox{\llap{##2}\hskip\labelsep ##1}%
                    \hbox{\strut}}}]}%
\def\@opargbegintheorem##1##2##3{%
  \item[\rlap{\vbox{\theorem@headerfont
                    \hbox{\llap{##2}\hskip\labelsep ##1\ (##3)}%
                    \hbox{\strut}}}]}}
\endgroup
%</thmb>
%    \end{macrocode}
% \changes{v2.1c}{90/02/17}{`endgroup outside macrocode env.}
% \end{macro}
%
%
%
% \subsubsection{The \texttt{margin} style}
%
% This style option is stored in the file |thm.sty|.
%    \begin{macrocode}
%<*thm>
\begingroup \makeatletter
\@ifundefined{theorem@style}{\input{theorem.sty}}{}
\ProvidesFile{thm.sty}
             [\FMithmInfo]
%    \end{macrocode}
%
% \begin{macro}{\th@margin}
%    Again this is only a variant of the theorem styles described
%    above without any new ideas.
%    \begin{macrocode}
\gdef\th@margin{\normalfont\slshape
  \def\@begintheorem##1##2{\item
       [\theorem@headerfont \llap{##2}\hskip\labelsep ##1]}%
\def\@opargbegintheorem##1##2##3{%
  \item[\theorem@headerfont \llap{##2}\hskip\labelsep ##1\ (##3)]}}
\endgroup
%</thm>
%    \end{macrocode}
% \end{macro}
%
%
%
%
% \subsection{Final Definitions}
%
% \begin{macro}{\theorempreskipamount}
% \begin{macro}{\theorempostskipamount}
%    The \textsf{skip} parameters that regulate the vertical empty space
%    before and after the {\envfont theorem} environment have
%    to be allocated as well.
% \changes{v2.0f}{89/07/20}{Unnecessary `global' removed.}
%    \begin{macrocode}
%<*package>
\newskip\theorempreskipamount
\newskip\theorempostskipamount
%    \end{macrocode}
%    Since we have used the same values for all theorem sets, we now can
%    assign them.
%    \begin{macrocode}
\global\setlength\theorempreskipamount{12pt plus 5pt minus 3pt}
\global\setlength\theorempostskipamount{8pt plus 3pt minus 1.5pt}
%    \end{macrocode}
% \end{macro}
% \end{macro}
%
%
% \begin{macro}{\@endtheorem}
%    The same holds for the macro |\@endtheorem|, which ends a
%    {\envfont theorem} environment. Since it is the same for all
%    theorem sets, it is removed from the macros |\th@|\meta{style}.
%    It simply ends the {\envfont trivlist} environment, which was
%    begun in |\@thm|.
%    \begin{macrocode}
\global\let\@endtheorem=\endtrivlist
%    \end{macrocode}
% \end{macro}
%
%
%
% \begin{macro}{\@preamblecmds}
%    All macros defined above are to be used only in the preamble.
%    Therefore, we insert them in |\@preamblecmds| which will disable
%    them at begin document. This is done by the internal \LaTeXe{}
%    command |\@onlypreamble|.
% \changes{v2.0g}{89/09/19}{`theoremheaderfont and `theorembodyfont
%    added.}
% \changes{v2.2a}{94/02/03}{Use \LaTeXe{} interface}
%    \begin{macrocode}
\@onlypreamble\@xnthm
\@onlypreamble\@ynthm
\@onlypreamble\@othm
\@onlypreamble\newtheorem
\@onlypreamble\theoremstyle
\@onlypreamble\theorembodyfont
\@onlypreamble\theoremheaderfont
%    \end{macrocode}
% \end{macro}
%
% Finally we declare the \texttt{plain} theorem style to be the default.
% \changes{v2.1b}{90/06/04}{Fixed missing `item by declaring default
%    style.}
%    \begin{macrocode}
\theoremstyle{plain}
%</package>
%    \end{macrocode}
%
% \Finale

\endinput
}{}
\ProvidesFile{thc.sty}
             [\FMithmInfo]
%    \end{macrocode}
%
% \begin{macro}{\th@change}
%    The \texttt{change} theorem style corresponds to the
%    \texttt{change break} style without a linebreak after the header.
%    To say it in another way, it's the same as the \texttt{plain}
%    style but with number and name interchanged and |\slshape| as the
%    default font.
%    \begin{macrocode}
\gdef\th@change{\normalfont\slshape
  \def\@begintheorem##1##2{\item
       [\hskip\labelsep \theorem@headerfont ##2\ ##1]}%
\def\@opargbegintheorem##1##2##3{%
  \item[\hskip\labelsep \theorem@headerfont ##2\ ##1\ (##3)]}}
\endgroup
%</thc>
%    \end{macrocode}
% \end{macro}
%
%
% \subsubsection{The \texttt{marginbreak} style}
%
% This style option is the one used most often at Mainz.  It is saved
% in the file |thmb.sty|.
%    \begin{macrocode}
%<*thmb>
\begingroup \makeatletter
\@ifundefined{theorem@style}{% \iffalse meta-comment
%
% Copyright (C) 1993-2024
%
% The LaTeX Project and any individual authors listed elsewhere
% in this file.
%
% This file is part of the Standard LaTeX `Tools Bundle'.
% -------------------------------------------------------
%
% It may be distributed and/or modified under the
% conditions of the LaTeX Project Public License, either version 1.3c
% of this license or (at your option) any later version.
% The latest version of this license is in
%    https://www.latex-project.org/lppl.txt
% and version 1.3c or later is part of all distributions of LaTeX
% version 2005/12/01 or later.
%
% The list of all files belonging to the LaTeX `Tools Bundle' is
% given in the file `manifest.txt'.
%
% \fi
%
%%
%
%
\def\FMithmInfo{2023/07/05 v2.2c Theorem extension package (FMi)}
%
% \ProvidesFile{theorem.dtx}[\FMithmInfo]
%\iffalse   % this is a METACOMMENT !
%%
%% Package `theorem' to use with LaTeX2e
%% Copyright (C) 1989-1995 Frank Mittelbach, all rights reserved.
%<+package>\@ifundefined{theorem@style}{}{\endinput}
%<+package>\NeedsTeXFormat{LaTeX2e}[1995/06/01]
%<+package>\ProvidesPackage{theorem}
%<+package>         [\FMithmInfo]
%
% \fi
%
% \changes{v1.0}{87/05/07}{First draft.}
% \changes{v1.0a}{87/05/07}{`thistheoremstyle' added.}
% \changes{v1.1a}{87/07/29}{New concept: redefine the internal
%                          \protect\LaTeX{} commands `@xnthm' etc.}
% \changes{v1.1b}{87/07/31}{`theoremfont' used to specify font.}
% \changes{v1.2a}{88/01/08}{`@endtheorem' now globally defined.}
% \changes{v1.2b}{88/01/11}{`newtheorem' only allowed in the preamble.}
% \changes{v1.3a}{88/02/14}{`theoremfont' now a toks register.}
%
% \changes{v2.0a}{88/11/16}{New implementation and new documentation.}
% \changes{v2.0a}{88/11/16}{No layout changes but internal modifications
%                           old substyles won't work.}
% \changes{v2.0b}{89/01/30}{`global' used for all definitions.}
% \changes{v2.0c}{89/06/15}{Documentation translated to English.
%                            Thanks to LLL and Nina}
% \changes{v2.1a}{90/02/15}{Support for new font selection.}
% \changes{v2.2a}{94/02/03}{Cleanup for LaTeX2e}
%
% \DoNotIndex{\@endparenv,\@ifdefinable,\@ifnextchar,\@ifundefined}
% \DoNotIndex{\@namedef,\@nameuse,\@spaces,\@tempa,\@warning}
% \DoNotIndex{\ }
% \DoNotIndex{\begingroup,\bfseries}
% \DoNotIndex{\csname}
% \DoNotIndex{\def,\do,\docdate}
% \DoNotIndex{\endcsname,\endgroup,\endinput,\endtrivlist,\expandafter}
% \DoNotIndex{\filedate,\fileversion}
% \DoNotIndex{\gdef,\global}
% \DoNotIndex{\hbox,\hskip}
% \DoNotIndex{\ignorespaces,\input,\itshape,\item}
% \DoNotIndex{\labelsep,\let,\llap}
% \DoNotIndex{\makeatletter}
% \DoNotIndex{\noexpand}
% \DoNotIndex{\refstepcounter,\relax,\rlap}
% \DoNotIndex{\setlength,\slshape,\space,\string,\strut}
% \DoNotIndex{\the,\trivlist,\typeout}
% \DoNotIndex{\vbox}
% \DoNotIndex{\xdef}
%
% \def\deflabel#1{\ttfamily #1\hfill}
% \def\deflist#1{\begin{list}{}{\settowidth\labelwidth{\ttfamily #1}%
%                               \setlength\leftmargin\labelwidth
%                               \addtolength\leftmargin\labelsep
%                               \let\makelabel\deflabel}}
% \def\enddeflist{\end{list}}
% \def\envfont{\normalfont\ttfamily}
%
% \MakeShortVerb{\|}
%
% \newcommand{\TUB}{{\sl TUGboat\/}}
%
% \GetFileInfo{theorem.dtx}
% \title{An Extension of the
%        \LaTeX{} theorem environment\thanks{This file
%        has version number \fileversion, last
%        revised \filedate.}}
% \author{Frank Mittelbach}
%
% \MaintainedByLaTeXTeam{tools}
% \maketitle
%
% \begin{abstract}
%    The macros described in this paper yield an extension of the
%    \LaTeX{} theorem mechanism. It is designed to satisfy the
%    different requirements of various journals. Thus, the
%    layout of the ``theorems'' can be manipulated by determining a
%    ``style''. This article describes not only the use, but
%    also the definition, of the necessary macros.
% \end{abstract}
%
% \section*{Preface to version 2.2}
%
% For \LaTeXe{} this package did not need any fundamental changes. I
% only modified the messages generated so that theorem layout styles
% will show up with the |\listfiles| command and cleaned the section
% on the New Font Selection Scheme since this is now included in
% \LaTeX.
%
% \section*{Preface to version 2.1}
%
% This version is identical to 2.0g described in \TUB\ 10\#3 except
% for some internal defaults which are now set depending on the used
% font selection scheme.
%
% This was done to avoid unpleasant surprises if the new font selection
% scheme is in force. For further details see
% section~\ref{sec:newfonts} and \cite{bk:GMS94}.
%
% \section{Introduction}
%
% For our purposes here, ``theorems'' are labelled enunciations,
% often set off from the main text by extra space and a font change.
% Theorems, corollaries, conjectures, definitions, and remarks
% are all instances of ``theorems''.  The ``header'' of these structures
% is composed of a label (such as \textsc{Theorem} or \textsc{Remark})
% and a number which serializes an item in the sequence of items
% with the same label.
%
% Shortly after the introduction of \LaTeX{} at the Fachbereich
% Mathematik in Mainz, the desire to manipulate the layout of
% ``theorems'' arose.  In Mainz, the following two conventions came
% into general use:
% \begin{enumerate}
%   \item The number of the theorem is shown in the margin.
%   \item There is a line break at
%         the end of the theorem header.
% \end{enumerate}
%
% Additionally, some journals require different formats which depend on
% the ``sort of theorem'': e.g.\ often remarks and definitions are set
% in |\upshape|, while |\itshape| is employed for main theorems.
%
% Confronted with these requirements, a theorem environment was
% developed in Mainz which allows separate determination of the
% layout of the ``theorems sets'', comparable to |\pagestyle|.
%
%
% \section{The user interface}
% \subsection{Defining new theorem sets}
%
% \DescribeMacro\newtheorem
% As in the original \LaTeX{} version, the command |\newtheorem|
% defines a new ``theorem set'' or ``theorem-like structure''.
% Two required arguments name the new environment and give the text
% to be typeset with each instance of the new ``set'', while
% an optional argument determines how
% the ``set'' is enumerated:
% \begin{description}
%    \item[\ttfamily \bslash newtheorem\{foo\}\{bar\}]
%       The theorem set {\envfont foo} (whose name is \texttt{bar})
%       uses its own counter.
%    \item[\ttfamily \bslash newtheorem\{foo2\}{[foo]}\{bar2\}]
%       The theorem set {\envfont foo2} (printed name \texttt{bar2})
%       uses the same counter as the theorem set \texttt{foo}.
%    \item[\ttfamily \bslash newtheorem\{foo3\}\{bar3\}{[section]}]
%       The theorem set {\envfont foo3} (printed name \texttt{bar3}) is
%       enumerated within the counter \texttt{section}, i.e.\ with every
%       new |\section| the enumeration begins again with 1, and
%       the enumeration is composed from the section-number and the
%       theorem counter itself.
% \end{description}
%
% \DescribeMacro\theoremstyle
% Additionally, the command |\theoremstyle| can define the layout
% of various, or all, theorem sets.  It should be noted that any
% theorem set defined by |\newtheorem| is typeset in the
% |\theoremstyle| that is current at the time of the definition.
% Thus, the following
% \begin{verbatim}
%    \theoremstyle{break}        \newtheorem{Cor}{Corollary}
%    \theoremstyle{plain}        \newtheorem{Exa}{Example}[section]
%\end{verbatim}                             ^^A No blank at line start
% leads to the result that the set {\envfont Cor} is formatted in the
% style \texttt{break}, while the set {\envfont Exa} and all the
% following ones are formatted in the style \texttt{plain}, unless
% another |\theoremstyle| follows.  Since the definitions installed by
% |\newtheorem| are global, one also can limit |\theoremstyle| locally
% by grouping braces.
%
% \DescribeMacro\theorembodyfont
% The choice of the font for the theorem body is completely
% independent of the chosen |\theoremstyle|; this has proven to
% be very advantageous.  For example,
% \begin{verbatim}
%   {\theorembodyfont{\upshape}          \newtheorem{Rem}{Remark}}
%\end{verbatim}
% defines a theorem set {\envfont Rem}, which will be set in
% |\upshape| in the current layout (which in our example is
% \texttt{plain}).  As with |\theoremstyle|, the |\theorembodyfont|
% chosen is that current at the time of |\newtheorem|.  If
% |\theorembodyfont| is not specified or one defines
% |\theorembodyfont{}|, then the font used will be that defined by the
% |\theoremstyle|.
%
% \DescribeMacro\theoremheaderfont It is also possible to customize
% the font used for the theorem headers.  This is, however, a global
% declaration, and therefore there should be at most one
% |\theoremheaderfont| declaration in the preamble.\footnote{If it is
% actually necessary to have different header fonts, one has to define
% new theorem styles (substituting the desired font) or specify the
% information directly in the \texttt{\bslash newtheorem} declaration
% (the unclean variant).}
%
% \DescribeMacro\theorempreskipamount
% \DescribeMacro\theorempostskipamount Two additional parameters
% affect the vertical space around the theorem environments:
% |\theorempreskipamount| and |\theorempostskipamount| define,
% respectively, the spacing before and after such an environment.
% These parameters apply for all theorem sets and can be manipulated
% with the ordinary length macros.  They are rubber lengths
% (`\textsf{skips}'), and therefore can contain \texttt{plus} and
% \texttt{minus} parts.
%
% Since the definition of theorem sets should---most sensibly---be
% placed in the preamble, we only allow installation there.
% It is therefore possible to release the memory used here after
% |\begin{document}|, in order to make room for other
% applications.
%
%
% \subsection{Existing theorem styles}
%
% The following theorem styles exist to date:
% \begin{deflist}{marginbreak}
%    \item[plain]
%       This theorem style emulates the original \LaTeX{} definition,
%       except that additionally the parameters
%       |\theorem...skipamount| are used.
%    \item[break]
%       In this style, the theorem header is followed by a line
%       break.
%    \item[marginbreak]
%       The theorem number is set in the margin, and there is a line
%       break as in \texttt{break}.
%    \item[changebreak]
%       Like \texttt{break}, but with header number and text
%       interchanged.
%    \item[change]
%       Header number and text are interchanged, without a line break.
%    \item[margin]
%       The number is set in the left margin, without a line break.
% \end{deflist}
% All styles (except \texttt{plain}) select |\slshape| as the default
% |\theorembodyfont|.
%
%
% \subsection{Examples}
%
% Given the above theorem sets {\envfont Cor}, {\envfont Exa}
% and {\envfont Rem},
% suppose that the preamble also contains the declarations:
% \begin{verbatim}
%    \theoremstyle{marginbreak}   \newtheorem{Lem}[Cor]{Lemma}
%    \theoremstyle{change}
%    \theorembodyfont{\itshape}        \newtheorem{Def}[Cor]{Definition}
%
%    \theoremheaderfont{\scshape}
%\end{verbatim}
% Then the following are some typical examples of the typeset output
% resulting from their use.
%
% \begin{Cor}
%    This is a sentence typeset in the theorem environment {\envfont
%    Cor}.
% \end{Cor}
%
% \begin{Exa}
%    This is a sentence typeset in the theorem environment {\envfont
%    Exa}.
% \end{Exa}
%
% \begin{Rem}
%    This is a sentence typeset in the theorem environment {\envfont
%    Rem}.
% \end{Rem}
%
% \begin{Lem}[Ben User]
%    This is a sentence typeset in the theorem environment {\envfont
%    Lem}.
% \end{Lem}
%
% \begin{Def}[Very impressive Definition]
%    This is a sentence typeset in the theorem environment {\envfont
%    Def}.
% \end{Def}
%
% The last two examples show the effect of the optional argument to
% a theorem environment (it is the text typeset in parentheses).
%
%
% \section{Special Considerations}
% \label{sec:newfonts}
%
% Theoremheader and body are implemented as a unit. This means that
% the |\theoremheaderfont|  will inherit characteristics of the
% |\theorembodyfont| in \LaTeXe. Thus, if for example
% |\theorembodyfont| is |\itshape| and |\theoremheaderfont|
% is |\bfseries| the font selected for the header will have the
% characteristics `bold extended italic'. If this is not desired one
% should set the |\theoremheaderfont| to something like
% \begin{verbatim}
%      \theoremheaderfont{\normalfont\bfseries}
%\end{verbatim}
% i.e.\ supplying all necessary font information explicitly.
%
%
%
% \section{Acknowledgements}
%
% The publication of this set of macros was only possible with the
% help of Christina Busse (translating the manuscript into English),
% Joachim Pense (playing the r\^ole of typist), Chris Rowley
% (looking everything over) and many others providing useful
% suggestions.
%
% \MaybeStop{%
%    \begin{thebibliography}{9}
%       \bibitem{bk:GMS94} \textsc{M.~Goossens}, \textsc{F.~Mittelbach}
%         and \textsc{A.~Samarin}.
%         \newblock The \LaTeX{} Companion.
%         \newblock
%         Addison-Wesley, Reading, Massachusetts, 1994.
%       \bibitem{latex:source} \textsc{Lamport, Leslie.} \newblock
%                               \texttt{latex.tex}, version 2.09, date
%                               Feb. 1990.
%    \end{thebibliography}
%    \PrintIndex
%    \PrintChanges}
%
%
% \section{The documentation driver file}
%
% The next bit of code contains the documentation driver file for
% \TeX{}, i.e., the file that will produce the documentation you are
% currently reading. It will be extracted from this file by the
% \texttt{docstrip} program. Since it is the first code in the file
% one can alternatively process this file directly with \LaTeXe{} to
% obtain the documentation.
%
%    \begin{macrocode}
%<*driver>
\documentclass{ltxdoc}

\usepackage{theorem}

%<+driver>% The next few lines define theorem sets which are used
%<+driver>% in the example section of the documentation.

\theoremstyle{break}       \newtheorem{Cor}{Corollary}
\theoremstyle{plain}       \newtheorem{Exa}{Example}[section]
{\theorembodyfont{\upshape}\newtheorem{Rem}{Remark}}
\theoremstyle{marginbreak} \newtheorem{Lem}[Cor]{Lemma}
\theoremstyle{change}
\theorembodyfont{\itshape} \newtheorem{Def}[Cor]{Definition}

\theoremheaderfont{\scshape}

\RecordChanges

\begin{document}
   \DocInput{theorem.dtx}
\end{document}
%</driver>
%    \end{macrocode}
%
% \section{Definition of the Macros}
%
% If the file has been loaded before, we abort immediately. If not the
% package announces itself (this is actually done at the very top of
% the file---the way it is done isn't good style so don't copy it).
% \changes{v2.0e}{89/07/19}{Spaces removed from `typeout'.}
% \changes{v2.2b}{95/11/19}{Announce moved to top of file}
%    \begin{macrocode}
%<*package>
%\@ifundefined{theorem@style}{}{\endinput}
%\def\FMithmInfo{1995/11/19 v2.2b Theorem extension package (FMi)}
%\ProvidesPackage{theorem}[\FMithmInfo]
%    \end{macrocode}
%
%
% \subsection{Definition of theorem styles and fonts}
%
% All the definitions in this file are done globally to allow
% inputting this file inside a group.
%
% \begin{macro}{\theoremstyle}
%    Before a theorem style can be installed, the chosen style must be
%    known. For that reason, we must test to see that
%    |\th@|\meta{style} is known or, more precisely, that it is
%    different from |\relax|. If the style is not known then
%    |\th@plain| is used.
%    \begin{macrocode}
\gdef\theoremstyle#1{%
   \@ifundefined{th@#1}{\@warning
          {Unknown theoremstyle `#1'. Using `plain'}%
          \theorem@style{plain}}%
%    \end{macrocode}
%    We save the theorem style to be used in the token register
%    |\theorem@style|.
%    \begin{macrocode}
      {\theorem@style{#1}}%
%    \end{macrocode}
%    Now we ``evaluate'' the theorem style: this means, we call the
%    macro |\th@|\meta{style} which will activate the relevant
%    definitions which are contained in a separate file.  This is done
%    in a group to suppress changes to the current font.  This could
%    otherwise pose problems together with the new font selection
%    scheme\footnote{When I printed the original article using the new
%    font selection scheme I ended with a document with slanted
%    typefaces (text headings and all) simply because one of the
%    theorem styles used \texttt{\bslash sl} at toplevel.} if the
%    |\th@|\meta{style} is evaluated a second time.
% \changes{v2.1a}{90/02/15}{Group added around `th@..}
% \changes{v2.1a}{90/02/15}{Reseting of `..@begintheorem removed.}
%    \begin{macrocode}
      \begingroup
        \csname th@\the\theorem@style \endcsname
      \endgroup}
%    \end{macrocode}
% \end{macro}
%
%
% \begin{macro}{\@begintheorem}
% \changes{v2.1a}{90/02/15}{Relaxed at toplevel.}
% \begin{macro}{\@opargbegintheorem}
% \changes{v2.1a}{90/02/15}{Relaxed at toplevel.}
%    We reset |\@begintheorem| and |\@opargbegintheorem| to
%    |\relax| since these commands are no longer necessary at
%    toplevel. This will save a few tokens.
%    \begin{macrocode}
\global\let\@begintheorem\relax
\global\let\@opargbegintheorem\relax
%    \end{macrocode}
% \end{macro}
% \end{macro}
%
% \begin{macro}{\theorem@style}
%    Obviously the \textsf{token} register used above has to be
%    allocated. To assure the utmost compatibility with the
%    original \LaTeX{} definition, we set the default theorem style to
%    \texttt{plain}, which implements the usual \LaTeX{} convention.
% \changes{v2.0f}{89/07/20}{Unnecessary `global' removed.}
%    \begin{macrocode}
\newtoks\theorem@style
\global\theorem@style{plain}
%    \end{macrocode}
% \end{macro}
%
% \begin{macro}{\theorembodyfont}
% \begin{macro}{\theorem@bodyfont}
%    If the |\theorembodyfont| is set by the user then it should not
%    interact with the default font set in the theorem style. When the
%    new font selection is in force this may happen if, for example,
%    the default is |\itshape| and the new |\theorembodyfont| is
%    |\sffamily|. So we add a |\reset@font| command in front of the
%    user definition.
% \changes{v2.0f}{89/07/20}{Unnecessary `global' removed.}
% \changes{v2.0g}{89/09/19}{`theoremfont to `theorembodyfont.}
% \changes{v2.1c}{92/06/22}{`reset@font added and logic changed.}
% \changes{v2.1d}{92/10/28}{Reset to empty if arg is empty}
%    \begin{macrocode}
\gdef\theorembodyfont#1{%
%    \end{macrocode}
%    We check if the argument supplied is empty and if so put nothing
%    into the |\theorem@bodyfont| token register to allow for
%    |\theorembodyfont{}| as a mean of using the default of the
%    current |\theoremstyle|.
%    \begin{macrocode}
   \def\@tempa{#1}%
   \ifx\@tempa\@empty
    \theorem@bodyfont{}%
   \else
    \theorem@bodyfont{\reset@font#1}%
   \fi
   }
\newtoks\theorem@bodyfont
\global\theorem@bodyfont{}
%    \end{macrocode}
% \end{macro}
% \end{macro}
%
%
% \begin{macro}{\theoremheaderfont}
% \changes{v2.0g}{89/09/19}{Macro added.}
%    The font for the theorem headers is handled differently because
%    this definition applies to all theorem styles.
%    \begin{macrocode}
\gdef\theoremheaderfont#1{\gdef\theorem@headerfont{#1}%
%    \end{macrocode}
%    After using the macro once it is redefined to produce an error
%    message.
%    \begin{macrocode}
       \gdef\theoremheaderfont##1{%
        \typeout{\string\theoremheaderfont\space should be used
                 only once.}}}
%    \end{macrocode}
% \end{macro}
%
% \begin{macro}{\theorem@headerfont}
% \changes{v2.0g}{89/09/19}{Macro added.}
%    To set the |\theorem@headerfont| default we first test if
%    the new fontselection scheme is in force.
% \changes{v2.1a}{90/02/15}{Support for new font selection.}
%    \begin{macrocode}
\ifx\upshape\undefined
%    \end{macrocode}
%    If not we define it to expand into |\bfseries|. We don't use
%    |\let| just in case a following style option redefines this
%    macro.
%    \begin{macrocode}
\gdef\theorem@headerfont{\bfseries}
%    \end{macrocode}
%    Otherwise we reset the current shape before calling |\bfseries|.
%    \begin{macrocode}
\else \gdef\theorem@headerfont{\normalfont\bfseries}\fi
%    \end{macrocode}
% \end{macro}
%
% \begin{macro}{\th@plain}
% \begin{macro}{\th@break}
% \begin{macro}{\th@marginbreak}
% \begin{macro}{\th@changebreak}
% \begin{macro}{\th@change}
% \begin{macro}{\th@margin}
%    The different styles are defined in macros such as
%    |\th@plain|.  Since memory space is precious in
%    ``non-Big-versions'', we have to avoid offering too many unused
%    definitions.  Therefore we define these styles in separate files
%    that can be loaded on demand.  Thus the commands themselves only
%    load these files. We use |\@input@|, a \LaTeXe{} internal command
%    that ensures that the file will be listed with |\listfiles|.
% \changes{v2.2a}{94/02/02}{Use `@input@ to load theorem layout files}
%    \begin{macrocode}
\gdef\th@plain{\@input@{thp.sty}}
\gdef\th@break{\@input@{thb.sty}}
\gdef\th@marginbreak{\@input@{thmb.sty}}
\gdef\th@changebreak{\@input@{thcb.sty}}
\gdef\th@change{\@input@{thc.sty}}
\gdef\th@margin{\@input@{thm.sty}}
%    \end{macrocode}
%    This list will be expanded when new styles become
%    available.  For testing,  just append new theorem substyles
%    as document options.
% \end{macro}
% \end{macro}
% \end{macro}
% \end{macro}
% \end{macro}
% \end{macro}
%
%
%
% \subsection{Definition of a new theorem set}
%
% As already pointed out, a new theorem environment can be defined in
% three different ways:
% \begin{verbatim}
%    \newtheorem{Lem}{Lemma}
%    \newtheorem{Lem}{Lemma}[section]
%    \newtheorem{Lem}[Theorem]{Lemma}
%\end{verbatim}
% The function of the macro |\newtheorem| is to recognize these cases
% and then to branch into one of the three macros |\@ynthm|, |\@xnthm|
% or |\@othm|.  This mechanism is adopted unchanged from
% \cite{latex:source}; the essential point here is that, for example,
% in the second case, the arguments \texttt{Lem}, \texttt{Lemma} and
% \texttt{section} are passed over to the macro |\@xnthm|.
%
% We inspect this case first because the others present fewer problems,
% and thus are easily derived from this one.
%
% \begin{macro}{\@xnthm}
%    For our example arguments, the macro |\@xnthm| must fulfill
%    the following:
%    \begin{itemize}
%       \item
%          Define a new \LaTeX-counter `Lem'
%       \item
%          reset this counter within a |\section|
%       \item
%          define the macro |\theLem|
%       \item
%          define the environment macros |\Lem| and
%          |\endLem| using the current
%          |\theoremstyle| and |\theorem@bodyfont|.
%    \end{itemize}
%    Obviously, all this should happen only if the first argument
%    of |\@xnthm| (i.e.\ \texttt{Lem} in our example) is chosen so
%    as not to conflict with any previously defined commands or
%    environments.  This test is performed by the \LaTeX{} macro
%    |\@ifdefinable|.
%    \begin{macrocode}
\gdef\@xnthm#1#2[#3]{\expandafter\@ifdefinable\csname #1\endcsname
%    \end{macrocode}
%    Therefore, the first argument of |\@ifdefinable| is the
%    expansion (in the example, |\Lem|) of
%    |\csname||#1\endcsname|.  The second argument is
%    executed only if the test has been completed successfully.
%    \begin{macrocode}
   {%
%    \end{macrocode}
%    Now we define the new counter. The names of the \LaTeX{}
%    macros employed should speak for themselves:
% \changes{v2.2b}{1995/10/07}{Use \cs{@newctr}}
% \changes{v2.2c}{1995/11/23}{Use \cs{@newctr} properly.}
%    \begin{macrocode}
    \@definecounter{#1}\@newctr{#1}[#3]%
%    \end{macrocode}
%    Using |\@newctr| will give a proper error message if the counter
%    in |#3| is not defined.
%    In defining `|\theLem|' we must generate the desired macro
%    name by use of |\expandafter| and |\csname|.
%    \begin{macrocode}
    \expandafter\xdef\csname the#1\endcsname
%    \end{macrocode}
%    An |\xdef| is used in order to make the definition global,
%    and to ensure that it contains the replacement texts of
%    |\@thmcountersep| and |\@thmcounter|.\footnote{These
%    two macros can be defined by the document style.  Their default
%    values produce a `\texttt{.}' as separation and an arabic
%    representation of the number.} However, not everything should be
%    expanded. For example, it saves space to use |\thesection|
%    instead of its---at times---lengthy expansion.
%    \begin{macrocode}
      {\expandafter \noexpand \csname the#3\endcsname
       \@thmcountersep \@thmcounter{#1}}%
%    \end{macrocode}
%    Thus with the defaults of \LaTeX{}, |\theLem| would be
%    replaced by the command sequence |\thesection.\arabic{Lem}|.
%
%    We will now look at the definition of the macro which is executed
%    at the beginning of the actual environment (in our example this
%    macro is |\Lem|). It should be noted that we use an
%    ``|\expandafter| trick'' to expand only certain parts of the
%    replacement text at the time of the definition.
%    \begin{macrocode}
    \def\@tempa{\global\@namedef{#1}}%
    \expandafter \@tempa \expandafter{%
%    \end{macrocode}
%    First, the macro that contains the current definitions of
%    |\@begintheorem| and |\@opargtheorem| should be called
%    up.  The name of this macro---as is already known---has the form
%    |\th@|\meta{theorem\ style}; therefore, it must be called by
%    \begin{macrocode}
      \csname th@\the \theorem@style
%    \end{macrocode}
%    In addition the default theorem font should be changeable, i.e.\
%    we have to insert the contents of |\theorem@bodyfont|. For
%    that reason, we expand even further, beyond |\endcsname|,
%    and thus insert the contents of the \textsf{token} register
%    |\theorem@bodyfont| in the replacement text.
%    \begin{macrocode}
            \expandafter \endcsname \the \theorem@bodyfont
%    \end{macrocode}
%    Now it is time to call the macro |\@thm| which takes over
%    the further processing.  It has two arguments: the current
%    counter name (in our example, \texttt{Lem}), and the text of the
%    label (in our example, \texttt{Lemma}).
%    \begin{macrocode}
     \@thm{#1}{#2}}%
%    \end{macrocode}
%    With this, the `sub-definition' is complete.  The macro
%    |\@endtheorem| ends a theorem environment and is, so far, nothing
%    but an |\endtrivlist|.  (Hence it is defined globally, and not
%    within the theorem styles.\footnote{This has to be changed as
%    soon as theorem styles that change \texttt{\bslash @endtheorem}
%    exist.  In such a case, all existing styles must be changed as
%    well since they will have to reset the macro.}) Therefore, we can
%    set it equivalent to the macro that ends the theorem set (in our
%    example, |\endLem|).  However, if some day theorem styles exist
%    that do change |\@endtheorem|, we would have to use the
%    commented-out line instead.
%    \begin{macrocode}
    \global \expandafter \let \csname end#1\endcsname \@endtheorem
%   \global\@namedef{end#1}{\@endtheorem}%
%    \end{macrocode}
%    With these commands all the required definitions are employed,
%    unless the test |\@ifdefinable| has failed.  Therefore, we
%    end the second argument of this macro and with it the definition
%    of |\@xnthm|.
%    \begin{macrocode}
   }}
%    \end{macrocode}
% \end{macro}
%
% \begin{macro}{\@ynthm}
%    The definition of |\@ynthm| is completely analogous. In
%    this case the new counter that is defined is not reset within
%    another counter; thus the definition of |\the...|
%    is simplified:
%    \begin{macrocode}
\gdef\@ynthm#1#2{\expandafter\@ifdefinable\csname #1\endcsname
   {\@definecounter{#1}%
    \expandafter\xdef\csname the#1\endcsname{\@thmcounter{#1}}%
%    \end{macrocode}
%    The rest of the definition corresponds literally to that of
%    |\@xnthm|:
%    \begin{macrocode}
    \def\@tempa{\global\@namedef{#1}}\expandafter \@tempa
     \expandafter{\csname th@\the \theorem@style \expandafter
     \endcsname \the\theorem@bodyfont \@thm{#1}{#2}}%
    \global \expandafter \let \csname end#1\endcsname \@endtheorem}}
%    \end{macrocode}
% \end{macro}
%
% \begin{macro}{\@othm}
%    The definition of |\@othm| does not contain anything new.
%    \begin{macrocode}
\gdef\@othm#1[#2]#3{%
%    \end{macrocode}
% \changes{v2.2b}{95/10/07}{Build in check for counter defined}
%    We do not define a new counter but instead use one that has
%    already been defined.  Thus the only definition we need is that
%    of this pseudo-counter (i.e.\ |\the|\meta{env.\ name}).
%    First we check if |#2| corresponds to a known counter name.
%    \begin{macrocode}
  \expandafter\ifx\csname c@#2\endcsname\relax
   \@nocounterr{#2}%
  \else
   \expandafter\@ifdefinable\csname #1\endcsname
   {\expandafter \xdef \csname the#1\endcsname
     {\expandafter \noexpand \csname the#2\endcsname}%
%    \end{macrocode}
%    All other parts of the definition can be adopted from
%    |\@xnthm|. We have to remember, though, that in this case
%    the name of the current counter and the theorem label have moved
%    to the second and third arguments.
%    \begin{macrocode}
    \def\@tempa{\global\@namedef{#1}}\expandafter \@tempa
     \expandafter{\csname th@\the \theorem@style \expandafter
     \endcsname \the\theorem@bodyfont \@thm{#2}{#3}}%
    \global \expandafter \let \csname end#1\endcsname \@endtheorem}%
  \fi}
%    \end{macrocode}
% \end{macro}
%
% \subsection{Macros that are employed in a {\envfont theorem}
%    environment}
%
% \begin{macro}{\@thm}
%    The macro |\@thm| has to increase the current counter.  Then,
%    depending on whether the environment has (or does not have) an
%    optional argument, it has to branch into either |\@begintheorem|
%    or |\@opargtheorem|.
%    \begin{macrocode}
\gdef\@thm#1#2{\refstepcounter{#1}%
%    \end{macrocode}
%    Now we start a {\envfont trivlist} environment, and give
%    |\@topsep| and |\@topsepadd| the values of the \textsf{skip}
%    registers |\theorempreskipamount| and |\theorempostskipamount|.
%    The value in |\@topsep| is the vertical space that is inserted by
%    the first (and only) |\item| in our |\trivlist| whilst
%    |\@topsepadd| is inserted by |\@endparenv| at the end of that
%    {\envfont trivlist} environment.  By using these registers, we
%    obtain the desired space around a {\envfont theorem} environment.
%    \begin{macrocode}
   \trivlist
   \@topsep \theorempreskipamount               % used by first \item
   \@topsepadd \theorempostskipamount           % used by \@endparenv
%    \end{macrocode}
%    Now we have to test whether an optional argument has been given.
% \changes{v2.0d}{89/07/18}{Forgotten percent added.}
%    \begin{macrocode}
   \@ifnextchar [%
%    \end{macrocode}
%    If there is an optional argument, we will call |\@ythm|, and
%    move the arguments read back into the input stream.
%    \begin{macrocode}
   {\@ythm{#1}{#2}}%
%    \end{macrocode}
%    If not, we call |\@begintheorem|. Its first argument is the
%    name of the theorem set (hence the second argument of
%    |\@thm|). Its second argument is the macro that produces the
%    current number.
%    \begin{macrocode}
   {\@begintheorem{#2}{\csname the#1\endcsname}\ignorespaces}}
%    \end{macrocode}
% \end{macro}
%
% \begin{macro}{\@xthm}
% \begin{macro}{\@ythm}
%    Both these macros were originally called by |\@thm|.  We do
%    not need |\@xthm| anymore, hence we reset it to
%    |\relax|. The definition of |\@ythm| has not changed at
%    all from its definition in \LaTeX. In order to make the macros
%    easier to understand, we will
%    nevertheless present it (commented out).
%    \begin{macrocode}
\global\let\@xthm\relax
% \def\@ythm#1#2[#3]{\@opargbegintheorem{#2}{\csname
%           the#1\endcsname}{#3}\ignorespaces}
%</package>
%    \end{macrocode}
%    The primitive |\ignorespaces| in |\@ythm| and
%    |\@thm| is needed to remove the spaces between the
%    |\begin{...}| and the actual text.
% \end{macro}
% \end{macro}
%
%
% \subsection{Definition of the theorem substyles}
%
% As already pointed out, the theorem substyles, defined below, are
% only loaded when necessary.  Note that all these substyles, except
% \texttt{plain}, have |\slshape| as the default body font.
%
% \subsubsection{The \texttt{plain} style}
%
% \label{dok:th@plain} As the following macros use |@|, we have
% to locally set the |\catcode| of this symbol to ``letter''.
% This happens within a group, so that we do not have to worry about
% which |\catcode| that symbol had before.
%    \begin{macrocode}
%<*thp>
\begingroup \makeatletter
%    \end{macrocode}
% Since we are now within a group, we must make all definitions
% globally.  First we make sure that \texttt{theorem.sty} is loaded.
% This will allow us to use this file as a document style option
% without having to call \texttt{theorem} itself as an option.  At the
% same time, we assure that at least version 2 is loaded, since
% |\theorem@style| was not defined in earlier versions.
%    \begin{macrocode}
\@ifundefined{theorem@style}{\input{theorem.sty}}{}
\ProvidesFile{thp.sty}
             [\FMithmInfo]
%    \end{macrocode}
%
%
% \begin{macro}{\th@plain}
% \begin{macro}{\@begintheorem}
% \begin{macro}{\@opargbegintheorem}
%    |\theoremstyle{plain}| corresponds to the original
%    definition, except that the distances to the surrounding text are
%    determined by the parameters |\theorempreskipamount| and
%    |\theorempostskipamount|.
%    First we set the default body font.
%    \begin{macrocode}
\gdef\th@plain{\normalfont\itshape
%    \end{macrocode}
%    Then we define |\@begintheorem| and
%    |\@opargbegintheorem|. These two macros define how the
%    header of a theorem is typeset.  |\@opargbegintheorem| will
%    be called if a \textsf{theorem} environment with an optional
%    argument is encountered; otherwise, the header is constructed by
%    calling |\@begintheorem|.  If one of these macros is
%    executed, we are within a \textsf{trivlist} environment started by
%    |\@thm|.  So the theorem header is produced with an
%    |\item| command.
%
%    Instead of specifying the header font directly, all standard
%    theorem styles use the |\theorem@headerfont| macro to allow
%    customization.  The extra space (|\labelsep|) is necessary
%    because of problems in the \textsf{trivlist} environment.
%    \begin{macrocode}
  \def\@begintheorem##1##2{%
        \item[\hskip\labelsep \theorem@headerfont ##1\ ##2]}%
%    \end{macrocode}
%    The definition of |\@opargbegintheorem| is completely
%    analogous. The only difference is the fact that there exists a
%    third argument (which is the optional parameter of the
%    environment and contains additional information about the
%    theorem).  Customarily we enclose it in parentheses.
%    \begin{macrocode}
\def\@opargbegintheorem##1##2##3{%
   \item[\hskip\labelsep \theorem@headerfont ##1\ ##2\ (##3)]}}
%    \end{macrocode}
% \end{macro}
% \end{macro}
% \end{macro}
%
% \noindent
% We conclude with an |\endgroup| to restore the
% |\catcode| of |@|.
%    \begin{macrocode}
\endgroup
%</thp>
%    \end{macrocode}
%
%
% \subsubsection{The \texttt{break} style}
%
% This style option is stored in the file |thb.sty|.
% For the next two lines see the documentation for |\th@plain| on
% page~\pageref{dok:th@plain}.
%    \begin{macrocode}
%<*thb>
\begingroup \makeatletter
\@ifundefined{theorem@style}{\input{theorem.sty}}{}
\ProvidesFile{thb.sty}
             [\FMithmInfo]
%    \end{macrocode}
%
% \begin{macro}{\th@break}
%    |\theoremstyle{break}| produces a line break after the name
%    of the theorem.  The font is |\slshape|.  Hence, we define
%    |\th@break| as follows:
%    \begin{macrocode}
\gdef\th@break{\normalfont\slshape
  \def\@begintheorem##1##2{\item[%
%    \end{macrocode}
%    We run into the following problem: it is not possible to create
%    the header with |\item[|\meta{title}|]| and then start
%    a new line by, for example, |\mbox{}\\|. Such a definition
%    will fail whenever a list environment follows immediately.  With
%    the above construction, the |\mbox{}| causes the switch
%    |@inlabel| (cf.\ definition of |\list| and
%    |\trivlist| in \cite{latex:source}) to be set to \texttt{false}
%    and so the following list will insert additional vertical
%    space (|\topskip|). This is quite annoying. Therefore, we
%    create the line break within the |\item|. In order to ensure
%    that the text will begin at the proper position in the following
%    line, we simply pretend that the label does not take any
%    room.\footnote{This will lead to problems whenever very high
%    symbols occurring in the line tower into the
%    heading. So, something else has to be done here sometime.}
%    \begin{macrocode}
       \rlap{\vbox{\hbox{\hskip \labelsep\theorem@headerfont ##1\ ##2}%
                   \hbox{\strut}}}]}%
%    \end{macrocode}
%    Again, the definition of |\@opargbegintheorem| is
%    completely analogous.
%    \begin{macrocode}
\def\@opargbegintheorem##1##2##3{%
  \item[\rlap{\vbox{\hbox{\hskip \labelsep \theorem@headerfont
                     ##1\ ##2\ (##3)}%
                    \hbox{\strut}}}]}}
\endgroup
%</thb>
%    \end{macrocode}
% \end{macro}
%
%
% \subsubsection{The \texttt{changebreak} style}
%
%    \begin{macrocode}
% This style option is stored in the file |thcb.sty|.
%    \begin{macrocode}
%<*thcb>
\begingroup \makeatletter
\@ifundefined{theorem@style}{\input{theorem.sty}}{}
\ProvidesFile{thcb.sty}
             [\FMithmInfo]
%    \end{macrocode}
%
% \begin{macro}{\th@changebreak}
%    The \texttt{change-break} theorem style is like \texttt{break}
%    but with interchange of theorem name and theorem number.  Thus we
%    define |\th@changebreak| as follows:
%    \begin{macrocode}
\gdef\th@changebreak{\normalfont\slshape
  \def\@begintheorem##1##2{\item
       [\rlap{\vbox{\hbox{\hskip\labelsep \theorem@headerfont ##2\ ##1}%
                    \hbox{\strut}}}]}%
\def\@opargbegintheorem##1##2##3{%
  \item[\rlap{\vbox{\hbox{\hskip\labelsep \theorem@headerfont
                     ##2\ ##1\ (##3)}%
                    \hbox{\strut}}}]}}
\endgroup
%</thcb>
%    \end{macrocode}
% \end{macro}
%
%
% \subsubsection{The \texttt{change} style}
%
% This style option is stored in the file |thc.sty|.
%    \begin{macrocode}
%<*thc>
\begingroup \makeatletter
\@ifundefined{theorem@style}{\input{theorem.sty}}{}
\ProvidesFile{thc.sty}
             [\FMithmInfo]
%    \end{macrocode}
%
% \begin{macro}{\th@change}
%    The \texttt{change} theorem style corresponds to the
%    \texttt{change break} style without a linebreak after the header.
%    To say it in another way, it's the same as the \texttt{plain}
%    style but with number and name interchanged and |\slshape| as the
%    default font.
%    \begin{macrocode}
\gdef\th@change{\normalfont\slshape
  \def\@begintheorem##1##2{\item
       [\hskip\labelsep \theorem@headerfont ##2\ ##1]}%
\def\@opargbegintheorem##1##2##3{%
  \item[\hskip\labelsep \theorem@headerfont ##2\ ##1\ (##3)]}}
\endgroup
%</thc>
%    \end{macrocode}
% \end{macro}
%
%
% \subsubsection{The \texttt{marginbreak} style}
%
% This style option is the one used most often at Mainz.  It is saved
% in the file |thmb.sty|.
%    \begin{macrocode}
%<*thmb>
\begingroup \makeatletter
\@ifundefined{theorem@style}{\input{theorem.sty}}{}
\ProvidesFile{thmb.sty}
             [\FMithmInfo]
%    \end{macrocode}
%
% \begin{macro}{\th@marginbreak}
%    The \texttt{margin break} style is nearly the same as the
%    \texttt{change break} style. The only difference is the placement
%    of the theorem number.  We use |\llap| to place it in the left
%    margin.
%
%    In this style |\labelsep| denotes the separation between the
%    number and the text.
%    \begin{macrocode}
\gdef\th@marginbreak{\normalfont\slshape
  \def\@begintheorem##1##2{\item
       [\rlap{\vbox{\theorem@headerfont
                    \hbox{\llap{##2}\hskip\labelsep ##1}%
                    \hbox{\strut}}}]}%
\def\@opargbegintheorem##1##2##3{%
  \item[\rlap{\vbox{\theorem@headerfont
                    \hbox{\llap{##2}\hskip\labelsep ##1\ (##3)}%
                    \hbox{\strut}}}]}}
\endgroup
%</thmb>
%    \end{macrocode}
% \changes{v2.1c}{90/02/17}{`endgroup outside macrocode env.}
% \end{macro}
%
%
%
% \subsubsection{The \texttt{margin} style}
%
% This style option is stored in the file |thm.sty|.
%    \begin{macrocode}
%<*thm>
\begingroup \makeatletter
\@ifundefined{theorem@style}{\input{theorem.sty}}{}
\ProvidesFile{thm.sty}
             [\FMithmInfo]
%    \end{macrocode}
%
% \begin{macro}{\th@margin}
%    Again this is only a variant of the theorem styles described
%    above without any new ideas.
%    \begin{macrocode}
\gdef\th@margin{\normalfont\slshape
  \def\@begintheorem##1##2{\item
       [\theorem@headerfont \llap{##2}\hskip\labelsep ##1]}%
\def\@opargbegintheorem##1##2##3{%
  \item[\theorem@headerfont \llap{##2}\hskip\labelsep ##1\ (##3)]}}
\endgroup
%</thm>
%    \end{macrocode}
% \end{macro}
%
%
%
%
% \subsection{Final Definitions}
%
% \begin{macro}{\theorempreskipamount}
% \begin{macro}{\theorempostskipamount}
%    The \textsf{skip} parameters that regulate the vertical empty space
%    before and after the {\envfont theorem} environment have
%    to be allocated as well.
% \changes{v2.0f}{89/07/20}{Unnecessary `global' removed.}
%    \begin{macrocode}
%<*package>
\newskip\theorempreskipamount
\newskip\theorempostskipamount
%    \end{macrocode}
%    Since we have used the same values for all theorem sets, we now can
%    assign them.
%    \begin{macrocode}
\global\setlength\theorempreskipamount{12pt plus 5pt minus 3pt}
\global\setlength\theorempostskipamount{8pt plus 3pt minus 1.5pt}
%    \end{macrocode}
% \end{macro}
% \end{macro}
%
%
% \begin{macro}{\@endtheorem}
%    The same holds for the macro |\@endtheorem|, which ends a
%    {\envfont theorem} environment. Since it is the same for all
%    theorem sets, it is removed from the macros |\th@|\meta{style}.
%    It simply ends the {\envfont trivlist} environment, which was
%    begun in |\@thm|.
%    \begin{macrocode}
\global\let\@endtheorem=\endtrivlist
%    \end{macrocode}
% \end{macro}
%
%
%
% \begin{macro}{\@preamblecmds}
%    All macros defined above are to be used only in the preamble.
%    Therefore, we insert them in |\@preamblecmds| which will disable
%    them at begin document. This is done by the internal \LaTeXe{}
%    command |\@onlypreamble|.
% \changes{v2.0g}{89/09/19}{`theoremheaderfont and `theorembodyfont
%    added.}
% \changes{v2.2a}{94/02/03}{Use \LaTeXe{} interface}
%    \begin{macrocode}
\@onlypreamble\@xnthm
\@onlypreamble\@ynthm
\@onlypreamble\@othm
\@onlypreamble\newtheorem
\@onlypreamble\theoremstyle
\@onlypreamble\theorembodyfont
\@onlypreamble\theoremheaderfont
%    \end{macrocode}
% \end{macro}
%
% Finally we declare the \texttt{plain} theorem style to be the default.
% \changes{v2.1b}{90/06/04}{Fixed missing `item by declaring default
%    style.}
%    \begin{macrocode}
\theoremstyle{plain}
%</package>
%    \end{macrocode}
%
% \Finale

\endinput
}{}
\ProvidesFile{thmb.sty}
             [\FMithmInfo]
%    \end{macrocode}
%
% \begin{macro}{\th@marginbreak}
%    The \texttt{margin break} style is nearly the same as the
%    \texttt{change break} style. The only difference is the placement
%    of the theorem number.  We use |\llap| to place it in the left
%    margin.
%
%    In this style |\labelsep| denotes the separation between the
%    number and the text.
%    \begin{macrocode}
\gdef\th@marginbreak{\normalfont\slshape
  \def\@begintheorem##1##2{\item
       [\rlap{\vbox{\theorem@headerfont
                    \hbox{\llap{##2}\hskip\labelsep ##1}%
                    \hbox{\strut}}}]}%
\def\@opargbegintheorem##1##2##3{%
  \item[\rlap{\vbox{\theorem@headerfont
                    \hbox{\llap{##2}\hskip\labelsep ##1\ (##3)}%
                    \hbox{\strut}}}]}}
\endgroup
%</thmb>
%    \end{macrocode}
% \changes{v2.1c}{90/02/17}{`endgroup outside macrocode env.}
% \end{macro}
%
%
%
% \subsubsection{The \texttt{margin} style}
%
% This style option is stored in the file |thm.sty|.
%    \begin{macrocode}
%<*thm>
\begingroup \makeatletter
\@ifundefined{theorem@style}{% \iffalse meta-comment
%
% Copyright (C) 1993-2024
%
% The LaTeX Project and any individual authors listed elsewhere
% in this file.
%
% This file is part of the Standard LaTeX `Tools Bundle'.
% -------------------------------------------------------
%
% It may be distributed and/or modified under the
% conditions of the LaTeX Project Public License, either version 1.3c
% of this license or (at your option) any later version.
% The latest version of this license is in
%    https://www.latex-project.org/lppl.txt
% and version 1.3c or later is part of all distributions of LaTeX
% version 2005/12/01 or later.
%
% The list of all files belonging to the LaTeX `Tools Bundle' is
% given in the file `manifest.txt'.
%
% \fi
%
%%
%
%
\def\FMithmInfo{2023/07/05 v2.2c Theorem extension package (FMi)}
%
% \ProvidesFile{theorem.dtx}[\FMithmInfo]
%\iffalse   % this is a METACOMMENT !
%%
%% Package `theorem' to use with LaTeX2e
%% Copyright (C) 1989-1995 Frank Mittelbach, all rights reserved.
%<+package>\@ifundefined{theorem@style}{}{\endinput}
%<+package>\NeedsTeXFormat{LaTeX2e}[1995/06/01]
%<+package>\ProvidesPackage{theorem}
%<+package>         [\FMithmInfo]
%
% \fi
%
% \changes{v1.0}{87/05/07}{First draft.}
% \changes{v1.0a}{87/05/07}{`thistheoremstyle' added.}
% \changes{v1.1a}{87/07/29}{New concept: redefine the internal
%                          \protect\LaTeX{} commands `@xnthm' etc.}
% \changes{v1.1b}{87/07/31}{`theoremfont' used to specify font.}
% \changes{v1.2a}{88/01/08}{`@endtheorem' now globally defined.}
% \changes{v1.2b}{88/01/11}{`newtheorem' only allowed in the preamble.}
% \changes{v1.3a}{88/02/14}{`theoremfont' now a toks register.}
%
% \changes{v2.0a}{88/11/16}{New implementation and new documentation.}
% \changes{v2.0a}{88/11/16}{No layout changes but internal modifications
%                           old substyles won't work.}
% \changes{v2.0b}{89/01/30}{`global' used for all definitions.}
% \changes{v2.0c}{89/06/15}{Documentation translated to English.
%                            Thanks to LLL and Nina}
% \changes{v2.1a}{90/02/15}{Support for new font selection.}
% \changes{v2.2a}{94/02/03}{Cleanup for LaTeX2e}
%
% \DoNotIndex{\@endparenv,\@ifdefinable,\@ifnextchar,\@ifundefined}
% \DoNotIndex{\@namedef,\@nameuse,\@spaces,\@tempa,\@warning}
% \DoNotIndex{\ }
% \DoNotIndex{\begingroup,\bfseries}
% \DoNotIndex{\csname}
% \DoNotIndex{\def,\do,\docdate}
% \DoNotIndex{\endcsname,\endgroup,\endinput,\endtrivlist,\expandafter}
% \DoNotIndex{\filedate,\fileversion}
% \DoNotIndex{\gdef,\global}
% \DoNotIndex{\hbox,\hskip}
% \DoNotIndex{\ignorespaces,\input,\itshape,\item}
% \DoNotIndex{\labelsep,\let,\llap}
% \DoNotIndex{\makeatletter}
% \DoNotIndex{\noexpand}
% \DoNotIndex{\refstepcounter,\relax,\rlap}
% \DoNotIndex{\setlength,\slshape,\space,\string,\strut}
% \DoNotIndex{\the,\trivlist,\typeout}
% \DoNotIndex{\vbox}
% \DoNotIndex{\xdef}
%
% \def\deflabel#1{\ttfamily #1\hfill}
% \def\deflist#1{\begin{list}{}{\settowidth\labelwidth{\ttfamily #1}%
%                               \setlength\leftmargin\labelwidth
%                               \addtolength\leftmargin\labelsep
%                               \let\makelabel\deflabel}}
% \def\enddeflist{\end{list}}
% \def\envfont{\normalfont\ttfamily}
%
% \MakeShortVerb{\|}
%
% \newcommand{\TUB}{{\sl TUGboat\/}}
%
% \GetFileInfo{theorem.dtx}
% \title{An Extension of the
%        \LaTeX{} theorem environment\thanks{This file
%        has version number \fileversion, last
%        revised \filedate.}}
% \author{Frank Mittelbach}
%
% \MaintainedByLaTeXTeam{tools}
% \maketitle
%
% \begin{abstract}
%    The macros described in this paper yield an extension of the
%    \LaTeX{} theorem mechanism. It is designed to satisfy the
%    different requirements of various journals. Thus, the
%    layout of the ``theorems'' can be manipulated by determining a
%    ``style''. This article describes not only the use, but
%    also the definition, of the necessary macros.
% \end{abstract}
%
% \section*{Preface to version 2.2}
%
% For \LaTeXe{} this package did not need any fundamental changes. I
% only modified the messages generated so that theorem layout styles
% will show up with the |\listfiles| command and cleaned the section
% on the New Font Selection Scheme since this is now included in
% \LaTeX.
%
% \section*{Preface to version 2.1}
%
% This version is identical to 2.0g described in \TUB\ 10\#3 except
% for some internal defaults which are now set depending on the used
% font selection scheme.
%
% This was done to avoid unpleasant surprises if the new font selection
% scheme is in force. For further details see
% section~\ref{sec:newfonts} and \cite{bk:GMS94}.
%
% \section{Introduction}
%
% For our purposes here, ``theorems'' are labelled enunciations,
% often set off from the main text by extra space and a font change.
% Theorems, corollaries, conjectures, definitions, and remarks
% are all instances of ``theorems''.  The ``header'' of these structures
% is composed of a label (such as \textsc{Theorem} or \textsc{Remark})
% and a number which serializes an item in the sequence of items
% with the same label.
%
% Shortly after the introduction of \LaTeX{} at the Fachbereich
% Mathematik in Mainz, the desire to manipulate the layout of
% ``theorems'' arose.  In Mainz, the following two conventions came
% into general use:
% \begin{enumerate}
%   \item The number of the theorem is shown in the margin.
%   \item There is a line break at
%         the end of the theorem header.
% \end{enumerate}
%
% Additionally, some journals require different formats which depend on
% the ``sort of theorem'': e.g.\ often remarks and definitions are set
% in |\upshape|, while |\itshape| is employed for main theorems.
%
% Confronted with these requirements, a theorem environment was
% developed in Mainz which allows separate determination of the
% layout of the ``theorems sets'', comparable to |\pagestyle|.
%
%
% \section{The user interface}
% \subsection{Defining new theorem sets}
%
% \DescribeMacro\newtheorem
% As in the original \LaTeX{} version, the command |\newtheorem|
% defines a new ``theorem set'' or ``theorem-like structure''.
% Two required arguments name the new environment and give the text
% to be typeset with each instance of the new ``set'', while
% an optional argument determines how
% the ``set'' is enumerated:
% \begin{description}
%    \item[\ttfamily \bslash newtheorem\{foo\}\{bar\}]
%       The theorem set {\envfont foo} (whose name is \texttt{bar})
%       uses its own counter.
%    \item[\ttfamily \bslash newtheorem\{foo2\}{[foo]}\{bar2\}]
%       The theorem set {\envfont foo2} (printed name \texttt{bar2})
%       uses the same counter as the theorem set \texttt{foo}.
%    \item[\ttfamily \bslash newtheorem\{foo3\}\{bar3\}{[section]}]
%       The theorem set {\envfont foo3} (printed name \texttt{bar3}) is
%       enumerated within the counter \texttt{section}, i.e.\ with every
%       new |\section| the enumeration begins again with 1, and
%       the enumeration is composed from the section-number and the
%       theorem counter itself.
% \end{description}
%
% \DescribeMacro\theoremstyle
% Additionally, the command |\theoremstyle| can define the layout
% of various, or all, theorem sets.  It should be noted that any
% theorem set defined by |\newtheorem| is typeset in the
% |\theoremstyle| that is current at the time of the definition.
% Thus, the following
% \begin{verbatim}
%    \theoremstyle{break}        \newtheorem{Cor}{Corollary}
%    \theoremstyle{plain}        \newtheorem{Exa}{Example}[section]
%\end{verbatim}                             ^^A No blank at line start
% leads to the result that the set {\envfont Cor} is formatted in the
% style \texttt{break}, while the set {\envfont Exa} and all the
% following ones are formatted in the style \texttt{plain}, unless
% another |\theoremstyle| follows.  Since the definitions installed by
% |\newtheorem| are global, one also can limit |\theoremstyle| locally
% by grouping braces.
%
% \DescribeMacro\theorembodyfont
% The choice of the font for the theorem body is completely
% independent of the chosen |\theoremstyle|; this has proven to
% be very advantageous.  For example,
% \begin{verbatim}
%   {\theorembodyfont{\upshape}          \newtheorem{Rem}{Remark}}
%\end{verbatim}
% defines a theorem set {\envfont Rem}, which will be set in
% |\upshape| in the current layout (which in our example is
% \texttt{plain}).  As with |\theoremstyle|, the |\theorembodyfont|
% chosen is that current at the time of |\newtheorem|.  If
% |\theorembodyfont| is not specified or one defines
% |\theorembodyfont{}|, then the font used will be that defined by the
% |\theoremstyle|.
%
% \DescribeMacro\theoremheaderfont It is also possible to customize
% the font used for the theorem headers.  This is, however, a global
% declaration, and therefore there should be at most one
% |\theoremheaderfont| declaration in the preamble.\footnote{If it is
% actually necessary to have different header fonts, one has to define
% new theorem styles (substituting the desired font) or specify the
% information directly in the \texttt{\bslash newtheorem} declaration
% (the unclean variant).}
%
% \DescribeMacro\theorempreskipamount
% \DescribeMacro\theorempostskipamount Two additional parameters
% affect the vertical space around the theorem environments:
% |\theorempreskipamount| and |\theorempostskipamount| define,
% respectively, the spacing before and after such an environment.
% These parameters apply for all theorem sets and can be manipulated
% with the ordinary length macros.  They are rubber lengths
% (`\textsf{skips}'), and therefore can contain \texttt{plus} and
% \texttt{minus} parts.
%
% Since the definition of theorem sets should---most sensibly---be
% placed in the preamble, we only allow installation there.
% It is therefore possible to release the memory used here after
% |\begin{document}|, in order to make room for other
% applications.
%
%
% \subsection{Existing theorem styles}
%
% The following theorem styles exist to date:
% \begin{deflist}{marginbreak}
%    \item[plain]
%       This theorem style emulates the original \LaTeX{} definition,
%       except that additionally the parameters
%       |\theorem...skipamount| are used.
%    \item[break]
%       In this style, the theorem header is followed by a line
%       break.
%    \item[marginbreak]
%       The theorem number is set in the margin, and there is a line
%       break as in \texttt{break}.
%    \item[changebreak]
%       Like \texttt{break}, but with header number and text
%       interchanged.
%    \item[change]
%       Header number and text are interchanged, without a line break.
%    \item[margin]
%       The number is set in the left margin, without a line break.
% \end{deflist}
% All styles (except \texttt{plain}) select |\slshape| as the default
% |\theorembodyfont|.
%
%
% \subsection{Examples}
%
% Given the above theorem sets {\envfont Cor}, {\envfont Exa}
% and {\envfont Rem},
% suppose that the preamble also contains the declarations:
% \begin{verbatim}
%    \theoremstyle{marginbreak}   \newtheorem{Lem}[Cor]{Lemma}
%    \theoremstyle{change}
%    \theorembodyfont{\itshape}        \newtheorem{Def}[Cor]{Definition}
%
%    \theoremheaderfont{\scshape}
%\end{verbatim}
% Then the following are some typical examples of the typeset output
% resulting from their use.
%
% \begin{Cor}
%    This is a sentence typeset in the theorem environment {\envfont
%    Cor}.
% \end{Cor}
%
% \begin{Exa}
%    This is a sentence typeset in the theorem environment {\envfont
%    Exa}.
% \end{Exa}
%
% \begin{Rem}
%    This is a sentence typeset in the theorem environment {\envfont
%    Rem}.
% \end{Rem}
%
% \begin{Lem}[Ben User]
%    This is a sentence typeset in the theorem environment {\envfont
%    Lem}.
% \end{Lem}
%
% \begin{Def}[Very impressive Definition]
%    This is a sentence typeset in the theorem environment {\envfont
%    Def}.
% \end{Def}
%
% The last two examples show the effect of the optional argument to
% a theorem environment (it is the text typeset in parentheses).
%
%
% \section{Special Considerations}
% \label{sec:newfonts}
%
% Theoremheader and body are implemented as a unit. This means that
% the |\theoremheaderfont|  will inherit characteristics of the
% |\theorembodyfont| in \LaTeXe. Thus, if for example
% |\theorembodyfont| is |\itshape| and |\theoremheaderfont|
% is |\bfseries| the font selected for the header will have the
% characteristics `bold extended italic'. If this is not desired one
% should set the |\theoremheaderfont| to something like
% \begin{verbatim}
%      \theoremheaderfont{\normalfont\bfseries}
%\end{verbatim}
% i.e.\ supplying all necessary font information explicitly.
%
%
%
% \section{Acknowledgements}
%
% The publication of this set of macros was only possible with the
% help of Christina Busse (translating the manuscript into English),
% Joachim Pense (playing the r\^ole of typist), Chris Rowley
% (looking everything over) and many others providing useful
% suggestions.
%
% \MaybeStop{%
%    \begin{thebibliography}{9}
%       \bibitem{bk:GMS94} \textsc{M.~Goossens}, \textsc{F.~Mittelbach}
%         and \textsc{A.~Samarin}.
%         \newblock The \LaTeX{} Companion.
%         \newblock
%         Addison-Wesley, Reading, Massachusetts, 1994.
%       \bibitem{latex:source} \textsc{Lamport, Leslie.} \newblock
%                               \texttt{latex.tex}, version 2.09, date
%                               Feb. 1990.
%    \end{thebibliography}
%    \PrintIndex
%    \PrintChanges}
%
%
% \section{The documentation driver file}
%
% The next bit of code contains the documentation driver file for
% \TeX{}, i.e., the file that will produce the documentation you are
% currently reading. It will be extracted from this file by the
% \texttt{docstrip} program. Since it is the first code in the file
% one can alternatively process this file directly with \LaTeXe{} to
% obtain the documentation.
%
%    \begin{macrocode}
%<*driver>
\documentclass{ltxdoc}

\usepackage{theorem}

%<+driver>% The next few lines define theorem sets which are used
%<+driver>% in the example section of the documentation.

\theoremstyle{break}       \newtheorem{Cor}{Corollary}
\theoremstyle{plain}       \newtheorem{Exa}{Example}[section]
{\theorembodyfont{\upshape}\newtheorem{Rem}{Remark}}
\theoremstyle{marginbreak} \newtheorem{Lem}[Cor]{Lemma}
\theoremstyle{change}
\theorembodyfont{\itshape} \newtheorem{Def}[Cor]{Definition}

\theoremheaderfont{\scshape}

\RecordChanges

\begin{document}
   \DocInput{theorem.dtx}
\end{document}
%</driver>
%    \end{macrocode}
%
% \section{Definition of the Macros}
%
% If the file has been loaded before, we abort immediately. If not the
% package announces itself (this is actually done at the very top of
% the file---the way it is done isn't good style so don't copy it).
% \changes{v2.0e}{89/07/19}{Spaces removed from `typeout'.}
% \changes{v2.2b}{95/11/19}{Announce moved to top of file}
%    \begin{macrocode}
%<*package>
%\@ifundefined{theorem@style}{}{\endinput}
%\def\FMithmInfo{1995/11/19 v2.2b Theorem extension package (FMi)}
%\ProvidesPackage{theorem}[\FMithmInfo]
%    \end{macrocode}
%
%
% \subsection{Definition of theorem styles and fonts}
%
% All the definitions in this file are done globally to allow
% inputting this file inside a group.
%
% \begin{macro}{\theoremstyle}
%    Before a theorem style can be installed, the chosen style must be
%    known. For that reason, we must test to see that
%    |\th@|\meta{style} is known or, more precisely, that it is
%    different from |\relax|. If the style is not known then
%    |\th@plain| is used.
%    \begin{macrocode}
\gdef\theoremstyle#1{%
   \@ifundefined{th@#1}{\@warning
          {Unknown theoremstyle `#1'. Using `plain'}%
          \theorem@style{plain}}%
%    \end{macrocode}
%    We save the theorem style to be used in the token register
%    |\theorem@style|.
%    \begin{macrocode}
      {\theorem@style{#1}}%
%    \end{macrocode}
%    Now we ``evaluate'' the theorem style: this means, we call the
%    macro |\th@|\meta{style} which will activate the relevant
%    definitions which are contained in a separate file.  This is done
%    in a group to suppress changes to the current font.  This could
%    otherwise pose problems together with the new font selection
%    scheme\footnote{When I printed the original article using the new
%    font selection scheme I ended with a document with slanted
%    typefaces (text headings and all) simply because one of the
%    theorem styles used \texttt{\bslash sl} at toplevel.} if the
%    |\th@|\meta{style} is evaluated a second time.
% \changes{v2.1a}{90/02/15}{Group added around `th@..}
% \changes{v2.1a}{90/02/15}{Reseting of `..@begintheorem removed.}
%    \begin{macrocode}
      \begingroup
        \csname th@\the\theorem@style \endcsname
      \endgroup}
%    \end{macrocode}
% \end{macro}
%
%
% \begin{macro}{\@begintheorem}
% \changes{v2.1a}{90/02/15}{Relaxed at toplevel.}
% \begin{macro}{\@opargbegintheorem}
% \changes{v2.1a}{90/02/15}{Relaxed at toplevel.}
%    We reset |\@begintheorem| and |\@opargbegintheorem| to
%    |\relax| since these commands are no longer necessary at
%    toplevel. This will save a few tokens.
%    \begin{macrocode}
\global\let\@begintheorem\relax
\global\let\@opargbegintheorem\relax
%    \end{macrocode}
% \end{macro}
% \end{macro}
%
% \begin{macro}{\theorem@style}
%    Obviously the \textsf{token} register used above has to be
%    allocated. To assure the utmost compatibility with the
%    original \LaTeX{} definition, we set the default theorem style to
%    \texttt{plain}, which implements the usual \LaTeX{} convention.
% \changes{v2.0f}{89/07/20}{Unnecessary `global' removed.}
%    \begin{macrocode}
\newtoks\theorem@style
\global\theorem@style{plain}
%    \end{macrocode}
% \end{macro}
%
% \begin{macro}{\theorembodyfont}
% \begin{macro}{\theorem@bodyfont}
%    If the |\theorembodyfont| is set by the user then it should not
%    interact with the default font set in the theorem style. When the
%    new font selection is in force this may happen if, for example,
%    the default is |\itshape| and the new |\theorembodyfont| is
%    |\sffamily|. So we add a |\reset@font| command in front of the
%    user definition.
% \changes{v2.0f}{89/07/20}{Unnecessary `global' removed.}
% \changes{v2.0g}{89/09/19}{`theoremfont to `theorembodyfont.}
% \changes{v2.1c}{92/06/22}{`reset@font added and logic changed.}
% \changes{v2.1d}{92/10/28}{Reset to empty if arg is empty}
%    \begin{macrocode}
\gdef\theorembodyfont#1{%
%    \end{macrocode}
%    We check if the argument supplied is empty and if so put nothing
%    into the |\theorem@bodyfont| token register to allow for
%    |\theorembodyfont{}| as a mean of using the default of the
%    current |\theoremstyle|.
%    \begin{macrocode}
   \def\@tempa{#1}%
   \ifx\@tempa\@empty
    \theorem@bodyfont{}%
   \else
    \theorem@bodyfont{\reset@font#1}%
   \fi
   }
\newtoks\theorem@bodyfont
\global\theorem@bodyfont{}
%    \end{macrocode}
% \end{macro}
% \end{macro}
%
%
% \begin{macro}{\theoremheaderfont}
% \changes{v2.0g}{89/09/19}{Macro added.}
%    The font for the theorem headers is handled differently because
%    this definition applies to all theorem styles.
%    \begin{macrocode}
\gdef\theoremheaderfont#1{\gdef\theorem@headerfont{#1}%
%    \end{macrocode}
%    After using the macro once it is redefined to produce an error
%    message.
%    \begin{macrocode}
       \gdef\theoremheaderfont##1{%
        \typeout{\string\theoremheaderfont\space should be used
                 only once.}}}
%    \end{macrocode}
% \end{macro}
%
% \begin{macro}{\theorem@headerfont}
% \changes{v2.0g}{89/09/19}{Macro added.}
%    To set the |\theorem@headerfont| default we first test if
%    the new fontselection scheme is in force.
% \changes{v2.1a}{90/02/15}{Support for new font selection.}
%    \begin{macrocode}
\ifx\upshape\undefined
%    \end{macrocode}
%    If not we define it to expand into |\bfseries|. We don't use
%    |\let| just in case a following style option redefines this
%    macro.
%    \begin{macrocode}
\gdef\theorem@headerfont{\bfseries}
%    \end{macrocode}
%    Otherwise we reset the current shape before calling |\bfseries|.
%    \begin{macrocode}
\else \gdef\theorem@headerfont{\normalfont\bfseries}\fi
%    \end{macrocode}
% \end{macro}
%
% \begin{macro}{\th@plain}
% \begin{macro}{\th@break}
% \begin{macro}{\th@marginbreak}
% \begin{macro}{\th@changebreak}
% \begin{macro}{\th@change}
% \begin{macro}{\th@margin}
%    The different styles are defined in macros such as
%    |\th@plain|.  Since memory space is precious in
%    ``non-Big-versions'', we have to avoid offering too many unused
%    definitions.  Therefore we define these styles in separate files
%    that can be loaded on demand.  Thus the commands themselves only
%    load these files. We use |\@input@|, a \LaTeXe{} internal command
%    that ensures that the file will be listed with |\listfiles|.
% \changes{v2.2a}{94/02/02}{Use `@input@ to load theorem layout files}
%    \begin{macrocode}
\gdef\th@plain{\@input@{thp.sty}}
\gdef\th@break{\@input@{thb.sty}}
\gdef\th@marginbreak{\@input@{thmb.sty}}
\gdef\th@changebreak{\@input@{thcb.sty}}
\gdef\th@change{\@input@{thc.sty}}
\gdef\th@margin{\@input@{thm.sty}}
%    \end{macrocode}
%    This list will be expanded when new styles become
%    available.  For testing,  just append new theorem substyles
%    as document options.
% \end{macro}
% \end{macro}
% \end{macro}
% \end{macro}
% \end{macro}
% \end{macro}
%
%
%
% \subsection{Definition of a new theorem set}
%
% As already pointed out, a new theorem environment can be defined in
% three different ways:
% \begin{verbatim}
%    \newtheorem{Lem}{Lemma}
%    \newtheorem{Lem}{Lemma}[section]
%    \newtheorem{Lem}[Theorem]{Lemma}
%\end{verbatim}
% The function of the macro |\newtheorem| is to recognize these cases
% and then to branch into one of the three macros |\@ynthm|, |\@xnthm|
% or |\@othm|.  This mechanism is adopted unchanged from
% \cite{latex:source}; the essential point here is that, for example,
% in the second case, the arguments \texttt{Lem}, \texttt{Lemma} and
% \texttt{section} are passed over to the macro |\@xnthm|.
%
% We inspect this case first because the others present fewer problems,
% and thus are easily derived from this one.
%
% \begin{macro}{\@xnthm}
%    For our example arguments, the macro |\@xnthm| must fulfill
%    the following:
%    \begin{itemize}
%       \item
%          Define a new \LaTeX-counter `Lem'
%       \item
%          reset this counter within a |\section|
%       \item
%          define the macro |\theLem|
%       \item
%          define the environment macros |\Lem| and
%          |\endLem| using the current
%          |\theoremstyle| and |\theorem@bodyfont|.
%    \end{itemize}
%    Obviously, all this should happen only if the first argument
%    of |\@xnthm| (i.e.\ \texttt{Lem} in our example) is chosen so
%    as not to conflict with any previously defined commands or
%    environments.  This test is performed by the \LaTeX{} macro
%    |\@ifdefinable|.
%    \begin{macrocode}
\gdef\@xnthm#1#2[#3]{\expandafter\@ifdefinable\csname #1\endcsname
%    \end{macrocode}
%    Therefore, the first argument of |\@ifdefinable| is the
%    expansion (in the example, |\Lem|) of
%    |\csname||#1\endcsname|.  The second argument is
%    executed only if the test has been completed successfully.
%    \begin{macrocode}
   {%
%    \end{macrocode}
%    Now we define the new counter. The names of the \LaTeX{}
%    macros employed should speak for themselves:
% \changes{v2.2b}{1995/10/07}{Use \cs{@newctr}}
% \changes{v2.2c}{1995/11/23}{Use \cs{@newctr} properly.}
%    \begin{macrocode}
    \@definecounter{#1}\@newctr{#1}[#3]%
%    \end{macrocode}
%    Using |\@newctr| will give a proper error message if the counter
%    in |#3| is not defined.
%    In defining `|\theLem|' we must generate the desired macro
%    name by use of |\expandafter| and |\csname|.
%    \begin{macrocode}
    \expandafter\xdef\csname the#1\endcsname
%    \end{macrocode}
%    An |\xdef| is used in order to make the definition global,
%    and to ensure that it contains the replacement texts of
%    |\@thmcountersep| and |\@thmcounter|.\footnote{These
%    two macros can be defined by the document style.  Their default
%    values produce a `\texttt{.}' as separation and an arabic
%    representation of the number.} However, not everything should be
%    expanded. For example, it saves space to use |\thesection|
%    instead of its---at times---lengthy expansion.
%    \begin{macrocode}
      {\expandafter \noexpand \csname the#3\endcsname
       \@thmcountersep \@thmcounter{#1}}%
%    \end{macrocode}
%    Thus with the defaults of \LaTeX{}, |\theLem| would be
%    replaced by the command sequence |\thesection.\arabic{Lem}|.
%
%    We will now look at the definition of the macro which is executed
%    at the beginning of the actual environment (in our example this
%    macro is |\Lem|). It should be noted that we use an
%    ``|\expandafter| trick'' to expand only certain parts of the
%    replacement text at the time of the definition.
%    \begin{macrocode}
    \def\@tempa{\global\@namedef{#1}}%
    \expandafter \@tempa \expandafter{%
%    \end{macrocode}
%    First, the macro that contains the current definitions of
%    |\@begintheorem| and |\@opargtheorem| should be called
%    up.  The name of this macro---as is already known---has the form
%    |\th@|\meta{theorem\ style}; therefore, it must be called by
%    \begin{macrocode}
      \csname th@\the \theorem@style
%    \end{macrocode}
%    In addition the default theorem font should be changeable, i.e.\
%    we have to insert the contents of |\theorem@bodyfont|. For
%    that reason, we expand even further, beyond |\endcsname|,
%    and thus insert the contents of the \textsf{token} register
%    |\theorem@bodyfont| in the replacement text.
%    \begin{macrocode}
            \expandafter \endcsname \the \theorem@bodyfont
%    \end{macrocode}
%    Now it is time to call the macro |\@thm| which takes over
%    the further processing.  It has two arguments: the current
%    counter name (in our example, \texttt{Lem}), and the text of the
%    label (in our example, \texttt{Lemma}).
%    \begin{macrocode}
     \@thm{#1}{#2}}%
%    \end{macrocode}
%    With this, the `sub-definition' is complete.  The macro
%    |\@endtheorem| ends a theorem environment and is, so far, nothing
%    but an |\endtrivlist|.  (Hence it is defined globally, and not
%    within the theorem styles.\footnote{This has to be changed as
%    soon as theorem styles that change \texttt{\bslash @endtheorem}
%    exist.  In such a case, all existing styles must be changed as
%    well since they will have to reset the macro.}) Therefore, we can
%    set it equivalent to the macro that ends the theorem set (in our
%    example, |\endLem|).  However, if some day theorem styles exist
%    that do change |\@endtheorem|, we would have to use the
%    commented-out line instead.
%    \begin{macrocode}
    \global \expandafter \let \csname end#1\endcsname \@endtheorem
%   \global\@namedef{end#1}{\@endtheorem}%
%    \end{macrocode}
%    With these commands all the required definitions are employed,
%    unless the test |\@ifdefinable| has failed.  Therefore, we
%    end the second argument of this macro and with it the definition
%    of |\@xnthm|.
%    \begin{macrocode}
   }}
%    \end{macrocode}
% \end{macro}
%
% \begin{macro}{\@ynthm}
%    The definition of |\@ynthm| is completely analogous. In
%    this case the new counter that is defined is not reset within
%    another counter; thus the definition of |\the...|
%    is simplified:
%    \begin{macrocode}
\gdef\@ynthm#1#2{\expandafter\@ifdefinable\csname #1\endcsname
   {\@definecounter{#1}%
    \expandafter\xdef\csname the#1\endcsname{\@thmcounter{#1}}%
%    \end{macrocode}
%    The rest of the definition corresponds literally to that of
%    |\@xnthm|:
%    \begin{macrocode}
    \def\@tempa{\global\@namedef{#1}}\expandafter \@tempa
     \expandafter{\csname th@\the \theorem@style \expandafter
     \endcsname \the\theorem@bodyfont \@thm{#1}{#2}}%
    \global \expandafter \let \csname end#1\endcsname \@endtheorem}}
%    \end{macrocode}
% \end{macro}
%
% \begin{macro}{\@othm}
%    The definition of |\@othm| does not contain anything new.
%    \begin{macrocode}
\gdef\@othm#1[#2]#3{%
%    \end{macrocode}
% \changes{v2.2b}{95/10/07}{Build in check for counter defined}
%    We do not define a new counter but instead use one that has
%    already been defined.  Thus the only definition we need is that
%    of this pseudo-counter (i.e.\ |\the|\meta{env.\ name}).
%    First we check if |#2| corresponds to a known counter name.
%    \begin{macrocode}
  \expandafter\ifx\csname c@#2\endcsname\relax
   \@nocounterr{#2}%
  \else
   \expandafter\@ifdefinable\csname #1\endcsname
   {\expandafter \xdef \csname the#1\endcsname
     {\expandafter \noexpand \csname the#2\endcsname}%
%    \end{macrocode}
%    All other parts of the definition can be adopted from
%    |\@xnthm|. We have to remember, though, that in this case
%    the name of the current counter and the theorem label have moved
%    to the second and third arguments.
%    \begin{macrocode}
    \def\@tempa{\global\@namedef{#1}}\expandafter \@tempa
     \expandafter{\csname th@\the \theorem@style \expandafter
     \endcsname \the\theorem@bodyfont \@thm{#2}{#3}}%
    \global \expandafter \let \csname end#1\endcsname \@endtheorem}%
  \fi}
%    \end{macrocode}
% \end{macro}
%
% \subsection{Macros that are employed in a {\envfont theorem}
%    environment}
%
% \begin{macro}{\@thm}
%    The macro |\@thm| has to increase the current counter.  Then,
%    depending on whether the environment has (or does not have) an
%    optional argument, it has to branch into either |\@begintheorem|
%    or |\@opargtheorem|.
%    \begin{macrocode}
\gdef\@thm#1#2{\refstepcounter{#1}%
%    \end{macrocode}
%    Now we start a {\envfont trivlist} environment, and give
%    |\@topsep| and |\@topsepadd| the values of the \textsf{skip}
%    registers |\theorempreskipamount| and |\theorempostskipamount|.
%    The value in |\@topsep| is the vertical space that is inserted by
%    the first (and only) |\item| in our |\trivlist| whilst
%    |\@topsepadd| is inserted by |\@endparenv| at the end of that
%    {\envfont trivlist} environment.  By using these registers, we
%    obtain the desired space around a {\envfont theorem} environment.
%    \begin{macrocode}
   \trivlist
   \@topsep \theorempreskipamount               % used by first \item
   \@topsepadd \theorempostskipamount           % used by \@endparenv
%    \end{macrocode}
%    Now we have to test whether an optional argument has been given.
% \changes{v2.0d}{89/07/18}{Forgotten percent added.}
%    \begin{macrocode}
   \@ifnextchar [%
%    \end{macrocode}
%    If there is an optional argument, we will call |\@ythm|, and
%    move the arguments read back into the input stream.
%    \begin{macrocode}
   {\@ythm{#1}{#2}}%
%    \end{macrocode}
%    If not, we call |\@begintheorem|. Its first argument is the
%    name of the theorem set (hence the second argument of
%    |\@thm|). Its second argument is the macro that produces the
%    current number.
%    \begin{macrocode}
   {\@begintheorem{#2}{\csname the#1\endcsname}\ignorespaces}}
%    \end{macrocode}
% \end{macro}
%
% \begin{macro}{\@xthm}
% \begin{macro}{\@ythm}
%    Both these macros were originally called by |\@thm|.  We do
%    not need |\@xthm| anymore, hence we reset it to
%    |\relax|. The definition of |\@ythm| has not changed at
%    all from its definition in \LaTeX. In order to make the macros
%    easier to understand, we will
%    nevertheless present it (commented out).
%    \begin{macrocode}
\global\let\@xthm\relax
% \def\@ythm#1#2[#3]{\@opargbegintheorem{#2}{\csname
%           the#1\endcsname}{#3}\ignorespaces}
%</package>
%    \end{macrocode}
%    The primitive |\ignorespaces| in |\@ythm| and
%    |\@thm| is needed to remove the spaces between the
%    |\begin{...}| and the actual text.
% \end{macro}
% \end{macro}
%
%
% \subsection{Definition of the theorem substyles}
%
% As already pointed out, the theorem substyles, defined below, are
% only loaded when necessary.  Note that all these substyles, except
% \texttt{plain}, have |\slshape| as the default body font.
%
% \subsubsection{The \texttt{plain} style}
%
% \label{dok:th@plain} As the following macros use |@|, we have
% to locally set the |\catcode| of this symbol to ``letter''.
% This happens within a group, so that we do not have to worry about
% which |\catcode| that symbol had before.
%    \begin{macrocode}
%<*thp>
\begingroup \makeatletter
%    \end{macrocode}
% Since we are now within a group, we must make all definitions
% globally.  First we make sure that \texttt{theorem.sty} is loaded.
% This will allow us to use this file as a document style option
% without having to call \texttt{theorem} itself as an option.  At the
% same time, we assure that at least version 2 is loaded, since
% |\theorem@style| was not defined in earlier versions.
%    \begin{macrocode}
\@ifundefined{theorem@style}{\input{theorem.sty}}{}
\ProvidesFile{thp.sty}
             [\FMithmInfo]
%    \end{macrocode}
%
%
% \begin{macro}{\th@plain}
% \begin{macro}{\@begintheorem}
% \begin{macro}{\@opargbegintheorem}
%    |\theoremstyle{plain}| corresponds to the original
%    definition, except that the distances to the surrounding text are
%    determined by the parameters |\theorempreskipamount| and
%    |\theorempostskipamount|.
%    First we set the default body font.
%    \begin{macrocode}
\gdef\th@plain{\normalfont\itshape
%    \end{macrocode}
%    Then we define |\@begintheorem| and
%    |\@opargbegintheorem|. These two macros define how the
%    header of a theorem is typeset.  |\@opargbegintheorem| will
%    be called if a \textsf{theorem} environment with an optional
%    argument is encountered; otherwise, the header is constructed by
%    calling |\@begintheorem|.  If one of these macros is
%    executed, we are within a \textsf{trivlist} environment started by
%    |\@thm|.  So the theorem header is produced with an
%    |\item| command.
%
%    Instead of specifying the header font directly, all standard
%    theorem styles use the |\theorem@headerfont| macro to allow
%    customization.  The extra space (|\labelsep|) is necessary
%    because of problems in the \textsf{trivlist} environment.
%    \begin{macrocode}
  \def\@begintheorem##1##2{%
        \item[\hskip\labelsep \theorem@headerfont ##1\ ##2]}%
%    \end{macrocode}
%    The definition of |\@opargbegintheorem| is completely
%    analogous. The only difference is the fact that there exists a
%    third argument (which is the optional parameter of the
%    environment and contains additional information about the
%    theorem).  Customarily we enclose it in parentheses.
%    \begin{macrocode}
\def\@opargbegintheorem##1##2##3{%
   \item[\hskip\labelsep \theorem@headerfont ##1\ ##2\ (##3)]}}
%    \end{macrocode}
% \end{macro}
% \end{macro}
% \end{macro}
%
% \noindent
% We conclude with an |\endgroup| to restore the
% |\catcode| of |@|.
%    \begin{macrocode}
\endgroup
%</thp>
%    \end{macrocode}
%
%
% \subsubsection{The \texttt{break} style}
%
% This style option is stored in the file |thb.sty|.
% For the next two lines see the documentation for |\th@plain| on
% page~\pageref{dok:th@plain}.
%    \begin{macrocode}
%<*thb>
\begingroup \makeatletter
\@ifundefined{theorem@style}{\input{theorem.sty}}{}
\ProvidesFile{thb.sty}
             [\FMithmInfo]
%    \end{macrocode}
%
% \begin{macro}{\th@break}
%    |\theoremstyle{break}| produces a line break after the name
%    of the theorem.  The font is |\slshape|.  Hence, we define
%    |\th@break| as follows:
%    \begin{macrocode}
\gdef\th@break{\normalfont\slshape
  \def\@begintheorem##1##2{\item[%
%    \end{macrocode}
%    We run into the following problem: it is not possible to create
%    the header with |\item[|\meta{title}|]| and then start
%    a new line by, for example, |\mbox{}\\|. Such a definition
%    will fail whenever a list environment follows immediately.  With
%    the above construction, the |\mbox{}| causes the switch
%    |@inlabel| (cf.\ definition of |\list| and
%    |\trivlist| in \cite{latex:source}) to be set to \texttt{false}
%    and so the following list will insert additional vertical
%    space (|\topskip|). This is quite annoying. Therefore, we
%    create the line break within the |\item|. In order to ensure
%    that the text will begin at the proper position in the following
%    line, we simply pretend that the label does not take any
%    room.\footnote{This will lead to problems whenever very high
%    symbols occurring in the line tower into the
%    heading. So, something else has to be done here sometime.}
%    \begin{macrocode}
       \rlap{\vbox{\hbox{\hskip \labelsep\theorem@headerfont ##1\ ##2}%
                   \hbox{\strut}}}]}%
%    \end{macrocode}
%    Again, the definition of |\@opargbegintheorem| is
%    completely analogous.
%    \begin{macrocode}
\def\@opargbegintheorem##1##2##3{%
  \item[\rlap{\vbox{\hbox{\hskip \labelsep \theorem@headerfont
                     ##1\ ##2\ (##3)}%
                    \hbox{\strut}}}]}}
\endgroup
%</thb>
%    \end{macrocode}
% \end{macro}
%
%
% \subsubsection{The \texttt{changebreak} style}
%
%    \begin{macrocode}
% This style option is stored in the file |thcb.sty|.
%    \begin{macrocode}
%<*thcb>
\begingroup \makeatletter
\@ifundefined{theorem@style}{\input{theorem.sty}}{}
\ProvidesFile{thcb.sty}
             [\FMithmInfo]
%    \end{macrocode}
%
% \begin{macro}{\th@changebreak}
%    The \texttt{change-break} theorem style is like \texttt{break}
%    but with interchange of theorem name and theorem number.  Thus we
%    define |\th@changebreak| as follows:
%    \begin{macrocode}
\gdef\th@changebreak{\normalfont\slshape
  \def\@begintheorem##1##2{\item
       [\rlap{\vbox{\hbox{\hskip\labelsep \theorem@headerfont ##2\ ##1}%
                    \hbox{\strut}}}]}%
\def\@opargbegintheorem##1##2##3{%
  \item[\rlap{\vbox{\hbox{\hskip\labelsep \theorem@headerfont
                     ##2\ ##1\ (##3)}%
                    \hbox{\strut}}}]}}
\endgroup
%</thcb>
%    \end{macrocode}
% \end{macro}
%
%
% \subsubsection{The \texttt{change} style}
%
% This style option is stored in the file |thc.sty|.
%    \begin{macrocode}
%<*thc>
\begingroup \makeatletter
\@ifundefined{theorem@style}{\input{theorem.sty}}{}
\ProvidesFile{thc.sty}
             [\FMithmInfo]
%    \end{macrocode}
%
% \begin{macro}{\th@change}
%    The \texttt{change} theorem style corresponds to the
%    \texttt{change break} style without a linebreak after the header.
%    To say it in another way, it's the same as the \texttt{plain}
%    style but with number and name interchanged and |\slshape| as the
%    default font.
%    \begin{macrocode}
\gdef\th@change{\normalfont\slshape
  \def\@begintheorem##1##2{\item
       [\hskip\labelsep \theorem@headerfont ##2\ ##1]}%
\def\@opargbegintheorem##1##2##3{%
  \item[\hskip\labelsep \theorem@headerfont ##2\ ##1\ (##3)]}}
\endgroup
%</thc>
%    \end{macrocode}
% \end{macro}
%
%
% \subsubsection{The \texttt{marginbreak} style}
%
% This style option is the one used most often at Mainz.  It is saved
% in the file |thmb.sty|.
%    \begin{macrocode}
%<*thmb>
\begingroup \makeatletter
\@ifundefined{theorem@style}{\input{theorem.sty}}{}
\ProvidesFile{thmb.sty}
             [\FMithmInfo]
%    \end{macrocode}
%
% \begin{macro}{\th@marginbreak}
%    The \texttt{margin break} style is nearly the same as the
%    \texttt{change break} style. The only difference is the placement
%    of the theorem number.  We use |\llap| to place it in the left
%    margin.
%
%    In this style |\labelsep| denotes the separation between the
%    number and the text.
%    \begin{macrocode}
\gdef\th@marginbreak{\normalfont\slshape
  \def\@begintheorem##1##2{\item
       [\rlap{\vbox{\theorem@headerfont
                    \hbox{\llap{##2}\hskip\labelsep ##1}%
                    \hbox{\strut}}}]}%
\def\@opargbegintheorem##1##2##3{%
  \item[\rlap{\vbox{\theorem@headerfont
                    \hbox{\llap{##2}\hskip\labelsep ##1\ (##3)}%
                    \hbox{\strut}}}]}}
\endgroup
%</thmb>
%    \end{macrocode}
% \changes{v2.1c}{90/02/17}{`endgroup outside macrocode env.}
% \end{macro}
%
%
%
% \subsubsection{The \texttt{margin} style}
%
% This style option is stored in the file |thm.sty|.
%    \begin{macrocode}
%<*thm>
\begingroup \makeatletter
\@ifundefined{theorem@style}{\input{theorem.sty}}{}
\ProvidesFile{thm.sty}
             [\FMithmInfo]
%    \end{macrocode}
%
% \begin{macro}{\th@margin}
%    Again this is only a variant of the theorem styles described
%    above without any new ideas.
%    \begin{macrocode}
\gdef\th@margin{\normalfont\slshape
  \def\@begintheorem##1##2{\item
       [\theorem@headerfont \llap{##2}\hskip\labelsep ##1]}%
\def\@opargbegintheorem##1##2##3{%
  \item[\theorem@headerfont \llap{##2}\hskip\labelsep ##1\ (##3)]}}
\endgroup
%</thm>
%    \end{macrocode}
% \end{macro}
%
%
%
%
% \subsection{Final Definitions}
%
% \begin{macro}{\theorempreskipamount}
% \begin{macro}{\theorempostskipamount}
%    The \textsf{skip} parameters that regulate the vertical empty space
%    before and after the {\envfont theorem} environment have
%    to be allocated as well.
% \changes{v2.0f}{89/07/20}{Unnecessary `global' removed.}
%    \begin{macrocode}
%<*package>
\newskip\theorempreskipamount
\newskip\theorempostskipamount
%    \end{macrocode}
%    Since we have used the same values for all theorem sets, we now can
%    assign them.
%    \begin{macrocode}
\global\setlength\theorempreskipamount{12pt plus 5pt minus 3pt}
\global\setlength\theorempostskipamount{8pt plus 3pt minus 1.5pt}
%    \end{macrocode}
% \end{macro}
% \end{macro}
%
%
% \begin{macro}{\@endtheorem}
%    The same holds for the macro |\@endtheorem|, which ends a
%    {\envfont theorem} environment. Since it is the same for all
%    theorem sets, it is removed from the macros |\th@|\meta{style}.
%    It simply ends the {\envfont trivlist} environment, which was
%    begun in |\@thm|.
%    \begin{macrocode}
\global\let\@endtheorem=\endtrivlist
%    \end{macrocode}
% \end{macro}
%
%
%
% \begin{macro}{\@preamblecmds}
%    All macros defined above are to be used only in the preamble.
%    Therefore, we insert them in |\@preamblecmds| which will disable
%    them at begin document. This is done by the internal \LaTeXe{}
%    command |\@onlypreamble|.
% \changes{v2.0g}{89/09/19}{`theoremheaderfont and `theorembodyfont
%    added.}
% \changes{v2.2a}{94/02/03}{Use \LaTeXe{} interface}
%    \begin{macrocode}
\@onlypreamble\@xnthm
\@onlypreamble\@ynthm
\@onlypreamble\@othm
\@onlypreamble\newtheorem
\@onlypreamble\theoremstyle
\@onlypreamble\theorembodyfont
\@onlypreamble\theoremheaderfont
%    \end{macrocode}
% \end{macro}
%
% Finally we declare the \texttt{plain} theorem style to be the default.
% \changes{v2.1b}{90/06/04}{Fixed missing `item by declaring default
%    style.}
%    \begin{macrocode}
\theoremstyle{plain}
%</package>
%    \end{macrocode}
%
% \Finale

\endinput
}{}
\ProvidesFile{thm.sty}
             [\FMithmInfo]
%    \end{macrocode}
%
% \begin{macro}{\th@margin}
%    Again this is only a variant of the theorem styles described
%    above without any new ideas.
%    \begin{macrocode}
\gdef\th@margin{\normalfont\slshape
  \def\@begintheorem##1##2{\item
       [\theorem@headerfont \llap{##2}\hskip\labelsep ##1]}%
\def\@opargbegintheorem##1##2##3{%
  \item[\theorem@headerfont \llap{##2}\hskip\labelsep ##1\ (##3)]}}
\endgroup
%</thm>
%    \end{macrocode}
% \end{macro}
%
%
%
%
% \subsection{Final Definitions}
%
% \begin{macro}{\theorempreskipamount}
% \begin{macro}{\theorempostskipamount}
%    The \textsf{skip} parameters that regulate the vertical empty space
%    before and after the {\envfont theorem} environment have
%    to be allocated as well.
% \changes{v2.0f}{89/07/20}{Unnecessary `global' removed.}
%    \begin{macrocode}
%<*package>
\newskip\theorempreskipamount
\newskip\theorempostskipamount
%    \end{macrocode}
%    Since we have used the same values for all theorem sets, we now can
%    assign them.
%    \begin{macrocode}
\global\setlength\theorempreskipamount{12pt plus 5pt minus 3pt}
\global\setlength\theorempostskipamount{8pt plus 3pt minus 1.5pt}
%    \end{macrocode}
% \end{macro}
% \end{macro}
%
%
% \begin{macro}{\@endtheorem}
%    The same holds for the macro |\@endtheorem|, which ends a
%    {\envfont theorem} environment. Since it is the same for all
%    theorem sets, it is removed from the macros |\th@|\meta{style}.
%    It simply ends the {\envfont trivlist} environment, which was
%    begun in |\@thm|.
%    \begin{macrocode}
\global\let\@endtheorem=\endtrivlist
%    \end{macrocode}
% \end{macro}
%
%
%
% \begin{macro}{\@preamblecmds}
%    All macros defined above are to be used only in the preamble.
%    Therefore, we insert them in |\@preamblecmds| which will disable
%    them at begin document. This is done by the internal \LaTeXe{}
%    command |\@onlypreamble|.
% \changes{v2.0g}{89/09/19}{`theoremheaderfont and `theorembodyfont
%    added.}
% \changes{v2.2a}{94/02/03}{Use \LaTeXe{} interface}
%    \begin{macrocode}
\@onlypreamble\@xnthm
\@onlypreamble\@ynthm
\@onlypreamble\@othm
\@onlypreamble\newtheorem
\@onlypreamble\theoremstyle
\@onlypreamble\theorembodyfont
\@onlypreamble\theoremheaderfont
%    \end{macrocode}
% \end{macro}
%
% Finally we declare the \texttt{plain} theorem style to be the default.
% \changes{v2.1b}{90/06/04}{Fixed missing `item by declaring default
%    style.}
%    \begin{macrocode}
\theoremstyle{plain}
%</package>
%    \end{macrocode}
%
% \Finale

\endinput
}{}
\ProvidesFile{thm.sty}
             [\FMithmInfo]
%    \end{macrocode}
%
% \begin{macro}{\th@margin}
%    Again this is only a variant of the theorem styles described
%    above without any new ideas.
%    \begin{macrocode}
\gdef\th@margin{\normalfont\slshape
  \def\@begintheorem##1##2{\item
       [\theorem@headerfont \llap{##2}\hskip\labelsep ##1]}%
\def\@opargbegintheorem##1##2##3{%
  \item[\theorem@headerfont \llap{##2}\hskip\labelsep ##1\ (##3)]}}
\endgroup
%</thm>
%    \end{macrocode}
% \end{macro}
%
%
%
%
% \subsection{Final Definitions}
%
% \begin{macro}{\theorempreskipamount}
% \begin{macro}{\theorempostskipamount}
%    The \textsf{skip} parameters that regulate the vertical empty space
%    before and after the {\envfont theorem} environment have
%    to be allocated as well.
% \changes{v2.0f}{89/07/20}{Unnecessary `global' removed.}
%    \begin{macrocode}
%<*package>
\newskip\theorempreskipamount
\newskip\theorempostskipamount
%    \end{macrocode}
%    Since we have used the same values for all theorem sets, we now can
%    assign them.
%    \begin{macrocode}
\global\setlength\theorempreskipamount{12pt plus 5pt minus 3pt}
\global\setlength\theorempostskipamount{8pt plus 3pt minus 1.5pt}
%    \end{macrocode}
% \end{macro}
% \end{macro}
%
%
% \begin{macro}{\@endtheorem}
%    The same holds for the macro |\@endtheorem|, which ends a
%    {\envfont theorem} environment. Since it is the same for all
%    theorem sets, it is removed from the macros |\th@|\meta{style}.
%    It simply ends the {\envfont trivlist} environment, which was
%    begun in |\@thm|.
%    \begin{macrocode}
\global\let\@endtheorem=\endtrivlist
%    \end{macrocode}
% \end{macro}
%
%
%
% \begin{macro}{\@preamblecmds}
%    All macros defined above are to be used only in the preamble.
%    Therefore, we insert them in |\@preamblecmds| which will disable
%    them at begin document. This is done by the internal \LaTeXe{}
%    command |\@onlypreamble|.
% \changes{v2.0g}{89/09/19}{`theoremheaderfont and `theorembodyfont
%    added.}
% \changes{v2.2a}{94/02/03}{Use \LaTeXe{} interface}
%    \begin{macrocode}
\@onlypreamble\@xnthm
\@onlypreamble\@ynthm
\@onlypreamble\@othm
\@onlypreamble\newtheorem
\@onlypreamble\theoremstyle
\@onlypreamble\theorembodyfont
\@onlypreamble\theoremheaderfont
%    \end{macrocode}
% \end{macro}
%
% Finally we declare the \texttt{plain} theorem style to be the default.
% \changes{v2.1b}{90/06/04}{Fixed missing `item by declaring default
%    style.}
%    \begin{macrocode}
\theoremstyle{plain}
%</package>
%    \end{macrocode}
%
% \Finale

\endinput
}{}
\ProvidesFile{thb.sty}
             [\FMithmInfo]
%    \end{macrocode}
%
% \begin{macro}{\th@break}
%    |\theoremstyle{break}| produces a line break after the name
%    of the theorem.  The font is |\slshape|.  Hence, we define
%    |\th@break| as follows:
%    \begin{macrocode}
\gdef\th@break{\normalfont\slshape
  \def\@begintheorem##1##2{\item[%
%    \end{macrocode}
%    We run into the following problem: it is not possible to create
%    the header with |\item[|\meta{title}|]| and then start
%    a new line by, for example, |\mbox{}\\|. Such a definition
%    will fail whenever a list environment follows immediately.  With
%    the above construction, the |\mbox{}| causes the switch
%    |@inlabel| (cf.\ definition of |\list| and
%    |\trivlist| in \cite{latex:source}) to be set to \texttt{false}
%    and so the following list will insert additional vertical
%    space (|\topskip|). This is quite annoying. Therefore, we
%    create the line break within the |\item|. In order to ensure
%    that the text will begin at the proper position in the following
%    line, we simply pretend that the label does not take any
%    room.\footnote{This will lead to problems whenever very high
%    symbols occurring in the line tower into the
%    heading. So, something else has to be done here sometime.}
%    \begin{macrocode}
       \rlap{\vbox{\hbox{\hskip \labelsep\theorem@headerfont ##1\ ##2}%
                   \hbox{\strut}}}]}%
%    \end{macrocode}
%    Again, the definition of |\@opargbegintheorem| is
%    completely analogous.
%    \begin{macrocode}
\def\@opargbegintheorem##1##2##3{%
  \item[\rlap{\vbox{\hbox{\hskip \labelsep \theorem@headerfont
                     ##1\ ##2\ (##3)}%
                    \hbox{\strut}}}]}}
\endgroup
%</thb>
%    \end{macrocode}
% \end{macro}
%
%
% \subsubsection{The \texttt{changebreak} style}
%
%    \begin{macrocode}
% This style option is stored in the file |thcb.sty|.
%    \begin{macrocode}
%<*thcb>
\begingroup \makeatletter
\@ifundefined{theorem@style}{% \iffalse meta-comment
%
% Copyright (C) 1993-2024
%
% The LaTeX Project and any individual authors listed elsewhere
% in this file.
%
% This file is part of the Standard LaTeX `Tools Bundle'.
% -------------------------------------------------------
%
% It may be distributed and/or modified under the
% conditions of the LaTeX Project Public License, either version 1.3c
% of this license or (at your option) any later version.
% The latest version of this license is in
%    https://www.latex-project.org/lppl.txt
% and version 1.3c or later is part of all distributions of LaTeX
% version 2005/12/01 or later.
%
% The list of all files belonging to the LaTeX `Tools Bundle' is
% given in the file `manifest.txt'.
%
% \fi
%
%%
%
%
\def\FMithmInfo{2023/07/05 v2.2c Theorem extension package (FMi)}
%
% \ProvidesFile{theorem.dtx}[\FMithmInfo]
%\iffalse   % this is a METACOMMENT !
%%
%% Package `theorem' to use with LaTeX2e
%% Copyright (C) 1989-1995 Frank Mittelbach, all rights reserved.
%<+package>\@ifundefined{theorem@style}{}{\endinput}
%<+package>\NeedsTeXFormat{LaTeX2e}[1995/06/01]
%<+package>\ProvidesPackage{theorem}
%<+package>         [\FMithmInfo]
%
% \fi
%
% \changes{v1.0}{87/05/07}{First draft.}
% \changes{v1.0a}{87/05/07}{`thistheoremstyle' added.}
% \changes{v1.1a}{87/07/29}{New concept: redefine the internal
%                          \protect\LaTeX{} commands `@xnthm' etc.}
% \changes{v1.1b}{87/07/31}{`theoremfont' used to specify font.}
% \changes{v1.2a}{88/01/08}{`@endtheorem' now globally defined.}
% \changes{v1.2b}{88/01/11}{`newtheorem' only allowed in the preamble.}
% \changes{v1.3a}{88/02/14}{`theoremfont' now a toks register.}
%
% \changes{v2.0a}{88/11/16}{New implementation and new documentation.}
% \changes{v2.0a}{88/11/16}{No layout changes but internal modifications
%                           old substyles won't work.}
% \changes{v2.0b}{89/01/30}{`global' used for all definitions.}
% \changes{v2.0c}{89/06/15}{Documentation translated to English.
%                            Thanks to LLL and Nina}
% \changes{v2.1a}{90/02/15}{Support for new font selection.}
% \changes{v2.2a}{94/02/03}{Cleanup for LaTeX2e}
%
% \DoNotIndex{\@endparenv,\@ifdefinable,\@ifnextchar,\@ifundefined}
% \DoNotIndex{\@namedef,\@nameuse,\@spaces,\@tempa,\@warning}
% \DoNotIndex{\ }
% \DoNotIndex{\begingroup,\bfseries}
% \DoNotIndex{\csname}
% \DoNotIndex{\def,\do,\docdate}
% \DoNotIndex{\endcsname,\endgroup,\endinput,\endtrivlist,\expandafter}
% \DoNotIndex{\filedate,\fileversion}
% \DoNotIndex{\gdef,\global}
% \DoNotIndex{\hbox,\hskip}
% \DoNotIndex{\ignorespaces,\input,\itshape,\item}
% \DoNotIndex{\labelsep,\let,\llap}
% \DoNotIndex{\makeatletter}
% \DoNotIndex{\noexpand}
% \DoNotIndex{\refstepcounter,\relax,\rlap}
% \DoNotIndex{\setlength,\slshape,\space,\string,\strut}
% \DoNotIndex{\the,\trivlist,\typeout}
% \DoNotIndex{\vbox}
% \DoNotIndex{\xdef}
%
% \def\deflabel#1{\ttfamily #1\hfill}
% \def\deflist#1{\begin{list}{}{\settowidth\labelwidth{\ttfamily #1}%
%                               \setlength\leftmargin\labelwidth
%                               \addtolength\leftmargin\labelsep
%                               \let\makelabel\deflabel}}
% \def\enddeflist{\end{list}}
% \def\envfont{\normalfont\ttfamily}
%
% \MakeShortVerb{\|}
%
% \newcommand{\TUB}{{\sl TUGboat\/}}
%
% \GetFileInfo{theorem.dtx}
% \title{An Extension of the
%        \LaTeX{} theorem environment\thanks{This file
%        has version number \fileversion, last
%        revised \filedate.}}
% \author{Frank Mittelbach}
%
% \MaintainedByLaTeXTeam{tools}
% \maketitle
%
% \begin{abstract}
%    The macros described in this paper yield an extension of the
%    \LaTeX{} theorem mechanism. It is designed to satisfy the
%    different requirements of various journals. Thus, the
%    layout of the ``theorems'' can be manipulated by determining a
%    ``style''. This article describes not only the use, but
%    also the definition, of the necessary macros.
% \end{abstract}
%
% \section*{Preface to version 2.2}
%
% For \LaTeXe{} this package did not need any fundamental changes. I
% only modified the messages generated so that theorem layout styles
% will show up with the |\listfiles| command and cleaned the section
% on the New Font Selection Scheme since this is now included in
% \LaTeX.
%
% \section*{Preface to version 2.1}
%
% This version is identical to 2.0g described in \TUB\ 10\#3 except
% for some internal defaults which are now set depending on the used
% font selection scheme.
%
% This was done to avoid unpleasant surprises if the new font selection
% scheme is in force. For further details see
% section~\ref{sec:newfonts} and \cite{bk:GMS94}.
%
% \section{Introduction}
%
% For our purposes here, ``theorems'' are labelled enunciations,
% often set off from the main text by extra space and a font change.
% Theorems, corollaries, conjectures, definitions, and remarks
% are all instances of ``theorems''.  The ``header'' of these structures
% is composed of a label (such as \textsc{Theorem} or \textsc{Remark})
% and a number which serializes an item in the sequence of items
% with the same label.
%
% Shortly after the introduction of \LaTeX{} at the Fachbereich
% Mathematik in Mainz, the desire to manipulate the layout of
% ``theorems'' arose.  In Mainz, the following two conventions came
% into general use:
% \begin{enumerate}
%   \item The number of the theorem is shown in the margin.
%   \item There is a line break at
%         the end of the theorem header.
% \end{enumerate}
%
% Additionally, some journals require different formats which depend on
% the ``sort of theorem'': e.g.\ often remarks and definitions are set
% in |\upshape|, while |\itshape| is employed for main theorems.
%
% Confronted with these requirements, a theorem environment was
% developed in Mainz which allows separate determination of the
% layout of the ``theorems sets'', comparable to |\pagestyle|.
%
%
% \section{The user interface}
% \subsection{Defining new theorem sets}
%
% \DescribeMacro\newtheorem
% As in the original \LaTeX{} version, the command |\newtheorem|
% defines a new ``theorem set'' or ``theorem-like structure''.
% Two required arguments name the new environment and give the text
% to be typeset with each instance of the new ``set'', while
% an optional argument determines how
% the ``set'' is enumerated:
% \begin{description}
%    \item[\ttfamily \bslash newtheorem\{foo\}\{bar\}]
%       The theorem set {\envfont foo} (whose name is \texttt{bar})
%       uses its own counter.
%    \item[\ttfamily \bslash newtheorem\{foo2\}{[foo]}\{bar2\}]
%       The theorem set {\envfont foo2} (printed name \texttt{bar2})
%       uses the same counter as the theorem set \texttt{foo}.
%    \item[\ttfamily \bslash newtheorem\{foo3\}\{bar3\}{[section]}]
%       The theorem set {\envfont foo3} (printed name \texttt{bar3}) is
%       enumerated within the counter \texttt{section}, i.e.\ with every
%       new |\section| the enumeration begins again with 1, and
%       the enumeration is composed from the section-number and the
%       theorem counter itself.
% \end{description}
%
% \DescribeMacro\theoremstyle
% Additionally, the command |\theoremstyle| can define the layout
% of various, or all, theorem sets.  It should be noted that any
% theorem set defined by |\newtheorem| is typeset in the
% |\theoremstyle| that is current at the time of the definition.
% Thus, the following
% \begin{verbatim}
%    \theoremstyle{break}        \newtheorem{Cor}{Corollary}
%    \theoremstyle{plain}        \newtheorem{Exa}{Example}[section]
%\end{verbatim}                             ^^A No blank at line start
% leads to the result that the set {\envfont Cor} is formatted in the
% style \texttt{break}, while the set {\envfont Exa} and all the
% following ones are formatted in the style \texttt{plain}, unless
% another |\theoremstyle| follows.  Since the definitions installed by
% |\newtheorem| are global, one also can limit |\theoremstyle| locally
% by grouping braces.
%
% \DescribeMacro\theorembodyfont
% The choice of the font for the theorem body is completely
% independent of the chosen |\theoremstyle|; this has proven to
% be very advantageous.  For example,
% \begin{verbatim}
%   {\theorembodyfont{\upshape}          \newtheorem{Rem}{Remark}}
%\end{verbatim}
% defines a theorem set {\envfont Rem}, which will be set in
% |\upshape| in the current layout (which in our example is
% \texttt{plain}).  As with |\theoremstyle|, the |\theorembodyfont|
% chosen is that current at the time of |\newtheorem|.  If
% |\theorembodyfont| is not specified or one defines
% |\theorembodyfont{}|, then the font used will be that defined by the
% |\theoremstyle|.
%
% \DescribeMacro\theoremheaderfont It is also possible to customize
% the font used for the theorem headers.  This is, however, a global
% declaration, and therefore there should be at most one
% |\theoremheaderfont| declaration in the preamble.\footnote{If it is
% actually necessary to have different header fonts, one has to define
% new theorem styles (substituting the desired font) or specify the
% information directly in the \texttt{\bslash newtheorem} declaration
% (the unclean variant).}
%
% \DescribeMacro\theorempreskipamount
% \DescribeMacro\theorempostskipamount Two additional parameters
% affect the vertical space around the theorem environments:
% |\theorempreskipamount| and |\theorempostskipamount| define,
% respectively, the spacing before and after such an environment.
% These parameters apply for all theorem sets and can be manipulated
% with the ordinary length macros.  They are rubber lengths
% (`\textsf{skips}'), and therefore can contain \texttt{plus} and
% \texttt{minus} parts.
%
% Since the definition of theorem sets should---most sensibly---be
% placed in the preamble, we only allow installation there.
% It is therefore possible to release the memory used here after
% |\begin{document}|, in order to make room for other
% applications.
%
%
% \subsection{Existing theorem styles}
%
% The following theorem styles exist to date:
% \begin{deflist}{marginbreak}
%    \item[plain]
%       This theorem style emulates the original \LaTeX{} definition,
%       except that additionally the parameters
%       |\theorem...skipamount| are used.
%    \item[break]
%       In this style, the theorem header is followed by a line
%       break.
%    \item[marginbreak]
%       The theorem number is set in the margin, and there is a line
%       break as in \texttt{break}.
%    \item[changebreak]
%       Like \texttt{break}, but with header number and text
%       interchanged.
%    \item[change]
%       Header number and text are interchanged, without a line break.
%    \item[margin]
%       The number is set in the left margin, without a line break.
% \end{deflist}
% All styles (except \texttt{plain}) select |\slshape| as the default
% |\theorembodyfont|.
%
%
% \subsection{Examples}
%
% Given the above theorem sets {\envfont Cor}, {\envfont Exa}
% and {\envfont Rem},
% suppose that the preamble also contains the declarations:
% \begin{verbatim}
%    \theoremstyle{marginbreak}   \newtheorem{Lem}[Cor]{Lemma}
%    \theoremstyle{change}
%    \theorembodyfont{\itshape}        \newtheorem{Def}[Cor]{Definition}
%
%    \theoremheaderfont{\scshape}
%\end{verbatim}
% Then the following are some typical examples of the typeset output
% resulting from their use.
%
% \begin{Cor}
%    This is a sentence typeset in the theorem environment {\envfont
%    Cor}.
% \end{Cor}
%
% \begin{Exa}
%    This is a sentence typeset in the theorem environment {\envfont
%    Exa}.
% \end{Exa}
%
% \begin{Rem}
%    This is a sentence typeset in the theorem environment {\envfont
%    Rem}.
% \end{Rem}
%
% \begin{Lem}[Ben User]
%    This is a sentence typeset in the theorem environment {\envfont
%    Lem}.
% \end{Lem}
%
% \begin{Def}[Very impressive Definition]
%    This is a sentence typeset in the theorem environment {\envfont
%    Def}.
% \end{Def}
%
% The last two examples show the effect of the optional argument to
% a theorem environment (it is the text typeset in parentheses).
%
%
% \section{Special Considerations}
% \label{sec:newfonts}
%
% Theoremheader and body are implemented as a unit. This means that
% the |\theoremheaderfont|  will inherit characteristics of the
% |\theorembodyfont| in \LaTeXe. Thus, if for example
% |\theorembodyfont| is |\itshape| and |\theoremheaderfont|
% is |\bfseries| the font selected for the header will have the
% characteristics `bold extended italic'. If this is not desired one
% should set the |\theoremheaderfont| to something like
% \begin{verbatim}
%      \theoremheaderfont{\normalfont\bfseries}
%\end{verbatim}
% i.e.\ supplying all necessary font information explicitly.
%
%
%
% \section{Acknowledgements}
%
% The publication of this set of macros was only possible with the
% help of Christina Busse (translating the manuscript into English),
% Joachim Pense (playing the r\^ole of typist), Chris Rowley
% (looking everything over) and many others providing useful
% suggestions.
%
% \MaybeStop{%
%    \begin{thebibliography}{9}
%       \bibitem{bk:GMS94} \textsc{M.~Goossens}, \textsc{F.~Mittelbach}
%         and \textsc{A.~Samarin}.
%         \newblock The \LaTeX{} Companion.
%         \newblock
%         Addison-Wesley, Reading, Massachusetts, 1994.
%       \bibitem{latex:source} \textsc{Lamport, Leslie.} \newblock
%                               \texttt{latex.tex}, version 2.09, date
%                               Feb. 1990.
%    \end{thebibliography}
%    \PrintIndex
%    \PrintChanges}
%
%
% \section{The documentation driver file}
%
% The next bit of code contains the documentation driver file for
% \TeX{}, i.e., the file that will produce the documentation you are
% currently reading. It will be extracted from this file by the
% \texttt{docstrip} program. Since it is the first code in the file
% one can alternatively process this file directly with \LaTeXe{} to
% obtain the documentation.
%
%    \begin{macrocode}
%<*driver>
\documentclass{ltxdoc}

\usepackage{theorem}

%<+driver>% The next few lines define theorem sets which are used
%<+driver>% in the example section of the documentation.

\theoremstyle{break}       \newtheorem{Cor}{Corollary}
\theoremstyle{plain}       \newtheorem{Exa}{Example}[section]
{\theorembodyfont{\upshape}\newtheorem{Rem}{Remark}}
\theoremstyle{marginbreak} \newtheorem{Lem}[Cor]{Lemma}
\theoremstyle{change}
\theorembodyfont{\itshape} \newtheorem{Def}[Cor]{Definition}

\theoremheaderfont{\scshape}

\RecordChanges

\begin{document}
   \DocInput{theorem.dtx}
\end{document}
%</driver>
%    \end{macrocode}
%
% \section{Definition of the Macros}
%
% If the file has been loaded before, we abort immediately. If not the
% package announces itself (this is actually done at the very top of
% the file---the way it is done isn't good style so don't copy it).
% \changes{v2.0e}{89/07/19}{Spaces removed from `typeout'.}
% \changes{v2.2b}{95/11/19}{Announce moved to top of file}
%    \begin{macrocode}
%<*package>
%\@ifundefined{theorem@style}{}{\endinput}
%\def\FMithmInfo{1995/11/19 v2.2b Theorem extension package (FMi)}
%\ProvidesPackage{theorem}[\FMithmInfo]
%    \end{macrocode}
%
%
% \subsection{Definition of theorem styles and fonts}
%
% All the definitions in this file are done globally to allow
% inputting this file inside a group.
%
% \begin{macro}{\theoremstyle}
%    Before a theorem style can be installed, the chosen style must be
%    known. For that reason, we must test to see that
%    |\th@|\meta{style} is known or, more precisely, that it is
%    different from |\relax|. If the style is not known then
%    |\th@plain| is used.
%    \begin{macrocode}
\gdef\theoremstyle#1{%
   \@ifundefined{th@#1}{\@warning
          {Unknown theoremstyle `#1'. Using `plain'}%
          \theorem@style{plain}}%
%    \end{macrocode}
%    We save the theorem style to be used in the token register
%    |\theorem@style|.
%    \begin{macrocode}
      {\theorem@style{#1}}%
%    \end{macrocode}
%    Now we ``evaluate'' the theorem style: this means, we call the
%    macro |\th@|\meta{style} which will activate the relevant
%    definitions which are contained in a separate file.  This is done
%    in a group to suppress changes to the current font.  This could
%    otherwise pose problems together with the new font selection
%    scheme\footnote{When I printed the original article using the new
%    font selection scheme I ended with a document with slanted
%    typefaces (text headings and all) simply because one of the
%    theorem styles used \texttt{\bslash sl} at toplevel.} if the
%    |\th@|\meta{style} is evaluated a second time.
% \changes{v2.1a}{90/02/15}{Group added around `th@..}
% \changes{v2.1a}{90/02/15}{Reseting of `..@begintheorem removed.}
%    \begin{macrocode}
      \begingroup
        \csname th@\the\theorem@style \endcsname
      \endgroup}
%    \end{macrocode}
% \end{macro}
%
%
% \begin{macro}{\@begintheorem}
% \changes{v2.1a}{90/02/15}{Relaxed at toplevel.}
% \begin{macro}{\@opargbegintheorem}
% \changes{v2.1a}{90/02/15}{Relaxed at toplevel.}
%    We reset |\@begintheorem| and |\@opargbegintheorem| to
%    |\relax| since these commands are no longer necessary at
%    toplevel. This will save a few tokens.
%    \begin{macrocode}
\global\let\@begintheorem\relax
\global\let\@opargbegintheorem\relax
%    \end{macrocode}
% \end{macro}
% \end{macro}
%
% \begin{macro}{\theorem@style}
%    Obviously the \textsf{token} register used above has to be
%    allocated. To assure the utmost compatibility with the
%    original \LaTeX{} definition, we set the default theorem style to
%    \texttt{plain}, which implements the usual \LaTeX{} convention.
% \changes{v2.0f}{89/07/20}{Unnecessary `global' removed.}
%    \begin{macrocode}
\newtoks\theorem@style
\global\theorem@style{plain}
%    \end{macrocode}
% \end{macro}
%
% \begin{macro}{\theorembodyfont}
% \begin{macro}{\theorem@bodyfont}
%    If the |\theorembodyfont| is set by the user then it should not
%    interact with the default font set in the theorem style. When the
%    new font selection is in force this may happen if, for example,
%    the default is |\itshape| and the new |\theorembodyfont| is
%    |\sffamily|. So we add a |\reset@font| command in front of the
%    user definition.
% \changes{v2.0f}{89/07/20}{Unnecessary `global' removed.}
% \changes{v2.0g}{89/09/19}{`theoremfont to `theorembodyfont.}
% \changes{v2.1c}{92/06/22}{`reset@font added and logic changed.}
% \changes{v2.1d}{92/10/28}{Reset to empty if arg is empty}
%    \begin{macrocode}
\gdef\theorembodyfont#1{%
%    \end{macrocode}
%    We check if the argument supplied is empty and if so put nothing
%    into the |\theorem@bodyfont| token register to allow for
%    |\theorembodyfont{}| as a mean of using the default of the
%    current |\theoremstyle|.
%    \begin{macrocode}
   \def\@tempa{#1}%
   \ifx\@tempa\@empty
    \theorem@bodyfont{}%
   \else
    \theorem@bodyfont{\reset@font#1}%
   \fi
   }
\newtoks\theorem@bodyfont
\global\theorem@bodyfont{}
%    \end{macrocode}
% \end{macro}
% \end{macro}
%
%
% \begin{macro}{\theoremheaderfont}
% \changes{v2.0g}{89/09/19}{Macro added.}
%    The font for the theorem headers is handled differently because
%    this definition applies to all theorem styles.
%    \begin{macrocode}
\gdef\theoremheaderfont#1{\gdef\theorem@headerfont{#1}%
%    \end{macrocode}
%    After using the macro once it is redefined to produce an error
%    message.
%    \begin{macrocode}
       \gdef\theoremheaderfont##1{%
        \typeout{\string\theoremheaderfont\space should be used
                 only once.}}}
%    \end{macrocode}
% \end{macro}
%
% \begin{macro}{\theorem@headerfont}
% \changes{v2.0g}{89/09/19}{Macro added.}
%    To set the |\theorem@headerfont| default we first test if
%    the new fontselection scheme is in force.
% \changes{v2.1a}{90/02/15}{Support for new font selection.}
%    \begin{macrocode}
\ifx\upshape\undefined
%    \end{macrocode}
%    If not we define it to expand into |\bfseries|. We don't use
%    |\let| just in case a following style option redefines this
%    macro.
%    \begin{macrocode}
\gdef\theorem@headerfont{\bfseries}
%    \end{macrocode}
%    Otherwise we reset the current shape before calling |\bfseries|.
%    \begin{macrocode}
\else \gdef\theorem@headerfont{\normalfont\bfseries}\fi
%    \end{macrocode}
% \end{macro}
%
% \begin{macro}{\th@plain}
% \begin{macro}{\th@break}
% \begin{macro}{\th@marginbreak}
% \begin{macro}{\th@changebreak}
% \begin{macro}{\th@change}
% \begin{macro}{\th@margin}
%    The different styles are defined in macros such as
%    |\th@plain|.  Since memory space is precious in
%    ``non-Big-versions'', we have to avoid offering too many unused
%    definitions.  Therefore we define these styles in separate files
%    that can be loaded on demand.  Thus the commands themselves only
%    load these files. We use |\@input@|, a \LaTeXe{} internal command
%    that ensures that the file will be listed with |\listfiles|.
% \changes{v2.2a}{94/02/02}{Use `@input@ to load theorem layout files}
%    \begin{macrocode}
\gdef\th@plain{\@input@{thp.sty}}
\gdef\th@break{\@input@{thb.sty}}
\gdef\th@marginbreak{\@input@{thmb.sty}}
\gdef\th@changebreak{\@input@{thcb.sty}}
\gdef\th@change{\@input@{thc.sty}}
\gdef\th@margin{\@input@{thm.sty}}
%    \end{macrocode}
%    This list will be expanded when new styles become
%    available.  For testing,  just append new theorem substyles
%    as document options.
% \end{macro}
% \end{macro}
% \end{macro}
% \end{macro}
% \end{macro}
% \end{macro}
%
%
%
% \subsection{Definition of a new theorem set}
%
% As already pointed out, a new theorem environment can be defined in
% three different ways:
% \begin{verbatim}
%    \newtheorem{Lem}{Lemma}
%    \newtheorem{Lem}{Lemma}[section]
%    \newtheorem{Lem}[Theorem]{Lemma}
%\end{verbatim}
% The function of the macro |\newtheorem| is to recognize these cases
% and then to branch into one of the three macros |\@ynthm|, |\@xnthm|
% or |\@othm|.  This mechanism is adopted unchanged from
% \cite{latex:source}; the essential point here is that, for example,
% in the second case, the arguments \texttt{Lem}, \texttt{Lemma} and
% \texttt{section} are passed over to the macro |\@xnthm|.
%
% We inspect this case first because the others present fewer problems,
% and thus are easily derived from this one.
%
% \begin{macro}{\@xnthm}
%    For our example arguments, the macro |\@xnthm| must fulfill
%    the following:
%    \begin{itemize}
%       \item
%          Define a new \LaTeX-counter `Lem'
%       \item
%          reset this counter within a |\section|
%       \item
%          define the macro |\theLem|
%       \item
%          define the environment macros |\Lem| and
%          |\endLem| using the current
%          |\theoremstyle| and |\theorem@bodyfont|.
%    \end{itemize}
%    Obviously, all this should happen only if the first argument
%    of |\@xnthm| (i.e.\ \texttt{Lem} in our example) is chosen so
%    as not to conflict with any previously defined commands or
%    environments.  This test is performed by the \LaTeX{} macro
%    |\@ifdefinable|.
%    \begin{macrocode}
\gdef\@xnthm#1#2[#3]{\expandafter\@ifdefinable\csname #1\endcsname
%    \end{macrocode}
%    Therefore, the first argument of |\@ifdefinable| is the
%    expansion (in the example, |\Lem|) of
%    |\csname||#1\endcsname|.  The second argument is
%    executed only if the test has been completed successfully.
%    \begin{macrocode}
   {%
%    \end{macrocode}
%    Now we define the new counter. The names of the \LaTeX{}
%    macros employed should speak for themselves:
% \changes{v2.2b}{1995/10/07}{Use \cs{@newctr}}
% \changes{v2.2c}{1995/11/23}{Use \cs{@newctr} properly.}
%    \begin{macrocode}
    \@definecounter{#1}\@newctr{#1}[#3]%
%    \end{macrocode}
%    Using |\@newctr| will give a proper error message if the counter
%    in |#3| is not defined.
%    In defining `|\theLem|' we must generate the desired macro
%    name by use of |\expandafter| and |\csname|.
%    \begin{macrocode}
    \expandafter\xdef\csname the#1\endcsname
%    \end{macrocode}
%    An |\xdef| is used in order to make the definition global,
%    and to ensure that it contains the replacement texts of
%    |\@thmcountersep| and |\@thmcounter|.\footnote{These
%    two macros can be defined by the document style.  Their default
%    values produce a `\texttt{.}' as separation and an arabic
%    representation of the number.} However, not everything should be
%    expanded. For example, it saves space to use |\thesection|
%    instead of its---at times---lengthy expansion.
%    \begin{macrocode}
      {\expandafter \noexpand \csname the#3\endcsname
       \@thmcountersep \@thmcounter{#1}}%
%    \end{macrocode}
%    Thus with the defaults of \LaTeX{}, |\theLem| would be
%    replaced by the command sequence |\thesection.\arabic{Lem}|.
%
%    We will now look at the definition of the macro which is executed
%    at the beginning of the actual environment (in our example this
%    macro is |\Lem|). It should be noted that we use an
%    ``|\expandafter| trick'' to expand only certain parts of the
%    replacement text at the time of the definition.
%    \begin{macrocode}
    \def\@tempa{\global\@namedef{#1}}%
    \expandafter \@tempa \expandafter{%
%    \end{macrocode}
%    First, the macro that contains the current definitions of
%    |\@begintheorem| and |\@opargtheorem| should be called
%    up.  The name of this macro---as is already known---has the form
%    |\th@|\meta{theorem\ style}; therefore, it must be called by
%    \begin{macrocode}
      \csname th@\the \theorem@style
%    \end{macrocode}
%    In addition the default theorem font should be changeable, i.e.\
%    we have to insert the contents of |\theorem@bodyfont|. For
%    that reason, we expand even further, beyond |\endcsname|,
%    and thus insert the contents of the \textsf{token} register
%    |\theorem@bodyfont| in the replacement text.
%    \begin{macrocode}
            \expandafter \endcsname \the \theorem@bodyfont
%    \end{macrocode}
%    Now it is time to call the macro |\@thm| which takes over
%    the further processing.  It has two arguments: the current
%    counter name (in our example, \texttt{Lem}), and the text of the
%    label (in our example, \texttt{Lemma}).
%    \begin{macrocode}
     \@thm{#1}{#2}}%
%    \end{macrocode}
%    With this, the `sub-definition' is complete.  The macro
%    |\@endtheorem| ends a theorem environment and is, so far, nothing
%    but an |\endtrivlist|.  (Hence it is defined globally, and not
%    within the theorem styles.\footnote{This has to be changed as
%    soon as theorem styles that change \texttt{\bslash @endtheorem}
%    exist.  In such a case, all existing styles must be changed as
%    well since they will have to reset the macro.}) Therefore, we can
%    set it equivalent to the macro that ends the theorem set (in our
%    example, |\endLem|).  However, if some day theorem styles exist
%    that do change |\@endtheorem|, we would have to use the
%    commented-out line instead.
%    \begin{macrocode}
    \global \expandafter \let \csname end#1\endcsname \@endtheorem
%   \global\@namedef{end#1}{\@endtheorem}%
%    \end{macrocode}
%    With these commands all the required definitions are employed,
%    unless the test |\@ifdefinable| has failed.  Therefore, we
%    end the second argument of this macro and with it the definition
%    of |\@xnthm|.
%    \begin{macrocode}
   }}
%    \end{macrocode}
% \end{macro}
%
% \begin{macro}{\@ynthm}
%    The definition of |\@ynthm| is completely analogous. In
%    this case the new counter that is defined is not reset within
%    another counter; thus the definition of |\the...|
%    is simplified:
%    \begin{macrocode}
\gdef\@ynthm#1#2{\expandafter\@ifdefinable\csname #1\endcsname
   {\@definecounter{#1}%
    \expandafter\xdef\csname the#1\endcsname{\@thmcounter{#1}}%
%    \end{macrocode}
%    The rest of the definition corresponds literally to that of
%    |\@xnthm|:
%    \begin{macrocode}
    \def\@tempa{\global\@namedef{#1}}\expandafter \@tempa
     \expandafter{\csname th@\the \theorem@style \expandafter
     \endcsname \the\theorem@bodyfont \@thm{#1}{#2}}%
    \global \expandafter \let \csname end#1\endcsname \@endtheorem}}
%    \end{macrocode}
% \end{macro}
%
% \begin{macro}{\@othm}
%    The definition of |\@othm| does not contain anything new.
%    \begin{macrocode}
\gdef\@othm#1[#2]#3{%
%    \end{macrocode}
% \changes{v2.2b}{95/10/07}{Build in check for counter defined}
%    We do not define a new counter but instead use one that has
%    already been defined.  Thus the only definition we need is that
%    of this pseudo-counter (i.e.\ |\the|\meta{env.\ name}).
%    First we check if |#2| corresponds to a known counter name.
%    \begin{macrocode}
  \expandafter\ifx\csname c@#2\endcsname\relax
   \@nocounterr{#2}%
  \else
   \expandafter\@ifdefinable\csname #1\endcsname
   {\expandafter \xdef \csname the#1\endcsname
     {\expandafter \noexpand \csname the#2\endcsname}%
%    \end{macrocode}
%    All other parts of the definition can be adopted from
%    |\@xnthm|. We have to remember, though, that in this case
%    the name of the current counter and the theorem label have moved
%    to the second and third arguments.
%    \begin{macrocode}
    \def\@tempa{\global\@namedef{#1}}\expandafter \@tempa
     \expandafter{\csname th@\the \theorem@style \expandafter
     \endcsname \the\theorem@bodyfont \@thm{#2}{#3}}%
    \global \expandafter \let \csname end#1\endcsname \@endtheorem}%
  \fi}
%    \end{macrocode}
% \end{macro}
%
% \subsection{Macros that are employed in a {\envfont theorem}
%    environment}
%
% \begin{macro}{\@thm}
%    The macro |\@thm| has to increase the current counter.  Then,
%    depending on whether the environment has (or does not have) an
%    optional argument, it has to branch into either |\@begintheorem|
%    or |\@opargtheorem|.
%    \begin{macrocode}
\gdef\@thm#1#2{\refstepcounter{#1}%
%    \end{macrocode}
%    Now we start a {\envfont trivlist} environment, and give
%    |\@topsep| and |\@topsepadd| the values of the \textsf{skip}
%    registers |\theorempreskipamount| and |\theorempostskipamount|.
%    The value in |\@topsep| is the vertical space that is inserted by
%    the first (and only) |\item| in our |\trivlist| whilst
%    |\@topsepadd| is inserted by |\@endparenv| at the end of that
%    {\envfont trivlist} environment.  By using these registers, we
%    obtain the desired space around a {\envfont theorem} environment.
%    \begin{macrocode}
   \trivlist
   \@topsep \theorempreskipamount               % used by first \item
   \@topsepadd \theorempostskipamount           % used by \@endparenv
%    \end{macrocode}
%    Now we have to test whether an optional argument has been given.
% \changes{v2.0d}{89/07/18}{Forgotten percent added.}
%    \begin{macrocode}
   \@ifnextchar [%
%    \end{macrocode}
%    If there is an optional argument, we will call |\@ythm|, and
%    move the arguments read back into the input stream.
%    \begin{macrocode}
   {\@ythm{#1}{#2}}%
%    \end{macrocode}
%    If not, we call |\@begintheorem|. Its first argument is the
%    name of the theorem set (hence the second argument of
%    |\@thm|). Its second argument is the macro that produces the
%    current number.
%    \begin{macrocode}
   {\@begintheorem{#2}{\csname the#1\endcsname}\ignorespaces}}
%    \end{macrocode}
% \end{macro}
%
% \begin{macro}{\@xthm}
% \begin{macro}{\@ythm}
%    Both these macros were originally called by |\@thm|.  We do
%    not need |\@xthm| anymore, hence we reset it to
%    |\relax|. The definition of |\@ythm| has not changed at
%    all from its definition in \LaTeX. In order to make the macros
%    easier to understand, we will
%    nevertheless present it (commented out).
%    \begin{macrocode}
\global\let\@xthm\relax
% \def\@ythm#1#2[#3]{\@opargbegintheorem{#2}{\csname
%           the#1\endcsname}{#3}\ignorespaces}
%</package>
%    \end{macrocode}
%    The primitive |\ignorespaces| in |\@ythm| and
%    |\@thm| is needed to remove the spaces between the
%    |\begin{...}| and the actual text.
% \end{macro}
% \end{macro}
%
%
% \subsection{Definition of the theorem substyles}
%
% As already pointed out, the theorem substyles, defined below, are
% only loaded when necessary.  Note that all these substyles, except
% \texttt{plain}, have |\slshape| as the default body font.
%
% \subsubsection{The \texttt{plain} style}
%
% \label{dok:th@plain} As the following macros use |@|, we have
% to locally set the |\catcode| of this symbol to ``letter''.
% This happens within a group, so that we do not have to worry about
% which |\catcode| that symbol had before.
%    \begin{macrocode}
%<*thp>
\begingroup \makeatletter
%    \end{macrocode}
% Since we are now within a group, we must make all definitions
% globally.  First we make sure that \texttt{theorem.sty} is loaded.
% This will allow us to use this file as a document style option
% without having to call \texttt{theorem} itself as an option.  At the
% same time, we assure that at least version 2 is loaded, since
% |\theorem@style| was not defined in earlier versions.
%    \begin{macrocode}
\@ifundefined{theorem@style}{% \iffalse meta-comment
%
% Copyright (C) 1993-2024
%
% The LaTeX Project and any individual authors listed elsewhere
% in this file.
%
% This file is part of the Standard LaTeX `Tools Bundle'.
% -------------------------------------------------------
%
% It may be distributed and/or modified under the
% conditions of the LaTeX Project Public License, either version 1.3c
% of this license or (at your option) any later version.
% The latest version of this license is in
%    https://www.latex-project.org/lppl.txt
% and version 1.3c or later is part of all distributions of LaTeX
% version 2005/12/01 or later.
%
% The list of all files belonging to the LaTeX `Tools Bundle' is
% given in the file `manifest.txt'.
%
% \fi
%
%%
%
%
\def\FMithmInfo{2023/07/05 v2.2c Theorem extension package (FMi)}
%
% \ProvidesFile{theorem.dtx}[\FMithmInfo]
%\iffalse   % this is a METACOMMENT !
%%
%% Package `theorem' to use with LaTeX2e
%% Copyright (C) 1989-1995 Frank Mittelbach, all rights reserved.
%<+package>\@ifundefined{theorem@style}{}{\endinput}
%<+package>\NeedsTeXFormat{LaTeX2e}[1995/06/01]
%<+package>\ProvidesPackage{theorem}
%<+package>         [\FMithmInfo]
%
% \fi
%
% \changes{v1.0}{87/05/07}{First draft.}
% \changes{v1.0a}{87/05/07}{`thistheoremstyle' added.}
% \changes{v1.1a}{87/07/29}{New concept: redefine the internal
%                          \protect\LaTeX{} commands `@xnthm' etc.}
% \changes{v1.1b}{87/07/31}{`theoremfont' used to specify font.}
% \changes{v1.2a}{88/01/08}{`@endtheorem' now globally defined.}
% \changes{v1.2b}{88/01/11}{`newtheorem' only allowed in the preamble.}
% \changes{v1.3a}{88/02/14}{`theoremfont' now a toks register.}
%
% \changes{v2.0a}{88/11/16}{New implementation and new documentation.}
% \changes{v2.0a}{88/11/16}{No layout changes but internal modifications
%                           old substyles won't work.}
% \changes{v2.0b}{89/01/30}{`global' used for all definitions.}
% \changes{v2.0c}{89/06/15}{Documentation translated to English.
%                            Thanks to LLL and Nina}
% \changes{v2.1a}{90/02/15}{Support for new font selection.}
% \changes{v2.2a}{94/02/03}{Cleanup for LaTeX2e}
%
% \DoNotIndex{\@endparenv,\@ifdefinable,\@ifnextchar,\@ifundefined}
% \DoNotIndex{\@namedef,\@nameuse,\@spaces,\@tempa,\@warning}
% \DoNotIndex{\ }
% \DoNotIndex{\begingroup,\bfseries}
% \DoNotIndex{\csname}
% \DoNotIndex{\def,\do,\docdate}
% \DoNotIndex{\endcsname,\endgroup,\endinput,\endtrivlist,\expandafter}
% \DoNotIndex{\filedate,\fileversion}
% \DoNotIndex{\gdef,\global}
% \DoNotIndex{\hbox,\hskip}
% \DoNotIndex{\ignorespaces,\input,\itshape,\item}
% \DoNotIndex{\labelsep,\let,\llap}
% \DoNotIndex{\makeatletter}
% \DoNotIndex{\noexpand}
% \DoNotIndex{\refstepcounter,\relax,\rlap}
% \DoNotIndex{\setlength,\slshape,\space,\string,\strut}
% \DoNotIndex{\the,\trivlist,\typeout}
% \DoNotIndex{\vbox}
% \DoNotIndex{\xdef}
%
% \def\deflabel#1{\ttfamily #1\hfill}
% \def\deflist#1{\begin{list}{}{\settowidth\labelwidth{\ttfamily #1}%
%                               \setlength\leftmargin\labelwidth
%                               \addtolength\leftmargin\labelsep
%                               \let\makelabel\deflabel}}
% \def\enddeflist{\end{list}}
% \def\envfont{\normalfont\ttfamily}
%
% \MakeShortVerb{\|}
%
% \newcommand{\TUB}{{\sl TUGboat\/}}
%
% \GetFileInfo{theorem.dtx}
% \title{An Extension of the
%        \LaTeX{} theorem environment\thanks{This file
%        has version number \fileversion, last
%        revised \filedate.}}
% \author{Frank Mittelbach}
%
% \MaintainedByLaTeXTeam{tools}
% \maketitle
%
% \begin{abstract}
%    The macros described in this paper yield an extension of the
%    \LaTeX{} theorem mechanism. It is designed to satisfy the
%    different requirements of various journals. Thus, the
%    layout of the ``theorems'' can be manipulated by determining a
%    ``style''. This article describes not only the use, but
%    also the definition, of the necessary macros.
% \end{abstract}
%
% \section*{Preface to version 2.2}
%
% For \LaTeXe{} this package did not need any fundamental changes. I
% only modified the messages generated so that theorem layout styles
% will show up with the |\listfiles| command and cleaned the section
% on the New Font Selection Scheme since this is now included in
% \LaTeX.
%
% \section*{Preface to version 2.1}
%
% This version is identical to 2.0g described in \TUB\ 10\#3 except
% for some internal defaults which are now set depending on the used
% font selection scheme.
%
% This was done to avoid unpleasant surprises if the new font selection
% scheme is in force. For further details see
% section~\ref{sec:newfonts} and \cite{bk:GMS94}.
%
% \section{Introduction}
%
% For our purposes here, ``theorems'' are labelled enunciations,
% often set off from the main text by extra space and a font change.
% Theorems, corollaries, conjectures, definitions, and remarks
% are all instances of ``theorems''.  The ``header'' of these structures
% is composed of a label (such as \textsc{Theorem} or \textsc{Remark})
% and a number which serializes an item in the sequence of items
% with the same label.
%
% Shortly after the introduction of \LaTeX{} at the Fachbereich
% Mathematik in Mainz, the desire to manipulate the layout of
% ``theorems'' arose.  In Mainz, the following two conventions came
% into general use:
% \begin{enumerate}
%   \item The number of the theorem is shown in the margin.
%   \item There is a line break at
%         the end of the theorem header.
% \end{enumerate}
%
% Additionally, some journals require different formats which depend on
% the ``sort of theorem'': e.g.\ often remarks and definitions are set
% in |\upshape|, while |\itshape| is employed for main theorems.
%
% Confronted with these requirements, a theorem environment was
% developed in Mainz which allows separate determination of the
% layout of the ``theorems sets'', comparable to |\pagestyle|.
%
%
% \section{The user interface}
% \subsection{Defining new theorem sets}
%
% \DescribeMacro\newtheorem
% As in the original \LaTeX{} version, the command |\newtheorem|
% defines a new ``theorem set'' or ``theorem-like structure''.
% Two required arguments name the new environment and give the text
% to be typeset with each instance of the new ``set'', while
% an optional argument determines how
% the ``set'' is enumerated:
% \begin{description}
%    \item[\ttfamily \bslash newtheorem\{foo\}\{bar\}]
%       The theorem set {\envfont foo} (whose name is \texttt{bar})
%       uses its own counter.
%    \item[\ttfamily \bslash newtheorem\{foo2\}{[foo]}\{bar2\}]
%       The theorem set {\envfont foo2} (printed name \texttt{bar2})
%       uses the same counter as the theorem set \texttt{foo}.
%    \item[\ttfamily \bslash newtheorem\{foo3\}\{bar3\}{[section]}]
%       The theorem set {\envfont foo3} (printed name \texttt{bar3}) is
%       enumerated within the counter \texttt{section}, i.e.\ with every
%       new |\section| the enumeration begins again with 1, and
%       the enumeration is composed from the section-number and the
%       theorem counter itself.
% \end{description}
%
% \DescribeMacro\theoremstyle
% Additionally, the command |\theoremstyle| can define the layout
% of various, or all, theorem sets.  It should be noted that any
% theorem set defined by |\newtheorem| is typeset in the
% |\theoremstyle| that is current at the time of the definition.
% Thus, the following
% \begin{verbatim}
%    \theoremstyle{break}        \newtheorem{Cor}{Corollary}
%    \theoremstyle{plain}        \newtheorem{Exa}{Example}[section]
%\end{verbatim}                             ^^A No blank at line start
% leads to the result that the set {\envfont Cor} is formatted in the
% style \texttt{break}, while the set {\envfont Exa} and all the
% following ones are formatted in the style \texttt{plain}, unless
% another |\theoremstyle| follows.  Since the definitions installed by
% |\newtheorem| are global, one also can limit |\theoremstyle| locally
% by grouping braces.
%
% \DescribeMacro\theorembodyfont
% The choice of the font for the theorem body is completely
% independent of the chosen |\theoremstyle|; this has proven to
% be very advantageous.  For example,
% \begin{verbatim}
%   {\theorembodyfont{\upshape}          \newtheorem{Rem}{Remark}}
%\end{verbatim}
% defines a theorem set {\envfont Rem}, which will be set in
% |\upshape| in the current layout (which in our example is
% \texttt{plain}).  As with |\theoremstyle|, the |\theorembodyfont|
% chosen is that current at the time of |\newtheorem|.  If
% |\theorembodyfont| is not specified or one defines
% |\theorembodyfont{}|, then the font used will be that defined by the
% |\theoremstyle|.
%
% \DescribeMacro\theoremheaderfont It is also possible to customize
% the font used for the theorem headers.  This is, however, a global
% declaration, and therefore there should be at most one
% |\theoremheaderfont| declaration in the preamble.\footnote{If it is
% actually necessary to have different header fonts, one has to define
% new theorem styles (substituting the desired font) or specify the
% information directly in the \texttt{\bslash newtheorem} declaration
% (the unclean variant).}
%
% \DescribeMacro\theorempreskipamount
% \DescribeMacro\theorempostskipamount Two additional parameters
% affect the vertical space around the theorem environments:
% |\theorempreskipamount| and |\theorempostskipamount| define,
% respectively, the spacing before and after such an environment.
% These parameters apply for all theorem sets and can be manipulated
% with the ordinary length macros.  They are rubber lengths
% (`\textsf{skips}'), and therefore can contain \texttt{plus} and
% \texttt{minus} parts.
%
% Since the definition of theorem sets should---most sensibly---be
% placed in the preamble, we only allow installation there.
% It is therefore possible to release the memory used here after
% |\begin{document}|, in order to make room for other
% applications.
%
%
% \subsection{Existing theorem styles}
%
% The following theorem styles exist to date:
% \begin{deflist}{marginbreak}
%    \item[plain]
%       This theorem style emulates the original \LaTeX{} definition,
%       except that additionally the parameters
%       |\theorem...skipamount| are used.
%    \item[break]
%       In this style, the theorem header is followed by a line
%       break.
%    \item[marginbreak]
%       The theorem number is set in the margin, and there is a line
%       break as in \texttt{break}.
%    \item[changebreak]
%       Like \texttt{break}, but with header number and text
%       interchanged.
%    \item[change]
%       Header number and text are interchanged, without a line break.
%    \item[margin]
%       The number is set in the left margin, without a line break.
% \end{deflist}
% All styles (except \texttt{plain}) select |\slshape| as the default
% |\theorembodyfont|.
%
%
% \subsection{Examples}
%
% Given the above theorem sets {\envfont Cor}, {\envfont Exa}
% and {\envfont Rem},
% suppose that the preamble also contains the declarations:
% \begin{verbatim}
%    \theoremstyle{marginbreak}   \newtheorem{Lem}[Cor]{Lemma}
%    \theoremstyle{change}
%    \theorembodyfont{\itshape}        \newtheorem{Def}[Cor]{Definition}
%
%    \theoremheaderfont{\scshape}
%\end{verbatim}
% Then the following are some typical examples of the typeset output
% resulting from their use.
%
% \begin{Cor}
%    This is a sentence typeset in the theorem environment {\envfont
%    Cor}.
% \end{Cor}
%
% \begin{Exa}
%    This is a sentence typeset in the theorem environment {\envfont
%    Exa}.
% \end{Exa}
%
% \begin{Rem}
%    This is a sentence typeset in the theorem environment {\envfont
%    Rem}.
% \end{Rem}
%
% \begin{Lem}[Ben User]
%    This is a sentence typeset in the theorem environment {\envfont
%    Lem}.
% \end{Lem}
%
% \begin{Def}[Very impressive Definition]
%    This is a sentence typeset in the theorem environment {\envfont
%    Def}.
% \end{Def}
%
% The last two examples show the effect of the optional argument to
% a theorem environment (it is the text typeset in parentheses).
%
%
% \section{Special Considerations}
% \label{sec:newfonts}
%
% Theoremheader and body are implemented as a unit. This means that
% the |\theoremheaderfont|  will inherit characteristics of the
% |\theorembodyfont| in \LaTeXe. Thus, if for example
% |\theorembodyfont| is |\itshape| and |\theoremheaderfont|
% is |\bfseries| the font selected for the header will have the
% characteristics `bold extended italic'. If this is not desired one
% should set the |\theoremheaderfont| to something like
% \begin{verbatim}
%      \theoremheaderfont{\normalfont\bfseries}
%\end{verbatim}
% i.e.\ supplying all necessary font information explicitly.
%
%
%
% \section{Acknowledgements}
%
% The publication of this set of macros was only possible with the
% help of Christina Busse (translating the manuscript into English),
% Joachim Pense (playing the r\^ole of typist), Chris Rowley
% (looking everything over) and many others providing useful
% suggestions.
%
% \MaybeStop{%
%    \begin{thebibliography}{9}
%       \bibitem{bk:GMS94} \textsc{M.~Goossens}, \textsc{F.~Mittelbach}
%         and \textsc{A.~Samarin}.
%         \newblock The \LaTeX{} Companion.
%         \newblock
%         Addison-Wesley, Reading, Massachusetts, 1994.
%       \bibitem{latex:source} \textsc{Lamport, Leslie.} \newblock
%                               \texttt{latex.tex}, version 2.09, date
%                               Feb. 1990.
%    \end{thebibliography}
%    \PrintIndex
%    \PrintChanges}
%
%
% \section{The documentation driver file}
%
% The next bit of code contains the documentation driver file for
% \TeX{}, i.e., the file that will produce the documentation you are
% currently reading. It will be extracted from this file by the
% \texttt{docstrip} program. Since it is the first code in the file
% one can alternatively process this file directly with \LaTeXe{} to
% obtain the documentation.
%
%    \begin{macrocode}
%<*driver>
\documentclass{ltxdoc}

\usepackage{theorem}

%<+driver>% The next few lines define theorem sets which are used
%<+driver>% in the example section of the documentation.

\theoremstyle{break}       \newtheorem{Cor}{Corollary}
\theoremstyle{plain}       \newtheorem{Exa}{Example}[section]
{\theorembodyfont{\upshape}\newtheorem{Rem}{Remark}}
\theoremstyle{marginbreak} \newtheorem{Lem}[Cor]{Lemma}
\theoremstyle{change}
\theorembodyfont{\itshape} \newtheorem{Def}[Cor]{Definition}

\theoremheaderfont{\scshape}

\RecordChanges

\begin{document}
   \DocInput{theorem.dtx}
\end{document}
%</driver>
%    \end{macrocode}
%
% \section{Definition of the Macros}
%
% If the file has been loaded before, we abort immediately. If not the
% package announces itself (this is actually done at the very top of
% the file---the way it is done isn't good style so don't copy it).
% \changes{v2.0e}{89/07/19}{Spaces removed from `typeout'.}
% \changes{v2.2b}{95/11/19}{Announce moved to top of file}
%    \begin{macrocode}
%<*package>
%\@ifundefined{theorem@style}{}{\endinput}
%\def\FMithmInfo{1995/11/19 v2.2b Theorem extension package (FMi)}
%\ProvidesPackage{theorem}[\FMithmInfo]
%    \end{macrocode}
%
%
% \subsection{Definition of theorem styles and fonts}
%
% All the definitions in this file are done globally to allow
% inputting this file inside a group.
%
% \begin{macro}{\theoremstyle}
%    Before a theorem style can be installed, the chosen style must be
%    known. For that reason, we must test to see that
%    |\th@|\meta{style} is known or, more precisely, that it is
%    different from |\relax|. If the style is not known then
%    |\th@plain| is used.
%    \begin{macrocode}
\gdef\theoremstyle#1{%
   \@ifundefined{th@#1}{\@warning
          {Unknown theoremstyle `#1'. Using `plain'}%
          \theorem@style{plain}}%
%    \end{macrocode}
%    We save the theorem style to be used in the token register
%    |\theorem@style|.
%    \begin{macrocode}
      {\theorem@style{#1}}%
%    \end{macrocode}
%    Now we ``evaluate'' the theorem style: this means, we call the
%    macro |\th@|\meta{style} which will activate the relevant
%    definitions which are contained in a separate file.  This is done
%    in a group to suppress changes to the current font.  This could
%    otherwise pose problems together with the new font selection
%    scheme\footnote{When I printed the original article using the new
%    font selection scheme I ended with a document with slanted
%    typefaces (text headings and all) simply because one of the
%    theorem styles used \texttt{\bslash sl} at toplevel.} if the
%    |\th@|\meta{style} is evaluated a second time.
% \changes{v2.1a}{90/02/15}{Group added around `th@..}
% \changes{v2.1a}{90/02/15}{Reseting of `..@begintheorem removed.}
%    \begin{macrocode}
      \begingroup
        \csname th@\the\theorem@style \endcsname
      \endgroup}
%    \end{macrocode}
% \end{macro}
%
%
% \begin{macro}{\@begintheorem}
% \changes{v2.1a}{90/02/15}{Relaxed at toplevel.}
% \begin{macro}{\@opargbegintheorem}
% \changes{v2.1a}{90/02/15}{Relaxed at toplevel.}
%    We reset |\@begintheorem| and |\@opargbegintheorem| to
%    |\relax| since these commands are no longer necessary at
%    toplevel. This will save a few tokens.
%    \begin{macrocode}
\global\let\@begintheorem\relax
\global\let\@opargbegintheorem\relax
%    \end{macrocode}
% \end{macro}
% \end{macro}
%
% \begin{macro}{\theorem@style}
%    Obviously the \textsf{token} register used above has to be
%    allocated. To assure the utmost compatibility with the
%    original \LaTeX{} definition, we set the default theorem style to
%    \texttt{plain}, which implements the usual \LaTeX{} convention.
% \changes{v2.0f}{89/07/20}{Unnecessary `global' removed.}
%    \begin{macrocode}
\newtoks\theorem@style
\global\theorem@style{plain}
%    \end{macrocode}
% \end{macro}
%
% \begin{macro}{\theorembodyfont}
% \begin{macro}{\theorem@bodyfont}
%    If the |\theorembodyfont| is set by the user then it should not
%    interact with the default font set in the theorem style. When the
%    new font selection is in force this may happen if, for example,
%    the default is |\itshape| and the new |\theorembodyfont| is
%    |\sffamily|. So we add a |\reset@font| command in front of the
%    user definition.
% \changes{v2.0f}{89/07/20}{Unnecessary `global' removed.}
% \changes{v2.0g}{89/09/19}{`theoremfont to `theorembodyfont.}
% \changes{v2.1c}{92/06/22}{`reset@font added and logic changed.}
% \changes{v2.1d}{92/10/28}{Reset to empty if arg is empty}
%    \begin{macrocode}
\gdef\theorembodyfont#1{%
%    \end{macrocode}
%    We check if the argument supplied is empty and if so put nothing
%    into the |\theorem@bodyfont| token register to allow for
%    |\theorembodyfont{}| as a mean of using the default of the
%    current |\theoremstyle|.
%    \begin{macrocode}
   \def\@tempa{#1}%
   \ifx\@tempa\@empty
    \theorem@bodyfont{}%
   \else
    \theorem@bodyfont{\reset@font#1}%
   \fi
   }
\newtoks\theorem@bodyfont
\global\theorem@bodyfont{}
%    \end{macrocode}
% \end{macro}
% \end{macro}
%
%
% \begin{macro}{\theoremheaderfont}
% \changes{v2.0g}{89/09/19}{Macro added.}
%    The font for the theorem headers is handled differently because
%    this definition applies to all theorem styles.
%    \begin{macrocode}
\gdef\theoremheaderfont#1{\gdef\theorem@headerfont{#1}%
%    \end{macrocode}
%    After using the macro once it is redefined to produce an error
%    message.
%    \begin{macrocode}
       \gdef\theoremheaderfont##1{%
        \typeout{\string\theoremheaderfont\space should be used
                 only once.}}}
%    \end{macrocode}
% \end{macro}
%
% \begin{macro}{\theorem@headerfont}
% \changes{v2.0g}{89/09/19}{Macro added.}
%    To set the |\theorem@headerfont| default we first test if
%    the new fontselection scheme is in force.
% \changes{v2.1a}{90/02/15}{Support for new font selection.}
%    \begin{macrocode}
\ifx\upshape\undefined
%    \end{macrocode}
%    If not we define it to expand into |\bfseries|. We don't use
%    |\let| just in case a following style option redefines this
%    macro.
%    \begin{macrocode}
\gdef\theorem@headerfont{\bfseries}
%    \end{macrocode}
%    Otherwise we reset the current shape before calling |\bfseries|.
%    \begin{macrocode}
\else \gdef\theorem@headerfont{\normalfont\bfseries}\fi
%    \end{macrocode}
% \end{macro}
%
% \begin{macro}{\th@plain}
% \begin{macro}{\th@break}
% \begin{macro}{\th@marginbreak}
% \begin{macro}{\th@changebreak}
% \begin{macro}{\th@change}
% \begin{macro}{\th@margin}
%    The different styles are defined in macros such as
%    |\th@plain|.  Since memory space is precious in
%    ``non-Big-versions'', we have to avoid offering too many unused
%    definitions.  Therefore we define these styles in separate files
%    that can be loaded on demand.  Thus the commands themselves only
%    load these files. We use |\@input@|, a \LaTeXe{} internal command
%    that ensures that the file will be listed with |\listfiles|.
% \changes{v2.2a}{94/02/02}{Use `@input@ to load theorem layout files}
%    \begin{macrocode}
\gdef\th@plain{\@input@{thp.sty}}
\gdef\th@break{\@input@{thb.sty}}
\gdef\th@marginbreak{\@input@{thmb.sty}}
\gdef\th@changebreak{\@input@{thcb.sty}}
\gdef\th@change{\@input@{thc.sty}}
\gdef\th@margin{\@input@{thm.sty}}
%    \end{macrocode}
%    This list will be expanded when new styles become
%    available.  For testing,  just append new theorem substyles
%    as document options.
% \end{macro}
% \end{macro}
% \end{macro}
% \end{macro}
% \end{macro}
% \end{macro}
%
%
%
% \subsection{Definition of a new theorem set}
%
% As already pointed out, a new theorem environment can be defined in
% three different ways:
% \begin{verbatim}
%    \newtheorem{Lem}{Lemma}
%    \newtheorem{Lem}{Lemma}[section]
%    \newtheorem{Lem}[Theorem]{Lemma}
%\end{verbatim}
% The function of the macro |\newtheorem| is to recognize these cases
% and then to branch into one of the three macros |\@ynthm|, |\@xnthm|
% or |\@othm|.  This mechanism is adopted unchanged from
% \cite{latex:source}; the essential point here is that, for example,
% in the second case, the arguments \texttt{Lem}, \texttt{Lemma} and
% \texttt{section} are passed over to the macro |\@xnthm|.
%
% We inspect this case first because the others present fewer problems,
% and thus are easily derived from this one.
%
% \begin{macro}{\@xnthm}
%    For our example arguments, the macro |\@xnthm| must fulfill
%    the following:
%    \begin{itemize}
%       \item
%          Define a new \LaTeX-counter `Lem'
%       \item
%          reset this counter within a |\section|
%       \item
%          define the macro |\theLem|
%       \item
%          define the environment macros |\Lem| and
%          |\endLem| using the current
%          |\theoremstyle| and |\theorem@bodyfont|.
%    \end{itemize}
%    Obviously, all this should happen only if the first argument
%    of |\@xnthm| (i.e.\ \texttt{Lem} in our example) is chosen so
%    as not to conflict with any previously defined commands or
%    environments.  This test is performed by the \LaTeX{} macro
%    |\@ifdefinable|.
%    \begin{macrocode}
\gdef\@xnthm#1#2[#3]{\expandafter\@ifdefinable\csname #1\endcsname
%    \end{macrocode}
%    Therefore, the first argument of |\@ifdefinable| is the
%    expansion (in the example, |\Lem|) of
%    |\csname||#1\endcsname|.  The second argument is
%    executed only if the test has been completed successfully.
%    \begin{macrocode}
   {%
%    \end{macrocode}
%    Now we define the new counter. The names of the \LaTeX{}
%    macros employed should speak for themselves:
% \changes{v2.2b}{1995/10/07}{Use \cs{@newctr}}
% \changes{v2.2c}{1995/11/23}{Use \cs{@newctr} properly.}
%    \begin{macrocode}
    \@definecounter{#1}\@newctr{#1}[#3]%
%    \end{macrocode}
%    Using |\@newctr| will give a proper error message if the counter
%    in |#3| is not defined.
%    In defining `|\theLem|' we must generate the desired macro
%    name by use of |\expandafter| and |\csname|.
%    \begin{macrocode}
    \expandafter\xdef\csname the#1\endcsname
%    \end{macrocode}
%    An |\xdef| is used in order to make the definition global,
%    and to ensure that it contains the replacement texts of
%    |\@thmcountersep| and |\@thmcounter|.\footnote{These
%    two macros can be defined by the document style.  Their default
%    values produce a `\texttt{.}' as separation and an arabic
%    representation of the number.} However, not everything should be
%    expanded. For example, it saves space to use |\thesection|
%    instead of its---at times---lengthy expansion.
%    \begin{macrocode}
      {\expandafter \noexpand \csname the#3\endcsname
       \@thmcountersep \@thmcounter{#1}}%
%    \end{macrocode}
%    Thus with the defaults of \LaTeX{}, |\theLem| would be
%    replaced by the command sequence |\thesection.\arabic{Lem}|.
%
%    We will now look at the definition of the macro which is executed
%    at the beginning of the actual environment (in our example this
%    macro is |\Lem|). It should be noted that we use an
%    ``|\expandafter| trick'' to expand only certain parts of the
%    replacement text at the time of the definition.
%    \begin{macrocode}
    \def\@tempa{\global\@namedef{#1}}%
    \expandafter \@tempa \expandafter{%
%    \end{macrocode}
%    First, the macro that contains the current definitions of
%    |\@begintheorem| and |\@opargtheorem| should be called
%    up.  The name of this macro---as is already known---has the form
%    |\th@|\meta{theorem\ style}; therefore, it must be called by
%    \begin{macrocode}
      \csname th@\the \theorem@style
%    \end{macrocode}
%    In addition the default theorem font should be changeable, i.e.\
%    we have to insert the contents of |\theorem@bodyfont|. For
%    that reason, we expand even further, beyond |\endcsname|,
%    and thus insert the contents of the \textsf{token} register
%    |\theorem@bodyfont| in the replacement text.
%    \begin{macrocode}
            \expandafter \endcsname \the \theorem@bodyfont
%    \end{macrocode}
%    Now it is time to call the macro |\@thm| which takes over
%    the further processing.  It has two arguments: the current
%    counter name (in our example, \texttt{Lem}), and the text of the
%    label (in our example, \texttt{Lemma}).
%    \begin{macrocode}
     \@thm{#1}{#2}}%
%    \end{macrocode}
%    With this, the `sub-definition' is complete.  The macro
%    |\@endtheorem| ends a theorem environment and is, so far, nothing
%    but an |\endtrivlist|.  (Hence it is defined globally, and not
%    within the theorem styles.\footnote{This has to be changed as
%    soon as theorem styles that change \texttt{\bslash @endtheorem}
%    exist.  In such a case, all existing styles must be changed as
%    well since they will have to reset the macro.}) Therefore, we can
%    set it equivalent to the macro that ends the theorem set (in our
%    example, |\endLem|).  However, if some day theorem styles exist
%    that do change |\@endtheorem|, we would have to use the
%    commented-out line instead.
%    \begin{macrocode}
    \global \expandafter \let \csname end#1\endcsname \@endtheorem
%   \global\@namedef{end#1}{\@endtheorem}%
%    \end{macrocode}
%    With these commands all the required definitions are employed,
%    unless the test |\@ifdefinable| has failed.  Therefore, we
%    end the second argument of this macro and with it the definition
%    of |\@xnthm|.
%    \begin{macrocode}
   }}
%    \end{macrocode}
% \end{macro}
%
% \begin{macro}{\@ynthm}
%    The definition of |\@ynthm| is completely analogous. In
%    this case the new counter that is defined is not reset within
%    another counter; thus the definition of |\the...|
%    is simplified:
%    \begin{macrocode}
\gdef\@ynthm#1#2{\expandafter\@ifdefinable\csname #1\endcsname
   {\@definecounter{#1}%
    \expandafter\xdef\csname the#1\endcsname{\@thmcounter{#1}}%
%    \end{macrocode}
%    The rest of the definition corresponds literally to that of
%    |\@xnthm|:
%    \begin{macrocode}
    \def\@tempa{\global\@namedef{#1}}\expandafter \@tempa
     \expandafter{\csname th@\the \theorem@style \expandafter
     \endcsname \the\theorem@bodyfont \@thm{#1}{#2}}%
    \global \expandafter \let \csname end#1\endcsname \@endtheorem}}
%    \end{macrocode}
% \end{macro}
%
% \begin{macro}{\@othm}
%    The definition of |\@othm| does not contain anything new.
%    \begin{macrocode}
\gdef\@othm#1[#2]#3{%
%    \end{macrocode}
% \changes{v2.2b}{95/10/07}{Build in check for counter defined}
%    We do not define a new counter but instead use one that has
%    already been defined.  Thus the only definition we need is that
%    of this pseudo-counter (i.e.\ |\the|\meta{env.\ name}).
%    First we check if |#2| corresponds to a known counter name.
%    \begin{macrocode}
  \expandafter\ifx\csname c@#2\endcsname\relax
   \@nocounterr{#2}%
  \else
   \expandafter\@ifdefinable\csname #1\endcsname
   {\expandafter \xdef \csname the#1\endcsname
     {\expandafter \noexpand \csname the#2\endcsname}%
%    \end{macrocode}
%    All other parts of the definition can be adopted from
%    |\@xnthm|. We have to remember, though, that in this case
%    the name of the current counter and the theorem label have moved
%    to the second and third arguments.
%    \begin{macrocode}
    \def\@tempa{\global\@namedef{#1}}\expandafter \@tempa
     \expandafter{\csname th@\the \theorem@style \expandafter
     \endcsname \the\theorem@bodyfont \@thm{#2}{#3}}%
    \global \expandafter \let \csname end#1\endcsname \@endtheorem}%
  \fi}
%    \end{macrocode}
% \end{macro}
%
% \subsection{Macros that are employed in a {\envfont theorem}
%    environment}
%
% \begin{macro}{\@thm}
%    The macro |\@thm| has to increase the current counter.  Then,
%    depending on whether the environment has (or does not have) an
%    optional argument, it has to branch into either |\@begintheorem|
%    or |\@opargtheorem|.
%    \begin{macrocode}
\gdef\@thm#1#2{\refstepcounter{#1}%
%    \end{macrocode}
%    Now we start a {\envfont trivlist} environment, and give
%    |\@topsep| and |\@topsepadd| the values of the \textsf{skip}
%    registers |\theorempreskipamount| and |\theorempostskipamount|.
%    The value in |\@topsep| is the vertical space that is inserted by
%    the first (and only) |\item| in our |\trivlist| whilst
%    |\@topsepadd| is inserted by |\@endparenv| at the end of that
%    {\envfont trivlist} environment.  By using these registers, we
%    obtain the desired space around a {\envfont theorem} environment.
%    \begin{macrocode}
   \trivlist
   \@topsep \theorempreskipamount               % used by first \item
   \@topsepadd \theorempostskipamount           % used by \@endparenv
%    \end{macrocode}
%    Now we have to test whether an optional argument has been given.
% \changes{v2.0d}{89/07/18}{Forgotten percent added.}
%    \begin{macrocode}
   \@ifnextchar [%
%    \end{macrocode}
%    If there is an optional argument, we will call |\@ythm|, and
%    move the arguments read back into the input stream.
%    \begin{macrocode}
   {\@ythm{#1}{#2}}%
%    \end{macrocode}
%    If not, we call |\@begintheorem|. Its first argument is the
%    name of the theorem set (hence the second argument of
%    |\@thm|). Its second argument is the macro that produces the
%    current number.
%    \begin{macrocode}
   {\@begintheorem{#2}{\csname the#1\endcsname}\ignorespaces}}
%    \end{macrocode}
% \end{macro}
%
% \begin{macro}{\@xthm}
% \begin{macro}{\@ythm}
%    Both these macros were originally called by |\@thm|.  We do
%    not need |\@xthm| anymore, hence we reset it to
%    |\relax|. The definition of |\@ythm| has not changed at
%    all from its definition in \LaTeX. In order to make the macros
%    easier to understand, we will
%    nevertheless present it (commented out).
%    \begin{macrocode}
\global\let\@xthm\relax
% \def\@ythm#1#2[#3]{\@opargbegintheorem{#2}{\csname
%           the#1\endcsname}{#3}\ignorespaces}
%</package>
%    \end{macrocode}
%    The primitive |\ignorespaces| in |\@ythm| and
%    |\@thm| is needed to remove the spaces between the
%    |\begin{...}| and the actual text.
% \end{macro}
% \end{macro}
%
%
% \subsection{Definition of the theorem substyles}
%
% As already pointed out, the theorem substyles, defined below, are
% only loaded when necessary.  Note that all these substyles, except
% \texttt{plain}, have |\slshape| as the default body font.
%
% \subsubsection{The \texttt{plain} style}
%
% \label{dok:th@plain} As the following macros use |@|, we have
% to locally set the |\catcode| of this symbol to ``letter''.
% This happens within a group, so that we do not have to worry about
% which |\catcode| that symbol had before.
%    \begin{macrocode}
%<*thp>
\begingroup \makeatletter
%    \end{macrocode}
% Since we are now within a group, we must make all definitions
% globally.  First we make sure that \texttt{theorem.sty} is loaded.
% This will allow us to use this file as a document style option
% without having to call \texttt{theorem} itself as an option.  At the
% same time, we assure that at least version 2 is loaded, since
% |\theorem@style| was not defined in earlier versions.
%    \begin{macrocode}
\@ifundefined{theorem@style}{% \iffalse meta-comment
%
% Copyright (C) 1993-2024
%
% The LaTeX Project and any individual authors listed elsewhere
% in this file.
%
% This file is part of the Standard LaTeX `Tools Bundle'.
% -------------------------------------------------------
%
% It may be distributed and/or modified under the
% conditions of the LaTeX Project Public License, either version 1.3c
% of this license or (at your option) any later version.
% The latest version of this license is in
%    https://www.latex-project.org/lppl.txt
% and version 1.3c or later is part of all distributions of LaTeX
% version 2005/12/01 or later.
%
% The list of all files belonging to the LaTeX `Tools Bundle' is
% given in the file `manifest.txt'.
%
% \fi
%
%%
%
%
\def\FMithmInfo{2023/07/05 v2.2c Theorem extension package (FMi)}
%
% \ProvidesFile{theorem.dtx}[\FMithmInfo]
%\iffalse   % this is a METACOMMENT !
%%
%% Package `theorem' to use with LaTeX2e
%% Copyright (C) 1989-1995 Frank Mittelbach, all rights reserved.
%<+package>\@ifundefined{theorem@style}{}{\endinput}
%<+package>\NeedsTeXFormat{LaTeX2e}[1995/06/01]
%<+package>\ProvidesPackage{theorem}
%<+package>         [\FMithmInfo]
%
% \fi
%
% \changes{v1.0}{87/05/07}{First draft.}
% \changes{v1.0a}{87/05/07}{`thistheoremstyle' added.}
% \changes{v1.1a}{87/07/29}{New concept: redefine the internal
%                          \protect\LaTeX{} commands `@xnthm' etc.}
% \changes{v1.1b}{87/07/31}{`theoremfont' used to specify font.}
% \changes{v1.2a}{88/01/08}{`@endtheorem' now globally defined.}
% \changes{v1.2b}{88/01/11}{`newtheorem' only allowed in the preamble.}
% \changes{v1.3a}{88/02/14}{`theoremfont' now a toks register.}
%
% \changes{v2.0a}{88/11/16}{New implementation and new documentation.}
% \changes{v2.0a}{88/11/16}{No layout changes but internal modifications
%                           old substyles won't work.}
% \changes{v2.0b}{89/01/30}{`global' used for all definitions.}
% \changes{v2.0c}{89/06/15}{Documentation translated to English.
%                            Thanks to LLL and Nina}
% \changes{v2.1a}{90/02/15}{Support for new font selection.}
% \changes{v2.2a}{94/02/03}{Cleanup for LaTeX2e}
%
% \DoNotIndex{\@endparenv,\@ifdefinable,\@ifnextchar,\@ifundefined}
% \DoNotIndex{\@namedef,\@nameuse,\@spaces,\@tempa,\@warning}
% \DoNotIndex{\ }
% \DoNotIndex{\begingroup,\bfseries}
% \DoNotIndex{\csname}
% \DoNotIndex{\def,\do,\docdate}
% \DoNotIndex{\endcsname,\endgroup,\endinput,\endtrivlist,\expandafter}
% \DoNotIndex{\filedate,\fileversion}
% \DoNotIndex{\gdef,\global}
% \DoNotIndex{\hbox,\hskip}
% \DoNotIndex{\ignorespaces,\input,\itshape,\item}
% \DoNotIndex{\labelsep,\let,\llap}
% \DoNotIndex{\makeatletter}
% \DoNotIndex{\noexpand}
% \DoNotIndex{\refstepcounter,\relax,\rlap}
% \DoNotIndex{\setlength,\slshape,\space,\string,\strut}
% \DoNotIndex{\the,\trivlist,\typeout}
% \DoNotIndex{\vbox}
% \DoNotIndex{\xdef}
%
% \def\deflabel#1{\ttfamily #1\hfill}
% \def\deflist#1{\begin{list}{}{\settowidth\labelwidth{\ttfamily #1}%
%                               \setlength\leftmargin\labelwidth
%                               \addtolength\leftmargin\labelsep
%                               \let\makelabel\deflabel}}
% \def\enddeflist{\end{list}}
% \def\envfont{\normalfont\ttfamily}
%
% \MakeShortVerb{\|}
%
% \newcommand{\TUB}{{\sl TUGboat\/}}
%
% \GetFileInfo{theorem.dtx}
% \title{An Extension of the
%        \LaTeX{} theorem environment\thanks{This file
%        has version number \fileversion, last
%        revised \filedate.}}
% \author{Frank Mittelbach}
%
% \MaintainedByLaTeXTeam{tools}
% \maketitle
%
% \begin{abstract}
%    The macros described in this paper yield an extension of the
%    \LaTeX{} theorem mechanism. It is designed to satisfy the
%    different requirements of various journals. Thus, the
%    layout of the ``theorems'' can be manipulated by determining a
%    ``style''. This article describes not only the use, but
%    also the definition, of the necessary macros.
% \end{abstract}
%
% \section*{Preface to version 2.2}
%
% For \LaTeXe{} this package did not need any fundamental changes. I
% only modified the messages generated so that theorem layout styles
% will show up with the |\listfiles| command and cleaned the section
% on the New Font Selection Scheme since this is now included in
% \LaTeX.
%
% \section*{Preface to version 2.1}
%
% This version is identical to 2.0g described in \TUB\ 10\#3 except
% for some internal defaults which are now set depending on the used
% font selection scheme.
%
% This was done to avoid unpleasant surprises if the new font selection
% scheme is in force. For further details see
% section~\ref{sec:newfonts} and \cite{bk:GMS94}.
%
% \section{Introduction}
%
% For our purposes here, ``theorems'' are labelled enunciations,
% often set off from the main text by extra space and a font change.
% Theorems, corollaries, conjectures, definitions, and remarks
% are all instances of ``theorems''.  The ``header'' of these structures
% is composed of a label (such as \textsc{Theorem} or \textsc{Remark})
% and a number which serializes an item in the sequence of items
% with the same label.
%
% Shortly after the introduction of \LaTeX{} at the Fachbereich
% Mathematik in Mainz, the desire to manipulate the layout of
% ``theorems'' arose.  In Mainz, the following two conventions came
% into general use:
% \begin{enumerate}
%   \item The number of the theorem is shown in the margin.
%   \item There is a line break at
%         the end of the theorem header.
% \end{enumerate}
%
% Additionally, some journals require different formats which depend on
% the ``sort of theorem'': e.g.\ often remarks and definitions are set
% in |\upshape|, while |\itshape| is employed for main theorems.
%
% Confronted with these requirements, a theorem environment was
% developed in Mainz which allows separate determination of the
% layout of the ``theorems sets'', comparable to |\pagestyle|.
%
%
% \section{The user interface}
% \subsection{Defining new theorem sets}
%
% \DescribeMacro\newtheorem
% As in the original \LaTeX{} version, the command |\newtheorem|
% defines a new ``theorem set'' or ``theorem-like structure''.
% Two required arguments name the new environment and give the text
% to be typeset with each instance of the new ``set'', while
% an optional argument determines how
% the ``set'' is enumerated:
% \begin{description}
%    \item[\ttfamily \bslash newtheorem\{foo\}\{bar\}]
%       The theorem set {\envfont foo} (whose name is \texttt{bar})
%       uses its own counter.
%    \item[\ttfamily \bslash newtheorem\{foo2\}{[foo]}\{bar2\}]
%       The theorem set {\envfont foo2} (printed name \texttt{bar2})
%       uses the same counter as the theorem set \texttt{foo}.
%    \item[\ttfamily \bslash newtheorem\{foo3\}\{bar3\}{[section]}]
%       The theorem set {\envfont foo3} (printed name \texttt{bar3}) is
%       enumerated within the counter \texttt{section}, i.e.\ with every
%       new |\section| the enumeration begins again with 1, and
%       the enumeration is composed from the section-number and the
%       theorem counter itself.
% \end{description}
%
% \DescribeMacro\theoremstyle
% Additionally, the command |\theoremstyle| can define the layout
% of various, or all, theorem sets.  It should be noted that any
% theorem set defined by |\newtheorem| is typeset in the
% |\theoremstyle| that is current at the time of the definition.
% Thus, the following
% \begin{verbatim}
%    \theoremstyle{break}        \newtheorem{Cor}{Corollary}
%    \theoremstyle{plain}        \newtheorem{Exa}{Example}[section]
%\end{verbatim}                             ^^A No blank at line start
% leads to the result that the set {\envfont Cor} is formatted in the
% style \texttt{break}, while the set {\envfont Exa} and all the
% following ones are formatted in the style \texttt{plain}, unless
% another |\theoremstyle| follows.  Since the definitions installed by
% |\newtheorem| are global, one also can limit |\theoremstyle| locally
% by grouping braces.
%
% \DescribeMacro\theorembodyfont
% The choice of the font for the theorem body is completely
% independent of the chosen |\theoremstyle|; this has proven to
% be very advantageous.  For example,
% \begin{verbatim}
%   {\theorembodyfont{\upshape}          \newtheorem{Rem}{Remark}}
%\end{verbatim}
% defines a theorem set {\envfont Rem}, which will be set in
% |\upshape| in the current layout (which in our example is
% \texttt{plain}).  As with |\theoremstyle|, the |\theorembodyfont|
% chosen is that current at the time of |\newtheorem|.  If
% |\theorembodyfont| is not specified or one defines
% |\theorembodyfont{}|, then the font used will be that defined by the
% |\theoremstyle|.
%
% \DescribeMacro\theoremheaderfont It is also possible to customize
% the font used for the theorem headers.  This is, however, a global
% declaration, and therefore there should be at most one
% |\theoremheaderfont| declaration in the preamble.\footnote{If it is
% actually necessary to have different header fonts, one has to define
% new theorem styles (substituting the desired font) or specify the
% information directly in the \texttt{\bslash newtheorem} declaration
% (the unclean variant).}
%
% \DescribeMacro\theorempreskipamount
% \DescribeMacro\theorempostskipamount Two additional parameters
% affect the vertical space around the theorem environments:
% |\theorempreskipamount| and |\theorempostskipamount| define,
% respectively, the spacing before and after such an environment.
% These parameters apply for all theorem sets and can be manipulated
% with the ordinary length macros.  They are rubber lengths
% (`\textsf{skips}'), and therefore can contain \texttt{plus} and
% \texttt{minus} parts.
%
% Since the definition of theorem sets should---most sensibly---be
% placed in the preamble, we only allow installation there.
% It is therefore possible to release the memory used here after
% |\begin{document}|, in order to make room for other
% applications.
%
%
% \subsection{Existing theorem styles}
%
% The following theorem styles exist to date:
% \begin{deflist}{marginbreak}
%    \item[plain]
%       This theorem style emulates the original \LaTeX{} definition,
%       except that additionally the parameters
%       |\theorem...skipamount| are used.
%    \item[break]
%       In this style, the theorem header is followed by a line
%       break.
%    \item[marginbreak]
%       The theorem number is set in the margin, and there is a line
%       break as in \texttt{break}.
%    \item[changebreak]
%       Like \texttt{break}, but with header number and text
%       interchanged.
%    \item[change]
%       Header number and text are interchanged, without a line break.
%    \item[margin]
%       The number is set in the left margin, without a line break.
% \end{deflist}
% All styles (except \texttt{plain}) select |\slshape| as the default
% |\theorembodyfont|.
%
%
% \subsection{Examples}
%
% Given the above theorem sets {\envfont Cor}, {\envfont Exa}
% and {\envfont Rem},
% suppose that the preamble also contains the declarations:
% \begin{verbatim}
%    \theoremstyle{marginbreak}   \newtheorem{Lem}[Cor]{Lemma}
%    \theoremstyle{change}
%    \theorembodyfont{\itshape}        \newtheorem{Def}[Cor]{Definition}
%
%    \theoremheaderfont{\scshape}
%\end{verbatim}
% Then the following are some typical examples of the typeset output
% resulting from their use.
%
% \begin{Cor}
%    This is a sentence typeset in the theorem environment {\envfont
%    Cor}.
% \end{Cor}
%
% \begin{Exa}
%    This is a sentence typeset in the theorem environment {\envfont
%    Exa}.
% \end{Exa}
%
% \begin{Rem}
%    This is a sentence typeset in the theorem environment {\envfont
%    Rem}.
% \end{Rem}
%
% \begin{Lem}[Ben User]
%    This is a sentence typeset in the theorem environment {\envfont
%    Lem}.
% \end{Lem}
%
% \begin{Def}[Very impressive Definition]
%    This is a sentence typeset in the theorem environment {\envfont
%    Def}.
% \end{Def}
%
% The last two examples show the effect of the optional argument to
% a theorem environment (it is the text typeset in parentheses).
%
%
% \section{Special Considerations}
% \label{sec:newfonts}
%
% Theoremheader and body are implemented as a unit. This means that
% the |\theoremheaderfont|  will inherit characteristics of the
% |\theorembodyfont| in \LaTeXe. Thus, if for example
% |\theorembodyfont| is |\itshape| and |\theoremheaderfont|
% is |\bfseries| the font selected for the header will have the
% characteristics `bold extended italic'. If this is not desired one
% should set the |\theoremheaderfont| to something like
% \begin{verbatim}
%      \theoremheaderfont{\normalfont\bfseries}
%\end{verbatim}
% i.e.\ supplying all necessary font information explicitly.
%
%
%
% \section{Acknowledgements}
%
% The publication of this set of macros was only possible with the
% help of Christina Busse (translating the manuscript into English),
% Joachim Pense (playing the r\^ole of typist), Chris Rowley
% (looking everything over) and many others providing useful
% suggestions.
%
% \MaybeStop{%
%    \begin{thebibliography}{9}
%       \bibitem{bk:GMS94} \textsc{M.~Goossens}, \textsc{F.~Mittelbach}
%         and \textsc{A.~Samarin}.
%         \newblock The \LaTeX{} Companion.
%         \newblock
%         Addison-Wesley, Reading, Massachusetts, 1994.
%       \bibitem{latex:source} \textsc{Lamport, Leslie.} \newblock
%                               \texttt{latex.tex}, version 2.09, date
%                               Feb. 1990.
%    \end{thebibliography}
%    \PrintIndex
%    \PrintChanges}
%
%
% \section{The documentation driver file}
%
% The next bit of code contains the documentation driver file for
% \TeX{}, i.e., the file that will produce the documentation you are
% currently reading. It will be extracted from this file by the
% \texttt{docstrip} program. Since it is the first code in the file
% one can alternatively process this file directly with \LaTeXe{} to
% obtain the documentation.
%
%    \begin{macrocode}
%<*driver>
\documentclass{ltxdoc}

\usepackage{theorem}

%<+driver>% The next few lines define theorem sets which are used
%<+driver>% in the example section of the documentation.

\theoremstyle{break}       \newtheorem{Cor}{Corollary}
\theoremstyle{plain}       \newtheorem{Exa}{Example}[section]
{\theorembodyfont{\upshape}\newtheorem{Rem}{Remark}}
\theoremstyle{marginbreak} \newtheorem{Lem}[Cor]{Lemma}
\theoremstyle{change}
\theorembodyfont{\itshape} \newtheorem{Def}[Cor]{Definition}

\theoremheaderfont{\scshape}

\RecordChanges

\begin{document}
   \DocInput{theorem.dtx}
\end{document}
%</driver>
%    \end{macrocode}
%
% \section{Definition of the Macros}
%
% If the file has been loaded before, we abort immediately. If not the
% package announces itself (this is actually done at the very top of
% the file---the way it is done isn't good style so don't copy it).
% \changes{v2.0e}{89/07/19}{Spaces removed from `typeout'.}
% \changes{v2.2b}{95/11/19}{Announce moved to top of file}
%    \begin{macrocode}
%<*package>
%\@ifundefined{theorem@style}{}{\endinput}
%\def\FMithmInfo{1995/11/19 v2.2b Theorem extension package (FMi)}
%\ProvidesPackage{theorem}[\FMithmInfo]
%    \end{macrocode}
%
%
% \subsection{Definition of theorem styles and fonts}
%
% All the definitions in this file are done globally to allow
% inputting this file inside a group.
%
% \begin{macro}{\theoremstyle}
%    Before a theorem style can be installed, the chosen style must be
%    known. For that reason, we must test to see that
%    |\th@|\meta{style} is known or, more precisely, that it is
%    different from |\relax|. If the style is not known then
%    |\th@plain| is used.
%    \begin{macrocode}
\gdef\theoremstyle#1{%
   \@ifundefined{th@#1}{\@warning
          {Unknown theoremstyle `#1'. Using `plain'}%
          \theorem@style{plain}}%
%    \end{macrocode}
%    We save the theorem style to be used in the token register
%    |\theorem@style|.
%    \begin{macrocode}
      {\theorem@style{#1}}%
%    \end{macrocode}
%    Now we ``evaluate'' the theorem style: this means, we call the
%    macro |\th@|\meta{style} which will activate the relevant
%    definitions which are contained in a separate file.  This is done
%    in a group to suppress changes to the current font.  This could
%    otherwise pose problems together with the new font selection
%    scheme\footnote{When I printed the original article using the new
%    font selection scheme I ended with a document with slanted
%    typefaces (text headings and all) simply because one of the
%    theorem styles used \texttt{\bslash sl} at toplevel.} if the
%    |\th@|\meta{style} is evaluated a second time.
% \changes{v2.1a}{90/02/15}{Group added around `th@..}
% \changes{v2.1a}{90/02/15}{Reseting of `..@begintheorem removed.}
%    \begin{macrocode}
      \begingroup
        \csname th@\the\theorem@style \endcsname
      \endgroup}
%    \end{macrocode}
% \end{macro}
%
%
% \begin{macro}{\@begintheorem}
% \changes{v2.1a}{90/02/15}{Relaxed at toplevel.}
% \begin{macro}{\@opargbegintheorem}
% \changes{v2.1a}{90/02/15}{Relaxed at toplevel.}
%    We reset |\@begintheorem| and |\@opargbegintheorem| to
%    |\relax| since these commands are no longer necessary at
%    toplevel. This will save a few tokens.
%    \begin{macrocode}
\global\let\@begintheorem\relax
\global\let\@opargbegintheorem\relax
%    \end{macrocode}
% \end{macro}
% \end{macro}
%
% \begin{macro}{\theorem@style}
%    Obviously the \textsf{token} register used above has to be
%    allocated. To assure the utmost compatibility with the
%    original \LaTeX{} definition, we set the default theorem style to
%    \texttt{plain}, which implements the usual \LaTeX{} convention.
% \changes{v2.0f}{89/07/20}{Unnecessary `global' removed.}
%    \begin{macrocode}
\newtoks\theorem@style
\global\theorem@style{plain}
%    \end{macrocode}
% \end{macro}
%
% \begin{macro}{\theorembodyfont}
% \begin{macro}{\theorem@bodyfont}
%    If the |\theorembodyfont| is set by the user then it should not
%    interact with the default font set in the theorem style. When the
%    new font selection is in force this may happen if, for example,
%    the default is |\itshape| and the new |\theorembodyfont| is
%    |\sffamily|. So we add a |\reset@font| command in front of the
%    user definition.
% \changes{v2.0f}{89/07/20}{Unnecessary `global' removed.}
% \changes{v2.0g}{89/09/19}{`theoremfont to `theorembodyfont.}
% \changes{v2.1c}{92/06/22}{`reset@font added and logic changed.}
% \changes{v2.1d}{92/10/28}{Reset to empty if arg is empty}
%    \begin{macrocode}
\gdef\theorembodyfont#1{%
%    \end{macrocode}
%    We check if the argument supplied is empty and if so put nothing
%    into the |\theorem@bodyfont| token register to allow for
%    |\theorembodyfont{}| as a mean of using the default of the
%    current |\theoremstyle|.
%    \begin{macrocode}
   \def\@tempa{#1}%
   \ifx\@tempa\@empty
    \theorem@bodyfont{}%
   \else
    \theorem@bodyfont{\reset@font#1}%
   \fi
   }
\newtoks\theorem@bodyfont
\global\theorem@bodyfont{}
%    \end{macrocode}
% \end{macro}
% \end{macro}
%
%
% \begin{macro}{\theoremheaderfont}
% \changes{v2.0g}{89/09/19}{Macro added.}
%    The font for the theorem headers is handled differently because
%    this definition applies to all theorem styles.
%    \begin{macrocode}
\gdef\theoremheaderfont#1{\gdef\theorem@headerfont{#1}%
%    \end{macrocode}
%    After using the macro once it is redefined to produce an error
%    message.
%    \begin{macrocode}
       \gdef\theoremheaderfont##1{%
        \typeout{\string\theoremheaderfont\space should be used
                 only once.}}}
%    \end{macrocode}
% \end{macro}
%
% \begin{macro}{\theorem@headerfont}
% \changes{v2.0g}{89/09/19}{Macro added.}
%    To set the |\theorem@headerfont| default we first test if
%    the new fontselection scheme is in force.
% \changes{v2.1a}{90/02/15}{Support for new font selection.}
%    \begin{macrocode}
\ifx\upshape\undefined
%    \end{macrocode}
%    If not we define it to expand into |\bfseries|. We don't use
%    |\let| just in case a following style option redefines this
%    macro.
%    \begin{macrocode}
\gdef\theorem@headerfont{\bfseries}
%    \end{macrocode}
%    Otherwise we reset the current shape before calling |\bfseries|.
%    \begin{macrocode}
\else \gdef\theorem@headerfont{\normalfont\bfseries}\fi
%    \end{macrocode}
% \end{macro}
%
% \begin{macro}{\th@plain}
% \begin{macro}{\th@break}
% \begin{macro}{\th@marginbreak}
% \begin{macro}{\th@changebreak}
% \begin{macro}{\th@change}
% \begin{macro}{\th@margin}
%    The different styles are defined in macros such as
%    |\th@plain|.  Since memory space is precious in
%    ``non-Big-versions'', we have to avoid offering too many unused
%    definitions.  Therefore we define these styles in separate files
%    that can be loaded on demand.  Thus the commands themselves only
%    load these files. We use |\@input@|, a \LaTeXe{} internal command
%    that ensures that the file will be listed with |\listfiles|.
% \changes{v2.2a}{94/02/02}{Use `@input@ to load theorem layout files}
%    \begin{macrocode}
\gdef\th@plain{\@input@{thp.sty}}
\gdef\th@break{\@input@{thb.sty}}
\gdef\th@marginbreak{\@input@{thmb.sty}}
\gdef\th@changebreak{\@input@{thcb.sty}}
\gdef\th@change{\@input@{thc.sty}}
\gdef\th@margin{\@input@{thm.sty}}
%    \end{macrocode}
%    This list will be expanded when new styles become
%    available.  For testing,  just append new theorem substyles
%    as document options.
% \end{macro}
% \end{macro}
% \end{macro}
% \end{macro}
% \end{macro}
% \end{macro}
%
%
%
% \subsection{Definition of a new theorem set}
%
% As already pointed out, a new theorem environment can be defined in
% three different ways:
% \begin{verbatim}
%    \newtheorem{Lem}{Lemma}
%    \newtheorem{Lem}{Lemma}[section]
%    \newtheorem{Lem}[Theorem]{Lemma}
%\end{verbatim}
% The function of the macro |\newtheorem| is to recognize these cases
% and then to branch into one of the three macros |\@ynthm|, |\@xnthm|
% or |\@othm|.  This mechanism is adopted unchanged from
% \cite{latex:source}; the essential point here is that, for example,
% in the second case, the arguments \texttt{Lem}, \texttt{Lemma} and
% \texttt{section} are passed over to the macro |\@xnthm|.
%
% We inspect this case first because the others present fewer problems,
% and thus are easily derived from this one.
%
% \begin{macro}{\@xnthm}
%    For our example arguments, the macro |\@xnthm| must fulfill
%    the following:
%    \begin{itemize}
%       \item
%          Define a new \LaTeX-counter `Lem'
%       \item
%          reset this counter within a |\section|
%       \item
%          define the macro |\theLem|
%       \item
%          define the environment macros |\Lem| and
%          |\endLem| using the current
%          |\theoremstyle| and |\theorem@bodyfont|.
%    \end{itemize}
%    Obviously, all this should happen only if the first argument
%    of |\@xnthm| (i.e.\ \texttt{Lem} in our example) is chosen so
%    as not to conflict with any previously defined commands or
%    environments.  This test is performed by the \LaTeX{} macro
%    |\@ifdefinable|.
%    \begin{macrocode}
\gdef\@xnthm#1#2[#3]{\expandafter\@ifdefinable\csname #1\endcsname
%    \end{macrocode}
%    Therefore, the first argument of |\@ifdefinable| is the
%    expansion (in the example, |\Lem|) of
%    |\csname||#1\endcsname|.  The second argument is
%    executed only if the test has been completed successfully.
%    \begin{macrocode}
   {%
%    \end{macrocode}
%    Now we define the new counter. The names of the \LaTeX{}
%    macros employed should speak for themselves:
% \changes{v2.2b}{1995/10/07}{Use \cs{@newctr}}
% \changes{v2.2c}{1995/11/23}{Use \cs{@newctr} properly.}
%    \begin{macrocode}
    \@definecounter{#1}\@newctr{#1}[#3]%
%    \end{macrocode}
%    Using |\@newctr| will give a proper error message if the counter
%    in |#3| is not defined.
%    In defining `|\theLem|' we must generate the desired macro
%    name by use of |\expandafter| and |\csname|.
%    \begin{macrocode}
    \expandafter\xdef\csname the#1\endcsname
%    \end{macrocode}
%    An |\xdef| is used in order to make the definition global,
%    and to ensure that it contains the replacement texts of
%    |\@thmcountersep| and |\@thmcounter|.\footnote{These
%    two macros can be defined by the document style.  Their default
%    values produce a `\texttt{.}' as separation and an arabic
%    representation of the number.} However, not everything should be
%    expanded. For example, it saves space to use |\thesection|
%    instead of its---at times---lengthy expansion.
%    \begin{macrocode}
      {\expandafter \noexpand \csname the#3\endcsname
       \@thmcountersep \@thmcounter{#1}}%
%    \end{macrocode}
%    Thus with the defaults of \LaTeX{}, |\theLem| would be
%    replaced by the command sequence |\thesection.\arabic{Lem}|.
%
%    We will now look at the definition of the macro which is executed
%    at the beginning of the actual environment (in our example this
%    macro is |\Lem|). It should be noted that we use an
%    ``|\expandafter| trick'' to expand only certain parts of the
%    replacement text at the time of the definition.
%    \begin{macrocode}
    \def\@tempa{\global\@namedef{#1}}%
    \expandafter \@tempa \expandafter{%
%    \end{macrocode}
%    First, the macro that contains the current definitions of
%    |\@begintheorem| and |\@opargtheorem| should be called
%    up.  The name of this macro---as is already known---has the form
%    |\th@|\meta{theorem\ style}; therefore, it must be called by
%    \begin{macrocode}
      \csname th@\the \theorem@style
%    \end{macrocode}
%    In addition the default theorem font should be changeable, i.e.\
%    we have to insert the contents of |\theorem@bodyfont|. For
%    that reason, we expand even further, beyond |\endcsname|,
%    and thus insert the contents of the \textsf{token} register
%    |\theorem@bodyfont| in the replacement text.
%    \begin{macrocode}
            \expandafter \endcsname \the \theorem@bodyfont
%    \end{macrocode}
%    Now it is time to call the macro |\@thm| which takes over
%    the further processing.  It has two arguments: the current
%    counter name (in our example, \texttt{Lem}), and the text of the
%    label (in our example, \texttt{Lemma}).
%    \begin{macrocode}
     \@thm{#1}{#2}}%
%    \end{macrocode}
%    With this, the `sub-definition' is complete.  The macro
%    |\@endtheorem| ends a theorem environment and is, so far, nothing
%    but an |\endtrivlist|.  (Hence it is defined globally, and not
%    within the theorem styles.\footnote{This has to be changed as
%    soon as theorem styles that change \texttt{\bslash @endtheorem}
%    exist.  In such a case, all existing styles must be changed as
%    well since they will have to reset the macro.}) Therefore, we can
%    set it equivalent to the macro that ends the theorem set (in our
%    example, |\endLem|).  However, if some day theorem styles exist
%    that do change |\@endtheorem|, we would have to use the
%    commented-out line instead.
%    \begin{macrocode}
    \global \expandafter \let \csname end#1\endcsname \@endtheorem
%   \global\@namedef{end#1}{\@endtheorem}%
%    \end{macrocode}
%    With these commands all the required definitions are employed,
%    unless the test |\@ifdefinable| has failed.  Therefore, we
%    end the second argument of this macro and with it the definition
%    of |\@xnthm|.
%    \begin{macrocode}
   }}
%    \end{macrocode}
% \end{macro}
%
% \begin{macro}{\@ynthm}
%    The definition of |\@ynthm| is completely analogous. In
%    this case the new counter that is defined is not reset within
%    another counter; thus the definition of |\the...|
%    is simplified:
%    \begin{macrocode}
\gdef\@ynthm#1#2{\expandafter\@ifdefinable\csname #1\endcsname
   {\@definecounter{#1}%
    \expandafter\xdef\csname the#1\endcsname{\@thmcounter{#1}}%
%    \end{macrocode}
%    The rest of the definition corresponds literally to that of
%    |\@xnthm|:
%    \begin{macrocode}
    \def\@tempa{\global\@namedef{#1}}\expandafter \@tempa
     \expandafter{\csname th@\the \theorem@style \expandafter
     \endcsname \the\theorem@bodyfont \@thm{#1}{#2}}%
    \global \expandafter \let \csname end#1\endcsname \@endtheorem}}
%    \end{macrocode}
% \end{macro}
%
% \begin{macro}{\@othm}
%    The definition of |\@othm| does not contain anything new.
%    \begin{macrocode}
\gdef\@othm#1[#2]#3{%
%    \end{macrocode}
% \changes{v2.2b}{95/10/07}{Build in check for counter defined}
%    We do not define a new counter but instead use one that has
%    already been defined.  Thus the only definition we need is that
%    of this pseudo-counter (i.e.\ |\the|\meta{env.\ name}).
%    First we check if |#2| corresponds to a known counter name.
%    \begin{macrocode}
  \expandafter\ifx\csname c@#2\endcsname\relax
   \@nocounterr{#2}%
  \else
   \expandafter\@ifdefinable\csname #1\endcsname
   {\expandafter \xdef \csname the#1\endcsname
     {\expandafter \noexpand \csname the#2\endcsname}%
%    \end{macrocode}
%    All other parts of the definition can be adopted from
%    |\@xnthm|. We have to remember, though, that in this case
%    the name of the current counter and the theorem label have moved
%    to the second and third arguments.
%    \begin{macrocode}
    \def\@tempa{\global\@namedef{#1}}\expandafter \@tempa
     \expandafter{\csname th@\the \theorem@style \expandafter
     \endcsname \the\theorem@bodyfont \@thm{#2}{#3}}%
    \global \expandafter \let \csname end#1\endcsname \@endtheorem}%
  \fi}
%    \end{macrocode}
% \end{macro}
%
% \subsection{Macros that are employed in a {\envfont theorem}
%    environment}
%
% \begin{macro}{\@thm}
%    The macro |\@thm| has to increase the current counter.  Then,
%    depending on whether the environment has (or does not have) an
%    optional argument, it has to branch into either |\@begintheorem|
%    or |\@opargtheorem|.
%    \begin{macrocode}
\gdef\@thm#1#2{\refstepcounter{#1}%
%    \end{macrocode}
%    Now we start a {\envfont trivlist} environment, and give
%    |\@topsep| and |\@topsepadd| the values of the \textsf{skip}
%    registers |\theorempreskipamount| and |\theorempostskipamount|.
%    The value in |\@topsep| is the vertical space that is inserted by
%    the first (and only) |\item| in our |\trivlist| whilst
%    |\@topsepadd| is inserted by |\@endparenv| at the end of that
%    {\envfont trivlist} environment.  By using these registers, we
%    obtain the desired space around a {\envfont theorem} environment.
%    \begin{macrocode}
   \trivlist
   \@topsep \theorempreskipamount               % used by first \item
   \@topsepadd \theorempostskipamount           % used by \@endparenv
%    \end{macrocode}
%    Now we have to test whether an optional argument has been given.
% \changes{v2.0d}{89/07/18}{Forgotten percent added.}
%    \begin{macrocode}
   \@ifnextchar [%
%    \end{macrocode}
%    If there is an optional argument, we will call |\@ythm|, and
%    move the arguments read back into the input stream.
%    \begin{macrocode}
   {\@ythm{#1}{#2}}%
%    \end{macrocode}
%    If not, we call |\@begintheorem|. Its first argument is the
%    name of the theorem set (hence the second argument of
%    |\@thm|). Its second argument is the macro that produces the
%    current number.
%    \begin{macrocode}
   {\@begintheorem{#2}{\csname the#1\endcsname}\ignorespaces}}
%    \end{macrocode}
% \end{macro}
%
% \begin{macro}{\@xthm}
% \begin{macro}{\@ythm}
%    Both these macros were originally called by |\@thm|.  We do
%    not need |\@xthm| anymore, hence we reset it to
%    |\relax|. The definition of |\@ythm| has not changed at
%    all from its definition in \LaTeX. In order to make the macros
%    easier to understand, we will
%    nevertheless present it (commented out).
%    \begin{macrocode}
\global\let\@xthm\relax
% \def\@ythm#1#2[#3]{\@opargbegintheorem{#2}{\csname
%           the#1\endcsname}{#3}\ignorespaces}
%</package>
%    \end{macrocode}
%    The primitive |\ignorespaces| in |\@ythm| and
%    |\@thm| is needed to remove the spaces between the
%    |\begin{...}| and the actual text.
% \end{macro}
% \end{macro}
%
%
% \subsection{Definition of the theorem substyles}
%
% As already pointed out, the theorem substyles, defined below, are
% only loaded when necessary.  Note that all these substyles, except
% \texttt{plain}, have |\slshape| as the default body font.
%
% \subsubsection{The \texttt{plain} style}
%
% \label{dok:th@plain} As the following macros use |@|, we have
% to locally set the |\catcode| of this symbol to ``letter''.
% This happens within a group, so that we do not have to worry about
% which |\catcode| that symbol had before.
%    \begin{macrocode}
%<*thp>
\begingroup \makeatletter
%    \end{macrocode}
% Since we are now within a group, we must make all definitions
% globally.  First we make sure that \texttt{theorem.sty} is loaded.
% This will allow us to use this file as a document style option
% without having to call \texttt{theorem} itself as an option.  At the
% same time, we assure that at least version 2 is loaded, since
% |\theorem@style| was not defined in earlier versions.
%    \begin{macrocode}
\@ifundefined{theorem@style}{\input{theorem.sty}}{}
\ProvidesFile{thp.sty}
             [\FMithmInfo]
%    \end{macrocode}
%
%
% \begin{macro}{\th@plain}
% \begin{macro}{\@begintheorem}
% \begin{macro}{\@opargbegintheorem}
%    |\theoremstyle{plain}| corresponds to the original
%    definition, except that the distances to the surrounding text are
%    determined by the parameters |\theorempreskipamount| and
%    |\theorempostskipamount|.
%    First we set the default body font.
%    \begin{macrocode}
\gdef\th@plain{\normalfont\itshape
%    \end{macrocode}
%    Then we define |\@begintheorem| and
%    |\@opargbegintheorem|. These two macros define how the
%    header of a theorem is typeset.  |\@opargbegintheorem| will
%    be called if a \textsf{theorem} environment with an optional
%    argument is encountered; otherwise, the header is constructed by
%    calling |\@begintheorem|.  If one of these macros is
%    executed, we are within a \textsf{trivlist} environment started by
%    |\@thm|.  So the theorem header is produced with an
%    |\item| command.
%
%    Instead of specifying the header font directly, all standard
%    theorem styles use the |\theorem@headerfont| macro to allow
%    customization.  The extra space (|\labelsep|) is necessary
%    because of problems in the \textsf{trivlist} environment.
%    \begin{macrocode}
  \def\@begintheorem##1##2{%
        \item[\hskip\labelsep \theorem@headerfont ##1\ ##2]}%
%    \end{macrocode}
%    The definition of |\@opargbegintheorem| is completely
%    analogous. The only difference is the fact that there exists a
%    third argument (which is the optional parameter of the
%    environment and contains additional information about the
%    theorem).  Customarily we enclose it in parentheses.
%    \begin{macrocode}
\def\@opargbegintheorem##1##2##3{%
   \item[\hskip\labelsep \theorem@headerfont ##1\ ##2\ (##3)]}}
%    \end{macrocode}
% \end{macro}
% \end{macro}
% \end{macro}
%
% \noindent
% We conclude with an |\endgroup| to restore the
% |\catcode| of |@|.
%    \begin{macrocode}
\endgroup
%</thp>
%    \end{macrocode}
%
%
% \subsubsection{The \texttt{break} style}
%
% This style option is stored in the file |thb.sty|.
% For the next two lines see the documentation for |\th@plain| on
% page~\pageref{dok:th@plain}.
%    \begin{macrocode}
%<*thb>
\begingroup \makeatletter
\@ifundefined{theorem@style}{\input{theorem.sty}}{}
\ProvidesFile{thb.sty}
             [\FMithmInfo]
%    \end{macrocode}
%
% \begin{macro}{\th@break}
%    |\theoremstyle{break}| produces a line break after the name
%    of the theorem.  The font is |\slshape|.  Hence, we define
%    |\th@break| as follows:
%    \begin{macrocode}
\gdef\th@break{\normalfont\slshape
  \def\@begintheorem##1##2{\item[%
%    \end{macrocode}
%    We run into the following problem: it is not possible to create
%    the header with |\item[|\meta{title}|]| and then start
%    a new line by, for example, |\mbox{}\\|. Such a definition
%    will fail whenever a list environment follows immediately.  With
%    the above construction, the |\mbox{}| causes the switch
%    |@inlabel| (cf.\ definition of |\list| and
%    |\trivlist| in \cite{latex:source}) to be set to \texttt{false}
%    and so the following list will insert additional vertical
%    space (|\topskip|). This is quite annoying. Therefore, we
%    create the line break within the |\item|. In order to ensure
%    that the text will begin at the proper position in the following
%    line, we simply pretend that the label does not take any
%    room.\footnote{This will lead to problems whenever very high
%    symbols occurring in the line tower into the
%    heading. So, something else has to be done here sometime.}
%    \begin{macrocode}
       \rlap{\vbox{\hbox{\hskip \labelsep\theorem@headerfont ##1\ ##2}%
                   \hbox{\strut}}}]}%
%    \end{macrocode}
%    Again, the definition of |\@opargbegintheorem| is
%    completely analogous.
%    \begin{macrocode}
\def\@opargbegintheorem##1##2##3{%
  \item[\rlap{\vbox{\hbox{\hskip \labelsep \theorem@headerfont
                     ##1\ ##2\ (##3)}%
                    \hbox{\strut}}}]}}
\endgroup
%</thb>
%    \end{macrocode}
% \end{macro}
%
%
% \subsubsection{The \texttt{changebreak} style}
%
%    \begin{macrocode}
% This style option is stored in the file |thcb.sty|.
%    \begin{macrocode}
%<*thcb>
\begingroup \makeatletter
\@ifundefined{theorem@style}{\input{theorem.sty}}{}
\ProvidesFile{thcb.sty}
             [\FMithmInfo]
%    \end{macrocode}
%
% \begin{macro}{\th@changebreak}
%    The \texttt{change-break} theorem style is like \texttt{break}
%    but with interchange of theorem name and theorem number.  Thus we
%    define |\th@changebreak| as follows:
%    \begin{macrocode}
\gdef\th@changebreak{\normalfont\slshape
  \def\@begintheorem##1##2{\item
       [\rlap{\vbox{\hbox{\hskip\labelsep \theorem@headerfont ##2\ ##1}%
                    \hbox{\strut}}}]}%
\def\@opargbegintheorem##1##2##3{%
  \item[\rlap{\vbox{\hbox{\hskip\labelsep \theorem@headerfont
                     ##2\ ##1\ (##3)}%
                    \hbox{\strut}}}]}}
\endgroup
%</thcb>
%    \end{macrocode}
% \end{macro}
%
%
% \subsubsection{The \texttt{change} style}
%
% This style option is stored in the file |thc.sty|.
%    \begin{macrocode}
%<*thc>
\begingroup \makeatletter
\@ifundefined{theorem@style}{\input{theorem.sty}}{}
\ProvidesFile{thc.sty}
             [\FMithmInfo]
%    \end{macrocode}
%
% \begin{macro}{\th@change}
%    The \texttt{change} theorem style corresponds to the
%    \texttt{change break} style without a linebreak after the header.
%    To say it in another way, it's the same as the \texttt{plain}
%    style but with number and name interchanged and |\slshape| as the
%    default font.
%    \begin{macrocode}
\gdef\th@change{\normalfont\slshape
  \def\@begintheorem##1##2{\item
       [\hskip\labelsep \theorem@headerfont ##2\ ##1]}%
\def\@opargbegintheorem##1##2##3{%
  \item[\hskip\labelsep \theorem@headerfont ##2\ ##1\ (##3)]}}
\endgroup
%</thc>
%    \end{macrocode}
% \end{macro}
%
%
% \subsubsection{The \texttt{marginbreak} style}
%
% This style option is the one used most often at Mainz.  It is saved
% in the file |thmb.sty|.
%    \begin{macrocode}
%<*thmb>
\begingroup \makeatletter
\@ifundefined{theorem@style}{\input{theorem.sty}}{}
\ProvidesFile{thmb.sty}
             [\FMithmInfo]
%    \end{macrocode}
%
% \begin{macro}{\th@marginbreak}
%    The \texttt{margin break} style is nearly the same as the
%    \texttt{change break} style. The only difference is the placement
%    of the theorem number.  We use |\llap| to place it in the left
%    margin.
%
%    In this style |\labelsep| denotes the separation between the
%    number and the text.
%    \begin{macrocode}
\gdef\th@marginbreak{\normalfont\slshape
  \def\@begintheorem##1##2{\item
       [\rlap{\vbox{\theorem@headerfont
                    \hbox{\llap{##2}\hskip\labelsep ##1}%
                    \hbox{\strut}}}]}%
\def\@opargbegintheorem##1##2##3{%
  \item[\rlap{\vbox{\theorem@headerfont
                    \hbox{\llap{##2}\hskip\labelsep ##1\ (##3)}%
                    \hbox{\strut}}}]}}
\endgroup
%</thmb>
%    \end{macrocode}
% \changes{v2.1c}{90/02/17}{`endgroup outside macrocode env.}
% \end{macro}
%
%
%
% \subsubsection{The \texttt{margin} style}
%
% This style option is stored in the file |thm.sty|.
%    \begin{macrocode}
%<*thm>
\begingroup \makeatletter
\@ifundefined{theorem@style}{\input{theorem.sty}}{}
\ProvidesFile{thm.sty}
             [\FMithmInfo]
%    \end{macrocode}
%
% \begin{macro}{\th@margin}
%    Again this is only a variant of the theorem styles described
%    above without any new ideas.
%    \begin{macrocode}
\gdef\th@margin{\normalfont\slshape
  \def\@begintheorem##1##2{\item
       [\theorem@headerfont \llap{##2}\hskip\labelsep ##1]}%
\def\@opargbegintheorem##1##2##3{%
  \item[\theorem@headerfont \llap{##2}\hskip\labelsep ##1\ (##3)]}}
\endgroup
%</thm>
%    \end{macrocode}
% \end{macro}
%
%
%
%
% \subsection{Final Definitions}
%
% \begin{macro}{\theorempreskipamount}
% \begin{macro}{\theorempostskipamount}
%    The \textsf{skip} parameters that regulate the vertical empty space
%    before and after the {\envfont theorem} environment have
%    to be allocated as well.
% \changes{v2.0f}{89/07/20}{Unnecessary `global' removed.}
%    \begin{macrocode}
%<*package>
\newskip\theorempreskipamount
\newskip\theorempostskipamount
%    \end{macrocode}
%    Since we have used the same values for all theorem sets, we now can
%    assign them.
%    \begin{macrocode}
\global\setlength\theorempreskipamount{12pt plus 5pt minus 3pt}
\global\setlength\theorempostskipamount{8pt plus 3pt minus 1.5pt}
%    \end{macrocode}
% \end{macro}
% \end{macro}
%
%
% \begin{macro}{\@endtheorem}
%    The same holds for the macro |\@endtheorem|, which ends a
%    {\envfont theorem} environment. Since it is the same for all
%    theorem sets, it is removed from the macros |\th@|\meta{style}.
%    It simply ends the {\envfont trivlist} environment, which was
%    begun in |\@thm|.
%    \begin{macrocode}
\global\let\@endtheorem=\endtrivlist
%    \end{macrocode}
% \end{macro}
%
%
%
% \begin{macro}{\@preamblecmds}
%    All macros defined above are to be used only in the preamble.
%    Therefore, we insert them in |\@preamblecmds| which will disable
%    them at begin document. This is done by the internal \LaTeXe{}
%    command |\@onlypreamble|.
% \changes{v2.0g}{89/09/19}{`theoremheaderfont and `theorembodyfont
%    added.}
% \changes{v2.2a}{94/02/03}{Use \LaTeXe{} interface}
%    \begin{macrocode}
\@onlypreamble\@xnthm
\@onlypreamble\@ynthm
\@onlypreamble\@othm
\@onlypreamble\newtheorem
\@onlypreamble\theoremstyle
\@onlypreamble\theorembodyfont
\@onlypreamble\theoremheaderfont
%    \end{macrocode}
% \end{macro}
%
% Finally we declare the \texttt{plain} theorem style to be the default.
% \changes{v2.1b}{90/06/04}{Fixed missing `item by declaring default
%    style.}
%    \begin{macrocode}
\theoremstyle{plain}
%</package>
%    \end{macrocode}
%
% \Finale

\endinput
}{}
\ProvidesFile{thp.sty}
             [\FMithmInfo]
%    \end{macrocode}
%
%
% \begin{macro}{\th@plain}
% \begin{macro}{\@begintheorem}
% \begin{macro}{\@opargbegintheorem}
%    |\theoremstyle{plain}| corresponds to the original
%    definition, except that the distances to the surrounding text are
%    determined by the parameters |\theorempreskipamount| and
%    |\theorempostskipamount|.
%    First we set the default body font.
%    \begin{macrocode}
\gdef\th@plain{\normalfont\itshape
%    \end{macrocode}
%    Then we define |\@begintheorem| and
%    |\@opargbegintheorem|. These two macros define how the
%    header of a theorem is typeset.  |\@opargbegintheorem| will
%    be called if a \textsf{theorem} environment with an optional
%    argument is encountered; otherwise, the header is constructed by
%    calling |\@begintheorem|.  If one of these macros is
%    executed, we are within a \textsf{trivlist} environment started by
%    |\@thm|.  So the theorem header is produced with an
%    |\item| command.
%
%    Instead of specifying the header font directly, all standard
%    theorem styles use the |\theorem@headerfont| macro to allow
%    customization.  The extra space (|\labelsep|) is necessary
%    because of problems in the \textsf{trivlist} environment.
%    \begin{macrocode}
  \def\@begintheorem##1##2{%
        \item[\hskip\labelsep \theorem@headerfont ##1\ ##2]}%
%    \end{macrocode}
%    The definition of |\@opargbegintheorem| is completely
%    analogous. The only difference is the fact that there exists a
%    third argument (which is the optional parameter of the
%    environment and contains additional information about the
%    theorem).  Customarily we enclose it in parentheses.
%    \begin{macrocode}
\def\@opargbegintheorem##1##2##3{%
   \item[\hskip\labelsep \theorem@headerfont ##1\ ##2\ (##3)]}}
%    \end{macrocode}
% \end{macro}
% \end{macro}
% \end{macro}
%
% \noindent
% We conclude with an |\endgroup| to restore the
% |\catcode| of |@|.
%    \begin{macrocode}
\endgroup
%</thp>
%    \end{macrocode}
%
%
% \subsubsection{The \texttt{break} style}
%
% This style option is stored in the file |thb.sty|.
% For the next two lines see the documentation for |\th@plain| on
% page~\pageref{dok:th@plain}.
%    \begin{macrocode}
%<*thb>
\begingroup \makeatletter
\@ifundefined{theorem@style}{% \iffalse meta-comment
%
% Copyright (C) 1993-2024
%
% The LaTeX Project and any individual authors listed elsewhere
% in this file.
%
% This file is part of the Standard LaTeX `Tools Bundle'.
% -------------------------------------------------------
%
% It may be distributed and/or modified under the
% conditions of the LaTeX Project Public License, either version 1.3c
% of this license or (at your option) any later version.
% The latest version of this license is in
%    https://www.latex-project.org/lppl.txt
% and version 1.3c or later is part of all distributions of LaTeX
% version 2005/12/01 or later.
%
% The list of all files belonging to the LaTeX `Tools Bundle' is
% given in the file `manifest.txt'.
%
% \fi
%
%%
%
%
\def\FMithmInfo{2023/07/05 v2.2c Theorem extension package (FMi)}
%
% \ProvidesFile{theorem.dtx}[\FMithmInfo]
%\iffalse   % this is a METACOMMENT !
%%
%% Package `theorem' to use with LaTeX2e
%% Copyright (C) 1989-1995 Frank Mittelbach, all rights reserved.
%<+package>\@ifundefined{theorem@style}{}{\endinput}
%<+package>\NeedsTeXFormat{LaTeX2e}[1995/06/01]
%<+package>\ProvidesPackage{theorem}
%<+package>         [\FMithmInfo]
%
% \fi
%
% \changes{v1.0}{87/05/07}{First draft.}
% \changes{v1.0a}{87/05/07}{`thistheoremstyle' added.}
% \changes{v1.1a}{87/07/29}{New concept: redefine the internal
%                          \protect\LaTeX{} commands `@xnthm' etc.}
% \changes{v1.1b}{87/07/31}{`theoremfont' used to specify font.}
% \changes{v1.2a}{88/01/08}{`@endtheorem' now globally defined.}
% \changes{v1.2b}{88/01/11}{`newtheorem' only allowed in the preamble.}
% \changes{v1.3a}{88/02/14}{`theoremfont' now a toks register.}
%
% \changes{v2.0a}{88/11/16}{New implementation and new documentation.}
% \changes{v2.0a}{88/11/16}{No layout changes but internal modifications
%                           old substyles won't work.}
% \changes{v2.0b}{89/01/30}{`global' used for all definitions.}
% \changes{v2.0c}{89/06/15}{Documentation translated to English.
%                            Thanks to LLL and Nina}
% \changes{v2.1a}{90/02/15}{Support for new font selection.}
% \changes{v2.2a}{94/02/03}{Cleanup for LaTeX2e}
%
% \DoNotIndex{\@endparenv,\@ifdefinable,\@ifnextchar,\@ifundefined}
% \DoNotIndex{\@namedef,\@nameuse,\@spaces,\@tempa,\@warning}
% \DoNotIndex{\ }
% \DoNotIndex{\begingroup,\bfseries}
% \DoNotIndex{\csname}
% \DoNotIndex{\def,\do,\docdate}
% \DoNotIndex{\endcsname,\endgroup,\endinput,\endtrivlist,\expandafter}
% \DoNotIndex{\filedate,\fileversion}
% \DoNotIndex{\gdef,\global}
% \DoNotIndex{\hbox,\hskip}
% \DoNotIndex{\ignorespaces,\input,\itshape,\item}
% \DoNotIndex{\labelsep,\let,\llap}
% \DoNotIndex{\makeatletter}
% \DoNotIndex{\noexpand}
% \DoNotIndex{\refstepcounter,\relax,\rlap}
% \DoNotIndex{\setlength,\slshape,\space,\string,\strut}
% \DoNotIndex{\the,\trivlist,\typeout}
% \DoNotIndex{\vbox}
% \DoNotIndex{\xdef}
%
% \def\deflabel#1{\ttfamily #1\hfill}
% \def\deflist#1{\begin{list}{}{\settowidth\labelwidth{\ttfamily #1}%
%                               \setlength\leftmargin\labelwidth
%                               \addtolength\leftmargin\labelsep
%                               \let\makelabel\deflabel}}
% \def\enddeflist{\end{list}}
% \def\envfont{\normalfont\ttfamily}
%
% \MakeShortVerb{\|}
%
% \newcommand{\TUB}{{\sl TUGboat\/}}
%
% \GetFileInfo{theorem.dtx}
% \title{An Extension of the
%        \LaTeX{} theorem environment\thanks{This file
%        has version number \fileversion, last
%        revised \filedate.}}
% \author{Frank Mittelbach}
%
% \MaintainedByLaTeXTeam{tools}
% \maketitle
%
% \begin{abstract}
%    The macros described in this paper yield an extension of the
%    \LaTeX{} theorem mechanism. It is designed to satisfy the
%    different requirements of various journals. Thus, the
%    layout of the ``theorems'' can be manipulated by determining a
%    ``style''. This article describes not only the use, but
%    also the definition, of the necessary macros.
% \end{abstract}
%
% \section*{Preface to version 2.2}
%
% For \LaTeXe{} this package did not need any fundamental changes. I
% only modified the messages generated so that theorem layout styles
% will show up with the |\listfiles| command and cleaned the section
% on the New Font Selection Scheme since this is now included in
% \LaTeX.
%
% \section*{Preface to version 2.1}
%
% This version is identical to 2.0g described in \TUB\ 10\#3 except
% for some internal defaults which are now set depending on the used
% font selection scheme.
%
% This was done to avoid unpleasant surprises if the new font selection
% scheme is in force. For further details see
% section~\ref{sec:newfonts} and \cite{bk:GMS94}.
%
% \section{Introduction}
%
% For our purposes here, ``theorems'' are labelled enunciations,
% often set off from the main text by extra space and a font change.
% Theorems, corollaries, conjectures, definitions, and remarks
% are all instances of ``theorems''.  The ``header'' of these structures
% is composed of a label (such as \textsc{Theorem} or \textsc{Remark})
% and a number which serializes an item in the sequence of items
% with the same label.
%
% Shortly after the introduction of \LaTeX{} at the Fachbereich
% Mathematik in Mainz, the desire to manipulate the layout of
% ``theorems'' arose.  In Mainz, the following two conventions came
% into general use:
% \begin{enumerate}
%   \item The number of the theorem is shown in the margin.
%   \item There is a line break at
%         the end of the theorem header.
% \end{enumerate}
%
% Additionally, some journals require different formats which depend on
% the ``sort of theorem'': e.g.\ often remarks and definitions are set
% in |\upshape|, while |\itshape| is employed for main theorems.
%
% Confronted with these requirements, a theorem environment was
% developed in Mainz which allows separate determination of the
% layout of the ``theorems sets'', comparable to |\pagestyle|.
%
%
% \section{The user interface}
% \subsection{Defining new theorem sets}
%
% \DescribeMacro\newtheorem
% As in the original \LaTeX{} version, the command |\newtheorem|
% defines a new ``theorem set'' or ``theorem-like structure''.
% Two required arguments name the new environment and give the text
% to be typeset with each instance of the new ``set'', while
% an optional argument determines how
% the ``set'' is enumerated:
% \begin{description}
%    \item[\ttfamily \bslash newtheorem\{foo\}\{bar\}]
%       The theorem set {\envfont foo} (whose name is \texttt{bar})
%       uses its own counter.
%    \item[\ttfamily \bslash newtheorem\{foo2\}{[foo]}\{bar2\}]
%       The theorem set {\envfont foo2} (printed name \texttt{bar2})
%       uses the same counter as the theorem set \texttt{foo}.
%    \item[\ttfamily \bslash newtheorem\{foo3\}\{bar3\}{[section]}]
%       The theorem set {\envfont foo3} (printed name \texttt{bar3}) is
%       enumerated within the counter \texttt{section}, i.e.\ with every
%       new |\section| the enumeration begins again with 1, and
%       the enumeration is composed from the section-number and the
%       theorem counter itself.
% \end{description}
%
% \DescribeMacro\theoremstyle
% Additionally, the command |\theoremstyle| can define the layout
% of various, or all, theorem sets.  It should be noted that any
% theorem set defined by |\newtheorem| is typeset in the
% |\theoremstyle| that is current at the time of the definition.
% Thus, the following
% \begin{verbatim}
%    \theoremstyle{break}        \newtheorem{Cor}{Corollary}
%    \theoremstyle{plain}        \newtheorem{Exa}{Example}[section]
%\end{verbatim}                             ^^A No blank at line start
% leads to the result that the set {\envfont Cor} is formatted in the
% style \texttt{break}, while the set {\envfont Exa} and all the
% following ones are formatted in the style \texttt{plain}, unless
% another |\theoremstyle| follows.  Since the definitions installed by
% |\newtheorem| are global, one also can limit |\theoremstyle| locally
% by grouping braces.
%
% \DescribeMacro\theorembodyfont
% The choice of the font for the theorem body is completely
% independent of the chosen |\theoremstyle|; this has proven to
% be very advantageous.  For example,
% \begin{verbatim}
%   {\theorembodyfont{\upshape}          \newtheorem{Rem}{Remark}}
%\end{verbatim}
% defines a theorem set {\envfont Rem}, which will be set in
% |\upshape| in the current layout (which in our example is
% \texttt{plain}).  As with |\theoremstyle|, the |\theorembodyfont|
% chosen is that current at the time of |\newtheorem|.  If
% |\theorembodyfont| is not specified or one defines
% |\theorembodyfont{}|, then the font used will be that defined by the
% |\theoremstyle|.
%
% \DescribeMacro\theoremheaderfont It is also possible to customize
% the font used for the theorem headers.  This is, however, a global
% declaration, and therefore there should be at most one
% |\theoremheaderfont| declaration in the preamble.\footnote{If it is
% actually necessary to have different header fonts, one has to define
% new theorem styles (substituting the desired font) or specify the
% information directly in the \texttt{\bslash newtheorem} declaration
% (the unclean variant).}
%
% \DescribeMacro\theorempreskipamount
% \DescribeMacro\theorempostskipamount Two additional parameters
% affect the vertical space around the theorem environments:
% |\theorempreskipamount| and |\theorempostskipamount| define,
% respectively, the spacing before and after such an environment.
% These parameters apply for all theorem sets and can be manipulated
% with the ordinary length macros.  They are rubber lengths
% (`\textsf{skips}'), and therefore can contain \texttt{plus} and
% \texttt{minus} parts.
%
% Since the definition of theorem sets should---most sensibly---be
% placed in the preamble, we only allow installation there.
% It is therefore possible to release the memory used here after
% |\begin{document}|, in order to make room for other
% applications.
%
%
% \subsection{Existing theorem styles}
%
% The following theorem styles exist to date:
% \begin{deflist}{marginbreak}
%    \item[plain]
%       This theorem style emulates the original \LaTeX{} definition,
%       except that additionally the parameters
%       |\theorem...skipamount| are used.
%    \item[break]
%       In this style, the theorem header is followed by a line
%       break.
%    \item[marginbreak]
%       The theorem number is set in the margin, and there is a line
%       break as in \texttt{break}.
%    \item[changebreak]
%       Like \texttt{break}, but with header number and text
%       interchanged.
%    \item[change]
%       Header number and text are interchanged, without a line break.
%    \item[margin]
%       The number is set in the left margin, without a line break.
% \end{deflist}
% All styles (except \texttt{plain}) select |\slshape| as the default
% |\theorembodyfont|.
%
%
% \subsection{Examples}
%
% Given the above theorem sets {\envfont Cor}, {\envfont Exa}
% and {\envfont Rem},
% suppose that the preamble also contains the declarations:
% \begin{verbatim}
%    \theoremstyle{marginbreak}   \newtheorem{Lem}[Cor]{Lemma}
%    \theoremstyle{change}
%    \theorembodyfont{\itshape}        \newtheorem{Def}[Cor]{Definition}
%
%    \theoremheaderfont{\scshape}
%\end{verbatim}
% Then the following are some typical examples of the typeset output
% resulting from their use.
%
% \begin{Cor}
%    This is a sentence typeset in the theorem environment {\envfont
%    Cor}.
% \end{Cor}
%
% \begin{Exa}
%    This is a sentence typeset in the theorem environment {\envfont
%    Exa}.
% \end{Exa}
%
% \begin{Rem}
%    This is a sentence typeset in the theorem environment {\envfont
%    Rem}.
% \end{Rem}
%
% \begin{Lem}[Ben User]
%    This is a sentence typeset in the theorem environment {\envfont
%    Lem}.
% \end{Lem}
%
% \begin{Def}[Very impressive Definition]
%    This is a sentence typeset in the theorem environment {\envfont
%    Def}.
% \end{Def}
%
% The last two examples show the effect of the optional argument to
% a theorem environment (it is the text typeset in parentheses).
%
%
% \section{Special Considerations}
% \label{sec:newfonts}
%
% Theoremheader and body are implemented as a unit. This means that
% the |\theoremheaderfont|  will inherit characteristics of the
% |\theorembodyfont| in \LaTeXe. Thus, if for example
% |\theorembodyfont| is |\itshape| and |\theoremheaderfont|
% is |\bfseries| the font selected for the header will have the
% characteristics `bold extended italic'. If this is not desired one
% should set the |\theoremheaderfont| to something like
% \begin{verbatim}
%      \theoremheaderfont{\normalfont\bfseries}
%\end{verbatim}
% i.e.\ supplying all necessary font information explicitly.
%
%
%
% \section{Acknowledgements}
%
% The publication of this set of macros was only possible with the
% help of Christina Busse (translating the manuscript into English),
% Joachim Pense (playing the r\^ole of typist), Chris Rowley
% (looking everything over) and many others providing useful
% suggestions.
%
% \MaybeStop{%
%    \begin{thebibliography}{9}
%       \bibitem{bk:GMS94} \textsc{M.~Goossens}, \textsc{F.~Mittelbach}
%         and \textsc{A.~Samarin}.
%         \newblock The \LaTeX{} Companion.
%         \newblock
%         Addison-Wesley, Reading, Massachusetts, 1994.
%       \bibitem{latex:source} \textsc{Lamport, Leslie.} \newblock
%                               \texttt{latex.tex}, version 2.09, date
%                               Feb. 1990.
%    \end{thebibliography}
%    \PrintIndex
%    \PrintChanges}
%
%
% \section{The documentation driver file}
%
% The next bit of code contains the documentation driver file for
% \TeX{}, i.e., the file that will produce the documentation you are
% currently reading. It will be extracted from this file by the
% \texttt{docstrip} program. Since it is the first code in the file
% one can alternatively process this file directly with \LaTeXe{} to
% obtain the documentation.
%
%    \begin{macrocode}
%<*driver>
\documentclass{ltxdoc}

\usepackage{theorem}

%<+driver>% The next few lines define theorem sets which are used
%<+driver>% in the example section of the documentation.

\theoremstyle{break}       \newtheorem{Cor}{Corollary}
\theoremstyle{plain}       \newtheorem{Exa}{Example}[section]
{\theorembodyfont{\upshape}\newtheorem{Rem}{Remark}}
\theoremstyle{marginbreak} \newtheorem{Lem}[Cor]{Lemma}
\theoremstyle{change}
\theorembodyfont{\itshape} \newtheorem{Def}[Cor]{Definition}

\theoremheaderfont{\scshape}

\RecordChanges

\begin{document}
   \DocInput{theorem.dtx}
\end{document}
%</driver>
%    \end{macrocode}
%
% \section{Definition of the Macros}
%
% If the file has been loaded before, we abort immediately. If not the
% package announces itself (this is actually done at the very top of
% the file---the way it is done isn't good style so don't copy it).
% \changes{v2.0e}{89/07/19}{Spaces removed from `typeout'.}
% \changes{v2.2b}{95/11/19}{Announce moved to top of file}
%    \begin{macrocode}
%<*package>
%\@ifundefined{theorem@style}{}{\endinput}
%\def\FMithmInfo{1995/11/19 v2.2b Theorem extension package (FMi)}
%\ProvidesPackage{theorem}[\FMithmInfo]
%    \end{macrocode}
%
%
% \subsection{Definition of theorem styles and fonts}
%
% All the definitions in this file are done globally to allow
% inputting this file inside a group.
%
% \begin{macro}{\theoremstyle}
%    Before a theorem style can be installed, the chosen style must be
%    known. For that reason, we must test to see that
%    |\th@|\meta{style} is known or, more precisely, that it is
%    different from |\relax|. If the style is not known then
%    |\th@plain| is used.
%    \begin{macrocode}
\gdef\theoremstyle#1{%
   \@ifundefined{th@#1}{\@warning
          {Unknown theoremstyle `#1'. Using `plain'}%
          \theorem@style{plain}}%
%    \end{macrocode}
%    We save the theorem style to be used in the token register
%    |\theorem@style|.
%    \begin{macrocode}
      {\theorem@style{#1}}%
%    \end{macrocode}
%    Now we ``evaluate'' the theorem style: this means, we call the
%    macro |\th@|\meta{style} which will activate the relevant
%    definitions which are contained in a separate file.  This is done
%    in a group to suppress changes to the current font.  This could
%    otherwise pose problems together with the new font selection
%    scheme\footnote{When I printed the original article using the new
%    font selection scheme I ended with a document with slanted
%    typefaces (text headings and all) simply because one of the
%    theorem styles used \texttt{\bslash sl} at toplevel.} if the
%    |\th@|\meta{style} is evaluated a second time.
% \changes{v2.1a}{90/02/15}{Group added around `th@..}
% \changes{v2.1a}{90/02/15}{Reseting of `..@begintheorem removed.}
%    \begin{macrocode}
      \begingroup
        \csname th@\the\theorem@style \endcsname
      \endgroup}
%    \end{macrocode}
% \end{macro}
%
%
% \begin{macro}{\@begintheorem}
% \changes{v2.1a}{90/02/15}{Relaxed at toplevel.}
% \begin{macro}{\@opargbegintheorem}
% \changes{v2.1a}{90/02/15}{Relaxed at toplevel.}
%    We reset |\@begintheorem| and |\@opargbegintheorem| to
%    |\relax| since these commands are no longer necessary at
%    toplevel. This will save a few tokens.
%    \begin{macrocode}
\global\let\@begintheorem\relax
\global\let\@opargbegintheorem\relax
%    \end{macrocode}
% \end{macro}
% \end{macro}
%
% \begin{macro}{\theorem@style}
%    Obviously the \textsf{token} register used above has to be
%    allocated. To assure the utmost compatibility with the
%    original \LaTeX{} definition, we set the default theorem style to
%    \texttt{plain}, which implements the usual \LaTeX{} convention.
% \changes{v2.0f}{89/07/20}{Unnecessary `global' removed.}
%    \begin{macrocode}
\newtoks\theorem@style
\global\theorem@style{plain}
%    \end{macrocode}
% \end{macro}
%
% \begin{macro}{\theorembodyfont}
% \begin{macro}{\theorem@bodyfont}
%    If the |\theorembodyfont| is set by the user then it should not
%    interact with the default font set in the theorem style. When the
%    new font selection is in force this may happen if, for example,
%    the default is |\itshape| and the new |\theorembodyfont| is
%    |\sffamily|. So we add a |\reset@font| command in front of the
%    user definition.
% \changes{v2.0f}{89/07/20}{Unnecessary `global' removed.}
% \changes{v2.0g}{89/09/19}{`theoremfont to `theorembodyfont.}
% \changes{v2.1c}{92/06/22}{`reset@font added and logic changed.}
% \changes{v2.1d}{92/10/28}{Reset to empty if arg is empty}
%    \begin{macrocode}
\gdef\theorembodyfont#1{%
%    \end{macrocode}
%    We check if the argument supplied is empty and if so put nothing
%    into the |\theorem@bodyfont| token register to allow for
%    |\theorembodyfont{}| as a mean of using the default of the
%    current |\theoremstyle|.
%    \begin{macrocode}
   \def\@tempa{#1}%
   \ifx\@tempa\@empty
    \theorem@bodyfont{}%
   \else
    \theorem@bodyfont{\reset@font#1}%
   \fi
   }
\newtoks\theorem@bodyfont
\global\theorem@bodyfont{}
%    \end{macrocode}
% \end{macro}
% \end{macro}
%
%
% \begin{macro}{\theoremheaderfont}
% \changes{v2.0g}{89/09/19}{Macro added.}
%    The font for the theorem headers is handled differently because
%    this definition applies to all theorem styles.
%    \begin{macrocode}
\gdef\theoremheaderfont#1{\gdef\theorem@headerfont{#1}%
%    \end{macrocode}
%    After using the macro once it is redefined to produce an error
%    message.
%    \begin{macrocode}
       \gdef\theoremheaderfont##1{%
        \typeout{\string\theoremheaderfont\space should be used
                 only once.}}}
%    \end{macrocode}
% \end{macro}
%
% \begin{macro}{\theorem@headerfont}
% \changes{v2.0g}{89/09/19}{Macro added.}
%    To set the |\theorem@headerfont| default we first test if
%    the new fontselection scheme is in force.
% \changes{v2.1a}{90/02/15}{Support for new font selection.}
%    \begin{macrocode}
\ifx\upshape\undefined
%    \end{macrocode}
%    If not we define it to expand into |\bfseries|. We don't use
%    |\let| just in case a following style option redefines this
%    macro.
%    \begin{macrocode}
\gdef\theorem@headerfont{\bfseries}
%    \end{macrocode}
%    Otherwise we reset the current shape before calling |\bfseries|.
%    \begin{macrocode}
\else \gdef\theorem@headerfont{\normalfont\bfseries}\fi
%    \end{macrocode}
% \end{macro}
%
% \begin{macro}{\th@plain}
% \begin{macro}{\th@break}
% \begin{macro}{\th@marginbreak}
% \begin{macro}{\th@changebreak}
% \begin{macro}{\th@change}
% \begin{macro}{\th@margin}
%    The different styles are defined in macros such as
%    |\th@plain|.  Since memory space is precious in
%    ``non-Big-versions'', we have to avoid offering too many unused
%    definitions.  Therefore we define these styles in separate files
%    that can be loaded on demand.  Thus the commands themselves only
%    load these files. We use |\@input@|, a \LaTeXe{} internal command
%    that ensures that the file will be listed with |\listfiles|.
% \changes{v2.2a}{94/02/02}{Use `@input@ to load theorem layout files}
%    \begin{macrocode}
\gdef\th@plain{\@input@{thp.sty}}
\gdef\th@break{\@input@{thb.sty}}
\gdef\th@marginbreak{\@input@{thmb.sty}}
\gdef\th@changebreak{\@input@{thcb.sty}}
\gdef\th@change{\@input@{thc.sty}}
\gdef\th@margin{\@input@{thm.sty}}
%    \end{macrocode}
%    This list will be expanded when new styles become
%    available.  For testing,  just append new theorem substyles
%    as document options.
% \end{macro}
% \end{macro}
% \end{macro}
% \end{macro}
% \end{macro}
% \end{macro}
%
%
%
% \subsection{Definition of a new theorem set}
%
% As already pointed out, a new theorem environment can be defined in
% three different ways:
% \begin{verbatim}
%    \newtheorem{Lem}{Lemma}
%    \newtheorem{Lem}{Lemma}[section]
%    \newtheorem{Lem}[Theorem]{Lemma}
%\end{verbatim}
% The function of the macro |\newtheorem| is to recognize these cases
% and then to branch into one of the three macros |\@ynthm|, |\@xnthm|
% or |\@othm|.  This mechanism is adopted unchanged from
% \cite{latex:source}; the essential point here is that, for example,
% in the second case, the arguments \texttt{Lem}, \texttt{Lemma} and
% \texttt{section} are passed over to the macro |\@xnthm|.
%
% We inspect this case first because the others present fewer problems,
% and thus are easily derived from this one.
%
% \begin{macro}{\@xnthm}
%    For our example arguments, the macro |\@xnthm| must fulfill
%    the following:
%    \begin{itemize}
%       \item
%          Define a new \LaTeX-counter `Lem'
%       \item
%          reset this counter within a |\section|
%       \item
%          define the macro |\theLem|
%       \item
%          define the environment macros |\Lem| and
%          |\endLem| using the current
%          |\theoremstyle| and |\theorem@bodyfont|.
%    \end{itemize}
%    Obviously, all this should happen only if the first argument
%    of |\@xnthm| (i.e.\ \texttt{Lem} in our example) is chosen so
%    as not to conflict with any previously defined commands or
%    environments.  This test is performed by the \LaTeX{} macro
%    |\@ifdefinable|.
%    \begin{macrocode}
\gdef\@xnthm#1#2[#3]{\expandafter\@ifdefinable\csname #1\endcsname
%    \end{macrocode}
%    Therefore, the first argument of |\@ifdefinable| is the
%    expansion (in the example, |\Lem|) of
%    |\csname||#1\endcsname|.  The second argument is
%    executed only if the test has been completed successfully.
%    \begin{macrocode}
   {%
%    \end{macrocode}
%    Now we define the new counter. The names of the \LaTeX{}
%    macros employed should speak for themselves:
% \changes{v2.2b}{1995/10/07}{Use \cs{@newctr}}
% \changes{v2.2c}{1995/11/23}{Use \cs{@newctr} properly.}
%    \begin{macrocode}
    \@definecounter{#1}\@newctr{#1}[#3]%
%    \end{macrocode}
%    Using |\@newctr| will give a proper error message if the counter
%    in |#3| is not defined.
%    In defining `|\theLem|' we must generate the desired macro
%    name by use of |\expandafter| and |\csname|.
%    \begin{macrocode}
    \expandafter\xdef\csname the#1\endcsname
%    \end{macrocode}
%    An |\xdef| is used in order to make the definition global,
%    and to ensure that it contains the replacement texts of
%    |\@thmcountersep| and |\@thmcounter|.\footnote{These
%    two macros can be defined by the document style.  Their default
%    values produce a `\texttt{.}' as separation and an arabic
%    representation of the number.} However, not everything should be
%    expanded. For example, it saves space to use |\thesection|
%    instead of its---at times---lengthy expansion.
%    \begin{macrocode}
      {\expandafter \noexpand \csname the#3\endcsname
       \@thmcountersep \@thmcounter{#1}}%
%    \end{macrocode}
%    Thus with the defaults of \LaTeX{}, |\theLem| would be
%    replaced by the command sequence |\thesection.\arabic{Lem}|.
%
%    We will now look at the definition of the macro which is executed
%    at the beginning of the actual environment (in our example this
%    macro is |\Lem|). It should be noted that we use an
%    ``|\expandafter| trick'' to expand only certain parts of the
%    replacement text at the time of the definition.
%    \begin{macrocode}
    \def\@tempa{\global\@namedef{#1}}%
    \expandafter \@tempa \expandafter{%
%    \end{macrocode}
%    First, the macro that contains the current definitions of
%    |\@begintheorem| and |\@opargtheorem| should be called
%    up.  The name of this macro---as is already known---has the form
%    |\th@|\meta{theorem\ style}; therefore, it must be called by
%    \begin{macrocode}
      \csname th@\the \theorem@style
%    \end{macrocode}
%    In addition the default theorem font should be changeable, i.e.\
%    we have to insert the contents of |\theorem@bodyfont|. For
%    that reason, we expand even further, beyond |\endcsname|,
%    and thus insert the contents of the \textsf{token} register
%    |\theorem@bodyfont| in the replacement text.
%    \begin{macrocode}
            \expandafter \endcsname \the \theorem@bodyfont
%    \end{macrocode}
%    Now it is time to call the macro |\@thm| which takes over
%    the further processing.  It has two arguments: the current
%    counter name (in our example, \texttt{Lem}), and the text of the
%    label (in our example, \texttt{Lemma}).
%    \begin{macrocode}
     \@thm{#1}{#2}}%
%    \end{macrocode}
%    With this, the `sub-definition' is complete.  The macro
%    |\@endtheorem| ends a theorem environment and is, so far, nothing
%    but an |\endtrivlist|.  (Hence it is defined globally, and not
%    within the theorem styles.\footnote{This has to be changed as
%    soon as theorem styles that change \texttt{\bslash @endtheorem}
%    exist.  In such a case, all existing styles must be changed as
%    well since they will have to reset the macro.}) Therefore, we can
%    set it equivalent to the macro that ends the theorem set (in our
%    example, |\endLem|).  However, if some day theorem styles exist
%    that do change |\@endtheorem|, we would have to use the
%    commented-out line instead.
%    \begin{macrocode}
    \global \expandafter \let \csname end#1\endcsname \@endtheorem
%   \global\@namedef{end#1}{\@endtheorem}%
%    \end{macrocode}
%    With these commands all the required definitions are employed,
%    unless the test |\@ifdefinable| has failed.  Therefore, we
%    end the second argument of this macro and with it the definition
%    of |\@xnthm|.
%    \begin{macrocode}
   }}
%    \end{macrocode}
% \end{macro}
%
% \begin{macro}{\@ynthm}
%    The definition of |\@ynthm| is completely analogous. In
%    this case the new counter that is defined is not reset within
%    another counter; thus the definition of |\the...|
%    is simplified:
%    \begin{macrocode}
\gdef\@ynthm#1#2{\expandafter\@ifdefinable\csname #1\endcsname
   {\@definecounter{#1}%
    \expandafter\xdef\csname the#1\endcsname{\@thmcounter{#1}}%
%    \end{macrocode}
%    The rest of the definition corresponds literally to that of
%    |\@xnthm|:
%    \begin{macrocode}
    \def\@tempa{\global\@namedef{#1}}\expandafter \@tempa
     \expandafter{\csname th@\the \theorem@style \expandafter
     \endcsname \the\theorem@bodyfont \@thm{#1}{#2}}%
    \global \expandafter \let \csname end#1\endcsname \@endtheorem}}
%    \end{macrocode}
% \end{macro}
%
% \begin{macro}{\@othm}
%    The definition of |\@othm| does not contain anything new.
%    \begin{macrocode}
\gdef\@othm#1[#2]#3{%
%    \end{macrocode}
% \changes{v2.2b}{95/10/07}{Build in check for counter defined}
%    We do not define a new counter but instead use one that has
%    already been defined.  Thus the only definition we need is that
%    of this pseudo-counter (i.e.\ |\the|\meta{env.\ name}).
%    First we check if |#2| corresponds to a known counter name.
%    \begin{macrocode}
  \expandafter\ifx\csname c@#2\endcsname\relax
   \@nocounterr{#2}%
  \else
   \expandafter\@ifdefinable\csname #1\endcsname
   {\expandafter \xdef \csname the#1\endcsname
     {\expandafter \noexpand \csname the#2\endcsname}%
%    \end{macrocode}
%    All other parts of the definition can be adopted from
%    |\@xnthm|. We have to remember, though, that in this case
%    the name of the current counter and the theorem label have moved
%    to the second and third arguments.
%    \begin{macrocode}
    \def\@tempa{\global\@namedef{#1}}\expandafter \@tempa
     \expandafter{\csname th@\the \theorem@style \expandafter
     \endcsname \the\theorem@bodyfont \@thm{#2}{#3}}%
    \global \expandafter \let \csname end#1\endcsname \@endtheorem}%
  \fi}
%    \end{macrocode}
% \end{macro}
%
% \subsection{Macros that are employed in a {\envfont theorem}
%    environment}
%
% \begin{macro}{\@thm}
%    The macro |\@thm| has to increase the current counter.  Then,
%    depending on whether the environment has (or does not have) an
%    optional argument, it has to branch into either |\@begintheorem|
%    or |\@opargtheorem|.
%    \begin{macrocode}
\gdef\@thm#1#2{\refstepcounter{#1}%
%    \end{macrocode}
%    Now we start a {\envfont trivlist} environment, and give
%    |\@topsep| and |\@topsepadd| the values of the \textsf{skip}
%    registers |\theorempreskipamount| and |\theorempostskipamount|.
%    The value in |\@topsep| is the vertical space that is inserted by
%    the first (and only) |\item| in our |\trivlist| whilst
%    |\@topsepadd| is inserted by |\@endparenv| at the end of that
%    {\envfont trivlist} environment.  By using these registers, we
%    obtain the desired space around a {\envfont theorem} environment.
%    \begin{macrocode}
   \trivlist
   \@topsep \theorempreskipamount               % used by first \item
   \@topsepadd \theorempostskipamount           % used by \@endparenv
%    \end{macrocode}
%    Now we have to test whether an optional argument has been given.
% \changes{v2.0d}{89/07/18}{Forgotten percent added.}
%    \begin{macrocode}
   \@ifnextchar [%
%    \end{macrocode}
%    If there is an optional argument, we will call |\@ythm|, and
%    move the arguments read back into the input stream.
%    \begin{macrocode}
   {\@ythm{#1}{#2}}%
%    \end{macrocode}
%    If not, we call |\@begintheorem|. Its first argument is the
%    name of the theorem set (hence the second argument of
%    |\@thm|). Its second argument is the macro that produces the
%    current number.
%    \begin{macrocode}
   {\@begintheorem{#2}{\csname the#1\endcsname}\ignorespaces}}
%    \end{macrocode}
% \end{macro}
%
% \begin{macro}{\@xthm}
% \begin{macro}{\@ythm}
%    Both these macros were originally called by |\@thm|.  We do
%    not need |\@xthm| anymore, hence we reset it to
%    |\relax|. The definition of |\@ythm| has not changed at
%    all from its definition in \LaTeX. In order to make the macros
%    easier to understand, we will
%    nevertheless present it (commented out).
%    \begin{macrocode}
\global\let\@xthm\relax
% \def\@ythm#1#2[#3]{\@opargbegintheorem{#2}{\csname
%           the#1\endcsname}{#3}\ignorespaces}
%</package>
%    \end{macrocode}
%    The primitive |\ignorespaces| in |\@ythm| and
%    |\@thm| is needed to remove the spaces between the
%    |\begin{...}| and the actual text.
% \end{macro}
% \end{macro}
%
%
% \subsection{Definition of the theorem substyles}
%
% As already pointed out, the theorem substyles, defined below, are
% only loaded when necessary.  Note that all these substyles, except
% \texttt{plain}, have |\slshape| as the default body font.
%
% \subsubsection{The \texttt{plain} style}
%
% \label{dok:th@plain} As the following macros use |@|, we have
% to locally set the |\catcode| of this symbol to ``letter''.
% This happens within a group, so that we do not have to worry about
% which |\catcode| that symbol had before.
%    \begin{macrocode}
%<*thp>
\begingroup \makeatletter
%    \end{macrocode}
% Since we are now within a group, we must make all definitions
% globally.  First we make sure that \texttt{theorem.sty} is loaded.
% This will allow us to use this file as a document style option
% without having to call \texttt{theorem} itself as an option.  At the
% same time, we assure that at least version 2 is loaded, since
% |\theorem@style| was not defined in earlier versions.
%    \begin{macrocode}
\@ifundefined{theorem@style}{\input{theorem.sty}}{}
\ProvidesFile{thp.sty}
             [\FMithmInfo]
%    \end{macrocode}
%
%
% \begin{macro}{\th@plain}
% \begin{macro}{\@begintheorem}
% \begin{macro}{\@opargbegintheorem}
%    |\theoremstyle{plain}| corresponds to the original
%    definition, except that the distances to the surrounding text are
%    determined by the parameters |\theorempreskipamount| and
%    |\theorempostskipamount|.
%    First we set the default body font.
%    \begin{macrocode}
\gdef\th@plain{\normalfont\itshape
%    \end{macrocode}
%    Then we define |\@begintheorem| and
%    |\@opargbegintheorem|. These two macros define how the
%    header of a theorem is typeset.  |\@opargbegintheorem| will
%    be called if a \textsf{theorem} environment with an optional
%    argument is encountered; otherwise, the header is constructed by
%    calling |\@begintheorem|.  If one of these macros is
%    executed, we are within a \textsf{trivlist} environment started by
%    |\@thm|.  So the theorem header is produced with an
%    |\item| command.
%
%    Instead of specifying the header font directly, all standard
%    theorem styles use the |\theorem@headerfont| macro to allow
%    customization.  The extra space (|\labelsep|) is necessary
%    because of problems in the \textsf{trivlist} environment.
%    \begin{macrocode}
  \def\@begintheorem##1##2{%
        \item[\hskip\labelsep \theorem@headerfont ##1\ ##2]}%
%    \end{macrocode}
%    The definition of |\@opargbegintheorem| is completely
%    analogous. The only difference is the fact that there exists a
%    third argument (which is the optional parameter of the
%    environment and contains additional information about the
%    theorem).  Customarily we enclose it in parentheses.
%    \begin{macrocode}
\def\@opargbegintheorem##1##2##3{%
   \item[\hskip\labelsep \theorem@headerfont ##1\ ##2\ (##3)]}}
%    \end{macrocode}
% \end{macro}
% \end{macro}
% \end{macro}
%
% \noindent
% We conclude with an |\endgroup| to restore the
% |\catcode| of |@|.
%    \begin{macrocode}
\endgroup
%</thp>
%    \end{macrocode}
%
%
% \subsubsection{The \texttt{break} style}
%
% This style option is stored in the file |thb.sty|.
% For the next two lines see the documentation for |\th@plain| on
% page~\pageref{dok:th@plain}.
%    \begin{macrocode}
%<*thb>
\begingroup \makeatletter
\@ifundefined{theorem@style}{\input{theorem.sty}}{}
\ProvidesFile{thb.sty}
             [\FMithmInfo]
%    \end{macrocode}
%
% \begin{macro}{\th@break}
%    |\theoremstyle{break}| produces a line break after the name
%    of the theorem.  The font is |\slshape|.  Hence, we define
%    |\th@break| as follows:
%    \begin{macrocode}
\gdef\th@break{\normalfont\slshape
  \def\@begintheorem##1##2{\item[%
%    \end{macrocode}
%    We run into the following problem: it is not possible to create
%    the header with |\item[|\meta{title}|]| and then start
%    a new line by, for example, |\mbox{}\\|. Such a definition
%    will fail whenever a list environment follows immediately.  With
%    the above construction, the |\mbox{}| causes the switch
%    |@inlabel| (cf.\ definition of |\list| and
%    |\trivlist| in \cite{latex:source}) to be set to \texttt{false}
%    and so the following list will insert additional vertical
%    space (|\topskip|). This is quite annoying. Therefore, we
%    create the line break within the |\item|. In order to ensure
%    that the text will begin at the proper position in the following
%    line, we simply pretend that the label does not take any
%    room.\footnote{This will lead to problems whenever very high
%    symbols occurring in the line tower into the
%    heading. So, something else has to be done here sometime.}
%    \begin{macrocode}
       \rlap{\vbox{\hbox{\hskip \labelsep\theorem@headerfont ##1\ ##2}%
                   \hbox{\strut}}}]}%
%    \end{macrocode}
%    Again, the definition of |\@opargbegintheorem| is
%    completely analogous.
%    \begin{macrocode}
\def\@opargbegintheorem##1##2##3{%
  \item[\rlap{\vbox{\hbox{\hskip \labelsep \theorem@headerfont
                     ##1\ ##2\ (##3)}%
                    \hbox{\strut}}}]}}
\endgroup
%</thb>
%    \end{macrocode}
% \end{macro}
%
%
% \subsubsection{The \texttt{changebreak} style}
%
%    \begin{macrocode}
% This style option is stored in the file |thcb.sty|.
%    \begin{macrocode}
%<*thcb>
\begingroup \makeatletter
\@ifundefined{theorem@style}{\input{theorem.sty}}{}
\ProvidesFile{thcb.sty}
             [\FMithmInfo]
%    \end{macrocode}
%
% \begin{macro}{\th@changebreak}
%    The \texttt{change-break} theorem style is like \texttt{break}
%    but with interchange of theorem name and theorem number.  Thus we
%    define |\th@changebreak| as follows:
%    \begin{macrocode}
\gdef\th@changebreak{\normalfont\slshape
  \def\@begintheorem##1##2{\item
       [\rlap{\vbox{\hbox{\hskip\labelsep \theorem@headerfont ##2\ ##1}%
                    \hbox{\strut}}}]}%
\def\@opargbegintheorem##1##2##3{%
  \item[\rlap{\vbox{\hbox{\hskip\labelsep \theorem@headerfont
                     ##2\ ##1\ (##3)}%
                    \hbox{\strut}}}]}}
\endgroup
%</thcb>
%    \end{macrocode}
% \end{macro}
%
%
% \subsubsection{The \texttt{change} style}
%
% This style option is stored in the file |thc.sty|.
%    \begin{macrocode}
%<*thc>
\begingroup \makeatletter
\@ifundefined{theorem@style}{\input{theorem.sty}}{}
\ProvidesFile{thc.sty}
             [\FMithmInfo]
%    \end{macrocode}
%
% \begin{macro}{\th@change}
%    The \texttt{change} theorem style corresponds to the
%    \texttt{change break} style without a linebreak after the header.
%    To say it in another way, it's the same as the \texttt{plain}
%    style but with number and name interchanged and |\slshape| as the
%    default font.
%    \begin{macrocode}
\gdef\th@change{\normalfont\slshape
  \def\@begintheorem##1##2{\item
       [\hskip\labelsep \theorem@headerfont ##2\ ##1]}%
\def\@opargbegintheorem##1##2##3{%
  \item[\hskip\labelsep \theorem@headerfont ##2\ ##1\ (##3)]}}
\endgroup
%</thc>
%    \end{macrocode}
% \end{macro}
%
%
% \subsubsection{The \texttt{marginbreak} style}
%
% This style option is the one used most often at Mainz.  It is saved
% in the file |thmb.sty|.
%    \begin{macrocode}
%<*thmb>
\begingroup \makeatletter
\@ifundefined{theorem@style}{\input{theorem.sty}}{}
\ProvidesFile{thmb.sty}
             [\FMithmInfo]
%    \end{macrocode}
%
% \begin{macro}{\th@marginbreak}
%    The \texttt{margin break} style is nearly the same as the
%    \texttt{change break} style. The only difference is the placement
%    of the theorem number.  We use |\llap| to place it in the left
%    margin.
%
%    In this style |\labelsep| denotes the separation between the
%    number and the text.
%    \begin{macrocode}
\gdef\th@marginbreak{\normalfont\slshape
  \def\@begintheorem##1##2{\item
       [\rlap{\vbox{\theorem@headerfont
                    \hbox{\llap{##2}\hskip\labelsep ##1}%
                    \hbox{\strut}}}]}%
\def\@opargbegintheorem##1##2##3{%
  \item[\rlap{\vbox{\theorem@headerfont
                    \hbox{\llap{##2}\hskip\labelsep ##1\ (##3)}%
                    \hbox{\strut}}}]}}
\endgroup
%</thmb>
%    \end{macrocode}
% \changes{v2.1c}{90/02/17}{`endgroup outside macrocode env.}
% \end{macro}
%
%
%
% \subsubsection{The \texttt{margin} style}
%
% This style option is stored in the file |thm.sty|.
%    \begin{macrocode}
%<*thm>
\begingroup \makeatletter
\@ifundefined{theorem@style}{\input{theorem.sty}}{}
\ProvidesFile{thm.sty}
             [\FMithmInfo]
%    \end{macrocode}
%
% \begin{macro}{\th@margin}
%    Again this is only a variant of the theorem styles described
%    above without any new ideas.
%    \begin{macrocode}
\gdef\th@margin{\normalfont\slshape
  \def\@begintheorem##1##2{\item
       [\theorem@headerfont \llap{##2}\hskip\labelsep ##1]}%
\def\@opargbegintheorem##1##2##3{%
  \item[\theorem@headerfont \llap{##2}\hskip\labelsep ##1\ (##3)]}}
\endgroup
%</thm>
%    \end{macrocode}
% \end{macro}
%
%
%
%
% \subsection{Final Definitions}
%
% \begin{macro}{\theorempreskipamount}
% \begin{macro}{\theorempostskipamount}
%    The \textsf{skip} parameters that regulate the vertical empty space
%    before and after the {\envfont theorem} environment have
%    to be allocated as well.
% \changes{v2.0f}{89/07/20}{Unnecessary `global' removed.}
%    \begin{macrocode}
%<*package>
\newskip\theorempreskipamount
\newskip\theorempostskipamount
%    \end{macrocode}
%    Since we have used the same values for all theorem sets, we now can
%    assign them.
%    \begin{macrocode}
\global\setlength\theorempreskipamount{12pt plus 5pt minus 3pt}
\global\setlength\theorempostskipamount{8pt plus 3pt minus 1.5pt}
%    \end{macrocode}
% \end{macro}
% \end{macro}
%
%
% \begin{macro}{\@endtheorem}
%    The same holds for the macro |\@endtheorem|, which ends a
%    {\envfont theorem} environment. Since it is the same for all
%    theorem sets, it is removed from the macros |\th@|\meta{style}.
%    It simply ends the {\envfont trivlist} environment, which was
%    begun in |\@thm|.
%    \begin{macrocode}
\global\let\@endtheorem=\endtrivlist
%    \end{macrocode}
% \end{macro}
%
%
%
% \begin{macro}{\@preamblecmds}
%    All macros defined above are to be used only in the preamble.
%    Therefore, we insert them in |\@preamblecmds| which will disable
%    them at begin document. This is done by the internal \LaTeXe{}
%    command |\@onlypreamble|.
% \changes{v2.0g}{89/09/19}{`theoremheaderfont and `theorembodyfont
%    added.}
% \changes{v2.2a}{94/02/03}{Use \LaTeXe{} interface}
%    \begin{macrocode}
\@onlypreamble\@xnthm
\@onlypreamble\@ynthm
\@onlypreamble\@othm
\@onlypreamble\newtheorem
\@onlypreamble\theoremstyle
\@onlypreamble\theorembodyfont
\@onlypreamble\theoremheaderfont
%    \end{macrocode}
% \end{macro}
%
% Finally we declare the \texttt{plain} theorem style to be the default.
% \changes{v2.1b}{90/06/04}{Fixed missing `item by declaring default
%    style.}
%    \begin{macrocode}
\theoremstyle{plain}
%</package>
%    \end{macrocode}
%
% \Finale

\endinput
}{}
\ProvidesFile{thb.sty}
             [\FMithmInfo]
%    \end{macrocode}
%
% \begin{macro}{\th@break}
%    |\theoremstyle{break}| produces a line break after the name
%    of the theorem.  The font is |\slshape|.  Hence, we define
%    |\th@break| as follows:
%    \begin{macrocode}
\gdef\th@break{\normalfont\slshape
  \def\@begintheorem##1##2{\item[%
%    \end{macrocode}
%    We run into the following problem: it is not possible to create
%    the header with |\item[|\meta{title}|]| and then start
%    a new line by, for example, |\mbox{}\\|. Such a definition
%    will fail whenever a list environment follows immediately.  With
%    the above construction, the |\mbox{}| causes the switch
%    |@inlabel| (cf.\ definition of |\list| and
%    |\trivlist| in \cite{latex:source}) to be set to \texttt{false}
%    and so the following list will insert additional vertical
%    space (|\topskip|). This is quite annoying. Therefore, we
%    create the line break within the |\item|. In order to ensure
%    that the text will begin at the proper position in the following
%    line, we simply pretend that the label does not take any
%    room.\footnote{This will lead to problems whenever very high
%    symbols occurring in the line tower into the
%    heading. So, something else has to be done here sometime.}
%    \begin{macrocode}
       \rlap{\vbox{\hbox{\hskip \labelsep\theorem@headerfont ##1\ ##2}%
                   \hbox{\strut}}}]}%
%    \end{macrocode}
%    Again, the definition of |\@opargbegintheorem| is
%    completely analogous.
%    \begin{macrocode}
\def\@opargbegintheorem##1##2##3{%
  \item[\rlap{\vbox{\hbox{\hskip \labelsep \theorem@headerfont
                     ##1\ ##2\ (##3)}%
                    \hbox{\strut}}}]}}
\endgroup
%</thb>
%    \end{macrocode}
% \end{macro}
%
%
% \subsubsection{The \texttt{changebreak} style}
%
%    \begin{macrocode}
% This style option is stored in the file |thcb.sty|.
%    \begin{macrocode}
%<*thcb>
\begingroup \makeatletter
\@ifundefined{theorem@style}{% \iffalse meta-comment
%
% Copyright (C) 1993-2024
%
% The LaTeX Project and any individual authors listed elsewhere
% in this file.
%
% This file is part of the Standard LaTeX `Tools Bundle'.
% -------------------------------------------------------
%
% It may be distributed and/or modified under the
% conditions of the LaTeX Project Public License, either version 1.3c
% of this license or (at your option) any later version.
% The latest version of this license is in
%    https://www.latex-project.org/lppl.txt
% and version 1.3c or later is part of all distributions of LaTeX
% version 2005/12/01 or later.
%
% The list of all files belonging to the LaTeX `Tools Bundle' is
% given in the file `manifest.txt'.
%
% \fi
%
%%
%
%
\def\FMithmInfo{2023/07/05 v2.2c Theorem extension package (FMi)}
%
% \ProvidesFile{theorem.dtx}[\FMithmInfo]
%\iffalse   % this is a METACOMMENT !
%%
%% Package `theorem' to use with LaTeX2e
%% Copyright (C) 1989-1995 Frank Mittelbach, all rights reserved.
%<+package>\@ifundefined{theorem@style}{}{\endinput}
%<+package>\NeedsTeXFormat{LaTeX2e}[1995/06/01]
%<+package>\ProvidesPackage{theorem}
%<+package>         [\FMithmInfo]
%
% \fi
%
% \changes{v1.0}{87/05/07}{First draft.}
% \changes{v1.0a}{87/05/07}{`thistheoremstyle' added.}
% \changes{v1.1a}{87/07/29}{New concept: redefine the internal
%                          \protect\LaTeX{} commands `@xnthm' etc.}
% \changes{v1.1b}{87/07/31}{`theoremfont' used to specify font.}
% \changes{v1.2a}{88/01/08}{`@endtheorem' now globally defined.}
% \changes{v1.2b}{88/01/11}{`newtheorem' only allowed in the preamble.}
% \changes{v1.3a}{88/02/14}{`theoremfont' now a toks register.}
%
% \changes{v2.0a}{88/11/16}{New implementation and new documentation.}
% \changes{v2.0a}{88/11/16}{No layout changes but internal modifications
%                           old substyles won't work.}
% \changes{v2.0b}{89/01/30}{`global' used for all definitions.}
% \changes{v2.0c}{89/06/15}{Documentation translated to English.
%                            Thanks to LLL and Nina}
% \changes{v2.1a}{90/02/15}{Support for new font selection.}
% \changes{v2.2a}{94/02/03}{Cleanup for LaTeX2e}
%
% \DoNotIndex{\@endparenv,\@ifdefinable,\@ifnextchar,\@ifundefined}
% \DoNotIndex{\@namedef,\@nameuse,\@spaces,\@tempa,\@warning}
% \DoNotIndex{\ }
% \DoNotIndex{\begingroup,\bfseries}
% \DoNotIndex{\csname}
% \DoNotIndex{\def,\do,\docdate}
% \DoNotIndex{\endcsname,\endgroup,\endinput,\endtrivlist,\expandafter}
% \DoNotIndex{\filedate,\fileversion}
% \DoNotIndex{\gdef,\global}
% \DoNotIndex{\hbox,\hskip}
% \DoNotIndex{\ignorespaces,\input,\itshape,\item}
% \DoNotIndex{\labelsep,\let,\llap}
% \DoNotIndex{\makeatletter}
% \DoNotIndex{\noexpand}
% \DoNotIndex{\refstepcounter,\relax,\rlap}
% \DoNotIndex{\setlength,\slshape,\space,\string,\strut}
% \DoNotIndex{\the,\trivlist,\typeout}
% \DoNotIndex{\vbox}
% \DoNotIndex{\xdef}
%
% \def\deflabel#1{\ttfamily #1\hfill}
% \def\deflist#1{\begin{list}{}{\settowidth\labelwidth{\ttfamily #1}%
%                               \setlength\leftmargin\labelwidth
%                               \addtolength\leftmargin\labelsep
%                               \let\makelabel\deflabel}}
% \def\enddeflist{\end{list}}
% \def\envfont{\normalfont\ttfamily}
%
% \MakeShortVerb{\|}
%
% \newcommand{\TUB}{{\sl TUGboat\/}}
%
% \GetFileInfo{theorem.dtx}
% \title{An Extension of the
%        \LaTeX{} theorem environment\thanks{This file
%        has version number \fileversion, last
%        revised \filedate.}}
% \author{Frank Mittelbach}
%
% \MaintainedByLaTeXTeam{tools}
% \maketitle
%
% \begin{abstract}
%    The macros described in this paper yield an extension of the
%    \LaTeX{} theorem mechanism. It is designed to satisfy the
%    different requirements of various journals. Thus, the
%    layout of the ``theorems'' can be manipulated by determining a
%    ``style''. This article describes not only the use, but
%    also the definition, of the necessary macros.
% \end{abstract}
%
% \section*{Preface to version 2.2}
%
% For \LaTeXe{} this package did not need any fundamental changes. I
% only modified the messages generated so that theorem layout styles
% will show up with the |\listfiles| command and cleaned the section
% on the New Font Selection Scheme since this is now included in
% \LaTeX.
%
% \section*{Preface to version 2.1}
%
% This version is identical to 2.0g described in \TUB\ 10\#3 except
% for some internal defaults which are now set depending on the used
% font selection scheme.
%
% This was done to avoid unpleasant surprises if the new font selection
% scheme is in force. For further details see
% section~\ref{sec:newfonts} and \cite{bk:GMS94}.
%
% \section{Introduction}
%
% For our purposes here, ``theorems'' are labelled enunciations,
% often set off from the main text by extra space and a font change.
% Theorems, corollaries, conjectures, definitions, and remarks
% are all instances of ``theorems''.  The ``header'' of these structures
% is composed of a label (such as \textsc{Theorem} or \textsc{Remark})
% and a number which serializes an item in the sequence of items
% with the same label.
%
% Shortly after the introduction of \LaTeX{} at the Fachbereich
% Mathematik in Mainz, the desire to manipulate the layout of
% ``theorems'' arose.  In Mainz, the following two conventions came
% into general use:
% \begin{enumerate}
%   \item The number of the theorem is shown in the margin.
%   \item There is a line break at
%         the end of the theorem header.
% \end{enumerate}
%
% Additionally, some journals require different formats which depend on
% the ``sort of theorem'': e.g.\ often remarks and definitions are set
% in |\upshape|, while |\itshape| is employed for main theorems.
%
% Confronted with these requirements, a theorem environment was
% developed in Mainz which allows separate determination of the
% layout of the ``theorems sets'', comparable to |\pagestyle|.
%
%
% \section{The user interface}
% \subsection{Defining new theorem sets}
%
% \DescribeMacro\newtheorem
% As in the original \LaTeX{} version, the command |\newtheorem|
% defines a new ``theorem set'' or ``theorem-like structure''.
% Two required arguments name the new environment and give the text
% to be typeset with each instance of the new ``set'', while
% an optional argument determines how
% the ``set'' is enumerated:
% \begin{description}
%    \item[\ttfamily \bslash newtheorem\{foo\}\{bar\}]
%       The theorem set {\envfont foo} (whose name is \texttt{bar})
%       uses its own counter.
%    \item[\ttfamily \bslash newtheorem\{foo2\}{[foo]}\{bar2\}]
%       The theorem set {\envfont foo2} (printed name \texttt{bar2})
%       uses the same counter as the theorem set \texttt{foo}.
%    \item[\ttfamily \bslash newtheorem\{foo3\}\{bar3\}{[section]}]
%       The theorem set {\envfont foo3} (printed name \texttt{bar3}) is
%       enumerated within the counter \texttt{section}, i.e.\ with every
%       new |\section| the enumeration begins again with 1, and
%       the enumeration is composed from the section-number and the
%       theorem counter itself.
% \end{description}
%
% \DescribeMacro\theoremstyle
% Additionally, the command |\theoremstyle| can define the layout
% of various, or all, theorem sets.  It should be noted that any
% theorem set defined by |\newtheorem| is typeset in the
% |\theoremstyle| that is current at the time of the definition.
% Thus, the following
% \begin{verbatim}
%    \theoremstyle{break}        \newtheorem{Cor}{Corollary}
%    \theoremstyle{plain}        \newtheorem{Exa}{Example}[section]
%\end{verbatim}                             ^^A No blank at line start
% leads to the result that the set {\envfont Cor} is formatted in the
% style \texttt{break}, while the set {\envfont Exa} and all the
% following ones are formatted in the style \texttt{plain}, unless
% another |\theoremstyle| follows.  Since the definitions installed by
% |\newtheorem| are global, one also can limit |\theoremstyle| locally
% by grouping braces.
%
% \DescribeMacro\theorembodyfont
% The choice of the font for the theorem body is completely
% independent of the chosen |\theoremstyle|; this has proven to
% be very advantageous.  For example,
% \begin{verbatim}
%   {\theorembodyfont{\upshape}          \newtheorem{Rem}{Remark}}
%\end{verbatim}
% defines a theorem set {\envfont Rem}, which will be set in
% |\upshape| in the current layout (which in our example is
% \texttt{plain}).  As with |\theoremstyle|, the |\theorembodyfont|
% chosen is that current at the time of |\newtheorem|.  If
% |\theorembodyfont| is not specified or one defines
% |\theorembodyfont{}|, then the font used will be that defined by the
% |\theoremstyle|.
%
% \DescribeMacro\theoremheaderfont It is also possible to customize
% the font used for the theorem headers.  This is, however, a global
% declaration, and therefore there should be at most one
% |\theoremheaderfont| declaration in the preamble.\footnote{If it is
% actually necessary to have different header fonts, one has to define
% new theorem styles (substituting the desired font) or specify the
% information directly in the \texttt{\bslash newtheorem} declaration
% (the unclean variant).}
%
% \DescribeMacro\theorempreskipamount
% \DescribeMacro\theorempostskipamount Two additional parameters
% affect the vertical space around the theorem environments:
% |\theorempreskipamount| and |\theorempostskipamount| define,
% respectively, the spacing before and after such an environment.
% These parameters apply for all theorem sets and can be manipulated
% with the ordinary length macros.  They are rubber lengths
% (`\textsf{skips}'), and therefore can contain \texttt{plus} and
% \texttt{minus} parts.
%
% Since the definition of theorem sets should---most sensibly---be
% placed in the preamble, we only allow installation there.
% It is therefore possible to release the memory used here after
% |\begin{document}|, in order to make room for other
% applications.
%
%
% \subsection{Existing theorem styles}
%
% The following theorem styles exist to date:
% \begin{deflist}{marginbreak}
%    \item[plain]
%       This theorem style emulates the original \LaTeX{} definition,
%       except that additionally the parameters
%       |\theorem...skipamount| are used.
%    \item[break]
%       In this style, the theorem header is followed by a line
%       break.
%    \item[marginbreak]
%       The theorem number is set in the margin, and there is a line
%       break as in \texttt{break}.
%    \item[changebreak]
%       Like \texttt{break}, but with header number and text
%       interchanged.
%    \item[change]
%       Header number and text are interchanged, without a line break.
%    \item[margin]
%       The number is set in the left margin, without a line break.
% \end{deflist}
% All styles (except \texttt{plain}) select |\slshape| as the default
% |\theorembodyfont|.
%
%
% \subsection{Examples}
%
% Given the above theorem sets {\envfont Cor}, {\envfont Exa}
% and {\envfont Rem},
% suppose that the preamble also contains the declarations:
% \begin{verbatim}
%    \theoremstyle{marginbreak}   \newtheorem{Lem}[Cor]{Lemma}
%    \theoremstyle{change}
%    \theorembodyfont{\itshape}        \newtheorem{Def}[Cor]{Definition}
%
%    \theoremheaderfont{\scshape}
%\end{verbatim}
% Then the following are some typical examples of the typeset output
% resulting from their use.
%
% \begin{Cor}
%    This is a sentence typeset in the theorem environment {\envfont
%    Cor}.
% \end{Cor}
%
% \begin{Exa}
%    This is a sentence typeset in the theorem environment {\envfont
%    Exa}.
% \end{Exa}
%
% \begin{Rem}
%    This is a sentence typeset in the theorem environment {\envfont
%    Rem}.
% \end{Rem}
%
% \begin{Lem}[Ben User]
%    This is a sentence typeset in the theorem environment {\envfont
%    Lem}.
% \end{Lem}
%
% \begin{Def}[Very impressive Definition]
%    This is a sentence typeset in the theorem environment {\envfont
%    Def}.
% \end{Def}
%
% The last two examples show the effect of the optional argument to
% a theorem environment (it is the text typeset in parentheses).
%
%
% \section{Special Considerations}
% \label{sec:newfonts}
%
% Theoremheader and body are implemented as a unit. This means that
% the |\theoremheaderfont|  will inherit characteristics of the
% |\theorembodyfont| in \LaTeXe. Thus, if for example
% |\theorembodyfont| is |\itshape| and |\theoremheaderfont|
% is |\bfseries| the font selected for the header will have the
% characteristics `bold extended italic'. If this is not desired one
% should set the |\theoremheaderfont| to something like
% \begin{verbatim}
%      \theoremheaderfont{\normalfont\bfseries}
%\end{verbatim}
% i.e.\ supplying all necessary font information explicitly.
%
%
%
% \section{Acknowledgements}
%
% The publication of this set of macros was only possible with the
% help of Christina Busse (translating the manuscript into English),
% Joachim Pense (playing the r\^ole of typist), Chris Rowley
% (looking everything over) and many others providing useful
% suggestions.
%
% \MaybeStop{%
%    \begin{thebibliography}{9}
%       \bibitem{bk:GMS94} \textsc{M.~Goossens}, \textsc{F.~Mittelbach}
%         and \textsc{A.~Samarin}.
%         \newblock The \LaTeX{} Companion.
%         \newblock
%         Addison-Wesley, Reading, Massachusetts, 1994.
%       \bibitem{latex:source} \textsc{Lamport, Leslie.} \newblock
%                               \texttt{latex.tex}, version 2.09, date
%                               Feb. 1990.
%    \end{thebibliography}
%    \PrintIndex
%    \PrintChanges}
%
%
% \section{The documentation driver file}
%
% The next bit of code contains the documentation driver file for
% \TeX{}, i.e., the file that will produce the documentation you are
% currently reading. It will be extracted from this file by the
% \texttt{docstrip} program. Since it is the first code in the file
% one can alternatively process this file directly with \LaTeXe{} to
% obtain the documentation.
%
%    \begin{macrocode}
%<*driver>
\documentclass{ltxdoc}

\usepackage{theorem}

%<+driver>% The next few lines define theorem sets which are used
%<+driver>% in the example section of the documentation.

\theoremstyle{break}       \newtheorem{Cor}{Corollary}
\theoremstyle{plain}       \newtheorem{Exa}{Example}[section]
{\theorembodyfont{\upshape}\newtheorem{Rem}{Remark}}
\theoremstyle{marginbreak} \newtheorem{Lem}[Cor]{Lemma}
\theoremstyle{change}
\theorembodyfont{\itshape} \newtheorem{Def}[Cor]{Definition}

\theoremheaderfont{\scshape}

\RecordChanges

\begin{document}
   \DocInput{theorem.dtx}
\end{document}
%</driver>
%    \end{macrocode}
%
% \section{Definition of the Macros}
%
% If the file has been loaded before, we abort immediately. If not the
% package announces itself (this is actually done at the very top of
% the file---the way it is done isn't good style so don't copy it).
% \changes{v2.0e}{89/07/19}{Spaces removed from `typeout'.}
% \changes{v2.2b}{95/11/19}{Announce moved to top of file}
%    \begin{macrocode}
%<*package>
%\@ifundefined{theorem@style}{}{\endinput}
%\def\FMithmInfo{1995/11/19 v2.2b Theorem extension package (FMi)}
%\ProvidesPackage{theorem}[\FMithmInfo]
%    \end{macrocode}
%
%
% \subsection{Definition of theorem styles and fonts}
%
% All the definitions in this file are done globally to allow
% inputting this file inside a group.
%
% \begin{macro}{\theoremstyle}
%    Before a theorem style can be installed, the chosen style must be
%    known. For that reason, we must test to see that
%    |\th@|\meta{style} is known or, more precisely, that it is
%    different from |\relax|. If the style is not known then
%    |\th@plain| is used.
%    \begin{macrocode}
\gdef\theoremstyle#1{%
   \@ifundefined{th@#1}{\@warning
          {Unknown theoremstyle `#1'. Using `plain'}%
          \theorem@style{plain}}%
%    \end{macrocode}
%    We save the theorem style to be used in the token register
%    |\theorem@style|.
%    \begin{macrocode}
      {\theorem@style{#1}}%
%    \end{macrocode}
%    Now we ``evaluate'' the theorem style: this means, we call the
%    macro |\th@|\meta{style} which will activate the relevant
%    definitions which are contained in a separate file.  This is done
%    in a group to suppress changes to the current font.  This could
%    otherwise pose problems together with the new font selection
%    scheme\footnote{When I printed the original article using the new
%    font selection scheme I ended with a document with slanted
%    typefaces (text headings and all) simply because one of the
%    theorem styles used \texttt{\bslash sl} at toplevel.} if the
%    |\th@|\meta{style} is evaluated a second time.
% \changes{v2.1a}{90/02/15}{Group added around `th@..}
% \changes{v2.1a}{90/02/15}{Reseting of `..@begintheorem removed.}
%    \begin{macrocode}
      \begingroup
        \csname th@\the\theorem@style \endcsname
      \endgroup}
%    \end{macrocode}
% \end{macro}
%
%
% \begin{macro}{\@begintheorem}
% \changes{v2.1a}{90/02/15}{Relaxed at toplevel.}
% \begin{macro}{\@opargbegintheorem}
% \changes{v2.1a}{90/02/15}{Relaxed at toplevel.}
%    We reset |\@begintheorem| and |\@opargbegintheorem| to
%    |\relax| since these commands are no longer necessary at
%    toplevel. This will save a few tokens.
%    \begin{macrocode}
\global\let\@begintheorem\relax
\global\let\@opargbegintheorem\relax
%    \end{macrocode}
% \end{macro}
% \end{macro}
%
% \begin{macro}{\theorem@style}
%    Obviously the \textsf{token} register used above has to be
%    allocated. To assure the utmost compatibility with the
%    original \LaTeX{} definition, we set the default theorem style to
%    \texttt{plain}, which implements the usual \LaTeX{} convention.
% \changes{v2.0f}{89/07/20}{Unnecessary `global' removed.}
%    \begin{macrocode}
\newtoks\theorem@style
\global\theorem@style{plain}
%    \end{macrocode}
% \end{macro}
%
% \begin{macro}{\theorembodyfont}
% \begin{macro}{\theorem@bodyfont}
%    If the |\theorembodyfont| is set by the user then it should not
%    interact with the default font set in the theorem style. When the
%    new font selection is in force this may happen if, for example,
%    the default is |\itshape| and the new |\theorembodyfont| is
%    |\sffamily|. So we add a |\reset@font| command in front of the
%    user definition.
% \changes{v2.0f}{89/07/20}{Unnecessary `global' removed.}
% \changes{v2.0g}{89/09/19}{`theoremfont to `theorembodyfont.}
% \changes{v2.1c}{92/06/22}{`reset@font added and logic changed.}
% \changes{v2.1d}{92/10/28}{Reset to empty if arg is empty}
%    \begin{macrocode}
\gdef\theorembodyfont#1{%
%    \end{macrocode}
%    We check if the argument supplied is empty and if so put nothing
%    into the |\theorem@bodyfont| token register to allow for
%    |\theorembodyfont{}| as a mean of using the default of the
%    current |\theoremstyle|.
%    \begin{macrocode}
   \def\@tempa{#1}%
   \ifx\@tempa\@empty
    \theorem@bodyfont{}%
   \else
    \theorem@bodyfont{\reset@font#1}%
   \fi
   }
\newtoks\theorem@bodyfont
\global\theorem@bodyfont{}
%    \end{macrocode}
% \end{macro}
% \end{macro}
%
%
% \begin{macro}{\theoremheaderfont}
% \changes{v2.0g}{89/09/19}{Macro added.}
%    The font for the theorem headers is handled differently because
%    this definition applies to all theorem styles.
%    \begin{macrocode}
\gdef\theoremheaderfont#1{\gdef\theorem@headerfont{#1}%
%    \end{macrocode}
%    After using the macro once it is redefined to produce an error
%    message.
%    \begin{macrocode}
       \gdef\theoremheaderfont##1{%
        \typeout{\string\theoremheaderfont\space should be used
                 only once.}}}
%    \end{macrocode}
% \end{macro}
%
% \begin{macro}{\theorem@headerfont}
% \changes{v2.0g}{89/09/19}{Macro added.}
%    To set the |\theorem@headerfont| default we first test if
%    the new fontselection scheme is in force.
% \changes{v2.1a}{90/02/15}{Support for new font selection.}
%    \begin{macrocode}
\ifx\upshape\undefined
%    \end{macrocode}
%    If not we define it to expand into |\bfseries|. We don't use
%    |\let| just in case a following style option redefines this
%    macro.
%    \begin{macrocode}
\gdef\theorem@headerfont{\bfseries}
%    \end{macrocode}
%    Otherwise we reset the current shape before calling |\bfseries|.
%    \begin{macrocode}
\else \gdef\theorem@headerfont{\normalfont\bfseries}\fi
%    \end{macrocode}
% \end{macro}
%
% \begin{macro}{\th@plain}
% \begin{macro}{\th@break}
% \begin{macro}{\th@marginbreak}
% \begin{macro}{\th@changebreak}
% \begin{macro}{\th@change}
% \begin{macro}{\th@margin}
%    The different styles are defined in macros such as
%    |\th@plain|.  Since memory space is precious in
%    ``non-Big-versions'', we have to avoid offering too many unused
%    definitions.  Therefore we define these styles in separate files
%    that can be loaded on demand.  Thus the commands themselves only
%    load these files. We use |\@input@|, a \LaTeXe{} internal command
%    that ensures that the file will be listed with |\listfiles|.
% \changes{v2.2a}{94/02/02}{Use `@input@ to load theorem layout files}
%    \begin{macrocode}
\gdef\th@plain{\@input@{thp.sty}}
\gdef\th@break{\@input@{thb.sty}}
\gdef\th@marginbreak{\@input@{thmb.sty}}
\gdef\th@changebreak{\@input@{thcb.sty}}
\gdef\th@change{\@input@{thc.sty}}
\gdef\th@margin{\@input@{thm.sty}}
%    \end{macrocode}
%    This list will be expanded when new styles become
%    available.  For testing,  just append new theorem substyles
%    as document options.
% \end{macro}
% \end{macro}
% \end{macro}
% \end{macro}
% \end{macro}
% \end{macro}
%
%
%
% \subsection{Definition of a new theorem set}
%
% As already pointed out, a new theorem environment can be defined in
% three different ways:
% \begin{verbatim}
%    \newtheorem{Lem}{Lemma}
%    \newtheorem{Lem}{Lemma}[section]
%    \newtheorem{Lem}[Theorem]{Lemma}
%\end{verbatim}
% The function of the macro |\newtheorem| is to recognize these cases
% and then to branch into one of the three macros |\@ynthm|, |\@xnthm|
% or |\@othm|.  This mechanism is adopted unchanged from
% \cite{latex:source}; the essential point here is that, for example,
% in the second case, the arguments \texttt{Lem}, \texttt{Lemma} and
% \texttt{section} are passed over to the macro |\@xnthm|.
%
% We inspect this case first because the others present fewer problems,
% and thus are easily derived from this one.
%
% \begin{macro}{\@xnthm}
%    For our example arguments, the macro |\@xnthm| must fulfill
%    the following:
%    \begin{itemize}
%       \item
%          Define a new \LaTeX-counter `Lem'
%       \item
%          reset this counter within a |\section|
%       \item
%          define the macro |\theLem|
%       \item
%          define the environment macros |\Lem| and
%          |\endLem| using the current
%          |\theoremstyle| and |\theorem@bodyfont|.
%    \end{itemize}
%    Obviously, all this should happen only if the first argument
%    of |\@xnthm| (i.e.\ \texttt{Lem} in our example) is chosen so
%    as not to conflict with any previously defined commands or
%    environments.  This test is performed by the \LaTeX{} macro
%    |\@ifdefinable|.
%    \begin{macrocode}
\gdef\@xnthm#1#2[#3]{\expandafter\@ifdefinable\csname #1\endcsname
%    \end{macrocode}
%    Therefore, the first argument of |\@ifdefinable| is the
%    expansion (in the example, |\Lem|) of
%    |\csname||#1\endcsname|.  The second argument is
%    executed only if the test has been completed successfully.
%    \begin{macrocode}
   {%
%    \end{macrocode}
%    Now we define the new counter. The names of the \LaTeX{}
%    macros employed should speak for themselves:
% \changes{v2.2b}{1995/10/07}{Use \cs{@newctr}}
% \changes{v2.2c}{1995/11/23}{Use \cs{@newctr} properly.}
%    \begin{macrocode}
    \@definecounter{#1}\@newctr{#1}[#3]%
%    \end{macrocode}
%    Using |\@newctr| will give a proper error message if the counter
%    in |#3| is not defined.
%    In defining `|\theLem|' we must generate the desired macro
%    name by use of |\expandafter| and |\csname|.
%    \begin{macrocode}
    \expandafter\xdef\csname the#1\endcsname
%    \end{macrocode}
%    An |\xdef| is used in order to make the definition global,
%    and to ensure that it contains the replacement texts of
%    |\@thmcountersep| and |\@thmcounter|.\footnote{These
%    two macros can be defined by the document style.  Their default
%    values produce a `\texttt{.}' as separation and an arabic
%    representation of the number.} However, not everything should be
%    expanded. For example, it saves space to use |\thesection|
%    instead of its---at times---lengthy expansion.
%    \begin{macrocode}
      {\expandafter \noexpand \csname the#3\endcsname
       \@thmcountersep \@thmcounter{#1}}%
%    \end{macrocode}
%    Thus with the defaults of \LaTeX{}, |\theLem| would be
%    replaced by the command sequence |\thesection.\arabic{Lem}|.
%
%    We will now look at the definition of the macro which is executed
%    at the beginning of the actual environment (in our example this
%    macro is |\Lem|). It should be noted that we use an
%    ``|\expandafter| trick'' to expand only certain parts of the
%    replacement text at the time of the definition.
%    \begin{macrocode}
    \def\@tempa{\global\@namedef{#1}}%
    \expandafter \@tempa \expandafter{%
%    \end{macrocode}
%    First, the macro that contains the current definitions of
%    |\@begintheorem| and |\@opargtheorem| should be called
%    up.  The name of this macro---as is already known---has the form
%    |\th@|\meta{theorem\ style}; therefore, it must be called by
%    \begin{macrocode}
      \csname th@\the \theorem@style
%    \end{macrocode}
%    In addition the default theorem font should be changeable, i.e.\
%    we have to insert the contents of |\theorem@bodyfont|. For
%    that reason, we expand even further, beyond |\endcsname|,
%    and thus insert the contents of the \textsf{token} register
%    |\theorem@bodyfont| in the replacement text.
%    \begin{macrocode}
            \expandafter \endcsname \the \theorem@bodyfont
%    \end{macrocode}
%    Now it is time to call the macro |\@thm| which takes over
%    the further processing.  It has two arguments: the current
%    counter name (in our example, \texttt{Lem}), and the text of the
%    label (in our example, \texttt{Lemma}).
%    \begin{macrocode}
     \@thm{#1}{#2}}%
%    \end{macrocode}
%    With this, the `sub-definition' is complete.  The macro
%    |\@endtheorem| ends a theorem environment and is, so far, nothing
%    but an |\endtrivlist|.  (Hence it is defined globally, and not
%    within the theorem styles.\footnote{This has to be changed as
%    soon as theorem styles that change \texttt{\bslash @endtheorem}
%    exist.  In such a case, all existing styles must be changed as
%    well since they will have to reset the macro.}) Therefore, we can
%    set it equivalent to the macro that ends the theorem set (in our
%    example, |\endLem|).  However, if some day theorem styles exist
%    that do change |\@endtheorem|, we would have to use the
%    commented-out line instead.
%    \begin{macrocode}
    \global \expandafter \let \csname end#1\endcsname \@endtheorem
%   \global\@namedef{end#1}{\@endtheorem}%
%    \end{macrocode}
%    With these commands all the required definitions are employed,
%    unless the test |\@ifdefinable| has failed.  Therefore, we
%    end the second argument of this macro and with it the definition
%    of |\@xnthm|.
%    \begin{macrocode}
   }}
%    \end{macrocode}
% \end{macro}
%
% \begin{macro}{\@ynthm}
%    The definition of |\@ynthm| is completely analogous. In
%    this case the new counter that is defined is not reset within
%    another counter; thus the definition of |\the...|
%    is simplified:
%    \begin{macrocode}
\gdef\@ynthm#1#2{\expandafter\@ifdefinable\csname #1\endcsname
   {\@definecounter{#1}%
    \expandafter\xdef\csname the#1\endcsname{\@thmcounter{#1}}%
%    \end{macrocode}
%    The rest of the definition corresponds literally to that of
%    |\@xnthm|:
%    \begin{macrocode}
    \def\@tempa{\global\@namedef{#1}}\expandafter \@tempa
     \expandafter{\csname th@\the \theorem@style \expandafter
     \endcsname \the\theorem@bodyfont \@thm{#1}{#2}}%
    \global \expandafter \let \csname end#1\endcsname \@endtheorem}}
%    \end{macrocode}
% \end{macro}
%
% \begin{macro}{\@othm}
%    The definition of |\@othm| does not contain anything new.
%    \begin{macrocode}
\gdef\@othm#1[#2]#3{%
%    \end{macrocode}
% \changes{v2.2b}{95/10/07}{Build in check for counter defined}
%    We do not define a new counter but instead use one that has
%    already been defined.  Thus the only definition we need is that
%    of this pseudo-counter (i.e.\ |\the|\meta{env.\ name}).
%    First we check if |#2| corresponds to a known counter name.
%    \begin{macrocode}
  \expandafter\ifx\csname c@#2\endcsname\relax
   \@nocounterr{#2}%
  \else
   \expandafter\@ifdefinable\csname #1\endcsname
   {\expandafter \xdef \csname the#1\endcsname
     {\expandafter \noexpand \csname the#2\endcsname}%
%    \end{macrocode}
%    All other parts of the definition can be adopted from
%    |\@xnthm|. We have to remember, though, that in this case
%    the name of the current counter and the theorem label have moved
%    to the second and third arguments.
%    \begin{macrocode}
    \def\@tempa{\global\@namedef{#1}}\expandafter \@tempa
     \expandafter{\csname th@\the \theorem@style \expandafter
     \endcsname \the\theorem@bodyfont \@thm{#2}{#3}}%
    \global \expandafter \let \csname end#1\endcsname \@endtheorem}%
  \fi}
%    \end{macrocode}
% \end{macro}
%
% \subsection{Macros that are employed in a {\envfont theorem}
%    environment}
%
% \begin{macro}{\@thm}
%    The macro |\@thm| has to increase the current counter.  Then,
%    depending on whether the environment has (or does not have) an
%    optional argument, it has to branch into either |\@begintheorem|
%    or |\@opargtheorem|.
%    \begin{macrocode}
\gdef\@thm#1#2{\refstepcounter{#1}%
%    \end{macrocode}
%    Now we start a {\envfont trivlist} environment, and give
%    |\@topsep| and |\@topsepadd| the values of the \textsf{skip}
%    registers |\theorempreskipamount| and |\theorempostskipamount|.
%    The value in |\@topsep| is the vertical space that is inserted by
%    the first (and only) |\item| in our |\trivlist| whilst
%    |\@topsepadd| is inserted by |\@endparenv| at the end of that
%    {\envfont trivlist} environment.  By using these registers, we
%    obtain the desired space around a {\envfont theorem} environment.
%    \begin{macrocode}
   \trivlist
   \@topsep \theorempreskipamount               % used by first \item
   \@topsepadd \theorempostskipamount           % used by \@endparenv
%    \end{macrocode}
%    Now we have to test whether an optional argument has been given.
% \changes{v2.0d}{89/07/18}{Forgotten percent added.}
%    \begin{macrocode}
   \@ifnextchar [%
%    \end{macrocode}
%    If there is an optional argument, we will call |\@ythm|, and
%    move the arguments read back into the input stream.
%    \begin{macrocode}
   {\@ythm{#1}{#2}}%
%    \end{macrocode}
%    If not, we call |\@begintheorem|. Its first argument is the
%    name of the theorem set (hence the second argument of
%    |\@thm|). Its second argument is the macro that produces the
%    current number.
%    \begin{macrocode}
   {\@begintheorem{#2}{\csname the#1\endcsname}\ignorespaces}}
%    \end{macrocode}
% \end{macro}
%
% \begin{macro}{\@xthm}
% \begin{macro}{\@ythm}
%    Both these macros were originally called by |\@thm|.  We do
%    not need |\@xthm| anymore, hence we reset it to
%    |\relax|. The definition of |\@ythm| has not changed at
%    all from its definition in \LaTeX. In order to make the macros
%    easier to understand, we will
%    nevertheless present it (commented out).
%    \begin{macrocode}
\global\let\@xthm\relax
% \def\@ythm#1#2[#3]{\@opargbegintheorem{#2}{\csname
%           the#1\endcsname}{#3}\ignorespaces}
%</package>
%    \end{macrocode}
%    The primitive |\ignorespaces| in |\@ythm| and
%    |\@thm| is needed to remove the spaces between the
%    |\begin{...}| and the actual text.
% \end{macro}
% \end{macro}
%
%
% \subsection{Definition of the theorem substyles}
%
% As already pointed out, the theorem substyles, defined below, are
% only loaded when necessary.  Note that all these substyles, except
% \texttt{plain}, have |\slshape| as the default body font.
%
% \subsubsection{The \texttt{plain} style}
%
% \label{dok:th@plain} As the following macros use |@|, we have
% to locally set the |\catcode| of this symbol to ``letter''.
% This happens within a group, so that we do not have to worry about
% which |\catcode| that symbol had before.
%    \begin{macrocode}
%<*thp>
\begingroup \makeatletter
%    \end{macrocode}
% Since we are now within a group, we must make all definitions
% globally.  First we make sure that \texttt{theorem.sty} is loaded.
% This will allow us to use this file as a document style option
% without having to call \texttt{theorem} itself as an option.  At the
% same time, we assure that at least version 2 is loaded, since
% |\theorem@style| was not defined in earlier versions.
%    \begin{macrocode}
\@ifundefined{theorem@style}{\input{theorem.sty}}{}
\ProvidesFile{thp.sty}
             [\FMithmInfo]
%    \end{macrocode}
%
%
% \begin{macro}{\th@plain}
% \begin{macro}{\@begintheorem}
% \begin{macro}{\@opargbegintheorem}
%    |\theoremstyle{plain}| corresponds to the original
%    definition, except that the distances to the surrounding text are
%    determined by the parameters |\theorempreskipamount| and
%    |\theorempostskipamount|.
%    First we set the default body font.
%    \begin{macrocode}
\gdef\th@plain{\normalfont\itshape
%    \end{macrocode}
%    Then we define |\@begintheorem| and
%    |\@opargbegintheorem|. These two macros define how the
%    header of a theorem is typeset.  |\@opargbegintheorem| will
%    be called if a \textsf{theorem} environment with an optional
%    argument is encountered; otherwise, the header is constructed by
%    calling |\@begintheorem|.  If one of these macros is
%    executed, we are within a \textsf{trivlist} environment started by
%    |\@thm|.  So the theorem header is produced with an
%    |\item| command.
%
%    Instead of specifying the header font directly, all standard
%    theorem styles use the |\theorem@headerfont| macro to allow
%    customization.  The extra space (|\labelsep|) is necessary
%    because of problems in the \textsf{trivlist} environment.
%    \begin{macrocode}
  \def\@begintheorem##1##2{%
        \item[\hskip\labelsep \theorem@headerfont ##1\ ##2]}%
%    \end{macrocode}
%    The definition of |\@opargbegintheorem| is completely
%    analogous. The only difference is the fact that there exists a
%    third argument (which is the optional parameter of the
%    environment and contains additional information about the
%    theorem).  Customarily we enclose it in parentheses.
%    \begin{macrocode}
\def\@opargbegintheorem##1##2##3{%
   \item[\hskip\labelsep \theorem@headerfont ##1\ ##2\ (##3)]}}
%    \end{macrocode}
% \end{macro}
% \end{macro}
% \end{macro}
%
% \noindent
% We conclude with an |\endgroup| to restore the
% |\catcode| of |@|.
%    \begin{macrocode}
\endgroup
%</thp>
%    \end{macrocode}
%
%
% \subsubsection{The \texttt{break} style}
%
% This style option is stored in the file |thb.sty|.
% For the next two lines see the documentation for |\th@plain| on
% page~\pageref{dok:th@plain}.
%    \begin{macrocode}
%<*thb>
\begingroup \makeatletter
\@ifundefined{theorem@style}{\input{theorem.sty}}{}
\ProvidesFile{thb.sty}
             [\FMithmInfo]
%    \end{macrocode}
%
% \begin{macro}{\th@break}
%    |\theoremstyle{break}| produces a line break after the name
%    of the theorem.  The font is |\slshape|.  Hence, we define
%    |\th@break| as follows:
%    \begin{macrocode}
\gdef\th@break{\normalfont\slshape
  \def\@begintheorem##1##2{\item[%
%    \end{macrocode}
%    We run into the following problem: it is not possible to create
%    the header with |\item[|\meta{title}|]| and then start
%    a new line by, for example, |\mbox{}\\|. Such a definition
%    will fail whenever a list environment follows immediately.  With
%    the above construction, the |\mbox{}| causes the switch
%    |@inlabel| (cf.\ definition of |\list| and
%    |\trivlist| in \cite{latex:source}) to be set to \texttt{false}
%    and so the following list will insert additional vertical
%    space (|\topskip|). This is quite annoying. Therefore, we
%    create the line break within the |\item|. In order to ensure
%    that the text will begin at the proper position in the following
%    line, we simply pretend that the label does not take any
%    room.\footnote{This will lead to problems whenever very high
%    symbols occurring in the line tower into the
%    heading. So, something else has to be done here sometime.}
%    \begin{macrocode}
       \rlap{\vbox{\hbox{\hskip \labelsep\theorem@headerfont ##1\ ##2}%
                   \hbox{\strut}}}]}%
%    \end{macrocode}
%    Again, the definition of |\@opargbegintheorem| is
%    completely analogous.
%    \begin{macrocode}
\def\@opargbegintheorem##1##2##3{%
  \item[\rlap{\vbox{\hbox{\hskip \labelsep \theorem@headerfont
                     ##1\ ##2\ (##3)}%
                    \hbox{\strut}}}]}}
\endgroup
%</thb>
%    \end{macrocode}
% \end{macro}
%
%
% \subsubsection{The \texttt{changebreak} style}
%
%    \begin{macrocode}
% This style option is stored in the file |thcb.sty|.
%    \begin{macrocode}
%<*thcb>
\begingroup \makeatletter
\@ifundefined{theorem@style}{\input{theorem.sty}}{}
\ProvidesFile{thcb.sty}
             [\FMithmInfo]
%    \end{macrocode}
%
% \begin{macro}{\th@changebreak}
%    The \texttt{change-break} theorem style is like \texttt{break}
%    but with interchange of theorem name and theorem number.  Thus we
%    define |\th@changebreak| as follows:
%    \begin{macrocode}
\gdef\th@changebreak{\normalfont\slshape
  \def\@begintheorem##1##2{\item
       [\rlap{\vbox{\hbox{\hskip\labelsep \theorem@headerfont ##2\ ##1}%
                    \hbox{\strut}}}]}%
\def\@opargbegintheorem##1##2##3{%
  \item[\rlap{\vbox{\hbox{\hskip\labelsep \theorem@headerfont
                     ##2\ ##1\ (##3)}%
                    \hbox{\strut}}}]}}
\endgroup
%</thcb>
%    \end{macrocode}
% \end{macro}
%
%
% \subsubsection{The \texttt{change} style}
%
% This style option is stored in the file |thc.sty|.
%    \begin{macrocode}
%<*thc>
\begingroup \makeatletter
\@ifundefined{theorem@style}{\input{theorem.sty}}{}
\ProvidesFile{thc.sty}
             [\FMithmInfo]
%    \end{macrocode}
%
% \begin{macro}{\th@change}
%    The \texttt{change} theorem style corresponds to the
%    \texttt{change break} style without a linebreak after the header.
%    To say it in another way, it's the same as the \texttt{plain}
%    style but with number and name interchanged and |\slshape| as the
%    default font.
%    \begin{macrocode}
\gdef\th@change{\normalfont\slshape
  \def\@begintheorem##1##2{\item
       [\hskip\labelsep \theorem@headerfont ##2\ ##1]}%
\def\@opargbegintheorem##1##2##3{%
  \item[\hskip\labelsep \theorem@headerfont ##2\ ##1\ (##3)]}}
\endgroup
%</thc>
%    \end{macrocode}
% \end{macro}
%
%
% \subsubsection{The \texttt{marginbreak} style}
%
% This style option is the one used most often at Mainz.  It is saved
% in the file |thmb.sty|.
%    \begin{macrocode}
%<*thmb>
\begingroup \makeatletter
\@ifundefined{theorem@style}{\input{theorem.sty}}{}
\ProvidesFile{thmb.sty}
             [\FMithmInfo]
%    \end{macrocode}
%
% \begin{macro}{\th@marginbreak}
%    The \texttt{margin break} style is nearly the same as the
%    \texttt{change break} style. The only difference is the placement
%    of the theorem number.  We use |\llap| to place it in the left
%    margin.
%
%    In this style |\labelsep| denotes the separation between the
%    number and the text.
%    \begin{macrocode}
\gdef\th@marginbreak{\normalfont\slshape
  \def\@begintheorem##1##2{\item
       [\rlap{\vbox{\theorem@headerfont
                    \hbox{\llap{##2}\hskip\labelsep ##1}%
                    \hbox{\strut}}}]}%
\def\@opargbegintheorem##1##2##3{%
  \item[\rlap{\vbox{\theorem@headerfont
                    \hbox{\llap{##2}\hskip\labelsep ##1\ (##3)}%
                    \hbox{\strut}}}]}}
\endgroup
%</thmb>
%    \end{macrocode}
% \changes{v2.1c}{90/02/17}{`endgroup outside macrocode env.}
% \end{macro}
%
%
%
% \subsubsection{The \texttt{margin} style}
%
% This style option is stored in the file |thm.sty|.
%    \begin{macrocode}
%<*thm>
\begingroup \makeatletter
\@ifundefined{theorem@style}{\input{theorem.sty}}{}
\ProvidesFile{thm.sty}
             [\FMithmInfo]
%    \end{macrocode}
%
% \begin{macro}{\th@margin}
%    Again this is only a variant of the theorem styles described
%    above without any new ideas.
%    \begin{macrocode}
\gdef\th@margin{\normalfont\slshape
  \def\@begintheorem##1##2{\item
       [\theorem@headerfont \llap{##2}\hskip\labelsep ##1]}%
\def\@opargbegintheorem##1##2##3{%
  \item[\theorem@headerfont \llap{##2}\hskip\labelsep ##1\ (##3)]}}
\endgroup
%</thm>
%    \end{macrocode}
% \end{macro}
%
%
%
%
% \subsection{Final Definitions}
%
% \begin{macro}{\theorempreskipamount}
% \begin{macro}{\theorempostskipamount}
%    The \textsf{skip} parameters that regulate the vertical empty space
%    before and after the {\envfont theorem} environment have
%    to be allocated as well.
% \changes{v2.0f}{89/07/20}{Unnecessary `global' removed.}
%    \begin{macrocode}
%<*package>
\newskip\theorempreskipamount
\newskip\theorempostskipamount
%    \end{macrocode}
%    Since we have used the same values for all theorem sets, we now can
%    assign them.
%    \begin{macrocode}
\global\setlength\theorempreskipamount{12pt plus 5pt minus 3pt}
\global\setlength\theorempostskipamount{8pt plus 3pt minus 1.5pt}
%    \end{macrocode}
% \end{macro}
% \end{macro}
%
%
% \begin{macro}{\@endtheorem}
%    The same holds for the macro |\@endtheorem|, which ends a
%    {\envfont theorem} environment. Since it is the same for all
%    theorem sets, it is removed from the macros |\th@|\meta{style}.
%    It simply ends the {\envfont trivlist} environment, which was
%    begun in |\@thm|.
%    \begin{macrocode}
\global\let\@endtheorem=\endtrivlist
%    \end{macrocode}
% \end{macro}
%
%
%
% \begin{macro}{\@preamblecmds}
%    All macros defined above are to be used only in the preamble.
%    Therefore, we insert them in |\@preamblecmds| which will disable
%    them at begin document. This is done by the internal \LaTeXe{}
%    command |\@onlypreamble|.
% \changes{v2.0g}{89/09/19}{`theoremheaderfont and `theorembodyfont
%    added.}
% \changes{v2.2a}{94/02/03}{Use \LaTeXe{} interface}
%    \begin{macrocode}
\@onlypreamble\@xnthm
\@onlypreamble\@ynthm
\@onlypreamble\@othm
\@onlypreamble\newtheorem
\@onlypreamble\theoremstyle
\@onlypreamble\theorembodyfont
\@onlypreamble\theoremheaderfont
%    \end{macrocode}
% \end{macro}
%
% Finally we declare the \texttt{plain} theorem style to be the default.
% \changes{v2.1b}{90/06/04}{Fixed missing `item by declaring default
%    style.}
%    \begin{macrocode}
\theoremstyle{plain}
%</package>
%    \end{macrocode}
%
% \Finale

\endinput
}{}
\ProvidesFile{thcb.sty}
             [\FMithmInfo]
%    \end{macrocode}
%
% \begin{macro}{\th@changebreak}
%    The \texttt{change-break} theorem style is like \texttt{break}
%    but with interchange of theorem name and theorem number.  Thus we
%    define |\th@changebreak| as follows:
%    \begin{macrocode}
\gdef\th@changebreak{\normalfont\slshape
  \def\@begintheorem##1##2{\item
       [\rlap{\vbox{\hbox{\hskip\labelsep \theorem@headerfont ##2\ ##1}%
                    \hbox{\strut}}}]}%
\def\@opargbegintheorem##1##2##3{%
  \item[\rlap{\vbox{\hbox{\hskip\labelsep \theorem@headerfont
                     ##2\ ##1\ (##3)}%
                    \hbox{\strut}}}]}}
\endgroup
%</thcb>
%    \end{macrocode}
% \end{macro}
%
%
% \subsubsection{The \texttt{change} style}
%
% This style option is stored in the file |thc.sty|.
%    \begin{macrocode}
%<*thc>
\begingroup \makeatletter
\@ifundefined{theorem@style}{% \iffalse meta-comment
%
% Copyright (C) 1993-2024
%
% The LaTeX Project and any individual authors listed elsewhere
% in this file.
%
% This file is part of the Standard LaTeX `Tools Bundle'.
% -------------------------------------------------------
%
% It may be distributed and/or modified under the
% conditions of the LaTeX Project Public License, either version 1.3c
% of this license or (at your option) any later version.
% The latest version of this license is in
%    https://www.latex-project.org/lppl.txt
% and version 1.3c or later is part of all distributions of LaTeX
% version 2005/12/01 or later.
%
% The list of all files belonging to the LaTeX `Tools Bundle' is
% given in the file `manifest.txt'.
%
% \fi
%
%%
%
%
\def\FMithmInfo{2023/07/05 v2.2c Theorem extension package (FMi)}
%
% \ProvidesFile{theorem.dtx}[\FMithmInfo]
%\iffalse   % this is a METACOMMENT !
%%
%% Package `theorem' to use with LaTeX2e
%% Copyright (C) 1989-1995 Frank Mittelbach, all rights reserved.
%<+package>\@ifundefined{theorem@style}{}{\endinput}
%<+package>\NeedsTeXFormat{LaTeX2e}[1995/06/01]
%<+package>\ProvidesPackage{theorem}
%<+package>         [\FMithmInfo]
%
% \fi
%
% \changes{v1.0}{87/05/07}{First draft.}
% \changes{v1.0a}{87/05/07}{`thistheoremstyle' added.}
% \changes{v1.1a}{87/07/29}{New concept: redefine the internal
%                          \protect\LaTeX{} commands `@xnthm' etc.}
% \changes{v1.1b}{87/07/31}{`theoremfont' used to specify font.}
% \changes{v1.2a}{88/01/08}{`@endtheorem' now globally defined.}
% \changes{v1.2b}{88/01/11}{`newtheorem' only allowed in the preamble.}
% \changes{v1.3a}{88/02/14}{`theoremfont' now a toks register.}
%
% \changes{v2.0a}{88/11/16}{New implementation and new documentation.}
% \changes{v2.0a}{88/11/16}{No layout changes but internal modifications
%                           old substyles won't work.}
% \changes{v2.0b}{89/01/30}{`global' used for all definitions.}
% \changes{v2.0c}{89/06/15}{Documentation translated to English.
%                            Thanks to LLL and Nina}
% \changes{v2.1a}{90/02/15}{Support for new font selection.}
% \changes{v2.2a}{94/02/03}{Cleanup for LaTeX2e}
%
% \DoNotIndex{\@endparenv,\@ifdefinable,\@ifnextchar,\@ifundefined}
% \DoNotIndex{\@namedef,\@nameuse,\@spaces,\@tempa,\@warning}
% \DoNotIndex{\ }
% \DoNotIndex{\begingroup,\bfseries}
% \DoNotIndex{\csname}
% \DoNotIndex{\def,\do,\docdate}
% \DoNotIndex{\endcsname,\endgroup,\endinput,\endtrivlist,\expandafter}
% \DoNotIndex{\filedate,\fileversion}
% \DoNotIndex{\gdef,\global}
% \DoNotIndex{\hbox,\hskip}
% \DoNotIndex{\ignorespaces,\input,\itshape,\item}
% \DoNotIndex{\labelsep,\let,\llap}
% \DoNotIndex{\makeatletter}
% \DoNotIndex{\noexpand}
% \DoNotIndex{\refstepcounter,\relax,\rlap}
% \DoNotIndex{\setlength,\slshape,\space,\string,\strut}
% \DoNotIndex{\the,\trivlist,\typeout}
% \DoNotIndex{\vbox}
% \DoNotIndex{\xdef}
%
% \def\deflabel#1{\ttfamily #1\hfill}
% \def\deflist#1{\begin{list}{}{\settowidth\labelwidth{\ttfamily #1}%
%                               \setlength\leftmargin\labelwidth
%                               \addtolength\leftmargin\labelsep
%                               \let\makelabel\deflabel}}
% \def\enddeflist{\end{list}}
% \def\envfont{\normalfont\ttfamily}
%
% \MakeShortVerb{\|}
%
% \newcommand{\TUB}{{\sl TUGboat\/}}
%
% \GetFileInfo{theorem.dtx}
% \title{An Extension of the
%        \LaTeX{} theorem environment\thanks{This file
%        has version number \fileversion, last
%        revised \filedate.}}
% \author{Frank Mittelbach}
%
% \MaintainedByLaTeXTeam{tools}
% \maketitle
%
% \begin{abstract}
%    The macros described in this paper yield an extension of the
%    \LaTeX{} theorem mechanism. It is designed to satisfy the
%    different requirements of various journals. Thus, the
%    layout of the ``theorems'' can be manipulated by determining a
%    ``style''. This article describes not only the use, but
%    also the definition, of the necessary macros.
% \end{abstract}
%
% \section*{Preface to version 2.2}
%
% For \LaTeXe{} this package did not need any fundamental changes. I
% only modified the messages generated so that theorem layout styles
% will show up with the |\listfiles| command and cleaned the section
% on the New Font Selection Scheme since this is now included in
% \LaTeX.
%
% \section*{Preface to version 2.1}
%
% This version is identical to 2.0g described in \TUB\ 10\#3 except
% for some internal defaults which are now set depending on the used
% font selection scheme.
%
% This was done to avoid unpleasant surprises if the new font selection
% scheme is in force. For further details see
% section~\ref{sec:newfonts} and \cite{bk:GMS94}.
%
% \section{Introduction}
%
% For our purposes here, ``theorems'' are labelled enunciations,
% often set off from the main text by extra space and a font change.
% Theorems, corollaries, conjectures, definitions, and remarks
% are all instances of ``theorems''.  The ``header'' of these structures
% is composed of a label (such as \textsc{Theorem} or \textsc{Remark})
% and a number which serializes an item in the sequence of items
% with the same label.
%
% Shortly after the introduction of \LaTeX{} at the Fachbereich
% Mathematik in Mainz, the desire to manipulate the layout of
% ``theorems'' arose.  In Mainz, the following two conventions came
% into general use:
% \begin{enumerate}
%   \item The number of the theorem is shown in the margin.
%   \item There is a line break at
%         the end of the theorem header.
% \end{enumerate}
%
% Additionally, some journals require different formats which depend on
% the ``sort of theorem'': e.g.\ often remarks and definitions are set
% in |\upshape|, while |\itshape| is employed for main theorems.
%
% Confronted with these requirements, a theorem environment was
% developed in Mainz which allows separate determination of the
% layout of the ``theorems sets'', comparable to |\pagestyle|.
%
%
% \section{The user interface}
% \subsection{Defining new theorem sets}
%
% \DescribeMacro\newtheorem
% As in the original \LaTeX{} version, the command |\newtheorem|
% defines a new ``theorem set'' or ``theorem-like structure''.
% Two required arguments name the new environment and give the text
% to be typeset with each instance of the new ``set'', while
% an optional argument determines how
% the ``set'' is enumerated:
% \begin{description}
%    \item[\ttfamily \bslash newtheorem\{foo\}\{bar\}]
%       The theorem set {\envfont foo} (whose name is \texttt{bar})
%       uses its own counter.
%    \item[\ttfamily \bslash newtheorem\{foo2\}{[foo]}\{bar2\}]
%       The theorem set {\envfont foo2} (printed name \texttt{bar2})
%       uses the same counter as the theorem set \texttt{foo}.
%    \item[\ttfamily \bslash newtheorem\{foo3\}\{bar3\}{[section]}]
%       The theorem set {\envfont foo3} (printed name \texttt{bar3}) is
%       enumerated within the counter \texttt{section}, i.e.\ with every
%       new |\section| the enumeration begins again with 1, and
%       the enumeration is composed from the section-number and the
%       theorem counter itself.
% \end{description}
%
% \DescribeMacro\theoremstyle
% Additionally, the command |\theoremstyle| can define the layout
% of various, or all, theorem sets.  It should be noted that any
% theorem set defined by |\newtheorem| is typeset in the
% |\theoremstyle| that is current at the time of the definition.
% Thus, the following
% \begin{verbatim}
%    \theoremstyle{break}        \newtheorem{Cor}{Corollary}
%    \theoremstyle{plain}        \newtheorem{Exa}{Example}[section]
%\end{verbatim}                             ^^A No blank at line start
% leads to the result that the set {\envfont Cor} is formatted in the
% style \texttt{break}, while the set {\envfont Exa} and all the
% following ones are formatted in the style \texttt{plain}, unless
% another |\theoremstyle| follows.  Since the definitions installed by
% |\newtheorem| are global, one also can limit |\theoremstyle| locally
% by grouping braces.
%
% \DescribeMacro\theorembodyfont
% The choice of the font for the theorem body is completely
% independent of the chosen |\theoremstyle|; this has proven to
% be very advantageous.  For example,
% \begin{verbatim}
%   {\theorembodyfont{\upshape}          \newtheorem{Rem}{Remark}}
%\end{verbatim}
% defines a theorem set {\envfont Rem}, which will be set in
% |\upshape| in the current layout (which in our example is
% \texttt{plain}).  As with |\theoremstyle|, the |\theorembodyfont|
% chosen is that current at the time of |\newtheorem|.  If
% |\theorembodyfont| is not specified or one defines
% |\theorembodyfont{}|, then the font used will be that defined by the
% |\theoremstyle|.
%
% \DescribeMacro\theoremheaderfont It is also possible to customize
% the font used for the theorem headers.  This is, however, a global
% declaration, and therefore there should be at most one
% |\theoremheaderfont| declaration in the preamble.\footnote{If it is
% actually necessary to have different header fonts, one has to define
% new theorem styles (substituting the desired font) or specify the
% information directly in the \texttt{\bslash newtheorem} declaration
% (the unclean variant).}
%
% \DescribeMacro\theorempreskipamount
% \DescribeMacro\theorempostskipamount Two additional parameters
% affect the vertical space around the theorem environments:
% |\theorempreskipamount| and |\theorempostskipamount| define,
% respectively, the spacing before and after such an environment.
% These parameters apply for all theorem sets and can be manipulated
% with the ordinary length macros.  They are rubber lengths
% (`\textsf{skips}'), and therefore can contain \texttt{plus} and
% \texttt{minus} parts.
%
% Since the definition of theorem sets should---most sensibly---be
% placed in the preamble, we only allow installation there.
% It is therefore possible to release the memory used here after
% |\begin{document}|, in order to make room for other
% applications.
%
%
% \subsection{Existing theorem styles}
%
% The following theorem styles exist to date:
% \begin{deflist}{marginbreak}
%    \item[plain]
%       This theorem style emulates the original \LaTeX{} definition,
%       except that additionally the parameters
%       |\theorem...skipamount| are used.
%    \item[break]
%       In this style, the theorem header is followed by a line
%       break.
%    \item[marginbreak]
%       The theorem number is set in the margin, and there is a line
%       break as in \texttt{break}.
%    \item[changebreak]
%       Like \texttt{break}, but with header number and text
%       interchanged.
%    \item[change]
%       Header number and text are interchanged, without a line break.
%    \item[margin]
%       The number is set in the left margin, without a line break.
% \end{deflist}
% All styles (except \texttt{plain}) select |\slshape| as the default
% |\theorembodyfont|.
%
%
% \subsection{Examples}
%
% Given the above theorem sets {\envfont Cor}, {\envfont Exa}
% and {\envfont Rem},
% suppose that the preamble also contains the declarations:
% \begin{verbatim}
%    \theoremstyle{marginbreak}   \newtheorem{Lem}[Cor]{Lemma}
%    \theoremstyle{change}
%    \theorembodyfont{\itshape}        \newtheorem{Def}[Cor]{Definition}
%
%    \theoremheaderfont{\scshape}
%\end{verbatim}
% Then the following are some typical examples of the typeset output
% resulting from their use.
%
% \begin{Cor}
%    This is a sentence typeset in the theorem environment {\envfont
%    Cor}.
% \end{Cor}
%
% \begin{Exa}
%    This is a sentence typeset in the theorem environment {\envfont
%    Exa}.
% \end{Exa}
%
% \begin{Rem}
%    This is a sentence typeset in the theorem environment {\envfont
%    Rem}.
% \end{Rem}
%
% \begin{Lem}[Ben User]
%    This is a sentence typeset in the theorem environment {\envfont
%    Lem}.
% \end{Lem}
%
% \begin{Def}[Very impressive Definition]
%    This is a sentence typeset in the theorem environment {\envfont
%    Def}.
% \end{Def}
%
% The last two examples show the effect of the optional argument to
% a theorem environment (it is the text typeset in parentheses).
%
%
% \section{Special Considerations}
% \label{sec:newfonts}
%
% Theoremheader and body are implemented as a unit. This means that
% the |\theoremheaderfont|  will inherit characteristics of the
% |\theorembodyfont| in \LaTeXe. Thus, if for example
% |\theorembodyfont| is |\itshape| and |\theoremheaderfont|
% is |\bfseries| the font selected for the header will have the
% characteristics `bold extended italic'. If this is not desired one
% should set the |\theoremheaderfont| to something like
% \begin{verbatim}
%      \theoremheaderfont{\normalfont\bfseries}
%\end{verbatim}
% i.e.\ supplying all necessary font information explicitly.
%
%
%
% \section{Acknowledgements}
%
% The publication of this set of macros was only possible with the
% help of Christina Busse (translating the manuscript into English),
% Joachim Pense (playing the r\^ole of typist), Chris Rowley
% (looking everything over) and many others providing useful
% suggestions.
%
% \MaybeStop{%
%    \begin{thebibliography}{9}
%       \bibitem{bk:GMS94} \textsc{M.~Goossens}, \textsc{F.~Mittelbach}
%         and \textsc{A.~Samarin}.
%         \newblock The \LaTeX{} Companion.
%         \newblock
%         Addison-Wesley, Reading, Massachusetts, 1994.
%       \bibitem{latex:source} \textsc{Lamport, Leslie.} \newblock
%                               \texttt{latex.tex}, version 2.09, date
%                               Feb. 1990.
%    \end{thebibliography}
%    \PrintIndex
%    \PrintChanges}
%
%
% \section{The documentation driver file}
%
% The next bit of code contains the documentation driver file for
% \TeX{}, i.e., the file that will produce the documentation you are
% currently reading. It will be extracted from this file by the
% \texttt{docstrip} program. Since it is the first code in the file
% one can alternatively process this file directly with \LaTeXe{} to
% obtain the documentation.
%
%    \begin{macrocode}
%<*driver>
\documentclass{ltxdoc}

\usepackage{theorem}

%<+driver>% The next few lines define theorem sets which are used
%<+driver>% in the example section of the documentation.

\theoremstyle{break}       \newtheorem{Cor}{Corollary}
\theoremstyle{plain}       \newtheorem{Exa}{Example}[section]
{\theorembodyfont{\upshape}\newtheorem{Rem}{Remark}}
\theoremstyle{marginbreak} \newtheorem{Lem}[Cor]{Lemma}
\theoremstyle{change}
\theorembodyfont{\itshape} \newtheorem{Def}[Cor]{Definition}

\theoremheaderfont{\scshape}

\RecordChanges

\begin{document}
   \DocInput{theorem.dtx}
\end{document}
%</driver>
%    \end{macrocode}
%
% \section{Definition of the Macros}
%
% If the file has been loaded before, we abort immediately. If not the
% package announces itself (this is actually done at the very top of
% the file---the way it is done isn't good style so don't copy it).
% \changes{v2.0e}{89/07/19}{Spaces removed from `typeout'.}
% \changes{v2.2b}{95/11/19}{Announce moved to top of file}
%    \begin{macrocode}
%<*package>
%\@ifundefined{theorem@style}{}{\endinput}
%\def\FMithmInfo{1995/11/19 v2.2b Theorem extension package (FMi)}
%\ProvidesPackage{theorem}[\FMithmInfo]
%    \end{macrocode}
%
%
% \subsection{Definition of theorem styles and fonts}
%
% All the definitions in this file are done globally to allow
% inputting this file inside a group.
%
% \begin{macro}{\theoremstyle}
%    Before a theorem style can be installed, the chosen style must be
%    known. For that reason, we must test to see that
%    |\th@|\meta{style} is known or, more precisely, that it is
%    different from |\relax|. If the style is not known then
%    |\th@plain| is used.
%    \begin{macrocode}
\gdef\theoremstyle#1{%
   \@ifundefined{th@#1}{\@warning
          {Unknown theoremstyle `#1'. Using `plain'}%
          \theorem@style{plain}}%
%    \end{macrocode}
%    We save the theorem style to be used in the token register
%    |\theorem@style|.
%    \begin{macrocode}
      {\theorem@style{#1}}%
%    \end{macrocode}
%    Now we ``evaluate'' the theorem style: this means, we call the
%    macro |\th@|\meta{style} which will activate the relevant
%    definitions which are contained in a separate file.  This is done
%    in a group to suppress changes to the current font.  This could
%    otherwise pose problems together with the new font selection
%    scheme\footnote{When I printed the original article using the new
%    font selection scheme I ended with a document with slanted
%    typefaces (text headings and all) simply because one of the
%    theorem styles used \texttt{\bslash sl} at toplevel.} if the
%    |\th@|\meta{style} is evaluated a second time.
% \changes{v2.1a}{90/02/15}{Group added around `th@..}
% \changes{v2.1a}{90/02/15}{Reseting of `..@begintheorem removed.}
%    \begin{macrocode}
      \begingroup
        \csname th@\the\theorem@style \endcsname
      \endgroup}
%    \end{macrocode}
% \end{macro}
%
%
% \begin{macro}{\@begintheorem}
% \changes{v2.1a}{90/02/15}{Relaxed at toplevel.}
% \begin{macro}{\@opargbegintheorem}
% \changes{v2.1a}{90/02/15}{Relaxed at toplevel.}
%    We reset |\@begintheorem| and |\@opargbegintheorem| to
%    |\relax| since these commands are no longer necessary at
%    toplevel. This will save a few tokens.
%    \begin{macrocode}
\global\let\@begintheorem\relax
\global\let\@opargbegintheorem\relax
%    \end{macrocode}
% \end{macro}
% \end{macro}
%
% \begin{macro}{\theorem@style}
%    Obviously the \textsf{token} register used above has to be
%    allocated. To assure the utmost compatibility with the
%    original \LaTeX{} definition, we set the default theorem style to
%    \texttt{plain}, which implements the usual \LaTeX{} convention.
% \changes{v2.0f}{89/07/20}{Unnecessary `global' removed.}
%    \begin{macrocode}
\newtoks\theorem@style
\global\theorem@style{plain}
%    \end{macrocode}
% \end{macro}
%
% \begin{macro}{\theorembodyfont}
% \begin{macro}{\theorem@bodyfont}
%    If the |\theorembodyfont| is set by the user then it should not
%    interact with the default font set in the theorem style. When the
%    new font selection is in force this may happen if, for example,
%    the default is |\itshape| and the new |\theorembodyfont| is
%    |\sffamily|. So we add a |\reset@font| command in front of the
%    user definition.
% \changes{v2.0f}{89/07/20}{Unnecessary `global' removed.}
% \changes{v2.0g}{89/09/19}{`theoremfont to `theorembodyfont.}
% \changes{v2.1c}{92/06/22}{`reset@font added and logic changed.}
% \changes{v2.1d}{92/10/28}{Reset to empty if arg is empty}
%    \begin{macrocode}
\gdef\theorembodyfont#1{%
%    \end{macrocode}
%    We check if the argument supplied is empty and if so put nothing
%    into the |\theorem@bodyfont| token register to allow for
%    |\theorembodyfont{}| as a mean of using the default of the
%    current |\theoremstyle|.
%    \begin{macrocode}
   \def\@tempa{#1}%
   \ifx\@tempa\@empty
    \theorem@bodyfont{}%
   \else
    \theorem@bodyfont{\reset@font#1}%
   \fi
   }
\newtoks\theorem@bodyfont
\global\theorem@bodyfont{}
%    \end{macrocode}
% \end{macro}
% \end{macro}
%
%
% \begin{macro}{\theoremheaderfont}
% \changes{v2.0g}{89/09/19}{Macro added.}
%    The font for the theorem headers is handled differently because
%    this definition applies to all theorem styles.
%    \begin{macrocode}
\gdef\theoremheaderfont#1{\gdef\theorem@headerfont{#1}%
%    \end{macrocode}
%    After using the macro once it is redefined to produce an error
%    message.
%    \begin{macrocode}
       \gdef\theoremheaderfont##1{%
        \typeout{\string\theoremheaderfont\space should be used
                 only once.}}}
%    \end{macrocode}
% \end{macro}
%
% \begin{macro}{\theorem@headerfont}
% \changes{v2.0g}{89/09/19}{Macro added.}
%    To set the |\theorem@headerfont| default we first test if
%    the new fontselection scheme is in force.
% \changes{v2.1a}{90/02/15}{Support for new font selection.}
%    \begin{macrocode}
\ifx\upshape\undefined
%    \end{macrocode}
%    If not we define it to expand into |\bfseries|. We don't use
%    |\let| just in case a following style option redefines this
%    macro.
%    \begin{macrocode}
\gdef\theorem@headerfont{\bfseries}
%    \end{macrocode}
%    Otherwise we reset the current shape before calling |\bfseries|.
%    \begin{macrocode}
\else \gdef\theorem@headerfont{\normalfont\bfseries}\fi
%    \end{macrocode}
% \end{macro}
%
% \begin{macro}{\th@plain}
% \begin{macro}{\th@break}
% \begin{macro}{\th@marginbreak}
% \begin{macro}{\th@changebreak}
% \begin{macro}{\th@change}
% \begin{macro}{\th@margin}
%    The different styles are defined in macros such as
%    |\th@plain|.  Since memory space is precious in
%    ``non-Big-versions'', we have to avoid offering too many unused
%    definitions.  Therefore we define these styles in separate files
%    that can be loaded on demand.  Thus the commands themselves only
%    load these files. We use |\@input@|, a \LaTeXe{} internal command
%    that ensures that the file will be listed with |\listfiles|.
% \changes{v2.2a}{94/02/02}{Use `@input@ to load theorem layout files}
%    \begin{macrocode}
\gdef\th@plain{\@input@{thp.sty}}
\gdef\th@break{\@input@{thb.sty}}
\gdef\th@marginbreak{\@input@{thmb.sty}}
\gdef\th@changebreak{\@input@{thcb.sty}}
\gdef\th@change{\@input@{thc.sty}}
\gdef\th@margin{\@input@{thm.sty}}
%    \end{macrocode}
%    This list will be expanded when new styles become
%    available.  For testing,  just append new theorem substyles
%    as document options.
% \end{macro}
% \end{macro}
% \end{macro}
% \end{macro}
% \end{macro}
% \end{macro}
%
%
%
% \subsection{Definition of a new theorem set}
%
% As already pointed out, a new theorem environment can be defined in
% three different ways:
% \begin{verbatim}
%    \newtheorem{Lem}{Lemma}
%    \newtheorem{Lem}{Lemma}[section]
%    \newtheorem{Lem}[Theorem]{Lemma}
%\end{verbatim}
% The function of the macro |\newtheorem| is to recognize these cases
% and then to branch into one of the three macros |\@ynthm|, |\@xnthm|
% or |\@othm|.  This mechanism is adopted unchanged from
% \cite{latex:source}; the essential point here is that, for example,
% in the second case, the arguments \texttt{Lem}, \texttt{Lemma} and
% \texttt{section} are passed over to the macro |\@xnthm|.
%
% We inspect this case first because the others present fewer problems,
% and thus are easily derived from this one.
%
% \begin{macro}{\@xnthm}
%    For our example arguments, the macro |\@xnthm| must fulfill
%    the following:
%    \begin{itemize}
%       \item
%          Define a new \LaTeX-counter `Lem'
%       \item
%          reset this counter within a |\section|
%       \item
%          define the macro |\theLem|
%       \item
%          define the environment macros |\Lem| and
%          |\endLem| using the current
%          |\theoremstyle| and |\theorem@bodyfont|.
%    \end{itemize}
%    Obviously, all this should happen only if the first argument
%    of |\@xnthm| (i.e.\ \texttt{Lem} in our example) is chosen so
%    as not to conflict with any previously defined commands or
%    environments.  This test is performed by the \LaTeX{} macro
%    |\@ifdefinable|.
%    \begin{macrocode}
\gdef\@xnthm#1#2[#3]{\expandafter\@ifdefinable\csname #1\endcsname
%    \end{macrocode}
%    Therefore, the first argument of |\@ifdefinable| is the
%    expansion (in the example, |\Lem|) of
%    |\csname||#1\endcsname|.  The second argument is
%    executed only if the test has been completed successfully.
%    \begin{macrocode}
   {%
%    \end{macrocode}
%    Now we define the new counter. The names of the \LaTeX{}
%    macros employed should speak for themselves:
% \changes{v2.2b}{1995/10/07}{Use \cs{@newctr}}
% \changes{v2.2c}{1995/11/23}{Use \cs{@newctr} properly.}
%    \begin{macrocode}
    \@definecounter{#1}\@newctr{#1}[#3]%
%    \end{macrocode}
%    Using |\@newctr| will give a proper error message if the counter
%    in |#3| is not defined.
%    In defining `|\theLem|' we must generate the desired macro
%    name by use of |\expandafter| and |\csname|.
%    \begin{macrocode}
    \expandafter\xdef\csname the#1\endcsname
%    \end{macrocode}
%    An |\xdef| is used in order to make the definition global,
%    and to ensure that it contains the replacement texts of
%    |\@thmcountersep| and |\@thmcounter|.\footnote{These
%    two macros can be defined by the document style.  Their default
%    values produce a `\texttt{.}' as separation and an arabic
%    representation of the number.} However, not everything should be
%    expanded. For example, it saves space to use |\thesection|
%    instead of its---at times---lengthy expansion.
%    \begin{macrocode}
      {\expandafter \noexpand \csname the#3\endcsname
       \@thmcountersep \@thmcounter{#1}}%
%    \end{macrocode}
%    Thus with the defaults of \LaTeX{}, |\theLem| would be
%    replaced by the command sequence |\thesection.\arabic{Lem}|.
%
%    We will now look at the definition of the macro which is executed
%    at the beginning of the actual environment (in our example this
%    macro is |\Lem|). It should be noted that we use an
%    ``|\expandafter| trick'' to expand only certain parts of the
%    replacement text at the time of the definition.
%    \begin{macrocode}
    \def\@tempa{\global\@namedef{#1}}%
    \expandafter \@tempa \expandafter{%
%    \end{macrocode}
%    First, the macro that contains the current definitions of
%    |\@begintheorem| and |\@opargtheorem| should be called
%    up.  The name of this macro---as is already known---has the form
%    |\th@|\meta{theorem\ style}; therefore, it must be called by
%    \begin{macrocode}
      \csname th@\the \theorem@style
%    \end{macrocode}
%    In addition the default theorem font should be changeable, i.e.\
%    we have to insert the contents of |\theorem@bodyfont|. For
%    that reason, we expand even further, beyond |\endcsname|,
%    and thus insert the contents of the \textsf{token} register
%    |\theorem@bodyfont| in the replacement text.
%    \begin{macrocode}
            \expandafter \endcsname \the \theorem@bodyfont
%    \end{macrocode}
%    Now it is time to call the macro |\@thm| which takes over
%    the further processing.  It has two arguments: the current
%    counter name (in our example, \texttt{Lem}), and the text of the
%    label (in our example, \texttt{Lemma}).
%    \begin{macrocode}
     \@thm{#1}{#2}}%
%    \end{macrocode}
%    With this, the `sub-definition' is complete.  The macro
%    |\@endtheorem| ends a theorem environment and is, so far, nothing
%    but an |\endtrivlist|.  (Hence it is defined globally, and not
%    within the theorem styles.\footnote{This has to be changed as
%    soon as theorem styles that change \texttt{\bslash @endtheorem}
%    exist.  In such a case, all existing styles must be changed as
%    well since they will have to reset the macro.}) Therefore, we can
%    set it equivalent to the macro that ends the theorem set (in our
%    example, |\endLem|).  However, if some day theorem styles exist
%    that do change |\@endtheorem|, we would have to use the
%    commented-out line instead.
%    \begin{macrocode}
    \global \expandafter \let \csname end#1\endcsname \@endtheorem
%   \global\@namedef{end#1}{\@endtheorem}%
%    \end{macrocode}
%    With these commands all the required definitions are employed,
%    unless the test |\@ifdefinable| has failed.  Therefore, we
%    end the second argument of this macro and with it the definition
%    of |\@xnthm|.
%    \begin{macrocode}
   }}
%    \end{macrocode}
% \end{macro}
%
% \begin{macro}{\@ynthm}
%    The definition of |\@ynthm| is completely analogous. In
%    this case the new counter that is defined is not reset within
%    another counter; thus the definition of |\the...|
%    is simplified:
%    \begin{macrocode}
\gdef\@ynthm#1#2{\expandafter\@ifdefinable\csname #1\endcsname
   {\@definecounter{#1}%
    \expandafter\xdef\csname the#1\endcsname{\@thmcounter{#1}}%
%    \end{macrocode}
%    The rest of the definition corresponds literally to that of
%    |\@xnthm|:
%    \begin{macrocode}
    \def\@tempa{\global\@namedef{#1}}\expandafter \@tempa
     \expandafter{\csname th@\the \theorem@style \expandafter
     \endcsname \the\theorem@bodyfont \@thm{#1}{#2}}%
    \global \expandafter \let \csname end#1\endcsname \@endtheorem}}
%    \end{macrocode}
% \end{macro}
%
% \begin{macro}{\@othm}
%    The definition of |\@othm| does not contain anything new.
%    \begin{macrocode}
\gdef\@othm#1[#2]#3{%
%    \end{macrocode}
% \changes{v2.2b}{95/10/07}{Build in check for counter defined}
%    We do not define a new counter but instead use one that has
%    already been defined.  Thus the only definition we need is that
%    of this pseudo-counter (i.e.\ |\the|\meta{env.\ name}).
%    First we check if |#2| corresponds to a known counter name.
%    \begin{macrocode}
  \expandafter\ifx\csname c@#2\endcsname\relax
   \@nocounterr{#2}%
  \else
   \expandafter\@ifdefinable\csname #1\endcsname
   {\expandafter \xdef \csname the#1\endcsname
     {\expandafter \noexpand \csname the#2\endcsname}%
%    \end{macrocode}
%    All other parts of the definition can be adopted from
%    |\@xnthm|. We have to remember, though, that in this case
%    the name of the current counter and the theorem label have moved
%    to the second and third arguments.
%    \begin{macrocode}
    \def\@tempa{\global\@namedef{#1}}\expandafter \@tempa
     \expandafter{\csname th@\the \theorem@style \expandafter
     \endcsname \the\theorem@bodyfont \@thm{#2}{#3}}%
    \global \expandafter \let \csname end#1\endcsname \@endtheorem}%
  \fi}
%    \end{macrocode}
% \end{macro}
%
% \subsection{Macros that are employed in a {\envfont theorem}
%    environment}
%
% \begin{macro}{\@thm}
%    The macro |\@thm| has to increase the current counter.  Then,
%    depending on whether the environment has (or does not have) an
%    optional argument, it has to branch into either |\@begintheorem|
%    or |\@opargtheorem|.
%    \begin{macrocode}
\gdef\@thm#1#2{\refstepcounter{#1}%
%    \end{macrocode}
%    Now we start a {\envfont trivlist} environment, and give
%    |\@topsep| and |\@topsepadd| the values of the \textsf{skip}
%    registers |\theorempreskipamount| and |\theorempostskipamount|.
%    The value in |\@topsep| is the vertical space that is inserted by
%    the first (and only) |\item| in our |\trivlist| whilst
%    |\@topsepadd| is inserted by |\@endparenv| at the end of that
%    {\envfont trivlist} environment.  By using these registers, we
%    obtain the desired space around a {\envfont theorem} environment.
%    \begin{macrocode}
   \trivlist
   \@topsep \theorempreskipamount               % used by first \item
   \@topsepadd \theorempostskipamount           % used by \@endparenv
%    \end{macrocode}
%    Now we have to test whether an optional argument has been given.
% \changes{v2.0d}{89/07/18}{Forgotten percent added.}
%    \begin{macrocode}
   \@ifnextchar [%
%    \end{macrocode}
%    If there is an optional argument, we will call |\@ythm|, and
%    move the arguments read back into the input stream.
%    \begin{macrocode}
   {\@ythm{#1}{#2}}%
%    \end{macrocode}
%    If not, we call |\@begintheorem|. Its first argument is the
%    name of the theorem set (hence the second argument of
%    |\@thm|). Its second argument is the macro that produces the
%    current number.
%    \begin{macrocode}
   {\@begintheorem{#2}{\csname the#1\endcsname}\ignorespaces}}
%    \end{macrocode}
% \end{macro}
%
% \begin{macro}{\@xthm}
% \begin{macro}{\@ythm}
%    Both these macros were originally called by |\@thm|.  We do
%    not need |\@xthm| anymore, hence we reset it to
%    |\relax|. The definition of |\@ythm| has not changed at
%    all from its definition in \LaTeX. In order to make the macros
%    easier to understand, we will
%    nevertheless present it (commented out).
%    \begin{macrocode}
\global\let\@xthm\relax
% \def\@ythm#1#2[#3]{\@opargbegintheorem{#2}{\csname
%           the#1\endcsname}{#3}\ignorespaces}
%</package>
%    \end{macrocode}
%    The primitive |\ignorespaces| in |\@ythm| and
%    |\@thm| is needed to remove the spaces between the
%    |\begin{...}| and the actual text.
% \end{macro}
% \end{macro}
%
%
% \subsection{Definition of the theorem substyles}
%
% As already pointed out, the theorem substyles, defined below, are
% only loaded when necessary.  Note that all these substyles, except
% \texttt{plain}, have |\slshape| as the default body font.
%
% \subsubsection{The \texttt{plain} style}
%
% \label{dok:th@plain} As the following macros use |@|, we have
% to locally set the |\catcode| of this symbol to ``letter''.
% This happens within a group, so that we do not have to worry about
% which |\catcode| that symbol had before.
%    \begin{macrocode}
%<*thp>
\begingroup \makeatletter
%    \end{macrocode}
% Since we are now within a group, we must make all definitions
% globally.  First we make sure that \texttt{theorem.sty} is loaded.
% This will allow us to use this file as a document style option
% without having to call \texttt{theorem} itself as an option.  At the
% same time, we assure that at least version 2 is loaded, since
% |\theorem@style| was not defined in earlier versions.
%    \begin{macrocode}
\@ifundefined{theorem@style}{\input{theorem.sty}}{}
\ProvidesFile{thp.sty}
             [\FMithmInfo]
%    \end{macrocode}
%
%
% \begin{macro}{\th@plain}
% \begin{macro}{\@begintheorem}
% \begin{macro}{\@opargbegintheorem}
%    |\theoremstyle{plain}| corresponds to the original
%    definition, except that the distances to the surrounding text are
%    determined by the parameters |\theorempreskipamount| and
%    |\theorempostskipamount|.
%    First we set the default body font.
%    \begin{macrocode}
\gdef\th@plain{\normalfont\itshape
%    \end{macrocode}
%    Then we define |\@begintheorem| and
%    |\@opargbegintheorem|. These two macros define how the
%    header of a theorem is typeset.  |\@opargbegintheorem| will
%    be called if a \textsf{theorem} environment with an optional
%    argument is encountered; otherwise, the header is constructed by
%    calling |\@begintheorem|.  If one of these macros is
%    executed, we are within a \textsf{trivlist} environment started by
%    |\@thm|.  So the theorem header is produced with an
%    |\item| command.
%
%    Instead of specifying the header font directly, all standard
%    theorem styles use the |\theorem@headerfont| macro to allow
%    customization.  The extra space (|\labelsep|) is necessary
%    because of problems in the \textsf{trivlist} environment.
%    \begin{macrocode}
  \def\@begintheorem##1##2{%
        \item[\hskip\labelsep \theorem@headerfont ##1\ ##2]}%
%    \end{macrocode}
%    The definition of |\@opargbegintheorem| is completely
%    analogous. The only difference is the fact that there exists a
%    third argument (which is the optional parameter of the
%    environment and contains additional information about the
%    theorem).  Customarily we enclose it in parentheses.
%    \begin{macrocode}
\def\@opargbegintheorem##1##2##3{%
   \item[\hskip\labelsep \theorem@headerfont ##1\ ##2\ (##3)]}}
%    \end{macrocode}
% \end{macro}
% \end{macro}
% \end{macro}
%
% \noindent
% We conclude with an |\endgroup| to restore the
% |\catcode| of |@|.
%    \begin{macrocode}
\endgroup
%</thp>
%    \end{macrocode}
%
%
% \subsubsection{The \texttt{break} style}
%
% This style option is stored in the file |thb.sty|.
% For the next two lines see the documentation for |\th@plain| on
% page~\pageref{dok:th@plain}.
%    \begin{macrocode}
%<*thb>
\begingroup \makeatletter
\@ifundefined{theorem@style}{\input{theorem.sty}}{}
\ProvidesFile{thb.sty}
             [\FMithmInfo]
%    \end{macrocode}
%
% \begin{macro}{\th@break}
%    |\theoremstyle{break}| produces a line break after the name
%    of the theorem.  The font is |\slshape|.  Hence, we define
%    |\th@break| as follows:
%    \begin{macrocode}
\gdef\th@break{\normalfont\slshape
  \def\@begintheorem##1##2{\item[%
%    \end{macrocode}
%    We run into the following problem: it is not possible to create
%    the header with |\item[|\meta{title}|]| and then start
%    a new line by, for example, |\mbox{}\\|. Such a definition
%    will fail whenever a list environment follows immediately.  With
%    the above construction, the |\mbox{}| causes the switch
%    |@inlabel| (cf.\ definition of |\list| and
%    |\trivlist| in \cite{latex:source}) to be set to \texttt{false}
%    and so the following list will insert additional vertical
%    space (|\topskip|). This is quite annoying. Therefore, we
%    create the line break within the |\item|. In order to ensure
%    that the text will begin at the proper position in the following
%    line, we simply pretend that the label does not take any
%    room.\footnote{This will lead to problems whenever very high
%    symbols occurring in the line tower into the
%    heading. So, something else has to be done here sometime.}
%    \begin{macrocode}
       \rlap{\vbox{\hbox{\hskip \labelsep\theorem@headerfont ##1\ ##2}%
                   \hbox{\strut}}}]}%
%    \end{macrocode}
%    Again, the definition of |\@opargbegintheorem| is
%    completely analogous.
%    \begin{macrocode}
\def\@opargbegintheorem##1##2##3{%
  \item[\rlap{\vbox{\hbox{\hskip \labelsep \theorem@headerfont
                     ##1\ ##2\ (##3)}%
                    \hbox{\strut}}}]}}
\endgroup
%</thb>
%    \end{macrocode}
% \end{macro}
%
%
% \subsubsection{The \texttt{changebreak} style}
%
%    \begin{macrocode}
% This style option is stored in the file |thcb.sty|.
%    \begin{macrocode}
%<*thcb>
\begingroup \makeatletter
\@ifundefined{theorem@style}{\input{theorem.sty}}{}
\ProvidesFile{thcb.sty}
             [\FMithmInfo]
%    \end{macrocode}
%
% \begin{macro}{\th@changebreak}
%    The \texttt{change-break} theorem style is like \texttt{break}
%    but with interchange of theorem name and theorem number.  Thus we
%    define |\th@changebreak| as follows:
%    \begin{macrocode}
\gdef\th@changebreak{\normalfont\slshape
  \def\@begintheorem##1##2{\item
       [\rlap{\vbox{\hbox{\hskip\labelsep \theorem@headerfont ##2\ ##1}%
                    \hbox{\strut}}}]}%
\def\@opargbegintheorem##1##2##3{%
  \item[\rlap{\vbox{\hbox{\hskip\labelsep \theorem@headerfont
                     ##2\ ##1\ (##3)}%
                    \hbox{\strut}}}]}}
\endgroup
%</thcb>
%    \end{macrocode}
% \end{macro}
%
%
% \subsubsection{The \texttt{change} style}
%
% This style option is stored in the file |thc.sty|.
%    \begin{macrocode}
%<*thc>
\begingroup \makeatletter
\@ifundefined{theorem@style}{\input{theorem.sty}}{}
\ProvidesFile{thc.sty}
             [\FMithmInfo]
%    \end{macrocode}
%
% \begin{macro}{\th@change}
%    The \texttt{change} theorem style corresponds to the
%    \texttt{change break} style without a linebreak after the header.
%    To say it in another way, it's the same as the \texttt{plain}
%    style but with number and name interchanged and |\slshape| as the
%    default font.
%    \begin{macrocode}
\gdef\th@change{\normalfont\slshape
  \def\@begintheorem##1##2{\item
       [\hskip\labelsep \theorem@headerfont ##2\ ##1]}%
\def\@opargbegintheorem##1##2##3{%
  \item[\hskip\labelsep \theorem@headerfont ##2\ ##1\ (##3)]}}
\endgroup
%</thc>
%    \end{macrocode}
% \end{macro}
%
%
% \subsubsection{The \texttt{marginbreak} style}
%
% This style option is the one used most often at Mainz.  It is saved
% in the file |thmb.sty|.
%    \begin{macrocode}
%<*thmb>
\begingroup \makeatletter
\@ifundefined{theorem@style}{\input{theorem.sty}}{}
\ProvidesFile{thmb.sty}
             [\FMithmInfo]
%    \end{macrocode}
%
% \begin{macro}{\th@marginbreak}
%    The \texttt{margin break} style is nearly the same as the
%    \texttt{change break} style. The only difference is the placement
%    of the theorem number.  We use |\llap| to place it in the left
%    margin.
%
%    In this style |\labelsep| denotes the separation between the
%    number and the text.
%    \begin{macrocode}
\gdef\th@marginbreak{\normalfont\slshape
  \def\@begintheorem##1##2{\item
       [\rlap{\vbox{\theorem@headerfont
                    \hbox{\llap{##2}\hskip\labelsep ##1}%
                    \hbox{\strut}}}]}%
\def\@opargbegintheorem##1##2##3{%
  \item[\rlap{\vbox{\theorem@headerfont
                    \hbox{\llap{##2}\hskip\labelsep ##1\ (##3)}%
                    \hbox{\strut}}}]}}
\endgroup
%</thmb>
%    \end{macrocode}
% \changes{v2.1c}{90/02/17}{`endgroup outside macrocode env.}
% \end{macro}
%
%
%
% \subsubsection{The \texttt{margin} style}
%
% This style option is stored in the file |thm.sty|.
%    \begin{macrocode}
%<*thm>
\begingroup \makeatletter
\@ifundefined{theorem@style}{\input{theorem.sty}}{}
\ProvidesFile{thm.sty}
             [\FMithmInfo]
%    \end{macrocode}
%
% \begin{macro}{\th@margin}
%    Again this is only a variant of the theorem styles described
%    above without any new ideas.
%    \begin{macrocode}
\gdef\th@margin{\normalfont\slshape
  \def\@begintheorem##1##2{\item
       [\theorem@headerfont \llap{##2}\hskip\labelsep ##1]}%
\def\@opargbegintheorem##1##2##3{%
  \item[\theorem@headerfont \llap{##2}\hskip\labelsep ##1\ (##3)]}}
\endgroup
%</thm>
%    \end{macrocode}
% \end{macro}
%
%
%
%
% \subsection{Final Definitions}
%
% \begin{macro}{\theorempreskipamount}
% \begin{macro}{\theorempostskipamount}
%    The \textsf{skip} parameters that regulate the vertical empty space
%    before and after the {\envfont theorem} environment have
%    to be allocated as well.
% \changes{v2.0f}{89/07/20}{Unnecessary `global' removed.}
%    \begin{macrocode}
%<*package>
\newskip\theorempreskipamount
\newskip\theorempostskipamount
%    \end{macrocode}
%    Since we have used the same values for all theorem sets, we now can
%    assign them.
%    \begin{macrocode}
\global\setlength\theorempreskipamount{12pt plus 5pt minus 3pt}
\global\setlength\theorempostskipamount{8pt plus 3pt minus 1.5pt}
%    \end{macrocode}
% \end{macro}
% \end{macro}
%
%
% \begin{macro}{\@endtheorem}
%    The same holds for the macro |\@endtheorem|, which ends a
%    {\envfont theorem} environment. Since it is the same for all
%    theorem sets, it is removed from the macros |\th@|\meta{style}.
%    It simply ends the {\envfont trivlist} environment, which was
%    begun in |\@thm|.
%    \begin{macrocode}
\global\let\@endtheorem=\endtrivlist
%    \end{macrocode}
% \end{macro}
%
%
%
% \begin{macro}{\@preamblecmds}
%    All macros defined above are to be used only in the preamble.
%    Therefore, we insert them in |\@preamblecmds| which will disable
%    them at begin document. This is done by the internal \LaTeXe{}
%    command |\@onlypreamble|.
% \changes{v2.0g}{89/09/19}{`theoremheaderfont and `theorembodyfont
%    added.}
% \changes{v2.2a}{94/02/03}{Use \LaTeXe{} interface}
%    \begin{macrocode}
\@onlypreamble\@xnthm
\@onlypreamble\@ynthm
\@onlypreamble\@othm
\@onlypreamble\newtheorem
\@onlypreamble\theoremstyle
\@onlypreamble\theorembodyfont
\@onlypreamble\theoremheaderfont
%    \end{macrocode}
% \end{macro}
%
% Finally we declare the \texttt{plain} theorem style to be the default.
% \changes{v2.1b}{90/06/04}{Fixed missing `item by declaring default
%    style.}
%    \begin{macrocode}
\theoremstyle{plain}
%</package>
%    \end{macrocode}
%
% \Finale

\endinput
}{}
\ProvidesFile{thc.sty}
             [\FMithmInfo]
%    \end{macrocode}
%
% \begin{macro}{\th@change}
%    The \texttt{change} theorem style corresponds to the
%    \texttt{change break} style without a linebreak after the header.
%    To say it in another way, it's the same as the \texttt{plain}
%    style but with number and name interchanged and |\slshape| as the
%    default font.
%    \begin{macrocode}
\gdef\th@change{\normalfont\slshape
  \def\@begintheorem##1##2{\item
       [\hskip\labelsep \theorem@headerfont ##2\ ##1]}%
\def\@opargbegintheorem##1##2##3{%
  \item[\hskip\labelsep \theorem@headerfont ##2\ ##1\ (##3)]}}
\endgroup
%</thc>
%    \end{macrocode}
% \end{macro}
%
%
% \subsubsection{The \texttt{marginbreak} style}
%
% This style option is the one used most often at Mainz.  It is saved
% in the file |thmb.sty|.
%    \begin{macrocode}
%<*thmb>
\begingroup \makeatletter
\@ifundefined{theorem@style}{% \iffalse meta-comment
%
% Copyright (C) 1993-2024
%
% The LaTeX Project and any individual authors listed elsewhere
% in this file.
%
% This file is part of the Standard LaTeX `Tools Bundle'.
% -------------------------------------------------------
%
% It may be distributed and/or modified under the
% conditions of the LaTeX Project Public License, either version 1.3c
% of this license or (at your option) any later version.
% The latest version of this license is in
%    https://www.latex-project.org/lppl.txt
% and version 1.3c or later is part of all distributions of LaTeX
% version 2005/12/01 or later.
%
% The list of all files belonging to the LaTeX `Tools Bundle' is
% given in the file `manifest.txt'.
%
% \fi
%
%%
%
%
\def\FMithmInfo{2023/07/05 v2.2c Theorem extension package (FMi)}
%
% \ProvidesFile{theorem.dtx}[\FMithmInfo]
%\iffalse   % this is a METACOMMENT !
%%
%% Package `theorem' to use with LaTeX2e
%% Copyright (C) 1989-1995 Frank Mittelbach, all rights reserved.
%<+package>\@ifundefined{theorem@style}{}{\endinput}
%<+package>\NeedsTeXFormat{LaTeX2e}[1995/06/01]
%<+package>\ProvidesPackage{theorem}
%<+package>         [\FMithmInfo]
%
% \fi
%
% \changes{v1.0}{87/05/07}{First draft.}
% \changes{v1.0a}{87/05/07}{`thistheoremstyle' added.}
% \changes{v1.1a}{87/07/29}{New concept: redefine the internal
%                          \protect\LaTeX{} commands `@xnthm' etc.}
% \changes{v1.1b}{87/07/31}{`theoremfont' used to specify font.}
% \changes{v1.2a}{88/01/08}{`@endtheorem' now globally defined.}
% \changes{v1.2b}{88/01/11}{`newtheorem' only allowed in the preamble.}
% \changes{v1.3a}{88/02/14}{`theoremfont' now a toks register.}
%
% \changes{v2.0a}{88/11/16}{New implementation and new documentation.}
% \changes{v2.0a}{88/11/16}{No layout changes but internal modifications
%                           old substyles won't work.}
% \changes{v2.0b}{89/01/30}{`global' used for all definitions.}
% \changes{v2.0c}{89/06/15}{Documentation translated to English.
%                            Thanks to LLL and Nina}
% \changes{v2.1a}{90/02/15}{Support for new font selection.}
% \changes{v2.2a}{94/02/03}{Cleanup for LaTeX2e}
%
% \DoNotIndex{\@endparenv,\@ifdefinable,\@ifnextchar,\@ifundefined}
% \DoNotIndex{\@namedef,\@nameuse,\@spaces,\@tempa,\@warning}
% \DoNotIndex{\ }
% \DoNotIndex{\begingroup,\bfseries}
% \DoNotIndex{\csname}
% \DoNotIndex{\def,\do,\docdate}
% \DoNotIndex{\endcsname,\endgroup,\endinput,\endtrivlist,\expandafter}
% \DoNotIndex{\filedate,\fileversion}
% \DoNotIndex{\gdef,\global}
% \DoNotIndex{\hbox,\hskip}
% \DoNotIndex{\ignorespaces,\input,\itshape,\item}
% \DoNotIndex{\labelsep,\let,\llap}
% \DoNotIndex{\makeatletter}
% \DoNotIndex{\noexpand}
% \DoNotIndex{\refstepcounter,\relax,\rlap}
% \DoNotIndex{\setlength,\slshape,\space,\string,\strut}
% \DoNotIndex{\the,\trivlist,\typeout}
% \DoNotIndex{\vbox}
% \DoNotIndex{\xdef}
%
% \def\deflabel#1{\ttfamily #1\hfill}
% \def\deflist#1{\begin{list}{}{\settowidth\labelwidth{\ttfamily #1}%
%                               \setlength\leftmargin\labelwidth
%                               \addtolength\leftmargin\labelsep
%                               \let\makelabel\deflabel}}
% \def\enddeflist{\end{list}}
% \def\envfont{\normalfont\ttfamily}
%
% \MakeShortVerb{\|}
%
% \newcommand{\TUB}{{\sl TUGboat\/}}
%
% \GetFileInfo{theorem.dtx}
% \title{An Extension of the
%        \LaTeX{} theorem environment\thanks{This file
%        has version number \fileversion, last
%        revised \filedate.}}
% \author{Frank Mittelbach}
%
% \MaintainedByLaTeXTeam{tools}
% \maketitle
%
% \begin{abstract}
%    The macros described in this paper yield an extension of the
%    \LaTeX{} theorem mechanism. It is designed to satisfy the
%    different requirements of various journals. Thus, the
%    layout of the ``theorems'' can be manipulated by determining a
%    ``style''. This article describes not only the use, but
%    also the definition, of the necessary macros.
% \end{abstract}
%
% \section*{Preface to version 2.2}
%
% For \LaTeXe{} this package did not need any fundamental changes. I
% only modified the messages generated so that theorem layout styles
% will show up with the |\listfiles| command and cleaned the section
% on the New Font Selection Scheme since this is now included in
% \LaTeX.
%
% \section*{Preface to version 2.1}
%
% This version is identical to 2.0g described in \TUB\ 10\#3 except
% for some internal defaults which are now set depending on the used
% font selection scheme.
%
% This was done to avoid unpleasant surprises if the new font selection
% scheme is in force. For further details see
% section~\ref{sec:newfonts} and \cite{bk:GMS94}.
%
% \section{Introduction}
%
% For our purposes here, ``theorems'' are labelled enunciations,
% often set off from the main text by extra space and a font change.
% Theorems, corollaries, conjectures, definitions, and remarks
% are all instances of ``theorems''.  The ``header'' of these structures
% is composed of a label (such as \textsc{Theorem} or \textsc{Remark})
% and a number which serializes an item in the sequence of items
% with the same label.
%
% Shortly after the introduction of \LaTeX{} at the Fachbereich
% Mathematik in Mainz, the desire to manipulate the layout of
% ``theorems'' arose.  In Mainz, the following two conventions came
% into general use:
% \begin{enumerate}
%   \item The number of the theorem is shown in the margin.
%   \item There is a line break at
%         the end of the theorem header.
% \end{enumerate}
%
% Additionally, some journals require different formats which depend on
% the ``sort of theorem'': e.g.\ often remarks and definitions are set
% in |\upshape|, while |\itshape| is employed for main theorems.
%
% Confronted with these requirements, a theorem environment was
% developed in Mainz which allows separate determination of the
% layout of the ``theorems sets'', comparable to |\pagestyle|.
%
%
% \section{The user interface}
% \subsection{Defining new theorem sets}
%
% \DescribeMacro\newtheorem
% As in the original \LaTeX{} version, the command |\newtheorem|
% defines a new ``theorem set'' or ``theorem-like structure''.
% Two required arguments name the new environment and give the text
% to be typeset with each instance of the new ``set'', while
% an optional argument determines how
% the ``set'' is enumerated:
% \begin{description}
%    \item[\ttfamily \bslash newtheorem\{foo\}\{bar\}]
%       The theorem set {\envfont foo} (whose name is \texttt{bar})
%       uses its own counter.
%    \item[\ttfamily \bslash newtheorem\{foo2\}{[foo]}\{bar2\}]
%       The theorem set {\envfont foo2} (printed name \texttt{bar2})
%       uses the same counter as the theorem set \texttt{foo}.
%    \item[\ttfamily \bslash newtheorem\{foo3\}\{bar3\}{[section]}]
%       The theorem set {\envfont foo3} (printed name \texttt{bar3}) is
%       enumerated within the counter \texttt{section}, i.e.\ with every
%       new |\section| the enumeration begins again with 1, and
%       the enumeration is composed from the section-number and the
%       theorem counter itself.
% \end{description}
%
% \DescribeMacro\theoremstyle
% Additionally, the command |\theoremstyle| can define the layout
% of various, or all, theorem sets.  It should be noted that any
% theorem set defined by |\newtheorem| is typeset in the
% |\theoremstyle| that is current at the time of the definition.
% Thus, the following
% \begin{verbatim}
%    \theoremstyle{break}        \newtheorem{Cor}{Corollary}
%    \theoremstyle{plain}        \newtheorem{Exa}{Example}[section]
%\end{verbatim}                             ^^A No blank at line start
% leads to the result that the set {\envfont Cor} is formatted in the
% style \texttt{break}, while the set {\envfont Exa} and all the
% following ones are formatted in the style \texttt{plain}, unless
% another |\theoremstyle| follows.  Since the definitions installed by
% |\newtheorem| are global, one also can limit |\theoremstyle| locally
% by grouping braces.
%
% \DescribeMacro\theorembodyfont
% The choice of the font for the theorem body is completely
% independent of the chosen |\theoremstyle|; this has proven to
% be very advantageous.  For example,
% \begin{verbatim}
%   {\theorembodyfont{\upshape}          \newtheorem{Rem}{Remark}}
%\end{verbatim}
% defines a theorem set {\envfont Rem}, which will be set in
% |\upshape| in the current layout (which in our example is
% \texttt{plain}).  As with |\theoremstyle|, the |\theorembodyfont|
% chosen is that current at the time of |\newtheorem|.  If
% |\theorembodyfont| is not specified or one defines
% |\theorembodyfont{}|, then the font used will be that defined by the
% |\theoremstyle|.
%
% \DescribeMacro\theoremheaderfont It is also possible to customize
% the font used for the theorem headers.  This is, however, a global
% declaration, and therefore there should be at most one
% |\theoremheaderfont| declaration in the preamble.\footnote{If it is
% actually necessary to have different header fonts, one has to define
% new theorem styles (substituting the desired font) or specify the
% information directly in the \texttt{\bslash newtheorem} declaration
% (the unclean variant).}
%
% \DescribeMacro\theorempreskipamount
% \DescribeMacro\theorempostskipamount Two additional parameters
% affect the vertical space around the theorem environments:
% |\theorempreskipamount| and |\theorempostskipamount| define,
% respectively, the spacing before and after such an environment.
% These parameters apply for all theorem sets and can be manipulated
% with the ordinary length macros.  They are rubber lengths
% (`\textsf{skips}'), and therefore can contain \texttt{plus} and
% \texttt{minus} parts.
%
% Since the definition of theorem sets should---most sensibly---be
% placed in the preamble, we only allow installation there.
% It is therefore possible to release the memory used here after
% |\begin{document}|, in order to make room for other
% applications.
%
%
% \subsection{Existing theorem styles}
%
% The following theorem styles exist to date:
% \begin{deflist}{marginbreak}
%    \item[plain]
%       This theorem style emulates the original \LaTeX{} definition,
%       except that additionally the parameters
%       |\theorem...skipamount| are used.
%    \item[break]
%       In this style, the theorem header is followed by a line
%       break.
%    \item[marginbreak]
%       The theorem number is set in the margin, and there is a line
%       break as in \texttt{break}.
%    \item[changebreak]
%       Like \texttt{break}, but with header number and text
%       interchanged.
%    \item[change]
%       Header number and text are interchanged, without a line break.
%    \item[margin]
%       The number is set in the left margin, without a line break.
% \end{deflist}
% All styles (except \texttt{plain}) select |\slshape| as the default
% |\theorembodyfont|.
%
%
% \subsection{Examples}
%
% Given the above theorem sets {\envfont Cor}, {\envfont Exa}
% and {\envfont Rem},
% suppose that the preamble also contains the declarations:
% \begin{verbatim}
%    \theoremstyle{marginbreak}   \newtheorem{Lem}[Cor]{Lemma}
%    \theoremstyle{change}
%    \theorembodyfont{\itshape}        \newtheorem{Def}[Cor]{Definition}
%
%    \theoremheaderfont{\scshape}
%\end{verbatim}
% Then the following are some typical examples of the typeset output
% resulting from their use.
%
% \begin{Cor}
%    This is a sentence typeset in the theorem environment {\envfont
%    Cor}.
% \end{Cor}
%
% \begin{Exa}
%    This is a sentence typeset in the theorem environment {\envfont
%    Exa}.
% \end{Exa}
%
% \begin{Rem}
%    This is a sentence typeset in the theorem environment {\envfont
%    Rem}.
% \end{Rem}
%
% \begin{Lem}[Ben User]
%    This is a sentence typeset in the theorem environment {\envfont
%    Lem}.
% \end{Lem}
%
% \begin{Def}[Very impressive Definition]
%    This is a sentence typeset in the theorem environment {\envfont
%    Def}.
% \end{Def}
%
% The last two examples show the effect of the optional argument to
% a theorem environment (it is the text typeset in parentheses).
%
%
% \section{Special Considerations}
% \label{sec:newfonts}
%
% Theoremheader and body are implemented as a unit. This means that
% the |\theoremheaderfont|  will inherit characteristics of the
% |\theorembodyfont| in \LaTeXe. Thus, if for example
% |\theorembodyfont| is |\itshape| and |\theoremheaderfont|
% is |\bfseries| the font selected for the header will have the
% characteristics `bold extended italic'. If this is not desired one
% should set the |\theoremheaderfont| to something like
% \begin{verbatim}
%      \theoremheaderfont{\normalfont\bfseries}
%\end{verbatim}
% i.e.\ supplying all necessary font information explicitly.
%
%
%
% \section{Acknowledgements}
%
% The publication of this set of macros was only possible with the
% help of Christina Busse (translating the manuscript into English),
% Joachim Pense (playing the r\^ole of typist), Chris Rowley
% (looking everything over) and many others providing useful
% suggestions.
%
% \MaybeStop{%
%    \begin{thebibliography}{9}
%       \bibitem{bk:GMS94} \textsc{M.~Goossens}, \textsc{F.~Mittelbach}
%         and \textsc{A.~Samarin}.
%         \newblock The \LaTeX{} Companion.
%         \newblock
%         Addison-Wesley, Reading, Massachusetts, 1994.
%       \bibitem{latex:source} \textsc{Lamport, Leslie.} \newblock
%                               \texttt{latex.tex}, version 2.09, date
%                               Feb. 1990.
%    \end{thebibliography}
%    \PrintIndex
%    \PrintChanges}
%
%
% \section{The documentation driver file}
%
% The next bit of code contains the documentation driver file for
% \TeX{}, i.e., the file that will produce the documentation you are
% currently reading. It will be extracted from this file by the
% \texttt{docstrip} program. Since it is the first code in the file
% one can alternatively process this file directly with \LaTeXe{} to
% obtain the documentation.
%
%    \begin{macrocode}
%<*driver>
\documentclass{ltxdoc}

\usepackage{theorem}

%<+driver>% The next few lines define theorem sets which are used
%<+driver>% in the example section of the documentation.

\theoremstyle{break}       \newtheorem{Cor}{Corollary}
\theoremstyle{plain}       \newtheorem{Exa}{Example}[section]
{\theorembodyfont{\upshape}\newtheorem{Rem}{Remark}}
\theoremstyle{marginbreak} \newtheorem{Lem}[Cor]{Lemma}
\theoremstyle{change}
\theorembodyfont{\itshape} \newtheorem{Def}[Cor]{Definition}

\theoremheaderfont{\scshape}

\RecordChanges

\begin{document}
   \DocInput{theorem.dtx}
\end{document}
%</driver>
%    \end{macrocode}
%
% \section{Definition of the Macros}
%
% If the file has been loaded before, we abort immediately. If not the
% package announces itself (this is actually done at the very top of
% the file---the way it is done isn't good style so don't copy it).
% \changes{v2.0e}{89/07/19}{Spaces removed from `typeout'.}
% \changes{v2.2b}{95/11/19}{Announce moved to top of file}
%    \begin{macrocode}
%<*package>
%\@ifundefined{theorem@style}{}{\endinput}
%\def\FMithmInfo{1995/11/19 v2.2b Theorem extension package (FMi)}
%\ProvidesPackage{theorem}[\FMithmInfo]
%    \end{macrocode}
%
%
% \subsection{Definition of theorem styles and fonts}
%
% All the definitions in this file are done globally to allow
% inputting this file inside a group.
%
% \begin{macro}{\theoremstyle}
%    Before a theorem style can be installed, the chosen style must be
%    known. For that reason, we must test to see that
%    |\th@|\meta{style} is known or, more precisely, that it is
%    different from |\relax|. If the style is not known then
%    |\th@plain| is used.
%    \begin{macrocode}
\gdef\theoremstyle#1{%
   \@ifundefined{th@#1}{\@warning
          {Unknown theoremstyle `#1'. Using `plain'}%
          \theorem@style{plain}}%
%    \end{macrocode}
%    We save the theorem style to be used in the token register
%    |\theorem@style|.
%    \begin{macrocode}
      {\theorem@style{#1}}%
%    \end{macrocode}
%    Now we ``evaluate'' the theorem style: this means, we call the
%    macro |\th@|\meta{style} which will activate the relevant
%    definitions which are contained in a separate file.  This is done
%    in a group to suppress changes to the current font.  This could
%    otherwise pose problems together with the new font selection
%    scheme\footnote{When I printed the original article using the new
%    font selection scheme I ended with a document with slanted
%    typefaces (text headings and all) simply because one of the
%    theorem styles used \texttt{\bslash sl} at toplevel.} if the
%    |\th@|\meta{style} is evaluated a second time.
% \changes{v2.1a}{90/02/15}{Group added around `th@..}
% \changes{v2.1a}{90/02/15}{Reseting of `..@begintheorem removed.}
%    \begin{macrocode}
      \begingroup
        \csname th@\the\theorem@style \endcsname
      \endgroup}
%    \end{macrocode}
% \end{macro}
%
%
% \begin{macro}{\@begintheorem}
% \changes{v2.1a}{90/02/15}{Relaxed at toplevel.}
% \begin{macro}{\@opargbegintheorem}
% \changes{v2.1a}{90/02/15}{Relaxed at toplevel.}
%    We reset |\@begintheorem| and |\@opargbegintheorem| to
%    |\relax| since these commands are no longer necessary at
%    toplevel. This will save a few tokens.
%    \begin{macrocode}
\global\let\@begintheorem\relax
\global\let\@opargbegintheorem\relax
%    \end{macrocode}
% \end{macro}
% \end{macro}
%
% \begin{macro}{\theorem@style}
%    Obviously the \textsf{token} register used above has to be
%    allocated. To assure the utmost compatibility with the
%    original \LaTeX{} definition, we set the default theorem style to
%    \texttt{plain}, which implements the usual \LaTeX{} convention.
% \changes{v2.0f}{89/07/20}{Unnecessary `global' removed.}
%    \begin{macrocode}
\newtoks\theorem@style
\global\theorem@style{plain}
%    \end{macrocode}
% \end{macro}
%
% \begin{macro}{\theorembodyfont}
% \begin{macro}{\theorem@bodyfont}
%    If the |\theorembodyfont| is set by the user then it should not
%    interact with the default font set in the theorem style. When the
%    new font selection is in force this may happen if, for example,
%    the default is |\itshape| and the new |\theorembodyfont| is
%    |\sffamily|. So we add a |\reset@font| command in front of the
%    user definition.
% \changes{v2.0f}{89/07/20}{Unnecessary `global' removed.}
% \changes{v2.0g}{89/09/19}{`theoremfont to `theorembodyfont.}
% \changes{v2.1c}{92/06/22}{`reset@font added and logic changed.}
% \changes{v2.1d}{92/10/28}{Reset to empty if arg is empty}
%    \begin{macrocode}
\gdef\theorembodyfont#1{%
%    \end{macrocode}
%    We check if the argument supplied is empty and if so put nothing
%    into the |\theorem@bodyfont| token register to allow for
%    |\theorembodyfont{}| as a mean of using the default of the
%    current |\theoremstyle|.
%    \begin{macrocode}
   \def\@tempa{#1}%
   \ifx\@tempa\@empty
    \theorem@bodyfont{}%
   \else
    \theorem@bodyfont{\reset@font#1}%
   \fi
   }
\newtoks\theorem@bodyfont
\global\theorem@bodyfont{}
%    \end{macrocode}
% \end{macro}
% \end{macro}
%
%
% \begin{macro}{\theoremheaderfont}
% \changes{v2.0g}{89/09/19}{Macro added.}
%    The font for the theorem headers is handled differently because
%    this definition applies to all theorem styles.
%    \begin{macrocode}
\gdef\theoremheaderfont#1{\gdef\theorem@headerfont{#1}%
%    \end{macrocode}
%    After using the macro once it is redefined to produce an error
%    message.
%    \begin{macrocode}
       \gdef\theoremheaderfont##1{%
        \typeout{\string\theoremheaderfont\space should be used
                 only once.}}}
%    \end{macrocode}
% \end{macro}
%
% \begin{macro}{\theorem@headerfont}
% \changes{v2.0g}{89/09/19}{Macro added.}
%    To set the |\theorem@headerfont| default we first test if
%    the new fontselection scheme is in force.
% \changes{v2.1a}{90/02/15}{Support for new font selection.}
%    \begin{macrocode}
\ifx\upshape\undefined
%    \end{macrocode}
%    If not we define it to expand into |\bfseries|. We don't use
%    |\let| just in case a following style option redefines this
%    macro.
%    \begin{macrocode}
\gdef\theorem@headerfont{\bfseries}
%    \end{macrocode}
%    Otherwise we reset the current shape before calling |\bfseries|.
%    \begin{macrocode}
\else \gdef\theorem@headerfont{\normalfont\bfseries}\fi
%    \end{macrocode}
% \end{macro}
%
% \begin{macro}{\th@plain}
% \begin{macro}{\th@break}
% \begin{macro}{\th@marginbreak}
% \begin{macro}{\th@changebreak}
% \begin{macro}{\th@change}
% \begin{macro}{\th@margin}
%    The different styles are defined in macros such as
%    |\th@plain|.  Since memory space is precious in
%    ``non-Big-versions'', we have to avoid offering too many unused
%    definitions.  Therefore we define these styles in separate files
%    that can be loaded on demand.  Thus the commands themselves only
%    load these files. We use |\@input@|, a \LaTeXe{} internal command
%    that ensures that the file will be listed with |\listfiles|.
% \changes{v2.2a}{94/02/02}{Use `@input@ to load theorem layout files}
%    \begin{macrocode}
\gdef\th@plain{\@input@{thp.sty}}
\gdef\th@break{\@input@{thb.sty}}
\gdef\th@marginbreak{\@input@{thmb.sty}}
\gdef\th@changebreak{\@input@{thcb.sty}}
\gdef\th@change{\@input@{thc.sty}}
\gdef\th@margin{\@input@{thm.sty}}
%    \end{macrocode}
%    This list will be expanded when new styles become
%    available.  For testing,  just append new theorem substyles
%    as document options.
% \end{macro}
% \end{macro}
% \end{macro}
% \end{macro}
% \end{macro}
% \end{macro}
%
%
%
% \subsection{Definition of a new theorem set}
%
% As already pointed out, a new theorem environment can be defined in
% three different ways:
% \begin{verbatim}
%    \newtheorem{Lem}{Lemma}
%    \newtheorem{Lem}{Lemma}[section]
%    \newtheorem{Lem}[Theorem]{Lemma}
%\end{verbatim}
% The function of the macro |\newtheorem| is to recognize these cases
% and then to branch into one of the three macros |\@ynthm|, |\@xnthm|
% or |\@othm|.  This mechanism is adopted unchanged from
% \cite{latex:source}; the essential point here is that, for example,
% in the second case, the arguments \texttt{Lem}, \texttt{Lemma} and
% \texttt{section} are passed over to the macro |\@xnthm|.
%
% We inspect this case first because the others present fewer problems,
% and thus are easily derived from this one.
%
% \begin{macro}{\@xnthm}
%    For our example arguments, the macro |\@xnthm| must fulfill
%    the following:
%    \begin{itemize}
%       \item
%          Define a new \LaTeX-counter `Lem'
%       \item
%          reset this counter within a |\section|
%       \item
%          define the macro |\theLem|
%       \item
%          define the environment macros |\Lem| and
%          |\endLem| using the current
%          |\theoremstyle| and |\theorem@bodyfont|.
%    \end{itemize}
%    Obviously, all this should happen only if the first argument
%    of |\@xnthm| (i.e.\ \texttt{Lem} in our example) is chosen so
%    as not to conflict with any previously defined commands or
%    environments.  This test is performed by the \LaTeX{} macro
%    |\@ifdefinable|.
%    \begin{macrocode}
\gdef\@xnthm#1#2[#3]{\expandafter\@ifdefinable\csname #1\endcsname
%    \end{macrocode}
%    Therefore, the first argument of |\@ifdefinable| is the
%    expansion (in the example, |\Lem|) of
%    |\csname||#1\endcsname|.  The second argument is
%    executed only if the test has been completed successfully.
%    \begin{macrocode}
   {%
%    \end{macrocode}
%    Now we define the new counter. The names of the \LaTeX{}
%    macros employed should speak for themselves:
% \changes{v2.2b}{1995/10/07}{Use \cs{@newctr}}
% \changes{v2.2c}{1995/11/23}{Use \cs{@newctr} properly.}
%    \begin{macrocode}
    \@definecounter{#1}\@newctr{#1}[#3]%
%    \end{macrocode}
%    Using |\@newctr| will give a proper error message if the counter
%    in |#3| is not defined.
%    In defining `|\theLem|' we must generate the desired macro
%    name by use of |\expandafter| and |\csname|.
%    \begin{macrocode}
    \expandafter\xdef\csname the#1\endcsname
%    \end{macrocode}
%    An |\xdef| is used in order to make the definition global,
%    and to ensure that it contains the replacement texts of
%    |\@thmcountersep| and |\@thmcounter|.\footnote{These
%    two macros can be defined by the document style.  Their default
%    values produce a `\texttt{.}' as separation and an arabic
%    representation of the number.} However, not everything should be
%    expanded. For example, it saves space to use |\thesection|
%    instead of its---at times---lengthy expansion.
%    \begin{macrocode}
      {\expandafter \noexpand \csname the#3\endcsname
       \@thmcountersep \@thmcounter{#1}}%
%    \end{macrocode}
%    Thus with the defaults of \LaTeX{}, |\theLem| would be
%    replaced by the command sequence |\thesection.\arabic{Lem}|.
%
%    We will now look at the definition of the macro which is executed
%    at the beginning of the actual environment (in our example this
%    macro is |\Lem|). It should be noted that we use an
%    ``|\expandafter| trick'' to expand only certain parts of the
%    replacement text at the time of the definition.
%    \begin{macrocode}
    \def\@tempa{\global\@namedef{#1}}%
    \expandafter \@tempa \expandafter{%
%    \end{macrocode}
%    First, the macro that contains the current definitions of
%    |\@begintheorem| and |\@opargtheorem| should be called
%    up.  The name of this macro---as is already known---has the form
%    |\th@|\meta{theorem\ style}; therefore, it must be called by
%    \begin{macrocode}
      \csname th@\the \theorem@style
%    \end{macrocode}
%    In addition the default theorem font should be changeable, i.e.\
%    we have to insert the contents of |\theorem@bodyfont|. For
%    that reason, we expand even further, beyond |\endcsname|,
%    and thus insert the contents of the \textsf{token} register
%    |\theorem@bodyfont| in the replacement text.
%    \begin{macrocode}
            \expandafter \endcsname \the \theorem@bodyfont
%    \end{macrocode}
%    Now it is time to call the macro |\@thm| which takes over
%    the further processing.  It has two arguments: the current
%    counter name (in our example, \texttt{Lem}), and the text of the
%    label (in our example, \texttt{Lemma}).
%    \begin{macrocode}
     \@thm{#1}{#2}}%
%    \end{macrocode}
%    With this, the `sub-definition' is complete.  The macro
%    |\@endtheorem| ends a theorem environment and is, so far, nothing
%    but an |\endtrivlist|.  (Hence it is defined globally, and not
%    within the theorem styles.\footnote{This has to be changed as
%    soon as theorem styles that change \texttt{\bslash @endtheorem}
%    exist.  In such a case, all existing styles must be changed as
%    well since they will have to reset the macro.}) Therefore, we can
%    set it equivalent to the macro that ends the theorem set (in our
%    example, |\endLem|).  However, if some day theorem styles exist
%    that do change |\@endtheorem|, we would have to use the
%    commented-out line instead.
%    \begin{macrocode}
    \global \expandafter \let \csname end#1\endcsname \@endtheorem
%   \global\@namedef{end#1}{\@endtheorem}%
%    \end{macrocode}
%    With these commands all the required definitions are employed,
%    unless the test |\@ifdefinable| has failed.  Therefore, we
%    end the second argument of this macro and with it the definition
%    of |\@xnthm|.
%    \begin{macrocode}
   }}
%    \end{macrocode}
% \end{macro}
%
% \begin{macro}{\@ynthm}
%    The definition of |\@ynthm| is completely analogous. In
%    this case the new counter that is defined is not reset within
%    another counter; thus the definition of |\the...|
%    is simplified:
%    \begin{macrocode}
\gdef\@ynthm#1#2{\expandafter\@ifdefinable\csname #1\endcsname
   {\@definecounter{#1}%
    \expandafter\xdef\csname the#1\endcsname{\@thmcounter{#1}}%
%    \end{macrocode}
%    The rest of the definition corresponds literally to that of
%    |\@xnthm|:
%    \begin{macrocode}
    \def\@tempa{\global\@namedef{#1}}\expandafter \@tempa
     \expandafter{\csname th@\the \theorem@style \expandafter
     \endcsname \the\theorem@bodyfont \@thm{#1}{#2}}%
    \global \expandafter \let \csname end#1\endcsname \@endtheorem}}
%    \end{macrocode}
% \end{macro}
%
% \begin{macro}{\@othm}
%    The definition of |\@othm| does not contain anything new.
%    \begin{macrocode}
\gdef\@othm#1[#2]#3{%
%    \end{macrocode}
% \changes{v2.2b}{95/10/07}{Build in check for counter defined}
%    We do not define a new counter but instead use one that has
%    already been defined.  Thus the only definition we need is that
%    of this pseudo-counter (i.e.\ |\the|\meta{env.\ name}).
%    First we check if |#2| corresponds to a known counter name.
%    \begin{macrocode}
  \expandafter\ifx\csname c@#2\endcsname\relax
   \@nocounterr{#2}%
  \else
   \expandafter\@ifdefinable\csname #1\endcsname
   {\expandafter \xdef \csname the#1\endcsname
     {\expandafter \noexpand \csname the#2\endcsname}%
%    \end{macrocode}
%    All other parts of the definition can be adopted from
%    |\@xnthm|. We have to remember, though, that in this case
%    the name of the current counter and the theorem label have moved
%    to the second and third arguments.
%    \begin{macrocode}
    \def\@tempa{\global\@namedef{#1}}\expandafter \@tempa
     \expandafter{\csname th@\the \theorem@style \expandafter
     \endcsname \the\theorem@bodyfont \@thm{#2}{#3}}%
    \global \expandafter \let \csname end#1\endcsname \@endtheorem}%
  \fi}
%    \end{macrocode}
% \end{macro}
%
% \subsection{Macros that are employed in a {\envfont theorem}
%    environment}
%
% \begin{macro}{\@thm}
%    The macro |\@thm| has to increase the current counter.  Then,
%    depending on whether the environment has (or does not have) an
%    optional argument, it has to branch into either |\@begintheorem|
%    or |\@opargtheorem|.
%    \begin{macrocode}
\gdef\@thm#1#2{\refstepcounter{#1}%
%    \end{macrocode}
%    Now we start a {\envfont trivlist} environment, and give
%    |\@topsep| and |\@topsepadd| the values of the \textsf{skip}
%    registers |\theorempreskipamount| and |\theorempostskipamount|.
%    The value in |\@topsep| is the vertical space that is inserted by
%    the first (and only) |\item| in our |\trivlist| whilst
%    |\@topsepadd| is inserted by |\@endparenv| at the end of that
%    {\envfont trivlist} environment.  By using these registers, we
%    obtain the desired space around a {\envfont theorem} environment.
%    \begin{macrocode}
   \trivlist
   \@topsep \theorempreskipamount               % used by first \item
   \@topsepadd \theorempostskipamount           % used by \@endparenv
%    \end{macrocode}
%    Now we have to test whether an optional argument has been given.
% \changes{v2.0d}{89/07/18}{Forgotten percent added.}
%    \begin{macrocode}
   \@ifnextchar [%
%    \end{macrocode}
%    If there is an optional argument, we will call |\@ythm|, and
%    move the arguments read back into the input stream.
%    \begin{macrocode}
   {\@ythm{#1}{#2}}%
%    \end{macrocode}
%    If not, we call |\@begintheorem|. Its first argument is the
%    name of the theorem set (hence the second argument of
%    |\@thm|). Its second argument is the macro that produces the
%    current number.
%    \begin{macrocode}
   {\@begintheorem{#2}{\csname the#1\endcsname}\ignorespaces}}
%    \end{macrocode}
% \end{macro}
%
% \begin{macro}{\@xthm}
% \begin{macro}{\@ythm}
%    Both these macros were originally called by |\@thm|.  We do
%    not need |\@xthm| anymore, hence we reset it to
%    |\relax|. The definition of |\@ythm| has not changed at
%    all from its definition in \LaTeX. In order to make the macros
%    easier to understand, we will
%    nevertheless present it (commented out).
%    \begin{macrocode}
\global\let\@xthm\relax
% \def\@ythm#1#2[#3]{\@opargbegintheorem{#2}{\csname
%           the#1\endcsname}{#3}\ignorespaces}
%</package>
%    \end{macrocode}
%    The primitive |\ignorespaces| in |\@ythm| and
%    |\@thm| is needed to remove the spaces between the
%    |\begin{...}| and the actual text.
% \end{macro}
% \end{macro}
%
%
% \subsection{Definition of the theorem substyles}
%
% As already pointed out, the theorem substyles, defined below, are
% only loaded when necessary.  Note that all these substyles, except
% \texttt{plain}, have |\slshape| as the default body font.
%
% \subsubsection{The \texttt{plain} style}
%
% \label{dok:th@plain} As the following macros use |@|, we have
% to locally set the |\catcode| of this symbol to ``letter''.
% This happens within a group, so that we do not have to worry about
% which |\catcode| that symbol had before.
%    \begin{macrocode}
%<*thp>
\begingroup \makeatletter
%    \end{macrocode}
% Since we are now within a group, we must make all definitions
% globally.  First we make sure that \texttt{theorem.sty} is loaded.
% This will allow us to use this file as a document style option
% without having to call \texttt{theorem} itself as an option.  At the
% same time, we assure that at least version 2 is loaded, since
% |\theorem@style| was not defined in earlier versions.
%    \begin{macrocode}
\@ifundefined{theorem@style}{\input{theorem.sty}}{}
\ProvidesFile{thp.sty}
             [\FMithmInfo]
%    \end{macrocode}
%
%
% \begin{macro}{\th@plain}
% \begin{macro}{\@begintheorem}
% \begin{macro}{\@opargbegintheorem}
%    |\theoremstyle{plain}| corresponds to the original
%    definition, except that the distances to the surrounding text are
%    determined by the parameters |\theorempreskipamount| and
%    |\theorempostskipamount|.
%    First we set the default body font.
%    \begin{macrocode}
\gdef\th@plain{\normalfont\itshape
%    \end{macrocode}
%    Then we define |\@begintheorem| and
%    |\@opargbegintheorem|. These two macros define how the
%    header of a theorem is typeset.  |\@opargbegintheorem| will
%    be called if a \textsf{theorem} environment with an optional
%    argument is encountered; otherwise, the header is constructed by
%    calling |\@begintheorem|.  If one of these macros is
%    executed, we are within a \textsf{trivlist} environment started by
%    |\@thm|.  So the theorem header is produced with an
%    |\item| command.
%
%    Instead of specifying the header font directly, all standard
%    theorem styles use the |\theorem@headerfont| macro to allow
%    customization.  The extra space (|\labelsep|) is necessary
%    because of problems in the \textsf{trivlist} environment.
%    \begin{macrocode}
  \def\@begintheorem##1##2{%
        \item[\hskip\labelsep \theorem@headerfont ##1\ ##2]}%
%    \end{macrocode}
%    The definition of |\@opargbegintheorem| is completely
%    analogous. The only difference is the fact that there exists a
%    third argument (which is the optional parameter of the
%    environment and contains additional information about the
%    theorem).  Customarily we enclose it in parentheses.
%    \begin{macrocode}
\def\@opargbegintheorem##1##2##3{%
   \item[\hskip\labelsep \theorem@headerfont ##1\ ##2\ (##3)]}}
%    \end{macrocode}
% \end{macro}
% \end{macro}
% \end{macro}
%
% \noindent
% We conclude with an |\endgroup| to restore the
% |\catcode| of |@|.
%    \begin{macrocode}
\endgroup
%</thp>
%    \end{macrocode}
%
%
% \subsubsection{The \texttt{break} style}
%
% This style option is stored in the file |thb.sty|.
% For the next two lines see the documentation for |\th@plain| on
% page~\pageref{dok:th@plain}.
%    \begin{macrocode}
%<*thb>
\begingroup \makeatletter
\@ifundefined{theorem@style}{\input{theorem.sty}}{}
\ProvidesFile{thb.sty}
             [\FMithmInfo]
%    \end{macrocode}
%
% \begin{macro}{\th@break}
%    |\theoremstyle{break}| produces a line break after the name
%    of the theorem.  The font is |\slshape|.  Hence, we define
%    |\th@break| as follows:
%    \begin{macrocode}
\gdef\th@break{\normalfont\slshape
  \def\@begintheorem##1##2{\item[%
%    \end{macrocode}
%    We run into the following problem: it is not possible to create
%    the header with |\item[|\meta{title}|]| and then start
%    a new line by, for example, |\mbox{}\\|. Such a definition
%    will fail whenever a list environment follows immediately.  With
%    the above construction, the |\mbox{}| causes the switch
%    |@inlabel| (cf.\ definition of |\list| and
%    |\trivlist| in \cite{latex:source}) to be set to \texttt{false}
%    and so the following list will insert additional vertical
%    space (|\topskip|). This is quite annoying. Therefore, we
%    create the line break within the |\item|. In order to ensure
%    that the text will begin at the proper position in the following
%    line, we simply pretend that the label does not take any
%    room.\footnote{This will lead to problems whenever very high
%    symbols occurring in the line tower into the
%    heading. So, something else has to be done here sometime.}
%    \begin{macrocode}
       \rlap{\vbox{\hbox{\hskip \labelsep\theorem@headerfont ##1\ ##2}%
                   \hbox{\strut}}}]}%
%    \end{macrocode}
%    Again, the definition of |\@opargbegintheorem| is
%    completely analogous.
%    \begin{macrocode}
\def\@opargbegintheorem##1##2##3{%
  \item[\rlap{\vbox{\hbox{\hskip \labelsep \theorem@headerfont
                     ##1\ ##2\ (##3)}%
                    \hbox{\strut}}}]}}
\endgroup
%</thb>
%    \end{macrocode}
% \end{macro}
%
%
% \subsubsection{The \texttt{changebreak} style}
%
%    \begin{macrocode}
% This style option is stored in the file |thcb.sty|.
%    \begin{macrocode}
%<*thcb>
\begingroup \makeatletter
\@ifundefined{theorem@style}{\input{theorem.sty}}{}
\ProvidesFile{thcb.sty}
             [\FMithmInfo]
%    \end{macrocode}
%
% \begin{macro}{\th@changebreak}
%    The \texttt{change-break} theorem style is like \texttt{break}
%    but with interchange of theorem name and theorem number.  Thus we
%    define |\th@changebreak| as follows:
%    \begin{macrocode}
\gdef\th@changebreak{\normalfont\slshape
  \def\@begintheorem##1##2{\item
       [\rlap{\vbox{\hbox{\hskip\labelsep \theorem@headerfont ##2\ ##1}%
                    \hbox{\strut}}}]}%
\def\@opargbegintheorem##1##2##3{%
  \item[\rlap{\vbox{\hbox{\hskip\labelsep \theorem@headerfont
                     ##2\ ##1\ (##3)}%
                    \hbox{\strut}}}]}}
\endgroup
%</thcb>
%    \end{macrocode}
% \end{macro}
%
%
% \subsubsection{The \texttt{change} style}
%
% This style option is stored in the file |thc.sty|.
%    \begin{macrocode}
%<*thc>
\begingroup \makeatletter
\@ifundefined{theorem@style}{\input{theorem.sty}}{}
\ProvidesFile{thc.sty}
             [\FMithmInfo]
%    \end{macrocode}
%
% \begin{macro}{\th@change}
%    The \texttt{change} theorem style corresponds to the
%    \texttt{change break} style without a linebreak after the header.
%    To say it in another way, it's the same as the \texttt{plain}
%    style but with number and name interchanged and |\slshape| as the
%    default font.
%    \begin{macrocode}
\gdef\th@change{\normalfont\slshape
  \def\@begintheorem##1##2{\item
       [\hskip\labelsep \theorem@headerfont ##2\ ##1]}%
\def\@opargbegintheorem##1##2##3{%
  \item[\hskip\labelsep \theorem@headerfont ##2\ ##1\ (##3)]}}
\endgroup
%</thc>
%    \end{macrocode}
% \end{macro}
%
%
% \subsubsection{The \texttt{marginbreak} style}
%
% This style option is the one used most often at Mainz.  It is saved
% in the file |thmb.sty|.
%    \begin{macrocode}
%<*thmb>
\begingroup \makeatletter
\@ifundefined{theorem@style}{\input{theorem.sty}}{}
\ProvidesFile{thmb.sty}
             [\FMithmInfo]
%    \end{macrocode}
%
% \begin{macro}{\th@marginbreak}
%    The \texttt{margin break} style is nearly the same as the
%    \texttt{change break} style. The only difference is the placement
%    of the theorem number.  We use |\llap| to place it in the left
%    margin.
%
%    In this style |\labelsep| denotes the separation between the
%    number and the text.
%    \begin{macrocode}
\gdef\th@marginbreak{\normalfont\slshape
  \def\@begintheorem##1##2{\item
       [\rlap{\vbox{\theorem@headerfont
                    \hbox{\llap{##2}\hskip\labelsep ##1}%
                    \hbox{\strut}}}]}%
\def\@opargbegintheorem##1##2##3{%
  \item[\rlap{\vbox{\theorem@headerfont
                    \hbox{\llap{##2}\hskip\labelsep ##1\ (##3)}%
                    \hbox{\strut}}}]}}
\endgroup
%</thmb>
%    \end{macrocode}
% \changes{v2.1c}{90/02/17}{`endgroup outside macrocode env.}
% \end{macro}
%
%
%
% \subsubsection{The \texttt{margin} style}
%
% This style option is stored in the file |thm.sty|.
%    \begin{macrocode}
%<*thm>
\begingroup \makeatletter
\@ifundefined{theorem@style}{\input{theorem.sty}}{}
\ProvidesFile{thm.sty}
             [\FMithmInfo]
%    \end{macrocode}
%
% \begin{macro}{\th@margin}
%    Again this is only a variant of the theorem styles described
%    above without any new ideas.
%    \begin{macrocode}
\gdef\th@margin{\normalfont\slshape
  \def\@begintheorem##1##2{\item
       [\theorem@headerfont \llap{##2}\hskip\labelsep ##1]}%
\def\@opargbegintheorem##1##2##3{%
  \item[\theorem@headerfont \llap{##2}\hskip\labelsep ##1\ (##3)]}}
\endgroup
%</thm>
%    \end{macrocode}
% \end{macro}
%
%
%
%
% \subsection{Final Definitions}
%
% \begin{macro}{\theorempreskipamount}
% \begin{macro}{\theorempostskipamount}
%    The \textsf{skip} parameters that regulate the vertical empty space
%    before and after the {\envfont theorem} environment have
%    to be allocated as well.
% \changes{v2.0f}{89/07/20}{Unnecessary `global' removed.}
%    \begin{macrocode}
%<*package>
\newskip\theorempreskipamount
\newskip\theorempostskipamount
%    \end{macrocode}
%    Since we have used the same values for all theorem sets, we now can
%    assign them.
%    \begin{macrocode}
\global\setlength\theorempreskipamount{12pt plus 5pt minus 3pt}
\global\setlength\theorempostskipamount{8pt plus 3pt minus 1.5pt}
%    \end{macrocode}
% \end{macro}
% \end{macro}
%
%
% \begin{macro}{\@endtheorem}
%    The same holds for the macro |\@endtheorem|, which ends a
%    {\envfont theorem} environment. Since it is the same for all
%    theorem sets, it is removed from the macros |\th@|\meta{style}.
%    It simply ends the {\envfont trivlist} environment, which was
%    begun in |\@thm|.
%    \begin{macrocode}
\global\let\@endtheorem=\endtrivlist
%    \end{macrocode}
% \end{macro}
%
%
%
% \begin{macro}{\@preamblecmds}
%    All macros defined above are to be used only in the preamble.
%    Therefore, we insert them in |\@preamblecmds| which will disable
%    them at begin document. This is done by the internal \LaTeXe{}
%    command |\@onlypreamble|.
% \changes{v2.0g}{89/09/19}{`theoremheaderfont and `theorembodyfont
%    added.}
% \changes{v2.2a}{94/02/03}{Use \LaTeXe{} interface}
%    \begin{macrocode}
\@onlypreamble\@xnthm
\@onlypreamble\@ynthm
\@onlypreamble\@othm
\@onlypreamble\newtheorem
\@onlypreamble\theoremstyle
\@onlypreamble\theorembodyfont
\@onlypreamble\theoremheaderfont
%    \end{macrocode}
% \end{macro}
%
% Finally we declare the \texttt{plain} theorem style to be the default.
% \changes{v2.1b}{90/06/04}{Fixed missing `item by declaring default
%    style.}
%    \begin{macrocode}
\theoremstyle{plain}
%</package>
%    \end{macrocode}
%
% \Finale

\endinput
}{}
\ProvidesFile{thmb.sty}
             [\FMithmInfo]
%    \end{macrocode}
%
% \begin{macro}{\th@marginbreak}
%    The \texttt{margin break} style is nearly the same as the
%    \texttt{change break} style. The only difference is the placement
%    of the theorem number.  We use |\llap| to place it in the left
%    margin.
%
%    In this style |\labelsep| denotes the separation between the
%    number and the text.
%    \begin{macrocode}
\gdef\th@marginbreak{\normalfont\slshape
  \def\@begintheorem##1##2{\item
       [\rlap{\vbox{\theorem@headerfont
                    \hbox{\llap{##2}\hskip\labelsep ##1}%
                    \hbox{\strut}}}]}%
\def\@opargbegintheorem##1##2##3{%
  \item[\rlap{\vbox{\theorem@headerfont
                    \hbox{\llap{##2}\hskip\labelsep ##1\ (##3)}%
                    \hbox{\strut}}}]}}
\endgroup
%</thmb>
%    \end{macrocode}
% \changes{v2.1c}{90/02/17}{`endgroup outside macrocode env.}
% \end{macro}
%
%
%
% \subsubsection{The \texttt{margin} style}
%
% This style option is stored in the file |thm.sty|.
%    \begin{macrocode}
%<*thm>
\begingroup \makeatletter
\@ifundefined{theorem@style}{% \iffalse meta-comment
%
% Copyright (C) 1993-2024
%
% The LaTeX Project and any individual authors listed elsewhere
% in this file.
%
% This file is part of the Standard LaTeX `Tools Bundle'.
% -------------------------------------------------------
%
% It may be distributed and/or modified under the
% conditions of the LaTeX Project Public License, either version 1.3c
% of this license or (at your option) any later version.
% The latest version of this license is in
%    https://www.latex-project.org/lppl.txt
% and version 1.3c or later is part of all distributions of LaTeX
% version 2005/12/01 or later.
%
% The list of all files belonging to the LaTeX `Tools Bundle' is
% given in the file `manifest.txt'.
%
% \fi
%
%%
%
%
\def\FMithmInfo{2023/07/05 v2.2c Theorem extension package (FMi)}
%
% \ProvidesFile{theorem.dtx}[\FMithmInfo]
%\iffalse   % this is a METACOMMENT !
%%
%% Package `theorem' to use with LaTeX2e
%% Copyright (C) 1989-1995 Frank Mittelbach, all rights reserved.
%<+package>\@ifundefined{theorem@style}{}{\endinput}
%<+package>\NeedsTeXFormat{LaTeX2e}[1995/06/01]
%<+package>\ProvidesPackage{theorem}
%<+package>         [\FMithmInfo]
%
% \fi
%
% \changes{v1.0}{87/05/07}{First draft.}
% \changes{v1.0a}{87/05/07}{`thistheoremstyle' added.}
% \changes{v1.1a}{87/07/29}{New concept: redefine the internal
%                          \protect\LaTeX{} commands `@xnthm' etc.}
% \changes{v1.1b}{87/07/31}{`theoremfont' used to specify font.}
% \changes{v1.2a}{88/01/08}{`@endtheorem' now globally defined.}
% \changes{v1.2b}{88/01/11}{`newtheorem' only allowed in the preamble.}
% \changes{v1.3a}{88/02/14}{`theoremfont' now a toks register.}
%
% \changes{v2.0a}{88/11/16}{New implementation and new documentation.}
% \changes{v2.0a}{88/11/16}{No layout changes but internal modifications
%                           old substyles won't work.}
% \changes{v2.0b}{89/01/30}{`global' used for all definitions.}
% \changes{v2.0c}{89/06/15}{Documentation translated to English.
%                            Thanks to LLL and Nina}
% \changes{v2.1a}{90/02/15}{Support for new font selection.}
% \changes{v2.2a}{94/02/03}{Cleanup for LaTeX2e}
%
% \DoNotIndex{\@endparenv,\@ifdefinable,\@ifnextchar,\@ifundefined}
% \DoNotIndex{\@namedef,\@nameuse,\@spaces,\@tempa,\@warning}
% \DoNotIndex{\ }
% \DoNotIndex{\begingroup,\bfseries}
% \DoNotIndex{\csname}
% \DoNotIndex{\def,\do,\docdate}
% \DoNotIndex{\endcsname,\endgroup,\endinput,\endtrivlist,\expandafter}
% \DoNotIndex{\filedate,\fileversion}
% \DoNotIndex{\gdef,\global}
% \DoNotIndex{\hbox,\hskip}
% \DoNotIndex{\ignorespaces,\input,\itshape,\item}
% \DoNotIndex{\labelsep,\let,\llap}
% \DoNotIndex{\makeatletter}
% \DoNotIndex{\noexpand}
% \DoNotIndex{\refstepcounter,\relax,\rlap}
% \DoNotIndex{\setlength,\slshape,\space,\string,\strut}
% \DoNotIndex{\the,\trivlist,\typeout}
% \DoNotIndex{\vbox}
% \DoNotIndex{\xdef}
%
% \def\deflabel#1{\ttfamily #1\hfill}
% \def\deflist#1{\begin{list}{}{\settowidth\labelwidth{\ttfamily #1}%
%                               \setlength\leftmargin\labelwidth
%                               \addtolength\leftmargin\labelsep
%                               \let\makelabel\deflabel}}
% \def\enddeflist{\end{list}}
% \def\envfont{\normalfont\ttfamily}
%
% \MakeShortVerb{\|}
%
% \newcommand{\TUB}{{\sl TUGboat\/}}
%
% \GetFileInfo{theorem.dtx}
% \title{An Extension of the
%        \LaTeX{} theorem environment\thanks{This file
%        has version number \fileversion, last
%        revised \filedate.}}
% \author{Frank Mittelbach}
%
% \MaintainedByLaTeXTeam{tools}
% \maketitle
%
% \begin{abstract}
%    The macros described in this paper yield an extension of the
%    \LaTeX{} theorem mechanism. It is designed to satisfy the
%    different requirements of various journals. Thus, the
%    layout of the ``theorems'' can be manipulated by determining a
%    ``style''. This article describes not only the use, but
%    also the definition, of the necessary macros.
% \end{abstract}
%
% \section*{Preface to version 2.2}
%
% For \LaTeXe{} this package did not need any fundamental changes. I
% only modified the messages generated so that theorem layout styles
% will show up with the |\listfiles| command and cleaned the section
% on the New Font Selection Scheme since this is now included in
% \LaTeX.
%
% \section*{Preface to version 2.1}
%
% This version is identical to 2.0g described in \TUB\ 10\#3 except
% for some internal defaults which are now set depending on the used
% font selection scheme.
%
% This was done to avoid unpleasant surprises if the new font selection
% scheme is in force. For further details see
% section~\ref{sec:newfonts} and \cite{bk:GMS94}.
%
% \section{Introduction}
%
% For our purposes here, ``theorems'' are labelled enunciations,
% often set off from the main text by extra space and a font change.
% Theorems, corollaries, conjectures, definitions, and remarks
% are all instances of ``theorems''.  The ``header'' of these structures
% is composed of a label (such as \textsc{Theorem} or \textsc{Remark})
% and a number which serializes an item in the sequence of items
% with the same label.
%
% Shortly after the introduction of \LaTeX{} at the Fachbereich
% Mathematik in Mainz, the desire to manipulate the layout of
% ``theorems'' arose.  In Mainz, the following two conventions came
% into general use:
% \begin{enumerate}
%   \item The number of the theorem is shown in the margin.
%   \item There is a line break at
%         the end of the theorem header.
% \end{enumerate}
%
% Additionally, some journals require different formats which depend on
% the ``sort of theorem'': e.g.\ often remarks and definitions are set
% in |\upshape|, while |\itshape| is employed for main theorems.
%
% Confronted with these requirements, a theorem environment was
% developed in Mainz which allows separate determination of the
% layout of the ``theorems sets'', comparable to |\pagestyle|.
%
%
% \section{The user interface}
% \subsection{Defining new theorem sets}
%
% \DescribeMacro\newtheorem
% As in the original \LaTeX{} version, the command |\newtheorem|
% defines a new ``theorem set'' or ``theorem-like structure''.
% Two required arguments name the new environment and give the text
% to be typeset with each instance of the new ``set'', while
% an optional argument determines how
% the ``set'' is enumerated:
% \begin{description}
%    \item[\ttfamily \bslash newtheorem\{foo\}\{bar\}]
%       The theorem set {\envfont foo} (whose name is \texttt{bar})
%       uses its own counter.
%    \item[\ttfamily \bslash newtheorem\{foo2\}{[foo]}\{bar2\}]
%       The theorem set {\envfont foo2} (printed name \texttt{bar2})
%       uses the same counter as the theorem set \texttt{foo}.
%    \item[\ttfamily \bslash newtheorem\{foo3\}\{bar3\}{[section]}]
%       The theorem set {\envfont foo3} (printed name \texttt{bar3}) is
%       enumerated within the counter \texttt{section}, i.e.\ with every
%       new |\section| the enumeration begins again with 1, and
%       the enumeration is composed from the section-number and the
%       theorem counter itself.
% \end{description}
%
% \DescribeMacro\theoremstyle
% Additionally, the command |\theoremstyle| can define the layout
% of various, or all, theorem sets.  It should be noted that any
% theorem set defined by |\newtheorem| is typeset in the
% |\theoremstyle| that is current at the time of the definition.
% Thus, the following
% \begin{verbatim}
%    \theoremstyle{break}        \newtheorem{Cor}{Corollary}
%    \theoremstyle{plain}        \newtheorem{Exa}{Example}[section]
%\end{verbatim}                             ^^A No blank at line start
% leads to the result that the set {\envfont Cor} is formatted in the
% style \texttt{break}, while the set {\envfont Exa} and all the
% following ones are formatted in the style \texttt{plain}, unless
% another |\theoremstyle| follows.  Since the definitions installed by
% |\newtheorem| are global, one also can limit |\theoremstyle| locally
% by grouping braces.
%
% \DescribeMacro\theorembodyfont
% The choice of the font for the theorem body is completely
% independent of the chosen |\theoremstyle|; this has proven to
% be very advantageous.  For example,
% \begin{verbatim}
%   {\theorembodyfont{\upshape}          \newtheorem{Rem}{Remark}}
%\end{verbatim}
% defines a theorem set {\envfont Rem}, which will be set in
% |\upshape| in the current layout (which in our example is
% \texttt{plain}).  As with |\theoremstyle|, the |\theorembodyfont|
% chosen is that current at the time of |\newtheorem|.  If
% |\theorembodyfont| is not specified or one defines
% |\theorembodyfont{}|, then the font used will be that defined by the
% |\theoremstyle|.
%
% \DescribeMacro\theoremheaderfont It is also possible to customize
% the font used for the theorem headers.  This is, however, a global
% declaration, and therefore there should be at most one
% |\theoremheaderfont| declaration in the preamble.\footnote{If it is
% actually necessary to have different header fonts, one has to define
% new theorem styles (substituting the desired font) or specify the
% information directly in the \texttt{\bslash newtheorem} declaration
% (the unclean variant).}
%
% \DescribeMacro\theorempreskipamount
% \DescribeMacro\theorempostskipamount Two additional parameters
% affect the vertical space around the theorem environments:
% |\theorempreskipamount| and |\theorempostskipamount| define,
% respectively, the spacing before and after such an environment.
% These parameters apply for all theorem sets and can be manipulated
% with the ordinary length macros.  They are rubber lengths
% (`\textsf{skips}'), and therefore can contain \texttt{plus} and
% \texttt{minus} parts.
%
% Since the definition of theorem sets should---most sensibly---be
% placed in the preamble, we only allow installation there.
% It is therefore possible to release the memory used here after
% |\begin{document}|, in order to make room for other
% applications.
%
%
% \subsection{Existing theorem styles}
%
% The following theorem styles exist to date:
% \begin{deflist}{marginbreak}
%    \item[plain]
%       This theorem style emulates the original \LaTeX{} definition,
%       except that additionally the parameters
%       |\theorem...skipamount| are used.
%    \item[break]
%       In this style, the theorem header is followed by a line
%       break.
%    \item[marginbreak]
%       The theorem number is set in the margin, and there is a line
%       break as in \texttt{break}.
%    \item[changebreak]
%       Like \texttt{break}, but with header number and text
%       interchanged.
%    \item[change]
%       Header number and text are interchanged, without a line break.
%    \item[margin]
%       The number is set in the left margin, without a line break.
% \end{deflist}
% All styles (except \texttt{plain}) select |\slshape| as the default
% |\theorembodyfont|.
%
%
% \subsection{Examples}
%
% Given the above theorem sets {\envfont Cor}, {\envfont Exa}
% and {\envfont Rem},
% suppose that the preamble also contains the declarations:
% \begin{verbatim}
%    \theoremstyle{marginbreak}   \newtheorem{Lem}[Cor]{Lemma}
%    \theoremstyle{change}
%    \theorembodyfont{\itshape}        \newtheorem{Def}[Cor]{Definition}
%
%    \theoremheaderfont{\scshape}
%\end{verbatim}
% Then the following are some typical examples of the typeset output
% resulting from their use.
%
% \begin{Cor}
%    This is a sentence typeset in the theorem environment {\envfont
%    Cor}.
% \end{Cor}
%
% \begin{Exa}
%    This is a sentence typeset in the theorem environment {\envfont
%    Exa}.
% \end{Exa}
%
% \begin{Rem}
%    This is a sentence typeset in the theorem environment {\envfont
%    Rem}.
% \end{Rem}
%
% \begin{Lem}[Ben User]
%    This is a sentence typeset in the theorem environment {\envfont
%    Lem}.
% \end{Lem}
%
% \begin{Def}[Very impressive Definition]
%    This is a sentence typeset in the theorem environment {\envfont
%    Def}.
% \end{Def}
%
% The last two examples show the effect of the optional argument to
% a theorem environment (it is the text typeset in parentheses).
%
%
% \section{Special Considerations}
% \label{sec:newfonts}
%
% Theoremheader and body are implemented as a unit. This means that
% the |\theoremheaderfont|  will inherit characteristics of the
% |\theorembodyfont| in \LaTeXe. Thus, if for example
% |\theorembodyfont| is |\itshape| and |\theoremheaderfont|
% is |\bfseries| the font selected for the header will have the
% characteristics `bold extended italic'. If this is not desired one
% should set the |\theoremheaderfont| to something like
% \begin{verbatim}
%      \theoremheaderfont{\normalfont\bfseries}
%\end{verbatim}
% i.e.\ supplying all necessary font information explicitly.
%
%
%
% \section{Acknowledgements}
%
% The publication of this set of macros was only possible with the
% help of Christina Busse (translating the manuscript into English),
% Joachim Pense (playing the r\^ole of typist), Chris Rowley
% (looking everything over) and many others providing useful
% suggestions.
%
% \MaybeStop{%
%    \begin{thebibliography}{9}
%       \bibitem{bk:GMS94} \textsc{M.~Goossens}, \textsc{F.~Mittelbach}
%         and \textsc{A.~Samarin}.
%         \newblock The \LaTeX{} Companion.
%         \newblock
%         Addison-Wesley, Reading, Massachusetts, 1994.
%       \bibitem{latex:source} \textsc{Lamport, Leslie.} \newblock
%                               \texttt{latex.tex}, version 2.09, date
%                               Feb. 1990.
%    \end{thebibliography}
%    \PrintIndex
%    \PrintChanges}
%
%
% \section{The documentation driver file}
%
% The next bit of code contains the documentation driver file for
% \TeX{}, i.e., the file that will produce the documentation you are
% currently reading. It will be extracted from this file by the
% \texttt{docstrip} program. Since it is the first code in the file
% one can alternatively process this file directly with \LaTeXe{} to
% obtain the documentation.
%
%    \begin{macrocode}
%<*driver>
\documentclass{ltxdoc}

\usepackage{theorem}

%<+driver>% The next few lines define theorem sets which are used
%<+driver>% in the example section of the documentation.

\theoremstyle{break}       \newtheorem{Cor}{Corollary}
\theoremstyle{plain}       \newtheorem{Exa}{Example}[section]
{\theorembodyfont{\upshape}\newtheorem{Rem}{Remark}}
\theoremstyle{marginbreak} \newtheorem{Lem}[Cor]{Lemma}
\theoremstyle{change}
\theorembodyfont{\itshape} \newtheorem{Def}[Cor]{Definition}

\theoremheaderfont{\scshape}

\RecordChanges

\begin{document}
   \DocInput{theorem.dtx}
\end{document}
%</driver>
%    \end{macrocode}
%
% \section{Definition of the Macros}
%
% If the file has been loaded before, we abort immediately. If not the
% package announces itself (this is actually done at the very top of
% the file---the way it is done isn't good style so don't copy it).
% \changes{v2.0e}{89/07/19}{Spaces removed from `typeout'.}
% \changes{v2.2b}{95/11/19}{Announce moved to top of file}
%    \begin{macrocode}
%<*package>
%\@ifundefined{theorem@style}{}{\endinput}
%\def\FMithmInfo{1995/11/19 v2.2b Theorem extension package (FMi)}
%\ProvidesPackage{theorem}[\FMithmInfo]
%    \end{macrocode}
%
%
% \subsection{Definition of theorem styles and fonts}
%
% All the definitions in this file are done globally to allow
% inputting this file inside a group.
%
% \begin{macro}{\theoremstyle}
%    Before a theorem style can be installed, the chosen style must be
%    known. For that reason, we must test to see that
%    |\th@|\meta{style} is known or, more precisely, that it is
%    different from |\relax|. If the style is not known then
%    |\th@plain| is used.
%    \begin{macrocode}
\gdef\theoremstyle#1{%
   \@ifundefined{th@#1}{\@warning
          {Unknown theoremstyle `#1'. Using `plain'}%
          \theorem@style{plain}}%
%    \end{macrocode}
%    We save the theorem style to be used in the token register
%    |\theorem@style|.
%    \begin{macrocode}
      {\theorem@style{#1}}%
%    \end{macrocode}
%    Now we ``evaluate'' the theorem style: this means, we call the
%    macro |\th@|\meta{style} which will activate the relevant
%    definitions which are contained in a separate file.  This is done
%    in a group to suppress changes to the current font.  This could
%    otherwise pose problems together with the new font selection
%    scheme\footnote{When I printed the original article using the new
%    font selection scheme I ended with a document with slanted
%    typefaces (text headings and all) simply because one of the
%    theorem styles used \texttt{\bslash sl} at toplevel.} if the
%    |\th@|\meta{style} is evaluated a second time.
% \changes{v2.1a}{90/02/15}{Group added around `th@..}
% \changes{v2.1a}{90/02/15}{Reseting of `..@begintheorem removed.}
%    \begin{macrocode}
      \begingroup
        \csname th@\the\theorem@style \endcsname
      \endgroup}
%    \end{macrocode}
% \end{macro}
%
%
% \begin{macro}{\@begintheorem}
% \changes{v2.1a}{90/02/15}{Relaxed at toplevel.}
% \begin{macro}{\@opargbegintheorem}
% \changes{v2.1a}{90/02/15}{Relaxed at toplevel.}
%    We reset |\@begintheorem| and |\@opargbegintheorem| to
%    |\relax| since these commands are no longer necessary at
%    toplevel. This will save a few tokens.
%    \begin{macrocode}
\global\let\@begintheorem\relax
\global\let\@opargbegintheorem\relax
%    \end{macrocode}
% \end{macro}
% \end{macro}
%
% \begin{macro}{\theorem@style}
%    Obviously the \textsf{token} register used above has to be
%    allocated. To assure the utmost compatibility with the
%    original \LaTeX{} definition, we set the default theorem style to
%    \texttt{plain}, which implements the usual \LaTeX{} convention.
% \changes{v2.0f}{89/07/20}{Unnecessary `global' removed.}
%    \begin{macrocode}
\newtoks\theorem@style
\global\theorem@style{plain}
%    \end{macrocode}
% \end{macro}
%
% \begin{macro}{\theorembodyfont}
% \begin{macro}{\theorem@bodyfont}
%    If the |\theorembodyfont| is set by the user then it should not
%    interact with the default font set in the theorem style. When the
%    new font selection is in force this may happen if, for example,
%    the default is |\itshape| and the new |\theorembodyfont| is
%    |\sffamily|. So we add a |\reset@font| command in front of the
%    user definition.
% \changes{v2.0f}{89/07/20}{Unnecessary `global' removed.}
% \changes{v2.0g}{89/09/19}{`theoremfont to `theorembodyfont.}
% \changes{v2.1c}{92/06/22}{`reset@font added and logic changed.}
% \changes{v2.1d}{92/10/28}{Reset to empty if arg is empty}
%    \begin{macrocode}
\gdef\theorembodyfont#1{%
%    \end{macrocode}
%    We check if the argument supplied is empty and if so put nothing
%    into the |\theorem@bodyfont| token register to allow for
%    |\theorembodyfont{}| as a mean of using the default of the
%    current |\theoremstyle|.
%    \begin{macrocode}
   \def\@tempa{#1}%
   \ifx\@tempa\@empty
    \theorem@bodyfont{}%
   \else
    \theorem@bodyfont{\reset@font#1}%
   \fi
   }
\newtoks\theorem@bodyfont
\global\theorem@bodyfont{}
%    \end{macrocode}
% \end{macro}
% \end{macro}
%
%
% \begin{macro}{\theoremheaderfont}
% \changes{v2.0g}{89/09/19}{Macro added.}
%    The font for the theorem headers is handled differently because
%    this definition applies to all theorem styles.
%    \begin{macrocode}
\gdef\theoremheaderfont#1{\gdef\theorem@headerfont{#1}%
%    \end{macrocode}
%    After using the macro once it is redefined to produce an error
%    message.
%    \begin{macrocode}
       \gdef\theoremheaderfont##1{%
        \typeout{\string\theoremheaderfont\space should be used
                 only once.}}}
%    \end{macrocode}
% \end{macro}
%
% \begin{macro}{\theorem@headerfont}
% \changes{v2.0g}{89/09/19}{Macro added.}
%    To set the |\theorem@headerfont| default we first test if
%    the new fontselection scheme is in force.
% \changes{v2.1a}{90/02/15}{Support for new font selection.}
%    \begin{macrocode}
\ifx\upshape\undefined
%    \end{macrocode}
%    If not we define it to expand into |\bfseries|. We don't use
%    |\let| just in case a following style option redefines this
%    macro.
%    \begin{macrocode}
\gdef\theorem@headerfont{\bfseries}
%    \end{macrocode}
%    Otherwise we reset the current shape before calling |\bfseries|.
%    \begin{macrocode}
\else \gdef\theorem@headerfont{\normalfont\bfseries}\fi
%    \end{macrocode}
% \end{macro}
%
% \begin{macro}{\th@plain}
% \begin{macro}{\th@break}
% \begin{macro}{\th@marginbreak}
% \begin{macro}{\th@changebreak}
% \begin{macro}{\th@change}
% \begin{macro}{\th@margin}
%    The different styles are defined in macros such as
%    |\th@plain|.  Since memory space is precious in
%    ``non-Big-versions'', we have to avoid offering too many unused
%    definitions.  Therefore we define these styles in separate files
%    that can be loaded on demand.  Thus the commands themselves only
%    load these files. We use |\@input@|, a \LaTeXe{} internal command
%    that ensures that the file will be listed with |\listfiles|.
% \changes{v2.2a}{94/02/02}{Use `@input@ to load theorem layout files}
%    \begin{macrocode}
\gdef\th@plain{\@input@{thp.sty}}
\gdef\th@break{\@input@{thb.sty}}
\gdef\th@marginbreak{\@input@{thmb.sty}}
\gdef\th@changebreak{\@input@{thcb.sty}}
\gdef\th@change{\@input@{thc.sty}}
\gdef\th@margin{\@input@{thm.sty}}
%    \end{macrocode}
%    This list will be expanded when new styles become
%    available.  For testing,  just append new theorem substyles
%    as document options.
% \end{macro}
% \end{macro}
% \end{macro}
% \end{macro}
% \end{macro}
% \end{macro}
%
%
%
% \subsection{Definition of a new theorem set}
%
% As already pointed out, a new theorem environment can be defined in
% three different ways:
% \begin{verbatim}
%    \newtheorem{Lem}{Lemma}
%    \newtheorem{Lem}{Lemma}[section]
%    \newtheorem{Lem}[Theorem]{Lemma}
%\end{verbatim}
% The function of the macro |\newtheorem| is to recognize these cases
% and then to branch into one of the three macros |\@ynthm|, |\@xnthm|
% or |\@othm|.  This mechanism is adopted unchanged from
% \cite{latex:source}; the essential point here is that, for example,
% in the second case, the arguments \texttt{Lem}, \texttt{Lemma} and
% \texttt{section} are passed over to the macro |\@xnthm|.
%
% We inspect this case first because the others present fewer problems,
% and thus are easily derived from this one.
%
% \begin{macro}{\@xnthm}
%    For our example arguments, the macro |\@xnthm| must fulfill
%    the following:
%    \begin{itemize}
%       \item
%          Define a new \LaTeX-counter `Lem'
%       \item
%          reset this counter within a |\section|
%       \item
%          define the macro |\theLem|
%       \item
%          define the environment macros |\Lem| and
%          |\endLem| using the current
%          |\theoremstyle| and |\theorem@bodyfont|.
%    \end{itemize}
%    Obviously, all this should happen only if the first argument
%    of |\@xnthm| (i.e.\ \texttt{Lem} in our example) is chosen so
%    as not to conflict with any previously defined commands or
%    environments.  This test is performed by the \LaTeX{} macro
%    |\@ifdefinable|.
%    \begin{macrocode}
\gdef\@xnthm#1#2[#3]{\expandafter\@ifdefinable\csname #1\endcsname
%    \end{macrocode}
%    Therefore, the first argument of |\@ifdefinable| is the
%    expansion (in the example, |\Lem|) of
%    |\csname||#1\endcsname|.  The second argument is
%    executed only if the test has been completed successfully.
%    \begin{macrocode}
   {%
%    \end{macrocode}
%    Now we define the new counter. The names of the \LaTeX{}
%    macros employed should speak for themselves:
% \changes{v2.2b}{1995/10/07}{Use \cs{@newctr}}
% \changes{v2.2c}{1995/11/23}{Use \cs{@newctr} properly.}
%    \begin{macrocode}
    \@definecounter{#1}\@newctr{#1}[#3]%
%    \end{macrocode}
%    Using |\@newctr| will give a proper error message if the counter
%    in |#3| is not defined.
%    In defining `|\theLem|' we must generate the desired macro
%    name by use of |\expandafter| and |\csname|.
%    \begin{macrocode}
    \expandafter\xdef\csname the#1\endcsname
%    \end{macrocode}
%    An |\xdef| is used in order to make the definition global,
%    and to ensure that it contains the replacement texts of
%    |\@thmcountersep| and |\@thmcounter|.\footnote{These
%    two macros can be defined by the document style.  Their default
%    values produce a `\texttt{.}' as separation and an arabic
%    representation of the number.} However, not everything should be
%    expanded. For example, it saves space to use |\thesection|
%    instead of its---at times---lengthy expansion.
%    \begin{macrocode}
      {\expandafter \noexpand \csname the#3\endcsname
       \@thmcountersep \@thmcounter{#1}}%
%    \end{macrocode}
%    Thus with the defaults of \LaTeX{}, |\theLem| would be
%    replaced by the command sequence |\thesection.\arabic{Lem}|.
%
%    We will now look at the definition of the macro which is executed
%    at the beginning of the actual environment (in our example this
%    macro is |\Lem|). It should be noted that we use an
%    ``|\expandafter| trick'' to expand only certain parts of the
%    replacement text at the time of the definition.
%    \begin{macrocode}
    \def\@tempa{\global\@namedef{#1}}%
    \expandafter \@tempa \expandafter{%
%    \end{macrocode}
%    First, the macro that contains the current definitions of
%    |\@begintheorem| and |\@opargtheorem| should be called
%    up.  The name of this macro---as is already known---has the form
%    |\th@|\meta{theorem\ style}; therefore, it must be called by
%    \begin{macrocode}
      \csname th@\the \theorem@style
%    \end{macrocode}
%    In addition the default theorem font should be changeable, i.e.\
%    we have to insert the contents of |\theorem@bodyfont|. For
%    that reason, we expand even further, beyond |\endcsname|,
%    and thus insert the contents of the \textsf{token} register
%    |\theorem@bodyfont| in the replacement text.
%    \begin{macrocode}
            \expandafter \endcsname \the \theorem@bodyfont
%    \end{macrocode}
%    Now it is time to call the macro |\@thm| which takes over
%    the further processing.  It has two arguments: the current
%    counter name (in our example, \texttt{Lem}), and the text of the
%    label (in our example, \texttt{Lemma}).
%    \begin{macrocode}
     \@thm{#1}{#2}}%
%    \end{macrocode}
%    With this, the `sub-definition' is complete.  The macro
%    |\@endtheorem| ends a theorem environment and is, so far, nothing
%    but an |\endtrivlist|.  (Hence it is defined globally, and not
%    within the theorem styles.\footnote{This has to be changed as
%    soon as theorem styles that change \texttt{\bslash @endtheorem}
%    exist.  In such a case, all existing styles must be changed as
%    well since they will have to reset the macro.}) Therefore, we can
%    set it equivalent to the macro that ends the theorem set (in our
%    example, |\endLem|).  However, if some day theorem styles exist
%    that do change |\@endtheorem|, we would have to use the
%    commented-out line instead.
%    \begin{macrocode}
    \global \expandafter \let \csname end#1\endcsname \@endtheorem
%   \global\@namedef{end#1}{\@endtheorem}%
%    \end{macrocode}
%    With these commands all the required definitions are employed,
%    unless the test |\@ifdefinable| has failed.  Therefore, we
%    end the second argument of this macro and with it the definition
%    of |\@xnthm|.
%    \begin{macrocode}
   }}
%    \end{macrocode}
% \end{macro}
%
% \begin{macro}{\@ynthm}
%    The definition of |\@ynthm| is completely analogous. In
%    this case the new counter that is defined is not reset within
%    another counter; thus the definition of |\the...|
%    is simplified:
%    \begin{macrocode}
\gdef\@ynthm#1#2{\expandafter\@ifdefinable\csname #1\endcsname
   {\@definecounter{#1}%
    \expandafter\xdef\csname the#1\endcsname{\@thmcounter{#1}}%
%    \end{macrocode}
%    The rest of the definition corresponds literally to that of
%    |\@xnthm|:
%    \begin{macrocode}
    \def\@tempa{\global\@namedef{#1}}\expandafter \@tempa
     \expandafter{\csname th@\the \theorem@style \expandafter
     \endcsname \the\theorem@bodyfont \@thm{#1}{#2}}%
    \global \expandafter \let \csname end#1\endcsname \@endtheorem}}
%    \end{macrocode}
% \end{macro}
%
% \begin{macro}{\@othm}
%    The definition of |\@othm| does not contain anything new.
%    \begin{macrocode}
\gdef\@othm#1[#2]#3{%
%    \end{macrocode}
% \changes{v2.2b}{95/10/07}{Build in check for counter defined}
%    We do not define a new counter but instead use one that has
%    already been defined.  Thus the only definition we need is that
%    of this pseudo-counter (i.e.\ |\the|\meta{env.\ name}).
%    First we check if |#2| corresponds to a known counter name.
%    \begin{macrocode}
  \expandafter\ifx\csname c@#2\endcsname\relax
   \@nocounterr{#2}%
  \else
   \expandafter\@ifdefinable\csname #1\endcsname
   {\expandafter \xdef \csname the#1\endcsname
     {\expandafter \noexpand \csname the#2\endcsname}%
%    \end{macrocode}
%    All other parts of the definition can be adopted from
%    |\@xnthm|. We have to remember, though, that in this case
%    the name of the current counter and the theorem label have moved
%    to the second and third arguments.
%    \begin{macrocode}
    \def\@tempa{\global\@namedef{#1}}\expandafter \@tempa
     \expandafter{\csname th@\the \theorem@style \expandafter
     \endcsname \the\theorem@bodyfont \@thm{#2}{#3}}%
    \global \expandafter \let \csname end#1\endcsname \@endtheorem}%
  \fi}
%    \end{macrocode}
% \end{macro}
%
% \subsection{Macros that are employed in a {\envfont theorem}
%    environment}
%
% \begin{macro}{\@thm}
%    The macro |\@thm| has to increase the current counter.  Then,
%    depending on whether the environment has (or does not have) an
%    optional argument, it has to branch into either |\@begintheorem|
%    or |\@opargtheorem|.
%    \begin{macrocode}
\gdef\@thm#1#2{\refstepcounter{#1}%
%    \end{macrocode}
%    Now we start a {\envfont trivlist} environment, and give
%    |\@topsep| and |\@topsepadd| the values of the \textsf{skip}
%    registers |\theorempreskipamount| and |\theorempostskipamount|.
%    The value in |\@topsep| is the vertical space that is inserted by
%    the first (and only) |\item| in our |\trivlist| whilst
%    |\@topsepadd| is inserted by |\@endparenv| at the end of that
%    {\envfont trivlist} environment.  By using these registers, we
%    obtain the desired space around a {\envfont theorem} environment.
%    \begin{macrocode}
   \trivlist
   \@topsep \theorempreskipamount               % used by first \item
   \@topsepadd \theorempostskipamount           % used by \@endparenv
%    \end{macrocode}
%    Now we have to test whether an optional argument has been given.
% \changes{v2.0d}{89/07/18}{Forgotten percent added.}
%    \begin{macrocode}
   \@ifnextchar [%
%    \end{macrocode}
%    If there is an optional argument, we will call |\@ythm|, and
%    move the arguments read back into the input stream.
%    \begin{macrocode}
   {\@ythm{#1}{#2}}%
%    \end{macrocode}
%    If not, we call |\@begintheorem|. Its first argument is the
%    name of the theorem set (hence the second argument of
%    |\@thm|). Its second argument is the macro that produces the
%    current number.
%    \begin{macrocode}
   {\@begintheorem{#2}{\csname the#1\endcsname}\ignorespaces}}
%    \end{macrocode}
% \end{macro}
%
% \begin{macro}{\@xthm}
% \begin{macro}{\@ythm}
%    Both these macros were originally called by |\@thm|.  We do
%    not need |\@xthm| anymore, hence we reset it to
%    |\relax|. The definition of |\@ythm| has not changed at
%    all from its definition in \LaTeX. In order to make the macros
%    easier to understand, we will
%    nevertheless present it (commented out).
%    \begin{macrocode}
\global\let\@xthm\relax
% \def\@ythm#1#2[#3]{\@opargbegintheorem{#2}{\csname
%           the#1\endcsname}{#3}\ignorespaces}
%</package>
%    \end{macrocode}
%    The primitive |\ignorespaces| in |\@ythm| and
%    |\@thm| is needed to remove the spaces between the
%    |\begin{...}| and the actual text.
% \end{macro}
% \end{macro}
%
%
% \subsection{Definition of the theorem substyles}
%
% As already pointed out, the theorem substyles, defined below, are
% only loaded when necessary.  Note that all these substyles, except
% \texttt{plain}, have |\slshape| as the default body font.
%
% \subsubsection{The \texttt{plain} style}
%
% \label{dok:th@plain} As the following macros use |@|, we have
% to locally set the |\catcode| of this symbol to ``letter''.
% This happens within a group, so that we do not have to worry about
% which |\catcode| that symbol had before.
%    \begin{macrocode}
%<*thp>
\begingroup \makeatletter
%    \end{macrocode}
% Since we are now within a group, we must make all definitions
% globally.  First we make sure that \texttt{theorem.sty} is loaded.
% This will allow us to use this file as a document style option
% without having to call \texttt{theorem} itself as an option.  At the
% same time, we assure that at least version 2 is loaded, since
% |\theorem@style| was not defined in earlier versions.
%    \begin{macrocode}
\@ifundefined{theorem@style}{\input{theorem.sty}}{}
\ProvidesFile{thp.sty}
             [\FMithmInfo]
%    \end{macrocode}
%
%
% \begin{macro}{\th@plain}
% \begin{macro}{\@begintheorem}
% \begin{macro}{\@opargbegintheorem}
%    |\theoremstyle{plain}| corresponds to the original
%    definition, except that the distances to the surrounding text are
%    determined by the parameters |\theorempreskipamount| and
%    |\theorempostskipamount|.
%    First we set the default body font.
%    \begin{macrocode}
\gdef\th@plain{\normalfont\itshape
%    \end{macrocode}
%    Then we define |\@begintheorem| and
%    |\@opargbegintheorem|. These two macros define how the
%    header of a theorem is typeset.  |\@opargbegintheorem| will
%    be called if a \textsf{theorem} environment with an optional
%    argument is encountered; otherwise, the header is constructed by
%    calling |\@begintheorem|.  If one of these macros is
%    executed, we are within a \textsf{trivlist} environment started by
%    |\@thm|.  So the theorem header is produced with an
%    |\item| command.
%
%    Instead of specifying the header font directly, all standard
%    theorem styles use the |\theorem@headerfont| macro to allow
%    customization.  The extra space (|\labelsep|) is necessary
%    because of problems in the \textsf{trivlist} environment.
%    \begin{macrocode}
  \def\@begintheorem##1##2{%
        \item[\hskip\labelsep \theorem@headerfont ##1\ ##2]}%
%    \end{macrocode}
%    The definition of |\@opargbegintheorem| is completely
%    analogous. The only difference is the fact that there exists a
%    third argument (which is the optional parameter of the
%    environment and contains additional information about the
%    theorem).  Customarily we enclose it in parentheses.
%    \begin{macrocode}
\def\@opargbegintheorem##1##2##3{%
   \item[\hskip\labelsep \theorem@headerfont ##1\ ##2\ (##3)]}}
%    \end{macrocode}
% \end{macro}
% \end{macro}
% \end{macro}
%
% \noindent
% We conclude with an |\endgroup| to restore the
% |\catcode| of |@|.
%    \begin{macrocode}
\endgroup
%</thp>
%    \end{macrocode}
%
%
% \subsubsection{The \texttt{break} style}
%
% This style option is stored in the file |thb.sty|.
% For the next two lines see the documentation for |\th@plain| on
% page~\pageref{dok:th@plain}.
%    \begin{macrocode}
%<*thb>
\begingroup \makeatletter
\@ifundefined{theorem@style}{\input{theorem.sty}}{}
\ProvidesFile{thb.sty}
             [\FMithmInfo]
%    \end{macrocode}
%
% \begin{macro}{\th@break}
%    |\theoremstyle{break}| produces a line break after the name
%    of the theorem.  The font is |\slshape|.  Hence, we define
%    |\th@break| as follows:
%    \begin{macrocode}
\gdef\th@break{\normalfont\slshape
  \def\@begintheorem##1##2{\item[%
%    \end{macrocode}
%    We run into the following problem: it is not possible to create
%    the header with |\item[|\meta{title}|]| and then start
%    a new line by, for example, |\mbox{}\\|. Such a definition
%    will fail whenever a list environment follows immediately.  With
%    the above construction, the |\mbox{}| causes the switch
%    |@inlabel| (cf.\ definition of |\list| and
%    |\trivlist| in \cite{latex:source}) to be set to \texttt{false}
%    and so the following list will insert additional vertical
%    space (|\topskip|). This is quite annoying. Therefore, we
%    create the line break within the |\item|. In order to ensure
%    that the text will begin at the proper position in the following
%    line, we simply pretend that the label does not take any
%    room.\footnote{This will lead to problems whenever very high
%    symbols occurring in the line tower into the
%    heading. So, something else has to be done here sometime.}
%    \begin{macrocode}
       \rlap{\vbox{\hbox{\hskip \labelsep\theorem@headerfont ##1\ ##2}%
                   \hbox{\strut}}}]}%
%    \end{macrocode}
%    Again, the definition of |\@opargbegintheorem| is
%    completely analogous.
%    \begin{macrocode}
\def\@opargbegintheorem##1##2##3{%
  \item[\rlap{\vbox{\hbox{\hskip \labelsep \theorem@headerfont
                     ##1\ ##2\ (##3)}%
                    \hbox{\strut}}}]}}
\endgroup
%</thb>
%    \end{macrocode}
% \end{macro}
%
%
% \subsubsection{The \texttt{changebreak} style}
%
%    \begin{macrocode}
% This style option is stored in the file |thcb.sty|.
%    \begin{macrocode}
%<*thcb>
\begingroup \makeatletter
\@ifundefined{theorem@style}{\input{theorem.sty}}{}
\ProvidesFile{thcb.sty}
             [\FMithmInfo]
%    \end{macrocode}
%
% \begin{macro}{\th@changebreak}
%    The \texttt{change-break} theorem style is like \texttt{break}
%    but with interchange of theorem name and theorem number.  Thus we
%    define |\th@changebreak| as follows:
%    \begin{macrocode}
\gdef\th@changebreak{\normalfont\slshape
  \def\@begintheorem##1##2{\item
       [\rlap{\vbox{\hbox{\hskip\labelsep \theorem@headerfont ##2\ ##1}%
                    \hbox{\strut}}}]}%
\def\@opargbegintheorem##1##2##3{%
  \item[\rlap{\vbox{\hbox{\hskip\labelsep \theorem@headerfont
                     ##2\ ##1\ (##3)}%
                    \hbox{\strut}}}]}}
\endgroup
%</thcb>
%    \end{macrocode}
% \end{macro}
%
%
% \subsubsection{The \texttt{change} style}
%
% This style option is stored in the file |thc.sty|.
%    \begin{macrocode}
%<*thc>
\begingroup \makeatletter
\@ifundefined{theorem@style}{\input{theorem.sty}}{}
\ProvidesFile{thc.sty}
             [\FMithmInfo]
%    \end{macrocode}
%
% \begin{macro}{\th@change}
%    The \texttt{change} theorem style corresponds to the
%    \texttt{change break} style without a linebreak after the header.
%    To say it in another way, it's the same as the \texttt{plain}
%    style but with number and name interchanged and |\slshape| as the
%    default font.
%    \begin{macrocode}
\gdef\th@change{\normalfont\slshape
  \def\@begintheorem##1##2{\item
       [\hskip\labelsep \theorem@headerfont ##2\ ##1]}%
\def\@opargbegintheorem##1##2##3{%
  \item[\hskip\labelsep \theorem@headerfont ##2\ ##1\ (##3)]}}
\endgroup
%</thc>
%    \end{macrocode}
% \end{macro}
%
%
% \subsubsection{The \texttt{marginbreak} style}
%
% This style option is the one used most often at Mainz.  It is saved
% in the file |thmb.sty|.
%    \begin{macrocode}
%<*thmb>
\begingroup \makeatletter
\@ifundefined{theorem@style}{\input{theorem.sty}}{}
\ProvidesFile{thmb.sty}
             [\FMithmInfo]
%    \end{macrocode}
%
% \begin{macro}{\th@marginbreak}
%    The \texttt{margin break} style is nearly the same as the
%    \texttt{change break} style. The only difference is the placement
%    of the theorem number.  We use |\llap| to place it in the left
%    margin.
%
%    In this style |\labelsep| denotes the separation between the
%    number and the text.
%    \begin{macrocode}
\gdef\th@marginbreak{\normalfont\slshape
  \def\@begintheorem##1##2{\item
       [\rlap{\vbox{\theorem@headerfont
                    \hbox{\llap{##2}\hskip\labelsep ##1}%
                    \hbox{\strut}}}]}%
\def\@opargbegintheorem##1##2##3{%
  \item[\rlap{\vbox{\theorem@headerfont
                    \hbox{\llap{##2}\hskip\labelsep ##1\ (##3)}%
                    \hbox{\strut}}}]}}
\endgroup
%</thmb>
%    \end{macrocode}
% \changes{v2.1c}{90/02/17}{`endgroup outside macrocode env.}
% \end{macro}
%
%
%
% \subsubsection{The \texttt{margin} style}
%
% This style option is stored in the file |thm.sty|.
%    \begin{macrocode}
%<*thm>
\begingroup \makeatletter
\@ifundefined{theorem@style}{\input{theorem.sty}}{}
\ProvidesFile{thm.sty}
             [\FMithmInfo]
%    \end{macrocode}
%
% \begin{macro}{\th@margin}
%    Again this is only a variant of the theorem styles described
%    above without any new ideas.
%    \begin{macrocode}
\gdef\th@margin{\normalfont\slshape
  \def\@begintheorem##1##2{\item
       [\theorem@headerfont \llap{##2}\hskip\labelsep ##1]}%
\def\@opargbegintheorem##1##2##3{%
  \item[\theorem@headerfont \llap{##2}\hskip\labelsep ##1\ (##3)]}}
\endgroup
%</thm>
%    \end{macrocode}
% \end{macro}
%
%
%
%
% \subsection{Final Definitions}
%
% \begin{macro}{\theorempreskipamount}
% \begin{macro}{\theorempostskipamount}
%    The \textsf{skip} parameters that regulate the vertical empty space
%    before and after the {\envfont theorem} environment have
%    to be allocated as well.
% \changes{v2.0f}{89/07/20}{Unnecessary `global' removed.}
%    \begin{macrocode}
%<*package>
\newskip\theorempreskipamount
\newskip\theorempostskipamount
%    \end{macrocode}
%    Since we have used the same values for all theorem sets, we now can
%    assign them.
%    \begin{macrocode}
\global\setlength\theorempreskipamount{12pt plus 5pt minus 3pt}
\global\setlength\theorempostskipamount{8pt plus 3pt minus 1.5pt}
%    \end{macrocode}
% \end{macro}
% \end{macro}
%
%
% \begin{macro}{\@endtheorem}
%    The same holds for the macro |\@endtheorem|, which ends a
%    {\envfont theorem} environment. Since it is the same for all
%    theorem sets, it is removed from the macros |\th@|\meta{style}.
%    It simply ends the {\envfont trivlist} environment, which was
%    begun in |\@thm|.
%    \begin{macrocode}
\global\let\@endtheorem=\endtrivlist
%    \end{macrocode}
% \end{macro}
%
%
%
% \begin{macro}{\@preamblecmds}
%    All macros defined above are to be used only in the preamble.
%    Therefore, we insert them in |\@preamblecmds| which will disable
%    them at begin document. This is done by the internal \LaTeXe{}
%    command |\@onlypreamble|.
% \changes{v2.0g}{89/09/19}{`theoremheaderfont and `theorembodyfont
%    added.}
% \changes{v2.2a}{94/02/03}{Use \LaTeXe{} interface}
%    \begin{macrocode}
\@onlypreamble\@xnthm
\@onlypreamble\@ynthm
\@onlypreamble\@othm
\@onlypreamble\newtheorem
\@onlypreamble\theoremstyle
\@onlypreamble\theorembodyfont
\@onlypreamble\theoremheaderfont
%    \end{macrocode}
% \end{macro}
%
% Finally we declare the \texttt{plain} theorem style to be the default.
% \changes{v2.1b}{90/06/04}{Fixed missing `item by declaring default
%    style.}
%    \begin{macrocode}
\theoremstyle{plain}
%</package>
%    \end{macrocode}
%
% \Finale

\endinput
}{}
\ProvidesFile{thm.sty}
             [\FMithmInfo]
%    \end{macrocode}
%
% \begin{macro}{\th@margin}
%    Again this is only a variant of the theorem styles described
%    above without any new ideas.
%    \begin{macrocode}
\gdef\th@margin{\normalfont\slshape
  \def\@begintheorem##1##2{\item
       [\theorem@headerfont \llap{##2}\hskip\labelsep ##1]}%
\def\@opargbegintheorem##1##2##3{%
  \item[\theorem@headerfont \llap{##2}\hskip\labelsep ##1\ (##3)]}}
\endgroup
%</thm>
%    \end{macrocode}
% \end{macro}
%
%
%
%
% \subsection{Final Definitions}
%
% \begin{macro}{\theorempreskipamount}
% \begin{macro}{\theorempostskipamount}
%    The \textsf{skip} parameters that regulate the vertical empty space
%    before and after the {\envfont theorem} environment have
%    to be allocated as well.
% \changes{v2.0f}{89/07/20}{Unnecessary `global' removed.}
%    \begin{macrocode}
%<*package>
\newskip\theorempreskipamount
\newskip\theorempostskipamount
%    \end{macrocode}
%    Since we have used the same values for all theorem sets, we now can
%    assign them.
%    \begin{macrocode}
\global\setlength\theorempreskipamount{12pt plus 5pt minus 3pt}
\global\setlength\theorempostskipamount{8pt plus 3pt minus 1.5pt}
%    \end{macrocode}
% \end{macro}
% \end{macro}
%
%
% \begin{macro}{\@endtheorem}
%    The same holds for the macro |\@endtheorem|, which ends a
%    {\envfont theorem} environment. Since it is the same for all
%    theorem sets, it is removed from the macros |\th@|\meta{style}.
%    It simply ends the {\envfont trivlist} environment, which was
%    begun in |\@thm|.
%    \begin{macrocode}
\global\let\@endtheorem=\endtrivlist
%    \end{macrocode}
% \end{macro}
%
%
%
% \begin{macro}{\@preamblecmds}
%    All macros defined above are to be used only in the preamble.
%    Therefore, we insert them in |\@preamblecmds| which will disable
%    them at begin document. This is done by the internal \LaTeXe{}
%    command |\@onlypreamble|.
% \changes{v2.0g}{89/09/19}{`theoremheaderfont and `theorembodyfont
%    added.}
% \changes{v2.2a}{94/02/03}{Use \LaTeXe{} interface}
%    \begin{macrocode}
\@onlypreamble\@xnthm
\@onlypreamble\@ynthm
\@onlypreamble\@othm
\@onlypreamble\newtheorem
\@onlypreamble\theoremstyle
\@onlypreamble\theorembodyfont
\@onlypreamble\theoremheaderfont
%    \end{macrocode}
% \end{macro}
%
% Finally we declare the \texttt{plain} theorem style to be the default.
% \changes{v2.1b}{90/06/04}{Fixed missing `item by declaring default
%    style.}
%    \begin{macrocode}
\theoremstyle{plain}
%</package>
%    \end{macrocode}
%
% \Finale

\endinput
}{}
\ProvidesFile{thp.sty}
             [\FMithmInfo]
%    \end{macrocode}
%
%
% \begin{macro}{\th@plain}
% \begin{macro}{\@begintheorem}
% \begin{macro}{\@opargbegintheorem}
%    |\theoremstyle{plain}| corresponds to the original
%    definition, except that the distances to the surrounding text are
%    determined by the parameters |\theorempreskipamount| and
%    |\theorempostskipamount|.
%    First we set the default body font.
%    \begin{macrocode}
\gdef\th@plain{\normalfont\itshape
%    \end{macrocode}
%    Then we define |\@begintheorem| and
%    |\@opargbegintheorem|. These two macros define how the
%    header of a theorem is typeset.  |\@opargbegintheorem| will
%    be called if a \textsf{theorem} environment with an optional
%    argument is encountered; otherwise, the header is constructed by
%    calling |\@begintheorem|.  If one of these macros is
%    executed, we are within a \textsf{trivlist} environment started by
%    |\@thm|.  So the theorem header is produced with an
%    |\item| command.
%
%    Instead of specifying the header font directly, all standard
%    theorem styles use the |\theorem@headerfont| macro to allow
%    customization.  The extra space (|\labelsep|) is necessary
%    because of problems in the \textsf{trivlist} environment.
%    \begin{macrocode}
  \def\@begintheorem##1##2{%
        \item[\hskip\labelsep \theorem@headerfont ##1\ ##2]}%
%    \end{macrocode}
%    The definition of |\@opargbegintheorem| is completely
%    analogous. The only difference is the fact that there exists a
%    third argument (which is the optional parameter of the
%    environment and contains additional information about the
%    theorem).  Customarily we enclose it in parentheses.
%    \begin{macrocode}
\def\@opargbegintheorem##1##2##3{%
   \item[\hskip\labelsep \theorem@headerfont ##1\ ##2\ (##3)]}}
%    \end{macrocode}
% \end{macro}
% \end{macro}
% \end{macro}
%
% \noindent
% We conclude with an |\endgroup| to restore the
% |\catcode| of |@|.
%    \begin{macrocode}
\endgroup
%</thp>
%    \end{macrocode}
%
%
% \subsubsection{The \texttt{break} style}
%
% This style option is stored in the file |thb.sty|.
% For the next two lines see the documentation for |\th@plain| on
% page~\pageref{dok:th@plain}.
%    \begin{macrocode}
%<*thb>
\begingroup \makeatletter
\@ifundefined{theorem@style}{% \iffalse meta-comment
%
% Copyright (C) 1993-2024
%
% The LaTeX Project and any individual authors listed elsewhere
% in this file.
%
% This file is part of the Standard LaTeX `Tools Bundle'.
% -------------------------------------------------------
%
% It may be distributed and/or modified under the
% conditions of the LaTeX Project Public License, either version 1.3c
% of this license or (at your option) any later version.
% The latest version of this license is in
%    https://www.latex-project.org/lppl.txt
% and version 1.3c or later is part of all distributions of LaTeX
% version 2005/12/01 or later.
%
% The list of all files belonging to the LaTeX `Tools Bundle' is
% given in the file `manifest.txt'.
%
% \fi
%
%%
%
%
\def\FMithmInfo{2023/07/05 v2.2c Theorem extension package (FMi)}
%
% \ProvidesFile{theorem.dtx}[\FMithmInfo]
%\iffalse   % this is a METACOMMENT !
%%
%% Package `theorem' to use with LaTeX2e
%% Copyright (C) 1989-1995 Frank Mittelbach, all rights reserved.
%<+package>\@ifundefined{theorem@style}{}{\endinput}
%<+package>\NeedsTeXFormat{LaTeX2e}[1995/06/01]
%<+package>\ProvidesPackage{theorem}
%<+package>         [\FMithmInfo]
%
% \fi
%
% \changes{v1.0}{87/05/07}{First draft.}
% \changes{v1.0a}{87/05/07}{`thistheoremstyle' added.}
% \changes{v1.1a}{87/07/29}{New concept: redefine the internal
%                          \protect\LaTeX{} commands `@xnthm' etc.}
% \changes{v1.1b}{87/07/31}{`theoremfont' used to specify font.}
% \changes{v1.2a}{88/01/08}{`@endtheorem' now globally defined.}
% \changes{v1.2b}{88/01/11}{`newtheorem' only allowed in the preamble.}
% \changes{v1.3a}{88/02/14}{`theoremfont' now a toks register.}
%
% \changes{v2.0a}{88/11/16}{New implementation and new documentation.}
% \changes{v2.0a}{88/11/16}{No layout changes but internal modifications
%                           old substyles won't work.}
% \changes{v2.0b}{89/01/30}{`global' used for all definitions.}
% \changes{v2.0c}{89/06/15}{Documentation translated to English.
%                            Thanks to LLL and Nina}
% \changes{v2.1a}{90/02/15}{Support for new font selection.}
% \changes{v2.2a}{94/02/03}{Cleanup for LaTeX2e}
%
% \DoNotIndex{\@endparenv,\@ifdefinable,\@ifnextchar,\@ifundefined}
% \DoNotIndex{\@namedef,\@nameuse,\@spaces,\@tempa,\@warning}
% \DoNotIndex{\ }
% \DoNotIndex{\begingroup,\bfseries}
% \DoNotIndex{\csname}
% \DoNotIndex{\def,\do,\docdate}
% \DoNotIndex{\endcsname,\endgroup,\endinput,\endtrivlist,\expandafter}
% \DoNotIndex{\filedate,\fileversion}
% \DoNotIndex{\gdef,\global}
% \DoNotIndex{\hbox,\hskip}
% \DoNotIndex{\ignorespaces,\input,\itshape,\item}
% \DoNotIndex{\labelsep,\let,\llap}
% \DoNotIndex{\makeatletter}
% \DoNotIndex{\noexpand}
% \DoNotIndex{\refstepcounter,\relax,\rlap}
% \DoNotIndex{\setlength,\slshape,\space,\string,\strut}
% \DoNotIndex{\the,\trivlist,\typeout}
% \DoNotIndex{\vbox}
% \DoNotIndex{\xdef}
%
% \def\deflabel#1{\ttfamily #1\hfill}
% \def\deflist#1{\begin{list}{}{\settowidth\labelwidth{\ttfamily #1}%
%                               \setlength\leftmargin\labelwidth
%                               \addtolength\leftmargin\labelsep
%                               \let\makelabel\deflabel}}
% \def\enddeflist{\end{list}}
% \def\envfont{\normalfont\ttfamily}
%
% \MakeShortVerb{\|}
%
% \newcommand{\TUB}{{\sl TUGboat\/}}
%
% \GetFileInfo{theorem.dtx}
% \title{An Extension of the
%        \LaTeX{} theorem environment\thanks{This file
%        has version number \fileversion, last
%        revised \filedate.}}
% \author{Frank Mittelbach}
%
% \MaintainedByLaTeXTeam{tools}
% \maketitle
%
% \begin{abstract}
%    The macros described in this paper yield an extension of the
%    \LaTeX{} theorem mechanism. It is designed to satisfy the
%    different requirements of various journals. Thus, the
%    layout of the ``theorems'' can be manipulated by determining a
%    ``style''. This article describes not only the use, but
%    also the definition, of the necessary macros.
% \end{abstract}
%
% \section*{Preface to version 2.2}
%
% For \LaTeXe{} this package did not need any fundamental changes. I
% only modified the messages generated so that theorem layout styles
% will show up with the |\listfiles| command and cleaned the section
% on the New Font Selection Scheme since this is now included in
% \LaTeX.
%
% \section*{Preface to version 2.1}
%
% This version is identical to 2.0g described in \TUB\ 10\#3 except
% for some internal defaults which are now set depending on the used
% font selection scheme.
%
% This was done to avoid unpleasant surprises if the new font selection
% scheme is in force. For further details see
% section~\ref{sec:newfonts} and \cite{bk:GMS94}.
%
% \section{Introduction}
%
% For our purposes here, ``theorems'' are labelled enunciations,
% often set off from the main text by extra space and a font change.
% Theorems, corollaries, conjectures, definitions, and remarks
% are all instances of ``theorems''.  The ``header'' of these structures
% is composed of a label (such as \textsc{Theorem} or \textsc{Remark})
% and a number which serializes an item in the sequence of items
% with the same label.
%
% Shortly after the introduction of \LaTeX{} at the Fachbereich
% Mathematik in Mainz, the desire to manipulate the layout of
% ``theorems'' arose.  In Mainz, the following two conventions came
% into general use:
% \begin{enumerate}
%   \item The number of the theorem is shown in the margin.
%   \item There is a line break at
%         the end of the theorem header.
% \end{enumerate}
%
% Additionally, some journals require different formats which depend on
% the ``sort of theorem'': e.g.\ often remarks and definitions are set
% in |\upshape|, while |\itshape| is employed for main theorems.
%
% Confronted with these requirements, a theorem environment was
% developed in Mainz which allows separate determination of the
% layout of the ``theorems sets'', comparable to |\pagestyle|.
%
%
% \section{The user interface}
% \subsection{Defining new theorem sets}
%
% \DescribeMacro\newtheorem
% As in the original \LaTeX{} version, the command |\newtheorem|
% defines a new ``theorem set'' or ``theorem-like structure''.
% Two required arguments name the new environment and give the text
% to be typeset with each instance of the new ``set'', while
% an optional argument determines how
% the ``set'' is enumerated:
% \begin{description}
%    \item[\ttfamily \bslash newtheorem\{foo\}\{bar\}]
%       The theorem set {\envfont foo} (whose name is \texttt{bar})
%       uses its own counter.
%    \item[\ttfamily \bslash newtheorem\{foo2\}{[foo]}\{bar2\}]
%       The theorem set {\envfont foo2} (printed name \texttt{bar2})
%       uses the same counter as the theorem set \texttt{foo}.
%    \item[\ttfamily \bslash newtheorem\{foo3\}\{bar3\}{[section]}]
%       The theorem set {\envfont foo3} (printed name \texttt{bar3}) is
%       enumerated within the counter \texttt{section}, i.e.\ with every
%       new |\section| the enumeration begins again with 1, and
%       the enumeration is composed from the section-number and the
%       theorem counter itself.
% \end{description}
%
% \DescribeMacro\theoremstyle
% Additionally, the command |\theoremstyle| can define the layout
% of various, or all, theorem sets.  It should be noted that any
% theorem set defined by |\newtheorem| is typeset in the
% |\theoremstyle| that is current at the time of the definition.
% Thus, the following
% \begin{verbatim}
%    \theoremstyle{break}        \newtheorem{Cor}{Corollary}
%    \theoremstyle{plain}        \newtheorem{Exa}{Example}[section]
%\end{verbatim}                             ^^A No blank at line start
% leads to the result that the set {\envfont Cor} is formatted in the
% style \texttt{break}, while the set {\envfont Exa} and all the
% following ones are formatted in the style \texttt{plain}, unless
% another |\theoremstyle| follows.  Since the definitions installed by
% |\newtheorem| are global, one also can limit |\theoremstyle| locally
% by grouping braces.
%
% \DescribeMacro\theorembodyfont
% The choice of the font for the theorem body is completely
% independent of the chosen |\theoremstyle|; this has proven to
% be very advantageous.  For example,
% \begin{verbatim}
%   {\theorembodyfont{\upshape}          \newtheorem{Rem}{Remark}}
%\end{verbatim}
% defines a theorem set {\envfont Rem}, which will be set in
% |\upshape| in the current layout (which in our example is
% \texttt{plain}).  As with |\theoremstyle|, the |\theorembodyfont|
% chosen is that current at the time of |\newtheorem|.  If
% |\theorembodyfont| is not specified or one defines
% |\theorembodyfont{}|, then the font used will be that defined by the
% |\theoremstyle|.
%
% \DescribeMacro\theoremheaderfont It is also possible to customize
% the font used for the theorem headers.  This is, however, a global
% declaration, and therefore there should be at most one
% |\theoremheaderfont| declaration in the preamble.\footnote{If it is
% actually necessary to have different header fonts, one has to define
% new theorem styles (substituting the desired font) or specify the
% information directly in the \texttt{\bslash newtheorem} declaration
% (the unclean variant).}
%
% \DescribeMacro\theorempreskipamount
% \DescribeMacro\theorempostskipamount Two additional parameters
% affect the vertical space around the theorem environments:
% |\theorempreskipamount| and |\theorempostskipamount| define,
% respectively, the spacing before and after such an environment.
% These parameters apply for all theorem sets and can be manipulated
% with the ordinary length macros.  They are rubber lengths
% (`\textsf{skips}'), and therefore can contain \texttt{plus} and
% \texttt{minus} parts.
%
% Since the definition of theorem sets should---most sensibly---be
% placed in the preamble, we only allow installation there.
% It is therefore possible to release the memory used here after
% |\begin{document}|, in order to make room for other
% applications.
%
%
% \subsection{Existing theorem styles}
%
% The following theorem styles exist to date:
% \begin{deflist}{marginbreak}
%    \item[plain]
%       This theorem style emulates the original \LaTeX{} definition,
%       except that additionally the parameters
%       |\theorem...skipamount| are used.
%    \item[break]
%       In this style, the theorem header is followed by a line
%       break.
%    \item[marginbreak]
%       The theorem number is set in the margin, and there is a line
%       break as in \texttt{break}.
%    \item[changebreak]
%       Like \texttt{break}, but with header number and text
%       interchanged.
%    \item[change]
%       Header number and text are interchanged, without a line break.
%    \item[margin]
%       The number is set in the left margin, without a line break.
% \end{deflist}
% All styles (except \texttt{plain}) select |\slshape| as the default
% |\theorembodyfont|.
%
%
% \subsection{Examples}
%
% Given the above theorem sets {\envfont Cor}, {\envfont Exa}
% and {\envfont Rem},
% suppose that the preamble also contains the declarations:
% \begin{verbatim}
%    \theoremstyle{marginbreak}   \newtheorem{Lem}[Cor]{Lemma}
%    \theoremstyle{change}
%    \theorembodyfont{\itshape}        \newtheorem{Def}[Cor]{Definition}
%
%    \theoremheaderfont{\scshape}
%\end{verbatim}
% Then the following are some typical examples of the typeset output
% resulting from their use.
%
% \begin{Cor}
%    This is a sentence typeset in the theorem environment {\envfont
%    Cor}.
% \end{Cor}
%
% \begin{Exa}
%    This is a sentence typeset in the theorem environment {\envfont
%    Exa}.
% \end{Exa}
%
% \begin{Rem}
%    This is a sentence typeset in the theorem environment {\envfont
%    Rem}.
% \end{Rem}
%
% \begin{Lem}[Ben User]
%    This is a sentence typeset in the theorem environment {\envfont
%    Lem}.
% \end{Lem}
%
% \begin{Def}[Very impressive Definition]
%    This is a sentence typeset in the theorem environment {\envfont
%    Def}.
% \end{Def}
%
% The last two examples show the effect of the optional argument to
% a theorem environment (it is the text typeset in parentheses).
%
%
% \section{Special Considerations}
% \label{sec:newfonts}
%
% Theoremheader and body are implemented as a unit. This means that
% the |\theoremheaderfont|  will inherit characteristics of the
% |\theorembodyfont| in \LaTeXe. Thus, if for example
% |\theorembodyfont| is |\itshape| and |\theoremheaderfont|
% is |\bfseries| the font selected for the header will have the
% characteristics `bold extended italic'. If this is not desired one
% should set the |\theoremheaderfont| to something like
% \begin{verbatim}
%      \theoremheaderfont{\normalfont\bfseries}
%\end{verbatim}
% i.e.\ supplying all necessary font information explicitly.
%
%
%
% \section{Acknowledgements}
%
% The publication of this set of macros was only possible with the
% help of Christina Busse (translating the manuscript into English),
% Joachim Pense (playing the r\^ole of typist), Chris Rowley
% (looking everything over) and many others providing useful
% suggestions.
%
% \MaybeStop{%
%    \begin{thebibliography}{9}
%       \bibitem{bk:GMS94} \textsc{M.~Goossens}, \textsc{F.~Mittelbach}
%         and \textsc{A.~Samarin}.
%         \newblock The \LaTeX{} Companion.
%         \newblock
%         Addison-Wesley, Reading, Massachusetts, 1994.
%       \bibitem{latex:source} \textsc{Lamport, Leslie.} \newblock
%                               \texttt{latex.tex}, version 2.09, date
%                               Feb. 1990.
%    \end{thebibliography}
%    \PrintIndex
%    \PrintChanges}
%
%
% \section{The documentation driver file}
%
% The next bit of code contains the documentation driver file for
% \TeX{}, i.e., the file that will produce the documentation you are
% currently reading. It will be extracted from this file by the
% \texttt{docstrip} program. Since it is the first code in the file
% one can alternatively process this file directly with \LaTeXe{} to
% obtain the documentation.
%
%    \begin{macrocode}
%<*driver>
\documentclass{ltxdoc}

\usepackage{theorem}

%<+driver>% The next few lines define theorem sets which are used
%<+driver>% in the example section of the documentation.

\theoremstyle{break}       \newtheorem{Cor}{Corollary}
\theoremstyle{plain}       \newtheorem{Exa}{Example}[section]
{\theorembodyfont{\upshape}\newtheorem{Rem}{Remark}}
\theoremstyle{marginbreak} \newtheorem{Lem}[Cor]{Lemma}
\theoremstyle{change}
\theorembodyfont{\itshape} \newtheorem{Def}[Cor]{Definition}

\theoremheaderfont{\scshape}

\RecordChanges

\begin{document}
   \DocInput{theorem.dtx}
\end{document}
%</driver>
%    \end{macrocode}
%
% \section{Definition of the Macros}
%
% If the file has been loaded before, we abort immediately. If not the
% package announces itself (this is actually done at the very top of
% the file---the way it is done isn't good style so don't copy it).
% \changes{v2.0e}{89/07/19}{Spaces removed from `typeout'.}
% \changes{v2.2b}{95/11/19}{Announce moved to top of file}
%    \begin{macrocode}
%<*package>
%\@ifundefined{theorem@style}{}{\endinput}
%\def\FMithmInfo{1995/11/19 v2.2b Theorem extension package (FMi)}
%\ProvidesPackage{theorem}[\FMithmInfo]
%    \end{macrocode}
%
%
% \subsection{Definition of theorem styles and fonts}
%
% All the definitions in this file are done globally to allow
% inputting this file inside a group.
%
% \begin{macro}{\theoremstyle}
%    Before a theorem style can be installed, the chosen style must be
%    known. For that reason, we must test to see that
%    |\th@|\meta{style} is known or, more precisely, that it is
%    different from |\relax|. If the style is not known then
%    |\th@plain| is used.
%    \begin{macrocode}
\gdef\theoremstyle#1{%
   \@ifundefined{th@#1}{\@warning
          {Unknown theoremstyle `#1'. Using `plain'}%
          \theorem@style{plain}}%
%    \end{macrocode}
%    We save the theorem style to be used in the token register
%    |\theorem@style|.
%    \begin{macrocode}
      {\theorem@style{#1}}%
%    \end{macrocode}
%    Now we ``evaluate'' the theorem style: this means, we call the
%    macro |\th@|\meta{style} which will activate the relevant
%    definitions which are contained in a separate file.  This is done
%    in a group to suppress changes to the current font.  This could
%    otherwise pose problems together with the new font selection
%    scheme\footnote{When I printed the original article using the new
%    font selection scheme I ended with a document with slanted
%    typefaces (text headings and all) simply because one of the
%    theorem styles used \texttt{\bslash sl} at toplevel.} if the
%    |\th@|\meta{style} is evaluated a second time.
% \changes{v2.1a}{90/02/15}{Group added around `th@..}
% \changes{v2.1a}{90/02/15}{Reseting of `..@begintheorem removed.}
%    \begin{macrocode}
      \begingroup
        \csname th@\the\theorem@style \endcsname
      \endgroup}
%    \end{macrocode}
% \end{macro}
%
%
% \begin{macro}{\@begintheorem}
% \changes{v2.1a}{90/02/15}{Relaxed at toplevel.}
% \begin{macro}{\@opargbegintheorem}
% \changes{v2.1a}{90/02/15}{Relaxed at toplevel.}
%    We reset |\@begintheorem| and |\@opargbegintheorem| to
%    |\relax| since these commands are no longer necessary at
%    toplevel. This will save a few tokens.
%    \begin{macrocode}
\global\let\@begintheorem\relax
\global\let\@opargbegintheorem\relax
%    \end{macrocode}
% \end{macro}
% \end{macro}
%
% \begin{macro}{\theorem@style}
%    Obviously the \textsf{token} register used above has to be
%    allocated. To assure the utmost compatibility with the
%    original \LaTeX{} definition, we set the default theorem style to
%    \texttt{plain}, which implements the usual \LaTeX{} convention.
% \changes{v2.0f}{89/07/20}{Unnecessary `global' removed.}
%    \begin{macrocode}
\newtoks\theorem@style
\global\theorem@style{plain}
%    \end{macrocode}
% \end{macro}
%
% \begin{macro}{\theorembodyfont}
% \begin{macro}{\theorem@bodyfont}
%    If the |\theorembodyfont| is set by the user then it should not
%    interact with the default font set in the theorem style. When the
%    new font selection is in force this may happen if, for example,
%    the default is |\itshape| and the new |\theorembodyfont| is
%    |\sffamily|. So we add a |\reset@font| command in front of the
%    user definition.
% \changes{v2.0f}{89/07/20}{Unnecessary `global' removed.}
% \changes{v2.0g}{89/09/19}{`theoremfont to `theorembodyfont.}
% \changes{v2.1c}{92/06/22}{`reset@font added and logic changed.}
% \changes{v2.1d}{92/10/28}{Reset to empty if arg is empty}
%    \begin{macrocode}
\gdef\theorembodyfont#1{%
%    \end{macrocode}
%    We check if the argument supplied is empty and if so put nothing
%    into the |\theorem@bodyfont| token register to allow for
%    |\theorembodyfont{}| as a mean of using the default of the
%    current |\theoremstyle|.
%    \begin{macrocode}
   \def\@tempa{#1}%
   \ifx\@tempa\@empty
    \theorem@bodyfont{}%
   \else
    \theorem@bodyfont{\reset@font#1}%
   \fi
   }
\newtoks\theorem@bodyfont
\global\theorem@bodyfont{}
%    \end{macrocode}
% \end{macro}
% \end{macro}
%
%
% \begin{macro}{\theoremheaderfont}
% \changes{v2.0g}{89/09/19}{Macro added.}
%    The font for the theorem headers is handled differently because
%    this definition applies to all theorem styles.
%    \begin{macrocode}
\gdef\theoremheaderfont#1{\gdef\theorem@headerfont{#1}%
%    \end{macrocode}
%    After using the macro once it is redefined to produce an error
%    message.
%    \begin{macrocode}
       \gdef\theoremheaderfont##1{%
        \typeout{\string\theoremheaderfont\space should be used
                 only once.}}}
%    \end{macrocode}
% \end{macro}
%
% \begin{macro}{\theorem@headerfont}
% \changes{v2.0g}{89/09/19}{Macro added.}
%    To set the |\theorem@headerfont| default we first test if
%    the new fontselection scheme is in force.
% \changes{v2.1a}{90/02/15}{Support for new font selection.}
%    \begin{macrocode}
\ifx\upshape\undefined
%    \end{macrocode}
%    If not we define it to expand into |\bfseries|. We don't use
%    |\let| just in case a following style option redefines this
%    macro.
%    \begin{macrocode}
\gdef\theorem@headerfont{\bfseries}
%    \end{macrocode}
%    Otherwise we reset the current shape before calling |\bfseries|.
%    \begin{macrocode}
\else \gdef\theorem@headerfont{\normalfont\bfseries}\fi
%    \end{macrocode}
% \end{macro}
%
% \begin{macro}{\th@plain}
% \begin{macro}{\th@break}
% \begin{macro}{\th@marginbreak}
% \begin{macro}{\th@changebreak}
% \begin{macro}{\th@change}
% \begin{macro}{\th@margin}
%    The different styles are defined in macros such as
%    |\th@plain|.  Since memory space is precious in
%    ``non-Big-versions'', we have to avoid offering too many unused
%    definitions.  Therefore we define these styles in separate files
%    that can be loaded on demand.  Thus the commands themselves only
%    load these files. We use |\@input@|, a \LaTeXe{} internal command
%    that ensures that the file will be listed with |\listfiles|.
% \changes{v2.2a}{94/02/02}{Use `@input@ to load theorem layout files}
%    \begin{macrocode}
\gdef\th@plain{\@input@{thp.sty}}
\gdef\th@break{\@input@{thb.sty}}
\gdef\th@marginbreak{\@input@{thmb.sty}}
\gdef\th@changebreak{\@input@{thcb.sty}}
\gdef\th@change{\@input@{thc.sty}}
\gdef\th@margin{\@input@{thm.sty}}
%    \end{macrocode}
%    This list will be expanded when new styles become
%    available.  For testing,  just append new theorem substyles
%    as document options.
% \end{macro}
% \end{macro}
% \end{macro}
% \end{macro}
% \end{macro}
% \end{macro}
%
%
%
% \subsection{Definition of a new theorem set}
%
% As already pointed out, a new theorem environment can be defined in
% three different ways:
% \begin{verbatim}
%    \newtheorem{Lem}{Lemma}
%    \newtheorem{Lem}{Lemma}[section]
%    \newtheorem{Lem}[Theorem]{Lemma}
%\end{verbatim}
% The function of the macro |\newtheorem| is to recognize these cases
% and then to branch into one of the three macros |\@ynthm|, |\@xnthm|
% or |\@othm|.  This mechanism is adopted unchanged from
% \cite{latex:source}; the essential point here is that, for example,
% in the second case, the arguments \texttt{Lem}, \texttt{Lemma} and
% \texttt{section} are passed over to the macro |\@xnthm|.
%
% We inspect this case first because the others present fewer problems,
% and thus are easily derived from this one.
%
% \begin{macro}{\@xnthm}
%    For our example arguments, the macro |\@xnthm| must fulfill
%    the following:
%    \begin{itemize}
%       \item
%          Define a new \LaTeX-counter `Lem'
%       \item
%          reset this counter within a |\section|
%       \item
%          define the macro |\theLem|
%       \item
%          define the environment macros |\Lem| and
%          |\endLem| using the current
%          |\theoremstyle| and |\theorem@bodyfont|.
%    \end{itemize}
%    Obviously, all this should happen only if the first argument
%    of |\@xnthm| (i.e.\ \texttt{Lem} in our example) is chosen so
%    as not to conflict with any previously defined commands or
%    environments.  This test is performed by the \LaTeX{} macro
%    |\@ifdefinable|.
%    \begin{macrocode}
\gdef\@xnthm#1#2[#3]{\expandafter\@ifdefinable\csname #1\endcsname
%    \end{macrocode}
%    Therefore, the first argument of |\@ifdefinable| is the
%    expansion (in the example, |\Lem|) of
%    |\csname||#1\endcsname|.  The second argument is
%    executed only if the test has been completed successfully.
%    \begin{macrocode}
   {%
%    \end{macrocode}
%    Now we define the new counter. The names of the \LaTeX{}
%    macros employed should speak for themselves:
% \changes{v2.2b}{1995/10/07}{Use \cs{@newctr}}
% \changes{v2.2c}{1995/11/23}{Use \cs{@newctr} properly.}
%    \begin{macrocode}
    \@definecounter{#1}\@newctr{#1}[#3]%
%    \end{macrocode}
%    Using |\@newctr| will give a proper error message if the counter
%    in |#3| is not defined.
%    In defining `|\theLem|' we must generate the desired macro
%    name by use of |\expandafter| and |\csname|.
%    \begin{macrocode}
    \expandafter\xdef\csname the#1\endcsname
%    \end{macrocode}
%    An |\xdef| is used in order to make the definition global,
%    and to ensure that it contains the replacement texts of
%    |\@thmcountersep| and |\@thmcounter|.\footnote{These
%    two macros can be defined by the document style.  Their default
%    values produce a `\texttt{.}' as separation and an arabic
%    representation of the number.} However, not everything should be
%    expanded. For example, it saves space to use |\thesection|
%    instead of its---at times---lengthy expansion.
%    \begin{macrocode}
      {\expandafter \noexpand \csname the#3\endcsname
       \@thmcountersep \@thmcounter{#1}}%
%    \end{macrocode}
%    Thus with the defaults of \LaTeX{}, |\theLem| would be
%    replaced by the command sequence |\thesection.\arabic{Lem}|.
%
%    We will now look at the definition of the macro which is executed
%    at the beginning of the actual environment (in our example this
%    macro is |\Lem|). It should be noted that we use an
%    ``|\expandafter| trick'' to expand only certain parts of the
%    replacement text at the time of the definition.
%    \begin{macrocode}
    \def\@tempa{\global\@namedef{#1}}%
    \expandafter \@tempa \expandafter{%
%    \end{macrocode}
%    First, the macro that contains the current definitions of
%    |\@begintheorem| and |\@opargtheorem| should be called
%    up.  The name of this macro---as is already known---has the form
%    |\th@|\meta{theorem\ style}; therefore, it must be called by
%    \begin{macrocode}
      \csname th@\the \theorem@style
%    \end{macrocode}
%    In addition the default theorem font should be changeable, i.e.\
%    we have to insert the contents of |\theorem@bodyfont|. For
%    that reason, we expand even further, beyond |\endcsname|,
%    and thus insert the contents of the \textsf{token} register
%    |\theorem@bodyfont| in the replacement text.
%    \begin{macrocode}
            \expandafter \endcsname \the \theorem@bodyfont
%    \end{macrocode}
%    Now it is time to call the macro |\@thm| which takes over
%    the further processing.  It has two arguments: the current
%    counter name (in our example, \texttt{Lem}), and the text of the
%    label (in our example, \texttt{Lemma}).
%    \begin{macrocode}
     \@thm{#1}{#2}}%
%    \end{macrocode}
%    With this, the `sub-definition' is complete.  The macro
%    |\@endtheorem| ends a theorem environment and is, so far, nothing
%    but an |\endtrivlist|.  (Hence it is defined globally, and not
%    within the theorem styles.\footnote{This has to be changed as
%    soon as theorem styles that change \texttt{\bslash @endtheorem}
%    exist.  In such a case, all existing styles must be changed as
%    well since they will have to reset the macro.}) Therefore, we can
%    set it equivalent to the macro that ends the theorem set (in our
%    example, |\endLem|).  However, if some day theorem styles exist
%    that do change |\@endtheorem|, we would have to use the
%    commented-out line instead.
%    \begin{macrocode}
    \global \expandafter \let \csname end#1\endcsname \@endtheorem
%   \global\@namedef{end#1}{\@endtheorem}%
%    \end{macrocode}
%    With these commands all the required definitions are employed,
%    unless the test |\@ifdefinable| has failed.  Therefore, we
%    end the second argument of this macro and with it the definition
%    of |\@xnthm|.
%    \begin{macrocode}
   }}
%    \end{macrocode}
% \end{macro}
%
% \begin{macro}{\@ynthm}
%    The definition of |\@ynthm| is completely analogous. In
%    this case the new counter that is defined is not reset within
%    another counter; thus the definition of |\the...|
%    is simplified:
%    \begin{macrocode}
\gdef\@ynthm#1#2{\expandafter\@ifdefinable\csname #1\endcsname
   {\@definecounter{#1}%
    \expandafter\xdef\csname the#1\endcsname{\@thmcounter{#1}}%
%    \end{macrocode}
%    The rest of the definition corresponds literally to that of
%    |\@xnthm|:
%    \begin{macrocode}
    \def\@tempa{\global\@namedef{#1}}\expandafter \@tempa
     \expandafter{\csname th@\the \theorem@style \expandafter
     \endcsname \the\theorem@bodyfont \@thm{#1}{#2}}%
    \global \expandafter \let \csname end#1\endcsname \@endtheorem}}
%    \end{macrocode}
% \end{macro}
%
% \begin{macro}{\@othm}
%    The definition of |\@othm| does not contain anything new.
%    \begin{macrocode}
\gdef\@othm#1[#2]#3{%
%    \end{macrocode}
% \changes{v2.2b}{95/10/07}{Build in check for counter defined}
%    We do not define a new counter but instead use one that has
%    already been defined.  Thus the only definition we need is that
%    of this pseudo-counter (i.e.\ |\the|\meta{env.\ name}).
%    First we check if |#2| corresponds to a known counter name.
%    \begin{macrocode}
  \expandafter\ifx\csname c@#2\endcsname\relax
   \@nocounterr{#2}%
  \else
   \expandafter\@ifdefinable\csname #1\endcsname
   {\expandafter \xdef \csname the#1\endcsname
     {\expandafter \noexpand \csname the#2\endcsname}%
%    \end{macrocode}
%    All other parts of the definition can be adopted from
%    |\@xnthm|. We have to remember, though, that in this case
%    the name of the current counter and the theorem label have moved
%    to the second and third arguments.
%    \begin{macrocode}
    \def\@tempa{\global\@namedef{#1}}\expandafter \@tempa
     \expandafter{\csname th@\the \theorem@style \expandafter
     \endcsname \the\theorem@bodyfont \@thm{#2}{#3}}%
    \global \expandafter \let \csname end#1\endcsname \@endtheorem}%
  \fi}
%    \end{macrocode}
% \end{macro}
%
% \subsection{Macros that are employed in a {\envfont theorem}
%    environment}
%
% \begin{macro}{\@thm}
%    The macro |\@thm| has to increase the current counter.  Then,
%    depending on whether the environment has (or does not have) an
%    optional argument, it has to branch into either |\@begintheorem|
%    or |\@opargtheorem|.
%    \begin{macrocode}
\gdef\@thm#1#2{\refstepcounter{#1}%
%    \end{macrocode}
%    Now we start a {\envfont trivlist} environment, and give
%    |\@topsep| and |\@topsepadd| the values of the \textsf{skip}
%    registers |\theorempreskipamount| and |\theorempostskipamount|.
%    The value in |\@topsep| is the vertical space that is inserted by
%    the first (and only) |\item| in our |\trivlist| whilst
%    |\@topsepadd| is inserted by |\@endparenv| at the end of that
%    {\envfont trivlist} environment.  By using these registers, we
%    obtain the desired space around a {\envfont theorem} environment.
%    \begin{macrocode}
   \trivlist
   \@topsep \theorempreskipamount               % used by first \item
   \@topsepadd \theorempostskipamount           % used by \@endparenv
%    \end{macrocode}
%    Now we have to test whether an optional argument has been given.
% \changes{v2.0d}{89/07/18}{Forgotten percent added.}
%    \begin{macrocode}
   \@ifnextchar [%
%    \end{macrocode}
%    If there is an optional argument, we will call |\@ythm|, and
%    move the arguments read back into the input stream.
%    \begin{macrocode}
   {\@ythm{#1}{#2}}%
%    \end{macrocode}
%    If not, we call |\@begintheorem|. Its first argument is the
%    name of the theorem set (hence the second argument of
%    |\@thm|). Its second argument is the macro that produces the
%    current number.
%    \begin{macrocode}
   {\@begintheorem{#2}{\csname the#1\endcsname}\ignorespaces}}
%    \end{macrocode}
% \end{macro}
%
% \begin{macro}{\@xthm}
% \begin{macro}{\@ythm}
%    Both these macros were originally called by |\@thm|.  We do
%    not need |\@xthm| anymore, hence we reset it to
%    |\relax|. The definition of |\@ythm| has not changed at
%    all from its definition in \LaTeX. In order to make the macros
%    easier to understand, we will
%    nevertheless present it (commented out).
%    \begin{macrocode}
\global\let\@xthm\relax
% \def\@ythm#1#2[#3]{\@opargbegintheorem{#2}{\csname
%           the#1\endcsname}{#3}\ignorespaces}
%</package>
%    \end{macrocode}
%    The primitive |\ignorespaces| in |\@ythm| and
%    |\@thm| is needed to remove the spaces between the
%    |\begin{...}| and the actual text.
% \end{macro}
% \end{macro}
%
%
% \subsection{Definition of the theorem substyles}
%
% As already pointed out, the theorem substyles, defined below, are
% only loaded when necessary.  Note that all these substyles, except
% \texttt{plain}, have |\slshape| as the default body font.
%
% \subsubsection{The \texttt{plain} style}
%
% \label{dok:th@plain} As the following macros use |@|, we have
% to locally set the |\catcode| of this symbol to ``letter''.
% This happens within a group, so that we do not have to worry about
% which |\catcode| that symbol had before.
%    \begin{macrocode}
%<*thp>
\begingroup \makeatletter
%    \end{macrocode}
% Since we are now within a group, we must make all definitions
% globally.  First we make sure that \texttt{theorem.sty} is loaded.
% This will allow us to use this file as a document style option
% without having to call \texttt{theorem} itself as an option.  At the
% same time, we assure that at least version 2 is loaded, since
% |\theorem@style| was not defined in earlier versions.
%    \begin{macrocode}
\@ifundefined{theorem@style}{% \iffalse meta-comment
%
% Copyright (C) 1993-2024
%
% The LaTeX Project and any individual authors listed elsewhere
% in this file.
%
% This file is part of the Standard LaTeX `Tools Bundle'.
% -------------------------------------------------------
%
% It may be distributed and/or modified under the
% conditions of the LaTeX Project Public License, either version 1.3c
% of this license or (at your option) any later version.
% The latest version of this license is in
%    https://www.latex-project.org/lppl.txt
% and version 1.3c or later is part of all distributions of LaTeX
% version 2005/12/01 or later.
%
% The list of all files belonging to the LaTeX `Tools Bundle' is
% given in the file `manifest.txt'.
%
% \fi
%
%%
%
%
\def\FMithmInfo{2023/07/05 v2.2c Theorem extension package (FMi)}
%
% \ProvidesFile{theorem.dtx}[\FMithmInfo]
%\iffalse   % this is a METACOMMENT !
%%
%% Package `theorem' to use with LaTeX2e
%% Copyright (C) 1989-1995 Frank Mittelbach, all rights reserved.
%<+package>\@ifundefined{theorem@style}{}{\endinput}
%<+package>\NeedsTeXFormat{LaTeX2e}[1995/06/01]
%<+package>\ProvidesPackage{theorem}
%<+package>         [\FMithmInfo]
%
% \fi
%
% \changes{v1.0}{87/05/07}{First draft.}
% \changes{v1.0a}{87/05/07}{`thistheoremstyle' added.}
% \changes{v1.1a}{87/07/29}{New concept: redefine the internal
%                          \protect\LaTeX{} commands `@xnthm' etc.}
% \changes{v1.1b}{87/07/31}{`theoremfont' used to specify font.}
% \changes{v1.2a}{88/01/08}{`@endtheorem' now globally defined.}
% \changes{v1.2b}{88/01/11}{`newtheorem' only allowed in the preamble.}
% \changes{v1.3a}{88/02/14}{`theoremfont' now a toks register.}
%
% \changes{v2.0a}{88/11/16}{New implementation and new documentation.}
% \changes{v2.0a}{88/11/16}{No layout changes but internal modifications
%                           old substyles won't work.}
% \changes{v2.0b}{89/01/30}{`global' used for all definitions.}
% \changes{v2.0c}{89/06/15}{Documentation translated to English.
%                            Thanks to LLL and Nina}
% \changes{v2.1a}{90/02/15}{Support for new font selection.}
% \changes{v2.2a}{94/02/03}{Cleanup for LaTeX2e}
%
% \DoNotIndex{\@endparenv,\@ifdefinable,\@ifnextchar,\@ifundefined}
% \DoNotIndex{\@namedef,\@nameuse,\@spaces,\@tempa,\@warning}
% \DoNotIndex{\ }
% \DoNotIndex{\begingroup,\bfseries}
% \DoNotIndex{\csname}
% \DoNotIndex{\def,\do,\docdate}
% \DoNotIndex{\endcsname,\endgroup,\endinput,\endtrivlist,\expandafter}
% \DoNotIndex{\filedate,\fileversion}
% \DoNotIndex{\gdef,\global}
% \DoNotIndex{\hbox,\hskip}
% \DoNotIndex{\ignorespaces,\input,\itshape,\item}
% \DoNotIndex{\labelsep,\let,\llap}
% \DoNotIndex{\makeatletter}
% \DoNotIndex{\noexpand}
% \DoNotIndex{\refstepcounter,\relax,\rlap}
% \DoNotIndex{\setlength,\slshape,\space,\string,\strut}
% \DoNotIndex{\the,\trivlist,\typeout}
% \DoNotIndex{\vbox}
% \DoNotIndex{\xdef}
%
% \def\deflabel#1{\ttfamily #1\hfill}
% \def\deflist#1{\begin{list}{}{\settowidth\labelwidth{\ttfamily #1}%
%                               \setlength\leftmargin\labelwidth
%                               \addtolength\leftmargin\labelsep
%                               \let\makelabel\deflabel}}
% \def\enddeflist{\end{list}}
% \def\envfont{\normalfont\ttfamily}
%
% \MakeShortVerb{\|}
%
% \newcommand{\TUB}{{\sl TUGboat\/}}
%
% \GetFileInfo{theorem.dtx}
% \title{An Extension of the
%        \LaTeX{} theorem environment\thanks{This file
%        has version number \fileversion, last
%        revised \filedate.}}
% \author{Frank Mittelbach}
%
% \MaintainedByLaTeXTeam{tools}
% \maketitle
%
% \begin{abstract}
%    The macros described in this paper yield an extension of the
%    \LaTeX{} theorem mechanism. It is designed to satisfy the
%    different requirements of various journals. Thus, the
%    layout of the ``theorems'' can be manipulated by determining a
%    ``style''. This article describes not only the use, but
%    also the definition, of the necessary macros.
% \end{abstract}
%
% \section*{Preface to version 2.2}
%
% For \LaTeXe{} this package did not need any fundamental changes. I
% only modified the messages generated so that theorem layout styles
% will show up with the |\listfiles| command and cleaned the section
% on the New Font Selection Scheme since this is now included in
% \LaTeX.
%
% \section*{Preface to version 2.1}
%
% This version is identical to 2.0g described in \TUB\ 10\#3 except
% for some internal defaults which are now set depending on the used
% font selection scheme.
%
% This was done to avoid unpleasant surprises if the new font selection
% scheme is in force. For further details see
% section~\ref{sec:newfonts} and \cite{bk:GMS94}.
%
% \section{Introduction}
%
% For our purposes here, ``theorems'' are labelled enunciations,
% often set off from the main text by extra space and a font change.
% Theorems, corollaries, conjectures, definitions, and remarks
% are all instances of ``theorems''.  The ``header'' of these structures
% is composed of a label (such as \textsc{Theorem} or \textsc{Remark})
% and a number which serializes an item in the sequence of items
% with the same label.
%
% Shortly after the introduction of \LaTeX{} at the Fachbereich
% Mathematik in Mainz, the desire to manipulate the layout of
% ``theorems'' arose.  In Mainz, the following two conventions came
% into general use:
% \begin{enumerate}
%   \item The number of the theorem is shown in the margin.
%   \item There is a line break at
%         the end of the theorem header.
% \end{enumerate}
%
% Additionally, some journals require different formats which depend on
% the ``sort of theorem'': e.g.\ often remarks and definitions are set
% in |\upshape|, while |\itshape| is employed for main theorems.
%
% Confronted with these requirements, a theorem environment was
% developed in Mainz which allows separate determination of the
% layout of the ``theorems sets'', comparable to |\pagestyle|.
%
%
% \section{The user interface}
% \subsection{Defining new theorem sets}
%
% \DescribeMacro\newtheorem
% As in the original \LaTeX{} version, the command |\newtheorem|
% defines a new ``theorem set'' or ``theorem-like structure''.
% Two required arguments name the new environment and give the text
% to be typeset with each instance of the new ``set'', while
% an optional argument determines how
% the ``set'' is enumerated:
% \begin{description}
%    \item[\ttfamily \bslash newtheorem\{foo\}\{bar\}]
%       The theorem set {\envfont foo} (whose name is \texttt{bar})
%       uses its own counter.
%    \item[\ttfamily \bslash newtheorem\{foo2\}{[foo]}\{bar2\}]
%       The theorem set {\envfont foo2} (printed name \texttt{bar2})
%       uses the same counter as the theorem set \texttt{foo}.
%    \item[\ttfamily \bslash newtheorem\{foo3\}\{bar3\}{[section]}]
%       The theorem set {\envfont foo3} (printed name \texttt{bar3}) is
%       enumerated within the counter \texttt{section}, i.e.\ with every
%       new |\section| the enumeration begins again with 1, and
%       the enumeration is composed from the section-number and the
%       theorem counter itself.
% \end{description}
%
% \DescribeMacro\theoremstyle
% Additionally, the command |\theoremstyle| can define the layout
% of various, or all, theorem sets.  It should be noted that any
% theorem set defined by |\newtheorem| is typeset in the
% |\theoremstyle| that is current at the time of the definition.
% Thus, the following
% \begin{verbatim}
%    \theoremstyle{break}        \newtheorem{Cor}{Corollary}
%    \theoremstyle{plain}        \newtheorem{Exa}{Example}[section]
%\end{verbatim}                             ^^A No blank at line start
% leads to the result that the set {\envfont Cor} is formatted in the
% style \texttt{break}, while the set {\envfont Exa} and all the
% following ones are formatted in the style \texttt{plain}, unless
% another |\theoremstyle| follows.  Since the definitions installed by
% |\newtheorem| are global, one also can limit |\theoremstyle| locally
% by grouping braces.
%
% \DescribeMacro\theorembodyfont
% The choice of the font for the theorem body is completely
% independent of the chosen |\theoremstyle|; this has proven to
% be very advantageous.  For example,
% \begin{verbatim}
%   {\theorembodyfont{\upshape}          \newtheorem{Rem}{Remark}}
%\end{verbatim}
% defines a theorem set {\envfont Rem}, which will be set in
% |\upshape| in the current layout (which in our example is
% \texttt{plain}).  As with |\theoremstyle|, the |\theorembodyfont|
% chosen is that current at the time of |\newtheorem|.  If
% |\theorembodyfont| is not specified or one defines
% |\theorembodyfont{}|, then the font used will be that defined by the
% |\theoremstyle|.
%
% \DescribeMacro\theoremheaderfont It is also possible to customize
% the font used for the theorem headers.  This is, however, a global
% declaration, and therefore there should be at most one
% |\theoremheaderfont| declaration in the preamble.\footnote{If it is
% actually necessary to have different header fonts, one has to define
% new theorem styles (substituting the desired font) or specify the
% information directly in the \texttt{\bslash newtheorem} declaration
% (the unclean variant).}
%
% \DescribeMacro\theorempreskipamount
% \DescribeMacro\theorempostskipamount Two additional parameters
% affect the vertical space around the theorem environments:
% |\theorempreskipamount| and |\theorempostskipamount| define,
% respectively, the spacing before and after such an environment.
% These parameters apply for all theorem sets and can be manipulated
% with the ordinary length macros.  They are rubber lengths
% (`\textsf{skips}'), and therefore can contain \texttt{plus} and
% \texttt{minus} parts.
%
% Since the definition of theorem sets should---most sensibly---be
% placed in the preamble, we only allow installation there.
% It is therefore possible to release the memory used here after
% |\begin{document}|, in order to make room for other
% applications.
%
%
% \subsection{Existing theorem styles}
%
% The following theorem styles exist to date:
% \begin{deflist}{marginbreak}
%    \item[plain]
%       This theorem style emulates the original \LaTeX{} definition,
%       except that additionally the parameters
%       |\theorem...skipamount| are used.
%    \item[break]
%       In this style, the theorem header is followed by a line
%       break.
%    \item[marginbreak]
%       The theorem number is set in the margin, and there is a line
%       break as in \texttt{break}.
%    \item[changebreak]
%       Like \texttt{break}, but with header number and text
%       interchanged.
%    \item[change]
%       Header number and text are interchanged, without a line break.
%    \item[margin]
%       The number is set in the left margin, without a line break.
% \end{deflist}
% All styles (except \texttt{plain}) select |\slshape| as the default
% |\theorembodyfont|.
%
%
% \subsection{Examples}
%
% Given the above theorem sets {\envfont Cor}, {\envfont Exa}
% and {\envfont Rem},
% suppose that the preamble also contains the declarations:
% \begin{verbatim}
%    \theoremstyle{marginbreak}   \newtheorem{Lem}[Cor]{Lemma}
%    \theoremstyle{change}
%    \theorembodyfont{\itshape}        \newtheorem{Def}[Cor]{Definition}
%
%    \theoremheaderfont{\scshape}
%\end{verbatim}
% Then the following are some typical examples of the typeset output
% resulting from their use.
%
% \begin{Cor}
%    This is a sentence typeset in the theorem environment {\envfont
%    Cor}.
% \end{Cor}
%
% \begin{Exa}
%    This is a sentence typeset in the theorem environment {\envfont
%    Exa}.
% \end{Exa}
%
% \begin{Rem}
%    This is a sentence typeset in the theorem environment {\envfont
%    Rem}.
% \end{Rem}
%
% \begin{Lem}[Ben User]
%    This is a sentence typeset in the theorem environment {\envfont
%    Lem}.
% \end{Lem}
%
% \begin{Def}[Very impressive Definition]
%    This is a sentence typeset in the theorem environment {\envfont
%    Def}.
% \end{Def}
%
% The last two examples show the effect of the optional argument to
% a theorem environment (it is the text typeset in parentheses).
%
%
% \section{Special Considerations}
% \label{sec:newfonts}
%
% Theoremheader and body are implemented as a unit. This means that
% the |\theoremheaderfont|  will inherit characteristics of the
% |\theorembodyfont| in \LaTeXe. Thus, if for example
% |\theorembodyfont| is |\itshape| and |\theoremheaderfont|
% is |\bfseries| the font selected for the header will have the
% characteristics `bold extended italic'. If this is not desired one
% should set the |\theoremheaderfont| to something like
% \begin{verbatim}
%      \theoremheaderfont{\normalfont\bfseries}
%\end{verbatim}
% i.e.\ supplying all necessary font information explicitly.
%
%
%
% \section{Acknowledgements}
%
% The publication of this set of macros was only possible with the
% help of Christina Busse (translating the manuscript into English),
% Joachim Pense (playing the r\^ole of typist), Chris Rowley
% (looking everything over) and many others providing useful
% suggestions.
%
% \MaybeStop{%
%    \begin{thebibliography}{9}
%       \bibitem{bk:GMS94} \textsc{M.~Goossens}, \textsc{F.~Mittelbach}
%         and \textsc{A.~Samarin}.
%         \newblock The \LaTeX{} Companion.
%         \newblock
%         Addison-Wesley, Reading, Massachusetts, 1994.
%       \bibitem{latex:source} \textsc{Lamport, Leslie.} \newblock
%                               \texttt{latex.tex}, version 2.09, date
%                               Feb. 1990.
%    \end{thebibliography}
%    \PrintIndex
%    \PrintChanges}
%
%
% \section{The documentation driver file}
%
% The next bit of code contains the documentation driver file for
% \TeX{}, i.e., the file that will produce the documentation you are
% currently reading. It will be extracted from this file by the
% \texttt{docstrip} program. Since it is the first code in the file
% one can alternatively process this file directly with \LaTeXe{} to
% obtain the documentation.
%
%    \begin{macrocode}
%<*driver>
\documentclass{ltxdoc}

\usepackage{theorem}

%<+driver>% The next few lines define theorem sets which are used
%<+driver>% in the example section of the documentation.

\theoremstyle{break}       \newtheorem{Cor}{Corollary}
\theoremstyle{plain}       \newtheorem{Exa}{Example}[section]
{\theorembodyfont{\upshape}\newtheorem{Rem}{Remark}}
\theoremstyle{marginbreak} \newtheorem{Lem}[Cor]{Lemma}
\theoremstyle{change}
\theorembodyfont{\itshape} \newtheorem{Def}[Cor]{Definition}

\theoremheaderfont{\scshape}

\RecordChanges

\begin{document}
   \DocInput{theorem.dtx}
\end{document}
%</driver>
%    \end{macrocode}
%
% \section{Definition of the Macros}
%
% If the file has been loaded before, we abort immediately. If not the
% package announces itself (this is actually done at the very top of
% the file---the way it is done isn't good style so don't copy it).
% \changes{v2.0e}{89/07/19}{Spaces removed from `typeout'.}
% \changes{v2.2b}{95/11/19}{Announce moved to top of file}
%    \begin{macrocode}
%<*package>
%\@ifundefined{theorem@style}{}{\endinput}
%\def\FMithmInfo{1995/11/19 v2.2b Theorem extension package (FMi)}
%\ProvidesPackage{theorem}[\FMithmInfo]
%    \end{macrocode}
%
%
% \subsection{Definition of theorem styles and fonts}
%
% All the definitions in this file are done globally to allow
% inputting this file inside a group.
%
% \begin{macro}{\theoremstyle}
%    Before a theorem style can be installed, the chosen style must be
%    known. For that reason, we must test to see that
%    |\th@|\meta{style} is known or, more precisely, that it is
%    different from |\relax|. If the style is not known then
%    |\th@plain| is used.
%    \begin{macrocode}
\gdef\theoremstyle#1{%
   \@ifundefined{th@#1}{\@warning
          {Unknown theoremstyle `#1'. Using `plain'}%
          \theorem@style{plain}}%
%    \end{macrocode}
%    We save the theorem style to be used in the token register
%    |\theorem@style|.
%    \begin{macrocode}
      {\theorem@style{#1}}%
%    \end{macrocode}
%    Now we ``evaluate'' the theorem style: this means, we call the
%    macro |\th@|\meta{style} which will activate the relevant
%    definitions which are contained in a separate file.  This is done
%    in a group to suppress changes to the current font.  This could
%    otherwise pose problems together with the new font selection
%    scheme\footnote{When I printed the original article using the new
%    font selection scheme I ended with a document with slanted
%    typefaces (text headings and all) simply because one of the
%    theorem styles used \texttt{\bslash sl} at toplevel.} if the
%    |\th@|\meta{style} is evaluated a second time.
% \changes{v2.1a}{90/02/15}{Group added around `th@..}
% \changes{v2.1a}{90/02/15}{Reseting of `..@begintheorem removed.}
%    \begin{macrocode}
      \begingroup
        \csname th@\the\theorem@style \endcsname
      \endgroup}
%    \end{macrocode}
% \end{macro}
%
%
% \begin{macro}{\@begintheorem}
% \changes{v2.1a}{90/02/15}{Relaxed at toplevel.}
% \begin{macro}{\@opargbegintheorem}
% \changes{v2.1a}{90/02/15}{Relaxed at toplevel.}
%    We reset |\@begintheorem| and |\@opargbegintheorem| to
%    |\relax| since these commands are no longer necessary at
%    toplevel. This will save a few tokens.
%    \begin{macrocode}
\global\let\@begintheorem\relax
\global\let\@opargbegintheorem\relax
%    \end{macrocode}
% \end{macro}
% \end{macro}
%
% \begin{macro}{\theorem@style}
%    Obviously the \textsf{token} register used above has to be
%    allocated. To assure the utmost compatibility with the
%    original \LaTeX{} definition, we set the default theorem style to
%    \texttt{plain}, which implements the usual \LaTeX{} convention.
% \changes{v2.0f}{89/07/20}{Unnecessary `global' removed.}
%    \begin{macrocode}
\newtoks\theorem@style
\global\theorem@style{plain}
%    \end{macrocode}
% \end{macro}
%
% \begin{macro}{\theorembodyfont}
% \begin{macro}{\theorem@bodyfont}
%    If the |\theorembodyfont| is set by the user then it should not
%    interact with the default font set in the theorem style. When the
%    new font selection is in force this may happen if, for example,
%    the default is |\itshape| and the new |\theorembodyfont| is
%    |\sffamily|. So we add a |\reset@font| command in front of the
%    user definition.
% \changes{v2.0f}{89/07/20}{Unnecessary `global' removed.}
% \changes{v2.0g}{89/09/19}{`theoremfont to `theorembodyfont.}
% \changes{v2.1c}{92/06/22}{`reset@font added and logic changed.}
% \changes{v2.1d}{92/10/28}{Reset to empty if arg is empty}
%    \begin{macrocode}
\gdef\theorembodyfont#1{%
%    \end{macrocode}
%    We check if the argument supplied is empty and if so put nothing
%    into the |\theorem@bodyfont| token register to allow for
%    |\theorembodyfont{}| as a mean of using the default of the
%    current |\theoremstyle|.
%    \begin{macrocode}
   \def\@tempa{#1}%
   \ifx\@tempa\@empty
    \theorem@bodyfont{}%
   \else
    \theorem@bodyfont{\reset@font#1}%
   \fi
   }
\newtoks\theorem@bodyfont
\global\theorem@bodyfont{}
%    \end{macrocode}
% \end{macro}
% \end{macro}
%
%
% \begin{macro}{\theoremheaderfont}
% \changes{v2.0g}{89/09/19}{Macro added.}
%    The font for the theorem headers is handled differently because
%    this definition applies to all theorem styles.
%    \begin{macrocode}
\gdef\theoremheaderfont#1{\gdef\theorem@headerfont{#1}%
%    \end{macrocode}
%    After using the macro once it is redefined to produce an error
%    message.
%    \begin{macrocode}
       \gdef\theoremheaderfont##1{%
        \typeout{\string\theoremheaderfont\space should be used
                 only once.}}}
%    \end{macrocode}
% \end{macro}
%
% \begin{macro}{\theorem@headerfont}
% \changes{v2.0g}{89/09/19}{Macro added.}
%    To set the |\theorem@headerfont| default we first test if
%    the new fontselection scheme is in force.
% \changes{v2.1a}{90/02/15}{Support for new font selection.}
%    \begin{macrocode}
\ifx\upshape\undefined
%    \end{macrocode}
%    If not we define it to expand into |\bfseries|. We don't use
%    |\let| just in case a following style option redefines this
%    macro.
%    \begin{macrocode}
\gdef\theorem@headerfont{\bfseries}
%    \end{macrocode}
%    Otherwise we reset the current shape before calling |\bfseries|.
%    \begin{macrocode}
\else \gdef\theorem@headerfont{\normalfont\bfseries}\fi
%    \end{macrocode}
% \end{macro}
%
% \begin{macro}{\th@plain}
% \begin{macro}{\th@break}
% \begin{macro}{\th@marginbreak}
% \begin{macro}{\th@changebreak}
% \begin{macro}{\th@change}
% \begin{macro}{\th@margin}
%    The different styles are defined in macros such as
%    |\th@plain|.  Since memory space is precious in
%    ``non-Big-versions'', we have to avoid offering too many unused
%    definitions.  Therefore we define these styles in separate files
%    that can be loaded on demand.  Thus the commands themselves only
%    load these files. We use |\@input@|, a \LaTeXe{} internal command
%    that ensures that the file will be listed with |\listfiles|.
% \changes{v2.2a}{94/02/02}{Use `@input@ to load theorem layout files}
%    \begin{macrocode}
\gdef\th@plain{\@input@{thp.sty}}
\gdef\th@break{\@input@{thb.sty}}
\gdef\th@marginbreak{\@input@{thmb.sty}}
\gdef\th@changebreak{\@input@{thcb.sty}}
\gdef\th@change{\@input@{thc.sty}}
\gdef\th@margin{\@input@{thm.sty}}
%    \end{macrocode}
%    This list will be expanded when new styles become
%    available.  For testing,  just append new theorem substyles
%    as document options.
% \end{macro}
% \end{macro}
% \end{macro}
% \end{macro}
% \end{macro}
% \end{macro}
%
%
%
% \subsection{Definition of a new theorem set}
%
% As already pointed out, a new theorem environment can be defined in
% three different ways:
% \begin{verbatim}
%    \newtheorem{Lem}{Lemma}
%    \newtheorem{Lem}{Lemma}[section]
%    \newtheorem{Lem}[Theorem]{Lemma}
%\end{verbatim}
% The function of the macro |\newtheorem| is to recognize these cases
% and then to branch into one of the three macros |\@ynthm|, |\@xnthm|
% or |\@othm|.  This mechanism is adopted unchanged from
% \cite{latex:source}; the essential point here is that, for example,
% in the second case, the arguments \texttt{Lem}, \texttt{Lemma} and
% \texttt{section} are passed over to the macro |\@xnthm|.
%
% We inspect this case first because the others present fewer problems,
% and thus are easily derived from this one.
%
% \begin{macro}{\@xnthm}
%    For our example arguments, the macro |\@xnthm| must fulfill
%    the following:
%    \begin{itemize}
%       \item
%          Define a new \LaTeX-counter `Lem'
%       \item
%          reset this counter within a |\section|
%       \item
%          define the macro |\theLem|
%       \item
%          define the environment macros |\Lem| and
%          |\endLem| using the current
%          |\theoremstyle| and |\theorem@bodyfont|.
%    \end{itemize}
%    Obviously, all this should happen only if the first argument
%    of |\@xnthm| (i.e.\ \texttt{Lem} in our example) is chosen so
%    as not to conflict with any previously defined commands or
%    environments.  This test is performed by the \LaTeX{} macro
%    |\@ifdefinable|.
%    \begin{macrocode}
\gdef\@xnthm#1#2[#3]{\expandafter\@ifdefinable\csname #1\endcsname
%    \end{macrocode}
%    Therefore, the first argument of |\@ifdefinable| is the
%    expansion (in the example, |\Lem|) of
%    |\csname||#1\endcsname|.  The second argument is
%    executed only if the test has been completed successfully.
%    \begin{macrocode}
   {%
%    \end{macrocode}
%    Now we define the new counter. The names of the \LaTeX{}
%    macros employed should speak for themselves:
% \changes{v2.2b}{1995/10/07}{Use \cs{@newctr}}
% \changes{v2.2c}{1995/11/23}{Use \cs{@newctr} properly.}
%    \begin{macrocode}
    \@definecounter{#1}\@newctr{#1}[#3]%
%    \end{macrocode}
%    Using |\@newctr| will give a proper error message if the counter
%    in |#3| is not defined.
%    In defining `|\theLem|' we must generate the desired macro
%    name by use of |\expandafter| and |\csname|.
%    \begin{macrocode}
    \expandafter\xdef\csname the#1\endcsname
%    \end{macrocode}
%    An |\xdef| is used in order to make the definition global,
%    and to ensure that it contains the replacement texts of
%    |\@thmcountersep| and |\@thmcounter|.\footnote{These
%    two macros can be defined by the document style.  Their default
%    values produce a `\texttt{.}' as separation and an arabic
%    representation of the number.} However, not everything should be
%    expanded. For example, it saves space to use |\thesection|
%    instead of its---at times---lengthy expansion.
%    \begin{macrocode}
      {\expandafter \noexpand \csname the#3\endcsname
       \@thmcountersep \@thmcounter{#1}}%
%    \end{macrocode}
%    Thus with the defaults of \LaTeX{}, |\theLem| would be
%    replaced by the command sequence |\thesection.\arabic{Lem}|.
%
%    We will now look at the definition of the macro which is executed
%    at the beginning of the actual environment (in our example this
%    macro is |\Lem|). It should be noted that we use an
%    ``|\expandafter| trick'' to expand only certain parts of the
%    replacement text at the time of the definition.
%    \begin{macrocode}
    \def\@tempa{\global\@namedef{#1}}%
    \expandafter \@tempa \expandafter{%
%    \end{macrocode}
%    First, the macro that contains the current definitions of
%    |\@begintheorem| and |\@opargtheorem| should be called
%    up.  The name of this macro---as is already known---has the form
%    |\th@|\meta{theorem\ style}; therefore, it must be called by
%    \begin{macrocode}
      \csname th@\the \theorem@style
%    \end{macrocode}
%    In addition the default theorem font should be changeable, i.e.\
%    we have to insert the contents of |\theorem@bodyfont|. For
%    that reason, we expand even further, beyond |\endcsname|,
%    and thus insert the contents of the \textsf{token} register
%    |\theorem@bodyfont| in the replacement text.
%    \begin{macrocode}
            \expandafter \endcsname \the \theorem@bodyfont
%    \end{macrocode}
%    Now it is time to call the macro |\@thm| which takes over
%    the further processing.  It has two arguments: the current
%    counter name (in our example, \texttt{Lem}), and the text of the
%    label (in our example, \texttt{Lemma}).
%    \begin{macrocode}
     \@thm{#1}{#2}}%
%    \end{macrocode}
%    With this, the `sub-definition' is complete.  The macro
%    |\@endtheorem| ends a theorem environment and is, so far, nothing
%    but an |\endtrivlist|.  (Hence it is defined globally, and not
%    within the theorem styles.\footnote{This has to be changed as
%    soon as theorem styles that change \texttt{\bslash @endtheorem}
%    exist.  In such a case, all existing styles must be changed as
%    well since they will have to reset the macro.}) Therefore, we can
%    set it equivalent to the macro that ends the theorem set (in our
%    example, |\endLem|).  However, if some day theorem styles exist
%    that do change |\@endtheorem|, we would have to use the
%    commented-out line instead.
%    \begin{macrocode}
    \global \expandafter \let \csname end#1\endcsname \@endtheorem
%   \global\@namedef{end#1}{\@endtheorem}%
%    \end{macrocode}
%    With these commands all the required definitions are employed,
%    unless the test |\@ifdefinable| has failed.  Therefore, we
%    end the second argument of this macro and with it the definition
%    of |\@xnthm|.
%    \begin{macrocode}
   }}
%    \end{macrocode}
% \end{macro}
%
% \begin{macro}{\@ynthm}
%    The definition of |\@ynthm| is completely analogous. In
%    this case the new counter that is defined is not reset within
%    another counter; thus the definition of |\the...|
%    is simplified:
%    \begin{macrocode}
\gdef\@ynthm#1#2{\expandafter\@ifdefinable\csname #1\endcsname
   {\@definecounter{#1}%
    \expandafter\xdef\csname the#1\endcsname{\@thmcounter{#1}}%
%    \end{macrocode}
%    The rest of the definition corresponds literally to that of
%    |\@xnthm|:
%    \begin{macrocode}
    \def\@tempa{\global\@namedef{#1}}\expandafter \@tempa
     \expandafter{\csname th@\the \theorem@style \expandafter
     \endcsname \the\theorem@bodyfont \@thm{#1}{#2}}%
    \global \expandafter \let \csname end#1\endcsname \@endtheorem}}
%    \end{macrocode}
% \end{macro}
%
% \begin{macro}{\@othm}
%    The definition of |\@othm| does not contain anything new.
%    \begin{macrocode}
\gdef\@othm#1[#2]#3{%
%    \end{macrocode}
% \changes{v2.2b}{95/10/07}{Build in check for counter defined}
%    We do not define a new counter but instead use one that has
%    already been defined.  Thus the only definition we need is that
%    of this pseudo-counter (i.e.\ |\the|\meta{env.\ name}).
%    First we check if |#2| corresponds to a known counter name.
%    \begin{macrocode}
  \expandafter\ifx\csname c@#2\endcsname\relax
   \@nocounterr{#2}%
  \else
   \expandafter\@ifdefinable\csname #1\endcsname
   {\expandafter \xdef \csname the#1\endcsname
     {\expandafter \noexpand \csname the#2\endcsname}%
%    \end{macrocode}
%    All other parts of the definition can be adopted from
%    |\@xnthm|. We have to remember, though, that in this case
%    the name of the current counter and the theorem label have moved
%    to the second and third arguments.
%    \begin{macrocode}
    \def\@tempa{\global\@namedef{#1}}\expandafter \@tempa
     \expandafter{\csname th@\the \theorem@style \expandafter
     \endcsname \the\theorem@bodyfont \@thm{#2}{#3}}%
    \global \expandafter \let \csname end#1\endcsname \@endtheorem}%
  \fi}
%    \end{macrocode}
% \end{macro}
%
% \subsection{Macros that are employed in a {\envfont theorem}
%    environment}
%
% \begin{macro}{\@thm}
%    The macro |\@thm| has to increase the current counter.  Then,
%    depending on whether the environment has (or does not have) an
%    optional argument, it has to branch into either |\@begintheorem|
%    or |\@opargtheorem|.
%    \begin{macrocode}
\gdef\@thm#1#2{\refstepcounter{#1}%
%    \end{macrocode}
%    Now we start a {\envfont trivlist} environment, and give
%    |\@topsep| and |\@topsepadd| the values of the \textsf{skip}
%    registers |\theorempreskipamount| and |\theorempostskipamount|.
%    The value in |\@topsep| is the vertical space that is inserted by
%    the first (and only) |\item| in our |\trivlist| whilst
%    |\@topsepadd| is inserted by |\@endparenv| at the end of that
%    {\envfont trivlist} environment.  By using these registers, we
%    obtain the desired space around a {\envfont theorem} environment.
%    \begin{macrocode}
   \trivlist
   \@topsep \theorempreskipamount               % used by first \item
   \@topsepadd \theorempostskipamount           % used by \@endparenv
%    \end{macrocode}
%    Now we have to test whether an optional argument has been given.
% \changes{v2.0d}{89/07/18}{Forgotten percent added.}
%    \begin{macrocode}
   \@ifnextchar [%
%    \end{macrocode}
%    If there is an optional argument, we will call |\@ythm|, and
%    move the arguments read back into the input stream.
%    \begin{macrocode}
   {\@ythm{#1}{#2}}%
%    \end{macrocode}
%    If not, we call |\@begintheorem|. Its first argument is the
%    name of the theorem set (hence the second argument of
%    |\@thm|). Its second argument is the macro that produces the
%    current number.
%    \begin{macrocode}
   {\@begintheorem{#2}{\csname the#1\endcsname}\ignorespaces}}
%    \end{macrocode}
% \end{macro}
%
% \begin{macro}{\@xthm}
% \begin{macro}{\@ythm}
%    Both these macros were originally called by |\@thm|.  We do
%    not need |\@xthm| anymore, hence we reset it to
%    |\relax|. The definition of |\@ythm| has not changed at
%    all from its definition in \LaTeX. In order to make the macros
%    easier to understand, we will
%    nevertheless present it (commented out).
%    \begin{macrocode}
\global\let\@xthm\relax
% \def\@ythm#1#2[#3]{\@opargbegintheorem{#2}{\csname
%           the#1\endcsname}{#3}\ignorespaces}
%</package>
%    \end{macrocode}
%    The primitive |\ignorespaces| in |\@ythm| and
%    |\@thm| is needed to remove the spaces between the
%    |\begin{...}| and the actual text.
% \end{macro}
% \end{macro}
%
%
% \subsection{Definition of the theorem substyles}
%
% As already pointed out, the theorem substyles, defined below, are
% only loaded when necessary.  Note that all these substyles, except
% \texttt{plain}, have |\slshape| as the default body font.
%
% \subsubsection{The \texttt{plain} style}
%
% \label{dok:th@plain} As the following macros use |@|, we have
% to locally set the |\catcode| of this symbol to ``letter''.
% This happens within a group, so that we do not have to worry about
% which |\catcode| that symbol had before.
%    \begin{macrocode}
%<*thp>
\begingroup \makeatletter
%    \end{macrocode}
% Since we are now within a group, we must make all definitions
% globally.  First we make sure that \texttt{theorem.sty} is loaded.
% This will allow us to use this file as a document style option
% without having to call \texttt{theorem} itself as an option.  At the
% same time, we assure that at least version 2 is loaded, since
% |\theorem@style| was not defined in earlier versions.
%    \begin{macrocode}
\@ifundefined{theorem@style}{\input{theorem.sty}}{}
\ProvidesFile{thp.sty}
             [\FMithmInfo]
%    \end{macrocode}
%
%
% \begin{macro}{\th@plain}
% \begin{macro}{\@begintheorem}
% \begin{macro}{\@opargbegintheorem}
%    |\theoremstyle{plain}| corresponds to the original
%    definition, except that the distances to the surrounding text are
%    determined by the parameters |\theorempreskipamount| and
%    |\theorempostskipamount|.
%    First we set the default body font.
%    \begin{macrocode}
\gdef\th@plain{\normalfont\itshape
%    \end{macrocode}
%    Then we define |\@begintheorem| and
%    |\@opargbegintheorem|. These two macros define how the
%    header of a theorem is typeset.  |\@opargbegintheorem| will
%    be called if a \textsf{theorem} environment with an optional
%    argument is encountered; otherwise, the header is constructed by
%    calling |\@begintheorem|.  If one of these macros is
%    executed, we are within a \textsf{trivlist} environment started by
%    |\@thm|.  So the theorem header is produced with an
%    |\item| command.
%
%    Instead of specifying the header font directly, all standard
%    theorem styles use the |\theorem@headerfont| macro to allow
%    customization.  The extra space (|\labelsep|) is necessary
%    because of problems in the \textsf{trivlist} environment.
%    \begin{macrocode}
  \def\@begintheorem##1##2{%
        \item[\hskip\labelsep \theorem@headerfont ##1\ ##2]}%
%    \end{macrocode}
%    The definition of |\@opargbegintheorem| is completely
%    analogous. The only difference is the fact that there exists a
%    third argument (which is the optional parameter of the
%    environment and contains additional information about the
%    theorem).  Customarily we enclose it in parentheses.
%    \begin{macrocode}
\def\@opargbegintheorem##1##2##3{%
   \item[\hskip\labelsep \theorem@headerfont ##1\ ##2\ (##3)]}}
%    \end{macrocode}
% \end{macro}
% \end{macro}
% \end{macro}
%
% \noindent
% We conclude with an |\endgroup| to restore the
% |\catcode| of |@|.
%    \begin{macrocode}
\endgroup
%</thp>
%    \end{macrocode}
%
%
% \subsubsection{The \texttt{break} style}
%
% This style option is stored in the file |thb.sty|.
% For the next two lines see the documentation for |\th@plain| on
% page~\pageref{dok:th@plain}.
%    \begin{macrocode}
%<*thb>
\begingroup \makeatletter
\@ifundefined{theorem@style}{\input{theorem.sty}}{}
\ProvidesFile{thb.sty}
             [\FMithmInfo]
%    \end{macrocode}
%
% \begin{macro}{\th@break}
%    |\theoremstyle{break}| produces a line break after the name
%    of the theorem.  The font is |\slshape|.  Hence, we define
%    |\th@break| as follows:
%    \begin{macrocode}
\gdef\th@break{\normalfont\slshape
  \def\@begintheorem##1##2{\item[%
%    \end{macrocode}
%    We run into the following problem: it is not possible to create
%    the header with |\item[|\meta{title}|]| and then start
%    a new line by, for example, |\mbox{}\\|. Such a definition
%    will fail whenever a list environment follows immediately.  With
%    the above construction, the |\mbox{}| causes the switch
%    |@inlabel| (cf.\ definition of |\list| and
%    |\trivlist| in \cite{latex:source}) to be set to \texttt{false}
%    and so the following list will insert additional vertical
%    space (|\topskip|). This is quite annoying. Therefore, we
%    create the line break within the |\item|. In order to ensure
%    that the text will begin at the proper position in the following
%    line, we simply pretend that the label does not take any
%    room.\footnote{This will lead to problems whenever very high
%    symbols occurring in the line tower into the
%    heading. So, something else has to be done here sometime.}
%    \begin{macrocode}
       \rlap{\vbox{\hbox{\hskip \labelsep\theorem@headerfont ##1\ ##2}%
                   \hbox{\strut}}}]}%
%    \end{macrocode}
%    Again, the definition of |\@opargbegintheorem| is
%    completely analogous.
%    \begin{macrocode}
\def\@opargbegintheorem##1##2##3{%
  \item[\rlap{\vbox{\hbox{\hskip \labelsep \theorem@headerfont
                     ##1\ ##2\ (##3)}%
                    \hbox{\strut}}}]}}
\endgroup
%</thb>
%    \end{macrocode}
% \end{macro}
%
%
% \subsubsection{The \texttt{changebreak} style}
%
%    \begin{macrocode}
% This style option is stored in the file |thcb.sty|.
%    \begin{macrocode}
%<*thcb>
\begingroup \makeatletter
\@ifundefined{theorem@style}{\input{theorem.sty}}{}
\ProvidesFile{thcb.sty}
             [\FMithmInfo]
%    \end{macrocode}
%
% \begin{macro}{\th@changebreak}
%    The \texttt{change-break} theorem style is like \texttt{break}
%    but with interchange of theorem name and theorem number.  Thus we
%    define |\th@changebreak| as follows:
%    \begin{macrocode}
\gdef\th@changebreak{\normalfont\slshape
  \def\@begintheorem##1##2{\item
       [\rlap{\vbox{\hbox{\hskip\labelsep \theorem@headerfont ##2\ ##1}%
                    \hbox{\strut}}}]}%
\def\@opargbegintheorem##1##2##3{%
  \item[\rlap{\vbox{\hbox{\hskip\labelsep \theorem@headerfont
                     ##2\ ##1\ (##3)}%
                    \hbox{\strut}}}]}}
\endgroup
%</thcb>
%    \end{macrocode}
% \end{macro}
%
%
% \subsubsection{The \texttt{change} style}
%
% This style option is stored in the file |thc.sty|.
%    \begin{macrocode}
%<*thc>
\begingroup \makeatletter
\@ifundefined{theorem@style}{\input{theorem.sty}}{}
\ProvidesFile{thc.sty}
             [\FMithmInfo]
%    \end{macrocode}
%
% \begin{macro}{\th@change}
%    The \texttt{change} theorem style corresponds to the
%    \texttt{change break} style without a linebreak after the header.
%    To say it in another way, it's the same as the \texttt{plain}
%    style but with number and name interchanged and |\slshape| as the
%    default font.
%    \begin{macrocode}
\gdef\th@change{\normalfont\slshape
  \def\@begintheorem##1##2{\item
       [\hskip\labelsep \theorem@headerfont ##2\ ##1]}%
\def\@opargbegintheorem##1##2##3{%
  \item[\hskip\labelsep \theorem@headerfont ##2\ ##1\ (##3)]}}
\endgroup
%</thc>
%    \end{macrocode}
% \end{macro}
%
%
% \subsubsection{The \texttt{marginbreak} style}
%
% This style option is the one used most often at Mainz.  It is saved
% in the file |thmb.sty|.
%    \begin{macrocode}
%<*thmb>
\begingroup \makeatletter
\@ifundefined{theorem@style}{\input{theorem.sty}}{}
\ProvidesFile{thmb.sty}
             [\FMithmInfo]
%    \end{macrocode}
%
% \begin{macro}{\th@marginbreak}
%    The \texttt{margin break} style is nearly the same as the
%    \texttt{change break} style. The only difference is the placement
%    of the theorem number.  We use |\llap| to place it in the left
%    margin.
%
%    In this style |\labelsep| denotes the separation between the
%    number and the text.
%    \begin{macrocode}
\gdef\th@marginbreak{\normalfont\slshape
  \def\@begintheorem##1##2{\item
       [\rlap{\vbox{\theorem@headerfont
                    \hbox{\llap{##2}\hskip\labelsep ##1}%
                    \hbox{\strut}}}]}%
\def\@opargbegintheorem##1##2##3{%
  \item[\rlap{\vbox{\theorem@headerfont
                    \hbox{\llap{##2}\hskip\labelsep ##1\ (##3)}%
                    \hbox{\strut}}}]}}
\endgroup
%</thmb>
%    \end{macrocode}
% \changes{v2.1c}{90/02/17}{`endgroup outside macrocode env.}
% \end{macro}
%
%
%
% \subsubsection{The \texttt{margin} style}
%
% This style option is stored in the file |thm.sty|.
%    \begin{macrocode}
%<*thm>
\begingroup \makeatletter
\@ifundefined{theorem@style}{\input{theorem.sty}}{}
\ProvidesFile{thm.sty}
             [\FMithmInfo]
%    \end{macrocode}
%
% \begin{macro}{\th@margin}
%    Again this is only a variant of the theorem styles described
%    above without any new ideas.
%    \begin{macrocode}
\gdef\th@margin{\normalfont\slshape
  \def\@begintheorem##1##2{\item
       [\theorem@headerfont \llap{##2}\hskip\labelsep ##1]}%
\def\@opargbegintheorem##1##2##3{%
  \item[\theorem@headerfont \llap{##2}\hskip\labelsep ##1\ (##3)]}}
\endgroup
%</thm>
%    \end{macrocode}
% \end{macro}
%
%
%
%
% \subsection{Final Definitions}
%
% \begin{macro}{\theorempreskipamount}
% \begin{macro}{\theorempostskipamount}
%    The \textsf{skip} parameters that regulate the vertical empty space
%    before and after the {\envfont theorem} environment have
%    to be allocated as well.
% \changes{v2.0f}{89/07/20}{Unnecessary `global' removed.}
%    \begin{macrocode}
%<*package>
\newskip\theorempreskipamount
\newskip\theorempostskipamount
%    \end{macrocode}
%    Since we have used the same values for all theorem sets, we now can
%    assign them.
%    \begin{macrocode}
\global\setlength\theorempreskipamount{12pt plus 5pt minus 3pt}
\global\setlength\theorempostskipamount{8pt plus 3pt minus 1.5pt}
%    \end{macrocode}
% \end{macro}
% \end{macro}
%
%
% \begin{macro}{\@endtheorem}
%    The same holds for the macro |\@endtheorem|, which ends a
%    {\envfont theorem} environment. Since it is the same for all
%    theorem sets, it is removed from the macros |\th@|\meta{style}.
%    It simply ends the {\envfont trivlist} environment, which was
%    begun in |\@thm|.
%    \begin{macrocode}
\global\let\@endtheorem=\endtrivlist
%    \end{macrocode}
% \end{macro}
%
%
%
% \begin{macro}{\@preamblecmds}
%    All macros defined above are to be used only in the preamble.
%    Therefore, we insert them in |\@preamblecmds| which will disable
%    them at begin document. This is done by the internal \LaTeXe{}
%    command |\@onlypreamble|.
% \changes{v2.0g}{89/09/19}{`theoremheaderfont and `theorembodyfont
%    added.}
% \changes{v2.2a}{94/02/03}{Use \LaTeXe{} interface}
%    \begin{macrocode}
\@onlypreamble\@xnthm
\@onlypreamble\@ynthm
\@onlypreamble\@othm
\@onlypreamble\newtheorem
\@onlypreamble\theoremstyle
\@onlypreamble\theorembodyfont
\@onlypreamble\theoremheaderfont
%    \end{macrocode}
% \end{macro}
%
% Finally we declare the \texttt{plain} theorem style to be the default.
% \changes{v2.1b}{90/06/04}{Fixed missing `item by declaring default
%    style.}
%    \begin{macrocode}
\theoremstyle{plain}
%</package>
%    \end{macrocode}
%
% \Finale

\endinput
}{}
\ProvidesFile{thp.sty}
             [\FMithmInfo]
%    \end{macrocode}
%
%
% \begin{macro}{\th@plain}
% \begin{macro}{\@begintheorem}
% \begin{macro}{\@opargbegintheorem}
%    |\theoremstyle{plain}| corresponds to the original
%    definition, except that the distances to the surrounding text are
%    determined by the parameters |\theorempreskipamount| and
%    |\theorempostskipamount|.
%    First we set the default body font.
%    \begin{macrocode}
\gdef\th@plain{\normalfont\itshape
%    \end{macrocode}
%    Then we define |\@begintheorem| and
%    |\@opargbegintheorem|. These two macros define how the
%    header of a theorem is typeset.  |\@opargbegintheorem| will
%    be called if a \textsf{theorem} environment with an optional
%    argument is encountered; otherwise, the header is constructed by
%    calling |\@begintheorem|.  If one of these macros is
%    executed, we are within a \textsf{trivlist} environment started by
%    |\@thm|.  So the theorem header is produced with an
%    |\item| command.
%
%    Instead of specifying the header font directly, all standard
%    theorem styles use the |\theorem@headerfont| macro to allow
%    customization.  The extra space (|\labelsep|) is necessary
%    because of problems in the \textsf{trivlist} environment.
%    \begin{macrocode}
  \def\@begintheorem##1##2{%
        \item[\hskip\labelsep \theorem@headerfont ##1\ ##2]}%
%    \end{macrocode}
%    The definition of |\@opargbegintheorem| is completely
%    analogous. The only difference is the fact that there exists a
%    third argument (which is the optional parameter of the
%    environment and contains additional information about the
%    theorem).  Customarily we enclose it in parentheses.
%    \begin{macrocode}
\def\@opargbegintheorem##1##2##3{%
   \item[\hskip\labelsep \theorem@headerfont ##1\ ##2\ (##3)]}}
%    \end{macrocode}
% \end{macro}
% \end{macro}
% \end{macro}
%
% \noindent
% We conclude with an |\endgroup| to restore the
% |\catcode| of |@|.
%    \begin{macrocode}
\endgroup
%</thp>
%    \end{macrocode}
%
%
% \subsubsection{The \texttt{break} style}
%
% This style option is stored in the file |thb.sty|.
% For the next two lines see the documentation for |\th@plain| on
% page~\pageref{dok:th@plain}.
%    \begin{macrocode}
%<*thb>
\begingroup \makeatletter
\@ifundefined{theorem@style}{% \iffalse meta-comment
%
% Copyright (C) 1993-2024
%
% The LaTeX Project and any individual authors listed elsewhere
% in this file.
%
% This file is part of the Standard LaTeX `Tools Bundle'.
% -------------------------------------------------------
%
% It may be distributed and/or modified under the
% conditions of the LaTeX Project Public License, either version 1.3c
% of this license or (at your option) any later version.
% The latest version of this license is in
%    https://www.latex-project.org/lppl.txt
% and version 1.3c or later is part of all distributions of LaTeX
% version 2005/12/01 or later.
%
% The list of all files belonging to the LaTeX `Tools Bundle' is
% given in the file `manifest.txt'.
%
% \fi
%
%%
%
%
\def\FMithmInfo{2023/07/05 v2.2c Theorem extension package (FMi)}
%
% \ProvidesFile{theorem.dtx}[\FMithmInfo]
%\iffalse   % this is a METACOMMENT !
%%
%% Package `theorem' to use with LaTeX2e
%% Copyright (C) 1989-1995 Frank Mittelbach, all rights reserved.
%<+package>\@ifundefined{theorem@style}{}{\endinput}
%<+package>\NeedsTeXFormat{LaTeX2e}[1995/06/01]
%<+package>\ProvidesPackage{theorem}
%<+package>         [\FMithmInfo]
%
% \fi
%
% \changes{v1.0}{87/05/07}{First draft.}
% \changes{v1.0a}{87/05/07}{`thistheoremstyle' added.}
% \changes{v1.1a}{87/07/29}{New concept: redefine the internal
%                          \protect\LaTeX{} commands `@xnthm' etc.}
% \changes{v1.1b}{87/07/31}{`theoremfont' used to specify font.}
% \changes{v1.2a}{88/01/08}{`@endtheorem' now globally defined.}
% \changes{v1.2b}{88/01/11}{`newtheorem' only allowed in the preamble.}
% \changes{v1.3a}{88/02/14}{`theoremfont' now a toks register.}
%
% \changes{v2.0a}{88/11/16}{New implementation and new documentation.}
% \changes{v2.0a}{88/11/16}{No layout changes but internal modifications
%                           old substyles won't work.}
% \changes{v2.0b}{89/01/30}{`global' used for all definitions.}
% \changes{v2.0c}{89/06/15}{Documentation translated to English.
%                            Thanks to LLL and Nina}
% \changes{v2.1a}{90/02/15}{Support for new font selection.}
% \changes{v2.2a}{94/02/03}{Cleanup for LaTeX2e}
%
% \DoNotIndex{\@endparenv,\@ifdefinable,\@ifnextchar,\@ifundefined}
% \DoNotIndex{\@namedef,\@nameuse,\@spaces,\@tempa,\@warning}
% \DoNotIndex{\ }
% \DoNotIndex{\begingroup,\bfseries}
% \DoNotIndex{\csname}
% \DoNotIndex{\def,\do,\docdate}
% \DoNotIndex{\endcsname,\endgroup,\endinput,\endtrivlist,\expandafter}
% \DoNotIndex{\filedate,\fileversion}
% \DoNotIndex{\gdef,\global}
% \DoNotIndex{\hbox,\hskip}
% \DoNotIndex{\ignorespaces,\input,\itshape,\item}
% \DoNotIndex{\labelsep,\let,\llap}
% \DoNotIndex{\makeatletter}
% \DoNotIndex{\noexpand}
% \DoNotIndex{\refstepcounter,\relax,\rlap}
% \DoNotIndex{\setlength,\slshape,\space,\string,\strut}
% \DoNotIndex{\the,\trivlist,\typeout}
% \DoNotIndex{\vbox}
% \DoNotIndex{\xdef}
%
% \def\deflabel#1{\ttfamily #1\hfill}
% \def\deflist#1{\begin{list}{}{\settowidth\labelwidth{\ttfamily #1}%
%                               \setlength\leftmargin\labelwidth
%                               \addtolength\leftmargin\labelsep
%                               \let\makelabel\deflabel}}
% \def\enddeflist{\end{list}}
% \def\envfont{\normalfont\ttfamily}
%
% \MakeShortVerb{\|}
%
% \newcommand{\TUB}{{\sl TUGboat\/}}
%
% \GetFileInfo{theorem.dtx}
% \title{An Extension of the
%        \LaTeX{} theorem environment\thanks{This file
%        has version number \fileversion, last
%        revised \filedate.}}
% \author{Frank Mittelbach}
%
% \MaintainedByLaTeXTeam{tools}
% \maketitle
%
% \begin{abstract}
%    The macros described in this paper yield an extension of the
%    \LaTeX{} theorem mechanism. It is designed to satisfy the
%    different requirements of various journals. Thus, the
%    layout of the ``theorems'' can be manipulated by determining a
%    ``style''. This article describes not only the use, but
%    also the definition, of the necessary macros.
% \end{abstract}
%
% \section*{Preface to version 2.2}
%
% For \LaTeXe{} this package did not need any fundamental changes. I
% only modified the messages generated so that theorem layout styles
% will show up with the |\listfiles| command and cleaned the section
% on the New Font Selection Scheme since this is now included in
% \LaTeX.
%
% \section*{Preface to version 2.1}
%
% This version is identical to 2.0g described in \TUB\ 10\#3 except
% for some internal defaults which are now set depending on the used
% font selection scheme.
%
% This was done to avoid unpleasant surprises if the new font selection
% scheme is in force. For further details see
% section~\ref{sec:newfonts} and \cite{bk:GMS94}.
%
% \section{Introduction}
%
% For our purposes here, ``theorems'' are labelled enunciations,
% often set off from the main text by extra space and a font change.
% Theorems, corollaries, conjectures, definitions, and remarks
% are all instances of ``theorems''.  The ``header'' of these structures
% is composed of a label (such as \textsc{Theorem} or \textsc{Remark})
% and a number which serializes an item in the sequence of items
% with the same label.
%
% Shortly after the introduction of \LaTeX{} at the Fachbereich
% Mathematik in Mainz, the desire to manipulate the layout of
% ``theorems'' arose.  In Mainz, the following two conventions came
% into general use:
% \begin{enumerate}
%   \item The number of the theorem is shown in the margin.
%   \item There is a line break at
%         the end of the theorem header.
% \end{enumerate}
%
% Additionally, some journals require different formats which depend on
% the ``sort of theorem'': e.g.\ often remarks and definitions are set
% in |\upshape|, while |\itshape| is employed for main theorems.
%
% Confronted with these requirements, a theorem environment was
% developed in Mainz which allows separate determination of the
% layout of the ``theorems sets'', comparable to |\pagestyle|.
%
%
% \section{The user interface}
% \subsection{Defining new theorem sets}
%
% \DescribeMacro\newtheorem
% As in the original \LaTeX{} version, the command |\newtheorem|
% defines a new ``theorem set'' or ``theorem-like structure''.
% Two required arguments name the new environment and give the text
% to be typeset with each instance of the new ``set'', while
% an optional argument determines how
% the ``set'' is enumerated:
% \begin{description}
%    \item[\ttfamily \bslash newtheorem\{foo\}\{bar\}]
%       The theorem set {\envfont foo} (whose name is \texttt{bar})
%       uses its own counter.
%    \item[\ttfamily \bslash newtheorem\{foo2\}{[foo]}\{bar2\}]
%       The theorem set {\envfont foo2} (printed name \texttt{bar2})
%       uses the same counter as the theorem set \texttt{foo}.
%    \item[\ttfamily \bslash newtheorem\{foo3\}\{bar3\}{[section]}]
%       The theorem set {\envfont foo3} (printed name \texttt{bar3}) is
%       enumerated within the counter \texttt{section}, i.e.\ with every
%       new |\section| the enumeration begins again with 1, and
%       the enumeration is composed from the section-number and the
%       theorem counter itself.
% \end{description}
%
% \DescribeMacro\theoremstyle
% Additionally, the command |\theoremstyle| can define the layout
% of various, or all, theorem sets.  It should be noted that any
% theorem set defined by |\newtheorem| is typeset in the
% |\theoremstyle| that is current at the time of the definition.
% Thus, the following
% \begin{verbatim}
%    \theoremstyle{break}        \newtheorem{Cor}{Corollary}
%    \theoremstyle{plain}        \newtheorem{Exa}{Example}[section]
%\end{verbatim}                             ^^A No blank at line start
% leads to the result that the set {\envfont Cor} is formatted in the
% style \texttt{break}, while the set {\envfont Exa} and all the
% following ones are formatted in the style \texttt{plain}, unless
% another |\theoremstyle| follows.  Since the definitions installed by
% |\newtheorem| are global, one also can limit |\theoremstyle| locally
% by grouping braces.
%
% \DescribeMacro\theorembodyfont
% The choice of the font for the theorem body is completely
% independent of the chosen |\theoremstyle|; this has proven to
% be very advantageous.  For example,
% \begin{verbatim}
%   {\theorembodyfont{\upshape}          \newtheorem{Rem}{Remark}}
%\end{verbatim}
% defines a theorem set {\envfont Rem}, which will be set in
% |\upshape| in the current layout (which in our example is
% \texttt{plain}).  As with |\theoremstyle|, the |\theorembodyfont|
% chosen is that current at the time of |\newtheorem|.  If
% |\theorembodyfont| is not specified or one defines
% |\theorembodyfont{}|, then the font used will be that defined by the
% |\theoremstyle|.
%
% \DescribeMacro\theoremheaderfont It is also possible to customize
% the font used for the theorem headers.  This is, however, a global
% declaration, and therefore there should be at most one
% |\theoremheaderfont| declaration in the preamble.\footnote{If it is
% actually necessary to have different header fonts, one has to define
% new theorem styles (substituting the desired font) or specify the
% information directly in the \texttt{\bslash newtheorem} declaration
% (the unclean variant).}
%
% \DescribeMacro\theorempreskipamount
% \DescribeMacro\theorempostskipamount Two additional parameters
% affect the vertical space around the theorem environments:
% |\theorempreskipamount| and |\theorempostskipamount| define,
% respectively, the spacing before and after such an environment.
% These parameters apply for all theorem sets and can be manipulated
% with the ordinary length macros.  They are rubber lengths
% (`\textsf{skips}'), and therefore can contain \texttt{plus} and
% \texttt{minus} parts.
%
% Since the definition of theorem sets should---most sensibly---be
% placed in the preamble, we only allow installation there.
% It is therefore possible to release the memory used here after
% |\begin{document}|, in order to make room for other
% applications.
%
%
% \subsection{Existing theorem styles}
%
% The following theorem styles exist to date:
% \begin{deflist}{marginbreak}
%    \item[plain]
%       This theorem style emulates the original \LaTeX{} definition,
%       except that additionally the parameters
%       |\theorem...skipamount| are used.
%    \item[break]
%       In this style, the theorem header is followed by a line
%       break.
%    \item[marginbreak]
%       The theorem number is set in the margin, and there is a line
%       break as in \texttt{break}.
%    \item[changebreak]
%       Like \texttt{break}, but with header number and text
%       interchanged.
%    \item[change]
%       Header number and text are interchanged, without a line break.
%    \item[margin]
%       The number is set in the left margin, without a line break.
% \end{deflist}
% All styles (except \texttt{plain}) select |\slshape| as the default
% |\theorembodyfont|.
%
%
% \subsection{Examples}
%
% Given the above theorem sets {\envfont Cor}, {\envfont Exa}
% and {\envfont Rem},
% suppose that the preamble also contains the declarations:
% \begin{verbatim}
%    \theoremstyle{marginbreak}   \newtheorem{Lem}[Cor]{Lemma}
%    \theoremstyle{change}
%    \theorembodyfont{\itshape}        \newtheorem{Def}[Cor]{Definition}
%
%    \theoremheaderfont{\scshape}
%\end{verbatim}
% Then the following are some typical examples of the typeset output
% resulting from their use.
%
% \begin{Cor}
%    This is a sentence typeset in the theorem environment {\envfont
%    Cor}.
% \end{Cor}
%
% \begin{Exa}
%    This is a sentence typeset in the theorem environment {\envfont
%    Exa}.
% \end{Exa}
%
% \begin{Rem}
%    This is a sentence typeset in the theorem environment {\envfont
%    Rem}.
% \end{Rem}
%
% \begin{Lem}[Ben User]
%    This is a sentence typeset in the theorem environment {\envfont
%    Lem}.
% \end{Lem}
%
% \begin{Def}[Very impressive Definition]
%    This is a sentence typeset in the theorem environment {\envfont
%    Def}.
% \end{Def}
%
% The last two examples show the effect of the optional argument to
% a theorem environment (it is the text typeset in parentheses).
%
%
% \section{Special Considerations}
% \label{sec:newfonts}
%
% Theoremheader and body are implemented as a unit. This means that
% the |\theoremheaderfont|  will inherit characteristics of the
% |\theorembodyfont| in \LaTeXe. Thus, if for example
% |\theorembodyfont| is |\itshape| and |\theoremheaderfont|
% is |\bfseries| the font selected for the header will have the
% characteristics `bold extended italic'. If this is not desired one
% should set the |\theoremheaderfont| to something like
% \begin{verbatim}
%      \theoremheaderfont{\normalfont\bfseries}
%\end{verbatim}
% i.e.\ supplying all necessary font information explicitly.
%
%
%
% \section{Acknowledgements}
%
% The publication of this set of macros was only possible with the
% help of Christina Busse (translating the manuscript into English),
% Joachim Pense (playing the r\^ole of typist), Chris Rowley
% (looking everything over) and many others providing useful
% suggestions.
%
% \MaybeStop{%
%    \begin{thebibliography}{9}
%       \bibitem{bk:GMS94} \textsc{M.~Goossens}, \textsc{F.~Mittelbach}
%         and \textsc{A.~Samarin}.
%         \newblock The \LaTeX{} Companion.
%         \newblock
%         Addison-Wesley, Reading, Massachusetts, 1994.
%       \bibitem{latex:source} \textsc{Lamport, Leslie.} \newblock
%                               \texttt{latex.tex}, version 2.09, date
%                               Feb. 1990.
%    \end{thebibliography}
%    \PrintIndex
%    \PrintChanges}
%
%
% \section{The documentation driver file}
%
% The next bit of code contains the documentation driver file for
% \TeX{}, i.e., the file that will produce the documentation you are
% currently reading. It will be extracted from this file by the
% \texttt{docstrip} program. Since it is the first code in the file
% one can alternatively process this file directly with \LaTeXe{} to
% obtain the documentation.
%
%    \begin{macrocode}
%<*driver>
\documentclass{ltxdoc}

\usepackage{theorem}

%<+driver>% The next few lines define theorem sets which are used
%<+driver>% in the example section of the documentation.

\theoremstyle{break}       \newtheorem{Cor}{Corollary}
\theoremstyle{plain}       \newtheorem{Exa}{Example}[section]
{\theorembodyfont{\upshape}\newtheorem{Rem}{Remark}}
\theoremstyle{marginbreak} \newtheorem{Lem}[Cor]{Lemma}
\theoremstyle{change}
\theorembodyfont{\itshape} \newtheorem{Def}[Cor]{Definition}

\theoremheaderfont{\scshape}

\RecordChanges

\begin{document}
   \DocInput{theorem.dtx}
\end{document}
%</driver>
%    \end{macrocode}
%
% \section{Definition of the Macros}
%
% If the file has been loaded before, we abort immediately. If not the
% package announces itself (this is actually done at the very top of
% the file---the way it is done isn't good style so don't copy it).
% \changes{v2.0e}{89/07/19}{Spaces removed from `typeout'.}
% \changes{v2.2b}{95/11/19}{Announce moved to top of file}
%    \begin{macrocode}
%<*package>
%\@ifundefined{theorem@style}{}{\endinput}
%\def\FMithmInfo{1995/11/19 v2.2b Theorem extension package (FMi)}
%\ProvidesPackage{theorem}[\FMithmInfo]
%    \end{macrocode}
%
%
% \subsection{Definition of theorem styles and fonts}
%
% All the definitions in this file are done globally to allow
% inputting this file inside a group.
%
% \begin{macro}{\theoremstyle}
%    Before a theorem style can be installed, the chosen style must be
%    known. For that reason, we must test to see that
%    |\th@|\meta{style} is known or, more precisely, that it is
%    different from |\relax|. If the style is not known then
%    |\th@plain| is used.
%    \begin{macrocode}
\gdef\theoremstyle#1{%
   \@ifundefined{th@#1}{\@warning
          {Unknown theoremstyle `#1'. Using `plain'}%
          \theorem@style{plain}}%
%    \end{macrocode}
%    We save the theorem style to be used in the token register
%    |\theorem@style|.
%    \begin{macrocode}
      {\theorem@style{#1}}%
%    \end{macrocode}
%    Now we ``evaluate'' the theorem style: this means, we call the
%    macro |\th@|\meta{style} which will activate the relevant
%    definitions which are contained in a separate file.  This is done
%    in a group to suppress changes to the current font.  This could
%    otherwise pose problems together with the new font selection
%    scheme\footnote{When I printed the original article using the new
%    font selection scheme I ended with a document with slanted
%    typefaces (text headings and all) simply because one of the
%    theorem styles used \texttt{\bslash sl} at toplevel.} if the
%    |\th@|\meta{style} is evaluated a second time.
% \changes{v2.1a}{90/02/15}{Group added around `th@..}
% \changes{v2.1a}{90/02/15}{Reseting of `..@begintheorem removed.}
%    \begin{macrocode}
      \begingroup
        \csname th@\the\theorem@style \endcsname
      \endgroup}
%    \end{macrocode}
% \end{macro}
%
%
% \begin{macro}{\@begintheorem}
% \changes{v2.1a}{90/02/15}{Relaxed at toplevel.}
% \begin{macro}{\@opargbegintheorem}
% \changes{v2.1a}{90/02/15}{Relaxed at toplevel.}
%    We reset |\@begintheorem| and |\@opargbegintheorem| to
%    |\relax| since these commands are no longer necessary at
%    toplevel. This will save a few tokens.
%    \begin{macrocode}
\global\let\@begintheorem\relax
\global\let\@opargbegintheorem\relax
%    \end{macrocode}
% \end{macro}
% \end{macro}
%
% \begin{macro}{\theorem@style}
%    Obviously the \textsf{token} register used above has to be
%    allocated. To assure the utmost compatibility with the
%    original \LaTeX{} definition, we set the default theorem style to
%    \texttt{plain}, which implements the usual \LaTeX{} convention.
% \changes{v2.0f}{89/07/20}{Unnecessary `global' removed.}
%    \begin{macrocode}
\newtoks\theorem@style
\global\theorem@style{plain}
%    \end{macrocode}
% \end{macro}
%
% \begin{macro}{\theorembodyfont}
% \begin{macro}{\theorem@bodyfont}
%    If the |\theorembodyfont| is set by the user then it should not
%    interact with the default font set in the theorem style. When the
%    new font selection is in force this may happen if, for example,
%    the default is |\itshape| and the new |\theorembodyfont| is
%    |\sffamily|. So we add a |\reset@font| command in front of the
%    user definition.
% \changes{v2.0f}{89/07/20}{Unnecessary `global' removed.}
% \changes{v2.0g}{89/09/19}{`theoremfont to `theorembodyfont.}
% \changes{v2.1c}{92/06/22}{`reset@font added and logic changed.}
% \changes{v2.1d}{92/10/28}{Reset to empty if arg is empty}
%    \begin{macrocode}
\gdef\theorembodyfont#1{%
%    \end{macrocode}
%    We check if the argument supplied is empty and if so put nothing
%    into the |\theorem@bodyfont| token register to allow for
%    |\theorembodyfont{}| as a mean of using the default of the
%    current |\theoremstyle|.
%    \begin{macrocode}
   \def\@tempa{#1}%
   \ifx\@tempa\@empty
    \theorem@bodyfont{}%
   \else
    \theorem@bodyfont{\reset@font#1}%
   \fi
   }
\newtoks\theorem@bodyfont
\global\theorem@bodyfont{}
%    \end{macrocode}
% \end{macro}
% \end{macro}
%
%
% \begin{macro}{\theoremheaderfont}
% \changes{v2.0g}{89/09/19}{Macro added.}
%    The font for the theorem headers is handled differently because
%    this definition applies to all theorem styles.
%    \begin{macrocode}
\gdef\theoremheaderfont#1{\gdef\theorem@headerfont{#1}%
%    \end{macrocode}
%    After using the macro once it is redefined to produce an error
%    message.
%    \begin{macrocode}
       \gdef\theoremheaderfont##1{%
        \typeout{\string\theoremheaderfont\space should be used
                 only once.}}}
%    \end{macrocode}
% \end{macro}
%
% \begin{macro}{\theorem@headerfont}
% \changes{v2.0g}{89/09/19}{Macro added.}
%    To set the |\theorem@headerfont| default we first test if
%    the new fontselection scheme is in force.
% \changes{v2.1a}{90/02/15}{Support for new font selection.}
%    \begin{macrocode}
\ifx\upshape\undefined
%    \end{macrocode}
%    If not we define it to expand into |\bfseries|. We don't use
%    |\let| just in case a following style option redefines this
%    macro.
%    \begin{macrocode}
\gdef\theorem@headerfont{\bfseries}
%    \end{macrocode}
%    Otherwise we reset the current shape before calling |\bfseries|.
%    \begin{macrocode}
\else \gdef\theorem@headerfont{\normalfont\bfseries}\fi
%    \end{macrocode}
% \end{macro}
%
% \begin{macro}{\th@plain}
% \begin{macro}{\th@break}
% \begin{macro}{\th@marginbreak}
% \begin{macro}{\th@changebreak}
% \begin{macro}{\th@change}
% \begin{macro}{\th@margin}
%    The different styles are defined in macros such as
%    |\th@plain|.  Since memory space is precious in
%    ``non-Big-versions'', we have to avoid offering too many unused
%    definitions.  Therefore we define these styles in separate files
%    that can be loaded on demand.  Thus the commands themselves only
%    load these files. We use |\@input@|, a \LaTeXe{} internal command
%    that ensures that the file will be listed with |\listfiles|.
% \changes{v2.2a}{94/02/02}{Use `@input@ to load theorem layout files}
%    \begin{macrocode}
\gdef\th@plain{\@input@{thp.sty}}
\gdef\th@break{\@input@{thb.sty}}
\gdef\th@marginbreak{\@input@{thmb.sty}}
\gdef\th@changebreak{\@input@{thcb.sty}}
\gdef\th@change{\@input@{thc.sty}}
\gdef\th@margin{\@input@{thm.sty}}
%    \end{macrocode}
%    This list will be expanded when new styles become
%    available.  For testing,  just append new theorem substyles
%    as document options.
% \end{macro}
% \end{macro}
% \end{macro}
% \end{macro}
% \end{macro}
% \end{macro}
%
%
%
% \subsection{Definition of a new theorem set}
%
% As already pointed out, a new theorem environment can be defined in
% three different ways:
% \begin{verbatim}
%    \newtheorem{Lem}{Lemma}
%    \newtheorem{Lem}{Lemma}[section]
%    \newtheorem{Lem}[Theorem]{Lemma}
%\end{verbatim}
% The function of the macro |\newtheorem| is to recognize these cases
% and then to branch into one of the three macros |\@ynthm|, |\@xnthm|
% or |\@othm|.  This mechanism is adopted unchanged from
% \cite{latex:source}; the essential point here is that, for example,
% in the second case, the arguments \texttt{Lem}, \texttt{Lemma} and
% \texttt{section} are passed over to the macro |\@xnthm|.
%
% We inspect this case first because the others present fewer problems,
% and thus are easily derived from this one.
%
% \begin{macro}{\@xnthm}
%    For our example arguments, the macro |\@xnthm| must fulfill
%    the following:
%    \begin{itemize}
%       \item
%          Define a new \LaTeX-counter `Lem'
%       \item
%          reset this counter within a |\section|
%       \item
%          define the macro |\theLem|
%       \item
%          define the environment macros |\Lem| and
%          |\endLem| using the current
%          |\theoremstyle| and |\theorem@bodyfont|.
%    \end{itemize}
%    Obviously, all this should happen only if the first argument
%    of |\@xnthm| (i.e.\ \texttt{Lem} in our example) is chosen so
%    as not to conflict with any previously defined commands or
%    environments.  This test is performed by the \LaTeX{} macro
%    |\@ifdefinable|.
%    \begin{macrocode}
\gdef\@xnthm#1#2[#3]{\expandafter\@ifdefinable\csname #1\endcsname
%    \end{macrocode}
%    Therefore, the first argument of |\@ifdefinable| is the
%    expansion (in the example, |\Lem|) of
%    |\csname||#1\endcsname|.  The second argument is
%    executed only if the test has been completed successfully.
%    \begin{macrocode}
   {%
%    \end{macrocode}
%    Now we define the new counter. The names of the \LaTeX{}
%    macros employed should speak for themselves:
% \changes{v2.2b}{1995/10/07}{Use \cs{@newctr}}
% \changes{v2.2c}{1995/11/23}{Use \cs{@newctr} properly.}
%    \begin{macrocode}
    \@definecounter{#1}\@newctr{#1}[#3]%
%    \end{macrocode}
%    Using |\@newctr| will give a proper error message if the counter
%    in |#3| is not defined.
%    In defining `|\theLem|' we must generate the desired macro
%    name by use of |\expandafter| and |\csname|.
%    \begin{macrocode}
    \expandafter\xdef\csname the#1\endcsname
%    \end{macrocode}
%    An |\xdef| is used in order to make the definition global,
%    and to ensure that it contains the replacement texts of
%    |\@thmcountersep| and |\@thmcounter|.\footnote{These
%    two macros can be defined by the document style.  Their default
%    values produce a `\texttt{.}' as separation and an arabic
%    representation of the number.} However, not everything should be
%    expanded. For example, it saves space to use |\thesection|
%    instead of its---at times---lengthy expansion.
%    \begin{macrocode}
      {\expandafter \noexpand \csname the#3\endcsname
       \@thmcountersep \@thmcounter{#1}}%
%    \end{macrocode}
%    Thus with the defaults of \LaTeX{}, |\theLem| would be
%    replaced by the command sequence |\thesection.\arabic{Lem}|.
%
%    We will now look at the definition of the macro which is executed
%    at the beginning of the actual environment (in our example this
%    macro is |\Lem|). It should be noted that we use an
%    ``|\expandafter| trick'' to expand only certain parts of the
%    replacement text at the time of the definition.
%    \begin{macrocode}
    \def\@tempa{\global\@namedef{#1}}%
    \expandafter \@tempa \expandafter{%
%    \end{macrocode}
%    First, the macro that contains the current definitions of
%    |\@begintheorem| and |\@opargtheorem| should be called
%    up.  The name of this macro---as is already known---has the form
%    |\th@|\meta{theorem\ style}; therefore, it must be called by
%    \begin{macrocode}
      \csname th@\the \theorem@style
%    \end{macrocode}
%    In addition the default theorem font should be changeable, i.e.\
%    we have to insert the contents of |\theorem@bodyfont|. For
%    that reason, we expand even further, beyond |\endcsname|,
%    and thus insert the contents of the \textsf{token} register
%    |\theorem@bodyfont| in the replacement text.
%    \begin{macrocode}
            \expandafter \endcsname \the \theorem@bodyfont
%    \end{macrocode}
%    Now it is time to call the macro |\@thm| which takes over
%    the further processing.  It has two arguments: the current
%    counter name (in our example, \texttt{Lem}), and the text of the
%    label (in our example, \texttt{Lemma}).
%    \begin{macrocode}
     \@thm{#1}{#2}}%
%    \end{macrocode}
%    With this, the `sub-definition' is complete.  The macro
%    |\@endtheorem| ends a theorem environment and is, so far, nothing
%    but an |\endtrivlist|.  (Hence it is defined globally, and not
%    within the theorem styles.\footnote{This has to be changed as
%    soon as theorem styles that change \texttt{\bslash @endtheorem}
%    exist.  In such a case, all existing styles must be changed as
%    well since they will have to reset the macro.}) Therefore, we can
%    set it equivalent to the macro that ends the theorem set (in our
%    example, |\endLem|).  However, if some day theorem styles exist
%    that do change |\@endtheorem|, we would have to use the
%    commented-out line instead.
%    \begin{macrocode}
    \global \expandafter \let \csname end#1\endcsname \@endtheorem
%   \global\@namedef{end#1}{\@endtheorem}%
%    \end{macrocode}
%    With these commands all the required definitions are employed,
%    unless the test |\@ifdefinable| has failed.  Therefore, we
%    end the second argument of this macro and with it the definition
%    of |\@xnthm|.
%    \begin{macrocode}
   }}
%    \end{macrocode}
% \end{macro}
%
% \begin{macro}{\@ynthm}
%    The definition of |\@ynthm| is completely analogous. In
%    this case the new counter that is defined is not reset within
%    another counter; thus the definition of |\the...|
%    is simplified:
%    \begin{macrocode}
\gdef\@ynthm#1#2{\expandafter\@ifdefinable\csname #1\endcsname
   {\@definecounter{#1}%
    \expandafter\xdef\csname the#1\endcsname{\@thmcounter{#1}}%
%    \end{macrocode}
%    The rest of the definition corresponds literally to that of
%    |\@xnthm|:
%    \begin{macrocode}
    \def\@tempa{\global\@namedef{#1}}\expandafter \@tempa
     \expandafter{\csname th@\the \theorem@style \expandafter
     \endcsname \the\theorem@bodyfont \@thm{#1}{#2}}%
    \global \expandafter \let \csname end#1\endcsname \@endtheorem}}
%    \end{macrocode}
% \end{macro}
%
% \begin{macro}{\@othm}
%    The definition of |\@othm| does not contain anything new.
%    \begin{macrocode}
\gdef\@othm#1[#2]#3{%
%    \end{macrocode}
% \changes{v2.2b}{95/10/07}{Build in check for counter defined}
%    We do not define a new counter but instead use one that has
%    already been defined.  Thus the only definition we need is that
%    of this pseudo-counter (i.e.\ |\the|\meta{env.\ name}).
%    First we check if |#2| corresponds to a known counter name.
%    \begin{macrocode}
  \expandafter\ifx\csname c@#2\endcsname\relax
   \@nocounterr{#2}%
  \else
   \expandafter\@ifdefinable\csname #1\endcsname
   {\expandafter \xdef \csname the#1\endcsname
     {\expandafter \noexpand \csname the#2\endcsname}%
%    \end{macrocode}
%    All other parts of the definition can be adopted from
%    |\@xnthm|. We have to remember, though, that in this case
%    the name of the current counter and the theorem label have moved
%    to the second and third arguments.
%    \begin{macrocode}
    \def\@tempa{\global\@namedef{#1}}\expandafter \@tempa
     \expandafter{\csname th@\the \theorem@style \expandafter
     \endcsname \the\theorem@bodyfont \@thm{#2}{#3}}%
    \global \expandafter \let \csname end#1\endcsname \@endtheorem}%
  \fi}
%    \end{macrocode}
% \end{macro}
%
% \subsection{Macros that are employed in a {\envfont theorem}
%    environment}
%
% \begin{macro}{\@thm}
%    The macro |\@thm| has to increase the current counter.  Then,
%    depending on whether the environment has (or does not have) an
%    optional argument, it has to branch into either |\@begintheorem|
%    or |\@opargtheorem|.
%    \begin{macrocode}
\gdef\@thm#1#2{\refstepcounter{#1}%
%    \end{macrocode}
%    Now we start a {\envfont trivlist} environment, and give
%    |\@topsep| and |\@topsepadd| the values of the \textsf{skip}
%    registers |\theorempreskipamount| and |\theorempostskipamount|.
%    The value in |\@topsep| is the vertical space that is inserted by
%    the first (and only) |\item| in our |\trivlist| whilst
%    |\@topsepadd| is inserted by |\@endparenv| at the end of that
%    {\envfont trivlist} environment.  By using these registers, we
%    obtain the desired space around a {\envfont theorem} environment.
%    \begin{macrocode}
   \trivlist
   \@topsep \theorempreskipamount               % used by first \item
   \@topsepadd \theorempostskipamount           % used by \@endparenv
%    \end{macrocode}
%    Now we have to test whether an optional argument has been given.
% \changes{v2.0d}{89/07/18}{Forgotten percent added.}
%    \begin{macrocode}
   \@ifnextchar [%
%    \end{macrocode}
%    If there is an optional argument, we will call |\@ythm|, and
%    move the arguments read back into the input stream.
%    \begin{macrocode}
   {\@ythm{#1}{#2}}%
%    \end{macrocode}
%    If not, we call |\@begintheorem|. Its first argument is the
%    name of the theorem set (hence the second argument of
%    |\@thm|). Its second argument is the macro that produces the
%    current number.
%    \begin{macrocode}
   {\@begintheorem{#2}{\csname the#1\endcsname}\ignorespaces}}
%    \end{macrocode}
% \end{macro}
%
% \begin{macro}{\@xthm}
% \begin{macro}{\@ythm}
%    Both these macros were originally called by |\@thm|.  We do
%    not need |\@xthm| anymore, hence we reset it to
%    |\relax|. The definition of |\@ythm| has not changed at
%    all from its definition in \LaTeX. In order to make the macros
%    easier to understand, we will
%    nevertheless present it (commented out).
%    \begin{macrocode}
\global\let\@xthm\relax
% \def\@ythm#1#2[#3]{\@opargbegintheorem{#2}{\csname
%           the#1\endcsname}{#3}\ignorespaces}
%</package>
%    \end{macrocode}
%    The primitive |\ignorespaces| in |\@ythm| and
%    |\@thm| is needed to remove the spaces between the
%    |\begin{...}| and the actual text.
% \end{macro}
% \end{macro}
%
%
% \subsection{Definition of the theorem substyles}
%
% As already pointed out, the theorem substyles, defined below, are
% only loaded when necessary.  Note that all these substyles, except
% \texttt{plain}, have |\slshape| as the default body font.
%
% \subsubsection{The \texttt{plain} style}
%
% \label{dok:th@plain} As the following macros use |@|, we have
% to locally set the |\catcode| of this symbol to ``letter''.
% This happens within a group, so that we do not have to worry about
% which |\catcode| that symbol had before.
%    \begin{macrocode}
%<*thp>
\begingroup \makeatletter
%    \end{macrocode}
% Since we are now within a group, we must make all definitions
% globally.  First we make sure that \texttt{theorem.sty} is loaded.
% This will allow us to use this file as a document style option
% without having to call \texttt{theorem} itself as an option.  At the
% same time, we assure that at least version 2 is loaded, since
% |\theorem@style| was not defined in earlier versions.
%    \begin{macrocode}
\@ifundefined{theorem@style}{\input{theorem.sty}}{}
\ProvidesFile{thp.sty}
             [\FMithmInfo]
%    \end{macrocode}
%
%
% \begin{macro}{\th@plain}
% \begin{macro}{\@begintheorem}
% \begin{macro}{\@opargbegintheorem}
%    |\theoremstyle{plain}| corresponds to the original
%    definition, except that the distances to the surrounding text are
%    determined by the parameters |\theorempreskipamount| and
%    |\theorempostskipamount|.
%    First we set the default body font.
%    \begin{macrocode}
\gdef\th@plain{\normalfont\itshape
%    \end{macrocode}
%    Then we define |\@begintheorem| and
%    |\@opargbegintheorem|. These two macros define how the
%    header of a theorem is typeset.  |\@opargbegintheorem| will
%    be called if a \textsf{theorem} environment with an optional
%    argument is encountered; otherwise, the header is constructed by
%    calling |\@begintheorem|.  If one of these macros is
%    executed, we are within a \textsf{trivlist} environment started by
%    |\@thm|.  So the theorem header is produced with an
%    |\item| command.
%
%    Instead of specifying the header font directly, all standard
%    theorem styles use the |\theorem@headerfont| macro to allow
%    customization.  The extra space (|\labelsep|) is necessary
%    because of problems in the \textsf{trivlist} environment.
%    \begin{macrocode}
  \def\@begintheorem##1##2{%
        \item[\hskip\labelsep \theorem@headerfont ##1\ ##2]}%
%    \end{macrocode}
%    The definition of |\@opargbegintheorem| is completely
%    analogous. The only difference is the fact that there exists a
%    third argument (which is the optional parameter of the
%    environment and contains additional information about the
%    theorem).  Customarily we enclose it in parentheses.
%    \begin{macrocode}
\def\@opargbegintheorem##1##2##3{%
   \item[\hskip\labelsep \theorem@headerfont ##1\ ##2\ (##3)]}}
%    \end{macrocode}
% \end{macro}
% \end{macro}
% \end{macro}
%
% \noindent
% We conclude with an |\endgroup| to restore the
% |\catcode| of |@|.
%    \begin{macrocode}
\endgroup
%</thp>
%    \end{macrocode}
%
%
% \subsubsection{The \texttt{break} style}
%
% This style option is stored in the file |thb.sty|.
% For the next two lines see the documentation for |\th@plain| on
% page~\pageref{dok:th@plain}.
%    \begin{macrocode}
%<*thb>
\begingroup \makeatletter
\@ifundefined{theorem@style}{\input{theorem.sty}}{}
\ProvidesFile{thb.sty}
             [\FMithmInfo]
%    \end{macrocode}
%
% \begin{macro}{\th@break}
%    |\theoremstyle{break}| produces a line break after the name
%    of the theorem.  The font is |\slshape|.  Hence, we define
%    |\th@break| as follows:
%    \begin{macrocode}
\gdef\th@break{\normalfont\slshape
  \def\@begintheorem##1##2{\item[%
%    \end{macrocode}
%    We run into the following problem: it is not possible to create
%    the header with |\item[|\meta{title}|]| and then start
%    a new line by, for example, |\mbox{}\\|. Such a definition
%    will fail whenever a list environment follows immediately.  With
%    the above construction, the |\mbox{}| causes the switch
%    |@inlabel| (cf.\ definition of |\list| and
%    |\trivlist| in \cite{latex:source}) to be set to \texttt{false}
%    and so the following list will insert additional vertical
%    space (|\topskip|). This is quite annoying. Therefore, we
%    create the line break within the |\item|. In order to ensure
%    that the text will begin at the proper position in the following
%    line, we simply pretend that the label does not take any
%    room.\footnote{This will lead to problems whenever very high
%    symbols occurring in the line tower into the
%    heading. So, something else has to be done here sometime.}
%    \begin{macrocode}
       \rlap{\vbox{\hbox{\hskip \labelsep\theorem@headerfont ##1\ ##2}%
                   \hbox{\strut}}}]}%
%    \end{macrocode}
%    Again, the definition of |\@opargbegintheorem| is
%    completely analogous.
%    \begin{macrocode}
\def\@opargbegintheorem##1##2##3{%
  \item[\rlap{\vbox{\hbox{\hskip \labelsep \theorem@headerfont
                     ##1\ ##2\ (##3)}%
                    \hbox{\strut}}}]}}
\endgroup
%</thb>
%    \end{macrocode}
% \end{macro}
%
%
% \subsubsection{The \texttt{changebreak} style}
%
%    \begin{macrocode}
% This style option is stored in the file |thcb.sty|.
%    \begin{macrocode}
%<*thcb>
\begingroup \makeatletter
\@ifundefined{theorem@style}{\input{theorem.sty}}{}
\ProvidesFile{thcb.sty}
             [\FMithmInfo]
%    \end{macrocode}
%
% \begin{macro}{\th@changebreak}
%    The \texttt{change-break} theorem style is like \texttt{break}
%    but with interchange of theorem name and theorem number.  Thus we
%    define |\th@changebreak| as follows:
%    \begin{macrocode}
\gdef\th@changebreak{\normalfont\slshape
  \def\@begintheorem##1##2{\item
       [\rlap{\vbox{\hbox{\hskip\labelsep \theorem@headerfont ##2\ ##1}%
                    \hbox{\strut}}}]}%
\def\@opargbegintheorem##1##2##3{%
  \item[\rlap{\vbox{\hbox{\hskip\labelsep \theorem@headerfont
                     ##2\ ##1\ (##3)}%
                    \hbox{\strut}}}]}}
\endgroup
%</thcb>
%    \end{macrocode}
% \end{macro}
%
%
% \subsubsection{The \texttt{change} style}
%
% This style option is stored in the file |thc.sty|.
%    \begin{macrocode}
%<*thc>
\begingroup \makeatletter
\@ifundefined{theorem@style}{\input{theorem.sty}}{}
\ProvidesFile{thc.sty}
             [\FMithmInfo]
%    \end{macrocode}
%
% \begin{macro}{\th@change}
%    The \texttt{change} theorem style corresponds to the
%    \texttt{change break} style without a linebreak after the header.
%    To say it in another way, it's the same as the \texttt{plain}
%    style but with number and name interchanged and |\slshape| as the
%    default font.
%    \begin{macrocode}
\gdef\th@change{\normalfont\slshape
  \def\@begintheorem##1##2{\item
       [\hskip\labelsep \theorem@headerfont ##2\ ##1]}%
\def\@opargbegintheorem##1##2##3{%
  \item[\hskip\labelsep \theorem@headerfont ##2\ ##1\ (##3)]}}
\endgroup
%</thc>
%    \end{macrocode}
% \end{macro}
%
%
% \subsubsection{The \texttt{marginbreak} style}
%
% This style option is the one used most often at Mainz.  It is saved
% in the file |thmb.sty|.
%    \begin{macrocode}
%<*thmb>
\begingroup \makeatletter
\@ifundefined{theorem@style}{\input{theorem.sty}}{}
\ProvidesFile{thmb.sty}
             [\FMithmInfo]
%    \end{macrocode}
%
% \begin{macro}{\th@marginbreak}
%    The \texttt{margin break} style is nearly the same as the
%    \texttt{change break} style. The only difference is the placement
%    of the theorem number.  We use |\llap| to place it in the left
%    margin.
%
%    In this style |\labelsep| denotes the separation between the
%    number and the text.
%    \begin{macrocode}
\gdef\th@marginbreak{\normalfont\slshape
  \def\@begintheorem##1##2{\item
       [\rlap{\vbox{\theorem@headerfont
                    \hbox{\llap{##2}\hskip\labelsep ##1}%
                    \hbox{\strut}}}]}%
\def\@opargbegintheorem##1##2##3{%
  \item[\rlap{\vbox{\theorem@headerfont
                    \hbox{\llap{##2}\hskip\labelsep ##1\ (##3)}%
                    \hbox{\strut}}}]}}
\endgroup
%</thmb>
%    \end{macrocode}
% \changes{v2.1c}{90/02/17}{`endgroup outside macrocode env.}
% \end{macro}
%
%
%
% \subsubsection{The \texttt{margin} style}
%
% This style option is stored in the file |thm.sty|.
%    \begin{macrocode}
%<*thm>
\begingroup \makeatletter
\@ifundefined{theorem@style}{\input{theorem.sty}}{}
\ProvidesFile{thm.sty}
             [\FMithmInfo]
%    \end{macrocode}
%
% \begin{macro}{\th@margin}
%    Again this is only a variant of the theorem styles described
%    above without any new ideas.
%    \begin{macrocode}
\gdef\th@margin{\normalfont\slshape
  \def\@begintheorem##1##2{\item
       [\theorem@headerfont \llap{##2}\hskip\labelsep ##1]}%
\def\@opargbegintheorem##1##2##3{%
  \item[\theorem@headerfont \llap{##2}\hskip\labelsep ##1\ (##3)]}}
\endgroup
%</thm>
%    \end{macrocode}
% \end{macro}
%
%
%
%
% \subsection{Final Definitions}
%
% \begin{macro}{\theorempreskipamount}
% \begin{macro}{\theorempostskipamount}
%    The \textsf{skip} parameters that regulate the vertical empty space
%    before and after the {\envfont theorem} environment have
%    to be allocated as well.
% \changes{v2.0f}{89/07/20}{Unnecessary `global' removed.}
%    \begin{macrocode}
%<*package>
\newskip\theorempreskipamount
\newskip\theorempostskipamount
%    \end{macrocode}
%    Since we have used the same values for all theorem sets, we now can
%    assign them.
%    \begin{macrocode}
\global\setlength\theorempreskipamount{12pt plus 5pt minus 3pt}
\global\setlength\theorempostskipamount{8pt plus 3pt minus 1.5pt}
%    \end{macrocode}
% \end{macro}
% \end{macro}
%
%
% \begin{macro}{\@endtheorem}
%    The same holds for the macro |\@endtheorem|, which ends a
%    {\envfont theorem} environment. Since it is the same for all
%    theorem sets, it is removed from the macros |\th@|\meta{style}.
%    It simply ends the {\envfont trivlist} environment, which was
%    begun in |\@thm|.
%    \begin{macrocode}
\global\let\@endtheorem=\endtrivlist
%    \end{macrocode}
% \end{macro}
%
%
%
% \begin{macro}{\@preamblecmds}
%    All macros defined above are to be used only in the preamble.
%    Therefore, we insert them in |\@preamblecmds| which will disable
%    them at begin document. This is done by the internal \LaTeXe{}
%    command |\@onlypreamble|.
% \changes{v2.0g}{89/09/19}{`theoremheaderfont and `theorembodyfont
%    added.}
% \changes{v2.2a}{94/02/03}{Use \LaTeXe{} interface}
%    \begin{macrocode}
\@onlypreamble\@xnthm
\@onlypreamble\@ynthm
\@onlypreamble\@othm
\@onlypreamble\newtheorem
\@onlypreamble\theoremstyle
\@onlypreamble\theorembodyfont
\@onlypreamble\theoremheaderfont
%    \end{macrocode}
% \end{macro}
%
% Finally we declare the \texttt{plain} theorem style to be the default.
% \changes{v2.1b}{90/06/04}{Fixed missing `item by declaring default
%    style.}
%    \begin{macrocode}
\theoremstyle{plain}
%</package>
%    \end{macrocode}
%
% \Finale

\endinput
}{}
\ProvidesFile{thb.sty}
             [\FMithmInfo]
%    \end{macrocode}
%
% \begin{macro}{\th@break}
%    |\theoremstyle{break}| produces a line break after the name
%    of the theorem.  The font is |\slshape|.  Hence, we define
%    |\th@break| as follows:
%    \begin{macrocode}
\gdef\th@break{\normalfont\slshape
  \def\@begintheorem##1##2{\item[%
%    \end{macrocode}
%    We run into the following problem: it is not possible to create
%    the header with |\item[|\meta{title}|]| and then start
%    a new line by, for example, |\mbox{}\\|. Such a definition
%    will fail whenever a list environment follows immediately.  With
%    the above construction, the |\mbox{}| causes the switch
%    |@inlabel| (cf.\ definition of |\list| and
%    |\trivlist| in \cite{latex:source}) to be set to \texttt{false}
%    and so the following list will insert additional vertical
%    space (|\topskip|). This is quite annoying. Therefore, we
%    create the line break within the |\item|. In order to ensure
%    that the text will begin at the proper position in the following
%    line, we simply pretend that the label does not take any
%    room.\footnote{This will lead to problems whenever very high
%    symbols occurring in the line tower into the
%    heading. So, something else has to be done here sometime.}
%    \begin{macrocode}
       \rlap{\vbox{\hbox{\hskip \labelsep\theorem@headerfont ##1\ ##2}%
                   \hbox{\strut}}}]}%
%    \end{macrocode}
%    Again, the definition of |\@opargbegintheorem| is
%    completely analogous.
%    \begin{macrocode}
\def\@opargbegintheorem##1##2##3{%
  \item[\rlap{\vbox{\hbox{\hskip \labelsep \theorem@headerfont
                     ##1\ ##2\ (##3)}%
                    \hbox{\strut}}}]}}
\endgroup
%</thb>
%    \end{macrocode}
% \end{macro}
%
%
% \subsubsection{The \texttt{changebreak} style}
%
%    \begin{macrocode}
% This style option is stored in the file |thcb.sty|.
%    \begin{macrocode}
%<*thcb>
\begingroup \makeatletter
\@ifundefined{theorem@style}{% \iffalse meta-comment
%
% Copyright (C) 1993-2024
%
% The LaTeX Project and any individual authors listed elsewhere
% in this file.
%
% This file is part of the Standard LaTeX `Tools Bundle'.
% -------------------------------------------------------
%
% It may be distributed and/or modified under the
% conditions of the LaTeX Project Public License, either version 1.3c
% of this license or (at your option) any later version.
% The latest version of this license is in
%    https://www.latex-project.org/lppl.txt
% and version 1.3c or later is part of all distributions of LaTeX
% version 2005/12/01 or later.
%
% The list of all files belonging to the LaTeX `Tools Bundle' is
% given in the file `manifest.txt'.
%
% \fi
%
%%
%
%
\def\FMithmInfo{2023/07/05 v2.2c Theorem extension package (FMi)}
%
% \ProvidesFile{theorem.dtx}[\FMithmInfo]
%\iffalse   % this is a METACOMMENT !
%%
%% Package `theorem' to use with LaTeX2e
%% Copyright (C) 1989-1995 Frank Mittelbach, all rights reserved.
%<+package>\@ifundefined{theorem@style}{}{\endinput}
%<+package>\NeedsTeXFormat{LaTeX2e}[1995/06/01]
%<+package>\ProvidesPackage{theorem}
%<+package>         [\FMithmInfo]
%
% \fi
%
% \changes{v1.0}{87/05/07}{First draft.}
% \changes{v1.0a}{87/05/07}{`thistheoremstyle' added.}
% \changes{v1.1a}{87/07/29}{New concept: redefine the internal
%                          \protect\LaTeX{} commands `@xnthm' etc.}
% \changes{v1.1b}{87/07/31}{`theoremfont' used to specify font.}
% \changes{v1.2a}{88/01/08}{`@endtheorem' now globally defined.}
% \changes{v1.2b}{88/01/11}{`newtheorem' only allowed in the preamble.}
% \changes{v1.3a}{88/02/14}{`theoremfont' now a toks register.}
%
% \changes{v2.0a}{88/11/16}{New implementation and new documentation.}
% \changes{v2.0a}{88/11/16}{No layout changes but internal modifications
%                           old substyles won't work.}
% \changes{v2.0b}{89/01/30}{`global' used for all definitions.}
% \changes{v2.0c}{89/06/15}{Documentation translated to English.
%                            Thanks to LLL and Nina}
% \changes{v2.1a}{90/02/15}{Support for new font selection.}
% \changes{v2.2a}{94/02/03}{Cleanup for LaTeX2e}
%
% \DoNotIndex{\@endparenv,\@ifdefinable,\@ifnextchar,\@ifundefined}
% \DoNotIndex{\@namedef,\@nameuse,\@spaces,\@tempa,\@warning}
% \DoNotIndex{\ }
% \DoNotIndex{\begingroup,\bfseries}
% \DoNotIndex{\csname}
% \DoNotIndex{\def,\do,\docdate}
% \DoNotIndex{\endcsname,\endgroup,\endinput,\endtrivlist,\expandafter}
% \DoNotIndex{\filedate,\fileversion}
% \DoNotIndex{\gdef,\global}
% \DoNotIndex{\hbox,\hskip}
% \DoNotIndex{\ignorespaces,\input,\itshape,\item}
% \DoNotIndex{\labelsep,\let,\llap}
% \DoNotIndex{\makeatletter}
% \DoNotIndex{\noexpand}
% \DoNotIndex{\refstepcounter,\relax,\rlap}
% \DoNotIndex{\setlength,\slshape,\space,\string,\strut}
% \DoNotIndex{\the,\trivlist,\typeout}
% \DoNotIndex{\vbox}
% \DoNotIndex{\xdef}
%
% \def\deflabel#1{\ttfamily #1\hfill}
% \def\deflist#1{\begin{list}{}{\settowidth\labelwidth{\ttfamily #1}%
%                               \setlength\leftmargin\labelwidth
%                               \addtolength\leftmargin\labelsep
%                               \let\makelabel\deflabel}}
% \def\enddeflist{\end{list}}
% \def\envfont{\normalfont\ttfamily}
%
% \MakeShortVerb{\|}
%
% \newcommand{\TUB}{{\sl TUGboat\/}}
%
% \GetFileInfo{theorem.dtx}
% \title{An Extension of the
%        \LaTeX{} theorem environment\thanks{This file
%        has version number \fileversion, last
%        revised \filedate.}}
% \author{Frank Mittelbach}
%
% \MaintainedByLaTeXTeam{tools}
% \maketitle
%
% \begin{abstract}
%    The macros described in this paper yield an extension of the
%    \LaTeX{} theorem mechanism. It is designed to satisfy the
%    different requirements of various journals. Thus, the
%    layout of the ``theorems'' can be manipulated by determining a
%    ``style''. This article describes not only the use, but
%    also the definition, of the necessary macros.
% \end{abstract}
%
% \section*{Preface to version 2.2}
%
% For \LaTeXe{} this package did not need any fundamental changes. I
% only modified the messages generated so that theorem layout styles
% will show up with the |\listfiles| command and cleaned the section
% on the New Font Selection Scheme since this is now included in
% \LaTeX.
%
% \section*{Preface to version 2.1}
%
% This version is identical to 2.0g described in \TUB\ 10\#3 except
% for some internal defaults which are now set depending on the used
% font selection scheme.
%
% This was done to avoid unpleasant surprises if the new font selection
% scheme is in force. For further details see
% section~\ref{sec:newfonts} and \cite{bk:GMS94}.
%
% \section{Introduction}
%
% For our purposes here, ``theorems'' are labelled enunciations,
% often set off from the main text by extra space and a font change.
% Theorems, corollaries, conjectures, definitions, and remarks
% are all instances of ``theorems''.  The ``header'' of these structures
% is composed of a label (such as \textsc{Theorem} or \textsc{Remark})
% and a number which serializes an item in the sequence of items
% with the same label.
%
% Shortly after the introduction of \LaTeX{} at the Fachbereich
% Mathematik in Mainz, the desire to manipulate the layout of
% ``theorems'' arose.  In Mainz, the following two conventions came
% into general use:
% \begin{enumerate}
%   \item The number of the theorem is shown in the margin.
%   \item There is a line break at
%         the end of the theorem header.
% \end{enumerate}
%
% Additionally, some journals require different formats which depend on
% the ``sort of theorem'': e.g.\ often remarks and definitions are set
% in |\upshape|, while |\itshape| is employed for main theorems.
%
% Confronted with these requirements, a theorem environment was
% developed in Mainz which allows separate determination of the
% layout of the ``theorems sets'', comparable to |\pagestyle|.
%
%
% \section{The user interface}
% \subsection{Defining new theorem sets}
%
% \DescribeMacro\newtheorem
% As in the original \LaTeX{} version, the command |\newtheorem|
% defines a new ``theorem set'' or ``theorem-like structure''.
% Two required arguments name the new environment and give the text
% to be typeset with each instance of the new ``set'', while
% an optional argument determines how
% the ``set'' is enumerated:
% \begin{description}
%    \item[\ttfamily \bslash newtheorem\{foo\}\{bar\}]
%       The theorem set {\envfont foo} (whose name is \texttt{bar})
%       uses its own counter.
%    \item[\ttfamily \bslash newtheorem\{foo2\}{[foo]}\{bar2\}]
%       The theorem set {\envfont foo2} (printed name \texttt{bar2})
%       uses the same counter as the theorem set \texttt{foo}.
%    \item[\ttfamily \bslash newtheorem\{foo3\}\{bar3\}{[section]}]
%       The theorem set {\envfont foo3} (printed name \texttt{bar3}) is
%       enumerated within the counter \texttt{section}, i.e.\ with every
%       new |\section| the enumeration begins again with 1, and
%       the enumeration is composed from the section-number and the
%       theorem counter itself.
% \end{description}
%
% \DescribeMacro\theoremstyle
% Additionally, the command |\theoremstyle| can define the layout
% of various, or all, theorem sets.  It should be noted that any
% theorem set defined by |\newtheorem| is typeset in the
% |\theoremstyle| that is current at the time of the definition.
% Thus, the following
% \begin{verbatim}
%    \theoremstyle{break}        \newtheorem{Cor}{Corollary}
%    \theoremstyle{plain}        \newtheorem{Exa}{Example}[section]
%\end{verbatim}                             ^^A No blank at line start
% leads to the result that the set {\envfont Cor} is formatted in the
% style \texttt{break}, while the set {\envfont Exa} and all the
% following ones are formatted in the style \texttt{plain}, unless
% another |\theoremstyle| follows.  Since the definitions installed by
% |\newtheorem| are global, one also can limit |\theoremstyle| locally
% by grouping braces.
%
% \DescribeMacro\theorembodyfont
% The choice of the font for the theorem body is completely
% independent of the chosen |\theoremstyle|; this has proven to
% be very advantageous.  For example,
% \begin{verbatim}
%   {\theorembodyfont{\upshape}          \newtheorem{Rem}{Remark}}
%\end{verbatim}
% defines a theorem set {\envfont Rem}, which will be set in
% |\upshape| in the current layout (which in our example is
% \texttt{plain}).  As with |\theoremstyle|, the |\theorembodyfont|
% chosen is that current at the time of |\newtheorem|.  If
% |\theorembodyfont| is not specified or one defines
% |\theorembodyfont{}|, then the font used will be that defined by the
% |\theoremstyle|.
%
% \DescribeMacro\theoremheaderfont It is also possible to customize
% the font used for the theorem headers.  This is, however, a global
% declaration, and therefore there should be at most one
% |\theoremheaderfont| declaration in the preamble.\footnote{If it is
% actually necessary to have different header fonts, one has to define
% new theorem styles (substituting the desired font) or specify the
% information directly in the \texttt{\bslash newtheorem} declaration
% (the unclean variant).}
%
% \DescribeMacro\theorempreskipamount
% \DescribeMacro\theorempostskipamount Two additional parameters
% affect the vertical space around the theorem environments:
% |\theorempreskipamount| and |\theorempostskipamount| define,
% respectively, the spacing before and after such an environment.
% These parameters apply for all theorem sets and can be manipulated
% with the ordinary length macros.  They are rubber lengths
% (`\textsf{skips}'), and therefore can contain \texttt{plus} and
% \texttt{minus} parts.
%
% Since the definition of theorem sets should---most sensibly---be
% placed in the preamble, we only allow installation there.
% It is therefore possible to release the memory used here after
% |\begin{document}|, in order to make room for other
% applications.
%
%
% \subsection{Existing theorem styles}
%
% The following theorem styles exist to date:
% \begin{deflist}{marginbreak}
%    \item[plain]
%       This theorem style emulates the original \LaTeX{} definition,
%       except that additionally the parameters
%       |\theorem...skipamount| are used.
%    \item[break]
%       In this style, the theorem header is followed by a line
%       break.
%    \item[marginbreak]
%       The theorem number is set in the margin, and there is a line
%       break as in \texttt{break}.
%    \item[changebreak]
%       Like \texttt{break}, but with header number and text
%       interchanged.
%    \item[change]
%       Header number and text are interchanged, without a line break.
%    \item[margin]
%       The number is set in the left margin, without a line break.
% \end{deflist}
% All styles (except \texttt{plain}) select |\slshape| as the default
% |\theorembodyfont|.
%
%
% \subsection{Examples}
%
% Given the above theorem sets {\envfont Cor}, {\envfont Exa}
% and {\envfont Rem},
% suppose that the preamble also contains the declarations:
% \begin{verbatim}
%    \theoremstyle{marginbreak}   \newtheorem{Lem}[Cor]{Lemma}
%    \theoremstyle{change}
%    \theorembodyfont{\itshape}        \newtheorem{Def}[Cor]{Definition}
%
%    \theoremheaderfont{\scshape}
%\end{verbatim}
% Then the following are some typical examples of the typeset output
% resulting from their use.
%
% \begin{Cor}
%    This is a sentence typeset in the theorem environment {\envfont
%    Cor}.
% \end{Cor}
%
% \begin{Exa}
%    This is a sentence typeset in the theorem environment {\envfont
%    Exa}.
% \end{Exa}
%
% \begin{Rem}
%    This is a sentence typeset in the theorem environment {\envfont
%    Rem}.
% \end{Rem}
%
% \begin{Lem}[Ben User]
%    This is a sentence typeset in the theorem environment {\envfont
%    Lem}.
% \end{Lem}
%
% \begin{Def}[Very impressive Definition]
%    This is a sentence typeset in the theorem environment {\envfont
%    Def}.
% \end{Def}
%
% The last two examples show the effect of the optional argument to
% a theorem environment (it is the text typeset in parentheses).
%
%
% \section{Special Considerations}
% \label{sec:newfonts}
%
% Theoremheader and body are implemented as a unit. This means that
% the |\theoremheaderfont|  will inherit characteristics of the
% |\theorembodyfont| in \LaTeXe. Thus, if for example
% |\theorembodyfont| is |\itshape| and |\theoremheaderfont|
% is |\bfseries| the font selected for the header will have the
% characteristics `bold extended italic'. If this is not desired one
% should set the |\theoremheaderfont| to something like
% \begin{verbatim}
%      \theoremheaderfont{\normalfont\bfseries}
%\end{verbatim}
% i.e.\ supplying all necessary font information explicitly.
%
%
%
% \section{Acknowledgements}
%
% The publication of this set of macros was only possible with the
% help of Christina Busse (translating the manuscript into English),
% Joachim Pense (playing the r\^ole of typist), Chris Rowley
% (looking everything over) and many others providing useful
% suggestions.
%
% \MaybeStop{%
%    \begin{thebibliography}{9}
%       \bibitem{bk:GMS94} \textsc{M.~Goossens}, \textsc{F.~Mittelbach}
%         and \textsc{A.~Samarin}.
%         \newblock The \LaTeX{} Companion.
%         \newblock
%         Addison-Wesley, Reading, Massachusetts, 1994.
%       \bibitem{latex:source} \textsc{Lamport, Leslie.} \newblock
%                               \texttt{latex.tex}, version 2.09, date
%                               Feb. 1990.
%    \end{thebibliography}
%    \PrintIndex
%    \PrintChanges}
%
%
% \section{The documentation driver file}
%
% The next bit of code contains the documentation driver file for
% \TeX{}, i.e., the file that will produce the documentation you are
% currently reading. It will be extracted from this file by the
% \texttt{docstrip} program. Since it is the first code in the file
% one can alternatively process this file directly with \LaTeXe{} to
% obtain the documentation.
%
%    \begin{macrocode}
%<*driver>
\documentclass{ltxdoc}

\usepackage{theorem}

%<+driver>% The next few lines define theorem sets which are used
%<+driver>% in the example section of the documentation.

\theoremstyle{break}       \newtheorem{Cor}{Corollary}
\theoremstyle{plain}       \newtheorem{Exa}{Example}[section]
{\theorembodyfont{\upshape}\newtheorem{Rem}{Remark}}
\theoremstyle{marginbreak} \newtheorem{Lem}[Cor]{Lemma}
\theoremstyle{change}
\theorembodyfont{\itshape} \newtheorem{Def}[Cor]{Definition}

\theoremheaderfont{\scshape}

\RecordChanges

\begin{document}
   \DocInput{theorem.dtx}
\end{document}
%</driver>
%    \end{macrocode}
%
% \section{Definition of the Macros}
%
% If the file has been loaded before, we abort immediately. If not the
% package announces itself (this is actually done at the very top of
% the file---the way it is done isn't good style so don't copy it).
% \changes{v2.0e}{89/07/19}{Spaces removed from `typeout'.}
% \changes{v2.2b}{95/11/19}{Announce moved to top of file}
%    \begin{macrocode}
%<*package>
%\@ifundefined{theorem@style}{}{\endinput}
%\def\FMithmInfo{1995/11/19 v2.2b Theorem extension package (FMi)}
%\ProvidesPackage{theorem}[\FMithmInfo]
%    \end{macrocode}
%
%
% \subsection{Definition of theorem styles and fonts}
%
% All the definitions in this file are done globally to allow
% inputting this file inside a group.
%
% \begin{macro}{\theoremstyle}
%    Before a theorem style can be installed, the chosen style must be
%    known. For that reason, we must test to see that
%    |\th@|\meta{style} is known or, more precisely, that it is
%    different from |\relax|. If the style is not known then
%    |\th@plain| is used.
%    \begin{macrocode}
\gdef\theoremstyle#1{%
   \@ifundefined{th@#1}{\@warning
          {Unknown theoremstyle `#1'. Using `plain'}%
          \theorem@style{plain}}%
%    \end{macrocode}
%    We save the theorem style to be used in the token register
%    |\theorem@style|.
%    \begin{macrocode}
      {\theorem@style{#1}}%
%    \end{macrocode}
%    Now we ``evaluate'' the theorem style: this means, we call the
%    macro |\th@|\meta{style} which will activate the relevant
%    definitions which are contained in a separate file.  This is done
%    in a group to suppress changes to the current font.  This could
%    otherwise pose problems together with the new font selection
%    scheme\footnote{When I printed the original article using the new
%    font selection scheme I ended with a document with slanted
%    typefaces (text headings and all) simply because one of the
%    theorem styles used \texttt{\bslash sl} at toplevel.} if the
%    |\th@|\meta{style} is evaluated a second time.
% \changes{v2.1a}{90/02/15}{Group added around `th@..}
% \changes{v2.1a}{90/02/15}{Reseting of `..@begintheorem removed.}
%    \begin{macrocode}
      \begingroup
        \csname th@\the\theorem@style \endcsname
      \endgroup}
%    \end{macrocode}
% \end{macro}
%
%
% \begin{macro}{\@begintheorem}
% \changes{v2.1a}{90/02/15}{Relaxed at toplevel.}
% \begin{macro}{\@opargbegintheorem}
% \changes{v2.1a}{90/02/15}{Relaxed at toplevel.}
%    We reset |\@begintheorem| and |\@opargbegintheorem| to
%    |\relax| since these commands are no longer necessary at
%    toplevel. This will save a few tokens.
%    \begin{macrocode}
\global\let\@begintheorem\relax
\global\let\@opargbegintheorem\relax
%    \end{macrocode}
% \end{macro}
% \end{macro}
%
% \begin{macro}{\theorem@style}
%    Obviously the \textsf{token} register used above has to be
%    allocated. To assure the utmost compatibility with the
%    original \LaTeX{} definition, we set the default theorem style to
%    \texttt{plain}, which implements the usual \LaTeX{} convention.
% \changes{v2.0f}{89/07/20}{Unnecessary `global' removed.}
%    \begin{macrocode}
\newtoks\theorem@style
\global\theorem@style{plain}
%    \end{macrocode}
% \end{macro}
%
% \begin{macro}{\theorembodyfont}
% \begin{macro}{\theorem@bodyfont}
%    If the |\theorembodyfont| is set by the user then it should not
%    interact with the default font set in the theorem style. When the
%    new font selection is in force this may happen if, for example,
%    the default is |\itshape| and the new |\theorembodyfont| is
%    |\sffamily|. So we add a |\reset@font| command in front of the
%    user definition.
% \changes{v2.0f}{89/07/20}{Unnecessary `global' removed.}
% \changes{v2.0g}{89/09/19}{`theoremfont to `theorembodyfont.}
% \changes{v2.1c}{92/06/22}{`reset@font added and logic changed.}
% \changes{v2.1d}{92/10/28}{Reset to empty if arg is empty}
%    \begin{macrocode}
\gdef\theorembodyfont#1{%
%    \end{macrocode}
%    We check if the argument supplied is empty and if so put nothing
%    into the |\theorem@bodyfont| token register to allow for
%    |\theorembodyfont{}| as a mean of using the default of the
%    current |\theoremstyle|.
%    \begin{macrocode}
   \def\@tempa{#1}%
   \ifx\@tempa\@empty
    \theorem@bodyfont{}%
   \else
    \theorem@bodyfont{\reset@font#1}%
   \fi
   }
\newtoks\theorem@bodyfont
\global\theorem@bodyfont{}
%    \end{macrocode}
% \end{macro}
% \end{macro}
%
%
% \begin{macro}{\theoremheaderfont}
% \changes{v2.0g}{89/09/19}{Macro added.}
%    The font for the theorem headers is handled differently because
%    this definition applies to all theorem styles.
%    \begin{macrocode}
\gdef\theoremheaderfont#1{\gdef\theorem@headerfont{#1}%
%    \end{macrocode}
%    After using the macro once it is redefined to produce an error
%    message.
%    \begin{macrocode}
       \gdef\theoremheaderfont##1{%
        \typeout{\string\theoremheaderfont\space should be used
                 only once.}}}
%    \end{macrocode}
% \end{macro}
%
% \begin{macro}{\theorem@headerfont}
% \changes{v2.0g}{89/09/19}{Macro added.}
%    To set the |\theorem@headerfont| default we first test if
%    the new fontselection scheme is in force.
% \changes{v2.1a}{90/02/15}{Support for new font selection.}
%    \begin{macrocode}
\ifx\upshape\undefined
%    \end{macrocode}
%    If not we define it to expand into |\bfseries|. We don't use
%    |\let| just in case a following style option redefines this
%    macro.
%    \begin{macrocode}
\gdef\theorem@headerfont{\bfseries}
%    \end{macrocode}
%    Otherwise we reset the current shape before calling |\bfseries|.
%    \begin{macrocode}
\else \gdef\theorem@headerfont{\normalfont\bfseries}\fi
%    \end{macrocode}
% \end{macro}
%
% \begin{macro}{\th@plain}
% \begin{macro}{\th@break}
% \begin{macro}{\th@marginbreak}
% \begin{macro}{\th@changebreak}
% \begin{macro}{\th@change}
% \begin{macro}{\th@margin}
%    The different styles are defined in macros such as
%    |\th@plain|.  Since memory space is precious in
%    ``non-Big-versions'', we have to avoid offering too many unused
%    definitions.  Therefore we define these styles in separate files
%    that can be loaded on demand.  Thus the commands themselves only
%    load these files. We use |\@input@|, a \LaTeXe{} internal command
%    that ensures that the file will be listed with |\listfiles|.
% \changes{v2.2a}{94/02/02}{Use `@input@ to load theorem layout files}
%    \begin{macrocode}
\gdef\th@plain{\@input@{thp.sty}}
\gdef\th@break{\@input@{thb.sty}}
\gdef\th@marginbreak{\@input@{thmb.sty}}
\gdef\th@changebreak{\@input@{thcb.sty}}
\gdef\th@change{\@input@{thc.sty}}
\gdef\th@margin{\@input@{thm.sty}}
%    \end{macrocode}
%    This list will be expanded when new styles become
%    available.  For testing,  just append new theorem substyles
%    as document options.
% \end{macro}
% \end{macro}
% \end{macro}
% \end{macro}
% \end{macro}
% \end{macro}
%
%
%
% \subsection{Definition of a new theorem set}
%
% As already pointed out, a new theorem environment can be defined in
% three different ways:
% \begin{verbatim}
%    \newtheorem{Lem}{Lemma}
%    \newtheorem{Lem}{Lemma}[section]
%    \newtheorem{Lem}[Theorem]{Lemma}
%\end{verbatim}
% The function of the macro |\newtheorem| is to recognize these cases
% and then to branch into one of the three macros |\@ynthm|, |\@xnthm|
% or |\@othm|.  This mechanism is adopted unchanged from
% \cite{latex:source}; the essential point here is that, for example,
% in the second case, the arguments \texttt{Lem}, \texttt{Lemma} and
% \texttt{section} are passed over to the macro |\@xnthm|.
%
% We inspect this case first because the others present fewer problems,
% and thus are easily derived from this one.
%
% \begin{macro}{\@xnthm}
%    For our example arguments, the macro |\@xnthm| must fulfill
%    the following:
%    \begin{itemize}
%       \item
%          Define a new \LaTeX-counter `Lem'
%       \item
%          reset this counter within a |\section|
%       \item
%          define the macro |\theLem|
%       \item
%          define the environment macros |\Lem| and
%          |\endLem| using the current
%          |\theoremstyle| and |\theorem@bodyfont|.
%    \end{itemize}
%    Obviously, all this should happen only if the first argument
%    of |\@xnthm| (i.e.\ \texttt{Lem} in our example) is chosen so
%    as not to conflict with any previously defined commands or
%    environments.  This test is performed by the \LaTeX{} macro
%    |\@ifdefinable|.
%    \begin{macrocode}
\gdef\@xnthm#1#2[#3]{\expandafter\@ifdefinable\csname #1\endcsname
%    \end{macrocode}
%    Therefore, the first argument of |\@ifdefinable| is the
%    expansion (in the example, |\Lem|) of
%    |\csname||#1\endcsname|.  The second argument is
%    executed only if the test has been completed successfully.
%    \begin{macrocode}
   {%
%    \end{macrocode}
%    Now we define the new counter. The names of the \LaTeX{}
%    macros employed should speak for themselves:
% \changes{v2.2b}{1995/10/07}{Use \cs{@newctr}}
% \changes{v2.2c}{1995/11/23}{Use \cs{@newctr} properly.}
%    \begin{macrocode}
    \@definecounter{#1}\@newctr{#1}[#3]%
%    \end{macrocode}
%    Using |\@newctr| will give a proper error message if the counter
%    in |#3| is not defined.
%    In defining `|\theLem|' we must generate the desired macro
%    name by use of |\expandafter| and |\csname|.
%    \begin{macrocode}
    \expandafter\xdef\csname the#1\endcsname
%    \end{macrocode}
%    An |\xdef| is used in order to make the definition global,
%    and to ensure that it contains the replacement texts of
%    |\@thmcountersep| and |\@thmcounter|.\footnote{These
%    two macros can be defined by the document style.  Their default
%    values produce a `\texttt{.}' as separation and an arabic
%    representation of the number.} However, not everything should be
%    expanded. For example, it saves space to use |\thesection|
%    instead of its---at times---lengthy expansion.
%    \begin{macrocode}
      {\expandafter \noexpand \csname the#3\endcsname
       \@thmcountersep \@thmcounter{#1}}%
%    \end{macrocode}
%    Thus with the defaults of \LaTeX{}, |\theLem| would be
%    replaced by the command sequence |\thesection.\arabic{Lem}|.
%
%    We will now look at the definition of the macro which is executed
%    at the beginning of the actual environment (in our example this
%    macro is |\Lem|). It should be noted that we use an
%    ``|\expandafter| trick'' to expand only certain parts of the
%    replacement text at the time of the definition.
%    \begin{macrocode}
    \def\@tempa{\global\@namedef{#1}}%
    \expandafter \@tempa \expandafter{%
%    \end{macrocode}
%    First, the macro that contains the current definitions of
%    |\@begintheorem| and |\@opargtheorem| should be called
%    up.  The name of this macro---as is already known---has the form
%    |\th@|\meta{theorem\ style}; therefore, it must be called by
%    \begin{macrocode}
      \csname th@\the \theorem@style
%    \end{macrocode}
%    In addition the default theorem font should be changeable, i.e.\
%    we have to insert the contents of |\theorem@bodyfont|. For
%    that reason, we expand even further, beyond |\endcsname|,
%    and thus insert the contents of the \textsf{token} register
%    |\theorem@bodyfont| in the replacement text.
%    \begin{macrocode}
            \expandafter \endcsname \the \theorem@bodyfont
%    \end{macrocode}
%    Now it is time to call the macro |\@thm| which takes over
%    the further processing.  It has two arguments: the current
%    counter name (in our example, \texttt{Lem}), and the text of the
%    label (in our example, \texttt{Lemma}).
%    \begin{macrocode}
     \@thm{#1}{#2}}%
%    \end{macrocode}
%    With this, the `sub-definition' is complete.  The macro
%    |\@endtheorem| ends a theorem environment and is, so far, nothing
%    but an |\endtrivlist|.  (Hence it is defined globally, and not
%    within the theorem styles.\footnote{This has to be changed as
%    soon as theorem styles that change \texttt{\bslash @endtheorem}
%    exist.  In such a case, all existing styles must be changed as
%    well since they will have to reset the macro.}) Therefore, we can
%    set it equivalent to the macro that ends the theorem set (in our
%    example, |\endLem|).  However, if some day theorem styles exist
%    that do change |\@endtheorem|, we would have to use the
%    commented-out line instead.
%    \begin{macrocode}
    \global \expandafter \let \csname end#1\endcsname \@endtheorem
%   \global\@namedef{end#1}{\@endtheorem}%
%    \end{macrocode}
%    With these commands all the required definitions are employed,
%    unless the test |\@ifdefinable| has failed.  Therefore, we
%    end the second argument of this macro and with it the definition
%    of |\@xnthm|.
%    \begin{macrocode}
   }}
%    \end{macrocode}
% \end{macro}
%
% \begin{macro}{\@ynthm}
%    The definition of |\@ynthm| is completely analogous. In
%    this case the new counter that is defined is not reset within
%    another counter; thus the definition of |\the...|
%    is simplified:
%    \begin{macrocode}
\gdef\@ynthm#1#2{\expandafter\@ifdefinable\csname #1\endcsname
   {\@definecounter{#1}%
    \expandafter\xdef\csname the#1\endcsname{\@thmcounter{#1}}%
%    \end{macrocode}
%    The rest of the definition corresponds literally to that of
%    |\@xnthm|:
%    \begin{macrocode}
    \def\@tempa{\global\@namedef{#1}}\expandafter \@tempa
     \expandafter{\csname th@\the \theorem@style \expandafter
     \endcsname \the\theorem@bodyfont \@thm{#1}{#2}}%
    \global \expandafter \let \csname end#1\endcsname \@endtheorem}}
%    \end{macrocode}
% \end{macro}
%
% \begin{macro}{\@othm}
%    The definition of |\@othm| does not contain anything new.
%    \begin{macrocode}
\gdef\@othm#1[#2]#3{%
%    \end{macrocode}
% \changes{v2.2b}{95/10/07}{Build in check for counter defined}
%    We do not define a new counter but instead use one that has
%    already been defined.  Thus the only definition we need is that
%    of this pseudo-counter (i.e.\ |\the|\meta{env.\ name}).
%    First we check if |#2| corresponds to a known counter name.
%    \begin{macrocode}
  \expandafter\ifx\csname c@#2\endcsname\relax
   \@nocounterr{#2}%
  \else
   \expandafter\@ifdefinable\csname #1\endcsname
   {\expandafter \xdef \csname the#1\endcsname
     {\expandafter \noexpand \csname the#2\endcsname}%
%    \end{macrocode}
%    All other parts of the definition can be adopted from
%    |\@xnthm|. We have to remember, though, that in this case
%    the name of the current counter and the theorem label have moved
%    to the second and third arguments.
%    \begin{macrocode}
    \def\@tempa{\global\@namedef{#1}}\expandafter \@tempa
     \expandafter{\csname th@\the \theorem@style \expandafter
     \endcsname \the\theorem@bodyfont \@thm{#2}{#3}}%
    \global \expandafter \let \csname end#1\endcsname \@endtheorem}%
  \fi}
%    \end{macrocode}
% \end{macro}
%
% \subsection{Macros that are employed in a {\envfont theorem}
%    environment}
%
% \begin{macro}{\@thm}
%    The macro |\@thm| has to increase the current counter.  Then,
%    depending on whether the environment has (or does not have) an
%    optional argument, it has to branch into either |\@begintheorem|
%    or |\@opargtheorem|.
%    \begin{macrocode}
\gdef\@thm#1#2{\refstepcounter{#1}%
%    \end{macrocode}
%    Now we start a {\envfont trivlist} environment, and give
%    |\@topsep| and |\@topsepadd| the values of the \textsf{skip}
%    registers |\theorempreskipamount| and |\theorempostskipamount|.
%    The value in |\@topsep| is the vertical space that is inserted by
%    the first (and only) |\item| in our |\trivlist| whilst
%    |\@topsepadd| is inserted by |\@endparenv| at the end of that
%    {\envfont trivlist} environment.  By using these registers, we
%    obtain the desired space around a {\envfont theorem} environment.
%    \begin{macrocode}
   \trivlist
   \@topsep \theorempreskipamount               % used by first \item
   \@topsepadd \theorempostskipamount           % used by \@endparenv
%    \end{macrocode}
%    Now we have to test whether an optional argument has been given.
% \changes{v2.0d}{89/07/18}{Forgotten percent added.}
%    \begin{macrocode}
   \@ifnextchar [%
%    \end{macrocode}
%    If there is an optional argument, we will call |\@ythm|, and
%    move the arguments read back into the input stream.
%    \begin{macrocode}
   {\@ythm{#1}{#2}}%
%    \end{macrocode}
%    If not, we call |\@begintheorem|. Its first argument is the
%    name of the theorem set (hence the second argument of
%    |\@thm|). Its second argument is the macro that produces the
%    current number.
%    \begin{macrocode}
   {\@begintheorem{#2}{\csname the#1\endcsname}\ignorespaces}}
%    \end{macrocode}
% \end{macro}
%
% \begin{macro}{\@xthm}
% \begin{macro}{\@ythm}
%    Both these macros were originally called by |\@thm|.  We do
%    not need |\@xthm| anymore, hence we reset it to
%    |\relax|. The definition of |\@ythm| has not changed at
%    all from its definition in \LaTeX. In order to make the macros
%    easier to understand, we will
%    nevertheless present it (commented out).
%    \begin{macrocode}
\global\let\@xthm\relax
% \def\@ythm#1#2[#3]{\@opargbegintheorem{#2}{\csname
%           the#1\endcsname}{#3}\ignorespaces}
%</package>
%    \end{macrocode}
%    The primitive |\ignorespaces| in |\@ythm| and
%    |\@thm| is needed to remove the spaces between the
%    |\begin{...}| and the actual text.
% \end{macro}
% \end{macro}
%
%
% \subsection{Definition of the theorem substyles}
%
% As already pointed out, the theorem substyles, defined below, are
% only loaded when necessary.  Note that all these substyles, except
% \texttt{plain}, have |\slshape| as the default body font.
%
% \subsubsection{The \texttt{plain} style}
%
% \label{dok:th@plain} As the following macros use |@|, we have
% to locally set the |\catcode| of this symbol to ``letter''.
% This happens within a group, so that we do not have to worry about
% which |\catcode| that symbol had before.
%    \begin{macrocode}
%<*thp>
\begingroup \makeatletter
%    \end{macrocode}
% Since we are now within a group, we must make all definitions
% globally.  First we make sure that \texttt{theorem.sty} is loaded.
% This will allow us to use this file as a document style option
% without having to call \texttt{theorem} itself as an option.  At the
% same time, we assure that at least version 2 is loaded, since
% |\theorem@style| was not defined in earlier versions.
%    \begin{macrocode}
\@ifundefined{theorem@style}{\input{theorem.sty}}{}
\ProvidesFile{thp.sty}
             [\FMithmInfo]
%    \end{macrocode}
%
%
% \begin{macro}{\th@plain}
% \begin{macro}{\@begintheorem}
% \begin{macro}{\@opargbegintheorem}
%    |\theoremstyle{plain}| corresponds to the original
%    definition, except that the distances to the surrounding text are
%    determined by the parameters |\theorempreskipamount| and
%    |\theorempostskipamount|.
%    First we set the default body font.
%    \begin{macrocode}
\gdef\th@plain{\normalfont\itshape
%    \end{macrocode}
%    Then we define |\@begintheorem| and
%    |\@opargbegintheorem|. These two macros define how the
%    header of a theorem is typeset.  |\@opargbegintheorem| will
%    be called if a \textsf{theorem} environment with an optional
%    argument is encountered; otherwise, the header is constructed by
%    calling |\@begintheorem|.  If one of these macros is
%    executed, we are within a \textsf{trivlist} environment started by
%    |\@thm|.  So the theorem header is produced with an
%    |\item| command.
%
%    Instead of specifying the header font directly, all standard
%    theorem styles use the |\theorem@headerfont| macro to allow
%    customization.  The extra space (|\labelsep|) is necessary
%    because of problems in the \textsf{trivlist} environment.
%    \begin{macrocode}
  \def\@begintheorem##1##2{%
        \item[\hskip\labelsep \theorem@headerfont ##1\ ##2]}%
%    \end{macrocode}
%    The definition of |\@opargbegintheorem| is completely
%    analogous. The only difference is the fact that there exists a
%    third argument (which is the optional parameter of the
%    environment and contains additional information about the
%    theorem).  Customarily we enclose it in parentheses.
%    \begin{macrocode}
\def\@opargbegintheorem##1##2##3{%
   \item[\hskip\labelsep \theorem@headerfont ##1\ ##2\ (##3)]}}
%    \end{macrocode}
% \end{macro}
% \end{macro}
% \end{macro}
%
% \noindent
% We conclude with an |\endgroup| to restore the
% |\catcode| of |@|.
%    \begin{macrocode}
\endgroup
%</thp>
%    \end{macrocode}
%
%
% \subsubsection{The \texttt{break} style}
%
% This style option is stored in the file |thb.sty|.
% For the next two lines see the documentation for |\th@plain| on
% page~\pageref{dok:th@plain}.
%    \begin{macrocode}
%<*thb>
\begingroup \makeatletter
\@ifundefined{theorem@style}{\input{theorem.sty}}{}
\ProvidesFile{thb.sty}
             [\FMithmInfo]
%    \end{macrocode}
%
% \begin{macro}{\th@break}
%    |\theoremstyle{break}| produces a line break after the name
%    of the theorem.  The font is |\slshape|.  Hence, we define
%    |\th@break| as follows:
%    \begin{macrocode}
\gdef\th@break{\normalfont\slshape
  \def\@begintheorem##1##2{\item[%
%    \end{macrocode}
%    We run into the following problem: it is not possible to create
%    the header with |\item[|\meta{title}|]| and then start
%    a new line by, for example, |\mbox{}\\|. Such a definition
%    will fail whenever a list environment follows immediately.  With
%    the above construction, the |\mbox{}| causes the switch
%    |@inlabel| (cf.\ definition of |\list| and
%    |\trivlist| in \cite{latex:source}) to be set to \texttt{false}
%    and so the following list will insert additional vertical
%    space (|\topskip|). This is quite annoying. Therefore, we
%    create the line break within the |\item|. In order to ensure
%    that the text will begin at the proper position in the following
%    line, we simply pretend that the label does not take any
%    room.\footnote{This will lead to problems whenever very high
%    symbols occurring in the line tower into the
%    heading. So, something else has to be done here sometime.}
%    \begin{macrocode}
       \rlap{\vbox{\hbox{\hskip \labelsep\theorem@headerfont ##1\ ##2}%
                   \hbox{\strut}}}]}%
%    \end{macrocode}
%    Again, the definition of |\@opargbegintheorem| is
%    completely analogous.
%    \begin{macrocode}
\def\@opargbegintheorem##1##2##3{%
  \item[\rlap{\vbox{\hbox{\hskip \labelsep \theorem@headerfont
                     ##1\ ##2\ (##3)}%
                    \hbox{\strut}}}]}}
\endgroup
%</thb>
%    \end{macrocode}
% \end{macro}
%
%
% \subsubsection{The \texttt{changebreak} style}
%
%    \begin{macrocode}
% This style option is stored in the file |thcb.sty|.
%    \begin{macrocode}
%<*thcb>
\begingroup \makeatletter
\@ifundefined{theorem@style}{\input{theorem.sty}}{}
\ProvidesFile{thcb.sty}
             [\FMithmInfo]
%    \end{macrocode}
%
% \begin{macro}{\th@changebreak}
%    The \texttt{change-break} theorem style is like \texttt{break}
%    but with interchange of theorem name and theorem number.  Thus we
%    define |\th@changebreak| as follows:
%    \begin{macrocode}
\gdef\th@changebreak{\normalfont\slshape
  \def\@begintheorem##1##2{\item
       [\rlap{\vbox{\hbox{\hskip\labelsep \theorem@headerfont ##2\ ##1}%
                    \hbox{\strut}}}]}%
\def\@opargbegintheorem##1##2##3{%
  \item[\rlap{\vbox{\hbox{\hskip\labelsep \theorem@headerfont
                     ##2\ ##1\ (##3)}%
                    \hbox{\strut}}}]}}
\endgroup
%</thcb>
%    \end{macrocode}
% \end{macro}
%
%
% \subsubsection{The \texttt{change} style}
%
% This style option is stored in the file |thc.sty|.
%    \begin{macrocode}
%<*thc>
\begingroup \makeatletter
\@ifundefined{theorem@style}{\input{theorem.sty}}{}
\ProvidesFile{thc.sty}
             [\FMithmInfo]
%    \end{macrocode}
%
% \begin{macro}{\th@change}
%    The \texttt{change} theorem style corresponds to the
%    \texttt{change break} style without a linebreak after the header.
%    To say it in another way, it's the same as the \texttt{plain}
%    style but with number and name interchanged and |\slshape| as the
%    default font.
%    \begin{macrocode}
\gdef\th@change{\normalfont\slshape
  \def\@begintheorem##1##2{\item
       [\hskip\labelsep \theorem@headerfont ##2\ ##1]}%
\def\@opargbegintheorem##1##2##3{%
  \item[\hskip\labelsep \theorem@headerfont ##2\ ##1\ (##3)]}}
\endgroup
%</thc>
%    \end{macrocode}
% \end{macro}
%
%
% \subsubsection{The \texttt{marginbreak} style}
%
% This style option is the one used most often at Mainz.  It is saved
% in the file |thmb.sty|.
%    \begin{macrocode}
%<*thmb>
\begingroup \makeatletter
\@ifundefined{theorem@style}{\input{theorem.sty}}{}
\ProvidesFile{thmb.sty}
             [\FMithmInfo]
%    \end{macrocode}
%
% \begin{macro}{\th@marginbreak}
%    The \texttt{margin break} style is nearly the same as the
%    \texttt{change break} style. The only difference is the placement
%    of the theorem number.  We use |\llap| to place it in the left
%    margin.
%
%    In this style |\labelsep| denotes the separation between the
%    number and the text.
%    \begin{macrocode}
\gdef\th@marginbreak{\normalfont\slshape
  \def\@begintheorem##1##2{\item
       [\rlap{\vbox{\theorem@headerfont
                    \hbox{\llap{##2}\hskip\labelsep ##1}%
                    \hbox{\strut}}}]}%
\def\@opargbegintheorem##1##2##3{%
  \item[\rlap{\vbox{\theorem@headerfont
                    \hbox{\llap{##2}\hskip\labelsep ##1\ (##3)}%
                    \hbox{\strut}}}]}}
\endgroup
%</thmb>
%    \end{macrocode}
% \changes{v2.1c}{90/02/17}{`endgroup outside macrocode env.}
% \end{macro}
%
%
%
% \subsubsection{The \texttt{margin} style}
%
% This style option is stored in the file |thm.sty|.
%    \begin{macrocode}
%<*thm>
\begingroup \makeatletter
\@ifundefined{theorem@style}{\input{theorem.sty}}{}
\ProvidesFile{thm.sty}
             [\FMithmInfo]
%    \end{macrocode}
%
% \begin{macro}{\th@margin}
%    Again this is only a variant of the theorem styles described
%    above without any new ideas.
%    \begin{macrocode}
\gdef\th@margin{\normalfont\slshape
  \def\@begintheorem##1##2{\item
       [\theorem@headerfont \llap{##2}\hskip\labelsep ##1]}%
\def\@opargbegintheorem##1##2##3{%
  \item[\theorem@headerfont \llap{##2}\hskip\labelsep ##1\ (##3)]}}
\endgroup
%</thm>
%    \end{macrocode}
% \end{macro}
%
%
%
%
% \subsection{Final Definitions}
%
% \begin{macro}{\theorempreskipamount}
% \begin{macro}{\theorempostskipamount}
%    The \textsf{skip} parameters that regulate the vertical empty space
%    before and after the {\envfont theorem} environment have
%    to be allocated as well.
% \changes{v2.0f}{89/07/20}{Unnecessary `global' removed.}
%    \begin{macrocode}
%<*package>
\newskip\theorempreskipamount
\newskip\theorempostskipamount
%    \end{macrocode}
%    Since we have used the same values for all theorem sets, we now can
%    assign them.
%    \begin{macrocode}
\global\setlength\theorempreskipamount{12pt plus 5pt minus 3pt}
\global\setlength\theorempostskipamount{8pt plus 3pt minus 1.5pt}
%    \end{macrocode}
% \end{macro}
% \end{macro}
%
%
% \begin{macro}{\@endtheorem}
%    The same holds for the macro |\@endtheorem|, which ends a
%    {\envfont theorem} environment. Since it is the same for all
%    theorem sets, it is removed from the macros |\th@|\meta{style}.
%    It simply ends the {\envfont trivlist} environment, which was
%    begun in |\@thm|.
%    \begin{macrocode}
\global\let\@endtheorem=\endtrivlist
%    \end{macrocode}
% \end{macro}
%
%
%
% \begin{macro}{\@preamblecmds}
%    All macros defined above are to be used only in the preamble.
%    Therefore, we insert them in |\@preamblecmds| which will disable
%    them at begin document. This is done by the internal \LaTeXe{}
%    command |\@onlypreamble|.
% \changes{v2.0g}{89/09/19}{`theoremheaderfont and `theorembodyfont
%    added.}
% \changes{v2.2a}{94/02/03}{Use \LaTeXe{} interface}
%    \begin{macrocode}
\@onlypreamble\@xnthm
\@onlypreamble\@ynthm
\@onlypreamble\@othm
\@onlypreamble\newtheorem
\@onlypreamble\theoremstyle
\@onlypreamble\theorembodyfont
\@onlypreamble\theoremheaderfont
%    \end{macrocode}
% \end{macro}
%
% Finally we declare the \texttt{plain} theorem style to be the default.
% \changes{v2.1b}{90/06/04}{Fixed missing `item by declaring default
%    style.}
%    \begin{macrocode}
\theoremstyle{plain}
%</package>
%    \end{macrocode}
%
% \Finale

\endinput
}{}
\ProvidesFile{thcb.sty}
             [\FMithmInfo]
%    \end{macrocode}
%
% \begin{macro}{\th@changebreak}
%    The \texttt{change-break} theorem style is like \texttt{break}
%    but with interchange of theorem name and theorem number.  Thus we
%    define |\th@changebreak| as follows:
%    \begin{macrocode}
\gdef\th@changebreak{\normalfont\slshape
  \def\@begintheorem##1##2{\item
       [\rlap{\vbox{\hbox{\hskip\labelsep \theorem@headerfont ##2\ ##1}%
                    \hbox{\strut}}}]}%
\def\@opargbegintheorem##1##2##3{%
  \item[\rlap{\vbox{\hbox{\hskip\labelsep \theorem@headerfont
                     ##2\ ##1\ (##3)}%
                    \hbox{\strut}}}]}}
\endgroup
%</thcb>
%    \end{macrocode}
% \end{macro}
%
%
% \subsubsection{The \texttt{change} style}
%
% This style option is stored in the file |thc.sty|.
%    \begin{macrocode}
%<*thc>
\begingroup \makeatletter
\@ifundefined{theorem@style}{% \iffalse meta-comment
%
% Copyright (C) 1993-2024
%
% The LaTeX Project and any individual authors listed elsewhere
% in this file.
%
% This file is part of the Standard LaTeX `Tools Bundle'.
% -------------------------------------------------------
%
% It may be distributed and/or modified under the
% conditions of the LaTeX Project Public License, either version 1.3c
% of this license or (at your option) any later version.
% The latest version of this license is in
%    https://www.latex-project.org/lppl.txt
% and version 1.3c or later is part of all distributions of LaTeX
% version 2005/12/01 or later.
%
% The list of all files belonging to the LaTeX `Tools Bundle' is
% given in the file `manifest.txt'.
%
% \fi
%
%%
%
%
\def\FMithmInfo{2023/07/05 v2.2c Theorem extension package (FMi)}
%
% \ProvidesFile{theorem.dtx}[\FMithmInfo]
%\iffalse   % this is a METACOMMENT !
%%
%% Package `theorem' to use with LaTeX2e
%% Copyright (C) 1989-1995 Frank Mittelbach, all rights reserved.
%<+package>\@ifundefined{theorem@style}{}{\endinput}
%<+package>\NeedsTeXFormat{LaTeX2e}[1995/06/01]
%<+package>\ProvidesPackage{theorem}
%<+package>         [\FMithmInfo]
%
% \fi
%
% \changes{v1.0}{87/05/07}{First draft.}
% \changes{v1.0a}{87/05/07}{`thistheoremstyle' added.}
% \changes{v1.1a}{87/07/29}{New concept: redefine the internal
%                          \protect\LaTeX{} commands `@xnthm' etc.}
% \changes{v1.1b}{87/07/31}{`theoremfont' used to specify font.}
% \changes{v1.2a}{88/01/08}{`@endtheorem' now globally defined.}
% \changes{v1.2b}{88/01/11}{`newtheorem' only allowed in the preamble.}
% \changes{v1.3a}{88/02/14}{`theoremfont' now a toks register.}
%
% \changes{v2.0a}{88/11/16}{New implementation and new documentation.}
% \changes{v2.0a}{88/11/16}{No layout changes but internal modifications
%                           old substyles won't work.}
% \changes{v2.0b}{89/01/30}{`global' used for all definitions.}
% \changes{v2.0c}{89/06/15}{Documentation translated to English.
%                            Thanks to LLL and Nina}
% \changes{v2.1a}{90/02/15}{Support for new font selection.}
% \changes{v2.2a}{94/02/03}{Cleanup for LaTeX2e}
%
% \DoNotIndex{\@endparenv,\@ifdefinable,\@ifnextchar,\@ifundefined}
% \DoNotIndex{\@namedef,\@nameuse,\@spaces,\@tempa,\@warning}
% \DoNotIndex{\ }
% \DoNotIndex{\begingroup,\bfseries}
% \DoNotIndex{\csname}
% \DoNotIndex{\def,\do,\docdate}
% \DoNotIndex{\endcsname,\endgroup,\endinput,\endtrivlist,\expandafter}
% \DoNotIndex{\filedate,\fileversion}
% \DoNotIndex{\gdef,\global}
% \DoNotIndex{\hbox,\hskip}
% \DoNotIndex{\ignorespaces,\input,\itshape,\item}
% \DoNotIndex{\labelsep,\let,\llap}
% \DoNotIndex{\makeatletter}
% \DoNotIndex{\noexpand}
% \DoNotIndex{\refstepcounter,\relax,\rlap}
% \DoNotIndex{\setlength,\slshape,\space,\string,\strut}
% \DoNotIndex{\the,\trivlist,\typeout}
% \DoNotIndex{\vbox}
% \DoNotIndex{\xdef}
%
% \def\deflabel#1{\ttfamily #1\hfill}
% \def\deflist#1{\begin{list}{}{\settowidth\labelwidth{\ttfamily #1}%
%                               \setlength\leftmargin\labelwidth
%                               \addtolength\leftmargin\labelsep
%                               \let\makelabel\deflabel}}
% \def\enddeflist{\end{list}}
% \def\envfont{\normalfont\ttfamily}
%
% \MakeShortVerb{\|}
%
% \newcommand{\TUB}{{\sl TUGboat\/}}
%
% \GetFileInfo{theorem.dtx}
% \title{An Extension of the
%        \LaTeX{} theorem environment\thanks{This file
%        has version number \fileversion, last
%        revised \filedate.}}
% \author{Frank Mittelbach}
%
% \MaintainedByLaTeXTeam{tools}
% \maketitle
%
% \begin{abstract}
%    The macros described in this paper yield an extension of the
%    \LaTeX{} theorem mechanism. It is designed to satisfy the
%    different requirements of various journals. Thus, the
%    layout of the ``theorems'' can be manipulated by determining a
%    ``style''. This article describes not only the use, but
%    also the definition, of the necessary macros.
% \end{abstract}
%
% \section*{Preface to version 2.2}
%
% For \LaTeXe{} this package did not need any fundamental changes. I
% only modified the messages generated so that theorem layout styles
% will show up with the |\listfiles| command and cleaned the section
% on the New Font Selection Scheme since this is now included in
% \LaTeX.
%
% \section*{Preface to version 2.1}
%
% This version is identical to 2.0g described in \TUB\ 10\#3 except
% for some internal defaults which are now set depending on the used
% font selection scheme.
%
% This was done to avoid unpleasant surprises if the new font selection
% scheme is in force. For further details see
% section~\ref{sec:newfonts} and \cite{bk:GMS94}.
%
% \section{Introduction}
%
% For our purposes here, ``theorems'' are labelled enunciations,
% often set off from the main text by extra space and a font change.
% Theorems, corollaries, conjectures, definitions, and remarks
% are all instances of ``theorems''.  The ``header'' of these structures
% is composed of a label (such as \textsc{Theorem} or \textsc{Remark})
% and a number which serializes an item in the sequence of items
% with the same label.
%
% Shortly after the introduction of \LaTeX{} at the Fachbereich
% Mathematik in Mainz, the desire to manipulate the layout of
% ``theorems'' arose.  In Mainz, the following two conventions came
% into general use:
% \begin{enumerate}
%   \item The number of the theorem is shown in the margin.
%   \item There is a line break at
%         the end of the theorem header.
% \end{enumerate}
%
% Additionally, some journals require different formats which depend on
% the ``sort of theorem'': e.g.\ often remarks and definitions are set
% in |\upshape|, while |\itshape| is employed for main theorems.
%
% Confronted with these requirements, a theorem environment was
% developed in Mainz which allows separate determination of the
% layout of the ``theorems sets'', comparable to |\pagestyle|.
%
%
% \section{The user interface}
% \subsection{Defining new theorem sets}
%
% \DescribeMacro\newtheorem
% As in the original \LaTeX{} version, the command |\newtheorem|
% defines a new ``theorem set'' or ``theorem-like structure''.
% Two required arguments name the new environment and give the text
% to be typeset with each instance of the new ``set'', while
% an optional argument determines how
% the ``set'' is enumerated:
% \begin{description}
%    \item[\ttfamily \bslash newtheorem\{foo\}\{bar\}]
%       The theorem set {\envfont foo} (whose name is \texttt{bar})
%       uses its own counter.
%    \item[\ttfamily \bslash newtheorem\{foo2\}{[foo]}\{bar2\}]
%       The theorem set {\envfont foo2} (printed name \texttt{bar2})
%       uses the same counter as the theorem set \texttt{foo}.
%    \item[\ttfamily \bslash newtheorem\{foo3\}\{bar3\}{[section]}]
%       The theorem set {\envfont foo3} (printed name \texttt{bar3}) is
%       enumerated within the counter \texttt{section}, i.e.\ with every
%       new |\section| the enumeration begins again with 1, and
%       the enumeration is composed from the section-number and the
%       theorem counter itself.
% \end{description}
%
% \DescribeMacro\theoremstyle
% Additionally, the command |\theoremstyle| can define the layout
% of various, or all, theorem sets.  It should be noted that any
% theorem set defined by |\newtheorem| is typeset in the
% |\theoremstyle| that is current at the time of the definition.
% Thus, the following
% \begin{verbatim}
%    \theoremstyle{break}        \newtheorem{Cor}{Corollary}
%    \theoremstyle{plain}        \newtheorem{Exa}{Example}[section]
%\end{verbatim}                             ^^A No blank at line start
% leads to the result that the set {\envfont Cor} is formatted in the
% style \texttt{break}, while the set {\envfont Exa} and all the
% following ones are formatted in the style \texttt{plain}, unless
% another |\theoremstyle| follows.  Since the definitions installed by
% |\newtheorem| are global, one also can limit |\theoremstyle| locally
% by grouping braces.
%
% \DescribeMacro\theorembodyfont
% The choice of the font for the theorem body is completely
% independent of the chosen |\theoremstyle|; this has proven to
% be very advantageous.  For example,
% \begin{verbatim}
%   {\theorembodyfont{\upshape}          \newtheorem{Rem}{Remark}}
%\end{verbatim}
% defines a theorem set {\envfont Rem}, which will be set in
% |\upshape| in the current layout (which in our example is
% \texttt{plain}).  As with |\theoremstyle|, the |\theorembodyfont|
% chosen is that current at the time of |\newtheorem|.  If
% |\theorembodyfont| is not specified or one defines
% |\theorembodyfont{}|, then the font used will be that defined by the
% |\theoremstyle|.
%
% \DescribeMacro\theoremheaderfont It is also possible to customize
% the font used for the theorem headers.  This is, however, a global
% declaration, and therefore there should be at most one
% |\theoremheaderfont| declaration in the preamble.\footnote{If it is
% actually necessary to have different header fonts, one has to define
% new theorem styles (substituting the desired font) or specify the
% information directly in the \texttt{\bslash newtheorem} declaration
% (the unclean variant).}
%
% \DescribeMacro\theorempreskipamount
% \DescribeMacro\theorempostskipamount Two additional parameters
% affect the vertical space around the theorem environments:
% |\theorempreskipamount| and |\theorempostskipamount| define,
% respectively, the spacing before and after such an environment.
% These parameters apply for all theorem sets and can be manipulated
% with the ordinary length macros.  They are rubber lengths
% (`\textsf{skips}'), and therefore can contain \texttt{plus} and
% \texttt{minus} parts.
%
% Since the definition of theorem sets should---most sensibly---be
% placed in the preamble, we only allow installation there.
% It is therefore possible to release the memory used here after
% |\begin{document}|, in order to make room for other
% applications.
%
%
% \subsection{Existing theorem styles}
%
% The following theorem styles exist to date:
% \begin{deflist}{marginbreak}
%    \item[plain]
%       This theorem style emulates the original \LaTeX{} definition,
%       except that additionally the parameters
%       |\theorem...skipamount| are used.
%    \item[break]
%       In this style, the theorem header is followed by a line
%       break.
%    \item[marginbreak]
%       The theorem number is set in the margin, and there is a line
%       break as in \texttt{break}.
%    \item[changebreak]
%       Like \texttt{break}, but with header number and text
%       interchanged.
%    \item[change]
%       Header number and text are interchanged, without a line break.
%    \item[margin]
%       The number is set in the left margin, without a line break.
% \end{deflist}
% All styles (except \texttt{plain}) select |\slshape| as the default
% |\theorembodyfont|.
%
%
% \subsection{Examples}
%
% Given the above theorem sets {\envfont Cor}, {\envfont Exa}
% and {\envfont Rem},
% suppose that the preamble also contains the declarations:
% \begin{verbatim}
%    \theoremstyle{marginbreak}   \newtheorem{Lem}[Cor]{Lemma}
%    \theoremstyle{change}
%    \theorembodyfont{\itshape}        \newtheorem{Def}[Cor]{Definition}
%
%    \theoremheaderfont{\scshape}
%\end{verbatim}
% Then the following are some typical examples of the typeset output
% resulting from their use.
%
% \begin{Cor}
%    This is a sentence typeset in the theorem environment {\envfont
%    Cor}.
% \end{Cor}
%
% \begin{Exa}
%    This is a sentence typeset in the theorem environment {\envfont
%    Exa}.
% \end{Exa}
%
% \begin{Rem}
%    This is a sentence typeset in the theorem environment {\envfont
%    Rem}.
% \end{Rem}
%
% \begin{Lem}[Ben User]
%    This is a sentence typeset in the theorem environment {\envfont
%    Lem}.
% \end{Lem}
%
% \begin{Def}[Very impressive Definition]
%    This is a sentence typeset in the theorem environment {\envfont
%    Def}.
% \end{Def}
%
% The last two examples show the effect of the optional argument to
% a theorem environment (it is the text typeset in parentheses).
%
%
% \section{Special Considerations}
% \label{sec:newfonts}
%
% Theoremheader and body are implemented as a unit. This means that
% the |\theoremheaderfont|  will inherit characteristics of the
% |\theorembodyfont| in \LaTeXe. Thus, if for example
% |\theorembodyfont| is |\itshape| and |\theoremheaderfont|
% is |\bfseries| the font selected for the header will have the
% characteristics `bold extended italic'. If this is not desired one
% should set the |\theoremheaderfont| to something like
% \begin{verbatim}
%      \theoremheaderfont{\normalfont\bfseries}
%\end{verbatim}
% i.e.\ supplying all necessary font information explicitly.
%
%
%
% \section{Acknowledgements}
%
% The publication of this set of macros was only possible with the
% help of Christina Busse (translating the manuscript into English),
% Joachim Pense (playing the r\^ole of typist), Chris Rowley
% (looking everything over) and many others providing useful
% suggestions.
%
% \MaybeStop{%
%    \begin{thebibliography}{9}
%       \bibitem{bk:GMS94} \textsc{M.~Goossens}, \textsc{F.~Mittelbach}
%         and \textsc{A.~Samarin}.
%         \newblock The \LaTeX{} Companion.
%         \newblock
%         Addison-Wesley, Reading, Massachusetts, 1994.
%       \bibitem{latex:source} \textsc{Lamport, Leslie.} \newblock
%                               \texttt{latex.tex}, version 2.09, date
%                               Feb. 1990.
%    \end{thebibliography}
%    \PrintIndex
%    \PrintChanges}
%
%
% \section{The documentation driver file}
%
% The next bit of code contains the documentation driver file for
% \TeX{}, i.e., the file that will produce the documentation you are
% currently reading. It will be extracted from this file by the
% \texttt{docstrip} program. Since it is the first code in the file
% one can alternatively process this file directly with \LaTeXe{} to
% obtain the documentation.
%
%    \begin{macrocode}
%<*driver>
\documentclass{ltxdoc}

\usepackage{theorem}

%<+driver>% The next few lines define theorem sets which are used
%<+driver>% in the example section of the documentation.

\theoremstyle{break}       \newtheorem{Cor}{Corollary}
\theoremstyle{plain}       \newtheorem{Exa}{Example}[section]
{\theorembodyfont{\upshape}\newtheorem{Rem}{Remark}}
\theoremstyle{marginbreak} \newtheorem{Lem}[Cor]{Lemma}
\theoremstyle{change}
\theorembodyfont{\itshape} \newtheorem{Def}[Cor]{Definition}

\theoremheaderfont{\scshape}

\RecordChanges

\begin{document}
   \DocInput{theorem.dtx}
\end{document}
%</driver>
%    \end{macrocode}
%
% \section{Definition of the Macros}
%
% If the file has been loaded before, we abort immediately. If not the
% package announces itself (this is actually done at the very top of
% the file---the way it is done isn't good style so don't copy it).
% \changes{v2.0e}{89/07/19}{Spaces removed from `typeout'.}
% \changes{v2.2b}{95/11/19}{Announce moved to top of file}
%    \begin{macrocode}
%<*package>
%\@ifundefined{theorem@style}{}{\endinput}
%\def\FMithmInfo{1995/11/19 v2.2b Theorem extension package (FMi)}
%\ProvidesPackage{theorem}[\FMithmInfo]
%    \end{macrocode}
%
%
% \subsection{Definition of theorem styles and fonts}
%
% All the definitions in this file are done globally to allow
% inputting this file inside a group.
%
% \begin{macro}{\theoremstyle}
%    Before a theorem style can be installed, the chosen style must be
%    known. For that reason, we must test to see that
%    |\th@|\meta{style} is known or, more precisely, that it is
%    different from |\relax|. If the style is not known then
%    |\th@plain| is used.
%    \begin{macrocode}
\gdef\theoremstyle#1{%
   \@ifundefined{th@#1}{\@warning
          {Unknown theoremstyle `#1'. Using `plain'}%
          \theorem@style{plain}}%
%    \end{macrocode}
%    We save the theorem style to be used in the token register
%    |\theorem@style|.
%    \begin{macrocode}
      {\theorem@style{#1}}%
%    \end{macrocode}
%    Now we ``evaluate'' the theorem style: this means, we call the
%    macro |\th@|\meta{style} which will activate the relevant
%    definitions which are contained in a separate file.  This is done
%    in a group to suppress changes to the current font.  This could
%    otherwise pose problems together with the new font selection
%    scheme\footnote{When I printed the original article using the new
%    font selection scheme I ended with a document with slanted
%    typefaces (text headings and all) simply because one of the
%    theorem styles used \texttt{\bslash sl} at toplevel.} if the
%    |\th@|\meta{style} is evaluated a second time.
% \changes{v2.1a}{90/02/15}{Group added around `th@..}
% \changes{v2.1a}{90/02/15}{Reseting of `..@begintheorem removed.}
%    \begin{macrocode}
      \begingroup
        \csname th@\the\theorem@style \endcsname
      \endgroup}
%    \end{macrocode}
% \end{macro}
%
%
% \begin{macro}{\@begintheorem}
% \changes{v2.1a}{90/02/15}{Relaxed at toplevel.}
% \begin{macro}{\@opargbegintheorem}
% \changes{v2.1a}{90/02/15}{Relaxed at toplevel.}
%    We reset |\@begintheorem| and |\@opargbegintheorem| to
%    |\relax| since these commands are no longer necessary at
%    toplevel. This will save a few tokens.
%    \begin{macrocode}
\global\let\@begintheorem\relax
\global\let\@opargbegintheorem\relax
%    \end{macrocode}
% \end{macro}
% \end{macro}
%
% \begin{macro}{\theorem@style}
%    Obviously the \textsf{token} register used above has to be
%    allocated. To assure the utmost compatibility with the
%    original \LaTeX{} definition, we set the default theorem style to
%    \texttt{plain}, which implements the usual \LaTeX{} convention.
% \changes{v2.0f}{89/07/20}{Unnecessary `global' removed.}
%    \begin{macrocode}
\newtoks\theorem@style
\global\theorem@style{plain}
%    \end{macrocode}
% \end{macro}
%
% \begin{macro}{\theorembodyfont}
% \begin{macro}{\theorem@bodyfont}
%    If the |\theorembodyfont| is set by the user then it should not
%    interact with the default font set in the theorem style. When the
%    new font selection is in force this may happen if, for example,
%    the default is |\itshape| and the new |\theorembodyfont| is
%    |\sffamily|. So we add a |\reset@font| command in front of the
%    user definition.
% \changes{v2.0f}{89/07/20}{Unnecessary `global' removed.}
% \changes{v2.0g}{89/09/19}{`theoremfont to `theorembodyfont.}
% \changes{v2.1c}{92/06/22}{`reset@font added and logic changed.}
% \changes{v2.1d}{92/10/28}{Reset to empty if arg is empty}
%    \begin{macrocode}
\gdef\theorembodyfont#1{%
%    \end{macrocode}
%    We check if the argument supplied is empty and if so put nothing
%    into the |\theorem@bodyfont| token register to allow for
%    |\theorembodyfont{}| as a mean of using the default of the
%    current |\theoremstyle|.
%    \begin{macrocode}
   \def\@tempa{#1}%
   \ifx\@tempa\@empty
    \theorem@bodyfont{}%
   \else
    \theorem@bodyfont{\reset@font#1}%
   \fi
   }
\newtoks\theorem@bodyfont
\global\theorem@bodyfont{}
%    \end{macrocode}
% \end{macro}
% \end{macro}
%
%
% \begin{macro}{\theoremheaderfont}
% \changes{v2.0g}{89/09/19}{Macro added.}
%    The font for the theorem headers is handled differently because
%    this definition applies to all theorem styles.
%    \begin{macrocode}
\gdef\theoremheaderfont#1{\gdef\theorem@headerfont{#1}%
%    \end{macrocode}
%    After using the macro once it is redefined to produce an error
%    message.
%    \begin{macrocode}
       \gdef\theoremheaderfont##1{%
        \typeout{\string\theoremheaderfont\space should be used
                 only once.}}}
%    \end{macrocode}
% \end{macro}
%
% \begin{macro}{\theorem@headerfont}
% \changes{v2.0g}{89/09/19}{Macro added.}
%    To set the |\theorem@headerfont| default we first test if
%    the new fontselection scheme is in force.
% \changes{v2.1a}{90/02/15}{Support for new font selection.}
%    \begin{macrocode}
\ifx\upshape\undefined
%    \end{macrocode}
%    If not we define it to expand into |\bfseries|. We don't use
%    |\let| just in case a following style option redefines this
%    macro.
%    \begin{macrocode}
\gdef\theorem@headerfont{\bfseries}
%    \end{macrocode}
%    Otherwise we reset the current shape before calling |\bfseries|.
%    \begin{macrocode}
\else \gdef\theorem@headerfont{\normalfont\bfseries}\fi
%    \end{macrocode}
% \end{macro}
%
% \begin{macro}{\th@plain}
% \begin{macro}{\th@break}
% \begin{macro}{\th@marginbreak}
% \begin{macro}{\th@changebreak}
% \begin{macro}{\th@change}
% \begin{macro}{\th@margin}
%    The different styles are defined in macros such as
%    |\th@plain|.  Since memory space is precious in
%    ``non-Big-versions'', we have to avoid offering too many unused
%    definitions.  Therefore we define these styles in separate files
%    that can be loaded on demand.  Thus the commands themselves only
%    load these files. We use |\@input@|, a \LaTeXe{} internal command
%    that ensures that the file will be listed with |\listfiles|.
% \changes{v2.2a}{94/02/02}{Use `@input@ to load theorem layout files}
%    \begin{macrocode}
\gdef\th@plain{\@input@{thp.sty}}
\gdef\th@break{\@input@{thb.sty}}
\gdef\th@marginbreak{\@input@{thmb.sty}}
\gdef\th@changebreak{\@input@{thcb.sty}}
\gdef\th@change{\@input@{thc.sty}}
\gdef\th@margin{\@input@{thm.sty}}
%    \end{macrocode}
%    This list will be expanded when new styles become
%    available.  For testing,  just append new theorem substyles
%    as document options.
% \end{macro}
% \end{macro}
% \end{macro}
% \end{macro}
% \end{macro}
% \end{macro}
%
%
%
% \subsection{Definition of a new theorem set}
%
% As already pointed out, a new theorem environment can be defined in
% three different ways:
% \begin{verbatim}
%    \newtheorem{Lem}{Lemma}
%    \newtheorem{Lem}{Lemma}[section]
%    \newtheorem{Lem}[Theorem]{Lemma}
%\end{verbatim}
% The function of the macro |\newtheorem| is to recognize these cases
% and then to branch into one of the three macros |\@ynthm|, |\@xnthm|
% or |\@othm|.  This mechanism is adopted unchanged from
% \cite{latex:source}; the essential point here is that, for example,
% in the second case, the arguments \texttt{Lem}, \texttt{Lemma} and
% \texttt{section} are passed over to the macro |\@xnthm|.
%
% We inspect this case first because the others present fewer problems,
% and thus are easily derived from this one.
%
% \begin{macro}{\@xnthm}
%    For our example arguments, the macro |\@xnthm| must fulfill
%    the following:
%    \begin{itemize}
%       \item
%          Define a new \LaTeX-counter `Lem'
%       \item
%          reset this counter within a |\section|
%       \item
%          define the macro |\theLem|
%       \item
%          define the environment macros |\Lem| and
%          |\endLem| using the current
%          |\theoremstyle| and |\theorem@bodyfont|.
%    \end{itemize}
%    Obviously, all this should happen only if the first argument
%    of |\@xnthm| (i.e.\ \texttt{Lem} in our example) is chosen so
%    as not to conflict with any previously defined commands or
%    environments.  This test is performed by the \LaTeX{} macro
%    |\@ifdefinable|.
%    \begin{macrocode}
\gdef\@xnthm#1#2[#3]{\expandafter\@ifdefinable\csname #1\endcsname
%    \end{macrocode}
%    Therefore, the first argument of |\@ifdefinable| is the
%    expansion (in the example, |\Lem|) of
%    |\csname||#1\endcsname|.  The second argument is
%    executed only if the test has been completed successfully.
%    \begin{macrocode}
   {%
%    \end{macrocode}
%    Now we define the new counter. The names of the \LaTeX{}
%    macros employed should speak for themselves:
% \changes{v2.2b}{1995/10/07}{Use \cs{@newctr}}
% \changes{v2.2c}{1995/11/23}{Use \cs{@newctr} properly.}
%    \begin{macrocode}
    \@definecounter{#1}\@newctr{#1}[#3]%
%    \end{macrocode}
%    Using |\@newctr| will give a proper error message if the counter
%    in |#3| is not defined.
%    In defining `|\theLem|' we must generate the desired macro
%    name by use of |\expandafter| and |\csname|.
%    \begin{macrocode}
    \expandafter\xdef\csname the#1\endcsname
%    \end{macrocode}
%    An |\xdef| is used in order to make the definition global,
%    and to ensure that it contains the replacement texts of
%    |\@thmcountersep| and |\@thmcounter|.\footnote{These
%    two macros can be defined by the document style.  Their default
%    values produce a `\texttt{.}' as separation and an arabic
%    representation of the number.} However, not everything should be
%    expanded. For example, it saves space to use |\thesection|
%    instead of its---at times---lengthy expansion.
%    \begin{macrocode}
      {\expandafter \noexpand \csname the#3\endcsname
       \@thmcountersep \@thmcounter{#1}}%
%    \end{macrocode}
%    Thus with the defaults of \LaTeX{}, |\theLem| would be
%    replaced by the command sequence |\thesection.\arabic{Lem}|.
%
%    We will now look at the definition of the macro which is executed
%    at the beginning of the actual environment (in our example this
%    macro is |\Lem|). It should be noted that we use an
%    ``|\expandafter| trick'' to expand only certain parts of the
%    replacement text at the time of the definition.
%    \begin{macrocode}
    \def\@tempa{\global\@namedef{#1}}%
    \expandafter \@tempa \expandafter{%
%    \end{macrocode}
%    First, the macro that contains the current definitions of
%    |\@begintheorem| and |\@opargtheorem| should be called
%    up.  The name of this macro---as is already known---has the form
%    |\th@|\meta{theorem\ style}; therefore, it must be called by
%    \begin{macrocode}
      \csname th@\the \theorem@style
%    \end{macrocode}
%    In addition the default theorem font should be changeable, i.e.\
%    we have to insert the contents of |\theorem@bodyfont|. For
%    that reason, we expand even further, beyond |\endcsname|,
%    and thus insert the contents of the \textsf{token} register
%    |\theorem@bodyfont| in the replacement text.
%    \begin{macrocode}
            \expandafter \endcsname \the \theorem@bodyfont
%    \end{macrocode}
%    Now it is time to call the macro |\@thm| which takes over
%    the further processing.  It has two arguments: the current
%    counter name (in our example, \texttt{Lem}), and the text of the
%    label (in our example, \texttt{Lemma}).
%    \begin{macrocode}
     \@thm{#1}{#2}}%
%    \end{macrocode}
%    With this, the `sub-definition' is complete.  The macro
%    |\@endtheorem| ends a theorem environment and is, so far, nothing
%    but an |\endtrivlist|.  (Hence it is defined globally, and not
%    within the theorem styles.\footnote{This has to be changed as
%    soon as theorem styles that change \texttt{\bslash @endtheorem}
%    exist.  In such a case, all existing styles must be changed as
%    well since they will have to reset the macro.}) Therefore, we can
%    set it equivalent to the macro that ends the theorem set (in our
%    example, |\endLem|).  However, if some day theorem styles exist
%    that do change |\@endtheorem|, we would have to use the
%    commented-out line instead.
%    \begin{macrocode}
    \global \expandafter \let \csname end#1\endcsname \@endtheorem
%   \global\@namedef{end#1}{\@endtheorem}%
%    \end{macrocode}
%    With these commands all the required definitions are employed,
%    unless the test |\@ifdefinable| has failed.  Therefore, we
%    end the second argument of this macro and with it the definition
%    of |\@xnthm|.
%    \begin{macrocode}
   }}
%    \end{macrocode}
% \end{macro}
%
% \begin{macro}{\@ynthm}
%    The definition of |\@ynthm| is completely analogous. In
%    this case the new counter that is defined is not reset within
%    another counter; thus the definition of |\the...|
%    is simplified:
%    \begin{macrocode}
\gdef\@ynthm#1#2{\expandafter\@ifdefinable\csname #1\endcsname
   {\@definecounter{#1}%
    \expandafter\xdef\csname the#1\endcsname{\@thmcounter{#1}}%
%    \end{macrocode}
%    The rest of the definition corresponds literally to that of
%    |\@xnthm|:
%    \begin{macrocode}
    \def\@tempa{\global\@namedef{#1}}\expandafter \@tempa
     \expandafter{\csname th@\the \theorem@style \expandafter
     \endcsname \the\theorem@bodyfont \@thm{#1}{#2}}%
    \global \expandafter \let \csname end#1\endcsname \@endtheorem}}
%    \end{macrocode}
% \end{macro}
%
% \begin{macro}{\@othm}
%    The definition of |\@othm| does not contain anything new.
%    \begin{macrocode}
\gdef\@othm#1[#2]#3{%
%    \end{macrocode}
% \changes{v2.2b}{95/10/07}{Build in check for counter defined}
%    We do not define a new counter but instead use one that has
%    already been defined.  Thus the only definition we need is that
%    of this pseudo-counter (i.e.\ |\the|\meta{env.\ name}).
%    First we check if |#2| corresponds to a known counter name.
%    \begin{macrocode}
  \expandafter\ifx\csname c@#2\endcsname\relax
   \@nocounterr{#2}%
  \else
   \expandafter\@ifdefinable\csname #1\endcsname
   {\expandafter \xdef \csname the#1\endcsname
     {\expandafter \noexpand \csname the#2\endcsname}%
%    \end{macrocode}
%    All other parts of the definition can be adopted from
%    |\@xnthm|. We have to remember, though, that in this case
%    the name of the current counter and the theorem label have moved
%    to the second and third arguments.
%    \begin{macrocode}
    \def\@tempa{\global\@namedef{#1}}\expandafter \@tempa
     \expandafter{\csname th@\the \theorem@style \expandafter
     \endcsname \the\theorem@bodyfont \@thm{#2}{#3}}%
    \global \expandafter \let \csname end#1\endcsname \@endtheorem}%
  \fi}
%    \end{macrocode}
% \end{macro}
%
% \subsection{Macros that are employed in a {\envfont theorem}
%    environment}
%
% \begin{macro}{\@thm}
%    The macro |\@thm| has to increase the current counter.  Then,
%    depending on whether the environment has (or does not have) an
%    optional argument, it has to branch into either |\@begintheorem|
%    or |\@opargtheorem|.
%    \begin{macrocode}
\gdef\@thm#1#2{\refstepcounter{#1}%
%    \end{macrocode}
%    Now we start a {\envfont trivlist} environment, and give
%    |\@topsep| and |\@topsepadd| the values of the \textsf{skip}
%    registers |\theorempreskipamount| and |\theorempostskipamount|.
%    The value in |\@topsep| is the vertical space that is inserted by
%    the first (and only) |\item| in our |\trivlist| whilst
%    |\@topsepadd| is inserted by |\@endparenv| at the end of that
%    {\envfont trivlist} environment.  By using these registers, we
%    obtain the desired space around a {\envfont theorem} environment.
%    \begin{macrocode}
   \trivlist
   \@topsep \theorempreskipamount               % used by first \item
   \@topsepadd \theorempostskipamount           % used by \@endparenv
%    \end{macrocode}
%    Now we have to test whether an optional argument has been given.
% \changes{v2.0d}{89/07/18}{Forgotten percent added.}
%    \begin{macrocode}
   \@ifnextchar [%
%    \end{macrocode}
%    If there is an optional argument, we will call |\@ythm|, and
%    move the arguments read back into the input stream.
%    \begin{macrocode}
   {\@ythm{#1}{#2}}%
%    \end{macrocode}
%    If not, we call |\@begintheorem|. Its first argument is the
%    name of the theorem set (hence the second argument of
%    |\@thm|). Its second argument is the macro that produces the
%    current number.
%    \begin{macrocode}
   {\@begintheorem{#2}{\csname the#1\endcsname}\ignorespaces}}
%    \end{macrocode}
% \end{macro}
%
% \begin{macro}{\@xthm}
% \begin{macro}{\@ythm}
%    Both these macros were originally called by |\@thm|.  We do
%    not need |\@xthm| anymore, hence we reset it to
%    |\relax|. The definition of |\@ythm| has not changed at
%    all from its definition in \LaTeX. In order to make the macros
%    easier to understand, we will
%    nevertheless present it (commented out).
%    \begin{macrocode}
\global\let\@xthm\relax
% \def\@ythm#1#2[#3]{\@opargbegintheorem{#2}{\csname
%           the#1\endcsname}{#3}\ignorespaces}
%</package>
%    \end{macrocode}
%    The primitive |\ignorespaces| in |\@ythm| and
%    |\@thm| is needed to remove the spaces between the
%    |\begin{...}| and the actual text.
% \end{macro}
% \end{macro}
%
%
% \subsection{Definition of the theorem substyles}
%
% As already pointed out, the theorem substyles, defined below, are
% only loaded when necessary.  Note that all these substyles, except
% \texttt{plain}, have |\slshape| as the default body font.
%
% \subsubsection{The \texttt{plain} style}
%
% \label{dok:th@plain} As the following macros use |@|, we have
% to locally set the |\catcode| of this symbol to ``letter''.
% This happens within a group, so that we do not have to worry about
% which |\catcode| that symbol had before.
%    \begin{macrocode}
%<*thp>
\begingroup \makeatletter
%    \end{macrocode}
% Since we are now within a group, we must make all definitions
% globally.  First we make sure that \texttt{theorem.sty} is loaded.
% This will allow us to use this file as a document style option
% without having to call \texttt{theorem} itself as an option.  At the
% same time, we assure that at least version 2 is loaded, since
% |\theorem@style| was not defined in earlier versions.
%    \begin{macrocode}
\@ifundefined{theorem@style}{\input{theorem.sty}}{}
\ProvidesFile{thp.sty}
             [\FMithmInfo]
%    \end{macrocode}
%
%
% \begin{macro}{\th@plain}
% \begin{macro}{\@begintheorem}
% \begin{macro}{\@opargbegintheorem}
%    |\theoremstyle{plain}| corresponds to the original
%    definition, except that the distances to the surrounding text are
%    determined by the parameters |\theorempreskipamount| and
%    |\theorempostskipamount|.
%    First we set the default body font.
%    \begin{macrocode}
\gdef\th@plain{\normalfont\itshape
%    \end{macrocode}
%    Then we define |\@begintheorem| and
%    |\@opargbegintheorem|. These two macros define how the
%    header of a theorem is typeset.  |\@opargbegintheorem| will
%    be called if a \textsf{theorem} environment with an optional
%    argument is encountered; otherwise, the header is constructed by
%    calling |\@begintheorem|.  If one of these macros is
%    executed, we are within a \textsf{trivlist} environment started by
%    |\@thm|.  So the theorem header is produced with an
%    |\item| command.
%
%    Instead of specifying the header font directly, all standard
%    theorem styles use the |\theorem@headerfont| macro to allow
%    customization.  The extra space (|\labelsep|) is necessary
%    because of problems in the \textsf{trivlist} environment.
%    \begin{macrocode}
  \def\@begintheorem##1##2{%
        \item[\hskip\labelsep \theorem@headerfont ##1\ ##2]}%
%    \end{macrocode}
%    The definition of |\@opargbegintheorem| is completely
%    analogous. The only difference is the fact that there exists a
%    third argument (which is the optional parameter of the
%    environment and contains additional information about the
%    theorem).  Customarily we enclose it in parentheses.
%    \begin{macrocode}
\def\@opargbegintheorem##1##2##3{%
   \item[\hskip\labelsep \theorem@headerfont ##1\ ##2\ (##3)]}}
%    \end{macrocode}
% \end{macro}
% \end{macro}
% \end{macro}
%
% \noindent
% We conclude with an |\endgroup| to restore the
% |\catcode| of |@|.
%    \begin{macrocode}
\endgroup
%</thp>
%    \end{macrocode}
%
%
% \subsubsection{The \texttt{break} style}
%
% This style option is stored in the file |thb.sty|.
% For the next two lines see the documentation for |\th@plain| on
% page~\pageref{dok:th@plain}.
%    \begin{macrocode}
%<*thb>
\begingroup \makeatletter
\@ifundefined{theorem@style}{\input{theorem.sty}}{}
\ProvidesFile{thb.sty}
             [\FMithmInfo]
%    \end{macrocode}
%
% \begin{macro}{\th@break}
%    |\theoremstyle{break}| produces a line break after the name
%    of the theorem.  The font is |\slshape|.  Hence, we define
%    |\th@break| as follows:
%    \begin{macrocode}
\gdef\th@break{\normalfont\slshape
  \def\@begintheorem##1##2{\item[%
%    \end{macrocode}
%    We run into the following problem: it is not possible to create
%    the header with |\item[|\meta{title}|]| and then start
%    a new line by, for example, |\mbox{}\\|. Such a definition
%    will fail whenever a list environment follows immediately.  With
%    the above construction, the |\mbox{}| causes the switch
%    |@inlabel| (cf.\ definition of |\list| and
%    |\trivlist| in \cite{latex:source}) to be set to \texttt{false}
%    and so the following list will insert additional vertical
%    space (|\topskip|). This is quite annoying. Therefore, we
%    create the line break within the |\item|. In order to ensure
%    that the text will begin at the proper position in the following
%    line, we simply pretend that the label does not take any
%    room.\footnote{This will lead to problems whenever very high
%    symbols occurring in the line tower into the
%    heading. So, something else has to be done here sometime.}
%    \begin{macrocode}
       \rlap{\vbox{\hbox{\hskip \labelsep\theorem@headerfont ##1\ ##2}%
                   \hbox{\strut}}}]}%
%    \end{macrocode}
%    Again, the definition of |\@opargbegintheorem| is
%    completely analogous.
%    \begin{macrocode}
\def\@opargbegintheorem##1##2##3{%
  \item[\rlap{\vbox{\hbox{\hskip \labelsep \theorem@headerfont
                     ##1\ ##2\ (##3)}%
                    \hbox{\strut}}}]}}
\endgroup
%</thb>
%    \end{macrocode}
% \end{macro}
%
%
% \subsubsection{The \texttt{changebreak} style}
%
%    \begin{macrocode}
% This style option is stored in the file |thcb.sty|.
%    \begin{macrocode}
%<*thcb>
\begingroup \makeatletter
\@ifundefined{theorem@style}{\input{theorem.sty}}{}
\ProvidesFile{thcb.sty}
             [\FMithmInfo]
%    \end{macrocode}
%
% \begin{macro}{\th@changebreak}
%    The \texttt{change-break} theorem style is like \texttt{break}
%    but with interchange of theorem name and theorem number.  Thus we
%    define |\th@changebreak| as follows:
%    \begin{macrocode}
\gdef\th@changebreak{\normalfont\slshape
  \def\@begintheorem##1##2{\item
       [\rlap{\vbox{\hbox{\hskip\labelsep \theorem@headerfont ##2\ ##1}%
                    \hbox{\strut}}}]}%
\def\@opargbegintheorem##1##2##3{%
  \item[\rlap{\vbox{\hbox{\hskip\labelsep \theorem@headerfont
                     ##2\ ##1\ (##3)}%
                    \hbox{\strut}}}]}}
\endgroup
%</thcb>
%    \end{macrocode}
% \end{macro}
%
%
% \subsubsection{The \texttt{change} style}
%
% This style option is stored in the file |thc.sty|.
%    \begin{macrocode}
%<*thc>
\begingroup \makeatletter
\@ifundefined{theorem@style}{\input{theorem.sty}}{}
\ProvidesFile{thc.sty}
             [\FMithmInfo]
%    \end{macrocode}
%
% \begin{macro}{\th@change}
%    The \texttt{change} theorem style corresponds to the
%    \texttt{change break} style without a linebreak after the header.
%    To say it in another way, it's the same as the \texttt{plain}
%    style but with number and name interchanged and |\slshape| as the
%    default font.
%    \begin{macrocode}
\gdef\th@change{\normalfont\slshape
  \def\@begintheorem##1##2{\item
       [\hskip\labelsep \theorem@headerfont ##2\ ##1]}%
\def\@opargbegintheorem##1##2##3{%
  \item[\hskip\labelsep \theorem@headerfont ##2\ ##1\ (##3)]}}
\endgroup
%</thc>
%    \end{macrocode}
% \end{macro}
%
%
% \subsubsection{The \texttt{marginbreak} style}
%
% This style option is the one used most often at Mainz.  It is saved
% in the file |thmb.sty|.
%    \begin{macrocode}
%<*thmb>
\begingroup \makeatletter
\@ifundefined{theorem@style}{\input{theorem.sty}}{}
\ProvidesFile{thmb.sty}
             [\FMithmInfo]
%    \end{macrocode}
%
% \begin{macro}{\th@marginbreak}
%    The \texttt{margin break} style is nearly the same as the
%    \texttt{change break} style. The only difference is the placement
%    of the theorem number.  We use |\llap| to place it in the left
%    margin.
%
%    In this style |\labelsep| denotes the separation between the
%    number and the text.
%    \begin{macrocode}
\gdef\th@marginbreak{\normalfont\slshape
  \def\@begintheorem##1##2{\item
       [\rlap{\vbox{\theorem@headerfont
                    \hbox{\llap{##2}\hskip\labelsep ##1}%
                    \hbox{\strut}}}]}%
\def\@opargbegintheorem##1##2##3{%
  \item[\rlap{\vbox{\theorem@headerfont
                    \hbox{\llap{##2}\hskip\labelsep ##1\ (##3)}%
                    \hbox{\strut}}}]}}
\endgroup
%</thmb>
%    \end{macrocode}
% \changes{v2.1c}{90/02/17}{`endgroup outside macrocode env.}
% \end{macro}
%
%
%
% \subsubsection{The \texttt{margin} style}
%
% This style option is stored in the file |thm.sty|.
%    \begin{macrocode}
%<*thm>
\begingroup \makeatletter
\@ifundefined{theorem@style}{\input{theorem.sty}}{}
\ProvidesFile{thm.sty}
             [\FMithmInfo]
%    \end{macrocode}
%
% \begin{macro}{\th@margin}
%    Again this is only a variant of the theorem styles described
%    above without any new ideas.
%    \begin{macrocode}
\gdef\th@margin{\normalfont\slshape
  \def\@begintheorem##1##2{\item
       [\theorem@headerfont \llap{##2}\hskip\labelsep ##1]}%
\def\@opargbegintheorem##1##2##3{%
  \item[\theorem@headerfont \llap{##2}\hskip\labelsep ##1\ (##3)]}}
\endgroup
%</thm>
%    \end{macrocode}
% \end{macro}
%
%
%
%
% \subsection{Final Definitions}
%
% \begin{macro}{\theorempreskipamount}
% \begin{macro}{\theorempostskipamount}
%    The \textsf{skip} parameters that regulate the vertical empty space
%    before and after the {\envfont theorem} environment have
%    to be allocated as well.
% \changes{v2.0f}{89/07/20}{Unnecessary `global' removed.}
%    \begin{macrocode}
%<*package>
\newskip\theorempreskipamount
\newskip\theorempostskipamount
%    \end{macrocode}
%    Since we have used the same values for all theorem sets, we now can
%    assign them.
%    \begin{macrocode}
\global\setlength\theorempreskipamount{12pt plus 5pt minus 3pt}
\global\setlength\theorempostskipamount{8pt plus 3pt minus 1.5pt}
%    \end{macrocode}
% \end{macro}
% \end{macro}
%
%
% \begin{macro}{\@endtheorem}
%    The same holds for the macro |\@endtheorem|, which ends a
%    {\envfont theorem} environment. Since it is the same for all
%    theorem sets, it is removed from the macros |\th@|\meta{style}.
%    It simply ends the {\envfont trivlist} environment, which was
%    begun in |\@thm|.
%    \begin{macrocode}
\global\let\@endtheorem=\endtrivlist
%    \end{macrocode}
% \end{macro}
%
%
%
% \begin{macro}{\@preamblecmds}
%    All macros defined above are to be used only in the preamble.
%    Therefore, we insert them in |\@preamblecmds| which will disable
%    them at begin document. This is done by the internal \LaTeXe{}
%    command |\@onlypreamble|.
% \changes{v2.0g}{89/09/19}{`theoremheaderfont and `theorembodyfont
%    added.}
% \changes{v2.2a}{94/02/03}{Use \LaTeXe{} interface}
%    \begin{macrocode}
\@onlypreamble\@xnthm
\@onlypreamble\@ynthm
\@onlypreamble\@othm
\@onlypreamble\newtheorem
\@onlypreamble\theoremstyle
\@onlypreamble\theorembodyfont
\@onlypreamble\theoremheaderfont
%    \end{macrocode}
% \end{macro}
%
% Finally we declare the \texttt{plain} theorem style to be the default.
% \changes{v2.1b}{90/06/04}{Fixed missing `item by declaring default
%    style.}
%    \begin{macrocode}
\theoremstyle{plain}
%</package>
%    \end{macrocode}
%
% \Finale

\endinput
}{}
\ProvidesFile{thc.sty}
             [\FMithmInfo]
%    \end{macrocode}
%
% \begin{macro}{\th@change}
%    The \texttt{change} theorem style corresponds to the
%    \texttt{change break} style without a linebreak after the header.
%    To say it in another way, it's the same as the \texttt{plain}
%    style but with number and name interchanged and |\slshape| as the
%    default font.
%    \begin{macrocode}
\gdef\th@change{\normalfont\slshape
  \def\@begintheorem##1##2{\item
       [\hskip\labelsep \theorem@headerfont ##2\ ##1]}%
\def\@opargbegintheorem##1##2##3{%
  \item[\hskip\labelsep \theorem@headerfont ##2\ ##1\ (##3)]}}
\endgroup
%</thc>
%    \end{macrocode}
% \end{macro}
%
%
% \subsubsection{The \texttt{marginbreak} style}
%
% This style option is the one used most often at Mainz.  It is saved
% in the file |thmb.sty|.
%    \begin{macrocode}
%<*thmb>
\begingroup \makeatletter
\@ifundefined{theorem@style}{% \iffalse meta-comment
%
% Copyright (C) 1993-2024
%
% The LaTeX Project and any individual authors listed elsewhere
% in this file.
%
% This file is part of the Standard LaTeX `Tools Bundle'.
% -------------------------------------------------------
%
% It may be distributed and/or modified under the
% conditions of the LaTeX Project Public License, either version 1.3c
% of this license or (at your option) any later version.
% The latest version of this license is in
%    https://www.latex-project.org/lppl.txt
% and version 1.3c or later is part of all distributions of LaTeX
% version 2005/12/01 or later.
%
% The list of all files belonging to the LaTeX `Tools Bundle' is
% given in the file `manifest.txt'.
%
% \fi
%
%%
%
%
\def\FMithmInfo{2023/07/05 v2.2c Theorem extension package (FMi)}
%
% \ProvidesFile{theorem.dtx}[\FMithmInfo]
%\iffalse   % this is a METACOMMENT !
%%
%% Package `theorem' to use with LaTeX2e
%% Copyright (C) 1989-1995 Frank Mittelbach, all rights reserved.
%<+package>\@ifundefined{theorem@style}{}{\endinput}
%<+package>\NeedsTeXFormat{LaTeX2e}[1995/06/01]
%<+package>\ProvidesPackage{theorem}
%<+package>         [\FMithmInfo]
%
% \fi
%
% \changes{v1.0}{87/05/07}{First draft.}
% \changes{v1.0a}{87/05/07}{`thistheoremstyle' added.}
% \changes{v1.1a}{87/07/29}{New concept: redefine the internal
%                          \protect\LaTeX{} commands `@xnthm' etc.}
% \changes{v1.1b}{87/07/31}{`theoremfont' used to specify font.}
% \changes{v1.2a}{88/01/08}{`@endtheorem' now globally defined.}
% \changes{v1.2b}{88/01/11}{`newtheorem' only allowed in the preamble.}
% \changes{v1.3a}{88/02/14}{`theoremfont' now a toks register.}
%
% \changes{v2.0a}{88/11/16}{New implementation and new documentation.}
% \changes{v2.0a}{88/11/16}{No layout changes but internal modifications
%                           old substyles won't work.}
% \changes{v2.0b}{89/01/30}{`global' used for all definitions.}
% \changes{v2.0c}{89/06/15}{Documentation translated to English.
%                            Thanks to LLL and Nina}
% \changes{v2.1a}{90/02/15}{Support for new font selection.}
% \changes{v2.2a}{94/02/03}{Cleanup for LaTeX2e}
%
% \DoNotIndex{\@endparenv,\@ifdefinable,\@ifnextchar,\@ifundefined}
% \DoNotIndex{\@namedef,\@nameuse,\@spaces,\@tempa,\@warning}
% \DoNotIndex{\ }
% \DoNotIndex{\begingroup,\bfseries}
% \DoNotIndex{\csname}
% \DoNotIndex{\def,\do,\docdate}
% \DoNotIndex{\endcsname,\endgroup,\endinput,\endtrivlist,\expandafter}
% \DoNotIndex{\filedate,\fileversion}
% \DoNotIndex{\gdef,\global}
% \DoNotIndex{\hbox,\hskip}
% \DoNotIndex{\ignorespaces,\input,\itshape,\item}
% \DoNotIndex{\labelsep,\let,\llap}
% \DoNotIndex{\makeatletter}
% \DoNotIndex{\noexpand}
% \DoNotIndex{\refstepcounter,\relax,\rlap}
% \DoNotIndex{\setlength,\slshape,\space,\string,\strut}
% \DoNotIndex{\the,\trivlist,\typeout}
% \DoNotIndex{\vbox}
% \DoNotIndex{\xdef}
%
% \def\deflabel#1{\ttfamily #1\hfill}
% \def\deflist#1{\begin{list}{}{\settowidth\labelwidth{\ttfamily #1}%
%                               \setlength\leftmargin\labelwidth
%                               \addtolength\leftmargin\labelsep
%                               \let\makelabel\deflabel}}
% \def\enddeflist{\end{list}}
% \def\envfont{\normalfont\ttfamily}
%
% \MakeShortVerb{\|}
%
% \newcommand{\TUB}{{\sl TUGboat\/}}
%
% \GetFileInfo{theorem.dtx}
% \title{An Extension of the
%        \LaTeX{} theorem environment\thanks{This file
%        has version number \fileversion, last
%        revised \filedate.}}
% \author{Frank Mittelbach}
%
% \MaintainedByLaTeXTeam{tools}
% \maketitle
%
% \begin{abstract}
%    The macros described in this paper yield an extension of the
%    \LaTeX{} theorem mechanism. It is designed to satisfy the
%    different requirements of various journals. Thus, the
%    layout of the ``theorems'' can be manipulated by determining a
%    ``style''. This article describes not only the use, but
%    also the definition, of the necessary macros.
% \end{abstract}
%
% \section*{Preface to version 2.2}
%
% For \LaTeXe{} this package did not need any fundamental changes. I
% only modified the messages generated so that theorem layout styles
% will show up with the |\listfiles| command and cleaned the section
% on the New Font Selection Scheme since this is now included in
% \LaTeX.
%
% \section*{Preface to version 2.1}
%
% This version is identical to 2.0g described in \TUB\ 10\#3 except
% for some internal defaults which are now set depending on the used
% font selection scheme.
%
% This was done to avoid unpleasant surprises if the new font selection
% scheme is in force. For further details see
% section~\ref{sec:newfonts} and \cite{bk:GMS94}.
%
% \section{Introduction}
%
% For our purposes here, ``theorems'' are labelled enunciations,
% often set off from the main text by extra space and a font change.
% Theorems, corollaries, conjectures, definitions, and remarks
% are all instances of ``theorems''.  The ``header'' of these structures
% is composed of a label (such as \textsc{Theorem} or \textsc{Remark})
% and a number which serializes an item in the sequence of items
% with the same label.
%
% Shortly after the introduction of \LaTeX{} at the Fachbereich
% Mathematik in Mainz, the desire to manipulate the layout of
% ``theorems'' arose.  In Mainz, the following two conventions came
% into general use:
% \begin{enumerate}
%   \item The number of the theorem is shown in the margin.
%   \item There is a line break at
%         the end of the theorem header.
% \end{enumerate}
%
% Additionally, some journals require different formats which depend on
% the ``sort of theorem'': e.g.\ often remarks and definitions are set
% in |\upshape|, while |\itshape| is employed for main theorems.
%
% Confronted with these requirements, a theorem environment was
% developed in Mainz which allows separate determination of the
% layout of the ``theorems sets'', comparable to |\pagestyle|.
%
%
% \section{The user interface}
% \subsection{Defining new theorem sets}
%
% \DescribeMacro\newtheorem
% As in the original \LaTeX{} version, the command |\newtheorem|
% defines a new ``theorem set'' or ``theorem-like structure''.
% Two required arguments name the new environment and give the text
% to be typeset with each instance of the new ``set'', while
% an optional argument determines how
% the ``set'' is enumerated:
% \begin{description}
%    \item[\ttfamily \bslash newtheorem\{foo\}\{bar\}]
%       The theorem set {\envfont foo} (whose name is \texttt{bar})
%       uses its own counter.
%    \item[\ttfamily \bslash newtheorem\{foo2\}{[foo]}\{bar2\}]
%       The theorem set {\envfont foo2} (printed name \texttt{bar2})
%       uses the same counter as the theorem set \texttt{foo}.
%    \item[\ttfamily \bslash newtheorem\{foo3\}\{bar3\}{[section]}]
%       The theorem set {\envfont foo3} (printed name \texttt{bar3}) is
%       enumerated within the counter \texttt{section}, i.e.\ with every
%       new |\section| the enumeration begins again with 1, and
%       the enumeration is composed from the section-number and the
%       theorem counter itself.
% \end{description}
%
% \DescribeMacro\theoremstyle
% Additionally, the command |\theoremstyle| can define the layout
% of various, or all, theorem sets.  It should be noted that any
% theorem set defined by |\newtheorem| is typeset in the
% |\theoremstyle| that is current at the time of the definition.
% Thus, the following
% \begin{verbatim}
%    \theoremstyle{break}        \newtheorem{Cor}{Corollary}
%    \theoremstyle{plain}        \newtheorem{Exa}{Example}[section]
%\end{verbatim}                             ^^A No blank at line start
% leads to the result that the set {\envfont Cor} is formatted in the
% style \texttt{break}, while the set {\envfont Exa} and all the
% following ones are formatted in the style \texttt{plain}, unless
% another |\theoremstyle| follows.  Since the definitions installed by
% |\newtheorem| are global, one also can limit |\theoremstyle| locally
% by grouping braces.
%
% \DescribeMacro\theorembodyfont
% The choice of the font for the theorem body is completely
% independent of the chosen |\theoremstyle|; this has proven to
% be very advantageous.  For example,
% \begin{verbatim}
%   {\theorembodyfont{\upshape}          \newtheorem{Rem}{Remark}}
%\end{verbatim}
% defines a theorem set {\envfont Rem}, which will be set in
% |\upshape| in the current layout (which in our example is
% \texttt{plain}).  As with |\theoremstyle|, the |\theorembodyfont|
% chosen is that current at the time of |\newtheorem|.  If
% |\theorembodyfont| is not specified or one defines
% |\theorembodyfont{}|, then the font used will be that defined by the
% |\theoremstyle|.
%
% \DescribeMacro\theoremheaderfont It is also possible to customize
% the font used for the theorem headers.  This is, however, a global
% declaration, and therefore there should be at most one
% |\theoremheaderfont| declaration in the preamble.\footnote{If it is
% actually necessary to have different header fonts, one has to define
% new theorem styles (substituting the desired font) or specify the
% information directly in the \texttt{\bslash newtheorem} declaration
% (the unclean variant).}
%
% \DescribeMacro\theorempreskipamount
% \DescribeMacro\theorempostskipamount Two additional parameters
% affect the vertical space around the theorem environments:
% |\theorempreskipamount| and |\theorempostskipamount| define,
% respectively, the spacing before and after such an environment.
% These parameters apply for all theorem sets and can be manipulated
% with the ordinary length macros.  They are rubber lengths
% (`\textsf{skips}'), and therefore can contain \texttt{plus} and
% \texttt{minus} parts.
%
% Since the definition of theorem sets should---most sensibly---be
% placed in the preamble, we only allow installation there.
% It is therefore possible to release the memory used here after
% |\begin{document}|, in order to make room for other
% applications.
%
%
% \subsection{Existing theorem styles}
%
% The following theorem styles exist to date:
% \begin{deflist}{marginbreak}
%    \item[plain]
%       This theorem style emulates the original \LaTeX{} definition,
%       except that additionally the parameters
%       |\theorem...skipamount| are used.
%    \item[break]
%       In this style, the theorem header is followed by a line
%       break.
%    \item[marginbreak]
%       The theorem number is set in the margin, and there is a line
%       break as in \texttt{break}.
%    \item[changebreak]
%       Like \texttt{break}, but with header number and text
%       interchanged.
%    \item[change]
%       Header number and text are interchanged, without a line break.
%    \item[margin]
%       The number is set in the left margin, without a line break.
% \end{deflist}
% All styles (except \texttt{plain}) select |\slshape| as the default
% |\theorembodyfont|.
%
%
% \subsection{Examples}
%
% Given the above theorem sets {\envfont Cor}, {\envfont Exa}
% and {\envfont Rem},
% suppose that the preamble also contains the declarations:
% \begin{verbatim}
%    \theoremstyle{marginbreak}   \newtheorem{Lem}[Cor]{Lemma}
%    \theoremstyle{change}
%    \theorembodyfont{\itshape}        \newtheorem{Def}[Cor]{Definition}
%
%    \theoremheaderfont{\scshape}
%\end{verbatim}
% Then the following are some typical examples of the typeset output
% resulting from their use.
%
% \begin{Cor}
%    This is a sentence typeset in the theorem environment {\envfont
%    Cor}.
% \end{Cor}
%
% \begin{Exa}
%    This is a sentence typeset in the theorem environment {\envfont
%    Exa}.
% \end{Exa}
%
% \begin{Rem}
%    This is a sentence typeset in the theorem environment {\envfont
%    Rem}.
% \end{Rem}
%
% \begin{Lem}[Ben User]
%    This is a sentence typeset in the theorem environment {\envfont
%    Lem}.
% \end{Lem}
%
% \begin{Def}[Very impressive Definition]
%    This is a sentence typeset in the theorem environment {\envfont
%    Def}.
% \end{Def}
%
% The last two examples show the effect of the optional argument to
% a theorem environment (it is the text typeset in parentheses).
%
%
% \section{Special Considerations}
% \label{sec:newfonts}
%
% Theoremheader and body are implemented as a unit. This means that
% the |\theoremheaderfont|  will inherit characteristics of the
% |\theorembodyfont| in \LaTeXe. Thus, if for example
% |\theorembodyfont| is |\itshape| and |\theoremheaderfont|
% is |\bfseries| the font selected for the header will have the
% characteristics `bold extended italic'. If this is not desired one
% should set the |\theoremheaderfont| to something like
% \begin{verbatim}
%      \theoremheaderfont{\normalfont\bfseries}
%\end{verbatim}
% i.e.\ supplying all necessary font information explicitly.
%
%
%
% \section{Acknowledgements}
%
% The publication of this set of macros was only possible with the
% help of Christina Busse (translating the manuscript into English),
% Joachim Pense (playing the r\^ole of typist), Chris Rowley
% (looking everything over) and many others providing useful
% suggestions.
%
% \MaybeStop{%
%    \begin{thebibliography}{9}
%       \bibitem{bk:GMS94} \textsc{M.~Goossens}, \textsc{F.~Mittelbach}
%         and \textsc{A.~Samarin}.
%         \newblock The \LaTeX{} Companion.
%         \newblock
%         Addison-Wesley, Reading, Massachusetts, 1994.
%       \bibitem{latex:source} \textsc{Lamport, Leslie.} \newblock
%                               \texttt{latex.tex}, version 2.09, date
%                               Feb. 1990.
%    \end{thebibliography}
%    \PrintIndex
%    \PrintChanges}
%
%
% \section{The documentation driver file}
%
% The next bit of code contains the documentation driver file for
% \TeX{}, i.e., the file that will produce the documentation you are
% currently reading. It will be extracted from this file by the
% \texttt{docstrip} program. Since it is the first code in the file
% one can alternatively process this file directly with \LaTeXe{} to
% obtain the documentation.
%
%    \begin{macrocode}
%<*driver>
\documentclass{ltxdoc}

\usepackage{theorem}

%<+driver>% The next few lines define theorem sets which are used
%<+driver>% in the example section of the documentation.

\theoremstyle{break}       \newtheorem{Cor}{Corollary}
\theoremstyle{plain}       \newtheorem{Exa}{Example}[section]
{\theorembodyfont{\upshape}\newtheorem{Rem}{Remark}}
\theoremstyle{marginbreak} \newtheorem{Lem}[Cor]{Lemma}
\theoremstyle{change}
\theorembodyfont{\itshape} \newtheorem{Def}[Cor]{Definition}

\theoremheaderfont{\scshape}

\RecordChanges

\begin{document}
   \DocInput{theorem.dtx}
\end{document}
%</driver>
%    \end{macrocode}
%
% \section{Definition of the Macros}
%
% If the file has been loaded before, we abort immediately. If not the
% package announces itself (this is actually done at the very top of
% the file---the way it is done isn't good style so don't copy it).
% \changes{v2.0e}{89/07/19}{Spaces removed from `typeout'.}
% \changes{v2.2b}{95/11/19}{Announce moved to top of file}
%    \begin{macrocode}
%<*package>
%\@ifundefined{theorem@style}{}{\endinput}
%\def\FMithmInfo{1995/11/19 v2.2b Theorem extension package (FMi)}
%\ProvidesPackage{theorem}[\FMithmInfo]
%    \end{macrocode}
%
%
% \subsection{Definition of theorem styles and fonts}
%
% All the definitions in this file are done globally to allow
% inputting this file inside a group.
%
% \begin{macro}{\theoremstyle}
%    Before a theorem style can be installed, the chosen style must be
%    known. For that reason, we must test to see that
%    |\th@|\meta{style} is known or, more precisely, that it is
%    different from |\relax|. If the style is not known then
%    |\th@plain| is used.
%    \begin{macrocode}
\gdef\theoremstyle#1{%
   \@ifundefined{th@#1}{\@warning
          {Unknown theoremstyle `#1'. Using `plain'}%
          \theorem@style{plain}}%
%    \end{macrocode}
%    We save the theorem style to be used in the token register
%    |\theorem@style|.
%    \begin{macrocode}
      {\theorem@style{#1}}%
%    \end{macrocode}
%    Now we ``evaluate'' the theorem style: this means, we call the
%    macro |\th@|\meta{style} which will activate the relevant
%    definitions which are contained in a separate file.  This is done
%    in a group to suppress changes to the current font.  This could
%    otherwise pose problems together with the new font selection
%    scheme\footnote{When I printed the original article using the new
%    font selection scheme I ended with a document with slanted
%    typefaces (text headings and all) simply because one of the
%    theorem styles used \texttt{\bslash sl} at toplevel.} if the
%    |\th@|\meta{style} is evaluated a second time.
% \changes{v2.1a}{90/02/15}{Group added around `th@..}
% \changes{v2.1a}{90/02/15}{Reseting of `..@begintheorem removed.}
%    \begin{macrocode}
      \begingroup
        \csname th@\the\theorem@style \endcsname
      \endgroup}
%    \end{macrocode}
% \end{macro}
%
%
% \begin{macro}{\@begintheorem}
% \changes{v2.1a}{90/02/15}{Relaxed at toplevel.}
% \begin{macro}{\@opargbegintheorem}
% \changes{v2.1a}{90/02/15}{Relaxed at toplevel.}
%    We reset |\@begintheorem| and |\@opargbegintheorem| to
%    |\relax| since these commands are no longer necessary at
%    toplevel. This will save a few tokens.
%    \begin{macrocode}
\global\let\@begintheorem\relax
\global\let\@opargbegintheorem\relax
%    \end{macrocode}
% \end{macro}
% \end{macro}
%
% \begin{macro}{\theorem@style}
%    Obviously the \textsf{token} register used above has to be
%    allocated. To assure the utmost compatibility with the
%    original \LaTeX{} definition, we set the default theorem style to
%    \texttt{plain}, which implements the usual \LaTeX{} convention.
% \changes{v2.0f}{89/07/20}{Unnecessary `global' removed.}
%    \begin{macrocode}
\newtoks\theorem@style
\global\theorem@style{plain}
%    \end{macrocode}
% \end{macro}
%
% \begin{macro}{\theorembodyfont}
% \begin{macro}{\theorem@bodyfont}
%    If the |\theorembodyfont| is set by the user then it should not
%    interact with the default font set in the theorem style. When the
%    new font selection is in force this may happen if, for example,
%    the default is |\itshape| and the new |\theorembodyfont| is
%    |\sffamily|. So we add a |\reset@font| command in front of the
%    user definition.
% \changes{v2.0f}{89/07/20}{Unnecessary `global' removed.}
% \changes{v2.0g}{89/09/19}{`theoremfont to `theorembodyfont.}
% \changes{v2.1c}{92/06/22}{`reset@font added and logic changed.}
% \changes{v2.1d}{92/10/28}{Reset to empty if arg is empty}
%    \begin{macrocode}
\gdef\theorembodyfont#1{%
%    \end{macrocode}
%    We check if the argument supplied is empty and if so put nothing
%    into the |\theorem@bodyfont| token register to allow for
%    |\theorembodyfont{}| as a mean of using the default of the
%    current |\theoremstyle|.
%    \begin{macrocode}
   \def\@tempa{#1}%
   \ifx\@tempa\@empty
    \theorem@bodyfont{}%
   \else
    \theorem@bodyfont{\reset@font#1}%
   \fi
   }
\newtoks\theorem@bodyfont
\global\theorem@bodyfont{}
%    \end{macrocode}
% \end{macro}
% \end{macro}
%
%
% \begin{macro}{\theoremheaderfont}
% \changes{v2.0g}{89/09/19}{Macro added.}
%    The font for the theorem headers is handled differently because
%    this definition applies to all theorem styles.
%    \begin{macrocode}
\gdef\theoremheaderfont#1{\gdef\theorem@headerfont{#1}%
%    \end{macrocode}
%    After using the macro once it is redefined to produce an error
%    message.
%    \begin{macrocode}
       \gdef\theoremheaderfont##1{%
        \typeout{\string\theoremheaderfont\space should be used
                 only once.}}}
%    \end{macrocode}
% \end{macro}
%
% \begin{macro}{\theorem@headerfont}
% \changes{v2.0g}{89/09/19}{Macro added.}
%    To set the |\theorem@headerfont| default we first test if
%    the new fontselection scheme is in force.
% \changes{v2.1a}{90/02/15}{Support for new font selection.}
%    \begin{macrocode}
\ifx\upshape\undefined
%    \end{macrocode}
%    If not we define it to expand into |\bfseries|. We don't use
%    |\let| just in case a following style option redefines this
%    macro.
%    \begin{macrocode}
\gdef\theorem@headerfont{\bfseries}
%    \end{macrocode}
%    Otherwise we reset the current shape before calling |\bfseries|.
%    \begin{macrocode}
\else \gdef\theorem@headerfont{\normalfont\bfseries}\fi
%    \end{macrocode}
% \end{macro}
%
% \begin{macro}{\th@plain}
% \begin{macro}{\th@break}
% \begin{macro}{\th@marginbreak}
% \begin{macro}{\th@changebreak}
% \begin{macro}{\th@change}
% \begin{macro}{\th@margin}
%    The different styles are defined in macros such as
%    |\th@plain|.  Since memory space is precious in
%    ``non-Big-versions'', we have to avoid offering too many unused
%    definitions.  Therefore we define these styles in separate files
%    that can be loaded on demand.  Thus the commands themselves only
%    load these files. We use |\@input@|, a \LaTeXe{} internal command
%    that ensures that the file will be listed with |\listfiles|.
% \changes{v2.2a}{94/02/02}{Use `@input@ to load theorem layout files}
%    \begin{macrocode}
\gdef\th@plain{\@input@{thp.sty}}
\gdef\th@break{\@input@{thb.sty}}
\gdef\th@marginbreak{\@input@{thmb.sty}}
\gdef\th@changebreak{\@input@{thcb.sty}}
\gdef\th@change{\@input@{thc.sty}}
\gdef\th@margin{\@input@{thm.sty}}
%    \end{macrocode}
%    This list will be expanded when new styles become
%    available.  For testing,  just append new theorem substyles
%    as document options.
% \end{macro}
% \end{macro}
% \end{macro}
% \end{macro}
% \end{macro}
% \end{macro}
%
%
%
% \subsection{Definition of a new theorem set}
%
% As already pointed out, a new theorem environment can be defined in
% three different ways:
% \begin{verbatim}
%    \newtheorem{Lem}{Lemma}
%    \newtheorem{Lem}{Lemma}[section]
%    \newtheorem{Lem}[Theorem]{Lemma}
%\end{verbatim}
% The function of the macro |\newtheorem| is to recognize these cases
% and then to branch into one of the three macros |\@ynthm|, |\@xnthm|
% or |\@othm|.  This mechanism is adopted unchanged from
% \cite{latex:source}; the essential point here is that, for example,
% in the second case, the arguments \texttt{Lem}, \texttt{Lemma} and
% \texttt{section} are passed over to the macro |\@xnthm|.
%
% We inspect this case first because the others present fewer problems,
% and thus are easily derived from this one.
%
% \begin{macro}{\@xnthm}
%    For our example arguments, the macro |\@xnthm| must fulfill
%    the following:
%    \begin{itemize}
%       \item
%          Define a new \LaTeX-counter `Lem'
%       \item
%          reset this counter within a |\section|
%       \item
%          define the macro |\theLem|
%       \item
%          define the environment macros |\Lem| and
%          |\endLem| using the current
%          |\theoremstyle| and |\theorem@bodyfont|.
%    \end{itemize}
%    Obviously, all this should happen only if the first argument
%    of |\@xnthm| (i.e.\ \texttt{Lem} in our example) is chosen so
%    as not to conflict with any previously defined commands or
%    environments.  This test is performed by the \LaTeX{} macro
%    |\@ifdefinable|.
%    \begin{macrocode}
\gdef\@xnthm#1#2[#3]{\expandafter\@ifdefinable\csname #1\endcsname
%    \end{macrocode}
%    Therefore, the first argument of |\@ifdefinable| is the
%    expansion (in the example, |\Lem|) of
%    |\csname||#1\endcsname|.  The second argument is
%    executed only if the test has been completed successfully.
%    \begin{macrocode}
   {%
%    \end{macrocode}
%    Now we define the new counter. The names of the \LaTeX{}
%    macros employed should speak for themselves:
% \changes{v2.2b}{1995/10/07}{Use \cs{@newctr}}
% \changes{v2.2c}{1995/11/23}{Use \cs{@newctr} properly.}
%    \begin{macrocode}
    \@definecounter{#1}\@newctr{#1}[#3]%
%    \end{macrocode}
%    Using |\@newctr| will give a proper error message if the counter
%    in |#3| is not defined.
%    In defining `|\theLem|' we must generate the desired macro
%    name by use of |\expandafter| and |\csname|.
%    \begin{macrocode}
    \expandafter\xdef\csname the#1\endcsname
%    \end{macrocode}
%    An |\xdef| is used in order to make the definition global,
%    and to ensure that it contains the replacement texts of
%    |\@thmcountersep| and |\@thmcounter|.\footnote{These
%    two macros can be defined by the document style.  Their default
%    values produce a `\texttt{.}' as separation and an arabic
%    representation of the number.} However, not everything should be
%    expanded. For example, it saves space to use |\thesection|
%    instead of its---at times---lengthy expansion.
%    \begin{macrocode}
      {\expandafter \noexpand \csname the#3\endcsname
       \@thmcountersep \@thmcounter{#1}}%
%    \end{macrocode}
%    Thus with the defaults of \LaTeX{}, |\theLem| would be
%    replaced by the command sequence |\thesection.\arabic{Lem}|.
%
%    We will now look at the definition of the macro which is executed
%    at the beginning of the actual environment (in our example this
%    macro is |\Lem|). It should be noted that we use an
%    ``|\expandafter| trick'' to expand only certain parts of the
%    replacement text at the time of the definition.
%    \begin{macrocode}
    \def\@tempa{\global\@namedef{#1}}%
    \expandafter \@tempa \expandafter{%
%    \end{macrocode}
%    First, the macro that contains the current definitions of
%    |\@begintheorem| and |\@opargtheorem| should be called
%    up.  The name of this macro---as is already known---has the form
%    |\th@|\meta{theorem\ style}; therefore, it must be called by
%    \begin{macrocode}
      \csname th@\the \theorem@style
%    \end{macrocode}
%    In addition the default theorem font should be changeable, i.e.\
%    we have to insert the contents of |\theorem@bodyfont|. For
%    that reason, we expand even further, beyond |\endcsname|,
%    and thus insert the contents of the \textsf{token} register
%    |\theorem@bodyfont| in the replacement text.
%    \begin{macrocode}
            \expandafter \endcsname \the \theorem@bodyfont
%    \end{macrocode}
%    Now it is time to call the macro |\@thm| which takes over
%    the further processing.  It has two arguments: the current
%    counter name (in our example, \texttt{Lem}), and the text of the
%    label (in our example, \texttt{Lemma}).
%    \begin{macrocode}
     \@thm{#1}{#2}}%
%    \end{macrocode}
%    With this, the `sub-definition' is complete.  The macro
%    |\@endtheorem| ends a theorem environment and is, so far, nothing
%    but an |\endtrivlist|.  (Hence it is defined globally, and not
%    within the theorem styles.\footnote{This has to be changed as
%    soon as theorem styles that change \texttt{\bslash @endtheorem}
%    exist.  In such a case, all existing styles must be changed as
%    well since they will have to reset the macro.}) Therefore, we can
%    set it equivalent to the macro that ends the theorem set (in our
%    example, |\endLem|).  However, if some day theorem styles exist
%    that do change |\@endtheorem|, we would have to use the
%    commented-out line instead.
%    \begin{macrocode}
    \global \expandafter \let \csname end#1\endcsname \@endtheorem
%   \global\@namedef{end#1}{\@endtheorem}%
%    \end{macrocode}
%    With these commands all the required definitions are employed,
%    unless the test |\@ifdefinable| has failed.  Therefore, we
%    end the second argument of this macro and with it the definition
%    of |\@xnthm|.
%    \begin{macrocode}
   }}
%    \end{macrocode}
% \end{macro}
%
% \begin{macro}{\@ynthm}
%    The definition of |\@ynthm| is completely analogous. In
%    this case the new counter that is defined is not reset within
%    another counter; thus the definition of |\the...|
%    is simplified:
%    \begin{macrocode}
\gdef\@ynthm#1#2{\expandafter\@ifdefinable\csname #1\endcsname
   {\@definecounter{#1}%
    \expandafter\xdef\csname the#1\endcsname{\@thmcounter{#1}}%
%    \end{macrocode}
%    The rest of the definition corresponds literally to that of
%    |\@xnthm|:
%    \begin{macrocode}
    \def\@tempa{\global\@namedef{#1}}\expandafter \@tempa
     \expandafter{\csname th@\the \theorem@style \expandafter
     \endcsname \the\theorem@bodyfont \@thm{#1}{#2}}%
    \global \expandafter \let \csname end#1\endcsname \@endtheorem}}
%    \end{macrocode}
% \end{macro}
%
% \begin{macro}{\@othm}
%    The definition of |\@othm| does not contain anything new.
%    \begin{macrocode}
\gdef\@othm#1[#2]#3{%
%    \end{macrocode}
% \changes{v2.2b}{95/10/07}{Build in check for counter defined}
%    We do not define a new counter but instead use one that has
%    already been defined.  Thus the only definition we need is that
%    of this pseudo-counter (i.e.\ |\the|\meta{env.\ name}).
%    First we check if |#2| corresponds to a known counter name.
%    \begin{macrocode}
  \expandafter\ifx\csname c@#2\endcsname\relax
   \@nocounterr{#2}%
  \else
   \expandafter\@ifdefinable\csname #1\endcsname
   {\expandafter \xdef \csname the#1\endcsname
     {\expandafter \noexpand \csname the#2\endcsname}%
%    \end{macrocode}
%    All other parts of the definition can be adopted from
%    |\@xnthm|. We have to remember, though, that in this case
%    the name of the current counter and the theorem label have moved
%    to the second and third arguments.
%    \begin{macrocode}
    \def\@tempa{\global\@namedef{#1}}\expandafter \@tempa
     \expandafter{\csname th@\the \theorem@style \expandafter
     \endcsname \the\theorem@bodyfont \@thm{#2}{#3}}%
    \global \expandafter \let \csname end#1\endcsname \@endtheorem}%
  \fi}
%    \end{macrocode}
% \end{macro}
%
% \subsection{Macros that are employed in a {\envfont theorem}
%    environment}
%
% \begin{macro}{\@thm}
%    The macro |\@thm| has to increase the current counter.  Then,
%    depending on whether the environment has (or does not have) an
%    optional argument, it has to branch into either |\@begintheorem|
%    or |\@opargtheorem|.
%    \begin{macrocode}
\gdef\@thm#1#2{\refstepcounter{#1}%
%    \end{macrocode}
%    Now we start a {\envfont trivlist} environment, and give
%    |\@topsep| and |\@topsepadd| the values of the \textsf{skip}
%    registers |\theorempreskipamount| and |\theorempostskipamount|.
%    The value in |\@topsep| is the vertical space that is inserted by
%    the first (and only) |\item| in our |\trivlist| whilst
%    |\@topsepadd| is inserted by |\@endparenv| at the end of that
%    {\envfont trivlist} environment.  By using these registers, we
%    obtain the desired space around a {\envfont theorem} environment.
%    \begin{macrocode}
   \trivlist
   \@topsep \theorempreskipamount               % used by first \item
   \@topsepadd \theorempostskipamount           % used by \@endparenv
%    \end{macrocode}
%    Now we have to test whether an optional argument has been given.
% \changes{v2.0d}{89/07/18}{Forgotten percent added.}
%    \begin{macrocode}
   \@ifnextchar [%
%    \end{macrocode}
%    If there is an optional argument, we will call |\@ythm|, and
%    move the arguments read back into the input stream.
%    \begin{macrocode}
   {\@ythm{#1}{#2}}%
%    \end{macrocode}
%    If not, we call |\@begintheorem|. Its first argument is the
%    name of the theorem set (hence the second argument of
%    |\@thm|). Its second argument is the macro that produces the
%    current number.
%    \begin{macrocode}
   {\@begintheorem{#2}{\csname the#1\endcsname}\ignorespaces}}
%    \end{macrocode}
% \end{macro}
%
% \begin{macro}{\@xthm}
% \begin{macro}{\@ythm}
%    Both these macros were originally called by |\@thm|.  We do
%    not need |\@xthm| anymore, hence we reset it to
%    |\relax|. The definition of |\@ythm| has not changed at
%    all from its definition in \LaTeX. In order to make the macros
%    easier to understand, we will
%    nevertheless present it (commented out).
%    \begin{macrocode}
\global\let\@xthm\relax
% \def\@ythm#1#2[#3]{\@opargbegintheorem{#2}{\csname
%           the#1\endcsname}{#3}\ignorespaces}
%</package>
%    \end{macrocode}
%    The primitive |\ignorespaces| in |\@ythm| and
%    |\@thm| is needed to remove the spaces between the
%    |\begin{...}| and the actual text.
% \end{macro}
% \end{macro}
%
%
% \subsection{Definition of the theorem substyles}
%
% As already pointed out, the theorem substyles, defined below, are
% only loaded when necessary.  Note that all these substyles, except
% \texttt{plain}, have |\slshape| as the default body font.
%
% \subsubsection{The \texttt{plain} style}
%
% \label{dok:th@plain} As the following macros use |@|, we have
% to locally set the |\catcode| of this symbol to ``letter''.
% This happens within a group, so that we do not have to worry about
% which |\catcode| that symbol had before.
%    \begin{macrocode}
%<*thp>
\begingroup \makeatletter
%    \end{macrocode}
% Since we are now within a group, we must make all definitions
% globally.  First we make sure that \texttt{theorem.sty} is loaded.
% This will allow us to use this file as a document style option
% without having to call \texttt{theorem} itself as an option.  At the
% same time, we assure that at least version 2 is loaded, since
% |\theorem@style| was not defined in earlier versions.
%    \begin{macrocode}
\@ifundefined{theorem@style}{\input{theorem.sty}}{}
\ProvidesFile{thp.sty}
             [\FMithmInfo]
%    \end{macrocode}
%
%
% \begin{macro}{\th@plain}
% \begin{macro}{\@begintheorem}
% \begin{macro}{\@opargbegintheorem}
%    |\theoremstyle{plain}| corresponds to the original
%    definition, except that the distances to the surrounding text are
%    determined by the parameters |\theorempreskipamount| and
%    |\theorempostskipamount|.
%    First we set the default body font.
%    \begin{macrocode}
\gdef\th@plain{\normalfont\itshape
%    \end{macrocode}
%    Then we define |\@begintheorem| and
%    |\@opargbegintheorem|. These two macros define how the
%    header of a theorem is typeset.  |\@opargbegintheorem| will
%    be called if a \textsf{theorem} environment with an optional
%    argument is encountered; otherwise, the header is constructed by
%    calling |\@begintheorem|.  If one of these macros is
%    executed, we are within a \textsf{trivlist} environment started by
%    |\@thm|.  So the theorem header is produced with an
%    |\item| command.
%
%    Instead of specifying the header font directly, all standard
%    theorem styles use the |\theorem@headerfont| macro to allow
%    customization.  The extra space (|\labelsep|) is necessary
%    because of problems in the \textsf{trivlist} environment.
%    \begin{macrocode}
  \def\@begintheorem##1##2{%
        \item[\hskip\labelsep \theorem@headerfont ##1\ ##2]}%
%    \end{macrocode}
%    The definition of |\@opargbegintheorem| is completely
%    analogous. The only difference is the fact that there exists a
%    third argument (which is the optional parameter of the
%    environment and contains additional information about the
%    theorem).  Customarily we enclose it in parentheses.
%    \begin{macrocode}
\def\@opargbegintheorem##1##2##3{%
   \item[\hskip\labelsep \theorem@headerfont ##1\ ##2\ (##3)]}}
%    \end{macrocode}
% \end{macro}
% \end{macro}
% \end{macro}
%
% \noindent
% We conclude with an |\endgroup| to restore the
% |\catcode| of |@|.
%    \begin{macrocode}
\endgroup
%</thp>
%    \end{macrocode}
%
%
% \subsubsection{The \texttt{break} style}
%
% This style option is stored in the file |thb.sty|.
% For the next two lines see the documentation for |\th@plain| on
% page~\pageref{dok:th@plain}.
%    \begin{macrocode}
%<*thb>
\begingroup \makeatletter
\@ifundefined{theorem@style}{\input{theorem.sty}}{}
\ProvidesFile{thb.sty}
             [\FMithmInfo]
%    \end{macrocode}
%
% \begin{macro}{\th@break}
%    |\theoremstyle{break}| produces a line break after the name
%    of the theorem.  The font is |\slshape|.  Hence, we define
%    |\th@break| as follows:
%    \begin{macrocode}
\gdef\th@break{\normalfont\slshape
  \def\@begintheorem##1##2{\item[%
%    \end{macrocode}
%    We run into the following problem: it is not possible to create
%    the header with |\item[|\meta{title}|]| and then start
%    a new line by, for example, |\mbox{}\\|. Such a definition
%    will fail whenever a list environment follows immediately.  With
%    the above construction, the |\mbox{}| causes the switch
%    |@inlabel| (cf.\ definition of |\list| and
%    |\trivlist| in \cite{latex:source}) to be set to \texttt{false}
%    and so the following list will insert additional vertical
%    space (|\topskip|). This is quite annoying. Therefore, we
%    create the line break within the |\item|. In order to ensure
%    that the text will begin at the proper position in the following
%    line, we simply pretend that the label does not take any
%    room.\footnote{This will lead to problems whenever very high
%    symbols occurring in the line tower into the
%    heading. So, something else has to be done here sometime.}
%    \begin{macrocode}
       \rlap{\vbox{\hbox{\hskip \labelsep\theorem@headerfont ##1\ ##2}%
                   \hbox{\strut}}}]}%
%    \end{macrocode}
%    Again, the definition of |\@opargbegintheorem| is
%    completely analogous.
%    \begin{macrocode}
\def\@opargbegintheorem##1##2##3{%
  \item[\rlap{\vbox{\hbox{\hskip \labelsep \theorem@headerfont
                     ##1\ ##2\ (##3)}%
                    \hbox{\strut}}}]}}
\endgroup
%</thb>
%    \end{macrocode}
% \end{macro}
%
%
% \subsubsection{The \texttt{changebreak} style}
%
%    \begin{macrocode}
% This style option is stored in the file |thcb.sty|.
%    \begin{macrocode}
%<*thcb>
\begingroup \makeatletter
\@ifundefined{theorem@style}{\input{theorem.sty}}{}
\ProvidesFile{thcb.sty}
             [\FMithmInfo]
%    \end{macrocode}
%
% \begin{macro}{\th@changebreak}
%    The \texttt{change-break} theorem style is like \texttt{break}
%    but with interchange of theorem name and theorem number.  Thus we
%    define |\th@changebreak| as follows:
%    \begin{macrocode}
\gdef\th@changebreak{\normalfont\slshape
  \def\@begintheorem##1##2{\item
       [\rlap{\vbox{\hbox{\hskip\labelsep \theorem@headerfont ##2\ ##1}%
                    \hbox{\strut}}}]}%
\def\@opargbegintheorem##1##2##3{%
  \item[\rlap{\vbox{\hbox{\hskip\labelsep \theorem@headerfont
                     ##2\ ##1\ (##3)}%
                    \hbox{\strut}}}]}}
\endgroup
%</thcb>
%    \end{macrocode}
% \end{macro}
%
%
% \subsubsection{The \texttt{change} style}
%
% This style option is stored in the file |thc.sty|.
%    \begin{macrocode}
%<*thc>
\begingroup \makeatletter
\@ifundefined{theorem@style}{\input{theorem.sty}}{}
\ProvidesFile{thc.sty}
             [\FMithmInfo]
%    \end{macrocode}
%
% \begin{macro}{\th@change}
%    The \texttt{change} theorem style corresponds to the
%    \texttt{change break} style without a linebreak after the header.
%    To say it in another way, it's the same as the \texttt{plain}
%    style but with number and name interchanged and |\slshape| as the
%    default font.
%    \begin{macrocode}
\gdef\th@change{\normalfont\slshape
  \def\@begintheorem##1##2{\item
       [\hskip\labelsep \theorem@headerfont ##2\ ##1]}%
\def\@opargbegintheorem##1##2##3{%
  \item[\hskip\labelsep \theorem@headerfont ##2\ ##1\ (##3)]}}
\endgroup
%</thc>
%    \end{macrocode}
% \end{macro}
%
%
% \subsubsection{The \texttt{marginbreak} style}
%
% This style option is the one used most often at Mainz.  It is saved
% in the file |thmb.sty|.
%    \begin{macrocode}
%<*thmb>
\begingroup \makeatletter
\@ifundefined{theorem@style}{\input{theorem.sty}}{}
\ProvidesFile{thmb.sty}
             [\FMithmInfo]
%    \end{macrocode}
%
% \begin{macro}{\th@marginbreak}
%    The \texttt{margin break} style is nearly the same as the
%    \texttt{change break} style. The only difference is the placement
%    of the theorem number.  We use |\llap| to place it in the left
%    margin.
%
%    In this style |\labelsep| denotes the separation between the
%    number and the text.
%    \begin{macrocode}
\gdef\th@marginbreak{\normalfont\slshape
  \def\@begintheorem##1##2{\item
       [\rlap{\vbox{\theorem@headerfont
                    \hbox{\llap{##2}\hskip\labelsep ##1}%
                    \hbox{\strut}}}]}%
\def\@opargbegintheorem##1##2##3{%
  \item[\rlap{\vbox{\theorem@headerfont
                    \hbox{\llap{##2}\hskip\labelsep ##1\ (##3)}%
                    \hbox{\strut}}}]}}
\endgroup
%</thmb>
%    \end{macrocode}
% \changes{v2.1c}{90/02/17}{`endgroup outside macrocode env.}
% \end{macro}
%
%
%
% \subsubsection{The \texttt{margin} style}
%
% This style option is stored in the file |thm.sty|.
%    \begin{macrocode}
%<*thm>
\begingroup \makeatletter
\@ifundefined{theorem@style}{\input{theorem.sty}}{}
\ProvidesFile{thm.sty}
             [\FMithmInfo]
%    \end{macrocode}
%
% \begin{macro}{\th@margin}
%    Again this is only a variant of the theorem styles described
%    above without any new ideas.
%    \begin{macrocode}
\gdef\th@margin{\normalfont\slshape
  \def\@begintheorem##1##2{\item
       [\theorem@headerfont \llap{##2}\hskip\labelsep ##1]}%
\def\@opargbegintheorem##1##2##3{%
  \item[\theorem@headerfont \llap{##2}\hskip\labelsep ##1\ (##3)]}}
\endgroup
%</thm>
%    \end{macrocode}
% \end{macro}
%
%
%
%
% \subsection{Final Definitions}
%
% \begin{macro}{\theorempreskipamount}
% \begin{macro}{\theorempostskipamount}
%    The \textsf{skip} parameters that regulate the vertical empty space
%    before and after the {\envfont theorem} environment have
%    to be allocated as well.
% \changes{v2.0f}{89/07/20}{Unnecessary `global' removed.}
%    \begin{macrocode}
%<*package>
\newskip\theorempreskipamount
\newskip\theorempostskipamount
%    \end{macrocode}
%    Since we have used the same values for all theorem sets, we now can
%    assign them.
%    \begin{macrocode}
\global\setlength\theorempreskipamount{12pt plus 5pt minus 3pt}
\global\setlength\theorempostskipamount{8pt plus 3pt minus 1.5pt}
%    \end{macrocode}
% \end{macro}
% \end{macro}
%
%
% \begin{macro}{\@endtheorem}
%    The same holds for the macro |\@endtheorem|, which ends a
%    {\envfont theorem} environment. Since it is the same for all
%    theorem sets, it is removed from the macros |\th@|\meta{style}.
%    It simply ends the {\envfont trivlist} environment, which was
%    begun in |\@thm|.
%    \begin{macrocode}
\global\let\@endtheorem=\endtrivlist
%    \end{macrocode}
% \end{macro}
%
%
%
% \begin{macro}{\@preamblecmds}
%    All macros defined above are to be used only in the preamble.
%    Therefore, we insert them in |\@preamblecmds| which will disable
%    them at begin document. This is done by the internal \LaTeXe{}
%    command |\@onlypreamble|.
% \changes{v2.0g}{89/09/19}{`theoremheaderfont and `theorembodyfont
%    added.}
% \changes{v2.2a}{94/02/03}{Use \LaTeXe{} interface}
%    \begin{macrocode}
\@onlypreamble\@xnthm
\@onlypreamble\@ynthm
\@onlypreamble\@othm
\@onlypreamble\newtheorem
\@onlypreamble\theoremstyle
\@onlypreamble\theorembodyfont
\@onlypreamble\theoremheaderfont
%    \end{macrocode}
% \end{macro}
%
% Finally we declare the \texttt{plain} theorem style to be the default.
% \changes{v2.1b}{90/06/04}{Fixed missing `item by declaring default
%    style.}
%    \begin{macrocode}
\theoremstyle{plain}
%</package>
%    \end{macrocode}
%
% \Finale

\endinput
}{}
\ProvidesFile{thmb.sty}
             [\FMithmInfo]
%    \end{macrocode}
%
% \begin{macro}{\th@marginbreak}
%    The \texttt{margin break} style is nearly the same as the
%    \texttt{change break} style. The only difference is the placement
%    of the theorem number.  We use |\llap| to place it in the left
%    margin.
%
%    In this style |\labelsep| denotes the separation between the
%    number and the text.
%    \begin{macrocode}
\gdef\th@marginbreak{\normalfont\slshape
  \def\@begintheorem##1##2{\item
       [\rlap{\vbox{\theorem@headerfont
                    \hbox{\llap{##2}\hskip\labelsep ##1}%
                    \hbox{\strut}}}]}%
\def\@opargbegintheorem##1##2##3{%
  \item[\rlap{\vbox{\theorem@headerfont
                    \hbox{\llap{##2}\hskip\labelsep ##1\ (##3)}%
                    \hbox{\strut}}}]}}
\endgroup
%</thmb>
%    \end{macrocode}
% \changes{v2.1c}{90/02/17}{`endgroup outside macrocode env.}
% \end{macro}
%
%
%
% \subsubsection{The \texttt{margin} style}
%
% This style option is stored in the file |thm.sty|.
%    \begin{macrocode}
%<*thm>
\begingroup \makeatletter
\@ifundefined{theorem@style}{% \iffalse meta-comment
%
% Copyright (C) 1993-2024
%
% The LaTeX Project and any individual authors listed elsewhere
% in this file.
%
% This file is part of the Standard LaTeX `Tools Bundle'.
% -------------------------------------------------------
%
% It may be distributed and/or modified under the
% conditions of the LaTeX Project Public License, either version 1.3c
% of this license or (at your option) any later version.
% The latest version of this license is in
%    https://www.latex-project.org/lppl.txt
% and version 1.3c or later is part of all distributions of LaTeX
% version 2005/12/01 or later.
%
% The list of all files belonging to the LaTeX `Tools Bundle' is
% given in the file `manifest.txt'.
%
% \fi
%
%%
%
%
\def\FMithmInfo{2023/07/05 v2.2c Theorem extension package (FMi)}
%
% \ProvidesFile{theorem.dtx}[\FMithmInfo]
%\iffalse   % this is a METACOMMENT !
%%
%% Package `theorem' to use with LaTeX2e
%% Copyright (C) 1989-1995 Frank Mittelbach, all rights reserved.
%<+package>\@ifundefined{theorem@style}{}{\endinput}
%<+package>\NeedsTeXFormat{LaTeX2e}[1995/06/01]
%<+package>\ProvidesPackage{theorem}
%<+package>         [\FMithmInfo]
%
% \fi
%
% \changes{v1.0}{87/05/07}{First draft.}
% \changes{v1.0a}{87/05/07}{`thistheoremstyle' added.}
% \changes{v1.1a}{87/07/29}{New concept: redefine the internal
%                          \protect\LaTeX{} commands `@xnthm' etc.}
% \changes{v1.1b}{87/07/31}{`theoremfont' used to specify font.}
% \changes{v1.2a}{88/01/08}{`@endtheorem' now globally defined.}
% \changes{v1.2b}{88/01/11}{`newtheorem' only allowed in the preamble.}
% \changes{v1.3a}{88/02/14}{`theoremfont' now a toks register.}
%
% \changes{v2.0a}{88/11/16}{New implementation and new documentation.}
% \changes{v2.0a}{88/11/16}{No layout changes but internal modifications
%                           old substyles won't work.}
% \changes{v2.0b}{89/01/30}{`global' used for all definitions.}
% \changes{v2.0c}{89/06/15}{Documentation translated to English.
%                            Thanks to LLL and Nina}
% \changes{v2.1a}{90/02/15}{Support for new font selection.}
% \changes{v2.2a}{94/02/03}{Cleanup for LaTeX2e}
%
% \DoNotIndex{\@endparenv,\@ifdefinable,\@ifnextchar,\@ifundefined}
% \DoNotIndex{\@namedef,\@nameuse,\@spaces,\@tempa,\@warning}
% \DoNotIndex{\ }
% \DoNotIndex{\begingroup,\bfseries}
% \DoNotIndex{\csname}
% \DoNotIndex{\def,\do,\docdate}
% \DoNotIndex{\endcsname,\endgroup,\endinput,\endtrivlist,\expandafter}
% \DoNotIndex{\filedate,\fileversion}
% \DoNotIndex{\gdef,\global}
% \DoNotIndex{\hbox,\hskip}
% \DoNotIndex{\ignorespaces,\input,\itshape,\item}
% \DoNotIndex{\labelsep,\let,\llap}
% \DoNotIndex{\makeatletter}
% \DoNotIndex{\noexpand}
% \DoNotIndex{\refstepcounter,\relax,\rlap}
% \DoNotIndex{\setlength,\slshape,\space,\string,\strut}
% \DoNotIndex{\the,\trivlist,\typeout}
% \DoNotIndex{\vbox}
% \DoNotIndex{\xdef}
%
% \def\deflabel#1{\ttfamily #1\hfill}
% \def\deflist#1{\begin{list}{}{\settowidth\labelwidth{\ttfamily #1}%
%                               \setlength\leftmargin\labelwidth
%                               \addtolength\leftmargin\labelsep
%                               \let\makelabel\deflabel}}
% \def\enddeflist{\end{list}}
% \def\envfont{\normalfont\ttfamily}
%
% \MakeShortVerb{\|}
%
% \newcommand{\TUB}{{\sl TUGboat\/}}
%
% \GetFileInfo{theorem.dtx}
% \title{An Extension of the
%        \LaTeX{} theorem environment\thanks{This file
%        has version number \fileversion, last
%        revised \filedate.}}
% \author{Frank Mittelbach}
%
% \MaintainedByLaTeXTeam{tools}
% \maketitle
%
% \begin{abstract}
%    The macros described in this paper yield an extension of the
%    \LaTeX{} theorem mechanism. It is designed to satisfy the
%    different requirements of various journals. Thus, the
%    layout of the ``theorems'' can be manipulated by determining a
%    ``style''. This article describes not only the use, but
%    also the definition, of the necessary macros.
% \end{abstract}
%
% \section*{Preface to version 2.2}
%
% For \LaTeXe{} this package did not need any fundamental changes. I
% only modified the messages generated so that theorem layout styles
% will show up with the |\listfiles| command and cleaned the section
% on the New Font Selection Scheme since this is now included in
% \LaTeX.
%
% \section*{Preface to version 2.1}
%
% This version is identical to 2.0g described in \TUB\ 10\#3 except
% for some internal defaults which are now set depending on the used
% font selection scheme.
%
% This was done to avoid unpleasant surprises if the new font selection
% scheme is in force. For further details see
% section~\ref{sec:newfonts} and \cite{bk:GMS94}.
%
% \section{Introduction}
%
% For our purposes here, ``theorems'' are labelled enunciations,
% often set off from the main text by extra space and a font change.
% Theorems, corollaries, conjectures, definitions, and remarks
% are all instances of ``theorems''.  The ``header'' of these structures
% is composed of a label (such as \textsc{Theorem} or \textsc{Remark})
% and a number which serializes an item in the sequence of items
% with the same label.
%
% Shortly after the introduction of \LaTeX{} at the Fachbereich
% Mathematik in Mainz, the desire to manipulate the layout of
% ``theorems'' arose.  In Mainz, the following two conventions came
% into general use:
% \begin{enumerate}
%   \item The number of the theorem is shown in the margin.
%   \item There is a line break at
%         the end of the theorem header.
% \end{enumerate}
%
% Additionally, some journals require different formats which depend on
% the ``sort of theorem'': e.g.\ often remarks and definitions are set
% in |\upshape|, while |\itshape| is employed for main theorems.
%
% Confronted with these requirements, a theorem environment was
% developed in Mainz which allows separate determination of the
% layout of the ``theorems sets'', comparable to |\pagestyle|.
%
%
% \section{The user interface}
% \subsection{Defining new theorem sets}
%
% \DescribeMacro\newtheorem
% As in the original \LaTeX{} version, the command |\newtheorem|
% defines a new ``theorem set'' or ``theorem-like structure''.
% Two required arguments name the new environment and give the text
% to be typeset with each instance of the new ``set'', while
% an optional argument determines how
% the ``set'' is enumerated:
% \begin{description}
%    \item[\ttfamily \bslash newtheorem\{foo\}\{bar\}]
%       The theorem set {\envfont foo} (whose name is \texttt{bar})
%       uses its own counter.
%    \item[\ttfamily \bslash newtheorem\{foo2\}{[foo]}\{bar2\}]
%       The theorem set {\envfont foo2} (printed name \texttt{bar2})
%       uses the same counter as the theorem set \texttt{foo}.
%    \item[\ttfamily \bslash newtheorem\{foo3\}\{bar3\}{[section]}]
%       The theorem set {\envfont foo3} (printed name \texttt{bar3}) is
%       enumerated within the counter \texttt{section}, i.e.\ with every
%       new |\section| the enumeration begins again with 1, and
%       the enumeration is composed from the section-number and the
%       theorem counter itself.
% \end{description}
%
% \DescribeMacro\theoremstyle
% Additionally, the command |\theoremstyle| can define the layout
% of various, or all, theorem sets.  It should be noted that any
% theorem set defined by |\newtheorem| is typeset in the
% |\theoremstyle| that is current at the time of the definition.
% Thus, the following
% \begin{verbatim}
%    \theoremstyle{break}        \newtheorem{Cor}{Corollary}
%    \theoremstyle{plain}        \newtheorem{Exa}{Example}[section]
%\end{verbatim}                             ^^A No blank at line start
% leads to the result that the set {\envfont Cor} is formatted in the
% style \texttt{break}, while the set {\envfont Exa} and all the
% following ones are formatted in the style \texttt{plain}, unless
% another |\theoremstyle| follows.  Since the definitions installed by
% |\newtheorem| are global, one also can limit |\theoremstyle| locally
% by grouping braces.
%
% \DescribeMacro\theorembodyfont
% The choice of the font for the theorem body is completely
% independent of the chosen |\theoremstyle|; this has proven to
% be very advantageous.  For example,
% \begin{verbatim}
%   {\theorembodyfont{\upshape}          \newtheorem{Rem}{Remark}}
%\end{verbatim}
% defines a theorem set {\envfont Rem}, which will be set in
% |\upshape| in the current layout (which in our example is
% \texttt{plain}).  As with |\theoremstyle|, the |\theorembodyfont|
% chosen is that current at the time of |\newtheorem|.  If
% |\theorembodyfont| is not specified or one defines
% |\theorembodyfont{}|, then the font used will be that defined by the
% |\theoremstyle|.
%
% \DescribeMacro\theoremheaderfont It is also possible to customize
% the font used for the theorem headers.  This is, however, a global
% declaration, and therefore there should be at most one
% |\theoremheaderfont| declaration in the preamble.\footnote{If it is
% actually necessary to have different header fonts, one has to define
% new theorem styles (substituting the desired font) or specify the
% information directly in the \texttt{\bslash newtheorem} declaration
% (the unclean variant).}
%
% \DescribeMacro\theorempreskipamount
% \DescribeMacro\theorempostskipamount Two additional parameters
% affect the vertical space around the theorem environments:
% |\theorempreskipamount| and |\theorempostskipamount| define,
% respectively, the spacing before and after such an environment.
% These parameters apply for all theorem sets and can be manipulated
% with the ordinary length macros.  They are rubber lengths
% (`\textsf{skips}'), and therefore can contain \texttt{plus} and
% \texttt{minus} parts.
%
% Since the definition of theorem sets should---most sensibly---be
% placed in the preamble, we only allow installation there.
% It is therefore possible to release the memory used here after
% |\begin{document}|, in order to make room for other
% applications.
%
%
% \subsection{Existing theorem styles}
%
% The following theorem styles exist to date:
% \begin{deflist}{marginbreak}
%    \item[plain]
%       This theorem style emulates the original \LaTeX{} definition,
%       except that additionally the parameters
%       |\theorem...skipamount| are used.
%    \item[break]
%       In this style, the theorem header is followed by a line
%       break.
%    \item[marginbreak]
%       The theorem number is set in the margin, and there is a line
%       break as in \texttt{break}.
%    \item[changebreak]
%       Like \texttt{break}, but with header number and text
%       interchanged.
%    \item[change]
%       Header number and text are interchanged, without a line break.
%    \item[margin]
%       The number is set in the left margin, without a line break.
% \end{deflist}
% All styles (except \texttt{plain}) select |\slshape| as the default
% |\theorembodyfont|.
%
%
% \subsection{Examples}
%
% Given the above theorem sets {\envfont Cor}, {\envfont Exa}
% and {\envfont Rem},
% suppose that the preamble also contains the declarations:
% \begin{verbatim}
%    \theoremstyle{marginbreak}   \newtheorem{Lem}[Cor]{Lemma}
%    \theoremstyle{change}
%    \theorembodyfont{\itshape}        \newtheorem{Def}[Cor]{Definition}
%
%    \theoremheaderfont{\scshape}
%\end{verbatim}
% Then the following are some typical examples of the typeset output
% resulting from their use.
%
% \begin{Cor}
%    This is a sentence typeset in the theorem environment {\envfont
%    Cor}.
% \end{Cor}
%
% \begin{Exa}
%    This is a sentence typeset in the theorem environment {\envfont
%    Exa}.
% \end{Exa}
%
% \begin{Rem}
%    This is a sentence typeset in the theorem environment {\envfont
%    Rem}.
% \end{Rem}
%
% \begin{Lem}[Ben User]
%    This is a sentence typeset in the theorem environment {\envfont
%    Lem}.
% \end{Lem}
%
% \begin{Def}[Very impressive Definition]
%    This is a sentence typeset in the theorem environment {\envfont
%    Def}.
% \end{Def}
%
% The last two examples show the effect of the optional argument to
% a theorem environment (it is the text typeset in parentheses).
%
%
% \section{Special Considerations}
% \label{sec:newfonts}
%
% Theoremheader and body are implemented as a unit. This means that
% the |\theoremheaderfont|  will inherit characteristics of the
% |\theorembodyfont| in \LaTeXe. Thus, if for example
% |\theorembodyfont| is |\itshape| and |\theoremheaderfont|
% is |\bfseries| the font selected for the header will have the
% characteristics `bold extended italic'. If this is not desired one
% should set the |\theoremheaderfont| to something like
% \begin{verbatim}
%      \theoremheaderfont{\normalfont\bfseries}
%\end{verbatim}
% i.e.\ supplying all necessary font information explicitly.
%
%
%
% \section{Acknowledgements}
%
% The publication of this set of macros was only possible with the
% help of Christina Busse (translating the manuscript into English),
% Joachim Pense (playing the r\^ole of typist), Chris Rowley
% (looking everything over) and many others providing useful
% suggestions.
%
% \MaybeStop{%
%    \begin{thebibliography}{9}
%       \bibitem{bk:GMS94} \textsc{M.~Goossens}, \textsc{F.~Mittelbach}
%         and \textsc{A.~Samarin}.
%         \newblock The \LaTeX{} Companion.
%         \newblock
%         Addison-Wesley, Reading, Massachusetts, 1994.
%       \bibitem{latex:source} \textsc{Lamport, Leslie.} \newblock
%                               \texttt{latex.tex}, version 2.09, date
%                               Feb. 1990.
%    \end{thebibliography}
%    \PrintIndex
%    \PrintChanges}
%
%
% \section{The documentation driver file}
%
% The next bit of code contains the documentation driver file for
% \TeX{}, i.e., the file that will produce the documentation you are
% currently reading. It will be extracted from this file by the
% \texttt{docstrip} program. Since it is the first code in the file
% one can alternatively process this file directly with \LaTeXe{} to
% obtain the documentation.
%
%    \begin{macrocode}
%<*driver>
\documentclass{ltxdoc}

\usepackage{theorem}

%<+driver>% The next few lines define theorem sets which are used
%<+driver>% in the example section of the documentation.

\theoremstyle{break}       \newtheorem{Cor}{Corollary}
\theoremstyle{plain}       \newtheorem{Exa}{Example}[section]
{\theorembodyfont{\upshape}\newtheorem{Rem}{Remark}}
\theoremstyle{marginbreak} \newtheorem{Lem}[Cor]{Lemma}
\theoremstyle{change}
\theorembodyfont{\itshape} \newtheorem{Def}[Cor]{Definition}

\theoremheaderfont{\scshape}

\RecordChanges

\begin{document}
   \DocInput{theorem.dtx}
\end{document}
%</driver>
%    \end{macrocode}
%
% \section{Definition of the Macros}
%
% If the file has been loaded before, we abort immediately. If not the
% package announces itself (this is actually done at the very top of
% the file---the way it is done isn't good style so don't copy it).
% \changes{v2.0e}{89/07/19}{Spaces removed from `typeout'.}
% \changes{v2.2b}{95/11/19}{Announce moved to top of file}
%    \begin{macrocode}
%<*package>
%\@ifundefined{theorem@style}{}{\endinput}
%\def\FMithmInfo{1995/11/19 v2.2b Theorem extension package (FMi)}
%\ProvidesPackage{theorem}[\FMithmInfo]
%    \end{macrocode}
%
%
% \subsection{Definition of theorem styles and fonts}
%
% All the definitions in this file are done globally to allow
% inputting this file inside a group.
%
% \begin{macro}{\theoremstyle}
%    Before a theorem style can be installed, the chosen style must be
%    known. For that reason, we must test to see that
%    |\th@|\meta{style} is known or, more precisely, that it is
%    different from |\relax|. If the style is not known then
%    |\th@plain| is used.
%    \begin{macrocode}
\gdef\theoremstyle#1{%
   \@ifundefined{th@#1}{\@warning
          {Unknown theoremstyle `#1'. Using `plain'}%
          \theorem@style{plain}}%
%    \end{macrocode}
%    We save the theorem style to be used in the token register
%    |\theorem@style|.
%    \begin{macrocode}
      {\theorem@style{#1}}%
%    \end{macrocode}
%    Now we ``evaluate'' the theorem style: this means, we call the
%    macro |\th@|\meta{style} which will activate the relevant
%    definitions which are contained in a separate file.  This is done
%    in a group to suppress changes to the current font.  This could
%    otherwise pose problems together with the new font selection
%    scheme\footnote{When I printed the original article using the new
%    font selection scheme I ended with a document with slanted
%    typefaces (text headings and all) simply because one of the
%    theorem styles used \texttt{\bslash sl} at toplevel.} if the
%    |\th@|\meta{style} is evaluated a second time.
% \changes{v2.1a}{90/02/15}{Group added around `th@..}
% \changes{v2.1a}{90/02/15}{Reseting of `..@begintheorem removed.}
%    \begin{macrocode}
      \begingroup
        \csname th@\the\theorem@style \endcsname
      \endgroup}
%    \end{macrocode}
% \end{macro}
%
%
% \begin{macro}{\@begintheorem}
% \changes{v2.1a}{90/02/15}{Relaxed at toplevel.}
% \begin{macro}{\@opargbegintheorem}
% \changes{v2.1a}{90/02/15}{Relaxed at toplevel.}
%    We reset |\@begintheorem| and |\@opargbegintheorem| to
%    |\relax| since these commands are no longer necessary at
%    toplevel. This will save a few tokens.
%    \begin{macrocode}
\global\let\@begintheorem\relax
\global\let\@opargbegintheorem\relax
%    \end{macrocode}
% \end{macro}
% \end{macro}
%
% \begin{macro}{\theorem@style}
%    Obviously the \textsf{token} register used above has to be
%    allocated. To assure the utmost compatibility with the
%    original \LaTeX{} definition, we set the default theorem style to
%    \texttt{plain}, which implements the usual \LaTeX{} convention.
% \changes{v2.0f}{89/07/20}{Unnecessary `global' removed.}
%    \begin{macrocode}
\newtoks\theorem@style
\global\theorem@style{plain}
%    \end{macrocode}
% \end{macro}
%
% \begin{macro}{\theorembodyfont}
% \begin{macro}{\theorem@bodyfont}
%    If the |\theorembodyfont| is set by the user then it should not
%    interact with the default font set in the theorem style. When the
%    new font selection is in force this may happen if, for example,
%    the default is |\itshape| and the new |\theorembodyfont| is
%    |\sffamily|. So we add a |\reset@font| command in front of the
%    user definition.
% \changes{v2.0f}{89/07/20}{Unnecessary `global' removed.}
% \changes{v2.0g}{89/09/19}{`theoremfont to `theorembodyfont.}
% \changes{v2.1c}{92/06/22}{`reset@font added and logic changed.}
% \changes{v2.1d}{92/10/28}{Reset to empty if arg is empty}
%    \begin{macrocode}
\gdef\theorembodyfont#1{%
%    \end{macrocode}
%    We check if the argument supplied is empty and if so put nothing
%    into the |\theorem@bodyfont| token register to allow for
%    |\theorembodyfont{}| as a mean of using the default of the
%    current |\theoremstyle|.
%    \begin{macrocode}
   \def\@tempa{#1}%
   \ifx\@tempa\@empty
    \theorem@bodyfont{}%
   \else
    \theorem@bodyfont{\reset@font#1}%
   \fi
   }
\newtoks\theorem@bodyfont
\global\theorem@bodyfont{}
%    \end{macrocode}
% \end{macro}
% \end{macro}
%
%
% \begin{macro}{\theoremheaderfont}
% \changes{v2.0g}{89/09/19}{Macro added.}
%    The font for the theorem headers is handled differently because
%    this definition applies to all theorem styles.
%    \begin{macrocode}
\gdef\theoremheaderfont#1{\gdef\theorem@headerfont{#1}%
%    \end{macrocode}
%    After using the macro once it is redefined to produce an error
%    message.
%    \begin{macrocode}
       \gdef\theoremheaderfont##1{%
        \typeout{\string\theoremheaderfont\space should be used
                 only once.}}}
%    \end{macrocode}
% \end{macro}
%
% \begin{macro}{\theorem@headerfont}
% \changes{v2.0g}{89/09/19}{Macro added.}
%    To set the |\theorem@headerfont| default we first test if
%    the new fontselection scheme is in force.
% \changes{v2.1a}{90/02/15}{Support for new font selection.}
%    \begin{macrocode}
\ifx\upshape\undefined
%    \end{macrocode}
%    If not we define it to expand into |\bfseries|. We don't use
%    |\let| just in case a following style option redefines this
%    macro.
%    \begin{macrocode}
\gdef\theorem@headerfont{\bfseries}
%    \end{macrocode}
%    Otherwise we reset the current shape before calling |\bfseries|.
%    \begin{macrocode}
\else \gdef\theorem@headerfont{\normalfont\bfseries}\fi
%    \end{macrocode}
% \end{macro}
%
% \begin{macro}{\th@plain}
% \begin{macro}{\th@break}
% \begin{macro}{\th@marginbreak}
% \begin{macro}{\th@changebreak}
% \begin{macro}{\th@change}
% \begin{macro}{\th@margin}
%    The different styles are defined in macros such as
%    |\th@plain|.  Since memory space is precious in
%    ``non-Big-versions'', we have to avoid offering too many unused
%    definitions.  Therefore we define these styles in separate files
%    that can be loaded on demand.  Thus the commands themselves only
%    load these files. We use |\@input@|, a \LaTeXe{} internal command
%    that ensures that the file will be listed with |\listfiles|.
% \changes{v2.2a}{94/02/02}{Use `@input@ to load theorem layout files}
%    \begin{macrocode}
\gdef\th@plain{\@input@{thp.sty}}
\gdef\th@break{\@input@{thb.sty}}
\gdef\th@marginbreak{\@input@{thmb.sty}}
\gdef\th@changebreak{\@input@{thcb.sty}}
\gdef\th@change{\@input@{thc.sty}}
\gdef\th@margin{\@input@{thm.sty}}
%    \end{macrocode}
%    This list will be expanded when new styles become
%    available.  For testing,  just append new theorem substyles
%    as document options.
% \end{macro}
% \end{macro}
% \end{macro}
% \end{macro}
% \end{macro}
% \end{macro}
%
%
%
% \subsection{Definition of a new theorem set}
%
% As already pointed out, a new theorem environment can be defined in
% three different ways:
% \begin{verbatim}
%    \newtheorem{Lem}{Lemma}
%    \newtheorem{Lem}{Lemma}[section]
%    \newtheorem{Lem}[Theorem]{Lemma}
%\end{verbatim}
% The function of the macro |\newtheorem| is to recognize these cases
% and then to branch into one of the three macros |\@ynthm|, |\@xnthm|
% or |\@othm|.  This mechanism is adopted unchanged from
% \cite{latex:source}; the essential point here is that, for example,
% in the second case, the arguments \texttt{Lem}, \texttt{Lemma} and
% \texttt{section} are passed over to the macro |\@xnthm|.
%
% We inspect this case first because the others present fewer problems,
% and thus are easily derived from this one.
%
% \begin{macro}{\@xnthm}
%    For our example arguments, the macro |\@xnthm| must fulfill
%    the following:
%    \begin{itemize}
%       \item
%          Define a new \LaTeX-counter `Lem'
%       \item
%          reset this counter within a |\section|
%       \item
%          define the macro |\theLem|
%       \item
%          define the environment macros |\Lem| and
%          |\endLem| using the current
%          |\theoremstyle| and |\theorem@bodyfont|.
%    \end{itemize}
%    Obviously, all this should happen only if the first argument
%    of |\@xnthm| (i.e.\ \texttt{Lem} in our example) is chosen so
%    as not to conflict with any previously defined commands or
%    environments.  This test is performed by the \LaTeX{} macro
%    |\@ifdefinable|.
%    \begin{macrocode}
\gdef\@xnthm#1#2[#3]{\expandafter\@ifdefinable\csname #1\endcsname
%    \end{macrocode}
%    Therefore, the first argument of |\@ifdefinable| is the
%    expansion (in the example, |\Lem|) of
%    |\csname||#1\endcsname|.  The second argument is
%    executed only if the test has been completed successfully.
%    \begin{macrocode}
   {%
%    \end{macrocode}
%    Now we define the new counter. The names of the \LaTeX{}
%    macros employed should speak for themselves:
% \changes{v2.2b}{1995/10/07}{Use \cs{@newctr}}
% \changes{v2.2c}{1995/11/23}{Use \cs{@newctr} properly.}
%    \begin{macrocode}
    \@definecounter{#1}\@newctr{#1}[#3]%
%    \end{macrocode}
%    Using |\@newctr| will give a proper error message if the counter
%    in |#3| is not defined.
%    In defining `|\theLem|' we must generate the desired macro
%    name by use of |\expandafter| and |\csname|.
%    \begin{macrocode}
    \expandafter\xdef\csname the#1\endcsname
%    \end{macrocode}
%    An |\xdef| is used in order to make the definition global,
%    and to ensure that it contains the replacement texts of
%    |\@thmcountersep| and |\@thmcounter|.\footnote{These
%    two macros can be defined by the document style.  Their default
%    values produce a `\texttt{.}' as separation and an arabic
%    representation of the number.} However, not everything should be
%    expanded. For example, it saves space to use |\thesection|
%    instead of its---at times---lengthy expansion.
%    \begin{macrocode}
      {\expandafter \noexpand \csname the#3\endcsname
       \@thmcountersep \@thmcounter{#1}}%
%    \end{macrocode}
%    Thus with the defaults of \LaTeX{}, |\theLem| would be
%    replaced by the command sequence |\thesection.\arabic{Lem}|.
%
%    We will now look at the definition of the macro which is executed
%    at the beginning of the actual environment (in our example this
%    macro is |\Lem|). It should be noted that we use an
%    ``|\expandafter| trick'' to expand only certain parts of the
%    replacement text at the time of the definition.
%    \begin{macrocode}
    \def\@tempa{\global\@namedef{#1}}%
    \expandafter \@tempa \expandafter{%
%    \end{macrocode}
%    First, the macro that contains the current definitions of
%    |\@begintheorem| and |\@opargtheorem| should be called
%    up.  The name of this macro---as is already known---has the form
%    |\th@|\meta{theorem\ style}; therefore, it must be called by
%    \begin{macrocode}
      \csname th@\the \theorem@style
%    \end{macrocode}
%    In addition the default theorem font should be changeable, i.e.\
%    we have to insert the contents of |\theorem@bodyfont|. For
%    that reason, we expand even further, beyond |\endcsname|,
%    and thus insert the contents of the \textsf{token} register
%    |\theorem@bodyfont| in the replacement text.
%    \begin{macrocode}
            \expandafter \endcsname \the \theorem@bodyfont
%    \end{macrocode}
%    Now it is time to call the macro |\@thm| which takes over
%    the further processing.  It has two arguments: the current
%    counter name (in our example, \texttt{Lem}), and the text of the
%    label (in our example, \texttt{Lemma}).
%    \begin{macrocode}
     \@thm{#1}{#2}}%
%    \end{macrocode}
%    With this, the `sub-definition' is complete.  The macro
%    |\@endtheorem| ends a theorem environment and is, so far, nothing
%    but an |\endtrivlist|.  (Hence it is defined globally, and not
%    within the theorem styles.\footnote{This has to be changed as
%    soon as theorem styles that change \texttt{\bslash @endtheorem}
%    exist.  In such a case, all existing styles must be changed as
%    well since they will have to reset the macro.}) Therefore, we can
%    set it equivalent to the macro that ends the theorem set (in our
%    example, |\endLem|).  However, if some day theorem styles exist
%    that do change |\@endtheorem|, we would have to use the
%    commented-out line instead.
%    \begin{macrocode}
    \global \expandafter \let \csname end#1\endcsname \@endtheorem
%   \global\@namedef{end#1}{\@endtheorem}%
%    \end{macrocode}
%    With these commands all the required definitions are employed,
%    unless the test |\@ifdefinable| has failed.  Therefore, we
%    end the second argument of this macro and with it the definition
%    of |\@xnthm|.
%    \begin{macrocode}
   }}
%    \end{macrocode}
% \end{macro}
%
% \begin{macro}{\@ynthm}
%    The definition of |\@ynthm| is completely analogous. In
%    this case the new counter that is defined is not reset within
%    another counter; thus the definition of |\the...|
%    is simplified:
%    \begin{macrocode}
\gdef\@ynthm#1#2{\expandafter\@ifdefinable\csname #1\endcsname
   {\@definecounter{#1}%
    \expandafter\xdef\csname the#1\endcsname{\@thmcounter{#1}}%
%    \end{macrocode}
%    The rest of the definition corresponds literally to that of
%    |\@xnthm|:
%    \begin{macrocode}
    \def\@tempa{\global\@namedef{#1}}\expandafter \@tempa
     \expandafter{\csname th@\the \theorem@style \expandafter
     \endcsname \the\theorem@bodyfont \@thm{#1}{#2}}%
    \global \expandafter \let \csname end#1\endcsname \@endtheorem}}
%    \end{macrocode}
% \end{macro}
%
% \begin{macro}{\@othm}
%    The definition of |\@othm| does not contain anything new.
%    \begin{macrocode}
\gdef\@othm#1[#2]#3{%
%    \end{macrocode}
% \changes{v2.2b}{95/10/07}{Build in check for counter defined}
%    We do not define a new counter but instead use one that has
%    already been defined.  Thus the only definition we need is that
%    of this pseudo-counter (i.e.\ |\the|\meta{env.\ name}).
%    First we check if |#2| corresponds to a known counter name.
%    \begin{macrocode}
  \expandafter\ifx\csname c@#2\endcsname\relax
   \@nocounterr{#2}%
  \else
   \expandafter\@ifdefinable\csname #1\endcsname
   {\expandafter \xdef \csname the#1\endcsname
     {\expandafter \noexpand \csname the#2\endcsname}%
%    \end{macrocode}
%    All other parts of the definition can be adopted from
%    |\@xnthm|. We have to remember, though, that in this case
%    the name of the current counter and the theorem label have moved
%    to the second and third arguments.
%    \begin{macrocode}
    \def\@tempa{\global\@namedef{#1}}\expandafter \@tempa
     \expandafter{\csname th@\the \theorem@style \expandafter
     \endcsname \the\theorem@bodyfont \@thm{#2}{#3}}%
    \global \expandafter \let \csname end#1\endcsname \@endtheorem}%
  \fi}
%    \end{macrocode}
% \end{macro}
%
% \subsection{Macros that are employed in a {\envfont theorem}
%    environment}
%
% \begin{macro}{\@thm}
%    The macro |\@thm| has to increase the current counter.  Then,
%    depending on whether the environment has (or does not have) an
%    optional argument, it has to branch into either |\@begintheorem|
%    or |\@opargtheorem|.
%    \begin{macrocode}
\gdef\@thm#1#2{\refstepcounter{#1}%
%    \end{macrocode}
%    Now we start a {\envfont trivlist} environment, and give
%    |\@topsep| and |\@topsepadd| the values of the \textsf{skip}
%    registers |\theorempreskipamount| and |\theorempostskipamount|.
%    The value in |\@topsep| is the vertical space that is inserted by
%    the first (and only) |\item| in our |\trivlist| whilst
%    |\@topsepadd| is inserted by |\@endparenv| at the end of that
%    {\envfont trivlist} environment.  By using these registers, we
%    obtain the desired space around a {\envfont theorem} environment.
%    \begin{macrocode}
   \trivlist
   \@topsep \theorempreskipamount               % used by first \item
   \@topsepadd \theorempostskipamount           % used by \@endparenv
%    \end{macrocode}
%    Now we have to test whether an optional argument has been given.
% \changes{v2.0d}{89/07/18}{Forgotten percent added.}
%    \begin{macrocode}
   \@ifnextchar [%
%    \end{macrocode}
%    If there is an optional argument, we will call |\@ythm|, and
%    move the arguments read back into the input stream.
%    \begin{macrocode}
   {\@ythm{#1}{#2}}%
%    \end{macrocode}
%    If not, we call |\@begintheorem|. Its first argument is the
%    name of the theorem set (hence the second argument of
%    |\@thm|). Its second argument is the macro that produces the
%    current number.
%    \begin{macrocode}
   {\@begintheorem{#2}{\csname the#1\endcsname}\ignorespaces}}
%    \end{macrocode}
% \end{macro}
%
% \begin{macro}{\@xthm}
% \begin{macro}{\@ythm}
%    Both these macros were originally called by |\@thm|.  We do
%    not need |\@xthm| anymore, hence we reset it to
%    |\relax|. The definition of |\@ythm| has not changed at
%    all from its definition in \LaTeX. In order to make the macros
%    easier to understand, we will
%    nevertheless present it (commented out).
%    \begin{macrocode}
\global\let\@xthm\relax
% \def\@ythm#1#2[#3]{\@opargbegintheorem{#2}{\csname
%           the#1\endcsname}{#3}\ignorespaces}
%</package>
%    \end{macrocode}
%    The primitive |\ignorespaces| in |\@ythm| and
%    |\@thm| is needed to remove the spaces between the
%    |\begin{...}| and the actual text.
% \end{macro}
% \end{macro}
%
%
% \subsection{Definition of the theorem substyles}
%
% As already pointed out, the theorem substyles, defined below, are
% only loaded when necessary.  Note that all these substyles, except
% \texttt{plain}, have |\slshape| as the default body font.
%
% \subsubsection{The \texttt{plain} style}
%
% \label{dok:th@plain} As the following macros use |@|, we have
% to locally set the |\catcode| of this symbol to ``letter''.
% This happens within a group, so that we do not have to worry about
% which |\catcode| that symbol had before.
%    \begin{macrocode}
%<*thp>
\begingroup \makeatletter
%    \end{macrocode}
% Since we are now within a group, we must make all definitions
% globally.  First we make sure that \texttt{theorem.sty} is loaded.
% This will allow us to use this file as a document style option
% without having to call \texttt{theorem} itself as an option.  At the
% same time, we assure that at least version 2 is loaded, since
% |\theorem@style| was not defined in earlier versions.
%    \begin{macrocode}
\@ifundefined{theorem@style}{\input{theorem.sty}}{}
\ProvidesFile{thp.sty}
             [\FMithmInfo]
%    \end{macrocode}
%
%
% \begin{macro}{\th@plain}
% \begin{macro}{\@begintheorem}
% \begin{macro}{\@opargbegintheorem}
%    |\theoremstyle{plain}| corresponds to the original
%    definition, except that the distances to the surrounding text are
%    determined by the parameters |\theorempreskipamount| and
%    |\theorempostskipamount|.
%    First we set the default body font.
%    \begin{macrocode}
\gdef\th@plain{\normalfont\itshape
%    \end{macrocode}
%    Then we define |\@begintheorem| and
%    |\@opargbegintheorem|. These two macros define how the
%    header of a theorem is typeset.  |\@opargbegintheorem| will
%    be called if a \textsf{theorem} environment with an optional
%    argument is encountered; otherwise, the header is constructed by
%    calling |\@begintheorem|.  If one of these macros is
%    executed, we are within a \textsf{trivlist} environment started by
%    |\@thm|.  So the theorem header is produced with an
%    |\item| command.
%
%    Instead of specifying the header font directly, all standard
%    theorem styles use the |\theorem@headerfont| macro to allow
%    customization.  The extra space (|\labelsep|) is necessary
%    because of problems in the \textsf{trivlist} environment.
%    \begin{macrocode}
  \def\@begintheorem##1##2{%
        \item[\hskip\labelsep \theorem@headerfont ##1\ ##2]}%
%    \end{macrocode}
%    The definition of |\@opargbegintheorem| is completely
%    analogous. The only difference is the fact that there exists a
%    third argument (which is the optional parameter of the
%    environment and contains additional information about the
%    theorem).  Customarily we enclose it in parentheses.
%    \begin{macrocode}
\def\@opargbegintheorem##1##2##3{%
   \item[\hskip\labelsep \theorem@headerfont ##1\ ##2\ (##3)]}}
%    \end{macrocode}
% \end{macro}
% \end{macro}
% \end{macro}
%
% \noindent
% We conclude with an |\endgroup| to restore the
% |\catcode| of |@|.
%    \begin{macrocode}
\endgroup
%</thp>
%    \end{macrocode}
%
%
% \subsubsection{The \texttt{break} style}
%
% This style option is stored in the file |thb.sty|.
% For the next two lines see the documentation for |\th@plain| on
% page~\pageref{dok:th@plain}.
%    \begin{macrocode}
%<*thb>
\begingroup \makeatletter
\@ifundefined{theorem@style}{\input{theorem.sty}}{}
\ProvidesFile{thb.sty}
             [\FMithmInfo]
%    \end{macrocode}
%
% \begin{macro}{\th@break}
%    |\theoremstyle{break}| produces a line break after the name
%    of the theorem.  The font is |\slshape|.  Hence, we define
%    |\th@break| as follows:
%    \begin{macrocode}
\gdef\th@break{\normalfont\slshape
  \def\@begintheorem##1##2{\item[%
%    \end{macrocode}
%    We run into the following problem: it is not possible to create
%    the header with |\item[|\meta{title}|]| and then start
%    a new line by, for example, |\mbox{}\\|. Such a definition
%    will fail whenever a list environment follows immediately.  With
%    the above construction, the |\mbox{}| causes the switch
%    |@inlabel| (cf.\ definition of |\list| and
%    |\trivlist| in \cite{latex:source}) to be set to \texttt{false}
%    and so the following list will insert additional vertical
%    space (|\topskip|). This is quite annoying. Therefore, we
%    create the line break within the |\item|. In order to ensure
%    that the text will begin at the proper position in the following
%    line, we simply pretend that the label does not take any
%    room.\footnote{This will lead to problems whenever very high
%    symbols occurring in the line tower into the
%    heading. So, something else has to be done here sometime.}
%    \begin{macrocode}
       \rlap{\vbox{\hbox{\hskip \labelsep\theorem@headerfont ##1\ ##2}%
                   \hbox{\strut}}}]}%
%    \end{macrocode}
%    Again, the definition of |\@opargbegintheorem| is
%    completely analogous.
%    \begin{macrocode}
\def\@opargbegintheorem##1##2##3{%
  \item[\rlap{\vbox{\hbox{\hskip \labelsep \theorem@headerfont
                     ##1\ ##2\ (##3)}%
                    \hbox{\strut}}}]}}
\endgroup
%</thb>
%    \end{macrocode}
% \end{macro}
%
%
% \subsubsection{The \texttt{changebreak} style}
%
%    \begin{macrocode}
% This style option is stored in the file |thcb.sty|.
%    \begin{macrocode}
%<*thcb>
\begingroup \makeatletter
\@ifundefined{theorem@style}{\input{theorem.sty}}{}
\ProvidesFile{thcb.sty}
             [\FMithmInfo]
%    \end{macrocode}
%
% \begin{macro}{\th@changebreak}
%    The \texttt{change-break} theorem style is like \texttt{break}
%    but with interchange of theorem name and theorem number.  Thus we
%    define |\th@changebreak| as follows:
%    \begin{macrocode}
\gdef\th@changebreak{\normalfont\slshape
  \def\@begintheorem##1##2{\item
       [\rlap{\vbox{\hbox{\hskip\labelsep \theorem@headerfont ##2\ ##1}%
                    \hbox{\strut}}}]}%
\def\@opargbegintheorem##1##2##3{%
  \item[\rlap{\vbox{\hbox{\hskip\labelsep \theorem@headerfont
                     ##2\ ##1\ (##3)}%
                    \hbox{\strut}}}]}}
\endgroup
%</thcb>
%    \end{macrocode}
% \end{macro}
%
%
% \subsubsection{The \texttt{change} style}
%
% This style option is stored in the file |thc.sty|.
%    \begin{macrocode}
%<*thc>
\begingroup \makeatletter
\@ifundefined{theorem@style}{\input{theorem.sty}}{}
\ProvidesFile{thc.sty}
             [\FMithmInfo]
%    \end{macrocode}
%
% \begin{macro}{\th@change}
%    The \texttt{change} theorem style corresponds to the
%    \texttt{change break} style without a linebreak after the header.
%    To say it in another way, it's the same as the \texttt{plain}
%    style but with number and name interchanged and |\slshape| as the
%    default font.
%    \begin{macrocode}
\gdef\th@change{\normalfont\slshape
  \def\@begintheorem##1##2{\item
       [\hskip\labelsep \theorem@headerfont ##2\ ##1]}%
\def\@opargbegintheorem##1##2##3{%
  \item[\hskip\labelsep \theorem@headerfont ##2\ ##1\ (##3)]}}
\endgroup
%</thc>
%    \end{macrocode}
% \end{macro}
%
%
% \subsubsection{The \texttt{marginbreak} style}
%
% This style option is the one used most often at Mainz.  It is saved
% in the file |thmb.sty|.
%    \begin{macrocode}
%<*thmb>
\begingroup \makeatletter
\@ifundefined{theorem@style}{\input{theorem.sty}}{}
\ProvidesFile{thmb.sty}
             [\FMithmInfo]
%    \end{macrocode}
%
% \begin{macro}{\th@marginbreak}
%    The \texttt{margin break} style is nearly the same as the
%    \texttt{change break} style. The only difference is the placement
%    of the theorem number.  We use |\llap| to place it in the left
%    margin.
%
%    In this style |\labelsep| denotes the separation between the
%    number and the text.
%    \begin{macrocode}
\gdef\th@marginbreak{\normalfont\slshape
  \def\@begintheorem##1##2{\item
       [\rlap{\vbox{\theorem@headerfont
                    \hbox{\llap{##2}\hskip\labelsep ##1}%
                    \hbox{\strut}}}]}%
\def\@opargbegintheorem##1##2##3{%
  \item[\rlap{\vbox{\theorem@headerfont
                    \hbox{\llap{##2}\hskip\labelsep ##1\ (##3)}%
                    \hbox{\strut}}}]}}
\endgroup
%</thmb>
%    \end{macrocode}
% \changes{v2.1c}{90/02/17}{`endgroup outside macrocode env.}
% \end{macro}
%
%
%
% \subsubsection{The \texttt{margin} style}
%
% This style option is stored in the file |thm.sty|.
%    \begin{macrocode}
%<*thm>
\begingroup \makeatletter
\@ifundefined{theorem@style}{\input{theorem.sty}}{}
\ProvidesFile{thm.sty}
             [\FMithmInfo]
%    \end{macrocode}
%
% \begin{macro}{\th@margin}
%    Again this is only a variant of the theorem styles described
%    above without any new ideas.
%    \begin{macrocode}
\gdef\th@margin{\normalfont\slshape
  \def\@begintheorem##1##2{\item
       [\theorem@headerfont \llap{##2}\hskip\labelsep ##1]}%
\def\@opargbegintheorem##1##2##3{%
  \item[\theorem@headerfont \llap{##2}\hskip\labelsep ##1\ (##3)]}}
\endgroup
%</thm>
%    \end{macrocode}
% \end{macro}
%
%
%
%
% \subsection{Final Definitions}
%
% \begin{macro}{\theorempreskipamount}
% \begin{macro}{\theorempostskipamount}
%    The \textsf{skip} parameters that regulate the vertical empty space
%    before and after the {\envfont theorem} environment have
%    to be allocated as well.
% \changes{v2.0f}{89/07/20}{Unnecessary `global' removed.}
%    \begin{macrocode}
%<*package>
\newskip\theorempreskipamount
\newskip\theorempostskipamount
%    \end{macrocode}
%    Since we have used the same values for all theorem sets, we now can
%    assign them.
%    \begin{macrocode}
\global\setlength\theorempreskipamount{12pt plus 5pt minus 3pt}
\global\setlength\theorempostskipamount{8pt plus 3pt minus 1.5pt}
%    \end{macrocode}
% \end{macro}
% \end{macro}
%
%
% \begin{macro}{\@endtheorem}
%    The same holds for the macro |\@endtheorem|, which ends a
%    {\envfont theorem} environment. Since it is the same for all
%    theorem sets, it is removed from the macros |\th@|\meta{style}.
%    It simply ends the {\envfont trivlist} environment, which was
%    begun in |\@thm|.
%    \begin{macrocode}
\global\let\@endtheorem=\endtrivlist
%    \end{macrocode}
% \end{macro}
%
%
%
% \begin{macro}{\@preamblecmds}
%    All macros defined above are to be used only in the preamble.
%    Therefore, we insert them in |\@preamblecmds| which will disable
%    them at begin document. This is done by the internal \LaTeXe{}
%    command |\@onlypreamble|.
% \changes{v2.0g}{89/09/19}{`theoremheaderfont and `theorembodyfont
%    added.}
% \changes{v2.2a}{94/02/03}{Use \LaTeXe{} interface}
%    \begin{macrocode}
\@onlypreamble\@xnthm
\@onlypreamble\@ynthm
\@onlypreamble\@othm
\@onlypreamble\newtheorem
\@onlypreamble\theoremstyle
\@onlypreamble\theorembodyfont
\@onlypreamble\theoremheaderfont
%    \end{macrocode}
% \end{macro}
%
% Finally we declare the \texttt{plain} theorem style to be the default.
% \changes{v2.1b}{90/06/04}{Fixed missing `item by declaring default
%    style.}
%    \begin{macrocode}
\theoremstyle{plain}
%</package>
%    \end{macrocode}
%
% \Finale

\endinput
}{}
\ProvidesFile{thm.sty}
             [\FMithmInfo]
%    \end{macrocode}
%
% \begin{macro}{\th@margin}
%    Again this is only a variant of the theorem styles described
%    above without any new ideas.
%    \begin{macrocode}
\gdef\th@margin{\normalfont\slshape
  \def\@begintheorem##1##2{\item
       [\theorem@headerfont \llap{##2}\hskip\labelsep ##1]}%
\def\@opargbegintheorem##1##2##3{%
  \item[\theorem@headerfont \llap{##2}\hskip\labelsep ##1\ (##3)]}}
\endgroup
%</thm>
%    \end{macrocode}
% \end{macro}
%
%
%
%
% \subsection{Final Definitions}
%
% \begin{macro}{\theorempreskipamount}
% \begin{macro}{\theorempostskipamount}
%    The \textsf{skip} parameters that regulate the vertical empty space
%    before and after the {\envfont theorem} environment have
%    to be allocated as well.
% \changes{v2.0f}{89/07/20}{Unnecessary `global' removed.}
%    \begin{macrocode}
%<*package>
\newskip\theorempreskipamount
\newskip\theorempostskipamount
%    \end{macrocode}
%    Since we have used the same values for all theorem sets, we now can
%    assign them.
%    \begin{macrocode}
\global\setlength\theorempreskipamount{12pt plus 5pt minus 3pt}
\global\setlength\theorempostskipamount{8pt plus 3pt minus 1.5pt}
%    \end{macrocode}
% \end{macro}
% \end{macro}
%
%
% \begin{macro}{\@endtheorem}
%    The same holds for the macro |\@endtheorem|, which ends a
%    {\envfont theorem} environment. Since it is the same for all
%    theorem sets, it is removed from the macros |\th@|\meta{style}.
%    It simply ends the {\envfont trivlist} environment, which was
%    begun in |\@thm|.
%    \begin{macrocode}
\global\let\@endtheorem=\endtrivlist
%    \end{macrocode}
% \end{macro}
%
%
%
% \begin{macro}{\@preamblecmds}
%    All macros defined above are to be used only in the preamble.
%    Therefore, we insert them in |\@preamblecmds| which will disable
%    them at begin document. This is done by the internal \LaTeXe{}
%    command |\@onlypreamble|.
% \changes{v2.0g}{89/09/19}{`theoremheaderfont and `theorembodyfont
%    added.}
% \changes{v2.2a}{94/02/03}{Use \LaTeXe{} interface}
%    \begin{macrocode}
\@onlypreamble\@xnthm
\@onlypreamble\@ynthm
\@onlypreamble\@othm
\@onlypreamble\newtheorem
\@onlypreamble\theoremstyle
\@onlypreamble\theorembodyfont
\@onlypreamble\theoremheaderfont
%    \end{macrocode}
% \end{macro}
%
% Finally we declare the \texttt{plain} theorem style to be the default.
% \changes{v2.1b}{90/06/04}{Fixed missing `item by declaring default
%    style.}
%    \begin{macrocode}
\theoremstyle{plain}
%</package>
%    \end{macrocode}
%
% \Finale

\endinput
}{}
\ProvidesFile{thb.sty}
             [\FMithmInfo]
%    \end{macrocode}
%
% \begin{macro}{\th@break}
%    |\theoremstyle{break}| produces a line break after the name
%    of the theorem.  The font is |\slshape|.  Hence, we define
%    |\th@break| as follows:
%    \begin{macrocode}
\gdef\th@break{\normalfont\slshape
  \def\@begintheorem##1##2{\item[%
%    \end{macrocode}
%    We run into the following problem: it is not possible to create
%    the header with |\item[|\meta{title}|]| and then start
%    a new line by, for example, |\mbox{}\\|. Such a definition
%    will fail whenever a list environment follows immediately.  With
%    the above construction, the |\mbox{}| causes the switch
%    |@inlabel| (cf.\ definition of |\list| and
%    |\trivlist| in \cite{latex:source}) to be set to \texttt{false}
%    and so the following list will insert additional vertical
%    space (|\topskip|). This is quite annoying. Therefore, we
%    create the line break within the |\item|. In order to ensure
%    that the text will begin at the proper position in the following
%    line, we simply pretend that the label does not take any
%    room.\footnote{This will lead to problems whenever very high
%    symbols occurring in the line tower into the
%    heading. So, something else has to be done here sometime.}
%    \begin{macrocode}
       \rlap{\vbox{\hbox{\hskip \labelsep\theorem@headerfont ##1\ ##2}%
                   \hbox{\strut}}}]}%
%    \end{macrocode}
%    Again, the definition of |\@opargbegintheorem| is
%    completely analogous.
%    \begin{macrocode}
\def\@opargbegintheorem##1##2##3{%
  \item[\rlap{\vbox{\hbox{\hskip \labelsep \theorem@headerfont
                     ##1\ ##2\ (##3)}%
                    \hbox{\strut}}}]}}
\endgroup
%</thb>
%    \end{macrocode}
% \end{macro}
%
%
% \subsubsection{The \texttt{changebreak} style}
%
%    \begin{macrocode}
% This style option is stored in the file |thcb.sty|.
%    \begin{macrocode}
%<*thcb>
\begingroup \makeatletter
\@ifundefined{theorem@style}{% \iffalse meta-comment
%
% Copyright (C) 1993-2024
%
% The LaTeX Project and any individual authors listed elsewhere
% in this file.
%
% This file is part of the Standard LaTeX `Tools Bundle'.
% -------------------------------------------------------
%
% It may be distributed and/or modified under the
% conditions of the LaTeX Project Public License, either version 1.3c
% of this license or (at your option) any later version.
% The latest version of this license is in
%    https://www.latex-project.org/lppl.txt
% and version 1.3c or later is part of all distributions of LaTeX
% version 2005/12/01 or later.
%
% The list of all files belonging to the LaTeX `Tools Bundle' is
% given in the file `manifest.txt'.
%
% \fi
%
%%
%
%
\def\FMithmInfo{2023/07/05 v2.2c Theorem extension package (FMi)}
%
% \ProvidesFile{theorem.dtx}[\FMithmInfo]
%\iffalse   % this is a METACOMMENT !
%%
%% Package `theorem' to use with LaTeX2e
%% Copyright (C) 1989-1995 Frank Mittelbach, all rights reserved.
%<+package>\@ifundefined{theorem@style}{}{\endinput}
%<+package>\NeedsTeXFormat{LaTeX2e}[1995/06/01]
%<+package>\ProvidesPackage{theorem}
%<+package>         [\FMithmInfo]
%
% \fi
%
% \changes{v1.0}{87/05/07}{First draft.}
% \changes{v1.0a}{87/05/07}{`thistheoremstyle' added.}
% \changes{v1.1a}{87/07/29}{New concept: redefine the internal
%                          \protect\LaTeX{} commands `@xnthm' etc.}
% \changes{v1.1b}{87/07/31}{`theoremfont' used to specify font.}
% \changes{v1.2a}{88/01/08}{`@endtheorem' now globally defined.}
% \changes{v1.2b}{88/01/11}{`newtheorem' only allowed in the preamble.}
% \changes{v1.3a}{88/02/14}{`theoremfont' now a toks register.}
%
% \changes{v2.0a}{88/11/16}{New implementation and new documentation.}
% \changes{v2.0a}{88/11/16}{No layout changes but internal modifications
%                           old substyles won't work.}
% \changes{v2.0b}{89/01/30}{`global' used for all definitions.}
% \changes{v2.0c}{89/06/15}{Documentation translated to English.
%                            Thanks to LLL and Nina}
% \changes{v2.1a}{90/02/15}{Support for new font selection.}
% \changes{v2.2a}{94/02/03}{Cleanup for LaTeX2e}
%
% \DoNotIndex{\@endparenv,\@ifdefinable,\@ifnextchar,\@ifundefined}
% \DoNotIndex{\@namedef,\@nameuse,\@spaces,\@tempa,\@warning}
% \DoNotIndex{\ }
% \DoNotIndex{\begingroup,\bfseries}
% \DoNotIndex{\csname}
% \DoNotIndex{\def,\do,\docdate}
% \DoNotIndex{\endcsname,\endgroup,\endinput,\endtrivlist,\expandafter}
% \DoNotIndex{\filedate,\fileversion}
% \DoNotIndex{\gdef,\global}
% \DoNotIndex{\hbox,\hskip}
% \DoNotIndex{\ignorespaces,\input,\itshape,\item}
% \DoNotIndex{\labelsep,\let,\llap}
% \DoNotIndex{\makeatletter}
% \DoNotIndex{\noexpand}
% \DoNotIndex{\refstepcounter,\relax,\rlap}
% \DoNotIndex{\setlength,\slshape,\space,\string,\strut}
% \DoNotIndex{\the,\trivlist,\typeout}
% \DoNotIndex{\vbox}
% \DoNotIndex{\xdef}
%
% \def\deflabel#1{\ttfamily #1\hfill}
% \def\deflist#1{\begin{list}{}{\settowidth\labelwidth{\ttfamily #1}%
%                               \setlength\leftmargin\labelwidth
%                               \addtolength\leftmargin\labelsep
%                               \let\makelabel\deflabel}}
% \def\enddeflist{\end{list}}
% \def\envfont{\normalfont\ttfamily}
%
% \MakeShortVerb{\|}
%
% \newcommand{\TUB}{{\sl TUGboat\/}}
%
% \GetFileInfo{theorem.dtx}
% \title{An Extension of the
%        \LaTeX{} theorem environment\thanks{This file
%        has version number \fileversion, last
%        revised \filedate.}}
% \author{Frank Mittelbach}
%
% \MaintainedByLaTeXTeam{tools}
% \maketitle
%
% \begin{abstract}
%    The macros described in this paper yield an extension of the
%    \LaTeX{} theorem mechanism. It is designed to satisfy the
%    different requirements of various journals. Thus, the
%    layout of the ``theorems'' can be manipulated by determining a
%    ``style''. This article describes not only the use, but
%    also the definition, of the necessary macros.
% \end{abstract}
%
% \section*{Preface to version 2.2}
%
% For \LaTeXe{} this package did not need any fundamental changes. I
% only modified the messages generated so that theorem layout styles
% will show up with the |\listfiles| command and cleaned the section
% on the New Font Selection Scheme since this is now included in
% \LaTeX.
%
% \section*{Preface to version 2.1}
%
% This version is identical to 2.0g described in \TUB\ 10\#3 except
% for some internal defaults which are now set depending on the used
% font selection scheme.
%
% This was done to avoid unpleasant surprises if the new font selection
% scheme is in force. For further details see
% section~\ref{sec:newfonts} and \cite{bk:GMS94}.
%
% \section{Introduction}
%
% For our purposes here, ``theorems'' are labelled enunciations,
% often set off from the main text by extra space and a font change.
% Theorems, corollaries, conjectures, definitions, and remarks
% are all instances of ``theorems''.  The ``header'' of these structures
% is composed of a label (such as \textsc{Theorem} or \textsc{Remark})
% and a number which serializes an item in the sequence of items
% with the same label.
%
% Shortly after the introduction of \LaTeX{} at the Fachbereich
% Mathematik in Mainz, the desire to manipulate the layout of
% ``theorems'' arose.  In Mainz, the following two conventions came
% into general use:
% \begin{enumerate}
%   \item The number of the theorem is shown in the margin.
%   \item There is a line break at
%         the end of the theorem header.
% \end{enumerate}
%
% Additionally, some journals require different formats which depend on
% the ``sort of theorem'': e.g.\ often remarks and definitions are set
% in |\upshape|, while |\itshape| is employed for main theorems.
%
% Confronted with these requirements, a theorem environment was
% developed in Mainz which allows separate determination of the
% layout of the ``theorems sets'', comparable to |\pagestyle|.
%
%
% \section{The user interface}
% \subsection{Defining new theorem sets}
%
% \DescribeMacro\newtheorem
% As in the original \LaTeX{} version, the command |\newtheorem|
% defines a new ``theorem set'' or ``theorem-like structure''.
% Two required arguments name the new environment and give the text
% to be typeset with each instance of the new ``set'', while
% an optional argument determines how
% the ``set'' is enumerated:
% \begin{description}
%    \item[\ttfamily \bslash newtheorem\{foo\}\{bar\}]
%       The theorem set {\envfont foo} (whose name is \texttt{bar})
%       uses its own counter.
%    \item[\ttfamily \bslash newtheorem\{foo2\}{[foo]}\{bar2\}]
%       The theorem set {\envfont foo2} (printed name \texttt{bar2})
%       uses the same counter as the theorem set \texttt{foo}.
%    \item[\ttfamily \bslash newtheorem\{foo3\}\{bar3\}{[section]}]
%       The theorem set {\envfont foo3} (printed name \texttt{bar3}) is
%       enumerated within the counter \texttt{section}, i.e.\ with every
%       new |\section| the enumeration begins again with 1, and
%       the enumeration is composed from the section-number and the
%       theorem counter itself.
% \end{description}
%
% \DescribeMacro\theoremstyle
% Additionally, the command |\theoremstyle| can define the layout
% of various, or all, theorem sets.  It should be noted that any
% theorem set defined by |\newtheorem| is typeset in the
% |\theoremstyle| that is current at the time of the definition.
% Thus, the following
% \begin{verbatim}
%    \theoremstyle{break}        \newtheorem{Cor}{Corollary}
%    \theoremstyle{plain}        \newtheorem{Exa}{Example}[section]
%\end{verbatim}                             ^^A No blank at line start
% leads to the result that the set {\envfont Cor} is formatted in the
% style \texttt{break}, while the set {\envfont Exa} and all the
% following ones are formatted in the style \texttt{plain}, unless
% another |\theoremstyle| follows.  Since the definitions installed by
% |\newtheorem| are global, one also can limit |\theoremstyle| locally
% by grouping braces.
%
% \DescribeMacro\theorembodyfont
% The choice of the font for the theorem body is completely
% independent of the chosen |\theoremstyle|; this has proven to
% be very advantageous.  For example,
% \begin{verbatim}
%   {\theorembodyfont{\upshape}          \newtheorem{Rem}{Remark}}
%\end{verbatim}
% defines a theorem set {\envfont Rem}, which will be set in
% |\upshape| in the current layout (which in our example is
% \texttt{plain}).  As with |\theoremstyle|, the |\theorembodyfont|
% chosen is that current at the time of |\newtheorem|.  If
% |\theorembodyfont| is not specified or one defines
% |\theorembodyfont{}|, then the font used will be that defined by the
% |\theoremstyle|.
%
% \DescribeMacro\theoremheaderfont It is also possible to customize
% the font used for the theorem headers.  This is, however, a global
% declaration, and therefore there should be at most one
% |\theoremheaderfont| declaration in the preamble.\footnote{If it is
% actually necessary to have different header fonts, one has to define
% new theorem styles (substituting the desired font) or specify the
% information directly in the \texttt{\bslash newtheorem} declaration
% (the unclean variant).}
%
% \DescribeMacro\theorempreskipamount
% \DescribeMacro\theorempostskipamount Two additional parameters
% affect the vertical space around the theorem environments:
% |\theorempreskipamount| and |\theorempostskipamount| define,
% respectively, the spacing before and after such an environment.
% These parameters apply for all theorem sets and can be manipulated
% with the ordinary length macros.  They are rubber lengths
% (`\textsf{skips}'), and therefore can contain \texttt{plus} and
% \texttt{minus} parts.
%
% Since the definition of theorem sets should---most sensibly---be
% placed in the preamble, we only allow installation there.
% It is therefore possible to release the memory used here after
% |\begin{document}|, in order to make room for other
% applications.
%
%
% \subsection{Existing theorem styles}
%
% The following theorem styles exist to date:
% \begin{deflist}{marginbreak}
%    \item[plain]
%       This theorem style emulates the original \LaTeX{} definition,
%       except that additionally the parameters
%       |\theorem...skipamount| are used.
%    \item[break]
%       In this style, the theorem header is followed by a line
%       break.
%    \item[marginbreak]
%       The theorem number is set in the margin, and there is a line
%       break as in \texttt{break}.
%    \item[changebreak]
%       Like \texttt{break}, but with header number and text
%       interchanged.
%    \item[change]
%       Header number and text are interchanged, without a line break.
%    \item[margin]
%       The number is set in the left margin, without a line break.
% \end{deflist}
% All styles (except \texttt{plain}) select |\slshape| as the default
% |\theorembodyfont|.
%
%
% \subsection{Examples}
%
% Given the above theorem sets {\envfont Cor}, {\envfont Exa}
% and {\envfont Rem},
% suppose that the preamble also contains the declarations:
% \begin{verbatim}
%    \theoremstyle{marginbreak}   \newtheorem{Lem}[Cor]{Lemma}
%    \theoremstyle{change}
%    \theorembodyfont{\itshape}        \newtheorem{Def}[Cor]{Definition}
%
%    \theoremheaderfont{\scshape}
%\end{verbatim}
% Then the following are some typical examples of the typeset output
% resulting from their use.
%
% \begin{Cor}
%    This is a sentence typeset in the theorem environment {\envfont
%    Cor}.
% \end{Cor}
%
% \begin{Exa}
%    This is a sentence typeset in the theorem environment {\envfont
%    Exa}.
% \end{Exa}
%
% \begin{Rem}
%    This is a sentence typeset in the theorem environment {\envfont
%    Rem}.
% \end{Rem}
%
% \begin{Lem}[Ben User]
%    This is a sentence typeset in the theorem environment {\envfont
%    Lem}.
% \end{Lem}
%
% \begin{Def}[Very impressive Definition]
%    This is a sentence typeset in the theorem environment {\envfont
%    Def}.
% \end{Def}
%
% The last two examples show the effect of the optional argument to
% a theorem environment (it is the text typeset in parentheses).
%
%
% \section{Special Considerations}
% \label{sec:newfonts}
%
% Theoremheader and body are implemented as a unit. This means that
% the |\theoremheaderfont|  will inherit characteristics of the
% |\theorembodyfont| in \LaTeXe. Thus, if for example
% |\theorembodyfont| is |\itshape| and |\theoremheaderfont|
% is |\bfseries| the font selected for the header will have the
% characteristics `bold extended italic'. If this is not desired one
% should set the |\theoremheaderfont| to something like
% \begin{verbatim}
%      \theoremheaderfont{\normalfont\bfseries}
%\end{verbatim}
% i.e.\ supplying all necessary font information explicitly.
%
%
%
% \section{Acknowledgements}
%
% The publication of this set of macros was only possible with the
% help of Christina Busse (translating the manuscript into English),
% Joachim Pense (playing the r\^ole of typist), Chris Rowley
% (looking everything over) and many others providing useful
% suggestions.
%
% \MaybeStop{%
%    \begin{thebibliography}{9}
%       \bibitem{bk:GMS94} \textsc{M.~Goossens}, \textsc{F.~Mittelbach}
%         and \textsc{A.~Samarin}.
%         \newblock The \LaTeX{} Companion.
%         \newblock
%         Addison-Wesley, Reading, Massachusetts, 1994.
%       \bibitem{latex:source} \textsc{Lamport, Leslie.} \newblock
%                               \texttt{latex.tex}, version 2.09, date
%                               Feb. 1990.
%    \end{thebibliography}
%    \PrintIndex
%    \PrintChanges}
%
%
% \section{The documentation driver file}
%
% The next bit of code contains the documentation driver file for
% \TeX{}, i.e., the file that will produce the documentation you are
% currently reading. It will be extracted from this file by the
% \texttt{docstrip} program. Since it is the first code in the file
% one can alternatively process this file directly with \LaTeXe{} to
% obtain the documentation.
%
%    \begin{macrocode}
%<*driver>
\documentclass{ltxdoc}

\usepackage{theorem}

%<+driver>% The next few lines define theorem sets which are used
%<+driver>% in the example section of the documentation.

\theoremstyle{break}       \newtheorem{Cor}{Corollary}
\theoremstyle{plain}       \newtheorem{Exa}{Example}[section]
{\theorembodyfont{\upshape}\newtheorem{Rem}{Remark}}
\theoremstyle{marginbreak} \newtheorem{Lem}[Cor]{Lemma}
\theoremstyle{change}
\theorembodyfont{\itshape} \newtheorem{Def}[Cor]{Definition}

\theoremheaderfont{\scshape}

\RecordChanges

\begin{document}
   \DocInput{theorem.dtx}
\end{document}
%</driver>
%    \end{macrocode}
%
% \section{Definition of the Macros}
%
% If the file has been loaded before, we abort immediately. If not the
% package announces itself (this is actually done at the very top of
% the file---the way it is done isn't good style so don't copy it).
% \changes{v2.0e}{89/07/19}{Spaces removed from `typeout'.}
% \changes{v2.2b}{95/11/19}{Announce moved to top of file}
%    \begin{macrocode}
%<*package>
%\@ifundefined{theorem@style}{}{\endinput}
%\def\FMithmInfo{1995/11/19 v2.2b Theorem extension package (FMi)}
%\ProvidesPackage{theorem}[\FMithmInfo]
%    \end{macrocode}
%
%
% \subsection{Definition of theorem styles and fonts}
%
% All the definitions in this file are done globally to allow
% inputting this file inside a group.
%
% \begin{macro}{\theoremstyle}
%    Before a theorem style can be installed, the chosen style must be
%    known. For that reason, we must test to see that
%    |\th@|\meta{style} is known or, more precisely, that it is
%    different from |\relax|. If the style is not known then
%    |\th@plain| is used.
%    \begin{macrocode}
\gdef\theoremstyle#1{%
   \@ifundefined{th@#1}{\@warning
          {Unknown theoremstyle `#1'. Using `plain'}%
          \theorem@style{plain}}%
%    \end{macrocode}
%    We save the theorem style to be used in the token register
%    |\theorem@style|.
%    \begin{macrocode}
      {\theorem@style{#1}}%
%    \end{macrocode}
%    Now we ``evaluate'' the theorem style: this means, we call the
%    macro |\th@|\meta{style} which will activate the relevant
%    definitions which are contained in a separate file.  This is done
%    in a group to suppress changes to the current font.  This could
%    otherwise pose problems together with the new font selection
%    scheme\footnote{When I printed the original article using the new
%    font selection scheme I ended with a document with slanted
%    typefaces (text headings and all) simply because one of the
%    theorem styles used \texttt{\bslash sl} at toplevel.} if the
%    |\th@|\meta{style} is evaluated a second time.
% \changes{v2.1a}{90/02/15}{Group added around `th@..}
% \changes{v2.1a}{90/02/15}{Reseting of `..@begintheorem removed.}
%    \begin{macrocode}
      \begingroup
        \csname th@\the\theorem@style \endcsname
      \endgroup}
%    \end{macrocode}
% \end{macro}
%
%
% \begin{macro}{\@begintheorem}
% \changes{v2.1a}{90/02/15}{Relaxed at toplevel.}
% \begin{macro}{\@opargbegintheorem}
% \changes{v2.1a}{90/02/15}{Relaxed at toplevel.}
%    We reset |\@begintheorem| and |\@opargbegintheorem| to
%    |\relax| since these commands are no longer necessary at
%    toplevel. This will save a few tokens.
%    \begin{macrocode}
\global\let\@begintheorem\relax
\global\let\@opargbegintheorem\relax
%    \end{macrocode}
% \end{macro}
% \end{macro}
%
% \begin{macro}{\theorem@style}
%    Obviously the \textsf{token} register used above has to be
%    allocated. To assure the utmost compatibility with the
%    original \LaTeX{} definition, we set the default theorem style to
%    \texttt{plain}, which implements the usual \LaTeX{} convention.
% \changes{v2.0f}{89/07/20}{Unnecessary `global' removed.}
%    \begin{macrocode}
\newtoks\theorem@style
\global\theorem@style{plain}
%    \end{macrocode}
% \end{macro}
%
% \begin{macro}{\theorembodyfont}
% \begin{macro}{\theorem@bodyfont}
%    If the |\theorembodyfont| is set by the user then it should not
%    interact with the default font set in the theorem style. When the
%    new font selection is in force this may happen if, for example,
%    the default is |\itshape| and the new |\theorembodyfont| is
%    |\sffamily|. So we add a |\reset@font| command in front of the
%    user definition.
% \changes{v2.0f}{89/07/20}{Unnecessary `global' removed.}
% \changes{v2.0g}{89/09/19}{`theoremfont to `theorembodyfont.}
% \changes{v2.1c}{92/06/22}{`reset@font added and logic changed.}
% \changes{v2.1d}{92/10/28}{Reset to empty if arg is empty}
%    \begin{macrocode}
\gdef\theorembodyfont#1{%
%    \end{macrocode}
%    We check if the argument supplied is empty and if so put nothing
%    into the |\theorem@bodyfont| token register to allow for
%    |\theorembodyfont{}| as a mean of using the default of the
%    current |\theoremstyle|.
%    \begin{macrocode}
   \def\@tempa{#1}%
   \ifx\@tempa\@empty
    \theorem@bodyfont{}%
   \else
    \theorem@bodyfont{\reset@font#1}%
   \fi
   }
\newtoks\theorem@bodyfont
\global\theorem@bodyfont{}
%    \end{macrocode}
% \end{macro}
% \end{macro}
%
%
% \begin{macro}{\theoremheaderfont}
% \changes{v2.0g}{89/09/19}{Macro added.}
%    The font for the theorem headers is handled differently because
%    this definition applies to all theorem styles.
%    \begin{macrocode}
\gdef\theoremheaderfont#1{\gdef\theorem@headerfont{#1}%
%    \end{macrocode}
%    After using the macro once it is redefined to produce an error
%    message.
%    \begin{macrocode}
       \gdef\theoremheaderfont##1{%
        \typeout{\string\theoremheaderfont\space should be used
                 only once.}}}
%    \end{macrocode}
% \end{macro}
%
% \begin{macro}{\theorem@headerfont}
% \changes{v2.0g}{89/09/19}{Macro added.}
%    To set the |\theorem@headerfont| default we first test if
%    the new fontselection scheme is in force.
% \changes{v2.1a}{90/02/15}{Support for new font selection.}
%    \begin{macrocode}
\ifx\upshape\undefined
%    \end{macrocode}
%    If not we define it to expand into |\bfseries|. We don't use
%    |\let| just in case a following style option redefines this
%    macro.
%    \begin{macrocode}
\gdef\theorem@headerfont{\bfseries}
%    \end{macrocode}
%    Otherwise we reset the current shape before calling |\bfseries|.
%    \begin{macrocode}
\else \gdef\theorem@headerfont{\normalfont\bfseries}\fi
%    \end{macrocode}
% \end{macro}
%
% \begin{macro}{\th@plain}
% \begin{macro}{\th@break}
% \begin{macro}{\th@marginbreak}
% \begin{macro}{\th@changebreak}
% \begin{macro}{\th@change}
% \begin{macro}{\th@margin}
%    The different styles are defined in macros such as
%    |\th@plain|.  Since memory space is precious in
%    ``non-Big-versions'', we have to avoid offering too many unused
%    definitions.  Therefore we define these styles in separate files
%    that can be loaded on demand.  Thus the commands themselves only
%    load these files. We use |\@input@|, a \LaTeXe{} internal command
%    that ensures that the file will be listed with |\listfiles|.
% \changes{v2.2a}{94/02/02}{Use `@input@ to load theorem layout files}
%    \begin{macrocode}
\gdef\th@plain{\@input@{thp.sty}}
\gdef\th@break{\@input@{thb.sty}}
\gdef\th@marginbreak{\@input@{thmb.sty}}
\gdef\th@changebreak{\@input@{thcb.sty}}
\gdef\th@change{\@input@{thc.sty}}
\gdef\th@margin{\@input@{thm.sty}}
%    \end{macrocode}
%    This list will be expanded when new styles become
%    available.  For testing,  just append new theorem substyles
%    as document options.
% \end{macro}
% \end{macro}
% \end{macro}
% \end{macro}
% \end{macro}
% \end{macro}
%
%
%
% \subsection{Definition of a new theorem set}
%
% As already pointed out, a new theorem environment can be defined in
% three different ways:
% \begin{verbatim}
%    \newtheorem{Lem}{Lemma}
%    \newtheorem{Lem}{Lemma}[section]
%    \newtheorem{Lem}[Theorem]{Lemma}
%\end{verbatim}
% The function of the macro |\newtheorem| is to recognize these cases
% and then to branch into one of the three macros |\@ynthm|, |\@xnthm|
% or |\@othm|.  This mechanism is adopted unchanged from
% \cite{latex:source}; the essential point here is that, for example,
% in the second case, the arguments \texttt{Lem}, \texttt{Lemma} and
% \texttt{section} are passed over to the macro |\@xnthm|.
%
% We inspect this case first because the others present fewer problems,
% and thus are easily derived from this one.
%
% \begin{macro}{\@xnthm}
%    For our example arguments, the macro |\@xnthm| must fulfill
%    the following:
%    \begin{itemize}
%       \item
%          Define a new \LaTeX-counter `Lem'
%       \item
%          reset this counter within a |\section|
%       \item
%          define the macro |\theLem|
%       \item
%          define the environment macros |\Lem| and
%          |\endLem| using the current
%          |\theoremstyle| and |\theorem@bodyfont|.
%    \end{itemize}
%    Obviously, all this should happen only if the first argument
%    of |\@xnthm| (i.e.\ \texttt{Lem} in our example) is chosen so
%    as not to conflict with any previously defined commands or
%    environments.  This test is performed by the \LaTeX{} macro
%    |\@ifdefinable|.
%    \begin{macrocode}
\gdef\@xnthm#1#2[#3]{\expandafter\@ifdefinable\csname #1\endcsname
%    \end{macrocode}
%    Therefore, the first argument of |\@ifdefinable| is the
%    expansion (in the example, |\Lem|) of
%    |\csname||#1\endcsname|.  The second argument is
%    executed only if the test has been completed successfully.
%    \begin{macrocode}
   {%
%    \end{macrocode}
%    Now we define the new counter. The names of the \LaTeX{}
%    macros employed should speak for themselves:
% \changes{v2.2b}{1995/10/07}{Use \cs{@newctr}}
% \changes{v2.2c}{1995/11/23}{Use \cs{@newctr} properly.}
%    \begin{macrocode}
    \@definecounter{#1}\@newctr{#1}[#3]%
%    \end{macrocode}
%    Using |\@newctr| will give a proper error message if the counter
%    in |#3| is not defined.
%    In defining `|\theLem|' we must generate the desired macro
%    name by use of |\expandafter| and |\csname|.
%    \begin{macrocode}
    \expandafter\xdef\csname the#1\endcsname
%    \end{macrocode}
%    An |\xdef| is used in order to make the definition global,
%    and to ensure that it contains the replacement texts of
%    |\@thmcountersep| and |\@thmcounter|.\footnote{These
%    two macros can be defined by the document style.  Their default
%    values produce a `\texttt{.}' as separation and an arabic
%    representation of the number.} However, not everything should be
%    expanded. For example, it saves space to use |\thesection|
%    instead of its---at times---lengthy expansion.
%    \begin{macrocode}
      {\expandafter \noexpand \csname the#3\endcsname
       \@thmcountersep \@thmcounter{#1}}%
%    \end{macrocode}
%    Thus with the defaults of \LaTeX{}, |\theLem| would be
%    replaced by the command sequence |\thesection.\arabic{Lem}|.
%
%    We will now look at the definition of the macro which is executed
%    at the beginning of the actual environment (in our example this
%    macro is |\Lem|). It should be noted that we use an
%    ``|\expandafter| trick'' to expand only certain parts of the
%    replacement text at the time of the definition.
%    \begin{macrocode}
    \def\@tempa{\global\@namedef{#1}}%
    \expandafter \@tempa \expandafter{%
%    \end{macrocode}
%    First, the macro that contains the current definitions of
%    |\@begintheorem| and |\@opargtheorem| should be called
%    up.  The name of this macro---as is already known---has the form
%    |\th@|\meta{theorem\ style}; therefore, it must be called by
%    \begin{macrocode}
      \csname th@\the \theorem@style
%    \end{macrocode}
%    In addition the default theorem font should be changeable, i.e.\
%    we have to insert the contents of |\theorem@bodyfont|. For
%    that reason, we expand even further, beyond |\endcsname|,
%    and thus insert the contents of the \textsf{token} register
%    |\theorem@bodyfont| in the replacement text.
%    \begin{macrocode}
            \expandafter \endcsname \the \theorem@bodyfont
%    \end{macrocode}
%    Now it is time to call the macro |\@thm| which takes over
%    the further processing.  It has two arguments: the current
%    counter name (in our example, \texttt{Lem}), and the text of the
%    label (in our example, \texttt{Lemma}).
%    \begin{macrocode}
     \@thm{#1}{#2}}%
%    \end{macrocode}
%    With this, the `sub-definition' is complete.  The macro
%    |\@endtheorem| ends a theorem environment and is, so far, nothing
%    but an |\endtrivlist|.  (Hence it is defined globally, and not
%    within the theorem styles.\footnote{This has to be changed as
%    soon as theorem styles that change \texttt{\bslash @endtheorem}
%    exist.  In such a case, all existing styles must be changed as
%    well since they will have to reset the macro.}) Therefore, we can
%    set it equivalent to the macro that ends the theorem set (in our
%    example, |\endLem|).  However, if some day theorem styles exist
%    that do change |\@endtheorem|, we would have to use the
%    commented-out line instead.
%    \begin{macrocode}
    \global \expandafter \let \csname end#1\endcsname \@endtheorem
%   \global\@namedef{end#1}{\@endtheorem}%
%    \end{macrocode}
%    With these commands all the required definitions are employed,
%    unless the test |\@ifdefinable| has failed.  Therefore, we
%    end the second argument of this macro and with it the definition
%    of |\@xnthm|.
%    \begin{macrocode}
   }}
%    \end{macrocode}
% \end{macro}
%
% \begin{macro}{\@ynthm}
%    The definition of |\@ynthm| is completely analogous. In
%    this case the new counter that is defined is not reset within
%    another counter; thus the definition of |\the...|
%    is simplified:
%    \begin{macrocode}
\gdef\@ynthm#1#2{\expandafter\@ifdefinable\csname #1\endcsname
   {\@definecounter{#1}%
    \expandafter\xdef\csname the#1\endcsname{\@thmcounter{#1}}%
%    \end{macrocode}
%    The rest of the definition corresponds literally to that of
%    |\@xnthm|:
%    \begin{macrocode}
    \def\@tempa{\global\@namedef{#1}}\expandafter \@tempa
     \expandafter{\csname th@\the \theorem@style \expandafter
     \endcsname \the\theorem@bodyfont \@thm{#1}{#2}}%
    \global \expandafter \let \csname end#1\endcsname \@endtheorem}}
%    \end{macrocode}
% \end{macro}
%
% \begin{macro}{\@othm}
%    The definition of |\@othm| does not contain anything new.
%    \begin{macrocode}
\gdef\@othm#1[#2]#3{%
%    \end{macrocode}
% \changes{v2.2b}{95/10/07}{Build in check for counter defined}
%    We do not define a new counter but instead use one that has
%    already been defined.  Thus the only definition we need is that
%    of this pseudo-counter (i.e.\ |\the|\meta{env.\ name}).
%    First we check if |#2| corresponds to a known counter name.
%    \begin{macrocode}
  \expandafter\ifx\csname c@#2\endcsname\relax
   \@nocounterr{#2}%
  \else
   \expandafter\@ifdefinable\csname #1\endcsname
   {\expandafter \xdef \csname the#1\endcsname
     {\expandafter \noexpand \csname the#2\endcsname}%
%    \end{macrocode}
%    All other parts of the definition can be adopted from
%    |\@xnthm|. We have to remember, though, that in this case
%    the name of the current counter and the theorem label have moved
%    to the second and third arguments.
%    \begin{macrocode}
    \def\@tempa{\global\@namedef{#1}}\expandafter \@tempa
     \expandafter{\csname th@\the \theorem@style \expandafter
     \endcsname \the\theorem@bodyfont \@thm{#2}{#3}}%
    \global \expandafter \let \csname end#1\endcsname \@endtheorem}%
  \fi}
%    \end{macrocode}
% \end{macro}
%
% \subsection{Macros that are employed in a {\envfont theorem}
%    environment}
%
% \begin{macro}{\@thm}
%    The macro |\@thm| has to increase the current counter.  Then,
%    depending on whether the environment has (or does not have) an
%    optional argument, it has to branch into either |\@begintheorem|
%    or |\@opargtheorem|.
%    \begin{macrocode}
\gdef\@thm#1#2{\refstepcounter{#1}%
%    \end{macrocode}
%    Now we start a {\envfont trivlist} environment, and give
%    |\@topsep| and |\@topsepadd| the values of the \textsf{skip}
%    registers |\theorempreskipamount| and |\theorempostskipamount|.
%    The value in |\@topsep| is the vertical space that is inserted by
%    the first (and only) |\item| in our |\trivlist| whilst
%    |\@topsepadd| is inserted by |\@endparenv| at the end of that
%    {\envfont trivlist} environment.  By using these registers, we
%    obtain the desired space around a {\envfont theorem} environment.
%    \begin{macrocode}
   \trivlist
   \@topsep \theorempreskipamount               % used by first \item
   \@topsepadd \theorempostskipamount           % used by \@endparenv
%    \end{macrocode}
%    Now we have to test whether an optional argument has been given.
% \changes{v2.0d}{89/07/18}{Forgotten percent added.}
%    \begin{macrocode}
   \@ifnextchar [%
%    \end{macrocode}
%    If there is an optional argument, we will call |\@ythm|, and
%    move the arguments read back into the input stream.
%    \begin{macrocode}
   {\@ythm{#1}{#2}}%
%    \end{macrocode}
%    If not, we call |\@begintheorem|. Its first argument is the
%    name of the theorem set (hence the second argument of
%    |\@thm|). Its second argument is the macro that produces the
%    current number.
%    \begin{macrocode}
   {\@begintheorem{#2}{\csname the#1\endcsname}\ignorespaces}}
%    \end{macrocode}
% \end{macro}
%
% \begin{macro}{\@xthm}
% \begin{macro}{\@ythm}
%    Both these macros were originally called by |\@thm|.  We do
%    not need |\@xthm| anymore, hence we reset it to
%    |\relax|. The definition of |\@ythm| has not changed at
%    all from its definition in \LaTeX. In order to make the macros
%    easier to understand, we will
%    nevertheless present it (commented out).
%    \begin{macrocode}
\global\let\@xthm\relax
% \def\@ythm#1#2[#3]{\@opargbegintheorem{#2}{\csname
%           the#1\endcsname}{#3}\ignorespaces}
%</package>
%    \end{macrocode}
%    The primitive |\ignorespaces| in |\@ythm| and
%    |\@thm| is needed to remove the spaces between the
%    |\begin{...}| and the actual text.
% \end{macro}
% \end{macro}
%
%
% \subsection{Definition of the theorem substyles}
%
% As already pointed out, the theorem substyles, defined below, are
% only loaded when necessary.  Note that all these substyles, except
% \texttt{plain}, have |\slshape| as the default body font.
%
% \subsubsection{The \texttt{plain} style}
%
% \label{dok:th@plain} As the following macros use |@|, we have
% to locally set the |\catcode| of this symbol to ``letter''.
% This happens within a group, so that we do not have to worry about
% which |\catcode| that symbol had before.
%    \begin{macrocode}
%<*thp>
\begingroup \makeatletter
%    \end{macrocode}
% Since we are now within a group, we must make all definitions
% globally.  First we make sure that \texttt{theorem.sty} is loaded.
% This will allow us to use this file as a document style option
% without having to call \texttt{theorem} itself as an option.  At the
% same time, we assure that at least version 2 is loaded, since
% |\theorem@style| was not defined in earlier versions.
%    \begin{macrocode}
\@ifundefined{theorem@style}{% \iffalse meta-comment
%
% Copyright (C) 1993-2024
%
% The LaTeX Project and any individual authors listed elsewhere
% in this file.
%
% This file is part of the Standard LaTeX `Tools Bundle'.
% -------------------------------------------------------
%
% It may be distributed and/or modified under the
% conditions of the LaTeX Project Public License, either version 1.3c
% of this license or (at your option) any later version.
% The latest version of this license is in
%    https://www.latex-project.org/lppl.txt
% and version 1.3c or later is part of all distributions of LaTeX
% version 2005/12/01 or later.
%
% The list of all files belonging to the LaTeX `Tools Bundle' is
% given in the file `manifest.txt'.
%
% \fi
%
%%
%
%
\def\FMithmInfo{2023/07/05 v2.2c Theorem extension package (FMi)}
%
% \ProvidesFile{theorem.dtx}[\FMithmInfo]
%\iffalse   % this is a METACOMMENT !
%%
%% Package `theorem' to use with LaTeX2e
%% Copyright (C) 1989-1995 Frank Mittelbach, all rights reserved.
%<+package>\@ifundefined{theorem@style}{}{\endinput}
%<+package>\NeedsTeXFormat{LaTeX2e}[1995/06/01]
%<+package>\ProvidesPackage{theorem}
%<+package>         [\FMithmInfo]
%
% \fi
%
% \changes{v1.0}{87/05/07}{First draft.}
% \changes{v1.0a}{87/05/07}{`thistheoremstyle' added.}
% \changes{v1.1a}{87/07/29}{New concept: redefine the internal
%                          \protect\LaTeX{} commands `@xnthm' etc.}
% \changes{v1.1b}{87/07/31}{`theoremfont' used to specify font.}
% \changes{v1.2a}{88/01/08}{`@endtheorem' now globally defined.}
% \changes{v1.2b}{88/01/11}{`newtheorem' only allowed in the preamble.}
% \changes{v1.3a}{88/02/14}{`theoremfont' now a toks register.}
%
% \changes{v2.0a}{88/11/16}{New implementation and new documentation.}
% \changes{v2.0a}{88/11/16}{No layout changes but internal modifications
%                           old substyles won't work.}
% \changes{v2.0b}{89/01/30}{`global' used for all definitions.}
% \changes{v2.0c}{89/06/15}{Documentation translated to English.
%                            Thanks to LLL and Nina}
% \changes{v2.1a}{90/02/15}{Support for new font selection.}
% \changes{v2.2a}{94/02/03}{Cleanup for LaTeX2e}
%
% \DoNotIndex{\@endparenv,\@ifdefinable,\@ifnextchar,\@ifundefined}
% \DoNotIndex{\@namedef,\@nameuse,\@spaces,\@tempa,\@warning}
% \DoNotIndex{\ }
% \DoNotIndex{\begingroup,\bfseries}
% \DoNotIndex{\csname}
% \DoNotIndex{\def,\do,\docdate}
% \DoNotIndex{\endcsname,\endgroup,\endinput,\endtrivlist,\expandafter}
% \DoNotIndex{\filedate,\fileversion}
% \DoNotIndex{\gdef,\global}
% \DoNotIndex{\hbox,\hskip}
% \DoNotIndex{\ignorespaces,\input,\itshape,\item}
% \DoNotIndex{\labelsep,\let,\llap}
% \DoNotIndex{\makeatletter}
% \DoNotIndex{\noexpand}
% \DoNotIndex{\refstepcounter,\relax,\rlap}
% \DoNotIndex{\setlength,\slshape,\space,\string,\strut}
% \DoNotIndex{\the,\trivlist,\typeout}
% \DoNotIndex{\vbox}
% \DoNotIndex{\xdef}
%
% \def\deflabel#1{\ttfamily #1\hfill}
% \def\deflist#1{\begin{list}{}{\settowidth\labelwidth{\ttfamily #1}%
%                               \setlength\leftmargin\labelwidth
%                               \addtolength\leftmargin\labelsep
%                               \let\makelabel\deflabel}}
% \def\enddeflist{\end{list}}
% \def\envfont{\normalfont\ttfamily}
%
% \MakeShortVerb{\|}
%
% \newcommand{\TUB}{{\sl TUGboat\/}}
%
% \GetFileInfo{theorem.dtx}
% \title{An Extension of the
%        \LaTeX{} theorem environment\thanks{This file
%        has version number \fileversion, last
%        revised \filedate.}}
% \author{Frank Mittelbach}
%
% \MaintainedByLaTeXTeam{tools}
% \maketitle
%
% \begin{abstract}
%    The macros described in this paper yield an extension of the
%    \LaTeX{} theorem mechanism. It is designed to satisfy the
%    different requirements of various journals. Thus, the
%    layout of the ``theorems'' can be manipulated by determining a
%    ``style''. This article describes not only the use, but
%    also the definition, of the necessary macros.
% \end{abstract}
%
% \section*{Preface to version 2.2}
%
% For \LaTeXe{} this package did not need any fundamental changes. I
% only modified the messages generated so that theorem layout styles
% will show up with the |\listfiles| command and cleaned the section
% on the New Font Selection Scheme since this is now included in
% \LaTeX.
%
% \section*{Preface to version 2.1}
%
% This version is identical to 2.0g described in \TUB\ 10\#3 except
% for some internal defaults which are now set depending on the used
% font selection scheme.
%
% This was done to avoid unpleasant surprises if the new font selection
% scheme is in force. For further details see
% section~\ref{sec:newfonts} and \cite{bk:GMS94}.
%
% \section{Introduction}
%
% For our purposes here, ``theorems'' are labelled enunciations,
% often set off from the main text by extra space and a font change.
% Theorems, corollaries, conjectures, definitions, and remarks
% are all instances of ``theorems''.  The ``header'' of these structures
% is composed of a label (such as \textsc{Theorem} or \textsc{Remark})
% and a number which serializes an item in the sequence of items
% with the same label.
%
% Shortly after the introduction of \LaTeX{} at the Fachbereich
% Mathematik in Mainz, the desire to manipulate the layout of
% ``theorems'' arose.  In Mainz, the following two conventions came
% into general use:
% \begin{enumerate}
%   \item The number of the theorem is shown in the margin.
%   \item There is a line break at
%         the end of the theorem header.
% \end{enumerate}
%
% Additionally, some journals require different formats which depend on
% the ``sort of theorem'': e.g.\ often remarks and definitions are set
% in |\upshape|, while |\itshape| is employed for main theorems.
%
% Confronted with these requirements, a theorem environment was
% developed in Mainz which allows separate determination of the
% layout of the ``theorems sets'', comparable to |\pagestyle|.
%
%
% \section{The user interface}
% \subsection{Defining new theorem sets}
%
% \DescribeMacro\newtheorem
% As in the original \LaTeX{} version, the command |\newtheorem|
% defines a new ``theorem set'' or ``theorem-like structure''.
% Two required arguments name the new environment and give the text
% to be typeset with each instance of the new ``set'', while
% an optional argument determines how
% the ``set'' is enumerated:
% \begin{description}
%    \item[\ttfamily \bslash newtheorem\{foo\}\{bar\}]
%       The theorem set {\envfont foo} (whose name is \texttt{bar})
%       uses its own counter.
%    \item[\ttfamily \bslash newtheorem\{foo2\}{[foo]}\{bar2\}]
%       The theorem set {\envfont foo2} (printed name \texttt{bar2})
%       uses the same counter as the theorem set \texttt{foo}.
%    \item[\ttfamily \bslash newtheorem\{foo3\}\{bar3\}{[section]}]
%       The theorem set {\envfont foo3} (printed name \texttt{bar3}) is
%       enumerated within the counter \texttt{section}, i.e.\ with every
%       new |\section| the enumeration begins again with 1, and
%       the enumeration is composed from the section-number and the
%       theorem counter itself.
% \end{description}
%
% \DescribeMacro\theoremstyle
% Additionally, the command |\theoremstyle| can define the layout
% of various, or all, theorem sets.  It should be noted that any
% theorem set defined by |\newtheorem| is typeset in the
% |\theoremstyle| that is current at the time of the definition.
% Thus, the following
% \begin{verbatim}
%    \theoremstyle{break}        \newtheorem{Cor}{Corollary}
%    \theoremstyle{plain}        \newtheorem{Exa}{Example}[section]
%\end{verbatim}                             ^^A No blank at line start
% leads to the result that the set {\envfont Cor} is formatted in the
% style \texttt{break}, while the set {\envfont Exa} and all the
% following ones are formatted in the style \texttt{plain}, unless
% another |\theoremstyle| follows.  Since the definitions installed by
% |\newtheorem| are global, one also can limit |\theoremstyle| locally
% by grouping braces.
%
% \DescribeMacro\theorembodyfont
% The choice of the font for the theorem body is completely
% independent of the chosen |\theoremstyle|; this has proven to
% be very advantageous.  For example,
% \begin{verbatim}
%   {\theorembodyfont{\upshape}          \newtheorem{Rem}{Remark}}
%\end{verbatim}
% defines a theorem set {\envfont Rem}, which will be set in
% |\upshape| in the current layout (which in our example is
% \texttt{plain}).  As with |\theoremstyle|, the |\theorembodyfont|
% chosen is that current at the time of |\newtheorem|.  If
% |\theorembodyfont| is not specified or one defines
% |\theorembodyfont{}|, then the font used will be that defined by the
% |\theoremstyle|.
%
% \DescribeMacro\theoremheaderfont It is also possible to customize
% the font used for the theorem headers.  This is, however, a global
% declaration, and therefore there should be at most one
% |\theoremheaderfont| declaration in the preamble.\footnote{If it is
% actually necessary to have different header fonts, one has to define
% new theorem styles (substituting the desired font) or specify the
% information directly in the \texttt{\bslash newtheorem} declaration
% (the unclean variant).}
%
% \DescribeMacro\theorempreskipamount
% \DescribeMacro\theorempostskipamount Two additional parameters
% affect the vertical space around the theorem environments:
% |\theorempreskipamount| and |\theorempostskipamount| define,
% respectively, the spacing before and after such an environment.
% These parameters apply for all theorem sets and can be manipulated
% with the ordinary length macros.  They are rubber lengths
% (`\textsf{skips}'), and therefore can contain \texttt{plus} and
% \texttt{minus} parts.
%
% Since the definition of theorem sets should---most sensibly---be
% placed in the preamble, we only allow installation there.
% It is therefore possible to release the memory used here after
% |\begin{document}|, in order to make room for other
% applications.
%
%
% \subsection{Existing theorem styles}
%
% The following theorem styles exist to date:
% \begin{deflist}{marginbreak}
%    \item[plain]
%       This theorem style emulates the original \LaTeX{} definition,
%       except that additionally the parameters
%       |\theorem...skipamount| are used.
%    \item[break]
%       In this style, the theorem header is followed by a line
%       break.
%    \item[marginbreak]
%       The theorem number is set in the margin, and there is a line
%       break as in \texttt{break}.
%    \item[changebreak]
%       Like \texttt{break}, but with header number and text
%       interchanged.
%    \item[change]
%       Header number and text are interchanged, without a line break.
%    \item[margin]
%       The number is set in the left margin, without a line break.
% \end{deflist}
% All styles (except \texttt{plain}) select |\slshape| as the default
% |\theorembodyfont|.
%
%
% \subsection{Examples}
%
% Given the above theorem sets {\envfont Cor}, {\envfont Exa}
% and {\envfont Rem},
% suppose that the preamble also contains the declarations:
% \begin{verbatim}
%    \theoremstyle{marginbreak}   \newtheorem{Lem}[Cor]{Lemma}
%    \theoremstyle{change}
%    \theorembodyfont{\itshape}        \newtheorem{Def}[Cor]{Definition}
%
%    \theoremheaderfont{\scshape}
%\end{verbatim}
% Then the following are some typical examples of the typeset output
% resulting from their use.
%
% \begin{Cor}
%    This is a sentence typeset in the theorem environment {\envfont
%    Cor}.
% \end{Cor}
%
% \begin{Exa}
%    This is a sentence typeset in the theorem environment {\envfont
%    Exa}.
% \end{Exa}
%
% \begin{Rem}
%    This is a sentence typeset in the theorem environment {\envfont
%    Rem}.
% \end{Rem}
%
% \begin{Lem}[Ben User]
%    This is a sentence typeset in the theorem environment {\envfont
%    Lem}.
% \end{Lem}
%
% \begin{Def}[Very impressive Definition]
%    This is a sentence typeset in the theorem environment {\envfont
%    Def}.
% \end{Def}
%
% The last two examples show the effect of the optional argument to
% a theorem environment (it is the text typeset in parentheses).
%
%
% \section{Special Considerations}
% \label{sec:newfonts}
%
% Theoremheader and body are implemented as a unit. This means that
% the |\theoremheaderfont|  will inherit characteristics of the
% |\theorembodyfont| in \LaTeXe. Thus, if for example
% |\theorembodyfont| is |\itshape| and |\theoremheaderfont|
% is |\bfseries| the font selected for the header will have the
% characteristics `bold extended italic'. If this is not desired one
% should set the |\theoremheaderfont| to something like
% \begin{verbatim}
%      \theoremheaderfont{\normalfont\bfseries}
%\end{verbatim}
% i.e.\ supplying all necessary font information explicitly.
%
%
%
% \section{Acknowledgements}
%
% The publication of this set of macros was only possible with the
% help of Christina Busse (translating the manuscript into English),
% Joachim Pense (playing the r\^ole of typist), Chris Rowley
% (looking everything over) and many others providing useful
% suggestions.
%
% \MaybeStop{%
%    \begin{thebibliography}{9}
%       \bibitem{bk:GMS94} \textsc{M.~Goossens}, \textsc{F.~Mittelbach}
%         and \textsc{A.~Samarin}.
%         \newblock The \LaTeX{} Companion.
%         \newblock
%         Addison-Wesley, Reading, Massachusetts, 1994.
%       \bibitem{latex:source} \textsc{Lamport, Leslie.} \newblock
%                               \texttt{latex.tex}, version 2.09, date
%                               Feb. 1990.
%    \end{thebibliography}
%    \PrintIndex
%    \PrintChanges}
%
%
% \section{The documentation driver file}
%
% The next bit of code contains the documentation driver file for
% \TeX{}, i.e., the file that will produce the documentation you are
% currently reading. It will be extracted from this file by the
% \texttt{docstrip} program. Since it is the first code in the file
% one can alternatively process this file directly with \LaTeXe{} to
% obtain the documentation.
%
%    \begin{macrocode}
%<*driver>
\documentclass{ltxdoc}

\usepackage{theorem}

%<+driver>% The next few lines define theorem sets which are used
%<+driver>% in the example section of the documentation.

\theoremstyle{break}       \newtheorem{Cor}{Corollary}
\theoremstyle{plain}       \newtheorem{Exa}{Example}[section]
{\theorembodyfont{\upshape}\newtheorem{Rem}{Remark}}
\theoremstyle{marginbreak} \newtheorem{Lem}[Cor]{Lemma}
\theoremstyle{change}
\theorembodyfont{\itshape} \newtheorem{Def}[Cor]{Definition}

\theoremheaderfont{\scshape}

\RecordChanges

\begin{document}
   \DocInput{theorem.dtx}
\end{document}
%</driver>
%    \end{macrocode}
%
% \section{Definition of the Macros}
%
% If the file has been loaded before, we abort immediately. If not the
% package announces itself (this is actually done at the very top of
% the file---the way it is done isn't good style so don't copy it).
% \changes{v2.0e}{89/07/19}{Spaces removed from `typeout'.}
% \changes{v2.2b}{95/11/19}{Announce moved to top of file}
%    \begin{macrocode}
%<*package>
%\@ifundefined{theorem@style}{}{\endinput}
%\def\FMithmInfo{1995/11/19 v2.2b Theorem extension package (FMi)}
%\ProvidesPackage{theorem}[\FMithmInfo]
%    \end{macrocode}
%
%
% \subsection{Definition of theorem styles and fonts}
%
% All the definitions in this file are done globally to allow
% inputting this file inside a group.
%
% \begin{macro}{\theoremstyle}
%    Before a theorem style can be installed, the chosen style must be
%    known. For that reason, we must test to see that
%    |\th@|\meta{style} is known or, more precisely, that it is
%    different from |\relax|. If the style is not known then
%    |\th@plain| is used.
%    \begin{macrocode}
\gdef\theoremstyle#1{%
   \@ifundefined{th@#1}{\@warning
          {Unknown theoremstyle `#1'. Using `plain'}%
          \theorem@style{plain}}%
%    \end{macrocode}
%    We save the theorem style to be used in the token register
%    |\theorem@style|.
%    \begin{macrocode}
      {\theorem@style{#1}}%
%    \end{macrocode}
%    Now we ``evaluate'' the theorem style: this means, we call the
%    macro |\th@|\meta{style} which will activate the relevant
%    definitions which are contained in a separate file.  This is done
%    in a group to suppress changes to the current font.  This could
%    otherwise pose problems together with the new font selection
%    scheme\footnote{When I printed the original article using the new
%    font selection scheme I ended with a document with slanted
%    typefaces (text headings and all) simply because one of the
%    theorem styles used \texttt{\bslash sl} at toplevel.} if the
%    |\th@|\meta{style} is evaluated a second time.
% \changes{v2.1a}{90/02/15}{Group added around `th@..}
% \changes{v2.1a}{90/02/15}{Reseting of `..@begintheorem removed.}
%    \begin{macrocode}
      \begingroup
        \csname th@\the\theorem@style \endcsname
      \endgroup}
%    \end{macrocode}
% \end{macro}
%
%
% \begin{macro}{\@begintheorem}
% \changes{v2.1a}{90/02/15}{Relaxed at toplevel.}
% \begin{macro}{\@opargbegintheorem}
% \changes{v2.1a}{90/02/15}{Relaxed at toplevel.}
%    We reset |\@begintheorem| and |\@opargbegintheorem| to
%    |\relax| since these commands are no longer necessary at
%    toplevel. This will save a few tokens.
%    \begin{macrocode}
\global\let\@begintheorem\relax
\global\let\@opargbegintheorem\relax
%    \end{macrocode}
% \end{macro}
% \end{macro}
%
% \begin{macro}{\theorem@style}
%    Obviously the \textsf{token} register used above has to be
%    allocated. To assure the utmost compatibility with the
%    original \LaTeX{} definition, we set the default theorem style to
%    \texttt{plain}, which implements the usual \LaTeX{} convention.
% \changes{v2.0f}{89/07/20}{Unnecessary `global' removed.}
%    \begin{macrocode}
\newtoks\theorem@style
\global\theorem@style{plain}
%    \end{macrocode}
% \end{macro}
%
% \begin{macro}{\theorembodyfont}
% \begin{macro}{\theorem@bodyfont}
%    If the |\theorembodyfont| is set by the user then it should not
%    interact with the default font set in the theorem style. When the
%    new font selection is in force this may happen if, for example,
%    the default is |\itshape| and the new |\theorembodyfont| is
%    |\sffamily|. So we add a |\reset@font| command in front of the
%    user definition.
% \changes{v2.0f}{89/07/20}{Unnecessary `global' removed.}
% \changes{v2.0g}{89/09/19}{`theoremfont to `theorembodyfont.}
% \changes{v2.1c}{92/06/22}{`reset@font added and logic changed.}
% \changes{v2.1d}{92/10/28}{Reset to empty if arg is empty}
%    \begin{macrocode}
\gdef\theorembodyfont#1{%
%    \end{macrocode}
%    We check if the argument supplied is empty and if so put nothing
%    into the |\theorem@bodyfont| token register to allow for
%    |\theorembodyfont{}| as a mean of using the default of the
%    current |\theoremstyle|.
%    \begin{macrocode}
   \def\@tempa{#1}%
   \ifx\@tempa\@empty
    \theorem@bodyfont{}%
   \else
    \theorem@bodyfont{\reset@font#1}%
   \fi
   }
\newtoks\theorem@bodyfont
\global\theorem@bodyfont{}
%    \end{macrocode}
% \end{macro}
% \end{macro}
%
%
% \begin{macro}{\theoremheaderfont}
% \changes{v2.0g}{89/09/19}{Macro added.}
%    The font for the theorem headers is handled differently because
%    this definition applies to all theorem styles.
%    \begin{macrocode}
\gdef\theoremheaderfont#1{\gdef\theorem@headerfont{#1}%
%    \end{macrocode}
%    After using the macro once it is redefined to produce an error
%    message.
%    \begin{macrocode}
       \gdef\theoremheaderfont##1{%
        \typeout{\string\theoremheaderfont\space should be used
                 only once.}}}
%    \end{macrocode}
% \end{macro}
%
% \begin{macro}{\theorem@headerfont}
% \changes{v2.0g}{89/09/19}{Macro added.}
%    To set the |\theorem@headerfont| default we first test if
%    the new fontselection scheme is in force.
% \changes{v2.1a}{90/02/15}{Support for new font selection.}
%    \begin{macrocode}
\ifx\upshape\undefined
%    \end{macrocode}
%    If not we define it to expand into |\bfseries|. We don't use
%    |\let| just in case a following style option redefines this
%    macro.
%    \begin{macrocode}
\gdef\theorem@headerfont{\bfseries}
%    \end{macrocode}
%    Otherwise we reset the current shape before calling |\bfseries|.
%    \begin{macrocode}
\else \gdef\theorem@headerfont{\normalfont\bfseries}\fi
%    \end{macrocode}
% \end{macro}
%
% \begin{macro}{\th@plain}
% \begin{macro}{\th@break}
% \begin{macro}{\th@marginbreak}
% \begin{macro}{\th@changebreak}
% \begin{macro}{\th@change}
% \begin{macro}{\th@margin}
%    The different styles are defined in macros such as
%    |\th@plain|.  Since memory space is precious in
%    ``non-Big-versions'', we have to avoid offering too many unused
%    definitions.  Therefore we define these styles in separate files
%    that can be loaded on demand.  Thus the commands themselves only
%    load these files. We use |\@input@|, a \LaTeXe{} internal command
%    that ensures that the file will be listed with |\listfiles|.
% \changes{v2.2a}{94/02/02}{Use `@input@ to load theorem layout files}
%    \begin{macrocode}
\gdef\th@plain{\@input@{thp.sty}}
\gdef\th@break{\@input@{thb.sty}}
\gdef\th@marginbreak{\@input@{thmb.sty}}
\gdef\th@changebreak{\@input@{thcb.sty}}
\gdef\th@change{\@input@{thc.sty}}
\gdef\th@margin{\@input@{thm.sty}}
%    \end{macrocode}
%    This list will be expanded when new styles become
%    available.  For testing,  just append new theorem substyles
%    as document options.
% \end{macro}
% \end{macro}
% \end{macro}
% \end{macro}
% \end{macro}
% \end{macro}
%
%
%
% \subsection{Definition of a new theorem set}
%
% As already pointed out, a new theorem environment can be defined in
% three different ways:
% \begin{verbatim}
%    \newtheorem{Lem}{Lemma}
%    \newtheorem{Lem}{Lemma}[section]
%    \newtheorem{Lem}[Theorem]{Lemma}
%\end{verbatim}
% The function of the macro |\newtheorem| is to recognize these cases
% and then to branch into one of the three macros |\@ynthm|, |\@xnthm|
% or |\@othm|.  This mechanism is adopted unchanged from
% \cite{latex:source}; the essential point here is that, for example,
% in the second case, the arguments \texttt{Lem}, \texttt{Lemma} and
% \texttt{section} are passed over to the macro |\@xnthm|.
%
% We inspect this case first because the others present fewer problems,
% and thus are easily derived from this one.
%
% \begin{macro}{\@xnthm}
%    For our example arguments, the macro |\@xnthm| must fulfill
%    the following:
%    \begin{itemize}
%       \item
%          Define a new \LaTeX-counter `Lem'
%       \item
%          reset this counter within a |\section|
%       \item
%          define the macro |\theLem|
%       \item
%          define the environment macros |\Lem| and
%          |\endLem| using the current
%          |\theoremstyle| and |\theorem@bodyfont|.
%    \end{itemize}
%    Obviously, all this should happen only if the first argument
%    of |\@xnthm| (i.e.\ \texttt{Lem} in our example) is chosen so
%    as not to conflict with any previously defined commands or
%    environments.  This test is performed by the \LaTeX{} macro
%    |\@ifdefinable|.
%    \begin{macrocode}
\gdef\@xnthm#1#2[#3]{\expandafter\@ifdefinable\csname #1\endcsname
%    \end{macrocode}
%    Therefore, the first argument of |\@ifdefinable| is the
%    expansion (in the example, |\Lem|) of
%    |\csname||#1\endcsname|.  The second argument is
%    executed only if the test has been completed successfully.
%    \begin{macrocode}
   {%
%    \end{macrocode}
%    Now we define the new counter. The names of the \LaTeX{}
%    macros employed should speak for themselves:
% \changes{v2.2b}{1995/10/07}{Use \cs{@newctr}}
% \changes{v2.2c}{1995/11/23}{Use \cs{@newctr} properly.}
%    \begin{macrocode}
    \@definecounter{#1}\@newctr{#1}[#3]%
%    \end{macrocode}
%    Using |\@newctr| will give a proper error message if the counter
%    in |#3| is not defined.
%    In defining `|\theLem|' we must generate the desired macro
%    name by use of |\expandafter| and |\csname|.
%    \begin{macrocode}
    \expandafter\xdef\csname the#1\endcsname
%    \end{macrocode}
%    An |\xdef| is used in order to make the definition global,
%    and to ensure that it contains the replacement texts of
%    |\@thmcountersep| and |\@thmcounter|.\footnote{These
%    two macros can be defined by the document style.  Their default
%    values produce a `\texttt{.}' as separation and an arabic
%    representation of the number.} However, not everything should be
%    expanded. For example, it saves space to use |\thesection|
%    instead of its---at times---lengthy expansion.
%    \begin{macrocode}
      {\expandafter \noexpand \csname the#3\endcsname
       \@thmcountersep \@thmcounter{#1}}%
%    \end{macrocode}
%    Thus with the defaults of \LaTeX{}, |\theLem| would be
%    replaced by the command sequence |\thesection.\arabic{Lem}|.
%
%    We will now look at the definition of the macro which is executed
%    at the beginning of the actual environment (in our example this
%    macro is |\Lem|). It should be noted that we use an
%    ``|\expandafter| trick'' to expand only certain parts of the
%    replacement text at the time of the definition.
%    \begin{macrocode}
    \def\@tempa{\global\@namedef{#1}}%
    \expandafter \@tempa \expandafter{%
%    \end{macrocode}
%    First, the macro that contains the current definitions of
%    |\@begintheorem| and |\@opargtheorem| should be called
%    up.  The name of this macro---as is already known---has the form
%    |\th@|\meta{theorem\ style}; therefore, it must be called by
%    \begin{macrocode}
      \csname th@\the \theorem@style
%    \end{macrocode}
%    In addition the default theorem font should be changeable, i.e.\
%    we have to insert the contents of |\theorem@bodyfont|. For
%    that reason, we expand even further, beyond |\endcsname|,
%    and thus insert the contents of the \textsf{token} register
%    |\theorem@bodyfont| in the replacement text.
%    \begin{macrocode}
            \expandafter \endcsname \the \theorem@bodyfont
%    \end{macrocode}
%    Now it is time to call the macro |\@thm| which takes over
%    the further processing.  It has two arguments: the current
%    counter name (in our example, \texttt{Lem}), and the text of the
%    label (in our example, \texttt{Lemma}).
%    \begin{macrocode}
     \@thm{#1}{#2}}%
%    \end{macrocode}
%    With this, the `sub-definition' is complete.  The macro
%    |\@endtheorem| ends a theorem environment and is, so far, nothing
%    but an |\endtrivlist|.  (Hence it is defined globally, and not
%    within the theorem styles.\footnote{This has to be changed as
%    soon as theorem styles that change \texttt{\bslash @endtheorem}
%    exist.  In such a case, all existing styles must be changed as
%    well since they will have to reset the macro.}) Therefore, we can
%    set it equivalent to the macro that ends the theorem set (in our
%    example, |\endLem|).  However, if some day theorem styles exist
%    that do change |\@endtheorem|, we would have to use the
%    commented-out line instead.
%    \begin{macrocode}
    \global \expandafter \let \csname end#1\endcsname \@endtheorem
%   \global\@namedef{end#1}{\@endtheorem}%
%    \end{macrocode}
%    With these commands all the required definitions are employed,
%    unless the test |\@ifdefinable| has failed.  Therefore, we
%    end the second argument of this macro and with it the definition
%    of |\@xnthm|.
%    \begin{macrocode}
   }}
%    \end{macrocode}
% \end{macro}
%
% \begin{macro}{\@ynthm}
%    The definition of |\@ynthm| is completely analogous. In
%    this case the new counter that is defined is not reset within
%    another counter; thus the definition of |\the...|
%    is simplified:
%    \begin{macrocode}
\gdef\@ynthm#1#2{\expandafter\@ifdefinable\csname #1\endcsname
   {\@definecounter{#1}%
    \expandafter\xdef\csname the#1\endcsname{\@thmcounter{#1}}%
%    \end{macrocode}
%    The rest of the definition corresponds literally to that of
%    |\@xnthm|:
%    \begin{macrocode}
    \def\@tempa{\global\@namedef{#1}}\expandafter \@tempa
     \expandafter{\csname th@\the \theorem@style \expandafter
     \endcsname \the\theorem@bodyfont \@thm{#1}{#2}}%
    \global \expandafter \let \csname end#1\endcsname \@endtheorem}}
%    \end{macrocode}
% \end{macro}
%
% \begin{macro}{\@othm}
%    The definition of |\@othm| does not contain anything new.
%    \begin{macrocode}
\gdef\@othm#1[#2]#3{%
%    \end{macrocode}
% \changes{v2.2b}{95/10/07}{Build in check for counter defined}
%    We do not define a new counter but instead use one that has
%    already been defined.  Thus the only definition we need is that
%    of this pseudo-counter (i.e.\ |\the|\meta{env.\ name}).
%    First we check if |#2| corresponds to a known counter name.
%    \begin{macrocode}
  \expandafter\ifx\csname c@#2\endcsname\relax
   \@nocounterr{#2}%
  \else
   \expandafter\@ifdefinable\csname #1\endcsname
   {\expandafter \xdef \csname the#1\endcsname
     {\expandafter \noexpand \csname the#2\endcsname}%
%    \end{macrocode}
%    All other parts of the definition can be adopted from
%    |\@xnthm|. We have to remember, though, that in this case
%    the name of the current counter and the theorem label have moved
%    to the second and third arguments.
%    \begin{macrocode}
    \def\@tempa{\global\@namedef{#1}}\expandafter \@tempa
     \expandafter{\csname th@\the \theorem@style \expandafter
     \endcsname \the\theorem@bodyfont \@thm{#2}{#3}}%
    \global \expandafter \let \csname end#1\endcsname \@endtheorem}%
  \fi}
%    \end{macrocode}
% \end{macro}
%
% \subsection{Macros that are employed in a {\envfont theorem}
%    environment}
%
% \begin{macro}{\@thm}
%    The macro |\@thm| has to increase the current counter.  Then,
%    depending on whether the environment has (or does not have) an
%    optional argument, it has to branch into either |\@begintheorem|
%    or |\@opargtheorem|.
%    \begin{macrocode}
\gdef\@thm#1#2{\refstepcounter{#1}%
%    \end{macrocode}
%    Now we start a {\envfont trivlist} environment, and give
%    |\@topsep| and |\@topsepadd| the values of the \textsf{skip}
%    registers |\theorempreskipamount| and |\theorempostskipamount|.
%    The value in |\@topsep| is the vertical space that is inserted by
%    the first (and only) |\item| in our |\trivlist| whilst
%    |\@topsepadd| is inserted by |\@endparenv| at the end of that
%    {\envfont trivlist} environment.  By using these registers, we
%    obtain the desired space around a {\envfont theorem} environment.
%    \begin{macrocode}
   \trivlist
   \@topsep \theorempreskipamount               % used by first \item
   \@topsepadd \theorempostskipamount           % used by \@endparenv
%    \end{macrocode}
%    Now we have to test whether an optional argument has been given.
% \changes{v2.0d}{89/07/18}{Forgotten percent added.}
%    \begin{macrocode}
   \@ifnextchar [%
%    \end{macrocode}
%    If there is an optional argument, we will call |\@ythm|, and
%    move the arguments read back into the input stream.
%    \begin{macrocode}
   {\@ythm{#1}{#2}}%
%    \end{macrocode}
%    If not, we call |\@begintheorem|. Its first argument is the
%    name of the theorem set (hence the second argument of
%    |\@thm|). Its second argument is the macro that produces the
%    current number.
%    \begin{macrocode}
   {\@begintheorem{#2}{\csname the#1\endcsname}\ignorespaces}}
%    \end{macrocode}
% \end{macro}
%
% \begin{macro}{\@xthm}
% \begin{macro}{\@ythm}
%    Both these macros were originally called by |\@thm|.  We do
%    not need |\@xthm| anymore, hence we reset it to
%    |\relax|. The definition of |\@ythm| has not changed at
%    all from its definition in \LaTeX. In order to make the macros
%    easier to understand, we will
%    nevertheless present it (commented out).
%    \begin{macrocode}
\global\let\@xthm\relax
% \def\@ythm#1#2[#3]{\@opargbegintheorem{#2}{\csname
%           the#1\endcsname}{#3}\ignorespaces}
%</package>
%    \end{macrocode}
%    The primitive |\ignorespaces| in |\@ythm| and
%    |\@thm| is needed to remove the spaces between the
%    |\begin{...}| and the actual text.
% \end{macro}
% \end{macro}
%
%
% \subsection{Definition of the theorem substyles}
%
% As already pointed out, the theorem substyles, defined below, are
% only loaded when necessary.  Note that all these substyles, except
% \texttt{plain}, have |\slshape| as the default body font.
%
% \subsubsection{The \texttt{plain} style}
%
% \label{dok:th@plain} As the following macros use |@|, we have
% to locally set the |\catcode| of this symbol to ``letter''.
% This happens within a group, so that we do not have to worry about
% which |\catcode| that symbol had before.
%    \begin{macrocode}
%<*thp>
\begingroup \makeatletter
%    \end{macrocode}
% Since we are now within a group, we must make all definitions
% globally.  First we make sure that \texttt{theorem.sty} is loaded.
% This will allow us to use this file as a document style option
% without having to call \texttt{theorem} itself as an option.  At the
% same time, we assure that at least version 2 is loaded, since
% |\theorem@style| was not defined in earlier versions.
%    \begin{macrocode}
\@ifundefined{theorem@style}{\input{theorem.sty}}{}
\ProvidesFile{thp.sty}
             [\FMithmInfo]
%    \end{macrocode}
%
%
% \begin{macro}{\th@plain}
% \begin{macro}{\@begintheorem}
% \begin{macro}{\@opargbegintheorem}
%    |\theoremstyle{plain}| corresponds to the original
%    definition, except that the distances to the surrounding text are
%    determined by the parameters |\theorempreskipamount| and
%    |\theorempostskipamount|.
%    First we set the default body font.
%    \begin{macrocode}
\gdef\th@plain{\normalfont\itshape
%    \end{macrocode}
%    Then we define |\@begintheorem| and
%    |\@opargbegintheorem|. These two macros define how the
%    header of a theorem is typeset.  |\@opargbegintheorem| will
%    be called if a \textsf{theorem} environment with an optional
%    argument is encountered; otherwise, the header is constructed by
%    calling |\@begintheorem|.  If one of these macros is
%    executed, we are within a \textsf{trivlist} environment started by
%    |\@thm|.  So the theorem header is produced with an
%    |\item| command.
%
%    Instead of specifying the header font directly, all standard
%    theorem styles use the |\theorem@headerfont| macro to allow
%    customization.  The extra space (|\labelsep|) is necessary
%    because of problems in the \textsf{trivlist} environment.
%    \begin{macrocode}
  \def\@begintheorem##1##2{%
        \item[\hskip\labelsep \theorem@headerfont ##1\ ##2]}%
%    \end{macrocode}
%    The definition of |\@opargbegintheorem| is completely
%    analogous. The only difference is the fact that there exists a
%    third argument (which is the optional parameter of the
%    environment and contains additional information about the
%    theorem).  Customarily we enclose it in parentheses.
%    \begin{macrocode}
\def\@opargbegintheorem##1##2##3{%
   \item[\hskip\labelsep \theorem@headerfont ##1\ ##2\ (##3)]}}
%    \end{macrocode}
% \end{macro}
% \end{macro}
% \end{macro}
%
% \noindent
% We conclude with an |\endgroup| to restore the
% |\catcode| of |@|.
%    \begin{macrocode}
\endgroup
%</thp>
%    \end{macrocode}
%
%
% \subsubsection{The \texttt{break} style}
%
% This style option is stored in the file |thb.sty|.
% For the next two lines see the documentation for |\th@plain| on
% page~\pageref{dok:th@plain}.
%    \begin{macrocode}
%<*thb>
\begingroup \makeatletter
\@ifundefined{theorem@style}{\input{theorem.sty}}{}
\ProvidesFile{thb.sty}
             [\FMithmInfo]
%    \end{macrocode}
%
% \begin{macro}{\th@break}
%    |\theoremstyle{break}| produces a line break after the name
%    of the theorem.  The font is |\slshape|.  Hence, we define
%    |\th@break| as follows:
%    \begin{macrocode}
\gdef\th@break{\normalfont\slshape
  \def\@begintheorem##1##2{\item[%
%    \end{macrocode}
%    We run into the following problem: it is not possible to create
%    the header with |\item[|\meta{title}|]| and then start
%    a new line by, for example, |\mbox{}\\|. Such a definition
%    will fail whenever a list environment follows immediately.  With
%    the above construction, the |\mbox{}| causes the switch
%    |@inlabel| (cf.\ definition of |\list| and
%    |\trivlist| in \cite{latex:source}) to be set to \texttt{false}
%    and so the following list will insert additional vertical
%    space (|\topskip|). This is quite annoying. Therefore, we
%    create the line break within the |\item|. In order to ensure
%    that the text will begin at the proper position in the following
%    line, we simply pretend that the label does not take any
%    room.\footnote{This will lead to problems whenever very high
%    symbols occurring in the line tower into the
%    heading. So, something else has to be done here sometime.}
%    \begin{macrocode}
       \rlap{\vbox{\hbox{\hskip \labelsep\theorem@headerfont ##1\ ##2}%
                   \hbox{\strut}}}]}%
%    \end{macrocode}
%    Again, the definition of |\@opargbegintheorem| is
%    completely analogous.
%    \begin{macrocode}
\def\@opargbegintheorem##1##2##3{%
  \item[\rlap{\vbox{\hbox{\hskip \labelsep \theorem@headerfont
                     ##1\ ##2\ (##3)}%
                    \hbox{\strut}}}]}}
\endgroup
%</thb>
%    \end{macrocode}
% \end{macro}
%
%
% \subsubsection{The \texttt{changebreak} style}
%
%    \begin{macrocode}
% This style option is stored in the file |thcb.sty|.
%    \begin{macrocode}
%<*thcb>
\begingroup \makeatletter
\@ifundefined{theorem@style}{\input{theorem.sty}}{}
\ProvidesFile{thcb.sty}
             [\FMithmInfo]
%    \end{macrocode}
%
% \begin{macro}{\th@changebreak}
%    The \texttt{change-break} theorem style is like \texttt{break}
%    but with interchange of theorem name and theorem number.  Thus we
%    define |\th@changebreak| as follows:
%    \begin{macrocode}
\gdef\th@changebreak{\normalfont\slshape
  \def\@begintheorem##1##2{\item
       [\rlap{\vbox{\hbox{\hskip\labelsep \theorem@headerfont ##2\ ##1}%
                    \hbox{\strut}}}]}%
\def\@opargbegintheorem##1##2##3{%
  \item[\rlap{\vbox{\hbox{\hskip\labelsep \theorem@headerfont
                     ##2\ ##1\ (##3)}%
                    \hbox{\strut}}}]}}
\endgroup
%</thcb>
%    \end{macrocode}
% \end{macro}
%
%
% \subsubsection{The \texttt{change} style}
%
% This style option is stored in the file |thc.sty|.
%    \begin{macrocode}
%<*thc>
\begingroup \makeatletter
\@ifundefined{theorem@style}{\input{theorem.sty}}{}
\ProvidesFile{thc.sty}
             [\FMithmInfo]
%    \end{macrocode}
%
% \begin{macro}{\th@change}
%    The \texttt{change} theorem style corresponds to the
%    \texttt{change break} style without a linebreak after the header.
%    To say it in another way, it's the same as the \texttt{plain}
%    style but with number and name interchanged and |\slshape| as the
%    default font.
%    \begin{macrocode}
\gdef\th@change{\normalfont\slshape
  \def\@begintheorem##1##2{\item
       [\hskip\labelsep \theorem@headerfont ##2\ ##1]}%
\def\@opargbegintheorem##1##2##3{%
  \item[\hskip\labelsep \theorem@headerfont ##2\ ##1\ (##3)]}}
\endgroup
%</thc>
%    \end{macrocode}
% \end{macro}
%
%
% \subsubsection{The \texttt{marginbreak} style}
%
% This style option is the one used most often at Mainz.  It is saved
% in the file |thmb.sty|.
%    \begin{macrocode}
%<*thmb>
\begingroup \makeatletter
\@ifundefined{theorem@style}{\input{theorem.sty}}{}
\ProvidesFile{thmb.sty}
             [\FMithmInfo]
%    \end{macrocode}
%
% \begin{macro}{\th@marginbreak}
%    The \texttt{margin break} style is nearly the same as the
%    \texttt{change break} style. The only difference is the placement
%    of the theorem number.  We use |\llap| to place it in the left
%    margin.
%
%    In this style |\labelsep| denotes the separation between the
%    number and the text.
%    \begin{macrocode}
\gdef\th@marginbreak{\normalfont\slshape
  \def\@begintheorem##1##2{\item
       [\rlap{\vbox{\theorem@headerfont
                    \hbox{\llap{##2}\hskip\labelsep ##1}%
                    \hbox{\strut}}}]}%
\def\@opargbegintheorem##1##2##3{%
  \item[\rlap{\vbox{\theorem@headerfont
                    \hbox{\llap{##2}\hskip\labelsep ##1\ (##3)}%
                    \hbox{\strut}}}]}}
\endgroup
%</thmb>
%    \end{macrocode}
% \changes{v2.1c}{90/02/17}{`endgroup outside macrocode env.}
% \end{macro}
%
%
%
% \subsubsection{The \texttt{margin} style}
%
% This style option is stored in the file |thm.sty|.
%    \begin{macrocode}
%<*thm>
\begingroup \makeatletter
\@ifundefined{theorem@style}{\input{theorem.sty}}{}
\ProvidesFile{thm.sty}
             [\FMithmInfo]
%    \end{macrocode}
%
% \begin{macro}{\th@margin}
%    Again this is only a variant of the theorem styles described
%    above without any new ideas.
%    \begin{macrocode}
\gdef\th@margin{\normalfont\slshape
  \def\@begintheorem##1##2{\item
       [\theorem@headerfont \llap{##2}\hskip\labelsep ##1]}%
\def\@opargbegintheorem##1##2##3{%
  \item[\theorem@headerfont \llap{##2}\hskip\labelsep ##1\ (##3)]}}
\endgroup
%</thm>
%    \end{macrocode}
% \end{macro}
%
%
%
%
% \subsection{Final Definitions}
%
% \begin{macro}{\theorempreskipamount}
% \begin{macro}{\theorempostskipamount}
%    The \textsf{skip} parameters that regulate the vertical empty space
%    before and after the {\envfont theorem} environment have
%    to be allocated as well.
% \changes{v2.0f}{89/07/20}{Unnecessary `global' removed.}
%    \begin{macrocode}
%<*package>
\newskip\theorempreskipamount
\newskip\theorempostskipamount
%    \end{macrocode}
%    Since we have used the same values for all theorem sets, we now can
%    assign them.
%    \begin{macrocode}
\global\setlength\theorempreskipamount{12pt plus 5pt minus 3pt}
\global\setlength\theorempostskipamount{8pt plus 3pt minus 1.5pt}
%    \end{macrocode}
% \end{macro}
% \end{macro}
%
%
% \begin{macro}{\@endtheorem}
%    The same holds for the macro |\@endtheorem|, which ends a
%    {\envfont theorem} environment. Since it is the same for all
%    theorem sets, it is removed from the macros |\th@|\meta{style}.
%    It simply ends the {\envfont trivlist} environment, which was
%    begun in |\@thm|.
%    \begin{macrocode}
\global\let\@endtheorem=\endtrivlist
%    \end{macrocode}
% \end{macro}
%
%
%
% \begin{macro}{\@preamblecmds}
%    All macros defined above are to be used only in the preamble.
%    Therefore, we insert them in |\@preamblecmds| which will disable
%    them at begin document. This is done by the internal \LaTeXe{}
%    command |\@onlypreamble|.
% \changes{v2.0g}{89/09/19}{`theoremheaderfont and `theorembodyfont
%    added.}
% \changes{v2.2a}{94/02/03}{Use \LaTeXe{} interface}
%    \begin{macrocode}
\@onlypreamble\@xnthm
\@onlypreamble\@ynthm
\@onlypreamble\@othm
\@onlypreamble\newtheorem
\@onlypreamble\theoremstyle
\@onlypreamble\theorembodyfont
\@onlypreamble\theoremheaderfont
%    \end{macrocode}
% \end{macro}
%
% Finally we declare the \texttt{plain} theorem style to be the default.
% \changes{v2.1b}{90/06/04}{Fixed missing `item by declaring default
%    style.}
%    \begin{macrocode}
\theoremstyle{plain}
%</package>
%    \end{macrocode}
%
% \Finale

\endinput
}{}
\ProvidesFile{thp.sty}
             [\FMithmInfo]
%    \end{macrocode}
%
%
% \begin{macro}{\th@plain}
% \begin{macro}{\@begintheorem}
% \begin{macro}{\@opargbegintheorem}
%    |\theoremstyle{plain}| corresponds to the original
%    definition, except that the distances to the surrounding text are
%    determined by the parameters |\theorempreskipamount| and
%    |\theorempostskipamount|.
%    First we set the default body font.
%    \begin{macrocode}
\gdef\th@plain{\normalfont\itshape
%    \end{macrocode}
%    Then we define |\@begintheorem| and
%    |\@opargbegintheorem|. These two macros define how the
%    header of a theorem is typeset.  |\@opargbegintheorem| will
%    be called if a \textsf{theorem} environment with an optional
%    argument is encountered; otherwise, the header is constructed by
%    calling |\@begintheorem|.  If one of these macros is
%    executed, we are within a \textsf{trivlist} environment started by
%    |\@thm|.  So the theorem header is produced with an
%    |\item| command.
%
%    Instead of specifying the header font directly, all standard
%    theorem styles use the |\theorem@headerfont| macro to allow
%    customization.  The extra space (|\labelsep|) is necessary
%    because of problems in the \textsf{trivlist} environment.
%    \begin{macrocode}
  \def\@begintheorem##1##2{%
        \item[\hskip\labelsep \theorem@headerfont ##1\ ##2]}%
%    \end{macrocode}
%    The definition of |\@opargbegintheorem| is completely
%    analogous. The only difference is the fact that there exists a
%    third argument (which is the optional parameter of the
%    environment and contains additional information about the
%    theorem).  Customarily we enclose it in parentheses.
%    \begin{macrocode}
\def\@opargbegintheorem##1##2##3{%
   \item[\hskip\labelsep \theorem@headerfont ##1\ ##2\ (##3)]}}
%    \end{macrocode}
% \end{macro}
% \end{macro}
% \end{macro}
%
% \noindent
% We conclude with an |\endgroup| to restore the
% |\catcode| of |@|.
%    \begin{macrocode}
\endgroup
%</thp>
%    \end{macrocode}
%
%
% \subsubsection{The \texttt{break} style}
%
% This style option is stored in the file |thb.sty|.
% For the next two lines see the documentation for |\th@plain| on
% page~\pageref{dok:th@plain}.
%    \begin{macrocode}
%<*thb>
\begingroup \makeatletter
\@ifundefined{theorem@style}{% \iffalse meta-comment
%
% Copyright (C) 1993-2024
%
% The LaTeX Project and any individual authors listed elsewhere
% in this file.
%
% This file is part of the Standard LaTeX `Tools Bundle'.
% -------------------------------------------------------
%
% It may be distributed and/or modified under the
% conditions of the LaTeX Project Public License, either version 1.3c
% of this license or (at your option) any later version.
% The latest version of this license is in
%    https://www.latex-project.org/lppl.txt
% and version 1.3c or later is part of all distributions of LaTeX
% version 2005/12/01 or later.
%
% The list of all files belonging to the LaTeX `Tools Bundle' is
% given in the file `manifest.txt'.
%
% \fi
%
%%
%
%
\def\FMithmInfo{2023/07/05 v2.2c Theorem extension package (FMi)}
%
% \ProvidesFile{theorem.dtx}[\FMithmInfo]
%\iffalse   % this is a METACOMMENT !
%%
%% Package `theorem' to use with LaTeX2e
%% Copyright (C) 1989-1995 Frank Mittelbach, all rights reserved.
%<+package>\@ifundefined{theorem@style}{}{\endinput}
%<+package>\NeedsTeXFormat{LaTeX2e}[1995/06/01]
%<+package>\ProvidesPackage{theorem}
%<+package>         [\FMithmInfo]
%
% \fi
%
% \changes{v1.0}{87/05/07}{First draft.}
% \changes{v1.0a}{87/05/07}{`thistheoremstyle' added.}
% \changes{v1.1a}{87/07/29}{New concept: redefine the internal
%                          \protect\LaTeX{} commands `@xnthm' etc.}
% \changes{v1.1b}{87/07/31}{`theoremfont' used to specify font.}
% \changes{v1.2a}{88/01/08}{`@endtheorem' now globally defined.}
% \changes{v1.2b}{88/01/11}{`newtheorem' only allowed in the preamble.}
% \changes{v1.3a}{88/02/14}{`theoremfont' now a toks register.}
%
% \changes{v2.0a}{88/11/16}{New implementation and new documentation.}
% \changes{v2.0a}{88/11/16}{No layout changes but internal modifications
%                           old substyles won't work.}
% \changes{v2.0b}{89/01/30}{`global' used for all definitions.}
% \changes{v2.0c}{89/06/15}{Documentation translated to English.
%                            Thanks to LLL and Nina}
% \changes{v2.1a}{90/02/15}{Support for new font selection.}
% \changes{v2.2a}{94/02/03}{Cleanup for LaTeX2e}
%
% \DoNotIndex{\@endparenv,\@ifdefinable,\@ifnextchar,\@ifundefined}
% \DoNotIndex{\@namedef,\@nameuse,\@spaces,\@tempa,\@warning}
% \DoNotIndex{\ }
% \DoNotIndex{\begingroup,\bfseries}
% \DoNotIndex{\csname}
% \DoNotIndex{\def,\do,\docdate}
% \DoNotIndex{\endcsname,\endgroup,\endinput,\endtrivlist,\expandafter}
% \DoNotIndex{\filedate,\fileversion}
% \DoNotIndex{\gdef,\global}
% \DoNotIndex{\hbox,\hskip}
% \DoNotIndex{\ignorespaces,\input,\itshape,\item}
% \DoNotIndex{\labelsep,\let,\llap}
% \DoNotIndex{\makeatletter}
% \DoNotIndex{\noexpand}
% \DoNotIndex{\refstepcounter,\relax,\rlap}
% \DoNotIndex{\setlength,\slshape,\space,\string,\strut}
% \DoNotIndex{\the,\trivlist,\typeout}
% \DoNotIndex{\vbox}
% \DoNotIndex{\xdef}
%
% \def\deflabel#1{\ttfamily #1\hfill}
% \def\deflist#1{\begin{list}{}{\settowidth\labelwidth{\ttfamily #1}%
%                               \setlength\leftmargin\labelwidth
%                               \addtolength\leftmargin\labelsep
%                               \let\makelabel\deflabel}}
% \def\enddeflist{\end{list}}
% \def\envfont{\normalfont\ttfamily}
%
% \MakeShortVerb{\|}
%
% \newcommand{\TUB}{{\sl TUGboat\/}}
%
% \GetFileInfo{theorem.dtx}
% \title{An Extension of the
%        \LaTeX{} theorem environment\thanks{This file
%        has version number \fileversion, last
%        revised \filedate.}}
% \author{Frank Mittelbach}
%
% \MaintainedByLaTeXTeam{tools}
% \maketitle
%
% \begin{abstract}
%    The macros described in this paper yield an extension of the
%    \LaTeX{} theorem mechanism. It is designed to satisfy the
%    different requirements of various journals. Thus, the
%    layout of the ``theorems'' can be manipulated by determining a
%    ``style''. This article describes not only the use, but
%    also the definition, of the necessary macros.
% \end{abstract}
%
% \section*{Preface to version 2.2}
%
% For \LaTeXe{} this package did not need any fundamental changes. I
% only modified the messages generated so that theorem layout styles
% will show up with the |\listfiles| command and cleaned the section
% on the New Font Selection Scheme since this is now included in
% \LaTeX.
%
% \section*{Preface to version 2.1}
%
% This version is identical to 2.0g described in \TUB\ 10\#3 except
% for some internal defaults which are now set depending on the used
% font selection scheme.
%
% This was done to avoid unpleasant surprises if the new font selection
% scheme is in force. For further details see
% section~\ref{sec:newfonts} and \cite{bk:GMS94}.
%
% \section{Introduction}
%
% For our purposes here, ``theorems'' are labelled enunciations,
% often set off from the main text by extra space and a font change.
% Theorems, corollaries, conjectures, definitions, and remarks
% are all instances of ``theorems''.  The ``header'' of these structures
% is composed of a label (such as \textsc{Theorem} or \textsc{Remark})
% and a number which serializes an item in the sequence of items
% with the same label.
%
% Shortly after the introduction of \LaTeX{} at the Fachbereich
% Mathematik in Mainz, the desire to manipulate the layout of
% ``theorems'' arose.  In Mainz, the following two conventions came
% into general use:
% \begin{enumerate}
%   \item The number of the theorem is shown in the margin.
%   \item There is a line break at
%         the end of the theorem header.
% \end{enumerate}
%
% Additionally, some journals require different formats which depend on
% the ``sort of theorem'': e.g.\ often remarks and definitions are set
% in |\upshape|, while |\itshape| is employed for main theorems.
%
% Confronted with these requirements, a theorem environment was
% developed in Mainz which allows separate determination of the
% layout of the ``theorems sets'', comparable to |\pagestyle|.
%
%
% \section{The user interface}
% \subsection{Defining new theorem sets}
%
% \DescribeMacro\newtheorem
% As in the original \LaTeX{} version, the command |\newtheorem|
% defines a new ``theorem set'' or ``theorem-like structure''.
% Two required arguments name the new environment and give the text
% to be typeset with each instance of the new ``set'', while
% an optional argument determines how
% the ``set'' is enumerated:
% \begin{description}
%    \item[\ttfamily \bslash newtheorem\{foo\}\{bar\}]
%       The theorem set {\envfont foo} (whose name is \texttt{bar})
%       uses its own counter.
%    \item[\ttfamily \bslash newtheorem\{foo2\}{[foo]}\{bar2\}]
%       The theorem set {\envfont foo2} (printed name \texttt{bar2})
%       uses the same counter as the theorem set \texttt{foo}.
%    \item[\ttfamily \bslash newtheorem\{foo3\}\{bar3\}{[section]}]
%       The theorem set {\envfont foo3} (printed name \texttt{bar3}) is
%       enumerated within the counter \texttt{section}, i.e.\ with every
%       new |\section| the enumeration begins again with 1, and
%       the enumeration is composed from the section-number and the
%       theorem counter itself.
% \end{description}
%
% \DescribeMacro\theoremstyle
% Additionally, the command |\theoremstyle| can define the layout
% of various, or all, theorem sets.  It should be noted that any
% theorem set defined by |\newtheorem| is typeset in the
% |\theoremstyle| that is current at the time of the definition.
% Thus, the following
% \begin{verbatim}
%    \theoremstyle{break}        \newtheorem{Cor}{Corollary}
%    \theoremstyle{plain}        \newtheorem{Exa}{Example}[section]
%\end{verbatim}                             ^^A No blank at line start
% leads to the result that the set {\envfont Cor} is formatted in the
% style \texttt{break}, while the set {\envfont Exa} and all the
% following ones are formatted in the style \texttt{plain}, unless
% another |\theoremstyle| follows.  Since the definitions installed by
% |\newtheorem| are global, one also can limit |\theoremstyle| locally
% by grouping braces.
%
% \DescribeMacro\theorembodyfont
% The choice of the font for the theorem body is completely
% independent of the chosen |\theoremstyle|; this has proven to
% be very advantageous.  For example,
% \begin{verbatim}
%   {\theorembodyfont{\upshape}          \newtheorem{Rem}{Remark}}
%\end{verbatim}
% defines a theorem set {\envfont Rem}, which will be set in
% |\upshape| in the current layout (which in our example is
% \texttt{plain}).  As with |\theoremstyle|, the |\theorembodyfont|
% chosen is that current at the time of |\newtheorem|.  If
% |\theorembodyfont| is not specified or one defines
% |\theorembodyfont{}|, then the font used will be that defined by the
% |\theoremstyle|.
%
% \DescribeMacro\theoremheaderfont It is also possible to customize
% the font used for the theorem headers.  This is, however, a global
% declaration, and therefore there should be at most one
% |\theoremheaderfont| declaration in the preamble.\footnote{If it is
% actually necessary to have different header fonts, one has to define
% new theorem styles (substituting the desired font) or specify the
% information directly in the \texttt{\bslash newtheorem} declaration
% (the unclean variant).}
%
% \DescribeMacro\theorempreskipamount
% \DescribeMacro\theorempostskipamount Two additional parameters
% affect the vertical space around the theorem environments:
% |\theorempreskipamount| and |\theorempostskipamount| define,
% respectively, the spacing before and after such an environment.
% These parameters apply for all theorem sets and can be manipulated
% with the ordinary length macros.  They are rubber lengths
% (`\textsf{skips}'), and therefore can contain \texttt{plus} and
% \texttt{minus} parts.
%
% Since the definition of theorem sets should---most sensibly---be
% placed in the preamble, we only allow installation there.
% It is therefore possible to release the memory used here after
% |\begin{document}|, in order to make room for other
% applications.
%
%
% \subsection{Existing theorem styles}
%
% The following theorem styles exist to date:
% \begin{deflist}{marginbreak}
%    \item[plain]
%       This theorem style emulates the original \LaTeX{} definition,
%       except that additionally the parameters
%       |\theorem...skipamount| are used.
%    \item[break]
%       In this style, the theorem header is followed by a line
%       break.
%    \item[marginbreak]
%       The theorem number is set in the margin, and there is a line
%       break as in \texttt{break}.
%    \item[changebreak]
%       Like \texttt{break}, but with header number and text
%       interchanged.
%    \item[change]
%       Header number and text are interchanged, without a line break.
%    \item[margin]
%       The number is set in the left margin, without a line break.
% \end{deflist}
% All styles (except \texttt{plain}) select |\slshape| as the default
% |\theorembodyfont|.
%
%
% \subsection{Examples}
%
% Given the above theorem sets {\envfont Cor}, {\envfont Exa}
% and {\envfont Rem},
% suppose that the preamble also contains the declarations:
% \begin{verbatim}
%    \theoremstyle{marginbreak}   \newtheorem{Lem}[Cor]{Lemma}
%    \theoremstyle{change}
%    \theorembodyfont{\itshape}        \newtheorem{Def}[Cor]{Definition}
%
%    \theoremheaderfont{\scshape}
%\end{verbatim}
% Then the following are some typical examples of the typeset output
% resulting from their use.
%
% \begin{Cor}
%    This is a sentence typeset in the theorem environment {\envfont
%    Cor}.
% \end{Cor}
%
% \begin{Exa}
%    This is a sentence typeset in the theorem environment {\envfont
%    Exa}.
% \end{Exa}
%
% \begin{Rem}
%    This is a sentence typeset in the theorem environment {\envfont
%    Rem}.
% \end{Rem}
%
% \begin{Lem}[Ben User]
%    This is a sentence typeset in the theorem environment {\envfont
%    Lem}.
% \end{Lem}
%
% \begin{Def}[Very impressive Definition]
%    This is a sentence typeset in the theorem environment {\envfont
%    Def}.
% \end{Def}
%
% The last two examples show the effect of the optional argument to
% a theorem environment (it is the text typeset in parentheses).
%
%
% \section{Special Considerations}
% \label{sec:newfonts}
%
% Theoremheader and body are implemented as a unit. This means that
% the |\theoremheaderfont|  will inherit characteristics of the
% |\theorembodyfont| in \LaTeXe. Thus, if for example
% |\theorembodyfont| is |\itshape| and |\theoremheaderfont|
% is |\bfseries| the font selected for the header will have the
% characteristics `bold extended italic'. If this is not desired one
% should set the |\theoremheaderfont| to something like
% \begin{verbatim}
%      \theoremheaderfont{\normalfont\bfseries}
%\end{verbatim}
% i.e.\ supplying all necessary font information explicitly.
%
%
%
% \section{Acknowledgements}
%
% The publication of this set of macros was only possible with the
% help of Christina Busse (translating the manuscript into English),
% Joachim Pense (playing the r\^ole of typist), Chris Rowley
% (looking everything over) and many others providing useful
% suggestions.
%
% \MaybeStop{%
%    \begin{thebibliography}{9}
%       \bibitem{bk:GMS94} \textsc{M.~Goossens}, \textsc{F.~Mittelbach}
%         and \textsc{A.~Samarin}.
%         \newblock The \LaTeX{} Companion.
%         \newblock
%         Addison-Wesley, Reading, Massachusetts, 1994.
%       \bibitem{latex:source} \textsc{Lamport, Leslie.} \newblock
%                               \texttt{latex.tex}, version 2.09, date
%                               Feb. 1990.
%    \end{thebibliography}
%    \PrintIndex
%    \PrintChanges}
%
%
% \section{The documentation driver file}
%
% The next bit of code contains the documentation driver file for
% \TeX{}, i.e., the file that will produce the documentation you are
% currently reading. It will be extracted from this file by the
% \texttt{docstrip} program. Since it is the first code in the file
% one can alternatively process this file directly with \LaTeXe{} to
% obtain the documentation.
%
%    \begin{macrocode}
%<*driver>
\documentclass{ltxdoc}

\usepackage{theorem}

%<+driver>% The next few lines define theorem sets which are used
%<+driver>% in the example section of the documentation.

\theoremstyle{break}       \newtheorem{Cor}{Corollary}
\theoremstyle{plain}       \newtheorem{Exa}{Example}[section]
{\theorembodyfont{\upshape}\newtheorem{Rem}{Remark}}
\theoremstyle{marginbreak} \newtheorem{Lem}[Cor]{Lemma}
\theoremstyle{change}
\theorembodyfont{\itshape} \newtheorem{Def}[Cor]{Definition}

\theoremheaderfont{\scshape}

\RecordChanges

\begin{document}
   \DocInput{theorem.dtx}
\end{document}
%</driver>
%    \end{macrocode}
%
% \section{Definition of the Macros}
%
% If the file has been loaded before, we abort immediately. If not the
% package announces itself (this is actually done at the very top of
% the file---the way it is done isn't good style so don't copy it).
% \changes{v2.0e}{89/07/19}{Spaces removed from `typeout'.}
% \changes{v2.2b}{95/11/19}{Announce moved to top of file}
%    \begin{macrocode}
%<*package>
%\@ifundefined{theorem@style}{}{\endinput}
%\def\FMithmInfo{1995/11/19 v2.2b Theorem extension package (FMi)}
%\ProvidesPackage{theorem}[\FMithmInfo]
%    \end{macrocode}
%
%
% \subsection{Definition of theorem styles and fonts}
%
% All the definitions in this file are done globally to allow
% inputting this file inside a group.
%
% \begin{macro}{\theoremstyle}
%    Before a theorem style can be installed, the chosen style must be
%    known. For that reason, we must test to see that
%    |\th@|\meta{style} is known or, more precisely, that it is
%    different from |\relax|. If the style is not known then
%    |\th@plain| is used.
%    \begin{macrocode}
\gdef\theoremstyle#1{%
   \@ifundefined{th@#1}{\@warning
          {Unknown theoremstyle `#1'. Using `plain'}%
          \theorem@style{plain}}%
%    \end{macrocode}
%    We save the theorem style to be used in the token register
%    |\theorem@style|.
%    \begin{macrocode}
      {\theorem@style{#1}}%
%    \end{macrocode}
%    Now we ``evaluate'' the theorem style: this means, we call the
%    macro |\th@|\meta{style} which will activate the relevant
%    definitions which are contained in a separate file.  This is done
%    in a group to suppress changes to the current font.  This could
%    otherwise pose problems together with the new font selection
%    scheme\footnote{When I printed the original article using the new
%    font selection scheme I ended with a document with slanted
%    typefaces (text headings and all) simply because one of the
%    theorem styles used \texttt{\bslash sl} at toplevel.} if the
%    |\th@|\meta{style} is evaluated a second time.
% \changes{v2.1a}{90/02/15}{Group added around `th@..}
% \changes{v2.1a}{90/02/15}{Reseting of `..@begintheorem removed.}
%    \begin{macrocode}
      \begingroup
        \csname th@\the\theorem@style \endcsname
      \endgroup}
%    \end{macrocode}
% \end{macro}
%
%
% \begin{macro}{\@begintheorem}
% \changes{v2.1a}{90/02/15}{Relaxed at toplevel.}
% \begin{macro}{\@opargbegintheorem}
% \changes{v2.1a}{90/02/15}{Relaxed at toplevel.}
%    We reset |\@begintheorem| and |\@opargbegintheorem| to
%    |\relax| since these commands are no longer necessary at
%    toplevel. This will save a few tokens.
%    \begin{macrocode}
\global\let\@begintheorem\relax
\global\let\@opargbegintheorem\relax
%    \end{macrocode}
% \end{macro}
% \end{macro}
%
% \begin{macro}{\theorem@style}
%    Obviously the \textsf{token} register used above has to be
%    allocated. To assure the utmost compatibility with the
%    original \LaTeX{} definition, we set the default theorem style to
%    \texttt{plain}, which implements the usual \LaTeX{} convention.
% \changes{v2.0f}{89/07/20}{Unnecessary `global' removed.}
%    \begin{macrocode}
\newtoks\theorem@style
\global\theorem@style{plain}
%    \end{macrocode}
% \end{macro}
%
% \begin{macro}{\theorembodyfont}
% \begin{macro}{\theorem@bodyfont}
%    If the |\theorembodyfont| is set by the user then it should not
%    interact with the default font set in the theorem style. When the
%    new font selection is in force this may happen if, for example,
%    the default is |\itshape| and the new |\theorembodyfont| is
%    |\sffamily|. So we add a |\reset@font| command in front of the
%    user definition.
% \changes{v2.0f}{89/07/20}{Unnecessary `global' removed.}
% \changes{v2.0g}{89/09/19}{`theoremfont to `theorembodyfont.}
% \changes{v2.1c}{92/06/22}{`reset@font added and logic changed.}
% \changes{v2.1d}{92/10/28}{Reset to empty if arg is empty}
%    \begin{macrocode}
\gdef\theorembodyfont#1{%
%    \end{macrocode}
%    We check if the argument supplied is empty and if so put nothing
%    into the |\theorem@bodyfont| token register to allow for
%    |\theorembodyfont{}| as a mean of using the default of the
%    current |\theoremstyle|.
%    \begin{macrocode}
   \def\@tempa{#1}%
   \ifx\@tempa\@empty
    \theorem@bodyfont{}%
   \else
    \theorem@bodyfont{\reset@font#1}%
   \fi
   }
\newtoks\theorem@bodyfont
\global\theorem@bodyfont{}
%    \end{macrocode}
% \end{macro}
% \end{macro}
%
%
% \begin{macro}{\theoremheaderfont}
% \changes{v2.0g}{89/09/19}{Macro added.}
%    The font for the theorem headers is handled differently because
%    this definition applies to all theorem styles.
%    \begin{macrocode}
\gdef\theoremheaderfont#1{\gdef\theorem@headerfont{#1}%
%    \end{macrocode}
%    After using the macro once it is redefined to produce an error
%    message.
%    \begin{macrocode}
       \gdef\theoremheaderfont##1{%
        \typeout{\string\theoremheaderfont\space should be used
                 only once.}}}
%    \end{macrocode}
% \end{macro}
%
% \begin{macro}{\theorem@headerfont}
% \changes{v2.0g}{89/09/19}{Macro added.}
%    To set the |\theorem@headerfont| default we first test if
%    the new fontselection scheme is in force.
% \changes{v2.1a}{90/02/15}{Support for new font selection.}
%    \begin{macrocode}
\ifx\upshape\undefined
%    \end{macrocode}
%    If not we define it to expand into |\bfseries|. We don't use
%    |\let| just in case a following style option redefines this
%    macro.
%    \begin{macrocode}
\gdef\theorem@headerfont{\bfseries}
%    \end{macrocode}
%    Otherwise we reset the current shape before calling |\bfseries|.
%    \begin{macrocode}
\else \gdef\theorem@headerfont{\normalfont\bfseries}\fi
%    \end{macrocode}
% \end{macro}
%
% \begin{macro}{\th@plain}
% \begin{macro}{\th@break}
% \begin{macro}{\th@marginbreak}
% \begin{macro}{\th@changebreak}
% \begin{macro}{\th@change}
% \begin{macro}{\th@margin}
%    The different styles are defined in macros such as
%    |\th@plain|.  Since memory space is precious in
%    ``non-Big-versions'', we have to avoid offering too many unused
%    definitions.  Therefore we define these styles in separate files
%    that can be loaded on demand.  Thus the commands themselves only
%    load these files. We use |\@input@|, a \LaTeXe{} internal command
%    that ensures that the file will be listed with |\listfiles|.
% \changes{v2.2a}{94/02/02}{Use `@input@ to load theorem layout files}
%    \begin{macrocode}
\gdef\th@plain{\@input@{thp.sty}}
\gdef\th@break{\@input@{thb.sty}}
\gdef\th@marginbreak{\@input@{thmb.sty}}
\gdef\th@changebreak{\@input@{thcb.sty}}
\gdef\th@change{\@input@{thc.sty}}
\gdef\th@margin{\@input@{thm.sty}}
%    \end{macrocode}
%    This list will be expanded when new styles become
%    available.  For testing,  just append new theorem substyles
%    as document options.
% \end{macro}
% \end{macro}
% \end{macro}
% \end{macro}
% \end{macro}
% \end{macro}
%
%
%
% \subsection{Definition of a new theorem set}
%
% As already pointed out, a new theorem environment can be defined in
% three different ways:
% \begin{verbatim}
%    \newtheorem{Lem}{Lemma}
%    \newtheorem{Lem}{Lemma}[section]
%    \newtheorem{Lem}[Theorem]{Lemma}
%\end{verbatim}
% The function of the macro |\newtheorem| is to recognize these cases
% and then to branch into one of the three macros |\@ynthm|, |\@xnthm|
% or |\@othm|.  This mechanism is adopted unchanged from
% \cite{latex:source}; the essential point here is that, for example,
% in the second case, the arguments \texttt{Lem}, \texttt{Lemma} and
% \texttt{section} are passed over to the macro |\@xnthm|.
%
% We inspect this case first because the others present fewer problems,
% and thus are easily derived from this one.
%
% \begin{macro}{\@xnthm}
%    For our example arguments, the macro |\@xnthm| must fulfill
%    the following:
%    \begin{itemize}
%       \item
%          Define a new \LaTeX-counter `Lem'
%       \item
%          reset this counter within a |\section|
%       \item
%          define the macro |\theLem|
%       \item
%          define the environment macros |\Lem| and
%          |\endLem| using the current
%          |\theoremstyle| and |\theorem@bodyfont|.
%    \end{itemize}
%    Obviously, all this should happen only if the first argument
%    of |\@xnthm| (i.e.\ \texttt{Lem} in our example) is chosen so
%    as not to conflict with any previously defined commands or
%    environments.  This test is performed by the \LaTeX{} macro
%    |\@ifdefinable|.
%    \begin{macrocode}
\gdef\@xnthm#1#2[#3]{\expandafter\@ifdefinable\csname #1\endcsname
%    \end{macrocode}
%    Therefore, the first argument of |\@ifdefinable| is the
%    expansion (in the example, |\Lem|) of
%    |\csname||#1\endcsname|.  The second argument is
%    executed only if the test has been completed successfully.
%    \begin{macrocode}
   {%
%    \end{macrocode}
%    Now we define the new counter. The names of the \LaTeX{}
%    macros employed should speak for themselves:
% \changes{v2.2b}{1995/10/07}{Use \cs{@newctr}}
% \changes{v2.2c}{1995/11/23}{Use \cs{@newctr} properly.}
%    \begin{macrocode}
    \@definecounter{#1}\@newctr{#1}[#3]%
%    \end{macrocode}
%    Using |\@newctr| will give a proper error message if the counter
%    in |#3| is not defined.
%    In defining `|\theLem|' we must generate the desired macro
%    name by use of |\expandafter| and |\csname|.
%    \begin{macrocode}
    \expandafter\xdef\csname the#1\endcsname
%    \end{macrocode}
%    An |\xdef| is used in order to make the definition global,
%    and to ensure that it contains the replacement texts of
%    |\@thmcountersep| and |\@thmcounter|.\footnote{These
%    two macros can be defined by the document style.  Their default
%    values produce a `\texttt{.}' as separation and an arabic
%    representation of the number.} However, not everything should be
%    expanded. For example, it saves space to use |\thesection|
%    instead of its---at times---lengthy expansion.
%    \begin{macrocode}
      {\expandafter \noexpand \csname the#3\endcsname
       \@thmcountersep \@thmcounter{#1}}%
%    \end{macrocode}
%    Thus with the defaults of \LaTeX{}, |\theLem| would be
%    replaced by the command sequence |\thesection.\arabic{Lem}|.
%
%    We will now look at the definition of the macro which is executed
%    at the beginning of the actual environment (in our example this
%    macro is |\Lem|). It should be noted that we use an
%    ``|\expandafter| trick'' to expand only certain parts of the
%    replacement text at the time of the definition.
%    \begin{macrocode}
    \def\@tempa{\global\@namedef{#1}}%
    \expandafter \@tempa \expandafter{%
%    \end{macrocode}
%    First, the macro that contains the current definitions of
%    |\@begintheorem| and |\@opargtheorem| should be called
%    up.  The name of this macro---as is already known---has the form
%    |\th@|\meta{theorem\ style}; therefore, it must be called by
%    \begin{macrocode}
      \csname th@\the \theorem@style
%    \end{macrocode}
%    In addition the default theorem font should be changeable, i.e.\
%    we have to insert the contents of |\theorem@bodyfont|. For
%    that reason, we expand even further, beyond |\endcsname|,
%    and thus insert the contents of the \textsf{token} register
%    |\theorem@bodyfont| in the replacement text.
%    \begin{macrocode}
            \expandafter \endcsname \the \theorem@bodyfont
%    \end{macrocode}
%    Now it is time to call the macro |\@thm| which takes over
%    the further processing.  It has two arguments: the current
%    counter name (in our example, \texttt{Lem}), and the text of the
%    label (in our example, \texttt{Lemma}).
%    \begin{macrocode}
     \@thm{#1}{#2}}%
%    \end{macrocode}
%    With this, the `sub-definition' is complete.  The macro
%    |\@endtheorem| ends a theorem environment and is, so far, nothing
%    but an |\endtrivlist|.  (Hence it is defined globally, and not
%    within the theorem styles.\footnote{This has to be changed as
%    soon as theorem styles that change \texttt{\bslash @endtheorem}
%    exist.  In such a case, all existing styles must be changed as
%    well since they will have to reset the macro.}) Therefore, we can
%    set it equivalent to the macro that ends the theorem set (in our
%    example, |\endLem|).  However, if some day theorem styles exist
%    that do change |\@endtheorem|, we would have to use the
%    commented-out line instead.
%    \begin{macrocode}
    \global \expandafter \let \csname end#1\endcsname \@endtheorem
%   \global\@namedef{end#1}{\@endtheorem}%
%    \end{macrocode}
%    With these commands all the required definitions are employed,
%    unless the test |\@ifdefinable| has failed.  Therefore, we
%    end the second argument of this macro and with it the definition
%    of |\@xnthm|.
%    \begin{macrocode}
   }}
%    \end{macrocode}
% \end{macro}
%
% \begin{macro}{\@ynthm}
%    The definition of |\@ynthm| is completely analogous. In
%    this case the new counter that is defined is not reset within
%    another counter; thus the definition of |\the...|
%    is simplified:
%    \begin{macrocode}
\gdef\@ynthm#1#2{\expandafter\@ifdefinable\csname #1\endcsname
   {\@definecounter{#1}%
    \expandafter\xdef\csname the#1\endcsname{\@thmcounter{#1}}%
%    \end{macrocode}
%    The rest of the definition corresponds literally to that of
%    |\@xnthm|:
%    \begin{macrocode}
    \def\@tempa{\global\@namedef{#1}}\expandafter \@tempa
     \expandafter{\csname th@\the \theorem@style \expandafter
     \endcsname \the\theorem@bodyfont \@thm{#1}{#2}}%
    \global \expandafter \let \csname end#1\endcsname \@endtheorem}}
%    \end{macrocode}
% \end{macro}
%
% \begin{macro}{\@othm}
%    The definition of |\@othm| does not contain anything new.
%    \begin{macrocode}
\gdef\@othm#1[#2]#3{%
%    \end{macrocode}
% \changes{v2.2b}{95/10/07}{Build in check for counter defined}
%    We do not define a new counter but instead use one that has
%    already been defined.  Thus the only definition we need is that
%    of this pseudo-counter (i.e.\ |\the|\meta{env.\ name}).
%    First we check if |#2| corresponds to a known counter name.
%    \begin{macrocode}
  \expandafter\ifx\csname c@#2\endcsname\relax
   \@nocounterr{#2}%
  \else
   \expandafter\@ifdefinable\csname #1\endcsname
   {\expandafter \xdef \csname the#1\endcsname
     {\expandafter \noexpand \csname the#2\endcsname}%
%    \end{macrocode}
%    All other parts of the definition can be adopted from
%    |\@xnthm|. We have to remember, though, that in this case
%    the name of the current counter and the theorem label have moved
%    to the second and third arguments.
%    \begin{macrocode}
    \def\@tempa{\global\@namedef{#1}}\expandafter \@tempa
     \expandafter{\csname th@\the \theorem@style \expandafter
     \endcsname \the\theorem@bodyfont \@thm{#2}{#3}}%
    \global \expandafter \let \csname end#1\endcsname \@endtheorem}%
  \fi}
%    \end{macrocode}
% \end{macro}
%
% \subsection{Macros that are employed in a {\envfont theorem}
%    environment}
%
% \begin{macro}{\@thm}
%    The macro |\@thm| has to increase the current counter.  Then,
%    depending on whether the environment has (or does not have) an
%    optional argument, it has to branch into either |\@begintheorem|
%    or |\@opargtheorem|.
%    \begin{macrocode}
\gdef\@thm#1#2{\refstepcounter{#1}%
%    \end{macrocode}
%    Now we start a {\envfont trivlist} environment, and give
%    |\@topsep| and |\@topsepadd| the values of the \textsf{skip}
%    registers |\theorempreskipamount| and |\theorempostskipamount|.
%    The value in |\@topsep| is the vertical space that is inserted by
%    the first (and only) |\item| in our |\trivlist| whilst
%    |\@topsepadd| is inserted by |\@endparenv| at the end of that
%    {\envfont trivlist} environment.  By using these registers, we
%    obtain the desired space around a {\envfont theorem} environment.
%    \begin{macrocode}
   \trivlist
   \@topsep \theorempreskipamount               % used by first \item
   \@topsepadd \theorempostskipamount           % used by \@endparenv
%    \end{macrocode}
%    Now we have to test whether an optional argument has been given.
% \changes{v2.0d}{89/07/18}{Forgotten percent added.}
%    \begin{macrocode}
   \@ifnextchar [%
%    \end{macrocode}
%    If there is an optional argument, we will call |\@ythm|, and
%    move the arguments read back into the input stream.
%    \begin{macrocode}
   {\@ythm{#1}{#2}}%
%    \end{macrocode}
%    If not, we call |\@begintheorem|. Its first argument is the
%    name of the theorem set (hence the second argument of
%    |\@thm|). Its second argument is the macro that produces the
%    current number.
%    \begin{macrocode}
   {\@begintheorem{#2}{\csname the#1\endcsname}\ignorespaces}}
%    \end{macrocode}
% \end{macro}
%
% \begin{macro}{\@xthm}
% \begin{macro}{\@ythm}
%    Both these macros were originally called by |\@thm|.  We do
%    not need |\@xthm| anymore, hence we reset it to
%    |\relax|. The definition of |\@ythm| has not changed at
%    all from its definition in \LaTeX. In order to make the macros
%    easier to understand, we will
%    nevertheless present it (commented out).
%    \begin{macrocode}
\global\let\@xthm\relax
% \def\@ythm#1#2[#3]{\@opargbegintheorem{#2}{\csname
%           the#1\endcsname}{#3}\ignorespaces}
%</package>
%    \end{macrocode}
%    The primitive |\ignorespaces| in |\@ythm| and
%    |\@thm| is needed to remove the spaces between the
%    |\begin{...}| and the actual text.
% \end{macro}
% \end{macro}
%
%
% \subsection{Definition of the theorem substyles}
%
% As already pointed out, the theorem substyles, defined below, are
% only loaded when necessary.  Note that all these substyles, except
% \texttt{plain}, have |\slshape| as the default body font.
%
% \subsubsection{The \texttt{plain} style}
%
% \label{dok:th@plain} As the following macros use |@|, we have
% to locally set the |\catcode| of this symbol to ``letter''.
% This happens within a group, so that we do not have to worry about
% which |\catcode| that symbol had before.
%    \begin{macrocode}
%<*thp>
\begingroup \makeatletter
%    \end{macrocode}
% Since we are now within a group, we must make all definitions
% globally.  First we make sure that \texttt{theorem.sty} is loaded.
% This will allow us to use this file as a document style option
% without having to call \texttt{theorem} itself as an option.  At the
% same time, we assure that at least version 2 is loaded, since
% |\theorem@style| was not defined in earlier versions.
%    \begin{macrocode}
\@ifundefined{theorem@style}{\input{theorem.sty}}{}
\ProvidesFile{thp.sty}
             [\FMithmInfo]
%    \end{macrocode}
%
%
% \begin{macro}{\th@plain}
% \begin{macro}{\@begintheorem}
% \begin{macro}{\@opargbegintheorem}
%    |\theoremstyle{plain}| corresponds to the original
%    definition, except that the distances to the surrounding text are
%    determined by the parameters |\theorempreskipamount| and
%    |\theorempostskipamount|.
%    First we set the default body font.
%    \begin{macrocode}
\gdef\th@plain{\normalfont\itshape
%    \end{macrocode}
%    Then we define |\@begintheorem| and
%    |\@opargbegintheorem|. These two macros define how the
%    header of a theorem is typeset.  |\@opargbegintheorem| will
%    be called if a \textsf{theorem} environment with an optional
%    argument is encountered; otherwise, the header is constructed by
%    calling |\@begintheorem|.  If one of these macros is
%    executed, we are within a \textsf{trivlist} environment started by
%    |\@thm|.  So the theorem header is produced with an
%    |\item| command.
%
%    Instead of specifying the header font directly, all standard
%    theorem styles use the |\theorem@headerfont| macro to allow
%    customization.  The extra space (|\labelsep|) is necessary
%    because of problems in the \textsf{trivlist} environment.
%    \begin{macrocode}
  \def\@begintheorem##1##2{%
        \item[\hskip\labelsep \theorem@headerfont ##1\ ##2]}%
%    \end{macrocode}
%    The definition of |\@opargbegintheorem| is completely
%    analogous. The only difference is the fact that there exists a
%    third argument (which is the optional parameter of the
%    environment and contains additional information about the
%    theorem).  Customarily we enclose it in parentheses.
%    \begin{macrocode}
\def\@opargbegintheorem##1##2##3{%
   \item[\hskip\labelsep \theorem@headerfont ##1\ ##2\ (##3)]}}
%    \end{macrocode}
% \end{macro}
% \end{macro}
% \end{macro}
%
% \noindent
% We conclude with an |\endgroup| to restore the
% |\catcode| of |@|.
%    \begin{macrocode}
\endgroup
%</thp>
%    \end{macrocode}
%
%
% \subsubsection{The \texttt{break} style}
%
% This style option is stored in the file |thb.sty|.
% For the next two lines see the documentation for |\th@plain| on
% page~\pageref{dok:th@plain}.
%    \begin{macrocode}
%<*thb>
\begingroup \makeatletter
\@ifundefined{theorem@style}{\input{theorem.sty}}{}
\ProvidesFile{thb.sty}
             [\FMithmInfo]
%    \end{macrocode}
%
% \begin{macro}{\th@break}
%    |\theoremstyle{break}| produces a line break after the name
%    of the theorem.  The font is |\slshape|.  Hence, we define
%    |\th@break| as follows:
%    \begin{macrocode}
\gdef\th@break{\normalfont\slshape
  \def\@begintheorem##1##2{\item[%
%    \end{macrocode}
%    We run into the following problem: it is not possible to create
%    the header with |\item[|\meta{title}|]| and then start
%    a new line by, for example, |\mbox{}\\|. Such a definition
%    will fail whenever a list environment follows immediately.  With
%    the above construction, the |\mbox{}| causes the switch
%    |@inlabel| (cf.\ definition of |\list| and
%    |\trivlist| in \cite{latex:source}) to be set to \texttt{false}
%    and so the following list will insert additional vertical
%    space (|\topskip|). This is quite annoying. Therefore, we
%    create the line break within the |\item|. In order to ensure
%    that the text will begin at the proper position in the following
%    line, we simply pretend that the label does not take any
%    room.\footnote{This will lead to problems whenever very high
%    symbols occurring in the line tower into the
%    heading. So, something else has to be done here sometime.}
%    \begin{macrocode}
       \rlap{\vbox{\hbox{\hskip \labelsep\theorem@headerfont ##1\ ##2}%
                   \hbox{\strut}}}]}%
%    \end{macrocode}
%    Again, the definition of |\@opargbegintheorem| is
%    completely analogous.
%    \begin{macrocode}
\def\@opargbegintheorem##1##2##3{%
  \item[\rlap{\vbox{\hbox{\hskip \labelsep \theorem@headerfont
                     ##1\ ##2\ (##3)}%
                    \hbox{\strut}}}]}}
\endgroup
%</thb>
%    \end{macrocode}
% \end{macro}
%
%
% \subsubsection{The \texttt{changebreak} style}
%
%    \begin{macrocode}
% This style option is stored in the file |thcb.sty|.
%    \begin{macrocode}
%<*thcb>
\begingroup \makeatletter
\@ifundefined{theorem@style}{\input{theorem.sty}}{}
\ProvidesFile{thcb.sty}
             [\FMithmInfo]
%    \end{macrocode}
%
% \begin{macro}{\th@changebreak}
%    The \texttt{change-break} theorem style is like \texttt{break}
%    but with interchange of theorem name and theorem number.  Thus we
%    define |\th@changebreak| as follows:
%    \begin{macrocode}
\gdef\th@changebreak{\normalfont\slshape
  \def\@begintheorem##1##2{\item
       [\rlap{\vbox{\hbox{\hskip\labelsep \theorem@headerfont ##2\ ##1}%
                    \hbox{\strut}}}]}%
\def\@opargbegintheorem##1##2##3{%
  \item[\rlap{\vbox{\hbox{\hskip\labelsep \theorem@headerfont
                     ##2\ ##1\ (##3)}%
                    \hbox{\strut}}}]}}
\endgroup
%</thcb>
%    \end{macrocode}
% \end{macro}
%
%
% \subsubsection{The \texttt{change} style}
%
% This style option is stored in the file |thc.sty|.
%    \begin{macrocode}
%<*thc>
\begingroup \makeatletter
\@ifundefined{theorem@style}{\input{theorem.sty}}{}
\ProvidesFile{thc.sty}
             [\FMithmInfo]
%    \end{macrocode}
%
% \begin{macro}{\th@change}
%    The \texttt{change} theorem style corresponds to the
%    \texttt{change break} style without a linebreak after the header.
%    To say it in another way, it's the same as the \texttt{plain}
%    style but with number and name interchanged and |\slshape| as the
%    default font.
%    \begin{macrocode}
\gdef\th@change{\normalfont\slshape
  \def\@begintheorem##1##2{\item
       [\hskip\labelsep \theorem@headerfont ##2\ ##1]}%
\def\@opargbegintheorem##1##2##3{%
  \item[\hskip\labelsep \theorem@headerfont ##2\ ##1\ (##3)]}}
\endgroup
%</thc>
%    \end{macrocode}
% \end{macro}
%
%
% \subsubsection{The \texttt{marginbreak} style}
%
% This style option is the one used most often at Mainz.  It is saved
% in the file |thmb.sty|.
%    \begin{macrocode}
%<*thmb>
\begingroup \makeatletter
\@ifundefined{theorem@style}{\input{theorem.sty}}{}
\ProvidesFile{thmb.sty}
             [\FMithmInfo]
%    \end{macrocode}
%
% \begin{macro}{\th@marginbreak}
%    The \texttt{margin break} style is nearly the same as the
%    \texttt{change break} style. The only difference is the placement
%    of the theorem number.  We use |\llap| to place it in the left
%    margin.
%
%    In this style |\labelsep| denotes the separation between the
%    number and the text.
%    \begin{macrocode}
\gdef\th@marginbreak{\normalfont\slshape
  \def\@begintheorem##1##2{\item
       [\rlap{\vbox{\theorem@headerfont
                    \hbox{\llap{##2}\hskip\labelsep ##1}%
                    \hbox{\strut}}}]}%
\def\@opargbegintheorem##1##2##3{%
  \item[\rlap{\vbox{\theorem@headerfont
                    \hbox{\llap{##2}\hskip\labelsep ##1\ (##3)}%
                    \hbox{\strut}}}]}}
\endgroup
%</thmb>
%    \end{macrocode}
% \changes{v2.1c}{90/02/17}{`endgroup outside macrocode env.}
% \end{macro}
%
%
%
% \subsubsection{The \texttt{margin} style}
%
% This style option is stored in the file |thm.sty|.
%    \begin{macrocode}
%<*thm>
\begingroup \makeatletter
\@ifundefined{theorem@style}{\input{theorem.sty}}{}
\ProvidesFile{thm.sty}
             [\FMithmInfo]
%    \end{macrocode}
%
% \begin{macro}{\th@margin}
%    Again this is only a variant of the theorem styles described
%    above without any new ideas.
%    \begin{macrocode}
\gdef\th@margin{\normalfont\slshape
  \def\@begintheorem##1##2{\item
       [\theorem@headerfont \llap{##2}\hskip\labelsep ##1]}%
\def\@opargbegintheorem##1##2##3{%
  \item[\theorem@headerfont \llap{##2}\hskip\labelsep ##1\ (##3)]}}
\endgroup
%</thm>
%    \end{macrocode}
% \end{macro}
%
%
%
%
% \subsection{Final Definitions}
%
% \begin{macro}{\theorempreskipamount}
% \begin{macro}{\theorempostskipamount}
%    The \textsf{skip} parameters that regulate the vertical empty space
%    before and after the {\envfont theorem} environment have
%    to be allocated as well.
% \changes{v2.0f}{89/07/20}{Unnecessary `global' removed.}
%    \begin{macrocode}
%<*package>
\newskip\theorempreskipamount
\newskip\theorempostskipamount
%    \end{macrocode}
%    Since we have used the same values for all theorem sets, we now can
%    assign them.
%    \begin{macrocode}
\global\setlength\theorempreskipamount{12pt plus 5pt minus 3pt}
\global\setlength\theorempostskipamount{8pt plus 3pt minus 1.5pt}
%    \end{macrocode}
% \end{macro}
% \end{macro}
%
%
% \begin{macro}{\@endtheorem}
%    The same holds for the macro |\@endtheorem|, which ends a
%    {\envfont theorem} environment. Since it is the same for all
%    theorem sets, it is removed from the macros |\th@|\meta{style}.
%    It simply ends the {\envfont trivlist} environment, which was
%    begun in |\@thm|.
%    \begin{macrocode}
\global\let\@endtheorem=\endtrivlist
%    \end{macrocode}
% \end{macro}
%
%
%
% \begin{macro}{\@preamblecmds}
%    All macros defined above are to be used only in the preamble.
%    Therefore, we insert them in |\@preamblecmds| which will disable
%    them at begin document. This is done by the internal \LaTeXe{}
%    command |\@onlypreamble|.
% \changes{v2.0g}{89/09/19}{`theoremheaderfont and `theorembodyfont
%    added.}
% \changes{v2.2a}{94/02/03}{Use \LaTeXe{} interface}
%    \begin{macrocode}
\@onlypreamble\@xnthm
\@onlypreamble\@ynthm
\@onlypreamble\@othm
\@onlypreamble\newtheorem
\@onlypreamble\theoremstyle
\@onlypreamble\theorembodyfont
\@onlypreamble\theoremheaderfont
%    \end{macrocode}
% \end{macro}
%
% Finally we declare the \texttt{plain} theorem style to be the default.
% \changes{v2.1b}{90/06/04}{Fixed missing `item by declaring default
%    style.}
%    \begin{macrocode}
\theoremstyle{plain}
%</package>
%    \end{macrocode}
%
% \Finale

\endinput
}{}
\ProvidesFile{thb.sty}
             [\FMithmInfo]
%    \end{macrocode}
%
% \begin{macro}{\th@break}
%    |\theoremstyle{break}| produces a line break after the name
%    of the theorem.  The font is |\slshape|.  Hence, we define
%    |\th@break| as follows:
%    \begin{macrocode}
\gdef\th@break{\normalfont\slshape
  \def\@begintheorem##1##2{\item[%
%    \end{macrocode}
%    We run into the following problem: it is not possible to create
%    the header with |\item[|\meta{title}|]| and then start
%    a new line by, for example, |\mbox{}\\|. Such a definition
%    will fail whenever a list environment follows immediately.  With
%    the above construction, the |\mbox{}| causes the switch
%    |@inlabel| (cf.\ definition of |\list| and
%    |\trivlist| in \cite{latex:source}) to be set to \texttt{false}
%    and so the following list will insert additional vertical
%    space (|\topskip|). This is quite annoying. Therefore, we
%    create the line break within the |\item|. In order to ensure
%    that the text will begin at the proper position in the following
%    line, we simply pretend that the label does not take any
%    room.\footnote{This will lead to problems whenever very high
%    symbols occurring in the line tower into the
%    heading. So, something else has to be done here sometime.}
%    \begin{macrocode}
       \rlap{\vbox{\hbox{\hskip \labelsep\theorem@headerfont ##1\ ##2}%
                   \hbox{\strut}}}]}%
%    \end{macrocode}
%    Again, the definition of |\@opargbegintheorem| is
%    completely analogous.
%    \begin{macrocode}
\def\@opargbegintheorem##1##2##3{%
  \item[\rlap{\vbox{\hbox{\hskip \labelsep \theorem@headerfont
                     ##1\ ##2\ (##3)}%
                    \hbox{\strut}}}]}}
\endgroup
%</thb>
%    \end{macrocode}
% \end{macro}
%
%
% \subsubsection{The \texttt{changebreak} style}
%
%    \begin{macrocode}
% This style option is stored in the file |thcb.sty|.
%    \begin{macrocode}
%<*thcb>
\begingroup \makeatletter
\@ifundefined{theorem@style}{% \iffalse meta-comment
%
% Copyright (C) 1993-2024
%
% The LaTeX Project and any individual authors listed elsewhere
% in this file.
%
% This file is part of the Standard LaTeX `Tools Bundle'.
% -------------------------------------------------------
%
% It may be distributed and/or modified under the
% conditions of the LaTeX Project Public License, either version 1.3c
% of this license or (at your option) any later version.
% The latest version of this license is in
%    https://www.latex-project.org/lppl.txt
% and version 1.3c or later is part of all distributions of LaTeX
% version 2005/12/01 or later.
%
% The list of all files belonging to the LaTeX `Tools Bundle' is
% given in the file `manifest.txt'.
%
% \fi
%
%%
%
%
\def\FMithmInfo{2023/07/05 v2.2c Theorem extension package (FMi)}
%
% \ProvidesFile{theorem.dtx}[\FMithmInfo]
%\iffalse   % this is a METACOMMENT !
%%
%% Package `theorem' to use with LaTeX2e
%% Copyright (C) 1989-1995 Frank Mittelbach, all rights reserved.
%<+package>\@ifundefined{theorem@style}{}{\endinput}
%<+package>\NeedsTeXFormat{LaTeX2e}[1995/06/01]
%<+package>\ProvidesPackage{theorem}
%<+package>         [\FMithmInfo]
%
% \fi
%
% \changes{v1.0}{87/05/07}{First draft.}
% \changes{v1.0a}{87/05/07}{`thistheoremstyle' added.}
% \changes{v1.1a}{87/07/29}{New concept: redefine the internal
%                          \protect\LaTeX{} commands `@xnthm' etc.}
% \changes{v1.1b}{87/07/31}{`theoremfont' used to specify font.}
% \changes{v1.2a}{88/01/08}{`@endtheorem' now globally defined.}
% \changes{v1.2b}{88/01/11}{`newtheorem' only allowed in the preamble.}
% \changes{v1.3a}{88/02/14}{`theoremfont' now a toks register.}
%
% \changes{v2.0a}{88/11/16}{New implementation and new documentation.}
% \changes{v2.0a}{88/11/16}{No layout changes but internal modifications
%                           old substyles won't work.}
% \changes{v2.0b}{89/01/30}{`global' used for all definitions.}
% \changes{v2.0c}{89/06/15}{Documentation translated to English.
%                            Thanks to LLL and Nina}
% \changes{v2.1a}{90/02/15}{Support for new font selection.}
% \changes{v2.2a}{94/02/03}{Cleanup for LaTeX2e}
%
% \DoNotIndex{\@endparenv,\@ifdefinable,\@ifnextchar,\@ifundefined}
% \DoNotIndex{\@namedef,\@nameuse,\@spaces,\@tempa,\@warning}
% \DoNotIndex{\ }
% \DoNotIndex{\begingroup,\bfseries}
% \DoNotIndex{\csname}
% \DoNotIndex{\def,\do,\docdate}
% \DoNotIndex{\endcsname,\endgroup,\endinput,\endtrivlist,\expandafter}
% \DoNotIndex{\filedate,\fileversion}
% \DoNotIndex{\gdef,\global}
% \DoNotIndex{\hbox,\hskip}
% \DoNotIndex{\ignorespaces,\input,\itshape,\item}
% \DoNotIndex{\labelsep,\let,\llap}
% \DoNotIndex{\makeatletter}
% \DoNotIndex{\noexpand}
% \DoNotIndex{\refstepcounter,\relax,\rlap}
% \DoNotIndex{\setlength,\slshape,\space,\string,\strut}
% \DoNotIndex{\the,\trivlist,\typeout}
% \DoNotIndex{\vbox}
% \DoNotIndex{\xdef}
%
% \def\deflabel#1{\ttfamily #1\hfill}
% \def\deflist#1{\begin{list}{}{\settowidth\labelwidth{\ttfamily #1}%
%                               \setlength\leftmargin\labelwidth
%                               \addtolength\leftmargin\labelsep
%                               \let\makelabel\deflabel}}
% \def\enddeflist{\end{list}}
% \def\envfont{\normalfont\ttfamily}
%
% \MakeShortVerb{\|}
%
% \newcommand{\TUB}{{\sl TUGboat\/}}
%
% \GetFileInfo{theorem.dtx}
% \title{An Extension of the
%        \LaTeX{} theorem environment\thanks{This file
%        has version number \fileversion, last
%        revised \filedate.}}
% \author{Frank Mittelbach}
%
% \MaintainedByLaTeXTeam{tools}
% \maketitle
%
% \begin{abstract}
%    The macros described in this paper yield an extension of the
%    \LaTeX{} theorem mechanism. It is designed to satisfy the
%    different requirements of various journals. Thus, the
%    layout of the ``theorems'' can be manipulated by determining a
%    ``style''. This article describes not only the use, but
%    also the definition, of the necessary macros.
% \end{abstract}
%
% \section*{Preface to version 2.2}
%
% For \LaTeXe{} this package did not need any fundamental changes. I
% only modified the messages generated so that theorem layout styles
% will show up with the |\listfiles| command and cleaned the section
% on the New Font Selection Scheme since this is now included in
% \LaTeX.
%
% \section*{Preface to version 2.1}
%
% This version is identical to 2.0g described in \TUB\ 10\#3 except
% for some internal defaults which are now set depending on the used
% font selection scheme.
%
% This was done to avoid unpleasant surprises if the new font selection
% scheme is in force. For further details see
% section~\ref{sec:newfonts} and \cite{bk:GMS94}.
%
% \section{Introduction}
%
% For our purposes here, ``theorems'' are labelled enunciations,
% often set off from the main text by extra space and a font change.
% Theorems, corollaries, conjectures, definitions, and remarks
% are all instances of ``theorems''.  The ``header'' of these structures
% is composed of a label (such as \textsc{Theorem} or \textsc{Remark})
% and a number which serializes an item in the sequence of items
% with the same label.
%
% Shortly after the introduction of \LaTeX{} at the Fachbereich
% Mathematik in Mainz, the desire to manipulate the layout of
% ``theorems'' arose.  In Mainz, the following two conventions came
% into general use:
% \begin{enumerate}
%   \item The number of the theorem is shown in the margin.
%   \item There is a line break at
%         the end of the theorem header.
% \end{enumerate}
%
% Additionally, some journals require different formats which depend on
% the ``sort of theorem'': e.g.\ often remarks and definitions are set
% in |\upshape|, while |\itshape| is employed for main theorems.
%
% Confronted with these requirements, a theorem environment was
% developed in Mainz which allows separate determination of the
% layout of the ``theorems sets'', comparable to |\pagestyle|.
%
%
% \section{The user interface}
% \subsection{Defining new theorem sets}
%
% \DescribeMacro\newtheorem
% As in the original \LaTeX{} version, the command |\newtheorem|
% defines a new ``theorem set'' or ``theorem-like structure''.
% Two required arguments name the new environment and give the text
% to be typeset with each instance of the new ``set'', while
% an optional argument determines how
% the ``set'' is enumerated:
% \begin{description}
%    \item[\ttfamily \bslash newtheorem\{foo\}\{bar\}]
%       The theorem set {\envfont foo} (whose name is \texttt{bar})
%       uses its own counter.
%    \item[\ttfamily \bslash newtheorem\{foo2\}{[foo]}\{bar2\}]
%       The theorem set {\envfont foo2} (printed name \texttt{bar2})
%       uses the same counter as the theorem set \texttt{foo}.
%    \item[\ttfamily \bslash newtheorem\{foo3\}\{bar3\}{[section]}]
%       The theorem set {\envfont foo3} (printed name \texttt{bar3}) is
%       enumerated within the counter \texttt{section}, i.e.\ with every
%       new |\section| the enumeration begins again with 1, and
%       the enumeration is composed from the section-number and the
%       theorem counter itself.
% \end{description}
%
% \DescribeMacro\theoremstyle
% Additionally, the command |\theoremstyle| can define the layout
% of various, or all, theorem sets.  It should be noted that any
% theorem set defined by |\newtheorem| is typeset in the
% |\theoremstyle| that is current at the time of the definition.
% Thus, the following
% \begin{verbatim}
%    \theoremstyle{break}        \newtheorem{Cor}{Corollary}
%    \theoremstyle{plain}        \newtheorem{Exa}{Example}[section]
%\end{verbatim}                             ^^A No blank at line start
% leads to the result that the set {\envfont Cor} is formatted in the
% style \texttt{break}, while the set {\envfont Exa} and all the
% following ones are formatted in the style \texttt{plain}, unless
% another |\theoremstyle| follows.  Since the definitions installed by
% |\newtheorem| are global, one also can limit |\theoremstyle| locally
% by grouping braces.
%
% \DescribeMacro\theorembodyfont
% The choice of the font for the theorem body is completely
% independent of the chosen |\theoremstyle|; this has proven to
% be very advantageous.  For example,
% \begin{verbatim}
%   {\theorembodyfont{\upshape}          \newtheorem{Rem}{Remark}}
%\end{verbatim}
% defines a theorem set {\envfont Rem}, which will be set in
% |\upshape| in the current layout (which in our example is
% \texttt{plain}).  As with |\theoremstyle|, the |\theorembodyfont|
% chosen is that current at the time of |\newtheorem|.  If
% |\theorembodyfont| is not specified or one defines
% |\theorembodyfont{}|, then the font used will be that defined by the
% |\theoremstyle|.
%
% \DescribeMacro\theoremheaderfont It is also possible to customize
% the font used for the theorem headers.  This is, however, a global
% declaration, and therefore there should be at most one
% |\theoremheaderfont| declaration in the preamble.\footnote{If it is
% actually necessary to have different header fonts, one has to define
% new theorem styles (substituting the desired font) or specify the
% information directly in the \texttt{\bslash newtheorem} declaration
% (the unclean variant).}
%
% \DescribeMacro\theorempreskipamount
% \DescribeMacro\theorempostskipamount Two additional parameters
% affect the vertical space around the theorem environments:
% |\theorempreskipamount| and |\theorempostskipamount| define,
% respectively, the spacing before and after such an environment.
% These parameters apply for all theorem sets and can be manipulated
% with the ordinary length macros.  They are rubber lengths
% (`\textsf{skips}'), and therefore can contain \texttt{plus} and
% \texttt{minus} parts.
%
% Since the definition of theorem sets should---most sensibly---be
% placed in the preamble, we only allow installation there.
% It is therefore possible to release the memory used here after
% |\begin{document}|, in order to make room for other
% applications.
%
%
% \subsection{Existing theorem styles}
%
% The following theorem styles exist to date:
% \begin{deflist}{marginbreak}
%    \item[plain]
%       This theorem style emulates the original \LaTeX{} definition,
%       except that additionally the parameters
%       |\theorem...skipamount| are used.
%    \item[break]
%       In this style, the theorem header is followed by a line
%       break.
%    \item[marginbreak]
%       The theorem number is set in the margin, and there is a line
%       break as in \texttt{break}.
%    \item[changebreak]
%       Like \texttt{break}, but with header number and text
%       interchanged.
%    \item[change]
%       Header number and text are interchanged, without a line break.
%    \item[margin]
%       The number is set in the left margin, without a line break.
% \end{deflist}
% All styles (except \texttt{plain}) select |\slshape| as the default
% |\theorembodyfont|.
%
%
% \subsection{Examples}
%
% Given the above theorem sets {\envfont Cor}, {\envfont Exa}
% and {\envfont Rem},
% suppose that the preamble also contains the declarations:
% \begin{verbatim}
%    \theoremstyle{marginbreak}   \newtheorem{Lem}[Cor]{Lemma}
%    \theoremstyle{change}
%    \theorembodyfont{\itshape}        \newtheorem{Def}[Cor]{Definition}
%
%    \theoremheaderfont{\scshape}
%\end{verbatim}
% Then the following are some typical examples of the typeset output
% resulting from their use.
%
% \begin{Cor}
%    This is a sentence typeset in the theorem environment {\envfont
%    Cor}.
% \end{Cor}
%
% \begin{Exa}
%    This is a sentence typeset in the theorem environment {\envfont
%    Exa}.
% \end{Exa}
%
% \begin{Rem}
%    This is a sentence typeset in the theorem environment {\envfont
%    Rem}.
% \end{Rem}
%
% \begin{Lem}[Ben User]
%    This is a sentence typeset in the theorem environment {\envfont
%    Lem}.
% \end{Lem}
%
% \begin{Def}[Very impressive Definition]
%    This is a sentence typeset in the theorem environment {\envfont
%    Def}.
% \end{Def}
%
% The last two examples show the effect of the optional argument to
% a theorem environment (it is the text typeset in parentheses).
%
%
% \section{Special Considerations}
% \label{sec:newfonts}
%
% Theoremheader and body are implemented as a unit. This means that
% the |\theoremheaderfont|  will inherit characteristics of the
% |\theorembodyfont| in \LaTeXe. Thus, if for example
% |\theorembodyfont| is |\itshape| and |\theoremheaderfont|
% is |\bfseries| the font selected for the header will have the
% characteristics `bold extended italic'. If this is not desired one
% should set the |\theoremheaderfont| to something like
% \begin{verbatim}
%      \theoremheaderfont{\normalfont\bfseries}
%\end{verbatim}
% i.e.\ supplying all necessary font information explicitly.
%
%
%
% \section{Acknowledgements}
%
% The publication of this set of macros was only possible with the
% help of Christina Busse (translating the manuscript into English),
% Joachim Pense (playing the r\^ole of typist), Chris Rowley
% (looking everything over) and many others providing useful
% suggestions.
%
% \MaybeStop{%
%    \begin{thebibliography}{9}
%       \bibitem{bk:GMS94} \textsc{M.~Goossens}, \textsc{F.~Mittelbach}
%         and \textsc{A.~Samarin}.
%         \newblock The \LaTeX{} Companion.
%         \newblock
%         Addison-Wesley, Reading, Massachusetts, 1994.
%       \bibitem{latex:source} \textsc{Lamport, Leslie.} \newblock
%                               \texttt{latex.tex}, version 2.09, date
%                               Feb. 1990.
%    \end{thebibliography}
%    \PrintIndex
%    \PrintChanges}
%
%
% \section{The documentation driver file}
%
% The next bit of code contains the documentation driver file for
% \TeX{}, i.e., the file that will produce the documentation you are
% currently reading. It will be extracted from this file by the
% \texttt{docstrip} program. Since it is the first code in the file
% one can alternatively process this file directly with \LaTeXe{} to
% obtain the documentation.
%
%    \begin{macrocode}
%<*driver>
\documentclass{ltxdoc}

\usepackage{theorem}

%<+driver>% The next few lines define theorem sets which are used
%<+driver>% in the example section of the documentation.

\theoremstyle{break}       \newtheorem{Cor}{Corollary}
\theoremstyle{plain}       \newtheorem{Exa}{Example}[section]
{\theorembodyfont{\upshape}\newtheorem{Rem}{Remark}}
\theoremstyle{marginbreak} \newtheorem{Lem}[Cor]{Lemma}
\theoremstyle{change}
\theorembodyfont{\itshape} \newtheorem{Def}[Cor]{Definition}

\theoremheaderfont{\scshape}

\RecordChanges

\begin{document}
   \DocInput{theorem.dtx}
\end{document}
%</driver>
%    \end{macrocode}
%
% \section{Definition of the Macros}
%
% If the file has been loaded before, we abort immediately. If not the
% package announces itself (this is actually done at the very top of
% the file---the way it is done isn't good style so don't copy it).
% \changes{v2.0e}{89/07/19}{Spaces removed from `typeout'.}
% \changes{v2.2b}{95/11/19}{Announce moved to top of file}
%    \begin{macrocode}
%<*package>
%\@ifundefined{theorem@style}{}{\endinput}
%\def\FMithmInfo{1995/11/19 v2.2b Theorem extension package (FMi)}
%\ProvidesPackage{theorem}[\FMithmInfo]
%    \end{macrocode}
%
%
% \subsection{Definition of theorem styles and fonts}
%
% All the definitions in this file are done globally to allow
% inputting this file inside a group.
%
% \begin{macro}{\theoremstyle}
%    Before a theorem style can be installed, the chosen style must be
%    known. For that reason, we must test to see that
%    |\th@|\meta{style} is known or, more precisely, that it is
%    different from |\relax|. If the style is not known then
%    |\th@plain| is used.
%    \begin{macrocode}
\gdef\theoremstyle#1{%
   \@ifundefined{th@#1}{\@warning
          {Unknown theoremstyle `#1'. Using `plain'}%
          \theorem@style{plain}}%
%    \end{macrocode}
%    We save the theorem style to be used in the token register
%    |\theorem@style|.
%    \begin{macrocode}
      {\theorem@style{#1}}%
%    \end{macrocode}
%    Now we ``evaluate'' the theorem style: this means, we call the
%    macro |\th@|\meta{style} which will activate the relevant
%    definitions which are contained in a separate file.  This is done
%    in a group to suppress changes to the current font.  This could
%    otherwise pose problems together with the new font selection
%    scheme\footnote{When I printed the original article using the new
%    font selection scheme I ended with a document with slanted
%    typefaces (text headings and all) simply because one of the
%    theorem styles used \texttt{\bslash sl} at toplevel.} if the
%    |\th@|\meta{style} is evaluated a second time.
% \changes{v2.1a}{90/02/15}{Group added around `th@..}
% \changes{v2.1a}{90/02/15}{Reseting of `..@begintheorem removed.}
%    \begin{macrocode}
      \begingroup
        \csname th@\the\theorem@style \endcsname
      \endgroup}
%    \end{macrocode}
% \end{macro}
%
%
% \begin{macro}{\@begintheorem}
% \changes{v2.1a}{90/02/15}{Relaxed at toplevel.}
% \begin{macro}{\@opargbegintheorem}
% \changes{v2.1a}{90/02/15}{Relaxed at toplevel.}
%    We reset |\@begintheorem| and |\@opargbegintheorem| to
%    |\relax| since these commands are no longer necessary at
%    toplevel. This will save a few tokens.
%    \begin{macrocode}
\global\let\@begintheorem\relax
\global\let\@opargbegintheorem\relax
%    \end{macrocode}
% \end{macro}
% \end{macro}
%
% \begin{macro}{\theorem@style}
%    Obviously the \textsf{token} register used above has to be
%    allocated. To assure the utmost compatibility with the
%    original \LaTeX{} definition, we set the default theorem style to
%    \texttt{plain}, which implements the usual \LaTeX{} convention.
% \changes{v2.0f}{89/07/20}{Unnecessary `global' removed.}
%    \begin{macrocode}
\newtoks\theorem@style
\global\theorem@style{plain}
%    \end{macrocode}
% \end{macro}
%
% \begin{macro}{\theorembodyfont}
% \begin{macro}{\theorem@bodyfont}
%    If the |\theorembodyfont| is set by the user then it should not
%    interact with the default font set in the theorem style. When the
%    new font selection is in force this may happen if, for example,
%    the default is |\itshape| and the new |\theorembodyfont| is
%    |\sffamily|. So we add a |\reset@font| command in front of the
%    user definition.
% \changes{v2.0f}{89/07/20}{Unnecessary `global' removed.}
% \changes{v2.0g}{89/09/19}{`theoremfont to `theorembodyfont.}
% \changes{v2.1c}{92/06/22}{`reset@font added and logic changed.}
% \changes{v2.1d}{92/10/28}{Reset to empty if arg is empty}
%    \begin{macrocode}
\gdef\theorembodyfont#1{%
%    \end{macrocode}
%    We check if the argument supplied is empty and if so put nothing
%    into the |\theorem@bodyfont| token register to allow for
%    |\theorembodyfont{}| as a mean of using the default of the
%    current |\theoremstyle|.
%    \begin{macrocode}
   \def\@tempa{#1}%
   \ifx\@tempa\@empty
    \theorem@bodyfont{}%
   \else
    \theorem@bodyfont{\reset@font#1}%
   \fi
   }
\newtoks\theorem@bodyfont
\global\theorem@bodyfont{}
%    \end{macrocode}
% \end{macro}
% \end{macro}
%
%
% \begin{macro}{\theoremheaderfont}
% \changes{v2.0g}{89/09/19}{Macro added.}
%    The font for the theorem headers is handled differently because
%    this definition applies to all theorem styles.
%    \begin{macrocode}
\gdef\theoremheaderfont#1{\gdef\theorem@headerfont{#1}%
%    \end{macrocode}
%    After using the macro once it is redefined to produce an error
%    message.
%    \begin{macrocode}
       \gdef\theoremheaderfont##1{%
        \typeout{\string\theoremheaderfont\space should be used
                 only once.}}}
%    \end{macrocode}
% \end{macro}
%
% \begin{macro}{\theorem@headerfont}
% \changes{v2.0g}{89/09/19}{Macro added.}
%    To set the |\theorem@headerfont| default we first test if
%    the new fontselection scheme is in force.
% \changes{v2.1a}{90/02/15}{Support for new font selection.}
%    \begin{macrocode}
\ifx\upshape\undefined
%    \end{macrocode}
%    If not we define it to expand into |\bfseries|. We don't use
%    |\let| just in case a following style option redefines this
%    macro.
%    \begin{macrocode}
\gdef\theorem@headerfont{\bfseries}
%    \end{macrocode}
%    Otherwise we reset the current shape before calling |\bfseries|.
%    \begin{macrocode}
\else \gdef\theorem@headerfont{\normalfont\bfseries}\fi
%    \end{macrocode}
% \end{macro}
%
% \begin{macro}{\th@plain}
% \begin{macro}{\th@break}
% \begin{macro}{\th@marginbreak}
% \begin{macro}{\th@changebreak}
% \begin{macro}{\th@change}
% \begin{macro}{\th@margin}
%    The different styles are defined in macros such as
%    |\th@plain|.  Since memory space is precious in
%    ``non-Big-versions'', we have to avoid offering too many unused
%    definitions.  Therefore we define these styles in separate files
%    that can be loaded on demand.  Thus the commands themselves only
%    load these files. We use |\@input@|, a \LaTeXe{} internal command
%    that ensures that the file will be listed with |\listfiles|.
% \changes{v2.2a}{94/02/02}{Use `@input@ to load theorem layout files}
%    \begin{macrocode}
\gdef\th@plain{\@input@{thp.sty}}
\gdef\th@break{\@input@{thb.sty}}
\gdef\th@marginbreak{\@input@{thmb.sty}}
\gdef\th@changebreak{\@input@{thcb.sty}}
\gdef\th@change{\@input@{thc.sty}}
\gdef\th@margin{\@input@{thm.sty}}
%    \end{macrocode}
%    This list will be expanded when new styles become
%    available.  For testing,  just append new theorem substyles
%    as document options.
% \end{macro}
% \end{macro}
% \end{macro}
% \end{macro}
% \end{macro}
% \end{macro}
%
%
%
% \subsection{Definition of a new theorem set}
%
% As already pointed out, a new theorem environment can be defined in
% three different ways:
% \begin{verbatim}
%    \newtheorem{Lem}{Lemma}
%    \newtheorem{Lem}{Lemma}[section]
%    \newtheorem{Lem}[Theorem]{Lemma}
%\end{verbatim}
% The function of the macro |\newtheorem| is to recognize these cases
% and then to branch into one of the three macros |\@ynthm|, |\@xnthm|
% or |\@othm|.  This mechanism is adopted unchanged from
% \cite{latex:source}; the essential point here is that, for example,
% in the second case, the arguments \texttt{Lem}, \texttt{Lemma} and
% \texttt{section} are passed over to the macro |\@xnthm|.
%
% We inspect this case first because the others present fewer problems,
% and thus are easily derived from this one.
%
% \begin{macro}{\@xnthm}
%    For our example arguments, the macro |\@xnthm| must fulfill
%    the following:
%    \begin{itemize}
%       \item
%          Define a new \LaTeX-counter `Lem'
%       \item
%          reset this counter within a |\section|
%       \item
%          define the macro |\theLem|
%       \item
%          define the environment macros |\Lem| and
%          |\endLem| using the current
%          |\theoremstyle| and |\theorem@bodyfont|.
%    \end{itemize}
%    Obviously, all this should happen only if the first argument
%    of |\@xnthm| (i.e.\ \texttt{Lem} in our example) is chosen so
%    as not to conflict with any previously defined commands or
%    environments.  This test is performed by the \LaTeX{} macro
%    |\@ifdefinable|.
%    \begin{macrocode}
\gdef\@xnthm#1#2[#3]{\expandafter\@ifdefinable\csname #1\endcsname
%    \end{macrocode}
%    Therefore, the first argument of |\@ifdefinable| is the
%    expansion (in the example, |\Lem|) of
%    |\csname||#1\endcsname|.  The second argument is
%    executed only if the test has been completed successfully.
%    \begin{macrocode}
   {%
%    \end{macrocode}
%    Now we define the new counter. The names of the \LaTeX{}
%    macros employed should speak for themselves:
% \changes{v2.2b}{1995/10/07}{Use \cs{@newctr}}
% \changes{v2.2c}{1995/11/23}{Use \cs{@newctr} properly.}
%    \begin{macrocode}
    \@definecounter{#1}\@newctr{#1}[#3]%
%    \end{macrocode}
%    Using |\@newctr| will give a proper error message if the counter
%    in |#3| is not defined.
%    In defining `|\theLem|' we must generate the desired macro
%    name by use of |\expandafter| and |\csname|.
%    \begin{macrocode}
    \expandafter\xdef\csname the#1\endcsname
%    \end{macrocode}
%    An |\xdef| is used in order to make the definition global,
%    and to ensure that it contains the replacement texts of
%    |\@thmcountersep| and |\@thmcounter|.\footnote{These
%    two macros can be defined by the document style.  Their default
%    values produce a `\texttt{.}' as separation and an arabic
%    representation of the number.} However, not everything should be
%    expanded. For example, it saves space to use |\thesection|
%    instead of its---at times---lengthy expansion.
%    \begin{macrocode}
      {\expandafter \noexpand \csname the#3\endcsname
       \@thmcountersep \@thmcounter{#1}}%
%    \end{macrocode}
%    Thus with the defaults of \LaTeX{}, |\theLem| would be
%    replaced by the command sequence |\thesection.\arabic{Lem}|.
%
%    We will now look at the definition of the macro which is executed
%    at the beginning of the actual environment (in our example this
%    macro is |\Lem|). It should be noted that we use an
%    ``|\expandafter| trick'' to expand only certain parts of the
%    replacement text at the time of the definition.
%    \begin{macrocode}
    \def\@tempa{\global\@namedef{#1}}%
    \expandafter \@tempa \expandafter{%
%    \end{macrocode}
%    First, the macro that contains the current definitions of
%    |\@begintheorem| and |\@opargtheorem| should be called
%    up.  The name of this macro---as is already known---has the form
%    |\th@|\meta{theorem\ style}; therefore, it must be called by
%    \begin{macrocode}
      \csname th@\the \theorem@style
%    \end{macrocode}
%    In addition the default theorem font should be changeable, i.e.\
%    we have to insert the contents of |\theorem@bodyfont|. For
%    that reason, we expand even further, beyond |\endcsname|,
%    and thus insert the contents of the \textsf{token} register
%    |\theorem@bodyfont| in the replacement text.
%    \begin{macrocode}
            \expandafter \endcsname \the \theorem@bodyfont
%    \end{macrocode}
%    Now it is time to call the macro |\@thm| which takes over
%    the further processing.  It has two arguments: the current
%    counter name (in our example, \texttt{Lem}), and the text of the
%    label (in our example, \texttt{Lemma}).
%    \begin{macrocode}
     \@thm{#1}{#2}}%
%    \end{macrocode}
%    With this, the `sub-definition' is complete.  The macro
%    |\@endtheorem| ends a theorem environment and is, so far, nothing
%    but an |\endtrivlist|.  (Hence it is defined globally, and not
%    within the theorem styles.\footnote{This has to be changed as
%    soon as theorem styles that change \texttt{\bslash @endtheorem}
%    exist.  In such a case, all existing styles must be changed as
%    well since they will have to reset the macro.}) Therefore, we can
%    set it equivalent to the macro that ends the theorem set (in our
%    example, |\endLem|).  However, if some day theorem styles exist
%    that do change |\@endtheorem|, we would have to use the
%    commented-out line instead.
%    \begin{macrocode}
    \global \expandafter \let \csname end#1\endcsname \@endtheorem
%   \global\@namedef{end#1}{\@endtheorem}%
%    \end{macrocode}
%    With these commands all the required definitions are employed,
%    unless the test |\@ifdefinable| has failed.  Therefore, we
%    end the second argument of this macro and with it the definition
%    of |\@xnthm|.
%    \begin{macrocode}
   }}
%    \end{macrocode}
% \end{macro}
%
% \begin{macro}{\@ynthm}
%    The definition of |\@ynthm| is completely analogous. In
%    this case the new counter that is defined is not reset within
%    another counter; thus the definition of |\the...|
%    is simplified:
%    \begin{macrocode}
\gdef\@ynthm#1#2{\expandafter\@ifdefinable\csname #1\endcsname
   {\@definecounter{#1}%
    \expandafter\xdef\csname the#1\endcsname{\@thmcounter{#1}}%
%    \end{macrocode}
%    The rest of the definition corresponds literally to that of
%    |\@xnthm|:
%    \begin{macrocode}
    \def\@tempa{\global\@namedef{#1}}\expandafter \@tempa
     \expandafter{\csname th@\the \theorem@style \expandafter
     \endcsname \the\theorem@bodyfont \@thm{#1}{#2}}%
    \global \expandafter \let \csname end#1\endcsname \@endtheorem}}
%    \end{macrocode}
% \end{macro}
%
% \begin{macro}{\@othm}
%    The definition of |\@othm| does not contain anything new.
%    \begin{macrocode}
\gdef\@othm#1[#2]#3{%
%    \end{macrocode}
% \changes{v2.2b}{95/10/07}{Build in check for counter defined}
%    We do not define a new counter but instead use one that has
%    already been defined.  Thus the only definition we need is that
%    of this pseudo-counter (i.e.\ |\the|\meta{env.\ name}).
%    First we check if |#2| corresponds to a known counter name.
%    \begin{macrocode}
  \expandafter\ifx\csname c@#2\endcsname\relax
   \@nocounterr{#2}%
  \else
   \expandafter\@ifdefinable\csname #1\endcsname
   {\expandafter \xdef \csname the#1\endcsname
     {\expandafter \noexpand \csname the#2\endcsname}%
%    \end{macrocode}
%    All other parts of the definition can be adopted from
%    |\@xnthm|. We have to remember, though, that in this case
%    the name of the current counter and the theorem label have moved
%    to the second and third arguments.
%    \begin{macrocode}
    \def\@tempa{\global\@namedef{#1}}\expandafter \@tempa
     \expandafter{\csname th@\the \theorem@style \expandafter
     \endcsname \the\theorem@bodyfont \@thm{#2}{#3}}%
    \global \expandafter \let \csname end#1\endcsname \@endtheorem}%
  \fi}
%    \end{macrocode}
% \end{macro}
%
% \subsection{Macros that are employed in a {\envfont theorem}
%    environment}
%
% \begin{macro}{\@thm}
%    The macro |\@thm| has to increase the current counter.  Then,
%    depending on whether the environment has (or does not have) an
%    optional argument, it has to branch into either |\@begintheorem|
%    or |\@opargtheorem|.
%    \begin{macrocode}
\gdef\@thm#1#2{\refstepcounter{#1}%
%    \end{macrocode}
%    Now we start a {\envfont trivlist} environment, and give
%    |\@topsep| and |\@topsepadd| the values of the \textsf{skip}
%    registers |\theorempreskipamount| and |\theorempostskipamount|.
%    The value in |\@topsep| is the vertical space that is inserted by
%    the first (and only) |\item| in our |\trivlist| whilst
%    |\@topsepadd| is inserted by |\@endparenv| at the end of that
%    {\envfont trivlist} environment.  By using these registers, we
%    obtain the desired space around a {\envfont theorem} environment.
%    \begin{macrocode}
   \trivlist
   \@topsep \theorempreskipamount               % used by first \item
   \@topsepadd \theorempostskipamount           % used by \@endparenv
%    \end{macrocode}
%    Now we have to test whether an optional argument has been given.
% \changes{v2.0d}{89/07/18}{Forgotten percent added.}
%    \begin{macrocode}
   \@ifnextchar [%
%    \end{macrocode}
%    If there is an optional argument, we will call |\@ythm|, and
%    move the arguments read back into the input stream.
%    \begin{macrocode}
   {\@ythm{#1}{#2}}%
%    \end{macrocode}
%    If not, we call |\@begintheorem|. Its first argument is the
%    name of the theorem set (hence the second argument of
%    |\@thm|). Its second argument is the macro that produces the
%    current number.
%    \begin{macrocode}
   {\@begintheorem{#2}{\csname the#1\endcsname}\ignorespaces}}
%    \end{macrocode}
% \end{macro}
%
% \begin{macro}{\@xthm}
% \begin{macro}{\@ythm}
%    Both these macros were originally called by |\@thm|.  We do
%    not need |\@xthm| anymore, hence we reset it to
%    |\relax|. The definition of |\@ythm| has not changed at
%    all from its definition in \LaTeX. In order to make the macros
%    easier to understand, we will
%    nevertheless present it (commented out).
%    \begin{macrocode}
\global\let\@xthm\relax
% \def\@ythm#1#2[#3]{\@opargbegintheorem{#2}{\csname
%           the#1\endcsname}{#3}\ignorespaces}
%</package>
%    \end{macrocode}
%    The primitive |\ignorespaces| in |\@ythm| and
%    |\@thm| is needed to remove the spaces between the
%    |\begin{...}| and the actual text.
% \end{macro}
% \end{macro}
%
%
% \subsection{Definition of the theorem substyles}
%
% As already pointed out, the theorem substyles, defined below, are
% only loaded when necessary.  Note that all these substyles, except
% \texttt{plain}, have |\slshape| as the default body font.
%
% \subsubsection{The \texttt{plain} style}
%
% \label{dok:th@plain} As the following macros use |@|, we have
% to locally set the |\catcode| of this symbol to ``letter''.
% This happens within a group, so that we do not have to worry about
% which |\catcode| that symbol had before.
%    \begin{macrocode}
%<*thp>
\begingroup \makeatletter
%    \end{macrocode}
% Since we are now within a group, we must make all definitions
% globally.  First we make sure that \texttt{theorem.sty} is loaded.
% This will allow us to use this file as a document style option
% without having to call \texttt{theorem} itself as an option.  At the
% same time, we assure that at least version 2 is loaded, since
% |\theorem@style| was not defined in earlier versions.
%    \begin{macrocode}
\@ifundefined{theorem@style}{\input{theorem.sty}}{}
\ProvidesFile{thp.sty}
             [\FMithmInfo]
%    \end{macrocode}
%
%
% \begin{macro}{\th@plain}
% \begin{macro}{\@begintheorem}
% \begin{macro}{\@opargbegintheorem}
%    |\theoremstyle{plain}| corresponds to the original
%    definition, except that the distances to the surrounding text are
%    determined by the parameters |\theorempreskipamount| and
%    |\theorempostskipamount|.
%    First we set the default body font.
%    \begin{macrocode}
\gdef\th@plain{\normalfont\itshape
%    \end{macrocode}
%    Then we define |\@begintheorem| and
%    |\@opargbegintheorem|. These two macros define how the
%    header of a theorem is typeset.  |\@opargbegintheorem| will
%    be called if a \textsf{theorem} environment with an optional
%    argument is encountered; otherwise, the header is constructed by
%    calling |\@begintheorem|.  If one of these macros is
%    executed, we are within a \textsf{trivlist} environment started by
%    |\@thm|.  So the theorem header is produced with an
%    |\item| command.
%
%    Instead of specifying the header font directly, all standard
%    theorem styles use the |\theorem@headerfont| macro to allow
%    customization.  The extra space (|\labelsep|) is necessary
%    because of problems in the \textsf{trivlist} environment.
%    \begin{macrocode}
  \def\@begintheorem##1##2{%
        \item[\hskip\labelsep \theorem@headerfont ##1\ ##2]}%
%    \end{macrocode}
%    The definition of |\@opargbegintheorem| is completely
%    analogous. The only difference is the fact that there exists a
%    third argument (which is the optional parameter of the
%    environment and contains additional information about the
%    theorem).  Customarily we enclose it in parentheses.
%    \begin{macrocode}
\def\@opargbegintheorem##1##2##3{%
   \item[\hskip\labelsep \theorem@headerfont ##1\ ##2\ (##3)]}}
%    \end{macrocode}
% \end{macro}
% \end{macro}
% \end{macro}
%
% \noindent
% We conclude with an |\endgroup| to restore the
% |\catcode| of |@|.
%    \begin{macrocode}
\endgroup
%</thp>
%    \end{macrocode}
%
%
% \subsubsection{The \texttt{break} style}
%
% This style option is stored in the file |thb.sty|.
% For the next two lines see the documentation for |\th@plain| on
% page~\pageref{dok:th@plain}.
%    \begin{macrocode}
%<*thb>
\begingroup \makeatletter
\@ifundefined{theorem@style}{\input{theorem.sty}}{}
\ProvidesFile{thb.sty}
             [\FMithmInfo]
%    \end{macrocode}
%
% \begin{macro}{\th@break}
%    |\theoremstyle{break}| produces a line break after the name
%    of the theorem.  The font is |\slshape|.  Hence, we define
%    |\th@break| as follows:
%    \begin{macrocode}
\gdef\th@break{\normalfont\slshape
  \def\@begintheorem##1##2{\item[%
%    \end{macrocode}
%    We run into the following problem: it is not possible to create
%    the header with |\item[|\meta{title}|]| and then start
%    a new line by, for example, |\mbox{}\\|. Such a definition
%    will fail whenever a list environment follows immediately.  With
%    the above construction, the |\mbox{}| causes the switch
%    |@inlabel| (cf.\ definition of |\list| and
%    |\trivlist| in \cite{latex:source}) to be set to \texttt{false}
%    and so the following list will insert additional vertical
%    space (|\topskip|). This is quite annoying. Therefore, we
%    create the line break within the |\item|. In order to ensure
%    that the text will begin at the proper position in the following
%    line, we simply pretend that the label does not take any
%    room.\footnote{This will lead to problems whenever very high
%    symbols occurring in the line tower into the
%    heading. So, something else has to be done here sometime.}
%    \begin{macrocode}
       \rlap{\vbox{\hbox{\hskip \labelsep\theorem@headerfont ##1\ ##2}%
                   \hbox{\strut}}}]}%
%    \end{macrocode}
%    Again, the definition of |\@opargbegintheorem| is
%    completely analogous.
%    \begin{macrocode}
\def\@opargbegintheorem##1##2##3{%
  \item[\rlap{\vbox{\hbox{\hskip \labelsep \theorem@headerfont
                     ##1\ ##2\ (##3)}%
                    \hbox{\strut}}}]}}
\endgroup
%</thb>
%    \end{macrocode}
% \end{macro}
%
%
% \subsubsection{The \texttt{changebreak} style}
%
%    \begin{macrocode}
% This style option is stored in the file |thcb.sty|.
%    \begin{macrocode}
%<*thcb>
\begingroup \makeatletter
\@ifundefined{theorem@style}{\input{theorem.sty}}{}
\ProvidesFile{thcb.sty}
             [\FMithmInfo]
%    \end{macrocode}
%
% \begin{macro}{\th@changebreak}
%    The \texttt{change-break} theorem style is like \texttt{break}
%    but with interchange of theorem name and theorem number.  Thus we
%    define |\th@changebreak| as follows:
%    \begin{macrocode}
\gdef\th@changebreak{\normalfont\slshape
  \def\@begintheorem##1##2{\item
       [\rlap{\vbox{\hbox{\hskip\labelsep \theorem@headerfont ##2\ ##1}%
                    \hbox{\strut}}}]}%
\def\@opargbegintheorem##1##2##3{%
  \item[\rlap{\vbox{\hbox{\hskip\labelsep \theorem@headerfont
                     ##2\ ##1\ (##3)}%
                    \hbox{\strut}}}]}}
\endgroup
%</thcb>
%    \end{macrocode}
% \end{macro}
%
%
% \subsubsection{The \texttt{change} style}
%
% This style option is stored in the file |thc.sty|.
%    \begin{macrocode}
%<*thc>
\begingroup \makeatletter
\@ifundefined{theorem@style}{\input{theorem.sty}}{}
\ProvidesFile{thc.sty}
             [\FMithmInfo]
%    \end{macrocode}
%
% \begin{macro}{\th@change}
%    The \texttt{change} theorem style corresponds to the
%    \texttt{change break} style without a linebreak after the header.
%    To say it in another way, it's the same as the \texttt{plain}
%    style but with number and name interchanged and |\slshape| as the
%    default font.
%    \begin{macrocode}
\gdef\th@change{\normalfont\slshape
  \def\@begintheorem##1##2{\item
       [\hskip\labelsep \theorem@headerfont ##2\ ##1]}%
\def\@opargbegintheorem##1##2##3{%
  \item[\hskip\labelsep \theorem@headerfont ##2\ ##1\ (##3)]}}
\endgroup
%</thc>
%    \end{macrocode}
% \end{macro}
%
%
% \subsubsection{The \texttt{marginbreak} style}
%
% This style option is the one used most often at Mainz.  It is saved
% in the file |thmb.sty|.
%    \begin{macrocode}
%<*thmb>
\begingroup \makeatletter
\@ifundefined{theorem@style}{\input{theorem.sty}}{}
\ProvidesFile{thmb.sty}
             [\FMithmInfo]
%    \end{macrocode}
%
% \begin{macro}{\th@marginbreak}
%    The \texttt{margin break} style is nearly the same as the
%    \texttt{change break} style. The only difference is the placement
%    of the theorem number.  We use |\llap| to place it in the left
%    margin.
%
%    In this style |\labelsep| denotes the separation between the
%    number and the text.
%    \begin{macrocode}
\gdef\th@marginbreak{\normalfont\slshape
  \def\@begintheorem##1##2{\item
       [\rlap{\vbox{\theorem@headerfont
                    \hbox{\llap{##2}\hskip\labelsep ##1}%
                    \hbox{\strut}}}]}%
\def\@opargbegintheorem##1##2##3{%
  \item[\rlap{\vbox{\theorem@headerfont
                    \hbox{\llap{##2}\hskip\labelsep ##1\ (##3)}%
                    \hbox{\strut}}}]}}
\endgroup
%</thmb>
%    \end{macrocode}
% \changes{v2.1c}{90/02/17}{`endgroup outside macrocode env.}
% \end{macro}
%
%
%
% \subsubsection{The \texttt{margin} style}
%
% This style option is stored in the file |thm.sty|.
%    \begin{macrocode}
%<*thm>
\begingroup \makeatletter
\@ifundefined{theorem@style}{\input{theorem.sty}}{}
\ProvidesFile{thm.sty}
             [\FMithmInfo]
%    \end{macrocode}
%
% \begin{macro}{\th@margin}
%    Again this is only a variant of the theorem styles described
%    above without any new ideas.
%    \begin{macrocode}
\gdef\th@margin{\normalfont\slshape
  \def\@begintheorem##1##2{\item
       [\theorem@headerfont \llap{##2}\hskip\labelsep ##1]}%
\def\@opargbegintheorem##1##2##3{%
  \item[\theorem@headerfont \llap{##2}\hskip\labelsep ##1\ (##3)]}}
\endgroup
%</thm>
%    \end{macrocode}
% \end{macro}
%
%
%
%
% \subsection{Final Definitions}
%
% \begin{macro}{\theorempreskipamount}
% \begin{macro}{\theorempostskipamount}
%    The \textsf{skip} parameters that regulate the vertical empty space
%    before and after the {\envfont theorem} environment have
%    to be allocated as well.
% \changes{v2.0f}{89/07/20}{Unnecessary `global' removed.}
%    \begin{macrocode}
%<*package>
\newskip\theorempreskipamount
\newskip\theorempostskipamount
%    \end{macrocode}
%    Since we have used the same values for all theorem sets, we now can
%    assign them.
%    \begin{macrocode}
\global\setlength\theorempreskipamount{12pt plus 5pt minus 3pt}
\global\setlength\theorempostskipamount{8pt plus 3pt minus 1.5pt}
%    \end{macrocode}
% \end{macro}
% \end{macro}
%
%
% \begin{macro}{\@endtheorem}
%    The same holds for the macro |\@endtheorem|, which ends a
%    {\envfont theorem} environment. Since it is the same for all
%    theorem sets, it is removed from the macros |\th@|\meta{style}.
%    It simply ends the {\envfont trivlist} environment, which was
%    begun in |\@thm|.
%    \begin{macrocode}
\global\let\@endtheorem=\endtrivlist
%    \end{macrocode}
% \end{macro}
%
%
%
% \begin{macro}{\@preamblecmds}
%    All macros defined above are to be used only in the preamble.
%    Therefore, we insert them in |\@preamblecmds| which will disable
%    them at begin document. This is done by the internal \LaTeXe{}
%    command |\@onlypreamble|.
% \changes{v2.0g}{89/09/19}{`theoremheaderfont and `theorembodyfont
%    added.}
% \changes{v2.2a}{94/02/03}{Use \LaTeXe{} interface}
%    \begin{macrocode}
\@onlypreamble\@xnthm
\@onlypreamble\@ynthm
\@onlypreamble\@othm
\@onlypreamble\newtheorem
\@onlypreamble\theoremstyle
\@onlypreamble\theorembodyfont
\@onlypreamble\theoremheaderfont
%    \end{macrocode}
% \end{macro}
%
% Finally we declare the \texttt{plain} theorem style to be the default.
% \changes{v2.1b}{90/06/04}{Fixed missing `item by declaring default
%    style.}
%    \begin{macrocode}
\theoremstyle{plain}
%</package>
%    \end{macrocode}
%
% \Finale

\endinput
}{}
\ProvidesFile{thcb.sty}
             [\FMithmInfo]
%    \end{macrocode}
%
% \begin{macro}{\th@changebreak}
%    The \texttt{change-break} theorem style is like \texttt{break}
%    but with interchange of theorem name and theorem number.  Thus we
%    define |\th@changebreak| as follows:
%    \begin{macrocode}
\gdef\th@changebreak{\normalfont\slshape
  \def\@begintheorem##1##2{\item
       [\rlap{\vbox{\hbox{\hskip\labelsep \theorem@headerfont ##2\ ##1}%
                    \hbox{\strut}}}]}%
\def\@opargbegintheorem##1##2##3{%
  \item[\rlap{\vbox{\hbox{\hskip\labelsep \theorem@headerfont
                     ##2\ ##1\ (##3)}%
                    \hbox{\strut}}}]}}
\endgroup
%</thcb>
%    \end{macrocode}
% \end{macro}
%
%
% \subsubsection{The \texttt{change} style}
%
% This style option is stored in the file |thc.sty|.
%    \begin{macrocode}
%<*thc>
\begingroup \makeatletter
\@ifundefined{theorem@style}{% \iffalse meta-comment
%
% Copyright (C) 1993-2024
%
% The LaTeX Project and any individual authors listed elsewhere
% in this file.
%
% This file is part of the Standard LaTeX `Tools Bundle'.
% -------------------------------------------------------
%
% It may be distributed and/or modified under the
% conditions of the LaTeX Project Public License, either version 1.3c
% of this license or (at your option) any later version.
% The latest version of this license is in
%    https://www.latex-project.org/lppl.txt
% and version 1.3c or later is part of all distributions of LaTeX
% version 2005/12/01 or later.
%
% The list of all files belonging to the LaTeX `Tools Bundle' is
% given in the file `manifest.txt'.
%
% \fi
%
%%
%
%
\def\FMithmInfo{2023/07/05 v2.2c Theorem extension package (FMi)}
%
% \ProvidesFile{theorem.dtx}[\FMithmInfo]
%\iffalse   % this is a METACOMMENT !
%%
%% Package `theorem' to use with LaTeX2e
%% Copyright (C) 1989-1995 Frank Mittelbach, all rights reserved.
%<+package>\@ifundefined{theorem@style}{}{\endinput}
%<+package>\NeedsTeXFormat{LaTeX2e}[1995/06/01]
%<+package>\ProvidesPackage{theorem}
%<+package>         [\FMithmInfo]
%
% \fi
%
% \changes{v1.0}{87/05/07}{First draft.}
% \changes{v1.0a}{87/05/07}{`thistheoremstyle' added.}
% \changes{v1.1a}{87/07/29}{New concept: redefine the internal
%                          \protect\LaTeX{} commands `@xnthm' etc.}
% \changes{v1.1b}{87/07/31}{`theoremfont' used to specify font.}
% \changes{v1.2a}{88/01/08}{`@endtheorem' now globally defined.}
% \changes{v1.2b}{88/01/11}{`newtheorem' only allowed in the preamble.}
% \changes{v1.3a}{88/02/14}{`theoremfont' now a toks register.}
%
% \changes{v2.0a}{88/11/16}{New implementation and new documentation.}
% \changes{v2.0a}{88/11/16}{No layout changes but internal modifications
%                           old substyles won't work.}
% \changes{v2.0b}{89/01/30}{`global' used for all definitions.}
% \changes{v2.0c}{89/06/15}{Documentation translated to English.
%                            Thanks to LLL and Nina}
% \changes{v2.1a}{90/02/15}{Support for new font selection.}
% \changes{v2.2a}{94/02/03}{Cleanup for LaTeX2e}
%
% \DoNotIndex{\@endparenv,\@ifdefinable,\@ifnextchar,\@ifundefined}
% \DoNotIndex{\@namedef,\@nameuse,\@spaces,\@tempa,\@warning}
% \DoNotIndex{\ }
% \DoNotIndex{\begingroup,\bfseries}
% \DoNotIndex{\csname}
% \DoNotIndex{\def,\do,\docdate}
% \DoNotIndex{\endcsname,\endgroup,\endinput,\endtrivlist,\expandafter}
% \DoNotIndex{\filedate,\fileversion}
% \DoNotIndex{\gdef,\global}
% \DoNotIndex{\hbox,\hskip}
% \DoNotIndex{\ignorespaces,\input,\itshape,\item}
% \DoNotIndex{\labelsep,\let,\llap}
% \DoNotIndex{\makeatletter}
% \DoNotIndex{\noexpand}
% \DoNotIndex{\refstepcounter,\relax,\rlap}
% \DoNotIndex{\setlength,\slshape,\space,\string,\strut}
% \DoNotIndex{\the,\trivlist,\typeout}
% \DoNotIndex{\vbox}
% \DoNotIndex{\xdef}
%
% \def\deflabel#1{\ttfamily #1\hfill}
% \def\deflist#1{\begin{list}{}{\settowidth\labelwidth{\ttfamily #1}%
%                               \setlength\leftmargin\labelwidth
%                               \addtolength\leftmargin\labelsep
%                               \let\makelabel\deflabel}}
% \def\enddeflist{\end{list}}
% \def\envfont{\normalfont\ttfamily}
%
% \MakeShortVerb{\|}
%
% \newcommand{\TUB}{{\sl TUGboat\/}}
%
% \GetFileInfo{theorem.dtx}
% \title{An Extension of the
%        \LaTeX{} theorem environment\thanks{This file
%        has version number \fileversion, last
%        revised \filedate.}}
% \author{Frank Mittelbach}
%
% \MaintainedByLaTeXTeam{tools}
% \maketitle
%
% \begin{abstract}
%    The macros described in this paper yield an extension of the
%    \LaTeX{} theorem mechanism. It is designed to satisfy the
%    different requirements of various journals. Thus, the
%    layout of the ``theorems'' can be manipulated by determining a
%    ``style''. This article describes not only the use, but
%    also the definition, of the necessary macros.
% \end{abstract}
%
% \section*{Preface to version 2.2}
%
% For \LaTeXe{} this package did not need any fundamental changes. I
% only modified the messages generated so that theorem layout styles
% will show up with the |\listfiles| command and cleaned the section
% on the New Font Selection Scheme since this is now included in
% \LaTeX.
%
% \section*{Preface to version 2.1}
%
% This version is identical to 2.0g described in \TUB\ 10\#3 except
% for some internal defaults which are now set depending on the used
% font selection scheme.
%
% This was done to avoid unpleasant surprises if the new font selection
% scheme is in force. For further details see
% section~\ref{sec:newfonts} and \cite{bk:GMS94}.
%
% \section{Introduction}
%
% For our purposes here, ``theorems'' are labelled enunciations,
% often set off from the main text by extra space and a font change.
% Theorems, corollaries, conjectures, definitions, and remarks
% are all instances of ``theorems''.  The ``header'' of these structures
% is composed of a label (such as \textsc{Theorem} or \textsc{Remark})
% and a number which serializes an item in the sequence of items
% with the same label.
%
% Shortly after the introduction of \LaTeX{} at the Fachbereich
% Mathematik in Mainz, the desire to manipulate the layout of
% ``theorems'' arose.  In Mainz, the following two conventions came
% into general use:
% \begin{enumerate}
%   \item The number of the theorem is shown in the margin.
%   \item There is a line break at
%         the end of the theorem header.
% \end{enumerate}
%
% Additionally, some journals require different formats which depend on
% the ``sort of theorem'': e.g.\ often remarks and definitions are set
% in |\upshape|, while |\itshape| is employed for main theorems.
%
% Confronted with these requirements, a theorem environment was
% developed in Mainz which allows separate determination of the
% layout of the ``theorems sets'', comparable to |\pagestyle|.
%
%
% \section{The user interface}
% \subsection{Defining new theorem sets}
%
% \DescribeMacro\newtheorem
% As in the original \LaTeX{} version, the command |\newtheorem|
% defines a new ``theorem set'' or ``theorem-like structure''.
% Two required arguments name the new environment and give the text
% to be typeset with each instance of the new ``set'', while
% an optional argument determines how
% the ``set'' is enumerated:
% \begin{description}
%    \item[\ttfamily \bslash newtheorem\{foo\}\{bar\}]
%       The theorem set {\envfont foo} (whose name is \texttt{bar})
%       uses its own counter.
%    \item[\ttfamily \bslash newtheorem\{foo2\}{[foo]}\{bar2\}]
%       The theorem set {\envfont foo2} (printed name \texttt{bar2})
%       uses the same counter as the theorem set \texttt{foo}.
%    \item[\ttfamily \bslash newtheorem\{foo3\}\{bar3\}{[section]}]
%       The theorem set {\envfont foo3} (printed name \texttt{bar3}) is
%       enumerated within the counter \texttt{section}, i.e.\ with every
%       new |\section| the enumeration begins again with 1, and
%       the enumeration is composed from the section-number and the
%       theorem counter itself.
% \end{description}
%
% \DescribeMacro\theoremstyle
% Additionally, the command |\theoremstyle| can define the layout
% of various, or all, theorem sets.  It should be noted that any
% theorem set defined by |\newtheorem| is typeset in the
% |\theoremstyle| that is current at the time of the definition.
% Thus, the following
% \begin{verbatim}
%    \theoremstyle{break}        \newtheorem{Cor}{Corollary}
%    \theoremstyle{plain}        \newtheorem{Exa}{Example}[section]
%\end{verbatim}                             ^^A No blank at line start
% leads to the result that the set {\envfont Cor} is formatted in the
% style \texttt{break}, while the set {\envfont Exa} and all the
% following ones are formatted in the style \texttt{plain}, unless
% another |\theoremstyle| follows.  Since the definitions installed by
% |\newtheorem| are global, one also can limit |\theoremstyle| locally
% by grouping braces.
%
% \DescribeMacro\theorembodyfont
% The choice of the font for the theorem body is completely
% independent of the chosen |\theoremstyle|; this has proven to
% be very advantageous.  For example,
% \begin{verbatim}
%   {\theorembodyfont{\upshape}          \newtheorem{Rem}{Remark}}
%\end{verbatim}
% defines a theorem set {\envfont Rem}, which will be set in
% |\upshape| in the current layout (which in our example is
% \texttt{plain}).  As with |\theoremstyle|, the |\theorembodyfont|
% chosen is that current at the time of |\newtheorem|.  If
% |\theorembodyfont| is not specified or one defines
% |\theorembodyfont{}|, then the font used will be that defined by the
% |\theoremstyle|.
%
% \DescribeMacro\theoremheaderfont It is also possible to customize
% the font used for the theorem headers.  This is, however, a global
% declaration, and therefore there should be at most one
% |\theoremheaderfont| declaration in the preamble.\footnote{If it is
% actually necessary to have different header fonts, one has to define
% new theorem styles (substituting the desired font) or specify the
% information directly in the \texttt{\bslash newtheorem} declaration
% (the unclean variant).}
%
% \DescribeMacro\theorempreskipamount
% \DescribeMacro\theorempostskipamount Two additional parameters
% affect the vertical space around the theorem environments:
% |\theorempreskipamount| and |\theorempostskipamount| define,
% respectively, the spacing before and after such an environment.
% These parameters apply for all theorem sets and can be manipulated
% with the ordinary length macros.  They are rubber lengths
% (`\textsf{skips}'), and therefore can contain \texttt{plus} and
% \texttt{minus} parts.
%
% Since the definition of theorem sets should---most sensibly---be
% placed in the preamble, we only allow installation there.
% It is therefore possible to release the memory used here after
% |\begin{document}|, in order to make room for other
% applications.
%
%
% \subsection{Existing theorem styles}
%
% The following theorem styles exist to date:
% \begin{deflist}{marginbreak}
%    \item[plain]
%       This theorem style emulates the original \LaTeX{} definition,
%       except that additionally the parameters
%       |\theorem...skipamount| are used.
%    \item[break]
%       In this style, the theorem header is followed by a line
%       break.
%    \item[marginbreak]
%       The theorem number is set in the margin, and there is a line
%       break as in \texttt{break}.
%    \item[changebreak]
%       Like \texttt{break}, but with header number and text
%       interchanged.
%    \item[change]
%       Header number and text are interchanged, without a line break.
%    \item[margin]
%       The number is set in the left margin, without a line break.
% \end{deflist}
% All styles (except \texttt{plain}) select |\slshape| as the default
% |\theorembodyfont|.
%
%
% \subsection{Examples}
%
% Given the above theorem sets {\envfont Cor}, {\envfont Exa}
% and {\envfont Rem},
% suppose that the preamble also contains the declarations:
% \begin{verbatim}
%    \theoremstyle{marginbreak}   \newtheorem{Lem}[Cor]{Lemma}
%    \theoremstyle{change}
%    \theorembodyfont{\itshape}        \newtheorem{Def}[Cor]{Definition}
%
%    \theoremheaderfont{\scshape}
%\end{verbatim}
% Then the following are some typical examples of the typeset output
% resulting from their use.
%
% \begin{Cor}
%    This is a sentence typeset in the theorem environment {\envfont
%    Cor}.
% \end{Cor}
%
% \begin{Exa}
%    This is a sentence typeset in the theorem environment {\envfont
%    Exa}.
% \end{Exa}
%
% \begin{Rem}
%    This is a sentence typeset in the theorem environment {\envfont
%    Rem}.
% \end{Rem}
%
% \begin{Lem}[Ben User]
%    This is a sentence typeset in the theorem environment {\envfont
%    Lem}.
% \end{Lem}
%
% \begin{Def}[Very impressive Definition]
%    This is a sentence typeset in the theorem environment {\envfont
%    Def}.
% \end{Def}
%
% The last two examples show the effect of the optional argument to
% a theorem environment (it is the text typeset in parentheses).
%
%
% \section{Special Considerations}
% \label{sec:newfonts}
%
% Theoremheader and body are implemented as a unit. This means that
% the |\theoremheaderfont|  will inherit characteristics of the
% |\theorembodyfont| in \LaTeXe. Thus, if for example
% |\theorembodyfont| is |\itshape| and |\theoremheaderfont|
% is |\bfseries| the font selected for the header will have the
% characteristics `bold extended italic'. If this is not desired one
% should set the |\theoremheaderfont| to something like
% \begin{verbatim}
%      \theoremheaderfont{\normalfont\bfseries}
%\end{verbatim}
% i.e.\ supplying all necessary font information explicitly.
%
%
%
% \section{Acknowledgements}
%
% The publication of this set of macros was only possible with the
% help of Christina Busse (translating the manuscript into English),
% Joachim Pense (playing the r\^ole of typist), Chris Rowley
% (looking everything over) and many others providing useful
% suggestions.
%
% \MaybeStop{%
%    \begin{thebibliography}{9}
%       \bibitem{bk:GMS94} \textsc{M.~Goossens}, \textsc{F.~Mittelbach}
%         and \textsc{A.~Samarin}.
%         \newblock The \LaTeX{} Companion.
%         \newblock
%         Addison-Wesley, Reading, Massachusetts, 1994.
%       \bibitem{latex:source} \textsc{Lamport, Leslie.} \newblock
%                               \texttt{latex.tex}, version 2.09, date
%                               Feb. 1990.
%    \end{thebibliography}
%    \PrintIndex
%    \PrintChanges}
%
%
% \section{The documentation driver file}
%
% The next bit of code contains the documentation driver file for
% \TeX{}, i.e., the file that will produce the documentation you are
% currently reading. It will be extracted from this file by the
% \texttt{docstrip} program. Since it is the first code in the file
% one can alternatively process this file directly with \LaTeXe{} to
% obtain the documentation.
%
%    \begin{macrocode}
%<*driver>
\documentclass{ltxdoc}

\usepackage{theorem}

%<+driver>% The next few lines define theorem sets which are used
%<+driver>% in the example section of the documentation.

\theoremstyle{break}       \newtheorem{Cor}{Corollary}
\theoremstyle{plain}       \newtheorem{Exa}{Example}[section]
{\theorembodyfont{\upshape}\newtheorem{Rem}{Remark}}
\theoremstyle{marginbreak} \newtheorem{Lem}[Cor]{Lemma}
\theoremstyle{change}
\theorembodyfont{\itshape} \newtheorem{Def}[Cor]{Definition}

\theoremheaderfont{\scshape}

\RecordChanges

\begin{document}
   \DocInput{theorem.dtx}
\end{document}
%</driver>
%    \end{macrocode}
%
% \section{Definition of the Macros}
%
% If the file has been loaded before, we abort immediately. If not the
% package announces itself (this is actually done at the very top of
% the file---the way it is done isn't good style so don't copy it).
% \changes{v2.0e}{89/07/19}{Spaces removed from `typeout'.}
% \changes{v2.2b}{95/11/19}{Announce moved to top of file}
%    \begin{macrocode}
%<*package>
%\@ifundefined{theorem@style}{}{\endinput}
%\def\FMithmInfo{1995/11/19 v2.2b Theorem extension package (FMi)}
%\ProvidesPackage{theorem}[\FMithmInfo]
%    \end{macrocode}
%
%
% \subsection{Definition of theorem styles and fonts}
%
% All the definitions in this file are done globally to allow
% inputting this file inside a group.
%
% \begin{macro}{\theoremstyle}
%    Before a theorem style can be installed, the chosen style must be
%    known. For that reason, we must test to see that
%    |\th@|\meta{style} is known or, more precisely, that it is
%    different from |\relax|. If the style is not known then
%    |\th@plain| is used.
%    \begin{macrocode}
\gdef\theoremstyle#1{%
   \@ifundefined{th@#1}{\@warning
          {Unknown theoremstyle `#1'. Using `plain'}%
          \theorem@style{plain}}%
%    \end{macrocode}
%    We save the theorem style to be used in the token register
%    |\theorem@style|.
%    \begin{macrocode}
      {\theorem@style{#1}}%
%    \end{macrocode}
%    Now we ``evaluate'' the theorem style: this means, we call the
%    macro |\th@|\meta{style} which will activate the relevant
%    definitions which are contained in a separate file.  This is done
%    in a group to suppress changes to the current font.  This could
%    otherwise pose problems together with the new font selection
%    scheme\footnote{When I printed the original article using the new
%    font selection scheme I ended with a document with slanted
%    typefaces (text headings and all) simply because one of the
%    theorem styles used \texttt{\bslash sl} at toplevel.} if the
%    |\th@|\meta{style} is evaluated a second time.
% \changes{v2.1a}{90/02/15}{Group added around `th@..}
% \changes{v2.1a}{90/02/15}{Reseting of `..@begintheorem removed.}
%    \begin{macrocode}
      \begingroup
        \csname th@\the\theorem@style \endcsname
      \endgroup}
%    \end{macrocode}
% \end{macro}
%
%
% \begin{macro}{\@begintheorem}
% \changes{v2.1a}{90/02/15}{Relaxed at toplevel.}
% \begin{macro}{\@opargbegintheorem}
% \changes{v2.1a}{90/02/15}{Relaxed at toplevel.}
%    We reset |\@begintheorem| and |\@opargbegintheorem| to
%    |\relax| since these commands are no longer necessary at
%    toplevel. This will save a few tokens.
%    \begin{macrocode}
\global\let\@begintheorem\relax
\global\let\@opargbegintheorem\relax
%    \end{macrocode}
% \end{macro}
% \end{macro}
%
% \begin{macro}{\theorem@style}
%    Obviously the \textsf{token} register used above has to be
%    allocated. To assure the utmost compatibility with the
%    original \LaTeX{} definition, we set the default theorem style to
%    \texttt{plain}, which implements the usual \LaTeX{} convention.
% \changes{v2.0f}{89/07/20}{Unnecessary `global' removed.}
%    \begin{macrocode}
\newtoks\theorem@style
\global\theorem@style{plain}
%    \end{macrocode}
% \end{macro}
%
% \begin{macro}{\theorembodyfont}
% \begin{macro}{\theorem@bodyfont}
%    If the |\theorembodyfont| is set by the user then it should not
%    interact with the default font set in the theorem style. When the
%    new font selection is in force this may happen if, for example,
%    the default is |\itshape| and the new |\theorembodyfont| is
%    |\sffamily|. So we add a |\reset@font| command in front of the
%    user definition.
% \changes{v2.0f}{89/07/20}{Unnecessary `global' removed.}
% \changes{v2.0g}{89/09/19}{`theoremfont to `theorembodyfont.}
% \changes{v2.1c}{92/06/22}{`reset@font added and logic changed.}
% \changes{v2.1d}{92/10/28}{Reset to empty if arg is empty}
%    \begin{macrocode}
\gdef\theorembodyfont#1{%
%    \end{macrocode}
%    We check if the argument supplied is empty and if so put nothing
%    into the |\theorem@bodyfont| token register to allow for
%    |\theorembodyfont{}| as a mean of using the default of the
%    current |\theoremstyle|.
%    \begin{macrocode}
   \def\@tempa{#1}%
   \ifx\@tempa\@empty
    \theorem@bodyfont{}%
   \else
    \theorem@bodyfont{\reset@font#1}%
   \fi
   }
\newtoks\theorem@bodyfont
\global\theorem@bodyfont{}
%    \end{macrocode}
% \end{macro}
% \end{macro}
%
%
% \begin{macro}{\theoremheaderfont}
% \changes{v2.0g}{89/09/19}{Macro added.}
%    The font for the theorem headers is handled differently because
%    this definition applies to all theorem styles.
%    \begin{macrocode}
\gdef\theoremheaderfont#1{\gdef\theorem@headerfont{#1}%
%    \end{macrocode}
%    After using the macro once it is redefined to produce an error
%    message.
%    \begin{macrocode}
       \gdef\theoremheaderfont##1{%
        \typeout{\string\theoremheaderfont\space should be used
                 only once.}}}
%    \end{macrocode}
% \end{macro}
%
% \begin{macro}{\theorem@headerfont}
% \changes{v2.0g}{89/09/19}{Macro added.}
%    To set the |\theorem@headerfont| default we first test if
%    the new fontselection scheme is in force.
% \changes{v2.1a}{90/02/15}{Support for new font selection.}
%    \begin{macrocode}
\ifx\upshape\undefined
%    \end{macrocode}
%    If not we define it to expand into |\bfseries|. We don't use
%    |\let| just in case a following style option redefines this
%    macro.
%    \begin{macrocode}
\gdef\theorem@headerfont{\bfseries}
%    \end{macrocode}
%    Otherwise we reset the current shape before calling |\bfseries|.
%    \begin{macrocode}
\else \gdef\theorem@headerfont{\normalfont\bfseries}\fi
%    \end{macrocode}
% \end{macro}
%
% \begin{macro}{\th@plain}
% \begin{macro}{\th@break}
% \begin{macro}{\th@marginbreak}
% \begin{macro}{\th@changebreak}
% \begin{macro}{\th@change}
% \begin{macro}{\th@margin}
%    The different styles are defined in macros such as
%    |\th@plain|.  Since memory space is precious in
%    ``non-Big-versions'', we have to avoid offering too many unused
%    definitions.  Therefore we define these styles in separate files
%    that can be loaded on demand.  Thus the commands themselves only
%    load these files. We use |\@input@|, a \LaTeXe{} internal command
%    that ensures that the file will be listed with |\listfiles|.
% \changes{v2.2a}{94/02/02}{Use `@input@ to load theorem layout files}
%    \begin{macrocode}
\gdef\th@plain{\@input@{thp.sty}}
\gdef\th@break{\@input@{thb.sty}}
\gdef\th@marginbreak{\@input@{thmb.sty}}
\gdef\th@changebreak{\@input@{thcb.sty}}
\gdef\th@change{\@input@{thc.sty}}
\gdef\th@margin{\@input@{thm.sty}}
%    \end{macrocode}
%    This list will be expanded when new styles become
%    available.  For testing,  just append new theorem substyles
%    as document options.
% \end{macro}
% \end{macro}
% \end{macro}
% \end{macro}
% \end{macro}
% \end{macro}
%
%
%
% \subsection{Definition of a new theorem set}
%
% As already pointed out, a new theorem environment can be defined in
% three different ways:
% \begin{verbatim}
%    \newtheorem{Lem}{Lemma}
%    \newtheorem{Lem}{Lemma}[section]
%    \newtheorem{Lem}[Theorem]{Lemma}
%\end{verbatim}
% The function of the macro |\newtheorem| is to recognize these cases
% and then to branch into one of the three macros |\@ynthm|, |\@xnthm|
% or |\@othm|.  This mechanism is adopted unchanged from
% \cite{latex:source}; the essential point here is that, for example,
% in the second case, the arguments \texttt{Lem}, \texttt{Lemma} and
% \texttt{section} are passed over to the macro |\@xnthm|.
%
% We inspect this case first because the others present fewer problems,
% and thus are easily derived from this one.
%
% \begin{macro}{\@xnthm}
%    For our example arguments, the macro |\@xnthm| must fulfill
%    the following:
%    \begin{itemize}
%       \item
%          Define a new \LaTeX-counter `Lem'
%       \item
%          reset this counter within a |\section|
%       \item
%          define the macro |\theLem|
%       \item
%          define the environment macros |\Lem| and
%          |\endLem| using the current
%          |\theoremstyle| and |\theorem@bodyfont|.
%    \end{itemize}
%    Obviously, all this should happen only if the first argument
%    of |\@xnthm| (i.e.\ \texttt{Lem} in our example) is chosen so
%    as not to conflict with any previously defined commands or
%    environments.  This test is performed by the \LaTeX{} macro
%    |\@ifdefinable|.
%    \begin{macrocode}
\gdef\@xnthm#1#2[#3]{\expandafter\@ifdefinable\csname #1\endcsname
%    \end{macrocode}
%    Therefore, the first argument of |\@ifdefinable| is the
%    expansion (in the example, |\Lem|) of
%    |\csname||#1\endcsname|.  The second argument is
%    executed only if the test has been completed successfully.
%    \begin{macrocode}
   {%
%    \end{macrocode}
%    Now we define the new counter. The names of the \LaTeX{}
%    macros employed should speak for themselves:
% \changes{v2.2b}{1995/10/07}{Use \cs{@newctr}}
% \changes{v2.2c}{1995/11/23}{Use \cs{@newctr} properly.}
%    \begin{macrocode}
    \@definecounter{#1}\@newctr{#1}[#3]%
%    \end{macrocode}
%    Using |\@newctr| will give a proper error message if the counter
%    in |#3| is not defined.
%    In defining `|\theLem|' we must generate the desired macro
%    name by use of |\expandafter| and |\csname|.
%    \begin{macrocode}
    \expandafter\xdef\csname the#1\endcsname
%    \end{macrocode}
%    An |\xdef| is used in order to make the definition global,
%    and to ensure that it contains the replacement texts of
%    |\@thmcountersep| and |\@thmcounter|.\footnote{These
%    two macros can be defined by the document style.  Their default
%    values produce a `\texttt{.}' as separation and an arabic
%    representation of the number.} However, not everything should be
%    expanded. For example, it saves space to use |\thesection|
%    instead of its---at times---lengthy expansion.
%    \begin{macrocode}
      {\expandafter \noexpand \csname the#3\endcsname
       \@thmcountersep \@thmcounter{#1}}%
%    \end{macrocode}
%    Thus with the defaults of \LaTeX{}, |\theLem| would be
%    replaced by the command sequence |\thesection.\arabic{Lem}|.
%
%    We will now look at the definition of the macro which is executed
%    at the beginning of the actual environment (in our example this
%    macro is |\Lem|). It should be noted that we use an
%    ``|\expandafter| trick'' to expand only certain parts of the
%    replacement text at the time of the definition.
%    \begin{macrocode}
    \def\@tempa{\global\@namedef{#1}}%
    \expandafter \@tempa \expandafter{%
%    \end{macrocode}
%    First, the macro that contains the current definitions of
%    |\@begintheorem| and |\@opargtheorem| should be called
%    up.  The name of this macro---as is already known---has the form
%    |\th@|\meta{theorem\ style}; therefore, it must be called by
%    \begin{macrocode}
      \csname th@\the \theorem@style
%    \end{macrocode}
%    In addition the default theorem font should be changeable, i.e.\
%    we have to insert the contents of |\theorem@bodyfont|. For
%    that reason, we expand even further, beyond |\endcsname|,
%    and thus insert the contents of the \textsf{token} register
%    |\theorem@bodyfont| in the replacement text.
%    \begin{macrocode}
            \expandafter \endcsname \the \theorem@bodyfont
%    \end{macrocode}
%    Now it is time to call the macro |\@thm| which takes over
%    the further processing.  It has two arguments: the current
%    counter name (in our example, \texttt{Lem}), and the text of the
%    label (in our example, \texttt{Lemma}).
%    \begin{macrocode}
     \@thm{#1}{#2}}%
%    \end{macrocode}
%    With this, the `sub-definition' is complete.  The macro
%    |\@endtheorem| ends a theorem environment and is, so far, nothing
%    but an |\endtrivlist|.  (Hence it is defined globally, and not
%    within the theorem styles.\footnote{This has to be changed as
%    soon as theorem styles that change \texttt{\bslash @endtheorem}
%    exist.  In such a case, all existing styles must be changed as
%    well since they will have to reset the macro.}) Therefore, we can
%    set it equivalent to the macro that ends the theorem set (in our
%    example, |\endLem|).  However, if some day theorem styles exist
%    that do change |\@endtheorem|, we would have to use the
%    commented-out line instead.
%    \begin{macrocode}
    \global \expandafter \let \csname end#1\endcsname \@endtheorem
%   \global\@namedef{end#1}{\@endtheorem}%
%    \end{macrocode}
%    With these commands all the required definitions are employed,
%    unless the test |\@ifdefinable| has failed.  Therefore, we
%    end the second argument of this macro and with it the definition
%    of |\@xnthm|.
%    \begin{macrocode}
   }}
%    \end{macrocode}
% \end{macro}
%
% \begin{macro}{\@ynthm}
%    The definition of |\@ynthm| is completely analogous. In
%    this case the new counter that is defined is not reset within
%    another counter; thus the definition of |\the...|
%    is simplified:
%    \begin{macrocode}
\gdef\@ynthm#1#2{\expandafter\@ifdefinable\csname #1\endcsname
   {\@definecounter{#1}%
    \expandafter\xdef\csname the#1\endcsname{\@thmcounter{#1}}%
%    \end{macrocode}
%    The rest of the definition corresponds literally to that of
%    |\@xnthm|:
%    \begin{macrocode}
    \def\@tempa{\global\@namedef{#1}}\expandafter \@tempa
     \expandafter{\csname th@\the \theorem@style \expandafter
     \endcsname \the\theorem@bodyfont \@thm{#1}{#2}}%
    \global \expandafter \let \csname end#1\endcsname \@endtheorem}}
%    \end{macrocode}
% \end{macro}
%
% \begin{macro}{\@othm}
%    The definition of |\@othm| does not contain anything new.
%    \begin{macrocode}
\gdef\@othm#1[#2]#3{%
%    \end{macrocode}
% \changes{v2.2b}{95/10/07}{Build in check for counter defined}
%    We do not define a new counter but instead use one that has
%    already been defined.  Thus the only definition we need is that
%    of this pseudo-counter (i.e.\ |\the|\meta{env.\ name}).
%    First we check if |#2| corresponds to a known counter name.
%    \begin{macrocode}
  \expandafter\ifx\csname c@#2\endcsname\relax
   \@nocounterr{#2}%
  \else
   \expandafter\@ifdefinable\csname #1\endcsname
   {\expandafter \xdef \csname the#1\endcsname
     {\expandafter \noexpand \csname the#2\endcsname}%
%    \end{macrocode}
%    All other parts of the definition can be adopted from
%    |\@xnthm|. We have to remember, though, that in this case
%    the name of the current counter and the theorem label have moved
%    to the second and third arguments.
%    \begin{macrocode}
    \def\@tempa{\global\@namedef{#1}}\expandafter \@tempa
     \expandafter{\csname th@\the \theorem@style \expandafter
     \endcsname \the\theorem@bodyfont \@thm{#2}{#3}}%
    \global \expandafter \let \csname end#1\endcsname \@endtheorem}%
  \fi}
%    \end{macrocode}
% \end{macro}
%
% \subsection{Macros that are employed in a {\envfont theorem}
%    environment}
%
% \begin{macro}{\@thm}
%    The macro |\@thm| has to increase the current counter.  Then,
%    depending on whether the environment has (or does not have) an
%    optional argument, it has to branch into either |\@begintheorem|
%    or |\@opargtheorem|.
%    \begin{macrocode}
\gdef\@thm#1#2{\refstepcounter{#1}%
%    \end{macrocode}
%    Now we start a {\envfont trivlist} environment, and give
%    |\@topsep| and |\@topsepadd| the values of the \textsf{skip}
%    registers |\theorempreskipamount| and |\theorempostskipamount|.
%    The value in |\@topsep| is the vertical space that is inserted by
%    the first (and only) |\item| in our |\trivlist| whilst
%    |\@topsepadd| is inserted by |\@endparenv| at the end of that
%    {\envfont trivlist} environment.  By using these registers, we
%    obtain the desired space around a {\envfont theorem} environment.
%    \begin{macrocode}
   \trivlist
   \@topsep \theorempreskipamount               % used by first \item
   \@topsepadd \theorempostskipamount           % used by \@endparenv
%    \end{macrocode}
%    Now we have to test whether an optional argument has been given.
% \changes{v2.0d}{89/07/18}{Forgotten percent added.}
%    \begin{macrocode}
   \@ifnextchar [%
%    \end{macrocode}
%    If there is an optional argument, we will call |\@ythm|, and
%    move the arguments read back into the input stream.
%    \begin{macrocode}
   {\@ythm{#1}{#2}}%
%    \end{macrocode}
%    If not, we call |\@begintheorem|. Its first argument is the
%    name of the theorem set (hence the second argument of
%    |\@thm|). Its second argument is the macro that produces the
%    current number.
%    \begin{macrocode}
   {\@begintheorem{#2}{\csname the#1\endcsname}\ignorespaces}}
%    \end{macrocode}
% \end{macro}
%
% \begin{macro}{\@xthm}
% \begin{macro}{\@ythm}
%    Both these macros were originally called by |\@thm|.  We do
%    not need |\@xthm| anymore, hence we reset it to
%    |\relax|. The definition of |\@ythm| has not changed at
%    all from its definition in \LaTeX. In order to make the macros
%    easier to understand, we will
%    nevertheless present it (commented out).
%    \begin{macrocode}
\global\let\@xthm\relax
% \def\@ythm#1#2[#3]{\@opargbegintheorem{#2}{\csname
%           the#1\endcsname}{#3}\ignorespaces}
%</package>
%    \end{macrocode}
%    The primitive |\ignorespaces| in |\@ythm| and
%    |\@thm| is needed to remove the spaces between the
%    |\begin{...}| and the actual text.
% \end{macro}
% \end{macro}
%
%
% \subsection{Definition of the theorem substyles}
%
% As already pointed out, the theorem substyles, defined below, are
% only loaded when necessary.  Note that all these substyles, except
% \texttt{plain}, have |\slshape| as the default body font.
%
% \subsubsection{The \texttt{plain} style}
%
% \label{dok:th@plain} As the following macros use |@|, we have
% to locally set the |\catcode| of this symbol to ``letter''.
% This happens within a group, so that we do not have to worry about
% which |\catcode| that symbol had before.
%    \begin{macrocode}
%<*thp>
\begingroup \makeatletter
%    \end{macrocode}
% Since we are now within a group, we must make all definitions
% globally.  First we make sure that \texttt{theorem.sty} is loaded.
% This will allow us to use this file as a document style option
% without having to call \texttt{theorem} itself as an option.  At the
% same time, we assure that at least version 2 is loaded, since
% |\theorem@style| was not defined in earlier versions.
%    \begin{macrocode}
\@ifundefined{theorem@style}{\input{theorem.sty}}{}
\ProvidesFile{thp.sty}
             [\FMithmInfo]
%    \end{macrocode}
%
%
% \begin{macro}{\th@plain}
% \begin{macro}{\@begintheorem}
% \begin{macro}{\@opargbegintheorem}
%    |\theoremstyle{plain}| corresponds to the original
%    definition, except that the distances to the surrounding text are
%    determined by the parameters |\theorempreskipamount| and
%    |\theorempostskipamount|.
%    First we set the default body font.
%    \begin{macrocode}
\gdef\th@plain{\normalfont\itshape
%    \end{macrocode}
%    Then we define |\@begintheorem| and
%    |\@opargbegintheorem|. These two macros define how the
%    header of a theorem is typeset.  |\@opargbegintheorem| will
%    be called if a \textsf{theorem} environment with an optional
%    argument is encountered; otherwise, the header is constructed by
%    calling |\@begintheorem|.  If one of these macros is
%    executed, we are within a \textsf{trivlist} environment started by
%    |\@thm|.  So the theorem header is produced with an
%    |\item| command.
%
%    Instead of specifying the header font directly, all standard
%    theorem styles use the |\theorem@headerfont| macro to allow
%    customization.  The extra space (|\labelsep|) is necessary
%    because of problems in the \textsf{trivlist} environment.
%    \begin{macrocode}
  \def\@begintheorem##1##2{%
        \item[\hskip\labelsep \theorem@headerfont ##1\ ##2]}%
%    \end{macrocode}
%    The definition of |\@opargbegintheorem| is completely
%    analogous. The only difference is the fact that there exists a
%    third argument (which is the optional parameter of the
%    environment and contains additional information about the
%    theorem).  Customarily we enclose it in parentheses.
%    \begin{macrocode}
\def\@opargbegintheorem##1##2##3{%
   \item[\hskip\labelsep \theorem@headerfont ##1\ ##2\ (##3)]}}
%    \end{macrocode}
% \end{macro}
% \end{macro}
% \end{macro}
%
% \noindent
% We conclude with an |\endgroup| to restore the
% |\catcode| of |@|.
%    \begin{macrocode}
\endgroup
%</thp>
%    \end{macrocode}
%
%
% \subsubsection{The \texttt{break} style}
%
% This style option is stored in the file |thb.sty|.
% For the next two lines see the documentation for |\th@plain| on
% page~\pageref{dok:th@plain}.
%    \begin{macrocode}
%<*thb>
\begingroup \makeatletter
\@ifundefined{theorem@style}{\input{theorem.sty}}{}
\ProvidesFile{thb.sty}
             [\FMithmInfo]
%    \end{macrocode}
%
% \begin{macro}{\th@break}
%    |\theoremstyle{break}| produces a line break after the name
%    of the theorem.  The font is |\slshape|.  Hence, we define
%    |\th@break| as follows:
%    \begin{macrocode}
\gdef\th@break{\normalfont\slshape
  \def\@begintheorem##1##2{\item[%
%    \end{macrocode}
%    We run into the following problem: it is not possible to create
%    the header with |\item[|\meta{title}|]| and then start
%    a new line by, for example, |\mbox{}\\|. Such a definition
%    will fail whenever a list environment follows immediately.  With
%    the above construction, the |\mbox{}| causes the switch
%    |@inlabel| (cf.\ definition of |\list| and
%    |\trivlist| in \cite{latex:source}) to be set to \texttt{false}
%    and so the following list will insert additional vertical
%    space (|\topskip|). This is quite annoying. Therefore, we
%    create the line break within the |\item|. In order to ensure
%    that the text will begin at the proper position in the following
%    line, we simply pretend that the label does not take any
%    room.\footnote{This will lead to problems whenever very high
%    symbols occurring in the line tower into the
%    heading. So, something else has to be done here sometime.}
%    \begin{macrocode}
       \rlap{\vbox{\hbox{\hskip \labelsep\theorem@headerfont ##1\ ##2}%
                   \hbox{\strut}}}]}%
%    \end{macrocode}
%    Again, the definition of |\@opargbegintheorem| is
%    completely analogous.
%    \begin{macrocode}
\def\@opargbegintheorem##1##2##3{%
  \item[\rlap{\vbox{\hbox{\hskip \labelsep \theorem@headerfont
                     ##1\ ##2\ (##3)}%
                    \hbox{\strut}}}]}}
\endgroup
%</thb>
%    \end{macrocode}
% \end{macro}
%
%
% \subsubsection{The \texttt{changebreak} style}
%
%    \begin{macrocode}
% This style option is stored in the file |thcb.sty|.
%    \begin{macrocode}
%<*thcb>
\begingroup \makeatletter
\@ifundefined{theorem@style}{\input{theorem.sty}}{}
\ProvidesFile{thcb.sty}
             [\FMithmInfo]
%    \end{macrocode}
%
% \begin{macro}{\th@changebreak}
%    The \texttt{change-break} theorem style is like \texttt{break}
%    but with interchange of theorem name and theorem number.  Thus we
%    define |\th@changebreak| as follows:
%    \begin{macrocode}
\gdef\th@changebreak{\normalfont\slshape
  \def\@begintheorem##1##2{\item
       [\rlap{\vbox{\hbox{\hskip\labelsep \theorem@headerfont ##2\ ##1}%
                    \hbox{\strut}}}]}%
\def\@opargbegintheorem##1##2##3{%
  \item[\rlap{\vbox{\hbox{\hskip\labelsep \theorem@headerfont
                     ##2\ ##1\ (##3)}%
                    \hbox{\strut}}}]}}
\endgroup
%</thcb>
%    \end{macrocode}
% \end{macro}
%
%
% \subsubsection{The \texttt{change} style}
%
% This style option is stored in the file |thc.sty|.
%    \begin{macrocode}
%<*thc>
\begingroup \makeatletter
\@ifundefined{theorem@style}{\input{theorem.sty}}{}
\ProvidesFile{thc.sty}
             [\FMithmInfo]
%    \end{macrocode}
%
% \begin{macro}{\th@change}
%    The \texttt{change} theorem style corresponds to the
%    \texttt{change break} style without a linebreak after the header.
%    To say it in another way, it's the same as the \texttt{plain}
%    style but with number and name interchanged and |\slshape| as the
%    default font.
%    \begin{macrocode}
\gdef\th@change{\normalfont\slshape
  \def\@begintheorem##1##2{\item
       [\hskip\labelsep \theorem@headerfont ##2\ ##1]}%
\def\@opargbegintheorem##1##2##3{%
  \item[\hskip\labelsep \theorem@headerfont ##2\ ##1\ (##3)]}}
\endgroup
%</thc>
%    \end{macrocode}
% \end{macro}
%
%
% \subsubsection{The \texttt{marginbreak} style}
%
% This style option is the one used most often at Mainz.  It is saved
% in the file |thmb.sty|.
%    \begin{macrocode}
%<*thmb>
\begingroup \makeatletter
\@ifundefined{theorem@style}{\input{theorem.sty}}{}
\ProvidesFile{thmb.sty}
             [\FMithmInfo]
%    \end{macrocode}
%
% \begin{macro}{\th@marginbreak}
%    The \texttt{margin break} style is nearly the same as the
%    \texttt{change break} style. The only difference is the placement
%    of the theorem number.  We use |\llap| to place it in the left
%    margin.
%
%    In this style |\labelsep| denotes the separation between the
%    number and the text.
%    \begin{macrocode}
\gdef\th@marginbreak{\normalfont\slshape
  \def\@begintheorem##1##2{\item
       [\rlap{\vbox{\theorem@headerfont
                    \hbox{\llap{##2}\hskip\labelsep ##1}%
                    \hbox{\strut}}}]}%
\def\@opargbegintheorem##1##2##3{%
  \item[\rlap{\vbox{\theorem@headerfont
                    \hbox{\llap{##2}\hskip\labelsep ##1\ (##3)}%
                    \hbox{\strut}}}]}}
\endgroup
%</thmb>
%    \end{macrocode}
% \changes{v2.1c}{90/02/17}{`endgroup outside macrocode env.}
% \end{macro}
%
%
%
% \subsubsection{The \texttt{margin} style}
%
% This style option is stored in the file |thm.sty|.
%    \begin{macrocode}
%<*thm>
\begingroup \makeatletter
\@ifundefined{theorem@style}{\input{theorem.sty}}{}
\ProvidesFile{thm.sty}
             [\FMithmInfo]
%    \end{macrocode}
%
% \begin{macro}{\th@margin}
%    Again this is only a variant of the theorem styles described
%    above without any new ideas.
%    \begin{macrocode}
\gdef\th@margin{\normalfont\slshape
  \def\@begintheorem##1##2{\item
       [\theorem@headerfont \llap{##2}\hskip\labelsep ##1]}%
\def\@opargbegintheorem##1##2##3{%
  \item[\theorem@headerfont \llap{##2}\hskip\labelsep ##1\ (##3)]}}
\endgroup
%</thm>
%    \end{macrocode}
% \end{macro}
%
%
%
%
% \subsection{Final Definitions}
%
% \begin{macro}{\theorempreskipamount}
% \begin{macro}{\theorempostskipamount}
%    The \textsf{skip} parameters that regulate the vertical empty space
%    before and after the {\envfont theorem} environment have
%    to be allocated as well.
% \changes{v2.0f}{89/07/20}{Unnecessary `global' removed.}
%    \begin{macrocode}
%<*package>
\newskip\theorempreskipamount
\newskip\theorempostskipamount
%    \end{macrocode}
%    Since we have used the same values for all theorem sets, we now can
%    assign them.
%    \begin{macrocode}
\global\setlength\theorempreskipamount{12pt plus 5pt minus 3pt}
\global\setlength\theorempostskipamount{8pt plus 3pt minus 1.5pt}
%    \end{macrocode}
% \end{macro}
% \end{macro}
%
%
% \begin{macro}{\@endtheorem}
%    The same holds for the macro |\@endtheorem|, which ends a
%    {\envfont theorem} environment. Since it is the same for all
%    theorem sets, it is removed from the macros |\th@|\meta{style}.
%    It simply ends the {\envfont trivlist} environment, which was
%    begun in |\@thm|.
%    \begin{macrocode}
\global\let\@endtheorem=\endtrivlist
%    \end{macrocode}
% \end{macro}
%
%
%
% \begin{macro}{\@preamblecmds}
%    All macros defined above are to be used only in the preamble.
%    Therefore, we insert them in |\@preamblecmds| which will disable
%    them at begin document. This is done by the internal \LaTeXe{}
%    command |\@onlypreamble|.
% \changes{v2.0g}{89/09/19}{`theoremheaderfont and `theorembodyfont
%    added.}
% \changes{v2.2a}{94/02/03}{Use \LaTeXe{} interface}
%    \begin{macrocode}
\@onlypreamble\@xnthm
\@onlypreamble\@ynthm
\@onlypreamble\@othm
\@onlypreamble\newtheorem
\@onlypreamble\theoremstyle
\@onlypreamble\theorembodyfont
\@onlypreamble\theoremheaderfont
%    \end{macrocode}
% \end{macro}
%
% Finally we declare the \texttt{plain} theorem style to be the default.
% \changes{v2.1b}{90/06/04}{Fixed missing `item by declaring default
%    style.}
%    \begin{macrocode}
\theoremstyle{plain}
%</package>
%    \end{macrocode}
%
% \Finale

\endinput
}{}
\ProvidesFile{thc.sty}
             [\FMithmInfo]
%    \end{macrocode}
%
% \begin{macro}{\th@change}
%    The \texttt{change} theorem style corresponds to the
%    \texttt{change break} style without a linebreak after the header.
%    To say it in another way, it's the same as the \texttt{plain}
%    style but with number and name interchanged and |\slshape| as the
%    default font.
%    \begin{macrocode}
\gdef\th@change{\normalfont\slshape
  \def\@begintheorem##1##2{\item
       [\hskip\labelsep \theorem@headerfont ##2\ ##1]}%
\def\@opargbegintheorem##1##2##3{%
  \item[\hskip\labelsep \theorem@headerfont ##2\ ##1\ (##3)]}}
\endgroup
%</thc>
%    \end{macrocode}
% \end{macro}
%
%
% \subsubsection{The \texttt{marginbreak} style}
%
% This style option is the one used most often at Mainz.  It is saved
% in the file |thmb.sty|.
%    \begin{macrocode}
%<*thmb>
\begingroup \makeatletter
\@ifundefined{theorem@style}{% \iffalse meta-comment
%
% Copyright (C) 1993-2024
%
% The LaTeX Project and any individual authors listed elsewhere
% in this file.
%
% This file is part of the Standard LaTeX `Tools Bundle'.
% -------------------------------------------------------
%
% It may be distributed and/or modified under the
% conditions of the LaTeX Project Public License, either version 1.3c
% of this license or (at your option) any later version.
% The latest version of this license is in
%    https://www.latex-project.org/lppl.txt
% and version 1.3c or later is part of all distributions of LaTeX
% version 2005/12/01 or later.
%
% The list of all files belonging to the LaTeX `Tools Bundle' is
% given in the file `manifest.txt'.
%
% \fi
%
%%
%
%
\def\FMithmInfo{2023/07/05 v2.2c Theorem extension package (FMi)}
%
% \ProvidesFile{theorem.dtx}[\FMithmInfo]
%\iffalse   % this is a METACOMMENT !
%%
%% Package `theorem' to use with LaTeX2e
%% Copyright (C) 1989-1995 Frank Mittelbach, all rights reserved.
%<+package>\@ifundefined{theorem@style}{}{\endinput}
%<+package>\NeedsTeXFormat{LaTeX2e}[1995/06/01]
%<+package>\ProvidesPackage{theorem}
%<+package>         [\FMithmInfo]
%
% \fi
%
% \changes{v1.0}{87/05/07}{First draft.}
% \changes{v1.0a}{87/05/07}{`thistheoremstyle' added.}
% \changes{v1.1a}{87/07/29}{New concept: redefine the internal
%                          \protect\LaTeX{} commands `@xnthm' etc.}
% \changes{v1.1b}{87/07/31}{`theoremfont' used to specify font.}
% \changes{v1.2a}{88/01/08}{`@endtheorem' now globally defined.}
% \changes{v1.2b}{88/01/11}{`newtheorem' only allowed in the preamble.}
% \changes{v1.3a}{88/02/14}{`theoremfont' now a toks register.}
%
% \changes{v2.0a}{88/11/16}{New implementation and new documentation.}
% \changes{v2.0a}{88/11/16}{No layout changes but internal modifications
%                           old substyles won't work.}
% \changes{v2.0b}{89/01/30}{`global' used for all definitions.}
% \changes{v2.0c}{89/06/15}{Documentation translated to English.
%                            Thanks to LLL and Nina}
% \changes{v2.1a}{90/02/15}{Support for new font selection.}
% \changes{v2.2a}{94/02/03}{Cleanup for LaTeX2e}
%
% \DoNotIndex{\@endparenv,\@ifdefinable,\@ifnextchar,\@ifundefined}
% \DoNotIndex{\@namedef,\@nameuse,\@spaces,\@tempa,\@warning}
% \DoNotIndex{\ }
% \DoNotIndex{\begingroup,\bfseries}
% \DoNotIndex{\csname}
% \DoNotIndex{\def,\do,\docdate}
% \DoNotIndex{\endcsname,\endgroup,\endinput,\endtrivlist,\expandafter}
% \DoNotIndex{\filedate,\fileversion}
% \DoNotIndex{\gdef,\global}
% \DoNotIndex{\hbox,\hskip}
% \DoNotIndex{\ignorespaces,\input,\itshape,\item}
% \DoNotIndex{\labelsep,\let,\llap}
% \DoNotIndex{\makeatletter}
% \DoNotIndex{\noexpand}
% \DoNotIndex{\refstepcounter,\relax,\rlap}
% \DoNotIndex{\setlength,\slshape,\space,\string,\strut}
% \DoNotIndex{\the,\trivlist,\typeout}
% \DoNotIndex{\vbox}
% \DoNotIndex{\xdef}
%
% \def\deflabel#1{\ttfamily #1\hfill}
% \def\deflist#1{\begin{list}{}{\settowidth\labelwidth{\ttfamily #1}%
%                               \setlength\leftmargin\labelwidth
%                               \addtolength\leftmargin\labelsep
%                               \let\makelabel\deflabel}}
% \def\enddeflist{\end{list}}
% \def\envfont{\normalfont\ttfamily}
%
% \MakeShortVerb{\|}
%
% \newcommand{\TUB}{{\sl TUGboat\/}}
%
% \GetFileInfo{theorem.dtx}
% \title{An Extension of the
%        \LaTeX{} theorem environment\thanks{This file
%        has version number \fileversion, last
%        revised \filedate.}}
% \author{Frank Mittelbach}
%
% \MaintainedByLaTeXTeam{tools}
% \maketitle
%
% \begin{abstract}
%    The macros described in this paper yield an extension of the
%    \LaTeX{} theorem mechanism. It is designed to satisfy the
%    different requirements of various journals. Thus, the
%    layout of the ``theorems'' can be manipulated by determining a
%    ``style''. This article describes not only the use, but
%    also the definition, of the necessary macros.
% \end{abstract}
%
% \section*{Preface to version 2.2}
%
% For \LaTeXe{} this package did not need any fundamental changes. I
% only modified the messages generated so that theorem layout styles
% will show up with the |\listfiles| command and cleaned the section
% on the New Font Selection Scheme since this is now included in
% \LaTeX.
%
% \section*{Preface to version 2.1}
%
% This version is identical to 2.0g described in \TUB\ 10\#3 except
% for some internal defaults which are now set depending on the used
% font selection scheme.
%
% This was done to avoid unpleasant surprises if the new font selection
% scheme is in force. For further details see
% section~\ref{sec:newfonts} and \cite{bk:GMS94}.
%
% \section{Introduction}
%
% For our purposes here, ``theorems'' are labelled enunciations,
% often set off from the main text by extra space and a font change.
% Theorems, corollaries, conjectures, definitions, and remarks
% are all instances of ``theorems''.  The ``header'' of these structures
% is composed of a label (such as \textsc{Theorem} or \textsc{Remark})
% and a number which serializes an item in the sequence of items
% with the same label.
%
% Shortly after the introduction of \LaTeX{} at the Fachbereich
% Mathematik in Mainz, the desire to manipulate the layout of
% ``theorems'' arose.  In Mainz, the following two conventions came
% into general use:
% \begin{enumerate}
%   \item The number of the theorem is shown in the margin.
%   \item There is a line break at
%         the end of the theorem header.
% \end{enumerate}
%
% Additionally, some journals require different formats which depend on
% the ``sort of theorem'': e.g.\ often remarks and definitions are set
% in |\upshape|, while |\itshape| is employed for main theorems.
%
% Confronted with these requirements, a theorem environment was
% developed in Mainz which allows separate determination of the
% layout of the ``theorems sets'', comparable to |\pagestyle|.
%
%
% \section{The user interface}
% \subsection{Defining new theorem sets}
%
% \DescribeMacro\newtheorem
% As in the original \LaTeX{} version, the command |\newtheorem|
% defines a new ``theorem set'' or ``theorem-like structure''.
% Two required arguments name the new environment and give the text
% to be typeset with each instance of the new ``set'', while
% an optional argument determines how
% the ``set'' is enumerated:
% \begin{description}
%    \item[\ttfamily \bslash newtheorem\{foo\}\{bar\}]
%       The theorem set {\envfont foo} (whose name is \texttt{bar})
%       uses its own counter.
%    \item[\ttfamily \bslash newtheorem\{foo2\}{[foo]}\{bar2\}]
%       The theorem set {\envfont foo2} (printed name \texttt{bar2})
%       uses the same counter as the theorem set \texttt{foo}.
%    \item[\ttfamily \bslash newtheorem\{foo3\}\{bar3\}{[section]}]
%       The theorem set {\envfont foo3} (printed name \texttt{bar3}) is
%       enumerated within the counter \texttt{section}, i.e.\ with every
%       new |\section| the enumeration begins again with 1, and
%       the enumeration is composed from the section-number and the
%       theorem counter itself.
% \end{description}
%
% \DescribeMacro\theoremstyle
% Additionally, the command |\theoremstyle| can define the layout
% of various, or all, theorem sets.  It should be noted that any
% theorem set defined by |\newtheorem| is typeset in the
% |\theoremstyle| that is current at the time of the definition.
% Thus, the following
% \begin{verbatim}
%    \theoremstyle{break}        \newtheorem{Cor}{Corollary}
%    \theoremstyle{plain}        \newtheorem{Exa}{Example}[section]
%\end{verbatim}                             ^^A No blank at line start
% leads to the result that the set {\envfont Cor} is formatted in the
% style \texttt{break}, while the set {\envfont Exa} and all the
% following ones are formatted in the style \texttt{plain}, unless
% another |\theoremstyle| follows.  Since the definitions installed by
% |\newtheorem| are global, one also can limit |\theoremstyle| locally
% by grouping braces.
%
% \DescribeMacro\theorembodyfont
% The choice of the font for the theorem body is completely
% independent of the chosen |\theoremstyle|; this has proven to
% be very advantageous.  For example,
% \begin{verbatim}
%   {\theorembodyfont{\upshape}          \newtheorem{Rem}{Remark}}
%\end{verbatim}
% defines a theorem set {\envfont Rem}, which will be set in
% |\upshape| in the current layout (which in our example is
% \texttt{plain}).  As with |\theoremstyle|, the |\theorembodyfont|
% chosen is that current at the time of |\newtheorem|.  If
% |\theorembodyfont| is not specified or one defines
% |\theorembodyfont{}|, then the font used will be that defined by the
% |\theoremstyle|.
%
% \DescribeMacro\theoremheaderfont It is also possible to customize
% the font used for the theorem headers.  This is, however, a global
% declaration, and therefore there should be at most one
% |\theoremheaderfont| declaration in the preamble.\footnote{If it is
% actually necessary to have different header fonts, one has to define
% new theorem styles (substituting the desired font) or specify the
% information directly in the \texttt{\bslash newtheorem} declaration
% (the unclean variant).}
%
% \DescribeMacro\theorempreskipamount
% \DescribeMacro\theorempostskipamount Two additional parameters
% affect the vertical space around the theorem environments:
% |\theorempreskipamount| and |\theorempostskipamount| define,
% respectively, the spacing before and after such an environment.
% These parameters apply for all theorem sets and can be manipulated
% with the ordinary length macros.  They are rubber lengths
% (`\textsf{skips}'), and therefore can contain \texttt{plus} and
% \texttt{minus} parts.
%
% Since the definition of theorem sets should---most sensibly---be
% placed in the preamble, we only allow installation there.
% It is therefore possible to release the memory used here after
% |\begin{document}|, in order to make room for other
% applications.
%
%
% \subsection{Existing theorem styles}
%
% The following theorem styles exist to date:
% \begin{deflist}{marginbreak}
%    \item[plain]
%       This theorem style emulates the original \LaTeX{} definition,
%       except that additionally the parameters
%       |\theorem...skipamount| are used.
%    \item[break]
%       In this style, the theorem header is followed by a line
%       break.
%    \item[marginbreak]
%       The theorem number is set in the margin, and there is a line
%       break as in \texttt{break}.
%    \item[changebreak]
%       Like \texttt{break}, but with header number and text
%       interchanged.
%    \item[change]
%       Header number and text are interchanged, without a line break.
%    \item[margin]
%       The number is set in the left margin, without a line break.
% \end{deflist}
% All styles (except \texttt{plain}) select |\slshape| as the default
% |\theorembodyfont|.
%
%
% \subsection{Examples}
%
% Given the above theorem sets {\envfont Cor}, {\envfont Exa}
% and {\envfont Rem},
% suppose that the preamble also contains the declarations:
% \begin{verbatim}
%    \theoremstyle{marginbreak}   \newtheorem{Lem}[Cor]{Lemma}
%    \theoremstyle{change}
%    \theorembodyfont{\itshape}        \newtheorem{Def}[Cor]{Definition}
%
%    \theoremheaderfont{\scshape}
%\end{verbatim}
% Then the following are some typical examples of the typeset output
% resulting from their use.
%
% \begin{Cor}
%    This is a sentence typeset in the theorem environment {\envfont
%    Cor}.
% \end{Cor}
%
% \begin{Exa}
%    This is a sentence typeset in the theorem environment {\envfont
%    Exa}.
% \end{Exa}
%
% \begin{Rem}
%    This is a sentence typeset in the theorem environment {\envfont
%    Rem}.
% \end{Rem}
%
% \begin{Lem}[Ben User]
%    This is a sentence typeset in the theorem environment {\envfont
%    Lem}.
% \end{Lem}
%
% \begin{Def}[Very impressive Definition]
%    This is a sentence typeset in the theorem environment {\envfont
%    Def}.
% \end{Def}
%
% The last two examples show the effect of the optional argument to
% a theorem environment (it is the text typeset in parentheses).
%
%
% \section{Special Considerations}
% \label{sec:newfonts}
%
% Theoremheader and body are implemented as a unit. This means that
% the |\theoremheaderfont|  will inherit characteristics of the
% |\theorembodyfont| in \LaTeXe. Thus, if for example
% |\theorembodyfont| is |\itshape| and |\theoremheaderfont|
% is |\bfseries| the font selected for the header will have the
% characteristics `bold extended italic'. If this is not desired one
% should set the |\theoremheaderfont| to something like
% \begin{verbatim}
%      \theoremheaderfont{\normalfont\bfseries}
%\end{verbatim}
% i.e.\ supplying all necessary font information explicitly.
%
%
%
% \section{Acknowledgements}
%
% The publication of this set of macros was only possible with the
% help of Christina Busse (translating the manuscript into English),
% Joachim Pense (playing the r\^ole of typist), Chris Rowley
% (looking everything over) and many others providing useful
% suggestions.
%
% \MaybeStop{%
%    \begin{thebibliography}{9}
%       \bibitem{bk:GMS94} \textsc{M.~Goossens}, \textsc{F.~Mittelbach}
%         and \textsc{A.~Samarin}.
%         \newblock The \LaTeX{} Companion.
%         \newblock
%         Addison-Wesley, Reading, Massachusetts, 1994.
%       \bibitem{latex:source} \textsc{Lamport, Leslie.} \newblock
%                               \texttt{latex.tex}, version 2.09, date
%                               Feb. 1990.
%    \end{thebibliography}
%    \PrintIndex
%    \PrintChanges}
%
%
% \section{The documentation driver file}
%
% The next bit of code contains the documentation driver file for
% \TeX{}, i.e., the file that will produce the documentation you are
% currently reading. It will be extracted from this file by the
% \texttt{docstrip} program. Since it is the first code in the file
% one can alternatively process this file directly with \LaTeXe{} to
% obtain the documentation.
%
%    \begin{macrocode}
%<*driver>
\documentclass{ltxdoc}

\usepackage{theorem}

%<+driver>% The next few lines define theorem sets which are used
%<+driver>% in the example section of the documentation.

\theoremstyle{break}       \newtheorem{Cor}{Corollary}
\theoremstyle{plain}       \newtheorem{Exa}{Example}[section]
{\theorembodyfont{\upshape}\newtheorem{Rem}{Remark}}
\theoremstyle{marginbreak} \newtheorem{Lem}[Cor]{Lemma}
\theoremstyle{change}
\theorembodyfont{\itshape} \newtheorem{Def}[Cor]{Definition}

\theoremheaderfont{\scshape}

\RecordChanges

\begin{document}
   \DocInput{theorem.dtx}
\end{document}
%</driver>
%    \end{macrocode}
%
% \section{Definition of the Macros}
%
% If the file has been loaded before, we abort immediately. If not the
% package announces itself (this is actually done at the very top of
% the file---the way it is done isn't good style so don't copy it).
% \changes{v2.0e}{89/07/19}{Spaces removed from `typeout'.}
% \changes{v2.2b}{95/11/19}{Announce moved to top of file}
%    \begin{macrocode}
%<*package>
%\@ifundefined{theorem@style}{}{\endinput}
%\def\FMithmInfo{1995/11/19 v2.2b Theorem extension package (FMi)}
%\ProvidesPackage{theorem}[\FMithmInfo]
%    \end{macrocode}
%
%
% \subsection{Definition of theorem styles and fonts}
%
% All the definitions in this file are done globally to allow
% inputting this file inside a group.
%
% \begin{macro}{\theoremstyle}
%    Before a theorem style can be installed, the chosen style must be
%    known. For that reason, we must test to see that
%    |\th@|\meta{style} is known or, more precisely, that it is
%    different from |\relax|. If the style is not known then
%    |\th@plain| is used.
%    \begin{macrocode}
\gdef\theoremstyle#1{%
   \@ifundefined{th@#1}{\@warning
          {Unknown theoremstyle `#1'. Using `plain'}%
          \theorem@style{plain}}%
%    \end{macrocode}
%    We save the theorem style to be used in the token register
%    |\theorem@style|.
%    \begin{macrocode}
      {\theorem@style{#1}}%
%    \end{macrocode}
%    Now we ``evaluate'' the theorem style: this means, we call the
%    macro |\th@|\meta{style} which will activate the relevant
%    definitions which are contained in a separate file.  This is done
%    in a group to suppress changes to the current font.  This could
%    otherwise pose problems together with the new font selection
%    scheme\footnote{When I printed the original article using the new
%    font selection scheme I ended with a document with slanted
%    typefaces (text headings and all) simply because one of the
%    theorem styles used \texttt{\bslash sl} at toplevel.} if the
%    |\th@|\meta{style} is evaluated a second time.
% \changes{v2.1a}{90/02/15}{Group added around `th@..}
% \changes{v2.1a}{90/02/15}{Reseting of `..@begintheorem removed.}
%    \begin{macrocode}
      \begingroup
        \csname th@\the\theorem@style \endcsname
      \endgroup}
%    \end{macrocode}
% \end{macro}
%
%
% \begin{macro}{\@begintheorem}
% \changes{v2.1a}{90/02/15}{Relaxed at toplevel.}
% \begin{macro}{\@opargbegintheorem}
% \changes{v2.1a}{90/02/15}{Relaxed at toplevel.}
%    We reset |\@begintheorem| and |\@opargbegintheorem| to
%    |\relax| since these commands are no longer necessary at
%    toplevel. This will save a few tokens.
%    \begin{macrocode}
\global\let\@begintheorem\relax
\global\let\@opargbegintheorem\relax
%    \end{macrocode}
% \end{macro}
% \end{macro}
%
% \begin{macro}{\theorem@style}
%    Obviously the \textsf{token} register used above has to be
%    allocated. To assure the utmost compatibility with the
%    original \LaTeX{} definition, we set the default theorem style to
%    \texttt{plain}, which implements the usual \LaTeX{} convention.
% \changes{v2.0f}{89/07/20}{Unnecessary `global' removed.}
%    \begin{macrocode}
\newtoks\theorem@style
\global\theorem@style{plain}
%    \end{macrocode}
% \end{macro}
%
% \begin{macro}{\theorembodyfont}
% \begin{macro}{\theorem@bodyfont}
%    If the |\theorembodyfont| is set by the user then it should not
%    interact with the default font set in the theorem style. When the
%    new font selection is in force this may happen if, for example,
%    the default is |\itshape| and the new |\theorembodyfont| is
%    |\sffamily|. So we add a |\reset@font| command in front of the
%    user definition.
% \changes{v2.0f}{89/07/20}{Unnecessary `global' removed.}
% \changes{v2.0g}{89/09/19}{`theoremfont to `theorembodyfont.}
% \changes{v2.1c}{92/06/22}{`reset@font added and logic changed.}
% \changes{v2.1d}{92/10/28}{Reset to empty if arg is empty}
%    \begin{macrocode}
\gdef\theorembodyfont#1{%
%    \end{macrocode}
%    We check if the argument supplied is empty and if so put nothing
%    into the |\theorem@bodyfont| token register to allow for
%    |\theorembodyfont{}| as a mean of using the default of the
%    current |\theoremstyle|.
%    \begin{macrocode}
   \def\@tempa{#1}%
   \ifx\@tempa\@empty
    \theorem@bodyfont{}%
   \else
    \theorem@bodyfont{\reset@font#1}%
   \fi
   }
\newtoks\theorem@bodyfont
\global\theorem@bodyfont{}
%    \end{macrocode}
% \end{macro}
% \end{macro}
%
%
% \begin{macro}{\theoremheaderfont}
% \changes{v2.0g}{89/09/19}{Macro added.}
%    The font for the theorem headers is handled differently because
%    this definition applies to all theorem styles.
%    \begin{macrocode}
\gdef\theoremheaderfont#1{\gdef\theorem@headerfont{#1}%
%    \end{macrocode}
%    After using the macro once it is redefined to produce an error
%    message.
%    \begin{macrocode}
       \gdef\theoremheaderfont##1{%
        \typeout{\string\theoremheaderfont\space should be used
                 only once.}}}
%    \end{macrocode}
% \end{macro}
%
% \begin{macro}{\theorem@headerfont}
% \changes{v2.0g}{89/09/19}{Macro added.}
%    To set the |\theorem@headerfont| default we first test if
%    the new fontselection scheme is in force.
% \changes{v2.1a}{90/02/15}{Support for new font selection.}
%    \begin{macrocode}
\ifx\upshape\undefined
%    \end{macrocode}
%    If not we define it to expand into |\bfseries|. We don't use
%    |\let| just in case a following style option redefines this
%    macro.
%    \begin{macrocode}
\gdef\theorem@headerfont{\bfseries}
%    \end{macrocode}
%    Otherwise we reset the current shape before calling |\bfseries|.
%    \begin{macrocode}
\else \gdef\theorem@headerfont{\normalfont\bfseries}\fi
%    \end{macrocode}
% \end{macro}
%
% \begin{macro}{\th@plain}
% \begin{macro}{\th@break}
% \begin{macro}{\th@marginbreak}
% \begin{macro}{\th@changebreak}
% \begin{macro}{\th@change}
% \begin{macro}{\th@margin}
%    The different styles are defined in macros such as
%    |\th@plain|.  Since memory space is precious in
%    ``non-Big-versions'', we have to avoid offering too many unused
%    definitions.  Therefore we define these styles in separate files
%    that can be loaded on demand.  Thus the commands themselves only
%    load these files. We use |\@input@|, a \LaTeXe{} internal command
%    that ensures that the file will be listed with |\listfiles|.
% \changes{v2.2a}{94/02/02}{Use `@input@ to load theorem layout files}
%    \begin{macrocode}
\gdef\th@plain{\@input@{thp.sty}}
\gdef\th@break{\@input@{thb.sty}}
\gdef\th@marginbreak{\@input@{thmb.sty}}
\gdef\th@changebreak{\@input@{thcb.sty}}
\gdef\th@change{\@input@{thc.sty}}
\gdef\th@margin{\@input@{thm.sty}}
%    \end{macrocode}
%    This list will be expanded when new styles become
%    available.  For testing,  just append new theorem substyles
%    as document options.
% \end{macro}
% \end{macro}
% \end{macro}
% \end{macro}
% \end{macro}
% \end{macro}
%
%
%
% \subsection{Definition of a new theorem set}
%
% As already pointed out, a new theorem environment can be defined in
% three different ways:
% \begin{verbatim}
%    \newtheorem{Lem}{Lemma}
%    \newtheorem{Lem}{Lemma}[section]
%    \newtheorem{Lem}[Theorem]{Lemma}
%\end{verbatim}
% The function of the macro |\newtheorem| is to recognize these cases
% and then to branch into one of the three macros |\@ynthm|, |\@xnthm|
% or |\@othm|.  This mechanism is adopted unchanged from
% \cite{latex:source}; the essential point here is that, for example,
% in the second case, the arguments \texttt{Lem}, \texttt{Lemma} and
% \texttt{section} are passed over to the macro |\@xnthm|.
%
% We inspect this case first because the others present fewer problems,
% and thus are easily derived from this one.
%
% \begin{macro}{\@xnthm}
%    For our example arguments, the macro |\@xnthm| must fulfill
%    the following:
%    \begin{itemize}
%       \item
%          Define a new \LaTeX-counter `Lem'
%       \item
%          reset this counter within a |\section|
%       \item
%          define the macro |\theLem|
%       \item
%          define the environment macros |\Lem| and
%          |\endLem| using the current
%          |\theoremstyle| and |\theorem@bodyfont|.
%    \end{itemize}
%    Obviously, all this should happen only if the first argument
%    of |\@xnthm| (i.e.\ \texttt{Lem} in our example) is chosen so
%    as not to conflict with any previously defined commands or
%    environments.  This test is performed by the \LaTeX{} macro
%    |\@ifdefinable|.
%    \begin{macrocode}
\gdef\@xnthm#1#2[#3]{\expandafter\@ifdefinable\csname #1\endcsname
%    \end{macrocode}
%    Therefore, the first argument of |\@ifdefinable| is the
%    expansion (in the example, |\Lem|) of
%    |\csname||#1\endcsname|.  The second argument is
%    executed only if the test has been completed successfully.
%    \begin{macrocode}
   {%
%    \end{macrocode}
%    Now we define the new counter. The names of the \LaTeX{}
%    macros employed should speak for themselves:
% \changes{v2.2b}{1995/10/07}{Use \cs{@newctr}}
% \changes{v2.2c}{1995/11/23}{Use \cs{@newctr} properly.}
%    \begin{macrocode}
    \@definecounter{#1}\@newctr{#1}[#3]%
%    \end{macrocode}
%    Using |\@newctr| will give a proper error message if the counter
%    in |#3| is not defined.
%    In defining `|\theLem|' we must generate the desired macro
%    name by use of |\expandafter| and |\csname|.
%    \begin{macrocode}
    \expandafter\xdef\csname the#1\endcsname
%    \end{macrocode}
%    An |\xdef| is used in order to make the definition global,
%    and to ensure that it contains the replacement texts of
%    |\@thmcountersep| and |\@thmcounter|.\footnote{These
%    two macros can be defined by the document style.  Their default
%    values produce a `\texttt{.}' as separation and an arabic
%    representation of the number.} However, not everything should be
%    expanded. For example, it saves space to use |\thesection|
%    instead of its---at times---lengthy expansion.
%    \begin{macrocode}
      {\expandafter \noexpand \csname the#3\endcsname
       \@thmcountersep \@thmcounter{#1}}%
%    \end{macrocode}
%    Thus with the defaults of \LaTeX{}, |\theLem| would be
%    replaced by the command sequence |\thesection.\arabic{Lem}|.
%
%    We will now look at the definition of the macro which is executed
%    at the beginning of the actual environment (in our example this
%    macro is |\Lem|). It should be noted that we use an
%    ``|\expandafter| trick'' to expand only certain parts of the
%    replacement text at the time of the definition.
%    \begin{macrocode}
    \def\@tempa{\global\@namedef{#1}}%
    \expandafter \@tempa \expandafter{%
%    \end{macrocode}
%    First, the macro that contains the current definitions of
%    |\@begintheorem| and |\@opargtheorem| should be called
%    up.  The name of this macro---as is already known---has the form
%    |\th@|\meta{theorem\ style}; therefore, it must be called by
%    \begin{macrocode}
      \csname th@\the \theorem@style
%    \end{macrocode}
%    In addition the default theorem font should be changeable, i.e.\
%    we have to insert the contents of |\theorem@bodyfont|. For
%    that reason, we expand even further, beyond |\endcsname|,
%    and thus insert the contents of the \textsf{token} register
%    |\theorem@bodyfont| in the replacement text.
%    \begin{macrocode}
            \expandafter \endcsname \the \theorem@bodyfont
%    \end{macrocode}
%    Now it is time to call the macro |\@thm| which takes over
%    the further processing.  It has two arguments: the current
%    counter name (in our example, \texttt{Lem}), and the text of the
%    label (in our example, \texttt{Lemma}).
%    \begin{macrocode}
     \@thm{#1}{#2}}%
%    \end{macrocode}
%    With this, the `sub-definition' is complete.  The macro
%    |\@endtheorem| ends a theorem environment and is, so far, nothing
%    but an |\endtrivlist|.  (Hence it is defined globally, and not
%    within the theorem styles.\footnote{This has to be changed as
%    soon as theorem styles that change \texttt{\bslash @endtheorem}
%    exist.  In such a case, all existing styles must be changed as
%    well since they will have to reset the macro.}) Therefore, we can
%    set it equivalent to the macro that ends the theorem set (in our
%    example, |\endLem|).  However, if some day theorem styles exist
%    that do change |\@endtheorem|, we would have to use the
%    commented-out line instead.
%    \begin{macrocode}
    \global \expandafter \let \csname end#1\endcsname \@endtheorem
%   \global\@namedef{end#1}{\@endtheorem}%
%    \end{macrocode}
%    With these commands all the required definitions are employed,
%    unless the test |\@ifdefinable| has failed.  Therefore, we
%    end the second argument of this macro and with it the definition
%    of |\@xnthm|.
%    \begin{macrocode}
   }}
%    \end{macrocode}
% \end{macro}
%
% \begin{macro}{\@ynthm}
%    The definition of |\@ynthm| is completely analogous. In
%    this case the new counter that is defined is not reset within
%    another counter; thus the definition of |\the...|
%    is simplified:
%    \begin{macrocode}
\gdef\@ynthm#1#2{\expandafter\@ifdefinable\csname #1\endcsname
   {\@definecounter{#1}%
    \expandafter\xdef\csname the#1\endcsname{\@thmcounter{#1}}%
%    \end{macrocode}
%    The rest of the definition corresponds literally to that of
%    |\@xnthm|:
%    \begin{macrocode}
    \def\@tempa{\global\@namedef{#1}}\expandafter \@tempa
     \expandafter{\csname th@\the \theorem@style \expandafter
     \endcsname \the\theorem@bodyfont \@thm{#1}{#2}}%
    \global \expandafter \let \csname end#1\endcsname \@endtheorem}}
%    \end{macrocode}
% \end{macro}
%
% \begin{macro}{\@othm}
%    The definition of |\@othm| does not contain anything new.
%    \begin{macrocode}
\gdef\@othm#1[#2]#3{%
%    \end{macrocode}
% \changes{v2.2b}{95/10/07}{Build in check for counter defined}
%    We do not define a new counter but instead use one that has
%    already been defined.  Thus the only definition we need is that
%    of this pseudo-counter (i.e.\ |\the|\meta{env.\ name}).
%    First we check if |#2| corresponds to a known counter name.
%    \begin{macrocode}
  \expandafter\ifx\csname c@#2\endcsname\relax
   \@nocounterr{#2}%
  \else
   \expandafter\@ifdefinable\csname #1\endcsname
   {\expandafter \xdef \csname the#1\endcsname
     {\expandafter \noexpand \csname the#2\endcsname}%
%    \end{macrocode}
%    All other parts of the definition can be adopted from
%    |\@xnthm|. We have to remember, though, that in this case
%    the name of the current counter and the theorem label have moved
%    to the second and third arguments.
%    \begin{macrocode}
    \def\@tempa{\global\@namedef{#1}}\expandafter \@tempa
     \expandafter{\csname th@\the \theorem@style \expandafter
     \endcsname \the\theorem@bodyfont \@thm{#2}{#3}}%
    \global \expandafter \let \csname end#1\endcsname \@endtheorem}%
  \fi}
%    \end{macrocode}
% \end{macro}
%
% \subsection{Macros that are employed in a {\envfont theorem}
%    environment}
%
% \begin{macro}{\@thm}
%    The macro |\@thm| has to increase the current counter.  Then,
%    depending on whether the environment has (or does not have) an
%    optional argument, it has to branch into either |\@begintheorem|
%    or |\@opargtheorem|.
%    \begin{macrocode}
\gdef\@thm#1#2{\refstepcounter{#1}%
%    \end{macrocode}
%    Now we start a {\envfont trivlist} environment, and give
%    |\@topsep| and |\@topsepadd| the values of the \textsf{skip}
%    registers |\theorempreskipamount| and |\theorempostskipamount|.
%    The value in |\@topsep| is the vertical space that is inserted by
%    the first (and only) |\item| in our |\trivlist| whilst
%    |\@topsepadd| is inserted by |\@endparenv| at the end of that
%    {\envfont trivlist} environment.  By using these registers, we
%    obtain the desired space around a {\envfont theorem} environment.
%    \begin{macrocode}
   \trivlist
   \@topsep \theorempreskipamount               % used by first \item
   \@topsepadd \theorempostskipamount           % used by \@endparenv
%    \end{macrocode}
%    Now we have to test whether an optional argument has been given.
% \changes{v2.0d}{89/07/18}{Forgotten percent added.}
%    \begin{macrocode}
   \@ifnextchar [%
%    \end{macrocode}
%    If there is an optional argument, we will call |\@ythm|, and
%    move the arguments read back into the input stream.
%    \begin{macrocode}
   {\@ythm{#1}{#2}}%
%    \end{macrocode}
%    If not, we call |\@begintheorem|. Its first argument is the
%    name of the theorem set (hence the second argument of
%    |\@thm|). Its second argument is the macro that produces the
%    current number.
%    \begin{macrocode}
   {\@begintheorem{#2}{\csname the#1\endcsname}\ignorespaces}}
%    \end{macrocode}
% \end{macro}
%
% \begin{macro}{\@xthm}
% \begin{macro}{\@ythm}
%    Both these macros were originally called by |\@thm|.  We do
%    not need |\@xthm| anymore, hence we reset it to
%    |\relax|. The definition of |\@ythm| has not changed at
%    all from its definition in \LaTeX. In order to make the macros
%    easier to understand, we will
%    nevertheless present it (commented out).
%    \begin{macrocode}
\global\let\@xthm\relax
% \def\@ythm#1#2[#3]{\@opargbegintheorem{#2}{\csname
%           the#1\endcsname}{#3}\ignorespaces}
%</package>
%    \end{macrocode}
%    The primitive |\ignorespaces| in |\@ythm| and
%    |\@thm| is needed to remove the spaces between the
%    |\begin{...}| and the actual text.
% \end{macro}
% \end{macro}
%
%
% \subsection{Definition of the theorem substyles}
%
% As already pointed out, the theorem substyles, defined below, are
% only loaded when necessary.  Note that all these substyles, except
% \texttt{plain}, have |\slshape| as the default body font.
%
% \subsubsection{The \texttt{plain} style}
%
% \label{dok:th@plain} As the following macros use |@|, we have
% to locally set the |\catcode| of this symbol to ``letter''.
% This happens within a group, so that we do not have to worry about
% which |\catcode| that symbol had before.
%    \begin{macrocode}
%<*thp>
\begingroup \makeatletter
%    \end{macrocode}
% Since we are now within a group, we must make all definitions
% globally.  First we make sure that \texttt{theorem.sty} is loaded.
% This will allow us to use this file as a document style option
% without having to call \texttt{theorem} itself as an option.  At the
% same time, we assure that at least version 2 is loaded, since
% |\theorem@style| was not defined in earlier versions.
%    \begin{macrocode}
\@ifundefined{theorem@style}{\input{theorem.sty}}{}
\ProvidesFile{thp.sty}
             [\FMithmInfo]
%    \end{macrocode}
%
%
% \begin{macro}{\th@plain}
% \begin{macro}{\@begintheorem}
% \begin{macro}{\@opargbegintheorem}
%    |\theoremstyle{plain}| corresponds to the original
%    definition, except that the distances to the surrounding text are
%    determined by the parameters |\theorempreskipamount| and
%    |\theorempostskipamount|.
%    First we set the default body font.
%    \begin{macrocode}
\gdef\th@plain{\normalfont\itshape
%    \end{macrocode}
%    Then we define |\@begintheorem| and
%    |\@opargbegintheorem|. These two macros define how the
%    header of a theorem is typeset.  |\@opargbegintheorem| will
%    be called if a \textsf{theorem} environment with an optional
%    argument is encountered; otherwise, the header is constructed by
%    calling |\@begintheorem|.  If one of these macros is
%    executed, we are within a \textsf{trivlist} environment started by
%    |\@thm|.  So the theorem header is produced with an
%    |\item| command.
%
%    Instead of specifying the header font directly, all standard
%    theorem styles use the |\theorem@headerfont| macro to allow
%    customization.  The extra space (|\labelsep|) is necessary
%    because of problems in the \textsf{trivlist} environment.
%    \begin{macrocode}
  \def\@begintheorem##1##2{%
        \item[\hskip\labelsep \theorem@headerfont ##1\ ##2]}%
%    \end{macrocode}
%    The definition of |\@opargbegintheorem| is completely
%    analogous. The only difference is the fact that there exists a
%    third argument (which is the optional parameter of the
%    environment and contains additional information about the
%    theorem).  Customarily we enclose it in parentheses.
%    \begin{macrocode}
\def\@opargbegintheorem##1##2##3{%
   \item[\hskip\labelsep \theorem@headerfont ##1\ ##2\ (##3)]}}
%    \end{macrocode}
% \end{macro}
% \end{macro}
% \end{macro}
%
% \noindent
% We conclude with an |\endgroup| to restore the
% |\catcode| of |@|.
%    \begin{macrocode}
\endgroup
%</thp>
%    \end{macrocode}
%
%
% \subsubsection{The \texttt{break} style}
%
% This style option is stored in the file |thb.sty|.
% For the next two lines see the documentation for |\th@plain| on
% page~\pageref{dok:th@plain}.
%    \begin{macrocode}
%<*thb>
\begingroup \makeatletter
\@ifundefined{theorem@style}{\input{theorem.sty}}{}
\ProvidesFile{thb.sty}
             [\FMithmInfo]
%    \end{macrocode}
%
% \begin{macro}{\th@break}
%    |\theoremstyle{break}| produces a line break after the name
%    of the theorem.  The font is |\slshape|.  Hence, we define
%    |\th@break| as follows:
%    \begin{macrocode}
\gdef\th@break{\normalfont\slshape
  \def\@begintheorem##1##2{\item[%
%    \end{macrocode}
%    We run into the following problem: it is not possible to create
%    the header with |\item[|\meta{title}|]| and then start
%    a new line by, for example, |\mbox{}\\|. Such a definition
%    will fail whenever a list environment follows immediately.  With
%    the above construction, the |\mbox{}| causes the switch
%    |@inlabel| (cf.\ definition of |\list| and
%    |\trivlist| in \cite{latex:source}) to be set to \texttt{false}
%    and so the following list will insert additional vertical
%    space (|\topskip|). This is quite annoying. Therefore, we
%    create the line break within the |\item|. In order to ensure
%    that the text will begin at the proper position in the following
%    line, we simply pretend that the label does not take any
%    room.\footnote{This will lead to problems whenever very high
%    symbols occurring in the line tower into the
%    heading. So, something else has to be done here sometime.}
%    \begin{macrocode}
       \rlap{\vbox{\hbox{\hskip \labelsep\theorem@headerfont ##1\ ##2}%
                   \hbox{\strut}}}]}%
%    \end{macrocode}
%    Again, the definition of |\@opargbegintheorem| is
%    completely analogous.
%    \begin{macrocode}
\def\@opargbegintheorem##1##2##3{%
  \item[\rlap{\vbox{\hbox{\hskip \labelsep \theorem@headerfont
                     ##1\ ##2\ (##3)}%
                    \hbox{\strut}}}]}}
\endgroup
%</thb>
%    \end{macrocode}
% \end{macro}
%
%
% \subsubsection{The \texttt{changebreak} style}
%
%    \begin{macrocode}
% This style option is stored in the file |thcb.sty|.
%    \begin{macrocode}
%<*thcb>
\begingroup \makeatletter
\@ifundefined{theorem@style}{\input{theorem.sty}}{}
\ProvidesFile{thcb.sty}
             [\FMithmInfo]
%    \end{macrocode}
%
% \begin{macro}{\th@changebreak}
%    The \texttt{change-break} theorem style is like \texttt{break}
%    but with interchange of theorem name and theorem number.  Thus we
%    define |\th@changebreak| as follows:
%    \begin{macrocode}
\gdef\th@changebreak{\normalfont\slshape
  \def\@begintheorem##1##2{\item
       [\rlap{\vbox{\hbox{\hskip\labelsep \theorem@headerfont ##2\ ##1}%
                    \hbox{\strut}}}]}%
\def\@opargbegintheorem##1##2##3{%
  \item[\rlap{\vbox{\hbox{\hskip\labelsep \theorem@headerfont
                     ##2\ ##1\ (##3)}%
                    \hbox{\strut}}}]}}
\endgroup
%</thcb>
%    \end{macrocode}
% \end{macro}
%
%
% \subsubsection{The \texttt{change} style}
%
% This style option is stored in the file |thc.sty|.
%    \begin{macrocode}
%<*thc>
\begingroup \makeatletter
\@ifundefined{theorem@style}{\input{theorem.sty}}{}
\ProvidesFile{thc.sty}
             [\FMithmInfo]
%    \end{macrocode}
%
% \begin{macro}{\th@change}
%    The \texttt{change} theorem style corresponds to the
%    \texttt{change break} style without a linebreak after the header.
%    To say it in another way, it's the same as the \texttt{plain}
%    style but with number and name interchanged and |\slshape| as the
%    default font.
%    \begin{macrocode}
\gdef\th@change{\normalfont\slshape
  \def\@begintheorem##1##2{\item
       [\hskip\labelsep \theorem@headerfont ##2\ ##1]}%
\def\@opargbegintheorem##1##2##3{%
  \item[\hskip\labelsep \theorem@headerfont ##2\ ##1\ (##3)]}}
\endgroup
%</thc>
%    \end{macrocode}
% \end{macro}
%
%
% \subsubsection{The \texttt{marginbreak} style}
%
% This style option is the one used most often at Mainz.  It is saved
% in the file |thmb.sty|.
%    \begin{macrocode}
%<*thmb>
\begingroup \makeatletter
\@ifundefined{theorem@style}{\input{theorem.sty}}{}
\ProvidesFile{thmb.sty}
             [\FMithmInfo]
%    \end{macrocode}
%
% \begin{macro}{\th@marginbreak}
%    The \texttt{margin break} style is nearly the same as the
%    \texttt{change break} style. The only difference is the placement
%    of the theorem number.  We use |\llap| to place it in the left
%    margin.
%
%    In this style |\labelsep| denotes the separation between the
%    number and the text.
%    \begin{macrocode}
\gdef\th@marginbreak{\normalfont\slshape
  \def\@begintheorem##1##2{\item
       [\rlap{\vbox{\theorem@headerfont
                    \hbox{\llap{##2}\hskip\labelsep ##1}%
                    \hbox{\strut}}}]}%
\def\@opargbegintheorem##1##2##3{%
  \item[\rlap{\vbox{\theorem@headerfont
                    \hbox{\llap{##2}\hskip\labelsep ##1\ (##3)}%
                    \hbox{\strut}}}]}}
\endgroup
%</thmb>
%    \end{macrocode}
% \changes{v2.1c}{90/02/17}{`endgroup outside macrocode env.}
% \end{macro}
%
%
%
% \subsubsection{The \texttt{margin} style}
%
% This style option is stored in the file |thm.sty|.
%    \begin{macrocode}
%<*thm>
\begingroup \makeatletter
\@ifundefined{theorem@style}{\input{theorem.sty}}{}
\ProvidesFile{thm.sty}
             [\FMithmInfo]
%    \end{macrocode}
%
% \begin{macro}{\th@margin}
%    Again this is only a variant of the theorem styles described
%    above without any new ideas.
%    \begin{macrocode}
\gdef\th@margin{\normalfont\slshape
  \def\@begintheorem##1##2{\item
       [\theorem@headerfont \llap{##2}\hskip\labelsep ##1]}%
\def\@opargbegintheorem##1##2##3{%
  \item[\theorem@headerfont \llap{##2}\hskip\labelsep ##1\ (##3)]}}
\endgroup
%</thm>
%    \end{macrocode}
% \end{macro}
%
%
%
%
% \subsection{Final Definitions}
%
% \begin{macro}{\theorempreskipamount}
% \begin{macro}{\theorempostskipamount}
%    The \textsf{skip} parameters that regulate the vertical empty space
%    before and after the {\envfont theorem} environment have
%    to be allocated as well.
% \changes{v2.0f}{89/07/20}{Unnecessary `global' removed.}
%    \begin{macrocode}
%<*package>
\newskip\theorempreskipamount
\newskip\theorempostskipamount
%    \end{macrocode}
%    Since we have used the same values for all theorem sets, we now can
%    assign them.
%    \begin{macrocode}
\global\setlength\theorempreskipamount{12pt plus 5pt minus 3pt}
\global\setlength\theorempostskipamount{8pt plus 3pt minus 1.5pt}
%    \end{macrocode}
% \end{macro}
% \end{macro}
%
%
% \begin{macro}{\@endtheorem}
%    The same holds for the macro |\@endtheorem|, which ends a
%    {\envfont theorem} environment. Since it is the same for all
%    theorem sets, it is removed from the macros |\th@|\meta{style}.
%    It simply ends the {\envfont trivlist} environment, which was
%    begun in |\@thm|.
%    \begin{macrocode}
\global\let\@endtheorem=\endtrivlist
%    \end{macrocode}
% \end{macro}
%
%
%
% \begin{macro}{\@preamblecmds}
%    All macros defined above are to be used only in the preamble.
%    Therefore, we insert them in |\@preamblecmds| which will disable
%    them at begin document. This is done by the internal \LaTeXe{}
%    command |\@onlypreamble|.
% \changes{v2.0g}{89/09/19}{`theoremheaderfont and `theorembodyfont
%    added.}
% \changes{v2.2a}{94/02/03}{Use \LaTeXe{} interface}
%    \begin{macrocode}
\@onlypreamble\@xnthm
\@onlypreamble\@ynthm
\@onlypreamble\@othm
\@onlypreamble\newtheorem
\@onlypreamble\theoremstyle
\@onlypreamble\theorembodyfont
\@onlypreamble\theoremheaderfont
%    \end{macrocode}
% \end{macro}
%
% Finally we declare the \texttt{plain} theorem style to be the default.
% \changes{v2.1b}{90/06/04}{Fixed missing `item by declaring default
%    style.}
%    \begin{macrocode}
\theoremstyle{plain}
%</package>
%    \end{macrocode}
%
% \Finale

\endinput
}{}
\ProvidesFile{thmb.sty}
             [\FMithmInfo]
%    \end{macrocode}
%
% \begin{macro}{\th@marginbreak}
%    The \texttt{margin break} style is nearly the same as the
%    \texttt{change break} style. The only difference is the placement
%    of the theorem number.  We use |\llap| to place it in the left
%    margin.
%
%    In this style |\labelsep| denotes the separation between the
%    number and the text.
%    \begin{macrocode}
\gdef\th@marginbreak{\normalfont\slshape
  \def\@begintheorem##1##2{\item
       [\rlap{\vbox{\theorem@headerfont
                    \hbox{\llap{##2}\hskip\labelsep ##1}%
                    \hbox{\strut}}}]}%
\def\@opargbegintheorem##1##2##3{%
  \item[\rlap{\vbox{\theorem@headerfont
                    \hbox{\llap{##2}\hskip\labelsep ##1\ (##3)}%
                    \hbox{\strut}}}]}}
\endgroup
%</thmb>
%    \end{macrocode}
% \changes{v2.1c}{90/02/17}{`endgroup outside macrocode env.}
% \end{macro}
%
%
%
% \subsubsection{The \texttt{margin} style}
%
% This style option is stored in the file |thm.sty|.
%    \begin{macrocode}
%<*thm>
\begingroup \makeatletter
\@ifundefined{theorem@style}{% \iffalse meta-comment
%
% Copyright (C) 1993-2024
%
% The LaTeX Project and any individual authors listed elsewhere
% in this file.
%
% This file is part of the Standard LaTeX `Tools Bundle'.
% -------------------------------------------------------
%
% It may be distributed and/or modified under the
% conditions of the LaTeX Project Public License, either version 1.3c
% of this license or (at your option) any later version.
% The latest version of this license is in
%    https://www.latex-project.org/lppl.txt
% and version 1.3c or later is part of all distributions of LaTeX
% version 2005/12/01 or later.
%
% The list of all files belonging to the LaTeX `Tools Bundle' is
% given in the file `manifest.txt'.
%
% \fi
%
%%
%
%
\def\FMithmInfo{2023/07/05 v2.2c Theorem extension package (FMi)}
%
% \ProvidesFile{theorem.dtx}[\FMithmInfo]
%\iffalse   % this is a METACOMMENT !
%%
%% Package `theorem' to use with LaTeX2e
%% Copyright (C) 1989-1995 Frank Mittelbach, all rights reserved.
%<+package>\@ifundefined{theorem@style}{}{\endinput}
%<+package>\NeedsTeXFormat{LaTeX2e}[1995/06/01]
%<+package>\ProvidesPackage{theorem}
%<+package>         [\FMithmInfo]
%
% \fi
%
% \changes{v1.0}{87/05/07}{First draft.}
% \changes{v1.0a}{87/05/07}{`thistheoremstyle' added.}
% \changes{v1.1a}{87/07/29}{New concept: redefine the internal
%                          \protect\LaTeX{} commands `@xnthm' etc.}
% \changes{v1.1b}{87/07/31}{`theoremfont' used to specify font.}
% \changes{v1.2a}{88/01/08}{`@endtheorem' now globally defined.}
% \changes{v1.2b}{88/01/11}{`newtheorem' only allowed in the preamble.}
% \changes{v1.3a}{88/02/14}{`theoremfont' now a toks register.}
%
% \changes{v2.0a}{88/11/16}{New implementation and new documentation.}
% \changes{v2.0a}{88/11/16}{No layout changes but internal modifications
%                           old substyles won't work.}
% \changes{v2.0b}{89/01/30}{`global' used for all definitions.}
% \changes{v2.0c}{89/06/15}{Documentation translated to English.
%                            Thanks to LLL and Nina}
% \changes{v2.1a}{90/02/15}{Support for new font selection.}
% \changes{v2.2a}{94/02/03}{Cleanup for LaTeX2e}
%
% \DoNotIndex{\@endparenv,\@ifdefinable,\@ifnextchar,\@ifundefined}
% \DoNotIndex{\@namedef,\@nameuse,\@spaces,\@tempa,\@warning}
% \DoNotIndex{\ }
% \DoNotIndex{\begingroup,\bfseries}
% \DoNotIndex{\csname}
% \DoNotIndex{\def,\do,\docdate}
% \DoNotIndex{\endcsname,\endgroup,\endinput,\endtrivlist,\expandafter}
% \DoNotIndex{\filedate,\fileversion}
% \DoNotIndex{\gdef,\global}
% \DoNotIndex{\hbox,\hskip}
% \DoNotIndex{\ignorespaces,\input,\itshape,\item}
% \DoNotIndex{\labelsep,\let,\llap}
% \DoNotIndex{\makeatletter}
% \DoNotIndex{\noexpand}
% \DoNotIndex{\refstepcounter,\relax,\rlap}
% \DoNotIndex{\setlength,\slshape,\space,\string,\strut}
% \DoNotIndex{\the,\trivlist,\typeout}
% \DoNotIndex{\vbox}
% \DoNotIndex{\xdef}
%
% \def\deflabel#1{\ttfamily #1\hfill}
% \def\deflist#1{\begin{list}{}{\settowidth\labelwidth{\ttfamily #1}%
%                               \setlength\leftmargin\labelwidth
%                               \addtolength\leftmargin\labelsep
%                               \let\makelabel\deflabel}}
% \def\enddeflist{\end{list}}
% \def\envfont{\normalfont\ttfamily}
%
% \MakeShortVerb{\|}
%
% \newcommand{\TUB}{{\sl TUGboat\/}}
%
% \GetFileInfo{theorem.dtx}
% \title{An Extension of the
%        \LaTeX{} theorem environment\thanks{This file
%        has version number \fileversion, last
%        revised \filedate.}}
% \author{Frank Mittelbach}
%
% \MaintainedByLaTeXTeam{tools}
% \maketitle
%
% \begin{abstract}
%    The macros described in this paper yield an extension of the
%    \LaTeX{} theorem mechanism. It is designed to satisfy the
%    different requirements of various journals. Thus, the
%    layout of the ``theorems'' can be manipulated by determining a
%    ``style''. This article describes not only the use, but
%    also the definition, of the necessary macros.
% \end{abstract}
%
% \section*{Preface to version 2.2}
%
% For \LaTeXe{} this package did not need any fundamental changes. I
% only modified the messages generated so that theorem layout styles
% will show up with the |\listfiles| command and cleaned the section
% on the New Font Selection Scheme since this is now included in
% \LaTeX.
%
% \section*{Preface to version 2.1}
%
% This version is identical to 2.0g described in \TUB\ 10\#3 except
% for some internal defaults which are now set depending on the used
% font selection scheme.
%
% This was done to avoid unpleasant surprises if the new font selection
% scheme is in force. For further details see
% section~\ref{sec:newfonts} and \cite{bk:GMS94}.
%
% \section{Introduction}
%
% For our purposes here, ``theorems'' are labelled enunciations,
% often set off from the main text by extra space and a font change.
% Theorems, corollaries, conjectures, definitions, and remarks
% are all instances of ``theorems''.  The ``header'' of these structures
% is composed of a label (such as \textsc{Theorem} or \textsc{Remark})
% and a number which serializes an item in the sequence of items
% with the same label.
%
% Shortly after the introduction of \LaTeX{} at the Fachbereich
% Mathematik in Mainz, the desire to manipulate the layout of
% ``theorems'' arose.  In Mainz, the following two conventions came
% into general use:
% \begin{enumerate}
%   \item The number of the theorem is shown in the margin.
%   \item There is a line break at
%         the end of the theorem header.
% \end{enumerate}
%
% Additionally, some journals require different formats which depend on
% the ``sort of theorem'': e.g.\ often remarks and definitions are set
% in |\upshape|, while |\itshape| is employed for main theorems.
%
% Confronted with these requirements, a theorem environment was
% developed in Mainz which allows separate determination of the
% layout of the ``theorems sets'', comparable to |\pagestyle|.
%
%
% \section{The user interface}
% \subsection{Defining new theorem sets}
%
% \DescribeMacro\newtheorem
% As in the original \LaTeX{} version, the command |\newtheorem|
% defines a new ``theorem set'' or ``theorem-like structure''.
% Two required arguments name the new environment and give the text
% to be typeset with each instance of the new ``set'', while
% an optional argument determines how
% the ``set'' is enumerated:
% \begin{description}
%    \item[\ttfamily \bslash newtheorem\{foo\}\{bar\}]
%       The theorem set {\envfont foo} (whose name is \texttt{bar})
%       uses its own counter.
%    \item[\ttfamily \bslash newtheorem\{foo2\}{[foo]}\{bar2\}]
%       The theorem set {\envfont foo2} (printed name \texttt{bar2})
%       uses the same counter as the theorem set \texttt{foo}.
%    \item[\ttfamily \bslash newtheorem\{foo3\}\{bar3\}{[section]}]
%       The theorem set {\envfont foo3} (printed name \texttt{bar3}) is
%       enumerated within the counter \texttt{section}, i.e.\ with every
%       new |\section| the enumeration begins again with 1, and
%       the enumeration is composed from the section-number and the
%       theorem counter itself.
% \end{description}
%
% \DescribeMacro\theoremstyle
% Additionally, the command |\theoremstyle| can define the layout
% of various, or all, theorem sets.  It should be noted that any
% theorem set defined by |\newtheorem| is typeset in the
% |\theoremstyle| that is current at the time of the definition.
% Thus, the following
% \begin{verbatim}
%    \theoremstyle{break}        \newtheorem{Cor}{Corollary}
%    \theoremstyle{plain}        \newtheorem{Exa}{Example}[section]
%\end{verbatim}                             ^^A No blank at line start
% leads to the result that the set {\envfont Cor} is formatted in the
% style \texttt{break}, while the set {\envfont Exa} and all the
% following ones are formatted in the style \texttt{plain}, unless
% another |\theoremstyle| follows.  Since the definitions installed by
% |\newtheorem| are global, one also can limit |\theoremstyle| locally
% by grouping braces.
%
% \DescribeMacro\theorembodyfont
% The choice of the font for the theorem body is completely
% independent of the chosen |\theoremstyle|; this has proven to
% be very advantageous.  For example,
% \begin{verbatim}
%   {\theorembodyfont{\upshape}          \newtheorem{Rem}{Remark}}
%\end{verbatim}
% defines a theorem set {\envfont Rem}, which will be set in
% |\upshape| in the current layout (which in our example is
% \texttt{plain}).  As with |\theoremstyle|, the |\theorembodyfont|
% chosen is that current at the time of |\newtheorem|.  If
% |\theorembodyfont| is not specified or one defines
% |\theorembodyfont{}|, then the font used will be that defined by the
% |\theoremstyle|.
%
% \DescribeMacro\theoremheaderfont It is also possible to customize
% the font used for the theorem headers.  This is, however, a global
% declaration, and therefore there should be at most one
% |\theoremheaderfont| declaration in the preamble.\footnote{If it is
% actually necessary to have different header fonts, one has to define
% new theorem styles (substituting the desired font) or specify the
% information directly in the \texttt{\bslash newtheorem} declaration
% (the unclean variant).}
%
% \DescribeMacro\theorempreskipamount
% \DescribeMacro\theorempostskipamount Two additional parameters
% affect the vertical space around the theorem environments:
% |\theorempreskipamount| and |\theorempostskipamount| define,
% respectively, the spacing before and after such an environment.
% These parameters apply for all theorem sets and can be manipulated
% with the ordinary length macros.  They are rubber lengths
% (`\textsf{skips}'), and therefore can contain \texttt{plus} and
% \texttt{minus} parts.
%
% Since the definition of theorem sets should---most sensibly---be
% placed in the preamble, we only allow installation there.
% It is therefore possible to release the memory used here after
% |\begin{document}|, in order to make room for other
% applications.
%
%
% \subsection{Existing theorem styles}
%
% The following theorem styles exist to date:
% \begin{deflist}{marginbreak}
%    \item[plain]
%       This theorem style emulates the original \LaTeX{} definition,
%       except that additionally the parameters
%       |\theorem...skipamount| are used.
%    \item[break]
%       In this style, the theorem header is followed by a line
%       break.
%    \item[marginbreak]
%       The theorem number is set in the margin, and there is a line
%       break as in \texttt{break}.
%    \item[changebreak]
%       Like \texttt{break}, but with header number and text
%       interchanged.
%    \item[change]
%       Header number and text are interchanged, without a line break.
%    \item[margin]
%       The number is set in the left margin, without a line break.
% \end{deflist}
% All styles (except \texttt{plain}) select |\slshape| as the default
% |\theorembodyfont|.
%
%
% \subsection{Examples}
%
% Given the above theorem sets {\envfont Cor}, {\envfont Exa}
% and {\envfont Rem},
% suppose that the preamble also contains the declarations:
% \begin{verbatim}
%    \theoremstyle{marginbreak}   \newtheorem{Lem}[Cor]{Lemma}
%    \theoremstyle{change}
%    \theorembodyfont{\itshape}        \newtheorem{Def}[Cor]{Definition}
%
%    \theoremheaderfont{\scshape}
%\end{verbatim}
% Then the following are some typical examples of the typeset output
% resulting from their use.
%
% \begin{Cor}
%    This is a sentence typeset in the theorem environment {\envfont
%    Cor}.
% \end{Cor}
%
% \begin{Exa}
%    This is a sentence typeset in the theorem environment {\envfont
%    Exa}.
% \end{Exa}
%
% \begin{Rem}
%    This is a sentence typeset in the theorem environment {\envfont
%    Rem}.
% \end{Rem}
%
% \begin{Lem}[Ben User]
%    This is a sentence typeset in the theorem environment {\envfont
%    Lem}.
% \end{Lem}
%
% \begin{Def}[Very impressive Definition]
%    This is a sentence typeset in the theorem environment {\envfont
%    Def}.
% \end{Def}
%
% The last two examples show the effect of the optional argument to
% a theorem environment (it is the text typeset in parentheses).
%
%
% \section{Special Considerations}
% \label{sec:newfonts}
%
% Theoremheader and body are implemented as a unit. This means that
% the |\theoremheaderfont|  will inherit characteristics of the
% |\theorembodyfont| in \LaTeXe. Thus, if for example
% |\theorembodyfont| is |\itshape| and |\theoremheaderfont|
% is |\bfseries| the font selected for the header will have the
% characteristics `bold extended italic'. If this is not desired one
% should set the |\theoremheaderfont| to something like
% \begin{verbatim}
%      \theoremheaderfont{\normalfont\bfseries}
%\end{verbatim}
% i.e.\ supplying all necessary font information explicitly.
%
%
%
% \section{Acknowledgements}
%
% The publication of this set of macros was only possible with the
% help of Christina Busse (translating the manuscript into English),
% Joachim Pense (playing the r\^ole of typist), Chris Rowley
% (looking everything over) and many others providing useful
% suggestions.
%
% \MaybeStop{%
%    \begin{thebibliography}{9}
%       \bibitem{bk:GMS94} \textsc{M.~Goossens}, \textsc{F.~Mittelbach}
%         and \textsc{A.~Samarin}.
%         \newblock The \LaTeX{} Companion.
%         \newblock
%         Addison-Wesley, Reading, Massachusetts, 1994.
%       \bibitem{latex:source} \textsc{Lamport, Leslie.} \newblock
%                               \texttt{latex.tex}, version 2.09, date
%                               Feb. 1990.
%    \end{thebibliography}
%    \PrintIndex
%    \PrintChanges}
%
%
% \section{The documentation driver file}
%
% The next bit of code contains the documentation driver file for
% \TeX{}, i.e., the file that will produce the documentation you are
% currently reading. It will be extracted from this file by the
% \texttt{docstrip} program. Since it is the first code in the file
% one can alternatively process this file directly with \LaTeXe{} to
% obtain the documentation.
%
%    \begin{macrocode}
%<*driver>
\documentclass{ltxdoc}

\usepackage{theorem}

%<+driver>% The next few lines define theorem sets which are used
%<+driver>% in the example section of the documentation.

\theoremstyle{break}       \newtheorem{Cor}{Corollary}
\theoremstyle{plain}       \newtheorem{Exa}{Example}[section]
{\theorembodyfont{\upshape}\newtheorem{Rem}{Remark}}
\theoremstyle{marginbreak} \newtheorem{Lem}[Cor]{Lemma}
\theoremstyle{change}
\theorembodyfont{\itshape} \newtheorem{Def}[Cor]{Definition}

\theoremheaderfont{\scshape}

\RecordChanges

\begin{document}
   \DocInput{theorem.dtx}
\end{document}
%</driver>
%    \end{macrocode}
%
% \section{Definition of the Macros}
%
% If the file has been loaded before, we abort immediately. If not the
% package announces itself (this is actually done at the very top of
% the file---the way it is done isn't good style so don't copy it).
% \changes{v2.0e}{89/07/19}{Spaces removed from `typeout'.}
% \changes{v2.2b}{95/11/19}{Announce moved to top of file}
%    \begin{macrocode}
%<*package>
%\@ifundefined{theorem@style}{}{\endinput}
%\def\FMithmInfo{1995/11/19 v2.2b Theorem extension package (FMi)}
%\ProvidesPackage{theorem}[\FMithmInfo]
%    \end{macrocode}
%
%
% \subsection{Definition of theorem styles and fonts}
%
% All the definitions in this file are done globally to allow
% inputting this file inside a group.
%
% \begin{macro}{\theoremstyle}
%    Before a theorem style can be installed, the chosen style must be
%    known. For that reason, we must test to see that
%    |\th@|\meta{style} is known or, more precisely, that it is
%    different from |\relax|. If the style is not known then
%    |\th@plain| is used.
%    \begin{macrocode}
\gdef\theoremstyle#1{%
   \@ifundefined{th@#1}{\@warning
          {Unknown theoremstyle `#1'. Using `plain'}%
          \theorem@style{plain}}%
%    \end{macrocode}
%    We save the theorem style to be used in the token register
%    |\theorem@style|.
%    \begin{macrocode}
      {\theorem@style{#1}}%
%    \end{macrocode}
%    Now we ``evaluate'' the theorem style: this means, we call the
%    macro |\th@|\meta{style} which will activate the relevant
%    definitions which are contained in a separate file.  This is done
%    in a group to suppress changes to the current font.  This could
%    otherwise pose problems together with the new font selection
%    scheme\footnote{When I printed the original article using the new
%    font selection scheme I ended with a document with slanted
%    typefaces (text headings and all) simply because one of the
%    theorem styles used \texttt{\bslash sl} at toplevel.} if the
%    |\th@|\meta{style} is evaluated a second time.
% \changes{v2.1a}{90/02/15}{Group added around `th@..}
% \changes{v2.1a}{90/02/15}{Reseting of `..@begintheorem removed.}
%    \begin{macrocode}
      \begingroup
        \csname th@\the\theorem@style \endcsname
      \endgroup}
%    \end{macrocode}
% \end{macro}
%
%
% \begin{macro}{\@begintheorem}
% \changes{v2.1a}{90/02/15}{Relaxed at toplevel.}
% \begin{macro}{\@opargbegintheorem}
% \changes{v2.1a}{90/02/15}{Relaxed at toplevel.}
%    We reset |\@begintheorem| and |\@opargbegintheorem| to
%    |\relax| since these commands are no longer necessary at
%    toplevel. This will save a few tokens.
%    \begin{macrocode}
\global\let\@begintheorem\relax
\global\let\@opargbegintheorem\relax
%    \end{macrocode}
% \end{macro}
% \end{macro}
%
% \begin{macro}{\theorem@style}
%    Obviously the \textsf{token} register used above has to be
%    allocated. To assure the utmost compatibility with the
%    original \LaTeX{} definition, we set the default theorem style to
%    \texttt{plain}, which implements the usual \LaTeX{} convention.
% \changes{v2.0f}{89/07/20}{Unnecessary `global' removed.}
%    \begin{macrocode}
\newtoks\theorem@style
\global\theorem@style{plain}
%    \end{macrocode}
% \end{macro}
%
% \begin{macro}{\theorembodyfont}
% \begin{macro}{\theorem@bodyfont}
%    If the |\theorembodyfont| is set by the user then it should not
%    interact with the default font set in the theorem style. When the
%    new font selection is in force this may happen if, for example,
%    the default is |\itshape| and the new |\theorembodyfont| is
%    |\sffamily|. So we add a |\reset@font| command in front of the
%    user definition.
% \changes{v2.0f}{89/07/20}{Unnecessary `global' removed.}
% \changes{v2.0g}{89/09/19}{`theoremfont to `theorembodyfont.}
% \changes{v2.1c}{92/06/22}{`reset@font added and logic changed.}
% \changes{v2.1d}{92/10/28}{Reset to empty if arg is empty}
%    \begin{macrocode}
\gdef\theorembodyfont#1{%
%    \end{macrocode}
%    We check if the argument supplied is empty and if so put nothing
%    into the |\theorem@bodyfont| token register to allow for
%    |\theorembodyfont{}| as a mean of using the default of the
%    current |\theoremstyle|.
%    \begin{macrocode}
   \def\@tempa{#1}%
   \ifx\@tempa\@empty
    \theorem@bodyfont{}%
   \else
    \theorem@bodyfont{\reset@font#1}%
   \fi
   }
\newtoks\theorem@bodyfont
\global\theorem@bodyfont{}
%    \end{macrocode}
% \end{macro}
% \end{macro}
%
%
% \begin{macro}{\theoremheaderfont}
% \changes{v2.0g}{89/09/19}{Macro added.}
%    The font for the theorem headers is handled differently because
%    this definition applies to all theorem styles.
%    \begin{macrocode}
\gdef\theoremheaderfont#1{\gdef\theorem@headerfont{#1}%
%    \end{macrocode}
%    After using the macro once it is redefined to produce an error
%    message.
%    \begin{macrocode}
       \gdef\theoremheaderfont##1{%
        \typeout{\string\theoremheaderfont\space should be used
                 only once.}}}
%    \end{macrocode}
% \end{macro}
%
% \begin{macro}{\theorem@headerfont}
% \changes{v2.0g}{89/09/19}{Macro added.}
%    To set the |\theorem@headerfont| default we first test if
%    the new fontselection scheme is in force.
% \changes{v2.1a}{90/02/15}{Support for new font selection.}
%    \begin{macrocode}
\ifx\upshape\undefined
%    \end{macrocode}
%    If not we define it to expand into |\bfseries|. We don't use
%    |\let| just in case a following style option redefines this
%    macro.
%    \begin{macrocode}
\gdef\theorem@headerfont{\bfseries}
%    \end{macrocode}
%    Otherwise we reset the current shape before calling |\bfseries|.
%    \begin{macrocode}
\else \gdef\theorem@headerfont{\normalfont\bfseries}\fi
%    \end{macrocode}
% \end{macro}
%
% \begin{macro}{\th@plain}
% \begin{macro}{\th@break}
% \begin{macro}{\th@marginbreak}
% \begin{macro}{\th@changebreak}
% \begin{macro}{\th@change}
% \begin{macro}{\th@margin}
%    The different styles are defined in macros such as
%    |\th@plain|.  Since memory space is precious in
%    ``non-Big-versions'', we have to avoid offering too many unused
%    definitions.  Therefore we define these styles in separate files
%    that can be loaded on demand.  Thus the commands themselves only
%    load these files. We use |\@input@|, a \LaTeXe{} internal command
%    that ensures that the file will be listed with |\listfiles|.
% \changes{v2.2a}{94/02/02}{Use `@input@ to load theorem layout files}
%    \begin{macrocode}
\gdef\th@plain{\@input@{thp.sty}}
\gdef\th@break{\@input@{thb.sty}}
\gdef\th@marginbreak{\@input@{thmb.sty}}
\gdef\th@changebreak{\@input@{thcb.sty}}
\gdef\th@change{\@input@{thc.sty}}
\gdef\th@margin{\@input@{thm.sty}}
%    \end{macrocode}
%    This list will be expanded when new styles become
%    available.  For testing,  just append new theorem substyles
%    as document options.
% \end{macro}
% \end{macro}
% \end{macro}
% \end{macro}
% \end{macro}
% \end{macro}
%
%
%
% \subsection{Definition of a new theorem set}
%
% As already pointed out, a new theorem environment can be defined in
% three different ways:
% \begin{verbatim}
%    \newtheorem{Lem}{Lemma}
%    \newtheorem{Lem}{Lemma}[section]
%    \newtheorem{Lem}[Theorem]{Lemma}
%\end{verbatim}
% The function of the macro |\newtheorem| is to recognize these cases
% and then to branch into one of the three macros |\@ynthm|, |\@xnthm|
% or |\@othm|.  This mechanism is adopted unchanged from
% \cite{latex:source}; the essential point here is that, for example,
% in the second case, the arguments \texttt{Lem}, \texttt{Lemma} and
% \texttt{section} are passed over to the macro |\@xnthm|.
%
% We inspect this case first because the others present fewer problems,
% and thus are easily derived from this one.
%
% \begin{macro}{\@xnthm}
%    For our example arguments, the macro |\@xnthm| must fulfill
%    the following:
%    \begin{itemize}
%       \item
%          Define a new \LaTeX-counter `Lem'
%       \item
%          reset this counter within a |\section|
%       \item
%          define the macro |\theLem|
%       \item
%          define the environment macros |\Lem| and
%          |\endLem| using the current
%          |\theoremstyle| and |\theorem@bodyfont|.
%    \end{itemize}
%    Obviously, all this should happen only if the first argument
%    of |\@xnthm| (i.e.\ \texttt{Lem} in our example) is chosen so
%    as not to conflict with any previously defined commands or
%    environments.  This test is performed by the \LaTeX{} macro
%    |\@ifdefinable|.
%    \begin{macrocode}
\gdef\@xnthm#1#2[#3]{\expandafter\@ifdefinable\csname #1\endcsname
%    \end{macrocode}
%    Therefore, the first argument of |\@ifdefinable| is the
%    expansion (in the example, |\Lem|) of
%    |\csname||#1\endcsname|.  The second argument is
%    executed only if the test has been completed successfully.
%    \begin{macrocode}
   {%
%    \end{macrocode}
%    Now we define the new counter. The names of the \LaTeX{}
%    macros employed should speak for themselves:
% \changes{v2.2b}{1995/10/07}{Use \cs{@newctr}}
% \changes{v2.2c}{1995/11/23}{Use \cs{@newctr} properly.}
%    \begin{macrocode}
    \@definecounter{#1}\@newctr{#1}[#3]%
%    \end{macrocode}
%    Using |\@newctr| will give a proper error message if the counter
%    in |#3| is not defined.
%    In defining `|\theLem|' we must generate the desired macro
%    name by use of |\expandafter| and |\csname|.
%    \begin{macrocode}
    \expandafter\xdef\csname the#1\endcsname
%    \end{macrocode}
%    An |\xdef| is used in order to make the definition global,
%    and to ensure that it contains the replacement texts of
%    |\@thmcountersep| and |\@thmcounter|.\footnote{These
%    two macros can be defined by the document style.  Their default
%    values produce a `\texttt{.}' as separation and an arabic
%    representation of the number.} However, not everything should be
%    expanded. For example, it saves space to use |\thesection|
%    instead of its---at times---lengthy expansion.
%    \begin{macrocode}
      {\expandafter \noexpand \csname the#3\endcsname
       \@thmcountersep \@thmcounter{#1}}%
%    \end{macrocode}
%    Thus with the defaults of \LaTeX{}, |\theLem| would be
%    replaced by the command sequence |\thesection.\arabic{Lem}|.
%
%    We will now look at the definition of the macro which is executed
%    at the beginning of the actual environment (in our example this
%    macro is |\Lem|). It should be noted that we use an
%    ``|\expandafter| trick'' to expand only certain parts of the
%    replacement text at the time of the definition.
%    \begin{macrocode}
    \def\@tempa{\global\@namedef{#1}}%
    \expandafter \@tempa \expandafter{%
%    \end{macrocode}
%    First, the macro that contains the current definitions of
%    |\@begintheorem| and |\@opargtheorem| should be called
%    up.  The name of this macro---as is already known---has the form
%    |\th@|\meta{theorem\ style}; therefore, it must be called by
%    \begin{macrocode}
      \csname th@\the \theorem@style
%    \end{macrocode}
%    In addition the default theorem font should be changeable, i.e.\
%    we have to insert the contents of |\theorem@bodyfont|. For
%    that reason, we expand even further, beyond |\endcsname|,
%    and thus insert the contents of the \textsf{token} register
%    |\theorem@bodyfont| in the replacement text.
%    \begin{macrocode}
            \expandafter \endcsname \the \theorem@bodyfont
%    \end{macrocode}
%    Now it is time to call the macro |\@thm| which takes over
%    the further processing.  It has two arguments: the current
%    counter name (in our example, \texttt{Lem}), and the text of the
%    label (in our example, \texttt{Lemma}).
%    \begin{macrocode}
     \@thm{#1}{#2}}%
%    \end{macrocode}
%    With this, the `sub-definition' is complete.  The macro
%    |\@endtheorem| ends a theorem environment and is, so far, nothing
%    but an |\endtrivlist|.  (Hence it is defined globally, and not
%    within the theorem styles.\footnote{This has to be changed as
%    soon as theorem styles that change \texttt{\bslash @endtheorem}
%    exist.  In such a case, all existing styles must be changed as
%    well since they will have to reset the macro.}) Therefore, we can
%    set it equivalent to the macro that ends the theorem set (in our
%    example, |\endLem|).  However, if some day theorem styles exist
%    that do change |\@endtheorem|, we would have to use the
%    commented-out line instead.
%    \begin{macrocode}
    \global \expandafter \let \csname end#1\endcsname \@endtheorem
%   \global\@namedef{end#1}{\@endtheorem}%
%    \end{macrocode}
%    With these commands all the required definitions are employed,
%    unless the test |\@ifdefinable| has failed.  Therefore, we
%    end the second argument of this macro and with it the definition
%    of |\@xnthm|.
%    \begin{macrocode}
   }}
%    \end{macrocode}
% \end{macro}
%
% \begin{macro}{\@ynthm}
%    The definition of |\@ynthm| is completely analogous. In
%    this case the new counter that is defined is not reset within
%    another counter; thus the definition of |\the...|
%    is simplified:
%    \begin{macrocode}
\gdef\@ynthm#1#2{\expandafter\@ifdefinable\csname #1\endcsname
   {\@definecounter{#1}%
    \expandafter\xdef\csname the#1\endcsname{\@thmcounter{#1}}%
%    \end{macrocode}
%    The rest of the definition corresponds literally to that of
%    |\@xnthm|:
%    \begin{macrocode}
    \def\@tempa{\global\@namedef{#1}}\expandafter \@tempa
     \expandafter{\csname th@\the \theorem@style \expandafter
     \endcsname \the\theorem@bodyfont \@thm{#1}{#2}}%
    \global \expandafter \let \csname end#1\endcsname \@endtheorem}}
%    \end{macrocode}
% \end{macro}
%
% \begin{macro}{\@othm}
%    The definition of |\@othm| does not contain anything new.
%    \begin{macrocode}
\gdef\@othm#1[#2]#3{%
%    \end{macrocode}
% \changes{v2.2b}{95/10/07}{Build in check for counter defined}
%    We do not define a new counter but instead use one that has
%    already been defined.  Thus the only definition we need is that
%    of this pseudo-counter (i.e.\ |\the|\meta{env.\ name}).
%    First we check if |#2| corresponds to a known counter name.
%    \begin{macrocode}
  \expandafter\ifx\csname c@#2\endcsname\relax
   \@nocounterr{#2}%
  \else
   \expandafter\@ifdefinable\csname #1\endcsname
   {\expandafter \xdef \csname the#1\endcsname
     {\expandafter \noexpand \csname the#2\endcsname}%
%    \end{macrocode}
%    All other parts of the definition can be adopted from
%    |\@xnthm|. We have to remember, though, that in this case
%    the name of the current counter and the theorem label have moved
%    to the second and third arguments.
%    \begin{macrocode}
    \def\@tempa{\global\@namedef{#1}}\expandafter \@tempa
     \expandafter{\csname th@\the \theorem@style \expandafter
     \endcsname \the\theorem@bodyfont \@thm{#2}{#3}}%
    \global \expandafter \let \csname end#1\endcsname \@endtheorem}%
  \fi}
%    \end{macrocode}
% \end{macro}
%
% \subsection{Macros that are employed in a {\envfont theorem}
%    environment}
%
% \begin{macro}{\@thm}
%    The macro |\@thm| has to increase the current counter.  Then,
%    depending on whether the environment has (or does not have) an
%    optional argument, it has to branch into either |\@begintheorem|
%    or |\@opargtheorem|.
%    \begin{macrocode}
\gdef\@thm#1#2{\refstepcounter{#1}%
%    \end{macrocode}
%    Now we start a {\envfont trivlist} environment, and give
%    |\@topsep| and |\@topsepadd| the values of the \textsf{skip}
%    registers |\theorempreskipamount| and |\theorempostskipamount|.
%    The value in |\@topsep| is the vertical space that is inserted by
%    the first (and only) |\item| in our |\trivlist| whilst
%    |\@topsepadd| is inserted by |\@endparenv| at the end of that
%    {\envfont trivlist} environment.  By using these registers, we
%    obtain the desired space around a {\envfont theorem} environment.
%    \begin{macrocode}
   \trivlist
   \@topsep \theorempreskipamount               % used by first \item
   \@topsepadd \theorempostskipamount           % used by \@endparenv
%    \end{macrocode}
%    Now we have to test whether an optional argument has been given.
% \changes{v2.0d}{89/07/18}{Forgotten percent added.}
%    \begin{macrocode}
   \@ifnextchar [%
%    \end{macrocode}
%    If there is an optional argument, we will call |\@ythm|, and
%    move the arguments read back into the input stream.
%    \begin{macrocode}
   {\@ythm{#1}{#2}}%
%    \end{macrocode}
%    If not, we call |\@begintheorem|. Its first argument is the
%    name of the theorem set (hence the second argument of
%    |\@thm|). Its second argument is the macro that produces the
%    current number.
%    \begin{macrocode}
   {\@begintheorem{#2}{\csname the#1\endcsname}\ignorespaces}}
%    \end{macrocode}
% \end{macro}
%
% \begin{macro}{\@xthm}
% \begin{macro}{\@ythm}
%    Both these macros were originally called by |\@thm|.  We do
%    not need |\@xthm| anymore, hence we reset it to
%    |\relax|. The definition of |\@ythm| has not changed at
%    all from its definition in \LaTeX. In order to make the macros
%    easier to understand, we will
%    nevertheless present it (commented out).
%    \begin{macrocode}
\global\let\@xthm\relax
% \def\@ythm#1#2[#3]{\@opargbegintheorem{#2}{\csname
%           the#1\endcsname}{#3}\ignorespaces}
%</package>
%    \end{macrocode}
%    The primitive |\ignorespaces| in |\@ythm| and
%    |\@thm| is needed to remove the spaces between the
%    |\begin{...}| and the actual text.
% \end{macro}
% \end{macro}
%
%
% \subsection{Definition of the theorem substyles}
%
% As already pointed out, the theorem substyles, defined below, are
% only loaded when necessary.  Note that all these substyles, except
% \texttt{plain}, have |\slshape| as the default body font.
%
% \subsubsection{The \texttt{plain} style}
%
% \label{dok:th@plain} As the following macros use |@|, we have
% to locally set the |\catcode| of this symbol to ``letter''.
% This happens within a group, so that we do not have to worry about
% which |\catcode| that symbol had before.
%    \begin{macrocode}
%<*thp>
\begingroup \makeatletter
%    \end{macrocode}
% Since we are now within a group, we must make all definitions
% globally.  First we make sure that \texttt{theorem.sty} is loaded.
% This will allow us to use this file as a document style option
% without having to call \texttt{theorem} itself as an option.  At the
% same time, we assure that at least version 2 is loaded, since
% |\theorem@style| was not defined in earlier versions.
%    \begin{macrocode}
\@ifundefined{theorem@style}{\input{theorem.sty}}{}
\ProvidesFile{thp.sty}
             [\FMithmInfo]
%    \end{macrocode}
%
%
% \begin{macro}{\th@plain}
% \begin{macro}{\@begintheorem}
% \begin{macro}{\@opargbegintheorem}
%    |\theoremstyle{plain}| corresponds to the original
%    definition, except that the distances to the surrounding text are
%    determined by the parameters |\theorempreskipamount| and
%    |\theorempostskipamount|.
%    First we set the default body font.
%    \begin{macrocode}
\gdef\th@plain{\normalfont\itshape
%    \end{macrocode}
%    Then we define |\@begintheorem| and
%    |\@opargbegintheorem|. These two macros define how the
%    header of a theorem is typeset.  |\@opargbegintheorem| will
%    be called if a \textsf{theorem} environment with an optional
%    argument is encountered; otherwise, the header is constructed by
%    calling |\@begintheorem|.  If one of these macros is
%    executed, we are within a \textsf{trivlist} environment started by
%    |\@thm|.  So the theorem header is produced with an
%    |\item| command.
%
%    Instead of specifying the header font directly, all standard
%    theorem styles use the |\theorem@headerfont| macro to allow
%    customization.  The extra space (|\labelsep|) is necessary
%    because of problems in the \textsf{trivlist} environment.
%    \begin{macrocode}
  \def\@begintheorem##1##2{%
        \item[\hskip\labelsep \theorem@headerfont ##1\ ##2]}%
%    \end{macrocode}
%    The definition of |\@opargbegintheorem| is completely
%    analogous. The only difference is the fact that there exists a
%    third argument (which is the optional parameter of the
%    environment and contains additional information about the
%    theorem).  Customarily we enclose it in parentheses.
%    \begin{macrocode}
\def\@opargbegintheorem##1##2##3{%
   \item[\hskip\labelsep \theorem@headerfont ##1\ ##2\ (##3)]}}
%    \end{macrocode}
% \end{macro}
% \end{macro}
% \end{macro}
%
% \noindent
% We conclude with an |\endgroup| to restore the
% |\catcode| of |@|.
%    \begin{macrocode}
\endgroup
%</thp>
%    \end{macrocode}
%
%
% \subsubsection{The \texttt{break} style}
%
% This style option is stored in the file |thb.sty|.
% For the next two lines see the documentation for |\th@plain| on
% page~\pageref{dok:th@plain}.
%    \begin{macrocode}
%<*thb>
\begingroup \makeatletter
\@ifundefined{theorem@style}{\input{theorem.sty}}{}
\ProvidesFile{thb.sty}
             [\FMithmInfo]
%    \end{macrocode}
%
% \begin{macro}{\th@break}
%    |\theoremstyle{break}| produces a line break after the name
%    of the theorem.  The font is |\slshape|.  Hence, we define
%    |\th@break| as follows:
%    \begin{macrocode}
\gdef\th@break{\normalfont\slshape
  \def\@begintheorem##1##2{\item[%
%    \end{macrocode}
%    We run into the following problem: it is not possible to create
%    the header with |\item[|\meta{title}|]| and then start
%    a new line by, for example, |\mbox{}\\|. Such a definition
%    will fail whenever a list environment follows immediately.  With
%    the above construction, the |\mbox{}| causes the switch
%    |@inlabel| (cf.\ definition of |\list| and
%    |\trivlist| in \cite{latex:source}) to be set to \texttt{false}
%    and so the following list will insert additional vertical
%    space (|\topskip|). This is quite annoying. Therefore, we
%    create the line break within the |\item|. In order to ensure
%    that the text will begin at the proper position in the following
%    line, we simply pretend that the label does not take any
%    room.\footnote{This will lead to problems whenever very high
%    symbols occurring in the line tower into the
%    heading. So, something else has to be done here sometime.}
%    \begin{macrocode}
       \rlap{\vbox{\hbox{\hskip \labelsep\theorem@headerfont ##1\ ##2}%
                   \hbox{\strut}}}]}%
%    \end{macrocode}
%    Again, the definition of |\@opargbegintheorem| is
%    completely analogous.
%    \begin{macrocode}
\def\@opargbegintheorem##1##2##3{%
  \item[\rlap{\vbox{\hbox{\hskip \labelsep \theorem@headerfont
                     ##1\ ##2\ (##3)}%
                    \hbox{\strut}}}]}}
\endgroup
%</thb>
%    \end{macrocode}
% \end{macro}
%
%
% \subsubsection{The \texttt{changebreak} style}
%
%    \begin{macrocode}
% This style option is stored in the file |thcb.sty|.
%    \begin{macrocode}
%<*thcb>
\begingroup \makeatletter
\@ifundefined{theorem@style}{\input{theorem.sty}}{}
\ProvidesFile{thcb.sty}
             [\FMithmInfo]
%    \end{macrocode}
%
% \begin{macro}{\th@changebreak}
%    The \texttt{change-break} theorem style is like \texttt{break}
%    but with interchange of theorem name and theorem number.  Thus we
%    define |\th@changebreak| as follows:
%    \begin{macrocode}
\gdef\th@changebreak{\normalfont\slshape
  \def\@begintheorem##1##2{\item
       [\rlap{\vbox{\hbox{\hskip\labelsep \theorem@headerfont ##2\ ##1}%
                    \hbox{\strut}}}]}%
\def\@opargbegintheorem##1##2##3{%
  \item[\rlap{\vbox{\hbox{\hskip\labelsep \theorem@headerfont
                     ##2\ ##1\ (##3)}%
                    \hbox{\strut}}}]}}
\endgroup
%</thcb>
%    \end{macrocode}
% \end{macro}
%
%
% \subsubsection{The \texttt{change} style}
%
% This style option is stored in the file |thc.sty|.
%    \begin{macrocode}
%<*thc>
\begingroup \makeatletter
\@ifundefined{theorem@style}{\input{theorem.sty}}{}
\ProvidesFile{thc.sty}
             [\FMithmInfo]
%    \end{macrocode}
%
% \begin{macro}{\th@change}
%    The \texttt{change} theorem style corresponds to the
%    \texttt{change break} style without a linebreak after the header.
%    To say it in another way, it's the same as the \texttt{plain}
%    style but with number and name interchanged and |\slshape| as the
%    default font.
%    \begin{macrocode}
\gdef\th@change{\normalfont\slshape
  \def\@begintheorem##1##2{\item
       [\hskip\labelsep \theorem@headerfont ##2\ ##1]}%
\def\@opargbegintheorem##1##2##3{%
  \item[\hskip\labelsep \theorem@headerfont ##2\ ##1\ (##3)]}}
\endgroup
%</thc>
%    \end{macrocode}
% \end{macro}
%
%
% \subsubsection{The \texttt{marginbreak} style}
%
% This style option is the one used most often at Mainz.  It is saved
% in the file |thmb.sty|.
%    \begin{macrocode}
%<*thmb>
\begingroup \makeatletter
\@ifundefined{theorem@style}{\input{theorem.sty}}{}
\ProvidesFile{thmb.sty}
             [\FMithmInfo]
%    \end{macrocode}
%
% \begin{macro}{\th@marginbreak}
%    The \texttt{margin break} style is nearly the same as the
%    \texttt{change break} style. The only difference is the placement
%    of the theorem number.  We use |\llap| to place it in the left
%    margin.
%
%    In this style |\labelsep| denotes the separation between the
%    number and the text.
%    \begin{macrocode}
\gdef\th@marginbreak{\normalfont\slshape
  \def\@begintheorem##1##2{\item
       [\rlap{\vbox{\theorem@headerfont
                    \hbox{\llap{##2}\hskip\labelsep ##1}%
                    \hbox{\strut}}}]}%
\def\@opargbegintheorem##1##2##3{%
  \item[\rlap{\vbox{\theorem@headerfont
                    \hbox{\llap{##2}\hskip\labelsep ##1\ (##3)}%
                    \hbox{\strut}}}]}}
\endgroup
%</thmb>
%    \end{macrocode}
% \changes{v2.1c}{90/02/17}{`endgroup outside macrocode env.}
% \end{macro}
%
%
%
% \subsubsection{The \texttt{margin} style}
%
% This style option is stored in the file |thm.sty|.
%    \begin{macrocode}
%<*thm>
\begingroup \makeatletter
\@ifundefined{theorem@style}{\input{theorem.sty}}{}
\ProvidesFile{thm.sty}
             [\FMithmInfo]
%    \end{macrocode}
%
% \begin{macro}{\th@margin}
%    Again this is only a variant of the theorem styles described
%    above without any new ideas.
%    \begin{macrocode}
\gdef\th@margin{\normalfont\slshape
  \def\@begintheorem##1##2{\item
       [\theorem@headerfont \llap{##2}\hskip\labelsep ##1]}%
\def\@opargbegintheorem##1##2##3{%
  \item[\theorem@headerfont \llap{##2}\hskip\labelsep ##1\ (##3)]}}
\endgroup
%</thm>
%    \end{macrocode}
% \end{macro}
%
%
%
%
% \subsection{Final Definitions}
%
% \begin{macro}{\theorempreskipamount}
% \begin{macro}{\theorempostskipamount}
%    The \textsf{skip} parameters that regulate the vertical empty space
%    before and after the {\envfont theorem} environment have
%    to be allocated as well.
% \changes{v2.0f}{89/07/20}{Unnecessary `global' removed.}
%    \begin{macrocode}
%<*package>
\newskip\theorempreskipamount
\newskip\theorempostskipamount
%    \end{macrocode}
%    Since we have used the same values for all theorem sets, we now can
%    assign them.
%    \begin{macrocode}
\global\setlength\theorempreskipamount{12pt plus 5pt minus 3pt}
\global\setlength\theorempostskipamount{8pt plus 3pt minus 1.5pt}
%    \end{macrocode}
% \end{macro}
% \end{macro}
%
%
% \begin{macro}{\@endtheorem}
%    The same holds for the macro |\@endtheorem|, which ends a
%    {\envfont theorem} environment. Since it is the same for all
%    theorem sets, it is removed from the macros |\th@|\meta{style}.
%    It simply ends the {\envfont trivlist} environment, which was
%    begun in |\@thm|.
%    \begin{macrocode}
\global\let\@endtheorem=\endtrivlist
%    \end{macrocode}
% \end{macro}
%
%
%
% \begin{macro}{\@preamblecmds}
%    All macros defined above are to be used only in the preamble.
%    Therefore, we insert them in |\@preamblecmds| which will disable
%    them at begin document. This is done by the internal \LaTeXe{}
%    command |\@onlypreamble|.
% \changes{v2.0g}{89/09/19}{`theoremheaderfont and `theorembodyfont
%    added.}
% \changes{v2.2a}{94/02/03}{Use \LaTeXe{} interface}
%    \begin{macrocode}
\@onlypreamble\@xnthm
\@onlypreamble\@ynthm
\@onlypreamble\@othm
\@onlypreamble\newtheorem
\@onlypreamble\theoremstyle
\@onlypreamble\theorembodyfont
\@onlypreamble\theoremheaderfont
%    \end{macrocode}
% \end{macro}
%
% Finally we declare the \texttt{plain} theorem style to be the default.
% \changes{v2.1b}{90/06/04}{Fixed missing `item by declaring default
%    style.}
%    \begin{macrocode}
\theoremstyle{plain}
%</package>
%    \end{macrocode}
%
% \Finale

\endinput
}{}
\ProvidesFile{thm.sty}
             [\FMithmInfo]
%    \end{macrocode}
%
% \begin{macro}{\th@margin}
%    Again this is only a variant of the theorem styles described
%    above without any new ideas.
%    \begin{macrocode}
\gdef\th@margin{\normalfont\slshape
  \def\@begintheorem##1##2{\item
       [\theorem@headerfont \llap{##2}\hskip\labelsep ##1]}%
\def\@opargbegintheorem##1##2##3{%
  \item[\theorem@headerfont \llap{##2}\hskip\labelsep ##1\ (##3)]}}
\endgroup
%</thm>
%    \end{macrocode}
% \end{macro}
%
%
%
%
% \subsection{Final Definitions}
%
% \begin{macro}{\theorempreskipamount}
% \begin{macro}{\theorempostskipamount}
%    The \textsf{skip} parameters that regulate the vertical empty space
%    before and after the {\envfont theorem} environment have
%    to be allocated as well.
% \changes{v2.0f}{89/07/20}{Unnecessary `global' removed.}
%    \begin{macrocode}
%<*package>
\newskip\theorempreskipamount
\newskip\theorempostskipamount
%    \end{macrocode}
%    Since we have used the same values for all theorem sets, we now can
%    assign them.
%    \begin{macrocode}
\global\setlength\theorempreskipamount{12pt plus 5pt minus 3pt}
\global\setlength\theorempostskipamount{8pt plus 3pt minus 1.5pt}
%    \end{macrocode}
% \end{macro}
% \end{macro}
%
%
% \begin{macro}{\@endtheorem}
%    The same holds for the macro |\@endtheorem|, which ends a
%    {\envfont theorem} environment. Since it is the same for all
%    theorem sets, it is removed from the macros |\th@|\meta{style}.
%    It simply ends the {\envfont trivlist} environment, which was
%    begun in |\@thm|.
%    \begin{macrocode}
\global\let\@endtheorem=\endtrivlist
%    \end{macrocode}
% \end{macro}
%
%
%
% \begin{macro}{\@preamblecmds}
%    All macros defined above are to be used only in the preamble.
%    Therefore, we insert them in |\@preamblecmds| which will disable
%    them at begin document. This is done by the internal \LaTeXe{}
%    command |\@onlypreamble|.
% \changes{v2.0g}{89/09/19}{`theoremheaderfont and `theorembodyfont
%    added.}
% \changes{v2.2a}{94/02/03}{Use \LaTeXe{} interface}
%    \begin{macrocode}
\@onlypreamble\@xnthm
\@onlypreamble\@ynthm
\@onlypreamble\@othm
\@onlypreamble\newtheorem
\@onlypreamble\theoremstyle
\@onlypreamble\theorembodyfont
\@onlypreamble\theoremheaderfont
%    \end{macrocode}
% \end{macro}
%
% Finally we declare the \texttt{plain} theorem style to be the default.
% \changes{v2.1b}{90/06/04}{Fixed missing `item by declaring default
%    style.}
%    \begin{macrocode}
\theoremstyle{plain}
%</package>
%    \end{macrocode}
%
% \Finale

\endinput
}{}
\ProvidesFile{thcb.sty}
             [\FMithmInfo]
%    \end{macrocode}
%
% \begin{macro}{\th@changebreak}
%    The \texttt{change-break} theorem style is like \texttt{break}
%    but with interchange of theorem name and theorem number.  Thus we
%    define |\th@changebreak| as follows:
%    \begin{macrocode}
\gdef\th@changebreak{\normalfont\slshape
  \def\@begintheorem##1##2{\item
       [\rlap{\vbox{\hbox{\hskip\labelsep \theorem@headerfont ##2\ ##1}%
                    \hbox{\strut}}}]}%
\def\@opargbegintheorem##1##2##3{%
  \item[\rlap{\vbox{\hbox{\hskip\labelsep \theorem@headerfont
                     ##2\ ##1\ (##3)}%
                    \hbox{\strut}}}]}}
\endgroup
%</thcb>
%    \end{macrocode}
% \end{macro}
%
%
% \subsubsection{The \texttt{change} style}
%
% This style option is stored in the file |thc.sty|.
%    \begin{macrocode}
%<*thc>
\begingroup \makeatletter
\@ifundefined{theorem@style}{% \iffalse meta-comment
%
% Copyright (C) 1993-2024
%
% The LaTeX Project and any individual authors listed elsewhere
% in this file.
%
% This file is part of the Standard LaTeX `Tools Bundle'.
% -------------------------------------------------------
%
% It may be distributed and/or modified under the
% conditions of the LaTeX Project Public License, either version 1.3c
% of this license or (at your option) any later version.
% The latest version of this license is in
%    https://www.latex-project.org/lppl.txt
% and version 1.3c or later is part of all distributions of LaTeX
% version 2005/12/01 or later.
%
% The list of all files belonging to the LaTeX `Tools Bundle' is
% given in the file `manifest.txt'.
%
% \fi
%
%%
%
%
\def\FMithmInfo{2023/07/05 v2.2c Theorem extension package (FMi)}
%
% \ProvidesFile{theorem.dtx}[\FMithmInfo]
%\iffalse   % this is a METACOMMENT !
%%
%% Package `theorem' to use with LaTeX2e
%% Copyright (C) 1989-1995 Frank Mittelbach, all rights reserved.
%<+package>\@ifundefined{theorem@style}{}{\endinput}
%<+package>\NeedsTeXFormat{LaTeX2e}[1995/06/01]
%<+package>\ProvidesPackage{theorem}
%<+package>         [\FMithmInfo]
%
% \fi
%
% \changes{v1.0}{87/05/07}{First draft.}
% \changes{v1.0a}{87/05/07}{`thistheoremstyle' added.}
% \changes{v1.1a}{87/07/29}{New concept: redefine the internal
%                          \protect\LaTeX{} commands `@xnthm' etc.}
% \changes{v1.1b}{87/07/31}{`theoremfont' used to specify font.}
% \changes{v1.2a}{88/01/08}{`@endtheorem' now globally defined.}
% \changes{v1.2b}{88/01/11}{`newtheorem' only allowed in the preamble.}
% \changes{v1.3a}{88/02/14}{`theoremfont' now a toks register.}
%
% \changes{v2.0a}{88/11/16}{New implementation and new documentation.}
% \changes{v2.0a}{88/11/16}{No layout changes but internal modifications
%                           old substyles won't work.}
% \changes{v2.0b}{89/01/30}{`global' used for all definitions.}
% \changes{v2.0c}{89/06/15}{Documentation translated to English.
%                            Thanks to LLL and Nina}
% \changes{v2.1a}{90/02/15}{Support for new font selection.}
% \changes{v2.2a}{94/02/03}{Cleanup for LaTeX2e}
%
% \DoNotIndex{\@endparenv,\@ifdefinable,\@ifnextchar,\@ifundefined}
% \DoNotIndex{\@namedef,\@nameuse,\@spaces,\@tempa,\@warning}
% \DoNotIndex{\ }
% \DoNotIndex{\begingroup,\bfseries}
% \DoNotIndex{\csname}
% \DoNotIndex{\def,\do,\docdate}
% \DoNotIndex{\endcsname,\endgroup,\endinput,\endtrivlist,\expandafter}
% \DoNotIndex{\filedate,\fileversion}
% \DoNotIndex{\gdef,\global}
% \DoNotIndex{\hbox,\hskip}
% \DoNotIndex{\ignorespaces,\input,\itshape,\item}
% \DoNotIndex{\labelsep,\let,\llap}
% \DoNotIndex{\makeatletter}
% \DoNotIndex{\noexpand}
% \DoNotIndex{\refstepcounter,\relax,\rlap}
% \DoNotIndex{\setlength,\slshape,\space,\string,\strut}
% \DoNotIndex{\the,\trivlist,\typeout}
% \DoNotIndex{\vbox}
% \DoNotIndex{\xdef}
%
% \def\deflabel#1{\ttfamily #1\hfill}
% \def\deflist#1{\begin{list}{}{\settowidth\labelwidth{\ttfamily #1}%
%                               \setlength\leftmargin\labelwidth
%                               \addtolength\leftmargin\labelsep
%                               \let\makelabel\deflabel}}
% \def\enddeflist{\end{list}}
% \def\envfont{\normalfont\ttfamily}
%
% \MakeShortVerb{\|}
%
% \newcommand{\TUB}{{\sl TUGboat\/}}
%
% \GetFileInfo{theorem.dtx}
% \title{An Extension of the
%        \LaTeX{} theorem environment\thanks{This file
%        has version number \fileversion, last
%        revised \filedate.}}
% \author{Frank Mittelbach}
%
% \MaintainedByLaTeXTeam{tools}
% \maketitle
%
% \begin{abstract}
%    The macros described in this paper yield an extension of the
%    \LaTeX{} theorem mechanism. It is designed to satisfy the
%    different requirements of various journals. Thus, the
%    layout of the ``theorems'' can be manipulated by determining a
%    ``style''. This article describes not only the use, but
%    also the definition, of the necessary macros.
% \end{abstract}
%
% \section*{Preface to version 2.2}
%
% For \LaTeXe{} this package did not need any fundamental changes. I
% only modified the messages generated so that theorem layout styles
% will show up with the |\listfiles| command and cleaned the section
% on the New Font Selection Scheme since this is now included in
% \LaTeX.
%
% \section*{Preface to version 2.1}
%
% This version is identical to 2.0g described in \TUB\ 10\#3 except
% for some internal defaults which are now set depending on the used
% font selection scheme.
%
% This was done to avoid unpleasant surprises if the new font selection
% scheme is in force. For further details see
% section~\ref{sec:newfonts} and \cite{bk:GMS94}.
%
% \section{Introduction}
%
% For our purposes here, ``theorems'' are labelled enunciations,
% often set off from the main text by extra space and a font change.
% Theorems, corollaries, conjectures, definitions, and remarks
% are all instances of ``theorems''.  The ``header'' of these structures
% is composed of a label (such as \textsc{Theorem} or \textsc{Remark})
% and a number which serializes an item in the sequence of items
% with the same label.
%
% Shortly after the introduction of \LaTeX{} at the Fachbereich
% Mathematik in Mainz, the desire to manipulate the layout of
% ``theorems'' arose.  In Mainz, the following two conventions came
% into general use:
% \begin{enumerate}
%   \item The number of the theorem is shown in the margin.
%   \item There is a line break at
%         the end of the theorem header.
% \end{enumerate}
%
% Additionally, some journals require different formats which depend on
% the ``sort of theorem'': e.g.\ often remarks and definitions are set
% in |\upshape|, while |\itshape| is employed for main theorems.
%
% Confronted with these requirements, a theorem environment was
% developed in Mainz which allows separate determination of the
% layout of the ``theorems sets'', comparable to |\pagestyle|.
%
%
% \section{The user interface}
% \subsection{Defining new theorem sets}
%
% \DescribeMacro\newtheorem
% As in the original \LaTeX{} version, the command |\newtheorem|
% defines a new ``theorem set'' or ``theorem-like structure''.
% Two required arguments name the new environment and give the text
% to be typeset with each instance of the new ``set'', while
% an optional argument determines how
% the ``set'' is enumerated:
% \begin{description}
%    \item[\ttfamily \bslash newtheorem\{foo\}\{bar\}]
%       The theorem set {\envfont foo} (whose name is \texttt{bar})
%       uses its own counter.
%    \item[\ttfamily \bslash newtheorem\{foo2\}{[foo]}\{bar2\}]
%       The theorem set {\envfont foo2} (printed name \texttt{bar2})
%       uses the same counter as the theorem set \texttt{foo}.
%    \item[\ttfamily \bslash newtheorem\{foo3\}\{bar3\}{[section]}]
%       The theorem set {\envfont foo3} (printed name \texttt{bar3}) is
%       enumerated within the counter \texttt{section}, i.e.\ with every
%       new |\section| the enumeration begins again with 1, and
%       the enumeration is composed from the section-number and the
%       theorem counter itself.
% \end{description}
%
% \DescribeMacro\theoremstyle
% Additionally, the command |\theoremstyle| can define the layout
% of various, or all, theorem sets.  It should be noted that any
% theorem set defined by |\newtheorem| is typeset in the
% |\theoremstyle| that is current at the time of the definition.
% Thus, the following
% \begin{verbatim}
%    \theoremstyle{break}        \newtheorem{Cor}{Corollary}
%    \theoremstyle{plain}        \newtheorem{Exa}{Example}[section]
%\end{verbatim}                             ^^A No blank at line start
% leads to the result that the set {\envfont Cor} is formatted in the
% style \texttt{break}, while the set {\envfont Exa} and all the
% following ones are formatted in the style \texttt{plain}, unless
% another |\theoremstyle| follows.  Since the definitions installed by
% |\newtheorem| are global, one also can limit |\theoremstyle| locally
% by grouping braces.
%
% \DescribeMacro\theorembodyfont
% The choice of the font for the theorem body is completely
% independent of the chosen |\theoremstyle|; this has proven to
% be very advantageous.  For example,
% \begin{verbatim}
%   {\theorembodyfont{\upshape}          \newtheorem{Rem}{Remark}}
%\end{verbatim}
% defines a theorem set {\envfont Rem}, which will be set in
% |\upshape| in the current layout (which in our example is
% \texttt{plain}).  As with |\theoremstyle|, the |\theorembodyfont|
% chosen is that current at the time of |\newtheorem|.  If
% |\theorembodyfont| is not specified or one defines
% |\theorembodyfont{}|, then the font used will be that defined by the
% |\theoremstyle|.
%
% \DescribeMacro\theoremheaderfont It is also possible to customize
% the font used for the theorem headers.  This is, however, a global
% declaration, and therefore there should be at most one
% |\theoremheaderfont| declaration in the preamble.\footnote{If it is
% actually necessary to have different header fonts, one has to define
% new theorem styles (substituting the desired font) or specify the
% information directly in the \texttt{\bslash newtheorem} declaration
% (the unclean variant).}
%
% \DescribeMacro\theorempreskipamount
% \DescribeMacro\theorempostskipamount Two additional parameters
% affect the vertical space around the theorem environments:
% |\theorempreskipamount| and |\theorempostskipamount| define,
% respectively, the spacing before and after such an environment.
% These parameters apply for all theorem sets and can be manipulated
% with the ordinary length macros.  They are rubber lengths
% (`\textsf{skips}'), and therefore can contain \texttt{plus} and
% \texttt{minus} parts.
%
% Since the definition of theorem sets should---most sensibly---be
% placed in the preamble, we only allow installation there.
% It is therefore possible to release the memory used here after
% |\begin{document}|, in order to make room for other
% applications.
%
%
% \subsection{Existing theorem styles}
%
% The following theorem styles exist to date:
% \begin{deflist}{marginbreak}
%    \item[plain]
%       This theorem style emulates the original \LaTeX{} definition,
%       except that additionally the parameters
%       |\theorem...skipamount| are used.
%    \item[break]
%       In this style, the theorem header is followed by a line
%       break.
%    \item[marginbreak]
%       The theorem number is set in the margin, and there is a line
%       break as in \texttt{break}.
%    \item[changebreak]
%       Like \texttt{break}, but with header number and text
%       interchanged.
%    \item[change]
%       Header number and text are interchanged, without a line break.
%    \item[margin]
%       The number is set in the left margin, without a line break.
% \end{deflist}
% All styles (except \texttt{plain}) select |\slshape| as the default
% |\theorembodyfont|.
%
%
% \subsection{Examples}
%
% Given the above theorem sets {\envfont Cor}, {\envfont Exa}
% and {\envfont Rem},
% suppose that the preamble also contains the declarations:
% \begin{verbatim}
%    \theoremstyle{marginbreak}   \newtheorem{Lem}[Cor]{Lemma}
%    \theoremstyle{change}
%    \theorembodyfont{\itshape}        \newtheorem{Def}[Cor]{Definition}
%
%    \theoremheaderfont{\scshape}
%\end{verbatim}
% Then the following are some typical examples of the typeset output
% resulting from their use.
%
% \begin{Cor}
%    This is a sentence typeset in the theorem environment {\envfont
%    Cor}.
% \end{Cor}
%
% \begin{Exa}
%    This is a sentence typeset in the theorem environment {\envfont
%    Exa}.
% \end{Exa}
%
% \begin{Rem}
%    This is a sentence typeset in the theorem environment {\envfont
%    Rem}.
% \end{Rem}
%
% \begin{Lem}[Ben User]
%    This is a sentence typeset in the theorem environment {\envfont
%    Lem}.
% \end{Lem}
%
% \begin{Def}[Very impressive Definition]
%    This is a sentence typeset in the theorem environment {\envfont
%    Def}.
% \end{Def}
%
% The last two examples show the effect of the optional argument to
% a theorem environment (it is the text typeset in parentheses).
%
%
% \section{Special Considerations}
% \label{sec:newfonts}
%
% Theoremheader and body are implemented as a unit. This means that
% the |\theoremheaderfont|  will inherit characteristics of the
% |\theorembodyfont| in \LaTeXe. Thus, if for example
% |\theorembodyfont| is |\itshape| and |\theoremheaderfont|
% is |\bfseries| the font selected for the header will have the
% characteristics `bold extended italic'. If this is not desired one
% should set the |\theoremheaderfont| to something like
% \begin{verbatim}
%      \theoremheaderfont{\normalfont\bfseries}
%\end{verbatim}
% i.e.\ supplying all necessary font information explicitly.
%
%
%
% \section{Acknowledgements}
%
% The publication of this set of macros was only possible with the
% help of Christina Busse (translating the manuscript into English),
% Joachim Pense (playing the r\^ole of typist), Chris Rowley
% (looking everything over) and many others providing useful
% suggestions.
%
% \MaybeStop{%
%    \begin{thebibliography}{9}
%       \bibitem{bk:GMS94} \textsc{M.~Goossens}, \textsc{F.~Mittelbach}
%         and \textsc{A.~Samarin}.
%         \newblock The \LaTeX{} Companion.
%         \newblock
%         Addison-Wesley, Reading, Massachusetts, 1994.
%       \bibitem{latex:source} \textsc{Lamport, Leslie.} \newblock
%                               \texttt{latex.tex}, version 2.09, date
%                               Feb. 1990.
%    \end{thebibliography}
%    \PrintIndex
%    \PrintChanges}
%
%
% \section{The documentation driver file}
%
% The next bit of code contains the documentation driver file for
% \TeX{}, i.e., the file that will produce the documentation you are
% currently reading. It will be extracted from this file by the
% \texttt{docstrip} program. Since it is the first code in the file
% one can alternatively process this file directly with \LaTeXe{} to
% obtain the documentation.
%
%    \begin{macrocode}
%<*driver>
\documentclass{ltxdoc}

\usepackage{theorem}

%<+driver>% The next few lines define theorem sets which are used
%<+driver>% in the example section of the documentation.

\theoremstyle{break}       \newtheorem{Cor}{Corollary}
\theoremstyle{plain}       \newtheorem{Exa}{Example}[section]
{\theorembodyfont{\upshape}\newtheorem{Rem}{Remark}}
\theoremstyle{marginbreak} \newtheorem{Lem}[Cor]{Lemma}
\theoremstyle{change}
\theorembodyfont{\itshape} \newtheorem{Def}[Cor]{Definition}

\theoremheaderfont{\scshape}

\RecordChanges

\begin{document}
   \DocInput{theorem.dtx}
\end{document}
%</driver>
%    \end{macrocode}
%
% \section{Definition of the Macros}
%
% If the file has been loaded before, we abort immediately. If not the
% package announces itself (this is actually done at the very top of
% the file---the way it is done isn't good style so don't copy it).
% \changes{v2.0e}{89/07/19}{Spaces removed from `typeout'.}
% \changes{v2.2b}{95/11/19}{Announce moved to top of file}
%    \begin{macrocode}
%<*package>
%\@ifundefined{theorem@style}{}{\endinput}
%\def\FMithmInfo{1995/11/19 v2.2b Theorem extension package (FMi)}
%\ProvidesPackage{theorem}[\FMithmInfo]
%    \end{macrocode}
%
%
% \subsection{Definition of theorem styles and fonts}
%
% All the definitions in this file are done globally to allow
% inputting this file inside a group.
%
% \begin{macro}{\theoremstyle}
%    Before a theorem style can be installed, the chosen style must be
%    known. For that reason, we must test to see that
%    |\th@|\meta{style} is known or, more precisely, that it is
%    different from |\relax|. If the style is not known then
%    |\th@plain| is used.
%    \begin{macrocode}
\gdef\theoremstyle#1{%
   \@ifundefined{th@#1}{\@warning
          {Unknown theoremstyle `#1'. Using `plain'}%
          \theorem@style{plain}}%
%    \end{macrocode}
%    We save the theorem style to be used in the token register
%    |\theorem@style|.
%    \begin{macrocode}
      {\theorem@style{#1}}%
%    \end{macrocode}
%    Now we ``evaluate'' the theorem style: this means, we call the
%    macro |\th@|\meta{style} which will activate the relevant
%    definitions which are contained in a separate file.  This is done
%    in a group to suppress changes to the current font.  This could
%    otherwise pose problems together with the new font selection
%    scheme\footnote{When I printed the original article using the new
%    font selection scheme I ended with a document with slanted
%    typefaces (text headings and all) simply because one of the
%    theorem styles used \texttt{\bslash sl} at toplevel.} if the
%    |\th@|\meta{style} is evaluated a second time.
% \changes{v2.1a}{90/02/15}{Group added around `th@..}
% \changes{v2.1a}{90/02/15}{Reseting of `..@begintheorem removed.}
%    \begin{macrocode}
      \begingroup
        \csname th@\the\theorem@style \endcsname
      \endgroup}
%    \end{macrocode}
% \end{macro}
%
%
% \begin{macro}{\@begintheorem}
% \changes{v2.1a}{90/02/15}{Relaxed at toplevel.}
% \begin{macro}{\@opargbegintheorem}
% \changes{v2.1a}{90/02/15}{Relaxed at toplevel.}
%    We reset |\@begintheorem| and |\@opargbegintheorem| to
%    |\relax| since these commands are no longer necessary at
%    toplevel. This will save a few tokens.
%    \begin{macrocode}
\global\let\@begintheorem\relax
\global\let\@opargbegintheorem\relax
%    \end{macrocode}
% \end{macro}
% \end{macro}
%
% \begin{macro}{\theorem@style}
%    Obviously the \textsf{token} register used above has to be
%    allocated. To assure the utmost compatibility with the
%    original \LaTeX{} definition, we set the default theorem style to
%    \texttt{plain}, which implements the usual \LaTeX{} convention.
% \changes{v2.0f}{89/07/20}{Unnecessary `global' removed.}
%    \begin{macrocode}
\newtoks\theorem@style
\global\theorem@style{plain}
%    \end{macrocode}
% \end{macro}
%
% \begin{macro}{\theorembodyfont}
% \begin{macro}{\theorem@bodyfont}
%    If the |\theorembodyfont| is set by the user then it should not
%    interact with the default font set in the theorem style. When the
%    new font selection is in force this may happen if, for example,
%    the default is |\itshape| and the new |\theorembodyfont| is
%    |\sffamily|. So we add a |\reset@font| command in front of the
%    user definition.
% \changes{v2.0f}{89/07/20}{Unnecessary `global' removed.}
% \changes{v2.0g}{89/09/19}{`theoremfont to `theorembodyfont.}
% \changes{v2.1c}{92/06/22}{`reset@font added and logic changed.}
% \changes{v2.1d}{92/10/28}{Reset to empty if arg is empty}
%    \begin{macrocode}
\gdef\theorembodyfont#1{%
%    \end{macrocode}
%    We check if the argument supplied is empty and if so put nothing
%    into the |\theorem@bodyfont| token register to allow for
%    |\theorembodyfont{}| as a mean of using the default of the
%    current |\theoremstyle|.
%    \begin{macrocode}
   \def\@tempa{#1}%
   \ifx\@tempa\@empty
    \theorem@bodyfont{}%
   \else
    \theorem@bodyfont{\reset@font#1}%
   \fi
   }
\newtoks\theorem@bodyfont
\global\theorem@bodyfont{}
%    \end{macrocode}
% \end{macro}
% \end{macro}
%
%
% \begin{macro}{\theoremheaderfont}
% \changes{v2.0g}{89/09/19}{Macro added.}
%    The font for the theorem headers is handled differently because
%    this definition applies to all theorem styles.
%    \begin{macrocode}
\gdef\theoremheaderfont#1{\gdef\theorem@headerfont{#1}%
%    \end{macrocode}
%    After using the macro once it is redefined to produce an error
%    message.
%    \begin{macrocode}
       \gdef\theoremheaderfont##1{%
        \typeout{\string\theoremheaderfont\space should be used
                 only once.}}}
%    \end{macrocode}
% \end{macro}
%
% \begin{macro}{\theorem@headerfont}
% \changes{v2.0g}{89/09/19}{Macro added.}
%    To set the |\theorem@headerfont| default we first test if
%    the new fontselection scheme is in force.
% \changes{v2.1a}{90/02/15}{Support for new font selection.}
%    \begin{macrocode}
\ifx\upshape\undefined
%    \end{macrocode}
%    If not we define it to expand into |\bfseries|. We don't use
%    |\let| just in case a following style option redefines this
%    macro.
%    \begin{macrocode}
\gdef\theorem@headerfont{\bfseries}
%    \end{macrocode}
%    Otherwise we reset the current shape before calling |\bfseries|.
%    \begin{macrocode}
\else \gdef\theorem@headerfont{\normalfont\bfseries}\fi
%    \end{macrocode}
% \end{macro}
%
% \begin{macro}{\th@plain}
% \begin{macro}{\th@break}
% \begin{macro}{\th@marginbreak}
% \begin{macro}{\th@changebreak}
% \begin{macro}{\th@change}
% \begin{macro}{\th@margin}
%    The different styles are defined in macros such as
%    |\th@plain|.  Since memory space is precious in
%    ``non-Big-versions'', we have to avoid offering too many unused
%    definitions.  Therefore we define these styles in separate files
%    that can be loaded on demand.  Thus the commands themselves only
%    load these files. We use |\@input@|, a \LaTeXe{} internal command
%    that ensures that the file will be listed with |\listfiles|.
% \changes{v2.2a}{94/02/02}{Use `@input@ to load theorem layout files}
%    \begin{macrocode}
\gdef\th@plain{\@input@{thp.sty}}
\gdef\th@break{\@input@{thb.sty}}
\gdef\th@marginbreak{\@input@{thmb.sty}}
\gdef\th@changebreak{\@input@{thcb.sty}}
\gdef\th@change{\@input@{thc.sty}}
\gdef\th@margin{\@input@{thm.sty}}
%    \end{macrocode}
%    This list will be expanded when new styles become
%    available.  For testing,  just append new theorem substyles
%    as document options.
% \end{macro}
% \end{macro}
% \end{macro}
% \end{macro}
% \end{macro}
% \end{macro}
%
%
%
% \subsection{Definition of a new theorem set}
%
% As already pointed out, a new theorem environment can be defined in
% three different ways:
% \begin{verbatim}
%    \newtheorem{Lem}{Lemma}
%    \newtheorem{Lem}{Lemma}[section]
%    \newtheorem{Lem}[Theorem]{Lemma}
%\end{verbatim}
% The function of the macro |\newtheorem| is to recognize these cases
% and then to branch into one of the three macros |\@ynthm|, |\@xnthm|
% or |\@othm|.  This mechanism is adopted unchanged from
% \cite{latex:source}; the essential point here is that, for example,
% in the second case, the arguments \texttt{Lem}, \texttt{Lemma} and
% \texttt{section} are passed over to the macro |\@xnthm|.
%
% We inspect this case first because the others present fewer problems,
% and thus are easily derived from this one.
%
% \begin{macro}{\@xnthm}
%    For our example arguments, the macro |\@xnthm| must fulfill
%    the following:
%    \begin{itemize}
%       \item
%          Define a new \LaTeX-counter `Lem'
%       \item
%          reset this counter within a |\section|
%       \item
%          define the macro |\theLem|
%       \item
%          define the environment macros |\Lem| and
%          |\endLem| using the current
%          |\theoremstyle| and |\theorem@bodyfont|.
%    \end{itemize}
%    Obviously, all this should happen only if the first argument
%    of |\@xnthm| (i.e.\ \texttt{Lem} in our example) is chosen so
%    as not to conflict with any previously defined commands or
%    environments.  This test is performed by the \LaTeX{} macro
%    |\@ifdefinable|.
%    \begin{macrocode}
\gdef\@xnthm#1#2[#3]{\expandafter\@ifdefinable\csname #1\endcsname
%    \end{macrocode}
%    Therefore, the first argument of |\@ifdefinable| is the
%    expansion (in the example, |\Lem|) of
%    |\csname||#1\endcsname|.  The second argument is
%    executed only if the test has been completed successfully.
%    \begin{macrocode}
   {%
%    \end{macrocode}
%    Now we define the new counter. The names of the \LaTeX{}
%    macros employed should speak for themselves:
% \changes{v2.2b}{1995/10/07}{Use \cs{@newctr}}
% \changes{v2.2c}{1995/11/23}{Use \cs{@newctr} properly.}
%    \begin{macrocode}
    \@definecounter{#1}\@newctr{#1}[#3]%
%    \end{macrocode}
%    Using |\@newctr| will give a proper error message if the counter
%    in |#3| is not defined.
%    In defining `|\theLem|' we must generate the desired macro
%    name by use of |\expandafter| and |\csname|.
%    \begin{macrocode}
    \expandafter\xdef\csname the#1\endcsname
%    \end{macrocode}
%    An |\xdef| is used in order to make the definition global,
%    and to ensure that it contains the replacement texts of
%    |\@thmcountersep| and |\@thmcounter|.\footnote{These
%    two macros can be defined by the document style.  Their default
%    values produce a `\texttt{.}' as separation and an arabic
%    representation of the number.} However, not everything should be
%    expanded. For example, it saves space to use |\thesection|
%    instead of its---at times---lengthy expansion.
%    \begin{macrocode}
      {\expandafter \noexpand \csname the#3\endcsname
       \@thmcountersep \@thmcounter{#1}}%
%    \end{macrocode}
%    Thus with the defaults of \LaTeX{}, |\theLem| would be
%    replaced by the command sequence |\thesection.\arabic{Lem}|.
%
%    We will now look at the definition of the macro which is executed
%    at the beginning of the actual environment (in our example this
%    macro is |\Lem|). It should be noted that we use an
%    ``|\expandafter| trick'' to expand only certain parts of the
%    replacement text at the time of the definition.
%    \begin{macrocode}
    \def\@tempa{\global\@namedef{#1}}%
    \expandafter \@tempa \expandafter{%
%    \end{macrocode}
%    First, the macro that contains the current definitions of
%    |\@begintheorem| and |\@opargtheorem| should be called
%    up.  The name of this macro---as is already known---has the form
%    |\th@|\meta{theorem\ style}; therefore, it must be called by
%    \begin{macrocode}
      \csname th@\the \theorem@style
%    \end{macrocode}
%    In addition the default theorem font should be changeable, i.e.\
%    we have to insert the contents of |\theorem@bodyfont|. For
%    that reason, we expand even further, beyond |\endcsname|,
%    and thus insert the contents of the \textsf{token} register
%    |\theorem@bodyfont| in the replacement text.
%    \begin{macrocode}
            \expandafter \endcsname \the \theorem@bodyfont
%    \end{macrocode}
%    Now it is time to call the macro |\@thm| which takes over
%    the further processing.  It has two arguments: the current
%    counter name (in our example, \texttt{Lem}), and the text of the
%    label (in our example, \texttt{Lemma}).
%    \begin{macrocode}
     \@thm{#1}{#2}}%
%    \end{macrocode}
%    With this, the `sub-definition' is complete.  The macro
%    |\@endtheorem| ends a theorem environment and is, so far, nothing
%    but an |\endtrivlist|.  (Hence it is defined globally, and not
%    within the theorem styles.\footnote{This has to be changed as
%    soon as theorem styles that change \texttt{\bslash @endtheorem}
%    exist.  In such a case, all existing styles must be changed as
%    well since they will have to reset the macro.}) Therefore, we can
%    set it equivalent to the macro that ends the theorem set (in our
%    example, |\endLem|).  However, if some day theorem styles exist
%    that do change |\@endtheorem|, we would have to use the
%    commented-out line instead.
%    \begin{macrocode}
    \global \expandafter \let \csname end#1\endcsname \@endtheorem
%   \global\@namedef{end#1}{\@endtheorem}%
%    \end{macrocode}
%    With these commands all the required definitions are employed,
%    unless the test |\@ifdefinable| has failed.  Therefore, we
%    end the second argument of this macro and with it the definition
%    of |\@xnthm|.
%    \begin{macrocode}
   }}
%    \end{macrocode}
% \end{macro}
%
% \begin{macro}{\@ynthm}
%    The definition of |\@ynthm| is completely analogous. In
%    this case the new counter that is defined is not reset within
%    another counter; thus the definition of |\the...|
%    is simplified:
%    \begin{macrocode}
\gdef\@ynthm#1#2{\expandafter\@ifdefinable\csname #1\endcsname
   {\@definecounter{#1}%
    \expandafter\xdef\csname the#1\endcsname{\@thmcounter{#1}}%
%    \end{macrocode}
%    The rest of the definition corresponds literally to that of
%    |\@xnthm|:
%    \begin{macrocode}
    \def\@tempa{\global\@namedef{#1}}\expandafter \@tempa
     \expandafter{\csname th@\the \theorem@style \expandafter
     \endcsname \the\theorem@bodyfont \@thm{#1}{#2}}%
    \global \expandafter \let \csname end#1\endcsname \@endtheorem}}
%    \end{macrocode}
% \end{macro}
%
% \begin{macro}{\@othm}
%    The definition of |\@othm| does not contain anything new.
%    \begin{macrocode}
\gdef\@othm#1[#2]#3{%
%    \end{macrocode}
% \changes{v2.2b}{95/10/07}{Build in check for counter defined}
%    We do not define a new counter but instead use one that has
%    already been defined.  Thus the only definition we need is that
%    of this pseudo-counter (i.e.\ |\the|\meta{env.\ name}).
%    First we check if |#2| corresponds to a known counter name.
%    \begin{macrocode}
  \expandafter\ifx\csname c@#2\endcsname\relax
   \@nocounterr{#2}%
  \else
   \expandafter\@ifdefinable\csname #1\endcsname
   {\expandafter \xdef \csname the#1\endcsname
     {\expandafter \noexpand \csname the#2\endcsname}%
%    \end{macrocode}
%    All other parts of the definition can be adopted from
%    |\@xnthm|. We have to remember, though, that in this case
%    the name of the current counter and the theorem label have moved
%    to the second and third arguments.
%    \begin{macrocode}
    \def\@tempa{\global\@namedef{#1}}\expandafter \@tempa
     \expandafter{\csname th@\the \theorem@style \expandafter
     \endcsname \the\theorem@bodyfont \@thm{#2}{#3}}%
    \global \expandafter \let \csname end#1\endcsname \@endtheorem}%
  \fi}
%    \end{macrocode}
% \end{macro}
%
% \subsection{Macros that are employed in a {\envfont theorem}
%    environment}
%
% \begin{macro}{\@thm}
%    The macro |\@thm| has to increase the current counter.  Then,
%    depending on whether the environment has (or does not have) an
%    optional argument, it has to branch into either |\@begintheorem|
%    or |\@opargtheorem|.
%    \begin{macrocode}
\gdef\@thm#1#2{\refstepcounter{#1}%
%    \end{macrocode}
%    Now we start a {\envfont trivlist} environment, and give
%    |\@topsep| and |\@topsepadd| the values of the \textsf{skip}
%    registers |\theorempreskipamount| and |\theorempostskipamount|.
%    The value in |\@topsep| is the vertical space that is inserted by
%    the first (and only) |\item| in our |\trivlist| whilst
%    |\@topsepadd| is inserted by |\@endparenv| at the end of that
%    {\envfont trivlist} environment.  By using these registers, we
%    obtain the desired space around a {\envfont theorem} environment.
%    \begin{macrocode}
   \trivlist
   \@topsep \theorempreskipamount               % used by first \item
   \@topsepadd \theorempostskipamount           % used by \@endparenv
%    \end{macrocode}
%    Now we have to test whether an optional argument has been given.
% \changes{v2.0d}{89/07/18}{Forgotten percent added.}
%    \begin{macrocode}
   \@ifnextchar [%
%    \end{macrocode}
%    If there is an optional argument, we will call |\@ythm|, and
%    move the arguments read back into the input stream.
%    \begin{macrocode}
   {\@ythm{#1}{#2}}%
%    \end{macrocode}
%    If not, we call |\@begintheorem|. Its first argument is the
%    name of the theorem set (hence the second argument of
%    |\@thm|). Its second argument is the macro that produces the
%    current number.
%    \begin{macrocode}
   {\@begintheorem{#2}{\csname the#1\endcsname}\ignorespaces}}
%    \end{macrocode}
% \end{macro}
%
% \begin{macro}{\@xthm}
% \begin{macro}{\@ythm}
%    Both these macros were originally called by |\@thm|.  We do
%    not need |\@xthm| anymore, hence we reset it to
%    |\relax|. The definition of |\@ythm| has not changed at
%    all from its definition in \LaTeX. In order to make the macros
%    easier to understand, we will
%    nevertheless present it (commented out).
%    \begin{macrocode}
\global\let\@xthm\relax
% \def\@ythm#1#2[#3]{\@opargbegintheorem{#2}{\csname
%           the#1\endcsname}{#3}\ignorespaces}
%</package>
%    \end{macrocode}
%    The primitive |\ignorespaces| in |\@ythm| and
%    |\@thm| is needed to remove the spaces between the
%    |\begin{...}| and the actual text.
% \end{macro}
% \end{macro}
%
%
% \subsection{Definition of the theorem substyles}
%
% As already pointed out, the theorem substyles, defined below, are
% only loaded when necessary.  Note that all these substyles, except
% \texttt{plain}, have |\slshape| as the default body font.
%
% \subsubsection{The \texttt{plain} style}
%
% \label{dok:th@plain} As the following macros use |@|, we have
% to locally set the |\catcode| of this symbol to ``letter''.
% This happens within a group, so that we do not have to worry about
% which |\catcode| that symbol had before.
%    \begin{macrocode}
%<*thp>
\begingroup \makeatletter
%    \end{macrocode}
% Since we are now within a group, we must make all definitions
% globally.  First we make sure that \texttt{theorem.sty} is loaded.
% This will allow us to use this file as a document style option
% without having to call \texttt{theorem} itself as an option.  At the
% same time, we assure that at least version 2 is loaded, since
% |\theorem@style| was not defined in earlier versions.
%    \begin{macrocode}
\@ifundefined{theorem@style}{% \iffalse meta-comment
%
% Copyright (C) 1993-2024
%
% The LaTeX Project and any individual authors listed elsewhere
% in this file.
%
% This file is part of the Standard LaTeX `Tools Bundle'.
% -------------------------------------------------------
%
% It may be distributed and/or modified under the
% conditions of the LaTeX Project Public License, either version 1.3c
% of this license or (at your option) any later version.
% The latest version of this license is in
%    https://www.latex-project.org/lppl.txt
% and version 1.3c or later is part of all distributions of LaTeX
% version 2005/12/01 or later.
%
% The list of all files belonging to the LaTeX `Tools Bundle' is
% given in the file `manifest.txt'.
%
% \fi
%
%%
%
%
\def\FMithmInfo{2023/07/05 v2.2c Theorem extension package (FMi)}
%
% \ProvidesFile{theorem.dtx}[\FMithmInfo]
%\iffalse   % this is a METACOMMENT !
%%
%% Package `theorem' to use with LaTeX2e
%% Copyright (C) 1989-1995 Frank Mittelbach, all rights reserved.
%<+package>\@ifundefined{theorem@style}{}{\endinput}
%<+package>\NeedsTeXFormat{LaTeX2e}[1995/06/01]
%<+package>\ProvidesPackage{theorem}
%<+package>         [\FMithmInfo]
%
% \fi
%
% \changes{v1.0}{87/05/07}{First draft.}
% \changes{v1.0a}{87/05/07}{`thistheoremstyle' added.}
% \changes{v1.1a}{87/07/29}{New concept: redefine the internal
%                          \protect\LaTeX{} commands `@xnthm' etc.}
% \changes{v1.1b}{87/07/31}{`theoremfont' used to specify font.}
% \changes{v1.2a}{88/01/08}{`@endtheorem' now globally defined.}
% \changes{v1.2b}{88/01/11}{`newtheorem' only allowed in the preamble.}
% \changes{v1.3a}{88/02/14}{`theoremfont' now a toks register.}
%
% \changes{v2.0a}{88/11/16}{New implementation and new documentation.}
% \changes{v2.0a}{88/11/16}{No layout changes but internal modifications
%                           old substyles won't work.}
% \changes{v2.0b}{89/01/30}{`global' used for all definitions.}
% \changes{v2.0c}{89/06/15}{Documentation translated to English.
%                            Thanks to LLL and Nina}
% \changes{v2.1a}{90/02/15}{Support for new font selection.}
% \changes{v2.2a}{94/02/03}{Cleanup for LaTeX2e}
%
% \DoNotIndex{\@endparenv,\@ifdefinable,\@ifnextchar,\@ifundefined}
% \DoNotIndex{\@namedef,\@nameuse,\@spaces,\@tempa,\@warning}
% \DoNotIndex{\ }
% \DoNotIndex{\begingroup,\bfseries}
% \DoNotIndex{\csname}
% \DoNotIndex{\def,\do,\docdate}
% \DoNotIndex{\endcsname,\endgroup,\endinput,\endtrivlist,\expandafter}
% \DoNotIndex{\filedate,\fileversion}
% \DoNotIndex{\gdef,\global}
% \DoNotIndex{\hbox,\hskip}
% \DoNotIndex{\ignorespaces,\input,\itshape,\item}
% \DoNotIndex{\labelsep,\let,\llap}
% \DoNotIndex{\makeatletter}
% \DoNotIndex{\noexpand}
% \DoNotIndex{\refstepcounter,\relax,\rlap}
% \DoNotIndex{\setlength,\slshape,\space,\string,\strut}
% \DoNotIndex{\the,\trivlist,\typeout}
% \DoNotIndex{\vbox}
% \DoNotIndex{\xdef}
%
% \def\deflabel#1{\ttfamily #1\hfill}
% \def\deflist#1{\begin{list}{}{\settowidth\labelwidth{\ttfamily #1}%
%                               \setlength\leftmargin\labelwidth
%                               \addtolength\leftmargin\labelsep
%                               \let\makelabel\deflabel}}
% \def\enddeflist{\end{list}}
% \def\envfont{\normalfont\ttfamily}
%
% \MakeShortVerb{\|}
%
% \newcommand{\TUB}{{\sl TUGboat\/}}
%
% \GetFileInfo{theorem.dtx}
% \title{An Extension of the
%        \LaTeX{} theorem environment\thanks{This file
%        has version number \fileversion, last
%        revised \filedate.}}
% \author{Frank Mittelbach}
%
% \MaintainedByLaTeXTeam{tools}
% \maketitle
%
% \begin{abstract}
%    The macros described in this paper yield an extension of the
%    \LaTeX{} theorem mechanism. It is designed to satisfy the
%    different requirements of various journals. Thus, the
%    layout of the ``theorems'' can be manipulated by determining a
%    ``style''. This article describes not only the use, but
%    also the definition, of the necessary macros.
% \end{abstract}
%
% \section*{Preface to version 2.2}
%
% For \LaTeXe{} this package did not need any fundamental changes. I
% only modified the messages generated so that theorem layout styles
% will show up with the |\listfiles| command and cleaned the section
% on the New Font Selection Scheme since this is now included in
% \LaTeX.
%
% \section*{Preface to version 2.1}
%
% This version is identical to 2.0g described in \TUB\ 10\#3 except
% for some internal defaults which are now set depending on the used
% font selection scheme.
%
% This was done to avoid unpleasant surprises if the new font selection
% scheme is in force. For further details see
% section~\ref{sec:newfonts} and \cite{bk:GMS94}.
%
% \section{Introduction}
%
% For our purposes here, ``theorems'' are labelled enunciations,
% often set off from the main text by extra space and a font change.
% Theorems, corollaries, conjectures, definitions, and remarks
% are all instances of ``theorems''.  The ``header'' of these structures
% is composed of a label (such as \textsc{Theorem} or \textsc{Remark})
% and a number which serializes an item in the sequence of items
% with the same label.
%
% Shortly after the introduction of \LaTeX{} at the Fachbereich
% Mathematik in Mainz, the desire to manipulate the layout of
% ``theorems'' arose.  In Mainz, the following two conventions came
% into general use:
% \begin{enumerate}
%   \item The number of the theorem is shown in the margin.
%   \item There is a line break at
%         the end of the theorem header.
% \end{enumerate}
%
% Additionally, some journals require different formats which depend on
% the ``sort of theorem'': e.g.\ often remarks and definitions are set
% in |\upshape|, while |\itshape| is employed for main theorems.
%
% Confronted with these requirements, a theorem environment was
% developed in Mainz which allows separate determination of the
% layout of the ``theorems sets'', comparable to |\pagestyle|.
%
%
% \section{The user interface}
% \subsection{Defining new theorem sets}
%
% \DescribeMacro\newtheorem
% As in the original \LaTeX{} version, the command |\newtheorem|
% defines a new ``theorem set'' or ``theorem-like structure''.
% Two required arguments name the new environment and give the text
% to be typeset with each instance of the new ``set'', while
% an optional argument determines how
% the ``set'' is enumerated:
% \begin{description}
%    \item[\ttfamily \bslash newtheorem\{foo\}\{bar\}]
%       The theorem set {\envfont foo} (whose name is \texttt{bar})
%       uses its own counter.
%    \item[\ttfamily \bslash newtheorem\{foo2\}{[foo]}\{bar2\}]
%       The theorem set {\envfont foo2} (printed name \texttt{bar2})
%       uses the same counter as the theorem set \texttt{foo}.
%    \item[\ttfamily \bslash newtheorem\{foo3\}\{bar3\}{[section]}]
%       The theorem set {\envfont foo3} (printed name \texttt{bar3}) is
%       enumerated within the counter \texttt{section}, i.e.\ with every
%       new |\section| the enumeration begins again with 1, and
%       the enumeration is composed from the section-number and the
%       theorem counter itself.
% \end{description}
%
% \DescribeMacro\theoremstyle
% Additionally, the command |\theoremstyle| can define the layout
% of various, or all, theorem sets.  It should be noted that any
% theorem set defined by |\newtheorem| is typeset in the
% |\theoremstyle| that is current at the time of the definition.
% Thus, the following
% \begin{verbatim}
%    \theoremstyle{break}        \newtheorem{Cor}{Corollary}
%    \theoremstyle{plain}        \newtheorem{Exa}{Example}[section]
%\end{verbatim}                             ^^A No blank at line start
% leads to the result that the set {\envfont Cor} is formatted in the
% style \texttt{break}, while the set {\envfont Exa} and all the
% following ones are formatted in the style \texttt{plain}, unless
% another |\theoremstyle| follows.  Since the definitions installed by
% |\newtheorem| are global, one also can limit |\theoremstyle| locally
% by grouping braces.
%
% \DescribeMacro\theorembodyfont
% The choice of the font for the theorem body is completely
% independent of the chosen |\theoremstyle|; this has proven to
% be very advantageous.  For example,
% \begin{verbatim}
%   {\theorembodyfont{\upshape}          \newtheorem{Rem}{Remark}}
%\end{verbatim}
% defines a theorem set {\envfont Rem}, which will be set in
% |\upshape| in the current layout (which in our example is
% \texttt{plain}).  As with |\theoremstyle|, the |\theorembodyfont|
% chosen is that current at the time of |\newtheorem|.  If
% |\theorembodyfont| is not specified or one defines
% |\theorembodyfont{}|, then the font used will be that defined by the
% |\theoremstyle|.
%
% \DescribeMacro\theoremheaderfont It is also possible to customize
% the font used for the theorem headers.  This is, however, a global
% declaration, and therefore there should be at most one
% |\theoremheaderfont| declaration in the preamble.\footnote{If it is
% actually necessary to have different header fonts, one has to define
% new theorem styles (substituting the desired font) or specify the
% information directly in the \texttt{\bslash newtheorem} declaration
% (the unclean variant).}
%
% \DescribeMacro\theorempreskipamount
% \DescribeMacro\theorempostskipamount Two additional parameters
% affect the vertical space around the theorem environments:
% |\theorempreskipamount| and |\theorempostskipamount| define,
% respectively, the spacing before and after such an environment.
% These parameters apply for all theorem sets and can be manipulated
% with the ordinary length macros.  They are rubber lengths
% (`\textsf{skips}'), and therefore can contain \texttt{plus} and
% \texttt{minus} parts.
%
% Since the definition of theorem sets should---most sensibly---be
% placed in the preamble, we only allow installation there.
% It is therefore possible to release the memory used here after
% |\begin{document}|, in order to make room for other
% applications.
%
%
% \subsection{Existing theorem styles}
%
% The following theorem styles exist to date:
% \begin{deflist}{marginbreak}
%    \item[plain]
%       This theorem style emulates the original \LaTeX{} definition,
%       except that additionally the parameters
%       |\theorem...skipamount| are used.
%    \item[break]
%       In this style, the theorem header is followed by a line
%       break.
%    \item[marginbreak]
%       The theorem number is set in the margin, and there is a line
%       break as in \texttt{break}.
%    \item[changebreak]
%       Like \texttt{break}, but with header number and text
%       interchanged.
%    \item[change]
%       Header number and text are interchanged, without a line break.
%    \item[margin]
%       The number is set in the left margin, without a line break.
% \end{deflist}
% All styles (except \texttt{plain}) select |\slshape| as the default
% |\theorembodyfont|.
%
%
% \subsection{Examples}
%
% Given the above theorem sets {\envfont Cor}, {\envfont Exa}
% and {\envfont Rem},
% suppose that the preamble also contains the declarations:
% \begin{verbatim}
%    \theoremstyle{marginbreak}   \newtheorem{Lem}[Cor]{Lemma}
%    \theoremstyle{change}
%    \theorembodyfont{\itshape}        \newtheorem{Def}[Cor]{Definition}
%
%    \theoremheaderfont{\scshape}
%\end{verbatim}
% Then the following are some typical examples of the typeset output
% resulting from their use.
%
% \begin{Cor}
%    This is a sentence typeset in the theorem environment {\envfont
%    Cor}.
% \end{Cor}
%
% \begin{Exa}
%    This is a sentence typeset in the theorem environment {\envfont
%    Exa}.
% \end{Exa}
%
% \begin{Rem}
%    This is a sentence typeset in the theorem environment {\envfont
%    Rem}.
% \end{Rem}
%
% \begin{Lem}[Ben User]
%    This is a sentence typeset in the theorem environment {\envfont
%    Lem}.
% \end{Lem}
%
% \begin{Def}[Very impressive Definition]
%    This is a sentence typeset in the theorem environment {\envfont
%    Def}.
% \end{Def}
%
% The last two examples show the effect of the optional argument to
% a theorem environment (it is the text typeset in parentheses).
%
%
% \section{Special Considerations}
% \label{sec:newfonts}
%
% Theoremheader and body are implemented as a unit. This means that
% the |\theoremheaderfont|  will inherit characteristics of the
% |\theorembodyfont| in \LaTeXe. Thus, if for example
% |\theorembodyfont| is |\itshape| and |\theoremheaderfont|
% is |\bfseries| the font selected for the header will have the
% characteristics `bold extended italic'. If this is not desired one
% should set the |\theoremheaderfont| to something like
% \begin{verbatim}
%      \theoremheaderfont{\normalfont\bfseries}
%\end{verbatim}
% i.e.\ supplying all necessary font information explicitly.
%
%
%
% \section{Acknowledgements}
%
% The publication of this set of macros was only possible with the
% help of Christina Busse (translating the manuscript into English),
% Joachim Pense (playing the r\^ole of typist), Chris Rowley
% (looking everything over) and many others providing useful
% suggestions.
%
% \MaybeStop{%
%    \begin{thebibliography}{9}
%       \bibitem{bk:GMS94} \textsc{M.~Goossens}, \textsc{F.~Mittelbach}
%         and \textsc{A.~Samarin}.
%         \newblock The \LaTeX{} Companion.
%         \newblock
%         Addison-Wesley, Reading, Massachusetts, 1994.
%       \bibitem{latex:source} \textsc{Lamport, Leslie.} \newblock
%                               \texttt{latex.tex}, version 2.09, date
%                               Feb. 1990.
%    \end{thebibliography}
%    \PrintIndex
%    \PrintChanges}
%
%
% \section{The documentation driver file}
%
% The next bit of code contains the documentation driver file for
% \TeX{}, i.e., the file that will produce the documentation you are
% currently reading. It will be extracted from this file by the
% \texttt{docstrip} program. Since it is the first code in the file
% one can alternatively process this file directly with \LaTeXe{} to
% obtain the documentation.
%
%    \begin{macrocode}
%<*driver>
\documentclass{ltxdoc}

\usepackage{theorem}

%<+driver>% The next few lines define theorem sets which are used
%<+driver>% in the example section of the documentation.

\theoremstyle{break}       \newtheorem{Cor}{Corollary}
\theoremstyle{plain}       \newtheorem{Exa}{Example}[section]
{\theorembodyfont{\upshape}\newtheorem{Rem}{Remark}}
\theoremstyle{marginbreak} \newtheorem{Lem}[Cor]{Lemma}
\theoremstyle{change}
\theorembodyfont{\itshape} \newtheorem{Def}[Cor]{Definition}

\theoremheaderfont{\scshape}

\RecordChanges

\begin{document}
   \DocInput{theorem.dtx}
\end{document}
%</driver>
%    \end{macrocode}
%
% \section{Definition of the Macros}
%
% If the file has been loaded before, we abort immediately. If not the
% package announces itself (this is actually done at the very top of
% the file---the way it is done isn't good style so don't copy it).
% \changes{v2.0e}{89/07/19}{Spaces removed from `typeout'.}
% \changes{v2.2b}{95/11/19}{Announce moved to top of file}
%    \begin{macrocode}
%<*package>
%\@ifundefined{theorem@style}{}{\endinput}
%\def\FMithmInfo{1995/11/19 v2.2b Theorem extension package (FMi)}
%\ProvidesPackage{theorem}[\FMithmInfo]
%    \end{macrocode}
%
%
% \subsection{Definition of theorem styles and fonts}
%
% All the definitions in this file are done globally to allow
% inputting this file inside a group.
%
% \begin{macro}{\theoremstyle}
%    Before a theorem style can be installed, the chosen style must be
%    known. For that reason, we must test to see that
%    |\th@|\meta{style} is known or, more precisely, that it is
%    different from |\relax|. If the style is not known then
%    |\th@plain| is used.
%    \begin{macrocode}
\gdef\theoremstyle#1{%
   \@ifundefined{th@#1}{\@warning
          {Unknown theoremstyle `#1'. Using `plain'}%
          \theorem@style{plain}}%
%    \end{macrocode}
%    We save the theorem style to be used in the token register
%    |\theorem@style|.
%    \begin{macrocode}
      {\theorem@style{#1}}%
%    \end{macrocode}
%    Now we ``evaluate'' the theorem style: this means, we call the
%    macro |\th@|\meta{style} which will activate the relevant
%    definitions which are contained in a separate file.  This is done
%    in a group to suppress changes to the current font.  This could
%    otherwise pose problems together with the new font selection
%    scheme\footnote{When I printed the original article using the new
%    font selection scheme I ended with a document with slanted
%    typefaces (text headings and all) simply because one of the
%    theorem styles used \texttt{\bslash sl} at toplevel.} if the
%    |\th@|\meta{style} is evaluated a second time.
% \changes{v2.1a}{90/02/15}{Group added around `th@..}
% \changes{v2.1a}{90/02/15}{Reseting of `..@begintheorem removed.}
%    \begin{macrocode}
      \begingroup
        \csname th@\the\theorem@style \endcsname
      \endgroup}
%    \end{macrocode}
% \end{macro}
%
%
% \begin{macro}{\@begintheorem}
% \changes{v2.1a}{90/02/15}{Relaxed at toplevel.}
% \begin{macro}{\@opargbegintheorem}
% \changes{v2.1a}{90/02/15}{Relaxed at toplevel.}
%    We reset |\@begintheorem| and |\@opargbegintheorem| to
%    |\relax| since these commands are no longer necessary at
%    toplevel. This will save a few tokens.
%    \begin{macrocode}
\global\let\@begintheorem\relax
\global\let\@opargbegintheorem\relax
%    \end{macrocode}
% \end{macro}
% \end{macro}
%
% \begin{macro}{\theorem@style}
%    Obviously the \textsf{token} register used above has to be
%    allocated. To assure the utmost compatibility with the
%    original \LaTeX{} definition, we set the default theorem style to
%    \texttt{plain}, which implements the usual \LaTeX{} convention.
% \changes{v2.0f}{89/07/20}{Unnecessary `global' removed.}
%    \begin{macrocode}
\newtoks\theorem@style
\global\theorem@style{plain}
%    \end{macrocode}
% \end{macro}
%
% \begin{macro}{\theorembodyfont}
% \begin{macro}{\theorem@bodyfont}
%    If the |\theorembodyfont| is set by the user then it should not
%    interact with the default font set in the theorem style. When the
%    new font selection is in force this may happen if, for example,
%    the default is |\itshape| and the new |\theorembodyfont| is
%    |\sffamily|. So we add a |\reset@font| command in front of the
%    user definition.
% \changes{v2.0f}{89/07/20}{Unnecessary `global' removed.}
% \changes{v2.0g}{89/09/19}{`theoremfont to `theorembodyfont.}
% \changes{v2.1c}{92/06/22}{`reset@font added and logic changed.}
% \changes{v2.1d}{92/10/28}{Reset to empty if arg is empty}
%    \begin{macrocode}
\gdef\theorembodyfont#1{%
%    \end{macrocode}
%    We check if the argument supplied is empty and if so put nothing
%    into the |\theorem@bodyfont| token register to allow for
%    |\theorembodyfont{}| as a mean of using the default of the
%    current |\theoremstyle|.
%    \begin{macrocode}
   \def\@tempa{#1}%
   \ifx\@tempa\@empty
    \theorem@bodyfont{}%
   \else
    \theorem@bodyfont{\reset@font#1}%
   \fi
   }
\newtoks\theorem@bodyfont
\global\theorem@bodyfont{}
%    \end{macrocode}
% \end{macro}
% \end{macro}
%
%
% \begin{macro}{\theoremheaderfont}
% \changes{v2.0g}{89/09/19}{Macro added.}
%    The font for the theorem headers is handled differently because
%    this definition applies to all theorem styles.
%    \begin{macrocode}
\gdef\theoremheaderfont#1{\gdef\theorem@headerfont{#1}%
%    \end{macrocode}
%    After using the macro once it is redefined to produce an error
%    message.
%    \begin{macrocode}
       \gdef\theoremheaderfont##1{%
        \typeout{\string\theoremheaderfont\space should be used
                 only once.}}}
%    \end{macrocode}
% \end{macro}
%
% \begin{macro}{\theorem@headerfont}
% \changes{v2.0g}{89/09/19}{Macro added.}
%    To set the |\theorem@headerfont| default we first test if
%    the new fontselection scheme is in force.
% \changes{v2.1a}{90/02/15}{Support for new font selection.}
%    \begin{macrocode}
\ifx\upshape\undefined
%    \end{macrocode}
%    If not we define it to expand into |\bfseries|. We don't use
%    |\let| just in case a following style option redefines this
%    macro.
%    \begin{macrocode}
\gdef\theorem@headerfont{\bfseries}
%    \end{macrocode}
%    Otherwise we reset the current shape before calling |\bfseries|.
%    \begin{macrocode}
\else \gdef\theorem@headerfont{\normalfont\bfseries}\fi
%    \end{macrocode}
% \end{macro}
%
% \begin{macro}{\th@plain}
% \begin{macro}{\th@break}
% \begin{macro}{\th@marginbreak}
% \begin{macro}{\th@changebreak}
% \begin{macro}{\th@change}
% \begin{macro}{\th@margin}
%    The different styles are defined in macros such as
%    |\th@plain|.  Since memory space is precious in
%    ``non-Big-versions'', we have to avoid offering too many unused
%    definitions.  Therefore we define these styles in separate files
%    that can be loaded on demand.  Thus the commands themselves only
%    load these files. We use |\@input@|, a \LaTeXe{} internal command
%    that ensures that the file will be listed with |\listfiles|.
% \changes{v2.2a}{94/02/02}{Use `@input@ to load theorem layout files}
%    \begin{macrocode}
\gdef\th@plain{\@input@{thp.sty}}
\gdef\th@break{\@input@{thb.sty}}
\gdef\th@marginbreak{\@input@{thmb.sty}}
\gdef\th@changebreak{\@input@{thcb.sty}}
\gdef\th@change{\@input@{thc.sty}}
\gdef\th@margin{\@input@{thm.sty}}
%    \end{macrocode}
%    This list will be expanded when new styles become
%    available.  For testing,  just append new theorem substyles
%    as document options.
% \end{macro}
% \end{macro}
% \end{macro}
% \end{macro}
% \end{macro}
% \end{macro}
%
%
%
% \subsection{Definition of a new theorem set}
%
% As already pointed out, a new theorem environment can be defined in
% three different ways:
% \begin{verbatim}
%    \newtheorem{Lem}{Lemma}
%    \newtheorem{Lem}{Lemma}[section]
%    \newtheorem{Lem}[Theorem]{Lemma}
%\end{verbatim}
% The function of the macro |\newtheorem| is to recognize these cases
% and then to branch into one of the three macros |\@ynthm|, |\@xnthm|
% or |\@othm|.  This mechanism is adopted unchanged from
% \cite{latex:source}; the essential point here is that, for example,
% in the second case, the arguments \texttt{Lem}, \texttt{Lemma} and
% \texttt{section} are passed over to the macro |\@xnthm|.
%
% We inspect this case first because the others present fewer problems,
% and thus are easily derived from this one.
%
% \begin{macro}{\@xnthm}
%    For our example arguments, the macro |\@xnthm| must fulfill
%    the following:
%    \begin{itemize}
%       \item
%          Define a new \LaTeX-counter `Lem'
%       \item
%          reset this counter within a |\section|
%       \item
%          define the macro |\theLem|
%       \item
%          define the environment macros |\Lem| and
%          |\endLem| using the current
%          |\theoremstyle| and |\theorem@bodyfont|.
%    \end{itemize}
%    Obviously, all this should happen only if the first argument
%    of |\@xnthm| (i.e.\ \texttt{Lem} in our example) is chosen so
%    as not to conflict with any previously defined commands or
%    environments.  This test is performed by the \LaTeX{} macro
%    |\@ifdefinable|.
%    \begin{macrocode}
\gdef\@xnthm#1#2[#3]{\expandafter\@ifdefinable\csname #1\endcsname
%    \end{macrocode}
%    Therefore, the first argument of |\@ifdefinable| is the
%    expansion (in the example, |\Lem|) of
%    |\csname||#1\endcsname|.  The second argument is
%    executed only if the test has been completed successfully.
%    \begin{macrocode}
   {%
%    \end{macrocode}
%    Now we define the new counter. The names of the \LaTeX{}
%    macros employed should speak for themselves:
% \changes{v2.2b}{1995/10/07}{Use \cs{@newctr}}
% \changes{v2.2c}{1995/11/23}{Use \cs{@newctr} properly.}
%    \begin{macrocode}
    \@definecounter{#1}\@newctr{#1}[#3]%
%    \end{macrocode}
%    Using |\@newctr| will give a proper error message if the counter
%    in |#3| is not defined.
%    In defining `|\theLem|' we must generate the desired macro
%    name by use of |\expandafter| and |\csname|.
%    \begin{macrocode}
    \expandafter\xdef\csname the#1\endcsname
%    \end{macrocode}
%    An |\xdef| is used in order to make the definition global,
%    and to ensure that it contains the replacement texts of
%    |\@thmcountersep| and |\@thmcounter|.\footnote{These
%    two macros can be defined by the document style.  Their default
%    values produce a `\texttt{.}' as separation and an arabic
%    representation of the number.} However, not everything should be
%    expanded. For example, it saves space to use |\thesection|
%    instead of its---at times---lengthy expansion.
%    \begin{macrocode}
      {\expandafter \noexpand \csname the#3\endcsname
       \@thmcountersep \@thmcounter{#1}}%
%    \end{macrocode}
%    Thus with the defaults of \LaTeX{}, |\theLem| would be
%    replaced by the command sequence |\thesection.\arabic{Lem}|.
%
%    We will now look at the definition of the macro which is executed
%    at the beginning of the actual environment (in our example this
%    macro is |\Lem|). It should be noted that we use an
%    ``|\expandafter| trick'' to expand only certain parts of the
%    replacement text at the time of the definition.
%    \begin{macrocode}
    \def\@tempa{\global\@namedef{#1}}%
    \expandafter \@tempa \expandafter{%
%    \end{macrocode}
%    First, the macro that contains the current definitions of
%    |\@begintheorem| and |\@opargtheorem| should be called
%    up.  The name of this macro---as is already known---has the form
%    |\th@|\meta{theorem\ style}; therefore, it must be called by
%    \begin{macrocode}
      \csname th@\the \theorem@style
%    \end{macrocode}
%    In addition the default theorem font should be changeable, i.e.\
%    we have to insert the contents of |\theorem@bodyfont|. For
%    that reason, we expand even further, beyond |\endcsname|,
%    and thus insert the contents of the \textsf{token} register
%    |\theorem@bodyfont| in the replacement text.
%    \begin{macrocode}
            \expandafter \endcsname \the \theorem@bodyfont
%    \end{macrocode}
%    Now it is time to call the macro |\@thm| which takes over
%    the further processing.  It has two arguments: the current
%    counter name (in our example, \texttt{Lem}), and the text of the
%    label (in our example, \texttt{Lemma}).
%    \begin{macrocode}
     \@thm{#1}{#2}}%
%    \end{macrocode}
%    With this, the `sub-definition' is complete.  The macro
%    |\@endtheorem| ends a theorem environment and is, so far, nothing
%    but an |\endtrivlist|.  (Hence it is defined globally, and not
%    within the theorem styles.\footnote{This has to be changed as
%    soon as theorem styles that change \texttt{\bslash @endtheorem}
%    exist.  In such a case, all existing styles must be changed as
%    well since they will have to reset the macro.}) Therefore, we can
%    set it equivalent to the macro that ends the theorem set (in our
%    example, |\endLem|).  However, if some day theorem styles exist
%    that do change |\@endtheorem|, we would have to use the
%    commented-out line instead.
%    \begin{macrocode}
    \global \expandafter \let \csname end#1\endcsname \@endtheorem
%   \global\@namedef{end#1}{\@endtheorem}%
%    \end{macrocode}
%    With these commands all the required definitions are employed,
%    unless the test |\@ifdefinable| has failed.  Therefore, we
%    end the second argument of this macro and with it the definition
%    of |\@xnthm|.
%    \begin{macrocode}
   }}
%    \end{macrocode}
% \end{macro}
%
% \begin{macro}{\@ynthm}
%    The definition of |\@ynthm| is completely analogous. In
%    this case the new counter that is defined is not reset within
%    another counter; thus the definition of |\the...|
%    is simplified:
%    \begin{macrocode}
\gdef\@ynthm#1#2{\expandafter\@ifdefinable\csname #1\endcsname
   {\@definecounter{#1}%
    \expandafter\xdef\csname the#1\endcsname{\@thmcounter{#1}}%
%    \end{macrocode}
%    The rest of the definition corresponds literally to that of
%    |\@xnthm|:
%    \begin{macrocode}
    \def\@tempa{\global\@namedef{#1}}\expandafter \@tempa
     \expandafter{\csname th@\the \theorem@style \expandafter
     \endcsname \the\theorem@bodyfont \@thm{#1}{#2}}%
    \global \expandafter \let \csname end#1\endcsname \@endtheorem}}
%    \end{macrocode}
% \end{macro}
%
% \begin{macro}{\@othm}
%    The definition of |\@othm| does not contain anything new.
%    \begin{macrocode}
\gdef\@othm#1[#2]#3{%
%    \end{macrocode}
% \changes{v2.2b}{95/10/07}{Build in check for counter defined}
%    We do not define a new counter but instead use one that has
%    already been defined.  Thus the only definition we need is that
%    of this pseudo-counter (i.e.\ |\the|\meta{env.\ name}).
%    First we check if |#2| corresponds to a known counter name.
%    \begin{macrocode}
  \expandafter\ifx\csname c@#2\endcsname\relax
   \@nocounterr{#2}%
  \else
   \expandafter\@ifdefinable\csname #1\endcsname
   {\expandafter \xdef \csname the#1\endcsname
     {\expandafter \noexpand \csname the#2\endcsname}%
%    \end{macrocode}
%    All other parts of the definition can be adopted from
%    |\@xnthm|. We have to remember, though, that in this case
%    the name of the current counter and the theorem label have moved
%    to the second and third arguments.
%    \begin{macrocode}
    \def\@tempa{\global\@namedef{#1}}\expandafter \@tempa
     \expandafter{\csname th@\the \theorem@style \expandafter
     \endcsname \the\theorem@bodyfont \@thm{#2}{#3}}%
    \global \expandafter \let \csname end#1\endcsname \@endtheorem}%
  \fi}
%    \end{macrocode}
% \end{macro}
%
% \subsection{Macros that are employed in a {\envfont theorem}
%    environment}
%
% \begin{macro}{\@thm}
%    The macro |\@thm| has to increase the current counter.  Then,
%    depending on whether the environment has (or does not have) an
%    optional argument, it has to branch into either |\@begintheorem|
%    or |\@opargtheorem|.
%    \begin{macrocode}
\gdef\@thm#1#2{\refstepcounter{#1}%
%    \end{macrocode}
%    Now we start a {\envfont trivlist} environment, and give
%    |\@topsep| and |\@topsepadd| the values of the \textsf{skip}
%    registers |\theorempreskipamount| and |\theorempostskipamount|.
%    The value in |\@topsep| is the vertical space that is inserted by
%    the first (and only) |\item| in our |\trivlist| whilst
%    |\@topsepadd| is inserted by |\@endparenv| at the end of that
%    {\envfont trivlist} environment.  By using these registers, we
%    obtain the desired space around a {\envfont theorem} environment.
%    \begin{macrocode}
   \trivlist
   \@topsep \theorempreskipamount               % used by first \item
   \@topsepadd \theorempostskipamount           % used by \@endparenv
%    \end{macrocode}
%    Now we have to test whether an optional argument has been given.
% \changes{v2.0d}{89/07/18}{Forgotten percent added.}
%    \begin{macrocode}
   \@ifnextchar [%
%    \end{macrocode}
%    If there is an optional argument, we will call |\@ythm|, and
%    move the arguments read back into the input stream.
%    \begin{macrocode}
   {\@ythm{#1}{#2}}%
%    \end{macrocode}
%    If not, we call |\@begintheorem|. Its first argument is the
%    name of the theorem set (hence the second argument of
%    |\@thm|). Its second argument is the macro that produces the
%    current number.
%    \begin{macrocode}
   {\@begintheorem{#2}{\csname the#1\endcsname}\ignorespaces}}
%    \end{macrocode}
% \end{macro}
%
% \begin{macro}{\@xthm}
% \begin{macro}{\@ythm}
%    Both these macros were originally called by |\@thm|.  We do
%    not need |\@xthm| anymore, hence we reset it to
%    |\relax|. The definition of |\@ythm| has not changed at
%    all from its definition in \LaTeX. In order to make the macros
%    easier to understand, we will
%    nevertheless present it (commented out).
%    \begin{macrocode}
\global\let\@xthm\relax
% \def\@ythm#1#2[#3]{\@opargbegintheorem{#2}{\csname
%           the#1\endcsname}{#3}\ignorespaces}
%</package>
%    \end{macrocode}
%    The primitive |\ignorespaces| in |\@ythm| and
%    |\@thm| is needed to remove the spaces between the
%    |\begin{...}| and the actual text.
% \end{macro}
% \end{macro}
%
%
% \subsection{Definition of the theorem substyles}
%
% As already pointed out, the theorem substyles, defined below, are
% only loaded when necessary.  Note that all these substyles, except
% \texttt{plain}, have |\slshape| as the default body font.
%
% \subsubsection{The \texttt{plain} style}
%
% \label{dok:th@plain} As the following macros use |@|, we have
% to locally set the |\catcode| of this symbol to ``letter''.
% This happens within a group, so that we do not have to worry about
% which |\catcode| that symbol had before.
%    \begin{macrocode}
%<*thp>
\begingroup \makeatletter
%    \end{macrocode}
% Since we are now within a group, we must make all definitions
% globally.  First we make sure that \texttt{theorem.sty} is loaded.
% This will allow us to use this file as a document style option
% without having to call \texttt{theorem} itself as an option.  At the
% same time, we assure that at least version 2 is loaded, since
% |\theorem@style| was not defined in earlier versions.
%    \begin{macrocode}
\@ifundefined{theorem@style}{\input{theorem.sty}}{}
\ProvidesFile{thp.sty}
             [\FMithmInfo]
%    \end{macrocode}
%
%
% \begin{macro}{\th@plain}
% \begin{macro}{\@begintheorem}
% \begin{macro}{\@opargbegintheorem}
%    |\theoremstyle{plain}| corresponds to the original
%    definition, except that the distances to the surrounding text are
%    determined by the parameters |\theorempreskipamount| and
%    |\theorempostskipamount|.
%    First we set the default body font.
%    \begin{macrocode}
\gdef\th@plain{\normalfont\itshape
%    \end{macrocode}
%    Then we define |\@begintheorem| and
%    |\@opargbegintheorem|. These two macros define how the
%    header of a theorem is typeset.  |\@opargbegintheorem| will
%    be called if a \textsf{theorem} environment with an optional
%    argument is encountered; otherwise, the header is constructed by
%    calling |\@begintheorem|.  If one of these macros is
%    executed, we are within a \textsf{trivlist} environment started by
%    |\@thm|.  So the theorem header is produced with an
%    |\item| command.
%
%    Instead of specifying the header font directly, all standard
%    theorem styles use the |\theorem@headerfont| macro to allow
%    customization.  The extra space (|\labelsep|) is necessary
%    because of problems in the \textsf{trivlist} environment.
%    \begin{macrocode}
  \def\@begintheorem##1##2{%
        \item[\hskip\labelsep \theorem@headerfont ##1\ ##2]}%
%    \end{macrocode}
%    The definition of |\@opargbegintheorem| is completely
%    analogous. The only difference is the fact that there exists a
%    third argument (which is the optional parameter of the
%    environment and contains additional information about the
%    theorem).  Customarily we enclose it in parentheses.
%    \begin{macrocode}
\def\@opargbegintheorem##1##2##3{%
   \item[\hskip\labelsep \theorem@headerfont ##1\ ##2\ (##3)]}}
%    \end{macrocode}
% \end{macro}
% \end{macro}
% \end{macro}
%
% \noindent
% We conclude with an |\endgroup| to restore the
% |\catcode| of |@|.
%    \begin{macrocode}
\endgroup
%</thp>
%    \end{macrocode}
%
%
% \subsubsection{The \texttt{break} style}
%
% This style option is stored in the file |thb.sty|.
% For the next two lines see the documentation for |\th@plain| on
% page~\pageref{dok:th@plain}.
%    \begin{macrocode}
%<*thb>
\begingroup \makeatletter
\@ifundefined{theorem@style}{\input{theorem.sty}}{}
\ProvidesFile{thb.sty}
             [\FMithmInfo]
%    \end{macrocode}
%
% \begin{macro}{\th@break}
%    |\theoremstyle{break}| produces a line break after the name
%    of the theorem.  The font is |\slshape|.  Hence, we define
%    |\th@break| as follows:
%    \begin{macrocode}
\gdef\th@break{\normalfont\slshape
  \def\@begintheorem##1##2{\item[%
%    \end{macrocode}
%    We run into the following problem: it is not possible to create
%    the header with |\item[|\meta{title}|]| and then start
%    a new line by, for example, |\mbox{}\\|. Such a definition
%    will fail whenever a list environment follows immediately.  With
%    the above construction, the |\mbox{}| causes the switch
%    |@inlabel| (cf.\ definition of |\list| and
%    |\trivlist| in \cite{latex:source}) to be set to \texttt{false}
%    and so the following list will insert additional vertical
%    space (|\topskip|). This is quite annoying. Therefore, we
%    create the line break within the |\item|. In order to ensure
%    that the text will begin at the proper position in the following
%    line, we simply pretend that the label does not take any
%    room.\footnote{This will lead to problems whenever very high
%    symbols occurring in the line tower into the
%    heading. So, something else has to be done here sometime.}
%    \begin{macrocode}
       \rlap{\vbox{\hbox{\hskip \labelsep\theorem@headerfont ##1\ ##2}%
                   \hbox{\strut}}}]}%
%    \end{macrocode}
%    Again, the definition of |\@opargbegintheorem| is
%    completely analogous.
%    \begin{macrocode}
\def\@opargbegintheorem##1##2##3{%
  \item[\rlap{\vbox{\hbox{\hskip \labelsep \theorem@headerfont
                     ##1\ ##2\ (##3)}%
                    \hbox{\strut}}}]}}
\endgroup
%</thb>
%    \end{macrocode}
% \end{macro}
%
%
% \subsubsection{The \texttt{changebreak} style}
%
%    \begin{macrocode}
% This style option is stored in the file |thcb.sty|.
%    \begin{macrocode}
%<*thcb>
\begingroup \makeatletter
\@ifundefined{theorem@style}{\input{theorem.sty}}{}
\ProvidesFile{thcb.sty}
             [\FMithmInfo]
%    \end{macrocode}
%
% \begin{macro}{\th@changebreak}
%    The \texttt{change-break} theorem style is like \texttt{break}
%    but with interchange of theorem name and theorem number.  Thus we
%    define |\th@changebreak| as follows:
%    \begin{macrocode}
\gdef\th@changebreak{\normalfont\slshape
  \def\@begintheorem##1##2{\item
       [\rlap{\vbox{\hbox{\hskip\labelsep \theorem@headerfont ##2\ ##1}%
                    \hbox{\strut}}}]}%
\def\@opargbegintheorem##1##2##3{%
  \item[\rlap{\vbox{\hbox{\hskip\labelsep \theorem@headerfont
                     ##2\ ##1\ (##3)}%
                    \hbox{\strut}}}]}}
\endgroup
%</thcb>
%    \end{macrocode}
% \end{macro}
%
%
% \subsubsection{The \texttt{change} style}
%
% This style option is stored in the file |thc.sty|.
%    \begin{macrocode}
%<*thc>
\begingroup \makeatletter
\@ifundefined{theorem@style}{\input{theorem.sty}}{}
\ProvidesFile{thc.sty}
             [\FMithmInfo]
%    \end{macrocode}
%
% \begin{macro}{\th@change}
%    The \texttt{change} theorem style corresponds to the
%    \texttt{change break} style without a linebreak after the header.
%    To say it in another way, it's the same as the \texttt{plain}
%    style but with number and name interchanged and |\slshape| as the
%    default font.
%    \begin{macrocode}
\gdef\th@change{\normalfont\slshape
  \def\@begintheorem##1##2{\item
       [\hskip\labelsep \theorem@headerfont ##2\ ##1]}%
\def\@opargbegintheorem##1##2##3{%
  \item[\hskip\labelsep \theorem@headerfont ##2\ ##1\ (##3)]}}
\endgroup
%</thc>
%    \end{macrocode}
% \end{macro}
%
%
% \subsubsection{The \texttt{marginbreak} style}
%
% This style option is the one used most often at Mainz.  It is saved
% in the file |thmb.sty|.
%    \begin{macrocode}
%<*thmb>
\begingroup \makeatletter
\@ifundefined{theorem@style}{\input{theorem.sty}}{}
\ProvidesFile{thmb.sty}
             [\FMithmInfo]
%    \end{macrocode}
%
% \begin{macro}{\th@marginbreak}
%    The \texttt{margin break} style is nearly the same as the
%    \texttt{change break} style. The only difference is the placement
%    of the theorem number.  We use |\llap| to place it in the left
%    margin.
%
%    In this style |\labelsep| denotes the separation between the
%    number and the text.
%    \begin{macrocode}
\gdef\th@marginbreak{\normalfont\slshape
  \def\@begintheorem##1##2{\item
       [\rlap{\vbox{\theorem@headerfont
                    \hbox{\llap{##2}\hskip\labelsep ##1}%
                    \hbox{\strut}}}]}%
\def\@opargbegintheorem##1##2##3{%
  \item[\rlap{\vbox{\theorem@headerfont
                    \hbox{\llap{##2}\hskip\labelsep ##1\ (##3)}%
                    \hbox{\strut}}}]}}
\endgroup
%</thmb>
%    \end{macrocode}
% \changes{v2.1c}{90/02/17}{`endgroup outside macrocode env.}
% \end{macro}
%
%
%
% \subsubsection{The \texttt{margin} style}
%
% This style option is stored in the file |thm.sty|.
%    \begin{macrocode}
%<*thm>
\begingroup \makeatletter
\@ifundefined{theorem@style}{\input{theorem.sty}}{}
\ProvidesFile{thm.sty}
             [\FMithmInfo]
%    \end{macrocode}
%
% \begin{macro}{\th@margin}
%    Again this is only a variant of the theorem styles described
%    above without any new ideas.
%    \begin{macrocode}
\gdef\th@margin{\normalfont\slshape
  \def\@begintheorem##1##2{\item
       [\theorem@headerfont \llap{##2}\hskip\labelsep ##1]}%
\def\@opargbegintheorem##1##2##3{%
  \item[\theorem@headerfont \llap{##2}\hskip\labelsep ##1\ (##3)]}}
\endgroup
%</thm>
%    \end{macrocode}
% \end{macro}
%
%
%
%
% \subsection{Final Definitions}
%
% \begin{macro}{\theorempreskipamount}
% \begin{macro}{\theorempostskipamount}
%    The \textsf{skip} parameters that regulate the vertical empty space
%    before and after the {\envfont theorem} environment have
%    to be allocated as well.
% \changes{v2.0f}{89/07/20}{Unnecessary `global' removed.}
%    \begin{macrocode}
%<*package>
\newskip\theorempreskipamount
\newskip\theorempostskipamount
%    \end{macrocode}
%    Since we have used the same values for all theorem sets, we now can
%    assign them.
%    \begin{macrocode}
\global\setlength\theorempreskipamount{12pt plus 5pt minus 3pt}
\global\setlength\theorempostskipamount{8pt plus 3pt minus 1.5pt}
%    \end{macrocode}
% \end{macro}
% \end{macro}
%
%
% \begin{macro}{\@endtheorem}
%    The same holds for the macro |\@endtheorem|, which ends a
%    {\envfont theorem} environment. Since it is the same for all
%    theorem sets, it is removed from the macros |\th@|\meta{style}.
%    It simply ends the {\envfont trivlist} environment, which was
%    begun in |\@thm|.
%    \begin{macrocode}
\global\let\@endtheorem=\endtrivlist
%    \end{macrocode}
% \end{macro}
%
%
%
% \begin{macro}{\@preamblecmds}
%    All macros defined above are to be used only in the preamble.
%    Therefore, we insert them in |\@preamblecmds| which will disable
%    them at begin document. This is done by the internal \LaTeXe{}
%    command |\@onlypreamble|.
% \changes{v2.0g}{89/09/19}{`theoremheaderfont and `theorembodyfont
%    added.}
% \changes{v2.2a}{94/02/03}{Use \LaTeXe{} interface}
%    \begin{macrocode}
\@onlypreamble\@xnthm
\@onlypreamble\@ynthm
\@onlypreamble\@othm
\@onlypreamble\newtheorem
\@onlypreamble\theoremstyle
\@onlypreamble\theorembodyfont
\@onlypreamble\theoremheaderfont
%    \end{macrocode}
% \end{macro}
%
% Finally we declare the \texttt{plain} theorem style to be the default.
% \changes{v2.1b}{90/06/04}{Fixed missing `item by declaring default
%    style.}
%    \begin{macrocode}
\theoremstyle{plain}
%</package>
%    \end{macrocode}
%
% \Finale

\endinput
}{}
\ProvidesFile{thp.sty}
             [\FMithmInfo]
%    \end{macrocode}
%
%
% \begin{macro}{\th@plain}
% \begin{macro}{\@begintheorem}
% \begin{macro}{\@opargbegintheorem}
%    |\theoremstyle{plain}| corresponds to the original
%    definition, except that the distances to the surrounding text are
%    determined by the parameters |\theorempreskipamount| and
%    |\theorempostskipamount|.
%    First we set the default body font.
%    \begin{macrocode}
\gdef\th@plain{\normalfont\itshape
%    \end{macrocode}
%    Then we define |\@begintheorem| and
%    |\@opargbegintheorem|. These two macros define how the
%    header of a theorem is typeset.  |\@opargbegintheorem| will
%    be called if a \textsf{theorem} environment with an optional
%    argument is encountered; otherwise, the header is constructed by
%    calling |\@begintheorem|.  If one of these macros is
%    executed, we are within a \textsf{trivlist} environment started by
%    |\@thm|.  So the theorem header is produced with an
%    |\item| command.
%
%    Instead of specifying the header font directly, all standard
%    theorem styles use the |\theorem@headerfont| macro to allow
%    customization.  The extra space (|\labelsep|) is necessary
%    because of problems in the \textsf{trivlist} environment.
%    \begin{macrocode}
  \def\@begintheorem##1##2{%
        \item[\hskip\labelsep \theorem@headerfont ##1\ ##2]}%
%    \end{macrocode}
%    The definition of |\@opargbegintheorem| is completely
%    analogous. The only difference is the fact that there exists a
%    third argument (which is the optional parameter of the
%    environment and contains additional information about the
%    theorem).  Customarily we enclose it in parentheses.
%    \begin{macrocode}
\def\@opargbegintheorem##1##2##3{%
   \item[\hskip\labelsep \theorem@headerfont ##1\ ##2\ (##3)]}}
%    \end{macrocode}
% \end{macro}
% \end{macro}
% \end{macro}
%
% \noindent
% We conclude with an |\endgroup| to restore the
% |\catcode| of |@|.
%    \begin{macrocode}
\endgroup
%</thp>
%    \end{macrocode}
%
%
% \subsubsection{The \texttt{break} style}
%
% This style option is stored in the file |thb.sty|.
% For the next two lines see the documentation for |\th@plain| on
% page~\pageref{dok:th@plain}.
%    \begin{macrocode}
%<*thb>
\begingroup \makeatletter
\@ifundefined{theorem@style}{% \iffalse meta-comment
%
% Copyright (C) 1993-2024
%
% The LaTeX Project and any individual authors listed elsewhere
% in this file.
%
% This file is part of the Standard LaTeX `Tools Bundle'.
% -------------------------------------------------------
%
% It may be distributed and/or modified under the
% conditions of the LaTeX Project Public License, either version 1.3c
% of this license or (at your option) any later version.
% The latest version of this license is in
%    https://www.latex-project.org/lppl.txt
% and version 1.3c or later is part of all distributions of LaTeX
% version 2005/12/01 or later.
%
% The list of all files belonging to the LaTeX `Tools Bundle' is
% given in the file `manifest.txt'.
%
% \fi
%
%%
%
%
\def\FMithmInfo{2023/07/05 v2.2c Theorem extension package (FMi)}
%
% \ProvidesFile{theorem.dtx}[\FMithmInfo]
%\iffalse   % this is a METACOMMENT !
%%
%% Package `theorem' to use with LaTeX2e
%% Copyright (C) 1989-1995 Frank Mittelbach, all rights reserved.
%<+package>\@ifundefined{theorem@style}{}{\endinput}
%<+package>\NeedsTeXFormat{LaTeX2e}[1995/06/01]
%<+package>\ProvidesPackage{theorem}
%<+package>         [\FMithmInfo]
%
% \fi
%
% \changes{v1.0}{87/05/07}{First draft.}
% \changes{v1.0a}{87/05/07}{`thistheoremstyle' added.}
% \changes{v1.1a}{87/07/29}{New concept: redefine the internal
%                          \protect\LaTeX{} commands `@xnthm' etc.}
% \changes{v1.1b}{87/07/31}{`theoremfont' used to specify font.}
% \changes{v1.2a}{88/01/08}{`@endtheorem' now globally defined.}
% \changes{v1.2b}{88/01/11}{`newtheorem' only allowed in the preamble.}
% \changes{v1.3a}{88/02/14}{`theoremfont' now a toks register.}
%
% \changes{v2.0a}{88/11/16}{New implementation and new documentation.}
% \changes{v2.0a}{88/11/16}{No layout changes but internal modifications
%                           old substyles won't work.}
% \changes{v2.0b}{89/01/30}{`global' used for all definitions.}
% \changes{v2.0c}{89/06/15}{Documentation translated to English.
%                            Thanks to LLL and Nina}
% \changes{v2.1a}{90/02/15}{Support for new font selection.}
% \changes{v2.2a}{94/02/03}{Cleanup for LaTeX2e}
%
% \DoNotIndex{\@endparenv,\@ifdefinable,\@ifnextchar,\@ifundefined}
% \DoNotIndex{\@namedef,\@nameuse,\@spaces,\@tempa,\@warning}
% \DoNotIndex{\ }
% \DoNotIndex{\begingroup,\bfseries}
% \DoNotIndex{\csname}
% \DoNotIndex{\def,\do,\docdate}
% \DoNotIndex{\endcsname,\endgroup,\endinput,\endtrivlist,\expandafter}
% \DoNotIndex{\filedate,\fileversion}
% \DoNotIndex{\gdef,\global}
% \DoNotIndex{\hbox,\hskip}
% \DoNotIndex{\ignorespaces,\input,\itshape,\item}
% \DoNotIndex{\labelsep,\let,\llap}
% \DoNotIndex{\makeatletter}
% \DoNotIndex{\noexpand}
% \DoNotIndex{\refstepcounter,\relax,\rlap}
% \DoNotIndex{\setlength,\slshape,\space,\string,\strut}
% \DoNotIndex{\the,\trivlist,\typeout}
% \DoNotIndex{\vbox}
% \DoNotIndex{\xdef}
%
% \def\deflabel#1{\ttfamily #1\hfill}
% \def\deflist#1{\begin{list}{}{\settowidth\labelwidth{\ttfamily #1}%
%                               \setlength\leftmargin\labelwidth
%                               \addtolength\leftmargin\labelsep
%                               \let\makelabel\deflabel}}
% \def\enddeflist{\end{list}}
% \def\envfont{\normalfont\ttfamily}
%
% \MakeShortVerb{\|}
%
% \newcommand{\TUB}{{\sl TUGboat\/}}
%
% \GetFileInfo{theorem.dtx}
% \title{An Extension of the
%        \LaTeX{} theorem environment\thanks{This file
%        has version number \fileversion, last
%        revised \filedate.}}
% \author{Frank Mittelbach}
%
% \MaintainedByLaTeXTeam{tools}
% \maketitle
%
% \begin{abstract}
%    The macros described in this paper yield an extension of the
%    \LaTeX{} theorem mechanism. It is designed to satisfy the
%    different requirements of various journals. Thus, the
%    layout of the ``theorems'' can be manipulated by determining a
%    ``style''. This article describes not only the use, but
%    also the definition, of the necessary macros.
% \end{abstract}
%
% \section*{Preface to version 2.2}
%
% For \LaTeXe{} this package did not need any fundamental changes. I
% only modified the messages generated so that theorem layout styles
% will show up with the |\listfiles| command and cleaned the section
% on the New Font Selection Scheme since this is now included in
% \LaTeX.
%
% \section*{Preface to version 2.1}
%
% This version is identical to 2.0g described in \TUB\ 10\#3 except
% for some internal defaults which are now set depending on the used
% font selection scheme.
%
% This was done to avoid unpleasant surprises if the new font selection
% scheme is in force. For further details see
% section~\ref{sec:newfonts} and \cite{bk:GMS94}.
%
% \section{Introduction}
%
% For our purposes here, ``theorems'' are labelled enunciations,
% often set off from the main text by extra space and a font change.
% Theorems, corollaries, conjectures, definitions, and remarks
% are all instances of ``theorems''.  The ``header'' of these structures
% is composed of a label (such as \textsc{Theorem} or \textsc{Remark})
% and a number which serializes an item in the sequence of items
% with the same label.
%
% Shortly after the introduction of \LaTeX{} at the Fachbereich
% Mathematik in Mainz, the desire to manipulate the layout of
% ``theorems'' arose.  In Mainz, the following two conventions came
% into general use:
% \begin{enumerate}
%   \item The number of the theorem is shown in the margin.
%   \item There is a line break at
%         the end of the theorem header.
% \end{enumerate}
%
% Additionally, some journals require different formats which depend on
% the ``sort of theorem'': e.g.\ often remarks and definitions are set
% in |\upshape|, while |\itshape| is employed for main theorems.
%
% Confronted with these requirements, a theorem environment was
% developed in Mainz which allows separate determination of the
% layout of the ``theorems sets'', comparable to |\pagestyle|.
%
%
% \section{The user interface}
% \subsection{Defining new theorem sets}
%
% \DescribeMacro\newtheorem
% As in the original \LaTeX{} version, the command |\newtheorem|
% defines a new ``theorem set'' or ``theorem-like structure''.
% Two required arguments name the new environment and give the text
% to be typeset with each instance of the new ``set'', while
% an optional argument determines how
% the ``set'' is enumerated:
% \begin{description}
%    \item[\ttfamily \bslash newtheorem\{foo\}\{bar\}]
%       The theorem set {\envfont foo} (whose name is \texttt{bar})
%       uses its own counter.
%    \item[\ttfamily \bslash newtheorem\{foo2\}{[foo]}\{bar2\}]
%       The theorem set {\envfont foo2} (printed name \texttt{bar2})
%       uses the same counter as the theorem set \texttt{foo}.
%    \item[\ttfamily \bslash newtheorem\{foo3\}\{bar3\}{[section]}]
%       The theorem set {\envfont foo3} (printed name \texttt{bar3}) is
%       enumerated within the counter \texttt{section}, i.e.\ with every
%       new |\section| the enumeration begins again with 1, and
%       the enumeration is composed from the section-number and the
%       theorem counter itself.
% \end{description}
%
% \DescribeMacro\theoremstyle
% Additionally, the command |\theoremstyle| can define the layout
% of various, or all, theorem sets.  It should be noted that any
% theorem set defined by |\newtheorem| is typeset in the
% |\theoremstyle| that is current at the time of the definition.
% Thus, the following
% \begin{verbatim}
%    \theoremstyle{break}        \newtheorem{Cor}{Corollary}
%    \theoremstyle{plain}        \newtheorem{Exa}{Example}[section]
%\end{verbatim}                             ^^A No blank at line start
% leads to the result that the set {\envfont Cor} is formatted in the
% style \texttt{break}, while the set {\envfont Exa} and all the
% following ones are formatted in the style \texttt{plain}, unless
% another |\theoremstyle| follows.  Since the definitions installed by
% |\newtheorem| are global, one also can limit |\theoremstyle| locally
% by grouping braces.
%
% \DescribeMacro\theorembodyfont
% The choice of the font for the theorem body is completely
% independent of the chosen |\theoremstyle|; this has proven to
% be very advantageous.  For example,
% \begin{verbatim}
%   {\theorembodyfont{\upshape}          \newtheorem{Rem}{Remark}}
%\end{verbatim}
% defines a theorem set {\envfont Rem}, which will be set in
% |\upshape| in the current layout (which in our example is
% \texttt{plain}).  As with |\theoremstyle|, the |\theorembodyfont|
% chosen is that current at the time of |\newtheorem|.  If
% |\theorembodyfont| is not specified or one defines
% |\theorembodyfont{}|, then the font used will be that defined by the
% |\theoremstyle|.
%
% \DescribeMacro\theoremheaderfont It is also possible to customize
% the font used for the theorem headers.  This is, however, a global
% declaration, and therefore there should be at most one
% |\theoremheaderfont| declaration in the preamble.\footnote{If it is
% actually necessary to have different header fonts, one has to define
% new theorem styles (substituting the desired font) or specify the
% information directly in the \texttt{\bslash newtheorem} declaration
% (the unclean variant).}
%
% \DescribeMacro\theorempreskipamount
% \DescribeMacro\theorempostskipamount Two additional parameters
% affect the vertical space around the theorem environments:
% |\theorempreskipamount| and |\theorempostskipamount| define,
% respectively, the spacing before and after such an environment.
% These parameters apply for all theorem sets and can be manipulated
% with the ordinary length macros.  They are rubber lengths
% (`\textsf{skips}'), and therefore can contain \texttt{plus} and
% \texttt{minus} parts.
%
% Since the definition of theorem sets should---most sensibly---be
% placed in the preamble, we only allow installation there.
% It is therefore possible to release the memory used here after
% |\begin{document}|, in order to make room for other
% applications.
%
%
% \subsection{Existing theorem styles}
%
% The following theorem styles exist to date:
% \begin{deflist}{marginbreak}
%    \item[plain]
%       This theorem style emulates the original \LaTeX{} definition,
%       except that additionally the parameters
%       |\theorem...skipamount| are used.
%    \item[break]
%       In this style, the theorem header is followed by a line
%       break.
%    \item[marginbreak]
%       The theorem number is set in the margin, and there is a line
%       break as in \texttt{break}.
%    \item[changebreak]
%       Like \texttt{break}, but with header number and text
%       interchanged.
%    \item[change]
%       Header number and text are interchanged, without a line break.
%    \item[margin]
%       The number is set in the left margin, without a line break.
% \end{deflist}
% All styles (except \texttt{plain}) select |\slshape| as the default
% |\theorembodyfont|.
%
%
% \subsection{Examples}
%
% Given the above theorem sets {\envfont Cor}, {\envfont Exa}
% and {\envfont Rem},
% suppose that the preamble also contains the declarations:
% \begin{verbatim}
%    \theoremstyle{marginbreak}   \newtheorem{Lem}[Cor]{Lemma}
%    \theoremstyle{change}
%    \theorembodyfont{\itshape}        \newtheorem{Def}[Cor]{Definition}
%
%    \theoremheaderfont{\scshape}
%\end{verbatim}
% Then the following are some typical examples of the typeset output
% resulting from their use.
%
% \begin{Cor}
%    This is a sentence typeset in the theorem environment {\envfont
%    Cor}.
% \end{Cor}
%
% \begin{Exa}
%    This is a sentence typeset in the theorem environment {\envfont
%    Exa}.
% \end{Exa}
%
% \begin{Rem}
%    This is a sentence typeset in the theorem environment {\envfont
%    Rem}.
% \end{Rem}
%
% \begin{Lem}[Ben User]
%    This is a sentence typeset in the theorem environment {\envfont
%    Lem}.
% \end{Lem}
%
% \begin{Def}[Very impressive Definition]
%    This is a sentence typeset in the theorem environment {\envfont
%    Def}.
% \end{Def}
%
% The last two examples show the effect of the optional argument to
% a theorem environment (it is the text typeset in parentheses).
%
%
% \section{Special Considerations}
% \label{sec:newfonts}
%
% Theoremheader and body are implemented as a unit. This means that
% the |\theoremheaderfont|  will inherit characteristics of the
% |\theorembodyfont| in \LaTeXe. Thus, if for example
% |\theorembodyfont| is |\itshape| and |\theoremheaderfont|
% is |\bfseries| the font selected for the header will have the
% characteristics `bold extended italic'. If this is not desired one
% should set the |\theoremheaderfont| to something like
% \begin{verbatim}
%      \theoremheaderfont{\normalfont\bfseries}
%\end{verbatim}
% i.e.\ supplying all necessary font information explicitly.
%
%
%
% \section{Acknowledgements}
%
% The publication of this set of macros was only possible with the
% help of Christina Busse (translating the manuscript into English),
% Joachim Pense (playing the r\^ole of typist), Chris Rowley
% (looking everything over) and many others providing useful
% suggestions.
%
% \MaybeStop{%
%    \begin{thebibliography}{9}
%       \bibitem{bk:GMS94} \textsc{M.~Goossens}, \textsc{F.~Mittelbach}
%         and \textsc{A.~Samarin}.
%         \newblock The \LaTeX{} Companion.
%         \newblock
%         Addison-Wesley, Reading, Massachusetts, 1994.
%       \bibitem{latex:source} \textsc{Lamport, Leslie.} \newblock
%                               \texttt{latex.tex}, version 2.09, date
%                               Feb. 1990.
%    \end{thebibliography}
%    \PrintIndex
%    \PrintChanges}
%
%
% \section{The documentation driver file}
%
% The next bit of code contains the documentation driver file for
% \TeX{}, i.e., the file that will produce the documentation you are
% currently reading. It will be extracted from this file by the
% \texttt{docstrip} program. Since it is the first code in the file
% one can alternatively process this file directly with \LaTeXe{} to
% obtain the documentation.
%
%    \begin{macrocode}
%<*driver>
\documentclass{ltxdoc}

\usepackage{theorem}

%<+driver>% The next few lines define theorem sets which are used
%<+driver>% in the example section of the documentation.

\theoremstyle{break}       \newtheorem{Cor}{Corollary}
\theoremstyle{plain}       \newtheorem{Exa}{Example}[section]
{\theorembodyfont{\upshape}\newtheorem{Rem}{Remark}}
\theoremstyle{marginbreak} \newtheorem{Lem}[Cor]{Lemma}
\theoremstyle{change}
\theorembodyfont{\itshape} \newtheorem{Def}[Cor]{Definition}

\theoremheaderfont{\scshape}

\RecordChanges

\begin{document}
   \DocInput{theorem.dtx}
\end{document}
%</driver>
%    \end{macrocode}
%
% \section{Definition of the Macros}
%
% If the file has been loaded before, we abort immediately. If not the
% package announces itself (this is actually done at the very top of
% the file---the way it is done isn't good style so don't copy it).
% \changes{v2.0e}{89/07/19}{Spaces removed from `typeout'.}
% \changes{v2.2b}{95/11/19}{Announce moved to top of file}
%    \begin{macrocode}
%<*package>
%\@ifundefined{theorem@style}{}{\endinput}
%\def\FMithmInfo{1995/11/19 v2.2b Theorem extension package (FMi)}
%\ProvidesPackage{theorem}[\FMithmInfo]
%    \end{macrocode}
%
%
% \subsection{Definition of theorem styles and fonts}
%
% All the definitions in this file are done globally to allow
% inputting this file inside a group.
%
% \begin{macro}{\theoremstyle}
%    Before a theorem style can be installed, the chosen style must be
%    known. For that reason, we must test to see that
%    |\th@|\meta{style} is known or, more precisely, that it is
%    different from |\relax|. If the style is not known then
%    |\th@plain| is used.
%    \begin{macrocode}
\gdef\theoremstyle#1{%
   \@ifundefined{th@#1}{\@warning
          {Unknown theoremstyle `#1'. Using `plain'}%
          \theorem@style{plain}}%
%    \end{macrocode}
%    We save the theorem style to be used in the token register
%    |\theorem@style|.
%    \begin{macrocode}
      {\theorem@style{#1}}%
%    \end{macrocode}
%    Now we ``evaluate'' the theorem style: this means, we call the
%    macro |\th@|\meta{style} which will activate the relevant
%    definitions which are contained in a separate file.  This is done
%    in a group to suppress changes to the current font.  This could
%    otherwise pose problems together with the new font selection
%    scheme\footnote{When I printed the original article using the new
%    font selection scheme I ended with a document with slanted
%    typefaces (text headings and all) simply because one of the
%    theorem styles used \texttt{\bslash sl} at toplevel.} if the
%    |\th@|\meta{style} is evaluated a second time.
% \changes{v2.1a}{90/02/15}{Group added around `th@..}
% \changes{v2.1a}{90/02/15}{Reseting of `..@begintheorem removed.}
%    \begin{macrocode}
      \begingroup
        \csname th@\the\theorem@style \endcsname
      \endgroup}
%    \end{macrocode}
% \end{macro}
%
%
% \begin{macro}{\@begintheorem}
% \changes{v2.1a}{90/02/15}{Relaxed at toplevel.}
% \begin{macro}{\@opargbegintheorem}
% \changes{v2.1a}{90/02/15}{Relaxed at toplevel.}
%    We reset |\@begintheorem| and |\@opargbegintheorem| to
%    |\relax| since these commands are no longer necessary at
%    toplevel. This will save a few tokens.
%    \begin{macrocode}
\global\let\@begintheorem\relax
\global\let\@opargbegintheorem\relax
%    \end{macrocode}
% \end{macro}
% \end{macro}
%
% \begin{macro}{\theorem@style}
%    Obviously the \textsf{token} register used above has to be
%    allocated. To assure the utmost compatibility with the
%    original \LaTeX{} definition, we set the default theorem style to
%    \texttt{plain}, which implements the usual \LaTeX{} convention.
% \changes{v2.0f}{89/07/20}{Unnecessary `global' removed.}
%    \begin{macrocode}
\newtoks\theorem@style
\global\theorem@style{plain}
%    \end{macrocode}
% \end{macro}
%
% \begin{macro}{\theorembodyfont}
% \begin{macro}{\theorem@bodyfont}
%    If the |\theorembodyfont| is set by the user then it should not
%    interact with the default font set in the theorem style. When the
%    new font selection is in force this may happen if, for example,
%    the default is |\itshape| and the new |\theorembodyfont| is
%    |\sffamily|. So we add a |\reset@font| command in front of the
%    user definition.
% \changes{v2.0f}{89/07/20}{Unnecessary `global' removed.}
% \changes{v2.0g}{89/09/19}{`theoremfont to `theorembodyfont.}
% \changes{v2.1c}{92/06/22}{`reset@font added and logic changed.}
% \changes{v2.1d}{92/10/28}{Reset to empty if arg is empty}
%    \begin{macrocode}
\gdef\theorembodyfont#1{%
%    \end{macrocode}
%    We check if the argument supplied is empty and if so put nothing
%    into the |\theorem@bodyfont| token register to allow for
%    |\theorembodyfont{}| as a mean of using the default of the
%    current |\theoremstyle|.
%    \begin{macrocode}
   \def\@tempa{#1}%
   \ifx\@tempa\@empty
    \theorem@bodyfont{}%
   \else
    \theorem@bodyfont{\reset@font#1}%
   \fi
   }
\newtoks\theorem@bodyfont
\global\theorem@bodyfont{}
%    \end{macrocode}
% \end{macro}
% \end{macro}
%
%
% \begin{macro}{\theoremheaderfont}
% \changes{v2.0g}{89/09/19}{Macro added.}
%    The font for the theorem headers is handled differently because
%    this definition applies to all theorem styles.
%    \begin{macrocode}
\gdef\theoremheaderfont#1{\gdef\theorem@headerfont{#1}%
%    \end{macrocode}
%    After using the macro once it is redefined to produce an error
%    message.
%    \begin{macrocode}
       \gdef\theoremheaderfont##1{%
        \typeout{\string\theoremheaderfont\space should be used
                 only once.}}}
%    \end{macrocode}
% \end{macro}
%
% \begin{macro}{\theorem@headerfont}
% \changes{v2.0g}{89/09/19}{Macro added.}
%    To set the |\theorem@headerfont| default we first test if
%    the new fontselection scheme is in force.
% \changes{v2.1a}{90/02/15}{Support for new font selection.}
%    \begin{macrocode}
\ifx\upshape\undefined
%    \end{macrocode}
%    If not we define it to expand into |\bfseries|. We don't use
%    |\let| just in case a following style option redefines this
%    macro.
%    \begin{macrocode}
\gdef\theorem@headerfont{\bfseries}
%    \end{macrocode}
%    Otherwise we reset the current shape before calling |\bfseries|.
%    \begin{macrocode}
\else \gdef\theorem@headerfont{\normalfont\bfseries}\fi
%    \end{macrocode}
% \end{macro}
%
% \begin{macro}{\th@plain}
% \begin{macro}{\th@break}
% \begin{macro}{\th@marginbreak}
% \begin{macro}{\th@changebreak}
% \begin{macro}{\th@change}
% \begin{macro}{\th@margin}
%    The different styles are defined in macros such as
%    |\th@plain|.  Since memory space is precious in
%    ``non-Big-versions'', we have to avoid offering too many unused
%    definitions.  Therefore we define these styles in separate files
%    that can be loaded on demand.  Thus the commands themselves only
%    load these files. We use |\@input@|, a \LaTeXe{} internal command
%    that ensures that the file will be listed with |\listfiles|.
% \changes{v2.2a}{94/02/02}{Use `@input@ to load theorem layout files}
%    \begin{macrocode}
\gdef\th@plain{\@input@{thp.sty}}
\gdef\th@break{\@input@{thb.sty}}
\gdef\th@marginbreak{\@input@{thmb.sty}}
\gdef\th@changebreak{\@input@{thcb.sty}}
\gdef\th@change{\@input@{thc.sty}}
\gdef\th@margin{\@input@{thm.sty}}
%    \end{macrocode}
%    This list will be expanded when new styles become
%    available.  For testing,  just append new theorem substyles
%    as document options.
% \end{macro}
% \end{macro}
% \end{macro}
% \end{macro}
% \end{macro}
% \end{macro}
%
%
%
% \subsection{Definition of a new theorem set}
%
% As already pointed out, a new theorem environment can be defined in
% three different ways:
% \begin{verbatim}
%    \newtheorem{Lem}{Lemma}
%    \newtheorem{Lem}{Lemma}[section]
%    \newtheorem{Lem}[Theorem]{Lemma}
%\end{verbatim}
% The function of the macro |\newtheorem| is to recognize these cases
% and then to branch into one of the three macros |\@ynthm|, |\@xnthm|
% or |\@othm|.  This mechanism is adopted unchanged from
% \cite{latex:source}; the essential point here is that, for example,
% in the second case, the arguments \texttt{Lem}, \texttt{Lemma} and
% \texttt{section} are passed over to the macro |\@xnthm|.
%
% We inspect this case first because the others present fewer problems,
% and thus are easily derived from this one.
%
% \begin{macro}{\@xnthm}
%    For our example arguments, the macro |\@xnthm| must fulfill
%    the following:
%    \begin{itemize}
%       \item
%          Define a new \LaTeX-counter `Lem'
%       \item
%          reset this counter within a |\section|
%       \item
%          define the macro |\theLem|
%       \item
%          define the environment macros |\Lem| and
%          |\endLem| using the current
%          |\theoremstyle| and |\theorem@bodyfont|.
%    \end{itemize}
%    Obviously, all this should happen only if the first argument
%    of |\@xnthm| (i.e.\ \texttt{Lem} in our example) is chosen so
%    as not to conflict with any previously defined commands or
%    environments.  This test is performed by the \LaTeX{} macro
%    |\@ifdefinable|.
%    \begin{macrocode}
\gdef\@xnthm#1#2[#3]{\expandafter\@ifdefinable\csname #1\endcsname
%    \end{macrocode}
%    Therefore, the first argument of |\@ifdefinable| is the
%    expansion (in the example, |\Lem|) of
%    |\csname||#1\endcsname|.  The second argument is
%    executed only if the test has been completed successfully.
%    \begin{macrocode}
   {%
%    \end{macrocode}
%    Now we define the new counter. The names of the \LaTeX{}
%    macros employed should speak for themselves:
% \changes{v2.2b}{1995/10/07}{Use \cs{@newctr}}
% \changes{v2.2c}{1995/11/23}{Use \cs{@newctr} properly.}
%    \begin{macrocode}
    \@definecounter{#1}\@newctr{#1}[#3]%
%    \end{macrocode}
%    Using |\@newctr| will give a proper error message if the counter
%    in |#3| is not defined.
%    In defining `|\theLem|' we must generate the desired macro
%    name by use of |\expandafter| and |\csname|.
%    \begin{macrocode}
    \expandafter\xdef\csname the#1\endcsname
%    \end{macrocode}
%    An |\xdef| is used in order to make the definition global,
%    and to ensure that it contains the replacement texts of
%    |\@thmcountersep| and |\@thmcounter|.\footnote{These
%    two macros can be defined by the document style.  Their default
%    values produce a `\texttt{.}' as separation and an arabic
%    representation of the number.} However, not everything should be
%    expanded. For example, it saves space to use |\thesection|
%    instead of its---at times---lengthy expansion.
%    \begin{macrocode}
      {\expandafter \noexpand \csname the#3\endcsname
       \@thmcountersep \@thmcounter{#1}}%
%    \end{macrocode}
%    Thus with the defaults of \LaTeX{}, |\theLem| would be
%    replaced by the command sequence |\thesection.\arabic{Lem}|.
%
%    We will now look at the definition of the macro which is executed
%    at the beginning of the actual environment (in our example this
%    macro is |\Lem|). It should be noted that we use an
%    ``|\expandafter| trick'' to expand only certain parts of the
%    replacement text at the time of the definition.
%    \begin{macrocode}
    \def\@tempa{\global\@namedef{#1}}%
    \expandafter \@tempa \expandafter{%
%    \end{macrocode}
%    First, the macro that contains the current definitions of
%    |\@begintheorem| and |\@opargtheorem| should be called
%    up.  The name of this macro---as is already known---has the form
%    |\th@|\meta{theorem\ style}; therefore, it must be called by
%    \begin{macrocode}
      \csname th@\the \theorem@style
%    \end{macrocode}
%    In addition the default theorem font should be changeable, i.e.\
%    we have to insert the contents of |\theorem@bodyfont|. For
%    that reason, we expand even further, beyond |\endcsname|,
%    and thus insert the contents of the \textsf{token} register
%    |\theorem@bodyfont| in the replacement text.
%    \begin{macrocode}
            \expandafter \endcsname \the \theorem@bodyfont
%    \end{macrocode}
%    Now it is time to call the macro |\@thm| which takes over
%    the further processing.  It has two arguments: the current
%    counter name (in our example, \texttt{Lem}), and the text of the
%    label (in our example, \texttt{Lemma}).
%    \begin{macrocode}
     \@thm{#1}{#2}}%
%    \end{macrocode}
%    With this, the `sub-definition' is complete.  The macro
%    |\@endtheorem| ends a theorem environment and is, so far, nothing
%    but an |\endtrivlist|.  (Hence it is defined globally, and not
%    within the theorem styles.\footnote{This has to be changed as
%    soon as theorem styles that change \texttt{\bslash @endtheorem}
%    exist.  In such a case, all existing styles must be changed as
%    well since they will have to reset the macro.}) Therefore, we can
%    set it equivalent to the macro that ends the theorem set (in our
%    example, |\endLem|).  However, if some day theorem styles exist
%    that do change |\@endtheorem|, we would have to use the
%    commented-out line instead.
%    \begin{macrocode}
    \global \expandafter \let \csname end#1\endcsname \@endtheorem
%   \global\@namedef{end#1}{\@endtheorem}%
%    \end{macrocode}
%    With these commands all the required definitions are employed,
%    unless the test |\@ifdefinable| has failed.  Therefore, we
%    end the second argument of this macro and with it the definition
%    of |\@xnthm|.
%    \begin{macrocode}
   }}
%    \end{macrocode}
% \end{macro}
%
% \begin{macro}{\@ynthm}
%    The definition of |\@ynthm| is completely analogous. In
%    this case the new counter that is defined is not reset within
%    another counter; thus the definition of |\the...|
%    is simplified:
%    \begin{macrocode}
\gdef\@ynthm#1#2{\expandafter\@ifdefinable\csname #1\endcsname
   {\@definecounter{#1}%
    \expandafter\xdef\csname the#1\endcsname{\@thmcounter{#1}}%
%    \end{macrocode}
%    The rest of the definition corresponds literally to that of
%    |\@xnthm|:
%    \begin{macrocode}
    \def\@tempa{\global\@namedef{#1}}\expandafter \@tempa
     \expandafter{\csname th@\the \theorem@style \expandafter
     \endcsname \the\theorem@bodyfont \@thm{#1}{#2}}%
    \global \expandafter \let \csname end#1\endcsname \@endtheorem}}
%    \end{macrocode}
% \end{macro}
%
% \begin{macro}{\@othm}
%    The definition of |\@othm| does not contain anything new.
%    \begin{macrocode}
\gdef\@othm#1[#2]#3{%
%    \end{macrocode}
% \changes{v2.2b}{95/10/07}{Build in check for counter defined}
%    We do not define a new counter but instead use one that has
%    already been defined.  Thus the only definition we need is that
%    of this pseudo-counter (i.e.\ |\the|\meta{env.\ name}).
%    First we check if |#2| corresponds to a known counter name.
%    \begin{macrocode}
  \expandafter\ifx\csname c@#2\endcsname\relax
   \@nocounterr{#2}%
  \else
   \expandafter\@ifdefinable\csname #1\endcsname
   {\expandafter \xdef \csname the#1\endcsname
     {\expandafter \noexpand \csname the#2\endcsname}%
%    \end{macrocode}
%    All other parts of the definition can be adopted from
%    |\@xnthm|. We have to remember, though, that in this case
%    the name of the current counter and the theorem label have moved
%    to the second and third arguments.
%    \begin{macrocode}
    \def\@tempa{\global\@namedef{#1}}\expandafter \@tempa
     \expandafter{\csname th@\the \theorem@style \expandafter
     \endcsname \the\theorem@bodyfont \@thm{#2}{#3}}%
    \global \expandafter \let \csname end#1\endcsname \@endtheorem}%
  \fi}
%    \end{macrocode}
% \end{macro}
%
% \subsection{Macros that are employed in a {\envfont theorem}
%    environment}
%
% \begin{macro}{\@thm}
%    The macro |\@thm| has to increase the current counter.  Then,
%    depending on whether the environment has (or does not have) an
%    optional argument, it has to branch into either |\@begintheorem|
%    or |\@opargtheorem|.
%    \begin{macrocode}
\gdef\@thm#1#2{\refstepcounter{#1}%
%    \end{macrocode}
%    Now we start a {\envfont trivlist} environment, and give
%    |\@topsep| and |\@topsepadd| the values of the \textsf{skip}
%    registers |\theorempreskipamount| and |\theorempostskipamount|.
%    The value in |\@topsep| is the vertical space that is inserted by
%    the first (and only) |\item| in our |\trivlist| whilst
%    |\@topsepadd| is inserted by |\@endparenv| at the end of that
%    {\envfont trivlist} environment.  By using these registers, we
%    obtain the desired space around a {\envfont theorem} environment.
%    \begin{macrocode}
   \trivlist
   \@topsep \theorempreskipamount               % used by first \item
   \@topsepadd \theorempostskipamount           % used by \@endparenv
%    \end{macrocode}
%    Now we have to test whether an optional argument has been given.
% \changes{v2.0d}{89/07/18}{Forgotten percent added.}
%    \begin{macrocode}
   \@ifnextchar [%
%    \end{macrocode}
%    If there is an optional argument, we will call |\@ythm|, and
%    move the arguments read back into the input stream.
%    \begin{macrocode}
   {\@ythm{#1}{#2}}%
%    \end{macrocode}
%    If not, we call |\@begintheorem|. Its first argument is the
%    name of the theorem set (hence the second argument of
%    |\@thm|). Its second argument is the macro that produces the
%    current number.
%    \begin{macrocode}
   {\@begintheorem{#2}{\csname the#1\endcsname}\ignorespaces}}
%    \end{macrocode}
% \end{macro}
%
% \begin{macro}{\@xthm}
% \begin{macro}{\@ythm}
%    Both these macros were originally called by |\@thm|.  We do
%    not need |\@xthm| anymore, hence we reset it to
%    |\relax|. The definition of |\@ythm| has not changed at
%    all from its definition in \LaTeX. In order to make the macros
%    easier to understand, we will
%    nevertheless present it (commented out).
%    \begin{macrocode}
\global\let\@xthm\relax
% \def\@ythm#1#2[#3]{\@opargbegintheorem{#2}{\csname
%           the#1\endcsname}{#3}\ignorespaces}
%</package>
%    \end{macrocode}
%    The primitive |\ignorespaces| in |\@ythm| and
%    |\@thm| is needed to remove the spaces between the
%    |\begin{...}| and the actual text.
% \end{macro}
% \end{macro}
%
%
% \subsection{Definition of the theorem substyles}
%
% As already pointed out, the theorem substyles, defined below, are
% only loaded when necessary.  Note that all these substyles, except
% \texttt{plain}, have |\slshape| as the default body font.
%
% \subsubsection{The \texttt{plain} style}
%
% \label{dok:th@plain} As the following macros use |@|, we have
% to locally set the |\catcode| of this symbol to ``letter''.
% This happens within a group, so that we do not have to worry about
% which |\catcode| that symbol had before.
%    \begin{macrocode}
%<*thp>
\begingroup \makeatletter
%    \end{macrocode}
% Since we are now within a group, we must make all definitions
% globally.  First we make sure that \texttt{theorem.sty} is loaded.
% This will allow us to use this file as a document style option
% without having to call \texttt{theorem} itself as an option.  At the
% same time, we assure that at least version 2 is loaded, since
% |\theorem@style| was not defined in earlier versions.
%    \begin{macrocode}
\@ifundefined{theorem@style}{\input{theorem.sty}}{}
\ProvidesFile{thp.sty}
             [\FMithmInfo]
%    \end{macrocode}
%
%
% \begin{macro}{\th@plain}
% \begin{macro}{\@begintheorem}
% \begin{macro}{\@opargbegintheorem}
%    |\theoremstyle{plain}| corresponds to the original
%    definition, except that the distances to the surrounding text are
%    determined by the parameters |\theorempreskipamount| and
%    |\theorempostskipamount|.
%    First we set the default body font.
%    \begin{macrocode}
\gdef\th@plain{\normalfont\itshape
%    \end{macrocode}
%    Then we define |\@begintheorem| and
%    |\@opargbegintheorem|. These two macros define how the
%    header of a theorem is typeset.  |\@opargbegintheorem| will
%    be called if a \textsf{theorem} environment with an optional
%    argument is encountered; otherwise, the header is constructed by
%    calling |\@begintheorem|.  If one of these macros is
%    executed, we are within a \textsf{trivlist} environment started by
%    |\@thm|.  So the theorem header is produced with an
%    |\item| command.
%
%    Instead of specifying the header font directly, all standard
%    theorem styles use the |\theorem@headerfont| macro to allow
%    customization.  The extra space (|\labelsep|) is necessary
%    because of problems in the \textsf{trivlist} environment.
%    \begin{macrocode}
  \def\@begintheorem##1##2{%
        \item[\hskip\labelsep \theorem@headerfont ##1\ ##2]}%
%    \end{macrocode}
%    The definition of |\@opargbegintheorem| is completely
%    analogous. The only difference is the fact that there exists a
%    third argument (which is the optional parameter of the
%    environment and contains additional information about the
%    theorem).  Customarily we enclose it in parentheses.
%    \begin{macrocode}
\def\@opargbegintheorem##1##2##3{%
   \item[\hskip\labelsep \theorem@headerfont ##1\ ##2\ (##3)]}}
%    \end{macrocode}
% \end{macro}
% \end{macro}
% \end{macro}
%
% \noindent
% We conclude with an |\endgroup| to restore the
% |\catcode| of |@|.
%    \begin{macrocode}
\endgroup
%</thp>
%    \end{macrocode}
%
%
% \subsubsection{The \texttt{break} style}
%
% This style option is stored in the file |thb.sty|.
% For the next two lines see the documentation for |\th@plain| on
% page~\pageref{dok:th@plain}.
%    \begin{macrocode}
%<*thb>
\begingroup \makeatletter
\@ifundefined{theorem@style}{\input{theorem.sty}}{}
\ProvidesFile{thb.sty}
             [\FMithmInfo]
%    \end{macrocode}
%
% \begin{macro}{\th@break}
%    |\theoremstyle{break}| produces a line break after the name
%    of the theorem.  The font is |\slshape|.  Hence, we define
%    |\th@break| as follows:
%    \begin{macrocode}
\gdef\th@break{\normalfont\slshape
  \def\@begintheorem##1##2{\item[%
%    \end{macrocode}
%    We run into the following problem: it is not possible to create
%    the header with |\item[|\meta{title}|]| and then start
%    a new line by, for example, |\mbox{}\\|. Such a definition
%    will fail whenever a list environment follows immediately.  With
%    the above construction, the |\mbox{}| causes the switch
%    |@inlabel| (cf.\ definition of |\list| and
%    |\trivlist| in \cite{latex:source}) to be set to \texttt{false}
%    and so the following list will insert additional vertical
%    space (|\topskip|). This is quite annoying. Therefore, we
%    create the line break within the |\item|. In order to ensure
%    that the text will begin at the proper position in the following
%    line, we simply pretend that the label does not take any
%    room.\footnote{This will lead to problems whenever very high
%    symbols occurring in the line tower into the
%    heading. So, something else has to be done here sometime.}
%    \begin{macrocode}
       \rlap{\vbox{\hbox{\hskip \labelsep\theorem@headerfont ##1\ ##2}%
                   \hbox{\strut}}}]}%
%    \end{macrocode}
%    Again, the definition of |\@opargbegintheorem| is
%    completely analogous.
%    \begin{macrocode}
\def\@opargbegintheorem##1##2##3{%
  \item[\rlap{\vbox{\hbox{\hskip \labelsep \theorem@headerfont
                     ##1\ ##2\ (##3)}%
                    \hbox{\strut}}}]}}
\endgroup
%</thb>
%    \end{macrocode}
% \end{macro}
%
%
% \subsubsection{The \texttt{changebreak} style}
%
%    \begin{macrocode}
% This style option is stored in the file |thcb.sty|.
%    \begin{macrocode}
%<*thcb>
\begingroup \makeatletter
\@ifundefined{theorem@style}{\input{theorem.sty}}{}
\ProvidesFile{thcb.sty}
             [\FMithmInfo]
%    \end{macrocode}
%
% \begin{macro}{\th@changebreak}
%    The \texttt{change-break} theorem style is like \texttt{break}
%    but with interchange of theorem name and theorem number.  Thus we
%    define |\th@changebreak| as follows:
%    \begin{macrocode}
\gdef\th@changebreak{\normalfont\slshape
  \def\@begintheorem##1##2{\item
       [\rlap{\vbox{\hbox{\hskip\labelsep \theorem@headerfont ##2\ ##1}%
                    \hbox{\strut}}}]}%
\def\@opargbegintheorem##1##2##3{%
  \item[\rlap{\vbox{\hbox{\hskip\labelsep \theorem@headerfont
                     ##2\ ##1\ (##3)}%
                    \hbox{\strut}}}]}}
\endgroup
%</thcb>
%    \end{macrocode}
% \end{macro}
%
%
% \subsubsection{The \texttt{change} style}
%
% This style option is stored in the file |thc.sty|.
%    \begin{macrocode}
%<*thc>
\begingroup \makeatletter
\@ifundefined{theorem@style}{\input{theorem.sty}}{}
\ProvidesFile{thc.sty}
             [\FMithmInfo]
%    \end{macrocode}
%
% \begin{macro}{\th@change}
%    The \texttt{change} theorem style corresponds to the
%    \texttt{change break} style without a linebreak after the header.
%    To say it in another way, it's the same as the \texttt{plain}
%    style but with number and name interchanged and |\slshape| as the
%    default font.
%    \begin{macrocode}
\gdef\th@change{\normalfont\slshape
  \def\@begintheorem##1##2{\item
       [\hskip\labelsep \theorem@headerfont ##2\ ##1]}%
\def\@opargbegintheorem##1##2##3{%
  \item[\hskip\labelsep \theorem@headerfont ##2\ ##1\ (##3)]}}
\endgroup
%</thc>
%    \end{macrocode}
% \end{macro}
%
%
% \subsubsection{The \texttt{marginbreak} style}
%
% This style option is the one used most often at Mainz.  It is saved
% in the file |thmb.sty|.
%    \begin{macrocode}
%<*thmb>
\begingroup \makeatletter
\@ifundefined{theorem@style}{\input{theorem.sty}}{}
\ProvidesFile{thmb.sty}
             [\FMithmInfo]
%    \end{macrocode}
%
% \begin{macro}{\th@marginbreak}
%    The \texttt{margin break} style is nearly the same as the
%    \texttt{change break} style. The only difference is the placement
%    of the theorem number.  We use |\llap| to place it in the left
%    margin.
%
%    In this style |\labelsep| denotes the separation between the
%    number and the text.
%    \begin{macrocode}
\gdef\th@marginbreak{\normalfont\slshape
  \def\@begintheorem##1##2{\item
       [\rlap{\vbox{\theorem@headerfont
                    \hbox{\llap{##2}\hskip\labelsep ##1}%
                    \hbox{\strut}}}]}%
\def\@opargbegintheorem##1##2##3{%
  \item[\rlap{\vbox{\theorem@headerfont
                    \hbox{\llap{##2}\hskip\labelsep ##1\ (##3)}%
                    \hbox{\strut}}}]}}
\endgroup
%</thmb>
%    \end{macrocode}
% \changes{v2.1c}{90/02/17}{`endgroup outside macrocode env.}
% \end{macro}
%
%
%
% \subsubsection{The \texttt{margin} style}
%
% This style option is stored in the file |thm.sty|.
%    \begin{macrocode}
%<*thm>
\begingroup \makeatletter
\@ifundefined{theorem@style}{\input{theorem.sty}}{}
\ProvidesFile{thm.sty}
             [\FMithmInfo]
%    \end{macrocode}
%
% \begin{macro}{\th@margin}
%    Again this is only a variant of the theorem styles described
%    above without any new ideas.
%    \begin{macrocode}
\gdef\th@margin{\normalfont\slshape
  \def\@begintheorem##1##2{\item
       [\theorem@headerfont \llap{##2}\hskip\labelsep ##1]}%
\def\@opargbegintheorem##1##2##3{%
  \item[\theorem@headerfont \llap{##2}\hskip\labelsep ##1\ (##3)]}}
\endgroup
%</thm>
%    \end{macrocode}
% \end{macro}
%
%
%
%
% \subsection{Final Definitions}
%
% \begin{macro}{\theorempreskipamount}
% \begin{macro}{\theorempostskipamount}
%    The \textsf{skip} parameters that regulate the vertical empty space
%    before and after the {\envfont theorem} environment have
%    to be allocated as well.
% \changes{v2.0f}{89/07/20}{Unnecessary `global' removed.}
%    \begin{macrocode}
%<*package>
\newskip\theorempreskipamount
\newskip\theorempostskipamount
%    \end{macrocode}
%    Since we have used the same values for all theorem sets, we now can
%    assign them.
%    \begin{macrocode}
\global\setlength\theorempreskipamount{12pt plus 5pt minus 3pt}
\global\setlength\theorempostskipamount{8pt plus 3pt minus 1.5pt}
%    \end{macrocode}
% \end{macro}
% \end{macro}
%
%
% \begin{macro}{\@endtheorem}
%    The same holds for the macro |\@endtheorem|, which ends a
%    {\envfont theorem} environment. Since it is the same for all
%    theorem sets, it is removed from the macros |\th@|\meta{style}.
%    It simply ends the {\envfont trivlist} environment, which was
%    begun in |\@thm|.
%    \begin{macrocode}
\global\let\@endtheorem=\endtrivlist
%    \end{macrocode}
% \end{macro}
%
%
%
% \begin{macro}{\@preamblecmds}
%    All macros defined above are to be used only in the preamble.
%    Therefore, we insert them in |\@preamblecmds| which will disable
%    them at begin document. This is done by the internal \LaTeXe{}
%    command |\@onlypreamble|.
% \changes{v2.0g}{89/09/19}{`theoremheaderfont and `theorembodyfont
%    added.}
% \changes{v2.2a}{94/02/03}{Use \LaTeXe{} interface}
%    \begin{macrocode}
\@onlypreamble\@xnthm
\@onlypreamble\@ynthm
\@onlypreamble\@othm
\@onlypreamble\newtheorem
\@onlypreamble\theoremstyle
\@onlypreamble\theorembodyfont
\@onlypreamble\theoremheaderfont
%    \end{macrocode}
% \end{macro}
%
% Finally we declare the \texttt{plain} theorem style to be the default.
% \changes{v2.1b}{90/06/04}{Fixed missing `item by declaring default
%    style.}
%    \begin{macrocode}
\theoremstyle{plain}
%</package>
%    \end{macrocode}
%
% \Finale

\endinput
}{}
\ProvidesFile{thb.sty}
             [\FMithmInfo]
%    \end{macrocode}
%
% \begin{macro}{\th@break}
%    |\theoremstyle{break}| produces a line break after the name
%    of the theorem.  The font is |\slshape|.  Hence, we define
%    |\th@break| as follows:
%    \begin{macrocode}
\gdef\th@break{\normalfont\slshape
  \def\@begintheorem##1##2{\item[%
%    \end{macrocode}
%    We run into the following problem: it is not possible to create
%    the header with |\item[|\meta{title}|]| and then start
%    a new line by, for example, |\mbox{}\\|. Such a definition
%    will fail whenever a list environment follows immediately.  With
%    the above construction, the |\mbox{}| causes the switch
%    |@inlabel| (cf.\ definition of |\list| and
%    |\trivlist| in \cite{latex:source}) to be set to \texttt{false}
%    and so the following list will insert additional vertical
%    space (|\topskip|). This is quite annoying. Therefore, we
%    create the line break within the |\item|. In order to ensure
%    that the text will begin at the proper position in the following
%    line, we simply pretend that the label does not take any
%    room.\footnote{This will lead to problems whenever very high
%    symbols occurring in the line tower into the
%    heading. So, something else has to be done here sometime.}
%    \begin{macrocode}
       \rlap{\vbox{\hbox{\hskip \labelsep\theorem@headerfont ##1\ ##2}%
                   \hbox{\strut}}}]}%
%    \end{macrocode}
%    Again, the definition of |\@opargbegintheorem| is
%    completely analogous.
%    \begin{macrocode}
\def\@opargbegintheorem##1##2##3{%
  \item[\rlap{\vbox{\hbox{\hskip \labelsep \theorem@headerfont
                     ##1\ ##2\ (##3)}%
                    \hbox{\strut}}}]}}
\endgroup
%</thb>
%    \end{macrocode}
% \end{macro}
%
%
% \subsubsection{The \texttt{changebreak} style}
%
%    \begin{macrocode}
% This style option is stored in the file |thcb.sty|.
%    \begin{macrocode}
%<*thcb>
\begingroup \makeatletter
\@ifundefined{theorem@style}{% \iffalse meta-comment
%
% Copyright (C) 1993-2024
%
% The LaTeX Project and any individual authors listed elsewhere
% in this file.
%
% This file is part of the Standard LaTeX `Tools Bundle'.
% -------------------------------------------------------
%
% It may be distributed and/or modified under the
% conditions of the LaTeX Project Public License, either version 1.3c
% of this license or (at your option) any later version.
% The latest version of this license is in
%    https://www.latex-project.org/lppl.txt
% and version 1.3c or later is part of all distributions of LaTeX
% version 2005/12/01 or later.
%
% The list of all files belonging to the LaTeX `Tools Bundle' is
% given in the file `manifest.txt'.
%
% \fi
%
%%
%
%
\def\FMithmInfo{2023/07/05 v2.2c Theorem extension package (FMi)}
%
% \ProvidesFile{theorem.dtx}[\FMithmInfo]
%\iffalse   % this is a METACOMMENT !
%%
%% Package `theorem' to use with LaTeX2e
%% Copyright (C) 1989-1995 Frank Mittelbach, all rights reserved.
%<+package>\@ifundefined{theorem@style}{}{\endinput}
%<+package>\NeedsTeXFormat{LaTeX2e}[1995/06/01]
%<+package>\ProvidesPackage{theorem}
%<+package>         [\FMithmInfo]
%
% \fi
%
% \changes{v1.0}{87/05/07}{First draft.}
% \changes{v1.0a}{87/05/07}{`thistheoremstyle' added.}
% \changes{v1.1a}{87/07/29}{New concept: redefine the internal
%                          \protect\LaTeX{} commands `@xnthm' etc.}
% \changes{v1.1b}{87/07/31}{`theoremfont' used to specify font.}
% \changes{v1.2a}{88/01/08}{`@endtheorem' now globally defined.}
% \changes{v1.2b}{88/01/11}{`newtheorem' only allowed in the preamble.}
% \changes{v1.3a}{88/02/14}{`theoremfont' now a toks register.}
%
% \changes{v2.0a}{88/11/16}{New implementation and new documentation.}
% \changes{v2.0a}{88/11/16}{No layout changes but internal modifications
%                           old substyles won't work.}
% \changes{v2.0b}{89/01/30}{`global' used for all definitions.}
% \changes{v2.0c}{89/06/15}{Documentation translated to English.
%                            Thanks to LLL and Nina}
% \changes{v2.1a}{90/02/15}{Support for new font selection.}
% \changes{v2.2a}{94/02/03}{Cleanup for LaTeX2e}
%
% \DoNotIndex{\@endparenv,\@ifdefinable,\@ifnextchar,\@ifundefined}
% \DoNotIndex{\@namedef,\@nameuse,\@spaces,\@tempa,\@warning}
% \DoNotIndex{\ }
% \DoNotIndex{\begingroup,\bfseries}
% \DoNotIndex{\csname}
% \DoNotIndex{\def,\do,\docdate}
% \DoNotIndex{\endcsname,\endgroup,\endinput,\endtrivlist,\expandafter}
% \DoNotIndex{\filedate,\fileversion}
% \DoNotIndex{\gdef,\global}
% \DoNotIndex{\hbox,\hskip}
% \DoNotIndex{\ignorespaces,\input,\itshape,\item}
% \DoNotIndex{\labelsep,\let,\llap}
% \DoNotIndex{\makeatletter}
% \DoNotIndex{\noexpand}
% \DoNotIndex{\refstepcounter,\relax,\rlap}
% \DoNotIndex{\setlength,\slshape,\space,\string,\strut}
% \DoNotIndex{\the,\trivlist,\typeout}
% \DoNotIndex{\vbox}
% \DoNotIndex{\xdef}
%
% \def\deflabel#1{\ttfamily #1\hfill}
% \def\deflist#1{\begin{list}{}{\settowidth\labelwidth{\ttfamily #1}%
%                               \setlength\leftmargin\labelwidth
%                               \addtolength\leftmargin\labelsep
%                               \let\makelabel\deflabel}}
% \def\enddeflist{\end{list}}
% \def\envfont{\normalfont\ttfamily}
%
% \MakeShortVerb{\|}
%
% \newcommand{\TUB}{{\sl TUGboat\/}}
%
% \GetFileInfo{theorem.dtx}
% \title{An Extension of the
%        \LaTeX{} theorem environment\thanks{This file
%        has version number \fileversion, last
%        revised \filedate.}}
% \author{Frank Mittelbach}
%
% \MaintainedByLaTeXTeam{tools}
% \maketitle
%
% \begin{abstract}
%    The macros described in this paper yield an extension of the
%    \LaTeX{} theorem mechanism. It is designed to satisfy the
%    different requirements of various journals. Thus, the
%    layout of the ``theorems'' can be manipulated by determining a
%    ``style''. This article describes not only the use, but
%    also the definition, of the necessary macros.
% \end{abstract}
%
% \section*{Preface to version 2.2}
%
% For \LaTeXe{} this package did not need any fundamental changes. I
% only modified the messages generated so that theorem layout styles
% will show up with the |\listfiles| command and cleaned the section
% on the New Font Selection Scheme since this is now included in
% \LaTeX.
%
% \section*{Preface to version 2.1}
%
% This version is identical to 2.0g described in \TUB\ 10\#3 except
% for some internal defaults which are now set depending on the used
% font selection scheme.
%
% This was done to avoid unpleasant surprises if the new font selection
% scheme is in force. For further details see
% section~\ref{sec:newfonts} and \cite{bk:GMS94}.
%
% \section{Introduction}
%
% For our purposes here, ``theorems'' are labelled enunciations,
% often set off from the main text by extra space and a font change.
% Theorems, corollaries, conjectures, definitions, and remarks
% are all instances of ``theorems''.  The ``header'' of these structures
% is composed of a label (such as \textsc{Theorem} or \textsc{Remark})
% and a number which serializes an item in the sequence of items
% with the same label.
%
% Shortly after the introduction of \LaTeX{} at the Fachbereich
% Mathematik in Mainz, the desire to manipulate the layout of
% ``theorems'' arose.  In Mainz, the following two conventions came
% into general use:
% \begin{enumerate}
%   \item The number of the theorem is shown in the margin.
%   \item There is a line break at
%         the end of the theorem header.
% \end{enumerate}
%
% Additionally, some journals require different formats which depend on
% the ``sort of theorem'': e.g.\ often remarks and definitions are set
% in |\upshape|, while |\itshape| is employed for main theorems.
%
% Confronted with these requirements, a theorem environment was
% developed in Mainz which allows separate determination of the
% layout of the ``theorems sets'', comparable to |\pagestyle|.
%
%
% \section{The user interface}
% \subsection{Defining new theorem sets}
%
% \DescribeMacro\newtheorem
% As in the original \LaTeX{} version, the command |\newtheorem|
% defines a new ``theorem set'' or ``theorem-like structure''.
% Two required arguments name the new environment and give the text
% to be typeset with each instance of the new ``set'', while
% an optional argument determines how
% the ``set'' is enumerated:
% \begin{description}
%    \item[\ttfamily \bslash newtheorem\{foo\}\{bar\}]
%       The theorem set {\envfont foo} (whose name is \texttt{bar})
%       uses its own counter.
%    \item[\ttfamily \bslash newtheorem\{foo2\}{[foo]}\{bar2\}]
%       The theorem set {\envfont foo2} (printed name \texttt{bar2})
%       uses the same counter as the theorem set \texttt{foo}.
%    \item[\ttfamily \bslash newtheorem\{foo3\}\{bar3\}{[section]}]
%       The theorem set {\envfont foo3} (printed name \texttt{bar3}) is
%       enumerated within the counter \texttt{section}, i.e.\ with every
%       new |\section| the enumeration begins again with 1, and
%       the enumeration is composed from the section-number and the
%       theorem counter itself.
% \end{description}
%
% \DescribeMacro\theoremstyle
% Additionally, the command |\theoremstyle| can define the layout
% of various, or all, theorem sets.  It should be noted that any
% theorem set defined by |\newtheorem| is typeset in the
% |\theoremstyle| that is current at the time of the definition.
% Thus, the following
% \begin{verbatim}
%    \theoremstyle{break}        \newtheorem{Cor}{Corollary}
%    \theoremstyle{plain}        \newtheorem{Exa}{Example}[section]
%\end{verbatim}                             ^^A No blank at line start
% leads to the result that the set {\envfont Cor} is formatted in the
% style \texttt{break}, while the set {\envfont Exa} and all the
% following ones are formatted in the style \texttt{plain}, unless
% another |\theoremstyle| follows.  Since the definitions installed by
% |\newtheorem| are global, one also can limit |\theoremstyle| locally
% by grouping braces.
%
% \DescribeMacro\theorembodyfont
% The choice of the font for the theorem body is completely
% independent of the chosen |\theoremstyle|; this has proven to
% be very advantageous.  For example,
% \begin{verbatim}
%   {\theorembodyfont{\upshape}          \newtheorem{Rem}{Remark}}
%\end{verbatim}
% defines a theorem set {\envfont Rem}, which will be set in
% |\upshape| in the current layout (which in our example is
% \texttt{plain}).  As with |\theoremstyle|, the |\theorembodyfont|
% chosen is that current at the time of |\newtheorem|.  If
% |\theorembodyfont| is not specified or one defines
% |\theorembodyfont{}|, then the font used will be that defined by the
% |\theoremstyle|.
%
% \DescribeMacro\theoremheaderfont It is also possible to customize
% the font used for the theorem headers.  This is, however, a global
% declaration, and therefore there should be at most one
% |\theoremheaderfont| declaration in the preamble.\footnote{If it is
% actually necessary to have different header fonts, one has to define
% new theorem styles (substituting the desired font) or specify the
% information directly in the \texttt{\bslash newtheorem} declaration
% (the unclean variant).}
%
% \DescribeMacro\theorempreskipamount
% \DescribeMacro\theorempostskipamount Two additional parameters
% affect the vertical space around the theorem environments:
% |\theorempreskipamount| and |\theorempostskipamount| define,
% respectively, the spacing before and after such an environment.
% These parameters apply for all theorem sets and can be manipulated
% with the ordinary length macros.  They are rubber lengths
% (`\textsf{skips}'), and therefore can contain \texttt{plus} and
% \texttt{minus} parts.
%
% Since the definition of theorem sets should---most sensibly---be
% placed in the preamble, we only allow installation there.
% It is therefore possible to release the memory used here after
% |\begin{document}|, in order to make room for other
% applications.
%
%
% \subsection{Existing theorem styles}
%
% The following theorem styles exist to date:
% \begin{deflist}{marginbreak}
%    \item[plain]
%       This theorem style emulates the original \LaTeX{} definition,
%       except that additionally the parameters
%       |\theorem...skipamount| are used.
%    \item[break]
%       In this style, the theorem header is followed by a line
%       break.
%    \item[marginbreak]
%       The theorem number is set in the margin, and there is a line
%       break as in \texttt{break}.
%    \item[changebreak]
%       Like \texttt{break}, but with header number and text
%       interchanged.
%    \item[change]
%       Header number and text are interchanged, without a line break.
%    \item[margin]
%       The number is set in the left margin, without a line break.
% \end{deflist}
% All styles (except \texttt{plain}) select |\slshape| as the default
% |\theorembodyfont|.
%
%
% \subsection{Examples}
%
% Given the above theorem sets {\envfont Cor}, {\envfont Exa}
% and {\envfont Rem},
% suppose that the preamble also contains the declarations:
% \begin{verbatim}
%    \theoremstyle{marginbreak}   \newtheorem{Lem}[Cor]{Lemma}
%    \theoremstyle{change}
%    \theorembodyfont{\itshape}        \newtheorem{Def}[Cor]{Definition}
%
%    \theoremheaderfont{\scshape}
%\end{verbatim}
% Then the following are some typical examples of the typeset output
% resulting from their use.
%
% \begin{Cor}
%    This is a sentence typeset in the theorem environment {\envfont
%    Cor}.
% \end{Cor}
%
% \begin{Exa}
%    This is a sentence typeset in the theorem environment {\envfont
%    Exa}.
% \end{Exa}
%
% \begin{Rem}
%    This is a sentence typeset in the theorem environment {\envfont
%    Rem}.
% \end{Rem}
%
% \begin{Lem}[Ben User]
%    This is a sentence typeset in the theorem environment {\envfont
%    Lem}.
% \end{Lem}
%
% \begin{Def}[Very impressive Definition]
%    This is a sentence typeset in the theorem environment {\envfont
%    Def}.
% \end{Def}
%
% The last two examples show the effect of the optional argument to
% a theorem environment (it is the text typeset in parentheses).
%
%
% \section{Special Considerations}
% \label{sec:newfonts}
%
% Theoremheader and body are implemented as a unit. This means that
% the |\theoremheaderfont|  will inherit characteristics of the
% |\theorembodyfont| in \LaTeXe. Thus, if for example
% |\theorembodyfont| is |\itshape| and |\theoremheaderfont|
% is |\bfseries| the font selected for the header will have the
% characteristics `bold extended italic'. If this is not desired one
% should set the |\theoremheaderfont| to something like
% \begin{verbatim}
%      \theoremheaderfont{\normalfont\bfseries}
%\end{verbatim}
% i.e.\ supplying all necessary font information explicitly.
%
%
%
% \section{Acknowledgements}
%
% The publication of this set of macros was only possible with the
% help of Christina Busse (translating the manuscript into English),
% Joachim Pense (playing the r\^ole of typist), Chris Rowley
% (looking everything over) and many others providing useful
% suggestions.
%
% \MaybeStop{%
%    \begin{thebibliography}{9}
%       \bibitem{bk:GMS94} \textsc{M.~Goossens}, \textsc{F.~Mittelbach}
%         and \textsc{A.~Samarin}.
%         \newblock The \LaTeX{} Companion.
%         \newblock
%         Addison-Wesley, Reading, Massachusetts, 1994.
%       \bibitem{latex:source} \textsc{Lamport, Leslie.} \newblock
%                               \texttt{latex.tex}, version 2.09, date
%                               Feb. 1990.
%    \end{thebibliography}
%    \PrintIndex
%    \PrintChanges}
%
%
% \section{The documentation driver file}
%
% The next bit of code contains the documentation driver file for
% \TeX{}, i.e., the file that will produce the documentation you are
% currently reading. It will be extracted from this file by the
% \texttt{docstrip} program. Since it is the first code in the file
% one can alternatively process this file directly with \LaTeXe{} to
% obtain the documentation.
%
%    \begin{macrocode}
%<*driver>
\documentclass{ltxdoc}

\usepackage{theorem}

%<+driver>% The next few lines define theorem sets which are used
%<+driver>% in the example section of the documentation.

\theoremstyle{break}       \newtheorem{Cor}{Corollary}
\theoremstyle{plain}       \newtheorem{Exa}{Example}[section]
{\theorembodyfont{\upshape}\newtheorem{Rem}{Remark}}
\theoremstyle{marginbreak} \newtheorem{Lem}[Cor]{Lemma}
\theoremstyle{change}
\theorembodyfont{\itshape} \newtheorem{Def}[Cor]{Definition}

\theoremheaderfont{\scshape}

\RecordChanges

\begin{document}
   \DocInput{theorem.dtx}
\end{document}
%</driver>
%    \end{macrocode}
%
% \section{Definition of the Macros}
%
% If the file has been loaded before, we abort immediately. If not the
% package announces itself (this is actually done at the very top of
% the file---the way it is done isn't good style so don't copy it).
% \changes{v2.0e}{89/07/19}{Spaces removed from `typeout'.}
% \changes{v2.2b}{95/11/19}{Announce moved to top of file}
%    \begin{macrocode}
%<*package>
%\@ifundefined{theorem@style}{}{\endinput}
%\def\FMithmInfo{1995/11/19 v2.2b Theorem extension package (FMi)}
%\ProvidesPackage{theorem}[\FMithmInfo]
%    \end{macrocode}
%
%
% \subsection{Definition of theorem styles and fonts}
%
% All the definitions in this file are done globally to allow
% inputting this file inside a group.
%
% \begin{macro}{\theoremstyle}
%    Before a theorem style can be installed, the chosen style must be
%    known. For that reason, we must test to see that
%    |\th@|\meta{style} is known or, more precisely, that it is
%    different from |\relax|. If the style is not known then
%    |\th@plain| is used.
%    \begin{macrocode}
\gdef\theoremstyle#1{%
   \@ifundefined{th@#1}{\@warning
          {Unknown theoremstyle `#1'. Using `plain'}%
          \theorem@style{plain}}%
%    \end{macrocode}
%    We save the theorem style to be used in the token register
%    |\theorem@style|.
%    \begin{macrocode}
      {\theorem@style{#1}}%
%    \end{macrocode}
%    Now we ``evaluate'' the theorem style: this means, we call the
%    macro |\th@|\meta{style} which will activate the relevant
%    definitions which are contained in a separate file.  This is done
%    in a group to suppress changes to the current font.  This could
%    otherwise pose problems together with the new font selection
%    scheme\footnote{When I printed the original article using the new
%    font selection scheme I ended with a document with slanted
%    typefaces (text headings and all) simply because one of the
%    theorem styles used \texttt{\bslash sl} at toplevel.} if the
%    |\th@|\meta{style} is evaluated a second time.
% \changes{v2.1a}{90/02/15}{Group added around `th@..}
% \changes{v2.1a}{90/02/15}{Reseting of `..@begintheorem removed.}
%    \begin{macrocode}
      \begingroup
        \csname th@\the\theorem@style \endcsname
      \endgroup}
%    \end{macrocode}
% \end{macro}
%
%
% \begin{macro}{\@begintheorem}
% \changes{v2.1a}{90/02/15}{Relaxed at toplevel.}
% \begin{macro}{\@opargbegintheorem}
% \changes{v2.1a}{90/02/15}{Relaxed at toplevel.}
%    We reset |\@begintheorem| and |\@opargbegintheorem| to
%    |\relax| since these commands are no longer necessary at
%    toplevel. This will save a few tokens.
%    \begin{macrocode}
\global\let\@begintheorem\relax
\global\let\@opargbegintheorem\relax
%    \end{macrocode}
% \end{macro}
% \end{macro}
%
% \begin{macro}{\theorem@style}
%    Obviously the \textsf{token} register used above has to be
%    allocated. To assure the utmost compatibility with the
%    original \LaTeX{} definition, we set the default theorem style to
%    \texttt{plain}, which implements the usual \LaTeX{} convention.
% \changes{v2.0f}{89/07/20}{Unnecessary `global' removed.}
%    \begin{macrocode}
\newtoks\theorem@style
\global\theorem@style{plain}
%    \end{macrocode}
% \end{macro}
%
% \begin{macro}{\theorembodyfont}
% \begin{macro}{\theorem@bodyfont}
%    If the |\theorembodyfont| is set by the user then it should not
%    interact with the default font set in the theorem style. When the
%    new font selection is in force this may happen if, for example,
%    the default is |\itshape| and the new |\theorembodyfont| is
%    |\sffamily|. So we add a |\reset@font| command in front of the
%    user definition.
% \changes{v2.0f}{89/07/20}{Unnecessary `global' removed.}
% \changes{v2.0g}{89/09/19}{`theoremfont to `theorembodyfont.}
% \changes{v2.1c}{92/06/22}{`reset@font added and logic changed.}
% \changes{v2.1d}{92/10/28}{Reset to empty if arg is empty}
%    \begin{macrocode}
\gdef\theorembodyfont#1{%
%    \end{macrocode}
%    We check if the argument supplied is empty and if so put nothing
%    into the |\theorem@bodyfont| token register to allow for
%    |\theorembodyfont{}| as a mean of using the default of the
%    current |\theoremstyle|.
%    \begin{macrocode}
   \def\@tempa{#1}%
   \ifx\@tempa\@empty
    \theorem@bodyfont{}%
   \else
    \theorem@bodyfont{\reset@font#1}%
   \fi
   }
\newtoks\theorem@bodyfont
\global\theorem@bodyfont{}
%    \end{macrocode}
% \end{macro}
% \end{macro}
%
%
% \begin{macro}{\theoremheaderfont}
% \changes{v2.0g}{89/09/19}{Macro added.}
%    The font for the theorem headers is handled differently because
%    this definition applies to all theorem styles.
%    \begin{macrocode}
\gdef\theoremheaderfont#1{\gdef\theorem@headerfont{#1}%
%    \end{macrocode}
%    After using the macro once it is redefined to produce an error
%    message.
%    \begin{macrocode}
       \gdef\theoremheaderfont##1{%
        \typeout{\string\theoremheaderfont\space should be used
                 only once.}}}
%    \end{macrocode}
% \end{macro}
%
% \begin{macro}{\theorem@headerfont}
% \changes{v2.0g}{89/09/19}{Macro added.}
%    To set the |\theorem@headerfont| default we first test if
%    the new fontselection scheme is in force.
% \changes{v2.1a}{90/02/15}{Support for new font selection.}
%    \begin{macrocode}
\ifx\upshape\undefined
%    \end{macrocode}
%    If not we define it to expand into |\bfseries|. We don't use
%    |\let| just in case a following style option redefines this
%    macro.
%    \begin{macrocode}
\gdef\theorem@headerfont{\bfseries}
%    \end{macrocode}
%    Otherwise we reset the current shape before calling |\bfseries|.
%    \begin{macrocode}
\else \gdef\theorem@headerfont{\normalfont\bfseries}\fi
%    \end{macrocode}
% \end{macro}
%
% \begin{macro}{\th@plain}
% \begin{macro}{\th@break}
% \begin{macro}{\th@marginbreak}
% \begin{macro}{\th@changebreak}
% \begin{macro}{\th@change}
% \begin{macro}{\th@margin}
%    The different styles are defined in macros such as
%    |\th@plain|.  Since memory space is precious in
%    ``non-Big-versions'', we have to avoid offering too many unused
%    definitions.  Therefore we define these styles in separate files
%    that can be loaded on demand.  Thus the commands themselves only
%    load these files. We use |\@input@|, a \LaTeXe{} internal command
%    that ensures that the file will be listed with |\listfiles|.
% \changes{v2.2a}{94/02/02}{Use `@input@ to load theorem layout files}
%    \begin{macrocode}
\gdef\th@plain{\@input@{thp.sty}}
\gdef\th@break{\@input@{thb.sty}}
\gdef\th@marginbreak{\@input@{thmb.sty}}
\gdef\th@changebreak{\@input@{thcb.sty}}
\gdef\th@change{\@input@{thc.sty}}
\gdef\th@margin{\@input@{thm.sty}}
%    \end{macrocode}
%    This list will be expanded when new styles become
%    available.  For testing,  just append new theorem substyles
%    as document options.
% \end{macro}
% \end{macro}
% \end{macro}
% \end{macro}
% \end{macro}
% \end{macro}
%
%
%
% \subsection{Definition of a new theorem set}
%
% As already pointed out, a new theorem environment can be defined in
% three different ways:
% \begin{verbatim}
%    \newtheorem{Lem}{Lemma}
%    \newtheorem{Lem}{Lemma}[section]
%    \newtheorem{Lem}[Theorem]{Lemma}
%\end{verbatim}
% The function of the macro |\newtheorem| is to recognize these cases
% and then to branch into one of the three macros |\@ynthm|, |\@xnthm|
% or |\@othm|.  This mechanism is adopted unchanged from
% \cite{latex:source}; the essential point here is that, for example,
% in the second case, the arguments \texttt{Lem}, \texttt{Lemma} and
% \texttt{section} are passed over to the macro |\@xnthm|.
%
% We inspect this case first because the others present fewer problems,
% and thus are easily derived from this one.
%
% \begin{macro}{\@xnthm}
%    For our example arguments, the macro |\@xnthm| must fulfill
%    the following:
%    \begin{itemize}
%       \item
%          Define a new \LaTeX-counter `Lem'
%       \item
%          reset this counter within a |\section|
%       \item
%          define the macro |\theLem|
%       \item
%          define the environment macros |\Lem| and
%          |\endLem| using the current
%          |\theoremstyle| and |\theorem@bodyfont|.
%    \end{itemize}
%    Obviously, all this should happen only if the first argument
%    of |\@xnthm| (i.e.\ \texttt{Lem} in our example) is chosen so
%    as not to conflict with any previously defined commands or
%    environments.  This test is performed by the \LaTeX{} macro
%    |\@ifdefinable|.
%    \begin{macrocode}
\gdef\@xnthm#1#2[#3]{\expandafter\@ifdefinable\csname #1\endcsname
%    \end{macrocode}
%    Therefore, the first argument of |\@ifdefinable| is the
%    expansion (in the example, |\Lem|) of
%    |\csname||#1\endcsname|.  The second argument is
%    executed only if the test has been completed successfully.
%    \begin{macrocode}
   {%
%    \end{macrocode}
%    Now we define the new counter. The names of the \LaTeX{}
%    macros employed should speak for themselves:
% \changes{v2.2b}{1995/10/07}{Use \cs{@newctr}}
% \changes{v2.2c}{1995/11/23}{Use \cs{@newctr} properly.}
%    \begin{macrocode}
    \@definecounter{#1}\@newctr{#1}[#3]%
%    \end{macrocode}
%    Using |\@newctr| will give a proper error message if the counter
%    in |#3| is not defined.
%    In defining `|\theLem|' we must generate the desired macro
%    name by use of |\expandafter| and |\csname|.
%    \begin{macrocode}
    \expandafter\xdef\csname the#1\endcsname
%    \end{macrocode}
%    An |\xdef| is used in order to make the definition global,
%    and to ensure that it contains the replacement texts of
%    |\@thmcountersep| and |\@thmcounter|.\footnote{These
%    two macros can be defined by the document style.  Their default
%    values produce a `\texttt{.}' as separation and an arabic
%    representation of the number.} However, not everything should be
%    expanded. For example, it saves space to use |\thesection|
%    instead of its---at times---lengthy expansion.
%    \begin{macrocode}
      {\expandafter \noexpand \csname the#3\endcsname
       \@thmcountersep \@thmcounter{#1}}%
%    \end{macrocode}
%    Thus with the defaults of \LaTeX{}, |\theLem| would be
%    replaced by the command sequence |\thesection.\arabic{Lem}|.
%
%    We will now look at the definition of the macro which is executed
%    at the beginning of the actual environment (in our example this
%    macro is |\Lem|). It should be noted that we use an
%    ``|\expandafter| trick'' to expand only certain parts of the
%    replacement text at the time of the definition.
%    \begin{macrocode}
    \def\@tempa{\global\@namedef{#1}}%
    \expandafter \@tempa \expandafter{%
%    \end{macrocode}
%    First, the macro that contains the current definitions of
%    |\@begintheorem| and |\@opargtheorem| should be called
%    up.  The name of this macro---as is already known---has the form
%    |\th@|\meta{theorem\ style}; therefore, it must be called by
%    \begin{macrocode}
      \csname th@\the \theorem@style
%    \end{macrocode}
%    In addition the default theorem font should be changeable, i.e.\
%    we have to insert the contents of |\theorem@bodyfont|. For
%    that reason, we expand even further, beyond |\endcsname|,
%    and thus insert the contents of the \textsf{token} register
%    |\theorem@bodyfont| in the replacement text.
%    \begin{macrocode}
            \expandafter \endcsname \the \theorem@bodyfont
%    \end{macrocode}
%    Now it is time to call the macro |\@thm| which takes over
%    the further processing.  It has two arguments: the current
%    counter name (in our example, \texttt{Lem}), and the text of the
%    label (in our example, \texttt{Lemma}).
%    \begin{macrocode}
     \@thm{#1}{#2}}%
%    \end{macrocode}
%    With this, the `sub-definition' is complete.  The macro
%    |\@endtheorem| ends a theorem environment and is, so far, nothing
%    but an |\endtrivlist|.  (Hence it is defined globally, and not
%    within the theorem styles.\footnote{This has to be changed as
%    soon as theorem styles that change \texttt{\bslash @endtheorem}
%    exist.  In such a case, all existing styles must be changed as
%    well since they will have to reset the macro.}) Therefore, we can
%    set it equivalent to the macro that ends the theorem set (in our
%    example, |\endLem|).  However, if some day theorem styles exist
%    that do change |\@endtheorem|, we would have to use the
%    commented-out line instead.
%    \begin{macrocode}
    \global \expandafter \let \csname end#1\endcsname \@endtheorem
%   \global\@namedef{end#1}{\@endtheorem}%
%    \end{macrocode}
%    With these commands all the required definitions are employed,
%    unless the test |\@ifdefinable| has failed.  Therefore, we
%    end the second argument of this macro and with it the definition
%    of |\@xnthm|.
%    \begin{macrocode}
   }}
%    \end{macrocode}
% \end{macro}
%
% \begin{macro}{\@ynthm}
%    The definition of |\@ynthm| is completely analogous. In
%    this case the new counter that is defined is not reset within
%    another counter; thus the definition of |\the...|
%    is simplified:
%    \begin{macrocode}
\gdef\@ynthm#1#2{\expandafter\@ifdefinable\csname #1\endcsname
   {\@definecounter{#1}%
    \expandafter\xdef\csname the#1\endcsname{\@thmcounter{#1}}%
%    \end{macrocode}
%    The rest of the definition corresponds literally to that of
%    |\@xnthm|:
%    \begin{macrocode}
    \def\@tempa{\global\@namedef{#1}}\expandafter \@tempa
     \expandafter{\csname th@\the \theorem@style \expandafter
     \endcsname \the\theorem@bodyfont \@thm{#1}{#2}}%
    \global \expandafter \let \csname end#1\endcsname \@endtheorem}}
%    \end{macrocode}
% \end{macro}
%
% \begin{macro}{\@othm}
%    The definition of |\@othm| does not contain anything new.
%    \begin{macrocode}
\gdef\@othm#1[#2]#3{%
%    \end{macrocode}
% \changes{v2.2b}{95/10/07}{Build in check for counter defined}
%    We do not define a new counter but instead use one that has
%    already been defined.  Thus the only definition we need is that
%    of this pseudo-counter (i.e.\ |\the|\meta{env.\ name}).
%    First we check if |#2| corresponds to a known counter name.
%    \begin{macrocode}
  \expandafter\ifx\csname c@#2\endcsname\relax
   \@nocounterr{#2}%
  \else
   \expandafter\@ifdefinable\csname #1\endcsname
   {\expandafter \xdef \csname the#1\endcsname
     {\expandafter \noexpand \csname the#2\endcsname}%
%    \end{macrocode}
%    All other parts of the definition can be adopted from
%    |\@xnthm|. We have to remember, though, that in this case
%    the name of the current counter and the theorem label have moved
%    to the second and third arguments.
%    \begin{macrocode}
    \def\@tempa{\global\@namedef{#1}}\expandafter \@tempa
     \expandafter{\csname th@\the \theorem@style \expandafter
     \endcsname \the\theorem@bodyfont \@thm{#2}{#3}}%
    \global \expandafter \let \csname end#1\endcsname \@endtheorem}%
  \fi}
%    \end{macrocode}
% \end{macro}
%
% \subsection{Macros that are employed in a {\envfont theorem}
%    environment}
%
% \begin{macro}{\@thm}
%    The macro |\@thm| has to increase the current counter.  Then,
%    depending on whether the environment has (or does not have) an
%    optional argument, it has to branch into either |\@begintheorem|
%    or |\@opargtheorem|.
%    \begin{macrocode}
\gdef\@thm#1#2{\refstepcounter{#1}%
%    \end{macrocode}
%    Now we start a {\envfont trivlist} environment, and give
%    |\@topsep| and |\@topsepadd| the values of the \textsf{skip}
%    registers |\theorempreskipamount| and |\theorempostskipamount|.
%    The value in |\@topsep| is the vertical space that is inserted by
%    the first (and only) |\item| in our |\trivlist| whilst
%    |\@topsepadd| is inserted by |\@endparenv| at the end of that
%    {\envfont trivlist} environment.  By using these registers, we
%    obtain the desired space around a {\envfont theorem} environment.
%    \begin{macrocode}
   \trivlist
   \@topsep \theorempreskipamount               % used by first \item
   \@topsepadd \theorempostskipamount           % used by \@endparenv
%    \end{macrocode}
%    Now we have to test whether an optional argument has been given.
% \changes{v2.0d}{89/07/18}{Forgotten percent added.}
%    \begin{macrocode}
   \@ifnextchar [%
%    \end{macrocode}
%    If there is an optional argument, we will call |\@ythm|, and
%    move the arguments read back into the input stream.
%    \begin{macrocode}
   {\@ythm{#1}{#2}}%
%    \end{macrocode}
%    If not, we call |\@begintheorem|. Its first argument is the
%    name of the theorem set (hence the second argument of
%    |\@thm|). Its second argument is the macro that produces the
%    current number.
%    \begin{macrocode}
   {\@begintheorem{#2}{\csname the#1\endcsname}\ignorespaces}}
%    \end{macrocode}
% \end{macro}
%
% \begin{macro}{\@xthm}
% \begin{macro}{\@ythm}
%    Both these macros were originally called by |\@thm|.  We do
%    not need |\@xthm| anymore, hence we reset it to
%    |\relax|. The definition of |\@ythm| has not changed at
%    all from its definition in \LaTeX. In order to make the macros
%    easier to understand, we will
%    nevertheless present it (commented out).
%    \begin{macrocode}
\global\let\@xthm\relax
% \def\@ythm#1#2[#3]{\@opargbegintheorem{#2}{\csname
%           the#1\endcsname}{#3}\ignorespaces}
%</package>
%    \end{macrocode}
%    The primitive |\ignorespaces| in |\@ythm| and
%    |\@thm| is needed to remove the spaces between the
%    |\begin{...}| and the actual text.
% \end{macro}
% \end{macro}
%
%
% \subsection{Definition of the theorem substyles}
%
% As already pointed out, the theorem substyles, defined below, are
% only loaded when necessary.  Note that all these substyles, except
% \texttt{plain}, have |\slshape| as the default body font.
%
% \subsubsection{The \texttt{plain} style}
%
% \label{dok:th@plain} As the following macros use |@|, we have
% to locally set the |\catcode| of this symbol to ``letter''.
% This happens within a group, so that we do not have to worry about
% which |\catcode| that symbol had before.
%    \begin{macrocode}
%<*thp>
\begingroup \makeatletter
%    \end{macrocode}
% Since we are now within a group, we must make all definitions
% globally.  First we make sure that \texttt{theorem.sty} is loaded.
% This will allow us to use this file as a document style option
% without having to call \texttt{theorem} itself as an option.  At the
% same time, we assure that at least version 2 is loaded, since
% |\theorem@style| was not defined in earlier versions.
%    \begin{macrocode}
\@ifundefined{theorem@style}{\input{theorem.sty}}{}
\ProvidesFile{thp.sty}
             [\FMithmInfo]
%    \end{macrocode}
%
%
% \begin{macro}{\th@plain}
% \begin{macro}{\@begintheorem}
% \begin{macro}{\@opargbegintheorem}
%    |\theoremstyle{plain}| corresponds to the original
%    definition, except that the distances to the surrounding text are
%    determined by the parameters |\theorempreskipamount| and
%    |\theorempostskipamount|.
%    First we set the default body font.
%    \begin{macrocode}
\gdef\th@plain{\normalfont\itshape
%    \end{macrocode}
%    Then we define |\@begintheorem| and
%    |\@opargbegintheorem|. These two macros define how the
%    header of a theorem is typeset.  |\@opargbegintheorem| will
%    be called if a \textsf{theorem} environment with an optional
%    argument is encountered; otherwise, the header is constructed by
%    calling |\@begintheorem|.  If one of these macros is
%    executed, we are within a \textsf{trivlist} environment started by
%    |\@thm|.  So the theorem header is produced with an
%    |\item| command.
%
%    Instead of specifying the header font directly, all standard
%    theorem styles use the |\theorem@headerfont| macro to allow
%    customization.  The extra space (|\labelsep|) is necessary
%    because of problems in the \textsf{trivlist} environment.
%    \begin{macrocode}
  \def\@begintheorem##1##2{%
        \item[\hskip\labelsep \theorem@headerfont ##1\ ##2]}%
%    \end{macrocode}
%    The definition of |\@opargbegintheorem| is completely
%    analogous. The only difference is the fact that there exists a
%    third argument (which is the optional parameter of the
%    environment and contains additional information about the
%    theorem).  Customarily we enclose it in parentheses.
%    \begin{macrocode}
\def\@opargbegintheorem##1##2##3{%
   \item[\hskip\labelsep \theorem@headerfont ##1\ ##2\ (##3)]}}
%    \end{macrocode}
% \end{macro}
% \end{macro}
% \end{macro}
%
% \noindent
% We conclude with an |\endgroup| to restore the
% |\catcode| of |@|.
%    \begin{macrocode}
\endgroup
%</thp>
%    \end{macrocode}
%
%
% \subsubsection{The \texttt{break} style}
%
% This style option is stored in the file |thb.sty|.
% For the next two lines see the documentation for |\th@plain| on
% page~\pageref{dok:th@plain}.
%    \begin{macrocode}
%<*thb>
\begingroup \makeatletter
\@ifundefined{theorem@style}{\input{theorem.sty}}{}
\ProvidesFile{thb.sty}
             [\FMithmInfo]
%    \end{macrocode}
%
% \begin{macro}{\th@break}
%    |\theoremstyle{break}| produces a line break after the name
%    of the theorem.  The font is |\slshape|.  Hence, we define
%    |\th@break| as follows:
%    \begin{macrocode}
\gdef\th@break{\normalfont\slshape
  \def\@begintheorem##1##2{\item[%
%    \end{macrocode}
%    We run into the following problem: it is not possible to create
%    the header with |\item[|\meta{title}|]| and then start
%    a new line by, for example, |\mbox{}\\|. Such a definition
%    will fail whenever a list environment follows immediately.  With
%    the above construction, the |\mbox{}| causes the switch
%    |@inlabel| (cf.\ definition of |\list| and
%    |\trivlist| in \cite{latex:source}) to be set to \texttt{false}
%    and so the following list will insert additional vertical
%    space (|\topskip|). This is quite annoying. Therefore, we
%    create the line break within the |\item|. In order to ensure
%    that the text will begin at the proper position in the following
%    line, we simply pretend that the label does not take any
%    room.\footnote{This will lead to problems whenever very high
%    symbols occurring in the line tower into the
%    heading. So, something else has to be done here sometime.}
%    \begin{macrocode}
       \rlap{\vbox{\hbox{\hskip \labelsep\theorem@headerfont ##1\ ##2}%
                   \hbox{\strut}}}]}%
%    \end{macrocode}
%    Again, the definition of |\@opargbegintheorem| is
%    completely analogous.
%    \begin{macrocode}
\def\@opargbegintheorem##1##2##3{%
  \item[\rlap{\vbox{\hbox{\hskip \labelsep \theorem@headerfont
                     ##1\ ##2\ (##3)}%
                    \hbox{\strut}}}]}}
\endgroup
%</thb>
%    \end{macrocode}
% \end{macro}
%
%
% \subsubsection{The \texttt{changebreak} style}
%
%    \begin{macrocode}
% This style option is stored in the file |thcb.sty|.
%    \begin{macrocode}
%<*thcb>
\begingroup \makeatletter
\@ifundefined{theorem@style}{\input{theorem.sty}}{}
\ProvidesFile{thcb.sty}
             [\FMithmInfo]
%    \end{macrocode}
%
% \begin{macro}{\th@changebreak}
%    The \texttt{change-break} theorem style is like \texttt{break}
%    but with interchange of theorem name and theorem number.  Thus we
%    define |\th@changebreak| as follows:
%    \begin{macrocode}
\gdef\th@changebreak{\normalfont\slshape
  \def\@begintheorem##1##2{\item
       [\rlap{\vbox{\hbox{\hskip\labelsep \theorem@headerfont ##2\ ##1}%
                    \hbox{\strut}}}]}%
\def\@opargbegintheorem##1##2##3{%
  \item[\rlap{\vbox{\hbox{\hskip\labelsep \theorem@headerfont
                     ##2\ ##1\ (##3)}%
                    \hbox{\strut}}}]}}
\endgroup
%</thcb>
%    \end{macrocode}
% \end{macro}
%
%
% \subsubsection{The \texttt{change} style}
%
% This style option is stored in the file |thc.sty|.
%    \begin{macrocode}
%<*thc>
\begingroup \makeatletter
\@ifundefined{theorem@style}{\input{theorem.sty}}{}
\ProvidesFile{thc.sty}
             [\FMithmInfo]
%    \end{macrocode}
%
% \begin{macro}{\th@change}
%    The \texttt{change} theorem style corresponds to the
%    \texttt{change break} style without a linebreak after the header.
%    To say it in another way, it's the same as the \texttt{plain}
%    style but with number and name interchanged and |\slshape| as the
%    default font.
%    \begin{macrocode}
\gdef\th@change{\normalfont\slshape
  \def\@begintheorem##1##2{\item
       [\hskip\labelsep \theorem@headerfont ##2\ ##1]}%
\def\@opargbegintheorem##1##2##3{%
  \item[\hskip\labelsep \theorem@headerfont ##2\ ##1\ (##3)]}}
\endgroup
%</thc>
%    \end{macrocode}
% \end{macro}
%
%
% \subsubsection{The \texttt{marginbreak} style}
%
% This style option is the one used most often at Mainz.  It is saved
% in the file |thmb.sty|.
%    \begin{macrocode}
%<*thmb>
\begingroup \makeatletter
\@ifundefined{theorem@style}{\input{theorem.sty}}{}
\ProvidesFile{thmb.sty}
             [\FMithmInfo]
%    \end{macrocode}
%
% \begin{macro}{\th@marginbreak}
%    The \texttt{margin break} style is nearly the same as the
%    \texttt{change break} style. The only difference is the placement
%    of the theorem number.  We use |\llap| to place it in the left
%    margin.
%
%    In this style |\labelsep| denotes the separation between the
%    number and the text.
%    \begin{macrocode}
\gdef\th@marginbreak{\normalfont\slshape
  \def\@begintheorem##1##2{\item
       [\rlap{\vbox{\theorem@headerfont
                    \hbox{\llap{##2}\hskip\labelsep ##1}%
                    \hbox{\strut}}}]}%
\def\@opargbegintheorem##1##2##3{%
  \item[\rlap{\vbox{\theorem@headerfont
                    \hbox{\llap{##2}\hskip\labelsep ##1\ (##3)}%
                    \hbox{\strut}}}]}}
\endgroup
%</thmb>
%    \end{macrocode}
% \changes{v2.1c}{90/02/17}{`endgroup outside macrocode env.}
% \end{macro}
%
%
%
% \subsubsection{The \texttt{margin} style}
%
% This style option is stored in the file |thm.sty|.
%    \begin{macrocode}
%<*thm>
\begingroup \makeatletter
\@ifundefined{theorem@style}{\input{theorem.sty}}{}
\ProvidesFile{thm.sty}
             [\FMithmInfo]
%    \end{macrocode}
%
% \begin{macro}{\th@margin}
%    Again this is only a variant of the theorem styles described
%    above without any new ideas.
%    \begin{macrocode}
\gdef\th@margin{\normalfont\slshape
  \def\@begintheorem##1##2{\item
       [\theorem@headerfont \llap{##2}\hskip\labelsep ##1]}%
\def\@opargbegintheorem##1##2##3{%
  \item[\theorem@headerfont \llap{##2}\hskip\labelsep ##1\ (##3)]}}
\endgroup
%</thm>
%    \end{macrocode}
% \end{macro}
%
%
%
%
% \subsection{Final Definitions}
%
% \begin{macro}{\theorempreskipamount}
% \begin{macro}{\theorempostskipamount}
%    The \textsf{skip} parameters that regulate the vertical empty space
%    before and after the {\envfont theorem} environment have
%    to be allocated as well.
% \changes{v2.0f}{89/07/20}{Unnecessary `global' removed.}
%    \begin{macrocode}
%<*package>
\newskip\theorempreskipamount
\newskip\theorempostskipamount
%    \end{macrocode}
%    Since we have used the same values for all theorem sets, we now can
%    assign them.
%    \begin{macrocode}
\global\setlength\theorempreskipamount{12pt plus 5pt minus 3pt}
\global\setlength\theorempostskipamount{8pt plus 3pt minus 1.5pt}
%    \end{macrocode}
% \end{macro}
% \end{macro}
%
%
% \begin{macro}{\@endtheorem}
%    The same holds for the macro |\@endtheorem|, which ends a
%    {\envfont theorem} environment. Since it is the same for all
%    theorem sets, it is removed from the macros |\th@|\meta{style}.
%    It simply ends the {\envfont trivlist} environment, which was
%    begun in |\@thm|.
%    \begin{macrocode}
\global\let\@endtheorem=\endtrivlist
%    \end{macrocode}
% \end{macro}
%
%
%
% \begin{macro}{\@preamblecmds}
%    All macros defined above are to be used only in the preamble.
%    Therefore, we insert them in |\@preamblecmds| which will disable
%    them at begin document. This is done by the internal \LaTeXe{}
%    command |\@onlypreamble|.
% \changes{v2.0g}{89/09/19}{`theoremheaderfont and `theorembodyfont
%    added.}
% \changes{v2.2a}{94/02/03}{Use \LaTeXe{} interface}
%    \begin{macrocode}
\@onlypreamble\@xnthm
\@onlypreamble\@ynthm
\@onlypreamble\@othm
\@onlypreamble\newtheorem
\@onlypreamble\theoremstyle
\@onlypreamble\theorembodyfont
\@onlypreamble\theoremheaderfont
%    \end{macrocode}
% \end{macro}
%
% Finally we declare the \texttt{plain} theorem style to be the default.
% \changes{v2.1b}{90/06/04}{Fixed missing `item by declaring default
%    style.}
%    \begin{macrocode}
\theoremstyle{plain}
%</package>
%    \end{macrocode}
%
% \Finale

\endinput
}{}
\ProvidesFile{thcb.sty}
             [\FMithmInfo]
%    \end{macrocode}
%
% \begin{macro}{\th@changebreak}
%    The \texttt{change-break} theorem style is like \texttt{break}
%    but with interchange of theorem name and theorem number.  Thus we
%    define |\th@changebreak| as follows:
%    \begin{macrocode}
\gdef\th@changebreak{\normalfont\slshape
  \def\@begintheorem##1##2{\item
       [\rlap{\vbox{\hbox{\hskip\labelsep \theorem@headerfont ##2\ ##1}%
                    \hbox{\strut}}}]}%
\def\@opargbegintheorem##1##2##3{%
  \item[\rlap{\vbox{\hbox{\hskip\labelsep \theorem@headerfont
                     ##2\ ##1\ (##3)}%
                    \hbox{\strut}}}]}}
\endgroup
%</thcb>
%    \end{macrocode}
% \end{macro}
%
%
% \subsubsection{The \texttt{change} style}
%
% This style option is stored in the file |thc.sty|.
%    \begin{macrocode}
%<*thc>
\begingroup \makeatletter
\@ifundefined{theorem@style}{% \iffalse meta-comment
%
% Copyright (C) 1993-2024
%
% The LaTeX Project and any individual authors listed elsewhere
% in this file.
%
% This file is part of the Standard LaTeX `Tools Bundle'.
% -------------------------------------------------------
%
% It may be distributed and/or modified under the
% conditions of the LaTeX Project Public License, either version 1.3c
% of this license or (at your option) any later version.
% The latest version of this license is in
%    https://www.latex-project.org/lppl.txt
% and version 1.3c or later is part of all distributions of LaTeX
% version 2005/12/01 or later.
%
% The list of all files belonging to the LaTeX `Tools Bundle' is
% given in the file `manifest.txt'.
%
% \fi
%
%%
%
%
\def\FMithmInfo{2023/07/05 v2.2c Theorem extension package (FMi)}
%
% \ProvidesFile{theorem.dtx}[\FMithmInfo]
%\iffalse   % this is a METACOMMENT !
%%
%% Package `theorem' to use with LaTeX2e
%% Copyright (C) 1989-1995 Frank Mittelbach, all rights reserved.
%<+package>\@ifundefined{theorem@style}{}{\endinput}
%<+package>\NeedsTeXFormat{LaTeX2e}[1995/06/01]
%<+package>\ProvidesPackage{theorem}
%<+package>         [\FMithmInfo]
%
% \fi
%
% \changes{v1.0}{87/05/07}{First draft.}
% \changes{v1.0a}{87/05/07}{`thistheoremstyle' added.}
% \changes{v1.1a}{87/07/29}{New concept: redefine the internal
%                          \protect\LaTeX{} commands `@xnthm' etc.}
% \changes{v1.1b}{87/07/31}{`theoremfont' used to specify font.}
% \changes{v1.2a}{88/01/08}{`@endtheorem' now globally defined.}
% \changes{v1.2b}{88/01/11}{`newtheorem' only allowed in the preamble.}
% \changes{v1.3a}{88/02/14}{`theoremfont' now a toks register.}
%
% \changes{v2.0a}{88/11/16}{New implementation and new documentation.}
% \changes{v2.0a}{88/11/16}{No layout changes but internal modifications
%                           old substyles won't work.}
% \changes{v2.0b}{89/01/30}{`global' used for all definitions.}
% \changes{v2.0c}{89/06/15}{Documentation translated to English.
%                            Thanks to LLL and Nina}
% \changes{v2.1a}{90/02/15}{Support for new font selection.}
% \changes{v2.2a}{94/02/03}{Cleanup for LaTeX2e}
%
% \DoNotIndex{\@endparenv,\@ifdefinable,\@ifnextchar,\@ifundefined}
% \DoNotIndex{\@namedef,\@nameuse,\@spaces,\@tempa,\@warning}
% \DoNotIndex{\ }
% \DoNotIndex{\begingroup,\bfseries}
% \DoNotIndex{\csname}
% \DoNotIndex{\def,\do,\docdate}
% \DoNotIndex{\endcsname,\endgroup,\endinput,\endtrivlist,\expandafter}
% \DoNotIndex{\filedate,\fileversion}
% \DoNotIndex{\gdef,\global}
% \DoNotIndex{\hbox,\hskip}
% \DoNotIndex{\ignorespaces,\input,\itshape,\item}
% \DoNotIndex{\labelsep,\let,\llap}
% \DoNotIndex{\makeatletter}
% \DoNotIndex{\noexpand}
% \DoNotIndex{\refstepcounter,\relax,\rlap}
% \DoNotIndex{\setlength,\slshape,\space,\string,\strut}
% \DoNotIndex{\the,\trivlist,\typeout}
% \DoNotIndex{\vbox}
% \DoNotIndex{\xdef}
%
% \def\deflabel#1{\ttfamily #1\hfill}
% \def\deflist#1{\begin{list}{}{\settowidth\labelwidth{\ttfamily #1}%
%                               \setlength\leftmargin\labelwidth
%                               \addtolength\leftmargin\labelsep
%                               \let\makelabel\deflabel}}
% \def\enddeflist{\end{list}}
% \def\envfont{\normalfont\ttfamily}
%
% \MakeShortVerb{\|}
%
% \newcommand{\TUB}{{\sl TUGboat\/}}
%
% \GetFileInfo{theorem.dtx}
% \title{An Extension of the
%        \LaTeX{} theorem environment\thanks{This file
%        has version number \fileversion, last
%        revised \filedate.}}
% \author{Frank Mittelbach}
%
% \MaintainedByLaTeXTeam{tools}
% \maketitle
%
% \begin{abstract}
%    The macros described in this paper yield an extension of the
%    \LaTeX{} theorem mechanism. It is designed to satisfy the
%    different requirements of various journals. Thus, the
%    layout of the ``theorems'' can be manipulated by determining a
%    ``style''. This article describes not only the use, but
%    also the definition, of the necessary macros.
% \end{abstract}
%
% \section*{Preface to version 2.2}
%
% For \LaTeXe{} this package did not need any fundamental changes. I
% only modified the messages generated so that theorem layout styles
% will show up with the |\listfiles| command and cleaned the section
% on the New Font Selection Scheme since this is now included in
% \LaTeX.
%
% \section*{Preface to version 2.1}
%
% This version is identical to 2.0g described in \TUB\ 10\#3 except
% for some internal defaults which are now set depending on the used
% font selection scheme.
%
% This was done to avoid unpleasant surprises if the new font selection
% scheme is in force. For further details see
% section~\ref{sec:newfonts} and \cite{bk:GMS94}.
%
% \section{Introduction}
%
% For our purposes here, ``theorems'' are labelled enunciations,
% often set off from the main text by extra space and a font change.
% Theorems, corollaries, conjectures, definitions, and remarks
% are all instances of ``theorems''.  The ``header'' of these structures
% is composed of a label (such as \textsc{Theorem} or \textsc{Remark})
% and a number which serializes an item in the sequence of items
% with the same label.
%
% Shortly after the introduction of \LaTeX{} at the Fachbereich
% Mathematik in Mainz, the desire to manipulate the layout of
% ``theorems'' arose.  In Mainz, the following two conventions came
% into general use:
% \begin{enumerate}
%   \item The number of the theorem is shown in the margin.
%   \item There is a line break at
%         the end of the theorem header.
% \end{enumerate}
%
% Additionally, some journals require different formats which depend on
% the ``sort of theorem'': e.g.\ often remarks and definitions are set
% in |\upshape|, while |\itshape| is employed for main theorems.
%
% Confronted with these requirements, a theorem environment was
% developed in Mainz which allows separate determination of the
% layout of the ``theorems sets'', comparable to |\pagestyle|.
%
%
% \section{The user interface}
% \subsection{Defining new theorem sets}
%
% \DescribeMacro\newtheorem
% As in the original \LaTeX{} version, the command |\newtheorem|
% defines a new ``theorem set'' or ``theorem-like structure''.
% Two required arguments name the new environment and give the text
% to be typeset with each instance of the new ``set'', while
% an optional argument determines how
% the ``set'' is enumerated:
% \begin{description}
%    \item[\ttfamily \bslash newtheorem\{foo\}\{bar\}]
%       The theorem set {\envfont foo} (whose name is \texttt{bar})
%       uses its own counter.
%    \item[\ttfamily \bslash newtheorem\{foo2\}{[foo]}\{bar2\}]
%       The theorem set {\envfont foo2} (printed name \texttt{bar2})
%       uses the same counter as the theorem set \texttt{foo}.
%    \item[\ttfamily \bslash newtheorem\{foo3\}\{bar3\}{[section]}]
%       The theorem set {\envfont foo3} (printed name \texttt{bar3}) is
%       enumerated within the counter \texttt{section}, i.e.\ with every
%       new |\section| the enumeration begins again with 1, and
%       the enumeration is composed from the section-number and the
%       theorem counter itself.
% \end{description}
%
% \DescribeMacro\theoremstyle
% Additionally, the command |\theoremstyle| can define the layout
% of various, or all, theorem sets.  It should be noted that any
% theorem set defined by |\newtheorem| is typeset in the
% |\theoremstyle| that is current at the time of the definition.
% Thus, the following
% \begin{verbatim}
%    \theoremstyle{break}        \newtheorem{Cor}{Corollary}
%    \theoremstyle{plain}        \newtheorem{Exa}{Example}[section]
%\end{verbatim}                             ^^A No blank at line start
% leads to the result that the set {\envfont Cor} is formatted in the
% style \texttt{break}, while the set {\envfont Exa} and all the
% following ones are formatted in the style \texttt{plain}, unless
% another |\theoremstyle| follows.  Since the definitions installed by
% |\newtheorem| are global, one also can limit |\theoremstyle| locally
% by grouping braces.
%
% \DescribeMacro\theorembodyfont
% The choice of the font for the theorem body is completely
% independent of the chosen |\theoremstyle|; this has proven to
% be very advantageous.  For example,
% \begin{verbatim}
%   {\theorembodyfont{\upshape}          \newtheorem{Rem}{Remark}}
%\end{verbatim}
% defines a theorem set {\envfont Rem}, which will be set in
% |\upshape| in the current layout (which in our example is
% \texttt{plain}).  As with |\theoremstyle|, the |\theorembodyfont|
% chosen is that current at the time of |\newtheorem|.  If
% |\theorembodyfont| is not specified or one defines
% |\theorembodyfont{}|, then the font used will be that defined by the
% |\theoremstyle|.
%
% \DescribeMacro\theoremheaderfont It is also possible to customize
% the font used for the theorem headers.  This is, however, a global
% declaration, and therefore there should be at most one
% |\theoremheaderfont| declaration in the preamble.\footnote{If it is
% actually necessary to have different header fonts, one has to define
% new theorem styles (substituting the desired font) or specify the
% information directly in the \texttt{\bslash newtheorem} declaration
% (the unclean variant).}
%
% \DescribeMacro\theorempreskipamount
% \DescribeMacro\theorempostskipamount Two additional parameters
% affect the vertical space around the theorem environments:
% |\theorempreskipamount| and |\theorempostskipamount| define,
% respectively, the spacing before and after such an environment.
% These parameters apply for all theorem sets and can be manipulated
% with the ordinary length macros.  They are rubber lengths
% (`\textsf{skips}'), and therefore can contain \texttt{plus} and
% \texttt{minus} parts.
%
% Since the definition of theorem sets should---most sensibly---be
% placed in the preamble, we only allow installation there.
% It is therefore possible to release the memory used here after
% |\begin{document}|, in order to make room for other
% applications.
%
%
% \subsection{Existing theorem styles}
%
% The following theorem styles exist to date:
% \begin{deflist}{marginbreak}
%    \item[plain]
%       This theorem style emulates the original \LaTeX{} definition,
%       except that additionally the parameters
%       |\theorem...skipamount| are used.
%    \item[break]
%       In this style, the theorem header is followed by a line
%       break.
%    \item[marginbreak]
%       The theorem number is set in the margin, and there is a line
%       break as in \texttt{break}.
%    \item[changebreak]
%       Like \texttt{break}, but with header number and text
%       interchanged.
%    \item[change]
%       Header number and text are interchanged, without a line break.
%    \item[margin]
%       The number is set in the left margin, without a line break.
% \end{deflist}
% All styles (except \texttt{plain}) select |\slshape| as the default
% |\theorembodyfont|.
%
%
% \subsection{Examples}
%
% Given the above theorem sets {\envfont Cor}, {\envfont Exa}
% and {\envfont Rem},
% suppose that the preamble also contains the declarations:
% \begin{verbatim}
%    \theoremstyle{marginbreak}   \newtheorem{Lem}[Cor]{Lemma}
%    \theoremstyle{change}
%    \theorembodyfont{\itshape}        \newtheorem{Def}[Cor]{Definition}
%
%    \theoremheaderfont{\scshape}
%\end{verbatim}
% Then the following are some typical examples of the typeset output
% resulting from their use.
%
% \begin{Cor}
%    This is a sentence typeset in the theorem environment {\envfont
%    Cor}.
% \end{Cor}
%
% \begin{Exa}
%    This is a sentence typeset in the theorem environment {\envfont
%    Exa}.
% \end{Exa}
%
% \begin{Rem}
%    This is a sentence typeset in the theorem environment {\envfont
%    Rem}.
% \end{Rem}
%
% \begin{Lem}[Ben User]
%    This is a sentence typeset in the theorem environment {\envfont
%    Lem}.
% \end{Lem}
%
% \begin{Def}[Very impressive Definition]
%    This is a sentence typeset in the theorem environment {\envfont
%    Def}.
% \end{Def}
%
% The last two examples show the effect of the optional argument to
% a theorem environment (it is the text typeset in parentheses).
%
%
% \section{Special Considerations}
% \label{sec:newfonts}
%
% Theoremheader and body are implemented as a unit. This means that
% the |\theoremheaderfont|  will inherit characteristics of the
% |\theorembodyfont| in \LaTeXe. Thus, if for example
% |\theorembodyfont| is |\itshape| and |\theoremheaderfont|
% is |\bfseries| the font selected for the header will have the
% characteristics `bold extended italic'. If this is not desired one
% should set the |\theoremheaderfont| to something like
% \begin{verbatim}
%      \theoremheaderfont{\normalfont\bfseries}
%\end{verbatim}
% i.e.\ supplying all necessary font information explicitly.
%
%
%
% \section{Acknowledgements}
%
% The publication of this set of macros was only possible with the
% help of Christina Busse (translating the manuscript into English),
% Joachim Pense (playing the r\^ole of typist), Chris Rowley
% (looking everything over) and many others providing useful
% suggestions.
%
% \MaybeStop{%
%    \begin{thebibliography}{9}
%       \bibitem{bk:GMS94} \textsc{M.~Goossens}, \textsc{F.~Mittelbach}
%         and \textsc{A.~Samarin}.
%         \newblock The \LaTeX{} Companion.
%         \newblock
%         Addison-Wesley, Reading, Massachusetts, 1994.
%       \bibitem{latex:source} \textsc{Lamport, Leslie.} \newblock
%                               \texttt{latex.tex}, version 2.09, date
%                               Feb. 1990.
%    \end{thebibliography}
%    \PrintIndex
%    \PrintChanges}
%
%
% \section{The documentation driver file}
%
% The next bit of code contains the documentation driver file for
% \TeX{}, i.e., the file that will produce the documentation you are
% currently reading. It will be extracted from this file by the
% \texttt{docstrip} program. Since it is the first code in the file
% one can alternatively process this file directly with \LaTeXe{} to
% obtain the documentation.
%
%    \begin{macrocode}
%<*driver>
\documentclass{ltxdoc}

\usepackage{theorem}

%<+driver>% The next few lines define theorem sets which are used
%<+driver>% in the example section of the documentation.

\theoremstyle{break}       \newtheorem{Cor}{Corollary}
\theoremstyle{plain}       \newtheorem{Exa}{Example}[section]
{\theorembodyfont{\upshape}\newtheorem{Rem}{Remark}}
\theoremstyle{marginbreak} \newtheorem{Lem}[Cor]{Lemma}
\theoremstyle{change}
\theorembodyfont{\itshape} \newtheorem{Def}[Cor]{Definition}

\theoremheaderfont{\scshape}

\RecordChanges

\begin{document}
   \DocInput{theorem.dtx}
\end{document}
%</driver>
%    \end{macrocode}
%
% \section{Definition of the Macros}
%
% If the file has been loaded before, we abort immediately. If not the
% package announces itself (this is actually done at the very top of
% the file---the way it is done isn't good style so don't copy it).
% \changes{v2.0e}{89/07/19}{Spaces removed from `typeout'.}
% \changes{v2.2b}{95/11/19}{Announce moved to top of file}
%    \begin{macrocode}
%<*package>
%\@ifundefined{theorem@style}{}{\endinput}
%\def\FMithmInfo{1995/11/19 v2.2b Theorem extension package (FMi)}
%\ProvidesPackage{theorem}[\FMithmInfo]
%    \end{macrocode}
%
%
% \subsection{Definition of theorem styles and fonts}
%
% All the definitions in this file are done globally to allow
% inputting this file inside a group.
%
% \begin{macro}{\theoremstyle}
%    Before a theorem style can be installed, the chosen style must be
%    known. For that reason, we must test to see that
%    |\th@|\meta{style} is known or, more precisely, that it is
%    different from |\relax|. If the style is not known then
%    |\th@plain| is used.
%    \begin{macrocode}
\gdef\theoremstyle#1{%
   \@ifundefined{th@#1}{\@warning
          {Unknown theoremstyle `#1'. Using `plain'}%
          \theorem@style{plain}}%
%    \end{macrocode}
%    We save the theorem style to be used in the token register
%    |\theorem@style|.
%    \begin{macrocode}
      {\theorem@style{#1}}%
%    \end{macrocode}
%    Now we ``evaluate'' the theorem style: this means, we call the
%    macro |\th@|\meta{style} which will activate the relevant
%    definitions which are contained in a separate file.  This is done
%    in a group to suppress changes to the current font.  This could
%    otherwise pose problems together with the new font selection
%    scheme\footnote{When I printed the original article using the new
%    font selection scheme I ended with a document with slanted
%    typefaces (text headings and all) simply because one of the
%    theorem styles used \texttt{\bslash sl} at toplevel.} if the
%    |\th@|\meta{style} is evaluated a second time.
% \changes{v2.1a}{90/02/15}{Group added around `th@..}
% \changes{v2.1a}{90/02/15}{Reseting of `..@begintheorem removed.}
%    \begin{macrocode}
      \begingroup
        \csname th@\the\theorem@style \endcsname
      \endgroup}
%    \end{macrocode}
% \end{macro}
%
%
% \begin{macro}{\@begintheorem}
% \changes{v2.1a}{90/02/15}{Relaxed at toplevel.}
% \begin{macro}{\@opargbegintheorem}
% \changes{v2.1a}{90/02/15}{Relaxed at toplevel.}
%    We reset |\@begintheorem| and |\@opargbegintheorem| to
%    |\relax| since these commands are no longer necessary at
%    toplevel. This will save a few tokens.
%    \begin{macrocode}
\global\let\@begintheorem\relax
\global\let\@opargbegintheorem\relax
%    \end{macrocode}
% \end{macro}
% \end{macro}
%
% \begin{macro}{\theorem@style}
%    Obviously the \textsf{token} register used above has to be
%    allocated. To assure the utmost compatibility with the
%    original \LaTeX{} definition, we set the default theorem style to
%    \texttt{plain}, which implements the usual \LaTeX{} convention.
% \changes{v2.0f}{89/07/20}{Unnecessary `global' removed.}
%    \begin{macrocode}
\newtoks\theorem@style
\global\theorem@style{plain}
%    \end{macrocode}
% \end{macro}
%
% \begin{macro}{\theorembodyfont}
% \begin{macro}{\theorem@bodyfont}
%    If the |\theorembodyfont| is set by the user then it should not
%    interact with the default font set in the theorem style. When the
%    new font selection is in force this may happen if, for example,
%    the default is |\itshape| and the new |\theorembodyfont| is
%    |\sffamily|. So we add a |\reset@font| command in front of the
%    user definition.
% \changes{v2.0f}{89/07/20}{Unnecessary `global' removed.}
% \changes{v2.0g}{89/09/19}{`theoremfont to `theorembodyfont.}
% \changes{v2.1c}{92/06/22}{`reset@font added and logic changed.}
% \changes{v2.1d}{92/10/28}{Reset to empty if arg is empty}
%    \begin{macrocode}
\gdef\theorembodyfont#1{%
%    \end{macrocode}
%    We check if the argument supplied is empty and if so put nothing
%    into the |\theorem@bodyfont| token register to allow for
%    |\theorembodyfont{}| as a mean of using the default of the
%    current |\theoremstyle|.
%    \begin{macrocode}
   \def\@tempa{#1}%
   \ifx\@tempa\@empty
    \theorem@bodyfont{}%
   \else
    \theorem@bodyfont{\reset@font#1}%
   \fi
   }
\newtoks\theorem@bodyfont
\global\theorem@bodyfont{}
%    \end{macrocode}
% \end{macro}
% \end{macro}
%
%
% \begin{macro}{\theoremheaderfont}
% \changes{v2.0g}{89/09/19}{Macro added.}
%    The font for the theorem headers is handled differently because
%    this definition applies to all theorem styles.
%    \begin{macrocode}
\gdef\theoremheaderfont#1{\gdef\theorem@headerfont{#1}%
%    \end{macrocode}
%    After using the macro once it is redefined to produce an error
%    message.
%    \begin{macrocode}
       \gdef\theoremheaderfont##1{%
        \typeout{\string\theoremheaderfont\space should be used
                 only once.}}}
%    \end{macrocode}
% \end{macro}
%
% \begin{macro}{\theorem@headerfont}
% \changes{v2.0g}{89/09/19}{Macro added.}
%    To set the |\theorem@headerfont| default we first test if
%    the new fontselection scheme is in force.
% \changes{v2.1a}{90/02/15}{Support for new font selection.}
%    \begin{macrocode}
\ifx\upshape\undefined
%    \end{macrocode}
%    If not we define it to expand into |\bfseries|. We don't use
%    |\let| just in case a following style option redefines this
%    macro.
%    \begin{macrocode}
\gdef\theorem@headerfont{\bfseries}
%    \end{macrocode}
%    Otherwise we reset the current shape before calling |\bfseries|.
%    \begin{macrocode}
\else \gdef\theorem@headerfont{\normalfont\bfseries}\fi
%    \end{macrocode}
% \end{macro}
%
% \begin{macro}{\th@plain}
% \begin{macro}{\th@break}
% \begin{macro}{\th@marginbreak}
% \begin{macro}{\th@changebreak}
% \begin{macro}{\th@change}
% \begin{macro}{\th@margin}
%    The different styles are defined in macros such as
%    |\th@plain|.  Since memory space is precious in
%    ``non-Big-versions'', we have to avoid offering too many unused
%    definitions.  Therefore we define these styles in separate files
%    that can be loaded on demand.  Thus the commands themselves only
%    load these files. We use |\@input@|, a \LaTeXe{} internal command
%    that ensures that the file will be listed with |\listfiles|.
% \changes{v2.2a}{94/02/02}{Use `@input@ to load theorem layout files}
%    \begin{macrocode}
\gdef\th@plain{\@input@{thp.sty}}
\gdef\th@break{\@input@{thb.sty}}
\gdef\th@marginbreak{\@input@{thmb.sty}}
\gdef\th@changebreak{\@input@{thcb.sty}}
\gdef\th@change{\@input@{thc.sty}}
\gdef\th@margin{\@input@{thm.sty}}
%    \end{macrocode}
%    This list will be expanded when new styles become
%    available.  For testing,  just append new theorem substyles
%    as document options.
% \end{macro}
% \end{macro}
% \end{macro}
% \end{macro}
% \end{macro}
% \end{macro}
%
%
%
% \subsection{Definition of a new theorem set}
%
% As already pointed out, a new theorem environment can be defined in
% three different ways:
% \begin{verbatim}
%    \newtheorem{Lem}{Lemma}
%    \newtheorem{Lem}{Lemma}[section]
%    \newtheorem{Lem}[Theorem]{Lemma}
%\end{verbatim}
% The function of the macro |\newtheorem| is to recognize these cases
% and then to branch into one of the three macros |\@ynthm|, |\@xnthm|
% or |\@othm|.  This mechanism is adopted unchanged from
% \cite{latex:source}; the essential point here is that, for example,
% in the second case, the arguments \texttt{Lem}, \texttt{Lemma} and
% \texttt{section} are passed over to the macro |\@xnthm|.
%
% We inspect this case first because the others present fewer problems,
% and thus are easily derived from this one.
%
% \begin{macro}{\@xnthm}
%    For our example arguments, the macro |\@xnthm| must fulfill
%    the following:
%    \begin{itemize}
%       \item
%          Define a new \LaTeX-counter `Lem'
%       \item
%          reset this counter within a |\section|
%       \item
%          define the macro |\theLem|
%       \item
%          define the environment macros |\Lem| and
%          |\endLem| using the current
%          |\theoremstyle| and |\theorem@bodyfont|.
%    \end{itemize}
%    Obviously, all this should happen only if the first argument
%    of |\@xnthm| (i.e.\ \texttt{Lem} in our example) is chosen so
%    as not to conflict with any previously defined commands or
%    environments.  This test is performed by the \LaTeX{} macro
%    |\@ifdefinable|.
%    \begin{macrocode}
\gdef\@xnthm#1#2[#3]{\expandafter\@ifdefinable\csname #1\endcsname
%    \end{macrocode}
%    Therefore, the first argument of |\@ifdefinable| is the
%    expansion (in the example, |\Lem|) of
%    |\csname||#1\endcsname|.  The second argument is
%    executed only if the test has been completed successfully.
%    \begin{macrocode}
   {%
%    \end{macrocode}
%    Now we define the new counter. The names of the \LaTeX{}
%    macros employed should speak for themselves:
% \changes{v2.2b}{1995/10/07}{Use \cs{@newctr}}
% \changes{v2.2c}{1995/11/23}{Use \cs{@newctr} properly.}
%    \begin{macrocode}
    \@definecounter{#1}\@newctr{#1}[#3]%
%    \end{macrocode}
%    Using |\@newctr| will give a proper error message if the counter
%    in |#3| is not defined.
%    In defining `|\theLem|' we must generate the desired macro
%    name by use of |\expandafter| and |\csname|.
%    \begin{macrocode}
    \expandafter\xdef\csname the#1\endcsname
%    \end{macrocode}
%    An |\xdef| is used in order to make the definition global,
%    and to ensure that it contains the replacement texts of
%    |\@thmcountersep| and |\@thmcounter|.\footnote{These
%    two macros can be defined by the document style.  Their default
%    values produce a `\texttt{.}' as separation and an arabic
%    representation of the number.} However, not everything should be
%    expanded. For example, it saves space to use |\thesection|
%    instead of its---at times---lengthy expansion.
%    \begin{macrocode}
      {\expandafter \noexpand \csname the#3\endcsname
       \@thmcountersep \@thmcounter{#1}}%
%    \end{macrocode}
%    Thus with the defaults of \LaTeX{}, |\theLem| would be
%    replaced by the command sequence |\thesection.\arabic{Lem}|.
%
%    We will now look at the definition of the macro which is executed
%    at the beginning of the actual environment (in our example this
%    macro is |\Lem|). It should be noted that we use an
%    ``|\expandafter| trick'' to expand only certain parts of the
%    replacement text at the time of the definition.
%    \begin{macrocode}
    \def\@tempa{\global\@namedef{#1}}%
    \expandafter \@tempa \expandafter{%
%    \end{macrocode}
%    First, the macro that contains the current definitions of
%    |\@begintheorem| and |\@opargtheorem| should be called
%    up.  The name of this macro---as is already known---has the form
%    |\th@|\meta{theorem\ style}; therefore, it must be called by
%    \begin{macrocode}
      \csname th@\the \theorem@style
%    \end{macrocode}
%    In addition the default theorem font should be changeable, i.e.\
%    we have to insert the contents of |\theorem@bodyfont|. For
%    that reason, we expand even further, beyond |\endcsname|,
%    and thus insert the contents of the \textsf{token} register
%    |\theorem@bodyfont| in the replacement text.
%    \begin{macrocode}
            \expandafter \endcsname \the \theorem@bodyfont
%    \end{macrocode}
%    Now it is time to call the macro |\@thm| which takes over
%    the further processing.  It has two arguments: the current
%    counter name (in our example, \texttt{Lem}), and the text of the
%    label (in our example, \texttt{Lemma}).
%    \begin{macrocode}
     \@thm{#1}{#2}}%
%    \end{macrocode}
%    With this, the `sub-definition' is complete.  The macro
%    |\@endtheorem| ends a theorem environment and is, so far, nothing
%    but an |\endtrivlist|.  (Hence it is defined globally, and not
%    within the theorem styles.\footnote{This has to be changed as
%    soon as theorem styles that change \texttt{\bslash @endtheorem}
%    exist.  In such a case, all existing styles must be changed as
%    well since they will have to reset the macro.}) Therefore, we can
%    set it equivalent to the macro that ends the theorem set (in our
%    example, |\endLem|).  However, if some day theorem styles exist
%    that do change |\@endtheorem|, we would have to use the
%    commented-out line instead.
%    \begin{macrocode}
    \global \expandafter \let \csname end#1\endcsname \@endtheorem
%   \global\@namedef{end#1}{\@endtheorem}%
%    \end{macrocode}
%    With these commands all the required definitions are employed,
%    unless the test |\@ifdefinable| has failed.  Therefore, we
%    end the second argument of this macro and with it the definition
%    of |\@xnthm|.
%    \begin{macrocode}
   }}
%    \end{macrocode}
% \end{macro}
%
% \begin{macro}{\@ynthm}
%    The definition of |\@ynthm| is completely analogous. In
%    this case the new counter that is defined is not reset within
%    another counter; thus the definition of |\the...|
%    is simplified:
%    \begin{macrocode}
\gdef\@ynthm#1#2{\expandafter\@ifdefinable\csname #1\endcsname
   {\@definecounter{#1}%
    \expandafter\xdef\csname the#1\endcsname{\@thmcounter{#1}}%
%    \end{macrocode}
%    The rest of the definition corresponds literally to that of
%    |\@xnthm|:
%    \begin{macrocode}
    \def\@tempa{\global\@namedef{#1}}\expandafter \@tempa
     \expandafter{\csname th@\the \theorem@style \expandafter
     \endcsname \the\theorem@bodyfont \@thm{#1}{#2}}%
    \global \expandafter \let \csname end#1\endcsname \@endtheorem}}
%    \end{macrocode}
% \end{macro}
%
% \begin{macro}{\@othm}
%    The definition of |\@othm| does not contain anything new.
%    \begin{macrocode}
\gdef\@othm#1[#2]#3{%
%    \end{macrocode}
% \changes{v2.2b}{95/10/07}{Build in check for counter defined}
%    We do not define a new counter but instead use one that has
%    already been defined.  Thus the only definition we need is that
%    of this pseudo-counter (i.e.\ |\the|\meta{env.\ name}).
%    First we check if |#2| corresponds to a known counter name.
%    \begin{macrocode}
  \expandafter\ifx\csname c@#2\endcsname\relax
   \@nocounterr{#2}%
  \else
   \expandafter\@ifdefinable\csname #1\endcsname
   {\expandafter \xdef \csname the#1\endcsname
     {\expandafter \noexpand \csname the#2\endcsname}%
%    \end{macrocode}
%    All other parts of the definition can be adopted from
%    |\@xnthm|. We have to remember, though, that in this case
%    the name of the current counter and the theorem label have moved
%    to the second and third arguments.
%    \begin{macrocode}
    \def\@tempa{\global\@namedef{#1}}\expandafter \@tempa
     \expandafter{\csname th@\the \theorem@style \expandafter
     \endcsname \the\theorem@bodyfont \@thm{#2}{#3}}%
    \global \expandafter \let \csname end#1\endcsname \@endtheorem}%
  \fi}
%    \end{macrocode}
% \end{macro}
%
% \subsection{Macros that are employed in a {\envfont theorem}
%    environment}
%
% \begin{macro}{\@thm}
%    The macro |\@thm| has to increase the current counter.  Then,
%    depending on whether the environment has (or does not have) an
%    optional argument, it has to branch into either |\@begintheorem|
%    or |\@opargtheorem|.
%    \begin{macrocode}
\gdef\@thm#1#2{\refstepcounter{#1}%
%    \end{macrocode}
%    Now we start a {\envfont trivlist} environment, and give
%    |\@topsep| and |\@topsepadd| the values of the \textsf{skip}
%    registers |\theorempreskipamount| and |\theorempostskipamount|.
%    The value in |\@topsep| is the vertical space that is inserted by
%    the first (and only) |\item| in our |\trivlist| whilst
%    |\@topsepadd| is inserted by |\@endparenv| at the end of that
%    {\envfont trivlist} environment.  By using these registers, we
%    obtain the desired space around a {\envfont theorem} environment.
%    \begin{macrocode}
   \trivlist
   \@topsep \theorempreskipamount               % used by first \item
   \@topsepadd \theorempostskipamount           % used by \@endparenv
%    \end{macrocode}
%    Now we have to test whether an optional argument has been given.
% \changes{v2.0d}{89/07/18}{Forgotten percent added.}
%    \begin{macrocode}
   \@ifnextchar [%
%    \end{macrocode}
%    If there is an optional argument, we will call |\@ythm|, and
%    move the arguments read back into the input stream.
%    \begin{macrocode}
   {\@ythm{#1}{#2}}%
%    \end{macrocode}
%    If not, we call |\@begintheorem|. Its first argument is the
%    name of the theorem set (hence the second argument of
%    |\@thm|). Its second argument is the macro that produces the
%    current number.
%    \begin{macrocode}
   {\@begintheorem{#2}{\csname the#1\endcsname}\ignorespaces}}
%    \end{macrocode}
% \end{macro}
%
% \begin{macro}{\@xthm}
% \begin{macro}{\@ythm}
%    Both these macros were originally called by |\@thm|.  We do
%    not need |\@xthm| anymore, hence we reset it to
%    |\relax|. The definition of |\@ythm| has not changed at
%    all from its definition in \LaTeX. In order to make the macros
%    easier to understand, we will
%    nevertheless present it (commented out).
%    \begin{macrocode}
\global\let\@xthm\relax
% \def\@ythm#1#2[#3]{\@opargbegintheorem{#2}{\csname
%           the#1\endcsname}{#3}\ignorespaces}
%</package>
%    \end{macrocode}
%    The primitive |\ignorespaces| in |\@ythm| and
%    |\@thm| is needed to remove the spaces between the
%    |\begin{...}| and the actual text.
% \end{macro}
% \end{macro}
%
%
% \subsection{Definition of the theorem substyles}
%
% As already pointed out, the theorem substyles, defined below, are
% only loaded when necessary.  Note that all these substyles, except
% \texttt{plain}, have |\slshape| as the default body font.
%
% \subsubsection{The \texttt{plain} style}
%
% \label{dok:th@plain} As the following macros use |@|, we have
% to locally set the |\catcode| of this symbol to ``letter''.
% This happens within a group, so that we do not have to worry about
% which |\catcode| that symbol had before.
%    \begin{macrocode}
%<*thp>
\begingroup \makeatletter
%    \end{macrocode}
% Since we are now within a group, we must make all definitions
% globally.  First we make sure that \texttt{theorem.sty} is loaded.
% This will allow us to use this file as a document style option
% without having to call \texttt{theorem} itself as an option.  At the
% same time, we assure that at least version 2 is loaded, since
% |\theorem@style| was not defined in earlier versions.
%    \begin{macrocode}
\@ifundefined{theorem@style}{\input{theorem.sty}}{}
\ProvidesFile{thp.sty}
             [\FMithmInfo]
%    \end{macrocode}
%
%
% \begin{macro}{\th@plain}
% \begin{macro}{\@begintheorem}
% \begin{macro}{\@opargbegintheorem}
%    |\theoremstyle{plain}| corresponds to the original
%    definition, except that the distances to the surrounding text are
%    determined by the parameters |\theorempreskipamount| and
%    |\theorempostskipamount|.
%    First we set the default body font.
%    \begin{macrocode}
\gdef\th@plain{\normalfont\itshape
%    \end{macrocode}
%    Then we define |\@begintheorem| and
%    |\@opargbegintheorem|. These two macros define how the
%    header of a theorem is typeset.  |\@opargbegintheorem| will
%    be called if a \textsf{theorem} environment with an optional
%    argument is encountered; otherwise, the header is constructed by
%    calling |\@begintheorem|.  If one of these macros is
%    executed, we are within a \textsf{trivlist} environment started by
%    |\@thm|.  So the theorem header is produced with an
%    |\item| command.
%
%    Instead of specifying the header font directly, all standard
%    theorem styles use the |\theorem@headerfont| macro to allow
%    customization.  The extra space (|\labelsep|) is necessary
%    because of problems in the \textsf{trivlist} environment.
%    \begin{macrocode}
  \def\@begintheorem##1##2{%
        \item[\hskip\labelsep \theorem@headerfont ##1\ ##2]}%
%    \end{macrocode}
%    The definition of |\@opargbegintheorem| is completely
%    analogous. The only difference is the fact that there exists a
%    third argument (which is the optional parameter of the
%    environment and contains additional information about the
%    theorem).  Customarily we enclose it in parentheses.
%    \begin{macrocode}
\def\@opargbegintheorem##1##2##3{%
   \item[\hskip\labelsep \theorem@headerfont ##1\ ##2\ (##3)]}}
%    \end{macrocode}
% \end{macro}
% \end{macro}
% \end{macro}
%
% \noindent
% We conclude with an |\endgroup| to restore the
% |\catcode| of |@|.
%    \begin{macrocode}
\endgroup
%</thp>
%    \end{macrocode}
%
%
% \subsubsection{The \texttt{break} style}
%
% This style option is stored in the file |thb.sty|.
% For the next two lines see the documentation for |\th@plain| on
% page~\pageref{dok:th@plain}.
%    \begin{macrocode}
%<*thb>
\begingroup \makeatletter
\@ifundefined{theorem@style}{\input{theorem.sty}}{}
\ProvidesFile{thb.sty}
             [\FMithmInfo]
%    \end{macrocode}
%
% \begin{macro}{\th@break}
%    |\theoremstyle{break}| produces a line break after the name
%    of the theorem.  The font is |\slshape|.  Hence, we define
%    |\th@break| as follows:
%    \begin{macrocode}
\gdef\th@break{\normalfont\slshape
  \def\@begintheorem##1##2{\item[%
%    \end{macrocode}
%    We run into the following problem: it is not possible to create
%    the header with |\item[|\meta{title}|]| and then start
%    a new line by, for example, |\mbox{}\\|. Such a definition
%    will fail whenever a list environment follows immediately.  With
%    the above construction, the |\mbox{}| causes the switch
%    |@inlabel| (cf.\ definition of |\list| and
%    |\trivlist| in \cite{latex:source}) to be set to \texttt{false}
%    and so the following list will insert additional vertical
%    space (|\topskip|). This is quite annoying. Therefore, we
%    create the line break within the |\item|. In order to ensure
%    that the text will begin at the proper position in the following
%    line, we simply pretend that the label does not take any
%    room.\footnote{This will lead to problems whenever very high
%    symbols occurring in the line tower into the
%    heading. So, something else has to be done here sometime.}
%    \begin{macrocode}
       \rlap{\vbox{\hbox{\hskip \labelsep\theorem@headerfont ##1\ ##2}%
                   \hbox{\strut}}}]}%
%    \end{macrocode}
%    Again, the definition of |\@opargbegintheorem| is
%    completely analogous.
%    \begin{macrocode}
\def\@opargbegintheorem##1##2##3{%
  \item[\rlap{\vbox{\hbox{\hskip \labelsep \theorem@headerfont
                     ##1\ ##2\ (##3)}%
                    \hbox{\strut}}}]}}
\endgroup
%</thb>
%    \end{macrocode}
% \end{macro}
%
%
% \subsubsection{The \texttt{changebreak} style}
%
%    \begin{macrocode}
% This style option is stored in the file |thcb.sty|.
%    \begin{macrocode}
%<*thcb>
\begingroup \makeatletter
\@ifundefined{theorem@style}{\input{theorem.sty}}{}
\ProvidesFile{thcb.sty}
             [\FMithmInfo]
%    \end{macrocode}
%
% \begin{macro}{\th@changebreak}
%    The \texttt{change-break} theorem style is like \texttt{break}
%    but with interchange of theorem name and theorem number.  Thus we
%    define |\th@changebreak| as follows:
%    \begin{macrocode}
\gdef\th@changebreak{\normalfont\slshape
  \def\@begintheorem##1##2{\item
       [\rlap{\vbox{\hbox{\hskip\labelsep \theorem@headerfont ##2\ ##1}%
                    \hbox{\strut}}}]}%
\def\@opargbegintheorem##1##2##3{%
  \item[\rlap{\vbox{\hbox{\hskip\labelsep \theorem@headerfont
                     ##2\ ##1\ (##3)}%
                    \hbox{\strut}}}]}}
\endgroup
%</thcb>
%    \end{macrocode}
% \end{macro}
%
%
% \subsubsection{The \texttt{change} style}
%
% This style option is stored in the file |thc.sty|.
%    \begin{macrocode}
%<*thc>
\begingroup \makeatletter
\@ifundefined{theorem@style}{\input{theorem.sty}}{}
\ProvidesFile{thc.sty}
             [\FMithmInfo]
%    \end{macrocode}
%
% \begin{macro}{\th@change}
%    The \texttt{change} theorem style corresponds to the
%    \texttt{change break} style without a linebreak after the header.
%    To say it in another way, it's the same as the \texttt{plain}
%    style but with number and name interchanged and |\slshape| as the
%    default font.
%    \begin{macrocode}
\gdef\th@change{\normalfont\slshape
  \def\@begintheorem##1##2{\item
       [\hskip\labelsep \theorem@headerfont ##2\ ##1]}%
\def\@opargbegintheorem##1##2##3{%
  \item[\hskip\labelsep \theorem@headerfont ##2\ ##1\ (##3)]}}
\endgroup
%</thc>
%    \end{macrocode}
% \end{macro}
%
%
% \subsubsection{The \texttt{marginbreak} style}
%
% This style option is the one used most often at Mainz.  It is saved
% in the file |thmb.sty|.
%    \begin{macrocode}
%<*thmb>
\begingroup \makeatletter
\@ifundefined{theorem@style}{\input{theorem.sty}}{}
\ProvidesFile{thmb.sty}
             [\FMithmInfo]
%    \end{macrocode}
%
% \begin{macro}{\th@marginbreak}
%    The \texttt{margin break} style is nearly the same as the
%    \texttt{change break} style. The only difference is the placement
%    of the theorem number.  We use |\llap| to place it in the left
%    margin.
%
%    In this style |\labelsep| denotes the separation between the
%    number and the text.
%    \begin{macrocode}
\gdef\th@marginbreak{\normalfont\slshape
  \def\@begintheorem##1##2{\item
       [\rlap{\vbox{\theorem@headerfont
                    \hbox{\llap{##2}\hskip\labelsep ##1}%
                    \hbox{\strut}}}]}%
\def\@opargbegintheorem##1##2##3{%
  \item[\rlap{\vbox{\theorem@headerfont
                    \hbox{\llap{##2}\hskip\labelsep ##1\ (##3)}%
                    \hbox{\strut}}}]}}
\endgroup
%</thmb>
%    \end{macrocode}
% \changes{v2.1c}{90/02/17}{`endgroup outside macrocode env.}
% \end{macro}
%
%
%
% \subsubsection{The \texttt{margin} style}
%
% This style option is stored in the file |thm.sty|.
%    \begin{macrocode}
%<*thm>
\begingroup \makeatletter
\@ifundefined{theorem@style}{\input{theorem.sty}}{}
\ProvidesFile{thm.sty}
             [\FMithmInfo]
%    \end{macrocode}
%
% \begin{macro}{\th@margin}
%    Again this is only a variant of the theorem styles described
%    above without any new ideas.
%    \begin{macrocode}
\gdef\th@margin{\normalfont\slshape
  \def\@begintheorem##1##2{\item
       [\theorem@headerfont \llap{##2}\hskip\labelsep ##1]}%
\def\@opargbegintheorem##1##2##3{%
  \item[\theorem@headerfont \llap{##2}\hskip\labelsep ##1\ (##3)]}}
\endgroup
%</thm>
%    \end{macrocode}
% \end{macro}
%
%
%
%
% \subsection{Final Definitions}
%
% \begin{macro}{\theorempreskipamount}
% \begin{macro}{\theorempostskipamount}
%    The \textsf{skip} parameters that regulate the vertical empty space
%    before and after the {\envfont theorem} environment have
%    to be allocated as well.
% \changes{v2.0f}{89/07/20}{Unnecessary `global' removed.}
%    \begin{macrocode}
%<*package>
\newskip\theorempreskipamount
\newskip\theorempostskipamount
%    \end{macrocode}
%    Since we have used the same values for all theorem sets, we now can
%    assign them.
%    \begin{macrocode}
\global\setlength\theorempreskipamount{12pt plus 5pt minus 3pt}
\global\setlength\theorempostskipamount{8pt plus 3pt minus 1.5pt}
%    \end{macrocode}
% \end{macro}
% \end{macro}
%
%
% \begin{macro}{\@endtheorem}
%    The same holds for the macro |\@endtheorem|, which ends a
%    {\envfont theorem} environment. Since it is the same for all
%    theorem sets, it is removed from the macros |\th@|\meta{style}.
%    It simply ends the {\envfont trivlist} environment, which was
%    begun in |\@thm|.
%    \begin{macrocode}
\global\let\@endtheorem=\endtrivlist
%    \end{macrocode}
% \end{macro}
%
%
%
% \begin{macro}{\@preamblecmds}
%    All macros defined above are to be used only in the preamble.
%    Therefore, we insert them in |\@preamblecmds| which will disable
%    them at begin document. This is done by the internal \LaTeXe{}
%    command |\@onlypreamble|.
% \changes{v2.0g}{89/09/19}{`theoremheaderfont and `theorembodyfont
%    added.}
% \changes{v2.2a}{94/02/03}{Use \LaTeXe{} interface}
%    \begin{macrocode}
\@onlypreamble\@xnthm
\@onlypreamble\@ynthm
\@onlypreamble\@othm
\@onlypreamble\newtheorem
\@onlypreamble\theoremstyle
\@onlypreamble\theorembodyfont
\@onlypreamble\theoremheaderfont
%    \end{macrocode}
% \end{macro}
%
% Finally we declare the \texttt{plain} theorem style to be the default.
% \changes{v2.1b}{90/06/04}{Fixed missing `item by declaring default
%    style.}
%    \begin{macrocode}
\theoremstyle{plain}
%</package>
%    \end{macrocode}
%
% \Finale

\endinput
}{}
\ProvidesFile{thc.sty}
             [\FMithmInfo]
%    \end{macrocode}
%
% \begin{macro}{\th@change}
%    The \texttt{change} theorem style corresponds to the
%    \texttt{change break} style without a linebreak after the header.
%    To say it in another way, it's the same as the \texttt{plain}
%    style but with number and name interchanged and |\slshape| as the
%    default font.
%    \begin{macrocode}
\gdef\th@change{\normalfont\slshape
  \def\@begintheorem##1##2{\item
       [\hskip\labelsep \theorem@headerfont ##2\ ##1]}%
\def\@opargbegintheorem##1##2##3{%
  \item[\hskip\labelsep \theorem@headerfont ##2\ ##1\ (##3)]}}
\endgroup
%</thc>
%    \end{macrocode}
% \end{macro}
%
%
% \subsubsection{The \texttt{marginbreak} style}
%
% This style option is the one used most often at Mainz.  It is saved
% in the file |thmb.sty|.
%    \begin{macrocode}
%<*thmb>
\begingroup \makeatletter
\@ifundefined{theorem@style}{% \iffalse meta-comment
%
% Copyright (C) 1993-2024
%
% The LaTeX Project and any individual authors listed elsewhere
% in this file.
%
% This file is part of the Standard LaTeX `Tools Bundle'.
% -------------------------------------------------------
%
% It may be distributed and/or modified under the
% conditions of the LaTeX Project Public License, either version 1.3c
% of this license or (at your option) any later version.
% The latest version of this license is in
%    https://www.latex-project.org/lppl.txt
% and version 1.3c or later is part of all distributions of LaTeX
% version 2005/12/01 or later.
%
% The list of all files belonging to the LaTeX `Tools Bundle' is
% given in the file `manifest.txt'.
%
% \fi
%
%%
%
%
\def\FMithmInfo{2023/07/05 v2.2c Theorem extension package (FMi)}
%
% \ProvidesFile{theorem.dtx}[\FMithmInfo]
%\iffalse   % this is a METACOMMENT !
%%
%% Package `theorem' to use with LaTeX2e
%% Copyright (C) 1989-1995 Frank Mittelbach, all rights reserved.
%<+package>\@ifundefined{theorem@style}{}{\endinput}
%<+package>\NeedsTeXFormat{LaTeX2e}[1995/06/01]
%<+package>\ProvidesPackage{theorem}
%<+package>         [\FMithmInfo]
%
% \fi
%
% \changes{v1.0}{87/05/07}{First draft.}
% \changes{v1.0a}{87/05/07}{`thistheoremstyle' added.}
% \changes{v1.1a}{87/07/29}{New concept: redefine the internal
%                          \protect\LaTeX{} commands `@xnthm' etc.}
% \changes{v1.1b}{87/07/31}{`theoremfont' used to specify font.}
% \changes{v1.2a}{88/01/08}{`@endtheorem' now globally defined.}
% \changes{v1.2b}{88/01/11}{`newtheorem' only allowed in the preamble.}
% \changes{v1.3a}{88/02/14}{`theoremfont' now a toks register.}
%
% \changes{v2.0a}{88/11/16}{New implementation and new documentation.}
% \changes{v2.0a}{88/11/16}{No layout changes but internal modifications
%                           old substyles won't work.}
% \changes{v2.0b}{89/01/30}{`global' used for all definitions.}
% \changes{v2.0c}{89/06/15}{Documentation translated to English.
%                            Thanks to LLL and Nina}
% \changes{v2.1a}{90/02/15}{Support for new font selection.}
% \changes{v2.2a}{94/02/03}{Cleanup for LaTeX2e}
%
% \DoNotIndex{\@endparenv,\@ifdefinable,\@ifnextchar,\@ifundefined}
% \DoNotIndex{\@namedef,\@nameuse,\@spaces,\@tempa,\@warning}
% \DoNotIndex{\ }
% \DoNotIndex{\begingroup,\bfseries}
% \DoNotIndex{\csname}
% \DoNotIndex{\def,\do,\docdate}
% \DoNotIndex{\endcsname,\endgroup,\endinput,\endtrivlist,\expandafter}
% \DoNotIndex{\filedate,\fileversion}
% \DoNotIndex{\gdef,\global}
% \DoNotIndex{\hbox,\hskip}
% \DoNotIndex{\ignorespaces,\input,\itshape,\item}
% \DoNotIndex{\labelsep,\let,\llap}
% \DoNotIndex{\makeatletter}
% \DoNotIndex{\noexpand}
% \DoNotIndex{\refstepcounter,\relax,\rlap}
% \DoNotIndex{\setlength,\slshape,\space,\string,\strut}
% \DoNotIndex{\the,\trivlist,\typeout}
% \DoNotIndex{\vbox}
% \DoNotIndex{\xdef}
%
% \def\deflabel#1{\ttfamily #1\hfill}
% \def\deflist#1{\begin{list}{}{\settowidth\labelwidth{\ttfamily #1}%
%                               \setlength\leftmargin\labelwidth
%                               \addtolength\leftmargin\labelsep
%                               \let\makelabel\deflabel}}
% \def\enddeflist{\end{list}}
% \def\envfont{\normalfont\ttfamily}
%
% \MakeShortVerb{\|}
%
% \newcommand{\TUB}{{\sl TUGboat\/}}
%
% \GetFileInfo{theorem.dtx}
% \title{An Extension of the
%        \LaTeX{} theorem environment\thanks{This file
%        has version number \fileversion, last
%        revised \filedate.}}
% \author{Frank Mittelbach}
%
% \MaintainedByLaTeXTeam{tools}
% \maketitle
%
% \begin{abstract}
%    The macros described in this paper yield an extension of the
%    \LaTeX{} theorem mechanism. It is designed to satisfy the
%    different requirements of various journals. Thus, the
%    layout of the ``theorems'' can be manipulated by determining a
%    ``style''. This article describes not only the use, but
%    also the definition, of the necessary macros.
% \end{abstract}
%
% \section*{Preface to version 2.2}
%
% For \LaTeXe{} this package did not need any fundamental changes. I
% only modified the messages generated so that theorem layout styles
% will show up with the |\listfiles| command and cleaned the section
% on the New Font Selection Scheme since this is now included in
% \LaTeX.
%
% \section*{Preface to version 2.1}
%
% This version is identical to 2.0g described in \TUB\ 10\#3 except
% for some internal defaults which are now set depending on the used
% font selection scheme.
%
% This was done to avoid unpleasant surprises if the new font selection
% scheme is in force. For further details see
% section~\ref{sec:newfonts} and \cite{bk:GMS94}.
%
% \section{Introduction}
%
% For our purposes here, ``theorems'' are labelled enunciations,
% often set off from the main text by extra space and a font change.
% Theorems, corollaries, conjectures, definitions, and remarks
% are all instances of ``theorems''.  The ``header'' of these structures
% is composed of a label (such as \textsc{Theorem} or \textsc{Remark})
% and a number which serializes an item in the sequence of items
% with the same label.
%
% Shortly after the introduction of \LaTeX{} at the Fachbereich
% Mathematik in Mainz, the desire to manipulate the layout of
% ``theorems'' arose.  In Mainz, the following two conventions came
% into general use:
% \begin{enumerate}
%   \item The number of the theorem is shown in the margin.
%   \item There is a line break at
%         the end of the theorem header.
% \end{enumerate}
%
% Additionally, some journals require different formats which depend on
% the ``sort of theorem'': e.g.\ often remarks and definitions are set
% in |\upshape|, while |\itshape| is employed for main theorems.
%
% Confronted with these requirements, a theorem environment was
% developed in Mainz which allows separate determination of the
% layout of the ``theorems sets'', comparable to |\pagestyle|.
%
%
% \section{The user interface}
% \subsection{Defining new theorem sets}
%
% \DescribeMacro\newtheorem
% As in the original \LaTeX{} version, the command |\newtheorem|
% defines a new ``theorem set'' or ``theorem-like structure''.
% Two required arguments name the new environment and give the text
% to be typeset with each instance of the new ``set'', while
% an optional argument determines how
% the ``set'' is enumerated:
% \begin{description}
%    \item[\ttfamily \bslash newtheorem\{foo\}\{bar\}]
%       The theorem set {\envfont foo} (whose name is \texttt{bar})
%       uses its own counter.
%    \item[\ttfamily \bslash newtheorem\{foo2\}{[foo]}\{bar2\}]
%       The theorem set {\envfont foo2} (printed name \texttt{bar2})
%       uses the same counter as the theorem set \texttt{foo}.
%    \item[\ttfamily \bslash newtheorem\{foo3\}\{bar3\}{[section]}]
%       The theorem set {\envfont foo3} (printed name \texttt{bar3}) is
%       enumerated within the counter \texttt{section}, i.e.\ with every
%       new |\section| the enumeration begins again with 1, and
%       the enumeration is composed from the section-number and the
%       theorem counter itself.
% \end{description}
%
% \DescribeMacro\theoremstyle
% Additionally, the command |\theoremstyle| can define the layout
% of various, or all, theorem sets.  It should be noted that any
% theorem set defined by |\newtheorem| is typeset in the
% |\theoremstyle| that is current at the time of the definition.
% Thus, the following
% \begin{verbatim}
%    \theoremstyle{break}        \newtheorem{Cor}{Corollary}
%    \theoremstyle{plain}        \newtheorem{Exa}{Example}[section]
%\end{verbatim}                             ^^A No blank at line start
% leads to the result that the set {\envfont Cor} is formatted in the
% style \texttt{break}, while the set {\envfont Exa} and all the
% following ones are formatted in the style \texttt{plain}, unless
% another |\theoremstyle| follows.  Since the definitions installed by
% |\newtheorem| are global, one also can limit |\theoremstyle| locally
% by grouping braces.
%
% \DescribeMacro\theorembodyfont
% The choice of the font for the theorem body is completely
% independent of the chosen |\theoremstyle|; this has proven to
% be very advantageous.  For example,
% \begin{verbatim}
%   {\theorembodyfont{\upshape}          \newtheorem{Rem}{Remark}}
%\end{verbatim}
% defines a theorem set {\envfont Rem}, which will be set in
% |\upshape| in the current layout (which in our example is
% \texttt{plain}).  As with |\theoremstyle|, the |\theorembodyfont|
% chosen is that current at the time of |\newtheorem|.  If
% |\theorembodyfont| is not specified or one defines
% |\theorembodyfont{}|, then the font used will be that defined by the
% |\theoremstyle|.
%
% \DescribeMacro\theoremheaderfont It is also possible to customize
% the font used for the theorem headers.  This is, however, a global
% declaration, and therefore there should be at most one
% |\theoremheaderfont| declaration in the preamble.\footnote{If it is
% actually necessary to have different header fonts, one has to define
% new theorem styles (substituting the desired font) or specify the
% information directly in the \texttt{\bslash newtheorem} declaration
% (the unclean variant).}
%
% \DescribeMacro\theorempreskipamount
% \DescribeMacro\theorempostskipamount Two additional parameters
% affect the vertical space around the theorem environments:
% |\theorempreskipamount| and |\theorempostskipamount| define,
% respectively, the spacing before and after such an environment.
% These parameters apply for all theorem sets and can be manipulated
% with the ordinary length macros.  They are rubber lengths
% (`\textsf{skips}'), and therefore can contain \texttt{plus} and
% \texttt{minus} parts.
%
% Since the definition of theorem sets should---most sensibly---be
% placed in the preamble, we only allow installation there.
% It is therefore possible to release the memory used here after
% |\begin{document}|, in order to make room for other
% applications.
%
%
% \subsection{Existing theorem styles}
%
% The following theorem styles exist to date:
% \begin{deflist}{marginbreak}
%    \item[plain]
%       This theorem style emulates the original \LaTeX{} definition,
%       except that additionally the parameters
%       |\theorem...skipamount| are used.
%    \item[break]
%       In this style, the theorem header is followed by a line
%       break.
%    \item[marginbreak]
%       The theorem number is set in the margin, and there is a line
%       break as in \texttt{break}.
%    \item[changebreak]
%       Like \texttt{break}, but with header number and text
%       interchanged.
%    \item[change]
%       Header number and text are interchanged, without a line break.
%    \item[margin]
%       The number is set in the left margin, without a line break.
% \end{deflist}
% All styles (except \texttt{plain}) select |\slshape| as the default
% |\theorembodyfont|.
%
%
% \subsection{Examples}
%
% Given the above theorem sets {\envfont Cor}, {\envfont Exa}
% and {\envfont Rem},
% suppose that the preamble also contains the declarations:
% \begin{verbatim}
%    \theoremstyle{marginbreak}   \newtheorem{Lem}[Cor]{Lemma}
%    \theoremstyle{change}
%    \theorembodyfont{\itshape}        \newtheorem{Def}[Cor]{Definition}
%
%    \theoremheaderfont{\scshape}
%\end{verbatim}
% Then the following are some typical examples of the typeset output
% resulting from their use.
%
% \begin{Cor}
%    This is a sentence typeset in the theorem environment {\envfont
%    Cor}.
% \end{Cor}
%
% \begin{Exa}
%    This is a sentence typeset in the theorem environment {\envfont
%    Exa}.
% \end{Exa}
%
% \begin{Rem}
%    This is a sentence typeset in the theorem environment {\envfont
%    Rem}.
% \end{Rem}
%
% \begin{Lem}[Ben User]
%    This is a sentence typeset in the theorem environment {\envfont
%    Lem}.
% \end{Lem}
%
% \begin{Def}[Very impressive Definition]
%    This is a sentence typeset in the theorem environment {\envfont
%    Def}.
% \end{Def}
%
% The last two examples show the effect of the optional argument to
% a theorem environment (it is the text typeset in parentheses).
%
%
% \section{Special Considerations}
% \label{sec:newfonts}
%
% Theoremheader and body are implemented as a unit. This means that
% the |\theoremheaderfont|  will inherit characteristics of the
% |\theorembodyfont| in \LaTeXe. Thus, if for example
% |\theorembodyfont| is |\itshape| and |\theoremheaderfont|
% is |\bfseries| the font selected for the header will have the
% characteristics `bold extended italic'. If this is not desired one
% should set the |\theoremheaderfont| to something like
% \begin{verbatim}
%      \theoremheaderfont{\normalfont\bfseries}
%\end{verbatim}
% i.e.\ supplying all necessary font information explicitly.
%
%
%
% \section{Acknowledgements}
%
% The publication of this set of macros was only possible with the
% help of Christina Busse (translating the manuscript into English),
% Joachim Pense (playing the r\^ole of typist), Chris Rowley
% (looking everything over) and many others providing useful
% suggestions.
%
% \MaybeStop{%
%    \begin{thebibliography}{9}
%       \bibitem{bk:GMS94} \textsc{M.~Goossens}, \textsc{F.~Mittelbach}
%         and \textsc{A.~Samarin}.
%         \newblock The \LaTeX{} Companion.
%         \newblock
%         Addison-Wesley, Reading, Massachusetts, 1994.
%       \bibitem{latex:source} \textsc{Lamport, Leslie.} \newblock
%                               \texttt{latex.tex}, version 2.09, date
%                               Feb. 1990.
%    \end{thebibliography}
%    \PrintIndex
%    \PrintChanges}
%
%
% \section{The documentation driver file}
%
% The next bit of code contains the documentation driver file for
% \TeX{}, i.e., the file that will produce the documentation you are
% currently reading. It will be extracted from this file by the
% \texttt{docstrip} program. Since it is the first code in the file
% one can alternatively process this file directly with \LaTeXe{} to
% obtain the documentation.
%
%    \begin{macrocode}
%<*driver>
\documentclass{ltxdoc}

\usepackage{theorem}

%<+driver>% The next few lines define theorem sets which are used
%<+driver>% in the example section of the documentation.

\theoremstyle{break}       \newtheorem{Cor}{Corollary}
\theoremstyle{plain}       \newtheorem{Exa}{Example}[section]
{\theorembodyfont{\upshape}\newtheorem{Rem}{Remark}}
\theoremstyle{marginbreak} \newtheorem{Lem}[Cor]{Lemma}
\theoremstyle{change}
\theorembodyfont{\itshape} \newtheorem{Def}[Cor]{Definition}

\theoremheaderfont{\scshape}

\RecordChanges

\begin{document}
   \DocInput{theorem.dtx}
\end{document}
%</driver>
%    \end{macrocode}
%
% \section{Definition of the Macros}
%
% If the file has been loaded before, we abort immediately. If not the
% package announces itself (this is actually done at the very top of
% the file---the way it is done isn't good style so don't copy it).
% \changes{v2.0e}{89/07/19}{Spaces removed from `typeout'.}
% \changes{v2.2b}{95/11/19}{Announce moved to top of file}
%    \begin{macrocode}
%<*package>
%\@ifundefined{theorem@style}{}{\endinput}
%\def\FMithmInfo{1995/11/19 v2.2b Theorem extension package (FMi)}
%\ProvidesPackage{theorem}[\FMithmInfo]
%    \end{macrocode}
%
%
% \subsection{Definition of theorem styles and fonts}
%
% All the definitions in this file are done globally to allow
% inputting this file inside a group.
%
% \begin{macro}{\theoremstyle}
%    Before a theorem style can be installed, the chosen style must be
%    known. For that reason, we must test to see that
%    |\th@|\meta{style} is known or, more precisely, that it is
%    different from |\relax|. If the style is not known then
%    |\th@plain| is used.
%    \begin{macrocode}
\gdef\theoremstyle#1{%
   \@ifundefined{th@#1}{\@warning
          {Unknown theoremstyle `#1'. Using `plain'}%
          \theorem@style{plain}}%
%    \end{macrocode}
%    We save the theorem style to be used in the token register
%    |\theorem@style|.
%    \begin{macrocode}
      {\theorem@style{#1}}%
%    \end{macrocode}
%    Now we ``evaluate'' the theorem style: this means, we call the
%    macro |\th@|\meta{style} which will activate the relevant
%    definitions which are contained in a separate file.  This is done
%    in a group to suppress changes to the current font.  This could
%    otherwise pose problems together with the new font selection
%    scheme\footnote{When I printed the original article using the new
%    font selection scheme I ended with a document with slanted
%    typefaces (text headings and all) simply because one of the
%    theorem styles used \texttt{\bslash sl} at toplevel.} if the
%    |\th@|\meta{style} is evaluated a second time.
% \changes{v2.1a}{90/02/15}{Group added around `th@..}
% \changes{v2.1a}{90/02/15}{Reseting of `..@begintheorem removed.}
%    \begin{macrocode}
      \begingroup
        \csname th@\the\theorem@style \endcsname
      \endgroup}
%    \end{macrocode}
% \end{macro}
%
%
% \begin{macro}{\@begintheorem}
% \changes{v2.1a}{90/02/15}{Relaxed at toplevel.}
% \begin{macro}{\@opargbegintheorem}
% \changes{v2.1a}{90/02/15}{Relaxed at toplevel.}
%    We reset |\@begintheorem| and |\@opargbegintheorem| to
%    |\relax| since these commands are no longer necessary at
%    toplevel. This will save a few tokens.
%    \begin{macrocode}
\global\let\@begintheorem\relax
\global\let\@opargbegintheorem\relax
%    \end{macrocode}
% \end{macro}
% \end{macro}
%
% \begin{macro}{\theorem@style}
%    Obviously the \textsf{token} register used above has to be
%    allocated. To assure the utmost compatibility with the
%    original \LaTeX{} definition, we set the default theorem style to
%    \texttt{plain}, which implements the usual \LaTeX{} convention.
% \changes{v2.0f}{89/07/20}{Unnecessary `global' removed.}
%    \begin{macrocode}
\newtoks\theorem@style
\global\theorem@style{plain}
%    \end{macrocode}
% \end{macro}
%
% \begin{macro}{\theorembodyfont}
% \begin{macro}{\theorem@bodyfont}
%    If the |\theorembodyfont| is set by the user then it should not
%    interact with the default font set in the theorem style. When the
%    new font selection is in force this may happen if, for example,
%    the default is |\itshape| and the new |\theorembodyfont| is
%    |\sffamily|. So we add a |\reset@font| command in front of the
%    user definition.
% \changes{v2.0f}{89/07/20}{Unnecessary `global' removed.}
% \changes{v2.0g}{89/09/19}{`theoremfont to `theorembodyfont.}
% \changes{v2.1c}{92/06/22}{`reset@font added and logic changed.}
% \changes{v2.1d}{92/10/28}{Reset to empty if arg is empty}
%    \begin{macrocode}
\gdef\theorembodyfont#1{%
%    \end{macrocode}
%    We check if the argument supplied is empty and if so put nothing
%    into the |\theorem@bodyfont| token register to allow for
%    |\theorembodyfont{}| as a mean of using the default of the
%    current |\theoremstyle|.
%    \begin{macrocode}
   \def\@tempa{#1}%
   \ifx\@tempa\@empty
    \theorem@bodyfont{}%
   \else
    \theorem@bodyfont{\reset@font#1}%
   \fi
   }
\newtoks\theorem@bodyfont
\global\theorem@bodyfont{}
%    \end{macrocode}
% \end{macro}
% \end{macro}
%
%
% \begin{macro}{\theoremheaderfont}
% \changes{v2.0g}{89/09/19}{Macro added.}
%    The font for the theorem headers is handled differently because
%    this definition applies to all theorem styles.
%    \begin{macrocode}
\gdef\theoremheaderfont#1{\gdef\theorem@headerfont{#1}%
%    \end{macrocode}
%    After using the macro once it is redefined to produce an error
%    message.
%    \begin{macrocode}
       \gdef\theoremheaderfont##1{%
        \typeout{\string\theoremheaderfont\space should be used
                 only once.}}}
%    \end{macrocode}
% \end{macro}
%
% \begin{macro}{\theorem@headerfont}
% \changes{v2.0g}{89/09/19}{Macro added.}
%    To set the |\theorem@headerfont| default we first test if
%    the new fontselection scheme is in force.
% \changes{v2.1a}{90/02/15}{Support for new font selection.}
%    \begin{macrocode}
\ifx\upshape\undefined
%    \end{macrocode}
%    If not we define it to expand into |\bfseries|. We don't use
%    |\let| just in case a following style option redefines this
%    macro.
%    \begin{macrocode}
\gdef\theorem@headerfont{\bfseries}
%    \end{macrocode}
%    Otherwise we reset the current shape before calling |\bfseries|.
%    \begin{macrocode}
\else \gdef\theorem@headerfont{\normalfont\bfseries}\fi
%    \end{macrocode}
% \end{macro}
%
% \begin{macro}{\th@plain}
% \begin{macro}{\th@break}
% \begin{macro}{\th@marginbreak}
% \begin{macro}{\th@changebreak}
% \begin{macro}{\th@change}
% \begin{macro}{\th@margin}
%    The different styles are defined in macros such as
%    |\th@plain|.  Since memory space is precious in
%    ``non-Big-versions'', we have to avoid offering too many unused
%    definitions.  Therefore we define these styles in separate files
%    that can be loaded on demand.  Thus the commands themselves only
%    load these files. We use |\@input@|, a \LaTeXe{} internal command
%    that ensures that the file will be listed with |\listfiles|.
% \changes{v2.2a}{94/02/02}{Use `@input@ to load theorem layout files}
%    \begin{macrocode}
\gdef\th@plain{\@input@{thp.sty}}
\gdef\th@break{\@input@{thb.sty}}
\gdef\th@marginbreak{\@input@{thmb.sty}}
\gdef\th@changebreak{\@input@{thcb.sty}}
\gdef\th@change{\@input@{thc.sty}}
\gdef\th@margin{\@input@{thm.sty}}
%    \end{macrocode}
%    This list will be expanded when new styles become
%    available.  For testing,  just append new theorem substyles
%    as document options.
% \end{macro}
% \end{macro}
% \end{macro}
% \end{macro}
% \end{macro}
% \end{macro}
%
%
%
% \subsection{Definition of a new theorem set}
%
% As already pointed out, a new theorem environment can be defined in
% three different ways:
% \begin{verbatim}
%    \newtheorem{Lem}{Lemma}
%    \newtheorem{Lem}{Lemma}[section]
%    \newtheorem{Lem}[Theorem]{Lemma}
%\end{verbatim}
% The function of the macro |\newtheorem| is to recognize these cases
% and then to branch into one of the three macros |\@ynthm|, |\@xnthm|
% or |\@othm|.  This mechanism is adopted unchanged from
% \cite{latex:source}; the essential point here is that, for example,
% in the second case, the arguments \texttt{Lem}, \texttt{Lemma} and
% \texttt{section} are passed over to the macro |\@xnthm|.
%
% We inspect this case first because the others present fewer problems,
% and thus are easily derived from this one.
%
% \begin{macro}{\@xnthm}
%    For our example arguments, the macro |\@xnthm| must fulfill
%    the following:
%    \begin{itemize}
%       \item
%          Define a new \LaTeX-counter `Lem'
%       \item
%          reset this counter within a |\section|
%       \item
%          define the macro |\theLem|
%       \item
%          define the environment macros |\Lem| and
%          |\endLem| using the current
%          |\theoremstyle| and |\theorem@bodyfont|.
%    \end{itemize}
%    Obviously, all this should happen only if the first argument
%    of |\@xnthm| (i.e.\ \texttt{Lem} in our example) is chosen so
%    as not to conflict with any previously defined commands or
%    environments.  This test is performed by the \LaTeX{} macro
%    |\@ifdefinable|.
%    \begin{macrocode}
\gdef\@xnthm#1#2[#3]{\expandafter\@ifdefinable\csname #1\endcsname
%    \end{macrocode}
%    Therefore, the first argument of |\@ifdefinable| is the
%    expansion (in the example, |\Lem|) of
%    |\csname||#1\endcsname|.  The second argument is
%    executed only if the test has been completed successfully.
%    \begin{macrocode}
   {%
%    \end{macrocode}
%    Now we define the new counter. The names of the \LaTeX{}
%    macros employed should speak for themselves:
% \changes{v2.2b}{1995/10/07}{Use \cs{@newctr}}
% \changes{v2.2c}{1995/11/23}{Use \cs{@newctr} properly.}
%    \begin{macrocode}
    \@definecounter{#1}\@newctr{#1}[#3]%
%    \end{macrocode}
%    Using |\@newctr| will give a proper error message if the counter
%    in |#3| is not defined.
%    In defining `|\theLem|' we must generate the desired macro
%    name by use of |\expandafter| and |\csname|.
%    \begin{macrocode}
    \expandafter\xdef\csname the#1\endcsname
%    \end{macrocode}
%    An |\xdef| is used in order to make the definition global,
%    and to ensure that it contains the replacement texts of
%    |\@thmcountersep| and |\@thmcounter|.\footnote{These
%    two macros can be defined by the document style.  Their default
%    values produce a `\texttt{.}' as separation and an arabic
%    representation of the number.} However, not everything should be
%    expanded. For example, it saves space to use |\thesection|
%    instead of its---at times---lengthy expansion.
%    \begin{macrocode}
      {\expandafter \noexpand \csname the#3\endcsname
       \@thmcountersep \@thmcounter{#1}}%
%    \end{macrocode}
%    Thus with the defaults of \LaTeX{}, |\theLem| would be
%    replaced by the command sequence |\thesection.\arabic{Lem}|.
%
%    We will now look at the definition of the macro which is executed
%    at the beginning of the actual environment (in our example this
%    macro is |\Lem|). It should be noted that we use an
%    ``|\expandafter| trick'' to expand only certain parts of the
%    replacement text at the time of the definition.
%    \begin{macrocode}
    \def\@tempa{\global\@namedef{#1}}%
    \expandafter \@tempa \expandafter{%
%    \end{macrocode}
%    First, the macro that contains the current definitions of
%    |\@begintheorem| and |\@opargtheorem| should be called
%    up.  The name of this macro---as is already known---has the form
%    |\th@|\meta{theorem\ style}; therefore, it must be called by
%    \begin{macrocode}
      \csname th@\the \theorem@style
%    \end{macrocode}
%    In addition the default theorem font should be changeable, i.e.\
%    we have to insert the contents of |\theorem@bodyfont|. For
%    that reason, we expand even further, beyond |\endcsname|,
%    and thus insert the contents of the \textsf{token} register
%    |\theorem@bodyfont| in the replacement text.
%    \begin{macrocode}
            \expandafter \endcsname \the \theorem@bodyfont
%    \end{macrocode}
%    Now it is time to call the macro |\@thm| which takes over
%    the further processing.  It has two arguments: the current
%    counter name (in our example, \texttt{Lem}), and the text of the
%    label (in our example, \texttt{Lemma}).
%    \begin{macrocode}
     \@thm{#1}{#2}}%
%    \end{macrocode}
%    With this, the `sub-definition' is complete.  The macro
%    |\@endtheorem| ends a theorem environment and is, so far, nothing
%    but an |\endtrivlist|.  (Hence it is defined globally, and not
%    within the theorem styles.\footnote{This has to be changed as
%    soon as theorem styles that change \texttt{\bslash @endtheorem}
%    exist.  In such a case, all existing styles must be changed as
%    well since they will have to reset the macro.}) Therefore, we can
%    set it equivalent to the macro that ends the theorem set (in our
%    example, |\endLem|).  However, if some day theorem styles exist
%    that do change |\@endtheorem|, we would have to use the
%    commented-out line instead.
%    \begin{macrocode}
    \global \expandafter \let \csname end#1\endcsname \@endtheorem
%   \global\@namedef{end#1}{\@endtheorem}%
%    \end{macrocode}
%    With these commands all the required definitions are employed,
%    unless the test |\@ifdefinable| has failed.  Therefore, we
%    end the second argument of this macro and with it the definition
%    of |\@xnthm|.
%    \begin{macrocode}
   }}
%    \end{macrocode}
% \end{macro}
%
% \begin{macro}{\@ynthm}
%    The definition of |\@ynthm| is completely analogous. In
%    this case the new counter that is defined is not reset within
%    another counter; thus the definition of |\the...|
%    is simplified:
%    \begin{macrocode}
\gdef\@ynthm#1#2{\expandafter\@ifdefinable\csname #1\endcsname
   {\@definecounter{#1}%
    \expandafter\xdef\csname the#1\endcsname{\@thmcounter{#1}}%
%    \end{macrocode}
%    The rest of the definition corresponds literally to that of
%    |\@xnthm|:
%    \begin{macrocode}
    \def\@tempa{\global\@namedef{#1}}\expandafter \@tempa
     \expandafter{\csname th@\the \theorem@style \expandafter
     \endcsname \the\theorem@bodyfont \@thm{#1}{#2}}%
    \global \expandafter \let \csname end#1\endcsname \@endtheorem}}
%    \end{macrocode}
% \end{macro}
%
% \begin{macro}{\@othm}
%    The definition of |\@othm| does not contain anything new.
%    \begin{macrocode}
\gdef\@othm#1[#2]#3{%
%    \end{macrocode}
% \changes{v2.2b}{95/10/07}{Build in check for counter defined}
%    We do not define a new counter but instead use one that has
%    already been defined.  Thus the only definition we need is that
%    of this pseudo-counter (i.e.\ |\the|\meta{env.\ name}).
%    First we check if |#2| corresponds to a known counter name.
%    \begin{macrocode}
  \expandafter\ifx\csname c@#2\endcsname\relax
   \@nocounterr{#2}%
  \else
   \expandafter\@ifdefinable\csname #1\endcsname
   {\expandafter \xdef \csname the#1\endcsname
     {\expandafter \noexpand \csname the#2\endcsname}%
%    \end{macrocode}
%    All other parts of the definition can be adopted from
%    |\@xnthm|. We have to remember, though, that in this case
%    the name of the current counter and the theorem label have moved
%    to the second and third arguments.
%    \begin{macrocode}
    \def\@tempa{\global\@namedef{#1}}\expandafter \@tempa
     \expandafter{\csname th@\the \theorem@style \expandafter
     \endcsname \the\theorem@bodyfont \@thm{#2}{#3}}%
    \global \expandafter \let \csname end#1\endcsname \@endtheorem}%
  \fi}
%    \end{macrocode}
% \end{macro}
%
% \subsection{Macros that are employed in a {\envfont theorem}
%    environment}
%
% \begin{macro}{\@thm}
%    The macro |\@thm| has to increase the current counter.  Then,
%    depending on whether the environment has (or does not have) an
%    optional argument, it has to branch into either |\@begintheorem|
%    or |\@opargtheorem|.
%    \begin{macrocode}
\gdef\@thm#1#2{\refstepcounter{#1}%
%    \end{macrocode}
%    Now we start a {\envfont trivlist} environment, and give
%    |\@topsep| and |\@topsepadd| the values of the \textsf{skip}
%    registers |\theorempreskipamount| and |\theorempostskipamount|.
%    The value in |\@topsep| is the vertical space that is inserted by
%    the first (and only) |\item| in our |\trivlist| whilst
%    |\@topsepadd| is inserted by |\@endparenv| at the end of that
%    {\envfont trivlist} environment.  By using these registers, we
%    obtain the desired space around a {\envfont theorem} environment.
%    \begin{macrocode}
   \trivlist
   \@topsep \theorempreskipamount               % used by first \item
   \@topsepadd \theorempostskipamount           % used by \@endparenv
%    \end{macrocode}
%    Now we have to test whether an optional argument has been given.
% \changes{v2.0d}{89/07/18}{Forgotten percent added.}
%    \begin{macrocode}
   \@ifnextchar [%
%    \end{macrocode}
%    If there is an optional argument, we will call |\@ythm|, and
%    move the arguments read back into the input stream.
%    \begin{macrocode}
   {\@ythm{#1}{#2}}%
%    \end{macrocode}
%    If not, we call |\@begintheorem|. Its first argument is the
%    name of the theorem set (hence the second argument of
%    |\@thm|). Its second argument is the macro that produces the
%    current number.
%    \begin{macrocode}
   {\@begintheorem{#2}{\csname the#1\endcsname}\ignorespaces}}
%    \end{macrocode}
% \end{macro}
%
% \begin{macro}{\@xthm}
% \begin{macro}{\@ythm}
%    Both these macros were originally called by |\@thm|.  We do
%    not need |\@xthm| anymore, hence we reset it to
%    |\relax|. The definition of |\@ythm| has not changed at
%    all from its definition in \LaTeX. In order to make the macros
%    easier to understand, we will
%    nevertheless present it (commented out).
%    \begin{macrocode}
\global\let\@xthm\relax
% \def\@ythm#1#2[#3]{\@opargbegintheorem{#2}{\csname
%           the#1\endcsname}{#3}\ignorespaces}
%</package>
%    \end{macrocode}
%    The primitive |\ignorespaces| in |\@ythm| and
%    |\@thm| is needed to remove the spaces between the
%    |\begin{...}| and the actual text.
% \end{macro}
% \end{macro}
%
%
% \subsection{Definition of the theorem substyles}
%
% As already pointed out, the theorem substyles, defined below, are
% only loaded when necessary.  Note that all these substyles, except
% \texttt{plain}, have |\slshape| as the default body font.
%
% \subsubsection{The \texttt{plain} style}
%
% \label{dok:th@plain} As the following macros use |@|, we have
% to locally set the |\catcode| of this symbol to ``letter''.
% This happens within a group, so that we do not have to worry about
% which |\catcode| that symbol had before.
%    \begin{macrocode}
%<*thp>
\begingroup \makeatletter
%    \end{macrocode}
% Since we are now within a group, we must make all definitions
% globally.  First we make sure that \texttt{theorem.sty} is loaded.
% This will allow us to use this file as a document style option
% without having to call \texttt{theorem} itself as an option.  At the
% same time, we assure that at least version 2 is loaded, since
% |\theorem@style| was not defined in earlier versions.
%    \begin{macrocode}
\@ifundefined{theorem@style}{\input{theorem.sty}}{}
\ProvidesFile{thp.sty}
             [\FMithmInfo]
%    \end{macrocode}
%
%
% \begin{macro}{\th@plain}
% \begin{macro}{\@begintheorem}
% \begin{macro}{\@opargbegintheorem}
%    |\theoremstyle{plain}| corresponds to the original
%    definition, except that the distances to the surrounding text are
%    determined by the parameters |\theorempreskipamount| and
%    |\theorempostskipamount|.
%    First we set the default body font.
%    \begin{macrocode}
\gdef\th@plain{\normalfont\itshape
%    \end{macrocode}
%    Then we define |\@begintheorem| and
%    |\@opargbegintheorem|. These two macros define how the
%    header of a theorem is typeset.  |\@opargbegintheorem| will
%    be called if a \textsf{theorem} environment with an optional
%    argument is encountered; otherwise, the header is constructed by
%    calling |\@begintheorem|.  If one of these macros is
%    executed, we are within a \textsf{trivlist} environment started by
%    |\@thm|.  So the theorem header is produced with an
%    |\item| command.
%
%    Instead of specifying the header font directly, all standard
%    theorem styles use the |\theorem@headerfont| macro to allow
%    customization.  The extra space (|\labelsep|) is necessary
%    because of problems in the \textsf{trivlist} environment.
%    \begin{macrocode}
  \def\@begintheorem##1##2{%
        \item[\hskip\labelsep \theorem@headerfont ##1\ ##2]}%
%    \end{macrocode}
%    The definition of |\@opargbegintheorem| is completely
%    analogous. The only difference is the fact that there exists a
%    third argument (which is the optional parameter of the
%    environment and contains additional information about the
%    theorem).  Customarily we enclose it in parentheses.
%    \begin{macrocode}
\def\@opargbegintheorem##1##2##3{%
   \item[\hskip\labelsep \theorem@headerfont ##1\ ##2\ (##3)]}}
%    \end{macrocode}
% \end{macro}
% \end{macro}
% \end{macro}
%
% \noindent
% We conclude with an |\endgroup| to restore the
% |\catcode| of |@|.
%    \begin{macrocode}
\endgroup
%</thp>
%    \end{macrocode}
%
%
% \subsubsection{The \texttt{break} style}
%
% This style option is stored in the file |thb.sty|.
% For the next two lines see the documentation for |\th@plain| on
% page~\pageref{dok:th@plain}.
%    \begin{macrocode}
%<*thb>
\begingroup \makeatletter
\@ifundefined{theorem@style}{\input{theorem.sty}}{}
\ProvidesFile{thb.sty}
             [\FMithmInfo]
%    \end{macrocode}
%
% \begin{macro}{\th@break}
%    |\theoremstyle{break}| produces a line break after the name
%    of the theorem.  The font is |\slshape|.  Hence, we define
%    |\th@break| as follows:
%    \begin{macrocode}
\gdef\th@break{\normalfont\slshape
  \def\@begintheorem##1##2{\item[%
%    \end{macrocode}
%    We run into the following problem: it is not possible to create
%    the header with |\item[|\meta{title}|]| and then start
%    a new line by, for example, |\mbox{}\\|. Such a definition
%    will fail whenever a list environment follows immediately.  With
%    the above construction, the |\mbox{}| causes the switch
%    |@inlabel| (cf.\ definition of |\list| and
%    |\trivlist| in \cite{latex:source}) to be set to \texttt{false}
%    and so the following list will insert additional vertical
%    space (|\topskip|). This is quite annoying. Therefore, we
%    create the line break within the |\item|. In order to ensure
%    that the text will begin at the proper position in the following
%    line, we simply pretend that the label does not take any
%    room.\footnote{This will lead to problems whenever very high
%    symbols occurring in the line tower into the
%    heading. So, something else has to be done here sometime.}
%    \begin{macrocode}
       \rlap{\vbox{\hbox{\hskip \labelsep\theorem@headerfont ##1\ ##2}%
                   \hbox{\strut}}}]}%
%    \end{macrocode}
%    Again, the definition of |\@opargbegintheorem| is
%    completely analogous.
%    \begin{macrocode}
\def\@opargbegintheorem##1##2##3{%
  \item[\rlap{\vbox{\hbox{\hskip \labelsep \theorem@headerfont
                     ##1\ ##2\ (##3)}%
                    \hbox{\strut}}}]}}
\endgroup
%</thb>
%    \end{macrocode}
% \end{macro}
%
%
% \subsubsection{The \texttt{changebreak} style}
%
%    \begin{macrocode}
% This style option is stored in the file |thcb.sty|.
%    \begin{macrocode}
%<*thcb>
\begingroup \makeatletter
\@ifundefined{theorem@style}{\input{theorem.sty}}{}
\ProvidesFile{thcb.sty}
             [\FMithmInfo]
%    \end{macrocode}
%
% \begin{macro}{\th@changebreak}
%    The \texttt{change-break} theorem style is like \texttt{break}
%    but with interchange of theorem name and theorem number.  Thus we
%    define |\th@changebreak| as follows:
%    \begin{macrocode}
\gdef\th@changebreak{\normalfont\slshape
  \def\@begintheorem##1##2{\item
       [\rlap{\vbox{\hbox{\hskip\labelsep \theorem@headerfont ##2\ ##1}%
                    \hbox{\strut}}}]}%
\def\@opargbegintheorem##1##2##3{%
  \item[\rlap{\vbox{\hbox{\hskip\labelsep \theorem@headerfont
                     ##2\ ##1\ (##3)}%
                    \hbox{\strut}}}]}}
\endgroup
%</thcb>
%    \end{macrocode}
% \end{macro}
%
%
% \subsubsection{The \texttt{change} style}
%
% This style option is stored in the file |thc.sty|.
%    \begin{macrocode}
%<*thc>
\begingroup \makeatletter
\@ifundefined{theorem@style}{\input{theorem.sty}}{}
\ProvidesFile{thc.sty}
             [\FMithmInfo]
%    \end{macrocode}
%
% \begin{macro}{\th@change}
%    The \texttt{change} theorem style corresponds to the
%    \texttt{change break} style without a linebreak after the header.
%    To say it in another way, it's the same as the \texttt{plain}
%    style but with number and name interchanged and |\slshape| as the
%    default font.
%    \begin{macrocode}
\gdef\th@change{\normalfont\slshape
  \def\@begintheorem##1##2{\item
       [\hskip\labelsep \theorem@headerfont ##2\ ##1]}%
\def\@opargbegintheorem##1##2##3{%
  \item[\hskip\labelsep \theorem@headerfont ##2\ ##1\ (##3)]}}
\endgroup
%</thc>
%    \end{macrocode}
% \end{macro}
%
%
% \subsubsection{The \texttt{marginbreak} style}
%
% This style option is the one used most often at Mainz.  It is saved
% in the file |thmb.sty|.
%    \begin{macrocode}
%<*thmb>
\begingroup \makeatletter
\@ifundefined{theorem@style}{\input{theorem.sty}}{}
\ProvidesFile{thmb.sty}
             [\FMithmInfo]
%    \end{macrocode}
%
% \begin{macro}{\th@marginbreak}
%    The \texttt{margin break} style is nearly the same as the
%    \texttt{change break} style. The only difference is the placement
%    of the theorem number.  We use |\llap| to place it in the left
%    margin.
%
%    In this style |\labelsep| denotes the separation between the
%    number and the text.
%    \begin{macrocode}
\gdef\th@marginbreak{\normalfont\slshape
  \def\@begintheorem##1##2{\item
       [\rlap{\vbox{\theorem@headerfont
                    \hbox{\llap{##2}\hskip\labelsep ##1}%
                    \hbox{\strut}}}]}%
\def\@opargbegintheorem##1##2##3{%
  \item[\rlap{\vbox{\theorem@headerfont
                    \hbox{\llap{##2}\hskip\labelsep ##1\ (##3)}%
                    \hbox{\strut}}}]}}
\endgroup
%</thmb>
%    \end{macrocode}
% \changes{v2.1c}{90/02/17}{`endgroup outside macrocode env.}
% \end{macro}
%
%
%
% \subsubsection{The \texttt{margin} style}
%
% This style option is stored in the file |thm.sty|.
%    \begin{macrocode}
%<*thm>
\begingroup \makeatletter
\@ifundefined{theorem@style}{\input{theorem.sty}}{}
\ProvidesFile{thm.sty}
             [\FMithmInfo]
%    \end{macrocode}
%
% \begin{macro}{\th@margin}
%    Again this is only a variant of the theorem styles described
%    above without any new ideas.
%    \begin{macrocode}
\gdef\th@margin{\normalfont\slshape
  \def\@begintheorem##1##2{\item
       [\theorem@headerfont \llap{##2}\hskip\labelsep ##1]}%
\def\@opargbegintheorem##1##2##3{%
  \item[\theorem@headerfont \llap{##2}\hskip\labelsep ##1\ (##3)]}}
\endgroup
%</thm>
%    \end{macrocode}
% \end{macro}
%
%
%
%
% \subsection{Final Definitions}
%
% \begin{macro}{\theorempreskipamount}
% \begin{macro}{\theorempostskipamount}
%    The \textsf{skip} parameters that regulate the vertical empty space
%    before and after the {\envfont theorem} environment have
%    to be allocated as well.
% \changes{v2.0f}{89/07/20}{Unnecessary `global' removed.}
%    \begin{macrocode}
%<*package>
\newskip\theorempreskipamount
\newskip\theorempostskipamount
%    \end{macrocode}
%    Since we have used the same values for all theorem sets, we now can
%    assign them.
%    \begin{macrocode}
\global\setlength\theorempreskipamount{12pt plus 5pt minus 3pt}
\global\setlength\theorempostskipamount{8pt plus 3pt minus 1.5pt}
%    \end{macrocode}
% \end{macro}
% \end{macro}
%
%
% \begin{macro}{\@endtheorem}
%    The same holds for the macro |\@endtheorem|, which ends a
%    {\envfont theorem} environment. Since it is the same for all
%    theorem sets, it is removed from the macros |\th@|\meta{style}.
%    It simply ends the {\envfont trivlist} environment, which was
%    begun in |\@thm|.
%    \begin{macrocode}
\global\let\@endtheorem=\endtrivlist
%    \end{macrocode}
% \end{macro}
%
%
%
% \begin{macro}{\@preamblecmds}
%    All macros defined above are to be used only in the preamble.
%    Therefore, we insert them in |\@preamblecmds| which will disable
%    them at begin document. This is done by the internal \LaTeXe{}
%    command |\@onlypreamble|.
% \changes{v2.0g}{89/09/19}{`theoremheaderfont and `theorembodyfont
%    added.}
% \changes{v2.2a}{94/02/03}{Use \LaTeXe{} interface}
%    \begin{macrocode}
\@onlypreamble\@xnthm
\@onlypreamble\@ynthm
\@onlypreamble\@othm
\@onlypreamble\newtheorem
\@onlypreamble\theoremstyle
\@onlypreamble\theorembodyfont
\@onlypreamble\theoremheaderfont
%    \end{macrocode}
% \end{macro}
%
% Finally we declare the \texttt{plain} theorem style to be the default.
% \changes{v2.1b}{90/06/04}{Fixed missing `item by declaring default
%    style.}
%    \begin{macrocode}
\theoremstyle{plain}
%</package>
%    \end{macrocode}
%
% \Finale

\endinput
}{}
\ProvidesFile{thmb.sty}
             [\FMithmInfo]
%    \end{macrocode}
%
% \begin{macro}{\th@marginbreak}
%    The \texttt{margin break} style is nearly the same as the
%    \texttt{change break} style. The only difference is the placement
%    of the theorem number.  We use |\llap| to place it in the left
%    margin.
%
%    In this style |\labelsep| denotes the separation between the
%    number and the text.
%    \begin{macrocode}
\gdef\th@marginbreak{\normalfont\slshape
  \def\@begintheorem##1##2{\item
       [\rlap{\vbox{\theorem@headerfont
                    \hbox{\llap{##2}\hskip\labelsep ##1}%
                    \hbox{\strut}}}]}%
\def\@opargbegintheorem##1##2##3{%
  \item[\rlap{\vbox{\theorem@headerfont
                    \hbox{\llap{##2}\hskip\labelsep ##1\ (##3)}%
                    \hbox{\strut}}}]}}
\endgroup
%</thmb>
%    \end{macrocode}
% \changes{v2.1c}{90/02/17}{`endgroup outside macrocode env.}
% \end{macro}
%
%
%
% \subsubsection{The \texttt{margin} style}
%
% This style option is stored in the file |thm.sty|.
%    \begin{macrocode}
%<*thm>
\begingroup \makeatletter
\@ifundefined{theorem@style}{% \iffalse meta-comment
%
% Copyright (C) 1993-2024
%
% The LaTeX Project and any individual authors listed elsewhere
% in this file.
%
% This file is part of the Standard LaTeX `Tools Bundle'.
% -------------------------------------------------------
%
% It may be distributed and/or modified under the
% conditions of the LaTeX Project Public License, either version 1.3c
% of this license or (at your option) any later version.
% The latest version of this license is in
%    https://www.latex-project.org/lppl.txt
% and version 1.3c or later is part of all distributions of LaTeX
% version 2005/12/01 or later.
%
% The list of all files belonging to the LaTeX `Tools Bundle' is
% given in the file `manifest.txt'.
%
% \fi
%
%%
%
%
\def\FMithmInfo{2023/07/05 v2.2c Theorem extension package (FMi)}
%
% \ProvidesFile{theorem.dtx}[\FMithmInfo]
%\iffalse   % this is a METACOMMENT !
%%
%% Package `theorem' to use with LaTeX2e
%% Copyright (C) 1989-1995 Frank Mittelbach, all rights reserved.
%<+package>\@ifundefined{theorem@style}{}{\endinput}
%<+package>\NeedsTeXFormat{LaTeX2e}[1995/06/01]
%<+package>\ProvidesPackage{theorem}
%<+package>         [\FMithmInfo]
%
% \fi
%
% \changes{v1.0}{87/05/07}{First draft.}
% \changes{v1.0a}{87/05/07}{`thistheoremstyle' added.}
% \changes{v1.1a}{87/07/29}{New concept: redefine the internal
%                          \protect\LaTeX{} commands `@xnthm' etc.}
% \changes{v1.1b}{87/07/31}{`theoremfont' used to specify font.}
% \changes{v1.2a}{88/01/08}{`@endtheorem' now globally defined.}
% \changes{v1.2b}{88/01/11}{`newtheorem' only allowed in the preamble.}
% \changes{v1.3a}{88/02/14}{`theoremfont' now a toks register.}
%
% \changes{v2.0a}{88/11/16}{New implementation and new documentation.}
% \changes{v2.0a}{88/11/16}{No layout changes but internal modifications
%                           old substyles won't work.}
% \changes{v2.0b}{89/01/30}{`global' used for all definitions.}
% \changes{v2.0c}{89/06/15}{Documentation translated to English.
%                            Thanks to LLL and Nina}
% \changes{v2.1a}{90/02/15}{Support for new font selection.}
% \changes{v2.2a}{94/02/03}{Cleanup for LaTeX2e}
%
% \DoNotIndex{\@endparenv,\@ifdefinable,\@ifnextchar,\@ifundefined}
% \DoNotIndex{\@namedef,\@nameuse,\@spaces,\@tempa,\@warning}
% \DoNotIndex{\ }
% \DoNotIndex{\begingroup,\bfseries}
% \DoNotIndex{\csname}
% \DoNotIndex{\def,\do,\docdate}
% \DoNotIndex{\endcsname,\endgroup,\endinput,\endtrivlist,\expandafter}
% \DoNotIndex{\filedate,\fileversion}
% \DoNotIndex{\gdef,\global}
% \DoNotIndex{\hbox,\hskip}
% \DoNotIndex{\ignorespaces,\input,\itshape,\item}
% \DoNotIndex{\labelsep,\let,\llap}
% \DoNotIndex{\makeatletter}
% \DoNotIndex{\noexpand}
% \DoNotIndex{\refstepcounter,\relax,\rlap}
% \DoNotIndex{\setlength,\slshape,\space,\string,\strut}
% \DoNotIndex{\the,\trivlist,\typeout}
% \DoNotIndex{\vbox}
% \DoNotIndex{\xdef}
%
% \def\deflabel#1{\ttfamily #1\hfill}
% \def\deflist#1{\begin{list}{}{\settowidth\labelwidth{\ttfamily #1}%
%                               \setlength\leftmargin\labelwidth
%                               \addtolength\leftmargin\labelsep
%                               \let\makelabel\deflabel}}
% \def\enddeflist{\end{list}}
% \def\envfont{\normalfont\ttfamily}
%
% \MakeShortVerb{\|}
%
% \newcommand{\TUB}{{\sl TUGboat\/}}
%
% \GetFileInfo{theorem.dtx}
% \title{An Extension of the
%        \LaTeX{} theorem environment\thanks{This file
%        has version number \fileversion, last
%        revised \filedate.}}
% \author{Frank Mittelbach}
%
% \MaintainedByLaTeXTeam{tools}
% \maketitle
%
% \begin{abstract}
%    The macros described in this paper yield an extension of the
%    \LaTeX{} theorem mechanism. It is designed to satisfy the
%    different requirements of various journals. Thus, the
%    layout of the ``theorems'' can be manipulated by determining a
%    ``style''. This article describes not only the use, but
%    also the definition, of the necessary macros.
% \end{abstract}
%
% \section*{Preface to version 2.2}
%
% For \LaTeXe{} this package did not need any fundamental changes. I
% only modified the messages generated so that theorem layout styles
% will show up with the |\listfiles| command and cleaned the section
% on the New Font Selection Scheme since this is now included in
% \LaTeX.
%
% \section*{Preface to version 2.1}
%
% This version is identical to 2.0g described in \TUB\ 10\#3 except
% for some internal defaults which are now set depending on the used
% font selection scheme.
%
% This was done to avoid unpleasant surprises if the new font selection
% scheme is in force. For further details see
% section~\ref{sec:newfonts} and \cite{bk:GMS94}.
%
% \section{Introduction}
%
% For our purposes here, ``theorems'' are labelled enunciations,
% often set off from the main text by extra space and a font change.
% Theorems, corollaries, conjectures, definitions, and remarks
% are all instances of ``theorems''.  The ``header'' of these structures
% is composed of a label (such as \textsc{Theorem} or \textsc{Remark})
% and a number which serializes an item in the sequence of items
% with the same label.
%
% Shortly after the introduction of \LaTeX{} at the Fachbereich
% Mathematik in Mainz, the desire to manipulate the layout of
% ``theorems'' arose.  In Mainz, the following two conventions came
% into general use:
% \begin{enumerate}
%   \item The number of the theorem is shown in the margin.
%   \item There is a line break at
%         the end of the theorem header.
% \end{enumerate}
%
% Additionally, some journals require different formats which depend on
% the ``sort of theorem'': e.g.\ often remarks and definitions are set
% in |\upshape|, while |\itshape| is employed for main theorems.
%
% Confronted with these requirements, a theorem environment was
% developed in Mainz which allows separate determination of the
% layout of the ``theorems sets'', comparable to |\pagestyle|.
%
%
% \section{The user interface}
% \subsection{Defining new theorem sets}
%
% \DescribeMacro\newtheorem
% As in the original \LaTeX{} version, the command |\newtheorem|
% defines a new ``theorem set'' or ``theorem-like structure''.
% Two required arguments name the new environment and give the text
% to be typeset with each instance of the new ``set'', while
% an optional argument determines how
% the ``set'' is enumerated:
% \begin{description}
%    \item[\ttfamily \bslash newtheorem\{foo\}\{bar\}]
%       The theorem set {\envfont foo} (whose name is \texttt{bar})
%       uses its own counter.
%    \item[\ttfamily \bslash newtheorem\{foo2\}{[foo]}\{bar2\}]
%       The theorem set {\envfont foo2} (printed name \texttt{bar2})
%       uses the same counter as the theorem set \texttt{foo}.
%    \item[\ttfamily \bslash newtheorem\{foo3\}\{bar3\}{[section]}]
%       The theorem set {\envfont foo3} (printed name \texttt{bar3}) is
%       enumerated within the counter \texttt{section}, i.e.\ with every
%       new |\section| the enumeration begins again with 1, and
%       the enumeration is composed from the section-number and the
%       theorem counter itself.
% \end{description}
%
% \DescribeMacro\theoremstyle
% Additionally, the command |\theoremstyle| can define the layout
% of various, or all, theorem sets.  It should be noted that any
% theorem set defined by |\newtheorem| is typeset in the
% |\theoremstyle| that is current at the time of the definition.
% Thus, the following
% \begin{verbatim}
%    \theoremstyle{break}        \newtheorem{Cor}{Corollary}
%    \theoremstyle{plain}        \newtheorem{Exa}{Example}[section]
%\end{verbatim}                             ^^A No blank at line start
% leads to the result that the set {\envfont Cor} is formatted in the
% style \texttt{break}, while the set {\envfont Exa} and all the
% following ones are formatted in the style \texttt{plain}, unless
% another |\theoremstyle| follows.  Since the definitions installed by
% |\newtheorem| are global, one also can limit |\theoremstyle| locally
% by grouping braces.
%
% \DescribeMacro\theorembodyfont
% The choice of the font for the theorem body is completely
% independent of the chosen |\theoremstyle|; this has proven to
% be very advantageous.  For example,
% \begin{verbatim}
%   {\theorembodyfont{\upshape}          \newtheorem{Rem}{Remark}}
%\end{verbatim}
% defines a theorem set {\envfont Rem}, which will be set in
% |\upshape| in the current layout (which in our example is
% \texttt{plain}).  As with |\theoremstyle|, the |\theorembodyfont|
% chosen is that current at the time of |\newtheorem|.  If
% |\theorembodyfont| is not specified or one defines
% |\theorembodyfont{}|, then the font used will be that defined by the
% |\theoremstyle|.
%
% \DescribeMacro\theoremheaderfont It is also possible to customize
% the font used for the theorem headers.  This is, however, a global
% declaration, and therefore there should be at most one
% |\theoremheaderfont| declaration in the preamble.\footnote{If it is
% actually necessary to have different header fonts, one has to define
% new theorem styles (substituting the desired font) or specify the
% information directly in the \texttt{\bslash newtheorem} declaration
% (the unclean variant).}
%
% \DescribeMacro\theorempreskipamount
% \DescribeMacro\theorempostskipamount Two additional parameters
% affect the vertical space around the theorem environments:
% |\theorempreskipamount| and |\theorempostskipamount| define,
% respectively, the spacing before and after such an environment.
% These parameters apply for all theorem sets and can be manipulated
% with the ordinary length macros.  They are rubber lengths
% (`\textsf{skips}'), and therefore can contain \texttt{plus} and
% \texttt{minus} parts.
%
% Since the definition of theorem sets should---most sensibly---be
% placed in the preamble, we only allow installation there.
% It is therefore possible to release the memory used here after
% |\begin{document}|, in order to make room for other
% applications.
%
%
% \subsection{Existing theorem styles}
%
% The following theorem styles exist to date:
% \begin{deflist}{marginbreak}
%    \item[plain]
%       This theorem style emulates the original \LaTeX{} definition,
%       except that additionally the parameters
%       |\theorem...skipamount| are used.
%    \item[break]
%       In this style, the theorem header is followed by a line
%       break.
%    \item[marginbreak]
%       The theorem number is set in the margin, and there is a line
%       break as in \texttt{break}.
%    \item[changebreak]
%       Like \texttt{break}, but with header number and text
%       interchanged.
%    \item[change]
%       Header number and text are interchanged, without a line break.
%    \item[margin]
%       The number is set in the left margin, without a line break.
% \end{deflist}
% All styles (except \texttt{plain}) select |\slshape| as the default
% |\theorembodyfont|.
%
%
% \subsection{Examples}
%
% Given the above theorem sets {\envfont Cor}, {\envfont Exa}
% and {\envfont Rem},
% suppose that the preamble also contains the declarations:
% \begin{verbatim}
%    \theoremstyle{marginbreak}   \newtheorem{Lem}[Cor]{Lemma}
%    \theoremstyle{change}
%    \theorembodyfont{\itshape}        \newtheorem{Def}[Cor]{Definition}
%
%    \theoremheaderfont{\scshape}
%\end{verbatim}
% Then the following are some typical examples of the typeset output
% resulting from their use.
%
% \begin{Cor}
%    This is a sentence typeset in the theorem environment {\envfont
%    Cor}.
% \end{Cor}
%
% \begin{Exa}
%    This is a sentence typeset in the theorem environment {\envfont
%    Exa}.
% \end{Exa}
%
% \begin{Rem}
%    This is a sentence typeset in the theorem environment {\envfont
%    Rem}.
% \end{Rem}
%
% \begin{Lem}[Ben User]
%    This is a sentence typeset in the theorem environment {\envfont
%    Lem}.
% \end{Lem}
%
% \begin{Def}[Very impressive Definition]
%    This is a sentence typeset in the theorem environment {\envfont
%    Def}.
% \end{Def}
%
% The last two examples show the effect of the optional argument to
% a theorem environment (it is the text typeset in parentheses).
%
%
% \section{Special Considerations}
% \label{sec:newfonts}
%
% Theoremheader and body are implemented as a unit. This means that
% the |\theoremheaderfont|  will inherit characteristics of the
% |\theorembodyfont| in \LaTeXe. Thus, if for example
% |\theorembodyfont| is |\itshape| and |\theoremheaderfont|
% is |\bfseries| the font selected for the header will have the
% characteristics `bold extended italic'. If this is not desired one
% should set the |\theoremheaderfont| to something like
% \begin{verbatim}
%      \theoremheaderfont{\normalfont\bfseries}
%\end{verbatim}
% i.e.\ supplying all necessary font information explicitly.
%
%
%
% \section{Acknowledgements}
%
% The publication of this set of macros was only possible with the
% help of Christina Busse (translating the manuscript into English),
% Joachim Pense (playing the r\^ole of typist), Chris Rowley
% (looking everything over) and many others providing useful
% suggestions.
%
% \MaybeStop{%
%    \begin{thebibliography}{9}
%       \bibitem{bk:GMS94} \textsc{M.~Goossens}, \textsc{F.~Mittelbach}
%         and \textsc{A.~Samarin}.
%         \newblock The \LaTeX{} Companion.
%         \newblock
%         Addison-Wesley, Reading, Massachusetts, 1994.
%       \bibitem{latex:source} \textsc{Lamport, Leslie.} \newblock
%                               \texttt{latex.tex}, version 2.09, date
%                               Feb. 1990.
%    \end{thebibliography}
%    \PrintIndex
%    \PrintChanges}
%
%
% \section{The documentation driver file}
%
% The next bit of code contains the documentation driver file for
% \TeX{}, i.e., the file that will produce the documentation you are
% currently reading. It will be extracted from this file by the
% \texttt{docstrip} program. Since it is the first code in the file
% one can alternatively process this file directly with \LaTeXe{} to
% obtain the documentation.
%
%    \begin{macrocode}
%<*driver>
\documentclass{ltxdoc}

\usepackage{theorem}

%<+driver>% The next few lines define theorem sets which are used
%<+driver>% in the example section of the documentation.

\theoremstyle{break}       \newtheorem{Cor}{Corollary}
\theoremstyle{plain}       \newtheorem{Exa}{Example}[section]
{\theorembodyfont{\upshape}\newtheorem{Rem}{Remark}}
\theoremstyle{marginbreak} \newtheorem{Lem}[Cor]{Lemma}
\theoremstyle{change}
\theorembodyfont{\itshape} \newtheorem{Def}[Cor]{Definition}

\theoremheaderfont{\scshape}

\RecordChanges

\begin{document}
   \DocInput{theorem.dtx}
\end{document}
%</driver>
%    \end{macrocode}
%
% \section{Definition of the Macros}
%
% If the file has been loaded before, we abort immediately. If not the
% package announces itself (this is actually done at the very top of
% the file---the way it is done isn't good style so don't copy it).
% \changes{v2.0e}{89/07/19}{Spaces removed from `typeout'.}
% \changes{v2.2b}{95/11/19}{Announce moved to top of file}
%    \begin{macrocode}
%<*package>
%\@ifundefined{theorem@style}{}{\endinput}
%\def\FMithmInfo{1995/11/19 v2.2b Theorem extension package (FMi)}
%\ProvidesPackage{theorem}[\FMithmInfo]
%    \end{macrocode}
%
%
% \subsection{Definition of theorem styles and fonts}
%
% All the definitions in this file are done globally to allow
% inputting this file inside a group.
%
% \begin{macro}{\theoremstyle}
%    Before a theorem style can be installed, the chosen style must be
%    known. For that reason, we must test to see that
%    |\th@|\meta{style} is known or, more precisely, that it is
%    different from |\relax|. If the style is not known then
%    |\th@plain| is used.
%    \begin{macrocode}
\gdef\theoremstyle#1{%
   \@ifundefined{th@#1}{\@warning
          {Unknown theoremstyle `#1'. Using `plain'}%
          \theorem@style{plain}}%
%    \end{macrocode}
%    We save the theorem style to be used in the token register
%    |\theorem@style|.
%    \begin{macrocode}
      {\theorem@style{#1}}%
%    \end{macrocode}
%    Now we ``evaluate'' the theorem style: this means, we call the
%    macro |\th@|\meta{style} which will activate the relevant
%    definitions which are contained in a separate file.  This is done
%    in a group to suppress changes to the current font.  This could
%    otherwise pose problems together with the new font selection
%    scheme\footnote{When I printed the original article using the new
%    font selection scheme I ended with a document with slanted
%    typefaces (text headings and all) simply because one of the
%    theorem styles used \texttt{\bslash sl} at toplevel.} if the
%    |\th@|\meta{style} is evaluated a second time.
% \changes{v2.1a}{90/02/15}{Group added around `th@..}
% \changes{v2.1a}{90/02/15}{Reseting of `..@begintheorem removed.}
%    \begin{macrocode}
      \begingroup
        \csname th@\the\theorem@style \endcsname
      \endgroup}
%    \end{macrocode}
% \end{macro}
%
%
% \begin{macro}{\@begintheorem}
% \changes{v2.1a}{90/02/15}{Relaxed at toplevel.}
% \begin{macro}{\@opargbegintheorem}
% \changes{v2.1a}{90/02/15}{Relaxed at toplevel.}
%    We reset |\@begintheorem| and |\@opargbegintheorem| to
%    |\relax| since these commands are no longer necessary at
%    toplevel. This will save a few tokens.
%    \begin{macrocode}
\global\let\@begintheorem\relax
\global\let\@opargbegintheorem\relax
%    \end{macrocode}
% \end{macro}
% \end{macro}
%
% \begin{macro}{\theorem@style}
%    Obviously the \textsf{token} register used above has to be
%    allocated. To assure the utmost compatibility with the
%    original \LaTeX{} definition, we set the default theorem style to
%    \texttt{plain}, which implements the usual \LaTeX{} convention.
% \changes{v2.0f}{89/07/20}{Unnecessary `global' removed.}
%    \begin{macrocode}
\newtoks\theorem@style
\global\theorem@style{plain}
%    \end{macrocode}
% \end{macro}
%
% \begin{macro}{\theorembodyfont}
% \begin{macro}{\theorem@bodyfont}
%    If the |\theorembodyfont| is set by the user then it should not
%    interact with the default font set in the theorem style. When the
%    new font selection is in force this may happen if, for example,
%    the default is |\itshape| and the new |\theorembodyfont| is
%    |\sffamily|. So we add a |\reset@font| command in front of the
%    user definition.
% \changes{v2.0f}{89/07/20}{Unnecessary `global' removed.}
% \changes{v2.0g}{89/09/19}{`theoremfont to `theorembodyfont.}
% \changes{v2.1c}{92/06/22}{`reset@font added and logic changed.}
% \changes{v2.1d}{92/10/28}{Reset to empty if arg is empty}
%    \begin{macrocode}
\gdef\theorembodyfont#1{%
%    \end{macrocode}
%    We check if the argument supplied is empty and if so put nothing
%    into the |\theorem@bodyfont| token register to allow for
%    |\theorembodyfont{}| as a mean of using the default of the
%    current |\theoremstyle|.
%    \begin{macrocode}
   \def\@tempa{#1}%
   \ifx\@tempa\@empty
    \theorem@bodyfont{}%
   \else
    \theorem@bodyfont{\reset@font#1}%
   \fi
   }
\newtoks\theorem@bodyfont
\global\theorem@bodyfont{}
%    \end{macrocode}
% \end{macro}
% \end{macro}
%
%
% \begin{macro}{\theoremheaderfont}
% \changes{v2.0g}{89/09/19}{Macro added.}
%    The font for the theorem headers is handled differently because
%    this definition applies to all theorem styles.
%    \begin{macrocode}
\gdef\theoremheaderfont#1{\gdef\theorem@headerfont{#1}%
%    \end{macrocode}
%    After using the macro once it is redefined to produce an error
%    message.
%    \begin{macrocode}
       \gdef\theoremheaderfont##1{%
        \typeout{\string\theoremheaderfont\space should be used
                 only once.}}}
%    \end{macrocode}
% \end{macro}
%
% \begin{macro}{\theorem@headerfont}
% \changes{v2.0g}{89/09/19}{Macro added.}
%    To set the |\theorem@headerfont| default we first test if
%    the new fontselection scheme is in force.
% \changes{v2.1a}{90/02/15}{Support for new font selection.}
%    \begin{macrocode}
\ifx\upshape\undefined
%    \end{macrocode}
%    If not we define it to expand into |\bfseries|. We don't use
%    |\let| just in case a following style option redefines this
%    macro.
%    \begin{macrocode}
\gdef\theorem@headerfont{\bfseries}
%    \end{macrocode}
%    Otherwise we reset the current shape before calling |\bfseries|.
%    \begin{macrocode}
\else \gdef\theorem@headerfont{\normalfont\bfseries}\fi
%    \end{macrocode}
% \end{macro}
%
% \begin{macro}{\th@plain}
% \begin{macro}{\th@break}
% \begin{macro}{\th@marginbreak}
% \begin{macro}{\th@changebreak}
% \begin{macro}{\th@change}
% \begin{macro}{\th@margin}
%    The different styles are defined in macros such as
%    |\th@plain|.  Since memory space is precious in
%    ``non-Big-versions'', we have to avoid offering too many unused
%    definitions.  Therefore we define these styles in separate files
%    that can be loaded on demand.  Thus the commands themselves only
%    load these files. We use |\@input@|, a \LaTeXe{} internal command
%    that ensures that the file will be listed with |\listfiles|.
% \changes{v2.2a}{94/02/02}{Use `@input@ to load theorem layout files}
%    \begin{macrocode}
\gdef\th@plain{\@input@{thp.sty}}
\gdef\th@break{\@input@{thb.sty}}
\gdef\th@marginbreak{\@input@{thmb.sty}}
\gdef\th@changebreak{\@input@{thcb.sty}}
\gdef\th@change{\@input@{thc.sty}}
\gdef\th@margin{\@input@{thm.sty}}
%    \end{macrocode}
%    This list will be expanded when new styles become
%    available.  For testing,  just append new theorem substyles
%    as document options.
% \end{macro}
% \end{macro}
% \end{macro}
% \end{macro}
% \end{macro}
% \end{macro}
%
%
%
% \subsection{Definition of a new theorem set}
%
% As already pointed out, a new theorem environment can be defined in
% three different ways:
% \begin{verbatim}
%    \newtheorem{Lem}{Lemma}
%    \newtheorem{Lem}{Lemma}[section]
%    \newtheorem{Lem}[Theorem]{Lemma}
%\end{verbatim}
% The function of the macro |\newtheorem| is to recognize these cases
% and then to branch into one of the three macros |\@ynthm|, |\@xnthm|
% or |\@othm|.  This mechanism is adopted unchanged from
% \cite{latex:source}; the essential point here is that, for example,
% in the second case, the arguments \texttt{Lem}, \texttt{Lemma} and
% \texttt{section} are passed over to the macro |\@xnthm|.
%
% We inspect this case first because the others present fewer problems,
% and thus are easily derived from this one.
%
% \begin{macro}{\@xnthm}
%    For our example arguments, the macro |\@xnthm| must fulfill
%    the following:
%    \begin{itemize}
%       \item
%          Define a new \LaTeX-counter `Lem'
%       \item
%          reset this counter within a |\section|
%       \item
%          define the macro |\theLem|
%       \item
%          define the environment macros |\Lem| and
%          |\endLem| using the current
%          |\theoremstyle| and |\theorem@bodyfont|.
%    \end{itemize}
%    Obviously, all this should happen only if the first argument
%    of |\@xnthm| (i.e.\ \texttt{Lem} in our example) is chosen so
%    as not to conflict with any previously defined commands or
%    environments.  This test is performed by the \LaTeX{} macro
%    |\@ifdefinable|.
%    \begin{macrocode}
\gdef\@xnthm#1#2[#3]{\expandafter\@ifdefinable\csname #1\endcsname
%    \end{macrocode}
%    Therefore, the first argument of |\@ifdefinable| is the
%    expansion (in the example, |\Lem|) of
%    |\csname||#1\endcsname|.  The second argument is
%    executed only if the test has been completed successfully.
%    \begin{macrocode}
   {%
%    \end{macrocode}
%    Now we define the new counter. The names of the \LaTeX{}
%    macros employed should speak for themselves:
% \changes{v2.2b}{1995/10/07}{Use \cs{@newctr}}
% \changes{v2.2c}{1995/11/23}{Use \cs{@newctr} properly.}
%    \begin{macrocode}
    \@definecounter{#1}\@newctr{#1}[#3]%
%    \end{macrocode}
%    Using |\@newctr| will give a proper error message if the counter
%    in |#3| is not defined.
%    In defining `|\theLem|' we must generate the desired macro
%    name by use of |\expandafter| and |\csname|.
%    \begin{macrocode}
    \expandafter\xdef\csname the#1\endcsname
%    \end{macrocode}
%    An |\xdef| is used in order to make the definition global,
%    and to ensure that it contains the replacement texts of
%    |\@thmcountersep| and |\@thmcounter|.\footnote{These
%    two macros can be defined by the document style.  Their default
%    values produce a `\texttt{.}' as separation and an arabic
%    representation of the number.} However, not everything should be
%    expanded. For example, it saves space to use |\thesection|
%    instead of its---at times---lengthy expansion.
%    \begin{macrocode}
      {\expandafter \noexpand \csname the#3\endcsname
       \@thmcountersep \@thmcounter{#1}}%
%    \end{macrocode}
%    Thus with the defaults of \LaTeX{}, |\theLem| would be
%    replaced by the command sequence |\thesection.\arabic{Lem}|.
%
%    We will now look at the definition of the macro which is executed
%    at the beginning of the actual environment (in our example this
%    macro is |\Lem|). It should be noted that we use an
%    ``|\expandafter| trick'' to expand only certain parts of the
%    replacement text at the time of the definition.
%    \begin{macrocode}
    \def\@tempa{\global\@namedef{#1}}%
    \expandafter \@tempa \expandafter{%
%    \end{macrocode}
%    First, the macro that contains the current definitions of
%    |\@begintheorem| and |\@opargtheorem| should be called
%    up.  The name of this macro---as is already known---has the form
%    |\th@|\meta{theorem\ style}; therefore, it must be called by
%    \begin{macrocode}
      \csname th@\the \theorem@style
%    \end{macrocode}
%    In addition the default theorem font should be changeable, i.e.\
%    we have to insert the contents of |\theorem@bodyfont|. For
%    that reason, we expand even further, beyond |\endcsname|,
%    and thus insert the contents of the \textsf{token} register
%    |\theorem@bodyfont| in the replacement text.
%    \begin{macrocode}
            \expandafter \endcsname \the \theorem@bodyfont
%    \end{macrocode}
%    Now it is time to call the macro |\@thm| which takes over
%    the further processing.  It has two arguments: the current
%    counter name (in our example, \texttt{Lem}), and the text of the
%    label (in our example, \texttt{Lemma}).
%    \begin{macrocode}
     \@thm{#1}{#2}}%
%    \end{macrocode}
%    With this, the `sub-definition' is complete.  The macro
%    |\@endtheorem| ends a theorem environment and is, so far, nothing
%    but an |\endtrivlist|.  (Hence it is defined globally, and not
%    within the theorem styles.\footnote{This has to be changed as
%    soon as theorem styles that change \texttt{\bslash @endtheorem}
%    exist.  In such a case, all existing styles must be changed as
%    well since they will have to reset the macro.}) Therefore, we can
%    set it equivalent to the macro that ends the theorem set (in our
%    example, |\endLem|).  However, if some day theorem styles exist
%    that do change |\@endtheorem|, we would have to use the
%    commented-out line instead.
%    \begin{macrocode}
    \global \expandafter \let \csname end#1\endcsname \@endtheorem
%   \global\@namedef{end#1}{\@endtheorem}%
%    \end{macrocode}
%    With these commands all the required definitions are employed,
%    unless the test |\@ifdefinable| has failed.  Therefore, we
%    end the second argument of this macro and with it the definition
%    of |\@xnthm|.
%    \begin{macrocode}
   }}
%    \end{macrocode}
% \end{macro}
%
% \begin{macro}{\@ynthm}
%    The definition of |\@ynthm| is completely analogous. In
%    this case the new counter that is defined is not reset within
%    another counter; thus the definition of |\the...|
%    is simplified:
%    \begin{macrocode}
\gdef\@ynthm#1#2{\expandafter\@ifdefinable\csname #1\endcsname
   {\@definecounter{#1}%
    \expandafter\xdef\csname the#1\endcsname{\@thmcounter{#1}}%
%    \end{macrocode}
%    The rest of the definition corresponds literally to that of
%    |\@xnthm|:
%    \begin{macrocode}
    \def\@tempa{\global\@namedef{#1}}\expandafter \@tempa
     \expandafter{\csname th@\the \theorem@style \expandafter
     \endcsname \the\theorem@bodyfont \@thm{#1}{#2}}%
    \global \expandafter \let \csname end#1\endcsname \@endtheorem}}
%    \end{macrocode}
% \end{macro}
%
% \begin{macro}{\@othm}
%    The definition of |\@othm| does not contain anything new.
%    \begin{macrocode}
\gdef\@othm#1[#2]#3{%
%    \end{macrocode}
% \changes{v2.2b}{95/10/07}{Build in check for counter defined}
%    We do not define a new counter but instead use one that has
%    already been defined.  Thus the only definition we need is that
%    of this pseudo-counter (i.e.\ |\the|\meta{env.\ name}).
%    First we check if |#2| corresponds to a known counter name.
%    \begin{macrocode}
  \expandafter\ifx\csname c@#2\endcsname\relax
   \@nocounterr{#2}%
  \else
   \expandafter\@ifdefinable\csname #1\endcsname
   {\expandafter \xdef \csname the#1\endcsname
     {\expandafter \noexpand \csname the#2\endcsname}%
%    \end{macrocode}
%    All other parts of the definition can be adopted from
%    |\@xnthm|. We have to remember, though, that in this case
%    the name of the current counter and the theorem label have moved
%    to the second and third arguments.
%    \begin{macrocode}
    \def\@tempa{\global\@namedef{#1}}\expandafter \@tempa
     \expandafter{\csname th@\the \theorem@style \expandafter
     \endcsname \the\theorem@bodyfont \@thm{#2}{#3}}%
    \global \expandafter \let \csname end#1\endcsname \@endtheorem}%
  \fi}
%    \end{macrocode}
% \end{macro}
%
% \subsection{Macros that are employed in a {\envfont theorem}
%    environment}
%
% \begin{macro}{\@thm}
%    The macro |\@thm| has to increase the current counter.  Then,
%    depending on whether the environment has (or does not have) an
%    optional argument, it has to branch into either |\@begintheorem|
%    or |\@opargtheorem|.
%    \begin{macrocode}
\gdef\@thm#1#2{\refstepcounter{#1}%
%    \end{macrocode}
%    Now we start a {\envfont trivlist} environment, and give
%    |\@topsep| and |\@topsepadd| the values of the \textsf{skip}
%    registers |\theorempreskipamount| and |\theorempostskipamount|.
%    The value in |\@topsep| is the vertical space that is inserted by
%    the first (and only) |\item| in our |\trivlist| whilst
%    |\@topsepadd| is inserted by |\@endparenv| at the end of that
%    {\envfont trivlist} environment.  By using these registers, we
%    obtain the desired space around a {\envfont theorem} environment.
%    \begin{macrocode}
   \trivlist
   \@topsep \theorempreskipamount               % used by first \item
   \@topsepadd \theorempostskipamount           % used by \@endparenv
%    \end{macrocode}
%    Now we have to test whether an optional argument has been given.
% \changes{v2.0d}{89/07/18}{Forgotten percent added.}
%    \begin{macrocode}
   \@ifnextchar [%
%    \end{macrocode}
%    If there is an optional argument, we will call |\@ythm|, and
%    move the arguments read back into the input stream.
%    \begin{macrocode}
   {\@ythm{#1}{#2}}%
%    \end{macrocode}
%    If not, we call |\@begintheorem|. Its first argument is the
%    name of the theorem set (hence the second argument of
%    |\@thm|). Its second argument is the macro that produces the
%    current number.
%    \begin{macrocode}
   {\@begintheorem{#2}{\csname the#1\endcsname}\ignorespaces}}
%    \end{macrocode}
% \end{macro}
%
% \begin{macro}{\@xthm}
% \begin{macro}{\@ythm}
%    Both these macros were originally called by |\@thm|.  We do
%    not need |\@xthm| anymore, hence we reset it to
%    |\relax|. The definition of |\@ythm| has not changed at
%    all from its definition in \LaTeX. In order to make the macros
%    easier to understand, we will
%    nevertheless present it (commented out).
%    \begin{macrocode}
\global\let\@xthm\relax
% \def\@ythm#1#2[#3]{\@opargbegintheorem{#2}{\csname
%           the#1\endcsname}{#3}\ignorespaces}
%</package>
%    \end{macrocode}
%    The primitive |\ignorespaces| in |\@ythm| and
%    |\@thm| is needed to remove the spaces between the
%    |\begin{...}| and the actual text.
% \end{macro}
% \end{macro}
%
%
% \subsection{Definition of the theorem substyles}
%
% As already pointed out, the theorem substyles, defined below, are
% only loaded when necessary.  Note that all these substyles, except
% \texttt{plain}, have |\slshape| as the default body font.
%
% \subsubsection{The \texttt{plain} style}
%
% \label{dok:th@plain} As the following macros use |@|, we have
% to locally set the |\catcode| of this symbol to ``letter''.
% This happens within a group, so that we do not have to worry about
% which |\catcode| that symbol had before.
%    \begin{macrocode}
%<*thp>
\begingroup \makeatletter
%    \end{macrocode}
% Since we are now within a group, we must make all definitions
% globally.  First we make sure that \texttt{theorem.sty} is loaded.
% This will allow us to use this file as a document style option
% without having to call \texttt{theorem} itself as an option.  At the
% same time, we assure that at least version 2 is loaded, since
% |\theorem@style| was not defined in earlier versions.
%    \begin{macrocode}
\@ifundefined{theorem@style}{\input{theorem.sty}}{}
\ProvidesFile{thp.sty}
             [\FMithmInfo]
%    \end{macrocode}
%
%
% \begin{macro}{\th@plain}
% \begin{macro}{\@begintheorem}
% \begin{macro}{\@opargbegintheorem}
%    |\theoremstyle{plain}| corresponds to the original
%    definition, except that the distances to the surrounding text are
%    determined by the parameters |\theorempreskipamount| and
%    |\theorempostskipamount|.
%    First we set the default body font.
%    \begin{macrocode}
\gdef\th@plain{\normalfont\itshape
%    \end{macrocode}
%    Then we define |\@begintheorem| and
%    |\@opargbegintheorem|. These two macros define how the
%    header of a theorem is typeset.  |\@opargbegintheorem| will
%    be called if a \textsf{theorem} environment with an optional
%    argument is encountered; otherwise, the header is constructed by
%    calling |\@begintheorem|.  If one of these macros is
%    executed, we are within a \textsf{trivlist} environment started by
%    |\@thm|.  So the theorem header is produced with an
%    |\item| command.
%
%    Instead of specifying the header font directly, all standard
%    theorem styles use the |\theorem@headerfont| macro to allow
%    customization.  The extra space (|\labelsep|) is necessary
%    because of problems in the \textsf{trivlist} environment.
%    \begin{macrocode}
  \def\@begintheorem##1##2{%
        \item[\hskip\labelsep \theorem@headerfont ##1\ ##2]}%
%    \end{macrocode}
%    The definition of |\@opargbegintheorem| is completely
%    analogous. The only difference is the fact that there exists a
%    third argument (which is the optional parameter of the
%    environment and contains additional information about the
%    theorem).  Customarily we enclose it in parentheses.
%    \begin{macrocode}
\def\@opargbegintheorem##1##2##3{%
   \item[\hskip\labelsep \theorem@headerfont ##1\ ##2\ (##3)]}}
%    \end{macrocode}
% \end{macro}
% \end{macro}
% \end{macro}
%
% \noindent
% We conclude with an |\endgroup| to restore the
% |\catcode| of |@|.
%    \begin{macrocode}
\endgroup
%</thp>
%    \end{macrocode}
%
%
% \subsubsection{The \texttt{break} style}
%
% This style option is stored in the file |thb.sty|.
% For the next two lines see the documentation for |\th@plain| on
% page~\pageref{dok:th@plain}.
%    \begin{macrocode}
%<*thb>
\begingroup \makeatletter
\@ifundefined{theorem@style}{\input{theorem.sty}}{}
\ProvidesFile{thb.sty}
             [\FMithmInfo]
%    \end{macrocode}
%
% \begin{macro}{\th@break}
%    |\theoremstyle{break}| produces a line break after the name
%    of the theorem.  The font is |\slshape|.  Hence, we define
%    |\th@break| as follows:
%    \begin{macrocode}
\gdef\th@break{\normalfont\slshape
  \def\@begintheorem##1##2{\item[%
%    \end{macrocode}
%    We run into the following problem: it is not possible to create
%    the header with |\item[|\meta{title}|]| and then start
%    a new line by, for example, |\mbox{}\\|. Such a definition
%    will fail whenever a list environment follows immediately.  With
%    the above construction, the |\mbox{}| causes the switch
%    |@inlabel| (cf.\ definition of |\list| and
%    |\trivlist| in \cite{latex:source}) to be set to \texttt{false}
%    and so the following list will insert additional vertical
%    space (|\topskip|). This is quite annoying. Therefore, we
%    create the line break within the |\item|. In order to ensure
%    that the text will begin at the proper position in the following
%    line, we simply pretend that the label does not take any
%    room.\footnote{This will lead to problems whenever very high
%    symbols occurring in the line tower into the
%    heading. So, something else has to be done here sometime.}
%    \begin{macrocode}
       \rlap{\vbox{\hbox{\hskip \labelsep\theorem@headerfont ##1\ ##2}%
                   \hbox{\strut}}}]}%
%    \end{macrocode}
%    Again, the definition of |\@opargbegintheorem| is
%    completely analogous.
%    \begin{macrocode}
\def\@opargbegintheorem##1##2##3{%
  \item[\rlap{\vbox{\hbox{\hskip \labelsep \theorem@headerfont
                     ##1\ ##2\ (##3)}%
                    \hbox{\strut}}}]}}
\endgroup
%</thb>
%    \end{macrocode}
% \end{macro}
%
%
% \subsubsection{The \texttt{changebreak} style}
%
%    \begin{macrocode}
% This style option is stored in the file |thcb.sty|.
%    \begin{macrocode}
%<*thcb>
\begingroup \makeatletter
\@ifundefined{theorem@style}{\input{theorem.sty}}{}
\ProvidesFile{thcb.sty}
             [\FMithmInfo]
%    \end{macrocode}
%
% \begin{macro}{\th@changebreak}
%    The \texttt{change-break} theorem style is like \texttt{break}
%    but with interchange of theorem name and theorem number.  Thus we
%    define |\th@changebreak| as follows:
%    \begin{macrocode}
\gdef\th@changebreak{\normalfont\slshape
  \def\@begintheorem##1##2{\item
       [\rlap{\vbox{\hbox{\hskip\labelsep \theorem@headerfont ##2\ ##1}%
                    \hbox{\strut}}}]}%
\def\@opargbegintheorem##1##2##3{%
  \item[\rlap{\vbox{\hbox{\hskip\labelsep \theorem@headerfont
                     ##2\ ##1\ (##3)}%
                    \hbox{\strut}}}]}}
\endgroup
%</thcb>
%    \end{macrocode}
% \end{macro}
%
%
% \subsubsection{The \texttt{change} style}
%
% This style option is stored in the file |thc.sty|.
%    \begin{macrocode}
%<*thc>
\begingroup \makeatletter
\@ifundefined{theorem@style}{\input{theorem.sty}}{}
\ProvidesFile{thc.sty}
             [\FMithmInfo]
%    \end{macrocode}
%
% \begin{macro}{\th@change}
%    The \texttt{change} theorem style corresponds to the
%    \texttt{change break} style without a linebreak after the header.
%    To say it in another way, it's the same as the \texttt{plain}
%    style but with number and name interchanged and |\slshape| as the
%    default font.
%    \begin{macrocode}
\gdef\th@change{\normalfont\slshape
  \def\@begintheorem##1##2{\item
       [\hskip\labelsep \theorem@headerfont ##2\ ##1]}%
\def\@opargbegintheorem##1##2##3{%
  \item[\hskip\labelsep \theorem@headerfont ##2\ ##1\ (##3)]}}
\endgroup
%</thc>
%    \end{macrocode}
% \end{macro}
%
%
% \subsubsection{The \texttt{marginbreak} style}
%
% This style option is the one used most often at Mainz.  It is saved
% in the file |thmb.sty|.
%    \begin{macrocode}
%<*thmb>
\begingroup \makeatletter
\@ifundefined{theorem@style}{\input{theorem.sty}}{}
\ProvidesFile{thmb.sty}
             [\FMithmInfo]
%    \end{macrocode}
%
% \begin{macro}{\th@marginbreak}
%    The \texttt{margin break} style is nearly the same as the
%    \texttt{change break} style. The only difference is the placement
%    of the theorem number.  We use |\llap| to place it in the left
%    margin.
%
%    In this style |\labelsep| denotes the separation between the
%    number and the text.
%    \begin{macrocode}
\gdef\th@marginbreak{\normalfont\slshape
  \def\@begintheorem##1##2{\item
       [\rlap{\vbox{\theorem@headerfont
                    \hbox{\llap{##2}\hskip\labelsep ##1}%
                    \hbox{\strut}}}]}%
\def\@opargbegintheorem##1##2##3{%
  \item[\rlap{\vbox{\theorem@headerfont
                    \hbox{\llap{##2}\hskip\labelsep ##1\ (##3)}%
                    \hbox{\strut}}}]}}
\endgroup
%</thmb>
%    \end{macrocode}
% \changes{v2.1c}{90/02/17}{`endgroup outside macrocode env.}
% \end{macro}
%
%
%
% \subsubsection{The \texttt{margin} style}
%
% This style option is stored in the file |thm.sty|.
%    \begin{macrocode}
%<*thm>
\begingroup \makeatletter
\@ifundefined{theorem@style}{\input{theorem.sty}}{}
\ProvidesFile{thm.sty}
             [\FMithmInfo]
%    \end{macrocode}
%
% \begin{macro}{\th@margin}
%    Again this is only a variant of the theorem styles described
%    above without any new ideas.
%    \begin{macrocode}
\gdef\th@margin{\normalfont\slshape
  \def\@begintheorem##1##2{\item
       [\theorem@headerfont \llap{##2}\hskip\labelsep ##1]}%
\def\@opargbegintheorem##1##2##3{%
  \item[\theorem@headerfont \llap{##2}\hskip\labelsep ##1\ (##3)]}}
\endgroup
%</thm>
%    \end{macrocode}
% \end{macro}
%
%
%
%
% \subsection{Final Definitions}
%
% \begin{macro}{\theorempreskipamount}
% \begin{macro}{\theorempostskipamount}
%    The \textsf{skip} parameters that regulate the vertical empty space
%    before and after the {\envfont theorem} environment have
%    to be allocated as well.
% \changes{v2.0f}{89/07/20}{Unnecessary `global' removed.}
%    \begin{macrocode}
%<*package>
\newskip\theorempreskipamount
\newskip\theorempostskipamount
%    \end{macrocode}
%    Since we have used the same values for all theorem sets, we now can
%    assign them.
%    \begin{macrocode}
\global\setlength\theorempreskipamount{12pt plus 5pt minus 3pt}
\global\setlength\theorempostskipamount{8pt plus 3pt minus 1.5pt}
%    \end{macrocode}
% \end{macro}
% \end{macro}
%
%
% \begin{macro}{\@endtheorem}
%    The same holds for the macro |\@endtheorem|, which ends a
%    {\envfont theorem} environment. Since it is the same for all
%    theorem sets, it is removed from the macros |\th@|\meta{style}.
%    It simply ends the {\envfont trivlist} environment, which was
%    begun in |\@thm|.
%    \begin{macrocode}
\global\let\@endtheorem=\endtrivlist
%    \end{macrocode}
% \end{macro}
%
%
%
% \begin{macro}{\@preamblecmds}
%    All macros defined above are to be used only in the preamble.
%    Therefore, we insert them in |\@preamblecmds| which will disable
%    them at begin document. This is done by the internal \LaTeXe{}
%    command |\@onlypreamble|.
% \changes{v2.0g}{89/09/19}{`theoremheaderfont and `theorembodyfont
%    added.}
% \changes{v2.2a}{94/02/03}{Use \LaTeXe{} interface}
%    \begin{macrocode}
\@onlypreamble\@xnthm
\@onlypreamble\@ynthm
\@onlypreamble\@othm
\@onlypreamble\newtheorem
\@onlypreamble\theoremstyle
\@onlypreamble\theorembodyfont
\@onlypreamble\theoremheaderfont
%    \end{macrocode}
% \end{macro}
%
% Finally we declare the \texttt{plain} theorem style to be the default.
% \changes{v2.1b}{90/06/04}{Fixed missing `item by declaring default
%    style.}
%    \begin{macrocode}
\theoremstyle{plain}
%</package>
%    \end{macrocode}
%
% \Finale

\endinput
}{}
\ProvidesFile{thm.sty}
             [\FMithmInfo]
%    \end{macrocode}
%
% \begin{macro}{\th@margin}
%    Again this is only a variant of the theorem styles described
%    above without any new ideas.
%    \begin{macrocode}
\gdef\th@margin{\normalfont\slshape
  \def\@begintheorem##1##2{\item
       [\theorem@headerfont \llap{##2}\hskip\labelsep ##1]}%
\def\@opargbegintheorem##1##2##3{%
  \item[\theorem@headerfont \llap{##2}\hskip\labelsep ##1\ (##3)]}}
\endgroup
%</thm>
%    \end{macrocode}
% \end{macro}
%
%
%
%
% \subsection{Final Definitions}
%
% \begin{macro}{\theorempreskipamount}
% \begin{macro}{\theorempostskipamount}
%    The \textsf{skip} parameters that regulate the vertical empty space
%    before and after the {\envfont theorem} environment have
%    to be allocated as well.
% \changes{v2.0f}{89/07/20}{Unnecessary `global' removed.}
%    \begin{macrocode}
%<*package>
\newskip\theorempreskipamount
\newskip\theorempostskipamount
%    \end{macrocode}
%    Since we have used the same values for all theorem sets, we now can
%    assign them.
%    \begin{macrocode}
\global\setlength\theorempreskipamount{12pt plus 5pt minus 3pt}
\global\setlength\theorempostskipamount{8pt plus 3pt minus 1.5pt}
%    \end{macrocode}
% \end{macro}
% \end{macro}
%
%
% \begin{macro}{\@endtheorem}
%    The same holds for the macro |\@endtheorem|, which ends a
%    {\envfont theorem} environment. Since it is the same for all
%    theorem sets, it is removed from the macros |\th@|\meta{style}.
%    It simply ends the {\envfont trivlist} environment, which was
%    begun in |\@thm|.
%    \begin{macrocode}
\global\let\@endtheorem=\endtrivlist
%    \end{macrocode}
% \end{macro}
%
%
%
% \begin{macro}{\@preamblecmds}
%    All macros defined above are to be used only in the preamble.
%    Therefore, we insert them in |\@preamblecmds| which will disable
%    them at begin document. This is done by the internal \LaTeXe{}
%    command |\@onlypreamble|.
% \changes{v2.0g}{89/09/19}{`theoremheaderfont and `theorembodyfont
%    added.}
% \changes{v2.2a}{94/02/03}{Use \LaTeXe{} interface}
%    \begin{macrocode}
\@onlypreamble\@xnthm
\@onlypreamble\@ynthm
\@onlypreamble\@othm
\@onlypreamble\newtheorem
\@onlypreamble\theoremstyle
\@onlypreamble\theorembodyfont
\@onlypreamble\theoremheaderfont
%    \end{macrocode}
% \end{macro}
%
% Finally we declare the \texttt{plain} theorem style to be the default.
% \changes{v2.1b}{90/06/04}{Fixed missing `item by declaring default
%    style.}
%    \begin{macrocode}
\theoremstyle{plain}
%</package>
%    \end{macrocode}
%
% \Finale

\endinput
}{}
\ProvidesFile{thc.sty}
             [\FMithmInfo]
%    \end{macrocode}
%
% \begin{macro}{\th@change}
%    The \texttt{change} theorem style corresponds to the
%    \texttt{change break} style without a linebreak after the header.
%    To say it in another way, it's the same as the \texttt{plain}
%    style but with number and name interchanged and |\slshape| as the
%    default font.
%    \begin{macrocode}
\gdef\th@change{\normalfont\slshape
  \def\@begintheorem##1##2{\item
       [\hskip\labelsep \theorem@headerfont ##2\ ##1]}%
\def\@opargbegintheorem##1##2##3{%
  \item[\hskip\labelsep \theorem@headerfont ##2\ ##1\ (##3)]}}
\endgroup
%</thc>
%    \end{macrocode}
% \end{macro}
%
%
% \subsubsection{The \texttt{marginbreak} style}
%
% This style option is the one used most often at Mainz.  It is saved
% in the file |thmb.sty|.
%    \begin{macrocode}
%<*thmb>
\begingroup \makeatletter
\@ifundefined{theorem@style}{% \iffalse meta-comment
%
% Copyright (C) 1993-2024
%
% The LaTeX Project and any individual authors listed elsewhere
% in this file.
%
% This file is part of the Standard LaTeX `Tools Bundle'.
% -------------------------------------------------------
%
% It may be distributed and/or modified under the
% conditions of the LaTeX Project Public License, either version 1.3c
% of this license or (at your option) any later version.
% The latest version of this license is in
%    https://www.latex-project.org/lppl.txt
% and version 1.3c or later is part of all distributions of LaTeX
% version 2005/12/01 or later.
%
% The list of all files belonging to the LaTeX `Tools Bundle' is
% given in the file `manifest.txt'.
%
% \fi
%
%%
%
%
\def\FMithmInfo{2023/07/05 v2.2c Theorem extension package (FMi)}
%
% \ProvidesFile{theorem.dtx}[\FMithmInfo]
%\iffalse   % this is a METACOMMENT !
%%
%% Package `theorem' to use with LaTeX2e
%% Copyright (C) 1989-1995 Frank Mittelbach, all rights reserved.
%<+package>\@ifundefined{theorem@style}{}{\endinput}
%<+package>\NeedsTeXFormat{LaTeX2e}[1995/06/01]
%<+package>\ProvidesPackage{theorem}
%<+package>         [\FMithmInfo]
%
% \fi
%
% \changes{v1.0}{87/05/07}{First draft.}
% \changes{v1.0a}{87/05/07}{`thistheoremstyle' added.}
% \changes{v1.1a}{87/07/29}{New concept: redefine the internal
%                          \protect\LaTeX{} commands `@xnthm' etc.}
% \changes{v1.1b}{87/07/31}{`theoremfont' used to specify font.}
% \changes{v1.2a}{88/01/08}{`@endtheorem' now globally defined.}
% \changes{v1.2b}{88/01/11}{`newtheorem' only allowed in the preamble.}
% \changes{v1.3a}{88/02/14}{`theoremfont' now a toks register.}
%
% \changes{v2.0a}{88/11/16}{New implementation and new documentation.}
% \changes{v2.0a}{88/11/16}{No layout changes but internal modifications
%                           old substyles won't work.}
% \changes{v2.0b}{89/01/30}{`global' used for all definitions.}
% \changes{v2.0c}{89/06/15}{Documentation translated to English.
%                            Thanks to LLL and Nina}
% \changes{v2.1a}{90/02/15}{Support for new font selection.}
% \changes{v2.2a}{94/02/03}{Cleanup for LaTeX2e}
%
% \DoNotIndex{\@endparenv,\@ifdefinable,\@ifnextchar,\@ifundefined}
% \DoNotIndex{\@namedef,\@nameuse,\@spaces,\@tempa,\@warning}
% \DoNotIndex{\ }
% \DoNotIndex{\begingroup,\bfseries}
% \DoNotIndex{\csname}
% \DoNotIndex{\def,\do,\docdate}
% \DoNotIndex{\endcsname,\endgroup,\endinput,\endtrivlist,\expandafter}
% \DoNotIndex{\filedate,\fileversion}
% \DoNotIndex{\gdef,\global}
% \DoNotIndex{\hbox,\hskip}
% \DoNotIndex{\ignorespaces,\input,\itshape,\item}
% \DoNotIndex{\labelsep,\let,\llap}
% \DoNotIndex{\makeatletter}
% \DoNotIndex{\noexpand}
% \DoNotIndex{\refstepcounter,\relax,\rlap}
% \DoNotIndex{\setlength,\slshape,\space,\string,\strut}
% \DoNotIndex{\the,\trivlist,\typeout}
% \DoNotIndex{\vbox}
% \DoNotIndex{\xdef}
%
% \def\deflabel#1{\ttfamily #1\hfill}
% \def\deflist#1{\begin{list}{}{\settowidth\labelwidth{\ttfamily #1}%
%                               \setlength\leftmargin\labelwidth
%                               \addtolength\leftmargin\labelsep
%                               \let\makelabel\deflabel}}
% \def\enddeflist{\end{list}}
% \def\envfont{\normalfont\ttfamily}
%
% \MakeShortVerb{\|}
%
% \newcommand{\TUB}{{\sl TUGboat\/}}
%
% \GetFileInfo{theorem.dtx}
% \title{An Extension of the
%        \LaTeX{} theorem environment\thanks{This file
%        has version number \fileversion, last
%        revised \filedate.}}
% \author{Frank Mittelbach}
%
% \MaintainedByLaTeXTeam{tools}
% \maketitle
%
% \begin{abstract}
%    The macros described in this paper yield an extension of the
%    \LaTeX{} theorem mechanism. It is designed to satisfy the
%    different requirements of various journals. Thus, the
%    layout of the ``theorems'' can be manipulated by determining a
%    ``style''. This article describes not only the use, but
%    also the definition, of the necessary macros.
% \end{abstract}
%
% \section*{Preface to version 2.2}
%
% For \LaTeXe{} this package did not need any fundamental changes. I
% only modified the messages generated so that theorem layout styles
% will show up with the |\listfiles| command and cleaned the section
% on the New Font Selection Scheme since this is now included in
% \LaTeX.
%
% \section*{Preface to version 2.1}
%
% This version is identical to 2.0g described in \TUB\ 10\#3 except
% for some internal defaults which are now set depending on the used
% font selection scheme.
%
% This was done to avoid unpleasant surprises if the new font selection
% scheme is in force. For further details see
% section~\ref{sec:newfonts} and \cite{bk:GMS94}.
%
% \section{Introduction}
%
% For our purposes here, ``theorems'' are labelled enunciations,
% often set off from the main text by extra space and a font change.
% Theorems, corollaries, conjectures, definitions, and remarks
% are all instances of ``theorems''.  The ``header'' of these structures
% is composed of a label (such as \textsc{Theorem} or \textsc{Remark})
% and a number which serializes an item in the sequence of items
% with the same label.
%
% Shortly after the introduction of \LaTeX{} at the Fachbereich
% Mathematik in Mainz, the desire to manipulate the layout of
% ``theorems'' arose.  In Mainz, the following two conventions came
% into general use:
% \begin{enumerate}
%   \item The number of the theorem is shown in the margin.
%   \item There is a line break at
%         the end of the theorem header.
% \end{enumerate}
%
% Additionally, some journals require different formats which depend on
% the ``sort of theorem'': e.g.\ often remarks and definitions are set
% in |\upshape|, while |\itshape| is employed for main theorems.
%
% Confronted with these requirements, a theorem environment was
% developed in Mainz which allows separate determination of the
% layout of the ``theorems sets'', comparable to |\pagestyle|.
%
%
% \section{The user interface}
% \subsection{Defining new theorem sets}
%
% \DescribeMacro\newtheorem
% As in the original \LaTeX{} version, the command |\newtheorem|
% defines a new ``theorem set'' or ``theorem-like structure''.
% Two required arguments name the new environment and give the text
% to be typeset with each instance of the new ``set'', while
% an optional argument determines how
% the ``set'' is enumerated:
% \begin{description}
%    \item[\ttfamily \bslash newtheorem\{foo\}\{bar\}]
%       The theorem set {\envfont foo} (whose name is \texttt{bar})
%       uses its own counter.
%    \item[\ttfamily \bslash newtheorem\{foo2\}{[foo]}\{bar2\}]
%       The theorem set {\envfont foo2} (printed name \texttt{bar2})
%       uses the same counter as the theorem set \texttt{foo}.
%    \item[\ttfamily \bslash newtheorem\{foo3\}\{bar3\}{[section]}]
%       The theorem set {\envfont foo3} (printed name \texttt{bar3}) is
%       enumerated within the counter \texttt{section}, i.e.\ with every
%       new |\section| the enumeration begins again with 1, and
%       the enumeration is composed from the section-number and the
%       theorem counter itself.
% \end{description}
%
% \DescribeMacro\theoremstyle
% Additionally, the command |\theoremstyle| can define the layout
% of various, or all, theorem sets.  It should be noted that any
% theorem set defined by |\newtheorem| is typeset in the
% |\theoremstyle| that is current at the time of the definition.
% Thus, the following
% \begin{verbatim}
%    \theoremstyle{break}        \newtheorem{Cor}{Corollary}
%    \theoremstyle{plain}        \newtheorem{Exa}{Example}[section]
%\end{verbatim}                             ^^A No blank at line start
% leads to the result that the set {\envfont Cor} is formatted in the
% style \texttt{break}, while the set {\envfont Exa} and all the
% following ones are formatted in the style \texttt{plain}, unless
% another |\theoremstyle| follows.  Since the definitions installed by
% |\newtheorem| are global, one also can limit |\theoremstyle| locally
% by grouping braces.
%
% \DescribeMacro\theorembodyfont
% The choice of the font for the theorem body is completely
% independent of the chosen |\theoremstyle|; this has proven to
% be very advantageous.  For example,
% \begin{verbatim}
%   {\theorembodyfont{\upshape}          \newtheorem{Rem}{Remark}}
%\end{verbatim}
% defines a theorem set {\envfont Rem}, which will be set in
% |\upshape| in the current layout (which in our example is
% \texttt{plain}).  As with |\theoremstyle|, the |\theorembodyfont|
% chosen is that current at the time of |\newtheorem|.  If
% |\theorembodyfont| is not specified or one defines
% |\theorembodyfont{}|, then the font used will be that defined by the
% |\theoremstyle|.
%
% \DescribeMacro\theoremheaderfont It is also possible to customize
% the font used for the theorem headers.  This is, however, a global
% declaration, and therefore there should be at most one
% |\theoremheaderfont| declaration in the preamble.\footnote{If it is
% actually necessary to have different header fonts, one has to define
% new theorem styles (substituting the desired font) or specify the
% information directly in the \texttt{\bslash newtheorem} declaration
% (the unclean variant).}
%
% \DescribeMacro\theorempreskipamount
% \DescribeMacro\theorempostskipamount Two additional parameters
% affect the vertical space around the theorem environments:
% |\theorempreskipamount| and |\theorempostskipamount| define,
% respectively, the spacing before and after such an environment.
% These parameters apply for all theorem sets and can be manipulated
% with the ordinary length macros.  They are rubber lengths
% (`\textsf{skips}'), and therefore can contain \texttt{plus} and
% \texttt{minus} parts.
%
% Since the definition of theorem sets should---most sensibly---be
% placed in the preamble, we only allow installation there.
% It is therefore possible to release the memory used here after
% |\begin{document}|, in order to make room for other
% applications.
%
%
% \subsection{Existing theorem styles}
%
% The following theorem styles exist to date:
% \begin{deflist}{marginbreak}
%    \item[plain]
%       This theorem style emulates the original \LaTeX{} definition,
%       except that additionally the parameters
%       |\theorem...skipamount| are used.
%    \item[break]
%       In this style, the theorem header is followed by a line
%       break.
%    \item[marginbreak]
%       The theorem number is set in the margin, and there is a line
%       break as in \texttt{break}.
%    \item[changebreak]
%       Like \texttt{break}, but with header number and text
%       interchanged.
%    \item[change]
%       Header number and text are interchanged, without a line break.
%    \item[margin]
%       The number is set in the left margin, without a line break.
% \end{deflist}
% All styles (except \texttt{plain}) select |\slshape| as the default
% |\theorembodyfont|.
%
%
% \subsection{Examples}
%
% Given the above theorem sets {\envfont Cor}, {\envfont Exa}
% and {\envfont Rem},
% suppose that the preamble also contains the declarations:
% \begin{verbatim}
%    \theoremstyle{marginbreak}   \newtheorem{Lem}[Cor]{Lemma}
%    \theoremstyle{change}
%    \theorembodyfont{\itshape}        \newtheorem{Def}[Cor]{Definition}
%
%    \theoremheaderfont{\scshape}
%\end{verbatim}
% Then the following are some typical examples of the typeset output
% resulting from their use.
%
% \begin{Cor}
%    This is a sentence typeset in the theorem environment {\envfont
%    Cor}.
% \end{Cor}
%
% \begin{Exa}
%    This is a sentence typeset in the theorem environment {\envfont
%    Exa}.
% \end{Exa}
%
% \begin{Rem}
%    This is a sentence typeset in the theorem environment {\envfont
%    Rem}.
% \end{Rem}
%
% \begin{Lem}[Ben User]
%    This is a sentence typeset in the theorem environment {\envfont
%    Lem}.
% \end{Lem}
%
% \begin{Def}[Very impressive Definition]
%    This is a sentence typeset in the theorem environment {\envfont
%    Def}.
% \end{Def}
%
% The last two examples show the effect of the optional argument to
% a theorem environment (it is the text typeset in parentheses).
%
%
% \section{Special Considerations}
% \label{sec:newfonts}
%
% Theoremheader and body are implemented as a unit. This means that
% the |\theoremheaderfont|  will inherit characteristics of the
% |\theorembodyfont| in \LaTeXe. Thus, if for example
% |\theorembodyfont| is |\itshape| and |\theoremheaderfont|
% is |\bfseries| the font selected for the header will have the
% characteristics `bold extended italic'. If this is not desired one
% should set the |\theoremheaderfont| to something like
% \begin{verbatim}
%      \theoremheaderfont{\normalfont\bfseries}
%\end{verbatim}
% i.e.\ supplying all necessary font information explicitly.
%
%
%
% \section{Acknowledgements}
%
% The publication of this set of macros was only possible with the
% help of Christina Busse (translating the manuscript into English),
% Joachim Pense (playing the r\^ole of typist), Chris Rowley
% (looking everything over) and many others providing useful
% suggestions.
%
% \MaybeStop{%
%    \begin{thebibliography}{9}
%       \bibitem{bk:GMS94} \textsc{M.~Goossens}, \textsc{F.~Mittelbach}
%         and \textsc{A.~Samarin}.
%         \newblock The \LaTeX{} Companion.
%         \newblock
%         Addison-Wesley, Reading, Massachusetts, 1994.
%       \bibitem{latex:source} \textsc{Lamport, Leslie.} \newblock
%                               \texttt{latex.tex}, version 2.09, date
%                               Feb. 1990.
%    \end{thebibliography}
%    \PrintIndex
%    \PrintChanges}
%
%
% \section{The documentation driver file}
%
% The next bit of code contains the documentation driver file for
% \TeX{}, i.e., the file that will produce the documentation you are
% currently reading. It will be extracted from this file by the
% \texttt{docstrip} program. Since it is the first code in the file
% one can alternatively process this file directly with \LaTeXe{} to
% obtain the documentation.
%
%    \begin{macrocode}
%<*driver>
\documentclass{ltxdoc}

\usepackage{theorem}

%<+driver>% The next few lines define theorem sets which are used
%<+driver>% in the example section of the documentation.

\theoremstyle{break}       \newtheorem{Cor}{Corollary}
\theoremstyle{plain}       \newtheorem{Exa}{Example}[section]
{\theorembodyfont{\upshape}\newtheorem{Rem}{Remark}}
\theoremstyle{marginbreak} \newtheorem{Lem}[Cor]{Lemma}
\theoremstyle{change}
\theorembodyfont{\itshape} \newtheorem{Def}[Cor]{Definition}

\theoremheaderfont{\scshape}

\RecordChanges

\begin{document}
   \DocInput{theorem.dtx}
\end{document}
%</driver>
%    \end{macrocode}
%
% \section{Definition of the Macros}
%
% If the file has been loaded before, we abort immediately. If not the
% package announces itself (this is actually done at the very top of
% the file---the way it is done isn't good style so don't copy it).
% \changes{v2.0e}{89/07/19}{Spaces removed from `typeout'.}
% \changes{v2.2b}{95/11/19}{Announce moved to top of file}
%    \begin{macrocode}
%<*package>
%\@ifundefined{theorem@style}{}{\endinput}
%\def\FMithmInfo{1995/11/19 v2.2b Theorem extension package (FMi)}
%\ProvidesPackage{theorem}[\FMithmInfo]
%    \end{macrocode}
%
%
% \subsection{Definition of theorem styles and fonts}
%
% All the definitions in this file are done globally to allow
% inputting this file inside a group.
%
% \begin{macro}{\theoremstyle}
%    Before a theorem style can be installed, the chosen style must be
%    known. For that reason, we must test to see that
%    |\th@|\meta{style} is known or, more precisely, that it is
%    different from |\relax|. If the style is not known then
%    |\th@plain| is used.
%    \begin{macrocode}
\gdef\theoremstyle#1{%
   \@ifundefined{th@#1}{\@warning
          {Unknown theoremstyle `#1'. Using `plain'}%
          \theorem@style{plain}}%
%    \end{macrocode}
%    We save the theorem style to be used in the token register
%    |\theorem@style|.
%    \begin{macrocode}
      {\theorem@style{#1}}%
%    \end{macrocode}
%    Now we ``evaluate'' the theorem style: this means, we call the
%    macro |\th@|\meta{style} which will activate the relevant
%    definitions which are contained in a separate file.  This is done
%    in a group to suppress changes to the current font.  This could
%    otherwise pose problems together with the new font selection
%    scheme\footnote{When I printed the original article using the new
%    font selection scheme I ended with a document with slanted
%    typefaces (text headings and all) simply because one of the
%    theorem styles used \texttt{\bslash sl} at toplevel.} if the
%    |\th@|\meta{style} is evaluated a second time.
% \changes{v2.1a}{90/02/15}{Group added around `th@..}
% \changes{v2.1a}{90/02/15}{Reseting of `..@begintheorem removed.}
%    \begin{macrocode}
      \begingroup
        \csname th@\the\theorem@style \endcsname
      \endgroup}
%    \end{macrocode}
% \end{macro}
%
%
% \begin{macro}{\@begintheorem}
% \changes{v2.1a}{90/02/15}{Relaxed at toplevel.}
% \begin{macro}{\@opargbegintheorem}
% \changes{v2.1a}{90/02/15}{Relaxed at toplevel.}
%    We reset |\@begintheorem| and |\@opargbegintheorem| to
%    |\relax| since these commands are no longer necessary at
%    toplevel. This will save a few tokens.
%    \begin{macrocode}
\global\let\@begintheorem\relax
\global\let\@opargbegintheorem\relax
%    \end{macrocode}
% \end{macro}
% \end{macro}
%
% \begin{macro}{\theorem@style}
%    Obviously the \textsf{token} register used above has to be
%    allocated. To assure the utmost compatibility with the
%    original \LaTeX{} definition, we set the default theorem style to
%    \texttt{plain}, which implements the usual \LaTeX{} convention.
% \changes{v2.0f}{89/07/20}{Unnecessary `global' removed.}
%    \begin{macrocode}
\newtoks\theorem@style
\global\theorem@style{plain}
%    \end{macrocode}
% \end{macro}
%
% \begin{macro}{\theorembodyfont}
% \begin{macro}{\theorem@bodyfont}
%    If the |\theorembodyfont| is set by the user then it should not
%    interact with the default font set in the theorem style. When the
%    new font selection is in force this may happen if, for example,
%    the default is |\itshape| and the new |\theorembodyfont| is
%    |\sffamily|. So we add a |\reset@font| command in front of the
%    user definition.
% \changes{v2.0f}{89/07/20}{Unnecessary `global' removed.}
% \changes{v2.0g}{89/09/19}{`theoremfont to `theorembodyfont.}
% \changes{v2.1c}{92/06/22}{`reset@font added and logic changed.}
% \changes{v2.1d}{92/10/28}{Reset to empty if arg is empty}
%    \begin{macrocode}
\gdef\theorembodyfont#1{%
%    \end{macrocode}
%    We check if the argument supplied is empty and if so put nothing
%    into the |\theorem@bodyfont| token register to allow for
%    |\theorembodyfont{}| as a mean of using the default of the
%    current |\theoremstyle|.
%    \begin{macrocode}
   \def\@tempa{#1}%
   \ifx\@tempa\@empty
    \theorem@bodyfont{}%
   \else
    \theorem@bodyfont{\reset@font#1}%
   \fi
   }
\newtoks\theorem@bodyfont
\global\theorem@bodyfont{}
%    \end{macrocode}
% \end{macro}
% \end{macro}
%
%
% \begin{macro}{\theoremheaderfont}
% \changes{v2.0g}{89/09/19}{Macro added.}
%    The font for the theorem headers is handled differently because
%    this definition applies to all theorem styles.
%    \begin{macrocode}
\gdef\theoremheaderfont#1{\gdef\theorem@headerfont{#1}%
%    \end{macrocode}
%    After using the macro once it is redefined to produce an error
%    message.
%    \begin{macrocode}
       \gdef\theoremheaderfont##1{%
        \typeout{\string\theoremheaderfont\space should be used
                 only once.}}}
%    \end{macrocode}
% \end{macro}
%
% \begin{macro}{\theorem@headerfont}
% \changes{v2.0g}{89/09/19}{Macro added.}
%    To set the |\theorem@headerfont| default we first test if
%    the new fontselection scheme is in force.
% \changes{v2.1a}{90/02/15}{Support for new font selection.}
%    \begin{macrocode}
\ifx\upshape\undefined
%    \end{macrocode}
%    If not we define it to expand into |\bfseries|. We don't use
%    |\let| just in case a following style option redefines this
%    macro.
%    \begin{macrocode}
\gdef\theorem@headerfont{\bfseries}
%    \end{macrocode}
%    Otherwise we reset the current shape before calling |\bfseries|.
%    \begin{macrocode}
\else \gdef\theorem@headerfont{\normalfont\bfseries}\fi
%    \end{macrocode}
% \end{macro}
%
% \begin{macro}{\th@plain}
% \begin{macro}{\th@break}
% \begin{macro}{\th@marginbreak}
% \begin{macro}{\th@changebreak}
% \begin{macro}{\th@change}
% \begin{macro}{\th@margin}
%    The different styles are defined in macros such as
%    |\th@plain|.  Since memory space is precious in
%    ``non-Big-versions'', we have to avoid offering too many unused
%    definitions.  Therefore we define these styles in separate files
%    that can be loaded on demand.  Thus the commands themselves only
%    load these files. We use |\@input@|, a \LaTeXe{} internal command
%    that ensures that the file will be listed with |\listfiles|.
% \changes{v2.2a}{94/02/02}{Use `@input@ to load theorem layout files}
%    \begin{macrocode}
\gdef\th@plain{\@input@{thp.sty}}
\gdef\th@break{\@input@{thb.sty}}
\gdef\th@marginbreak{\@input@{thmb.sty}}
\gdef\th@changebreak{\@input@{thcb.sty}}
\gdef\th@change{\@input@{thc.sty}}
\gdef\th@margin{\@input@{thm.sty}}
%    \end{macrocode}
%    This list will be expanded when new styles become
%    available.  For testing,  just append new theorem substyles
%    as document options.
% \end{macro}
% \end{macro}
% \end{macro}
% \end{macro}
% \end{macro}
% \end{macro}
%
%
%
% \subsection{Definition of a new theorem set}
%
% As already pointed out, a new theorem environment can be defined in
% three different ways:
% \begin{verbatim}
%    \newtheorem{Lem}{Lemma}
%    \newtheorem{Lem}{Lemma}[section]
%    \newtheorem{Lem}[Theorem]{Lemma}
%\end{verbatim}
% The function of the macro |\newtheorem| is to recognize these cases
% and then to branch into one of the three macros |\@ynthm|, |\@xnthm|
% or |\@othm|.  This mechanism is adopted unchanged from
% \cite{latex:source}; the essential point here is that, for example,
% in the second case, the arguments \texttt{Lem}, \texttt{Lemma} and
% \texttt{section} are passed over to the macro |\@xnthm|.
%
% We inspect this case first because the others present fewer problems,
% and thus are easily derived from this one.
%
% \begin{macro}{\@xnthm}
%    For our example arguments, the macro |\@xnthm| must fulfill
%    the following:
%    \begin{itemize}
%       \item
%          Define a new \LaTeX-counter `Lem'
%       \item
%          reset this counter within a |\section|
%       \item
%          define the macro |\theLem|
%       \item
%          define the environment macros |\Lem| and
%          |\endLem| using the current
%          |\theoremstyle| and |\theorem@bodyfont|.
%    \end{itemize}
%    Obviously, all this should happen only if the first argument
%    of |\@xnthm| (i.e.\ \texttt{Lem} in our example) is chosen so
%    as not to conflict with any previously defined commands or
%    environments.  This test is performed by the \LaTeX{} macro
%    |\@ifdefinable|.
%    \begin{macrocode}
\gdef\@xnthm#1#2[#3]{\expandafter\@ifdefinable\csname #1\endcsname
%    \end{macrocode}
%    Therefore, the first argument of |\@ifdefinable| is the
%    expansion (in the example, |\Lem|) of
%    |\csname||#1\endcsname|.  The second argument is
%    executed only if the test has been completed successfully.
%    \begin{macrocode}
   {%
%    \end{macrocode}
%    Now we define the new counter. The names of the \LaTeX{}
%    macros employed should speak for themselves:
% \changes{v2.2b}{1995/10/07}{Use \cs{@newctr}}
% \changes{v2.2c}{1995/11/23}{Use \cs{@newctr} properly.}
%    \begin{macrocode}
    \@definecounter{#1}\@newctr{#1}[#3]%
%    \end{macrocode}
%    Using |\@newctr| will give a proper error message if the counter
%    in |#3| is not defined.
%    In defining `|\theLem|' we must generate the desired macro
%    name by use of |\expandafter| and |\csname|.
%    \begin{macrocode}
    \expandafter\xdef\csname the#1\endcsname
%    \end{macrocode}
%    An |\xdef| is used in order to make the definition global,
%    and to ensure that it contains the replacement texts of
%    |\@thmcountersep| and |\@thmcounter|.\footnote{These
%    two macros can be defined by the document style.  Their default
%    values produce a `\texttt{.}' as separation and an arabic
%    representation of the number.} However, not everything should be
%    expanded. For example, it saves space to use |\thesection|
%    instead of its---at times---lengthy expansion.
%    \begin{macrocode}
      {\expandafter \noexpand \csname the#3\endcsname
       \@thmcountersep \@thmcounter{#1}}%
%    \end{macrocode}
%    Thus with the defaults of \LaTeX{}, |\theLem| would be
%    replaced by the command sequence |\thesection.\arabic{Lem}|.
%
%    We will now look at the definition of the macro which is executed
%    at the beginning of the actual environment (in our example this
%    macro is |\Lem|). It should be noted that we use an
%    ``|\expandafter| trick'' to expand only certain parts of the
%    replacement text at the time of the definition.
%    \begin{macrocode}
    \def\@tempa{\global\@namedef{#1}}%
    \expandafter \@tempa \expandafter{%
%    \end{macrocode}
%    First, the macro that contains the current definitions of
%    |\@begintheorem| and |\@opargtheorem| should be called
%    up.  The name of this macro---as is already known---has the form
%    |\th@|\meta{theorem\ style}; therefore, it must be called by
%    \begin{macrocode}
      \csname th@\the \theorem@style
%    \end{macrocode}
%    In addition the default theorem font should be changeable, i.e.\
%    we have to insert the contents of |\theorem@bodyfont|. For
%    that reason, we expand even further, beyond |\endcsname|,
%    and thus insert the contents of the \textsf{token} register
%    |\theorem@bodyfont| in the replacement text.
%    \begin{macrocode}
            \expandafter \endcsname \the \theorem@bodyfont
%    \end{macrocode}
%    Now it is time to call the macro |\@thm| which takes over
%    the further processing.  It has two arguments: the current
%    counter name (in our example, \texttt{Lem}), and the text of the
%    label (in our example, \texttt{Lemma}).
%    \begin{macrocode}
     \@thm{#1}{#2}}%
%    \end{macrocode}
%    With this, the `sub-definition' is complete.  The macro
%    |\@endtheorem| ends a theorem environment and is, so far, nothing
%    but an |\endtrivlist|.  (Hence it is defined globally, and not
%    within the theorem styles.\footnote{This has to be changed as
%    soon as theorem styles that change \texttt{\bslash @endtheorem}
%    exist.  In such a case, all existing styles must be changed as
%    well since they will have to reset the macro.}) Therefore, we can
%    set it equivalent to the macro that ends the theorem set (in our
%    example, |\endLem|).  However, if some day theorem styles exist
%    that do change |\@endtheorem|, we would have to use the
%    commented-out line instead.
%    \begin{macrocode}
    \global \expandafter \let \csname end#1\endcsname \@endtheorem
%   \global\@namedef{end#1}{\@endtheorem}%
%    \end{macrocode}
%    With these commands all the required definitions are employed,
%    unless the test |\@ifdefinable| has failed.  Therefore, we
%    end the second argument of this macro and with it the definition
%    of |\@xnthm|.
%    \begin{macrocode}
   }}
%    \end{macrocode}
% \end{macro}
%
% \begin{macro}{\@ynthm}
%    The definition of |\@ynthm| is completely analogous. In
%    this case the new counter that is defined is not reset within
%    another counter; thus the definition of |\the...|
%    is simplified:
%    \begin{macrocode}
\gdef\@ynthm#1#2{\expandafter\@ifdefinable\csname #1\endcsname
   {\@definecounter{#1}%
    \expandafter\xdef\csname the#1\endcsname{\@thmcounter{#1}}%
%    \end{macrocode}
%    The rest of the definition corresponds literally to that of
%    |\@xnthm|:
%    \begin{macrocode}
    \def\@tempa{\global\@namedef{#1}}\expandafter \@tempa
     \expandafter{\csname th@\the \theorem@style \expandafter
     \endcsname \the\theorem@bodyfont \@thm{#1}{#2}}%
    \global \expandafter \let \csname end#1\endcsname \@endtheorem}}
%    \end{macrocode}
% \end{macro}
%
% \begin{macro}{\@othm}
%    The definition of |\@othm| does not contain anything new.
%    \begin{macrocode}
\gdef\@othm#1[#2]#3{%
%    \end{macrocode}
% \changes{v2.2b}{95/10/07}{Build in check for counter defined}
%    We do not define a new counter but instead use one that has
%    already been defined.  Thus the only definition we need is that
%    of this pseudo-counter (i.e.\ |\the|\meta{env.\ name}).
%    First we check if |#2| corresponds to a known counter name.
%    \begin{macrocode}
  \expandafter\ifx\csname c@#2\endcsname\relax
   \@nocounterr{#2}%
  \else
   \expandafter\@ifdefinable\csname #1\endcsname
   {\expandafter \xdef \csname the#1\endcsname
     {\expandafter \noexpand \csname the#2\endcsname}%
%    \end{macrocode}
%    All other parts of the definition can be adopted from
%    |\@xnthm|. We have to remember, though, that in this case
%    the name of the current counter and the theorem label have moved
%    to the second and third arguments.
%    \begin{macrocode}
    \def\@tempa{\global\@namedef{#1}}\expandafter \@tempa
     \expandafter{\csname th@\the \theorem@style \expandafter
     \endcsname \the\theorem@bodyfont \@thm{#2}{#3}}%
    \global \expandafter \let \csname end#1\endcsname \@endtheorem}%
  \fi}
%    \end{macrocode}
% \end{macro}
%
% \subsection{Macros that are employed in a {\envfont theorem}
%    environment}
%
% \begin{macro}{\@thm}
%    The macro |\@thm| has to increase the current counter.  Then,
%    depending on whether the environment has (or does not have) an
%    optional argument, it has to branch into either |\@begintheorem|
%    or |\@opargtheorem|.
%    \begin{macrocode}
\gdef\@thm#1#2{\refstepcounter{#1}%
%    \end{macrocode}
%    Now we start a {\envfont trivlist} environment, and give
%    |\@topsep| and |\@topsepadd| the values of the \textsf{skip}
%    registers |\theorempreskipamount| and |\theorempostskipamount|.
%    The value in |\@topsep| is the vertical space that is inserted by
%    the first (and only) |\item| in our |\trivlist| whilst
%    |\@topsepadd| is inserted by |\@endparenv| at the end of that
%    {\envfont trivlist} environment.  By using these registers, we
%    obtain the desired space around a {\envfont theorem} environment.
%    \begin{macrocode}
   \trivlist
   \@topsep \theorempreskipamount               % used by first \item
   \@topsepadd \theorempostskipamount           % used by \@endparenv
%    \end{macrocode}
%    Now we have to test whether an optional argument has been given.
% \changes{v2.0d}{89/07/18}{Forgotten percent added.}
%    \begin{macrocode}
   \@ifnextchar [%
%    \end{macrocode}
%    If there is an optional argument, we will call |\@ythm|, and
%    move the arguments read back into the input stream.
%    \begin{macrocode}
   {\@ythm{#1}{#2}}%
%    \end{macrocode}
%    If not, we call |\@begintheorem|. Its first argument is the
%    name of the theorem set (hence the second argument of
%    |\@thm|). Its second argument is the macro that produces the
%    current number.
%    \begin{macrocode}
   {\@begintheorem{#2}{\csname the#1\endcsname}\ignorespaces}}
%    \end{macrocode}
% \end{macro}
%
% \begin{macro}{\@xthm}
% \begin{macro}{\@ythm}
%    Both these macros were originally called by |\@thm|.  We do
%    not need |\@xthm| anymore, hence we reset it to
%    |\relax|. The definition of |\@ythm| has not changed at
%    all from its definition in \LaTeX. In order to make the macros
%    easier to understand, we will
%    nevertheless present it (commented out).
%    \begin{macrocode}
\global\let\@xthm\relax
% \def\@ythm#1#2[#3]{\@opargbegintheorem{#2}{\csname
%           the#1\endcsname}{#3}\ignorespaces}
%</package>
%    \end{macrocode}
%    The primitive |\ignorespaces| in |\@ythm| and
%    |\@thm| is needed to remove the spaces between the
%    |\begin{...}| and the actual text.
% \end{macro}
% \end{macro}
%
%
% \subsection{Definition of the theorem substyles}
%
% As already pointed out, the theorem substyles, defined below, are
% only loaded when necessary.  Note that all these substyles, except
% \texttt{plain}, have |\slshape| as the default body font.
%
% \subsubsection{The \texttt{plain} style}
%
% \label{dok:th@plain} As the following macros use |@|, we have
% to locally set the |\catcode| of this symbol to ``letter''.
% This happens within a group, so that we do not have to worry about
% which |\catcode| that symbol had before.
%    \begin{macrocode}
%<*thp>
\begingroup \makeatletter
%    \end{macrocode}
% Since we are now within a group, we must make all definitions
% globally.  First we make sure that \texttt{theorem.sty} is loaded.
% This will allow us to use this file as a document style option
% without having to call \texttt{theorem} itself as an option.  At the
% same time, we assure that at least version 2 is loaded, since
% |\theorem@style| was not defined in earlier versions.
%    \begin{macrocode}
\@ifundefined{theorem@style}{% \iffalse meta-comment
%
% Copyright (C) 1993-2024
%
% The LaTeX Project and any individual authors listed elsewhere
% in this file.
%
% This file is part of the Standard LaTeX `Tools Bundle'.
% -------------------------------------------------------
%
% It may be distributed and/or modified under the
% conditions of the LaTeX Project Public License, either version 1.3c
% of this license or (at your option) any later version.
% The latest version of this license is in
%    https://www.latex-project.org/lppl.txt
% and version 1.3c or later is part of all distributions of LaTeX
% version 2005/12/01 or later.
%
% The list of all files belonging to the LaTeX `Tools Bundle' is
% given in the file `manifest.txt'.
%
% \fi
%
%%
%
%
\def\FMithmInfo{2023/07/05 v2.2c Theorem extension package (FMi)}
%
% \ProvidesFile{theorem.dtx}[\FMithmInfo]
%\iffalse   % this is a METACOMMENT !
%%
%% Package `theorem' to use with LaTeX2e
%% Copyright (C) 1989-1995 Frank Mittelbach, all rights reserved.
%<+package>\@ifundefined{theorem@style}{}{\endinput}
%<+package>\NeedsTeXFormat{LaTeX2e}[1995/06/01]
%<+package>\ProvidesPackage{theorem}
%<+package>         [\FMithmInfo]
%
% \fi
%
% \changes{v1.0}{87/05/07}{First draft.}
% \changes{v1.0a}{87/05/07}{`thistheoremstyle' added.}
% \changes{v1.1a}{87/07/29}{New concept: redefine the internal
%                          \protect\LaTeX{} commands `@xnthm' etc.}
% \changes{v1.1b}{87/07/31}{`theoremfont' used to specify font.}
% \changes{v1.2a}{88/01/08}{`@endtheorem' now globally defined.}
% \changes{v1.2b}{88/01/11}{`newtheorem' only allowed in the preamble.}
% \changes{v1.3a}{88/02/14}{`theoremfont' now a toks register.}
%
% \changes{v2.0a}{88/11/16}{New implementation and new documentation.}
% \changes{v2.0a}{88/11/16}{No layout changes but internal modifications
%                           old substyles won't work.}
% \changes{v2.0b}{89/01/30}{`global' used for all definitions.}
% \changes{v2.0c}{89/06/15}{Documentation translated to English.
%                            Thanks to LLL and Nina}
% \changes{v2.1a}{90/02/15}{Support for new font selection.}
% \changes{v2.2a}{94/02/03}{Cleanup for LaTeX2e}
%
% \DoNotIndex{\@endparenv,\@ifdefinable,\@ifnextchar,\@ifundefined}
% \DoNotIndex{\@namedef,\@nameuse,\@spaces,\@tempa,\@warning}
% \DoNotIndex{\ }
% \DoNotIndex{\begingroup,\bfseries}
% \DoNotIndex{\csname}
% \DoNotIndex{\def,\do,\docdate}
% \DoNotIndex{\endcsname,\endgroup,\endinput,\endtrivlist,\expandafter}
% \DoNotIndex{\filedate,\fileversion}
% \DoNotIndex{\gdef,\global}
% \DoNotIndex{\hbox,\hskip}
% \DoNotIndex{\ignorespaces,\input,\itshape,\item}
% \DoNotIndex{\labelsep,\let,\llap}
% \DoNotIndex{\makeatletter}
% \DoNotIndex{\noexpand}
% \DoNotIndex{\refstepcounter,\relax,\rlap}
% \DoNotIndex{\setlength,\slshape,\space,\string,\strut}
% \DoNotIndex{\the,\trivlist,\typeout}
% \DoNotIndex{\vbox}
% \DoNotIndex{\xdef}
%
% \def\deflabel#1{\ttfamily #1\hfill}
% \def\deflist#1{\begin{list}{}{\settowidth\labelwidth{\ttfamily #1}%
%                               \setlength\leftmargin\labelwidth
%                               \addtolength\leftmargin\labelsep
%                               \let\makelabel\deflabel}}
% \def\enddeflist{\end{list}}
% \def\envfont{\normalfont\ttfamily}
%
% \MakeShortVerb{\|}
%
% \newcommand{\TUB}{{\sl TUGboat\/}}
%
% \GetFileInfo{theorem.dtx}
% \title{An Extension of the
%        \LaTeX{} theorem environment\thanks{This file
%        has version number \fileversion, last
%        revised \filedate.}}
% \author{Frank Mittelbach}
%
% \MaintainedByLaTeXTeam{tools}
% \maketitle
%
% \begin{abstract}
%    The macros described in this paper yield an extension of the
%    \LaTeX{} theorem mechanism. It is designed to satisfy the
%    different requirements of various journals. Thus, the
%    layout of the ``theorems'' can be manipulated by determining a
%    ``style''. This article describes not only the use, but
%    also the definition, of the necessary macros.
% \end{abstract}
%
% \section*{Preface to version 2.2}
%
% For \LaTeXe{} this package did not need any fundamental changes. I
% only modified the messages generated so that theorem layout styles
% will show up with the |\listfiles| command and cleaned the section
% on the New Font Selection Scheme since this is now included in
% \LaTeX.
%
% \section*{Preface to version 2.1}
%
% This version is identical to 2.0g described in \TUB\ 10\#3 except
% for some internal defaults which are now set depending on the used
% font selection scheme.
%
% This was done to avoid unpleasant surprises if the new font selection
% scheme is in force. For further details see
% section~\ref{sec:newfonts} and \cite{bk:GMS94}.
%
% \section{Introduction}
%
% For our purposes here, ``theorems'' are labelled enunciations,
% often set off from the main text by extra space and a font change.
% Theorems, corollaries, conjectures, definitions, and remarks
% are all instances of ``theorems''.  The ``header'' of these structures
% is composed of a label (such as \textsc{Theorem} or \textsc{Remark})
% and a number which serializes an item in the sequence of items
% with the same label.
%
% Shortly after the introduction of \LaTeX{} at the Fachbereich
% Mathematik in Mainz, the desire to manipulate the layout of
% ``theorems'' arose.  In Mainz, the following two conventions came
% into general use:
% \begin{enumerate}
%   \item The number of the theorem is shown in the margin.
%   \item There is a line break at
%         the end of the theorem header.
% \end{enumerate}
%
% Additionally, some journals require different formats which depend on
% the ``sort of theorem'': e.g.\ often remarks and definitions are set
% in |\upshape|, while |\itshape| is employed for main theorems.
%
% Confronted with these requirements, a theorem environment was
% developed in Mainz which allows separate determination of the
% layout of the ``theorems sets'', comparable to |\pagestyle|.
%
%
% \section{The user interface}
% \subsection{Defining new theorem sets}
%
% \DescribeMacro\newtheorem
% As in the original \LaTeX{} version, the command |\newtheorem|
% defines a new ``theorem set'' or ``theorem-like structure''.
% Two required arguments name the new environment and give the text
% to be typeset with each instance of the new ``set'', while
% an optional argument determines how
% the ``set'' is enumerated:
% \begin{description}
%    \item[\ttfamily \bslash newtheorem\{foo\}\{bar\}]
%       The theorem set {\envfont foo} (whose name is \texttt{bar})
%       uses its own counter.
%    \item[\ttfamily \bslash newtheorem\{foo2\}{[foo]}\{bar2\}]
%       The theorem set {\envfont foo2} (printed name \texttt{bar2})
%       uses the same counter as the theorem set \texttt{foo}.
%    \item[\ttfamily \bslash newtheorem\{foo3\}\{bar3\}{[section]}]
%       The theorem set {\envfont foo3} (printed name \texttt{bar3}) is
%       enumerated within the counter \texttt{section}, i.e.\ with every
%       new |\section| the enumeration begins again with 1, and
%       the enumeration is composed from the section-number and the
%       theorem counter itself.
% \end{description}
%
% \DescribeMacro\theoremstyle
% Additionally, the command |\theoremstyle| can define the layout
% of various, or all, theorem sets.  It should be noted that any
% theorem set defined by |\newtheorem| is typeset in the
% |\theoremstyle| that is current at the time of the definition.
% Thus, the following
% \begin{verbatim}
%    \theoremstyle{break}        \newtheorem{Cor}{Corollary}
%    \theoremstyle{plain}        \newtheorem{Exa}{Example}[section]
%\end{verbatim}                             ^^A No blank at line start
% leads to the result that the set {\envfont Cor} is formatted in the
% style \texttt{break}, while the set {\envfont Exa} and all the
% following ones are formatted in the style \texttt{plain}, unless
% another |\theoremstyle| follows.  Since the definitions installed by
% |\newtheorem| are global, one also can limit |\theoremstyle| locally
% by grouping braces.
%
% \DescribeMacro\theorembodyfont
% The choice of the font for the theorem body is completely
% independent of the chosen |\theoremstyle|; this has proven to
% be very advantageous.  For example,
% \begin{verbatim}
%   {\theorembodyfont{\upshape}          \newtheorem{Rem}{Remark}}
%\end{verbatim}
% defines a theorem set {\envfont Rem}, which will be set in
% |\upshape| in the current layout (which in our example is
% \texttt{plain}).  As with |\theoremstyle|, the |\theorembodyfont|
% chosen is that current at the time of |\newtheorem|.  If
% |\theorembodyfont| is not specified or one defines
% |\theorembodyfont{}|, then the font used will be that defined by the
% |\theoremstyle|.
%
% \DescribeMacro\theoremheaderfont It is also possible to customize
% the font used for the theorem headers.  This is, however, a global
% declaration, and therefore there should be at most one
% |\theoremheaderfont| declaration in the preamble.\footnote{If it is
% actually necessary to have different header fonts, one has to define
% new theorem styles (substituting the desired font) or specify the
% information directly in the \texttt{\bslash newtheorem} declaration
% (the unclean variant).}
%
% \DescribeMacro\theorempreskipamount
% \DescribeMacro\theorempostskipamount Two additional parameters
% affect the vertical space around the theorem environments:
% |\theorempreskipamount| and |\theorempostskipamount| define,
% respectively, the spacing before and after such an environment.
% These parameters apply for all theorem sets and can be manipulated
% with the ordinary length macros.  They are rubber lengths
% (`\textsf{skips}'), and therefore can contain \texttt{plus} and
% \texttt{minus} parts.
%
% Since the definition of theorem sets should---most sensibly---be
% placed in the preamble, we only allow installation there.
% It is therefore possible to release the memory used here after
% |\begin{document}|, in order to make room for other
% applications.
%
%
% \subsection{Existing theorem styles}
%
% The following theorem styles exist to date:
% \begin{deflist}{marginbreak}
%    \item[plain]
%       This theorem style emulates the original \LaTeX{} definition,
%       except that additionally the parameters
%       |\theorem...skipamount| are used.
%    \item[break]
%       In this style, the theorem header is followed by a line
%       break.
%    \item[marginbreak]
%       The theorem number is set in the margin, and there is a line
%       break as in \texttt{break}.
%    \item[changebreak]
%       Like \texttt{break}, but with header number and text
%       interchanged.
%    \item[change]
%       Header number and text are interchanged, without a line break.
%    \item[margin]
%       The number is set in the left margin, without a line break.
% \end{deflist}
% All styles (except \texttt{plain}) select |\slshape| as the default
% |\theorembodyfont|.
%
%
% \subsection{Examples}
%
% Given the above theorem sets {\envfont Cor}, {\envfont Exa}
% and {\envfont Rem},
% suppose that the preamble also contains the declarations:
% \begin{verbatim}
%    \theoremstyle{marginbreak}   \newtheorem{Lem}[Cor]{Lemma}
%    \theoremstyle{change}
%    \theorembodyfont{\itshape}        \newtheorem{Def}[Cor]{Definition}
%
%    \theoremheaderfont{\scshape}
%\end{verbatim}
% Then the following are some typical examples of the typeset output
% resulting from their use.
%
% \begin{Cor}
%    This is a sentence typeset in the theorem environment {\envfont
%    Cor}.
% \end{Cor}
%
% \begin{Exa}
%    This is a sentence typeset in the theorem environment {\envfont
%    Exa}.
% \end{Exa}
%
% \begin{Rem}
%    This is a sentence typeset in the theorem environment {\envfont
%    Rem}.
% \end{Rem}
%
% \begin{Lem}[Ben User]
%    This is a sentence typeset in the theorem environment {\envfont
%    Lem}.
% \end{Lem}
%
% \begin{Def}[Very impressive Definition]
%    This is a sentence typeset in the theorem environment {\envfont
%    Def}.
% \end{Def}
%
% The last two examples show the effect of the optional argument to
% a theorem environment (it is the text typeset in parentheses).
%
%
% \section{Special Considerations}
% \label{sec:newfonts}
%
% Theoremheader and body are implemented as a unit. This means that
% the |\theoremheaderfont|  will inherit characteristics of the
% |\theorembodyfont| in \LaTeXe. Thus, if for example
% |\theorembodyfont| is |\itshape| and |\theoremheaderfont|
% is |\bfseries| the font selected for the header will have the
% characteristics `bold extended italic'. If this is not desired one
% should set the |\theoremheaderfont| to something like
% \begin{verbatim}
%      \theoremheaderfont{\normalfont\bfseries}
%\end{verbatim}
% i.e.\ supplying all necessary font information explicitly.
%
%
%
% \section{Acknowledgements}
%
% The publication of this set of macros was only possible with the
% help of Christina Busse (translating the manuscript into English),
% Joachim Pense (playing the r\^ole of typist), Chris Rowley
% (looking everything over) and many others providing useful
% suggestions.
%
% \MaybeStop{%
%    \begin{thebibliography}{9}
%       \bibitem{bk:GMS94} \textsc{M.~Goossens}, \textsc{F.~Mittelbach}
%         and \textsc{A.~Samarin}.
%         \newblock The \LaTeX{} Companion.
%         \newblock
%         Addison-Wesley, Reading, Massachusetts, 1994.
%       \bibitem{latex:source} \textsc{Lamport, Leslie.} \newblock
%                               \texttt{latex.tex}, version 2.09, date
%                               Feb. 1990.
%    \end{thebibliography}
%    \PrintIndex
%    \PrintChanges}
%
%
% \section{The documentation driver file}
%
% The next bit of code contains the documentation driver file for
% \TeX{}, i.e., the file that will produce the documentation you are
% currently reading. It will be extracted from this file by the
% \texttt{docstrip} program. Since it is the first code in the file
% one can alternatively process this file directly with \LaTeXe{} to
% obtain the documentation.
%
%    \begin{macrocode}
%<*driver>
\documentclass{ltxdoc}

\usepackage{theorem}

%<+driver>% The next few lines define theorem sets which are used
%<+driver>% in the example section of the documentation.

\theoremstyle{break}       \newtheorem{Cor}{Corollary}
\theoremstyle{plain}       \newtheorem{Exa}{Example}[section]
{\theorembodyfont{\upshape}\newtheorem{Rem}{Remark}}
\theoremstyle{marginbreak} \newtheorem{Lem}[Cor]{Lemma}
\theoremstyle{change}
\theorembodyfont{\itshape} \newtheorem{Def}[Cor]{Definition}

\theoremheaderfont{\scshape}

\RecordChanges

\begin{document}
   \DocInput{theorem.dtx}
\end{document}
%</driver>
%    \end{macrocode}
%
% \section{Definition of the Macros}
%
% If the file has been loaded before, we abort immediately. If not the
% package announces itself (this is actually done at the very top of
% the file---the way it is done isn't good style so don't copy it).
% \changes{v2.0e}{89/07/19}{Spaces removed from `typeout'.}
% \changes{v2.2b}{95/11/19}{Announce moved to top of file}
%    \begin{macrocode}
%<*package>
%\@ifundefined{theorem@style}{}{\endinput}
%\def\FMithmInfo{1995/11/19 v2.2b Theorem extension package (FMi)}
%\ProvidesPackage{theorem}[\FMithmInfo]
%    \end{macrocode}
%
%
% \subsection{Definition of theorem styles and fonts}
%
% All the definitions in this file are done globally to allow
% inputting this file inside a group.
%
% \begin{macro}{\theoremstyle}
%    Before a theorem style can be installed, the chosen style must be
%    known. For that reason, we must test to see that
%    |\th@|\meta{style} is known or, more precisely, that it is
%    different from |\relax|. If the style is not known then
%    |\th@plain| is used.
%    \begin{macrocode}
\gdef\theoremstyle#1{%
   \@ifundefined{th@#1}{\@warning
          {Unknown theoremstyle `#1'. Using `plain'}%
          \theorem@style{plain}}%
%    \end{macrocode}
%    We save the theorem style to be used in the token register
%    |\theorem@style|.
%    \begin{macrocode}
      {\theorem@style{#1}}%
%    \end{macrocode}
%    Now we ``evaluate'' the theorem style: this means, we call the
%    macro |\th@|\meta{style} which will activate the relevant
%    definitions which are contained in a separate file.  This is done
%    in a group to suppress changes to the current font.  This could
%    otherwise pose problems together with the new font selection
%    scheme\footnote{When I printed the original article using the new
%    font selection scheme I ended with a document with slanted
%    typefaces (text headings and all) simply because one of the
%    theorem styles used \texttt{\bslash sl} at toplevel.} if the
%    |\th@|\meta{style} is evaluated a second time.
% \changes{v2.1a}{90/02/15}{Group added around `th@..}
% \changes{v2.1a}{90/02/15}{Reseting of `..@begintheorem removed.}
%    \begin{macrocode}
      \begingroup
        \csname th@\the\theorem@style \endcsname
      \endgroup}
%    \end{macrocode}
% \end{macro}
%
%
% \begin{macro}{\@begintheorem}
% \changes{v2.1a}{90/02/15}{Relaxed at toplevel.}
% \begin{macro}{\@opargbegintheorem}
% \changes{v2.1a}{90/02/15}{Relaxed at toplevel.}
%    We reset |\@begintheorem| and |\@opargbegintheorem| to
%    |\relax| since these commands are no longer necessary at
%    toplevel. This will save a few tokens.
%    \begin{macrocode}
\global\let\@begintheorem\relax
\global\let\@opargbegintheorem\relax
%    \end{macrocode}
% \end{macro}
% \end{macro}
%
% \begin{macro}{\theorem@style}
%    Obviously the \textsf{token} register used above has to be
%    allocated. To assure the utmost compatibility with the
%    original \LaTeX{} definition, we set the default theorem style to
%    \texttt{plain}, which implements the usual \LaTeX{} convention.
% \changes{v2.0f}{89/07/20}{Unnecessary `global' removed.}
%    \begin{macrocode}
\newtoks\theorem@style
\global\theorem@style{plain}
%    \end{macrocode}
% \end{macro}
%
% \begin{macro}{\theorembodyfont}
% \begin{macro}{\theorem@bodyfont}
%    If the |\theorembodyfont| is set by the user then it should not
%    interact with the default font set in the theorem style. When the
%    new font selection is in force this may happen if, for example,
%    the default is |\itshape| and the new |\theorembodyfont| is
%    |\sffamily|. So we add a |\reset@font| command in front of the
%    user definition.
% \changes{v2.0f}{89/07/20}{Unnecessary `global' removed.}
% \changes{v2.0g}{89/09/19}{`theoremfont to `theorembodyfont.}
% \changes{v2.1c}{92/06/22}{`reset@font added and logic changed.}
% \changes{v2.1d}{92/10/28}{Reset to empty if arg is empty}
%    \begin{macrocode}
\gdef\theorembodyfont#1{%
%    \end{macrocode}
%    We check if the argument supplied is empty and if so put nothing
%    into the |\theorem@bodyfont| token register to allow for
%    |\theorembodyfont{}| as a mean of using the default of the
%    current |\theoremstyle|.
%    \begin{macrocode}
   \def\@tempa{#1}%
   \ifx\@tempa\@empty
    \theorem@bodyfont{}%
   \else
    \theorem@bodyfont{\reset@font#1}%
   \fi
   }
\newtoks\theorem@bodyfont
\global\theorem@bodyfont{}
%    \end{macrocode}
% \end{macro}
% \end{macro}
%
%
% \begin{macro}{\theoremheaderfont}
% \changes{v2.0g}{89/09/19}{Macro added.}
%    The font for the theorem headers is handled differently because
%    this definition applies to all theorem styles.
%    \begin{macrocode}
\gdef\theoremheaderfont#1{\gdef\theorem@headerfont{#1}%
%    \end{macrocode}
%    After using the macro once it is redefined to produce an error
%    message.
%    \begin{macrocode}
       \gdef\theoremheaderfont##1{%
        \typeout{\string\theoremheaderfont\space should be used
                 only once.}}}
%    \end{macrocode}
% \end{macro}
%
% \begin{macro}{\theorem@headerfont}
% \changes{v2.0g}{89/09/19}{Macro added.}
%    To set the |\theorem@headerfont| default we first test if
%    the new fontselection scheme is in force.
% \changes{v2.1a}{90/02/15}{Support for new font selection.}
%    \begin{macrocode}
\ifx\upshape\undefined
%    \end{macrocode}
%    If not we define it to expand into |\bfseries|. We don't use
%    |\let| just in case a following style option redefines this
%    macro.
%    \begin{macrocode}
\gdef\theorem@headerfont{\bfseries}
%    \end{macrocode}
%    Otherwise we reset the current shape before calling |\bfseries|.
%    \begin{macrocode}
\else \gdef\theorem@headerfont{\normalfont\bfseries}\fi
%    \end{macrocode}
% \end{macro}
%
% \begin{macro}{\th@plain}
% \begin{macro}{\th@break}
% \begin{macro}{\th@marginbreak}
% \begin{macro}{\th@changebreak}
% \begin{macro}{\th@change}
% \begin{macro}{\th@margin}
%    The different styles are defined in macros such as
%    |\th@plain|.  Since memory space is precious in
%    ``non-Big-versions'', we have to avoid offering too many unused
%    definitions.  Therefore we define these styles in separate files
%    that can be loaded on demand.  Thus the commands themselves only
%    load these files. We use |\@input@|, a \LaTeXe{} internal command
%    that ensures that the file will be listed with |\listfiles|.
% \changes{v2.2a}{94/02/02}{Use `@input@ to load theorem layout files}
%    \begin{macrocode}
\gdef\th@plain{\@input@{thp.sty}}
\gdef\th@break{\@input@{thb.sty}}
\gdef\th@marginbreak{\@input@{thmb.sty}}
\gdef\th@changebreak{\@input@{thcb.sty}}
\gdef\th@change{\@input@{thc.sty}}
\gdef\th@margin{\@input@{thm.sty}}
%    \end{macrocode}
%    This list will be expanded when new styles become
%    available.  For testing,  just append new theorem substyles
%    as document options.
% \end{macro}
% \end{macro}
% \end{macro}
% \end{macro}
% \end{macro}
% \end{macro}
%
%
%
% \subsection{Definition of a new theorem set}
%
% As already pointed out, a new theorem environment can be defined in
% three different ways:
% \begin{verbatim}
%    \newtheorem{Lem}{Lemma}
%    \newtheorem{Lem}{Lemma}[section]
%    \newtheorem{Lem}[Theorem]{Lemma}
%\end{verbatim}
% The function of the macro |\newtheorem| is to recognize these cases
% and then to branch into one of the three macros |\@ynthm|, |\@xnthm|
% or |\@othm|.  This mechanism is adopted unchanged from
% \cite{latex:source}; the essential point here is that, for example,
% in the second case, the arguments \texttt{Lem}, \texttt{Lemma} and
% \texttt{section} are passed over to the macro |\@xnthm|.
%
% We inspect this case first because the others present fewer problems,
% and thus are easily derived from this one.
%
% \begin{macro}{\@xnthm}
%    For our example arguments, the macro |\@xnthm| must fulfill
%    the following:
%    \begin{itemize}
%       \item
%          Define a new \LaTeX-counter `Lem'
%       \item
%          reset this counter within a |\section|
%       \item
%          define the macro |\theLem|
%       \item
%          define the environment macros |\Lem| and
%          |\endLem| using the current
%          |\theoremstyle| and |\theorem@bodyfont|.
%    \end{itemize}
%    Obviously, all this should happen only if the first argument
%    of |\@xnthm| (i.e.\ \texttt{Lem} in our example) is chosen so
%    as not to conflict with any previously defined commands or
%    environments.  This test is performed by the \LaTeX{} macro
%    |\@ifdefinable|.
%    \begin{macrocode}
\gdef\@xnthm#1#2[#3]{\expandafter\@ifdefinable\csname #1\endcsname
%    \end{macrocode}
%    Therefore, the first argument of |\@ifdefinable| is the
%    expansion (in the example, |\Lem|) of
%    |\csname||#1\endcsname|.  The second argument is
%    executed only if the test has been completed successfully.
%    \begin{macrocode}
   {%
%    \end{macrocode}
%    Now we define the new counter. The names of the \LaTeX{}
%    macros employed should speak for themselves:
% \changes{v2.2b}{1995/10/07}{Use \cs{@newctr}}
% \changes{v2.2c}{1995/11/23}{Use \cs{@newctr} properly.}
%    \begin{macrocode}
    \@definecounter{#1}\@newctr{#1}[#3]%
%    \end{macrocode}
%    Using |\@newctr| will give a proper error message if the counter
%    in |#3| is not defined.
%    In defining `|\theLem|' we must generate the desired macro
%    name by use of |\expandafter| and |\csname|.
%    \begin{macrocode}
    \expandafter\xdef\csname the#1\endcsname
%    \end{macrocode}
%    An |\xdef| is used in order to make the definition global,
%    and to ensure that it contains the replacement texts of
%    |\@thmcountersep| and |\@thmcounter|.\footnote{These
%    two macros can be defined by the document style.  Their default
%    values produce a `\texttt{.}' as separation and an arabic
%    representation of the number.} However, not everything should be
%    expanded. For example, it saves space to use |\thesection|
%    instead of its---at times---lengthy expansion.
%    \begin{macrocode}
      {\expandafter \noexpand \csname the#3\endcsname
       \@thmcountersep \@thmcounter{#1}}%
%    \end{macrocode}
%    Thus with the defaults of \LaTeX{}, |\theLem| would be
%    replaced by the command sequence |\thesection.\arabic{Lem}|.
%
%    We will now look at the definition of the macro which is executed
%    at the beginning of the actual environment (in our example this
%    macro is |\Lem|). It should be noted that we use an
%    ``|\expandafter| trick'' to expand only certain parts of the
%    replacement text at the time of the definition.
%    \begin{macrocode}
    \def\@tempa{\global\@namedef{#1}}%
    \expandafter \@tempa \expandafter{%
%    \end{macrocode}
%    First, the macro that contains the current definitions of
%    |\@begintheorem| and |\@opargtheorem| should be called
%    up.  The name of this macro---as is already known---has the form
%    |\th@|\meta{theorem\ style}; therefore, it must be called by
%    \begin{macrocode}
      \csname th@\the \theorem@style
%    \end{macrocode}
%    In addition the default theorem font should be changeable, i.e.\
%    we have to insert the contents of |\theorem@bodyfont|. For
%    that reason, we expand even further, beyond |\endcsname|,
%    and thus insert the contents of the \textsf{token} register
%    |\theorem@bodyfont| in the replacement text.
%    \begin{macrocode}
            \expandafter \endcsname \the \theorem@bodyfont
%    \end{macrocode}
%    Now it is time to call the macro |\@thm| which takes over
%    the further processing.  It has two arguments: the current
%    counter name (in our example, \texttt{Lem}), and the text of the
%    label (in our example, \texttt{Lemma}).
%    \begin{macrocode}
     \@thm{#1}{#2}}%
%    \end{macrocode}
%    With this, the `sub-definition' is complete.  The macro
%    |\@endtheorem| ends a theorem environment and is, so far, nothing
%    but an |\endtrivlist|.  (Hence it is defined globally, and not
%    within the theorem styles.\footnote{This has to be changed as
%    soon as theorem styles that change \texttt{\bslash @endtheorem}
%    exist.  In such a case, all existing styles must be changed as
%    well since they will have to reset the macro.}) Therefore, we can
%    set it equivalent to the macro that ends the theorem set (in our
%    example, |\endLem|).  However, if some day theorem styles exist
%    that do change |\@endtheorem|, we would have to use the
%    commented-out line instead.
%    \begin{macrocode}
    \global \expandafter \let \csname end#1\endcsname \@endtheorem
%   \global\@namedef{end#1}{\@endtheorem}%
%    \end{macrocode}
%    With these commands all the required definitions are employed,
%    unless the test |\@ifdefinable| has failed.  Therefore, we
%    end the second argument of this macro and with it the definition
%    of |\@xnthm|.
%    \begin{macrocode}
   }}
%    \end{macrocode}
% \end{macro}
%
% \begin{macro}{\@ynthm}
%    The definition of |\@ynthm| is completely analogous. In
%    this case the new counter that is defined is not reset within
%    another counter; thus the definition of |\the...|
%    is simplified:
%    \begin{macrocode}
\gdef\@ynthm#1#2{\expandafter\@ifdefinable\csname #1\endcsname
   {\@definecounter{#1}%
    \expandafter\xdef\csname the#1\endcsname{\@thmcounter{#1}}%
%    \end{macrocode}
%    The rest of the definition corresponds literally to that of
%    |\@xnthm|:
%    \begin{macrocode}
    \def\@tempa{\global\@namedef{#1}}\expandafter \@tempa
     \expandafter{\csname th@\the \theorem@style \expandafter
     \endcsname \the\theorem@bodyfont \@thm{#1}{#2}}%
    \global \expandafter \let \csname end#1\endcsname \@endtheorem}}
%    \end{macrocode}
% \end{macro}
%
% \begin{macro}{\@othm}
%    The definition of |\@othm| does not contain anything new.
%    \begin{macrocode}
\gdef\@othm#1[#2]#3{%
%    \end{macrocode}
% \changes{v2.2b}{95/10/07}{Build in check for counter defined}
%    We do not define a new counter but instead use one that has
%    already been defined.  Thus the only definition we need is that
%    of this pseudo-counter (i.e.\ |\the|\meta{env.\ name}).
%    First we check if |#2| corresponds to a known counter name.
%    \begin{macrocode}
  \expandafter\ifx\csname c@#2\endcsname\relax
   \@nocounterr{#2}%
  \else
   \expandafter\@ifdefinable\csname #1\endcsname
   {\expandafter \xdef \csname the#1\endcsname
     {\expandafter \noexpand \csname the#2\endcsname}%
%    \end{macrocode}
%    All other parts of the definition can be adopted from
%    |\@xnthm|. We have to remember, though, that in this case
%    the name of the current counter and the theorem label have moved
%    to the second and third arguments.
%    \begin{macrocode}
    \def\@tempa{\global\@namedef{#1}}\expandafter \@tempa
     \expandafter{\csname th@\the \theorem@style \expandafter
     \endcsname \the\theorem@bodyfont \@thm{#2}{#3}}%
    \global \expandafter \let \csname end#1\endcsname \@endtheorem}%
  \fi}
%    \end{macrocode}
% \end{macro}
%
% \subsection{Macros that are employed in a {\envfont theorem}
%    environment}
%
% \begin{macro}{\@thm}
%    The macro |\@thm| has to increase the current counter.  Then,
%    depending on whether the environment has (or does not have) an
%    optional argument, it has to branch into either |\@begintheorem|
%    or |\@opargtheorem|.
%    \begin{macrocode}
\gdef\@thm#1#2{\refstepcounter{#1}%
%    \end{macrocode}
%    Now we start a {\envfont trivlist} environment, and give
%    |\@topsep| and |\@topsepadd| the values of the \textsf{skip}
%    registers |\theorempreskipamount| and |\theorempostskipamount|.
%    The value in |\@topsep| is the vertical space that is inserted by
%    the first (and only) |\item| in our |\trivlist| whilst
%    |\@topsepadd| is inserted by |\@endparenv| at the end of that
%    {\envfont trivlist} environment.  By using these registers, we
%    obtain the desired space around a {\envfont theorem} environment.
%    \begin{macrocode}
   \trivlist
   \@topsep \theorempreskipamount               % used by first \item
   \@topsepadd \theorempostskipamount           % used by \@endparenv
%    \end{macrocode}
%    Now we have to test whether an optional argument has been given.
% \changes{v2.0d}{89/07/18}{Forgotten percent added.}
%    \begin{macrocode}
   \@ifnextchar [%
%    \end{macrocode}
%    If there is an optional argument, we will call |\@ythm|, and
%    move the arguments read back into the input stream.
%    \begin{macrocode}
   {\@ythm{#1}{#2}}%
%    \end{macrocode}
%    If not, we call |\@begintheorem|. Its first argument is the
%    name of the theorem set (hence the second argument of
%    |\@thm|). Its second argument is the macro that produces the
%    current number.
%    \begin{macrocode}
   {\@begintheorem{#2}{\csname the#1\endcsname}\ignorespaces}}
%    \end{macrocode}
% \end{macro}
%
% \begin{macro}{\@xthm}
% \begin{macro}{\@ythm}
%    Both these macros were originally called by |\@thm|.  We do
%    not need |\@xthm| anymore, hence we reset it to
%    |\relax|. The definition of |\@ythm| has not changed at
%    all from its definition in \LaTeX. In order to make the macros
%    easier to understand, we will
%    nevertheless present it (commented out).
%    \begin{macrocode}
\global\let\@xthm\relax
% \def\@ythm#1#2[#3]{\@opargbegintheorem{#2}{\csname
%           the#1\endcsname}{#3}\ignorespaces}
%</package>
%    \end{macrocode}
%    The primitive |\ignorespaces| in |\@ythm| and
%    |\@thm| is needed to remove the spaces between the
%    |\begin{...}| and the actual text.
% \end{macro}
% \end{macro}
%
%
% \subsection{Definition of the theorem substyles}
%
% As already pointed out, the theorem substyles, defined below, are
% only loaded when necessary.  Note that all these substyles, except
% \texttt{plain}, have |\slshape| as the default body font.
%
% \subsubsection{The \texttt{plain} style}
%
% \label{dok:th@plain} As the following macros use |@|, we have
% to locally set the |\catcode| of this symbol to ``letter''.
% This happens within a group, so that we do not have to worry about
% which |\catcode| that symbol had before.
%    \begin{macrocode}
%<*thp>
\begingroup \makeatletter
%    \end{macrocode}
% Since we are now within a group, we must make all definitions
% globally.  First we make sure that \texttt{theorem.sty} is loaded.
% This will allow us to use this file as a document style option
% without having to call \texttt{theorem} itself as an option.  At the
% same time, we assure that at least version 2 is loaded, since
% |\theorem@style| was not defined in earlier versions.
%    \begin{macrocode}
\@ifundefined{theorem@style}{\input{theorem.sty}}{}
\ProvidesFile{thp.sty}
             [\FMithmInfo]
%    \end{macrocode}
%
%
% \begin{macro}{\th@plain}
% \begin{macro}{\@begintheorem}
% \begin{macro}{\@opargbegintheorem}
%    |\theoremstyle{plain}| corresponds to the original
%    definition, except that the distances to the surrounding text are
%    determined by the parameters |\theorempreskipamount| and
%    |\theorempostskipamount|.
%    First we set the default body font.
%    \begin{macrocode}
\gdef\th@plain{\normalfont\itshape
%    \end{macrocode}
%    Then we define |\@begintheorem| and
%    |\@opargbegintheorem|. These two macros define how the
%    header of a theorem is typeset.  |\@opargbegintheorem| will
%    be called if a \textsf{theorem} environment with an optional
%    argument is encountered; otherwise, the header is constructed by
%    calling |\@begintheorem|.  If one of these macros is
%    executed, we are within a \textsf{trivlist} environment started by
%    |\@thm|.  So the theorem header is produced with an
%    |\item| command.
%
%    Instead of specifying the header font directly, all standard
%    theorem styles use the |\theorem@headerfont| macro to allow
%    customization.  The extra space (|\labelsep|) is necessary
%    because of problems in the \textsf{trivlist} environment.
%    \begin{macrocode}
  \def\@begintheorem##1##2{%
        \item[\hskip\labelsep \theorem@headerfont ##1\ ##2]}%
%    \end{macrocode}
%    The definition of |\@opargbegintheorem| is completely
%    analogous. The only difference is the fact that there exists a
%    third argument (which is the optional parameter of the
%    environment and contains additional information about the
%    theorem).  Customarily we enclose it in parentheses.
%    \begin{macrocode}
\def\@opargbegintheorem##1##2##3{%
   \item[\hskip\labelsep \theorem@headerfont ##1\ ##2\ (##3)]}}
%    \end{macrocode}
% \end{macro}
% \end{macro}
% \end{macro}
%
% \noindent
% We conclude with an |\endgroup| to restore the
% |\catcode| of |@|.
%    \begin{macrocode}
\endgroup
%</thp>
%    \end{macrocode}
%
%
% \subsubsection{The \texttt{break} style}
%
% This style option is stored in the file |thb.sty|.
% For the next two lines see the documentation for |\th@plain| on
% page~\pageref{dok:th@plain}.
%    \begin{macrocode}
%<*thb>
\begingroup \makeatletter
\@ifundefined{theorem@style}{\input{theorem.sty}}{}
\ProvidesFile{thb.sty}
             [\FMithmInfo]
%    \end{macrocode}
%
% \begin{macro}{\th@break}
%    |\theoremstyle{break}| produces a line break after the name
%    of the theorem.  The font is |\slshape|.  Hence, we define
%    |\th@break| as follows:
%    \begin{macrocode}
\gdef\th@break{\normalfont\slshape
  \def\@begintheorem##1##2{\item[%
%    \end{macrocode}
%    We run into the following problem: it is not possible to create
%    the header with |\item[|\meta{title}|]| and then start
%    a new line by, for example, |\mbox{}\\|. Such a definition
%    will fail whenever a list environment follows immediately.  With
%    the above construction, the |\mbox{}| causes the switch
%    |@inlabel| (cf.\ definition of |\list| and
%    |\trivlist| in \cite{latex:source}) to be set to \texttt{false}
%    and so the following list will insert additional vertical
%    space (|\topskip|). This is quite annoying. Therefore, we
%    create the line break within the |\item|. In order to ensure
%    that the text will begin at the proper position in the following
%    line, we simply pretend that the label does not take any
%    room.\footnote{This will lead to problems whenever very high
%    symbols occurring in the line tower into the
%    heading. So, something else has to be done here sometime.}
%    \begin{macrocode}
       \rlap{\vbox{\hbox{\hskip \labelsep\theorem@headerfont ##1\ ##2}%
                   \hbox{\strut}}}]}%
%    \end{macrocode}
%    Again, the definition of |\@opargbegintheorem| is
%    completely analogous.
%    \begin{macrocode}
\def\@opargbegintheorem##1##2##3{%
  \item[\rlap{\vbox{\hbox{\hskip \labelsep \theorem@headerfont
                     ##1\ ##2\ (##3)}%
                    \hbox{\strut}}}]}}
\endgroup
%</thb>
%    \end{macrocode}
% \end{macro}
%
%
% \subsubsection{The \texttt{changebreak} style}
%
%    \begin{macrocode}
% This style option is stored in the file |thcb.sty|.
%    \begin{macrocode}
%<*thcb>
\begingroup \makeatletter
\@ifundefined{theorem@style}{\input{theorem.sty}}{}
\ProvidesFile{thcb.sty}
             [\FMithmInfo]
%    \end{macrocode}
%
% \begin{macro}{\th@changebreak}
%    The \texttt{change-break} theorem style is like \texttt{break}
%    but with interchange of theorem name and theorem number.  Thus we
%    define |\th@changebreak| as follows:
%    \begin{macrocode}
\gdef\th@changebreak{\normalfont\slshape
  \def\@begintheorem##1##2{\item
       [\rlap{\vbox{\hbox{\hskip\labelsep \theorem@headerfont ##2\ ##1}%
                    \hbox{\strut}}}]}%
\def\@opargbegintheorem##1##2##3{%
  \item[\rlap{\vbox{\hbox{\hskip\labelsep \theorem@headerfont
                     ##2\ ##1\ (##3)}%
                    \hbox{\strut}}}]}}
\endgroup
%</thcb>
%    \end{macrocode}
% \end{macro}
%
%
% \subsubsection{The \texttt{change} style}
%
% This style option is stored in the file |thc.sty|.
%    \begin{macrocode}
%<*thc>
\begingroup \makeatletter
\@ifundefined{theorem@style}{\input{theorem.sty}}{}
\ProvidesFile{thc.sty}
             [\FMithmInfo]
%    \end{macrocode}
%
% \begin{macro}{\th@change}
%    The \texttt{change} theorem style corresponds to the
%    \texttt{change break} style without a linebreak after the header.
%    To say it in another way, it's the same as the \texttt{plain}
%    style but with number and name interchanged and |\slshape| as the
%    default font.
%    \begin{macrocode}
\gdef\th@change{\normalfont\slshape
  \def\@begintheorem##1##2{\item
       [\hskip\labelsep \theorem@headerfont ##2\ ##1]}%
\def\@opargbegintheorem##1##2##3{%
  \item[\hskip\labelsep \theorem@headerfont ##2\ ##1\ (##3)]}}
\endgroup
%</thc>
%    \end{macrocode}
% \end{macro}
%
%
% \subsubsection{The \texttt{marginbreak} style}
%
% This style option is the one used most often at Mainz.  It is saved
% in the file |thmb.sty|.
%    \begin{macrocode}
%<*thmb>
\begingroup \makeatletter
\@ifundefined{theorem@style}{\input{theorem.sty}}{}
\ProvidesFile{thmb.sty}
             [\FMithmInfo]
%    \end{macrocode}
%
% \begin{macro}{\th@marginbreak}
%    The \texttt{margin break} style is nearly the same as the
%    \texttt{change break} style. The only difference is the placement
%    of the theorem number.  We use |\llap| to place it in the left
%    margin.
%
%    In this style |\labelsep| denotes the separation between the
%    number and the text.
%    \begin{macrocode}
\gdef\th@marginbreak{\normalfont\slshape
  \def\@begintheorem##1##2{\item
       [\rlap{\vbox{\theorem@headerfont
                    \hbox{\llap{##2}\hskip\labelsep ##1}%
                    \hbox{\strut}}}]}%
\def\@opargbegintheorem##1##2##3{%
  \item[\rlap{\vbox{\theorem@headerfont
                    \hbox{\llap{##2}\hskip\labelsep ##1\ (##3)}%
                    \hbox{\strut}}}]}}
\endgroup
%</thmb>
%    \end{macrocode}
% \changes{v2.1c}{90/02/17}{`endgroup outside macrocode env.}
% \end{macro}
%
%
%
% \subsubsection{The \texttt{margin} style}
%
% This style option is stored in the file |thm.sty|.
%    \begin{macrocode}
%<*thm>
\begingroup \makeatletter
\@ifundefined{theorem@style}{\input{theorem.sty}}{}
\ProvidesFile{thm.sty}
             [\FMithmInfo]
%    \end{macrocode}
%
% \begin{macro}{\th@margin}
%    Again this is only a variant of the theorem styles described
%    above without any new ideas.
%    \begin{macrocode}
\gdef\th@margin{\normalfont\slshape
  \def\@begintheorem##1##2{\item
       [\theorem@headerfont \llap{##2}\hskip\labelsep ##1]}%
\def\@opargbegintheorem##1##2##3{%
  \item[\theorem@headerfont \llap{##2}\hskip\labelsep ##1\ (##3)]}}
\endgroup
%</thm>
%    \end{macrocode}
% \end{macro}
%
%
%
%
% \subsection{Final Definitions}
%
% \begin{macro}{\theorempreskipamount}
% \begin{macro}{\theorempostskipamount}
%    The \textsf{skip} parameters that regulate the vertical empty space
%    before and after the {\envfont theorem} environment have
%    to be allocated as well.
% \changes{v2.0f}{89/07/20}{Unnecessary `global' removed.}
%    \begin{macrocode}
%<*package>
\newskip\theorempreskipamount
\newskip\theorempostskipamount
%    \end{macrocode}
%    Since we have used the same values for all theorem sets, we now can
%    assign them.
%    \begin{macrocode}
\global\setlength\theorempreskipamount{12pt plus 5pt minus 3pt}
\global\setlength\theorempostskipamount{8pt plus 3pt minus 1.5pt}
%    \end{macrocode}
% \end{macro}
% \end{macro}
%
%
% \begin{macro}{\@endtheorem}
%    The same holds for the macro |\@endtheorem|, which ends a
%    {\envfont theorem} environment. Since it is the same for all
%    theorem sets, it is removed from the macros |\th@|\meta{style}.
%    It simply ends the {\envfont trivlist} environment, which was
%    begun in |\@thm|.
%    \begin{macrocode}
\global\let\@endtheorem=\endtrivlist
%    \end{macrocode}
% \end{macro}
%
%
%
% \begin{macro}{\@preamblecmds}
%    All macros defined above are to be used only in the preamble.
%    Therefore, we insert them in |\@preamblecmds| which will disable
%    them at begin document. This is done by the internal \LaTeXe{}
%    command |\@onlypreamble|.
% \changes{v2.0g}{89/09/19}{`theoremheaderfont and `theorembodyfont
%    added.}
% \changes{v2.2a}{94/02/03}{Use \LaTeXe{} interface}
%    \begin{macrocode}
\@onlypreamble\@xnthm
\@onlypreamble\@ynthm
\@onlypreamble\@othm
\@onlypreamble\newtheorem
\@onlypreamble\theoremstyle
\@onlypreamble\theorembodyfont
\@onlypreamble\theoremheaderfont
%    \end{macrocode}
% \end{macro}
%
% Finally we declare the \texttt{plain} theorem style to be the default.
% \changes{v2.1b}{90/06/04}{Fixed missing `item by declaring default
%    style.}
%    \begin{macrocode}
\theoremstyle{plain}
%</package>
%    \end{macrocode}
%
% \Finale

\endinput
}{}
\ProvidesFile{thp.sty}
             [\FMithmInfo]
%    \end{macrocode}
%
%
% \begin{macro}{\th@plain}
% \begin{macro}{\@begintheorem}
% \begin{macro}{\@opargbegintheorem}
%    |\theoremstyle{plain}| corresponds to the original
%    definition, except that the distances to the surrounding text are
%    determined by the parameters |\theorempreskipamount| and
%    |\theorempostskipamount|.
%    First we set the default body font.
%    \begin{macrocode}
\gdef\th@plain{\normalfont\itshape
%    \end{macrocode}
%    Then we define |\@begintheorem| and
%    |\@opargbegintheorem|. These two macros define how the
%    header of a theorem is typeset.  |\@opargbegintheorem| will
%    be called if a \textsf{theorem} environment with an optional
%    argument is encountered; otherwise, the header is constructed by
%    calling |\@begintheorem|.  If one of these macros is
%    executed, we are within a \textsf{trivlist} environment started by
%    |\@thm|.  So the theorem header is produced with an
%    |\item| command.
%
%    Instead of specifying the header font directly, all standard
%    theorem styles use the |\theorem@headerfont| macro to allow
%    customization.  The extra space (|\labelsep|) is necessary
%    because of problems in the \textsf{trivlist} environment.
%    \begin{macrocode}
  \def\@begintheorem##1##2{%
        \item[\hskip\labelsep \theorem@headerfont ##1\ ##2]}%
%    \end{macrocode}
%    The definition of |\@opargbegintheorem| is completely
%    analogous. The only difference is the fact that there exists a
%    third argument (which is the optional parameter of the
%    environment and contains additional information about the
%    theorem).  Customarily we enclose it in parentheses.
%    \begin{macrocode}
\def\@opargbegintheorem##1##2##3{%
   \item[\hskip\labelsep \theorem@headerfont ##1\ ##2\ (##3)]}}
%    \end{macrocode}
% \end{macro}
% \end{macro}
% \end{macro}
%
% \noindent
% We conclude with an |\endgroup| to restore the
% |\catcode| of |@|.
%    \begin{macrocode}
\endgroup
%</thp>
%    \end{macrocode}
%
%
% \subsubsection{The \texttt{break} style}
%
% This style option is stored in the file |thb.sty|.
% For the next two lines see the documentation for |\th@plain| on
% page~\pageref{dok:th@plain}.
%    \begin{macrocode}
%<*thb>
\begingroup \makeatletter
\@ifundefined{theorem@style}{% \iffalse meta-comment
%
% Copyright (C) 1993-2024
%
% The LaTeX Project and any individual authors listed elsewhere
% in this file.
%
% This file is part of the Standard LaTeX `Tools Bundle'.
% -------------------------------------------------------
%
% It may be distributed and/or modified under the
% conditions of the LaTeX Project Public License, either version 1.3c
% of this license or (at your option) any later version.
% The latest version of this license is in
%    https://www.latex-project.org/lppl.txt
% and version 1.3c or later is part of all distributions of LaTeX
% version 2005/12/01 or later.
%
% The list of all files belonging to the LaTeX `Tools Bundle' is
% given in the file `manifest.txt'.
%
% \fi
%
%%
%
%
\def\FMithmInfo{2023/07/05 v2.2c Theorem extension package (FMi)}
%
% \ProvidesFile{theorem.dtx}[\FMithmInfo]
%\iffalse   % this is a METACOMMENT !
%%
%% Package `theorem' to use with LaTeX2e
%% Copyright (C) 1989-1995 Frank Mittelbach, all rights reserved.
%<+package>\@ifundefined{theorem@style}{}{\endinput}
%<+package>\NeedsTeXFormat{LaTeX2e}[1995/06/01]
%<+package>\ProvidesPackage{theorem}
%<+package>         [\FMithmInfo]
%
% \fi
%
% \changes{v1.0}{87/05/07}{First draft.}
% \changes{v1.0a}{87/05/07}{`thistheoremstyle' added.}
% \changes{v1.1a}{87/07/29}{New concept: redefine the internal
%                          \protect\LaTeX{} commands `@xnthm' etc.}
% \changes{v1.1b}{87/07/31}{`theoremfont' used to specify font.}
% \changes{v1.2a}{88/01/08}{`@endtheorem' now globally defined.}
% \changes{v1.2b}{88/01/11}{`newtheorem' only allowed in the preamble.}
% \changes{v1.3a}{88/02/14}{`theoremfont' now a toks register.}
%
% \changes{v2.0a}{88/11/16}{New implementation and new documentation.}
% \changes{v2.0a}{88/11/16}{No layout changes but internal modifications
%                           old substyles won't work.}
% \changes{v2.0b}{89/01/30}{`global' used for all definitions.}
% \changes{v2.0c}{89/06/15}{Documentation translated to English.
%                            Thanks to LLL and Nina}
% \changes{v2.1a}{90/02/15}{Support for new font selection.}
% \changes{v2.2a}{94/02/03}{Cleanup for LaTeX2e}
%
% \DoNotIndex{\@endparenv,\@ifdefinable,\@ifnextchar,\@ifundefined}
% \DoNotIndex{\@namedef,\@nameuse,\@spaces,\@tempa,\@warning}
% \DoNotIndex{\ }
% \DoNotIndex{\begingroup,\bfseries}
% \DoNotIndex{\csname}
% \DoNotIndex{\def,\do,\docdate}
% \DoNotIndex{\endcsname,\endgroup,\endinput,\endtrivlist,\expandafter}
% \DoNotIndex{\filedate,\fileversion}
% \DoNotIndex{\gdef,\global}
% \DoNotIndex{\hbox,\hskip}
% \DoNotIndex{\ignorespaces,\input,\itshape,\item}
% \DoNotIndex{\labelsep,\let,\llap}
% \DoNotIndex{\makeatletter}
% \DoNotIndex{\noexpand}
% \DoNotIndex{\refstepcounter,\relax,\rlap}
% \DoNotIndex{\setlength,\slshape,\space,\string,\strut}
% \DoNotIndex{\the,\trivlist,\typeout}
% \DoNotIndex{\vbox}
% \DoNotIndex{\xdef}
%
% \def\deflabel#1{\ttfamily #1\hfill}
% \def\deflist#1{\begin{list}{}{\settowidth\labelwidth{\ttfamily #1}%
%                               \setlength\leftmargin\labelwidth
%                               \addtolength\leftmargin\labelsep
%                               \let\makelabel\deflabel}}
% \def\enddeflist{\end{list}}
% \def\envfont{\normalfont\ttfamily}
%
% \MakeShortVerb{\|}
%
% \newcommand{\TUB}{{\sl TUGboat\/}}
%
% \GetFileInfo{theorem.dtx}
% \title{An Extension of the
%        \LaTeX{} theorem environment\thanks{This file
%        has version number \fileversion, last
%        revised \filedate.}}
% \author{Frank Mittelbach}
%
% \MaintainedByLaTeXTeam{tools}
% \maketitle
%
% \begin{abstract}
%    The macros described in this paper yield an extension of the
%    \LaTeX{} theorem mechanism. It is designed to satisfy the
%    different requirements of various journals. Thus, the
%    layout of the ``theorems'' can be manipulated by determining a
%    ``style''. This article describes not only the use, but
%    also the definition, of the necessary macros.
% \end{abstract}
%
% \section*{Preface to version 2.2}
%
% For \LaTeXe{} this package did not need any fundamental changes. I
% only modified the messages generated so that theorem layout styles
% will show up with the |\listfiles| command and cleaned the section
% on the New Font Selection Scheme since this is now included in
% \LaTeX.
%
% \section*{Preface to version 2.1}
%
% This version is identical to 2.0g described in \TUB\ 10\#3 except
% for some internal defaults which are now set depending on the used
% font selection scheme.
%
% This was done to avoid unpleasant surprises if the new font selection
% scheme is in force. For further details see
% section~\ref{sec:newfonts} and \cite{bk:GMS94}.
%
% \section{Introduction}
%
% For our purposes here, ``theorems'' are labelled enunciations,
% often set off from the main text by extra space and a font change.
% Theorems, corollaries, conjectures, definitions, and remarks
% are all instances of ``theorems''.  The ``header'' of these structures
% is composed of a label (such as \textsc{Theorem} or \textsc{Remark})
% and a number which serializes an item in the sequence of items
% with the same label.
%
% Shortly after the introduction of \LaTeX{} at the Fachbereich
% Mathematik in Mainz, the desire to manipulate the layout of
% ``theorems'' arose.  In Mainz, the following two conventions came
% into general use:
% \begin{enumerate}
%   \item The number of the theorem is shown in the margin.
%   \item There is a line break at
%         the end of the theorem header.
% \end{enumerate}
%
% Additionally, some journals require different formats which depend on
% the ``sort of theorem'': e.g.\ often remarks and definitions are set
% in |\upshape|, while |\itshape| is employed for main theorems.
%
% Confronted with these requirements, a theorem environment was
% developed in Mainz which allows separate determination of the
% layout of the ``theorems sets'', comparable to |\pagestyle|.
%
%
% \section{The user interface}
% \subsection{Defining new theorem sets}
%
% \DescribeMacro\newtheorem
% As in the original \LaTeX{} version, the command |\newtheorem|
% defines a new ``theorem set'' or ``theorem-like structure''.
% Two required arguments name the new environment and give the text
% to be typeset with each instance of the new ``set'', while
% an optional argument determines how
% the ``set'' is enumerated:
% \begin{description}
%    \item[\ttfamily \bslash newtheorem\{foo\}\{bar\}]
%       The theorem set {\envfont foo} (whose name is \texttt{bar})
%       uses its own counter.
%    \item[\ttfamily \bslash newtheorem\{foo2\}{[foo]}\{bar2\}]
%       The theorem set {\envfont foo2} (printed name \texttt{bar2})
%       uses the same counter as the theorem set \texttt{foo}.
%    \item[\ttfamily \bslash newtheorem\{foo3\}\{bar3\}{[section]}]
%       The theorem set {\envfont foo3} (printed name \texttt{bar3}) is
%       enumerated within the counter \texttt{section}, i.e.\ with every
%       new |\section| the enumeration begins again with 1, and
%       the enumeration is composed from the section-number and the
%       theorem counter itself.
% \end{description}
%
% \DescribeMacro\theoremstyle
% Additionally, the command |\theoremstyle| can define the layout
% of various, or all, theorem sets.  It should be noted that any
% theorem set defined by |\newtheorem| is typeset in the
% |\theoremstyle| that is current at the time of the definition.
% Thus, the following
% \begin{verbatim}
%    \theoremstyle{break}        \newtheorem{Cor}{Corollary}
%    \theoremstyle{plain}        \newtheorem{Exa}{Example}[section]
%\end{verbatim}                             ^^A No blank at line start
% leads to the result that the set {\envfont Cor} is formatted in the
% style \texttt{break}, while the set {\envfont Exa} and all the
% following ones are formatted in the style \texttt{plain}, unless
% another |\theoremstyle| follows.  Since the definitions installed by
% |\newtheorem| are global, one also can limit |\theoremstyle| locally
% by grouping braces.
%
% \DescribeMacro\theorembodyfont
% The choice of the font for the theorem body is completely
% independent of the chosen |\theoremstyle|; this has proven to
% be very advantageous.  For example,
% \begin{verbatim}
%   {\theorembodyfont{\upshape}          \newtheorem{Rem}{Remark}}
%\end{verbatim}
% defines a theorem set {\envfont Rem}, which will be set in
% |\upshape| in the current layout (which in our example is
% \texttt{plain}).  As with |\theoremstyle|, the |\theorembodyfont|
% chosen is that current at the time of |\newtheorem|.  If
% |\theorembodyfont| is not specified or one defines
% |\theorembodyfont{}|, then the font used will be that defined by the
% |\theoremstyle|.
%
% \DescribeMacro\theoremheaderfont It is also possible to customize
% the font used for the theorem headers.  This is, however, a global
% declaration, and therefore there should be at most one
% |\theoremheaderfont| declaration in the preamble.\footnote{If it is
% actually necessary to have different header fonts, one has to define
% new theorem styles (substituting the desired font) or specify the
% information directly in the \texttt{\bslash newtheorem} declaration
% (the unclean variant).}
%
% \DescribeMacro\theorempreskipamount
% \DescribeMacro\theorempostskipamount Two additional parameters
% affect the vertical space around the theorem environments:
% |\theorempreskipamount| and |\theorempostskipamount| define,
% respectively, the spacing before and after such an environment.
% These parameters apply for all theorem sets and can be manipulated
% with the ordinary length macros.  They are rubber lengths
% (`\textsf{skips}'), and therefore can contain \texttt{plus} and
% \texttt{minus} parts.
%
% Since the definition of theorem sets should---most sensibly---be
% placed in the preamble, we only allow installation there.
% It is therefore possible to release the memory used here after
% |\begin{document}|, in order to make room for other
% applications.
%
%
% \subsection{Existing theorem styles}
%
% The following theorem styles exist to date:
% \begin{deflist}{marginbreak}
%    \item[plain]
%       This theorem style emulates the original \LaTeX{} definition,
%       except that additionally the parameters
%       |\theorem...skipamount| are used.
%    \item[break]
%       In this style, the theorem header is followed by a line
%       break.
%    \item[marginbreak]
%       The theorem number is set in the margin, and there is a line
%       break as in \texttt{break}.
%    \item[changebreak]
%       Like \texttt{break}, but with header number and text
%       interchanged.
%    \item[change]
%       Header number and text are interchanged, without a line break.
%    \item[margin]
%       The number is set in the left margin, without a line break.
% \end{deflist}
% All styles (except \texttt{plain}) select |\slshape| as the default
% |\theorembodyfont|.
%
%
% \subsection{Examples}
%
% Given the above theorem sets {\envfont Cor}, {\envfont Exa}
% and {\envfont Rem},
% suppose that the preamble also contains the declarations:
% \begin{verbatim}
%    \theoremstyle{marginbreak}   \newtheorem{Lem}[Cor]{Lemma}
%    \theoremstyle{change}
%    \theorembodyfont{\itshape}        \newtheorem{Def}[Cor]{Definition}
%
%    \theoremheaderfont{\scshape}
%\end{verbatim}
% Then the following are some typical examples of the typeset output
% resulting from their use.
%
% \begin{Cor}
%    This is a sentence typeset in the theorem environment {\envfont
%    Cor}.
% \end{Cor}
%
% \begin{Exa}
%    This is a sentence typeset in the theorem environment {\envfont
%    Exa}.
% \end{Exa}
%
% \begin{Rem}
%    This is a sentence typeset in the theorem environment {\envfont
%    Rem}.
% \end{Rem}
%
% \begin{Lem}[Ben User]
%    This is a sentence typeset in the theorem environment {\envfont
%    Lem}.
% \end{Lem}
%
% \begin{Def}[Very impressive Definition]
%    This is a sentence typeset in the theorem environment {\envfont
%    Def}.
% \end{Def}
%
% The last two examples show the effect of the optional argument to
% a theorem environment (it is the text typeset in parentheses).
%
%
% \section{Special Considerations}
% \label{sec:newfonts}
%
% Theoremheader and body are implemented as a unit. This means that
% the |\theoremheaderfont|  will inherit characteristics of the
% |\theorembodyfont| in \LaTeXe. Thus, if for example
% |\theorembodyfont| is |\itshape| and |\theoremheaderfont|
% is |\bfseries| the font selected for the header will have the
% characteristics `bold extended italic'. If this is not desired one
% should set the |\theoremheaderfont| to something like
% \begin{verbatim}
%      \theoremheaderfont{\normalfont\bfseries}
%\end{verbatim}
% i.e.\ supplying all necessary font information explicitly.
%
%
%
% \section{Acknowledgements}
%
% The publication of this set of macros was only possible with the
% help of Christina Busse (translating the manuscript into English),
% Joachim Pense (playing the r\^ole of typist), Chris Rowley
% (looking everything over) and many others providing useful
% suggestions.
%
% \MaybeStop{%
%    \begin{thebibliography}{9}
%       \bibitem{bk:GMS94} \textsc{M.~Goossens}, \textsc{F.~Mittelbach}
%         and \textsc{A.~Samarin}.
%         \newblock The \LaTeX{} Companion.
%         \newblock
%         Addison-Wesley, Reading, Massachusetts, 1994.
%       \bibitem{latex:source} \textsc{Lamport, Leslie.} \newblock
%                               \texttt{latex.tex}, version 2.09, date
%                               Feb. 1990.
%    \end{thebibliography}
%    \PrintIndex
%    \PrintChanges}
%
%
% \section{The documentation driver file}
%
% The next bit of code contains the documentation driver file for
% \TeX{}, i.e., the file that will produce the documentation you are
% currently reading. It will be extracted from this file by the
% \texttt{docstrip} program. Since it is the first code in the file
% one can alternatively process this file directly with \LaTeXe{} to
% obtain the documentation.
%
%    \begin{macrocode}
%<*driver>
\documentclass{ltxdoc}

\usepackage{theorem}

%<+driver>% The next few lines define theorem sets which are used
%<+driver>% in the example section of the documentation.

\theoremstyle{break}       \newtheorem{Cor}{Corollary}
\theoremstyle{plain}       \newtheorem{Exa}{Example}[section]
{\theorembodyfont{\upshape}\newtheorem{Rem}{Remark}}
\theoremstyle{marginbreak} \newtheorem{Lem}[Cor]{Lemma}
\theoremstyle{change}
\theorembodyfont{\itshape} \newtheorem{Def}[Cor]{Definition}

\theoremheaderfont{\scshape}

\RecordChanges

\begin{document}
   \DocInput{theorem.dtx}
\end{document}
%</driver>
%    \end{macrocode}
%
% \section{Definition of the Macros}
%
% If the file has been loaded before, we abort immediately. If not the
% package announces itself (this is actually done at the very top of
% the file---the way it is done isn't good style so don't copy it).
% \changes{v2.0e}{89/07/19}{Spaces removed from `typeout'.}
% \changes{v2.2b}{95/11/19}{Announce moved to top of file}
%    \begin{macrocode}
%<*package>
%\@ifundefined{theorem@style}{}{\endinput}
%\def\FMithmInfo{1995/11/19 v2.2b Theorem extension package (FMi)}
%\ProvidesPackage{theorem}[\FMithmInfo]
%    \end{macrocode}
%
%
% \subsection{Definition of theorem styles and fonts}
%
% All the definitions in this file are done globally to allow
% inputting this file inside a group.
%
% \begin{macro}{\theoremstyle}
%    Before a theorem style can be installed, the chosen style must be
%    known. For that reason, we must test to see that
%    |\th@|\meta{style} is known or, more precisely, that it is
%    different from |\relax|. If the style is not known then
%    |\th@plain| is used.
%    \begin{macrocode}
\gdef\theoremstyle#1{%
   \@ifundefined{th@#1}{\@warning
          {Unknown theoremstyle `#1'. Using `plain'}%
          \theorem@style{plain}}%
%    \end{macrocode}
%    We save the theorem style to be used in the token register
%    |\theorem@style|.
%    \begin{macrocode}
      {\theorem@style{#1}}%
%    \end{macrocode}
%    Now we ``evaluate'' the theorem style: this means, we call the
%    macro |\th@|\meta{style} which will activate the relevant
%    definitions which are contained in a separate file.  This is done
%    in a group to suppress changes to the current font.  This could
%    otherwise pose problems together with the new font selection
%    scheme\footnote{When I printed the original article using the new
%    font selection scheme I ended with a document with slanted
%    typefaces (text headings and all) simply because one of the
%    theorem styles used \texttt{\bslash sl} at toplevel.} if the
%    |\th@|\meta{style} is evaluated a second time.
% \changes{v2.1a}{90/02/15}{Group added around `th@..}
% \changes{v2.1a}{90/02/15}{Reseting of `..@begintheorem removed.}
%    \begin{macrocode}
      \begingroup
        \csname th@\the\theorem@style \endcsname
      \endgroup}
%    \end{macrocode}
% \end{macro}
%
%
% \begin{macro}{\@begintheorem}
% \changes{v2.1a}{90/02/15}{Relaxed at toplevel.}
% \begin{macro}{\@opargbegintheorem}
% \changes{v2.1a}{90/02/15}{Relaxed at toplevel.}
%    We reset |\@begintheorem| and |\@opargbegintheorem| to
%    |\relax| since these commands are no longer necessary at
%    toplevel. This will save a few tokens.
%    \begin{macrocode}
\global\let\@begintheorem\relax
\global\let\@opargbegintheorem\relax
%    \end{macrocode}
% \end{macro}
% \end{macro}
%
% \begin{macro}{\theorem@style}
%    Obviously the \textsf{token} register used above has to be
%    allocated. To assure the utmost compatibility with the
%    original \LaTeX{} definition, we set the default theorem style to
%    \texttt{plain}, which implements the usual \LaTeX{} convention.
% \changes{v2.0f}{89/07/20}{Unnecessary `global' removed.}
%    \begin{macrocode}
\newtoks\theorem@style
\global\theorem@style{plain}
%    \end{macrocode}
% \end{macro}
%
% \begin{macro}{\theorembodyfont}
% \begin{macro}{\theorem@bodyfont}
%    If the |\theorembodyfont| is set by the user then it should not
%    interact with the default font set in the theorem style. When the
%    new font selection is in force this may happen if, for example,
%    the default is |\itshape| and the new |\theorembodyfont| is
%    |\sffamily|. So we add a |\reset@font| command in front of the
%    user definition.
% \changes{v2.0f}{89/07/20}{Unnecessary `global' removed.}
% \changes{v2.0g}{89/09/19}{`theoremfont to `theorembodyfont.}
% \changes{v2.1c}{92/06/22}{`reset@font added and logic changed.}
% \changes{v2.1d}{92/10/28}{Reset to empty if arg is empty}
%    \begin{macrocode}
\gdef\theorembodyfont#1{%
%    \end{macrocode}
%    We check if the argument supplied is empty and if so put nothing
%    into the |\theorem@bodyfont| token register to allow for
%    |\theorembodyfont{}| as a mean of using the default of the
%    current |\theoremstyle|.
%    \begin{macrocode}
   \def\@tempa{#1}%
   \ifx\@tempa\@empty
    \theorem@bodyfont{}%
   \else
    \theorem@bodyfont{\reset@font#1}%
   \fi
   }
\newtoks\theorem@bodyfont
\global\theorem@bodyfont{}
%    \end{macrocode}
% \end{macro}
% \end{macro}
%
%
% \begin{macro}{\theoremheaderfont}
% \changes{v2.0g}{89/09/19}{Macro added.}
%    The font for the theorem headers is handled differently because
%    this definition applies to all theorem styles.
%    \begin{macrocode}
\gdef\theoremheaderfont#1{\gdef\theorem@headerfont{#1}%
%    \end{macrocode}
%    After using the macro once it is redefined to produce an error
%    message.
%    \begin{macrocode}
       \gdef\theoremheaderfont##1{%
        \typeout{\string\theoremheaderfont\space should be used
                 only once.}}}
%    \end{macrocode}
% \end{macro}
%
% \begin{macro}{\theorem@headerfont}
% \changes{v2.0g}{89/09/19}{Macro added.}
%    To set the |\theorem@headerfont| default we first test if
%    the new fontselection scheme is in force.
% \changes{v2.1a}{90/02/15}{Support for new font selection.}
%    \begin{macrocode}
\ifx\upshape\undefined
%    \end{macrocode}
%    If not we define it to expand into |\bfseries|. We don't use
%    |\let| just in case a following style option redefines this
%    macro.
%    \begin{macrocode}
\gdef\theorem@headerfont{\bfseries}
%    \end{macrocode}
%    Otherwise we reset the current shape before calling |\bfseries|.
%    \begin{macrocode}
\else \gdef\theorem@headerfont{\normalfont\bfseries}\fi
%    \end{macrocode}
% \end{macro}
%
% \begin{macro}{\th@plain}
% \begin{macro}{\th@break}
% \begin{macro}{\th@marginbreak}
% \begin{macro}{\th@changebreak}
% \begin{macro}{\th@change}
% \begin{macro}{\th@margin}
%    The different styles are defined in macros such as
%    |\th@plain|.  Since memory space is precious in
%    ``non-Big-versions'', we have to avoid offering too many unused
%    definitions.  Therefore we define these styles in separate files
%    that can be loaded on demand.  Thus the commands themselves only
%    load these files. We use |\@input@|, a \LaTeXe{} internal command
%    that ensures that the file will be listed with |\listfiles|.
% \changes{v2.2a}{94/02/02}{Use `@input@ to load theorem layout files}
%    \begin{macrocode}
\gdef\th@plain{\@input@{thp.sty}}
\gdef\th@break{\@input@{thb.sty}}
\gdef\th@marginbreak{\@input@{thmb.sty}}
\gdef\th@changebreak{\@input@{thcb.sty}}
\gdef\th@change{\@input@{thc.sty}}
\gdef\th@margin{\@input@{thm.sty}}
%    \end{macrocode}
%    This list will be expanded when new styles become
%    available.  For testing,  just append new theorem substyles
%    as document options.
% \end{macro}
% \end{macro}
% \end{macro}
% \end{macro}
% \end{macro}
% \end{macro}
%
%
%
% \subsection{Definition of a new theorem set}
%
% As already pointed out, a new theorem environment can be defined in
% three different ways:
% \begin{verbatim}
%    \newtheorem{Lem}{Lemma}
%    \newtheorem{Lem}{Lemma}[section]
%    \newtheorem{Lem}[Theorem]{Lemma}
%\end{verbatim}
% The function of the macro |\newtheorem| is to recognize these cases
% and then to branch into one of the three macros |\@ynthm|, |\@xnthm|
% or |\@othm|.  This mechanism is adopted unchanged from
% \cite{latex:source}; the essential point here is that, for example,
% in the second case, the arguments \texttt{Lem}, \texttt{Lemma} and
% \texttt{section} are passed over to the macro |\@xnthm|.
%
% We inspect this case first because the others present fewer problems,
% and thus are easily derived from this one.
%
% \begin{macro}{\@xnthm}
%    For our example arguments, the macro |\@xnthm| must fulfill
%    the following:
%    \begin{itemize}
%       \item
%          Define a new \LaTeX-counter `Lem'
%       \item
%          reset this counter within a |\section|
%       \item
%          define the macro |\theLem|
%       \item
%          define the environment macros |\Lem| and
%          |\endLem| using the current
%          |\theoremstyle| and |\theorem@bodyfont|.
%    \end{itemize}
%    Obviously, all this should happen only if the first argument
%    of |\@xnthm| (i.e.\ \texttt{Lem} in our example) is chosen so
%    as not to conflict with any previously defined commands or
%    environments.  This test is performed by the \LaTeX{} macro
%    |\@ifdefinable|.
%    \begin{macrocode}
\gdef\@xnthm#1#2[#3]{\expandafter\@ifdefinable\csname #1\endcsname
%    \end{macrocode}
%    Therefore, the first argument of |\@ifdefinable| is the
%    expansion (in the example, |\Lem|) of
%    |\csname||#1\endcsname|.  The second argument is
%    executed only if the test has been completed successfully.
%    \begin{macrocode}
   {%
%    \end{macrocode}
%    Now we define the new counter. The names of the \LaTeX{}
%    macros employed should speak for themselves:
% \changes{v2.2b}{1995/10/07}{Use \cs{@newctr}}
% \changes{v2.2c}{1995/11/23}{Use \cs{@newctr} properly.}
%    \begin{macrocode}
    \@definecounter{#1}\@newctr{#1}[#3]%
%    \end{macrocode}
%    Using |\@newctr| will give a proper error message if the counter
%    in |#3| is not defined.
%    In defining `|\theLem|' we must generate the desired macro
%    name by use of |\expandafter| and |\csname|.
%    \begin{macrocode}
    \expandafter\xdef\csname the#1\endcsname
%    \end{macrocode}
%    An |\xdef| is used in order to make the definition global,
%    and to ensure that it contains the replacement texts of
%    |\@thmcountersep| and |\@thmcounter|.\footnote{These
%    two macros can be defined by the document style.  Their default
%    values produce a `\texttt{.}' as separation and an arabic
%    representation of the number.} However, not everything should be
%    expanded. For example, it saves space to use |\thesection|
%    instead of its---at times---lengthy expansion.
%    \begin{macrocode}
      {\expandafter \noexpand \csname the#3\endcsname
       \@thmcountersep \@thmcounter{#1}}%
%    \end{macrocode}
%    Thus with the defaults of \LaTeX{}, |\theLem| would be
%    replaced by the command sequence |\thesection.\arabic{Lem}|.
%
%    We will now look at the definition of the macro which is executed
%    at the beginning of the actual environment (in our example this
%    macro is |\Lem|). It should be noted that we use an
%    ``|\expandafter| trick'' to expand only certain parts of the
%    replacement text at the time of the definition.
%    \begin{macrocode}
    \def\@tempa{\global\@namedef{#1}}%
    \expandafter \@tempa \expandafter{%
%    \end{macrocode}
%    First, the macro that contains the current definitions of
%    |\@begintheorem| and |\@opargtheorem| should be called
%    up.  The name of this macro---as is already known---has the form
%    |\th@|\meta{theorem\ style}; therefore, it must be called by
%    \begin{macrocode}
      \csname th@\the \theorem@style
%    \end{macrocode}
%    In addition the default theorem font should be changeable, i.e.\
%    we have to insert the contents of |\theorem@bodyfont|. For
%    that reason, we expand even further, beyond |\endcsname|,
%    and thus insert the contents of the \textsf{token} register
%    |\theorem@bodyfont| in the replacement text.
%    \begin{macrocode}
            \expandafter \endcsname \the \theorem@bodyfont
%    \end{macrocode}
%    Now it is time to call the macro |\@thm| which takes over
%    the further processing.  It has two arguments: the current
%    counter name (in our example, \texttt{Lem}), and the text of the
%    label (in our example, \texttt{Lemma}).
%    \begin{macrocode}
     \@thm{#1}{#2}}%
%    \end{macrocode}
%    With this, the `sub-definition' is complete.  The macro
%    |\@endtheorem| ends a theorem environment and is, so far, nothing
%    but an |\endtrivlist|.  (Hence it is defined globally, and not
%    within the theorem styles.\footnote{This has to be changed as
%    soon as theorem styles that change \texttt{\bslash @endtheorem}
%    exist.  In such a case, all existing styles must be changed as
%    well since they will have to reset the macro.}) Therefore, we can
%    set it equivalent to the macro that ends the theorem set (in our
%    example, |\endLem|).  However, if some day theorem styles exist
%    that do change |\@endtheorem|, we would have to use the
%    commented-out line instead.
%    \begin{macrocode}
    \global \expandafter \let \csname end#1\endcsname \@endtheorem
%   \global\@namedef{end#1}{\@endtheorem}%
%    \end{macrocode}
%    With these commands all the required definitions are employed,
%    unless the test |\@ifdefinable| has failed.  Therefore, we
%    end the second argument of this macro and with it the definition
%    of |\@xnthm|.
%    \begin{macrocode}
   }}
%    \end{macrocode}
% \end{macro}
%
% \begin{macro}{\@ynthm}
%    The definition of |\@ynthm| is completely analogous. In
%    this case the new counter that is defined is not reset within
%    another counter; thus the definition of |\the...|
%    is simplified:
%    \begin{macrocode}
\gdef\@ynthm#1#2{\expandafter\@ifdefinable\csname #1\endcsname
   {\@definecounter{#1}%
    \expandafter\xdef\csname the#1\endcsname{\@thmcounter{#1}}%
%    \end{macrocode}
%    The rest of the definition corresponds literally to that of
%    |\@xnthm|:
%    \begin{macrocode}
    \def\@tempa{\global\@namedef{#1}}\expandafter \@tempa
     \expandafter{\csname th@\the \theorem@style \expandafter
     \endcsname \the\theorem@bodyfont \@thm{#1}{#2}}%
    \global \expandafter \let \csname end#1\endcsname \@endtheorem}}
%    \end{macrocode}
% \end{macro}
%
% \begin{macro}{\@othm}
%    The definition of |\@othm| does not contain anything new.
%    \begin{macrocode}
\gdef\@othm#1[#2]#3{%
%    \end{macrocode}
% \changes{v2.2b}{95/10/07}{Build in check for counter defined}
%    We do not define a new counter but instead use one that has
%    already been defined.  Thus the only definition we need is that
%    of this pseudo-counter (i.e.\ |\the|\meta{env.\ name}).
%    First we check if |#2| corresponds to a known counter name.
%    \begin{macrocode}
  \expandafter\ifx\csname c@#2\endcsname\relax
   \@nocounterr{#2}%
  \else
   \expandafter\@ifdefinable\csname #1\endcsname
   {\expandafter \xdef \csname the#1\endcsname
     {\expandafter \noexpand \csname the#2\endcsname}%
%    \end{macrocode}
%    All other parts of the definition can be adopted from
%    |\@xnthm|. We have to remember, though, that in this case
%    the name of the current counter and the theorem label have moved
%    to the second and third arguments.
%    \begin{macrocode}
    \def\@tempa{\global\@namedef{#1}}\expandafter \@tempa
     \expandafter{\csname th@\the \theorem@style \expandafter
     \endcsname \the\theorem@bodyfont \@thm{#2}{#3}}%
    \global \expandafter \let \csname end#1\endcsname \@endtheorem}%
  \fi}
%    \end{macrocode}
% \end{macro}
%
% \subsection{Macros that are employed in a {\envfont theorem}
%    environment}
%
% \begin{macro}{\@thm}
%    The macro |\@thm| has to increase the current counter.  Then,
%    depending on whether the environment has (or does not have) an
%    optional argument, it has to branch into either |\@begintheorem|
%    or |\@opargtheorem|.
%    \begin{macrocode}
\gdef\@thm#1#2{\refstepcounter{#1}%
%    \end{macrocode}
%    Now we start a {\envfont trivlist} environment, and give
%    |\@topsep| and |\@topsepadd| the values of the \textsf{skip}
%    registers |\theorempreskipamount| and |\theorempostskipamount|.
%    The value in |\@topsep| is the vertical space that is inserted by
%    the first (and only) |\item| in our |\trivlist| whilst
%    |\@topsepadd| is inserted by |\@endparenv| at the end of that
%    {\envfont trivlist} environment.  By using these registers, we
%    obtain the desired space around a {\envfont theorem} environment.
%    \begin{macrocode}
   \trivlist
   \@topsep \theorempreskipamount               % used by first \item
   \@topsepadd \theorempostskipamount           % used by \@endparenv
%    \end{macrocode}
%    Now we have to test whether an optional argument has been given.
% \changes{v2.0d}{89/07/18}{Forgotten percent added.}
%    \begin{macrocode}
   \@ifnextchar [%
%    \end{macrocode}
%    If there is an optional argument, we will call |\@ythm|, and
%    move the arguments read back into the input stream.
%    \begin{macrocode}
   {\@ythm{#1}{#2}}%
%    \end{macrocode}
%    If not, we call |\@begintheorem|. Its first argument is the
%    name of the theorem set (hence the second argument of
%    |\@thm|). Its second argument is the macro that produces the
%    current number.
%    \begin{macrocode}
   {\@begintheorem{#2}{\csname the#1\endcsname}\ignorespaces}}
%    \end{macrocode}
% \end{macro}
%
% \begin{macro}{\@xthm}
% \begin{macro}{\@ythm}
%    Both these macros were originally called by |\@thm|.  We do
%    not need |\@xthm| anymore, hence we reset it to
%    |\relax|. The definition of |\@ythm| has not changed at
%    all from its definition in \LaTeX. In order to make the macros
%    easier to understand, we will
%    nevertheless present it (commented out).
%    \begin{macrocode}
\global\let\@xthm\relax
% \def\@ythm#1#2[#3]{\@opargbegintheorem{#2}{\csname
%           the#1\endcsname}{#3}\ignorespaces}
%</package>
%    \end{macrocode}
%    The primitive |\ignorespaces| in |\@ythm| and
%    |\@thm| is needed to remove the spaces between the
%    |\begin{...}| and the actual text.
% \end{macro}
% \end{macro}
%
%
% \subsection{Definition of the theorem substyles}
%
% As already pointed out, the theorem substyles, defined below, are
% only loaded when necessary.  Note that all these substyles, except
% \texttt{plain}, have |\slshape| as the default body font.
%
% \subsubsection{The \texttt{plain} style}
%
% \label{dok:th@plain} As the following macros use |@|, we have
% to locally set the |\catcode| of this symbol to ``letter''.
% This happens within a group, so that we do not have to worry about
% which |\catcode| that symbol had before.
%    \begin{macrocode}
%<*thp>
\begingroup \makeatletter
%    \end{macrocode}
% Since we are now within a group, we must make all definitions
% globally.  First we make sure that \texttt{theorem.sty} is loaded.
% This will allow us to use this file as a document style option
% without having to call \texttt{theorem} itself as an option.  At the
% same time, we assure that at least version 2 is loaded, since
% |\theorem@style| was not defined in earlier versions.
%    \begin{macrocode}
\@ifundefined{theorem@style}{\input{theorem.sty}}{}
\ProvidesFile{thp.sty}
             [\FMithmInfo]
%    \end{macrocode}
%
%
% \begin{macro}{\th@plain}
% \begin{macro}{\@begintheorem}
% \begin{macro}{\@opargbegintheorem}
%    |\theoremstyle{plain}| corresponds to the original
%    definition, except that the distances to the surrounding text are
%    determined by the parameters |\theorempreskipamount| and
%    |\theorempostskipamount|.
%    First we set the default body font.
%    \begin{macrocode}
\gdef\th@plain{\normalfont\itshape
%    \end{macrocode}
%    Then we define |\@begintheorem| and
%    |\@opargbegintheorem|. These two macros define how the
%    header of a theorem is typeset.  |\@opargbegintheorem| will
%    be called if a \textsf{theorem} environment with an optional
%    argument is encountered; otherwise, the header is constructed by
%    calling |\@begintheorem|.  If one of these macros is
%    executed, we are within a \textsf{trivlist} environment started by
%    |\@thm|.  So the theorem header is produced with an
%    |\item| command.
%
%    Instead of specifying the header font directly, all standard
%    theorem styles use the |\theorem@headerfont| macro to allow
%    customization.  The extra space (|\labelsep|) is necessary
%    because of problems in the \textsf{trivlist} environment.
%    \begin{macrocode}
  \def\@begintheorem##1##2{%
        \item[\hskip\labelsep \theorem@headerfont ##1\ ##2]}%
%    \end{macrocode}
%    The definition of |\@opargbegintheorem| is completely
%    analogous. The only difference is the fact that there exists a
%    third argument (which is the optional parameter of the
%    environment and contains additional information about the
%    theorem).  Customarily we enclose it in parentheses.
%    \begin{macrocode}
\def\@opargbegintheorem##1##2##3{%
   \item[\hskip\labelsep \theorem@headerfont ##1\ ##2\ (##3)]}}
%    \end{macrocode}
% \end{macro}
% \end{macro}
% \end{macro}
%
% \noindent
% We conclude with an |\endgroup| to restore the
% |\catcode| of |@|.
%    \begin{macrocode}
\endgroup
%</thp>
%    \end{macrocode}
%
%
% \subsubsection{The \texttt{break} style}
%
% This style option is stored in the file |thb.sty|.
% For the next two lines see the documentation for |\th@plain| on
% page~\pageref{dok:th@plain}.
%    \begin{macrocode}
%<*thb>
\begingroup \makeatletter
\@ifundefined{theorem@style}{\input{theorem.sty}}{}
\ProvidesFile{thb.sty}
             [\FMithmInfo]
%    \end{macrocode}
%
% \begin{macro}{\th@break}
%    |\theoremstyle{break}| produces a line break after the name
%    of the theorem.  The font is |\slshape|.  Hence, we define
%    |\th@break| as follows:
%    \begin{macrocode}
\gdef\th@break{\normalfont\slshape
  \def\@begintheorem##1##2{\item[%
%    \end{macrocode}
%    We run into the following problem: it is not possible to create
%    the header with |\item[|\meta{title}|]| and then start
%    a new line by, for example, |\mbox{}\\|. Such a definition
%    will fail whenever a list environment follows immediately.  With
%    the above construction, the |\mbox{}| causes the switch
%    |@inlabel| (cf.\ definition of |\list| and
%    |\trivlist| in \cite{latex:source}) to be set to \texttt{false}
%    and so the following list will insert additional vertical
%    space (|\topskip|). This is quite annoying. Therefore, we
%    create the line break within the |\item|. In order to ensure
%    that the text will begin at the proper position in the following
%    line, we simply pretend that the label does not take any
%    room.\footnote{This will lead to problems whenever very high
%    symbols occurring in the line tower into the
%    heading. So, something else has to be done here sometime.}
%    \begin{macrocode}
       \rlap{\vbox{\hbox{\hskip \labelsep\theorem@headerfont ##1\ ##2}%
                   \hbox{\strut}}}]}%
%    \end{macrocode}
%    Again, the definition of |\@opargbegintheorem| is
%    completely analogous.
%    \begin{macrocode}
\def\@opargbegintheorem##1##2##3{%
  \item[\rlap{\vbox{\hbox{\hskip \labelsep \theorem@headerfont
                     ##1\ ##2\ (##3)}%
                    \hbox{\strut}}}]}}
\endgroup
%</thb>
%    \end{macrocode}
% \end{macro}
%
%
% \subsubsection{The \texttt{changebreak} style}
%
%    \begin{macrocode}
% This style option is stored in the file |thcb.sty|.
%    \begin{macrocode}
%<*thcb>
\begingroup \makeatletter
\@ifundefined{theorem@style}{\input{theorem.sty}}{}
\ProvidesFile{thcb.sty}
             [\FMithmInfo]
%    \end{macrocode}
%
% \begin{macro}{\th@changebreak}
%    The \texttt{change-break} theorem style is like \texttt{break}
%    but with interchange of theorem name and theorem number.  Thus we
%    define |\th@changebreak| as follows:
%    \begin{macrocode}
\gdef\th@changebreak{\normalfont\slshape
  \def\@begintheorem##1##2{\item
       [\rlap{\vbox{\hbox{\hskip\labelsep \theorem@headerfont ##2\ ##1}%
                    \hbox{\strut}}}]}%
\def\@opargbegintheorem##1##2##3{%
  \item[\rlap{\vbox{\hbox{\hskip\labelsep \theorem@headerfont
                     ##2\ ##1\ (##3)}%
                    \hbox{\strut}}}]}}
\endgroup
%</thcb>
%    \end{macrocode}
% \end{macro}
%
%
% \subsubsection{The \texttt{change} style}
%
% This style option is stored in the file |thc.sty|.
%    \begin{macrocode}
%<*thc>
\begingroup \makeatletter
\@ifundefined{theorem@style}{\input{theorem.sty}}{}
\ProvidesFile{thc.sty}
             [\FMithmInfo]
%    \end{macrocode}
%
% \begin{macro}{\th@change}
%    The \texttt{change} theorem style corresponds to the
%    \texttt{change break} style without a linebreak after the header.
%    To say it in another way, it's the same as the \texttt{plain}
%    style but with number and name interchanged and |\slshape| as the
%    default font.
%    \begin{macrocode}
\gdef\th@change{\normalfont\slshape
  \def\@begintheorem##1##2{\item
       [\hskip\labelsep \theorem@headerfont ##2\ ##1]}%
\def\@opargbegintheorem##1##2##3{%
  \item[\hskip\labelsep \theorem@headerfont ##2\ ##1\ (##3)]}}
\endgroup
%</thc>
%    \end{macrocode}
% \end{macro}
%
%
% \subsubsection{The \texttt{marginbreak} style}
%
% This style option is the one used most often at Mainz.  It is saved
% in the file |thmb.sty|.
%    \begin{macrocode}
%<*thmb>
\begingroup \makeatletter
\@ifundefined{theorem@style}{\input{theorem.sty}}{}
\ProvidesFile{thmb.sty}
             [\FMithmInfo]
%    \end{macrocode}
%
% \begin{macro}{\th@marginbreak}
%    The \texttt{margin break} style is nearly the same as the
%    \texttt{change break} style. The only difference is the placement
%    of the theorem number.  We use |\llap| to place it in the left
%    margin.
%
%    In this style |\labelsep| denotes the separation between the
%    number and the text.
%    \begin{macrocode}
\gdef\th@marginbreak{\normalfont\slshape
  \def\@begintheorem##1##2{\item
       [\rlap{\vbox{\theorem@headerfont
                    \hbox{\llap{##2}\hskip\labelsep ##1}%
                    \hbox{\strut}}}]}%
\def\@opargbegintheorem##1##2##3{%
  \item[\rlap{\vbox{\theorem@headerfont
                    \hbox{\llap{##2}\hskip\labelsep ##1\ (##3)}%
                    \hbox{\strut}}}]}}
\endgroup
%</thmb>
%    \end{macrocode}
% \changes{v2.1c}{90/02/17}{`endgroup outside macrocode env.}
% \end{macro}
%
%
%
% \subsubsection{The \texttt{margin} style}
%
% This style option is stored in the file |thm.sty|.
%    \begin{macrocode}
%<*thm>
\begingroup \makeatletter
\@ifundefined{theorem@style}{\input{theorem.sty}}{}
\ProvidesFile{thm.sty}
             [\FMithmInfo]
%    \end{macrocode}
%
% \begin{macro}{\th@margin}
%    Again this is only a variant of the theorem styles described
%    above without any new ideas.
%    \begin{macrocode}
\gdef\th@margin{\normalfont\slshape
  \def\@begintheorem##1##2{\item
       [\theorem@headerfont \llap{##2}\hskip\labelsep ##1]}%
\def\@opargbegintheorem##1##2##3{%
  \item[\theorem@headerfont \llap{##2}\hskip\labelsep ##1\ (##3)]}}
\endgroup
%</thm>
%    \end{macrocode}
% \end{macro}
%
%
%
%
% \subsection{Final Definitions}
%
% \begin{macro}{\theorempreskipamount}
% \begin{macro}{\theorempostskipamount}
%    The \textsf{skip} parameters that regulate the vertical empty space
%    before and after the {\envfont theorem} environment have
%    to be allocated as well.
% \changes{v2.0f}{89/07/20}{Unnecessary `global' removed.}
%    \begin{macrocode}
%<*package>
\newskip\theorempreskipamount
\newskip\theorempostskipamount
%    \end{macrocode}
%    Since we have used the same values for all theorem sets, we now can
%    assign them.
%    \begin{macrocode}
\global\setlength\theorempreskipamount{12pt plus 5pt minus 3pt}
\global\setlength\theorempostskipamount{8pt plus 3pt minus 1.5pt}
%    \end{macrocode}
% \end{macro}
% \end{macro}
%
%
% \begin{macro}{\@endtheorem}
%    The same holds for the macro |\@endtheorem|, which ends a
%    {\envfont theorem} environment. Since it is the same for all
%    theorem sets, it is removed from the macros |\th@|\meta{style}.
%    It simply ends the {\envfont trivlist} environment, which was
%    begun in |\@thm|.
%    \begin{macrocode}
\global\let\@endtheorem=\endtrivlist
%    \end{macrocode}
% \end{macro}
%
%
%
% \begin{macro}{\@preamblecmds}
%    All macros defined above are to be used only in the preamble.
%    Therefore, we insert them in |\@preamblecmds| which will disable
%    them at begin document. This is done by the internal \LaTeXe{}
%    command |\@onlypreamble|.
% \changes{v2.0g}{89/09/19}{`theoremheaderfont and `theorembodyfont
%    added.}
% \changes{v2.2a}{94/02/03}{Use \LaTeXe{} interface}
%    \begin{macrocode}
\@onlypreamble\@xnthm
\@onlypreamble\@ynthm
\@onlypreamble\@othm
\@onlypreamble\newtheorem
\@onlypreamble\theoremstyle
\@onlypreamble\theorembodyfont
\@onlypreamble\theoremheaderfont
%    \end{macrocode}
% \end{macro}
%
% Finally we declare the \texttt{plain} theorem style to be the default.
% \changes{v2.1b}{90/06/04}{Fixed missing `item by declaring default
%    style.}
%    \begin{macrocode}
\theoremstyle{plain}
%</package>
%    \end{macrocode}
%
% \Finale

\endinput
}{}
\ProvidesFile{thb.sty}
             [\FMithmInfo]
%    \end{macrocode}
%
% \begin{macro}{\th@break}
%    |\theoremstyle{break}| produces a line break after the name
%    of the theorem.  The font is |\slshape|.  Hence, we define
%    |\th@break| as follows:
%    \begin{macrocode}
\gdef\th@break{\normalfont\slshape
  \def\@begintheorem##1##2{\item[%
%    \end{macrocode}
%    We run into the following problem: it is not possible to create
%    the header with |\item[|\meta{title}|]| and then start
%    a new line by, for example, |\mbox{}\\|. Such a definition
%    will fail whenever a list environment follows immediately.  With
%    the above construction, the |\mbox{}| causes the switch
%    |@inlabel| (cf.\ definition of |\list| and
%    |\trivlist| in \cite{latex:source}) to be set to \texttt{false}
%    and so the following list will insert additional vertical
%    space (|\topskip|). This is quite annoying. Therefore, we
%    create the line break within the |\item|. In order to ensure
%    that the text will begin at the proper position in the following
%    line, we simply pretend that the label does not take any
%    room.\footnote{This will lead to problems whenever very high
%    symbols occurring in the line tower into the
%    heading. So, something else has to be done here sometime.}
%    \begin{macrocode}
       \rlap{\vbox{\hbox{\hskip \labelsep\theorem@headerfont ##1\ ##2}%
                   \hbox{\strut}}}]}%
%    \end{macrocode}
%    Again, the definition of |\@opargbegintheorem| is
%    completely analogous.
%    \begin{macrocode}
\def\@opargbegintheorem##1##2##3{%
  \item[\rlap{\vbox{\hbox{\hskip \labelsep \theorem@headerfont
                     ##1\ ##2\ (##3)}%
                    \hbox{\strut}}}]}}
\endgroup
%</thb>
%    \end{macrocode}
% \end{macro}
%
%
% \subsubsection{The \texttt{changebreak} style}
%
%    \begin{macrocode}
% This style option is stored in the file |thcb.sty|.
%    \begin{macrocode}
%<*thcb>
\begingroup \makeatletter
\@ifundefined{theorem@style}{% \iffalse meta-comment
%
% Copyright (C) 1993-2024
%
% The LaTeX Project and any individual authors listed elsewhere
% in this file.
%
% This file is part of the Standard LaTeX `Tools Bundle'.
% -------------------------------------------------------
%
% It may be distributed and/or modified under the
% conditions of the LaTeX Project Public License, either version 1.3c
% of this license or (at your option) any later version.
% The latest version of this license is in
%    https://www.latex-project.org/lppl.txt
% and version 1.3c or later is part of all distributions of LaTeX
% version 2005/12/01 or later.
%
% The list of all files belonging to the LaTeX `Tools Bundle' is
% given in the file `manifest.txt'.
%
% \fi
%
%%
%
%
\def\FMithmInfo{2023/07/05 v2.2c Theorem extension package (FMi)}
%
% \ProvidesFile{theorem.dtx}[\FMithmInfo]
%\iffalse   % this is a METACOMMENT !
%%
%% Package `theorem' to use with LaTeX2e
%% Copyright (C) 1989-1995 Frank Mittelbach, all rights reserved.
%<+package>\@ifundefined{theorem@style}{}{\endinput}
%<+package>\NeedsTeXFormat{LaTeX2e}[1995/06/01]
%<+package>\ProvidesPackage{theorem}
%<+package>         [\FMithmInfo]
%
% \fi
%
% \changes{v1.0}{87/05/07}{First draft.}
% \changes{v1.0a}{87/05/07}{`thistheoremstyle' added.}
% \changes{v1.1a}{87/07/29}{New concept: redefine the internal
%                          \protect\LaTeX{} commands `@xnthm' etc.}
% \changes{v1.1b}{87/07/31}{`theoremfont' used to specify font.}
% \changes{v1.2a}{88/01/08}{`@endtheorem' now globally defined.}
% \changes{v1.2b}{88/01/11}{`newtheorem' only allowed in the preamble.}
% \changes{v1.3a}{88/02/14}{`theoremfont' now a toks register.}
%
% \changes{v2.0a}{88/11/16}{New implementation and new documentation.}
% \changes{v2.0a}{88/11/16}{No layout changes but internal modifications
%                           old substyles won't work.}
% \changes{v2.0b}{89/01/30}{`global' used for all definitions.}
% \changes{v2.0c}{89/06/15}{Documentation translated to English.
%                            Thanks to LLL and Nina}
% \changes{v2.1a}{90/02/15}{Support for new font selection.}
% \changes{v2.2a}{94/02/03}{Cleanup for LaTeX2e}
%
% \DoNotIndex{\@endparenv,\@ifdefinable,\@ifnextchar,\@ifundefined}
% \DoNotIndex{\@namedef,\@nameuse,\@spaces,\@tempa,\@warning}
% \DoNotIndex{\ }
% \DoNotIndex{\begingroup,\bfseries}
% \DoNotIndex{\csname}
% \DoNotIndex{\def,\do,\docdate}
% \DoNotIndex{\endcsname,\endgroup,\endinput,\endtrivlist,\expandafter}
% \DoNotIndex{\filedate,\fileversion}
% \DoNotIndex{\gdef,\global}
% \DoNotIndex{\hbox,\hskip}
% \DoNotIndex{\ignorespaces,\input,\itshape,\item}
% \DoNotIndex{\labelsep,\let,\llap}
% \DoNotIndex{\makeatletter}
% \DoNotIndex{\noexpand}
% \DoNotIndex{\refstepcounter,\relax,\rlap}
% \DoNotIndex{\setlength,\slshape,\space,\string,\strut}
% \DoNotIndex{\the,\trivlist,\typeout}
% \DoNotIndex{\vbox}
% \DoNotIndex{\xdef}
%
% \def\deflabel#1{\ttfamily #1\hfill}
% \def\deflist#1{\begin{list}{}{\settowidth\labelwidth{\ttfamily #1}%
%                               \setlength\leftmargin\labelwidth
%                               \addtolength\leftmargin\labelsep
%                               \let\makelabel\deflabel}}
% \def\enddeflist{\end{list}}
% \def\envfont{\normalfont\ttfamily}
%
% \MakeShortVerb{\|}
%
% \newcommand{\TUB}{{\sl TUGboat\/}}
%
% \GetFileInfo{theorem.dtx}
% \title{An Extension of the
%        \LaTeX{} theorem environment\thanks{This file
%        has version number \fileversion, last
%        revised \filedate.}}
% \author{Frank Mittelbach}
%
% \MaintainedByLaTeXTeam{tools}
% \maketitle
%
% \begin{abstract}
%    The macros described in this paper yield an extension of the
%    \LaTeX{} theorem mechanism. It is designed to satisfy the
%    different requirements of various journals. Thus, the
%    layout of the ``theorems'' can be manipulated by determining a
%    ``style''. This article describes not only the use, but
%    also the definition, of the necessary macros.
% \end{abstract}
%
% \section*{Preface to version 2.2}
%
% For \LaTeXe{} this package did not need any fundamental changes. I
% only modified the messages generated so that theorem layout styles
% will show up with the |\listfiles| command and cleaned the section
% on the New Font Selection Scheme since this is now included in
% \LaTeX.
%
% \section*{Preface to version 2.1}
%
% This version is identical to 2.0g described in \TUB\ 10\#3 except
% for some internal defaults which are now set depending on the used
% font selection scheme.
%
% This was done to avoid unpleasant surprises if the new font selection
% scheme is in force. For further details see
% section~\ref{sec:newfonts} and \cite{bk:GMS94}.
%
% \section{Introduction}
%
% For our purposes here, ``theorems'' are labelled enunciations,
% often set off from the main text by extra space and a font change.
% Theorems, corollaries, conjectures, definitions, and remarks
% are all instances of ``theorems''.  The ``header'' of these structures
% is composed of a label (such as \textsc{Theorem} or \textsc{Remark})
% and a number which serializes an item in the sequence of items
% with the same label.
%
% Shortly after the introduction of \LaTeX{} at the Fachbereich
% Mathematik in Mainz, the desire to manipulate the layout of
% ``theorems'' arose.  In Mainz, the following two conventions came
% into general use:
% \begin{enumerate}
%   \item The number of the theorem is shown in the margin.
%   \item There is a line break at
%         the end of the theorem header.
% \end{enumerate}
%
% Additionally, some journals require different formats which depend on
% the ``sort of theorem'': e.g.\ often remarks and definitions are set
% in |\upshape|, while |\itshape| is employed for main theorems.
%
% Confronted with these requirements, a theorem environment was
% developed in Mainz which allows separate determination of the
% layout of the ``theorems sets'', comparable to |\pagestyle|.
%
%
% \section{The user interface}
% \subsection{Defining new theorem sets}
%
% \DescribeMacro\newtheorem
% As in the original \LaTeX{} version, the command |\newtheorem|
% defines a new ``theorem set'' or ``theorem-like structure''.
% Two required arguments name the new environment and give the text
% to be typeset with each instance of the new ``set'', while
% an optional argument determines how
% the ``set'' is enumerated:
% \begin{description}
%    \item[\ttfamily \bslash newtheorem\{foo\}\{bar\}]
%       The theorem set {\envfont foo} (whose name is \texttt{bar})
%       uses its own counter.
%    \item[\ttfamily \bslash newtheorem\{foo2\}{[foo]}\{bar2\}]
%       The theorem set {\envfont foo2} (printed name \texttt{bar2})
%       uses the same counter as the theorem set \texttt{foo}.
%    \item[\ttfamily \bslash newtheorem\{foo3\}\{bar3\}{[section]}]
%       The theorem set {\envfont foo3} (printed name \texttt{bar3}) is
%       enumerated within the counter \texttt{section}, i.e.\ with every
%       new |\section| the enumeration begins again with 1, and
%       the enumeration is composed from the section-number and the
%       theorem counter itself.
% \end{description}
%
% \DescribeMacro\theoremstyle
% Additionally, the command |\theoremstyle| can define the layout
% of various, or all, theorem sets.  It should be noted that any
% theorem set defined by |\newtheorem| is typeset in the
% |\theoremstyle| that is current at the time of the definition.
% Thus, the following
% \begin{verbatim}
%    \theoremstyle{break}        \newtheorem{Cor}{Corollary}
%    \theoremstyle{plain}        \newtheorem{Exa}{Example}[section]
%\end{verbatim}                             ^^A No blank at line start
% leads to the result that the set {\envfont Cor} is formatted in the
% style \texttt{break}, while the set {\envfont Exa} and all the
% following ones are formatted in the style \texttt{plain}, unless
% another |\theoremstyle| follows.  Since the definitions installed by
% |\newtheorem| are global, one also can limit |\theoremstyle| locally
% by grouping braces.
%
% \DescribeMacro\theorembodyfont
% The choice of the font for the theorem body is completely
% independent of the chosen |\theoremstyle|; this has proven to
% be very advantageous.  For example,
% \begin{verbatim}
%   {\theorembodyfont{\upshape}          \newtheorem{Rem}{Remark}}
%\end{verbatim}
% defines a theorem set {\envfont Rem}, which will be set in
% |\upshape| in the current layout (which in our example is
% \texttt{plain}).  As with |\theoremstyle|, the |\theorembodyfont|
% chosen is that current at the time of |\newtheorem|.  If
% |\theorembodyfont| is not specified or one defines
% |\theorembodyfont{}|, then the font used will be that defined by the
% |\theoremstyle|.
%
% \DescribeMacro\theoremheaderfont It is also possible to customize
% the font used for the theorem headers.  This is, however, a global
% declaration, and therefore there should be at most one
% |\theoremheaderfont| declaration in the preamble.\footnote{If it is
% actually necessary to have different header fonts, one has to define
% new theorem styles (substituting the desired font) or specify the
% information directly in the \texttt{\bslash newtheorem} declaration
% (the unclean variant).}
%
% \DescribeMacro\theorempreskipamount
% \DescribeMacro\theorempostskipamount Two additional parameters
% affect the vertical space around the theorem environments:
% |\theorempreskipamount| and |\theorempostskipamount| define,
% respectively, the spacing before and after such an environment.
% These parameters apply for all theorem sets and can be manipulated
% with the ordinary length macros.  They are rubber lengths
% (`\textsf{skips}'), and therefore can contain \texttt{plus} and
% \texttt{minus} parts.
%
% Since the definition of theorem sets should---most sensibly---be
% placed in the preamble, we only allow installation there.
% It is therefore possible to release the memory used here after
% |\begin{document}|, in order to make room for other
% applications.
%
%
% \subsection{Existing theorem styles}
%
% The following theorem styles exist to date:
% \begin{deflist}{marginbreak}
%    \item[plain]
%       This theorem style emulates the original \LaTeX{} definition,
%       except that additionally the parameters
%       |\theorem...skipamount| are used.
%    \item[break]
%       In this style, the theorem header is followed by a line
%       break.
%    \item[marginbreak]
%       The theorem number is set in the margin, and there is a line
%       break as in \texttt{break}.
%    \item[changebreak]
%       Like \texttt{break}, but with header number and text
%       interchanged.
%    \item[change]
%       Header number and text are interchanged, without a line break.
%    \item[margin]
%       The number is set in the left margin, without a line break.
% \end{deflist}
% All styles (except \texttt{plain}) select |\slshape| as the default
% |\theorembodyfont|.
%
%
% \subsection{Examples}
%
% Given the above theorem sets {\envfont Cor}, {\envfont Exa}
% and {\envfont Rem},
% suppose that the preamble also contains the declarations:
% \begin{verbatim}
%    \theoremstyle{marginbreak}   \newtheorem{Lem}[Cor]{Lemma}
%    \theoremstyle{change}
%    \theorembodyfont{\itshape}        \newtheorem{Def}[Cor]{Definition}
%
%    \theoremheaderfont{\scshape}
%\end{verbatim}
% Then the following are some typical examples of the typeset output
% resulting from their use.
%
% \begin{Cor}
%    This is a sentence typeset in the theorem environment {\envfont
%    Cor}.
% \end{Cor}
%
% \begin{Exa}
%    This is a sentence typeset in the theorem environment {\envfont
%    Exa}.
% \end{Exa}
%
% \begin{Rem}
%    This is a sentence typeset in the theorem environment {\envfont
%    Rem}.
% \end{Rem}
%
% \begin{Lem}[Ben User]
%    This is a sentence typeset in the theorem environment {\envfont
%    Lem}.
% \end{Lem}
%
% \begin{Def}[Very impressive Definition]
%    This is a sentence typeset in the theorem environment {\envfont
%    Def}.
% \end{Def}
%
% The last two examples show the effect of the optional argument to
% a theorem environment (it is the text typeset in parentheses).
%
%
% \section{Special Considerations}
% \label{sec:newfonts}
%
% Theoremheader and body are implemented as a unit. This means that
% the |\theoremheaderfont|  will inherit characteristics of the
% |\theorembodyfont| in \LaTeXe. Thus, if for example
% |\theorembodyfont| is |\itshape| and |\theoremheaderfont|
% is |\bfseries| the font selected for the header will have the
% characteristics `bold extended italic'. If this is not desired one
% should set the |\theoremheaderfont| to something like
% \begin{verbatim}
%      \theoremheaderfont{\normalfont\bfseries}
%\end{verbatim}
% i.e.\ supplying all necessary font information explicitly.
%
%
%
% \section{Acknowledgements}
%
% The publication of this set of macros was only possible with the
% help of Christina Busse (translating the manuscript into English),
% Joachim Pense (playing the r\^ole of typist), Chris Rowley
% (looking everything over) and many others providing useful
% suggestions.
%
% \MaybeStop{%
%    \begin{thebibliography}{9}
%       \bibitem{bk:GMS94} \textsc{M.~Goossens}, \textsc{F.~Mittelbach}
%         and \textsc{A.~Samarin}.
%         \newblock The \LaTeX{} Companion.
%         \newblock
%         Addison-Wesley, Reading, Massachusetts, 1994.
%       \bibitem{latex:source} \textsc{Lamport, Leslie.} \newblock
%                               \texttt{latex.tex}, version 2.09, date
%                               Feb. 1990.
%    \end{thebibliography}
%    \PrintIndex
%    \PrintChanges}
%
%
% \section{The documentation driver file}
%
% The next bit of code contains the documentation driver file for
% \TeX{}, i.e., the file that will produce the documentation you are
% currently reading. It will be extracted from this file by the
% \texttt{docstrip} program. Since it is the first code in the file
% one can alternatively process this file directly with \LaTeXe{} to
% obtain the documentation.
%
%    \begin{macrocode}
%<*driver>
\documentclass{ltxdoc}

\usepackage{theorem}

%<+driver>% The next few lines define theorem sets which are used
%<+driver>% in the example section of the documentation.

\theoremstyle{break}       \newtheorem{Cor}{Corollary}
\theoremstyle{plain}       \newtheorem{Exa}{Example}[section]
{\theorembodyfont{\upshape}\newtheorem{Rem}{Remark}}
\theoremstyle{marginbreak} \newtheorem{Lem}[Cor]{Lemma}
\theoremstyle{change}
\theorembodyfont{\itshape} \newtheorem{Def}[Cor]{Definition}

\theoremheaderfont{\scshape}

\RecordChanges

\begin{document}
   \DocInput{theorem.dtx}
\end{document}
%</driver>
%    \end{macrocode}
%
% \section{Definition of the Macros}
%
% If the file has been loaded before, we abort immediately. If not the
% package announces itself (this is actually done at the very top of
% the file---the way it is done isn't good style so don't copy it).
% \changes{v2.0e}{89/07/19}{Spaces removed from `typeout'.}
% \changes{v2.2b}{95/11/19}{Announce moved to top of file}
%    \begin{macrocode}
%<*package>
%\@ifundefined{theorem@style}{}{\endinput}
%\def\FMithmInfo{1995/11/19 v2.2b Theorem extension package (FMi)}
%\ProvidesPackage{theorem}[\FMithmInfo]
%    \end{macrocode}
%
%
% \subsection{Definition of theorem styles and fonts}
%
% All the definitions in this file are done globally to allow
% inputting this file inside a group.
%
% \begin{macro}{\theoremstyle}
%    Before a theorem style can be installed, the chosen style must be
%    known. For that reason, we must test to see that
%    |\th@|\meta{style} is known or, more precisely, that it is
%    different from |\relax|. If the style is not known then
%    |\th@plain| is used.
%    \begin{macrocode}
\gdef\theoremstyle#1{%
   \@ifundefined{th@#1}{\@warning
          {Unknown theoremstyle `#1'. Using `plain'}%
          \theorem@style{plain}}%
%    \end{macrocode}
%    We save the theorem style to be used in the token register
%    |\theorem@style|.
%    \begin{macrocode}
      {\theorem@style{#1}}%
%    \end{macrocode}
%    Now we ``evaluate'' the theorem style: this means, we call the
%    macro |\th@|\meta{style} which will activate the relevant
%    definitions which are contained in a separate file.  This is done
%    in a group to suppress changes to the current font.  This could
%    otherwise pose problems together with the new font selection
%    scheme\footnote{When I printed the original article using the new
%    font selection scheme I ended with a document with slanted
%    typefaces (text headings and all) simply because one of the
%    theorem styles used \texttt{\bslash sl} at toplevel.} if the
%    |\th@|\meta{style} is evaluated a second time.
% \changes{v2.1a}{90/02/15}{Group added around `th@..}
% \changes{v2.1a}{90/02/15}{Reseting of `..@begintheorem removed.}
%    \begin{macrocode}
      \begingroup
        \csname th@\the\theorem@style \endcsname
      \endgroup}
%    \end{macrocode}
% \end{macro}
%
%
% \begin{macro}{\@begintheorem}
% \changes{v2.1a}{90/02/15}{Relaxed at toplevel.}
% \begin{macro}{\@opargbegintheorem}
% \changes{v2.1a}{90/02/15}{Relaxed at toplevel.}
%    We reset |\@begintheorem| and |\@opargbegintheorem| to
%    |\relax| since these commands are no longer necessary at
%    toplevel. This will save a few tokens.
%    \begin{macrocode}
\global\let\@begintheorem\relax
\global\let\@opargbegintheorem\relax
%    \end{macrocode}
% \end{macro}
% \end{macro}
%
% \begin{macro}{\theorem@style}
%    Obviously the \textsf{token} register used above has to be
%    allocated. To assure the utmost compatibility with the
%    original \LaTeX{} definition, we set the default theorem style to
%    \texttt{plain}, which implements the usual \LaTeX{} convention.
% \changes{v2.0f}{89/07/20}{Unnecessary `global' removed.}
%    \begin{macrocode}
\newtoks\theorem@style
\global\theorem@style{plain}
%    \end{macrocode}
% \end{macro}
%
% \begin{macro}{\theorembodyfont}
% \begin{macro}{\theorem@bodyfont}
%    If the |\theorembodyfont| is set by the user then it should not
%    interact with the default font set in the theorem style. When the
%    new font selection is in force this may happen if, for example,
%    the default is |\itshape| and the new |\theorembodyfont| is
%    |\sffamily|. So we add a |\reset@font| command in front of the
%    user definition.
% \changes{v2.0f}{89/07/20}{Unnecessary `global' removed.}
% \changes{v2.0g}{89/09/19}{`theoremfont to `theorembodyfont.}
% \changes{v2.1c}{92/06/22}{`reset@font added and logic changed.}
% \changes{v2.1d}{92/10/28}{Reset to empty if arg is empty}
%    \begin{macrocode}
\gdef\theorembodyfont#1{%
%    \end{macrocode}
%    We check if the argument supplied is empty and if so put nothing
%    into the |\theorem@bodyfont| token register to allow for
%    |\theorembodyfont{}| as a mean of using the default of the
%    current |\theoremstyle|.
%    \begin{macrocode}
   \def\@tempa{#1}%
   \ifx\@tempa\@empty
    \theorem@bodyfont{}%
   \else
    \theorem@bodyfont{\reset@font#1}%
   \fi
   }
\newtoks\theorem@bodyfont
\global\theorem@bodyfont{}
%    \end{macrocode}
% \end{macro}
% \end{macro}
%
%
% \begin{macro}{\theoremheaderfont}
% \changes{v2.0g}{89/09/19}{Macro added.}
%    The font for the theorem headers is handled differently because
%    this definition applies to all theorem styles.
%    \begin{macrocode}
\gdef\theoremheaderfont#1{\gdef\theorem@headerfont{#1}%
%    \end{macrocode}
%    After using the macro once it is redefined to produce an error
%    message.
%    \begin{macrocode}
       \gdef\theoremheaderfont##1{%
        \typeout{\string\theoremheaderfont\space should be used
                 only once.}}}
%    \end{macrocode}
% \end{macro}
%
% \begin{macro}{\theorem@headerfont}
% \changes{v2.0g}{89/09/19}{Macro added.}
%    To set the |\theorem@headerfont| default we first test if
%    the new fontselection scheme is in force.
% \changes{v2.1a}{90/02/15}{Support for new font selection.}
%    \begin{macrocode}
\ifx\upshape\undefined
%    \end{macrocode}
%    If not we define it to expand into |\bfseries|. We don't use
%    |\let| just in case a following style option redefines this
%    macro.
%    \begin{macrocode}
\gdef\theorem@headerfont{\bfseries}
%    \end{macrocode}
%    Otherwise we reset the current shape before calling |\bfseries|.
%    \begin{macrocode}
\else \gdef\theorem@headerfont{\normalfont\bfseries}\fi
%    \end{macrocode}
% \end{macro}
%
% \begin{macro}{\th@plain}
% \begin{macro}{\th@break}
% \begin{macro}{\th@marginbreak}
% \begin{macro}{\th@changebreak}
% \begin{macro}{\th@change}
% \begin{macro}{\th@margin}
%    The different styles are defined in macros such as
%    |\th@plain|.  Since memory space is precious in
%    ``non-Big-versions'', we have to avoid offering too many unused
%    definitions.  Therefore we define these styles in separate files
%    that can be loaded on demand.  Thus the commands themselves only
%    load these files. We use |\@input@|, a \LaTeXe{} internal command
%    that ensures that the file will be listed with |\listfiles|.
% \changes{v2.2a}{94/02/02}{Use `@input@ to load theorem layout files}
%    \begin{macrocode}
\gdef\th@plain{\@input@{thp.sty}}
\gdef\th@break{\@input@{thb.sty}}
\gdef\th@marginbreak{\@input@{thmb.sty}}
\gdef\th@changebreak{\@input@{thcb.sty}}
\gdef\th@change{\@input@{thc.sty}}
\gdef\th@margin{\@input@{thm.sty}}
%    \end{macrocode}
%    This list will be expanded when new styles become
%    available.  For testing,  just append new theorem substyles
%    as document options.
% \end{macro}
% \end{macro}
% \end{macro}
% \end{macro}
% \end{macro}
% \end{macro}
%
%
%
% \subsection{Definition of a new theorem set}
%
% As already pointed out, a new theorem environment can be defined in
% three different ways:
% \begin{verbatim}
%    \newtheorem{Lem}{Lemma}
%    \newtheorem{Lem}{Lemma}[section]
%    \newtheorem{Lem}[Theorem]{Lemma}
%\end{verbatim}
% The function of the macro |\newtheorem| is to recognize these cases
% and then to branch into one of the three macros |\@ynthm|, |\@xnthm|
% or |\@othm|.  This mechanism is adopted unchanged from
% \cite{latex:source}; the essential point here is that, for example,
% in the second case, the arguments \texttt{Lem}, \texttt{Lemma} and
% \texttt{section} are passed over to the macro |\@xnthm|.
%
% We inspect this case first because the others present fewer problems,
% and thus are easily derived from this one.
%
% \begin{macro}{\@xnthm}
%    For our example arguments, the macro |\@xnthm| must fulfill
%    the following:
%    \begin{itemize}
%       \item
%          Define a new \LaTeX-counter `Lem'
%       \item
%          reset this counter within a |\section|
%       \item
%          define the macro |\theLem|
%       \item
%          define the environment macros |\Lem| and
%          |\endLem| using the current
%          |\theoremstyle| and |\theorem@bodyfont|.
%    \end{itemize}
%    Obviously, all this should happen only if the first argument
%    of |\@xnthm| (i.e.\ \texttt{Lem} in our example) is chosen so
%    as not to conflict with any previously defined commands or
%    environments.  This test is performed by the \LaTeX{} macro
%    |\@ifdefinable|.
%    \begin{macrocode}
\gdef\@xnthm#1#2[#3]{\expandafter\@ifdefinable\csname #1\endcsname
%    \end{macrocode}
%    Therefore, the first argument of |\@ifdefinable| is the
%    expansion (in the example, |\Lem|) of
%    |\csname||#1\endcsname|.  The second argument is
%    executed only if the test has been completed successfully.
%    \begin{macrocode}
   {%
%    \end{macrocode}
%    Now we define the new counter. The names of the \LaTeX{}
%    macros employed should speak for themselves:
% \changes{v2.2b}{1995/10/07}{Use \cs{@newctr}}
% \changes{v2.2c}{1995/11/23}{Use \cs{@newctr} properly.}
%    \begin{macrocode}
    \@definecounter{#1}\@newctr{#1}[#3]%
%    \end{macrocode}
%    Using |\@newctr| will give a proper error message if the counter
%    in |#3| is not defined.
%    In defining `|\theLem|' we must generate the desired macro
%    name by use of |\expandafter| and |\csname|.
%    \begin{macrocode}
    \expandafter\xdef\csname the#1\endcsname
%    \end{macrocode}
%    An |\xdef| is used in order to make the definition global,
%    and to ensure that it contains the replacement texts of
%    |\@thmcountersep| and |\@thmcounter|.\footnote{These
%    two macros can be defined by the document style.  Their default
%    values produce a `\texttt{.}' as separation and an arabic
%    representation of the number.} However, not everything should be
%    expanded. For example, it saves space to use |\thesection|
%    instead of its---at times---lengthy expansion.
%    \begin{macrocode}
      {\expandafter \noexpand \csname the#3\endcsname
       \@thmcountersep \@thmcounter{#1}}%
%    \end{macrocode}
%    Thus with the defaults of \LaTeX{}, |\theLem| would be
%    replaced by the command sequence |\thesection.\arabic{Lem}|.
%
%    We will now look at the definition of the macro which is executed
%    at the beginning of the actual environment (in our example this
%    macro is |\Lem|). It should be noted that we use an
%    ``|\expandafter| trick'' to expand only certain parts of the
%    replacement text at the time of the definition.
%    \begin{macrocode}
    \def\@tempa{\global\@namedef{#1}}%
    \expandafter \@tempa \expandafter{%
%    \end{macrocode}
%    First, the macro that contains the current definitions of
%    |\@begintheorem| and |\@opargtheorem| should be called
%    up.  The name of this macro---as is already known---has the form
%    |\th@|\meta{theorem\ style}; therefore, it must be called by
%    \begin{macrocode}
      \csname th@\the \theorem@style
%    \end{macrocode}
%    In addition the default theorem font should be changeable, i.e.\
%    we have to insert the contents of |\theorem@bodyfont|. For
%    that reason, we expand even further, beyond |\endcsname|,
%    and thus insert the contents of the \textsf{token} register
%    |\theorem@bodyfont| in the replacement text.
%    \begin{macrocode}
            \expandafter \endcsname \the \theorem@bodyfont
%    \end{macrocode}
%    Now it is time to call the macro |\@thm| which takes over
%    the further processing.  It has two arguments: the current
%    counter name (in our example, \texttt{Lem}), and the text of the
%    label (in our example, \texttt{Lemma}).
%    \begin{macrocode}
     \@thm{#1}{#2}}%
%    \end{macrocode}
%    With this, the `sub-definition' is complete.  The macro
%    |\@endtheorem| ends a theorem environment and is, so far, nothing
%    but an |\endtrivlist|.  (Hence it is defined globally, and not
%    within the theorem styles.\footnote{This has to be changed as
%    soon as theorem styles that change \texttt{\bslash @endtheorem}
%    exist.  In such a case, all existing styles must be changed as
%    well since they will have to reset the macro.}) Therefore, we can
%    set it equivalent to the macro that ends the theorem set (in our
%    example, |\endLem|).  However, if some day theorem styles exist
%    that do change |\@endtheorem|, we would have to use the
%    commented-out line instead.
%    \begin{macrocode}
    \global \expandafter \let \csname end#1\endcsname \@endtheorem
%   \global\@namedef{end#1}{\@endtheorem}%
%    \end{macrocode}
%    With these commands all the required definitions are employed,
%    unless the test |\@ifdefinable| has failed.  Therefore, we
%    end the second argument of this macro and with it the definition
%    of |\@xnthm|.
%    \begin{macrocode}
   }}
%    \end{macrocode}
% \end{macro}
%
% \begin{macro}{\@ynthm}
%    The definition of |\@ynthm| is completely analogous. In
%    this case the new counter that is defined is not reset within
%    another counter; thus the definition of |\the...|
%    is simplified:
%    \begin{macrocode}
\gdef\@ynthm#1#2{\expandafter\@ifdefinable\csname #1\endcsname
   {\@definecounter{#1}%
    \expandafter\xdef\csname the#1\endcsname{\@thmcounter{#1}}%
%    \end{macrocode}
%    The rest of the definition corresponds literally to that of
%    |\@xnthm|:
%    \begin{macrocode}
    \def\@tempa{\global\@namedef{#1}}\expandafter \@tempa
     \expandafter{\csname th@\the \theorem@style \expandafter
     \endcsname \the\theorem@bodyfont \@thm{#1}{#2}}%
    \global \expandafter \let \csname end#1\endcsname \@endtheorem}}
%    \end{macrocode}
% \end{macro}
%
% \begin{macro}{\@othm}
%    The definition of |\@othm| does not contain anything new.
%    \begin{macrocode}
\gdef\@othm#1[#2]#3{%
%    \end{macrocode}
% \changes{v2.2b}{95/10/07}{Build in check for counter defined}
%    We do not define a new counter but instead use one that has
%    already been defined.  Thus the only definition we need is that
%    of this pseudo-counter (i.e.\ |\the|\meta{env.\ name}).
%    First we check if |#2| corresponds to a known counter name.
%    \begin{macrocode}
  \expandafter\ifx\csname c@#2\endcsname\relax
   \@nocounterr{#2}%
  \else
   \expandafter\@ifdefinable\csname #1\endcsname
   {\expandafter \xdef \csname the#1\endcsname
     {\expandafter \noexpand \csname the#2\endcsname}%
%    \end{macrocode}
%    All other parts of the definition can be adopted from
%    |\@xnthm|. We have to remember, though, that in this case
%    the name of the current counter and the theorem label have moved
%    to the second and third arguments.
%    \begin{macrocode}
    \def\@tempa{\global\@namedef{#1}}\expandafter \@tempa
     \expandafter{\csname th@\the \theorem@style \expandafter
     \endcsname \the\theorem@bodyfont \@thm{#2}{#3}}%
    \global \expandafter \let \csname end#1\endcsname \@endtheorem}%
  \fi}
%    \end{macrocode}
% \end{macro}
%
% \subsection{Macros that are employed in a {\envfont theorem}
%    environment}
%
% \begin{macro}{\@thm}
%    The macro |\@thm| has to increase the current counter.  Then,
%    depending on whether the environment has (or does not have) an
%    optional argument, it has to branch into either |\@begintheorem|
%    or |\@opargtheorem|.
%    \begin{macrocode}
\gdef\@thm#1#2{\refstepcounter{#1}%
%    \end{macrocode}
%    Now we start a {\envfont trivlist} environment, and give
%    |\@topsep| and |\@topsepadd| the values of the \textsf{skip}
%    registers |\theorempreskipamount| and |\theorempostskipamount|.
%    The value in |\@topsep| is the vertical space that is inserted by
%    the first (and only) |\item| in our |\trivlist| whilst
%    |\@topsepadd| is inserted by |\@endparenv| at the end of that
%    {\envfont trivlist} environment.  By using these registers, we
%    obtain the desired space around a {\envfont theorem} environment.
%    \begin{macrocode}
   \trivlist
   \@topsep \theorempreskipamount               % used by first \item
   \@topsepadd \theorempostskipamount           % used by \@endparenv
%    \end{macrocode}
%    Now we have to test whether an optional argument has been given.
% \changes{v2.0d}{89/07/18}{Forgotten percent added.}
%    \begin{macrocode}
   \@ifnextchar [%
%    \end{macrocode}
%    If there is an optional argument, we will call |\@ythm|, and
%    move the arguments read back into the input stream.
%    \begin{macrocode}
   {\@ythm{#1}{#2}}%
%    \end{macrocode}
%    If not, we call |\@begintheorem|. Its first argument is the
%    name of the theorem set (hence the second argument of
%    |\@thm|). Its second argument is the macro that produces the
%    current number.
%    \begin{macrocode}
   {\@begintheorem{#2}{\csname the#1\endcsname}\ignorespaces}}
%    \end{macrocode}
% \end{macro}
%
% \begin{macro}{\@xthm}
% \begin{macro}{\@ythm}
%    Both these macros were originally called by |\@thm|.  We do
%    not need |\@xthm| anymore, hence we reset it to
%    |\relax|. The definition of |\@ythm| has not changed at
%    all from its definition in \LaTeX. In order to make the macros
%    easier to understand, we will
%    nevertheless present it (commented out).
%    \begin{macrocode}
\global\let\@xthm\relax
% \def\@ythm#1#2[#3]{\@opargbegintheorem{#2}{\csname
%           the#1\endcsname}{#3}\ignorespaces}
%</package>
%    \end{macrocode}
%    The primitive |\ignorespaces| in |\@ythm| and
%    |\@thm| is needed to remove the spaces between the
%    |\begin{...}| and the actual text.
% \end{macro}
% \end{macro}
%
%
% \subsection{Definition of the theorem substyles}
%
% As already pointed out, the theorem substyles, defined below, are
% only loaded when necessary.  Note that all these substyles, except
% \texttt{plain}, have |\slshape| as the default body font.
%
% \subsubsection{The \texttt{plain} style}
%
% \label{dok:th@plain} As the following macros use |@|, we have
% to locally set the |\catcode| of this symbol to ``letter''.
% This happens within a group, so that we do not have to worry about
% which |\catcode| that symbol had before.
%    \begin{macrocode}
%<*thp>
\begingroup \makeatletter
%    \end{macrocode}
% Since we are now within a group, we must make all definitions
% globally.  First we make sure that \texttt{theorem.sty} is loaded.
% This will allow us to use this file as a document style option
% without having to call \texttt{theorem} itself as an option.  At the
% same time, we assure that at least version 2 is loaded, since
% |\theorem@style| was not defined in earlier versions.
%    \begin{macrocode}
\@ifundefined{theorem@style}{\input{theorem.sty}}{}
\ProvidesFile{thp.sty}
             [\FMithmInfo]
%    \end{macrocode}
%
%
% \begin{macro}{\th@plain}
% \begin{macro}{\@begintheorem}
% \begin{macro}{\@opargbegintheorem}
%    |\theoremstyle{plain}| corresponds to the original
%    definition, except that the distances to the surrounding text are
%    determined by the parameters |\theorempreskipamount| and
%    |\theorempostskipamount|.
%    First we set the default body font.
%    \begin{macrocode}
\gdef\th@plain{\normalfont\itshape
%    \end{macrocode}
%    Then we define |\@begintheorem| and
%    |\@opargbegintheorem|. These two macros define how the
%    header of a theorem is typeset.  |\@opargbegintheorem| will
%    be called if a \textsf{theorem} environment with an optional
%    argument is encountered; otherwise, the header is constructed by
%    calling |\@begintheorem|.  If one of these macros is
%    executed, we are within a \textsf{trivlist} environment started by
%    |\@thm|.  So the theorem header is produced with an
%    |\item| command.
%
%    Instead of specifying the header font directly, all standard
%    theorem styles use the |\theorem@headerfont| macro to allow
%    customization.  The extra space (|\labelsep|) is necessary
%    because of problems in the \textsf{trivlist} environment.
%    \begin{macrocode}
  \def\@begintheorem##1##2{%
        \item[\hskip\labelsep \theorem@headerfont ##1\ ##2]}%
%    \end{macrocode}
%    The definition of |\@opargbegintheorem| is completely
%    analogous. The only difference is the fact that there exists a
%    third argument (which is the optional parameter of the
%    environment and contains additional information about the
%    theorem).  Customarily we enclose it in parentheses.
%    \begin{macrocode}
\def\@opargbegintheorem##1##2##3{%
   \item[\hskip\labelsep \theorem@headerfont ##1\ ##2\ (##3)]}}
%    \end{macrocode}
% \end{macro}
% \end{macro}
% \end{macro}
%
% \noindent
% We conclude with an |\endgroup| to restore the
% |\catcode| of |@|.
%    \begin{macrocode}
\endgroup
%</thp>
%    \end{macrocode}
%
%
% \subsubsection{The \texttt{break} style}
%
% This style option is stored in the file |thb.sty|.
% For the next two lines see the documentation for |\th@plain| on
% page~\pageref{dok:th@plain}.
%    \begin{macrocode}
%<*thb>
\begingroup \makeatletter
\@ifundefined{theorem@style}{\input{theorem.sty}}{}
\ProvidesFile{thb.sty}
             [\FMithmInfo]
%    \end{macrocode}
%
% \begin{macro}{\th@break}
%    |\theoremstyle{break}| produces a line break after the name
%    of the theorem.  The font is |\slshape|.  Hence, we define
%    |\th@break| as follows:
%    \begin{macrocode}
\gdef\th@break{\normalfont\slshape
  \def\@begintheorem##1##2{\item[%
%    \end{macrocode}
%    We run into the following problem: it is not possible to create
%    the header with |\item[|\meta{title}|]| and then start
%    a new line by, for example, |\mbox{}\\|. Such a definition
%    will fail whenever a list environment follows immediately.  With
%    the above construction, the |\mbox{}| causes the switch
%    |@inlabel| (cf.\ definition of |\list| and
%    |\trivlist| in \cite{latex:source}) to be set to \texttt{false}
%    and so the following list will insert additional vertical
%    space (|\topskip|). This is quite annoying. Therefore, we
%    create the line break within the |\item|. In order to ensure
%    that the text will begin at the proper position in the following
%    line, we simply pretend that the label does not take any
%    room.\footnote{This will lead to problems whenever very high
%    symbols occurring in the line tower into the
%    heading. So, something else has to be done here sometime.}
%    \begin{macrocode}
       \rlap{\vbox{\hbox{\hskip \labelsep\theorem@headerfont ##1\ ##2}%
                   \hbox{\strut}}}]}%
%    \end{macrocode}
%    Again, the definition of |\@opargbegintheorem| is
%    completely analogous.
%    \begin{macrocode}
\def\@opargbegintheorem##1##2##3{%
  \item[\rlap{\vbox{\hbox{\hskip \labelsep \theorem@headerfont
                     ##1\ ##2\ (##3)}%
                    \hbox{\strut}}}]}}
\endgroup
%</thb>
%    \end{macrocode}
% \end{macro}
%
%
% \subsubsection{The \texttt{changebreak} style}
%
%    \begin{macrocode}
% This style option is stored in the file |thcb.sty|.
%    \begin{macrocode}
%<*thcb>
\begingroup \makeatletter
\@ifundefined{theorem@style}{\input{theorem.sty}}{}
\ProvidesFile{thcb.sty}
             [\FMithmInfo]
%    \end{macrocode}
%
% \begin{macro}{\th@changebreak}
%    The \texttt{change-break} theorem style is like \texttt{break}
%    but with interchange of theorem name and theorem number.  Thus we
%    define |\th@changebreak| as follows:
%    \begin{macrocode}
\gdef\th@changebreak{\normalfont\slshape
  \def\@begintheorem##1##2{\item
       [\rlap{\vbox{\hbox{\hskip\labelsep \theorem@headerfont ##2\ ##1}%
                    \hbox{\strut}}}]}%
\def\@opargbegintheorem##1##2##3{%
  \item[\rlap{\vbox{\hbox{\hskip\labelsep \theorem@headerfont
                     ##2\ ##1\ (##3)}%
                    \hbox{\strut}}}]}}
\endgroup
%</thcb>
%    \end{macrocode}
% \end{macro}
%
%
% \subsubsection{The \texttt{change} style}
%
% This style option is stored in the file |thc.sty|.
%    \begin{macrocode}
%<*thc>
\begingroup \makeatletter
\@ifundefined{theorem@style}{\input{theorem.sty}}{}
\ProvidesFile{thc.sty}
             [\FMithmInfo]
%    \end{macrocode}
%
% \begin{macro}{\th@change}
%    The \texttt{change} theorem style corresponds to the
%    \texttt{change break} style without a linebreak after the header.
%    To say it in another way, it's the same as the \texttt{plain}
%    style but with number and name interchanged and |\slshape| as the
%    default font.
%    \begin{macrocode}
\gdef\th@change{\normalfont\slshape
  \def\@begintheorem##1##2{\item
       [\hskip\labelsep \theorem@headerfont ##2\ ##1]}%
\def\@opargbegintheorem##1##2##3{%
  \item[\hskip\labelsep \theorem@headerfont ##2\ ##1\ (##3)]}}
\endgroup
%</thc>
%    \end{macrocode}
% \end{macro}
%
%
% \subsubsection{The \texttt{marginbreak} style}
%
% This style option is the one used most often at Mainz.  It is saved
% in the file |thmb.sty|.
%    \begin{macrocode}
%<*thmb>
\begingroup \makeatletter
\@ifundefined{theorem@style}{\input{theorem.sty}}{}
\ProvidesFile{thmb.sty}
             [\FMithmInfo]
%    \end{macrocode}
%
% \begin{macro}{\th@marginbreak}
%    The \texttt{margin break} style is nearly the same as the
%    \texttt{change break} style. The only difference is the placement
%    of the theorem number.  We use |\llap| to place it in the left
%    margin.
%
%    In this style |\labelsep| denotes the separation between the
%    number and the text.
%    \begin{macrocode}
\gdef\th@marginbreak{\normalfont\slshape
  \def\@begintheorem##1##2{\item
       [\rlap{\vbox{\theorem@headerfont
                    \hbox{\llap{##2}\hskip\labelsep ##1}%
                    \hbox{\strut}}}]}%
\def\@opargbegintheorem##1##2##3{%
  \item[\rlap{\vbox{\theorem@headerfont
                    \hbox{\llap{##2}\hskip\labelsep ##1\ (##3)}%
                    \hbox{\strut}}}]}}
\endgroup
%</thmb>
%    \end{macrocode}
% \changes{v2.1c}{90/02/17}{`endgroup outside macrocode env.}
% \end{macro}
%
%
%
% \subsubsection{The \texttt{margin} style}
%
% This style option is stored in the file |thm.sty|.
%    \begin{macrocode}
%<*thm>
\begingroup \makeatletter
\@ifundefined{theorem@style}{\input{theorem.sty}}{}
\ProvidesFile{thm.sty}
             [\FMithmInfo]
%    \end{macrocode}
%
% \begin{macro}{\th@margin}
%    Again this is only a variant of the theorem styles described
%    above without any new ideas.
%    \begin{macrocode}
\gdef\th@margin{\normalfont\slshape
  \def\@begintheorem##1##2{\item
       [\theorem@headerfont \llap{##2}\hskip\labelsep ##1]}%
\def\@opargbegintheorem##1##2##3{%
  \item[\theorem@headerfont \llap{##2}\hskip\labelsep ##1\ (##3)]}}
\endgroup
%</thm>
%    \end{macrocode}
% \end{macro}
%
%
%
%
% \subsection{Final Definitions}
%
% \begin{macro}{\theorempreskipamount}
% \begin{macro}{\theorempostskipamount}
%    The \textsf{skip} parameters that regulate the vertical empty space
%    before and after the {\envfont theorem} environment have
%    to be allocated as well.
% \changes{v2.0f}{89/07/20}{Unnecessary `global' removed.}
%    \begin{macrocode}
%<*package>
\newskip\theorempreskipamount
\newskip\theorempostskipamount
%    \end{macrocode}
%    Since we have used the same values for all theorem sets, we now can
%    assign them.
%    \begin{macrocode}
\global\setlength\theorempreskipamount{12pt plus 5pt minus 3pt}
\global\setlength\theorempostskipamount{8pt plus 3pt minus 1.5pt}
%    \end{macrocode}
% \end{macro}
% \end{macro}
%
%
% \begin{macro}{\@endtheorem}
%    The same holds for the macro |\@endtheorem|, which ends a
%    {\envfont theorem} environment. Since it is the same for all
%    theorem sets, it is removed from the macros |\th@|\meta{style}.
%    It simply ends the {\envfont trivlist} environment, which was
%    begun in |\@thm|.
%    \begin{macrocode}
\global\let\@endtheorem=\endtrivlist
%    \end{macrocode}
% \end{macro}
%
%
%
% \begin{macro}{\@preamblecmds}
%    All macros defined above are to be used only in the preamble.
%    Therefore, we insert them in |\@preamblecmds| which will disable
%    them at begin document. This is done by the internal \LaTeXe{}
%    command |\@onlypreamble|.
% \changes{v2.0g}{89/09/19}{`theoremheaderfont and `theorembodyfont
%    added.}
% \changes{v2.2a}{94/02/03}{Use \LaTeXe{} interface}
%    \begin{macrocode}
\@onlypreamble\@xnthm
\@onlypreamble\@ynthm
\@onlypreamble\@othm
\@onlypreamble\newtheorem
\@onlypreamble\theoremstyle
\@onlypreamble\theorembodyfont
\@onlypreamble\theoremheaderfont
%    \end{macrocode}
% \end{macro}
%
% Finally we declare the \texttt{plain} theorem style to be the default.
% \changes{v2.1b}{90/06/04}{Fixed missing `item by declaring default
%    style.}
%    \begin{macrocode}
\theoremstyle{plain}
%</package>
%    \end{macrocode}
%
% \Finale

\endinput
}{}
\ProvidesFile{thcb.sty}
             [\FMithmInfo]
%    \end{macrocode}
%
% \begin{macro}{\th@changebreak}
%    The \texttt{change-break} theorem style is like \texttt{break}
%    but with interchange of theorem name and theorem number.  Thus we
%    define |\th@changebreak| as follows:
%    \begin{macrocode}
\gdef\th@changebreak{\normalfont\slshape
  \def\@begintheorem##1##2{\item
       [\rlap{\vbox{\hbox{\hskip\labelsep \theorem@headerfont ##2\ ##1}%
                    \hbox{\strut}}}]}%
\def\@opargbegintheorem##1##2##3{%
  \item[\rlap{\vbox{\hbox{\hskip\labelsep \theorem@headerfont
                     ##2\ ##1\ (##3)}%
                    \hbox{\strut}}}]}}
\endgroup
%</thcb>
%    \end{macrocode}
% \end{macro}
%
%
% \subsubsection{The \texttt{change} style}
%
% This style option is stored in the file |thc.sty|.
%    \begin{macrocode}
%<*thc>
\begingroup \makeatletter
\@ifundefined{theorem@style}{% \iffalse meta-comment
%
% Copyright (C) 1993-2024
%
% The LaTeX Project and any individual authors listed elsewhere
% in this file.
%
% This file is part of the Standard LaTeX `Tools Bundle'.
% -------------------------------------------------------
%
% It may be distributed and/or modified under the
% conditions of the LaTeX Project Public License, either version 1.3c
% of this license or (at your option) any later version.
% The latest version of this license is in
%    https://www.latex-project.org/lppl.txt
% and version 1.3c or later is part of all distributions of LaTeX
% version 2005/12/01 or later.
%
% The list of all files belonging to the LaTeX `Tools Bundle' is
% given in the file `manifest.txt'.
%
% \fi
%
%%
%
%
\def\FMithmInfo{2023/07/05 v2.2c Theorem extension package (FMi)}
%
% \ProvidesFile{theorem.dtx}[\FMithmInfo]
%\iffalse   % this is a METACOMMENT !
%%
%% Package `theorem' to use with LaTeX2e
%% Copyright (C) 1989-1995 Frank Mittelbach, all rights reserved.
%<+package>\@ifundefined{theorem@style}{}{\endinput}
%<+package>\NeedsTeXFormat{LaTeX2e}[1995/06/01]
%<+package>\ProvidesPackage{theorem}
%<+package>         [\FMithmInfo]
%
% \fi
%
% \changes{v1.0}{87/05/07}{First draft.}
% \changes{v1.0a}{87/05/07}{`thistheoremstyle' added.}
% \changes{v1.1a}{87/07/29}{New concept: redefine the internal
%                          \protect\LaTeX{} commands `@xnthm' etc.}
% \changes{v1.1b}{87/07/31}{`theoremfont' used to specify font.}
% \changes{v1.2a}{88/01/08}{`@endtheorem' now globally defined.}
% \changes{v1.2b}{88/01/11}{`newtheorem' only allowed in the preamble.}
% \changes{v1.3a}{88/02/14}{`theoremfont' now a toks register.}
%
% \changes{v2.0a}{88/11/16}{New implementation and new documentation.}
% \changes{v2.0a}{88/11/16}{No layout changes but internal modifications
%                           old substyles won't work.}
% \changes{v2.0b}{89/01/30}{`global' used for all definitions.}
% \changes{v2.0c}{89/06/15}{Documentation translated to English.
%                            Thanks to LLL and Nina}
% \changes{v2.1a}{90/02/15}{Support for new font selection.}
% \changes{v2.2a}{94/02/03}{Cleanup for LaTeX2e}
%
% \DoNotIndex{\@endparenv,\@ifdefinable,\@ifnextchar,\@ifundefined}
% \DoNotIndex{\@namedef,\@nameuse,\@spaces,\@tempa,\@warning}
% \DoNotIndex{\ }
% \DoNotIndex{\begingroup,\bfseries}
% \DoNotIndex{\csname}
% \DoNotIndex{\def,\do,\docdate}
% \DoNotIndex{\endcsname,\endgroup,\endinput,\endtrivlist,\expandafter}
% \DoNotIndex{\filedate,\fileversion}
% \DoNotIndex{\gdef,\global}
% \DoNotIndex{\hbox,\hskip}
% \DoNotIndex{\ignorespaces,\input,\itshape,\item}
% \DoNotIndex{\labelsep,\let,\llap}
% \DoNotIndex{\makeatletter}
% \DoNotIndex{\noexpand}
% \DoNotIndex{\refstepcounter,\relax,\rlap}
% \DoNotIndex{\setlength,\slshape,\space,\string,\strut}
% \DoNotIndex{\the,\trivlist,\typeout}
% \DoNotIndex{\vbox}
% \DoNotIndex{\xdef}
%
% \def\deflabel#1{\ttfamily #1\hfill}
% \def\deflist#1{\begin{list}{}{\settowidth\labelwidth{\ttfamily #1}%
%                               \setlength\leftmargin\labelwidth
%                               \addtolength\leftmargin\labelsep
%                               \let\makelabel\deflabel}}
% \def\enddeflist{\end{list}}
% \def\envfont{\normalfont\ttfamily}
%
% \MakeShortVerb{\|}
%
% \newcommand{\TUB}{{\sl TUGboat\/}}
%
% \GetFileInfo{theorem.dtx}
% \title{An Extension of the
%        \LaTeX{} theorem environment\thanks{This file
%        has version number \fileversion, last
%        revised \filedate.}}
% \author{Frank Mittelbach}
%
% \MaintainedByLaTeXTeam{tools}
% \maketitle
%
% \begin{abstract}
%    The macros described in this paper yield an extension of the
%    \LaTeX{} theorem mechanism. It is designed to satisfy the
%    different requirements of various journals. Thus, the
%    layout of the ``theorems'' can be manipulated by determining a
%    ``style''. This article describes not only the use, but
%    also the definition, of the necessary macros.
% \end{abstract}
%
% \section*{Preface to version 2.2}
%
% For \LaTeXe{} this package did not need any fundamental changes. I
% only modified the messages generated so that theorem layout styles
% will show up with the |\listfiles| command and cleaned the section
% on the New Font Selection Scheme since this is now included in
% \LaTeX.
%
% \section*{Preface to version 2.1}
%
% This version is identical to 2.0g described in \TUB\ 10\#3 except
% for some internal defaults which are now set depending on the used
% font selection scheme.
%
% This was done to avoid unpleasant surprises if the new font selection
% scheme is in force. For further details see
% section~\ref{sec:newfonts} and \cite{bk:GMS94}.
%
% \section{Introduction}
%
% For our purposes here, ``theorems'' are labelled enunciations,
% often set off from the main text by extra space and a font change.
% Theorems, corollaries, conjectures, definitions, and remarks
% are all instances of ``theorems''.  The ``header'' of these structures
% is composed of a label (such as \textsc{Theorem} or \textsc{Remark})
% and a number which serializes an item in the sequence of items
% with the same label.
%
% Shortly after the introduction of \LaTeX{} at the Fachbereich
% Mathematik in Mainz, the desire to manipulate the layout of
% ``theorems'' arose.  In Mainz, the following two conventions came
% into general use:
% \begin{enumerate}
%   \item The number of the theorem is shown in the margin.
%   \item There is a line break at
%         the end of the theorem header.
% \end{enumerate}
%
% Additionally, some journals require different formats which depend on
% the ``sort of theorem'': e.g.\ often remarks and definitions are set
% in |\upshape|, while |\itshape| is employed for main theorems.
%
% Confronted with these requirements, a theorem environment was
% developed in Mainz which allows separate determination of the
% layout of the ``theorems sets'', comparable to |\pagestyle|.
%
%
% \section{The user interface}
% \subsection{Defining new theorem sets}
%
% \DescribeMacro\newtheorem
% As in the original \LaTeX{} version, the command |\newtheorem|
% defines a new ``theorem set'' or ``theorem-like structure''.
% Two required arguments name the new environment and give the text
% to be typeset with each instance of the new ``set'', while
% an optional argument determines how
% the ``set'' is enumerated:
% \begin{description}
%    \item[\ttfamily \bslash newtheorem\{foo\}\{bar\}]
%       The theorem set {\envfont foo} (whose name is \texttt{bar})
%       uses its own counter.
%    \item[\ttfamily \bslash newtheorem\{foo2\}{[foo]}\{bar2\}]
%       The theorem set {\envfont foo2} (printed name \texttt{bar2})
%       uses the same counter as the theorem set \texttt{foo}.
%    \item[\ttfamily \bslash newtheorem\{foo3\}\{bar3\}{[section]}]
%       The theorem set {\envfont foo3} (printed name \texttt{bar3}) is
%       enumerated within the counter \texttt{section}, i.e.\ with every
%       new |\section| the enumeration begins again with 1, and
%       the enumeration is composed from the section-number and the
%       theorem counter itself.
% \end{description}
%
% \DescribeMacro\theoremstyle
% Additionally, the command |\theoremstyle| can define the layout
% of various, or all, theorem sets.  It should be noted that any
% theorem set defined by |\newtheorem| is typeset in the
% |\theoremstyle| that is current at the time of the definition.
% Thus, the following
% \begin{verbatim}
%    \theoremstyle{break}        \newtheorem{Cor}{Corollary}
%    \theoremstyle{plain}        \newtheorem{Exa}{Example}[section]
%\end{verbatim}                             ^^A No blank at line start
% leads to the result that the set {\envfont Cor} is formatted in the
% style \texttt{break}, while the set {\envfont Exa} and all the
% following ones are formatted in the style \texttt{plain}, unless
% another |\theoremstyle| follows.  Since the definitions installed by
% |\newtheorem| are global, one also can limit |\theoremstyle| locally
% by grouping braces.
%
% \DescribeMacro\theorembodyfont
% The choice of the font for the theorem body is completely
% independent of the chosen |\theoremstyle|; this has proven to
% be very advantageous.  For example,
% \begin{verbatim}
%   {\theorembodyfont{\upshape}          \newtheorem{Rem}{Remark}}
%\end{verbatim}
% defines a theorem set {\envfont Rem}, which will be set in
% |\upshape| in the current layout (which in our example is
% \texttt{plain}).  As with |\theoremstyle|, the |\theorembodyfont|
% chosen is that current at the time of |\newtheorem|.  If
% |\theorembodyfont| is not specified or one defines
% |\theorembodyfont{}|, then the font used will be that defined by the
% |\theoremstyle|.
%
% \DescribeMacro\theoremheaderfont It is also possible to customize
% the font used for the theorem headers.  This is, however, a global
% declaration, and therefore there should be at most one
% |\theoremheaderfont| declaration in the preamble.\footnote{If it is
% actually necessary to have different header fonts, one has to define
% new theorem styles (substituting the desired font) or specify the
% information directly in the \texttt{\bslash newtheorem} declaration
% (the unclean variant).}
%
% \DescribeMacro\theorempreskipamount
% \DescribeMacro\theorempostskipamount Two additional parameters
% affect the vertical space around the theorem environments:
% |\theorempreskipamount| and |\theorempostskipamount| define,
% respectively, the spacing before and after such an environment.
% These parameters apply for all theorem sets and can be manipulated
% with the ordinary length macros.  They are rubber lengths
% (`\textsf{skips}'), and therefore can contain \texttt{plus} and
% \texttt{minus} parts.
%
% Since the definition of theorem sets should---most sensibly---be
% placed in the preamble, we only allow installation there.
% It is therefore possible to release the memory used here after
% |\begin{document}|, in order to make room for other
% applications.
%
%
% \subsection{Existing theorem styles}
%
% The following theorem styles exist to date:
% \begin{deflist}{marginbreak}
%    \item[plain]
%       This theorem style emulates the original \LaTeX{} definition,
%       except that additionally the parameters
%       |\theorem...skipamount| are used.
%    \item[break]
%       In this style, the theorem header is followed by a line
%       break.
%    \item[marginbreak]
%       The theorem number is set in the margin, and there is a line
%       break as in \texttt{break}.
%    \item[changebreak]
%       Like \texttt{break}, but with header number and text
%       interchanged.
%    \item[change]
%       Header number and text are interchanged, without a line break.
%    \item[margin]
%       The number is set in the left margin, without a line break.
% \end{deflist}
% All styles (except \texttt{plain}) select |\slshape| as the default
% |\theorembodyfont|.
%
%
% \subsection{Examples}
%
% Given the above theorem sets {\envfont Cor}, {\envfont Exa}
% and {\envfont Rem},
% suppose that the preamble also contains the declarations:
% \begin{verbatim}
%    \theoremstyle{marginbreak}   \newtheorem{Lem}[Cor]{Lemma}
%    \theoremstyle{change}
%    \theorembodyfont{\itshape}        \newtheorem{Def}[Cor]{Definition}
%
%    \theoremheaderfont{\scshape}
%\end{verbatim}
% Then the following are some typical examples of the typeset output
% resulting from their use.
%
% \begin{Cor}
%    This is a sentence typeset in the theorem environment {\envfont
%    Cor}.
% \end{Cor}
%
% \begin{Exa}
%    This is a sentence typeset in the theorem environment {\envfont
%    Exa}.
% \end{Exa}
%
% \begin{Rem}
%    This is a sentence typeset in the theorem environment {\envfont
%    Rem}.
% \end{Rem}
%
% \begin{Lem}[Ben User]
%    This is a sentence typeset in the theorem environment {\envfont
%    Lem}.
% \end{Lem}
%
% \begin{Def}[Very impressive Definition]
%    This is a sentence typeset in the theorem environment {\envfont
%    Def}.
% \end{Def}
%
% The last two examples show the effect of the optional argument to
% a theorem environment (it is the text typeset in parentheses).
%
%
% \section{Special Considerations}
% \label{sec:newfonts}
%
% Theoremheader and body are implemented as a unit. This means that
% the |\theoremheaderfont|  will inherit characteristics of the
% |\theorembodyfont| in \LaTeXe. Thus, if for example
% |\theorembodyfont| is |\itshape| and |\theoremheaderfont|
% is |\bfseries| the font selected for the header will have the
% characteristics `bold extended italic'. If this is not desired one
% should set the |\theoremheaderfont| to something like
% \begin{verbatim}
%      \theoremheaderfont{\normalfont\bfseries}
%\end{verbatim}
% i.e.\ supplying all necessary font information explicitly.
%
%
%
% \section{Acknowledgements}
%
% The publication of this set of macros was only possible with the
% help of Christina Busse (translating the manuscript into English),
% Joachim Pense (playing the r\^ole of typist), Chris Rowley
% (looking everything over) and many others providing useful
% suggestions.
%
% \MaybeStop{%
%    \begin{thebibliography}{9}
%       \bibitem{bk:GMS94} \textsc{M.~Goossens}, \textsc{F.~Mittelbach}
%         and \textsc{A.~Samarin}.
%         \newblock The \LaTeX{} Companion.
%         \newblock
%         Addison-Wesley, Reading, Massachusetts, 1994.
%       \bibitem{latex:source} \textsc{Lamport, Leslie.} \newblock
%                               \texttt{latex.tex}, version 2.09, date
%                               Feb. 1990.
%    \end{thebibliography}
%    \PrintIndex
%    \PrintChanges}
%
%
% \section{The documentation driver file}
%
% The next bit of code contains the documentation driver file for
% \TeX{}, i.e., the file that will produce the documentation you are
% currently reading. It will be extracted from this file by the
% \texttt{docstrip} program. Since it is the first code in the file
% one can alternatively process this file directly with \LaTeXe{} to
% obtain the documentation.
%
%    \begin{macrocode}
%<*driver>
\documentclass{ltxdoc}

\usepackage{theorem}

%<+driver>% The next few lines define theorem sets which are used
%<+driver>% in the example section of the documentation.

\theoremstyle{break}       \newtheorem{Cor}{Corollary}
\theoremstyle{plain}       \newtheorem{Exa}{Example}[section]
{\theorembodyfont{\upshape}\newtheorem{Rem}{Remark}}
\theoremstyle{marginbreak} \newtheorem{Lem}[Cor]{Lemma}
\theoremstyle{change}
\theorembodyfont{\itshape} \newtheorem{Def}[Cor]{Definition}

\theoremheaderfont{\scshape}

\RecordChanges

\begin{document}
   \DocInput{theorem.dtx}
\end{document}
%</driver>
%    \end{macrocode}
%
% \section{Definition of the Macros}
%
% If the file has been loaded before, we abort immediately. If not the
% package announces itself (this is actually done at the very top of
% the file---the way it is done isn't good style so don't copy it).
% \changes{v2.0e}{89/07/19}{Spaces removed from `typeout'.}
% \changes{v2.2b}{95/11/19}{Announce moved to top of file}
%    \begin{macrocode}
%<*package>
%\@ifundefined{theorem@style}{}{\endinput}
%\def\FMithmInfo{1995/11/19 v2.2b Theorem extension package (FMi)}
%\ProvidesPackage{theorem}[\FMithmInfo]
%    \end{macrocode}
%
%
% \subsection{Definition of theorem styles and fonts}
%
% All the definitions in this file are done globally to allow
% inputting this file inside a group.
%
% \begin{macro}{\theoremstyle}
%    Before a theorem style can be installed, the chosen style must be
%    known. For that reason, we must test to see that
%    |\th@|\meta{style} is known or, more precisely, that it is
%    different from |\relax|. If the style is not known then
%    |\th@plain| is used.
%    \begin{macrocode}
\gdef\theoremstyle#1{%
   \@ifundefined{th@#1}{\@warning
          {Unknown theoremstyle `#1'. Using `plain'}%
          \theorem@style{plain}}%
%    \end{macrocode}
%    We save the theorem style to be used in the token register
%    |\theorem@style|.
%    \begin{macrocode}
      {\theorem@style{#1}}%
%    \end{macrocode}
%    Now we ``evaluate'' the theorem style: this means, we call the
%    macro |\th@|\meta{style} which will activate the relevant
%    definitions which are contained in a separate file.  This is done
%    in a group to suppress changes to the current font.  This could
%    otherwise pose problems together with the new font selection
%    scheme\footnote{When I printed the original article using the new
%    font selection scheme I ended with a document with slanted
%    typefaces (text headings and all) simply because one of the
%    theorem styles used \texttt{\bslash sl} at toplevel.} if the
%    |\th@|\meta{style} is evaluated a second time.
% \changes{v2.1a}{90/02/15}{Group added around `th@..}
% \changes{v2.1a}{90/02/15}{Reseting of `..@begintheorem removed.}
%    \begin{macrocode}
      \begingroup
        \csname th@\the\theorem@style \endcsname
      \endgroup}
%    \end{macrocode}
% \end{macro}
%
%
% \begin{macro}{\@begintheorem}
% \changes{v2.1a}{90/02/15}{Relaxed at toplevel.}
% \begin{macro}{\@opargbegintheorem}
% \changes{v2.1a}{90/02/15}{Relaxed at toplevel.}
%    We reset |\@begintheorem| and |\@opargbegintheorem| to
%    |\relax| since these commands are no longer necessary at
%    toplevel. This will save a few tokens.
%    \begin{macrocode}
\global\let\@begintheorem\relax
\global\let\@opargbegintheorem\relax
%    \end{macrocode}
% \end{macro}
% \end{macro}
%
% \begin{macro}{\theorem@style}
%    Obviously the \textsf{token} register used above has to be
%    allocated. To assure the utmost compatibility with the
%    original \LaTeX{} definition, we set the default theorem style to
%    \texttt{plain}, which implements the usual \LaTeX{} convention.
% \changes{v2.0f}{89/07/20}{Unnecessary `global' removed.}
%    \begin{macrocode}
\newtoks\theorem@style
\global\theorem@style{plain}
%    \end{macrocode}
% \end{macro}
%
% \begin{macro}{\theorembodyfont}
% \begin{macro}{\theorem@bodyfont}
%    If the |\theorembodyfont| is set by the user then it should not
%    interact with the default font set in the theorem style. When the
%    new font selection is in force this may happen if, for example,
%    the default is |\itshape| and the new |\theorembodyfont| is
%    |\sffamily|. So we add a |\reset@font| command in front of the
%    user definition.
% \changes{v2.0f}{89/07/20}{Unnecessary `global' removed.}
% \changes{v2.0g}{89/09/19}{`theoremfont to `theorembodyfont.}
% \changes{v2.1c}{92/06/22}{`reset@font added and logic changed.}
% \changes{v2.1d}{92/10/28}{Reset to empty if arg is empty}
%    \begin{macrocode}
\gdef\theorembodyfont#1{%
%    \end{macrocode}
%    We check if the argument supplied is empty and if so put nothing
%    into the |\theorem@bodyfont| token register to allow for
%    |\theorembodyfont{}| as a mean of using the default of the
%    current |\theoremstyle|.
%    \begin{macrocode}
   \def\@tempa{#1}%
   \ifx\@tempa\@empty
    \theorem@bodyfont{}%
   \else
    \theorem@bodyfont{\reset@font#1}%
   \fi
   }
\newtoks\theorem@bodyfont
\global\theorem@bodyfont{}
%    \end{macrocode}
% \end{macro}
% \end{macro}
%
%
% \begin{macro}{\theoremheaderfont}
% \changes{v2.0g}{89/09/19}{Macro added.}
%    The font for the theorem headers is handled differently because
%    this definition applies to all theorem styles.
%    \begin{macrocode}
\gdef\theoremheaderfont#1{\gdef\theorem@headerfont{#1}%
%    \end{macrocode}
%    After using the macro once it is redefined to produce an error
%    message.
%    \begin{macrocode}
       \gdef\theoremheaderfont##1{%
        \typeout{\string\theoremheaderfont\space should be used
                 only once.}}}
%    \end{macrocode}
% \end{macro}
%
% \begin{macro}{\theorem@headerfont}
% \changes{v2.0g}{89/09/19}{Macro added.}
%    To set the |\theorem@headerfont| default we first test if
%    the new fontselection scheme is in force.
% \changes{v2.1a}{90/02/15}{Support for new font selection.}
%    \begin{macrocode}
\ifx\upshape\undefined
%    \end{macrocode}
%    If not we define it to expand into |\bfseries|. We don't use
%    |\let| just in case a following style option redefines this
%    macro.
%    \begin{macrocode}
\gdef\theorem@headerfont{\bfseries}
%    \end{macrocode}
%    Otherwise we reset the current shape before calling |\bfseries|.
%    \begin{macrocode}
\else \gdef\theorem@headerfont{\normalfont\bfseries}\fi
%    \end{macrocode}
% \end{macro}
%
% \begin{macro}{\th@plain}
% \begin{macro}{\th@break}
% \begin{macro}{\th@marginbreak}
% \begin{macro}{\th@changebreak}
% \begin{macro}{\th@change}
% \begin{macro}{\th@margin}
%    The different styles are defined in macros such as
%    |\th@plain|.  Since memory space is precious in
%    ``non-Big-versions'', we have to avoid offering too many unused
%    definitions.  Therefore we define these styles in separate files
%    that can be loaded on demand.  Thus the commands themselves only
%    load these files. We use |\@input@|, a \LaTeXe{} internal command
%    that ensures that the file will be listed with |\listfiles|.
% \changes{v2.2a}{94/02/02}{Use `@input@ to load theorem layout files}
%    \begin{macrocode}
\gdef\th@plain{\@input@{thp.sty}}
\gdef\th@break{\@input@{thb.sty}}
\gdef\th@marginbreak{\@input@{thmb.sty}}
\gdef\th@changebreak{\@input@{thcb.sty}}
\gdef\th@change{\@input@{thc.sty}}
\gdef\th@margin{\@input@{thm.sty}}
%    \end{macrocode}
%    This list will be expanded when new styles become
%    available.  For testing,  just append new theorem substyles
%    as document options.
% \end{macro}
% \end{macro}
% \end{macro}
% \end{macro}
% \end{macro}
% \end{macro}
%
%
%
% \subsection{Definition of a new theorem set}
%
% As already pointed out, a new theorem environment can be defined in
% three different ways:
% \begin{verbatim}
%    \newtheorem{Lem}{Lemma}
%    \newtheorem{Lem}{Lemma}[section]
%    \newtheorem{Lem}[Theorem]{Lemma}
%\end{verbatim}
% The function of the macro |\newtheorem| is to recognize these cases
% and then to branch into one of the three macros |\@ynthm|, |\@xnthm|
% or |\@othm|.  This mechanism is adopted unchanged from
% \cite{latex:source}; the essential point here is that, for example,
% in the second case, the arguments \texttt{Lem}, \texttt{Lemma} and
% \texttt{section} are passed over to the macro |\@xnthm|.
%
% We inspect this case first because the others present fewer problems,
% and thus are easily derived from this one.
%
% \begin{macro}{\@xnthm}
%    For our example arguments, the macro |\@xnthm| must fulfill
%    the following:
%    \begin{itemize}
%       \item
%          Define a new \LaTeX-counter `Lem'
%       \item
%          reset this counter within a |\section|
%       \item
%          define the macro |\theLem|
%       \item
%          define the environment macros |\Lem| and
%          |\endLem| using the current
%          |\theoremstyle| and |\theorem@bodyfont|.
%    \end{itemize}
%    Obviously, all this should happen only if the first argument
%    of |\@xnthm| (i.e.\ \texttt{Lem} in our example) is chosen so
%    as not to conflict with any previously defined commands or
%    environments.  This test is performed by the \LaTeX{} macro
%    |\@ifdefinable|.
%    \begin{macrocode}
\gdef\@xnthm#1#2[#3]{\expandafter\@ifdefinable\csname #1\endcsname
%    \end{macrocode}
%    Therefore, the first argument of |\@ifdefinable| is the
%    expansion (in the example, |\Lem|) of
%    |\csname||#1\endcsname|.  The second argument is
%    executed only if the test has been completed successfully.
%    \begin{macrocode}
   {%
%    \end{macrocode}
%    Now we define the new counter. The names of the \LaTeX{}
%    macros employed should speak for themselves:
% \changes{v2.2b}{1995/10/07}{Use \cs{@newctr}}
% \changes{v2.2c}{1995/11/23}{Use \cs{@newctr} properly.}
%    \begin{macrocode}
    \@definecounter{#1}\@newctr{#1}[#3]%
%    \end{macrocode}
%    Using |\@newctr| will give a proper error message if the counter
%    in |#3| is not defined.
%    In defining `|\theLem|' we must generate the desired macro
%    name by use of |\expandafter| and |\csname|.
%    \begin{macrocode}
    \expandafter\xdef\csname the#1\endcsname
%    \end{macrocode}
%    An |\xdef| is used in order to make the definition global,
%    and to ensure that it contains the replacement texts of
%    |\@thmcountersep| and |\@thmcounter|.\footnote{These
%    two macros can be defined by the document style.  Their default
%    values produce a `\texttt{.}' as separation and an arabic
%    representation of the number.} However, not everything should be
%    expanded. For example, it saves space to use |\thesection|
%    instead of its---at times---lengthy expansion.
%    \begin{macrocode}
      {\expandafter \noexpand \csname the#3\endcsname
       \@thmcountersep \@thmcounter{#1}}%
%    \end{macrocode}
%    Thus with the defaults of \LaTeX{}, |\theLem| would be
%    replaced by the command sequence |\thesection.\arabic{Lem}|.
%
%    We will now look at the definition of the macro which is executed
%    at the beginning of the actual environment (in our example this
%    macro is |\Lem|). It should be noted that we use an
%    ``|\expandafter| trick'' to expand only certain parts of the
%    replacement text at the time of the definition.
%    \begin{macrocode}
    \def\@tempa{\global\@namedef{#1}}%
    \expandafter \@tempa \expandafter{%
%    \end{macrocode}
%    First, the macro that contains the current definitions of
%    |\@begintheorem| and |\@opargtheorem| should be called
%    up.  The name of this macro---as is already known---has the form
%    |\th@|\meta{theorem\ style}; therefore, it must be called by
%    \begin{macrocode}
      \csname th@\the \theorem@style
%    \end{macrocode}
%    In addition the default theorem font should be changeable, i.e.\
%    we have to insert the contents of |\theorem@bodyfont|. For
%    that reason, we expand even further, beyond |\endcsname|,
%    and thus insert the contents of the \textsf{token} register
%    |\theorem@bodyfont| in the replacement text.
%    \begin{macrocode}
            \expandafter \endcsname \the \theorem@bodyfont
%    \end{macrocode}
%    Now it is time to call the macro |\@thm| which takes over
%    the further processing.  It has two arguments: the current
%    counter name (in our example, \texttt{Lem}), and the text of the
%    label (in our example, \texttt{Lemma}).
%    \begin{macrocode}
     \@thm{#1}{#2}}%
%    \end{macrocode}
%    With this, the `sub-definition' is complete.  The macro
%    |\@endtheorem| ends a theorem environment and is, so far, nothing
%    but an |\endtrivlist|.  (Hence it is defined globally, and not
%    within the theorem styles.\footnote{This has to be changed as
%    soon as theorem styles that change \texttt{\bslash @endtheorem}
%    exist.  In such a case, all existing styles must be changed as
%    well since they will have to reset the macro.}) Therefore, we can
%    set it equivalent to the macro that ends the theorem set (in our
%    example, |\endLem|).  However, if some day theorem styles exist
%    that do change |\@endtheorem|, we would have to use the
%    commented-out line instead.
%    \begin{macrocode}
    \global \expandafter \let \csname end#1\endcsname \@endtheorem
%   \global\@namedef{end#1}{\@endtheorem}%
%    \end{macrocode}
%    With these commands all the required definitions are employed,
%    unless the test |\@ifdefinable| has failed.  Therefore, we
%    end the second argument of this macro and with it the definition
%    of |\@xnthm|.
%    \begin{macrocode}
   }}
%    \end{macrocode}
% \end{macro}
%
% \begin{macro}{\@ynthm}
%    The definition of |\@ynthm| is completely analogous. In
%    this case the new counter that is defined is not reset within
%    another counter; thus the definition of |\the...|
%    is simplified:
%    \begin{macrocode}
\gdef\@ynthm#1#2{\expandafter\@ifdefinable\csname #1\endcsname
   {\@definecounter{#1}%
    \expandafter\xdef\csname the#1\endcsname{\@thmcounter{#1}}%
%    \end{macrocode}
%    The rest of the definition corresponds literally to that of
%    |\@xnthm|:
%    \begin{macrocode}
    \def\@tempa{\global\@namedef{#1}}\expandafter \@tempa
     \expandafter{\csname th@\the \theorem@style \expandafter
     \endcsname \the\theorem@bodyfont \@thm{#1}{#2}}%
    \global \expandafter \let \csname end#1\endcsname \@endtheorem}}
%    \end{macrocode}
% \end{macro}
%
% \begin{macro}{\@othm}
%    The definition of |\@othm| does not contain anything new.
%    \begin{macrocode}
\gdef\@othm#1[#2]#3{%
%    \end{macrocode}
% \changes{v2.2b}{95/10/07}{Build in check for counter defined}
%    We do not define a new counter but instead use one that has
%    already been defined.  Thus the only definition we need is that
%    of this pseudo-counter (i.e.\ |\the|\meta{env.\ name}).
%    First we check if |#2| corresponds to a known counter name.
%    \begin{macrocode}
  \expandafter\ifx\csname c@#2\endcsname\relax
   \@nocounterr{#2}%
  \else
   \expandafter\@ifdefinable\csname #1\endcsname
   {\expandafter \xdef \csname the#1\endcsname
     {\expandafter \noexpand \csname the#2\endcsname}%
%    \end{macrocode}
%    All other parts of the definition can be adopted from
%    |\@xnthm|. We have to remember, though, that in this case
%    the name of the current counter and the theorem label have moved
%    to the second and third arguments.
%    \begin{macrocode}
    \def\@tempa{\global\@namedef{#1}}\expandafter \@tempa
     \expandafter{\csname th@\the \theorem@style \expandafter
     \endcsname \the\theorem@bodyfont \@thm{#2}{#3}}%
    \global \expandafter \let \csname end#1\endcsname \@endtheorem}%
  \fi}
%    \end{macrocode}
% \end{macro}
%
% \subsection{Macros that are employed in a {\envfont theorem}
%    environment}
%
% \begin{macro}{\@thm}
%    The macro |\@thm| has to increase the current counter.  Then,
%    depending on whether the environment has (or does not have) an
%    optional argument, it has to branch into either |\@begintheorem|
%    or |\@opargtheorem|.
%    \begin{macrocode}
\gdef\@thm#1#2{\refstepcounter{#1}%
%    \end{macrocode}
%    Now we start a {\envfont trivlist} environment, and give
%    |\@topsep| and |\@topsepadd| the values of the \textsf{skip}
%    registers |\theorempreskipamount| and |\theorempostskipamount|.
%    The value in |\@topsep| is the vertical space that is inserted by
%    the first (and only) |\item| in our |\trivlist| whilst
%    |\@topsepadd| is inserted by |\@endparenv| at the end of that
%    {\envfont trivlist} environment.  By using these registers, we
%    obtain the desired space around a {\envfont theorem} environment.
%    \begin{macrocode}
   \trivlist
   \@topsep \theorempreskipamount               % used by first \item
   \@topsepadd \theorempostskipamount           % used by \@endparenv
%    \end{macrocode}
%    Now we have to test whether an optional argument has been given.
% \changes{v2.0d}{89/07/18}{Forgotten percent added.}
%    \begin{macrocode}
   \@ifnextchar [%
%    \end{macrocode}
%    If there is an optional argument, we will call |\@ythm|, and
%    move the arguments read back into the input stream.
%    \begin{macrocode}
   {\@ythm{#1}{#2}}%
%    \end{macrocode}
%    If not, we call |\@begintheorem|. Its first argument is the
%    name of the theorem set (hence the second argument of
%    |\@thm|). Its second argument is the macro that produces the
%    current number.
%    \begin{macrocode}
   {\@begintheorem{#2}{\csname the#1\endcsname}\ignorespaces}}
%    \end{macrocode}
% \end{macro}
%
% \begin{macro}{\@xthm}
% \begin{macro}{\@ythm}
%    Both these macros were originally called by |\@thm|.  We do
%    not need |\@xthm| anymore, hence we reset it to
%    |\relax|. The definition of |\@ythm| has not changed at
%    all from its definition in \LaTeX. In order to make the macros
%    easier to understand, we will
%    nevertheless present it (commented out).
%    \begin{macrocode}
\global\let\@xthm\relax
% \def\@ythm#1#2[#3]{\@opargbegintheorem{#2}{\csname
%           the#1\endcsname}{#3}\ignorespaces}
%</package>
%    \end{macrocode}
%    The primitive |\ignorespaces| in |\@ythm| and
%    |\@thm| is needed to remove the spaces between the
%    |\begin{...}| and the actual text.
% \end{macro}
% \end{macro}
%
%
% \subsection{Definition of the theorem substyles}
%
% As already pointed out, the theorem substyles, defined below, are
% only loaded when necessary.  Note that all these substyles, except
% \texttt{plain}, have |\slshape| as the default body font.
%
% \subsubsection{The \texttt{plain} style}
%
% \label{dok:th@plain} As the following macros use |@|, we have
% to locally set the |\catcode| of this symbol to ``letter''.
% This happens within a group, so that we do not have to worry about
% which |\catcode| that symbol had before.
%    \begin{macrocode}
%<*thp>
\begingroup \makeatletter
%    \end{macrocode}
% Since we are now within a group, we must make all definitions
% globally.  First we make sure that \texttt{theorem.sty} is loaded.
% This will allow us to use this file as a document style option
% without having to call \texttt{theorem} itself as an option.  At the
% same time, we assure that at least version 2 is loaded, since
% |\theorem@style| was not defined in earlier versions.
%    \begin{macrocode}
\@ifundefined{theorem@style}{\input{theorem.sty}}{}
\ProvidesFile{thp.sty}
             [\FMithmInfo]
%    \end{macrocode}
%
%
% \begin{macro}{\th@plain}
% \begin{macro}{\@begintheorem}
% \begin{macro}{\@opargbegintheorem}
%    |\theoremstyle{plain}| corresponds to the original
%    definition, except that the distances to the surrounding text are
%    determined by the parameters |\theorempreskipamount| and
%    |\theorempostskipamount|.
%    First we set the default body font.
%    \begin{macrocode}
\gdef\th@plain{\normalfont\itshape
%    \end{macrocode}
%    Then we define |\@begintheorem| and
%    |\@opargbegintheorem|. These two macros define how the
%    header of a theorem is typeset.  |\@opargbegintheorem| will
%    be called if a \textsf{theorem} environment with an optional
%    argument is encountered; otherwise, the header is constructed by
%    calling |\@begintheorem|.  If one of these macros is
%    executed, we are within a \textsf{trivlist} environment started by
%    |\@thm|.  So the theorem header is produced with an
%    |\item| command.
%
%    Instead of specifying the header font directly, all standard
%    theorem styles use the |\theorem@headerfont| macro to allow
%    customization.  The extra space (|\labelsep|) is necessary
%    because of problems in the \textsf{trivlist} environment.
%    \begin{macrocode}
  \def\@begintheorem##1##2{%
        \item[\hskip\labelsep \theorem@headerfont ##1\ ##2]}%
%    \end{macrocode}
%    The definition of |\@opargbegintheorem| is completely
%    analogous. The only difference is the fact that there exists a
%    third argument (which is the optional parameter of the
%    environment and contains additional information about the
%    theorem).  Customarily we enclose it in parentheses.
%    \begin{macrocode}
\def\@opargbegintheorem##1##2##3{%
   \item[\hskip\labelsep \theorem@headerfont ##1\ ##2\ (##3)]}}
%    \end{macrocode}
% \end{macro}
% \end{macro}
% \end{macro}
%
% \noindent
% We conclude with an |\endgroup| to restore the
% |\catcode| of |@|.
%    \begin{macrocode}
\endgroup
%</thp>
%    \end{macrocode}
%
%
% \subsubsection{The \texttt{break} style}
%
% This style option is stored in the file |thb.sty|.
% For the next two lines see the documentation for |\th@plain| on
% page~\pageref{dok:th@plain}.
%    \begin{macrocode}
%<*thb>
\begingroup \makeatletter
\@ifundefined{theorem@style}{\input{theorem.sty}}{}
\ProvidesFile{thb.sty}
             [\FMithmInfo]
%    \end{macrocode}
%
% \begin{macro}{\th@break}
%    |\theoremstyle{break}| produces a line break after the name
%    of the theorem.  The font is |\slshape|.  Hence, we define
%    |\th@break| as follows:
%    \begin{macrocode}
\gdef\th@break{\normalfont\slshape
  \def\@begintheorem##1##2{\item[%
%    \end{macrocode}
%    We run into the following problem: it is not possible to create
%    the header with |\item[|\meta{title}|]| and then start
%    a new line by, for example, |\mbox{}\\|. Such a definition
%    will fail whenever a list environment follows immediately.  With
%    the above construction, the |\mbox{}| causes the switch
%    |@inlabel| (cf.\ definition of |\list| and
%    |\trivlist| in \cite{latex:source}) to be set to \texttt{false}
%    and so the following list will insert additional vertical
%    space (|\topskip|). This is quite annoying. Therefore, we
%    create the line break within the |\item|. In order to ensure
%    that the text will begin at the proper position in the following
%    line, we simply pretend that the label does not take any
%    room.\footnote{This will lead to problems whenever very high
%    symbols occurring in the line tower into the
%    heading. So, something else has to be done here sometime.}
%    \begin{macrocode}
       \rlap{\vbox{\hbox{\hskip \labelsep\theorem@headerfont ##1\ ##2}%
                   \hbox{\strut}}}]}%
%    \end{macrocode}
%    Again, the definition of |\@opargbegintheorem| is
%    completely analogous.
%    \begin{macrocode}
\def\@opargbegintheorem##1##2##3{%
  \item[\rlap{\vbox{\hbox{\hskip \labelsep \theorem@headerfont
                     ##1\ ##2\ (##3)}%
                    \hbox{\strut}}}]}}
\endgroup
%</thb>
%    \end{macrocode}
% \end{macro}
%
%
% \subsubsection{The \texttt{changebreak} style}
%
%    \begin{macrocode}
% This style option is stored in the file |thcb.sty|.
%    \begin{macrocode}
%<*thcb>
\begingroup \makeatletter
\@ifundefined{theorem@style}{\input{theorem.sty}}{}
\ProvidesFile{thcb.sty}
             [\FMithmInfo]
%    \end{macrocode}
%
% \begin{macro}{\th@changebreak}
%    The \texttt{change-break} theorem style is like \texttt{break}
%    but with interchange of theorem name and theorem number.  Thus we
%    define |\th@changebreak| as follows:
%    \begin{macrocode}
\gdef\th@changebreak{\normalfont\slshape
  \def\@begintheorem##1##2{\item
       [\rlap{\vbox{\hbox{\hskip\labelsep \theorem@headerfont ##2\ ##1}%
                    \hbox{\strut}}}]}%
\def\@opargbegintheorem##1##2##3{%
  \item[\rlap{\vbox{\hbox{\hskip\labelsep \theorem@headerfont
                     ##2\ ##1\ (##3)}%
                    \hbox{\strut}}}]}}
\endgroup
%</thcb>
%    \end{macrocode}
% \end{macro}
%
%
% \subsubsection{The \texttt{change} style}
%
% This style option is stored in the file |thc.sty|.
%    \begin{macrocode}
%<*thc>
\begingroup \makeatletter
\@ifundefined{theorem@style}{\input{theorem.sty}}{}
\ProvidesFile{thc.sty}
             [\FMithmInfo]
%    \end{macrocode}
%
% \begin{macro}{\th@change}
%    The \texttt{change} theorem style corresponds to the
%    \texttt{change break} style without a linebreak after the header.
%    To say it in another way, it's the same as the \texttt{plain}
%    style but with number and name interchanged and |\slshape| as the
%    default font.
%    \begin{macrocode}
\gdef\th@change{\normalfont\slshape
  \def\@begintheorem##1##2{\item
       [\hskip\labelsep \theorem@headerfont ##2\ ##1]}%
\def\@opargbegintheorem##1##2##3{%
  \item[\hskip\labelsep \theorem@headerfont ##2\ ##1\ (##3)]}}
\endgroup
%</thc>
%    \end{macrocode}
% \end{macro}
%
%
% \subsubsection{The \texttt{marginbreak} style}
%
% This style option is the one used most often at Mainz.  It is saved
% in the file |thmb.sty|.
%    \begin{macrocode}
%<*thmb>
\begingroup \makeatletter
\@ifundefined{theorem@style}{\input{theorem.sty}}{}
\ProvidesFile{thmb.sty}
             [\FMithmInfo]
%    \end{macrocode}
%
% \begin{macro}{\th@marginbreak}
%    The \texttt{margin break} style is nearly the same as the
%    \texttt{change break} style. The only difference is the placement
%    of the theorem number.  We use |\llap| to place it in the left
%    margin.
%
%    In this style |\labelsep| denotes the separation between the
%    number and the text.
%    \begin{macrocode}
\gdef\th@marginbreak{\normalfont\slshape
  \def\@begintheorem##1##2{\item
       [\rlap{\vbox{\theorem@headerfont
                    \hbox{\llap{##2}\hskip\labelsep ##1}%
                    \hbox{\strut}}}]}%
\def\@opargbegintheorem##1##2##3{%
  \item[\rlap{\vbox{\theorem@headerfont
                    \hbox{\llap{##2}\hskip\labelsep ##1\ (##3)}%
                    \hbox{\strut}}}]}}
\endgroup
%</thmb>
%    \end{macrocode}
% \changes{v2.1c}{90/02/17}{`endgroup outside macrocode env.}
% \end{macro}
%
%
%
% \subsubsection{The \texttt{margin} style}
%
% This style option is stored in the file |thm.sty|.
%    \begin{macrocode}
%<*thm>
\begingroup \makeatletter
\@ifundefined{theorem@style}{\input{theorem.sty}}{}
\ProvidesFile{thm.sty}
             [\FMithmInfo]
%    \end{macrocode}
%
% \begin{macro}{\th@margin}
%    Again this is only a variant of the theorem styles described
%    above without any new ideas.
%    \begin{macrocode}
\gdef\th@margin{\normalfont\slshape
  \def\@begintheorem##1##2{\item
       [\theorem@headerfont \llap{##2}\hskip\labelsep ##1]}%
\def\@opargbegintheorem##1##2##3{%
  \item[\theorem@headerfont \llap{##2}\hskip\labelsep ##1\ (##3)]}}
\endgroup
%</thm>
%    \end{macrocode}
% \end{macro}
%
%
%
%
% \subsection{Final Definitions}
%
% \begin{macro}{\theorempreskipamount}
% \begin{macro}{\theorempostskipamount}
%    The \textsf{skip} parameters that regulate the vertical empty space
%    before and after the {\envfont theorem} environment have
%    to be allocated as well.
% \changes{v2.0f}{89/07/20}{Unnecessary `global' removed.}
%    \begin{macrocode}
%<*package>
\newskip\theorempreskipamount
\newskip\theorempostskipamount
%    \end{macrocode}
%    Since we have used the same values for all theorem sets, we now can
%    assign them.
%    \begin{macrocode}
\global\setlength\theorempreskipamount{12pt plus 5pt minus 3pt}
\global\setlength\theorempostskipamount{8pt plus 3pt minus 1.5pt}
%    \end{macrocode}
% \end{macro}
% \end{macro}
%
%
% \begin{macro}{\@endtheorem}
%    The same holds for the macro |\@endtheorem|, which ends a
%    {\envfont theorem} environment. Since it is the same for all
%    theorem sets, it is removed from the macros |\th@|\meta{style}.
%    It simply ends the {\envfont trivlist} environment, which was
%    begun in |\@thm|.
%    \begin{macrocode}
\global\let\@endtheorem=\endtrivlist
%    \end{macrocode}
% \end{macro}
%
%
%
% \begin{macro}{\@preamblecmds}
%    All macros defined above are to be used only in the preamble.
%    Therefore, we insert them in |\@preamblecmds| which will disable
%    them at begin document. This is done by the internal \LaTeXe{}
%    command |\@onlypreamble|.
% \changes{v2.0g}{89/09/19}{`theoremheaderfont and `theorembodyfont
%    added.}
% \changes{v2.2a}{94/02/03}{Use \LaTeXe{} interface}
%    \begin{macrocode}
\@onlypreamble\@xnthm
\@onlypreamble\@ynthm
\@onlypreamble\@othm
\@onlypreamble\newtheorem
\@onlypreamble\theoremstyle
\@onlypreamble\theorembodyfont
\@onlypreamble\theoremheaderfont
%    \end{macrocode}
% \end{macro}
%
% Finally we declare the \texttt{plain} theorem style to be the default.
% \changes{v2.1b}{90/06/04}{Fixed missing `item by declaring default
%    style.}
%    \begin{macrocode}
\theoremstyle{plain}
%</package>
%    \end{macrocode}
%
% \Finale

\endinput
}{}
\ProvidesFile{thc.sty}
             [\FMithmInfo]
%    \end{macrocode}
%
% \begin{macro}{\th@change}
%    The \texttt{change} theorem style corresponds to the
%    \texttt{change break} style without a linebreak after the header.
%    To say it in another way, it's the same as the \texttt{plain}
%    style but with number and name interchanged and |\slshape| as the
%    default font.
%    \begin{macrocode}
\gdef\th@change{\normalfont\slshape
  \def\@begintheorem##1##2{\item
       [\hskip\labelsep \theorem@headerfont ##2\ ##1]}%
\def\@opargbegintheorem##1##2##3{%
  \item[\hskip\labelsep \theorem@headerfont ##2\ ##1\ (##3)]}}
\endgroup
%</thc>
%    \end{macrocode}
% \end{macro}
%
%
% \subsubsection{The \texttt{marginbreak} style}
%
% This style option is the one used most often at Mainz.  It is saved
% in the file |thmb.sty|.
%    \begin{macrocode}
%<*thmb>
\begingroup \makeatletter
\@ifundefined{theorem@style}{% \iffalse meta-comment
%
% Copyright (C) 1993-2024
%
% The LaTeX Project and any individual authors listed elsewhere
% in this file.
%
% This file is part of the Standard LaTeX `Tools Bundle'.
% -------------------------------------------------------
%
% It may be distributed and/or modified under the
% conditions of the LaTeX Project Public License, either version 1.3c
% of this license or (at your option) any later version.
% The latest version of this license is in
%    https://www.latex-project.org/lppl.txt
% and version 1.3c or later is part of all distributions of LaTeX
% version 2005/12/01 or later.
%
% The list of all files belonging to the LaTeX `Tools Bundle' is
% given in the file `manifest.txt'.
%
% \fi
%
%%
%
%
\def\FMithmInfo{2023/07/05 v2.2c Theorem extension package (FMi)}
%
% \ProvidesFile{theorem.dtx}[\FMithmInfo]
%\iffalse   % this is a METACOMMENT !
%%
%% Package `theorem' to use with LaTeX2e
%% Copyright (C) 1989-1995 Frank Mittelbach, all rights reserved.
%<+package>\@ifundefined{theorem@style}{}{\endinput}
%<+package>\NeedsTeXFormat{LaTeX2e}[1995/06/01]
%<+package>\ProvidesPackage{theorem}
%<+package>         [\FMithmInfo]
%
% \fi
%
% \changes{v1.0}{87/05/07}{First draft.}
% \changes{v1.0a}{87/05/07}{`thistheoremstyle' added.}
% \changes{v1.1a}{87/07/29}{New concept: redefine the internal
%                          \protect\LaTeX{} commands `@xnthm' etc.}
% \changes{v1.1b}{87/07/31}{`theoremfont' used to specify font.}
% \changes{v1.2a}{88/01/08}{`@endtheorem' now globally defined.}
% \changes{v1.2b}{88/01/11}{`newtheorem' only allowed in the preamble.}
% \changes{v1.3a}{88/02/14}{`theoremfont' now a toks register.}
%
% \changes{v2.0a}{88/11/16}{New implementation and new documentation.}
% \changes{v2.0a}{88/11/16}{No layout changes but internal modifications
%                           old substyles won't work.}
% \changes{v2.0b}{89/01/30}{`global' used for all definitions.}
% \changes{v2.0c}{89/06/15}{Documentation translated to English.
%                            Thanks to LLL and Nina}
% \changes{v2.1a}{90/02/15}{Support for new font selection.}
% \changes{v2.2a}{94/02/03}{Cleanup for LaTeX2e}
%
% \DoNotIndex{\@endparenv,\@ifdefinable,\@ifnextchar,\@ifundefined}
% \DoNotIndex{\@namedef,\@nameuse,\@spaces,\@tempa,\@warning}
% \DoNotIndex{\ }
% \DoNotIndex{\begingroup,\bfseries}
% \DoNotIndex{\csname}
% \DoNotIndex{\def,\do,\docdate}
% \DoNotIndex{\endcsname,\endgroup,\endinput,\endtrivlist,\expandafter}
% \DoNotIndex{\filedate,\fileversion}
% \DoNotIndex{\gdef,\global}
% \DoNotIndex{\hbox,\hskip}
% \DoNotIndex{\ignorespaces,\input,\itshape,\item}
% \DoNotIndex{\labelsep,\let,\llap}
% \DoNotIndex{\makeatletter}
% \DoNotIndex{\noexpand}
% \DoNotIndex{\refstepcounter,\relax,\rlap}
% \DoNotIndex{\setlength,\slshape,\space,\string,\strut}
% \DoNotIndex{\the,\trivlist,\typeout}
% \DoNotIndex{\vbox}
% \DoNotIndex{\xdef}
%
% \def\deflabel#1{\ttfamily #1\hfill}
% \def\deflist#1{\begin{list}{}{\settowidth\labelwidth{\ttfamily #1}%
%                               \setlength\leftmargin\labelwidth
%                               \addtolength\leftmargin\labelsep
%                               \let\makelabel\deflabel}}
% \def\enddeflist{\end{list}}
% \def\envfont{\normalfont\ttfamily}
%
% \MakeShortVerb{\|}
%
% \newcommand{\TUB}{{\sl TUGboat\/}}
%
% \GetFileInfo{theorem.dtx}
% \title{An Extension of the
%        \LaTeX{} theorem environment\thanks{This file
%        has version number \fileversion, last
%        revised \filedate.}}
% \author{Frank Mittelbach}
%
% \MaintainedByLaTeXTeam{tools}
% \maketitle
%
% \begin{abstract}
%    The macros described in this paper yield an extension of the
%    \LaTeX{} theorem mechanism. It is designed to satisfy the
%    different requirements of various journals. Thus, the
%    layout of the ``theorems'' can be manipulated by determining a
%    ``style''. This article describes not only the use, but
%    also the definition, of the necessary macros.
% \end{abstract}
%
% \section*{Preface to version 2.2}
%
% For \LaTeXe{} this package did not need any fundamental changes. I
% only modified the messages generated so that theorem layout styles
% will show up with the |\listfiles| command and cleaned the section
% on the New Font Selection Scheme since this is now included in
% \LaTeX.
%
% \section*{Preface to version 2.1}
%
% This version is identical to 2.0g described in \TUB\ 10\#3 except
% for some internal defaults which are now set depending on the used
% font selection scheme.
%
% This was done to avoid unpleasant surprises if the new font selection
% scheme is in force. For further details see
% section~\ref{sec:newfonts} and \cite{bk:GMS94}.
%
% \section{Introduction}
%
% For our purposes here, ``theorems'' are labelled enunciations,
% often set off from the main text by extra space and a font change.
% Theorems, corollaries, conjectures, definitions, and remarks
% are all instances of ``theorems''.  The ``header'' of these structures
% is composed of a label (such as \textsc{Theorem} or \textsc{Remark})
% and a number which serializes an item in the sequence of items
% with the same label.
%
% Shortly after the introduction of \LaTeX{} at the Fachbereich
% Mathematik in Mainz, the desire to manipulate the layout of
% ``theorems'' arose.  In Mainz, the following two conventions came
% into general use:
% \begin{enumerate}
%   \item The number of the theorem is shown in the margin.
%   \item There is a line break at
%         the end of the theorem header.
% \end{enumerate}
%
% Additionally, some journals require different formats which depend on
% the ``sort of theorem'': e.g.\ often remarks and definitions are set
% in |\upshape|, while |\itshape| is employed for main theorems.
%
% Confronted with these requirements, a theorem environment was
% developed in Mainz which allows separate determination of the
% layout of the ``theorems sets'', comparable to |\pagestyle|.
%
%
% \section{The user interface}
% \subsection{Defining new theorem sets}
%
% \DescribeMacro\newtheorem
% As in the original \LaTeX{} version, the command |\newtheorem|
% defines a new ``theorem set'' or ``theorem-like structure''.
% Two required arguments name the new environment and give the text
% to be typeset with each instance of the new ``set'', while
% an optional argument determines how
% the ``set'' is enumerated:
% \begin{description}
%    \item[\ttfamily \bslash newtheorem\{foo\}\{bar\}]
%       The theorem set {\envfont foo} (whose name is \texttt{bar})
%       uses its own counter.
%    \item[\ttfamily \bslash newtheorem\{foo2\}{[foo]}\{bar2\}]
%       The theorem set {\envfont foo2} (printed name \texttt{bar2})
%       uses the same counter as the theorem set \texttt{foo}.
%    \item[\ttfamily \bslash newtheorem\{foo3\}\{bar3\}{[section]}]
%       The theorem set {\envfont foo3} (printed name \texttt{bar3}) is
%       enumerated within the counter \texttt{section}, i.e.\ with every
%       new |\section| the enumeration begins again with 1, and
%       the enumeration is composed from the section-number and the
%       theorem counter itself.
% \end{description}
%
% \DescribeMacro\theoremstyle
% Additionally, the command |\theoremstyle| can define the layout
% of various, or all, theorem sets.  It should be noted that any
% theorem set defined by |\newtheorem| is typeset in the
% |\theoremstyle| that is current at the time of the definition.
% Thus, the following
% \begin{verbatim}
%    \theoremstyle{break}        \newtheorem{Cor}{Corollary}
%    \theoremstyle{plain}        \newtheorem{Exa}{Example}[section]
%\end{verbatim}                             ^^A No blank at line start
% leads to the result that the set {\envfont Cor} is formatted in the
% style \texttt{break}, while the set {\envfont Exa} and all the
% following ones are formatted in the style \texttt{plain}, unless
% another |\theoremstyle| follows.  Since the definitions installed by
% |\newtheorem| are global, one also can limit |\theoremstyle| locally
% by grouping braces.
%
% \DescribeMacro\theorembodyfont
% The choice of the font for the theorem body is completely
% independent of the chosen |\theoremstyle|; this has proven to
% be very advantageous.  For example,
% \begin{verbatim}
%   {\theorembodyfont{\upshape}          \newtheorem{Rem}{Remark}}
%\end{verbatim}
% defines a theorem set {\envfont Rem}, which will be set in
% |\upshape| in the current layout (which in our example is
% \texttt{plain}).  As with |\theoremstyle|, the |\theorembodyfont|
% chosen is that current at the time of |\newtheorem|.  If
% |\theorembodyfont| is not specified or one defines
% |\theorembodyfont{}|, then the font used will be that defined by the
% |\theoremstyle|.
%
% \DescribeMacro\theoremheaderfont It is also possible to customize
% the font used for the theorem headers.  This is, however, a global
% declaration, and therefore there should be at most one
% |\theoremheaderfont| declaration in the preamble.\footnote{If it is
% actually necessary to have different header fonts, one has to define
% new theorem styles (substituting the desired font) or specify the
% information directly in the \texttt{\bslash newtheorem} declaration
% (the unclean variant).}
%
% \DescribeMacro\theorempreskipamount
% \DescribeMacro\theorempostskipamount Two additional parameters
% affect the vertical space around the theorem environments:
% |\theorempreskipamount| and |\theorempostskipamount| define,
% respectively, the spacing before and after such an environment.
% These parameters apply for all theorem sets and can be manipulated
% with the ordinary length macros.  They are rubber lengths
% (`\textsf{skips}'), and therefore can contain \texttt{plus} and
% \texttt{minus} parts.
%
% Since the definition of theorem sets should---most sensibly---be
% placed in the preamble, we only allow installation there.
% It is therefore possible to release the memory used here after
% |\begin{document}|, in order to make room for other
% applications.
%
%
% \subsection{Existing theorem styles}
%
% The following theorem styles exist to date:
% \begin{deflist}{marginbreak}
%    \item[plain]
%       This theorem style emulates the original \LaTeX{} definition,
%       except that additionally the parameters
%       |\theorem...skipamount| are used.
%    \item[break]
%       In this style, the theorem header is followed by a line
%       break.
%    \item[marginbreak]
%       The theorem number is set in the margin, and there is a line
%       break as in \texttt{break}.
%    \item[changebreak]
%       Like \texttt{break}, but with header number and text
%       interchanged.
%    \item[change]
%       Header number and text are interchanged, without a line break.
%    \item[margin]
%       The number is set in the left margin, without a line break.
% \end{deflist}
% All styles (except \texttt{plain}) select |\slshape| as the default
% |\theorembodyfont|.
%
%
% \subsection{Examples}
%
% Given the above theorem sets {\envfont Cor}, {\envfont Exa}
% and {\envfont Rem},
% suppose that the preamble also contains the declarations:
% \begin{verbatim}
%    \theoremstyle{marginbreak}   \newtheorem{Lem}[Cor]{Lemma}
%    \theoremstyle{change}
%    \theorembodyfont{\itshape}        \newtheorem{Def}[Cor]{Definition}
%
%    \theoremheaderfont{\scshape}
%\end{verbatim}
% Then the following are some typical examples of the typeset output
% resulting from their use.
%
% \begin{Cor}
%    This is a sentence typeset in the theorem environment {\envfont
%    Cor}.
% \end{Cor}
%
% \begin{Exa}
%    This is a sentence typeset in the theorem environment {\envfont
%    Exa}.
% \end{Exa}
%
% \begin{Rem}
%    This is a sentence typeset in the theorem environment {\envfont
%    Rem}.
% \end{Rem}
%
% \begin{Lem}[Ben User]
%    This is a sentence typeset in the theorem environment {\envfont
%    Lem}.
% \end{Lem}
%
% \begin{Def}[Very impressive Definition]
%    This is a sentence typeset in the theorem environment {\envfont
%    Def}.
% \end{Def}
%
% The last two examples show the effect of the optional argument to
% a theorem environment (it is the text typeset in parentheses).
%
%
% \section{Special Considerations}
% \label{sec:newfonts}
%
% Theoremheader and body are implemented as a unit. This means that
% the |\theoremheaderfont|  will inherit characteristics of the
% |\theorembodyfont| in \LaTeXe. Thus, if for example
% |\theorembodyfont| is |\itshape| and |\theoremheaderfont|
% is |\bfseries| the font selected for the header will have the
% characteristics `bold extended italic'. If this is not desired one
% should set the |\theoremheaderfont| to something like
% \begin{verbatim}
%      \theoremheaderfont{\normalfont\bfseries}
%\end{verbatim}
% i.e.\ supplying all necessary font information explicitly.
%
%
%
% \section{Acknowledgements}
%
% The publication of this set of macros was only possible with the
% help of Christina Busse (translating the manuscript into English),
% Joachim Pense (playing the r\^ole of typist), Chris Rowley
% (looking everything over) and many others providing useful
% suggestions.
%
% \MaybeStop{%
%    \begin{thebibliography}{9}
%       \bibitem{bk:GMS94} \textsc{M.~Goossens}, \textsc{F.~Mittelbach}
%         and \textsc{A.~Samarin}.
%         \newblock The \LaTeX{} Companion.
%         \newblock
%         Addison-Wesley, Reading, Massachusetts, 1994.
%       \bibitem{latex:source} \textsc{Lamport, Leslie.} \newblock
%                               \texttt{latex.tex}, version 2.09, date
%                               Feb. 1990.
%    \end{thebibliography}
%    \PrintIndex
%    \PrintChanges}
%
%
% \section{The documentation driver file}
%
% The next bit of code contains the documentation driver file for
% \TeX{}, i.e., the file that will produce the documentation you are
% currently reading. It will be extracted from this file by the
% \texttt{docstrip} program. Since it is the first code in the file
% one can alternatively process this file directly with \LaTeXe{} to
% obtain the documentation.
%
%    \begin{macrocode}
%<*driver>
\documentclass{ltxdoc}

\usepackage{theorem}

%<+driver>% The next few lines define theorem sets which are used
%<+driver>% in the example section of the documentation.

\theoremstyle{break}       \newtheorem{Cor}{Corollary}
\theoremstyle{plain}       \newtheorem{Exa}{Example}[section]
{\theorembodyfont{\upshape}\newtheorem{Rem}{Remark}}
\theoremstyle{marginbreak} \newtheorem{Lem}[Cor]{Lemma}
\theoremstyle{change}
\theorembodyfont{\itshape} \newtheorem{Def}[Cor]{Definition}

\theoremheaderfont{\scshape}

\RecordChanges

\begin{document}
   \DocInput{theorem.dtx}
\end{document}
%</driver>
%    \end{macrocode}
%
% \section{Definition of the Macros}
%
% If the file has been loaded before, we abort immediately. If not the
% package announces itself (this is actually done at the very top of
% the file---the way it is done isn't good style so don't copy it).
% \changes{v2.0e}{89/07/19}{Spaces removed from `typeout'.}
% \changes{v2.2b}{95/11/19}{Announce moved to top of file}
%    \begin{macrocode}
%<*package>
%\@ifundefined{theorem@style}{}{\endinput}
%\def\FMithmInfo{1995/11/19 v2.2b Theorem extension package (FMi)}
%\ProvidesPackage{theorem}[\FMithmInfo]
%    \end{macrocode}
%
%
% \subsection{Definition of theorem styles and fonts}
%
% All the definitions in this file are done globally to allow
% inputting this file inside a group.
%
% \begin{macro}{\theoremstyle}
%    Before a theorem style can be installed, the chosen style must be
%    known. For that reason, we must test to see that
%    |\th@|\meta{style} is known or, more precisely, that it is
%    different from |\relax|. If the style is not known then
%    |\th@plain| is used.
%    \begin{macrocode}
\gdef\theoremstyle#1{%
   \@ifundefined{th@#1}{\@warning
          {Unknown theoremstyle `#1'. Using `plain'}%
          \theorem@style{plain}}%
%    \end{macrocode}
%    We save the theorem style to be used in the token register
%    |\theorem@style|.
%    \begin{macrocode}
      {\theorem@style{#1}}%
%    \end{macrocode}
%    Now we ``evaluate'' the theorem style: this means, we call the
%    macro |\th@|\meta{style} which will activate the relevant
%    definitions which are contained in a separate file.  This is done
%    in a group to suppress changes to the current font.  This could
%    otherwise pose problems together with the new font selection
%    scheme\footnote{When I printed the original article using the new
%    font selection scheme I ended with a document with slanted
%    typefaces (text headings and all) simply because one of the
%    theorem styles used \texttt{\bslash sl} at toplevel.} if the
%    |\th@|\meta{style} is evaluated a second time.
% \changes{v2.1a}{90/02/15}{Group added around `th@..}
% \changes{v2.1a}{90/02/15}{Reseting of `..@begintheorem removed.}
%    \begin{macrocode}
      \begingroup
        \csname th@\the\theorem@style \endcsname
      \endgroup}
%    \end{macrocode}
% \end{macro}
%
%
% \begin{macro}{\@begintheorem}
% \changes{v2.1a}{90/02/15}{Relaxed at toplevel.}
% \begin{macro}{\@opargbegintheorem}
% \changes{v2.1a}{90/02/15}{Relaxed at toplevel.}
%    We reset |\@begintheorem| and |\@opargbegintheorem| to
%    |\relax| since these commands are no longer necessary at
%    toplevel. This will save a few tokens.
%    \begin{macrocode}
\global\let\@begintheorem\relax
\global\let\@opargbegintheorem\relax
%    \end{macrocode}
% \end{macro}
% \end{macro}
%
% \begin{macro}{\theorem@style}
%    Obviously the \textsf{token} register used above has to be
%    allocated. To assure the utmost compatibility with the
%    original \LaTeX{} definition, we set the default theorem style to
%    \texttt{plain}, which implements the usual \LaTeX{} convention.
% \changes{v2.0f}{89/07/20}{Unnecessary `global' removed.}
%    \begin{macrocode}
\newtoks\theorem@style
\global\theorem@style{plain}
%    \end{macrocode}
% \end{macro}
%
% \begin{macro}{\theorembodyfont}
% \begin{macro}{\theorem@bodyfont}
%    If the |\theorembodyfont| is set by the user then it should not
%    interact with the default font set in the theorem style. When the
%    new font selection is in force this may happen if, for example,
%    the default is |\itshape| and the new |\theorembodyfont| is
%    |\sffamily|. So we add a |\reset@font| command in front of the
%    user definition.
% \changes{v2.0f}{89/07/20}{Unnecessary `global' removed.}
% \changes{v2.0g}{89/09/19}{`theoremfont to `theorembodyfont.}
% \changes{v2.1c}{92/06/22}{`reset@font added and logic changed.}
% \changes{v2.1d}{92/10/28}{Reset to empty if arg is empty}
%    \begin{macrocode}
\gdef\theorembodyfont#1{%
%    \end{macrocode}
%    We check if the argument supplied is empty and if so put nothing
%    into the |\theorem@bodyfont| token register to allow for
%    |\theorembodyfont{}| as a mean of using the default of the
%    current |\theoremstyle|.
%    \begin{macrocode}
   \def\@tempa{#1}%
   \ifx\@tempa\@empty
    \theorem@bodyfont{}%
   \else
    \theorem@bodyfont{\reset@font#1}%
   \fi
   }
\newtoks\theorem@bodyfont
\global\theorem@bodyfont{}
%    \end{macrocode}
% \end{macro}
% \end{macro}
%
%
% \begin{macro}{\theoremheaderfont}
% \changes{v2.0g}{89/09/19}{Macro added.}
%    The font for the theorem headers is handled differently because
%    this definition applies to all theorem styles.
%    \begin{macrocode}
\gdef\theoremheaderfont#1{\gdef\theorem@headerfont{#1}%
%    \end{macrocode}
%    After using the macro once it is redefined to produce an error
%    message.
%    \begin{macrocode}
       \gdef\theoremheaderfont##1{%
        \typeout{\string\theoremheaderfont\space should be used
                 only once.}}}
%    \end{macrocode}
% \end{macro}
%
% \begin{macro}{\theorem@headerfont}
% \changes{v2.0g}{89/09/19}{Macro added.}
%    To set the |\theorem@headerfont| default we first test if
%    the new fontselection scheme is in force.
% \changes{v2.1a}{90/02/15}{Support for new font selection.}
%    \begin{macrocode}
\ifx\upshape\undefined
%    \end{macrocode}
%    If not we define it to expand into |\bfseries|. We don't use
%    |\let| just in case a following style option redefines this
%    macro.
%    \begin{macrocode}
\gdef\theorem@headerfont{\bfseries}
%    \end{macrocode}
%    Otherwise we reset the current shape before calling |\bfseries|.
%    \begin{macrocode}
\else \gdef\theorem@headerfont{\normalfont\bfseries}\fi
%    \end{macrocode}
% \end{macro}
%
% \begin{macro}{\th@plain}
% \begin{macro}{\th@break}
% \begin{macro}{\th@marginbreak}
% \begin{macro}{\th@changebreak}
% \begin{macro}{\th@change}
% \begin{macro}{\th@margin}
%    The different styles are defined in macros such as
%    |\th@plain|.  Since memory space is precious in
%    ``non-Big-versions'', we have to avoid offering too many unused
%    definitions.  Therefore we define these styles in separate files
%    that can be loaded on demand.  Thus the commands themselves only
%    load these files. We use |\@input@|, a \LaTeXe{} internal command
%    that ensures that the file will be listed with |\listfiles|.
% \changes{v2.2a}{94/02/02}{Use `@input@ to load theorem layout files}
%    \begin{macrocode}
\gdef\th@plain{\@input@{thp.sty}}
\gdef\th@break{\@input@{thb.sty}}
\gdef\th@marginbreak{\@input@{thmb.sty}}
\gdef\th@changebreak{\@input@{thcb.sty}}
\gdef\th@change{\@input@{thc.sty}}
\gdef\th@margin{\@input@{thm.sty}}
%    \end{macrocode}
%    This list will be expanded when new styles become
%    available.  For testing,  just append new theorem substyles
%    as document options.
% \end{macro}
% \end{macro}
% \end{macro}
% \end{macro}
% \end{macro}
% \end{macro}
%
%
%
% \subsection{Definition of a new theorem set}
%
% As already pointed out, a new theorem environment can be defined in
% three different ways:
% \begin{verbatim}
%    \newtheorem{Lem}{Lemma}
%    \newtheorem{Lem}{Lemma}[section]
%    \newtheorem{Lem}[Theorem]{Lemma}
%\end{verbatim}
% The function of the macro |\newtheorem| is to recognize these cases
% and then to branch into one of the three macros |\@ynthm|, |\@xnthm|
% or |\@othm|.  This mechanism is adopted unchanged from
% \cite{latex:source}; the essential point here is that, for example,
% in the second case, the arguments \texttt{Lem}, \texttt{Lemma} and
% \texttt{section} are passed over to the macro |\@xnthm|.
%
% We inspect this case first because the others present fewer problems,
% and thus are easily derived from this one.
%
% \begin{macro}{\@xnthm}
%    For our example arguments, the macro |\@xnthm| must fulfill
%    the following:
%    \begin{itemize}
%       \item
%          Define a new \LaTeX-counter `Lem'
%       \item
%          reset this counter within a |\section|
%       \item
%          define the macro |\theLem|
%       \item
%          define the environment macros |\Lem| and
%          |\endLem| using the current
%          |\theoremstyle| and |\theorem@bodyfont|.
%    \end{itemize}
%    Obviously, all this should happen only if the first argument
%    of |\@xnthm| (i.e.\ \texttt{Lem} in our example) is chosen so
%    as not to conflict with any previously defined commands or
%    environments.  This test is performed by the \LaTeX{} macro
%    |\@ifdefinable|.
%    \begin{macrocode}
\gdef\@xnthm#1#2[#3]{\expandafter\@ifdefinable\csname #1\endcsname
%    \end{macrocode}
%    Therefore, the first argument of |\@ifdefinable| is the
%    expansion (in the example, |\Lem|) of
%    |\csname||#1\endcsname|.  The second argument is
%    executed only if the test has been completed successfully.
%    \begin{macrocode}
   {%
%    \end{macrocode}
%    Now we define the new counter. The names of the \LaTeX{}
%    macros employed should speak for themselves:
% \changes{v2.2b}{1995/10/07}{Use \cs{@newctr}}
% \changes{v2.2c}{1995/11/23}{Use \cs{@newctr} properly.}
%    \begin{macrocode}
    \@definecounter{#1}\@newctr{#1}[#3]%
%    \end{macrocode}
%    Using |\@newctr| will give a proper error message if the counter
%    in |#3| is not defined.
%    In defining `|\theLem|' we must generate the desired macro
%    name by use of |\expandafter| and |\csname|.
%    \begin{macrocode}
    \expandafter\xdef\csname the#1\endcsname
%    \end{macrocode}
%    An |\xdef| is used in order to make the definition global,
%    and to ensure that it contains the replacement texts of
%    |\@thmcountersep| and |\@thmcounter|.\footnote{These
%    two macros can be defined by the document style.  Their default
%    values produce a `\texttt{.}' as separation and an arabic
%    representation of the number.} However, not everything should be
%    expanded. For example, it saves space to use |\thesection|
%    instead of its---at times---lengthy expansion.
%    \begin{macrocode}
      {\expandafter \noexpand \csname the#3\endcsname
       \@thmcountersep \@thmcounter{#1}}%
%    \end{macrocode}
%    Thus with the defaults of \LaTeX{}, |\theLem| would be
%    replaced by the command sequence |\thesection.\arabic{Lem}|.
%
%    We will now look at the definition of the macro which is executed
%    at the beginning of the actual environment (in our example this
%    macro is |\Lem|). It should be noted that we use an
%    ``|\expandafter| trick'' to expand only certain parts of the
%    replacement text at the time of the definition.
%    \begin{macrocode}
    \def\@tempa{\global\@namedef{#1}}%
    \expandafter \@tempa \expandafter{%
%    \end{macrocode}
%    First, the macro that contains the current definitions of
%    |\@begintheorem| and |\@opargtheorem| should be called
%    up.  The name of this macro---as is already known---has the form
%    |\th@|\meta{theorem\ style}; therefore, it must be called by
%    \begin{macrocode}
      \csname th@\the \theorem@style
%    \end{macrocode}
%    In addition the default theorem font should be changeable, i.e.\
%    we have to insert the contents of |\theorem@bodyfont|. For
%    that reason, we expand even further, beyond |\endcsname|,
%    and thus insert the contents of the \textsf{token} register
%    |\theorem@bodyfont| in the replacement text.
%    \begin{macrocode}
            \expandafter \endcsname \the \theorem@bodyfont
%    \end{macrocode}
%    Now it is time to call the macro |\@thm| which takes over
%    the further processing.  It has two arguments: the current
%    counter name (in our example, \texttt{Lem}), and the text of the
%    label (in our example, \texttt{Lemma}).
%    \begin{macrocode}
     \@thm{#1}{#2}}%
%    \end{macrocode}
%    With this, the `sub-definition' is complete.  The macro
%    |\@endtheorem| ends a theorem environment and is, so far, nothing
%    but an |\endtrivlist|.  (Hence it is defined globally, and not
%    within the theorem styles.\footnote{This has to be changed as
%    soon as theorem styles that change \texttt{\bslash @endtheorem}
%    exist.  In such a case, all existing styles must be changed as
%    well since they will have to reset the macro.}) Therefore, we can
%    set it equivalent to the macro that ends the theorem set (in our
%    example, |\endLem|).  However, if some day theorem styles exist
%    that do change |\@endtheorem|, we would have to use the
%    commented-out line instead.
%    \begin{macrocode}
    \global \expandafter \let \csname end#1\endcsname \@endtheorem
%   \global\@namedef{end#1}{\@endtheorem}%
%    \end{macrocode}
%    With these commands all the required definitions are employed,
%    unless the test |\@ifdefinable| has failed.  Therefore, we
%    end the second argument of this macro and with it the definition
%    of |\@xnthm|.
%    \begin{macrocode}
   }}
%    \end{macrocode}
% \end{macro}
%
% \begin{macro}{\@ynthm}
%    The definition of |\@ynthm| is completely analogous. In
%    this case the new counter that is defined is not reset within
%    another counter; thus the definition of |\the...|
%    is simplified:
%    \begin{macrocode}
\gdef\@ynthm#1#2{\expandafter\@ifdefinable\csname #1\endcsname
   {\@definecounter{#1}%
    \expandafter\xdef\csname the#1\endcsname{\@thmcounter{#1}}%
%    \end{macrocode}
%    The rest of the definition corresponds literally to that of
%    |\@xnthm|:
%    \begin{macrocode}
    \def\@tempa{\global\@namedef{#1}}\expandafter \@tempa
     \expandafter{\csname th@\the \theorem@style \expandafter
     \endcsname \the\theorem@bodyfont \@thm{#1}{#2}}%
    \global \expandafter \let \csname end#1\endcsname \@endtheorem}}
%    \end{macrocode}
% \end{macro}
%
% \begin{macro}{\@othm}
%    The definition of |\@othm| does not contain anything new.
%    \begin{macrocode}
\gdef\@othm#1[#2]#3{%
%    \end{macrocode}
% \changes{v2.2b}{95/10/07}{Build in check for counter defined}
%    We do not define a new counter but instead use one that has
%    already been defined.  Thus the only definition we need is that
%    of this pseudo-counter (i.e.\ |\the|\meta{env.\ name}).
%    First we check if |#2| corresponds to a known counter name.
%    \begin{macrocode}
  \expandafter\ifx\csname c@#2\endcsname\relax
   \@nocounterr{#2}%
  \else
   \expandafter\@ifdefinable\csname #1\endcsname
   {\expandafter \xdef \csname the#1\endcsname
     {\expandafter \noexpand \csname the#2\endcsname}%
%    \end{macrocode}
%    All other parts of the definition can be adopted from
%    |\@xnthm|. We have to remember, though, that in this case
%    the name of the current counter and the theorem label have moved
%    to the second and third arguments.
%    \begin{macrocode}
    \def\@tempa{\global\@namedef{#1}}\expandafter \@tempa
     \expandafter{\csname th@\the \theorem@style \expandafter
     \endcsname \the\theorem@bodyfont \@thm{#2}{#3}}%
    \global \expandafter \let \csname end#1\endcsname \@endtheorem}%
  \fi}
%    \end{macrocode}
% \end{macro}
%
% \subsection{Macros that are employed in a {\envfont theorem}
%    environment}
%
% \begin{macro}{\@thm}
%    The macro |\@thm| has to increase the current counter.  Then,
%    depending on whether the environment has (or does not have) an
%    optional argument, it has to branch into either |\@begintheorem|
%    or |\@opargtheorem|.
%    \begin{macrocode}
\gdef\@thm#1#2{\refstepcounter{#1}%
%    \end{macrocode}
%    Now we start a {\envfont trivlist} environment, and give
%    |\@topsep| and |\@topsepadd| the values of the \textsf{skip}
%    registers |\theorempreskipamount| and |\theorempostskipamount|.
%    The value in |\@topsep| is the vertical space that is inserted by
%    the first (and only) |\item| in our |\trivlist| whilst
%    |\@topsepadd| is inserted by |\@endparenv| at the end of that
%    {\envfont trivlist} environment.  By using these registers, we
%    obtain the desired space around a {\envfont theorem} environment.
%    \begin{macrocode}
   \trivlist
   \@topsep \theorempreskipamount               % used by first \item
   \@topsepadd \theorempostskipamount           % used by \@endparenv
%    \end{macrocode}
%    Now we have to test whether an optional argument has been given.
% \changes{v2.0d}{89/07/18}{Forgotten percent added.}
%    \begin{macrocode}
   \@ifnextchar [%
%    \end{macrocode}
%    If there is an optional argument, we will call |\@ythm|, and
%    move the arguments read back into the input stream.
%    \begin{macrocode}
   {\@ythm{#1}{#2}}%
%    \end{macrocode}
%    If not, we call |\@begintheorem|. Its first argument is the
%    name of the theorem set (hence the second argument of
%    |\@thm|). Its second argument is the macro that produces the
%    current number.
%    \begin{macrocode}
   {\@begintheorem{#2}{\csname the#1\endcsname}\ignorespaces}}
%    \end{macrocode}
% \end{macro}
%
% \begin{macro}{\@xthm}
% \begin{macro}{\@ythm}
%    Both these macros were originally called by |\@thm|.  We do
%    not need |\@xthm| anymore, hence we reset it to
%    |\relax|. The definition of |\@ythm| has not changed at
%    all from its definition in \LaTeX. In order to make the macros
%    easier to understand, we will
%    nevertheless present it (commented out).
%    \begin{macrocode}
\global\let\@xthm\relax
% \def\@ythm#1#2[#3]{\@opargbegintheorem{#2}{\csname
%           the#1\endcsname}{#3}\ignorespaces}
%</package>
%    \end{macrocode}
%    The primitive |\ignorespaces| in |\@ythm| and
%    |\@thm| is needed to remove the spaces between the
%    |\begin{...}| and the actual text.
% \end{macro}
% \end{macro}
%
%
% \subsection{Definition of the theorem substyles}
%
% As already pointed out, the theorem substyles, defined below, are
% only loaded when necessary.  Note that all these substyles, except
% \texttt{plain}, have |\slshape| as the default body font.
%
% \subsubsection{The \texttt{plain} style}
%
% \label{dok:th@plain} As the following macros use |@|, we have
% to locally set the |\catcode| of this symbol to ``letter''.
% This happens within a group, so that we do not have to worry about
% which |\catcode| that symbol had before.
%    \begin{macrocode}
%<*thp>
\begingroup \makeatletter
%    \end{macrocode}
% Since we are now within a group, we must make all definitions
% globally.  First we make sure that \texttt{theorem.sty} is loaded.
% This will allow us to use this file as a document style option
% without having to call \texttt{theorem} itself as an option.  At the
% same time, we assure that at least version 2 is loaded, since
% |\theorem@style| was not defined in earlier versions.
%    \begin{macrocode}
\@ifundefined{theorem@style}{\input{theorem.sty}}{}
\ProvidesFile{thp.sty}
             [\FMithmInfo]
%    \end{macrocode}
%
%
% \begin{macro}{\th@plain}
% \begin{macro}{\@begintheorem}
% \begin{macro}{\@opargbegintheorem}
%    |\theoremstyle{plain}| corresponds to the original
%    definition, except that the distances to the surrounding text are
%    determined by the parameters |\theorempreskipamount| and
%    |\theorempostskipamount|.
%    First we set the default body font.
%    \begin{macrocode}
\gdef\th@plain{\normalfont\itshape
%    \end{macrocode}
%    Then we define |\@begintheorem| and
%    |\@opargbegintheorem|. These two macros define how the
%    header of a theorem is typeset.  |\@opargbegintheorem| will
%    be called if a \textsf{theorem} environment with an optional
%    argument is encountered; otherwise, the header is constructed by
%    calling |\@begintheorem|.  If one of these macros is
%    executed, we are within a \textsf{trivlist} environment started by
%    |\@thm|.  So the theorem header is produced with an
%    |\item| command.
%
%    Instead of specifying the header font directly, all standard
%    theorem styles use the |\theorem@headerfont| macro to allow
%    customization.  The extra space (|\labelsep|) is necessary
%    because of problems in the \textsf{trivlist} environment.
%    \begin{macrocode}
  \def\@begintheorem##1##2{%
        \item[\hskip\labelsep \theorem@headerfont ##1\ ##2]}%
%    \end{macrocode}
%    The definition of |\@opargbegintheorem| is completely
%    analogous. The only difference is the fact that there exists a
%    third argument (which is the optional parameter of the
%    environment and contains additional information about the
%    theorem).  Customarily we enclose it in parentheses.
%    \begin{macrocode}
\def\@opargbegintheorem##1##2##3{%
   \item[\hskip\labelsep \theorem@headerfont ##1\ ##2\ (##3)]}}
%    \end{macrocode}
% \end{macro}
% \end{macro}
% \end{macro}
%
% \noindent
% We conclude with an |\endgroup| to restore the
% |\catcode| of |@|.
%    \begin{macrocode}
\endgroup
%</thp>
%    \end{macrocode}
%
%
% \subsubsection{The \texttt{break} style}
%
% This style option is stored in the file |thb.sty|.
% For the next two lines see the documentation for |\th@plain| on
% page~\pageref{dok:th@plain}.
%    \begin{macrocode}
%<*thb>
\begingroup \makeatletter
\@ifundefined{theorem@style}{\input{theorem.sty}}{}
\ProvidesFile{thb.sty}
             [\FMithmInfo]
%    \end{macrocode}
%
% \begin{macro}{\th@break}
%    |\theoremstyle{break}| produces a line break after the name
%    of the theorem.  The font is |\slshape|.  Hence, we define
%    |\th@break| as follows:
%    \begin{macrocode}
\gdef\th@break{\normalfont\slshape
  \def\@begintheorem##1##2{\item[%
%    \end{macrocode}
%    We run into the following problem: it is not possible to create
%    the header with |\item[|\meta{title}|]| and then start
%    a new line by, for example, |\mbox{}\\|. Such a definition
%    will fail whenever a list environment follows immediately.  With
%    the above construction, the |\mbox{}| causes the switch
%    |@inlabel| (cf.\ definition of |\list| and
%    |\trivlist| in \cite{latex:source}) to be set to \texttt{false}
%    and so the following list will insert additional vertical
%    space (|\topskip|). This is quite annoying. Therefore, we
%    create the line break within the |\item|. In order to ensure
%    that the text will begin at the proper position in the following
%    line, we simply pretend that the label does not take any
%    room.\footnote{This will lead to problems whenever very high
%    symbols occurring in the line tower into the
%    heading. So, something else has to be done here sometime.}
%    \begin{macrocode}
       \rlap{\vbox{\hbox{\hskip \labelsep\theorem@headerfont ##1\ ##2}%
                   \hbox{\strut}}}]}%
%    \end{macrocode}
%    Again, the definition of |\@opargbegintheorem| is
%    completely analogous.
%    \begin{macrocode}
\def\@opargbegintheorem##1##2##3{%
  \item[\rlap{\vbox{\hbox{\hskip \labelsep \theorem@headerfont
                     ##1\ ##2\ (##3)}%
                    \hbox{\strut}}}]}}
\endgroup
%</thb>
%    \end{macrocode}
% \end{macro}
%
%
% \subsubsection{The \texttt{changebreak} style}
%
%    \begin{macrocode}
% This style option is stored in the file |thcb.sty|.
%    \begin{macrocode}
%<*thcb>
\begingroup \makeatletter
\@ifundefined{theorem@style}{\input{theorem.sty}}{}
\ProvidesFile{thcb.sty}
             [\FMithmInfo]
%    \end{macrocode}
%
% \begin{macro}{\th@changebreak}
%    The \texttt{change-break} theorem style is like \texttt{break}
%    but with interchange of theorem name and theorem number.  Thus we
%    define |\th@changebreak| as follows:
%    \begin{macrocode}
\gdef\th@changebreak{\normalfont\slshape
  \def\@begintheorem##1##2{\item
       [\rlap{\vbox{\hbox{\hskip\labelsep \theorem@headerfont ##2\ ##1}%
                    \hbox{\strut}}}]}%
\def\@opargbegintheorem##1##2##3{%
  \item[\rlap{\vbox{\hbox{\hskip\labelsep \theorem@headerfont
                     ##2\ ##1\ (##3)}%
                    \hbox{\strut}}}]}}
\endgroup
%</thcb>
%    \end{macrocode}
% \end{macro}
%
%
% \subsubsection{The \texttt{change} style}
%
% This style option is stored in the file |thc.sty|.
%    \begin{macrocode}
%<*thc>
\begingroup \makeatletter
\@ifundefined{theorem@style}{\input{theorem.sty}}{}
\ProvidesFile{thc.sty}
             [\FMithmInfo]
%    \end{macrocode}
%
% \begin{macro}{\th@change}
%    The \texttt{change} theorem style corresponds to the
%    \texttt{change break} style without a linebreak after the header.
%    To say it in another way, it's the same as the \texttt{plain}
%    style but with number and name interchanged and |\slshape| as the
%    default font.
%    \begin{macrocode}
\gdef\th@change{\normalfont\slshape
  \def\@begintheorem##1##2{\item
       [\hskip\labelsep \theorem@headerfont ##2\ ##1]}%
\def\@opargbegintheorem##1##2##3{%
  \item[\hskip\labelsep \theorem@headerfont ##2\ ##1\ (##3)]}}
\endgroup
%</thc>
%    \end{macrocode}
% \end{macro}
%
%
% \subsubsection{The \texttt{marginbreak} style}
%
% This style option is the one used most often at Mainz.  It is saved
% in the file |thmb.sty|.
%    \begin{macrocode}
%<*thmb>
\begingroup \makeatletter
\@ifundefined{theorem@style}{\input{theorem.sty}}{}
\ProvidesFile{thmb.sty}
             [\FMithmInfo]
%    \end{macrocode}
%
% \begin{macro}{\th@marginbreak}
%    The \texttt{margin break} style is nearly the same as the
%    \texttt{change break} style. The only difference is the placement
%    of the theorem number.  We use |\llap| to place it in the left
%    margin.
%
%    In this style |\labelsep| denotes the separation between the
%    number and the text.
%    \begin{macrocode}
\gdef\th@marginbreak{\normalfont\slshape
  \def\@begintheorem##1##2{\item
       [\rlap{\vbox{\theorem@headerfont
                    \hbox{\llap{##2}\hskip\labelsep ##1}%
                    \hbox{\strut}}}]}%
\def\@opargbegintheorem##1##2##3{%
  \item[\rlap{\vbox{\theorem@headerfont
                    \hbox{\llap{##2}\hskip\labelsep ##1\ (##3)}%
                    \hbox{\strut}}}]}}
\endgroup
%</thmb>
%    \end{macrocode}
% \changes{v2.1c}{90/02/17}{`endgroup outside macrocode env.}
% \end{macro}
%
%
%
% \subsubsection{The \texttt{margin} style}
%
% This style option is stored in the file |thm.sty|.
%    \begin{macrocode}
%<*thm>
\begingroup \makeatletter
\@ifundefined{theorem@style}{\input{theorem.sty}}{}
\ProvidesFile{thm.sty}
             [\FMithmInfo]
%    \end{macrocode}
%
% \begin{macro}{\th@margin}
%    Again this is only a variant of the theorem styles described
%    above without any new ideas.
%    \begin{macrocode}
\gdef\th@margin{\normalfont\slshape
  \def\@begintheorem##1##2{\item
       [\theorem@headerfont \llap{##2}\hskip\labelsep ##1]}%
\def\@opargbegintheorem##1##2##3{%
  \item[\theorem@headerfont \llap{##2}\hskip\labelsep ##1\ (##3)]}}
\endgroup
%</thm>
%    \end{macrocode}
% \end{macro}
%
%
%
%
% \subsection{Final Definitions}
%
% \begin{macro}{\theorempreskipamount}
% \begin{macro}{\theorempostskipamount}
%    The \textsf{skip} parameters that regulate the vertical empty space
%    before and after the {\envfont theorem} environment have
%    to be allocated as well.
% \changes{v2.0f}{89/07/20}{Unnecessary `global' removed.}
%    \begin{macrocode}
%<*package>
\newskip\theorempreskipamount
\newskip\theorempostskipamount
%    \end{macrocode}
%    Since we have used the same values for all theorem sets, we now can
%    assign them.
%    \begin{macrocode}
\global\setlength\theorempreskipamount{12pt plus 5pt minus 3pt}
\global\setlength\theorempostskipamount{8pt plus 3pt minus 1.5pt}
%    \end{macrocode}
% \end{macro}
% \end{macro}
%
%
% \begin{macro}{\@endtheorem}
%    The same holds for the macro |\@endtheorem|, which ends a
%    {\envfont theorem} environment. Since it is the same for all
%    theorem sets, it is removed from the macros |\th@|\meta{style}.
%    It simply ends the {\envfont trivlist} environment, which was
%    begun in |\@thm|.
%    \begin{macrocode}
\global\let\@endtheorem=\endtrivlist
%    \end{macrocode}
% \end{macro}
%
%
%
% \begin{macro}{\@preamblecmds}
%    All macros defined above are to be used only in the preamble.
%    Therefore, we insert them in |\@preamblecmds| which will disable
%    them at begin document. This is done by the internal \LaTeXe{}
%    command |\@onlypreamble|.
% \changes{v2.0g}{89/09/19}{`theoremheaderfont and `theorembodyfont
%    added.}
% \changes{v2.2a}{94/02/03}{Use \LaTeXe{} interface}
%    \begin{macrocode}
\@onlypreamble\@xnthm
\@onlypreamble\@ynthm
\@onlypreamble\@othm
\@onlypreamble\newtheorem
\@onlypreamble\theoremstyle
\@onlypreamble\theorembodyfont
\@onlypreamble\theoremheaderfont
%    \end{macrocode}
% \end{macro}
%
% Finally we declare the \texttt{plain} theorem style to be the default.
% \changes{v2.1b}{90/06/04}{Fixed missing `item by declaring default
%    style.}
%    \begin{macrocode}
\theoremstyle{plain}
%</package>
%    \end{macrocode}
%
% \Finale

\endinput
}{}
\ProvidesFile{thmb.sty}
             [\FMithmInfo]
%    \end{macrocode}
%
% \begin{macro}{\th@marginbreak}
%    The \texttt{margin break} style is nearly the same as the
%    \texttt{change break} style. The only difference is the placement
%    of the theorem number.  We use |\llap| to place it in the left
%    margin.
%
%    In this style |\labelsep| denotes the separation between the
%    number and the text.
%    \begin{macrocode}
\gdef\th@marginbreak{\normalfont\slshape
  \def\@begintheorem##1##2{\item
       [\rlap{\vbox{\theorem@headerfont
                    \hbox{\llap{##2}\hskip\labelsep ##1}%
                    \hbox{\strut}}}]}%
\def\@opargbegintheorem##1##2##3{%
  \item[\rlap{\vbox{\theorem@headerfont
                    \hbox{\llap{##2}\hskip\labelsep ##1\ (##3)}%
                    \hbox{\strut}}}]}}
\endgroup
%</thmb>
%    \end{macrocode}
% \changes{v2.1c}{90/02/17}{`endgroup outside macrocode env.}
% \end{macro}
%
%
%
% \subsubsection{The \texttt{margin} style}
%
% This style option is stored in the file |thm.sty|.
%    \begin{macrocode}
%<*thm>
\begingroup \makeatletter
\@ifundefined{theorem@style}{% \iffalse meta-comment
%
% Copyright (C) 1993-2024
%
% The LaTeX Project and any individual authors listed elsewhere
% in this file.
%
% This file is part of the Standard LaTeX `Tools Bundle'.
% -------------------------------------------------------
%
% It may be distributed and/or modified under the
% conditions of the LaTeX Project Public License, either version 1.3c
% of this license or (at your option) any later version.
% The latest version of this license is in
%    https://www.latex-project.org/lppl.txt
% and version 1.3c or later is part of all distributions of LaTeX
% version 2005/12/01 or later.
%
% The list of all files belonging to the LaTeX `Tools Bundle' is
% given in the file `manifest.txt'.
%
% \fi
%
%%
%
%
\def\FMithmInfo{2023/07/05 v2.2c Theorem extension package (FMi)}
%
% \ProvidesFile{theorem.dtx}[\FMithmInfo]
%\iffalse   % this is a METACOMMENT !
%%
%% Package `theorem' to use with LaTeX2e
%% Copyright (C) 1989-1995 Frank Mittelbach, all rights reserved.
%<+package>\@ifundefined{theorem@style}{}{\endinput}
%<+package>\NeedsTeXFormat{LaTeX2e}[1995/06/01]
%<+package>\ProvidesPackage{theorem}
%<+package>         [\FMithmInfo]
%
% \fi
%
% \changes{v1.0}{87/05/07}{First draft.}
% \changes{v1.0a}{87/05/07}{`thistheoremstyle' added.}
% \changes{v1.1a}{87/07/29}{New concept: redefine the internal
%                          \protect\LaTeX{} commands `@xnthm' etc.}
% \changes{v1.1b}{87/07/31}{`theoremfont' used to specify font.}
% \changes{v1.2a}{88/01/08}{`@endtheorem' now globally defined.}
% \changes{v1.2b}{88/01/11}{`newtheorem' only allowed in the preamble.}
% \changes{v1.3a}{88/02/14}{`theoremfont' now a toks register.}
%
% \changes{v2.0a}{88/11/16}{New implementation and new documentation.}
% \changes{v2.0a}{88/11/16}{No layout changes but internal modifications
%                           old substyles won't work.}
% \changes{v2.0b}{89/01/30}{`global' used for all definitions.}
% \changes{v2.0c}{89/06/15}{Documentation translated to English.
%                            Thanks to LLL and Nina}
% \changes{v2.1a}{90/02/15}{Support for new font selection.}
% \changes{v2.2a}{94/02/03}{Cleanup for LaTeX2e}
%
% \DoNotIndex{\@endparenv,\@ifdefinable,\@ifnextchar,\@ifundefined}
% \DoNotIndex{\@namedef,\@nameuse,\@spaces,\@tempa,\@warning}
% \DoNotIndex{\ }
% \DoNotIndex{\begingroup,\bfseries}
% \DoNotIndex{\csname}
% \DoNotIndex{\def,\do,\docdate}
% \DoNotIndex{\endcsname,\endgroup,\endinput,\endtrivlist,\expandafter}
% \DoNotIndex{\filedate,\fileversion}
% \DoNotIndex{\gdef,\global}
% \DoNotIndex{\hbox,\hskip}
% \DoNotIndex{\ignorespaces,\input,\itshape,\item}
% \DoNotIndex{\labelsep,\let,\llap}
% \DoNotIndex{\makeatletter}
% \DoNotIndex{\noexpand}
% \DoNotIndex{\refstepcounter,\relax,\rlap}
% \DoNotIndex{\setlength,\slshape,\space,\string,\strut}
% \DoNotIndex{\the,\trivlist,\typeout}
% \DoNotIndex{\vbox}
% \DoNotIndex{\xdef}
%
% \def\deflabel#1{\ttfamily #1\hfill}
% \def\deflist#1{\begin{list}{}{\settowidth\labelwidth{\ttfamily #1}%
%                               \setlength\leftmargin\labelwidth
%                               \addtolength\leftmargin\labelsep
%                               \let\makelabel\deflabel}}
% \def\enddeflist{\end{list}}
% \def\envfont{\normalfont\ttfamily}
%
% \MakeShortVerb{\|}
%
% \newcommand{\TUB}{{\sl TUGboat\/}}
%
% \GetFileInfo{theorem.dtx}
% \title{An Extension of the
%        \LaTeX{} theorem environment\thanks{This file
%        has version number \fileversion, last
%        revised \filedate.}}
% \author{Frank Mittelbach}
%
% \MaintainedByLaTeXTeam{tools}
% \maketitle
%
% \begin{abstract}
%    The macros described in this paper yield an extension of the
%    \LaTeX{} theorem mechanism. It is designed to satisfy the
%    different requirements of various journals. Thus, the
%    layout of the ``theorems'' can be manipulated by determining a
%    ``style''. This article describes not only the use, but
%    also the definition, of the necessary macros.
% \end{abstract}
%
% \section*{Preface to version 2.2}
%
% For \LaTeXe{} this package did not need any fundamental changes. I
% only modified the messages generated so that theorem layout styles
% will show up with the |\listfiles| command and cleaned the section
% on the New Font Selection Scheme since this is now included in
% \LaTeX.
%
% \section*{Preface to version 2.1}
%
% This version is identical to 2.0g described in \TUB\ 10\#3 except
% for some internal defaults which are now set depending on the used
% font selection scheme.
%
% This was done to avoid unpleasant surprises if the new font selection
% scheme is in force. For further details see
% section~\ref{sec:newfonts} and \cite{bk:GMS94}.
%
% \section{Introduction}
%
% For our purposes here, ``theorems'' are labelled enunciations,
% often set off from the main text by extra space and a font change.
% Theorems, corollaries, conjectures, definitions, and remarks
% are all instances of ``theorems''.  The ``header'' of these structures
% is composed of a label (such as \textsc{Theorem} or \textsc{Remark})
% and a number which serializes an item in the sequence of items
% with the same label.
%
% Shortly after the introduction of \LaTeX{} at the Fachbereich
% Mathematik in Mainz, the desire to manipulate the layout of
% ``theorems'' arose.  In Mainz, the following two conventions came
% into general use:
% \begin{enumerate}
%   \item The number of the theorem is shown in the margin.
%   \item There is a line break at
%         the end of the theorem header.
% \end{enumerate}
%
% Additionally, some journals require different formats which depend on
% the ``sort of theorem'': e.g.\ often remarks and definitions are set
% in |\upshape|, while |\itshape| is employed for main theorems.
%
% Confronted with these requirements, a theorem environment was
% developed in Mainz which allows separate determination of the
% layout of the ``theorems sets'', comparable to |\pagestyle|.
%
%
% \section{The user interface}
% \subsection{Defining new theorem sets}
%
% \DescribeMacro\newtheorem
% As in the original \LaTeX{} version, the command |\newtheorem|
% defines a new ``theorem set'' or ``theorem-like structure''.
% Two required arguments name the new environment and give the text
% to be typeset with each instance of the new ``set'', while
% an optional argument determines how
% the ``set'' is enumerated:
% \begin{description}
%    \item[\ttfamily \bslash newtheorem\{foo\}\{bar\}]
%       The theorem set {\envfont foo} (whose name is \texttt{bar})
%       uses its own counter.
%    \item[\ttfamily \bslash newtheorem\{foo2\}{[foo]}\{bar2\}]
%       The theorem set {\envfont foo2} (printed name \texttt{bar2})
%       uses the same counter as the theorem set \texttt{foo}.
%    \item[\ttfamily \bslash newtheorem\{foo3\}\{bar3\}{[section]}]
%       The theorem set {\envfont foo3} (printed name \texttt{bar3}) is
%       enumerated within the counter \texttt{section}, i.e.\ with every
%       new |\section| the enumeration begins again with 1, and
%       the enumeration is composed from the section-number and the
%       theorem counter itself.
% \end{description}
%
% \DescribeMacro\theoremstyle
% Additionally, the command |\theoremstyle| can define the layout
% of various, or all, theorem sets.  It should be noted that any
% theorem set defined by |\newtheorem| is typeset in the
% |\theoremstyle| that is current at the time of the definition.
% Thus, the following
% \begin{verbatim}
%    \theoremstyle{break}        \newtheorem{Cor}{Corollary}
%    \theoremstyle{plain}        \newtheorem{Exa}{Example}[section]
%\end{verbatim}                             ^^A No blank at line start
% leads to the result that the set {\envfont Cor} is formatted in the
% style \texttt{break}, while the set {\envfont Exa} and all the
% following ones are formatted in the style \texttt{plain}, unless
% another |\theoremstyle| follows.  Since the definitions installed by
% |\newtheorem| are global, one also can limit |\theoremstyle| locally
% by grouping braces.
%
% \DescribeMacro\theorembodyfont
% The choice of the font for the theorem body is completely
% independent of the chosen |\theoremstyle|; this has proven to
% be very advantageous.  For example,
% \begin{verbatim}
%   {\theorembodyfont{\upshape}          \newtheorem{Rem}{Remark}}
%\end{verbatim}
% defines a theorem set {\envfont Rem}, which will be set in
% |\upshape| in the current layout (which in our example is
% \texttt{plain}).  As with |\theoremstyle|, the |\theorembodyfont|
% chosen is that current at the time of |\newtheorem|.  If
% |\theorembodyfont| is not specified or one defines
% |\theorembodyfont{}|, then the font used will be that defined by the
% |\theoremstyle|.
%
% \DescribeMacro\theoremheaderfont It is also possible to customize
% the font used for the theorem headers.  This is, however, a global
% declaration, and therefore there should be at most one
% |\theoremheaderfont| declaration in the preamble.\footnote{If it is
% actually necessary to have different header fonts, one has to define
% new theorem styles (substituting the desired font) or specify the
% information directly in the \texttt{\bslash newtheorem} declaration
% (the unclean variant).}
%
% \DescribeMacro\theorempreskipamount
% \DescribeMacro\theorempostskipamount Two additional parameters
% affect the vertical space around the theorem environments:
% |\theorempreskipamount| and |\theorempostskipamount| define,
% respectively, the spacing before and after such an environment.
% These parameters apply for all theorem sets and can be manipulated
% with the ordinary length macros.  They are rubber lengths
% (`\textsf{skips}'), and therefore can contain \texttt{plus} and
% \texttt{minus} parts.
%
% Since the definition of theorem sets should---most sensibly---be
% placed in the preamble, we only allow installation there.
% It is therefore possible to release the memory used here after
% |\begin{document}|, in order to make room for other
% applications.
%
%
% \subsection{Existing theorem styles}
%
% The following theorem styles exist to date:
% \begin{deflist}{marginbreak}
%    \item[plain]
%       This theorem style emulates the original \LaTeX{} definition,
%       except that additionally the parameters
%       |\theorem...skipamount| are used.
%    \item[break]
%       In this style, the theorem header is followed by a line
%       break.
%    \item[marginbreak]
%       The theorem number is set in the margin, and there is a line
%       break as in \texttt{break}.
%    \item[changebreak]
%       Like \texttt{break}, but with header number and text
%       interchanged.
%    \item[change]
%       Header number and text are interchanged, without a line break.
%    \item[margin]
%       The number is set in the left margin, without a line break.
% \end{deflist}
% All styles (except \texttt{plain}) select |\slshape| as the default
% |\theorembodyfont|.
%
%
% \subsection{Examples}
%
% Given the above theorem sets {\envfont Cor}, {\envfont Exa}
% and {\envfont Rem},
% suppose that the preamble also contains the declarations:
% \begin{verbatim}
%    \theoremstyle{marginbreak}   \newtheorem{Lem}[Cor]{Lemma}
%    \theoremstyle{change}
%    \theorembodyfont{\itshape}        \newtheorem{Def}[Cor]{Definition}
%
%    \theoremheaderfont{\scshape}
%\end{verbatim}
% Then the following are some typical examples of the typeset output
% resulting from their use.
%
% \begin{Cor}
%    This is a sentence typeset in the theorem environment {\envfont
%    Cor}.
% \end{Cor}
%
% \begin{Exa}
%    This is a sentence typeset in the theorem environment {\envfont
%    Exa}.
% \end{Exa}
%
% \begin{Rem}
%    This is a sentence typeset in the theorem environment {\envfont
%    Rem}.
% \end{Rem}
%
% \begin{Lem}[Ben User]
%    This is a sentence typeset in the theorem environment {\envfont
%    Lem}.
% \end{Lem}
%
% \begin{Def}[Very impressive Definition]
%    This is a sentence typeset in the theorem environment {\envfont
%    Def}.
% \end{Def}
%
% The last two examples show the effect of the optional argument to
% a theorem environment (it is the text typeset in parentheses).
%
%
% \section{Special Considerations}
% \label{sec:newfonts}
%
% Theoremheader and body are implemented as a unit. This means that
% the |\theoremheaderfont|  will inherit characteristics of the
% |\theorembodyfont| in \LaTeXe. Thus, if for example
% |\theorembodyfont| is |\itshape| and |\theoremheaderfont|
% is |\bfseries| the font selected for the header will have the
% characteristics `bold extended italic'. If this is not desired one
% should set the |\theoremheaderfont| to something like
% \begin{verbatim}
%      \theoremheaderfont{\normalfont\bfseries}
%\end{verbatim}
% i.e.\ supplying all necessary font information explicitly.
%
%
%
% \section{Acknowledgements}
%
% The publication of this set of macros was only possible with the
% help of Christina Busse (translating the manuscript into English),
% Joachim Pense (playing the r\^ole of typist), Chris Rowley
% (looking everything over) and many others providing useful
% suggestions.
%
% \MaybeStop{%
%    \begin{thebibliography}{9}
%       \bibitem{bk:GMS94} \textsc{M.~Goossens}, \textsc{F.~Mittelbach}
%         and \textsc{A.~Samarin}.
%         \newblock The \LaTeX{} Companion.
%         \newblock
%         Addison-Wesley, Reading, Massachusetts, 1994.
%       \bibitem{latex:source} \textsc{Lamport, Leslie.} \newblock
%                               \texttt{latex.tex}, version 2.09, date
%                               Feb. 1990.
%    \end{thebibliography}
%    \PrintIndex
%    \PrintChanges}
%
%
% \section{The documentation driver file}
%
% The next bit of code contains the documentation driver file for
% \TeX{}, i.e., the file that will produce the documentation you are
% currently reading. It will be extracted from this file by the
% \texttt{docstrip} program. Since it is the first code in the file
% one can alternatively process this file directly with \LaTeXe{} to
% obtain the documentation.
%
%    \begin{macrocode}
%<*driver>
\documentclass{ltxdoc}

\usepackage{theorem}

%<+driver>% The next few lines define theorem sets which are used
%<+driver>% in the example section of the documentation.

\theoremstyle{break}       \newtheorem{Cor}{Corollary}
\theoremstyle{plain}       \newtheorem{Exa}{Example}[section]
{\theorembodyfont{\upshape}\newtheorem{Rem}{Remark}}
\theoremstyle{marginbreak} \newtheorem{Lem}[Cor]{Lemma}
\theoremstyle{change}
\theorembodyfont{\itshape} \newtheorem{Def}[Cor]{Definition}

\theoremheaderfont{\scshape}

\RecordChanges

\begin{document}
   \DocInput{theorem.dtx}
\end{document}
%</driver>
%    \end{macrocode}
%
% \section{Definition of the Macros}
%
% If the file has been loaded before, we abort immediately. If not the
% package announces itself (this is actually done at the very top of
% the file---the way it is done isn't good style so don't copy it).
% \changes{v2.0e}{89/07/19}{Spaces removed from `typeout'.}
% \changes{v2.2b}{95/11/19}{Announce moved to top of file}
%    \begin{macrocode}
%<*package>
%\@ifundefined{theorem@style}{}{\endinput}
%\def\FMithmInfo{1995/11/19 v2.2b Theorem extension package (FMi)}
%\ProvidesPackage{theorem}[\FMithmInfo]
%    \end{macrocode}
%
%
% \subsection{Definition of theorem styles and fonts}
%
% All the definitions in this file are done globally to allow
% inputting this file inside a group.
%
% \begin{macro}{\theoremstyle}
%    Before a theorem style can be installed, the chosen style must be
%    known. For that reason, we must test to see that
%    |\th@|\meta{style} is known or, more precisely, that it is
%    different from |\relax|. If the style is not known then
%    |\th@plain| is used.
%    \begin{macrocode}
\gdef\theoremstyle#1{%
   \@ifundefined{th@#1}{\@warning
          {Unknown theoremstyle `#1'. Using `plain'}%
          \theorem@style{plain}}%
%    \end{macrocode}
%    We save the theorem style to be used in the token register
%    |\theorem@style|.
%    \begin{macrocode}
      {\theorem@style{#1}}%
%    \end{macrocode}
%    Now we ``evaluate'' the theorem style: this means, we call the
%    macro |\th@|\meta{style} which will activate the relevant
%    definitions which are contained in a separate file.  This is done
%    in a group to suppress changes to the current font.  This could
%    otherwise pose problems together with the new font selection
%    scheme\footnote{When I printed the original article using the new
%    font selection scheme I ended with a document with slanted
%    typefaces (text headings and all) simply because one of the
%    theorem styles used \texttt{\bslash sl} at toplevel.} if the
%    |\th@|\meta{style} is evaluated a second time.
% \changes{v2.1a}{90/02/15}{Group added around `th@..}
% \changes{v2.1a}{90/02/15}{Reseting of `..@begintheorem removed.}
%    \begin{macrocode}
      \begingroup
        \csname th@\the\theorem@style \endcsname
      \endgroup}
%    \end{macrocode}
% \end{macro}
%
%
% \begin{macro}{\@begintheorem}
% \changes{v2.1a}{90/02/15}{Relaxed at toplevel.}
% \begin{macro}{\@opargbegintheorem}
% \changes{v2.1a}{90/02/15}{Relaxed at toplevel.}
%    We reset |\@begintheorem| and |\@opargbegintheorem| to
%    |\relax| since these commands are no longer necessary at
%    toplevel. This will save a few tokens.
%    \begin{macrocode}
\global\let\@begintheorem\relax
\global\let\@opargbegintheorem\relax
%    \end{macrocode}
% \end{macro}
% \end{macro}
%
% \begin{macro}{\theorem@style}
%    Obviously the \textsf{token} register used above has to be
%    allocated. To assure the utmost compatibility with the
%    original \LaTeX{} definition, we set the default theorem style to
%    \texttt{plain}, which implements the usual \LaTeX{} convention.
% \changes{v2.0f}{89/07/20}{Unnecessary `global' removed.}
%    \begin{macrocode}
\newtoks\theorem@style
\global\theorem@style{plain}
%    \end{macrocode}
% \end{macro}
%
% \begin{macro}{\theorembodyfont}
% \begin{macro}{\theorem@bodyfont}
%    If the |\theorembodyfont| is set by the user then it should not
%    interact with the default font set in the theorem style. When the
%    new font selection is in force this may happen if, for example,
%    the default is |\itshape| and the new |\theorembodyfont| is
%    |\sffamily|. So we add a |\reset@font| command in front of the
%    user definition.
% \changes{v2.0f}{89/07/20}{Unnecessary `global' removed.}
% \changes{v2.0g}{89/09/19}{`theoremfont to `theorembodyfont.}
% \changes{v2.1c}{92/06/22}{`reset@font added and logic changed.}
% \changes{v2.1d}{92/10/28}{Reset to empty if arg is empty}
%    \begin{macrocode}
\gdef\theorembodyfont#1{%
%    \end{macrocode}
%    We check if the argument supplied is empty and if so put nothing
%    into the |\theorem@bodyfont| token register to allow for
%    |\theorembodyfont{}| as a mean of using the default of the
%    current |\theoremstyle|.
%    \begin{macrocode}
   \def\@tempa{#1}%
   \ifx\@tempa\@empty
    \theorem@bodyfont{}%
   \else
    \theorem@bodyfont{\reset@font#1}%
   \fi
   }
\newtoks\theorem@bodyfont
\global\theorem@bodyfont{}
%    \end{macrocode}
% \end{macro}
% \end{macro}
%
%
% \begin{macro}{\theoremheaderfont}
% \changes{v2.0g}{89/09/19}{Macro added.}
%    The font for the theorem headers is handled differently because
%    this definition applies to all theorem styles.
%    \begin{macrocode}
\gdef\theoremheaderfont#1{\gdef\theorem@headerfont{#1}%
%    \end{macrocode}
%    After using the macro once it is redefined to produce an error
%    message.
%    \begin{macrocode}
       \gdef\theoremheaderfont##1{%
        \typeout{\string\theoremheaderfont\space should be used
                 only once.}}}
%    \end{macrocode}
% \end{macro}
%
% \begin{macro}{\theorem@headerfont}
% \changes{v2.0g}{89/09/19}{Macro added.}
%    To set the |\theorem@headerfont| default we first test if
%    the new fontselection scheme is in force.
% \changes{v2.1a}{90/02/15}{Support for new font selection.}
%    \begin{macrocode}
\ifx\upshape\undefined
%    \end{macrocode}
%    If not we define it to expand into |\bfseries|. We don't use
%    |\let| just in case a following style option redefines this
%    macro.
%    \begin{macrocode}
\gdef\theorem@headerfont{\bfseries}
%    \end{macrocode}
%    Otherwise we reset the current shape before calling |\bfseries|.
%    \begin{macrocode}
\else \gdef\theorem@headerfont{\normalfont\bfseries}\fi
%    \end{macrocode}
% \end{macro}
%
% \begin{macro}{\th@plain}
% \begin{macro}{\th@break}
% \begin{macro}{\th@marginbreak}
% \begin{macro}{\th@changebreak}
% \begin{macro}{\th@change}
% \begin{macro}{\th@margin}
%    The different styles are defined in macros such as
%    |\th@plain|.  Since memory space is precious in
%    ``non-Big-versions'', we have to avoid offering too many unused
%    definitions.  Therefore we define these styles in separate files
%    that can be loaded on demand.  Thus the commands themselves only
%    load these files. We use |\@input@|, a \LaTeXe{} internal command
%    that ensures that the file will be listed with |\listfiles|.
% \changes{v2.2a}{94/02/02}{Use `@input@ to load theorem layout files}
%    \begin{macrocode}
\gdef\th@plain{\@input@{thp.sty}}
\gdef\th@break{\@input@{thb.sty}}
\gdef\th@marginbreak{\@input@{thmb.sty}}
\gdef\th@changebreak{\@input@{thcb.sty}}
\gdef\th@change{\@input@{thc.sty}}
\gdef\th@margin{\@input@{thm.sty}}
%    \end{macrocode}
%    This list will be expanded when new styles become
%    available.  For testing,  just append new theorem substyles
%    as document options.
% \end{macro}
% \end{macro}
% \end{macro}
% \end{macro}
% \end{macro}
% \end{macro}
%
%
%
% \subsection{Definition of a new theorem set}
%
% As already pointed out, a new theorem environment can be defined in
% three different ways:
% \begin{verbatim}
%    \newtheorem{Lem}{Lemma}
%    \newtheorem{Lem}{Lemma}[section]
%    \newtheorem{Lem}[Theorem]{Lemma}
%\end{verbatim}
% The function of the macro |\newtheorem| is to recognize these cases
% and then to branch into one of the three macros |\@ynthm|, |\@xnthm|
% or |\@othm|.  This mechanism is adopted unchanged from
% \cite{latex:source}; the essential point here is that, for example,
% in the second case, the arguments \texttt{Lem}, \texttt{Lemma} and
% \texttt{section} are passed over to the macro |\@xnthm|.
%
% We inspect this case first because the others present fewer problems,
% and thus are easily derived from this one.
%
% \begin{macro}{\@xnthm}
%    For our example arguments, the macro |\@xnthm| must fulfill
%    the following:
%    \begin{itemize}
%       \item
%          Define a new \LaTeX-counter `Lem'
%       \item
%          reset this counter within a |\section|
%       \item
%          define the macro |\theLem|
%       \item
%          define the environment macros |\Lem| and
%          |\endLem| using the current
%          |\theoremstyle| and |\theorem@bodyfont|.
%    \end{itemize}
%    Obviously, all this should happen only if the first argument
%    of |\@xnthm| (i.e.\ \texttt{Lem} in our example) is chosen so
%    as not to conflict with any previously defined commands or
%    environments.  This test is performed by the \LaTeX{} macro
%    |\@ifdefinable|.
%    \begin{macrocode}
\gdef\@xnthm#1#2[#3]{\expandafter\@ifdefinable\csname #1\endcsname
%    \end{macrocode}
%    Therefore, the first argument of |\@ifdefinable| is the
%    expansion (in the example, |\Lem|) of
%    |\csname||#1\endcsname|.  The second argument is
%    executed only if the test has been completed successfully.
%    \begin{macrocode}
   {%
%    \end{macrocode}
%    Now we define the new counter. The names of the \LaTeX{}
%    macros employed should speak for themselves:
% \changes{v2.2b}{1995/10/07}{Use \cs{@newctr}}
% \changes{v2.2c}{1995/11/23}{Use \cs{@newctr} properly.}
%    \begin{macrocode}
    \@definecounter{#1}\@newctr{#1}[#3]%
%    \end{macrocode}
%    Using |\@newctr| will give a proper error message if the counter
%    in |#3| is not defined.
%    In defining `|\theLem|' we must generate the desired macro
%    name by use of |\expandafter| and |\csname|.
%    \begin{macrocode}
    \expandafter\xdef\csname the#1\endcsname
%    \end{macrocode}
%    An |\xdef| is used in order to make the definition global,
%    and to ensure that it contains the replacement texts of
%    |\@thmcountersep| and |\@thmcounter|.\footnote{These
%    two macros can be defined by the document style.  Their default
%    values produce a `\texttt{.}' as separation and an arabic
%    representation of the number.} However, not everything should be
%    expanded. For example, it saves space to use |\thesection|
%    instead of its---at times---lengthy expansion.
%    \begin{macrocode}
      {\expandafter \noexpand \csname the#3\endcsname
       \@thmcountersep \@thmcounter{#1}}%
%    \end{macrocode}
%    Thus with the defaults of \LaTeX{}, |\theLem| would be
%    replaced by the command sequence |\thesection.\arabic{Lem}|.
%
%    We will now look at the definition of the macro which is executed
%    at the beginning of the actual environment (in our example this
%    macro is |\Lem|). It should be noted that we use an
%    ``|\expandafter| trick'' to expand only certain parts of the
%    replacement text at the time of the definition.
%    \begin{macrocode}
    \def\@tempa{\global\@namedef{#1}}%
    \expandafter \@tempa \expandafter{%
%    \end{macrocode}
%    First, the macro that contains the current definitions of
%    |\@begintheorem| and |\@opargtheorem| should be called
%    up.  The name of this macro---as is already known---has the form
%    |\th@|\meta{theorem\ style}; therefore, it must be called by
%    \begin{macrocode}
      \csname th@\the \theorem@style
%    \end{macrocode}
%    In addition the default theorem font should be changeable, i.e.\
%    we have to insert the contents of |\theorem@bodyfont|. For
%    that reason, we expand even further, beyond |\endcsname|,
%    and thus insert the contents of the \textsf{token} register
%    |\theorem@bodyfont| in the replacement text.
%    \begin{macrocode}
            \expandafter \endcsname \the \theorem@bodyfont
%    \end{macrocode}
%    Now it is time to call the macro |\@thm| which takes over
%    the further processing.  It has two arguments: the current
%    counter name (in our example, \texttt{Lem}), and the text of the
%    label (in our example, \texttt{Lemma}).
%    \begin{macrocode}
     \@thm{#1}{#2}}%
%    \end{macrocode}
%    With this, the `sub-definition' is complete.  The macro
%    |\@endtheorem| ends a theorem environment and is, so far, nothing
%    but an |\endtrivlist|.  (Hence it is defined globally, and not
%    within the theorem styles.\footnote{This has to be changed as
%    soon as theorem styles that change \texttt{\bslash @endtheorem}
%    exist.  In such a case, all existing styles must be changed as
%    well since they will have to reset the macro.}) Therefore, we can
%    set it equivalent to the macro that ends the theorem set (in our
%    example, |\endLem|).  However, if some day theorem styles exist
%    that do change |\@endtheorem|, we would have to use the
%    commented-out line instead.
%    \begin{macrocode}
    \global \expandafter \let \csname end#1\endcsname \@endtheorem
%   \global\@namedef{end#1}{\@endtheorem}%
%    \end{macrocode}
%    With these commands all the required definitions are employed,
%    unless the test |\@ifdefinable| has failed.  Therefore, we
%    end the second argument of this macro and with it the definition
%    of |\@xnthm|.
%    \begin{macrocode}
   }}
%    \end{macrocode}
% \end{macro}
%
% \begin{macro}{\@ynthm}
%    The definition of |\@ynthm| is completely analogous. In
%    this case the new counter that is defined is not reset within
%    another counter; thus the definition of |\the...|
%    is simplified:
%    \begin{macrocode}
\gdef\@ynthm#1#2{\expandafter\@ifdefinable\csname #1\endcsname
   {\@definecounter{#1}%
    \expandafter\xdef\csname the#1\endcsname{\@thmcounter{#1}}%
%    \end{macrocode}
%    The rest of the definition corresponds literally to that of
%    |\@xnthm|:
%    \begin{macrocode}
    \def\@tempa{\global\@namedef{#1}}\expandafter \@tempa
     \expandafter{\csname th@\the \theorem@style \expandafter
     \endcsname \the\theorem@bodyfont \@thm{#1}{#2}}%
    \global \expandafter \let \csname end#1\endcsname \@endtheorem}}
%    \end{macrocode}
% \end{macro}
%
% \begin{macro}{\@othm}
%    The definition of |\@othm| does not contain anything new.
%    \begin{macrocode}
\gdef\@othm#1[#2]#3{%
%    \end{macrocode}
% \changes{v2.2b}{95/10/07}{Build in check for counter defined}
%    We do not define a new counter but instead use one that has
%    already been defined.  Thus the only definition we need is that
%    of this pseudo-counter (i.e.\ |\the|\meta{env.\ name}).
%    First we check if |#2| corresponds to a known counter name.
%    \begin{macrocode}
  \expandafter\ifx\csname c@#2\endcsname\relax
   \@nocounterr{#2}%
  \else
   \expandafter\@ifdefinable\csname #1\endcsname
   {\expandafter \xdef \csname the#1\endcsname
     {\expandafter \noexpand \csname the#2\endcsname}%
%    \end{macrocode}
%    All other parts of the definition can be adopted from
%    |\@xnthm|. We have to remember, though, that in this case
%    the name of the current counter and the theorem label have moved
%    to the second and third arguments.
%    \begin{macrocode}
    \def\@tempa{\global\@namedef{#1}}\expandafter \@tempa
     \expandafter{\csname th@\the \theorem@style \expandafter
     \endcsname \the\theorem@bodyfont \@thm{#2}{#3}}%
    \global \expandafter \let \csname end#1\endcsname \@endtheorem}%
  \fi}
%    \end{macrocode}
% \end{macro}
%
% \subsection{Macros that are employed in a {\envfont theorem}
%    environment}
%
% \begin{macro}{\@thm}
%    The macro |\@thm| has to increase the current counter.  Then,
%    depending on whether the environment has (or does not have) an
%    optional argument, it has to branch into either |\@begintheorem|
%    or |\@opargtheorem|.
%    \begin{macrocode}
\gdef\@thm#1#2{\refstepcounter{#1}%
%    \end{macrocode}
%    Now we start a {\envfont trivlist} environment, and give
%    |\@topsep| and |\@topsepadd| the values of the \textsf{skip}
%    registers |\theorempreskipamount| and |\theorempostskipamount|.
%    The value in |\@topsep| is the vertical space that is inserted by
%    the first (and only) |\item| in our |\trivlist| whilst
%    |\@topsepadd| is inserted by |\@endparenv| at the end of that
%    {\envfont trivlist} environment.  By using these registers, we
%    obtain the desired space around a {\envfont theorem} environment.
%    \begin{macrocode}
   \trivlist
   \@topsep \theorempreskipamount               % used by first \item
   \@topsepadd \theorempostskipamount           % used by \@endparenv
%    \end{macrocode}
%    Now we have to test whether an optional argument has been given.
% \changes{v2.0d}{89/07/18}{Forgotten percent added.}
%    \begin{macrocode}
   \@ifnextchar [%
%    \end{macrocode}
%    If there is an optional argument, we will call |\@ythm|, and
%    move the arguments read back into the input stream.
%    \begin{macrocode}
   {\@ythm{#1}{#2}}%
%    \end{macrocode}
%    If not, we call |\@begintheorem|. Its first argument is the
%    name of the theorem set (hence the second argument of
%    |\@thm|). Its second argument is the macro that produces the
%    current number.
%    \begin{macrocode}
   {\@begintheorem{#2}{\csname the#1\endcsname}\ignorespaces}}
%    \end{macrocode}
% \end{macro}
%
% \begin{macro}{\@xthm}
% \begin{macro}{\@ythm}
%    Both these macros were originally called by |\@thm|.  We do
%    not need |\@xthm| anymore, hence we reset it to
%    |\relax|. The definition of |\@ythm| has not changed at
%    all from its definition in \LaTeX. In order to make the macros
%    easier to understand, we will
%    nevertheless present it (commented out).
%    \begin{macrocode}
\global\let\@xthm\relax
% \def\@ythm#1#2[#3]{\@opargbegintheorem{#2}{\csname
%           the#1\endcsname}{#3}\ignorespaces}
%</package>
%    \end{macrocode}
%    The primitive |\ignorespaces| in |\@ythm| and
%    |\@thm| is needed to remove the spaces between the
%    |\begin{...}| and the actual text.
% \end{macro}
% \end{macro}
%
%
% \subsection{Definition of the theorem substyles}
%
% As already pointed out, the theorem substyles, defined below, are
% only loaded when necessary.  Note that all these substyles, except
% \texttt{plain}, have |\slshape| as the default body font.
%
% \subsubsection{The \texttt{plain} style}
%
% \label{dok:th@plain} As the following macros use |@|, we have
% to locally set the |\catcode| of this symbol to ``letter''.
% This happens within a group, so that we do not have to worry about
% which |\catcode| that symbol had before.
%    \begin{macrocode}
%<*thp>
\begingroup \makeatletter
%    \end{macrocode}
% Since we are now within a group, we must make all definitions
% globally.  First we make sure that \texttt{theorem.sty} is loaded.
% This will allow us to use this file as a document style option
% without having to call \texttt{theorem} itself as an option.  At the
% same time, we assure that at least version 2 is loaded, since
% |\theorem@style| was not defined in earlier versions.
%    \begin{macrocode}
\@ifundefined{theorem@style}{\input{theorem.sty}}{}
\ProvidesFile{thp.sty}
             [\FMithmInfo]
%    \end{macrocode}
%
%
% \begin{macro}{\th@plain}
% \begin{macro}{\@begintheorem}
% \begin{macro}{\@opargbegintheorem}
%    |\theoremstyle{plain}| corresponds to the original
%    definition, except that the distances to the surrounding text are
%    determined by the parameters |\theorempreskipamount| and
%    |\theorempostskipamount|.
%    First we set the default body font.
%    \begin{macrocode}
\gdef\th@plain{\normalfont\itshape
%    \end{macrocode}
%    Then we define |\@begintheorem| and
%    |\@opargbegintheorem|. These two macros define how the
%    header of a theorem is typeset.  |\@opargbegintheorem| will
%    be called if a \textsf{theorem} environment with an optional
%    argument is encountered; otherwise, the header is constructed by
%    calling |\@begintheorem|.  If one of these macros is
%    executed, we are within a \textsf{trivlist} environment started by
%    |\@thm|.  So the theorem header is produced with an
%    |\item| command.
%
%    Instead of specifying the header font directly, all standard
%    theorem styles use the |\theorem@headerfont| macro to allow
%    customization.  The extra space (|\labelsep|) is necessary
%    because of problems in the \textsf{trivlist} environment.
%    \begin{macrocode}
  \def\@begintheorem##1##2{%
        \item[\hskip\labelsep \theorem@headerfont ##1\ ##2]}%
%    \end{macrocode}
%    The definition of |\@opargbegintheorem| is completely
%    analogous. The only difference is the fact that there exists a
%    third argument (which is the optional parameter of the
%    environment and contains additional information about the
%    theorem).  Customarily we enclose it in parentheses.
%    \begin{macrocode}
\def\@opargbegintheorem##1##2##3{%
   \item[\hskip\labelsep \theorem@headerfont ##1\ ##2\ (##3)]}}
%    \end{macrocode}
% \end{macro}
% \end{macro}
% \end{macro}
%
% \noindent
% We conclude with an |\endgroup| to restore the
% |\catcode| of |@|.
%    \begin{macrocode}
\endgroup
%</thp>
%    \end{macrocode}
%
%
% \subsubsection{The \texttt{break} style}
%
% This style option is stored in the file |thb.sty|.
% For the next two lines see the documentation for |\th@plain| on
% page~\pageref{dok:th@plain}.
%    \begin{macrocode}
%<*thb>
\begingroup \makeatletter
\@ifundefined{theorem@style}{\input{theorem.sty}}{}
\ProvidesFile{thb.sty}
             [\FMithmInfo]
%    \end{macrocode}
%
% \begin{macro}{\th@break}
%    |\theoremstyle{break}| produces a line break after the name
%    of the theorem.  The font is |\slshape|.  Hence, we define
%    |\th@break| as follows:
%    \begin{macrocode}
\gdef\th@break{\normalfont\slshape
  \def\@begintheorem##1##2{\item[%
%    \end{macrocode}
%    We run into the following problem: it is not possible to create
%    the header with |\item[|\meta{title}|]| and then start
%    a new line by, for example, |\mbox{}\\|. Such a definition
%    will fail whenever a list environment follows immediately.  With
%    the above construction, the |\mbox{}| causes the switch
%    |@inlabel| (cf.\ definition of |\list| and
%    |\trivlist| in \cite{latex:source}) to be set to \texttt{false}
%    and so the following list will insert additional vertical
%    space (|\topskip|). This is quite annoying. Therefore, we
%    create the line break within the |\item|. In order to ensure
%    that the text will begin at the proper position in the following
%    line, we simply pretend that the label does not take any
%    room.\footnote{This will lead to problems whenever very high
%    symbols occurring in the line tower into the
%    heading. So, something else has to be done here sometime.}
%    \begin{macrocode}
       \rlap{\vbox{\hbox{\hskip \labelsep\theorem@headerfont ##1\ ##2}%
                   \hbox{\strut}}}]}%
%    \end{macrocode}
%    Again, the definition of |\@opargbegintheorem| is
%    completely analogous.
%    \begin{macrocode}
\def\@opargbegintheorem##1##2##3{%
  \item[\rlap{\vbox{\hbox{\hskip \labelsep \theorem@headerfont
                     ##1\ ##2\ (##3)}%
                    \hbox{\strut}}}]}}
\endgroup
%</thb>
%    \end{macrocode}
% \end{macro}
%
%
% \subsubsection{The \texttt{changebreak} style}
%
%    \begin{macrocode}
% This style option is stored in the file |thcb.sty|.
%    \begin{macrocode}
%<*thcb>
\begingroup \makeatletter
\@ifundefined{theorem@style}{\input{theorem.sty}}{}
\ProvidesFile{thcb.sty}
             [\FMithmInfo]
%    \end{macrocode}
%
% \begin{macro}{\th@changebreak}
%    The \texttt{change-break} theorem style is like \texttt{break}
%    but with interchange of theorem name and theorem number.  Thus we
%    define |\th@changebreak| as follows:
%    \begin{macrocode}
\gdef\th@changebreak{\normalfont\slshape
  \def\@begintheorem##1##2{\item
       [\rlap{\vbox{\hbox{\hskip\labelsep \theorem@headerfont ##2\ ##1}%
                    \hbox{\strut}}}]}%
\def\@opargbegintheorem##1##2##3{%
  \item[\rlap{\vbox{\hbox{\hskip\labelsep \theorem@headerfont
                     ##2\ ##1\ (##3)}%
                    \hbox{\strut}}}]}}
\endgroup
%</thcb>
%    \end{macrocode}
% \end{macro}
%
%
% \subsubsection{The \texttt{change} style}
%
% This style option is stored in the file |thc.sty|.
%    \begin{macrocode}
%<*thc>
\begingroup \makeatletter
\@ifundefined{theorem@style}{\input{theorem.sty}}{}
\ProvidesFile{thc.sty}
             [\FMithmInfo]
%    \end{macrocode}
%
% \begin{macro}{\th@change}
%    The \texttt{change} theorem style corresponds to the
%    \texttt{change break} style without a linebreak after the header.
%    To say it in another way, it's the same as the \texttt{plain}
%    style but with number and name interchanged and |\slshape| as the
%    default font.
%    \begin{macrocode}
\gdef\th@change{\normalfont\slshape
  \def\@begintheorem##1##2{\item
       [\hskip\labelsep \theorem@headerfont ##2\ ##1]}%
\def\@opargbegintheorem##1##2##3{%
  \item[\hskip\labelsep \theorem@headerfont ##2\ ##1\ (##3)]}}
\endgroup
%</thc>
%    \end{macrocode}
% \end{macro}
%
%
% \subsubsection{The \texttt{marginbreak} style}
%
% This style option is the one used most often at Mainz.  It is saved
% in the file |thmb.sty|.
%    \begin{macrocode}
%<*thmb>
\begingroup \makeatletter
\@ifundefined{theorem@style}{\input{theorem.sty}}{}
\ProvidesFile{thmb.sty}
             [\FMithmInfo]
%    \end{macrocode}
%
% \begin{macro}{\th@marginbreak}
%    The \texttt{margin break} style is nearly the same as the
%    \texttt{change break} style. The only difference is the placement
%    of the theorem number.  We use |\llap| to place it in the left
%    margin.
%
%    In this style |\labelsep| denotes the separation between the
%    number and the text.
%    \begin{macrocode}
\gdef\th@marginbreak{\normalfont\slshape
  \def\@begintheorem##1##2{\item
       [\rlap{\vbox{\theorem@headerfont
                    \hbox{\llap{##2}\hskip\labelsep ##1}%
                    \hbox{\strut}}}]}%
\def\@opargbegintheorem##1##2##3{%
  \item[\rlap{\vbox{\theorem@headerfont
                    \hbox{\llap{##2}\hskip\labelsep ##1\ (##3)}%
                    \hbox{\strut}}}]}}
\endgroup
%</thmb>
%    \end{macrocode}
% \changes{v2.1c}{90/02/17}{`endgroup outside macrocode env.}
% \end{macro}
%
%
%
% \subsubsection{The \texttt{margin} style}
%
% This style option is stored in the file |thm.sty|.
%    \begin{macrocode}
%<*thm>
\begingroup \makeatletter
\@ifundefined{theorem@style}{\input{theorem.sty}}{}
\ProvidesFile{thm.sty}
             [\FMithmInfo]
%    \end{macrocode}
%
% \begin{macro}{\th@margin}
%    Again this is only a variant of the theorem styles described
%    above without any new ideas.
%    \begin{macrocode}
\gdef\th@margin{\normalfont\slshape
  \def\@begintheorem##1##2{\item
       [\theorem@headerfont \llap{##2}\hskip\labelsep ##1]}%
\def\@opargbegintheorem##1##2##3{%
  \item[\theorem@headerfont \llap{##2}\hskip\labelsep ##1\ (##3)]}}
\endgroup
%</thm>
%    \end{macrocode}
% \end{macro}
%
%
%
%
% \subsection{Final Definitions}
%
% \begin{macro}{\theorempreskipamount}
% \begin{macro}{\theorempostskipamount}
%    The \textsf{skip} parameters that regulate the vertical empty space
%    before and after the {\envfont theorem} environment have
%    to be allocated as well.
% \changes{v2.0f}{89/07/20}{Unnecessary `global' removed.}
%    \begin{macrocode}
%<*package>
\newskip\theorempreskipamount
\newskip\theorempostskipamount
%    \end{macrocode}
%    Since we have used the same values for all theorem sets, we now can
%    assign them.
%    \begin{macrocode}
\global\setlength\theorempreskipamount{12pt plus 5pt minus 3pt}
\global\setlength\theorempostskipamount{8pt plus 3pt minus 1.5pt}
%    \end{macrocode}
% \end{macro}
% \end{macro}
%
%
% \begin{macro}{\@endtheorem}
%    The same holds for the macro |\@endtheorem|, which ends a
%    {\envfont theorem} environment. Since it is the same for all
%    theorem sets, it is removed from the macros |\th@|\meta{style}.
%    It simply ends the {\envfont trivlist} environment, which was
%    begun in |\@thm|.
%    \begin{macrocode}
\global\let\@endtheorem=\endtrivlist
%    \end{macrocode}
% \end{macro}
%
%
%
% \begin{macro}{\@preamblecmds}
%    All macros defined above are to be used only in the preamble.
%    Therefore, we insert them in |\@preamblecmds| which will disable
%    them at begin document. This is done by the internal \LaTeXe{}
%    command |\@onlypreamble|.
% \changes{v2.0g}{89/09/19}{`theoremheaderfont and `theorembodyfont
%    added.}
% \changes{v2.2a}{94/02/03}{Use \LaTeXe{} interface}
%    \begin{macrocode}
\@onlypreamble\@xnthm
\@onlypreamble\@ynthm
\@onlypreamble\@othm
\@onlypreamble\newtheorem
\@onlypreamble\theoremstyle
\@onlypreamble\theorembodyfont
\@onlypreamble\theoremheaderfont
%    \end{macrocode}
% \end{macro}
%
% Finally we declare the \texttt{plain} theorem style to be the default.
% \changes{v2.1b}{90/06/04}{Fixed missing `item by declaring default
%    style.}
%    \begin{macrocode}
\theoremstyle{plain}
%</package>
%    \end{macrocode}
%
% \Finale

\endinput
}{}
\ProvidesFile{thm.sty}
             [\FMithmInfo]
%    \end{macrocode}
%
% \begin{macro}{\th@margin}
%    Again this is only a variant of the theorem styles described
%    above without any new ideas.
%    \begin{macrocode}
\gdef\th@margin{\normalfont\slshape
  \def\@begintheorem##1##2{\item
       [\theorem@headerfont \llap{##2}\hskip\labelsep ##1]}%
\def\@opargbegintheorem##1##2##3{%
  \item[\theorem@headerfont \llap{##2}\hskip\labelsep ##1\ (##3)]}}
\endgroup
%</thm>
%    \end{macrocode}
% \end{macro}
%
%
%
%
% \subsection{Final Definitions}
%
% \begin{macro}{\theorempreskipamount}
% \begin{macro}{\theorempostskipamount}
%    The \textsf{skip} parameters that regulate the vertical empty space
%    before and after the {\envfont theorem} environment have
%    to be allocated as well.
% \changes{v2.0f}{89/07/20}{Unnecessary `global' removed.}
%    \begin{macrocode}
%<*package>
\newskip\theorempreskipamount
\newskip\theorempostskipamount
%    \end{macrocode}
%    Since we have used the same values for all theorem sets, we now can
%    assign them.
%    \begin{macrocode}
\global\setlength\theorempreskipamount{12pt plus 5pt minus 3pt}
\global\setlength\theorempostskipamount{8pt plus 3pt minus 1.5pt}
%    \end{macrocode}
% \end{macro}
% \end{macro}
%
%
% \begin{macro}{\@endtheorem}
%    The same holds for the macro |\@endtheorem|, which ends a
%    {\envfont theorem} environment. Since it is the same for all
%    theorem sets, it is removed from the macros |\th@|\meta{style}.
%    It simply ends the {\envfont trivlist} environment, which was
%    begun in |\@thm|.
%    \begin{macrocode}
\global\let\@endtheorem=\endtrivlist
%    \end{macrocode}
% \end{macro}
%
%
%
% \begin{macro}{\@preamblecmds}
%    All macros defined above are to be used only in the preamble.
%    Therefore, we insert them in |\@preamblecmds| which will disable
%    them at begin document. This is done by the internal \LaTeXe{}
%    command |\@onlypreamble|.
% \changes{v2.0g}{89/09/19}{`theoremheaderfont and `theorembodyfont
%    added.}
% \changes{v2.2a}{94/02/03}{Use \LaTeXe{} interface}
%    \begin{macrocode}
\@onlypreamble\@xnthm
\@onlypreamble\@ynthm
\@onlypreamble\@othm
\@onlypreamble\newtheorem
\@onlypreamble\theoremstyle
\@onlypreamble\theorembodyfont
\@onlypreamble\theoremheaderfont
%    \end{macrocode}
% \end{macro}
%
% Finally we declare the \texttt{plain} theorem style to be the default.
% \changes{v2.1b}{90/06/04}{Fixed missing `item by declaring default
%    style.}
%    \begin{macrocode}
\theoremstyle{plain}
%</package>
%    \end{macrocode}
%
% \Finale

\endinput
}{}
\ProvidesFile{thmb.sty}
             [\FMithmInfo]
%    \end{macrocode}
%
% \begin{macro}{\th@marginbreak}
%    The \texttt{margin break} style is nearly the same as the
%    \texttt{change break} style. The only difference is the placement
%    of the theorem number.  We use |\llap| to place it in the left
%    margin.
%
%    In this style |\labelsep| denotes the separation between the
%    number and the text.
%    \begin{macrocode}
\gdef\th@marginbreak{\normalfont\slshape
  \def\@begintheorem##1##2{\item
       [\rlap{\vbox{\theorem@headerfont
                    \hbox{\llap{##2}\hskip\labelsep ##1}%
                    \hbox{\strut}}}]}%
\def\@opargbegintheorem##1##2##3{%
  \item[\rlap{\vbox{\theorem@headerfont
                    \hbox{\llap{##2}\hskip\labelsep ##1\ (##3)}%
                    \hbox{\strut}}}]}}
\endgroup
%</thmb>
%    \end{macrocode}
% \changes{v2.1c}{90/02/17}{`endgroup outside macrocode env.}
% \end{macro}
%
%
%
% \subsubsection{The \texttt{margin} style}
%
% This style option is stored in the file |thm.sty|.
%    \begin{macrocode}
%<*thm>
\begingroup \makeatletter
\@ifundefined{theorem@style}{% \iffalse meta-comment
%
% Copyright (C) 1993-2024
%
% The LaTeX Project and any individual authors listed elsewhere
% in this file.
%
% This file is part of the Standard LaTeX `Tools Bundle'.
% -------------------------------------------------------
%
% It may be distributed and/or modified under the
% conditions of the LaTeX Project Public License, either version 1.3c
% of this license or (at your option) any later version.
% The latest version of this license is in
%    https://www.latex-project.org/lppl.txt
% and version 1.3c or later is part of all distributions of LaTeX
% version 2005/12/01 or later.
%
% The list of all files belonging to the LaTeX `Tools Bundle' is
% given in the file `manifest.txt'.
%
% \fi
%
%%
%
%
\def\FMithmInfo{2023/07/05 v2.2c Theorem extension package (FMi)}
%
% \ProvidesFile{theorem.dtx}[\FMithmInfo]
%\iffalse   % this is a METACOMMENT !
%%
%% Package `theorem' to use with LaTeX2e
%% Copyright (C) 1989-1995 Frank Mittelbach, all rights reserved.
%<+package>\@ifundefined{theorem@style}{}{\endinput}
%<+package>\NeedsTeXFormat{LaTeX2e}[1995/06/01]
%<+package>\ProvidesPackage{theorem}
%<+package>         [\FMithmInfo]
%
% \fi
%
% \changes{v1.0}{87/05/07}{First draft.}
% \changes{v1.0a}{87/05/07}{`thistheoremstyle' added.}
% \changes{v1.1a}{87/07/29}{New concept: redefine the internal
%                          \protect\LaTeX{} commands `@xnthm' etc.}
% \changes{v1.1b}{87/07/31}{`theoremfont' used to specify font.}
% \changes{v1.2a}{88/01/08}{`@endtheorem' now globally defined.}
% \changes{v1.2b}{88/01/11}{`newtheorem' only allowed in the preamble.}
% \changes{v1.3a}{88/02/14}{`theoremfont' now a toks register.}
%
% \changes{v2.0a}{88/11/16}{New implementation and new documentation.}
% \changes{v2.0a}{88/11/16}{No layout changes but internal modifications
%                           old substyles won't work.}
% \changes{v2.0b}{89/01/30}{`global' used for all definitions.}
% \changes{v2.0c}{89/06/15}{Documentation translated to English.
%                            Thanks to LLL and Nina}
% \changes{v2.1a}{90/02/15}{Support for new font selection.}
% \changes{v2.2a}{94/02/03}{Cleanup for LaTeX2e}
%
% \DoNotIndex{\@endparenv,\@ifdefinable,\@ifnextchar,\@ifundefined}
% \DoNotIndex{\@namedef,\@nameuse,\@spaces,\@tempa,\@warning}
% \DoNotIndex{\ }
% \DoNotIndex{\begingroup,\bfseries}
% \DoNotIndex{\csname}
% \DoNotIndex{\def,\do,\docdate}
% \DoNotIndex{\endcsname,\endgroup,\endinput,\endtrivlist,\expandafter}
% \DoNotIndex{\filedate,\fileversion}
% \DoNotIndex{\gdef,\global}
% \DoNotIndex{\hbox,\hskip}
% \DoNotIndex{\ignorespaces,\input,\itshape,\item}
% \DoNotIndex{\labelsep,\let,\llap}
% \DoNotIndex{\makeatletter}
% \DoNotIndex{\noexpand}
% \DoNotIndex{\refstepcounter,\relax,\rlap}
% \DoNotIndex{\setlength,\slshape,\space,\string,\strut}
% \DoNotIndex{\the,\trivlist,\typeout}
% \DoNotIndex{\vbox}
% \DoNotIndex{\xdef}
%
% \def\deflabel#1{\ttfamily #1\hfill}
% \def\deflist#1{\begin{list}{}{\settowidth\labelwidth{\ttfamily #1}%
%                               \setlength\leftmargin\labelwidth
%                               \addtolength\leftmargin\labelsep
%                               \let\makelabel\deflabel}}
% \def\enddeflist{\end{list}}
% \def\envfont{\normalfont\ttfamily}
%
% \MakeShortVerb{\|}
%
% \newcommand{\TUB}{{\sl TUGboat\/}}
%
% \GetFileInfo{theorem.dtx}
% \title{An Extension of the
%        \LaTeX{} theorem environment\thanks{This file
%        has version number \fileversion, last
%        revised \filedate.}}
% \author{Frank Mittelbach}
%
% \MaintainedByLaTeXTeam{tools}
% \maketitle
%
% \begin{abstract}
%    The macros described in this paper yield an extension of the
%    \LaTeX{} theorem mechanism. It is designed to satisfy the
%    different requirements of various journals. Thus, the
%    layout of the ``theorems'' can be manipulated by determining a
%    ``style''. This article describes not only the use, but
%    also the definition, of the necessary macros.
% \end{abstract}
%
% \section*{Preface to version 2.2}
%
% For \LaTeXe{} this package did not need any fundamental changes. I
% only modified the messages generated so that theorem layout styles
% will show up with the |\listfiles| command and cleaned the section
% on the New Font Selection Scheme since this is now included in
% \LaTeX.
%
% \section*{Preface to version 2.1}
%
% This version is identical to 2.0g described in \TUB\ 10\#3 except
% for some internal defaults which are now set depending on the used
% font selection scheme.
%
% This was done to avoid unpleasant surprises if the new font selection
% scheme is in force. For further details see
% section~\ref{sec:newfonts} and \cite{bk:GMS94}.
%
% \section{Introduction}
%
% For our purposes here, ``theorems'' are labelled enunciations,
% often set off from the main text by extra space and a font change.
% Theorems, corollaries, conjectures, definitions, and remarks
% are all instances of ``theorems''.  The ``header'' of these structures
% is composed of a label (such as \textsc{Theorem} or \textsc{Remark})
% and a number which serializes an item in the sequence of items
% with the same label.
%
% Shortly after the introduction of \LaTeX{} at the Fachbereich
% Mathematik in Mainz, the desire to manipulate the layout of
% ``theorems'' arose.  In Mainz, the following two conventions came
% into general use:
% \begin{enumerate}
%   \item The number of the theorem is shown in the margin.
%   \item There is a line break at
%         the end of the theorem header.
% \end{enumerate}
%
% Additionally, some journals require different formats which depend on
% the ``sort of theorem'': e.g.\ often remarks and definitions are set
% in |\upshape|, while |\itshape| is employed for main theorems.
%
% Confronted with these requirements, a theorem environment was
% developed in Mainz which allows separate determination of the
% layout of the ``theorems sets'', comparable to |\pagestyle|.
%
%
% \section{The user interface}
% \subsection{Defining new theorem sets}
%
% \DescribeMacro\newtheorem
% As in the original \LaTeX{} version, the command |\newtheorem|
% defines a new ``theorem set'' or ``theorem-like structure''.
% Two required arguments name the new environment and give the text
% to be typeset with each instance of the new ``set'', while
% an optional argument determines how
% the ``set'' is enumerated:
% \begin{description}
%    \item[\ttfamily \bslash newtheorem\{foo\}\{bar\}]
%       The theorem set {\envfont foo} (whose name is \texttt{bar})
%       uses its own counter.
%    \item[\ttfamily \bslash newtheorem\{foo2\}{[foo]}\{bar2\}]
%       The theorem set {\envfont foo2} (printed name \texttt{bar2})
%       uses the same counter as the theorem set \texttt{foo}.
%    \item[\ttfamily \bslash newtheorem\{foo3\}\{bar3\}{[section]}]
%       The theorem set {\envfont foo3} (printed name \texttt{bar3}) is
%       enumerated within the counter \texttt{section}, i.e.\ with every
%       new |\section| the enumeration begins again with 1, and
%       the enumeration is composed from the section-number and the
%       theorem counter itself.
% \end{description}
%
% \DescribeMacro\theoremstyle
% Additionally, the command |\theoremstyle| can define the layout
% of various, or all, theorem sets.  It should be noted that any
% theorem set defined by |\newtheorem| is typeset in the
% |\theoremstyle| that is current at the time of the definition.
% Thus, the following
% \begin{verbatim}
%    \theoremstyle{break}        \newtheorem{Cor}{Corollary}
%    \theoremstyle{plain}        \newtheorem{Exa}{Example}[section]
%\end{verbatim}                             ^^A No blank at line start
% leads to the result that the set {\envfont Cor} is formatted in the
% style \texttt{break}, while the set {\envfont Exa} and all the
% following ones are formatted in the style \texttt{plain}, unless
% another |\theoremstyle| follows.  Since the definitions installed by
% |\newtheorem| are global, one also can limit |\theoremstyle| locally
% by grouping braces.
%
% \DescribeMacro\theorembodyfont
% The choice of the font for the theorem body is completely
% independent of the chosen |\theoremstyle|; this has proven to
% be very advantageous.  For example,
% \begin{verbatim}
%   {\theorembodyfont{\upshape}          \newtheorem{Rem}{Remark}}
%\end{verbatim}
% defines a theorem set {\envfont Rem}, which will be set in
% |\upshape| in the current layout (which in our example is
% \texttt{plain}).  As with |\theoremstyle|, the |\theorembodyfont|
% chosen is that current at the time of |\newtheorem|.  If
% |\theorembodyfont| is not specified or one defines
% |\theorembodyfont{}|, then the font used will be that defined by the
% |\theoremstyle|.
%
% \DescribeMacro\theoremheaderfont It is also possible to customize
% the font used for the theorem headers.  This is, however, a global
% declaration, and therefore there should be at most one
% |\theoremheaderfont| declaration in the preamble.\footnote{If it is
% actually necessary to have different header fonts, one has to define
% new theorem styles (substituting the desired font) or specify the
% information directly in the \texttt{\bslash newtheorem} declaration
% (the unclean variant).}
%
% \DescribeMacro\theorempreskipamount
% \DescribeMacro\theorempostskipamount Two additional parameters
% affect the vertical space around the theorem environments:
% |\theorempreskipamount| and |\theorempostskipamount| define,
% respectively, the spacing before and after such an environment.
% These parameters apply for all theorem sets and can be manipulated
% with the ordinary length macros.  They are rubber lengths
% (`\textsf{skips}'), and therefore can contain \texttt{plus} and
% \texttt{minus} parts.
%
% Since the definition of theorem sets should---most sensibly---be
% placed in the preamble, we only allow installation there.
% It is therefore possible to release the memory used here after
% |\begin{document}|, in order to make room for other
% applications.
%
%
% \subsection{Existing theorem styles}
%
% The following theorem styles exist to date:
% \begin{deflist}{marginbreak}
%    \item[plain]
%       This theorem style emulates the original \LaTeX{} definition,
%       except that additionally the parameters
%       |\theorem...skipamount| are used.
%    \item[break]
%       In this style, the theorem header is followed by a line
%       break.
%    \item[marginbreak]
%       The theorem number is set in the margin, and there is a line
%       break as in \texttt{break}.
%    \item[changebreak]
%       Like \texttt{break}, but with header number and text
%       interchanged.
%    \item[change]
%       Header number and text are interchanged, without a line break.
%    \item[margin]
%       The number is set in the left margin, without a line break.
% \end{deflist}
% All styles (except \texttt{plain}) select |\slshape| as the default
% |\theorembodyfont|.
%
%
% \subsection{Examples}
%
% Given the above theorem sets {\envfont Cor}, {\envfont Exa}
% and {\envfont Rem},
% suppose that the preamble also contains the declarations:
% \begin{verbatim}
%    \theoremstyle{marginbreak}   \newtheorem{Lem}[Cor]{Lemma}
%    \theoremstyle{change}
%    \theorembodyfont{\itshape}        \newtheorem{Def}[Cor]{Definition}
%
%    \theoremheaderfont{\scshape}
%\end{verbatim}
% Then the following are some typical examples of the typeset output
% resulting from their use.
%
% \begin{Cor}
%    This is a sentence typeset in the theorem environment {\envfont
%    Cor}.
% \end{Cor}
%
% \begin{Exa}
%    This is a sentence typeset in the theorem environment {\envfont
%    Exa}.
% \end{Exa}
%
% \begin{Rem}
%    This is a sentence typeset in the theorem environment {\envfont
%    Rem}.
% \end{Rem}
%
% \begin{Lem}[Ben User]
%    This is a sentence typeset in the theorem environment {\envfont
%    Lem}.
% \end{Lem}
%
% \begin{Def}[Very impressive Definition]
%    This is a sentence typeset in the theorem environment {\envfont
%    Def}.
% \end{Def}
%
% The last two examples show the effect of the optional argument to
% a theorem environment (it is the text typeset in parentheses).
%
%
% \section{Special Considerations}
% \label{sec:newfonts}
%
% Theoremheader and body are implemented as a unit. This means that
% the |\theoremheaderfont|  will inherit characteristics of the
% |\theorembodyfont| in \LaTeXe. Thus, if for example
% |\theorembodyfont| is |\itshape| and |\theoremheaderfont|
% is |\bfseries| the font selected for the header will have the
% characteristics `bold extended italic'. If this is not desired one
% should set the |\theoremheaderfont| to something like
% \begin{verbatim}
%      \theoremheaderfont{\normalfont\bfseries}
%\end{verbatim}
% i.e.\ supplying all necessary font information explicitly.
%
%
%
% \section{Acknowledgements}
%
% The publication of this set of macros was only possible with the
% help of Christina Busse (translating the manuscript into English),
% Joachim Pense (playing the r\^ole of typist), Chris Rowley
% (looking everything over) and many others providing useful
% suggestions.
%
% \MaybeStop{%
%    \begin{thebibliography}{9}
%       \bibitem{bk:GMS94} \textsc{M.~Goossens}, \textsc{F.~Mittelbach}
%         and \textsc{A.~Samarin}.
%         \newblock The \LaTeX{} Companion.
%         \newblock
%         Addison-Wesley, Reading, Massachusetts, 1994.
%       \bibitem{latex:source} \textsc{Lamport, Leslie.} \newblock
%                               \texttt{latex.tex}, version 2.09, date
%                               Feb. 1990.
%    \end{thebibliography}
%    \PrintIndex
%    \PrintChanges}
%
%
% \section{The documentation driver file}
%
% The next bit of code contains the documentation driver file for
% \TeX{}, i.e., the file that will produce the documentation you are
% currently reading. It will be extracted from this file by the
% \texttt{docstrip} program. Since it is the first code in the file
% one can alternatively process this file directly with \LaTeXe{} to
% obtain the documentation.
%
%    \begin{macrocode}
%<*driver>
\documentclass{ltxdoc}

\usepackage{theorem}

%<+driver>% The next few lines define theorem sets which are used
%<+driver>% in the example section of the documentation.

\theoremstyle{break}       \newtheorem{Cor}{Corollary}
\theoremstyle{plain}       \newtheorem{Exa}{Example}[section]
{\theorembodyfont{\upshape}\newtheorem{Rem}{Remark}}
\theoremstyle{marginbreak} \newtheorem{Lem}[Cor]{Lemma}
\theoremstyle{change}
\theorembodyfont{\itshape} \newtheorem{Def}[Cor]{Definition}

\theoremheaderfont{\scshape}

\RecordChanges

\begin{document}
   \DocInput{theorem.dtx}
\end{document}
%</driver>
%    \end{macrocode}
%
% \section{Definition of the Macros}
%
% If the file has been loaded before, we abort immediately. If not the
% package announces itself (this is actually done at the very top of
% the file---the way it is done isn't good style so don't copy it).
% \changes{v2.0e}{89/07/19}{Spaces removed from `typeout'.}
% \changes{v2.2b}{95/11/19}{Announce moved to top of file}
%    \begin{macrocode}
%<*package>
%\@ifundefined{theorem@style}{}{\endinput}
%\def\FMithmInfo{1995/11/19 v2.2b Theorem extension package (FMi)}
%\ProvidesPackage{theorem}[\FMithmInfo]
%    \end{macrocode}
%
%
% \subsection{Definition of theorem styles and fonts}
%
% All the definitions in this file are done globally to allow
% inputting this file inside a group.
%
% \begin{macro}{\theoremstyle}
%    Before a theorem style can be installed, the chosen style must be
%    known. For that reason, we must test to see that
%    |\th@|\meta{style} is known or, more precisely, that it is
%    different from |\relax|. If the style is not known then
%    |\th@plain| is used.
%    \begin{macrocode}
\gdef\theoremstyle#1{%
   \@ifundefined{th@#1}{\@warning
          {Unknown theoremstyle `#1'. Using `plain'}%
          \theorem@style{plain}}%
%    \end{macrocode}
%    We save the theorem style to be used in the token register
%    |\theorem@style|.
%    \begin{macrocode}
      {\theorem@style{#1}}%
%    \end{macrocode}
%    Now we ``evaluate'' the theorem style: this means, we call the
%    macro |\th@|\meta{style} which will activate the relevant
%    definitions which are contained in a separate file.  This is done
%    in a group to suppress changes to the current font.  This could
%    otherwise pose problems together with the new font selection
%    scheme\footnote{When I printed the original article using the new
%    font selection scheme I ended with a document with slanted
%    typefaces (text headings and all) simply because one of the
%    theorem styles used \texttt{\bslash sl} at toplevel.} if the
%    |\th@|\meta{style} is evaluated a second time.
% \changes{v2.1a}{90/02/15}{Group added around `th@..}
% \changes{v2.1a}{90/02/15}{Reseting of `..@begintheorem removed.}
%    \begin{macrocode}
      \begingroup
        \csname th@\the\theorem@style \endcsname
      \endgroup}
%    \end{macrocode}
% \end{macro}
%
%
% \begin{macro}{\@begintheorem}
% \changes{v2.1a}{90/02/15}{Relaxed at toplevel.}
% \begin{macro}{\@opargbegintheorem}
% \changes{v2.1a}{90/02/15}{Relaxed at toplevel.}
%    We reset |\@begintheorem| and |\@opargbegintheorem| to
%    |\relax| since these commands are no longer necessary at
%    toplevel. This will save a few tokens.
%    \begin{macrocode}
\global\let\@begintheorem\relax
\global\let\@opargbegintheorem\relax
%    \end{macrocode}
% \end{macro}
% \end{macro}
%
% \begin{macro}{\theorem@style}
%    Obviously the \textsf{token} register used above has to be
%    allocated. To assure the utmost compatibility with the
%    original \LaTeX{} definition, we set the default theorem style to
%    \texttt{plain}, which implements the usual \LaTeX{} convention.
% \changes{v2.0f}{89/07/20}{Unnecessary `global' removed.}
%    \begin{macrocode}
\newtoks\theorem@style
\global\theorem@style{plain}
%    \end{macrocode}
% \end{macro}
%
% \begin{macro}{\theorembodyfont}
% \begin{macro}{\theorem@bodyfont}
%    If the |\theorembodyfont| is set by the user then it should not
%    interact with the default font set in the theorem style. When the
%    new font selection is in force this may happen if, for example,
%    the default is |\itshape| and the new |\theorembodyfont| is
%    |\sffamily|. So we add a |\reset@font| command in front of the
%    user definition.
% \changes{v2.0f}{89/07/20}{Unnecessary `global' removed.}
% \changes{v2.0g}{89/09/19}{`theoremfont to `theorembodyfont.}
% \changes{v2.1c}{92/06/22}{`reset@font added and logic changed.}
% \changes{v2.1d}{92/10/28}{Reset to empty if arg is empty}
%    \begin{macrocode}
\gdef\theorembodyfont#1{%
%    \end{macrocode}
%    We check if the argument supplied is empty and if so put nothing
%    into the |\theorem@bodyfont| token register to allow for
%    |\theorembodyfont{}| as a mean of using the default of the
%    current |\theoremstyle|.
%    \begin{macrocode}
   \def\@tempa{#1}%
   \ifx\@tempa\@empty
    \theorem@bodyfont{}%
   \else
    \theorem@bodyfont{\reset@font#1}%
   \fi
   }
\newtoks\theorem@bodyfont
\global\theorem@bodyfont{}
%    \end{macrocode}
% \end{macro}
% \end{macro}
%
%
% \begin{macro}{\theoremheaderfont}
% \changes{v2.0g}{89/09/19}{Macro added.}
%    The font for the theorem headers is handled differently because
%    this definition applies to all theorem styles.
%    \begin{macrocode}
\gdef\theoremheaderfont#1{\gdef\theorem@headerfont{#1}%
%    \end{macrocode}
%    After using the macro once it is redefined to produce an error
%    message.
%    \begin{macrocode}
       \gdef\theoremheaderfont##1{%
        \typeout{\string\theoremheaderfont\space should be used
                 only once.}}}
%    \end{macrocode}
% \end{macro}
%
% \begin{macro}{\theorem@headerfont}
% \changes{v2.0g}{89/09/19}{Macro added.}
%    To set the |\theorem@headerfont| default we first test if
%    the new fontselection scheme is in force.
% \changes{v2.1a}{90/02/15}{Support for new font selection.}
%    \begin{macrocode}
\ifx\upshape\undefined
%    \end{macrocode}
%    If not we define it to expand into |\bfseries|. We don't use
%    |\let| just in case a following style option redefines this
%    macro.
%    \begin{macrocode}
\gdef\theorem@headerfont{\bfseries}
%    \end{macrocode}
%    Otherwise we reset the current shape before calling |\bfseries|.
%    \begin{macrocode}
\else \gdef\theorem@headerfont{\normalfont\bfseries}\fi
%    \end{macrocode}
% \end{macro}
%
% \begin{macro}{\th@plain}
% \begin{macro}{\th@break}
% \begin{macro}{\th@marginbreak}
% \begin{macro}{\th@changebreak}
% \begin{macro}{\th@change}
% \begin{macro}{\th@margin}
%    The different styles are defined in macros such as
%    |\th@plain|.  Since memory space is precious in
%    ``non-Big-versions'', we have to avoid offering too many unused
%    definitions.  Therefore we define these styles in separate files
%    that can be loaded on demand.  Thus the commands themselves only
%    load these files. We use |\@input@|, a \LaTeXe{} internal command
%    that ensures that the file will be listed with |\listfiles|.
% \changes{v2.2a}{94/02/02}{Use `@input@ to load theorem layout files}
%    \begin{macrocode}
\gdef\th@plain{\@input@{thp.sty}}
\gdef\th@break{\@input@{thb.sty}}
\gdef\th@marginbreak{\@input@{thmb.sty}}
\gdef\th@changebreak{\@input@{thcb.sty}}
\gdef\th@change{\@input@{thc.sty}}
\gdef\th@margin{\@input@{thm.sty}}
%    \end{macrocode}
%    This list will be expanded when new styles become
%    available.  For testing,  just append new theorem substyles
%    as document options.
% \end{macro}
% \end{macro}
% \end{macro}
% \end{macro}
% \end{macro}
% \end{macro}
%
%
%
% \subsection{Definition of a new theorem set}
%
% As already pointed out, a new theorem environment can be defined in
% three different ways:
% \begin{verbatim}
%    \newtheorem{Lem}{Lemma}
%    \newtheorem{Lem}{Lemma}[section]
%    \newtheorem{Lem}[Theorem]{Lemma}
%\end{verbatim}
% The function of the macro |\newtheorem| is to recognize these cases
% and then to branch into one of the three macros |\@ynthm|, |\@xnthm|
% or |\@othm|.  This mechanism is adopted unchanged from
% \cite{latex:source}; the essential point here is that, for example,
% in the second case, the arguments \texttt{Lem}, \texttt{Lemma} and
% \texttt{section} are passed over to the macro |\@xnthm|.
%
% We inspect this case first because the others present fewer problems,
% and thus are easily derived from this one.
%
% \begin{macro}{\@xnthm}
%    For our example arguments, the macro |\@xnthm| must fulfill
%    the following:
%    \begin{itemize}
%       \item
%          Define a new \LaTeX-counter `Lem'
%       \item
%          reset this counter within a |\section|
%       \item
%          define the macro |\theLem|
%       \item
%          define the environment macros |\Lem| and
%          |\endLem| using the current
%          |\theoremstyle| and |\theorem@bodyfont|.
%    \end{itemize}
%    Obviously, all this should happen only if the first argument
%    of |\@xnthm| (i.e.\ \texttt{Lem} in our example) is chosen so
%    as not to conflict with any previously defined commands or
%    environments.  This test is performed by the \LaTeX{} macro
%    |\@ifdefinable|.
%    \begin{macrocode}
\gdef\@xnthm#1#2[#3]{\expandafter\@ifdefinable\csname #1\endcsname
%    \end{macrocode}
%    Therefore, the first argument of |\@ifdefinable| is the
%    expansion (in the example, |\Lem|) of
%    |\csname||#1\endcsname|.  The second argument is
%    executed only if the test has been completed successfully.
%    \begin{macrocode}
   {%
%    \end{macrocode}
%    Now we define the new counter. The names of the \LaTeX{}
%    macros employed should speak for themselves:
% \changes{v2.2b}{1995/10/07}{Use \cs{@newctr}}
% \changes{v2.2c}{1995/11/23}{Use \cs{@newctr} properly.}
%    \begin{macrocode}
    \@definecounter{#1}\@newctr{#1}[#3]%
%    \end{macrocode}
%    Using |\@newctr| will give a proper error message if the counter
%    in |#3| is not defined.
%    In defining `|\theLem|' we must generate the desired macro
%    name by use of |\expandafter| and |\csname|.
%    \begin{macrocode}
    \expandafter\xdef\csname the#1\endcsname
%    \end{macrocode}
%    An |\xdef| is used in order to make the definition global,
%    and to ensure that it contains the replacement texts of
%    |\@thmcountersep| and |\@thmcounter|.\footnote{These
%    two macros can be defined by the document style.  Their default
%    values produce a `\texttt{.}' as separation and an arabic
%    representation of the number.} However, not everything should be
%    expanded. For example, it saves space to use |\thesection|
%    instead of its---at times---lengthy expansion.
%    \begin{macrocode}
      {\expandafter \noexpand \csname the#3\endcsname
       \@thmcountersep \@thmcounter{#1}}%
%    \end{macrocode}
%    Thus with the defaults of \LaTeX{}, |\theLem| would be
%    replaced by the command sequence |\thesection.\arabic{Lem}|.
%
%    We will now look at the definition of the macro which is executed
%    at the beginning of the actual environment (in our example this
%    macro is |\Lem|). It should be noted that we use an
%    ``|\expandafter| trick'' to expand only certain parts of the
%    replacement text at the time of the definition.
%    \begin{macrocode}
    \def\@tempa{\global\@namedef{#1}}%
    \expandafter \@tempa \expandafter{%
%    \end{macrocode}
%    First, the macro that contains the current definitions of
%    |\@begintheorem| and |\@opargtheorem| should be called
%    up.  The name of this macro---as is already known---has the form
%    |\th@|\meta{theorem\ style}; therefore, it must be called by
%    \begin{macrocode}
      \csname th@\the \theorem@style
%    \end{macrocode}
%    In addition the default theorem font should be changeable, i.e.\
%    we have to insert the contents of |\theorem@bodyfont|. For
%    that reason, we expand even further, beyond |\endcsname|,
%    and thus insert the contents of the \textsf{token} register
%    |\theorem@bodyfont| in the replacement text.
%    \begin{macrocode}
            \expandafter \endcsname \the \theorem@bodyfont
%    \end{macrocode}
%    Now it is time to call the macro |\@thm| which takes over
%    the further processing.  It has two arguments: the current
%    counter name (in our example, \texttt{Lem}), and the text of the
%    label (in our example, \texttt{Lemma}).
%    \begin{macrocode}
     \@thm{#1}{#2}}%
%    \end{macrocode}
%    With this, the `sub-definition' is complete.  The macro
%    |\@endtheorem| ends a theorem environment and is, so far, nothing
%    but an |\endtrivlist|.  (Hence it is defined globally, and not
%    within the theorem styles.\footnote{This has to be changed as
%    soon as theorem styles that change \texttt{\bslash @endtheorem}
%    exist.  In such a case, all existing styles must be changed as
%    well since they will have to reset the macro.}) Therefore, we can
%    set it equivalent to the macro that ends the theorem set (in our
%    example, |\endLem|).  However, if some day theorem styles exist
%    that do change |\@endtheorem|, we would have to use the
%    commented-out line instead.
%    \begin{macrocode}
    \global \expandafter \let \csname end#1\endcsname \@endtheorem
%   \global\@namedef{end#1}{\@endtheorem}%
%    \end{macrocode}
%    With these commands all the required definitions are employed,
%    unless the test |\@ifdefinable| has failed.  Therefore, we
%    end the second argument of this macro and with it the definition
%    of |\@xnthm|.
%    \begin{macrocode}
   }}
%    \end{macrocode}
% \end{macro}
%
% \begin{macro}{\@ynthm}
%    The definition of |\@ynthm| is completely analogous. In
%    this case the new counter that is defined is not reset within
%    another counter; thus the definition of |\the...|
%    is simplified:
%    \begin{macrocode}
\gdef\@ynthm#1#2{\expandafter\@ifdefinable\csname #1\endcsname
   {\@definecounter{#1}%
    \expandafter\xdef\csname the#1\endcsname{\@thmcounter{#1}}%
%    \end{macrocode}
%    The rest of the definition corresponds literally to that of
%    |\@xnthm|:
%    \begin{macrocode}
    \def\@tempa{\global\@namedef{#1}}\expandafter \@tempa
     \expandafter{\csname th@\the \theorem@style \expandafter
     \endcsname \the\theorem@bodyfont \@thm{#1}{#2}}%
    \global \expandafter \let \csname end#1\endcsname \@endtheorem}}
%    \end{macrocode}
% \end{macro}
%
% \begin{macro}{\@othm}
%    The definition of |\@othm| does not contain anything new.
%    \begin{macrocode}
\gdef\@othm#1[#2]#3{%
%    \end{macrocode}
% \changes{v2.2b}{95/10/07}{Build in check for counter defined}
%    We do not define a new counter but instead use one that has
%    already been defined.  Thus the only definition we need is that
%    of this pseudo-counter (i.e.\ |\the|\meta{env.\ name}).
%    First we check if |#2| corresponds to a known counter name.
%    \begin{macrocode}
  \expandafter\ifx\csname c@#2\endcsname\relax
   \@nocounterr{#2}%
  \else
   \expandafter\@ifdefinable\csname #1\endcsname
   {\expandafter \xdef \csname the#1\endcsname
     {\expandafter \noexpand \csname the#2\endcsname}%
%    \end{macrocode}
%    All other parts of the definition can be adopted from
%    |\@xnthm|. We have to remember, though, that in this case
%    the name of the current counter and the theorem label have moved
%    to the second and third arguments.
%    \begin{macrocode}
    \def\@tempa{\global\@namedef{#1}}\expandafter \@tempa
     \expandafter{\csname th@\the \theorem@style \expandafter
     \endcsname \the\theorem@bodyfont \@thm{#2}{#3}}%
    \global \expandafter \let \csname end#1\endcsname \@endtheorem}%
  \fi}
%    \end{macrocode}
% \end{macro}
%
% \subsection{Macros that are employed in a {\envfont theorem}
%    environment}
%
% \begin{macro}{\@thm}
%    The macro |\@thm| has to increase the current counter.  Then,
%    depending on whether the environment has (or does not have) an
%    optional argument, it has to branch into either |\@begintheorem|
%    or |\@opargtheorem|.
%    \begin{macrocode}
\gdef\@thm#1#2{\refstepcounter{#1}%
%    \end{macrocode}
%    Now we start a {\envfont trivlist} environment, and give
%    |\@topsep| and |\@topsepadd| the values of the \textsf{skip}
%    registers |\theorempreskipamount| and |\theorempostskipamount|.
%    The value in |\@topsep| is the vertical space that is inserted by
%    the first (and only) |\item| in our |\trivlist| whilst
%    |\@topsepadd| is inserted by |\@endparenv| at the end of that
%    {\envfont trivlist} environment.  By using these registers, we
%    obtain the desired space around a {\envfont theorem} environment.
%    \begin{macrocode}
   \trivlist
   \@topsep \theorempreskipamount               % used by first \item
   \@topsepadd \theorempostskipamount           % used by \@endparenv
%    \end{macrocode}
%    Now we have to test whether an optional argument has been given.
% \changes{v2.0d}{89/07/18}{Forgotten percent added.}
%    \begin{macrocode}
   \@ifnextchar [%
%    \end{macrocode}
%    If there is an optional argument, we will call |\@ythm|, and
%    move the arguments read back into the input stream.
%    \begin{macrocode}
   {\@ythm{#1}{#2}}%
%    \end{macrocode}
%    If not, we call |\@begintheorem|. Its first argument is the
%    name of the theorem set (hence the second argument of
%    |\@thm|). Its second argument is the macro that produces the
%    current number.
%    \begin{macrocode}
   {\@begintheorem{#2}{\csname the#1\endcsname}\ignorespaces}}
%    \end{macrocode}
% \end{macro}
%
% \begin{macro}{\@xthm}
% \begin{macro}{\@ythm}
%    Both these macros were originally called by |\@thm|.  We do
%    not need |\@xthm| anymore, hence we reset it to
%    |\relax|. The definition of |\@ythm| has not changed at
%    all from its definition in \LaTeX. In order to make the macros
%    easier to understand, we will
%    nevertheless present it (commented out).
%    \begin{macrocode}
\global\let\@xthm\relax
% \def\@ythm#1#2[#3]{\@opargbegintheorem{#2}{\csname
%           the#1\endcsname}{#3}\ignorespaces}
%</package>
%    \end{macrocode}
%    The primitive |\ignorespaces| in |\@ythm| and
%    |\@thm| is needed to remove the spaces between the
%    |\begin{...}| and the actual text.
% \end{macro}
% \end{macro}
%
%
% \subsection{Definition of the theorem substyles}
%
% As already pointed out, the theorem substyles, defined below, are
% only loaded when necessary.  Note that all these substyles, except
% \texttt{plain}, have |\slshape| as the default body font.
%
% \subsubsection{The \texttt{plain} style}
%
% \label{dok:th@plain} As the following macros use |@|, we have
% to locally set the |\catcode| of this symbol to ``letter''.
% This happens within a group, so that we do not have to worry about
% which |\catcode| that symbol had before.
%    \begin{macrocode}
%<*thp>
\begingroup \makeatletter
%    \end{macrocode}
% Since we are now within a group, we must make all definitions
% globally.  First we make sure that \texttt{theorem.sty} is loaded.
% This will allow us to use this file as a document style option
% without having to call \texttt{theorem} itself as an option.  At the
% same time, we assure that at least version 2 is loaded, since
% |\theorem@style| was not defined in earlier versions.
%    \begin{macrocode}
\@ifundefined{theorem@style}{% \iffalse meta-comment
%
% Copyright (C) 1993-2024
%
% The LaTeX Project and any individual authors listed elsewhere
% in this file.
%
% This file is part of the Standard LaTeX `Tools Bundle'.
% -------------------------------------------------------
%
% It may be distributed and/or modified under the
% conditions of the LaTeX Project Public License, either version 1.3c
% of this license or (at your option) any later version.
% The latest version of this license is in
%    https://www.latex-project.org/lppl.txt
% and version 1.3c or later is part of all distributions of LaTeX
% version 2005/12/01 or later.
%
% The list of all files belonging to the LaTeX `Tools Bundle' is
% given in the file `manifest.txt'.
%
% \fi
%
%%
%
%
\def\FMithmInfo{2023/07/05 v2.2c Theorem extension package (FMi)}
%
% \ProvidesFile{theorem.dtx}[\FMithmInfo]
%\iffalse   % this is a METACOMMENT !
%%
%% Package `theorem' to use with LaTeX2e
%% Copyright (C) 1989-1995 Frank Mittelbach, all rights reserved.
%<+package>\@ifundefined{theorem@style}{}{\endinput}
%<+package>\NeedsTeXFormat{LaTeX2e}[1995/06/01]
%<+package>\ProvidesPackage{theorem}
%<+package>         [\FMithmInfo]
%
% \fi
%
% \changes{v1.0}{87/05/07}{First draft.}
% \changes{v1.0a}{87/05/07}{`thistheoremstyle' added.}
% \changes{v1.1a}{87/07/29}{New concept: redefine the internal
%                          \protect\LaTeX{} commands `@xnthm' etc.}
% \changes{v1.1b}{87/07/31}{`theoremfont' used to specify font.}
% \changes{v1.2a}{88/01/08}{`@endtheorem' now globally defined.}
% \changes{v1.2b}{88/01/11}{`newtheorem' only allowed in the preamble.}
% \changes{v1.3a}{88/02/14}{`theoremfont' now a toks register.}
%
% \changes{v2.0a}{88/11/16}{New implementation and new documentation.}
% \changes{v2.0a}{88/11/16}{No layout changes but internal modifications
%                           old substyles won't work.}
% \changes{v2.0b}{89/01/30}{`global' used for all definitions.}
% \changes{v2.0c}{89/06/15}{Documentation translated to English.
%                            Thanks to LLL and Nina}
% \changes{v2.1a}{90/02/15}{Support for new font selection.}
% \changes{v2.2a}{94/02/03}{Cleanup for LaTeX2e}
%
% \DoNotIndex{\@endparenv,\@ifdefinable,\@ifnextchar,\@ifundefined}
% \DoNotIndex{\@namedef,\@nameuse,\@spaces,\@tempa,\@warning}
% \DoNotIndex{\ }
% \DoNotIndex{\begingroup,\bfseries}
% \DoNotIndex{\csname}
% \DoNotIndex{\def,\do,\docdate}
% \DoNotIndex{\endcsname,\endgroup,\endinput,\endtrivlist,\expandafter}
% \DoNotIndex{\filedate,\fileversion}
% \DoNotIndex{\gdef,\global}
% \DoNotIndex{\hbox,\hskip}
% \DoNotIndex{\ignorespaces,\input,\itshape,\item}
% \DoNotIndex{\labelsep,\let,\llap}
% \DoNotIndex{\makeatletter}
% \DoNotIndex{\noexpand}
% \DoNotIndex{\refstepcounter,\relax,\rlap}
% \DoNotIndex{\setlength,\slshape,\space,\string,\strut}
% \DoNotIndex{\the,\trivlist,\typeout}
% \DoNotIndex{\vbox}
% \DoNotIndex{\xdef}
%
% \def\deflabel#1{\ttfamily #1\hfill}
% \def\deflist#1{\begin{list}{}{\settowidth\labelwidth{\ttfamily #1}%
%                               \setlength\leftmargin\labelwidth
%                               \addtolength\leftmargin\labelsep
%                               \let\makelabel\deflabel}}
% \def\enddeflist{\end{list}}
% \def\envfont{\normalfont\ttfamily}
%
% \MakeShortVerb{\|}
%
% \newcommand{\TUB}{{\sl TUGboat\/}}
%
% \GetFileInfo{theorem.dtx}
% \title{An Extension of the
%        \LaTeX{} theorem environment\thanks{This file
%        has version number \fileversion, last
%        revised \filedate.}}
% \author{Frank Mittelbach}
%
% \MaintainedByLaTeXTeam{tools}
% \maketitle
%
% \begin{abstract}
%    The macros described in this paper yield an extension of the
%    \LaTeX{} theorem mechanism. It is designed to satisfy the
%    different requirements of various journals. Thus, the
%    layout of the ``theorems'' can be manipulated by determining a
%    ``style''. This article describes not only the use, but
%    also the definition, of the necessary macros.
% \end{abstract}
%
% \section*{Preface to version 2.2}
%
% For \LaTeXe{} this package did not need any fundamental changes. I
% only modified the messages generated so that theorem layout styles
% will show up with the |\listfiles| command and cleaned the section
% on the New Font Selection Scheme since this is now included in
% \LaTeX.
%
% \section*{Preface to version 2.1}
%
% This version is identical to 2.0g described in \TUB\ 10\#3 except
% for some internal defaults which are now set depending on the used
% font selection scheme.
%
% This was done to avoid unpleasant surprises if the new font selection
% scheme is in force. For further details see
% section~\ref{sec:newfonts} and \cite{bk:GMS94}.
%
% \section{Introduction}
%
% For our purposes here, ``theorems'' are labelled enunciations,
% often set off from the main text by extra space and a font change.
% Theorems, corollaries, conjectures, definitions, and remarks
% are all instances of ``theorems''.  The ``header'' of these structures
% is composed of a label (such as \textsc{Theorem} or \textsc{Remark})
% and a number which serializes an item in the sequence of items
% with the same label.
%
% Shortly after the introduction of \LaTeX{} at the Fachbereich
% Mathematik in Mainz, the desire to manipulate the layout of
% ``theorems'' arose.  In Mainz, the following two conventions came
% into general use:
% \begin{enumerate}
%   \item The number of the theorem is shown in the margin.
%   \item There is a line break at
%         the end of the theorem header.
% \end{enumerate}
%
% Additionally, some journals require different formats which depend on
% the ``sort of theorem'': e.g.\ often remarks and definitions are set
% in |\upshape|, while |\itshape| is employed for main theorems.
%
% Confronted with these requirements, a theorem environment was
% developed in Mainz which allows separate determination of the
% layout of the ``theorems sets'', comparable to |\pagestyle|.
%
%
% \section{The user interface}
% \subsection{Defining new theorem sets}
%
% \DescribeMacro\newtheorem
% As in the original \LaTeX{} version, the command |\newtheorem|
% defines a new ``theorem set'' or ``theorem-like structure''.
% Two required arguments name the new environment and give the text
% to be typeset with each instance of the new ``set'', while
% an optional argument determines how
% the ``set'' is enumerated:
% \begin{description}
%    \item[\ttfamily \bslash newtheorem\{foo\}\{bar\}]
%       The theorem set {\envfont foo} (whose name is \texttt{bar})
%       uses its own counter.
%    \item[\ttfamily \bslash newtheorem\{foo2\}{[foo]}\{bar2\}]
%       The theorem set {\envfont foo2} (printed name \texttt{bar2})
%       uses the same counter as the theorem set \texttt{foo}.
%    \item[\ttfamily \bslash newtheorem\{foo3\}\{bar3\}{[section]}]
%       The theorem set {\envfont foo3} (printed name \texttt{bar3}) is
%       enumerated within the counter \texttt{section}, i.e.\ with every
%       new |\section| the enumeration begins again with 1, and
%       the enumeration is composed from the section-number and the
%       theorem counter itself.
% \end{description}
%
% \DescribeMacro\theoremstyle
% Additionally, the command |\theoremstyle| can define the layout
% of various, or all, theorem sets.  It should be noted that any
% theorem set defined by |\newtheorem| is typeset in the
% |\theoremstyle| that is current at the time of the definition.
% Thus, the following
% \begin{verbatim}
%    \theoremstyle{break}        \newtheorem{Cor}{Corollary}
%    \theoremstyle{plain}        \newtheorem{Exa}{Example}[section]
%\end{verbatim}                             ^^A No blank at line start
% leads to the result that the set {\envfont Cor} is formatted in the
% style \texttt{break}, while the set {\envfont Exa} and all the
% following ones are formatted in the style \texttt{plain}, unless
% another |\theoremstyle| follows.  Since the definitions installed by
% |\newtheorem| are global, one also can limit |\theoremstyle| locally
% by grouping braces.
%
% \DescribeMacro\theorembodyfont
% The choice of the font for the theorem body is completely
% independent of the chosen |\theoremstyle|; this has proven to
% be very advantageous.  For example,
% \begin{verbatim}
%   {\theorembodyfont{\upshape}          \newtheorem{Rem}{Remark}}
%\end{verbatim}
% defines a theorem set {\envfont Rem}, which will be set in
% |\upshape| in the current layout (which in our example is
% \texttt{plain}).  As with |\theoremstyle|, the |\theorembodyfont|
% chosen is that current at the time of |\newtheorem|.  If
% |\theorembodyfont| is not specified or one defines
% |\theorembodyfont{}|, then the font used will be that defined by the
% |\theoremstyle|.
%
% \DescribeMacro\theoremheaderfont It is also possible to customize
% the font used for the theorem headers.  This is, however, a global
% declaration, and therefore there should be at most one
% |\theoremheaderfont| declaration in the preamble.\footnote{If it is
% actually necessary to have different header fonts, one has to define
% new theorem styles (substituting the desired font) or specify the
% information directly in the \texttt{\bslash newtheorem} declaration
% (the unclean variant).}
%
% \DescribeMacro\theorempreskipamount
% \DescribeMacro\theorempostskipamount Two additional parameters
% affect the vertical space around the theorem environments:
% |\theorempreskipamount| and |\theorempostskipamount| define,
% respectively, the spacing before and after such an environment.
% These parameters apply for all theorem sets and can be manipulated
% with the ordinary length macros.  They are rubber lengths
% (`\textsf{skips}'), and therefore can contain \texttt{plus} and
% \texttt{minus} parts.
%
% Since the definition of theorem sets should---most sensibly---be
% placed in the preamble, we only allow installation there.
% It is therefore possible to release the memory used here after
% |\begin{document}|, in order to make room for other
% applications.
%
%
% \subsection{Existing theorem styles}
%
% The following theorem styles exist to date:
% \begin{deflist}{marginbreak}
%    \item[plain]
%       This theorem style emulates the original \LaTeX{} definition,
%       except that additionally the parameters
%       |\theorem...skipamount| are used.
%    \item[break]
%       In this style, the theorem header is followed by a line
%       break.
%    \item[marginbreak]
%       The theorem number is set in the margin, and there is a line
%       break as in \texttt{break}.
%    \item[changebreak]
%       Like \texttt{break}, but with header number and text
%       interchanged.
%    \item[change]
%       Header number and text are interchanged, without a line break.
%    \item[margin]
%       The number is set in the left margin, without a line break.
% \end{deflist}
% All styles (except \texttt{plain}) select |\slshape| as the default
% |\theorembodyfont|.
%
%
% \subsection{Examples}
%
% Given the above theorem sets {\envfont Cor}, {\envfont Exa}
% and {\envfont Rem},
% suppose that the preamble also contains the declarations:
% \begin{verbatim}
%    \theoremstyle{marginbreak}   \newtheorem{Lem}[Cor]{Lemma}
%    \theoremstyle{change}
%    \theorembodyfont{\itshape}        \newtheorem{Def}[Cor]{Definition}
%
%    \theoremheaderfont{\scshape}
%\end{verbatim}
% Then the following are some typical examples of the typeset output
% resulting from their use.
%
% \begin{Cor}
%    This is a sentence typeset in the theorem environment {\envfont
%    Cor}.
% \end{Cor}
%
% \begin{Exa}
%    This is a sentence typeset in the theorem environment {\envfont
%    Exa}.
% \end{Exa}
%
% \begin{Rem}
%    This is a sentence typeset in the theorem environment {\envfont
%    Rem}.
% \end{Rem}
%
% \begin{Lem}[Ben User]
%    This is a sentence typeset in the theorem environment {\envfont
%    Lem}.
% \end{Lem}
%
% \begin{Def}[Very impressive Definition]
%    This is a sentence typeset in the theorem environment {\envfont
%    Def}.
% \end{Def}
%
% The last two examples show the effect of the optional argument to
% a theorem environment (it is the text typeset in parentheses).
%
%
% \section{Special Considerations}
% \label{sec:newfonts}
%
% Theoremheader and body are implemented as a unit. This means that
% the |\theoremheaderfont|  will inherit characteristics of the
% |\theorembodyfont| in \LaTeXe. Thus, if for example
% |\theorembodyfont| is |\itshape| and |\theoremheaderfont|
% is |\bfseries| the font selected for the header will have the
% characteristics `bold extended italic'. If this is not desired one
% should set the |\theoremheaderfont| to something like
% \begin{verbatim}
%      \theoremheaderfont{\normalfont\bfseries}
%\end{verbatim}
% i.e.\ supplying all necessary font information explicitly.
%
%
%
% \section{Acknowledgements}
%
% The publication of this set of macros was only possible with the
% help of Christina Busse (translating the manuscript into English),
% Joachim Pense (playing the r\^ole of typist), Chris Rowley
% (looking everything over) and many others providing useful
% suggestions.
%
% \MaybeStop{%
%    \begin{thebibliography}{9}
%       \bibitem{bk:GMS94} \textsc{M.~Goossens}, \textsc{F.~Mittelbach}
%         and \textsc{A.~Samarin}.
%         \newblock The \LaTeX{} Companion.
%         \newblock
%         Addison-Wesley, Reading, Massachusetts, 1994.
%       \bibitem{latex:source} \textsc{Lamport, Leslie.} \newblock
%                               \texttt{latex.tex}, version 2.09, date
%                               Feb. 1990.
%    \end{thebibliography}
%    \PrintIndex
%    \PrintChanges}
%
%
% \section{The documentation driver file}
%
% The next bit of code contains the documentation driver file for
% \TeX{}, i.e., the file that will produce the documentation you are
% currently reading. It will be extracted from this file by the
% \texttt{docstrip} program. Since it is the first code in the file
% one can alternatively process this file directly with \LaTeXe{} to
% obtain the documentation.
%
%    \begin{macrocode}
%<*driver>
\documentclass{ltxdoc}

\usepackage{theorem}

%<+driver>% The next few lines define theorem sets which are used
%<+driver>% in the example section of the documentation.

\theoremstyle{break}       \newtheorem{Cor}{Corollary}
\theoremstyle{plain}       \newtheorem{Exa}{Example}[section]
{\theorembodyfont{\upshape}\newtheorem{Rem}{Remark}}
\theoremstyle{marginbreak} \newtheorem{Lem}[Cor]{Lemma}
\theoremstyle{change}
\theorembodyfont{\itshape} \newtheorem{Def}[Cor]{Definition}

\theoremheaderfont{\scshape}

\RecordChanges

\begin{document}
   \DocInput{theorem.dtx}
\end{document}
%</driver>
%    \end{macrocode}
%
% \section{Definition of the Macros}
%
% If the file has been loaded before, we abort immediately. If not the
% package announces itself (this is actually done at the very top of
% the file---the way it is done isn't good style so don't copy it).
% \changes{v2.0e}{89/07/19}{Spaces removed from `typeout'.}
% \changes{v2.2b}{95/11/19}{Announce moved to top of file}
%    \begin{macrocode}
%<*package>
%\@ifundefined{theorem@style}{}{\endinput}
%\def\FMithmInfo{1995/11/19 v2.2b Theorem extension package (FMi)}
%\ProvidesPackage{theorem}[\FMithmInfo]
%    \end{macrocode}
%
%
% \subsection{Definition of theorem styles and fonts}
%
% All the definitions in this file are done globally to allow
% inputting this file inside a group.
%
% \begin{macro}{\theoremstyle}
%    Before a theorem style can be installed, the chosen style must be
%    known. For that reason, we must test to see that
%    |\th@|\meta{style} is known or, more precisely, that it is
%    different from |\relax|. If the style is not known then
%    |\th@plain| is used.
%    \begin{macrocode}
\gdef\theoremstyle#1{%
   \@ifundefined{th@#1}{\@warning
          {Unknown theoremstyle `#1'. Using `plain'}%
          \theorem@style{plain}}%
%    \end{macrocode}
%    We save the theorem style to be used in the token register
%    |\theorem@style|.
%    \begin{macrocode}
      {\theorem@style{#1}}%
%    \end{macrocode}
%    Now we ``evaluate'' the theorem style: this means, we call the
%    macro |\th@|\meta{style} which will activate the relevant
%    definitions which are contained in a separate file.  This is done
%    in a group to suppress changes to the current font.  This could
%    otherwise pose problems together with the new font selection
%    scheme\footnote{When I printed the original article using the new
%    font selection scheme I ended with a document with slanted
%    typefaces (text headings and all) simply because one of the
%    theorem styles used \texttt{\bslash sl} at toplevel.} if the
%    |\th@|\meta{style} is evaluated a second time.
% \changes{v2.1a}{90/02/15}{Group added around `th@..}
% \changes{v2.1a}{90/02/15}{Reseting of `..@begintheorem removed.}
%    \begin{macrocode}
      \begingroup
        \csname th@\the\theorem@style \endcsname
      \endgroup}
%    \end{macrocode}
% \end{macro}
%
%
% \begin{macro}{\@begintheorem}
% \changes{v2.1a}{90/02/15}{Relaxed at toplevel.}
% \begin{macro}{\@opargbegintheorem}
% \changes{v2.1a}{90/02/15}{Relaxed at toplevel.}
%    We reset |\@begintheorem| and |\@opargbegintheorem| to
%    |\relax| since these commands are no longer necessary at
%    toplevel. This will save a few tokens.
%    \begin{macrocode}
\global\let\@begintheorem\relax
\global\let\@opargbegintheorem\relax
%    \end{macrocode}
% \end{macro}
% \end{macro}
%
% \begin{macro}{\theorem@style}
%    Obviously the \textsf{token} register used above has to be
%    allocated. To assure the utmost compatibility with the
%    original \LaTeX{} definition, we set the default theorem style to
%    \texttt{plain}, which implements the usual \LaTeX{} convention.
% \changes{v2.0f}{89/07/20}{Unnecessary `global' removed.}
%    \begin{macrocode}
\newtoks\theorem@style
\global\theorem@style{plain}
%    \end{macrocode}
% \end{macro}
%
% \begin{macro}{\theorembodyfont}
% \begin{macro}{\theorem@bodyfont}
%    If the |\theorembodyfont| is set by the user then it should not
%    interact with the default font set in the theorem style. When the
%    new font selection is in force this may happen if, for example,
%    the default is |\itshape| and the new |\theorembodyfont| is
%    |\sffamily|. So we add a |\reset@font| command in front of the
%    user definition.
% \changes{v2.0f}{89/07/20}{Unnecessary `global' removed.}
% \changes{v2.0g}{89/09/19}{`theoremfont to `theorembodyfont.}
% \changes{v2.1c}{92/06/22}{`reset@font added and logic changed.}
% \changes{v2.1d}{92/10/28}{Reset to empty if arg is empty}
%    \begin{macrocode}
\gdef\theorembodyfont#1{%
%    \end{macrocode}
%    We check if the argument supplied is empty and if so put nothing
%    into the |\theorem@bodyfont| token register to allow for
%    |\theorembodyfont{}| as a mean of using the default of the
%    current |\theoremstyle|.
%    \begin{macrocode}
   \def\@tempa{#1}%
   \ifx\@tempa\@empty
    \theorem@bodyfont{}%
   \else
    \theorem@bodyfont{\reset@font#1}%
   \fi
   }
\newtoks\theorem@bodyfont
\global\theorem@bodyfont{}
%    \end{macrocode}
% \end{macro}
% \end{macro}
%
%
% \begin{macro}{\theoremheaderfont}
% \changes{v2.0g}{89/09/19}{Macro added.}
%    The font for the theorem headers is handled differently because
%    this definition applies to all theorem styles.
%    \begin{macrocode}
\gdef\theoremheaderfont#1{\gdef\theorem@headerfont{#1}%
%    \end{macrocode}
%    After using the macro once it is redefined to produce an error
%    message.
%    \begin{macrocode}
       \gdef\theoremheaderfont##1{%
        \typeout{\string\theoremheaderfont\space should be used
                 only once.}}}
%    \end{macrocode}
% \end{macro}
%
% \begin{macro}{\theorem@headerfont}
% \changes{v2.0g}{89/09/19}{Macro added.}
%    To set the |\theorem@headerfont| default we first test if
%    the new fontselection scheme is in force.
% \changes{v2.1a}{90/02/15}{Support for new font selection.}
%    \begin{macrocode}
\ifx\upshape\undefined
%    \end{macrocode}
%    If not we define it to expand into |\bfseries|. We don't use
%    |\let| just in case a following style option redefines this
%    macro.
%    \begin{macrocode}
\gdef\theorem@headerfont{\bfseries}
%    \end{macrocode}
%    Otherwise we reset the current shape before calling |\bfseries|.
%    \begin{macrocode}
\else \gdef\theorem@headerfont{\normalfont\bfseries}\fi
%    \end{macrocode}
% \end{macro}
%
% \begin{macro}{\th@plain}
% \begin{macro}{\th@break}
% \begin{macro}{\th@marginbreak}
% \begin{macro}{\th@changebreak}
% \begin{macro}{\th@change}
% \begin{macro}{\th@margin}
%    The different styles are defined in macros such as
%    |\th@plain|.  Since memory space is precious in
%    ``non-Big-versions'', we have to avoid offering too many unused
%    definitions.  Therefore we define these styles in separate files
%    that can be loaded on demand.  Thus the commands themselves only
%    load these files. We use |\@input@|, a \LaTeXe{} internal command
%    that ensures that the file will be listed with |\listfiles|.
% \changes{v2.2a}{94/02/02}{Use `@input@ to load theorem layout files}
%    \begin{macrocode}
\gdef\th@plain{\@input@{thp.sty}}
\gdef\th@break{\@input@{thb.sty}}
\gdef\th@marginbreak{\@input@{thmb.sty}}
\gdef\th@changebreak{\@input@{thcb.sty}}
\gdef\th@change{\@input@{thc.sty}}
\gdef\th@margin{\@input@{thm.sty}}
%    \end{macrocode}
%    This list will be expanded when new styles become
%    available.  For testing,  just append new theorem substyles
%    as document options.
% \end{macro}
% \end{macro}
% \end{macro}
% \end{macro}
% \end{macro}
% \end{macro}
%
%
%
% \subsection{Definition of a new theorem set}
%
% As already pointed out, a new theorem environment can be defined in
% three different ways:
% \begin{verbatim}
%    \newtheorem{Lem}{Lemma}
%    \newtheorem{Lem}{Lemma}[section]
%    \newtheorem{Lem}[Theorem]{Lemma}
%\end{verbatim}
% The function of the macro |\newtheorem| is to recognize these cases
% and then to branch into one of the three macros |\@ynthm|, |\@xnthm|
% or |\@othm|.  This mechanism is adopted unchanged from
% \cite{latex:source}; the essential point here is that, for example,
% in the second case, the arguments \texttt{Lem}, \texttt{Lemma} and
% \texttt{section} are passed over to the macro |\@xnthm|.
%
% We inspect this case first because the others present fewer problems,
% and thus are easily derived from this one.
%
% \begin{macro}{\@xnthm}
%    For our example arguments, the macro |\@xnthm| must fulfill
%    the following:
%    \begin{itemize}
%       \item
%          Define a new \LaTeX-counter `Lem'
%       \item
%          reset this counter within a |\section|
%       \item
%          define the macro |\theLem|
%       \item
%          define the environment macros |\Lem| and
%          |\endLem| using the current
%          |\theoremstyle| and |\theorem@bodyfont|.
%    \end{itemize}
%    Obviously, all this should happen only if the first argument
%    of |\@xnthm| (i.e.\ \texttt{Lem} in our example) is chosen so
%    as not to conflict with any previously defined commands or
%    environments.  This test is performed by the \LaTeX{} macro
%    |\@ifdefinable|.
%    \begin{macrocode}
\gdef\@xnthm#1#2[#3]{\expandafter\@ifdefinable\csname #1\endcsname
%    \end{macrocode}
%    Therefore, the first argument of |\@ifdefinable| is the
%    expansion (in the example, |\Lem|) of
%    |\csname||#1\endcsname|.  The second argument is
%    executed only if the test has been completed successfully.
%    \begin{macrocode}
   {%
%    \end{macrocode}
%    Now we define the new counter. The names of the \LaTeX{}
%    macros employed should speak for themselves:
% \changes{v2.2b}{1995/10/07}{Use \cs{@newctr}}
% \changes{v2.2c}{1995/11/23}{Use \cs{@newctr} properly.}
%    \begin{macrocode}
    \@definecounter{#1}\@newctr{#1}[#3]%
%    \end{macrocode}
%    Using |\@newctr| will give a proper error message if the counter
%    in |#3| is not defined.
%    In defining `|\theLem|' we must generate the desired macro
%    name by use of |\expandafter| and |\csname|.
%    \begin{macrocode}
    \expandafter\xdef\csname the#1\endcsname
%    \end{macrocode}
%    An |\xdef| is used in order to make the definition global,
%    and to ensure that it contains the replacement texts of
%    |\@thmcountersep| and |\@thmcounter|.\footnote{These
%    two macros can be defined by the document style.  Their default
%    values produce a `\texttt{.}' as separation and an arabic
%    representation of the number.} However, not everything should be
%    expanded. For example, it saves space to use |\thesection|
%    instead of its---at times---lengthy expansion.
%    \begin{macrocode}
      {\expandafter \noexpand \csname the#3\endcsname
       \@thmcountersep \@thmcounter{#1}}%
%    \end{macrocode}
%    Thus with the defaults of \LaTeX{}, |\theLem| would be
%    replaced by the command sequence |\thesection.\arabic{Lem}|.
%
%    We will now look at the definition of the macro which is executed
%    at the beginning of the actual environment (in our example this
%    macro is |\Lem|). It should be noted that we use an
%    ``|\expandafter| trick'' to expand only certain parts of the
%    replacement text at the time of the definition.
%    \begin{macrocode}
    \def\@tempa{\global\@namedef{#1}}%
    \expandafter \@tempa \expandafter{%
%    \end{macrocode}
%    First, the macro that contains the current definitions of
%    |\@begintheorem| and |\@opargtheorem| should be called
%    up.  The name of this macro---as is already known---has the form
%    |\th@|\meta{theorem\ style}; therefore, it must be called by
%    \begin{macrocode}
      \csname th@\the \theorem@style
%    \end{macrocode}
%    In addition the default theorem font should be changeable, i.e.\
%    we have to insert the contents of |\theorem@bodyfont|. For
%    that reason, we expand even further, beyond |\endcsname|,
%    and thus insert the contents of the \textsf{token} register
%    |\theorem@bodyfont| in the replacement text.
%    \begin{macrocode}
            \expandafter \endcsname \the \theorem@bodyfont
%    \end{macrocode}
%    Now it is time to call the macro |\@thm| which takes over
%    the further processing.  It has two arguments: the current
%    counter name (in our example, \texttt{Lem}), and the text of the
%    label (in our example, \texttt{Lemma}).
%    \begin{macrocode}
     \@thm{#1}{#2}}%
%    \end{macrocode}
%    With this, the `sub-definition' is complete.  The macro
%    |\@endtheorem| ends a theorem environment and is, so far, nothing
%    but an |\endtrivlist|.  (Hence it is defined globally, and not
%    within the theorem styles.\footnote{This has to be changed as
%    soon as theorem styles that change \texttt{\bslash @endtheorem}
%    exist.  In such a case, all existing styles must be changed as
%    well since they will have to reset the macro.}) Therefore, we can
%    set it equivalent to the macro that ends the theorem set (in our
%    example, |\endLem|).  However, if some day theorem styles exist
%    that do change |\@endtheorem|, we would have to use the
%    commented-out line instead.
%    \begin{macrocode}
    \global \expandafter \let \csname end#1\endcsname \@endtheorem
%   \global\@namedef{end#1}{\@endtheorem}%
%    \end{macrocode}
%    With these commands all the required definitions are employed,
%    unless the test |\@ifdefinable| has failed.  Therefore, we
%    end the second argument of this macro and with it the definition
%    of |\@xnthm|.
%    \begin{macrocode}
   }}
%    \end{macrocode}
% \end{macro}
%
% \begin{macro}{\@ynthm}
%    The definition of |\@ynthm| is completely analogous. In
%    this case the new counter that is defined is not reset within
%    another counter; thus the definition of |\the...|
%    is simplified:
%    \begin{macrocode}
\gdef\@ynthm#1#2{\expandafter\@ifdefinable\csname #1\endcsname
   {\@definecounter{#1}%
    \expandafter\xdef\csname the#1\endcsname{\@thmcounter{#1}}%
%    \end{macrocode}
%    The rest of the definition corresponds literally to that of
%    |\@xnthm|:
%    \begin{macrocode}
    \def\@tempa{\global\@namedef{#1}}\expandafter \@tempa
     \expandafter{\csname th@\the \theorem@style \expandafter
     \endcsname \the\theorem@bodyfont \@thm{#1}{#2}}%
    \global \expandafter \let \csname end#1\endcsname \@endtheorem}}
%    \end{macrocode}
% \end{macro}
%
% \begin{macro}{\@othm}
%    The definition of |\@othm| does not contain anything new.
%    \begin{macrocode}
\gdef\@othm#1[#2]#3{%
%    \end{macrocode}
% \changes{v2.2b}{95/10/07}{Build in check for counter defined}
%    We do not define a new counter but instead use one that has
%    already been defined.  Thus the only definition we need is that
%    of this pseudo-counter (i.e.\ |\the|\meta{env.\ name}).
%    First we check if |#2| corresponds to a known counter name.
%    \begin{macrocode}
  \expandafter\ifx\csname c@#2\endcsname\relax
   \@nocounterr{#2}%
  \else
   \expandafter\@ifdefinable\csname #1\endcsname
   {\expandafter \xdef \csname the#1\endcsname
     {\expandafter \noexpand \csname the#2\endcsname}%
%    \end{macrocode}
%    All other parts of the definition can be adopted from
%    |\@xnthm|. We have to remember, though, that in this case
%    the name of the current counter and the theorem label have moved
%    to the second and third arguments.
%    \begin{macrocode}
    \def\@tempa{\global\@namedef{#1}}\expandafter \@tempa
     \expandafter{\csname th@\the \theorem@style \expandafter
     \endcsname \the\theorem@bodyfont \@thm{#2}{#3}}%
    \global \expandafter \let \csname end#1\endcsname \@endtheorem}%
  \fi}
%    \end{macrocode}
% \end{macro}
%
% \subsection{Macros that are employed in a {\envfont theorem}
%    environment}
%
% \begin{macro}{\@thm}
%    The macro |\@thm| has to increase the current counter.  Then,
%    depending on whether the environment has (or does not have) an
%    optional argument, it has to branch into either |\@begintheorem|
%    or |\@opargtheorem|.
%    \begin{macrocode}
\gdef\@thm#1#2{\refstepcounter{#1}%
%    \end{macrocode}
%    Now we start a {\envfont trivlist} environment, and give
%    |\@topsep| and |\@topsepadd| the values of the \textsf{skip}
%    registers |\theorempreskipamount| and |\theorempostskipamount|.
%    The value in |\@topsep| is the vertical space that is inserted by
%    the first (and only) |\item| in our |\trivlist| whilst
%    |\@topsepadd| is inserted by |\@endparenv| at the end of that
%    {\envfont trivlist} environment.  By using these registers, we
%    obtain the desired space around a {\envfont theorem} environment.
%    \begin{macrocode}
   \trivlist
   \@topsep \theorempreskipamount               % used by first \item
   \@topsepadd \theorempostskipamount           % used by \@endparenv
%    \end{macrocode}
%    Now we have to test whether an optional argument has been given.
% \changes{v2.0d}{89/07/18}{Forgotten percent added.}
%    \begin{macrocode}
   \@ifnextchar [%
%    \end{macrocode}
%    If there is an optional argument, we will call |\@ythm|, and
%    move the arguments read back into the input stream.
%    \begin{macrocode}
   {\@ythm{#1}{#2}}%
%    \end{macrocode}
%    If not, we call |\@begintheorem|. Its first argument is the
%    name of the theorem set (hence the second argument of
%    |\@thm|). Its second argument is the macro that produces the
%    current number.
%    \begin{macrocode}
   {\@begintheorem{#2}{\csname the#1\endcsname}\ignorespaces}}
%    \end{macrocode}
% \end{macro}
%
% \begin{macro}{\@xthm}
% \begin{macro}{\@ythm}
%    Both these macros were originally called by |\@thm|.  We do
%    not need |\@xthm| anymore, hence we reset it to
%    |\relax|. The definition of |\@ythm| has not changed at
%    all from its definition in \LaTeX. In order to make the macros
%    easier to understand, we will
%    nevertheless present it (commented out).
%    \begin{macrocode}
\global\let\@xthm\relax
% \def\@ythm#1#2[#3]{\@opargbegintheorem{#2}{\csname
%           the#1\endcsname}{#3}\ignorespaces}
%</package>
%    \end{macrocode}
%    The primitive |\ignorespaces| in |\@ythm| and
%    |\@thm| is needed to remove the spaces between the
%    |\begin{...}| and the actual text.
% \end{macro}
% \end{macro}
%
%
% \subsection{Definition of the theorem substyles}
%
% As already pointed out, the theorem substyles, defined below, are
% only loaded when necessary.  Note that all these substyles, except
% \texttt{plain}, have |\slshape| as the default body font.
%
% \subsubsection{The \texttt{plain} style}
%
% \label{dok:th@plain} As the following macros use |@|, we have
% to locally set the |\catcode| of this symbol to ``letter''.
% This happens within a group, so that we do not have to worry about
% which |\catcode| that symbol had before.
%    \begin{macrocode}
%<*thp>
\begingroup \makeatletter
%    \end{macrocode}
% Since we are now within a group, we must make all definitions
% globally.  First we make sure that \texttt{theorem.sty} is loaded.
% This will allow us to use this file as a document style option
% without having to call \texttt{theorem} itself as an option.  At the
% same time, we assure that at least version 2 is loaded, since
% |\theorem@style| was not defined in earlier versions.
%    \begin{macrocode}
\@ifundefined{theorem@style}{\input{theorem.sty}}{}
\ProvidesFile{thp.sty}
             [\FMithmInfo]
%    \end{macrocode}
%
%
% \begin{macro}{\th@plain}
% \begin{macro}{\@begintheorem}
% \begin{macro}{\@opargbegintheorem}
%    |\theoremstyle{plain}| corresponds to the original
%    definition, except that the distances to the surrounding text are
%    determined by the parameters |\theorempreskipamount| and
%    |\theorempostskipamount|.
%    First we set the default body font.
%    \begin{macrocode}
\gdef\th@plain{\normalfont\itshape
%    \end{macrocode}
%    Then we define |\@begintheorem| and
%    |\@opargbegintheorem|. These two macros define how the
%    header of a theorem is typeset.  |\@opargbegintheorem| will
%    be called if a \textsf{theorem} environment with an optional
%    argument is encountered; otherwise, the header is constructed by
%    calling |\@begintheorem|.  If one of these macros is
%    executed, we are within a \textsf{trivlist} environment started by
%    |\@thm|.  So the theorem header is produced with an
%    |\item| command.
%
%    Instead of specifying the header font directly, all standard
%    theorem styles use the |\theorem@headerfont| macro to allow
%    customization.  The extra space (|\labelsep|) is necessary
%    because of problems in the \textsf{trivlist} environment.
%    \begin{macrocode}
  \def\@begintheorem##1##2{%
        \item[\hskip\labelsep \theorem@headerfont ##1\ ##2]}%
%    \end{macrocode}
%    The definition of |\@opargbegintheorem| is completely
%    analogous. The only difference is the fact that there exists a
%    third argument (which is the optional parameter of the
%    environment and contains additional information about the
%    theorem).  Customarily we enclose it in parentheses.
%    \begin{macrocode}
\def\@opargbegintheorem##1##2##3{%
   \item[\hskip\labelsep \theorem@headerfont ##1\ ##2\ (##3)]}}
%    \end{macrocode}
% \end{macro}
% \end{macro}
% \end{macro}
%
% \noindent
% We conclude with an |\endgroup| to restore the
% |\catcode| of |@|.
%    \begin{macrocode}
\endgroup
%</thp>
%    \end{macrocode}
%
%
% \subsubsection{The \texttt{break} style}
%
% This style option is stored in the file |thb.sty|.
% For the next two lines see the documentation for |\th@plain| on
% page~\pageref{dok:th@plain}.
%    \begin{macrocode}
%<*thb>
\begingroup \makeatletter
\@ifundefined{theorem@style}{\input{theorem.sty}}{}
\ProvidesFile{thb.sty}
             [\FMithmInfo]
%    \end{macrocode}
%
% \begin{macro}{\th@break}
%    |\theoremstyle{break}| produces a line break after the name
%    of the theorem.  The font is |\slshape|.  Hence, we define
%    |\th@break| as follows:
%    \begin{macrocode}
\gdef\th@break{\normalfont\slshape
  \def\@begintheorem##1##2{\item[%
%    \end{macrocode}
%    We run into the following problem: it is not possible to create
%    the header with |\item[|\meta{title}|]| and then start
%    a new line by, for example, |\mbox{}\\|. Such a definition
%    will fail whenever a list environment follows immediately.  With
%    the above construction, the |\mbox{}| causes the switch
%    |@inlabel| (cf.\ definition of |\list| and
%    |\trivlist| in \cite{latex:source}) to be set to \texttt{false}
%    and so the following list will insert additional vertical
%    space (|\topskip|). This is quite annoying. Therefore, we
%    create the line break within the |\item|. In order to ensure
%    that the text will begin at the proper position in the following
%    line, we simply pretend that the label does not take any
%    room.\footnote{This will lead to problems whenever very high
%    symbols occurring in the line tower into the
%    heading. So, something else has to be done here sometime.}
%    \begin{macrocode}
       \rlap{\vbox{\hbox{\hskip \labelsep\theorem@headerfont ##1\ ##2}%
                   \hbox{\strut}}}]}%
%    \end{macrocode}
%    Again, the definition of |\@opargbegintheorem| is
%    completely analogous.
%    \begin{macrocode}
\def\@opargbegintheorem##1##2##3{%
  \item[\rlap{\vbox{\hbox{\hskip \labelsep \theorem@headerfont
                     ##1\ ##2\ (##3)}%
                    \hbox{\strut}}}]}}
\endgroup
%</thb>
%    \end{macrocode}
% \end{macro}
%
%
% \subsubsection{The \texttt{changebreak} style}
%
%    \begin{macrocode}
% This style option is stored in the file |thcb.sty|.
%    \begin{macrocode}
%<*thcb>
\begingroup \makeatletter
\@ifundefined{theorem@style}{\input{theorem.sty}}{}
\ProvidesFile{thcb.sty}
             [\FMithmInfo]
%    \end{macrocode}
%
% \begin{macro}{\th@changebreak}
%    The \texttt{change-break} theorem style is like \texttt{break}
%    but with interchange of theorem name and theorem number.  Thus we
%    define |\th@changebreak| as follows:
%    \begin{macrocode}
\gdef\th@changebreak{\normalfont\slshape
  \def\@begintheorem##1##2{\item
       [\rlap{\vbox{\hbox{\hskip\labelsep \theorem@headerfont ##2\ ##1}%
                    \hbox{\strut}}}]}%
\def\@opargbegintheorem##1##2##3{%
  \item[\rlap{\vbox{\hbox{\hskip\labelsep \theorem@headerfont
                     ##2\ ##1\ (##3)}%
                    \hbox{\strut}}}]}}
\endgroup
%</thcb>
%    \end{macrocode}
% \end{macro}
%
%
% \subsubsection{The \texttt{change} style}
%
% This style option is stored in the file |thc.sty|.
%    \begin{macrocode}
%<*thc>
\begingroup \makeatletter
\@ifundefined{theorem@style}{\input{theorem.sty}}{}
\ProvidesFile{thc.sty}
             [\FMithmInfo]
%    \end{macrocode}
%
% \begin{macro}{\th@change}
%    The \texttt{change} theorem style corresponds to the
%    \texttt{change break} style without a linebreak after the header.
%    To say it in another way, it's the same as the \texttt{plain}
%    style but with number and name interchanged and |\slshape| as the
%    default font.
%    \begin{macrocode}
\gdef\th@change{\normalfont\slshape
  \def\@begintheorem##1##2{\item
       [\hskip\labelsep \theorem@headerfont ##2\ ##1]}%
\def\@opargbegintheorem##1##2##3{%
  \item[\hskip\labelsep \theorem@headerfont ##2\ ##1\ (##3)]}}
\endgroup
%</thc>
%    \end{macrocode}
% \end{macro}
%
%
% \subsubsection{The \texttt{marginbreak} style}
%
% This style option is the one used most often at Mainz.  It is saved
% in the file |thmb.sty|.
%    \begin{macrocode}
%<*thmb>
\begingroup \makeatletter
\@ifundefined{theorem@style}{\input{theorem.sty}}{}
\ProvidesFile{thmb.sty}
             [\FMithmInfo]
%    \end{macrocode}
%
% \begin{macro}{\th@marginbreak}
%    The \texttt{margin break} style is nearly the same as the
%    \texttt{change break} style. The only difference is the placement
%    of the theorem number.  We use |\llap| to place it in the left
%    margin.
%
%    In this style |\labelsep| denotes the separation between the
%    number and the text.
%    \begin{macrocode}
\gdef\th@marginbreak{\normalfont\slshape
  \def\@begintheorem##1##2{\item
       [\rlap{\vbox{\theorem@headerfont
                    \hbox{\llap{##2}\hskip\labelsep ##1}%
                    \hbox{\strut}}}]}%
\def\@opargbegintheorem##1##2##3{%
  \item[\rlap{\vbox{\theorem@headerfont
                    \hbox{\llap{##2}\hskip\labelsep ##1\ (##3)}%
                    \hbox{\strut}}}]}}
\endgroup
%</thmb>
%    \end{macrocode}
% \changes{v2.1c}{90/02/17}{`endgroup outside macrocode env.}
% \end{macro}
%
%
%
% \subsubsection{The \texttt{margin} style}
%
% This style option is stored in the file |thm.sty|.
%    \begin{macrocode}
%<*thm>
\begingroup \makeatletter
\@ifundefined{theorem@style}{\input{theorem.sty}}{}
\ProvidesFile{thm.sty}
             [\FMithmInfo]
%    \end{macrocode}
%
% \begin{macro}{\th@margin}
%    Again this is only a variant of the theorem styles described
%    above without any new ideas.
%    \begin{macrocode}
\gdef\th@margin{\normalfont\slshape
  \def\@begintheorem##1##2{\item
       [\theorem@headerfont \llap{##2}\hskip\labelsep ##1]}%
\def\@opargbegintheorem##1##2##3{%
  \item[\theorem@headerfont \llap{##2}\hskip\labelsep ##1\ (##3)]}}
\endgroup
%</thm>
%    \end{macrocode}
% \end{macro}
%
%
%
%
% \subsection{Final Definitions}
%
% \begin{macro}{\theorempreskipamount}
% \begin{macro}{\theorempostskipamount}
%    The \textsf{skip} parameters that regulate the vertical empty space
%    before and after the {\envfont theorem} environment have
%    to be allocated as well.
% \changes{v2.0f}{89/07/20}{Unnecessary `global' removed.}
%    \begin{macrocode}
%<*package>
\newskip\theorempreskipamount
\newskip\theorempostskipamount
%    \end{macrocode}
%    Since we have used the same values for all theorem sets, we now can
%    assign them.
%    \begin{macrocode}
\global\setlength\theorempreskipamount{12pt plus 5pt minus 3pt}
\global\setlength\theorempostskipamount{8pt plus 3pt minus 1.5pt}
%    \end{macrocode}
% \end{macro}
% \end{macro}
%
%
% \begin{macro}{\@endtheorem}
%    The same holds for the macro |\@endtheorem|, which ends a
%    {\envfont theorem} environment. Since it is the same for all
%    theorem sets, it is removed from the macros |\th@|\meta{style}.
%    It simply ends the {\envfont trivlist} environment, which was
%    begun in |\@thm|.
%    \begin{macrocode}
\global\let\@endtheorem=\endtrivlist
%    \end{macrocode}
% \end{macro}
%
%
%
% \begin{macro}{\@preamblecmds}
%    All macros defined above are to be used only in the preamble.
%    Therefore, we insert them in |\@preamblecmds| which will disable
%    them at begin document. This is done by the internal \LaTeXe{}
%    command |\@onlypreamble|.
% \changes{v2.0g}{89/09/19}{`theoremheaderfont and `theorembodyfont
%    added.}
% \changes{v2.2a}{94/02/03}{Use \LaTeXe{} interface}
%    \begin{macrocode}
\@onlypreamble\@xnthm
\@onlypreamble\@ynthm
\@onlypreamble\@othm
\@onlypreamble\newtheorem
\@onlypreamble\theoremstyle
\@onlypreamble\theorembodyfont
\@onlypreamble\theoremheaderfont
%    \end{macrocode}
% \end{macro}
%
% Finally we declare the \texttt{plain} theorem style to be the default.
% \changes{v2.1b}{90/06/04}{Fixed missing `item by declaring default
%    style.}
%    \begin{macrocode}
\theoremstyle{plain}
%</package>
%    \end{macrocode}
%
% \Finale

\endinput
}{}
\ProvidesFile{thp.sty}
             [\FMithmInfo]
%    \end{macrocode}
%
%
% \begin{macro}{\th@plain}
% \begin{macro}{\@begintheorem}
% \begin{macro}{\@opargbegintheorem}
%    |\theoremstyle{plain}| corresponds to the original
%    definition, except that the distances to the surrounding text are
%    determined by the parameters |\theorempreskipamount| and
%    |\theorempostskipamount|.
%    First we set the default body font.
%    \begin{macrocode}
\gdef\th@plain{\normalfont\itshape
%    \end{macrocode}
%    Then we define |\@begintheorem| and
%    |\@opargbegintheorem|. These two macros define how the
%    header of a theorem is typeset.  |\@opargbegintheorem| will
%    be called if a \textsf{theorem} environment with an optional
%    argument is encountered; otherwise, the header is constructed by
%    calling |\@begintheorem|.  If one of these macros is
%    executed, we are within a \textsf{trivlist} environment started by
%    |\@thm|.  So the theorem header is produced with an
%    |\item| command.
%
%    Instead of specifying the header font directly, all standard
%    theorem styles use the |\theorem@headerfont| macro to allow
%    customization.  The extra space (|\labelsep|) is necessary
%    because of problems in the \textsf{trivlist} environment.
%    \begin{macrocode}
  \def\@begintheorem##1##2{%
        \item[\hskip\labelsep \theorem@headerfont ##1\ ##2]}%
%    \end{macrocode}
%    The definition of |\@opargbegintheorem| is completely
%    analogous. The only difference is the fact that there exists a
%    third argument (which is the optional parameter of the
%    environment and contains additional information about the
%    theorem).  Customarily we enclose it in parentheses.
%    \begin{macrocode}
\def\@opargbegintheorem##1##2##3{%
   \item[\hskip\labelsep \theorem@headerfont ##1\ ##2\ (##3)]}}
%    \end{macrocode}
% \end{macro}
% \end{macro}
% \end{macro}
%
% \noindent
% We conclude with an |\endgroup| to restore the
% |\catcode| of |@|.
%    \begin{macrocode}
\endgroup
%</thp>
%    \end{macrocode}
%
%
% \subsubsection{The \texttt{break} style}
%
% This style option is stored in the file |thb.sty|.
% For the next two lines see the documentation for |\th@plain| on
% page~\pageref{dok:th@plain}.
%    \begin{macrocode}
%<*thb>
\begingroup \makeatletter
\@ifundefined{theorem@style}{% \iffalse meta-comment
%
% Copyright (C) 1993-2024
%
% The LaTeX Project and any individual authors listed elsewhere
% in this file.
%
% This file is part of the Standard LaTeX `Tools Bundle'.
% -------------------------------------------------------
%
% It may be distributed and/or modified under the
% conditions of the LaTeX Project Public License, either version 1.3c
% of this license or (at your option) any later version.
% The latest version of this license is in
%    https://www.latex-project.org/lppl.txt
% and version 1.3c or later is part of all distributions of LaTeX
% version 2005/12/01 or later.
%
% The list of all files belonging to the LaTeX `Tools Bundle' is
% given in the file `manifest.txt'.
%
% \fi
%
%%
%
%
\def\FMithmInfo{2023/07/05 v2.2c Theorem extension package (FMi)}
%
% \ProvidesFile{theorem.dtx}[\FMithmInfo]
%\iffalse   % this is a METACOMMENT !
%%
%% Package `theorem' to use with LaTeX2e
%% Copyright (C) 1989-1995 Frank Mittelbach, all rights reserved.
%<+package>\@ifundefined{theorem@style}{}{\endinput}
%<+package>\NeedsTeXFormat{LaTeX2e}[1995/06/01]
%<+package>\ProvidesPackage{theorem}
%<+package>         [\FMithmInfo]
%
% \fi
%
% \changes{v1.0}{87/05/07}{First draft.}
% \changes{v1.0a}{87/05/07}{`thistheoremstyle' added.}
% \changes{v1.1a}{87/07/29}{New concept: redefine the internal
%                          \protect\LaTeX{} commands `@xnthm' etc.}
% \changes{v1.1b}{87/07/31}{`theoremfont' used to specify font.}
% \changes{v1.2a}{88/01/08}{`@endtheorem' now globally defined.}
% \changes{v1.2b}{88/01/11}{`newtheorem' only allowed in the preamble.}
% \changes{v1.3a}{88/02/14}{`theoremfont' now a toks register.}
%
% \changes{v2.0a}{88/11/16}{New implementation and new documentation.}
% \changes{v2.0a}{88/11/16}{No layout changes but internal modifications
%                           old substyles won't work.}
% \changes{v2.0b}{89/01/30}{`global' used for all definitions.}
% \changes{v2.0c}{89/06/15}{Documentation translated to English.
%                            Thanks to LLL and Nina}
% \changes{v2.1a}{90/02/15}{Support for new font selection.}
% \changes{v2.2a}{94/02/03}{Cleanup for LaTeX2e}
%
% \DoNotIndex{\@endparenv,\@ifdefinable,\@ifnextchar,\@ifundefined}
% \DoNotIndex{\@namedef,\@nameuse,\@spaces,\@tempa,\@warning}
% \DoNotIndex{\ }
% \DoNotIndex{\begingroup,\bfseries}
% \DoNotIndex{\csname}
% \DoNotIndex{\def,\do,\docdate}
% \DoNotIndex{\endcsname,\endgroup,\endinput,\endtrivlist,\expandafter}
% \DoNotIndex{\filedate,\fileversion}
% \DoNotIndex{\gdef,\global}
% \DoNotIndex{\hbox,\hskip}
% \DoNotIndex{\ignorespaces,\input,\itshape,\item}
% \DoNotIndex{\labelsep,\let,\llap}
% \DoNotIndex{\makeatletter}
% \DoNotIndex{\noexpand}
% \DoNotIndex{\refstepcounter,\relax,\rlap}
% \DoNotIndex{\setlength,\slshape,\space,\string,\strut}
% \DoNotIndex{\the,\trivlist,\typeout}
% \DoNotIndex{\vbox}
% \DoNotIndex{\xdef}
%
% \def\deflabel#1{\ttfamily #1\hfill}
% \def\deflist#1{\begin{list}{}{\settowidth\labelwidth{\ttfamily #1}%
%                               \setlength\leftmargin\labelwidth
%                               \addtolength\leftmargin\labelsep
%                               \let\makelabel\deflabel}}
% \def\enddeflist{\end{list}}
% \def\envfont{\normalfont\ttfamily}
%
% \MakeShortVerb{\|}
%
% \newcommand{\TUB}{{\sl TUGboat\/}}
%
% \GetFileInfo{theorem.dtx}
% \title{An Extension of the
%        \LaTeX{} theorem environment\thanks{This file
%        has version number \fileversion, last
%        revised \filedate.}}
% \author{Frank Mittelbach}
%
% \MaintainedByLaTeXTeam{tools}
% \maketitle
%
% \begin{abstract}
%    The macros described in this paper yield an extension of the
%    \LaTeX{} theorem mechanism. It is designed to satisfy the
%    different requirements of various journals. Thus, the
%    layout of the ``theorems'' can be manipulated by determining a
%    ``style''. This article describes not only the use, but
%    also the definition, of the necessary macros.
% \end{abstract}
%
% \section*{Preface to version 2.2}
%
% For \LaTeXe{} this package did not need any fundamental changes. I
% only modified the messages generated so that theorem layout styles
% will show up with the |\listfiles| command and cleaned the section
% on the New Font Selection Scheme since this is now included in
% \LaTeX.
%
% \section*{Preface to version 2.1}
%
% This version is identical to 2.0g described in \TUB\ 10\#3 except
% for some internal defaults which are now set depending on the used
% font selection scheme.
%
% This was done to avoid unpleasant surprises if the new font selection
% scheme is in force. For further details see
% section~\ref{sec:newfonts} and \cite{bk:GMS94}.
%
% \section{Introduction}
%
% For our purposes here, ``theorems'' are labelled enunciations,
% often set off from the main text by extra space and a font change.
% Theorems, corollaries, conjectures, definitions, and remarks
% are all instances of ``theorems''.  The ``header'' of these structures
% is composed of a label (such as \textsc{Theorem} or \textsc{Remark})
% and a number which serializes an item in the sequence of items
% with the same label.
%
% Shortly after the introduction of \LaTeX{} at the Fachbereich
% Mathematik in Mainz, the desire to manipulate the layout of
% ``theorems'' arose.  In Mainz, the following two conventions came
% into general use:
% \begin{enumerate}
%   \item The number of the theorem is shown in the margin.
%   \item There is a line break at
%         the end of the theorem header.
% \end{enumerate}
%
% Additionally, some journals require different formats which depend on
% the ``sort of theorem'': e.g.\ often remarks and definitions are set
% in |\upshape|, while |\itshape| is employed for main theorems.
%
% Confronted with these requirements, a theorem environment was
% developed in Mainz which allows separate determination of the
% layout of the ``theorems sets'', comparable to |\pagestyle|.
%
%
% \section{The user interface}
% \subsection{Defining new theorem sets}
%
% \DescribeMacro\newtheorem
% As in the original \LaTeX{} version, the command |\newtheorem|
% defines a new ``theorem set'' or ``theorem-like structure''.
% Two required arguments name the new environment and give the text
% to be typeset with each instance of the new ``set'', while
% an optional argument determines how
% the ``set'' is enumerated:
% \begin{description}
%    \item[\ttfamily \bslash newtheorem\{foo\}\{bar\}]
%       The theorem set {\envfont foo} (whose name is \texttt{bar})
%       uses its own counter.
%    \item[\ttfamily \bslash newtheorem\{foo2\}{[foo]}\{bar2\}]
%       The theorem set {\envfont foo2} (printed name \texttt{bar2})
%       uses the same counter as the theorem set \texttt{foo}.
%    \item[\ttfamily \bslash newtheorem\{foo3\}\{bar3\}{[section]}]
%       The theorem set {\envfont foo3} (printed name \texttt{bar3}) is
%       enumerated within the counter \texttt{section}, i.e.\ with every
%       new |\section| the enumeration begins again with 1, and
%       the enumeration is composed from the section-number and the
%       theorem counter itself.
% \end{description}
%
% \DescribeMacro\theoremstyle
% Additionally, the command |\theoremstyle| can define the layout
% of various, or all, theorem sets.  It should be noted that any
% theorem set defined by |\newtheorem| is typeset in the
% |\theoremstyle| that is current at the time of the definition.
% Thus, the following
% \begin{verbatim}
%    \theoremstyle{break}        \newtheorem{Cor}{Corollary}
%    \theoremstyle{plain}        \newtheorem{Exa}{Example}[section]
%\end{verbatim}                             ^^A No blank at line start
% leads to the result that the set {\envfont Cor} is formatted in the
% style \texttt{break}, while the set {\envfont Exa} and all the
% following ones are formatted in the style \texttt{plain}, unless
% another |\theoremstyle| follows.  Since the definitions installed by
% |\newtheorem| are global, one also can limit |\theoremstyle| locally
% by grouping braces.
%
% \DescribeMacro\theorembodyfont
% The choice of the font for the theorem body is completely
% independent of the chosen |\theoremstyle|; this has proven to
% be very advantageous.  For example,
% \begin{verbatim}
%   {\theorembodyfont{\upshape}          \newtheorem{Rem}{Remark}}
%\end{verbatim}
% defines a theorem set {\envfont Rem}, which will be set in
% |\upshape| in the current layout (which in our example is
% \texttt{plain}).  As with |\theoremstyle|, the |\theorembodyfont|
% chosen is that current at the time of |\newtheorem|.  If
% |\theorembodyfont| is not specified or one defines
% |\theorembodyfont{}|, then the font used will be that defined by the
% |\theoremstyle|.
%
% \DescribeMacro\theoremheaderfont It is also possible to customize
% the font used for the theorem headers.  This is, however, a global
% declaration, and therefore there should be at most one
% |\theoremheaderfont| declaration in the preamble.\footnote{If it is
% actually necessary to have different header fonts, one has to define
% new theorem styles (substituting the desired font) or specify the
% information directly in the \texttt{\bslash newtheorem} declaration
% (the unclean variant).}
%
% \DescribeMacro\theorempreskipamount
% \DescribeMacro\theorempostskipamount Two additional parameters
% affect the vertical space around the theorem environments:
% |\theorempreskipamount| and |\theorempostskipamount| define,
% respectively, the spacing before and after such an environment.
% These parameters apply for all theorem sets and can be manipulated
% with the ordinary length macros.  They are rubber lengths
% (`\textsf{skips}'), and therefore can contain \texttt{plus} and
% \texttt{minus} parts.
%
% Since the definition of theorem sets should---most sensibly---be
% placed in the preamble, we only allow installation there.
% It is therefore possible to release the memory used here after
% |\begin{document}|, in order to make room for other
% applications.
%
%
% \subsection{Existing theorem styles}
%
% The following theorem styles exist to date:
% \begin{deflist}{marginbreak}
%    \item[plain]
%       This theorem style emulates the original \LaTeX{} definition,
%       except that additionally the parameters
%       |\theorem...skipamount| are used.
%    \item[break]
%       In this style, the theorem header is followed by a line
%       break.
%    \item[marginbreak]
%       The theorem number is set in the margin, and there is a line
%       break as in \texttt{break}.
%    \item[changebreak]
%       Like \texttt{break}, but with header number and text
%       interchanged.
%    \item[change]
%       Header number and text are interchanged, without a line break.
%    \item[margin]
%       The number is set in the left margin, without a line break.
% \end{deflist}
% All styles (except \texttt{plain}) select |\slshape| as the default
% |\theorembodyfont|.
%
%
% \subsection{Examples}
%
% Given the above theorem sets {\envfont Cor}, {\envfont Exa}
% and {\envfont Rem},
% suppose that the preamble also contains the declarations:
% \begin{verbatim}
%    \theoremstyle{marginbreak}   \newtheorem{Lem}[Cor]{Lemma}
%    \theoremstyle{change}
%    \theorembodyfont{\itshape}        \newtheorem{Def}[Cor]{Definition}
%
%    \theoremheaderfont{\scshape}
%\end{verbatim}
% Then the following are some typical examples of the typeset output
% resulting from their use.
%
% \begin{Cor}
%    This is a sentence typeset in the theorem environment {\envfont
%    Cor}.
% \end{Cor}
%
% \begin{Exa}
%    This is a sentence typeset in the theorem environment {\envfont
%    Exa}.
% \end{Exa}
%
% \begin{Rem}
%    This is a sentence typeset in the theorem environment {\envfont
%    Rem}.
% \end{Rem}
%
% \begin{Lem}[Ben User]
%    This is a sentence typeset in the theorem environment {\envfont
%    Lem}.
% \end{Lem}
%
% \begin{Def}[Very impressive Definition]
%    This is a sentence typeset in the theorem environment {\envfont
%    Def}.
% \end{Def}
%
% The last two examples show the effect of the optional argument to
% a theorem environment (it is the text typeset in parentheses).
%
%
% \section{Special Considerations}
% \label{sec:newfonts}
%
% Theoremheader and body are implemented as a unit. This means that
% the |\theoremheaderfont|  will inherit characteristics of the
% |\theorembodyfont| in \LaTeXe. Thus, if for example
% |\theorembodyfont| is |\itshape| and |\theoremheaderfont|
% is |\bfseries| the font selected for the header will have the
% characteristics `bold extended italic'. If this is not desired one
% should set the |\theoremheaderfont| to something like
% \begin{verbatim}
%      \theoremheaderfont{\normalfont\bfseries}
%\end{verbatim}
% i.e.\ supplying all necessary font information explicitly.
%
%
%
% \section{Acknowledgements}
%
% The publication of this set of macros was only possible with the
% help of Christina Busse (translating the manuscript into English),
% Joachim Pense (playing the r\^ole of typist), Chris Rowley
% (looking everything over) and many others providing useful
% suggestions.
%
% \MaybeStop{%
%    \begin{thebibliography}{9}
%       \bibitem{bk:GMS94} \textsc{M.~Goossens}, \textsc{F.~Mittelbach}
%         and \textsc{A.~Samarin}.
%         \newblock The \LaTeX{} Companion.
%         \newblock
%         Addison-Wesley, Reading, Massachusetts, 1994.
%       \bibitem{latex:source} \textsc{Lamport, Leslie.} \newblock
%                               \texttt{latex.tex}, version 2.09, date
%                               Feb. 1990.
%    \end{thebibliography}
%    \PrintIndex
%    \PrintChanges}
%
%
% \section{The documentation driver file}
%
% The next bit of code contains the documentation driver file for
% \TeX{}, i.e., the file that will produce the documentation you are
% currently reading. It will be extracted from this file by the
% \texttt{docstrip} program. Since it is the first code in the file
% one can alternatively process this file directly with \LaTeXe{} to
% obtain the documentation.
%
%    \begin{macrocode}
%<*driver>
\documentclass{ltxdoc}

\usepackage{theorem}

%<+driver>% The next few lines define theorem sets which are used
%<+driver>% in the example section of the documentation.

\theoremstyle{break}       \newtheorem{Cor}{Corollary}
\theoremstyle{plain}       \newtheorem{Exa}{Example}[section]
{\theorembodyfont{\upshape}\newtheorem{Rem}{Remark}}
\theoremstyle{marginbreak} \newtheorem{Lem}[Cor]{Lemma}
\theoremstyle{change}
\theorembodyfont{\itshape} \newtheorem{Def}[Cor]{Definition}

\theoremheaderfont{\scshape}

\RecordChanges

\begin{document}
   \DocInput{theorem.dtx}
\end{document}
%</driver>
%    \end{macrocode}
%
% \section{Definition of the Macros}
%
% If the file has been loaded before, we abort immediately. If not the
% package announces itself (this is actually done at the very top of
% the file---the way it is done isn't good style so don't copy it).
% \changes{v2.0e}{89/07/19}{Spaces removed from `typeout'.}
% \changes{v2.2b}{95/11/19}{Announce moved to top of file}
%    \begin{macrocode}
%<*package>
%\@ifundefined{theorem@style}{}{\endinput}
%\def\FMithmInfo{1995/11/19 v2.2b Theorem extension package (FMi)}
%\ProvidesPackage{theorem}[\FMithmInfo]
%    \end{macrocode}
%
%
% \subsection{Definition of theorem styles and fonts}
%
% All the definitions in this file are done globally to allow
% inputting this file inside a group.
%
% \begin{macro}{\theoremstyle}
%    Before a theorem style can be installed, the chosen style must be
%    known. For that reason, we must test to see that
%    |\th@|\meta{style} is known or, more precisely, that it is
%    different from |\relax|. If the style is not known then
%    |\th@plain| is used.
%    \begin{macrocode}
\gdef\theoremstyle#1{%
   \@ifundefined{th@#1}{\@warning
          {Unknown theoremstyle `#1'. Using `plain'}%
          \theorem@style{plain}}%
%    \end{macrocode}
%    We save the theorem style to be used in the token register
%    |\theorem@style|.
%    \begin{macrocode}
      {\theorem@style{#1}}%
%    \end{macrocode}
%    Now we ``evaluate'' the theorem style: this means, we call the
%    macro |\th@|\meta{style} which will activate the relevant
%    definitions which are contained in a separate file.  This is done
%    in a group to suppress changes to the current font.  This could
%    otherwise pose problems together with the new font selection
%    scheme\footnote{When I printed the original article using the new
%    font selection scheme I ended with a document with slanted
%    typefaces (text headings and all) simply because one of the
%    theorem styles used \texttt{\bslash sl} at toplevel.} if the
%    |\th@|\meta{style} is evaluated a second time.
% \changes{v2.1a}{90/02/15}{Group added around `th@..}
% \changes{v2.1a}{90/02/15}{Reseting of `..@begintheorem removed.}
%    \begin{macrocode}
      \begingroup
        \csname th@\the\theorem@style \endcsname
      \endgroup}
%    \end{macrocode}
% \end{macro}
%
%
% \begin{macro}{\@begintheorem}
% \changes{v2.1a}{90/02/15}{Relaxed at toplevel.}
% \begin{macro}{\@opargbegintheorem}
% \changes{v2.1a}{90/02/15}{Relaxed at toplevel.}
%    We reset |\@begintheorem| and |\@opargbegintheorem| to
%    |\relax| since these commands are no longer necessary at
%    toplevel. This will save a few tokens.
%    \begin{macrocode}
\global\let\@begintheorem\relax
\global\let\@opargbegintheorem\relax
%    \end{macrocode}
% \end{macro}
% \end{macro}
%
% \begin{macro}{\theorem@style}
%    Obviously the \textsf{token} register used above has to be
%    allocated. To assure the utmost compatibility with the
%    original \LaTeX{} definition, we set the default theorem style to
%    \texttt{plain}, which implements the usual \LaTeX{} convention.
% \changes{v2.0f}{89/07/20}{Unnecessary `global' removed.}
%    \begin{macrocode}
\newtoks\theorem@style
\global\theorem@style{plain}
%    \end{macrocode}
% \end{macro}
%
% \begin{macro}{\theorembodyfont}
% \begin{macro}{\theorem@bodyfont}
%    If the |\theorembodyfont| is set by the user then it should not
%    interact with the default font set in the theorem style. When the
%    new font selection is in force this may happen if, for example,
%    the default is |\itshape| and the new |\theorembodyfont| is
%    |\sffamily|. So we add a |\reset@font| command in front of the
%    user definition.
% \changes{v2.0f}{89/07/20}{Unnecessary `global' removed.}
% \changes{v2.0g}{89/09/19}{`theoremfont to `theorembodyfont.}
% \changes{v2.1c}{92/06/22}{`reset@font added and logic changed.}
% \changes{v2.1d}{92/10/28}{Reset to empty if arg is empty}
%    \begin{macrocode}
\gdef\theorembodyfont#1{%
%    \end{macrocode}
%    We check if the argument supplied is empty and if so put nothing
%    into the |\theorem@bodyfont| token register to allow for
%    |\theorembodyfont{}| as a mean of using the default of the
%    current |\theoremstyle|.
%    \begin{macrocode}
   \def\@tempa{#1}%
   \ifx\@tempa\@empty
    \theorem@bodyfont{}%
   \else
    \theorem@bodyfont{\reset@font#1}%
   \fi
   }
\newtoks\theorem@bodyfont
\global\theorem@bodyfont{}
%    \end{macrocode}
% \end{macro}
% \end{macro}
%
%
% \begin{macro}{\theoremheaderfont}
% \changes{v2.0g}{89/09/19}{Macro added.}
%    The font for the theorem headers is handled differently because
%    this definition applies to all theorem styles.
%    \begin{macrocode}
\gdef\theoremheaderfont#1{\gdef\theorem@headerfont{#1}%
%    \end{macrocode}
%    After using the macro once it is redefined to produce an error
%    message.
%    \begin{macrocode}
       \gdef\theoremheaderfont##1{%
        \typeout{\string\theoremheaderfont\space should be used
                 only once.}}}
%    \end{macrocode}
% \end{macro}
%
% \begin{macro}{\theorem@headerfont}
% \changes{v2.0g}{89/09/19}{Macro added.}
%    To set the |\theorem@headerfont| default we first test if
%    the new fontselection scheme is in force.
% \changes{v2.1a}{90/02/15}{Support for new font selection.}
%    \begin{macrocode}
\ifx\upshape\undefined
%    \end{macrocode}
%    If not we define it to expand into |\bfseries|. We don't use
%    |\let| just in case a following style option redefines this
%    macro.
%    \begin{macrocode}
\gdef\theorem@headerfont{\bfseries}
%    \end{macrocode}
%    Otherwise we reset the current shape before calling |\bfseries|.
%    \begin{macrocode}
\else \gdef\theorem@headerfont{\normalfont\bfseries}\fi
%    \end{macrocode}
% \end{macro}
%
% \begin{macro}{\th@plain}
% \begin{macro}{\th@break}
% \begin{macro}{\th@marginbreak}
% \begin{macro}{\th@changebreak}
% \begin{macro}{\th@change}
% \begin{macro}{\th@margin}
%    The different styles are defined in macros such as
%    |\th@plain|.  Since memory space is precious in
%    ``non-Big-versions'', we have to avoid offering too many unused
%    definitions.  Therefore we define these styles in separate files
%    that can be loaded on demand.  Thus the commands themselves only
%    load these files. We use |\@input@|, a \LaTeXe{} internal command
%    that ensures that the file will be listed with |\listfiles|.
% \changes{v2.2a}{94/02/02}{Use `@input@ to load theorem layout files}
%    \begin{macrocode}
\gdef\th@plain{\@input@{thp.sty}}
\gdef\th@break{\@input@{thb.sty}}
\gdef\th@marginbreak{\@input@{thmb.sty}}
\gdef\th@changebreak{\@input@{thcb.sty}}
\gdef\th@change{\@input@{thc.sty}}
\gdef\th@margin{\@input@{thm.sty}}
%    \end{macrocode}
%    This list will be expanded when new styles become
%    available.  For testing,  just append new theorem substyles
%    as document options.
% \end{macro}
% \end{macro}
% \end{macro}
% \end{macro}
% \end{macro}
% \end{macro}
%
%
%
% \subsection{Definition of a new theorem set}
%
% As already pointed out, a new theorem environment can be defined in
% three different ways:
% \begin{verbatim}
%    \newtheorem{Lem}{Lemma}
%    \newtheorem{Lem}{Lemma}[section]
%    \newtheorem{Lem}[Theorem]{Lemma}
%\end{verbatim}
% The function of the macro |\newtheorem| is to recognize these cases
% and then to branch into one of the three macros |\@ynthm|, |\@xnthm|
% or |\@othm|.  This mechanism is adopted unchanged from
% \cite{latex:source}; the essential point here is that, for example,
% in the second case, the arguments \texttt{Lem}, \texttt{Lemma} and
% \texttt{section} are passed over to the macro |\@xnthm|.
%
% We inspect this case first because the others present fewer problems,
% and thus are easily derived from this one.
%
% \begin{macro}{\@xnthm}
%    For our example arguments, the macro |\@xnthm| must fulfill
%    the following:
%    \begin{itemize}
%       \item
%          Define a new \LaTeX-counter `Lem'
%       \item
%          reset this counter within a |\section|
%       \item
%          define the macro |\theLem|
%       \item
%          define the environment macros |\Lem| and
%          |\endLem| using the current
%          |\theoremstyle| and |\theorem@bodyfont|.
%    \end{itemize}
%    Obviously, all this should happen only if the first argument
%    of |\@xnthm| (i.e.\ \texttt{Lem} in our example) is chosen so
%    as not to conflict with any previously defined commands or
%    environments.  This test is performed by the \LaTeX{} macro
%    |\@ifdefinable|.
%    \begin{macrocode}
\gdef\@xnthm#1#2[#3]{\expandafter\@ifdefinable\csname #1\endcsname
%    \end{macrocode}
%    Therefore, the first argument of |\@ifdefinable| is the
%    expansion (in the example, |\Lem|) of
%    |\csname||#1\endcsname|.  The second argument is
%    executed only if the test has been completed successfully.
%    \begin{macrocode}
   {%
%    \end{macrocode}
%    Now we define the new counter. The names of the \LaTeX{}
%    macros employed should speak for themselves:
% \changes{v2.2b}{1995/10/07}{Use \cs{@newctr}}
% \changes{v2.2c}{1995/11/23}{Use \cs{@newctr} properly.}
%    \begin{macrocode}
    \@definecounter{#1}\@newctr{#1}[#3]%
%    \end{macrocode}
%    Using |\@newctr| will give a proper error message if the counter
%    in |#3| is not defined.
%    In defining `|\theLem|' we must generate the desired macro
%    name by use of |\expandafter| and |\csname|.
%    \begin{macrocode}
    \expandafter\xdef\csname the#1\endcsname
%    \end{macrocode}
%    An |\xdef| is used in order to make the definition global,
%    and to ensure that it contains the replacement texts of
%    |\@thmcountersep| and |\@thmcounter|.\footnote{These
%    two macros can be defined by the document style.  Their default
%    values produce a `\texttt{.}' as separation and an arabic
%    representation of the number.} However, not everything should be
%    expanded. For example, it saves space to use |\thesection|
%    instead of its---at times---lengthy expansion.
%    \begin{macrocode}
      {\expandafter \noexpand \csname the#3\endcsname
       \@thmcountersep \@thmcounter{#1}}%
%    \end{macrocode}
%    Thus with the defaults of \LaTeX{}, |\theLem| would be
%    replaced by the command sequence |\thesection.\arabic{Lem}|.
%
%    We will now look at the definition of the macro which is executed
%    at the beginning of the actual environment (in our example this
%    macro is |\Lem|). It should be noted that we use an
%    ``|\expandafter| trick'' to expand only certain parts of the
%    replacement text at the time of the definition.
%    \begin{macrocode}
    \def\@tempa{\global\@namedef{#1}}%
    \expandafter \@tempa \expandafter{%
%    \end{macrocode}
%    First, the macro that contains the current definitions of
%    |\@begintheorem| and |\@opargtheorem| should be called
%    up.  The name of this macro---as is already known---has the form
%    |\th@|\meta{theorem\ style}; therefore, it must be called by
%    \begin{macrocode}
      \csname th@\the \theorem@style
%    \end{macrocode}
%    In addition the default theorem font should be changeable, i.e.\
%    we have to insert the contents of |\theorem@bodyfont|. For
%    that reason, we expand even further, beyond |\endcsname|,
%    and thus insert the contents of the \textsf{token} register
%    |\theorem@bodyfont| in the replacement text.
%    \begin{macrocode}
            \expandafter \endcsname \the \theorem@bodyfont
%    \end{macrocode}
%    Now it is time to call the macro |\@thm| which takes over
%    the further processing.  It has two arguments: the current
%    counter name (in our example, \texttt{Lem}), and the text of the
%    label (in our example, \texttt{Lemma}).
%    \begin{macrocode}
     \@thm{#1}{#2}}%
%    \end{macrocode}
%    With this, the `sub-definition' is complete.  The macro
%    |\@endtheorem| ends a theorem environment and is, so far, nothing
%    but an |\endtrivlist|.  (Hence it is defined globally, and not
%    within the theorem styles.\footnote{This has to be changed as
%    soon as theorem styles that change \texttt{\bslash @endtheorem}
%    exist.  In such a case, all existing styles must be changed as
%    well since they will have to reset the macro.}) Therefore, we can
%    set it equivalent to the macro that ends the theorem set (in our
%    example, |\endLem|).  However, if some day theorem styles exist
%    that do change |\@endtheorem|, we would have to use the
%    commented-out line instead.
%    \begin{macrocode}
    \global \expandafter \let \csname end#1\endcsname \@endtheorem
%   \global\@namedef{end#1}{\@endtheorem}%
%    \end{macrocode}
%    With these commands all the required definitions are employed,
%    unless the test |\@ifdefinable| has failed.  Therefore, we
%    end the second argument of this macro and with it the definition
%    of |\@xnthm|.
%    \begin{macrocode}
   }}
%    \end{macrocode}
% \end{macro}
%
% \begin{macro}{\@ynthm}
%    The definition of |\@ynthm| is completely analogous. In
%    this case the new counter that is defined is not reset within
%    another counter; thus the definition of |\the...|
%    is simplified:
%    \begin{macrocode}
\gdef\@ynthm#1#2{\expandafter\@ifdefinable\csname #1\endcsname
   {\@definecounter{#1}%
    \expandafter\xdef\csname the#1\endcsname{\@thmcounter{#1}}%
%    \end{macrocode}
%    The rest of the definition corresponds literally to that of
%    |\@xnthm|:
%    \begin{macrocode}
    \def\@tempa{\global\@namedef{#1}}\expandafter \@tempa
     \expandafter{\csname th@\the \theorem@style \expandafter
     \endcsname \the\theorem@bodyfont \@thm{#1}{#2}}%
    \global \expandafter \let \csname end#1\endcsname \@endtheorem}}
%    \end{macrocode}
% \end{macro}
%
% \begin{macro}{\@othm}
%    The definition of |\@othm| does not contain anything new.
%    \begin{macrocode}
\gdef\@othm#1[#2]#3{%
%    \end{macrocode}
% \changes{v2.2b}{95/10/07}{Build in check for counter defined}
%    We do not define a new counter but instead use one that has
%    already been defined.  Thus the only definition we need is that
%    of this pseudo-counter (i.e.\ |\the|\meta{env.\ name}).
%    First we check if |#2| corresponds to a known counter name.
%    \begin{macrocode}
  \expandafter\ifx\csname c@#2\endcsname\relax
   \@nocounterr{#2}%
  \else
   \expandafter\@ifdefinable\csname #1\endcsname
   {\expandafter \xdef \csname the#1\endcsname
     {\expandafter \noexpand \csname the#2\endcsname}%
%    \end{macrocode}
%    All other parts of the definition can be adopted from
%    |\@xnthm|. We have to remember, though, that in this case
%    the name of the current counter and the theorem label have moved
%    to the second and third arguments.
%    \begin{macrocode}
    \def\@tempa{\global\@namedef{#1}}\expandafter \@tempa
     \expandafter{\csname th@\the \theorem@style \expandafter
     \endcsname \the\theorem@bodyfont \@thm{#2}{#3}}%
    \global \expandafter \let \csname end#1\endcsname \@endtheorem}%
  \fi}
%    \end{macrocode}
% \end{macro}
%
% \subsection{Macros that are employed in a {\envfont theorem}
%    environment}
%
% \begin{macro}{\@thm}
%    The macro |\@thm| has to increase the current counter.  Then,
%    depending on whether the environment has (or does not have) an
%    optional argument, it has to branch into either |\@begintheorem|
%    or |\@opargtheorem|.
%    \begin{macrocode}
\gdef\@thm#1#2{\refstepcounter{#1}%
%    \end{macrocode}
%    Now we start a {\envfont trivlist} environment, and give
%    |\@topsep| and |\@topsepadd| the values of the \textsf{skip}
%    registers |\theorempreskipamount| and |\theorempostskipamount|.
%    The value in |\@topsep| is the vertical space that is inserted by
%    the first (and only) |\item| in our |\trivlist| whilst
%    |\@topsepadd| is inserted by |\@endparenv| at the end of that
%    {\envfont trivlist} environment.  By using these registers, we
%    obtain the desired space around a {\envfont theorem} environment.
%    \begin{macrocode}
   \trivlist
   \@topsep \theorempreskipamount               % used by first \item
   \@topsepadd \theorempostskipamount           % used by \@endparenv
%    \end{macrocode}
%    Now we have to test whether an optional argument has been given.
% \changes{v2.0d}{89/07/18}{Forgotten percent added.}
%    \begin{macrocode}
   \@ifnextchar [%
%    \end{macrocode}
%    If there is an optional argument, we will call |\@ythm|, and
%    move the arguments read back into the input stream.
%    \begin{macrocode}
   {\@ythm{#1}{#2}}%
%    \end{macrocode}
%    If not, we call |\@begintheorem|. Its first argument is the
%    name of the theorem set (hence the second argument of
%    |\@thm|). Its second argument is the macro that produces the
%    current number.
%    \begin{macrocode}
   {\@begintheorem{#2}{\csname the#1\endcsname}\ignorespaces}}
%    \end{macrocode}
% \end{macro}
%
% \begin{macro}{\@xthm}
% \begin{macro}{\@ythm}
%    Both these macros were originally called by |\@thm|.  We do
%    not need |\@xthm| anymore, hence we reset it to
%    |\relax|. The definition of |\@ythm| has not changed at
%    all from its definition in \LaTeX. In order to make the macros
%    easier to understand, we will
%    nevertheless present it (commented out).
%    \begin{macrocode}
\global\let\@xthm\relax
% \def\@ythm#1#2[#3]{\@opargbegintheorem{#2}{\csname
%           the#1\endcsname}{#3}\ignorespaces}
%</package>
%    \end{macrocode}
%    The primitive |\ignorespaces| in |\@ythm| and
%    |\@thm| is needed to remove the spaces between the
%    |\begin{...}| and the actual text.
% \end{macro}
% \end{macro}
%
%
% \subsection{Definition of the theorem substyles}
%
% As already pointed out, the theorem substyles, defined below, are
% only loaded when necessary.  Note that all these substyles, except
% \texttt{plain}, have |\slshape| as the default body font.
%
% \subsubsection{The \texttt{plain} style}
%
% \label{dok:th@plain} As the following macros use |@|, we have
% to locally set the |\catcode| of this symbol to ``letter''.
% This happens within a group, so that we do not have to worry about
% which |\catcode| that symbol had before.
%    \begin{macrocode}
%<*thp>
\begingroup \makeatletter
%    \end{macrocode}
% Since we are now within a group, we must make all definitions
% globally.  First we make sure that \texttt{theorem.sty} is loaded.
% This will allow us to use this file as a document style option
% without having to call \texttt{theorem} itself as an option.  At the
% same time, we assure that at least version 2 is loaded, since
% |\theorem@style| was not defined in earlier versions.
%    \begin{macrocode}
\@ifundefined{theorem@style}{\input{theorem.sty}}{}
\ProvidesFile{thp.sty}
             [\FMithmInfo]
%    \end{macrocode}
%
%
% \begin{macro}{\th@plain}
% \begin{macro}{\@begintheorem}
% \begin{macro}{\@opargbegintheorem}
%    |\theoremstyle{plain}| corresponds to the original
%    definition, except that the distances to the surrounding text are
%    determined by the parameters |\theorempreskipamount| and
%    |\theorempostskipamount|.
%    First we set the default body font.
%    \begin{macrocode}
\gdef\th@plain{\normalfont\itshape
%    \end{macrocode}
%    Then we define |\@begintheorem| and
%    |\@opargbegintheorem|. These two macros define how the
%    header of a theorem is typeset.  |\@opargbegintheorem| will
%    be called if a \textsf{theorem} environment with an optional
%    argument is encountered; otherwise, the header is constructed by
%    calling |\@begintheorem|.  If one of these macros is
%    executed, we are within a \textsf{trivlist} environment started by
%    |\@thm|.  So the theorem header is produced with an
%    |\item| command.
%
%    Instead of specifying the header font directly, all standard
%    theorem styles use the |\theorem@headerfont| macro to allow
%    customization.  The extra space (|\labelsep|) is necessary
%    because of problems in the \textsf{trivlist} environment.
%    \begin{macrocode}
  \def\@begintheorem##1##2{%
        \item[\hskip\labelsep \theorem@headerfont ##1\ ##2]}%
%    \end{macrocode}
%    The definition of |\@opargbegintheorem| is completely
%    analogous. The only difference is the fact that there exists a
%    third argument (which is the optional parameter of the
%    environment and contains additional information about the
%    theorem).  Customarily we enclose it in parentheses.
%    \begin{macrocode}
\def\@opargbegintheorem##1##2##3{%
   \item[\hskip\labelsep \theorem@headerfont ##1\ ##2\ (##3)]}}
%    \end{macrocode}
% \end{macro}
% \end{macro}
% \end{macro}
%
% \noindent
% We conclude with an |\endgroup| to restore the
% |\catcode| of |@|.
%    \begin{macrocode}
\endgroup
%</thp>
%    \end{macrocode}
%
%
% \subsubsection{The \texttt{break} style}
%
% This style option is stored in the file |thb.sty|.
% For the next two lines see the documentation for |\th@plain| on
% page~\pageref{dok:th@plain}.
%    \begin{macrocode}
%<*thb>
\begingroup \makeatletter
\@ifundefined{theorem@style}{\input{theorem.sty}}{}
\ProvidesFile{thb.sty}
             [\FMithmInfo]
%    \end{macrocode}
%
% \begin{macro}{\th@break}
%    |\theoremstyle{break}| produces a line break after the name
%    of the theorem.  The font is |\slshape|.  Hence, we define
%    |\th@break| as follows:
%    \begin{macrocode}
\gdef\th@break{\normalfont\slshape
  \def\@begintheorem##1##2{\item[%
%    \end{macrocode}
%    We run into the following problem: it is not possible to create
%    the header with |\item[|\meta{title}|]| and then start
%    a new line by, for example, |\mbox{}\\|. Such a definition
%    will fail whenever a list environment follows immediately.  With
%    the above construction, the |\mbox{}| causes the switch
%    |@inlabel| (cf.\ definition of |\list| and
%    |\trivlist| in \cite{latex:source}) to be set to \texttt{false}
%    and so the following list will insert additional vertical
%    space (|\topskip|). This is quite annoying. Therefore, we
%    create the line break within the |\item|. In order to ensure
%    that the text will begin at the proper position in the following
%    line, we simply pretend that the label does not take any
%    room.\footnote{This will lead to problems whenever very high
%    symbols occurring in the line tower into the
%    heading. So, something else has to be done here sometime.}
%    \begin{macrocode}
       \rlap{\vbox{\hbox{\hskip \labelsep\theorem@headerfont ##1\ ##2}%
                   \hbox{\strut}}}]}%
%    \end{macrocode}
%    Again, the definition of |\@opargbegintheorem| is
%    completely analogous.
%    \begin{macrocode}
\def\@opargbegintheorem##1##2##3{%
  \item[\rlap{\vbox{\hbox{\hskip \labelsep \theorem@headerfont
                     ##1\ ##2\ (##3)}%
                    \hbox{\strut}}}]}}
\endgroup
%</thb>
%    \end{macrocode}
% \end{macro}
%
%
% \subsubsection{The \texttt{changebreak} style}
%
%    \begin{macrocode}
% This style option is stored in the file |thcb.sty|.
%    \begin{macrocode}
%<*thcb>
\begingroup \makeatletter
\@ifundefined{theorem@style}{\input{theorem.sty}}{}
\ProvidesFile{thcb.sty}
             [\FMithmInfo]
%    \end{macrocode}
%
% \begin{macro}{\th@changebreak}
%    The \texttt{change-break} theorem style is like \texttt{break}
%    but with interchange of theorem name and theorem number.  Thus we
%    define |\th@changebreak| as follows:
%    \begin{macrocode}
\gdef\th@changebreak{\normalfont\slshape
  \def\@begintheorem##1##2{\item
       [\rlap{\vbox{\hbox{\hskip\labelsep \theorem@headerfont ##2\ ##1}%
                    \hbox{\strut}}}]}%
\def\@opargbegintheorem##1##2##3{%
  \item[\rlap{\vbox{\hbox{\hskip\labelsep \theorem@headerfont
                     ##2\ ##1\ (##3)}%
                    \hbox{\strut}}}]}}
\endgroup
%</thcb>
%    \end{macrocode}
% \end{macro}
%
%
% \subsubsection{The \texttt{change} style}
%
% This style option is stored in the file |thc.sty|.
%    \begin{macrocode}
%<*thc>
\begingroup \makeatletter
\@ifundefined{theorem@style}{\input{theorem.sty}}{}
\ProvidesFile{thc.sty}
             [\FMithmInfo]
%    \end{macrocode}
%
% \begin{macro}{\th@change}
%    The \texttt{change} theorem style corresponds to the
%    \texttt{change break} style without a linebreak after the header.
%    To say it in another way, it's the same as the \texttt{plain}
%    style but with number and name interchanged and |\slshape| as the
%    default font.
%    \begin{macrocode}
\gdef\th@change{\normalfont\slshape
  \def\@begintheorem##1##2{\item
       [\hskip\labelsep \theorem@headerfont ##2\ ##1]}%
\def\@opargbegintheorem##1##2##3{%
  \item[\hskip\labelsep \theorem@headerfont ##2\ ##1\ (##3)]}}
\endgroup
%</thc>
%    \end{macrocode}
% \end{macro}
%
%
% \subsubsection{The \texttt{marginbreak} style}
%
% This style option is the one used most often at Mainz.  It is saved
% in the file |thmb.sty|.
%    \begin{macrocode}
%<*thmb>
\begingroup \makeatletter
\@ifundefined{theorem@style}{\input{theorem.sty}}{}
\ProvidesFile{thmb.sty}
             [\FMithmInfo]
%    \end{macrocode}
%
% \begin{macro}{\th@marginbreak}
%    The \texttt{margin break} style is nearly the same as the
%    \texttt{change break} style. The only difference is the placement
%    of the theorem number.  We use |\llap| to place it in the left
%    margin.
%
%    In this style |\labelsep| denotes the separation between the
%    number and the text.
%    \begin{macrocode}
\gdef\th@marginbreak{\normalfont\slshape
  \def\@begintheorem##1##2{\item
       [\rlap{\vbox{\theorem@headerfont
                    \hbox{\llap{##2}\hskip\labelsep ##1}%
                    \hbox{\strut}}}]}%
\def\@opargbegintheorem##1##2##3{%
  \item[\rlap{\vbox{\theorem@headerfont
                    \hbox{\llap{##2}\hskip\labelsep ##1\ (##3)}%
                    \hbox{\strut}}}]}}
\endgroup
%</thmb>
%    \end{macrocode}
% \changes{v2.1c}{90/02/17}{`endgroup outside macrocode env.}
% \end{macro}
%
%
%
% \subsubsection{The \texttt{margin} style}
%
% This style option is stored in the file |thm.sty|.
%    \begin{macrocode}
%<*thm>
\begingroup \makeatletter
\@ifundefined{theorem@style}{\input{theorem.sty}}{}
\ProvidesFile{thm.sty}
             [\FMithmInfo]
%    \end{macrocode}
%
% \begin{macro}{\th@margin}
%    Again this is only a variant of the theorem styles described
%    above without any new ideas.
%    \begin{macrocode}
\gdef\th@margin{\normalfont\slshape
  \def\@begintheorem##1##2{\item
       [\theorem@headerfont \llap{##2}\hskip\labelsep ##1]}%
\def\@opargbegintheorem##1##2##3{%
  \item[\theorem@headerfont \llap{##2}\hskip\labelsep ##1\ (##3)]}}
\endgroup
%</thm>
%    \end{macrocode}
% \end{macro}
%
%
%
%
% \subsection{Final Definitions}
%
% \begin{macro}{\theorempreskipamount}
% \begin{macro}{\theorempostskipamount}
%    The \textsf{skip} parameters that regulate the vertical empty space
%    before and after the {\envfont theorem} environment have
%    to be allocated as well.
% \changes{v2.0f}{89/07/20}{Unnecessary `global' removed.}
%    \begin{macrocode}
%<*package>
\newskip\theorempreskipamount
\newskip\theorempostskipamount
%    \end{macrocode}
%    Since we have used the same values for all theorem sets, we now can
%    assign them.
%    \begin{macrocode}
\global\setlength\theorempreskipamount{12pt plus 5pt minus 3pt}
\global\setlength\theorempostskipamount{8pt plus 3pt minus 1.5pt}
%    \end{macrocode}
% \end{macro}
% \end{macro}
%
%
% \begin{macro}{\@endtheorem}
%    The same holds for the macro |\@endtheorem|, which ends a
%    {\envfont theorem} environment. Since it is the same for all
%    theorem sets, it is removed from the macros |\th@|\meta{style}.
%    It simply ends the {\envfont trivlist} environment, which was
%    begun in |\@thm|.
%    \begin{macrocode}
\global\let\@endtheorem=\endtrivlist
%    \end{macrocode}
% \end{macro}
%
%
%
% \begin{macro}{\@preamblecmds}
%    All macros defined above are to be used only in the preamble.
%    Therefore, we insert them in |\@preamblecmds| which will disable
%    them at begin document. This is done by the internal \LaTeXe{}
%    command |\@onlypreamble|.
% \changes{v2.0g}{89/09/19}{`theoremheaderfont and `theorembodyfont
%    added.}
% \changes{v2.2a}{94/02/03}{Use \LaTeXe{} interface}
%    \begin{macrocode}
\@onlypreamble\@xnthm
\@onlypreamble\@ynthm
\@onlypreamble\@othm
\@onlypreamble\newtheorem
\@onlypreamble\theoremstyle
\@onlypreamble\theorembodyfont
\@onlypreamble\theoremheaderfont
%    \end{macrocode}
% \end{macro}
%
% Finally we declare the \texttt{plain} theorem style to be the default.
% \changes{v2.1b}{90/06/04}{Fixed missing `item by declaring default
%    style.}
%    \begin{macrocode}
\theoremstyle{plain}
%</package>
%    \end{macrocode}
%
% \Finale

\endinput
}{}
\ProvidesFile{thb.sty}
             [\FMithmInfo]
%    \end{macrocode}
%
% \begin{macro}{\th@break}
%    |\theoremstyle{break}| produces a line break after the name
%    of the theorem.  The font is |\slshape|.  Hence, we define
%    |\th@break| as follows:
%    \begin{macrocode}
\gdef\th@break{\normalfont\slshape
  \def\@begintheorem##1##2{\item[%
%    \end{macrocode}
%    We run into the following problem: it is not possible to create
%    the header with |\item[|\meta{title}|]| and then start
%    a new line by, for example, |\mbox{}\\|. Such a definition
%    will fail whenever a list environment follows immediately.  With
%    the above construction, the |\mbox{}| causes the switch
%    |@inlabel| (cf.\ definition of |\list| and
%    |\trivlist| in \cite{latex:source}) to be set to \texttt{false}
%    and so the following list will insert additional vertical
%    space (|\topskip|). This is quite annoying. Therefore, we
%    create the line break within the |\item|. In order to ensure
%    that the text will begin at the proper position in the following
%    line, we simply pretend that the label does not take any
%    room.\footnote{This will lead to problems whenever very high
%    symbols occurring in the line tower into the
%    heading. So, something else has to be done here sometime.}
%    \begin{macrocode}
       \rlap{\vbox{\hbox{\hskip \labelsep\theorem@headerfont ##1\ ##2}%
                   \hbox{\strut}}}]}%
%    \end{macrocode}
%    Again, the definition of |\@opargbegintheorem| is
%    completely analogous.
%    \begin{macrocode}
\def\@opargbegintheorem##1##2##3{%
  \item[\rlap{\vbox{\hbox{\hskip \labelsep \theorem@headerfont
                     ##1\ ##2\ (##3)}%
                    \hbox{\strut}}}]}}
\endgroup
%</thb>
%    \end{macrocode}
% \end{macro}
%
%
% \subsubsection{The \texttt{changebreak} style}
%
%    \begin{macrocode}
% This style option is stored in the file |thcb.sty|.
%    \begin{macrocode}
%<*thcb>
\begingroup \makeatletter
\@ifundefined{theorem@style}{% \iffalse meta-comment
%
% Copyright (C) 1993-2024
%
% The LaTeX Project and any individual authors listed elsewhere
% in this file.
%
% This file is part of the Standard LaTeX `Tools Bundle'.
% -------------------------------------------------------
%
% It may be distributed and/or modified under the
% conditions of the LaTeX Project Public License, either version 1.3c
% of this license or (at your option) any later version.
% The latest version of this license is in
%    https://www.latex-project.org/lppl.txt
% and version 1.3c or later is part of all distributions of LaTeX
% version 2005/12/01 or later.
%
% The list of all files belonging to the LaTeX `Tools Bundle' is
% given in the file `manifest.txt'.
%
% \fi
%
%%
%
%
\def\FMithmInfo{2023/07/05 v2.2c Theorem extension package (FMi)}
%
% \ProvidesFile{theorem.dtx}[\FMithmInfo]
%\iffalse   % this is a METACOMMENT !
%%
%% Package `theorem' to use with LaTeX2e
%% Copyright (C) 1989-1995 Frank Mittelbach, all rights reserved.
%<+package>\@ifundefined{theorem@style}{}{\endinput}
%<+package>\NeedsTeXFormat{LaTeX2e}[1995/06/01]
%<+package>\ProvidesPackage{theorem}
%<+package>         [\FMithmInfo]
%
% \fi
%
% \changes{v1.0}{87/05/07}{First draft.}
% \changes{v1.0a}{87/05/07}{`thistheoremstyle' added.}
% \changes{v1.1a}{87/07/29}{New concept: redefine the internal
%                          \protect\LaTeX{} commands `@xnthm' etc.}
% \changes{v1.1b}{87/07/31}{`theoremfont' used to specify font.}
% \changes{v1.2a}{88/01/08}{`@endtheorem' now globally defined.}
% \changes{v1.2b}{88/01/11}{`newtheorem' only allowed in the preamble.}
% \changes{v1.3a}{88/02/14}{`theoremfont' now a toks register.}
%
% \changes{v2.0a}{88/11/16}{New implementation and new documentation.}
% \changes{v2.0a}{88/11/16}{No layout changes but internal modifications
%                           old substyles won't work.}
% \changes{v2.0b}{89/01/30}{`global' used for all definitions.}
% \changes{v2.0c}{89/06/15}{Documentation translated to English.
%                            Thanks to LLL and Nina}
% \changes{v2.1a}{90/02/15}{Support for new font selection.}
% \changes{v2.2a}{94/02/03}{Cleanup for LaTeX2e}
%
% \DoNotIndex{\@endparenv,\@ifdefinable,\@ifnextchar,\@ifundefined}
% \DoNotIndex{\@namedef,\@nameuse,\@spaces,\@tempa,\@warning}
% \DoNotIndex{\ }
% \DoNotIndex{\begingroup,\bfseries}
% \DoNotIndex{\csname}
% \DoNotIndex{\def,\do,\docdate}
% \DoNotIndex{\endcsname,\endgroup,\endinput,\endtrivlist,\expandafter}
% \DoNotIndex{\filedate,\fileversion}
% \DoNotIndex{\gdef,\global}
% \DoNotIndex{\hbox,\hskip}
% \DoNotIndex{\ignorespaces,\input,\itshape,\item}
% \DoNotIndex{\labelsep,\let,\llap}
% \DoNotIndex{\makeatletter}
% \DoNotIndex{\noexpand}
% \DoNotIndex{\refstepcounter,\relax,\rlap}
% \DoNotIndex{\setlength,\slshape,\space,\string,\strut}
% \DoNotIndex{\the,\trivlist,\typeout}
% \DoNotIndex{\vbox}
% \DoNotIndex{\xdef}
%
% \def\deflabel#1{\ttfamily #1\hfill}
% \def\deflist#1{\begin{list}{}{\settowidth\labelwidth{\ttfamily #1}%
%                               \setlength\leftmargin\labelwidth
%                               \addtolength\leftmargin\labelsep
%                               \let\makelabel\deflabel}}
% \def\enddeflist{\end{list}}
% \def\envfont{\normalfont\ttfamily}
%
% \MakeShortVerb{\|}
%
% \newcommand{\TUB}{{\sl TUGboat\/}}
%
% \GetFileInfo{theorem.dtx}
% \title{An Extension of the
%        \LaTeX{} theorem environment\thanks{This file
%        has version number \fileversion, last
%        revised \filedate.}}
% \author{Frank Mittelbach}
%
% \MaintainedByLaTeXTeam{tools}
% \maketitle
%
% \begin{abstract}
%    The macros described in this paper yield an extension of the
%    \LaTeX{} theorem mechanism. It is designed to satisfy the
%    different requirements of various journals. Thus, the
%    layout of the ``theorems'' can be manipulated by determining a
%    ``style''. This article describes not only the use, but
%    also the definition, of the necessary macros.
% \end{abstract}
%
% \section*{Preface to version 2.2}
%
% For \LaTeXe{} this package did not need any fundamental changes. I
% only modified the messages generated so that theorem layout styles
% will show up with the |\listfiles| command and cleaned the section
% on the New Font Selection Scheme since this is now included in
% \LaTeX.
%
% \section*{Preface to version 2.1}
%
% This version is identical to 2.0g described in \TUB\ 10\#3 except
% for some internal defaults which are now set depending on the used
% font selection scheme.
%
% This was done to avoid unpleasant surprises if the new font selection
% scheme is in force. For further details see
% section~\ref{sec:newfonts} and \cite{bk:GMS94}.
%
% \section{Introduction}
%
% For our purposes here, ``theorems'' are labelled enunciations,
% often set off from the main text by extra space and a font change.
% Theorems, corollaries, conjectures, definitions, and remarks
% are all instances of ``theorems''.  The ``header'' of these structures
% is composed of a label (such as \textsc{Theorem} or \textsc{Remark})
% and a number which serializes an item in the sequence of items
% with the same label.
%
% Shortly after the introduction of \LaTeX{} at the Fachbereich
% Mathematik in Mainz, the desire to manipulate the layout of
% ``theorems'' arose.  In Mainz, the following two conventions came
% into general use:
% \begin{enumerate}
%   \item The number of the theorem is shown in the margin.
%   \item There is a line break at
%         the end of the theorem header.
% \end{enumerate}
%
% Additionally, some journals require different formats which depend on
% the ``sort of theorem'': e.g.\ often remarks and definitions are set
% in |\upshape|, while |\itshape| is employed for main theorems.
%
% Confronted with these requirements, a theorem environment was
% developed in Mainz which allows separate determination of the
% layout of the ``theorems sets'', comparable to |\pagestyle|.
%
%
% \section{The user interface}
% \subsection{Defining new theorem sets}
%
% \DescribeMacro\newtheorem
% As in the original \LaTeX{} version, the command |\newtheorem|
% defines a new ``theorem set'' or ``theorem-like structure''.
% Two required arguments name the new environment and give the text
% to be typeset with each instance of the new ``set'', while
% an optional argument determines how
% the ``set'' is enumerated:
% \begin{description}
%    \item[\ttfamily \bslash newtheorem\{foo\}\{bar\}]
%       The theorem set {\envfont foo} (whose name is \texttt{bar})
%       uses its own counter.
%    \item[\ttfamily \bslash newtheorem\{foo2\}{[foo]}\{bar2\}]
%       The theorem set {\envfont foo2} (printed name \texttt{bar2})
%       uses the same counter as the theorem set \texttt{foo}.
%    \item[\ttfamily \bslash newtheorem\{foo3\}\{bar3\}{[section]}]
%       The theorem set {\envfont foo3} (printed name \texttt{bar3}) is
%       enumerated within the counter \texttt{section}, i.e.\ with every
%       new |\section| the enumeration begins again with 1, and
%       the enumeration is composed from the section-number and the
%       theorem counter itself.
% \end{description}
%
% \DescribeMacro\theoremstyle
% Additionally, the command |\theoremstyle| can define the layout
% of various, or all, theorem sets.  It should be noted that any
% theorem set defined by |\newtheorem| is typeset in the
% |\theoremstyle| that is current at the time of the definition.
% Thus, the following
% \begin{verbatim}
%    \theoremstyle{break}        \newtheorem{Cor}{Corollary}
%    \theoremstyle{plain}        \newtheorem{Exa}{Example}[section]
%\end{verbatim}                             ^^A No blank at line start
% leads to the result that the set {\envfont Cor} is formatted in the
% style \texttt{break}, while the set {\envfont Exa} and all the
% following ones are formatted in the style \texttt{plain}, unless
% another |\theoremstyle| follows.  Since the definitions installed by
% |\newtheorem| are global, one also can limit |\theoremstyle| locally
% by grouping braces.
%
% \DescribeMacro\theorembodyfont
% The choice of the font for the theorem body is completely
% independent of the chosen |\theoremstyle|; this has proven to
% be very advantageous.  For example,
% \begin{verbatim}
%   {\theorembodyfont{\upshape}          \newtheorem{Rem}{Remark}}
%\end{verbatim}
% defines a theorem set {\envfont Rem}, which will be set in
% |\upshape| in the current layout (which in our example is
% \texttt{plain}).  As with |\theoremstyle|, the |\theorembodyfont|
% chosen is that current at the time of |\newtheorem|.  If
% |\theorembodyfont| is not specified or one defines
% |\theorembodyfont{}|, then the font used will be that defined by the
% |\theoremstyle|.
%
% \DescribeMacro\theoremheaderfont It is also possible to customize
% the font used for the theorem headers.  This is, however, a global
% declaration, and therefore there should be at most one
% |\theoremheaderfont| declaration in the preamble.\footnote{If it is
% actually necessary to have different header fonts, one has to define
% new theorem styles (substituting the desired font) or specify the
% information directly in the \texttt{\bslash newtheorem} declaration
% (the unclean variant).}
%
% \DescribeMacro\theorempreskipamount
% \DescribeMacro\theorempostskipamount Two additional parameters
% affect the vertical space around the theorem environments:
% |\theorempreskipamount| and |\theorempostskipamount| define,
% respectively, the spacing before and after such an environment.
% These parameters apply for all theorem sets and can be manipulated
% with the ordinary length macros.  They are rubber lengths
% (`\textsf{skips}'), and therefore can contain \texttt{plus} and
% \texttt{minus} parts.
%
% Since the definition of theorem sets should---most sensibly---be
% placed in the preamble, we only allow installation there.
% It is therefore possible to release the memory used here after
% |\begin{document}|, in order to make room for other
% applications.
%
%
% \subsection{Existing theorem styles}
%
% The following theorem styles exist to date:
% \begin{deflist}{marginbreak}
%    \item[plain]
%       This theorem style emulates the original \LaTeX{} definition,
%       except that additionally the parameters
%       |\theorem...skipamount| are used.
%    \item[break]
%       In this style, the theorem header is followed by a line
%       break.
%    \item[marginbreak]
%       The theorem number is set in the margin, and there is a line
%       break as in \texttt{break}.
%    \item[changebreak]
%       Like \texttt{break}, but with header number and text
%       interchanged.
%    \item[change]
%       Header number and text are interchanged, without a line break.
%    \item[margin]
%       The number is set in the left margin, without a line break.
% \end{deflist}
% All styles (except \texttt{plain}) select |\slshape| as the default
% |\theorembodyfont|.
%
%
% \subsection{Examples}
%
% Given the above theorem sets {\envfont Cor}, {\envfont Exa}
% and {\envfont Rem},
% suppose that the preamble also contains the declarations:
% \begin{verbatim}
%    \theoremstyle{marginbreak}   \newtheorem{Lem}[Cor]{Lemma}
%    \theoremstyle{change}
%    \theorembodyfont{\itshape}        \newtheorem{Def}[Cor]{Definition}
%
%    \theoremheaderfont{\scshape}
%\end{verbatim}
% Then the following are some typical examples of the typeset output
% resulting from their use.
%
% \begin{Cor}
%    This is a sentence typeset in the theorem environment {\envfont
%    Cor}.
% \end{Cor}
%
% \begin{Exa}
%    This is a sentence typeset in the theorem environment {\envfont
%    Exa}.
% \end{Exa}
%
% \begin{Rem}
%    This is a sentence typeset in the theorem environment {\envfont
%    Rem}.
% \end{Rem}
%
% \begin{Lem}[Ben User]
%    This is a sentence typeset in the theorem environment {\envfont
%    Lem}.
% \end{Lem}
%
% \begin{Def}[Very impressive Definition]
%    This is a sentence typeset in the theorem environment {\envfont
%    Def}.
% \end{Def}
%
% The last two examples show the effect of the optional argument to
% a theorem environment (it is the text typeset in parentheses).
%
%
% \section{Special Considerations}
% \label{sec:newfonts}
%
% Theoremheader and body are implemented as a unit. This means that
% the |\theoremheaderfont|  will inherit characteristics of the
% |\theorembodyfont| in \LaTeXe. Thus, if for example
% |\theorembodyfont| is |\itshape| and |\theoremheaderfont|
% is |\bfseries| the font selected for the header will have the
% characteristics `bold extended italic'. If this is not desired one
% should set the |\theoremheaderfont| to something like
% \begin{verbatim}
%      \theoremheaderfont{\normalfont\bfseries}
%\end{verbatim}
% i.e.\ supplying all necessary font information explicitly.
%
%
%
% \section{Acknowledgements}
%
% The publication of this set of macros was only possible with the
% help of Christina Busse (translating the manuscript into English),
% Joachim Pense (playing the r\^ole of typist), Chris Rowley
% (looking everything over) and many others providing useful
% suggestions.
%
% \MaybeStop{%
%    \begin{thebibliography}{9}
%       \bibitem{bk:GMS94} \textsc{M.~Goossens}, \textsc{F.~Mittelbach}
%         and \textsc{A.~Samarin}.
%         \newblock The \LaTeX{} Companion.
%         \newblock
%         Addison-Wesley, Reading, Massachusetts, 1994.
%       \bibitem{latex:source} \textsc{Lamport, Leslie.} \newblock
%                               \texttt{latex.tex}, version 2.09, date
%                               Feb. 1990.
%    \end{thebibliography}
%    \PrintIndex
%    \PrintChanges}
%
%
% \section{The documentation driver file}
%
% The next bit of code contains the documentation driver file for
% \TeX{}, i.e., the file that will produce the documentation you are
% currently reading. It will be extracted from this file by the
% \texttt{docstrip} program. Since it is the first code in the file
% one can alternatively process this file directly with \LaTeXe{} to
% obtain the documentation.
%
%    \begin{macrocode}
%<*driver>
\documentclass{ltxdoc}

\usepackage{theorem}

%<+driver>% The next few lines define theorem sets which are used
%<+driver>% in the example section of the documentation.

\theoremstyle{break}       \newtheorem{Cor}{Corollary}
\theoremstyle{plain}       \newtheorem{Exa}{Example}[section]
{\theorembodyfont{\upshape}\newtheorem{Rem}{Remark}}
\theoremstyle{marginbreak} \newtheorem{Lem}[Cor]{Lemma}
\theoremstyle{change}
\theorembodyfont{\itshape} \newtheorem{Def}[Cor]{Definition}

\theoremheaderfont{\scshape}

\RecordChanges

\begin{document}
   \DocInput{theorem.dtx}
\end{document}
%</driver>
%    \end{macrocode}
%
% \section{Definition of the Macros}
%
% If the file has been loaded before, we abort immediately. If not the
% package announces itself (this is actually done at the very top of
% the file---the way it is done isn't good style so don't copy it).
% \changes{v2.0e}{89/07/19}{Spaces removed from `typeout'.}
% \changes{v2.2b}{95/11/19}{Announce moved to top of file}
%    \begin{macrocode}
%<*package>
%\@ifundefined{theorem@style}{}{\endinput}
%\def\FMithmInfo{1995/11/19 v2.2b Theorem extension package (FMi)}
%\ProvidesPackage{theorem}[\FMithmInfo]
%    \end{macrocode}
%
%
% \subsection{Definition of theorem styles and fonts}
%
% All the definitions in this file are done globally to allow
% inputting this file inside a group.
%
% \begin{macro}{\theoremstyle}
%    Before a theorem style can be installed, the chosen style must be
%    known. For that reason, we must test to see that
%    |\th@|\meta{style} is known or, more precisely, that it is
%    different from |\relax|. If the style is not known then
%    |\th@plain| is used.
%    \begin{macrocode}
\gdef\theoremstyle#1{%
   \@ifundefined{th@#1}{\@warning
          {Unknown theoremstyle `#1'. Using `plain'}%
          \theorem@style{plain}}%
%    \end{macrocode}
%    We save the theorem style to be used in the token register
%    |\theorem@style|.
%    \begin{macrocode}
      {\theorem@style{#1}}%
%    \end{macrocode}
%    Now we ``evaluate'' the theorem style: this means, we call the
%    macro |\th@|\meta{style} which will activate the relevant
%    definitions which are contained in a separate file.  This is done
%    in a group to suppress changes to the current font.  This could
%    otherwise pose problems together with the new font selection
%    scheme\footnote{When I printed the original article using the new
%    font selection scheme I ended with a document with slanted
%    typefaces (text headings and all) simply because one of the
%    theorem styles used \texttt{\bslash sl} at toplevel.} if the
%    |\th@|\meta{style} is evaluated a second time.
% \changes{v2.1a}{90/02/15}{Group added around `th@..}
% \changes{v2.1a}{90/02/15}{Reseting of `..@begintheorem removed.}
%    \begin{macrocode}
      \begingroup
        \csname th@\the\theorem@style \endcsname
      \endgroup}
%    \end{macrocode}
% \end{macro}
%
%
% \begin{macro}{\@begintheorem}
% \changes{v2.1a}{90/02/15}{Relaxed at toplevel.}
% \begin{macro}{\@opargbegintheorem}
% \changes{v2.1a}{90/02/15}{Relaxed at toplevel.}
%    We reset |\@begintheorem| and |\@opargbegintheorem| to
%    |\relax| since these commands are no longer necessary at
%    toplevel. This will save a few tokens.
%    \begin{macrocode}
\global\let\@begintheorem\relax
\global\let\@opargbegintheorem\relax
%    \end{macrocode}
% \end{macro}
% \end{macro}
%
% \begin{macro}{\theorem@style}
%    Obviously the \textsf{token} register used above has to be
%    allocated. To assure the utmost compatibility with the
%    original \LaTeX{} definition, we set the default theorem style to
%    \texttt{plain}, which implements the usual \LaTeX{} convention.
% \changes{v2.0f}{89/07/20}{Unnecessary `global' removed.}
%    \begin{macrocode}
\newtoks\theorem@style
\global\theorem@style{plain}
%    \end{macrocode}
% \end{macro}
%
% \begin{macro}{\theorembodyfont}
% \begin{macro}{\theorem@bodyfont}
%    If the |\theorembodyfont| is set by the user then it should not
%    interact with the default font set in the theorem style. When the
%    new font selection is in force this may happen if, for example,
%    the default is |\itshape| and the new |\theorembodyfont| is
%    |\sffamily|. So we add a |\reset@font| command in front of the
%    user definition.
% \changes{v2.0f}{89/07/20}{Unnecessary `global' removed.}
% \changes{v2.0g}{89/09/19}{`theoremfont to `theorembodyfont.}
% \changes{v2.1c}{92/06/22}{`reset@font added and logic changed.}
% \changes{v2.1d}{92/10/28}{Reset to empty if arg is empty}
%    \begin{macrocode}
\gdef\theorembodyfont#1{%
%    \end{macrocode}
%    We check if the argument supplied is empty and if so put nothing
%    into the |\theorem@bodyfont| token register to allow for
%    |\theorembodyfont{}| as a mean of using the default of the
%    current |\theoremstyle|.
%    \begin{macrocode}
   \def\@tempa{#1}%
   \ifx\@tempa\@empty
    \theorem@bodyfont{}%
   \else
    \theorem@bodyfont{\reset@font#1}%
   \fi
   }
\newtoks\theorem@bodyfont
\global\theorem@bodyfont{}
%    \end{macrocode}
% \end{macro}
% \end{macro}
%
%
% \begin{macro}{\theoremheaderfont}
% \changes{v2.0g}{89/09/19}{Macro added.}
%    The font for the theorem headers is handled differently because
%    this definition applies to all theorem styles.
%    \begin{macrocode}
\gdef\theoremheaderfont#1{\gdef\theorem@headerfont{#1}%
%    \end{macrocode}
%    After using the macro once it is redefined to produce an error
%    message.
%    \begin{macrocode}
       \gdef\theoremheaderfont##1{%
        \typeout{\string\theoremheaderfont\space should be used
                 only once.}}}
%    \end{macrocode}
% \end{macro}
%
% \begin{macro}{\theorem@headerfont}
% \changes{v2.0g}{89/09/19}{Macro added.}
%    To set the |\theorem@headerfont| default we first test if
%    the new fontselection scheme is in force.
% \changes{v2.1a}{90/02/15}{Support for new font selection.}
%    \begin{macrocode}
\ifx\upshape\undefined
%    \end{macrocode}
%    If not we define it to expand into |\bfseries|. We don't use
%    |\let| just in case a following style option redefines this
%    macro.
%    \begin{macrocode}
\gdef\theorem@headerfont{\bfseries}
%    \end{macrocode}
%    Otherwise we reset the current shape before calling |\bfseries|.
%    \begin{macrocode}
\else \gdef\theorem@headerfont{\normalfont\bfseries}\fi
%    \end{macrocode}
% \end{macro}
%
% \begin{macro}{\th@plain}
% \begin{macro}{\th@break}
% \begin{macro}{\th@marginbreak}
% \begin{macro}{\th@changebreak}
% \begin{macro}{\th@change}
% \begin{macro}{\th@margin}
%    The different styles are defined in macros such as
%    |\th@plain|.  Since memory space is precious in
%    ``non-Big-versions'', we have to avoid offering too many unused
%    definitions.  Therefore we define these styles in separate files
%    that can be loaded on demand.  Thus the commands themselves only
%    load these files. We use |\@input@|, a \LaTeXe{} internal command
%    that ensures that the file will be listed with |\listfiles|.
% \changes{v2.2a}{94/02/02}{Use `@input@ to load theorem layout files}
%    \begin{macrocode}
\gdef\th@plain{\@input@{thp.sty}}
\gdef\th@break{\@input@{thb.sty}}
\gdef\th@marginbreak{\@input@{thmb.sty}}
\gdef\th@changebreak{\@input@{thcb.sty}}
\gdef\th@change{\@input@{thc.sty}}
\gdef\th@margin{\@input@{thm.sty}}
%    \end{macrocode}
%    This list will be expanded when new styles become
%    available.  For testing,  just append new theorem substyles
%    as document options.
% \end{macro}
% \end{macro}
% \end{macro}
% \end{macro}
% \end{macro}
% \end{macro}
%
%
%
% \subsection{Definition of a new theorem set}
%
% As already pointed out, a new theorem environment can be defined in
% three different ways:
% \begin{verbatim}
%    \newtheorem{Lem}{Lemma}
%    \newtheorem{Lem}{Lemma}[section]
%    \newtheorem{Lem}[Theorem]{Lemma}
%\end{verbatim}
% The function of the macro |\newtheorem| is to recognize these cases
% and then to branch into one of the three macros |\@ynthm|, |\@xnthm|
% or |\@othm|.  This mechanism is adopted unchanged from
% \cite{latex:source}; the essential point here is that, for example,
% in the second case, the arguments \texttt{Lem}, \texttt{Lemma} and
% \texttt{section} are passed over to the macro |\@xnthm|.
%
% We inspect this case first because the others present fewer problems,
% and thus are easily derived from this one.
%
% \begin{macro}{\@xnthm}
%    For our example arguments, the macro |\@xnthm| must fulfill
%    the following:
%    \begin{itemize}
%       \item
%          Define a new \LaTeX-counter `Lem'
%       \item
%          reset this counter within a |\section|
%       \item
%          define the macro |\theLem|
%       \item
%          define the environment macros |\Lem| and
%          |\endLem| using the current
%          |\theoremstyle| and |\theorem@bodyfont|.
%    \end{itemize}
%    Obviously, all this should happen only if the first argument
%    of |\@xnthm| (i.e.\ \texttt{Lem} in our example) is chosen so
%    as not to conflict with any previously defined commands or
%    environments.  This test is performed by the \LaTeX{} macro
%    |\@ifdefinable|.
%    \begin{macrocode}
\gdef\@xnthm#1#2[#3]{\expandafter\@ifdefinable\csname #1\endcsname
%    \end{macrocode}
%    Therefore, the first argument of |\@ifdefinable| is the
%    expansion (in the example, |\Lem|) of
%    |\csname||#1\endcsname|.  The second argument is
%    executed only if the test has been completed successfully.
%    \begin{macrocode}
   {%
%    \end{macrocode}
%    Now we define the new counter. The names of the \LaTeX{}
%    macros employed should speak for themselves:
% \changes{v2.2b}{1995/10/07}{Use \cs{@newctr}}
% \changes{v2.2c}{1995/11/23}{Use \cs{@newctr} properly.}
%    \begin{macrocode}
    \@definecounter{#1}\@newctr{#1}[#3]%
%    \end{macrocode}
%    Using |\@newctr| will give a proper error message if the counter
%    in |#3| is not defined.
%    In defining `|\theLem|' we must generate the desired macro
%    name by use of |\expandafter| and |\csname|.
%    \begin{macrocode}
    \expandafter\xdef\csname the#1\endcsname
%    \end{macrocode}
%    An |\xdef| is used in order to make the definition global,
%    and to ensure that it contains the replacement texts of
%    |\@thmcountersep| and |\@thmcounter|.\footnote{These
%    two macros can be defined by the document style.  Their default
%    values produce a `\texttt{.}' as separation and an arabic
%    representation of the number.} However, not everything should be
%    expanded. For example, it saves space to use |\thesection|
%    instead of its---at times---lengthy expansion.
%    \begin{macrocode}
      {\expandafter \noexpand \csname the#3\endcsname
       \@thmcountersep \@thmcounter{#1}}%
%    \end{macrocode}
%    Thus with the defaults of \LaTeX{}, |\theLem| would be
%    replaced by the command sequence |\thesection.\arabic{Lem}|.
%
%    We will now look at the definition of the macro which is executed
%    at the beginning of the actual environment (in our example this
%    macro is |\Lem|). It should be noted that we use an
%    ``|\expandafter| trick'' to expand only certain parts of the
%    replacement text at the time of the definition.
%    \begin{macrocode}
    \def\@tempa{\global\@namedef{#1}}%
    \expandafter \@tempa \expandafter{%
%    \end{macrocode}
%    First, the macro that contains the current definitions of
%    |\@begintheorem| and |\@opargtheorem| should be called
%    up.  The name of this macro---as is already known---has the form
%    |\th@|\meta{theorem\ style}; therefore, it must be called by
%    \begin{macrocode}
      \csname th@\the \theorem@style
%    \end{macrocode}
%    In addition the default theorem font should be changeable, i.e.\
%    we have to insert the contents of |\theorem@bodyfont|. For
%    that reason, we expand even further, beyond |\endcsname|,
%    and thus insert the contents of the \textsf{token} register
%    |\theorem@bodyfont| in the replacement text.
%    \begin{macrocode}
            \expandafter \endcsname \the \theorem@bodyfont
%    \end{macrocode}
%    Now it is time to call the macro |\@thm| which takes over
%    the further processing.  It has two arguments: the current
%    counter name (in our example, \texttt{Lem}), and the text of the
%    label (in our example, \texttt{Lemma}).
%    \begin{macrocode}
     \@thm{#1}{#2}}%
%    \end{macrocode}
%    With this, the `sub-definition' is complete.  The macro
%    |\@endtheorem| ends a theorem environment and is, so far, nothing
%    but an |\endtrivlist|.  (Hence it is defined globally, and not
%    within the theorem styles.\footnote{This has to be changed as
%    soon as theorem styles that change \texttt{\bslash @endtheorem}
%    exist.  In such a case, all existing styles must be changed as
%    well since they will have to reset the macro.}) Therefore, we can
%    set it equivalent to the macro that ends the theorem set (in our
%    example, |\endLem|).  However, if some day theorem styles exist
%    that do change |\@endtheorem|, we would have to use the
%    commented-out line instead.
%    \begin{macrocode}
    \global \expandafter \let \csname end#1\endcsname \@endtheorem
%   \global\@namedef{end#1}{\@endtheorem}%
%    \end{macrocode}
%    With these commands all the required definitions are employed,
%    unless the test |\@ifdefinable| has failed.  Therefore, we
%    end the second argument of this macro and with it the definition
%    of |\@xnthm|.
%    \begin{macrocode}
   }}
%    \end{macrocode}
% \end{macro}
%
% \begin{macro}{\@ynthm}
%    The definition of |\@ynthm| is completely analogous. In
%    this case the new counter that is defined is not reset within
%    another counter; thus the definition of |\the...|
%    is simplified:
%    \begin{macrocode}
\gdef\@ynthm#1#2{\expandafter\@ifdefinable\csname #1\endcsname
   {\@definecounter{#1}%
    \expandafter\xdef\csname the#1\endcsname{\@thmcounter{#1}}%
%    \end{macrocode}
%    The rest of the definition corresponds literally to that of
%    |\@xnthm|:
%    \begin{macrocode}
    \def\@tempa{\global\@namedef{#1}}\expandafter \@tempa
     \expandafter{\csname th@\the \theorem@style \expandafter
     \endcsname \the\theorem@bodyfont \@thm{#1}{#2}}%
    \global \expandafter \let \csname end#1\endcsname \@endtheorem}}
%    \end{macrocode}
% \end{macro}
%
% \begin{macro}{\@othm}
%    The definition of |\@othm| does not contain anything new.
%    \begin{macrocode}
\gdef\@othm#1[#2]#3{%
%    \end{macrocode}
% \changes{v2.2b}{95/10/07}{Build in check for counter defined}
%    We do not define a new counter but instead use one that has
%    already been defined.  Thus the only definition we need is that
%    of this pseudo-counter (i.e.\ |\the|\meta{env.\ name}).
%    First we check if |#2| corresponds to a known counter name.
%    \begin{macrocode}
  \expandafter\ifx\csname c@#2\endcsname\relax
   \@nocounterr{#2}%
  \else
   \expandafter\@ifdefinable\csname #1\endcsname
   {\expandafter \xdef \csname the#1\endcsname
     {\expandafter \noexpand \csname the#2\endcsname}%
%    \end{macrocode}
%    All other parts of the definition can be adopted from
%    |\@xnthm|. We have to remember, though, that in this case
%    the name of the current counter and the theorem label have moved
%    to the second and third arguments.
%    \begin{macrocode}
    \def\@tempa{\global\@namedef{#1}}\expandafter \@tempa
     \expandafter{\csname th@\the \theorem@style \expandafter
     \endcsname \the\theorem@bodyfont \@thm{#2}{#3}}%
    \global \expandafter \let \csname end#1\endcsname \@endtheorem}%
  \fi}
%    \end{macrocode}
% \end{macro}
%
% \subsection{Macros that are employed in a {\envfont theorem}
%    environment}
%
% \begin{macro}{\@thm}
%    The macro |\@thm| has to increase the current counter.  Then,
%    depending on whether the environment has (or does not have) an
%    optional argument, it has to branch into either |\@begintheorem|
%    or |\@opargtheorem|.
%    \begin{macrocode}
\gdef\@thm#1#2{\refstepcounter{#1}%
%    \end{macrocode}
%    Now we start a {\envfont trivlist} environment, and give
%    |\@topsep| and |\@topsepadd| the values of the \textsf{skip}
%    registers |\theorempreskipamount| and |\theorempostskipamount|.
%    The value in |\@topsep| is the vertical space that is inserted by
%    the first (and only) |\item| in our |\trivlist| whilst
%    |\@topsepadd| is inserted by |\@endparenv| at the end of that
%    {\envfont trivlist} environment.  By using these registers, we
%    obtain the desired space around a {\envfont theorem} environment.
%    \begin{macrocode}
   \trivlist
   \@topsep \theorempreskipamount               % used by first \item
   \@topsepadd \theorempostskipamount           % used by \@endparenv
%    \end{macrocode}
%    Now we have to test whether an optional argument has been given.
% \changes{v2.0d}{89/07/18}{Forgotten percent added.}
%    \begin{macrocode}
   \@ifnextchar [%
%    \end{macrocode}
%    If there is an optional argument, we will call |\@ythm|, and
%    move the arguments read back into the input stream.
%    \begin{macrocode}
   {\@ythm{#1}{#2}}%
%    \end{macrocode}
%    If not, we call |\@begintheorem|. Its first argument is the
%    name of the theorem set (hence the second argument of
%    |\@thm|). Its second argument is the macro that produces the
%    current number.
%    \begin{macrocode}
   {\@begintheorem{#2}{\csname the#1\endcsname}\ignorespaces}}
%    \end{macrocode}
% \end{macro}
%
% \begin{macro}{\@xthm}
% \begin{macro}{\@ythm}
%    Both these macros were originally called by |\@thm|.  We do
%    not need |\@xthm| anymore, hence we reset it to
%    |\relax|. The definition of |\@ythm| has not changed at
%    all from its definition in \LaTeX. In order to make the macros
%    easier to understand, we will
%    nevertheless present it (commented out).
%    \begin{macrocode}
\global\let\@xthm\relax
% \def\@ythm#1#2[#3]{\@opargbegintheorem{#2}{\csname
%           the#1\endcsname}{#3}\ignorespaces}
%</package>
%    \end{macrocode}
%    The primitive |\ignorespaces| in |\@ythm| and
%    |\@thm| is needed to remove the spaces between the
%    |\begin{...}| and the actual text.
% \end{macro}
% \end{macro}
%
%
% \subsection{Definition of the theorem substyles}
%
% As already pointed out, the theorem substyles, defined below, are
% only loaded when necessary.  Note that all these substyles, except
% \texttt{plain}, have |\slshape| as the default body font.
%
% \subsubsection{The \texttt{plain} style}
%
% \label{dok:th@plain} As the following macros use |@|, we have
% to locally set the |\catcode| of this symbol to ``letter''.
% This happens within a group, so that we do not have to worry about
% which |\catcode| that symbol had before.
%    \begin{macrocode}
%<*thp>
\begingroup \makeatletter
%    \end{macrocode}
% Since we are now within a group, we must make all definitions
% globally.  First we make sure that \texttt{theorem.sty} is loaded.
% This will allow us to use this file as a document style option
% without having to call \texttt{theorem} itself as an option.  At the
% same time, we assure that at least version 2 is loaded, since
% |\theorem@style| was not defined in earlier versions.
%    \begin{macrocode}
\@ifundefined{theorem@style}{\input{theorem.sty}}{}
\ProvidesFile{thp.sty}
             [\FMithmInfo]
%    \end{macrocode}
%
%
% \begin{macro}{\th@plain}
% \begin{macro}{\@begintheorem}
% \begin{macro}{\@opargbegintheorem}
%    |\theoremstyle{plain}| corresponds to the original
%    definition, except that the distances to the surrounding text are
%    determined by the parameters |\theorempreskipamount| and
%    |\theorempostskipamount|.
%    First we set the default body font.
%    \begin{macrocode}
\gdef\th@plain{\normalfont\itshape
%    \end{macrocode}
%    Then we define |\@begintheorem| and
%    |\@opargbegintheorem|. These two macros define how the
%    header of a theorem is typeset.  |\@opargbegintheorem| will
%    be called if a \textsf{theorem} environment with an optional
%    argument is encountered; otherwise, the header is constructed by
%    calling |\@begintheorem|.  If one of these macros is
%    executed, we are within a \textsf{trivlist} environment started by
%    |\@thm|.  So the theorem header is produced with an
%    |\item| command.
%
%    Instead of specifying the header font directly, all standard
%    theorem styles use the |\theorem@headerfont| macro to allow
%    customization.  The extra space (|\labelsep|) is necessary
%    because of problems in the \textsf{trivlist} environment.
%    \begin{macrocode}
  \def\@begintheorem##1##2{%
        \item[\hskip\labelsep \theorem@headerfont ##1\ ##2]}%
%    \end{macrocode}
%    The definition of |\@opargbegintheorem| is completely
%    analogous. The only difference is the fact that there exists a
%    third argument (which is the optional parameter of the
%    environment and contains additional information about the
%    theorem).  Customarily we enclose it in parentheses.
%    \begin{macrocode}
\def\@opargbegintheorem##1##2##3{%
   \item[\hskip\labelsep \theorem@headerfont ##1\ ##2\ (##3)]}}
%    \end{macrocode}
% \end{macro}
% \end{macro}
% \end{macro}
%
% \noindent
% We conclude with an |\endgroup| to restore the
% |\catcode| of |@|.
%    \begin{macrocode}
\endgroup
%</thp>
%    \end{macrocode}
%
%
% \subsubsection{The \texttt{break} style}
%
% This style option is stored in the file |thb.sty|.
% For the next two lines see the documentation for |\th@plain| on
% page~\pageref{dok:th@plain}.
%    \begin{macrocode}
%<*thb>
\begingroup \makeatletter
\@ifundefined{theorem@style}{\input{theorem.sty}}{}
\ProvidesFile{thb.sty}
             [\FMithmInfo]
%    \end{macrocode}
%
% \begin{macro}{\th@break}
%    |\theoremstyle{break}| produces a line break after the name
%    of the theorem.  The font is |\slshape|.  Hence, we define
%    |\th@break| as follows:
%    \begin{macrocode}
\gdef\th@break{\normalfont\slshape
  \def\@begintheorem##1##2{\item[%
%    \end{macrocode}
%    We run into the following problem: it is not possible to create
%    the header with |\item[|\meta{title}|]| and then start
%    a new line by, for example, |\mbox{}\\|. Such a definition
%    will fail whenever a list environment follows immediately.  With
%    the above construction, the |\mbox{}| causes the switch
%    |@inlabel| (cf.\ definition of |\list| and
%    |\trivlist| in \cite{latex:source}) to be set to \texttt{false}
%    and so the following list will insert additional vertical
%    space (|\topskip|). This is quite annoying. Therefore, we
%    create the line break within the |\item|. In order to ensure
%    that the text will begin at the proper position in the following
%    line, we simply pretend that the label does not take any
%    room.\footnote{This will lead to problems whenever very high
%    symbols occurring in the line tower into the
%    heading. So, something else has to be done here sometime.}
%    \begin{macrocode}
       \rlap{\vbox{\hbox{\hskip \labelsep\theorem@headerfont ##1\ ##2}%
                   \hbox{\strut}}}]}%
%    \end{macrocode}
%    Again, the definition of |\@opargbegintheorem| is
%    completely analogous.
%    \begin{macrocode}
\def\@opargbegintheorem##1##2##3{%
  \item[\rlap{\vbox{\hbox{\hskip \labelsep \theorem@headerfont
                     ##1\ ##2\ (##3)}%
                    \hbox{\strut}}}]}}
\endgroup
%</thb>
%    \end{macrocode}
% \end{macro}
%
%
% \subsubsection{The \texttt{changebreak} style}
%
%    \begin{macrocode}
% This style option is stored in the file |thcb.sty|.
%    \begin{macrocode}
%<*thcb>
\begingroup \makeatletter
\@ifundefined{theorem@style}{\input{theorem.sty}}{}
\ProvidesFile{thcb.sty}
             [\FMithmInfo]
%    \end{macrocode}
%
% \begin{macro}{\th@changebreak}
%    The \texttt{change-break} theorem style is like \texttt{break}
%    but with interchange of theorem name and theorem number.  Thus we
%    define |\th@changebreak| as follows:
%    \begin{macrocode}
\gdef\th@changebreak{\normalfont\slshape
  \def\@begintheorem##1##2{\item
       [\rlap{\vbox{\hbox{\hskip\labelsep \theorem@headerfont ##2\ ##1}%
                    \hbox{\strut}}}]}%
\def\@opargbegintheorem##1##2##3{%
  \item[\rlap{\vbox{\hbox{\hskip\labelsep \theorem@headerfont
                     ##2\ ##1\ (##3)}%
                    \hbox{\strut}}}]}}
\endgroup
%</thcb>
%    \end{macrocode}
% \end{macro}
%
%
% \subsubsection{The \texttt{change} style}
%
% This style option is stored in the file |thc.sty|.
%    \begin{macrocode}
%<*thc>
\begingroup \makeatletter
\@ifundefined{theorem@style}{\input{theorem.sty}}{}
\ProvidesFile{thc.sty}
             [\FMithmInfo]
%    \end{macrocode}
%
% \begin{macro}{\th@change}
%    The \texttt{change} theorem style corresponds to the
%    \texttt{change break} style without a linebreak after the header.
%    To say it in another way, it's the same as the \texttt{plain}
%    style but with number and name interchanged and |\slshape| as the
%    default font.
%    \begin{macrocode}
\gdef\th@change{\normalfont\slshape
  \def\@begintheorem##1##2{\item
       [\hskip\labelsep \theorem@headerfont ##2\ ##1]}%
\def\@opargbegintheorem##1##2##3{%
  \item[\hskip\labelsep \theorem@headerfont ##2\ ##1\ (##3)]}}
\endgroup
%</thc>
%    \end{macrocode}
% \end{macro}
%
%
% \subsubsection{The \texttt{marginbreak} style}
%
% This style option is the one used most often at Mainz.  It is saved
% in the file |thmb.sty|.
%    \begin{macrocode}
%<*thmb>
\begingroup \makeatletter
\@ifundefined{theorem@style}{\input{theorem.sty}}{}
\ProvidesFile{thmb.sty}
             [\FMithmInfo]
%    \end{macrocode}
%
% \begin{macro}{\th@marginbreak}
%    The \texttt{margin break} style is nearly the same as the
%    \texttt{change break} style. The only difference is the placement
%    of the theorem number.  We use |\llap| to place it in the left
%    margin.
%
%    In this style |\labelsep| denotes the separation between the
%    number and the text.
%    \begin{macrocode}
\gdef\th@marginbreak{\normalfont\slshape
  \def\@begintheorem##1##2{\item
       [\rlap{\vbox{\theorem@headerfont
                    \hbox{\llap{##2}\hskip\labelsep ##1}%
                    \hbox{\strut}}}]}%
\def\@opargbegintheorem##1##2##3{%
  \item[\rlap{\vbox{\theorem@headerfont
                    \hbox{\llap{##2}\hskip\labelsep ##1\ (##3)}%
                    \hbox{\strut}}}]}}
\endgroup
%</thmb>
%    \end{macrocode}
% \changes{v2.1c}{90/02/17}{`endgroup outside macrocode env.}
% \end{macro}
%
%
%
% \subsubsection{The \texttt{margin} style}
%
% This style option is stored in the file |thm.sty|.
%    \begin{macrocode}
%<*thm>
\begingroup \makeatletter
\@ifundefined{theorem@style}{\input{theorem.sty}}{}
\ProvidesFile{thm.sty}
             [\FMithmInfo]
%    \end{macrocode}
%
% \begin{macro}{\th@margin}
%    Again this is only a variant of the theorem styles described
%    above without any new ideas.
%    \begin{macrocode}
\gdef\th@margin{\normalfont\slshape
  \def\@begintheorem##1##2{\item
       [\theorem@headerfont \llap{##2}\hskip\labelsep ##1]}%
\def\@opargbegintheorem##1##2##3{%
  \item[\theorem@headerfont \llap{##2}\hskip\labelsep ##1\ (##3)]}}
\endgroup
%</thm>
%    \end{macrocode}
% \end{macro}
%
%
%
%
% \subsection{Final Definitions}
%
% \begin{macro}{\theorempreskipamount}
% \begin{macro}{\theorempostskipamount}
%    The \textsf{skip} parameters that regulate the vertical empty space
%    before and after the {\envfont theorem} environment have
%    to be allocated as well.
% \changes{v2.0f}{89/07/20}{Unnecessary `global' removed.}
%    \begin{macrocode}
%<*package>
\newskip\theorempreskipamount
\newskip\theorempostskipamount
%    \end{macrocode}
%    Since we have used the same values for all theorem sets, we now can
%    assign them.
%    \begin{macrocode}
\global\setlength\theorempreskipamount{12pt plus 5pt minus 3pt}
\global\setlength\theorempostskipamount{8pt plus 3pt minus 1.5pt}
%    \end{macrocode}
% \end{macro}
% \end{macro}
%
%
% \begin{macro}{\@endtheorem}
%    The same holds for the macro |\@endtheorem|, which ends a
%    {\envfont theorem} environment. Since it is the same for all
%    theorem sets, it is removed from the macros |\th@|\meta{style}.
%    It simply ends the {\envfont trivlist} environment, which was
%    begun in |\@thm|.
%    \begin{macrocode}
\global\let\@endtheorem=\endtrivlist
%    \end{macrocode}
% \end{macro}
%
%
%
% \begin{macro}{\@preamblecmds}
%    All macros defined above are to be used only in the preamble.
%    Therefore, we insert them in |\@preamblecmds| which will disable
%    them at begin document. This is done by the internal \LaTeXe{}
%    command |\@onlypreamble|.
% \changes{v2.0g}{89/09/19}{`theoremheaderfont and `theorembodyfont
%    added.}
% \changes{v2.2a}{94/02/03}{Use \LaTeXe{} interface}
%    \begin{macrocode}
\@onlypreamble\@xnthm
\@onlypreamble\@ynthm
\@onlypreamble\@othm
\@onlypreamble\newtheorem
\@onlypreamble\theoremstyle
\@onlypreamble\theorembodyfont
\@onlypreamble\theoremheaderfont
%    \end{macrocode}
% \end{macro}
%
% Finally we declare the \texttt{plain} theorem style to be the default.
% \changes{v2.1b}{90/06/04}{Fixed missing `item by declaring default
%    style.}
%    \begin{macrocode}
\theoremstyle{plain}
%</package>
%    \end{macrocode}
%
% \Finale

\endinput
}{}
\ProvidesFile{thcb.sty}
             [\FMithmInfo]
%    \end{macrocode}
%
% \begin{macro}{\th@changebreak}
%    The \texttt{change-break} theorem style is like \texttt{break}
%    but with interchange of theorem name and theorem number.  Thus we
%    define |\th@changebreak| as follows:
%    \begin{macrocode}
\gdef\th@changebreak{\normalfont\slshape
  \def\@begintheorem##1##2{\item
       [\rlap{\vbox{\hbox{\hskip\labelsep \theorem@headerfont ##2\ ##1}%
                    \hbox{\strut}}}]}%
\def\@opargbegintheorem##1##2##3{%
  \item[\rlap{\vbox{\hbox{\hskip\labelsep \theorem@headerfont
                     ##2\ ##1\ (##3)}%
                    \hbox{\strut}}}]}}
\endgroup
%</thcb>
%    \end{macrocode}
% \end{macro}
%
%
% \subsubsection{The \texttt{change} style}
%
% This style option is stored in the file |thc.sty|.
%    \begin{macrocode}
%<*thc>
\begingroup \makeatletter
\@ifundefined{theorem@style}{% \iffalse meta-comment
%
% Copyright (C) 1993-2024
%
% The LaTeX Project and any individual authors listed elsewhere
% in this file.
%
% This file is part of the Standard LaTeX `Tools Bundle'.
% -------------------------------------------------------
%
% It may be distributed and/or modified under the
% conditions of the LaTeX Project Public License, either version 1.3c
% of this license or (at your option) any later version.
% The latest version of this license is in
%    https://www.latex-project.org/lppl.txt
% and version 1.3c or later is part of all distributions of LaTeX
% version 2005/12/01 or later.
%
% The list of all files belonging to the LaTeX `Tools Bundle' is
% given in the file `manifest.txt'.
%
% \fi
%
%%
%
%
\def\FMithmInfo{2023/07/05 v2.2c Theorem extension package (FMi)}
%
% \ProvidesFile{theorem.dtx}[\FMithmInfo]
%\iffalse   % this is a METACOMMENT !
%%
%% Package `theorem' to use with LaTeX2e
%% Copyright (C) 1989-1995 Frank Mittelbach, all rights reserved.
%<+package>\@ifundefined{theorem@style}{}{\endinput}
%<+package>\NeedsTeXFormat{LaTeX2e}[1995/06/01]
%<+package>\ProvidesPackage{theorem}
%<+package>         [\FMithmInfo]
%
% \fi
%
% \changes{v1.0}{87/05/07}{First draft.}
% \changes{v1.0a}{87/05/07}{`thistheoremstyle' added.}
% \changes{v1.1a}{87/07/29}{New concept: redefine the internal
%                          \protect\LaTeX{} commands `@xnthm' etc.}
% \changes{v1.1b}{87/07/31}{`theoremfont' used to specify font.}
% \changes{v1.2a}{88/01/08}{`@endtheorem' now globally defined.}
% \changes{v1.2b}{88/01/11}{`newtheorem' only allowed in the preamble.}
% \changes{v1.3a}{88/02/14}{`theoremfont' now a toks register.}
%
% \changes{v2.0a}{88/11/16}{New implementation and new documentation.}
% \changes{v2.0a}{88/11/16}{No layout changes but internal modifications
%                           old substyles won't work.}
% \changes{v2.0b}{89/01/30}{`global' used for all definitions.}
% \changes{v2.0c}{89/06/15}{Documentation translated to English.
%                            Thanks to LLL and Nina}
% \changes{v2.1a}{90/02/15}{Support for new font selection.}
% \changes{v2.2a}{94/02/03}{Cleanup for LaTeX2e}
%
% \DoNotIndex{\@endparenv,\@ifdefinable,\@ifnextchar,\@ifundefined}
% \DoNotIndex{\@namedef,\@nameuse,\@spaces,\@tempa,\@warning}
% \DoNotIndex{\ }
% \DoNotIndex{\begingroup,\bfseries}
% \DoNotIndex{\csname}
% \DoNotIndex{\def,\do,\docdate}
% \DoNotIndex{\endcsname,\endgroup,\endinput,\endtrivlist,\expandafter}
% \DoNotIndex{\filedate,\fileversion}
% \DoNotIndex{\gdef,\global}
% \DoNotIndex{\hbox,\hskip}
% \DoNotIndex{\ignorespaces,\input,\itshape,\item}
% \DoNotIndex{\labelsep,\let,\llap}
% \DoNotIndex{\makeatletter}
% \DoNotIndex{\noexpand}
% \DoNotIndex{\refstepcounter,\relax,\rlap}
% \DoNotIndex{\setlength,\slshape,\space,\string,\strut}
% \DoNotIndex{\the,\trivlist,\typeout}
% \DoNotIndex{\vbox}
% \DoNotIndex{\xdef}
%
% \def\deflabel#1{\ttfamily #1\hfill}
% \def\deflist#1{\begin{list}{}{\settowidth\labelwidth{\ttfamily #1}%
%                               \setlength\leftmargin\labelwidth
%                               \addtolength\leftmargin\labelsep
%                               \let\makelabel\deflabel}}
% \def\enddeflist{\end{list}}
% \def\envfont{\normalfont\ttfamily}
%
% \MakeShortVerb{\|}
%
% \newcommand{\TUB}{{\sl TUGboat\/}}
%
% \GetFileInfo{theorem.dtx}
% \title{An Extension of the
%        \LaTeX{} theorem environment\thanks{This file
%        has version number \fileversion, last
%        revised \filedate.}}
% \author{Frank Mittelbach}
%
% \MaintainedByLaTeXTeam{tools}
% \maketitle
%
% \begin{abstract}
%    The macros described in this paper yield an extension of the
%    \LaTeX{} theorem mechanism. It is designed to satisfy the
%    different requirements of various journals. Thus, the
%    layout of the ``theorems'' can be manipulated by determining a
%    ``style''. This article describes not only the use, but
%    also the definition, of the necessary macros.
% \end{abstract}
%
% \section*{Preface to version 2.2}
%
% For \LaTeXe{} this package did not need any fundamental changes. I
% only modified the messages generated so that theorem layout styles
% will show up with the |\listfiles| command and cleaned the section
% on the New Font Selection Scheme since this is now included in
% \LaTeX.
%
% \section*{Preface to version 2.1}
%
% This version is identical to 2.0g described in \TUB\ 10\#3 except
% for some internal defaults which are now set depending on the used
% font selection scheme.
%
% This was done to avoid unpleasant surprises if the new font selection
% scheme is in force. For further details see
% section~\ref{sec:newfonts} and \cite{bk:GMS94}.
%
% \section{Introduction}
%
% For our purposes here, ``theorems'' are labelled enunciations,
% often set off from the main text by extra space and a font change.
% Theorems, corollaries, conjectures, definitions, and remarks
% are all instances of ``theorems''.  The ``header'' of these structures
% is composed of a label (such as \textsc{Theorem} or \textsc{Remark})
% and a number which serializes an item in the sequence of items
% with the same label.
%
% Shortly after the introduction of \LaTeX{} at the Fachbereich
% Mathematik in Mainz, the desire to manipulate the layout of
% ``theorems'' arose.  In Mainz, the following two conventions came
% into general use:
% \begin{enumerate}
%   \item The number of the theorem is shown in the margin.
%   \item There is a line break at
%         the end of the theorem header.
% \end{enumerate}
%
% Additionally, some journals require different formats which depend on
% the ``sort of theorem'': e.g.\ often remarks and definitions are set
% in |\upshape|, while |\itshape| is employed for main theorems.
%
% Confronted with these requirements, a theorem environment was
% developed in Mainz which allows separate determination of the
% layout of the ``theorems sets'', comparable to |\pagestyle|.
%
%
% \section{The user interface}
% \subsection{Defining new theorem sets}
%
% \DescribeMacro\newtheorem
% As in the original \LaTeX{} version, the command |\newtheorem|
% defines a new ``theorem set'' or ``theorem-like structure''.
% Two required arguments name the new environment and give the text
% to be typeset with each instance of the new ``set'', while
% an optional argument determines how
% the ``set'' is enumerated:
% \begin{description}
%    \item[\ttfamily \bslash newtheorem\{foo\}\{bar\}]
%       The theorem set {\envfont foo} (whose name is \texttt{bar})
%       uses its own counter.
%    \item[\ttfamily \bslash newtheorem\{foo2\}{[foo]}\{bar2\}]
%       The theorem set {\envfont foo2} (printed name \texttt{bar2})
%       uses the same counter as the theorem set \texttt{foo}.
%    \item[\ttfamily \bslash newtheorem\{foo3\}\{bar3\}{[section]}]
%       The theorem set {\envfont foo3} (printed name \texttt{bar3}) is
%       enumerated within the counter \texttt{section}, i.e.\ with every
%       new |\section| the enumeration begins again with 1, and
%       the enumeration is composed from the section-number and the
%       theorem counter itself.
% \end{description}
%
% \DescribeMacro\theoremstyle
% Additionally, the command |\theoremstyle| can define the layout
% of various, or all, theorem sets.  It should be noted that any
% theorem set defined by |\newtheorem| is typeset in the
% |\theoremstyle| that is current at the time of the definition.
% Thus, the following
% \begin{verbatim}
%    \theoremstyle{break}        \newtheorem{Cor}{Corollary}
%    \theoremstyle{plain}        \newtheorem{Exa}{Example}[section]
%\end{verbatim}                             ^^A No blank at line start
% leads to the result that the set {\envfont Cor} is formatted in the
% style \texttt{break}, while the set {\envfont Exa} and all the
% following ones are formatted in the style \texttt{plain}, unless
% another |\theoremstyle| follows.  Since the definitions installed by
% |\newtheorem| are global, one also can limit |\theoremstyle| locally
% by grouping braces.
%
% \DescribeMacro\theorembodyfont
% The choice of the font for the theorem body is completely
% independent of the chosen |\theoremstyle|; this has proven to
% be very advantageous.  For example,
% \begin{verbatim}
%   {\theorembodyfont{\upshape}          \newtheorem{Rem}{Remark}}
%\end{verbatim}
% defines a theorem set {\envfont Rem}, which will be set in
% |\upshape| in the current layout (which in our example is
% \texttt{plain}).  As with |\theoremstyle|, the |\theorembodyfont|
% chosen is that current at the time of |\newtheorem|.  If
% |\theorembodyfont| is not specified or one defines
% |\theorembodyfont{}|, then the font used will be that defined by the
% |\theoremstyle|.
%
% \DescribeMacro\theoremheaderfont It is also possible to customize
% the font used for the theorem headers.  This is, however, a global
% declaration, and therefore there should be at most one
% |\theoremheaderfont| declaration in the preamble.\footnote{If it is
% actually necessary to have different header fonts, one has to define
% new theorem styles (substituting the desired font) or specify the
% information directly in the \texttt{\bslash newtheorem} declaration
% (the unclean variant).}
%
% \DescribeMacro\theorempreskipamount
% \DescribeMacro\theorempostskipamount Two additional parameters
% affect the vertical space around the theorem environments:
% |\theorempreskipamount| and |\theorempostskipamount| define,
% respectively, the spacing before and after such an environment.
% These parameters apply for all theorem sets and can be manipulated
% with the ordinary length macros.  They are rubber lengths
% (`\textsf{skips}'), and therefore can contain \texttt{plus} and
% \texttt{minus} parts.
%
% Since the definition of theorem sets should---most sensibly---be
% placed in the preamble, we only allow installation there.
% It is therefore possible to release the memory used here after
% |\begin{document}|, in order to make room for other
% applications.
%
%
% \subsection{Existing theorem styles}
%
% The following theorem styles exist to date:
% \begin{deflist}{marginbreak}
%    \item[plain]
%       This theorem style emulates the original \LaTeX{} definition,
%       except that additionally the parameters
%       |\theorem...skipamount| are used.
%    \item[break]
%       In this style, the theorem header is followed by a line
%       break.
%    \item[marginbreak]
%       The theorem number is set in the margin, and there is a line
%       break as in \texttt{break}.
%    \item[changebreak]
%       Like \texttt{break}, but with header number and text
%       interchanged.
%    \item[change]
%       Header number and text are interchanged, without a line break.
%    \item[margin]
%       The number is set in the left margin, without a line break.
% \end{deflist}
% All styles (except \texttt{plain}) select |\slshape| as the default
% |\theorembodyfont|.
%
%
% \subsection{Examples}
%
% Given the above theorem sets {\envfont Cor}, {\envfont Exa}
% and {\envfont Rem},
% suppose that the preamble also contains the declarations:
% \begin{verbatim}
%    \theoremstyle{marginbreak}   \newtheorem{Lem}[Cor]{Lemma}
%    \theoremstyle{change}
%    \theorembodyfont{\itshape}        \newtheorem{Def}[Cor]{Definition}
%
%    \theoremheaderfont{\scshape}
%\end{verbatim}
% Then the following are some typical examples of the typeset output
% resulting from their use.
%
% \begin{Cor}
%    This is a sentence typeset in the theorem environment {\envfont
%    Cor}.
% \end{Cor}
%
% \begin{Exa}
%    This is a sentence typeset in the theorem environment {\envfont
%    Exa}.
% \end{Exa}
%
% \begin{Rem}
%    This is a sentence typeset in the theorem environment {\envfont
%    Rem}.
% \end{Rem}
%
% \begin{Lem}[Ben User]
%    This is a sentence typeset in the theorem environment {\envfont
%    Lem}.
% \end{Lem}
%
% \begin{Def}[Very impressive Definition]
%    This is a sentence typeset in the theorem environment {\envfont
%    Def}.
% \end{Def}
%
% The last two examples show the effect of the optional argument to
% a theorem environment (it is the text typeset in parentheses).
%
%
% \section{Special Considerations}
% \label{sec:newfonts}
%
% Theoremheader and body are implemented as a unit. This means that
% the |\theoremheaderfont|  will inherit characteristics of the
% |\theorembodyfont| in \LaTeXe. Thus, if for example
% |\theorembodyfont| is |\itshape| and |\theoremheaderfont|
% is |\bfseries| the font selected for the header will have the
% characteristics `bold extended italic'. If this is not desired one
% should set the |\theoremheaderfont| to something like
% \begin{verbatim}
%      \theoremheaderfont{\normalfont\bfseries}
%\end{verbatim}
% i.e.\ supplying all necessary font information explicitly.
%
%
%
% \section{Acknowledgements}
%
% The publication of this set of macros was only possible with the
% help of Christina Busse (translating the manuscript into English),
% Joachim Pense (playing the r\^ole of typist), Chris Rowley
% (looking everything over) and many others providing useful
% suggestions.
%
% \MaybeStop{%
%    \begin{thebibliography}{9}
%       \bibitem{bk:GMS94} \textsc{M.~Goossens}, \textsc{F.~Mittelbach}
%         and \textsc{A.~Samarin}.
%         \newblock The \LaTeX{} Companion.
%         \newblock
%         Addison-Wesley, Reading, Massachusetts, 1994.
%       \bibitem{latex:source} \textsc{Lamport, Leslie.} \newblock
%                               \texttt{latex.tex}, version 2.09, date
%                               Feb. 1990.
%    \end{thebibliography}
%    \PrintIndex
%    \PrintChanges}
%
%
% \section{The documentation driver file}
%
% The next bit of code contains the documentation driver file for
% \TeX{}, i.e., the file that will produce the documentation you are
% currently reading. It will be extracted from this file by the
% \texttt{docstrip} program. Since it is the first code in the file
% one can alternatively process this file directly with \LaTeXe{} to
% obtain the documentation.
%
%    \begin{macrocode}
%<*driver>
\documentclass{ltxdoc}

\usepackage{theorem}

%<+driver>% The next few lines define theorem sets which are used
%<+driver>% in the example section of the documentation.

\theoremstyle{break}       \newtheorem{Cor}{Corollary}
\theoremstyle{plain}       \newtheorem{Exa}{Example}[section]
{\theorembodyfont{\upshape}\newtheorem{Rem}{Remark}}
\theoremstyle{marginbreak} \newtheorem{Lem}[Cor]{Lemma}
\theoremstyle{change}
\theorembodyfont{\itshape} \newtheorem{Def}[Cor]{Definition}

\theoremheaderfont{\scshape}

\RecordChanges

\begin{document}
   \DocInput{theorem.dtx}
\end{document}
%</driver>
%    \end{macrocode}
%
% \section{Definition of the Macros}
%
% If the file has been loaded before, we abort immediately. If not the
% package announces itself (this is actually done at the very top of
% the file---the way it is done isn't good style so don't copy it).
% \changes{v2.0e}{89/07/19}{Spaces removed from `typeout'.}
% \changes{v2.2b}{95/11/19}{Announce moved to top of file}
%    \begin{macrocode}
%<*package>
%\@ifundefined{theorem@style}{}{\endinput}
%\def\FMithmInfo{1995/11/19 v2.2b Theorem extension package (FMi)}
%\ProvidesPackage{theorem}[\FMithmInfo]
%    \end{macrocode}
%
%
% \subsection{Definition of theorem styles and fonts}
%
% All the definitions in this file are done globally to allow
% inputting this file inside a group.
%
% \begin{macro}{\theoremstyle}
%    Before a theorem style can be installed, the chosen style must be
%    known. For that reason, we must test to see that
%    |\th@|\meta{style} is known or, more precisely, that it is
%    different from |\relax|. If the style is not known then
%    |\th@plain| is used.
%    \begin{macrocode}
\gdef\theoremstyle#1{%
   \@ifundefined{th@#1}{\@warning
          {Unknown theoremstyle `#1'. Using `plain'}%
          \theorem@style{plain}}%
%    \end{macrocode}
%    We save the theorem style to be used in the token register
%    |\theorem@style|.
%    \begin{macrocode}
      {\theorem@style{#1}}%
%    \end{macrocode}
%    Now we ``evaluate'' the theorem style: this means, we call the
%    macro |\th@|\meta{style} which will activate the relevant
%    definitions which are contained in a separate file.  This is done
%    in a group to suppress changes to the current font.  This could
%    otherwise pose problems together with the new font selection
%    scheme\footnote{When I printed the original article using the new
%    font selection scheme I ended with a document with slanted
%    typefaces (text headings and all) simply because one of the
%    theorem styles used \texttt{\bslash sl} at toplevel.} if the
%    |\th@|\meta{style} is evaluated a second time.
% \changes{v2.1a}{90/02/15}{Group added around `th@..}
% \changes{v2.1a}{90/02/15}{Reseting of `..@begintheorem removed.}
%    \begin{macrocode}
      \begingroup
        \csname th@\the\theorem@style \endcsname
      \endgroup}
%    \end{macrocode}
% \end{macro}
%
%
% \begin{macro}{\@begintheorem}
% \changes{v2.1a}{90/02/15}{Relaxed at toplevel.}
% \begin{macro}{\@opargbegintheorem}
% \changes{v2.1a}{90/02/15}{Relaxed at toplevel.}
%    We reset |\@begintheorem| and |\@opargbegintheorem| to
%    |\relax| since these commands are no longer necessary at
%    toplevel. This will save a few tokens.
%    \begin{macrocode}
\global\let\@begintheorem\relax
\global\let\@opargbegintheorem\relax
%    \end{macrocode}
% \end{macro}
% \end{macro}
%
% \begin{macro}{\theorem@style}
%    Obviously the \textsf{token} register used above has to be
%    allocated. To assure the utmost compatibility with the
%    original \LaTeX{} definition, we set the default theorem style to
%    \texttt{plain}, which implements the usual \LaTeX{} convention.
% \changes{v2.0f}{89/07/20}{Unnecessary `global' removed.}
%    \begin{macrocode}
\newtoks\theorem@style
\global\theorem@style{plain}
%    \end{macrocode}
% \end{macro}
%
% \begin{macro}{\theorembodyfont}
% \begin{macro}{\theorem@bodyfont}
%    If the |\theorembodyfont| is set by the user then it should not
%    interact with the default font set in the theorem style. When the
%    new font selection is in force this may happen if, for example,
%    the default is |\itshape| and the new |\theorembodyfont| is
%    |\sffamily|. So we add a |\reset@font| command in front of the
%    user definition.
% \changes{v2.0f}{89/07/20}{Unnecessary `global' removed.}
% \changes{v2.0g}{89/09/19}{`theoremfont to `theorembodyfont.}
% \changes{v2.1c}{92/06/22}{`reset@font added and logic changed.}
% \changes{v2.1d}{92/10/28}{Reset to empty if arg is empty}
%    \begin{macrocode}
\gdef\theorembodyfont#1{%
%    \end{macrocode}
%    We check if the argument supplied is empty and if so put nothing
%    into the |\theorem@bodyfont| token register to allow for
%    |\theorembodyfont{}| as a mean of using the default of the
%    current |\theoremstyle|.
%    \begin{macrocode}
   \def\@tempa{#1}%
   \ifx\@tempa\@empty
    \theorem@bodyfont{}%
   \else
    \theorem@bodyfont{\reset@font#1}%
   \fi
   }
\newtoks\theorem@bodyfont
\global\theorem@bodyfont{}
%    \end{macrocode}
% \end{macro}
% \end{macro}
%
%
% \begin{macro}{\theoremheaderfont}
% \changes{v2.0g}{89/09/19}{Macro added.}
%    The font for the theorem headers is handled differently because
%    this definition applies to all theorem styles.
%    \begin{macrocode}
\gdef\theoremheaderfont#1{\gdef\theorem@headerfont{#1}%
%    \end{macrocode}
%    After using the macro once it is redefined to produce an error
%    message.
%    \begin{macrocode}
       \gdef\theoremheaderfont##1{%
        \typeout{\string\theoremheaderfont\space should be used
                 only once.}}}
%    \end{macrocode}
% \end{macro}
%
% \begin{macro}{\theorem@headerfont}
% \changes{v2.0g}{89/09/19}{Macro added.}
%    To set the |\theorem@headerfont| default we first test if
%    the new fontselection scheme is in force.
% \changes{v2.1a}{90/02/15}{Support for new font selection.}
%    \begin{macrocode}
\ifx\upshape\undefined
%    \end{macrocode}
%    If not we define it to expand into |\bfseries|. We don't use
%    |\let| just in case a following style option redefines this
%    macro.
%    \begin{macrocode}
\gdef\theorem@headerfont{\bfseries}
%    \end{macrocode}
%    Otherwise we reset the current shape before calling |\bfseries|.
%    \begin{macrocode}
\else \gdef\theorem@headerfont{\normalfont\bfseries}\fi
%    \end{macrocode}
% \end{macro}
%
% \begin{macro}{\th@plain}
% \begin{macro}{\th@break}
% \begin{macro}{\th@marginbreak}
% \begin{macro}{\th@changebreak}
% \begin{macro}{\th@change}
% \begin{macro}{\th@margin}
%    The different styles are defined in macros such as
%    |\th@plain|.  Since memory space is precious in
%    ``non-Big-versions'', we have to avoid offering too many unused
%    definitions.  Therefore we define these styles in separate files
%    that can be loaded on demand.  Thus the commands themselves only
%    load these files. We use |\@input@|, a \LaTeXe{} internal command
%    that ensures that the file will be listed with |\listfiles|.
% \changes{v2.2a}{94/02/02}{Use `@input@ to load theorem layout files}
%    \begin{macrocode}
\gdef\th@plain{\@input@{thp.sty}}
\gdef\th@break{\@input@{thb.sty}}
\gdef\th@marginbreak{\@input@{thmb.sty}}
\gdef\th@changebreak{\@input@{thcb.sty}}
\gdef\th@change{\@input@{thc.sty}}
\gdef\th@margin{\@input@{thm.sty}}
%    \end{macrocode}
%    This list will be expanded when new styles become
%    available.  For testing,  just append new theorem substyles
%    as document options.
% \end{macro}
% \end{macro}
% \end{macro}
% \end{macro}
% \end{macro}
% \end{macro}
%
%
%
% \subsection{Definition of a new theorem set}
%
% As already pointed out, a new theorem environment can be defined in
% three different ways:
% \begin{verbatim}
%    \newtheorem{Lem}{Lemma}
%    \newtheorem{Lem}{Lemma}[section]
%    \newtheorem{Lem}[Theorem]{Lemma}
%\end{verbatim}
% The function of the macro |\newtheorem| is to recognize these cases
% and then to branch into one of the three macros |\@ynthm|, |\@xnthm|
% or |\@othm|.  This mechanism is adopted unchanged from
% \cite{latex:source}; the essential point here is that, for example,
% in the second case, the arguments \texttt{Lem}, \texttt{Lemma} and
% \texttt{section} are passed over to the macro |\@xnthm|.
%
% We inspect this case first because the others present fewer problems,
% and thus are easily derived from this one.
%
% \begin{macro}{\@xnthm}
%    For our example arguments, the macro |\@xnthm| must fulfill
%    the following:
%    \begin{itemize}
%       \item
%          Define a new \LaTeX-counter `Lem'
%       \item
%          reset this counter within a |\section|
%       \item
%          define the macro |\theLem|
%       \item
%          define the environment macros |\Lem| and
%          |\endLem| using the current
%          |\theoremstyle| and |\theorem@bodyfont|.
%    \end{itemize}
%    Obviously, all this should happen only if the first argument
%    of |\@xnthm| (i.e.\ \texttt{Lem} in our example) is chosen so
%    as not to conflict with any previously defined commands or
%    environments.  This test is performed by the \LaTeX{} macro
%    |\@ifdefinable|.
%    \begin{macrocode}
\gdef\@xnthm#1#2[#3]{\expandafter\@ifdefinable\csname #1\endcsname
%    \end{macrocode}
%    Therefore, the first argument of |\@ifdefinable| is the
%    expansion (in the example, |\Lem|) of
%    |\csname||#1\endcsname|.  The second argument is
%    executed only if the test has been completed successfully.
%    \begin{macrocode}
   {%
%    \end{macrocode}
%    Now we define the new counter. The names of the \LaTeX{}
%    macros employed should speak for themselves:
% \changes{v2.2b}{1995/10/07}{Use \cs{@newctr}}
% \changes{v2.2c}{1995/11/23}{Use \cs{@newctr} properly.}
%    \begin{macrocode}
    \@definecounter{#1}\@newctr{#1}[#3]%
%    \end{macrocode}
%    Using |\@newctr| will give a proper error message if the counter
%    in |#3| is not defined.
%    In defining `|\theLem|' we must generate the desired macro
%    name by use of |\expandafter| and |\csname|.
%    \begin{macrocode}
    \expandafter\xdef\csname the#1\endcsname
%    \end{macrocode}
%    An |\xdef| is used in order to make the definition global,
%    and to ensure that it contains the replacement texts of
%    |\@thmcountersep| and |\@thmcounter|.\footnote{These
%    two macros can be defined by the document style.  Their default
%    values produce a `\texttt{.}' as separation and an arabic
%    representation of the number.} However, not everything should be
%    expanded. For example, it saves space to use |\thesection|
%    instead of its---at times---lengthy expansion.
%    \begin{macrocode}
      {\expandafter \noexpand \csname the#3\endcsname
       \@thmcountersep \@thmcounter{#1}}%
%    \end{macrocode}
%    Thus with the defaults of \LaTeX{}, |\theLem| would be
%    replaced by the command sequence |\thesection.\arabic{Lem}|.
%
%    We will now look at the definition of the macro which is executed
%    at the beginning of the actual environment (in our example this
%    macro is |\Lem|). It should be noted that we use an
%    ``|\expandafter| trick'' to expand only certain parts of the
%    replacement text at the time of the definition.
%    \begin{macrocode}
    \def\@tempa{\global\@namedef{#1}}%
    \expandafter \@tempa \expandafter{%
%    \end{macrocode}
%    First, the macro that contains the current definitions of
%    |\@begintheorem| and |\@opargtheorem| should be called
%    up.  The name of this macro---as is already known---has the form
%    |\th@|\meta{theorem\ style}; therefore, it must be called by
%    \begin{macrocode}
      \csname th@\the \theorem@style
%    \end{macrocode}
%    In addition the default theorem font should be changeable, i.e.\
%    we have to insert the contents of |\theorem@bodyfont|. For
%    that reason, we expand even further, beyond |\endcsname|,
%    and thus insert the contents of the \textsf{token} register
%    |\theorem@bodyfont| in the replacement text.
%    \begin{macrocode}
            \expandafter \endcsname \the \theorem@bodyfont
%    \end{macrocode}
%    Now it is time to call the macro |\@thm| which takes over
%    the further processing.  It has two arguments: the current
%    counter name (in our example, \texttt{Lem}), and the text of the
%    label (in our example, \texttt{Lemma}).
%    \begin{macrocode}
     \@thm{#1}{#2}}%
%    \end{macrocode}
%    With this, the `sub-definition' is complete.  The macro
%    |\@endtheorem| ends a theorem environment and is, so far, nothing
%    but an |\endtrivlist|.  (Hence it is defined globally, and not
%    within the theorem styles.\footnote{This has to be changed as
%    soon as theorem styles that change \texttt{\bslash @endtheorem}
%    exist.  In such a case, all existing styles must be changed as
%    well since they will have to reset the macro.}) Therefore, we can
%    set it equivalent to the macro that ends the theorem set (in our
%    example, |\endLem|).  However, if some day theorem styles exist
%    that do change |\@endtheorem|, we would have to use the
%    commented-out line instead.
%    \begin{macrocode}
    \global \expandafter \let \csname end#1\endcsname \@endtheorem
%   \global\@namedef{end#1}{\@endtheorem}%
%    \end{macrocode}
%    With these commands all the required definitions are employed,
%    unless the test |\@ifdefinable| has failed.  Therefore, we
%    end the second argument of this macro and with it the definition
%    of |\@xnthm|.
%    \begin{macrocode}
   }}
%    \end{macrocode}
% \end{macro}
%
% \begin{macro}{\@ynthm}
%    The definition of |\@ynthm| is completely analogous. In
%    this case the new counter that is defined is not reset within
%    another counter; thus the definition of |\the...|
%    is simplified:
%    \begin{macrocode}
\gdef\@ynthm#1#2{\expandafter\@ifdefinable\csname #1\endcsname
   {\@definecounter{#1}%
    \expandafter\xdef\csname the#1\endcsname{\@thmcounter{#1}}%
%    \end{macrocode}
%    The rest of the definition corresponds literally to that of
%    |\@xnthm|:
%    \begin{macrocode}
    \def\@tempa{\global\@namedef{#1}}\expandafter \@tempa
     \expandafter{\csname th@\the \theorem@style \expandafter
     \endcsname \the\theorem@bodyfont \@thm{#1}{#2}}%
    \global \expandafter \let \csname end#1\endcsname \@endtheorem}}
%    \end{macrocode}
% \end{macro}
%
% \begin{macro}{\@othm}
%    The definition of |\@othm| does not contain anything new.
%    \begin{macrocode}
\gdef\@othm#1[#2]#3{%
%    \end{macrocode}
% \changes{v2.2b}{95/10/07}{Build in check for counter defined}
%    We do not define a new counter but instead use one that has
%    already been defined.  Thus the only definition we need is that
%    of this pseudo-counter (i.e.\ |\the|\meta{env.\ name}).
%    First we check if |#2| corresponds to a known counter name.
%    \begin{macrocode}
  \expandafter\ifx\csname c@#2\endcsname\relax
   \@nocounterr{#2}%
  \else
   \expandafter\@ifdefinable\csname #1\endcsname
   {\expandafter \xdef \csname the#1\endcsname
     {\expandafter \noexpand \csname the#2\endcsname}%
%    \end{macrocode}
%    All other parts of the definition can be adopted from
%    |\@xnthm|. We have to remember, though, that in this case
%    the name of the current counter and the theorem label have moved
%    to the second and third arguments.
%    \begin{macrocode}
    \def\@tempa{\global\@namedef{#1}}\expandafter \@tempa
     \expandafter{\csname th@\the \theorem@style \expandafter
     \endcsname \the\theorem@bodyfont \@thm{#2}{#3}}%
    \global \expandafter \let \csname end#1\endcsname \@endtheorem}%
  \fi}
%    \end{macrocode}
% \end{macro}
%
% \subsection{Macros that are employed in a {\envfont theorem}
%    environment}
%
% \begin{macro}{\@thm}
%    The macro |\@thm| has to increase the current counter.  Then,
%    depending on whether the environment has (or does not have) an
%    optional argument, it has to branch into either |\@begintheorem|
%    or |\@opargtheorem|.
%    \begin{macrocode}
\gdef\@thm#1#2{\refstepcounter{#1}%
%    \end{macrocode}
%    Now we start a {\envfont trivlist} environment, and give
%    |\@topsep| and |\@topsepadd| the values of the \textsf{skip}
%    registers |\theorempreskipamount| and |\theorempostskipamount|.
%    The value in |\@topsep| is the vertical space that is inserted by
%    the first (and only) |\item| in our |\trivlist| whilst
%    |\@topsepadd| is inserted by |\@endparenv| at the end of that
%    {\envfont trivlist} environment.  By using these registers, we
%    obtain the desired space around a {\envfont theorem} environment.
%    \begin{macrocode}
   \trivlist
   \@topsep \theorempreskipamount               % used by first \item
   \@topsepadd \theorempostskipamount           % used by \@endparenv
%    \end{macrocode}
%    Now we have to test whether an optional argument has been given.
% \changes{v2.0d}{89/07/18}{Forgotten percent added.}
%    \begin{macrocode}
   \@ifnextchar [%
%    \end{macrocode}
%    If there is an optional argument, we will call |\@ythm|, and
%    move the arguments read back into the input stream.
%    \begin{macrocode}
   {\@ythm{#1}{#2}}%
%    \end{macrocode}
%    If not, we call |\@begintheorem|. Its first argument is the
%    name of the theorem set (hence the second argument of
%    |\@thm|). Its second argument is the macro that produces the
%    current number.
%    \begin{macrocode}
   {\@begintheorem{#2}{\csname the#1\endcsname}\ignorespaces}}
%    \end{macrocode}
% \end{macro}
%
% \begin{macro}{\@xthm}
% \begin{macro}{\@ythm}
%    Both these macros were originally called by |\@thm|.  We do
%    not need |\@xthm| anymore, hence we reset it to
%    |\relax|. The definition of |\@ythm| has not changed at
%    all from its definition in \LaTeX. In order to make the macros
%    easier to understand, we will
%    nevertheless present it (commented out).
%    \begin{macrocode}
\global\let\@xthm\relax
% \def\@ythm#1#2[#3]{\@opargbegintheorem{#2}{\csname
%           the#1\endcsname}{#3}\ignorespaces}
%</package>
%    \end{macrocode}
%    The primitive |\ignorespaces| in |\@ythm| and
%    |\@thm| is needed to remove the spaces between the
%    |\begin{...}| and the actual text.
% \end{macro}
% \end{macro}
%
%
% \subsection{Definition of the theorem substyles}
%
% As already pointed out, the theorem substyles, defined below, are
% only loaded when necessary.  Note that all these substyles, except
% \texttt{plain}, have |\slshape| as the default body font.
%
% \subsubsection{The \texttt{plain} style}
%
% \label{dok:th@plain} As the following macros use |@|, we have
% to locally set the |\catcode| of this symbol to ``letter''.
% This happens within a group, so that we do not have to worry about
% which |\catcode| that symbol had before.
%    \begin{macrocode}
%<*thp>
\begingroup \makeatletter
%    \end{macrocode}
% Since we are now within a group, we must make all definitions
% globally.  First we make sure that \texttt{theorem.sty} is loaded.
% This will allow us to use this file as a document style option
% without having to call \texttt{theorem} itself as an option.  At the
% same time, we assure that at least version 2 is loaded, since
% |\theorem@style| was not defined in earlier versions.
%    \begin{macrocode}
\@ifundefined{theorem@style}{\input{theorem.sty}}{}
\ProvidesFile{thp.sty}
             [\FMithmInfo]
%    \end{macrocode}
%
%
% \begin{macro}{\th@plain}
% \begin{macro}{\@begintheorem}
% \begin{macro}{\@opargbegintheorem}
%    |\theoremstyle{plain}| corresponds to the original
%    definition, except that the distances to the surrounding text are
%    determined by the parameters |\theorempreskipamount| and
%    |\theorempostskipamount|.
%    First we set the default body font.
%    \begin{macrocode}
\gdef\th@plain{\normalfont\itshape
%    \end{macrocode}
%    Then we define |\@begintheorem| and
%    |\@opargbegintheorem|. These two macros define how the
%    header of a theorem is typeset.  |\@opargbegintheorem| will
%    be called if a \textsf{theorem} environment with an optional
%    argument is encountered; otherwise, the header is constructed by
%    calling |\@begintheorem|.  If one of these macros is
%    executed, we are within a \textsf{trivlist} environment started by
%    |\@thm|.  So the theorem header is produced with an
%    |\item| command.
%
%    Instead of specifying the header font directly, all standard
%    theorem styles use the |\theorem@headerfont| macro to allow
%    customization.  The extra space (|\labelsep|) is necessary
%    because of problems in the \textsf{trivlist} environment.
%    \begin{macrocode}
  \def\@begintheorem##1##2{%
        \item[\hskip\labelsep \theorem@headerfont ##1\ ##2]}%
%    \end{macrocode}
%    The definition of |\@opargbegintheorem| is completely
%    analogous. The only difference is the fact that there exists a
%    third argument (which is the optional parameter of the
%    environment and contains additional information about the
%    theorem).  Customarily we enclose it in parentheses.
%    \begin{macrocode}
\def\@opargbegintheorem##1##2##3{%
   \item[\hskip\labelsep \theorem@headerfont ##1\ ##2\ (##3)]}}
%    \end{macrocode}
% \end{macro}
% \end{macro}
% \end{macro}
%
% \noindent
% We conclude with an |\endgroup| to restore the
% |\catcode| of |@|.
%    \begin{macrocode}
\endgroup
%</thp>
%    \end{macrocode}
%
%
% \subsubsection{The \texttt{break} style}
%
% This style option is stored in the file |thb.sty|.
% For the next two lines see the documentation for |\th@plain| on
% page~\pageref{dok:th@plain}.
%    \begin{macrocode}
%<*thb>
\begingroup \makeatletter
\@ifundefined{theorem@style}{\input{theorem.sty}}{}
\ProvidesFile{thb.sty}
             [\FMithmInfo]
%    \end{macrocode}
%
% \begin{macro}{\th@break}
%    |\theoremstyle{break}| produces a line break after the name
%    of the theorem.  The font is |\slshape|.  Hence, we define
%    |\th@break| as follows:
%    \begin{macrocode}
\gdef\th@break{\normalfont\slshape
  \def\@begintheorem##1##2{\item[%
%    \end{macrocode}
%    We run into the following problem: it is not possible to create
%    the header with |\item[|\meta{title}|]| and then start
%    a new line by, for example, |\mbox{}\\|. Such a definition
%    will fail whenever a list environment follows immediately.  With
%    the above construction, the |\mbox{}| causes the switch
%    |@inlabel| (cf.\ definition of |\list| and
%    |\trivlist| in \cite{latex:source}) to be set to \texttt{false}
%    and so the following list will insert additional vertical
%    space (|\topskip|). This is quite annoying. Therefore, we
%    create the line break within the |\item|. In order to ensure
%    that the text will begin at the proper position in the following
%    line, we simply pretend that the label does not take any
%    room.\footnote{This will lead to problems whenever very high
%    symbols occurring in the line tower into the
%    heading. So, something else has to be done here sometime.}
%    \begin{macrocode}
       \rlap{\vbox{\hbox{\hskip \labelsep\theorem@headerfont ##1\ ##2}%
                   \hbox{\strut}}}]}%
%    \end{macrocode}
%    Again, the definition of |\@opargbegintheorem| is
%    completely analogous.
%    \begin{macrocode}
\def\@opargbegintheorem##1##2##3{%
  \item[\rlap{\vbox{\hbox{\hskip \labelsep \theorem@headerfont
                     ##1\ ##2\ (##3)}%
                    \hbox{\strut}}}]}}
\endgroup
%</thb>
%    \end{macrocode}
% \end{macro}
%
%
% \subsubsection{The \texttt{changebreak} style}
%
%    \begin{macrocode}
% This style option is stored in the file |thcb.sty|.
%    \begin{macrocode}
%<*thcb>
\begingroup \makeatletter
\@ifundefined{theorem@style}{\input{theorem.sty}}{}
\ProvidesFile{thcb.sty}
             [\FMithmInfo]
%    \end{macrocode}
%
% \begin{macro}{\th@changebreak}
%    The \texttt{change-break} theorem style is like \texttt{break}
%    but with interchange of theorem name and theorem number.  Thus we
%    define |\th@changebreak| as follows:
%    \begin{macrocode}
\gdef\th@changebreak{\normalfont\slshape
  \def\@begintheorem##1##2{\item
       [\rlap{\vbox{\hbox{\hskip\labelsep \theorem@headerfont ##2\ ##1}%
                    \hbox{\strut}}}]}%
\def\@opargbegintheorem##1##2##3{%
  \item[\rlap{\vbox{\hbox{\hskip\labelsep \theorem@headerfont
                     ##2\ ##1\ (##3)}%
                    \hbox{\strut}}}]}}
\endgroup
%</thcb>
%    \end{macrocode}
% \end{macro}
%
%
% \subsubsection{The \texttt{change} style}
%
% This style option is stored in the file |thc.sty|.
%    \begin{macrocode}
%<*thc>
\begingroup \makeatletter
\@ifundefined{theorem@style}{\input{theorem.sty}}{}
\ProvidesFile{thc.sty}
             [\FMithmInfo]
%    \end{macrocode}
%
% \begin{macro}{\th@change}
%    The \texttt{change} theorem style corresponds to the
%    \texttt{change break} style without a linebreak after the header.
%    To say it in another way, it's the same as the \texttt{plain}
%    style but with number and name interchanged and |\slshape| as the
%    default font.
%    \begin{macrocode}
\gdef\th@change{\normalfont\slshape
  \def\@begintheorem##1##2{\item
       [\hskip\labelsep \theorem@headerfont ##2\ ##1]}%
\def\@opargbegintheorem##1##2##3{%
  \item[\hskip\labelsep \theorem@headerfont ##2\ ##1\ (##3)]}}
\endgroup
%</thc>
%    \end{macrocode}
% \end{macro}
%
%
% \subsubsection{The \texttt{marginbreak} style}
%
% This style option is the one used most often at Mainz.  It is saved
% in the file |thmb.sty|.
%    \begin{macrocode}
%<*thmb>
\begingroup \makeatletter
\@ifundefined{theorem@style}{\input{theorem.sty}}{}
\ProvidesFile{thmb.sty}
             [\FMithmInfo]
%    \end{macrocode}
%
% \begin{macro}{\th@marginbreak}
%    The \texttt{margin break} style is nearly the same as the
%    \texttt{change break} style. The only difference is the placement
%    of the theorem number.  We use |\llap| to place it in the left
%    margin.
%
%    In this style |\labelsep| denotes the separation between the
%    number and the text.
%    \begin{macrocode}
\gdef\th@marginbreak{\normalfont\slshape
  \def\@begintheorem##1##2{\item
       [\rlap{\vbox{\theorem@headerfont
                    \hbox{\llap{##2}\hskip\labelsep ##1}%
                    \hbox{\strut}}}]}%
\def\@opargbegintheorem##1##2##3{%
  \item[\rlap{\vbox{\theorem@headerfont
                    \hbox{\llap{##2}\hskip\labelsep ##1\ (##3)}%
                    \hbox{\strut}}}]}}
\endgroup
%</thmb>
%    \end{macrocode}
% \changes{v2.1c}{90/02/17}{`endgroup outside macrocode env.}
% \end{macro}
%
%
%
% \subsubsection{The \texttt{margin} style}
%
% This style option is stored in the file |thm.sty|.
%    \begin{macrocode}
%<*thm>
\begingroup \makeatletter
\@ifundefined{theorem@style}{\input{theorem.sty}}{}
\ProvidesFile{thm.sty}
             [\FMithmInfo]
%    \end{macrocode}
%
% \begin{macro}{\th@margin}
%    Again this is only a variant of the theorem styles described
%    above without any new ideas.
%    \begin{macrocode}
\gdef\th@margin{\normalfont\slshape
  \def\@begintheorem##1##2{\item
       [\theorem@headerfont \llap{##2}\hskip\labelsep ##1]}%
\def\@opargbegintheorem##1##2##3{%
  \item[\theorem@headerfont \llap{##2}\hskip\labelsep ##1\ (##3)]}}
\endgroup
%</thm>
%    \end{macrocode}
% \end{macro}
%
%
%
%
% \subsection{Final Definitions}
%
% \begin{macro}{\theorempreskipamount}
% \begin{macro}{\theorempostskipamount}
%    The \textsf{skip} parameters that regulate the vertical empty space
%    before and after the {\envfont theorem} environment have
%    to be allocated as well.
% \changes{v2.0f}{89/07/20}{Unnecessary `global' removed.}
%    \begin{macrocode}
%<*package>
\newskip\theorempreskipamount
\newskip\theorempostskipamount
%    \end{macrocode}
%    Since we have used the same values for all theorem sets, we now can
%    assign them.
%    \begin{macrocode}
\global\setlength\theorempreskipamount{12pt plus 5pt minus 3pt}
\global\setlength\theorempostskipamount{8pt plus 3pt minus 1.5pt}
%    \end{macrocode}
% \end{macro}
% \end{macro}
%
%
% \begin{macro}{\@endtheorem}
%    The same holds for the macro |\@endtheorem|, which ends a
%    {\envfont theorem} environment. Since it is the same for all
%    theorem sets, it is removed from the macros |\th@|\meta{style}.
%    It simply ends the {\envfont trivlist} environment, which was
%    begun in |\@thm|.
%    \begin{macrocode}
\global\let\@endtheorem=\endtrivlist
%    \end{macrocode}
% \end{macro}
%
%
%
% \begin{macro}{\@preamblecmds}
%    All macros defined above are to be used only in the preamble.
%    Therefore, we insert them in |\@preamblecmds| which will disable
%    them at begin document. This is done by the internal \LaTeXe{}
%    command |\@onlypreamble|.
% \changes{v2.0g}{89/09/19}{`theoremheaderfont and `theorembodyfont
%    added.}
% \changes{v2.2a}{94/02/03}{Use \LaTeXe{} interface}
%    \begin{macrocode}
\@onlypreamble\@xnthm
\@onlypreamble\@ynthm
\@onlypreamble\@othm
\@onlypreamble\newtheorem
\@onlypreamble\theoremstyle
\@onlypreamble\theorembodyfont
\@onlypreamble\theoremheaderfont
%    \end{macrocode}
% \end{macro}
%
% Finally we declare the \texttt{plain} theorem style to be the default.
% \changes{v2.1b}{90/06/04}{Fixed missing `item by declaring default
%    style.}
%    \begin{macrocode}
\theoremstyle{plain}
%</package>
%    \end{macrocode}
%
% \Finale

\endinput
}{}
\ProvidesFile{thc.sty}
             [\FMithmInfo]
%    \end{macrocode}
%
% \begin{macro}{\th@change}
%    The \texttt{change} theorem style corresponds to the
%    \texttt{change break} style without a linebreak after the header.
%    To say it in another way, it's the same as the \texttt{plain}
%    style but with number and name interchanged and |\slshape| as the
%    default font.
%    \begin{macrocode}
\gdef\th@change{\normalfont\slshape
  \def\@begintheorem##1##2{\item
       [\hskip\labelsep \theorem@headerfont ##2\ ##1]}%
\def\@opargbegintheorem##1##2##3{%
  \item[\hskip\labelsep \theorem@headerfont ##2\ ##1\ (##3)]}}
\endgroup
%</thc>
%    \end{macrocode}
% \end{macro}
%
%
% \subsubsection{The \texttt{marginbreak} style}
%
% This style option is the one used most often at Mainz.  It is saved
% in the file |thmb.sty|.
%    \begin{macrocode}
%<*thmb>
\begingroup \makeatletter
\@ifundefined{theorem@style}{% \iffalse meta-comment
%
% Copyright (C) 1993-2024
%
% The LaTeX Project and any individual authors listed elsewhere
% in this file.
%
% This file is part of the Standard LaTeX `Tools Bundle'.
% -------------------------------------------------------
%
% It may be distributed and/or modified under the
% conditions of the LaTeX Project Public License, either version 1.3c
% of this license or (at your option) any later version.
% The latest version of this license is in
%    https://www.latex-project.org/lppl.txt
% and version 1.3c or later is part of all distributions of LaTeX
% version 2005/12/01 or later.
%
% The list of all files belonging to the LaTeX `Tools Bundle' is
% given in the file `manifest.txt'.
%
% \fi
%
%%
%
%
\def\FMithmInfo{2023/07/05 v2.2c Theorem extension package (FMi)}
%
% \ProvidesFile{theorem.dtx}[\FMithmInfo]
%\iffalse   % this is a METACOMMENT !
%%
%% Package `theorem' to use with LaTeX2e
%% Copyright (C) 1989-1995 Frank Mittelbach, all rights reserved.
%<+package>\@ifundefined{theorem@style}{}{\endinput}
%<+package>\NeedsTeXFormat{LaTeX2e}[1995/06/01]
%<+package>\ProvidesPackage{theorem}
%<+package>         [\FMithmInfo]
%
% \fi
%
% \changes{v1.0}{87/05/07}{First draft.}
% \changes{v1.0a}{87/05/07}{`thistheoremstyle' added.}
% \changes{v1.1a}{87/07/29}{New concept: redefine the internal
%                          \protect\LaTeX{} commands `@xnthm' etc.}
% \changes{v1.1b}{87/07/31}{`theoremfont' used to specify font.}
% \changes{v1.2a}{88/01/08}{`@endtheorem' now globally defined.}
% \changes{v1.2b}{88/01/11}{`newtheorem' only allowed in the preamble.}
% \changes{v1.3a}{88/02/14}{`theoremfont' now a toks register.}
%
% \changes{v2.0a}{88/11/16}{New implementation and new documentation.}
% \changes{v2.0a}{88/11/16}{No layout changes but internal modifications
%                           old substyles won't work.}
% \changes{v2.0b}{89/01/30}{`global' used for all definitions.}
% \changes{v2.0c}{89/06/15}{Documentation translated to English.
%                            Thanks to LLL and Nina}
% \changes{v2.1a}{90/02/15}{Support for new font selection.}
% \changes{v2.2a}{94/02/03}{Cleanup for LaTeX2e}
%
% \DoNotIndex{\@endparenv,\@ifdefinable,\@ifnextchar,\@ifundefined}
% \DoNotIndex{\@namedef,\@nameuse,\@spaces,\@tempa,\@warning}
% \DoNotIndex{\ }
% \DoNotIndex{\begingroup,\bfseries}
% \DoNotIndex{\csname}
% \DoNotIndex{\def,\do,\docdate}
% \DoNotIndex{\endcsname,\endgroup,\endinput,\endtrivlist,\expandafter}
% \DoNotIndex{\filedate,\fileversion}
% \DoNotIndex{\gdef,\global}
% \DoNotIndex{\hbox,\hskip}
% \DoNotIndex{\ignorespaces,\input,\itshape,\item}
% \DoNotIndex{\labelsep,\let,\llap}
% \DoNotIndex{\makeatletter}
% \DoNotIndex{\noexpand}
% \DoNotIndex{\refstepcounter,\relax,\rlap}
% \DoNotIndex{\setlength,\slshape,\space,\string,\strut}
% \DoNotIndex{\the,\trivlist,\typeout}
% \DoNotIndex{\vbox}
% \DoNotIndex{\xdef}
%
% \def\deflabel#1{\ttfamily #1\hfill}
% \def\deflist#1{\begin{list}{}{\settowidth\labelwidth{\ttfamily #1}%
%                               \setlength\leftmargin\labelwidth
%                               \addtolength\leftmargin\labelsep
%                               \let\makelabel\deflabel}}
% \def\enddeflist{\end{list}}
% \def\envfont{\normalfont\ttfamily}
%
% \MakeShortVerb{\|}
%
% \newcommand{\TUB}{{\sl TUGboat\/}}
%
% \GetFileInfo{theorem.dtx}
% \title{An Extension of the
%        \LaTeX{} theorem environment\thanks{This file
%        has version number \fileversion, last
%        revised \filedate.}}
% \author{Frank Mittelbach}
%
% \MaintainedByLaTeXTeam{tools}
% \maketitle
%
% \begin{abstract}
%    The macros described in this paper yield an extension of the
%    \LaTeX{} theorem mechanism. It is designed to satisfy the
%    different requirements of various journals. Thus, the
%    layout of the ``theorems'' can be manipulated by determining a
%    ``style''. This article describes not only the use, but
%    also the definition, of the necessary macros.
% \end{abstract}
%
% \section*{Preface to version 2.2}
%
% For \LaTeXe{} this package did not need any fundamental changes. I
% only modified the messages generated so that theorem layout styles
% will show up with the |\listfiles| command and cleaned the section
% on the New Font Selection Scheme since this is now included in
% \LaTeX.
%
% \section*{Preface to version 2.1}
%
% This version is identical to 2.0g described in \TUB\ 10\#3 except
% for some internal defaults which are now set depending on the used
% font selection scheme.
%
% This was done to avoid unpleasant surprises if the new font selection
% scheme is in force. For further details see
% section~\ref{sec:newfonts} and \cite{bk:GMS94}.
%
% \section{Introduction}
%
% For our purposes here, ``theorems'' are labelled enunciations,
% often set off from the main text by extra space and a font change.
% Theorems, corollaries, conjectures, definitions, and remarks
% are all instances of ``theorems''.  The ``header'' of these structures
% is composed of a label (such as \textsc{Theorem} or \textsc{Remark})
% and a number which serializes an item in the sequence of items
% with the same label.
%
% Shortly after the introduction of \LaTeX{} at the Fachbereich
% Mathematik in Mainz, the desire to manipulate the layout of
% ``theorems'' arose.  In Mainz, the following two conventions came
% into general use:
% \begin{enumerate}
%   \item The number of the theorem is shown in the margin.
%   \item There is a line break at
%         the end of the theorem header.
% \end{enumerate}
%
% Additionally, some journals require different formats which depend on
% the ``sort of theorem'': e.g.\ often remarks and definitions are set
% in |\upshape|, while |\itshape| is employed for main theorems.
%
% Confronted with these requirements, a theorem environment was
% developed in Mainz which allows separate determination of the
% layout of the ``theorems sets'', comparable to |\pagestyle|.
%
%
% \section{The user interface}
% \subsection{Defining new theorem sets}
%
% \DescribeMacro\newtheorem
% As in the original \LaTeX{} version, the command |\newtheorem|
% defines a new ``theorem set'' or ``theorem-like structure''.
% Two required arguments name the new environment and give the text
% to be typeset with each instance of the new ``set'', while
% an optional argument determines how
% the ``set'' is enumerated:
% \begin{description}
%    \item[\ttfamily \bslash newtheorem\{foo\}\{bar\}]
%       The theorem set {\envfont foo} (whose name is \texttt{bar})
%       uses its own counter.
%    \item[\ttfamily \bslash newtheorem\{foo2\}{[foo]}\{bar2\}]
%       The theorem set {\envfont foo2} (printed name \texttt{bar2})
%       uses the same counter as the theorem set \texttt{foo}.
%    \item[\ttfamily \bslash newtheorem\{foo3\}\{bar3\}{[section]}]
%       The theorem set {\envfont foo3} (printed name \texttt{bar3}) is
%       enumerated within the counter \texttt{section}, i.e.\ with every
%       new |\section| the enumeration begins again with 1, and
%       the enumeration is composed from the section-number and the
%       theorem counter itself.
% \end{description}
%
% \DescribeMacro\theoremstyle
% Additionally, the command |\theoremstyle| can define the layout
% of various, or all, theorem sets.  It should be noted that any
% theorem set defined by |\newtheorem| is typeset in the
% |\theoremstyle| that is current at the time of the definition.
% Thus, the following
% \begin{verbatim}
%    \theoremstyle{break}        \newtheorem{Cor}{Corollary}
%    \theoremstyle{plain}        \newtheorem{Exa}{Example}[section]
%\end{verbatim}                             ^^A No blank at line start
% leads to the result that the set {\envfont Cor} is formatted in the
% style \texttt{break}, while the set {\envfont Exa} and all the
% following ones are formatted in the style \texttt{plain}, unless
% another |\theoremstyle| follows.  Since the definitions installed by
% |\newtheorem| are global, one also can limit |\theoremstyle| locally
% by grouping braces.
%
% \DescribeMacro\theorembodyfont
% The choice of the font for the theorem body is completely
% independent of the chosen |\theoremstyle|; this has proven to
% be very advantageous.  For example,
% \begin{verbatim}
%   {\theorembodyfont{\upshape}          \newtheorem{Rem}{Remark}}
%\end{verbatim}
% defines a theorem set {\envfont Rem}, which will be set in
% |\upshape| in the current layout (which in our example is
% \texttt{plain}).  As with |\theoremstyle|, the |\theorembodyfont|
% chosen is that current at the time of |\newtheorem|.  If
% |\theorembodyfont| is not specified or one defines
% |\theorembodyfont{}|, then the font used will be that defined by the
% |\theoremstyle|.
%
% \DescribeMacro\theoremheaderfont It is also possible to customize
% the font used for the theorem headers.  This is, however, a global
% declaration, and therefore there should be at most one
% |\theoremheaderfont| declaration in the preamble.\footnote{If it is
% actually necessary to have different header fonts, one has to define
% new theorem styles (substituting the desired font) or specify the
% information directly in the \texttt{\bslash newtheorem} declaration
% (the unclean variant).}
%
% \DescribeMacro\theorempreskipamount
% \DescribeMacro\theorempostskipamount Two additional parameters
% affect the vertical space around the theorem environments:
% |\theorempreskipamount| and |\theorempostskipamount| define,
% respectively, the spacing before and after such an environment.
% These parameters apply for all theorem sets and can be manipulated
% with the ordinary length macros.  They are rubber lengths
% (`\textsf{skips}'), and therefore can contain \texttt{plus} and
% \texttt{minus} parts.
%
% Since the definition of theorem sets should---most sensibly---be
% placed in the preamble, we only allow installation there.
% It is therefore possible to release the memory used here after
% |\begin{document}|, in order to make room for other
% applications.
%
%
% \subsection{Existing theorem styles}
%
% The following theorem styles exist to date:
% \begin{deflist}{marginbreak}
%    \item[plain]
%       This theorem style emulates the original \LaTeX{} definition,
%       except that additionally the parameters
%       |\theorem...skipamount| are used.
%    \item[break]
%       In this style, the theorem header is followed by a line
%       break.
%    \item[marginbreak]
%       The theorem number is set in the margin, and there is a line
%       break as in \texttt{break}.
%    \item[changebreak]
%       Like \texttt{break}, but with header number and text
%       interchanged.
%    \item[change]
%       Header number and text are interchanged, without a line break.
%    \item[margin]
%       The number is set in the left margin, without a line break.
% \end{deflist}
% All styles (except \texttt{plain}) select |\slshape| as the default
% |\theorembodyfont|.
%
%
% \subsection{Examples}
%
% Given the above theorem sets {\envfont Cor}, {\envfont Exa}
% and {\envfont Rem},
% suppose that the preamble also contains the declarations:
% \begin{verbatim}
%    \theoremstyle{marginbreak}   \newtheorem{Lem}[Cor]{Lemma}
%    \theoremstyle{change}
%    \theorembodyfont{\itshape}        \newtheorem{Def}[Cor]{Definition}
%
%    \theoremheaderfont{\scshape}
%\end{verbatim}
% Then the following are some typical examples of the typeset output
% resulting from their use.
%
% \begin{Cor}
%    This is a sentence typeset in the theorem environment {\envfont
%    Cor}.
% \end{Cor}
%
% \begin{Exa}
%    This is a sentence typeset in the theorem environment {\envfont
%    Exa}.
% \end{Exa}
%
% \begin{Rem}
%    This is a sentence typeset in the theorem environment {\envfont
%    Rem}.
% \end{Rem}
%
% \begin{Lem}[Ben User]
%    This is a sentence typeset in the theorem environment {\envfont
%    Lem}.
% \end{Lem}
%
% \begin{Def}[Very impressive Definition]
%    This is a sentence typeset in the theorem environment {\envfont
%    Def}.
% \end{Def}
%
% The last two examples show the effect of the optional argument to
% a theorem environment (it is the text typeset in parentheses).
%
%
% \section{Special Considerations}
% \label{sec:newfonts}
%
% Theoremheader and body are implemented as a unit. This means that
% the |\theoremheaderfont|  will inherit characteristics of the
% |\theorembodyfont| in \LaTeXe. Thus, if for example
% |\theorembodyfont| is |\itshape| and |\theoremheaderfont|
% is |\bfseries| the font selected for the header will have the
% characteristics `bold extended italic'. If this is not desired one
% should set the |\theoremheaderfont| to something like
% \begin{verbatim}
%      \theoremheaderfont{\normalfont\bfseries}
%\end{verbatim}
% i.e.\ supplying all necessary font information explicitly.
%
%
%
% \section{Acknowledgements}
%
% The publication of this set of macros was only possible with the
% help of Christina Busse (translating the manuscript into English),
% Joachim Pense (playing the r\^ole of typist), Chris Rowley
% (looking everything over) and many others providing useful
% suggestions.
%
% \MaybeStop{%
%    \begin{thebibliography}{9}
%       \bibitem{bk:GMS94} \textsc{M.~Goossens}, \textsc{F.~Mittelbach}
%         and \textsc{A.~Samarin}.
%         \newblock The \LaTeX{} Companion.
%         \newblock
%         Addison-Wesley, Reading, Massachusetts, 1994.
%       \bibitem{latex:source} \textsc{Lamport, Leslie.} \newblock
%                               \texttt{latex.tex}, version 2.09, date
%                               Feb. 1990.
%    \end{thebibliography}
%    \PrintIndex
%    \PrintChanges}
%
%
% \section{The documentation driver file}
%
% The next bit of code contains the documentation driver file for
% \TeX{}, i.e., the file that will produce the documentation you are
% currently reading. It will be extracted from this file by the
% \texttt{docstrip} program. Since it is the first code in the file
% one can alternatively process this file directly with \LaTeXe{} to
% obtain the documentation.
%
%    \begin{macrocode}
%<*driver>
\documentclass{ltxdoc}

\usepackage{theorem}

%<+driver>% The next few lines define theorem sets which are used
%<+driver>% in the example section of the documentation.

\theoremstyle{break}       \newtheorem{Cor}{Corollary}
\theoremstyle{plain}       \newtheorem{Exa}{Example}[section]
{\theorembodyfont{\upshape}\newtheorem{Rem}{Remark}}
\theoremstyle{marginbreak} \newtheorem{Lem}[Cor]{Lemma}
\theoremstyle{change}
\theorembodyfont{\itshape} \newtheorem{Def}[Cor]{Definition}

\theoremheaderfont{\scshape}

\RecordChanges

\begin{document}
   \DocInput{theorem.dtx}
\end{document}
%</driver>
%    \end{macrocode}
%
% \section{Definition of the Macros}
%
% If the file has been loaded before, we abort immediately. If not the
% package announces itself (this is actually done at the very top of
% the file---the way it is done isn't good style so don't copy it).
% \changes{v2.0e}{89/07/19}{Spaces removed from `typeout'.}
% \changes{v2.2b}{95/11/19}{Announce moved to top of file}
%    \begin{macrocode}
%<*package>
%\@ifundefined{theorem@style}{}{\endinput}
%\def\FMithmInfo{1995/11/19 v2.2b Theorem extension package (FMi)}
%\ProvidesPackage{theorem}[\FMithmInfo]
%    \end{macrocode}
%
%
% \subsection{Definition of theorem styles and fonts}
%
% All the definitions in this file are done globally to allow
% inputting this file inside a group.
%
% \begin{macro}{\theoremstyle}
%    Before a theorem style can be installed, the chosen style must be
%    known. For that reason, we must test to see that
%    |\th@|\meta{style} is known or, more precisely, that it is
%    different from |\relax|. If the style is not known then
%    |\th@plain| is used.
%    \begin{macrocode}
\gdef\theoremstyle#1{%
   \@ifundefined{th@#1}{\@warning
          {Unknown theoremstyle `#1'. Using `plain'}%
          \theorem@style{plain}}%
%    \end{macrocode}
%    We save the theorem style to be used in the token register
%    |\theorem@style|.
%    \begin{macrocode}
      {\theorem@style{#1}}%
%    \end{macrocode}
%    Now we ``evaluate'' the theorem style: this means, we call the
%    macro |\th@|\meta{style} which will activate the relevant
%    definitions which are contained in a separate file.  This is done
%    in a group to suppress changes to the current font.  This could
%    otherwise pose problems together with the new font selection
%    scheme\footnote{When I printed the original article using the new
%    font selection scheme I ended with a document with slanted
%    typefaces (text headings and all) simply because one of the
%    theorem styles used \texttt{\bslash sl} at toplevel.} if the
%    |\th@|\meta{style} is evaluated a second time.
% \changes{v2.1a}{90/02/15}{Group added around `th@..}
% \changes{v2.1a}{90/02/15}{Reseting of `..@begintheorem removed.}
%    \begin{macrocode}
      \begingroup
        \csname th@\the\theorem@style \endcsname
      \endgroup}
%    \end{macrocode}
% \end{macro}
%
%
% \begin{macro}{\@begintheorem}
% \changes{v2.1a}{90/02/15}{Relaxed at toplevel.}
% \begin{macro}{\@opargbegintheorem}
% \changes{v2.1a}{90/02/15}{Relaxed at toplevel.}
%    We reset |\@begintheorem| and |\@opargbegintheorem| to
%    |\relax| since these commands are no longer necessary at
%    toplevel. This will save a few tokens.
%    \begin{macrocode}
\global\let\@begintheorem\relax
\global\let\@opargbegintheorem\relax
%    \end{macrocode}
% \end{macro}
% \end{macro}
%
% \begin{macro}{\theorem@style}
%    Obviously the \textsf{token} register used above has to be
%    allocated. To assure the utmost compatibility with the
%    original \LaTeX{} definition, we set the default theorem style to
%    \texttt{plain}, which implements the usual \LaTeX{} convention.
% \changes{v2.0f}{89/07/20}{Unnecessary `global' removed.}
%    \begin{macrocode}
\newtoks\theorem@style
\global\theorem@style{plain}
%    \end{macrocode}
% \end{macro}
%
% \begin{macro}{\theorembodyfont}
% \begin{macro}{\theorem@bodyfont}
%    If the |\theorembodyfont| is set by the user then it should not
%    interact with the default font set in the theorem style. When the
%    new font selection is in force this may happen if, for example,
%    the default is |\itshape| and the new |\theorembodyfont| is
%    |\sffamily|. So we add a |\reset@font| command in front of the
%    user definition.
% \changes{v2.0f}{89/07/20}{Unnecessary `global' removed.}
% \changes{v2.0g}{89/09/19}{`theoremfont to `theorembodyfont.}
% \changes{v2.1c}{92/06/22}{`reset@font added and logic changed.}
% \changes{v2.1d}{92/10/28}{Reset to empty if arg is empty}
%    \begin{macrocode}
\gdef\theorembodyfont#1{%
%    \end{macrocode}
%    We check if the argument supplied is empty and if so put nothing
%    into the |\theorem@bodyfont| token register to allow for
%    |\theorembodyfont{}| as a mean of using the default of the
%    current |\theoremstyle|.
%    \begin{macrocode}
   \def\@tempa{#1}%
   \ifx\@tempa\@empty
    \theorem@bodyfont{}%
   \else
    \theorem@bodyfont{\reset@font#1}%
   \fi
   }
\newtoks\theorem@bodyfont
\global\theorem@bodyfont{}
%    \end{macrocode}
% \end{macro}
% \end{macro}
%
%
% \begin{macro}{\theoremheaderfont}
% \changes{v2.0g}{89/09/19}{Macro added.}
%    The font for the theorem headers is handled differently because
%    this definition applies to all theorem styles.
%    \begin{macrocode}
\gdef\theoremheaderfont#1{\gdef\theorem@headerfont{#1}%
%    \end{macrocode}
%    After using the macro once it is redefined to produce an error
%    message.
%    \begin{macrocode}
       \gdef\theoremheaderfont##1{%
        \typeout{\string\theoremheaderfont\space should be used
                 only once.}}}
%    \end{macrocode}
% \end{macro}
%
% \begin{macro}{\theorem@headerfont}
% \changes{v2.0g}{89/09/19}{Macro added.}
%    To set the |\theorem@headerfont| default we first test if
%    the new fontselection scheme is in force.
% \changes{v2.1a}{90/02/15}{Support for new font selection.}
%    \begin{macrocode}
\ifx\upshape\undefined
%    \end{macrocode}
%    If not we define it to expand into |\bfseries|. We don't use
%    |\let| just in case a following style option redefines this
%    macro.
%    \begin{macrocode}
\gdef\theorem@headerfont{\bfseries}
%    \end{macrocode}
%    Otherwise we reset the current shape before calling |\bfseries|.
%    \begin{macrocode}
\else \gdef\theorem@headerfont{\normalfont\bfseries}\fi
%    \end{macrocode}
% \end{macro}
%
% \begin{macro}{\th@plain}
% \begin{macro}{\th@break}
% \begin{macro}{\th@marginbreak}
% \begin{macro}{\th@changebreak}
% \begin{macro}{\th@change}
% \begin{macro}{\th@margin}
%    The different styles are defined in macros such as
%    |\th@plain|.  Since memory space is precious in
%    ``non-Big-versions'', we have to avoid offering too many unused
%    definitions.  Therefore we define these styles in separate files
%    that can be loaded on demand.  Thus the commands themselves only
%    load these files. We use |\@input@|, a \LaTeXe{} internal command
%    that ensures that the file will be listed with |\listfiles|.
% \changes{v2.2a}{94/02/02}{Use `@input@ to load theorem layout files}
%    \begin{macrocode}
\gdef\th@plain{\@input@{thp.sty}}
\gdef\th@break{\@input@{thb.sty}}
\gdef\th@marginbreak{\@input@{thmb.sty}}
\gdef\th@changebreak{\@input@{thcb.sty}}
\gdef\th@change{\@input@{thc.sty}}
\gdef\th@margin{\@input@{thm.sty}}
%    \end{macrocode}
%    This list will be expanded when new styles become
%    available.  For testing,  just append new theorem substyles
%    as document options.
% \end{macro}
% \end{macro}
% \end{macro}
% \end{macro}
% \end{macro}
% \end{macro}
%
%
%
% \subsection{Definition of a new theorem set}
%
% As already pointed out, a new theorem environment can be defined in
% three different ways:
% \begin{verbatim}
%    \newtheorem{Lem}{Lemma}
%    \newtheorem{Lem}{Lemma}[section]
%    \newtheorem{Lem}[Theorem]{Lemma}
%\end{verbatim}
% The function of the macro |\newtheorem| is to recognize these cases
% and then to branch into one of the three macros |\@ynthm|, |\@xnthm|
% or |\@othm|.  This mechanism is adopted unchanged from
% \cite{latex:source}; the essential point here is that, for example,
% in the second case, the arguments \texttt{Lem}, \texttt{Lemma} and
% \texttt{section} are passed over to the macro |\@xnthm|.
%
% We inspect this case first because the others present fewer problems,
% and thus are easily derived from this one.
%
% \begin{macro}{\@xnthm}
%    For our example arguments, the macro |\@xnthm| must fulfill
%    the following:
%    \begin{itemize}
%       \item
%          Define a new \LaTeX-counter `Lem'
%       \item
%          reset this counter within a |\section|
%       \item
%          define the macro |\theLem|
%       \item
%          define the environment macros |\Lem| and
%          |\endLem| using the current
%          |\theoremstyle| and |\theorem@bodyfont|.
%    \end{itemize}
%    Obviously, all this should happen only if the first argument
%    of |\@xnthm| (i.e.\ \texttt{Lem} in our example) is chosen so
%    as not to conflict with any previously defined commands or
%    environments.  This test is performed by the \LaTeX{} macro
%    |\@ifdefinable|.
%    \begin{macrocode}
\gdef\@xnthm#1#2[#3]{\expandafter\@ifdefinable\csname #1\endcsname
%    \end{macrocode}
%    Therefore, the first argument of |\@ifdefinable| is the
%    expansion (in the example, |\Lem|) of
%    |\csname||#1\endcsname|.  The second argument is
%    executed only if the test has been completed successfully.
%    \begin{macrocode}
   {%
%    \end{macrocode}
%    Now we define the new counter. The names of the \LaTeX{}
%    macros employed should speak for themselves:
% \changes{v2.2b}{1995/10/07}{Use \cs{@newctr}}
% \changes{v2.2c}{1995/11/23}{Use \cs{@newctr} properly.}
%    \begin{macrocode}
    \@definecounter{#1}\@newctr{#1}[#3]%
%    \end{macrocode}
%    Using |\@newctr| will give a proper error message if the counter
%    in |#3| is not defined.
%    In defining `|\theLem|' we must generate the desired macro
%    name by use of |\expandafter| and |\csname|.
%    \begin{macrocode}
    \expandafter\xdef\csname the#1\endcsname
%    \end{macrocode}
%    An |\xdef| is used in order to make the definition global,
%    and to ensure that it contains the replacement texts of
%    |\@thmcountersep| and |\@thmcounter|.\footnote{These
%    two macros can be defined by the document style.  Their default
%    values produce a `\texttt{.}' as separation and an arabic
%    representation of the number.} However, not everything should be
%    expanded. For example, it saves space to use |\thesection|
%    instead of its---at times---lengthy expansion.
%    \begin{macrocode}
      {\expandafter \noexpand \csname the#3\endcsname
       \@thmcountersep \@thmcounter{#1}}%
%    \end{macrocode}
%    Thus with the defaults of \LaTeX{}, |\theLem| would be
%    replaced by the command sequence |\thesection.\arabic{Lem}|.
%
%    We will now look at the definition of the macro which is executed
%    at the beginning of the actual environment (in our example this
%    macro is |\Lem|). It should be noted that we use an
%    ``|\expandafter| trick'' to expand only certain parts of the
%    replacement text at the time of the definition.
%    \begin{macrocode}
    \def\@tempa{\global\@namedef{#1}}%
    \expandafter \@tempa \expandafter{%
%    \end{macrocode}
%    First, the macro that contains the current definitions of
%    |\@begintheorem| and |\@opargtheorem| should be called
%    up.  The name of this macro---as is already known---has the form
%    |\th@|\meta{theorem\ style}; therefore, it must be called by
%    \begin{macrocode}
      \csname th@\the \theorem@style
%    \end{macrocode}
%    In addition the default theorem font should be changeable, i.e.\
%    we have to insert the contents of |\theorem@bodyfont|. For
%    that reason, we expand even further, beyond |\endcsname|,
%    and thus insert the contents of the \textsf{token} register
%    |\theorem@bodyfont| in the replacement text.
%    \begin{macrocode}
            \expandafter \endcsname \the \theorem@bodyfont
%    \end{macrocode}
%    Now it is time to call the macro |\@thm| which takes over
%    the further processing.  It has two arguments: the current
%    counter name (in our example, \texttt{Lem}), and the text of the
%    label (in our example, \texttt{Lemma}).
%    \begin{macrocode}
     \@thm{#1}{#2}}%
%    \end{macrocode}
%    With this, the `sub-definition' is complete.  The macro
%    |\@endtheorem| ends a theorem environment and is, so far, nothing
%    but an |\endtrivlist|.  (Hence it is defined globally, and not
%    within the theorem styles.\footnote{This has to be changed as
%    soon as theorem styles that change \texttt{\bslash @endtheorem}
%    exist.  In such a case, all existing styles must be changed as
%    well since they will have to reset the macro.}) Therefore, we can
%    set it equivalent to the macro that ends the theorem set (in our
%    example, |\endLem|).  However, if some day theorem styles exist
%    that do change |\@endtheorem|, we would have to use the
%    commented-out line instead.
%    \begin{macrocode}
    \global \expandafter \let \csname end#1\endcsname \@endtheorem
%   \global\@namedef{end#1}{\@endtheorem}%
%    \end{macrocode}
%    With these commands all the required definitions are employed,
%    unless the test |\@ifdefinable| has failed.  Therefore, we
%    end the second argument of this macro and with it the definition
%    of |\@xnthm|.
%    \begin{macrocode}
   }}
%    \end{macrocode}
% \end{macro}
%
% \begin{macro}{\@ynthm}
%    The definition of |\@ynthm| is completely analogous. In
%    this case the new counter that is defined is not reset within
%    another counter; thus the definition of |\the...|
%    is simplified:
%    \begin{macrocode}
\gdef\@ynthm#1#2{\expandafter\@ifdefinable\csname #1\endcsname
   {\@definecounter{#1}%
    \expandafter\xdef\csname the#1\endcsname{\@thmcounter{#1}}%
%    \end{macrocode}
%    The rest of the definition corresponds literally to that of
%    |\@xnthm|:
%    \begin{macrocode}
    \def\@tempa{\global\@namedef{#1}}\expandafter \@tempa
     \expandafter{\csname th@\the \theorem@style \expandafter
     \endcsname \the\theorem@bodyfont \@thm{#1}{#2}}%
    \global \expandafter \let \csname end#1\endcsname \@endtheorem}}
%    \end{macrocode}
% \end{macro}
%
% \begin{macro}{\@othm}
%    The definition of |\@othm| does not contain anything new.
%    \begin{macrocode}
\gdef\@othm#1[#2]#3{%
%    \end{macrocode}
% \changes{v2.2b}{95/10/07}{Build in check for counter defined}
%    We do not define a new counter but instead use one that has
%    already been defined.  Thus the only definition we need is that
%    of this pseudo-counter (i.e.\ |\the|\meta{env.\ name}).
%    First we check if |#2| corresponds to a known counter name.
%    \begin{macrocode}
  \expandafter\ifx\csname c@#2\endcsname\relax
   \@nocounterr{#2}%
  \else
   \expandafter\@ifdefinable\csname #1\endcsname
   {\expandafter \xdef \csname the#1\endcsname
     {\expandafter \noexpand \csname the#2\endcsname}%
%    \end{macrocode}
%    All other parts of the definition can be adopted from
%    |\@xnthm|. We have to remember, though, that in this case
%    the name of the current counter and the theorem label have moved
%    to the second and third arguments.
%    \begin{macrocode}
    \def\@tempa{\global\@namedef{#1}}\expandafter \@tempa
     \expandafter{\csname th@\the \theorem@style \expandafter
     \endcsname \the\theorem@bodyfont \@thm{#2}{#3}}%
    \global \expandafter \let \csname end#1\endcsname \@endtheorem}%
  \fi}
%    \end{macrocode}
% \end{macro}
%
% \subsection{Macros that are employed in a {\envfont theorem}
%    environment}
%
% \begin{macro}{\@thm}
%    The macro |\@thm| has to increase the current counter.  Then,
%    depending on whether the environment has (or does not have) an
%    optional argument, it has to branch into either |\@begintheorem|
%    or |\@opargtheorem|.
%    \begin{macrocode}
\gdef\@thm#1#2{\refstepcounter{#1}%
%    \end{macrocode}
%    Now we start a {\envfont trivlist} environment, and give
%    |\@topsep| and |\@topsepadd| the values of the \textsf{skip}
%    registers |\theorempreskipamount| and |\theorempostskipamount|.
%    The value in |\@topsep| is the vertical space that is inserted by
%    the first (and only) |\item| in our |\trivlist| whilst
%    |\@topsepadd| is inserted by |\@endparenv| at the end of that
%    {\envfont trivlist} environment.  By using these registers, we
%    obtain the desired space around a {\envfont theorem} environment.
%    \begin{macrocode}
   \trivlist
   \@topsep \theorempreskipamount               % used by first \item
   \@topsepadd \theorempostskipamount           % used by \@endparenv
%    \end{macrocode}
%    Now we have to test whether an optional argument has been given.
% \changes{v2.0d}{89/07/18}{Forgotten percent added.}
%    \begin{macrocode}
   \@ifnextchar [%
%    \end{macrocode}
%    If there is an optional argument, we will call |\@ythm|, and
%    move the arguments read back into the input stream.
%    \begin{macrocode}
   {\@ythm{#1}{#2}}%
%    \end{macrocode}
%    If not, we call |\@begintheorem|. Its first argument is the
%    name of the theorem set (hence the second argument of
%    |\@thm|). Its second argument is the macro that produces the
%    current number.
%    \begin{macrocode}
   {\@begintheorem{#2}{\csname the#1\endcsname}\ignorespaces}}
%    \end{macrocode}
% \end{macro}
%
% \begin{macro}{\@xthm}
% \begin{macro}{\@ythm}
%    Both these macros were originally called by |\@thm|.  We do
%    not need |\@xthm| anymore, hence we reset it to
%    |\relax|. The definition of |\@ythm| has not changed at
%    all from its definition in \LaTeX. In order to make the macros
%    easier to understand, we will
%    nevertheless present it (commented out).
%    \begin{macrocode}
\global\let\@xthm\relax
% \def\@ythm#1#2[#3]{\@opargbegintheorem{#2}{\csname
%           the#1\endcsname}{#3}\ignorespaces}
%</package>
%    \end{macrocode}
%    The primitive |\ignorespaces| in |\@ythm| and
%    |\@thm| is needed to remove the spaces between the
%    |\begin{...}| and the actual text.
% \end{macro}
% \end{macro}
%
%
% \subsection{Definition of the theorem substyles}
%
% As already pointed out, the theorem substyles, defined below, are
% only loaded when necessary.  Note that all these substyles, except
% \texttt{plain}, have |\slshape| as the default body font.
%
% \subsubsection{The \texttt{plain} style}
%
% \label{dok:th@plain} As the following macros use |@|, we have
% to locally set the |\catcode| of this symbol to ``letter''.
% This happens within a group, so that we do not have to worry about
% which |\catcode| that symbol had before.
%    \begin{macrocode}
%<*thp>
\begingroup \makeatletter
%    \end{macrocode}
% Since we are now within a group, we must make all definitions
% globally.  First we make sure that \texttt{theorem.sty} is loaded.
% This will allow us to use this file as a document style option
% without having to call \texttt{theorem} itself as an option.  At the
% same time, we assure that at least version 2 is loaded, since
% |\theorem@style| was not defined in earlier versions.
%    \begin{macrocode}
\@ifundefined{theorem@style}{\input{theorem.sty}}{}
\ProvidesFile{thp.sty}
             [\FMithmInfo]
%    \end{macrocode}
%
%
% \begin{macro}{\th@plain}
% \begin{macro}{\@begintheorem}
% \begin{macro}{\@opargbegintheorem}
%    |\theoremstyle{plain}| corresponds to the original
%    definition, except that the distances to the surrounding text are
%    determined by the parameters |\theorempreskipamount| and
%    |\theorempostskipamount|.
%    First we set the default body font.
%    \begin{macrocode}
\gdef\th@plain{\normalfont\itshape
%    \end{macrocode}
%    Then we define |\@begintheorem| and
%    |\@opargbegintheorem|. These two macros define how the
%    header of a theorem is typeset.  |\@opargbegintheorem| will
%    be called if a \textsf{theorem} environment with an optional
%    argument is encountered; otherwise, the header is constructed by
%    calling |\@begintheorem|.  If one of these macros is
%    executed, we are within a \textsf{trivlist} environment started by
%    |\@thm|.  So the theorem header is produced with an
%    |\item| command.
%
%    Instead of specifying the header font directly, all standard
%    theorem styles use the |\theorem@headerfont| macro to allow
%    customization.  The extra space (|\labelsep|) is necessary
%    because of problems in the \textsf{trivlist} environment.
%    \begin{macrocode}
  \def\@begintheorem##1##2{%
        \item[\hskip\labelsep \theorem@headerfont ##1\ ##2]}%
%    \end{macrocode}
%    The definition of |\@opargbegintheorem| is completely
%    analogous. The only difference is the fact that there exists a
%    third argument (which is the optional parameter of the
%    environment and contains additional information about the
%    theorem).  Customarily we enclose it in parentheses.
%    \begin{macrocode}
\def\@opargbegintheorem##1##2##3{%
   \item[\hskip\labelsep \theorem@headerfont ##1\ ##2\ (##3)]}}
%    \end{macrocode}
% \end{macro}
% \end{macro}
% \end{macro}
%
% \noindent
% We conclude with an |\endgroup| to restore the
% |\catcode| of |@|.
%    \begin{macrocode}
\endgroup
%</thp>
%    \end{macrocode}
%
%
% \subsubsection{The \texttt{break} style}
%
% This style option is stored in the file |thb.sty|.
% For the next two lines see the documentation for |\th@plain| on
% page~\pageref{dok:th@plain}.
%    \begin{macrocode}
%<*thb>
\begingroup \makeatletter
\@ifundefined{theorem@style}{\input{theorem.sty}}{}
\ProvidesFile{thb.sty}
             [\FMithmInfo]
%    \end{macrocode}
%
% \begin{macro}{\th@break}
%    |\theoremstyle{break}| produces a line break after the name
%    of the theorem.  The font is |\slshape|.  Hence, we define
%    |\th@break| as follows:
%    \begin{macrocode}
\gdef\th@break{\normalfont\slshape
  \def\@begintheorem##1##2{\item[%
%    \end{macrocode}
%    We run into the following problem: it is not possible to create
%    the header with |\item[|\meta{title}|]| and then start
%    a new line by, for example, |\mbox{}\\|. Such a definition
%    will fail whenever a list environment follows immediately.  With
%    the above construction, the |\mbox{}| causes the switch
%    |@inlabel| (cf.\ definition of |\list| and
%    |\trivlist| in \cite{latex:source}) to be set to \texttt{false}
%    and so the following list will insert additional vertical
%    space (|\topskip|). This is quite annoying. Therefore, we
%    create the line break within the |\item|. In order to ensure
%    that the text will begin at the proper position in the following
%    line, we simply pretend that the label does not take any
%    room.\footnote{This will lead to problems whenever very high
%    symbols occurring in the line tower into the
%    heading. So, something else has to be done here sometime.}
%    \begin{macrocode}
       \rlap{\vbox{\hbox{\hskip \labelsep\theorem@headerfont ##1\ ##2}%
                   \hbox{\strut}}}]}%
%    \end{macrocode}
%    Again, the definition of |\@opargbegintheorem| is
%    completely analogous.
%    \begin{macrocode}
\def\@opargbegintheorem##1##2##3{%
  \item[\rlap{\vbox{\hbox{\hskip \labelsep \theorem@headerfont
                     ##1\ ##2\ (##3)}%
                    \hbox{\strut}}}]}}
\endgroup
%</thb>
%    \end{macrocode}
% \end{macro}
%
%
% \subsubsection{The \texttt{changebreak} style}
%
%    \begin{macrocode}
% This style option is stored in the file |thcb.sty|.
%    \begin{macrocode}
%<*thcb>
\begingroup \makeatletter
\@ifundefined{theorem@style}{\input{theorem.sty}}{}
\ProvidesFile{thcb.sty}
             [\FMithmInfo]
%    \end{macrocode}
%
% \begin{macro}{\th@changebreak}
%    The \texttt{change-break} theorem style is like \texttt{break}
%    but with interchange of theorem name and theorem number.  Thus we
%    define |\th@changebreak| as follows:
%    \begin{macrocode}
\gdef\th@changebreak{\normalfont\slshape
  \def\@begintheorem##1##2{\item
       [\rlap{\vbox{\hbox{\hskip\labelsep \theorem@headerfont ##2\ ##1}%
                    \hbox{\strut}}}]}%
\def\@opargbegintheorem##1##2##3{%
  \item[\rlap{\vbox{\hbox{\hskip\labelsep \theorem@headerfont
                     ##2\ ##1\ (##3)}%
                    \hbox{\strut}}}]}}
\endgroup
%</thcb>
%    \end{macrocode}
% \end{macro}
%
%
% \subsubsection{The \texttt{change} style}
%
% This style option is stored in the file |thc.sty|.
%    \begin{macrocode}
%<*thc>
\begingroup \makeatletter
\@ifundefined{theorem@style}{\input{theorem.sty}}{}
\ProvidesFile{thc.sty}
             [\FMithmInfo]
%    \end{macrocode}
%
% \begin{macro}{\th@change}
%    The \texttt{change} theorem style corresponds to the
%    \texttt{change break} style without a linebreak after the header.
%    To say it in another way, it's the same as the \texttt{plain}
%    style but with number and name interchanged and |\slshape| as the
%    default font.
%    \begin{macrocode}
\gdef\th@change{\normalfont\slshape
  \def\@begintheorem##1##2{\item
       [\hskip\labelsep \theorem@headerfont ##2\ ##1]}%
\def\@opargbegintheorem##1##2##3{%
  \item[\hskip\labelsep \theorem@headerfont ##2\ ##1\ (##3)]}}
\endgroup
%</thc>
%    \end{macrocode}
% \end{macro}
%
%
% \subsubsection{The \texttt{marginbreak} style}
%
% This style option is the one used most often at Mainz.  It is saved
% in the file |thmb.sty|.
%    \begin{macrocode}
%<*thmb>
\begingroup \makeatletter
\@ifundefined{theorem@style}{\input{theorem.sty}}{}
\ProvidesFile{thmb.sty}
             [\FMithmInfo]
%    \end{macrocode}
%
% \begin{macro}{\th@marginbreak}
%    The \texttt{margin break} style is nearly the same as the
%    \texttt{change break} style. The only difference is the placement
%    of the theorem number.  We use |\llap| to place it in the left
%    margin.
%
%    In this style |\labelsep| denotes the separation between the
%    number and the text.
%    \begin{macrocode}
\gdef\th@marginbreak{\normalfont\slshape
  \def\@begintheorem##1##2{\item
       [\rlap{\vbox{\theorem@headerfont
                    \hbox{\llap{##2}\hskip\labelsep ##1}%
                    \hbox{\strut}}}]}%
\def\@opargbegintheorem##1##2##3{%
  \item[\rlap{\vbox{\theorem@headerfont
                    \hbox{\llap{##2}\hskip\labelsep ##1\ (##3)}%
                    \hbox{\strut}}}]}}
\endgroup
%</thmb>
%    \end{macrocode}
% \changes{v2.1c}{90/02/17}{`endgroup outside macrocode env.}
% \end{macro}
%
%
%
% \subsubsection{The \texttt{margin} style}
%
% This style option is stored in the file |thm.sty|.
%    \begin{macrocode}
%<*thm>
\begingroup \makeatletter
\@ifundefined{theorem@style}{\input{theorem.sty}}{}
\ProvidesFile{thm.sty}
             [\FMithmInfo]
%    \end{macrocode}
%
% \begin{macro}{\th@margin}
%    Again this is only a variant of the theorem styles described
%    above without any new ideas.
%    \begin{macrocode}
\gdef\th@margin{\normalfont\slshape
  \def\@begintheorem##1##2{\item
       [\theorem@headerfont \llap{##2}\hskip\labelsep ##1]}%
\def\@opargbegintheorem##1##2##3{%
  \item[\theorem@headerfont \llap{##2}\hskip\labelsep ##1\ (##3)]}}
\endgroup
%</thm>
%    \end{macrocode}
% \end{macro}
%
%
%
%
% \subsection{Final Definitions}
%
% \begin{macro}{\theorempreskipamount}
% \begin{macro}{\theorempostskipamount}
%    The \textsf{skip} parameters that regulate the vertical empty space
%    before and after the {\envfont theorem} environment have
%    to be allocated as well.
% \changes{v2.0f}{89/07/20}{Unnecessary `global' removed.}
%    \begin{macrocode}
%<*package>
\newskip\theorempreskipamount
\newskip\theorempostskipamount
%    \end{macrocode}
%    Since we have used the same values for all theorem sets, we now can
%    assign them.
%    \begin{macrocode}
\global\setlength\theorempreskipamount{12pt plus 5pt minus 3pt}
\global\setlength\theorempostskipamount{8pt plus 3pt minus 1.5pt}
%    \end{macrocode}
% \end{macro}
% \end{macro}
%
%
% \begin{macro}{\@endtheorem}
%    The same holds for the macro |\@endtheorem|, which ends a
%    {\envfont theorem} environment. Since it is the same for all
%    theorem sets, it is removed from the macros |\th@|\meta{style}.
%    It simply ends the {\envfont trivlist} environment, which was
%    begun in |\@thm|.
%    \begin{macrocode}
\global\let\@endtheorem=\endtrivlist
%    \end{macrocode}
% \end{macro}
%
%
%
% \begin{macro}{\@preamblecmds}
%    All macros defined above are to be used only in the preamble.
%    Therefore, we insert them in |\@preamblecmds| which will disable
%    them at begin document. This is done by the internal \LaTeXe{}
%    command |\@onlypreamble|.
% \changes{v2.0g}{89/09/19}{`theoremheaderfont and `theorembodyfont
%    added.}
% \changes{v2.2a}{94/02/03}{Use \LaTeXe{} interface}
%    \begin{macrocode}
\@onlypreamble\@xnthm
\@onlypreamble\@ynthm
\@onlypreamble\@othm
\@onlypreamble\newtheorem
\@onlypreamble\theoremstyle
\@onlypreamble\theorembodyfont
\@onlypreamble\theoremheaderfont
%    \end{macrocode}
% \end{macro}
%
% Finally we declare the \texttt{plain} theorem style to be the default.
% \changes{v2.1b}{90/06/04}{Fixed missing `item by declaring default
%    style.}
%    \begin{macrocode}
\theoremstyle{plain}
%</package>
%    \end{macrocode}
%
% \Finale

\endinput
}{}
\ProvidesFile{thmb.sty}
             [\FMithmInfo]
%    \end{macrocode}
%
% \begin{macro}{\th@marginbreak}
%    The \texttt{margin break} style is nearly the same as the
%    \texttt{change break} style. The only difference is the placement
%    of the theorem number.  We use |\llap| to place it in the left
%    margin.
%
%    In this style |\labelsep| denotes the separation between the
%    number and the text.
%    \begin{macrocode}
\gdef\th@marginbreak{\normalfont\slshape
  \def\@begintheorem##1##2{\item
       [\rlap{\vbox{\theorem@headerfont
                    \hbox{\llap{##2}\hskip\labelsep ##1}%
                    \hbox{\strut}}}]}%
\def\@opargbegintheorem##1##2##3{%
  \item[\rlap{\vbox{\theorem@headerfont
                    \hbox{\llap{##2}\hskip\labelsep ##1\ (##3)}%
                    \hbox{\strut}}}]}}
\endgroup
%</thmb>
%    \end{macrocode}
% \changes{v2.1c}{90/02/17}{`endgroup outside macrocode env.}
% \end{macro}
%
%
%
% \subsubsection{The \texttt{margin} style}
%
% This style option is stored in the file |thm.sty|.
%    \begin{macrocode}
%<*thm>
\begingroup \makeatletter
\@ifundefined{theorem@style}{% \iffalse meta-comment
%
% Copyright (C) 1993-2024
%
% The LaTeX Project and any individual authors listed elsewhere
% in this file.
%
% This file is part of the Standard LaTeX `Tools Bundle'.
% -------------------------------------------------------
%
% It may be distributed and/or modified under the
% conditions of the LaTeX Project Public License, either version 1.3c
% of this license or (at your option) any later version.
% The latest version of this license is in
%    https://www.latex-project.org/lppl.txt
% and version 1.3c or later is part of all distributions of LaTeX
% version 2005/12/01 or later.
%
% The list of all files belonging to the LaTeX `Tools Bundle' is
% given in the file `manifest.txt'.
%
% \fi
%
%%
%
%
\def\FMithmInfo{2023/07/05 v2.2c Theorem extension package (FMi)}
%
% \ProvidesFile{theorem.dtx}[\FMithmInfo]
%\iffalse   % this is a METACOMMENT !
%%
%% Package `theorem' to use with LaTeX2e
%% Copyright (C) 1989-1995 Frank Mittelbach, all rights reserved.
%<+package>\@ifundefined{theorem@style}{}{\endinput}
%<+package>\NeedsTeXFormat{LaTeX2e}[1995/06/01]
%<+package>\ProvidesPackage{theorem}
%<+package>         [\FMithmInfo]
%
% \fi
%
% \changes{v1.0}{87/05/07}{First draft.}
% \changes{v1.0a}{87/05/07}{`thistheoremstyle' added.}
% \changes{v1.1a}{87/07/29}{New concept: redefine the internal
%                          \protect\LaTeX{} commands `@xnthm' etc.}
% \changes{v1.1b}{87/07/31}{`theoremfont' used to specify font.}
% \changes{v1.2a}{88/01/08}{`@endtheorem' now globally defined.}
% \changes{v1.2b}{88/01/11}{`newtheorem' only allowed in the preamble.}
% \changes{v1.3a}{88/02/14}{`theoremfont' now a toks register.}
%
% \changes{v2.0a}{88/11/16}{New implementation and new documentation.}
% \changes{v2.0a}{88/11/16}{No layout changes but internal modifications
%                           old substyles won't work.}
% \changes{v2.0b}{89/01/30}{`global' used for all definitions.}
% \changes{v2.0c}{89/06/15}{Documentation translated to English.
%                            Thanks to LLL and Nina}
% \changes{v2.1a}{90/02/15}{Support for new font selection.}
% \changes{v2.2a}{94/02/03}{Cleanup for LaTeX2e}
%
% \DoNotIndex{\@endparenv,\@ifdefinable,\@ifnextchar,\@ifundefined}
% \DoNotIndex{\@namedef,\@nameuse,\@spaces,\@tempa,\@warning}
% \DoNotIndex{\ }
% \DoNotIndex{\begingroup,\bfseries}
% \DoNotIndex{\csname}
% \DoNotIndex{\def,\do,\docdate}
% \DoNotIndex{\endcsname,\endgroup,\endinput,\endtrivlist,\expandafter}
% \DoNotIndex{\filedate,\fileversion}
% \DoNotIndex{\gdef,\global}
% \DoNotIndex{\hbox,\hskip}
% \DoNotIndex{\ignorespaces,\input,\itshape,\item}
% \DoNotIndex{\labelsep,\let,\llap}
% \DoNotIndex{\makeatletter}
% \DoNotIndex{\noexpand}
% \DoNotIndex{\refstepcounter,\relax,\rlap}
% \DoNotIndex{\setlength,\slshape,\space,\string,\strut}
% \DoNotIndex{\the,\trivlist,\typeout}
% \DoNotIndex{\vbox}
% \DoNotIndex{\xdef}
%
% \def\deflabel#1{\ttfamily #1\hfill}
% \def\deflist#1{\begin{list}{}{\settowidth\labelwidth{\ttfamily #1}%
%                               \setlength\leftmargin\labelwidth
%                               \addtolength\leftmargin\labelsep
%                               \let\makelabel\deflabel}}
% \def\enddeflist{\end{list}}
% \def\envfont{\normalfont\ttfamily}
%
% \MakeShortVerb{\|}
%
% \newcommand{\TUB}{{\sl TUGboat\/}}
%
% \GetFileInfo{theorem.dtx}
% \title{An Extension of the
%        \LaTeX{} theorem environment\thanks{This file
%        has version number \fileversion, last
%        revised \filedate.}}
% \author{Frank Mittelbach}
%
% \MaintainedByLaTeXTeam{tools}
% \maketitle
%
% \begin{abstract}
%    The macros described in this paper yield an extension of the
%    \LaTeX{} theorem mechanism. It is designed to satisfy the
%    different requirements of various journals. Thus, the
%    layout of the ``theorems'' can be manipulated by determining a
%    ``style''. This article describes not only the use, but
%    also the definition, of the necessary macros.
% \end{abstract}
%
% \section*{Preface to version 2.2}
%
% For \LaTeXe{} this package did not need any fundamental changes. I
% only modified the messages generated so that theorem layout styles
% will show up with the |\listfiles| command and cleaned the section
% on the New Font Selection Scheme since this is now included in
% \LaTeX.
%
% \section*{Preface to version 2.1}
%
% This version is identical to 2.0g described in \TUB\ 10\#3 except
% for some internal defaults which are now set depending on the used
% font selection scheme.
%
% This was done to avoid unpleasant surprises if the new font selection
% scheme is in force. For further details see
% section~\ref{sec:newfonts} and \cite{bk:GMS94}.
%
% \section{Introduction}
%
% For our purposes here, ``theorems'' are labelled enunciations,
% often set off from the main text by extra space and a font change.
% Theorems, corollaries, conjectures, definitions, and remarks
% are all instances of ``theorems''.  The ``header'' of these structures
% is composed of a label (such as \textsc{Theorem} or \textsc{Remark})
% and a number which serializes an item in the sequence of items
% with the same label.
%
% Shortly after the introduction of \LaTeX{} at the Fachbereich
% Mathematik in Mainz, the desire to manipulate the layout of
% ``theorems'' arose.  In Mainz, the following two conventions came
% into general use:
% \begin{enumerate}
%   \item The number of the theorem is shown in the margin.
%   \item There is a line break at
%         the end of the theorem header.
% \end{enumerate}
%
% Additionally, some journals require different formats which depend on
% the ``sort of theorem'': e.g.\ often remarks and definitions are set
% in |\upshape|, while |\itshape| is employed for main theorems.
%
% Confronted with these requirements, a theorem environment was
% developed in Mainz which allows separate determination of the
% layout of the ``theorems sets'', comparable to |\pagestyle|.
%
%
% \section{The user interface}
% \subsection{Defining new theorem sets}
%
% \DescribeMacro\newtheorem
% As in the original \LaTeX{} version, the command |\newtheorem|
% defines a new ``theorem set'' or ``theorem-like structure''.
% Two required arguments name the new environment and give the text
% to be typeset with each instance of the new ``set'', while
% an optional argument determines how
% the ``set'' is enumerated:
% \begin{description}
%    \item[\ttfamily \bslash newtheorem\{foo\}\{bar\}]
%       The theorem set {\envfont foo} (whose name is \texttt{bar})
%       uses its own counter.
%    \item[\ttfamily \bslash newtheorem\{foo2\}{[foo]}\{bar2\}]
%       The theorem set {\envfont foo2} (printed name \texttt{bar2})
%       uses the same counter as the theorem set \texttt{foo}.
%    \item[\ttfamily \bslash newtheorem\{foo3\}\{bar3\}{[section]}]
%       The theorem set {\envfont foo3} (printed name \texttt{bar3}) is
%       enumerated within the counter \texttt{section}, i.e.\ with every
%       new |\section| the enumeration begins again with 1, and
%       the enumeration is composed from the section-number and the
%       theorem counter itself.
% \end{description}
%
% \DescribeMacro\theoremstyle
% Additionally, the command |\theoremstyle| can define the layout
% of various, or all, theorem sets.  It should be noted that any
% theorem set defined by |\newtheorem| is typeset in the
% |\theoremstyle| that is current at the time of the definition.
% Thus, the following
% \begin{verbatim}
%    \theoremstyle{break}        \newtheorem{Cor}{Corollary}
%    \theoremstyle{plain}        \newtheorem{Exa}{Example}[section]
%\end{verbatim}                             ^^A No blank at line start
% leads to the result that the set {\envfont Cor} is formatted in the
% style \texttt{break}, while the set {\envfont Exa} and all the
% following ones are formatted in the style \texttt{plain}, unless
% another |\theoremstyle| follows.  Since the definitions installed by
% |\newtheorem| are global, one also can limit |\theoremstyle| locally
% by grouping braces.
%
% \DescribeMacro\theorembodyfont
% The choice of the font for the theorem body is completely
% independent of the chosen |\theoremstyle|; this has proven to
% be very advantageous.  For example,
% \begin{verbatim}
%   {\theorembodyfont{\upshape}          \newtheorem{Rem}{Remark}}
%\end{verbatim}
% defines a theorem set {\envfont Rem}, which will be set in
% |\upshape| in the current layout (which in our example is
% \texttt{plain}).  As with |\theoremstyle|, the |\theorembodyfont|
% chosen is that current at the time of |\newtheorem|.  If
% |\theorembodyfont| is not specified or one defines
% |\theorembodyfont{}|, then the font used will be that defined by the
% |\theoremstyle|.
%
% \DescribeMacro\theoremheaderfont It is also possible to customize
% the font used for the theorem headers.  This is, however, a global
% declaration, and therefore there should be at most one
% |\theoremheaderfont| declaration in the preamble.\footnote{If it is
% actually necessary to have different header fonts, one has to define
% new theorem styles (substituting the desired font) or specify the
% information directly in the \texttt{\bslash newtheorem} declaration
% (the unclean variant).}
%
% \DescribeMacro\theorempreskipamount
% \DescribeMacro\theorempostskipamount Two additional parameters
% affect the vertical space around the theorem environments:
% |\theorempreskipamount| and |\theorempostskipamount| define,
% respectively, the spacing before and after such an environment.
% These parameters apply for all theorem sets and can be manipulated
% with the ordinary length macros.  They are rubber lengths
% (`\textsf{skips}'), and therefore can contain \texttt{plus} and
% \texttt{minus} parts.
%
% Since the definition of theorem sets should---most sensibly---be
% placed in the preamble, we only allow installation there.
% It is therefore possible to release the memory used here after
% |\begin{document}|, in order to make room for other
% applications.
%
%
% \subsection{Existing theorem styles}
%
% The following theorem styles exist to date:
% \begin{deflist}{marginbreak}
%    \item[plain]
%       This theorem style emulates the original \LaTeX{} definition,
%       except that additionally the parameters
%       |\theorem...skipamount| are used.
%    \item[break]
%       In this style, the theorem header is followed by a line
%       break.
%    \item[marginbreak]
%       The theorem number is set in the margin, and there is a line
%       break as in \texttt{break}.
%    \item[changebreak]
%       Like \texttt{break}, but with header number and text
%       interchanged.
%    \item[change]
%       Header number and text are interchanged, without a line break.
%    \item[margin]
%       The number is set in the left margin, without a line break.
% \end{deflist}
% All styles (except \texttt{plain}) select |\slshape| as the default
% |\theorembodyfont|.
%
%
% \subsection{Examples}
%
% Given the above theorem sets {\envfont Cor}, {\envfont Exa}
% and {\envfont Rem},
% suppose that the preamble also contains the declarations:
% \begin{verbatim}
%    \theoremstyle{marginbreak}   \newtheorem{Lem}[Cor]{Lemma}
%    \theoremstyle{change}
%    \theorembodyfont{\itshape}        \newtheorem{Def}[Cor]{Definition}
%
%    \theoremheaderfont{\scshape}
%\end{verbatim}
% Then the following are some typical examples of the typeset output
% resulting from their use.
%
% \begin{Cor}
%    This is a sentence typeset in the theorem environment {\envfont
%    Cor}.
% \end{Cor}
%
% \begin{Exa}
%    This is a sentence typeset in the theorem environment {\envfont
%    Exa}.
% \end{Exa}
%
% \begin{Rem}
%    This is a sentence typeset in the theorem environment {\envfont
%    Rem}.
% \end{Rem}
%
% \begin{Lem}[Ben User]
%    This is a sentence typeset in the theorem environment {\envfont
%    Lem}.
% \end{Lem}
%
% \begin{Def}[Very impressive Definition]
%    This is a sentence typeset in the theorem environment {\envfont
%    Def}.
% \end{Def}
%
% The last two examples show the effect of the optional argument to
% a theorem environment (it is the text typeset in parentheses).
%
%
% \section{Special Considerations}
% \label{sec:newfonts}
%
% Theoremheader and body are implemented as a unit. This means that
% the |\theoremheaderfont|  will inherit characteristics of the
% |\theorembodyfont| in \LaTeXe. Thus, if for example
% |\theorembodyfont| is |\itshape| and |\theoremheaderfont|
% is |\bfseries| the font selected for the header will have the
% characteristics `bold extended italic'. If this is not desired one
% should set the |\theoremheaderfont| to something like
% \begin{verbatim}
%      \theoremheaderfont{\normalfont\bfseries}
%\end{verbatim}
% i.e.\ supplying all necessary font information explicitly.
%
%
%
% \section{Acknowledgements}
%
% The publication of this set of macros was only possible with the
% help of Christina Busse (translating the manuscript into English),
% Joachim Pense (playing the r\^ole of typist), Chris Rowley
% (looking everything over) and many others providing useful
% suggestions.
%
% \MaybeStop{%
%    \begin{thebibliography}{9}
%       \bibitem{bk:GMS94} \textsc{M.~Goossens}, \textsc{F.~Mittelbach}
%         and \textsc{A.~Samarin}.
%         \newblock The \LaTeX{} Companion.
%         \newblock
%         Addison-Wesley, Reading, Massachusetts, 1994.
%       \bibitem{latex:source} \textsc{Lamport, Leslie.} \newblock
%                               \texttt{latex.tex}, version 2.09, date
%                               Feb. 1990.
%    \end{thebibliography}
%    \PrintIndex
%    \PrintChanges}
%
%
% \section{The documentation driver file}
%
% The next bit of code contains the documentation driver file for
% \TeX{}, i.e., the file that will produce the documentation you are
% currently reading. It will be extracted from this file by the
% \texttt{docstrip} program. Since it is the first code in the file
% one can alternatively process this file directly with \LaTeXe{} to
% obtain the documentation.
%
%    \begin{macrocode}
%<*driver>
\documentclass{ltxdoc}

\usepackage{theorem}

%<+driver>% The next few lines define theorem sets which are used
%<+driver>% in the example section of the documentation.

\theoremstyle{break}       \newtheorem{Cor}{Corollary}
\theoremstyle{plain}       \newtheorem{Exa}{Example}[section]
{\theorembodyfont{\upshape}\newtheorem{Rem}{Remark}}
\theoremstyle{marginbreak} \newtheorem{Lem}[Cor]{Lemma}
\theoremstyle{change}
\theorembodyfont{\itshape} \newtheorem{Def}[Cor]{Definition}

\theoremheaderfont{\scshape}

\RecordChanges

\begin{document}
   \DocInput{theorem.dtx}
\end{document}
%</driver>
%    \end{macrocode}
%
% \section{Definition of the Macros}
%
% If the file has been loaded before, we abort immediately. If not the
% package announces itself (this is actually done at the very top of
% the file---the way it is done isn't good style so don't copy it).
% \changes{v2.0e}{89/07/19}{Spaces removed from `typeout'.}
% \changes{v2.2b}{95/11/19}{Announce moved to top of file}
%    \begin{macrocode}
%<*package>
%\@ifundefined{theorem@style}{}{\endinput}
%\def\FMithmInfo{1995/11/19 v2.2b Theorem extension package (FMi)}
%\ProvidesPackage{theorem}[\FMithmInfo]
%    \end{macrocode}
%
%
% \subsection{Definition of theorem styles and fonts}
%
% All the definitions in this file are done globally to allow
% inputting this file inside a group.
%
% \begin{macro}{\theoremstyle}
%    Before a theorem style can be installed, the chosen style must be
%    known. For that reason, we must test to see that
%    |\th@|\meta{style} is known or, more precisely, that it is
%    different from |\relax|. If the style is not known then
%    |\th@plain| is used.
%    \begin{macrocode}
\gdef\theoremstyle#1{%
   \@ifundefined{th@#1}{\@warning
          {Unknown theoremstyle `#1'. Using `plain'}%
          \theorem@style{plain}}%
%    \end{macrocode}
%    We save the theorem style to be used in the token register
%    |\theorem@style|.
%    \begin{macrocode}
      {\theorem@style{#1}}%
%    \end{macrocode}
%    Now we ``evaluate'' the theorem style: this means, we call the
%    macro |\th@|\meta{style} which will activate the relevant
%    definitions which are contained in a separate file.  This is done
%    in a group to suppress changes to the current font.  This could
%    otherwise pose problems together with the new font selection
%    scheme\footnote{When I printed the original article using the new
%    font selection scheme I ended with a document with slanted
%    typefaces (text headings and all) simply because one of the
%    theorem styles used \texttt{\bslash sl} at toplevel.} if the
%    |\th@|\meta{style} is evaluated a second time.
% \changes{v2.1a}{90/02/15}{Group added around `th@..}
% \changes{v2.1a}{90/02/15}{Reseting of `..@begintheorem removed.}
%    \begin{macrocode}
      \begingroup
        \csname th@\the\theorem@style \endcsname
      \endgroup}
%    \end{macrocode}
% \end{macro}
%
%
% \begin{macro}{\@begintheorem}
% \changes{v2.1a}{90/02/15}{Relaxed at toplevel.}
% \begin{macro}{\@opargbegintheorem}
% \changes{v2.1a}{90/02/15}{Relaxed at toplevel.}
%    We reset |\@begintheorem| and |\@opargbegintheorem| to
%    |\relax| since these commands are no longer necessary at
%    toplevel. This will save a few tokens.
%    \begin{macrocode}
\global\let\@begintheorem\relax
\global\let\@opargbegintheorem\relax
%    \end{macrocode}
% \end{macro}
% \end{macro}
%
% \begin{macro}{\theorem@style}
%    Obviously the \textsf{token} register used above has to be
%    allocated. To assure the utmost compatibility with the
%    original \LaTeX{} definition, we set the default theorem style to
%    \texttt{plain}, which implements the usual \LaTeX{} convention.
% \changes{v2.0f}{89/07/20}{Unnecessary `global' removed.}
%    \begin{macrocode}
\newtoks\theorem@style
\global\theorem@style{plain}
%    \end{macrocode}
% \end{macro}
%
% \begin{macro}{\theorembodyfont}
% \begin{macro}{\theorem@bodyfont}
%    If the |\theorembodyfont| is set by the user then it should not
%    interact with the default font set in the theorem style. When the
%    new font selection is in force this may happen if, for example,
%    the default is |\itshape| and the new |\theorembodyfont| is
%    |\sffamily|. So we add a |\reset@font| command in front of the
%    user definition.
% \changes{v2.0f}{89/07/20}{Unnecessary `global' removed.}
% \changes{v2.0g}{89/09/19}{`theoremfont to `theorembodyfont.}
% \changes{v2.1c}{92/06/22}{`reset@font added and logic changed.}
% \changes{v2.1d}{92/10/28}{Reset to empty if arg is empty}
%    \begin{macrocode}
\gdef\theorembodyfont#1{%
%    \end{macrocode}
%    We check if the argument supplied is empty and if so put nothing
%    into the |\theorem@bodyfont| token register to allow for
%    |\theorembodyfont{}| as a mean of using the default of the
%    current |\theoremstyle|.
%    \begin{macrocode}
   \def\@tempa{#1}%
   \ifx\@tempa\@empty
    \theorem@bodyfont{}%
   \else
    \theorem@bodyfont{\reset@font#1}%
   \fi
   }
\newtoks\theorem@bodyfont
\global\theorem@bodyfont{}
%    \end{macrocode}
% \end{macro}
% \end{macro}
%
%
% \begin{macro}{\theoremheaderfont}
% \changes{v2.0g}{89/09/19}{Macro added.}
%    The font for the theorem headers is handled differently because
%    this definition applies to all theorem styles.
%    \begin{macrocode}
\gdef\theoremheaderfont#1{\gdef\theorem@headerfont{#1}%
%    \end{macrocode}
%    After using the macro once it is redefined to produce an error
%    message.
%    \begin{macrocode}
       \gdef\theoremheaderfont##1{%
        \typeout{\string\theoremheaderfont\space should be used
                 only once.}}}
%    \end{macrocode}
% \end{macro}
%
% \begin{macro}{\theorem@headerfont}
% \changes{v2.0g}{89/09/19}{Macro added.}
%    To set the |\theorem@headerfont| default we first test if
%    the new fontselection scheme is in force.
% \changes{v2.1a}{90/02/15}{Support for new font selection.}
%    \begin{macrocode}
\ifx\upshape\undefined
%    \end{macrocode}
%    If not we define it to expand into |\bfseries|. We don't use
%    |\let| just in case a following style option redefines this
%    macro.
%    \begin{macrocode}
\gdef\theorem@headerfont{\bfseries}
%    \end{macrocode}
%    Otherwise we reset the current shape before calling |\bfseries|.
%    \begin{macrocode}
\else \gdef\theorem@headerfont{\normalfont\bfseries}\fi
%    \end{macrocode}
% \end{macro}
%
% \begin{macro}{\th@plain}
% \begin{macro}{\th@break}
% \begin{macro}{\th@marginbreak}
% \begin{macro}{\th@changebreak}
% \begin{macro}{\th@change}
% \begin{macro}{\th@margin}
%    The different styles are defined in macros such as
%    |\th@plain|.  Since memory space is precious in
%    ``non-Big-versions'', we have to avoid offering too many unused
%    definitions.  Therefore we define these styles in separate files
%    that can be loaded on demand.  Thus the commands themselves only
%    load these files. We use |\@input@|, a \LaTeXe{} internal command
%    that ensures that the file will be listed with |\listfiles|.
% \changes{v2.2a}{94/02/02}{Use `@input@ to load theorem layout files}
%    \begin{macrocode}
\gdef\th@plain{\@input@{thp.sty}}
\gdef\th@break{\@input@{thb.sty}}
\gdef\th@marginbreak{\@input@{thmb.sty}}
\gdef\th@changebreak{\@input@{thcb.sty}}
\gdef\th@change{\@input@{thc.sty}}
\gdef\th@margin{\@input@{thm.sty}}
%    \end{macrocode}
%    This list will be expanded when new styles become
%    available.  For testing,  just append new theorem substyles
%    as document options.
% \end{macro}
% \end{macro}
% \end{macro}
% \end{macro}
% \end{macro}
% \end{macro}
%
%
%
% \subsection{Definition of a new theorem set}
%
% As already pointed out, a new theorem environment can be defined in
% three different ways:
% \begin{verbatim}
%    \newtheorem{Lem}{Lemma}
%    \newtheorem{Lem}{Lemma}[section]
%    \newtheorem{Lem}[Theorem]{Lemma}
%\end{verbatim}
% The function of the macro |\newtheorem| is to recognize these cases
% and then to branch into one of the three macros |\@ynthm|, |\@xnthm|
% or |\@othm|.  This mechanism is adopted unchanged from
% \cite{latex:source}; the essential point here is that, for example,
% in the second case, the arguments \texttt{Lem}, \texttt{Lemma} and
% \texttt{section} are passed over to the macro |\@xnthm|.
%
% We inspect this case first because the others present fewer problems,
% and thus are easily derived from this one.
%
% \begin{macro}{\@xnthm}
%    For our example arguments, the macro |\@xnthm| must fulfill
%    the following:
%    \begin{itemize}
%       \item
%          Define a new \LaTeX-counter `Lem'
%       \item
%          reset this counter within a |\section|
%       \item
%          define the macro |\theLem|
%       \item
%          define the environment macros |\Lem| and
%          |\endLem| using the current
%          |\theoremstyle| and |\theorem@bodyfont|.
%    \end{itemize}
%    Obviously, all this should happen only if the first argument
%    of |\@xnthm| (i.e.\ \texttt{Lem} in our example) is chosen so
%    as not to conflict with any previously defined commands or
%    environments.  This test is performed by the \LaTeX{} macro
%    |\@ifdefinable|.
%    \begin{macrocode}
\gdef\@xnthm#1#2[#3]{\expandafter\@ifdefinable\csname #1\endcsname
%    \end{macrocode}
%    Therefore, the first argument of |\@ifdefinable| is the
%    expansion (in the example, |\Lem|) of
%    |\csname||#1\endcsname|.  The second argument is
%    executed only if the test has been completed successfully.
%    \begin{macrocode}
   {%
%    \end{macrocode}
%    Now we define the new counter. The names of the \LaTeX{}
%    macros employed should speak for themselves:
% \changes{v2.2b}{1995/10/07}{Use \cs{@newctr}}
% \changes{v2.2c}{1995/11/23}{Use \cs{@newctr} properly.}
%    \begin{macrocode}
    \@definecounter{#1}\@newctr{#1}[#3]%
%    \end{macrocode}
%    Using |\@newctr| will give a proper error message if the counter
%    in |#3| is not defined.
%    In defining `|\theLem|' we must generate the desired macro
%    name by use of |\expandafter| and |\csname|.
%    \begin{macrocode}
    \expandafter\xdef\csname the#1\endcsname
%    \end{macrocode}
%    An |\xdef| is used in order to make the definition global,
%    and to ensure that it contains the replacement texts of
%    |\@thmcountersep| and |\@thmcounter|.\footnote{These
%    two macros can be defined by the document style.  Their default
%    values produce a `\texttt{.}' as separation and an arabic
%    representation of the number.} However, not everything should be
%    expanded. For example, it saves space to use |\thesection|
%    instead of its---at times---lengthy expansion.
%    \begin{macrocode}
      {\expandafter \noexpand \csname the#3\endcsname
       \@thmcountersep \@thmcounter{#1}}%
%    \end{macrocode}
%    Thus with the defaults of \LaTeX{}, |\theLem| would be
%    replaced by the command sequence |\thesection.\arabic{Lem}|.
%
%    We will now look at the definition of the macro which is executed
%    at the beginning of the actual environment (in our example this
%    macro is |\Lem|). It should be noted that we use an
%    ``|\expandafter| trick'' to expand only certain parts of the
%    replacement text at the time of the definition.
%    \begin{macrocode}
    \def\@tempa{\global\@namedef{#1}}%
    \expandafter \@tempa \expandafter{%
%    \end{macrocode}
%    First, the macro that contains the current definitions of
%    |\@begintheorem| and |\@opargtheorem| should be called
%    up.  The name of this macro---as is already known---has the form
%    |\th@|\meta{theorem\ style}; therefore, it must be called by
%    \begin{macrocode}
      \csname th@\the \theorem@style
%    \end{macrocode}
%    In addition the default theorem font should be changeable, i.e.\
%    we have to insert the contents of |\theorem@bodyfont|. For
%    that reason, we expand even further, beyond |\endcsname|,
%    and thus insert the contents of the \textsf{token} register
%    |\theorem@bodyfont| in the replacement text.
%    \begin{macrocode}
            \expandafter \endcsname \the \theorem@bodyfont
%    \end{macrocode}
%    Now it is time to call the macro |\@thm| which takes over
%    the further processing.  It has two arguments: the current
%    counter name (in our example, \texttt{Lem}), and the text of the
%    label (in our example, \texttt{Lemma}).
%    \begin{macrocode}
     \@thm{#1}{#2}}%
%    \end{macrocode}
%    With this, the `sub-definition' is complete.  The macro
%    |\@endtheorem| ends a theorem environment and is, so far, nothing
%    but an |\endtrivlist|.  (Hence it is defined globally, and not
%    within the theorem styles.\footnote{This has to be changed as
%    soon as theorem styles that change \texttt{\bslash @endtheorem}
%    exist.  In such a case, all existing styles must be changed as
%    well since they will have to reset the macro.}) Therefore, we can
%    set it equivalent to the macro that ends the theorem set (in our
%    example, |\endLem|).  However, if some day theorem styles exist
%    that do change |\@endtheorem|, we would have to use the
%    commented-out line instead.
%    \begin{macrocode}
    \global \expandafter \let \csname end#1\endcsname \@endtheorem
%   \global\@namedef{end#1}{\@endtheorem}%
%    \end{macrocode}
%    With these commands all the required definitions are employed,
%    unless the test |\@ifdefinable| has failed.  Therefore, we
%    end the second argument of this macro and with it the definition
%    of |\@xnthm|.
%    \begin{macrocode}
   }}
%    \end{macrocode}
% \end{macro}
%
% \begin{macro}{\@ynthm}
%    The definition of |\@ynthm| is completely analogous. In
%    this case the new counter that is defined is not reset within
%    another counter; thus the definition of |\the...|
%    is simplified:
%    \begin{macrocode}
\gdef\@ynthm#1#2{\expandafter\@ifdefinable\csname #1\endcsname
   {\@definecounter{#1}%
    \expandafter\xdef\csname the#1\endcsname{\@thmcounter{#1}}%
%    \end{macrocode}
%    The rest of the definition corresponds literally to that of
%    |\@xnthm|:
%    \begin{macrocode}
    \def\@tempa{\global\@namedef{#1}}\expandafter \@tempa
     \expandafter{\csname th@\the \theorem@style \expandafter
     \endcsname \the\theorem@bodyfont \@thm{#1}{#2}}%
    \global \expandafter \let \csname end#1\endcsname \@endtheorem}}
%    \end{macrocode}
% \end{macro}
%
% \begin{macro}{\@othm}
%    The definition of |\@othm| does not contain anything new.
%    \begin{macrocode}
\gdef\@othm#1[#2]#3{%
%    \end{macrocode}
% \changes{v2.2b}{95/10/07}{Build in check for counter defined}
%    We do not define a new counter but instead use one that has
%    already been defined.  Thus the only definition we need is that
%    of this pseudo-counter (i.e.\ |\the|\meta{env.\ name}).
%    First we check if |#2| corresponds to a known counter name.
%    \begin{macrocode}
  \expandafter\ifx\csname c@#2\endcsname\relax
   \@nocounterr{#2}%
  \else
   \expandafter\@ifdefinable\csname #1\endcsname
   {\expandafter \xdef \csname the#1\endcsname
     {\expandafter \noexpand \csname the#2\endcsname}%
%    \end{macrocode}
%    All other parts of the definition can be adopted from
%    |\@xnthm|. We have to remember, though, that in this case
%    the name of the current counter and the theorem label have moved
%    to the second and third arguments.
%    \begin{macrocode}
    \def\@tempa{\global\@namedef{#1}}\expandafter \@tempa
     \expandafter{\csname th@\the \theorem@style \expandafter
     \endcsname \the\theorem@bodyfont \@thm{#2}{#3}}%
    \global \expandafter \let \csname end#1\endcsname \@endtheorem}%
  \fi}
%    \end{macrocode}
% \end{macro}
%
% \subsection{Macros that are employed in a {\envfont theorem}
%    environment}
%
% \begin{macro}{\@thm}
%    The macro |\@thm| has to increase the current counter.  Then,
%    depending on whether the environment has (or does not have) an
%    optional argument, it has to branch into either |\@begintheorem|
%    or |\@opargtheorem|.
%    \begin{macrocode}
\gdef\@thm#1#2{\refstepcounter{#1}%
%    \end{macrocode}
%    Now we start a {\envfont trivlist} environment, and give
%    |\@topsep| and |\@topsepadd| the values of the \textsf{skip}
%    registers |\theorempreskipamount| and |\theorempostskipamount|.
%    The value in |\@topsep| is the vertical space that is inserted by
%    the first (and only) |\item| in our |\trivlist| whilst
%    |\@topsepadd| is inserted by |\@endparenv| at the end of that
%    {\envfont trivlist} environment.  By using these registers, we
%    obtain the desired space around a {\envfont theorem} environment.
%    \begin{macrocode}
   \trivlist
   \@topsep \theorempreskipamount               % used by first \item
   \@topsepadd \theorempostskipamount           % used by \@endparenv
%    \end{macrocode}
%    Now we have to test whether an optional argument has been given.
% \changes{v2.0d}{89/07/18}{Forgotten percent added.}
%    \begin{macrocode}
   \@ifnextchar [%
%    \end{macrocode}
%    If there is an optional argument, we will call |\@ythm|, and
%    move the arguments read back into the input stream.
%    \begin{macrocode}
   {\@ythm{#1}{#2}}%
%    \end{macrocode}
%    If not, we call |\@begintheorem|. Its first argument is the
%    name of the theorem set (hence the second argument of
%    |\@thm|). Its second argument is the macro that produces the
%    current number.
%    \begin{macrocode}
   {\@begintheorem{#2}{\csname the#1\endcsname}\ignorespaces}}
%    \end{macrocode}
% \end{macro}
%
% \begin{macro}{\@xthm}
% \begin{macro}{\@ythm}
%    Both these macros were originally called by |\@thm|.  We do
%    not need |\@xthm| anymore, hence we reset it to
%    |\relax|. The definition of |\@ythm| has not changed at
%    all from its definition in \LaTeX. In order to make the macros
%    easier to understand, we will
%    nevertheless present it (commented out).
%    \begin{macrocode}
\global\let\@xthm\relax
% \def\@ythm#1#2[#3]{\@opargbegintheorem{#2}{\csname
%           the#1\endcsname}{#3}\ignorespaces}
%</package>
%    \end{macrocode}
%    The primitive |\ignorespaces| in |\@ythm| and
%    |\@thm| is needed to remove the spaces between the
%    |\begin{...}| and the actual text.
% \end{macro}
% \end{macro}
%
%
% \subsection{Definition of the theorem substyles}
%
% As already pointed out, the theorem substyles, defined below, are
% only loaded when necessary.  Note that all these substyles, except
% \texttt{plain}, have |\slshape| as the default body font.
%
% \subsubsection{The \texttt{plain} style}
%
% \label{dok:th@plain} As the following macros use |@|, we have
% to locally set the |\catcode| of this symbol to ``letter''.
% This happens within a group, so that we do not have to worry about
% which |\catcode| that symbol had before.
%    \begin{macrocode}
%<*thp>
\begingroup \makeatletter
%    \end{macrocode}
% Since we are now within a group, we must make all definitions
% globally.  First we make sure that \texttt{theorem.sty} is loaded.
% This will allow us to use this file as a document style option
% without having to call \texttt{theorem} itself as an option.  At the
% same time, we assure that at least version 2 is loaded, since
% |\theorem@style| was not defined in earlier versions.
%    \begin{macrocode}
\@ifundefined{theorem@style}{\input{theorem.sty}}{}
\ProvidesFile{thp.sty}
             [\FMithmInfo]
%    \end{macrocode}
%
%
% \begin{macro}{\th@plain}
% \begin{macro}{\@begintheorem}
% \begin{macro}{\@opargbegintheorem}
%    |\theoremstyle{plain}| corresponds to the original
%    definition, except that the distances to the surrounding text are
%    determined by the parameters |\theorempreskipamount| and
%    |\theorempostskipamount|.
%    First we set the default body font.
%    \begin{macrocode}
\gdef\th@plain{\normalfont\itshape
%    \end{macrocode}
%    Then we define |\@begintheorem| and
%    |\@opargbegintheorem|. These two macros define how the
%    header of a theorem is typeset.  |\@opargbegintheorem| will
%    be called if a \textsf{theorem} environment with an optional
%    argument is encountered; otherwise, the header is constructed by
%    calling |\@begintheorem|.  If one of these macros is
%    executed, we are within a \textsf{trivlist} environment started by
%    |\@thm|.  So the theorem header is produced with an
%    |\item| command.
%
%    Instead of specifying the header font directly, all standard
%    theorem styles use the |\theorem@headerfont| macro to allow
%    customization.  The extra space (|\labelsep|) is necessary
%    because of problems in the \textsf{trivlist} environment.
%    \begin{macrocode}
  \def\@begintheorem##1##2{%
        \item[\hskip\labelsep \theorem@headerfont ##1\ ##2]}%
%    \end{macrocode}
%    The definition of |\@opargbegintheorem| is completely
%    analogous. The only difference is the fact that there exists a
%    third argument (which is the optional parameter of the
%    environment and contains additional information about the
%    theorem).  Customarily we enclose it in parentheses.
%    \begin{macrocode}
\def\@opargbegintheorem##1##2##3{%
   \item[\hskip\labelsep \theorem@headerfont ##1\ ##2\ (##3)]}}
%    \end{macrocode}
% \end{macro}
% \end{macro}
% \end{macro}
%
% \noindent
% We conclude with an |\endgroup| to restore the
% |\catcode| of |@|.
%    \begin{macrocode}
\endgroup
%</thp>
%    \end{macrocode}
%
%
% \subsubsection{The \texttt{break} style}
%
% This style option is stored in the file |thb.sty|.
% For the next two lines see the documentation for |\th@plain| on
% page~\pageref{dok:th@plain}.
%    \begin{macrocode}
%<*thb>
\begingroup \makeatletter
\@ifundefined{theorem@style}{\input{theorem.sty}}{}
\ProvidesFile{thb.sty}
             [\FMithmInfo]
%    \end{macrocode}
%
% \begin{macro}{\th@break}
%    |\theoremstyle{break}| produces a line break after the name
%    of the theorem.  The font is |\slshape|.  Hence, we define
%    |\th@break| as follows:
%    \begin{macrocode}
\gdef\th@break{\normalfont\slshape
  \def\@begintheorem##1##2{\item[%
%    \end{macrocode}
%    We run into the following problem: it is not possible to create
%    the header with |\item[|\meta{title}|]| and then start
%    a new line by, for example, |\mbox{}\\|. Such a definition
%    will fail whenever a list environment follows immediately.  With
%    the above construction, the |\mbox{}| causes the switch
%    |@inlabel| (cf.\ definition of |\list| and
%    |\trivlist| in \cite{latex:source}) to be set to \texttt{false}
%    and so the following list will insert additional vertical
%    space (|\topskip|). This is quite annoying. Therefore, we
%    create the line break within the |\item|. In order to ensure
%    that the text will begin at the proper position in the following
%    line, we simply pretend that the label does not take any
%    room.\footnote{This will lead to problems whenever very high
%    symbols occurring in the line tower into the
%    heading. So, something else has to be done here sometime.}
%    \begin{macrocode}
       \rlap{\vbox{\hbox{\hskip \labelsep\theorem@headerfont ##1\ ##2}%
                   \hbox{\strut}}}]}%
%    \end{macrocode}
%    Again, the definition of |\@opargbegintheorem| is
%    completely analogous.
%    \begin{macrocode}
\def\@opargbegintheorem##1##2##3{%
  \item[\rlap{\vbox{\hbox{\hskip \labelsep \theorem@headerfont
                     ##1\ ##2\ (##3)}%
                    \hbox{\strut}}}]}}
\endgroup
%</thb>
%    \end{macrocode}
% \end{macro}
%
%
% \subsubsection{The \texttt{changebreak} style}
%
%    \begin{macrocode}
% This style option is stored in the file |thcb.sty|.
%    \begin{macrocode}
%<*thcb>
\begingroup \makeatletter
\@ifundefined{theorem@style}{\input{theorem.sty}}{}
\ProvidesFile{thcb.sty}
             [\FMithmInfo]
%    \end{macrocode}
%
% \begin{macro}{\th@changebreak}
%    The \texttt{change-break} theorem style is like \texttt{break}
%    but with interchange of theorem name and theorem number.  Thus we
%    define |\th@changebreak| as follows:
%    \begin{macrocode}
\gdef\th@changebreak{\normalfont\slshape
  \def\@begintheorem##1##2{\item
       [\rlap{\vbox{\hbox{\hskip\labelsep \theorem@headerfont ##2\ ##1}%
                    \hbox{\strut}}}]}%
\def\@opargbegintheorem##1##2##3{%
  \item[\rlap{\vbox{\hbox{\hskip\labelsep \theorem@headerfont
                     ##2\ ##1\ (##3)}%
                    \hbox{\strut}}}]}}
\endgroup
%</thcb>
%    \end{macrocode}
% \end{macro}
%
%
% \subsubsection{The \texttt{change} style}
%
% This style option is stored in the file |thc.sty|.
%    \begin{macrocode}
%<*thc>
\begingroup \makeatletter
\@ifundefined{theorem@style}{\input{theorem.sty}}{}
\ProvidesFile{thc.sty}
             [\FMithmInfo]
%    \end{macrocode}
%
% \begin{macro}{\th@change}
%    The \texttt{change} theorem style corresponds to the
%    \texttt{change break} style without a linebreak after the header.
%    To say it in another way, it's the same as the \texttt{plain}
%    style but with number and name interchanged and |\slshape| as the
%    default font.
%    \begin{macrocode}
\gdef\th@change{\normalfont\slshape
  \def\@begintheorem##1##2{\item
       [\hskip\labelsep \theorem@headerfont ##2\ ##1]}%
\def\@opargbegintheorem##1##2##3{%
  \item[\hskip\labelsep \theorem@headerfont ##2\ ##1\ (##3)]}}
\endgroup
%</thc>
%    \end{macrocode}
% \end{macro}
%
%
% \subsubsection{The \texttt{marginbreak} style}
%
% This style option is the one used most often at Mainz.  It is saved
% in the file |thmb.sty|.
%    \begin{macrocode}
%<*thmb>
\begingroup \makeatletter
\@ifundefined{theorem@style}{\input{theorem.sty}}{}
\ProvidesFile{thmb.sty}
             [\FMithmInfo]
%    \end{macrocode}
%
% \begin{macro}{\th@marginbreak}
%    The \texttt{margin break} style is nearly the same as the
%    \texttt{change break} style. The only difference is the placement
%    of the theorem number.  We use |\llap| to place it in the left
%    margin.
%
%    In this style |\labelsep| denotes the separation between the
%    number and the text.
%    \begin{macrocode}
\gdef\th@marginbreak{\normalfont\slshape
  \def\@begintheorem##1##2{\item
       [\rlap{\vbox{\theorem@headerfont
                    \hbox{\llap{##2}\hskip\labelsep ##1}%
                    \hbox{\strut}}}]}%
\def\@opargbegintheorem##1##2##3{%
  \item[\rlap{\vbox{\theorem@headerfont
                    \hbox{\llap{##2}\hskip\labelsep ##1\ (##3)}%
                    \hbox{\strut}}}]}}
\endgroup
%</thmb>
%    \end{macrocode}
% \changes{v2.1c}{90/02/17}{`endgroup outside macrocode env.}
% \end{macro}
%
%
%
% \subsubsection{The \texttt{margin} style}
%
% This style option is stored in the file |thm.sty|.
%    \begin{macrocode}
%<*thm>
\begingroup \makeatletter
\@ifundefined{theorem@style}{\input{theorem.sty}}{}
\ProvidesFile{thm.sty}
             [\FMithmInfo]
%    \end{macrocode}
%
% \begin{macro}{\th@margin}
%    Again this is only a variant of the theorem styles described
%    above without any new ideas.
%    \begin{macrocode}
\gdef\th@margin{\normalfont\slshape
  \def\@begintheorem##1##2{\item
       [\theorem@headerfont \llap{##2}\hskip\labelsep ##1]}%
\def\@opargbegintheorem##1##2##3{%
  \item[\theorem@headerfont \llap{##2}\hskip\labelsep ##1\ (##3)]}}
\endgroup
%</thm>
%    \end{macrocode}
% \end{macro}
%
%
%
%
% \subsection{Final Definitions}
%
% \begin{macro}{\theorempreskipamount}
% \begin{macro}{\theorempostskipamount}
%    The \textsf{skip} parameters that regulate the vertical empty space
%    before and after the {\envfont theorem} environment have
%    to be allocated as well.
% \changes{v2.0f}{89/07/20}{Unnecessary `global' removed.}
%    \begin{macrocode}
%<*package>
\newskip\theorempreskipamount
\newskip\theorempostskipamount
%    \end{macrocode}
%    Since we have used the same values for all theorem sets, we now can
%    assign them.
%    \begin{macrocode}
\global\setlength\theorempreskipamount{12pt plus 5pt minus 3pt}
\global\setlength\theorempostskipamount{8pt plus 3pt minus 1.5pt}
%    \end{macrocode}
% \end{macro}
% \end{macro}
%
%
% \begin{macro}{\@endtheorem}
%    The same holds for the macro |\@endtheorem|, which ends a
%    {\envfont theorem} environment. Since it is the same for all
%    theorem sets, it is removed from the macros |\th@|\meta{style}.
%    It simply ends the {\envfont trivlist} environment, which was
%    begun in |\@thm|.
%    \begin{macrocode}
\global\let\@endtheorem=\endtrivlist
%    \end{macrocode}
% \end{macro}
%
%
%
% \begin{macro}{\@preamblecmds}
%    All macros defined above are to be used only in the preamble.
%    Therefore, we insert them in |\@preamblecmds| which will disable
%    them at begin document. This is done by the internal \LaTeXe{}
%    command |\@onlypreamble|.
% \changes{v2.0g}{89/09/19}{`theoremheaderfont and `theorembodyfont
%    added.}
% \changes{v2.2a}{94/02/03}{Use \LaTeXe{} interface}
%    \begin{macrocode}
\@onlypreamble\@xnthm
\@onlypreamble\@ynthm
\@onlypreamble\@othm
\@onlypreamble\newtheorem
\@onlypreamble\theoremstyle
\@onlypreamble\theorembodyfont
\@onlypreamble\theoremheaderfont
%    \end{macrocode}
% \end{macro}
%
% Finally we declare the \texttt{plain} theorem style to be the default.
% \changes{v2.1b}{90/06/04}{Fixed missing `item by declaring default
%    style.}
%    \begin{macrocode}
\theoremstyle{plain}
%</package>
%    \end{macrocode}
%
% \Finale

\endinput
}{}
\ProvidesFile{thm.sty}
             [\FMithmInfo]
%    \end{macrocode}
%
% \begin{macro}{\th@margin}
%    Again this is only a variant of the theorem styles described
%    above without any new ideas.
%    \begin{macrocode}
\gdef\th@margin{\normalfont\slshape
  \def\@begintheorem##1##2{\item
       [\theorem@headerfont \llap{##2}\hskip\labelsep ##1]}%
\def\@opargbegintheorem##1##2##3{%
  \item[\theorem@headerfont \llap{##2}\hskip\labelsep ##1\ (##3)]}}
\endgroup
%</thm>
%    \end{macrocode}
% \end{macro}
%
%
%
%
% \subsection{Final Definitions}
%
% \begin{macro}{\theorempreskipamount}
% \begin{macro}{\theorempostskipamount}
%    The \textsf{skip} parameters that regulate the vertical empty space
%    before and after the {\envfont theorem} environment have
%    to be allocated as well.
% \changes{v2.0f}{89/07/20}{Unnecessary `global' removed.}
%    \begin{macrocode}
%<*package>
\newskip\theorempreskipamount
\newskip\theorempostskipamount
%    \end{macrocode}
%    Since we have used the same values for all theorem sets, we now can
%    assign them.
%    \begin{macrocode}
\global\setlength\theorempreskipamount{12pt plus 5pt minus 3pt}
\global\setlength\theorempostskipamount{8pt plus 3pt minus 1.5pt}
%    \end{macrocode}
% \end{macro}
% \end{macro}
%
%
% \begin{macro}{\@endtheorem}
%    The same holds for the macro |\@endtheorem|, which ends a
%    {\envfont theorem} environment. Since it is the same for all
%    theorem sets, it is removed from the macros |\th@|\meta{style}.
%    It simply ends the {\envfont trivlist} environment, which was
%    begun in |\@thm|.
%    \begin{macrocode}
\global\let\@endtheorem=\endtrivlist
%    \end{macrocode}
% \end{macro}
%
%
%
% \begin{macro}{\@preamblecmds}
%    All macros defined above are to be used only in the preamble.
%    Therefore, we insert them in |\@preamblecmds| which will disable
%    them at begin document. This is done by the internal \LaTeXe{}
%    command |\@onlypreamble|.
% \changes{v2.0g}{89/09/19}{`theoremheaderfont and `theorembodyfont
%    added.}
% \changes{v2.2a}{94/02/03}{Use \LaTeXe{} interface}
%    \begin{macrocode}
\@onlypreamble\@xnthm
\@onlypreamble\@ynthm
\@onlypreamble\@othm
\@onlypreamble\newtheorem
\@onlypreamble\theoremstyle
\@onlypreamble\theorembodyfont
\@onlypreamble\theoremheaderfont
%    \end{macrocode}
% \end{macro}
%
% Finally we declare the \texttt{plain} theorem style to be the default.
% \changes{v2.1b}{90/06/04}{Fixed missing `item by declaring default
%    style.}
%    \begin{macrocode}
\theoremstyle{plain}
%</package>
%    \end{macrocode}
%
% \Finale

\endinput
}{}
\ProvidesFile{thm.sty}
             [\FMithmInfo]
%    \end{macrocode}
%
% \begin{macro}{\th@margin}
%    Again this is only a variant of the theorem styles described
%    above without any new ideas.
%    \begin{macrocode}
\gdef\th@margin{\normalfont\slshape
  \def\@begintheorem##1##2{\item
       [\theorem@headerfont \llap{##2}\hskip\labelsep ##1]}%
\def\@opargbegintheorem##1##2##3{%
  \item[\theorem@headerfont \llap{##2}\hskip\labelsep ##1\ (##3)]}}
\endgroup
%</thm>
%    \end{macrocode}
% \end{macro}
%
%
%
%
% \subsection{Final Definitions}
%
% \begin{macro}{\theorempreskipamount}
% \begin{macro}{\theorempostskipamount}
%    The \textsf{skip} parameters that regulate the vertical empty space
%    before and after the {\envfont theorem} environment have
%    to be allocated as well.
% \changes{v2.0f}{89/07/20}{Unnecessary `global' removed.}
%    \begin{macrocode}
%<*package>
\newskip\theorempreskipamount
\newskip\theorempostskipamount
%    \end{macrocode}
%    Since we have used the same values for all theorem sets, we now can
%    assign them.
%    \begin{macrocode}
\global\setlength\theorempreskipamount{12pt plus 5pt minus 3pt}
\global\setlength\theorempostskipamount{8pt plus 3pt minus 1.5pt}
%    \end{macrocode}
% \end{macro}
% \end{macro}
%
%
% \begin{macro}{\@endtheorem}
%    The same holds for the macro |\@endtheorem|, which ends a
%    {\envfont theorem} environment. Since it is the same for all
%    theorem sets, it is removed from the macros |\th@|\meta{style}.
%    It simply ends the {\envfont trivlist} environment, which was
%    begun in |\@thm|.
%    \begin{macrocode}
\global\let\@endtheorem=\endtrivlist
%    \end{macrocode}
% \end{macro}
%
%
%
% \begin{macro}{\@preamblecmds}
%    All macros defined above are to be used only in the preamble.
%    Therefore, we insert them in |\@preamblecmds| which will disable
%    them at begin document. This is done by the internal \LaTeXe{}
%    command |\@onlypreamble|.
% \changes{v2.0g}{89/09/19}{`theoremheaderfont and `theorembodyfont
%    added.}
% \changes{v2.2a}{94/02/03}{Use \LaTeXe{} interface}
%    \begin{macrocode}
\@onlypreamble\@xnthm
\@onlypreamble\@ynthm
\@onlypreamble\@othm
\@onlypreamble\newtheorem
\@onlypreamble\theoremstyle
\@onlypreamble\theorembodyfont
\@onlypreamble\theoremheaderfont
%    \end{macrocode}
% \end{macro}
%
% Finally we declare the \texttt{plain} theorem style to be the default.
% \changes{v2.1b}{90/06/04}{Fixed missing `item by declaring default
%    style.}
%    \begin{macrocode}
\theoremstyle{plain}
%</package>
%    \end{macrocode}
%
% \Finale

\endinput
}{}
\ProvidesFile{thcb.sty}
             [\FMithmInfo]
%    \end{macrocode}
%
% \begin{macro}{\th@changebreak}
%    The \texttt{change-break} theorem style is like \texttt{break}
%    but with interchange of theorem name and theorem number.  Thus we
%    define |\th@changebreak| as follows:
%    \begin{macrocode}
\gdef\th@changebreak{\normalfont\slshape
  \def\@begintheorem##1##2{\item
       [\rlap{\vbox{\hbox{\hskip\labelsep \theorem@headerfont ##2\ ##1}%
                    \hbox{\strut}}}]}%
\def\@opargbegintheorem##1##2##3{%
  \item[\rlap{\vbox{\hbox{\hskip\labelsep \theorem@headerfont
                     ##2\ ##1\ (##3)}%
                    \hbox{\strut}}}]}}
\endgroup
%</thcb>
%    \end{macrocode}
% \end{macro}
%
%
% \subsubsection{The \texttt{change} style}
%
% This style option is stored in the file |thc.sty|.
%    \begin{macrocode}
%<*thc>
\begingroup \makeatletter
\@ifundefined{theorem@style}{% \iffalse meta-comment
%
% Copyright (C) 1993-2024
%
% The LaTeX Project and any individual authors listed elsewhere
% in this file.
%
% This file is part of the Standard LaTeX `Tools Bundle'.
% -------------------------------------------------------
%
% It may be distributed and/or modified under the
% conditions of the LaTeX Project Public License, either version 1.3c
% of this license or (at your option) any later version.
% The latest version of this license is in
%    https://www.latex-project.org/lppl.txt
% and version 1.3c or later is part of all distributions of LaTeX
% version 2005/12/01 or later.
%
% The list of all files belonging to the LaTeX `Tools Bundle' is
% given in the file `manifest.txt'.
%
% \fi
%
%%
%
%
\def\FMithmInfo{2023/07/05 v2.2c Theorem extension package (FMi)}
%
% \ProvidesFile{theorem.dtx}[\FMithmInfo]
%\iffalse   % this is a METACOMMENT !
%%
%% Package `theorem' to use with LaTeX2e
%% Copyright (C) 1989-1995 Frank Mittelbach, all rights reserved.
%<+package>\@ifundefined{theorem@style}{}{\endinput}
%<+package>\NeedsTeXFormat{LaTeX2e}[1995/06/01]
%<+package>\ProvidesPackage{theorem}
%<+package>         [\FMithmInfo]
%
% \fi
%
% \changes{v1.0}{87/05/07}{First draft.}
% \changes{v1.0a}{87/05/07}{`thistheoremstyle' added.}
% \changes{v1.1a}{87/07/29}{New concept: redefine the internal
%                          \protect\LaTeX{} commands `@xnthm' etc.}
% \changes{v1.1b}{87/07/31}{`theoremfont' used to specify font.}
% \changes{v1.2a}{88/01/08}{`@endtheorem' now globally defined.}
% \changes{v1.2b}{88/01/11}{`newtheorem' only allowed in the preamble.}
% \changes{v1.3a}{88/02/14}{`theoremfont' now a toks register.}
%
% \changes{v2.0a}{88/11/16}{New implementation and new documentation.}
% \changes{v2.0a}{88/11/16}{No layout changes but internal modifications
%                           old substyles won't work.}
% \changes{v2.0b}{89/01/30}{`global' used for all definitions.}
% \changes{v2.0c}{89/06/15}{Documentation translated to English.
%                            Thanks to LLL and Nina}
% \changes{v2.1a}{90/02/15}{Support for new font selection.}
% \changes{v2.2a}{94/02/03}{Cleanup for LaTeX2e}
%
% \DoNotIndex{\@endparenv,\@ifdefinable,\@ifnextchar,\@ifundefined}
% \DoNotIndex{\@namedef,\@nameuse,\@spaces,\@tempa,\@warning}
% \DoNotIndex{\ }
% \DoNotIndex{\begingroup,\bfseries}
% \DoNotIndex{\csname}
% \DoNotIndex{\def,\do,\docdate}
% \DoNotIndex{\endcsname,\endgroup,\endinput,\endtrivlist,\expandafter}
% \DoNotIndex{\filedate,\fileversion}
% \DoNotIndex{\gdef,\global}
% \DoNotIndex{\hbox,\hskip}
% \DoNotIndex{\ignorespaces,\input,\itshape,\item}
% \DoNotIndex{\labelsep,\let,\llap}
% \DoNotIndex{\makeatletter}
% \DoNotIndex{\noexpand}
% \DoNotIndex{\refstepcounter,\relax,\rlap}
% \DoNotIndex{\setlength,\slshape,\space,\string,\strut}
% \DoNotIndex{\the,\trivlist,\typeout}
% \DoNotIndex{\vbox}
% \DoNotIndex{\xdef}
%
% \def\deflabel#1{\ttfamily #1\hfill}
% \def\deflist#1{\begin{list}{}{\settowidth\labelwidth{\ttfamily #1}%
%                               \setlength\leftmargin\labelwidth
%                               \addtolength\leftmargin\labelsep
%                               \let\makelabel\deflabel}}
% \def\enddeflist{\end{list}}
% \def\envfont{\normalfont\ttfamily}
%
% \MakeShortVerb{\|}
%
% \newcommand{\TUB}{{\sl TUGboat\/}}
%
% \GetFileInfo{theorem.dtx}
% \title{An Extension of the
%        \LaTeX{} theorem environment\thanks{This file
%        has version number \fileversion, last
%        revised \filedate.}}
% \author{Frank Mittelbach}
%
% \MaintainedByLaTeXTeam{tools}
% \maketitle
%
% \begin{abstract}
%    The macros described in this paper yield an extension of the
%    \LaTeX{} theorem mechanism. It is designed to satisfy the
%    different requirements of various journals. Thus, the
%    layout of the ``theorems'' can be manipulated by determining a
%    ``style''. This article describes not only the use, but
%    also the definition, of the necessary macros.
% \end{abstract}
%
% \section*{Preface to version 2.2}
%
% For \LaTeXe{} this package did not need any fundamental changes. I
% only modified the messages generated so that theorem layout styles
% will show up with the |\listfiles| command and cleaned the section
% on the New Font Selection Scheme since this is now included in
% \LaTeX.
%
% \section*{Preface to version 2.1}
%
% This version is identical to 2.0g described in \TUB\ 10\#3 except
% for some internal defaults which are now set depending on the used
% font selection scheme.
%
% This was done to avoid unpleasant surprises if the new font selection
% scheme is in force. For further details see
% section~\ref{sec:newfonts} and \cite{bk:GMS94}.
%
% \section{Introduction}
%
% For our purposes here, ``theorems'' are labelled enunciations,
% often set off from the main text by extra space and a font change.
% Theorems, corollaries, conjectures, definitions, and remarks
% are all instances of ``theorems''.  The ``header'' of these structures
% is composed of a label (such as \textsc{Theorem} or \textsc{Remark})
% and a number which serializes an item in the sequence of items
% with the same label.
%
% Shortly after the introduction of \LaTeX{} at the Fachbereich
% Mathematik in Mainz, the desire to manipulate the layout of
% ``theorems'' arose.  In Mainz, the following two conventions came
% into general use:
% \begin{enumerate}
%   \item The number of the theorem is shown in the margin.
%   \item There is a line break at
%         the end of the theorem header.
% \end{enumerate}
%
% Additionally, some journals require different formats which depend on
% the ``sort of theorem'': e.g.\ often remarks and definitions are set
% in |\upshape|, while |\itshape| is employed for main theorems.
%
% Confronted with these requirements, a theorem environment was
% developed in Mainz which allows separate determination of the
% layout of the ``theorems sets'', comparable to |\pagestyle|.
%
%
% \section{The user interface}
% \subsection{Defining new theorem sets}
%
% \DescribeMacro\newtheorem
% As in the original \LaTeX{} version, the command |\newtheorem|
% defines a new ``theorem set'' or ``theorem-like structure''.
% Two required arguments name the new environment and give the text
% to be typeset with each instance of the new ``set'', while
% an optional argument determines how
% the ``set'' is enumerated:
% \begin{description}
%    \item[\ttfamily \bslash newtheorem\{foo\}\{bar\}]
%       The theorem set {\envfont foo} (whose name is \texttt{bar})
%       uses its own counter.
%    \item[\ttfamily \bslash newtheorem\{foo2\}{[foo]}\{bar2\}]
%       The theorem set {\envfont foo2} (printed name \texttt{bar2})
%       uses the same counter as the theorem set \texttt{foo}.
%    \item[\ttfamily \bslash newtheorem\{foo3\}\{bar3\}{[section]}]
%       The theorem set {\envfont foo3} (printed name \texttt{bar3}) is
%       enumerated within the counter \texttt{section}, i.e.\ with every
%       new |\section| the enumeration begins again with 1, and
%       the enumeration is composed from the section-number and the
%       theorem counter itself.
% \end{description}
%
% \DescribeMacro\theoremstyle
% Additionally, the command |\theoremstyle| can define the layout
% of various, or all, theorem sets.  It should be noted that any
% theorem set defined by |\newtheorem| is typeset in the
% |\theoremstyle| that is current at the time of the definition.
% Thus, the following
% \begin{verbatim}
%    \theoremstyle{break}        \newtheorem{Cor}{Corollary}
%    \theoremstyle{plain}        \newtheorem{Exa}{Example}[section]
%\end{verbatim}                             ^^A No blank at line start
% leads to the result that the set {\envfont Cor} is formatted in the
% style \texttt{break}, while the set {\envfont Exa} and all the
% following ones are formatted in the style \texttt{plain}, unless
% another |\theoremstyle| follows.  Since the definitions installed by
% |\newtheorem| are global, one also can limit |\theoremstyle| locally
% by grouping braces.
%
% \DescribeMacro\theorembodyfont
% The choice of the font for the theorem body is completely
% independent of the chosen |\theoremstyle|; this has proven to
% be very advantageous.  For example,
% \begin{verbatim}
%   {\theorembodyfont{\upshape}          \newtheorem{Rem}{Remark}}
%\end{verbatim}
% defines a theorem set {\envfont Rem}, which will be set in
% |\upshape| in the current layout (which in our example is
% \texttt{plain}).  As with |\theoremstyle|, the |\theorembodyfont|
% chosen is that current at the time of |\newtheorem|.  If
% |\theorembodyfont| is not specified or one defines
% |\theorembodyfont{}|, then the font used will be that defined by the
% |\theoremstyle|.
%
% \DescribeMacro\theoremheaderfont It is also possible to customize
% the font used for the theorem headers.  This is, however, a global
% declaration, and therefore there should be at most one
% |\theoremheaderfont| declaration in the preamble.\footnote{If it is
% actually necessary to have different header fonts, one has to define
% new theorem styles (substituting the desired font) or specify the
% information directly in the \texttt{\bslash newtheorem} declaration
% (the unclean variant).}
%
% \DescribeMacro\theorempreskipamount
% \DescribeMacro\theorempostskipamount Two additional parameters
% affect the vertical space around the theorem environments:
% |\theorempreskipamount| and |\theorempostskipamount| define,
% respectively, the spacing before and after such an environment.
% These parameters apply for all theorem sets and can be manipulated
% with the ordinary length macros.  They are rubber lengths
% (`\textsf{skips}'), and therefore can contain \texttt{plus} and
% \texttt{minus} parts.
%
% Since the definition of theorem sets should---most sensibly---be
% placed in the preamble, we only allow installation there.
% It is therefore possible to release the memory used here after
% |\begin{document}|, in order to make room for other
% applications.
%
%
% \subsection{Existing theorem styles}
%
% The following theorem styles exist to date:
% \begin{deflist}{marginbreak}
%    \item[plain]
%       This theorem style emulates the original \LaTeX{} definition,
%       except that additionally the parameters
%       |\theorem...skipamount| are used.
%    \item[break]
%       In this style, the theorem header is followed by a line
%       break.
%    \item[marginbreak]
%       The theorem number is set in the margin, and there is a line
%       break as in \texttt{break}.
%    \item[changebreak]
%       Like \texttt{break}, but with header number and text
%       interchanged.
%    \item[change]
%       Header number and text are interchanged, without a line break.
%    \item[margin]
%       The number is set in the left margin, without a line break.
% \end{deflist}
% All styles (except \texttt{plain}) select |\slshape| as the default
% |\theorembodyfont|.
%
%
% \subsection{Examples}
%
% Given the above theorem sets {\envfont Cor}, {\envfont Exa}
% and {\envfont Rem},
% suppose that the preamble also contains the declarations:
% \begin{verbatim}
%    \theoremstyle{marginbreak}   \newtheorem{Lem}[Cor]{Lemma}
%    \theoremstyle{change}
%    \theorembodyfont{\itshape}        \newtheorem{Def}[Cor]{Definition}
%
%    \theoremheaderfont{\scshape}
%\end{verbatim}
% Then the following are some typical examples of the typeset output
% resulting from their use.
%
% \begin{Cor}
%    This is a sentence typeset in the theorem environment {\envfont
%    Cor}.
% \end{Cor}
%
% \begin{Exa}
%    This is a sentence typeset in the theorem environment {\envfont
%    Exa}.
% \end{Exa}
%
% \begin{Rem}
%    This is a sentence typeset in the theorem environment {\envfont
%    Rem}.
% \end{Rem}
%
% \begin{Lem}[Ben User]
%    This is a sentence typeset in the theorem environment {\envfont
%    Lem}.
% \end{Lem}
%
% \begin{Def}[Very impressive Definition]
%    This is a sentence typeset in the theorem environment {\envfont
%    Def}.
% \end{Def}
%
% The last two examples show the effect of the optional argument to
% a theorem environment (it is the text typeset in parentheses).
%
%
% \section{Special Considerations}
% \label{sec:newfonts}
%
% Theoremheader and body are implemented as a unit. This means that
% the |\theoremheaderfont|  will inherit characteristics of the
% |\theorembodyfont| in \LaTeXe. Thus, if for example
% |\theorembodyfont| is |\itshape| and |\theoremheaderfont|
% is |\bfseries| the font selected for the header will have the
% characteristics `bold extended italic'. If this is not desired one
% should set the |\theoremheaderfont| to something like
% \begin{verbatim}
%      \theoremheaderfont{\normalfont\bfseries}
%\end{verbatim}
% i.e.\ supplying all necessary font information explicitly.
%
%
%
% \section{Acknowledgements}
%
% The publication of this set of macros was only possible with the
% help of Christina Busse (translating the manuscript into English),
% Joachim Pense (playing the r\^ole of typist), Chris Rowley
% (looking everything over) and many others providing useful
% suggestions.
%
% \MaybeStop{%
%    \begin{thebibliography}{9}
%       \bibitem{bk:GMS94} \textsc{M.~Goossens}, \textsc{F.~Mittelbach}
%         and \textsc{A.~Samarin}.
%         \newblock The \LaTeX{} Companion.
%         \newblock
%         Addison-Wesley, Reading, Massachusetts, 1994.
%       \bibitem{latex:source} \textsc{Lamport, Leslie.} \newblock
%                               \texttt{latex.tex}, version 2.09, date
%                               Feb. 1990.
%    \end{thebibliography}
%    \PrintIndex
%    \PrintChanges}
%
%
% \section{The documentation driver file}
%
% The next bit of code contains the documentation driver file for
% \TeX{}, i.e., the file that will produce the documentation you are
% currently reading. It will be extracted from this file by the
% \texttt{docstrip} program. Since it is the first code in the file
% one can alternatively process this file directly with \LaTeXe{} to
% obtain the documentation.
%
%    \begin{macrocode}
%<*driver>
\documentclass{ltxdoc}

\usepackage{theorem}

%<+driver>% The next few lines define theorem sets which are used
%<+driver>% in the example section of the documentation.

\theoremstyle{break}       \newtheorem{Cor}{Corollary}
\theoremstyle{plain}       \newtheorem{Exa}{Example}[section]
{\theorembodyfont{\upshape}\newtheorem{Rem}{Remark}}
\theoremstyle{marginbreak} \newtheorem{Lem}[Cor]{Lemma}
\theoremstyle{change}
\theorembodyfont{\itshape} \newtheorem{Def}[Cor]{Definition}

\theoremheaderfont{\scshape}

\RecordChanges

\begin{document}
   \DocInput{theorem.dtx}
\end{document}
%</driver>
%    \end{macrocode}
%
% \section{Definition of the Macros}
%
% If the file has been loaded before, we abort immediately. If not the
% package announces itself (this is actually done at the very top of
% the file---the way it is done isn't good style so don't copy it).
% \changes{v2.0e}{89/07/19}{Spaces removed from `typeout'.}
% \changes{v2.2b}{95/11/19}{Announce moved to top of file}
%    \begin{macrocode}
%<*package>
%\@ifundefined{theorem@style}{}{\endinput}
%\def\FMithmInfo{1995/11/19 v2.2b Theorem extension package (FMi)}
%\ProvidesPackage{theorem}[\FMithmInfo]
%    \end{macrocode}
%
%
% \subsection{Definition of theorem styles and fonts}
%
% All the definitions in this file are done globally to allow
% inputting this file inside a group.
%
% \begin{macro}{\theoremstyle}
%    Before a theorem style can be installed, the chosen style must be
%    known. For that reason, we must test to see that
%    |\th@|\meta{style} is known or, more precisely, that it is
%    different from |\relax|. If the style is not known then
%    |\th@plain| is used.
%    \begin{macrocode}
\gdef\theoremstyle#1{%
   \@ifundefined{th@#1}{\@warning
          {Unknown theoremstyle `#1'. Using `plain'}%
          \theorem@style{plain}}%
%    \end{macrocode}
%    We save the theorem style to be used in the token register
%    |\theorem@style|.
%    \begin{macrocode}
      {\theorem@style{#1}}%
%    \end{macrocode}
%    Now we ``evaluate'' the theorem style: this means, we call the
%    macro |\th@|\meta{style} which will activate the relevant
%    definitions which are contained in a separate file.  This is done
%    in a group to suppress changes to the current font.  This could
%    otherwise pose problems together with the new font selection
%    scheme\footnote{When I printed the original article using the new
%    font selection scheme I ended with a document with slanted
%    typefaces (text headings and all) simply because one of the
%    theorem styles used \texttt{\bslash sl} at toplevel.} if the
%    |\th@|\meta{style} is evaluated a second time.
% \changes{v2.1a}{90/02/15}{Group added around `th@..}
% \changes{v2.1a}{90/02/15}{Reseting of `..@begintheorem removed.}
%    \begin{macrocode}
      \begingroup
        \csname th@\the\theorem@style \endcsname
      \endgroup}
%    \end{macrocode}
% \end{macro}
%
%
% \begin{macro}{\@begintheorem}
% \changes{v2.1a}{90/02/15}{Relaxed at toplevel.}
% \begin{macro}{\@opargbegintheorem}
% \changes{v2.1a}{90/02/15}{Relaxed at toplevel.}
%    We reset |\@begintheorem| and |\@opargbegintheorem| to
%    |\relax| since these commands are no longer necessary at
%    toplevel. This will save a few tokens.
%    \begin{macrocode}
\global\let\@begintheorem\relax
\global\let\@opargbegintheorem\relax
%    \end{macrocode}
% \end{macro}
% \end{macro}
%
% \begin{macro}{\theorem@style}
%    Obviously the \textsf{token} register used above has to be
%    allocated. To assure the utmost compatibility with the
%    original \LaTeX{} definition, we set the default theorem style to
%    \texttt{plain}, which implements the usual \LaTeX{} convention.
% \changes{v2.0f}{89/07/20}{Unnecessary `global' removed.}
%    \begin{macrocode}
\newtoks\theorem@style
\global\theorem@style{plain}
%    \end{macrocode}
% \end{macro}
%
% \begin{macro}{\theorembodyfont}
% \begin{macro}{\theorem@bodyfont}
%    If the |\theorembodyfont| is set by the user then it should not
%    interact with the default font set in the theorem style. When the
%    new font selection is in force this may happen if, for example,
%    the default is |\itshape| and the new |\theorembodyfont| is
%    |\sffamily|. So we add a |\reset@font| command in front of the
%    user definition.
% \changes{v2.0f}{89/07/20}{Unnecessary `global' removed.}
% \changes{v2.0g}{89/09/19}{`theoremfont to `theorembodyfont.}
% \changes{v2.1c}{92/06/22}{`reset@font added and logic changed.}
% \changes{v2.1d}{92/10/28}{Reset to empty if arg is empty}
%    \begin{macrocode}
\gdef\theorembodyfont#1{%
%    \end{macrocode}
%    We check if the argument supplied is empty and if so put nothing
%    into the |\theorem@bodyfont| token register to allow for
%    |\theorembodyfont{}| as a mean of using the default of the
%    current |\theoremstyle|.
%    \begin{macrocode}
   \def\@tempa{#1}%
   \ifx\@tempa\@empty
    \theorem@bodyfont{}%
   \else
    \theorem@bodyfont{\reset@font#1}%
   \fi
   }
\newtoks\theorem@bodyfont
\global\theorem@bodyfont{}
%    \end{macrocode}
% \end{macro}
% \end{macro}
%
%
% \begin{macro}{\theoremheaderfont}
% \changes{v2.0g}{89/09/19}{Macro added.}
%    The font for the theorem headers is handled differently because
%    this definition applies to all theorem styles.
%    \begin{macrocode}
\gdef\theoremheaderfont#1{\gdef\theorem@headerfont{#1}%
%    \end{macrocode}
%    After using the macro once it is redefined to produce an error
%    message.
%    \begin{macrocode}
       \gdef\theoremheaderfont##1{%
        \typeout{\string\theoremheaderfont\space should be used
                 only once.}}}
%    \end{macrocode}
% \end{macro}
%
% \begin{macro}{\theorem@headerfont}
% \changes{v2.0g}{89/09/19}{Macro added.}
%    To set the |\theorem@headerfont| default we first test if
%    the new fontselection scheme is in force.
% \changes{v2.1a}{90/02/15}{Support for new font selection.}
%    \begin{macrocode}
\ifx\upshape\undefined
%    \end{macrocode}
%    If not we define it to expand into |\bfseries|. We don't use
%    |\let| just in case a following style option redefines this
%    macro.
%    \begin{macrocode}
\gdef\theorem@headerfont{\bfseries}
%    \end{macrocode}
%    Otherwise we reset the current shape before calling |\bfseries|.
%    \begin{macrocode}
\else \gdef\theorem@headerfont{\normalfont\bfseries}\fi
%    \end{macrocode}
% \end{macro}
%
% \begin{macro}{\th@plain}
% \begin{macro}{\th@break}
% \begin{macro}{\th@marginbreak}
% \begin{macro}{\th@changebreak}
% \begin{macro}{\th@change}
% \begin{macro}{\th@margin}
%    The different styles are defined in macros such as
%    |\th@plain|.  Since memory space is precious in
%    ``non-Big-versions'', we have to avoid offering too many unused
%    definitions.  Therefore we define these styles in separate files
%    that can be loaded on demand.  Thus the commands themselves only
%    load these files. We use |\@input@|, a \LaTeXe{} internal command
%    that ensures that the file will be listed with |\listfiles|.
% \changes{v2.2a}{94/02/02}{Use `@input@ to load theorem layout files}
%    \begin{macrocode}
\gdef\th@plain{\@input@{thp.sty}}
\gdef\th@break{\@input@{thb.sty}}
\gdef\th@marginbreak{\@input@{thmb.sty}}
\gdef\th@changebreak{\@input@{thcb.sty}}
\gdef\th@change{\@input@{thc.sty}}
\gdef\th@margin{\@input@{thm.sty}}
%    \end{macrocode}
%    This list will be expanded when new styles become
%    available.  For testing,  just append new theorem substyles
%    as document options.
% \end{macro}
% \end{macro}
% \end{macro}
% \end{macro}
% \end{macro}
% \end{macro}
%
%
%
% \subsection{Definition of a new theorem set}
%
% As already pointed out, a new theorem environment can be defined in
% three different ways:
% \begin{verbatim}
%    \newtheorem{Lem}{Lemma}
%    \newtheorem{Lem}{Lemma}[section]
%    \newtheorem{Lem}[Theorem]{Lemma}
%\end{verbatim}
% The function of the macro |\newtheorem| is to recognize these cases
% and then to branch into one of the three macros |\@ynthm|, |\@xnthm|
% or |\@othm|.  This mechanism is adopted unchanged from
% \cite{latex:source}; the essential point here is that, for example,
% in the second case, the arguments \texttt{Lem}, \texttt{Lemma} and
% \texttt{section} are passed over to the macro |\@xnthm|.
%
% We inspect this case first because the others present fewer problems,
% and thus are easily derived from this one.
%
% \begin{macro}{\@xnthm}
%    For our example arguments, the macro |\@xnthm| must fulfill
%    the following:
%    \begin{itemize}
%       \item
%          Define a new \LaTeX-counter `Lem'
%       \item
%          reset this counter within a |\section|
%       \item
%          define the macro |\theLem|
%       \item
%          define the environment macros |\Lem| and
%          |\endLem| using the current
%          |\theoremstyle| and |\theorem@bodyfont|.
%    \end{itemize}
%    Obviously, all this should happen only if the first argument
%    of |\@xnthm| (i.e.\ \texttt{Lem} in our example) is chosen so
%    as not to conflict with any previously defined commands or
%    environments.  This test is performed by the \LaTeX{} macro
%    |\@ifdefinable|.
%    \begin{macrocode}
\gdef\@xnthm#1#2[#3]{\expandafter\@ifdefinable\csname #1\endcsname
%    \end{macrocode}
%    Therefore, the first argument of |\@ifdefinable| is the
%    expansion (in the example, |\Lem|) of
%    |\csname||#1\endcsname|.  The second argument is
%    executed only if the test has been completed successfully.
%    \begin{macrocode}
   {%
%    \end{macrocode}
%    Now we define the new counter. The names of the \LaTeX{}
%    macros employed should speak for themselves:
% \changes{v2.2b}{1995/10/07}{Use \cs{@newctr}}
% \changes{v2.2c}{1995/11/23}{Use \cs{@newctr} properly.}
%    \begin{macrocode}
    \@definecounter{#1}\@newctr{#1}[#3]%
%    \end{macrocode}
%    Using |\@newctr| will give a proper error message if the counter
%    in |#3| is not defined.
%    In defining `|\theLem|' we must generate the desired macro
%    name by use of |\expandafter| and |\csname|.
%    \begin{macrocode}
    \expandafter\xdef\csname the#1\endcsname
%    \end{macrocode}
%    An |\xdef| is used in order to make the definition global,
%    and to ensure that it contains the replacement texts of
%    |\@thmcountersep| and |\@thmcounter|.\footnote{These
%    two macros can be defined by the document style.  Their default
%    values produce a `\texttt{.}' as separation and an arabic
%    representation of the number.} However, not everything should be
%    expanded. For example, it saves space to use |\thesection|
%    instead of its---at times---lengthy expansion.
%    \begin{macrocode}
      {\expandafter \noexpand \csname the#3\endcsname
       \@thmcountersep \@thmcounter{#1}}%
%    \end{macrocode}
%    Thus with the defaults of \LaTeX{}, |\theLem| would be
%    replaced by the command sequence |\thesection.\arabic{Lem}|.
%
%    We will now look at the definition of the macro which is executed
%    at the beginning of the actual environment (in our example this
%    macro is |\Lem|). It should be noted that we use an
%    ``|\expandafter| trick'' to expand only certain parts of the
%    replacement text at the time of the definition.
%    \begin{macrocode}
    \def\@tempa{\global\@namedef{#1}}%
    \expandafter \@tempa \expandafter{%
%    \end{macrocode}
%    First, the macro that contains the current definitions of
%    |\@begintheorem| and |\@opargtheorem| should be called
%    up.  The name of this macro---as is already known---has the form
%    |\th@|\meta{theorem\ style}; therefore, it must be called by
%    \begin{macrocode}
      \csname th@\the \theorem@style
%    \end{macrocode}
%    In addition the default theorem font should be changeable, i.e.\
%    we have to insert the contents of |\theorem@bodyfont|. For
%    that reason, we expand even further, beyond |\endcsname|,
%    and thus insert the contents of the \textsf{token} register
%    |\theorem@bodyfont| in the replacement text.
%    \begin{macrocode}
            \expandafter \endcsname \the \theorem@bodyfont
%    \end{macrocode}
%    Now it is time to call the macro |\@thm| which takes over
%    the further processing.  It has two arguments: the current
%    counter name (in our example, \texttt{Lem}), and the text of the
%    label (in our example, \texttt{Lemma}).
%    \begin{macrocode}
     \@thm{#1}{#2}}%
%    \end{macrocode}
%    With this, the `sub-definition' is complete.  The macro
%    |\@endtheorem| ends a theorem environment and is, so far, nothing
%    but an |\endtrivlist|.  (Hence it is defined globally, and not
%    within the theorem styles.\footnote{This has to be changed as
%    soon as theorem styles that change \texttt{\bslash @endtheorem}
%    exist.  In such a case, all existing styles must be changed as
%    well since they will have to reset the macro.}) Therefore, we can
%    set it equivalent to the macro that ends the theorem set (in our
%    example, |\endLem|).  However, if some day theorem styles exist
%    that do change |\@endtheorem|, we would have to use the
%    commented-out line instead.
%    \begin{macrocode}
    \global \expandafter \let \csname end#1\endcsname \@endtheorem
%   \global\@namedef{end#1}{\@endtheorem}%
%    \end{macrocode}
%    With these commands all the required definitions are employed,
%    unless the test |\@ifdefinable| has failed.  Therefore, we
%    end the second argument of this macro and with it the definition
%    of |\@xnthm|.
%    \begin{macrocode}
   }}
%    \end{macrocode}
% \end{macro}
%
% \begin{macro}{\@ynthm}
%    The definition of |\@ynthm| is completely analogous. In
%    this case the new counter that is defined is not reset within
%    another counter; thus the definition of |\the...|
%    is simplified:
%    \begin{macrocode}
\gdef\@ynthm#1#2{\expandafter\@ifdefinable\csname #1\endcsname
   {\@definecounter{#1}%
    \expandafter\xdef\csname the#1\endcsname{\@thmcounter{#1}}%
%    \end{macrocode}
%    The rest of the definition corresponds literally to that of
%    |\@xnthm|:
%    \begin{macrocode}
    \def\@tempa{\global\@namedef{#1}}\expandafter \@tempa
     \expandafter{\csname th@\the \theorem@style \expandafter
     \endcsname \the\theorem@bodyfont \@thm{#1}{#2}}%
    \global \expandafter \let \csname end#1\endcsname \@endtheorem}}
%    \end{macrocode}
% \end{macro}
%
% \begin{macro}{\@othm}
%    The definition of |\@othm| does not contain anything new.
%    \begin{macrocode}
\gdef\@othm#1[#2]#3{%
%    \end{macrocode}
% \changes{v2.2b}{95/10/07}{Build in check for counter defined}
%    We do not define a new counter but instead use one that has
%    already been defined.  Thus the only definition we need is that
%    of this pseudo-counter (i.e.\ |\the|\meta{env.\ name}).
%    First we check if |#2| corresponds to a known counter name.
%    \begin{macrocode}
  \expandafter\ifx\csname c@#2\endcsname\relax
   \@nocounterr{#2}%
  \else
   \expandafter\@ifdefinable\csname #1\endcsname
   {\expandafter \xdef \csname the#1\endcsname
     {\expandafter \noexpand \csname the#2\endcsname}%
%    \end{macrocode}
%    All other parts of the definition can be adopted from
%    |\@xnthm|. We have to remember, though, that in this case
%    the name of the current counter and the theorem label have moved
%    to the second and third arguments.
%    \begin{macrocode}
    \def\@tempa{\global\@namedef{#1}}\expandafter \@tempa
     \expandafter{\csname th@\the \theorem@style \expandafter
     \endcsname \the\theorem@bodyfont \@thm{#2}{#3}}%
    \global \expandafter \let \csname end#1\endcsname \@endtheorem}%
  \fi}
%    \end{macrocode}
% \end{macro}
%
% \subsection{Macros that are employed in a {\envfont theorem}
%    environment}
%
% \begin{macro}{\@thm}
%    The macro |\@thm| has to increase the current counter.  Then,
%    depending on whether the environment has (or does not have) an
%    optional argument, it has to branch into either |\@begintheorem|
%    or |\@opargtheorem|.
%    \begin{macrocode}
\gdef\@thm#1#2{\refstepcounter{#1}%
%    \end{macrocode}
%    Now we start a {\envfont trivlist} environment, and give
%    |\@topsep| and |\@topsepadd| the values of the \textsf{skip}
%    registers |\theorempreskipamount| and |\theorempostskipamount|.
%    The value in |\@topsep| is the vertical space that is inserted by
%    the first (and only) |\item| in our |\trivlist| whilst
%    |\@topsepadd| is inserted by |\@endparenv| at the end of that
%    {\envfont trivlist} environment.  By using these registers, we
%    obtain the desired space around a {\envfont theorem} environment.
%    \begin{macrocode}
   \trivlist
   \@topsep \theorempreskipamount               % used by first \item
   \@topsepadd \theorempostskipamount           % used by \@endparenv
%    \end{macrocode}
%    Now we have to test whether an optional argument has been given.
% \changes{v2.0d}{89/07/18}{Forgotten percent added.}
%    \begin{macrocode}
   \@ifnextchar [%
%    \end{macrocode}
%    If there is an optional argument, we will call |\@ythm|, and
%    move the arguments read back into the input stream.
%    \begin{macrocode}
   {\@ythm{#1}{#2}}%
%    \end{macrocode}
%    If not, we call |\@begintheorem|. Its first argument is the
%    name of the theorem set (hence the second argument of
%    |\@thm|). Its second argument is the macro that produces the
%    current number.
%    \begin{macrocode}
   {\@begintheorem{#2}{\csname the#1\endcsname}\ignorespaces}}
%    \end{macrocode}
% \end{macro}
%
% \begin{macro}{\@xthm}
% \begin{macro}{\@ythm}
%    Both these macros were originally called by |\@thm|.  We do
%    not need |\@xthm| anymore, hence we reset it to
%    |\relax|. The definition of |\@ythm| has not changed at
%    all from its definition in \LaTeX. In order to make the macros
%    easier to understand, we will
%    nevertheless present it (commented out).
%    \begin{macrocode}
\global\let\@xthm\relax
% \def\@ythm#1#2[#3]{\@opargbegintheorem{#2}{\csname
%           the#1\endcsname}{#3}\ignorespaces}
%</package>
%    \end{macrocode}
%    The primitive |\ignorespaces| in |\@ythm| and
%    |\@thm| is needed to remove the spaces between the
%    |\begin{...}| and the actual text.
% \end{macro}
% \end{macro}
%
%
% \subsection{Definition of the theorem substyles}
%
% As already pointed out, the theorem substyles, defined below, are
% only loaded when necessary.  Note that all these substyles, except
% \texttt{plain}, have |\slshape| as the default body font.
%
% \subsubsection{The \texttt{plain} style}
%
% \label{dok:th@plain} As the following macros use |@|, we have
% to locally set the |\catcode| of this symbol to ``letter''.
% This happens within a group, so that we do not have to worry about
% which |\catcode| that symbol had before.
%    \begin{macrocode}
%<*thp>
\begingroup \makeatletter
%    \end{macrocode}
% Since we are now within a group, we must make all definitions
% globally.  First we make sure that \texttt{theorem.sty} is loaded.
% This will allow us to use this file as a document style option
% without having to call \texttt{theorem} itself as an option.  At the
% same time, we assure that at least version 2 is loaded, since
% |\theorem@style| was not defined in earlier versions.
%    \begin{macrocode}
\@ifundefined{theorem@style}{% \iffalse meta-comment
%
% Copyright (C) 1993-2024
%
% The LaTeX Project and any individual authors listed elsewhere
% in this file.
%
% This file is part of the Standard LaTeX `Tools Bundle'.
% -------------------------------------------------------
%
% It may be distributed and/or modified under the
% conditions of the LaTeX Project Public License, either version 1.3c
% of this license or (at your option) any later version.
% The latest version of this license is in
%    https://www.latex-project.org/lppl.txt
% and version 1.3c or later is part of all distributions of LaTeX
% version 2005/12/01 or later.
%
% The list of all files belonging to the LaTeX `Tools Bundle' is
% given in the file `manifest.txt'.
%
% \fi
%
%%
%
%
\def\FMithmInfo{2023/07/05 v2.2c Theorem extension package (FMi)}
%
% \ProvidesFile{theorem.dtx}[\FMithmInfo]
%\iffalse   % this is a METACOMMENT !
%%
%% Package `theorem' to use with LaTeX2e
%% Copyright (C) 1989-1995 Frank Mittelbach, all rights reserved.
%<+package>\@ifundefined{theorem@style}{}{\endinput}
%<+package>\NeedsTeXFormat{LaTeX2e}[1995/06/01]
%<+package>\ProvidesPackage{theorem}
%<+package>         [\FMithmInfo]
%
% \fi
%
% \changes{v1.0}{87/05/07}{First draft.}
% \changes{v1.0a}{87/05/07}{`thistheoremstyle' added.}
% \changes{v1.1a}{87/07/29}{New concept: redefine the internal
%                          \protect\LaTeX{} commands `@xnthm' etc.}
% \changes{v1.1b}{87/07/31}{`theoremfont' used to specify font.}
% \changes{v1.2a}{88/01/08}{`@endtheorem' now globally defined.}
% \changes{v1.2b}{88/01/11}{`newtheorem' only allowed in the preamble.}
% \changes{v1.3a}{88/02/14}{`theoremfont' now a toks register.}
%
% \changes{v2.0a}{88/11/16}{New implementation and new documentation.}
% \changes{v2.0a}{88/11/16}{No layout changes but internal modifications
%                           old substyles won't work.}
% \changes{v2.0b}{89/01/30}{`global' used for all definitions.}
% \changes{v2.0c}{89/06/15}{Documentation translated to English.
%                            Thanks to LLL and Nina}
% \changes{v2.1a}{90/02/15}{Support for new font selection.}
% \changes{v2.2a}{94/02/03}{Cleanup for LaTeX2e}
%
% \DoNotIndex{\@endparenv,\@ifdefinable,\@ifnextchar,\@ifundefined}
% \DoNotIndex{\@namedef,\@nameuse,\@spaces,\@tempa,\@warning}
% \DoNotIndex{\ }
% \DoNotIndex{\begingroup,\bfseries}
% \DoNotIndex{\csname}
% \DoNotIndex{\def,\do,\docdate}
% \DoNotIndex{\endcsname,\endgroup,\endinput,\endtrivlist,\expandafter}
% \DoNotIndex{\filedate,\fileversion}
% \DoNotIndex{\gdef,\global}
% \DoNotIndex{\hbox,\hskip}
% \DoNotIndex{\ignorespaces,\input,\itshape,\item}
% \DoNotIndex{\labelsep,\let,\llap}
% \DoNotIndex{\makeatletter}
% \DoNotIndex{\noexpand}
% \DoNotIndex{\refstepcounter,\relax,\rlap}
% \DoNotIndex{\setlength,\slshape,\space,\string,\strut}
% \DoNotIndex{\the,\trivlist,\typeout}
% \DoNotIndex{\vbox}
% \DoNotIndex{\xdef}
%
% \def\deflabel#1{\ttfamily #1\hfill}
% \def\deflist#1{\begin{list}{}{\settowidth\labelwidth{\ttfamily #1}%
%                               \setlength\leftmargin\labelwidth
%                               \addtolength\leftmargin\labelsep
%                               \let\makelabel\deflabel}}
% \def\enddeflist{\end{list}}
% \def\envfont{\normalfont\ttfamily}
%
% \MakeShortVerb{\|}
%
% \newcommand{\TUB}{{\sl TUGboat\/}}
%
% \GetFileInfo{theorem.dtx}
% \title{An Extension of the
%        \LaTeX{} theorem environment\thanks{This file
%        has version number \fileversion, last
%        revised \filedate.}}
% \author{Frank Mittelbach}
%
% \MaintainedByLaTeXTeam{tools}
% \maketitle
%
% \begin{abstract}
%    The macros described in this paper yield an extension of the
%    \LaTeX{} theorem mechanism. It is designed to satisfy the
%    different requirements of various journals. Thus, the
%    layout of the ``theorems'' can be manipulated by determining a
%    ``style''. This article describes not only the use, but
%    also the definition, of the necessary macros.
% \end{abstract}
%
% \section*{Preface to version 2.2}
%
% For \LaTeXe{} this package did not need any fundamental changes. I
% only modified the messages generated so that theorem layout styles
% will show up with the |\listfiles| command and cleaned the section
% on the New Font Selection Scheme since this is now included in
% \LaTeX.
%
% \section*{Preface to version 2.1}
%
% This version is identical to 2.0g described in \TUB\ 10\#3 except
% for some internal defaults which are now set depending on the used
% font selection scheme.
%
% This was done to avoid unpleasant surprises if the new font selection
% scheme is in force. For further details see
% section~\ref{sec:newfonts} and \cite{bk:GMS94}.
%
% \section{Introduction}
%
% For our purposes here, ``theorems'' are labelled enunciations,
% often set off from the main text by extra space and a font change.
% Theorems, corollaries, conjectures, definitions, and remarks
% are all instances of ``theorems''.  The ``header'' of these structures
% is composed of a label (such as \textsc{Theorem} or \textsc{Remark})
% and a number which serializes an item in the sequence of items
% with the same label.
%
% Shortly after the introduction of \LaTeX{} at the Fachbereich
% Mathematik in Mainz, the desire to manipulate the layout of
% ``theorems'' arose.  In Mainz, the following two conventions came
% into general use:
% \begin{enumerate}
%   \item The number of the theorem is shown in the margin.
%   \item There is a line break at
%         the end of the theorem header.
% \end{enumerate}
%
% Additionally, some journals require different formats which depend on
% the ``sort of theorem'': e.g.\ often remarks and definitions are set
% in |\upshape|, while |\itshape| is employed for main theorems.
%
% Confronted with these requirements, a theorem environment was
% developed in Mainz which allows separate determination of the
% layout of the ``theorems sets'', comparable to |\pagestyle|.
%
%
% \section{The user interface}
% \subsection{Defining new theorem sets}
%
% \DescribeMacro\newtheorem
% As in the original \LaTeX{} version, the command |\newtheorem|
% defines a new ``theorem set'' or ``theorem-like structure''.
% Two required arguments name the new environment and give the text
% to be typeset with each instance of the new ``set'', while
% an optional argument determines how
% the ``set'' is enumerated:
% \begin{description}
%    \item[\ttfamily \bslash newtheorem\{foo\}\{bar\}]
%       The theorem set {\envfont foo} (whose name is \texttt{bar})
%       uses its own counter.
%    \item[\ttfamily \bslash newtheorem\{foo2\}{[foo]}\{bar2\}]
%       The theorem set {\envfont foo2} (printed name \texttt{bar2})
%       uses the same counter as the theorem set \texttt{foo}.
%    \item[\ttfamily \bslash newtheorem\{foo3\}\{bar3\}{[section]}]
%       The theorem set {\envfont foo3} (printed name \texttt{bar3}) is
%       enumerated within the counter \texttt{section}, i.e.\ with every
%       new |\section| the enumeration begins again with 1, and
%       the enumeration is composed from the section-number and the
%       theorem counter itself.
% \end{description}
%
% \DescribeMacro\theoremstyle
% Additionally, the command |\theoremstyle| can define the layout
% of various, or all, theorem sets.  It should be noted that any
% theorem set defined by |\newtheorem| is typeset in the
% |\theoremstyle| that is current at the time of the definition.
% Thus, the following
% \begin{verbatim}
%    \theoremstyle{break}        \newtheorem{Cor}{Corollary}
%    \theoremstyle{plain}        \newtheorem{Exa}{Example}[section]
%\end{verbatim}                             ^^A No blank at line start
% leads to the result that the set {\envfont Cor} is formatted in the
% style \texttt{break}, while the set {\envfont Exa} and all the
% following ones are formatted in the style \texttt{plain}, unless
% another |\theoremstyle| follows.  Since the definitions installed by
% |\newtheorem| are global, one also can limit |\theoremstyle| locally
% by grouping braces.
%
% \DescribeMacro\theorembodyfont
% The choice of the font for the theorem body is completely
% independent of the chosen |\theoremstyle|; this has proven to
% be very advantageous.  For example,
% \begin{verbatim}
%   {\theorembodyfont{\upshape}          \newtheorem{Rem}{Remark}}
%\end{verbatim}
% defines a theorem set {\envfont Rem}, which will be set in
% |\upshape| in the current layout (which in our example is
% \texttt{plain}).  As with |\theoremstyle|, the |\theorembodyfont|
% chosen is that current at the time of |\newtheorem|.  If
% |\theorembodyfont| is not specified or one defines
% |\theorembodyfont{}|, then the font used will be that defined by the
% |\theoremstyle|.
%
% \DescribeMacro\theoremheaderfont It is also possible to customize
% the font used for the theorem headers.  This is, however, a global
% declaration, and therefore there should be at most one
% |\theoremheaderfont| declaration in the preamble.\footnote{If it is
% actually necessary to have different header fonts, one has to define
% new theorem styles (substituting the desired font) or specify the
% information directly in the \texttt{\bslash newtheorem} declaration
% (the unclean variant).}
%
% \DescribeMacro\theorempreskipamount
% \DescribeMacro\theorempostskipamount Two additional parameters
% affect the vertical space around the theorem environments:
% |\theorempreskipamount| and |\theorempostskipamount| define,
% respectively, the spacing before and after such an environment.
% These parameters apply for all theorem sets and can be manipulated
% with the ordinary length macros.  They are rubber lengths
% (`\textsf{skips}'), and therefore can contain \texttt{plus} and
% \texttt{minus} parts.
%
% Since the definition of theorem sets should---most sensibly---be
% placed in the preamble, we only allow installation there.
% It is therefore possible to release the memory used here after
% |\begin{document}|, in order to make room for other
% applications.
%
%
% \subsection{Existing theorem styles}
%
% The following theorem styles exist to date:
% \begin{deflist}{marginbreak}
%    \item[plain]
%       This theorem style emulates the original \LaTeX{} definition,
%       except that additionally the parameters
%       |\theorem...skipamount| are used.
%    \item[break]
%       In this style, the theorem header is followed by a line
%       break.
%    \item[marginbreak]
%       The theorem number is set in the margin, and there is a line
%       break as in \texttt{break}.
%    \item[changebreak]
%       Like \texttt{break}, but with header number and text
%       interchanged.
%    \item[change]
%       Header number and text are interchanged, without a line break.
%    \item[margin]
%       The number is set in the left margin, without a line break.
% \end{deflist}
% All styles (except \texttt{plain}) select |\slshape| as the default
% |\theorembodyfont|.
%
%
% \subsection{Examples}
%
% Given the above theorem sets {\envfont Cor}, {\envfont Exa}
% and {\envfont Rem},
% suppose that the preamble also contains the declarations:
% \begin{verbatim}
%    \theoremstyle{marginbreak}   \newtheorem{Lem}[Cor]{Lemma}
%    \theoremstyle{change}
%    \theorembodyfont{\itshape}        \newtheorem{Def}[Cor]{Definition}
%
%    \theoremheaderfont{\scshape}
%\end{verbatim}
% Then the following are some typical examples of the typeset output
% resulting from their use.
%
% \begin{Cor}
%    This is a sentence typeset in the theorem environment {\envfont
%    Cor}.
% \end{Cor}
%
% \begin{Exa}
%    This is a sentence typeset in the theorem environment {\envfont
%    Exa}.
% \end{Exa}
%
% \begin{Rem}
%    This is a sentence typeset in the theorem environment {\envfont
%    Rem}.
% \end{Rem}
%
% \begin{Lem}[Ben User]
%    This is a sentence typeset in the theorem environment {\envfont
%    Lem}.
% \end{Lem}
%
% \begin{Def}[Very impressive Definition]
%    This is a sentence typeset in the theorem environment {\envfont
%    Def}.
% \end{Def}
%
% The last two examples show the effect of the optional argument to
% a theorem environment (it is the text typeset in parentheses).
%
%
% \section{Special Considerations}
% \label{sec:newfonts}
%
% Theoremheader and body are implemented as a unit. This means that
% the |\theoremheaderfont|  will inherit characteristics of the
% |\theorembodyfont| in \LaTeXe. Thus, if for example
% |\theorembodyfont| is |\itshape| and |\theoremheaderfont|
% is |\bfseries| the font selected for the header will have the
% characteristics `bold extended italic'. If this is not desired one
% should set the |\theoremheaderfont| to something like
% \begin{verbatim}
%      \theoremheaderfont{\normalfont\bfseries}
%\end{verbatim}
% i.e.\ supplying all necessary font information explicitly.
%
%
%
% \section{Acknowledgements}
%
% The publication of this set of macros was only possible with the
% help of Christina Busse (translating the manuscript into English),
% Joachim Pense (playing the r\^ole of typist), Chris Rowley
% (looking everything over) and many others providing useful
% suggestions.
%
% \MaybeStop{%
%    \begin{thebibliography}{9}
%       \bibitem{bk:GMS94} \textsc{M.~Goossens}, \textsc{F.~Mittelbach}
%         and \textsc{A.~Samarin}.
%         \newblock The \LaTeX{} Companion.
%         \newblock
%         Addison-Wesley, Reading, Massachusetts, 1994.
%       \bibitem{latex:source} \textsc{Lamport, Leslie.} \newblock
%                               \texttt{latex.tex}, version 2.09, date
%                               Feb. 1990.
%    \end{thebibliography}
%    \PrintIndex
%    \PrintChanges}
%
%
% \section{The documentation driver file}
%
% The next bit of code contains the documentation driver file for
% \TeX{}, i.e., the file that will produce the documentation you are
% currently reading. It will be extracted from this file by the
% \texttt{docstrip} program. Since it is the first code in the file
% one can alternatively process this file directly with \LaTeXe{} to
% obtain the documentation.
%
%    \begin{macrocode}
%<*driver>
\documentclass{ltxdoc}

\usepackage{theorem}

%<+driver>% The next few lines define theorem sets which are used
%<+driver>% in the example section of the documentation.

\theoremstyle{break}       \newtheorem{Cor}{Corollary}
\theoremstyle{plain}       \newtheorem{Exa}{Example}[section]
{\theorembodyfont{\upshape}\newtheorem{Rem}{Remark}}
\theoremstyle{marginbreak} \newtheorem{Lem}[Cor]{Lemma}
\theoremstyle{change}
\theorembodyfont{\itshape} \newtheorem{Def}[Cor]{Definition}

\theoremheaderfont{\scshape}

\RecordChanges

\begin{document}
   \DocInput{theorem.dtx}
\end{document}
%</driver>
%    \end{macrocode}
%
% \section{Definition of the Macros}
%
% If the file has been loaded before, we abort immediately. If not the
% package announces itself (this is actually done at the very top of
% the file---the way it is done isn't good style so don't copy it).
% \changes{v2.0e}{89/07/19}{Spaces removed from `typeout'.}
% \changes{v2.2b}{95/11/19}{Announce moved to top of file}
%    \begin{macrocode}
%<*package>
%\@ifundefined{theorem@style}{}{\endinput}
%\def\FMithmInfo{1995/11/19 v2.2b Theorem extension package (FMi)}
%\ProvidesPackage{theorem}[\FMithmInfo]
%    \end{macrocode}
%
%
% \subsection{Definition of theorem styles and fonts}
%
% All the definitions in this file are done globally to allow
% inputting this file inside a group.
%
% \begin{macro}{\theoremstyle}
%    Before a theorem style can be installed, the chosen style must be
%    known. For that reason, we must test to see that
%    |\th@|\meta{style} is known or, more precisely, that it is
%    different from |\relax|. If the style is not known then
%    |\th@plain| is used.
%    \begin{macrocode}
\gdef\theoremstyle#1{%
   \@ifundefined{th@#1}{\@warning
          {Unknown theoremstyle `#1'. Using `plain'}%
          \theorem@style{plain}}%
%    \end{macrocode}
%    We save the theorem style to be used in the token register
%    |\theorem@style|.
%    \begin{macrocode}
      {\theorem@style{#1}}%
%    \end{macrocode}
%    Now we ``evaluate'' the theorem style: this means, we call the
%    macro |\th@|\meta{style} which will activate the relevant
%    definitions which are contained in a separate file.  This is done
%    in a group to suppress changes to the current font.  This could
%    otherwise pose problems together with the new font selection
%    scheme\footnote{When I printed the original article using the new
%    font selection scheme I ended with a document with slanted
%    typefaces (text headings and all) simply because one of the
%    theorem styles used \texttt{\bslash sl} at toplevel.} if the
%    |\th@|\meta{style} is evaluated a second time.
% \changes{v2.1a}{90/02/15}{Group added around `th@..}
% \changes{v2.1a}{90/02/15}{Reseting of `..@begintheorem removed.}
%    \begin{macrocode}
      \begingroup
        \csname th@\the\theorem@style \endcsname
      \endgroup}
%    \end{macrocode}
% \end{macro}
%
%
% \begin{macro}{\@begintheorem}
% \changes{v2.1a}{90/02/15}{Relaxed at toplevel.}
% \begin{macro}{\@opargbegintheorem}
% \changes{v2.1a}{90/02/15}{Relaxed at toplevel.}
%    We reset |\@begintheorem| and |\@opargbegintheorem| to
%    |\relax| since these commands are no longer necessary at
%    toplevel. This will save a few tokens.
%    \begin{macrocode}
\global\let\@begintheorem\relax
\global\let\@opargbegintheorem\relax
%    \end{macrocode}
% \end{macro}
% \end{macro}
%
% \begin{macro}{\theorem@style}
%    Obviously the \textsf{token} register used above has to be
%    allocated. To assure the utmost compatibility with the
%    original \LaTeX{} definition, we set the default theorem style to
%    \texttt{plain}, which implements the usual \LaTeX{} convention.
% \changes{v2.0f}{89/07/20}{Unnecessary `global' removed.}
%    \begin{macrocode}
\newtoks\theorem@style
\global\theorem@style{plain}
%    \end{macrocode}
% \end{macro}
%
% \begin{macro}{\theorembodyfont}
% \begin{macro}{\theorem@bodyfont}
%    If the |\theorembodyfont| is set by the user then it should not
%    interact with the default font set in the theorem style. When the
%    new font selection is in force this may happen if, for example,
%    the default is |\itshape| and the new |\theorembodyfont| is
%    |\sffamily|. So we add a |\reset@font| command in front of the
%    user definition.
% \changes{v2.0f}{89/07/20}{Unnecessary `global' removed.}
% \changes{v2.0g}{89/09/19}{`theoremfont to `theorembodyfont.}
% \changes{v2.1c}{92/06/22}{`reset@font added and logic changed.}
% \changes{v2.1d}{92/10/28}{Reset to empty if arg is empty}
%    \begin{macrocode}
\gdef\theorembodyfont#1{%
%    \end{macrocode}
%    We check if the argument supplied is empty and if so put nothing
%    into the |\theorem@bodyfont| token register to allow for
%    |\theorembodyfont{}| as a mean of using the default of the
%    current |\theoremstyle|.
%    \begin{macrocode}
   \def\@tempa{#1}%
   \ifx\@tempa\@empty
    \theorem@bodyfont{}%
   \else
    \theorem@bodyfont{\reset@font#1}%
   \fi
   }
\newtoks\theorem@bodyfont
\global\theorem@bodyfont{}
%    \end{macrocode}
% \end{macro}
% \end{macro}
%
%
% \begin{macro}{\theoremheaderfont}
% \changes{v2.0g}{89/09/19}{Macro added.}
%    The font for the theorem headers is handled differently because
%    this definition applies to all theorem styles.
%    \begin{macrocode}
\gdef\theoremheaderfont#1{\gdef\theorem@headerfont{#1}%
%    \end{macrocode}
%    After using the macro once it is redefined to produce an error
%    message.
%    \begin{macrocode}
       \gdef\theoremheaderfont##1{%
        \typeout{\string\theoremheaderfont\space should be used
                 only once.}}}
%    \end{macrocode}
% \end{macro}
%
% \begin{macro}{\theorem@headerfont}
% \changes{v2.0g}{89/09/19}{Macro added.}
%    To set the |\theorem@headerfont| default we first test if
%    the new fontselection scheme is in force.
% \changes{v2.1a}{90/02/15}{Support for new font selection.}
%    \begin{macrocode}
\ifx\upshape\undefined
%    \end{macrocode}
%    If not we define it to expand into |\bfseries|. We don't use
%    |\let| just in case a following style option redefines this
%    macro.
%    \begin{macrocode}
\gdef\theorem@headerfont{\bfseries}
%    \end{macrocode}
%    Otherwise we reset the current shape before calling |\bfseries|.
%    \begin{macrocode}
\else \gdef\theorem@headerfont{\normalfont\bfseries}\fi
%    \end{macrocode}
% \end{macro}
%
% \begin{macro}{\th@plain}
% \begin{macro}{\th@break}
% \begin{macro}{\th@marginbreak}
% \begin{macro}{\th@changebreak}
% \begin{macro}{\th@change}
% \begin{macro}{\th@margin}
%    The different styles are defined in macros such as
%    |\th@plain|.  Since memory space is precious in
%    ``non-Big-versions'', we have to avoid offering too many unused
%    definitions.  Therefore we define these styles in separate files
%    that can be loaded on demand.  Thus the commands themselves only
%    load these files. We use |\@input@|, a \LaTeXe{} internal command
%    that ensures that the file will be listed with |\listfiles|.
% \changes{v2.2a}{94/02/02}{Use `@input@ to load theorem layout files}
%    \begin{macrocode}
\gdef\th@plain{\@input@{thp.sty}}
\gdef\th@break{\@input@{thb.sty}}
\gdef\th@marginbreak{\@input@{thmb.sty}}
\gdef\th@changebreak{\@input@{thcb.sty}}
\gdef\th@change{\@input@{thc.sty}}
\gdef\th@margin{\@input@{thm.sty}}
%    \end{macrocode}
%    This list will be expanded when new styles become
%    available.  For testing,  just append new theorem substyles
%    as document options.
% \end{macro}
% \end{macro}
% \end{macro}
% \end{macro}
% \end{macro}
% \end{macro}
%
%
%
% \subsection{Definition of a new theorem set}
%
% As already pointed out, a new theorem environment can be defined in
% three different ways:
% \begin{verbatim}
%    \newtheorem{Lem}{Lemma}
%    \newtheorem{Lem}{Lemma}[section]
%    \newtheorem{Lem}[Theorem]{Lemma}
%\end{verbatim}
% The function of the macro |\newtheorem| is to recognize these cases
% and then to branch into one of the three macros |\@ynthm|, |\@xnthm|
% or |\@othm|.  This mechanism is adopted unchanged from
% \cite{latex:source}; the essential point here is that, for example,
% in the second case, the arguments \texttt{Lem}, \texttt{Lemma} and
% \texttt{section} are passed over to the macro |\@xnthm|.
%
% We inspect this case first because the others present fewer problems,
% and thus are easily derived from this one.
%
% \begin{macro}{\@xnthm}
%    For our example arguments, the macro |\@xnthm| must fulfill
%    the following:
%    \begin{itemize}
%       \item
%          Define a new \LaTeX-counter `Lem'
%       \item
%          reset this counter within a |\section|
%       \item
%          define the macro |\theLem|
%       \item
%          define the environment macros |\Lem| and
%          |\endLem| using the current
%          |\theoremstyle| and |\theorem@bodyfont|.
%    \end{itemize}
%    Obviously, all this should happen only if the first argument
%    of |\@xnthm| (i.e.\ \texttt{Lem} in our example) is chosen so
%    as not to conflict with any previously defined commands or
%    environments.  This test is performed by the \LaTeX{} macro
%    |\@ifdefinable|.
%    \begin{macrocode}
\gdef\@xnthm#1#2[#3]{\expandafter\@ifdefinable\csname #1\endcsname
%    \end{macrocode}
%    Therefore, the first argument of |\@ifdefinable| is the
%    expansion (in the example, |\Lem|) of
%    |\csname||#1\endcsname|.  The second argument is
%    executed only if the test has been completed successfully.
%    \begin{macrocode}
   {%
%    \end{macrocode}
%    Now we define the new counter. The names of the \LaTeX{}
%    macros employed should speak for themselves:
% \changes{v2.2b}{1995/10/07}{Use \cs{@newctr}}
% \changes{v2.2c}{1995/11/23}{Use \cs{@newctr} properly.}
%    \begin{macrocode}
    \@definecounter{#1}\@newctr{#1}[#3]%
%    \end{macrocode}
%    Using |\@newctr| will give a proper error message if the counter
%    in |#3| is not defined.
%    In defining `|\theLem|' we must generate the desired macro
%    name by use of |\expandafter| and |\csname|.
%    \begin{macrocode}
    \expandafter\xdef\csname the#1\endcsname
%    \end{macrocode}
%    An |\xdef| is used in order to make the definition global,
%    and to ensure that it contains the replacement texts of
%    |\@thmcountersep| and |\@thmcounter|.\footnote{These
%    two macros can be defined by the document style.  Their default
%    values produce a `\texttt{.}' as separation and an arabic
%    representation of the number.} However, not everything should be
%    expanded. For example, it saves space to use |\thesection|
%    instead of its---at times---lengthy expansion.
%    \begin{macrocode}
      {\expandafter \noexpand \csname the#3\endcsname
       \@thmcountersep \@thmcounter{#1}}%
%    \end{macrocode}
%    Thus with the defaults of \LaTeX{}, |\theLem| would be
%    replaced by the command sequence |\thesection.\arabic{Lem}|.
%
%    We will now look at the definition of the macro which is executed
%    at the beginning of the actual environment (in our example this
%    macro is |\Lem|). It should be noted that we use an
%    ``|\expandafter| trick'' to expand only certain parts of the
%    replacement text at the time of the definition.
%    \begin{macrocode}
    \def\@tempa{\global\@namedef{#1}}%
    \expandafter \@tempa \expandafter{%
%    \end{macrocode}
%    First, the macro that contains the current definitions of
%    |\@begintheorem| and |\@opargtheorem| should be called
%    up.  The name of this macro---as is already known---has the form
%    |\th@|\meta{theorem\ style}; therefore, it must be called by
%    \begin{macrocode}
      \csname th@\the \theorem@style
%    \end{macrocode}
%    In addition the default theorem font should be changeable, i.e.\
%    we have to insert the contents of |\theorem@bodyfont|. For
%    that reason, we expand even further, beyond |\endcsname|,
%    and thus insert the contents of the \textsf{token} register
%    |\theorem@bodyfont| in the replacement text.
%    \begin{macrocode}
            \expandafter \endcsname \the \theorem@bodyfont
%    \end{macrocode}
%    Now it is time to call the macro |\@thm| which takes over
%    the further processing.  It has two arguments: the current
%    counter name (in our example, \texttt{Lem}), and the text of the
%    label (in our example, \texttt{Lemma}).
%    \begin{macrocode}
     \@thm{#1}{#2}}%
%    \end{macrocode}
%    With this, the `sub-definition' is complete.  The macro
%    |\@endtheorem| ends a theorem environment and is, so far, nothing
%    but an |\endtrivlist|.  (Hence it is defined globally, and not
%    within the theorem styles.\footnote{This has to be changed as
%    soon as theorem styles that change \texttt{\bslash @endtheorem}
%    exist.  In such a case, all existing styles must be changed as
%    well since they will have to reset the macro.}) Therefore, we can
%    set it equivalent to the macro that ends the theorem set (in our
%    example, |\endLem|).  However, if some day theorem styles exist
%    that do change |\@endtheorem|, we would have to use the
%    commented-out line instead.
%    \begin{macrocode}
    \global \expandafter \let \csname end#1\endcsname \@endtheorem
%   \global\@namedef{end#1}{\@endtheorem}%
%    \end{macrocode}
%    With these commands all the required definitions are employed,
%    unless the test |\@ifdefinable| has failed.  Therefore, we
%    end the second argument of this macro and with it the definition
%    of |\@xnthm|.
%    \begin{macrocode}
   }}
%    \end{macrocode}
% \end{macro}
%
% \begin{macro}{\@ynthm}
%    The definition of |\@ynthm| is completely analogous. In
%    this case the new counter that is defined is not reset within
%    another counter; thus the definition of |\the...|
%    is simplified:
%    \begin{macrocode}
\gdef\@ynthm#1#2{\expandafter\@ifdefinable\csname #1\endcsname
   {\@definecounter{#1}%
    \expandafter\xdef\csname the#1\endcsname{\@thmcounter{#1}}%
%    \end{macrocode}
%    The rest of the definition corresponds literally to that of
%    |\@xnthm|:
%    \begin{macrocode}
    \def\@tempa{\global\@namedef{#1}}\expandafter \@tempa
     \expandafter{\csname th@\the \theorem@style \expandafter
     \endcsname \the\theorem@bodyfont \@thm{#1}{#2}}%
    \global \expandafter \let \csname end#1\endcsname \@endtheorem}}
%    \end{macrocode}
% \end{macro}
%
% \begin{macro}{\@othm}
%    The definition of |\@othm| does not contain anything new.
%    \begin{macrocode}
\gdef\@othm#1[#2]#3{%
%    \end{macrocode}
% \changes{v2.2b}{95/10/07}{Build in check for counter defined}
%    We do not define a new counter but instead use one that has
%    already been defined.  Thus the only definition we need is that
%    of this pseudo-counter (i.e.\ |\the|\meta{env.\ name}).
%    First we check if |#2| corresponds to a known counter name.
%    \begin{macrocode}
  \expandafter\ifx\csname c@#2\endcsname\relax
   \@nocounterr{#2}%
  \else
   \expandafter\@ifdefinable\csname #1\endcsname
   {\expandafter \xdef \csname the#1\endcsname
     {\expandafter \noexpand \csname the#2\endcsname}%
%    \end{macrocode}
%    All other parts of the definition can be adopted from
%    |\@xnthm|. We have to remember, though, that in this case
%    the name of the current counter and the theorem label have moved
%    to the second and third arguments.
%    \begin{macrocode}
    \def\@tempa{\global\@namedef{#1}}\expandafter \@tempa
     \expandafter{\csname th@\the \theorem@style \expandafter
     \endcsname \the\theorem@bodyfont \@thm{#2}{#3}}%
    \global \expandafter \let \csname end#1\endcsname \@endtheorem}%
  \fi}
%    \end{macrocode}
% \end{macro}
%
% \subsection{Macros that are employed in a {\envfont theorem}
%    environment}
%
% \begin{macro}{\@thm}
%    The macro |\@thm| has to increase the current counter.  Then,
%    depending on whether the environment has (or does not have) an
%    optional argument, it has to branch into either |\@begintheorem|
%    or |\@opargtheorem|.
%    \begin{macrocode}
\gdef\@thm#1#2{\refstepcounter{#1}%
%    \end{macrocode}
%    Now we start a {\envfont trivlist} environment, and give
%    |\@topsep| and |\@topsepadd| the values of the \textsf{skip}
%    registers |\theorempreskipamount| and |\theorempostskipamount|.
%    The value in |\@topsep| is the vertical space that is inserted by
%    the first (and only) |\item| in our |\trivlist| whilst
%    |\@topsepadd| is inserted by |\@endparenv| at the end of that
%    {\envfont trivlist} environment.  By using these registers, we
%    obtain the desired space around a {\envfont theorem} environment.
%    \begin{macrocode}
   \trivlist
   \@topsep \theorempreskipamount               % used by first \item
   \@topsepadd \theorempostskipamount           % used by \@endparenv
%    \end{macrocode}
%    Now we have to test whether an optional argument has been given.
% \changes{v2.0d}{89/07/18}{Forgotten percent added.}
%    \begin{macrocode}
   \@ifnextchar [%
%    \end{macrocode}
%    If there is an optional argument, we will call |\@ythm|, and
%    move the arguments read back into the input stream.
%    \begin{macrocode}
   {\@ythm{#1}{#2}}%
%    \end{macrocode}
%    If not, we call |\@begintheorem|. Its first argument is the
%    name of the theorem set (hence the second argument of
%    |\@thm|). Its second argument is the macro that produces the
%    current number.
%    \begin{macrocode}
   {\@begintheorem{#2}{\csname the#1\endcsname}\ignorespaces}}
%    \end{macrocode}
% \end{macro}
%
% \begin{macro}{\@xthm}
% \begin{macro}{\@ythm}
%    Both these macros were originally called by |\@thm|.  We do
%    not need |\@xthm| anymore, hence we reset it to
%    |\relax|. The definition of |\@ythm| has not changed at
%    all from its definition in \LaTeX. In order to make the macros
%    easier to understand, we will
%    nevertheless present it (commented out).
%    \begin{macrocode}
\global\let\@xthm\relax
% \def\@ythm#1#2[#3]{\@opargbegintheorem{#2}{\csname
%           the#1\endcsname}{#3}\ignorespaces}
%</package>
%    \end{macrocode}
%    The primitive |\ignorespaces| in |\@ythm| and
%    |\@thm| is needed to remove the spaces between the
%    |\begin{...}| and the actual text.
% \end{macro}
% \end{macro}
%
%
% \subsection{Definition of the theorem substyles}
%
% As already pointed out, the theorem substyles, defined below, are
% only loaded when necessary.  Note that all these substyles, except
% \texttt{plain}, have |\slshape| as the default body font.
%
% \subsubsection{The \texttt{plain} style}
%
% \label{dok:th@plain} As the following macros use |@|, we have
% to locally set the |\catcode| of this symbol to ``letter''.
% This happens within a group, so that we do not have to worry about
% which |\catcode| that symbol had before.
%    \begin{macrocode}
%<*thp>
\begingroup \makeatletter
%    \end{macrocode}
% Since we are now within a group, we must make all definitions
% globally.  First we make sure that \texttt{theorem.sty} is loaded.
% This will allow us to use this file as a document style option
% without having to call \texttt{theorem} itself as an option.  At the
% same time, we assure that at least version 2 is loaded, since
% |\theorem@style| was not defined in earlier versions.
%    \begin{macrocode}
\@ifundefined{theorem@style}{% \iffalse meta-comment
%
% Copyright (C) 1993-2024
%
% The LaTeX Project and any individual authors listed elsewhere
% in this file.
%
% This file is part of the Standard LaTeX `Tools Bundle'.
% -------------------------------------------------------
%
% It may be distributed and/or modified under the
% conditions of the LaTeX Project Public License, either version 1.3c
% of this license or (at your option) any later version.
% The latest version of this license is in
%    https://www.latex-project.org/lppl.txt
% and version 1.3c or later is part of all distributions of LaTeX
% version 2005/12/01 or later.
%
% The list of all files belonging to the LaTeX `Tools Bundle' is
% given in the file `manifest.txt'.
%
% \fi
%
%%
%
%
\def\FMithmInfo{2023/07/05 v2.2c Theorem extension package (FMi)}
%
% \ProvidesFile{theorem.dtx}[\FMithmInfo]
%\iffalse   % this is a METACOMMENT !
%%
%% Package `theorem' to use with LaTeX2e
%% Copyright (C) 1989-1995 Frank Mittelbach, all rights reserved.
%<+package>\@ifundefined{theorem@style}{}{\endinput}
%<+package>\NeedsTeXFormat{LaTeX2e}[1995/06/01]
%<+package>\ProvidesPackage{theorem}
%<+package>         [\FMithmInfo]
%
% \fi
%
% \changes{v1.0}{87/05/07}{First draft.}
% \changes{v1.0a}{87/05/07}{`thistheoremstyle' added.}
% \changes{v1.1a}{87/07/29}{New concept: redefine the internal
%                          \protect\LaTeX{} commands `@xnthm' etc.}
% \changes{v1.1b}{87/07/31}{`theoremfont' used to specify font.}
% \changes{v1.2a}{88/01/08}{`@endtheorem' now globally defined.}
% \changes{v1.2b}{88/01/11}{`newtheorem' only allowed in the preamble.}
% \changes{v1.3a}{88/02/14}{`theoremfont' now a toks register.}
%
% \changes{v2.0a}{88/11/16}{New implementation and new documentation.}
% \changes{v2.0a}{88/11/16}{No layout changes but internal modifications
%                           old substyles won't work.}
% \changes{v2.0b}{89/01/30}{`global' used for all definitions.}
% \changes{v2.0c}{89/06/15}{Documentation translated to English.
%                            Thanks to LLL and Nina}
% \changes{v2.1a}{90/02/15}{Support for new font selection.}
% \changes{v2.2a}{94/02/03}{Cleanup for LaTeX2e}
%
% \DoNotIndex{\@endparenv,\@ifdefinable,\@ifnextchar,\@ifundefined}
% \DoNotIndex{\@namedef,\@nameuse,\@spaces,\@tempa,\@warning}
% \DoNotIndex{\ }
% \DoNotIndex{\begingroup,\bfseries}
% \DoNotIndex{\csname}
% \DoNotIndex{\def,\do,\docdate}
% \DoNotIndex{\endcsname,\endgroup,\endinput,\endtrivlist,\expandafter}
% \DoNotIndex{\filedate,\fileversion}
% \DoNotIndex{\gdef,\global}
% \DoNotIndex{\hbox,\hskip}
% \DoNotIndex{\ignorespaces,\input,\itshape,\item}
% \DoNotIndex{\labelsep,\let,\llap}
% \DoNotIndex{\makeatletter}
% \DoNotIndex{\noexpand}
% \DoNotIndex{\refstepcounter,\relax,\rlap}
% \DoNotIndex{\setlength,\slshape,\space,\string,\strut}
% \DoNotIndex{\the,\trivlist,\typeout}
% \DoNotIndex{\vbox}
% \DoNotIndex{\xdef}
%
% \def\deflabel#1{\ttfamily #1\hfill}
% \def\deflist#1{\begin{list}{}{\settowidth\labelwidth{\ttfamily #1}%
%                               \setlength\leftmargin\labelwidth
%                               \addtolength\leftmargin\labelsep
%                               \let\makelabel\deflabel}}
% \def\enddeflist{\end{list}}
% \def\envfont{\normalfont\ttfamily}
%
% \MakeShortVerb{\|}
%
% \newcommand{\TUB}{{\sl TUGboat\/}}
%
% \GetFileInfo{theorem.dtx}
% \title{An Extension of the
%        \LaTeX{} theorem environment\thanks{This file
%        has version number \fileversion, last
%        revised \filedate.}}
% \author{Frank Mittelbach}
%
% \MaintainedByLaTeXTeam{tools}
% \maketitle
%
% \begin{abstract}
%    The macros described in this paper yield an extension of the
%    \LaTeX{} theorem mechanism. It is designed to satisfy the
%    different requirements of various journals. Thus, the
%    layout of the ``theorems'' can be manipulated by determining a
%    ``style''. This article describes not only the use, but
%    also the definition, of the necessary macros.
% \end{abstract}
%
% \section*{Preface to version 2.2}
%
% For \LaTeXe{} this package did not need any fundamental changes. I
% only modified the messages generated so that theorem layout styles
% will show up with the |\listfiles| command and cleaned the section
% on the New Font Selection Scheme since this is now included in
% \LaTeX.
%
% \section*{Preface to version 2.1}
%
% This version is identical to 2.0g described in \TUB\ 10\#3 except
% for some internal defaults which are now set depending on the used
% font selection scheme.
%
% This was done to avoid unpleasant surprises if the new font selection
% scheme is in force. For further details see
% section~\ref{sec:newfonts} and \cite{bk:GMS94}.
%
% \section{Introduction}
%
% For our purposes here, ``theorems'' are labelled enunciations,
% often set off from the main text by extra space and a font change.
% Theorems, corollaries, conjectures, definitions, and remarks
% are all instances of ``theorems''.  The ``header'' of these structures
% is composed of a label (such as \textsc{Theorem} or \textsc{Remark})
% and a number which serializes an item in the sequence of items
% with the same label.
%
% Shortly after the introduction of \LaTeX{} at the Fachbereich
% Mathematik in Mainz, the desire to manipulate the layout of
% ``theorems'' arose.  In Mainz, the following two conventions came
% into general use:
% \begin{enumerate}
%   \item The number of the theorem is shown in the margin.
%   \item There is a line break at
%         the end of the theorem header.
% \end{enumerate}
%
% Additionally, some journals require different formats which depend on
% the ``sort of theorem'': e.g.\ often remarks and definitions are set
% in |\upshape|, while |\itshape| is employed for main theorems.
%
% Confronted with these requirements, a theorem environment was
% developed in Mainz which allows separate determination of the
% layout of the ``theorems sets'', comparable to |\pagestyle|.
%
%
% \section{The user interface}
% \subsection{Defining new theorem sets}
%
% \DescribeMacro\newtheorem
% As in the original \LaTeX{} version, the command |\newtheorem|
% defines a new ``theorem set'' or ``theorem-like structure''.
% Two required arguments name the new environment and give the text
% to be typeset with each instance of the new ``set'', while
% an optional argument determines how
% the ``set'' is enumerated:
% \begin{description}
%    \item[\ttfamily \bslash newtheorem\{foo\}\{bar\}]
%       The theorem set {\envfont foo} (whose name is \texttt{bar})
%       uses its own counter.
%    \item[\ttfamily \bslash newtheorem\{foo2\}{[foo]}\{bar2\}]
%       The theorem set {\envfont foo2} (printed name \texttt{bar2})
%       uses the same counter as the theorem set \texttt{foo}.
%    \item[\ttfamily \bslash newtheorem\{foo3\}\{bar3\}{[section]}]
%       The theorem set {\envfont foo3} (printed name \texttt{bar3}) is
%       enumerated within the counter \texttt{section}, i.e.\ with every
%       new |\section| the enumeration begins again with 1, and
%       the enumeration is composed from the section-number and the
%       theorem counter itself.
% \end{description}
%
% \DescribeMacro\theoremstyle
% Additionally, the command |\theoremstyle| can define the layout
% of various, or all, theorem sets.  It should be noted that any
% theorem set defined by |\newtheorem| is typeset in the
% |\theoremstyle| that is current at the time of the definition.
% Thus, the following
% \begin{verbatim}
%    \theoremstyle{break}        \newtheorem{Cor}{Corollary}
%    \theoremstyle{plain}        \newtheorem{Exa}{Example}[section]
%\end{verbatim}                             ^^A No blank at line start
% leads to the result that the set {\envfont Cor} is formatted in the
% style \texttt{break}, while the set {\envfont Exa} and all the
% following ones are formatted in the style \texttt{plain}, unless
% another |\theoremstyle| follows.  Since the definitions installed by
% |\newtheorem| are global, one also can limit |\theoremstyle| locally
% by grouping braces.
%
% \DescribeMacro\theorembodyfont
% The choice of the font for the theorem body is completely
% independent of the chosen |\theoremstyle|; this has proven to
% be very advantageous.  For example,
% \begin{verbatim}
%   {\theorembodyfont{\upshape}          \newtheorem{Rem}{Remark}}
%\end{verbatim}
% defines a theorem set {\envfont Rem}, which will be set in
% |\upshape| in the current layout (which in our example is
% \texttt{plain}).  As with |\theoremstyle|, the |\theorembodyfont|
% chosen is that current at the time of |\newtheorem|.  If
% |\theorembodyfont| is not specified or one defines
% |\theorembodyfont{}|, then the font used will be that defined by the
% |\theoremstyle|.
%
% \DescribeMacro\theoremheaderfont It is also possible to customize
% the font used for the theorem headers.  This is, however, a global
% declaration, and therefore there should be at most one
% |\theoremheaderfont| declaration in the preamble.\footnote{If it is
% actually necessary to have different header fonts, one has to define
% new theorem styles (substituting the desired font) or specify the
% information directly in the \texttt{\bslash newtheorem} declaration
% (the unclean variant).}
%
% \DescribeMacro\theorempreskipamount
% \DescribeMacro\theorempostskipamount Two additional parameters
% affect the vertical space around the theorem environments:
% |\theorempreskipamount| and |\theorempostskipamount| define,
% respectively, the spacing before and after such an environment.
% These parameters apply for all theorem sets and can be manipulated
% with the ordinary length macros.  They are rubber lengths
% (`\textsf{skips}'), and therefore can contain \texttt{plus} and
% \texttt{minus} parts.
%
% Since the definition of theorem sets should---most sensibly---be
% placed in the preamble, we only allow installation there.
% It is therefore possible to release the memory used here after
% |\begin{document}|, in order to make room for other
% applications.
%
%
% \subsection{Existing theorem styles}
%
% The following theorem styles exist to date:
% \begin{deflist}{marginbreak}
%    \item[plain]
%       This theorem style emulates the original \LaTeX{} definition,
%       except that additionally the parameters
%       |\theorem...skipamount| are used.
%    \item[break]
%       In this style, the theorem header is followed by a line
%       break.
%    \item[marginbreak]
%       The theorem number is set in the margin, and there is a line
%       break as in \texttt{break}.
%    \item[changebreak]
%       Like \texttt{break}, but with header number and text
%       interchanged.
%    \item[change]
%       Header number and text are interchanged, without a line break.
%    \item[margin]
%       The number is set in the left margin, without a line break.
% \end{deflist}
% All styles (except \texttt{plain}) select |\slshape| as the default
% |\theorembodyfont|.
%
%
% \subsection{Examples}
%
% Given the above theorem sets {\envfont Cor}, {\envfont Exa}
% and {\envfont Rem},
% suppose that the preamble also contains the declarations:
% \begin{verbatim}
%    \theoremstyle{marginbreak}   \newtheorem{Lem}[Cor]{Lemma}
%    \theoremstyle{change}
%    \theorembodyfont{\itshape}        \newtheorem{Def}[Cor]{Definition}
%
%    \theoremheaderfont{\scshape}
%\end{verbatim}
% Then the following are some typical examples of the typeset output
% resulting from their use.
%
% \begin{Cor}
%    This is a sentence typeset in the theorem environment {\envfont
%    Cor}.
% \end{Cor}
%
% \begin{Exa}
%    This is a sentence typeset in the theorem environment {\envfont
%    Exa}.
% \end{Exa}
%
% \begin{Rem}
%    This is a sentence typeset in the theorem environment {\envfont
%    Rem}.
% \end{Rem}
%
% \begin{Lem}[Ben User]
%    This is a sentence typeset in the theorem environment {\envfont
%    Lem}.
% \end{Lem}
%
% \begin{Def}[Very impressive Definition]
%    This is a sentence typeset in the theorem environment {\envfont
%    Def}.
% \end{Def}
%
% The last two examples show the effect of the optional argument to
% a theorem environment (it is the text typeset in parentheses).
%
%
% \section{Special Considerations}
% \label{sec:newfonts}
%
% Theoremheader and body are implemented as a unit. This means that
% the |\theoremheaderfont|  will inherit characteristics of the
% |\theorembodyfont| in \LaTeXe. Thus, if for example
% |\theorembodyfont| is |\itshape| and |\theoremheaderfont|
% is |\bfseries| the font selected for the header will have the
% characteristics `bold extended italic'. If this is not desired one
% should set the |\theoremheaderfont| to something like
% \begin{verbatim}
%      \theoremheaderfont{\normalfont\bfseries}
%\end{verbatim}
% i.e.\ supplying all necessary font information explicitly.
%
%
%
% \section{Acknowledgements}
%
% The publication of this set of macros was only possible with the
% help of Christina Busse (translating the manuscript into English),
% Joachim Pense (playing the r\^ole of typist), Chris Rowley
% (looking everything over) and many others providing useful
% suggestions.
%
% \MaybeStop{%
%    \begin{thebibliography}{9}
%       \bibitem{bk:GMS94} \textsc{M.~Goossens}, \textsc{F.~Mittelbach}
%         and \textsc{A.~Samarin}.
%         \newblock The \LaTeX{} Companion.
%         \newblock
%         Addison-Wesley, Reading, Massachusetts, 1994.
%       \bibitem{latex:source} \textsc{Lamport, Leslie.} \newblock
%                               \texttt{latex.tex}, version 2.09, date
%                               Feb. 1990.
%    \end{thebibliography}
%    \PrintIndex
%    \PrintChanges}
%
%
% \section{The documentation driver file}
%
% The next bit of code contains the documentation driver file for
% \TeX{}, i.e., the file that will produce the documentation you are
% currently reading. It will be extracted from this file by the
% \texttt{docstrip} program. Since it is the first code in the file
% one can alternatively process this file directly with \LaTeXe{} to
% obtain the documentation.
%
%    \begin{macrocode}
%<*driver>
\documentclass{ltxdoc}

\usepackage{theorem}

%<+driver>% The next few lines define theorem sets which are used
%<+driver>% in the example section of the documentation.

\theoremstyle{break}       \newtheorem{Cor}{Corollary}
\theoremstyle{plain}       \newtheorem{Exa}{Example}[section]
{\theorembodyfont{\upshape}\newtheorem{Rem}{Remark}}
\theoremstyle{marginbreak} \newtheorem{Lem}[Cor]{Lemma}
\theoremstyle{change}
\theorembodyfont{\itshape} \newtheorem{Def}[Cor]{Definition}

\theoremheaderfont{\scshape}

\RecordChanges

\begin{document}
   \DocInput{theorem.dtx}
\end{document}
%</driver>
%    \end{macrocode}
%
% \section{Definition of the Macros}
%
% If the file has been loaded before, we abort immediately. If not the
% package announces itself (this is actually done at the very top of
% the file---the way it is done isn't good style so don't copy it).
% \changes{v2.0e}{89/07/19}{Spaces removed from `typeout'.}
% \changes{v2.2b}{95/11/19}{Announce moved to top of file}
%    \begin{macrocode}
%<*package>
%\@ifundefined{theorem@style}{}{\endinput}
%\def\FMithmInfo{1995/11/19 v2.2b Theorem extension package (FMi)}
%\ProvidesPackage{theorem}[\FMithmInfo]
%    \end{macrocode}
%
%
% \subsection{Definition of theorem styles and fonts}
%
% All the definitions in this file are done globally to allow
% inputting this file inside a group.
%
% \begin{macro}{\theoremstyle}
%    Before a theorem style can be installed, the chosen style must be
%    known. For that reason, we must test to see that
%    |\th@|\meta{style} is known or, more precisely, that it is
%    different from |\relax|. If the style is not known then
%    |\th@plain| is used.
%    \begin{macrocode}
\gdef\theoremstyle#1{%
   \@ifundefined{th@#1}{\@warning
          {Unknown theoremstyle `#1'. Using `plain'}%
          \theorem@style{plain}}%
%    \end{macrocode}
%    We save the theorem style to be used in the token register
%    |\theorem@style|.
%    \begin{macrocode}
      {\theorem@style{#1}}%
%    \end{macrocode}
%    Now we ``evaluate'' the theorem style: this means, we call the
%    macro |\th@|\meta{style} which will activate the relevant
%    definitions which are contained in a separate file.  This is done
%    in a group to suppress changes to the current font.  This could
%    otherwise pose problems together with the new font selection
%    scheme\footnote{When I printed the original article using the new
%    font selection scheme I ended with a document with slanted
%    typefaces (text headings and all) simply because one of the
%    theorem styles used \texttt{\bslash sl} at toplevel.} if the
%    |\th@|\meta{style} is evaluated a second time.
% \changes{v2.1a}{90/02/15}{Group added around `th@..}
% \changes{v2.1a}{90/02/15}{Reseting of `..@begintheorem removed.}
%    \begin{macrocode}
      \begingroup
        \csname th@\the\theorem@style \endcsname
      \endgroup}
%    \end{macrocode}
% \end{macro}
%
%
% \begin{macro}{\@begintheorem}
% \changes{v2.1a}{90/02/15}{Relaxed at toplevel.}
% \begin{macro}{\@opargbegintheorem}
% \changes{v2.1a}{90/02/15}{Relaxed at toplevel.}
%    We reset |\@begintheorem| and |\@opargbegintheorem| to
%    |\relax| since these commands are no longer necessary at
%    toplevel. This will save a few tokens.
%    \begin{macrocode}
\global\let\@begintheorem\relax
\global\let\@opargbegintheorem\relax
%    \end{macrocode}
% \end{macro}
% \end{macro}
%
% \begin{macro}{\theorem@style}
%    Obviously the \textsf{token} register used above has to be
%    allocated. To assure the utmost compatibility with the
%    original \LaTeX{} definition, we set the default theorem style to
%    \texttt{plain}, which implements the usual \LaTeX{} convention.
% \changes{v2.0f}{89/07/20}{Unnecessary `global' removed.}
%    \begin{macrocode}
\newtoks\theorem@style
\global\theorem@style{plain}
%    \end{macrocode}
% \end{macro}
%
% \begin{macro}{\theorembodyfont}
% \begin{macro}{\theorem@bodyfont}
%    If the |\theorembodyfont| is set by the user then it should not
%    interact with the default font set in the theorem style. When the
%    new font selection is in force this may happen if, for example,
%    the default is |\itshape| and the new |\theorembodyfont| is
%    |\sffamily|. So we add a |\reset@font| command in front of the
%    user definition.
% \changes{v2.0f}{89/07/20}{Unnecessary `global' removed.}
% \changes{v2.0g}{89/09/19}{`theoremfont to `theorembodyfont.}
% \changes{v2.1c}{92/06/22}{`reset@font added and logic changed.}
% \changes{v2.1d}{92/10/28}{Reset to empty if arg is empty}
%    \begin{macrocode}
\gdef\theorembodyfont#1{%
%    \end{macrocode}
%    We check if the argument supplied is empty and if so put nothing
%    into the |\theorem@bodyfont| token register to allow for
%    |\theorembodyfont{}| as a mean of using the default of the
%    current |\theoremstyle|.
%    \begin{macrocode}
   \def\@tempa{#1}%
   \ifx\@tempa\@empty
    \theorem@bodyfont{}%
   \else
    \theorem@bodyfont{\reset@font#1}%
   \fi
   }
\newtoks\theorem@bodyfont
\global\theorem@bodyfont{}
%    \end{macrocode}
% \end{macro}
% \end{macro}
%
%
% \begin{macro}{\theoremheaderfont}
% \changes{v2.0g}{89/09/19}{Macro added.}
%    The font for the theorem headers is handled differently because
%    this definition applies to all theorem styles.
%    \begin{macrocode}
\gdef\theoremheaderfont#1{\gdef\theorem@headerfont{#1}%
%    \end{macrocode}
%    After using the macro once it is redefined to produce an error
%    message.
%    \begin{macrocode}
       \gdef\theoremheaderfont##1{%
        \typeout{\string\theoremheaderfont\space should be used
                 only once.}}}
%    \end{macrocode}
% \end{macro}
%
% \begin{macro}{\theorem@headerfont}
% \changes{v2.0g}{89/09/19}{Macro added.}
%    To set the |\theorem@headerfont| default we first test if
%    the new fontselection scheme is in force.
% \changes{v2.1a}{90/02/15}{Support for new font selection.}
%    \begin{macrocode}
\ifx\upshape\undefined
%    \end{macrocode}
%    If not we define it to expand into |\bfseries|. We don't use
%    |\let| just in case a following style option redefines this
%    macro.
%    \begin{macrocode}
\gdef\theorem@headerfont{\bfseries}
%    \end{macrocode}
%    Otherwise we reset the current shape before calling |\bfseries|.
%    \begin{macrocode}
\else \gdef\theorem@headerfont{\normalfont\bfseries}\fi
%    \end{macrocode}
% \end{macro}
%
% \begin{macro}{\th@plain}
% \begin{macro}{\th@break}
% \begin{macro}{\th@marginbreak}
% \begin{macro}{\th@changebreak}
% \begin{macro}{\th@change}
% \begin{macro}{\th@margin}
%    The different styles are defined in macros such as
%    |\th@plain|.  Since memory space is precious in
%    ``non-Big-versions'', we have to avoid offering too many unused
%    definitions.  Therefore we define these styles in separate files
%    that can be loaded on demand.  Thus the commands themselves only
%    load these files. We use |\@input@|, a \LaTeXe{} internal command
%    that ensures that the file will be listed with |\listfiles|.
% \changes{v2.2a}{94/02/02}{Use `@input@ to load theorem layout files}
%    \begin{macrocode}
\gdef\th@plain{\@input@{thp.sty}}
\gdef\th@break{\@input@{thb.sty}}
\gdef\th@marginbreak{\@input@{thmb.sty}}
\gdef\th@changebreak{\@input@{thcb.sty}}
\gdef\th@change{\@input@{thc.sty}}
\gdef\th@margin{\@input@{thm.sty}}
%    \end{macrocode}
%    This list will be expanded when new styles become
%    available.  For testing,  just append new theorem substyles
%    as document options.
% \end{macro}
% \end{macro}
% \end{macro}
% \end{macro}
% \end{macro}
% \end{macro}
%
%
%
% \subsection{Definition of a new theorem set}
%
% As already pointed out, a new theorem environment can be defined in
% three different ways:
% \begin{verbatim}
%    \newtheorem{Lem}{Lemma}
%    \newtheorem{Lem}{Lemma}[section]
%    \newtheorem{Lem}[Theorem]{Lemma}
%\end{verbatim}
% The function of the macro |\newtheorem| is to recognize these cases
% and then to branch into one of the three macros |\@ynthm|, |\@xnthm|
% or |\@othm|.  This mechanism is adopted unchanged from
% \cite{latex:source}; the essential point here is that, for example,
% in the second case, the arguments \texttt{Lem}, \texttt{Lemma} and
% \texttt{section} are passed over to the macro |\@xnthm|.
%
% We inspect this case first because the others present fewer problems,
% and thus are easily derived from this one.
%
% \begin{macro}{\@xnthm}
%    For our example arguments, the macro |\@xnthm| must fulfill
%    the following:
%    \begin{itemize}
%       \item
%          Define a new \LaTeX-counter `Lem'
%       \item
%          reset this counter within a |\section|
%       \item
%          define the macro |\theLem|
%       \item
%          define the environment macros |\Lem| and
%          |\endLem| using the current
%          |\theoremstyle| and |\theorem@bodyfont|.
%    \end{itemize}
%    Obviously, all this should happen only if the first argument
%    of |\@xnthm| (i.e.\ \texttt{Lem} in our example) is chosen so
%    as not to conflict with any previously defined commands or
%    environments.  This test is performed by the \LaTeX{} macro
%    |\@ifdefinable|.
%    \begin{macrocode}
\gdef\@xnthm#1#2[#3]{\expandafter\@ifdefinable\csname #1\endcsname
%    \end{macrocode}
%    Therefore, the first argument of |\@ifdefinable| is the
%    expansion (in the example, |\Lem|) of
%    |\csname||#1\endcsname|.  The second argument is
%    executed only if the test has been completed successfully.
%    \begin{macrocode}
   {%
%    \end{macrocode}
%    Now we define the new counter. The names of the \LaTeX{}
%    macros employed should speak for themselves:
% \changes{v2.2b}{1995/10/07}{Use \cs{@newctr}}
% \changes{v2.2c}{1995/11/23}{Use \cs{@newctr} properly.}
%    \begin{macrocode}
    \@definecounter{#1}\@newctr{#1}[#3]%
%    \end{macrocode}
%    Using |\@newctr| will give a proper error message if the counter
%    in |#3| is not defined.
%    In defining `|\theLem|' we must generate the desired macro
%    name by use of |\expandafter| and |\csname|.
%    \begin{macrocode}
    \expandafter\xdef\csname the#1\endcsname
%    \end{macrocode}
%    An |\xdef| is used in order to make the definition global,
%    and to ensure that it contains the replacement texts of
%    |\@thmcountersep| and |\@thmcounter|.\footnote{These
%    two macros can be defined by the document style.  Their default
%    values produce a `\texttt{.}' as separation and an arabic
%    representation of the number.} However, not everything should be
%    expanded. For example, it saves space to use |\thesection|
%    instead of its---at times---lengthy expansion.
%    \begin{macrocode}
      {\expandafter \noexpand \csname the#3\endcsname
       \@thmcountersep \@thmcounter{#1}}%
%    \end{macrocode}
%    Thus with the defaults of \LaTeX{}, |\theLem| would be
%    replaced by the command sequence |\thesection.\arabic{Lem}|.
%
%    We will now look at the definition of the macro which is executed
%    at the beginning of the actual environment (in our example this
%    macro is |\Lem|). It should be noted that we use an
%    ``|\expandafter| trick'' to expand only certain parts of the
%    replacement text at the time of the definition.
%    \begin{macrocode}
    \def\@tempa{\global\@namedef{#1}}%
    \expandafter \@tempa \expandafter{%
%    \end{macrocode}
%    First, the macro that contains the current definitions of
%    |\@begintheorem| and |\@opargtheorem| should be called
%    up.  The name of this macro---as is already known---has the form
%    |\th@|\meta{theorem\ style}; therefore, it must be called by
%    \begin{macrocode}
      \csname th@\the \theorem@style
%    \end{macrocode}
%    In addition the default theorem font should be changeable, i.e.\
%    we have to insert the contents of |\theorem@bodyfont|. For
%    that reason, we expand even further, beyond |\endcsname|,
%    and thus insert the contents of the \textsf{token} register
%    |\theorem@bodyfont| in the replacement text.
%    \begin{macrocode}
            \expandafter \endcsname \the \theorem@bodyfont
%    \end{macrocode}
%    Now it is time to call the macro |\@thm| which takes over
%    the further processing.  It has two arguments: the current
%    counter name (in our example, \texttt{Lem}), and the text of the
%    label (in our example, \texttt{Lemma}).
%    \begin{macrocode}
     \@thm{#1}{#2}}%
%    \end{macrocode}
%    With this, the `sub-definition' is complete.  The macro
%    |\@endtheorem| ends a theorem environment and is, so far, nothing
%    but an |\endtrivlist|.  (Hence it is defined globally, and not
%    within the theorem styles.\footnote{This has to be changed as
%    soon as theorem styles that change \texttt{\bslash @endtheorem}
%    exist.  In such a case, all existing styles must be changed as
%    well since they will have to reset the macro.}) Therefore, we can
%    set it equivalent to the macro that ends the theorem set (in our
%    example, |\endLem|).  However, if some day theorem styles exist
%    that do change |\@endtheorem|, we would have to use the
%    commented-out line instead.
%    \begin{macrocode}
    \global \expandafter \let \csname end#1\endcsname \@endtheorem
%   \global\@namedef{end#1}{\@endtheorem}%
%    \end{macrocode}
%    With these commands all the required definitions are employed,
%    unless the test |\@ifdefinable| has failed.  Therefore, we
%    end the second argument of this macro and with it the definition
%    of |\@xnthm|.
%    \begin{macrocode}
   }}
%    \end{macrocode}
% \end{macro}
%
% \begin{macro}{\@ynthm}
%    The definition of |\@ynthm| is completely analogous. In
%    this case the new counter that is defined is not reset within
%    another counter; thus the definition of |\the...|
%    is simplified:
%    \begin{macrocode}
\gdef\@ynthm#1#2{\expandafter\@ifdefinable\csname #1\endcsname
   {\@definecounter{#1}%
    \expandafter\xdef\csname the#1\endcsname{\@thmcounter{#1}}%
%    \end{macrocode}
%    The rest of the definition corresponds literally to that of
%    |\@xnthm|:
%    \begin{macrocode}
    \def\@tempa{\global\@namedef{#1}}\expandafter \@tempa
     \expandafter{\csname th@\the \theorem@style \expandafter
     \endcsname \the\theorem@bodyfont \@thm{#1}{#2}}%
    \global \expandafter \let \csname end#1\endcsname \@endtheorem}}
%    \end{macrocode}
% \end{macro}
%
% \begin{macro}{\@othm}
%    The definition of |\@othm| does not contain anything new.
%    \begin{macrocode}
\gdef\@othm#1[#2]#3{%
%    \end{macrocode}
% \changes{v2.2b}{95/10/07}{Build in check for counter defined}
%    We do not define a new counter but instead use one that has
%    already been defined.  Thus the only definition we need is that
%    of this pseudo-counter (i.e.\ |\the|\meta{env.\ name}).
%    First we check if |#2| corresponds to a known counter name.
%    \begin{macrocode}
  \expandafter\ifx\csname c@#2\endcsname\relax
   \@nocounterr{#2}%
  \else
   \expandafter\@ifdefinable\csname #1\endcsname
   {\expandafter \xdef \csname the#1\endcsname
     {\expandafter \noexpand \csname the#2\endcsname}%
%    \end{macrocode}
%    All other parts of the definition can be adopted from
%    |\@xnthm|. We have to remember, though, that in this case
%    the name of the current counter and the theorem label have moved
%    to the second and third arguments.
%    \begin{macrocode}
    \def\@tempa{\global\@namedef{#1}}\expandafter \@tempa
     \expandafter{\csname th@\the \theorem@style \expandafter
     \endcsname \the\theorem@bodyfont \@thm{#2}{#3}}%
    \global \expandafter \let \csname end#1\endcsname \@endtheorem}%
  \fi}
%    \end{macrocode}
% \end{macro}
%
% \subsection{Macros that are employed in a {\envfont theorem}
%    environment}
%
% \begin{macro}{\@thm}
%    The macro |\@thm| has to increase the current counter.  Then,
%    depending on whether the environment has (or does not have) an
%    optional argument, it has to branch into either |\@begintheorem|
%    or |\@opargtheorem|.
%    \begin{macrocode}
\gdef\@thm#1#2{\refstepcounter{#1}%
%    \end{macrocode}
%    Now we start a {\envfont trivlist} environment, and give
%    |\@topsep| and |\@topsepadd| the values of the \textsf{skip}
%    registers |\theorempreskipamount| and |\theorempostskipamount|.
%    The value in |\@topsep| is the vertical space that is inserted by
%    the first (and only) |\item| in our |\trivlist| whilst
%    |\@topsepadd| is inserted by |\@endparenv| at the end of that
%    {\envfont trivlist} environment.  By using these registers, we
%    obtain the desired space around a {\envfont theorem} environment.
%    \begin{macrocode}
   \trivlist
   \@topsep \theorempreskipamount               % used by first \item
   \@topsepadd \theorempostskipamount           % used by \@endparenv
%    \end{macrocode}
%    Now we have to test whether an optional argument has been given.
% \changes{v2.0d}{89/07/18}{Forgotten percent added.}
%    \begin{macrocode}
   \@ifnextchar [%
%    \end{macrocode}
%    If there is an optional argument, we will call |\@ythm|, and
%    move the arguments read back into the input stream.
%    \begin{macrocode}
   {\@ythm{#1}{#2}}%
%    \end{macrocode}
%    If not, we call |\@begintheorem|. Its first argument is the
%    name of the theorem set (hence the second argument of
%    |\@thm|). Its second argument is the macro that produces the
%    current number.
%    \begin{macrocode}
   {\@begintheorem{#2}{\csname the#1\endcsname}\ignorespaces}}
%    \end{macrocode}
% \end{macro}
%
% \begin{macro}{\@xthm}
% \begin{macro}{\@ythm}
%    Both these macros were originally called by |\@thm|.  We do
%    not need |\@xthm| anymore, hence we reset it to
%    |\relax|. The definition of |\@ythm| has not changed at
%    all from its definition in \LaTeX. In order to make the macros
%    easier to understand, we will
%    nevertheless present it (commented out).
%    \begin{macrocode}
\global\let\@xthm\relax
% \def\@ythm#1#2[#3]{\@opargbegintheorem{#2}{\csname
%           the#1\endcsname}{#3}\ignorespaces}
%</package>
%    \end{macrocode}
%    The primitive |\ignorespaces| in |\@ythm| and
%    |\@thm| is needed to remove the spaces between the
%    |\begin{...}| and the actual text.
% \end{macro}
% \end{macro}
%
%
% \subsection{Definition of the theorem substyles}
%
% As already pointed out, the theorem substyles, defined below, are
% only loaded when necessary.  Note that all these substyles, except
% \texttt{plain}, have |\slshape| as the default body font.
%
% \subsubsection{The \texttt{plain} style}
%
% \label{dok:th@plain} As the following macros use |@|, we have
% to locally set the |\catcode| of this symbol to ``letter''.
% This happens within a group, so that we do not have to worry about
% which |\catcode| that symbol had before.
%    \begin{macrocode}
%<*thp>
\begingroup \makeatletter
%    \end{macrocode}
% Since we are now within a group, we must make all definitions
% globally.  First we make sure that \texttt{theorem.sty} is loaded.
% This will allow us to use this file as a document style option
% without having to call \texttt{theorem} itself as an option.  At the
% same time, we assure that at least version 2 is loaded, since
% |\theorem@style| was not defined in earlier versions.
%    \begin{macrocode}
\@ifundefined{theorem@style}{\input{theorem.sty}}{}
\ProvidesFile{thp.sty}
             [\FMithmInfo]
%    \end{macrocode}
%
%
% \begin{macro}{\th@plain}
% \begin{macro}{\@begintheorem}
% \begin{macro}{\@opargbegintheorem}
%    |\theoremstyle{plain}| corresponds to the original
%    definition, except that the distances to the surrounding text are
%    determined by the parameters |\theorempreskipamount| and
%    |\theorempostskipamount|.
%    First we set the default body font.
%    \begin{macrocode}
\gdef\th@plain{\normalfont\itshape
%    \end{macrocode}
%    Then we define |\@begintheorem| and
%    |\@opargbegintheorem|. These two macros define how the
%    header of a theorem is typeset.  |\@opargbegintheorem| will
%    be called if a \textsf{theorem} environment with an optional
%    argument is encountered; otherwise, the header is constructed by
%    calling |\@begintheorem|.  If one of these macros is
%    executed, we are within a \textsf{trivlist} environment started by
%    |\@thm|.  So the theorem header is produced with an
%    |\item| command.
%
%    Instead of specifying the header font directly, all standard
%    theorem styles use the |\theorem@headerfont| macro to allow
%    customization.  The extra space (|\labelsep|) is necessary
%    because of problems in the \textsf{trivlist} environment.
%    \begin{macrocode}
  \def\@begintheorem##1##2{%
        \item[\hskip\labelsep \theorem@headerfont ##1\ ##2]}%
%    \end{macrocode}
%    The definition of |\@opargbegintheorem| is completely
%    analogous. The only difference is the fact that there exists a
%    third argument (which is the optional parameter of the
%    environment and contains additional information about the
%    theorem).  Customarily we enclose it in parentheses.
%    \begin{macrocode}
\def\@opargbegintheorem##1##2##3{%
   \item[\hskip\labelsep \theorem@headerfont ##1\ ##2\ (##3)]}}
%    \end{macrocode}
% \end{macro}
% \end{macro}
% \end{macro}
%
% \noindent
% We conclude with an |\endgroup| to restore the
% |\catcode| of |@|.
%    \begin{macrocode}
\endgroup
%</thp>
%    \end{macrocode}
%
%
% \subsubsection{The \texttt{break} style}
%
% This style option is stored in the file |thb.sty|.
% For the next two lines see the documentation for |\th@plain| on
% page~\pageref{dok:th@plain}.
%    \begin{macrocode}
%<*thb>
\begingroup \makeatletter
\@ifundefined{theorem@style}{\input{theorem.sty}}{}
\ProvidesFile{thb.sty}
             [\FMithmInfo]
%    \end{macrocode}
%
% \begin{macro}{\th@break}
%    |\theoremstyle{break}| produces a line break after the name
%    of the theorem.  The font is |\slshape|.  Hence, we define
%    |\th@break| as follows:
%    \begin{macrocode}
\gdef\th@break{\normalfont\slshape
  \def\@begintheorem##1##2{\item[%
%    \end{macrocode}
%    We run into the following problem: it is not possible to create
%    the header with |\item[|\meta{title}|]| and then start
%    a new line by, for example, |\mbox{}\\|. Such a definition
%    will fail whenever a list environment follows immediately.  With
%    the above construction, the |\mbox{}| causes the switch
%    |@inlabel| (cf.\ definition of |\list| and
%    |\trivlist| in \cite{latex:source}) to be set to \texttt{false}
%    and so the following list will insert additional vertical
%    space (|\topskip|). This is quite annoying. Therefore, we
%    create the line break within the |\item|. In order to ensure
%    that the text will begin at the proper position in the following
%    line, we simply pretend that the label does not take any
%    room.\footnote{This will lead to problems whenever very high
%    symbols occurring in the line tower into the
%    heading. So, something else has to be done here sometime.}
%    \begin{macrocode}
       \rlap{\vbox{\hbox{\hskip \labelsep\theorem@headerfont ##1\ ##2}%
                   \hbox{\strut}}}]}%
%    \end{macrocode}
%    Again, the definition of |\@opargbegintheorem| is
%    completely analogous.
%    \begin{macrocode}
\def\@opargbegintheorem##1##2##3{%
  \item[\rlap{\vbox{\hbox{\hskip \labelsep \theorem@headerfont
                     ##1\ ##2\ (##3)}%
                    \hbox{\strut}}}]}}
\endgroup
%</thb>
%    \end{macrocode}
% \end{macro}
%
%
% \subsubsection{The \texttt{changebreak} style}
%
%    \begin{macrocode}
% This style option is stored in the file |thcb.sty|.
%    \begin{macrocode}
%<*thcb>
\begingroup \makeatletter
\@ifundefined{theorem@style}{\input{theorem.sty}}{}
\ProvidesFile{thcb.sty}
             [\FMithmInfo]
%    \end{macrocode}
%
% \begin{macro}{\th@changebreak}
%    The \texttt{change-break} theorem style is like \texttt{break}
%    but with interchange of theorem name and theorem number.  Thus we
%    define |\th@changebreak| as follows:
%    \begin{macrocode}
\gdef\th@changebreak{\normalfont\slshape
  \def\@begintheorem##1##2{\item
       [\rlap{\vbox{\hbox{\hskip\labelsep \theorem@headerfont ##2\ ##1}%
                    \hbox{\strut}}}]}%
\def\@opargbegintheorem##1##2##3{%
  \item[\rlap{\vbox{\hbox{\hskip\labelsep \theorem@headerfont
                     ##2\ ##1\ (##3)}%
                    \hbox{\strut}}}]}}
\endgroup
%</thcb>
%    \end{macrocode}
% \end{macro}
%
%
% \subsubsection{The \texttt{change} style}
%
% This style option is stored in the file |thc.sty|.
%    \begin{macrocode}
%<*thc>
\begingroup \makeatletter
\@ifundefined{theorem@style}{\input{theorem.sty}}{}
\ProvidesFile{thc.sty}
             [\FMithmInfo]
%    \end{macrocode}
%
% \begin{macro}{\th@change}
%    The \texttt{change} theorem style corresponds to the
%    \texttt{change break} style without a linebreak after the header.
%    To say it in another way, it's the same as the \texttt{plain}
%    style but with number and name interchanged and |\slshape| as the
%    default font.
%    \begin{macrocode}
\gdef\th@change{\normalfont\slshape
  \def\@begintheorem##1##2{\item
       [\hskip\labelsep \theorem@headerfont ##2\ ##1]}%
\def\@opargbegintheorem##1##2##3{%
  \item[\hskip\labelsep \theorem@headerfont ##2\ ##1\ (##3)]}}
\endgroup
%</thc>
%    \end{macrocode}
% \end{macro}
%
%
% \subsubsection{The \texttt{marginbreak} style}
%
% This style option is the one used most often at Mainz.  It is saved
% in the file |thmb.sty|.
%    \begin{macrocode}
%<*thmb>
\begingroup \makeatletter
\@ifundefined{theorem@style}{\input{theorem.sty}}{}
\ProvidesFile{thmb.sty}
             [\FMithmInfo]
%    \end{macrocode}
%
% \begin{macro}{\th@marginbreak}
%    The \texttt{margin break} style is nearly the same as the
%    \texttt{change break} style. The only difference is the placement
%    of the theorem number.  We use |\llap| to place it in the left
%    margin.
%
%    In this style |\labelsep| denotes the separation between the
%    number and the text.
%    \begin{macrocode}
\gdef\th@marginbreak{\normalfont\slshape
  \def\@begintheorem##1##2{\item
       [\rlap{\vbox{\theorem@headerfont
                    \hbox{\llap{##2}\hskip\labelsep ##1}%
                    \hbox{\strut}}}]}%
\def\@opargbegintheorem##1##2##3{%
  \item[\rlap{\vbox{\theorem@headerfont
                    \hbox{\llap{##2}\hskip\labelsep ##1\ (##3)}%
                    \hbox{\strut}}}]}}
\endgroup
%</thmb>
%    \end{macrocode}
% \changes{v2.1c}{90/02/17}{`endgroup outside macrocode env.}
% \end{macro}
%
%
%
% \subsubsection{The \texttt{margin} style}
%
% This style option is stored in the file |thm.sty|.
%    \begin{macrocode}
%<*thm>
\begingroup \makeatletter
\@ifundefined{theorem@style}{\input{theorem.sty}}{}
\ProvidesFile{thm.sty}
             [\FMithmInfo]
%    \end{macrocode}
%
% \begin{macro}{\th@margin}
%    Again this is only a variant of the theorem styles described
%    above without any new ideas.
%    \begin{macrocode}
\gdef\th@margin{\normalfont\slshape
  \def\@begintheorem##1##2{\item
       [\theorem@headerfont \llap{##2}\hskip\labelsep ##1]}%
\def\@opargbegintheorem##1##2##3{%
  \item[\theorem@headerfont \llap{##2}\hskip\labelsep ##1\ (##3)]}}
\endgroup
%</thm>
%    \end{macrocode}
% \end{macro}
%
%
%
%
% \subsection{Final Definitions}
%
% \begin{macro}{\theorempreskipamount}
% \begin{macro}{\theorempostskipamount}
%    The \textsf{skip} parameters that regulate the vertical empty space
%    before and after the {\envfont theorem} environment have
%    to be allocated as well.
% \changes{v2.0f}{89/07/20}{Unnecessary `global' removed.}
%    \begin{macrocode}
%<*package>
\newskip\theorempreskipamount
\newskip\theorempostskipamount
%    \end{macrocode}
%    Since we have used the same values for all theorem sets, we now can
%    assign them.
%    \begin{macrocode}
\global\setlength\theorempreskipamount{12pt plus 5pt minus 3pt}
\global\setlength\theorempostskipamount{8pt plus 3pt minus 1.5pt}
%    \end{macrocode}
% \end{macro}
% \end{macro}
%
%
% \begin{macro}{\@endtheorem}
%    The same holds for the macro |\@endtheorem|, which ends a
%    {\envfont theorem} environment. Since it is the same for all
%    theorem sets, it is removed from the macros |\th@|\meta{style}.
%    It simply ends the {\envfont trivlist} environment, which was
%    begun in |\@thm|.
%    \begin{macrocode}
\global\let\@endtheorem=\endtrivlist
%    \end{macrocode}
% \end{macro}
%
%
%
% \begin{macro}{\@preamblecmds}
%    All macros defined above are to be used only in the preamble.
%    Therefore, we insert them in |\@preamblecmds| which will disable
%    them at begin document. This is done by the internal \LaTeXe{}
%    command |\@onlypreamble|.
% \changes{v2.0g}{89/09/19}{`theoremheaderfont and `theorembodyfont
%    added.}
% \changes{v2.2a}{94/02/03}{Use \LaTeXe{} interface}
%    \begin{macrocode}
\@onlypreamble\@xnthm
\@onlypreamble\@ynthm
\@onlypreamble\@othm
\@onlypreamble\newtheorem
\@onlypreamble\theoremstyle
\@onlypreamble\theorembodyfont
\@onlypreamble\theoremheaderfont
%    \end{macrocode}
% \end{macro}
%
% Finally we declare the \texttt{plain} theorem style to be the default.
% \changes{v2.1b}{90/06/04}{Fixed missing `item by declaring default
%    style.}
%    \begin{macrocode}
\theoremstyle{plain}
%</package>
%    \end{macrocode}
%
% \Finale

\endinput
}{}
\ProvidesFile{thp.sty}
             [\FMithmInfo]
%    \end{macrocode}
%
%
% \begin{macro}{\th@plain}
% \begin{macro}{\@begintheorem}
% \begin{macro}{\@opargbegintheorem}
%    |\theoremstyle{plain}| corresponds to the original
%    definition, except that the distances to the surrounding text are
%    determined by the parameters |\theorempreskipamount| and
%    |\theorempostskipamount|.
%    First we set the default body font.
%    \begin{macrocode}
\gdef\th@plain{\normalfont\itshape
%    \end{macrocode}
%    Then we define |\@begintheorem| and
%    |\@opargbegintheorem|. These two macros define how the
%    header of a theorem is typeset.  |\@opargbegintheorem| will
%    be called if a \textsf{theorem} environment with an optional
%    argument is encountered; otherwise, the header is constructed by
%    calling |\@begintheorem|.  If one of these macros is
%    executed, we are within a \textsf{trivlist} environment started by
%    |\@thm|.  So the theorem header is produced with an
%    |\item| command.
%
%    Instead of specifying the header font directly, all standard
%    theorem styles use the |\theorem@headerfont| macro to allow
%    customization.  The extra space (|\labelsep|) is necessary
%    because of problems in the \textsf{trivlist} environment.
%    \begin{macrocode}
  \def\@begintheorem##1##2{%
        \item[\hskip\labelsep \theorem@headerfont ##1\ ##2]}%
%    \end{macrocode}
%    The definition of |\@opargbegintheorem| is completely
%    analogous. The only difference is the fact that there exists a
%    third argument (which is the optional parameter of the
%    environment and contains additional information about the
%    theorem).  Customarily we enclose it in parentheses.
%    \begin{macrocode}
\def\@opargbegintheorem##1##2##3{%
   \item[\hskip\labelsep \theorem@headerfont ##1\ ##2\ (##3)]}}
%    \end{macrocode}
% \end{macro}
% \end{macro}
% \end{macro}
%
% \noindent
% We conclude with an |\endgroup| to restore the
% |\catcode| of |@|.
%    \begin{macrocode}
\endgroup
%</thp>
%    \end{macrocode}
%
%
% \subsubsection{The \texttt{break} style}
%
% This style option is stored in the file |thb.sty|.
% For the next two lines see the documentation for |\th@plain| on
% page~\pageref{dok:th@plain}.
%    \begin{macrocode}
%<*thb>
\begingroup \makeatletter
\@ifundefined{theorem@style}{% \iffalse meta-comment
%
% Copyright (C) 1993-2024
%
% The LaTeX Project and any individual authors listed elsewhere
% in this file.
%
% This file is part of the Standard LaTeX `Tools Bundle'.
% -------------------------------------------------------
%
% It may be distributed and/or modified under the
% conditions of the LaTeX Project Public License, either version 1.3c
% of this license or (at your option) any later version.
% The latest version of this license is in
%    https://www.latex-project.org/lppl.txt
% and version 1.3c or later is part of all distributions of LaTeX
% version 2005/12/01 or later.
%
% The list of all files belonging to the LaTeX `Tools Bundle' is
% given in the file `manifest.txt'.
%
% \fi
%
%%
%
%
\def\FMithmInfo{2023/07/05 v2.2c Theorem extension package (FMi)}
%
% \ProvidesFile{theorem.dtx}[\FMithmInfo]
%\iffalse   % this is a METACOMMENT !
%%
%% Package `theorem' to use with LaTeX2e
%% Copyright (C) 1989-1995 Frank Mittelbach, all rights reserved.
%<+package>\@ifundefined{theorem@style}{}{\endinput}
%<+package>\NeedsTeXFormat{LaTeX2e}[1995/06/01]
%<+package>\ProvidesPackage{theorem}
%<+package>         [\FMithmInfo]
%
% \fi
%
% \changes{v1.0}{87/05/07}{First draft.}
% \changes{v1.0a}{87/05/07}{`thistheoremstyle' added.}
% \changes{v1.1a}{87/07/29}{New concept: redefine the internal
%                          \protect\LaTeX{} commands `@xnthm' etc.}
% \changes{v1.1b}{87/07/31}{`theoremfont' used to specify font.}
% \changes{v1.2a}{88/01/08}{`@endtheorem' now globally defined.}
% \changes{v1.2b}{88/01/11}{`newtheorem' only allowed in the preamble.}
% \changes{v1.3a}{88/02/14}{`theoremfont' now a toks register.}
%
% \changes{v2.0a}{88/11/16}{New implementation and new documentation.}
% \changes{v2.0a}{88/11/16}{No layout changes but internal modifications
%                           old substyles won't work.}
% \changes{v2.0b}{89/01/30}{`global' used for all definitions.}
% \changes{v2.0c}{89/06/15}{Documentation translated to English.
%                            Thanks to LLL and Nina}
% \changes{v2.1a}{90/02/15}{Support for new font selection.}
% \changes{v2.2a}{94/02/03}{Cleanup for LaTeX2e}
%
% \DoNotIndex{\@endparenv,\@ifdefinable,\@ifnextchar,\@ifundefined}
% \DoNotIndex{\@namedef,\@nameuse,\@spaces,\@tempa,\@warning}
% \DoNotIndex{\ }
% \DoNotIndex{\begingroup,\bfseries}
% \DoNotIndex{\csname}
% \DoNotIndex{\def,\do,\docdate}
% \DoNotIndex{\endcsname,\endgroup,\endinput,\endtrivlist,\expandafter}
% \DoNotIndex{\filedate,\fileversion}
% \DoNotIndex{\gdef,\global}
% \DoNotIndex{\hbox,\hskip}
% \DoNotIndex{\ignorespaces,\input,\itshape,\item}
% \DoNotIndex{\labelsep,\let,\llap}
% \DoNotIndex{\makeatletter}
% \DoNotIndex{\noexpand}
% \DoNotIndex{\refstepcounter,\relax,\rlap}
% \DoNotIndex{\setlength,\slshape,\space,\string,\strut}
% \DoNotIndex{\the,\trivlist,\typeout}
% \DoNotIndex{\vbox}
% \DoNotIndex{\xdef}
%
% \def\deflabel#1{\ttfamily #1\hfill}
% \def\deflist#1{\begin{list}{}{\settowidth\labelwidth{\ttfamily #1}%
%                               \setlength\leftmargin\labelwidth
%                               \addtolength\leftmargin\labelsep
%                               \let\makelabel\deflabel}}
% \def\enddeflist{\end{list}}
% \def\envfont{\normalfont\ttfamily}
%
% \MakeShortVerb{\|}
%
% \newcommand{\TUB}{{\sl TUGboat\/}}
%
% \GetFileInfo{theorem.dtx}
% \title{An Extension of the
%        \LaTeX{} theorem environment\thanks{This file
%        has version number \fileversion, last
%        revised \filedate.}}
% \author{Frank Mittelbach}
%
% \MaintainedByLaTeXTeam{tools}
% \maketitle
%
% \begin{abstract}
%    The macros described in this paper yield an extension of the
%    \LaTeX{} theorem mechanism. It is designed to satisfy the
%    different requirements of various journals. Thus, the
%    layout of the ``theorems'' can be manipulated by determining a
%    ``style''. This article describes not only the use, but
%    also the definition, of the necessary macros.
% \end{abstract}
%
% \section*{Preface to version 2.2}
%
% For \LaTeXe{} this package did not need any fundamental changes. I
% only modified the messages generated so that theorem layout styles
% will show up with the |\listfiles| command and cleaned the section
% on the New Font Selection Scheme since this is now included in
% \LaTeX.
%
% \section*{Preface to version 2.1}
%
% This version is identical to 2.0g described in \TUB\ 10\#3 except
% for some internal defaults which are now set depending on the used
% font selection scheme.
%
% This was done to avoid unpleasant surprises if the new font selection
% scheme is in force. For further details see
% section~\ref{sec:newfonts} and \cite{bk:GMS94}.
%
% \section{Introduction}
%
% For our purposes here, ``theorems'' are labelled enunciations,
% often set off from the main text by extra space and a font change.
% Theorems, corollaries, conjectures, definitions, and remarks
% are all instances of ``theorems''.  The ``header'' of these structures
% is composed of a label (such as \textsc{Theorem} or \textsc{Remark})
% and a number which serializes an item in the sequence of items
% with the same label.
%
% Shortly after the introduction of \LaTeX{} at the Fachbereich
% Mathematik in Mainz, the desire to manipulate the layout of
% ``theorems'' arose.  In Mainz, the following two conventions came
% into general use:
% \begin{enumerate}
%   \item The number of the theorem is shown in the margin.
%   \item There is a line break at
%         the end of the theorem header.
% \end{enumerate}
%
% Additionally, some journals require different formats which depend on
% the ``sort of theorem'': e.g.\ often remarks and definitions are set
% in |\upshape|, while |\itshape| is employed for main theorems.
%
% Confronted with these requirements, a theorem environment was
% developed in Mainz which allows separate determination of the
% layout of the ``theorems sets'', comparable to |\pagestyle|.
%
%
% \section{The user interface}
% \subsection{Defining new theorem sets}
%
% \DescribeMacro\newtheorem
% As in the original \LaTeX{} version, the command |\newtheorem|
% defines a new ``theorem set'' or ``theorem-like structure''.
% Two required arguments name the new environment and give the text
% to be typeset with each instance of the new ``set'', while
% an optional argument determines how
% the ``set'' is enumerated:
% \begin{description}
%    \item[\ttfamily \bslash newtheorem\{foo\}\{bar\}]
%       The theorem set {\envfont foo} (whose name is \texttt{bar})
%       uses its own counter.
%    \item[\ttfamily \bslash newtheorem\{foo2\}{[foo]}\{bar2\}]
%       The theorem set {\envfont foo2} (printed name \texttt{bar2})
%       uses the same counter as the theorem set \texttt{foo}.
%    \item[\ttfamily \bslash newtheorem\{foo3\}\{bar3\}{[section]}]
%       The theorem set {\envfont foo3} (printed name \texttt{bar3}) is
%       enumerated within the counter \texttt{section}, i.e.\ with every
%       new |\section| the enumeration begins again with 1, and
%       the enumeration is composed from the section-number and the
%       theorem counter itself.
% \end{description}
%
% \DescribeMacro\theoremstyle
% Additionally, the command |\theoremstyle| can define the layout
% of various, or all, theorem sets.  It should be noted that any
% theorem set defined by |\newtheorem| is typeset in the
% |\theoremstyle| that is current at the time of the definition.
% Thus, the following
% \begin{verbatim}
%    \theoremstyle{break}        \newtheorem{Cor}{Corollary}
%    \theoremstyle{plain}        \newtheorem{Exa}{Example}[section]
%\end{verbatim}                             ^^A No blank at line start
% leads to the result that the set {\envfont Cor} is formatted in the
% style \texttt{break}, while the set {\envfont Exa} and all the
% following ones are formatted in the style \texttt{plain}, unless
% another |\theoremstyle| follows.  Since the definitions installed by
% |\newtheorem| are global, one also can limit |\theoremstyle| locally
% by grouping braces.
%
% \DescribeMacro\theorembodyfont
% The choice of the font for the theorem body is completely
% independent of the chosen |\theoremstyle|; this has proven to
% be very advantageous.  For example,
% \begin{verbatim}
%   {\theorembodyfont{\upshape}          \newtheorem{Rem}{Remark}}
%\end{verbatim}
% defines a theorem set {\envfont Rem}, which will be set in
% |\upshape| in the current layout (which in our example is
% \texttt{plain}).  As with |\theoremstyle|, the |\theorembodyfont|
% chosen is that current at the time of |\newtheorem|.  If
% |\theorembodyfont| is not specified or one defines
% |\theorembodyfont{}|, then the font used will be that defined by the
% |\theoremstyle|.
%
% \DescribeMacro\theoremheaderfont It is also possible to customize
% the font used for the theorem headers.  This is, however, a global
% declaration, and therefore there should be at most one
% |\theoremheaderfont| declaration in the preamble.\footnote{If it is
% actually necessary to have different header fonts, one has to define
% new theorem styles (substituting the desired font) or specify the
% information directly in the \texttt{\bslash newtheorem} declaration
% (the unclean variant).}
%
% \DescribeMacro\theorempreskipamount
% \DescribeMacro\theorempostskipamount Two additional parameters
% affect the vertical space around the theorem environments:
% |\theorempreskipamount| and |\theorempostskipamount| define,
% respectively, the spacing before and after such an environment.
% These parameters apply for all theorem sets and can be manipulated
% with the ordinary length macros.  They are rubber lengths
% (`\textsf{skips}'), and therefore can contain \texttt{plus} and
% \texttt{minus} parts.
%
% Since the definition of theorem sets should---most sensibly---be
% placed in the preamble, we only allow installation there.
% It is therefore possible to release the memory used here after
% |\begin{document}|, in order to make room for other
% applications.
%
%
% \subsection{Existing theorem styles}
%
% The following theorem styles exist to date:
% \begin{deflist}{marginbreak}
%    \item[plain]
%       This theorem style emulates the original \LaTeX{} definition,
%       except that additionally the parameters
%       |\theorem...skipamount| are used.
%    \item[break]
%       In this style, the theorem header is followed by a line
%       break.
%    \item[marginbreak]
%       The theorem number is set in the margin, and there is a line
%       break as in \texttt{break}.
%    \item[changebreak]
%       Like \texttt{break}, but with header number and text
%       interchanged.
%    \item[change]
%       Header number and text are interchanged, without a line break.
%    \item[margin]
%       The number is set in the left margin, without a line break.
% \end{deflist}
% All styles (except \texttt{plain}) select |\slshape| as the default
% |\theorembodyfont|.
%
%
% \subsection{Examples}
%
% Given the above theorem sets {\envfont Cor}, {\envfont Exa}
% and {\envfont Rem},
% suppose that the preamble also contains the declarations:
% \begin{verbatim}
%    \theoremstyle{marginbreak}   \newtheorem{Lem}[Cor]{Lemma}
%    \theoremstyle{change}
%    \theorembodyfont{\itshape}        \newtheorem{Def}[Cor]{Definition}
%
%    \theoremheaderfont{\scshape}
%\end{verbatim}
% Then the following are some typical examples of the typeset output
% resulting from their use.
%
% \begin{Cor}
%    This is a sentence typeset in the theorem environment {\envfont
%    Cor}.
% \end{Cor}
%
% \begin{Exa}
%    This is a sentence typeset in the theorem environment {\envfont
%    Exa}.
% \end{Exa}
%
% \begin{Rem}
%    This is a sentence typeset in the theorem environment {\envfont
%    Rem}.
% \end{Rem}
%
% \begin{Lem}[Ben User]
%    This is a sentence typeset in the theorem environment {\envfont
%    Lem}.
% \end{Lem}
%
% \begin{Def}[Very impressive Definition]
%    This is a sentence typeset in the theorem environment {\envfont
%    Def}.
% \end{Def}
%
% The last two examples show the effect of the optional argument to
% a theorem environment (it is the text typeset in parentheses).
%
%
% \section{Special Considerations}
% \label{sec:newfonts}
%
% Theoremheader and body are implemented as a unit. This means that
% the |\theoremheaderfont|  will inherit characteristics of the
% |\theorembodyfont| in \LaTeXe. Thus, if for example
% |\theorembodyfont| is |\itshape| and |\theoremheaderfont|
% is |\bfseries| the font selected for the header will have the
% characteristics `bold extended italic'. If this is not desired one
% should set the |\theoremheaderfont| to something like
% \begin{verbatim}
%      \theoremheaderfont{\normalfont\bfseries}
%\end{verbatim}
% i.e.\ supplying all necessary font information explicitly.
%
%
%
% \section{Acknowledgements}
%
% The publication of this set of macros was only possible with the
% help of Christina Busse (translating the manuscript into English),
% Joachim Pense (playing the r\^ole of typist), Chris Rowley
% (looking everything over) and many others providing useful
% suggestions.
%
% \MaybeStop{%
%    \begin{thebibliography}{9}
%       \bibitem{bk:GMS94} \textsc{M.~Goossens}, \textsc{F.~Mittelbach}
%         and \textsc{A.~Samarin}.
%         \newblock The \LaTeX{} Companion.
%         \newblock
%         Addison-Wesley, Reading, Massachusetts, 1994.
%       \bibitem{latex:source} \textsc{Lamport, Leslie.} \newblock
%                               \texttt{latex.tex}, version 2.09, date
%                               Feb. 1990.
%    \end{thebibliography}
%    \PrintIndex
%    \PrintChanges}
%
%
% \section{The documentation driver file}
%
% The next bit of code contains the documentation driver file for
% \TeX{}, i.e., the file that will produce the documentation you are
% currently reading. It will be extracted from this file by the
% \texttt{docstrip} program. Since it is the first code in the file
% one can alternatively process this file directly with \LaTeXe{} to
% obtain the documentation.
%
%    \begin{macrocode}
%<*driver>
\documentclass{ltxdoc}

\usepackage{theorem}

%<+driver>% The next few lines define theorem sets which are used
%<+driver>% in the example section of the documentation.

\theoremstyle{break}       \newtheorem{Cor}{Corollary}
\theoremstyle{plain}       \newtheorem{Exa}{Example}[section]
{\theorembodyfont{\upshape}\newtheorem{Rem}{Remark}}
\theoremstyle{marginbreak} \newtheorem{Lem}[Cor]{Lemma}
\theoremstyle{change}
\theorembodyfont{\itshape} \newtheorem{Def}[Cor]{Definition}

\theoremheaderfont{\scshape}

\RecordChanges

\begin{document}
   \DocInput{theorem.dtx}
\end{document}
%</driver>
%    \end{macrocode}
%
% \section{Definition of the Macros}
%
% If the file has been loaded before, we abort immediately. If not the
% package announces itself (this is actually done at the very top of
% the file---the way it is done isn't good style so don't copy it).
% \changes{v2.0e}{89/07/19}{Spaces removed from `typeout'.}
% \changes{v2.2b}{95/11/19}{Announce moved to top of file}
%    \begin{macrocode}
%<*package>
%\@ifundefined{theorem@style}{}{\endinput}
%\def\FMithmInfo{1995/11/19 v2.2b Theorem extension package (FMi)}
%\ProvidesPackage{theorem}[\FMithmInfo]
%    \end{macrocode}
%
%
% \subsection{Definition of theorem styles and fonts}
%
% All the definitions in this file are done globally to allow
% inputting this file inside a group.
%
% \begin{macro}{\theoremstyle}
%    Before a theorem style can be installed, the chosen style must be
%    known. For that reason, we must test to see that
%    |\th@|\meta{style} is known or, more precisely, that it is
%    different from |\relax|. If the style is not known then
%    |\th@plain| is used.
%    \begin{macrocode}
\gdef\theoremstyle#1{%
   \@ifundefined{th@#1}{\@warning
          {Unknown theoremstyle `#1'. Using `plain'}%
          \theorem@style{plain}}%
%    \end{macrocode}
%    We save the theorem style to be used in the token register
%    |\theorem@style|.
%    \begin{macrocode}
      {\theorem@style{#1}}%
%    \end{macrocode}
%    Now we ``evaluate'' the theorem style: this means, we call the
%    macro |\th@|\meta{style} which will activate the relevant
%    definitions which are contained in a separate file.  This is done
%    in a group to suppress changes to the current font.  This could
%    otherwise pose problems together with the new font selection
%    scheme\footnote{When I printed the original article using the new
%    font selection scheme I ended with a document with slanted
%    typefaces (text headings and all) simply because one of the
%    theorem styles used \texttt{\bslash sl} at toplevel.} if the
%    |\th@|\meta{style} is evaluated a second time.
% \changes{v2.1a}{90/02/15}{Group added around `th@..}
% \changes{v2.1a}{90/02/15}{Reseting of `..@begintheorem removed.}
%    \begin{macrocode}
      \begingroup
        \csname th@\the\theorem@style \endcsname
      \endgroup}
%    \end{macrocode}
% \end{macro}
%
%
% \begin{macro}{\@begintheorem}
% \changes{v2.1a}{90/02/15}{Relaxed at toplevel.}
% \begin{macro}{\@opargbegintheorem}
% \changes{v2.1a}{90/02/15}{Relaxed at toplevel.}
%    We reset |\@begintheorem| and |\@opargbegintheorem| to
%    |\relax| since these commands are no longer necessary at
%    toplevel. This will save a few tokens.
%    \begin{macrocode}
\global\let\@begintheorem\relax
\global\let\@opargbegintheorem\relax
%    \end{macrocode}
% \end{macro}
% \end{macro}
%
% \begin{macro}{\theorem@style}
%    Obviously the \textsf{token} register used above has to be
%    allocated. To assure the utmost compatibility with the
%    original \LaTeX{} definition, we set the default theorem style to
%    \texttt{plain}, which implements the usual \LaTeX{} convention.
% \changes{v2.0f}{89/07/20}{Unnecessary `global' removed.}
%    \begin{macrocode}
\newtoks\theorem@style
\global\theorem@style{plain}
%    \end{macrocode}
% \end{macro}
%
% \begin{macro}{\theorembodyfont}
% \begin{macro}{\theorem@bodyfont}
%    If the |\theorembodyfont| is set by the user then it should not
%    interact with the default font set in the theorem style. When the
%    new font selection is in force this may happen if, for example,
%    the default is |\itshape| and the new |\theorembodyfont| is
%    |\sffamily|. So we add a |\reset@font| command in front of the
%    user definition.
% \changes{v2.0f}{89/07/20}{Unnecessary `global' removed.}
% \changes{v2.0g}{89/09/19}{`theoremfont to `theorembodyfont.}
% \changes{v2.1c}{92/06/22}{`reset@font added and logic changed.}
% \changes{v2.1d}{92/10/28}{Reset to empty if arg is empty}
%    \begin{macrocode}
\gdef\theorembodyfont#1{%
%    \end{macrocode}
%    We check if the argument supplied is empty and if so put nothing
%    into the |\theorem@bodyfont| token register to allow for
%    |\theorembodyfont{}| as a mean of using the default of the
%    current |\theoremstyle|.
%    \begin{macrocode}
   \def\@tempa{#1}%
   \ifx\@tempa\@empty
    \theorem@bodyfont{}%
   \else
    \theorem@bodyfont{\reset@font#1}%
   \fi
   }
\newtoks\theorem@bodyfont
\global\theorem@bodyfont{}
%    \end{macrocode}
% \end{macro}
% \end{macro}
%
%
% \begin{macro}{\theoremheaderfont}
% \changes{v2.0g}{89/09/19}{Macro added.}
%    The font for the theorem headers is handled differently because
%    this definition applies to all theorem styles.
%    \begin{macrocode}
\gdef\theoremheaderfont#1{\gdef\theorem@headerfont{#1}%
%    \end{macrocode}
%    After using the macro once it is redefined to produce an error
%    message.
%    \begin{macrocode}
       \gdef\theoremheaderfont##1{%
        \typeout{\string\theoremheaderfont\space should be used
                 only once.}}}
%    \end{macrocode}
% \end{macro}
%
% \begin{macro}{\theorem@headerfont}
% \changes{v2.0g}{89/09/19}{Macro added.}
%    To set the |\theorem@headerfont| default we first test if
%    the new fontselection scheme is in force.
% \changes{v2.1a}{90/02/15}{Support for new font selection.}
%    \begin{macrocode}
\ifx\upshape\undefined
%    \end{macrocode}
%    If not we define it to expand into |\bfseries|. We don't use
%    |\let| just in case a following style option redefines this
%    macro.
%    \begin{macrocode}
\gdef\theorem@headerfont{\bfseries}
%    \end{macrocode}
%    Otherwise we reset the current shape before calling |\bfseries|.
%    \begin{macrocode}
\else \gdef\theorem@headerfont{\normalfont\bfseries}\fi
%    \end{macrocode}
% \end{macro}
%
% \begin{macro}{\th@plain}
% \begin{macro}{\th@break}
% \begin{macro}{\th@marginbreak}
% \begin{macro}{\th@changebreak}
% \begin{macro}{\th@change}
% \begin{macro}{\th@margin}
%    The different styles are defined in macros such as
%    |\th@plain|.  Since memory space is precious in
%    ``non-Big-versions'', we have to avoid offering too many unused
%    definitions.  Therefore we define these styles in separate files
%    that can be loaded on demand.  Thus the commands themselves only
%    load these files. We use |\@input@|, a \LaTeXe{} internal command
%    that ensures that the file will be listed with |\listfiles|.
% \changes{v2.2a}{94/02/02}{Use `@input@ to load theorem layout files}
%    \begin{macrocode}
\gdef\th@plain{\@input@{thp.sty}}
\gdef\th@break{\@input@{thb.sty}}
\gdef\th@marginbreak{\@input@{thmb.sty}}
\gdef\th@changebreak{\@input@{thcb.sty}}
\gdef\th@change{\@input@{thc.sty}}
\gdef\th@margin{\@input@{thm.sty}}
%    \end{macrocode}
%    This list will be expanded when new styles become
%    available.  For testing,  just append new theorem substyles
%    as document options.
% \end{macro}
% \end{macro}
% \end{macro}
% \end{macro}
% \end{macro}
% \end{macro}
%
%
%
% \subsection{Definition of a new theorem set}
%
% As already pointed out, a new theorem environment can be defined in
% three different ways:
% \begin{verbatim}
%    \newtheorem{Lem}{Lemma}
%    \newtheorem{Lem}{Lemma}[section]
%    \newtheorem{Lem}[Theorem]{Lemma}
%\end{verbatim}
% The function of the macro |\newtheorem| is to recognize these cases
% and then to branch into one of the three macros |\@ynthm|, |\@xnthm|
% or |\@othm|.  This mechanism is adopted unchanged from
% \cite{latex:source}; the essential point here is that, for example,
% in the second case, the arguments \texttt{Lem}, \texttt{Lemma} and
% \texttt{section} are passed over to the macro |\@xnthm|.
%
% We inspect this case first because the others present fewer problems,
% and thus are easily derived from this one.
%
% \begin{macro}{\@xnthm}
%    For our example arguments, the macro |\@xnthm| must fulfill
%    the following:
%    \begin{itemize}
%       \item
%          Define a new \LaTeX-counter `Lem'
%       \item
%          reset this counter within a |\section|
%       \item
%          define the macro |\theLem|
%       \item
%          define the environment macros |\Lem| and
%          |\endLem| using the current
%          |\theoremstyle| and |\theorem@bodyfont|.
%    \end{itemize}
%    Obviously, all this should happen only if the first argument
%    of |\@xnthm| (i.e.\ \texttt{Lem} in our example) is chosen so
%    as not to conflict with any previously defined commands or
%    environments.  This test is performed by the \LaTeX{} macro
%    |\@ifdefinable|.
%    \begin{macrocode}
\gdef\@xnthm#1#2[#3]{\expandafter\@ifdefinable\csname #1\endcsname
%    \end{macrocode}
%    Therefore, the first argument of |\@ifdefinable| is the
%    expansion (in the example, |\Lem|) of
%    |\csname||#1\endcsname|.  The second argument is
%    executed only if the test has been completed successfully.
%    \begin{macrocode}
   {%
%    \end{macrocode}
%    Now we define the new counter. The names of the \LaTeX{}
%    macros employed should speak for themselves:
% \changes{v2.2b}{1995/10/07}{Use \cs{@newctr}}
% \changes{v2.2c}{1995/11/23}{Use \cs{@newctr} properly.}
%    \begin{macrocode}
    \@definecounter{#1}\@newctr{#1}[#3]%
%    \end{macrocode}
%    Using |\@newctr| will give a proper error message if the counter
%    in |#3| is not defined.
%    In defining `|\theLem|' we must generate the desired macro
%    name by use of |\expandafter| and |\csname|.
%    \begin{macrocode}
    \expandafter\xdef\csname the#1\endcsname
%    \end{macrocode}
%    An |\xdef| is used in order to make the definition global,
%    and to ensure that it contains the replacement texts of
%    |\@thmcountersep| and |\@thmcounter|.\footnote{These
%    two macros can be defined by the document style.  Their default
%    values produce a `\texttt{.}' as separation and an arabic
%    representation of the number.} However, not everything should be
%    expanded. For example, it saves space to use |\thesection|
%    instead of its---at times---lengthy expansion.
%    \begin{macrocode}
      {\expandafter \noexpand \csname the#3\endcsname
       \@thmcountersep \@thmcounter{#1}}%
%    \end{macrocode}
%    Thus with the defaults of \LaTeX{}, |\theLem| would be
%    replaced by the command sequence |\thesection.\arabic{Lem}|.
%
%    We will now look at the definition of the macro which is executed
%    at the beginning of the actual environment (in our example this
%    macro is |\Lem|). It should be noted that we use an
%    ``|\expandafter| trick'' to expand only certain parts of the
%    replacement text at the time of the definition.
%    \begin{macrocode}
    \def\@tempa{\global\@namedef{#1}}%
    \expandafter \@tempa \expandafter{%
%    \end{macrocode}
%    First, the macro that contains the current definitions of
%    |\@begintheorem| and |\@opargtheorem| should be called
%    up.  The name of this macro---as is already known---has the form
%    |\th@|\meta{theorem\ style}; therefore, it must be called by
%    \begin{macrocode}
      \csname th@\the \theorem@style
%    \end{macrocode}
%    In addition the default theorem font should be changeable, i.e.\
%    we have to insert the contents of |\theorem@bodyfont|. For
%    that reason, we expand even further, beyond |\endcsname|,
%    and thus insert the contents of the \textsf{token} register
%    |\theorem@bodyfont| in the replacement text.
%    \begin{macrocode}
            \expandafter \endcsname \the \theorem@bodyfont
%    \end{macrocode}
%    Now it is time to call the macro |\@thm| which takes over
%    the further processing.  It has two arguments: the current
%    counter name (in our example, \texttt{Lem}), and the text of the
%    label (in our example, \texttt{Lemma}).
%    \begin{macrocode}
     \@thm{#1}{#2}}%
%    \end{macrocode}
%    With this, the `sub-definition' is complete.  The macro
%    |\@endtheorem| ends a theorem environment and is, so far, nothing
%    but an |\endtrivlist|.  (Hence it is defined globally, and not
%    within the theorem styles.\footnote{This has to be changed as
%    soon as theorem styles that change \texttt{\bslash @endtheorem}
%    exist.  In such a case, all existing styles must be changed as
%    well since they will have to reset the macro.}) Therefore, we can
%    set it equivalent to the macro that ends the theorem set (in our
%    example, |\endLem|).  However, if some day theorem styles exist
%    that do change |\@endtheorem|, we would have to use the
%    commented-out line instead.
%    \begin{macrocode}
    \global \expandafter \let \csname end#1\endcsname \@endtheorem
%   \global\@namedef{end#1}{\@endtheorem}%
%    \end{macrocode}
%    With these commands all the required definitions are employed,
%    unless the test |\@ifdefinable| has failed.  Therefore, we
%    end the second argument of this macro and with it the definition
%    of |\@xnthm|.
%    \begin{macrocode}
   }}
%    \end{macrocode}
% \end{macro}
%
% \begin{macro}{\@ynthm}
%    The definition of |\@ynthm| is completely analogous. In
%    this case the new counter that is defined is not reset within
%    another counter; thus the definition of |\the...|
%    is simplified:
%    \begin{macrocode}
\gdef\@ynthm#1#2{\expandafter\@ifdefinable\csname #1\endcsname
   {\@definecounter{#1}%
    \expandafter\xdef\csname the#1\endcsname{\@thmcounter{#1}}%
%    \end{macrocode}
%    The rest of the definition corresponds literally to that of
%    |\@xnthm|:
%    \begin{macrocode}
    \def\@tempa{\global\@namedef{#1}}\expandafter \@tempa
     \expandafter{\csname th@\the \theorem@style \expandafter
     \endcsname \the\theorem@bodyfont \@thm{#1}{#2}}%
    \global \expandafter \let \csname end#1\endcsname \@endtheorem}}
%    \end{macrocode}
% \end{macro}
%
% \begin{macro}{\@othm}
%    The definition of |\@othm| does not contain anything new.
%    \begin{macrocode}
\gdef\@othm#1[#2]#3{%
%    \end{macrocode}
% \changes{v2.2b}{95/10/07}{Build in check for counter defined}
%    We do not define a new counter but instead use one that has
%    already been defined.  Thus the only definition we need is that
%    of this pseudo-counter (i.e.\ |\the|\meta{env.\ name}).
%    First we check if |#2| corresponds to a known counter name.
%    \begin{macrocode}
  \expandafter\ifx\csname c@#2\endcsname\relax
   \@nocounterr{#2}%
  \else
   \expandafter\@ifdefinable\csname #1\endcsname
   {\expandafter \xdef \csname the#1\endcsname
     {\expandafter \noexpand \csname the#2\endcsname}%
%    \end{macrocode}
%    All other parts of the definition can be adopted from
%    |\@xnthm|. We have to remember, though, that in this case
%    the name of the current counter and the theorem label have moved
%    to the second and third arguments.
%    \begin{macrocode}
    \def\@tempa{\global\@namedef{#1}}\expandafter \@tempa
     \expandafter{\csname th@\the \theorem@style \expandafter
     \endcsname \the\theorem@bodyfont \@thm{#2}{#3}}%
    \global \expandafter \let \csname end#1\endcsname \@endtheorem}%
  \fi}
%    \end{macrocode}
% \end{macro}
%
% \subsection{Macros that are employed in a {\envfont theorem}
%    environment}
%
% \begin{macro}{\@thm}
%    The macro |\@thm| has to increase the current counter.  Then,
%    depending on whether the environment has (or does not have) an
%    optional argument, it has to branch into either |\@begintheorem|
%    or |\@opargtheorem|.
%    \begin{macrocode}
\gdef\@thm#1#2{\refstepcounter{#1}%
%    \end{macrocode}
%    Now we start a {\envfont trivlist} environment, and give
%    |\@topsep| and |\@topsepadd| the values of the \textsf{skip}
%    registers |\theorempreskipamount| and |\theorempostskipamount|.
%    The value in |\@topsep| is the vertical space that is inserted by
%    the first (and only) |\item| in our |\trivlist| whilst
%    |\@topsepadd| is inserted by |\@endparenv| at the end of that
%    {\envfont trivlist} environment.  By using these registers, we
%    obtain the desired space around a {\envfont theorem} environment.
%    \begin{macrocode}
   \trivlist
   \@topsep \theorempreskipamount               % used by first \item
   \@topsepadd \theorempostskipamount           % used by \@endparenv
%    \end{macrocode}
%    Now we have to test whether an optional argument has been given.
% \changes{v2.0d}{89/07/18}{Forgotten percent added.}
%    \begin{macrocode}
   \@ifnextchar [%
%    \end{macrocode}
%    If there is an optional argument, we will call |\@ythm|, and
%    move the arguments read back into the input stream.
%    \begin{macrocode}
   {\@ythm{#1}{#2}}%
%    \end{macrocode}
%    If not, we call |\@begintheorem|. Its first argument is the
%    name of the theorem set (hence the second argument of
%    |\@thm|). Its second argument is the macro that produces the
%    current number.
%    \begin{macrocode}
   {\@begintheorem{#2}{\csname the#1\endcsname}\ignorespaces}}
%    \end{macrocode}
% \end{macro}
%
% \begin{macro}{\@xthm}
% \begin{macro}{\@ythm}
%    Both these macros were originally called by |\@thm|.  We do
%    not need |\@xthm| anymore, hence we reset it to
%    |\relax|. The definition of |\@ythm| has not changed at
%    all from its definition in \LaTeX. In order to make the macros
%    easier to understand, we will
%    nevertheless present it (commented out).
%    \begin{macrocode}
\global\let\@xthm\relax
% \def\@ythm#1#2[#3]{\@opargbegintheorem{#2}{\csname
%           the#1\endcsname}{#3}\ignorespaces}
%</package>
%    \end{macrocode}
%    The primitive |\ignorespaces| in |\@ythm| and
%    |\@thm| is needed to remove the spaces between the
%    |\begin{...}| and the actual text.
% \end{macro}
% \end{macro}
%
%
% \subsection{Definition of the theorem substyles}
%
% As already pointed out, the theorem substyles, defined below, are
% only loaded when necessary.  Note that all these substyles, except
% \texttt{plain}, have |\slshape| as the default body font.
%
% \subsubsection{The \texttt{plain} style}
%
% \label{dok:th@plain} As the following macros use |@|, we have
% to locally set the |\catcode| of this symbol to ``letter''.
% This happens within a group, so that we do not have to worry about
% which |\catcode| that symbol had before.
%    \begin{macrocode}
%<*thp>
\begingroup \makeatletter
%    \end{macrocode}
% Since we are now within a group, we must make all definitions
% globally.  First we make sure that \texttt{theorem.sty} is loaded.
% This will allow us to use this file as a document style option
% without having to call \texttt{theorem} itself as an option.  At the
% same time, we assure that at least version 2 is loaded, since
% |\theorem@style| was not defined in earlier versions.
%    \begin{macrocode}
\@ifundefined{theorem@style}{\input{theorem.sty}}{}
\ProvidesFile{thp.sty}
             [\FMithmInfo]
%    \end{macrocode}
%
%
% \begin{macro}{\th@plain}
% \begin{macro}{\@begintheorem}
% \begin{macro}{\@opargbegintheorem}
%    |\theoremstyle{plain}| corresponds to the original
%    definition, except that the distances to the surrounding text are
%    determined by the parameters |\theorempreskipamount| and
%    |\theorempostskipamount|.
%    First we set the default body font.
%    \begin{macrocode}
\gdef\th@plain{\normalfont\itshape
%    \end{macrocode}
%    Then we define |\@begintheorem| and
%    |\@opargbegintheorem|. These two macros define how the
%    header of a theorem is typeset.  |\@opargbegintheorem| will
%    be called if a \textsf{theorem} environment with an optional
%    argument is encountered; otherwise, the header is constructed by
%    calling |\@begintheorem|.  If one of these macros is
%    executed, we are within a \textsf{trivlist} environment started by
%    |\@thm|.  So the theorem header is produced with an
%    |\item| command.
%
%    Instead of specifying the header font directly, all standard
%    theorem styles use the |\theorem@headerfont| macro to allow
%    customization.  The extra space (|\labelsep|) is necessary
%    because of problems in the \textsf{trivlist} environment.
%    \begin{macrocode}
  \def\@begintheorem##1##2{%
        \item[\hskip\labelsep \theorem@headerfont ##1\ ##2]}%
%    \end{macrocode}
%    The definition of |\@opargbegintheorem| is completely
%    analogous. The only difference is the fact that there exists a
%    third argument (which is the optional parameter of the
%    environment and contains additional information about the
%    theorem).  Customarily we enclose it in parentheses.
%    \begin{macrocode}
\def\@opargbegintheorem##1##2##3{%
   \item[\hskip\labelsep \theorem@headerfont ##1\ ##2\ (##3)]}}
%    \end{macrocode}
% \end{macro}
% \end{macro}
% \end{macro}
%
% \noindent
% We conclude with an |\endgroup| to restore the
% |\catcode| of |@|.
%    \begin{macrocode}
\endgroup
%</thp>
%    \end{macrocode}
%
%
% \subsubsection{The \texttt{break} style}
%
% This style option is stored in the file |thb.sty|.
% For the next two lines see the documentation for |\th@plain| on
% page~\pageref{dok:th@plain}.
%    \begin{macrocode}
%<*thb>
\begingroup \makeatletter
\@ifundefined{theorem@style}{\input{theorem.sty}}{}
\ProvidesFile{thb.sty}
             [\FMithmInfo]
%    \end{macrocode}
%
% \begin{macro}{\th@break}
%    |\theoremstyle{break}| produces a line break after the name
%    of the theorem.  The font is |\slshape|.  Hence, we define
%    |\th@break| as follows:
%    \begin{macrocode}
\gdef\th@break{\normalfont\slshape
  \def\@begintheorem##1##2{\item[%
%    \end{macrocode}
%    We run into the following problem: it is not possible to create
%    the header with |\item[|\meta{title}|]| and then start
%    a new line by, for example, |\mbox{}\\|. Such a definition
%    will fail whenever a list environment follows immediately.  With
%    the above construction, the |\mbox{}| causes the switch
%    |@inlabel| (cf.\ definition of |\list| and
%    |\trivlist| in \cite{latex:source}) to be set to \texttt{false}
%    and so the following list will insert additional vertical
%    space (|\topskip|). This is quite annoying. Therefore, we
%    create the line break within the |\item|. In order to ensure
%    that the text will begin at the proper position in the following
%    line, we simply pretend that the label does not take any
%    room.\footnote{This will lead to problems whenever very high
%    symbols occurring in the line tower into the
%    heading. So, something else has to be done here sometime.}
%    \begin{macrocode}
       \rlap{\vbox{\hbox{\hskip \labelsep\theorem@headerfont ##1\ ##2}%
                   \hbox{\strut}}}]}%
%    \end{macrocode}
%    Again, the definition of |\@opargbegintheorem| is
%    completely analogous.
%    \begin{macrocode}
\def\@opargbegintheorem##1##2##3{%
  \item[\rlap{\vbox{\hbox{\hskip \labelsep \theorem@headerfont
                     ##1\ ##2\ (##3)}%
                    \hbox{\strut}}}]}}
\endgroup
%</thb>
%    \end{macrocode}
% \end{macro}
%
%
% \subsubsection{The \texttt{changebreak} style}
%
%    \begin{macrocode}
% This style option is stored in the file |thcb.sty|.
%    \begin{macrocode}
%<*thcb>
\begingroup \makeatletter
\@ifundefined{theorem@style}{\input{theorem.sty}}{}
\ProvidesFile{thcb.sty}
             [\FMithmInfo]
%    \end{macrocode}
%
% \begin{macro}{\th@changebreak}
%    The \texttt{change-break} theorem style is like \texttt{break}
%    but with interchange of theorem name and theorem number.  Thus we
%    define |\th@changebreak| as follows:
%    \begin{macrocode}
\gdef\th@changebreak{\normalfont\slshape
  \def\@begintheorem##1##2{\item
       [\rlap{\vbox{\hbox{\hskip\labelsep \theorem@headerfont ##2\ ##1}%
                    \hbox{\strut}}}]}%
\def\@opargbegintheorem##1##2##3{%
  \item[\rlap{\vbox{\hbox{\hskip\labelsep \theorem@headerfont
                     ##2\ ##1\ (##3)}%
                    \hbox{\strut}}}]}}
\endgroup
%</thcb>
%    \end{macrocode}
% \end{macro}
%
%
% \subsubsection{The \texttt{change} style}
%
% This style option is stored in the file |thc.sty|.
%    \begin{macrocode}
%<*thc>
\begingroup \makeatletter
\@ifundefined{theorem@style}{\input{theorem.sty}}{}
\ProvidesFile{thc.sty}
             [\FMithmInfo]
%    \end{macrocode}
%
% \begin{macro}{\th@change}
%    The \texttt{change} theorem style corresponds to the
%    \texttt{change break} style without a linebreak after the header.
%    To say it in another way, it's the same as the \texttt{plain}
%    style but with number and name interchanged and |\slshape| as the
%    default font.
%    \begin{macrocode}
\gdef\th@change{\normalfont\slshape
  \def\@begintheorem##1##2{\item
       [\hskip\labelsep \theorem@headerfont ##2\ ##1]}%
\def\@opargbegintheorem##1##2##3{%
  \item[\hskip\labelsep \theorem@headerfont ##2\ ##1\ (##3)]}}
\endgroup
%</thc>
%    \end{macrocode}
% \end{macro}
%
%
% \subsubsection{The \texttt{marginbreak} style}
%
% This style option is the one used most often at Mainz.  It is saved
% in the file |thmb.sty|.
%    \begin{macrocode}
%<*thmb>
\begingroup \makeatletter
\@ifundefined{theorem@style}{\input{theorem.sty}}{}
\ProvidesFile{thmb.sty}
             [\FMithmInfo]
%    \end{macrocode}
%
% \begin{macro}{\th@marginbreak}
%    The \texttt{margin break} style is nearly the same as the
%    \texttt{change break} style. The only difference is the placement
%    of the theorem number.  We use |\llap| to place it in the left
%    margin.
%
%    In this style |\labelsep| denotes the separation between the
%    number and the text.
%    \begin{macrocode}
\gdef\th@marginbreak{\normalfont\slshape
  \def\@begintheorem##1##2{\item
       [\rlap{\vbox{\theorem@headerfont
                    \hbox{\llap{##2}\hskip\labelsep ##1}%
                    \hbox{\strut}}}]}%
\def\@opargbegintheorem##1##2##3{%
  \item[\rlap{\vbox{\theorem@headerfont
                    \hbox{\llap{##2}\hskip\labelsep ##1\ (##3)}%
                    \hbox{\strut}}}]}}
\endgroup
%</thmb>
%    \end{macrocode}
% \changes{v2.1c}{90/02/17}{`endgroup outside macrocode env.}
% \end{macro}
%
%
%
% \subsubsection{The \texttt{margin} style}
%
% This style option is stored in the file |thm.sty|.
%    \begin{macrocode}
%<*thm>
\begingroup \makeatletter
\@ifundefined{theorem@style}{\input{theorem.sty}}{}
\ProvidesFile{thm.sty}
             [\FMithmInfo]
%    \end{macrocode}
%
% \begin{macro}{\th@margin}
%    Again this is only a variant of the theorem styles described
%    above without any new ideas.
%    \begin{macrocode}
\gdef\th@margin{\normalfont\slshape
  \def\@begintheorem##1##2{\item
       [\theorem@headerfont \llap{##2}\hskip\labelsep ##1]}%
\def\@opargbegintheorem##1##2##3{%
  \item[\theorem@headerfont \llap{##2}\hskip\labelsep ##1\ (##3)]}}
\endgroup
%</thm>
%    \end{macrocode}
% \end{macro}
%
%
%
%
% \subsection{Final Definitions}
%
% \begin{macro}{\theorempreskipamount}
% \begin{macro}{\theorempostskipamount}
%    The \textsf{skip} parameters that regulate the vertical empty space
%    before and after the {\envfont theorem} environment have
%    to be allocated as well.
% \changes{v2.0f}{89/07/20}{Unnecessary `global' removed.}
%    \begin{macrocode}
%<*package>
\newskip\theorempreskipamount
\newskip\theorempostskipamount
%    \end{macrocode}
%    Since we have used the same values for all theorem sets, we now can
%    assign them.
%    \begin{macrocode}
\global\setlength\theorempreskipamount{12pt plus 5pt minus 3pt}
\global\setlength\theorempostskipamount{8pt plus 3pt minus 1.5pt}
%    \end{macrocode}
% \end{macro}
% \end{macro}
%
%
% \begin{macro}{\@endtheorem}
%    The same holds for the macro |\@endtheorem|, which ends a
%    {\envfont theorem} environment. Since it is the same for all
%    theorem sets, it is removed from the macros |\th@|\meta{style}.
%    It simply ends the {\envfont trivlist} environment, which was
%    begun in |\@thm|.
%    \begin{macrocode}
\global\let\@endtheorem=\endtrivlist
%    \end{macrocode}
% \end{macro}
%
%
%
% \begin{macro}{\@preamblecmds}
%    All macros defined above are to be used only in the preamble.
%    Therefore, we insert them in |\@preamblecmds| which will disable
%    them at begin document. This is done by the internal \LaTeXe{}
%    command |\@onlypreamble|.
% \changes{v2.0g}{89/09/19}{`theoremheaderfont and `theorembodyfont
%    added.}
% \changes{v2.2a}{94/02/03}{Use \LaTeXe{} interface}
%    \begin{macrocode}
\@onlypreamble\@xnthm
\@onlypreamble\@ynthm
\@onlypreamble\@othm
\@onlypreamble\newtheorem
\@onlypreamble\theoremstyle
\@onlypreamble\theorembodyfont
\@onlypreamble\theoremheaderfont
%    \end{macrocode}
% \end{macro}
%
% Finally we declare the \texttt{plain} theorem style to be the default.
% \changes{v2.1b}{90/06/04}{Fixed missing `item by declaring default
%    style.}
%    \begin{macrocode}
\theoremstyle{plain}
%</package>
%    \end{macrocode}
%
% \Finale

\endinput
}{}
\ProvidesFile{thb.sty}
             [\FMithmInfo]
%    \end{macrocode}
%
% \begin{macro}{\th@break}
%    |\theoremstyle{break}| produces a line break after the name
%    of the theorem.  The font is |\slshape|.  Hence, we define
%    |\th@break| as follows:
%    \begin{macrocode}
\gdef\th@break{\normalfont\slshape
  \def\@begintheorem##1##2{\item[%
%    \end{macrocode}
%    We run into the following problem: it is not possible to create
%    the header with |\item[|\meta{title}|]| and then start
%    a new line by, for example, |\mbox{}\\|. Such a definition
%    will fail whenever a list environment follows immediately.  With
%    the above construction, the |\mbox{}| causes the switch
%    |@inlabel| (cf.\ definition of |\list| and
%    |\trivlist| in \cite{latex:source}) to be set to \texttt{false}
%    and so the following list will insert additional vertical
%    space (|\topskip|). This is quite annoying. Therefore, we
%    create the line break within the |\item|. In order to ensure
%    that the text will begin at the proper position in the following
%    line, we simply pretend that the label does not take any
%    room.\footnote{This will lead to problems whenever very high
%    symbols occurring in the line tower into the
%    heading. So, something else has to be done here sometime.}
%    \begin{macrocode}
       \rlap{\vbox{\hbox{\hskip \labelsep\theorem@headerfont ##1\ ##2}%
                   \hbox{\strut}}}]}%
%    \end{macrocode}
%    Again, the definition of |\@opargbegintheorem| is
%    completely analogous.
%    \begin{macrocode}
\def\@opargbegintheorem##1##2##3{%
  \item[\rlap{\vbox{\hbox{\hskip \labelsep \theorem@headerfont
                     ##1\ ##2\ (##3)}%
                    \hbox{\strut}}}]}}
\endgroup
%</thb>
%    \end{macrocode}
% \end{macro}
%
%
% \subsubsection{The \texttt{changebreak} style}
%
%    \begin{macrocode}
% This style option is stored in the file |thcb.sty|.
%    \begin{macrocode}
%<*thcb>
\begingroup \makeatletter
\@ifundefined{theorem@style}{% \iffalse meta-comment
%
% Copyright (C) 1993-2024
%
% The LaTeX Project and any individual authors listed elsewhere
% in this file.
%
% This file is part of the Standard LaTeX `Tools Bundle'.
% -------------------------------------------------------
%
% It may be distributed and/or modified under the
% conditions of the LaTeX Project Public License, either version 1.3c
% of this license or (at your option) any later version.
% The latest version of this license is in
%    https://www.latex-project.org/lppl.txt
% and version 1.3c or later is part of all distributions of LaTeX
% version 2005/12/01 or later.
%
% The list of all files belonging to the LaTeX `Tools Bundle' is
% given in the file `manifest.txt'.
%
% \fi
%
%%
%
%
\def\FMithmInfo{2023/07/05 v2.2c Theorem extension package (FMi)}
%
% \ProvidesFile{theorem.dtx}[\FMithmInfo]
%\iffalse   % this is a METACOMMENT !
%%
%% Package `theorem' to use with LaTeX2e
%% Copyright (C) 1989-1995 Frank Mittelbach, all rights reserved.
%<+package>\@ifundefined{theorem@style}{}{\endinput}
%<+package>\NeedsTeXFormat{LaTeX2e}[1995/06/01]
%<+package>\ProvidesPackage{theorem}
%<+package>         [\FMithmInfo]
%
% \fi
%
% \changes{v1.0}{87/05/07}{First draft.}
% \changes{v1.0a}{87/05/07}{`thistheoremstyle' added.}
% \changes{v1.1a}{87/07/29}{New concept: redefine the internal
%                          \protect\LaTeX{} commands `@xnthm' etc.}
% \changes{v1.1b}{87/07/31}{`theoremfont' used to specify font.}
% \changes{v1.2a}{88/01/08}{`@endtheorem' now globally defined.}
% \changes{v1.2b}{88/01/11}{`newtheorem' only allowed in the preamble.}
% \changes{v1.3a}{88/02/14}{`theoremfont' now a toks register.}
%
% \changes{v2.0a}{88/11/16}{New implementation and new documentation.}
% \changes{v2.0a}{88/11/16}{No layout changes but internal modifications
%                           old substyles won't work.}
% \changes{v2.0b}{89/01/30}{`global' used for all definitions.}
% \changes{v2.0c}{89/06/15}{Documentation translated to English.
%                            Thanks to LLL and Nina}
% \changes{v2.1a}{90/02/15}{Support for new font selection.}
% \changes{v2.2a}{94/02/03}{Cleanup for LaTeX2e}
%
% \DoNotIndex{\@endparenv,\@ifdefinable,\@ifnextchar,\@ifundefined}
% \DoNotIndex{\@namedef,\@nameuse,\@spaces,\@tempa,\@warning}
% \DoNotIndex{\ }
% \DoNotIndex{\begingroup,\bfseries}
% \DoNotIndex{\csname}
% \DoNotIndex{\def,\do,\docdate}
% \DoNotIndex{\endcsname,\endgroup,\endinput,\endtrivlist,\expandafter}
% \DoNotIndex{\filedate,\fileversion}
% \DoNotIndex{\gdef,\global}
% \DoNotIndex{\hbox,\hskip}
% \DoNotIndex{\ignorespaces,\input,\itshape,\item}
% \DoNotIndex{\labelsep,\let,\llap}
% \DoNotIndex{\makeatletter}
% \DoNotIndex{\noexpand}
% \DoNotIndex{\refstepcounter,\relax,\rlap}
% \DoNotIndex{\setlength,\slshape,\space,\string,\strut}
% \DoNotIndex{\the,\trivlist,\typeout}
% \DoNotIndex{\vbox}
% \DoNotIndex{\xdef}
%
% \def\deflabel#1{\ttfamily #1\hfill}
% \def\deflist#1{\begin{list}{}{\settowidth\labelwidth{\ttfamily #1}%
%                               \setlength\leftmargin\labelwidth
%                               \addtolength\leftmargin\labelsep
%                               \let\makelabel\deflabel}}
% \def\enddeflist{\end{list}}
% \def\envfont{\normalfont\ttfamily}
%
% \MakeShortVerb{\|}
%
% \newcommand{\TUB}{{\sl TUGboat\/}}
%
% \GetFileInfo{theorem.dtx}
% \title{An Extension of the
%        \LaTeX{} theorem environment\thanks{This file
%        has version number \fileversion, last
%        revised \filedate.}}
% \author{Frank Mittelbach}
%
% \MaintainedByLaTeXTeam{tools}
% \maketitle
%
% \begin{abstract}
%    The macros described in this paper yield an extension of the
%    \LaTeX{} theorem mechanism. It is designed to satisfy the
%    different requirements of various journals. Thus, the
%    layout of the ``theorems'' can be manipulated by determining a
%    ``style''. This article describes not only the use, but
%    also the definition, of the necessary macros.
% \end{abstract}
%
% \section*{Preface to version 2.2}
%
% For \LaTeXe{} this package did not need any fundamental changes. I
% only modified the messages generated so that theorem layout styles
% will show up with the |\listfiles| command and cleaned the section
% on the New Font Selection Scheme since this is now included in
% \LaTeX.
%
% \section*{Preface to version 2.1}
%
% This version is identical to 2.0g described in \TUB\ 10\#3 except
% for some internal defaults which are now set depending on the used
% font selection scheme.
%
% This was done to avoid unpleasant surprises if the new font selection
% scheme is in force. For further details see
% section~\ref{sec:newfonts} and \cite{bk:GMS94}.
%
% \section{Introduction}
%
% For our purposes here, ``theorems'' are labelled enunciations,
% often set off from the main text by extra space and a font change.
% Theorems, corollaries, conjectures, definitions, and remarks
% are all instances of ``theorems''.  The ``header'' of these structures
% is composed of a label (such as \textsc{Theorem} or \textsc{Remark})
% and a number which serializes an item in the sequence of items
% with the same label.
%
% Shortly after the introduction of \LaTeX{} at the Fachbereich
% Mathematik in Mainz, the desire to manipulate the layout of
% ``theorems'' arose.  In Mainz, the following two conventions came
% into general use:
% \begin{enumerate}
%   \item The number of the theorem is shown in the margin.
%   \item There is a line break at
%         the end of the theorem header.
% \end{enumerate}
%
% Additionally, some journals require different formats which depend on
% the ``sort of theorem'': e.g.\ often remarks and definitions are set
% in |\upshape|, while |\itshape| is employed for main theorems.
%
% Confronted with these requirements, a theorem environment was
% developed in Mainz which allows separate determination of the
% layout of the ``theorems sets'', comparable to |\pagestyle|.
%
%
% \section{The user interface}
% \subsection{Defining new theorem sets}
%
% \DescribeMacro\newtheorem
% As in the original \LaTeX{} version, the command |\newtheorem|
% defines a new ``theorem set'' or ``theorem-like structure''.
% Two required arguments name the new environment and give the text
% to be typeset with each instance of the new ``set'', while
% an optional argument determines how
% the ``set'' is enumerated:
% \begin{description}
%    \item[\ttfamily \bslash newtheorem\{foo\}\{bar\}]
%       The theorem set {\envfont foo} (whose name is \texttt{bar})
%       uses its own counter.
%    \item[\ttfamily \bslash newtheorem\{foo2\}{[foo]}\{bar2\}]
%       The theorem set {\envfont foo2} (printed name \texttt{bar2})
%       uses the same counter as the theorem set \texttt{foo}.
%    \item[\ttfamily \bslash newtheorem\{foo3\}\{bar3\}{[section]}]
%       The theorem set {\envfont foo3} (printed name \texttt{bar3}) is
%       enumerated within the counter \texttt{section}, i.e.\ with every
%       new |\section| the enumeration begins again with 1, and
%       the enumeration is composed from the section-number and the
%       theorem counter itself.
% \end{description}
%
% \DescribeMacro\theoremstyle
% Additionally, the command |\theoremstyle| can define the layout
% of various, or all, theorem sets.  It should be noted that any
% theorem set defined by |\newtheorem| is typeset in the
% |\theoremstyle| that is current at the time of the definition.
% Thus, the following
% \begin{verbatim}
%    \theoremstyle{break}        \newtheorem{Cor}{Corollary}
%    \theoremstyle{plain}        \newtheorem{Exa}{Example}[section]
%\end{verbatim}                             ^^A No blank at line start
% leads to the result that the set {\envfont Cor} is formatted in the
% style \texttt{break}, while the set {\envfont Exa} and all the
% following ones are formatted in the style \texttt{plain}, unless
% another |\theoremstyle| follows.  Since the definitions installed by
% |\newtheorem| are global, one also can limit |\theoremstyle| locally
% by grouping braces.
%
% \DescribeMacro\theorembodyfont
% The choice of the font for the theorem body is completely
% independent of the chosen |\theoremstyle|; this has proven to
% be very advantageous.  For example,
% \begin{verbatim}
%   {\theorembodyfont{\upshape}          \newtheorem{Rem}{Remark}}
%\end{verbatim}
% defines a theorem set {\envfont Rem}, which will be set in
% |\upshape| in the current layout (which in our example is
% \texttt{plain}).  As with |\theoremstyle|, the |\theorembodyfont|
% chosen is that current at the time of |\newtheorem|.  If
% |\theorembodyfont| is not specified or one defines
% |\theorembodyfont{}|, then the font used will be that defined by the
% |\theoremstyle|.
%
% \DescribeMacro\theoremheaderfont It is also possible to customize
% the font used for the theorem headers.  This is, however, a global
% declaration, and therefore there should be at most one
% |\theoremheaderfont| declaration in the preamble.\footnote{If it is
% actually necessary to have different header fonts, one has to define
% new theorem styles (substituting the desired font) or specify the
% information directly in the \texttt{\bslash newtheorem} declaration
% (the unclean variant).}
%
% \DescribeMacro\theorempreskipamount
% \DescribeMacro\theorempostskipamount Two additional parameters
% affect the vertical space around the theorem environments:
% |\theorempreskipamount| and |\theorempostskipamount| define,
% respectively, the spacing before and after such an environment.
% These parameters apply for all theorem sets and can be manipulated
% with the ordinary length macros.  They are rubber lengths
% (`\textsf{skips}'), and therefore can contain \texttt{plus} and
% \texttt{minus} parts.
%
% Since the definition of theorem sets should---most sensibly---be
% placed in the preamble, we only allow installation there.
% It is therefore possible to release the memory used here after
% |\begin{document}|, in order to make room for other
% applications.
%
%
% \subsection{Existing theorem styles}
%
% The following theorem styles exist to date:
% \begin{deflist}{marginbreak}
%    \item[plain]
%       This theorem style emulates the original \LaTeX{} definition,
%       except that additionally the parameters
%       |\theorem...skipamount| are used.
%    \item[break]
%       In this style, the theorem header is followed by a line
%       break.
%    \item[marginbreak]
%       The theorem number is set in the margin, and there is a line
%       break as in \texttt{break}.
%    \item[changebreak]
%       Like \texttt{break}, but with header number and text
%       interchanged.
%    \item[change]
%       Header number and text are interchanged, without a line break.
%    \item[margin]
%       The number is set in the left margin, without a line break.
% \end{deflist}
% All styles (except \texttt{plain}) select |\slshape| as the default
% |\theorembodyfont|.
%
%
% \subsection{Examples}
%
% Given the above theorem sets {\envfont Cor}, {\envfont Exa}
% and {\envfont Rem},
% suppose that the preamble also contains the declarations:
% \begin{verbatim}
%    \theoremstyle{marginbreak}   \newtheorem{Lem}[Cor]{Lemma}
%    \theoremstyle{change}
%    \theorembodyfont{\itshape}        \newtheorem{Def}[Cor]{Definition}
%
%    \theoremheaderfont{\scshape}
%\end{verbatim}
% Then the following are some typical examples of the typeset output
% resulting from their use.
%
% \begin{Cor}
%    This is a sentence typeset in the theorem environment {\envfont
%    Cor}.
% \end{Cor}
%
% \begin{Exa}
%    This is a sentence typeset in the theorem environment {\envfont
%    Exa}.
% \end{Exa}
%
% \begin{Rem}
%    This is a sentence typeset in the theorem environment {\envfont
%    Rem}.
% \end{Rem}
%
% \begin{Lem}[Ben User]
%    This is a sentence typeset in the theorem environment {\envfont
%    Lem}.
% \end{Lem}
%
% \begin{Def}[Very impressive Definition]
%    This is a sentence typeset in the theorem environment {\envfont
%    Def}.
% \end{Def}
%
% The last two examples show the effect of the optional argument to
% a theorem environment (it is the text typeset in parentheses).
%
%
% \section{Special Considerations}
% \label{sec:newfonts}
%
% Theoremheader and body are implemented as a unit. This means that
% the |\theoremheaderfont|  will inherit characteristics of the
% |\theorembodyfont| in \LaTeXe. Thus, if for example
% |\theorembodyfont| is |\itshape| and |\theoremheaderfont|
% is |\bfseries| the font selected for the header will have the
% characteristics `bold extended italic'. If this is not desired one
% should set the |\theoremheaderfont| to something like
% \begin{verbatim}
%      \theoremheaderfont{\normalfont\bfseries}
%\end{verbatim}
% i.e.\ supplying all necessary font information explicitly.
%
%
%
% \section{Acknowledgements}
%
% The publication of this set of macros was only possible with the
% help of Christina Busse (translating the manuscript into English),
% Joachim Pense (playing the r\^ole of typist), Chris Rowley
% (looking everything over) and many others providing useful
% suggestions.
%
% \MaybeStop{%
%    \begin{thebibliography}{9}
%       \bibitem{bk:GMS94} \textsc{M.~Goossens}, \textsc{F.~Mittelbach}
%         and \textsc{A.~Samarin}.
%         \newblock The \LaTeX{} Companion.
%         \newblock
%         Addison-Wesley, Reading, Massachusetts, 1994.
%       \bibitem{latex:source} \textsc{Lamport, Leslie.} \newblock
%                               \texttt{latex.tex}, version 2.09, date
%                               Feb. 1990.
%    \end{thebibliography}
%    \PrintIndex
%    \PrintChanges}
%
%
% \section{The documentation driver file}
%
% The next bit of code contains the documentation driver file for
% \TeX{}, i.e., the file that will produce the documentation you are
% currently reading. It will be extracted from this file by the
% \texttt{docstrip} program. Since it is the first code in the file
% one can alternatively process this file directly with \LaTeXe{} to
% obtain the documentation.
%
%    \begin{macrocode}
%<*driver>
\documentclass{ltxdoc}

\usepackage{theorem}

%<+driver>% The next few lines define theorem sets which are used
%<+driver>% in the example section of the documentation.

\theoremstyle{break}       \newtheorem{Cor}{Corollary}
\theoremstyle{plain}       \newtheorem{Exa}{Example}[section]
{\theorembodyfont{\upshape}\newtheorem{Rem}{Remark}}
\theoremstyle{marginbreak} \newtheorem{Lem}[Cor]{Lemma}
\theoremstyle{change}
\theorembodyfont{\itshape} \newtheorem{Def}[Cor]{Definition}

\theoremheaderfont{\scshape}

\RecordChanges

\begin{document}
   \DocInput{theorem.dtx}
\end{document}
%</driver>
%    \end{macrocode}
%
% \section{Definition of the Macros}
%
% If the file has been loaded before, we abort immediately. If not the
% package announces itself (this is actually done at the very top of
% the file---the way it is done isn't good style so don't copy it).
% \changes{v2.0e}{89/07/19}{Spaces removed from `typeout'.}
% \changes{v2.2b}{95/11/19}{Announce moved to top of file}
%    \begin{macrocode}
%<*package>
%\@ifundefined{theorem@style}{}{\endinput}
%\def\FMithmInfo{1995/11/19 v2.2b Theorem extension package (FMi)}
%\ProvidesPackage{theorem}[\FMithmInfo]
%    \end{macrocode}
%
%
% \subsection{Definition of theorem styles and fonts}
%
% All the definitions in this file are done globally to allow
% inputting this file inside a group.
%
% \begin{macro}{\theoremstyle}
%    Before a theorem style can be installed, the chosen style must be
%    known. For that reason, we must test to see that
%    |\th@|\meta{style} is known or, more precisely, that it is
%    different from |\relax|. If the style is not known then
%    |\th@plain| is used.
%    \begin{macrocode}
\gdef\theoremstyle#1{%
   \@ifundefined{th@#1}{\@warning
          {Unknown theoremstyle `#1'. Using `plain'}%
          \theorem@style{plain}}%
%    \end{macrocode}
%    We save the theorem style to be used in the token register
%    |\theorem@style|.
%    \begin{macrocode}
      {\theorem@style{#1}}%
%    \end{macrocode}
%    Now we ``evaluate'' the theorem style: this means, we call the
%    macro |\th@|\meta{style} which will activate the relevant
%    definitions which are contained in a separate file.  This is done
%    in a group to suppress changes to the current font.  This could
%    otherwise pose problems together with the new font selection
%    scheme\footnote{When I printed the original article using the new
%    font selection scheme I ended with a document with slanted
%    typefaces (text headings and all) simply because one of the
%    theorem styles used \texttt{\bslash sl} at toplevel.} if the
%    |\th@|\meta{style} is evaluated a second time.
% \changes{v2.1a}{90/02/15}{Group added around `th@..}
% \changes{v2.1a}{90/02/15}{Reseting of `..@begintheorem removed.}
%    \begin{macrocode}
      \begingroup
        \csname th@\the\theorem@style \endcsname
      \endgroup}
%    \end{macrocode}
% \end{macro}
%
%
% \begin{macro}{\@begintheorem}
% \changes{v2.1a}{90/02/15}{Relaxed at toplevel.}
% \begin{macro}{\@opargbegintheorem}
% \changes{v2.1a}{90/02/15}{Relaxed at toplevel.}
%    We reset |\@begintheorem| and |\@opargbegintheorem| to
%    |\relax| since these commands are no longer necessary at
%    toplevel. This will save a few tokens.
%    \begin{macrocode}
\global\let\@begintheorem\relax
\global\let\@opargbegintheorem\relax
%    \end{macrocode}
% \end{macro}
% \end{macro}
%
% \begin{macro}{\theorem@style}
%    Obviously the \textsf{token} register used above has to be
%    allocated. To assure the utmost compatibility with the
%    original \LaTeX{} definition, we set the default theorem style to
%    \texttt{plain}, which implements the usual \LaTeX{} convention.
% \changes{v2.0f}{89/07/20}{Unnecessary `global' removed.}
%    \begin{macrocode}
\newtoks\theorem@style
\global\theorem@style{plain}
%    \end{macrocode}
% \end{macro}
%
% \begin{macro}{\theorembodyfont}
% \begin{macro}{\theorem@bodyfont}
%    If the |\theorembodyfont| is set by the user then it should not
%    interact with the default font set in the theorem style. When the
%    new font selection is in force this may happen if, for example,
%    the default is |\itshape| and the new |\theorembodyfont| is
%    |\sffamily|. So we add a |\reset@font| command in front of the
%    user definition.
% \changes{v2.0f}{89/07/20}{Unnecessary `global' removed.}
% \changes{v2.0g}{89/09/19}{`theoremfont to `theorembodyfont.}
% \changes{v2.1c}{92/06/22}{`reset@font added and logic changed.}
% \changes{v2.1d}{92/10/28}{Reset to empty if arg is empty}
%    \begin{macrocode}
\gdef\theorembodyfont#1{%
%    \end{macrocode}
%    We check if the argument supplied is empty and if so put nothing
%    into the |\theorem@bodyfont| token register to allow for
%    |\theorembodyfont{}| as a mean of using the default of the
%    current |\theoremstyle|.
%    \begin{macrocode}
   \def\@tempa{#1}%
   \ifx\@tempa\@empty
    \theorem@bodyfont{}%
   \else
    \theorem@bodyfont{\reset@font#1}%
   \fi
   }
\newtoks\theorem@bodyfont
\global\theorem@bodyfont{}
%    \end{macrocode}
% \end{macro}
% \end{macro}
%
%
% \begin{macro}{\theoremheaderfont}
% \changes{v2.0g}{89/09/19}{Macro added.}
%    The font for the theorem headers is handled differently because
%    this definition applies to all theorem styles.
%    \begin{macrocode}
\gdef\theoremheaderfont#1{\gdef\theorem@headerfont{#1}%
%    \end{macrocode}
%    After using the macro once it is redefined to produce an error
%    message.
%    \begin{macrocode}
       \gdef\theoremheaderfont##1{%
        \typeout{\string\theoremheaderfont\space should be used
                 only once.}}}
%    \end{macrocode}
% \end{macro}
%
% \begin{macro}{\theorem@headerfont}
% \changes{v2.0g}{89/09/19}{Macro added.}
%    To set the |\theorem@headerfont| default we first test if
%    the new fontselection scheme is in force.
% \changes{v2.1a}{90/02/15}{Support for new font selection.}
%    \begin{macrocode}
\ifx\upshape\undefined
%    \end{macrocode}
%    If not we define it to expand into |\bfseries|. We don't use
%    |\let| just in case a following style option redefines this
%    macro.
%    \begin{macrocode}
\gdef\theorem@headerfont{\bfseries}
%    \end{macrocode}
%    Otherwise we reset the current shape before calling |\bfseries|.
%    \begin{macrocode}
\else \gdef\theorem@headerfont{\normalfont\bfseries}\fi
%    \end{macrocode}
% \end{macro}
%
% \begin{macro}{\th@plain}
% \begin{macro}{\th@break}
% \begin{macro}{\th@marginbreak}
% \begin{macro}{\th@changebreak}
% \begin{macro}{\th@change}
% \begin{macro}{\th@margin}
%    The different styles are defined in macros such as
%    |\th@plain|.  Since memory space is precious in
%    ``non-Big-versions'', we have to avoid offering too many unused
%    definitions.  Therefore we define these styles in separate files
%    that can be loaded on demand.  Thus the commands themselves only
%    load these files. We use |\@input@|, a \LaTeXe{} internal command
%    that ensures that the file will be listed with |\listfiles|.
% \changes{v2.2a}{94/02/02}{Use `@input@ to load theorem layout files}
%    \begin{macrocode}
\gdef\th@plain{\@input@{thp.sty}}
\gdef\th@break{\@input@{thb.sty}}
\gdef\th@marginbreak{\@input@{thmb.sty}}
\gdef\th@changebreak{\@input@{thcb.sty}}
\gdef\th@change{\@input@{thc.sty}}
\gdef\th@margin{\@input@{thm.sty}}
%    \end{macrocode}
%    This list will be expanded when new styles become
%    available.  For testing,  just append new theorem substyles
%    as document options.
% \end{macro}
% \end{macro}
% \end{macro}
% \end{macro}
% \end{macro}
% \end{macro}
%
%
%
% \subsection{Definition of a new theorem set}
%
% As already pointed out, a new theorem environment can be defined in
% three different ways:
% \begin{verbatim}
%    \newtheorem{Lem}{Lemma}
%    \newtheorem{Lem}{Lemma}[section]
%    \newtheorem{Lem}[Theorem]{Lemma}
%\end{verbatim}
% The function of the macro |\newtheorem| is to recognize these cases
% and then to branch into one of the three macros |\@ynthm|, |\@xnthm|
% or |\@othm|.  This mechanism is adopted unchanged from
% \cite{latex:source}; the essential point here is that, for example,
% in the second case, the arguments \texttt{Lem}, \texttt{Lemma} and
% \texttt{section} are passed over to the macro |\@xnthm|.
%
% We inspect this case first because the others present fewer problems,
% and thus are easily derived from this one.
%
% \begin{macro}{\@xnthm}
%    For our example arguments, the macro |\@xnthm| must fulfill
%    the following:
%    \begin{itemize}
%       \item
%          Define a new \LaTeX-counter `Lem'
%       \item
%          reset this counter within a |\section|
%       \item
%          define the macro |\theLem|
%       \item
%          define the environment macros |\Lem| and
%          |\endLem| using the current
%          |\theoremstyle| and |\theorem@bodyfont|.
%    \end{itemize}
%    Obviously, all this should happen only if the first argument
%    of |\@xnthm| (i.e.\ \texttt{Lem} in our example) is chosen so
%    as not to conflict with any previously defined commands or
%    environments.  This test is performed by the \LaTeX{} macro
%    |\@ifdefinable|.
%    \begin{macrocode}
\gdef\@xnthm#1#2[#3]{\expandafter\@ifdefinable\csname #1\endcsname
%    \end{macrocode}
%    Therefore, the first argument of |\@ifdefinable| is the
%    expansion (in the example, |\Lem|) of
%    |\csname||#1\endcsname|.  The second argument is
%    executed only if the test has been completed successfully.
%    \begin{macrocode}
   {%
%    \end{macrocode}
%    Now we define the new counter. The names of the \LaTeX{}
%    macros employed should speak for themselves:
% \changes{v2.2b}{1995/10/07}{Use \cs{@newctr}}
% \changes{v2.2c}{1995/11/23}{Use \cs{@newctr} properly.}
%    \begin{macrocode}
    \@definecounter{#1}\@newctr{#1}[#3]%
%    \end{macrocode}
%    Using |\@newctr| will give a proper error message if the counter
%    in |#3| is not defined.
%    In defining `|\theLem|' we must generate the desired macro
%    name by use of |\expandafter| and |\csname|.
%    \begin{macrocode}
    \expandafter\xdef\csname the#1\endcsname
%    \end{macrocode}
%    An |\xdef| is used in order to make the definition global,
%    and to ensure that it contains the replacement texts of
%    |\@thmcountersep| and |\@thmcounter|.\footnote{These
%    two macros can be defined by the document style.  Their default
%    values produce a `\texttt{.}' as separation and an arabic
%    representation of the number.} However, not everything should be
%    expanded. For example, it saves space to use |\thesection|
%    instead of its---at times---lengthy expansion.
%    \begin{macrocode}
      {\expandafter \noexpand \csname the#3\endcsname
       \@thmcountersep \@thmcounter{#1}}%
%    \end{macrocode}
%    Thus with the defaults of \LaTeX{}, |\theLem| would be
%    replaced by the command sequence |\thesection.\arabic{Lem}|.
%
%    We will now look at the definition of the macro which is executed
%    at the beginning of the actual environment (in our example this
%    macro is |\Lem|). It should be noted that we use an
%    ``|\expandafter| trick'' to expand only certain parts of the
%    replacement text at the time of the definition.
%    \begin{macrocode}
    \def\@tempa{\global\@namedef{#1}}%
    \expandafter \@tempa \expandafter{%
%    \end{macrocode}
%    First, the macro that contains the current definitions of
%    |\@begintheorem| and |\@opargtheorem| should be called
%    up.  The name of this macro---as is already known---has the form
%    |\th@|\meta{theorem\ style}; therefore, it must be called by
%    \begin{macrocode}
      \csname th@\the \theorem@style
%    \end{macrocode}
%    In addition the default theorem font should be changeable, i.e.\
%    we have to insert the contents of |\theorem@bodyfont|. For
%    that reason, we expand even further, beyond |\endcsname|,
%    and thus insert the contents of the \textsf{token} register
%    |\theorem@bodyfont| in the replacement text.
%    \begin{macrocode}
            \expandafter \endcsname \the \theorem@bodyfont
%    \end{macrocode}
%    Now it is time to call the macro |\@thm| which takes over
%    the further processing.  It has two arguments: the current
%    counter name (in our example, \texttt{Lem}), and the text of the
%    label (in our example, \texttt{Lemma}).
%    \begin{macrocode}
     \@thm{#1}{#2}}%
%    \end{macrocode}
%    With this, the `sub-definition' is complete.  The macro
%    |\@endtheorem| ends a theorem environment and is, so far, nothing
%    but an |\endtrivlist|.  (Hence it is defined globally, and not
%    within the theorem styles.\footnote{This has to be changed as
%    soon as theorem styles that change \texttt{\bslash @endtheorem}
%    exist.  In such a case, all existing styles must be changed as
%    well since they will have to reset the macro.}) Therefore, we can
%    set it equivalent to the macro that ends the theorem set (in our
%    example, |\endLem|).  However, if some day theorem styles exist
%    that do change |\@endtheorem|, we would have to use the
%    commented-out line instead.
%    \begin{macrocode}
    \global \expandafter \let \csname end#1\endcsname \@endtheorem
%   \global\@namedef{end#1}{\@endtheorem}%
%    \end{macrocode}
%    With these commands all the required definitions are employed,
%    unless the test |\@ifdefinable| has failed.  Therefore, we
%    end the second argument of this macro and with it the definition
%    of |\@xnthm|.
%    \begin{macrocode}
   }}
%    \end{macrocode}
% \end{macro}
%
% \begin{macro}{\@ynthm}
%    The definition of |\@ynthm| is completely analogous. In
%    this case the new counter that is defined is not reset within
%    another counter; thus the definition of |\the...|
%    is simplified:
%    \begin{macrocode}
\gdef\@ynthm#1#2{\expandafter\@ifdefinable\csname #1\endcsname
   {\@definecounter{#1}%
    \expandafter\xdef\csname the#1\endcsname{\@thmcounter{#1}}%
%    \end{macrocode}
%    The rest of the definition corresponds literally to that of
%    |\@xnthm|:
%    \begin{macrocode}
    \def\@tempa{\global\@namedef{#1}}\expandafter \@tempa
     \expandafter{\csname th@\the \theorem@style \expandafter
     \endcsname \the\theorem@bodyfont \@thm{#1}{#2}}%
    \global \expandafter \let \csname end#1\endcsname \@endtheorem}}
%    \end{macrocode}
% \end{macro}
%
% \begin{macro}{\@othm}
%    The definition of |\@othm| does not contain anything new.
%    \begin{macrocode}
\gdef\@othm#1[#2]#3{%
%    \end{macrocode}
% \changes{v2.2b}{95/10/07}{Build in check for counter defined}
%    We do not define a new counter but instead use one that has
%    already been defined.  Thus the only definition we need is that
%    of this pseudo-counter (i.e.\ |\the|\meta{env.\ name}).
%    First we check if |#2| corresponds to a known counter name.
%    \begin{macrocode}
  \expandafter\ifx\csname c@#2\endcsname\relax
   \@nocounterr{#2}%
  \else
   \expandafter\@ifdefinable\csname #1\endcsname
   {\expandafter \xdef \csname the#1\endcsname
     {\expandafter \noexpand \csname the#2\endcsname}%
%    \end{macrocode}
%    All other parts of the definition can be adopted from
%    |\@xnthm|. We have to remember, though, that in this case
%    the name of the current counter and the theorem label have moved
%    to the second and third arguments.
%    \begin{macrocode}
    \def\@tempa{\global\@namedef{#1}}\expandafter \@tempa
     \expandafter{\csname th@\the \theorem@style \expandafter
     \endcsname \the\theorem@bodyfont \@thm{#2}{#3}}%
    \global \expandafter \let \csname end#1\endcsname \@endtheorem}%
  \fi}
%    \end{macrocode}
% \end{macro}
%
% \subsection{Macros that are employed in a {\envfont theorem}
%    environment}
%
% \begin{macro}{\@thm}
%    The macro |\@thm| has to increase the current counter.  Then,
%    depending on whether the environment has (or does not have) an
%    optional argument, it has to branch into either |\@begintheorem|
%    or |\@opargtheorem|.
%    \begin{macrocode}
\gdef\@thm#1#2{\refstepcounter{#1}%
%    \end{macrocode}
%    Now we start a {\envfont trivlist} environment, and give
%    |\@topsep| and |\@topsepadd| the values of the \textsf{skip}
%    registers |\theorempreskipamount| and |\theorempostskipamount|.
%    The value in |\@topsep| is the vertical space that is inserted by
%    the first (and only) |\item| in our |\trivlist| whilst
%    |\@topsepadd| is inserted by |\@endparenv| at the end of that
%    {\envfont trivlist} environment.  By using these registers, we
%    obtain the desired space around a {\envfont theorem} environment.
%    \begin{macrocode}
   \trivlist
   \@topsep \theorempreskipamount               % used by first \item
   \@topsepadd \theorempostskipamount           % used by \@endparenv
%    \end{macrocode}
%    Now we have to test whether an optional argument has been given.
% \changes{v2.0d}{89/07/18}{Forgotten percent added.}
%    \begin{macrocode}
   \@ifnextchar [%
%    \end{macrocode}
%    If there is an optional argument, we will call |\@ythm|, and
%    move the arguments read back into the input stream.
%    \begin{macrocode}
   {\@ythm{#1}{#2}}%
%    \end{macrocode}
%    If not, we call |\@begintheorem|. Its first argument is the
%    name of the theorem set (hence the second argument of
%    |\@thm|). Its second argument is the macro that produces the
%    current number.
%    \begin{macrocode}
   {\@begintheorem{#2}{\csname the#1\endcsname}\ignorespaces}}
%    \end{macrocode}
% \end{macro}
%
% \begin{macro}{\@xthm}
% \begin{macro}{\@ythm}
%    Both these macros were originally called by |\@thm|.  We do
%    not need |\@xthm| anymore, hence we reset it to
%    |\relax|. The definition of |\@ythm| has not changed at
%    all from its definition in \LaTeX. In order to make the macros
%    easier to understand, we will
%    nevertheless present it (commented out).
%    \begin{macrocode}
\global\let\@xthm\relax
% \def\@ythm#1#2[#3]{\@opargbegintheorem{#2}{\csname
%           the#1\endcsname}{#3}\ignorespaces}
%</package>
%    \end{macrocode}
%    The primitive |\ignorespaces| in |\@ythm| and
%    |\@thm| is needed to remove the spaces between the
%    |\begin{...}| and the actual text.
% \end{macro}
% \end{macro}
%
%
% \subsection{Definition of the theorem substyles}
%
% As already pointed out, the theorem substyles, defined below, are
% only loaded when necessary.  Note that all these substyles, except
% \texttt{plain}, have |\slshape| as the default body font.
%
% \subsubsection{The \texttt{plain} style}
%
% \label{dok:th@plain} As the following macros use |@|, we have
% to locally set the |\catcode| of this symbol to ``letter''.
% This happens within a group, so that we do not have to worry about
% which |\catcode| that symbol had before.
%    \begin{macrocode}
%<*thp>
\begingroup \makeatletter
%    \end{macrocode}
% Since we are now within a group, we must make all definitions
% globally.  First we make sure that \texttt{theorem.sty} is loaded.
% This will allow us to use this file as a document style option
% without having to call \texttt{theorem} itself as an option.  At the
% same time, we assure that at least version 2 is loaded, since
% |\theorem@style| was not defined in earlier versions.
%    \begin{macrocode}
\@ifundefined{theorem@style}{\input{theorem.sty}}{}
\ProvidesFile{thp.sty}
             [\FMithmInfo]
%    \end{macrocode}
%
%
% \begin{macro}{\th@plain}
% \begin{macro}{\@begintheorem}
% \begin{macro}{\@opargbegintheorem}
%    |\theoremstyle{plain}| corresponds to the original
%    definition, except that the distances to the surrounding text are
%    determined by the parameters |\theorempreskipamount| and
%    |\theorempostskipamount|.
%    First we set the default body font.
%    \begin{macrocode}
\gdef\th@plain{\normalfont\itshape
%    \end{macrocode}
%    Then we define |\@begintheorem| and
%    |\@opargbegintheorem|. These two macros define how the
%    header of a theorem is typeset.  |\@opargbegintheorem| will
%    be called if a \textsf{theorem} environment with an optional
%    argument is encountered; otherwise, the header is constructed by
%    calling |\@begintheorem|.  If one of these macros is
%    executed, we are within a \textsf{trivlist} environment started by
%    |\@thm|.  So the theorem header is produced with an
%    |\item| command.
%
%    Instead of specifying the header font directly, all standard
%    theorem styles use the |\theorem@headerfont| macro to allow
%    customization.  The extra space (|\labelsep|) is necessary
%    because of problems in the \textsf{trivlist} environment.
%    \begin{macrocode}
  \def\@begintheorem##1##2{%
        \item[\hskip\labelsep \theorem@headerfont ##1\ ##2]}%
%    \end{macrocode}
%    The definition of |\@opargbegintheorem| is completely
%    analogous. The only difference is the fact that there exists a
%    third argument (which is the optional parameter of the
%    environment and contains additional information about the
%    theorem).  Customarily we enclose it in parentheses.
%    \begin{macrocode}
\def\@opargbegintheorem##1##2##3{%
   \item[\hskip\labelsep \theorem@headerfont ##1\ ##2\ (##3)]}}
%    \end{macrocode}
% \end{macro}
% \end{macro}
% \end{macro}
%
% \noindent
% We conclude with an |\endgroup| to restore the
% |\catcode| of |@|.
%    \begin{macrocode}
\endgroup
%</thp>
%    \end{macrocode}
%
%
% \subsubsection{The \texttt{break} style}
%
% This style option is stored in the file |thb.sty|.
% For the next two lines see the documentation for |\th@plain| on
% page~\pageref{dok:th@plain}.
%    \begin{macrocode}
%<*thb>
\begingroup \makeatletter
\@ifundefined{theorem@style}{\input{theorem.sty}}{}
\ProvidesFile{thb.sty}
             [\FMithmInfo]
%    \end{macrocode}
%
% \begin{macro}{\th@break}
%    |\theoremstyle{break}| produces a line break after the name
%    of the theorem.  The font is |\slshape|.  Hence, we define
%    |\th@break| as follows:
%    \begin{macrocode}
\gdef\th@break{\normalfont\slshape
  \def\@begintheorem##1##2{\item[%
%    \end{macrocode}
%    We run into the following problem: it is not possible to create
%    the header with |\item[|\meta{title}|]| and then start
%    a new line by, for example, |\mbox{}\\|. Such a definition
%    will fail whenever a list environment follows immediately.  With
%    the above construction, the |\mbox{}| causes the switch
%    |@inlabel| (cf.\ definition of |\list| and
%    |\trivlist| in \cite{latex:source}) to be set to \texttt{false}
%    and so the following list will insert additional vertical
%    space (|\topskip|). This is quite annoying. Therefore, we
%    create the line break within the |\item|. In order to ensure
%    that the text will begin at the proper position in the following
%    line, we simply pretend that the label does not take any
%    room.\footnote{This will lead to problems whenever very high
%    symbols occurring in the line tower into the
%    heading. So, something else has to be done here sometime.}
%    \begin{macrocode}
       \rlap{\vbox{\hbox{\hskip \labelsep\theorem@headerfont ##1\ ##2}%
                   \hbox{\strut}}}]}%
%    \end{macrocode}
%    Again, the definition of |\@opargbegintheorem| is
%    completely analogous.
%    \begin{macrocode}
\def\@opargbegintheorem##1##2##3{%
  \item[\rlap{\vbox{\hbox{\hskip \labelsep \theorem@headerfont
                     ##1\ ##2\ (##3)}%
                    \hbox{\strut}}}]}}
\endgroup
%</thb>
%    \end{macrocode}
% \end{macro}
%
%
% \subsubsection{The \texttt{changebreak} style}
%
%    \begin{macrocode}
% This style option is stored in the file |thcb.sty|.
%    \begin{macrocode}
%<*thcb>
\begingroup \makeatletter
\@ifundefined{theorem@style}{\input{theorem.sty}}{}
\ProvidesFile{thcb.sty}
             [\FMithmInfo]
%    \end{macrocode}
%
% \begin{macro}{\th@changebreak}
%    The \texttt{change-break} theorem style is like \texttt{break}
%    but with interchange of theorem name and theorem number.  Thus we
%    define |\th@changebreak| as follows:
%    \begin{macrocode}
\gdef\th@changebreak{\normalfont\slshape
  \def\@begintheorem##1##2{\item
       [\rlap{\vbox{\hbox{\hskip\labelsep \theorem@headerfont ##2\ ##1}%
                    \hbox{\strut}}}]}%
\def\@opargbegintheorem##1##2##3{%
  \item[\rlap{\vbox{\hbox{\hskip\labelsep \theorem@headerfont
                     ##2\ ##1\ (##3)}%
                    \hbox{\strut}}}]}}
\endgroup
%</thcb>
%    \end{macrocode}
% \end{macro}
%
%
% \subsubsection{The \texttt{change} style}
%
% This style option is stored in the file |thc.sty|.
%    \begin{macrocode}
%<*thc>
\begingroup \makeatletter
\@ifundefined{theorem@style}{\input{theorem.sty}}{}
\ProvidesFile{thc.sty}
             [\FMithmInfo]
%    \end{macrocode}
%
% \begin{macro}{\th@change}
%    The \texttt{change} theorem style corresponds to the
%    \texttt{change break} style without a linebreak after the header.
%    To say it in another way, it's the same as the \texttt{plain}
%    style but with number and name interchanged and |\slshape| as the
%    default font.
%    \begin{macrocode}
\gdef\th@change{\normalfont\slshape
  \def\@begintheorem##1##2{\item
       [\hskip\labelsep \theorem@headerfont ##2\ ##1]}%
\def\@opargbegintheorem##1##2##3{%
  \item[\hskip\labelsep \theorem@headerfont ##2\ ##1\ (##3)]}}
\endgroup
%</thc>
%    \end{macrocode}
% \end{macro}
%
%
% \subsubsection{The \texttt{marginbreak} style}
%
% This style option is the one used most often at Mainz.  It is saved
% in the file |thmb.sty|.
%    \begin{macrocode}
%<*thmb>
\begingroup \makeatletter
\@ifundefined{theorem@style}{\input{theorem.sty}}{}
\ProvidesFile{thmb.sty}
             [\FMithmInfo]
%    \end{macrocode}
%
% \begin{macro}{\th@marginbreak}
%    The \texttt{margin break} style is nearly the same as the
%    \texttt{change break} style. The only difference is the placement
%    of the theorem number.  We use |\llap| to place it in the left
%    margin.
%
%    In this style |\labelsep| denotes the separation between the
%    number and the text.
%    \begin{macrocode}
\gdef\th@marginbreak{\normalfont\slshape
  \def\@begintheorem##1##2{\item
       [\rlap{\vbox{\theorem@headerfont
                    \hbox{\llap{##2}\hskip\labelsep ##1}%
                    \hbox{\strut}}}]}%
\def\@opargbegintheorem##1##2##3{%
  \item[\rlap{\vbox{\theorem@headerfont
                    \hbox{\llap{##2}\hskip\labelsep ##1\ (##3)}%
                    \hbox{\strut}}}]}}
\endgroup
%</thmb>
%    \end{macrocode}
% \changes{v2.1c}{90/02/17}{`endgroup outside macrocode env.}
% \end{macro}
%
%
%
% \subsubsection{The \texttt{margin} style}
%
% This style option is stored in the file |thm.sty|.
%    \begin{macrocode}
%<*thm>
\begingroup \makeatletter
\@ifundefined{theorem@style}{\input{theorem.sty}}{}
\ProvidesFile{thm.sty}
             [\FMithmInfo]
%    \end{macrocode}
%
% \begin{macro}{\th@margin}
%    Again this is only a variant of the theorem styles described
%    above without any new ideas.
%    \begin{macrocode}
\gdef\th@margin{\normalfont\slshape
  \def\@begintheorem##1##2{\item
       [\theorem@headerfont \llap{##2}\hskip\labelsep ##1]}%
\def\@opargbegintheorem##1##2##3{%
  \item[\theorem@headerfont \llap{##2}\hskip\labelsep ##1\ (##3)]}}
\endgroup
%</thm>
%    \end{macrocode}
% \end{macro}
%
%
%
%
% \subsection{Final Definitions}
%
% \begin{macro}{\theorempreskipamount}
% \begin{macro}{\theorempostskipamount}
%    The \textsf{skip} parameters that regulate the vertical empty space
%    before and after the {\envfont theorem} environment have
%    to be allocated as well.
% \changes{v2.0f}{89/07/20}{Unnecessary `global' removed.}
%    \begin{macrocode}
%<*package>
\newskip\theorempreskipamount
\newskip\theorempostskipamount
%    \end{macrocode}
%    Since we have used the same values for all theorem sets, we now can
%    assign them.
%    \begin{macrocode}
\global\setlength\theorempreskipamount{12pt plus 5pt minus 3pt}
\global\setlength\theorempostskipamount{8pt plus 3pt minus 1.5pt}
%    \end{macrocode}
% \end{macro}
% \end{macro}
%
%
% \begin{macro}{\@endtheorem}
%    The same holds for the macro |\@endtheorem|, which ends a
%    {\envfont theorem} environment. Since it is the same for all
%    theorem sets, it is removed from the macros |\th@|\meta{style}.
%    It simply ends the {\envfont trivlist} environment, which was
%    begun in |\@thm|.
%    \begin{macrocode}
\global\let\@endtheorem=\endtrivlist
%    \end{macrocode}
% \end{macro}
%
%
%
% \begin{macro}{\@preamblecmds}
%    All macros defined above are to be used only in the preamble.
%    Therefore, we insert them in |\@preamblecmds| which will disable
%    them at begin document. This is done by the internal \LaTeXe{}
%    command |\@onlypreamble|.
% \changes{v2.0g}{89/09/19}{`theoremheaderfont and `theorembodyfont
%    added.}
% \changes{v2.2a}{94/02/03}{Use \LaTeXe{} interface}
%    \begin{macrocode}
\@onlypreamble\@xnthm
\@onlypreamble\@ynthm
\@onlypreamble\@othm
\@onlypreamble\newtheorem
\@onlypreamble\theoremstyle
\@onlypreamble\theorembodyfont
\@onlypreamble\theoremheaderfont
%    \end{macrocode}
% \end{macro}
%
% Finally we declare the \texttt{plain} theorem style to be the default.
% \changes{v2.1b}{90/06/04}{Fixed missing `item by declaring default
%    style.}
%    \begin{macrocode}
\theoremstyle{plain}
%</package>
%    \end{macrocode}
%
% \Finale

\endinput
}{}
\ProvidesFile{thcb.sty}
             [\FMithmInfo]
%    \end{macrocode}
%
% \begin{macro}{\th@changebreak}
%    The \texttt{change-break} theorem style is like \texttt{break}
%    but with interchange of theorem name and theorem number.  Thus we
%    define |\th@changebreak| as follows:
%    \begin{macrocode}
\gdef\th@changebreak{\normalfont\slshape
  \def\@begintheorem##1##2{\item
       [\rlap{\vbox{\hbox{\hskip\labelsep \theorem@headerfont ##2\ ##1}%
                    \hbox{\strut}}}]}%
\def\@opargbegintheorem##1##2##3{%
  \item[\rlap{\vbox{\hbox{\hskip\labelsep \theorem@headerfont
                     ##2\ ##1\ (##3)}%
                    \hbox{\strut}}}]}}
\endgroup
%</thcb>
%    \end{macrocode}
% \end{macro}
%
%
% \subsubsection{The \texttt{change} style}
%
% This style option is stored in the file |thc.sty|.
%    \begin{macrocode}
%<*thc>
\begingroup \makeatletter
\@ifundefined{theorem@style}{% \iffalse meta-comment
%
% Copyright (C) 1993-2024
%
% The LaTeX Project and any individual authors listed elsewhere
% in this file.
%
% This file is part of the Standard LaTeX `Tools Bundle'.
% -------------------------------------------------------
%
% It may be distributed and/or modified under the
% conditions of the LaTeX Project Public License, either version 1.3c
% of this license or (at your option) any later version.
% The latest version of this license is in
%    https://www.latex-project.org/lppl.txt
% and version 1.3c or later is part of all distributions of LaTeX
% version 2005/12/01 or later.
%
% The list of all files belonging to the LaTeX `Tools Bundle' is
% given in the file `manifest.txt'.
%
% \fi
%
%%
%
%
\def\FMithmInfo{2023/07/05 v2.2c Theorem extension package (FMi)}
%
% \ProvidesFile{theorem.dtx}[\FMithmInfo]
%\iffalse   % this is a METACOMMENT !
%%
%% Package `theorem' to use with LaTeX2e
%% Copyright (C) 1989-1995 Frank Mittelbach, all rights reserved.
%<+package>\@ifundefined{theorem@style}{}{\endinput}
%<+package>\NeedsTeXFormat{LaTeX2e}[1995/06/01]
%<+package>\ProvidesPackage{theorem}
%<+package>         [\FMithmInfo]
%
% \fi
%
% \changes{v1.0}{87/05/07}{First draft.}
% \changes{v1.0a}{87/05/07}{`thistheoremstyle' added.}
% \changes{v1.1a}{87/07/29}{New concept: redefine the internal
%                          \protect\LaTeX{} commands `@xnthm' etc.}
% \changes{v1.1b}{87/07/31}{`theoremfont' used to specify font.}
% \changes{v1.2a}{88/01/08}{`@endtheorem' now globally defined.}
% \changes{v1.2b}{88/01/11}{`newtheorem' only allowed in the preamble.}
% \changes{v1.3a}{88/02/14}{`theoremfont' now a toks register.}
%
% \changes{v2.0a}{88/11/16}{New implementation and new documentation.}
% \changes{v2.0a}{88/11/16}{No layout changes but internal modifications
%                           old substyles won't work.}
% \changes{v2.0b}{89/01/30}{`global' used for all definitions.}
% \changes{v2.0c}{89/06/15}{Documentation translated to English.
%                            Thanks to LLL and Nina}
% \changes{v2.1a}{90/02/15}{Support for new font selection.}
% \changes{v2.2a}{94/02/03}{Cleanup for LaTeX2e}
%
% \DoNotIndex{\@endparenv,\@ifdefinable,\@ifnextchar,\@ifundefined}
% \DoNotIndex{\@namedef,\@nameuse,\@spaces,\@tempa,\@warning}
% \DoNotIndex{\ }
% \DoNotIndex{\begingroup,\bfseries}
% \DoNotIndex{\csname}
% \DoNotIndex{\def,\do,\docdate}
% \DoNotIndex{\endcsname,\endgroup,\endinput,\endtrivlist,\expandafter}
% \DoNotIndex{\filedate,\fileversion}
% \DoNotIndex{\gdef,\global}
% \DoNotIndex{\hbox,\hskip}
% \DoNotIndex{\ignorespaces,\input,\itshape,\item}
% \DoNotIndex{\labelsep,\let,\llap}
% \DoNotIndex{\makeatletter}
% \DoNotIndex{\noexpand}
% \DoNotIndex{\refstepcounter,\relax,\rlap}
% \DoNotIndex{\setlength,\slshape,\space,\string,\strut}
% \DoNotIndex{\the,\trivlist,\typeout}
% \DoNotIndex{\vbox}
% \DoNotIndex{\xdef}
%
% \def\deflabel#1{\ttfamily #1\hfill}
% \def\deflist#1{\begin{list}{}{\settowidth\labelwidth{\ttfamily #1}%
%                               \setlength\leftmargin\labelwidth
%                               \addtolength\leftmargin\labelsep
%                               \let\makelabel\deflabel}}
% \def\enddeflist{\end{list}}
% \def\envfont{\normalfont\ttfamily}
%
% \MakeShortVerb{\|}
%
% \newcommand{\TUB}{{\sl TUGboat\/}}
%
% \GetFileInfo{theorem.dtx}
% \title{An Extension of the
%        \LaTeX{} theorem environment\thanks{This file
%        has version number \fileversion, last
%        revised \filedate.}}
% \author{Frank Mittelbach}
%
% \MaintainedByLaTeXTeam{tools}
% \maketitle
%
% \begin{abstract}
%    The macros described in this paper yield an extension of the
%    \LaTeX{} theorem mechanism. It is designed to satisfy the
%    different requirements of various journals. Thus, the
%    layout of the ``theorems'' can be manipulated by determining a
%    ``style''. This article describes not only the use, but
%    also the definition, of the necessary macros.
% \end{abstract}
%
% \section*{Preface to version 2.2}
%
% For \LaTeXe{} this package did not need any fundamental changes. I
% only modified the messages generated so that theorem layout styles
% will show up with the |\listfiles| command and cleaned the section
% on the New Font Selection Scheme since this is now included in
% \LaTeX.
%
% \section*{Preface to version 2.1}
%
% This version is identical to 2.0g described in \TUB\ 10\#3 except
% for some internal defaults which are now set depending on the used
% font selection scheme.
%
% This was done to avoid unpleasant surprises if the new font selection
% scheme is in force. For further details see
% section~\ref{sec:newfonts} and \cite{bk:GMS94}.
%
% \section{Introduction}
%
% For our purposes here, ``theorems'' are labelled enunciations,
% often set off from the main text by extra space and a font change.
% Theorems, corollaries, conjectures, definitions, and remarks
% are all instances of ``theorems''.  The ``header'' of these structures
% is composed of a label (such as \textsc{Theorem} or \textsc{Remark})
% and a number which serializes an item in the sequence of items
% with the same label.
%
% Shortly after the introduction of \LaTeX{} at the Fachbereich
% Mathematik in Mainz, the desire to manipulate the layout of
% ``theorems'' arose.  In Mainz, the following two conventions came
% into general use:
% \begin{enumerate}
%   \item The number of the theorem is shown in the margin.
%   \item There is a line break at
%         the end of the theorem header.
% \end{enumerate}
%
% Additionally, some journals require different formats which depend on
% the ``sort of theorem'': e.g.\ often remarks and definitions are set
% in |\upshape|, while |\itshape| is employed for main theorems.
%
% Confronted with these requirements, a theorem environment was
% developed in Mainz which allows separate determination of the
% layout of the ``theorems sets'', comparable to |\pagestyle|.
%
%
% \section{The user interface}
% \subsection{Defining new theorem sets}
%
% \DescribeMacro\newtheorem
% As in the original \LaTeX{} version, the command |\newtheorem|
% defines a new ``theorem set'' or ``theorem-like structure''.
% Two required arguments name the new environment and give the text
% to be typeset with each instance of the new ``set'', while
% an optional argument determines how
% the ``set'' is enumerated:
% \begin{description}
%    \item[\ttfamily \bslash newtheorem\{foo\}\{bar\}]
%       The theorem set {\envfont foo} (whose name is \texttt{bar})
%       uses its own counter.
%    \item[\ttfamily \bslash newtheorem\{foo2\}{[foo]}\{bar2\}]
%       The theorem set {\envfont foo2} (printed name \texttt{bar2})
%       uses the same counter as the theorem set \texttt{foo}.
%    \item[\ttfamily \bslash newtheorem\{foo3\}\{bar3\}{[section]}]
%       The theorem set {\envfont foo3} (printed name \texttt{bar3}) is
%       enumerated within the counter \texttt{section}, i.e.\ with every
%       new |\section| the enumeration begins again with 1, and
%       the enumeration is composed from the section-number and the
%       theorem counter itself.
% \end{description}
%
% \DescribeMacro\theoremstyle
% Additionally, the command |\theoremstyle| can define the layout
% of various, or all, theorem sets.  It should be noted that any
% theorem set defined by |\newtheorem| is typeset in the
% |\theoremstyle| that is current at the time of the definition.
% Thus, the following
% \begin{verbatim}
%    \theoremstyle{break}        \newtheorem{Cor}{Corollary}
%    \theoremstyle{plain}        \newtheorem{Exa}{Example}[section]
%\end{verbatim}                             ^^A No blank at line start
% leads to the result that the set {\envfont Cor} is formatted in the
% style \texttt{break}, while the set {\envfont Exa} and all the
% following ones are formatted in the style \texttt{plain}, unless
% another |\theoremstyle| follows.  Since the definitions installed by
% |\newtheorem| are global, one also can limit |\theoremstyle| locally
% by grouping braces.
%
% \DescribeMacro\theorembodyfont
% The choice of the font for the theorem body is completely
% independent of the chosen |\theoremstyle|; this has proven to
% be very advantageous.  For example,
% \begin{verbatim}
%   {\theorembodyfont{\upshape}          \newtheorem{Rem}{Remark}}
%\end{verbatim}
% defines a theorem set {\envfont Rem}, which will be set in
% |\upshape| in the current layout (which in our example is
% \texttt{plain}).  As with |\theoremstyle|, the |\theorembodyfont|
% chosen is that current at the time of |\newtheorem|.  If
% |\theorembodyfont| is not specified or one defines
% |\theorembodyfont{}|, then the font used will be that defined by the
% |\theoremstyle|.
%
% \DescribeMacro\theoremheaderfont It is also possible to customize
% the font used for the theorem headers.  This is, however, a global
% declaration, and therefore there should be at most one
% |\theoremheaderfont| declaration in the preamble.\footnote{If it is
% actually necessary to have different header fonts, one has to define
% new theorem styles (substituting the desired font) or specify the
% information directly in the \texttt{\bslash newtheorem} declaration
% (the unclean variant).}
%
% \DescribeMacro\theorempreskipamount
% \DescribeMacro\theorempostskipamount Two additional parameters
% affect the vertical space around the theorem environments:
% |\theorempreskipamount| and |\theorempostskipamount| define,
% respectively, the spacing before and after such an environment.
% These parameters apply for all theorem sets and can be manipulated
% with the ordinary length macros.  They are rubber lengths
% (`\textsf{skips}'), and therefore can contain \texttt{plus} and
% \texttt{minus} parts.
%
% Since the definition of theorem sets should---most sensibly---be
% placed in the preamble, we only allow installation there.
% It is therefore possible to release the memory used here after
% |\begin{document}|, in order to make room for other
% applications.
%
%
% \subsection{Existing theorem styles}
%
% The following theorem styles exist to date:
% \begin{deflist}{marginbreak}
%    \item[plain]
%       This theorem style emulates the original \LaTeX{} definition,
%       except that additionally the parameters
%       |\theorem...skipamount| are used.
%    \item[break]
%       In this style, the theorem header is followed by a line
%       break.
%    \item[marginbreak]
%       The theorem number is set in the margin, and there is a line
%       break as in \texttt{break}.
%    \item[changebreak]
%       Like \texttt{break}, but with header number and text
%       interchanged.
%    \item[change]
%       Header number and text are interchanged, without a line break.
%    \item[margin]
%       The number is set in the left margin, without a line break.
% \end{deflist}
% All styles (except \texttt{plain}) select |\slshape| as the default
% |\theorembodyfont|.
%
%
% \subsection{Examples}
%
% Given the above theorem sets {\envfont Cor}, {\envfont Exa}
% and {\envfont Rem},
% suppose that the preamble also contains the declarations:
% \begin{verbatim}
%    \theoremstyle{marginbreak}   \newtheorem{Lem}[Cor]{Lemma}
%    \theoremstyle{change}
%    \theorembodyfont{\itshape}        \newtheorem{Def}[Cor]{Definition}
%
%    \theoremheaderfont{\scshape}
%\end{verbatim}
% Then the following are some typical examples of the typeset output
% resulting from their use.
%
% \begin{Cor}
%    This is a sentence typeset in the theorem environment {\envfont
%    Cor}.
% \end{Cor}
%
% \begin{Exa}
%    This is a sentence typeset in the theorem environment {\envfont
%    Exa}.
% \end{Exa}
%
% \begin{Rem}
%    This is a sentence typeset in the theorem environment {\envfont
%    Rem}.
% \end{Rem}
%
% \begin{Lem}[Ben User]
%    This is a sentence typeset in the theorem environment {\envfont
%    Lem}.
% \end{Lem}
%
% \begin{Def}[Very impressive Definition]
%    This is a sentence typeset in the theorem environment {\envfont
%    Def}.
% \end{Def}
%
% The last two examples show the effect of the optional argument to
% a theorem environment (it is the text typeset in parentheses).
%
%
% \section{Special Considerations}
% \label{sec:newfonts}
%
% Theoremheader and body are implemented as a unit. This means that
% the |\theoremheaderfont|  will inherit characteristics of the
% |\theorembodyfont| in \LaTeXe. Thus, if for example
% |\theorembodyfont| is |\itshape| and |\theoremheaderfont|
% is |\bfseries| the font selected for the header will have the
% characteristics `bold extended italic'. If this is not desired one
% should set the |\theoremheaderfont| to something like
% \begin{verbatim}
%      \theoremheaderfont{\normalfont\bfseries}
%\end{verbatim}
% i.e.\ supplying all necessary font information explicitly.
%
%
%
% \section{Acknowledgements}
%
% The publication of this set of macros was only possible with the
% help of Christina Busse (translating the manuscript into English),
% Joachim Pense (playing the r\^ole of typist), Chris Rowley
% (looking everything over) and many others providing useful
% suggestions.
%
% \MaybeStop{%
%    \begin{thebibliography}{9}
%       \bibitem{bk:GMS94} \textsc{M.~Goossens}, \textsc{F.~Mittelbach}
%         and \textsc{A.~Samarin}.
%         \newblock The \LaTeX{} Companion.
%         \newblock
%         Addison-Wesley, Reading, Massachusetts, 1994.
%       \bibitem{latex:source} \textsc{Lamport, Leslie.} \newblock
%                               \texttt{latex.tex}, version 2.09, date
%                               Feb. 1990.
%    \end{thebibliography}
%    \PrintIndex
%    \PrintChanges}
%
%
% \section{The documentation driver file}
%
% The next bit of code contains the documentation driver file for
% \TeX{}, i.e., the file that will produce the documentation you are
% currently reading. It will be extracted from this file by the
% \texttt{docstrip} program. Since it is the first code in the file
% one can alternatively process this file directly with \LaTeXe{} to
% obtain the documentation.
%
%    \begin{macrocode}
%<*driver>
\documentclass{ltxdoc}

\usepackage{theorem}

%<+driver>% The next few lines define theorem sets which are used
%<+driver>% in the example section of the documentation.

\theoremstyle{break}       \newtheorem{Cor}{Corollary}
\theoremstyle{plain}       \newtheorem{Exa}{Example}[section]
{\theorembodyfont{\upshape}\newtheorem{Rem}{Remark}}
\theoremstyle{marginbreak} \newtheorem{Lem}[Cor]{Lemma}
\theoremstyle{change}
\theorembodyfont{\itshape} \newtheorem{Def}[Cor]{Definition}

\theoremheaderfont{\scshape}

\RecordChanges

\begin{document}
   \DocInput{theorem.dtx}
\end{document}
%</driver>
%    \end{macrocode}
%
% \section{Definition of the Macros}
%
% If the file has been loaded before, we abort immediately. If not the
% package announces itself (this is actually done at the very top of
% the file---the way it is done isn't good style so don't copy it).
% \changes{v2.0e}{89/07/19}{Spaces removed from `typeout'.}
% \changes{v2.2b}{95/11/19}{Announce moved to top of file}
%    \begin{macrocode}
%<*package>
%\@ifundefined{theorem@style}{}{\endinput}
%\def\FMithmInfo{1995/11/19 v2.2b Theorem extension package (FMi)}
%\ProvidesPackage{theorem}[\FMithmInfo]
%    \end{macrocode}
%
%
% \subsection{Definition of theorem styles and fonts}
%
% All the definitions in this file are done globally to allow
% inputting this file inside a group.
%
% \begin{macro}{\theoremstyle}
%    Before a theorem style can be installed, the chosen style must be
%    known. For that reason, we must test to see that
%    |\th@|\meta{style} is known or, more precisely, that it is
%    different from |\relax|. If the style is not known then
%    |\th@plain| is used.
%    \begin{macrocode}
\gdef\theoremstyle#1{%
   \@ifundefined{th@#1}{\@warning
          {Unknown theoremstyle `#1'. Using `plain'}%
          \theorem@style{plain}}%
%    \end{macrocode}
%    We save the theorem style to be used in the token register
%    |\theorem@style|.
%    \begin{macrocode}
      {\theorem@style{#1}}%
%    \end{macrocode}
%    Now we ``evaluate'' the theorem style: this means, we call the
%    macro |\th@|\meta{style} which will activate the relevant
%    definitions which are contained in a separate file.  This is done
%    in a group to suppress changes to the current font.  This could
%    otherwise pose problems together with the new font selection
%    scheme\footnote{When I printed the original article using the new
%    font selection scheme I ended with a document with slanted
%    typefaces (text headings and all) simply because one of the
%    theorem styles used \texttt{\bslash sl} at toplevel.} if the
%    |\th@|\meta{style} is evaluated a second time.
% \changes{v2.1a}{90/02/15}{Group added around `th@..}
% \changes{v2.1a}{90/02/15}{Reseting of `..@begintheorem removed.}
%    \begin{macrocode}
      \begingroup
        \csname th@\the\theorem@style \endcsname
      \endgroup}
%    \end{macrocode}
% \end{macro}
%
%
% \begin{macro}{\@begintheorem}
% \changes{v2.1a}{90/02/15}{Relaxed at toplevel.}
% \begin{macro}{\@opargbegintheorem}
% \changes{v2.1a}{90/02/15}{Relaxed at toplevel.}
%    We reset |\@begintheorem| and |\@opargbegintheorem| to
%    |\relax| since these commands are no longer necessary at
%    toplevel. This will save a few tokens.
%    \begin{macrocode}
\global\let\@begintheorem\relax
\global\let\@opargbegintheorem\relax
%    \end{macrocode}
% \end{macro}
% \end{macro}
%
% \begin{macro}{\theorem@style}
%    Obviously the \textsf{token} register used above has to be
%    allocated. To assure the utmost compatibility with the
%    original \LaTeX{} definition, we set the default theorem style to
%    \texttt{plain}, which implements the usual \LaTeX{} convention.
% \changes{v2.0f}{89/07/20}{Unnecessary `global' removed.}
%    \begin{macrocode}
\newtoks\theorem@style
\global\theorem@style{plain}
%    \end{macrocode}
% \end{macro}
%
% \begin{macro}{\theorembodyfont}
% \begin{macro}{\theorem@bodyfont}
%    If the |\theorembodyfont| is set by the user then it should not
%    interact with the default font set in the theorem style. When the
%    new font selection is in force this may happen if, for example,
%    the default is |\itshape| and the new |\theorembodyfont| is
%    |\sffamily|. So we add a |\reset@font| command in front of the
%    user definition.
% \changes{v2.0f}{89/07/20}{Unnecessary `global' removed.}
% \changes{v2.0g}{89/09/19}{`theoremfont to `theorembodyfont.}
% \changes{v2.1c}{92/06/22}{`reset@font added and logic changed.}
% \changes{v2.1d}{92/10/28}{Reset to empty if arg is empty}
%    \begin{macrocode}
\gdef\theorembodyfont#1{%
%    \end{macrocode}
%    We check if the argument supplied is empty and if so put nothing
%    into the |\theorem@bodyfont| token register to allow for
%    |\theorembodyfont{}| as a mean of using the default of the
%    current |\theoremstyle|.
%    \begin{macrocode}
   \def\@tempa{#1}%
   \ifx\@tempa\@empty
    \theorem@bodyfont{}%
   \else
    \theorem@bodyfont{\reset@font#1}%
   \fi
   }
\newtoks\theorem@bodyfont
\global\theorem@bodyfont{}
%    \end{macrocode}
% \end{macro}
% \end{macro}
%
%
% \begin{macro}{\theoremheaderfont}
% \changes{v2.0g}{89/09/19}{Macro added.}
%    The font for the theorem headers is handled differently because
%    this definition applies to all theorem styles.
%    \begin{macrocode}
\gdef\theoremheaderfont#1{\gdef\theorem@headerfont{#1}%
%    \end{macrocode}
%    After using the macro once it is redefined to produce an error
%    message.
%    \begin{macrocode}
       \gdef\theoremheaderfont##1{%
        \typeout{\string\theoremheaderfont\space should be used
                 only once.}}}
%    \end{macrocode}
% \end{macro}
%
% \begin{macro}{\theorem@headerfont}
% \changes{v2.0g}{89/09/19}{Macro added.}
%    To set the |\theorem@headerfont| default we first test if
%    the new fontselection scheme is in force.
% \changes{v2.1a}{90/02/15}{Support for new font selection.}
%    \begin{macrocode}
\ifx\upshape\undefined
%    \end{macrocode}
%    If not we define it to expand into |\bfseries|. We don't use
%    |\let| just in case a following style option redefines this
%    macro.
%    \begin{macrocode}
\gdef\theorem@headerfont{\bfseries}
%    \end{macrocode}
%    Otherwise we reset the current shape before calling |\bfseries|.
%    \begin{macrocode}
\else \gdef\theorem@headerfont{\normalfont\bfseries}\fi
%    \end{macrocode}
% \end{macro}
%
% \begin{macro}{\th@plain}
% \begin{macro}{\th@break}
% \begin{macro}{\th@marginbreak}
% \begin{macro}{\th@changebreak}
% \begin{macro}{\th@change}
% \begin{macro}{\th@margin}
%    The different styles are defined in macros such as
%    |\th@plain|.  Since memory space is precious in
%    ``non-Big-versions'', we have to avoid offering too many unused
%    definitions.  Therefore we define these styles in separate files
%    that can be loaded on demand.  Thus the commands themselves only
%    load these files. We use |\@input@|, a \LaTeXe{} internal command
%    that ensures that the file will be listed with |\listfiles|.
% \changes{v2.2a}{94/02/02}{Use `@input@ to load theorem layout files}
%    \begin{macrocode}
\gdef\th@plain{\@input@{thp.sty}}
\gdef\th@break{\@input@{thb.sty}}
\gdef\th@marginbreak{\@input@{thmb.sty}}
\gdef\th@changebreak{\@input@{thcb.sty}}
\gdef\th@change{\@input@{thc.sty}}
\gdef\th@margin{\@input@{thm.sty}}
%    \end{macrocode}
%    This list will be expanded when new styles become
%    available.  For testing,  just append new theorem substyles
%    as document options.
% \end{macro}
% \end{macro}
% \end{macro}
% \end{macro}
% \end{macro}
% \end{macro}
%
%
%
% \subsection{Definition of a new theorem set}
%
% As already pointed out, a new theorem environment can be defined in
% three different ways:
% \begin{verbatim}
%    \newtheorem{Lem}{Lemma}
%    \newtheorem{Lem}{Lemma}[section]
%    \newtheorem{Lem}[Theorem]{Lemma}
%\end{verbatim}
% The function of the macro |\newtheorem| is to recognize these cases
% and then to branch into one of the three macros |\@ynthm|, |\@xnthm|
% or |\@othm|.  This mechanism is adopted unchanged from
% \cite{latex:source}; the essential point here is that, for example,
% in the second case, the arguments \texttt{Lem}, \texttt{Lemma} and
% \texttt{section} are passed over to the macro |\@xnthm|.
%
% We inspect this case first because the others present fewer problems,
% and thus are easily derived from this one.
%
% \begin{macro}{\@xnthm}
%    For our example arguments, the macro |\@xnthm| must fulfill
%    the following:
%    \begin{itemize}
%       \item
%          Define a new \LaTeX-counter `Lem'
%       \item
%          reset this counter within a |\section|
%       \item
%          define the macro |\theLem|
%       \item
%          define the environment macros |\Lem| and
%          |\endLem| using the current
%          |\theoremstyle| and |\theorem@bodyfont|.
%    \end{itemize}
%    Obviously, all this should happen only if the first argument
%    of |\@xnthm| (i.e.\ \texttt{Lem} in our example) is chosen so
%    as not to conflict with any previously defined commands or
%    environments.  This test is performed by the \LaTeX{} macro
%    |\@ifdefinable|.
%    \begin{macrocode}
\gdef\@xnthm#1#2[#3]{\expandafter\@ifdefinable\csname #1\endcsname
%    \end{macrocode}
%    Therefore, the first argument of |\@ifdefinable| is the
%    expansion (in the example, |\Lem|) of
%    |\csname||#1\endcsname|.  The second argument is
%    executed only if the test has been completed successfully.
%    \begin{macrocode}
   {%
%    \end{macrocode}
%    Now we define the new counter. The names of the \LaTeX{}
%    macros employed should speak for themselves:
% \changes{v2.2b}{1995/10/07}{Use \cs{@newctr}}
% \changes{v2.2c}{1995/11/23}{Use \cs{@newctr} properly.}
%    \begin{macrocode}
    \@definecounter{#1}\@newctr{#1}[#3]%
%    \end{macrocode}
%    Using |\@newctr| will give a proper error message if the counter
%    in |#3| is not defined.
%    In defining `|\theLem|' we must generate the desired macro
%    name by use of |\expandafter| and |\csname|.
%    \begin{macrocode}
    \expandafter\xdef\csname the#1\endcsname
%    \end{macrocode}
%    An |\xdef| is used in order to make the definition global,
%    and to ensure that it contains the replacement texts of
%    |\@thmcountersep| and |\@thmcounter|.\footnote{These
%    two macros can be defined by the document style.  Their default
%    values produce a `\texttt{.}' as separation and an arabic
%    representation of the number.} However, not everything should be
%    expanded. For example, it saves space to use |\thesection|
%    instead of its---at times---lengthy expansion.
%    \begin{macrocode}
      {\expandafter \noexpand \csname the#3\endcsname
       \@thmcountersep \@thmcounter{#1}}%
%    \end{macrocode}
%    Thus with the defaults of \LaTeX{}, |\theLem| would be
%    replaced by the command sequence |\thesection.\arabic{Lem}|.
%
%    We will now look at the definition of the macro which is executed
%    at the beginning of the actual environment (in our example this
%    macro is |\Lem|). It should be noted that we use an
%    ``|\expandafter| trick'' to expand only certain parts of the
%    replacement text at the time of the definition.
%    \begin{macrocode}
    \def\@tempa{\global\@namedef{#1}}%
    \expandafter \@tempa \expandafter{%
%    \end{macrocode}
%    First, the macro that contains the current definitions of
%    |\@begintheorem| and |\@opargtheorem| should be called
%    up.  The name of this macro---as is already known---has the form
%    |\th@|\meta{theorem\ style}; therefore, it must be called by
%    \begin{macrocode}
      \csname th@\the \theorem@style
%    \end{macrocode}
%    In addition the default theorem font should be changeable, i.e.\
%    we have to insert the contents of |\theorem@bodyfont|. For
%    that reason, we expand even further, beyond |\endcsname|,
%    and thus insert the contents of the \textsf{token} register
%    |\theorem@bodyfont| in the replacement text.
%    \begin{macrocode}
            \expandafter \endcsname \the \theorem@bodyfont
%    \end{macrocode}
%    Now it is time to call the macro |\@thm| which takes over
%    the further processing.  It has two arguments: the current
%    counter name (in our example, \texttt{Lem}), and the text of the
%    label (in our example, \texttt{Lemma}).
%    \begin{macrocode}
     \@thm{#1}{#2}}%
%    \end{macrocode}
%    With this, the `sub-definition' is complete.  The macro
%    |\@endtheorem| ends a theorem environment and is, so far, nothing
%    but an |\endtrivlist|.  (Hence it is defined globally, and not
%    within the theorem styles.\footnote{This has to be changed as
%    soon as theorem styles that change \texttt{\bslash @endtheorem}
%    exist.  In such a case, all existing styles must be changed as
%    well since they will have to reset the macro.}) Therefore, we can
%    set it equivalent to the macro that ends the theorem set (in our
%    example, |\endLem|).  However, if some day theorem styles exist
%    that do change |\@endtheorem|, we would have to use the
%    commented-out line instead.
%    \begin{macrocode}
    \global \expandafter \let \csname end#1\endcsname \@endtheorem
%   \global\@namedef{end#1}{\@endtheorem}%
%    \end{macrocode}
%    With these commands all the required definitions are employed,
%    unless the test |\@ifdefinable| has failed.  Therefore, we
%    end the second argument of this macro and with it the definition
%    of |\@xnthm|.
%    \begin{macrocode}
   }}
%    \end{macrocode}
% \end{macro}
%
% \begin{macro}{\@ynthm}
%    The definition of |\@ynthm| is completely analogous. In
%    this case the new counter that is defined is not reset within
%    another counter; thus the definition of |\the...|
%    is simplified:
%    \begin{macrocode}
\gdef\@ynthm#1#2{\expandafter\@ifdefinable\csname #1\endcsname
   {\@definecounter{#1}%
    \expandafter\xdef\csname the#1\endcsname{\@thmcounter{#1}}%
%    \end{macrocode}
%    The rest of the definition corresponds literally to that of
%    |\@xnthm|:
%    \begin{macrocode}
    \def\@tempa{\global\@namedef{#1}}\expandafter \@tempa
     \expandafter{\csname th@\the \theorem@style \expandafter
     \endcsname \the\theorem@bodyfont \@thm{#1}{#2}}%
    \global \expandafter \let \csname end#1\endcsname \@endtheorem}}
%    \end{macrocode}
% \end{macro}
%
% \begin{macro}{\@othm}
%    The definition of |\@othm| does not contain anything new.
%    \begin{macrocode}
\gdef\@othm#1[#2]#3{%
%    \end{macrocode}
% \changes{v2.2b}{95/10/07}{Build in check for counter defined}
%    We do not define a new counter but instead use one that has
%    already been defined.  Thus the only definition we need is that
%    of this pseudo-counter (i.e.\ |\the|\meta{env.\ name}).
%    First we check if |#2| corresponds to a known counter name.
%    \begin{macrocode}
  \expandafter\ifx\csname c@#2\endcsname\relax
   \@nocounterr{#2}%
  \else
   \expandafter\@ifdefinable\csname #1\endcsname
   {\expandafter \xdef \csname the#1\endcsname
     {\expandafter \noexpand \csname the#2\endcsname}%
%    \end{macrocode}
%    All other parts of the definition can be adopted from
%    |\@xnthm|. We have to remember, though, that in this case
%    the name of the current counter and the theorem label have moved
%    to the second and third arguments.
%    \begin{macrocode}
    \def\@tempa{\global\@namedef{#1}}\expandafter \@tempa
     \expandafter{\csname th@\the \theorem@style \expandafter
     \endcsname \the\theorem@bodyfont \@thm{#2}{#3}}%
    \global \expandafter \let \csname end#1\endcsname \@endtheorem}%
  \fi}
%    \end{macrocode}
% \end{macro}
%
% \subsection{Macros that are employed in a {\envfont theorem}
%    environment}
%
% \begin{macro}{\@thm}
%    The macro |\@thm| has to increase the current counter.  Then,
%    depending on whether the environment has (or does not have) an
%    optional argument, it has to branch into either |\@begintheorem|
%    or |\@opargtheorem|.
%    \begin{macrocode}
\gdef\@thm#1#2{\refstepcounter{#1}%
%    \end{macrocode}
%    Now we start a {\envfont trivlist} environment, and give
%    |\@topsep| and |\@topsepadd| the values of the \textsf{skip}
%    registers |\theorempreskipamount| and |\theorempostskipamount|.
%    The value in |\@topsep| is the vertical space that is inserted by
%    the first (and only) |\item| in our |\trivlist| whilst
%    |\@topsepadd| is inserted by |\@endparenv| at the end of that
%    {\envfont trivlist} environment.  By using these registers, we
%    obtain the desired space around a {\envfont theorem} environment.
%    \begin{macrocode}
   \trivlist
   \@topsep \theorempreskipamount               % used by first \item
   \@topsepadd \theorempostskipamount           % used by \@endparenv
%    \end{macrocode}
%    Now we have to test whether an optional argument has been given.
% \changes{v2.0d}{89/07/18}{Forgotten percent added.}
%    \begin{macrocode}
   \@ifnextchar [%
%    \end{macrocode}
%    If there is an optional argument, we will call |\@ythm|, and
%    move the arguments read back into the input stream.
%    \begin{macrocode}
   {\@ythm{#1}{#2}}%
%    \end{macrocode}
%    If not, we call |\@begintheorem|. Its first argument is the
%    name of the theorem set (hence the second argument of
%    |\@thm|). Its second argument is the macro that produces the
%    current number.
%    \begin{macrocode}
   {\@begintheorem{#2}{\csname the#1\endcsname}\ignorespaces}}
%    \end{macrocode}
% \end{macro}
%
% \begin{macro}{\@xthm}
% \begin{macro}{\@ythm}
%    Both these macros were originally called by |\@thm|.  We do
%    not need |\@xthm| anymore, hence we reset it to
%    |\relax|. The definition of |\@ythm| has not changed at
%    all from its definition in \LaTeX. In order to make the macros
%    easier to understand, we will
%    nevertheless present it (commented out).
%    \begin{macrocode}
\global\let\@xthm\relax
% \def\@ythm#1#2[#3]{\@opargbegintheorem{#2}{\csname
%           the#1\endcsname}{#3}\ignorespaces}
%</package>
%    \end{macrocode}
%    The primitive |\ignorespaces| in |\@ythm| and
%    |\@thm| is needed to remove the spaces between the
%    |\begin{...}| and the actual text.
% \end{macro}
% \end{macro}
%
%
% \subsection{Definition of the theorem substyles}
%
% As already pointed out, the theorem substyles, defined below, are
% only loaded when necessary.  Note that all these substyles, except
% \texttt{plain}, have |\slshape| as the default body font.
%
% \subsubsection{The \texttt{plain} style}
%
% \label{dok:th@plain} As the following macros use |@|, we have
% to locally set the |\catcode| of this symbol to ``letter''.
% This happens within a group, so that we do not have to worry about
% which |\catcode| that symbol had before.
%    \begin{macrocode}
%<*thp>
\begingroup \makeatletter
%    \end{macrocode}
% Since we are now within a group, we must make all definitions
% globally.  First we make sure that \texttt{theorem.sty} is loaded.
% This will allow us to use this file as a document style option
% without having to call \texttt{theorem} itself as an option.  At the
% same time, we assure that at least version 2 is loaded, since
% |\theorem@style| was not defined in earlier versions.
%    \begin{macrocode}
\@ifundefined{theorem@style}{\input{theorem.sty}}{}
\ProvidesFile{thp.sty}
             [\FMithmInfo]
%    \end{macrocode}
%
%
% \begin{macro}{\th@plain}
% \begin{macro}{\@begintheorem}
% \begin{macro}{\@opargbegintheorem}
%    |\theoremstyle{plain}| corresponds to the original
%    definition, except that the distances to the surrounding text are
%    determined by the parameters |\theorempreskipamount| and
%    |\theorempostskipamount|.
%    First we set the default body font.
%    \begin{macrocode}
\gdef\th@plain{\normalfont\itshape
%    \end{macrocode}
%    Then we define |\@begintheorem| and
%    |\@opargbegintheorem|. These two macros define how the
%    header of a theorem is typeset.  |\@opargbegintheorem| will
%    be called if a \textsf{theorem} environment with an optional
%    argument is encountered; otherwise, the header is constructed by
%    calling |\@begintheorem|.  If one of these macros is
%    executed, we are within a \textsf{trivlist} environment started by
%    |\@thm|.  So the theorem header is produced with an
%    |\item| command.
%
%    Instead of specifying the header font directly, all standard
%    theorem styles use the |\theorem@headerfont| macro to allow
%    customization.  The extra space (|\labelsep|) is necessary
%    because of problems in the \textsf{trivlist} environment.
%    \begin{macrocode}
  \def\@begintheorem##1##2{%
        \item[\hskip\labelsep \theorem@headerfont ##1\ ##2]}%
%    \end{macrocode}
%    The definition of |\@opargbegintheorem| is completely
%    analogous. The only difference is the fact that there exists a
%    third argument (which is the optional parameter of the
%    environment and contains additional information about the
%    theorem).  Customarily we enclose it in parentheses.
%    \begin{macrocode}
\def\@opargbegintheorem##1##2##3{%
   \item[\hskip\labelsep \theorem@headerfont ##1\ ##2\ (##3)]}}
%    \end{macrocode}
% \end{macro}
% \end{macro}
% \end{macro}
%
% \noindent
% We conclude with an |\endgroup| to restore the
% |\catcode| of |@|.
%    \begin{macrocode}
\endgroup
%</thp>
%    \end{macrocode}
%
%
% \subsubsection{The \texttt{break} style}
%
% This style option is stored in the file |thb.sty|.
% For the next two lines see the documentation for |\th@plain| on
% page~\pageref{dok:th@plain}.
%    \begin{macrocode}
%<*thb>
\begingroup \makeatletter
\@ifundefined{theorem@style}{\input{theorem.sty}}{}
\ProvidesFile{thb.sty}
             [\FMithmInfo]
%    \end{macrocode}
%
% \begin{macro}{\th@break}
%    |\theoremstyle{break}| produces a line break after the name
%    of the theorem.  The font is |\slshape|.  Hence, we define
%    |\th@break| as follows:
%    \begin{macrocode}
\gdef\th@break{\normalfont\slshape
  \def\@begintheorem##1##2{\item[%
%    \end{macrocode}
%    We run into the following problem: it is not possible to create
%    the header with |\item[|\meta{title}|]| and then start
%    a new line by, for example, |\mbox{}\\|. Such a definition
%    will fail whenever a list environment follows immediately.  With
%    the above construction, the |\mbox{}| causes the switch
%    |@inlabel| (cf.\ definition of |\list| and
%    |\trivlist| in \cite{latex:source}) to be set to \texttt{false}
%    and so the following list will insert additional vertical
%    space (|\topskip|). This is quite annoying. Therefore, we
%    create the line break within the |\item|. In order to ensure
%    that the text will begin at the proper position in the following
%    line, we simply pretend that the label does not take any
%    room.\footnote{This will lead to problems whenever very high
%    symbols occurring in the line tower into the
%    heading. So, something else has to be done here sometime.}
%    \begin{macrocode}
       \rlap{\vbox{\hbox{\hskip \labelsep\theorem@headerfont ##1\ ##2}%
                   \hbox{\strut}}}]}%
%    \end{macrocode}
%    Again, the definition of |\@opargbegintheorem| is
%    completely analogous.
%    \begin{macrocode}
\def\@opargbegintheorem##1##2##3{%
  \item[\rlap{\vbox{\hbox{\hskip \labelsep \theorem@headerfont
                     ##1\ ##2\ (##3)}%
                    \hbox{\strut}}}]}}
\endgroup
%</thb>
%    \end{macrocode}
% \end{macro}
%
%
% \subsubsection{The \texttt{changebreak} style}
%
%    \begin{macrocode}
% This style option is stored in the file |thcb.sty|.
%    \begin{macrocode}
%<*thcb>
\begingroup \makeatletter
\@ifundefined{theorem@style}{\input{theorem.sty}}{}
\ProvidesFile{thcb.sty}
             [\FMithmInfo]
%    \end{macrocode}
%
% \begin{macro}{\th@changebreak}
%    The \texttt{change-break} theorem style is like \texttt{break}
%    but with interchange of theorem name and theorem number.  Thus we
%    define |\th@changebreak| as follows:
%    \begin{macrocode}
\gdef\th@changebreak{\normalfont\slshape
  \def\@begintheorem##1##2{\item
       [\rlap{\vbox{\hbox{\hskip\labelsep \theorem@headerfont ##2\ ##1}%
                    \hbox{\strut}}}]}%
\def\@opargbegintheorem##1##2##3{%
  \item[\rlap{\vbox{\hbox{\hskip\labelsep \theorem@headerfont
                     ##2\ ##1\ (##3)}%
                    \hbox{\strut}}}]}}
\endgroup
%</thcb>
%    \end{macrocode}
% \end{macro}
%
%
% \subsubsection{The \texttt{change} style}
%
% This style option is stored in the file |thc.sty|.
%    \begin{macrocode}
%<*thc>
\begingroup \makeatletter
\@ifundefined{theorem@style}{\input{theorem.sty}}{}
\ProvidesFile{thc.sty}
             [\FMithmInfo]
%    \end{macrocode}
%
% \begin{macro}{\th@change}
%    The \texttt{change} theorem style corresponds to the
%    \texttt{change break} style without a linebreak after the header.
%    To say it in another way, it's the same as the \texttt{plain}
%    style but with number and name interchanged and |\slshape| as the
%    default font.
%    \begin{macrocode}
\gdef\th@change{\normalfont\slshape
  \def\@begintheorem##1##2{\item
       [\hskip\labelsep \theorem@headerfont ##2\ ##1]}%
\def\@opargbegintheorem##1##2##3{%
  \item[\hskip\labelsep \theorem@headerfont ##2\ ##1\ (##3)]}}
\endgroup
%</thc>
%    \end{macrocode}
% \end{macro}
%
%
% \subsubsection{The \texttt{marginbreak} style}
%
% This style option is the one used most often at Mainz.  It is saved
% in the file |thmb.sty|.
%    \begin{macrocode}
%<*thmb>
\begingroup \makeatletter
\@ifundefined{theorem@style}{\input{theorem.sty}}{}
\ProvidesFile{thmb.sty}
             [\FMithmInfo]
%    \end{macrocode}
%
% \begin{macro}{\th@marginbreak}
%    The \texttt{margin break} style is nearly the same as the
%    \texttt{change break} style. The only difference is the placement
%    of the theorem number.  We use |\llap| to place it in the left
%    margin.
%
%    In this style |\labelsep| denotes the separation between the
%    number and the text.
%    \begin{macrocode}
\gdef\th@marginbreak{\normalfont\slshape
  \def\@begintheorem##1##2{\item
       [\rlap{\vbox{\theorem@headerfont
                    \hbox{\llap{##2}\hskip\labelsep ##1}%
                    \hbox{\strut}}}]}%
\def\@opargbegintheorem##1##2##3{%
  \item[\rlap{\vbox{\theorem@headerfont
                    \hbox{\llap{##2}\hskip\labelsep ##1\ (##3)}%
                    \hbox{\strut}}}]}}
\endgroup
%</thmb>
%    \end{macrocode}
% \changes{v2.1c}{90/02/17}{`endgroup outside macrocode env.}
% \end{macro}
%
%
%
% \subsubsection{The \texttt{margin} style}
%
% This style option is stored in the file |thm.sty|.
%    \begin{macrocode}
%<*thm>
\begingroup \makeatletter
\@ifundefined{theorem@style}{\input{theorem.sty}}{}
\ProvidesFile{thm.sty}
             [\FMithmInfo]
%    \end{macrocode}
%
% \begin{macro}{\th@margin}
%    Again this is only a variant of the theorem styles described
%    above without any new ideas.
%    \begin{macrocode}
\gdef\th@margin{\normalfont\slshape
  \def\@begintheorem##1##2{\item
       [\theorem@headerfont \llap{##2}\hskip\labelsep ##1]}%
\def\@opargbegintheorem##1##2##3{%
  \item[\theorem@headerfont \llap{##2}\hskip\labelsep ##1\ (##3)]}}
\endgroup
%</thm>
%    \end{macrocode}
% \end{macro}
%
%
%
%
% \subsection{Final Definitions}
%
% \begin{macro}{\theorempreskipamount}
% \begin{macro}{\theorempostskipamount}
%    The \textsf{skip} parameters that regulate the vertical empty space
%    before and after the {\envfont theorem} environment have
%    to be allocated as well.
% \changes{v2.0f}{89/07/20}{Unnecessary `global' removed.}
%    \begin{macrocode}
%<*package>
\newskip\theorempreskipamount
\newskip\theorempostskipamount
%    \end{macrocode}
%    Since we have used the same values for all theorem sets, we now can
%    assign them.
%    \begin{macrocode}
\global\setlength\theorempreskipamount{12pt plus 5pt minus 3pt}
\global\setlength\theorempostskipamount{8pt plus 3pt minus 1.5pt}
%    \end{macrocode}
% \end{macro}
% \end{macro}
%
%
% \begin{macro}{\@endtheorem}
%    The same holds for the macro |\@endtheorem|, which ends a
%    {\envfont theorem} environment. Since it is the same for all
%    theorem sets, it is removed from the macros |\th@|\meta{style}.
%    It simply ends the {\envfont trivlist} environment, which was
%    begun in |\@thm|.
%    \begin{macrocode}
\global\let\@endtheorem=\endtrivlist
%    \end{macrocode}
% \end{macro}
%
%
%
% \begin{macro}{\@preamblecmds}
%    All macros defined above are to be used only in the preamble.
%    Therefore, we insert them in |\@preamblecmds| which will disable
%    them at begin document. This is done by the internal \LaTeXe{}
%    command |\@onlypreamble|.
% \changes{v2.0g}{89/09/19}{`theoremheaderfont and `theorembodyfont
%    added.}
% \changes{v2.2a}{94/02/03}{Use \LaTeXe{} interface}
%    \begin{macrocode}
\@onlypreamble\@xnthm
\@onlypreamble\@ynthm
\@onlypreamble\@othm
\@onlypreamble\newtheorem
\@onlypreamble\theoremstyle
\@onlypreamble\theorembodyfont
\@onlypreamble\theoremheaderfont
%    \end{macrocode}
% \end{macro}
%
% Finally we declare the \texttt{plain} theorem style to be the default.
% \changes{v2.1b}{90/06/04}{Fixed missing `item by declaring default
%    style.}
%    \begin{macrocode}
\theoremstyle{plain}
%</package>
%    \end{macrocode}
%
% \Finale

\endinput
}{}
\ProvidesFile{thc.sty}
             [\FMithmInfo]
%    \end{macrocode}
%
% \begin{macro}{\th@change}
%    The \texttt{change} theorem style corresponds to the
%    \texttt{change break} style without a linebreak after the header.
%    To say it in another way, it's the same as the \texttt{plain}
%    style but with number and name interchanged and |\slshape| as the
%    default font.
%    \begin{macrocode}
\gdef\th@change{\normalfont\slshape
  \def\@begintheorem##1##2{\item
       [\hskip\labelsep \theorem@headerfont ##2\ ##1]}%
\def\@opargbegintheorem##1##2##3{%
  \item[\hskip\labelsep \theorem@headerfont ##2\ ##1\ (##3)]}}
\endgroup
%</thc>
%    \end{macrocode}
% \end{macro}
%
%
% \subsubsection{The \texttt{marginbreak} style}
%
% This style option is the one used most often at Mainz.  It is saved
% in the file |thmb.sty|.
%    \begin{macrocode}
%<*thmb>
\begingroup \makeatletter
\@ifundefined{theorem@style}{% \iffalse meta-comment
%
% Copyright (C) 1993-2024
%
% The LaTeX Project and any individual authors listed elsewhere
% in this file.
%
% This file is part of the Standard LaTeX `Tools Bundle'.
% -------------------------------------------------------
%
% It may be distributed and/or modified under the
% conditions of the LaTeX Project Public License, either version 1.3c
% of this license or (at your option) any later version.
% The latest version of this license is in
%    https://www.latex-project.org/lppl.txt
% and version 1.3c or later is part of all distributions of LaTeX
% version 2005/12/01 or later.
%
% The list of all files belonging to the LaTeX `Tools Bundle' is
% given in the file `manifest.txt'.
%
% \fi
%
%%
%
%
\def\FMithmInfo{2023/07/05 v2.2c Theorem extension package (FMi)}
%
% \ProvidesFile{theorem.dtx}[\FMithmInfo]
%\iffalse   % this is a METACOMMENT !
%%
%% Package `theorem' to use with LaTeX2e
%% Copyright (C) 1989-1995 Frank Mittelbach, all rights reserved.
%<+package>\@ifundefined{theorem@style}{}{\endinput}
%<+package>\NeedsTeXFormat{LaTeX2e}[1995/06/01]
%<+package>\ProvidesPackage{theorem}
%<+package>         [\FMithmInfo]
%
% \fi
%
% \changes{v1.0}{87/05/07}{First draft.}
% \changes{v1.0a}{87/05/07}{`thistheoremstyle' added.}
% \changes{v1.1a}{87/07/29}{New concept: redefine the internal
%                          \protect\LaTeX{} commands `@xnthm' etc.}
% \changes{v1.1b}{87/07/31}{`theoremfont' used to specify font.}
% \changes{v1.2a}{88/01/08}{`@endtheorem' now globally defined.}
% \changes{v1.2b}{88/01/11}{`newtheorem' only allowed in the preamble.}
% \changes{v1.3a}{88/02/14}{`theoremfont' now a toks register.}
%
% \changes{v2.0a}{88/11/16}{New implementation and new documentation.}
% \changes{v2.0a}{88/11/16}{No layout changes but internal modifications
%                           old substyles won't work.}
% \changes{v2.0b}{89/01/30}{`global' used for all definitions.}
% \changes{v2.0c}{89/06/15}{Documentation translated to English.
%                            Thanks to LLL and Nina}
% \changes{v2.1a}{90/02/15}{Support for new font selection.}
% \changes{v2.2a}{94/02/03}{Cleanup for LaTeX2e}
%
% \DoNotIndex{\@endparenv,\@ifdefinable,\@ifnextchar,\@ifundefined}
% \DoNotIndex{\@namedef,\@nameuse,\@spaces,\@tempa,\@warning}
% \DoNotIndex{\ }
% \DoNotIndex{\begingroup,\bfseries}
% \DoNotIndex{\csname}
% \DoNotIndex{\def,\do,\docdate}
% \DoNotIndex{\endcsname,\endgroup,\endinput,\endtrivlist,\expandafter}
% \DoNotIndex{\filedate,\fileversion}
% \DoNotIndex{\gdef,\global}
% \DoNotIndex{\hbox,\hskip}
% \DoNotIndex{\ignorespaces,\input,\itshape,\item}
% \DoNotIndex{\labelsep,\let,\llap}
% \DoNotIndex{\makeatletter}
% \DoNotIndex{\noexpand}
% \DoNotIndex{\refstepcounter,\relax,\rlap}
% \DoNotIndex{\setlength,\slshape,\space,\string,\strut}
% \DoNotIndex{\the,\trivlist,\typeout}
% \DoNotIndex{\vbox}
% \DoNotIndex{\xdef}
%
% \def\deflabel#1{\ttfamily #1\hfill}
% \def\deflist#1{\begin{list}{}{\settowidth\labelwidth{\ttfamily #1}%
%                               \setlength\leftmargin\labelwidth
%                               \addtolength\leftmargin\labelsep
%                               \let\makelabel\deflabel}}
% \def\enddeflist{\end{list}}
% \def\envfont{\normalfont\ttfamily}
%
% \MakeShortVerb{\|}
%
% \newcommand{\TUB}{{\sl TUGboat\/}}
%
% \GetFileInfo{theorem.dtx}
% \title{An Extension of the
%        \LaTeX{} theorem environment\thanks{This file
%        has version number \fileversion, last
%        revised \filedate.}}
% \author{Frank Mittelbach}
%
% \MaintainedByLaTeXTeam{tools}
% \maketitle
%
% \begin{abstract}
%    The macros described in this paper yield an extension of the
%    \LaTeX{} theorem mechanism. It is designed to satisfy the
%    different requirements of various journals. Thus, the
%    layout of the ``theorems'' can be manipulated by determining a
%    ``style''. This article describes not only the use, but
%    also the definition, of the necessary macros.
% \end{abstract}
%
% \section*{Preface to version 2.2}
%
% For \LaTeXe{} this package did not need any fundamental changes. I
% only modified the messages generated so that theorem layout styles
% will show up with the |\listfiles| command and cleaned the section
% on the New Font Selection Scheme since this is now included in
% \LaTeX.
%
% \section*{Preface to version 2.1}
%
% This version is identical to 2.0g described in \TUB\ 10\#3 except
% for some internal defaults which are now set depending on the used
% font selection scheme.
%
% This was done to avoid unpleasant surprises if the new font selection
% scheme is in force. For further details see
% section~\ref{sec:newfonts} and \cite{bk:GMS94}.
%
% \section{Introduction}
%
% For our purposes here, ``theorems'' are labelled enunciations,
% often set off from the main text by extra space and a font change.
% Theorems, corollaries, conjectures, definitions, and remarks
% are all instances of ``theorems''.  The ``header'' of these structures
% is composed of a label (such as \textsc{Theorem} or \textsc{Remark})
% and a number which serializes an item in the sequence of items
% with the same label.
%
% Shortly after the introduction of \LaTeX{} at the Fachbereich
% Mathematik in Mainz, the desire to manipulate the layout of
% ``theorems'' arose.  In Mainz, the following two conventions came
% into general use:
% \begin{enumerate}
%   \item The number of the theorem is shown in the margin.
%   \item There is a line break at
%         the end of the theorem header.
% \end{enumerate}
%
% Additionally, some journals require different formats which depend on
% the ``sort of theorem'': e.g.\ often remarks and definitions are set
% in |\upshape|, while |\itshape| is employed for main theorems.
%
% Confronted with these requirements, a theorem environment was
% developed in Mainz which allows separate determination of the
% layout of the ``theorems sets'', comparable to |\pagestyle|.
%
%
% \section{The user interface}
% \subsection{Defining new theorem sets}
%
% \DescribeMacro\newtheorem
% As in the original \LaTeX{} version, the command |\newtheorem|
% defines a new ``theorem set'' or ``theorem-like structure''.
% Two required arguments name the new environment and give the text
% to be typeset with each instance of the new ``set'', while
% an optional argument determines how
% the ``set'' is enumerated:
% \begin{description}
%    \item[\ttfamily \bslash newtheorem\{foo\}\{bar\}]
%       The theorem set {\envfont foo} (whose name is \texttt{bar})
%       uses its own counter.
%    \item[\ttfamily \bslash newtheorem\{foo2\}{[foo]}\{bar2\}]
%       The theorem set {\envfont foo2} (printed name \texttt{bar2})
%       uses the same counter as the theorem set \texttt{foo}.
%    \item[\ttfamily \bslash newtheorem\{foo3\}\{bar3\}{[section]}]
%       The theorem set {\envfont foo3} (printed name \texttt{bar3}) is
%       enumerated within the counter \texttt{section}, i.e.\ with every
%       new |\section| the enumeration begins again with 1, and
%       the enumeration is composed from the section-number and the
%       theorem counter itself.
% \end{description}
%
% \DescribeMacro\theoremstyle
% Additionally, the command |\theoremstyle| can define the layout
% of various, or all, theorem sets.  It should be noted that any
% theorem set defined by |\newtheorem| is typeset in the
% |\theoremstyle| that is current at the time of the definition.
% Thus, the following
% \begin{verbatim}
%    \theoremstyle{break}        \newtheorem{Cor}{Corollary}
%    \theoremstyle{plain}        \newtheorem{Exa}{Example}[section]
%\end{verbatim}                             ^^A No blank at line start
% leads to the result that the set {\envfont Cor} is formatted in the
% style \texttt{break}, while the set {\envfont Exa} and all the
% following ones are formatted in the style \texttt{plain}, unless
% another |\theoremstyle| follows.  Since the definitions installed by
% |\newtheorem| are global, one also can limit |\theoremstyle| locally
% by grouping braces.
%
% \DescribeMacro\theorembodyfont
% The choice of the font for the theorem body is completely
% independent of the chosen |\theoremstyle|; this has proven to
% be very advantageous.  For example,
% \begin{verbatim}
%   {\theorembodyfont{\upshape}          \newtheorem{Rem}{Remark}}
%\end{verbatim}
% defines a theorem set {\envfont Rem}, which will be set in
% |\upshape| in the current layout (which in our example is
% \texttt{plain}).  As with |\theoremstyle|, the |\theorembodyfont|
% chosen is that current at the time of |\newtheorem|.  If
% |\theorembodyfont| is not specified or one defines
% |\theorembodyfont{}|, then the font used will be that defined by the
% |\theoremstyle|.
%
% \DescribeMacro\theoremheaderfont It is also possible to customize
% the font used for the theorem headers.  This is, however, a global
% declaration, and therefore there should be at most one
% |\theoremheaderfont| declaration in the preamble.\footnote{If it is
% actually necessary to have different header fonts, one has to define
% new theorem styles (substituting the desired font) or specify the
% information directly in the \texttt{\bslash newtheorem} declaration
% (the unclean variant).}
%
% \DescribeMacro\theorempreskipamount
% \DescribeMacro\theorempostskipamount Two additional parameters
% affect the vertical space around the theorem environments:
% |\theorempreskipamount| and |\theorempostskipamount| define,
% respectively, the spacing before and after such an environment.
% These parameters apply for all theorem sets and can be manipulated
% with the ordinary length macros.  They are rubber lengths
% (`\textsf{skips}'), and therefore can contain \texttt{plus} and
% \texttt{minus} parts.
%
% Since the definition of theorem sets should---most sensibly---be
% placed in the preamble, we only allow installation there.
% It is therefore possible to release the memory used here after
% |\begin{document}|, in order to make room for other
% applications.
%
%
% \subsection{Existing theorem styles}
%
% The following theorem styles exist to date:
% \begin{deflist}{marginbreak}
%    \item[plain]
%       This theorem style emulates the original \LaTeX{} definition,
%       except that additionally the parameters
%       |\theorem...skipamount| are used.
%    \item[break]
%       In this style, the theorem header is followed by a line
%       break.
%    \item[marginbreak]
%       The theorem number is set in the margin, and there is a line
%       break as in \texttt{break}.
%    \item[changebreak]
%       Like \texttt{break}, but with header number and text
%       interchanged.
%    \item[change]
%       Header number and text are interchanged, without a line break.
%    \item[margin]
%       The number is set in the left margin, without a line break.
% \end{deflist}
% All styles (except \texttt{plain}) select |\slshape| as the default
% |\theorembodyfont|.
%
%
% \subsection{Examples}
%
% Given the above theorem sets {\envfont Cor}, {\envfont Exa}
% and {\envfont Rem},
% suppose that the preamble also contains the declarations:
% \begin{verbatim}
%    \theoremstyle{marginbreak}   \newtheorem{Lem}[Cor]{Lemma}
%    \theoremstyle{change}
%    \theorembodyfont{\itshape}        \newtheorem{Def}[Cor]{Definition}
%
%    \theoremheaderfont{\scshape}
%\end{verbatim}
% Then the following are some typical examples of the typeset output
% resulting from their use.
%
% \begin{Cor}
%    This is a sentence typeset in the theorem environment {\envfont
%    Cor}.
% \end{Cor}
%
% \begin{Exa}
%    This is a sentence typeset in the theorem environment {\envfont
%    Exa}.
% \end{Exa}
%
% \begin{Rem}
%    This is a sentence typeset in the theorem environment {\envfont
%    Rem}.
% \end{Rem}
%
% \begin{Lem}[Ben User]
%    This is a sentence typeset in the theorem environment {\envfont
%    Lem}.
% \end{Lem}
%
% \begin{Def}[Very impressive Definition]
%    This is a sentence typeset in the theorem environment {\envfont
%    Def}.
% \end{Def}
%
% The last two examples show the effect of the optional argument to
% a theorem environment (it is the text typeset in parentheses).
%
%
% \section{Special Considerations}
% \label{sec:newfonts}
%
% Theoremheader and body are implemented as a unit. This means that
% the |\theoremheaderfont|  will inherit characteristics of the
% |\theorembodyfont| in \LaTeXe. Thus, if for example
% |\theorembodyfont| is |\itshape| and |\theoremheaderfont|
% is |\bfseries| the font selected for the header will have the
% characteristics `bold extended italic'. If this is not desired one
% should set the |\theoremheaderfont| to something like
% \begin{verbatim}
%      \theoremheaderfont{\normalfont\bfseries}
%\end{verbatim}
% i.e.\ supplying all necessary font information explicitly.
%
%
%
% \section{Acknowledgements}
%
% The publication of this set of macros was only possible with the
% help of Christina Busse (translating the manuscript into English),
% Joachim Pense (playing the r\^ole of typist), Chris Rowley
% (looking everything over) and many others providing useful
% suggestions.
%
% \MaybeStop{%
%    \begin{thebibliography}{9}
%       \bibitem{bk:GMS94} \textsc{M.~Goossens}, \textsc{F.~Mittelbach}
%         and \textsc{A.~Samarin}.
%         \newblock The \LaTeX{} Companion.
%         \newblock
%         Addison-Wesley, Reading, Massachusetts, 1994.
%       \bibitem{latex:source} \textsc{Lamport, Leslie.} \newblock
%                               \texttt{latex.tex}, version 2.09, date
%                               Feb. 1990.
%    \end{thebibliography}
%    \PrintIndex
%    \PrintChanges}
%
%
% \section{The documentation driver file}
%
% The next bit of code contains the documentation driver file for
% \TeX{}, i.e., the file that will produce the documentation you are
% currently reading. It will be extracted from this file by the
% \texttt{docstrip} program. Since it is the first code in the file
% one can alternatively process this file directly with \LaTeXe{} to
% obtain the documentation.
%
%    \begin{macrocode}
%<*driver>
\documentclass{ltxdoc}

\usepackage{theorem}

%<+driver>% The next few lines define theorem sets which are used
%<+driver>% in the example section of the documentation.

\theoremstyle{break}       \newtheorem{Cor}{Corollary}
\theoremstyle{plain}       \newtheorem{Exa}{Example}[section]
{\theorembodyfont{\upshape}\newtheorem{Rem}{Remark}}
\theoremstyle{marginbreak} \newtheorem{Lem}[Cor]{Lemma}
\theoremstyle{change}
\theorembodyfont{\itshape} \newtheorem{Def}[Cor]{Definition}

\theoremheaderfont{\scshape}

\RecordChanges

\begin{document}
   \DocInput{theorem.dtx}
\end{document}
%</driver>
%    \end{macrocode}
%
% \section{Definition of the Macros}
%
% If the file has been loaded before, we abort immediately. If not the
% package announces itself (this is actually done at the very top of
% the file---the way it is done isn't good style so don't copy it).
% \changes{v2.0e}{89/07/19}{Spaces removed from `typeout'.}
% \changes{v2.2b}{95/11/19}{Announce moved to top of file}
%    \begin{macrocode}
%<*package>
%\@ifundefined{theorem@style}{}{\endinput}
%\def\FMithmInfo{1995/11/19 v2.2b Theorem extension package (FMi)}
%\ProvidesPackage{theorem}[\FMithmInfo]
%    \end{macrocode}
%
%
% \subsection{Definition of theorem styles and fonts}
%
% All the definitions in this file are done globally to allow
% inputting this file inside a group.
%
% \begin{macro}{\theoremstyle}
%    Before a theorem style can be installed, the chosen style must be
%    known. For that reason, we must test to see that
%    |\th@|\meta{style} is known or, more precisely, that it is
%    different from |\relax|. If the style is not known then
%    |\th@plain| is used.
%    \begin{macrocode}
\gdef\theoremstyle#1{%
   \@ifundefined{th@#1}{\@warning
          {Unknown theoremstyle `#1'. Using `plain'}%
          \theorem@style{plain}}%
%    \end{macrocode}
%    We save the theorem style to be used in the token register
%    |\theorem@style|.
%    \begin{macrocode}
      {\theorem@style{#1}}%
%    \end{macrocode}
%    Now we ``evaluate'' the theorem style: this means, we call the
%    macro |\th@|\meta{style} which will activate the relevant
%    definitions which are contained in a separate file.  This is done
%    in a group to suppress changes to the current font.  This could
%    otherwise pose problems together with the new font selection
%    scheme\footnote{When I printed the original article using the new
%    font selection scheme I ended with a document with slanted
%    typefaces (text headings and all) simply because one of the
%    theorem styles used \texttt{\bslash sl} at toplevel.} if the
%    |\th@|\meta{style} is evaluated a second time.
% \changes{v2.1a}{90/02/15}{Group added around `th@..}
% \changes{v2.1a}{90/02/15}{Reseting of `..@begintheorem removed.}
%    \begin{macrocode}
      \begingroup
        \csname th@\the\theorem@style \endcsname
      \endgroup}
%    \end{macrocode}
% \end{macro}
%
%
% \begin{macro}{\@begintheorem}
% \changes{v2.1a}{90/02/15}{Relaxed at toplevel.}
% \begin{macro}{\@opargbegintheorem}
% \changes{v2.1a}{90/02/15}{Relaxed at toplevel.}
%    We reset |\@begintheorem| and |\@opargbegintheorem| to
%    |\relax| since these commands are no longer necessary at
%    toplevel. This will save a few tokens.
%    \begin{macrocode}
\global\let\@begintheorem\relax
\global\let\@opargbegintheorem\relax
%    \end{macrocode}
% \end{macro}
% \end{macro}
%
% \begin{macro}{\theorem@style}
%    Obviously the \textsf{token} register used above has to be
%    allocated. To assure the utmost compatibility with the
%    original \LaTeX{} definition, we set the default theorem style to
%    \texttt{plain}, which implements the usual \LaTeX{} convention.
% \changes{v2.0f}{89/07/20}{Unnecessary `global' removed.}
%    \begin{macrocode}
\newtoks\theorem@style
\global\theorem@style{plain}
%    \end{macrocode}
% \end{macro}
%
% \begin{macro}{\theorembodyfont}
% \begin{macro}{\theorem@bodyfont}
%    If the |\theorembodyfont| is set by the user then it should not
%    interact with the default font set in the theorem style. When the
%    new font selection is in force this may happen if, for example,
%    the default is |\itshape| and the new |\theorembodyfont| is
%    |\sffamily|. So we add a |\reset@font| command in front of the
%    user definition.
% \changes{v2.0f}{89/07/20}{Unnecessary `global' removed.}
% \changes{v2.0g}{89/09/19}{`theoremfont to `theorembodyfont.}
% \changes{v2.1c}{92/06/22}{`reset@font added and logic changed.}
% \changes{v2.1d}{92/10/28}{Reset to empty if arg is empty}
%    \begin{macrocode}
\gdef\theorembodyfont#1{%
%    \end{macrocode}
%    We check if the argument supplied is empty and if so put nothing
%    into the |\theorem@bodyfont| token register to allow for
%    |\theorembodyfont{}| as a mean of using the default of the
%    current |\theoremstyle|.
%    \begin{macrocode}
   \def\@tempa{#1}%
   \ifx\@tempa\@empty
    \theorem@bodyfont{}%
   \else
    \theorem@bodyfont{\reset@font#1}%
   \fi
   }
\newtoks\theorem@bodyfont
\global\theorem@bodyfont{}
%    \end{macrocode}
% \end{macro}
% \end{macro}
%
%
% \begin{macro}{\theoremheaderfont}
% \changes{v2.0g}{89/09/19}{Macro added.}
%    The font for the theorem headers is handled differently because
%    this definition applies to all theorem styles.
%    \begin{macrocode}
\gdef\theoremheaderfont#1{\gdef\theorem@headerfont{#1}%
%    \end{macrocode}
%    After using the macro once it is redefined to produce an error
%    message.
%    \begin{macrocode}
       \gdef\theoremheaderfont##1{%
        \typeout{\string\theoremheaderfont\space should be used
                 only once.}}}
%    \end{macrocode}
% \end{macro}
%
% \begin{macro}{\theorem@headerfont}
% \changes{v2.0g}{89/09/19}{Macro added.}
%    To set the |\theorem@headerfont| default we first test if
%    the new fontselection scheme is in force.
% \changes{v2.1a}{90/02/15}{Support for new font selection.}
%    \begin{macrocode}
\ifx\upshape\undefined
%    \end{macrocode}
%    If not we define it to expand into |\bfseries|. We don't use
%    |\let| just in case a following style option redefines this
%    macro.
%    \begin{macrocode}
\gdef\theorem@headerfont{\bfseries}
%    \end{macrocode}
%    Otherwise we reset the current shape before calling |\bfseries|.
%    \begin{macrocode}
\else \gdef\theorem@headerfont{\normalfont\bfseries}\fi
%    \end{macrocode}
% \end{macro}
%
% \begin{macro}{\th@plain}
% \begin{macro}{\th@break}
% \begin{macro}{\th@marginbreak}
% \begin{macro}{\th@changebreak}
% \begin{macro}{\th@change}
% \begin{macro}{\th@margin}
%    The different styles are defined in macros such as
%    |\th@plain|.  Since memory space is precious in
%    ``non-Big-versions'', we have to avoid offering too many unused
%    definitions.  Therefore we define these styles in separate files
%    that can be loaded on demand.  Thus the commands themselves only
%    load these files. We use |\@input@|, a \LaTeXe{} internal command
%    that ensures that the file will be listed with |\listfiles|.
% \changes{v2.2a}{94/02/02}{Use `@input@ to load theorem layout files}
%    \begin{macrocode}
\gdef\th@plain{\@input@{thp.sty}}
\gdef\th@break{\@input@{thb.sty}}
\gdef\th@marginbreak{\@input@{thmb.sty}}
\gdef\th@changebreak{\@input@{thcb.sty}}
\gdef\th@change{\@input@{thc.sty}}
\gdef\th@margin{\@input@{thm.sty}}
%    \end{macrocode}
%    This list will be expanded when new styles become
%    available.  For testing,  just append new theorem substyles
%    as document options.
% \end{macro}
% \end{macro}
% \end{macro}
% \end{macro}
% \end{macro}
% \end{macro}
%
%
%
% \subsection{Definition of a new theorem set}
%
% As already pointed out, a new theorem environment can be defined in
% three different ways:
% \begin{verbatim}
%    \newtheorem{Lem}{Lemma}
%    \newtheorem{Lem}{Lemma}[section]
%    \newtheorem{Lem}[Theorem]{Lemma}
%\end{verbatim}
% The function of the macro |\newtheorem| is to recognize these cases
% and then to branch into one of the three macros |\@ynthm|, |\@xnthm|
% or |\@othm|.  This mechanism is adopted unchanged from
% \cite{latex:source}; the essential point here is that, for example,
% in the second case, the arguments \texttt{Lem}, \texttt{Lemma} and
% \texttt{section} are passed over to the macro |\@xnthm|.
%
% We inspect this case first because the others present fewer problems,
% and thus are easily derived from this one.
%
% \begin{macro}{\@xnthm}
%    For our example arguments, the macro |\@xnthm| must fulfill
%    the following:
%    \begin{itemize}
%       \item
%          Define a new \LaTeX-counter `Lem'
%       \item
%          reset this counter within a |\section|
%       \item
%          define the macro |\theLem|
%       \item
%          define the environment macros |\Lem| and
%          |\endLem| using the current
%          |\theoremstyle| and |\theorem@bodyfont|.
%    \end{itemize}
%    Obviously, all this should happen only if the first argument
%    of |\@xnthm| (i.e.\ \texttt{Lem} in our example) is chosen so
%    as not to conflict with any previously defined commands or
%    environments.  This test is performed by the \LaTeX{} macro
%    |\@ifdefinable|.
%    \begin{macrocode}
\gdef\@xnthm#1#2[#3]{\expandafter\@ifdefinable\csname #1\endcsname
%    \end{macrocode}
%    Therefore, the first argument of |\@ifdefinable| is the
%    expansion (in the example, |\Lem|) of
%    |\csname||#1\endcsname|.  The second argument is
%    executed only if the test has been completed successfully.
%    \begin{macrocode}
   {%
%    \end{macrocode}
%    Now we define the new counter. The names of the \LaTeX{}
%    macros employed should speak for themselves:
% \changes{v2.2b}{1995/10/07}{Use \cs{@newctr}}
% \changes{v2.2c}{1995/11/23}{Use \cs{@newctr} properly.}
%    \begin{macrocode}
    \@definecounter{#1}\@newctr{#1}[#3]%
%    \end{macrocode}
%    Using |\@newctr| will give a proper error message if the counter
%    in |#3| is not defined.
%    In defining `|\theLem|' we must generate the desired macro
%    name by use of |\expandafter| and |\csname|.
%    \begin{macrocode}
    \expandafter\xdef\csname the#1\endcsname
%    \end{macrocode}
%    An |\xdef| is used in order to make the definition global,
%    and to ensure that it contains the replacement texts of
%    |\@thmcountersep| and |\@thmcounter|.\footnote{These
%    two macros can be defined by the document style.  Their default
%    values produce a `\texttt{.}' as separation and an arabic
%    representation of the number.} However, not everything should be
%    expanded. For example, it saves space to use |\thesection|
%    instead of its---at times---lengthy expansion.
%    \begin{macrocode}
      {\expandafter \noexpand \csname the#3\endcsname
       \@thmcountersep \@thmcounter{#1}}%
%    \end{macrocode}
%    Thus with the defaults of \LaTeX{}, |\theLem| would be
%    replaced by the command sequence |\thesection.\arabic{Lem}|.
%
%    We will now look at the definition of the macro which is executed
%    at the beginning of the actual environment (in our example this
%    macro is |\Lem|). It should be noted that we use an
%    ``|\expandafter| trick'' to expand only certain parts of the
%    replacement text at the time of the definition.
%    \begin{macrocode}
    \def\@tempa{\global\@namedef{#1}}%
    \expandafter \@tempa \expandafter{%
%    \end{macrocode}
%    First, the macro that contains the current definitions of
%    |\@begintheorem| and |\@opargtheorem| should be called
%    up.  The name of this macro---as is already known---has the form
%    |\th@|\meta{theorem\ style}; therefore, it must be called by
%    \begin{macrocode}
      \csname th@\the \theorem@style
%    \end{macrocode}
%    In addition the default theorem font should be changeable, i.e.\
%    we have to insert the contents of |\theorem@bodyfont|. For
%    that reason, we expand even further, beyond |\endcsname|,
%    and thus insert the contents of the \textsf{token} register
%    |\theorem@bodyfont| in the replacement text.
%    \begin{macrocode}
            \expandafter \endcsname \the \theorem@bodyfont
%    \end{macrocode}
%    Now it is time to call the macro |\@thm| which takes over
%    the further processing.  It has two arguments: the current
%    counter name (in our example, \texttt{Lem}), and the text of the
%    label (in our example, \texttt{Lemma}).
%    \begin{macrocode}
     \@thm{#1}{#2}}%
%    \end{macrocode}
%    With this, the `sub-definition' is complete.  The macro
%    |\@endtheorem| ends a theorem environment and is, so far, nothing
%    but an |\endtrivlist|.  (Hence it is defined globally, and not
%    within the theorem styles.\footnote{This has to be changed as
%    soon as theorem styles that change \texttt{\bslash @endtheorem}
%    exist.  In such a case, all existing styles must be changed as
%    well since they will have to reset the macro.}) Therefore, we can
%    set it equivalent to the macro that ends the theorem set (in our
%    example, |\endLem|).  However, if some day theorem styles exist
%    that do change |\@endtheorem|, we would have to use the
%    commented-out line instead.
%    \begin{macrocode}
    \global \expandafter \let \csname end#1\endcsname \@endtheorem
%   \global\@namedef{end#1}{\@endtheorem}%
%    \end{macrocode}
%    With these commands all the required definitions are employed,
%    unless the test |\@ifdefinable| has failed.  Therefore, we
%    end the second argument of this macro and with it the definition
%    of |\@xnthm|.
%    \begin{macrocode}
   }}
%    \end{macrocode}
% \end{macro}
%
% \begin{macro}{\@ynthm}
%    The definition of |\@ynthm| is completely analogous. In
%    this case the new counter that is defined is not reset within
%    another counter; thus the definition of |\the...|
%    is simplified:
%    \begin{macrocode}
\gdef\@ynthm#1#2{\expandafter\@ifdefinable\csname #1\endcsname
   {\@definecounter{#1}%
    \expandafter\xdef\csname the#1\endcsname{\@thmcounter{#1}}%
%    \end{macrocode}
%    The rest of the definition corresponds literally to that of
%    |\@xnthm|:
%    \begin{macrocode}
    \def\@tempa{\global\@namedef{#1}}\expandafter \@tempa
     \expandafter{\csname th@\the \theorem@style \expandafter
     \endcsname \the\theorem@bodyfont \@thm{#1}{#2}}%
    \global \expandafter \let \csname end#1\endcsname \@endtheorem}}
%    \end{macrocode}
% \end{macro}
%
% \begin{macro}{\@othm}
%    The definition of |\@othm| does not contain anything new.
%    \begin{macrocode}
\gdef\@othm#1[#2]#3{%
%    \end{macrocode}
% \changes{v2.2b}{95/10/07}{Build in check for counter defined}
%    We do not define a new counter but instead use one that has
%    already been defined.  Thus the only definition we need is that
%    of this pseudo-counter (i.e.\ |\the|\meta{env.\ name}).
%    First we check if |#2| corresponds to a known counter name.
%    \begin{macrocode}
  \expandafter\ifx\csname c@#2\endcsname\relax
   \@nocounterr{#2}%
  \else
   \expandafter\@ifdefinable\csname #1\endcsname
   {\expandafter \xdef \csname the#1\endcsname
     {\expandafter \noexpand \csname the#2\endcsname}%
%    \end{macrocode}
%    All other parts of the definition can be adopted from
%    |\@xnthm|. We have to remember, though, that in this case
%    the name of the current counter and the theorem label have moved
%    to the second and third arguments.
%    \begin{macrocode}
    \def\@tempa{\global\@namedef{#1}}\expandafter \@tempa
     \expandafter{\csname th@\the \theorem@style \expandafter
     \endcsname \the\theorem@bodyfont \@thm{#2}{#3}}%
    \global \expandafter \let \csname end#1\endcsname \@endtheorem}%
  \fi}
%    \end{macrocode}
% \end{macro}
%
% \subsection{Macros that are employed in a {\envfont theorem}
%    environment}
%
% \begin{macro}{\@thm}
%    The macro |\@thm| has to increase the current counter.  Then,
%    depending on whether the environment has (or does not have) an
%    optional argument, it has to branch into either |\@begintheorem|
%    or |\@opargtheorem|.
%    \begin{macrocode}
\gdef\@thm#1#2{\refstepcounter{#1}%
%    \end{macrocode}
%    Now we start a {\envfont trivlist} environment, and give
%    |\@topsep| and |\@topsepadd| the values of the \textsf{skip}
%    registers |\theorempreskipamount| and |\theorempostskipamount|.
%    The value in |\@topsep| is the vertical space that is inserted by
%    the first (and only) |\item| in our |\trivlist| whilst
%    |\@topsepadd| is inserted by |\@endparenv| at the end of that
%    {\envfont trivlist} environment.  By using these registers, we
%    obtain the desired space around a {\envfont theorem} environment.
%    \begin{macrocode}
   \trivlist
   \@topsep \theorempreskipamount               % used by first \item
   \@topsepadd \theorempostskipamount           % used by \@endparenv
%    \end{macrocode}
%    Now we have to test whether an optional argument has been given.
% \changes{v2.0d}{89/07/18}{Forgotten percent added.}
%    \begin{macrocode}
   \@ifnextchar [%
%    \end{macrocode}
%    If there is an optional argument, we will call |\@ythm|, and
%    move the arguments read back into the input stream.
%    \begin{macrocode}
   {\@ythm{#1}{#2}}%
%    \end{macrocode}
%    If not, we call |\@begintheorem|. Its first argument is the
%    name of the theorem set (hence the second argument of
%    |\@thm|). Its second argument is the macro that produces the
%    current number.
%    \begin{macrocode}
   {\@begintheorem{#2}{\csname the#1\endcsname}\ignorespaces}}
%    \end{macrocode}
% \end{macro}
%
% \begin{macro}{\@xthm}
% \begin{macro}{\@ythm}
%    Both these macros were originally called by |\@thm|.  We do
%    not need |\@xthm| anymore, hence we reset it to
%    |\relax|. The definition of |\@ythm| has not changed at
%    all from its definition in \LaTeX. In order to make the macros
%    easier to understand, we will
%    nevertheless present it (commented out).
%    \begin{macrocode}
\global\let\@xthm\relax
% \def\@ythm#1#2[#3]{\@opargbegintheorem{#2}{\csname
%           the#1\endcsname}{#3}\ignorespaces}
%</package>
%    \end{macrocode}
%    The primitive |\ignorespaces| in |\@ythm| and
%    |\@thm| is needed to remove the spaces between the
%    |\begin{...}| and the actual text.
% \end{macro}
% \end{macro}
%
%
% \subsection{Definition of the theorem substyles}
%
% As already pointed out, the theorem substyles, defined below, are
% only loaded when necessary.  Note that all these substyles, except
% \texttt{plain}, have |\slshape| as the default body font.
%
% \subsubsection{The \texttt{plain} style}
%
% \label{dok:th@plain} As the following macros use |@|, we have
% to locally set the |\catcode| of this symbol to ``letter''.
% This happens within a group, so that we do not have to worry about
% which |\catcode| that symbol had before.
%    \begin{macrocode}
%<*thp>
\begingroup \makeatletter
%    \end{macrocode}
% Since we are now within a group, we must make all definitions
% globally.  First we make sure that \texttt{theorem.sty} is loaded.
% This will allow us to use this file as a document style option
% without having to call \texttt{theorem} itself as an option.  At the
% same time, we assure that at least version 2 is loaded, since
% |\theorem@style| was not defined in earlier versions.
%    \begin{macrocode}
\@ifundefined{theorem@style}{\input{theorem.sty}}{}
\ProvidesFile{thp.sty}
             [\FMithmInfo]
%    \end{macrocode}
%
%
% \begin{macro}{\th@plain}
% \begin{macro}{\@begintheorem}
% \begin{macro}{\@opargbegintheorem}
%    |\theoremstyle{plain}| corresponds to the original
%    definition, except that the distances to the surrounding text are
%    determined by the parameters |\theorempreskipamount| and
%    |\theorempostskipamount|.
%    First we set the default body font.
%    \begin{macrocode}
\gdef\th@plain{\normalfont\itshape
%    \end{macrocode}
%    Then we define |\@begintheorem| and
%    |\@opargbegintheorem|. These two macros define how the
%    header of a theorem is typeset.  |\@opargbegintheorem| will
%    be called if a \textsf{theorem} environment with an optional
%    argument is encountered; otherwise, the header is constructed by
%    calling |\@begintheorem|.  If one of these macros is
%    executed, we are within a \textsf{trivlist} environment started by
%    |\@thm|.  So the theorem header is produced with an
%    |\item| command.
%
%    Instead of specifying the header font directly, all standard
%    theorem styles use the |\theorem@headerfont| macro to allow
%    customization.  The extra space (|\labelsep|) is necessary
%    because of problems in the \textsf{trivlist} environment.
%    \begin{macrocode}
  \def\@begintheorem##1##2{%
        \item[\hskip\labelsep \theorem@headerfont ##1\ ##2]}%
%    \end{macrocode}
%    The definition of |\@opargbegintheorem| is completely
%    analogous. The only difference is the fact that there exists a
%    third argument (which is the optional parameter of the
%    environment and contains additional information about the
%    theorem).  Customarily we enclose it in parentheses.
%    \begin{macrocode}
\def\@opargbegintheorem##1##2##3{%
   \item[\hskip\labelsep \theorem@headerfont ##1\ ##2\ (##3)]}}
%    \end{macrocode}
% \end{macro}
% \end{macro}
% \end{macro}
%
% \noindent
% We conclude with an |\endgroup| to restore the
% |\catcode| of |@|.
%    \begin{macrocode}
\endgroup
%</thp>
%    \end{macrocode}
%
%
% \subsubsection{The \texttt{break} style}
%
% This style option is stored in the file |thb.sty|.
% For the next two lines see the documentation for |\th@plain| on
% page~\pageref{dok:th@plain}.
%    \begin{macrocode}
%<*thb>
\begingroup \makeatletter
\@ifundefined{theorem@style}{\input{theorem.sty}}{}
\ProvidesFile{thb.sty}
             [\FMithmInfo]
%    \end{macrocode}
%
% \begin{macro}{\th@break}
%    |\theoremstyle{break}| produces a line break after the name
%    of the theorem.  The font is |\slshape|.  Hence, we define
%    |\th@break| as follows:
%    \begin{macrocode}
\gdef\th@break{\normalfont\slshape
  \def\@begintheorem##1##2{\item[%
%    \end{macrocode}
%    We run into the following problem: it is not possible to create
%    the header with |\item[|\meta{title}|]| and then start
%    a new line by, for example, |\mbox{}\\|. Such a definition
%    will fail whenever a list environment follows immediately.  With
%    the above construction, the |\mbox{}| causes the switch
%    |@inlabel| (cf.\ definition of |\list| and
%    |\trivlist| in \cite{latex:source}) to be set to \texttt{false}
%    and so the following list will insert additional vertical
%    space (|\topskip|). This is quite annoying. Therefore, we
%    create the line break within the |\item|. In order to ensure
%    that the text will begin at the proper position in the following
%    line, we simply pretend that the label does not take any
%    room.\footnote{This will lead to problems whenever very high
%    symbols occurring in the line tower into the
%    heading. So, something else has to be done here sometime.}
%    \begin{macrocode}
       \rlap{\vbox{\hbox{\hskip \labelsep\theorem@headerfont ##1\ ##2}%
                   \hbox{\strut}}}]}%
%    \end{macrocode}
%    Again, the definition of |\@opargbegintheorem| is
%    completely analogous.
%    \begin{macrocode}
\def\@opargbegintheorem##1##2##3{%
  \item[\rlap{\vbox{\hbox{\hskip \labelsep \theorem@headerfont
                     ##1\ ##2\ (##3)}%
                    \hbox{\strut}}}]}}
\endgroup
%</thb>
%    \end{macrocode}
% \end{macro}
%
%
% \subsubsection{The \texttt{changebreak} style}
%
%    \begin{macrocode}
% This style option is stored in the file |thcb.sty|.
%    \begin{macrocode}
%<*thcb>
\begingroup \makeatletter
\@ifundefined{theorem@style}{\input{theorem.sty}}{}
\ProvidesFile{thcb.sty}
             [\FMithmInfo]
%    \end{macrocode}
%
% \begin{macro}{\th@changebreak}
%    The \texttt{change-break} theorem style is like \texttt{break}
%    but with interchange of theorem name and theorem number.  Thus we
%    define |\th@changebreak| as follows:
%    \begin{macrocode}
\gdef\th@changebreak{\normalfont\slshape
  \def\@begintheorem##1##2{\item
       [\rlap{\vbox{\hbox{\hskip\labelsep \theorem@headerfont ##2\ ##1}%
                    \hbox{\strut}}}]}%
\def\@opargbegintheorem##1##2##3{%
  \item[\rlap{\vbox{\hbox{\hskip\labelsep \theorem@headerfont
                     ##2\ ##1\ (##3)}%
                    \hbox{\strut}}}]}}
\endgroup
%</thcb>
%    \end{macrocode}
% \end{macro}
%
%
% \subsubsection{The \texttt{change} style}
%
% This style option is stored in the file |thc.sty|.
%    \begin{macrocode}
%<*thc>
\begingroup \makeatletter
\@ifundefined{theorem@style}{\input{theorem.sty}}{}
\ProvidesFile{thc.sty}
             [\FMithmInfo]
%    \end{macrocode}
%
% \begin{macro}{\th@change}
%    The \texttt{change} theorem style corresponds to the
%    \texttt{change break} style without a linebreak after the header.
%    To say it in another way, it's the same as the \texttt{plain}
%    style but with number and name interchanged and |\slshape| as the
%    default font.
%    \begin{macrocode}
\gdef\th@change{\normalfont\slshape
  \def\@begintheorem##1##2{\item
       [\hskip\labelsep \theorem@headerfont ##2\ ##1]}%
\def\@opargbegintheorem##1##2##3{%
  \item[\hskip\labelsep \theorem@headerfont ##2\ ##1\ (##3)]}}
\endgroup
%</thc>
%    \end{macrocode}
% \end{macro}
%
%
% \subsubsection{The \texttt{marginbreak} style}
%
% This style option is the one used most often at Mainz.  It is saved
% in the file |thmb.sty|.
%    \begin{macrocode}
%<*thmb>
\begingroup \makeatletter
\@ifundefined{theorem@style}{\input{theorem.sty}}{}
\ProvidesFile{thmb.sty}
             [\FMithmInfo]
%    \end{macrocode}
%
% \begin{macro}{\th@marginbreak}
%    The \texttt{margin break} style is nearly the same as the
%    \texttt{change break} style. The only difference is the placement
%    of the theorem number.  We use |\llap| to place it in the left
%    margin.
%
%    In this style |\labelsep| denotes the separation between the
%    number and the text.
%    \begin{macrocode}
\gdef\th@marginbreak{\normalfont\slshape
  \def\@begintheorem##1##2{\item
       [\rlap{\vbox{\theorem@headerfont
                    \hbox{\llap{##2}\hskip\labelsep ##1}%
                    \hbox{\strut}}}]}%
\def\@opargbegintheorem##1##2##3{%
  \item[\rlap{\vbox{\theorem@headerfont
                    \hbox{\llap{##2}\hskip\labelsep ##1\ (##3)}%
                    \hbox{\strut}}}]}}
\endgroup
%</thmb>
%    \end{macrocode}
% \changes{v2.1c}{90/02/17}{`endgroup outside macrocode env.}
% \end{macro}
%
%
%
% \subsubsection{The \texttt{margin} style}
%
% This style option is stored in the file |thm.sty|.
%    \begin{macrocode}
%<*thm>
\begingroup \makeatletter
\@ifundefined{theorem@style}{\input{theorem.sty}}{}
\ProvidesFile{thm.sty}
             [\FMithmInfo]
%    \end{macrocode}
%
% \begin{macro}{\th@margin}
%    Again this is only a variant of the theorem styles described
%    above without any new ideas.
%    \begin{macrocode}
\gdef\th@margin{\normalfont\slshape
  \def\@begintheorem##1##2{\item
       [\theorem@headerfont \llap{##2}\hskip\labelsep ##1]}%
\def\@opargbegintheorem##1##2##3{%
  \item[\theorem@headerfont \llap{##2}\hskip\labelsep ##1\ (##3)]}}
\endgroup
%</thm>
%    \end{macrocode}
% \end{macro}
%
%
%
%
% \subsection{Final Definitions}
%
% \begin{macro}{\theorempreskipamount}
% \begin{macro}{\theorempostskipamount}
%    The \textsf{skip} parameters that regulate the vertical empty space
%    before and after the {\envfont theorem} environment have
%    to be allocated as well.
% \changes{v2.0f}{89/07/20}{Unnecessary `global' removed.}
%    \begin{macrocode}
%<*package>
\newskip\theorempreskipamount
\newskip\theorempostskipamount
%    \end{macrocode}
%    Since we have used the same values for all theorem sets, we now can
%    assign them.
%    \begin{macrocode}
\global\setlength\theorempreskipamount{12pt plus 5pt minus 3pt}
\global\setlength\theorempostskipamount{8pt plus 3pt minus 1.5pt}
%    \end{macrocode}
% \end{macro}
% \end{macro}
%
%
% \begin{macro}{\@endtheorem}
%    The same holds for the macro |\@endtheorem|, which ends a
%    {\envfont theorem} environment. Since it is the same for all
%    theorem sets, it is removed from the macros |\th@|\meta{style}.
%    It simply ends the {\envfont trivlist} environment, which was
%    begun in |\@thm|.
%    \begin{macrocode}
\global\let\@endtheorem=\endtrivlist
%    \end{macrocode}
% \end{macro}
%
%
%
% \begin{macro}{\@preamblecmds}
%    All macros defined above are to be used only in the preamble.
%    Therefore, we insert them in |\@preamblecmds| which will disable
%    them at begin document. This is done by the internal \LaTeXe{}
%    command |\@onlypreamble|.
% \changes{v2.0g}{89/09/19}{`theoremheaderfont and `theorembodyfont
%    added.}
% \changes{v2.2a}{94/02/03}{Use \LaTeXe{} interface}
%    \begin{macrocode}
\@onlypreamble\@xnthm
\@onlypreamble\@ynthm
\@onlypreamble\@othm
\@onlypreamble\newtheorem
\@onlypreamble\theoremstyle
\@onlypreamble\theorembodyfont
\@onlypreamble\theoremheaderfont
%    \end{macrocode}
% \end{macro}
%
% Finally we declare the \texttt{plain} theorem style to be the default.
% \changes{v2.1b}{90/06/04}{Fixed missing `item by declaring default
%    style.}
%    \begin{macrocode}
\theoremstyle{plain}
%</package>
%    \end{macrocode}
%
% \Finale

\endinput
}{}
\ProvidesFile{thmb.sty}
             [\FMithmInfo]
%    \end{macrocode}
%
% \begin{macro}{\th@marginbreak}
%    The \texttt{margin break} style is nearly the same as the
%    \texttt{change break} style. The only difference is the placement
%    of the theorem number.  We use |\llap| to place it in the left
%    margin.
%
%    In this style |\labelsep| denotes the separation between the
%    number and the text.
%    \begin{macrocode}
\gdef\th@marginbreak{\normalfont\slshape
  \def\@begintheorem##1##2{\item
       [\rlap{\vbox{\theorem@headerfont
                    \hbox{\llap{##2}\hskip\labelsep ##1}%
                    \hbox{\strut}}}]}%
\def\@opargbegintheorem##1##2##3{%
  \item[\rlap{\vbox{\theorem@headerfont
                    \hbox{\llap{##2}\hskip\labelsep ##1\ (##3)}%
                    \hbox{\strut}}}]}}
\endgroup
%</thmb>
%    \end{macrocode}
% \changes{v2.1c}{90/02/17}{`endgroup outside macrocode env.}
% \end{macro}
%
%
%
% \subsubsection{The \texttt{margin} style}
%
% This style option is stored in the file |thm.sty|.
%    \begin{macrocode}
%<*thm>
\begingroup \makeatletter
\@ifundefined{theorem@style}{% \iffalse meta-comment
%
% Copyright (C) 1993-2024
%
% The LaTeX Project and any individual authors listed elsewhere
% in this file.
%
% This file is part of the Standard LaTeX `Tools Bundle'.
% -------------------------------------------------------
%
% It may be distributed and/or modified under the
% conditions of the LaTeX Project Public License, either version 1.3c
% of this license or (at your option) any later version.
% The latest version of this license is in
%    https://www.latex-project.org/lppl.txt
% and version 1.3c or later is part of all distributions of LaTeX
% version 2005/12/01 or later.
%
% The list of all files belonging to the LaTeX `Tools Bundle' is
% given in the file `manifest.txt'.
%
% \fi
%
%%
%
%
\def\FMithmInfo{2023/07/05 v2.2c Theorem extension package (FMi)}
%
% \ProvidesFile{theorem.dtx}[\FMithmInfo]
%\iffalse   % this is a METACOMMENT !
%%
%% Package `theorem' to use with LaTeX2e
%% Copyright (C) 1989-1995 Frank Mittelbach, all rights reserved.
%<+package>\@ifundefined{theorem@style}{}{\endinput}
%<+package>\NeedsTeXFormat{LaTeX2e}[1995/06/01]
%<+package>\ProvidesPackage{theorem}
%<+package>         [\FMithmInfo]
%
% \fi
%
% \changes{v1.0}{87/05/07}{First draft.}
% \changes{v1.0a}{87/05/07}{`thistheoremstyle' added.}
% \changes{v1.1a}{87/07/29}{New concept: redefine the internal
%                          \protect\LaTeX{} commands `@xnthm' etc.}
% \changes{v1.1b}{87/07/31}{`theoremfont' used to specify font.}
% \changes{v1.2a}{88/01/08}{`@endtheorem' now globally defined.}
% \changes{v1.2b}{88/01/11}{`newtheorem' only allowed in the preamble.}
% \changes{v1.3a}{88/02/14}{`theoremfont' now a toks register.}
%
% \changes{v2.0a}{88/11/16}{New implementation and new documentation.}
% \changes{v2.0a}{88/11/16}{No layout changes but internal modifications
%                           old substyles won't work.}
% \changes{v2.0b}{89/01/30}{`global' used for all definitions.}
% \changes{v2.0c}{89/06/15}{Documentation translated to English.
%                            Thanks to LLL and Nina}
% \changes{v2.1a}{90/02/15}{Support for new font selection.}
% \changes{v2.2a}{94/02/03}{Cleanup for LaTeX2e}
%
% \DoNotIndex{\@endparenv,\@ifdefinable,\@ifnextchar,\@ifundefined}
% \DoNotIndex{\@namedef,\@nameuse,\@spaces,\@tempa,\@warning}
% \DoNotIndex{\ }
% \DoNotIndex{\begingroup,\bfseries}
% \DoNotIndex{\csname}
% \DoNotIndex{\def,\do,\docdate}
% \DoNotIndex{\endcsname,\endgroup,\endinput,\endtrivlist,\expandafter}
% \DoNotIndex{\filedate,\fileversion}
% \DoNotIndex{\gdef,\global}
% \DoNotIndex{\hbox,\hskip}
% \DoNotIndex{\ignorespaces,\input,\itshape,\item}
% \DoNotIndex{\labelsep,\let,\llap}
% \DoNotIndex{\makeatletter}
% \DoNotIndex{\noexpand}
% \DoNotIndex{\refstepcounter,\relax,\rlap}
% \DoNotIndex{\setlength,\slshape,\space,\string,\strut}
% \DoNotIndex{\the,\trivlist,\typeout}
% \DoNotIndex{\vbox}
% \DoNotIndex{\xdef}
%
% \def\deflabel#1{\ttfamily #1\hfill}
% \def\deflist#1{\begin{list}{}{\settowidth\labelwidth{\ttfamily #1}%
%                               \setlength\leftmargin\labelwidth
%                               \addtolength\leftmargin\labelsep
%                               \let\makelabel\deflabel}}
% \def\enddeflist{\end{list}}
% \def\envfont{\normalfont\ttfamily}
%
% \MakeShortVerb{\|}
%
% \newcommand{\TUB}{{\sl TUGboat\/}}
%
% \GetFileInfo{theorem.dtx}
% \title{An Extension of the
%        \LaTeX{} theorem environment\thanks{This file
%        has version number \fileversion, last
%        revised \filedate.}}
% \author{Frank Mittelbach}
%
% \MaintainedByLaTeXTeam{tools}
% \maketitle
%
% \begin{abstract}
%    The macros described in this paper yield an extension of the
%    \LaTeX{} theorem mechanism. It is designed to satisfy the
%    different requirements of various journals. Thus, the
%    layout of the ``theorems'' can be manipulated by determining a
%    ``style''. This article describes not only the use, but
%    also the definition, of the necessary macros.
% \end{abstract}
%
% \section*{Preface to version 2.2}
%
% For \LaTeXe{} this package did not need any fundamental changes. I
% only modified the messages generated so that theorem layout styles
% will show up with the |\listfiles| command and cleaned the section
% on the New Font Selection Scheme since this is now included in
% \LaTeX.
%
% \section*{Preface to version 2.1}
%
% This version is identical to 2.0g described in \TUB\ 10\#3 except
% for some internal defaults which are now set depending on the used
% font selection scheme.
%
% This was done to avoid unpleasant surprises if the new font selection
% scheme is in force. For further details see
% section~\ref{sec:newfonts} and \cite{bk:GMS94}.
%
% \section{Introduction}
%
% For our purposes here, ``theorems'' are labelled enunciations,
% often set off from the main text by extra space and a font change.
% Theorems, corollaries, conjectures, definitions, and remarks
% are all instances of ``theorems''.  The ``header'' of these structures
% is composed of a label (such as \textsc{Theorem} or \textsc{Remark})
% and a number which serializes an item in the sequence of items
% with the same label.
%
% Shortly after the introduction of \LaTeX{} at the Fachbereich
% Mathematik in Mainz, the desire to manipulate the layout of
% ``theorems'' arose.  In Mainz, the following two conventions came
% into general use:
% \begin{enumerate}
%   \item The number of the theorem is shown in the margin.
%   \item There is a line break at
%         the end of the theorem header.
% \end{enumerate}
%
% Additionally, some journals require different formats which depend on
% the ``sort of theorem'': e.g.\ often remarks and definitions are set
% in |\upshape|, while |\itshape| is employed for main theorems.
%
% Confronted with these requirements, a theorem environment was
% developed in Mainz which allows separate determination of the
% layout of the ``theorems sets'', comparable to |\pagestyle|.
%
%
% \section{The user interface}
% \subsection{Defining new theorem sets}
%
% \DescribeMacro\newtheorem
% As in the original \LaTeX{} version, the command |\newtheorem|
% defines a new ``theorem set'' or ``theorem-like structure''.
% Two required arguments name the new environment and give the text
% to be typeset with each instance of the new ``set'', while
% an optional argument determines how
% the ``set'' is enumerated:
% \begin{description}
%    \item[\ttfamily \bslash newtheorem\{foo\}\{bar\}]
%       The theorem set {\envfont foo} (whose name is \texttt{bar})
%       uses its own counter.
%    \item[\ttfamily \bslash newtheorem\{foo2\}{[foo]}\{bar2\}]
%       The theorem set {\envfont foo2} (printed name \texttt{bar2})
%       uses the same counter as the theorem set \texttt{foo}.
%    \item[\ttfamily \bslash newtheorem\{foo3\}\{bar3\}{[section]}]
%       The theorem set {\envfont foo3} (printed name \texttt{bar3}) is
%       enumerated within the counter \texttt{section}, i.e.\ with every
%       new |\section| the enumeration begins again with 1, and
%       the enumeration is composed from the section-number and the
%       theorem counter itself.
% \end{description}
%
% \DescribeMacro\theoremstyle
% Additionally, the command |\theoremstyle| can define the layout
% of various, or all, theorem sets.  It should be noted that any
% theorem set defined by |\newtheorem| is typeset in the
% |\theoremstyle| that is current at the time of the definition.
% Thus, the following
% \begin{verbatim}
%    \theoremstyle{break}        \newtheorem{Cor}{Corollary}
%    \theoremstyle{plain}        \newtheorem{Exa}{Example}[section]
%\end{verbatim}                             ^^A No blank at line start
% leads to the result that the set {\envfont Cor} is formatted in the
% style \texttt{break}, while the set {\envfont Exa} and all the
% following ones are formatted in the style \texttt{plain}, unless
% another |\theoremstyle| follows.  Since the definitions installed by
% |\newtheorem| are global, one also can limit |\theoremstyle| locally
% by grouping braces.
%
% \DescribeMacro\theorembodyfont
% The choice of the font for the theorem body is completely
% independent of the chosen |\theoremstyle|; this has proven to
% be very advantageous.  For example,
% \begin{verbatim}
%   {\theorembodyfont{\upshape}          \newtheorem{Rem}{Remark}}
%\end{verbatim}
% defines a theorem set {\envfont Rem}, which will be set in
% |\upshape| in the current layout (which in our example is
% \texttt{plain}).  As with |\theoremstyle|, the |\theorembodyfont|
% chosen is that current at the time of |\newtheorem|.  If
% |\theorembodyfont| is not specified or one defines
% |\theorembodyfont{}|, then the font used will be that defined by the
% |\theoremstyle|.
%
% \DescribeMacro\theoremheaderfont It is also possible to customize
% the font used for the theorem headers.  This is, however, a global
% declaration, and therefore there should be at most one
% |\theoremheaderfont| declaration in the preamble.\footnote{If it is
% actually necessary to have different header fonts, one has to define
% new theorem styles (substituting the desired font) or specify the
% information directly in the \texttt{\bslash newtheorem} declaration
% (the unclean variant).}
%
% \DescribeMacro\theorempreskipamount
% \DescribeMacro\theorempostskipamount Two additional parameters
% affect the vertical space around the theorem environments:
% |\theorempreskipamount| and |\theorempostskipamount| define,
% respectively, the spacing before and after such an environment.
% These parameters apply for all theorem sets and can be manipulated
% with the ordinary length macros.  They are rubber lengths
% (`\textsf{skips}'), and therefore can contain \texttt{plus} and
% \texttt{minus} parts.
%
% Since the definition of theorem sets should---most sensibly---be
% placed in the preamble, we only allow installation there.
% It is therefore possible to release the memory used here after
% |\begin{document}|, in order to make room for other
% applications.
%
%
% \subsection{Existing theorem styles}
%
% The following theorem styles exist to date:
% \begin{deflist}{marginbreak}
%    \item[plain]
%       This theorem style emulates the original \LaTeX{} definition,
%       except that additionally the parameters
%       |\theorem...skipamount| are used.
%    \item[break]
%       In this style, the theorem header is followed by a line
%       break.
%    \item[marginbreak]
%       The theorem number is set in the margin, and there is a line
%       break as in \texttt{break}.
%    \item[changebreak]
%       Like \texttt{break}, but with header number and text
%       interchanged.
%    \item[change]
%       Header number and text are interchanged, without a line break.
%    \item[margin]
%       The number is set in the left margin, without a line break.
% \end{deflist}
% All styles (except \texttt{plain}) select |\slshape| as the default
% |\theorembodyfont|.
%
%
% \subsection{Examples}
%
% Given the above theorem sets {\envfont Cor}, {\envfont Exa}
% and {\envfont Rem},
% suppose that the preamble also contains the declarations:
% \begin{verbatim}
%    \theoremstyle{marginbreak}   \newtheorem{Lem}[Cor]{Lemma}
%    \theoremstyle{change}
%    \theorembodyfont{\itshape}        \newtheorem{Def}[Cor]{Definition}
%
%    \theoremheaderfont{\scshape}
%\end{verbatim}
% Then the following are some typical examples of the typeset output
% resulting from their use.
%
% \begin{Cor}
%    This is a sentence typeset in the theorem environment {\envfont
%    Cor}.
% \end{Cor}
%
% \begin{Exa}
%    This is a sentence typeset in the theorem environment {\envfont
%    Exa}.
% \end{Exa}
%
% \begin{Rem}
%    This is a sentence typeset in the theorem environment {\envfont
%    Rem}.
% \end{Rem}
%
% \begin{Lem}[Ben User]
%    This is a sentence typeset in the theorem environment {\envfont
%    Lem}.
% \end{Lem}
%
% \begin{Def}[Very impressive Definition]
%    This is a sentence typeset in the theorem environment {\envfont
%    Def}.
% \end{Def}
%
% The last two examples show the effect of the optional argument to
% a theorem environment (it is the text typeset in parentheses).
%
%
% \section{Special Considerations}
% \label{sec:newfonts}
%
% Theoremheader and body are implemented as a unit. This means that
% the |\theoremheaderfont|  will inherit characteristics of the
% |\theorembodyfont| in \LaTeXe. Thus, if for example
% |\theorembodyfont| is |\itshape| and |\theoremheaderfont|
% is |\bfseries| the font selected for the header will have the
% characteristics `bold extended italic'. If this is not desired one
% should set the |\theoremheaderfont| to something like
% \begin{verbatim}
%      \theoremheaderfont{\normalfont\bfseries}
%\end{verbatim}
% i.e.\ supplying all necessary font information explicitly.
%
%
%
% \section{Acknowledgements}
%
% The publication of this set of macros was only possible with the
% help of Christina Busse (translating the manuscript into English),
% Joachim Pense (playing the r\^ole of typist), Chris Rowley
% (looking everything over) and many others providing useful
% suggestions.
%
% \MaybeStop{%
%    \begin{thebibliography}{9}
%       \bibitem{bk:GMS94} \textsc{M.~Goossens}, \textsc{F.~Mittelbach}
%         and \textsc{A.~Samarin}.
%         \newblock The \LaTeX{} Companion.
%         \newblock
%         Addison-Wesley, Reading, Massachusetts, 1994.
%       \bibitem{latex:source} \textsc{Lamport, Leslie.} \newblock
%                               \texttt{latex.tex}, version 2.09, date
%                               Feb. 1990.
%    \end{thebibliography}
%    \PrintIndex
%    \PrintChanges}
%
%
% \section{The documentation driver file}
%
% The next bit of code contains the documentation driver file for
% \TeX{}, i.e., the file that will produce the documentation you are
% currently reading. It will be extracted from this file by the
% \texttt{docstrip} program. Since it is the first code in the file
% one can alternatively process this file directly with \LaTeXe{} to
% obtain the documentation.
%
%    \begin{macrocode}
%<*driver>
\documentclass{ltxdoc}

\usepackage{theorem}

%<+driver>% The next few lines define theorem sets which are used
%<+driver>% in the example section of the documentation.

\theoremstyle{break}       \newtheorem{Cor}{Corollary}
\theoremstyle{plain}       \newtheorem{Exa}{Example}[section]
{\theorembodyfont{\upshape}\newtheorem{Rem}{Remark}}
\theoremstyle{marginbreak} \newtheorem{Lem}[Cor]{Lemma}
\theoremstyle{change}
\theorembodyfont{\itshape} \newtheorem{Def}[Cor]{Definition}

\theoremheaderfont{\scshape}

\RecordChanges

\begin{document}
   \DocInput{theorem.dtx}
\end{document}
%</driver>
%    \end{macrocode}
%
% \section{Definition of the Macros}
%
% If the file has been loaded before, we abort immediately. If not the
% package announces itself (this is actually done at the very top of
% the file---the way it is done isn't good style so don't copy it).
% \changes{v2.0e}{89/07/19}{Spaces removed from `typeout'.}
% \changes{v2.2b}{95/11/19}{Announce moved to top of file}
%    \begin{macrocode}
%<*package>
%\@ifundefined{theorem@style}{}{\endinput}
%\def\FMithmInfo{1995/11/19 v2.2b Theorem extension package (FMi)}
%\ProvidesPackage{theorem}[\FMithmInfo]
%    \end{macrocode}
%
%
% \subsection{Definition of theorem styles and fonts}
%
% All the definitions in this file are done globally to allow
% inputting this file inside a group.
%
% \begin{macro}{\theoremstyle}
%    Before a theorem style can be installed, the chosen style must be
%    known. For that reason, we must test to see that
%    |\th@|\meta{style} is known or, more precisely, that it is
%    different from |\relax|. If the style is not known then
%    |\th@plain| is used.
%    \begin{macrocode}
\gdef\theoremstyle#1{%
   \@ifundefined{th@#1}{\@warning
          {Unknown theoremstyle `#1'. Using `plain'}%
          \theorem@style{plain}}%
%    \end{macrocode}
%    We save the theorem style to be used in the token register
%    |\theorem@style|.
%    \begin{macrocode}
      {\theorem@style{#1}}%
%    \end{macrocode}
%    Now we ``evaluate'' the theorem style: this means, we call the
%    macro |\th@|\meta{style} which will activate the relevant
%    definitions which are contained in a separate file.  This is done
%    in a group to suppress changes to the current font.  This could
%    otherwise pose problems together with the new font selection
%    scheme\footnote{When I printed the original article using the new
%    font selection scheme I ended with a document with slanted
%    typefaces (text headings and all) simply because one of the
%    theorem styles used \texttt{\bslash sl} at toplevel.} if the
%    |\th@|\meta{style} is evaluated a second time.
% \changes{v2.1a}{90/02/15}{Group added around `th@..}
% \changes{v2.1a}{90/02/15}{Reseting of `..@begintheorem removed.}
%    \begin{macrocode}
      \begingroup
        \csname th@\the\theorem@style \endcsname
      \endgroup}
%    \end{macrocode}
% \end{macro}
%
%
% \begin{macro}{\@begintheorem}
% \changes{v2.1a}{90/02/15}{Relaxed at toplevel.}
% \begin{macro}{\@opargbegintheorem}
% \changes{v2.1a}{90/02/15}{Relaxed at toplevel.}
%    We reset |\@begintheorem| and |\@opargbegintheorem| to
%    |\relax| since these commands are no longer necessary at
%    toplevel. This will save a few tokens.
%    \begin{macrocode}
\global\let\@begintheorem\relax
\global\let\@opargbegintheorem\relax
%    \end{macrocode}
% \end{macro}
% \end{macro}
%
% \begin{macro}{\theorem@style}
%    Obviously the \textsf{token} register used above has to be
%    allocated. To assure the utmost compatibility with the
%    original \LaTeX{} definition, we set the default theorem style to
%    \texttt{plain}, which implements the usual \LaTeX{} convention.
% \changes{v2.0f}{89/07/20}{Unnecessary `global' removed.}
%    \begin{macrocode}
\newtoks\theorem@style
\global\theorem@style{plain}
%    \end{macrocode}
% \end{macro}
%
% \begin{macro}{\theorembodyfont}
% \begin{macro}{\theorem@bodyfont}
%    If the |\theorembodyfont| is set by the user then it should not
%    interact with the default font set in the theorem style. When the
%    new font selection is in force this may happen if, for example,
%    the default is |\itshape| and the new |\theorembodyfont| is
%    |\sffamily|. So we add a |\reset@font| command in front of the
%    user definition.
% \changes{v2.0f}{89/07/20}{Unnecessary `global' removed.}
% \changes{v2.0g}{89/09/19}{`theoremfont to `theorembodyfont.}
% \changes{v2.1c}{92/06/22}{`reset@font added and logic changed.}
% \changes{v2.1d}{92/10/28}{Reset to empty if arg is empty}
%    \begin{macrocode}
\gdef\theorembodyfont#1{%
%    \end{macrocode}
%    We check if the argument supplied is empty and if so put nothing
%    into the |\theorem@bodyfont| token register to allow for
%    |\theorembodyfont{}| as a mean of using the default of the
%    current |\theoremstyle|.
%    \begin{macrocode}
   \def\@tempa{#1}%
   \ifx\@tempa\@empty
    \theorem@bodyfont{}%
   \else
    \theorem@bodyfont{\reset@font#1}%
   \fi
   }
\newtoks\theorem@bodyfont
\global\theorem@bodyfont{}
%    \end{macrocode}
% \end{macro}
% \end{macro}
%
%
% \begin{macro}{\theoremheaderfont}
% \changes{v2.0g}{89/09/19}{Macro added.}
%    The font for the theorem headers is handled differently because
%    this definition applies to all theorem styles.
%    \begin{macrocode}
\gdef\theoremheaderfont#1{\gdef\theorem@headerfont{#1}%
%    \end{macrocode}
%    After using the macro once it is redefined to produce an error
%    message.
%    \begin{macrocode}
       \gdef\theoremheaderfont##1{%
        \typeout{\string\theoremheaderfont\space should be used
                 only once.}}}
%    \end{macrocode}
% \end{macro}
%
% \begin{macro}{\theorem@headerfont}
% \changes{v2.0g}{89/09/19}{Macro added.}
%    To set the |\theorem@headerfont| default we first test if
%    the new fontselection scheme is in force.
% \changes{v2.1a}{90/02/15}{Support for new font selection.}
%    \begin{macrocode}
\ifx\upshape\undefined
%    \end{macrocode}
%    If not we define it to expand into |\bfseries|. We don't use
%    |\let| just in case a following style option redefines this
%    macro.
%    \begin{macrocode}
\gdef\theorem@headerfont{\bfseries}
%    \end{macrocode}
%    Otherwise we reset the current shape before calling |\bfseries|.
%    \begin{macrocode}
\else \gdef\theorem@headerfont{\normalfont\bfseries}\fi
%    \end{macrocode}
% \end{macro}
%
% \begin{macro}{\th@plain}
% \begin{macro}{\th@break}
% \begin{macro}{\th@marginbreak}
% \begin{macro}{\th@changebreak}
% \begin{macro}{\th@change}
% \begin{macro}{\th@margin}
%    The different styles are defined in macros such as
%    |\th@plain|.  Since memory space is precious in
%    ``non-Big-versions'', we have to avoid offering too many unused
%    definitions.  Therefore we define these styles in separate files
%    that can be loaded on demand.  Thus the commands themselves only
%    load these files. We use |\@input@|, a \LaTeXe{} internal command
%    that ensures that the file will be listed with |\listfiles|.
% \changes{v2.2a}{94/02/02}{Use `@input@ to load theorem layout files}
%    \begin{macrocode}
\gdef\th@plain{\@input@{thp.sty}}
\gdef\th@break{\@input@{thb.sty}}
\gdef\th@marginbreak{\@input@{thmb.sty}}
\gdef\th@changebreak{\@input@{thcb.sty}}
\gdef\th@change{\@input@{thc.sty}}
\gdef\th@margin{\@input@{thm.sty}}
%    \end{macrocode}
%    This list will be expanded when new styles become
%    available.  For testing,  just append new theorem substyles
%    as document options.
% \end{macro}
% \end{macro}
% \end{macro}
% \end{macro}
% \end{macro}
% \end{macro}
%
%
%
% \subsection{Definition of a new theorem set}
%
% As already pointed out, a new theorem environment can be defined in
% three different ways:
% \begin{verbatim}
%    \newtheorem{Lem}{Lemma}
%    \newtheorem{Lem}{Lemma}[section]
%    \newtheorem{Lem}[Theorem]{Lemma}
%\end{verbatim}
% The function of the macro |\newtheorem| is to recognize these cases
% and then to branch into one of the three macros |\@ynthm|, |\@xnthm|
% or |\@othm|.  This mechanism is adopted unchanged from
% \cite{latex:source}; the essential point here is that, for example,
% in the second case, the arguments \texttt{Lem}, \texttt{Lemma} and
% \texttt{section} are passed over to the macro |\@xnthm|.
%
% We inspect this case first because the others present fewer problems,
% and thus are easily derived from this one.
%
% \begin{macro}{\@xnthm}
%    For our example arguments, the macro |\@xnthm| must fulfill
%    the following:
%    \begin{itemize}
%       \item
%          Define a new \LaTeX-counter `Lem'
%       \item
%          reset this counter within a |\section|
%       \item
%          define the macro |\theLem|
%       \item
%          define the environment macros |\Lem| and
%          |\endLem| using the current
%          |\theoremstyle| and |\theorem@bodyfont|.
%    \end{itemize}
%    Obviously, all this should happen only if the first argument
%    of |\@xnthm| (i.e.\ \texttt{Lem} in our example) is chosen so
%    as not to conflict with any previously defined commands or
%    environments.  This test is performed by the \LaTeX{} macro
%    |\@ifdefinable|.
%    \begin{macrocode}
\gdef\@xnthm#1#2[#3]{\expandafter\@ifdefinable\csname #1\endcsname
%    \end{macrocode}
%    Therefore, the first argument of |\@ifdefinable| is the
%    expansion (in the example, |\Lem|) of
%    |\csname||#1\endcsname|.  The second argument is
%    executed only if the test has been completed successfully.
%    \begin{macrocode}
   {%
%    \end{macrocode}
%    Now we define the new counter. The names of the \LaTeX{}
%    macros employed should speak for themselves:
% \changes{v2.2b}{1995/10/07}{Use \cs{@newctr}}
% \changes{v2.2c}{1995/11/23}{Use \cs{@newctr} properly.}
%    \begin{macrocode}
    \@definecounter{#1}\@newctr{#1}[#3]%
%    \end{macrocode}
%    Using |\@newctr| will give a proper error message if the counter
%    in |#3| is not defined.
%    In defining `|\theLem|' we must generate the desired macro
%    name by use of |\expandafter| and |\csname|.
%    \begin{macrocode}
    \expandafter\xdef\csname the#1\endcsname
%    \end{macrocode}
%    An |\xdef| is used in order to make the definition global,
%    and to ensure that it contains the replacement texts of
%    |\@thmcountersep| and |\@thmcounter|.\footnote{These
%    two macros can be defined by the document style.  Their default
%    values produce a `\texttt{.}' as separation and an arabic
%    representation of the number.} However, not everything should be
%    expanded. For example, it saves space to use |\thesection|
%    instead of its---at times---lengthy expansion.
%    \begin{macrocode}
      {\expandafter \noexpand \csname the#3\endcsname
       \@thmcountersep \@thmcounter{#1}}%
%    \end{macrocode}
%    Thus with the defaults of \LaTeX{}, |\theLem| would be
%    replaced by the command sequence |\thesection.\arabic{Lem}|.
%
%    We will now look at the definition of the macro which is executed
%    at the beginning of the actual environment (in our example this
%    macro is |\Lem|). It should be noted that we use an
%    ``|\expandafter| trick'' to expand only certain parts of the
%    replacement text at the time of the definition.
%    \begin{macrocode}
    \def\@tempa{\global\@namedef{#1}}%
    \expandafter \@tempa \expandafter{%
%    \end{macrocode}
%    First, the macro that contains the current definitions of
%    |\@begintheorem| and |\@opargtheorem| should be called
%    up.  The name of this macro---as is already known---has the form
%    |\th@|\meta{theorem\ style}; therefore, it must be called by
%    \begin{macrocode}
      \csname th@\the \theorem@style
%    \end{macrocode}
%    In addition the default theorem font should be changeable, i.e.\
%    we have to insert the contents of |\theorem@bodyfont|. For
%    that reason, we expand even further, beyond |\endcsname|,
%    and thus insert the contents of the \textsf{token} register
%    |\theorem@bodyfont| in the replacement text.
%    \begin{macrocode}
            \expandafter \endcsname \the \theorem@bodyfont
%    \end{macrocode}
%    Now it is time to call the macro |\@thm| which takes over
%    the further processing.  It has two arguments: the current
%    counter name (in our example, \texttt{Lem}), and the text of the
%    label (in our example, \texttt{Lemma}).
%    \begin{macrocode}
     \@thm{#1}{#2}}%
%    \end{macrocode}
%    With this, the `sub-definition' is complete.  The macro
%    |\@endtheorem| ends a theorem environment and is, so far, nothing
%    but an |\endtrivlist|.  (Hence it is defined globally, and not
%    within the theorem styles.\footnote{This has to be changed as
%    soon as theorem styles that change \texttt{\bslash @endtheorem}
%    exist.  In such a case, all existing styles must be changed as
%    well since they will have to reset the macro.}) Therefore, we can
%    set it equivalent to the macro that ends the theorem set (in our
%    example, |\endLem|).  However, if some day theorem styles exist
%    that do change |\@endtheorem|, we would have to use the
%    commented-out line instead.
%    \begin{macrocode}
    \global \expandafter \let \csname end#1\endcsname \@endtheorem
%   \global\@namedef{end#1}{\@endtheorem}%
%    \end{macrocode}
%    With these commands all the required definitions are employed,
%    unless the test |\@ifdefinable| has failed.  Therefore, we
%    end the second argument of this macro and with it the definition
%    of |\@xnthm|.
%    \begin{macrocode}
   }}
%    \end{macrocode}
% \end{macro}
%
% \begin{macro}{\@ynthm}
%    The definition of |\@ynthm| is completely analogous. In
%    this case the new counter that is defined is not reset within
%    another counter; thus the definition of |\the...|
%    is simplified:
%    \begin{macrocode}
\gdef\@ynthm#1#2{\expandafter\@ifdefinable\csname #1\endcsname
   {\@definecounter{#1}%
    \expandafter\xdef\csname the#1\endcsname{\@thmcounter{#1}}%
%    \end{macrocode}
%    The rest of the definition corresponds literally to that of
%    |\@xnthm|:
%    \begin{macrocode}
    \def\@tempa{\global\@namedef{#1}}\expandafter \@tempa
     \expandafter{\csname th@\the \theorem@style \expandafter
     \endcsname \the\theorem@bodyfont \@thm{#1}{#2}}%
    \global \expandafter \let \csname end#1\endcsname \@endtheorem}}
%    \end{macrocode}
% \end{macro}
%
% \begin{macro}{\@othm}
%    The definition of |\@othm| does not contain anything new.
%    \begin{macrocode}
\gdef\@othm#1[#2]#3{%
%    \end{macrocode}
% \changes{v2.2b}{95/10/07}{Build in check for counter defined}
%    We do not define a new counter but instead use one that has
%    already been defined.  Thus the only definition we need is that
%    of this pseudo-counter (i.e.\ |\the|\meta{env.\ name}).
%    First we check if |#2| corresponds to a known counter name.
%    \begin{macrocode}
  \expandafter\ifx\csname c@#2\endcsname\relax
   \@nocounterr{#2}%
  \else
   \expandafter\@ifdefinable\csname #1\endcsname
   {\expandafter \xdef \csname the#1\endcsname
     {\expandafter \noexpand \csname the#2\endcsname}%
%    \end{macrocode}
%    All other parts of the definition can be adopted from
%    |\@xnthm|. We have to remember, though, that in this case
%    the name of the current counter and the theorem label have moved
%    to the second and third arguments.
%    \begin{macrocode}
    \def\@tempa{\global\@namedef{#1}}\expandafter \@tempa
     \expandafter{\csname th@\the \theorem@style \expandafter
     \endcsname \the\theorem@bodyfont \@thm{#2}{#3}}%
    \global \expandafter \let \csname end#1\endcsname \@endtheorem}%
  \fi}
%    \end{macrocode}
% \end{macro}
%
% \subsection{Macros that are employed in a {\envfont theorem}
%    environment}
%
% \begin{macro}{\@thm}
%    The macro |\@thm| has to increase the current counter.  Then,
%    depending on whether the environment has (or does not have) an
%    optional argument, it has to branch into either |\@begintheorem|
%    or |\@opargtheorem|.
%    \begin{macrocode}
\gdef\@thm#1#2{\refstepcounter{#1}%
%    \end{macrocode}
%    Now we start a {\envfont trivlist} environment, and give
%    |\@topsep| and |\@topsepadd| the values of the \textsf{skip}
%    registers |\theorempreskipamount| and |\theorempostskipamount|.
%    The value in |\@topsep| is the vertical space that is inserted by
%    the first (and only) |\item| in our |\trivlist| whilst
%    |\@topsepadd| is inserted by |\@endparenv| at the end of that
%    {\envfont trivlist} environment.  By using these registers, we
%    obtain the desired space around a {\envfont theorem} environment.
%    \begin{macrocode}
   \trivlist
   \@topsep \theorempreskipamount               % used by first \item
   \@topsepadd \theorempostskipamount           % used by \@endparenv
%    \end{macrocode}
%    Now we have to test whether an optional argument has been given.
% \changes{v2.0d}{89/07/18}{Forgotten percent added.}
%    \begin{macrocode}
   \@ifnextchar [%
%    \end{macrocode}
%    If there is an optional argument, we will call |\@ythm|, and
%    move the arguments read back into the input stream.
%    \begin{macrocode}
   {\@ythm{#1}{#2}}%
%    \end{macrocode}
%    If not, we call |\@begintheorem|. Its first argument is the
%    name of the theorem set (hence the second argument of
%    |\@thm|). Its second argument is the macro that produces the
%    current number.
%    \begin{macrocode}
   {\@begintheorem{#2}{\csname the#1\endcsname}\ignorespaces}}
%    \end{macrocode}
% \end{macro}
%
% \begin{macro}{\@xthm}
% \begin{macro}{\@ythm}
%    Both these macros were originally called by |\@thm|.  We do
%    not need |\@xthm| anymore, hence we reset it to
%    |\relax|. The definition of |\@ythm| has not changed at
%    all from its definition in \LaTeX. In order to make the macros
%    easier to understand, we will
%    nevertheless present it (commented out).
%    \begin{macrocode}
\global\let\@xthm\relax
% \def\@ythm#1#2[#3]{\@opargbegintheorem{#2}{\csname
%           the#1\endcsname}{#3}\ignorespaces}
%</package>
%    \end{macrocode}
%    The primitive |\ignorespaces| in |\@ythm| and
%    |\@thm| is needed to remove the spaces between the
%    |\begin{...}| and the actual text.
% \end{macro}
% \end{macro}
%
%
% \subsection{Definition of the theorem substyles}
%
% As already pointed out, the theorem substyles, defined below, are
% only loaded when necessary.  Note that all these substyles, except
% \texttt{plain}, have |\slshape| as the default body font.
%
% \subsubsection{The \texttt{plain} style}
%
% \label{dok:th@plain} As the following macros use |@|, we have
% to locally set the |\catcode| of this symbol to ``letter''.
% This happens within a group, so that we do not have to worry about
% which |\catcode| that symbol had before.
%    \begin{macrocode}
%<*thp>
\begingroup \makeatletter
%    \end{macrocode}
% Since we are now within a group, we must make all definitions
% globally.  First we make sure that \texttt{theorem.sty} is loaded.
% This will allow us to use this file as a document style option
% without having to call \texttt{theorem} itself as an option.  At the
% same time, we assure that at least version 2 is loaded, since
% |\theorem@style| was not defined in earlier versions.
%    \begin{macrocode}
\@ifundefined{theorem@style}{\input{theorem.sty}}{}
\ProvidesFile{thp.sty}
             [\FMithmInfo]
%    \end{macrocode}
%
%
% \begin{macro}{\th@plain}
% \begin{macro}{\@begintheorem}
% \begin{macro}{\@opargbegintheorem}
%    |\theoremstyle{plain}| corresponds to the original
%    definition, except that the distances to the surrounding text are
%    determined by the parameters |\theorempreskipamount| and
%    |\theorempostskipamount|.
%    First we set the default body font.
%    \begin{macrocode}
\gdef\th@plain{\normalfont\itshape
%    \end{macrocode}
%    Then we define |\@begintheorem| and
%    |\@opargbegintheorem|. These two macros define how the
%    header of a theorem is typeset.  |\@opargbegintheorem| will
%    be called if a \textsf{theorem} environment with an optional
%    argument is encountered; otherwise, the header is constructed by
%    calling |\@begintheorem|.  If one of these macros is
%    executed, we are within a \textsf{trivlist} environment started by
%    |\@thm|.  So the theorem header is produced with an
%    |\item| command.
%
%    Instead of specifying the header font directly, all standard
%    theorem styles use the |\theorem@headerfont| macro to allow
%    customization.  The extra space (|\labelsep|) is necessary
%    because of problems in the \textsf{trivlist} environment.
%    \begin{macrocode}
  \def\@begintheorem##1##2{%
        \item[\hskip\labelsep \theorem@headerfont ##1\ ##2]}%
%    \end{macrocode}
%    The definition of |\@opargbegintheorem| is completely
%    analogous. The only difference is the fact that there exists a
%    third argument (which is the optional parameter of the
%    environment and contains additional information about the
%    theorem).  Customarily we enclose it in parentheses.
%    \begin{macrocode}
\def\@opargbegintheorem##1##2##3{%
   \item[\hskip\labelsep \theorem@headerfont ##1\ ##2\ (##3)]}}
%    \end{macrocode}
% \end{macro}
% \end{macro}
% \end{macro}
%
% \noindent
% We conclude with an |\endgroup| to restore the
% |\catcode| of |@|.
%    \begin{macrocode}
\endgroup
%</thp>
%    \end{macrocode}
%
%
% \subsubsection{The \texttt{break} style}
%
% This style option is stored in the file |thb.sty|.
% For the next two lines see the documentation for |\th@plain| on
% page~\pageref{dok:th@plain}.
%    \begin{macrocode}
%<*thb>
\begingroup \makeatletter
\@ifundefined{theorem@style}{\input{theorem.sty}}{}
\ProvidesFile{thb.sty}
             [\FMithmInfo]
%    \end{macrocode}
%
% \begin{macro}{\th@break}
%    |\theoremstyle{break}| produces a line break after the name
%    of the theorem.  The font is |\slshape|.  Hence, we define
%    |\th@break| as follows:
%    \begin{macrocode}
\gdef\th@break{\normalfont\slshape
  \def\@begintheorem##1##2{\item[%
%    \end{macrocode}
%    We run into the following problem: it is not possible to create
%    the header with |\item[|\meta{title}|]| and then start
%    a new line by, for example, |\mbox{}\\|. Such a definition
%    will fail whenever a list environment follows immediately.  With
%    the above construction, the |\mbox{}| causes the switch
%    |@inlabel| (cf.\ definition of |\list| and
%    |\trivlist| in \cite{latex:source}) to be set to \texttt{false}
%    and so the following list will insert additional vertical
%    space (|\topskip|). This is quite annoying. Therefore, we
%    create the line break within the |\item|. In order to ensure
%    that the text will begin at the proper position in the following
%    line, we simply pretend that the label does not take any
%    room.\footnote{This will lead to problems whenever very high
%    symbols occurring in the line tower into the
%    heading. So, something else has to be done here sometime.}
%    \begin{macrocode}
       \rlap{\vbox{\hbox{\hskip \labelsep\theorem@headerfont ##1\ ##2}%
                   \hbox{\strut}}}]}%
%    \end{macrocode}
%    Again, the definition of |\@opargbegintheorem| is
%    completely analogous.
%    \begin{macrocode}
\def\@opargbegintheorem##1##2##3{%
  \item[\rlap{\vbox{\hbox{\hskip \labelsep \theorem@headerfont
                     ##1\ ##2\ (##3)}%
                    \hbox{\strut}}}]}}
\endgroup
%</thb>
%    \end{macrocode}
% \end{macro}
%
%
% \subsubsection{The \texttt{changebreak} style}
%
%    \begin{macrocode}
% This style option is stored in the file |thcb.sty|.
%    \begin{macrocode}
%<*thcb>
\begingroup \makeatletter
\@ifundefined{theorem@style}{\input{theorem.sty}}{}
\ProvidesFile{thcb.sty}
             [\FMithmInfo]
%    \end{macrocode}
%
% \begin{macro}{\th@changebreak}
%    The \texttt{change-break} theorem style is like \texttt{break}
%    but with interchange of theorem name and theorem number.  Thus we
%    define |\th@changebreak| as follows:
%    \begin{macrocode}
\gdef\th@changebreak{\normalfont\slshape
  \def\@begintheorem##1##2{\item
       [\rlap{\vbox{\hbox{\hskip\labelsep \theorem@headerfont ##2\ ##1}%
                    \hbox{\strut}}}]}%
\def\@opargbegintheorem##1##2##3{%
  \item[\rlap{\vbox{\hbox{\hskip\labelsep \theorem@headerfont
                     ##2\ ##1\ (##3)}%
                    \hbox{\strut}}}]}}
\endgroup
%</thcb>
%    \end{macrocode}
% \end{macro}
%
%
% \subsubsection{The \texttt{change} style}
%
% This style option is stored in the file |thc.sty|.
%    \begin{macrocode}
%<*thc>
\begingroup \makeatletter
\@ifundefined{theorem@style}{\input{theorem.sty}}{}
\ProvidesFile{thc.sty}
             [\FMithmInfo]
%    \end{macrocode}
%
% \begin{macro}{\th@change}
%    The \texttt{change} theorem style corresponds to the
%    \texttt{change break} style without a linebreak after the header.
%    To say it in another way, it's the same as the \texttt{plain}
%    style but with number and name interchanged and |\slshape| as the
%    default font.
%    \begin{macrocode}
\gdef\th@change{\normalfont\slshape
  \def\@begintheorem##1##2{\item
       [\hskip\labelsep \theorem@headerfont ##2\ ##1]}%
\def\@opargbegintheorem##1##2##3{%
  \item[\hskip\labelsep \theorem@headerfont ##2\ ##1\ (##3)]}}
\endgroup
%</thc>
%    \end{macrocode}
% \end{macro}
%
%
% \subsubsection{The \texttt{marginbreak} style}
%
% This style option is the one used most often at Mainz.  It is saved
% in the file |thmb.sty|.
%    \begin{macrocode}
%<*thmb>
\begingroup \makeatletter
\@ifundefined{theorem@style}{\input{theorem.sty}}{}
\ProvidesFile{thmb.sty}
             [\FMithmInfo]
%    \end{macrocode}
%
% \begin{macro}{\th@marginbreak}
%    The \texttt{margin break} style is nearly the same as the
%    \texttt{change break} style. The only difference is the placement
%    of the theorem number.  We use |\llap| to place it in the left
%    margin.
%
%    In this style |\labelsep| denotes the separation between the
%    number and the text.
%    \begin{macrocode}
\gdef\th@marginbreak{\normalfont\slshape
  \def\@begintheorem##1##2{\item
       [\rlap{\vbox{\theorem@headerfont
                    \hbox{\llap{##2}\hskip\labelsep ##1}%
                    \hbox{\strut}}}]}%
\def\@opargbegintheorem##1##2##3{%
  \item[\rlap{\vbox{\theorem@headerfont
                    \hbox{\llap{##2}\hskip\labelsep ##1\ (##3)}%
                    \hbox{\strut}}}]}}
\endgroup
%</thmb>
%    \end{macrocode}
% \changes{v2.1c}{90/02/17}{`endgroup outside macrocode env.}
% \end{macro}
%
%
%
% \subsubsection{The \texttt{margin} style}
%
% This style option is stored in the file |thm.sty|.
%    \begin{macrocode}
%<*thm>
\begingroup \makeatletter
\@ifundefined{theorem@style}{\input{theorem.sty}}{}
\ProvidesFile{thm.sty}
             [\FMithmInfo]
%    \end{macrocode}
%
% \begin{macro}{\th@margin}
%    Again this is only a variant of the theorem styles described
%    above without any new ideas.
%    \begin{macrocode}
\gdef\th@margin{\normalfont\slshape
  \def\@begintheorem##1##2{\item
       [\theorem@headerfont \llap{##2}\hskip\labelsep ##1]}%
\def\@opargbegintheorem##1##2##3{%
  \item[\theorem@headerfont \llap{##2}\hskip\labelsep ##1\ (##3)]}}
\endgroup
%</thm>
%    \end{macrocode}
% \end{macro}
%
%
%
%
% \subsection{Final Definitions}
%
% \begin{macro}{\theorempreskipamount}
% \begin{macro}{\theorempostskipamount}
%    The \textsf{skip} parameters that regulate the vertical empty space
%    before and after the {\envfont theorem} environment have
%    to be allocated as well.
% \changes{v2.0f}{89/07/20}{Unnecessary `global' removed.}
%    \begin{macrocode}
%<*package>
\newskip\theorempreskipamount
\newskip\theorempostskipamount
%    \end{macrocode}
%    Since we have used the same values for all theorem sets, we now can
%    assign them.
%    \begin{macrocode}
\global\setlength\theorempreskipamount{12pt plus 5pt minus 3pt}
\global\setlength\theorempostskipamount{8pt plus 3pt minus 1.5pt}
%    \end{macrocode}
% \end{macro}
% \end{macro}
%
%
% \begin{macro}{\@endtheorem}
%    The same holds for the macro |\@endtheorem|, which ends a
%    {\envfont theorem} environment. Since it is the same for all
%    theorem sets, it is removed from the macros |\th@|\meta{style}.
%    It simply ends the {\envfont trivlist} environment, which was
%    begun in |\@thm|.
%    \begin{macrocode}
\global\let\@endtheorem=\endtrivlist
%    \end{macrocode}
% \end{macro}
%
%
%
% \begin{macro}{\@preamblecmds}
%    All macros defined above are to be used only in the preamble.
%    Therefore, we insert them in |\@preamblecmds| which will disable
%    them at begin document. This is done by the internal \LaTeXe{}
%    command |\@onlypreamble|.
% \changes{v2.0g}{89/09/19}{`theoremheaderfont and `theorembodyfont
%    added.}
% \changes{v2.2a}{94/02/03}{Use \LaTeXe{} interface}
%    \begin{macrocode}
\@onlypreamble\@xnthm
\@onlypreamble\@ynthm
\@onlypreamble\@othm
\@onlypreamble\newtheorem
\@onlypreamble\theoremstyle
\@onlypreamble\theorembodyfont
\@onlypreamble\theoremheaderfont
%    \end{macrocode}
% \end{macro}
%
% Finally we declare the \texttt{plain} theorem style to be the default.
% \changes{v2.1b}{90/06/04}{Fixed missing `item by declaring default
%    style.}
%    \begin{macrocode}
\theoremstyle{plain}
%</package>
%    \end{macrocode}
%
% \Finale

\endinput
}{}
\ProvidesFile{thm.sty}
             [\FMithmInfo]
%    \end{macrocode}
%
% \begin{macro}{\th@margin}
%    Again this is only a variant of the theorem styles described
%    above without any new ideas.
%    \begin{macrocode}
\gdef\th@margin{\normalfont\slshape
  \def\@begintheorem##1##2{\item
       [\theorem@headerfont \llap{##2}\hskip\labelsep ##1]}%
\def\@opargbegintheorem##1##2##3{%
  \item[\theorem@headerfont \llap{##2}\hskip\labelsep ##1\ (##3)]}}
\endgroup
%</thm>
%    \end{macrocode}
% \end{macro}
%
%
%
%
% \subsection{Final Definitions}
%
% \begin{macro}{\theorempreskipamount}
% \begin{macro}{\theorempostskipamount}
%    The \textsf{skip} parameters that regulate the vertical empty space
%    before and after the {\envfont theorem} environment have
%    to be allocated as well.
% \changes{v2.0f}{89/07/20}{Unnecessary `global' removed.}
%    \begin{macrocode}
%<*package>
\newskip\theorempreskipamount
\newskip\theorempostskipamount
%    \end{macrocode}
%    Since we have used the same values for all theorem sets, we now can
%    assign them.
%    \begin{macrocode}
\global\setlength\theorempreskipamount{12pt plus 5pt minus 3pt}
\global\setlength\theorempostskipamount{8pt plus 3pt minus 1.5pt}
%    \end{macrocode}
% \end{macro}
% \end{macro}
%
%
% \begin{macro}{\@endtheorem}
%    The same holds for the macro |\@endtheorem|, which ends a
%    {\envfont theorem} environment. Since it is the same for all
%    theorem sets, it is removed from the macros |\th@|\meta{style}.
%    It simply ends the {\envfont trivlist} environment, which was
%    begun in |\@thm|.
%    \begin{macrocode}
\global\let\@endtheorem=\endtrivlist
%    \end{macrocode}
% \end{macro}
%
%
%
% \begin{macro}{\@preamblecmds}
%    All macros defined above are to be used only in the preamble.
%    Therefore, we insert them in |\@preamblecmds| which will disable
%    them at begin document. This is done by the internal \LaTeXe{}
%    command |\@onlypreamble|.
% \changes{v2.0g}{89/09/19}{`theoremheaderfont and `theorembodyfont
%    added.}
% \changes{v2.2a}{94/02/03}{Use \LaTeXe{} interface}
%    \begin{macrocode}
\@onlypreamble\@xnthm
\@onlypreamble\@ynthm
\@onlypreamble\@othm
\@onlypreamble\newtheorem
\@onlypreamble\theoremstyle
\@onlypreamble\theorembodyfont
\@onlypreamble\theoremheaderfont
%    \end{macrocode}
% \end{macro}
%
% Finally we declare the \texttt{plain} theorem style to be the default.
% \changes{v2.1b}{90/06/04}{Fixed missing `item by declaring default
%    style.}
%    \begin{macrocode}
\theoremstyle{plain}
%</package>
%    \end{macrocode}
%
% \Finale

\endinput
}{}
\ProvidesFile{thp.sty}
             [\FMithmInfo]
%    \end{macrocode}
%
%
% \begin{macro}{\th@plain}
% \begin{macro}{\@begintheorem}
% \begin{macro}{\@opargbegintheorem}
%    |\theoremstyle{plain}| corresponds to the original
%    definition, except that the distances to the surrounding text are
%    determined by the parameters |\theorempreskipamount| and
%    |\theorempostskipamount|.
%    First we set the default body font.
%    \begin{macrocode}
\gdef\th@plain{\normalfont\itshape
%    \end{macrocode}
%    Then we define |\@begintheorem| and
%    |\@opargbegintheorem|. These two macros define how the
%    header of a theorem is typeset.  |\@opargbegintheorem| will
%    be called if a \textsf{theorem} environment with an optional
%    argument is encountered; otherwise, the header is constructed by
%    calling |\@begintheorem|.  If one of these macros is
%    executed, we are within a \textsf{trivlist} environment started by
%    |\@thm|.  So the theorem header is produced with an
%    |\item| command.
%
%    Instead of specifying the header font directly, all standard
%    theorem styles use the |\theorem@headerfont| macro to allow
%    customization.  The extra space (|\labelsep|) is necessary
%    because of problems in the \textsf{trivlist} environment.
%    \begin{macrocode}
  \def\@begintheorem##1##2{%
        \item[\hskip\labelsep \theorem@headerfont ##1\ ##2]}%
%    \end{macrocode}
%    The definition of |\@opargbegintheorem| is completely
%    analogous. The only difference is the fact that there exists a
%    third argument (which is the optional parameter of the
%    environment and contains additional information about the
%    theorem).  Customarily we enclose it in parentheses.
%    \begin{macrocode}
\def\@opargbegintheorem##1##2##3{%
   \item[\hskip\labelsep \theorem@headerfont ##1\ ##2\ (##3)]}}
%    \end{macrocode}
% \end{macro}
% \end{macro}
% \end{macro}
%
% \noindent
% We conclude with an |\endgroup| to restore the
% |\catcode| of |@|.
%    \begin{macrocode}
\endgroup
%</thp>
%    \end{macrocode}
%
%
% \subsubsection{The \texttt{break} style}
%
% This style option is stored in the file |thb.sty|.
% For the next two lines see the documentation for |\th@plain| on
% page~\pageref{dok:th@plain}.
%    \begin{macrocode}
%<*thb>
\begingroup \makeatletter
\@ifundefined{theorem@style}{% \iffalse meta-comment
%
% Copyright (C) 1993-2024
%
% The LaTeX Project and any individual authors listed elsewhere
% in this file.
%
% This file is part of the Standard LaTeX `Tools Bundle'.
% -------------------------------------------------------
%
% It may be distributed and/or modified under the
% conditions of the LaTeX Project Public License, either version 1.3c
% of this license or (at your option) any later version.
% The latest version of this license is in
%    https://www.latex-project.org/lppl.txt
% and version 1.3c or later is part of all distributions of LaTeX
% version 2005/12/01 or later.
%
% The list of all files belonging to the LaTeX `Tools Bundle' is
% given in the file `manifest.txt'.
%
% \fi
%
%%
%
%
\def\FMithmInfo{2023/07/05 v2.2c Theorem extension package (FMi)}
%
% \ProvidesFile{theorem.dtx}[\FMithmInfo]
%\iffalse   % this is a METACOMMENT !
%%
%% Package `theorem' to use with LaTeX2e
%% Copyright (C) 1989-1995 Frank Mittelbach, all rights reserved.
%<+package>\@ifundefined{theorem@style}{}{\endinput}
%<+package>\NeedsTeXFormat{LaTeX2e}[1995/06/01]
%<+package>\ProvidesPackage{theorem}
%<+package>         [\FMithmInfo]
%
% \fi
%
% \changes{v1.0}{87/05/07}{First draft.}
% \changes{v1.0a}{87/05/07}{`thistheoremstyle' added.}
% \changes{v1.1a}{87/07/29}{New concept: redefine the internal
%                          \protect\LaTeX{} commands `@xnthm' etc.}
% \changes{v1.1b}{87/07/31}{`theoremfont' used to specify font.}
% \changes{v1.2a}{88/01/08}{`@endtheorem' now globally defined.}
% \changes{v1.2b}{88/01/11}{`newtheorem' only allowed in the preamble.}
% \changes{v1.3a}{88/02/14}{`theoremfont' now a toks register.}
%
% \changes{v2.0a}{88/11/16}{New implementation and new documentation.}
% \changes{v2.0a}{88/11/16}{No layout changes but internal modifications
%                           old substyles won't work.}
% \changes{v2.0b}{89/01/30}{`global' used for all definitions.}
% \changes{v2.0c}{89/06/15}{Documentation translated to English.
%                            Thanks to LLL and Nina}
% \changes{v2.1a}{90/02/15}{Support for new font selection.}
% \changes{v2.2a}{94/02/03}{Cleanup for LaTeX2e}
%
% \DoNotIndex{\@endparenv,\@ifdefinable,\@ifnextchar,\@ifundefined}
% \DoNotIndex{\@namedef,\@nameuse,\@spaces,\@tempa,\@warning}
% \DoNotIndex{\ }
% \DoNotIndex{\begingroup,\bfseries}
% \DoNotIndex{\csname}
% \DoNotIndex{\def,\do,\docdate}
% \DoNotIndex{\endcsname,\endgroup,\endinput,\endtrivlist,\expandafter}
% \DoNotIndex{\filedate,\fileversion}
% \DoNotIndex{\gdef,\global}
% \DoNotIndex{\hbox,\hskip}
% \DoNotIndex{\ignorespaces,\input,\itshape,\item}
% \DoNotIndex{\labelsep,\let,\llap}
% \DoNotIndex{\makeatletter}
% \DoNotIndex{\noexpand}
% \DoNotIndex{\refstepcounter,\relax,\rlap}
% \DoNotIndex{\setlength,\slshape,\space,\string,\strut}
% \DoNotIndex{\the,\trivlist,\typeout}
% \DoNotIndex{\vbox}
% \DoNotIndex{\xdef}
%
% \def\deflabel#1{\ttfamily #1\hfill}
% \def\deflist#1{\begin{list}{}{\settowidth\labelwidth{\ttfamily #1}%
%                               \setlength\leftmargin\labelwidth
%                               \addtolength\leftmargin\labelsep
%                               \let\makelabel\deflabel}}
% \def\enddeflist{\end{list}}
% \def\envfont{\normalfont\ttfamily}
%
% \MakeShortVerb{\|}
%
% \newcommand{\TUB}{{\sl TUGboat\/}}
%
% \GetFileInfo{theorem.dtx}
% \title{An Extension of the
%        \LaTeX{} theorem environment\thanks{This file
%        has version number \fileversion, last
%        revised \filedate.}}
% \author{Frank Mittelbach}
%
% \MaintainedByLaTeXTeam{tools}
% \maketitle
%
% \begin{abstract}
%    The macros described in this paper yield an extension of the
%    \LaTeX{} theorem mechanism. It is designed to satisfy the
%    different requirements of various journals. Thus, the
%    layout of the ``theorems'' can be manipulated by determining a
%    ``style''. This article describes not only the use, but
%    also the definition, of the necessary macros.
% \end{abstract}
%
% \section*{Preface to version 2.2}
%
% For \LaTeXe{} this package did not need any fundamental changes. I
% only modified the messages generated so that theorem layout styles
% will show up with the |\listfiles| command and cleaned the section
% on the New Font Selection Scheme since this is now included in
% \LaTeX.
%
% \section*{Preface to version 2.1}
%
% This version is identical to 2.0g described in \TUB\ 10\#3 except
% for some internal defaults which are now set depending on the used
% font selection scheme.
%
% This was done to avoid unpleasant surprises if the new font selection
% scheme is in force. For further details see
% section~\ref{sec:newfonts} and \cite{bk:GMS94}.
%
% \section{Introduction}
%
% For our purposes here, ``theorems'' are labelled enunciations,
% often set off from the main text by extra space and a font change.
% Theorems, corollaries, conjectures, definitions, and remarks
% are all instances of ``theorems''.  The ``header'' of these structures
% is composed of a label (such as \textsc{Theorem} or \textsc{Remark})
% and a number which serializes an item in the sequence of items
% with the same label.
%
% Shortly after the introduction of \LaTeX{} at the Fachbereich
% Mathematik in Mainz, the desire to manipulate the layout of
% ``theorems'' arose.  In Mainz, the following two conventions came
% into general use:
% \begin{enumerate}
%   \item The number of the theorem is shown in the margin.
%   \item There is a line break at
%         the end of the theorem header.
% \end{enumerate}
%
% Additionally, some journals require different formats which depend on
% the ``sort of theorem'': e.g.\ often remarks and definitions are set
% in |\upshape|, while |\itshape| is employed for main theorems.
%
% Confronted with these requirements, a theorem environment was
% developed in Mainz which allows separate determination of the
% layout of the ``theorems sets'', comparable to |\pagestyle|.
%
%
% \section{The user interface}
% \subsection{Defining new theorem sets}
%
% \DescribeMacro\newtheorem
% As in the original \LaTeX{} version, the command |\newtheorem|
% defines a new ``theorem set'' or ``theorem-like structure''.
% Two required arguments name the new environment and give the text
% to be typeset with each instance of the new ``set'', while
% an optional argument determines how
% the ``set'' is enumerated:
% \begin{description}
%    \item[\ttfamily \bslash newtheorem\{foo\}\{bar\}]
%       The theorem set {\envfont foo} (whose name is \texttt{bar})
%       uses its own counter.
%    \item[\ttfamily \bslash newtheorem\{foo2\}{[foo]}\{bar2\}]
%       The theorem set {\envfont foo2} (printed name \texttt{bar2})
%       uses the same counter as the theorem set \texttt{foo}.
%    \item[\ttfamily \bslash newtheorem\{foo3\}\{bar3\}{[section]}]
%       The theorem set {\envfont foo3} (printed name \texttt{bar3}) is
%       enumerated within the counter \texttt{section}, i.e.\ with every
%       new |\section| the enumeration begins again with 1, and
%       the enumeration is composed from the section-number and the
%       theorem counter itself.
% \end{description}
%
% \DescribeMacro\theoremstyle
% Additionally, the command |\theoremstyle| can define the layout
% of various, or all, theorem sets.  It should be noted that any
% theorem set defined by |\newtheorem| is typeset in the
% |\theoremstyle| that is current at the time of the definition.
% Thus, the following
% \begin{verbatim}
%    \theoremstyle{break}        \newtheorem{Cor}{Corollary}
%    \theoremstyle{plain}        \newtheorem{Exa}{Example}[section]
%\end{verbatim}                             ^^A No blank at line start
% leads to the result that the set {\envfont Cor} is formatted in the
% style \texttt{break}, while the set {\envfont Exa} and all the
% following ones are formatted in the style \texttt{plain}, unless
% another |\theoremstyle| follows.  Since the definitions installed by
% |\newtheorem| are global, one also can limit |\theoremstyle| locally
% by grouping braces.
%
% \DescribeMacro\theorembodyfont
% The choice of the font for the theorem body is completely
% independent of the chosen |\theoremstyle|; this has proven to
% be very advantageous.  For example,
% \begin{verbatim}
%   {\theorembodyfont{\upshape}          \newtheorem{Rem}{Remark}}
%\end{verbatim}
% defines a theorem set {\envfont Rem}, which will be set in
% |\upshape| in the current layout (which in our example is
% \texttt{plain}).  As with |\theoremstyle|, the |\theorembodyfont|
% chosen is that current at the time of |\newtheorem|.  If
% |\theorembodyfont| is not specified or one defines
% |\theorembodyfont{}|, then the font used will be that defined by the
% |\theoremstyle|.
%
% \DescribeMacro\theoremheaderfont It is also possible to customize
% the font used for the theorem headers.  This is, however, a global
% declaration, and therefore there should be at most one
% |\theoremheaderfont| declaration in the preamble.\footnote{If it is
% actually necessary to have different header fonts, one has to define
% new theorem styles (substituting the desired font) or specify the
% information directly in the \texttt{\bslash newtheorem} declaration
% (the unclean variant).}
%
% \DescribeMacro\theorempreskipamount
% \DescribeMacro\theorempostskipamount Two additional parameters
% affect the vertical space around the theorem environments:
% |\theorempreskipamount| and |\theorempostskipamount| define,
% respectively, the spacing before and after such an environment.
% These parameters apply for all theorem sets and can be manipulated
% with the ordinary length macros.  They are rubber lengths
% (`\textsf{skips}'), and therefore can contain \texttt{plus} and
% \texttt{minus} parts.
%
% Since the definition of theorem sets should---most sensibly---be
% placed in the preamble, we only allow installation there.
% It is therefore possible to release the memory used here after
% |\begin{document}|, in order to make room for other
% applications.
%
%
% \subsection{Existing theorem styles}
%
% The following theorem styles exist to date:
% \begin{deflist}{marginbreak}
%    \item[plain]
%       This theorem style emulates the original \LaTeX{} definition,
%       except that additionally the parameters
%       |\theorem...skipamount| are used.
%    \item[break]
%       In this style, the theorem header is followed by a line
%       break.
%    \item[marginbreak]
%       The theorem number is set in the margin, and there is a line
%       break as in \texttt{break}.
%    \item[changebreak]
%       Like \texttt{break}, but with header number and text
%       interchanged.
%    \item[change]
%       Header number and text are interchanged, without a line break.
%    \item[margin]
%       The number is set in the left margin, without a line break.
% \end{deflist}
% All styles (except \texttt{plain}) select |\slshape| as the default
% |\theorembodyfont|.
%
%
% \subsection{Examples}
%
% Given the above theorem sets {\envfont Cor}, {\envfont Exa}
% and {\envfont Rem},
% suppose that the preamble also contains the declarations:
% \begin{verbatim}
%    \theoremstyle{marginbreak}   \newtheorem{Lem}[Cor]{Lemma}
%    \theoremstyle{change}
%    \theorembodyfont{\itshape}        \newtheorem{Def}[Cor]{Definition}
%
%    \theoremheaderfont{\scshape}
%\end{verbatim}
% Then the following are some typical examples of the typeset output
% resulting from their use.
%
% \begin{Cor}
%    This is a sentence typeset in the theorem environment {\envfont
%    Cor}.
% \end{Cor}
%
% \begin{Exa}
%    This is a sentence typeset in the theorem environment {\envfont
%    Exa}.
% \end{Exa}
%
% \begin{Rem}
%    This is a sentence typeset in the theorem environment {\envfont
%    Rem}.
% \end{Rem}
%
% \begin{Lem}[Ben User]
%    This is a sentence typeset in the theorem environment {\envfont
%    Lem}.
% \end{Lem}
%
% \begin{Def}[Very impressive Definition]
%    This is a sentence typeset in the theorem environment {\envfont
%    Def}.
% \end{Def}
%
% The last two examples show the effect of the optional argument to
% a theorem environment (it is the text typeset in parentheses).
%
%
% \section{Special Considerations}
% \label{sec:newfonts}
%
% Theoremheader and body are implemented as a unit. This means that
% the |\theoremheaderfont|  will inherit characteristics of the
% |\theorembodyfont| in \LaTeXe. Thus, if for example
% |\theorembodyfont| is |\itshape| and |\theoremheaderfont|
% is |\bfseries| the font selected for the header will have the
% characteristics `bold extended italic'. If this is not desired one
% should set the |\theoremheaderfont| to something like
% \begin{verbatim}
%      \theoremheaderfont{\normalfont\bfseries}
%\end{verbatim}
% i.e.\ supplying all necessary font information explicitly.
%
%
%
% \section{Acknowledgements}
%
% The publication of this set of macros was only possible with the
% help of Christina Busse (translating the manuscript into English),
% Joachim Pense (playing the r\^ole of typist), Chris Rowley
% (looking everything over) and many others providing useful
% suggestions.
%
% \MaybeStop{%
%    \begin{thebibliography}{9}
%       \bibitem{bk:GMS94} \textsc{M.~Goossens}, \textsc{F.~Mittelbach}
%         and \textsc{A.~Samarin}.
%         \newblock The \LaTeX{} Companion.
%         \newblock
%         Addison-Wesley, Reading, Massachusetts, 1994.
%       \bibitem{latex:source} \textsc{Lamport, Leslie.} \newblock
%                               \texttt{latex.tex}, version 2.09, date
%                               Feb. 1990.
%    \end{thebibliography}
%    \PrintIndex
%    \PrintChanges}
%
%
% \section{The documentation driver file}
%
% The next bit of code contains the documentation driver file for
% \TeX{}, i.e., the file that will produce the documentation you are
% currently reading. It will be extracted from this file by the
% \texttt{docstrip} program. Since it is the first code in the file
% one can alternatively process this file directly with \LaTeXe{} to
% obtain the documentation.
%
%    \begin{macrocode}
%<*driver>
\documentclass{ltxdoc}

\usepackage{theorem}

%<+driver>% The next few lines define theorem sets which are used
%<+driver>% in the example section of the documentation.

\theoremstyle{break}       \newtheorem{Cor}{Corollary}
\theoremstyle{plain}       \newtheorem{Exa}{Example}[section]
{\theorembodyfont{\upshape}\newtheorem{Rem}{Remark}}
\theoremstyle{marginbreak} \newtheorem{Lem}[Cor]{Lemma}
\theoremstyle{change}
\theorembodyfont{\itshape} \newtheorem{Def}[Cor]{Definition}

\theoremheaderfont{\scshape}

\RecordChanges

\begin{document}
   \DocInput{theorem.dtx}
\end{document}
%</driver>
%    \end{macrocode}
%
% \section{Definition of the Macros}
%
% If the file has been loaded before, we abort immediately. If not the
% package announces itself (this is actually done at the very top of
% the file---the way it is done isn't good style so don't copy it).
% \changes{v2.0e}{89/07/19}{Spaces removed from `typeout'.}
% \changes{v2.2b}{95/11/19}{Announce moved to top of file}
%    \begin{macrocode}
%<*package>
%\@ifundefined{theorem@style}{}{\endinput}
%\def\FMithmInfo{1995/11/19 v2.2b Theorem extension package (FMi)}
%\ProvidesPackage{theorem}[\FMithmInfo]
%    \end{macrocode}
%
%
% \subsection{Definition of theorem styles and fonts}
%
% All the definitions in this file are done globally to allow
% inputting this file inside a group.
%
% \begin{macro}{\theoremstyle}
%    Before a theorem style can be installed, the chosen style must be
%    known. For that reason, we must test to see that
%    |\th@|\meta{style} is known or, more precisely, that it is
%    different from |\relax|. If the style is not known then
%    |\th@plain| is used.
%    \begin{macrocode}
\gdef\theoremstyle#1{%
   \@ifundefined{th@#1}{\@warning
          {Unknown theoremstyle `#1'. Using `plain'}%
          \theorem@style{plain}}%
%    \end{macrocode}
%    We save the theorem style to be used in the token register
%    |\theorem@style|.
%    \begin{macrocode}
      {\theorem@style{#1}}%
%    \end{macrocode}
%    Now we ``evaluate'' the theorem style: this means, we call the
%    macro |\th@|\meta{style} which will activate the relevant
%    definitions which are contained in a separate file.  This is done
%    in a group to suppress changes to the current font.  This could
%    otherwise pose problems together with the new font selection
%    scheme\footnote{When I printed the original article using the new
%    font selection scheme I ended with a document with slanted
%    typefaces (text headings and all) simply because one of the
%    theorem styles used \texttt{\bslash sl} at toplevel.} if the
%    |\th@|\meta{style} is evaluated a second time.
% \changes{v2.1a}{90/02/15}{Group added around `th@..}
% \changes{v2.1a}{90/02/15}{Reseting of `..@begintheorem removed.}
%    \begin{macrocode}
      \begingroup
        \csname th@\the\theorem@style \endcsname
      \endgroup}
%    \end{macrocode}
% \end{macro}
%
%
% \begin{macro}{\@begintheorem}
% \changes{v2.1a}{90/02/15}{Relaxed at toplevel.}
% \begin{macro}{\@opargbegintheorem}
% \changes{v2.1a}{90/02/15}{Relaxed at toplevel.}
%    We reset |\@begintheorem| and |\@opargbegintheorem| to
%    |\relax| since these commands are no longer necessary at
%    toplevel. This will save a few tokens.
%    \begin{macrocode}
\global\let\@begintheorem\relax
\global\let\@opargbegintheorem\relax
%    \end{macrocode}
% \end{macro}
% \end{macro}
%
% \begin{macro}{\theorem@style}
%    Obviously the \textsf{token} register used above has to be
%    allocated. To assure the utmost compatibility with the
%    original \LaTeX{} definition, we set the default theorem style to
%    \texttt{plain}, which implements the usual \LaTeX{} convention.
% \changes{v2.0f}{89/07/20}{Unnecessary `global' removed.}
%    \begin{macrocode}
\newtoks\theorem@style
\global\theorem@style{plain}
%    \end{macrocode}
% \end{macro}
%
% \begin{macro}{\theorembodyfont}
% \begin{macro}{\theorem@bodyfont}
%    If the |\theorembodyfont| is set by the user then it should not
%    interact with the default font set in the theorem style. When the
%    new font selection is in force this may happen if, for example,
%    the default is |\itshape| and the new |\theorembodyfont| is
%    |\sffamily|. So we add a |\reset@font| command in front of the
%    user definition.
% \changes{v2.0f}{89/07/20}{Unnecessary `global' removed.}
% \changes{v2.0g}{89/09/19}{`theoremfont to `theorembodyfont.}
% \changes{v2.1c}{92/06/22}{`reset@font added and logic changed.}
% \changes{v2.1d}{92/10/28}{Reset to empty if arg is empty}
%    \begin{macrocode}
\gdef\theorembodyfont#1{%
%    \end{macrocode}
%    We check if the argument supplied is empty and if so put nothing
%    into the |\theorem@bodyfont| token register to allow for
%    |\theorembodyfont{}| as a mean of using the default of the
%    current |\theoremstyle|.
%    \begin{macrocode}
   \def\@tempa{#1}%
   \ifx\@tempa\@empty
    \theorem@bodyfont{}%
   \else
    \theorem@bodyfont{\reset@font#1}%
   \fi
   }
\newtoks\theorem@bodyfont
\global\theorem@bodyfont{}
%    \end{macrocode}
% \end{macro}
% \end{macro}
%
%
% \begin{macro}{\theoremheaderfont}
% \changes{v2.0g}{89/09/19}{Macro added.}
%    The font for the theorem headers is handled differently because
%    this definition applies to all theorem styles.
%    \begin{macrocode}
\gdef\theoremheaderfont#1{\gdef\theorem@headerfont{#1}%
%    \end{macrocode}
%    After using the macro once it is redefined to produce an error
%    message.
%    \begin{macrocode}
       \gdef\theoremheaderfont##1{%
        \typeout{\string\theoremheaderfont\space should be used
                 only once.}}}
%    \end{macrocode}
% \end{macro}
%
% \begin{macro}{\theorem@headerfont}
% \changes{v2.0g}{89/09/19}{Macro added.}
%    To set the |\theorem@headerfont| default we first test if
%    the new fontselection scheme is in force.
% \changes{v2.1a}{90/02/15}{Support for new font selection.}
%    \begin{macrocode}
\ifx\upshape\undefined
%    \end{macrocode}
%    If not we define it to expand into |\bfseries|. We don't use
%    |\let| just in case a following style option redefines this
%    macro.
%    \begin{macrocode}
\gdef\theorem@headerfont{\bfseries}
%    \end{macrocode}
%    Otherwise we reset the current shape before calling |\bfseries|.
%    \begin{macrocode}
\else \gdef\theorem@headerfont{\normalfont\bfseries}\fi
%    \end{macrocode}
% \end{macro}
%
% \begin{macro}{\th@plain}
% \begin{macro}{\th@break}
% \begin{macro}{\th@marginbreak}
% \begin{macro}{\th@changebreak}
% \begin{macro}{\th@change}
% \begin{macro}{\th@margin}
%    The different styles are defined in macros such as
%    |\th@plain|.  Since memory space is precious in
%    ``non-Big-versions'', we have to avoid offering too many unused
%    definitions.  Therefore we define these styles in separate files
%    that can be loaded on demand.  Thus the commands themselves only
%    load these files. We use |\@input@|, a \LaTeXe{} internal command
%    that ensures that the file will be listed with |\listfiles|.
% \changes{v2.2a}{94/02/02}{Use `@input@ to load theorem layout files}
%    \begin{macrocode}
\gdef\th@plain{\@input@{thp.sty}}
\gdef\th@break{\@input@{thb.sty}}
\gdef\th@marginbreak{\@input@{thmb.sty}}
\gdef\th@changebreak{\@input@{thcb.sty}}
\gdef\th@change{\@input@{thc.sty}}
\gdef\th@margin{\@input@{thm.sty}}
%    \end{macrocode}
%    This list will be expanded when new styles become
%    available.  For testing,  just append new theorem substyles
%    as document options.
% \end{macro}
% \end{macro}
% \end{macro}
% \end{macro}
% \end{macro}
% \end{macro}
%
%
%
% \subsection{Definition of a new theorem set}
%
% As already pointed out, a new theorem environment can be defined in
% three different ways:
% \begin{verbatim}
%    \newtheorem{Lem}{Lemma}
%    \newtheorem{Lem}{Lemma}[section]
%    \newtheorem{Lem}[Theorem]{Lemma}
%\end{verbatim}
% The function of the macro |\newtheorem| is to recognize these cases
% and then to branch into one of the three macros |\@ynthm|, |\@xnthm|
% or |\@othm|.  This mechanism is adopted unchanged from
% \cite{latex:source}; the essential point here is that, for example,
% in the second case, the arguments \texttt{Lem}, \texttt{Lemma} and
% \texttt{section} are passed over to the macro |\@xnthm|.
%
% We inspect this case first because the others present fewer problems,
% and thus are easily derived from this one.
%
% \begin{macro}{\@xnthm}
%    For our example arguments, the macro |\@xnthm| must fulfill
%    the following:
%    \begin{itemize}
%       \item
%          Define a new \LaTeX-counter `Lem'
%       \item
%          reset this counter within a |\section|
%       \item
%          define the macro |\theLem|
%       \item
%          define the environment macros |\Lem| and
%          |\endLem| using the current
%          |\theoremstyle| and |\theorem@bodyfont|.
%    \end{itemize}
%    Obviously, all this should happen only if the first argument
%    of |\@xnthm| (i.e.\ \texttt{Lem} in our example) is chosen so
%    as not to conflict with any previously defined commands or
%    environments.  This test is performed by the \LaTeX{} macro
%    |\@ifdefinable|.
%    \begin{macrocode}
\gdef\@xnthm#1#2[#3]{\expandafter\@ifdefinable\csname #1\endcsname
%    \end{macrocode}
%    Therefore, the first argument of |\@ifdefinable| is the
%    expansion (in the example, |\Lem|) of
%    |\csname||#1\endcsname|.  The second argument is
%    executed only if the test has been completed successfully.
%    \begin{macrocode}
   {%
%    \end{macrocode}
%    Now we define the new counter. The names of the \LaTeX{}
%    macros employed should speak for themselves:
% \changes{v2.2b}{1995/10/07}{Use \cs{@newctr}}
% \changes{v2.2c}{1995/11/23}{Use \cs{@newctr} properly.}
%    \begin{macrocode}
    \@definecounter{#1}\@newctr{#1}[#3]%
%    \end{macrocode}
%    Using |\@newctr| will give a proper error message if the counter
%    in |#3| is not defined.
%    In defining `|\theLem|' we must generate the desired macro
%    name by use of |\expandafter| and |\csname|.
%    \begin{macrocode}
    \expandafter\xdef\csname the#1\endcsname
%    \end{macrocode}
%    An |\xdef| is used in order to make the definition global,
%    and to ensure that it contains the replacement texts of
%    |\@thmcountersep| and |\@thmcounter|.\footnote{These
%    two macros can be defined by the document style.  Their default
%    values produce a `\texttt{.}' as separation and an arabic
%    representation of the number.} However, not everything should be
%    expanded. For example, it saves space to use |\thesection|
%    instead of its---at times---lengthy expansion.
%    \begin{macrocode}
      {\expandafter \noexpand \csname the#3\endcsname
       \@thmcountersep \@thmcounter{#1}}%
%    \end{macrocode}
%    Thus with the defaults of \LaTeX{}, |\theLem| would be
%    replaced by the command sequence |\thesection.\arabic{Lem}|.
%
%    We will now look at the definition of the macro which is executed
%    at the beginning of the actual environment (in our example this
%    macro is |\Lem|). It should be noted that we use an
%    ``|\expandafter| trick'' to expand only certain parts of the
%    replacement text at the time of the definition.
%    \begin{macrocode}
    \def\@tempa{\global\@namedef{#1}}%
    \expandafter \@tempa \expandafter{%
%    \end{macrocode}
%    First, the macro that contains the current definitions of
%    |\@begintheorem| and |\@opargtheorem| should be called
%    up.  The name of this macro---as is already known---has the form
%    |\th@|\meta{theorem\ style}; therefore, it must be called by
%    \begin{macrocode}
      \csname th@\the \theorem@style
%    \end{macrocode}
%    In addition the default theorem font should be changeable, i.e.\
%    we have to insert the contents of |\theorem@bodyfont|. For
%    that reason, we expand even further, beyond |\endcsname|,
%    and thus insert the contents of the \textsf{token} register
%    |\theorem@bodyfont| in the replacement text.
%    \begin{macrocode}
            \expandafter \endcsname \the \theorem@bodyfont
%    \end{macrocode}
%    Now it is time to call the macro |\@thm| which takes over
%    the further processing.  It has two arguments: the current
%    counter name (in our example, \texttt{Lem}), and the text of the
%    label (in our example, \texttt{Lemma}).
%    \begin{macrocode}
     \@thm{#1}{#2}}%
%    \end{macrocode}
%    With this, the `sub-definition' is complete.  The macro
%    |\@endtheorem| ends a theorem environment and is, so far, nothing
%    but an |\endtrivlist|.  (Hence it is defined globally, and not
%    within the theorem styles.\footnote{This has to be changed as
%    soon as theorem styles that change \texttt{\bslash @endtheorem}
%    exist.  In such a case, all existing styles must be changed as
%    well since they will have to reset the macro.}) Therefore, we can
%    set it equivalent to the macro that ends the theorem set (in our
%    example, |\endLem|).  However, if some day theorem styles exist
%    that do change |\@endtheorem|, we would have to use the
%    commented-out line instead.
%    \begin{macrocode}
    \global \expandafter \let \csname end#1\endcsname \@endtheorem
%   \global\@namedef{end#1}{\@endtheorem}%
%    \end{macrocode}
%    With these commands all the required definitions are employed,
%    unless the test |\@ifdefinable| has failed.  Therefore, we
%    end the second argument of this macro and with it the definition
%    of |\@xnthm|.
%    \begin{macrocode}
   }}
%    \end{macrocode}
% \end{macro}
%
% \begin{macro}{\@ynthm}
%    The definition of |\@ynthm| is completely analogous. In
%    this case the new counter that is defined is not reset within
%    another counter; thus the definition of |\the...|
%    is simplified:
%    \begin{macrocode}
\gdef\@ynthm#1#2{\expandafter\@ifdefinable\csname #1\endcsname
   {\@definecounter{#1}%
    \expandafter\xdef\csname the#1\endcsname{\@thmcounter{#1}}%
%    \end{macrocode}
%    The rest of the definition corresponds literally to that of
%    |\@xnthm|:
%    \begin{macrocode}
    \def\@tempa{\global\@namedef{#1}}\expandafter \@tempa
     \expandafter{\csname th@\the \theorem@style \expandafter
     \endcsname \the\theorem@bodyfont \@thm{#1}{#2}}%
    \global \expandafter \let \csname end#1\endcsname \@endtheorem}}
%    \end{macrocode}
% \end{macro}
%
% \begin{macro}{\@othm}
%    The definition of |\@othm| does not contain anything new.
%    \begin{macrocode}
\gdef\@othm#1[#2]#3{%
%    \end{macrocode}
% \changes{v2.2b}{95/10/07}{Build in check for counter defined}
%    We do not define a new counter but instead use one that has
%    already been defined.  Thus the only definition we need is that
%    of this pseudo-counter (i.e.\ |\the|\meta{env.\ name}).
%    First we check if |#2| corresponds to a known counter name.
%    \begin{macrocode}
  \expandafter\ifx\csname c@#2\endcsname\relax
   \@nocounterr{#2}%
  \else
   \expandafter\@ifdefinable\csname #1\endcsname
   {\expandafter \xdef \csname the#1\endcsname
     {\expandafter \noexpand \csname the#2\endcsname}%
%    \end{macrocode}
%    All other parts of the definition can be adopted from
%    |\@xnthm|. We have to remember, though, that in this case
%    the name of the current counter and the theorem label have moved
%    to the second and third arguments.
%    \begin{macrocode}
    \def\@tempa{\global\@namedef{#1}}\expandafter \@tempa
     \expandafter{\csname th@\the \theorem@style \expandafter
     \endcsname \the\theorem@bodyfont \@thm{#2}{#3}}%
    \global \expandafter \let \csname end#1\endcsname \@endtheorem}%
  \fi}
%    \end{macrocode}
% \end{macro}
%
% \subsection{Macros that are employed in a {\envfont theorem}
%    environment}
%
% \begin{macro}{\@thm}
%    The macro |\@thm| has to increase the current counter.  Then,
%    depending on whether the environment has (or does not have) an
%    optional argument, it has to branch into either |\@begintheorem|
%    or |\@opargtheorem|.
%    \begin{macrocode}
\gdef\@thm#1#2{\refstepcounter{#1}%
%    \end{macrocode}
%    Now we start a {\envfont trivlist} environment, and give
%    |\@topsep| and |\@topsepadd| the values of the \textsf{skip}
%    registers |\theorempreskipamount| and |\theorempostskipamount|.
%    The value in |\@topsep| is the vertical space that is inserted by
%    the first (and only) |\item| in our |\trivlist| whilst
%    |\@topsepadd| is inserted by |\@endparenv| at the end of that
%    {\envfont trivlist} environment.  By using these registers, we
%    obtain the desired space around a {\envfont theorem} environment.
%    \begin{macrocode}
   \trivlist
   \@topsep \theorempreskipamount               % used by first \item
   \@topsepadd \theorempostskipamount           % used by \@endparenv
%    \end{macrocode}
%    Now we have to test whether an optional argument has been given.
% \changes{v2.0d}{89/07/18}{Forgotten percent added.}
%    \begin{macrocode}
   \@ifnextchar [%
%    \end{macrocode}
%    If there is an optional argument, we will call |\@ythm|, and
%    move the arguments read back into the input stream.
%    \begin{macrocode}
   {\@ythm{#1}{#2}}%
%    \end{macrocode}
%    If not, we call |\@begintheorem|. Its first argument is the
%    name of the theorem set (hence the second argument of
%    |\@thm|). Its second argument is the macro that produces the
%    current number.
%    \begin{macrocode}
   {\@begintheorem{#2}{\csname the#1\endcsname}\ignorespaces}}
%    \end{macrocode}
% \end{macro}
%
% \begin{macro}{\@xthm}
% \begin{macro}{\@ythm}
%    Both these macros were originally called by |\@thm|.  We do
%    not need |\@xthm| anymore, hence we reset it to
%    |\relax|. The definition of |\@ythm| has not changed at
%    all from its definition in \LaTeX. In order to make the macros
%    easier to understand, we will
%    nevertheless present it (commented out).
%    \begin{macrocode}
\global\let\@xthm\relax
% \def\@ythm#1#2[#3]{\@opargbegintheorem{#2}{\csname
%           the#1\endcsname}{#3}\ignorespaces}
%</package>
%    \end{macrocode}
%    The primitive |\ignorespaces| in |\@ythm| and
%    |\@thm| is needed to remove the spaces between the
%    |\begin{...}| and the actual text.
% \end{macro}
% \end{macro}
%
%
% \subsection{Definition of the theorem substyles}
%
% As already pointed out, the theorem substyles, defined below, are
% only loaded when necessary.  Note that all these substyles, except
% \texttt{plain}, have |\slshape| as the default body font.
%
% \subsubsection{The \texttt{plain} style}
%
% \label{dok:th@plain} As the following macros use |@|, we have
% to locally set the |\catcode| of this symbol to ``letter''.
% This happens within a group, so that we do not have to worry about
% which |\catcode| that symbol had before.
%    \begin{macrocode}
%<*thp>
\begingroup \makeatletter
%    \end{macrocode}
% Since we are now within a group, we must make all definitions
% globally.  First we make sure that \texttt{theorem.sty} is loaded.
% This will allow us to use this file as a document style option
% without having to call \texttt{theorem} itself as an option.  At the
% same time, we assure that at least version 2 is loaded, since
% |\theorem@style| was not defined in earlier versions.
%    \begin{macrocode}
\@ifundefined{theorem@style}{% \iffalse meta-comment
%
% Copyright (C) 1993-2024
%
% The LaTeX Project and any individual authors listed elsewhere
% in this file.
%
% This file is part of the Standard LaTeX `Tools Bundle'.
% -------------------------------------------------------
%
% It may be distributed and/or modified under the
% conditions of the LaTeX Project Public License, either version 1.3c
% of this license or (at your option) any later version.
% The latest version of this license is in
%    https://www.latex-project.org/lppl.txt
% and version 1.3c or later is part of all distributions of LaTeX
% version 2005/12/01 or later.
%
% The list of all files belonging to the LaTeX `Tools Bundle' is
% given in the file `manifest.txt'.
%
% \fi
%
%%
%
%
\def\FMithmInfo{2023/07/05 v2.2c Theorem extension package (FMi)}
%
% \ProvidesFile{theorem.dtx}[\FMithmInfo]
%\iffalse   % this is a METACOMMENT !
%%
%% Package `theorem' to use with LaTeX2e
%% Copyright (C) 1989-1995 Frank Mittelbach, all rights reserved.
%<+package>\@ifundefined{theorem@style}{}{\endinput}
%<+package>\NeedsTeXFormat{LaTeX2e}[1995/06/01]
%<+package>\ProvidesPackage{theorem}
%<+package>         [\FMithmInfo]
%
% \fi
%
% \changes{v1.0}{87/05/07}{First draft.}
% \changes{v1.0a}{87/05/07}{`thistheoremstyle' added.}
% \changes{v1.1a}{87/07/29}{New concept: redefine the internal
%                          \protect\LaTeX{} commands `@xnthm' etc.}
% \changes{v1.1b}{87/07/31}{`theoremfont' used to specify font.}
% \changes{v1.2a}{88/01/08}{`@endtheorem' now globally defined.}
% \changes{v1.2b}{88/01/11}{`newtheorem' only allowed in the preamble.}
% \changes{v1.3a}{88/02/14}{`theoremfont' now a toks register.}
%
% \changes{v2.0a}{88/11/16}{New implementation and new documentation.}
% \changes{v2.0a}{88/11/16}{No layout changes but internal modifications
%                           old substyles won't work.}
% \changes{v2.0b}{89/01/30}{`global' used for all definitions.}
% \changes{v2.0c}{89/06/15}{Documentation translated to English.
%                            Thanks to LLL and Nina}
% \changes{v2.1a}{90/02/15}{Support for new font selection.}
% \changes{v2.2a}{94/02/03}{Cleanup for LaTeX2e}
%
% \DoNotIndex{\@endparenv,\@ifdefinable,\@ifnextchar,\@ifundefined}
% \DoNotIndex{\@namedef,\@nameuse,\@spaces,\@tempa,\@warning}
% \DoNotIndex{\ }
% \DoNotIndex{\begingroup,\bfseries}
% \DoNotIndex{\csname}
% \DoNotIndex{\def,\do,\docdate}
% \DoNotIndex{\endcsname,\endgroup,\endinput,\endtrivlist,\expandafter}
% \DoNotIndex{\filedate,\fileversion}
% \DoNotIndex{\gdef,\global}
% \DoNotIndex{\hbox,\hskip}
% \DoNotIndex{\ignorespaces,\input,\itshape,\item}
% \DoNotIndex{\labelsep,\let,\llap}
% \DoNotIndex{\makeatletter}
% \DoNotIndex{\noexpand}
% \DoNotIndex{\refstepcounter,\relax,\rlap}
% \DoNotIndex{\setlength,\slshape,\space,\string,\strut}
% \DoNotIndex{\the,\trivlist,\typeout}
% \DoNotIndex{\vbox}
% \DoNotIndex{\xdef}
%
% \def\deflabel#1{\ttfamily #1\hfill}
% \def\deflist#1{\begin{list}{}{\settowidth\labelwidth{\ttfamily #1}%
%                               \setlength\leftmargin\labelwidth
%                               \addtolength\leftmargin\labelsep
%                               \let\makelabel\deflabel}}
% \def\enddeflist{\end{list}}
% \def\envfont{\normalfont\ttfamily}
%
% \MakeShortVerb{\|}
%
% \newcommand{\TUB}{{\sl TUGboat\/}}
%
% \GetFileInfo{theorem.dtx}
% \title{An Extension of the
%        \LaTeX{} theorem environment\thanks{This file
%        has version number \fileversion, last
%        revised \filedate.}}
% \author{Frank Mittelbach}
%
% \MaintainedByLaTeXTeam{tools}
% \maketitle
%
% \begin{abstract}
%    The macros described in this paper yield an extension of the
%    \LaTeX{} theorem mechanism. It is designed to satisfy the
%    different requirements of various journals. Thus, the
%    layout of the ``theorems'' can be manipulated by determining a
%    ``style''. This article describes not only the use, but
%    also the definition, of the necessary macros.
% \end{abstract}
%
% \section*{Preface to version 2.2}
%
% For \LaTeXe{} this package did not need any fundamental changes. I
% only modified the messages generated so that theorem layout styles
% will show up with the |\listfiles| command and cleaned the section
% on the New Font Selection Scheme since this is now included in
% \LaTeX.
%
% \section*{Preface to version 2.1}
%
% This version is identical to 2.0g described in \TUB\ 10\#3 except
% for some internal defaults which are now set depending on the used
% font selection scheme.
%
% This was done to avoid unpleasant surprises if the new font selection
% scheme is in force. For further details see
% section~\ref{sec:newfonts} and \cite{bk:GMS94}.
%
% \section{Introduction}
%
% For our purposes here, ``theorems'' are labelled enunciations,
% often set off from the main text by extra space and a font change.
% Theorems, corollaries, conjectures, definitions, and remarks
% are all instances of ``theorems''.  The ``header'' of these structures
% is composed of a label (such as \textsc{Theorem} or \textsc{Remark})
% and a number which serializes an item in the sequence of items
% with the same label.
%
% Shortly after the introduction of \LaTeX{} at the Fachbereich
% Mathematik in Mainz, the desire to manipulate the layout of
% ``theorems'' arose.  In Mainz, the following two conventions came
% into general use:
% \begin{enumerate}
%   \item The number of the theorem is shown in the margin.
%   \item There is a line break at
%         the end of the theorem header.
% \end{enumerate}
%
% Additionally, some journals require different formats which depend on
% the ``sort of theorem'': e.g.\ often remarks and definitions are set
% in |\upshape|, while |\itshape| is employed for main theorems.
%
% Confronted with these requirements, a theorem environment was
% developed in Mainz which allows separate determination of the
% layout of the ``theorems sets'', comparable to |\pagestyle|.
%
%
% \section{The user interface}
% \subsection{Defining new theorem sets}
%
% \DescribeMacro\newtheorem
% As in the original \LaTeX{} version, the command |\newtheorem|
% defines a new ``theorem set'' or ``theorem-like structure''.
% Two required arguments name the new environment and give the text
% to be typeset with each instance of the new ``set'', while
% an optional argument determines how
% the ``set'' is enumerated:
% \begin{description}
%    \item[\ttfamily \bslash newtheorem\{foo\}\{bar\}]
%       The theorem set {\envfont foo} (whose name is \texttt{bar})
%       uses its own counter.
%    \item[\ttfamily \bslash newtheorem\{foo2\}{[foo]}\{bar2\}]
%       The theorem set {\envfont foo2} (printed name \texttt{bar2})
%       uses the same counter as the theorem set \texttt{foo}.
%    \item[\ttfamily \bslash newtheorem\{foo3\}\{bar3\}{[section]}]
%       The theorem set {\envfont foo3} (printed name \texttt{bar3}) is
%       enumerated within the counter \texttt{section}, i.e.\ with every
%       new |\section| the enumeration begins again with 1, and
%       the enumeration is composed from the section-number and the
%       theorem counter itself.
% \end{description}
%
% \DescribeMacro\theoremstyle
% Additionally, the command |\theoremstyle| can define the layout
% of various, or all, theorem sets.  It should be noted that any
% theorem set defined by |\newtheorem| is typeset in the
% |\theoremstyle| that is current at the time of the definition.
% Thus, the following
% \begin{verbatim}
%    \theoremstyle{break}        \newtheorem{Cor}{Corollary}
%    \theoremstyle{plain}        \newtheorem{Exa}{Example}[section]
%\end{verbatim}                             ^^A No blank at line start
% leads to the result that the set {\envfont Cor} is formatted in the
% style \texttt{break}, while the set {\envfont Exa} and all the
% following ones are formatted in the style \texttt{plain}, unless
% another |\theoremstyle| follows.  Since the definitions installed by
% |\newtheorem| are global, one also can limit |\theoremstyle| locally
% by grouping braces.
%
% \DescribeMacro\theorembodyfont
% The choice of the font for the theorem body is completely
% independent of the chosen |\theoremstyle|; this has proven to
% be very advantageous.  For example,
% \begin{verbatim}
%   {\theorembodyfont{\upshape}          \newtheorem{Rem}{Remark}}
%\end{verbatim}
% defines a theorem set {\envfont Rem}, which will be set in
% |\upshape| in the current layout (which in our example is
% \texttt{plain}).  As with |\theoremstyle|, the |\theorembodyfont|
% chosen is that current at the time of |\newtheorem|.  If
% |\theorembodyfont| is not specified or one defines
% |\theorembodyfont{}|, then the font used will be that defined by the
% |\theoremstyle|.
%
% \DescribeMacro\theoremheaderfont It is also possible to customize
% the font used for the theorem headers.  This is, however, a global
% declaration, and therefore there should be at most one
% |\theoremheaderfont| declaration in the preamble.\footnote{If it is
% actually necessary to have different header fonts, one has to define
% new theorem styles (substituting the desired font) or specify the
% information directly in the \texttt{\bslash newtheorem} declaration
% (the unclean variant).}
%
% \DescribeMacro\theorempreskipamount
% \DescribeMacro\theorempostskipamount Two additional parameters
% affect the vertical space around the theorem environments:
% |\theorempreskipamount| and |\theorempostskipamount| define,
% respectively, the spacing before and after such an environment.
% These parameters apply for all theorem sets and can be manipulated
% with the ordinary length macros.  They are rubber lengths
% (`\textsf{skips}'), and therefore can contain \texttt{plus} and
% \texttt{minus} parts.
%
% Since the definition of theorem sets should---most sensibly---be
% placed in the preamble, we only allow installation there.
% It is therefore possible to release the memory used here after
% |\begin{document}|, in order to make room for other
% applications.
%
%
% \subsection{Existing theorem styles}
%
% The following theorem styles exist to date:
% \begin{deflist}{marginbreak}
%    \item[plain]
%       This theorem style emulates the original \LaTeX{} definition,
%       except that additionally the parameters
%       |\theorem...skipamount| are used.
%    \item[break]
%       In this style, the theorem header is followed by a line
%       break.
%    \item[marginbreak]
%       The theorem number is set in the margin, and there is a line
%       break as in \texttt{break}.
%    \item[changebreak]
%       Like \texttt{break}, but with header number and text
%       interchanged.
%    \item[change]
%       Header number and text are interchanged, without a line break.
%    \item[margin]
%       The number is set in the left margin, without a line break.
% \end{deflist}
% All styles (except \texttt{plain}) select |\slshape| as the default
% |\theorembodyfont|.
%
%
% \subsection{Examples}
%
% Given the above theorem sets {\envfont Cor}, {\envfont Exa}
% and {\envfont Rem},
% suppose that the preamble also contains the declarations:
% \begin{verbatim}
%    \theoremstyle{marginbreak}   \newtheorem{Lem}[Cor]{Lemma}
%    \theoremstyle{change}
%    \theorembodyfont{\itshape}        \newtheorem{Def}[Cor]{Definition}
%
%    \theoremheaderfont{\scshape}
%\end{verbatim}
% Then the following are some typical examples of the typeset output
% resulting from their use.
%
% \begin{Cor}
%    This is a sentence typeset in the theorem environment {\envfont
%    Cor}.
% \end{Cor}
%
% \begin{Exa}
%    This is a sentence typeset in the theorem environment {\envfont
%    Exa}.
% \end{Exa}
%
% \begin{Rem}
%    This is a sentence typeset in the theorem environment {\envfont
%    Rem}.
% \end{Rem}
%
% \begin{Lem}[Ben User]
%    This is a sentence typeset in the theorem environment {\envfont
%    Lem}.
% \end{Lem}
%
% \begin{Def}[Very impressive Definition]
%    This is a sentence typeset in the theorem environment {\envfont
%    Def}.
% \end{Def}
%
% The last two examples show the effect of the optional argument to
% a theorem environment (it is the text typeset in parentheses).
%
%
% \section{Special Considerations}
% \label{sec:newfonts}
%
% Theoremheader and body are implemented as a unit. This means that
% the |\theoremheaderfont|  will inherit characteristics of the
% |\theorembodyfont| in \LaTeXe. Thus, if for example
% |\theorembodyfont| is |\itshape| and |\theoremheaderfont|
% is |\bfseries| the font selected for the header will have the
% characteristics `bold extended italic'. If this is not desired one
% should set the |\theoremheaderfont| to something like
% \begin{verbatim}
%      \theoremheaderfont{\normalfont\bfseries}
%\end{verbatim}
% i.e.\ supplying all necessary font information explicitly.
%
%
%
% \section{Acknowledgements}
%
% The publication of this set of macros was only possible with the
% help of Christina Busse (translating the manuscript into English),
% Joachim Pense (playing the r\^ole of typist), Chris Rowley
% (looking everything over) and many others providing useful
% suggestions.
%
% \MaybeStop{%
%    \begin{thebibliography}{9}
%       \bibitem{bk:GMS94} \textsc{M.~Goossens}, \textsc{F.~Mittelbach}
%         and \textsc{A.~Samarin}.
%         \newblock The \LaTeX{} Companion.
%         \newblock
%         Addison-Wesley, Reading, Massachusetts, 1994.
%       \bibitem{latex:source} \textsc{Lamport, Leslie.} \newblock
%                               \texttt{latex.tex}, version 2.09, date
%                               Feb. 1990.
%    \end{thebibliography}
%    \PrintIndex
%    \PrintChanges}
%
%
% \section{The documentation driver file}
%
% The next bit of code contains the documentation driver file for
% \TeX{}, i.e., the file that will produce the documentation you are
% currently reading. It will be extracted from this file by the
% \texttt{docstrip} program. Since it is the first code in the file
% one can alternatively process this file directly with \LaTeXe{} to
% obtain the documentation.
%
%    \begin{macrocode}
%<*driver>
\documentclass{ltxdoc}

\usepackage{theorem}

%<+driver>% The next few lines define theorem sets which are used
%<+driver>% in the example section of the documentation.

\theoremstyle{break}       \newtheorem{Cor}{Corollary}
\theoremstyle{plain}       \newtheorem{Exa}{Example}[section]
{\theorembodyfont{\upshape}\newtheorem{Rem}{Remark}}
\theoremstyle{marginbreak} \newtheorem{Lem}[Cor]{Lemma}
\theoremstyle{change}
\theorembodyfont{\itshape} \newtheorem{Def}[Cor]{Definition}

\theoremheaderfont{\scshape}

\RecordChanges

\begin{document}
   \DocInput{theorem.dtx}
\end{document}
%</driver>
%    \end{macrocode}
%
% \section{Definition of the Macros}
%
% If the file has been loaded before, we abort immediately. If not the
% package announces itself (this is actually done at the very top of
% the file---the way it is done isn't good style so don't copy it).
% \changes{v2.0e}{89/07/19}{Spaces removed from `typeout'.}
% \changes{v2.2b}{95/11/19}{Announce moved to top of file}
%    \begin{macrocode}
%<*package>
%\@ifundefined{theorem@style}{}{\endinput}
%\def\FMithmInfo{1995/11/19 v2.2b Theorem extension package (FMi)}
%\ProvidesPackage{theorem}[\FMithmInfo]
%    \end{macrocode}
%
%
% \subsection{Definition of theorem styles and fonts}
%
% All the definitions in this file are done globally to allow
% inputting this file inside a group.
%
% \begin{macro}{\theoremstyle}
%    Before a theorem style can be installed, the chosen style must be
%    known. For that reason, we must test to see that
%    |\th@|\meta{style} is known or, more precisely, that it is
%    different from |\relax|. If the style is not known then
%    |\th@plain| is used.
%    \begin{macrocode}
\gdef\theoremstyle#1{%
   \@ifundefined{th@#1}{\@warning
          {Unknown theoremstyle `#1'. Using `plain'}%
          \theorem@style{plain}}%
%    \end{macrocode}
%    We save the theorem style to be used in the token register
%    |\theorem@style|.
%    \begin{macrocode}
      {\theorem@style{#1}}%
%    \end{macrocode}
%    Now we ``evaluate'' the theorem style: this means, we call the
%    macro |\th@|\meta{style} which will activate the relevant
%    definitions which are contained in a separate file.  This is done
%    in a group to suppress changes to the current font.  This could
%    otherwise pose problems together with the new font selection
%    scheme\footnote{When I printed the original article using the new
%    font selection scheme I ended with a document with slanted
%    typefaces (text headings and all) simply because one of the
%    theorem styles used \texttt{\bslash sl} at toplevel.} if the
%    |\th@|\meta{style} is evaluated a second time.
% \changes{v2.1a}{90/02/15}{Group added around `th@..}
% \changes{v2.1a}{90/02/15}{Reseting of `..@begintheorem removed.}
%    \begin{macrocode}
      \begingroup
        \csname th@\the\theorem@style \endcsname
      \endgroup}
%    \end{macrocode}
% \end{macro}
%
%
% \begin{macro}{\@begintheorem}
% \changes{v2.1a}{90/02/15}{Relaxed at toplevel.}
% \begin{macro}{\@opargbegintheorem}
% \changes{v2.1a}{90/02/15}{Relaxed at toplevel.}
%    We reset |\@begintheorem| and |\@opargbegintheorem| to
%    |\relax| since these commands are no longer necessary at
%    toplevel. This will save a few tokens.
%    \begin{macrocode}
\global\let\@begintheorem\relax
\global\let\@opargbegintheorem\relax
%    \end{macrocode}
% \end{macro}
% \end{macro}
%
% \begin{macro}{\theorem@style}
%    Obviously the \textsf{token} register used above has to be
%    allocated. To assure the utmost compatibility with the
%    original \LaTeX{} definition, we set the default theorem style to
%    \texttt{plain}, which implements the usual \LaTeX{} convention.
% \changes{v2.0f}{89/07/20}{Unnecessary `global' removed.}
%    \begin{macrocode}
\newtoks\theorem@style
\global\theorem@style{plain}
%    \end{macrocode}
% \end{macro}
%
% \begin{macro}{\theorembodyfont}
% \begin{macro}{\theorem@bodyfont}
%    If the |\theorembodyfont| is set by the user then it should not
%    interact with the default font set in the theorem style. When the
%    new font selection is in force this may happen if, for example,
%    the default is |\itshape| and the new |\theorembodyfont| is
%    |\sffamily|. So we add a |\reset@font| command in front of the
%    user definition.
% \changes{v2.0f}{89/07/20}{Unnecessary `global' removed.}
% \changes{v2.0g}{89/09/19}{`theoremfont to `theorembodyfont.}
% \changes{v2.1c}{92/06/22}{`reset@font added and logic changed.}
% \changes{v2.1d}{92/10/28}{Reset to empty if arg is empty}
%    \begin{macrocode}
\gdef\theorembodyfont#1{%
%    \end{macrocode}
%    We check if the argument supplied is empty and if so put nothing
%    into the |\theorem@bodyfont| token register to allow for
%    |\theorembodyfont{}| as a mean of using the default of the
%    current |\theoremstyle|.
%    \begin{macrocode}
   \def\@tempa{#1}%
   \ifx\@tempa\@empty
    \theorem@bodyfont{}%
   \else
    \theorem@bodyfont{\reset@font#1}%
   \fi
   }
\newtoks\theorem@bodyfont
\global\theorem@bodyfont{}
%    \end{macrocode}
% \end{macro}
% \end{macro}
%
%
% \begin{macro}{\theoremheaderfont}
% \changes{v2.0g}{89/09/19}{Macro added.}
%    The font for the theorem headers is handled differently because
%    this definition applies to all theorem styles.
%    \begin{macrocode}
\gdef\theoremheaderfont#1{\gdef\theorem@headerfont{#1}%
%    \end{macrocode}
%    After using the macro once it is redefined to produce an error
%    message.
%    \begin{macrocode}
       \gdef\theoremheaderfont##1{%
        \typeout{\string\theoremheaderfont\space should be used
                 only once.}}}
%    \end{macrocode}
% \end{macro}
%
% \begin{macro}{\theorem@headerfont}
% \changes{v2.0g}{89/09/19}{Macro added.}
%    To set the |\theorem@headerfont| default we first test if
%    the new fontselection scheme is in force.
% \changes{v2.1a}{90/02/15}{Support for new font selection.}
%    \begin{macrocode}
\ifx\upshape\undefined
%    \end{macrocode}
%    If not we define it to expand into |\bfseries|. We don't use
%    |\let| just in case a following style option redefines this
%    macro.
%    \begin{macrocode}
\gdef\theorem@headerfont{\bfseries}
%    \end{macrocode}
%    Otherwise we reset the current shape before calling |\bfseries|.
%    \begin{macrocode}
\else \gdef\theorem@headerfont{\normalfont\bfseries}\fi
%    \end{macrocode}
% \end{macro}
%
% \begin{macro}{\th@plain}
% \begin{macro}{\th@break}
% \begin{macro}{\th@marginbreak}
% \begin{macro}{\th@changebreak}
% \begin{macro}{\th@change}
% \begin{macro}{\th@margin}
%    The different styles are defined in macros such as
%    |\th@plain|.  Since memory space is precious in
%    ``non-Big-versions'', we have to avoid offering too many unused
%    definitions.  Therefore we define these styles in separate files
%    that can be loaded on demand.  Thus the commands themselves only
%    load these files. We use |\@input@|, a \LaTeXe{} internal command
%    that ensures that the file will be listed with |\listfiles|.
% \changes{v2.2a}{94/02/02}{Use `@input@ to load theorem layout files}
%    \begin{macrocode}
\gdef\th@plain{\@input@{thp.sty}}
\gdef\th@break{\@input@{thb.sty}}
\gdef\th@marginbreak{\@input@{thmb.sty}}
\gdef\th@changebreak{\@input@{thcb.sty}}
\gdef\th@change{\@input@{thc.sty}}
\gdef\th@margin{\@input@{thm.sty}}
%    \end{macrocode}
%    This list will be expanded when new styles become
%    available.  For testing,  just append new theorem substyles
%    as document options.
% \end{macro}
% \end{macro}
% \end{macro}
% \end{macro}
% \end{macro}
% \end{macro}
%
%
%
% \subsection{Definition of a new theorem set}
%
% As already pointed out, a new theorem environment can be defined in
% three different ways:
% \begin{verbatim}
%    \newtheorem{Lem}{Lemma}
%    \newtheorem{Lem}{Lemma}[section]
%    \newtheorem{Lem}[Theorem]{Lemma}
%\end{verbatim}
% The function of the macro |\newtheorem| is to recognize these cases
% and then to branch into one of the three macros |\@ynthm|, |\@xnthm|
% or |\@othm|.  This mechanism is adopted unchanged from
% \cite{latex:source}; the essential point here is that, for example,
% in the second case, the arguments \texttt{Lem}, \texttt{Lemma} and
% \texttt{section} are passed over to the macro |\@xnthm|.
%
% We inspect this case first because the others present fewer problems,
% and thus are easily derived from this one.
%
% \begin{macro}{\@xnthm}
%    For our example arguments, the macro |\@xnthm| must fulfill
%    the following:
%    \begin{itemize}
%       \item
%          Define a new \LaTeX-counter `Lem'
%       \item
%          reset this counter within a |\section|
%       \item
%          define the macro |\theLem|
%       \item
%          define the environment macros |\Lem| and
%          |\endLem| using the current
%          |\theoremstyle| and |\theorem@bodyfont|.
%    \end{itemize}
%    Obviously, all this should happen only if the first argument
%    of |\@xnthm| (i.e.\ \texttt{Lem} in our example) is chosen so
%    as not to conflict with any previously defined commands or
%    environments.  This test is performed by the \LaTeX{} macro
%    |\@ifdefinable|.
%    \begin{macrocode}
\gdef\@xnthm#1#2[#3]{\expandafter\@ifdefinable\csname #1\endcsname
%    \end{macrocode}
%    Therefore, the first argument of |\@ifdefinable| is the
%    expansion (in the example, |\Lem|) of
%    |\csname||#1\endcsname|.  The second argument is
%    executed only if the test has been completed successfully.
%    \begin{macrocode}
   {%
%    \end{macrocode}
%    Now we define the new counter. The names of the \LaTeX{}
%    macros employed should speak for themselves:
% \changes{v2.2b}{1995/10/07}{Use \cs{@newctr}}
% \changes{v2.2c}{1995/11/23}{Use \cs{@newctr} properly.}
%    \begin{macrocode}
    \@definecounter{#1}\@newctr{#1}[#3]%
%    \end{macrocode}
%    Using |\@newctr| will give a proper error message if the counter
%    in |#3| is not defined.
%    In defining `|\theLem|' we must generate the desired macro
%    name by use of |\expandafter| and |\csname|.
%    \begin{macrocode}
    \expandafter\xdef\csname the#1\endcsname
%    \end{macrocode}
%    An |\xdef| is used in order to make the definition global,
%    and to ensure that it contains the replacement texts of
%    |\@thmcountersep| and |\@thmcounter|.\footnote{These
%    two macros can be defined by the document style.  Their default
%    values produce a `\texttt{.}' as separation and an arabic
%    representation of the number.} However, not everything should be
%    expanded. For example, it saves space to use |\thesection|
%    instead of its---at times---lengthy expansion.
%    \begin{macrocode}
      {\expandafter \noexpand \csname the#3\endcsname
       \@thmcountersep \@thmcounter{#1}}%
%    \end{macrocode}
%    Thus with the defaults of \LaTeX{}, |\theLem| would be
%    replaced by the command sequence |\thesection.\arabic{Lem}|.
%
%    We will now look at the definition of the macro which is executed
%    at the beginning of the actual environment (in our example this
%    macro is |\Lem|). It should be noted that we use an
%    ``|\expandafter| trick'' to expand only certain parts of the
%    replacement text at the time of the definition.
%    \begin{macrocode}
    \def\@tempa{\global\@namedef{#1}}%
    \expandafter \@tempa \expandafter{%
%    \end{macrocode}
%    First, the macro that contains the current definitions of
%    |\@begintheorem| and |\@opargtheorem| should be called
%    up.  The name of this macro---as is already known---has the form
%    |\th@|\meta{theorem\ style}; therefore, it must be called by
%    \begin{macrocode}
      \csname th@\the \theorem@style
%    \end{macrocode}
%    In addition the default theorem font should be changeable, i.e.\
%    we have to insert the contents of |\theorem@bodyfont|. For
%    that reason, we expand even further, beyond |\endcsname|,
%    and thus insert the contents of the \textsf{token} register
%    |\theorem@bodyfont| in the replacement text.
%    \begin{macrocode}
            \expandafter \endcsname \the \theorem@bodyfont
%    \end{macrocode}
%    Now it is time to call the macro |\@thm| which takes over
%    the further processing.  It has two arguments: the current
%    counter name (in our example, \texttt{Lem}), and the text of the
%    label (in our example, \texttt{Lemma}).
%    \begin{macrocode}
     \@thm{#1}{#2}}%
%    \end{macrocode}
%    With this, the `sub-definition' is complete.  The macro
%    |\@endtheorem| ends a theorem environment and is, so far, nothing
%    but an |\endtrivlist|.  (Hence it is defined globally, and not
%    within the theorem styles.\footnote{This has to be changed as
%    soon as theorem styles that change \texttt{\bslash @endtheorem}
%    exist.  In such a case, all existing styles must be changed as
%    well since they will have to reset the macro.}) Therefore, we can
%    set it equivalent to the macro that ends the theorem set (in our
%    example, |\endLem|).  However, if some day theorem styles exist
%    that do change |\@endtheorem|, we would have to use the
%    commented-out line instead.
%    \begin{macrocode}
    \global \expandafter \let \csname end#1\endcsname \@endtheorem
%   \global\@namedef{end#1}{\@endtheorem}%
%    \end{macrocode}
%    With these commands all the required definitions are employed,
%    unless the test |\@ifdefinable| has failed.  Therefore, we
%    end the second argument of this macro and with it the definition
%    of |\@xnthm|.
%    \begin{macrocode}
   }}
%    \end{macrocode}
% \end{macro}
%
% \begin{macro}{\@ynthm}
%    The definition of |\@ynthm| is completely analogous. In
%    this case the new counter that is defined is not reset within
%    another counter; thus the definition of |\the...|
%    is simplified:
%    \begin{macrocode}
\gdef\@ynthm#1#2{\expandafter\@ifdefinable\csname #1\endcsname
   {\@definecounter{#1}%
    \expandafter\xdef\csname the#1\endcsname{\@thmcounter{#1}}%
%    \end{macrocode}
%    The rest of the definition corresponds literally to that of
%    |\@xnthm|:
%    \begin{macrocode}
    \def\@tempa{\global\@namedef{#1}}\expandafter \@tempa
     \expandafter{\csname th@\the \theorem@style \expandafter
     \endcsname \the\theorem@bodyfont \@thm{#1}{#2}}%
    \global \expandafter \let \csname end#1\endcsname \@endtheorem}}
%    \end{macrocode}
% \end{macro}
%
% \begin{macro}{\@othm}
%    The definition of |\@othm| does not contain anything new.
%    \begin{macrocode}
\gdef\@othm#1[#2]#3{%
%    \end{macrocode}
% \changes{v2.2b}{95/10/07}{Build in check for counter defined}
%    We do not define a new counter but instead use one that has
%    already been defined.  Thus the only definition we need is that
%    of this pseudo-counter (i.e.\ |\the|\meta{env.\ name}).
%    First we check if |#2| corresponds to a known counter name.
%    \begin{macrocode}
  \expandafter\ifx\csname c@#2\endcsname\relax
   \@nocounterr{#2}%
  \else
   \expandafter\@ifdefinable\csname #1\endcsname
   {\expandafter \xdef \csname the#1\endcsname
     {\expandafter \noexpand \csname the#2\endcsname}%
%    \end{macrocode}
%    All other parts of the definition can be adopted from
%    |\@xnthm|. We have to remember, though, that in this case
%    the name of the current counter and the theorem label have moved
%    to the second and third arguments.
%    \begin{macrocode}
    \def\@tempa{\global\@namedef{#1}}\expandafter \@tempa
     \expandafter{\csname th@\the \theorem@style \expandafter
     \endcsname \the\theorem@bodyfont \@thm{#2}{#3}}%
    \global \expandafter \let \csname end#1\endcsname \@endtheorem}%
  \fi}
%    \end{macrocode}
% \end{macro}
%
% \subsection{Macros that are employed in a {\envfont theorem}
%    environment}
%
% \begin{macro}{\@thm}
%    The macro |\@thm| has to increase the current counter.  Then,
%    depending on whether the environment has (or does not have) an
%    optional argument, it has to branch into either |\@begintheorem|
%    or |\@opargtheorem|.
%    \begin{macrocode}
\gdef\@thm#1#2{\refstepcounter{#1}%
%    \end{macrocode}
%    Now we start a {\envfont trivlist} environment, and give
%    |\@topsep| and |\@topsepadd| the values of the \textsf{skip}
%    registers |\theorempreskipamount| and |\theorempostskipamount|.
%    The value in |\@topsep| is the vertical space that is inserted by
%    the first (and only) |\item| in our |\trivlist| whilst
%    |\@topsepadd| is inserted by |\@endparenv| at the end of that
%    {\envfont trivlist} environment.  By using these registers, we
%    obtain the desired space around a {\envfont theorem} environment.
%    \begin{macrocode}
   \trivlist
   \@topsep \theorempreskipamount               % used by first \item
   \@topsepadd \theorempostskipamount           % used by \@endparenv
%    \end{macrocode}
%    Now we have to test whether an optional argument has been given.
% \changes{v2.0d}{89/07/18}{Forgotten percent added.}
%    \begin{macrocode}
   \@ifnextchar [%
%    \end{macrocode}
%    If there is an optional argument, we will call |\@ythm|, and
%    move the arguments read back into the input stream.
%    \begin{macrocode}
   {\@ythm{#1}{#2}}%
%    \end{macrocode}
%    If not, we call |\@begintheorem|. Its first argument is the
%    name of the theorem set (hence the second argument of
%    |\@thm|). Its second argument is the macro that produces the
%    current number.
%    \begin{macrocode}
   {\@begintheorem{#2}{\csname the#1\endcsname}\ignorespaces}}
%    \end{macrocode}
% \end{macro}
%
% \begin{macro}{\@xthm}
% \begin{macro}{\@ythm}
%    Both these macros were originally called by |\@thm|.  We do
%    not need |\@xthm| anymore, hence we reset it to
%    |\relax|. The definition of |\@ythm| has not changed at
%    all from its definition in \LaTeX. In order to make the macros
%    easier to understand, we will
%    nevertheless present it (commented out).
%    \begin{macrocode}
\global\let\@xthm\relax
% \def\@ythm#1#2[#3]{\@opargbegintheorem{#2}{\csname
%           the#1\endcsname}{#3}\ignorespaces}
%</package>
%    \end{macrocode}
%    The primitive |\ignorespaces| in |\@ythm| and
%    |\@thm| is needed to remove the spaces between the
%    |\begin{...}| and the actual text.
% \end{macro}
% \end{macro}
%
%
% \subsection{Definition of the theorem substyles}
%
% As already pointed out, the theorem substyles, defined below, are
% only loaded when necessary.  Note that all these substyles, except
% \texttt{plain}, have |\slshape| as the default body font.
%
% \subsubsection{The \texttt{plain} style}
%
% \label{dok:th@plain} As the following macros use |@|, we have
% to locally set the |\catcode| of this symbol to ``letter''.
% This happens within a group, so that we do not have to worry about
% which |\catcode| that symbol had before.
%    \begin{macrocode}
%<*thp>
\begingroup \makeatletter
%    \end{macrocode}
% Since we are now within a group, we must make all definitions
% globally.  First we make sure that \texttt{theorem.sty} is loaded.
% This will allow us to use this file as a document style option
% without having to call \texttt{theorem} itself as an option.  At the
% same time, we assure that at least version 2 is loaded, since
% |\theorem@style| was not defined in earlier versions.
%    \begin{macrocode}
\@ifundefined{theorem@style}{\input{theorem.sty}}{}
\ProvidesFile{thp.sty}
             [\FMithmInfo]
%    \end{macrocode}
%
%
% \begin{macro}{\th@plain}
% \begin{macro}{\@begintheorem}
% \begin{macro}{\@opargbegintheorem}
%    |\theoremstyle{plain}| corresponds to the original
%    definition, except that the distances to the surrounding text are
%    determined by the parameters |\theorempreskipamount| and
%    |\theorempostskipamount|.
%    First we set the default body font.
%    \begin{macrocode}
\gdef\th@plain{\normalfont\itshape
%    \end{macrocode}
%    Then we define |\@begintheorem| and
%    |\@opargbegintheorem|. These two macros define how the
%    header of a theorem is typeset.  |\@opargbegintheorem| will
%    be called if a \textsf{theorem} environment with an optional
%    argument is encountered; otherwise, the header is constructed by
%    calling |\@begintheorem|.  If one of these macros is
%    executed, we are within a \textsf{trivlist} environment started by
%    |\@thm|.  So the theorem header is produced with an
%    |\item| command.
%
%    Instead of specifying the header font directly, all standard
%    theorem styles use the |\theorem@headerfont| macro to allow
%    customization.  The extra space (|\labelsep|) is necessary
%    because of problems in the \textsf{trivlist} environment.
%    \begin{macrocode}
  \def\@begintheorem##1##2{%
        \item[\hskip\labelsep \theorem@headerfont ##1\ ##2]}%
%    \end{macrocode}
%    The definition of |\@opargbegintheorem| is completely
%    analogous. The only difference is the fact that there exists a
%    third argument (which is the optional parameter of the
%    environment and contains additional information about the
%    theorem).  Customarily we enclose it in parentheses.
%    \begin{macrocode}
\def\@opargbegintheorem##1##2##3{%
   \item[\hskip\labelsep \theorem@headerfont ##1\ ##2\ (##3)]}}
%    \end{macrocode}
% \end{macro}
% \end{macro}
% \end{macro}
%
% \noindent
% We conclude with an |\endgroup| to restore the
% |\catcode| of |@|.
%    \begin{macrocode}
\endgroup
%</thp>
%    \end{macrocode}
%
%
% \subsubsection{The \texttt{break} style}
%
% This style option is stored in the file |thb.sty|.
% For the next two lines see the documentation for |\th@plain| on
% page~\pageref{dok:th@plain}.
%    \begin{macrocode}
%<*thb>
\begingroup \makeatletter
\@ifundefined{theorem@style}{\input{theorem.sty}}{}
\ProvidesFile{thb.sty}
             [\FMithmInfo]
%    \end{macrocode}
%
% \begin{macro}{\th@break}
%    |\theoremstyle{break}| produces a line break after the name
%    of the theorem.  The font is |\slshape|.  Hence, we define
%    |\th@break| as follows:
%    \begin{macrocode}
\gdef\th@break{\normalfont\slshape
  \def\@begintheorem##1##2{\item[%
%    \end{macrocode}
%    We run into the following problem: it is not possible to create
%    the header with |\item[|\meta{title}|]| and then start
%    a new line by, for example, |\mbox{}\\|. Such a definition
%    will fail whenever a list environment follows immediately.  With
%    the above construction, the |\mbox{}| causes the switch
%    |@inlabel| (cf.\ definition of |\list| and
%    |\trivlist| in \cite{latex:source}) to be set to \texttt{false}
%    and so the following list will insert additional vertical
%    space (|\topskip|). This is quite annoying. Therefore, we
%    create the line break within the |\item|. In order to ensure
%    that the text will begin at the proper position in the following
%    line, we simply pretend that the label does not take any
%    room.\footnote{This will lead to problems whenever very high
%    symbols occurring in the line tower into the
%    heading. So, something else has to be done here sometime.}
%    \begin{macrocode}
       \rlap{\vbox{\hbox{\hskip \labelsep\theorem@headerfont ##1\ ##2}%
                   \hbox{\strut}}}]}%
%    \end{macrocode}
%    Again, the definition of |\@opargbegintheorem| is
%    completely analogous.
%    \begin{macrocode}
\def\@opargbegintheorem##1##2##3{%
  \item[\rlap{\vbox{\hbox{\hskip \labelsep \theorem@headerfont
                     ##1\ ##2\ (##3)}%
                    \hbox{\strut}}}]}}
\endgroup
%</thb>
%    \end{macrocode}
% \end{macro}
%
%
% \subsubsection{The \texttt{changebreak} style}
%
%    \begin{macrocode}
% This style option is stored in the file |thcb.sty|.
%    \begin{macrocode}
%<*thcb>
\begingroup \makeatletter
\@ifundefined{theorem@style}{\input{theorem.sty}}{}
\ProvidesFile{thcb.sty}
             [\FMithmInfo]
%    \end{macrocode}
%
% \begin{macro}{\th@changebreak}
%    The \texttt{change-break} theorem style is like \texttt{break}
%    but with interchange of theorem name and theorem number.  Thus we
%    define |\th@changebreak| as follows:
%    \begin{macrocode}
\gdef\th@changebreak{\normalfont\slshape
  \def\@begintheorem##1##2{\item
       [\rlap{\vbox{\hbox{\hskip\labelsep \theorem@headerfont ##2\ ##1}%
                    \hbox{\strut}}}]}%
\def\@opargbegintheorem##1##2##3{%
  \item[\rlap{\vbox{\hbox{\hskip\labelsep \theorem@headerfont
                     ##2\ ##1\ (##3)}%
                    \hbox{\strut}}}]}}
\endgroup
%</thcb>
%    \end{macrocode}
% \end{macro}
%
%
% \subsubsection{The \texttt{change} style}
%
% This style option is stored in the file |thc.sty|.
%    \begin{macrocode}
%<*thc>
\begingroup \makeatletter
\@ifundefined{theorem@style}{\input{theorem.sty}}{}
\ProvidesFile{thc.sty}
             [\FMithmInfo]
%    \end{macrocode}
%
% \begin{macro}{\th@change}
%    The \texttt{change} theorem style corresponds to the
%    \texttt{change break} style without a linebreak after the header.
%    To say it in another way, it's the same as the \texttt{plain}
%    style but with number and name interchanged and |\slshape| as the
%    default font.
%    \begin{macrocode}
\gdef\th@change{\normalfont\slshape
  \def\@begintheorem##1##2{\item
       [\hskip\labelsep \theorem@headerfont ##2\ ##1]}%
\def\@opargbegintheorem##1##2##3{%
  \item[\hskip\labelsep \theorem@headerfont ##2\ ##1\ (##3)]}}
\endgroup
%</thc>
%    \end{macrocode}
% \end{macro}
%
%
% \subsubsection{The \texttt{marginbreak} style}
%
% This style option is the one used most often at Mainz.  It is saved
% in the file |thmb.sty|.
%    \begin{macrocode}
%<*thmb>
\begingroup \makeatletter
\@ifundefined{theorem@style}{\input{theorem.sty}}{}
\ProvidesFile{thmb.sty}
             [\FMithmInfo]
%    \end{macrocode}
%
% \begin{macro}{\th@marginbreak}
%    The \texttt{margin break} style is nearly the same as the
%    \texttt{change break} style. The only difference is the placement
%    of the theorem number.  We use |\llap| to place it in the left
%    margin.
%
%    In this style |\labelsep| denotes the separation between the
%    number and the text.
%    \begin{macrocode}
\gdef\th@marginbreak{\normalfont\slshape
  \def\@begintheorem##1##2{\item
       [\rlap{\vbox{\theorem@headerfont
                    \hbox{\llap{##2}\hskip\labelsep ##1}%
                    \hbox{\strut}}}]}%
\def\@opargbegintheorem##1##2##3{%
  \item[\rlap{\vbox{\theorem@headerfont
                    \hbox{\llap{##2}\hskip\labelsep ##1\ (##3)}%
                    \hbox{\strut}}}]}}
\endgroup
%</thmb>
%    \end{macrocode}
% \changes{v2.1c}{90/02/17}{`endgroup outside macrocode env.}
% \end{macro}
%
%
%
% \subsubsection{The \texttt{margin} style}
%
% This style option is stored in the file |thm.sty|.
%    \begin{macrocode}
%<*thm>
\begingroup \makeatletter
\@ifundefined{theorem@style}{\input{theorem.sty}}{}
\ProvidesFile{thm.sty}
             [\FMithmInfo]
%    \end{macrocode}
%
% \begin{macro}{\th@margin}
%    Again this is only a variant of the theorem styles described
%    above without any new ideas.
%    \begin{macrocode}
\gdef\th@margin{\normalfont\slshape
  \def\@begintheorem##1##2{\item
       [\theorem@headerfont \llap{##2}\hskip\labelsep ##1]}%
\def\@opargbegintheorem##1##2##3{%
  \item[\theorem@headerfont \llap{##2}\hskip\labelsep ##1\ (##3)]}}
\endgroup
%</thm>
%    \end{macrocode}
% \end{macro}
%
%
%
%
% \subsection{Final Definitions}
%
% \begin{macro}{\theorempreskipamount}
% \begin{macro}{\theorempostskipamount}
%    The \textsf{skip} parameters that regulate the vertical empty space
%    before and after the {\envfont theorem} environment have
%    to be allocated as well.
% \changes{v2.0f}{89/07/20}{Unnecessary `global' removed.}
%    \begin{macrocode}
%<*package>
\newskip\theorempreskipamount
\newskip\theorempostskipamount
%    \end{macrocode}
%    Since we have used the same values for all theorem sets, we now can
%    assign them.
%    \begin{macrocode}
\global\setlength\theorempreskipamount{12pt plus 5pt minus 3pt}
\global\setlength\theorempostskipamount{8pt plus 3pt minus 1.5pt}
%    \end{macrocode}
% \end{macro}
% \end{macro}
%
%
% \begin{macro}{\@endtheorem}
%    The same holds for the macro |\@endtheorem|, which ends a
%    {\envfont theorem} environment. Since it is the same for all
%    theorem sets, it is removed from the macros |\th@|\meta{style}.
%    It simply ends the {\envfont trivlist} environment, which was
%    begun in |\@thm|.
%    \begin{macrocode}
\global\let\@endtheorem=\endtrivlist
%    \end{macrocode}
% \end{macro}
%
%
%
% \begin{macro}{\@preamblecmds}
%    All macros defined above are to be used only in the preamble.
%    Therefore, we insert them in |\@preamblecmds| which will disable
%    them at begin document. This is done by the internal \LaTeXe{}
%    command |\@onlypreamble|.
% \changes{v2.0g}{89/09/19}{`theoremheaderfont and `theorembodyfont
%    added.}
% \changes{v2.2a}{94/02/03}{Use \LaTeXe{} interface}
%    \begin{macrocode}
\@onlypreamble\@xnthm
\@onlypreamble\@ynthm
\@onlypreamble\@othm
\@onlypreamble\newtheorem
\@onlypreamble\theoremstyle
\@onlypreamble\theorembodyfont
\@onlypreamble\theoremheaderfont
%    \end{macrocode}
% \end{macro}
%
% Finally we declare the \texttt{plain} theorem style to be the default.
% \changes{v2.1b}{90/06/04}{Fixed missing `item by declaring default
%    style.}
%    \begin{macrocode}
\theoremstyle{plain}
%</package>
%    \end{macrocode}
%
% \Finale

\endinput
}{}
\ProvidesFile{thp.sty}
             [\FMithmInfo]
%    \end{macrocode}
%
%
% \begin{macro}{\th@plain}
% \begin{macro}{\@begintheorem}
% \begin{macro}{\@opargbegintheorem}
%    |\theoremstyle{plain}| corresponds to the original
%    definition, except that the distances to the surrounding text are
%    determined by the parameters |\theorempreskipamount| and
%    |\theorempostskipamount|.
%    First we set the default body font.
%    \begin{macrocode}
\gdef\th@plain{\normalfont\itshape
%    \end{macrocode}
%    Then we define |\@begintheorem| and
%    |\@opargbegintheorem|. These two macros define how the
%    header of a theorem is typeset.  |\@opargbegintheorem| will
%    be called if a \textsf{theorem} environment with an optional
%    argument is encountered; otherwise, the header is constructed by
%    calling |\@begintheorem|.  If one of these macros is
%    executed, we are within a \textsf{trivlist} environment started by
%    |\@thm|.  So the theorem header is produced with an
%    |\item| command.
%
%    Instead of specifying the header font directly, all standard
%    theorem styles use the |\theorem@headerfont| macro to allow
%    customization.  The extra space (|\labelsep|) is necessary
%    because of problems in the \textsf{trivlist} environment.
%    \begin{macrocode}
  \def\@begintheorem##1##2{%
        \item[\hskip\labelsep \theorem@headerfont ##1\ ##2]}%
%    \end{macrocode}
%    The definition of |\@opargbegintheorem| is completely
%    analogous. The only difference is the fact that there exists a
%    third argument (which is the optional parameter of the
%    environment and contains additional information about the
%    theorem).  Customarily we enclose it in parentheses.
%    \begin{macrocode}
\def\@opargbegintheorem##1##2##3{%
   \item[\hskip\labelsep \theorem@headerfont ##1\ ##2\ (##3)]}}
%    \end{macrocode}
% \end{macro}
% \end{macro}
% \end{macro}
%
% \noindent
% We conclude with an |\endgroup| to restore the
% |\catcode| of |@|.
%    \begin{macrocode}
\endgroup
%</thp>
%    \end{macrocode}
%
%
% \subsubsection{The \texttt{break} style}
%
% This style option is stored in the file |thb.sty|.
% For the next two lines see the documentation for |\th@plain| on
% page~\pageref{dok:th@plain}.
%    \begin{macrocode}
%<*thb>
\begingroup \makeatletter
\@ifundefined{theorem@style}{% \iffalse meta-comment
%
% Copyright (C) 1993-2024
%
% The LaTeX Project and any individual authors listed elsewhere
% in this file.
%
% This file is part of the Standard LaTeX `Tools Bundle'.
% -------------------------------------------------------
%
% It may be distributed and/or modified under the
% conditions of the LaTeX Project Public License, either version 1.3c
% of this license or (at your option) any later version.
% The latest version of this license is in
%    https://www.latex-project.org/lppl.txt
% and version 1.3c or later is part of all distributions of LaTeX
% version 2005/12/01 or later.
%
% The list of all files belonging to the LaTeX `Tools Bundle' is
% given in the file `manifest.txt'.
%
% \fi
%
%%
%
%
\def\FMithmInfo{2023/07/05 v2.2c Theorem extension package (FMi)}
%
% \ProvidesFile{theorem.dtx}[\FMithmInfo]
%\iffalse   % this is a METACOMMENT !
%%
%% Package `theorem' to use with LaTeX2e
%% Copyright (C) 1989-1995 Frank Mittelbach, all rights reserved.
%<+package>\@ifundefined{theorem@style}{}{\endinput}
%<+package>\NeedsTeXFormat{LaTeX2e}[1995/06/01]
%<+package>\ProvidesPackage{theorem}
%<+package>         [\FMithmInfo]
%
% \fi
%
% \changes{v1.0}{87/05/07}{First draft.}
% \changes{v1.0a}{87/05/07}{`thistheoremstyle' added.}
% \changes{v1.1a}{87/07/29}{New concept: redefine the internal
%                          \protect\LaTeX{} commands `@xnthm' etc.}
% \changes{v1.1b}{87/07/31}{`theoremfont' used to specify font.}
% \changes{v1.2a}{88/01/08}{`@endtheorem' now globally defined.}
% \changes{v1.2b}{88/01/11}{`newtheorem' only allowed in the preamble.}
% \changes{v1.3a}{88/02/14}{`theoremfont' now a toks register.}
%
% \changes{v2.0a}{88/11/16}{New implementation and new documentation.}
% \changes{v2.0a}{88/11/16}{No layout changes but internal modifications
%                           old substyles won't work.}
% \changes{v2.0b}{89/01/30}{`global' used for all definitions.}
% \changes{v2.0c}{89/06/15}{Documentation translated to English.
%                            Thanks to LLL and Nina}
% \changes{v2.1a}{90/02/15}{Support for new font selection.}
% \changes{v2.2a}{94/02/03}{Cleanup for LaTeX2e}
%
% \DoNotIndex{\@endparenv,\@ifdefinable,\@ifnextchar,\@ifundefined}
% \DoNotIndex{\@namedef,\@nameuse,\@spaces,\@tempa,\@warning}
% \DoNotIndex{\ }
% \DoNotIndex{\begingroup,\bfseries}
% \DoNotIndex{\csname}
% \DoNotIndex{\def,\do,\docdate}
% \DoNotIndex{\endcsname,\endgroup,\endinput,\endtrivlist,\expandafter}
% \DoNotIndex{\filedate,\fileversion}
% \DoNotIndex{\gdef,\global}
% \DoNotIndex{\hbox,\hskip}
% \DoNotIndex{\ignorespaces,\input,\itshape,\item}
% \DoNotIndex{\labelsep,\let,\llap}
% \DoNotIndex{\makeatletter}
% \DoNotIndex{\noexpand}
% \DoNotIndex{\refstepcounter,\relax,\rlap}
% \DoNotIndex{\setlength,\slshape,\space,\string,\strut}
% \DoNotIndex{\the,\trivlist,\typeout}
% \DoNotIndex{\vbox}
% \DoNotIndex{\xdef}
%
% \def\deflabel#1{\ttfamily #1\hfill}
% \def\deflist#1{\begin{list}{}{\settowidth\labelwidth{\ttfamily #1}%
%                               \setlength\leftmargin\labelwidth
%                               \addtolength\leftmargin\labelsep
%                               \let\makelabel\deflabel}}
% \def\enddeflist{\end{list}}
% \def\envfont{\normalfont\ttfamily}
%
% \MakeShortVerb{\|}
%
% \newcommand{\TUB}{{\sl TUGboat\/}}
%
% \GetFileInfo{theorem.dtx}
% \title{An Extension of the
%        \LaTeX{} theorem environment\thanks{This file
%        has version number \fileversion, last
%        revised \filedate.}}
% \author{Frank Mittelbach}
%
% \MaintainedByLaTeXTeam{tools}
% \maketitle
%
% \begin{abstract}
%    The macros described in this paper yield an extension of the
%    \LaTeX{} theorem mechanism. It is designed to satisfy the
%    different requirements of various journals. Thus, the
%    layout of the ``theorems'' can be manipulated by determining a
%    ``style''. This article describes not only the use, but
%    also the definition, of the necessary macros.
% \end{abstract}
%
% \section*{Preface to version 2.2}
%
% For \LaTeXe{} this package did not need any fundamental changes. I
% only modified the messages generated so that theorem layout styles
% will show up with the |\listfiles| command and cleaned the section
% on the New Font Selection Scheme since this is now included in
% \LaTeX.
%
% \section*{Preface to version 2.1}
%
% This version is identical to 2.0g described in \TUB\ 10\#3 except
% for some internal defaults which are now set depending on the used
% font selection scheme.
%
% This was done to avoid unpleasant surprises if the new font selection
% scheme is in force. For further details see
% section~\ref{sec:newfonts} and \cite{bk:GMS94}.
%
% \section{Introduction}
%
% For our purposes here, ``theorems'' are labelled enunciations,
% often set off from the main text by extra space and a font change.
% Theorems, corollaries, conjectures, definitions, and remarks
% are all instances of ``theorems''.  The ``header'' of these structures
% is composed of a label (such as \textsc{Theorem} or \textsc{Remark})
% and a number which serializes an item in the sequence of items
% with the same label.
%
% Shortly after the introduction of \LaTeX{} at the Fachbereich
% Mathematik in Mainz, the desire to manipulate the layout of
% ``theorems'' arose.  In Mainz, the following two conventions came
% into general use:
% \begin{enumerate}
%   \item The number of the theorem is shown in the margin.
%   \item There is a line break at
%         the end of the theorem header.
% \end{enumerate}
%
% Additionally, some journals require different formats which depend on
% the ``sort of theorem'': e.g.\ often remarks and definitions are set
% in |\upshape|, while |\itshape| is employed for main theorems.
%
% Confronted with these requirements, a theorem environment was
% developed in Mainz which allows separate determination of the
% layout of the ``theorems sets'', comparable to |\pagestyle|.
%
%
% \section{The user interface}
% \subsection{Defining new theorem sets}
%
% \DescribeMacro\newtheorem
% As in the original \LaTeX{} version, the command |\newtheorem|
% defines a new ``theorem set'' or ``theorem-like structure''.
% Two required arguments name the new environment and give the text
% to be typeset with each instance of the new ``set'', while
% an optional argument determines how
% the ``set'' is enumerated:
% \begin{description}
%    \item[\ttfamily \bslash newtheorem\{foo\}\{bar\}]
%       The theorem set {\envfont foo} (whose name is \texttt{bar})
%       uses its own counter.
%    \item[\ttfamily \bslash newtheorem\{foo2\}{[foo]}\{bar2\}]
%       The theorem set {\envfont foo2} (printed name \texttt{bar2})
%       uses the same counter as the theorem set \texttt{foo}.
%    \item[\ttfamily \bslash newtheorem\{foo3\}\{bar3\}{[section]}]
%       The theorem set {\envfont foo3} (printed name \texttt{bar3}) is
%       enumerated within the counter \texttt{section}, i.e.\ with every
%       new |\section| the enumeration begins again with 1, and
%       the enumeration is composed from the section-number and the
%       theorem counter itself.
% \end{description}
%
% \DescribeMacro\theoremstyle
% Additionally, the command |\theoremstyle| can define the layout
% of various, or all, theorem sets.  It should be noted that any
% theorem set defined by |\newtheorem| is typeset in the
% |\theoremstyle| that is current at the time of the definition.
% Thus, the following
% \begin{verbatim}
%    \theoremstyle{break}        \newtheorem{Cor}{Corollary}
%    \theoremstyle{plain}        \newtheorem{Exa}{Example}[section]
%\end{verbatim}                             ^^A No blank at line start
% leads to the result that the set {\envfont Cor} is formatted in the
% style \texttt{break}, while the set {\envfont Exa} and all the
% following ones are formatted in the style \texttt{plain}, unless
% another |\theoremstyle| follows.  Since the definitions installed by
% |\newtheorem| are global, one also can limit |\theoremstyle| locally
% by grouping braces.
%
% \DescribeMacro\theorembodyfont
% The choice of the font for the theorem body is completely
% independent of the chosen |\theoremstyle|; this has proven to
% be very advantageous.  For example,
% \begin{verbatim}
%   {\theorembodyfont{\upshape}          \newtheorem{Rem}{Remark}}
%\end{verbatim}
% defines a theorem set {\envfont Rem}, which will be set in
% |\upshape| in the current layout (which in our example is
% \texttt{plain}).  As with |\theoremstyle|, the |\theorembodyfont|
% chosen is that current at the time of |\newtheorem|.  If
% |\theorembodyfont| is not specified or one defines
% |\theorembodyfont{}|, then the font used will be that defined by the
% |\theoremstyle|.
%
% \DescribeMacro\theoremheaderfont It is also possible to customize
% the font used for the theorem headers.  This is, however, a global
% declaration, and therefore there should be at most one
% |\theoremheaderfont| declaration in the preamble.\footnote{If it is
% actually necessary to have different header fonts, one has to define
% new theorem styles (substituting the desired font) or specify the
% information directly in the \texttt{\bslash newtheorem} declaration
% (the unclean variant).}
%
% \DescribeMacro\theorempreskipamount
% \DescribeMacro\theorempostskipamount Two additional parameters
% affect the vertical space around the theorem environments:
% |\theorempreskipamount| and |\theorempostskipamount| define,
% respectively, the spacing before and after such an environment.
% These parameters apply for all theorem sets and can be manipulated
% with the ordinary length macros.  They are rubber lengths
% (`\textsf{skips}'), and therefore can contain \texttt{plus} and
% \texttt{minus} parts.
%
% Since the definition of theorem sets should---most sensibly---be
% placed in the preamble, we only allow installation there.
% It is therefore possible to release the memory used here after
% |\begin{document}|, in order to make room for other
% applications.
%
%
% \subsection{Existing theorem styles}
%
% The following theorem styles exist to date:
% \begin{deflist}{marginbreak}
%    \item[plain]
%       This theorem style emulates the original \LaTeX{} definition,
%       except that additionally the parameters
%       |\theorem...skipamount| are used.
%    \item[break]
%       In this style, the theorem header is followed by a line
%       break.
%    \item[marginbreak]
%       The theorem number is set in the margin, and there is a line
%       break as in \texttt{break}.
%    \item[changebreak]
%       Like \texttt{break}, but with header number and text
%       interchanged.
%    \item[change]
%       Header number and text are interchanged, without a line break.
%    \item[margin]
%       The number is set in the left margin, without a line break.
% \end{deflist}
% All styles (except \texttt{plain}) select |\slshape| as the default
% |\theorembodyfont|.
%
%
% \subsection{Examples}
%
% Given the above theorem sets {\envfont Cor}, {\envfont Exa}
% and {\envfont Rem},
% suppose that the preamble also contains the declarations:
% \begin{verbatim}
%    \theoremstyle{marginbreak}   \newtheorem{Lem}[Cor]{Lemma}
%    \theoremstyle{change}
%    \theorembodyfont{\itshape}        \newtheorem{Def}[Cor]{Definition}
%
%    \theoremheaderfont{\scshape}
%\end{verbatim}
% Then the following are some typical examples of the typeset output
% resulting from their use.
%
% \begin{Cor}
%    This is a sentence typeset in the theorem environment {\envfont
%    Cor}.
% \end{Cor}
%
% \begin{Exa}
%    This is a sentence typeset in the theorem environment {\envfont
%    Exa}.
% \end{Exa}
%
% \begin{Rem}
%    This is a sentence typeset in the theorem environment {\envfont
%    Rem}.
% \end{Rem}
%
% \begin{Lem}[Ben User]
%    This is a sentence typeset in the theorem environment {\envfont
%    Lem}.
% \end{Lem}
%
% \begin{Def}[Very impressive Definition]
%    This is a sentence typeset in the theorem environment {\envfont
%    Def}.
% \end{Def}
%
% The last two examples show the effect of the optional argument to
% a theorem environment (it is the text typeset in parentheses).
%
%
% \section{Special Considerations}
% \label{sec:newfonts}
%
% Theoremheader and body are implemented as a unit. This means that
% the |\theoremheaderfont|  will inherit characteristics of the
% |\theorembodyfont| in \LaTeXe. Thus, if for example
% |\theorembodyfont| is |\itshape| and |\theoremheaderfont|
% is |\bfseries| the font selected for the header will have the
% characteristics `bold extended italic'. If this is not desired one
% should set the |\theoremheaderfont| to something like
% \begin{verbatim}
%      \theoremheaderfont{\normalfont\bfseries}
%\end{verbatim}
% i.e.\ supplying all necessary font information explicitly.
%
%
%
% \section{Acknowledgements}
%
% The publication of this set of macros was only possible with the
% help of Christina Busse (translating the manuscript into English),
% Joachim Pense (playing the r\^ole of typist), Chris Rowley
% (looking everything over) and many others providing useful
% suggestions.
%
% \MaybeStop{%
%    \begin{thebibliography}{9}
%       \bibitem{bk:GMS94} \textsc{M.~Goossens}, \textsc{F.~Mittelbach}
%         and \textsc{A.~Samarin}.
%         \newblock The \LaTeX{} Companion.
%         \newblock
%         Addison-Wesley, Reading, Massachusetts, 1994.
%       \bibitem{latex:source} \textsc{Lamport, Leslie.} \newblock
%                               \texttt{latex.tex}, version 2.09, date
%                               Feb. 1990.
%    \end{thebibliography}
%    \PrintIndex
%    \PrintChanges}
%
%
% \section{The documentation driver file}
%
% The next bit of code contains the documentation driver file for
% \TeX{}, i.e., the file that will produce the documentation you are
% currently reading. It will be extracted from this file by the
% \texttt{docstrip} program. Since it is the first code in the file
% one can alternatively process this file directly with \LaTeXe{} to
% obtain the documentation.
%
%    \begin{macrocode}
%<*driver>
\documentclass{ltxdoc}

\usepackage{theorem}

%<+driver>% The next few lines define theorem sets which are used
%<+driver>% in the example section of the documentation.

\theoremstyle{break}       \newtheorem{Cor}{Corollary}
\theoremstyle{plain}       \newtheorem{Exa}{Example}[section]
{\theorembodyfont{\upshape}\newtheorem{Rem}{Remark}}
\theoremstyle{marginbreak} \newtheorem{Lem}[Cor]{Lemma}
\theoremstyle{change}
\theorembodyfont{\itshape} \newtheorem{Def}[Cor]{Definition}

\theoremheaderfont{\scshape}

\RecordChanges

\begin{document}
   \DocInput{theorem.dtx}
\end{document}
%</driver>
%    \end{macrocode}
%
% \section{Definition of the Macros}
%
% If the file has been loaded before, we abort immediately. If not the
% package announces itself (this is actually done at the very top of
% the file---the way it is done isn't good style so don't copy it).
% \changes{v2.0e}{89/07/19}{Spaces removed from `typeout'.}
% \changes{v2.2b}{95/11/19}{Announce moved to top of file}
%    \begin{macrocode}
%<*package>
%\@ifundefined{theorem@style}{}{\endinput}
%\def\FMithmInfo{1995/11/19 v2.2b Theorem extension package (FMi)}
%\ProvidesPackage{theorem}[\FMithmInfo]
%    \end{macrocode}
%
%
% \subsection{Definition of theorem styles and fonts}
%
% All the definitions in this file are done globally to allow
% inputting this file inside a group.
%
% \begin{macro}{\theoremstyle}
%    Before a theorem style can be installed, the chosen style must be
%    known. For that reason, we must test to see that
%    |\th@|\meta{style} is known or, more precisely, that it is
%    different from |\relax|. If the style is not known then
%    |\th@plain| is used.
%    \begin{macrocode}
\gdef\theoremstyle#1{%
   \@ifundefined{th@#1}{\@warning
          {Unknown theoremstyle `#1'. Using `plain'}%
          \theorem@style{plain}}%
%    \end{macrocode}
%    We save the theorem style to be used in the token register
%    |\theorem@style|.
%    \begin{macrocode}
      {\theorem@style{#1}}%
%    \end{macrocode}
%    Now we ``evaluate'' the theorem style: this means, we call the
%    macro |\th@|\meta{style} which will activate the relevant
%    definitions which are contained in a separate file.  This is done
%    in a group to suppress changes to the current font.  This could
%    otherwise pose problems together with the new font selection
%    scheme\footnote{When I printed the original article using the new
%    font selection scheme I ended with a document with slanted
%    typefaces (text headings and all) simply because one of the
%    theorem styles used \texttt{\bslash sl} at toplevel.} if the
%    |\th@|\meta{style} is evaluated a second time.
% \changes{v2.1a}{90/02/15}{Group added around `th@..}
% \changes{v2.1a}{90/02/15}{Reseting of `..@begintheorem removed.}
%    \begin{macrocode}
      \begingroup
        \csname th@\the\theorem@style \endcsname
      \endgroup}
%    \end{macrocode}
% \end{macro}
%
%
% \begin{macro}{\@begintheorem}
% \changes{v2.1a}{90/02/15}{Relaxed at toplevel.}
% \begin{macro}{\@opargbegintheorem}
% \changes{v2.1a}{90/02/15}{Relaxed at toplevel.}
%    We reset |\@begintheorem| and |\@opargbegintheorem| to
%    |\relax| since these commands are no longer necessary at
%    toplevel. This will save a few tokens.
%    \begin{macrocode}
\global\let\@begintheorem\relax
\global\let\@opargbegintheorem\relax
%    \end{macrocode}
% \end{macro}
% \end{macro}
%
% \begin{macro}{\theorem@style}
%    Obviously the \textsf{token} register used above has to be
%    allocated. To assure the utmost compatibility with the
%    original \LaTeX{} definition, we set the default theorem style to
%    \texttt{plain}, which implements the usual \LaTeX{} convention.
% \changes{v2.0f}{89/07/20}{Unnecessary `global' removed.}
%    \begin{macrocode}
\newtoks\theorem@style
\global\theorem@style{plain}
%    \end{macrocode}
% \end{macro}
%
% \begin{macro}{\theorembodyfont}
% \begin{macro}{\theorem@bodyfont}
%    If the |\theorembodyfont| is set by the user then it should not
%    interact with the default font set in the theorem style. When the
%    new font selection is in force this may happen if, for example,
%    the default is |\itshape| and the new |\theorembodyfont| is
%    |\sffamily|. So we add a |\reset@font| command in front of the
%    user definition.
% \changes{v2.0f}{89/07/20}{Unnecessary `global' removed.}
% \changes{v2.0g}{89/09/19}{`theoremfont to `theorembodyfont.}
% \changes{v2.1c}{92/06/22}{`reset@font added and logic changed.}
% \changes{v2.1d}{92/10/28}{Reset to empty if arg is empty}
%    \begin{macrocode}
\gdef\theorembodyfont#1{%
%    \end{macrocode}
%    We check if the argument supplied is empty and if so put nothing
%    into the |\theorem@bodyfont| token register to allow for
%    |\theorembodyfont{}| as a mean of using the default of the
%    current |\theoremstyle|.
%    \begin{macrocode}
   \def\@tempa{#1}%
   \ifx\@tempa\@empty
    \theorem@bodyfont{}%
   \else
    \theorem@bodyfont{\reset@font#1}%
   \fi
   }
\newtoks\theorem@bodyfont
\global\theorem@bodyfont{}
%    \end{macrocode}
% \end{macro}
% \end{macro}
%
%
% \begin{macro}{\theoremheaderfont}
% \changes{v2.0g}{89/09/19}{Macro added.}
%    The font for the theorem headers is handled differently because
%    this definition applies to all theorem styles.
%    \begin{macrocode}
\gdef\theoremheaderfont#1{\gdef\theorem@headerfont{#1}%
%    \end{macrocode}
%    After using the macro once it is redefined to produce an error
%    message.
%    \begin{macrocode}
       \gdef\theoremheaderfont##1{%
        \typeout{\string\theoremheaderfont\space should be used
                 only once.}}}
%    \end{macrocode}
% \end{macro}
%
% \begin{macro}{\theorem@headerfont}
% \changes{v2.0g}{89/09/19}{Macro added.}
%    To set the |\theorem@headerfont| default we first test if
%    the new fontselection scheme is in force.
% \changes{v2.1a}{90/02/15}{Support for new font selection.}
%    \begin{macrocode}
\ifx\upshape\undefined
%    \end{macrocode}
%    If not we define it to expand into |\bfseries|. We don't use
%    |\let| just in case a following style option redefines this
%    macro.
%    \begin{macrocode}
\gdef\theorem@headerfont{\bfseries}
%    \end{macrocode}
%    Otherwise we reset the current shape before calling |\bfseries|.
%    \begin{macrocode}
\else \gdef\theorem@headerfont{\normalfont\bfseries}\fi
%    \end{macrocode}
% \end{macro}
%
% \begin{macro}{\th@plain}
% \begin{macro}{\th@break}
% \begin{macro}{\th@marginbreak}
% \begin{macro}{\th@changebreak}
% \begin{macro}{\th@change}
% \begin{macro}{\th@margin}
%    The different styles are defined in macros such as
%    |\th@plain|.  Since memory space is precious in
%    ``non-Big-versions'', we have to avoid offering too many unused
%    definitions.  Therefore we define these styles in separate files
%    that can be loaded on demand.  Thus the commands themselves only
%    load these files. We use |\@input@|, a \LaTeXe{} internal command
%    that ensures that the file will be listed with |\listfiles|.
% \changes{v2.2a}{94/02/02}{Use `@input@ to load theorem layout files}
%    \begin{macrocode}
\gdef\th@plain{\@input@{thp.sty}}
\gdef\th@break{\@input@{thb.sty}}
\gdef\th@marginbreak{\@input@{thmb.sty}}
\gdef\th@changebreak{\@input@{thcb.sty}}
\gdef\th@change{\@input@{thc.sty}}
\gdef\th@margin{\@input@{thm.sty}}
%    \end{macrocode}
%    This list will be expanded when new styles become
%    available.  For testing,  just append new theorem substyles
%    as document options.
% \end{macro}
% \end{macro}
% \end{macro}
% \end{macro}
% \end{macro}
% \end{macro}
%
%
%
% \subsection{Definition of a new theorem set}
%
% As already pointed out, a new theorem environment can be defined in
% three different ways:
% \begin{verbatim}
%    \newtheorem{Lem}{Lemma}
%    \newtheorem{Lem}{Lemma}[section]
%    \newtheorem{Lem}[Theorem]{Lemma}
%\end{verbatim}
% The function of the macro |\newtheorem| is to recognize these cases
% and then to branch into one of the three macros |\@ynthm|, |\@xnthm|
% or |\@othm|.  This mechanism is adopted unchanged from
% \cite{latex:source}; the essential point here is that, for example,
% in the second case, the arguments \texttt{Lem}, \texttt{Lemma} and
% \texttt{section} are passed over to the macro |\@xnthm|.
%
% We inspect this case first because the others present fewer problems,
% and thus are easily derived from this one.
%
% \begin{macro}{\@xnthm}
%    For our example arguments, the macro |\@xnthm| must fulfill
%    the following:
%    \begin{itemize}
%       \item
%          Define a new \LaTeX-counter `Lem'
%       \item
%          reset this counter within a |\section|
%       \item
%          define the macro |\theLem|
%       \item
%          define the environment macros |\Lem| and
%          |\endLem| using the current
%          |\theoremstyle| and |\theorem@bodyfont|.
%    \end{itemize}
%    Obviously, all this should happen only if the first argument
%    of |\@xnthm| (i.e.\ \texttt{Lem} in our example) is chosen so
%    as not to conflict with any previously defined commands or
%    environments.  This test is performed by the \LaTeX{} macro
%    |\@ifdefinable|.
%    \begin{macrocode}
\gdef\@xnthm#1#2[#3]{\expandafter\@ifdefinable\csname #1\endcsname
%    \end{macrocode}
%    Therefore, the first argument of |\@ifdefinable| is the
%    expansion (in the example, |\Lem|) of
%    |\csname||#1\endcsname|.  The second argument is
%    executed only if the test has been completed successfully.
%    \begin{macrocode}
   {%
%    \end{macrocode}
%    Now we define the new counter. The names of the \LaTeX{}
%    macros employed should speak for themselves:
% \changes{v2.2b}{1995/10/07}{Use \cs{@newctr}}
% \changes{v2.2c}{1995/11/23}{Use \cs{@newctr} properly.}
%    \begin{macrocode}
    \@definecounter{#1}\@newctr{#1}[#3]%
%    \end{macrocode}
%    Using |\@newctr| will give a proper error message if the counter
%    in |#3| is not defined.
%    In defining `|\theLem|' we must generate the desired macro
%    name by use of |\expandafter| and |\csname|.
%    \begin{macrocode}
    \expandafter\xdef\csname the#1\endcsname
%    \end{macrocode}
%    An |\xdef| is used in order to make the definition global,
%    and to ensure that it contains the replacement texts of
%    |\@thmcountersep| and |\@thmcounter|.\footnote{These
%    two macros can be defined by the document style.  Their default
%    values produce a `\texttt{.}' as separation and an arabic
%    representation of the number.} However, not everything should be
%    expanded. For example, it saves space to use |\thesection|
%    instead of its---at times---lengthy expansion.
%    \begin{macrocode}
      {\expandafter \noexpand \csname the#3\endcsname
       \@thmcountersep \@thmcounter{#1}}%
%    \end{macrocode}
%    Thus with the defaults of \LaTeX{}, |\theLem| would be
%    replaced by the command sequence |\thesection.\arabic{Lem}|.
%
%    We will now look at the definition of the macro which is executed
%    at the beginning of the actual environment (in our example this
%    macro is |\Lem|). It should be noted that we use an
%    ``|\expandafter| trick'' to expand only certain parts of the
%    replacement text at the time of the definition.
%    \begin{macrocode}
    \def\@tempa{\global\@namedef{#1}}%
    \expandafter \@tempa \expandafter{%
%    \end{macrocode}
%    First, the macro that contains the current definitions of
%    |\@begintheorem| and |\@opargtheorem| should be called
%    up.  The name of this macro---as is already known---has the form
%    |\th@|\meta{theorem\ style}; therefore, it must be called by
%    \begin{macrocode}
      \csname th@\the \theorem@style
%    \end{macrocode}
%    In addition the default theorem font should be changeable, i.e.\
%    we have to insert the contents of |\theorem@bodyfont|. For
%    that reason, we expand even further, beyond |\endcsname|,
%    and thus insert the contents of the \textsf{token} register
%    |\theorem@bodyfont| in the replacement text.
%    \begin{macrocode}
            \expandafter \endcsname \the \theorem@bodyfont
%    \end{macrocode}
%    Now it is time to call the macro |\@thm| which takes over
%    the further processing.  It has two arguments: the current
%    counter name (in our example, \texttt{Lem}), and the text of the
%    label (in our example, \texttt{Lemma}).
%    \begin{macrocode}
     \@thm{#1}{#2}}%
%    \end{macrocode}
%    With this, the `sub-definition' is complete.  The macro
%    |\@endtheorem| ends a theorem environment and is, so far, nothing
%    but an |\endtrivlist|.  (Hence it is defined globally, and not
%    within the theorem styles.\footnote{This has to be changed as
%    soon as theorem styles that change \texttt{\bslash @endtheorem}
%    exist.  In such a case, all existing styles must be changed as
%    well since they will have to reset the macro.}) Therefore, we can
%    set it equivalent to the macro that ends the theorem set (in our
%    example, |\endLem|).  However, if some day theorem styles exist
%    that do change |\@endtheorem|, we would have to use the
%    commented-out line instead.
%    \begin{macrocode}
    \global \expandafter \let \csname end#1\endcsname \@endtheorem
%   \global\@namedef{end#1}{\@endtheorem}%
%    \end{macrocode}
%    With these commands all the required definitions are employed,
%    unless the test |\@ifdefinable| has failed.  Therefore, we
%    end the second argument of this macro and with it the definition
%    of |\@xnthm|.
%    \begin{macrocode}
   }}
%    \end{macrocode}
% \end{macro}
%
% \begin{macro}{\@ynthm}
%    The definition of |\@ynthm| is completely analogous. In
%    this case the new counter that is defined is not reset within
%    another counter; thus the definition of |\the...|
%    is simplified:
%    \begin{macrocode}
\gdef\@ynthm#1#2{\expandafter\@ifdefinable\csname #1\endcsname
   {\@definecounter{#1}%
    \expandafter\xdef\csname the#1\endcsname{\@thmcounter{#1}}%
%    \end{macrocode}
%    The rest of the definition corresponds literally to that of
%    |\@xnthm|:
%    \begin{macrocode}
    \def\@tempa{\global\@namedef{#1}}\expandafter \@tempa
     \expandafter{\csname th@\the \theorem@style \expandafter
     \endcsname \the\theorem@bodyfont \@thm{#1}{#2}}%
    \global \expandafter \let \csname end#1\endcsname \@endtheorem}}
%    \end{macrocode}
% \end{macro}
%
% \begin{macro}{\@othm}
%    The definition of |\@othm| does not contain anything new.
%    \begin{macrocode}
\gdef\@othm#1[#2]#3{%
%    \end{macrocode}
% \changes{v2.2b}{95/10/07}{Build in check for counter defined}
%    We do not define a new counter but instead use one that has
%    already been defined.  Thus the only definition we need is that
%    of this pseudo-counter (i.e.\ |\the|\meta{env.\ name}).
%    First we check if |#2| corresponds to a known counter name.
%    \begin{macrocode}
  \expandafter\ifx\csname c@#2\endcsname\relax
   \@nocounterr{#2}%
  \else
   \expandafter\@ifdefinable\csname #1\endcsname
   {\expandafter \xdef \csname the#1\endcsname
     {\expandafter \noexpand \csname the#2\endcsname}%
%    \end{macrocode}
%    All other parts of the definition can be adopted from
%    |\@xnthm|. We have to remember, though, that in this case
%    the name of the current counter and the theorem label have moved
%    to the second and third arguments.
%    \begin{macrocode}
    \def\@tempa{\global\@namedef{#1}}\expandafter \@tempa
     \expandafter{\csname th@\the \theorem@style \expandafter
     \endcsname \the\theorem@bodyfont \@thm{#2}{#3}}%
    \global \expandafter \let \csname end#1\endcsname \@endtheorem}%
  \fi}
%    \end{macrocode}
% \end{macro}
%
% \subsection{Macros that are employed in a {\envfont theorem}
%    environment}
%
% \begin{macro}{\@thm}
%    The macro |\@thm| has to increase the current counter.  Then,
%    depending on whether the environment has (or does not have) an
%    optional argument, it has to branch into either |\@begintheorem|
%    or |\@opargtheorem|.
%    \begin{macrocode}
\gdef\@thm#1#2{\refstepcounter{#1}%
%    \end{macrocode}
%    Now we start a {\envfont trivlist} environment, and give
%    |\@topsep| and |\@topsepadd| the values of the \textsf{skip}
%    registers |\theorempreskipamount| and |\theorempostskipamount|.
%    The value in |\@topsep| is the vertical space that is inserted by
%    the first (and only) |\item| in our |\trivlist| whilst
%    |\@topsepadd| is inserted by |\@endparenv| at the end of that
%    {\envfont trivlist} environment.  By using these registers, we
%    obtain the desired space around a {\envfont theorem} environment.
%    \begin{macrocode}
   \trivlist
   \@topsep \theorempreskipamount               % used by first \item
   \@topsepadd \theorempostskipamount           % used by \@endparenv
%    \end{macrocode}
%    Now we have to test whether an optional argument has been given.
% \changes{v2.0d}{89/07/18}{Forgotten percent added.}
%    \begin{macrocode}
   \@ifnextchar [%
%    \end{macrocode}
%    If there is an optional argument, we will call |\@ythm|, and
%    move the arguments read back into the input stream.
%    \begin{macrocode}
   {\@ythm{#1}{#2}}%
%    \end{macrocode}
%    If not, we call |\@begintheorem|. Its first argument is the
%    name of the theorem set (hence the second argument of
%    |\@thm|). Its second argument is the macro that produces the
%    current number.
%    \begin{macrocode}
   {\@begintheorem{#2}{\csname the#1\endcsname}\ignorespaces}}
%    \end{macrocode}
% \end{macro}
%
% \begin{macro}{\@xthm}
% \begin{macro}{\@ythm}
%    Both these macros were originally called by |\@thm|.  We do
%    not need |\@xthm| anymore, hence we reset it to
%    |\relax|. The definition of |\@ythm| has not changed at
%    all from its definition in \LaTeX. In order to make the macros
%    easier to understand, we will
%    nevertheless present it (commented out).
%    \begin{macrocode}
\global\let\@xthm\relax
% \def\@ythm#1#2[#3]{\@opargbegintheorem{#2}{\csname
%           the#1\endcsname}{#3}\ignorespaces}
%</package>
%    \end{macrocode}
%    The primitive |\ignorespaces| in |\@ythm| and
%    |\@thm| is needed to remove the spaces between the
%    |\begin{...}| and the actual text.
% \end{macro}
% \end{macro}
%
%
% \subsection{Definition of the theorem substyles}
%
% As already pointed out, the theorem substyles, defined below, are
% only loaded when necessary.  Note that all these substyles, except
% \texttt{plain}, have |\slshape| as the default body font.
%
% \subsubsection{The \texttt{plain} style}
%
% \label{dok:th@plain} As the following macros use |@|, we have
% to locally set the |\catcode| of this symbol to ``letter''.
% This happens within a group, so that we do not have to worry about
% which |\catcode| that symbol had before.
%    \begin{macrocode}
%<*thp>
\begingroup \makeatletter
%    \end{macrocode}
% Since we are now within a group, we must make all definitions
% globally.  First we make sure that \texttt{theorem.sty} is loaded.
% This will allow us to use this file as a document style option
% without having to call \texttt{theorem} itself as an option.  At the
% same time, we assure that at least version 2 is loaded, since
% |\theorem@style| was not defined in earlier versions.
%    \begin{macrocode}
\@ifundefined{theorem@style}{\input{theorem.sty}}{}
\ProvidesFile{thp.sty}
             [\FMithmInfo]
%    \end{macrocode}
%
%
% \begin{macro}{\th@plain}
% \begin{macro}{\@begintheorem}
% \begin{macro}{\@opargbegintheorem}
%    |\theoremstyle{plain}| corresponds to the original
%    definition, except that the distances to the surrounding text are
%    determined by the parameters |\theorempreskipamount| and
%    |\theorempostskipamount|.
%    First we set the default body font.
%    \begin{macrocode}
\gdef\th@plain{\normalfont\itshape
%    \end{macrocode}
%    Then we define |\@begintheorem| and
%    |\@opargbegintheorem|. These two macros define how the
%    header of a theorem is typeset.  |\@opargbegintheorem| will
%    be called if a \textsf{theorem} environment with an optional
%    argument is encountered; otherwise, the header is constructed by
%    calling |\@begintheorem|.  If one of these macros is
%    executed, we are within a \textsf{trivlist} environment started by
%    |\@thm|.  So the theorem header is produced with an
%    |\item| command.
%
%    Instead of specifying the header font directly, all standard
%    theorem styles use the |\theorem@headerfont| macro to allow
%    customization.  The extra space (|\labelsep|) is necessary
%    because of problems in the \textsf{trivlist} environment.
%    \begin{macrocode}
  \def\@begintheorem##1##2{%
        \item[\hskip\labelsep \theorem@headerfont ##1\ ##2]}%
%    \end{macrocode}
%    The definition of |\@opargbegintheorem| is completely
%    analogous. The only difference is the fact that there exists a
%    third argument (which is the optional parameter of the
%    environment and contains additional information about the
%    theorem).  Customarily we enclose it in parentheses.
%    \begin{macrocode}
\def\@opargbegintheorem##1##2##3{%
   \item[\hskip\labelsep \theorem@headerfont ##1\ ##2\ (##3)]}}
%    \end{macrocode}
% \end{macro}
% \end{macro}
% \end{macro}
%
% \noindent
% We conclude with an |\endgroup| to restore the
% |\catcode| of |@|.
%    \begin{macrocode}
\endgroup
%</thp>
%    \end{macrocode}
%
%
% \subsubsection{The \texttt{break} style}
%
% This style option is stored in the file |thb.sty|.
% For the next two lines see the documentation for |\th@plain| on
% page~\pageref{dok:th@plain}.
%    \begin{macrocode}
%<*thb>
\begingroup \makeatletter
\@ifundefined{theorem@style}{\input{theorem.sty}}{}
\ProvidesFile{thb.sty}
             [\FMithmInfo]
%    \end{macrocode}
%
% \begin{macro}{\th@break}
%    |\theoremstyle{break}| produces a line break after the name
%    of the theorem.  The font is |\slshape|.  Hence, we define
%    |\th@break| as follows:
%    \begin{macrocode}
\gdef\th@break{\normalfont\slshape
  \def\@begintheorem##1##2{\item[%
%    \end{macrocode}
%    We run into the following problem: it is not possible to create
%    the header with |\item[|\meta{title}|]| and then start
%    a new line by, for example, |\mbox{}\\|. Such a definition
%    will fail whenever a list environment follows immediately.  With
%    the above construction, the |\mbox{}| causes the switch
%    |@inlabel| (cf.\ definition of |\list| and
%    |\trivlist| in \cite{latex:source}) to be set to \texttt{false}
%    and so the following list will insert additional vertical
%    space (|\topskip|). This is quite annoying. Therefore, we
%    create the line break within the |\item|. In order to ensure
%    that the text will begin at the proper position in the following
%    line, we simply pretend that the label does not take any
%    room.\footnote{This will lead to problems whenever very high
%    symbols occurring in the line tower into the
%    heading. So, something else has to be done here sometime.}
%    \begin{macrocode}
       \rlap{\vbox{\hbox{\hskip \labelsep\theorem@headerfont ##1\ ##2}%
                   \hbox{\strut}}}]}%
%    \end{macrocode}
%    Again, the definition of |\@opargbegintheorem| is
%    completely analogous.
%    \begin{macrocode}
\def\@opargbegintheorem##1##2##3{%
  \item[\rlap{\vbox{\hbox{\hskip \labelsep \theorem@headerfont
                     ##1\ ##2\ (##3)}%
                    \hbox{\strut}}}]}}
\endgroup
%</thb>
%    \end{macrocode}
% \end{macro}
%
%
% \subsubsection{The \texttt{changebreak} style}
%
%    \begin{macrocode}
% This style option is stored in the file |thcb.sty|.
%    \begin{macrocode}
%<*thcb>
\begingroup \makeatletter
\@ifundefined{theorem@style}{\input{theorem.sty}}{}
\ProvidesFile{thcb.sty}
             [\FMithmInfo]
%    \end{macrocode}
%
% \begin{macro}{\th@changebreak}
%    The \texttt{change-break} theorem style is like \texttt{break}
%    but with interchange of theorem name and theorem number.  Thus we
%    define |\th@changebreak| as follows:
%    \begin{macrocode}
\gdef\th@changebreak{\normalfont\slshape
  \def\@begintheorem##1##2{\item
       [\rlap{\vbox{\hbox{\hskip\labelsep \theorem@headerfont ##2\ ##1}%
                    \hbox{\strut}}}]}%
\def\@opargbegintheorem##1##2##3{%
  \item[\rlap{\vbox{\hbox{\hskip\labelsep \theorem@headerfont
                     ##2\ ##1\ (##3)}%
                    \hbox{\strut}}}]}}
\endgroup
%</thcb>
%    \end{macrocode}
% \end{macro}
%
%
% \subsubsection{The \texttt{change} style}
%
% This style option is stored in the file |thc.sty|.
%    \begin{macrocode}
%<*thc>
\begingroup \makeatletter
\@ifundefined{theorem@style}{\input{theorem.sty}}{}
\ProvidesFile{thc.sty}
             [\FMithmInfo]
%    \end{macrocode}
%
% \begin{macro}{\th@change}
%    The \texttt{change} theorem style corresponds to the
%    \texttt{change break} style without a linebreak after the header.
%    To say it in another way, it's the same as the \texttt{plain}
%    style but with number and name interchanged and |\slshape| as the
%    default font.
%    \begin{macrocode}
\gdef\th@change{\normalfont\slshape
  \def\@begintheorem##1##2{\item
       [\hskip\labelsep \theorem@headerfont ##2\ ##1]}%
\def\@opargbegintheorem##1##2##3{%
  \item[\hskip\labelsep \theorem@headerfont ##2\ ##1\ (##3)]}}
\endgroup
%</thc>
%    \end{macrocode}
% \end{macro}
%
%
% \subsubsection{The \texttt{marginbreak} style}
%
% This style option is the one used most often at Mainz.  It is saved
% in the file |thmb.sty|.
%    \begin{macrocode}
%<*thmb>
\begingroup \makeatletter
\@ifundefined{theorem@style}{\input{theorem.sty}}{}
\ProvidesFile{thmb.sty}
             [\FMithmInfo]
%    \end{macrocode}
%
% \begin{macro}{\th@marginbreak}
%    The \texttt{margin break} style is nearly the same as the
%    \texttt{change break} style. The only difference is the placement
%    of the theorem number.  We use |\llap| to place it in the left
%    margin.
%
%    In this style |\labelsep| denotes the separation between the
%    number and the text.
%    \begin{macrocode}
\gdef\th@marginbreak{\normalfont\slshape
  \def\@begintheorem##1##2{\item
       [\rlap{\vbox{\theorem@headerfont
                    \hbox{\llap{##2}\hskip\labelsep ##1}%
                    \hbox{\strut}}}]}%
\def\@opargbegintheorem##1##2##3{%
  \item[\rlap{\vbox{\theorem@headerfont
                    \hbox{\llap{##2}\hskip\labelsep ##1\ (##3)}%
                    \hbox{\strut}}}]}}
\endgroup
%</thmb>
%    \end{macrocode}
% \changes{v2.1c}{90/02/17}{`endgroup outside macrocode env.}
% \end{macro}
%
%
%
% \subsubsection{The \texttt{margin} style}
%
% This style option is stored in the file |thm.sty|.
%    \begin{macrocode}
%<*thm>
\begingroup \makeatletter
\@ifundefined{theorem@style}{\input{theorem.sty}}{}
\ProvidesFile{thm.sty}
             [\FMithmInfo]
%    \end{macrocode}
%
% \begin{macro}{\th@margin}
%    Again this is only a variant of the theorem styles described
%    above without any new ideas.
%    \begin{macrocode}
\gdef\th@margin{\normalfont\slshape
  \def\@begintheorem##1##2{\item
       [\theorem@headerfont \llap{##2}\hskip\labelsep ##1]}%
\def\@opargbegintheorem##1##2##3{%
  \item[\theorem@headerfont \llap{##2}\hskip\labelsep ##1\ (##3)]}}
\endgroup
%</thm>
%    \end{macrocode}
% \end{macro}
%
%
%
%
% \subsection{Final Definitions}
%
% \begin{macro}{\theorempreskipamount}
% \begin{macro}{\theorempostskipamount}
%    The \textsf{skip} parameters that regulate the vertical empty space
%    before and after the {\envfont theorem} environment have
%    to be allocated as well.
% \changes{v2.0f}{89/07/20}{Unnecessary `global' removed.}
%    \begin{macrocode}
%<*package>
\newskip\theorempreskipamount
\newskip\theorempostskipamount
%    \end{macrocode}
%    Since we have used the same values for all theorem sets, we now can
%    assign them.
%    \begin{macrocode}
\global\setlength\theorempreskipamount{12pt plus 5pt minus 3pt}
\global\setlength\theorempostskipamount{8pt plus 3pt minus 1.5pt}
%    \end{macrocode}
% \end{macro}
% \end{macro}
%
%
% \begin{macro}{\@endtheorem}
%    The same holds for the macro |\@endtheorem|, which ends a
%    {\envfont theorem} environment. Since it is the same for all
%    theorem sets, it is removed from the macros |\th@|\meta{style}.
%    It simply ends the {\envfont trivlist} environment, which was
%    begun in |\@thm|.
%    \begin{macrocode}
\global\let\@endtheorem=\endtrivlist
%    \end{macrocode}
% \end{macro}
%
%
%
% \begin{macro}{\@preamblecmds}
%    All macros defined above are to be used only in the preamble.
%    Therefore, we insert them in |\@preamblecmds| which will disable
%    them at begin document. This is done by the internal \LaTeXe{}
%    command |\@onlypreamble|.
% \changes{v2.0g}{89/09/19}{`theoremheaderfont and `theorembodyfont
%    added.}
% \changes{v2.2a}{94/02/03}{Use \LaTeXe{} interface}
%    \begin{macrocode}
\@onlypreamble\@xnthm
\@onlypreamble\@ynthm
\@onlypreamble\@othm
\@onlypreamble\newtheorem
\@onlypreamble\theoremstyle
\@onlypreamble\theorembodyfont
\@onlypreamble\theoremheaderfont
%    \end{macrocode}
% \end{macro}
%
% Finally we declare the \texttt{plain} theorem style to be the default.
% \changes{v2.1b}{90/06/04}{Fixed missing `item by declaring default
%    style.}
%    \begin{macrocode}
\theoremstyle{plain}
%</package>
%    \end{macrocode}
%
% \Finale

\endinput
}{}
\ProvidesFile{thb.sty}
             [\FMithmInfo]
%    \end{macrocode}
%
% \begin{macro}{\th@break}
%    |\theoremstyle{break}| produces a line break after the name
%    of the theorem.  The font is |\slshape|.  Hence, we define
%    |\th@break| as follows:
%    \begin{macrocode}
\gdef\th@break{\normalfont\slshape
  \def\@begintheorem##1##2{\item[%
%    \end{macrocode}
%    We run into the following problem: it is not possible to create
%    the header with |\item[|\meta{title}|]| and then start
%    a new line by, for example, |\mbox{}\\|. Such a definition
%    will fail whenever a list environment follows immediately.  With
%    the above construction, the |\mbox{}| causes the switch
%    |@inlabel| (cf.\ definition of |\list| and
%    |\trivlist| in \cite{latex:source}) to be set to \texttt{false}
%    and so the following list will insert additional vertical
%    space (|\topskip|). This is quite annoying. Therefore, we
%    create the line break within the |\item|. In order to ensure
%    that the text will begin at the proper position in the following
%    line, we simply pretend that the label does not take any
%    room.\footnote{This will lead to problems whenever very high
%    symbols occurring in the line tower into the
%    heading. So, something else has to be done here sometime.}
%    \begin{macrocode}
       \rlap{\vbox{\hbox{\hskip \labelsep\theorem@headerfont ##1\ ##2}%
                   \hbox{\strut}}}]}%
%    \end{macrocode}
%    Again, the definition of |\@opargbegintheorem| is
%    completely analogous.
%    \begin{macrocode}
\def\@opargbegintheorem##1##2##3{%
  \item[\rlap{\vbox{\hbox{\hskip \labelsep \theorem@headerfont
                     ##1\ ##2\ (##3)}%
                    \hbox{\strut}}}]}}
\endgroup
%</thb>
%    \end{macrocode}
% \end{macro}
%
%
% \subsubsection{The \texttt{changebreak} style}
%
%    \begin{macrocode}
% This style option is stored in the file |thcb.sty|.
%    \begin{macrocode}
%<*thcb>
\begingroup \makeatletter
\@ifundefined{theorem@style}{% \iffalse meta-comment
%
% Copyright (C) 1993-2024
%
% The LaTeX Project and any individual authors listed elsewhere
% in this file.
%
% This file is part of the Standard LaTeX `Tools Bundle'.
% -------------------------------------------------------
%
% It may be distributed and/or modified under the
% conditions of the LaTeX Project Public License, either version 1.3c
% of this license or (at your option) any later version.
% The latest version of this license is in
%    https://www.latex-project.org/lppl.txt
% and version 1.3c or later is part of all distributions of LaTeX
% version 2005/12/01 or later.
%
% The list of all files belonging to the LaTeX `Tools Bundle' is
% given in the file `manifest.txt'.
%
% \fi
%
%%
%
%
\def\FMithmInfo{2023/07/05 v2.2c Theorem extension package (FMi)}
%
% \ProvidesFile{theorem.dtx}[\FMithmInfo]
%\iffalse   % this is a METACOMMENT !
%%
%% Package `theorem' to use with LaTeX2e
%% Copyright (C) 1989-1995 Frank Mittelbach, all rights reserved.
%<+package>\@ifundefined{theorem@style}{}{\endinput}
%<+package>\NeedsTeXFormat{LaTeX2e}[1995/06/01]
%<+package>\ProvidesPackage{theorem}
%<+package>         [\FMithmInfo]
%
% \fi
%
% \changes{v1.0}{87/05/07}{First draft.}
% \changes{v1.0a}{87/05/07}{`thistheoremstyle' added.}
% \changes{v1.1a}{87/07/29}{New concept: redefine the internal
%                          \protect\LaTeX{} commands `@xnthm' etc.}
% \changes{v1.1b}{87/07/31}{`theoremfont' used to specify font.}
% \changes{v1.2a}{88/01/08}{`@endtheorem' now globally defined.}
% \changes{v1.2b}{88/01/11}{`newtheorem' only allowed in the preamble.}
% \changes{v1.3a}{88/02/14}{`theoremfont' now a toks register.}
%
% \changes{v2.0a}{88/11/16}{New implementation and new documentation.}
% \changes{v2.0a}{88/11/16}{No layout changes but internal modifications
%                           old substyles won't work.}
% \changes{v2.0b}{89/01/30}{`global' used for all definitions.}
% \changes{v2.0c}{89/06/15}{Documentation translated to English.
%                            Thanks to LLL and Nina}
% \changes{v2.1a}{90/02/15}{Support for new font selection.}
% \changes{v2.2a}{94/02/03}{Cleanup for LaTeX2e}
%
% \DoNotIndex{\@endparenv,\@ifdefinable,\@ifnextchar,\@ifundefined}
% \DoNotIndex{\@namedef,\@nameuse,\@spaces,\@tempa,\@warning}
% \DoNotIndex{\ }
% \DoNotIndex{\begingroup,\bfseries}
% \DoNotIndex{\csname}
% \DoNotIndex{\def,\do,\docdate}
% \DoNotIndex{\endcsname,\endgroup,\endinput,\endtrivlist,\expandafter}
% \DoNotIndex{\filedate,\fileversion}
% \DoNotIndex{\gdef,\global}
% \DoNotIndex{\hbox,\hskip}
% \DoNotIndex{\ignorespaces,\input,\itshape,\item}
% \DoNotIndex{\labelsep,\let,\llap}
% \DoNotIndex{\makeatletter}
% \DoNotIndex{\noexpand}
% \DoNotIndex{\refstepcounter,\relax,\rlap}
% \DoNotIndex{\setlength,\slshape,\space,\string,\strut}
% \DoNotIndex{\the,\trivlist,\typeout}
% \DoNotIndex{\vbox}
% \DoNotIndex{\xdef}
%
% \def\deflabel#1{\ttfamily #1\hfill}
% \def\deflist#1{\begin{list}{}{\settowidth\labelwidth{\ttfamily #1}%
%                               \setlength\leftmargin\labelwidth
%                               \addtolength\leftmargin\labelsep
%                               \let\makelabel\deflabel}}
% \def\enddeflist{\end{list}}
% \def\envfont{\normalfont\ttfamily}
%
% \MakeShortVerb{\|}
%
% \newcommand{\TUB}{{\sl TUGboat\/}}
%
% \GetFileInfo{theorem.dtx}
% \title{An Extension of the
%        \LaTeX{} theorem environment\thanks{This file
%        has version number \fileversion, last
%        revised \filedate.}}
% \author{Frank Mittelbach}
%
% \MaintainedByLaTeXTeam{tools}
% \maketitle
%
% \begin{abstract}
%    The macros described in this paper yield an extension of the
%    \LaTeX{} theorem mechanism. It is designed to satisfy the
%    different requirements of various journals. Thus, the
%    layout of the ``theorems'' can be manipulated by determining a
%    ``style''. This article describes not only the use, but
%    also the definition, of the necessary macros.
% \end{abstract}
%
% \section*{Preface to version 2.2}
%
% For \LaTeXe{} this package did not need any fundamental changes. I
% only modified the messages generated so that theorem layout styles
% will show up with the |\listfiles| command and cleaned the section
% on the New Font Selection Scheme since this is now included in
% \LaTeX.
%
% \section*{Preface to version 2.1}
%
% This version is identical to 2.0g described in \TUB\ 10\#3 except
% for some internal defaults which are now set depending on the used
% font selection scheme.
%
% This was done to avoid unpleasant surprises if the new font selection
% scheme is in force. For further details see
% section~\ref{sec:newfonts} and \cite{bk:GMS94}.
%
% \section{Introduction}
%
% For our purposes here, ``theorems'' are labelled enunciations,
% often set off from the main text by extra space and a font change.
% Theorems, corollaries, conjectures, definitions, and remarks
% are all instances of ``theorems''.  The ``header'' of these structures
% is composed of a label (such as \textsc{Theorem} or \textsc{Remark})
% and a number which serializes an item in the sequence of items
% with the same label.
%
% Shortly after the introduction of \LaTeX{} at the Fachbereich
% Mathematik in Mainz, the desire to manipulate the layout of
% ``theorems'' arose.  In Mainz, the following two conventions came
% into general use:
% \begin{enumerate}
%   \item The number of the theorem is shown in the margin.
%   \item There is a line break at
%         the end of the theorem header.
% \end{enumerate}
%
% Additionally, some journals require different formats which depend on
% the ``sort of theorem'': e.g.\ often remarks and definitions are set
% in |\upshape|, while |\itshape| is employed for main theorems.
%
% Confronted with these requirements, a theorem environment was
% developed in Mainz which allows separate determination of the
% layout of the ``theorems sets'', comparable to |\pagestyle|.
%
%
% \section{The user interface}
% \subsection{Defining new theorem sets}
%
% \DescribeMacro\newtheorem
% As in the original \LaTeX{} version, the command |\newtheorem|
% defines a new ``theorem set'' or ``theorem-like structure''.
% Two required arguments name the new environment and give the text
% to be typeset with each instance of the new ``set'', while
% an optional argument determines how
% the ``set'' is enumerated:
% \begin{description}
%    \item[\ttfamily \bslash newtheorem\{foo\}\{bar\}]
%       The theorem set {\envfont foo} (whose name is \texttt{bar})
%       uses its own counter.
%    \item[\ttfamily \bslash newtheorem\{foo2\}{[foo]}\{bar2\}]
%       The theorem set {\envfont foo2} (printed name \texttt{bar2})
%       uses the same counter as the theorem set \texttt{foo}.
%    \item[\ttfamily \bslash newtheorem\{foo3\}\{bar3\}{[section]}]
%       The theorem set {\envfont foo3} (printed name \texttt{bar3}) is
%       enumerated within the counter \texttt{section}, i.e.\ with every
%       new |\section| the enumeration begins again with 1, and
%       the enumeration is composed from the section-number and the
%       theorem counter itself.
% \end{description}
%
% \DescribeMacro\theoremstyle
% Additionally, the command |\theoremstyle| can define the layout
% of various, or all, theorem sets.  It should be noted that any
% theorem set defined by |\newtheorem| is typeset in the
% |\theoremstyle| that is current at the time of the definition.
% Thus, the following
% \begin{verbatim}
%    \theoremstyle{break}        \newtheorem{Cor}{Corollary}
%    \theoremstyle{plain}        \newtheorem{Exa}{Example}[section]
%\end{verbatim}                             ^^A No blank at line start
% leads to the result that the set {\envfont Cor} is formatted in the
% style \texttt{break}, while the set {\envfont Exa} and all the
% following ones are formatted in the style \texttt{plain}, unless
% another |\theoremstyle| follows.  Since the definitions installed by
% |\newtheorem| are global, one also can limit |\theoremstyle| locally
% by grouping braces.
%
% \DescribeMacro\theorembodyfont
% The choice of the font for the theorem body is completely
% independent of the chosen |\theoremstyle|; this has proven to
% be very advantageous.  For example,
% \begin{verbatim}
%   {\theorembodyfont{\upshape}          \newtheorem{Rem}{Remark}}
%\end{verbatim}
% defines a theorem set {\envfont Rem}, which will be set in
% |\upshape| in the current layout (which in our example is
% \texttt{plain}).  As with |\theoremstyle|, the |\theorembodyfont|
% chosen is that current at the time of |\newtheorem|.  If
% |\theorembodyfont| is not specified or one defines
% |\theorembodyfont{}|, then the font used will be that defined by the
% |\theoremstyle|.
%
% \DescribeMacro\theoremheaderfont It is also possible to customize
% the font used for the theorem headers.  This is, however, a global
% declaration, and therefore there should be at most one
% |\theoremheaderfont| declaration in the preamble.\footnote{If it is
% actually necessary to have different header fonts, one has to define
% new theorem styles (substituting the desired font) or specify the
% information directly in the \texttt{\bslash newtheorem} declaration
% (the unclean variant).}
%
% \DescribeMacro\theorempreskipamount
% \DescribeMacro\theorempostskipamount Two additional parameters
% affect the vertical space around the theorem environments:
% |\theorempreskipamount| and |\theorempostskipamount| define,
% respectively, the spacing before and after such an environment.
% These parameters apply for all theorem sets and can be manipulated
% with the ordinary length macros.  They are rubber lengths
% (`\textsf{skips}'), and therefore can contain \texttt{plus} and
% \texttt{minus} parts.
%
% Since the definition of theorem sets should---most sensibly---be
% placed in the preamble, we only allow installation there.
% It is therefore possible to release the memory used here after
% |\begin{document}|, in order to make room for other
% applications.
%
%
% \subsection{Existing theorem styles}
%
% The following theorem styles exist to date:
% \begin{deflist}{marginbreak}
%    \item[plain]
%       This theorem style emulates the original \LaTeX{} definition,
%       except that additionally the parameters
%       |\theorem...skipamount| are used.
%    \item[break]
%       In this style, the theorem header is followed by a line
%       break.
%    \item[marginbreak]
%       The theorem number is set in the margin, and there is a line
%       break as in \texttt{break}.
%    \item[changebreak]
%       Like \texttt{break}, but with header number and text
%       interchanged.
%    \item[change]
%       Header number and text are interchanged, without a line break.
%    \item[margin]
%       The number is set in the left margin, without a line break.
% \end{deflist}
% All styles (except \texttt{plain}) select |\slshape| as the default
% |\theorembodyfont|.
%
%
% \subsection{Examples}
%
% Given the above theorem sets {\envfont Cor}, {\envfont Exa}
% and {\envfont Rem},
% suppose that the preamble also contains the declarations:
% \begin{verbatim}
%    \theoremstyle{marginbreak}   \newtheorem{Lem}[Cor]{Lemma}
%    \theoremstyle{change}
%    \theorembodyfont{\itshape}        \newtheorem{Def}[Cor]{Definition}
%
%    \theoremheaderfont{\scshape}
%\end{verbatim}
% Then the following are some typical examples of the typeset output
% resulting from their use.
%
% \begin{Cor}
%    This is a sentence typeset in the theorem environment {\envfont
%    Cor}.
% \end{Cor}
%
% \begin{Exa}
%    This is a sentence typeset in the theorem environment {\envfont
%    Exa}.
% \end{Exa}
%
% \begin{Rem}
%    This is a sentence typeset in the theorem environment {\envfont
%    Rem}.
% \end{Rem}
%
% \begin{Lem}[Ben User]
%    This is a sentence typeset in the theorem environment {\envfont
%    Lem}.
% \end{Lem}
%
% \begin{Def}[Very impressive Definition]
%    This is a sentence typeset in the theorem environment {\envfont
%    Def}.
% \end{Def}
%
% The last two examples show the effect of the optional argument to
% a theorem environment (it is the text typeset in parentheses).
%
%
% \section{Special Considerations}
% \label{sec:newfonts}
%
% Theoremheader and body are implemented as a unit. This means that
% the |\theoremheaderfont|  will inherit characteristics of the
% |\theorembodyfont| in \LaTeXe. Thus, if for example
% |\theorembodyfont| is |\itshape| and |\theoremheaderfont|
% is |\bfseries| the font selected for the header will have the
% characteristics `bold extended italic'. If this is not desired one
% should set the |\theoremheaderfont| to something like
% \begin{verbatim}
%      \theoremheaderfont{\normalfont\bfseries}
%\end{verbatim}
% i.e.\ supplying all necessary font information explicitly.
%
%
%
% \section{Acknowledgements}
%
% The publication of this set of macros was only possible with the
% help of Christina Busse (translating the manuscript into English),
% Joachim Pense (playing the r\^ole of typist), Chris Rowley
% (looking everything over) and many others providing useful
% suggestions.
%
% \MaybeStop{%
%    \begin{thebibliography}{9}
%       \bibitem{bk:GMS94} \textsc{M.~Goossens}, \textsc{F.~Mittelbach}
%         and \textsc{A.~Samarin}.
%         \newblock The \LaTeX{} Companion.
%         \newblock
%         Addison-Wesley, Reading, Massachusetts, 1994.
%       \bibitem{latex:source} \textsc{Lamport, Leslie.} \newblock
%                               \texttt{latex.tex}, version 2.09, date
%                               Feb. 1990.
%    \end{thebibliography}
%    \PrintIndex
%    \PrintChanges}
%
%
% \section{The documentation driver file}
%
% The next bit of code contains the documentation driver file for
% \TeX{}, i.e., the file that will produce the documentation you are
% currently reading. It will be extracted from this file by the
% \texttt{docstrip} program. Since it is the first code in the file
% one can alternatively process this file directly with \LaTeXe{} to
% obtain the documentation.
%
%    \begin{macrocode}
%<*driver>
\documentclass{ltxdoc}

\usepackage{theorem}

%<+driver>% The next few lines define theorem sets which are used
%<+driver>% in the example section of the documentation.

\theoremstyle{break}       \newtheorem{Cor}{Corollary}
\theoremstyle{plain}       \newtheorem{Exa}{Example}[section]
{\theorembodyfont{\upshape}\newtheorem{Rem}{Remark}}
\theoremstyle{marginbreak} \newtheorem{Lem}[Cor]{Lemma}
\theoremstyle{change}
\theorembodyfont{\itshape} \newtheorem{Def}[Cor]{Definition}

\theoremheaderfont{\scshape}

\RecordChanges

\begin{document}
   \DocInput{theorem.dtx}
\end{document}
%</driver>
%    \end{macrocode}
%
% \section{Definition of the Macros}
%
% If the file has been loaded before, we abort immediately. If not the
% package announces itself (this is actually done at the very top of
% the file---the way it is done isn't good style so don't copy it).
% \changes{v2.0e}{89/07/19}{Spaces removed from `typeout'.}
% \changes{v2.2b}{95/11/19}{Announce moved to top of file}
%    \begin{macrocode}
%<*package>
%\@ifundefined{theorem@style}{}{\endinput}
%\def\FMithmInfo{1995/11/19 v2.2b Theorem extension package (FMi)}
%\ProvidesPackage{theorem}[\FMithmInfo]
%    \end{macrocode}
%
%
% \subsection{Definition of theorem styles and fonts}
%
% All the definitions in this file are done globally to allow
% inputting this file inside a group.
%
% \begin{macro}{\theoremstyle}
%    Before a theorem style can be installed, the chosen style must be
%    known. For that reason, we must test to see that
%    |\th@|\meta{style} is known or, more precisely, that it is
%    different from |\relax|. If the style is not known then
%    |\th@plain| is used.
%    \begin{macrocode}
\gdef\theoremstyle#1{%
   \@ifundefined{th@#1}{\@warning
          {Unknown theoremstyle `#1'. Using `plain'}%
          \theorem@style{plain}}%
%    \end{macrocode}
%    We save the theorem style to be used in the token register
%    |\theorem@style|.
%    \begin{macrocode}
      {\theorem@style{#1}}%
%    \end{macrocode}
%    Now we ``evaluate'' the theorem style: this means, we call the
%    macro |\th@|\meta{style} which will activate the relevant
%    definitions which are contained in a separate file.  This is done
%    in a group to suppress changes to the current font.  This could
%    otherwise pose problems together with the new font selection
%    scheme\footnote{When I printed the original article using the new
%    font selection scheme I ended with a document with slanted
%    typefaces (text headings and all) simply because one of the
%    theorem styles used \texttt{\bslash sl} at toplevel.} if the
%    |\th@|\meta{style} is evaluated a second time.
% \changes{v2.1a}{90/02/15}{Group added around `th@..}
% \changes{v2.1a}{90/02/15}{Reseting of `..@begintheorem removed.}
%    \begin{macrocode}
      \begingroup
        \csname th@\the\theorem@style \endcsname
      \endgroup}
%    \end{macrocode}
% \end{macro}
%
%
% \begin{macro}{\@begintheorem}
% \changes{v2.1a}{90/02/15}{Relaxed at toplevel.}
% \begin{macro}{\@opargbegintheorem}
% \changes{v2.1a}{90/02/15}{Relaxed at toplevel.}
%    We reset |\@begintheorem| and |\@opargbegintheorem| to
%    |\relax| since these commands are no longer necessary at
%    toplevel. This will save a few tokens.
%    \begin{macrocode}
\global\let\@begintheorem\relax
\global\let\@opargbegintheorem\relax
%    \end{macrocode}
% \end{macro}
% \end{macro}
%
% \begin{macro}{\theorem@style}
%    Obviously the \textsf{token} register used above has to be
%    allocated. To assure the utmost compatibility with the
%    original \LaTeX{} definition, we set the default theorem style to
%    \texttt{plain}, which implements the usual \LaTeX{} convention.
% \changes{v2.0f}{89/07/20}{Unnecessary `global' removed.}
%    \begin{macrocode}
\newtoks\theorem@style
\global\theorem@style{plain}
%    \end{macrocode}
% \end{macro}
%
% \begin{macro}{\theorembodyfont}
% \begin{macro}{\theorem@bodyfont}
%    If the |\theorembodyfont| is set by the user then it should not
%    interact with the default font set in the theorem style. When the
%    new font selection is in force this may happen if, for example,
%    the default is |\itshape| and the new |\theorembodyfont| is
%    |\sffamily|. So we add a |\reset@font| command in front of the
%    user definition.
% \changes{v2.0f}{89/07/20}{Unnecessary `global' removed.}
% \changes{v2.0g}{89/09/19}{`theoremfont to `theorembodyfont.}
% \changes{v2.1c}{92/06/22}{`reset@font added and logic changed.}
% \changes{v2.1d}{92/10/28}{Reset to empty if arg is empty}
%    \begin{macrocode}
\gdef\theorembodyfont#1{%
%    \end{macrocode}
%    We check if the argument supplied is empty and if so put nothing
%    into the |\theorem@bodyfont| token register to allow for
%    |\theorembodyfont{}| as a mean of using the default of the
%    current |\theoremstyle|.
%    \begin{macrocode}
   \def\@tempa{#1}%
   \ifx\@tempa\@empty
    \theorem@bodyfont{}%
   \else
    \theorem@bodyfont{\reset@font#1}%
   \fi
   }
\newtoks\theorem@bodyfont
\global\theorem@bodyfont{}
%    \end{macrocode}
% \end{macro}
% \end{macro}
%
%
% \begin{macro}{\theoremheaderfont}
% \changes{v2.0g}{89/09/19}{Macro added.}
%    The font for the theorem headers is handled differently because
%    this definition applies to all theorem styles.
%    \begin{macrocode}
\gdef\theoremheaderfont#1{\gdef\theorem@headerfont{#1}%
%    \end{macrocode}
%    After using the macro once it is redefined to produce an error
%    message.
%    \begin{macrocode}
       \gdef\theoremheaderfont##1{%
        \typeout{\string\theoremheaderfont\space should be used
                 only once.}}}
%    \end{macrocode}
% \end{macro}
%
% \begin{macro}{\theorem@headerfont}
% \changes{v2.0g}{89/09/19}{Macro added.}
%    To set the |\theorem@headerfont| default we first test if
%    the new fontselection scheme is in force.
% \changes{v2.1a}{90/02/15}{Support for new font selection.}
%    \begin{macrocode}
\ifx\upshape\undefined
%    \end{macrocode}
%    If not we define it to expand into |\bfseries|. We don't use
%    |\let| just in case a following style option redefines this
%    macro.
%    \begin{macrocode}
\gdef\theorem@headerfont{\bfseries}
%    \end{macrocode}
%    Otherwise we reset the current shape before calling |\bfseries|.
%    \begin{macrocode}
\else \gdef\theorem@headerfont{\normalfont\bfseries}\fi
%    \end{macrocode}
% \end{macro}
%
% \begin{macro}{\th@plain}
% \begin{macro}{\th@break}
% \begin{macro}{\th@marginbreak}
% \begin{macro}{\th@changebreak}
% \begin{macro}{\th@change}
% \begin{macro}{\th@margin}
%    The different styles are defined in macros such as
%    |\th@plain|.  Since memory space is precious in
%    ``non-Big-versions'', we have to avoid offering too many unused
%    definitions.  Therefore we define these styles in separate files
%    that can be loaded on demand.  Thus the commands themselves only
%    load these files. We use |\@input@|, a \LaTeXe{} internal command
%    that ensures that the file will be listed with |\listfiles|.
% \changes{v2.2a}{94/02/02}{Use `@input@ to load theorem layout files}
%    \begin{macrocode}
\gdef\th@plain{\@input@{thp.sty}}
\gdef\th@break{\@input@{thb.sty}}
\gdef\th@marginbreak{\@input@{thmb.sty}}
\gdef\th@changebreak{\@input@{thcb.sty}}
\gdef\th@change{\@input@{thc.sty}}
\gdef\th@margin{\@input@{thm.sty}}
%    \end{macrocode}
%    This list will be expanded when new styles become
%    available.  For testing,  just append new theorem substyles
%    as document options.
% \end{macro}
% \end{macro}
% \end{macro}
% \end{macro}
% \end{macro}
% \end{macro}
%
%
%
% \subsection{Definition of a new theorem set}
%
% As already pointed out, a new theorem environment can be defined in
% three different ways:
% \begin{verbatim}
%    \newtheorem{Lem}{Lemma}
%    \newtheorem{Lem}{Lemma}[section]
%    \newtheorem{Lem}[Theorem]{Lemma}
%\end{verbatim}
% The function of the macro |\newtheorem| is to recognize these cases
% and then to branch into one of the three macros |\@ynthm|, |\@xnthm|
% or |\@othm|.  This mechanism is adopted unchanged from
% \cite{latex:source}; the essential point here is that, for example,
% in the second case, the arguments \texttt{Lem}, \texttt{Lemma} and
% \texttt{section} are passed over to the macro |\@xnthm|.
%
% We inspect this case first because the others present fewer problems,
% and thus are easily derived from this one.
%
% \begin{macro}{\@xnthm}
%    For our example arguments, the macro |\@xnthm| must fulfill
%    the following:
%    \begin{itemize}
%       \item
%          Define a new \LaTeX-counter `Lem'
%       \item
%          reset this counter within a |\section|
%       \item
%          define the macro |\theLem|
%       \item
%          define the environment macros |\Lem| and
%          |\endLem| using the current
%          |\theoremstyle| and |\theorem@bodyfont|.
%    \end{itemize}
%    Obviously, all this should happen only if the first argument
%    of |\@xnthm| (i.e.\ \texttt{Lem} in our example) is chosen so
%    as not to conflict with any previously defined commands or
%    environments.  This test is performed by the \LaTeX{} macro
%    |\@ifdefinable|.
%    \begin{macrocode}
\gdef\@xnthm#1#2[#3]{\expandafter\@ifdefinable\csname #1\endcsname
%    \end{macrocode}
%    Therefore, the first argument of |\@ifdefinable| is the
%    expansion (in the example, |\Lem|) of
%    |\csname||#1\endcsname|.  The second argument is
%    executed only if the test has been completed successfully.
%    \begin{macrocode}
   {%
%    \end{macrocode}
%    Now we define the new counter. The names of the \LaTeX{}
%    macros employed should speak for themselves:
% \changes{v2.2b}{1995/10/07}{Use \cs{@newctr}}
% \changes{v2.2c}{1995/11/23}{Use \cs{@newctr} properly.}
%    \begin{macrocode}
    \@definecounter{#1}\@newctr{#1}[#3]%
%    \end{macrocode}
%    Using |\@newctr| will give a proper error message if the counter
%    in |#3| is not defined.
%    In defining `|\theLem|' we must generate the desired macro
%    name by use of |\expandafter| and |\csname|.
%    \begin{macrocode}
    \expandafter\xdef\csname the#1\endcsname
%    \end{macrocode}
%    An |\xdef| is used in order to make the definition global,
%    and to ensure that it contains the replacement texts of
%    |\@thmcountersep| and |\@thmcounter|.\footnote{These
%    two macros can be defined by the document style.  Their default
%    values produce a `\texttt{.}' as separation and an arabic
%    representation of the number.} However, not everything should be
%    expanded. For example, it saves space to use |\thesection|
%    instead of its---at times---lengthy expansion.
%    \begin{macrocode}
      {\expandafter \noexpand \csname the#3\endcsname
       \@thmcountersep \@thmcounter{#1}}%
%    \end{macrocode}
%    Thus with the defaults of \LaTeX{}, |\theLem| would be
%    replaced by the command sequence |\thesection.\arabic{Lem}|.
%
%    We will now look at the definition of the macro which is executed
%    at the beginning of the actual environment (in our example this
%    macro is |\Lem|). It should be noted that we use an
%    ``|\expandafter| trick'' to expand only certain parts of the
%    replacement text at the time of the definition.
%    \begin{macrocode}
    \def\@tempa{\global\@namedef{#1}}%
    \expandafter \@tempa \expandafter{%
%    \end{macrocode}
%    First, the macro that contains the current definitions of
%    |\@begintheorem| and |\@opargtheorem| should be called
%    up.  The name of this macro---as is already known---has the form
%    |\th@|\meta{theorem\ style}; therefore, it must be called by
%    \begin{macrocode}
      \csname th@\the \theorem@style
%    \end{macrocode}
%    In addition the default theorem font should be changeable, i.e.\
%    we have to insert the contents of |\theorem@bodyfont|. For
%    that reason, we expand even further, beyond |\endcsname|,
%    and thus insert the contents of the \textsf{token} register
%    |\theorem@bodyfont| in the replacement text.
%    \begin{macrocode}
            \expandafter \endcsname \the \theorem@bodyfont
%    \end{macrocode}
%    Now it is time to call the macro |\@thm| which takes over
%    the further processing.  It has two arguments: the current
%    counter name (in our example, \texttt{Lem}), and the text of the
%    label (in our example, \texttt{Lemma}).
%    \begin{macrocode}
     \@thm{#1}{#2}}%
%    \end{macrocode}
%    With this, the `sub-definition' is complete.  The macro
%    |\@endtheorem| ends a theorem environment and is, so far, nothing
%    but an |\endtrivlist|.  (Hence it is defined globally, and not
%    within the theorem styles.\footnote{This has to be changed as
%    soon as theorem styles that change \texttt{\bslash @endtheorem}
%    exist.  In such a case, all existing styles must be changed as
%    well since they will have to reset the macro.}) Therefore, we can
%    set it equivalent to the macro that ends the theorem set (in our
%    example, |\endLem|).  However, if some day theorem styles exist
%    that do change |\@endtheorem|, we would have to use the
%    commented-out line instead.
%    \begin{macrocode}
    \global \expandafter \let \csname end#1\endcsname \@endtheorem
%   \global\@namedef{end#1}{\@endtheorem}%
%    \end{macrocode}
%    With these commands all the required definitions are employed,
%    unless the test |\@ifdefinable| has failed.  Therefore, we
%    end the second argument of this macro and with it the definition
%    of |\@xnthm|.
%    \begin{macrocode}
   }}
%    \end{macrocode}
% \end{macro}
%
% \begin{macro}{\@ynthm}
%    The definition of |\@ynthm| is completely analogous. In
%    this case the new counter that is defined is not reset within
%    another counter; thus the definition of |\the...|
%    is simplified:
%    \begin{macrocode}
\gdef\@ynthm#1#2{\expandafter\@ifdefinable\csname #1\endcsname
   {\@definecounter{#1}%
    \expandafter\xdef\csname the#1\endcsname{\@thmcounter{#1}}%
%    \end{macrocode}
%    The rest of the definition corresponds literally to that of
%    |\@xnthm|:
%    \begin{macrocode}
    \def\@tempa{\global\@namedef{#1}}\expandafter \@tempa
     \expandafter{\csname th@\the \theorem@style \expandafter
     \endcsname \the\theorem@bodyfont \@thm{#1}{#2}}%
    \global \expandafter \let \csname end#1\endcsname \@endtheorem}}
%    \end{macrocode}
% \end{macro}
%
% \begin{macro}{\@othm}
%    The definition of |\@othm| does not contain anything new.
%    \begin{macrocode}
\gdef\@othm#1[#2]#3{%
%    \end{macrocode}
% \changes{v2.2b}{95/10/07}{Build in check for counter defined}
%    We do not define a new counter but instead use one that has
%    already been defined.  Thus the only definition we need is that
%    of this pseudo-counter (i.e.\ |\the|\meta{env.\ name}).
%    First we check if |#2| corresponds to a known counter name.
%    \begin{macrocode}
  \expandafter\ifx\csname c@#2\endcsname\relax
   \@nocounterr{#2}%
  \else
   \expandafter\@ifdefinable\csname #1\endcsname
   {\expandafter \xdef \csname the#1\endcsname
     {\expandafter \noexpand \csname the#2\endcsname}%
%    \end{macrocode}
%    All other parts of the definition can be adopted from
%    |\@xnthm|. We have to remember, though, that in this case
%    the name of the current counter and the theorem label have moved
%    to the second and third arguments.
%    \begin{macrocode}
    \def\@tempa{\global\@namedef{#1}}\expandafter \@tempa
     \expandafter{\csname th@\the \theorem@style \expandafter
     \endcsname \the\theorem@bodyfont \@thm{#2}{#3}}%
    \global \expandafter \let \csname end#1\endcsname \@endtheorem}%
  \fi}
%    \end{macrocode}
% \end{macro}
%
% \subsection{Macros that are employed in a {\envfont theorem}
%    environment}
%
% \begin{macro}{\@thm}
%    The macro |\@thm| has to increase the current counter.  Then,
%    depending on whether the environment has (or does not have) an
%    optional argument, it has to branch into either |\@begintheorem|
%    or |\@opargtheorem|.
%    \begin{macrocode}
\gdef\@thm#1#2{\refstepcounter{#1}%
%    \end{macrocode}
%    Now we start a {\envfont trivlist} environment, and give
%    |\@topsep| and |\@topsepadd| the values of the \textsf{skip}
%    registers |\theorempreskipamount| and |\theorempostskipamount|.
%    The value in |\@topsep| is the vertical space that is inserted by
%    the first (and only) |\item| in our |\trivlist| whilst
%    |\@topsepadd| is inserted by |\@endparenv| at the end of that
%    {\envfont trivlist} environment.  By using these registers, we
%    obtain the desired space around a {\envfont theorem} environment.
%    \begin{macrocode}
   \trivlist
   \@topsep \theorempreskipamount               % used by first \item
   \@topsepadd \theorempostskipamount           % used by \@endparenv
%    \end{macrocode}
%    Now we have to test whether an optional argument has been given.
% \changes{v2.0d}{89/07/18}{Forgotten percent added.}
%    \begin{macrocode}
   \@ifnextchar [%
%    \end{macrocode}
%    If there is an optional argument, we will call |\@ythm|, and
%    move the arguments read back into the input stream.
%    \begin{macrocode}
   {\@ythm{#1}{#2}}%
%    \end{macrocode}
%    If not, we call |\@begintheorem|. Its first argument is the
%    name of the theorem set (hence the second argument of
%    |\@thm|). Its second argument is the macro that produces the
%    current number.
%    \begin{macrocode}
   {\@begintheorem{#2}{\csname the#1\endcsname}\ignorespaces}}
%    \end{macrocode}
% \end{macro}
%
% \begin{macro}{\@xthm}
% \begin{macro}{\@ythm}
%    Both these macros were originally called by |\@thm|.  We do
%    not need |\@xthm| anymore, hence we reset it to
%    |\relax|. The definition of |\@ythm| has not changed at
%    all from its definition in \LaTeX. In order to make the macros
%    easier to understand, we will
%    nevertheless present it (commented out).
%    \begin{macrocode}
\global\let\@xthm\relax
% \def\@ythm#1#2[#3]{\@opargbegintheorem{#2}{\csname
%           the#1\endcsname}{#3}\ignorespaces}
%</package>
%    \end{macrocode}
%    The primitive |\ignorespaces| in |\@ythm| and
%    |\@thm| is needed to remove the spaces between the
%    |\begin{...}| and the actual text.
% \end{macro}
% \end{macro}
%
%
% \subsection{Definition of the theorem substyles}
%
% As already pointed out, the theorem substyles, defined below, are
% only loaded when necessary.  Note that all these substyles, except
% \texttt{plain}, have |\slshape| as the default body font.
%
% \subsubsection{The \texttt{plain} style}
%
% \label{dok:th@plain} As the following macros use |@|, we have
% to locally set the |\catcode| of this symbol to ``letter''.
% This happens within a group, so that we do not have to worry about
% which |\catcode| that symbol had before.
%    \begin{macrocode}
%<*thp>
\begingroup \makeatletter
%    \end{macrocode}
% Since we are now within a group, we must make all definitions
% globally.  First we make sure that \texttt{theorem.sty} is loaded.
% This will allow us to use this file as a document style option
% without having to call \texttt{theorem} itself as an option.  At the
% same time, we assure that at least version 2 is loaded, since
% |\theorem@style| was not defined in earlier versions.
%    \begin{macrocode}
\@ifundefined{theorem@style}{\input{theorem.sty}}{}
\ProvidesFile{thp.sty}
             [\FMithmInfo]
%    \end{macrocode}
%
%
% \begin{macro}{\th@plain}
% \begin{macro}{\@begintheorem}
% \begin{macro}{\@opargbegintheorem}
%    |\theoremstyle{plain}| corresponds to the original
%    definition, except that the distances to the surrounding text are
%    determined by the parameters |\theorempreskipamount| and
%    |\theorempostskipamount|.
%    First we set the default body font.
%    \begin{macrocode}
\gdef\th@plain{\normalfont\itshape
%    \end{macrocode}
%    Then we define |\@begintheorem| and
%    |\@opargbegintheorem|. These two macros define how the
%    header of a theorem is typeset.  |\@opargbegintheorem| will
%    be called if a \textsf{theorem} environment with an optional
%    argument is encountered; otherwise, the header is constructed by
%    calling |\@begintheorem|.  If one of these macros is
%    executed, we are within a \textsf{trivlist} environment started by
%    |\@thm|.  So the theorem header is produced with an
%    |\item| command.
%
%    Instead of specifying the header font directly, all standard
%    theorem styles use the |\theorem@headerfont| macro to allow
%    customization.  The extra space (|\labelsep|) is necessary
%    because of problems in the \textsf{trivlist} environment.
%    \begin{macrocode}
  \def\@begintheorem##1##2{%
        \item[\hskip\labelsep \theorem@headerfont ##1\ ##2]}%
%    \end{macrocode}
%    The definition of |\@opargbegintheorem| is completely
%    analogous. The only difference is the fact that there exists a
%    third argument (which is the optional parameter of the
%    environment and contains additional information about the
%    theorem).  Customarily we enclose it in parentheses.
%    \begin{macrocode}
\def\@opargbegintheorem##1##2##3{%
   \item[\hskip\labelsep \theorem@headerfont ##1\ ##2\ (##3)]}}
%    \end{macrocode}
% \end{macro}
% \end{macro}
% \end{macro}
%
% \noindent
% We conclude with an |\endgroup| to restore the
% |\catcode| of |@|.
%    \begin{macrocode}
\endgroup
%</thp>
%    \end{macrocode}
%
%
% \subsubsection{The \texttt{break} style}
%
% This style option is stored in the file |thb.sty|.
% For the next two lines see the documentation for |\th@plain| on
% page~\pageref{dok:th@plain}.
%    \begin{macrocode}
%<*thb>
\begingroup \makeatletter
\@ifundefined{theorem@style}{\input{theorem.sty}}{}
\ProvidesFile{thb.sty}
             [\FMithmInfo]
%    \end{macrocode}
%
% \begin{macro}{\th@break}
%    |\theoremstyle{break}| produces a line break after the name
%    of the theorem.  The font is |\slshape|.  Hence, we define
%    |\th@break| as follows:
%    \begin{macrocode}
\gdef\th@break{\normalfont\slshape
  \def\@begintheorem##1##2{\item[%
%    \end{macrocode}
%    We run into the following problem: it is not possible to create
%    the header with |\item[|\meta{title}|]| and then start
%    a new line by, for example, |\mbox{}\\|. Such a definition
%    will fail whenever a list environment follows immediately.  With
%    the above construction, the |\mbox{}| causes the switch
%    |@inlabel| (cf.\ definition of |\list| and
%    |\trivlist| in \cite{latex:source}) to be set to \texttt{false}
%    and so the following list will insert additional vertical
%    space (|\topskip|). This is quite annoying. Therefore, we
%    create the line break within the |\item|. In order to ensure
%    that the text will begin at the proper position in the following
%    line, we simply pretend that the label does not take any
%    room.\footnote{This will lead to problems whenever very high
%    symbols occurring in the line tower into the
%    heading. So, something else has to be done here sometime.}
%    \begin{macrocode}
       \rlap{\vbox{\hbox{\hskip \labelsep\theorem@headerfont ##1\ ##2}%
                   \hbox{\strut}}}]}%
%    \end{macrocode}
%    Again, the definition of |\@opargbegintheorem| is
%    completely analogous.
%    \begin{macrocode}
\def\@opargbegintheorem##1##2##3{%
  \item[\rlap{\vbox{\hbox{\hskip \labelsep \theorem@headerfont
                     ##1\ ##2\ (##3)}%
                    \hbox{\strut}}}]}}
\endgroup
%</thb>
%    \end{macrocode}
% \end{macro}
%
%
% \subsubsection{The \texttt{changebreak} style}
%
%    \begin{macrocode}
% This style option is stored in the file |thcb.sty|.
%    \begin{macrocode}
%<*thcb>
\begingroup \makeatletter
\@ifundefined{theorem@style}{\input{theorem.sty}}{}
\ProvidesFile{thcb.sty}
             [\FMithmInfo]
%    \end{macrocode}
%
% \begin{macro}{\th@changebreak}
%    The \texttt{change-break} theorem style is like \texttt{break}
%    but with interchange of theorem name and theorem number.  Thus we
%    define |\th@changebreak| as follows:
%    \begin{macrocode}
\gdef\th@changebreak{\normalfont\slshape
  \def\@begintheorem##1##2{\item
       [\rlap{\vbox{\hbox{\hskip\labelsep \theorem@headerfont ##2\ ##1}%
                    \hbox{\strut}}}]}%
\def\@opargbegintheorem##1##2##3{%
  \item[\rlap{\vbox{\hbox{\hskip\labelsep \theorem@headerfont
                     ##2\ ##1\ (##3)}%
                    \hbox{\strut}}}]}}
\endgroup
%</thcb>
%    \end{macrocode}
% \end{macro}
%
%
% \subsubsection{The \texttt{change} style}
%
% This style option is stored in the file |thc.sty|.
%    \begin{macrocode}
%<*thc>
\begingroup \makeatletter
\@ifundefined{theorem@style}{\input{theorem.sty}}{}
\ProvidesFile{thc.sty}
             [\FMithmInfo]
%    \end{macrocode}
%
% \begin{macro}{\th@change}
%    The \texttt{change} theorem style corresponds to the
%    \texttt{change break} style without a linebreak after the header.
%    To say it in another way, it's the same as the \texttt{plain}
%    style but with number and name interchanged and |\slshape| as the
%    default font.
%    \begin{macrocode}
\gdef\th@change{\normalfont\slshape
  \def\@begintheorem##1##2{\item
       [\hskip\labelsep \theorem@headerfont ##2\ ##1]}%
\def\@opargbegintheorem##1##2##3{%
  \item[\hskip\labelsep \theorem@headerfont ##2\ ##1\ (##3)]}}
\endgroup
%</thc>
%    \end{macrocode}
% \end{macro}
%
%
% \subsubsection{The \texttt{marginbreak} style}
%
% This style option is the one used most often at Mainz.  It is saved
% in the file |thmb.sty|.
%    \begin{macrocode}
%<*thmb>
\begingroup \makeatletter
\@ifundefined{theorem@style}{\input{theorem.sty}}{}
\ProvidesFile{thmb.sty}
             [\FMithmInfo]
%    \end{macrocode}
%
% \begin{macro}{\th@marginbreak}
%    The \texttt{margin break} style is nearly the same as the
%    \texttt{change break} style. The only difference is the placement
%    of the theorem number.  We use |\llap| to place it in the left
%    margin.
%
%    In this style |\labelsep| denotes the separation between the
%    number and the text.
%    \begin{macrocode}
\gdef\th@marginbreak{\normalfont\slshape
  \def\@begintheorem##1##2{\item
       [\rlap{\vbox{\theorem@headerfont
                    \hbox{\llap{##2}\hskip\labelsep ##1}%
                    \hbox{\strut}}}]}%
\def\@opargbegintheorem##1##2##3{%
  \item[\rlap{\vbox{\theorem@headerfont
                    \hbox{\llap{##2}\hskip\labelsep ##1\ (##3)}%
                    \hbox{\strut}}}]}}
\endgroup
%</thmb>
%    \end{macrocode}
% \changes{v2.1c}{90/02/17}{`endgroup outside macrocode env.}
% \end{macro}
%
%
%
% \subsubsection{The \texttt{margin} style}
%
% This style option is stored in the file |thm.sty|.
%    \begin{macrocode}
%<*thm>
\begingroup \makeatletter
\@ifundefined{theorem@style}{\input{theorem.sty}}{}
\ProvidesFile{thm.sty}
             [\FMithmInfo]
%    \end{macrocode}
%
% \begin{macro}{\th@margin}
%    Again this is only a variant of the theorem styles described
%    above without any new ideas.
%    \begin{macrocode}
\gdef\th@margin{\normalfont\slshape
  \def\@begintheorem##1##2{\item
       [\theorem@headerfont \llap{##2}\hskip\labelsep ##1]}%
\def\@opargbegintheorem##1##2##3{%
  \item[\theorem@headerfont \llap{##2}\hskip\labelsep ##1\ (##3)]}}
\endgroup
%</thm>
%    \end{macrocode}
% \end{macro}
%
%
%
%
% \subsection{Final Definitions}
%
% \begin{macro}{\theorempreskipamount}
% \begin{macro}{\theorempostskipamount}
%    The \textsf{skip} parameters that regulate the vertical empty space
%    before and after the {\envfont theorem} environment have
%    to be allocated as well.
% \changes{v2.0f}{89/07/20}{Unnecessary `global' removed.}
%    \begin{macrocode}
%<*package>
\newskip\theorempreskipamount
\newskip\theorempostskipamount
%    \end{macrocode}
%    Since we have used the same values for all theorem sets, we now can
%    assign them.
%    \begin{macrocode}
\global\setlength\theorempreskipamount{12pt plus 5pt minus 3pt}
\global\setlength\theorempostskipamount{8pt plus 3pt minus 1.5pt}
%    \end{macrocode}
% \end{macro}
% \end{macro}
%
%
% \begin{macro}{\@endtheorem}
%    The same holds for the macro |\@endtheorem|, which ends a
%    {\envfont theorem} environment. Since it is the same for all
%    theorem sets, it is removed from the macros |\th@|\meta{style}.
%    It simply ends the {\envfont trivlist} environment, which was
%    begun in |\@thm|.
%    \begin{macrocode}
\global\let\@endtheorem=\endtrivlist
%    \end{macrocode}
% \end{macro}
%
%
%
% \begin{macro}{\@preamblecmds}
%    All macros defined above are to be used only in the preamble.
%    Therefore, we insert them in |\@preamblecmds| which will disable
%    them at begin document. This is done by the internal \LaTeXe{}
%    command |\@onlypreamble|.
% \changes{v2.0g}{89/09/19}{`theoremheaderfont and `theorembodyfont
%    added.}
% \changes{v2.2a}{94/02/03}{Use \LaTeXe{} interface}
%    \begin{macrocode}
\@onlypreamble\@xnthm
\@onlypreamble\@ynthm
\@onlypreamble\@othm
\@onlypreamble\newtheorem
\@onlypreamble\theoremstyle
\@onlypreamble\theorembodyfont
\@onlypreamble\theoremheaderfont
%    \end{macrocode}
% \end{macro}
%
% Finally we declare the \texttt{plain} theorem style to be the default.
% \changes{v2.1b}{90/06/04}{Fixed missing `item by declaring default
%    style.}
%    \begin{macrocode}
\theoremstyle{plain}
%</package>
%    \end{macrocode}
%
% \Finale

\endinput
}{}
\ProvidesFile{thcb.sty}
             [\FMithmInfo]
%    \end{macrocode}
%
% \begin{macro}{\th@changebreak}
%    The \texttt{change-break} theorem style is like \texttt{break}
%    but with interchange of theorem name and theorem number.  Thus we
%    define |\th@changebreak| as follows:
%    \begin{macrocode}
\gdef\th@changebreak{\normalfont\slshape
  \def\@begintheorem##1##2{\item
       [\rlap{\vbox{\hbox{\hskip\labelsep \theorem@headerfont ##2\ ##1}%
                    \hbox{\strut}}}]}%
\def\@opargbegintheorem##1##2##3{%
  \item[\rlap{\vbox{\hbox{\hskip\labelsep \theorem@headerfont
                     ##2\ ##1\ (##3)}%
                    \hbox{\strut}}}]}}
\endgroup
%</thcb>
%    \end{macrocode}
% \end{macro}
%
%
% \subsubsection{The \texttt{change} style}
%
% This style option is stored in the file |thc.sty|.
%    \begin{macrocode}
%<*thc>
\begingroup \makeatletter
\@ifundefined{theorem@style}{% \iffalse meta-comment
%
% Copyright (C) 1993-2024
%
% The LaTeX Project and any individual authors listed elsewhere
% in this file.
%
% This file is part of the Standard LaTeX `Tools Bundle'.
% -------------------------------------------------------
%
% It may be distributed and/or modified under the
% conditions of the LaTeX Project Public License, either version 1.3c
% of this license or (at your option) any later version.
% The latest version of this license is in
%    https://www.latex-project.org/lppl.txt
% and version 1.3c or later is part of all distributions of LaTeX
% version 2005/12/01 or later.
%
% The list of all files belonging to the LaTeX `Tools Bundle' is
% given in the file `manifest.txt'.
%
% \fi
%
%%
%
%
\def\FMithmInfo{2023/07/05 v2.2c Theorem extension package (FMi)}
%
% \ProvidesFile{theorem.dtx}[\FMithmInfo]
%\iffalse   % this is a METACOMMENT !
%%
%% Package `theorem' to use with LaTeX2e
%% Copyright (C) 1989-1995 Frank Mittelbach, all rights reserved.
%<+package>\@ifundefined{theorem@style}{}{\endinput}
%<+package>\NeedsTeXFormat{LaTeX2e}[1995/06/01]
%<+package>\ProvidesPackage{theorem}
%<+package>         [\FMithmInfo]
%
% \fi
%
% \changes{v1.0}{87/05/07}{First draft.}
% \changes{v1.0a}{87/05/07}{`thistheoremstyle' added.}
% \changes{v1.1a}{87/07/29}{New concept: redefine the internal
%                          \protect\LaTeX{} commands `@xnthm' etc.}
% \changes{v1.1b}{87/07/31}{`theoremfont' used to specify font.}
% \changes{v1.2a}{88/01/08}{`@endtheorem' now globally defined.}
% \changes{v1.2b}{88/01/11}{`newtheorem' only allowed in the preamble.}
% \changes{v1.3a}{88/02/14}{`theoremfont' now a toks register.}
%
% \changes{v2.0a}{88/11/16}{New implementation and new documentation.}
% \changes{v2.0a}{88/11/16}{No layout changes but internal modifications
%                           old substyles won't work.}
% \changes{v2.0b}{89/01/30}{`global' used for all definitions.}
% \changes{v2.0c}{89/06/15}{Documentation translated to English.
%                            Thanks to LLL and Nina}
% \changes{v2.1a}{90/02/15}{Support for new font selection.}
% \changes{v2.2a}{94/02/03}{Cleanup for LaTeX2e}
%
% \DoNotIndex{\@endparenv,\@ifdefinable,\@ifnextchar,\@ifundefined}
% \DoNotIndex{\@namedef,\@nameuse,\@spaces,\@tempa,\@warning}
% \DoNotIndex{\ }
% \DoNotIndex{\begingroup,\bfseries}
% \DoNotIndex{\csname}
% \DoNotIndex{\def,\do,\docdate}
% \DoNotIndex{\endcsname,\endgroup,\endinput,\endtrivlist,\expandafter}
% \DoNotIndex{\filedate,\fileversion}
% \DoNotIndex{\gdef,\global}
% \DoNotIndex{\hbox,\hskip}
% \DoNotIndex{\ignorespaces,\input,\itshape,\item}
% \DoNotIndex{\labelsep,\let,\llap}
% \DoNotIndex{\makeatletter}
% \DoNotIndex{\noexpand}
% \DoNotIndex{\refstepcounter,\relax,\rlap}
% \DoNotIndex{\setlength,\slshape,\space,\string,\strut}
% \DoNotIndex{\the,\trivlist,\typeout}
% \DoNotIndex{\vbox}
% \DoNotIndex{\xdef}
%
% \def\deflabel#1{\ttfamily #1\hfill}
% \def\deflist#1{\begin{list}{}{\settowidth\labelwidth{\ttfamily #1}%
%                               \setlength\leftmargin\labelwidth
%                               \addtolength\leftmargin\labelsep
%                               \let\makelabel\deflabel}}
% \def\enddeflist{\end{list}}
% \def\envfont{\normalfont\ttfamily}
%
% \MakeShortVerb{\|}
%
% \newcommand{\TUB}{{\sl TUGboat\/}}
%
% \GetFileInfo{theorem.dtx}
% \title{An Extension of the
%        \LaTeX{} theorem environment\thanks{This file
%        has version number \fileversion, last
%        revised \filedate.}}
% \author{Frank Mittelbach}
%
% \MaintainedByLaTeXTeam{tools}
% \maketitle
%
% \begin{abstract}
%    The macros described in this paper yield an extension of the
%    \LaTeX{} theorem mechanism. It is designed to satisfy the
%    different requirements of various journals. Thus, the
%    layout of the ``theorems'' can be manipulated by determining a
%    ``style''. This article describes not only the use, but
%    also the definition, of the necessary macros.
% \end{abstract}
%
% \section*{Preface to version 2.2}
%
% For \LaTeXe{} this package did not need any fundamental changes. I
% only modified the messages generated so that theorem layout styles
% will show up with the |\listfiles| command and cleaned the section
% on the New Font Selection Scheme since this is now included in
% \LaTeX.
%
% \section*{Preface to version 2.1}
%
% This version is identical to 2.0g described in \TUB\ 10\#3 except
% for some internal defaults which are now set depending on the used
% font selection scheme.
%
% This was done to avoid unpleasant surprises if the new font selection
% scheme is in force. For further details see
% section~\ref{sec:newfonts} and \cite{bk:GMS94}.
%
% \section{Introduction}
%
% For our purposes here, ``theorems'' are labelled enunciations,
% often set off from the main text by extra space and a font change.
% Theorems, corollaries, conjectures, definitions, and remarks
% are all instances of ``theorems''.  The ``header'' of these structures
% is composed of a label (such as \textsc{Theorem} or \textsc{Remark})
% and a number which serializes an item in the sequence of items
% with the same label.
%
% Shortly after the introduction of \LaTeX{} at the Fachbereich
% Mathematik in Mainz, the desire to manipulate the layout of
% ``theorems'' arose.  In Mainz, the following two conventions came
% into general use:
% \begin{enumerate}
%   \item The number of the theorem is shown in the margin.
%   \item There is a line break at
%         the end of the theorem header.
% \end{enumerate}
%
% Additionally, some journals require different formats which depend on
% the ``sort of theorem'': e.g.\ often remarks and definitions are set
% in |\upshape|, while |\itshape| is employed for main theorems.
%
% Confronted with these requirements, a theorem environment was
% developed in Mainz which allows separate determination of the
% layout of the ``theorems sets'', comparable to |\pagestyle|.
%
%
% \section{The user interface}
% \subsection{Defining new theorem sets}
%
% \DescribeMacro\newtheorem
% As in the original \LaTeX{} version, the command |\newtheorem|
% defines a new ``theorem set'' or ``theorem-like structure''.
% Two required arguments name the new environment and give the text
% to be typeset with each instance of the new ``set'', while
% an optional argument determines how
% the ``set'' is enumerated:
% \begin{description}
%    \item[\ttfamily \bslash newtheorem\{foo\}\{bar\}]
%       The theorem set {\envfont foo} (whose name is \texttt{bar})
%       uses its own counter.
%    \item[\ttfamily \bslash newtheorem\{foo2\}{[foo]}\{bar2\}]
%       The theorem set {\envfont foo2} (printed name \texttt{bar2})
%       uses the same counter as the theorem set \texttt{foo}.
%    \item[\ttfamily \bslash newtheorem\{foo3\}\{bar3\}{[section]}]
%       The theorem set {\envfont foo3} (printed name \texttt{bar3}) is
%       enumerated within the counter \texttt{section}, i.e.\ with every
%       new |\section| the enumeration begins again with 1, and
%       the enumeration is composed from the section-number and the
%       theorem counter itself.
% \end{description}
%
% \DescribeMacro\theoremstyle
% Additionally, the command |\theoremstyle| can define the layout
% of various, or all, theorem sets.  It should be noted that any
% theorem set defined by |\newtheorem| is typeset in the
% |\theoremstyle| that is current at the time of the definition.
% Thus, the following
% \begin{verbatim}
%    \theoremstyle{break}        \newtheorem{Cor}{Corollary}
%    \theoremstyle{plain}        \newtheorem{Exa}{Example}[section]
%\end{verbatim}                             ^^A No blank at line start
% leads to the result that the set {\envfont Cor} is formatted in the
% style \texttt{break}, while the set {\envfont Exa} and all the
% following ones are formatted in the style \texttt{plain}, unless
% another |\theoremstyle| follows.  Since the definitions installed by
% |\newtheorem| are global, one also can limit |\theoremstyle| locally
% by grouping braces.
%
% \DescribeMacro\theorembodyfont
% The choice of the font for the theorem body is completely
% independent of the chosen |\theoremstyle|; this has proven to
% be very advantageous.  For example,
% \begin{verbatim}
%   {\theorembodyfont{\upshape}          \newtheorem{Rem}{Remark}}
%\end{verbatim}
% defines a theorem set {\envfont Rem}, which will be set in
% |\upshape| in the current layout (which in our example is
% \texttt{plain}).  As with |\theoremstyle|, the |\theorembodyfont|
% chosen is that current at the time of |\newtheorem|.  If
% |\theorembodyfont| is not specified or one defines
% |\theorembodyfont{}|, then the font used will be that defined by the
% |\theoremstyle|.
%
% \DescribeMacro\theoremheaderfont It is also possible to customize
% the font used for the theorem headers.  This is, however, a global
% declaration, and therefore there should be at most one
% |\theoremheaderfont| declaration in the preamble.\footnote{If it is
% actually necessary to have different header fonts, one has to define
% new theorem styles (substituting the desired font) or specify the
% information directly in the \texttt{\bslash newtheorem} declaration
% (the unclean variant).}
%
% \DescribeMacro\theorempreskipamount
% \DescribeMacro\theorempostskipamount Two additional parameters
% affect the vertical space around the theorem environments:
% |\theorempreskipamount| and |\theorempostskipamount| define,
% respectively, the spacing before and after such an environment.
% These parameters apply for all theorem sets and can be manipulated
% with the ordinary length macros.  They are rubber lengths
% (`\textsf{skips}'), and therefore can contain \texttt{plus} and
% \texttt{minus} parts.
%
% Since the definition of theorem sets should---most sensibly---be
% placed in the preamble, we only allow installation there.
% It is therefore possible to release the memory used here after
% |\begin{document}|, in order to make room for other
% applications.
%
%
% \subsection{Existing theorem styles}
%
% The following theorem styles exist to date:
% \begin{deflist}{marginbreak}
%    \item[plain]
%       This theorem style emulates the original \LaTeX{} definition,
%       except that additionally the parameters
%       |\theorem...skipamount| are used.
%    \item[break]
%       In this style, the theorem header is followed by a line
%       break.
%    \item[marginbreak]
%       The theorem number is set in the margin, and there is a line
%       break as in \texttt{break}.
%    \item[changebreak]
%       Like \texttt{break}, but with header number and text
%       interchanged.
%    \item[change]
%       Header number and text are interchanged, without a line break.
%    \item[margin]
%       The number is set in the left margin, without a line break.
% \end{deflist}
% All styles (except \texttt{plain}) select |\slshape| as the default
% |\theorembodyfont|.
%
%
% \subsection{Examples}
%
% Given the above theorem sets {\envfont Cor}, {\envfont Exa}
% and {\envfont Rem},
% suppose that the preamble also contains the declarations:
% \begin{verbatim}
%    \theoremstyle{marginbreak}   \newtheorem{Lem}[Cor]{Lemma}
%    \theoremstyle{change}
%    \theorembodyfont{\itshape}        \newtheorem{Def}[Cor]{Definition}
%
%    \theoremheaderfont{\scshape}
%\end{verbatim}
% Then the following are some typical examples of the typeset output
% resulting from their use.
%
% \begin{Cor}
%    This is a sentence typeset in the theorem environment {\envfont
%    Cor}.
% \end{Cor}
%
% \begin{Exa}
%    This is a sentence typeset in the theorem environment {\envfont
%    Exa}.
% \end{Exa}
%
% \begin{Rem}
%    This is a sentence typeset in the theorem environment {\envfont
%    Rem}.
% \end{Rem}
%
% \begin{Lem}[Ben User]
%    This is a sentence typeset in the theorem environment {\envfont
%    Lem}.
% \end{Lem}
%
% \begin{Def}[Very impressive Definition]
%    This is a sentence typeset in the theorem environment {\envfont
%    Def}.
% \end{Def}
%
% The last two examples show the effect of the optional argument to
% a theorem environment (it is the text typeset in parentheses).
%
%
% \section{Special Considerations}
% \label{sec:newfonts}
%
% Theoremheader and body are implemented as a unit. This means that
% the |\theoremheaderfont|  will inherit characteristics of the
% |\theorembodyfont| in \LaTeXe. Thus, if for example
% |\theorembodyfont| is |\itshape| and |\theoremheaderfont|
% is |\bfseries| the font selected for the header will have the
% characteristics `bold extended italic'. If this is not desired one
% should set the |\theoremheaderfont| to something like
% \begin{verbatim}
%      \theoremheaderfont{\normalfont\bfseries}
%\end{verbatim}
% i.e.\ supplying all necessary font information explicitly.
%
%
%
% \section{Acknowledgements}
%
% The publication of this set of macros was only possible with the
% help of Christina Busse (translating the manuscript into English),
% Joachim Pense (playing the r\^ole of typist), Chris Rowley
% (looking everything over) and many others providing useful
% suggestions.
%
% \MaybeStop{%
%    \begin{thebibliography}{9}
%       \bibitem{bk:GMS94} \textsc{M.~Goossens}, \textsc{F.~Mittelbach}
%         and \textsc{A.~Samarin}.
%         \newblock The \LaTeX{} Companion.
%         \newblock
%         Addison-Wesley, Reading, Massachusetts, 1994.
%       \bibitem{latex:source} \textsc{Lamport, Leslie.} \newblock
%                               \texttt{latex.tex}, version 2.09, date
%                               Feb. 1990.
%    \end{thebibliography}
%    \PrintIndex
%    \PrintChanges}
%
%
% \section{The documentation driver file}
%
% The next bit of code contains the documentation driver file for
% \TeX{}, i.e., the file that will produce the documentation you are
% currently reading. It will be extracted from this file by the
% \texttt{docstrip} program. Since it is the first code in the file
% one can alternatively process this file directly with \LaTeXe{} to
% obtain the documentation.
%
%    \begin{macrocode}
%<*driver>
\documentclass{ltxdoc}

\usepackage{theorem}

%<+driver>% The next few lines define theorem sets which are used
%<+driver>% in the example section of the documentation.

\theoremstyle{break}       \newtheorem{Cor}{Corollary}
\theoremstyle{plain}       \newtheorem{Exa}{Example}[section]
{\theorembodyfont{\upshape}\newtheorem{Rem}{Remark}}
\theoremstyle{marginbreak} \newtheorem{Lem}[Cor]{Lemma}
\theoremstyle{change}
\theorembodyfont{\itshape} \newtheorem{Def}[Cor]{Definition}

\theoremheaderfont{\scshape}

\RecordChanges

\begin{document}
   \DocInput{theorem.dtx}
\end{document}
%</driver>
%    \end{macrocode}
%
% \section{Definition of the Macros}
%
% If the file has been loaded before, we abort immediately. If not the
% package announces itself (this is actually done at the very top of
% the file---the way it is done isn't good style so don't copy it).
% \changes{v2.0e}{89/07/19}{Spaces removed from `typeout'.}
% \changes{v2.2b}{95/11/19}{Announce moved to top of file}
%    \begin{macrocode}
%<*package>
%\@ifundefined{theorem@style}{}{\endinput}
%\def\FMithmInfo{1995/11/19 v2.2b Theorem extension package (FMi)}
%\ProvidesPackage{theorem}[\FMithmInfo]
%    \end{macrocode}
%
%
% \subsection{Definition of theorem styles and fonts}
%
% All the definitions in this file are done globally to allow
% inputting this file inside a group.
%
% \begin{macro}{\theoremstyle}
%    Before a theorem style can be installed, the chosen style must be
%    known. For that reason, we must test to see that
%    |\th@|\meta{style} is known or, more precisely, that it is
%    different from |\relax|. If the style is not known then
%    |\th@plain| is used.
%    \begin{macrocode}
\gdef\theoremstyle#1{%
   \@ifundefined{th@#1}{\@warning
          {Unknown theoremstyle `#1'. Using `plain'}%
          \theorem@style{plain}}%
%    \end{macrocode}
%    We save the theorem style to be used in the token register
%    |\theorem@style|.
%    \begin{macrocode}
      {\theorem@style{#1}}%
%    \end{macrocode}
%    Now we ``evaluate'' the theorem style: this means, we call the
%    macro |\th@|\meta{style} which will activate the relevant
%    definitions which are contained in a separate file.  This is done
%    in a group to suppress changes to the current font.  This could
%    otherwise pose problems together with the new font selection
%    scheme\footnote{When I printed the original article using the new
%    font selection scheme I ended with a document with slanted
%    typefaces (text headings and all) simply because one of the
%    theorem styles used \texttt{\bslash sl} at toplevel.} if the
%    |\th@|\meta{style} is evaluated a second time.
% \changes{v2.1a}{90/02/15}{Group added around `th@..}
% \changes{v2.1a}{90/02/15}{Reseting of `..@begintheorem removed.}
%    \begin{macrocode}
      \begingroup
        \csname th@\the\theorem@style \endcsname
      \endgroup}
%    \end{macrocode}
% \end{macro}
%
%
% \begin{macro}{\@begintheorem}
% \changes{v2.1a}{90/02/15}{Relaxed at toplevel.}
% \begin{macro}{\@opargbegintheorem}
% \changes{v2.1a}{90/02/15}{Relaxed at toplevel.}
%    We reset |\@begintheorem| and |\@opargbegintheorem| to
%    |\relax| since these commands are no longer necessary at
%    toplevel. This will save a few tokens.
%    \begin{macrocode}
\global\let\@begintheorem\relax
\global\let\@opargbegintheorem\relax
%    \end{macrocode}
% \end{macro}
% \end{macro}
%
% \begin{macro}{\theorem@style}
%    Obviously the \textsf{token} register used above has to be
%    allocated. To assure the utmost compatibility with the
%    original \LaTeX{} definition, we set the default theorem style to
%    \texttt{plain}, which implements the usual \LaTeX{} convention.
% \changes{v2.0f}{89/07/20}{Unnecessary `global' removed.}
%    \begin{macrocode}
\newtoks\theorem@style
\global\theorem@style{plain}
%    \end{macrocode}
% \end{macro}
%
% \begin{macro}{\theorembodyfont}
% \begin{macro}{\theorem@bodyfont}
%    If the |\theorembodyfont| is set by the user then it should not
%    interact with the default font set in the theorem style. When the
%    new font selection is in force this may happen if, for example,
%    the default is |\itshape| and the new |\theorembodyfont| is
%    |\sffamily|. So we add a |\reset@font| command in front of the
%    user definition.
% \changes{v2.0f}{89/07/20}{Unnecessary `global' removed.}
% \changes{v2.0g}{89/09/19}{`theoremfont to `theorembodyfont.}
% \changes{v2.1c}{92/06/22}{`reset@font added and logic changed.}
% \changes{v2.1d}{92/10/28}{Reset to empty if arg is empty}
%    \begin{macrocode}
\gdef\theorembodyfont#1{%
%    \end{macrocode}
%    We check if the argument supplied is empty and if so put nothing
%    into the |\theorem@bodyfont| token register to allow for
%    |\theorembodyfont{}| as a mean of using the default of the
%    current |\theoremstyle|.
%    \begin{macrocode}
   \def\@tempa{#1}%
   \ifx\@tempa\@empty
    \theorem@bodyfont{}%
   \else
    \theorem@bodyfont{\reset@font#1}%
   \fi
   }
\newtoks\theorem@bodyfont
\global\theorem@bodyfont{}
%    \end{macrocode}
% \end{macro}
% \end{macro}
%
%
% \begin{macro}{\theoremheaderfont}
% \changes{v2.0g}{89/09/19}{Macro added.}
%    The font for the theorem headers is handled differently because
%    this definition applies to all theorem styles.
%    \begin{macrocode}
\gdef\theoremheaderfont#1{\gdef\theorem@headerfont{#1}%
%    \end{macrocode}
%    After using the macro once it is redefined to produce an error
%    message.
%    \begin{macrocode}
       \gdef\theoremheaderfont##1{%
        \typeout{\string\theoremheaderfont\space should be used
                 only once.}}}
%    \end{macrocode}
% \end{macro}
%
% \begin{macro}{\theorem@headerfont}
% \changes{v2.0g}{89/09/19}{Macro added.}
%    To set the |\theorem@headerfont| default we first test if
%    the new fontselection scheme is in force.
% \changes{v2.1a}{90/02/15}{Support for new font selection.}
%    \begin{macrocode}
\ifx\upshape\undefined
%    \end{macrocode}
%    If not we define it to expand into |\bfseries|. We don't use
%    |\let| just in case a following style option redefines this
%    macro.
%    \begin{macrocode}
\gdef\theorem@headerfont{\bfseries}
%    \end{macrocode}
%    Otherwise we reset the current shape before calling |\bfseries|.
%    \begin{macrocode}
\else \gdef\theorem@headerfont{\normalfont\bfseries}\fi
%    \end{macrocode}
% \end{macro}
%
% \begin{macro}{\th@plain}
% \begin{macro}{\th@break}
% \begin{macro}{\th@marginbreak}
% \begin{macro}{\th@changebreak}
% \begin{macro}{\th@change}
% \begin{macro}{\th@margin}
%    The different styles are defined in macros such as
%    |\th@plain|.  Since memory space is precious in
%    ``non-Big-versions'', we have to avoid offering too many unused
%    definitions.  Therefore we define these styles in separate files
%    that can be loaded on demand.  Thus the commands themselves only
%    load these files. We use |\@input@|, a \LaTeXe{} internal command
%    that ensures that the file will be listed with |\listfiles|.
% \changes{v2.2a}{94/02/02}{Use `@input@ to load theorem layout files}
%    \begin{macrocode}
\gdef\th@plain{\@input@{thp.sty}}
\gdef\th@break{\@input@{thb.sty}}
\gdef\th@marginbreak{\@input@{thmb.sty}}
\gdef\th@changebreak{\@input@{thcb.sty}}
\gdef\th@change{\@input@{thc.sty}}
\gdef\th@margin{\@input@{thm.sty}}
%    \end{macrocode}
%    This list will be expanded when new styles become
%    available.  For testing,  just append new theorem substyles
%    as document options.
% \end{macro}
% \end{macro}
% \end{macro}
% \end{macro}
% \end{macro}
% \end{macro}
%
%
%
% \subsection{Definition of a new theorem set}
%
% As already pointed out, a new theorem environment can be defined in
% three different ways:
% \begin{verbatim}
%    \newtheorem{Lem}{Lemma}
%    \newtheorem{Lem}{Lemma}[section]
%    \newtheorem{Lem}[Theorem]{Lemma}
%\end{verbatim}
% The function of the macro |\newtheorem| is to recognize these cases
% and then to branch into one of the three macros |\@ynthm|, |\@xnthm|
% or |\@othm|.  This mechanism is adopted unchanged from
% \cite{latex:source}; the essential point here is that, for example,
% in the second case, the arguments \texttt{Lem}, \texttt{Lemma} and
% \texttt{section} are passed over to the macro |\@xnthm|.
%
% We inspect this case first because the others present fewer problems,
% and thus are easily derived from this one.
%
% \begin{macro}{\@xnthm}
%    For our example arguments, the macro |\@xnthm| must fulfill
%    the following:
%    \begin{itemize}
%       \item
%          Define a new \LaTeX-counter `Lem'
%       \item
%          reset this counter within a |\section|
%       \item
%          define the macro |\theLem|
%       \item
%          define the environment macros |\Lem| and
%          |\endLem| using the current
%          |\theoremstyle| and |\theorem@bodyfont|.
%    \end{itemize}
%    Obviously, all this should happen only if the first argument
%    of |\@xnthm| (i.e.\ \texttt{Lem} in our example) is chosen so
%    as not to conflict with any previously defined commands or
%    environments.  This test is performed by the \LaTeX{} macro
%    |\@ifdefinable|.
%    \begin{macrocode}
\gdef\@xnthm#1#2[#3]{\expandafter\@ifdefinable\csname #1\endcsname
%    \end{macrocode}
%    Therefore, the first argument of |\@ifdefinable| is the
%    expansion (in the example, |\Lem|) of
%    |\csname||#1\endcsname|.  The second argument is
%    executed only if the test has been completed successfully.
%    \begin{macrocode}
   {%
%    \end{macrocode}
%    Now we define the new counter. The names of the \LaTeX{}
%    macros employed should speak for themselves:
% \changes{v2.2b}{1995/10/07}{Use \cs{@newctr}}
% \changes{v2.2c}{1995/11/23}{Use \cs{@newctr} properly.}
%    \begin{macrocode}
    \@definecounter{#1}\@newctr{#1}[#3]%
%    \end{macrocode}
%    Using |\@newctr| will give a proper error message if the counter
%    in |#3| is not defined.
%    In defining `|\theLem|' we must generate the desired macro
%    name by use of |\expandafter| and |\csname|.
%    \begin{macrocode}
    \expandafter\xdef\csname the#1\endcsname
%    \end{macrocode}
%    An |\xdef| is used in order to make the definition global,
%    and to ensure that it contains the replacement texts of
%    |\@thmcountersep| and |\@thmcounter|.\footnote{These
%    two macros can be defined by the document style.  Their default
%    values produce a `\texttt{.}' as separation and an arabic
%    representation of the number.} However, not everything should be
%    expanded. For example, it saves space to use |\thesection|
%    instead of its---at times---lengthy expansion.
%    \begin{macrocode}
      {\expandafter \noexpand \csname the#3\endcsname
       \@thmcountersep \@thmcounter{#1}}%
%    \end{macrocode}
%    Thus with the defaults of \LaTeX{}, |\theLem| would be
%    replaced by the command sequence |\thesection.\arabic{Lem}|.
%
%    We will now look at the definition of the macro which is executed
%    at the beginning of the actual environment (in our example this
%    macro is |\Lem|). It should be noted that we use an
%    ``|\expandafter| trick'' to expand only certain parts of the
%    replacement text at the time of the definition.
%    \begin{macrocode}
    \def\@tempa{\global\@namedef{#1}}%
    \expandafter \@tempa \expandafter{%
%    \end{macrocode}
%    First, the macro that contains the current definitions of
%    |\@begintheorem| and |\@opargtheorem| should be called
%    up.  The name of this macro---as is already known---has the form
%    |\th@|\meta{theorem\ style}; therefore, it must be called by
%    \begin{macrocode}
      \csname th@\the \theorem@style
%    \end{macrocode}
%    In addition the default theorem font should be changeable, i.e.\
%    we have to insert the contents of |\theorem@bodyfont|. For
%    that reason, we expand even further, beyond |\endcsname|,
%    and thus insert the contents of the \textsf{token} register
%    |\theorem@bodyfont| in the replacement text.
%    \begin{macrocode}
            \expandafter \endcsname \the \theorem@bodyfont
%    \end{macrocode}
%    Now it is time to call the macro |\@thm| which takes over
%    the further processing.  It has two arguments: the current
%    counter name (in our example, \texttt{Lem}), and the text of the
%    label (in our example, \texttt{Lemma}).
%    \begin{macrocode}
     \@thm{#1}{#2}}%
%    \end{macrocode}
%    With this, the `sub-definition' is complete.  The macro
%    |\@endtheorem| ends a theorem environment and is, so far, nothing
%    but an |\endtrivlist|.  (Hence it is defined globally, and not
%    within the theorem styles.\footnote{This has to be changed as
%    soon as theorem styles that change \texttt{\bslash @endtheorem}
%    exist.  In such a case, all existing styles must be changed as
%    well since they will have to reset the macro.}) Therefore, we can
%    set it equivalent to the macro that ends the theorem set (in our
%    example, |\endLem|).  However, if some day theorem styles exist
%    that do change |\@endtheorem|, we would have to use the
%    commented-out line instead.
%    \begin{macrocode}
    \global \expandafter \let \csname end#1\endcsname \@endtheorem
%   \global\@namedef{end#1}{\@endtheorem}%
%    \end{macrocode}
%    With these commands all the required definitions are employed,
%    unless the test |\@ifdefinable| has failed.  Therefore, we
%    end the second argument of this macro and with it the definition
%    of |\@xnthm|.
%    \begin{macrocode}
   }}
%    \end{macrocode}
% \end{macro}
%
% \begin{macro}{\@ynthm}
%    The definition of |\@ynthm| is completely analogous. In
%    this case the new counter that is defined is not reset within
%    another counter; thus the definition of |\the...|
%    is simplified:
%    \begin{macrocode}
\gdef\@ynthm#1#2{\expandafter\@ifdefinable\csname #1\endcsname
   {\@definecounter{#1}%
    \expandafter\xdef\csname the#1\endcsname{\@thmcounter{#1}}%
%    \end{macrocode}
%    The rest of the definition corresponds literally to that of
%    |\@xnthm|:
%    \begin{macrocode}
    \def\@tempa{\global\@namedef{#1}}\expandafter \@tempa
     \expandafter{\csname th@\the \theorem@style \expandafter
     \endcsname \the\theorem@bodyfont \@thm{#1}{#2}}%
    \global \expandafter \let \csname end#1\endcsname \@endtheorem}}
%    \end{macrocode}
% \end{macro}
%
% \begin{macro}{\@othm}
%    The definition of |\@othm| does not contain anything new.
%    \begin{macrocode}
\gdef\@othm#1[#2]#3{%
%    \end{macrocode}
% \changes{v2.2b}{95/10/07}{Build in check for counter defined}
%    We do not define a new counter but instead use one that has
%    already been defined.  Thus the only definition we need is that
%    of this pseudo-counter (i.e.\ |\the|\meta{env.\ name}).
%    First we check if |#2| corresponds to a known counter name.
%    \begin{macrocode}
  \expandafter\ifx\csname c@#2\endcsname\relax
   \@nocounterr{#2}%
  \else
   \expandafter\@ifdefinable\csname #1\endcsname
   {\expandafter \xdef \csname the#1\endcsname
     {\expandafter \noexpand \csname the#2\endcsname}%
%    \end{macrocode}
%    All other parts of the definition can be adopted from
%    |\@xnthm|. We have to remember, though, that in this case
%    the name of the current counter and the theorem label have moved
%    to the second and third arguments.
%    \begin{macrocode}
    \def\@tempa{\global\@namedef{#1}}\expandafter \@tempa
     \expandafter{\csname th@\the \theorem@style \expandafter
     \endcsname \the\theorem@bodyfont \@thm{#2}{#3}}%
    \global \expandafter \let \csname end#1\endcsname \@endtheorem}%
  \fi}
%    \end{macrocode}
% \end{macro}
%
% \subsection{Macros that are employed in a {\envfont theorem}
%    environment}
%
% \begin{macro}{\@thm}
%    The macro |\@thm| has to increase the current counter.  Then,
%    depending on whether the environment has (or does not have) an
%    optional argument, it has to branch into either |\@begintheorem|
%    or |\@opargtheorem|.
%    \begin{macrocode}
\gdef\@thm#1#2{\refstepcounter{#1}%
%    \end{macrocode}
%    Now we start a {\envfont trivlist} environment, and give
%    |\@topsep| and |\@topsepadd| the values of the \textsf{skip}
%    registers |\theorempreskipamount| and |\theorempostskipamount|.
%    The value in |\@topsep| is the vertical space that is inserted by
%    the first (and only) |\item| in our |\trivlist| whilst
%    |\@topsepadd| is inserted by |\@endparenv| at the end of that
%    {\envfont trivlist} environment.  By using these registers, we
%    obtain the desired space around a {\envfont theorem} environment.
%    \begin{macrocode}
   \trivlist
   \@topsep \theorempreskipamount               % used by first \item
   \@topsepadd \theorempostskipamount           % used by \@endparenv
%    \end{macrocode}
%    Now we have to test whether an optional argument has been given.
% \changes{v2.0d}{89/07/18}{Forgotten percent added.}
%    \begin{macrocode}
   \@ifnextchar [%
%    \end{macrocode}
%    If there is an optional argument, we will call |\@ythm|, and
%    move the arguments read back into the input stream.
%    \begin{macrocode}
   {\@ythm{#1}{#2}}%
%    \end{macrocode}
%    If not, we call |\@begintheorem|. Its first argument is the
%    name of the theorem set (hence the second argument of
%    |\@thm|). Its second argument is the macro that produces the
%    current number.
%    \begin{macrocode}
   {\@begintheorem{#2}{\csname the#1\endcsname}\ignorespaces}}
%    \end{macrocode}
% \end{macro}
%
% \begin{macro}{\@xthm}
% \begin{macro}{\@ythm}
%    Both these macros were originally called by |\@thm|.  We do
%    not need |\@xthm| anymore, hence we reset it to
%    |\relax|. The definition of |\@ythm| has not changed at
%    all from its definition in \LaTeX. In order to make the macros
%    easier to understand, we will
%    nevertheless present it (commented out).
%    \begin{macrocode}
\global\let\@xthm\relax
% \def\@ythm#1#2[#3]{\@opargbegintheorem{#2}{\csname
%           the#1\endcsname}{#3}\ignorespaces}
%</package>
%    \end{macrocode}
%    The primitive |\ignorespaces| in |\@ythm| and
%    |\@thm| is needed to remove the spaces between the
%    |\begin{...}| and the actual text.
% \end{macro}
% \end{macro}
%
%
% \subsection{Definition of the theorem substyles}
%
% As already pointed out, the theorem substyles, defined below, are
% only loaded when necessary.  Note that all these substyles, except
% \texttt{plain}, have |\slshape| as the default body font.
%
% \subsubsection{The \texttt{plain} style}
%
% \label{dok:th@plain} As the following macros use |@|, we have
% to locally set the |\catcode| of this symbol to ``letter''.
% This happens within a group, so that we do not have to worry about
% which |\catcode| that symbol had before.
%    \begin{macrocode}
%<*thp>
\begingroup \makeatletter
%    \end{macrocode}
% Since we are now within a group, we must make all definitions
% globally.  First we make sure that \texttt{theorem.sty} is loaded.
% This will allow us to use this file as a document style option
% without having to call \texttt{theorem} itself as an option.  At the
% same time, we assure that at least version 2 is loaded, since
% |\theorem@style| was not defined in earlier versions.
%    \begin{macrocode}
\@ifundefined{theorem@style}{\input{theorem.sty}}{}
\ProvidesFile{thp.sty}
             [\FMithmInfo]
%    \end{macrocode}
%
%
% \begin{macro}{\th@plain}
% \begin{macro}{\@begintheorem}
% \begin{macro}{\@opargbegintheorem}
%    |\theoremstyle{plain}| corresponds to the original
%    definition, except that the distances to the surrounding text are
%    determined by the parameters |\theorempreskipamount| and
%    |\theorempostskipamount|.
%    First we set the default body font.
%    \begin{macrocode}
\gdef\th@plain{\normalfont\itshape
%    \end{macrocode}
%    Then we define |\@begintheorem| and
%    |\@opargbegintheorem|. These two macros define how the
%    header of a theorem is typeset.  |\@opargbegintheorem| will
%    be called if a \textsf{theorem} environment with an optional
%    argument is encountered; otherwise, the header is constructed by
%    calling |\@begintheorem|.  If one of these macros is
%    executed, we are within a \textsf{trivlist} environment started by
%    |\@thm|.  So the theorem header is produced with an
%    |\item| command.
%
%    Instead of specifying the header font directly, all standard
%    theorem styles use the |\theorem@headerfont| macro to allow
%    customization.  The extra space (|\labelsep|) is necessary
%    because of problems in the \textsf{trivlist} environment.
%    \begin{macrocode}
  \def\@begintheorem##1##2{%
        \item[\hskip\labelsep \theorem@headerfont ##1\ ##2]}%
%    \end{macrocode}
%    The definition of |\@opargbegintheorem| is completely
%    analogous. The only difference is the fact that there exists a
%    third argument (which is the optional parameter of the
%    environment and contains additional information about the
%    theorem).  Customarily we enclose it in parentheses.
%    \begin{macrocode}
\def\@opargbegintheorem##1##2##3{%
   \item[\hskip\labelsep \theorem@headerfont ##1\ ##2\ (##3)]}}
%    \end{macrocode}
% \end{macro}
% \end{macro}
% \end{macro}
%
% \noindent
% We conclude with an |\endgroup| to restore the
% |\catcode| of |@|.
%    \begin{macrocode}
\endgroup
%</thp>
%    \end{macrocode}
%
%
% \subsubsection{The \texttt{break} style}
%
% This style option is stored in the file |thb.sty|.
% For the next two lines see the documentation for |\th@plain| on
% page~\pageref{dok:th@plain}.
%    \begin{macrocode}
%<*thb>
\begingroup \makeatletter
\@ifundefined{theorem@style}{\input{theorem.sty}}{}
\ProvidesFile{thb.sty}
             [\FMithmInfo]
%    \end{macrocode}
%
% \begin{macro}{\th@break}
%    |\theoremstyle{break}| produces a line break after the name
%    of the theorem.  The font is |\slshape|.  Hence, we define
%    |\th@break| as follows:
%    \begin{macrocode}
\gdef\th@break{\normalfont\slshape
  \def\@begintheorem##1##2{\item[%
%    \end{macrocode}
%    We run into the following problem: it is not possible to create
%    the header with |\item[|\meta{title}|]| and then start
%    a new line by, for example, |\mbox{}\\|. Such a definition
%    will fail whenever a list environment follows immediately.  With
%    the above construction, the |\mbox{}| causes the switch
%    |@inlabel| (cf.\ definition of |\list| and
%    |\trivlist| in \cite{latex:source}) to be set to \texttt{false}
%    and so the following list will insert additional vertical
%    space (|\topskip|). This is quite annoying. Therefore, we
%    create the line break within the |\item|. In order to ensure
%    that the text will begin at the proper position in the following
%    line, we simply pretend that the label does not take any
%    room.\footnote{This will lead to problems whenever very high
%    symbols occurring in the line tower into the
%    heading. So, something else has to be done here sometime.}
%    \begin{macrocode}
       \rlap{\vbox{\hbox{\hskip \labelsep\theorem@headerfont ##1\ ##2}%
                   \hbox{\strut}}}]}%
%    \end{macrocode}
%    Again, the definition of |\@opargbegintheorem| is
%    completely analogous.
%    \begin{macrocode}
\def\@opargbegintheorem##1##2##3{%
  \item[\rlap{\vbox{\hbox{\hskip \labelsep \theorem@headerfont
                     ##1\ ##2\ (##3)}%
                    \hbox{\strut}}}]}}
\endgroup
%</thb>
%    \end{macrocode}
% \end{macro}
%
%
% \subsubsection{The \texttt{changebreak} style}
%
%    \begin{macrocode}
% This style option is stored in the file |thcb.sty|.
%    \begin{macrocode}
%<*thcb>
\begingroup \makeatletter
\@ifundefined{theorem@style}{\input{theorem.sty}}{}
\ProvidesFile{thcb.sty}
             [\FMithmInfo]
%    \end{macrocode}
%
% \begin{macro}{\th@changebreak}
%    The \texttt{change-break} theorem style is like \texttt{break}
%    but with interchange of theorem name and theorem number.  Thus we
%    define |\th@changebreak| as follows:
%    \begin{macrocode}
\gdef\th@changebreak{\normalfont\slshape
  \def\@begintheorem##1##2{\item
       [\rlap{\vbox{\hbox{\hskip\labelsep \theorem@headerfont ##2\ ##1}%
                    \hbox{\strut}}}]}%
\def\@opargbegintheorem##1##2##3{%
  \item[\rlap{\vbox{\hbox{\hskip\labelsep \theorem@headerfont
                     ##2\ ##1\ (##3)}%
                    \hbox{\strut}}}]}}
\endgroup
%</thcb>
%    \end{macrocode}
% \end{macro}
%
%
% \subsubsection{The \texttt{change} style}
%
% This style option is stored in the file |thc.sty|.
%    \begin{macrocode}
%<*thc>
\begingroup \makeatletter
\@ifundefined{theorem@style}{\input{theorem.sty}}{}
\ProvidesFile{thc.sty}
             [\FMithmInfo]
%    \end{macrocode}
%
% \begin{macro}{\th@change}
%    The \texttt{change} theorem style corresponds to the
%    \texttt{change break} style without a linebreak after the header.
%    To say it in another way, it's the same as the \texttt{plain}
%    style but with number and name interchanged and |\slshape| as the
%    default font.
%    \begin{macrocode}
\gdef\th@change{\normalfont\slshape
  \def\@begintheorem##1##2{\item
       [\hskip\labelsep \theorem@headerfont ##2\ ##1]}%
\def\@opargbegintheorem##1##2##3{%
  \item[\hskip\labelsep \theorem@headerfont ##2\ ##1\ (##3)]}}
\endgroup
%</thc>
%    \end{macrocode}
% \end{macro}
%
%
% \subsubsection{The \texttt{marginbreak} style}
%
% This style option is the one used most often at Mainz.  It is saved
% in the file |thmb.sty|.
%    \begin{macrocode}
%<*thmb>
\begingroup \makeatletter
\@ifundefined{theorem@style}{\input{theorem.sty}}{}
\ProvidesFile{thmb.sty}
             [\FMithmInfo]
%    \end{macrocode}
%
% \begin{macro}{\th@marginbreak}
%    The \texttt{margin break} style is nearly the same as the
%    \texttt{change break} style. The only difference is the placement
%    of the theorem number.  We use |\llap| to place it in the left
%    margin.
%
%    In this style |\labelsep| denotes the separation between the
%    number and the text.
%    \begin{macrocode}
\gdef\th@marginbreak{\normalfont\slshape
  \def\@begintheorem##1##2{\item
       [\rlap{\vbox{\theorem@headerfont
                    \hbox{\llap{##2}\hskip\labelsep ##1}%
                    \hbox{\strut}}}]}%
\def\@opargbegintheorem##1##2##3{%
  \item[\rlap{\vbox{\theorem@headerfont
                    \hbox{\llap{##2}\hskip\labelsep ##1\ (##3)}%
                    \hbox{\strut}}}]}}
\endgroup
%</thmb>
%    \end{macrocode}
% \changes{v2.1c}{90/02/17}{`endgroup outside macrocode env.}
% \end{macro}
%
%
%
% \subsubsection{The \texttt{margin} style}
%
% This style option is stored in the file |thm.sty|.
%    \begin{macrocode}
%<*thm>
\begingroup \makeatletter
\@ifundefined{theorem@style}{\input{theorem.sty}}{}
\ProvidesFile{thm.sty}
             [\FMithmInfo]
%    \end{macrocode}
%
% \begin{macro}{\th@margin}
%    Again this is only a variant of the theorem styles described
%    above without any new ideas.
%    \begin{macrocode}
\gdef\th@margin{\normalfont\slshape
  \def\@begintheorem##1##2{\item
       [\theorem@headerfont \llap{##2}\hskip\labelsep ##1]}%
\def\@opargbegintheorem##1##2##3{%
  \item[\theorem@headerfont \llap{##2}\hskip\labelsep ##1\ (##3)]}}
\endgroup
%</thm>
%    \end{macrocode}
% \end{macro}
%
%
%
%
% \subsection{Final Definitions}
%
% \begin{macro}{\theorempreskipamount}
% \begin{macro}{\theorempostskipamount}
%    The \textsf{skip} parameters that regulate the vertical empty space
%    before and after the {\envfont theorem} environment have
%    to be allocated as well.
% \changes{v2.0f}{89/07/20}{Unnecessary `global' removed.}
%    \begin{macrocode}
%<*package>
\newskip\theorempreskipamount
\newskip\theorempostskipamount
%    \end{macrocode}
%    Since we have used the same values for all theorem sets, we now can
%    assign them.
%    \begin{macrocode}
\global\setlength\theorempreskipamount{12pt plus 5pt minus 3pt}
\global\setlength\theorempostskipamount{8pt plus 3pt minus 1.5pt}
%    \end{macrocode}
% \end{macro}
% \end{macro}
%
%
% \begin{macro}{\@endtheorem}
%    The same holds for the macro |\@endtheorem|, which ends a
%    {\envfont theorem} environment. Since it is the same for all
%    theorem sets, it is removed from the macros |\th@|\meta{style}.
%    It simply ends the {\envfont trivlist} environment, which was
%    begun in |\@thm|.
%    \begin{macrocode}
\global\let\@endtheorem=\endtrivlist
%    \end{macrocode}
% \end{macro}
%
%
%
% \begin{macro}{\@preamblecmds}
%    All macros defined above are to be used only in the preamble.
%    Therefore, we insert them in |\@preamblecmds| which will disable
%    them at begin document. This is done by the internal \LaTeXe{}
%    command |\@onlypreamble|.
% \changes{v2.0g}{89/09/19}{`theoremheaderfont and `theorembodyfont
%    added.}
% \changes{v2.2a}{94/02/03}{Use \LaTeXe{} interface}
%    \begin{macrocode}
\@onlypreamble\@xnthm
\@onlypreamble\@ynthm
\@onlypreamble\@othm
\@onlypreamble\newtheorem
\@onlypreamble\theoremstyle
\@onlypreamble\theorembodyfont
\@onlypreamble\theoremheaderfont
%    \end{macrocode}
% \end{macro}
%
% Finally we declare the \texttt{plain} theorem style to be the default.
% \changes{v2.1b}{90/06/04}{Fixed missing `item by declaring default
%    style.}
%    \begin{macrocode}
\theoremstyle{plain}
%</package>
%    \end{macrocode}
%
% \Finale

\endinput
}{}
\ProvidesFile{thc.sty}
             [\FMithmInfo]
%    \end{macrocode}
%
% \begin{macro}{\th@change}
%    The \texttt{change} theorem style corresponds to the
%    \texttt{change break} style without a linebreak after the header.
%    To say it in another way, it's the same as the \texttt{plain}
%    style but with number and name interchanged and |\slshape| as the
%    default font.
%    \begin{macrocode}
\gdef\th@change{\normalfont\slshape
  \def\@begintheorem##1##2{\item
       [\hskip\labelsep \theorem@headerfont ##2\ ##1]}%
\def\@opargbegintheorem##1##2##3{%
  \item[\hskip\labelsep \theorem@headerfont ##2\ ##1\ (##3)]}}
\endgroup
%</thc>
%    \end{macrocode}
% \end{macro}
%
%
% \subsubsection{The \texttt{marginbreak} style}
%
% This style option is the one used most often at Mainz.  It is saved
% in the file |thmb.sty|.
%    \begin{macrocode}
%<*thmb>
\begingroup \makeatletter
\@ifundefined{theorem@style}{% \iffalse meta-comment
%
% Copyright (C) 1993-2024
%
% The LaTeX Project and any individual authors listed elsewhere
% in this file.
%
% This file is part of the Standard LaTeX `Tools Bundle'.
% -------------------------------------------------------
%
% It may be distributed and/or modified under the
% conditions of the LaTeX Project Public License, either version 1.3c
% of this license or (at your option) any later version.
% The latest version of this license is in
%    https://www.latex-project.org/lppl.txt
% and version 1.3c or later is part of all distributions of LaTeX
% version 2005/12/01 or later.
%
% The list of all files belonging to the LaTeX `Tools Bundle' is
% given in the file `manifest.txt'.
%
% \fi
%
%%
%
%
\def\FMithmInfo{2023/07/05 v2.2c Theorem extension package (FMi)}
%
% \ProvidesFile{theorem.dtx}[\FMithmInfo]
%\iffalse   % this is a METACOMMENT !
%%
%% Package `theorem' to use with LaTeX2e
%% Copyright (C) 1989-1995 Frank Mittelbach, all rights reserved.
%<+package>\@ifundefined{theorem@style}{}{\endinput}
%<+package>\NeedsTeXFormat{LaTeX2e}[1995/06/01]
%<+package>\ProvidesPackage{theorem}
%<+package>         [\FMithmInfo]
%
% \fi
%
% \changes{v1.0}{87/05/07}{First draft.}
% \changes{v1.0a}{87/05/07}{`thistheoremstyle' added.}
% \changes{v1.1a}{87/07/29}{New concept: redefine the internal
%                          \protect\LaTeX{} commands `@xnthm' etc.}
% \changes{v1.1b}{87/07/31}{`theoremfont' used to specify font.}
% \changes{v1.2a}{88/01/08}{`@endtheorem' now globally defined.}
% \changes{v1.2b}{88/01/11}{`newtheorem' only allowed in the preamble.}
% \changes{v1.3a}{88/02/14}{`theoremfont' now a toks register.}
%
% \changes{v2.0a}{88/11/16}{New implementation and new documentation.}
% \changes{v2.0a}{88/11/16}{No layout changes but internal modifications
%                           old substyles won't work.}
% \changes{v2.0b}{89/01/30}{`global' used for all definitions.}
% \changes{v2.0c}{89/06/15}{Documentation translated to English.
%                            Thanks to LLL and Nina}
% \changes{v2.1a}{90/02/15}{Support for new font selection.}
% \changes{v2.2a}{94/02/03}{Cleanup for LaTeX2e}
%
% \DoNotIndex{\@endparenv,\@ifdefinable,\@ifnextchar,\@ifundefined}
% \DoNotIndex{\@namedef,\@nameuse,\@spaces,\@tempa,\@warning}
% \DoNotIndex{\ }
% \DoNotIndex{\begingroup,\bfseries}
% \DoNotIndex{\csname}
% \DoNotIndex{\def,\do,\docdate}
% \DoNotIndex{\endcsname,\endgroup,\endinput,\endtrivlist,\expandafter}
% \DoNotIndex{\filedate,\fileversion}
% \DoNotIndex{\gdef,\global}
% \DoNotIndex{\hbox,\hskip}
% \DoNotIndex{\ignorespaces,\input,\itshape,\item}
% \DoNotIndex{\labelsep,\let,\llap}
% \DoNotIndex{\makeatletter}
% \DoNotIndex{\noexpand}
% \DoNotIndex{\refstepcounter,\relax,\rlap}
% \DoNotIndex{\setlength,\slshape,\space,\string,\strut}
% \DoNotIndex{\the,\trivlist,\typeout}
% \DoNotIndex{\vbox}
% \DoNotIndex{\xdef}
%
% \def\deflabel#1{\ttfamily #1\hfill}
% \def\deflist#1{\begin{list}{}{\settowidth\labelwidth{\ttfamily #1}%
%                               \setlength\leftmargin\labelwidth
%                               \addtolength\leftmargin\labelsep
%                               \let\makelabel\deflabel}}
% \def\enddeflist{\end{list}}
% \def\envfont{\normalfont\ttfamily}
%
% \MakeShortVerb{\|}
%
% \newcommand{\TUB}{{\sl TUGboat\/}}
%
% \GetFileInfo{theorem.dtx}
% \title{An Extension of the
%        \LaTeX{} theorem environment\thanks{This file
%        has version number \fileversion, last
%        revised \filedate.}}
% \author{Frank Mittelbach}
%
% \MaintainedByLaTeXTeam{tools}
% \maketitle
%
% \begin{abstract}
%    The macros described in this paper yield an extension of the
%    \LaTeX{} theorem mechanism. It is designed to satisfy the
%    different requirements of various journals. Thus, the
%    layout of the ``theorems'' can be manipulated by determining a
%    ``style''. This article describes not only the use, but
%    also the definition, of the necessary macros.
% \end{abstract}
%
% \section*{Preface to version 2.2}
%
% For \LaTeXe{} this package did not need any fundamental changes. I
% only modified the messages generated so that theorem layout styles
% will show up with the |\listfiles| command and cleaned the section
% on the New Font Selection Scheme since this is now included in
% \LaTeX.
%
% \section*{Preface to version 2.1}
%
% This version is identical to 2.0g described in \TUB\ 10\#3 except
% for some internal defaults which are now set depending on the used
% font selection scheme.
%
% This was done to avoid unpleasant surprises if the new font selection
% scheme is in force. For further details see
% section~\ref{sec:newfonts} and \cite{bk:GMS94}.
%
% \section{Introduction}
%
% For our purposes here, ``theorems'' are labelled enunciations,
% often set off from the main text by extra space and a font change.
% Theorems, corollaries, conjectures, definitions, and remarks
% are all instances of ``theorems''.  The ``header'' of these structures
% is composed of a label (such as \textsc{Theorem} or \textsc{Remark})
% and a number which serializes an item in the sequence of items
% with the same label.
%
% Shortly after the introduction of \LaTeX{} at the Fachbereich
% Mathematik in Mainz, the desire to manipulate the layout of
% ``theorems'' arose.  In Mainz, the following two conventions came
% into general use:
% \begin{enumerate}
%   \item The number of the theorem is shown in the margin.
%   \item There is a line break at
%         the end of the theorem header.
% \end{enumerate}
%
% Additionally, some journals require different formats which depend on
% the ``sort of theorem'': e.g.\ often remarks and definitions are set
% in |\upshape|, while |\itshape| is employed for main theorems.
%
% Confronted with these requirements, a theorem environment was
% developed in Mainz which allows separate determination of the
% layout of the ``theorems sets'', comparable to |\pagestyle|.
%
%
% \section{The user interface}
% \subsection{Defining new theorem sets}
%
% \DescribeMacro\newtheorem
% As in the original \LaTeX{} version, the command |\newtheorem|
% defines a new ``theorem set'' or ``theorem-like structure''.
% Two required arguments name the new environment and give the text
% to be typeset with each instance of the new ``set'', while
% an optional argument determines how
% the ``set'' is enumerated:
% \begin{description}
%    \item[\ttfamily \bslash newtheorem\{foo\}\{bar\}]
%       The theorem set {\envfont foo} (whose name is \texttt{bar})
%       uses its own counter.
%    \item[\ttfamily \bslash newtheorem\{foo2\}{[foo]}\{bar2\}]
%       The theorem set {\envfont foo2} (printed name \texttt{bar2})
%       uses the same counter as the theorem set \texttt{foo}.
%    \item[\ttfamily \bslash newtheorem\{foo3\}\{bar3\}{[section]}]
%       The theorem set {\envfont foo3} (printed name \texttt{bar3}) is
%       enumerated within the counter \texttt{section}, i.e.\ with every
%       new |\section| the enumeration begins again with 1, and
%       the enumeration is composed from the section-number and the
%       theorem counter itself.
% \end{description}
%
% \DescribeMacro\theoremstyle
% Additionally, the command |\theoremstyle| can define the layout
% of various, or all, theorem sets.  It should be noted that any
% theorem set defined by |\newtheorem| is typeset in the
% |\theoremstyle| that is current at the time of the definition.
% Thus, the following
% \begin{verbatim}
%    \theoremstyle{break}        \newtheorem{Cor}{Corollary}
%    \theoremstyle{plain}        \newtheorem{Exa}{Example}[section]
%\end{verbatim}                             ^^A No blank at line start
% leads to the result that the set {\envfont Cor} is formatted in the
% style \texttt{break}, while the set {\envfont Exa} and all the
% following ones are formatted in the style \texttt{plain}, unless
% another |\theoremstyle| follows.  Since the definitions installed by
% |\newtheorem| are global, one also can limit |\theoremstyle| locally
% by grouping braces.
%
% \DescribeMacro\theorembodyfont
% The choice of the font for the theorem body is completely
% independent of the chosen |\theoremstyle|; this has proven to
% be very advantageous.  For example,
% \begin{verbatim}
%   {\theorembodyfont{\upshape}          \newtheorem{Rem}{Remark}}
%\end{verbatim}
% defines a theorem set {\envfont Rem}, which will be set in
% |\upshape| in the current layout (which in our example is
% \texttt{plain}).  As with |\theoremstyle|, the |\theorembodyfont|
% chosen is that current at the time of |\newtheorem|.  If
% |\theorembodyfont| is not specified or one defines
% |\theorembodyfont{}|, then the font used will be that defined by the
% |\theoremstyle|.
%
% \DescribeMacro\theoremheaderfont It is also possible to customize
% the font used for the theorem headers.  This is, however, a global
% declaration, and therefore there should be at most one
% |\theoremheaderfont| declaration in the preamble.\footnote{If it is
% actually necessary to have different header fonts, one has to define
% new theorem styles (substituting the desired font) or specify the
% information directly in the \texttt{\bslash newtheorem} declaration
% (the unclean variant).}
%
% \DescribeMacro\theorempreskipamount
% \DescribeMacro\theorempostskipamount Two additional parameters
% affect the vertical space around the theorem environments:
% |\theorempreskipamount| and |\theorempostskipamount| define,
% respectively, the spacing before and after such an environment.
% These parameters apply for all theorem sets and can be manipulated
% with the ordinary length macros.  They are rubber lengths
% (`\textsf{skips}'), and therefore can contain \texttt{plus} and
% \texttt{minus} parts.
%
% Since the definition of theorem sets should---most sensibly---be
% placed in the preamble, we only allow installation there.
% It is therefore possible to release the memory used here after
% |\begin{document}|, in order to make room for other
% applications.
%
%
% \subsection{Existing theorem styles}
%
% The following theorem styles exist to date:
% \begin{deflist}{marginbreak}
%    \item[plain]
%       This theorem style emulates the original \LaTeX{} definition,
%       except that additionally the parameters
%       |\theorem...skipamount| are used.
%    \item[break]
%       In this style, the theorem header is followed by a line
%       break.
%    \item[marginbreak]
%       The theorem number is set in the margin, and there is a line
%       break as in \texttt{break}.
%    \item[changebreak]
%       Like \texttt{break}, but with header number and text
%       interchanged.
%    \item[change]
%       Header number and text are interchanged, without a line break.
%    \item[margin]
%       The number is set in the left margin, without a line break.
% \end{deflist}
% All styles (except \texttt{plain}) select |\slshape| as the default
% |\theorembodyfont|.
%
%
% \subsection{Examples}
%
% Given the above theorem sets {\envfont Cor}, {\envfont Exa}
% and {\envfont Rem},
% suppose that the preamble also contains the declarations:
% \begin{verbatim}
%    \theoremstyle{marginbreak}   \newtheorem{Lem}[Cor]{Lemma}
%    \theoremstyle{change}
%    \theorembodyfont{\itshape}        \newtheorem{Def}[Cor]{Definition}
%
%    \theoremheaderfont{\scshape}
%\end{verbatim}
% Then the following are some typical examples of the typeset output
% resulting from their use.
%
% \begin{Cor}
%    This is a sentence typeset in the theorem environment {\envfont
%    Cor}.
% \end{Cor}
%
% \begin{Exa}
%    This is a sentence typeset in the theorem environment {\envfont
%    Exa}.
% \end{Exa}
%
% \begin{Rem}
%    This is a sentence typeset in the theorem environment {\envfont
%    Rem}.
% \end{Rem}
%
% \begin{Lem}[Ben User]
%    This is a sentence typeset in the theorem environment {\envfont
%    Lem}.
% \end{Lem}
%
% \begin{Def}[Very impressive Definition]
%    This is a sentence typeset in the theorem environment {\envfont
%    Def}.
% \end{Def}
%
% The last two examples show the effect of the optional argument to
% a theorem environment (it is the text typeset in parentheses).
%
%
% \section{Special Considerations}
% \label{sec:newfonts}
%
% Theoremheader and body are implemented as a unit. This means that
% the |\theoremheaderfont|  will inherit characteristics of the
% |\theorembodyfont| in \LaTeXe. Thus, if for example
% |\theorembodyfont| is |\itshape| and |\theoremheaderfont|
% is |\bfseries| the font selected for the header will have the
% characteristics `bold extended italic'. If this is not desired one
% should set the |\theoremheaderfont| to something like
% \begin{verbatim}
%      \theoremheaderfont{\normalfont\bfseries}
%\end{verbatim}
% i.e.\ supplying all necessary font information explicitly.
%
%
%
% \section{Acknowledgements}
%
% The publication of this set of macros was only possible with the
% help of Christina Busse (translating the manuscript into English),
% Joachim Pense (playing the r\^ole of typist), Chris Rowley
% (looking everything over) and many others providing useful
% suggestions.
%
% \MaybeStop{%
%    \begin{thebibliography}{9}
%       \bibitem{bk:GMS94} \textsc{M.~Goossens}, \textsc{F.~Mittelbach}
%         and \textsc{A.~Samarin}.
%         \newblock The \LaTeX{} Companion.
%         \newblock
%         Addison-Wesley, Reading, Massachusetts, 1994.
%       \bibitem{latex:source} \textsc{Lamport, Leslie.} \newblock
%                               \texttt{latex.tex}, version 2.09, date
%                               Feb. 1990.
%    \end{thebibliography}
%    \PrintIndex
%    \PrintChanges}
%
%
% \section{The documentation driver file}
%
% The next bit of code contains the documentation driver file for
% \TeX{}, i.e., the file that will produce the documentation you are
% currently reading. It will be extracted from this file by the
% \texttt{docstrip} program. Since it is the first code in the file
% one can alternatively process this file directly with \LaTeXe{} to
% obtain the documentation.
%
%    \begin{macrocode}
%<*driver>
\documentclass{ltxdoc}

\usepackage{theorem}

%<+driver>% The next few lines define theorem sets which are used
%<+driver>% in the example section of the documentation.

\theoremstyle{break}       \newtheorem{Cor}{Corollary}
\theoremstyle{plain}       \newtheorem{Exa}{Example}[section]
{\theorembodyfont{\upshape}\newtheorem{Rem}{Remark}}
\theoremstyle{marginbreak} \newtheorem{Lem}[Cor]{Lemma}
\theoremstyle{change}
\theorembodyfont{\itshape} \newtheorem{Def}[Cor]{Definition}

\theoremheaderfont{\scshape}

\RecordChanges

\begin{document}
   \DocInput{theorem.dtx}
\end{document}
%</driver>
%    \end{macrocode}
%
% \section{Definition of the Macros}
%
% If the file has been loaded before, we abort immediately. If not the
% package announces itself (this is actually done at the very top of
% the file---the way it is done isn't good style so don't copy it).
% \changes{v2.0e}{89/07/19}{Spaces removed from `typeout'.}
% \changes{v2.2b}{95/11/19}{Announce moved to top of file}
%    \begin{macrocode}
%<*package>
%\@ifundefined{theorem@style}{}{\endinput}
%\def\FMithmInfo{1995/11/19 v2.2b Theorem extension package (FMi)}
%\ProvidesPackage{theorem}[\FMithmInfo]
%    \end{macrocode}
%
%
% \subsection{Definition of theorem styles and fonts}
%
% All the definitions in this file are done globally to allow
% inputting this file inside a group.
%
% \begin{macro}{\theoremstyle}
%    Before a theorem style can be installed, the chosen style must be
%    known. For that reason, we must test to see that
%    |\th@|\meta{style} is known or, more precisely, that it is
%    different from |\relax|. If the style is not known then
%    |\th@plain| is used.
%    \begin{macrocode}
\gdef\theoremstyle#1{%
   \@ifundefined{th@#1}{\@warning
          {Unknown theoremstyle `#1'. Using `plain'}%
          \theorem@style{plain}}%
%    \end{macrocode}
%    We save the theorem style to be used in the token register
%    |\theorem@style|.
%    \begin{macrocode}
      {\theorem@style{#1}}%
%    \end{macrocode}
%    Now we ``evaluate'' the theorem style: this means, we call the
%    macro |\th@|\meta{style} which will activate the relevant
%    definitions which are contained in a separate file.  This is done
%    in a group to suppress changes to the current font.  This could
%    otherwise pose problems together with the new font selection
%    scheme\footnote{When I printed the original article using the new
%    font selection scheme I ended with a document with slanted
%    typefaces (text headings and all) simply because one of the
%    theorem styles used \texttt{\bslash sl} at toplevel.} if the
%    |\th@|\meta{style} is evaluated a second time.
% \changes{v2.1a}{90/02/15}{Group added around `th@..}
% \changes{v2.1a}{90/02/15}{Reseting of `..@begintheorem removed.}
%    \begin{macrocode}
      \begingroup
        \csname th@\the\theorem@style \endcsname
      \endgroup}
%    \end{macrocode}
% \end{macro}
%
%
% \begin{macro}{\@begintheorem}
% \changes{v2.1a}{90/02/15}{Relaxed at toplevel.}
% \begin{macro}{\@opargbegintheorem}
% \changes{v2.1a}{90/02/15}{Relaxed at toplevel.}
%    We reset |\@begintheorem| and |\@opargbegintheorem| to
%    |\relax| since these commands are no longer necessary at
%    toplevel. This will save a few tokens.
%    \begin{macrocode}
\global\let\@begintheorem\relax
\global\let\@opargbegintheorem\relax
%    \end{macrocode}
% \end{macro}
% \end{macro}
%
% \begin{macro}{\theorem@style}
%    Obviously the \textsf{token} register used above has to be
%    allocated. To assure the utmost compatibility with the
%    original \LaTeX{} definition, we set the default theorem style to
%    \texttt{plain}, which implements the usual \LaTeX{} convention.
% \changes{v2.0f}{89/07/20}{Unnecessary `global' removed.}
%    \begin{macrocode}
\newtoks\theorem@style
\global\theorem@style{plain}
%    \end{macrocode}
% \end{macro}
%
% \begin{macro}{\theorembodyfont}
% \begin{macro}{\theorem@bodyfont}
%    If the |\theorembodyfont| is set by the user then it should not
%    interact with the default font set in the theorem style. When the
%    new font selection is in force this may happen if, for example,
%    the default is |\itshape| and the new |\theorembodyfont| is
%    |\sffamily|. So we add a |\reset@font| command in front of the
%    user definition.
% \changes{v2.0f}{89/07/20}{Unnecessary `global' removed.}
% \changes{v2.0g}{89/09/19}{`theoremfont to `theorembodyfont.}
% \changes{v2.1c}{92/06/22}{`reset@font added and logic changed.}
% \changes{v2.1d}{92/10/28}{Reset to empty if arg is empty}
%    \begin{macrocode}
\gdef\theorembodyfont#1{%
%    \end{macrocode}
%    We check if the argument supplied is empty and if so put nothing
%    into the |\theorem@bodyfont| token register to allow for
%    |\theorembodyfont{}| as a mean of using the default of the
%    current |\theoremstyle|.
%    \begin{macrocode}
   \def\@tempa{#1}%
   \ifx\@tempa\@empty
    \theorem@bodyfont{}%
   \else
    \theorem@bodyfont{\reset@font#1}%
   \fi
   }
\newtoks\theorem@bodyfont
\global\theorem@bodyfont{}
%    \end{macrocode}
% \end{macro}
% \end{macro}
%
%
% \begin{macro}{\theoremheaderfont}
% \changes{v2.0g}{89/09/19}{Macro added.}
%    The font for the theorem headers is handled differently because
%    this definition applies to all theorem styles.
%    \begin{macrocode}
\gdef\theoremheaderfont#1{\gdef\theorem@headerfont{#1}%
%    \end{macrocode}
%    After using the macro once it is redefined to produce an error
%    message.
%    \begin{macrocode}
       \gdef\theoremheaderfont##1{%
        \typeout{\string\theoremheaderfont\space should be used
                 only once.}}}
%    \end{macrocode}
% \end{macro}
%
% \begin{macro}{\theorem@headerfont}
% \changes{v2.0g}{89/09/19}{Macro added.}
%    To set the |\theorem@headerfont| default we first test if
%    the new fontselection scheme is in force.
% \changes{v2.1a}{90/02/15}{Support for new font selection.}
%    \begin{macrocode}
\ifx\upshape\undefined
%    \end{macrocode}
%    If not we define it to expand into |\bfseries|. We don't use
%    |\let| just in case a following style option redefines this
%    macro.
%    \begin{macrocode}
\gdef\theorem@headerfont{\bfseries}
%    \end{macrocode}
%    Otherwise we reset the current shape before calling |\bfseries|.
%    \begin{macrocode}
\else \gdef\theorem@headerfont{\normalfont\bfseries}\fi
%    \end{macrocode}
% \end{macro}
%
% \begin{macro}{\th@plain}
% \begin{macro}{\th@break}
% \begin{macro}{\th@marginbreak}
% \begin{macro}{\th@changebreak}
% \begin{macro}{\th@change}
% \begin{macro}{\th@margin}
%    The different styles are defined in macros such as
%    |\th@plain|.  Since memory space is precious in
%    ``non-Big-versions'', we have to avoid offering too many unused
%    definitions.  Therefore we define these styles in separate files
%    that can be loaded on demand.  Thus the commands themselves only
%    load these files. We use |\@input@|, a \LaTeXe{} internal command
%    that ensures that the file will be listed with |\listfiles|.
% \changes{v2.2a}{94/02/02}{Use `@input@ to load theorem layout files}
%    \begin{macrocode}
\gdef\th@plain{\@input@{thp.sty}}
\gdef\th@break{\@input@{thb.sty}}
\gdef\th@marginbreak{\@input@{thmb.sty}}
\gdef\th@changebreak{\@input@{thcb.sty}}
\gdef\th@change{\@input@{thc.sty}}
\gdef\th@margin{\@input@{thm.sty}}
%    \end{macrocode}
%    This list will be expanded when new styles become
%    available.  For testing,  just append new theorem substyles
%    as document options.
% \end{macro}
% \end{macro}
% \end{macro}
% \end{macro}
% \end{macro}
% \end{macro}
%
%
%
% \subsection{Definition of a new theorem set}
%
% As already pointed out, a new theorem environment can be defined in
% three different ways:
% \begin{verbatim}
%    \newtheorem{Lem}{Lemma}
%    \newtheorem{Lem}{Lemma}[section]
%    \newtheorem{Lem}[Theorem]{Lemma}
%\end{verbatim}
% The function of the macro |\newtheorem| is to recognize these cases
% and then to branch into one of the three macros |\@ynthm|, |\@xnthm|
% or |\@othm|.  This mechanism is adopted unchanged from
% \cite{latex:source}; the essential point here is that, for example,
% in the second case, the arguments \texttt{Lem}, \texttt{Lemma} and
% \texttt{section} are passed over to the macro |\@xnthm|.
%
% We inspect this case first because the others present fewer problems,
% and thus are easily derived from this one.
%
% \begin{macro}{\@xnthm}
%    For our example arguments, the macro |\@xnthm| must fulfill
%    the following:
%    \begin{itemize}
%       \item
%          Define a new \LaTeX-counter `Lem'
%       \item
%          reset this counter within a |\section|
%       \item
%          define the macro |\theLem|
%       \item
%          define the environment macros |\Lem| and
%          |\endLem| using the current
%          |\theoremstyle| and |\theorem@bodyfont|.
%    \end{itemize}
%    Obviously, all this should happen only if the first argument
%    of |\@xnthm| (i.e.\ \texttt{Lem} in our example) is chosen so
%    as not to conflict with any previously defined commands or
%    environments.  This test is performed by the \LaTeX{} macro
%    |\@ifdefinable|.
%    \begin{macrocode}
\gdef\@xnthm#1#2[#3]{\expandafter\@ifdefinable\csname #1\endcsname
%    \end{macrocode}
%    Therefore, the first argument of |\@ifdefinable| is the
%    expansion (in the example, |\Lem|) of
%    |\csname||#1\endcsname|.  The second argument is
%    executed only if the test has been completed successfully.
%    \begin{macrocode}
   {%
%    \end{macrocode}
%    Now we define the new counter. The names of the \LaTeX{}
%    macros employed should speak for themselves:
% \changes{v2.2b}{1995/10/07}{Use \cs{@newctr}}
% \changes{v2.2c}{1995/11/23}{Use \cs{@newctr} properly.}
%    \begin{macrocode}
    \@definecounter{#1}\@newctr{#1}[#3]%
%    \end{macrocode}
%    Using |\@newctr| will give a proper error message if the counter
%    in |#3| is not defined.
%    In defining `|\theLem|' we must generate the desired macro
%    name by use of |\expandafter| and |\csname|.
%    \begin{macrocode}
    \expandafter\xdef\csname the#1\endcsname
%    \end{macrocode}
%    An |\xdef| is used in order to make the definition global,
%    and to ensure that it contains the replacement texts of
%    |\@thmcountersep| and |\@thmcounter|.\footnote{These
%    two macros can be defined by the document style.  Their default
%    values produce a `\texttt{.}' as separation and an arabic
%    representation of the number.} However, not everything should be
%    expanded. For example, it saves space to use |\thesection|
%    instead of its---at times---lengthy expansion.
%    \begin{macrocode}
      {\expandafter \noexpand \csname the#3\endcsname
       \@thmcountersep \@thmcounter{#1}}%
%    \end{macrocode}
%    Thus with the defaults of \LaTeX{}, |\theLem| would be
%    replaced by the command sequence |\thesection.\arabic{Lem}|.
%
%    We will now look at the definition of the macro which is executed
%    at the beginning of the actual environment (in our example this
%    macro is |\Lem|). It should be noted that we use an
%    ``|\expandafter| trick'' to expand only certain parts of the
%    replacement text at the time of the definition.
%    \begin{macrocode}
    \def\@tempa{\global\@namedef{#1}}%
    \expandafter \@tempa \expandafter{%
%    \end{macrocode}
%    First, the macro that contains the current definitions of
%    |\@begintheorem| and |\@opargtheorem| should be called
%    up.  The name of this macro---as is already known---has the form
%    |\th@|\meta{theorem\ style}; therefore, it must be called by
%    \begin{macrocode}
      \csname th@\the \theorem@style
%    \end{macrocode}
%    In addition the default theorem font should be changeable, i.e.\
%    we have to insert the contents of |\theorem@bodyfont|. For
%    that reason, we expand even further, beyond |\endcsname|,
%    and thus insert the contents of the \textsf{token} register
%    |\theorem@bodyfont| in the replacement text.
%    \begin{macrocode}
            \expandafter \endcsname \the \theorem@bodyfont
%    \end{macrocode}
%    Now it is time to call the macro |\@thm| which takes over
%    the further processing.  It has two arguments: the current
%    counter name (in our example, \texttt{Lem}), and the text of the
%    label (in our example, \texttt{Lemma}).
%    \begin{macrocode}
     \@thm{#1}{#2}}%
%    \end{macrocode}
%    With this, the `sub-definition' is complete.  The macro
%    |\@endtheorem| ends a theorem environment and is, so far, nothing
%    but an |\endtrivlist|.  (Hence it is defined globally, and not
%    within the theorem styles.\footnote{This has to be changed as
%    soon as theorem styles that change \texttt{\bslash @endtheorem}
%    exist.  In such a case, all existing styles must be changed as
%    well since they will have to reset the macro.}) Therefore, we can
%    set it equivalent to the macro that ends the theorem set (in our
%    example, |\endLem|).  However, if some day theorem styles exist
%    that do change |\@endtheorem|, we would have to use the
%    commented-out line instead.
%    \begin{macrocode}
    \global \expandafter \let \csname end#1\endcsname \@endtheorem
%   \global\@namedef{end#1}{\@endtheorem}%
%    \end{macrocode}
%    With these commands all the required definitions are employed,
%    unless the test |\@ifdefinable| has failed.  Therefore, we
%    end the second argument of this macro and with it the definition
%    of |\@xnthm|.
%    \begin{macrocode}
   }}
%    \end{macrocode}
% \end{macro}
%
% \begin{macro}{\@ynthm}
%    The definition of |\@ynthm| is completely analogous. In
%    this case the new counter that is defined is not reset within
%    another counter; thus the definition of |\the...|
%    is simplified:
%    \begin{macrocode}
\gdef\@ynthm#1#2{\expandafter\@ifdefinable\csname #1\endcsname
   {\@definecounter{#1}%
    \expandafter\xdef\csname the#1\endcsname{\@thmcounter{#1}}%
%    \end{macrocode}
%    The rest of the definition corresponds literally to that of
%    |\@xnthm|:
%    \begin{macrocode}
    \def\@tempa{\global\@namedef{#1}}\expandafter \@tempa
     \expandafter{\csname th@\the \theorem@style \expandafter
     \endcsname \the\theorem@bodyfont \@thm{#1}{#2}}%
    \global \expandafter \let \csname end#1\endcsname \@endtheorem}}
%    \end{macrocode}
% \end{macro}
%
% \begin{macro}{\@othm}
%    The definition of |\@othm| does not contain anything new.
%    \begin{macrocode}
\gdef\@othm#1[#2]#3{%
%    \end{macrocode}
% \changes{v2.2b}{95/10/07}{Build in check for counter defined}
%    We do not define a new counter but instead use one that has
%    already been defined.  Thus the only definition we need is that
%    of this pseudo-counter (i.e.\ |\the|\meta{env.\ name}).
%    First we check if |#2| corresponds to a known counter name.
%    \begin{macrocode}
  \expandafter\ifx\csname c@#2\endcsname\relax
   \@nocounterr{#2}%
  \else
   \expandafter\@ifdefinable\csname #1\endcsname
   {\expandafter \xdef \csname the#1\endcsname
     {\expandafter \noexpand \csname the#2\endcsname}%
%    \end{macrocode}
%    All other parts of the definition can be adopted from
%    |\@xnthm|. We have to remember, though, that in this case
%    the name of the current counter and the theorem label have moved
%    to the second and third arguments.
%    \begin{macrocode}
    \def\@tempa{\global\@namedef{#1}}\expandafter \@tempa
     \expandafter{\csname th@\the \theorem@style \expandafter
     \endcsname \the\theorem@bodyfont \@thm{#2}{#3}}%
    \global \expandafter \let \csname end#1\endcsname \@endtheorem}%
  \fi}
%    \end{macrocode}
% \end{macro}
%
% \subsection{Macros that are employed in a {\envfont theorem}
%    environment}
%
% \begin{macro}{\@thm}
%    The macro |\@thm| has to increase the current counter.  Then,
%    depending on whether the environment has (or does not have) an
%    optional argument, it has to branch into either |\@begintheorem|
%    or |\@opargtheorem|.
%    \begin{macrocode}
\gdef\@thm#1#2{\refstepcounter{#1}%
%    \end{macrocode}
%    Now we start a {\envfont trivlist} environment, and give
%    |\@topsep| and |\@topsepadd| the values of the \textsf{skip}
%    registers |\theorempreskipamount| and |\theorempostskipamount|.
%    The value in |\@topsep| is the vertical space that is inserted by
%    the first (and only) |\item| in our |\trivlist| whilst
%    |\@topsepadd| is inserted by |\@endparenv| at the end of that
%    {\envfont trivlist} environment.  By using these registers, we
%    obtain the desired space around a {\envfont theorem} environment.
%    \begin{macrocode}
   \trivlist
   \@topsep \theorempreskipamount               % used by first \item
   \@topsepadd \theorempostskipamount           % used by \@endparenv
%    \end{macrocode}
%    Now we have to test whether an optional argument has been given.
% \changes{v2.0d}{89/07/18}{Forgotten percent added.}
%    \begin{macrocode}
   \@ifnextchar [%
%    \end{macrocode}
%    If there is an optional argument, we will call |\@ythm|, and
%    move the arguments read back into the input stream.
%    \begin{macrocode}
   {\@ythm{#1}{#2}}%
%    \end{macrocode}
%    If not, we call |\@begintheorem|. Its first argument is the
%    name of the theorem set (hence the second argument of
%    |\@thm|). Its second argument is the macro that produces the
%    current number.
%    \begin{macrocode}
   {\@begintheorem{#2}{\csname the#1\endcsname}\ignorespaces}}
%    \end{macrocode}
% \end{macro}
%
% \begin{macro}{\@xthm}
% \begin{macro}{\@ythm}
%    Both these macros were originally called by |\@thm|.  We do
%    not need |\@xthm| anymore, hence we reset it to
%    |\relax|. The definition of |\@ythm| has not changed at
%    all from its definition in \LaTeX. In order to make the macros
%    easier to understand, we will
%    nevertheless present it (commented out).
%    \begin{macrocode}
\global\let\@xthm\relax
% \def\@ythm#1#2[#3]{\@opargbegintheorem{#2}{\csname
%           the#1\endcsname}{#3}\ignorespaces}
%</package>
%    \end{macrocode}
%    The primitive |\ignorespaces| in |\@ythm| and
%    |\@thm| is needed to remove the spaces between the
%    |\begin{...}| and the actual text.
% \end{macro}
% \end{macro}
%
%
% \subsection{Definition of the theorem substyles}
%
% As already pointed out, the theorem substyles, defined below, are
% only loaded when necessary.  Note that all these substyles, except
% \texttt{plain}, have |\slshape| as the default body font.
%
% \subsubsection{The \texttt{plain} style}
%
% \label{dok:th@plain} As the following macros use |@|, we have
% to locally set the |\catcode| of this symbol to ``letter''.
% This happens within a group, so that we do not have to worry about
% which |\catcode| that symbol had before.
%    \begin{macrocode}
%<*thp>
\begingroup \makeatletter
%    \end{macrocode}
% Since we are now within a group, we must make all definitions
% globally.  First we make sure that \texttt{theorem.sty} is loaded.
% This will allow us to use this file as a document style option
% without having to call \texttt{theorem} itself as an option.  At the
% same time, we assure that at least version 2 is loaded, since
% |\theorem@style| was not defined in earlier versions.
%    \begin{macrocode}
\@ifundefined{theorem@style}{\input{theorem.sty}}{}
\ProvidesFile{thp.sty}
             [\FMithmInfo]
%    \end{macrocode}
%
%
% \begin{macro}{\th@plain}
% \begin{macro}{\@begintheorem}
% \begin{macro}{\@opargbegintheorem}
%    |\theoremstyle{plain}| corresponds to the original
%    definition, except that the distances to the surrounding text are
%    determined by the parameters |\theorempreskipamount| and
%    |\theorempostskipamount|.
%    First we set the default body font.
%    \begin{macrocode}
\gdef\th@plain{\normalfont\itshape
%    \end{macrocode}
%    Then we define |\@begintheorem| and
%    |\@opargbegintheorem|. These two macros define how the
%    header of a theorem is typeset.  |\@opargbegintheorem| will
%    be called if a \textsf{theorem} environment with an optional
%    argument is encountered; otherwise, the header is constructed by
%    calling |\@begintheorem|.  If one of these macros is
%    executed, we are within a \textsf{trivlist} environment started by
%    |\@thm|.  So the theorem header is produced with an
%    |\item| command.
%
%    Instead of specifying the header font directly, all standard
%    theorem styles use the |\theorem@headerfont| macro to allow
%    customization.  The extra space (|\labelsep|) is necessary
%    because of problems in the \textsf{trivlist} environment.
%    \begin{macrocode}
  \def\@begintheorem##1##2{%
        \item[\hskip\labelsep \theorem@headerfont ##1\ ##2]}%
%    \end{macrocode}
%    The definition of |\@opargbegintheorem| is completely
%    analogous. The only difference is the fact that there exists a
%    third argument (which is the optional parameter of the
%    environment and contains additional information about the
%    theorem).  Customarily we enclose it in parentheses.
%    \begin{macrocode}
\def\@opargbegintheorem##1##2##3{%
   \item[\hskip\labelsep \theorem@headerfont ##1\ ##2\ (##3)]}}
%    \end{macrocode}
% \end{macro}
% \end{macro}
% \end{macro}
%
% \noindent
% We conclude with an |\endgroup| to restore the
% |\catcode| of |@|.
%    \begin{macrocode}
\endgroup
%</thp>
%    \end{macrocode}
%
%
% \subsubsection{The \texttt{break} style}
%
% This style option is stored in the file |thb.sty|.
% For the next two lines see the documentation for |\th@plain| on
% page~\pageref{dok:th@plain}.
%    \begin{macrocode}
%<*thb>
\begingroup \makeatletter
\@ifundefined{theorem@style}{\input{theorem.sty}}{}
\ProvidesFile{thb.sty}
             [\FMithmInfo]
%    \end{macrocode}
%
% \begin{macro}{\th@break}
%    |\theoremstyle{break}| produces a line break after the name
%    of the theorem.  The font is |\slshape|.  Hence, we define
%    |\th@break| as follows:
%    \begin{macrocode}
\gdef\th@break{\normalfont\slshape
  \def\@begintheorem##1##2{\item[%
%    \end{macrocode}
%    We run into the following problem: it is not possible to create
%    the header with |\item[|\meta{title}|]| and then start
%    a new line by, for example, |\mbox{}\\|. Such a definition
%    will fail whenever a list environment follows immediately.  With
%    the above construction, the |\mbox{}| causes the switch
%    |@inlabel| (cf.\ definition of |\list| and
%    |\trivlist| in \cite{latex:source}) to be set to \texttt{false}
%    and so the following list will insert additional vertical
%    space (|\topskip|). This is quite annoying. Therefore, we
%    create the line break within the |\item|. In order to ensure
%    that the text will begin at the proper position in the following
%    line, we simply pretend that the label does not take any
%    room.\footnote{This will lead to problems whenever very high
%    symbols occurring in the line tower into the
%    heading. So, something else has to be done here sometime.}
%    \begin{macrocode}
       \rlap{\vbox{\hbox{\hskip \labelsep\theorem@headerfont ##1\ ##2}%
                   \hbox{\strut}}}]}%
%    \end{macrocode}
%    Again, the definition of |\@opargbegintheorem| is
%    completely analogous.
%    \begin{macrocode}
\def\@opargbegintheorem##1##2##3{%
  \item[\rlap{\vbox{\hbox{\hskip \labelsep \theorem@headerfont
                     ##1\ ##2\ (##3)}%
                    \hbox{\strut}}}]}}
\endgroup
%</thb>
%    \end{macrocode}
% \end{macro}
%
%
% \subsubsection{The \texttt{changebreak} style}
%
%    \begin{macrocode}
% This style option is stored in the file |thcb.sty|.
%    \begin{macrocode}
%<*thcb>
\begingroup \makeatletter
\@ifundefined{theorem@style}{\input{theorem.sty}}{}
\ProvidesFile{thcb.sty}
             [\FMithmInfo]
%    \end{macrocode}
%
% \begin{macro}{\th@changebreak}
%    The \texttt{change-break} theorem style is like \texttt{break}
%    but with interchange of theorem name and theorem number.  Thus we
%    define |\th@changebreak| as follows:
%    \begin{macrocode}
\gdef\th@changebreak{\normalfont\slshape
  \def\@begintheorem##1##2{\item
       [\rlap{\vbox{\hbox{\hskip\labelsep \theorem@headerfont ##2\ ##1}%
                    \hbox{\strut}}}]}%
\def\@opargbegintheorem##1##2##3{%
  \item[\rlap{\vbox{\hbox{\hskip\labelsep \theorem@headerfont
                     ##2\ ##1\ (##3)}%
                    \hbox{\strut}}}]}}
\endgroup
%</thcb>
%    \end{macrocode}
% \end{macro}
%
%
% \subsubsection{The \texttt{change} style}
%
% This style option is stored in the file |thc.sty|.
%    \begin{macrocode}
%<*thc>
\begingroup \makeatletter
\@ifundefined{theorem@style}{\input{theorem.sty}}{}
\ProvidesFile{thc.sty}
             [\FMithmInfo]
%    \end{macrocode}
%
% \begin{macro}{\th@change}
%    The \texttt{change} theorem style corresponds to the
%    \texttt{change break} style without a linebreak after the header.
%    To say it in another way, it's the same as the \texttt{plain}
%    style but with number and name interchanged and |\slshape| as the
%    default font.
%    \begin{macrocode}
\gdef\th@change{\normalfont\slshape
  \def\@begintheorem##1##2{\item
       [\hskip\labelsep \theorem@headerfont ##2\ ##1]}%
\def\@opargbegintheorem##1##2##3{%
  \item[\hskip\labelsep \theorem@headerfont ##2\ ##1\ (##3)]}}
\endgroup
%</thc>
%    \end{macrocode}
% \end{macro}
%
%
% \subsubsection{The \texttt{marginbreak} style}
%
% This style option is the one used most often at Mainz.  It is saved
% in the file |thmb.sty|.
%    \begin{macrocode}
%<*thmb>
\begingroup \makeatletter
\@ifundefined{theorem@style}{\input{theorem.sty}}{}
\ProvidesFile{thmb.sty}
             [\FMithmInfo]
%    \end{macrocode}
%
% \begin{macro}{\th@marginbreak}
%    The \texttt{margin break} style is nearly the same as the
%    \texttt{change break} style. The only difference is the placement
%    of the theorem number.  We use |\llap| to place it in the left
%    margin.
%
%    In this style |\labelsep| denotes the separation between the
%    number and the text.
%    \begin{macrocode}
\gdef\th@marginbreak{\normalfont\slshape
  \def\@begintheorem##1##2{\item
       [\rlap{\vbox{\theorem@headerfont
                    \hbox{\llap{##2}\hskip\labelsep ##1}%
                    \hbox{\strut}}}]}%
\def\@opargbegintheorem##1##2##3{%
  \item[\rlap{\vbox{\theorem@headerfont
                    \hbox{\llap{##2}\hskip\labelsep ##1\ (##3)}%
                    \hbox{\strut}}}]}}
\endgroup
%</thmb>
%    \end{macrocode}
% \changes{v2.1c}{90/02/17}{`endgroup outside macrocode env.}
% \end{macro}
%
%
%
% \subsubsection{The \texttt{margin} style}
%
% This style option is stored in the file |thm.sty|.
%    \begin{macrocode}
%<*thm>
\begingroup \makeatletter
\@ifundefined{theorem@style}{\input{theorem.sty}}{}
\ProvidesFile{thm.sty}
             [\FMithmInfo]
%    \end{macrocode}
%
% \begin{macro}{\th@margin}
%    Again this is only a variant of the theorem styles described
%    above without any new ideas.
%    \begin{macrocode}
\gdef\th@margin{\normalfont\slshape
  \def\@begintheorem##1##2{\item
       [\theorem@headerfont \llap{##2}\hskip\labelsep ##1]}%
\def\@opargbegintheorem##1##2##3{%
  \item[\theorem@headerfont \llap{##2}\hskip\labelsep ##1\ (##3)]}}
\endgroup
%</thm>
%    \end{macrocode}
% \end{macro}
%
%
%
%
% \subsection{Final Definitions}
%
% \begin{macro}{\theorempreskipamount}
% \begin{macro}{\theorempostskipamount}
%    The \textsf{skip} parameters that regulate the vertical empty space
%    before and after the {\envfont theorem} environment have
%    to be allocated as well.
% \changes{v2.0f}{89/07/20}{Unnecessary `global' removed.}
%    \begin{macrocode}
%<*package>
\newskip\theorempreskipamount
\newskip\theorempostskipamount
%    \end{macrocode}
%    Since we have used the same values for all theorem sets, we now can
%    assign them.
%    \begin{macrocode}
\global\setlength\theorempreskipamount{12pt plus 5pt minus 3pt}
\global\setlength\theorempostskipamount{8pt plus 3pt minus 1.5pt}
%    \end{macrocode}
% \end{macro}
% \end{macro}
%
%
% \begin{macro}{\@endtheorem}
%    The same holds for the macro |\@endtheorem|, which ends a
%    {\envfont theorem} environment. Since it is the same for all
%    theorem sets, it is removed from the macros |\th@|\meta{style}.
%    It simply ends the {\envfont trivlist} environment, which was
%    begun in |\@thm|.
%    \begin{macrocode}
\global\let\@endtheorem=\endtrivlist
%    \end{macrocode}
% \end{macro}
%
%
%
% \begin{macro}{\@preamblecmds}
%    All macros defined above are to be used only in the preamble.
%    Therefore, we insert them in |\@preamblecmds| which will disable
%    them at begin document. This is done by the internal \LaTeXe{}
%    command |\@onlypreamble|.
% \changes{v2.0g}{89/09/19}{`theoremheaderfont and `theorembodyfont
%    added.}
% \changes{v2.2a}{94/02/03}{Use \LaTeXe{} interface}
%    \begin{macrocode}
\@onlypreamble\@xnthm
\@onlypreamble\@ynthm
\@onlypreamble\@othm
\@onlypreamble\newtheorem
\@onlypreamble\theoremstyle
\@onlypreamble\theorembodyfont
\@onlypreamble\theoremheaderfont
%    \end{macrocode}
% \end{macro}
%
% Finally we declare the \texttt{plain} theorem style to be the default.
% \changes{v2.1b}{90/06/04}{Fixed missing `item by declaring default
%    style.}
%    \begin{macrocode}
\theoremstyle{plain}
%</package>
%    \end{macrocode}
%
% \Finale

\endinput
}{}
\ProvidesFile{thmb.sty}
             [\FMithmInfo]
%    \end{macrocode}
%
% \begin{macro}{\th@marginbreak}
%    The \texttt{margin break} style is nearly the same as the
%    \texttt{change break} style. The only difference is the placement
%    of the theorem number.  We use |\llap| to place it in the left
%    margin.
%
%    In this style |\labelsep| denotes the separation between the
%    number and the text.
%    \begin{macrocode}
\gdef\th@marginbreak{\normalfont\slshape
  \def\@begintheorem##1##2{\item
       [\rlap{\vbox{\theorem@headerfont
                    \hbox{\llap{##2}\hskip\labelsep ##1}%
                    \hbox{\strut}}}]}%
\def\@opargbegintheorem##1##2##3{%
  \item[\rlap{\vbox{\theorem@headerfont
                    \hbox{\llap{##2}\hskip\labelsep ##1\ (##3)}%
                    \hbox{\strut}}}]}}
\endgroup
%</thmb>
%    \end{macrocode}
% \changes{v2.1c}{90/02/17}{`endgroup outside macrocode env.}
% \end{macro}
%
%
%
% \subsubsection{The \texttt{margin} style}
%
% This style option is stored in the file |thm.sty|.
%    \begin{macrocode}
%<*thm>
\begingroup \makeatletter
\@ifundefined{theorem@style}{% \iffalse meta-comment
%
% Copyright (C) 1993-2024
%
% The LaTeX Project and any individual authors listed elsewhere
% in this file.
%
% This file is part of the Standard LaTeX `Tools Bundle'.
% -------------------------------------------------------
%
% It may be distributed and/or modified under the
% conditions of the LaTeX Project Public License, either version 1.3c
% of this license or (at your option) any later version.
% The latest version of this license is in
%    https://www.latex-project.org/lppl.txt
% and version 1.3c or later is part of all distributions of LaTeX
% version 2005/12/01 or later.
%
% The list of all files belonging to the LaTeX `Tools Bundle' is
% given in the file `manifest.txt'.
%
% \fi
%
%%
%
%
\def\FMithmInfo{2023/07/05 v2.2c Theorem extension package (FMi)}
%
% \ProvidesFile{theorem.dtx}[\FMithmInfo]
%\iffalse   % this is a METACOMMENT !
%%
%% Package `theorem' to use with LaTeX2e
%% Copyright (C) 1989-1995 Frank Mittelbach, all rights reserved.
%<+package>\@ifundefined{theorem@style}{}{\endinput}
%<+package>\NeedsTeXFormat{LaTeX2e}[1995/06/01]
%<+package>\ProvidesPackage{theorem}
%<+package>         [\FMithmInfo]
%
% \fi
%
% \changes{v1.0}{87/05/07}{First draft.}
% \changes{v1.0a}{87/05/07}{`thistheoremstyle' added.}
% \changes{v1.1a}{87/07/29}{New concept: redefine the internal
%                          \protect\LaTeX{} commands `@xnthm' etc.}
% \changes{v1.1b}{87/07/31}{`theoremfont' used to specify font.}
% \changes{v1.2a}{88/01/08}{`@endtheorem' now globally defined.}
% \changes{v1.2b}{88/01/11}{`newtheorem' only allowed in the preamble.}
% \changes{v1.3a}{88/02/14}{`theoremfont' now a toks register.}
%
% \changes{v2.0a}{88/11/16}{New implementation and new documentation.}
% \changes{v2.0a}{88/11/16}{No layout changes but internal modifications
%                           old substyles won't work.}
% \changes{v2.0b}{89/01/30}{`global' used for all definitions.}
% \changes{v2.0c}{89/06/15}{Documentation translated to English.
%                            Thanks to LLL and Nina}
% \changes{v2.1a}{90/02/15}{Support for new font selection.}
% \changes{v2.2a}{94/02/03}{Cleanup for LaTeX2e}
%
% \DoNotIndex{\@endparenv,\@ifdefinable,\@ifnextchar,\@ifundefined}
% \DoNotIndex{\@namedef,\@nameuse,\@spaces,\@tempa,\@warning}
% \DoNotIndex{\ }
% \DoNotIndex{\begingroup,\bfseries}
% \DoNotIndex{\csname}
% \DoNotIndex{\def,\do,\docdate}
% \DoNotIndex{\endcsname,\endgroup,\endinput,\endtrivlist,\expandafter}
% \DoNotIndex{\filedate,\fileversion}
% \DoNotIndex{\gdef,\global}
% \DoNotIndex{\hbox,\hskip}
% \DoNotIndex{\ignorespaces,\input,\itshape,\item}
% \DoNotIndex{\labelsep,\let,\llap}
% \DoNotIndex{\makeatletter}
% \DoNotIndex{\noexpand}
% \DoNotIndex{\refstepcounter,\relax,\rlap}
% \DoNotIndex{\setlength,\slshape,\space,\string,\strut}
% \DoNotIndex{\the,\trivlist,\typeout}
% \DoNotIndex{\vbox}
% \DoNotIndex{\xdef}
%
% \def\deflabel#1{\ttfamily #1\hfill}
% \def\deflist#1{\begin{list}{}{\settowidth\labelwidth{\ttfamily #1}%
%                               \setlength\leftmargin\labelwidth
%                               \addtolength\leftmargin\labelsep
%                               \let\makelabel\deflabel}}
% \def\enddeflist{\end{list}}
% \def\envfont{\normalfont\ttfamily}
%
% \MakeShortVerb{\|}
%
% \newcommand{\TUB}{{\sl TUGboat\/}}
%
% \GetFileInfo{theorem.dtx}
% \title{An Extension of the
%        \LaTeX{} theorem environment\thanks{This file
%        has version number \fileversion, last
%        revised \filedate.}}
% \author{Frank Mittelbach}
%
% \MaintainedByLaTeXTeam{tools}
% \maketitle
%
% \begin{abstract}
%    The macros described in this paper yield an extension of the
%    \LaTeX{} theorem mechanism. It is designed to satisfy the
%    different requirements of various journals. Thus, the
%    layout of the ``theorems'' can be manipulated by determining a
%    ``style''. This article describes not only the use, but
%    also the definition, of the necessary macros.
% \end{abstract}
%
% \section*{Preface to version 2.2}
%
% For \LaTeXe{} this package did not need any fundamental changes. I
% only modified the messages generated so that theorem layout styles
% will show up with the |\listfiles| command and cleaned the section
% on the New Font Selection Scheme since this is now included in
% \LaTeX.
%
% \section*{Preface to version 2.1}
%
% This version is identical to 2.0g described in \TUB\ 10\#3 except
% for some internal defaults which are now set depending on the used
% font selection scheme.
%
% This was done to avoid unpleasant surprises if the new font selection
% scheme is in force. For further details see
% section~\ref{sec:newfonts} and \cite{bk:GMS94}.
%
% \section{Introduction}
%
% For our purposes here, ``theorems'' are labelled enunciations,
% often set off from the main text by extra space and a font change.
% Theorems, corollaries, conjectures, definitions, and remarks
% are all instances of ``theorems''.  The ``header'' of these structures
% is composed of a label (such as \textsc{Theorem} or \textsc{Remark})
% and a number which serializes an item in the sequence of items
% with the same label.
%
% Shortly after the introduction of \LaTeX{} at the Fachbereich
% Mathematik in Mainz, the desire to manipulate the layout of
% ``theorems'' arose.  In Mainz, the following two conventions came
% into general use:
% \begin{enumerate}
%   \item The number of the theorem is shown in the margin.
%   \item There is a line break at
%         the end of the theorem header.
% \end{enumerate}
%
% Additionally, some journals require different formats which depend on
% the ``sort of theorem'': e.g.\ often remarks and definitions are set
% in |\upshape|, while |\itshape| is employed for main theorems.
%
% Confronted with these requirements, a theorem environment was
% developed in Mainz which allows separate determination of the
% layout of the ``theorems sets'', comparable to |\pagestyle|.
%
%
% \section{The user interface}
% \subsection{Defining new theorem sets}
%
% \DescribeMacro\newtheorem
% As in the original \LaTeX{} version, the command |\newtheorem|
% defines a new ``theorem set'' or ``theorem-like structure''.
% Two required arguments name the new environment and give the text
% to be typeset with each instance of the new ``set'', while
% an optional argument determines how
% the ``set'' is enumerated:
% \begin{description}
%    \item[\ttfamily \bslash newtheorem\{foo\}\{bar\}]
%       The theorem set {\envfont foo} (whose name is \texttt{bar})
%       uses its own counter.
%    \item[\ttfamily \bslash newtheorem\{foo2\}{[foo]}\{bar2\}]
%       The theorem set {\envfont foo2} (printed name \texttt{bar2})
%       uses the same counter as the theorem set \texttt{foo}.
%    \item[\ttfamily \bslash newtheorem\{foo3\}\{bar3\}{[section]}]
%       The theorem set {\envfont foo3} (printed name \texttt{bar3}) is
%       enumerated within the counter \texttt{section}, i.e.\ with every
%       new |\section| the enumeration begins again with 1, and
%       the enumeration is composed from the section-number and the
%       theorem counter itself.
% \end{description}
%
% \DescribeMacro\theoremstyle
% Additionally, the command |\theoremstyle| can define the layout
% of various, or all, theorem sets.  It should be noted that any
% theorem set defined by |\newtheorem| is typeset in the
% |\theoremstyle| that is current at the time of the definition.
% Thus, the following
% \begin{verbatim}
%    \theoremstyle{break}        \newtheorem{Cor}{Corollary}
%    \theoremstyle{plain}        \newtheorem{Exa}{Example}[section]
%\end{verbatim}                             ^^A No blank at line start
% leads to the result that the set {\envfont Cor} is formatted in the
% style \texttt{break}, while the set {\envfont Exa} and all the
% following ones are formatted in the style \texttt{plain}, unless
% another |\theoremstyle| follows.  Since the definitions installed by
% |\newtheorem| are global, one also can limit |\theoremstyle| locally
% by grouping braces.
%
% \DescribeMacro\theorembodyfont
% The choice of the font for the theorem body is completely
% independent of the chosen |\theoremstyle|; this has proven to
% be very advantageous.  For example,
% \begin{verbatim}
%   {\theorembodyfont{\upshape}          \newtheorem{Rem}{Remark}}
%\end{verbatim}
% defines a theorem set {\envfont Rem}, which will be set in
% |\upshape| in the current layout (which in our example is
% \texttt{plain}).  As with |\theoremstyle|, the |\theorembodyfont|
% chosen is that current at the time of |\newtheorem|.  If
% |\theorembodyfont| is not specified or one defines
% |\theorembodyfont{}|, then the font used will be that defined by the
% |\theoremstyle|.
%
% \DescribeMacro\theoremheaderfont It is also possible to customize
% the font used for the theorem headers.  This is, however, a global
% declaration, and therefore there should be at most one
% |\theoremheaderfont| declaration in the preamble.\footnote{If it is
% actually necessary to have different header fonts, one has to define
% new theorem styles (substituting the desired font) or specify the
% information directly in the \texttt{\bslash newtheorem} declaration
% (the unclean variant).}
%
% \DescribeMacro\theorempreskipamount
% \DescribeMacro\theorempostskipamount Two additional parameters
% affect the vertical space around the theorem environments:
% |\theorempreskipamount| and |\theorempostskipamount| define,
% respectively, the spacing before and after such an environment.
% These parameters apply for all theorem sets and can be manipulated
% with the ordinary length macros.  They are rubber lengths
% (`\textsf{skips}'), and therefore can contain \texttt{plus} and
% \texttt{minus} parts.
%
% Since the definition of theorem sets should---most sensibly---be
% placed in the preamble, we only allow installation there.
% It is therefore possible to release the memory used here after
% |\begin{document}|, in order to make room for other
% applications.
%
%
% \subsection{Existing theorem styles}
%
% The following theorem styles exist to date:
% \begin{deflist}{marginbreak}
%    \item[plain]
%       This theorem style emulates the original \LaTeX{} definition,
%       except that additionally the parameters
%       |\theorem...skipamount| are used.
%    \item[break]
%       In this style, the theorem header is followed by a line
%       break.
%    \item[marginbreak]
%       The theorem number is set in the margin, and there is a line
%       break as in \texttt{break}.
%    \item[changebreak]
%       Like \texttt{break}, but with header number and text
%       interchanged.
%    \item[change]
%       Header number and text are interchanged, without a line break.
%    \item[margin]
%       The number is set in the left margin, without a line break.
% \end{deflist}
% All styles (except \texttt{plain}) select |\slshape| as the default
% |\theorembodyfont|.
%
%
% \subsection{Examples}
%
% Given the above theorem sets {\envfont Cor}, {\envfont Exa}
% and {\envfont Rem},
% suppose that the preamble also contains the declarations:
% \begin{verbatim}
%    \theoremstyle{marginbreak}   \newtheorem{Lem}[Cor]{Lemma}
%    \theoremstyle{change}
%    \theorembodyfont{\itshape}        \newtheorem{Def}[Cor]{Definition}
%
%    \theoremheaderfont{\scshape}
%\end{verbatim}
% Then the following are some typical examples of the typeset output
% resulting from their use.
%
% \begin{Cor}
%    This is a sentence typeset in the theorem environment {\envfont
%    Cor}.
% \end{Cor}
%
% \begin{Exa}
%    This is a sentence typeset in the theorem environment {\envfont
%    Exa}.
% \end{Exa}
%
% \begin{Rem}
%    This is a sentence typeset in the theorem environment {\envfont
%    Rem}.
% \end{Rem}
%
% \begin{Lem}[Ben User]
%    This is a sentence typeset in the theorem environment {\envfont
%    Lem}.
% \end{Lem}
%
% \begin{Def}[Very impressive Definition]
%    This is a sentence typeset in the theorem environment {\envfont
%    Def}.
% \end{Def}
%
% The last two examples show the effect of the optional argument to
% a theorem environment (it is the text typeset in parentheses).
%
%
% \section{Special Considerations}
% \label{sec:newfonts}
%
% Theoremheader and body are implemented as a unit. This means that
% the |\theoremheaderfont|  will inherit characteristics of the
% |\theorembodyfont| in \LaTeXe. Thus, if for example
% |\theorembodyfont| is |\itshape| and |\theoremheaderfont|
% is |\bfseries| the font selected for the header will have the
% characteristics `bold extended italic'. If this is not desired one
% should set the |\theoremheaderfont| to something like
% \begin{verbatim}
%      \theoremheaderfont{\normalfont\bfseries}
%\end{verbatim}
% i.e.\ supplying all necessary font information explicitly.
%
%
%
% \section{Acknowledgements}
%
% The publication of this set of macros was only possible with the
% help of Christina Busse (translating the manuscript into English),
% Joachim Pense (playing the r\^ole of typist), Chris Rowley
% (looking everything over) and many others providing useful
% suggestions.
%
% \MaybeStop{%
%    \begin{thebibliography}{9}
%       \bibitem{bk:GMS94} \textsc{M.~Goossens}, \textsc{F.~Mittelbach}
%         and \textsc{A.~Samarin}.
%         \newblock The \LaTeX{} Companion.
%         \newblock
%         Addison-Wesley, Reading, Massachusetts, 1994.
%       \bibitem{latex:source} \textsc{Lamport, Leslie.} \newblock
%                               \texttt{latex.tex}, version 2.09, date
%                               Feb. 1990.
%    \end{thebibliography}
%    \PrintIndex
%    \PrintChanges}
%
%
% \section{The documentation driver file}
%
% The next bit of code contains the documentation driver file for
% \TeX{}, i.e., the file that will produce the documentation you are
% currently reading. It will be extracted from this file by the
% \texttt{docstrip} program. Since it is the first code in the file
% one can alternatively process this file directly with \LaTeXe{} to
% obtain the documentation.
%
%    \begin{macrocode}
%<*driver>
\documentclass{ltxdoc}

\usepackage{theorem}

%<+driver>% The next few lines define theorem sets which are used
%<+driver>% in the example section of the documentation.

\theoremstyle{break}       \newtheorem{Cor}{Corollary}
\theoremstyle{plain}       \newtheorem{Exa}{Example}[section]
{\theorembodyfont{\upshape}\newtheorem{Rem}{Remark}}
\theoremstyle{marginbreak} \newtheorem{Lem}[Cor]{Lemma}
\theoremstyle{change}
\theorembodyfont{\itshape} \newtheorem{Def}[Cor]{Definition}

\theoremheaderfont{\scshape}

\RecordChanges

\begin{document}
   \DocInput{theorem.dtx}
\end{document}
%</driver>
%    \end{macrocode}
%
% \section{Definition of the Macros}
%
% If the file has been loaded before, we abort immediately. If not the
% package announces itself (this is actually done at the very top of
% the file---the way it is done isn't good style so don't copy it).
% \changes{v2.0e}{89/07/19}{Spaces removed from `typeout'.}
% \changes{v2.2b}{95/11/19}{Announce moved to top of file}
%    \begin{macrocode}
%<*package>
%\@ifundefined{theorem@style}{}{\endinput}
%\def\FMithmInfo{1995/11/19 v2.2b Theorem extension package (FMi)}
%\ProvidesPackage{theorem}[\FMithmInfo]
%    \end{macrocode}
%
%
% \subsection{Definition of theorem styles and fonts}
%
% All the definitions in this file are done globally to allow
% inputting this file inside a group.
%
% \begin{macro}{\theoremstyle}
%    Before a theorem style can be installed, the chosen style must be
%    known. For that reason, we must test to see that
%    |\th@|\meta{style} is known or, more precisely, that it is
%    different from |\relax|. If the style is not known then
%    |\th@plain| is used.
%    \begin{macrocode}
\gdef\theoremstyle#1{%
   \@ifundefined{th@#1}{\@warning
          {Unknown theoremstyle `#1'. Using `plain'}%
          \theorem@style{plain}}%
%    \end{macrocode}
%    We save the theorem style to be used in the token register
%    |\theorem@style|.
%    \begin{macrocode}
      {\theorem@style{#1}}%
%    \end{macrocode}
%    Now we ``evaluate'' the theorem style: this means, we call the
%    macro |\th@|\meta{style} which will activate the relevant
%    definitions which are contained in a separate file.  This is done
%    in a group to suppress changes to the current font.  This could
%    otherwise pose problems together with the new font selection
%    scheme\footnote{When I printed the original article using the new
%    font selection scheme I ended with a document with slanted
%    typefaces (text headings and all) simply because one of the
%    theorem styles used \texttt{\bslash sl} at toplevel.} if the
%    |\th@|\meta{style} is evaluated a second time.
% \changes{v2.1a}{90/02/15}{Group added around `th@..}
% \changes{v2.1a}{90/02/15}{Reseting of `..@begintheorem removed.}
%    \begin{macrocode}
      \begingroup
        \csname th@\the\theorem@style \endcsname
      \endgroup}
%    \end{macrocode}
% \end{macro}
%
%
% \begin{macro}{\@begintheorem}
% \changes{v2.1a}{90/02/15}{Relaxed at toplevel.}
% \begin{macro}{\@opargbegintheorem}
% \changes{v2.1a}{90/02/15}{Relaxed at toplevel.}
%    We reset |\@begintheorem| and |\@opargbegintheorem| to
%    |\relax| since these commands are no longer necessary at
%    toplevel. This will save a few tokens.
%    \begin{macrocode}
\global\let\@begintheorem\relax
\global\let\@opargbegintheorem\relax
%    \end{macrocode}
% \end{macro}
% \end{macro}
%
% \begin{macro}{\theorem@style}
%    Obviously the \textsf{token} register used above has to be
%    allocated. To assure the utmost compatibility with the
%    original \LaTeX{} definition, we set the default theorem style to
%    \texttt{plain}, which implements the usual \LaTeX{} convention.
% \changes{v2.0f}{89/07/20}{Unnecessary `global' removed.}
%    \begin{macrocode}
\newtoks\theorem@style
\global\theorem@style{plain}
%    \end{macrocode}
% \end{macro}
%
% \begin{macro}{\theorembodyfont}
% \begin{macro}{\theorem@bodyfont}
%    If the |\theorembodyfont| is set by the user then it should not
%    interact with the default font set in the theorem style. When the
%    new font selection is in force this may happen if, for example,
%    the default is |\itshape| and the new |\theorembodyfont| is
%    |\sffamily|. So we add a |\reset@font| command in front of the
%    user definition.
% \changes{v2.0f}{89/07/20}{Unnecessary `global' removed.}
% \changes{v2.0g}{89/09/19}{`theoremfont to `theorembodyfont.}
% \changes{v2.1c}{92/06/22}{`reset@font added and logic changed.}
% \changes{v2.1d}{92/10/28}{Reset to empty if arg is empty}
%    \begin{macrocode}
\gdef\theorembodyfont#1{%
%    \end{macrocode}
%    We check if the argument supplied is empty and if so put nothing
%    into the |\theorem@bodyfont| token register to allow for
%    |\theorembodyfont{}| as a mean of using the default of the
%    current |\theoremstyle|.
%    \begin{macrocode}
   \def\@tempa{#1}%
   \ifx\@tempa\@empty
    \theorem@bodyfont{}%
   \else
    \theorem@bodyfont{\reset@font#1}%
   \fi
   }
\newtoks\theorem@bodyfont
\global\theorem@bodyfont{}
%    \end{macrocode}
% \end{macro}
% \end{macro}
%
%
% \begin{macro}{\theoremheaderfont}
% \changes{v2.0g}{89/09/19}{Macro added.}
%    The font for the theorem headers is handled differently because
%    this definition applies to all theorem styles.
%    \begin{macrocode}
\gdef\theoremheaderfont#1{\gdef\theorem@headerfont{#1}%
%    \end{macrocode}
%    After using the macro once it is redefined to produce an error
%    message.
%    \begin{macrocode}
       \gdef\theoremheaderfont##1{%
        \typeout{\string\theoremheaderfont\space should be used
                 only once.}}}
%    \end{macrocode}
% \end{macro}
%
% \begin{macro}{\theorem@headerfont}
% \changes{v2.0g}{89/09/19}{Macro added.}
%    To set the |\theorem@headerfont| default we first test if
%    the new fontselection scheme is in force.
% \changes{v2.1a}{90/02/15}{Support for new font selection.}
%    \begin{macrocode}
\ifx\upshape\undefined
%    \end{macrocode}
%    If not we define it to expand into |\bfseries|. We don't use
%    |\let| just in case a following style option redefines this
%    macro.
%    \begin{macrocode}
\gdef\theorem@headerfont{\bfseries}
%    \end{macrocode}
%    Otherwise we reset the current shape before calling |\bfseries|.
%    \begin{macrocode}
\else \gdef\theorem@headerfont{\normalfont\bfseries}\fi
%    \end{macrocode}
% \end{macro}
%
% \begin{macro}{\th@plain}
% \begin{macro}{\th@break}
% \begin{macro}{\th@marginbreak}
% \begin{macro}{\th@changebreak}
% \begin{macro}{\th@change}
% \begin{macro}{\th@margin}
%    The different styles are defined in macros such as
%    |\th@plain|.  Since memory space is precious in
%    ``non-Big-versions'', we have to avoid offering too many unused
%    definitions.  Therefore we define these styles in separate files
%    that can be loaded on demand.  Thus the commands themselves only
%    load these files. We use |\@input@|, a \LaTeXe{} internal command
%    that ensures that the file will be listed with |\listfiles|.
% \changes{v2.2a}{94/02/02}{Use `@input@ to load theorem layout files}
%    \begin{macrocode}
\gdef\th@plain{\@input@{thp.sty}}
\gdef\th@break{\@input@{thb.sty}}
\gdef\th@marginbreak{\@input@{thmb.sty}}
\gdef\th@changebreak{\@input@{thcb.sty}}
\gdef\th@change{\@input@{thc.sty}}
\gdef\th@margin{\@input@{thm.sty}}
%    \end{macrocode}
%    This list will be expanded when new styles become
%    available.  For testing,  just append new theorem substyles
%    as document options.
% \end{macro}
% \end{macro}
% \end{macro}
% \end{macro}
% \end{macro}
% \end{macro}
%
%
%
% \subsection{Definition of a new theorem set}
%
% As already pointed out, a new theorem environment can be defined in
% three different ways:
% \begin{verbatim}
%    \newtheorem{Lem}{Lemma}
%    \newtheorem{Lem}{Lemma}[section]
%    \newtheorem{Lem}[Theorem]{Lemma}
%\end{verbatim}
% The function of the macro |\newtheorem| is to recognize these cases
% and then to branch into one of the three macros |\@ynthm|, |\@xnthm|
% or |\@othm|.  This mechanism is adopted unchanged from
% \cite{latex:source}; the essential point here is that, for example,
% in the second case, the arguments \texttt{Lem}, \texttt{Lemma} and
% \texttt{section} are passed over to the macro |\@xnthm|.
%
% We inspect this case first because the others present fewer problems,
% and thus are easily derived from this one.
%
% \begin{macro}{\@xnthm}
%    For our example arguments, the macro |\@xnthm| must fulfill
%    the following:
%    \begin{itemize}
%       \item
%          Define a new \LaTeX-counter `Lem'
%       \item
%          reset this counter within a |\section|
%       \item
%          define the macro |\theLem|
%       \item
%          define the environment macros |\Lem| and
%          |\endLem| using the current
%          |\theoremstyle| and |\theorem@bodyfont|.
%    \end{itemize}
%    Obviously, all this should happen only if the first argument
%    of |\@xnthm| (i.e.\ \texttt{Lem} in our example) is chosen so
%    as not to conflict with any previously defined commands or
%    environments.  This test is performed by the \LaTeX{} macro
%    |\@ifdefinable|.
%    \begin{macrocode}
\gdef\@xnthm#1#2[#3]{\expandafter\@ifdefinable\csname #1\endcsname
%    \end{macrocode}
%    Therefore, the first argument of |\@ifdefinable| is the
%    expansion (in the example, |\Lem|) of
%    |\csname||#1\endcsname|.  The second argument is
%    executed only if the test has been completed successfully.
%    \begin{macrocode}
   {%
%    \end{macrocode}
%    Now we define the new counter. The names of the \LaTeX{}
%    macros employed should speak for themselves:
% \changes{v2.2b}{1995/10/07}{Use \cs{@newctr}}
% \changes{v2.2c}{1995/11/23}{Use \cs{@newctr} properly.}
%    \begin{macrocode}
    \@definecounter{#1}\@newctr{#1}[#3]%
%    \end{macrocode}
%    Using |\@newctr| will give a proper error message if the counter
%    in |#3| is not defined.
%    In defining `|\theLem|' we must generate the desired macro
%    name by use of |\expandafter| and |\csname|.
%    \begin{macrocode}
    \expandafter\xdef\csname the#1\endcsname
%    \end{macrocode}
%    An |\xdef| is used in order to make the definition global,
%    and to ensure that it contains the replacement texts of
%    |\@thmcountersep| and |\@thmcounter|.\footnote{These
%    two macros can be defined by the document style.  Their default
%    values produce a `\texttt{.}' as separation and an arabic
%    representation of the number.} However, not everything should be
%    expanded. For example, it saves space to use |\thesection|
%    instead of its---at times---lengthy expansion.
%    \begin{macrocode}
      {\expandafter \noexpand \csname the#3\endcsname
       \@thmcountersep \@thmcounter{#1}}%
%    \end{macrocode}
%    Thus with the defaults of \LaTeX{}, |\theLem| would be
%    replaced by the command sequence |\thesection.\arabic{Lem}|.
%
%    We will now look at the definition of the macro which is executed
%    at the beginning of the actual environment (in our example this
%    macro is |\Lem|). It should be noted that we use an
%    ``|\expandafter| trick'' to expand only certain parts of the
%    replacement text at the time of the definition.
%    \begin{macrocode}
    \def\@tempa{\global\@namedef{#1}}%
    \expandafter \@tempa \expandafter{%
%    \end{macrocode}
%    First, the macro that contains the current definitions of
%    |\@begintheorem| and |\@opargtheorem| should be called
%    up.  The name of this macro---as is already known---has the form
%    |\th@|\meta{theorem\ style}; therefore, it must be called by
%    \begin{macrocode}
      \csname th@\the \theorem@style
%    \end{macrocode}
%    In addition the default theorem font should be changeable, i.e.\
%    we have to insert the contents of |\theorem@bodyfont|. For
%    that reason, we expand even further, beyond |\endcsname|,
%    and thus insert the contents of the \textsf{token} register
%    |\theorem@bodyfont| in the replacement text.
%    \begin{macrocode}
            \expandafter \endcsname \the \theorem@bodyfont
%    \end{macrocode}
%    Now it is time to call the macro |\@thm| which takes over
%    the further processing.  It has two arguments: the current
%    counter name (in our example, \texttt{Lem}), and the text of the
%    label (in our example, \texttt{Lemma}).
%    \begin{macrocode}
     \@thm{#1}{#2}}%
%    \end{macrocode}
%    With this, the `sub-definition' is complete.  The macro
%    |\@endtheorem| ends a theorem environment and is, so far, nothing
%    but an |\endtrivlist|.  (Hence it is defined globally, and not
%    within the theorem styles.\footnote{This has to be changed as
%    soon as theorem styles that change \texttt{\bslash @endtheorem}
%    exist.  In such a case, all existing styles must be changed as
%    well since they will have to reset the macro.}) Therefore, we can
%    set it equivalent to the macro that ends the theorem set (in our
%    example, |\endLem|).  However, if some day theorem styles exist
%    that do change |\@endtheorem|, we would have to use the
%    commented-out line instead.
%    \begin{macrocode}
    \global \expandafter \let \csname end#1\endcsname \@endtheorem
%   \global\@namedef{end#1}{\@endtheorem}%
%    \end{macrocode}
%    With these commands all the required definitions are employed,
%    unless the test |\@ifdefinable| has failed.  Therefore, we
%    end the second argument of this macro and with it the definition
%    of |\@xnthm|.
%    \begin{macrocode}
   }}
%    \end{macrocode}
% \end{macro}
%
% \begin{macro}{\@ynthm}
%    The definition of |\@ynthm| is completely analogous. In
%    this case the new counter that is defined is not reset within
%    another counter; thus the definition of |\the...|
%    is simplified:
%    \begin{macrocode}
\gdef\@ynthm#1#2{\expandafter\@ifdefinable\csname #1\endcsname
   {\@definecounter{#1}%
    \expandafter\xdef\csname the#1\endcsname{\@thmcounter{#1}}%
%    \end{macrocode}
%    The rest of the definition corresponds literally to that of
%    |\@xnthm|:
%    \begin{macrocode}
    \def\@tempa{\global\@namedef{#1}}\expandafter \@tempa
     \expandafter{\csname th@\the \theorem@style \expandafter
     \endcsname \the\theorem@bodyfont \@thm{#1}{#2}}%
    \global \expandafter \let \csname end#1\endcsname \@endtheorem}}
%    \end{macrocode}
% \end{macro}
%
% \begin{macro}{\@othm}
%    The definition of |\@othm| does not contain anything new.
%    \begin{macrocode}
\gdef\@othm#1[#2]#3{%
%    \end{macrocode}
% \changes{v2.2b}{95/10/07}{Build in check for counter defined}
%    We do not define a new counter but instead use one that has
%    already been defined.  Thus the only definition we need is that
%    of this pseudo-counter (i.e.\ |\the|\meta{env.\ name}).
%    First we check if |#2| corresponds to a known counter name.
%    \begin{macrocode}
  \expandafter\ifx\csname c@#2\endcsname\relax
   \@nocounterr{#2}%
  \else
   \expandafter\@ifdefinable\csname #1\endcsname
   {\expandafter \xdef \csname the#1\endcsname
     {\expandafter \noexpand \csname the#2\endcsname}%
%    \end{macrocode}
%    All other parts of the definition can be adopted from
%    |\@xnthm|. We have to remember, though, that in this case
%    the name of the current counter and the theorem label have moved
%    to the second and third arguments.
%    \begin{macrocode}
    \def\@tempa{\global\@namedef{#1}}\expandafter \@tempa
     \expandafter{\csname th@\the \theorem@style \expandafter
     \endcsname \the\theorem@bodyfont \@thm{#2}{#3}}%
    \global \expandafter \let \csname end#1\endcsname \@endtheorem}%
  \fi}
%    \end{macrocode}
% \end{macro}
%
% \subsection{Macros that are employed in a {\envfont theorem}
%    environment}
%
% \begin{macro}{\@thm}
%    The macro |\@thm| has to increase the current counter.  Then,
%    depending on whether the environment has (or does not have) an
%    optional argument, it has to branch into either |\@begintheorem|
%    or |\@opargtheorem|.
%    \begin{macrocode}
\gdef\@thm#1#2{\refstepcounter{#1}%
%    \end{macrocode}
%    Now we start a {\envfont trivlist} environment, and give
%    |\@topsep| and |\@topsepadd| the values of the \textsf{skip}
%    registers |\theorempreskipamount| and |\theorempostskipamount|.
%    The value in |\@topsep| is the vertical space that is inserted by
%    the first (and only) |\item| in our |\trivlist| whilst
%    |\@topsepadd| is inserted by |\@endparenv| at the end of that
%    {\envfont trivlist} environment.  By using these registers, we
%    obtain the desired space around a {\envfont theorem} environment.
%    \begin{macrocode}
   \trivlist
   \@topsep \theorempreskipamount               % used by first \item
   \@topsepadd \theorempostskipamount           % used by \@endparenv
%    \end{macrocode}
%    Now we have to test whether an optional argument has been given.
% \changes{v2.0d}{89/07/18}{Forgotten percent added.}
%    \begin{macrocode}
   \@ifnextchar [%
%    \end{macrocode}
%    If there is an optional argument, we will call |\@ythm|, and
%    move the arguments read back into the input stream.
%    \begin{macrocode}
   {\@ythm{#1}{#2}}%
%    \end{macrocode}
%    If not, we call |\@begintheorem|. Its first argument is the
%    name of the theorem set (hence the second argument of
%    |\@thm|). Its second argument is the macro that produces the
%    current number.
%    \begin{macrocode}
   {\@begintheorem{#2}{\csname the#1\endcsname}\ignorespaces}}
%    \end{macrocode}
% \end{macro}
%
% \begin{macro}{\@xthm}
% \begin{macro}{\@ythm}
%    Both these macros were originally called by |\@thm|.  We do
%    not need |\@xthm| anymore, hence we reset it to
%    |\relax|. The definition of |\@ythm| has not changed at
%    all from its definition in \LaTeX. In order to make the macros
%    easier to understand, we will
%    nevertheless present it (commented out).
%    \begin{macrocode}
\global\let\@xthm\relax
% \def\@ythm#1#2[#3]{\@opargbegintheorem{#2}{\csname
%           the#1\endcsname}{#3}\ignorespaces}
%</package>
%    \end{macrocode}
%    The primitive |\ignorespaces| in |\@ythm| and
%    |\@thm| is needed to remove the spaces between the
%    |\begin{...}| and the actual text.
% \end{macro}
% \end{macro}
%
%
% \subsection{Definition of the theorem substyles}
%
% As already pointed out, the theorem substyles, defined below, are
% only loaded when necessary.  Note that all these substyles, except
% \texttt{plain}, have |\slshape| as the default body font.
%
% \subsubsection{The \texttt{plain} style}
%
% \label{dok:th@plain} As the following macros use |@|, we have
% to locally set the |\catcode| of this symbol to ``letter''.
% This happens within a group, so that we do not have to worry about
% which |\catcode| that symbol had before.
%    \begin{macrocode}
%<*thp>
\begingroup \makeatletter
%    \end{macrocode}
% Since we are now within a group, we must make all definitions
% globally.  First we make sure that \texttt{theorem.sty} is loaded.
% This will allow us to use this file as a document style option
% without having to call \texttt{theorem} itself as an option.  At the
% same time, we assure that at least version 2 is loaded, since
% |\theorem@style| was not defined in earlier versions.
%    \begin{macrocode}
\@ifundefined{theorem@style}{\input{theorem.sty}}{}
\ProvidesFile{thp.sty}
             [\FMithmInfo]
%    \end{macrocode}
%
%
% \begin{macro}{\th@plain}
% \begin{macro}{\@begintheorem}
% \begin{macro}{\@opargbegintheorem}
%    |\theoremstyle{plain}| corresponds to the original
%    definition, except that the distances to the surrounding text are
%    determined by the parameters |\theorempreskipamount| and
%    |\theorempostskipamount|.
%    First we set the default body font.
%    \begin{macrocode}
\gdef\th@plain{\normalfont\itshape
%    \end{macrocode}
%    Then we define |\@begintheorem| and
%    |\@opargbegintheorem|. These two macros define how the
%    header of a theorem is typeset.  |\@opargbegintheorem| will
%    be called if a \textsf{theorem} environment with an optional
%    argument is encountered; otherwise, the header is constructed by
%    calling |\@begintheorem|.  If one of these macros is
%    executed, we are within a \textsf{trivlist} environment started by
%    |\@thm|.  So the theorem header is produced with an
%    |\item| command.
%
%    Instead of specifying the header font directly, all standard
%    theorem styles use the |\theorem@headerfont| macro to allow
%    customization.  The extra space (|\labelsep|) is necessary
%    because of problems in the \textsf{trivlist} environment.
%    \begin{macrocode}
  \def\@begintheorem##1##2{%
        \item[\hskip\labelsep \theorem@headerfont ##1\ ##2]}%
%    \end{macrocode}
%    The definition of |\@opargbegintheorem| is completely
%    analogous. The only difference is the fact that there exists a
%    third argument (which is the optional parameter of the
%    environment and contains additional information about the
%    theorem).  Customarily we enclose it in parentheses.
%    \begin{macrocode}
\def\@opargbegintheorem##1##2##3{%
   \item[\hskip\labelsep \theorem@headerfont ##1\ ##2\ (##3)]}}
%    \end{macrocode}
% \end{macro}
% \end{macro}
% \end{macro}
%
% \noindent
% We conclude with an |\endgroup| to restore the
% |\catcode| of |@|.
%    \begin{macrocode}
\endgroup
%</thp>
%    \end{macrocode}
%
%
% \subsubsection{The \texttt{break} style}
%
% This style option is stored in the file |thb.sty|.
% For the next two lines see the documentation for |\th@plain| on
% page~\pageref{dok:th@plain}.
%    \begin{macrocode}
%<*thb>
\begingroup \makeatletter
\@ifundefined{theorem@style}{\input{theorem.sty}}{}
\ProvidesFile{thb.sty}
             [\FMithmInfo]
%    \end{macrocode}
%
% \begin{macro}{\th@break}
%    |\theoremstyle{break}| produces a line break after the name
%    of the theorem.  The font is |\slshape|.  Hence, we define
%    |\th@break| as follows:
%    \begin{macrocode}
\gdef\th@break{\normalfont\slshape
  \def\@begintheorem##1##2{\item[%
%    \end{macrocode}
%    We run into the following problem: it is not possible to create
%    the header with |\item[|\meta{title}|]| and then start
%    a new line by, for example, |\mbox{}\\|. Such a definition
%    will fail whenever a list environment follows immediately.  With
%    the above construction, the |\mbox{}| causes the switch
%    |@inlabel| (cf.\ definition of |\list| and
%    |\trivlist| in \cite{latex:source}) to be set to \texttt{false}
%    and so the following list will insert additional vertical
%    space (|\topskip|). This is quite annoying. Therefore, we
%    create the line break within the |\item|. In order to ensure
%    that the text will begin at the proper position in the following
%    line, we simply pretend that the label does not take any
%    room.\footnote{This will lead to problems whenever very high
%    symbols occurring in the line tower into the
%    heading. So, something else has to be done here sometime.}
%    \begin{macrocode}
       \rlap{\vbox{\hbox{\hskip \labelsep\theorem@headerfont ##1\ ##2}%
                   \hbox{\strut}}}]}%
%    \end{macrocode}
%    Again, the definition of |\@opargbegintheorem| is
%    completely analogous.
%    \begin{macrocode}
\def\@opargbegintheorem##1##2##3{%
  \item[\rlap{\vbox{\hbox{\hskip \labelsep \theorem@headerfont
                     ##1\ ##2\ (##3)}%
                    \hbox{\strut}}}]}}
\endgroup
%</thb>
%    \end{macrocode}
% \end{macro}
%
%
% \subsubsection{The \texttt{changebreak} style}
%
%    \begin{macrocode}
% This style option is stored in the file |thcb.sty|.
%    \begin{macrocode}
%<*thcb>
\begingroup \makeatletter
\@ifundefined{theorem@style}{\input{theorem.sty}}{}
\ProvidesFile{thcb.sty}
             [\FMithmInfo]
%    \end{macrocode}
%
% \begin{macro}{\th@changebreak}
%    The \texttt{change-break} theorem style is like \texttt{break}
%    but with interchange of theorem name and theorem number.  Thus we
%    define |\th@changebreak| as follows:
%    \begin{macrocode}
\gdef\th@changebreak{\normalfont\slshape
  \def\@begintheorem##1##2{\item
       [\rlap{\vbox{\hbox{\hskip\labelsep \theorem@headerfont ##2\ ##1}%
                    \hbox{\strut}}}]}%
\def\@opargbegintheorem##1##2##3{%
  \item[\rlap{\vbox{\hbox{\hskip\labelsep \theorem@headerfont
                     ##2\ ##1\ (##3)}%
                    \hbox{\strut}}}]}}
\endgroup
%</thcb>
%    \end{macrocode}
% \end{macro}
%
%
% \subsubsection{The \texttt{change} style}
%
% This style option is stored in the file |thc.sty|.
%    \begin{macrocode}
%<*thc>
\begingroup \makeatletter
\@ifundefined{theorem@style}{\input{theorem.sty}}{}
\ProvidesFile{thc.sty}
             [\FMithmInfo]
%    \end{macrocode}
%
% \begin{macro}{\th@change}
%    The \texttt{change} theorem style corresponds to the
%    \texttt{change break} style without a linebreak after the header.
%    To say it in another way, it's the same as the \texttt{plain}
%    style but with number and name interchanged and |\slshape| as the
%    default font.
%    \begin{macrocode}
\gdef\th@change{\normalfont\slshape
  \def\@begintheorem##1##2{\item
       [\hskip\labelsep \theorem@headerfont ##2\ ##1]}%
\def\@opargbegintheorem##1##2##3{%
  \item[\hskip\labelsep \theorem@headerfont ##2\ ##1\ (##3)]}}
\endgroup
%</thc>
%    \end{macrocode}
% \end{macro}
%
%
% \subsubsection{The \texttt{marginbreak} style}
%
% This style option is the one used most often at Mainz.  It is saved
% in the file |thmb.sty|.
%    \begin{macrocode}
%<*thmb>
\begingroup \makeatletter
\@ifundefined{theorem@style}{\input{theorem.sty}}{}
\ProvidesFile{thmb.sty}
             [\FMithmInfo]
%    \end{macrocode}
%
% \begin{macro}{\th@marginbreak}
%    The \texttt{margin break} style is nearly the same as the
%    \texttt{change break} style. The only difference is the placement
%    of the theorem number.  We use |\llap| to place it in the left
%    margin.
%
%    In this style |\labelsep| denotes the separation between the
%    number and the text.
%    \begin{macrocode}
\gdef\th@marginbreak{\normalfont\slshape
  \def\@begintheorem##1##2{\item
       [\rlap{\vbox{\theorem@headerfont
                    \hbox{\llap{##2}\hskip\labelsep ##1}%
                    \hbox{\strut}}}]}%
\def\@opargbegintheorem##1##2##3{%
  \item[\rlap{\vbox{\theorem@headerfont
                    \hbox{\llap{##2}\hskip\labelsep ##1\ (##3)}%
                    \hbox{\strut}}}]}}
\endgroup
%</thmb>
%    \end{macrocode}
% \changes{v2.1c}{90/02/17}{`endgroup outside macrocode env.}
% \end{macro}
%
%
%
% \subsubsection{The \texttt{margin} style}
%
% This style option is stored in the file |thm.sty|.
%    \begin{macrocode}
%<*thm>
\begingroup \makeatletter
\@ifundefined{theorem@style}{\input{theorem.sty}}{}
\ProvidesFile{thm.sty}
             [\FMithmInfo]
%    \end{macrocode}
%
% \begin{macro}{\th@margin}
%    Again this is only a variant of the theorem styles described
%    above without any new ideas.
%    \begin{macrocode}
\gdef\th@margin{\normalfont\slshape
  \def\@begintheorem##1##2{\item
       [\theorem@headerfont \llap{##2}\hskip\labelsep ##1]}%
\def\@opargbegintheorem##1##2##3{%
  \item[\theorem@headerfont \llap{##2}\hskip\labelsep ##1\ (##3)]}}
\endgroup
%</thm>
%    \end{macrocode}
% \end{macro}
%
%
%
%
% \subsection{Final Definitions}
%
% \begin{macro}{\theorempreskipamount}
% \begin{macro}{\theorempostskipamount}
%    The \textsf{skip} parameters that regulate the vertical empty space
%    before and after the {\envfont theorem} environment have
%    to be allocated as well.
% \changes{v2.0f}{89/07/20}{Unnecessary `global' removed.}
%    \begin{macrocode}
%<*package>
\newskip\theorempreskipamount
\newskip\theorempostskipamount
%    \end{macrocode}
%    Since we have used the same values for all theorem sets, we now can
%    assign them.
%    \begin{macrocode}
\global\setlength\theorempreskipamount{12pt plus 5pt minus 3pt}
\global\setlength\theorempostskipamount{8pt plus 3pt minus 1.5pt}
%    \end{macrocode}
% \end{macro}
% \end{macro}
%
%
% \begin{macro}{\@endtheorem}
%    The same holds for the macro |\@endtheorem|, which ends a
%    {\envfont theorem} environment. Since it is the same for all
%    theorem sets, it is removed from the macros |\th@|\meta{style}.
%    It simply ends the {\envfont trivlist} environment, which was
%    begun in |\@thm|.
%    \begin{macrocode}
\global\let\@endtheorem=\endtrivlist
%    \end{macrocode}
% \end{macro}
%
%
%
% \begin{macro}{\@preamblecmds}
%    All macros defined above are to be used only in the preamble.
%    Therefore, we insert them in |\@preamblecmds| which will disable
%    them at begin document. This is done by the internal \LaTeXe{}
%    command |\@onlypreamble|.
% \changes{v2.0g}{89/09/19}{`theoremheaderfont and `theorembodyfont
%    added.}
% \changes{v2.2a}{94/02/03}{Use \LaTeXe{} interface}
%    \begin{macrocode}
\@onlypreamble\@xnthm
\@onlypreamble\@ynthm
\@onlypreamble\@othm
\@onlypreamble\newtheorem
\@onlypreamble\theoremstyle
\@onlypreamble\theorembodyfont
\@onlypreamble\theoremheaderfont
%    \end{macrocode}
% \end{macro}
%
% Finally we declare the \texttt{plain} theorem style to be the default.
% \changes{v2.1b}{90/06/04}{Fixed missing `item by declaring default
%    style.}
%    \begin{macrocode}
\theoremstyle{plain}
%</package>
%    \end{macrocode}
%
% \Finale

\endinput
}{}
\ProvidesFile{thm.sty}
             [\FMithmInfo]
%    \end{macrocode}
%
% \begin{macro}{\th@margin}
%    Again this is only a variant of the theorem styles described
%    above without any new ideas.
%    \begin{macrocode}
\gdef\th@margin{\normalfont\slshape
  \def\@begintheorem##1##2{\item
       [\theorem@headerfont \llap{##2}\hskip\labelsep ##1]}%
\def\@opargbegintheorem##1##2##3{%
  \item[\theorem@headerfont \llap{##2}\hskip\labelsep ##1\ (##3)]}}
\endgroup
%</thm>
%    \end{macrocode}
% \end{macro}
%
%
%
%
% \subsection{Final Definitions}
%
% \begin{macro}{\theorempreskipamount}
% \begin{macro}{\theorempostskipamount}
%    The \textsf{skip} parameters that regulate the vertical empty space
%    before and after the {\envfont theorem} environment have
%    to be allocated as well.
% \changes{v2.0f}{89/07/20}{Unnecessary `global' removed.}
%    \begin{macrocode}
%<*package>
\newskip\theorempreskipamount
\newskip\theorempostskipamount
%    \end{macrocode}
%    Since we have used the same values for all theorem sets, we now can
%    assign them.
%    \begin{macrocode}
\global\setlength\theorempreskipamount{12pt plus 5pt minus 3pt}
\global\setlength\theorempostskipamount{8pt plus 3pt minus 1.5pt}
%    \end{macrocode}
% \end{macro}
% \end{macro}
%
%
% \begin{macro}{\@endtheorem}
%    The same holds for the macro |\@endtheorem|, which ends a
%    {\envfont theorem} environment. Since it is the same for all
%    theorem sets, it is removed from the macros |\th@|\meta{style}.
%    It simply ends the {\envfont trivlist} environment, which was
%    begun in |\@thm|.
%    \begin{macrocode}
\global\let\@endtheorem=\endtrivlist
%    \end{macrocode}
% \end{macro}
%
%
%
% \begin{macro}{\@preamblecmds}
%    All macros defined above are to be used only in the preamble.
%    Therefore, we insert them in |\@preamblecmds| which will disable
%    them at begin document. This is done by the internal \LaTeXe{}
%    command |\@onlypreamble|.
% \changes{v2.0g}{89/09/19}{`theoremheaderfont and `theorembodyfont
%    added.}
% \changes{v2.2a}{94/02/03}{Use \LaTeXe{} interface}
%    \begin{macrocode}
\@onlypreamble\@xnthm
\@onlypreamble\@ynthm
\@onlypreamble\@othm
\@onlypreamble\newtheorem
\@onlypreamble\theoremstyle
\@onlypreamble\theorembodyfont
\@onlypreamble\theoremheaderfont
%    \end{macrocode}
% \end{macro}
%
% Finally we declare the \texttt{plain} theorem style to be the default.
% \changes{v2.1b}{90/06/04}{Fixed missing `item by declaring default
%    style.}
%    \begin{macrocode}
\theoremstyle{plain}
%</package>
%    \end{macrocode}
%
% \Finale

\endinput
}{}
\ProvidesFile{thb.sty}
             [\FMithmInfo]
%    \end{macrocode}
%
% \begin{macro}{\th@break}
%    |\theoremstyle{break}| produces a line break after the name
%    of the theorem.  The font is |\slshape|.  Hence, we define
%    |\th@break| as follows:
%    \begin{macrocode}
\gdef\th@break{\normalfont\slshape
  \def\@begintheorem##1##2{\item[%
%    \end{macrocode}
%    We run into the following problem: it is not possible to create
%    the header with |\item[|\meta{title}|]| and then start
%    a new line by, for example, |\mbox{}\\|. Such a definition
%    will fail whenever a list environment follows immediately.  With
%    the above construction, the |\mbox{}| causes the switch
%    |@inlabel| (cf.\ definition of |\list| and
%    |\trivlist| in \cite{latex:source}) to be set to \texttt{false}
%    and so the following list will insert additional vertical
%    space (|\topskip|). This is quite annoying. Therefore, we
%    create the line break within the |\item|. In order to ensure
%    that the text will begin at the proper position in the following
%    line, we simply pretend that the label does not take any
%    room.\footnote{This will lead to problems whenever very high
%    symbols occurring in the line tower into the
%    heading. So, something else has to be done here sometime.}
%    \begin{macrocode}
       \rlap{\vbox{\hbox{\hskip \labelsep\theorem@headerfont ##1\ ##2}%
                   \hbox{\strut}}}]}%
%    \end{macrocode}
%    Again, the definition of |\@opargbegintheorem| is
%    completely analogous.
%    \begin{macrocode}
\def\@opargbegintheorem##1##2##3{%
  \item[\rlap{\vbox{\hbox{\hskip \labelsep \theorem@headerfont
                     ##1\ ##2\ (##3)}%
                    \hbox{\strut}}}]}}
\endgroup
%</thb>
%    \end{macrocode}
% \end{macro}
%
%
% \subsubsection{The \texttt{changebreak} style}
%
%    \begin{macrocode}
% This style option is stored in the file |thcb.sty|.
%    \begin{macrocode}
%<*thcb>
\begingroup \makeatletter
\@ifundefined{theorem@style}{% \iffalse meta-comment
%
% Copyright (C) 1993-2024
%
% The LaTeX Project and any individual authors listed elsewhere
% in this file.
%
% This file is part of the Standard LaTeX `Tools Bundle'.
% -------------------------------------------------------
%
% It may be distributed and/or modified under the
% conditions of the LaTeX Project Public License, either version 1.3c
% of this license or (at your option) any later version.
% The latest version of this license is in
%    https://www.latex-project.org/lppl.txt
% and version 1.3c or later is part of all distributions of LaTeX
% version 2005/12/01 or later.
%
% The list of all files belonging to the LaTeX `Tools Bundle' is
% given in the file `manifest.txt'.
%
% \fi
%
%%
%
%
\def\FMithmInfo{2023/07/05 v2.2c Theorem extension package (FMi)}
%
% \ProvidesFile{theorem.dtx}[\FMithmInfo]
%\iffalse   % this is a METACOMMENT !
%%
%% Package `theorem' to use with LaTeX2e
%% Copyright (C) 1989-1995 Frank Mittelbach, all rights reserved.
%<+package>\@ifundefined{theorem@style}{}{\endinput}
%<+package>\NeedsTeXFormat{LaTeX2e}[1995/06/01]
%<+package>\ProvidesPackage{theorem}
%<+package>         [\FMithmInfo]
%
% \fi
%
% \changes{v1.0}{87/05/07}{First draft.}
% \changes{v1.0a}{87/05/07}{`thistheoremstyle' added.}
% \changes{v1.1a}{87/07/29}{New concept: redefine the internal
%                          \protect\LaTeX{} commands `@xnthm' etc.}
% \changes{v1.1b}{87/07/31}{`theoremfont' used to specify font.}
% \changes{v1.2a}{88/01/08}{`@endtheorem' now globally defined.}
% \changes{v1.2b}{88/01/11}{`newtheorem' only allowed in the preamble.}
% \changes{v1.3a}{88/02/14}{`theoremfont' now a toks register.}
%
% \changes{v2.0a}{88/11/16}{New implementation and new documentation.}
% \changes{v2.0a}{88/11/16}{No layout changes but internal modifications
%                           old substyles won't work.}
% \changes{v2.0b}{89/01/30}{`global' used for all definitions.}
% \changes{v2.0c}{89/06/15}{Documentation translated to English.
%                            Thanks to LLL and Nina}
% \changes{v2.1a}{90/02/15}{Support for new font selection.}
% \changes{v2.2a}{94/02/03}{Cleanup for LaTeX2e}
%
% \DoNotIndex{\@endparenv,\@ifdefinable,\@ifnextchar,\@ifundefined}
% \DoNotIndex{\@namedef,\@nameuse,\@spaces,\@tempa,\@warning}
% \DoNotIndex{\ }
% \DoNotIndex{\begingroup,\bfseries}
% \DoNotIndex{\csname}
% \DoNotIndex{\def,\do,\docdate}
% \DoNotIndex{\endcsname,\endgroup,\endinput,\endtrivlist,\expandafter}
% \DoNotIndex{\filedate,\fileversion}
% \DoNotIndex{\gdef,\global}
% \DoNotIndex{\hbox,\hskip}
% \DoNotIndex{\ignorespaces,\input,\itshape,\item}
% \DoNotIndex{\labelsep,\let,\llap}
% \DoNotIndex{\makeatletter}
% \DoNotIndex{\noexpand}
% \DoNotIndex{\refstepcounter,\relax,\rlap}
% \DoNotIndex{\setlength,\slshape,\space,\string,\strut}
% \DoNotIndex{\the,\trivlist,\typeout}
% \DoNotIndex{\vbox}
% \DoNotIndex{\xdef}
%
% \def\deflabel#1{\ttfamily #1\hfill}
% \def\deflist#1{\begin{list}{}{\settowidth\labelwidth{\ttfamily #1}%
%                               \setlength\leftmargin\labelwidth
%                               \addtolength\leftmargin\labelsep
%                               \let\makelabel\deflabel}}
% \def\enddeflist{\end{list}}
% \def\envfont{\normalfont\ttfamily}
%
% \MakeShortVerb{\|}
%
% \newcommand{\TUB}{{\sl TUGboat\/}}
%
% \GetFileInfo{theorem.dtx}
% \title{An Extension of the
%        \LaTeX{} theorem environment\thanks{This file
%        has version number \fileversion, last
%        revised \filedate.}}
% \author{Frank Mittelbach}
%
% \MaintainedByLaTeXTeam{tools}
% \maketitle
%
% \begin{abstract}
%    The macros described in this paper yield an extension of the
%    \LaTeX{} theorem mechanism. It is designed to satisfy the
%    different requirements of various journals. Thus, the
%    layout of the ``theorems'' can be manipulated by determining a
%    ``style''. This article describes not only the use, but
%    also the definition, of the necessary macros.
% \end{abstract}
%
% \section*{Preface to version 2.2}
%
% For \LaTeXe{} this package did not need any fundamental changes. I
% only modified the messages generated so that theorem layout styles
% will show up with the |\listfiles| command and cleaned the section
% on the New Font Selection Scheme since this is now included in
% \LaTeX.
%
% \section*{Preface to version 2.1}
%
% This version is identical to 2.0g described in \TUB\ 10\#3 except
% for some internal defaults which are now set depending on the used
% font selection scheme.
%
% This was done to avoid unpleasant surprises if the new font selection
% scheme is in force. For further details see
% section~\ref{sec:newfonts} and \cite{bk:GMS94}.
%
% \section{Introduction}
%
% For our purposes here, ``theorems'' are labelled enunciations,
% often set off from the main text by extra space and a font change.
% Theorems, corollaries, conjectures, definitions, and remarks
% are all instances of ``theorems''.  The ``header'' of these structures
% is composed of a label (such as \textsc{Theorem} or \textsc{Remark})
% and a number which serializes an item in the sequence of items
% with the same label.
%
% Shortly after the introduction of \LaTeX{} at the Fachbereich
% Mathematik in Mainz, the desire to manipulate the layout of
% ``theorems'' arose.  In Mainz, the following two conventions came
% into general use:
% \begin{enumerate}
%   \item The number of the theorem is shown in the margin.
%   \item There is a line break at
%         the end of the theorem header.
% \end{enumerate}
%
% Additionally, some journals require different formats which depend on
% the ``sort of theorem'': e.g.\ often remarks and definitions are set
% in |\upshape|, while |\itshape| is employed for main theorems.
%
% Confronted with these requirements, a theorem environment was
% developed in Mainz which allows separate determination of the
% layout of the ``theorems sets'', comparable to |\pagestyle|.
%
%
% \section{The user interface}
% \subsection{Defining new theorem sets}
%
% \DescribeMacro\newtheorem
% As in the original \LaTeX{} version, the command |\newtheorem|
% defines a new ``theorem set'' or ``theorem-like structure''.
% Two required arguments name the new environment and give the text
% to be typeset with each instance of the new ``set'', while
% an optional argument determines how
% the ``set'' is enumerated:
% \begin{description}
%    \item[\ttfamily \bslash newtheorem\{foo\}\{bar\}]
%       The theorem set {\envfont foo} (whose name is \texttt{bar})
%       uses its own counter.
%    \item[\ttfamily \bslash newtheorem\{foo2\}{[foo]}\{bar2\}]
%       The theorem set {\envfont foo2} (printed name \texttt{bar2})
%       uses the same counter as the theorem set \texttt{foo}.
%    \item[\ttfamily \bslash newtheorem\{foo3\}\{bar3\}{[section]}]
%       The theorem set {\envfont foo3} (printed name \texttt{bar3}) is
%       enumerated within the counter \texttt{section}, i.e.\ with every
%       new |\section| the enumeration begins again with 1, and
%       the enumeration is composed from the section-number and the
%       theorem counter itself.
% \end{description}
%
% \DescribeMacro\theoremstyle
% Additionally, the command |\theoremstyle| can define the layout
% of various, or all, theorem sets.  It should be noted that any
% theorem set defined by |\newtheorem| is typeset in the
% |\theoremstyle| that is current at the time of the definition.
% Thus, the following
% \begin{verbatim}
%    \theoremstyle{break}        \newtheorem{Cor}{Corollary}
%    \theoremstyle{plain}        \newtheorem{Exa}{Example}[section]
%\end{verbatim}                             ^^A No blank at line start
% leads to the result that the set {\envfont Cor} is formatted in the
% style \texttt{break}, while the set {\envfont Exa} and all the
% following ones are formatted in the style \texttt{plain}, unless
% another |\theoremstyle| follows.  Since the definitions installed by
% |\newtheorem| are global, one also can limit |\theoremstyle| locally
% by grouping braces.
%
% \DescribeMacro\theorembodyfont
% The choice of the font for the theorem body is completely
% independent of the chosen |\theoremstyle|; this has proven to
% be very advantageous.  For example,
% \begin{verbatim}
%   {\theorembodyfont{\upshape}          \newtheorem{Rem}{Remark}}
%\end{verbatim}
% defines a theorem set {\envfont Rem}, which will be set in
% |\upshape| in the current layout (which in our example is
% \texttt{plain}).  As with |\theoremstyle|, the |\theorembodyfont|
% chosen is that current at the time of |\newtheorem|.  If
% |\theorembodyfont| is not specified or one defines
% |\theorembodyfont{}|, then the font used will be that defined by the
% |\theoremstyle|.
%
% \DescribeMacro\theoremheaderfont It is also possible to customize
% the font used for the theorem headers.  This is, however, a global
% declaration, and therefore there should be at most one
% |\theoremheaderfont| declaration in the preamble.\footnote{If it is
% actually necessary to have different header fonts, one has to define
% new theorem styles (substituting the desired font) or specify the
% information directly in the \texttt{\bslash newtheorem} declaration
% (the unclean variant).}
%
% \DescribeMacro\theorempreskipamount
% \DescribeMacro\theorempostskipamount Two additional parameters
% affect the vertical space around the theorem environments:
% |\theorempreskipamount| and |\theorempostskipamount| define,
% respectively, the spacing before and after such an environment.
% These parameters apply for all theorem sets and can be manipulated
% with the ordinary length macros.  They are rubber lengths
% (`\textsf{skips}'), and therefore can contain \texttt{plus} and
% \texttt{minus} parts.
%
% Since the definition of theorem sets should---most sensibly---be
% placed in the preamble, we only allow installation there.
% It is therefore possible to release the memory used here after
% |\begin{document}|, in order to make room for other
% applications.
%
%
% \subsection{Existing theorem styles}
%
% The following theorem styles exist to date:
% \begin{deflist}{marginbreak}
%    \item[plain]
%       This theorem style emulates the original \LaTeX{} definition,
%       except that additionally the parameters
%       |\theorem...skipamount| are used.
%    \item[break]
%       In this style, the theorem header is followed by a line
%       break.
%    \item[marginbreak]
%       The theorem number is set in the margin, and there is a line
%       break as in \texttt{break}.
%    \item[changebreak]
%       Like \texttt{break}, but with header number and text
%       interchanged.
%    \item[change]
%       Header number and text are interchanged, without a line break.
%    \item[margin]
%       The number is set in the left margin, without a line break.
% \end{deflist}
% All styles (except \texttt{plain}) select |\slshape| as the default
% |\theorembodyfont|.
%
%
% \subsection{Examples}
%
% Given the above theorem sets {\envfont Cor}, {\envfont Exa}
% and {\envfont Rem},
% suppose that the preamble also contains the declarations:
% \begin{verbatim}
%    \theoremstyle{marginbreak}   \newtheorem{Lem}[Cor]{Lemma}
%    \theoremstyle{change}
%    \theorembodyfont{\itshape}        \newtheorem{Def}[Cor]{Definition}
%
%    \theoremheaderfont{\scshape}
%\end{verbatim}
% Then the following are some typical examples of the typeset output
% resulting from their use.
%
% \begin{Cor}
%    This is a sentence typeset in the theorem environment {\envfont
%    Cor}.
% \end{Cor}
%
% \begin{Exa}
%    This is a sentence typeset in the theorem environment {\envfont
%    Exa}.
% \end{Exa}
%
% \begin{Rem}
%    This is a sentence typeset in the theorem environment {\envfont
%    Rem}.
% \end{Rem}
%
% \begin{Lem}[Ben User]
%    This is a sentence typeset in the theorem environment {\envfont
%    Lem}.
% \end{Lem}
%
% \begin{Def}[Very impressive Definition]
%    This is a sentence typeset in the theorem environment {\envfont
%    Def}.
% \end{Def}
%
% The last two examples show the effect of the optional argument to
% a theorem environment (it is the text typeset in parentheses).
%
%
% \section{Special Considerations}
% \label{sec:newfonts}
%
% Theoremheader and body are implemented as a unit. This means that
% the |\theoremheaderfont|  will inherit characteristics of the
% |\theorembodyfont| in \LaTeXe. Thus, if for example
% |\theorembodyfont| is |\itshape| and |\theoremheaderfont|
% is |\bfseries| the font selected for the header will have the
% characteristics `bold extended italic'. If this is not desired one
% should set the |\theoremheaderfont| to something like
% \begin{verbatim}
%      \theoremheaderfont{\normalfont\bfseries}
%\end{verbatim}
% i.e.\ supplying all necessary font information explicitly.
%
%
%
% \section{Acknowledgements}
%
% The publication of this set of macros was only possible with the
% help of Christina Busse (translating the manuscript into English),
% Joachim Pense (playing the r\^ole of typist), Chris Rowley
% (looking everything over) and many others providing useful
% suggestions.
%
% \MaybeStop{%
%    \begin{thebibliography}{9}
%       \bibitem{bk:GMS94} \textsc{M.~Goossens}, \textsc{F.~Mittelbach}
%         and \textsc{A.~Samarin}.
%         \newblock The \LaTeX{} Companion.
%         \newblock
%         Addison-Wesley, Reading, Massachusetts, 1994.
%       \bibitem{latex:source} \textsc{Lamport, Leslie.} \newblock
%                               \texttt{latex.tex}, version 2.09, date
%                               Feb. 1990.
%    \end{thebibliography}
%    \PrintIndex
%    \PrintChanges}
%
%
% \section{The documentation driver file}
%
% The next bit of code contains the documentation driver file for
% \TeX{}, i.e., the file that will produce the documentation you are
% currently reading. It will be extracted from this file by the
% \texttt{docstrip} program. Since it is the first code in the file
% one can alternatively process this file directly with \LaTeXe{} to
% obtain the documentation.
%
%    \begin{macrocode}
%<*driver>
\documentclass{ltxdoc}

\usepackage{theorem}

%<+driver>% The next few lines define theorem sets which are used
%<+driver>% in the example section of the documentation.

\theoremstyle{break}       \newtheorem{Cor}{Corollary}
\theoremstyle{plain}       \newtheorem{Exa}{Example}[section]
{\theorembodyfont{\upshape}\newtheorem{Rem}{Remark}}
\theoremstyle{marginbreak} \newtheorem{Lem}[Cor]{Lemma}
\theoremstyle{change}
\theorembodyfont{\itshape} \newtheorem{Def}[Cor]{Definition}

\theoremheaderfont{\scshape}

\RecordChanges

\begin{document}
   \DocInput{theorem.dtx}
\end{document}
%</driver>
%    \end{macrocode}
%
% \section{Definition of the Macros}
%
% If the file has been loaded before, we abort immediately. If not the
% package announces itself (this is actually done at the very top of
% the file---the way it is done isn't good style so don't copy it).
% \changes{v2.0e}{89/07/19}{Spaces removed from `typeout'.}
% \changes{v2.2b}{95/11/19}{Announce moved to top of file}
%    \begin{macrocode}
%<*package>
%\@ifundefined{theorem@style}{}{\endinput}
%\def\FMithmInfo{1995/11/19 v2.2b Theorem extension package (FMi)}
%\ProvidesPackage{theorem}[\FMithmInfo]
%    \end{macrocode}
%
%
% \subsection{Definition of theorem styles and fonts}
%
% All the definitions in this file are done globally to allow
% inputting this file inside a group.
%
% \begin{macro}{\theoremstyle}
%    Before a theorem style can be installed, the chosen style must be
%    known. For that reason, we must test to see that
%    |\th@|\meta{style} is known or, more precisely, that it is
%    different from |\relax|. If the style is not known then
%    |\th@plain| is used.
%    \begin{macrocode}
\gdef\theoremstyle#1{%
   \@ifundefined{th@#1}{\@warning
          {Unknown theoremstyle `#1'. Using `plain'}%
          \theorem@style{plain}}%
%    \end{macrocode}
%    We save the theorem style to be used in the token register
%    |\theorem@style|.
%    \begin{macrocode}
      {\theorem@style{#1}}%
%    \end{macrocode}
%    Now we ``evaluate'' the theorem style: this means, we call the
%    macro |\th@|\meta{style} which will activate the relevant
%    definitions which are contained in a separate file.  This is done
%    in a group to suppress changes to the current font.  This could
%    otherwise pose problems together with the new font selection
%    scheme\footnote{When I printed the original article using the new
%    font selection scheme I ended with a document with slanted
%    typefaces (text headings and all) simply because one of the
%    theorem styles used \texttt{\bslash sl} at toplevel.} if the
%    |\th@|\meta{style} is evaluated a second time.
% \changes{v2.1a}{90/02/15}{Group added around `th@..}
% \changes{v2.1a}{90/02/15}{Reseting of `..@begintheorem removed.}
%    \begin{macrocode}
      \begingroup
        \csname th@\the\theorem@style \endcsname
      \endgroup}
%    \end{macrocode}
% \end{macro}
%
%
% \begin{macro}{\@begintheorem}
% \changes{v2.1a}{90/02/15}{Relaxed at toplevel.}
% \begin{macro}{\@opargbegintheorem}
% \changes{v2.1a}{90/02/15}{Relaxed at toplevel.}
%    We reset |\@begintheorem| and |\@opargbegintheorem| to
%    |\relax| since these commands are no longer necessary at
%    toplevel. This will save a few tokens.
%    \begin{macrocode}
\global\let\@begintheorem\relax
\global\let\@opargbegintheorem\relax
%    \end{macrocode}
% \end{macro}
% \end{macro}
%
% \begin{macro}{\theorem@style}
%    Obviously the \textsf{token} register used above has to be
%    allocated. To assure the utmost compatibility with the
%    original \LaTeX{} definition, we set the default theorem style to
%    \texttt{plain}, which implements the usual \LaTeX{} convention.
% \changes{v2.0f}{89/07/20}{Unnecessary `global' removed.}
%    \begin{macrocode}
\newtoks\theorem@style
\global\theorem@style{plain}
%    \end{macrocode}
% \end{macro}
%
% \begin{macro}{\theorembodyfont}
% \begin{macro}{\theorem@bodyfont}
%    If the |\theorembodyfont| is set by the user then it should not
%    interact with the default font set in the theorem style. When the
%    new font selection is in force this may happen if, for example,
%    the default is |\itshape| and the new |\theorembodyfont| is
%    |\sffamily|. So we add a |\reset@font| command in front of the
%    user definition.
% \changes{v2.0f}{89/07/20}{Unnecessary `global' removed.}
% \changes{v2.0g}{89/09/19}{`theoremfont to `theorembodyfont.}
% \changes{v2.1c}{92/06/22}{`reset@font added and logic changed.}
% \changes{v2.1d}{92/10/28}{Reset to empty if arg is empty}
%    \begin{macrocode}
\gdef\theorembodyfont#1{%
%    \end{macrocode}
%    We check if the argument supplied is empty and if so put nothing
%    into the |\theorem@bodyfont| token register to allow for
%    |\theorembodyfont{}| as a mean of using the default of the
%    current |\theoremstyle|.
%    \begin{macrocode}
   \def\@tempa{#1}%
   \ifx\@tempa\@empty
    \theorem@bodyfont{}%
   \else
    \theorem@bodyfont{\reset@font#1}%
   \fi
   }
\newtoks\theorem@bodyfont
\global\theorem@bodyfont{}
%    \end{macrocode}
% \end{macro}
% \end{macro}
%
%
% \begin{macro}{\theoremheaderfont}
% \changes{v2.0g}{89/09/19}{Macro added.}
%    The font for the theorem headers is handled differently because
%    this definition applies to all theorem styles.
%    \begin{macrocode}
\gdef\theoremheaderfont#1{\gdef\theorem@headerfont{#1}%
%    \end{macrocode}
%    After using the macro once it is redefined to produce an error
%    message.
%    \begin{macrocode}
       \gdef\theoremheaderfont##1{%
        \typeout{\string\theoremheaderfont\space should be used
                 only once.}}}
%    \end{macrocode}
% \end{macro}
%
% \begin{macro}{\theorem@headerfont}
% \changes{v2.0g}{89/09/19}{Macro added.}
%    To set the |\theorem@headerfont| default we first test if
%    the new fontselection scheme is in force.
% \changes{v2.1a}{90/02/15}{Support for new font selection.}
%    \begin{macrocode}
\ifx\upshape\undefined
%    \end{macrocode}
%    If not we define it to expand into |\bfseries|. We don't use
%    |\let| just in case a following style option redefines this
%    macro.
%    \begin{macrocode}
\gdef\theorem@headerfont{\bfseries}
%    \end{macrocode}
%    Otherwise we reset the current shape before calling |\bfseries|.
%    \begin{macrocode}
\else \gdef\theorem@headerfont{\normalfont\bfseries}\fi
%    \end{macrocode}
% \end{macro}
%
% \begin{macro}{\th@plain}
% \begin{macro}{\th@break}
% \begin{macro}{\th@marginbreak}
% \begin{macro}{\th@changebreak}
% \begin{macro}{\th@change}
% \begin{macro}{\th@margin}
%    The different styles are defined in macros such as
%    |\th@plain|.  Since memory space is precious in
%    ``non-Big-versions'', we have to avoid offering too many unused
%    definitions.  Therefore we define these styles in separate files
%    that can be loaded on demand.  Thus the commands themselves only
%    load these files. We use |\@input@|, a \LaTeXe{} internal command
%    that ensures that the file will be listed with |\listfiles|.
% \changes{v2.2a}{94/02/02}{Use `@input@ to load theorem layout files}
%    \begin{macrocode}
\gdef\th@plain{\@input@{thp.sty}}
\gdef\th@break{\@input@{thb.sty}}
\gdef\th@marginbreak{\@input@{thmb.sty}}
\gdef\th@changebreak{\@input@{thcb.sty}}
\gdef\th@change{\@input@{thc.sty}}
\gdef\th@margin{\@input@{thm.sty}}
%    \end{macrocode}
%    This list will be expanded when new styles become
%    available.  For testing,  just append new theorem substyles
%    as document options.
% \end{macro}
% \end{macro}
% \end{macro}
% \end{macro}
% \end{macro}
% \end{macro}
%
%
%
% \subsection{Definition of a new theorem set}
%
% As already pointed out, a new theorem environment can be defined in
% three different ways:
% \begin{verbatim}
%    \newtheorem{Lem}{Lemma}
%    \newtheorem{Lem}{Lemma}[section]
%    \newtheorem{Lem}[Theorem]{Lemma}
%\end{verbatim}
% The function of the macro |\newtheorem| is to recognize these cases
% and then to branch into one of the three macros |\@ynthm|, |\@xnthm|
% or |\@othm|.  This mechanism is adopted unchanged from
% \cite{latex:source}; the essential point here is that, for example,
% in the second case, the arguments \texttt{Lem}, \texttt{Lemma} and
% \texttt{section} are passed over to the macro |\@xnthm|.
%
% We inspect this case first because the others present fewer problems,
% and thus are easily derived from this one.
%
% \begin{macro}{\@xnthm}
%    For our example arguments, the macro |\@xnthm| must fulfill
%    the following:
%    \begin{itemize}
%       \item
%          Define a new \LaTeX-counter `Lem'
%       \item
%          reset this counter within a |\section|
%       \item
%          define the macro |\theLem|
%       \item
%          define the environment macros |\Lem| and
%          |\endLem| using the current
%          |\theoremstyle| and |\theorem@bodyfont|.
%    \end{itemize}
%    Obviously, all this should happen only if the first argument
%    of |\@xnthm| (i.e.\ \texttt{Lem} in our example) is chosen so
%    as not to conflict with any previously defined commands or
%    environments.  This test is performed by the \LaTeX{} macro
%    |\@ifdefinable|.
%    \begin{macrocode}
\gdef\@xnthm#1#2[#3]{\expandafter\@ifdefinable\csname #1\endcsname
%    \end{macrocode}
%    Therefore, the first argument of |\@ifdefinable| is the
%    expansion (in the example, |\Lem|) of
%    |\csname||#1\endcsname|.  The second argument is
%    executed only if the test has been completed successfully.
%    \begin{macrocode}
   {%
%    \end{macrocode}
%    Now we define the new counter. The names of the \LaTeX{}
%    macros employed should speak for themselves:
% \changes{v2.2b}{1995/10/07}{Use \cs{@newctr}}
% \changes{v2.2c}{1995/11/23}{Use \cs{@newctr} properly.}
%    \begin{macrocode}
    \@definecounter{#1}\@newctr{#1}[#3]%
%    \end{macrocode}
%    Using |\@newctr| will give a proper error message if the counter
%    in |#3| is not defined.
%    In defining `|\theLem|' we must generate the desired macro
%    name by use of |\expandafter| and |\csname|.
%    \begin{macrocode}
    \expandafter\xdef\csname the#1\endcsname
%    \end{macrocode}
%    An |\xdef| is used in order to make the definition global,
%    and to ensure that it contains the replacement texts of
%    |\@thmcountersep| and |\@thmcounter|.\footnote{These
%    two macros can be defined by the document style.  Their default
%    values produce a `\texttt{.}' as separation and an arabic
%    representation of the number.} However, not everything should be
%    expanded. For example, it saves space to use |\thesection|
%    instead of its---at times---lengthy expansion.
%    \begin{macrocode}
      {\expandafter \noexpand \csname the#3\endcsname
       \@thmcountersep \@thmcounter{#1}}%
%    \end{macrocode}
%    Thus with the defaults of \LaTeX{}, |\theLem| would be
%    replaced by the command sequence |\thesection.\arabic{Lem}|.
%
%    We will now look at the definition of the macro which is executed
%    at the beginning of the actual environment (in our example this
%    macro is |\Lem|). It should be noted that we use an
%    ``|\expandafter| trick'' to expand only certain parts of the
%    replacement text at the time of the definition.
%    \begin{macrocode}
    \def\@tempa{\global\@namedef{#1}}%
    \expandafter \@tempa \expandafter{%
%    \end{macrocode}
%    First, the macro that contains the current definitions of
%    |\@begintheorem| and |\@opargtheorem| should be called
%    up.  The name of this macro---as is already known---has the form
%    |\th@|\meta{theorem\ style}; therefore, it must be called by
%    \begin{macrocode}
      \csname th@\the \theorem@style
%    \end{macrocode}
%    In addition the default theorem font should be changeable, i.e.\
%    we have to insert the contents of |\theorem@bodyfont|. For
%    that reason, we expand even further, beyond |\endcsname|,
%    and thus insert the contents of the \textsf{token} register
%    |\theorem@bodyfont| in the replacement text.
%    \begin{macrocode}
            \expandafter \endcsname \the \theorem@bodyfont
%    \end{macrocode}
%    Now it is time to call the macro |\@thm| which takes over
%    the further processing.  It has two arguments: the current
%    counter name (in our example, \texttt{Lem}), and the text of the
%    label (in our example, \texttt{Lemma}).
%    \begin{macrocode}
     \@thm{#1}{#2}}%
%    \end{macrocode}
%    With this, the `sub-definition' is complete.  The macro
%    |\@endtheorem| ends a theorem environment and is, so far, nothing
%    but an |\endtrivlist|.  (Hence it is defined globally, and not
%    within the theorem styles.\footnote{This has to be changed as
%    soon as theorem styles that change \texttt{\bslash @endtheorem}
%    exist.  In such a case, all existing styles must be changed as
%    well since they will have to reset the macro.}) Therefore, we can
%    set it equivalent to the macro that ends the theorem set (in our
%    example, |\endLem|).  However, if some day theorem styles exist
%    that do change |\@endtheorem|, we would have to use the
%    commented-out line instead.
%    \begin{macrocode}
    \global \expandafter \let \csname end#1\endcsname \@endtheorem
%   \global\@namedef{end#1}{\@endtheorem}%
%    \end{macrocode}
%    With these commands all the required definitions are employed,
%    unless the test |\@ifdefinable| has failed.  Therefore, we
%    end the second argument of this macro and with it the definition
%    of |\@xnthm|.
%    \begin{macrocode}
   }}
%    \end{macrocode}
% \end{macro}
%
% \begin{macro}{\@ynthm}
%    The definition of |\@ynthm| is completely analogous. In
%    this case the new counter that is defined is not reset within
%    another counter; thus the definition of |\the...|
%    is simplified:
%    \begin{macrocode}
\gdef\@ynthm#1#2{\expandafter\@ifdefinable\csname #1\endcsname
   {\@definecounter{#1}%
    \expandafter\xdef\csname the#1\endcsname{\@thmcounter{#1}}%
%    \end{macrocode}
%    The rest of the definition corresponds literally to that of
%    |\@xnthm|:
%    \begin{macrocode}
    \def\@tempa{\global\@namedef{#1}}\expandafter \@tempa
     \expandafter{\csname th@\the \theorem@style \expandafter
     \endcsname \the\theorem@bodyfont \@thm{#1}{#2}}%
    \global \expandafter \let \csname end#1\endcsname \@endtheorem}}
%    \end{macrocode}
% \end{macro}
%
% \begin{macro}{\@othm}
%    The definition of |\@othm| does not contain anything new.
%    \begin{macrocode}
\gdef\@othm#1[#2]#3{%
%    \end{macrocode}
% \changes{v2.2b}{95/10/07}{Build in check for counter defined}
%    We do not define a new counter but instead use one that has
%    already been defined.  Thus the only definition we need is that
%    of this pseudo-counter (i.e.\ |\the|\meta{env.\ name}).
%    First we check if |#2| corresponds to a known counter name.
%    \begin{macrocode}
  \expandafter\ifx\csname c@#2\endcsname\relax
   \@nocounterr{#2}%
  \else
   \expandafter\@ifdefinable\csname #1\endcsname
   {\expandafter \xdef \csname the#1\endcsname
     {\expandafter \noexpand \csname the#2\endcsname}%
%    \end{macrocode}
%    All other parts of the definition can be adopted from
%    |\@xnthm|. We have to remember, though, that in this case
%    the name of the current counter and the theorem label have moved
%    to the second and third arguments.
%    \begin{macrocode}
    \def\@tempa{\global\@namedef{#1}}\expandafter \@tempa
     \expandafter{\csname th@\the \theorem@style \expandafter
     \endcsname \the\theorem@bodyfont \@thm{#2}{#3}}%
    \global \expandafter \let \csname end#1\endcsname \@endtheorem}%
  \fi}
%    \end{macrocode}
% \end{macro}
%
% \subsection{Macros that are employed in a {\envfont theorem}
%    environment}
%
% \begin{macro}{\@thm}
%    The macro |\@thm| has to increase the current counter.  Then,
%    depending on whether the environment has (or does not have) an
%    optional argument, it has to branch into either |\@begintheorem|
%    or |\@opargtheorem|.
%    \begin{macrocode}
\gdef\@thm#1#2{\refstepcounter{#1}%
%    \end{macrocode}
%    Now we start a {\envfont trivlist} environment, and give
%    |\@topsep| and |\@topsepadd| the values of the \textsf{skip}
%    registers |\theorempreskipamount| and |\theorempostskipamount|.
%    The value in |\@topsep| is the vertical space that is inserted by
%    the first (and only) |\item| in our |\trivlist| whilst
%    |\@topsepadd| is inserted by |\@endparenv| at the end of that
%    {\envfont trivlist} environment.  By using these registers, we
%    obtain the desired space around a {\envfont theorem} environment.
%    \begin{macrocode}
   \trivlist
   \@topsep \theorempreskipamount               % used by first \item
   \@topsepadd \theorempostskipamount           % used by \@endparenv
%    \end{macrocode}
%    Now we have to test whether an optional argument has been given.
% \changes{v2.0d}{89/07/18}{Forgotten percent added.}
%    \begin{macrocode}
   \@ifnextchar [%
%    \end{macrocode}
%    If there is an optional argument, we will call |\@ythm|, and
%    move the arguments read back into the input stream.
%    \begin{macrocode}
   {\@ythm{#1}{#2}}%
%    \end{macrocode}
%    If not, we call |\@begintheorem|. Its first argument is the
%    name of the theorem set (hence the second argument of
%    |\@thm|). Its second argument is the macro that produces the
%    current number.
%    \begin{macrocode}
   {\@begintheorem{#2}{\csname the#1\endcsname}\ignorespaces}}
%    \end{macrocode}
% \end{macro}
%
% \begin{macro}{\@xthm}
% \begin{macro}{\@ythm}
%    Both these macros were originally called by |\@thm|.  We do
%    not need |\@xthm| anymore, hence we reset it to
%    |\relax|. The definition of |\@ythm| has not changed at
%    all from its definition in \LaTeX. In order to make the macros
%    easier to understand, we will
%    nevertheless present it (commented out).
%    \begin{macrocode}
\global\let\@xthm\relax
% \def\@ythm#1#2[#3]{\@opargbegintheorem{#2}{\csname
%           the#1\endcsname}{#3}\ignorespaces}
%</package>
%    \end{macrocode}
%    The primitive |\ignorespaces| in |\@ythm| and
%    |\@thm| is needed to remove the spaces between the
%    |\begin{...}| and the actual text.
% \end{macro}
% \end{macro}
%
%
% \subsection{Definition of the theorem substyles}
%
% As already pointed out, the theorem substyles, defined below, are
% only loaded when necessary.  Note that all these substyles, except
% \texttt{plain}, have |\slshape| as the default body font.
%
% \subsubsection{The \texttt{plain} style}
%
% \label{dok:th@plain} As the following macros use |@|, we have
% to locally set the |\catcode| of this symbol to ``letter''.
% This happens within a group, so that we do not have to worry about
% which |\catcode| that symbol had before.
%    \begin{macrocode}
%<*thp>
\begingroup \makeatletter
%    \end{macrocode}
% Since we are now within a group, we must make all definitions
% globally.  First we make sure that \texttt{theorem.sty} is loaded.
% This will allow us to use this file as a document style option
% without having to call \texttt{theorem} itself as an option.  At the
% same time, we assure that at least version 2 is loaded, since
% |\theorem@style| was not defined in earlier versions.
%    \begin{macrocode}
\@ifundefined{theorem@style}{% \iffalse meta-comment
%
% Copyright (C) 1993-2024
%
% The LaTeX Project and any individual authors listed elsewhere
% in this file.
%
% This file is part of the Standard LaTeX `Tools Bundle'.
% -------------------------------------------------------
%
% It may be distributed and/or modified under the
% conditions of the LaTeX Project Public License, either version 1.3c
% of this license or (at your option) any later version.
% The latest version of this license is in
%    https://www.latex-project.org/lppl.txt
% and version 1.3c or later is part of all distributions of LaTeX
% version 2005/12/01 or later.
%
% The list of all files belonging to the LaTeX `Tools Bundle' is
% given in the file `manifest.txt'.
%
% \fi
%
%%
%
%
\def\FMithmInfo{2023/07/05 v2.2c Theorem extension package (FMi)}
%
% \ProvidesFile{theorem.dtx}[\FMithmInfo]
%\iffalse   % this is a METACOMMENT !
%%
%% Package `theorem' to use with LaTeX2e
%% Copyright (C) 1989-1995 Frank Mittelbach, all rights reserved.
%<+package>\@ifundefined{theorem@style}{}{\endinput}
%<+package>\NeedsTeXFormat{LaTeX2e}[1995/06/01]
%<+package>\ProvidesPackage{theorem}
%<+package>         [\FMithmInfo]
%
% \fi
%
% \changes{v1.0}{87/05/07}{First draft.}
% \changes{v1.0a}{87/05/07}{`thistheoremstyle' added.}
% \changes{v1.1a}{87/07/29}{New concept: redefine the internal
%                          \protect\LaTeX{} commands `@xnthm' etc.}
% \changes{v1.1b}{87/07/31}{`theoremfont' used to specify font.}
% \changes{v1.2a}{88/01/08}{`@endtheorem' now globally defined.}
% \changes{v1.2b}{88/01/11}{`newtheorem' only allowed in the preamble.}
% \changes{v1.3a}{88/02/14}{`theoremfont' now a toks register.}
%
% \changes{v2.0a}{88/11/16}{New implementation and new documentation.}
% \changes{v2.0a}{88/11/16}{No layout changes but internal modifications
%                           old substyles won't work.}
% \changes{v2.0b}{89/01/30}{`global' used for all definitions.}
% \changes{v2.0c}{89/06/15}{Documentation translated to English.
%                            Thanks to LLL and Nina}
% \changes{v2.1a}{90/02/15}{Support for new font selection.}
% \changes{v2.2a}{94/02/03}{Cleanup for LaTeX2e}
%
% \DoNotIndex{\@endparenv,\@ifdefinable,\@ifnextchar,\@ifundefined}
% \DoNotIndex{\@namedef,\@nameuse,\@spaces,\@tempa,\@warning}
% \DoNotIndex{\ }
% \DoNotIndex{\begingroup,\bfseries}
% \DoNotIndex{\csname}
% \DoNotIndex{\def,\do,\docdate}
% \DoNotIndex{\endcsname,\endgroup,\endinput,\endtrivlist,\expandafter}
% \DoNotIndex{\filedate,\fileversion}
% \DoNotIndex{\gdef,\global}
% \DoNotIndex{\hbox,\hskip}
% \DoNotIndex{\ignorespaces,\input,\itshape,\item}
% \DoNotIndex{\labelsep,\let,\llap}
% \DoNotIndex{\makeatletter}
% \DoNotIndex{\noexpand}
% \DoNotIndex{\refstepcounter,\relax,\rlap}
% \DoNotIndex{\setlength,\slshape,\space,\string,\strut}
% \DoNotIndex{\the,\trivlist,\typeout}
% \DoNotIndex{\vbox}
% \DoNotIndex{\xdef}
%
% \def\deflabel#1{\ttfamily #1\hfill}
% \def\deflist#1{\begin{list}{}{\settowidth\labelwidth{\ttfamily #1}%
%                               \setlength\leftmargin\labelwidth
%                               \addtolength\leftmargin\labelsep
%                               \let\makelabel\deflabel}}
% \def\enddeflist{\end{list}}
% \def\envfont{\normalfont\ttfamily}
%
% \MakeShortVerb{\|}
%
% \newcommand{\TUB}{{\sl TUGboat\/}}
%
% \GetFileInfo{theorem.dtx}
% \title{An Extension of the
%        \LaTeX{} theorem environment\thanks{This file
%        has version number \fileversion, last
%        revised \filedate.}}
% \author{Frank Mittelbach}
%
% \MaintainedByLaTeXTeam{tools}
% \maketitle
%
% \begin{abstract}
%    The macros described in this paper yield an extension of the
%    \LaTeX{} theorem mechanism. It is designed to satisfy the
%    different requirements of various journals. Thus, the
%    layout of the ``theorems'' can be manipulated by determining a
%    ``style''. This article describes not only the use, but
%    also the definition, of the necessary macros.
% \end{abstract}
%
% \section*{Preface to version 2.2}
%
% For \LaTeXe{} this package did not need any fundamental changes. I
% only modified the messages generated so that theorem layout styles
% will show up with the |\listfiles| command and cleaned the section
% on the New Font Selection Scheme since this is now included in
% \LaTeX.
%
% \section*{Preface to version 2.1}
%
% This version is identical to 2.0g described in \TUB\ 10\#3 except
% for some internal defaults which are now set depending on the used
% font selection scheme.
%
% This was done to avoid unpleasant surprises if the new font selection
% scheme is in force. For further details see
% section~\ref{sec:newfonts} and \cite{bk:GMS94}.
%
% \section{Introduction}
%
% For our purposes here, ``theorems'' are labelled enunciations,
% often set off from the main text by extra space and a font change.
% Theorems, corollaries, conjectures, definitions, and remarks
% are all instances of ``theorems''.  The ``header'' of these structures
% is composed of a label (such as \textsc{Theorem} or \textsc{Remark})
% and a number which serializes an item in the sequence of items
% with the same label.
%
% Shortly after the introduction of \LaTeX{} at the Fachbereich
% Mathematik in Mainz, the desire to manipulate the layout of
% ``theorems'' arose.  In Mainz, the following two conventions came
% into general use:
% \begin{enumerate}
%   \item The number of the theorem is shown in the margin.
%   \item There is a line break at
%         the end of the theorem header.
% \end{enumerate}
%
% Additionally, some journals require different formats which depend on
% the ``sort of theorem'': e.g.\ often remarks and definitions are set
% in |\upshape|, while |\itshape| is employed for main theorems.
%
% Confronted with these requirements, a theorem environment was
% developed in Mainz which allows separate determination of the
% layout of the ``theorems sets'', comparable to |\pagestyle|.
%
%
% \section{The user interface}
% \subsection{Defining new theorem sets}
%
% \DescribeMacro\newtheorem
% As in the original \LaTeX{} version, the command |\newtheorem|
% defines a new ``theorem set'' or ``theorem-like structure''.
% Two required arguments name the new environment and give the text
% to be typeset with each instance of the new ``set'', while
% an optional argument determines how
% the ``set'' is enumerated:
% \begin{description}
%    \item[\ttfamily \bslash newtheorem\{foo\}\{bar\}]
%       The theorem set {\envfont foo} (whose name is \texttt{bar})
%       uses its own counter.
%    \item[\ttfamily \bslash newtheorem\{foo2\}{[foo]}\{bar2\}]
%       The theorem set {\envfont foo2} (printed name \texttt{bar2})
%       uses the same counter as the theorem set \texttt{foo}.
%    \item[\ttfamily \bslash newtheorem\{foo3\}\{bar3\}{[section]}]
%       The theorem set {\envfont foo3} (printed name \texttt{bar3}) is
%       enumerated within the counter \texttt{section}, i.e.\ with every
%       new |\section| the enumeration begins again with 1, and
%       the enumeration is composed from the section-number and the
%       theorem counter itself.
% \end{description}
%
% \DescribeMacro\theoremstyle
% Additionally, the command |\theoremstyle| can define the layout
% of various, or all, theorem sets.  It should be noted that any
% theorem set defined by |\newtheorem| is typeset in the
% |\theoremstyle| that is current at the time of the definition.
% Thus, the following
% \begin{verbatim}
%    \theoremstyle{break}        \newtheorem{Cor}{Corollary}
%    \theoremstyle{plain}        \newtheorem{Exa}{Example}[section]
%\end{verbatim}                             ^^A No blank at line start
% leads to the result that the set {\envfont Cor} is formatted in the
% style \texttt{break}, while the set {\envfont Exa} and all the
% following ones are formatted in the style \texttt{plain}, unless
% another |\theoremstyle| follows.  Since the definitions installed by
% |\newtheorem| are global, one also can limit |\theoremstyle| locally
% by grouping braces.
%
% \DescribeMacro\theorembodyfont
% The choice of the font for the theorem body is completely
% independent of the chosen |\theoremstyle|; this has proven to
% be very advantageous.  For example,
% \begin{verbatim}
%   {\theorembodyfont{\upshape}          \newtheorem{Rem}{Remark}}
%\end{verbatim}
% defines a theorem set {\envfont Rem}, which will be set in
% |\upshape| in the current layout (which in our example is
% \texttt{plain}).  As with |\theoremstyle|, the |\theorembodyfont|
% chosen is that current at the time of |\newtheorem|.  If
% |\theorembodyfont| is not specified or one defines
% |\theorembodyfont{}|, then the font used will be that defined by the
% |\theoremstyle|.
%
% \DescribeMacro\theoremheaderfont It is also possible to customize
% the font used for the theorem headers.  This is, however, a global
% declaration, and therefore there should be at most one
% |\theoremheaderfont| declaration in the preamble.\footnote{If it is
% actually necessary to have different header fonts, one has to define
% new theorem styles (substituting the desired font) or specify the
% information directly in the \texttt{\bslash newtheorem} declaration
% (the unclean variant).}
%
% \DescribeMacro\theorempreskipamount
% \DescribeMacro\theorempostskipamount Two additional parameters
% affect the vertical space around the theorem environments:
% |\theorempreskipamount| and |\theorempostskipamount| define,
% respectively, the spacing before and after such an environment.
% These parameters apply for all theorem sets and can be manipulated
% with the ordinary length macros.  They are rubber lengths
% (`\textsf{skips}'), and therefore can contain \texttt{plus} and
% \texttt{minus} parts.
%
% Since the definition of theorem sets should---most sensibly---be
% placed in the preamble, we only allow installation there.
% It is therefore possible to release the memory used here after
% |\begin{document}|, in order to make room for other
% applications.
%
%
% \subsection{Existing theorem styles}
%
% The following theorem styles exist to date:
% \begin{deflist}{marginbreak}
%    \item[plain]
%       This theorem style emulates the original \LaTeX{} definition,
%       except that additionally the parameters
%       |\theorem...skipamount| are used.
%    \item[break]
%       In this style, the theorem header is followed by a line
%       break.
%    \item[marginbreak]
%       The theorem number is set in the margin, and there is a line
%       break as in \texttt{break}.
%    \item[changebreak]
%       Like \texttt{break}, but with header number and text
%       interchanged.
%    \item[change]
%       Header number and text are interchanged, without a line break.
%    \item[margin]
%       The number is set in the left margin, without a line break.
% \end{deflist}
% All styles (except \texttt{plain}) select |\slshape| as the default
% |\theorembodyfont|.
%
%
% \subsection{Examples}
%
% Given the above theorem sets {\envfont Cor}, {\envfont Exa}
% and {\envfont Rem},
% suppose that the preamble also contains the declarations:
% \begin{verbatim}
%    \theoremstyle{marginbreak}   \newtheorem{Lem}[Cor]{Lemma}
%    \theoremstyle{change}
%    \theorembodyfont{\itshape}        \newtheorem{Def}[Cor]{Definition}
%
%    \theoremheaderfont{\scshape}
%\end{verbatim}
% Then the following are some typical examples of the typeset output
% resulting from their use.
%
% \begin{Cor}
%    This is a sentence typeset in the theorem environment {\envfont
%    Cor}.
% \end{Cor}
%
% \begin{Exa}
%    This is a sentence typeset in the theorem environment {\envfont
%    Exa}.
% \end{Exa}
%
% \begin{Rem}
%    This is a sentence typeset in the theorem environment {\envfont
%    Rem}.
% \end{Rem}
%
% \begin{Lem}[Ben User]
%    This is a sentence typeset in the theorem environment {\envfont
%    Lem}.
% \end{Lem}
%
% \begin{Def}[Very impressive Definition]
%    This is a sentence typeset in the theorem environment {\envfont
%    Def}.
% \end{Def}
%
% The last two examples show the effect of the optional argument to
% a theorem environment (it is the text typeset in parentheses).
%
%
% \section{Special Considerations}
% \label{sec:newfonts}
%
% Theoremheader and body are implemented as a unit. This means that
% the |\theoremheaderfont|  will inherit characteristics of the
% |\theorembodyfont| in \LaTeXe. Thus, if for example
% |\theorembodyfont| is |\itshape| and |\theoremheaderfont|
% is |\bfseries| the font selected for the header will have the
% characteristics `bold extended italic'. If this is not desired one
% should set the |\theoremheaderfont| to something like
% \begin{verbatim}
%      \theoremheaderfont{\normalfont\bfseries}
%\end{verbatim}
% i.e.\ supplying all necessary font information explicitly.
%
%
%
% \section{Acknowledgements}
%
% The publication of this set of macros was only possible with the
% help of Christina Busse (translating the manuscript into English),
% Joachim Pense (playing the r\^ole of typist), Chris Rowley
% (looking everything over) and many others providing useful
% suggestions.
%
% \MaybeStop{%
%    \begin{thebibliography}{9}
%       \bibitem{bk:GMS94} \textsc{M.~Goossens}, \textsc{F.~Mittelbach}
%         and \textsc{A.~Samarin}.
%         \newblock The \LaTeX{} Companion.
%         \newblock
%         Addison-Wesley, Reading, Massachusetts, 1994.
%       \bibitem{latex:source} \textsc{Lamport, Leslie.} \newblock
%                               \texttt{latex.tex}, version 2.09, date
%                               Feb. 1990.
%    \end{thebibliography}
%    \PrintIndex
%    \PrintChanges}
%
%
% \section{The documentation driver file}
%
% The next bit of code contains the documentation driver file for
% \TeX{}, i.e., the file that will produce the documentation you are
% currently reading. It will be extracted from this file by the
% \texttt{docstrip} program. Since it is the first code in the file
% one can alternatively process this file directly with \LaTeXe{} to
% obtain the documentation.
%
%    \begin{macrocode}
%<*driver>
\documentclass{ltxdoc}

\usepackage{theorem}

%<+driver>% The next few lines define theorem sets which are used
%<+driver>% in the example section of the documentation.

\theoremstyle{break}       \newtheorem{Cor}{Corollary}
\theoremstyle{plain}       \newtheorem{Exa}{Example}[section]
{\theorembodyfont{\upshape}\newtheorem{Rem}{Remark}}
\theoremstyle{marginbreak} \newtheorem{Lem}[Cor]{Lemma}
\theoremstyle{change}
\theorembodyfont{\itshape} \newtheorem{Def}[Cor]{Definition}

\theoremheaderfont{\scshape}

\RecordChanges

\begin{document}
   \DocInput{theorem.dtx}
\end{document}
%</driver>
%    \end{macrocode}
%
% \section{Definition of the Macros}
%
% If the file has been loaded before, we abort immediately. If not the
% package announces itself (this is actually done at the very top of
% the file---the way it is done isn't good style so don't copy it).
% \changes{v2.0e}{89/07/19}{Spaces removed from `typeout'.}
% \changes{v2.2b}{95/11/19}{Announce moved to top of file}
%    \begin{macrocode}
%<*package>
%\@ifundefined{theorem@style}{}{\endinput}
%\def\FMithmInfo{1995/11/19 v2.2b Theorem extension package (FMi)}
%\ProvidesPackage{theorem}[\FMithmInfo]
%    \end{macrocode}
%
%
% \subsection{Definition of theorem styles and fonts}
%
% All the definitions in this file are done globally to allow
% inputting this file inside a group.
%
% \begin{macro}{\theoremstyle}
%    Before a theorem style can be installed, the chosen style must be
%    known. For that reason, we must test to see that
%    |\th@|\meta{style} is known or, more precisely, that it is
%    different from |\relax|. If the style is not known then
%    |\th@plain| is used.
%    \begin{macrocode}
\gdef\theoremstyle#1{%
   \@ifundefined{th@#1}{\@warning
          {Unknown theoremstyle `#1'. Using `plain'}%
          \theorem@style{plain}}%
%    \end{macrocode}
%    We save the theorem style to be used in the token register
%    |\theorem@style|.
%    \begin{macrocode}
      {\theorem@style{#1}}%
%    \end{macrocode}
%    Now we ``evaluate'' the theorem style: this means, we call the
%    macro |\th@|\meta{style} which will activate the relevant
%    definitions which are contained in a separate file.  This is done
%    in a group to suppress changes to the current font.  This could
%    otherwise pose problems together with the new font selection
%    scheme\footnote{When I printed the original article using the new
%    font selection scheme I ended with a document with slanted
%    typefaces (text headings and all) simply because one of the
%    theorem styles used \texttt{\bslash sl} at toplevel.} if the
%    |\th@|\meta{style} is evaluated a second time.
% \changes{v2.1a}{90/02/15}{Group added around `th@..}
% \changes{v2.1a}{90/02/15}{Reseting of `..@begintheorem removed.}
%    \begin{macrocode}
      \begingroup
        \csname th@\the\theorem@style \endcsname
      \endgroup}
%    \end{macrocode}
% \end{macro}
%
%
% \begin{macro}{\@begintheorem}
% \changes{v2.1a}{90/02/15}{Relaxed at toplevel.}
% \begin{macro}{\@opargbegintheorem}
% \changes{v2.1a}{90/02/15}{Relaxed at toplevel.}
%    We reset |\@begintheorem| and |\@opargbegintheorem| to
%    |\relax| since these commands are no longer necessary at
%    toplevel. This will save a few tokens.
%    \begin{macrocode}
\global\let\@begintheorem\relax
\global\let\@opargbegintheorem\relax
%    \end{macrocode}
% \end{macro}
% \end{macro}
%
% \begin{macro}{\theorem@style}
%    Obviously the \textsf{token} register used above has to be
%    allocated. To assure the utmost compatibility with the
%    original \LaTeX{} definition, we set the default theorem style to
%    \texttt{plain}, which implements the usual \LaTeX{} convention.
% \changes{v2.0f}{89/07/20}{Unnecessary `global' removed.}
%    \begin{macrocode}
\newtoks\theorem@style
\global\theorem@style{plain}
%    \end{macrocode}
% \end{macro}
%
% \begin{macro}{\theorembodyfont}
% \begin{macro}{\theorem@bodyfont}
%    If the |\theorembodyfont| is set by the user then it should not
%    interact with the default font set in the theorem style. When the
%    new font selection is in force this may happen if, for example,
%    the default is |\itshape| and the new |\theorembodyfont| is
%    |\sffamily|. So we add a |\reset@font| command in front of the
%    user definition.
% \changes{v2.0f}{89/07/20}{Unnecessary `global' removed.}
% \changes{v2.0g}{89/09/19}{`theoremfont to `theorembodyfont.}
% \changes{v2.1c}{92/06/22}{`reset@font added and logic changed.}
% \changes{v2.1d}{92/10/28}{Reset to empty if arg is empty}
%    \begin{macrocode}
\gdef\theorembodyfont#1{%
%    \end{macrocode}
%    We check if the argument supplied is empty and if so put nothing
%    into the |\theorem@bodyfont| token register to allow for
%    |\theorembodyfont{}| as a mean of using the default of the
%    current |\theoremstyle|.
%    \begin{macrocode}
   \def\@tempa{#1}%
   \ifx\@tempa\@empty
    \theorem@bodyfont{}%
   \else
    \theorem@bodyfont{\reset@font#1}%
   \fi
   }
\newtoks\theorem@bodyfont
\global\theorem@bodyfont{}
%    \end{macrocode}
% \end{macro}
% \end{macro}
%
%
% \begin{macro}{\theoremheaderfont}
% \changes{v2.0g}{89/09/19}{Macro added.}
%    The font for the theorem headers is handled differently because
%    this definition applies to all theorem styles.
%    \begin{macrocode}
\gdef\theoremheaderfont#1{\gdef\theorem@headerfont{#1}%
%    \end{macrocode}
%    After using the macro once it is redefined to produce an error
%    message.
%    \begin{macrocode}
       \gdef\theoremheaderfont##1{%
        \typeout{\string\theoremheaderfont\space should be used
                 only once.}}}
%    \end{macrocode}
% \end{macro}
%
% \begin{macro}{\theorem@headerfont}
% \changes{v2.0g}{89/09/19}{Macro added.}
%    To set the |\theorem@headerfont| default we first test if
%    the new fontselection scheme is in force.
% \changes{v2.1a}{90/02/15}{Support for new font selection.}
%    \begin{macrocode}
\ifx\upshape\undefined
%    \end{macrocode}
%    If not we define it to expand into |\bfseries|. We don't use
%    |\let| just in case a following style option redefines this
%    macro.
%    \begin{macrocode}
\gdef\theorem@headerfont{\bfseries}
%    \end{macrocode}
%    Otherwise we reset the current shape before calling |\bfseries|.
%    \begin{macrocode}
\else \gdef\theorem@headerfont{\normalfont\bfseries}\fi
%    \end{macrocode}
% \end{macro}
%
% \begin{macro}{\th@plain}
% \begin{macro}{\th@break}
% \begin{macro}{\th@marginbreak}
% \begin{macro}{\th@changebreak}
% \begin{macro}{\th@change}
% \begin{macro}{\th@margin}
%    The different styles are defined in macros such as
%    |\th@plain|.  Since memory space is precious in
%    ``non-Big-versions'', we have to avoid offering too many unused
%    definitions.  Therefore we define these styles in separate files
%    that can be loaded on demand.  Thus the commands themselves only
%    load these files. We use |\@input@|, a \LaTeXe{} internal command
%    that ensures that the file will be listed with |\listfiles|.
% \changes{v2.2a}{94/02/02}{Use `@input@ to load theorem layout files}
%    \begin{macrocode}
\gdef\th@plain{\@input@{thp.sty}}
\gdef\th@break{\@input@{thb.sty}}
\gdef\th@marginbreak{\@input@{thmb.sty}}
\gdef\th@changebreak{\@input@{thcb.sty}}
\gdef\th@change{\@input@{thc.sty}}
\gdef\th@margin{\@input@{thm.sty}}
%    \end{macrocode}
%    This list will be expanded when new styles become
%    available.  For testing,  just append new theorem substyles
%    as document options.
% \end{macro}
% \end{macro}
% \end{macro}
% \end{macro}
% \end{macro}
% \end{macro}
%
%
%
% \subsection{Definition of a new theorem set}
%
% As already pointed out, a new theorem environment can be defined in
% three different ways:
% \begin{verbatim}
%    \newtheorem{Lem}{Lemma}
%    \newtheorem{Lem}{Lemma}[section]
%    \newtheorem{Lem}[Theorem]{Lemma}
%\end{verbatim}
% The function of the macro |\newtheorem| is to recognize these cases
% and then to branch into one of the three macros |\@ynthm|, |\@xnthm|
% or |\@othm|.  This mechanism is adopted unchanged from
% \cite{latex:source}; the essential point here is that, for example,
% in the second case, the arguments \texttt{Lem}, \texttt{Lemma} and
% \texttt{section} are passed over to the macro |\@xnthm|.
%
% We inspect this case first because the others present fewer problems,
% and thus are easily derived from this one.
%
% \begin{macro}{\@xnthm}
%    For our example arguments, the macro |\@xnthm| must fulfill
%    the following:
%    \begin{itemize}
%       \item
%          Define a new \LaTeX-counter `Lem'
%       \item
%          reset this counter within a |\section|
%       \item
%          define the macro |\theLem|
%       \item
%          define the environment macros |\Lem| and
%          |\endLem| using the current
%          |\theoremstyle| and |\theorem@bodyfont|.
%    \end{itemize}
%    Obviously, all this should happen only if the first argument
%    of |\@xnthm| (i.e.\ \texttt{Lem} in our example) is chosen so
%    as not to conflict with any previously defined commands or
%    environments.  This test is performed by the \LaTeX{} macro
%    |\@ifdefinable|.
%    \begin{macrocode}
\gdef\@xnthm#1#2[#3]{\expandafter\@ifdefinable\csname #1\endcsname
%    \end{macrocode}
%    Therefore, the first argument of |\@ifdefinable| is the
%    expansion (in the example, |\Lem|) of
%    |\csname||#1\endcsname|.  The second argument is
%    executed only if the test has been completed successfully.
%    \begin{macrocode}
   {%
%    \end{macrocode}
%    Now we define the new counter. The names of the \LaTeX{}
%    macros employed should speak for themselves:
% \changes{v2.2b}{1995/10/07}{Use \cs{@newctr}}
% \changes{v2.2c}{1995/11/23}{Use \cs{@newctr} properly.}
%    \begin{macrocode}
    \@definecounter{#1}\@newctr{#1}[#3]%
%    \end{macrocode}
%    Using |\@newctr| will give a proper error message if the counter
%    in |#3| is not defined.
%    In defining `|\theLem|' we must generate the desired macro
%    name by use of |\expandafter| and |\csname|.
%    \begin{macrocode}
    \expandafter\xdef\csname the#1\endcsname
%    \end{macrocode}
%    An |\xdef| is used in order to make the definition global,
%    and to ensure that it contains the replacement texts of
%    |\@thmcountersep| and |\@thmcounter|.\footnote{These
%    two macros can be defined by the document style.  Their default
%    values produce a `\texttt{.}' as separation and an arabic
%    representation of the number.} However, not everything should be
%    expanded. For example, it saves space to use |\thesection|
%    instead of its---at times---lengthy expansion.
%    \begin{macrocode}
      {\expandafter \noexpand \csname the#3\endcsname
       \@thmcountersep \@thmcounter{#1}}%
%    \end{macrocode}
%    Thus with the defaults of \LaTeX{}, |\theLem| would be
%    replaced by the command sequence |\thesection.\arabic{Lem}|.
%
%    We will now look at the definition of the macro which is executed
%    at the beginning of the actual environment (in our example this
%    macro is |\Lem|). It should be noted that we use an
%    ``|\expandafter| trick'' to expand only certain parts of the
%    replacement text at the time of the definition.
%    \begin{macrocode}
    \def\@tempa{\global\@namedef{#1}}%
    \expandafter \@tempa \expandafter{%
%    \end{macrocode}
%    First, the macro that contains the current definitions of
%    |\@begintheorem| and |\@opargtheorem| should be called
%    up.  The name of this macro---as is already known---has the form
%    |\th@|\meta{theorem\ style}; therefore, it must be called by
%    \begin{macrocode}
      \csname th@\the \theorem@style
%    \end{macrocode}
%    In addition the default theorem font should be changeable, i.e.\
%    we have to insert the contents of |\theorem@bodyfont|. For
%    that reason, we expand even further, beyond |\endcsname|,
%    and thus insert the contents of the \textsf{token} register
%    |\theorem@bodyfont| in the replacement text.
%    \begin{macrocode}
            \expandafter \endcsname \the \theorem@bodyfont
%    \end{macrocode}
%    Now it is time to call the macro |\@thm| which takes over
%    the further processing.  It has two arguments: the current
%    counter name (in our example, \texttt{Lem}), and the text of the
%    label (in our example, \texttt{Lemma}).
%    \begin{macrocode}
     \@thm{#1}{#2}}%
%    \end{macrocode}
%    With this, the `sub-definition' is complete.  The macro
%    |\@endtheorem| ends a theorem environment and is, so far, nothing
%    but an |\endtrivlist|.  (Hence it is defined globally, and not
%    within the theorem styles.\footnote{This has to be changed as
%    soon as theorem styles that change \texttt{\bslash @endtheorem}
%    exist.  In such a case, all existing styles must be changed as
%    well since they will have to reset the macro.}) Therefore, we can
%    set it equivalent to the macro that ends the theorem set (in our
%    example, |\endLem|).  However, if some day theorem styles exist
%    that do change |\@endtheorem|, we would have to use the
%    commented-out line instead.
%    \begin{macrocode}
    \global \expandafter \let \csname end#1\endcsname \@endtheorem
%   \global\@namedef{end#1}{\@endtheorem}%
%    \end{macrocode}
%    With these commands all the required definitions are employed,
%    unless the test |\@ifdefinable| has failed.  Therefore, we
%    end the second argument of this macro and with it the definition
%    of |\@xnthm|.
%    \begin{macrocode}
   }}
%    \end{macrocode}
% \end{macro}
%
% \begin{macro}{\@ynthm}
%    The definition of |\@ynthm| is completely analogous. In
%    this case the new counter that is defined is not reset within
%    another counter; thus the definition of |\the...|
%    is simplified:
%    \begin{macrocode}
\gdef\@ynthm#1#2{\expandafter\@ifdefinable\csname #1\endcsname
   {\@definecounter{#1}%
    \expandafter\xdef\csname the#1\endcsname{\@thmcounter{#1}}%
%    \end{macrocode}
%    The rest of the definition corresponds literally to that of
%    |\@xnthm|:
%    \begin{macrocode}
    \def\@tempa{\global\@namedef{#1}}\expandafter \@tempa
     \expandafter{\csname th@\the \theorem@style \expandafter
     \endcsname \the\theorem@bodyfont \@thm{#1}{#2}}%
    \global \expandafter \let \csname end#1\endcsname \@endtheorem}}
%    \end{macrocode}
% \end{macro}
%
% \begin{macro}{\@othm}
%    The definition of |\@othm| does not contain anything new.
%    \begin{macrocode}
\gdef\@othm#1[#2]#3{%
%    \end{macrocode}
% \changes{v2.2b}{95/10/07}{Build in check for counter defined}
%    We do not define a new counter but instead use one that has
%    already been defined.  Thus the only definition we need is that
%    of this pseudo-counter (i.e.\ |\the|\meta{env.\ name}).
%    First we check if |#2| corresponds to a known counter name.
%    \begin{macrocode}
  \expandafter\ifx\csname c@#2\endcsname\relax
   \@nocounterr{#2}%
  \else
   \expandafter\@ifdefinable\csname #1\endcsname
   {\expandafter \xdef \csname the#1\endcsname
     {\expandafter \noexpand \csname the#2\endcsname}%
%    \end{macrocode}
%    All other parts of the definition can be adopted from
%    |\@xnthm|. We have to remember, though, that in this case
%    the name of the current counter and the theorem label have moved
%    to the second and third arguments.
%    \begin{macrocode}
    \def\@tempa{\global\@namedef{#1}}\expandafter \@tempa
     \expandafter{\csname th@\the \theorem@style \expandafter
     \endcsname \the\theorem@bodyfont \@thm{#2}{#3}}%
    \global \expandafter \let \csname end#1\endcsname \@endtheorem}%
  \fi}
%    \end{macrocode}
% \end{macro}
%
% \subsection{Macros that are employed in a {\envfont theorem}
%    environment}
%
% \begin{macro}{\@thm}
%    The macro |\@thm| has to increase the current counter.  Then,
%    depending on whether the environment has (or does not have) an
%    optional argument, it has to branch into either |\@begintheorem|
%    or |\@opargtheorem|.
%    \begin{macrocode}
\gdef\@thm#1#2{\refstepcounter{#1}%
%    \end{macrocode}
%    Now we start a {\envfont trivlist} environment, and give
%    |\@topsep| and |\@topsepadd| the values of the \textsf{skip}
%    registers |\theorempreskipamount| and |\theorempostskipamount|.
%    The value in |\@topsep| is the vertical space that is inserted by
%    the first (and only) |\item| in our |\trivlist| whilst
%    |\@topsepadd| is inserted by |\@endparenv| at the end of that
%    {\envfont trivlist} environment.  By using these registers, we
%    obtain the desired space around a {\envfont theorem} environment.
%    \begin{macrocode}
   \trivlist
   \@topsep \theorempreskipamount               % used by first \item
   \@topsepadd \theorempostskipamount           % used by \@endparenv
%    \end{macrocode}
%    Now we have to test whether an optional argument has been given.
% \changes{v2.0d}{89/07/18}{Forgotten percent added.}
%    \begin{macrocode}
   \@ifnextchar [%
%    \end{macrocode}
%    If there is an optional argument, we will call |\@ythm|, and
%    move the arguments read back into the input stream.
%    \begin{macrocode}
   {\@ythm{#1}{#2}}%
%    \end{macrocode}
%    If not, we call |\@begintheorem|. Its first argument is the
%    name of the theorem set (hence the second argument of
%    |\@thm|). Its second argument is the macro that produces the
%    current number.
%    \begin{macrocode}
   {\@begintheorem{#2}{\csname the#1\endcsname}\ignorespaces}}
%    \end{macrocode}
% \end{macro}
%
% \begin{macro}{\@xthm}
% \begin{macro}{\@ythm}
%    Both these macros were originally called by |\@thm|.  We do
%    not need |\@xthm| anymore, hence we reset it to
%    |\relax|. The definition of |\@ythm| has not changed at
%    all from its definition in \LaTeX. In order to make the macros
%    easier to understand, we will
%    nevertheless present it (commented out).
%    \begin{macrocode}
\global\let\@xthm\relax
% \def\@ythm#1#2[#3]{\@opargbegintheorem{#2}{\csname
%           the#1\endcsname}{#3}\ignorespaces}
%</package>
%    \end{macrocode}
%    The primitive |\ignorespaces| in |\@ythm| and
%    |\@thm| is needed to remove the spaces between the
%    |\begin{...}| and the actual text.
% \end{macro}
% \end{macro}
%
%
% \subsection{Definition of the theorem substyles}
%
% As already pointed out, the theorem substyles, defined below, are
% only loaded when necessary.  Note that all these substyles, except
% \texttt{plain}, have |\slshape| as the default body font.
%
% \subsubsection{The \texttt{plain} style}
%
% \label{dok:th@plain} As the following macros use |@|, we have
% to locally set the |\catcode| of this symbol to ``letter''.
% This happens within a group, so that we do not have to worry about
% which |\catcode| that symbol had before.
%    \begin{macrocode}
%<*thp>
\begingroup \makeatletter
%    \end{macrocode}
% Since we are now within a group, we must make all definitions
% globally.  First we make sure that \texttt{theorem.sty} is loaded.
% This will allow us to use this file as a document style option
% without having to call \texttt{theorem} itself as an option.  At the
% same time, we assure that at least version 2 is loaded, since
% |\theorem@style| was not defined in earlier versions.
%    \begin{macrocode}
\@ifundefined{theorem@style}{\input{theorem.sty}}{}
\ProvidesFile{thp.sty}
             [\FMithmInfo]
%    \end{macrocode}
%
%
% \begin{macro}{\th@plain}
% \begin{macro}{\@begintheorem}
% \begin{macro}{\@opargbegintheorem}
%    |\theoremstyle{plain}| corresponds to the original
%    definition, except that the distances to the surrounding text are
%    determined by the parameters |\theorempreskipamount| and
%    |\theorempostskipamount|.
%    First we set the default body font.
%    \begin{macrocode}
\gdef\th@plain{\normalfont\itshape
%    \end{macrocode}
%    Then we define |\@begintheorem| and
%    |\@opargbegintheorem|. These two macros define how the
%    header of a theorem is typeset.  |\@opargbegintheorem| will
%    be called if a \textsf{theorem} environment with an optional
%    argument is encountered; otherwise, the header is constructed by
%    calling |\@begintheorem|.  If one of these macros is
%    executed, we are within a \textsf{trivlist} environment started by
%    |\@thm|.  So the theorem header is produced with an
%    |\item| command.
%
%    Instead of specifying the header font directly, all standard
%    theorem styles use the |\theorem@headerfont| macro to allow
%    customization.  The extra space (|\labelsep|) is necessary
%    because of problems in the \textsf{trivlist} environment.
%    \begin{macrocode}
  \def\@begintheorem##1##2{%
        \item[\hskip\labelsep \theorem@headerfont ##1\ ##2]}%
%    \end{macrocode}
%    The definition of |\@opargbegintheorem| is completely
%    analogous. The only difference is the fact that there exists a
%    third argument (which is the optional parameter of the
%    environment and contains additional information about the
%    theorem).  Customarily we enclose it in parentheses.
%    \begin{macrocode}
\def\@opargbegintheorem##1##2##3{%
   \item[\hskip\labelsep \theorem@headerfont ##1\ ##2\ (##3)]}}
%    \end{macrocode}
% \end{macro}
% \end{macro}
% \end{macro}
%
% \noindent
% We conclude with an |\endgroup| to restore the
% |\catcode| of |@|.
%    \begin{macrocode}
\endgroup
%</thp>
%    \end{macrocode}
%
%
% \subsubsection{The \texttt{break} style}
%
% This style option is stored in the file |thb.sty|.
% For the next two lines see the documentation for |\th@plain| on
% page~\pageref{dok:th@plain}.
%    \begin{macrocode}
%<*thb>
\begingroup \makeatletter
\@ifundefined{theorem@style}{\input{theorem.sty}}{}
\ProvidesFile{thb.sty}
             [\FMithmInfo]
%    \end{macrocode}
%
% \begin{macro}{\th@break}
%    |\theoremstyle{break}| produces a line break after the name
%    of the theorem.  The font is |\slshape|.  Hence, we define
%    |\th@break| as follows:
%    \begin{macrocode}
\gdef\th@break{\normalfont\slshape
  \def\@begintheorem##1##2{\item[%
%    \end{macrocode}
%    We run into the following problem: it is not possible to create
%    the header with |\item[|\meta{title}|]| and then start
%    a new line by, for example, |\mbox{}\\|. Such a definition
%    will fail whenever a list environment follows immediately.  With
%    the above construction, the |\mbox{}| causes the switch
%    |@inlabel| (cf.\ definition of |\list| and
%    |\trivlist| in \cite{latex:source}) to be set to \texttt{false}
%    and so the following list will insert additional vertical
%    space (|\topskip|). This is quite annoying. Therefore, we
%    create the line break within the |\item|. In order to ensure
%    that the text will begin at the proper position in the following
%    line, we simply pretend that the label does not take any
%    room.\footnote{This will lead to problems whenever very high
%    symbols occurring in the line tower into the
%    heading. So, something else has to be done here sometime.}
%    \begin{macrocode}
       \rlap{\vbox{\hbox{\hskip \labelsep\theorem@headerfont ##1\ ##2}%
                   \hbox{\strut}}}]}%
%    \end{macrocode}
%    Again, the definition of |\@opargbegintheorem| is
%    completely analogous.
%    \begin{macrocode}
\def\@opargbegintheorem##1##2##3{%
  \item[\rlap{\vbox{\hbox{\hskip \labelsep \theorem@headerfont
                     ##1\ ##2\ (##3)}%
                    \hbox{\strut}}}]}}
\endgroup
%</thb>
%    \end{macrocode}
% \end{macro}
%
%
% \subsubsection{The \texttt{changebreak} style}
%
%    \begin{macrocode}
% This style option is stored in the file |thcb.sty|.
%    \begin{macrocode}
%<*thcb>
\begingroup \makeatletter
\@ifundefined{theorem@style}{\input{theorem.sty}}{}
\ProvidesFile{thcb.sty}
             [\FMithmInfo]
%    \end{macrocode}
%
% \begin{macro}{\th@changebreak}
%    The \texttt{change-break} theorem style is like \texttt{break}
%    but with interchange of theorem name and theorem number.  Thus we
%    define |\th@changebreak| as follows:
%    \begin{macrocode}
\gdef\th@changebreak{\normalfont\slshape
  \def\@begintheorem##1##2{\item
       [\rlap{\vbox{\hbox{\hskip\labelsep \theorem@headerfont ##2\ ##1}%
                    \hbox{\strut}}}]}%
\def\@opargbegintheorem##1##2##3{%
  \item[\rlap{\vbox{\hbox{\hskip\labelsep \theorem@headerfont
                     ##2\ ##1\ (##3)}%
                    \hbox{\strut}}}]}}
\endgroup
%</thcb>
%    \end{macrocode}
% \end{macro}
%
%
% \subsubsection{The \texttt{change} style}
%
% This style option is stored in the file |thc.sty|.
%    \begin{macrocode}
%<*thc>
\begingroup \makeatletter
\@ifundefined{theorem@style}{\input{theorem.sty}}{}
\ProvidesFile{thc.sty}
             [\FMithmInfo]
%    \end{macrocode}
%
% \begin{macro}{\th@change}
%    The \texttt{change} theorem style corresponds to the
%    \texttt{change break} style without a linebreak after the header.
%    To say it in another way, it's the same as the \texttt{plain}
%    style but with number and name interchanged and |\slshape| as the
%    default font.
%    \begin{macrocode}
\gdef\th@change{\normalfont\slshape
  \def\@begintheorem##1##2{\item
       [\hskip\labelsep \theorem@headerfont ##2\ ##1]}%
\def\@opargbegintheorem##1##2##3{%
  \item[\hskip\labelsep \theorem@headerfont ##2\ ##1\ (##3)]}}
\endgroup
%</thc>
%    \end{macrocode}
% \end{macro}
%
%
% \subsubsection{The \texttt{marginbreak} style}
%
% This style option is the one used most often at Mainz.  It is saved
% in the file |thmb.sty|.
%    \begin{macrocode}
%<*thmb>
\begingroup \makeatletter
\@ifundefined{theorem@style}{\input{theorem.sty}}{}
\ProvidesFile{thmb.sty}
             [\FMithmInfo]
%    \end{macrocode}
%
% \begin{macro}{\th@marginbreak}
%    The \texttt{margin break} style is nearly the same as the
%    \texttt{change break} style. The only difference is the placement
%    of the theorem number.  We use |\llap| to place it in the left
%    margin.
%
%    In this style |\labelsep| denotes the separation between the
%    number and the text.
%    \begin{macrocode}
\gdef\th@marginbreak{\normalfont\slshape
  \def\@begintheorem##1##2{\item
       [\rlap{\vbox{\theorem@headerfont
                    \hbox{\llap{##2}\hskip\labelsep ##1}%
                    \hbox{\strut}}}]}%
\def\@opargbegintheorem##1##2##3{%
  \item[\rlap{\vbox{\theorem@headerfont
                    \hbox{\llap{##2}\hskip\labelsep ##1\ (##3)}%
                    \hbox{\strut}}}]}}
\endgroup
%</thmb>
%    \end{macrocode}
% \changes{v2.1c}{90/02/17}{`endgroup outside macrocode env.}
% \end{macro}
%
%
%
% \subsubsection{The \texttt{margin} style}
%
% This style option is stored in the file |thm.sty|.
%    \begin{macrocode}
%<*thm>
\begingroup \makeatletter
\@ifundefined{theorem@style}{\input{theorem.sty}}{}
\ProvidesFile{thm.sty}
             [\FMithmInfo]
%    \end{macrocode}
%
% \begin{macro}{\th@margin}
%    Again this is only a variant of the theorem styles described
%    above without any new ideas.
%    \begin{macrocode}
\gdef\th@margin{\normalfont\slshape
  \def\@begintheorem##1##2{\item
       [\theorem@headerfont \llap{##2}\hskip\labelsep ##1]}%
\def\@opargbegintheorem##1##2##3{%
  \item[\theorem@headerfont \llap{##2}\hskip\labelsep ##1\ (##3)]}}
\endgroup
%</thm>
%    \end{macrocode}
% \end{macro}
%
%
%
%
% \subsection{Final Definitions}
%
% \begin{macro}{\theorempreskipamount}
% \begin{macro}{\theorempostskipamount}
%    The \textsf{skip} parameters that regulate the vertical empty space
%    before and after the {\envfont theorem} environment have
%    to be allocated as well.
% \changes{v2.0f}{89/07/20}{Unnecessary `global' removed.}
%    \begin{macrocode}
%<*package>
\newskip\theorempreskipamount
\newskip\theorempostskipamount
%    \end{macrocode}
%    Since we have used the same values for all theorem sets, we now can
%    assign them.
%    \begin{macrocode}
\global\setlength\theorempreskipamount{12pt plus 5pt minus 3pt}
\global\setlength\theorempostskipamount{8pt plus 3pt minus 1.5pt}
%    \end{macrocode}
% \end{macro}
% \end{macro}
%
%
% \begin{macro}{\@endtheorem}
%    The same holds for the macro |\@endtheorem|, which ends a
%    {\envfont theorem} environment. Since it is the same for all
%    theorem sets, it is removed from the macros |\th@|\meta{style}.
%    It simply ends the {\envfont trivlist} environment, which was
%    begun in |\@thm|.
%    \begin{macrocode}
\global\let\@endtheorem=\endtrivlist
%    \end{macrocode}
% \end{macro}
%
%
%
% \begin{macro}{\@preamblecmds}
%    All macros defined above are to be used only in the preamble.
%    Therefore, we insert them in |\@preamblecmds| which will disable
%    them at begin document. This is done by the internal \LaTeXe{}
%    command |\@onlypreamble|.
% \changes{v2.0g}{89/09/19}{`theoremheaderfont and `theorembodyfont
%    added.}
% \changes{v2.2a}{94/02/03}{Use \LaTeXe{} interface}
%    \begin{macrocode}
\@onlypreamble\@xnthm
\@onlypreamble\@ynthm
\@onlypreamble\@othm
\@onlypreamble\newtheorem
\@onlypreamble\theoremstyle
\@onlypreamble\theorembodyfont
\@onlypreamble\theoremheaderfont
%    \end{macrocode}
% \end{macro}
%
% Finally we declare the \texttt{plain} theorem style to be the default.
% \changes{v2.1b}{90/06/04}{Fixed missing `item by declaring default
%    style.}
%    \begin{macrocode}
\theoremstyle{plain}
%</package>
%    \end{macrocode}
%
% \Finale

\endinput
}{}
\ProvidesFile{thp.sty}
             [\FMithmInfo]
%    \end{macrocode}
%
%
% \begin{macro}{\th@plain}
% \begin{macro}{\@begintheorem}
% \begin{macro}{\@opargbegintheorem}
%    |\theoremstyle{plain}| corresponds to the original
%    definition, except that the distances to the surrounding text are
%    determined by the parameters |\theorempreskipamount| and
%    |\theorempostskipamount|.
%    First we set the default body font.
%    \begin{macrocode}
\gdef\th@plain{\normalfont\itshape
%    \end{macrocode}
%    Then we define |\@begintheorem| and
%    |\@opargbegintheorem|. These two macros define how the
%    header of a theorem is typeset.  |\@opargbegintheorem| will
%    be called if a \textsf{theorem} environment with an optional
%    argument is encountered; otherwise, the header is constructed by
%    calling |\@begintheorem|.  If one of these macros is
%    executed, we are within a \textsf{trivlist} environment started by
%    |\@thm|.  So the theorem header is produced with an
%    |\item| command.
%
%    Instead of specifying the header font directly, all standard
%    theorem styles use the |\theorem@headerfont| macro to allow
%    customization.  The extra space (|\labelsep|) is necessary
%    because of problems in the \textsf{trivlist} environment.
%    \begin{macrocode}
  \def\@begintheorem##1##2{%
        \item[\hskip\labelsep \theorem@headerfont ##1\ ##2]}%
%    \end{macrocode}
%    The definition of |\@opargbegintheorem| is completely
%    analogous. The only difference is the fact that there exists a
%    third argument (which is the optional parameter of the
%    environment and contains additional information about the
%    theorem).  Customarily we enclose it in parentheses.
%    \begin{macrocode}
\def\@opargbegintheorem##1##2##3{%
   \item[\hskip\labelsep \theorem@headerfont ##1\ ##2\ (##3)]}}
%    \end{macrocode}
% \end{macro}
% \end{macro}
% \end{macro}
%
% \noindent
% We conclude with an |\endgroup| to restore the
% |\catcode| of |@|.
%    \begin{macrocode}
\endgroup
%</thp>
%    \end{macrocode}
%
%
% \subsubsection{The \texttt{break} style}
%
% This style option is stored in the file |thb.sty|.
% For the next two lines see the documentation for |\th@plain| on
% page~\pageref{dok:th@plain}.
%    \begin{macrocode}
%<*thb>
\begingroup \makeatletter
\@ifundefined{theorem@style}{% \iffalse meta-comment
%
% Copyright (C) 1993-2024
%
% The LaTeX Project and any individual authors listed elsewhere
% in this file.
%
% This file is part of the Standard LaTeX `Tools Bundle'.
% -------------------------------------------------------
%
% It may be distributed and/or modified under the
% conditions of the LaTeX Project Public License, either version 1.3c
% of this license or (at your option) any later version.
% The latest version of this license is in
%    https://www.latex-project.org/lppl.txt
% and version 1.3c or later is part of all distributions of LaTeX
% version 2005/12/01 or later.
%
% The list of all files belonging to the LaTeX `Tools Bundle' is
% given in the file `manifest.txt'.
%
% \fi
%
%%
%
%
\def\FMithmInfo{2023/07/05 v2.2c Theorem extension package (FMi)}
%
% \ProvidesFile{theorem.dtx}[\FMithmInfo]
%\iffalse   % this is a METACOMMENT !
%%
%% Package `theorem' to use with LaTeX2e
%% Copyright (C) 1989-1995 Frank Mittelbach, all rights reserved.
%<+package>\@ifundefined{theorem@style}{}{\endinput}
%<+package>\NeedsTeXFormat{LaTeX2e}[1995/06/01]
%<+package>\ProvidesPackage{theorem}
%<+package>         [\FMithmInfo]
%
% \fi
%
% \changes{v1.0}{87/05/07}{First draft.}
% \changes{v1.0a}{87/05/07}{`thistheoremstyle' added.}
% \changes{v1.1a}{87/07/29}{New concept: redefine the internal
%                          \protect\LaTeX{} commands `@xnthm' etc.}
% \changes{v1.1b}{87/07/31}{`theoremfont' used to specify font.}
% \changes{v1.2a}{88/01/08}{`@endtheorem' now globally defined.}
% \changes{v1.2b}{88/01/11}{`newtheorem' only allowed in the preamble.}
% \changes{v1.3a}{88/02/14}{`theoremfont' now a toks register.}
%
% \changes{v2.0a}{88/11/16}{New implementation and new documentation.}
% \changes{v2.0a}{88/11/16}{No layout changes but internal modifications
%                           old substyles won't work.}
% \changes{v2.0b}{89/01/30}{`global' used for all definitions.}
% \changes{v2.0c}{89/06/15}{Documentation translated to English.
%                            Thanks to LLL and Nina}
% \changes{v2.1a}{90/02/15}{Support for new font selection.}
% \changes{v2.2a}{94/02/03}{Cleanup for LaTeX2e}
%
% \DoNotIndex{\@endparenv,\@ifdefinable,\@ifnextchar,\@ifundefined}
% \DoNotIndex{\@namedef,\@nameuse,\@spaces,\@tempa,\@warning}
% \DoNotIndex{\ }
% \DoNotIndex{\begingroup,\bfseries}
% \DoNotIndex{\csname}
% \DoNotIndex{\def,\do,\docdate}
% \DoNotIndex{\endcsname,\endgroup,\endinput,\endtrivlist,\expandafter}
% \DoNotIndex{\filedate,\fileversion}
% \DoNotIndex{\gdef,\global}
% \DoNotIndex{\hbox,\hskip}
% \DoNotIndex{\ignorespaces,\input,\itshape,\item}
% \DoNotIndex{\labelsep,\let,\llap}
% \DoNotIndex{\makeatletter}
% \DoNotIndex{\noexpand}
% \DoNotIndex{\refstepcounter,\relax,\rlap}
% \DoNotIndex{\setlength,\slshape,\space,\string,\strut}
% \DoNotIndex{\the,\trivlist,\typeout}
% \DoNotIndex{\vbox}
% \DoNotIndex{\xdef}
%
% \def\deflabel#1{\ttfamily #1\hfill}
% \def\deflist#1{\begin{list}{}{\settowidth\labelwidth{\ttfamily #1}%
%                               \setlength\leftmargin\labelwidth
%                               \addtolength\leftmargin\labelsep
%                               \let\makelabel\deflabel}}
% \def\enddeflist{\end{list}}
% \def\envfont{\normalfont\ttfamily}
%
% \MakeShortVerb{\|}
%
% \newcommand{\TUB}{{\sl TUGboat\/}}
%
% \GetFileInfo{theorem.dtx}
% \title{An Extension of the
%        \LaTeX{} theorem environment\thanks{This file
%        has version number \fileversion, last
%        revised \filedate.}}
% \author{Frank Mittelbach}
%
% \MaintainedByLaTeXTeam{tools}
% \maketitle
%
% \begin{abstract}
%    The macros described in this paper yield an extension of the
%    \LaTeX{} theorem mechanism. It is designed to satisfy the
%    different requirements of various journals. Thus, the
%    layout of the ``theorems'' can be manipulated by determining a
%    ``style''. This article describes not only the use, but
%    also the definition, of the necessary macros.
% \end{abstract}
%
% \section*{Preface to version 2.2}
%
% For \LaTeXe{} this package did not need any fundamental changes. I
% only modified the messages generated so that theorem layout styles
% will show up with the |\listfiles| command and cleaned the section
% on the New Font Selection Scheme since this is now included in
% \LaTeX.
%
% \section*{Preface to version 2.1}
%
% This version is identical to 2.0g described in \TUB\ 10\#3 except
% for some internal defaults which are now set depending on the used
% font selection scheme.
%
% This was done to avoid unpleasant surprises if the new font selection
% scheme is in force. For further details see
% section~\ref{sec:newfonts} and \cite{bk:GMS94}.
%
% \section{Introduction}
%
% For our purposes here, ``theorems'' are labelled enunciations,
% often set off from the main text by extra space and a font change.
% Theorems, corollaries, conjectures, definitions, and remarks
% are all instances of ``theorems''.  The ``header'' of these structures
% is composed of a label (such as \textsc{Theorem} or \textsc{Remark})
% and a number which serializes an item in the sequence of items
% with the same label.
%
% Shortly after the introduction of \LaTeX{} at the Fachbereich
% Mathematik in Mainz, the desire to manipulate the layout of
% ``theorems'' arose.  In Mainz, the following two conventions came
% into general use:
% \begin{enumerate}
%   \item The number of the theorem is shown in the margin.
%   \item There is a line break at
%         the end of the theorem header.
% \end{enumerate}
%
% Additionally, some journals require different formats which depend on
% the ``sort of theorem'': e.g.\ often remarks and definitions are set
% in |\upshape|, while |\itshape| is employed for main theorems.
%
% Confronted with these requirements, a theorem environment was
% developed in Mainz which allows separate determination of the
% layout of the ``theorems sets'', comparable to |\pagestyle|.
%
%
% \section{The user interface}
% \subsection{Defining new theorem sets}
%
% \DescribeMacro\newtheorem
% As in the original \LaTeX{} version, the command |\newtheorem|
% defines a new ``theorem set'' or ``theorem-like structure''.
% Two required arguments name the new environment and give the text
% to be typeset with each instance of the new ``set'', while
% an optional argument determines how
% the ``set'' is enumerated:
% \begin{description}
%    \item[\ttfamily \bslash newtheorem\{foo\}\{bar\}]
%       The theorem set {\envfont foo} (whose name is \texttt{bar})
%       uses its own counter.
%    \item[\ttfamily \bslash newtheorem\{foo2\}{[foo]}\{bar2\}]
%       The theorem set {\envfont foo2} (printed name \texttt{bar2})
%       uses the same counter as the theorem set \texttt{foo}.
%    \item[\ttfamily \bslash newtheorem\{foo3\}\{bar3\}{[section]}]
%       The theorem set {\envfont foo3} (printed name \texttt{bar3}) is
%       enumerated within the counter \texttt{section}, i.e.\ with every
%       new |\section| the enumeration begins again with 1, and
%       the enumeration is composed from the section-number and the
%       theorem counter itself.
% \end{description}
%
% \DescribeMacro\theoremstyle
% Additionally, the command |\theoremstyle| can define the layout
% of various, or all, theorem sets.  It should be noted that any
% theorem set defined by |\newtheorem| is typeset in the
% |\theoremstyle| that is current at the time of the definition.
% Thus, the following
% \begin{verbatim}
%    \theoremstyle{break}        \newtheorem{Cor}{Corollary}
%    \theoremstyle{plain}        \newtheorem{Exa}{Example}[section]
%\end{verbatim}                             ^^A No blank at line start
% leads to the result that the set {\envfont Cor} is formatted in the
% style \texttt{break}, while the set {\envfont Exa} and all the
% following ones are formatted in the style \texttt{plain}, unless
% another |\theoremstyle| follows.  Since the definitions installed by
% |\newtheorem| are global, one also can limit |\theoremstyle| locally
% by grouping braces.
%
% \DescribeMacro\theorembodyfont
% The choice of the font for the theorem body is completely
% independent of the chosen |\theoremstyle|; this has proven to
% be very advantageous.  For example,
% \begin{verbatim}
%   {\theorembodyfont{\upshape}          \newtheorem{Rem}{Remark}}
%\end{verbatim}
% defines a theorem set {\envfont Rem}, which will be set in
% |\upshape| in the current layout (which in our example is
% \texttt{plain}).  As with |\theoremstyle|, the |\theorembodyfont|
% chosen is that current at the time of |\newtheorem|.  If
% |\theorembodyfont| is not specified or one defines
% |\theorembodyfont{}|, then the font used will be that defined by the
% |\theoremstyle|.
%
% \DescribeMacro\theoremheaderfont It is also possible to customize
% the font used for the theorem headers.  This is, however, a global
% declaration, and therefore there should be at most one
% |\theoremheaderfont| declaration in the preamble.\footnote{If it is
% actually necessary to have different header fonts, one has to define
% new theorem styles (substituting the desired font) or specify the
% information directly in the \texttt{\bslash newtheorem} declaration
% (the unclean variant).}
%
% \DescribeMacro\theorempreskipamount
% \DescribeMacro\theorempostskipamount Two additional parameters
% affect the vertical space around the theorem environments:
% |\theorempreskipamount| and |\theorempostskipamount| define,
% respectively, the spacing before and after such an environment.
% These parameters apply for all theorem sets and can be manipulated
% with the ordinary length macros.  They are rubber lengths
% (`\textsf{skips}'), and therefore can contain \texttt{plus} and
% \texttt{minus} parts.
%
% Since the definition of theorem sets should---most sensibly---be
% placed in the preamble, we only allow installation there.
% It is therefore possible to release the memory used here after
% |\begin{document}|, in order to make room for other
% applications.
%
%
% \subsection{Existing theorem styles}
%
% The following theorem styles exist to date:
% \begin{deflist}{marginbreak}
%    \item[plain]
%       This theorem style emulates the original \LaTeX{} definition,
%       except that additionally the parameters
%       |\theorem...skipamount| are used.
%    \item[break]
%       In this style, the theorem header is followed by a line
%       break.
%    \item[marginbreak]
%       The theorem number is set in the margin, and there is a line
%       break as in \texttt{break}.
%    \item[changebreak]
%       Like \texttt{break}, but with header number and text
%       interchanged.
%    \item[change]
%       Header number and text are interchanged, without a line break.
%    \item[margin]
%       The number is set in the left margin, without a line break.
% \end{deflist}
% All styles (except \texttt{plain}) select |\slshape| as the default
% |\theorembodyfont|.
%
%
% \subsection{Examples}
%
% Given the above theorem sets {\envfont Cor}, {\envfont Exa}
% and {\envfont Rem},
% suppose that the preamble also contains the declarations:
% \begin{verbatim}
%    \theoremstyle{marginbreak}   \newtheorem{Lem}[Cor]{Lemma}
%    \theoremstyle{change}
%    \theorembodyfont{\itshape}        \newtheorem{Def}[Cor]{Definition}
%
%    \theoremheaderfont{\scshape}
%\end{verbatim}
% Then the following are some typical examples of the typeset output
% resulting from their use.
%
% \begin{Cor}
%    This is a sentence typeset in the theorem environment {\envfont
%    Cor}.
% \end{Cor}
%
% \begin{Exa}
%    This is a sentence typeset in the theorem environment {\envfont
%    Exa}.
% \end{Exa}
%
% \begin{Rem}
%    This is a sentence typeset in the theorem environment {\envfont
%    Rem}.
% \end{Rem}
%
% \begin{Lem}[Ben User]
%    This is a sentence typeset in the theorem environment {\envfont
%    Lem}.
% \end{Lem}
%
% \begin{Def}[Very impressive Definition]
%    This is a sentence typeset in the theorem environment {\envfont
%    Def}.
% \end{Def}
%
% The last two examples show the effect of the optional argument to
% a theorem environment (it is the text typeset in parentheses).
%
%
% \section{Special Considerations}
% \label{sec:newfonts}
%
% Theoremheader and body are implemented as a unit. This means that
% the |\theoremheaderfont|  will inherit characteristics of the
% |\theorembodyfont| in \LaTeXe. Thus, if for example
% |\theorembodyfont| is |\itshape| and |\theoremheaderfont|
% is |\bfseries| the font selected for the header will have the
% characteristics `bold extended italic'. If this is not desired one
% should set the |\theoremheaderfont| to something like
% \begin{verbatim}
%      \theoremheaderfont{\normalfont\bfseries}
%\end{verbatim}
% i.e.\ supplying all necessary font information explicitly.
%
%
%
% \section{Acknowledgements}
%
% The publication of this set of macros was only possible with the
% help of Christina Busse (translating the manuscript into English),
% Joachim Pense (playing the r\^ole of typist), Chris Rowley
% (looking everything over) and many others providing useful
% suggestions.
%
% \MaybeStop{%
%    \begin{thebibliography}{9}
%       \bibitem{bk:GMS94} \textsc{M.~Goossens}, \textsc{F.~Mittelbach}
%         and \textsc{A.~Samarin}.
%         \newblock The \LaTeX{} Companion.
%         \newblock
%         Addison-Wesley, Reading, Massachusetts, 1994.
%       \bibitem{latex:source} \textsc{Lamport, Leslie.} \newblock
%                               \texttt{latex.tex}, version 2.09, date
%                               Feb. 1990.
%    \end{thebibliography}
%    \PrintIndex
%    \PrintChanges}
%
%
% \section{The documentation driver file}
%
% The next bit of code contains the documentation driver file for
% \TeX{}, i.e., the file that will produce the documentation you are
% currently reading. It will be extracted from this file by the
% \texttt{docstrip} program. Since it is the first code in the file
% one can alternatively process this file directly with \LaTeXe{} to
% obtain the documentation.
%
%    \begin{macrocode}
%<*driver>
\documentclass{ltxdoc}

\usepackage{theorem}

%<+driver>% The next few lines define theorem sets which are used
%<+driver>% in the example section of the documentation.

\theoremstyle{break}       \newtheorem{Cor}{Corollary}
\theoremstyle{plain}       \newtheorem{Exa}{Example}[section]
{\theorembodyfont{\upshape}\newtheorem{Rem}{Remark}}
\theoremstyle{marginbreak} \newtheorem{Lem}[Cor]{Lemma}
\theoremstyle{change}
\theorembodyfont{\itshape} \newtheorem{Def}[Cor]{Definition}

\theoremheaderfont{\scshape}

\RecordChanges

\begin{document}
   \DocInput{theorem.dtx}
\end{document}
%</driver>
%    \end{macrocode}
%
% \section{Definition of the Macros}
%
% If the file has been loaded before, we abort immediately. If not the
% package announces itself (this is actually done at the very top of
% the file---the way it is done isn't good style so don't copy it).
% \changes{v2.0e}{89/07/19}{Spaces removed from `typeout'.}
% \changes{v2.2b}{95/11/19}{Announce moved to top of file}
%    \begin{macrocode}
%<*package>
%\@ifundefined{theorem@style}{}{\endinput}
%\def\FMithmInfo{1995/11/19 v2.2b Theorem extension package (FMi)}
%\ProvidesPackage{theorem}[\FMithmInfo]
%    \end{macrocode}
%
%
% \subsection{Definition of theorem styles and fonts}
%
% All the definitions in this file are done globally to allow
% inputting this file inside a group.
%
% \begin{macro}{\theoremstyle}
%    Before a theorem style can be installed, the chosen style must be
%    known. For that reason, we must test to see that
%    |\th@|\meta{style} is known or, more precisely, that it is
%    different from |\relax|. If the style is not known then
%    |\th@plain| is used.
%    \begin{macrocode}
\gdef\theoremstyle#1{%
   \@ifundefined{th@#1}{\@warning
          {Unknown theoremstyle `#1'. Using `plain'}%
          \theorem@style{plain}}%
%    \end{macrocode}
%    We save the theorem style to be used in the token register
%    |\theorem@style|.
%    \begin{macrocode}
      {\theorem@style{#1}}%
%    \end{macrocode}
%    Now we ``evaluate'' the theorem style: this means, we call the
%    macro |\th@|\meta{style} which will activate the relevant
%    definitions which are contained in a separate file.  This is done
%    in a group to suppress changes to the current font.  This could
%    otherwise pose problems together with the new font selection
%    scheme\footnote{When I printed the original article using the new
%    font selection scheme I ended with a document with slanted
%    typefaces (text headings and all) simply because one of the
%    theorem styles used \texttt{\bslash sl} at toplevel.} if the
%    |\th@|\meta{style} is evaluated a second time.
% \changes{v2.1a}{90/02/15}{Group added around `th@..}
% \changes{v2.1a}{90/02/15}{Reseting of `..@begintheorem removed.}
%    \begin{macrocode}
      \begingroup
        \csname th@\the\theorem@style \endcsname
      \endgroup}
%    \end{macrocode}
% \end{macro}
%
%
% \begin{macro}{\@begintheorem}
% \changes{v2.1a}{90/02/15}{Relaxed at toplevel.}
% \begin{macro}{\@opargbegintheorem}
% \changes{v2.1a}{90/02/15}{Relaxed at toplevel.}
%    We reset |\@begintheorem| and |\@opargbegintheorem| to
%    |\relax| since these commands are no longer necessary at
%    toplevel. This will save a few tokens.
%    \begin{macrocode}
\global\let\@begintheorem\relax
\global\let\@opargbegintheorem\relax
%    \end{macrocode}
% \end{macro}
% \end{macro}
%
% \begin{macro}{\theorem@style}
%    Obviously the \textsf{token} register used above has to be
%    allocated. To assure the utmost compatibility with the
%    original \LaTeX{} definition, we set the default theorem style to
%    \texttt{plain}, which implements the usual \LaTeX{} convention.
% \changes{v2.0f}{89/07/20}{Unnecessary `global' removed.}
%    \begin{macrocode}
\newtoks\theorem@style
\global\theorem@style{plain}
%    \end{macrocode}
% \end{macro}
%
% \begin{macro}{\theorembodyfont}
% \begin{macro}{\theorem@bodyfont}
%    If the |\theorembodyfont| is set by the user then it should not
%    interact with the default font set in the theorem style. When the
%    new font selection is in force this may happen if, for example,
%    the default is |\itshape| and the new |\theorembodyfont| is
%    |\sffamily|. So we add a |\reset@font| command in front of the
%    user definition.
% \changes{v2.0f}{89/07/20}{Unnecessary `global' removed.}
% \changes{v2.0g}{89/09/19}{`theoremfont to `theorembodyfont.}
% \changes{v2.1c}{92/06/22}{`reset@font added and logic changed.}
% \changes{v2.1d}{92/10/28}{Reset to empty if arg is empty}
%    \begin{macrocode}
\gdef\theorembodyfont#1{%
%    \end{macrocode}
%    We check if the argument supplied is empty and if so put nothing
%    into the |\theorem@bodyfont| token register to allow for
%    |\theorembodyfont{}| as a mean of using the default of the
%    current |\theoremstyle|.
%    \begin{macrocode}
   \def\@tempa{#1}%
   \ifx\@tempa\@empty
    \theorem@bodyfont{}%
   \else
    \theorem@bodyfont{\reset@font#1}%
   \fi
   }
\newtoks\theorem@bodyfont
\global\theorem@bodyfont{}
%    \end{macrocode}
% \end{macro}
% \end{macro}
%
%
% \begin{macro}{\theoremheaderfont}
% \changes{v2.0g}{89/09/19}{Macro added.}
%    The font for the theorem headers is handled differently because
%    this definition applies to all theorem styles.
%    \begin{macrocode}
\gdef\theoremheaderfont#1{\gdef\theorem@headerfont{#1}%
%    \end{macrocode}
%    After using the macro once it is redefined to produce an error
%    message.
%    \begin{macrocode}
       \gdef\theoremheaderfont##1{%
        \typeout{\string\theoremheaderfont\space should be used
                 only once.}}}
%    \end{macrocode}
% \end{macro}
%
% \begin{macro}{\theorem@headerfont}
% \changes{v2.0g}{89/09/19}{Macro added.}
%    To set the |\theorem@headerfont| default we first test if
%    the new fontselection scheme is in force.
% \changes{v2.1a}{90/02/15}{Support for new font selection.}
%    \begin{macrocode}
\ifx\upshape\undefined
%    \end{macrocode}
%    If not we define it to expand into |\bfseries|. We don't use
%    |\let| just in case a following style option redefines this
%    macro.
%    \begin{macrocode}
\gdef\theorem@headerfont{\bfseries}
%    \end{macrocode}
%    Otherwise we reset the current shape before calling |\bfseries|.
%    \begin{macrocode}
\else \gdef\theorem@headerfont{\normalfont\bfseries}\fi
%    \end{macrocode}
% \end{macro}
%
% \begin{macro}{\th@plain}
% \begin{macro}{\th@break}
% \begin{macro}{\th@marginbreak}
% \begin{macro}{\th@changebreak}
% \begin{macro}{\th@change}
% \begin{macro}{\th@margin}
%    The different styles are defined in macros such as
%    |\th@plain|.  Since memory space is precious in
%    ``non-Big-versions'', we have to avoid offering too many unused
%    definitions.  Therefore we define these styles in separate files
%    that can be loaded on demand.  Thus the commands themselves only
%    load these files. We use |\@input@|, a \LaTeXe{} internal command
%    that ensures that the file will be listed with |\listfiles|.
% \changes{v2.2a}{94/02/02}{Use `@input@ to load theorem layout files}
%    \begin{macrocode}
\gdef\th@plain{\@input@{thp.sty}}
\gdef\th@break{\@input@{thb.sty}}
\gdef\th@marginbreak{\@input@{thmb.sty}}
\gdef\th@changebreak{\@input@{thcb.sty}}
\gdef\th@change{\@input@{thc.sty}}
\gdef\th@margin{\@input@{thm.sty}}
%    \end{macrocode}
%    This list will be expanded when new styles become
%    available.  For testing,  just append new theorem substyles
%    as document options.
% \end{macro}
% \end{macro}
% \end{macro}
% \end{macro}
% \end{macro}
% \end{macro}
%
%
%
% \subsection{Definition of a new theorem set}
%
% As already pointed out, a new theorem environment can be defined in
% three different ways:
% \begin{verbatim}
%    \newtheorem{Lem}{Lemma}
%    \newtheorem{Lem}{Lemma}[section]
%    \newtheorem{Lem}[Theorem]{Lemma}
%\end{verbatim}
% The function of the macro |\newtheorem| is to recognize these cases
% and then to branch into one of the three macros |\@ynthm|, |\@xnthm|
% or |\@othm|.  This mechanism is adopted unchanged from
% \cite{latex:source}; the essential point here is that, for example,
% in the second case, the arguments \texttt{Lem}, \texttt{Lemma} and
% \texttt{section} are passed over to the macro |\@xnthm|.
%
% We inspect this case first because the others present fewer problems,
% and thus are easily derived from this one.
%
% \begin{macro}{\@xnthm}
%    For our example arguments, the macro |\@xnthm| must fulfill
%    the following:
%    \begin{itemize}
%       \item
%          Define a new \LaTeX-counter `Lem'
%       \item
%          reset this counter within a |\section|
%       \item
%          define the macro |\theLem|
%       \item
%          define the environment macros |\Lem| and
%          |\endLem| using the current
%          |\theoremstyle| and |\theorem@bodyfont|.
%    \end{itemize}
%    Obviously, all this should happen only if the first argument
%    of |\@xnthm| (i.e.\ \texttt{Lem} in our example) is chosen so
%    as not to conflict with any previously defined commands or
%    environments.  This test is performed by the \LaTeX{} macro
%    |\@ifdefinable|.
%    \begin{macrocode}
\gdef\@xnthm#1#2[#3]{\expandafter\@ifdefinable\csname #1\endcsname
%    \end{macrocode}
%    Therefore, the first argument of |\@ifdefinable| is the
%    expansion (in the example, |\Lem|) of
%    |\csname||#1\endcsname|.  The second argument is
%    executed only if the test has been completed successfully.
%    \begin{macrocode}
   {%
%    \end{macrocode}
%    Now we define the new counter. The names of the \LaTeX{}
%    macros employed should speak for themselves:
% \changes{v2.2b}{1995/10/07}{Use \cs{@newctr}}
% \changes{v2.2c}{1995/11/23}{Use \cs{@newctr} properly.}
%    \begin{macrocode}
    \@definecounter{#1}\@newctr{#1}[#3]%
%    \end{macrocode}
%    Using |\@newctr| will give a proper error message if the counter
%    in |#3| is not defined.
%    In defining `|\theLem|' we must generate the desired macro
%    name by use of |\expandafter| and |\csname|.
%    \begin{macrocode}
    \expandafter\xdef\csname the#1\endcsname
%    \end{macrocode}
%    An |\xdef| is used in order to make the definition global,
%    and to ensure that it contains the replacement texts of
%    |\@thmcountersep| and |\@thmcounter|.\footnote{These
%    two macros can be defined by the document style.  Their default
%    values produce a `\texttt{.}' as separation and an arabic
%    representation of the number.} However, not everything should be
%    expanded. For example, it saves space to use |\thesection|
%    instead of its---at times---lengthy expansion.
%    \begin{macrocode}
      {\expandafter \noexpand \csname the#3\endcsname
       \@thmcountersep \@thmcounter{#1}}%
%    \end{macrocode}
%    Thus with the defaults of \LaTeX{}, |\theLem| would be
%    replaced by the command sequence |\thesection.\arabic{Lem}|.
%
%    We will now look at the definition of the macro which is executed
%    at the beginning of the actual environment (in our example this
%    macro is |\Lem|). It should be noted that we use an
%    ``|\expandafter| trick'' to expand only certain parts of the
%    replacement text at the time of the definition.
%    \begin{macrocode}
    \def\@tempa{\global\@namedef{#1}}%
    \expandafter \@tempa \expandafter{%
%    \end{macrocode}
%    First, the macro that contains the current definitions of
%    |\@begintheorem| and |\@opargtheorem| should be called
%    up.  The name of this macro---as is already known---has the form
%    |\th@|\meta{theorem\ style}; therefore, it must be called by
%    \begin{macrocode}
      \csname th@\the \theorem@style
%    \end{macrocode}
%    In addition the default theorem font should be changeable, i.e.\
%    we have to insert the contents of |\theorem@bodyfont|. For
%    that reason, we expand even further, beyond |\endcsname|,
%    and thus insert the contents of the \textsf{token} register
%    |\theorem@bodyfont| in the replacement text.
%    \begin{macrocode}
            \expandafter \endcsname \the \theorem@bodyfont
%    \end{macrocode}
%    Now it is time to call the macro |\@thm| which takes over
%    the further processing.  It has two arguments: the current
%    counter name (in our example, \texttt{Lem}), and the text of the
%    label (in our example, \texttt{Lemma}).
%    \begin{macrocode}
     \@thm{#1}{#2}}%
%    \end{macrocode}
%    With this, the `sub-definition' is complete.  The macro
%    |\@endtheorem| ends a theorem environment and is, so far, nothing
%    but an |\endtrivlist|.  (Hence it is defined globally, and not
%    within the theorem styles.\footnote{This has to be changed as
%    soon as theorem styles that change \texttt{\bslash @endtheorem}
%    exist.  In such a case, all existing styles must be changed as
%    well since they will have to reset the macro.}) Therefore, we can
%    set it equivalent to the macro that ends the theorem set (in our
%    example, |\endLem|).  However, if some day theorem styles exist
%    that do change |\@endtheorem|, we would have to use the
%    commented-out line instead.
%    \begin{macrocode}
    \global \expandafter \let \csname end#1\endcsname \@endtheorem
%   \global\@namedef{end#1}{\@endtheorem}%
%    \end{macrocode}
%    With these commands all the required definitions are employed,
%    unless the test |\@ifdefinable| has failed.  Therefore, we
%    end the second argument of this macro and with it the definition
%    of |\@xnthm|.
%    \begin{macrocode}
   }}
%    \end{macrocode}
% \end{macro}
%
% \begin{macro}{\@ynthm}
%    The definition of |\@ynthm| is completely analogous. In
%    this case the new counter that is defined is not reset within
%    another counter; thus the definition of |\the...|
%    is simplified:
%    \begin{macrocode}
\gdef\@ynthm#1#2{\expandafter\@ifdefinable\csname #1\endcsname
   {\@definecounter{#1}%
    \expandafter\xdef\csname the#1\endcsname{\@thmcounter{#1}}%
%    \end{macrocode}
%    The rest of the definition corresponds literally to that of
%    |\@xnthm|:
%    \begin{macrocode}
    \def\@tempa{\global\@namedef{#1}}\expandafter \@tempa
     \expandafter{\csname th@\the \theorem@style \expandafter
     \endcsname \the\theorem@bodyfont \@thm{#1}{#2}}%
    \global \expandafter \let \csname end#1\endcsname \@endtheorem}}
%    \end{macrocode}
% \end{macro}
%
% \begin{macro}{\@othm}
%    The definition of |\@othm| does not contain anything new.
%    \begin{macrocode}
\gdef\@othm#1[#2]#3{%
%    \end{macrocode}
% \changes{v2.2b}{95/10/07}{Build in check for counter defined}
%    We do not define a new counter but instead use one that has
%    already been defined.  Thus the only definition we need is that
%    of this pseudo-counter (i.e.\ |\the|\meta{env.\ name}).
%    First we check if |#2| corresponds to a known counter name.
%    \begin{macrocode}
  \expandafter\ifx\csname c@#2\endcsname\relax
   \@nocounterr{#2}%
  \else
   \expandafter\@ifdefinable\csname #1\endcsname
   {\expandafter \xdef \csname the#1\endcsname
     {\expandafter \noexpand \csname the#2\endcsname}%
%    \end{macrocode}
%    All other parts of the definition can be adopted from
%    |\@xnthm|. We have to remember, though, that in this case
%    the name of the current counter and the theorem label have moved
%    to the second and third arguments.
%    \begin{macrocode}
    \def\@tempa{\global\@namedef{#1}}\expandafter \@tempa
     \expandafter{\csname th@\the \theorem@style \expandafter
     \endcsname \the\theorem@bodyfont \@thm{#2}{#3}}%
    \global \expandafter \let \csname end#1\endcsname \@endtheorem}%
  \fi}
%    \end{macrocode}
% \end{macro}
%
% \subsection{Macros that are employed in a {\envfont theorem}
%    environment}
%
% \begin{macro}{\@thm}
%    The macro |\@thm| has to increase the current counter.  Then,
%    depending on whether the environment has (or does not have) an
%    optional argument, it has to branch into either |\@begintheorem|
%    or |\@opargtheorem|.
%    \begin{macrocode}
\gdef\@thm#1#2{\refstepcounter{#1}%
%    \end{macrocode}
%    Now we start a {\envfont trivlist} environment, and give
%    |\@topsep| and |\@topsepadd| the values of the \textsf{skip}
%    registers |\theorempreskipamount| and |\theorempostskipamount|.
%    The value in |\@topsep| is the vertical space that is inserted by
%    the first (and only) |\item| in our |\trivlist| whilst
%    |\@topsepadd| is inserted by |\@endparenv| at the end of that
%    {\envfont trivlist} environment.  By using these registers, we
%    obtain the desired space around a {\envfont theorem} environment.
%    \begin{macrocode}
   \trivlist
   \@topsep \theorempreskipamount               % used by first \item
   \@topsepadd \theorempostskipamount           % used by \@endparenv
%    \end{macrocode}
%    Now we have to test whether an optional argument has been given.
% \changes{v2.0d}{89/07/18}{Forgotten percent added.}
%    \begin{macrocode}
   \@ifnextchar [%
%    \end{macrocode}
%    If there is an optional argument, we will call |\@ythm|, and
%    move the arguments read back into the input stream.
%    \begin{macrocode}
   {\@ythm{#1}{#2}}%
%    \end{macrocode}
%    If not, we call |\@begintheorem|. Its first argument is the
%    name of the theorem set (hence the second argument of
%    |\@thm|). Its second argument is the macro that produces the
%    current number.
%    \begin{macrocode}
   {\@begintheorem{#2}{\csname the#1\endcsname}\ignorespaces}}
%    \end{macrocode}
% \end{macro}
%
% \begin{macro}{\@xthm}
% \begin{macro}{\@ythm}
%    Both these macros were originally called by |\@thm|.  We do
%    not need |\@xthm| anymore, hence we reset it to
%    |\relax|. The definition of |\@ythm| has not changed at
%    all from its definition in \LaTeX. In order to make the macros
%    easier to understand, we will
%    nevertheless present it (commented out).
%    \begin{macrocode}
\global\let\@xthm\relax
% \def\@ythm#1#2[#3]{\@opargbegintheorem{#2}{\csname
%           the#1\endcsname}{#3}\ignorespaces}
%</package>
%    \end{macrocode}
%    The primitive |\ignorespaces| in |\@ythm| and
%    |\@thm| is needed to remove the spaces between the
%    |\begin{...}| and the actual text.
% \end{macro}
% \end{macro}
%
%
% \subsection{Definition of the theorem substyles}
%
% As already pointed out, the theorem substyles, defined below, are
% only loaded when necessary.  Note that all these substyles, except
% \texttt{plain}, have |\slshape| as the default body font.
%
% \subsubsection{The \texttt{plain} style}
%
% \label{dok:th@plain} As the following macros use |@|, we have
% to locally set the |\catcode| of this symbol to ``letter''.
% This happens within a group, so that we do not have to worry about
% which |\catcode| that symbol had before.
%    \begin{macrocode}
%<*thp>
\begingroup \makeatletter
%    \end{macrocode}
% Since we are now within a group, we must make all definitions
% globally.  First we make sure that \texttt{theorem.sty} is loaded.
% This will allow us to use this file as a document style option
% without having to call \texttt{theorem} itself as an option.  At the
% same time, we assure that at least version 2 is loaded, since
% |\theorem@style| was not defined in earlier versions.
%    \begin{macrocode}
\@ifundefined{theorem@style}{\input{theorem.sty}}{}
\ProvidesFile{thp.sty}
             [\FMithmInfo]
%    \end{macrocode}
%
%
% \begin{macro}{\th@plain}
% \begin{macro}{\@begintheorem}
% \begin{macro}{\@opargbegintheorem}
%    |\theoremstyle{plain}| corresponds to the original
%    definition, except that the distances to the surrounding text are
%    determined by the parameters |\theorempreskipamount| and
%    |\theorempostskipamount|.
%    First we set the default body font.
%    \begin{macrocode}
\gdef\th@plain{\normalfont\itshape
%    \end{macrocode}
%    Then we define |\@begintheorem| and
%    |\@opargbegintheorem|. These two macros define how the
%    header of a theorem is typeset.  |\@opargbegintheorem| will
%    be called if a \textsf{theorem} environment with an optional
%    argument is encountered; otherwise, the header is constructed by
%    calling |\@begintheorem|.  If one of these macros is
%    executed, we are within a \textsf{trivlist} environment started by
%    |\@thm|.  So the theorem header is produced with an
%    |\item| command.
%
%    Instead of specifying the header font directly, all standard
%    theorem styles use the |\theorem@headerfont| macro to allow
%    customization.  The extra space (|\labelsep|) is necessary
%    because of problems in the \textsf{trivlist} environment.
%    \begin{macrocode}
  \def\@begintheorem##1##2{%
        \item[\hskip\labelsep \theorem@headerfont ##1\ ##2]}%
%    \end{macrocode}
%    The definition of |\@opargbegintheorem| is completely
%    analogous. The only difference is the fact that there exists a
%    third argument (which is the optional parameter of the
%    environment and contains additional information about the
%    theorem).  Customarily we enclose it in parentheses.
%    \begin{macrocode}
\def\@opargbegintheorem##1##2##3{%
   \item[\hskip\labelsep \theorem@headerfont ##1\ ##2\ (##3)]}}
%    \end{macrocode}
% \end{macro}
% \end{macro}
% \end{macro}
%
% \noindent
% We conclude with an |\endgroup| to restore the
% |\catcode| of |@|.
%    \begin{macrocode}
\endgroup
%</thp>
%    \end{macrocode}
%
%
% \subsubsection{The \texttt{break} style}
%
% This style option is stored in the file |thb.sty|.
% For the next two lines see the documentation for |\th@plain| on
% page~\pageref{dok:th@plain}.
%    \begin{macrocode}
%<*thb>
\begingroup \makeatletter
\@ifundefined{theorem@style}{\input{theorem.sty}}{}
\ProvidesFile{thb.sty}
             [\FMithmInfo]
%    \end{macrocode}
%
% \begin{macro}{\th@break}
%    |\theoremstyle{break}| produces a line break after the name
%    of the theorem.  The font is |\slshape|.  Hence, we define
%    |\th@break| as follows:
%    \begin{macrocode}
\gdef\th@break{\normalfont\slshape
  \def\@begintheorem##1##2{\item[%
%    \end{macrocode}
%    We run into the following problem: it is not possible to create
%    the header with |\item[|\meta{title}|]| and then start
%    a new line by, for example, |\mbox{}\\|. Such a definition
%    will fail whenever a list environment follows immediately.  With
%    the above construction, the |\mbox{}| causes the switch
%    |@inlabel| (cf.\ definition of |\list| and
%    |\trivlist| in \cite{latex:source}) to be set to \texttt{false}
%    and so the following list will insert additional vertical
%    space (|\topskip|). This is quite annoying. Therefore, we
%    create the line break within the |\item|. In order to ensure
%    that the text will begin at the proper position in the following
%    line, we simply pretend that the label does not take any
%    room.\footnote{This will lead to problems whenever very high
%    symbols occurring in the line tower into the
%    heading. So, something else has to be done here sometime.}
%    \begin{macrocode}
       \rlap{\vbox{\hbox{\hskip \labelsep\theorem@headerfont ##1\ ##2}%
                   \hbox{\strut}}}]}%
%    \end{macrocode}
%    Again, the definition of |\@opargbegintheorem| is
%    completely analogous.
%    \begin{macrocode}
\def\@opargbegintheorem##1##2##3{%
  \item[\rlap{\vbox{\hbox{\hskip \labelsep \theorem@headerfont
                     ##1\ ##2\ (##3)}%
                    \hbox{\strut}}}]}}
\endgroup
%</thb>
%    \end{macrocode}
% \end{macro}
%
%
% \subsubsection{The \texttt{changebreak} style}
%
%    \begin{macrocode}
% This style option is stored in the file |thcb.sty|.
%    \begin{macrocode}
%<*thcb>
\begingroup \makeatletter
\@ifundefined{theorem@style}{\input{theorem.sty}}{}
\ProvidesFile{thcb.sty}
             [\FMithmInfo]
%    \end{macrocode}
%
% \begin{macro}{\th@changebreak}
%    The \texttt{change-break} theorem style is like \texttt{break}
%    but with interchange of theorem name and theorem number.  Thus we
%    define |\th@changebreak| as follows:
%    \begin{macrocode}
\gdef\th@changebreak{\normalfont\slshape
  \def\@begintheorem##1##2{\item
       [\rlap{\vbox{\hbox{\hskip\labelsep \theorem@headerfont ##2\ ##1}%
                    \hbox{\strut}}}]}%
\def\@opargbegintheorem##1##2##3{%
  \item[\rlap{\vbox{\hbox{\hskip\labelsep \theorem@headerfont
                     ##2\ ##1\ (##3)}%
                    \hbox{\strut}}}]}}
\endgroup
%</thcb>
%    \end{macrocode}
% \end{macro}
%
%
% \subsubsection{The \texttt{change} style}
%
% This style option is stored in the file |thc.sty|.
%    \begin{macrocode}
%<*thc>
\begingroup \makeatletter
\@ifundefined{theorem@style}{\input{theorem.sty}}{}
\ProvidesFile{thc.sty}
             [\FMithmInfo]
%    \end{macrocode}
%
% \begin{macro}{\th@change}
%    The \texttt{change} theorem style corresponds to the
%    \texttt{change break} style without a linebreak after the header.
%    To say it in another way, it's the same as the \texttt{plain}
%    style but with number and name interchanged and |\slshape| as the
%    default font.
%    \begin{macrocode}
\gdef\th@change{\normalfont\slshape
  \def\@begintheorem##1##2{\item
       [\hskip\labelsep \theorem@headerfont ##2\ ##1]}%
\def\@opargbegintheorem##1##2##3{%
  \item[\hskip\labelsep \theorem@headerfont ##2\ ##1\ (##3)]}}
\endgroup
%</thc>
%    \end{macrocode}
% \end{macro}
%
%
% \subsubsection{The \texttt{marginbreak} style}
%
% This style option is the one used most often at Mainz.  It is saved
% in the file |thmb.sty|.
%    \begin{macrocode}
%<*thmb>
\begingroup \makeatletter
\@ifundefined{theorem@style}{\input{theorem.sty}}{}
\ProvidesFile{thmb.sty}
             [\FMithmInfo]
%    \end{macrocode}
%
% \begin{macro}{\th@marginbreak}
%    The \texttt{margin break} style is nearly the same as the
%    \texttt{change break} style. The only difference is the placement
%    of the theorem number.  We use |\llap| to place it in the left
%    margin.
%
%    In this style |\labelsep| denotes the separation between the
%    number and the text.
%    \begin{macrocode}
\gdef\th@marginbreak{\normalfont\slshape
  \def\@begintheorem##1##2{\item
       [\rlap{\vbox{\theorem@headerfont
                    \hbox{\llap{##2}\hskip\labelsep ##1}%
                    \hbox{\strut}}}]}%
\def\@opargbegintheorem##1##2##3{%
  \item[\rlap{\vbox{\theorem@headerfont
                    \hbox{\llap{##2}\hskip\labelsep ##1\ (##3)}%
                    \hbox{\strut}}}]}}
\endgroup
%</thmb>
%    \end{macrocode}
% \changes{v2.1c}{90/02/17}{`endgroup outside macrocode env.}
% \end{macro}
%
%
%
% \subsubsection{The \texttt{margin} style}
%
% This style option is stored in the file |thm.sty|.
%    \begin{macrocode}
%<*thm>
\begingroup \makeatletter
\@ifundefined{theorem@style}{\input{theorem.sty}}{}
\ProvidesFile{thm.sty}
             [\FMithmInfo]
%    \end{macrocode}
%
% \begin{macro}{\th@margin}
%    Again this is only a variant of the theorem styles described
%    above without any new ideas.
%    \begin{macrocode}
\gdef\th@margin{\normalfont\slshape
  \def\@begintheorem##1##2{\item
       [\theorem@headerfont \llap{##2}\hskip\labelsep ##1]}%
\def\@opargbegintheorem##1##2##3{%
  \item[\theorem@headerfont \llap{##2}\hskip\labelsep ##1\ (##3)]}}
\endgroup
%</thm>
%    \end{macrocode}
% \end{macro}
%
%
%
%
% \subsection{Final Definitions}
%
% \begin{macro}{\theorempreskipamount}
% \begin{macro}{\theorempostskipamount}
%    The \textsf{skip} parameters that regulate the vertical empty space
%    before and after the {\envfont theorem} environment have
%    to be allocated as well.
% \changes{v2.0f}{89/07/20}{Unnecessary `global' removed.}
%    \begin{macrocode}
%<*package>
\newskip\theorempreskipamount
\newskip\theorempostskipamount
%    \end{macrocode}
%    Since we have used the same values for all theorem sets, we now can
%    assign them.
%    \begin{macrocode}
\global\setlength\theorempreskipamount{12pt plus 5pt minus 3pt}
\global\setlength\theorempostskipamount{8pt plus 3pt minus 1.5pt}
%    \end{macrocode}
% \end{macro}
% \end{macro}
%
%
% \begin{macro}{\@endtheorem}
%    The same holds for the macro |\@endtheorem|, which ends a
%    {\envfont theorem} environment. Since it is the same for all
%    theorem sets, it is removed from the macros |\th@|\meta{style}.
%    It simply ends the {\envfont trivlist} environment, which was
%    begun in |\@thm|.
%    \begin{macrocode}
\global\let\@endtheorem=\endtrivlist
%    \end{macrocode}
% \end{macro}
%
%
%
% \begin{macro}{\@preamblecmds}
%    All macros defined above are to be used only in the preamble.
%    Therefore, we insert them in |\@preamblecmds| which will disable
%    them at begin document. This is done by the internal \LaTeXe{}
%    command |\@onlypreamble|.
% \changes{v2.0g}{89/09/19}{`theoremheaderfont and `theorembodyfont
%    added.}
% \changes{v2.2a}{94/02/03}{Use \LaTeXe{} interface}
%    \begin{macrocode}
\@onlypreamble\@xnthm
\@onlypreamble\@ynthm
\@onlypreamble\@othm
\@onlypreamble\newtheorem
\@onlypreamble\theoremstyle
\@onlypreamble\theorembodyfont
\@onlypreamble\theoremheaderfont
%    \end{macrocode}
% \end{macro}
%
% Finally we declare the \texttt{plain} theorem style to be the default.
% \changes{v2.1b}{90/06/04}{Fixed missing `item by declaring default
%    style.}
%    \begin{macrocode}
\theoremstyle{plain}
%</package>
%    \end{macrocode}
%
% \Finale

\endinput
}{}
\ProvidesFile{thb.sty}
             [\FMithmInfo]
%    \end{macrocode}
%
% \begin{macro}{\th@break}
%    |\theoremstyle{break}| produces a line break after the name
%    of the theorem.  The font is |\slshape|.  Hence, we define
%    |\th@break| as follows:
%    \begin{macrocode}
\gdef\th@break{\normalfont\slshape
  \def\@begintheorem##1##2{\item[%
%    \end{macrocode}
%    We run into the following problem: it is not possible to create
%    the header with |\item[|\meta{title}|]| and then start
%    a new line by, for example, |\mbox{}\\|. Such a definition
%    will fail whenever a list environment follows immediately.  With
%    the above construction, the |\mbox{}| causes the switch
%    |@inlabel| (cf.\ definition of |\list| and
%    |\trivlist| in \cite{latex:source}) to be set to \texttt{false}
%    and so the following list will insert additional vertical
%    space (|\topskip|). This is quite annoying. Therefore, we
%    create the line break within the |\item|. In order to ensure
%    that the text will begin at the proper position in the following
%    line, we simply pretend that the label does not take any
%    room.\footnote{This will lead to problems whenever very high
%    symbols occurring in the line tower into the
%    heading. So, something else has to be done here sometime.}
%    \begin{macrocode}
       \rlap{\vbox{\hbox{\hskip \labelsep\theorem@headerfont ##1\ ##2}%
                   \hbox{\strut}}}]}%
%    \end{macrocode}
%    Again, the definition of |\@opargbegintheorem| is
%    completely analogous.
%    \begin{macrocode}
\def\@opargbegintheorem##1##2##3{%
  \item[\rlap{\vbox{\hbox{\hskip \labelsep \theorem@headerfont
                     ##1\ ##2\ (##3)}%
                    \hbox{\strut}}}]}}
\endgroup
%</thb>
%    \end{macrocode}
% \end{macro}
%
%
% \subsubsection{The \texttt{changebreak} style}
%
%    \begin{macrocode}
% This style option is stored in the file |thcb.sty|.
%    \begin{macrocode}
%<*thcb>
\begingroup \makeatletter
\@ifundefined{theorem@style}{% \iffalse meta-comment
%
% Copyright (C) 1993-2024
%
% The LaTeX Project and any individual authors listed elsewhere
% in this file.
%
% This file is part of the Standard LaTeX `Tools Bundle'.
% -------------------------------------------------------
%
% It may be distributed and/or modified under the
% conditions of the LaTeX Project Public License, either version 1.3c
% of this license or (at your option) any later version.
% The latest version of this license is in
%    https://www.latex-project.org/lppl.txt
% and version 1.3c or later is part of all distributions of LaTeX
% version 2005/12/01 or later.
%
% The list of all files belonging to the LaTeX `Tools Bundle' is
% given in the file `manifest.txt'.
%
% \fi
%
%%
%
%
\def\FMithmInfo{2023/07/05 v2.2c Theorem extension package (FMi)}
%
% \ProvidesFile{theorem.dtx}[\FMithmInfo]
%\iffalse   % this is a METACOMMENT !
%%
%% Package `theorem' to use with LaTeX2e
%% Copyright (C) 1989-1995 Frank Mittelbach, all rights reserved.
%<+package>\@ifundefined{theorem@style}{}{\endinput}
%<+package>\NeedsTeXFormat{LaTeX2e}[1995/06/01]
%<+package>\ProvidesPackage{theorem}
%<+package>         [\FMithmInfo]
%
% \fi
%
% \changes{v1.0}{87/05/07}{First draft.}
% \changes{v1.0a}{87/05/07}{`thistheoremstyle' added.}
% \changes{v1.1a}{87/07/29}{New concept: redefine the internal
%                          \protect\LaTeX{} commands `@xnthm' etc.}
% \changes{v1.1b}{87/07/31}{`theoremfont' used to specify font.}
% \changes{v1.2a}{88/01/08}{`@endtheorem' now globally defined.}
% \changes{v1.2b}{88/01/11}{`newtheorem' only allowed in the preamble.}
% \changes{v1.3a}{88/02/14}{`theoremfont' now a toks register.}
%
% \changes{v2.0a}{88/11/16}{New implementation and new documentation.}
% \changes{v2.0a}{88/11/16}{No layout changes but internal modifications
%                           old substyles won't work.}
% \changes{v2.0b}{89/01/30}{`global' used for all definitions.}
% \changes{v2.0c}{89/06/15}{Documentation translated to English.
%                            Thanks to LLL and Nina}
% \changes{v2.1a}{90/02/15}{Support for new font selection.}
% \changes{v2.2a}{94/02/03}{Cleanup for LaTeX2e}
%
% \DoNotIndex{\@endparenv,\@ifdefinable,\@ifnextchar,\@ifundefined}
% \DoNotIndex{\@namedef,\@nameuse,\@spaces,\@tempa,\@warning}
% \DoNotIndex{\ }
% \DoNotIndex{\begingroup,\bfseries}
% \DoNotIndex{\csname}
% \DoNotIndex{\def,\do,\docdate}
% \DoNotIndex{\endcsname,\endgroup,\endinput,\endtrivlist,\expandafter}
% \DoNotIndex{\filedate,\fileversion}
% \DoNotIndex{\gdef,\global}
% \DoNotIndex{\hbox,\hskip}
% \DoNotIndex{\ignorespaces,\input,\itshape,\item}
% \DoNotIndex{\labelsep,\let,\llap}
% \DoNotIndex{\makeatletter}
% \DoNotIndex{\noexpand}
% \DoNotIndex{\refstepcounter,\relax,\rlap}
% \DoNotIndex{\setlength,\slshape,\space,\string,\strut}
% \DoNotIndex{\the,\trivlist,\typeout}
% \DoNotIndex{\vbox}
% \DoNotIndex{\xdef}
%
% \def\deflabel#1{\ttfamily #1\hfill}
% \def\deflist#1{\begin{list}{}{\settowidth\labelwidth{\ttfamily #1}%
%                               \setlength\leftmargin\labelwidth
%                               \addtolength\leftmargin\labelsep
%                               \let\makelabel\deflabel}}
% \def\enddeflist{\end{list}}
% \def\envfont{\normalfont\ttfamily}
%
% \MakeShortVerb{\|}
%
% \newcommand{\TUB}{{\sl TUGboat\/}}
%
% \GetFileInfo{theorem.dtx}
% \title{An Extension of the
%        \LaTeX{} theorem environment\thanks{This file
%        has version number \fileversion, last
%        revised \filedate.}}
% \author{Frank Mittelbach}
%
% \MaintainedByLaTeXTeam{tools}
% \maketitle
%
% \begin{abstract}
%    The macros described in this paper yield an extension of the
%    \LaTeX{} theorem mechanism. It is designed to satisfy the
%    different requirements of various journals. Thus, the
%    layout of the ``theorems'' can be manipulated by determining a
%    ``style''. This article describes not only the use, but
%    also the definition, of the necessary macros.
% \end{abstract}
%
% \section*{Preface to version 2.2}
%
% For \LaTeXe{} this package did not need any fundamental changes. I
% only modified the messages generated so that theorem layout styles
% will show up with the |\listfiles| command and cleaned the section
% on the New Font Selection Scheme since this is now included in
% \LaTeX.
%
% \section*{Preface to version 2.1}
%
% This version is identical to 2.0g described in \TUB\ 10\#3 except
% for some internal defaults which are now set depending on the used
% font selection scheme.
%
% This was done to avoid unpleasant surprises if the new font selection
% scheme is in force. For further details see
% section~\ref{sec:newfonts} and \cite{bk:GMS94}.
%
% \section{Introduction}
%
% For our purposes here, ``theorems'' are labelled enunciations,
% often set off from the main text by extra space and a font change.
% Theorems, corollaries, conjectures, definitions, and remarks
% are all instances of ``theorems''.  The ``header'' of these structures
% is composed of a label (such as \textsc{Theorem} or \textsc{Remark})
% and a number which serializes an item in the sequence of items
% with the same label.
%
% Shortly after the introduction of \LaTeX{} at the Fachbereich
% Mathematik in Mainz, the desire to manipulate the layout of
% ``theorems'' arose.  In Mainz, the following two conventions came
% into general use:
% \begin{enumerate}
%   \item The number of the theorem is shown in the margin.
%   \item There is a line break at
%         the end of the theorem header.
% \end{enumerate}
%
% Additionally, some journals require different formats which depend on
% the ``sort of theorem'': e.g.\ often remarks and definitions are set
% in |\upshape|, while |\itshape| is employed for main theorems.
%
% Confronted with these requirements, a theorem environment was
% developed in Mainz which allows separate determination of the
% layout of the ``theorems sets'', comparable to |\pagestyle|.
%
%
% \section{The user interface}
% \subsection{Defining new theorem sets}
%
% \DescribeMacro\newtheorem
% As in the original \LaTeX{} version, the command |\newtheorem|
% defines a new ``theorem set'' or ``theorem-like structure''.
% Two required arguments name the new environment and give the text
% to be typeset with each instance of the new ``set'', while
% an optional argument determines how
% the ``set'' is enumerated:
% \begin{description}
%    \item[\ttfamily \bslash newtheorem\{foo\}\{bar\}]
%       The theorem set {\envfont foo} (whose name is \texttt{bar})
%       uses its own counter.
%    \item[\ttfamily \bslash newtheorem\{foo2\}{[foo]}\{bar2\}]
%       The theorem set {\envfont foo2} (printed name \texttt{bar2})
%       uses the same counter as the theorem set \texttt{foo}.
%    \item[\ttfamily \bslash newtheorem\{foo3\}\{bar3\}{[section]}]
%       The theorem set {\envfont foo3} (printed name \texttt{bar3}) is
%       enumerated within the counter \texttt{section}, i.e.\ with every
%       new |\section| the enumeration begins again with 1, and
%       the enumeration is composed from the section-number and the
%       theorem counter itself.
% \end{description}
%
% \DescribeMacro\theoremstyle
% Additionally, the command |\theoremstyle| can define the layout
% of various, or all, theorem sets.  It should be noted that any
% theorem set defined by |\newtheorem| is typeset in the
% |\theoremstyle| that is current at the time of the definition.
% Thus, the following
% \begin{verbatim}
%    \theoremstyle{break}        \newtheorem{Cor}{Corollary}
%    \theoremstyle{plain}        \newtheorem{Exa}{Example}[section]
%\end{verbatim}                             ^^A No blank at line start
% leads to the result that the set {\envfont Cor} is formatted in the
% style \texttt{break}, while the set {\envfont Exa} and all the
% following ones are formatted in the style \texttt{plain}, unless
% another |\theoremstyle| follows.  Since the definitions installed by
% |\newtheorem| are global, one also can limit |\theoremstyle| locally
% by grouping braces.
%
% \DescribeMacro\theorembodyfont
% The choice of the font for the theorem body is completely
% independent of the chosen |\theoremstyle|; this has proven to
% be very advantageous.  For example,
% \begin{verbatim}
%   {\theorembodyfont{\upshape}          \newtheorem{Rem}{Remark}}
%\end{verbatim}
% defines a theorem set {\envfont Rem}, which will be set in
% |\upshape| in the current layout (which in our example is
% \texttt{plain}).  As with |\theoremstyle|, the |\theorembodyfont|
% chosen is that current at the time of |\newtheorem|.  If
% |\theorembodyfont| is not specified or one defines
% |\theorembodyfont{}|, then the font used will be that defined by the
% |\theoremstyle|.
%
% \DescribeMacro\theoremheaderfont It is also possible to customize
% the font used for the theorem headers.  This is, however, a global
% declaration, and therefore there should be at most one
% |\theoremheaderfont| declaration in the preamble.\footnote{If it is
% actually necessary to have different header fonts, one has to define
% new theorem styles (substituting the desired font) or specify the
% information directly in the \texttt{\bslash newtheorem} declaration
% (the unclean variant).}
%
% \DescribeMacro\theorempreskipamount
% \DescribeMacro\theorempostskipamount Two additional parameters
% affect the vertical space around the theorem environments:
% |\theorempreskipamount| and |\theorempostskipamount| define,
% respectively, the spacing before and after such an environment.
% These parameters apply for all theorem sets and can be manipulated
% with the ordinary length macros.  They are rubber lengths
% (`\textsf{skips}'), and therefore can contain \texttt{plus} and
% \texttt{minus} parts.
%
% Since the definition of theorem sets should---most sensibly---be
% placed in the preamble, we only allow installation there.
% It is therefore possible to release the memory used here after
% |\begin{document}|, in order to make room for other
% applications.
%
%
% \subsection{Existing theorem styles}
%
% The following theorem styles exist to date:
% \begin{deflist}{marginbreak}
%    \item[plain]
%       This theorem style emulates the original \LaTeX{} definition,
%       except that additionally the parameters
%       |\theorem...skipamount| are used.
%    \item[break]
%       In this style, the theorem header is followed by a line
%       break.
%    \item[marginbreak]
%       The theorem number is set in the margin, and there is a line
%       break as in \texttt{break}.
%    \item[changebreak]
%       Like \texttt{break}, but with header number and text
%       interchanged.
%    \item[change]
%       Header number and text are interchanged, without a line break.
%    \item[margin]
%       The number is set in the left margin, without a line break.
% \end{deflist}
% All styles (except \texttt{plain}) select |\slshape| as the default
% |\theorembodyfont|.
%
%
% \subsection{Examples}
%
% Given the above theorem sets {\envfont Cor}, {\envfont Exa}
% and {\envfont Rem},
% suppose that the preamble also contains the declarations:
% \begin{verbatim}
%    \theoremstyle{marginbreak}   \newtheorem{Lem}[Cor]{Lemma}
%    \theoremstyle{change}
%    \theorembodyfont{\itshape}        \newtheorem{Def}[Cor]{Definition}
%
%    \theoremheaderfont{\scshape}
%\end{verbatim}
% Then the following are some typical examples of the typeset output
% resulting from their use.
%
% \begin{Cor}
%    This is a sentence typeset in the theorem environment {\envfont
%    Cor}.
% \end{Cor}
%
% \begin{Exa}
%    This is a sentence typeset in the theorem environment {\envfont
%    Exa}.
% \end{Exa}
%
% \begin{Rem}
%    This is a sentence typeset in the theorem environment {\envfont
%    Rem}.
% \end{Rem}
%
% \begin{Lem}[Ben User]
%    This is a sentence typeset in the theorem environment {\envfont
%    Lem}.
% \end{Lem}
%
% \begin{Def}[Very impressive Definition]
%    This is a sentence typeset in the theorem environment {\envfont
%    Def}.
% \end{Def}
%
% The last two examples show the effect of the optional argument to
% a theorem environment (it is the text typeset in parentheses).
%
%
% \section{Special Considerations}
% \label{sec:newfonts}
%
% Theoremheader and body are implemented as a unit. This means that
% the |\theoremheaderfont|  will inherit characteristics of the
% |\theorembodyfont| in \LaTeXe. Thus, if for example
% |\theorembodyfont| is |\itshape| and |\theoremheaderfont|
% is |\bfseries| the font selected for the header will have the
% characteristics `bold extended italic'. If this is not desired one
% should set the |\theoremheaderfont| to something like
% \begin{verbatim}
%      \theoremheaderfont{\normalfont\bfseries}
%\end{verbatim}
% i.e.\ supplying all necessary font information explicitly.
%
%
%
% \section{Acknowledgements}
%
% The publication of this set of macros was only possible with the
% help of Christina Busse (translating the manuscript into English),
% Joachim Pense (playing the r\^ole of typist), Chris Rowley
% (looking everything over) and many others providing useful
% suggestions.
%
% \MaybeStop{%
%    \begin{thebibliography}{9}
%       \bibitem{bk:GMS94} \textsc{M.~Goossens}, \textsc{F.~Mittelbach}
%         and \textsc{A.~Samarin}.
%         \newblock The \LaTeX{} Companion.
%         \newblock
%         Addison-Wesley, Reading, Massachusetts, 1994.
%       \bibitem{latex:source} \textsc{Lamport, Leslie.} \newblock
%                               \texttt{latex.tex}, version 2.09, date
%                               Feb. 1990.
%    \end{thebibliography}
%    \PrintIndex
%    \PrintChanges}
%
%
% \section{The documentation driver file}
%
% The next bit of code contains the documentation driver file for
% \TeX{}, i.e., the file that will produce the documentation you are
% currently reading. It will be extracted from this file by the
% \texttt{docstrip} program. Since it is the first code in the file
% one can alternatively process this file directly with \LaTeXe{} to
% obtain the documentation.
%
%    \begin{macrocode}
%<*driver>
\documentclass{ltxdoc}

\usepackage{theorem}

%<+driver>% The next few lines define theorem sets which are used
%<+driver>% in the example section of the documentation.

\theoremstyle{break}       \newtheorem{Cor}{Corollary}
\theoremstyle{plain}       \newtheorem{Exa}{Example}[section]
{\theorembodyfont{\upshape}\newtheorem{Rem}{Remark}}
\theoremstyle{marginbreak} \newtheorem{Lem}[Cor]{Lemma}
\theoremstyle{change}
\theorembodyfont{\itshape} \newtheorem{Def}[Cor]{Definition}

\theoremheaderfont{\scshape}

\RecordChanges

\begin{document}
   \DocInput{theorem.dtx}
\end{document}
%</driver>
%    \end{macrocode}
%
% \section{Definition of the Macros}
%
% If the file has been loaded before, we abort immediately. If not the
% package announces itself (this is actually done at the very top of
% the file---the way it is done isn't good style so don't copy it).
% \changes{v2.0e}{89/07/19}{Spaces removed from `typeout'.}
% \changes{v2.2b}{95/11/19}{Announce moved to top of file}
%    \begin{macrocode}
%<*package>
%\@ifundefined{theorem@style}{}{\endinput}
%\def\FMithmInfo{1995/11/19 v2.2b Theorem extension package (FMi)}
%\ProvidesPackage{theorem}[\FMithmInfo]
%    \end{macrocode}
%
%
% \subsection{Definition of theorem styles and fonts}
%
% All the definitions in this file are done globally to allow
% inputting this file inside a group.
%
% \begin{macro}{\theoremstyle}
%    Before a theorem style can be installed, the chosen style must be
%    known. For that reason, we must test to see that
%    |\th@|\meta{style} is known or, more precisely, that it is
%    different from |\relax|. If the style is not known then
%    |\th@plain| is used.
%    \begin{macrocode}
\gdef\theoremstyle#1{%
   \@ifundefined{th@#1}{\@warning
          {Unknown theoremstyle `#1'. Using `plain'}%
          \theorem@style{plain}}%
%    \end{macrocode}
%    We save the theorem style to be used in the token register
%    |\theorem@style|.
%    \begin{macrocode}
      {\theorem@style{#1}}%
%    \end{macrocode}
%    Now we ``evaluate'' the theorem style: this means, we call the
%    macro |\th@|\meta{style} which will activate the relevant
%    definitions which are contained in a separate file.  This is done
%    in a group to suppress changes to the current font.  This could
%    otherwise pose problems together with the new font selection
%    scheme\footnote{When I printed the original article using the new
%    font selection scheme I ended with a document with slanted
%    typefaces (text headings and all) simply because one of the
%    theorem styles used \texttt{\bslash sl} at toplevel.} if the
%    |\th@|\meta{style} is evaluated a second time.
% \changes{v2.1a}{90/02/15}{Group added around `th@..}
% \changes{v2.1a}{90/02/15}{Reseting of `..@begintheorem removed.}
%    \begin{macrocode}
      \begingroup
        \csname th@\the\theorem@style \endcsname
      \endgroup}
%    \end{macrocode}
% \end{macro}
%
%
% \begin{macro}{\@begintheorem}
% \changes{v2.1a}{90/02/15}{Relaxed at toplevel.}
% \begin{macro}{\@opargbegintheorem}
% \changes{v2.1a}{90/02/15}{Relaxed at toplevel.}
%    We reset |\@begintheorem| and |\@opargbegintheorem| to
%    |\relax| since these commands are no longer necessary at
%    toplevel. This will save a few tokens.
%    \begin{macrocode}
\global\let\@begintheorem\relax
\global\let\@opargbegintheorem\relax
%    \end{macrocode}
% \end{macro}
% \end{macro}
%
% \begin{macro}{\theorem@style}
%    Obviously the \textsf{token} register used above has to be
%    allocated. To assure the utmost compatibility with the
%    original \LaTeX{} definition, we set the default theorem style to
%    \texttt{plain}, which implements the usual \LaTeX{} convention.
% \changes{v2.0f}{89/07/20}{Unnecessary `global' removed.}
%    \begin{macrocode}
\newtoks\theorem@style
\global\theorem@style{plain}
%    \end{macrocode}
% \end{macro}
%
% \begin{macro}{\theorembodyfont}
% \begin{macro}{\theorem@bodyfont}
%    If the |\theorembodyfont| is set by the user then it should not
%    interact with the default font set in the theorem style. When the
%    new font selection is in force this may happen if, for example,
%    the default is |\itshape| and the new |\theorembodyfont| is
%    |\sffamily|. So we add a |\reset@font| command in front of the
%    user definition.
% \changes{v2.0f}{89/07/20}{Unnecessary `global' removed.}
% \changes{v2.0g}{89/09/19}{`theoremfont to `theorembodyfont.}
% \changes{v2.1c}{92/06/22}{`reset@font added and logic changed.}
% \changes{v2.1d}{92/10/28}{Reset to empty if arg is empty}
%    \begin{macrocode}
\gdef\theorembodyfont#1{%
%    \end{macrocode}
%    We check if the argument supplied is empty and if so put nothing
%    into the |\theorem@bodyfont| token register to allow for
%    |\theorembodyfont{}| as a mean of using the default of the
%    current |\theoremstyle|.
%    \begin{macrocode}
   \def\@tempa{#1}%
   \ifx\@tempa\@empty
    \theorem@bodyfont{}%
   \else
    \theorem@bodyfont{\reset@font#1}%
   \fi
   }
\newtoks\theorem@bodyfont
\global\theorem@bodyfont{}
%    \end{macrocode}
% \end{macro}
% \end{macro}
%
%
% \begin{macro}{\theoremheaderfont}
% \changes{v2.0g}{89/09/19}{Macro added.}
%    The font for the theorem headers is handled differently because
%    this definition applies to all theorem styles.
%    \begin{macrocode}
\gdef\theoremheaderfont#1{\gdef\theorem@headerfont{#1}%
%    \end{macrocode}
%    After using the macro once it is redefined to produce an error
%    message.
%    \begin{macrocode}
       \gdef\theoremheaderfont##1{%
        \typeout{\string\theoremheaderfont\space should be used
                 only once.}}}
%    \end{macrocode}
% \end{macro}
%
% \begin{macro}{\theorem@headerfont}
% \changes{v2.0g}{89/09/19}{Macro added.}
%    To set the |\theorem@headerfont| default we first test if
%    the new fontselection scheme is in force.
% \changes{v2.1a}{90/02/15}{Support for new font selection.}
%    \begin{macrocode}
\ifx\upshape\undefined
%    \end{macrocode}
%    If not we define it to expand into |\bfseries|. We don't use
%    |\let| just in case a following style option redefines this
%    macro.
%    \begin{macrocode}
\gdef\theorem@headerfont{\bfseries}
%    \end{macrocode}
%    Otherwise we reset the current shape before calling |\bfseries|.
%    \begin{macrocode}
\else \gdef\theorem@headerfont{\normalfont\bfseries}\fi
%    \end{macrocode}
% \end{macro}
%
% \begin{macro}{\th@plain}
% \begin{macro}{\th@break}
% \begin{macro}{\th@marginbreak}
% \begin{macro}{\th@changebreak}
% \begin{macro}{\th@change}
% \begin{macro}{\th@margin}
%    The different styles are defined in macros such as
%    |\th@plain|.  Since memory space is precious in
%    ``non-Big-versions'', we have to avoid offering too many unused
%    definitions.  Therefore we define these styles in separate files
%    that can be loaded on demand.  Thus the commands themselves only
%    load these files. We use |\@input@|, a \LaTeXe{} internal command
%    that ensures that the file will be listed with |\listfiles|.
% \changes{v2.2a}{94/02/02}{Use `@input@ to load theorem layout files}
%    \begin{macrocode}
\gdef\th@plain{\@input@{thp.sty}}
\gdef\th@break{\@input@{thb.sty}}
\gdef\th@marginbreak{\@input@{thmb.sty}}
\gdef\th@changebreak{\@input@{thcb.sty}}
\gdef\th@change{\@input@{thc.sty}}
\gdef\th@margin{\@input@{thm.sty}}
%    \end{macrocode}
%    This list will be expanded when new styles become
%    available.  For testing,  just append new theorem substyles
%    as document options.
% \end{macro}
% \end{macro}
% \end{macro}
% \end{macro}
% \end{macro}
% \end{macro}
%
%
%
% \subsection{Definition of a new theorem set}
%
% As already pointed out, a new theorem environment can be defined in
% three different ways:
% \begin{verbatim}
%    \newtheorem{Lem}{Lemma}
%    \newtheorem{Lem}{Lemma}[section]
%    \newtheorem{Lem}[Theorem]{Lemma}
%\end{verbatim}
% The function of the macro |\newtheorem| is to recognize these cases
% and then to branch into one of the three macros |\@ynthm|, |\@xnthm|
% or |\@othm|.  This mechanism is adopted unchanged from
% \cite{latex:source}; the essential point here is that, for example,
% in the second case, the arguments \texttt{Lem}, \texttt{Lemma} and
% \texttt{section} are passed over to the macro |\@xnthm|.
%
% We inspect this case first because the others present fewer problems,
% and thus are easily derived from this one.
%
% \begin{macro}{\@xnthm}
%    For our example arguments, the macro |\@xnthm| must fulfill
%    the following:
%    \begin{itemize}
%       \item
%          Define a new \LaTeX-counter `Lem'
%       \item
%          reset this counter within a |\section|
%       \item
%          define the macro |\theLem|
%       \item
%          define the environment macros |\Lem| and
%          |\endLem| using the current
%          |\theoremstyle| and |\theorem@bodyfont|.
%    \end{itemize}
%    Obviously, all this should happen only if the first argument
%    of |\@xnthm| (i.e.\ \texttt{Lem} in our example) is chosen so
%    as not to conflict with any previously defined commands or
%    environments.  This test is performed by the \LaTeX{} macro
%    |\@ifdefinable|.
%    \begin{macrocode}
\gdef\@xnthm#1#2[#3]{\expandafter\@ifdefinable\csname #1\endcsname
%    \end{macrocode}
%    Therefore, the first argument of |\@ifdefinable| is the
%    expansion (in the example, |\Lem|) of
%    |\csname||#1\endcsname|.  The second argument is
%    executed only if the test has been completed successfully.
%    \begin{macrocode}
   {%
%    \end{macrocode}
%    Now we define the new counter. The names of the \LaTeX{}
%    macros employed should speak for themselves:
% \changes{v2.2b}{1995/10/07}{Use \cs{@newctr}}
% \changes{v2.2c}{1995/11/23}{Use \cs{@newctr} properly.}
%    \begin{macrocode}
    \@definecounter{#1}\@newctr{#1}[#3]%
%    \end{macrocode}
%    Using |\@newctr| will give a proper error message if the counter
%    in |#3| is not defined.
%    In defining `|\theLem|' we must generate the desired macro
%    name by use of |\expandafter| and |\csname|.
%    \begin{macrocode}
    \expandafter\xdef\csname the#1\endcsname
%    \end{macrocode}
%    An |\xdef| is used in order to make the definition global,
%    and to ensure that it contains the replacement texts of
%    |\@thmcountersep| and |\@thmcounter|.\footnote{These
%    two macros can be defined by the document style.  Their default
%    values produce a `\texttt{.}' as separation and an arabic
%    representation of the number.} However, not everything should be
%    expanded. For example, it saves space to use |\thesection|
%    instead of its---at times---lengthy expansion.
%    \begin{macrocode}
      {\expandafter \noexpand \csname the#3\endcsname
       \@thmcountersep \@thmcounter{#1}}%
%    \end{macrocode}
%    Thus with the defaults of \LaTeX{}, |\theLem| would be
%    replaced by the command sequence |\thesection.\arabic{Lem}|.
%
%    We will now look at the definition of the macro which is executed
%    at the beginning of the actual environment (in our example this
%    macro is |\Lem|). It should be noted that we use an
%    ``|\expandafter| trick'' to expand only certain parts of the
%    replacement text at the time of the definition.
%    \begin{macrocode}
    \def\@tempa{\global\@namedef{#1}}%
    \expandafter \@tempa \expandafter{%
%    \end{macrocode}
%    First, the macro that contains the current definitions of
%    |\@begintheorem| and |\@opargtheorem| should be called
%    up.  The name of this macro---as is already known---has the form
%    |\th@|\meta{theorem\ style}; therefore, it must be called by
%    \begin{macrocode}
      \csname th@\the \theorem@style
%    \end{macrocode}
%    In addition the default theorem font should be changeable, i.e.\
%    we have to insert the contents of |\theorem@bodyfont|. For
%    that reason, we expand even further, beyond |\endcsname|,
%    and thus insert the contents of the \textsf{token} register
%    |\theorem@bodyfont| in the replacement text.
%    \begin{macrocode}
            \expandafter \endcsname \the \theorem@bodyfont
%    \end{macrocode}
%    Now it is time to call the macro |\@thm| which takes over
%    the further processing.  It has two arguments: the current
%    counter name (in our example, \texttt{Lem}), and the text of the
%    label (in our example, \texttt{Lemma}).
%    \begin{macrocode}
     \@thm{#1}{#2}}%
%    \end{macrocode}
%    With this, the `sub-definition' is complete.  The macro
%    |\@endtheorem| ends a theorem environment and is, so far, nothing
%    but an |\endtrivlist|.  (Hence it is defined globally, and not
%    within the theorem styles.\footnote{This has to be changed as
%    soon as theorem styles that change \texttt{\bslash @endtheorem}
%    exist.  In such a case, all existing styles must be changed as
%    well since they will have to reset the macro.}) Therefore, we can
%    set it equivalent to the macro that ends the theorem set (in our
%    example, |\endLem|).  However, if some day theorem styles exist
%    that do change |\@endtheorem|, we would have to use the
%    commented-out line instead.
%    \begin{macrocode}
    \global \expandafter \let \csname end#1\endcsname \@endtheorem
%   \global\@namedef{end#1}{\@endtheorem}%
%    \end{macrocode}
%    With these commands all the required definitions are employed,
%    unless the test |\@ifdefinable| has failed.  Therefore, we
%    end the second argument of this macro and with it the definition
%    of |\@xnthm|.
%    \begin{macrocode}
   }}
%    \end{macrocode}
% \end{macro}
%
% \begin{macro}{\@ynthm}
%    The definition of |\@ynthm| is completely analogous. In
%    this case the new counter that is defined is not reset within
%    another counter; thus the definition of |\the...|
%    is simplified:
%    \begin{macrocode}
\gdef\@ynthm#1#2{\expandafter\@ifdefinable\csname #1\endcsname
   {\@definecounter{#1}%
    \expandafter\xdef\csname the#1\endcsname{\@thmcounter{#1}}%
%    \end{macrocode}
%    The rest of the definition corresponds literally to that of
%    |\@xnthm|:
%    \begin{macrocode}
    \def\@tempa{\global\@namedef{#1}}\expandafter \@tempa
     \expandafter{\csname th@\the \theorem@style \expandafter
     \endcsname \the\theorem@bodyfont \@thm{#1}{#2}}%
    \global \expandafter \let \csname end#1\endcsname \@endtheorem}}
%    \end{macrocode}
% \end{macro}
%
% \begin{macro}{\@othm}
%    The definition of |\@othm| does not contain anything new.
%    \begin{macrocode}
\gdef\@othm#1[#2]#3{%
%    \end{macrocode}
% \changes{v2.2b}{95/10/07}{Build in check for counter defined}
%    We do not define a new counter but instead use one that has
%    already been defined.  Thus the only definition we need is that
%    of this pseudo-counter (i.e.\ |\the|\meta{env.\ name}).
%    First we check if |#2| corresponds to a known counter name.
%    \begin{macrocode}
  \expandafter\ifx\csname c@#2\endcsname\relax
   \@nocounterr{#2}%
  \else
   \expandafter\@ifdefinable\csname #1\endcsname
   {\expandafter \xdef \csname the#1\endcsname
     {\expandafter \noexpand \csname the#2\endcsname}%
%    \end{macrocode}
%    All other parts of the definition can be adopted from
%    |\@xnthm|. We have to remember, though, that in this case
%    the name of the current counter and the theorem label have moved
%    to the second and third arguments.
%    \begin{macrocode}
    \def\@tempa{\global\@namedef{#1}}\expandafter \@tempa
     \expandafter{\csname th@\the \theorem@style \expandafter
     \endcsname \the\theorem@bodyfont \@thm{#2}{#3}}%
    \global \expandafter \let \csname end#1\endcsname \@endtheorem}%
  \fi}
%    \end{macrocode}
% \end{macro}
%
% \subsection{Macros that are employed in a {\envfont theorem}
%    environment}
%
% \begin{macro}{\@thm}
%    The macro |\@thm| has to increase the current counter.  Then,
%    depending on whether the environment has (or does not have) an
%    optional argument, it has to branch into either |\@begintheorem|
%    or |\@opargtheorem|.
%    \begin{macrocode}
\gdef\@thm#1#2{\refstepcounter{#1}%
%    \end{macrocode}
%    Now we start a {\envfont trivlist} environment, and give
%    |\@topsep| and |\@topsepadd| the values of the \textsf{skip}
%    registers |\theorempreskipamount| and |\theorempostskipamount|.
%    The value in |\@topsep| is the vertical space that is inserted by
%    the first (and only) |\item| in our |\trivlist| whilst
%    |\@topsepadd| is inserted by |\@endparenv| at the end of that
%    {\envfont trivlist} environment.  By using these registers, we
%    obtain the desired space around a {\envfont theorem} environment.
%    \begin{macrocode}
   \trivlist
   \@topsep \theorempreskipamount               % used by first \item
   \@topsepadd \theorempostskipamount           % used by \@endparenv
%    \end{macrocode}
%    Now we have to test whether an optional argument has been given.
% \changes{v2.0d}{89/07/18}{Forgotten percent added.}
%    \begin{macrocode}
   \@ifnextchar [%
%    \end{macrocode}
%    If there is an optional argument, we will call |\@ythm|, and
%    move the arguments read back into the input stream.
%    \begin{macrocode}
   {\@ythm{#1}{#2}}%
%    \end{macrocode}
%    If not, we call |\@begintheorem|. Its first argument is the
%    name of the theorem set (hence the second argument of
%    |\@thm|). Its second argument is the macro that produces the
%    current number.
%    \begin{macrocode}
   {\@begintheorem{#2}{\csname the#1\endcsname}\ignorespaces}}
%    \end{macrocode}
% \end{macro}
%
% \begin{macro}{\@xthm}
% \begin{macro}{\@ythm}
%    Both these macros were originally called by |\@thm|.  We do
%    not need |\@xthm| anymore, hence we reset it to
%    |\relax|. The definition of |\@ythm| has not changed at
%    all from its definition in \LaTeX. In order to make the macros
%    easier to understand, we will
%    nevertheless present it (commented out).
%    \begin{macrocode}
\global\let\@xthm\relax
% \def\@ythm#1#2[#3]{\@opargbegintheorem{#2}{\csname
%           the#1\endcsname}{#3}\ignorespaces}
%</package>
%    \end{macrocode}
%    The primitive |\ignorespaces| in |\@ythm| and
%    |\@thm| is needed to remove the spaces between the
%    |\begin{...}| and the actual text.
% \end{macro}
% \end{macro}
%
%
% \subsection{Definition of the theorem substyles}
%
% As already pointed out, the theorem substyles, defined below, are
% only loaded when necessary.  Note that all these substyles, except
% \texttt{plain}, have |\slshape| as the default body font.
%
% \subsubsection{The \texttt{plain} style}
%
% \label{dok:th@plain} As the following macros use |@|, we have
% to locally set the |\catcode| of this symbol to ``letter''.
% This happens within a group, so that we do not have to worry about
% which |\catcode| that symbol had before.
%    \begin{macrocode}
%<*thp>
\begingroup \makeatletter
%    \end{macrocode}
% Since we are now within a group, we must make all definitions
% globally.  First we make sure that \texttt{theorem.sty} is loaded.
% This will allow us to use this file as a document style option
% without having to call \texttt{theorem} itself as an option.  At the
% same time, we assure that at least version 2 is loaded, since
% |\theorem@style| was not defined in earlier versions.
%    \begin{macrocode}
\@ifundefined{theorem@style}{\input{theorem.sty}}{}
\ProvidesFile{thp.sty}
             [\FMithmInfo]
%    \end{macrocode}
%
%
% \begin{macro}{\th@plain}
% \begin{macro}{\@begintheorem}
% \begin{macro}{\@opargbegintheorem}
%    |\theoremstyle{plain}| corresponds to the original
%    definition, except that the distances to the surrounding text are
%    determined by the parameters |\theorempreskipamount| and
%    |\theorempostskipamount|.
%    First we set the default body font.
%    \begin{macrocode}
\gdef\th@plain{\normalfont\itshape
%    \end{macrocode}
%    Then we define |\@begintheorem| and
%    |\@opargbegintheorem|. These two macros define how the
%    header of a theorem is typeset.  |\@opargbegintheorem| will
%    be called if a \textsf{theorem} environment with an optional
%    argument is encountered; otherwise, the header is constructed by
%    calling |\@begintheorem|.  If one of these macros is
%    executed, we are within a \textsf{trivlist} environment started by
%    |\@thm|.  So the theorem header is produced with an
%    |\item| command.
%
%    Instead of specifying the header font directly, all standard
%    theorem styles use the |\theorem@headerfont| macro to allow
%    customization.  The extra space (|\labelsep|) is necessary
%    because of problems in the \textsf{trivlist} environment.
%    \begin{macrocode}
  \def\@begintheorem##1##2{%
        \item[\hskip\labelsep \theorem@headerfont ##1\ ##2]}%
%    \end{macrocode}
%    The definition of |\@opargbegintheorem| is completely
%    analogous. The only difference is the fact that there exists a
%    third argument (which is the optional parameter of the
%    environment and contains additional information about the
%    theorem).  Customarily we enclose it in parentheses.
%    \begin{macrocode}
\def\@opargbegintheorem##1##2##3{%
   \item[\hskip\labelsep \theorem@headerfont ##1\ ##2\ (##3)]}}
%    \end{macrocode}
% \end{macro}
% \end{macro}
% \end{macro}
%
% \noindent
% We conclude with an |\endgroup| to restore the
% |\catcode| of |@|.
%    \begin{macrocode}
\endgroup
%</thp>
%    \end{macrocode}
%
%
% \subsubsection{The \texttt{break} style}
%
% This style option is stored in the file |thb.sty|.
% For the next two lines see the documentation for |\th@plain| on
% page~\pageref{dok:th@plain}.
%    \begin{macrocode}
%<*thb>
\begingroup \makeatletter
\@ifundefined{theorem@style}{\input{theorem.sty}}{}
\ProvidesFile{thb.sty}
             [\FMithmInfo]
%    \end{macrocode}
%
% \begin{macro}{\th@break}
%    |\theoremstyle{break}| produces a line break after the name
%    of the theorem.  The font is |\slshape|.  Hence, we define
%    |\th@break| as follows:
%    \begin{macrocode}
\gdef\th@break{\normalfont\slshape
  \def\@begintheorem##1##2{\item[%
%    \end{macrocode}
%    We run into the following problem: it is not possible to create
%    the header with |\item[|\meta{title}|]| and then start
%    a new line by, for example, |\mbox{}\\|. Such a definition
%    will fail whenever a list environment follows immediately.  With
%    the above construction, the |\mbox{}| causes the switch
%    |@inlabel| (cf.\ definition of |\list| and
%    |\trivlist| in \cite{latex:source}) to be set to \texttt{false}
%    and so the following list will insert additional vertical
%    space (|\topskip|). This is quite annoying. Therefore, we
%    create the line break within the |\item|. In order to ensure
%    that the text will begin at the proper position in the following
%    line, we simply pretend that the label does not take any
%    room.\footnote{This will lead to problems whenever very high
%    symbols occurring in the line tower into the
%    heading. So, something else has to be done here sometime.}
%    \begin{macrocode}
       \rlap{\vbox{\hbox{\hskip \labelsep\theorem@headerfont ##1\ ##2}%
                   \hbox{\strut}}}]}%
%    \end{macrocode}
%    Again, the definition of |\@opargbegintheorem| is
%    completely analogous.
%    \begin{macrocode}
\def\@opargbegintheorem##1##2##3{%
  \item[\rlap{\vbox{\hbox{\hskip \labelsep \theorem@headerfont
                     ##1\ ##2\ (##3)}%
                    \hbox{\strut}}}]}}
\endgroup
%</thb>
%    \end{macrocode}
% \end{macro}
%
%
% \subsubsection{The \texttt{changebreak} style}
%
%    \begin{macrocode}
% This style option is stored in the file |thcb.sty|.
%    \begin{macrocode}
%<*thcb>
\begingroup \makeatletter
\@ifundefined{theorem@style}{\input{theorem.sty}}{}
\ProvidesFile{thcb.sty}
             [\FMithmInfo]
%    \end{macrocode}
%
% \begin{macro}{\th@changebreak}
%    The \texttt{change-break} theorem style is like \texttt{break}
%    but with interchange of theorem name and theorem number.  Thus we
%    define |\th@changebreak| as follows:
%    \begin{macrocode}
\gdef\th@changebreak{\normalfont\slshape
  \def\@begintheorem##1##2{\item
       [\rlap{\vbox{\hbox{\hskip\labelsep \theorem@headerfont ##2\ ##1}%
                    \hbox{\strut}}}]}%
\def\@opargbegintheorem##1##2##3{%
  \item[\rlap{\vbox{\hbox{\hskip\labelsep \theorem@headerfont
                     ##2\ ##1\ (##3)}%
                    \hbox{\strut}}}]}}
\endgroup
%</thcb>
%    \end{macrocode}
% \end{macro}
%
%
% \subsubsection{The \texttt{change} style}
%
% This style option is stored in the file |thc.sty|.
%    \begin{macrocode}
%<*thc>
\begingroup \makeatletter
\@ifundefined{theorem@style}{\input{theorem.sty}}{}
\ProvidesFile{thc.sty}
             [\FMithmInfo]
%    \end{macrocode}
%
% \begin{macro}{\th@change}
%    The \texttt{change} theorem style corresponds to the
%    \texttt{change break} style without a linebreak after the header.
%    To say it in another way, it's the same as the \texttt{plain}
%    style but with number and name interchanged and |\slshape| as the
%    default font.
%    \begin{macrocode}
\gdef\th@change{\normalfont\slshape
  \def\@begintheorem##1##2{\item
       [\hskip\labelsep \theorem@headerfont ##2\ ##1]}%
\def\@opargbegintheorem##1##2##3{%
  \item[\hskip\labelsep \theorem@headerfont ##2\ ##1\ (##3)]}}
\endgroup
%</thc>
%    \end{macrocode}
% \end{macro}
%
%
% \subsubsection{The \texttt{marginbreak} style}
%
% This style option is the one used most often at Mainz.  It is saved
% in the file |thmb.sty|.
%    \begin{macrocode}
%<*thmb>
\begingroup \makeatletter
\@ifundefined{theorem@style}{\input{theorem.sty}}{}
\ProvidesFile{thmb.sty}
             [\FMithmInfo]
%    \end{macrocode}
%
% \begin{macro}{\th@marginbreak}
%    The \texttt{margin break} style is nearly the same as the
%    \texttt{change break} style. The only difference is the placement
%    of the theorem number.  We use |\llap| to place it in the left
%    margin.
%
%    In this style |\labelsep| denotes the separation between the
%    number and the text.
%    \begin{macrocode}
\gdef\th@marginbreak{\normalfont\slshape
  \def\@begintheorem##1##2{\item
       [\rlap{\vbox{\theorem@headerfont
                    \hbox{\llap{##2}\hskip\labelsep ##1}%
                    \hbox{\strut}}}]}%
\def\@opargbegintheorem##1##2##3{%
  \item[\rlap{\vbox{\theorem@headerfont
                    \hbox{\llap{##2}\hskip\labelsep ##1\ (##3)}%
                    \hbox{\strut}}}]}}
\endgroup
%</thmb>
%    \end{macrocode}
% \changes{v2.1c}{90/02/17}{`endgroup outside macrocode env.}
% \end{macro}
%
%
%
% \subsubsection{The \texttt{margin} style}
%
% This style option is stored in the file |thm.sty|.
%    \begin{macrocode}
%<*thm>
\begingroup \makeatletter
\@ifundefined{theorem@style}{\input{theorem.sty}}{}
\ProvidesFile{thm.sty}
             [\FMithmInfo]
%    \end{macrocode}
%
% \begin{macro}{\th@margin}
%    Again this is only a variant of the theorem styles described
%    above without any new ideas.
%    \begin{macrocode}
\gdef\th@margin{\normalfont\slshape
  \def\@begintheorem##1##2{\item
       [\theorem@headerfont \llap{##2}\hskip\labelsep ##1]}%
\def\@opargbegintheorem##1##2##3{%
  \item[\theorem@headerfont \llap{##2}\hskip\labelsep ##1\ (##3)]}}
\endgroup
%</thm>
%    \end{macrocode}
% \end{macro}
%
%
%
%
% \subsection{Final Definitions}
%
% \begin{macro}{\theorempreskipamount}
% \begin{macro}{\theorempostskipamount}
%    The \textsf{skip} parameters that regulate the vertical empty space
%    before and after the {\envfont theorem} environment have
%    to be allocated as well.
% \changes{v2.0f}{89/07/20}{Unnecessary `global' removed.}
%    \begin{macrocode}
%<*package>
\newskip\theorempreskipamount
\newskip\theorempostskipamount
%    \end{macrocode}
%    Since we have used the same values for all theorem sets, we now can
%    assign them.
%    \begin{macrocode}
\global\setlength\theorempreskipamount{12pt plus 5pt minus 3pt}
\global\setlength\theorempostskipamount{8pt plus 3pt minus 1.5pt}
%    \end{macrocode}
% \end{macro}
% \end{macro}
%
%
% \begin{macro}{\@endtheorem}
%    The same holds for the macro |\@endtheorem|, which ends a
%    {\envfont theorem} environment. Since it is the same for all
%    theorem sets, it is removed from the macros |\th@|\meta{style}.
%    It simply ends the {\envfont trivlist} environment, which was
%    begun in |\@thm|.
%    \begin{macrocode}
\global\let\@endtheorem=\endtrivlist
%    \end{macrocode}
% \end{macro}
%
%
%
% \begin{macro}{\@preamblecmds}
%    All macros defined above are to be used only in the preamble.
%    Therefore, we insert them in |\@preamblecmds| which will disable
%    them at begin document. This is done by the internal \LaTeXe{}
%    command |\@onlypreamble|.
% \changes{v2.0g}{89/09/19}{`theoremheaderfont and `theorembodyfont
%    added.}
% \changes{v2.2a}{94/02/03}{Use \LaTeXe{} interface}
%    \begin{macrocode}
\@onlypreamble\@xnthm
\@onlypreamble\@ynthm
\@onlypreamble\@othm
\@onlypreamble\newtheorem
\@onlypreamble\theoremstyle
\@onlypreamble\theorembodyfont
\@onlypreamble\theoremheaderfont
%    \end{macrocode}
% \end{macro}
%
% Finally we declare the \texttt{plain} theorem style to be the default.
% \changes{v2.1b}{90/06/04}{Fixed missing `item by declaring default
%    style.}
%    \begin{macrocode}
\theoremstyle{plain}
%</package>
%    \end{macrocode}
%
% \Finale

\endinput
}{}
\ProvidesFile{thcb.sty}
             [\FMithmInfo]
%    \end{macrocode}
%
% \begin{macro}{\th@changebreak}
%    The \texttt{change-break} theorem style is like \texttt{break}
%    but with interchange of theorem name and theorem number.  Thus we
%    define |\th@changebreak| as follows:
%    \begin{macrocode}
\gdef\th@changebreak{\normalfont\slshape
  \def\@begintheorem##1##2{\item
       [\rlap{\vbox{\hbox{\hskip\labelsep \theorem@headerfont ##2\ ##1}%
                    \hbox{\strut}}}]}%
\def\@opargbegintheorem##1##2##3{%
  \item[\rlap{\vbox{\hbox{\hskip\labelsep \theorem@headerfont
                     ##2\ ##1\ (##3)}%
                    \hbox{\strut}}}]}}
\endgroup
%</thcb>
%    \end{macrocode}
% \end{macro}
%
%
% \subsubsection{The \texttt{change} style}
%
% This style option is stored in the file |thc.sty|.
%    \begin{macrocode}
%<*thc>
\begingroup \makeatletter
\@ifundefined{theorem@style}{% \iffalse meta-comment
%
% Copyright (C) 1993-2024
%
% The LaTeX Project and any individual authors listed elsewhere
% in this file.
%
% This file is part of the Standard LaTeX `Tools Bundle'.
% -------------------------------------------------------
%
% It may be distributed and/or modified under the
% conditions of the LaTeX Project Public License, either version 1.3c
% of this license or (at your option) any later version.
% The latest version of this license is in
%    https://www.latex-project.org/lppl.txt
% and version 1.3c or later is part of all distributions of LaTeX
% version 2005/12/01 or later.
%
% The list of all files belonging to the LaTeX `Tools Bundle' is
% given in the file `manifest.txt'.
%
% \fi
%
%%
%
%
\def\FMithmInfo{2023/07/05 v2.2c Theorem extension package (FMi)}
%
% \ProvidesFile{theorem.dtx}[\FMithmInfo]
%\iffalse   % this is a METACOMMENT !
%%
%% Package `theorem' to use with LaTeX2e
%% Copyright (C) 1989-1995 Frank Mittelbach, all rights reserved.
%<+package>\@ifundefined{theorem@style}{}{\endinput}
%<+package>\NeedsTeXFormat{LaTeX2e}[1995/06/01]
%<+package>\ProvidesPackage{theorem}
%<+package>         [\FMithmInfo]
%
% \fi
%
% \changes{v1.0}{87/05/07}{First draft.}
% \changes{v1.0a}{87/05/07}{`thistheoremstyle' added.}
% \changes{v1.1a}{87/07/29}{New concept: redefine the internal
%                          \protect\LaTeX{} commands `@xnthm' etc.}
% \changes{v1.1b}{87/07/31}{`theoremfont' used to specify font.}
% \changes{v1.2a}{88/01/08}{`@endtheorem' now globally defined.}
% \changes{v1.2b}{88/01/11}{`newtheorem' only allowed in the preamble.}
% \changes{v1.3a}{88/02/14}{`theoremfont' now a toks register.}
%
% \changes{v2.0a}{88/11/16}{New implementation and new documentation.}
% \changes{v2.0a}{88/11/16}{No layout changes but internal modifications
%                           old substyles won't work.}
% \changes{v2.0b}{89/01/30}{`global' used for all definitions.}
% \changes{v2.0c}{89/06/15}{Documentation translated to English.
%                            Thanks to LLL and Nina}
% \changes{v2.1a}{90/02/15}{Support for new font selection.}
% \changes{v2.2a}{94/02/03}{Cleanup for LaTeX2e}
%
% \DoNotIndex{\@endparenv,\@ifdefinable,\@ifnextchar,\@ifundefined}
% \DoNotIndex{\@namedef,\@nameuse,\@spaces,\@tempa,\@warning}
% \DoNotIndex{\ }
% \DoNotIndex{\begingroup,\bfseries}
% \DoNotIndex{\csname}
% \DoNotIndex{\def,\do,\docdate}
% \DoNotIndex{\endcsname,\endgroup,\endinput,\endtrivlist,\expandafter}
% \DoNotIndex{\filedate,\fileversion}
% \DoNotIndex{\gdef,\global}
% \DoNotIndex{\hbox,\hskip}
% \DoNotIndex{\ignorespaces,\input,\itshape,\item}
% \DoNotIndex{\labelsep,\let,\llap}
% \DoNotIndex{\makeatletter}
% \DoNotIndex{\noexpand}
% \DoNotIndex{\refstepcounter,\relax,\rlap}
% \DoNotIndex{\setlength,\slshape,\space,\string,\strut}
% \DoNotIndex{\the,\trivlist,\typeout}
% \DoNotIndex{\vbox}
% \DoNotIndex{\xdef}
%
% \def\deflabel#1{\ttfamily #1\hfill}
% \def\deflist#1{\begin{list}{}{\settowidth\labelwidth{\ttfamily #1}%
%                               \setlength\leftmargin\labelwidth
%                               \addtolength\leftmargin\labelsep
%                               \let\makelabel\deflabel}}
% \def\enddeflist{\end{list}}
% \def\envfont{\normalfont\ttfamily}
%
% \MakeShortVerb{\|}
%
% \newcommand{\TUB}{{\sl TUGboat\/}}
%
% \GetFileInfo{theorem.dtx}
% \title{An Extension of the
%        \LaTeX{} theorem environment\thanks{This file
%        has version number \fileversion, last
%        revised \filedate.}}
% \author{Frank Mittelbach}
%
% \MaintainedByLaTeXTeam{tools}
% \maketitle
%
% \begin{abstract}
%    The macros described in this paper yield an extension of the
%    \LaTeX{} theorem mechanism. It is designed to satisfy the
%    different requirements of various journals. Thus, the
%    layout of the ``theorems'' can be manipulated by determining a
%    ``style''. This article describes not only the use, but
%    also the definition, of the necessary macros.
% \end{abstract}
%
% \section*{Preface to version 2.2}
%
% For \LaTeXe{} this package did not need any fundamental changes. I
% only modified the messages generated so that theorem layout styles
% will show up with the |\listfiles| command and cleaned the section
% on the New Font Selection Scheme since this is now included in
% \LaTeX.
%
% \section*{Preface to version 2.1}
%
% This version is identical to 2.0g described in \TUB\ 10\#3 except
% for some internal defaults which are now set depending on the used
% font selection scheme.
%
% This was done to avoid unpleasant surprises if the new font selection
% scheme is in force. For further details see
% section~\ref{sec:newfonts} and \cite{bk:GMS94}.
%
% \section{Introduction}
%
% For our purposes here, ``theorems'' are labelled enunciations,
% often set off from the main text by extra space and a font change.
% Theorems, corollaries, conjectures, definitions, and remarks
% are all instances of ``theorems''.  The ``header'' of these structures
% is composed of a label (such as \textsc{Theorem} or \textsc{Remark})
% and a number which serializes an item in the sequence of items
% with the same label.
%
% Shortly after the introduction of \LaTeX{} at the Fachbereich
% Mathematik in Mainz, the desire to manipulate the layout of
% ``theorems'' arose.  In Mainz, the following two conventions came
% into general use:
% \begin{enumerate}
%   \item The number of the theorem is shown in the margin.
%   \item There is a line break at
%         the end of the theorem header.
% \end{enumerate}
%
% Additionally, some journals require different formats which depend on
% the ``sort of theorem'': e.g.\ often remarks and definitions are set
% in |\upshape|, while |\itshape| is employed for main theorems.
%
% Confronted with these requirements, a theorem environment was
% developed in Mainz which allows separate determination of the
% layout of the ``theorems sets'', comparable to |\pagestyle|.
%
%
% \section{The user interface}
% \subsection{Defining new theorem sets}
%
% \DescribeMacro\newtheorem
% As in the original \LaTeX{} version, the command |\newtheorem|
% defines a new ``theorem set'' or ``theorem-like structure''.
% Two required arguments name the new environment and give the text
% to be typeset with each instance of the new ``set'', while
% an optional argument determines how
% the ``set'' is enumerated:
% \begin{description}
%    \item[\ttfamily \bslash newtheorem\{foo\}\{bar\}]
%       The theorem set {\envfont foo} (whose name is \texttt{bar})
%       uses its own counter.
%    \item[\ttfamily \bslash newtheorem\{foo2\}{[foo]}\{bar2\}]
%       The theorem set {\envfont foo2} (printed name \texttt{bar2})
%       uses the same counter as the theorem set \texttt{foo}.
%    \item[\ttfamily \bslash newtheorem\{foo3\}\{bar3\}{[section]}]
%       The theorem set {\envfont foo3} (printed name \texttt{bar3}) is
%       enumerated within the counter \texttt{section}, i.e.\ with every
%       new |\section| the enumeration begins again with 1, and
%       the enumeration is composed from the section-number and the
%       theorem counter itself.
% \end{description}
%
% \DescribeMacro\theoremstyle
% Additionally, the command |\theoremstyle| can define the layout
% of various, or all, theorem sets.  It should be noted that any
% theorem set defined by |\newtheorem| is typeset in the
% |\theoremstyle| that is current at the time of the definition.
% Thus, the following
% \begin{verbatim}
%    \theoremstyle{break}        \newtheorem{Cor}{Corollary}
%    \theoremstyle{plain}        \newtheorem{Exa}{Example}[section]
%\end{verbatim}                             ^^A No blank at line start
% leads to the result that the set {\envfont Cor} is formatted in the
% style \texttt{break}, while the set {\envfont Exa} and all the
% following ones are formatted in the style \texttt{plain}, unless
% another |\theoremstyle| follows.  Since the definitions installed by
% |\newtheorem| are global, one also can limit |\theoremstyle| locally
% by grouping braces.
%
% \DescribeMacro\theorembodyfont
% The choice of the font for the theorem body is completely
% independent of the chosen |\theoremstyle|; this has proven to
% be very advantageous.  For example,
% \begin{verbatim}
%   {\theorembodyfont{\upshape}          \newtheorem{Rem}{Remark}}
%\end{verbatim}
% defines a theorem set {\envfont Rem}, which will be set in
% |\upshape| in the current layout (which in our example is
% \texttt{plain}).  As with |\theoremstyle|, the |\theorembodyfont|
% chosen is that current at the time of |\newtheorem|.  If
% |\theorembodyfont| is not specified or one defines
% |\theorembodyfont{}|, then the font used will be that defined by the
% |\theoremstyle|.
%
% \DescribeMacro\theoremheaderfont It is also possible to customize
% the font used for the theorem headers.  This is, however, a global
% declaration, and therefore there should be at most one
% |\theoremheaderfont| declaration in the preamble.\footnote{If it is
% actually necessary to have different header fonts, one has to define
% new theorem styles (substituting the desired font) or specify the
% information directly in the \texttt{\bslash newtheorem} declaration
% (the unclean variant).}
%
% \DescribeMacro\theorempreskipamount
% \DescribeMacro\theorempostskipamount Two additional parameters
% affect the vertical space around the theorem environments:
% |\theorempreskipamount| and |\theorempostskipamount| define,
% respectively, the spacing before and after such an environment.
% These parameters apply for all theorem sets and can be manipulated
% with the ordinary length macros.  They are rubber lengths
% (`\textsf{skips}'), and therefore can contain \texttt{plus} and
% \texttt{minus} parts.
%
% Since the definition of theorem sets should---most sensibly---be
% placed in the preamble, we only allow installation there.
% It is therefore possible to release the memory used here after
% |\begin{document}|, in order to make room for other
% applications.
%
%
% \subsection{Existing theorem styles}
%
% The following theorem styles exist to date:
% \begin{deflist}{marginbreak}
%    \item[plain]
%       This theorem style emulates the original \LaTeX{} definition,
%       except that additionally the parameters
%       |\theorem...skipamount| are used.
%    \item[break]
%       In this style, the theorem header is followed by a line
%       break.
%    \item[marginbreak]
%       The theorem number is set in the margin, and there is a line
%       break as in \texttt{break}.
%    \item[changebreak]
%       Like \texttt{break}, but with header number and text
%       interchanged.
%    \item[change]
%       Header number and text are interchanged, without a line break.
%    \item[margin]
%       The number is set in the left margin, without a line break.
% \end{deflist}
% All styles (except \texttt{plain}) select |\slshape| as the default
% |\theorembodyfont|.
%
%
% \subsection{Examples}
%
% Given the above theorem sets {\envfont Cor}, {\envfont Exa}
% and {\envfont Rem},
% suppose that the preamble also contains the declarations:
% \begin{verbatim}
%    \theoremstyle{marginbreak}   \newtheorem{Lem}[Cor]{Lemma}
%    \theoremstyle{change}
%    \theorembodyfont{\itshape}        \newtheorem{Def}[Cor]{Definition}
%
%    \theoremheaderfont{\scshape}
%\end{verbatim}
% Then the following are some typical examples of the typeset output
% resulting from their use.
%
% \begin{Cor}
%    This is a sentence typeset in the theorem environment {\envfont
%    Cor}.
% \end{Cor}
%
% \begin{Exa}
%    This is a sentence typeset in the theorem environment {\envfont
%    Exa}.
% \end{Exa}
%
% \begin{Rem}
%    This is a sentence typeset in the theorem environment {\envfont
%    Rem}.
% \end{Rem}
%
% \begin{Lem}[Ben User]
%    This is a sentence typeset in the theorem environment {\envfont
%    Lem}.
% \end{Lem}
%
% \begin{Def}[Very impressive Definition]
%    This is a sentence typeset in the theorem environment {\envfont
%    Def}.
% \end{Def}
%
% The last two examples show the effect of the optional argument to
% a theorem environment (it is the text typeset in parentheses).
%
%
% \section{Special Considerations}
% \label{sec:newfonts}
%
% Theoremheader and body are implemented as a unit. This means that
% the |\theoremheaderfont|  will inherit characteristics of the
% |\theorembodyfont| in \LaTeXe. Thus, if for example
% |\theorembodyfont| is |\itshape| and |\theoremheaderfont|
% is |\bfseries| the font selected for the header will have the
% characteristics `bold extended italic'. If this is not desired one
% should set the |\theoremheaderfont| to something like
% \begin{verbatim}
%      \theoremheaderfont{\normalfont\bfseries}
%\end{verbatim}
% i.e.\ supplying all necessary font information explicitly.
%
%
%
% \section{Acknowledgements}
%
% The publication of this set of macros was only possible with the
% help of Christina Busse (translating the manuscript into English),
% Joachim Pense (playing the r\^ole of typist), Chris Rowley
% (looking everything over) and many others providing useful
% suggestions.
%
% \MaybeStop{%
%    \begin{thebibliography}{9}
%       \bibitem{bk:GMS94} \textsc{M.~Goossens}, \textsc{F.~Mittelbach}
%         and \textsc{A.~Samarin}.
%         \newblock The \LaTeX{} Companion.
%         \newblock
%         Addison-Wesley, Reading, Massachusetts, 1994.
%       \bibitem{latex:source} \textsc{Lamport, Leslie.} \newblock
%                               \texttt{latex.tex}, version 2.09, date
%                               Feb. 1990.
%    \end{thebibliography}
%    \PrintIndex
%    \PrintChanges}
%
%
% \section{The documentation driver file}
%
% The next bit of code contains the documentation driver file for
% \TeX{}, i.e., the file that will produce the documentation you are
% currently reading. It will be extracted from this file by the
% \texttt{docstrip} program. Since it is the first code in the file
% one can alternatively process this file directly with \LaTeXe{} to
% obtain the documentation.
%
%    \begin{macrocode}
%<*driver>
\documentclass{ltxdoc}

\usepackage{theorem}

%<+driver>% The next few lines define theorem sets which are used
%<+driver>% in the example section of the documentation.

\theoremstyle{break}       \newtheorem{Cor}{Corollary}
\theoremstyle{plain}       \newtheorem{Exa}{Example}[section]
{\theorembodyfont{\upshape}\newtheorem{Rem}{Remark}}
\theoremstyle{marginbreak} \newtheorem{Lem}[Cor]{Lemma}
\theoremstyle{change}
\theorembodyfont{\itshape} \newtheorem{Def}[Cor]{Definition}

\theoremheaderfont{\scshape}

\RecordChanges

\begin{document}
   \DocInput{theorem.dtx}
\end{document}
%</driver>
%    \end{macrocode}
%
% \section{Definition of the Macros}
%
% If the file has been loaded before, we abort immediately. If not the
% package announces itself (this is actually done at the very top of
% the file---the way it is done isn't good style so don't copy it).
% \changes{v2.0e}{89/07/19}{Spaces removed from `typeout'.}
% \changes{v2.2b}{95/11/19}{Announce moved to top of file}
%    \begin{macrocode}
%<*package>
%\@ifundefined{theorem@style}{}{\endinput}
%\def\FMithmInfo{1995/11/19 v2.2b Theorem extension package (FMi)}
%\ProvidesPackage{theorem}[\FMithmInfo]
%    \end{macrocode}
%
%
% \subsection{Definition of theorem styles and fonts}
%
% All the definitions in this file are done globally to allow
% inputting this file inside a group.
%
% \begin{macro}{\theoremstyle}
%    Before a theorem style can be installed, the chosen style must be
%    known. For that reason, we must test to see that
%    |\th@|\meta{style} is known or, more precisely, that it is
%    different from |\relax|. If the style is not known then
%    |\th@plain| is used.
%    \begin{macrocode}
\gdef\theoremstyle#1{%
   \@ifundefined{th@#1}{\@warning
          {Unknown theoremstyle `#1'. Using `plain'}%
          \theorem@style{plain}}%
%    \end{macrocode}
%    We save the theorem style to be used in the token register
%    |\theorem@style|.
%    \begin{macrocode}
      {\theorem@style{#1}}%
%    \end{macrocode}
%    Now we ``evaluate'' the theorem style: this means, we call the
%    macro |\th@|\meta{style} which will activate the relevant
%    definitions which are contained in a separate file.  This is done
%    in a group to suppress changes to the current font.  This could
%    otherwise pose problems together with the new font selection
%    scheme\footnote{When I printed the original article using the new
%    font selection scheme I ended with a document with slanted
%    typefaces (text headings and all) simply because one of the
%    theorem styles used \texttt{\bslash sl} at toplevel.} if the
%    |\th@|\meta{style} is evaluated a second time.
% \changes{v2.1a}{90/02/15}{Group added around `th@..}
% \changes{v2.1a}{90/02/15}{Reseting of `..@begintheorem removed.}
%    \begin{macrocode}
      \begingroup
        \csname th@\the\theorem@style \endcsname
      \endgroup}
%    \end{macrocode}
% \end{macro}
%
%
% \begin{macro}{\@begintheorem}
% \changes{v2.1a}{90/02/15}{Relaxed at toplevel.}
% \begin{macro}{\@opargbegintheorem}
% \changes{v2.1a}{90/02/15}{Relaxed at toplevel.}
%    We reset |\@begintheorem| and |\@opargbegintheorem| to
%    |\relax| since these commands are no longer necessary at
%    toplevel. This will save a few tokens.
%    \begin{macrocode}
\global\let\@begintheorem\relax
\global\let\@opargbegintheorem\relax
%    \end{macrocode}
% \end{macro}
% \end{macro}
%
% \begin{macro}{\theorem@style}
%    Obviously the \textsf{token} register used above has to be
%    allocated. To assure the utmost compatibility with the
%    original \LaTeX{} definition, we set the default theorem style to
%    \texttt{plain}, which implements the usual \LaTeX{} convention.
% \changes{v2.0f}{89/07/20}{Unnecessary `global' removed.}
%    \begin{macrocode}
\newtoks\theorem@style
\global\theorem@style{plain}
%    \end{macrocode}
% \end{macro}
%
% \begin{macro}{\theorembodyfont}
% \begin{macro}{\theorem@bodyfont}
%    If the |\theorembodyfont| is set by the user then it should not
%    interact with the default font set in the theorem style. When the
%    new font selection is in force this may happen if, for example,
%    the default is |\itshape| and the new |\theorembodyfont| is
%    |\sffamily|. So we add a |\reset@font| command in front of the
%    user definition.
% \changes{v2.0f}{89/07/20}{Unnecessary `global' removed.}
% \changes{v2.0g}{89/09/19}{`theoremfont to `theorembodyfont.}
% \changes{v2.1c}{92/06/22}{`reset@font added and logic changed.}
% \changes{v2.1d}{92/10/28}{Reset to empty if arg is empty}
%    \begin{macrocode}
\gdef\theorembodyfont#1{%
%    \end{macrocode}
%    We check if the argument supplied is empty and if so put nothing
%    into the |\theorem@bodyfont| token register to allow for
%    |\theorembodyfont{}| as a mean of using the default of the
%    current |\theoremstyle|.
%    \begin{macrocode}
   \def\@tempa{#1}%
   \ifx\@tempa\@empty
    \theorem@bodyfont{}%
   \else
    \theorem@bodyfont{\reset@font#1}%
   \fi
   }
\newtoks\theorem@bodyfont
\global\theorem@bodyfont{}
%    \end{macrocode}
% \end{macro}
% \end{macro}
%
%
% \begin{macro}{\theoremheaderfont}
% \changes{v2.0g}{89/09/19}{Macro added.}
%    The font for the theorem headers is handled differently because
%    this definition applies to all theorem styles.
%    \begin{macrocode}
\gdef\theoremheaderfont#1{\gdef\theorem@headerfont{#1}%
%    \end{macrocode}
%    After using the macro once it is redefined to produce an error
%    message.
%    \begin{macrocode}
       \gdef\theoremheaderfont##1{%
        \typeout{\string\theoremheaderfont\space should be used
                 only once.}}}
%    \end{macrocode}
% \end{macro}
%
% \begin{macro}{\theorem@headerfont}
% \changes{v2.0g}{89/09/19}{Macro added.}
%    To set the |\theorem@headerfont| default we first test if
%    the new fontselection scheme is in force.
% \changes{v2.1a}{90/02/15}{Support for new font selection.}
%    \begin{macrocode}
\ifx\upshape\undefined
%    \end{macrocode}
%    If not we define it to expand into |\bfseries|. We don't use
%    |\let| just in case a following style option redefines this
%    macro.
%    \begin{macrocode}
\gdef\theorem@headerfont{\bfseries}
%    \end{macrocode}
%    Otherwise we reset the current shape before calling |\bfseries|.
%    \begin{macrocode}
\else \gdef\theorem@headerfont{\normalfont\bfseries}\fi
%    \end{macrocode}
% \end{macro}
%
% \begin{macro}{\th@plain}
% \begin{macro}{\th@break}
% \begin{macro}{\th@marginbreak}
% \begin{macro}{\th@changebreak}
% \begin{macro}{\th@change}
% \begin{macro}{\th@margin}
%    The different styles are defined in macros such as
%    |\th@plain|.  Since memory space is precious in
%    ``non-Big-versions'', we have to avoid offering too many unused
%    definitions.  Therefore we define these styles in separate files
%    that can be loaded on demand.  Thus the commands themselves only
%    load these files. We use |\@input@|, a \LaTeXe{} internal command
%    that ensures that the file will be listed with |\listfiles|.
% \changes{v2.2a}{94/02/02}{Use `@input@ to load theorem layout files}
%    \begin{macrocode}
\gdef\th@plain{\@input@{thp.sty}}
\gdef\th@break{\@input@{thb.sty}}
\gdef\th@marginbreak{\@input@{thmb.sty}}
\gdef\th@changebreak{\@input@{thcb.sty}}
\gdef\th@change{\@input@{thc.sty}}
\gdef\th@margin{\@input@{thm.sty}}
%    \end{macrocode}
%    This list will be expanded when new styles become
%    available.  For testing,  just append new theorem substyles
%    as document options.
% \end{macro}
% \end{macro}
% \end{macro}
% \end{macro}
% \end{macro}
% \end{macro}
%
%
%
% \subsection{Definition of a new theorem set}
%
% As already pointed out, a new theorem environment can be defined in
% three different ways:
% \begin{verbatim}
%    \newtheorem{Lem}{Lemma}
%    \newtheorem{Lem}{Lemma}[section]
%    \newtheorem{Lem}[Theorem]{Lemma}
%\end{verbatim}
% The function of the macro |\newtheorem| is to recognize these cases
% and then to branch into one of the three macros |\@ynthm|, |\@xnthm|
% or |\@othm|.  This mechanism is adopted unchanged from
% \cite{latex:source}; the essential point here is that, for example,
% in the second case, the arguments \texttt{Lem}, \texttt{Lemma} and
% \texttt{section} are passed over to the macro |\@xnthm|.
%
% We inspect this case first because the others present fewer problems,
% and thus are easily derived from this one.
%
% \begin{macro}{\@xnthm}
%    For our example arguments, the macro |\@xnthm| must fulfill
%    the following:
%    \begin{itemize}
%       \item
%          Define a new \LaTeX-counter `Lem'
%       \item
%          reset this counter within a |\section|
%       \item
%          define the macro |\theLem|
%       \item
%          define the environment macros |\Lem| and
%          |\endLem| using the current
%          |\theoremstyle| and |\theorem@bodyfont|.
%    \end{itemize}
%    Obviously, all this should happen only if the first argument
%    of |\@xnthm| (i.e.\ \texttt{Lem} in our example) is chosen so
%    as not to conflict with any previously defined commands or
%    environments.  This test is performed by the \LaTeX{} macro
%    |\@ifdefinable|.
%    \begin{macrocode}
\gdef\@xnthm#1#2[#3]{\expandafter\@ifdefinable\csname #1\endcsname
%    \end{macrocode}
%    Therefore, the first argument of |\@ifdefinable| is the
%    expansion (in the example, |\Lem|) of
%    |\csname||#1\endcsname|.  The second argument is
%    executed only if the test has been completed successfully.
%    \begin{macrocode}
   {%
%    \end{macrocode}
%    Now we define the new counter. The names of the \LaTeX{}
%    macros employed should speak for themselves:
% \changes{v2.2b}{1995/10/07}{Use \cs{@newctr}}
% \changes{v2.2c}{1995/11/23}{Use \cs{@newctr} properly.}
%    \begin{macrocode}
    \@definecounter{#1}\@newctr{#1}[#3]%
%    \end{macrocode}
%    Using |\@newctr| will give a proper error message if the counter
%    in |#3| is not defined.
%    In defining `|\theLem|' we must generate the desired macro
%    name by use of |\expandafter| and |\csname|.
%    \begin{macrocode}
    \expandafter\xdef\csname the#1\endcsname
%    \end{macrocode}
%    An |\xdef| is used in order to make the definition global,
%    and to ensure that it contains the replacement texts of
%    |\@thmcountersep| and |\@thmcounter|.\footnote{These
%    two macros can be defined by the document style.  Their default
%    values produce a `\texttt{.}' as separation and an arabic
%    representation of the number.} However, not everything should be
%    expanded. For example, it saves space to use |\thesection|
%    instead of its---at times---lengthy expansion.
%    \begin{macrocode}
      {\expandafter \noexpand \csname the#3\endcsname
       \@thmcountersep \@thmcounter{#1}}%
%    \end{macrocode}
%    Thus with the defaults of \LaTeX{}, |\theLem| would be
%    replaced by the command sequence |\thesection.\arabic{Lem}|.
%
%    We will now look at the definition of the macro which is executed
%    at the beginning of the actual environment (in our example this
%    macro is |\Lem|). It should be noted that we use an
%    ``|\expandafter| trick'' to expand only certain parts of the
%    replacement text at the time of the definition.
%    \begin{macrocode}
    \def\@tempa{\global\@namedef{#1}}%
    \expandafter \@tempa \expandafter{%
%    \end{macrocode}
%    First, the macro that contains the current definitions of
%    |\@begintheorem| and |\@opargtheorem| should be called
%    up.  The name of this macro---as is already known---has the form
%    |\th@|\meta{theorem\ style}; therefore, it must be called by
%    \begin{macrocode}
      \csname th@\the \theorem@style
%    \end{macrocode}
%    In addition the default theorem font should be changeable, i.e.\
%    we have to insert the contents of |\theorem@bodyfont|. For
%    that reason, we expand even further, beyond |\endcsname|,
%    and thus insert the contents of the \textsf{token} register
%    |\theorem@bodyfont| in the replacement text.
%    \begin{macrocode}
            \expandafter \endcsname \the \theorem@bodyfont
%    \end{macrocode}
%    Now it is time to call the macro |\@thm| which takes over
%    the further processing.  It has two arguments: the current
%    counter name (in our example, \texttt{Lem}), and the text of the
%    label (in our example, \texttt{Lemma}).
%    \begin{macrocode}
     \@thm{#1}{#2}}%
%    \end{macrocode}
%    With this, the `sub-definition' is complete.  The macro
%    |\@endtheorem| ends a theorem environment and is, so far, nothing
%    but an |\endtrivlist|.  (Hence it is defined globally, and not
%    within the theorem styles.\footnote{This has to be changed as
%    soon as theorem styles that change \texttt{\bslash @endtheorem}
%    exist.  In such a case, all existing styles must be changed as
%    well since they will have to reset the macro.}) Therefore, we can
%    set it equivalent to the macro that ends the theorem set (in our
%    example, |\endLem|).  However, if some day theorem styles exist
%    that do change |\@endtheorem|, we would have to use the
%    commented-out line instead.
%    \begin{macrocode}
    \global \expandafter \let \csname end#1\endcsname \@endtheorem
%   \global\@namedef{end#1}{\@endtheorem}%
%    \end{macrocode}
%    With these commands all the required definitions are employed,
%    unless the test |\@ifdefinable| has failed.  Therefore, we
%    end the second argument of this macro and with it the definition
%    of |\@xnthm|.
%    \begin{macrocode}
   }}
%    \end{macrocode}
% \end{macro}
%
% \begin{macro}{\@ynthm}
%    The definition of |\@ynthm| is completely analogous. In
%    this case the new counter that is defined is not reset within
%    another counter; thus the definition of |\the...|
%    is simplified:
%    \begin{macrocode}
\gdef\@ynthm#1#2{\expandafter\@ifdefinable\csname #1\endcsname
   {\@definecounter{#1}%
    \expandafter\xdef\csname the#1\endcsname{\@thmcounter{#1}}%
%    \end{macrocode}
%    The rest of the definition corresponds literally to that of
%    |\@xnthm|:
%    \begin{macrocode}
    \def\@tempa{\global\@namedef{#1}}\expandafter \@tempa
     \expandafter{\csname th@\the \theorem@style \expandafter
     \endcsname \the\theorem@bodyfont \@thm{#1}{#2}}%
    \global \expandafter \let \csname end#1\endcsname \@endtheorem}}
%    \end{macrocode}
% \end{macro}
%
% \begin{macro}{\@othm}
%    The definition of |\@othm| does not contain anything new.
%    \begin{macrocode}
\gdef\@othm#1[#2]#3{%
%    \end{macrocode}
% \changes{v2.2b}{95/10/07}{Build in check for counter defined}
%    We do not define a new counter but instead use one that has
%    already been defined.  Thus the only definition we need is that
%    of this pseudo-counter (i.e.\ |\the|\meta{env.\ name}).
%    First we check if |#2| corresponds to a known counter name.
%    \begin{macrocode}
  \expandafter\ifx\csname c@#2\endcsname\relax
   \@nocounterr{#2}%
  \else
   \expandafter\@ifdefinable\csname #1\endcsname
   {\expandafter \xdef \csname the#1\endcsname
     {\expandafter \noexpand \csname the#2\endcsname}%
%    \end{macrocode}
%    All other parts of the definition can be adopted from
%    |\@xnthm|. We have to remember, though, that in this case
%    the name of the current counter and the theorem label have moved
%    to the second and third arguments.
%    \begin{macrocode}
    \def\@tempa{\global\@namedef{#1}}\expandafter \@tempa
     \expandafter{\csname th@\the \theorem@style \expandafter
     \endcsname \the\theorem@bodyfont \@thm{#2}{#3}}%
    \global \expandafter \let \csname end#1\endcsname \@endtheorem}%
  \fi}
%    \end{macrocode}
% \end{macro}
%
% \subsection{Macros that are employed in a {\envfont theorem}
%    environment}
%
% \begin{macro}{\@thm}
%    The macro |\@thm| has to increase the current counter.  Then,
%    depending on whether the environment has (or does not have) an
%    optional argument, it has to branch into either |\@begintheorem|
%    or |\@opargtheorem|.
%    \begin{macrocode}
\gdef\@thm#1#2{\refstepcounter{#1}%
%    \end{macrocode}
%    Now we start a {\envfont trivlist} environment, and give
%    |\@topsep| and |\@topsepadd| the values of the \textsf{skip}
%    registers |\theorempreskipamount| and |\theorempostskipamount|.
%    The value in |\@topsep| is the vertical space that is inserted by
%    the first (and only) |\item| in our |\trivlist| whilst
%    |\@topsepadd| is inserted by |\@endparenv| at the end of that
%    {\envfont trivlist} environment.  By using these registers, we
%    obtain the desired space around a {\envfont theorem} environment.
%    \begin{macrocode}
   \trivlist
   \@topsep \theorempreskipamount               % used by first \item
   \@topsepadd \theorempostskipamount           % used by \@endparenv
%    \end{macrocode}
%    Now we have to test whether an optional argument has been given.
% \changes{v2.0d}{89/07/18}{Forgotten percent added.}
%    \begin{macrocode}
   \@ifnextchar [%
%    \end{macrocode}
%    If there is an optional argument, we will call |\@ythm|, and
%    move the arguments read back into the input stream.
%    \begin{macrocode}
   {\@ythm{#1}{#2}}%
%    \end{macrocode}
%    If not, we call |\@begintheorem|. Its first argument is the
%    name of the theorem set (hence the second argument of
%    |\@thm|). Its second argument is the macro that produces the
%    current number.
%    \begin{macrocode}
   {\@begintheorem{#2}{\csname the#1\endcsname}\ignorespaces}}
%    \end{macrocode}
% \end{macro}
%
% \begin{macro}{\@xthm}
% \begin{macro}{\@ythm}
%    Both these macros were originally called by |\@thm|.  We do
%    not need |\@xthm| anymore, hence we reset it to
%    |\relax|. The definition of |\@ythm| has not changed at
%    all from its definition in \LaTeX. In order to make the macros
%    easier to understand, we will
%    nevertheless present it (commented out).
%    \begin{macrocode}
\global\let\@xthm\relax
% \def\@ythm#1#2[#3]{\@opargbegintheorem{#2}{\csname
%           the#1\endcsname}{#3}\ignorespaces}
%</package>
%    \end{macrocode}
%    The primitive |\ignorespaces| in |\@ythm| and
%    |\@thm| is needed to remove the spaces between the
%    |\begin{...}| and the actual text.
% \end{macro}
% \end{macro}
%
%
% \subsection{Definition of the theorem substyles}
%
% As already pointed out, the theorem substyles, defined below, are
% only loaded when necessary.  Note that all these substyles, except
% \texttt{plain}, have |\slshape| as the default body font.
%
% \subsubsection{The \texttt{plain} style}
%
% \label{dok:th@plain} As the following macros use |@|, we have
% to locally set the |\catcode| of this symbol to ``letter''.
% This happens within a group, so that we do not have to worry about
% which |\catcode| that symbol had before.
%    \begin{macrocode}
%<*thp>
\begingroup \makeatletter
%    \end{macrocode}
% Since we are now within a group, we must make all definitions
% globally.  First we make sure that \texttt{theorem.sty} is loaded.
% This will allow us to use this file as a document style option
% without having to call \texttt{theorem} itself as an option.  At the
% same time, we assure that at least version 2 is loaded, since
% |\theorem@style| was not defined in earlier versions.
%    \begin{macrocode}
\@ifundefined{theorem@style}{\input{theorem.sty}}{}
\ProvidesFile{thp.sty}
             [\FMithmInfo]
%    \end{macrocode}
%
%
% \begin{macro}{\th@plain}
% \begin{macro}{\@begintheorem}
% \begin{macro}{\@opargbegintheorem}
%    |\theoremstyle{plain}| corresponds to the original
%    definition, except that the distances to the surrounding text are
%    determined by the parameters |\theorempreskipamount| and
%    |\theorempostskipamount|.
%    First we set the default body font.
%    \begin{macrocode}
\gdef\th@plain{\normalfont\itshape
%    \end{macrocode}
%    Then we define |\@begintheorem| and
%    |\@opargbegintheorem|. These two macros define how the
%    header of a theorem is typeset.  |\@opargbegintheorem| will
%    be called if a \textsf{theorem} environment with an optional
%    argument is encountered; otherwise, the header is constructed by
%    calling |\@begintheorem|.  If one of these macros is
%    executed, we are within a \textsf{trivlist} environment started by
%    |\@thm|.  So the theorem header is produced with an
%    |\item| command.
%
%    Instead of specifying the header font directly, all standard
%    theorem styles use the |\theorem@headerfont| macro to allow
%    customization.  The extra space (|\labelsep|) is necessary
%    because of problems in the \textsf{trivlist} environment.
%    \begin{macrocode}
  \def\@begintheorem##1##2{%
        \item[\hskip\labelsep \theorem@headerfont ##1\ ##2]}%
%    \end{macrocode}
%    The definition of |\@opargbegintheorem| is completely
%    analogous. The only difference is the fact that there exists a
%    third argument (which is the optional parameter of the
%    environment and contains additional information about the
%    theorem).  Customarily we enclose it in parentheses.
%    \begin{macrocode}
\def\@opargbegintheorem##1##2##3{%
   \item[\hskip\labelsep \theorem@headerfont ##1\ ##2\ (##3)]}}
%    \end{macrocode}
% \end{macro}
% \end{macro}
% \end{macro}
%
% \noindent
% We conclude with an |\endgroup| to restore the
% |\catcode| of |@|.
%    \begin{macrocode}
\endgroup
%</thp>
%    \end{macrocode}
%
%
% \subsubsection{The \texttt{break} style}
%
% This style option is stored in the file |thb.sty|.
% For the next two lines see the documentation for |\th@plain| on
% page~\pageref{dok:th@plain}.
%    \begin{macrocode}
%<*thb>
\begingroup \makeatletter
\@ifundefined{theorem@style}{\input{theorem.sty}}{}
\ProvidesFile{thb.sty}
             [\FMithmInfo]
%    \end{macrocode}
%
% \begin{macro}{\th@break}
%    |\theoremstyle{break}| produces a line break after the name
%    of the theorem.  The font is |\slshape|.  Hence, we define
%    |\th@break| as follows:
%    \begin{macrocode}
\gdef\th@break{\normalfont\slshape
  \def\@begintheorem##1##2{\item[%
%    \end{macrocode}
%    We run into the following problem: it is not possible to create
%    the header with |\item[|\meta{title}|]| and then start
%    a new line by, for example, |\mbox{}\\|. Such a definition
%    will fail whenever a list environment follows immediately.  With
%    the above construction, the |\mbox{}| causes the switch
%    |@inlabel| (cf.\ definition of |\list| and
%    |\trivlist| in \cite{latex:source}) to be set to \texttt{false}
%    and so the following list will insert additional vertical
%    space (|\topskip|). This is quite annoying. Therefore, we
%    create the line break within the |\item|. In order to ensure
%    that the text will begin at the proper position in the following
%    line, we simply pretend that the label does not take any
%    room.\footnote{This will lead to problems whenever very high
%    symbols occurring in the line tower into the
%    heading. So, something else has to be done here sometime.}
%    \begin{macrocode}
       \rlap{\vbox{\hbox{\hskip \labelsep\theorem@headerfont ##1\ ##2}%
                   \hbox{\strut}}}]}%
%    \end{macrocode}
%    Again, the definition of |\@opargbegintheorem| is
%    completely analogous.
%    \begin{macrocode}
\def\@opargbegintheorem##1##2##3{%
  \item[\rlap{\vbox{\hbox{\hskip \labelsep \theorem@headerfont
                     ##1\ ##2\ (##3)}%
                    \hbox{\strut}}}]}}
\endgroup
%</thb>
%    \end{macrocode}
% \end{macro}
%
%
% \subsubsection{The \texttt{changebreak} style}
%
%    \begin{macrocode}
% This style option is stored in the file |thcb.sty|.
%    \begin{macrocode}
%<*thcb>
\begingroup \makeatletter
\@ifundefined{theorem@style}{\input{theorem.sty}}{}
\ProvidesFile{thcb.sty}
             [\FMithmInfo]
%    \end{macrocode}
%
% \begin{macro}{\th@changebreak}
%    The \texttt{change-break} theorem style is like \texttt{break}
%    but with interchange of theorem name and theorem number.  Thus we
%    define |\th@changebreak| as follows:
%    \begin{macrocode}
\gdef\th@changebreak{\normalfont\slshape
  \def\@begintheorem##1##2{\item
       [\rlap{\vbox{\hbox{\hskip\labelsep \theorem@headerfont ##2\ ##1}%
                    \hbox{\strut}}}]}%
\def\@opargbegintheorem##1##2##3{%
  \item[\rlap{\vbox{\hbox{\hskip\labelsep \theorem@headerfont
                     ##2\ ##1\ (##3)}%
                    \hbox{\strut}}}]}}
\endgroup
%</thcb>
%    \end{macrocode}
% \end{macro}
%
%
% \subsubsection{The \texttt{change} style}
%
% This style option is stored in the file |thc.sty|.
%    \begin{macrocode}
%<*thc>
\begingroup \makeatletter
\@ifundefined{theorem@style}{\input{theorem.sty}}{}
\ProvidesFile{thc.sty}
             [\FMithmInfo]
%    \end{macrocode}
%
% \begin{macro}{\th@change}
%    The \texttt{change} theorem style corresponds to the
%    \texttt{change break} style without a linebreak after the header.
%    To say it in another way, it's the same as the \texttt{plain}
%    style but with number and name interchanged and |\slshape| as the
%    default font.
%    \begin{macrocode}
\gdef\th@change{\normalfont\slshape
  \def\@begintheorem##1##2{\item
       [\hskip\labelsep \theorem@headerfont ##2\ ##1]}%
\def\@opargbegintheorem##1##2##3{%
  \item[\hskip\labelsep \theorem@headerfont ##2\ ##1\ (##3)]}}
\endgroup
%</thc>
%    \end{macrocode}
% \end{macro}
%
%
% \subsubsection{The \texttt{marginbreak} style}
%
% This style option is the one used most often at Mainz.  It is saved
% in the file |thmb.sty|.
%    \begin{macrocode}
%<*thmb>
\begingroup \makeatletter
\@ifundefined{theorem@style}{\input{theorem.sty}}{}
\ProvidesFile{thmb.sty}
             [\FMithmInfo]
%    \end{macrocode}
%
% \begin{macro}{\th@marginbreak}
%    The \texttt{margin break} style is nearly the same as the
%    \texttt{change break} style. The only difference is the placement
%    of the theorem number.  We use |\llap| to place it in the left
%    margin.
%
%    In this style |\labelsep| denotes the separation between the
%    number and the text.
%    \begin{macrocode}
\gdef\th@marginbreak{\normalfont\slshape
  \def\@begintheorem##1##2{\item
       [\rlap{\vbox{\theorem@headerfont
                    \hbox{\llap{##2}\hskip\labelsep ##1}%
                    \hbox{\strut}}}]}%
\def\@opargbegintheorem##1##2##3{%
  \item[\rlap{\vbox{\theorem@headerfont
                    \hbox{\llap{##2}\hskip\labelsep ##1\ (##3)}%
                    \hbox{\strut}}}]}}
\endgroup
%</thmb>
%    \end{macrocode}
% \changes{v2.1c}{90/02/17}{`endgroup outside macrocode env.}
% \end{macro}
%
%
%
% \subsubsection{The \texttt{margin} style}
%
% This style option is stored in the file |thm.sty|.
%    \begin{macrocode}
%<*thm>
\begingroup \makeatletter
\@ifundefined{theorem@style}{\input{theorem.sty}}{}
\ProvidesFile{thm.sty}
             [\FMithmInfo]
%    \end{macrocode}
%
% \begin{macro}{\th@margin}
%    Again this is only a variant of the theorem styles described
%    above without any new ideas.
%    \begin{macrocode}
\gdef\th@margin{\normalfont\slshape
  \def\@begintheorem##1##2{\item
       [\theorem@headerfont \llap{##2}\hskip\labelsep ##1]}%
\def\@opargbegintheorem##1##2##3{%
  \item[\theorem@headerfont \llap{##2}\hskip\labelsep ##1\ (##3)]}}
\endgroup
%</thm>
%    \end{macrocode}
% \end{macro}
%
%
%
%
% \subsection{Final Definitions}
%
% \begin{macro}{\theorempreskipamount}
% \begin{macro}{\theorempostskipamount}
%    The \textsf{skip} parameters that regulate the vertical empty space
%    before and after the {\envfont theorem} environment have
%    to be allocated as well.
% \changes{v2.0f}{89/07/20}{Unnecessary `global' removed.}
%    \begin{macrocode}
%<*package>
\newskip\theorempreskipamount
\newskip\theorempostskipamount
%    \end{macrocode}
%    Since we have used the same values for all theorem sets, we now can
%    assign them.
%    \begin{macrocode}
\global\setlength\theorempreskipamount{12pt plus 5pt minus 3pt}
\global\setlength\theorempostskipamount{8pt plus 3pt minus 1.5pt}
%    \end{macrocode}
% \end{macro}
% \end{macro}
%
%
% \begin{macro}{\@endtheorem}
%    The same holds for the macro |\@endtheorem|, which ends a
%    {\envfont theorem} environment. Since it is the same for all
%    theorem sets, it is removed from the macros |\th@|\meta{style}.
%    It simply ends the {\envfont trivlist} environment, which was
%    begun in |\@thm|.
%    \begin{macrocode}
\global\let\@endtheorem=\endtrivlist
%    \end{macrocode}
% \end{macro}
%
%
%
% \begin{macro}{\@preamblecmds}
%    All macros defined above are to be used only in the preamble.
%    Therefore, we insert them in |\@preamblecmds| which will disable
%    them at begin document. This is done by the internal \LaTeXe{}
%    command |\@onlypreamble|.
% \changes{v2.0g}{89/09/19}{`theoremheaderfont and `theorembodyfont
%    added.}
% \changes{v2.2a}{94/02/03}{Use \LaTeXe{} interface}
%    \begin{macrocode}
\@onlypreamble\@xnthm
\@onlypreamble\@ynthm
\@onlypreamble\@othm
\@onlypreamble\newtheorem
\@onlypreamble\theoremstyle
\@onlypreamble\theorembodyfont
\@onlypreamble\theoremheaderfont
%    \end{macrocode}
% \end{macro}
%
% Finally we declare the \texttt{plain} theorem style to be the default.
% \changes{v2.1b}{90/06/04}{Fixed missing `item by declaring default
%    style.}
%    \begin{macrocode}
\theoremstyle{plain}
%</package>
%    \end{macrocode}
%
% \Finale

\endinput
}{}
\ProvidesFile{thc.sty}
             [\FMithmInfo]
%    \end{macrocode}
%
% \begin{macro}{\th@change}
%    The \texttt{change} theorem style corresponds to the
%    \texttt{change break} style without a linebreak after the header.
%    To say it in another way, it's the same as the \texttt{plain}
%    style but with number and name interchanged and |\slshape| as the
%    default font.
%    \begin{macrocode}
\gdef\th@change{\normalfont\slshape
  \def\@begintheorem##1##2{\item
       [\hskip\labelsep \theorem@headerfont ##2\ ##1]}%
\def\@opargbegintheorem##1##2##3{%
  \item[\hskip\labelsep \theorem@headerfont ##2\ ##1\ (##3)]}}
\endgroup
%</thc>
%    \end{macrocode}
% \end{macro}
%
%
% \subsubsection{The \texttt{marginbreak} style}
%
% This style option is the one used most often at Mainz.  It is saved
% in the file |thmb.sty|.
%    \begin{macrocode}
%<*thmb>
\begingroup \makeatletter
\@ifundefined{theorem@style}{% \iffalse meta-comment
%
% Copyright (C) 1993-2024
%
% The LaTeX Project and any individual authors listed elsewhere
% in this file.
%
% This file is part of the Standard LaTeX `Tools Bundle'.
% -------------------------------------------------------
%
% It may be distributed and/or modified under the
% conditions of the LaTeX Project Public License, either version 1.3c
% of this license or (at your option) any later version.
% The latest version of this license is in
%    https://www.latex-project.org/lppl.txt
% and version 1.3c or later is part of all distributions of LaTeX
% version 2005/12/01 or later.
%
% The list of all files belonging to the LaTeX `Tools Bundle' is
% given in the file `manifest.txt'.
%
% \fi
%
%%
%
%
\def\FMithmInfo{2023/07/05 v2.2c Theorem extension package (FMi)}
%
% \ProvidesFile{theorem.dtx}[\FMithmInfo]
%\iffalse   % this is a METACOMMENT !
%%
%% Package `theorem' to use with LaTeX2e
%% Copyright (C) 1989-1995 Frank Mittelbach, all rights reserved.
%<+package>\@ifundefined{theorem@style}{}{\endinput}
%<+package>\NeedsTeXFormat{LaTeX2e}[1995/06/01]
%<+package>\ProvidesPackage{theorem}
%<+package>         [\FMithmInfo]
%
% \fi
%
% \changes{v1.0}{87/05/07}{First draft.}
% \changes{v1.0a}{87/05/07}{`thistheoremstyle' added.}
% \changes{v1.1a}{87/07/29}{New concept: redefine the internal
%                          \protect\LaTeX{} commands `@xnthm' etc.}
% \changes{v1.1b}{87/07/31}{`theoremfont' used to specify font.}
% \changes{v1.2a}{88/01/08}{`@endtheorem' now globally defined.}
% \changes{v1.2b}{88/01/11}{`newtheorem' only allowed in the preamble.}
% \changes{v1.3a}{88/02/14}{`theoremfont' now a toks register.}
%
% \changes{v2.0a}{88/11/16}{New implementation and new documentation.}
% \changes{v2.0a}{88/11/16}{No layout changes but internal modifications
%                           old substyles won't work.}
% \changes{v2.0b}{89/01/30}{`global' used for all definitions.}
% \changes{v2.0c}{89/06/15}{Documentation translated to English.
%                            Thanks to LLL and Nina}
% \changes{v2.1a}{90/02/15}{Support for new font selection.}
% \changes{v2.2a}{94/02/03}{Cleanup for LaTeX2e}
%
% \DoNotIndex{\@endparenv,\@ifdefinable,\@ifnextchar,\@ifundefined}
% \DoNotIndex{\@namedef,\@nameuse,\@spaces,\@tempa,\@warning}
% \DoNotIndex{\ }
% \DoNotIndex{\begingroup,\bfseries}
% \DoNotIndex{\csname}
% \DoNotIndex{\def,\do,\docdate}
% \DoNotIndex{\endcsname,\endgroup,\endinput,\endtrivlist,\expandafter}
% \DoNotIndex{\filedate,\fileversion}
% \DoNotIndex{\gdef,\global}
% \DoNotIndex{\hbox,\hskip}
% \DoNotIndex{\ignorespaces,\input,\itshape,\item}
% \DoNotIndex{\labelsep,\let,\llap}
% \DoNotIndex{\makeatletter}
% \DoNotIndex{\noexpand}
% \DoNotIndex{\refstepcounter,\relax,\rlap}
% \DoNotIndex{\setlength,\slshape,\space,\string,\strut}
% \DoNotIndex{\the,\trivlist,\typeout}
% \DoNotIndex{\vbox}
% \DoNotIndex{\xdef}
%
% \def\deflabel#1{\ttfamily #1\hfill}
% \def\deflist#1{\begin{list}{}{\settowidth\labelwidth{\ttfamily #1}%
%                               \setlength\leftmargin\labelwidth
%                               \addtolength\leftmargin\labelsep
%                               \let\makelabel\deflabel}}
% \def\enddeflist{\end{list}}
% \def\envfont{\normalfont\ttfamily}
%
% \MakeShortVerb{\|}
%
% \newcommand{\TUB}{{\sl TUGboat\/}}
%
% \GetFileInfo{theorem.dtx}
% \title{An Extension of the
%        \LaTeX{} theorem environment\thanks{This file
%        has version number \fileversion, last
%        revised \filedate.}}
% \author{Frank Mittelbach}
%
% \MaintainedByLaTeXTeam{tools}
% \maketitle
%
% \begin{abstract}
%    The macros described in this paper yield an extension of the
%    \LaTeX{} theorem mechanism. It is designed to satisfy the
%    different requirements of various journals. Thus, the
%    layout of the ``theorems'' can be manipulated by determining a
%    ``style''. This article describes not only the use, but
%    also the definition, of the necessary macros.
% \end{abstract}
%
% \section*{Preface to version 2.2}
%
% For \LaTeXe{} this package did not need any fundamental changes. I
% only modified the messages generated so that theorem layout styles
% will show up with the |\listfiles| command and cleaned the section
% on the New Font Selection Scheme since this is now included in
% \LaTeX.
%
% \section*{Preface to version 2.1}
%
% This version is identical to 2.0g described in \TUB\ 10\#3 except
% for some internal defaults which are now set depending on the used
% font selection scheme.
%
% This was done to avoid unpleasant surprises if the new font selection
% scheme is in force. For further details see
% section~\ref{sec:newfonts} and \cite{bk:GMS94}.
%
% \section{Introduction}
%
% For our purposes here, ``theorems'' are labelled enunciations,
% often set off from the main text by extra space and a font change.
% Theorems, corollaries, conjectures, definitions, and remarks
% are all instances of ``theorems''.  The ``header'' of these structures
% is composed of a label (such as \textsc{Theorem} or \textsc{Remark})
% and a number which serializes an item in the sequence of items
% with the same label.
%
% Shortly after the introduction of \LaTeX{} at the Fachbereich
% Mathematik in Mainz, the desire to manipulate the layout of
% ``theorems'' arose.  In Mainz, the following two conventions came
% into general use:
% \begin{enumerate}
%   \item The number of the theorem is shown in the margin.
%   \item There is a line break at
%         the end of the theorem header.
% \end{enumerate}
%
% Additionally, some journals require different formats which depend on
% the ``sort of theorem'': e.g.\ often remarks and definitions are set
% in |\upshape|, while |\itshape| is employed for main theorems.
%
% Confronted with these requirements, a theorem environment was
% developed in Mainz which allows separate determination of the
% layout of the ``theorems sets'', comparable to |\pagestyle|.
%
%
% \section{The user interface}
% \subsection{Defining new theorem sets}
%
% \DescribeMacro\newtheorem
% As in the original \LaTeX{} version, the command |\newtheorem|
% defines a new ``theorem set'' or ``theorem-like structure''.
% Two required arguments name the new environment and give the text
% to be typeset with each instance of the new ``set'', while
% an optional argument determines how
% the ``set'' is enumerated:
% \begin{description}
%    \item[\ttfamily \bslash newtheorem\{foo\}\{bar\}]
%       The theorem set {\envfont foo} (whose name is \texttt{bar})
%       uses its own counter.
%    \item[\ttfamily \bslash newtheorem\{foo2\}{[foo]}\{bar2\}]
%       The theorem set {\envfont foo2} (printed name \texttt{bar2})
%       uses the same counter as the theorem set \texttt{foo}.
%    \item[\ttfamily \bslash newtheorem\{foo3\}\{bar3\}{[section]}]
%       The theorem set {\envfont foo3} (printed name \texttt{bar3}) is
%       enumerated within the counter \texttt{section}, i.e.\ with every
%       new |\section| the enumeration begins again with 1, and
%       the enumeration is composed from the section-number and the
%       theorem counter itself.
% \end{description}
%
% \DescribeMacro\theoremstyle
% Additionally, the command |\theoremstyle| can define the layout
% of various, or all, theorem sets.  It should be noted that any
% theorem set defined by |\newtheorem| is typeset in the
% |\theoremstyle| that is current at the time of the definition.
% Thus, the following
% \begin{verbatim}
%    \theoremstyle{break}        \newtheorem{Cor}{Corollary}
%    \theoremstyle{plain}        \newtheorem{Exa}{Example}[section]
%\end{verbatim}                             ^^A No blank at line start
% leads to the result that the set {\envfont Cor} is formatted in the
% style \texttt{break}, while the set {\envfont Exa} and all the
% following ones are formatted in the style \texttt{plain}, unless
% another |\theoremstyle| follows.  Since the definitions installed by
% |\newtheorem| are global, one also can limit |\theoremstyle| locally
% by grouping braces.
%
% \DescribeMacro\theorembodyfont
% The choice of the font for the theorem body is completely
% independent of the chosen |\theoremstyle|; this has proven to
% be very advantageous.  For example,
% \begin{verbatim}
%   {\theorembodyfont{\upshape}          \newtheorem{Rem}{Remark}}
%\end{verbatim}
% defines a theorem set {\envfont Rem}, which will be set in
% |\upshape| in the current layout (which in our example is
% \texttt{plain}).  As with |\theoremstyle|, the |\theorembodyfont|
% chosen is that current at the time of |\newtheorem|.  If
% |\theorembodyfont| is not specified or one defines
% |\theorembodyfont{}|, then the font used will be that defined by the
% |\theoremstyle|.
%
% \DescribeMacro\theoremheaderfont It is also possible to customize
% the font used for the theorem headers.  This is, however, a global
% declaration, and therefore there should be at most one
% |\theoremheaderfont| declaration in the preamble.\footnote{If it is
% actually necessary to have different header fonts, one has to define
% new theorem styles (substituting the desired font) or specify the
% information directly in the \texttt{\bslash newtheorem} declaration
% (the unclean variant).}
%
% \DescribeMacro\theorempreskipamount
% \DescribeMacro\theorempostskipamount Two additional parameters
% affect the vertical space around the theorem environments:
% |\theorempreskipamount| and |\theorempostskipamount| define,
% respectively, the spacing before and after such an environment.
% These parameters apply for all theorem sets and can be manipulated
% with the ordinary length macros.  They are rubber lengths
% (`\textsf{skips}'), and therefore can contain \texttt{plus} and
% \texttt{minus} parts.
%
% Since the definition of theorem sets should---most sensibly---be
% placed in the preamble, we only allow installation there.
% It is therefore possible to release the memory used here after
% |\begin{document}|, in order to make room for other
% applications.
%
%
% \subsection{Existing theorem styles}
%
% The following theorem styles exist to date:
% \begin{deflist}{marginbreak}
%    \item[plain]
%       This theorem style emulates the original \LaTeX{} definition,
%       except that additionally the parameters
%       |\theorem...skipamount| are used.
%    \item[break]
%       In this style, the theorem header is followed by a line
%       break.
%    \item[marginbreak]
%       The theorem number is set in the margin, and there is a line
%       break as in \texttt{break}.
%    \item[changebreak]
%       Like \texttt{break}, but with header number and text
%       interchanged.
%    \item[change]
%       Header number and text are interchanged, without a line break.
%    \item[margin]
%       The number is set in the left margin, without a line break.
% \end{deflist}
% All styles (except \texttt{plain}) select |\slshape| as the default
% |\theorembodyfont|.
%
%
% \subsection{Examples}
%
% Given the above theorem sets {\envfont Cor}, {\envfont Exa}
% and {\envfont Rem},
% suppose that the preamble also contains the declarations:
% \begin{verbatim}
%    \theoremstyle{marginbreak}   \newtheorem{Lem}[Cor]{Lemma}
%    \theoremstyle{change}
%    \theorembodyfont{\itshape}        \newtheorem{Def}[Cor]{Definition}
%
%    \theoremheaderfont{\scshape}
%\end{verbatim}
% Then the following are some typical examples of the typeset output
% resulting from their use.
%
% \begin{Cor}
%    This is a sentence typeset in the theorem environment {\envfont
%    Cor}.
% \end{Cor}
%
% \begin{Exa}
%    This is a sentence typeset in the theorem environment {\envfont
%    Exa}.
% \end{Exa}
%
% \begin{Rem}
%    This is a sentence typeset in the theorem environment {\envfont
%    Rem}.
% \end{Rem}
%
% \begin{Lem}[Ben User]
%    This is a sentence typeset in the theorem environment {\envfont
%    Lem}.
% \end{Lem}
%
% \begin{Def}[Very impressive Definition]
%    This is a sentence typeset in the theorem environment {\envfont
%    Def}.
% \end{Def}
%
% The last two examples show the effect of the optional argument to
% a theorem environment (it is the text typeset in parentheses).
%
%
% \section{Special Considerations}
% \label{sec:newfonts}
%
% Theoremheader and body are implemented as a unit. This means that
% the |\theoremheaderfont|  will inherit characteristics of the
% |\theorembodyfont| in \LaTeXe. Thus, if for example
% |\theorembodyfont| is |\itshape| and |\theoremheaderfont|
% is |\bfseries| the font selected for the header will have the
% characteristics `bold extended italic'. If this is not desired one
% should set the |\theoremheaderfont| to something like
% \begin{verbatim}
%      \theoremheaderfont{\normalfont\bfseries}
%\end{verbatim}
% i.e.\ supplying all necessary font information explicitly.
%
%
%
% \section{Acknowledgements}
%
% The publication of this set of macros was only possible with the
% help of Christina Busse (translating the manuscript into English),
% Joachim Pense (playing the r\^ole of typist), Chris Rowley
% (looking everything over) and many others providing useful
% suggestions.
%
% \MaybeStop{%
%    \begin{thebibliography}{9}
%       \bibitem{bk:GMS94} \textsc{M.~Goossens}, \textsc{F.~Mittelbach}
%         and \textsc{A.~Samarin}.
%         \newblock The \LaTeX{} Companion.
%         \newblock
%         Addison-Wesley, Reading, Massachusetts, 1994.
%       \bibitem{latex:source} \textsc{Lamport, Leslie.} \newblock
%                               \texttt{latex.tex}, version 2.09, date
%                               Feb. 1990.
%    \end{thebibliography}
%    \PrintIndex
%    \PrintChanges}
%
%
% \section{The documentation driver file}
%
% The next bit of code contains the documentation driver file for
% \TeX{}, i.e., the file that will produce the documentation you are
% currently reading. It will be extracted from this file by the
% \texttt{docstrip} program. Since it is the first code in the file
% one can alternatively process this file directly with \LaTeXe{} to
% obtain the documentation.
%
%    \begin{macrocode}
%<*driver>
\documentclass{ltxdoc}

\usepackage{theorem}

%<+driver>% The next few lines define theorem sets which are used
%<+driver>% in the example section of the documentation.

\theoremstyle{break}       \newtheorem{Cor}{Corollary}
\theoremstyle{plain}       \newtheorem{Exa}{Example}[section]
{\theorembodyfont{\upshape}\newtheorem{Rem}{Remark}}
\theoremstyle{marginbreak} \newtheorem{Lem}[Cor]{Lemma}
\theoremstyle{change}
\theorembodyfont{\itshape} \newtheorem{Def}[Cor]{Definition}

\theoremheaderfont{\scshape}

\RecordChanges

\begin{document}
   \DocInput{theorem.dtx}
\end{document}
%</driver>
%    \end{macrocode}
%
% \section{Definition of the Macros}
%
% If the file has been loaded before, we abort immediately. If not the
% package announces itself (this is actually done at the very top of
% the file---the way it is done isn't good style so don't copy it).
% \changes{v2.0e}{89/07/19}{Spaces removed from `typeout'.}
% \changes{v2.2b}{95/11/19}{Announce moved to top of file}
%    \begin{macrocode}
%<*package>
%\@ifundefined{theorem@style}{}{\endinput}
%\def\FMithmInfo{1995/11/19 v2.2b Theorem extension package (FMi)}
%\ProvidesPackage{theorem}[\FMithmInfo]
%    \end{macrocode}
%
%
% \subsection{Definition of theorem styles and fonts}
%
% All the definitions in this file are done globally to allow
% inputting this file inside a group.
%
% \begin{macro}{\theoremstyle}
%    Before a theorem style can be installed, the chosen style must be
%    known. For that reason, we must test to see that
%    |\th@|\meta{style} is known or, more precisely, that it is
%    different from |\relax|. If the style is not known then
%    |\th@plain| is used.
%    \begin{macrocode}
\gdef\theoremstyle#1{%
   \@ifundefined{th@#1}{\@warning
          {Unknown theoremstyle `#1'. Using `plain'}%
          \theorem@style{plain}}%
%    \end{macrocode}
%    We save the theorem style to be used in the token register
%    |\theorem@style|.
%    \begin{macrocode}
      {\theorem@style{#1}}%
%    \end{macrocode}
%    Now we ``evaluate'' the theorem style: this means, we call the
%    macro |\th@|\meta{style} which will activate the relevant
%    definitions which are contained in a separate file.  This is done
%    in a group to suppress changes to the current font.  This could
%    otherwise pose problems together with the new font selection
%    scheme\footnote{When I printed the original article using the new
%    font selection scheme I ended with a document with slanted
%    typefaces (text headings and all) simply because one of the
%    theorem styles used \texttt{\bslash sl} at toplevel.} if the
%    |\th@|\meta{style} is evaluated a second time.
% \changes{v2.1a}{90/02/15}{Group added around `th@..}
% \changes{v2.1a}{90/02/15}{Reseting of `..@begintheorem removed.}
%    \begin{macrocode}
      \begingroup
        \csname th@\the\theorem@style \endcsname
      \endgroup}
%    \end{macrocode}
% \end{macro}
%
%
% \begin{macro}{\@begintheorem}
% \changes{v2.1a}{90/02/15}{Relaxed at toplevel.}
% \begin{macro}{\@opargbegintheorem}
% \changes{v2.1a}{90/02/15}{Relaxed at toplevel.}
%    We reset |\@begintheorem| and |\@opargbegintheorem| to
%    |\relax| since these commands are no longer necessary at
%    toplevel. This will save a few tokens.
%    \begin{macrocode}
\global\let\@begintheorem\relax
\global\let\@opargbegintheorem\relax
%    \end{macrocode}
% \end{macro}
% \end{macro}
%
% \begin{macro}{\theorem@style}
%    Obviously the \textsf{token} register used above has to be
%    allocated. To assure the utmost compatibility with the
%    original \LaTeX{} definition, we set the default theorem style to
%    \texttt{plain}, which implements the usual \LaTeX{} convention.
% \changes{v2.0f}{89/07/20}{Unnecessary `global' removed.}
%    \begin{macrocode}
\newtoks\theorem@style
\global\theorem@style{plain}
%    \end{macrocode}
% \end{macro}
%
% \begin{macro}{\theorembodyfont}
% \begin{macro}{\theorem@bodyfont}
%    If the |\theorembodyfont| is set by the user then it should not
%    interact with the default font set in the theorem style. When the
%    new font selection is in force this may happen if, for example,
%    the default is |\itshape| and the new |\theorembodyfont| is
%    |\sffamily|. So we add a |\reset@font| command in front of the
%    user definition.
% \changes{v2.0f}{89/07/20}{Unnecessary `global' removed.}
% \changes{v2.0g}{89/09/19}{`theoremfont to `theorembodyfont.}
% \changes{v2.1c}{92/06/22}{`reset@font added and logic changed.}
% \changes{v2.1d}{92/10/28}{Reset to empty if arg is empty}
%    \begin{macrocode}
\gdef\theorembodyfont#1{%
%    \end{macrocode}
%    We check if the argument supplied is empty and if so put nothing
%    into the |\theorem@bodyfont| token register to allow for
%    |\theorembodyfont{}| as a mean of using the default of the
%    current |\theoremstyle|.
%    \begin{macrocode}
   \def\@tempa{#1}%
   \ifx\@tempa\@empty
    \theorem@bodyfont{}%
   \else
    \theorem@bodyfont{\reset@font#1}%
   \fi
   }
\newtoks\theorem@bodyfont
\global\theorem@bodyfont{}
%    \end{macrocode}
% \end{macro}
% \end{macro}
%
%
% \begin{macro}{\theoremheaderfont}
% \changes{v2.0g}{89/09/19}{Macro added.}
%    The font for the theorem headers is handled differently because
%    this definition applies to all theorem styles.
%    \begin{macrocode}
\gdef\theoremheaderfont#1{\gdef\theorem@headerfont{#1}%
%    \end{macrocode}
%    After using the macro once it is redefined to produce an error
%    message.
%    \begin{macrocode}
       \gdef\theoremheaderfont##1{%
        \typeout{\string\theoremheaderfont\space should be used
                 only once.}}}
%    \end{macrocode}
% \end{macro}
%
% \begin{macro}{\theorem@headerfont}
% \changes{v2.0g}{89/09/19}{Macro added.}
%    To set the |\theorem@headerfont| default we first test if
%    the new fontselection scheme is in force.
% \changes{v2.1a}{90/02/15}{Support for new font selection.}
%    \begin{macrocode}
\ifx\upshape\undefined
%    \end{macrocode}
%    If not we define it to expand into |\bfseries|. We don't use
%    |\let| just in case a following style option redefines this
%    macro.
%    \begin{macrocode}
\gdef\theorem@headerfont{\bfseries}
%    \end{macrocode}
%    Otherwise we reset the current shape before calling |\bfseries|.
%    \begin{macrocode}
\else \gdef\theorem@headerfont{\normalfont\bfseries}\fi
%    \end{macrocode}
% \end{macro}
%
% \begin{macro}{\th@plain}
% \begin{macro}{\th@break}
% \begin{macro}{\th@marginbreak}
% \begin{macro}{\th@changebreak}
% \begin{macro}{\th@change}
% \begin{macro}{\th@margin}
%    The different styles are defined in macros such as
%    |\th@plain|.  Since memory space is precious in
%    ``non-Big-versions'', we have to avoid offering too many unused
%    definitions.  Therefore we define these styles in separate files
%    that can be loaded on demand.  Thus the commands themselves only
%    load these files. We use |\@input@|, a \LaTeXe{} internal command
%    that ensures that the file will be listed with |\listfiles|.
% \changes{v2.2a}{94/02/02}{Use `@input@ to load theorem layout files}
%    \begin{macrocode}
\gdef\th@plain{\@input@{thp.sty}}
\gdef\th@break{\@input@{thb.sty}}
\gdef\th@marginbreak{\@input@{thmb.sty}}
\gdef\th@changebreak{\@input@{thcb.sty}}
\gdef\th@change{\@input@{thc.sty}}
\gdef\th@margin{\@input@{thm.sty}}
%    \end{macrocode}
%    This list will be expanded when new styles become
%    available.  For testing,  just append new theorem substyles
%    as document options.
% \end{macro}
% \end{macro}
% \end{macro}
% \end{macro}
% \end{macro}
% \end{macro}
%
%
%
% \subsection{Definition of a new theorem set}
%
% As already pointed out, a new theorem environment can be defined in
% three different ways:
% \begin{verbatim}
%    \newtheorem{Lem}{Lemma}
%    \newtheorem{Lem}{Lemma}[section]
%    \newtheorem{Lem}[Theorem]{Lemma}
%\end{verbatim}
% The function of the macro |\newtheorem| is to recognize these cases
% and then to branch into one of the three macros |\@ynthm|, |\@xnthm|
% or |\@othm|.  This mechanism is adopted unchanged from
% \cite{latex:source}; the essential point here is that, for example,
% in the second case, the arguments \texttt{Lem}, \texttt{Lemma} and
% \texttt{section} are passed over to the macro |\@xnthm|.
%
% We inspect this case first because the others present fewer problems,
% and thus are easily derived from this one.
%
% \begin{macro}{\@xnthm}
%    For our example arguments, the macro |\@xnthm| must fulfill
%    the following:
%    \begin{itemize}
%       \item
%          Define a new \LaTeX-counter `Lem'
%       \item
%          reset this counter within a |\section|
%       \item
%          define the macro |\theLem|
%       \item
%          define the environment macros |\Lem| and
%          |\endLem| using the current
%          |\theoremstyle| and |\theorem@bodyfont|.
%    \end{itemize}
%    Obviously, all this should happen only if the first argument
%    of |\@xnthm| (i.e.\ \texttt{Lem} in our example) is chosen so
%    as not to conflict with any previously defined commands or
%    environments.  This test is performed by the \LaTeX{} macro
%    |\@ifdefinable|.
%    \begin{macrocode}
\gdef\@xnthm#1#2[#3]{\expandafter\@ifdefinable\csname #1\endcsname
%    \end{macrocode}
%    Therefore, the first argument of |\@ifdefinable| is the
%    expansion (in the example, |\Lem|) of
%    |\csname||#1\endcsname|.  The second argument is
%    executed only if the test has been completed successfully.
%    \begin{macrocode}
   {%
%    \end{macrocode}
%    Now we define the new counter. The names of the \LaTeX{}
%    macros employed should speak for themselves:
% \changes{v2.2b}{1995/10/07}{Use \cs{@newctr}}
% \changes{v2.2c}{1995/11/23}{Use \cs{@newctr} properly.}
%    \begin{macrocode}
    \@definecounter{#1}\@newctr{#1}[#3]%
%    \end{macrocode}
%    Using |\@newctr| will give a proper error message if the counter
%    in |#3| is not defined.
%    In defining `|\theLem|' we must generate the desired macro
%    name by use of |\expandafter| and |\csname|.
%    \begin{macrocode}
    \expandafter\xdef\csname the#1\endcsname
%    \end{macrocode}
%    An |\xdef| is used in order to make the definition global,
%    and to ensure that it contains the replacement texts of
%    |\@thmcountersep| and |\@thmcounter|.\footnote{These
%    two macros can be defined by the document style.  Their default
%    values produce a `\texttt{.}' as separation and an arabic
%    representation of the number.} However, not everything should be
%    expanded. For example, it saves space to use |\thesection|
%    instead of its---at times---lengthy expansion.
%    \begin{macrocode}
      {\expandafter \noexpand \csname the#3\endcsname
       \@thmcountersep \@thmcounter{#1}}%
%    \end{macrocode}
%    Thus with the defaults of \LaTeX{}, |\theLem| would be
%    replaced by the command sequence |\thesection.\arabic{Lem}|.
%
%    We will now look at the definition of the macro which is executed
%    at the beginning of the actual environment (in our example this
%    macro is |\Lem|). It should be noted that we use an
%    ``|\expandafter| trick'' to expand only certain parts of the
%    replacement text at the time of the definition.
%    \begin{macrocode}
    \def\@tempa{\global\@namedef{#1}}%
    \expandafter \@tempa \expandafter{%
%    \end{macrocode}
%    First, the macro that contains the current definitions of
%    |\@begintheorem| and |\@opargtheorem| should be called
%    up.  The name of this macro---as is already known---has the form
%    |\th@|\meta{theorem\ style}; therefore, it must be called by
%    \begin{macrocode}
      \csname th@\the \theorem@style
%    \end{macrocode}
%    In addition the default theorem font should be changeable, i.e.\
%    we have to insert the contents of |\theorem@bodyfont|. For
%    that reason, we expand even further, beyond |\endcsname|,
%    and thus insert the contents of the \textsf{token} register
%    |\theorem@bodyfont| in the replacement text.
%    \begin{macrocode}
            \expandafter \endcsname \the \theorem@bodyfont
%    \end{macrocode}
%    Now it is time to call the macro |\@thm| which takes over
%    the further processing.  It has two arguments: the current
%    counter name (in our example, \texttt{Lem}), and the text of the
%    label (in our example, \texttt{Lemma}).
%    \begin{macrocode}
     \@thm{#1}{#2}}%
%    \end{macrocode}
%    With this, the `sub-definition' is complete.  The macro
%    |\@endtheorem| ends a theorem environment and is, so far, nothing
%    but an |\endtrivlist|.  (Hence it is defined globally, and not
%    within the theorem styles.\footnote{This has to be changed as
%    soon as theorem styles that change \texttt{\bslash @endtheorem}
%    exist.  In such a case, all existing styles must be changed as
%    well since they will have to reset the macro.}) Therefore, we can
%    set it equivalent to the macro that ends the theorem set (in our
%    example, |\endLem|).  However, if some day theorem styles exist
%    that do change |\@endtheorem|, we would have to use the
%    commented-out line instead.
%    \begin{macrocode}
    \global \expandafter \let \csname end#1\endcsname \@endtheorem
%   \global\@namedef{end#1}{\@endtheorem}%
%    \end{macrocode}
%    With these commands all the required definitions are employed,
%    unless the test |\@ifdefinable| has failed.  Therefore, we
%    end the second argument of this macro and with it the definition
%    of |\@xnthm|.
%    \begin{macrocode}
   }}
%    \end{macrocode}
% \end{macro}
%
% \begin{macro}{\@ynthm}
%    The definition of |\@ynthm| is completely analogous. In
%    this case the new counter that is defined is not reset within
%    another counter; thus the definition of |\the...|
%    is simplified:
%    \begin{macrocode}
\gdef\@ynthm#1#2{\expandafter\@ifdefinable\csname #1\endcsname
   {\@definecounter{#1}%
    \expandafter\xdef\csname the#1\endcsname{\@thmcounter{#1}}%
%    \end{macrocode}
%    The rest of the definition corresponds literally to that of
%    |\@xnthm|:
%    \begin{macrocode}
    \def\@tempa{\global\@namedef{#1}}\expandafter \@tempa
     \expandafter{\csname th@\the \theorem@style \expandafter
     \endcsname \the\theorem@bodyfont \@thm{#1}{#2}}%
    \global \expandafter \let \csname end#1\endcsname \@endtheorem}}
%    \end{macrocode}
% \end{macro}
%
% \begin{macro}{\@othm}
%    The definition of |\@othm| does not contain anything new.
%    \begin{macrocode}
\gdef\@othm#1[#2]#3{%
%    \end{macrocode}
% \changes{v2.2b}{95/10/07}{Build in check for counter defined}
%    We do not define a new counter but instead use one that has
%    already been defined.  Thus the only definition we need is that
%    of this pseudo-counter (i.e.\ |\the|\meta{env.\ name}).
%    First we check if |#2| corresponds to a known counter name.
%    \begin{macrocode}
  \expandafter\ifx\csname c@#2\endcsname\relax
   \@nocounterr{#2}%
  \else
   \expandafter\@ifdefinable\csname #1\endcsname
   {\expandafter \xdef \csname the#1\endcsname
     {\expandafter \noexpand \csname the#2\endcsname}%
%    \end{macrocode}
%    All other parts of the definition can be adopted from
%    |\@xnthm|. We have to remember, though, that in this case
%    the name of the current counter and the theorem label have moved
%    to the second and third arguments.
%    \begin{macrocode}
    \def\@tempa{\global\@namedef{#1}}\expandafter \@tempa
     \expandafter{\csname th@\the \theorem@style \expandafter
     \endcsname \the\theorem@bodyfont \@thm{#2}{#3}}%
    \global \expandafter \let \csname end#1\endcsname \@endtheorem}%
  \fi}
%    \end{macrocode}
% \end{macro}
%
% \subsection{Macros that are employed in a {\envfont theorem}
%    environment}
%
% \begin{macro}{\@thm}
%    The macro |\@thm| has to increase the current counter.  Then,
%    depending on whether the environment has (or does not have) an
%    optional argument, it has to branch into either |\@begintheorem|
%    or |\@opargtheorem|.
%    \begin{macrocode}
\gdef\@thm#1#2{\refstepcounter{#1}%
%    \end{macrocode}
%    Now we start a {\envfont trivlist} environment, and give
%    |\@topsep| and |\@topsepadd| the values of the \textsf{skip}
%    registers |\theorempreskipamount| and |\theorempostskipamount|.
%    The value in |\@topsep| is the vertical space that is inserted by
%    the first (and only) |\item| in our |\trivlist| whilst
%    |\@topsepadd| is inserted by |\@endparenv| at the end of that
%    {\envfont trivlist} environment.  By using these registers, we
%    obtain the desired space around a {\envfont theorem} environment.
%    \begin{macrocode}
   \trivlist
   \@topsep \theorempreskipamount               % used by first \item
   \@topsepadd \theorempostskipamount           % used by \@endparenv
%    \end{macrocode}
%    Now we have to test whether an optional argument has been given.
% \changes{v2.0d}{89/07/18}{Forgotten percent added.}
%    \begin{macrocode}
   \@ifnextchar [%
%    \end{macrocode}
%    If there is an optional argument, we will call |\@ythm|, and
%    move the arguments read back into the input stream.
%    \begin{macrocode}
   {\@ythm{#1}{#2}}%
%    \end{macrocode}
%    If not, we call |\@begintheorem|. Its first argument is the
%    name of the theorem set (hence the second argument of
%    |\@thm|). Its second argument is the macro that produces the
%    current number.
%    \begin{macrocode}
   {\@begintheorem{#2}{\csname the#1\endcsname}\ignorespaces}}
%    \end{macrocode}
% \end{macro}
%
% \begin{macro}{\@xthm}
% \begin{macro}{\@ythm}
%    Both these macros were originally called by |\@thm|.  We do
%    not need |\@xthm| anymore, hence we reset it to
%    |\relax|. The definition of |\@ythm| has not changed at
%    all from its definition in \LaTeX. In order to make the macros
%    easier to understand, we will
%    nevertheless present it (commented out).
%    \begin{macrocode}
\global\let\@xthm\relax
% \def\@ythm#1#2[#3]{\@opargbegintheorem{#2}{\csname
%           the#1\endcsname}{#3}\ignorespaces}
%</package>
%    \end{macrocode}
%    The primitive |\ignorespaces| in |\@ythm| and
%    |\@thm| is needed to remove the spaces between the
%    |\begin{...}| and the actual text.
% \end{macro}
% \end{macro}
%
%
% \subsection{Definition of the theorem substyles}
%
% As already pointed out, the theorem substyles, defined below, are
% only loaded when necessary.  Note that all these substyles, except
% \texttt{plain}, have |\slshape| as the default body font.
%
% \subsubsection{The \texttt{plain} style}
%
% \label{dok:th@plain} As the following macros use |@|, we have
% to locally set the |\catcode| of this symbol to ``letter''.
% This happens within a group, so that we do not have to worry about
% which |\catcode| that symbol had before.
%    \begin{macrocode}
%<*thp>
\begingroup \makeatletter
%    \end{macrocode}
% Since we are now within a group, we must make all definitions
% globally.  First we make sure that \texttt{theorem.sty} is loaded.
% This will allow us to use this file as a document style option
% without having to call \texttt{theorem} itself as an option.  At the
% same time, we assure that at least version 2 is loaded, since
% |\theorem@style| was not defined in earlier versions.
%    \begin{macrocode}
\@ifundefined{theorem@style}{\input{theorem.sty}}{}
\ProvidesFile{thp.sty}
             [\FMithmInfo]
%    \end{macrocode}
%
%
% \begin{macro}{\th@plain}
% \begin{macro}{\@begintheorem}
% \begin{macro}{\@opargbegintheorem}
%    |\theoremstyle{plain}| corresponds to the original
%    definition, except that the distances to the surrounding text are
%    determined by the parameters |\theorempreskipamount| and
%    |\theorempostskipamount|.
%    First we set the default body font.
%    \begin{macrocode}
\gdef\th@plain{\normalfont\itshape
%    \end{macrocode}
%    Then we define |\@begintheorem| and
%    |\@opargbegintheorem|. These two macros define how the
%    header of a theorem is typeset.  |\@opargbegintheorem| will
%    be called if a \textsf{theorem} environment with an optional
%    argument is encountered; otherwise, the header is constructed by
%    calling |\@begintheorem|.  If one of these macros is
%    executed, we are within a \textsf{trivlist} environment started by
%    |\@thm|.  So the theorem header is produced with an
%    |\item| command.
%
%    Instead of specifying the header font directly, all standard
%    theorem styles use the |\theorem@headerfont| macro to allow
%    customization.  The extra space (|\labelsep|) is necessary
%    because of problems in the \textsf{trivlist} environment.
%    \begin{macrocode}
  \def\@begintheorem##1##2{%
        \item[\hskip\labelsep \theorem@headerfont ##1\ ##2]}%
%    \end{macrocode}
%    The definition of |\@opargbegintheorem| is completely
%    analogous. The only difference is the fact that there exists a
%    third argument (which is the optional parameter of the
%    environment and contains additional information about the
%    theorem).  Customarily we enclose it in parentheses.
%    \begin{macrocode}
\def\@opargbegintheorem##1##2##3{%
   \item[\hskip\labelsep \theorem@headerfont ##1\ ##2\ (##3)]}}
%    \end{macrocode}
% \end{macro}
% \end{macro}
% \end{macro}
%
% \noindent
% We conclude with an |\endgroup| to restore the
% |\catcode| of |@|.
%    \begin{macrocode}
\endgroup
%</thp>
%    \end{macrocode}
%
%
% \subsubsection{The \texttt{break} style}
%
% This style option is stored in the file |thb.sty|.
% For the next two lines see the documentation for |\th@plain| on
% page~\pageref{dok:th@plain}.
%    \begin{macrocode}
%<*thb>
\begingroup \makeatletter
\@ifundefined{theorem@style}{\input{theorem.sty}}{}
\ProvidesFile{thb.sty}
             [\FMithmInfo]
%    \end{macrocode}
%
% \begin{macro}{\th@break}
%    |\theoremstyle{break}| produces a line break after the name
%    of the theorem.  The font is |\slshape|.  Hence, we define
%    |\th@break| as follows:
%    \begin{macrocode}
\gdef\th@break{\normalfont\slshape
  \def\@begintheorem##1##2{\item[%
%    \end{macrocode}
%    We run into the following problem: it is not possible to create
%    the header with |\item[|\meta{title}|]| and then start
%    a new line by, for example, |\mbox{}\\|. Such a definition
%    will fail whenever a list environment follows immediately.  With
%    the above construction, the |\mbox{}| causes the switch
%    |@inlabel| (cf.\ definition of |\list| and
%    |\trivlist| in \cite{latex:source}) to be set to \texttt{false}
%    and so the following list will insert additional vertical
%    space (|\topskip|). This is quite annoying. Therefore, we
%    create the line break within the |\item|. In order to ensure
%    that the text will begin at the proper position in the following
%    line, we simply pretend that the label does not take any
%    room.\footnote{This will lead to problems whenever very high
%    symbols occurring in the line tower into the
%    heading. So, something else has to be done here sometime.}
%    \begin{macrocode}
       \rlap{\vbox{\hbox{\hskip \labelsep\theorem@headerfont ##1\ ##2}%
                   \hbox{\strut}}}]}%
%    \end{macrocode}
%    Again, the definition of |\@opargbegintheorem| is
%    completely analogous.
%    \begin{macrocode}
\def\@opargbegintheorem##1##2##3{%
  \item[\rlap{\vbox{\hbox{\hskip \labelsep \theorem@headerfont
                     ##1\ ##2\ (##3)}%
                    \hbox{\strut}}}]}}
\endgroup
%</thb>
%    \end{macrocode}
% \end{macro}
%
%
% \subsubsection{The \texttt{changebreak} style}
%
%    \begin{macrocode}
% This style option is stored in the file |thcb.sty|.
%    \begin{macrocode}
%<*thcb>
\begingroup \makeatletter
\@ifundefined{theorem@style}{\input{theorem.sty}}{}
\ProvidesFile{thcb.sty}
             [\FMithmInfo]
%    \end{macrocode}
%
% \begin{macro}{\th@changebreak}
%    The \texttt{change-break} theorem style is like \texttt{break}
%    but with interchange of theorem name and theorem number.  Thus we
%    define |\th@changebreak| as follows:
%    \begin{macrocode}
\gdef\th@changebreak{\normalfont\slshape
  \def\@begintheorem##1##2{\item
       [\rlap{\vbox{\hbox{\hskip\labelsep \theorem@headerfont ##2\ ##1}%
                    \hbox{\strut}}}]}%
\def\@opargbegintheorem##1##2##3{%
  \item[\rlap{\vbox{\hbox{\hskip\labelsep \theorem@headerfont
                     ##2\ ##1\ (##3)}%
                    \hbox{\strut}}}]}}
\endgroup
%</thcb>
%    \end{macrocode}
% \end{macro}
%
%
% \subsubsection{The \texttt{change} style}
%
% This style option is stored in the file |thc.sty|.
%    \begin{macrocode}
%<*thc>
\begingroup \makeatletter
\@ifundefined{theorem@style}{\input{theorem.sty}}{}
\ProvidesFile{thc.sty}
             [\FMithmInfo]
%    \end{macrocode}
%
% \begin{macro}{\th@change}
%    The \texttt{change} theorem style corresponds to the
%    \texttt{change break} style without a linebreak after the header.
%    To say it in another way, it's the same as the \texttt{plain}
%    style but with number and name interchanged and |\slshape| as the
%    default font.
%    \begin{macrocode}
\gdef\th@change{\normalfont\slshape
  \def\@begintheorem##1##2{\item
       [\hskip\labelsep \theorem@headerfont ##2\ ##1]}%
\def\@opargbegintheorem##1##2##3{%
  \item[\hskip\labelsep \theorem@headerfont ##2\ ##1\ (##3)]}}
\endgroup
%</thc>
%    \end{macrocode}
% \end{macro}
%
%
% \subsubsection{The \texttt{marginbreak} style}
%
% This style option is the one used most often at Mainz.  It is saved
% in the file |thmb.sty|.
%    \begin{macrocode}
%<*thmb>
\begingroup \makeatletter
\@ifundefined{theorem@style}{\input{theorem.sty}}{}
\ProvidesFile{thmb.sty}
             [\FMithmInfo]
%    \end{macrocode}
%
% \begin{macro}{\th@marginbreak}
%    The \texttt{margin break} style is nearly the same as the
%    \texttt{change break} style. The only difference is the placement
%    of the theorem number.  We use |\llap| to place it in the left
%    margin.
%
%    In this style |\labelsep| denotes the separation between the
%    number and the text.
%    \begin{macrocode}
\gdef\th@marginbreak{\normalfont\slshape
  \def\@begintheorem##1##2{\item
       [\rlap{\vbox{\theorem@headerfont
                    \hbox{\llap{##2}\hskip\labelsep ##1}%
                    \hbox{\strut}}}]}%
\def\@opargbegintheorem##1##2##3{%
  \item[\rlap{\vbox{\theorem@headerfont
                    \hbox{\llap{##2}\hskip\labelsep ##1\ (##3)}%
                    \hbox{\strut}}}]}}
\endgroup
%</thmb>
%    \end{macrocode}
% \changes{v2.1c}{90/02/17}{`endgroup outside macrocode env.}
% \end{macro}
%
%
%
% \subsubsection{The \texttt{margin} style}
%
% This style option is stored in the file |thm.sty|.
%    \begin{macrocode}
%<*thm>
\begingroup \makeatletter
\@ifundefined{theorem@style}{\input{theorem.sty}}{}
\ProvidesFile{thm.sty}
             [\FMithmInfo]
%    \end{macrocode}
%
% \begin{macro}{\th@margin}
%    Again this is only a variant of the theorem styles described
%    above without any new ideas.
%    \begin{macrocode}
\gdef\th@margin{\normalfont\slshape
  \def\@begintheorem##1##2{\item
       [\theorem@headerfont \llap{##2}\hskip\labelsep ##1]}%
\def\@opargbegintheorem##1##2##3{%
  \item[\theorem@headerfont \llap{##2}\hskip\labelsep ##1\ (##3)]}}
\endgroup
%</thm>
%    \end{macrocode}
% \end{macro}
%
%
%
%
% \subsection{Final Definitions}
%
% \begin{macro}{\theorempreskipamount}
% \begin{macro}{\theorempostskipamount}
%    The \textsf{skip} parameters that regulate the vertical empty space
%    before and after the {\envfont theorem} environment have
%    to be allocated as well.
% \changes{v2.0f}{89/07/20}{Unnecessary `global' removed.}
%    \begin{macrocode}
%<*package>
\newskip\theorempreskipamount
\newskip\theorempostskipamount
%    \end{macrocode}
%    Since we have used the same values for all theorem sets, we now can
%    assign them.
%    \begin{macrocode}
\global\setlength\theorempreskipamount{12pt plus 5pt minus 3pt}
\global\setlength\theorempostskipamount{8pt plus 3pt minus 1.5pt}
%    \end{macrocode}
% \end{macro}
% \end{macro}
%
%
% \begin{macro}{\@endtheorem}
%    The same holds for the macro |\@endtheorem|, which ends a
%    {\envfont theorem} environment. Since it is the same for all
%    theorem sets, it is removed from the macros |\th@|\meta{style}.
%    It simply ends the {\envfont trivlist} environment, which was
%    begun in |\@thm|.
%    \begin{macrocode}
\global\let\@endtheorem=\endtrivlist
%    \end{macrocode}
% \end{macro}
%
%
%
% \begin{macro}{\@preamblecmds}
%    All macros defined above are to be used only in the preamble.
%    Therefore, we insert them in |\@preamblecmds| which will disable
%    them at begin document. This is done by the internal \LaTeXe{}
%    command |\@onlypreamble|.
% \changes{v2.0g}{89/09/19}{`theoremheaderfont and `theorembodyfont
%    added.}
% \changes{v2.2a}{94/02/03}{Use \LaTeXe{} interface}
%    \begin{macrocode}
\@onlypreamble\@xnthm
\@onlypreamble\@ynthm
\@onlypreamble\@othm
\@onlypreamble\newtheorem
\@onlypreamble\theoremstyle
\@onlypreamble\theorembodyfont
\@onlypreamble\theoremheaderfont
%    \end{macrocode}
% \end{macro}
%
% Finally we declare the \texttt{plain} theorem style to be the default.
% \changes{v2.1b}{90/06/04}{Fixed missing `item by declaring default
%    style.}
%    \begin{macrocode}
\theoremstyle{plain}
%</package>
%    \end{macrocode}
%
% \Finale

\endinput
}{}
\ProvidesFile{thmb.sty}
             [\FMithmInfo]
%    \end{macrocode}
%
% \begin{macro}{\th@marginbreak}
%    The \texttt{margin break} style is nearly the same as the
%    \texttt{change break} style. The only difference is the placement
%    of the theorem number.  We use |\llap| to place it in the left
%    margin.
%
%    In this style |\labelsep| denotes the separation between the
%    number and the text.
%    \begin{macrocode}
\gdef\th@marginbreak{\normalfont\slshape
  \def\@begintheorem##1##2{\item
       [\rlap{\vbox{\theorem@headerfont
                    \hbox{\llap{##2}\hskip\labelsep ##1}%
                    \hbox{\strut}}}]}%
\def\@opargbegintheorem##1##2##3{%
  \item[\rlap{\vbox{\theorem@headerfont
                    \hbox{\llap{##2}\hskip\labelsep ##1\ (##3)}%
                    \hbox{\strut}}}]}}
\endgroup
%</thmb>
%    \end{macrocode}
% \changes{v2.1c}{90/02/17}{`endgroup outside macrocode env.}
% \end{macro}
%
%
%
% \subsubsection{The \texttt{margin} style}
%
% This style option is stored in the file |thm.sty|.
%    \begin{macrocode}
%<*thm>
\begingroup \makeatletter
\@ifundefined{theorem@style}{% \iffalse meta-comment
%
% Copyright (C) 1993-2024
%
% The LaTeX Project and any individual authors listed elsewhere
% in this file.
%
% This file is part of the Standard LaTeX `Tools Bundle'.
% -------------------------------------------------------
%
% It may be distributed and/or modified under the
% conditions of the LaTeX Project Public License, either version 1.3c
% of this license or (at your option) any later version.
% The latest version of this license is in
%    https://www.latex-project.org/lppl.txt
% and version 1.3c or later is part of all distributions of LaTeX
% version 2005/12/01 or later.
%
% The list of all files belonging to the LaTeX `Tools Bundle' is
% given in the file `manifest.txt'.
%
% \fi
%
%%
%
%
\def\FMithmInfo{2023/07/05 v2.2c Theorem extension package (FMi)}
%
% \ProvidesFile{theorem.dtx}[\FMithmInfo]
%\iffalse   % this is a METACOMMENT !
%%
%% Package `theorem' to use with LaTeX2e
%% Copyright (C) 1989-1995 Frank Mittelbach, all rights reserved.
%<+package>\@ifundefined{theorem@style}{}{\endinput}
%<+package>\NeedsTeXFormat{LaTeX2e}[1995/06/01]
%<+package>\ProvidesPackage{theorem}
%<+package>         [\FMithmInfo]
%
% \fi
%
% \changes{v1.0}{87/05/07}{First draft.}
% \changes{v1.0a}{87/05/07}{`thistheoremstyle' added.}
% \changes{v1.1a}{87/07/29}{New concept: redefine the internal
%                          \protect\LaTeX{} commands `@xnthm' etc.}
% \changes{v1.1b}{87/07/31}{`theoremfont' used to specify font.}
% \changes{v1.2a}{88/01/08}{`@endtheorem' now globally defined.}
% \changes{v1.2b}{88/01/11}{`newtheorem' only allowed in the preamble.}
% \changes{v1.3a}{88/02/14}{`theoremfont' now a toks register.}
%
% \changes{v2.0a}{88/11/16}{New implementation and new documentation.}
% \changes{v2.0a}{88/11/16}{No layout changes but internal modifications
%                           old substyles won't work.}
% \changes{v2.0b}{89/01/30}{`global' used for all definitions.}
% \changes{v2.0c}{89/06/15}{Documentation translated to English.
%                            Thanks to LLL and Nina}
% \changes{v2.1a}{90/02/15}{Support for new font selection.}
% \changes{v2.2a}{94/02/03}{Cleanup for LaTeX2e}
%
% \DoNotIndex{\@endparenv,\@ifdefinable,\@ifnextchar,\@ifundefined}
% \DoNotIndex{\@namedef,\@nameuse,\@spaces,\@tempa,\@warning}
% \DoNotIndex{\ }
% \DoNotIndex{\begingroup,\bfseries}
% \DoNotIndex{\csname}
% \DoNotIndex{\def,\do,\docdate}
% \DoNotIndex{\endcsname,\endgroup,\endinput,\endtrivlist,\expandafter}
% \DoNotIndex{\filedate,\fileversion}
% \DoNotIndex{\gdef,\global}
% \DoNotIndex{\hbox,\hskip}
% \DoNotIndex{\ignorespaces,\input,\itshape,\item}
% \DoNotIndex{\labelsep,\let,\llap}
% \DoNotIndex{\makeatletter}
% \DoNotIndex{\noexpand}
% \DoNotIndex{\refstepcounter,\relax,\rlap}
% \DoNotIndex{\setlength,\slshape,\space,\string,\strut}
% \DoNotIndex{\the,\trivlist,\typeout}
% \DoNotIndex{\vbox}
% \DoNotIndex{\xdef}
%
% \def\deflabel#1{\ttfamily #1\hfill}
% \def\deflist#1{\begin{list}{}{\settowidth\labelwidth{\ttfamily #1}%
%                               \setlength\leftmargin\labelwidth
%                               \addtolength\leftmargin\labelsep
%                               \let\makelabel\deflabel}}
% \def\enddeflist{\end{list}}
% \def\envfont{\normalfont\ttfamily}
%
% \MakeShortVerb{\|}
%
% \newcommand{\TUB}{{\sl TUGboat\/}}
%
% \GetFileInfo{theorem.dtx}
% \title{An Extension of the
%        \LaTeX{} theorem environment\thanks{This file
%        has version number \fileversion, last
%        revised \filedate.}}
% \author{Frank Mittelbach}
%
% \MaintainedByLaTeXTeam{tools}
% \maketitle
%
% \begin{abstract}
%    The macros described in this paper yield an extension of the
%    \LaTeX{} theorem mechanism. It is designed to satisfy the
%    different requirements of various journals. Thus, the
%    layout of the ``theorems'' can be manipulated by determining a
%    ``style''. This article describes not only the use, but
%    also the definition, of the necessary macros.
% \end{abstract}
%
% \section*{Preface to version 2.2}
%
% For \LaTeXe{} this package did not need any fundamental changes. I
% only modified the messages generated so that theorem layout styles
% will show up with the |\listfiles| command and cleaned the section
% on the New Font Selection Scheme since this is now included in
% \LaTeX.
%
% \section*{Preface to version 2.1}
%
% This version is identical to 2.0g described in \TUB\ 10\#3 except
% for some internal defaults which are now set depending on the used
% font selection scheme.
%
% This was done to avoid unpleasant surprises if the new font selection
% scheme is in force. For further details see
% section~\ref{sec:newfonts} and \cite{bk:GMS94}.
%
% \section{Introduction}
%
% For our purposes here, ``theorems'' are labelled enunciations,
% often set off from the main text by extra space and a font change.
% Theorems, corollaries, conjectures, definitions, and remarks
% are all instances of ``theorems''.  The ``header'' of these structures
% is composed of a label (such as \textsc{Theorem} or \textsc{Remark})
% and a number which serializes an item in the sequence of items
% with the same label.
%
% Shortly after the introduction of \LaTeX{} at the Fachbereich
% Mathematik in Mainz, the desire to manipulate the layout of
% ``theorems'' arose.  In Mainz, the following two conventions came
% into general use:
% \begin{enumerate}
%   \item The number of the theorem is shown in the margin.
%   \item There is a line break at
%         the end of the theorem header.
% \end{enumerate}
%
% Additionally, some journals require different formats which depend on
% the ``sort of theorem'': e.g.\ often remarks and definitions are set
% in |\upshape|, while |\itshape| is employed for main theorems.
%
% Confronted with these requirements, a theorem environment was
% developed in Mainz which allows separate determination of the
% layout of the ``theorems sets'', comparable to |\pagestyle|.
%
%
% \section{The user interface}
% \subsection{Defining new theorem sets}
%
% \DescribeMacro\newtheorem
% As in the original \LaTeX{} version, the command |\newtheorem|
% defines a new ``theorem set'' or ``theorem-like structure''.
% Two required arguments name the new environment and give the text
% to be typeset with each instance of the new ``set'', while
% an optional argument determines how
% the ``set'' is enumerated:
% \begin{description}
%    \item[\ttfamily \bslash newtheorem\{foo\}\{bar\}]
%       The theorem set {\envfont foo} (whose name is \texttt{bar})
%       uses its own counter.
%    \item[\ttfamily \bslash newtheorem\{foo2\}{[foo]}\{bar2\}]
%       The theorem set {\envfont foo2} (printed name \texttt{bar2})
%       uses the same counter as the theorem set \texttt{foo}.
%    \item[\ttfamily \bslash newtheorem\{foo3\}\{bar3\}{[section]}]
%       The theorem set {\envfont foo3} (printed name \texttt{bar3}) is
%       enumerated within the counter \texttt{section}, i.e.\ with every
%       new |\section| the enumeration begins again with 1, and
%       the enumeration is composed from the section-number and the
%       theorem counter itself.
% \end{description}
%
% \DescribeMacro\theoremstyle
% Additionally, the command |\theoremstyle| can define the layout
% of various, or all, theorem sets.  It should be noted that any
% theorem set defined by |\newtheorem| is typeset in the
% |\theoremstyle| that is current at the time of the definition.
% Thus, the following
% \begin{verbatim}
%    \theoremstyle{break}        \newtheorem{Cor}{Corollary}
%    \theoremstyle{plain}        \newtheorem{Exa}{Example}[section]
%\end{verbatim}                             ^^A No blank at line start
% leads to the result that the set {\envfont Cor} is formatted in the
% style \texttt{break}, while the set {\envfont Exa} and all the
% following ones are formatted in the style \texttt{plain}, unless
% another |\theoremstyle| follows.  Since the definitions installed by
% |\newtheorem| are global, one also can limit |\theoremstyle| locally
% by grouping braces.
%
% \DescribeMacro\theorembodyfont
% The choice of the font for the theorem body is completely
% independent of the chosen |\theoremstyle|; this has proven to
% be very advantageous.  For example,
% \begin{verbatim}
%   {\theorembodyfont{\upshape}          \newtheorem{Rem}{Remark}}
%\end{verbatim}
% defines a theorem set {\envfont Rem}, which will be set in
% |\upshape| in the current layout (which in our example is
% \texttt{plain}).  As with |\theoremstyle|, the |\theorembodyfont|
% chosen is that current at the time of |\newtheorem|.  If
% |\theorembodyfont| is not specified or one defines
% |\theorembodyfont{}|, then the font used will be that defined by the
% |\theoremstyle|.
%
% \DescribeMacro\theoremheaderfont It is also possible to customize
% the font used for the theorem headers.  This is, however, a global
% declaration, and therefore there should be at most one
% |\theoremheaderfont| declaration in the preamble.\footnote{If it is
% actually necessary to have different header fonts, one has to define
% new theorem styles (substituting the desired font) or specify the
% information directly in the \texttt{\bslash newtheorem} declaration
% (the unclean variant).}
%
% \DescribeMacro\theorempreskipamount
% \DescribeMacro\theorempostskipamount Two additional parameters
% affect the vertical space around the theorem environments:
% |\theorempreskipamount| and |\theorempostskipamount| define,
% respectively, the spacing before and after such an environment.
% These parameters apply for all theorem sets and can be manipulated
% with the ordinary length macros.  They are rubber lengths
% (`\textsf{skips}'), and therefore can contain \texttt{plus} and
% \texttt{minus} parts.
%
% Since the definition of theorem sets should---most sensibly---be
% placed in the preamble, we only allow installation there.
% It is therefore possible to release the memory used here after
% |\begin{document}|, in order to make room for other
% applications.
%
%
% \subsection{Existing theorem styles}
%
% The following theorem styles exist to date:
% \begin{deflist}{marginbreak}
%    \item[plain]
%       This theorem style emulates the original \LaTeX{} definition,
%       except that additionally the parameters
%       |\theorem...skipamount| are used.
%    \item[break]
%       In this style, the theorem header is followed by a line
%       break.
%    \item[marginbreak]
%       The theorem number is set in the margin, and there is a line
%       break as in \texttt{break}.
%    \item[changebreak]
%       Like \texttt{break}, but with header number and text
%       interchanged.
%    \item[change]
%       Header number and text are interchanged, without a line break.
%    \item[margin]
%       The number is set in the left margin, without a line break.
% \end{deflist}
% All styles (except \texttt{plain}) select |\slshape| as the default
% |\theorembodyfont|.
%
%
% \subsection{Examples}
%
% Given the above theorem sets {\envfont Cor}, {\envfont Exa}
% and {\envfont Rem},
% suppose that the preamble also contains the declarations:
% \begin{verbatim}
%    \theoremstyle{marginbreak}   \newtheorem{Lem}[Cor]{Lemma}
%    \theoremstyle{change}
%    \theorembodyfont{\itshape}        \newtheorem{Def}[Cor]{Definition}
%
%    \theoremheaderfont{\scshape}
%\end{verbatim}
% Then the following are some typical examples of the typeset output
% resulting from their use.
%
% \begin{Cor}
%    This is a sentence typeset in the theorem environment {\envfont
%    Cor}.
% \end{Cor}
%
% \begin{Exa}
%    This is a sentence typeset in the theorem environment {\envfont
%    Exa}.
% \end{Exa}
%
% \begin{Rem}
%    This is a sentence typeset in the theorem environment {\envfont
%    Rem}.
% \end{Rem}
%
% \begin{Lem}[Ben User]
%    This is a sentence typeset in the theorem environment {\envfont
%    Lem}.
% \end{Lem}
%
% \begin{Def}[Very impressive Definition]
%    This is a sentence typeset in the theorem environment {\envfont
%    Def}.
% \end{Def}
%
% The last two examples show the effect of the optional argument to
% a theorem environment (it is the text typeset in parentheses).
%
%
% \section{Special Considerations}
% \label{sec:newfonts}
%
% Theoremheader and body are implemented as a unit. This means that
% the |\theoremheaderfont|  will inherit characteristics of the
% |\theorembodyfont| in \LaTeXe. Thus, if for example
% |\theorembodyfont| is |\itshape| and |\theoremheaderfont|
% is |\bfseries| the font selected for the header will have the
% characteristics `bold extended italic'. If this is not desired one
% should set the |\theoremheaderfont| to something like
% \begin{verbatim}
%      \theoremheaderfont{\normalfont\bfseries}
%\end{verbatim}
% i.e.\ supplying all necessary font information explicitly.
%
%
%
% \section{Acknowledgements}
%
% The publication of this set of macros was only possible with the
% help of Christina Busse (translating the manuscript into English),
% Joachim Pense (playing the r\^ole of typist), Chris Rowley
% (looking everything over) and many others providing useful
% suggestions.
%
% \MaybeStop{%
%    \begin{thebibliography}{9}
%       \bibitem{bk:GMS94} \textsc{M.~Goossens}, \textsc{F.~Mittelbach}
%         and \textsc{A.~Samarin}.
%         \newblock The \LaTeX{} Companion.
%         \newblock
%         Addison-Wesley, Reading, Massachusetts, 1994.
%       \bibitem{latex:source} \textsc{Lamport, Leslie.} \newblock
%                               \texttt{latex.tex}, version 2.09, date
%                               Feb. 1990.
%    \end{thebibliography}
%    \PrintIndex
%    \PrintChanges}
%
%
% \section{The documentation driver file}
%
% The next bit of code contains the documentation driver file for
% \TeX{}, i.e., the file that will produce the documentation you are
% currently reading. It will be extracted from this file by the
% \texttt{docstrip} program. Since it is the first code in the file
% one can alternatively process this file directly with \LaTeXe{} to
% obtain the documentation.
%
%    \begin{macrocode}
%<*driver>
\documentclass{ltxdoc}

\usepackage{theorem}

%<+driver>% The next few lines define theorem sets which are used
%<+driver>% in the example section of the documentation.

\theoremstyle{break}       \newtheorem{Cor}{Corollary}
\theoremstyle{plain}       \newtheorem{Exa}{Example}[section]
{\theorembodyfont{\upshape}\newtheorem{Rem}{Remark}}
\theoremstyle{marginbreak} \newtheorem{Lem}[Cor]{Lemma}
\theoremstyle{change}
\theorembodyfont{\itshape} \newtheorem{Def}[Cor]{Definition}

\theoremheaderfont{\scshape}

\RecordChanges

\begin{document}
   \DocInput{theorem.dtx}
\end{document}
%</driver>
%    \end{macrocode}
%
% \section{Definition of the Macros}
%
% If the file has been loaded before, we abort immediately. If not the
% package announces itself (this is actually done at the very top of
% the file---the way it is done isn't good style so don't copy it).
% \changes{v2.0e}{89/07/19}{Spaces removed from `typeout'.}
% \changes{v2.2b}{95/11/19}{Announce moved to top of file}
%    \begin{macrocode}
%<*package>
%\@ifundefined{theorem@style}{}{\endinput}
%\def\FMithmInfo{1995/11/19 v2.2b Theorem extension package (FMi)}
%\ProvidesPackage{theorem}[\FMithmInfo]
%    \end{macrocode}
%
%
% \subsection{Definition of theorem styles and fonts}
%
% All the definitions in this file are done globally to allow
% inputting this file inside a group.
%
% \begin{macro}{\theoremstyle}
%    Before a theorem style can be installed, the chosen style must be
%    known. For that reason, we must test to see that
%    |\th@|\meta{style} is known or, more precisely, that it is
%    different from |\relax|. If the style is not known then
%    |\th@plain| is used.
%    \begin{macrocode}
\gdef\theoremstyle#1{%
   \@ifundefined{th@#1}{\@warning
          {Unknown theoremstyle `#1'. Using `plain'}%
          \theorem@style{plain}}%
%    \end{macrocode}
%    We save the theorem style to be used in the token register
%    |\theorem@style|.
%    \begin{macrocode}
      {\theorem@style{#1}}%
%    \end{macrocode}
%    Now we ``evaluate'' the theorem style: this means, we call the
%    macro |\th@|\meta{style} which will activate the relevant
%    definitions which are contained in a separate file.  This is done
%    in a group to suppress changes to the current font.  This could
%    otherwise pose problems together with the new font selection
%    scheme\footnote{When I printed the original article using the new
%    font selection scheme I ended with a document with slanted
%    typefaces (text headings and all) simply because one of the
%    theorem styles used \texttt{\bslash sl} at toplevel.} if the
%    |\th@|\meta{style} is evaluated a second time.
% \changes{v2.1a}{90/02/15}{Group added around `th@..}
% \changes{v2.1a}{90/02/15}{Reseting of `..@begintheorem removed.}
%    \begin{macrocode}
      \begingroup
        \csname th@\the\theorem@style \endcsname
      \endgroup}
%    \end{macrocode}
% \end{macro}
%
%
% \begin{macro}{\@begintheorem}
% \changes{v2.1a}{90/02/15}{Relaxed at toplevel.}
% \begin{macro}{\@opargbegintheorem}
% \changes{v2.1a}{90/02/15}{Relaxed at toplevel.}
%    We reset |\@begintheorem| and |\@opargbegintheorem| to
%    |\relax| since these commands are no longer necessary at
%    toplevel. This will save a few tokens.
%    \begin{macrocode}
\global\let\@begintheorem\relax
\global\let\@opargbegintheorem\relax
%    \end{macrocode}
% \end{macro}
% \end{macro}
%
% \begin{macro}{\theorem@style}
%    Obviously the \textsf{token} register used above has to be
%    allocated. To assure the utmost compatibility with the
%    original \LaTeX{} definition, we set the default theorem style to
%    \texttt{plain}, which implements the usual \LaTeX{} convention.
% \changes{v2.0f}{89/07/20}{Unnecessary `global' removed.}
%    \begin{macrocode}
\newtoks\theorem@style
\global\theorem@style{plain}
%    \end{macrocode}
% \end{macro}
%
% \begin{macro}{\theorembodyfont}
% \begin{macro}{\theorem@bodyfont}
%    If the |\theorembodyfont| is set by the user then it should not
%    interact with the default font set in the theorem style. When the
%    new font selection is in force this may happen if, for example,
%    the default is |\itshape| and the new |\theorembodyfont| is
%    |\sffamily|. So we add a |\reset@font| command in front of the
%    user definition.
% \changes{v2.0f}{89/07/20}{Unnecessary `global' removed.}
% \changes{v2.0g}{89/09/19}{`theoremfont to `theorembodyfont.}
% \changes{v2.1c}{92/06/22}{`reset@font added and logic changed.}
% \changes{v2.1d}{92/10/28}{Reset to empty if arg is empty}
%    \begin{macrocode}
\gdef\theorembodyfont#1{%
%    \end{macrocode}
%    We check if the argument supplied is empty and if so put nothing
%    into the |\theorem@bodyfont| token register to allow for
%    |\theorembodyfont{}| as a mean of using the default of the
%    current |\theoremstyle|.
%    \begin{macrocode}
   \def\@tempa{#1}%
   \ifx\@tempa\@empty
    \theorem@bodyfont{}%
   \else
    \theorem@bodyfont{\reset@font#1}%
   \fi
   }
\newtoks\theorem@bodyfont
\global\theorem@bodyfont{}
%    \end{macrocode}
% \end{macro}
% \end{macro}
%
%
% \begin{macro}{\theoremheaderfont}
% \changes{v2.0g}{89/09/19}{Macro added.}
%    The font for the theorem headers is handled differently because
%    this definition applies to all theorem styles.
%    \begin{macrocode}
\gdef\theoremheaderfont#1{\gdef\theorem@headerfont{#1}%
%    \end{macrocode}
%    After using the macro once it is redefined to produce an error
%    message.
%    \begin{macrocode}
       \gdef\theoremheaderfont##1{%
        \typeout{\string\theoremheaderfont\space should be used
                 only once.}}}
%    \end{macrocode}
% \end{macro}
%
% \begin{macro}{\theorem@headerfont}
% \changes{v2.0g}{89/09/19}{Macro added.}
%    To set the |\theorem@headerfont| default we first test if
%    the new fontselection scheme is in force.
% \changes{v2.1a}{90/02/15}{Support for new font selection.}
%    \begin{macrocode}
\ifx\upshape\undefined
%    \end{macrocode}
%    If not we define it to expand into |\bfseries|. We don't use
%    |\let| just in case a following style option redefines this
%    macro.
%    \begin{macrocode}
\gdef\theorem@headerfont{\bfseries}
%    \end{macrocode}
%    Otherwise we reset the current shape before calling |\bfseries|.
%    \begin{macrocode}
\else \gdef\theorem@headerfont{\normalfont\bfseries}\fi
%    \end{macrocode}
% \end{macro}
%
% \begin{macro}{\th@plain}
% \begin{macro}{\th@break}
% \begin{macro}{\th@marginbreak}
% \begin{macro}{\th@changebreak}
% \begin{macro}{\th@change}
% \begin{macro}{\th@margin}
%    The different styles are defined in macros such as
%    |\th@plain|.  Since memory space is precious in
%    ``non-Big-versions'', we have to avoid offering too many unused
%    definitions.  Therefore we define these styles in separate files
%    that can be loaded on demand.  Thus the commands themselves only
%    load these files. We use |\@input@|, a \LaTeXe{} internal command
%    that ensures that the file will be listed with |\listfiles|.
% \changes{v2.2a}{94/02/02}{Use `@input@ to load theorem layout files}
%    \begin{macrocode}
\gdef\th@plain{\@input@{thp.sty}}
\gdef\th@break{\@input@{thb.sty}}
\gdef\th@marginbreak{\@input@{thmb.sty}}
\gdef\th@changebreak{\@input@{thcb.sty}}
\gdef\th@change{\@input@{thc.sty}}
\gdef\th@margin{\@input@{thm.sty}}
%    \end{macrocode}
%    This list will be expanded when new styles become
%    available.  For testing,  just append new theorem substyles
%    as document options.
% \end{macro}
% \end{macro}
% \end{macro}
% \end{macro}
% \end{macro}
% \end{macro}
%
%
%
% \subsection{Definition of a new theorem set}
%
% As already pointed out, a new theorem environment can be defined in
% three different ways:
% \begin{verbatim}
%    \newtheorem{Lem}{Lemma}
%    \newtheorem{Lem}{Lemma}[section]
%    \newtheorem{Lem}[Theorem]{Lemma}
%\end{verbatim}
% The function of the macro |\newtheorem| is to recognize these cases
% and then to branch into one of the three macros |\@ynthm|, |\@xnthm|
% or |\@othm|.  This mechanism is adopted unchanged from
% \cite{latex:source}; the essential point here is that, for example,
% in the second case, the arguments \texttt{Lem}, \texttt{Lemma} and
% \texttt{section} are passed over to the macro |\@xnthm|.
%
% We inspect this case first because the others present fewer problems,
% and thus are easily derived from this one.
%
% \begin{macro}{\@xnthm}
%    For our example arguments, the macro |\@xnthm| must fulfill
%    the following:
%    \begin{itemize}
%       \item
%          Define a new \LaTeX-counter `Lem'
%       \item
%          reset this counter within a |\section|
%       \item
%          define the macro |\theLem|
%       \item
%          define the environment macros |\Lem| and
%          |\endLem| using the current
%          |\theoremstyle| and |\theorem@bodyfont|.
%    \end{itemize}
%    Obviously, all this should happen only if the first argument
%    of |\@xnthm| (i.e.\ \texttt{Lem} in our example) is chosen so
%    as not to conflict with any previously defined commands or
%    environments.  This test is performed by the \LaTeX{} macro
%    |\@ifdefinable|.
%    \begin{macrocode}
\gdef\@xnthm#1#2[#3]{\expandafter\@ifdefinable\csname #1\endcsname
%    \end{macrocode}
%    Therefore, the first argument of |\@ifdefinable| is the
%    expansion (in the example, |\Lem|) of
%    |\csname||#1\endcsname|.  The second argument is
%    executed only if the test has been completed successfully.
%    \begin{macrocode}
   {%
%    \end{macrocode}
%    Now we define the new counter. The names of the \LaTeX{}
%    macros employed should speak for themselves:
% \changes{v2.2b}{1995/10/07}{Use \cs{@newctr}}
% \changes{v2.2c}{1995/11/23}{Use \cs{@newctr} properly.}
%    \begin{macrocode}
    \@definecounter{#1}\@newctr{#1}[#3]%
%    \end{macrocode}
%    Using |\@newctr| will give a proper error message if the counter
%    in |#3| is not defined.
%    In defining `|\theLem|' we must generate the desired macro
%    name by use of |\expandafter| and |\csname|.
%    \begin{macrocode}
    \expandafter\xdef\csname the#1\endcsname
%    \end{macrocode}
%    An |\xdef| is used in order to make the definition global,
%    and to ensure that it contains the replacement texts of
%    |\@thmcountersep| and |\@thmcounter|.\footnote{These
%    two macros can be defined by the document style.  Their default
%    values produce a `\texttt{.}' as separation and an arabic
%    representation of the number.} However, not everything should be
%    expanded. For example, it saves space to use |\thesection|
%    instead of its---at times---lengthy expansion.
%    \begin{macrocode}
      {\expandafter \noexpand \csname the#3\endcsname
       \@thmcountersep \@thmcounter{#1}}%
%    \end{macrocode}
%    Thus with the defaults of \LaTeX{}, |\theLem| would be
%    replaced by the command sequence |\thesection.\arabic{Lem}|.
%
%    We will now look at the definition of the macro which is executed
%    at the beginning of the actual environment (in our example this
%    macro is |\Lem|). It should be noted that we use an
%    ``|\expandafter| trick'' to expand only certain parts of the
%    replacement text at the time of the definition.
%    \begin{macrocode}
    \def\@tempa{\global\@namedef{#1}}%
    \expandafter \@tempa \expandafter{%
%    \end{macrocode}
%    First, the macro that contains the current definitions of
%    |\@begintheorem| and |\@opargtheorem| should be called
%    up.  The name of this macro---as is already known---has the form
%    |\th@|\meta{theorem\ style}; therefore, it must be called by
%    \begin{macrocode}
      \csname th@\the \theorem@style
%    \end{macrocode}
%    In addition the default theorem font should be changeable, i.e.\
%    we have to insert the contents of |\theorem@bodyfont|. For
%    that reason, we expand even further, beyond |\endcsname|,
%    and thus insert the contents of the \textsf{token} register
%    |\theorem@bodyfont| in the replacement text.
%    \begin{macrocode}
            \expandafter \endcsname \the \theorem@bodyfont
%    \end{macrocode}
%    Now it is time to call the macro |\@thm| which takes over
%    the further processing.  It has two arguments: the current
%    counter name (in our example, \texttt{Lem}), and the text of the
%    label (in our example, \texttt{Lemma}).
%    \begin{macrocode}
     \@thm{#1}{#2}}%
%    \end{macrocode}
%    With this, the `sub-definition' is complete.  The macro
%    |\@endtheorem| ends a theorem environment and is, so far, nothing
%    but an |\endtrivlist|.  (Hence it is defined globally, and not
%    within the theorem styles.\footnote{This has to be changed as
%    soon as theorem styles that change \texttt{\bslash @endtheorem}
%    exist.  In such a case, all existing styles must be changed as
%    well since they will have to reset the macro.}) Therefore, we can
%    set it equivalent to the macro that ends the theorem set (in our
%    example, |\endLem|).  However, if some day theorem styles exist
%    that do change |\@endtheorem|, we would have to use the
%    commented-out line instead.
%    \begin{macrocode}
    \global \expandafter \let \csname end#1\endcsname \@endtheorem
%   \global\@namedef{end#1}{\@endtheorem}%
%    \end{macrocode}
%    With these commands all the required definitions are employed,
%    unless the test |\@ifdefinable| has failed.  Therefore, we
%    end the second argument of this macro and with it the definition
%    of |\@xnthm|.
%    \begin{macrocode}
   }}
%    \end{macrocode}
% \end{macro}
%
% \begin{macro}{\@ynthm}
%    The definition of |\@ynthm| is completely analogous. In
%    this case the new counter that is defined is not reset within
%    another counter; thus the definition of |\the...|
%    is simplified:
%    \begin{macrocode}
\gdef\@ynthm#1#2{\expandafter\@ifdefinable\csname #1\endcsname
   {\@definecounter{#1}%
    \expandafter\xdef\csname the#1\endcsname{\@thmcounter{#1}}%
%    \end{macrocode}
%    The rest of the definition corresponds literally to that of
%    |\@xnthm|:
%    \begin{macrocode}
    \def\@tempa{\global\@namedef{#1}}\expandafter \@tempa
     \expandafter{\csname th@\the \theorem@style \expandafter
     \endcsname \the\theorem@bodyfont \@thm{#1}{#2}}%
    \global \expandafter \let \csname end#1\endcsname \@endtheorem}}
%    \end{macrocode}
% \end{macro}
%
% \begin{macro}{\@othm}
%    The definition of |\@othm| does not contain anything new.
%    \begin{macrocode}
\gdef\@othm#1[#2]#3{%
%    \end{macrocode}
% \changes{v2.2b}{95/10/07}{Build in check for counter defined}
%    We do not define a new counter but instead use one that has
%    already been defined.  Thus the only definition we need is that
%    of this pseudo-counter (i.e.\ |\the|\meta{env.\ name}).
%    First we check if |#2| corresponds to a known counter name.
%    \begin{macrocode}
  \expandafter\ifx\csname c@#2\endcsname\relax
   \@nocounterr{#2}%
  \else
   \expandafter\@ifdefinable\csname #1\endcsname
   {\expandafter \xdef \csname the#1\endcsname
     {\expandafter \noexpand \csname the#2\endcsname}%
%    \end{macrocode}
%    All other parts of the definition can be adopted from
%    |\@xnthm|. We have to remember, though, that in this case
%    the name of the current counter and the theorem label have moved
%    to the second and third arguments.
%    \begin{macrocode}
    \def\@tempa{\global\@namedef{#1}}\expandafter \@tempa
     \expandafter{\csname th@\the \theorem@style \expandafter
     \endcsname \the\theorem@bodyfont \@thm{#2}{#3}}%
    \global \expandafter \let \csname end#1\endcsname \@endtheorem}%
  \fi}
%    \end{macrocode}
% \end{macro}
%
% \subsection{Macros that are employed in a {\envfont theorem}
%    environment}
%
% \begin{macro}{\@thm}
%    The macro |\@thm| has to increase the current counter.  Then,
%    depending on whether the environment has (or does not have) an
%    optional argument, it has to branch into either |\@begintheorem|
%    or |\@opargtheorem|.
%    \begin{macrocode}
\gdef\@thm#1#2{\refstepcounter{#1}%
%    \end{macrocode}
%    Now we start a {\envfont trivlist} environment, and give
%    |\@topsep| and |\@topsepadd| the values of the \textsf{skip}
%    registers |\theorempreskipamount| and |\theorempostskipamount|.
%    The value in |\@topsep| is the vertical space that is inserted by
%    the first (and only) |\item| in our |\trivlist| whilst
%    |\@topsepadd| is inserted by |\@endparenv| at the end of that
%    {\envfont trivlist} environment.  By using these registers, we
%    obtain the desired space around a {\envfont theorem} environment.
%    \begin{macrocode}
   \trivlist
   \@topsep \theorempreskipamount               % used by first \item
   \@topsepadd \theorempostskipamount           % used by \@endparenv
%    \end{macrocode}
%    Now we have to test whether an optional argument has been given.
% \changes{v2.0d}{89/07/18}{Forgotten percent added.}
%    \begin{macrocode}
   \@ifnextchar [%
%    \end{macrocode}
%    If there is an optional argument, we will call |\@ythm|, and
%    move the arguments read back into the input stream.
%    \begin{macrocode}
   {\@ythm{#1}{#2}}%
%    \end{macrocode}
%    If not, we call |\@begintheorem|. Its first argument is the
%    name of the theorem set (hence the second argument of
%    |\@thm|). Its second argument is the macro that produces the
%    current number.
%    \begin{macrocode}
   {\@begintheorem{#2}{\csname the#1\endcsname}\ignorespaces}}
%    \end{macrocode}
% \end{macro}
%
% \begin{macro}{\@xthm}
% \begin{macro}{\@ythm}
%    Both these macros were originally called by |\@thm|.  We do
%    not need |\@xthm| anymore, hence we reset it to
%    |\relax|. The definition of |\@ythm| has not changed at
%    all from its definition in \LaTeX. In order to make the macros
%    easier to understand, we will
%    nevertheless present it (commented out).
%    \begin{macrocode}
\global\let\@xthm\relax
% \def\@ythm#1#2[#3]{\@opargbegintheorem{#2}{\csname
%           the#1\endcsname}{#3}\ignorespaces}
%</package>
%    \end{macrocode}
%    The primitive |\ignorespaces| in |\@ythm| and
%    |\@thm| is needed to remove the spaces between the
%    |\begin{...}| and the actual text.
% \end{macro}
% \end{macro}
%
%
% \subsection{Definition of the theorem substyles}
%
% As already pointed out, the theorem substyles, defined below, are
% only loaded when necessary.  Note that all these substyles, except
% \texttt{plain}, have |\slshape| as the default body font.
%
% \subsubsection{The \texttt{plain} style}
%
% \label{dok:th@plain} As the following macros use |@|, we have
% to locally set the |\catcode| of this symbol to ``letter''.
% This happens within a group, so that we do not have to worry about
% which |\catcode| that symbol had before.
%    \begin{macrocode}
%<*thp>
\begingroup \makeatletter
%    \end{macrocode}
% Since we are now within a group, we must make all definitions
% globally.  First we make sure that \texttt{theorem.sty} is loaded.
% This will allow us to use this file as a document style option
% without having to call \texttt{theorem} itself as an option.  At the
% same time, we assure that at least version 2 is loaded, since
% |\theorem@style| was not defined in earlier versions.
%    \begin{macrocode}
\@ifundefined{theorem@style}{\input{theorem.sty}}{}
\ProvidesFile{thp.sty}
             [\FMithmInfo]
%    \end{macrocode}
%
%
% \begin{macro}{\th@plain}
% \begin{macro}{\@begintheorem}
% \begin{macro}{\@opargbegintheorem}
%    |\theoremstyle{plain}| corresponds to the original
%    definition, except that the distances to the surrounding text are
%    determined by the parameters |\theorempreskipamount| and
%    |\theorempostskipamount|.
%    First we set the default body font.
%    \begin{macrocode}
\gdef\th@plain{\normalfont\itshape
%    \end{macrocode}
%    Then we define |\@begintheorem| and
%    |\@opargbegintheorem|. These two macros define how the
%    header of a theorem is typeset.  |\@opargbegintheorem| will
%    be called if a \textsf{theorem} environment with an optional
%    argument is encountered; otherwise, the header is constructed by
%    calling |\@begintheorem|.  If one of these macros is
%    executed, we are within a \textsf{trivlist} environment started by
%    |\@thm|.  So the theorem header is produced with an
%    |\item| command.
%
%    Instead of specifying the header font directly, all standard
%    theorem styles use the |\theorem@headerfont| macro to allow
%    customization.  The extra space (|\labelsep|) is necessary
%    because of problems in the \textsf{trivlist} environment.
%    \begin{macrocode}
  \def\@begintheorem##1##2{%
        \item[\hskip\labelsep \theorem@headerfont ##1\ ##2]}%
%    \end{macrocode}
%    The definition of |\@opargbegintheorem| is completely
%    analogous. The only difference is the fact that there exists a
%    third argument (which is the optional parameter of the
%    environment and contains additional information about the
%    theorem).  Customarily we enclose it in parentheses.
%    \begin{macrocode}
\def\@opargbegintheorem##1##2##3{%
   \item[\hskip\labelsep \theorem@headerfont ##1\ ##2\ (##3)]}}
%    \end{macrocode}
% \end{macro}
% \end{macro}
% \end{macro}
%
% \noindent
% We conclude with an |\endgroup| to restore the
% |\catcode| of |@|.
%    \begin{macrocode}
\endgroup
%</thp>
%    \end{macrocode}
%
%
% \subsubsection{The \texttt{break} style}
%
% This style option is stored in the file |thb.sty|.
% For the next two lines see the documentation for |\th@plain| on
% page~\pageref{dok:th@plain}.
%    \begin{macrocode}
%<*thb>
\begingroup \makeatletter
\@ifundefined{theorem@style}{\input{theorem.sty}}{}
\ProvidesFile{thb.sty}
             [\FMithmInfo]
%    \end{macrocode}
%
% \begin{macro}{\th@break}
%    |\theoremstyle{break}| produces a line break after the name
%    of the theorem.  The font is |\slshape|.  Hence, we define
%    |\th@break| as follows:
%    \begin{macrocode}
\gdef\th@break{\normalfont\slshape
  \def\@begintheorem##1##2{\item[%
%    \end{macrocode}
%    We run into the following problem: it is not possible to create
%    the header with |\item[|\meta{title}|]| and then start
%    a new line by, for example, |\mbox{}\\|. Such a definition
%    will fail whenever a list environment follows immediately.  With
%    the above construction, the |\mbox{}| causes the switch
%    |@inlabel| (cf.\ definition of |\list| and
%    |\trivlist| in \cite{latex:source}) to be set to \texttt{false}
%    and so the following list will insert additional vertical
%    space (|\topskip|). This is quite annoying. Therefore, we
%    create the line break within the |\item|. In order to ensure
%    that the text will begin at the proper position in the following
%    line, we simply pretend that the label does not take any
%    room.\footnote{This will lead to problems whenever very high
%    symbols occurring in the line tower into the
%    heading. So, something else has to be done here sometime.}
%    \begin{macrocode}
       \rlap{\vbox{\hbox{\hskip \labelsep\theorem@headerfont ##1\ ##2}%
                   \hbox{\strut}}}]}%
%    \end{macrocode}
%    Again, the definition of |\@opargbegintheorem| is
%    completely analogous.
%    \begin{macrocode}
\def\@opargbegintheorem##1##2##3{%
  \item[\rlap{\vbox{\hbox{\hskip \labelsep \theorem@headerfont
                     ##1\ ##2\ (##3)}%
                    \hbox{\strut}}}]}}
\endgroup
%</thb>
%    \end{macrocode}
% \end{macro}
%
%
% \subsubsection{The \texttt{changebreak} style}
%
%    \begin{macrocode}
% This style option is stored in the file |thcb.sty|.
%    \begin{macrocode}
%<*thcb>
\begingroup \makeatletter
\@ifundefined{theorem@style}{\input{theorem.sty}}{}
\ProvidesFile{thcb.sty}
             [\FMithmInfo]
%    \end{macrocode}
%
% \begin{macro}{\th@changebreak}
%    The \texttt{change-break} theorem style is like \texttt{break}
%    but with interchange of theorem name and theorem number.  Thus we
%    define |\th@changebreak| as follows:
%    \begin{macrocode}
\gdef\th@changebreak{\normalfont\slshape
  \def\@begintheorem##1##2{\item
       [\rlap{\vbox{\hbox{\hskip\labelsep \theorem@headerfont ##2\ ##1}%
                    \hbox{\strut}}}]}%
\def\@opargbegintheorem##1##2##3{%
  \item[\rlap{\vbox{\hbox{\hskip\labelsep \theorem@headerfont
                     ##2\ ##1\ (##3)}%
                    \hbox{\strut}}}]}}
\endgroup
%</thcb>
%    \end{macrocode}
% \end{macro}
%
%
% \subsubsection{The \texttt{change} style}
%
% This style option is stored in the file |thc.sty|.
%    \begin{macrocode}
%<*thc>
\begingroup \makeatletter
\@ifundefined{theorem@style}{\input{theorem.sty}}{}
\ProvidesFile{thc.sty}
             [\FMithmInfo]
%    \end{macrocode}
%
% \begin{macro}{\th@change}
%    The \texttt{change} theorem style corresponds to the
%    \texttt{change break} style without a linebreak after the header.
%    To say it in another way, it's the same as the \texttt{plain}
%    style but with number and name interchanged and |\slshape| as the
%    default font.
%    \begin{macrocode}
\gdef\th@change{\normalfont\slshape
  \def\@begintheorem##1##2{\item
       [\hskip\labelsep \theorem@headerfont ##2\ ##1]}%
\def\@opargbegintheorem##1##2##3{%
  \item[\hskip\labelsep \theorem@headerfont ##2\ ##1\ (##3)]}}
\endgroup
%</thc>
%    \end{macrocode}
% \end{macro}
%
%
% \subsubsection{The \texttt{marginbreak} style}
%
% This style option is the one used most often at Mainz.  It is saved
% in the file |thmb.sty|.
%    \begin{macrocode}
%<*thmb>
\begingroup \makeatletter
\@ifundefined{theorem@style}{\input{theorem.sty}}{}
\ProvidesFile{thmb.sty}
             [\FMithmInfo]
%    \end{macrocode}
%
% \begin{macro}{\th@marginbreak}
%    The \texttt{margin break} style is nearly the same as the
%    \texttt{change break} style. The only difference is the placement
%    of the theorem number.  We use |\llap| to place it in the left
%    margin.
%
%    In this style |\labelsep| denotes the separation between the
%    number and the text.
%    \begin{macrocode}
\gdef\th@marginbreak{\normalfont\slshape
  \def\@begintheorem##1##2{\item
       [\rlap{\vbox{\theorem@headerfont
                    \hbox{\llap{##2}\hskip\labelsep ##1}%
                    \hbox{\strut}}}]}%
\def\@opargbegintheorem##1##2##3{%
  \item[\rlap{\vbox{\theorem@headerfont
                    \hbox{\llap{##2}\hskip\labelsep ##1\ (##3)}%
                    \hbox{\strut}}}]}}
\endgroup
%</thmb>
%    \end{macrocode}
% \changes{v2.1c}{90/02/17}{`endgroup outside macrocode env.}
% \end{macro}
%
%
%
% \subsubsection{The \texttt{margin} style}
%
% This style option is stored in the file |thm.sty|.
%    \begin{macrocode}
%<*thm>
\begingroup \makeatletter
\@ifundefined{theorem@style}{\input{theorem.sty}}{}
\ProvidesFile{thm.sty}
             [\FMithmInfo]
%    \end{macrocode}
%
% \begin{macro}{\th@margin}
%    Again this is only a variant of the theorem styles described
%    above without any new ideas.
%    \begin{macrocode}
\gdef\th@margin{\normalfont\slshape
  \def\@begintheorem##1##2{\item
       [\theorem@headerfont \llap{##2}\hskip\labelsep ##1]}%
\def\@opargbegintheorem##1##2##3{%
  \item[\theorem@headerfont \llap{##2}\hskip\labelsep ##1\ (##3)]}}
\endgroup
%</thm>
%    \end{macrocode}
% \end{macro}
%
%
%
%
% \subsection{Final Definitions}
%
% \begin{macro}{\theorempreskipamount}
% \begin{macro}{\theorempostskipamount}
%    The \textsf{skip} parameters that regulate the vertical empty space
%    before and after the {\envfont theorem} environment have
%    to be allocated as well.
% \changes{v2.0f}{89/07/20}{Unnecessary `global' removed.}
%    \begin{macrocode}
%<*package>
\newskip\theorempreskipamount
\newskip\theorempostskipamount
%    \end{macrocode}
%    Since we have used the same values for all theorem sets, we now can
%    assign them.
%    \begin{macrocode}
\global\setlength\theorempreskipamount{12pt plus 5pt minus 3pt}
\global\setlength\theorempostskipamount{8pt plus 3pt minus 1.5pt}
%    \end{macrocode}
% \end{macro}
% \end{macro}
%
%
% \begin{macro}{\@endtheorem}
%    The same holds for the macro |\@endtheorem|, which ends a
%    {\envfont theorem} environment. Since it is the same for all
%    theorem sets, it is removed from the macros |\th@|\meta{style}.
%    It simply ends the {\envfont trivlist} environment, which was
%    begun in |\@thm|.
%    \begin{macrocode}
\global\let\@endtheorem=\endtrivlist
%    \end{macrocode}
% \end{macro}
%
%
%
% \begin{macro}{\@preamblecmds}
%    All macros defined above are to be used only in the preamble.
%    Therefore, we insert them in |\@preamblecmds| which will disable
%    them at begin document. This is done by the internal \LaTeXe{}
%    command |\@onlypreamble|.
% \changes{v2.0g}{89/09/19}{`theoremheaderfont and `theorembodyfont
%    added.}
% \changes{v2.2a}{94/02/03}{Use \LaTeXe{} interface}
%    \begin{macrocode}
\@onlypreamble\@xnthm
\@onlypreamble\@ynthm
\@onlypreamble\@othm
\@onlypreamble\newtheorem
\@onlypreamble\theoremstyle
\@onlypreamble\theorembodyfont
\@onlypreamble\theoremheaderfont
%    \end{macrocode}
% \end{macro}
%
% Finally we declare the \texttt{plain} theorem style to be the default.
% \changes{v2.1b}{90/06/04}{Fixed missing `item by declaring default
%    style.}
%    \begin{macrocode}
\theoremstyle{plain}
%</package>
%    \end{macrocode}
%
% \Finale

\endinput
}{}
\ProvidesFile{thm.sty}
             [\FMithmInfo]
%    \end{macrocode}
%
% \begin{macro}{\th@margin}
%    Again this is only a variant of the theorem styles described
%    above without any new ideas.
%    \begin{macrocode}
\gdef\th@margin{\normalfont\slshape
  \def\@begintheorem##1##2{\item
       [\theorem@headerfont \llap{##2}\hskip\labelsep ##1]}%
\def\@opargbegintheorem##1##2##3{%
  \item[\theorem@headerfont \llap{##2}\hskip\labelsep ##1\ (##3)]}}
\endgroup
%</thm>
%    \end{macrocode}
% \end{macro}
%
%
%
%
% \subsection{Final Definitions}
%
% \begin{macro}{\theorempreskipamount}
% \begin{macro}{\theorempostskipamount}
%    The \textsf{skip} parameters that regulate the vertical empty space
%    before and after the {\envfont theorem} environment have
%    to be allocated as well.
% \changes{v2.0f}{89/07/20}{Unnecessary `global' removed.}
%    \begin{macrocode}
%<*package>
\newskip\theorempreskipamount
\newskip\theorempostskipamount
%    \end{macrocode}
%    Since we have used the same values for all theorem sets, we now can
%    assign them.
%    \begin{macrocode}
\global\setlength\theorempreskipamount{12pt plus 5pt minus 3pt}
\global\setlength\theorempostskipamount{8pt plus 3pt minus 1.5pt}
%    \end{macrocode}
% \end{macro}
% \end{macro}
%
%
% \begin{macro}{\@endtheorem}
%    The same holds for the macro |\@endtheorem|, which ends a
%    {\envfont theorem} environment. Since it is the same for all
%    theorem sets, it is removed from the macros |\th@|\meta{style}.
%    It simply ends the {\envfont trivlist} environment, which was
%    begun in |\@thm|.
%    \begin{macrocode}
\global\let\@endtheorem=\endtrivlist
%    \end{macrocode}
% \end{macro}
%
%
%
% \begin{macro}{\@preamblecmds}
%    All macros defined above are to be used only in the preamble.
%    Therefore, we insert them in |\@preamblecmds| which will disable
%    them at begin document. This is done by the internal \LaTeXe{}
%    command |\@onlypreamble|.
% \changes{v2.0g}{89/09/19}{`theoremheaderfont and `theorembodyfont
%    added.}
% \changes{v2.2a}{94/02/03}{Use \LaTeXe{} interface}
%    \begin{macrocode}
\@onlypreamble\@xnthm
\@onlypreamble\@ynthm
\@onlypreamble\@othm
\@onlypreamble\newtheorem
\@onlypreamble\theoremstyle
\@onlypreamble\theorembodyfont
\@onlypreamble\theoremheaderfont
%    \end{macrocode}
% \end{macro}
%
% Finally we declare the \texttt{plain} theorem style to be the default.
% \changes{v2.1b}{90/06/04}{Fixed missing `item by declaring default
%    style.}
%    \begin{macrocode}
\theoremstyle{plain}
%</package>
%    \end{macrocode}
%
% \Finale

\endinput
}{}
\ProvidesFile{thcb.sty}
             [\FMithmInfo]
%    \end{macrocode}
%
% \begin{macro}{\th@changebreak}
%    The \texttt{change-break} theorem style is like \texttt{break}
%    but with interchange of theorem name and theorem number.  Thus we
%    define |\th@changebreak| as follows:
%    \begin{macrocode}
\gdef\th@changebreak{\normalfont\slshape
  \def\@begintheorem##1##2{\item
       [\rlap{\vbox{\hbox{\hskip\labelsep \theorem@headerfont ##2\ ##1}%
                    \hbox{\strut}}}]}%
\def\@opargbegintheorem##1##2##3{%
  \item[\rlap{\vbox{\hbox{\hskip\labelsep \theorem@headerfont
                     ##2\ ##1\ (##3)}%
                    \hbox{\strut}}}]}}
\endgroup
%</thcb>
%    \end{macrocode}
% \end{macro}
%
%
% \subsubsection{The \texttt{change} style}
%
% This style option is stored in the file |thc.sty|.
%    \begin{macrocode}
%<*thc>
\begingroup \makeatletter
\@ifundefined{theorem@style}{% \iffalse meta-comment
%
% Copyright (C) 1993-2024
%
% The LaTeX Project and any individual authors listed elsewhere
% in this file.
%
% This file is part of the Standard LaTeX `Tools Bundle'.
% -------------------------------------------------------
%
% It may be distributed and/or modified under the
% conditions of the LaTeX Project Public License, either version 1.3c
% of this license or (at your option) any later version.
% The latest version of this license is in
%    https://www.latex-project.org/lppl.txt
% and version 1.3c or later is part of all distributions of LaTeX
% version 2005/12/01 or later.
%
% The list of all files belonging to the LaTeX `Tools Bundle' is
% given in the file `manifest.txt'.
%
% \fi
%
%%
%
%
\def\FMithmInfo{2023/07/05 v2.2c Theorem extension package (FMi)}
%
% \ProvidesFile{theorem.dtx}[\FMithmInfo]
%\iffalse   % this is a METACOMMENT !
%%
%% Package `theorem' to use with LaTeX2e
%% Copyright (C) 1989-1995 Frank Mittelbach, all rights reserved.
%<+package>\@ifundefined{theorem@style}{}{\endinput}
%<+package>\NeedsTeXFormat{LaTeX2e}[1995/06/01]
%<+package>\ProvidesPackage{theorem}
%<+package>         [\FMithmInfo]
%
% \fi
%
% \changes{v1.0}{87/05/07}{First draft.}
% \changes{v1.0a}{87/05/07}{`thistheoremstyle' added.}
% \changes{v1.1a}{87/07/29}{New concept: redefine the internal
%                          \protect\LaTeX{} commands `@xnthm' etc.}
% \changes{v1.1b}{87/07/31}{`theoremfont' used to specify font.}
% \changes{v1.2a}{88/01/08}{`@endtheorem' now globally defined.}
% \changes{v1.2b}{88/01/11}{`newtheorem' only allowed in the preamble.}
% \changes{v1.3a}{88/02/14}{`theoremfont' now a toks register.}
%
% \changes{v2.0a}{88/11/16}{New implementation and new documentation.}
% \changes{v2.0a}{88/11/16}{No layout changes but internal modifications
%                           old substyles won't work.}
% \changes{v2.0b}{89/01/30}{`global' used for all definitions.}
% \changes{v2.0c}{89/06/15}{Documentation translated to English.
%                            Thanks to LLL and Nina}
% \changes{v2.1a}{90/02/15}{Support for new font selection.}
% \changes{v2.2a}{94/02/03}{Cleanup for LaTeX2e}
%
% \DoNotIndex{\@endparenv,\@ifdefinable,\@ifnextchar,\@ifundefined}
% \DoNotIndex{\@namedef,\@nameuse,\@spaces,\@tempa,\@warning}
% \DoNotIndex{\ }
% \DoNotIndex{\begingroup,\bfseries}
% \DoNotIndex{\csname}
% \DoNotIndex{\def,\do,\docdate}
% \DoNotIndex{\endcsname,\endgroup,\endinput,\endtrivlist,\expandafter}
% \DoNotIndex{\filedate,\fileversion}
% \DoNotIndex{\gdef,\global}
% \DoNotIndex{\hbox,\hskip}
% \DoNotIndex{\ignorespaces,\input,\itshape,\item}
% \DoNotIndex{\labelsep,\let,\llap}
% \DoNotIndex{\makeatletter}
% \DoNotIndex{\noexpand}
% \DoNotIndex{\refstepcounter,\relax,\rlap}
% \DoNotIndex{\setlength,\slshape,\space,\string,\strut}
% \DoNotIndex{\the,\trivlist,\typeout}
% \DoNotIndex{\vbox}
% \DoNotIndex{\xdef}
%
% \def\deflabel#1{\ttfamily #1\hfill}
% \def\deflist#1{\begin{list}{}{\settowidth\labelwidth{\ttfamily #1}%
%                               \setlength\leftmargin\labelwidth
%                               \addtolength\leftmargin\labelsep
%                               \let\makelabel\deflabel}}
% \def\enddeflist{\end{list}}
% \def\envfont{\normalfont\ttfamily}
%
% \MakeShortVerb{\|}
%
% \newcommand{\TUB}{{\sl TUGboat\/}}
%
% \GetFileInfo{theorem.dtx}
% \title{An Extension of the
%        \LaTeX{} theorem environment\thanks{This file
%        has version number \fileversion, last
%        revised \filedate.}}
% \author{Frank Mittelbach}
%
% \MaintainedByLaTeXTeam{tools}
% \maketitle
%
% \begin{abstract}
%    The macros described in this paper yield an extension of the
%    \LaTeX{} theorem mechanism. It is designed to satisfy the
%    different requirements of various journals. Thus, the
%    layout of the ``theorems'' can be manipulated by determining a
%    ``style''. This article describes not only the use, but
%    also the definition, of the necessary macros.
% \end{abstract}
%
% \section*{Preface to version 2.2}
%
% For \LaTeXe{} this package did not need any fundamental changes. I
% only modified the messages generated so that theorem layout styles
% will show up with the |\listfiles| command and cleaned the section
% on the New Font Selection Scheme since this is now included in
% \LaTeX.
%
% \section*{Preface to version 2.1}
%
% This version is identical to 2.0g described in \TUB\ 10\#3 except
% for some internal defaults which are now set depending on the used
% font selection scheme.
%
% This was done to avoid unpleasant surprises if the new font selection
% scheme is in force. For further details see
% section~\ref{sec:newfonts} and \cite{bk:GMS94}.
%
% \section{Introduction}
%
% For our purposes here, ``theorems'' are labelled enunciations,
% often set off from the main text by extra space and a font change.
% Theorems, corollaries, conjectures, definitions, and remarks
% are all instances of ``theorems''.  The ``header'' of these structures
% is composed of a label (such as \textsc{Theorem} or \textsc{Remark})
% and a number which serializes an item in the sequence of items
% with the same label.
%
% Shortly after the introduction of \LaTeX{} at the Fachbereich
% Mathematik in Mainz, the desire to manipulate the layout of
% ``theorems'' arose.  In Mainz, the following two conventions came
% into general use:
% \begin{enumerate}
%   \item The number of the theorem is shown in the margin.
%   \item There is a line break at
%         the end of the theorem header.
% \end{enumerate}
%
% Additionally, some journals require different formats which depend on
% the ``sort of theorem'': e.g.\ often remarks and definitions are set
% in |\upshape|, while |\itshape| is employed for main theorems.
%
% Confronted with these requirements, a theorem environment was
% developed in Mainz which allows separate determination of the
% layout of the ``theorems sets'', comparable to |\pagestyle|.
%
%
% \section{The user interface}
% \subsection{Defining new theorem sets}
%
% \DescribeMacro\newtheorem
% As in the original \LaTeX{} version, the command |\newtheorem|
% defines a new ``theorem set'' or ``theorem-like structure''.
% Two required arguments name the new environment and give the text
% to be typeset with each instance of the new ``set'', while
% an optional argument determines how
% the ``set'' is enumerated:
% \begin{description}
%    \item[\ttfamily \bslash newtheorem\{foo\}\{bar\}]
%       The theorem set {\envfont foo} (whose name is \texttt{bar})
%       uses its own counter.
%    \item[\ttfamily \bslash newtheorem\{foo2\}{[foo]}\{bar2\}]
%       The theorem set {\envfont foo2} (printed name \texttt{bar2})
%       uses the same counter as the theorem set \texttt{foo}.
%    \item[\ttfamily \bslash newtheorem\{foo3\}\{bar3\}{[section]}]
%       The theorem set {\envfont foo3} (printed name \texttt{bar3}) is
%       enumerated within the counter \texttt{section}, i.e.\ with every
%       new |\section| the enumeration begins again with 1, and
%       the enumeration is composed from the section-number and the
%       theorem counter itself.
% \end{description}
%
% \DescribeMacro\theoremstyle
% Additionally, the command |\theoremstyle| can define the layout
% of various, or all, theorem sets.  It should be noted that any
% theorem set defined by |\newtheorem| is typeset in the
% |\theoremstyle| that is current at the time of the definition.
% Thus, the following
% \begin{verbatim}
%    \theoremstyle{break}        \newtheorem{Cor}{Corollary}
%    \theoremstyle{plain}        \newtheorem{Exa}{Example}[section]
%\end{verbatim}                             ^^A No blank at line start
% leads to the result that the set {\envfont Cor} is formatted in the
% style \texttt{break}, while the set {\envfont Exa} and all the
% following ones are formatted in the style \texttt{plain}, unless
% another |\theoremstyle| follows.  Since the definitions installed by
% |\newtheorem| are global, one also can limit |\theoremstyle| locally
% by grouping braces.
%
% \DescribeMacro\theorembodyfont
% The choice of the font for the theorem body is completely
% independent of the chosen |\theoremstyle|; this has proven to
% be very advantageous.  For example,
% \begin{verbatim}
%   {\theorembodyfont{\upshape}          \newtheorem{Rem}{Remark}}
%\end{verbatim}
% defines a theorem set {\envfont Rem}, which will be set in
% |\upshape| in the current layout (which in our example is
% \texttt{plain}).  As with |\theoremstyle|, the |\theorembodyfont|
% chosen is that current at the time of |\newtheorem|.  If
% |\theorembodyfont| is not specified or one defines
% |\theorembodyfont{}|, then the font used will be that defined by the
% |\theoremstyle|.
%
% \DescribeMacro\theoremheaderfont It is also possible to customize
% the font used for the theorem headers.  This is, however, a global
% declaration, and therefore there should be at most one
% |\theoremheaderfont| declaration in the preamble.\footnote{If it is
% actually necessary to have different header fonts, one has to define
% new theorem styles (substituting the desired font) or specify the
% information directly in the \texttt{\bslash newtheorem} declaration
% (the unclean variant).}
%
% \DescribeMacro\theorempreskipamount
% \DescribeMacro\theorempostskipamount Two additional parameters
% affect the vertical space around the theorem environments:
% |\theorempreskipamount| and |\theorempostskipamount| define,
% respectively, the spacing before and after such an environment.
% These parameters apply for all theorem sets and can be manipulated
% with the ordinary length macros.  They are rubber lengths
% (`\textsf{skips}'), and therefore can contain \texttt{plus} and
% \texttt{minus} parts.
%
% Since the definition of theorem sets should---most sensibly---be
% placed in the preamble, we only allow installation there.
% It is therefore possible to release the memory used here after
% |\begin{document}|, in order to make room for other
% applications.
%
%
% \subsection{Existing theorem styles}
%
% The following theorem styles exist to date:
% \begin{deflist}{marginbreak}
%    \item[plain]
%       This theorem style emulates the original \LaTeX{} definition,
%       except that additionally the parameters
%       |\theorem...skipamount| are used.
%    \item[break]
%       In this style, the theorem header is followed by a line
%       break.
%    \item[marginbreak]
%       The theorem number is set in the margin, and there is a line
%       break as in \texttt{break}.
%    \item[changebreak]
%       Like \texttt{break}, but with header number and text
%       interchanged.
%    \item[change]
%       Header number and text are interchanged, without a line break.
%    \item[margin]
%       The number is set in the left margin, without a line break.
% \end{deflist}
% All styles (except \texttt{plain}) select |\slshape| as the default
% |\theorembodyfont|.
%
%
% \subsection{Examples}
%
% Given the above theorem sets {\envfont Cor}, {\envfont Exa}
% and {\envfont Rem},
% suppose that the preamble also contains the declarations:
% \begin{verbatim}
%    \theoremstyle{marginbreak}   \newtheorem{Lem}[Cor]{Lemma}
%    \theoremstyle{change}
%    \theorembodyfont{\itshape}        \newtheorem{Def}[Cor]{Definition}
%
%    \theoremheaderfont{\scshape}
%\end{verbatim}
% Then the following are some typical examples of the typeset output
% resulting from their use.
%
% \begin{Cor}
%    This is a sentence typeset in the theorem environment {\envfont
%    Cor}.
% \end{Cor}
%
% \begin{Exa}
%    This is a sentence typeset in the theorem environment {\envfont
%    Exa}.
% \end{Exa}
%
% \begin{Rem}
%    This is a sentence typeset in the theorem environment {\envfont
%    Rem}.
% \end{Rem}
%
% \begin{Lem}[Ben User]
%    This is a sentence typeset in the theorem environment {\envfont
%    Lem}.
% \end{Lem}
%
% \begin{Def}[Very impressive Definition]
%    This is a sentence typeset in the theorem environment {\envfont
%    Def}.
% \end{Def}
%
% The last two examples show the effect of the optional argument to
% a theorem environment (it is the text typeset in parentheses).
%
%
% \section{Special Considerations}
% \label{sec:newfonts}
%
% Theoremheader and body are implemented as a unit. This means that
% the |\theoremheaderfont|  will inherit characteristics of the
% |\theorembodyfont| in \LaTeXe. Thus, if for example
% |\theorembodyfont| is |\itshape| and |\theoremheaderfont|
% is |\bfseries| the font selected for the header will have the
% characteristics `bold extended italic'. If this is not desired one
% should set the |\theoremheaderfont| to something like
% \begin{verbatim}
%      \theoremheaderfont{\normalfont\bfseries}
%\end{verbatim}
% i.e.\ supplying all necessary font information explicitly.
%
%
%
% \section{Acknowledgements}
%
% The publication of this set of macros was only possible with the
% help of Christina Busse (translating the manuscript into English),
% Joachim Pense (playing the r\^ole of typist), Chris Rowley
% (looking everything over) and many others providing useful
% suggestions.
%
% \MaybeStop{%
%    \begin{thebibliography}{9}
%       \bibitem{bk:GMS94} \textsc{M.~Goossens}, \textsc{F.~Mittelbach}
%         and \textsc{A.~Samarin}.
%         \newblock The \LaTeX{} Companion.
%         \newblock
%         Addison-Wesley, Reading, Massachusetts, 1994.
%       \bibitem{latex:source} \textsc{Lamport, Leslie.} \newblock
%                               \texttt{latex.tex}, version 2.09, date
%                               Feb. 1990.
%    \end{thebibliography}
%    \PrintIndex
%    \PrintChanges}
%
%
% \section{The documentation driver file}
%
% The next bit of code contains the documentation driver file for
% \TeX{}, i.e., the file that will produce the documentation you are
% currently reading. It will be extracted from this file by the
% \texttt{docstrip} program. Since it is the first code in the file
% one can alternatively process this file directly with \LaTeXe{} to
% obtain the documentation.
%
%    \begin{macrocode}
%<*driver>
\documentclass{ltxdoc}

\usepackage{theorem}

%<+driver>% The next few lines define theorem sets which are used
%<+driver>% in the example section of the documentation.

\theoremstyle{break}       \newtheorem{Cor}{Corollary}
\theoremstyle{plain}       \newtheorem{Exa}{Example}[section]
{\theorembodyfont{\upshape}\newtheorem{Rem}{Remark}}
\theoremstyle{marginbreak} \newtheorem{Lem}[Cor]{Lemma}
\theoremstyle{change}
\theorembodyfont{\itshape} \newtheorem{Def}[Cor]{Definition}

\theoremheaderfont{\scshape}

\RecordChanges

\begin{document}
   \DocInput{theorem.dtx}
\end{document}
%</driver>
%    \end{macrocode}
%
% \section{Definition of the Macros}
%
% If the file has been loaded before, we abort immediately. If not the
% package announces itself (this is actually done at the very top of
% the file---the way it is done isn't good style so don't copy it).
% \changes{v2.0e}{89/07/19}{Spaces removed from `typeout'.}
% \changes{v2.2b}{95/11/19}{Announce moved to top of file}
%    \begin{macrocode}
%<*package>
%\@ifundefined{theorem@style}{}{\endinput}
%\def\FMithmInfo{1995/11/19 v2.2b Theorem extension package (FMi)}
%\ProvidesPackage{theorem}[\FMithmInfo]
%    \end{macrocode}
%
%
% \subsection{Definition of theorem styles and fonts}
%
% All the definitions in this file are done globally to allow
% inputting this file inside a group.
%
% \begin{macro}{\theoremstyle}
%    Before a theorem style can be installed, the chosen style must be
%    known. For that reason, we must test to see that
%    |\th@|\meta{style} is known or, more precisely, that it is
%    different from |\relax|. If the style is not known then
%    |\th@plain| is used.
%    \begin{macrocode}
\gdef\theoremstyle#1{%
   \@ifundefined{th@#1}{\@warning
          {Unknown theoremstyle `#1'. Using `plain'}%
          \theorem@style{plain}}%
%    \end{macrocode}
%    We save the theorem style to be used in the token register
%    |\theorem@style|.
%    \begin{macrocode}
      {\theorem@style{#1}}%
%    \end{macrocode}
%    Now we ``evaluate'' the theorem style: this means, we call the
%    macro |\th@|\meta{style} which will activate the relevant
%    definitions which are contained in a separate file.  This is done
%    in a group to suppress changes to the current font.  This could
%    otherwise pose problems together with the new font selection
%    scheme\footnote{When I printed the original article using the new
%    font selection scheme I ended with a document with slanted
%    typefaces (text headings and all) simply because one of the
%    theorem styles used \texttt{\bslash sl} at toplevel.} if the
%    |\th@|\meta{style} is evaluated a second time.
% \changes{v2.1a}{90/02/15}{Group added around `th@..}
% \changes{v2.1a}{90/02/15}{Reseting of `..@begintheorem removed.}
%    \begin{macrocode}
      \begingroup
        \csname th@\the\theorem@style \endcsname
      \endgroup}
%    \end{macrocode}
% \end{macro}
%
%
% \begin{macro}{\@begintheorem}
% \changes{v2.1a}{90/02/15}{Relaxed at toplevel.}
% \begin{macro}{\@opargbegintheorem}
% \changes{v2.1a}{90/02/15}{Relaxed at toplevel.}
%    We reset |\@begintheorem| and |\@opargbegintheorem| to
%    |\relax| since these commands are no longer necessary at
%    toplevel. This will save a few tokens.
%    \begin{macrocode}
\global\let\@begintheorem\relax
\global\let\@opargbegintheorem\relax
%    \end{macrocode}
% \end{macro}
% \end{macro}
%
% \begin{macro}{\theorem@style}
%    Obviously the \textsf{token} register used above has to be
%    allocated. To assure the utmost compatibility with the
%    original \LaTeX{} definition, we set the default theorem style to
%    \texttt{plain}, which implements the usual \LaTeX{} convention.
% \changes{v2.0f}{89/07/20}{Unnecessary `global' removed.}
%    \begin{macrocode}
\newtoks\theorem@style
\global\theorem@style{plain}
%    \end{macrocode}
% \end{macro}
%
% \begin{macro}{\theorembodyfont}
% \begin{macro}{\theorem@bodyfont}
%    If the |\theorembodyfont| is set by the user then it should not
%    interact with the default font set in the theorem style. When the
%    new font selection is in force this may happen if, for example,
%    the default is |\itshape| and the new |\theorembodyfont| is
%    |\sffamily|. So we add a |\reset@font| command in front of the
%    user definition.
% \changes{v2.0f}{89/07/20}{Unnecessary `global' removed.}
% \changes{v2.0g}{89/09/19}{`theoremfont to `theorembodyfont.}
% \changes{v2.1c}{92/06/22}{`reset@font added and logic changed.}
% \changes{v2.1d}{92/10/28}{Reset to empty if arg is empty}
%    \begin{macrocode}
\gdef\theorembodyfont#1{%
%    \end{macrocode}
%    We check if the argument supplied is empty and if so put nothing
%    into the |\theorem@bodyfont| token register to allow for
%    |\theorembodyfont{}| as a mean of using the default of the
%    current |\theoremstyle|.
%    \begin{macrocode}
   \def\@tempa{#1}%
   \ifx\@tempa\@empty
    \theorem@bodyfont{}%
   \else
    \theorem@bodyfont{\reset@font#1}%
   \fi
   }
\newtoks\theorem@bodyfont
\global\theorem@bodyfont{}
%    \end{macrocode}
% \end{macro}
% \end{macro}
%
%
% \begin{macro}{\theoremheaderfont}
% \changes{v2.0g}{89/09/19}{Macro added.}
%    The font for the theorem headers is handled differently because
%    this definition applies to all theorem styles.
%    \begin{macrocode}
\gdef\theoremheaderfont#1{\gdef\theorem@headerfont{#1}%
%    \end{macrocode}
%    After using the macro once it is redefined to produce an error
%    message.
%    \begin{macrocode}
       \gdef\theoremheaderfont##1{%
        \typeout{\string\theoremheaderfont\space should be used
                 only once.}}}
%    \end{macrocode}
% \end{macro}
%
% \begin{macro}{\theorem@headerfont}
% \changes{v2.0g}{89/09/19}{Macro added.}
%    To set the |\theorem@headerfont| default we first test if
%    the new fontselection scheme is in force.
% \changes{v2.1a}{90/02/15}{Support for new font selection.}
%    \begin{macrocode}
\ifx\upshape\undefined
%    \end{macrocode}
%    If not we define it to expand into |\bfseries|. We don't use
%    |\let| just in case a following style option redefines this
%    macro.
%    \begin{macrocode}
\gdef\theorem@headerfont{\bfseries}
%    \end{macrocode}
%    Otherwise we reset the current shape before calling |\bfseries|.
%    \begin{macrocode}
\else \gdef\theorem@headerfont{\normalfont\bfseries}\fi
%    \end{macrocode}
% \end{macro}
%
% \begin{macro}{\th@plain}
% \begin{macro}{\th@break}
% \begin{macro}{\th@marginbreak}
% \begin{macro}{\th@changebreak}
% \begin{macro}{\th@change}
% \begin{macro}{\th@margin}
%    The different styles are defined in macros such as
%    |\th@plain|.  Since memory space is precious in
%    ``non-Big-versions'', we have to avoid offering too many unused
%    definitions.  Therefore we define these styles in separate files
%    that can be loaded on demand.  Thus the commands themselves only
%    load these files. We use |\@input@|, a \LaTeXe{} internal command
%    that ensures that the file will be listed with |\listfiles|.
% \changes{v2.2a}{94/02/02}{Use `@input@ to load theorem layout files}
%    \begin{macrocode}
\gdef\th@plain{\@input@{thp.sty}}
\gdef\th@break{\@input@{thb.sty}}
\gdef\th@marginbreak{\@input@{thmb.sty}}
\gdef\th@changebreak{\@input@{thcb.sty}}
\gdef\th@change{\@input@{thc.sty}}
\gdef\th@margin{\@input@{thm.sty}}
%    \end{macrocode}
%    This list will be expanded when new styles become
%    available.  For testing,  just append new theorem substyles
%    as document options.
% \end{macro}
% \end{macro}
% \end{macro}
% \end{macro}
% \end{macro}
% \end{macro}
%
%
%
% \subsection{Definition of a new theorem set}
%
% As already pointed out, a new theorem environment can be defined in
% three different ways:
% \begin{verbatim}
%    \newtheorem{Lem}{Lemma}
%    \newtheorem{Lem}{Lemma}[section]
%    \newtheorem{Lem}[Theorem]{Lemma}
%\end{verbatim}
% The function of the macro |\newtheorem| is to recognize these cases
% and then to branch into one of the three macros |\@ynthm|, |\@xnthm|
% or |\@othm|.  This mechanism is adopted unchanged from
% \cite{latex:source}; the essential point here is that, for example,
% in the second case, the arguments \texttt{Lem}, \texttt{Lemma} and
% \texttt{section} are passed over to the macro |\@xnthm|.
%
% We inspect this case first because the others present fewer problems,
% and thus are easily derived from this one.
%
% \begin{macro}{\@xnthm}
%    For our example arguments, the macro |\@xnthm| must fulfill
%    the following:
%    \begin{itemize}
%       \item
%          Define a new \LaTeX-counter `Lem'
%       \item
%          reset this counter within a |\section|
%       \item
%          define the macro |\theLem|
%       \item
%          define the environment macros |\Lem| and
%          |\endLem| using the current
%          |\theoremstyle| and |\theorem@bodyfont|.
%    \end{itemize}
%    Obviously, all this should happen only if the first argument
%    of |\@xnthm| (i.e.\ \texttt{Lem} in our example) is chosen so
%    as not to conflict with any previously defined commands or
%    environments.  This test is performed by the \LaTeX{} macro
%    |\@ifdefinable|.
%    \begin{macrocode}
\gdef\@xnthm#1#2[#3]{\expandafter\@ifdefinable\csname #1\endcsname
%    \end{macrocode}
%    Therefore, the first argument of |\@ifdefinable| is the
%    expansion (in the example, |\Lem|) of
%    |\csname||#1\endcsname|.  The second argument is
%    executed only if the test has been completed successfully.
%    \begin{macrocode}
   {%
%    \end{macrocode}
%    Now we define the new counter. The names of the \LaTeX{}
%    macros employed should speak for themselves:
% \changes{v2.2b}{1995/10/07}{Use \cs{@newctr}}
% \changes{v2.2c}{1995/11/23}{Use \cs{@newctr} properly.}
%    \begin{macrocode}
    \@definecounter{#1}\@newctr{#1}[#3]%
%    \end{macrocode}
%    Using |\@newctr| will give a proper error message if the counter
%    in |#3| is not defined.
%    In defining `|\theLem|' we must generate the desired macro
%    name by use of |\expandafter| and |\csname|.
%    \begin{macrocode}
    \expandafter\xdef\csname the#1\endcsname
%    \end{macrocode}
%    An |\xdef| is used in order to make the definition global,
%    and to ensure that it contains the replacement texts of
%    |\@thmcountersep| and |\@thmcounter|.\footnote{These
%    two macros can be defined by the document style.  Their default
%    values produce a `\texttt{.}' as separation and an arabic
%    representation of the number.} However, not everything should be
%    expanded. For example, it saves space to use |\thesection|
%    instead of its---at times---lengthy expansion.
%    \begin{macrocode}
      {\expandafter \noexpand \csname the#3\endcsname
       \@thmcountersep \@thmcounter{#1}}%
%    \end{macrocode}
%    Thus with the defaults of \LaTeX{}, |\theLem| would be
%    replaced by the command sequence |\thesection.\arabic{Lem}|.
%
%    We will now look at the definition of the macro which is executed
%    at the beginning of the actual environment (in our example this
%    macro is |\Lem|). It should be noted that we use an
%    ``|\expandafter| trick'' to expand only certain parts of the
%    replacement text at the time of the definition.
%    \begin{macrocode}
    \def\@tempa{\global\@namedef{#1}}%
    \expandafter \@tempa \expandafter{%
%    \end{macrocode}
%    First, the macro that contains the current definitions of
%    |\@begintheorem| and |\@opargtheorem| should be called
%    up.  The name of this macro---as is already known---has the form
%    |\th@|\meta{theorem\ style}; therefore, it must be called by
%    \begin{macrocode}
      \csname th@\the \theorem@style
%    \end{macrocode}
%    In addition the default theorem font should be changeable, i.e.\
%    we have to insert the contents of |\theorem@bodyfont|. For
%    that reason, we expand even further, beyond |\endcsname|,
%    and thus insert the contents of the \textsf{token} register
%    |\theorem@bodyfont| in the replacement text.
%    \begin{macrocode}
            \expandafter \endcsname \the \theorem@bodyfont
%    \end{macrocode}
%    Now it is time to call the macro |\@thm| which takes over
%    the further processing.  It has two arguments: the current
%    counter name (in our example, \texttt{Lem}), and the text of the
%    label (in our example, \texttt{Lemma}).
%    \begin{macrocode}
     \@thm{#1}{#2}}%
%    \end{macrocode}
%    With this, the `sub-definition' is complete.  The macro
%    |\@endtheorem| ends a theorem environment and is, so far, nothing
%    but an |\endtrivlist|.  (Hence it is defined globally, and not
%    within the theorem styles.\footnote{This has to be changed as
%    soon as theorem styles that change \texttt{\bslash @endtheorem}
%    exist.  In such a case, all existing styles must be changed as
%    well since they will have to reset the macro.}) Therefore, we can
%    set it equivalent to the macro that ends the theorem set (in our
%    example, |\endLem|).  However, if some day theorem styles exist
%    that do change |\@endtheorem|, we would have to use the
%    commented-out line instead.
%    \begin{macrocode}
    \global \expandafter \let \csname end#1\endcsname \@endtheorem
%   \global\@namedef{end#1}{\@endtheorem}%
%    \end{macrocode}
%    With these commands all the required definitions are employed,
%    unless the test |\@ifdefinable| has failed.  Therefore, we
%    end the second argument of this macro and with it the definition
%    of |\@xnthm|.
%    \begin{macrocode}
   }}
%    \end{macrocode}
% \end{macro}
%
% \begin{macro}{\@ynthm}
%    The definition of |\@ynthm| is completely analogous. In
%    this case the new counter that is defined is not reset within
%    another counter; thus the definition of |\the...|
%    is simplified:
%    \begin{macrocode}
\gdef\@ynthm#1#2{\expandafter\@ifdefinable\csname #1\endcsname
   {\@definecounter{#1}%
    \expandafter\xdef\csname the#1\endcsname{\@thmcounter{#1}}%
%    \end{macrocode}
%    The rest of the definition corresponds literally to that of
%    |\@xnthm|:
%    \begin{macrocode}
    \def\@tempa{\global\@namedef{#1}}\expandafter \@tempa
     \expandafter{\csname th@\the \theorem@style \expandafter
     \endcsname \the\theorem@bodyfont \@thm{#1}{#2}}%
    \global \expandafter \let \csname end#1\endcsname \@endtheorem}}
%    \end{macrocode}
% \end{macro}
%
% \begin{macro}{\@othm}
%    The definition of |\@othm| does not contain anything new.
%    \begin{macrocode}
\gdef\@othm#1[#2]#3{%
%    \end{macrocode}
% \changes{v2.2b}{95/10/07}{Build in check for counter defined}
%    We do not define a new counter but instead use one that has
%    already been defined.  Thus the only definition we need is that
%    of this pseudo-counter (i.e.\ |\the|\meta{env.\ name}).
%    First we check if |#2| corresponds to a known counter name.
%    \begin{macrocode}
  \expandafter\ifx\csname c@#2\endcsname\relax
   \@nocounterr{#2}%
  \else
   \expandafter\@ifdefinable\csname #1\endcsname
   {\expandafter \xdef \csname the#1\endcsname
     {\expandafter \noexpand \csname the#2\endcsname}%
%    \end{macrocode}
%    All other parts of the definition can be adopted from
%    |\@xnthm|. We have to remember, though, that in this case
%    the name of the current counter and the theorem label have moved
%    to the second and third arguments.
%    \begin{macrocode}
    \def\@tempa{\global\@namedef{#1}}\expandafter \@tempa
     \expandafter{\csname th@\the \theorem@style \expandafter
     \endcsname \the\theorem@bodyfont \@thm{#2}{#3}}%
    \global \expandafter \let \csname end#1\endcsname \@endtheorem}%
  \fi}
%    \end{macrocode}
% \end{macro}
%
% \subsection{Macros that are employed in a {\envfont theorem}
%    environment}
%
% \begin{macro}{\@thm}
%    The macro |\@thm| has to increase the current counter.  Then,
%    depending on whether the environment has (or does not have) an
%    optional argument, it has to branch into either |\@begintheorem|
%    or |\@opargtheorem|.
%    \begin{macrocode}
\gdef\@thm#1#2{\refstepcounter{#1}%
%    \end{macrocode}
%    Now we start a {\envfont trivlist} environment, and give
%    |\@topsep| and |\@topsepadd| the values of the \textsf{skip}
%    registers |\theorempreskipamount| and |\theorempostskipamount|.
%    The value in |\@topsep| is the vertical space that is inserted by
%    the first (and only) |\item| in our |\trivlist| whilst
%    |\@topsepadd| is inserted by |\@endparenv| at the end of that
%    {\envfont trivlist} environment.  By using these registers, we
%    obtain the desired space around a {\envfont theorem} environment.
%    \begin{macrocode}
   \trivlist
   \@topsep \theorempreskipamount               % used by first \item
   \@topsepadd \theorempostskipamount           % used by \@endparenv
%    \end{macrocode}
%    Now we have to test whether an optional argument has been given.
% \changes{v2.0d}{89/07/18}{Forgotten percent added.}
%    \begin{macrocode}
   \@ifnextchar [%
%    \end{macrocode}
%    If there is an optional argument, we will call |\@ythm|, and
%    move the arguments read back into the input stream.
%    \begin{macrocode}
   {\@ythm{#1}{#2}}%
%    \end{macrocode}
%    If not, we call |\@begintheorem|. Its first argument is the
%    name of the theorem set (hence the second argument of
%    |\@thm|). Its second argument is the macro that produces the
%    current number.
%    \begin{macrocode}
   {\@begintheorem{#2}{\csname the#1\endcsname}\ignorespaces}}
%    \end{macrocode}
% \end{macro}
%
% \begin{macro}{\@xthm}
% \begin{macro}{\@ythm}
%    Both these macros were originally called by |\@thm|.  We do
%    not need |\@xthm| anymore, hence we reset it to
%    |\relax|. The definition of |\@ythm| has not changed at
%    all from its definition in \LaTeX. In order to make the macros
%    easier to understand, we will
%    nevertheless present it (commented out).
%    \begin{macrocode}
\global\let\@xthm\relax
% \def\@ythm#1#2[#3]{\@opargbegintheorem{#2}{\csname
%           the#1\endcsname}{#3}\ignorespaces}
%</package>
%    \end{macrocode}
%    The primitive |\ignorespaces| in |\@ythm| and
%    |\@thm| is needed to remove the spaces between the
%    |\begin{...}| and the actual text.
% \end{macro}
% \end{macro}
%
%
% \subsection{Definition of the theorem substyles}
%
% As already pointed out, the theorem substyles, defined below, are
% only loaded when necessary.  Note that all these substyles, except
% \texttt{plain}, have |\slshape| as the default body font.
%
% \subsubsection{The \texttt{plain} style}
%
% \label{dok:th@plain} As the following macros use |@|, we have
% to locally set the |\catcode| of this symbol to ``letter''.
% This happens within a group, so that we do not have to worry about
% which |\catcode| that symbol had before.
%    \begin{macrocode}
%<*thp>
\begingroup \makeatletter
%    \end{macrocode}
% Since we are now within a group, we must make all definitions
% globally.  First we make sure that \texttt{theorem.sty} is loaded.
% This will allow us to use this file as a document style option
% without having to call \texttt{theorem} itself as an option.  At the
% same time, we assure that at least version 2 is loaded, since
% |\theorem@style| was not defined in earlier versions.
%    \begin{macrocode}
\@ifundefined{theorem@style}{% \iffalse meta-comment
%
% Copyright (C) 1993-2024
%
% The LaTeX Project and any individual authors listed elsewhere
% in this file.
%
% This file is part of the Standard LaTeX `Tools Bundle'.
% -------------------------------------------------------
%
% It may be distributed and/or modified under the
% conditions of the LaTeX Project Public License, either version 1.3c
% of this license or (at your option) any later version.
% The latest version of this license is in
%    https://www.latex-project.org/lppl.txt
% and version 1.3c or later is part of all distributions of LaTeX
% version 2005/12/01 or later.
%
% The list of all files belonging to the LaTeX `Tools Bundle' is
% given in the file `manifest.txt'.
%
% \fi
%
%%
%
%
\def\FMithmInfo{2023/07/05 v2.2c Theorem extension package (FMi)}
%
% \ProvidesFile{theorem.dtx}[\FMithmInfo]
%\iffalse   % this is a METACOMMENT !
%%
%% Package `theorem' to use with LaTeX2e
%% Copyright (C) 1989-1995 Frank Mittelbach, all rights reserved.
%<+package>\@ifundefined{theorem@style}{}{\endinput}
%<+package>\NeedsTeXFormat{LaTeX2e}[1995/06/01]
%<+package>\ProvidesPackage{theorem}
%<+package>         [\FMithmInfo]
%
% \fi
%
% \changes{v1.0}{87/05/07}{First draft.}
% \changes{v1.0a}{87/05/07}{`thistheoremstyle' added.}
% \changes{v1.1a}{87/07/29}{New concept: redefine the internal
%                          \protect\LaTeX{} commands `@xnthm' etc.}
% \changes{v1.1b}{87/07/31}{`theoremfont' used to specify font.}
% \changes{v1.2a}{88/01/08}{`@endtheorem' now globally defined.}
% \changes{v1.2b}{88/01/11}{`newtheorem' only allowed in the preamble.}
% \changes{v1.3a}{88/02/14}{`theoremfont' now a toks register.}
%
% \changes{v2.0a}{88/11/16}{New implementation and new documentation.}
% \changes{v2.0a}{88/11/16}{No layout changes but internal modifications
%                           old substyles won't work.}
% \changes{v2.0b}{89/01/30}{`global' used for all definitions.}
% \changes{v2.0c}{89/06/15}{Documentation translated to English.
%                            Thanks to LLL and Nina}
% \changes{v2.1a}{90/02/15}{Support for new font selection.}
% \changes{v2.2a}{94/02/03}{Cleanup for LaTeX2e}
%
% \DoNotIndex{\@endparenv,\@ifdefinable,\@ifnextchar,\@ifundefined}
% \DoNotIndex{\@namedef,\@nameuse,\@spaces,\@tempa,\@warning}
% \DoNotIndex{\ }
% \DoNotIndex{\begingroup,\bfseries}
% \DoNotIndex{\csname}
% \DoNotIndex{\def,\do,\docdate}
% \DoNotIndex{\endcsname,\endgroup,\endinput,\endtrivlist,\expandafter}
% \DoNotIndex{\filedate,\fileversion}
% \DoNotIndex{\gdef,\global}
% \DoNotIndex{\hbox,\hskip}
% \DoNotIndex{\ignorespaces,\input,\itshape,\item}
% \DoNotIndex{\labelsep,\let,\llap}
% \DoNotIndex{\makeatletter}
% \DoNotIndex{\noexpand}
% \DoNotIndex{\refstepcounter,\relax,\rlap}
% \DoNotIndex{\setlength,\slshape,\space,\string,\strut}
% \DoNotIndex{\the,\trivlist,\typeout}
% \DoNotIndex{\vbox}
% \DoNotIndex{\xdef}
%
% \def\deflabel#1{\ttfamily #1\hfill}
% \def\deflist#1{\begin{list}{}{\settowidth\labelwidth{\ttfamily #1}%
%                               \setlength\leftmargin\labelwidth
%                               \addtolength\leftmargin\labelsep
%                               \let\makelabel\deflabel}}
% \def\enddeflist{\end{list}}
% \def\envfont{\normalfont\ttfamily}
%
% \MakeShortVerb{\|}
%
% \newcommand{\TUB}{{\sl TUGboat\/}}
%
% \GetFileInfo{theorem.dtx}
% \title{An Extension of the
%        \LaTeX{} theorem environment\thanks{This file
%        has version number \fileversion, last
%        revised \filedate.}}
% \author{Frank Mittelbach}
%
% \MaintainedByLaTeXTeam{tools}
% \maketitle
%
% \begin{abstract}
%    The macros described in this paper yield an extension of the
%    \LaTeX{} theorem mechanism. It is designed to satisfy the
%    different requirements of various journals. Thus, the
%    layout of the ``theorems'' can be manipulated by determining a
%    ``style''. This article describes not only the use, but
%    also the definition, of the necessary macros.
% \end{abstract}
%
% \section*{Preface to version 2.2}
%
% For \LaTeXe{} this package did not need any fundamental changes. I
% only modified the messages generated so that theorem layout styles
% will show up with the |\listfiles| command and cleaned the section
% on the New Font Selection Scheme since this is now included in
% \LaTeX.
%
% \section*{Preface to version 2.1}
%
% This version is identical to 2.0g described in \TUB\ 10\#3 except
% for some internal defaults which are now set depending on the used
% font selection scheme.
%
% This was done to avoid unpleasant surprises if the new font selection
% scheme is in force. For further details see
% section~\ref{sec:newfonts} and \cite{bk:GMS94}.
%
% \section{Introduction}
%
% For our purposes here, ``theorems'' are labelled enunciations,
% often set off from the main text by extra space and a font change.
% Theorems, corollaries, conjectures, definitions, and remarks
% are all instances of ``theorems''.  The ``header'' of these structures
% is composed of a label (such as \textsc{Theorem} or \textsc{Remark})
% and a number which serializes an item in the sequence of items
% with the same label.
%
% Shortly after the introduction of \LaTeX{} at the Fachbereich
% Mathematik in Mainz, the desire to manipulate the layout of
% ``theorems'' arose.  In Mainz, the following two conventions came
% into general use:
% \begin{enumerate}
%   \item The number of the theorem is shown in the margin.
%   \item There is a line break at
%         the end of the theorem header.
% \end{enumerate}
%
% Additionally, some journals require different formats which depend on
% the ``sort of theorem'': e.g.\ often remarks and definitions are set
% in |\upshape|, while |\itshape| is employed for main theorems.
%
% Confronted with these requirements, a theorem environment was
% developed in Mainz which allows separate determination of the
% layout of the ``theorems sets'', comparable to |\pagestyle|.
%
%
% \section{The user interface}
% \subsection{Defining new theorem sets}
%
% \DescribeMacro\newtheorem
% As in the original \LaTeX{} version, the command |\newtheorem|
% defines a new ``theorem set'' or ``theorem-like structure''.
% Two required arguments name the new environment and give the text
% to be typeset with each instance of the new ``set'', while
% an optional argument determines how
% the ``set'' is enumerated:
% \begin{description}
%    \item[\ttfamily \bslash newtheorem\{foo\}\{bar\}]
%       The theorem set {\envfont foo} (whose name is \texttt{bar})
%       uses its own counter.
%    \item[\ttfamily \bslash newtheorem\{foo2\}{[foo]}\{bar2\}]
%       The theorem set {\envfont foo2} (printed name \texttt{bar2})
%       uses the same counter as the theorem set \texttt{foo}.
%    \item[\ttfamily \bslash newtheorem\{foo3\}\{bar3\}{[section]}]
%       The theorem set {\envfont foo3} (printed name \texttt{bar3}) is
%       enumerated within the counter \texttt{section}, i.e.\ with every
%       new |\section| the enumeration begins again with 1, and
%       the enumeration is composed from the section-number and the
%       theorem counter itself.
% \end{description}
%
% \DescribeMacro\theoremstyle
% Additionally, the command |\theoremstyle| can define the layout
% of various, or all, theorem sets.  It should be noted that any
% theorem set defined by |\newtheorem| is typeset in the
% |\theoremstyle| that is current at the time of the definition.
% Thus, the following
% \begin{verbatim}
%    \theoremstyle{break}        \newtheorem{Cor}{Corollary}
%    \theoremstyle{plain}        \newtheorem{Exa}{Example}[section]
%\end{verbatim}                             ^^A No blank at line start
% leads to the result that the set {\envfont Cor} is formatted in the
% style \texttt{break}, while the set {\envfont Exa} and all the
% following ones are formatted in the style \texttt{plain}, unless
% another |\theoremstyle| follows.  Since the definitions installed by
% |\newtheorem| are global, one also can limit |\theoremstyle| locally
% by grouping braces.
%
% \DescribeMacro\theorembodyfont
% The choice of the font for the theorem body is completely
% independent of the chosen |\theoremstyle|; this has proven to
% be very advantageous.  For example,
% \begin{verbatim}
%   {\theorembodyfont{\upshape}          \newtheorem{Rem}{Remark}}
%\end{verbatim}
% defines a theorem set {\envfont Rem}, which will be set in
% |\upshape| in the current layout (which in our example is
% \texttt{plain}).  As with |\theoremstyle|, the |\theorembodyfont|
% chosen is that current at the time of |\newtheorem|.  If
% |\theorembodyfont| is not specified or one defines
% |\theorembodyfont{}|, then the font used will be that defined by the
% |\theoremstyle|.
%
% \DescribeMacro\theoremheaderfont It is also possible to customize
% the font used for the theorem headers.  This is, however, a global
% declaration, and therefore there should be at most one
% |\theoremheaderfont| declaration in the preamble.\footnote{If it is
% actually necessary to have different header fonts, one has to define
% new theorem styles (substituting the desired font) or specify the
% information directly in the \texttt{\bslash newtheorem} declaration
% (the unclean variant).}
%
% \DescribeMacro\theorempreskipamount
% \DescribeMacro\theorempostskipamount Two additional parameters
% affect the vertical space around the theorem environments:
% |\theorempreskipamount| and |\theorempostskipamount| define,
% respectively, the spacing before and after such an environment.
% These parameters apply for all theorem sets and can be manipulated
% with the ordinary length macros.  They are rubber lengths
% (`\textsf{skips}'), and therefore can contain \texttt{plus} and
% \texttt{minus} parts.
%
% Since the definition of theorem sets should---most sensibly---be
% placed in the preamble, we only allow installation there.
% It is therefore possible to release the memory used here after
% |\begin{document}|, in order to make room for other
% applications.
%
%
% \subsection{Existing theorem styles}
%
% The following theorem styles exist to date:
% \begin{deflist}{marginbreak}
%    \item[plain]
%       This theorem style emulates the original \LaTeX{} definition,
%       except that additionally the parameters
%       |\theorem...skipamount| are used.
%    \item[break]
%       In this style, the theorem header is followed by a line
%       break.
%    \item[marginbreak]
%       The theorem number is set in the margin, and there is a line
%       break as in \texttt{break}.
%    \item[changebreak]
%       Like \texttt{break}, but with header number and text
%       interchanged.
%    \item[change]
%       Header number and text are interchanged, without a line break.
%    \item[margin]
%       The number is set in the left margin, without a line break.
% \end{deflist}
% All styles (except \texttt{plain}) select |\slshape| as the default
% |\theorembodyfont|.
%
%
% \subsection{Examples}
%
% Given the above theorem sets {\envfont Cor}, {\envfont Exa}
% and {\envfont Rem},
% suppose that the preamble also contains the declarations:
% \begin{verbatim}
%    \theoremstyle{marginbreak}   \newtheorem{Lem}[Cor]{Lemma}
%    \theoremstyle{change}
%    \theorembodyfont{\itshape}        \newtheorem{Def}[Cor]{Definition}
%
%    \theoremheaderfont{\scshape}
%\end{verbatim}
% Then the following are some typical examples of the typeset output
% resulting from their use.
%
% \begin{Cor}
%    This is a sentence typeset in the theorem environment {\envfont
%    Cor}.
% \end{Cor}
%
% \begin{Exa}
%    This is a sentence typeset in the theorem environment {\envfont
%    Exa}.
% \end{Exa}
%
% \begin{Rem}
%    This is a sentence typeset in the theorem environment {\envfont
%    Rem}.
% \end{Rem}
%
% \begin{Lem}[Ben User]
%    This is a sentence typeset in the theorem environment {\envfont
%    Lem}.
% \end{Lem}
%
% \begin{Def}[Very impressive Definition]
%    This is a sentence typeset in the theorem environment {\envfont
%    Def}.
% \end{Def}
%
% The last two examples show the effect of the optional argument to
% a theorem environment (it is the text typeset in parentheses).
%
%
% \section{Special Considerations}
% \label{sec:newfonts}
%
% Theoremheader and body are implemented as a unit. This means that
% the |\theoremheaderfont|  will inherit characteristics of the
% |\theorembodyfont| in \LaTeXe. Thus, if for example
% |\theorembodyfont| is |\itshape| and |\theoremheaderfont|
% is |\bfseries| the font selected for the header will have the
% characteristics `bold extended italic'. If this is not desired one
% should set the |\theoremheaderfont| to something like
% \begin{verbatim}
%      \theoremheaderfont{\normalfont\bfseries}
%\end{verbatim}
% i.e.\ supplying all necessary font information explicitly.
%
%
%
% \section{Acknowledgements}
%
% The publication of this set of macros was only possible with the
% help of Christina Busse (translating the manuscript into English),
% Joachim Pense (playing the r\^ole of typist), Chris Rowley
% (looking everything over) and many others providing useful
% suggestions.
%
% \MaybeStop{%
%    \begin{thebibliography}{9}
%       \bibitem{bk:GMS94} \textsc{M.~Goossens}, \textsc{F.~Mittelbach}
%         and \textsc{A.~Samarin}.
%         \newblock The \LaTeX{} Companion.
%         \newblock
%         Addison-Wesley, Reading, Massachusetts, 1994.
%       \bibitem{latex:source} \textsc{Lamport, Leslie.} \newblock
%                               \texttt{latex.tex}, version 2.09, date
%                               Feb. 1990.
%    \end{thebibliography}
%    \PrintIndex
%    \PrintChanges}
%
%
% \section{The documentation driver file}
%
% The next bit of code contains the documentation driver file for
% \TeX{}, i.e., the file that will produce the documentation you are
% currently reading. It will be extracted from this file by the
% \texttt{docstrip} program. Since it is the first code in the file
% one can alternatively process this file directly with \LaTeXe{} to
% obtain the documentation.
%
%    \begin{macrocode}
%<*driver>
\documentclass{ltxdoc}

\usepackage{theorem}

%<+driver>% The next few lines define theorem sets which are used
%<+driver>% in the example section of the documentation.

\theoremstyle{break}       \newtheorem{Cor}{Corollary}
\theoremstyle{plain}       \newtheorem{Exa}{Example}[section]
{\theorembodyfont{\upshape}\newtheorem{Rem}{Remark}}
\theoremstyle{marginbreak} \newtheorem{Lem}[Cor]{Lemma}
\theoremstyle{change}
\theorembodyfont{\itshape} \newtheorem{Def}[Cor]{Definition}

\theoremheaderfont{\scshape}

\RecordChanges

\begin{document}
   \DocInput{theorem.dtx}
\end{document}
%</driver>
%    \end{macrocode}
%
% \section{Definition of the Macros}
%
% If the file has been loaded before, we abort immediately. If not the
% package announces itself (this is actually done at the very top of
% the file---the way it is done isn't good style so don't copy it).
% \changes{v2.0e}{89/07/19}{Spaces removed from `typeout'.}
% \changes{v2.2b}{95/11/19}{Announce moved to top of file}
%    \begin{macrocode}
%<*package>
%\@ifundefined{theorem@style}{}{\endinput}
%\def\FMithmInfo{1995/11/19 v2.2b Theorem extension package (FMi)}
%\ProvidesPackage{theorem}[\FMithmInfo]
%    \end{macrocode}
%
%
% \subsection{Definition of theorem styles and fonts}
%
% All the definitions in this file are done globally to allow
% inputting this file inside a group.
%
% \begin{macro}{\theoremstyle}
%    Before a theorem style can be installed, the chosen style must be
%    known. For that reason, we must test to see that
%    |\th@|\meta{style} is known or, more precisely, that it is
%    different from |\relax|. If the style is not known then
%    |\th@plain| is used.
%    \begin{macrocode}
\gdef\theoremstyle#1{%
   \@ifundefined{th@#1}{\@warning
          {Unknown theoremstyle `#1'. Using `plain'}%
          \theorem@style{plain}}%
%    \end{macrocode}
%    We save the theorem style to be used in the token register
%    |\theorem@style|.
%    \begin{macrocode}
      {\theorem@style{#1}}%
%    \end{macrocode}
%    Now we ``evaluate'' the theorem style: this means, we call the
%    macro |\th@|\meta{style} which will activate the relevant
%    definitions which are contained in a separate file.  This is done
%    in a group to suppress changes to the current font.  This could
%    otherwise pose problems together with the new font selection
%    scheme\footnote{When I printed the original article using the new
%    font selection scheme I ended with a document with slanted
%    typefaces (text headings and all) simply because one of the
%    theorem styles used \texttt{\bslash sl} at toplevel.} if the
%    |\th@|\meta{style} is evaluated a second time.
% \changes{v2.1a}{90/02/15}{Group added around `th@..}
% \changes{v2.1a}{90/02/15}{Reseting of `..@begintheorem removed.}
%    \begin{macrocode}
      \begingroup
        \csname th@\the\theorem@style \endcsname
      \endgroup}
%    \end{macrocode}
% \end{macro}
%
%
% \begin{macro}{\@begintheorem}
% \changes{v2.1a}{90/02/15}{Relaxed at toplevel.}
% \begin{macro}{\@opargbegintheorem}
% \changes{v2.1a}{90/02/15}{Relaxed at toplevel.}
%    We reset |\@begintheorem| and |\@opargbegintheorem| to
%    |\relax| since these commands are no longer necessary at
%    toplevel. This will save a few tokens.
%    \begin{macrocode}
\global\let\@begintheorem\relax
\global\let\@opargbegintheorem\relax
%    \end{macrocode}
% \end{macro}
% \end{macro}
%
% \begin{macro}{\theorem@style}
%    Obviously the \textsf{token} register used above has to be
%    allocated. To assure the utmost compatibility with the
%    original \LaTeX{} definition, we set the default theorem style to
%    \texttt{plain}, which implements the usual \LaTeX{} convention.
% \changes{v2.0f}{89/07/20}{Unnecessary `global' removed.}
%    \begin{macrocode}
\newtoks\theorem@style
\global\theorem@style{plain}
%    \end{macrocode}
% \end{macro}
%
% \begin{macro}{\theorembodyfont}
% \begin{macro}{\theorem@bodyfont}
%    If the |\theorembodyfont| is set by the user then it should not
%    interact with the default font set in the theorem style. When the
%    new font selection is in force this may happen if, for example,
%    the default is |\itshape| and the new |\theorembodyfont| is
%    |\sffamily|. So we add a |\reset@font| command in front of the
%    user definition.
% \changes{v2.0f}{89/07/20}{Unnecessary `global' removed.}
% \changes{v2.0g}{89/09/19}{`theoremfont to `theorembodyfont.}
% \changes{v2.1c}{92/06/22}{`reset@font added and logic changed.}
% \changes{v2.1d}{92/10/28}{Reset to empty if arg is empty}
%    \begin{macrocode}
\gdef\theorembodyfont#1{%
%    \end{macrocode}
%    We check if the argument supplied is empty and if so put nothing
%    into the |\theorem@bodyfont| token register to allow for
%    |\theorembodyfont{}| as a mean of using the default of the
%    current |\theoremstyle|.
%    \begin{macrocode}
   \def\@tempa{#1}%
   \ifx\@tempa\@empty
    \theorem@bodyfont{}%
   \else
    \theorem@bodyfont{\reset@font#1}%
   \fi
   }
\newtoks\theorem@bodyfont
\global\theorem@bodyfont{}
%    \end{macrocode}
% \end{macro}
% \end{macro}
%
%
% \begin{macro}{\theoremheaderfont}
% \changes{v2.0g}{89/09/19}{Macro added.}
%    The font for the theorem headers is handled differently because
%    this definition applies to all theorem styles.
%    \begin{macrocode}
\gdef\theoremheaderfont#1{\gdef\theorem@headerfont{#1}%
%    \end{macrocode}
%    After using the macro once it is redefined to produce an error
%    message.
%    \begin{macrocode}
       \gdef\theoremheaderfont##1{%
        \typeout{\string\theoremheaderfont\space should be used
                 only once.}}}
%    \end{macrocode}
% \end{macro}
%
% \begin{macro}{\theorem@headerfont}
% \changes{v2.0g}{89/09/19}{Macro added.}
%    To set the |\theorem@headerfont| default we first test if
%    the new fontselection scheme is in force.
% \changes{v2.1a}{90/02/15}{Support for new font selection.}
%    \begin{macrocode}
\ifx\upshape\undefined
%    \end{macrocode}
%    If not we define it to expand into |\bfseries|. We don't use
%    |\let| just in case a following style option redefines this
%    macro.
%    \begin{macrocode}
\gdef\theorem@headerfont{\bfseries}
%    \end{macrocode}
%    Otherwise we reset the current shape before calling |\bfseries|.
%    \begin{macrocode}
\else \gdef\theorem@headerfont{\normalfont\bfseries}\fi
%    \end{macrocode}
% \end{macro}
%
% \begin{macro}{\th@plain}
% \begin{macro}{\th@break}
% \begin{macro}{\th@marginbreak}
% \begin{macro}{\th@changebreak}
% \begin{macro}{\th@change}
% \begin{macro}{\th@margin}
%    The different styles are defined in macros such as
%    |\th@plain|.  Since memory space is precious in
%    ``non-Big-versions'', we have to avoid offering too many unused
%    definitions.  Therefore we define these styles in separate files
%    that can be loaded on demand.  Thus the commands themselves only
%    load these files. We use |\@input@|, a \LaTeXe{} internal command
%    that ensures that the file will be listed with |\listfiles|.
% \changes{v2.2a}{94/02/02}{Use `@input@ to load theorem layout files}
%    \begin{macrocode}
\gdef\th@plain{\@input@{thp.sty}}
\gdef\th@break{\@input@{thb.sty}}
\gdef\th@marginbreak{\@input@{thmb.sty}}
\gdef\th@changebreak{\@input@{thcb.sty}}
\gdef\th@change{\@input@{thc.sty}}
\gdef\th@margin{\@input@{thm.sty}}
%    \end{macrocode}
%    This list will be expanded when new styles become
%    available.  For testing,  just append new theorem substyles
%    as document options.
% \end{macro}
% \end{macro}
% \end{macro}
% \end{macro}
% \end{macro}
% \end{macro}
%
%
%
% \subsection{Definition of a new theorem set}
%
% As already pointed out, a new theorem environment can be defined in
% three different ways:
% \begin{verbatim}
%    \newtheorem{Lem}{Lemma}
%    \newtheorem{Lem}{Lemma}[section]
%    \newtheorem{Lem}[Theorem]{Lemma}
%\end{verbatim}
% The function of the macro |\newtheorem| is to recognize these cases
% and then to branch into one of the three macros |\@ynthm|, |\@xnthm|
% or |\@othm|.  This mechanism is adopted unchanged from
% \cite{latex:source}; the essential point here is that, for example,
% in the second case, the arguments \texttt{Lem}, \texttt{Lemma} and
% \texttt{section} are passed over to the macro |\@xnthm|.
%
% We inspect this case first because the others present fewer problems,
% and thus are easily derived from this one.
%
% \begin{macro}{\@xnthm}
%    For our example arguments, the macro |\@xnthm| must fulfill
%    the following:
%    \begin{itemize}
%       \item
%          Define a new \LaTeX-counter `Lem'
%       \item
%          reset this counter within a |\section|
%       \item
%          define the macro |\theLem|
%       \item
%          define the environment macros |\Lem| and
%          |\endLem| using the current
%          |\theoremstyle| and |\theorem@bodyfont|.
%    \end{itemize}
%    Obviously, all this should happen only if the first argument
%    of |\@xnthm| (i.e.\ \texttt{Lem} in our example) is chosen so
%    as not to conflict with any previously defined commands or
%    environments.  This test is performed by the \LaTeX{} macro
%    |\@ifdefinable|.
%    \begin{macrocode}
\gdef\@xnthm#1#2[#3]{\expandafter\@ifdefinable\csname #1\endcsname
%    \end{macrocode}
%    Therefore, the first argument of |\@ifdefinable| is the
%    expansion (in the example, |\Lem|) of
%    |\csname||#1\endcsname|.  The second argument is
%    executed only if the test has been completed successfully.
%    \begin{macrocode}
   {%
%    \end{macrocode}
%    Now we define the new counter. The names of the \LaTeX{}
%    macros employed should speak for themselves:
% \changes{v2.2b}{1995/10/07}{Use \cs{@newctr}}
% \changes{v2.2c}{1995/11/23}{Use \cs{@newctr} properly.}
%    \begin{macrocode}
    \@definecounter{#1}\@newctr{#1}[#3]%
%    \end{macrocode}
%    Using |\@newctr| will give a proper error message if the counter
%    in |#3| is not defined.
%    In defining `|\theLem|' we must generate the desired macro
%    name by use of |\expandafter| and |\csname|.
%    \begin{macrocode}
    \expandafter\xdef\csname the#1\endcsname
%    \end{macrocode}
%    An |\xdef| is used in order to make the definition global,
%    and to ensure that it contains the replacement texts of
%    |\@thmcountersep| and |\@thmcounter|.\footnote{These
%    two macros can be defined by the document style.  Their default
%    values produce a `\texttt{.}' as separation and an arabic
%    representation of the number.} However, not everything should be
%    expanded. For example, it saves space to use |\thesection|
%    instead of its---at times---lengthy expansion.
%    \begin{macrocode}
      {\expandafter \noexpand \csname the#3\endcsname
       \@thmcountersep \@thmcounter{#1}}%
%    \end{macrocode}
%    Thus with the defaults of \LaTeX{}, |\theLem| would be
%    replaced by the command sequence |\thesection.\arabic{Lem}|.
%
%    We will now look at the definition of the macro which is executed
%    at the beginning of the actual environment (in our example this
%    macro is |\Lem|). It should be noted that we use an
%    ``|\expandafter| trick'' to expand only certain parts of the
%    replacement text at the time of the definition.
%    \begin{macrocode}
    \def\@tempa{\global\@namedef{#1}}%
    \expandafter \@tempa \expandafter{%
%    \end{macrocode}
%    First, the macro that contains the current definitions of
%    |\@begintheorem| and |\@opargtheorem| should be called
%    up.  The name of this macro---as is already known---has the form
%    |\th@|\meta{theorem\ style}; therefore, it must be called by
%    \begin{macrocode}
      \csname th@\the \theorem@style
%    \end{macrocode}
%    In addition the default theorem font should be changeable, i.e.\
%    we have to insert the contents of |\theorem@bodyfont|. For
%    that reason, we expand even further, beyond |\endcsname|,
%    and thus insert the contents of the \textsf{token} register
%    |\theorem@bodyfont| in the replacement text.
%    \begin{macrocode}
            \expandafter \endcsname \the \theorem@bodyfont
%    \end{macrocode}
%    Now it is time to call the macro |\@thm| which takes over
%    the further processing.  It has two arguments: the current
%    counter name (in our example, \texttt{Lem}), and the text of the
%    label (in our example, \texttt{Lemma}).
%    \begin{macrocode}
     \@thm{#1}{#2}}%
%    \end{macrocode}
%    With this, the `sub-definition' is complete.  The macro
%    |\@endtheorem| ends a theorem environment and is, so far, nothing
%    but an |\endtrivlist|.  (Hence it is defined globally, and not
%    within the theorem styles.\footnote{This has to be changed as
%    soon as theorem styles that change \texttt{\bslash @endtheorem}
%    exist.  In such a case, all existing styles must be changed as
%    well since they will have to reset the macro.}) Therefore, we can
%    set it equivalent to the macro that ends the theorem set (in our
%    example, |\endLem|).  However, if some day theorem styles exist
%    that do change |\@endtheorem|, we would have to use the
%    commented-out line instead.
%    \begin{macrocode}
    \global \expandafter \let \csname end#1\endcsname \@endtheorem
%   \global\@namedef{end#1}{\@endtheorem}%
%    \end{macrocode}
%    With these commands all the required definitions are employed,
%    unless the test |\@ifdefinable| has failed.  Therefore, we
%    end the second argument of this macro and with it the definition
%    of |\@xnthm|.
%    \begin{macrocode}
   }}
%    \end{macrocode}
% \end{macro}
%
% \begin{macro}{\@ynthm}
%    The definition of |\@ynthm| is completely analogous. In
%    this case the new counter that is defined is not reset within
%    another counter; thus the definition of |\the...|
%    is simplified:
%    \begin{macrocode}
\gdef\@ynthm#1#2{\expandafter\@ifdefinable\csname #1\endcsname
   {\@definecounter{#1}%
    \expandafter\xdef\csname the#1\endcsname{\@thmcounter{#1}}%
%    \end{macrocode}
%    The rest of the definition corresponds literally to that of
%    |\@xnthm|:
%    \begin{macrocode}
    \def\@tempa{\global\@namedef{#1}}\expandafter \@tempa
     \expandafter{\csname th@\the \theorem@style \expandafter
     \endcsname \the\theorem@bodyfont \@thm{#1}{#2}}%
    \global \expandafter \let \csname end#1\endcsname \@endtheorem}}
%    \end{macrocode}
% \end{macro}
%
% \begin{macro}{\@othm}
%    The definition of |\@othm| does not contain anything new.
%    \begin{macrocode}
\gdef\@othm#1[#2]#3{%
%    \end{macrocode}
% \changes{v2.2b}{95/10/07}{Build in check for counter defined}
%    We do not define a new counter but instead use one that has
%    already been defined.  Thus the only definition we need is that
%    of this pseudo-counter (i.e.\ |\the|\meta{env.\ name}).
%    First we check if |#2| corresponds to a known counter name.
%    \begin{macrocode}
  \expandafter\ifx\csname c@#2\endcsname\relax
   \@nocounterr{#2}%
  \else
   \expandafter\@ifdefinable\csname #1\endcsname
   {\expandafter \xdef \csname the#1\endcsname
     {\expandafter \noexpand \csname the#2\endcsname}%
%    \end{macrocode}
%    All other parts of the definition can be adopted from
%    |\@xnthm|. We have to remember, though, that in this case
%    the name of the current counter and the theorem label have moved
%    to the second and third arguments.
%    \begin{macrocode}
    \def\@tempa{\global\@namedef{#1}}\expandafter \@tempa
     \expandafter{\csname th@\the \theorem@style \expandafter
     \endcsname \the\theorem@bodyfont \@thm{#2}{#3}}%
    \global \expandafter \let \csname end#1\endcsname \@endtheorem}%
  \fi}
%    \end{macrocode}
% \end{macro}
%
% \subsection{Macros that are employed in a {\envfont theorem}
%    environment}
%
% \begin{macro}{\@thm}
%    The macro |\@thm| has to increase the current counter.  Then,
%    depending on whether the environment has (or does not have) an
%    optional argument, it has to branch into either |\@begintheorem|
%    or |\@opargtheorem|.
%    \begin{macrocode}
\gdef\@thm#1#2{\refstepcounter{#1}%
%    \end{macrocode}
%    Now we start a {\envfont trivlist} environment, and give
%    |\@topsep| and |\@topsepadd| the values of the \textsf{skip}
%    registers |\theorempreskipamount| and |\theorempostskipamount|.
%    The value in |\@topsep| is the vertical space that is inserted by
%    the first (and only) |\item| in our |\trivlist| whilst
%    |\@topsepadd| is inserted by |\@endparenv| at the end of that
%    {\envfont trivlist} environment.  By using these registers, we
%    obtain the desired space around a {\envfont theorem} environment.
%    \begin{macrocode}
   \trivlist
   \@topsep \theorempreskipamount               % used by first \item
   \@topsepadd \theorempostskipamount           % used by \@endparenv
%    \end{macrocode}
%    Now we have to test whether an optional argument has been given.
% \changes{v2.0d}{89/07/18}{Forgotten percent added.}
%    \begin{macrocode}
   \@ifnextchar [%
%    \end{macrocode}
%    If there is an optional argument, we will call |\@ythm|, and
%    move the arguments read back into the input stream.
%    \begin{macrocode}
   {\@ythm{#1}{#2}}%
%    \end{macrocode}
%    If not, we call |\@begintheorem|. Its first argument is the
%    name of the theorem set (hence the second argument of
%    |\@thm|). Its second argument is the macro that produces the
%    current number.
%    \begin{macrocode}
   {\@begintheorem{#2}{\csname the#1\endcsname}\ignorespaces}}
%    \end{macrocode}
% \end{macro}
%
% \begin{macro}{\@xthm}
% \begin{macro}{\@ythm}
%    Both these macros were originally called by |\@thm|.  We do
%    not need |\@xthm| anymore, hence we reset it to
%    |\relax|. The definition of |\@ythm| has not changed at
%    all from its definition in \LaTeX. In order to make the macros
%    easier to understand, we will
%    nevertheless present it (commented out).
%    \begin{macrocode}
\global\let\@xthm\relax
% \def\@ythm#1#2[#3]{\@opargbegintheorem{#2}{\csname
%           the#1\endcsname}{#3}\ignorespaces}
%</package>
%    \end{macrocode}
%    The primitive |\ignorespaces| in |\@ythm| and
%    |\@thm| is needed to remove the spaces between the
%    |\begin{...}| and the actual text.
% \end{macro}
% \end{macro}
%
%
% \subsection{Definition of the theorem substyles}
%
% As already pointed out, the theorem substyles, defined below, are
% only loaded when necessary.  Note that all these substyles, except
% \texttt{plain}, have |\slshape| as the default body font.
%
% \subsubsection{The \texttt{plain} style}
%
% \label{dok:th@plain} As the following macros use |@|, we have
% to locally set the |\catcode| of this symbol to ``letter''.
% This happens within a group, so that we do not have to worry about
% which |\catcode| that symbol had before.
%    \begin{macrocode}
%<*thp>
\begingroup \makeatletter
%    \end{macrocode}
% Since we are now within a group, we must make all definitions
% globally.  First we make sure that \texttt{theorem.sty} is loaded.
% This will allow us to use this file as a document style option
% without having to call \texttt{theorem} itself as an option.  At the
% same time, we assure that at least version 2 is loaded, since
% |\theorem@style| was not defined in earlier versions.
%    \begin{macrocode}
\@ifundefined{theorem@style}{\input{theorem.sty}}{}
\ProvidesFile{thp.sty}
             [\FMithmInfo]
%    \end{macrocode}
%
%
% \begin{macro}{\th@plain}
% \begin{macro}{\@begintheorem}
% \begin{macro}{\@opargbegintheorem}
%    |\theoremstyle{plain}| corresponds to the original
%    definition, except that the distances to the surrounding text are
%    determined by the parameters |\theorempreskipamount| and
%    |\theorempostskipamount|.
%    First we set the default body font.
%    \begin{macrocode}
\gdef\th@plain{\normalfont\itshape
%    \end{macrocode}
%    Then we define |\@begintheorem| and
%    |\@opargbegintheorem|. These two macros define how the
%    header of a theorem is typeset.  |\@opargbegintheorem| will
%    be called if a \textsf{theorem} environment with an optional
%    argument is encountered; otherwise, the header is constructed by
%    calling |\@begintheorem|.  If one of these macros is
%    executed, we are within a \textsf{trivlist} environment started by
%    |\@thm|.  So the theorem header is produced with an
%    |\item| command.
%
%    Instead of specifying the header font directly, all standard
%    theorem styles use the |\theorem@headerfont| macro to allow
%    customization.  The extra space (|\labelsep|) is necessary
%    because of problems in the \textsf{trivlist} environment.
%    \begin{macrocode}
  \def\@begintheorem##1##2{%
        \item[\hskip\labelsep \theorem@headerfont ##1\ ##2]}%
%    \end{macrocode}
%    The definition of |\@opargbegintheorem| is completely
%    analogous. The only difference is the fact that there exists a
%    third argument (which is the optional parameter of the
%    environment and contains additional information about the
%    theorem).  Customarily we enclose it in parentheses.
%    \begin{macrocode}
\def\@opargbegintheorem##1##2##3{%
   \item[\hskip\labelsep \theorem@headerfont ##1\ ##2\ (##3)]}}
%    \end{macrocode}
% \end{macro}
% \end{macro}
% \end{macro}
%
% \noindent
% We conclude with an |\endgroup| to restore the
% |\catcode| of |@|.
%    \begin{macrocode}
\endgroup
%</thp>
%    \end{macrocode}
%
%
% \subsubsection{The \texttt{break} style}
%
% This style option is stored in the file |thb.sty|.
% For the next two lines see the documentation for |\th@plain| on
% page~\pageref{dok:th@plain}.
%    \begin{macrocode}
%<*thb>
\begingroup \makeatletter
\@ifundefined{theorem@style}{\input{theorem.sty}}{}
\ProvidesFile{thb.sty}
             [\FMithmInfo]
%    \end{macrocode}
%
% \begin{macro}{\th@break}
%    |\theoremstyle{break}| produces a line break after the name
%    of the theorem.  The font is |\slshape|.  Hence, we define
%    |\th@break| as follows:
%    \begin{macrocode}
\gdef\th@break{\normalfont\slshape
  \def\@begintheorem##1##2{\item[%
%    \end{macrocode}
%    We run into the following problem: it is not possible to create
%    the header with |\item[|\meta{title}|]| and then start
%    a new line by, for example, |\mbox{}\\|. Such a definition
%    will fail whenever a list environment follows immediately.  With
%    the above construction, the |\mbox{}| causes the switch
%    |@inlabel| (cf.\ definition of |\list| and
%    |\trivlist| in \cite{latex:source}) to be set to \texttt{false}
%    and so the following list will insert additional vertical
%    space (|\topskip|). This is quite annoying. Therefore, we
%    create the line break within the |\item|. In order to ensure
%    that the text will begin at the proper position in the following
%    line, we simply pretend that the label does not take any
%    room.\footnote{This will lead to problems whenever very high
%    symbols occurring in the line tower into the
%    heading. So, something else has to be done here sometime.}
%    \begin{macrocode}
       \rlap{\vbox{\hbox{\hskip \labelsep\theorem@headerfont ##1\ ##2}%
                   \hbox{\strut}}}]}%
%    \end{macrocode}
%    Again, the definition of |\@opargbegintheorem| is
%    completely analogous.
%    \begin{macrocode}
\def\@opargbegintheorem##1##2##3{%
  \item[\rlap{\vbox{\hbox{\hskip \labelsep \theorem@headerfont
                     ##1\ ##2\ (##3)}%
                    \hbox{\strut}}}]}}
\endgroup
%</thb>
%    \end{macrocode}
% \end{macro}
%
%
% \subsubsection{The \texttt{changebreak} style}
%
%    \begin{macrocode}
% This style option is stored in the file |thcb.sty|.
%    \begin{macrocode}
%<*thcb>
\begingroup \makeatletter
\@ifundefined{theorem@style}{\input{theorem.sty}}{}
\ProvidesFile{thcb.sty}
             [\FMithmInfo]
%    \end{macrocode}
%
% \begin{macro}{\th@changebreak}
%    The \texttt{change-break} theorem style is like \texttt{break}
%    but with interchange of theorem name and theorem number.  Thus we
%    define |\th@changebreak| as follows:
%    \begin{macrocode}
\gdef\th@changebreak{\normalfont\slshape
  \def\@begintheorem##1##2{\item
       [\rlap{\vbox{\hbox{\hskip\labelsep \theorem@headerfont ##2\ ##1}%
                    \hbox{\strut}}}]}%
\def\@opargbegintheorem##1##2##3{%
  \item[\rlap{\vbox{\hbox{\hskip\labelsep \theorem@headerfont
                     ##2\ ##1\ (##3)}%
                    \hbox{\strut}}}]}}
\endgroup
%</thcb>
%    \end{macrocode}
% \end{macro}
%
%
% \subsubsection{The \texttt{change} style}
%
% This style option is stored in the file |thc.sty|.
%    \begin{macrocode}
%<*thc>
\begingroup \makeatletter
\@ifundefined{theorem@style}{\input{theorem.sty}}{}
\ProvidesFile{thc.sty}
             [\FMithmInfo]
%    \end{macrocode}
%
% \begin{macro}{\th@change}
%    The \texttt{change} theorem style corresponds to the
%    \texttt{change break} style without a linebreak after the header.
%    To say it in another way, it's the same as the \texttt{plain}
%    style but with number and name interchanged and |\slshape| as the
%    default font.
%    \begin{macrocode}
\gdef\th@change{\normalfont\slshape
  \def\@begintheorem##1##2{\item
       [\hskip\labelsep \theorem@headerfont ##2\ ##1]}%
\def\@opargbegintheorem##1##2##3{%
  \item[\hskip\labelsep \theorem@headerfont ##2\ ##1\ (##3)]}}
\endgroup
%</thc>
%    \end{macrocode}
% \end{macro}
%
%
% \subsubsection{The \texttt{marginbreak} style}
%
% This style option is the one used most often at Mainz.  It is saved
% in the file |thmb.sty|.
%    \begin{macrocode}
%<*thmb>
\begingroup \makeatletter
\@ifundefined{theorem@style}{\input{theorem.sty}}{}
\ProvidesFile{thmb.sty}
             [\FMithmInfo]
%    \end{macrocode}
%
% \begin{macro}{\th@marginbreak}
%    The \texttt{margin break} style is nearly the same as the
%    \texttt{change break} style. The only difference is the placement
%    of the theorem number.  We use |\llap| to place it in the left
%    margin.
%
%    In this style |\labelsep| denotes the separation between the
%    number and the text.
%    \begin{macrocode}
\gdef\th@marginbreak{\normalfont\slshape
  \def\@begintheorem##1##2{\item
       [\rlap{\vbox{\theorem@headerfont
                    \hbox{\llap{##2}\hskip\labelsep ##1}%
                    \hbox{\strut}}}]}%
\def\@opargbegintheorem##1##2##3{%
  \item[\rlap{\vbox{\theorem@headerfont
                    \hbox{\llap{##2}\hskip\labelsep ##1\ (##3)}%
                    \hbox{\strut}}}]}}
\endgroup
%</thmb>
%    \end{macrocode}
% \changes{v2.1c}{90/02/17}{`endgroup outside macrocode env.}
% \end{macro}
%
%
%
% \subsubsection{The \texttt{margin} style}
%
% This style option is stored in the file |thm.sty|.
%    \begin{macrocode}
%<*thm>
\begingroup \makeatletter
\@ifundefined{theorem@style}{\input{theorem.sty}}{}
\ProvidesFile{thm.sty}
             [\FMithmInfo]
%    \end{macrocode}
%
% \begin{macro}{\th@margin}
%    Again this is only a variant of the theorem styles described
%    above without any new ideas.
%    \begin{macrocode}
\gdef\th@margin{\normalfont\slshape
  \def\@begintheorem##1##2{\item
       [\theorem@headerfont \llap{##2}\hskip\labelsep ##1]}%
\def\@opargbegintheorem##1##2##3{%
  \item[\theorem@headerfont \llap{##2}\hskip\labelsep ##1\ (##3)]}}
\endgroup
%</thm>
%    \end{macrocode}
% \end{macro}
%
%
%
%
% \subsection{Final Definitions}
%
% \begin{macro}{\theorempreskipamount}
% \begin{macro}{\theorempostskipamount}
%    The \textsf{skip} parameters that regulate the vertical empty space
%    before and after the {\envfont theorem} environment have
%    to be allocated as well.
% \changes{v2.0f}{89/07/20}{Unnecessary `global' removed.}
%    \begin{macrocode}
%<*package>
\newskip\theorempreskipamount
\newskip\theorempostskipamount
%    \end{macrocode}
%    Since we have used the same values for all theorem sets, we now can
%    assign them.
%    \begin{macrocode}
\global\setlength\theorempreskipamount{12pt plus 5pt minus 3pt}
\global\setlength\theorempostskipamount{8pt plus 3pt minus 1.5pt}
%    \end{macrocode}
% \end{macro}
% \end{macro}
%
%
% \begin{macro}{\@endtheorem}
%    The same holds for the macro |\@endtheorem|, which ends a
%    {\envfont theorem} environment. Since it is the same for all
%    theorem sets, it is removed from the macros |\th@|\meta{style}.
%    It simply ends the {\envfont trivlist} environment, which was
%    begun in |\@thm|.
%    \begin{macrocode}
\global\let\@endtheorem=\endtrivlist
%    \end{macrocode}
% \end{macro}
%
%
%
% \begin{macro}{\@preamblecmds}
%    All macros defined above are to be used only in the preamble.
%    Therefore, we insert them in |\@preamblecmds| which will disable
%    them at begin document. This is done by the internal \LaTeXe{}
%    command |\@onlypreamble|.
% \changes{v2.0g}{89/09/19}{`theoremheaderfont and `theorembodyfont
%    added.}
% \changes{v2.2a}{94/02/03}{Use \LaTeXe{} interface}
%    \begin{macrocode}
\@onlypreamble\@xnthm
\@onlypreamble\@ynthm
\@onlypreamble\@othm
\@onlypreamble\newtheorem
\@onlypreamble\theoremstyle
\@onlypreamble\theorembodyfont
\@onlypreamble\theoremheaderfont
%    \end{macrocode}
% \end{macro}
%
% Finally we declare the \texttt{plain} theorem style to be the default.
% \changes{v2.1b}{90/06/04}{Fixed missing `item by declaring default
%    style.}
%    \begin{macrocode}
\theoremstyle{plain}
%</package>
%    \end{macrocode}
%
% \Finale

\endinput
}{}
\ProvidesFile{thp.sty}
             [\FMithmInfo]
%    \end{macrocode}
%
%
% \begin{macro}{\th@plain}
% \begin{macro}{\@begintheorem}
% \begin{macro}{\@opargbegintheorem}
%    |\theoremstyle{plain}| corresponds to the original
%    definition, except that the distances to the surrounding text are
%    determined by the parameters |\theorempreskipamount| and
%    |\theorempostskipamount|.
%    First we set the default body font.
%    \begin{macrocode}
\gdef\th@plain{\normalfont\itshape
%    \end{macrocode}
%    Then we define |\@begintheorem| and
%    |\@opargbegintheorem|. These two macros define how the
%    header of a theorem is typeset.  |\@opargbegintheorem| will
%    be called if a \textsf{theorem} environment with an optional
%    argument is encountered; otherwise, the header is constructed by
%    calling |\@begintheorem|.  If one of these macros is
%    executed, we are within a \textsf{trivlist} environment started by
%    |\@thm|.  So the theorem header is produced with an
%    |\item| command.
%
%    Instead of specifying the header font directly, all standard
%    theorem styles use the |\theorem@headerfont| macro to allow
%    customization.  The extra space (|\labelsep|) is necessary
%    because of problems in the \textsf{trivlist} environment.
%    \begin{macrocode}
  \def\@begintheorem##1##2{%
        \item[\hskip\labelsep \theorem@headerfont ##1\ ##2]}%
%    \end{macrocode}
%    The definition of |\@opargbegintheorem| is completely
%    analogous. The only difference is the fact that there exists a
%    third argument (which is the optional parameter of the
%    environment and contains additional information about the
%    theorem).  Customarily we enclose it in parentheses.
%    \begin{macrocode}
\def\@opargbegintheorem##1##2##3{%
   \item[\hskip\labelsep \theorem@headerfont ##1\ ##2\ (##3)]}}
%    \end{macrocode}
% \end{macro}
% \end{macro}
% \end{macro}
%
% \noindent
% We conclude with an |\endgroup| to restore the
% |\catcode| of |@|.
%    \begin{macrocode}
\endgroup
%</thp>
%    \end{macrocode}
%
%
% \subsubsection{The \texttt{break} style}
%
% This style option is stored in the file |thb.sty|.
% For the next two lines see the documentation for |\th@plain| on
% page~\pageref{dok:th@plain}.
%    \begin{macrocode}
%<*thb>
\begingroup \makeatletter
\@ifundefined{theorem@style}{% \iffalse meta-comment
%
% Copyright (C) 1993-2024
%
% The LaTeX Project and any individual authors listed elsewhere
% in this file.
%
% This file is part of the Standard LaTeX `Tools Bundle'.
% -------------------------------------------------------
%
% It may be distributed and/or modified under the
% conditions of the LaTeX Project Public License, either version 1.3c
% of this license or (at your option) any later version.
% The latest version of this license is in
%    https://www.latex-project.org/lppl.txt
% and version 1.3c or later is part of all distributions of LaTeX
% version 2005/12/01 or later.
%
% The list of all files belonging to the LaTeX `Tools Bundle' is
% given in the file `manifest.txt'.
%
% \fi
%
%%
%
%
\def\FMithmInfo{2023/07/05 v2.2c Theorem extension package (FMi)}
%
% \ProvidesFile{theorem.dtx}[\FMithmInfo]
%\iffalse   % this is a METACOMMENT !
%%
%% Package `theorem' to use with LaTeX2e
%% Copyright (C) 1989-1995 Frank Mittelbach, all rights reserved.
%<+package>\@ifundefined{theorem@style}{}{\endinput}
%<+package>\NeedsTeXFormat{LaTeX2e}[1995/06/01]
%<+package>\ProvidesPackage{theorem}
%<+package>         [\FMithmInfo]
%
% \fi
%
% \changes{v1.0}{87/05/07}{First draft.}
% \changes{v1.0a}{87/05/07}{`thistheoremstyle' added.}
% \changes{v1.1a}{87/07/29}{New concept: redefine the internal
%                          \protect\LaTeX{} commands `@xnthm' etc.}
% \changes{v1.1b}{87/07/31}{`theoremfont' used to specify font.}
% \changes{v1.2a}{88/01/08}{`@endtheorem' now globally defined.}
% \changes{v1.2b}{88/01/11}{`newtheorem' only allowed in the preamble.}
% \changes{v1.3a}{88/02/14}{`theoremfont' now a toks register.}
%
% \changes{v2.0a}{88/11/16}{New implementation and new documentation.}
% \changes{v2.0a}{88/11/16}{No layout changes but internal modifications
%                           old substyles won't work.}
% \changes{v2.0b}{89/01/30}{`global' used for all definitions.}
% \changes{v2.0c}{89/06/15}{Documentation translated to English.
%                            Thanks to LLL and Nina}
% \changes{v2.1a}{90/02/15}{Support for new font selection.}
% \changes{v2.2a}{94/02/03}{Cleanup for LaTeX2e}
%
% \DoNotIndex{\@endparenv,\@ifdefinable,\@ifnextchar,\@ifundefined}
% \DoNotIndex{\@namedef,\@nameuse,\@spaces,\@tempa,\@warning}
% \DoNotIndex{\ }
% \DoNotIndex{\begingroup,\bfseries}
% \DoNotIndex{\csname}
% \DoNotIndex{\def,\do,\docdate}
% \DoNotIndex{\endcsname,\endgroup,\endinput,\endtrivlist,\expandafter}
% \DoNotIndex{\filedate,\fileversion}
% \DoNotIndex{\gdef,\global}
% \DoNotIndex{\hbox,\hskip}
% \DoNotIndex{\ignorespaces,\input,\itshape,\item}
% \DoNotIndex{\labelsep,\let,\llap}
% \DoNotIndex{\makeatletter}
% \DoNotIndex{\noexpand}
% \DoNotIndex{\refstepcounter,\relax,\rlap}
% \DoNotIndex{\setlength,\slshape,\space,\string,\strut}
% \DoNotIndex{\the,\trivlist,\typeout}
% \DoNotIndex{\vbox}
% \DoNotIndex{\xdef}
%
% \def\deflabel#1{\ttfamily #1\hfill}
% \def\deflist#1{\begin{list}{}{\settowidth\labelwidth{\ttfamily #1}%
%                               \setlength\leftmargin\labelwidth
%                               \addtolength\leftmargin\labelsep
%                               \let\makelabel\deflabel}}
% \def\enddeflist{\end{list}}
% \def\envfont{\normalfont\ttfamily}
%
% \MakeShortVerb{\|}
%
% \newcommand{\TUB}{{\sl TUGboat\/}}
%
% \GetFileInfo{theorem.dtx}
% \title{An Extension of the
%        \LaTeX{} theorem environment\thanks{This file
%        has version number \fileversion, last
%        revised \filedate.}}
% \author{Frank Mittelbach}
%
% \MaintainedByLaTeXTeam{tools}
% \maketitle
%
% \begin{abstract}
%    The macros described in this paper yield an extension of the
%    \LaTeX{} theorem mechanism. It is designed to satisfy the
%    different requirements of various journals. Thus, the
%    layout of the ``theorems'' can be manipulated by determining a
%    ``style''. This article describes not only the use, but
%    also the definition, of the necessary macros.
% \end{abstract}
%
% \section*{Preface to version 2.2}
%
% For \LaTeXe{} this package did not need any fundamental changes. I
% only modified the messages generated so that theorem layout styles
% will show up with the |\listfiles| command and cleaned the section
% on the New Font Selection Scheme since this is now included in
% \LaTeX.
%
% \section*{Preface to version 2.1}
%
% This version is identical to 2.0g described in \TUB\ 10\#3 except
% for some internal defaults which are now set depending on the used
% font selection scheme.
%
% This was done to avoid unpleasant surprises if the new font selection
% scheme is in force. For further details see
% section~\ref{sec:newfonts} and \cite{bk:GMS94}.
%
% \section{Introduction}
%
% For our purposes here, ``theorems'' are labelled enunciations,
% often set off from the main text by extra space and a font change.
% Theorems, corollaries, conjectures, definitions, and remarks
% are all instances of ``theorems''.  The ``header'' of these structures
% is composed of a label (such as \textsc{Theorem} or \textsc{Remark})
% and a number which serializes an item in the sequence of items
% with the same label.
%
% Shortly after the introduction of \LaTeX{} at the Fachbereich
% Mathematik in Mainz, the desire to manipulate the layout of
% ``theorems'' arose.  In Mainz, the following two conventions came
% into general use:
% \begin{enumerate}
%   \item The number of the theorem is shown in the margin.
%   \item There is a line break at
%         the end of the theorem header.
% \end{enumerate}
%
% Additionally, some journals require different formats which depend on
% the ``sort of theorem'': e.g.\ often remarks and definitions are set
% in |\upshape|, while |\itshape| is employed for main theorems.
%
% Confronted with these requirements, a theorem environment was
% developed in Mainz which allows separate determination of the
% layout of the ``theorems sets'', comparable to |\pagestyle|.
%
%
% \section{The user interface}
% \subsection{Defining new theorem sets}
%
% \DescribeMacro\newtheorem
% As in the original \LaTeX{} version, the command |\newtheorem|
% defines a new ``theorem set'' or ``theorem-like structure''.
% Two required arguments name the new environment and give the text
% to be typeset with each instance of the new ``set'', while
% an optional argument determines how
% the ``set'' is enumerated:
% \begin{description}
%    \item[\ttfamily \bslash newtheorem\{foo\}\{bar\}]
%       The theorem set {\envfont foo} (whose name is \texttt{bar})
%       uses its own counter.
%    \item[\ttfamily \bslash newtheorem\{foo2\}{[foo]}\{bar2\}]
%       The theorem set {\envfont foo2} (printed name \texttt{bar2})
%       uses the same counter as the theorem set \texttt{foo}.
%    \item[\ttfamily \bslash newtheorem\{foo3\}\{bar3\}{[section]}]
%       The theorem set {\envfont foo3} (printed name \texttt{bar3}) is
%       enumerated within the counter \texttt{section}, i.e.\ with every
%       new |\section| the enumeration begins again with 1, and
%       the enumeration is composed from the section-number and the
%       theorem counter itself.
% \end{description}
%
% \DescribeMacro\theoremstyle
% Additionally, the command |\theoremstyle| can define the layout
% of various, or all, theorem sets.  It should be noted that any
% theorem set defined by |\newtheorem| is typeset in the
% |\theoremstyle| that is current at the time of the definition.
% Thus, the following
% \begin{verbatim}
%    \theoremstyle{break}        \newtheorem{Cor}{Corollary}
%    \theoremstyle{plain}        \newtheorem{Exa}{Example}[section]
%\end{verbatim}                             ^^A No blank at line start
% leads to the result that the set {\envfont Cor} is formatted in the
% style \texttt{break}, while the set {\envfont Exa} and all the
% following ones are formatted in the style \texttt{plain}, unless
% another |\theoremstyle| follows.  Since the definitions installed by
% |\newtheorem| are global, one also can limit |\theoremstyle| locally
% by grouping braces.
%
% \DescribeMacro\theorembodyfont
% The choice of the font for the theorem body is completely
% independent of the chosen |\theoremstyle|; this has proven to
% be very advantageous.  For example,
% \begin{verbatim}
%   {\theorembodyfont{\upshape}          \newtheorem{Rem}{Remark}}
%\end{verbatim}
% defines a theorem set {\envfont Rem}, which will be set in
% |\upshape| in the current layout (which in our example is
% \texttt{plain}).  As with |\theoremstyle|, the |\theorembodyfont|
% chosen is that current at the time of |\newtheorem|.  If
% |\theorembodyfont| is not specified or one defines
% |\theorembodyfont{}|, then the font used will be that defined by the
% |\theoremstyle|.
%
% \DescribeMacro\theoremheaderfont It is also possible to customize
% the font used for the theorem headers.  This is, however, a global
% declaration, and therefore there should be at most one
% |\theoremheaderfont| declaration in the preamble.\footnote{If it is
% actually necessary to have different header fonts, one has to define
% new theorem styles (substituting the desired font) or specify the
% information directly in the \texttt{\bslash newtheorem} declaration
% (the unclean variant).}
%
% \DescribeMacro\theorempreskipamount
% \DescribeMacro\theorempostskipamount Two additional parameters
% affect the vertical space around the theorem environments:
% |\theorempreskipamount| and |\theorempostskipamount| define,
% respectively, the spacing before and after such an environment.
% These parameters apply for all theorem sets and can be manipulated
% with the ordinary length macros.  They are rubber lengths
% (`\textsf{skips}'), and therefore can contain \texttt{plus} and
% \texttt{minus} parts.
%
% Since the definition of theorem sets should---most sensibly---be
% placed in the preamble, we only allow installation there.
% It is therefore possible to release the memory used here after
% |\begin{document}|, in order to make room for other
% applications.
%
%
% \subsection{Existing theorem styles}
%
% The following theorem styles exist to date:
% \begin{deflist}{marginbreak}
%    \item[plain]
%       This theorem style emulates the original \LaTeX{} definition,
%       except that additionally the parameters
%       |\theorem...skipamount| are used.
%    \item[break]
%       In this style, the theorem header is followed by a line
%       break.
%    \item[marginbreak]
%       The theorem number is set in the margin, and there is a line
%       break as in \texttt{break}.
%    \item[changebreak]
%       Like \texttt{break}, but with header number and text
%       interchanged.
%    \item[change]
%       Header number and text are interchanged, without a line break.
%    \item[margin]
%       The number is set in the left margin, without a line break.
% \end{deflist}
% All styles (except \texttt{plain}) select |\slshape| as the default
% |\theorembodyfont|.
%
%
% \subsection{Examples}
%
% Given the above theorem sets {\envfont Cor}, {\envfont Exa}
% and {\envfont Rem},
% suppose that the preamble also contains the declarations:
% \begin{verbatim}
%    \theoremstyle{marginbreak}   \newtheorem{Lem}[Cor]{Lemma}
%    \theoremstyle{change}
%    \theorembodyfont{\itshape}        \newtheorem{Def}[Cor]{Definition}
%
%    \theoremheaderfont{\scshape}
%\end{verbatim}
% Then the following are some typical examples of the typeset output
% resulting from their use.
%
% \begin{Cor}
%    This is a sentence typeset in the theorem environment {\envfont
%    Cor}.
% \end{Cor}
%
% \begin{Exa}
%    This is a sentence typeset in the theorem environment {\envfont
%    Exa}.
% \end{Exa}
%
% \begin{Rem}
%    This is a sentence typeset in the theorem environment {\envfont
%    Rem}.
% \end{Rem}
%
% \begin{Lem}[Ben User]
%    This is a sentence typeset in the theorem environment {\envfont
%    Lem}.
% \end{Lem}
%
% \begin{Def}[Very impressive Definition]
%    This is a sentence typeset in the theorem environment {\envfont
%    Def}.
% \end{Def}
%
% The last two examples show the effect of the optional argument to
% a theorem environment (it is the text typeset in parentheses).
%
%
% \section{Special Considerations}
% \label{sec:newfonts}
%
% Theoremheader and body are implemented as a unit. This means that
% the |\theoremheaderfont|  will inherit characteristics of the
% |\theorembodyfont| in \LaTeXe. Thus, if for example
% |\theorembodyfont| is |\itshape| and |\theoremheaderfont|
% is |\bfseries| the font selected for the header will have the
% characteristics `bold extended italic'. If this is not desired one
% should set the |\theoremheaderfont| to something like
% \begin{verbatim}
%      \theoremheaderfont{\normalfont\bfseries}
%\end{verbatim}
% i.e.\ supplying all necessary font information explicitly.
%
%
%
% \section{Acknowledgements}
%
% The publication of this set of macros was only possible with the
% help of Christina Busse (translating the manuscript into English),
% Joachim Pense (playing the r\^ole of typist), Chris Rowley
% (looking everything over) and many others providing useful
% suggestions.
%
% \MaybeStop{%
%    \begin{thebibliography}{9}
%       \bibitem{bk:GMS94} \textsc{M.~Goossens}, \textsc{F.~Mittelbach}
%         and \textsc{A.~Samarin}.
%         \newblock The \LaTeX{} Companion.
%         \newblock
%         Addison-Wesley, Reading, Massachusetts, 1994.
%       \bibitem{latex:source} \textsc{Lamport, Leslie.} \newblock
%                               \texttt{latex.tex}, version 2.09, date
%                               Feb. 1990.
%    \end{thebibliography}
%    \PrintIndex
%    \PrintChanges}
%
%
% \section{The documentation driver file}
%
% The next bit of code contains the documentation driver file for
% \TeX{}, i.e., the file that will produce the documentation you are
% currently reading. It will be extracted from this file by the
% \texttt{docstrip} program. Since it is the first code in the file
% one can alternatively process this file directly with \LaTeXe{} to
% obtain the documentation.
%
%    \begin{macrocode}
%<*driver>
\documentclass{ltxdoc}

\usepackage{theorem}

%<+driver>% The next few lines define theorem sets which are used
%<+driver>% in the example section of the documentation.

\theoremstyle{break}       \newtheorem{Cor}{Corollary}
\theoremstyle{plain}       \newtheorem{Exa}{Example}[section]
{\theorembodyfont{\upshape}\newtheorem{Rem}{Remark}}
\theoremstyle{marginbreak} \newtheorem{Lem}[Cor]{Lemma}
\theoremstyle{change}
\theorembodyfont{\itshape} \newtheorem{Def}[Cor]{Definition}

\theoremheaderfont{\scshape}

\RecordChanges

\begin{document}
   \DocInput{theorem.dtx}
\end{document}
%</driver>
%    \end{macrocode}
%
% \section{Definition of the Macros}
%
% If the file has been loaded before, we abort immediately. If not the
% package announces itself (this is actually done at the very top of
% the file---the way it is done isn't good style so don't copy it).
% \changes{v2.0e}{89/07/19}{Spaces removed from `typeout'.}
% \changes{v2.2b}{95/11/19}{Announce moved to top of file}
%    \begin{macrocode}
%<*package>
%\@ifundefined{theorem@style}{}{\endinput}
%\def\FMithmInfo{1995/11/19 v2.2b Theorem extension package (FMi)}
%\ProvidesPackage{theorem}[\FMithmInfo]
%    \end{macrocode}
%
%
% \subsection{Definition of theorem styles and fonts}
%
% All the definitions in this file are done globally to allow
% inputting this file inside a group.
%
% \begin{macro}{\theoremstyle}
%    Before a theorem style can be installed, the chosen style must be
%    known. For that reason, we must test to see that
%    |\th@|\meta{style} is known or, more precisely, that it is
%    different from |\relax|. If the style is not known then
%    |\th@plain| is used.
%    \begin{macrocode}
\gdef\theoremstyle#1{%
   \@ifundefined{th@#1}{\@warning
          {Unknown theoremstyle `#1'. Using `plain'}%
          \theorem@style{plain}}%
%    \end{macrocode}
%    We save the theorem style to be used in the token register
%    |\theorem@style|.
%    \begin{macrocode}
      {\theorem@style{#1}}%
%    \end{macrocode}
%    Now we ``evaluate'' the theorem style: this means, we call the
%    macro |\th@|\meta{style} which will activate the relevant
%    definitions which are contained in a separate file.  This is done
%    in a group to suppress changes to the current font.  This could
%    otherwise pose problems together with the new font selection
%    scheme\footnote{When I printed the original article using the new
%    font selection scheme I ended with a document with slanted
%    typefaces (text headings and all) simply because one of the
%    theorem styles used \texttt{\bslash sl} at toplevel.} if the
%    |\th@|\meta{style} is evaluated a second time.
% \changes{v2.1a}{90/02/15}{Group added around `th@..}
% \changes{v2.1a}{90/02/15}{Reseting of `..@begintheorem removed.}
%    \begin{macrocode}
      \begingroup
        \csname th@\the\theorem@style \endcsname
      \endgroup}
%    \end{macrocode}
% \end{macro}
%
%
% \begin{macro}{\@begintheorem}
% \changes{v2.1a}{90/02/15}{Relaxed at toplevel.}
% \begin{macro}{\@opargbegintheorem}
% \changes{v2.1a}{90/02/15}{Relaxed at toplevel.}
%    We reset |\@begintheorem| and |\@opargbegintheorem| to
%    |\relax| since these commands are no longer necessary at
%    toplevel. This will save a few tokens.
%    \begin{macrocode}
\global\let\@begintheorem\relax
\global\let\@opargbegintheorem\relax
%    \end{macrocode}
% \end{macro}
% \end{macro}
%
% \begin{macro}{\theorem@style}
%    Obviously the \textsf{token} register used above has to be
%    allocated. To assure the utmost compatibility with the
%    original \LaTeX{} definition, we set the default theorem style to
%    \texttt{plain}, which implements the usual \LaTeX{} convention.
% \changes{v2.0f}{89/07/20}{Unnecessary `global' removed.}
%    \begin{macrocode}
\newtoks\theorem@style
\global\theorem@style{plain}
%    \end{macrocode}
% \end{macro}
%
% \begin{macro}{\theorembodyfont}
% \begin{macro}{\theorem@bodyfont}
%    If the |\theorembodyfont| is set by the user then it should not
%    interact with the default font set in the theorem style. When the
%    new font selection is in force this may happen if, for example,
%    the default is |\itshape| and the new |\theorembodyfont| is
%    |\sffamily|. So we add a |\reset@font| command in front of the
%    user definition.
% \changes{v2.0f}{89/07/20}{Unnecessary `global' removed.}
% \changes{v2.0g}{89/09/19}{`theoremfont to `theorembodyfont.}
% \changes{v2.1c}{92/06/22}{`reset@font added and logic changed.}
% \changes{v2.1d}{92/10/28}{Reset to empty if arg is empty}
%    \begin{macrocode}
\gdef\theorembodyfont#1{%
%    \end{macrocode}
%    We check if the argument supplied is empty and if so put nothing
%    into the |\theorem@bodyfont| token register to allow for
%    |\theorembodyfont{}| as a mean of using the default of the
%    current |\theoremstyle|.
%    \begin{macrocode}
   \def\@tempa{#1}%
   \ifx\@tempa\@empty
    \theorem@bodyfont{}%
   \else
    \theorem@bodyfont{\reset@font#1}%
   \fi
   }
\newtoks\theorem@bodyfont
\global\theorem@bodyfont{}
%    \end{macrocode}
% \end{macro}
% \end{macro}
%
%
% \begin{macro}{\theoremheaderfont}
% \changes{v2.0g}{89/09/19}{Macro added.}
%    The font for the theorem headers is handled differently because
%    this definition applies to all theorem styles.
%    \begin{macrocode}
\gdef\theoremheaderfont#1{\gdef\theorem@headerfont{#1}%
%    \end{macrocode}
%    After using the macro once it is redefined to produce an error
%    message.
%    \begin{macrocode}
       \gdef\theoremheaderfont##1{%
        \typeout{\string\theoremheaderfont\space should be used
                 only once.}}}
%    \end{macrocode}
% \end{macro}
%
% \begin{macro}{\theorem@headerfont}
% \changes{v2.0g}{89/09/19}{Macro added.}
%    To set the |\theorem@headerfont| default we first test if
%    the new fontselection scheme is in force.
% \changes{v2.1a}{90/02/15}{Support for new font selection.}
%    \begin{macrocode}
\ifx\upshape\undefined
%    \end{macrocode}
%    If not we define it to expand into |\bfseries|. We don't use
%    |\let| just in case a following style option redefines this
%    macro.
%    \begin{macrocode}
\gdef\theorem@headerfont{\bfseries}
%    \end{macrocode}
%    Otherwise we reset the current shape before calling |\bfseries|.
%    \begin{macrocode}
\else \gdef\theorem@headerfont{\normalfont\bfseries}\fi
%    \end{macrocode}
% \end{macro}
%
% \begin{macro}{\th@plain}
% \begin{macro}{\th@break}
% \begin{macro}{\th@marginbreak}
% \begin{macro}{\th@changebreak}
% \begin{macro}{\th@change}
% \begin{macro}{\th@margin}
%    The different styles are defined in macros such as
%    |\th@plain|.  Since memory space is precious in
%    ``non-Big-versions'', we have to avoid offering too many unused
%    definitions.  Therefore we define these styles in separate files
%    that can be loaded on demand.  Thus the commands themselves only
%    load these files. We use |\@input@|, a \LaTeXe{} internal command
%    that ensures that the file will be listed with |\listfiles|.
% \changes{v2.2a}{94/02/02}{Use `@input@ to load theorem layout files}
%    \begin{macrocode}
\gdef\th@plain{\@input@{thp.sty}}
\gdef\th@break{\@input@{thb.sty}}
\gdef\th@marginbreak{\@input@{thmb.sty}}
\gdef\th@changebreak{\@input@{thcb.sty}}
\gdef\th@change{\@input@{thc.sty}}
\gdef\th@margin{\@input@{thm.sty}}
%    \end{macrocode}
%    This list will be expanded when new styles become
%    available.  For testing,  just append new theorem substyles
%    as document options.
% \end{macro}
% \end{macro}
% \end{macro}
% \end{macro}
% \end{macro}
% \end{macro}
%
%
%
% \subsection{Definition of a new theorem set}
%
% As already pointed out, a new theorem environment can be defined in
% three different ways:
% \begin{verbatim}
%    \newtheorem{Lem}{Lemma}
%    \newtheorem{Lem}{Lemma}[section]
%    \newtheorem{Lem}[Theorem]{Lemma}
%\end{verbatim}
% The function of the macro |\newtheorem| is to recognize these cases
% and then to branch into one of the three macros |\@ynthm|, |\@xnthm|
% or |\@othm|.  This mechanism is adopted unchanged from
% \cite{latex:source}; the essential point here is that, for example,
% in the second case, the arguments \texttt{Lem}, \texttt{Lemma} and
% \texttt{section} are passed over to the macro |\@xnthm|.
%
% We inspect this case first because the others present fewer problems,
% and thus are easily derived from this one.
%
% \begin{macro}{\@xnthm}
%    For our example arguments, the macro |\@xnthm| must fulfill
%    the following:
%    \begin{itemize}
%       \item
%          Define a new \LaTeX-counter `Lem'
%       \item
%          reset this counter within a |\section|
%       \item
%          define the macro |\theLem|
%       \item
%          define the environment macros |\Lem| and
%          |\endLem| using the current
%          |\theoremstyle| and |\theorem@bodyfont|.
%    \end{itemize}
%    Obviously, all this should happen only if the first argument
%    of |\@xnthm| (i.e.\ \texttt{Lem} in our example) is chosen so
%    as not to conflict with any previously defined commands or
%    environments.  This test is performed by the \LaTeX{} macro
%    |\@ifdefinable|.
%    \begin{macrocode}
\gdef\@xnthm#1#2[#3]{\expandafter\@ifdefinable\csname #1\endcsname
%    \end{macrocode}
%    Therefore, the first argument of |\@ifdefinable| is the
%    expansion (in the example, |\Lem|) of
%    |\csname||#1\endcsname|.  The second argument is
%    executed only if the test has been completed successfully.
%    \begin{macrocode}
   {%
%    \end{macrocode}
%    Now we define the new counter. The names of the \LaTeX{}
%    macros employed should speak for themselves:
% \changes{v2.2b}{1995/10/07}{Use \cs{@newctr}}
% \changes{v2.2c}{1995/11/23}{Use \cs{@newctr} properly.}
%    \begin{macrocode}
    \@definecounter{#1}\@newctr{#1}[#3]%
%    \end{macrocode}
%    Using |\@newctr| will give a proper error message if the counter
%    in |#3| is not defined.
%    In defining `|\theLem|' we must generate the desired macro
%    name by use of |\expandafter| and |\csname|.
%    \begin{macrocode}
    \expandafter\xdef\csname the#1\endcsname
%    \end{macrocode}
%    An |\xdef| is used in order to make the definition global,
%    and to ensure that it contains the replacement texts of
%    |\@thmcountersep| and |\@thmcounter|.\footnote{These
%    two macros can be defined by the document style.  Their default
%    values produce a `\texttt{.}' as separation and an arabic
%    representation of the number.} However, not everything should be
%    expanded. For example, it saves space to use |\thesection|
%    instead of its---at times---lengthy expansion.
%    \begin{macrocode}
      {\expandafter \noexpand \csname the#3\endcsname
       \@thmcountersep \@thmcounter{#1}}%
%    \end{macrocode}
%    Thus with the defaults of \LaTeX{}, |\theLem| would be
%    replaced by the command sequence |\thesection.\arabic{Lem}|.
%
%    We will now look at the definition of the macro which is executed
%    at the beginning of the actual environment (in our example this
%    macro is |\Lem|). It should be noted that we use an
%    ``|\expandafter| trick'' to expand only certain parts of the
%    replacement text at the time of the definition.
%    \begin{macrocode}
    \def\@tempa{\global\@namedef{#1}}%
    \expandafter \@tempa \expandafter{%
%    \end{macrocode}
%    First, the macro that contains the current definitions of
%    |\@begintheorem| and |\@opargtheorem| should be called
%    up.  The name of this macro---as is already known---has the form
%    |\th@|\meta{theorem\ style}; therefore, it must be called by
%    \begin{macrocode}
      \csname th@\the \theorem@style
%    \end{macrocode}
%    In addition the default theorem font should be changeable, i.e.\
%    we have to insert the contents of |\theorem@bodyfont|. For
%    that reason, we expand even further, beyond |\endcsname|,
%    and thus insert the contents of the \textsf{token} register
%    |\theorem@bodyfont| in the replacement text.
%    \begin{macrocode}
            \expandafter \endcsname \the \theorem@bodyfont
%    \end{macrocode}
%    Now it is time to call the macro |\@thm| which takes over
%    the further processing.  It has two arguments: the current
%    counter name (in our example, \texttt{Lem}), and the text of the
%    label (in our example, \texttt{Lemma}).
%    \begin{macrocode}
     \@thm{#1}{#2}}%
%    \end{macrocode}
%    With this, the `sub-definition' is complete.  The macro
%    |\@endtheorem| ends a theorem environment and is, so far, nothing
%    but an |\endtrivlist|.  (Hence it is defined globally, and not
%    within the theorem styles.\footnote{This has to be changed as
%    soon as theorem styles that change \texttt{\bslash @endtheorem}
%    exist.  In such a case, all existing styles must be changed as
%    well since they will have to reset the macro.}) Therefore, we can
%    set it equivalent to the macro that ends the theorem set (in our
%    example, |\endLem|).  However, if some day theorem styles exist
%    that do change |\@endtheorem|, we would have to use the
%    commented-out line instead.
%    \begin{macrocode}
    \global \expandafter \let \csname end#1\endcsname \@endtheorem
%   \global\@namedef{end#1}{\@endtheorem}%
%    \end{macrocode}
%    With these commands all the required definitions are employed,
%    unless the test |\@ifdefinable| has failed.  Therefore, we
%    end the second argument of this macro and with it the definition
%    of |\@xnthm|.
%    \begin{macrocode}
   }}
%    \end{macrocode}
% \end{macro}
%
% \begin{macro}{\@ynthm}
%    The definition of |\@ynthm| is completely analogous. In
%    this case the new counter that is defined is not reset within
%    another counter; thus the definition of |\the...|
%    is simplified:
%    \begin{macrocode}
\gdef\@ynthm#1#2{\expandafter\@ifdefinable\csname #1\endcsname
   {\@definecounter{#1}%
    \expandafter\xdef\csname the#1\endcsname{\@thmcounter{#1}}%
%    \end{macrocode}
%    The rest of the definition corresponds literally to that of
%    |\@xnthm|:
%    \begin{macrocode}
    \def\@tempa{\global\@namedef{#1}}\expandafter \@tempa
     \expandafter{\csname th@\the \theorem@style \expandafter
     \endcsname \the\theorem@bodyfont \@thm{#1}{#2}}%
    \global \expandafter \let \csname end#1\endcsname \@endtheorem}}
%    \end{macrocode}
% \end{macro}
%
% \begin{macro}{\@othm}
%    The definition of |\@othm| does not contain anything new.
%    \begin{macrocode}
\gdef\@othm#1[#2]#3{%
%    \end{macrocode}
% \changes{v2.2b}{95/10/07}{Build in check for counter defined}
%    We do not define a new counter but instead use one that has
%    already been defined.  Thus the only definition we need is that
%    of this pseudo-counter (i.e.\ |\the|\meta{env.\ name}).
%    First we check if |#2| corresponds to a known counter name.
%    \begin{macrocode}
  \expandafter\ifx\csname c@#2\endcsname\relax
   \@nocounterr{#2}%
  \else
   \expandafter\@ifdefinable\csname #1\endcsname
   {\expandafter \xdef \csname the#1\endcsname
     {\expandafter \noexpand \csname the#2\endcsname}%
%    \end{macrocode}
%    All other parts of the definition can be adopted from
%    |\@xnthm|. We have to remember, though, that in this case
%    the name of the current counter and the theorem label have moved
%    to the second and third arguments.
%    \begin{macrocode}
    \def\@tempa{\global\@namedef{#1}}\expandafter \@tempa
     \expandafter{\csname th@\the \theorem@style \expandafter
     \endcsname \the\theorem@bodyfont \@thm{#2}{#3}}%
    \global \expandafter \let \csname end#1\endcsname \@endtheorem}%
  \fi}
%    \end{macrocode}
% \end{macro}
%
% \subsection{Macros that are employed in a {\envfont theorem}
%    environment}
%
% \begin{macro}{\@thm}
%    The macro |\@thm| has to increase the current counter.  Then,
%    depending on whether the environment has (or does not have) an
%    optional argument, it has to branch into either |\@begintheorem|
%    or |\@opargtheorem|.
%    \begin{macrocode}
\gdef\@thm#1#2{\refstepcounter{#1}%
%    \end{macrocode}
%    Now we start a {\envfont trivlist} environment, and give
%    |\@topsep| and |\@topsepadd| the values of the \textsf{skip}
%    registers |\theorempreskipamount| and |\theorempostskipamount|.
%    The value in |\@topsep| is the vertical space that is inserted by
%    the first (and only) |\item| in our |\trivlist| whilst
%    |\@topsepadd| is inserted by |\@endparenv| at the end of that
%    {\envfont trivlist} environment.  By using these registers, we
%    obtain the desired space around a {\envfont theorem} environment.
%    \begin{macrocode}
   \trivlist
   \@topsep \theorempreskipamount               % used by first \item
   \@topsepadd \theorempostskipamount           % used by \@endparenv
%    \end{macrocode}
%    Now we have to test whether an optional argument has been given.
% \changes{v2.0d}{89/07/18}{Forgotten percent added.}
%    \begin{macrocode}
   \@ifnextchar [%
%    \end{macrocode}
%    If there is an optional argument, we will call |\@ythm|, and
%    move the arguments read back into the input stream.
%    \begin{macrocode}
   {\@ythm{#1}{#2}}%
%    \end{macrocode}
%    If not, we call |\@begintheorem|. Its first argument is the
%    name of the theorem set (hence the second argument of
%    |\@thm|). Its second argument is the macro that produces the
%    current number.
%    \begin{macrocode}
   {\@begintheorem{#2}{\csname the#1\endcsname}\ignorespaces}}
%    \end{macrocode}
% \end{macro}
%
% \begin{macro}{\@xthm}
% \begin{macro}{\@ythm}
%    Both these macros were originally called by |\@thm|.  We do
%    not need |\@xthm| anymore, hence we reset it to
%    |\relax|. The definition of |\@ythm| has not changed at
%    all from its definition in \LaTeX. In order to make the macros
%    easier to understand, we will
%    nevertheless present it (commented out).
%    \begin{macrocode}
\global\let\@xthm\relax
% \def\@ythm#1#2[#3]{\@opargbegintheorem{#2}{\csname
%           the#1\endcsname}{#3}\ignorespaces}
%</package>
%    \end{macrocode}
%    The primitive |\ignorespaces| in |\@ythm| and
%    |\@thm| is needed to remove the spaces between the
%    |\begin{...}| and the actual text.
% \end{macro}
% \end{macro}
%
%
% \subsection{Definition of the theorem substyles}
%
% As already pointed out, the theorem substyles, defined below, are
% only loaded when necessary.  Note that all these substyles, except
% \texttt{plain}, have |\slshape| as the default body font.
%
% \subsubsection{The \texttt{plain} style}
%
% \label{dok:th@plain} As the following macros use |@|, we have
% to locally set the |\catcode| of this symbol to ``letter''.
% This happens within a group, so that we do not have to worry about
% which |\catcode| that symbol had before.
%    \begin{macrocode}
%<*thp>
\begingroup \makeatletter
%    \end{macrocode}
% Since we are now within a group, we must make all definitions
% globally.  First we make sure that \texttt{theorem.sty} is loaded.
% This will allow us to use this file as a document style option
% without having to call \texttt{theorem} itself as an option.  At the
% same time, we assure that at least version 2 is loaded, since
% |\theorem@style| was not defined in earlier versions.
%    \begin{macrocode}
\@ifundefined{theorem@style}{\input{theorem.sty}}{}
\ProvidesFile{thp.sty}
             [\FMithmInfo]
%    \end{macrocode}
%
%
% \begin{macro}{\th@plain}
% \begin{macro}{\@begintheorem}
% \begin{macro}{\@opargbegintheorem}
%    |\theoremstyle{plain}| corresponds to the original
%    definition, except that the distances to the surrounding text are
%    determined by the parameters |\theorempreskipamount| and
%    |\theorempostskipamount|.
%    First we set the default body font.
%    \begin{macrocode}
\gdef\th@plain{\normalfont\itshape
%    \end{macrocode}
%    Then we define |\@begintheorem| and
%    |\@opargbegintheorem|. These two macros define how the
%    header of a theorem is typeset.  |\@opargbegintheorem| will
%    be called if a \textsf{theorem} environment with an optional
%    argument is encountered; otherwise, the header is constructed by
%    calling |\@begintheorem|.  If one of these macros is
%    executed, we are within a \textsf{trivlist} environment started by
%    |\@thm|.  So the theorem header is produced with an
%    |\item| command.
%
%    Instead of specifying the header font directly, all standard
%    theorem styles use the |\theorem@headerfont| macro to allow
%    customization.  The extra space (|\labelsep|) is necessary
%    because of problems in the \textsf{trivlist} environment.
%    \begin{macrocode}
  \def\@begintheorem##1##2{%
        \item[\hskip\labelsep \theorem@headerfont ##1\ ##2]}%
%    \end{macrocode}
%    The definition of |\@opargbegintheorem| is completely
%    analogous. The only difference is the fact that there exists a
%    third argument (which is the optional parameter of the
%    environment and contains additional information about the
%    theorem).  Customarily we enclose it in parentheses.
%    \begin{macrocode}
\def\@opargbegintheorem##1##2##3{%
   \item[\hskip\labelsep \theorem@headerfont ##1\ ##2\ (##3)]}}
%    \end{macrocode}
% \end{macro}
% \end{macro}
% \end{macro}
%
% \noindent
% We conclude with an |\endgroup| to restore the
% |\catcode| of |@|.
%    \begin{macrocode}
\endgroup
%</thp>
%    \end{macrocode}
%
%
% \subsubsection{The \texttt{break} style}
%
% This style option is stored in the file |thb.sty|.
% For the next two lines see the documentation for |\th@plain| on
% page~\pageref{dok:th@plain}.
%    \begin{macrocode}
%<*thb>
\begingroup \makeatletter
\@ifundefined{theorem@style}{\input{theorem.sty}}{}
\ProvidesFile{thb.sty}
             [\FMithmInfo]
%    \end{macrocode}
%
% \begin{macro}{\th@break}
%    |\theoremstyle{break}| produces a line break after the name
%    of the theorem.  The font is |\slshape|.  Hence, we define
%    |\th@break| as follows:
%    \begin{macrocode}
\gdef\th@break{\normalfont\slshape
  \def\@begintheorem##1##2{\item[%
%    \end{macrocode}
%    We run into the following problem: it is not possible to create
%    the header with |\item[|\meta{title}|]| and then start
%    a new line by, for example, |\mbox{}\\|. Such a definition
%    will fail whenever a list environment follows immediately.  With
%    the above construction, the |\mbox{}| causes the switch
%    |@inlabel| (cf.\ definition of |\list| and
%    |\trivlist| in \cite{latex:source}) to be set to \texttt{false}
%    and so the following list will insert additional vertical
%    space (|\topskip|). This is quite annoying. Therefore, we
%    create the line break within the |\item|. In order to ensure
%    that the text will begin at the proper position in the following
%    line, we simply pretend that the label does not take any
%    room.\footnote{This will lead to problems whenever very high
%    symbols occurring in the line tower into the
%    heading. So, something else has to be done here sometime.}
%    \begin{macrocode}
       \rlap{\vbox{\hbox{\hskip \labelsep\theorem@headerfont ##1\ ##2}%
                   \hbox{\strut}}}]}%
%    \end{macrocode}
%    Again, the definition of |\@opargbegintheorem| is
%    completely analogous.
%    \begin{macrocode}
\def\@opargbegintheorem##1##2##3{%
  \item[\rlap{\vbox{\hbox{\hskip \labelsep \theorem@headerfont
                     ##1\ ##2\ (##3)}%
                    \hbox{\strut}}}]}}
\endgroup
%</thb>
%    \end{macrocode}
% \end{macro}
%
%
% \subsubsection{The \texttt{changebreak} style}
%
%    \begin{macrocode}
% This style option is stored in the file |thcb.sty|.
%    \begin{macrocode}
%<*thcb>
\begingroup \makeatletter
\@ifundefined{theorem@style}{\input{theorem.sty}}{}
\ProvidesFile{thcb.sty}
             [\FMithmInfo]
%    \end{macrocode}
%
% \begin{macro}{\th@changebreak}
%    The \texttt{change-break} theorem style is like \texttt{break}
%    but with interchange of theorem name and theorem number.  Thus we
%    define |\th@changebreak| as follows:
%    \begin{macrocode}
\gdef\th@changebreak{\normalfont\slshape
  \def\@begintheorem##1##2{\item
       [\rlap{\vbox{\hbox{\hskip\labelsep \theorem@headerfont ##2\ ##1}%
                    \hbox{\strut}}}]}%
\def\@opargbegintheorem##1##2##3{%
  \item[\rlap{\vbox{\hbox{\hskip\labelsep \theorem@headerfont
                     ##2\ ##1\ (##3)}%
                    \hbox{\strut}}}]}}
\endgroup
%</thcb>
%    \end{macrocode}
% \end{macro}
%
%
% \subsubsection{The \texttt{change} style}
%
% This style option is stored in the file |thc.sty|.
%    \begin{macrocode}
%<*thc>
\begingroup \makeatletter
\@ifundefined{theorem@style}{\input{theorem.sty}}{}
\ProvidesFile{thc.sty}
             [\FMithmInfo]
%    \end{macrocode}
%
% \begin{macro}{\th@change}
%    The \texttt{change} theorem style corresponds to the
%    \texttt{change break} style without a linebreak after the header.
%    To say it in another way, it's the same as the \texttt{plain}
%    style but with number and name interchanged and |\slshape| as the
%    default font.
%    \begin{macrocode}
\gdef\th@change{\normalfont\slshape
  \def\@begintheorem##1##2{\item
       [\hskip\labelsep \theorem@headerfont ##2\ ##1]}%
\def\@opargbegintheorem##1##2##3{%
  \item[\hskip\labelsep \theorem@headerfont ##2\ ##1\ (##3)]}}
\endgroup
%</thc>
%    \end{macrocode}
% \end{macro}
%
%
% \subsubsection{The \texttt{marginbreak} style}
%
% This style option is the one used most often at Mainz.  It is saved
% in the file |thmb.sty|.
%    \begin{macrocode}
%<*thmb>
\begingroup \makeatletter
\@ifundefined{theorem@style}{\input{theorem.sty}}{}
\ProvidesFile{thmb.sty}
             [\FMithmInfo]
%    \end{macrocode}
%
% \begin{macro}{\th@marginbreak}
%    The \texttt{margin break} style is nearly the same as the
%    \texttt{change break} style. The only difference is the placement
%    of the theorem number.  We use |\llap| to place it in the left
%    margin.
%
%    In this style |\labelsep| denotes the separation between the
%    number and the text.
%    \begin{macrocode}
\gdef\th@marginbreak{\normalfont\slshape
  \def\@begintheorem##1##2{\item
       [\rlap{\vbox{\theorem@headerfont
                    \hbox{\llap{##2}\hskip\labelsep ##1}%
                    \hbox{\strut}}}]}%
\def\@opargbegintheorem##1##2##3{%
  \item[\rlap{\vbox{\theorem@headerfont
                    \hbox{\llap{##2}\hskip\labelsep ##1\ (##3)}%
                    \hbox{\strut}}}]}}
\endgroup
%</thmb>
%    \end{macrocode}
% \changes{v2.1c}{90/02/17}{`endgroup outside macrocode env.}
% \end{macro}
%
%
%
% \subsubsection{The \texttt{margin} style}
%
% This style option is stored in the file |thm.sty|.
%    \begin{macrocode}
%<*thm>
\begingroup \makeatletter
\@ifundefined{theorem@style}{\input{theorem.sty}}{}
\ProvidesFile{thm.sty}
             [\FMithmInfo]
%    \end{macrocode}
%
% \begin{macro}{\th@margin}
%    Again this is only a variant of the theorem styles described
%    above without any new ideas.
%    \begin{macrocode}
\gdef\th@margin{\normalfont\slshape
  \def\@begintheorem##1##2{\item
       [\theorem@headerfont \llap{##2}\hskip\labelsep ##1]}%
\def\@opargbegintheorem##1##2##3{%
  \item[\theorem@headerfont \llap{##2}\hskip\labelsep ##1\ (##3)]}}
\endgroup
%</thm>
%    \end{macrocode}
% \end{macro}
%
%
%
%
% \subsection{Final Definitions}
%
% \begin{macro}{\theorempreskipamount}
% \begin{macro}{\theorempostskipamount}
%    The \textsf{skip} parameters that regulate the vertical empty space
%    before and after the {\envfont theorem} environment have
%    to be allocated as well.
% \changes{v2.0f}{89/07/20}{Unnecessary `global' removed.}
%    \begin{macrocode}
%<*package>
\newskip\theorempreskipamount
\newskip\theorempostskipamount
%    \end{macrocode}
%    Since we have used the same values for all theorem sets, we now can
%    assign them.
%    \begin{macrocode}
\global\setlength\theorempreskipamount{12pt plus 5pt minus 3pt}
\global\setlength\theorempostskipamount{8pt plus 3pt minus 1.5pt}
%    \end{macrocode}
% \end{macro}
% \end{macro}
%
%
% \begin{macro}{\@endtheorem}
%    The same holds for the macro |\@endtheorem|, which ends a
%    {\envfont theorem} environment. Since it is the same for all
%    theorem sets, it is removed from the macros |\th@|\meta{style}.
%    It simply ends the {\envfont trivlist} environment, which was
%    begun in |\@thm|.
%    \begin{macrocode}
\global\let\@endtheorem=\endtrivlist
%    \end{macrocode}
% \end{macro}
%
%
%
% \begin{macro}{\@preamblecmds}
%    All macros defined above are to be used only in the preamble.
%    Therefore, we insert them in |\@preamblecmds| which will disable
%    them at begin document. This is done by the internal \LaTeXe{}
%    command |\@onlypreamble|.
% \changes{v2.0g}{89/09/19}{`theoremheaderfont and `theorembodyfont
%    added.}
% \changes{v2.2a}{94/02/03}{Use \LaTeXe{} interface}
%    \begin{macrocode}
\@onlypreamble\@xnthm
\@onlypreamble\@ynthm
\@onlypreamble\@othm
\@onlypreamble\newtheorem
\@onlypreamble\theoremstyle
\@onlypreamble\theorembodyfont
\@onlypreamble\theoremheaderfont
%    \end{macrocode}
% \end{macro}
%
% Finally we declare the \texttt{plain} theorem style to be the default.
% \changes{v2.1b}{90/06/04}{Fixed missing `item by declaring default
%    style.}
%    \begin{macrocode}
\theoremstyle{plain}
%</package>
%    \end{macrocode}
%
% \Finale

\endinput
}{}
\ProvidesFile{thb.sty}
             [\FMithmInfo]
%    \end{macrocode}
%
% \begin{macro}{\th@break}
%    |\theoremstyle{break}| produces a line break after the name
%    of the theorem.  The font is |\slshape|.  Hence, we define
%    |\th@break| as follows:
%    \begin{macrocode}
\gdef\th@break{\normalfont\slshape
  \def\@begintheorem##1##2{\item[%
%    \end{macrocode}
%    We run into the following problem: it is not possible to create
%    the header with |\item[|\meta{title}|]| and then start
%    a new line by, for example, |\mbox{}\\|. Such a definition
%    will fail whenever a list environment follows immediately.  With
%    the above construction, the |\mbox{}| causes the switch
%    |@inlabel| (cf.\ definition of |\list| and
%    |\trivlist| in \cite{latex:source}) to be set to \texttt{false}
%    and so the following list will insert additional vertical
%    space (|\topskip|). This is quite annoying. Therefore, we
%    create the line break within the |\item|. In order to ensure
%    that the text will begin at the proper position in the following
%    line, we simply pretend that the label does not take any
%    room.\footnote{This will lead to problems whenever very high
%    symbols occurring in the line tower into the
%    heading. So, something else has to be done here sometime.}
%    \begin{macrocode}
       \rlap{\vbox{\hbox{\hskip \labelsep\theorem@headerfont ##1\ ##2}%
                   \hbox{\strut}}}]}%
%    \end{macrocode}
%    Again, the definition of |\@opargbegintheorem| is
%    completely analogous.
%    \begin{macrocode}
\def\@opargbegintheorem##1##2##3{%
  \item[\rlap{\vbox{\hbox{\hskip \labelsep \theorem@headerfont
                     ##1\ ##2\ (##3)}%
                    \hbox{\strut}}}]}}
\endgroup
%</thb>
%    \end{macrocode}
% \end{macro}
%
%
% \subsubsection{The \texttt{changebreak} style}
%
%    \begin{macrocode}
% This style option is stored in the file |thcb.sty|.
%    \begin{macrocode}
%<*thcb>
\begingroup \makeatletter
\@ifundefined{theorem@style}{% \iffalse meta-comment
%
% Copyright (C) 1993-2024
%
% The LaTeX Project and any individual authors listed elsewhere
% in this file.
%
% This file is part of the Standard LaTeX `Tools Bundle'.
% -------------------------------------------------------
%
% It may be distributed and/or modified under the
% conditions of the LaTeX Project Public License, either version 1.3c
% of this license or (at your option) any later version.
% The latest version of this license is in
%    https://www.latex-project.org/lppl.txt
% and version 1.3c or later is part of all distributions of LaTeX
% version 2005/12/01 or later.
%
% The list of all files belonging to the LaTeX `Tools Bundle' is
% given in the file `manifest.txt'.
%
% \fi
%
%%
%
%
\def\FMithmInfo{2023/07/05 v2.2c Theorem extension package (FMi)}
%
% \ProvidesFile{theorem.dtx}[\FMithmInfo]
%\iffalse   % this is a METACOMMENT !
%%
%% Package `theorem' to use with LaTeX2e
%% Copyright (C) 1989-1995 Frank Mittelbach, all rights reserved.
%<+package>\@ifundefined{theorem@style}{}{\endinput}
%<+package>\NeedsTeXFormat{LaTeX2e}[1995/06/01]
%<+package>\ProvidesPackage{theorem}
%<+package>         [\FMithmInfo]
%
% \fi
%
% \changes{v1.0}{87/05/07}{First draft.}
% \changes{v1.0a}{87/05/07}{`thistheoremstyle' added.}
% \changes{v1.1a}{87/07/29}{New concept: redefine the internal
%                          \protect\LaTeX{} commands `@xnthm' etc.}
% \changes{v1.1b}{87/07/31}{`theoremfont' used to specify font.}
% \changes{v1.2a}{88/01/08}{`@endtheorem' now globally defined.}
% \changes{v1.2b}{88/01/11}{`newtheorem' only allowed in the preamble.}
% \changes{v1.3a}{88/02/14}{`theoremfont' now a toks register.}
%
% \changes{v2.0a}{88/11/16}{New implementation and new documentation.}
% \changes{v2.0a}{88/11/16}{No layout changes but internal modifications
%                           old substyles won't work.}
% \changes{v2.0b}{89/01/30}{`global' used for all definitions.}
% \changes{v2.0c}{89/06/15}{Documentation translated to English.
%                            Thanks to LLL and Nina}
% \changes{v2.1a}{90/02/15}{Support for new font selection.}
% \changes{v2.2a}{94/02/03}{Cleanup for LaTeX2e}
%
% \DoNotIndex{\@endparenv,\@ifdefinable,\@ifnextchar,\@ifundefined}
% \DoNotIndex{\@namedef,\@nameuse,\@spaces,\@tempa,\@warning}
% \DoNotIndex{\ }
% \DoNotIndex{\begingroup,\bfseries}
% \DoNotIndex{\csname}
% \DoNotIndex{\def,\do,\docdate}
% \DoNotIndex{\endcsname,\endgroup,\endinput,\endtrivlist,\expandafter}
% \DoNotIndex{\filedate,\fileversion}
% \DoNotIndex{\gdef,\global}
% \DoNotIndex{\hbox,\hskip}
% \DoNotIndex{\ignorespaces,\input,\itshape,\item}
% \DoNotIndex{\labelsep,\let,\llap}
% \DoNotIndex{\makeatletter}
% \DoNotIndex{\noexpand}
% \DoNotIndex{\refstepcounter,\relax,\rlap}
% \DoNotIndex{\setlength,\slshape,\space,\string,\strut}
% \DoNotIndex{\the,\trivlist,\typeout}
% \DoNotIndex{\vbox}
% \DoNotIndex{\xdef}
%
% \def\deflabel#1{\ttfamily #1\hfill}
% \def\deflist#1{\begin{list}{}{\settowidth\labelwidth{\ttfamily #1}%
%                               \setlength\leftmargin\labelwidth
%                               \addtolength\leftmargin\labelsep
%                               \let\makelabel\deflabel}}
% \def\enddeflist{\end{list}}
% \def\envfont{\normalfont\ttfamily}
%
% \MakeShortVerb{\|}
%
% \newcommand{\TUB}{{\sl TUGboat\/}}
%
% \GetFileInfo{theorem.dtx}
% \title{An Extension of the
%        \LaTeX{} theorem environment\thanks{This file
%        has version number \fileversion, last
%        revised \filedate.}}
% \author{Frank Mittelbach}
%
% \MaintainedByLaTeXTeam{tools}
% \maketitle
%
% \begin{abstract}
%    The macros described in this paper yield an extension of the
%    \LaTeX{} theorem mechanism. It is designed to satisfy the
%    different requirements of various journals. Thus, the
%    layout of the ``theorems'' can be manipulated by determining a
%    ``style''. This article describes not only the use, but
%    also the definition, of the necessary macros.
% \end{abstract}
%
% \section*{Preface to version 2.2}
%
% For \LaTeXe{} this package did not need any fundamental changes. I
% only modified the messages generated so that theorem layout styles
% will show up with the |\listfiles| command and cleaned the section
% on the New Font Selection Scheme since this is now included in
% \LaTeX.
%
% \section*{Preface to version 2.1}
%
% This version is identical to 2.0g described in \TUB\ 10\#3 except
% for some internal defaults which are now set depending on the used
% font selection scheme.
%
% This was done to avoid unpleasant surprises if the new font selection
% scheme is in force. For further details see
% section~\ref{sec:newfonts} and \cite{bk:GMS94}.
%
% \section{Introduction}
%
% For our purposes here, ``theorems'' are labelled enunciations,
% often set off from the main text by extra space and a font change.
% Theorems, corollaries, conjectures, definitions, and remarks
% are all instances of ``theorems''.  The ``header'' of these structures
% is composed of a label (such as \textsc{Theorem} or \textsc{Remark})
% and a number which serializes an item in the sequence of items
% with the same label.
%
% Shortly after the introduction of \LaTeX{} at the Fachbereich
% Mathematik in Mainz, the desire to manipulate the layout of
% ``theorems'' arose.  In Mainz, the following two conventions came
% into general use:
% \begin{enumerate}
%   \item The number of the theorem is shown in the margin.
%   \item There is a line break at
%         the end of the theorem header.
% \end{enumerate}
%
% Additionally, some journals require different formats which depend on
% the ``sort of theorem'': e.g.\ often remarks and definitions are set
% in |\upshape|, while |\itshape| is employed for main theorems.
%
% Confronted with these requirements, a theorem environment was
% developed in Mainz which allows separate determination of the
% layout of the ``theorems sets'', comparable to |\pagestyle|.
%
%
% \section{The user interface}
% \subsection{Defining new theorem sets}
%
% \DescribeMacro\newtheorem
% As in the original \LaTeX{} version, the command |\newtheorem|
% defines a new ``theorem set'' or ``theorem-like structure''.
% Two required arguments name the new environment and give the text
% to be typeset with each instance of the new ``set'', while
% an optional argument determines how
% the ``set'' is enumerated:
% \begin{description}
%    \item[\ttfamily \bslash newtheorem\{foo\}\{bar\}]
%       The theorem set {\envfont foo} (whose name is \texttt{bar})
%       uses its own counter.
%    \item[\ttfamily \bslash newtheorem\{foo2\}{[foo]}\{bar2\}]
%       The theorem set {\envfont foo2} (printed name \texttt{bar2})
%       uses the same counter as the theorem set \texttt{foo}.
%    \item[\ttfamily \bslash newtheorem\{foo3\}\{bar3\}{[section]}]
%       The theorem set {\envfont foo3} (printed name \texttt{bar3}) is
%       enumerated within the counter \texttt{section}, i.e.\ with every
%       new |\section| the enumeration begins again with 1, and
%       the enumeration is composed from the section-number and the
%       theorem counter itself.
% \end{description}
%
% \DescribeMacro\theoremstyle
% Additionally, the command |\theoremstyle| can define the layout
% of various, or all, theorem sets.  It should be noted that any
% theorem set defined by |\newtheorem| is typeset in the
% |\theoremstyle| that is current at the time of the definition.
% Thus, the following
% \begin{verbatim}
%    \theoremstyle{break}        \newtheorem{Cor}{Corollary}
%    \theoremstyle{plain}        \newtheorem{Exa}{Example}[section]
%\end{verbatim}                             ^^A No blank at line start
% leads to the result that the set {\envfont Cor} is formatted in the
% style \texttt{break}, while the set {\envfont Exa} and all the
% following ones are formatted in the style \texttt{plain}, unless
% another |\theoremstyle| follows.  Since the definitions installed by
% |\newtheorem| are global, one also can limit |\theoremstyle| locally
% by grouping braces.
%
% \DescribeMacro\theorembodyfont
% The choice of the font for the theorem body is completely
% independent of the chosen |\theoremstyle|; this has proven to
% be very advantageous.  For example,
% \begin{verbatim}
%   {\theorembodyfont{\upshape}          \newtheorem{Rem}{Remark}}
%\end{verbatim}
% defines a theorem set {\envfont Rem}, which will be set in
% |\upshape| in the current layout (which in our example is
% \texttt{plain}).  As with |\theoremstyle|, the |\theorembodyfont|
% chosen is that current at the time of |\newtheorem|.  If
% |\theorembodyfont| is not specified or one defines
% |\theorembodyfont{}|, then the font used will be that defined by the
% |\theoremstyle|.
%
% \DescribeMacro\theoremheaderfont It is also possible to customize
% the font used for the theorem headers.  This is, however, a global
% declaration, and therefore there should be at most one
% |\theoremheaderfont| declaration in the preamble.\footnote{If it is
% actually necessary to have different header fonts, one has to define
% new theorem styles (substituting the desired font) or specify the
% information directly in the \texttt{\bslash newtheorem} declaration
% (the unclean variant).}
%
% \DescribeMacro\theorempreskipamount
% \DescribeMacro\theorempostskipamount Two additional parameters
% affect the vertical space around the theorem environments:
% |\theorempreskipamount| and |\theorempostskipamount| define,
% respectively, the spacing before and after such an environment.
% These parameters apply for all theorem sets and can be manipulated
% with the ordinary length macros.  They are rubber lengths
% (`\textsf{skips}'), and therefore can contain \texttt{plus} and
% \texttt{minus} parts.
%
% Since the definition of theorem sets should---most sensibly---be
% placed in the preamble, we only allow installation there.
% It is therefore possible to release the memory used here after
% |\begin{document}|, in order to make room for other
% applications.
%
%
% \subsection{Existing theorem styles}
%
% The following theorem styles exist to date:
% \begin{deflist}{marginbreak}
%    \item[plain]
%       This theorem style emulates the original \LaTeX{} definition,
%       except that additionally the parameters
%       |\theorem...skipamount| are used.
%    \item[break]
%       In this style, the theorem header is followed by a line
%       break.
%    \item[marginbreak]
%       The theorem number is set in the margin, and there is a line
%       break as in \texttt{break}.
%    \item[changebreak]
%       Like \texttt{break}, but with header number and text
%       interchanged.
%    \item[change]
%       Header number and text are interchanged, without a line break.
%    \item[margin]
%       The number is set in the left margin, without a line break.
% \end{deflist}
% All styles (except \texttt{plain}) select |\slshape| as the default
% |\theorembodyfont|.
%
%
% \subsection{Examples}
%
% Given the above theorem sets {\envfont Cor}, {\envfont Exa}
% and {\envfont Rem},
% suppose that the preamble also contains the declarations:
% \begin{verbatim}
%    \theoremstyle{marginbreak}   \newtheorem{Lem}[Cor]{Lemma}
%    \theoremstyle{change}
%    \theorembodyfont{\itshape}        \newtheorem{Def}[Cor]{Definition}
%
%    \theoremheaderfont{\scshape}
%\end{verbatim}
% Then the following are some typical examples of the typeset output
% resulting from their use.
%
% \begin{Cor}
%    This is a sentence typeset in the theorem environment {\envfont
%    Cor}.
% \end{Cor}
%
% \begin{Exa}
%    This is a sentence typeset in the theorem environment {\envfont
%    Exa}.
% \end{Exa}
%
% \begin{Rem}
%    This is a sentence typeset in the theorem environment {\envfont
%    Rem}.
% \end{Rem}
%
% \begin{Lem}[Ben User]
%    This is a sentence typeset in the theorem environment {\envfont
%    Lem}.
% \end{Lem}
%
% \begin{Def}[Very impressive Definition]
%    This is a sentence typeset in the theorem environment {\envfont
%    Def}.
% \end{Def}
%
% The last two examples show the effect of the optional argument to
% a theorem environment (it is the text typeset in parentheses).
%
%
% \section{Special Considerations}
% \label{sec:newfonts}
%
% Theoremheader and body are implemented as a unit. This means that
% the |\theoremheaderfont|  will inherit characteristics of the
% |\theorembodyfont| in \LaTeXe. Thus, if for example
% |\theorembodyfont| is |\itshape| and |\theoremheaderfont|
% is |\bfseries| the font selected for the header will have the
% characteristics `bold extended italic'. If this is not desired one
% should set the |\theoremheaderfont| to something like
% \begin{verbatim}
%      \theoremheaderfont{\normalfont\bfseries}
%\end{verbatim}
% i.e.\ supplying all necessary font information explicitly.
%
%
%
% \section{Acknowledgements}
%
% The publication of this set of macros was only possible with the
% help of Christina Busse (translating the manuscript into English),
% Joachim Pense (playing the r\^ole of typist), Chris Rowley
% (looking everything over) and many others providing useful
% suggestions.
%
% \MaybeStop{%
%    \begin{thebibliography}{9}
%       \bibitem{bk:GMS94} \textsc{M.~Goossens}, \textsc{F.~Mittelbach}
%         and \textsc{A.~Samarin}.
%         \newblock The \LaTeX{} Companion.
%         \newblock
%         Addison-Wesley, Reading, Massachusetts, 1994.
%       \bibitem{latex:source} \textsc{Lamport, Leslie.} \newblock
%                               \texttt{latex.tex}, version 2.09, date
%                               Feb. 1990.
%    \end{thebibliography}
%    \PrintIndex
%    \PrintChanges}
%
%
% \section{The documentation driver file}
%
% The next bit of code contains the documentation driver file for
% \TeX{}, i.e., the file that will produce the documentation you are
% currently reading. It will be extracted from this file by the
% \texttt{docstrip} program. Since it is the first code in the file
% one can alternatively process this file directly with \LaTeXe{} to
% obtain the documentation.
%
%    \begin{macrocode}
%<*driver>
\documentclass{ltxdoc}

\usepackage{theorem}

%<+driver>% The next few lines define theorem sets which are used
%<+driver>% in the example section of the documentation.

\theoremstyle{break}       \newtheorem{Cor}{Corollary}
\theoremstyle{plain}       \newtheorem{Exa}{Example}[section]
{\theorembodyfont{\upshape}\newtheorem{Rem}{Remark}}
\theoremstyle{marginbreak} \newtheorem{Lem}[Cor]{Lemma}
\theoremstyle{change}
\theorembodyfont{\itshape} \newtheorem{Def}[Cor]{Definition}

\theoremheaderfont{\scshape}

\RecordChanges

\begin{document}
   \DocInput{theorem.dtx}
\end{document}
%</driver>
%    \end{macrocode}
%
% \section{Definition of the Macros}
%
% If the file has been loaded before, we abort immediately. If not the
% package announces itself (this is actually done at the very top of
% the file---the way it is done isn't good style so don't copy it).
% \changes{v2.0e}{89/07/19}{Spaces removed from `typeout'.}
% \changes{v2.2b}{95/11/19}{Announce moved to top of file}
%    \begin{macrocode}
%<*package>
%\@ifundefined{theorem@style}{}{\endinput}
%\def\FMithmInfo{1995/11/19 v2.2b Theorem extension package (FMi)}
%\ProvidesPackage{theorem}[\FMithmInfo]
%    \end{macrocode}
%
%
% \subsection{Definition of theorem styles and fonts}
%
% All the definitions in this file are done globally to allow
% inputting this file inside a group.
%
% \begin{macro}{\theoremstyle}
%    Before a theorem style can be installed, the chosen style must be
%    known. For that reason, we must test to see that
%    |\th@|\meta{style} is known or, more precisely, that it is
%    different from |\relax|. If the style is not known then
%    |\th@plain| is used.
%    \begin{macrocode}
\gdef\theoremstyle#1{%
   \@ifundefined{th@#1}{\@warning
          {Unknown theoremstyle `#1'. Using `plain'}%
          \theorem@style{plain}}%
%    \end{macrocode}
%    We save the theorem style to be used in the token register
%    |\theorem@style|.
%    \begin{macrocode}
      {\theorem@style{#1}}%
%    \end{macrocode}
%    Now we ``evaluate'' the theorem style: this means, we call the
%    macro |\th@|\meta{style} which will activate the relevant
%    definitions which are contained in a separate file.  This is done
%    in a group to suppress changes to the current font.  This could
%    otherwise pose problems together with the new font selection
%    scheme\footnote{When I printed the original article using the new
%    font selection scheme I ended with a document with slanted
%    typefaces (text headings and all) simply because one of the
%    theorem styles used \texttt{\bslash sl} at toplevel.} if the
%    |\th@|\meta{style} is evaluated a second time.
% \changes{v2.1a}{90/02/15}{Group added around `th@..}
% \changes{v2.1a}{90/02/15}{Reseting of `..@begintheorem removed.}
%    \begin{macrocode}
      \begingroup
        \csname th@\the\theorem@style \endcsname
      \endgroup}
%    \end{macrocode}
% \end{macro}
%
%
% \begin{macro}{\@begintheorem}
% \changes{v2.1a}{90/02/15}{Relaxed at toplevel.}
% \begin{macro}{\@opargbegintheorem}
% \changes{v2.1a}{90/02/15}{Relaxed at toplevel.}
%    We reset |\@begintheorem| and |\@opargbegintheorem| to
%    |\relax| since these commands are no longer necessary at
%    toplevel. This will save a few tokens.
%    \begin{macrocode}
\global\let\@begintheorem\relax
\global\let\@opargbegintheorem\relax
%    \end{macrocode}
% \end{macro}
% \end{macro}
%
% \begin{macro}{\theorem@style}
%    Obviously the \textsf{token} register used above has to be
%    allocated. To assure the utmost compatibility with the
%    original \LaTeX{} definition, we set the default theorem style to
%    \texttt{plain}, which implements the usual \LaTeX{} convention.
% \changes{v2.0f}{89/07/20}{Unnecessary `global' removed.}
%    \begin{macrocode}
\newtoks\theorem@style
\global\theorem@style{plain}
%    \end{macrocode}
% \end{macro}
%
% \begin{macro}{\theorembodyfont}
% \begin{macro}{\theorem@bodyfont}
%    If the |\theorembodyfont| is set by the user then it should not
%    interact with the default font set in the theorem style. When the
%    new font selection is in force this may happen if, for example,
%    the default is |\itshape| and the new |\theorembodyfont| is
%    |\sffamily|. So we add a |\reset@font| command in front of the
%    user definition.
% \changes{v2.0f}{89/07/20}{Unnecessary `global' removed.}
% \changes{v2.0g}{89/09/19}{`theoremfont to `theorembodyfont.}
% \changes{v2.1c}{92/06/22}{`reset@font added and logic changed.}
% \changes{v2.1d}{92/10/28}{Reset to empty if arg is empty}
%    \begin{macrocode}
\gdef\theorembodyfont#1{%
%    \end{macrocode}
%    We check if the argument supplied is empty and if so put nothing
%    into the |\theorem@bodyfont| token register to allow for
%    |\theorembodyfont{}| as a mean of using the default of the
%    current |\theoremstyle|.
%    \begin{macrocode}
   \def\@tempa{#1}%
   \ifx\@tempa\@empty
    \theorem@bodyfont{}%
   \else
    \theorem@bodyfont{\reset@font#1}%
   \fi
   }
\newtoks\theorem@bodyfont
\global\theorem@bodyfont{}
%    \end{macrocode}
% \end{macro}
% \end{macro}
%
%
% \begin{macro}{\theoremheaderfont}
% \changes{v2.0g}{89/09/19}{Macro added.}
%    The font for the theorem headers is handled differently because
%    this definition applies to all theorem styles.
%    \begin{macrocode}
\gdef\theoremheaderfont#1{\gdef\theorem@headerfont{#1}%
%    \end{macrocode}
%    After using the macro once it is redefined to produce an error
%    message.
%    \begin{macrocode}
       \gdef\theoremheaderfont##1{%
        \typeout{\string\theoremheaderfont\space should be used
                 only once.}}}
%    \end{macrocode}
% \end{macro}
%
% \begin{macro}{\theorem@headerfont}
% \changes{v2.0g}{89/09/19}{Macro added.}
%    To set the |\theorem@headerfont| default we first test if
%    the new fontselection scheme is in force.
% \changes{v2.1a}{90/02/15}{Support for new font selection.}
%    \begin{macrocode}
\ifx\upshape\undefined
%    \end{macrocode}
%    If not we define it to expand into |\bfseries|. We don't use
%    |\let| just in case a following style option redefines this
%    macro.
%    \begin{macrocode}
\gdef\theorem@headerfont{\bfseries}
%    \end{macrocode}
%    Otherwise we reset the current shape before calling |\bfseries|.
%    \begin{macrocode}
\else \gdef\theorem@headerfont{\normalfont\bfseries}\fi
%    \end{macrocode}
% \end{macro}
%
% \begin{macro}{\th@plain}
% \begin{macro}{\th@break}
% \begin{macro}{\th@marginbreak}
% \begin{macro}{\th@changebreak}
% \begin{macro}{\th@change}
% \begin{macro}{\th@margin}
%    The different styles are defined in macros such as
%    |\th@plain|.  Since memory space is precious in
%    ``non-Big-versions'', we have to avoid offering too many unused
%    definitions.  Therefore we define these styles in separate files
%    that can be loaded on demand.  Thus the commands themselves only
%    load these files. We use |\@input@|, a \LaTeXe{} internal command
%    that ensures that the file will be listed with |\listfiles|.
% \changes{v2.2a}{94/02/02}{Use `@input@ to load theorem layout files}
%    \begin{macrocode}
\gdef\th@plain{\@input@{thp.sty}}
\gdef\th@break{\@input@{thb.sty}}
\gdef\th@marginbreak{\@input@{thmb.sty}}
\gdef\th@changebreak{\@input@{thcb.sty}}
\gdef\th@change{\@input@{thc.sty}}
\gdef\th@margin{\@input@{thm.sty}}
%    \end{macrocode}
%    This list will be expanded when new styles become
%    available.  For testing,  just append new theorem substyles
%    as document options.
% \end{macro}
% \end{macro}
% \end{macro}
% \end{macro}
% \end{macro}
% \end{macro}
%
%
%
% \subsection{Definition of a new theorem set}
%
% As already pointed out, a new theorem environment can be defined in
% three different ways:
% \begin{verbatim}
%    \newtheorem{Lem}{Lemma}
%    \newtheorem{Lem}{Lemma}[section]
%    \newtheorem{Lem}[Theorem]{Lemma}
%\end{verbatim}
% The function of the macro |\newtheorem| is to recognize these cases
% and then to branch into one of the three macros |\@ynthm|, |\@xnthm|
% or |\@othm|.  This mechanism is adopted unchanged from
% \cite{latex:source}; the essential point here is that, for example,
% in the second case, the arguments \texttt{Lem}, \texttt{Lemma} and
% \texttt{section} are passed over to the macro |\@xnthm|.
%
% We inspect this case first because the others present fewer problems,
% and thus are easily derived from this one.
%
% \begin{macro}{\@xnthm}
%    For our example arguments, the macro |\@xnthm| must fulfill
%    the following:
%    \begin{itemize}
%       \item
%          Define a new \LaTeX-counter `Lem'
%       \item
%          reset this counter within a |\section|
%       \item
%          define the macro |\theLem|
%       \item
%          define the environment macros |\Lem| and
%          |\endLem| using the current
%          |\theoremstyle| and |\theorem@bodyfont|.
%    \end{itemize}
%    Obviously, all this should happen only if the first argument
%    of |\@xnthm| (i.e.\ \texttt{Lem} in our example) is chosen so
%    as not to conflict with any previously defined commands or
%    environments.  This test is performed by the \LaTeX{} macro
%    |\@ifdefinable|.
%    \begin{macrocode}
\gdef\@xnthm#1#2[#3]{\expandafter\@ifdefinable\csname #1\endcsname
%    \end{macrocode}
%    Therefore, the first argument of |\@ifdefinable| is the
%    expansion (in the example, |\Lem|) of
%    |\csname||#1\endcsname|.  The second argument is
%    executed only if the test has been completed successfully.
%    \begin{macrocode}
   {%
%    \end{macrocode}
%    Now we define the new counter. The names of the \LaTeX{}
%    macros employed should speak for themselves:
% \changes{v2.2b}{1995/10/07}{Use \cs{@newctr}}
% \changes{v2.2c}{1995/11/23}{Use \cs{@newctr} properly.}
%    \begin{macrocode}
    \@definecounter{#1}\@newctr{#1}[#3]%
%    \end{macrocode}
%    Using |\@newctr| will give a proper error message if the counter
%    in |#3| is not defined.
%    In defining `|\theLem|' we must generate the desired macro
%    name by use of |\expandafter| and |\csname|.
%    \begin{macrocode}
    \expandafter\xdef\csname the#1\endcsname
%    \end{macrocode}
%    An |\xdef| is used in order to make the definition global,
%    and to ensure that it contains the replacement texts of
%    |\@thmcountersep| and |\@thmcounter|.\footnote{These
%    two macros can be defined by the document style.  Their default
%    values produce a `\texttt{.}' as separation and an arabic
%    representation of the number.} However, not everything should be
%    expanded. For example, it saves space to use |\thesection|
%    instead of its---at times---lengthy expansion.
%    \begin{macrocode}
      {\expandafter \noexpand \csname the#3\endcsname
       \@thmcountersep \@thmcounter{#1}}%
%    \end{macrocode}
%    Thus with the defaults of \LaTeX{}, |\theLem| would be
%    replaced by the command sequence |\thesection.\arabic{Lem}|.
%
%    We will now look at the definition of the macro which is executed
%    at the beginning of the actual environment (in our example this
%    macro is |\Lem|). It should be noted that we use an
%    ``|\expandafter| trick'' to expand only certain parts of the
%    replacement text at the time of the definition.
%    \begin{macrocode}
    \def\@tempa{\global\@namedef{#1}}%
    \expandafter \@tempa \expandafter{%
%    \end{macrocode}
%    First, the macro that contains the current definitions of
%    |\@begintheorem| and |\@opargtheorem| should be called
%    up.  The name of this macro---as is already known---has the form
%    |\th@|\meta{theorem\ style}; therefore, it must be called by
%    \begin{macrocode}
      \csname th@\the \theorem@style
%    \end{macrocode}
%    In addition the default theorem font should be changeable, i.e.\
%    we have to insert the contents of |\theorem@bodyfont|. For
%    that reason, we expand even further, beyond |\endcsname|,
%    and thus insert the contents of the \textsf{token} register
%    |\theorem@bodyfont| in the replacement text.
%    \begin{macrocode}
            \expandafter \endcsname \the \theorem@bodyfont
%    \end{macrocode}
%    Now it is time to call the macro |\@thm| which takes over
%    the further processing.  It has two arguments: the current
%    counter name (in our example, \texttt{Lem}), and the text of the
%    label (in our example, \texttt{Lemma}).
%    \begin{macrocode}
     \@thm{#1}{#2}}%
%    \end{macrocode}
%    With this, the `sub-definition' is complete.  The macro
%    |\@endtheorem| ends a theorem environment and is, so far, nothing
%    but an |\endtrivlist|.  (Hence it is defined globally, and not
%    within the theorem styles.\footnote{This has to be changed as
%    soon as theorem styles that change \texttt{\bslash @endtheorem}
%    exist.  In such a case, all existing styles must be changed as
%    well since they will have to reset the macro.}) Therefore, we can
%    set it equivalent to the macro that ends the theorem set (in our
%    example, |\endLem|).  However, if some day theorem styles exist
%    that do change |\@endtheorem|, we would have to use the
%    commented-out line instead.
%    \begin{macrocode}
    \global \expandafter \let \csname end#1\endcsname \@endtheorem
%   \global\@namedef{end#1}{\@endtheorem}%
%    \end{macrocode}
%    With these commands all the required definitions are employed,
%    unless the test |\@ifdefinable| has failed.  Therefore, we
%    end the second argument of this macro and with it the definition
%    of |\@xnthm|.
%    \begin{macrocode}
   }}
%    \end{macrocode}
% \end{macro}
%
% \begin{macro}{\@ynthm}
%    The definition of |\@ynthm| is completely analogous. In
%    this case the new counter that is defined is not reset within
%    another counter; thus the definition of |\the...|
%    is simplified:
%    \begin{macrocode}
\gdef\@ynthm#1#2{\expandafter\@ifdefinable\csname #1\endcsname
   {\@definecounter{#1}%
    \expandafter\xdef\csname the#1\endcsname{\@thmcounter{#1}}%
%    \end{macrocode}
%    The rest of the definition corresponds literally to that of
%    |\@xnthm|:
%    \begin{macrocode}
    \def\@tempa{\global\@namedef{#1}}\expandafter \@tempa
     \expandafter{\csname th@\the \theorem@style \expandafter
     \endcsname \the\theorem@bodyfont \@thm{#1}{#2}}%
    \global \expandafter \let \csname end#1\endcsname \@endtheorem}}
%    \end{macrocode}
% \end{macro}
%
% \begin{macro}{\@othm}
%    The definition of |\@othm| does not contain anything new.
%    \begin{macrocode}
\gdef\@othm#1[#2]#3{%
%    \end{macrocode}
% \changes{v2.2b}{95/10/07}{Build in check for counter defined}
%    We do not define a new counter but instead use one that has
%    already been defined.  Thus the only definition we need is that
%    of this pseudo-counter (i.e.\ |\the|\meta{env.\ name}).
%    First we check if |#2| corresponds to a known counter name.
%    \begin{macrocode}
  \expandafter\ifx\csname c@#2\endcsname\relax
   \@nocounterr{#2}%
  \else
   \expandafter\@ifdefinable\csname #1\endcsname
   {\expandafter \xdef \csname the#1\endcsname
     {\expandafter \noexpand \csname the#2\endcsname}%
%    \end{macrocode}
%    All other parts of the definition can be adopted from
%    |\@xnthm|. We have to remember, though, that in this case
%    the name of the current counter and the theorem label have moved
%    to the second and third arguments.
%    \begin{macrocode}
    \def\@tempa{\global\@namedef{#1}}\expandafter \@tempa
     \expandafter{\csname th@\the \theorem@style \expandafter
     \endcsname \the\theorem@bodyfont \@thm{#2}{#3}}%
    \global \expandafter \let \csname end#1\endcsname \@endtheorem}%
  \fi}
%    \end{macrocode}
% \end{macro}
%
% \subsection{Macros that are employed in a {\envfont theorem}
%    environment}
%
% \begin{macro}{\@thm}
%    The macro |\@thm| has to increase the current counter.  Then,
%    depending on whether the environment has (or does not have) an
%    optional argument, it has to branch into either |\@begintheorem|
%    or |\@opargtheorem|.
%    \begin{macrocode}
\gdef\@thm#1#2{\refstepcounter{#1}%
%    \end{macrocode}
%    Now we start a {\envfont trivlist} environment, and give
%    |\@topsep| and |\@topsepadd| the values of the \textsf{skip}
%    registers |\theorempreskipamount| and |\theorempostskipamount|.
%    The value in |\@topsep| is the vertical space that is inserted by
%    the first (and only) |\item| in our |\trivlist| whilst
%    |\@topsepadd| is inserted by |\@endparenv| at the end of that
%    {\envfont trivlist} environment.  By using these registers, we
%    obtain the desired space around a {\envfont theorem} environment.
%    \begin{macrocode}
   \trivlist
   \@topsep \theorempreskipamount               % used by first \item
   \@topsepadd \theorempostskipamount           % used by \@endparenv
%    \end{macrocode}
%    Now we have to test whether an optional argument has been given.
% \changes{v2.0d}{89/07/18}{Forgotten percent added.}
%    \begin{macrocode}
   \@ifnextchar [%
%    \end{macrocode}
%    If there is an optional argument, we will call |\@ythm|, and
%    move the arguments read back into the input stream.
%    \begin{macrocode}
   {\@ythm{#1}{#2}}%
%    \end{macrocode}
%    If not, we call |\@begintheorem|. Its first argument is the
%    name of the theorem set (hence the second argument of
%    |\@thm|). Its second argument is the macro that produces the
%    current number.
%    \begin{macrocode}
   {\@begintheorem{#2}{\csname the#1\endcsname}\ignorespaces}}
%    \end{macrocode}
% \end{macro}
%
% \begin{macro}{\@xthm}
% \begin{macro}{\@ythm}
%    Both these macros were originally called by |\@thm|.  We do
%    not need |\@xthm| anymore, hence we reset it to
%    |\relax|. The definition of |\@ythm| has not changed at
%    all from its definition in \LaTeX. In order to make the macros
%    easier to understand, we will
%    nevertheless present it (commented out).
%    \begin{macrocode}
\global\let\@xthm\relax
% \def\@ythm#1#2[#3]{\@opargbegintheorem{#2}{\csname
%           the#1\endcsname}{#3}\ignorespaces}
%</package>
%    \end{macrocode}
%    The primitive |\ignorespaces| in |\@ythm| and
%    |\@thm| is needed to remove the spaces between the
%    |\begin{...}| and the actual text.
% \end{macro}
% \end{macro}
%
%
% \subsection{Definition of the theorem substyles}
%
% As already pointed out, the theorem substyles, defined below, are
% only loaded when necessary.  Note that all these substyles, except
% \texttt{plain}, have |\slshape| as the default body font.
%
% \subsubsection{The \texttt{plain} style}
%
% \label{dok:th@plain} As the following macros use |@|, we have
% to locally set the |\catcode| of this symbol to ``letter''.
% This happens within a group, so that we do not have to worry about
% which |\catcode| that symbol had before.
%    \begin{macrocode}
%<*thp>
\begingroup \makeatletter
%    \end{macrocode}
% Since we are now within a group, we must make all definitions
% globally.  First we make sure that \texttt{theorem.sty} is loaded.
% This will allow us to use this file as a document style option
% without having to call \texttt{theorem} itself as an option.  At the
% same time, we assure that at least version 2 is loaded, since
% |\theorem@style| was not defined in earlier versions.
%    \begin{macrocode}
\@ifundefined{theorem@style}{\input{theorem.sty}}{}
\ProvidesFile{thp.sty}
             [\FMithmInfo]
%    \end{macrocode}
%
%
% \begin{macro}{\th@plain}
% \begin{macro}{\@begintheorem}
% \begin{macro}{\@opargbegintheorem}
%    |\theoremstyle{plain}| corresponds to the original
%    definition, except that the distances to the surrounding text are
%    determined by the parameters |\theorempreskipamount| and
%    |\theorempostskipamount|.
%    First we set the default body font.
%    \begin{macrocode}
\gdef\th@plain{\normalfont\itshape
%    \end{macrocode}
%    Then we define |\@begintheorem| and
%    |\@opargbegintheorem|. These two macros define how the
%    header of a theorem is typeset.  |\@opargbegintheorem| will
%    be called if a \textsf{theorem} environment with an optional
%    argument is encountered; otherwise, the header is constructed by
%    calling |\@begintheorem|.  If one of these macros is
%    executed, we are within a \textsf{trivlist} environment started by
%    |\@thm|.  So the theorem header is produced with an
%    |\item| command.
%
%    Instead of specifying the header font directly, all standard
%    theorem styles use the |\theorem@headerfont| macro to allow
%    customization.  The extra space (|\labelsep|) is necessary
%    because of problems in the \textsf{trivlist} environment.
%    \begin{macrocode}
  \def\@begintheorem##1##2{%
        \item[\hskip\labelsep \theorem@headerfont ##1\ ##2]}%
%    \end{macrocode}
%    The definition of |\@opargbegintheorem| is completely
%    analogous. The only difference is the fact that there exists a
%    third argument (which is the optional parameter of the
%    environment and contains additional information about the
%    theorem).  Customarily we enclose it in parentheses.
%    \begin{macrocode}
\def\@opargbegintheorem##1##2##3{%
   \item[\hskip\labelsep \theorem@headerfont ##1\ ##2\ (##3)]}}
%    \end{macrocode}
% \end{macro}
% \end{macro}
% \end{macro}
%
% \noindent
% We conclude with an |\endgroup| to restore the
% |\catcode| of |@|.
%    \begin{macrocode}
\endgroup
%</thp>
%    \end{macrocode}
%
%
% \subsubsection{The \texttt{break} style}
%
% This style option is stored in the file |thb.sty|.
% For the next two lines see the documentation for |\th@plain| on
% page~\pageref{dok:th@plain}.
%    \begin{macrocode}
%<*thb>
\begingroup \makeatletter
\@ifundefined{theorem@style}{\input{theorem.sty}}{}
\ProvidesFile{thb.sty}
             [\FMithmInfo]
%    \end{macrocode}
%
% \begin{macro}{\th@break}
%    |\theoremstyle{break}| produces a line break after the name
%    of the theorem.  The font is |\slshape|.  Hence, we define
%    |\th@break| as follows:
%    \begin{macrocode}
\gdef\th@break{\normalfont\slshape
  \def\@begintheorem##1##2{\item[%
%    \end{macrocode}
%    We run into the following problem: it is not possible to create
%    the header with |\item[|\meta{title}|]| and then start
%    a new line by, for example, |\mbox{}\\|. Such a definition
%    will fail whenever a list environment follows immediately.  With
%    the above construction, the |\mbox{}| causes the switch
%    |@inlabel| (cf.\ definition of |\list| and
%    |\trivlist| in \cite{latex:source}) to be set to \texttt{false}
%    and so the following list will insert additional vertical
%    space (|\topskip|). This is quite annoying. Therefore, we
%    create the line break within the |\item|. In order to ensure
%    that the text will begin at the proper position in the following
%    line, we simply pretend that the label does not take any
%    room.\footnote{This will lead to problems whenever very high
%    symbols occurring in the line tower into the
%    heading. So, something else has to be done here sometime.}
%    \begin{macrocode}
       \rlap{\vbox{\hbox{\hskip \labelsep\theorem@headerfont ##1\ ##2}%
                   \hbox{\strut}}}]}%
%    \end{macrocode}
%    Again, the definition of |\@opargbegintheorem| is
%    completely analogous.
%    \begin{macrocode}
\def\@opargbegintheorem##1##2##3{%
  \item[\rlap{\vbox{\hbox{\hskip \labelsep \theorem@headerfont
                     ##1\ ##2\ (##3)}%
                    \hbox{\strut}}}]}}
\endgroup
%</thb>
%    \end{macrocode}
% \end{macro}
%
%
% \subsubsection{The \texttt{changebreak} style}
%
%    \begin{macrocode}
% This style option is stored in the file |thcb.sty|.
%    \begin{macrocode}
%<*thcb>
\begingroup \makeatletter
\@ifundefined{theorem@style}{\input{theorem.sty}}{}
\ProvidesFile{thcb.sty}
             [\FMithmInfo]
%    \end{macrocode}
%
% \begin{macro}{\th@changebreak}
%    The \texttt{change-break} theorem style is like \texttt{break}
%    but with interchange of theorem name and theorem number.  Thus we
%    define |\th@changebreak| as follows:
%    \begin{macrocode}
\gdef\th@changebreak{\normalfont\slshape
  \def\@begintheorem##1##2{\item
       [\rlap{\vbox{\hbox{\hskip\labelsep \theorem@headerfont ##2\ ##1}%
                    \hbox{\strut}}}]}%
\def\@opargbegintheorem##1##2##3{%
  \item[\rlap{\vbox{\hbox{\hskip\labelsep \theorem@headerfont
                     ##2\ ##1\ (##3)}%
                    \hbox{\strut}}}]}}
\endgroup
%</thcb>
%    \end{macrocode}
% \end{macro}
%
%
% \subsubsection{The \texttt{change} style}
%
% This style option is stored in the file |thc.sty|.
%    \begin{macrocode}
%<*thc>
\begingroup \makeatletter
\@ifundefined{theorem@style}{\input{theorem.sty}}{}
\ProvidesFile{thc.sty}
             [\FMithmInfo]
%    \end{macrocode}
%
% \begin{macro}{\th@change}
%    The \texttt{change} theorem style corresponds to the
%    \texttt{change break} style without a linebreak after the header.
%    To say it in another way, it's the same as the \texttt{plain}
%    style but with number and name interchanged and |\slshape| as the
%    default font.
%    \begin{macrocode}
\gdef\th@change{\normalfont\slshape
  \def\@begintheorem##1##2{\item
       [\hskip\labelsep \theorem@headerfont ##2\ ##1]}%
\def\@opargbegintheorem##1##2##3{%
  \item[\hskip\labelsep \theorem@headerfont ##2\ ##1\ (##3)]}}
\endgroup
%</thc>
%    \end{macrocode}
% \end{macro}
%
%
% \subsubsection{The \texttt{marginbreak} style}
%
% This style option is the one used most often at Mainz.  It is saved
% in the file |thmb.sty|.
%    \begin{macrocode}
%<*thmb>
\begingroup \makeatletter
\@ifundefined{theorem@style}{\input{theorem.sty}}{}
\ProvidesFile{thmb.sty}
             [\FMithmInfo]
%    \end{macrocode}
%
% \begin{macro}{\th@marginbreak}
%    The \texttt{margin break} style is nearly the same as the
%    \texttt{change break} style. The only difference is the placement
%    of the theorem number.  We use |\llap| to place it in the left
%    margin.
%
%    In this style |\labelsep| denotes the separation between the
%    number and the text.
%    \begin{macrocode}
\gdef\th@marginbreak{\normalfont\slshape
  \def\@begintheorem##1##2{\item
       [\rlap{\vbox{\theorem@headerfont
                    \hbox{\llap{##2}\hskip\labelsep ##1}%
                    \hbox{\strut}}}]}%
\def\@opargbegintheorem##1##2##3{%
  \item[\rlap{\vbox{\theorem@headerfont
                    \hbox{\llap{##2}\hskip\labelsep ##1\ (##3)}%
                    \hbox{\strut}}}]}}
\endgroup
%</thmb>
%    \end{macrocode}
% \changes{v2.1c}{90/02/17}{`endgroup outside macrocode env.}
% \end{macro}
%
%
%
% \subsubsection{The \texttt{margin} style}
%
% This style option is stored in the file |thm.sty|.
%    \begin{macrocode}
%<*thm>
\begingroup \makeatletter
\@ifundefined{theorem@style}{\input{theorem.sty}}{}
\ProvidesFile{thm.sty}
             [\FMithmInfo]
%    \end{macrocode}
%
% \begin{macro}{\th@margin}
%    Again this is only a variant of the theorem styles described
%    above without any new ideas.
%    \begin{macrocode}
\gdef\th@margin{\normalfont\slshape
  \def\@begintheorem##1##2{\item
       [\theorem@headerfont \llap{##2}\hskip\labelsep ##1]}%
\def\@opargbegintheorem##1##2##3{%
  \item[\theorem@headerfont \llap{##2}\hskip\labelsep ##1\ (##3)]}}
\endgroup
%</thm>
%    \end{macrocode}
% \end{macro}
%
%
%
%
% \subsection{Final Definitions}
%
% \begin{macro}{\theorempreskipamount}
% \begin{macro}{\theorempostskipamount}
%    The \textsf{skip} parameters that regulate the vertical empty space
%    before and after the {\envfont theorem} environment have
%    to be allocated as well.
% \changes{v2.0f}{89/07/20}{Unnecessary `global' removed.}
%    \begin{macrocode}
%<*package>
\newskip\theorempreskipamount
\newskip\theorempostskipamount
%    \end{macrocode}
%    Since we have used the same values for all theorem sets, we now can
%    assign them.
%    \begin{macrocode}
\global\setlength\theorempreskipamount{12pt plus 5pt minus 3pt}
\global\setlength\theorempostskipamount{8pt plus 3pt minus 1.5pt}
%    \end{macrocode}
% \end{macro}
% \end{macro}
%
%
% \begin{macro}{\@endtheorem}
%    The same holds for the macro |\@endtheorem|, which ends a
%    {\envfont theorem} environment. Since it is the same for all
%    theorem sets, it is removed from the macros |\th@|\meta{style}.
%    It simply ends the {\envfont trivlist} environment, which was
%    begun in |\@thm|.
%    \begin{macrocode}
\global\let\@endtheorem=\endtrivlist
%    \end{macrocode}
% \end{macro}
%
%
%
% \begin{macro}{\@preamblecmds}
%    All macros defined above are to be used only in the preamble.
%    Therefore, we insert them in |\@preamblecmds| which will disable
%    them at begin document. This is done by the internal \LaTeXe{}
%    command |\@onlypreamble|.
% \changes{v2.0g}{89/09/19}{`theoremheaderfont and `theorembodyfont
%    added.}
% \changes{v2.2a}{94/02/03}{Use \LaTeXe{} interface}
%    \begin{macrocode}
\@onlypreamble\@xnthm
\@onlypreamble\@ynthm
\@onlypreamble\@othm
\@onlypreamble\newtheorem
\@onlypreamble\theoremstyle
\@onlypreamble\theorembodyfont
\@onlypreamble\theoremheaderfont
%    \end{macrocode}
% \end{macro}
%
% Finally we declare the \texttt{plain} theorem style to be the default.
% \changes{v2.1b}{90/06/04}{Fixed missing `item by declaring default
%    style.}
%    \begin{macrocode}
\theoremstyle{plain}
%</package>
%    \end{macrocode}
%
% \Finale

\endinput
}{}
\ProvidesFile{thcb.sty}
             [\FMithmInfo]
%    \end{macrocode}
%
% \begin{macro}{\th@changebreak}
%    The \texttt{change-break} theorem style is like \texttt{break}
%    but with interchange of theorem name and theorem number.  Thus we
%    define |\th@changebreak| as follows:
%    \begin{macrocode}
\gdef\th@changebreak{\normalfont\slshape
  \def\@begintheorem##1##2{\item
       [\rlap{\vbox{\hbox{\hskip\labelsep \theorem@headerfont ##2\ ##1}%
                    \hbox{\strut}}}]}%
\def\@opargbegintheorem##1##2##3{%
  \item[\rlap{\vbox{\hbox{\hskip\labelsep \theorem@headerfont
                     ##2\ ##1\ (##3)}%
                    \hbox{\strut}}}]}}
\endgroup
%</thcb>
%    \end{macrocode}
% \end{macro}
%
%
% \subsubsection{The \texttt{change} style}
%
% This style option is stored in the file |thc.sty|.
%    \begin{macrocode}
%<*thc>
\begingroup \makeatletter
\@ifundefined{theorem@style}{% \iffalse meta-comment
%
% Copyright (C) 1993-2024
%
% The LaTeX Project and any individual authors listed elsewhere
% in this file.
%
% This file is part of the Standard LaTeX `Tools Bundle'.
% -------------------------------------------------------
%
% It may be distributed and/or modified under the
% conditions of the LaTeX Project Public License, either version 1.3c
% of this license or (at your option) any later version.
% The latest version of this license is in
%    https://www.latex-project.org/lppl.txt
% and version 1.3c or later is part of all distributions of LaTeX
% version 2005/12/01 or later.
%
% The list of all files belonging to the LaTeX `Tools Bundle' is
% given in the file `manifest.txt'.
%
% \fi
%
%%
%
%
\def\FMithmInfo{2023/07/05 v2.2c Theorem extension package (FMi)}
%
% \ProvidesFile{theorem.dtx}[\FMithmInfo]
%\iffalse   % this is a METACOMMENT !
%%
%% Package `theorem' to use with LaTeX2e
%% Copyright (C) 1989-1995 Frank Mittelbach, all rights reserved.
%<+package>\@ifundefined{theorem@style}{}{\endinput}
%<+package>\NeedsTeXFormat{LaTeX2e}[1995/06/01]
%<+package>\ProvidesPackage{theorem}
%<+package>         [\FMithmInfo]
%
% \fi
%
% \changes{v1.0}{87/05/07}{First draft.}
% \changes{v1.0a}{87/05/07}{`thistheoremstyle' added.}
% \changes{v1.1a}{87/07/29}{New concept: redefine the internal
%                          \protect\LaTeX{} commands `@xnthm' etc.}
% \changes{v1.1b}{87/07/31}{`theoremfont' used to specify font.}
% \changes{v1.2a}{88/01/08}{`@endtheorem' now globally defined.}
% \changes{v1.2b}{88/01/11}{`newtheorem' only allowed in the preamble.}
% \changes{v1.3a}{88/02/14}{`theoremfont' now a toks register.}
%
% \changes{v2.0a}{88/11/16}{New implementation and new documentation.}
% \changes{v2.0a}{88/11/16}{No layout changes but internal modifications
%                           old substyles won't work.}
% \changes{v2.0b}{89/01/30}{`global' used for all definitions.}
% \changes{v2.0c}{89/06/15}{Documentation translated to English.
%                            Thanks to LLL and Nina}
% \changes{v2.1a}{90/02/15}{Support for new font selection.}
% \changes{v2.2a}{94/02/03}{Cleanup for LaTeX2e}
%
% \DoNotIndex{\@endparenv,\@ifdefinable,\@ifnextchar,\@ifundefined}
% \DoNotIndex{\@namedef,\@nameuse,\@spaces,\@tempa,\@warning}
% \DoNotIndex{\ }
% \DoNotIndex{\begingroup,\bfseries}
% \DoNotIndex{\csname}
% \DoNotIndex{\def,\do,\docdate}
% \DoNotIndex{\endcsname,\endgroup,\endinput,\endtrivlist,\expandafter}
% \DoNotIndex{\filedate,\fileversion}
% \DoNotIndex{\gdef,\global}
% \DoNotIndex{\hbox,\hskip}
% \DoNotIndex{\ignorespaces,\input,\itshape,\item}
% \DoNotIndex{\labelsep,\let,\llap}
% \DoNotIndex{\makeatletter}
% \DoNotIndex{\noexpand}
% \DoNotIndex{\refstepcounter,\relax,\rlap}
% \DoNotIndex{\setlength,\slshape,\space,\string,\strut}
% \DoNotIndex{\the,\trivlist,\typeout}
% \DoNotIndex{\vbox}
% \DoNotIndex{\xdef}
%
% \def\deflabel#1{\ttfamily #1\hfill}
% \def\deflist#1{\begin{list}{}{\settowidth\labelwidth{\ttfamily #1}%
%                               \setlength\leftmargin\labelwidth
%                               \addtolength\leftmargin\labelsep
%                               \let\makelabel\deflabel}}
% \def\enddeflist{\end{list}}
% \def\envfont{\normalfont\ttfamily}
%
% \MakeShortVerb{\|}
%
% \newcommand{\TUB}{{\sl TUGboat\/}}
%
% \GetFileInfo{theorem.dtx}
% \title{An Extension of the
%        \LaTeX{} theorem environment\thanks{This file
%        has version number \fileversion, last
%        revised \filedate.}}
% \author{Frank Mittelbach}
%
% \MaintainedByLaTeXTeam{tools}
% \maketitle
%
% \begin{abstract}
%    The macros described in this paper yield an extension of the
%    \LaTeX{} theorem mechanism. It is designed to satisfy the
%    different requirements of various journals. Thus, the
%    layout of the ``theorems'' can be manipulated by determining a
%    ``style''. This article describes not only the use, but
%    also the definition, of the necessary macros.
% \end{abstract}
%
% \section*{Preface to version 2.2}
%
% For \LaTeXe{} this package did not need any fundamental changes. I
% only modified the messages generated so that theorem layout styles
% will show up with the |\listfiles| command and cleaned the section
% on the New Font Selection Scheme since this is now included in
% \LaTeX.
%
% \section*{Preface to version 2.1}
%
% This version is identical to 2.0g described in \TUB\ 10\#3 except
% for some internal defaults which are now set depending on the used
% font selection scheme.
%
% This was done to avoid unpleasant surprises if the new font selection
% scheme is in force. For further details see
% section~\ref{sec:newfonts} and \cite{bk:GMS94}.
%
% \section{Introduction}
%
% For our purposes here, ``theorems'' are labelled enunciations,
% often set off from the main text by extra space and a font change.
% Theorems, corollaries, conjectures, definitions, and remarks
% are all instances of ``theorems''.  The ``header'' of these structures
% is composed of a label (such as \textsc{Theorem} or \textsc{Remark})
% and a number which serializes an item in the sequence of items
% with the same label.
%
% Shortly after the introduction of \LaTeX{} at the Fachbereich
% Mathematik in Mainz, the desire to manipulate the layout of
% ``theorems'' arose.  In Mainz, the following two conventions came
% into general use:
% \begin{enumerate}
%   \item The number of the theorem is shown in the margin.
%   \item There is a line break at
%         the end of the theorem header.
% \end{enumerate}
%
% Additionally, some journals require different formats which depend on
% the ``sort of theorem'': e.g.\ often remarks and definitions are set
% in |\upshape|, while |\itshape| is employed for main theorems.
%
% Confronted with these requirements, a theorem environment was
% developed in Mainz which allows separate determination of the
% layout of the ``theorems sets'', comparable to |\pagestyle|.
%
%
% \section{The user interface}
% \subsection{Defining new theorem sets}
%
% \DescribeMacro\newtheorem
% As in the original \LaTeX{} version, the command |\newtheorem|
% defines a new ``theorem set'' or ``theorem-like structure''.
% Two required arguments name the new environment and give the text
% to be typeset with each instance of the new ``set'', while
% an optional argument determines how
% the ``set'' is enumerated:
% \begin{description}
%    \item[\ttfamily \bslash newtheorem\{foo\}\{bar\}]
%       The theorem set {\envfont foo} (whose name is \texttt{bar})
%       uses its own counter.
%    \item[\ttfamily \bslash newtheorem\{foo2\}{[foo]}\{bar2\}]
%       The theorem set {\envfont foo2} (printed name \texttt{bar2})
%       uses the same counter as the theorem set \texttt{foo}.
%    \item[\ttfamily \bslash newtheorem\{foo3\}\{bar3\}{[section]}]
%       The theorem set {\envfont foo3} (printed name \texttt{bar3}) is
%       enumerated within the counter \texttt{section}, i.e.\ with every
%       new |\section| the enumeration begins again with 1, and
%       the enumeration is composed from the section-number and the
%       theorem counter itself.
% \end{description}
%
% \DescribeMacro\theoremstyle
% Additionally, the command |\theoremstyle| can define the layout
% of various, or all, theorem sets.  It should be noted that any
% theorem set defined by |\newtheorem| is typeset in the
% |\theoremstyle| that is current at the time of the definition.
% Thus, the following
% \begin{verbatim}
%    \theoremstyle{break}        \newtheorem{Cor}{Corollary}
%    \theoremstyle{plain}        \newtheorem{Exa}{Example}[section]
%\end{verbatim}                             ^^A No blank at line start
% leads to the result that the set {\envfont Cor} is formatted in the
% style \texttt{break}, while the set {\envfont Exa} and all the
% following ones are formatted in the style \texttt{plain}, unless
% another |\theoremstyle| follows.  Since the definitions installed by
% |\newtheorem| are global, one also can limit |\theoremstyle| locally
% by grouping braces.
%
% \DescribeMacro\theorembodyfont
% The choice of the font for the theorem body is completely
% independent of the chosen |\theoremstyle|; this has proven to
% be very advantageous.  For example,
% \begin{verbatim}
%   {\theorembodyfont{\upshape}          \newtheorem{Rem}{Remark}}
%\end{verbatim}
% defines a theorem set {\envfont Rem}, which will be set in
% |\upshape| in the current layout (which in our example is
% \texttt{plain}).  As with |\theoremstyle|, the |\theorembodyfont|
% chosen is that current at the time of |\newtheorem|.  If
% |\theorembodyfont| is not specified or one defines
% |\theorembodyfont{}|, then the font used will be that defined by the
% |\theoremstyle|.
%
% \DescribeMacro\theoremheaderfont It is also possible to customize
% the font used for the theorem headers.  This is, however, a global
% declaration, and therefore there should be at most one
% |\theoremheaderfont| declaration in the preamble.\footnote{If it is
% actually necessary to have different header fonts, one has to define
% new theorem styles (substituting the desired font) or specify the
% information directly in the \texttt{\bslash newtheorem} declaration
% (the unclean variant).}
%
% \DescribeMacro\theorempreskipamount
% \DescribeMacro\theorempostskipamount Two additional parameters
% affect the vertical space around the theorem environments:
% |\theorempreskipamount| and |\theorempostskipamount| define,
% respectively, the spacing before and after such an environment.
% These parameters apply for all theorem sets and can be manipulated
% with the ordinary length macros.  They are rubber lengths
% (`\textsf{skips}'), and therefore can contain \texttt{plus} and
% \texttt{minus} parts.
%
% Since the definition of theorem sets should---most sensibly---be
% placed in the preamble, we only allow installation there.
% It is therefore possible to release the memory used here after
% |\begin{document}|, in order to make room for other
% applications.
%
%
% \subsection{Existing theorem styles}
%
% The following theorem styles exist to date:
% \begin{deflist}{marginbreak}
%    \item[plain]
%       This theorem style emulates the original \LaTeX{} definition,
%       except that additionally the parameters
%       |\theorem...skipamount| are used.
%    \item[break]
%       In this style, the theorem header is followed by a line
%       break.
%    \item[marginbreak]
%       The theorem number is set in the margin, and there is a line
%       break as in \texttt{break}.
%    \item[changebreak]
%       Like \texttt{break}, but with header number and text
%       interchanged.
%    \item[change]
%       Header number and text are interchanged, without a line break.
%    \item[margin]
%       The number is set in the left margin, without a line break.
% \end{deflist}
% All styles (except \texttt{plain}) select |\slshape| as the default
% |\theorembodyfont|.
%
%
% \subsection{Examples}
%
% Given the above theorem sets {\envfont Cor}, {\envfont Exa}
% and {\envfont Rem},
% suppose that the preamble also contains the declarations:
% \begin{verbatim}
%    \theoremstyle{marginbreak}   \newtheorem{Lem}[Cor]{Lemma}
%    \theoremstyle{change}
%    \theorembodyfont{\itshape}        \newtheorem{Def}[Cor]{Definition}
%
%    \theoremheaderfont{\scshape}
%\end{verbatim}
% Then the following are some typical examples of the typeset output
% resulting from their use.
%
% \begin{Cor}
%    This is a sentence typeset in the theorem environment {\envfont
%    Cor}.
% \end{Cor}
%
% \begin{Exa}
%    This is a sentence typeset in the theorem environment {\envfont
%    Exa}.
% \end{Exa}
%
% \begin{Rem}
%    This is a sentence typeset in the theorem environment {\envfont
%    Rem}.
% \end{Rem}
%
% \begin{Lem}[Ben User]
%    This is a sentence typeset in the theorem environment {\envfont
%    Lem}.
% \end{Lem}
%
% \begin{Def}[Very impressive Definition]
%    This is a sentence typeset in the theorem environment {\envfont
%    Def}.
% \end{Def}
%
% The last two examples show the effect of the optional argument to
% a theorem environment (it is the text typeset in parentheses).
%
%
% \section{Special Considerations}
% \label{sec:newfonts}
%
% Theoremheader and body are implemented as a unit. This means that
% the |\theoremheaderfont|  will inherit characteristics of the
% |\theorembodyfont| in \LaTeXe. Thus, if for example
% |\theorembodyfont| is |\itshape| and |\theoremheaderfont|
% is |\bfseries| the font selected for the header will have the
% characteristics `bold extended italic'. If this is not desired one
% should set the |\theoremheaderfont| to something like
% \begin{verbatim}
%      \theoremheaderfont{\normalfont\bfseries}
%\end{verbatim}
% i.e.\ supplying all necessary font information explicitly.
%
%
%
% \section{Acknowledgements}
%
% The publication of this set of macros was only possible with the
% help of Christina Busse (translating the manuscript into English),
% Joachim Pense (playing the r\^ole of typist), Chris Rowley
% (looking everything over) and many others providing useful
% suggestions.
%
% \MaybeStop{%
%    \begin{thebibliography}{9}
%       \bibitem{bk:GMS94} \textsc{M.~Goossens}, \textsc{F.~Mittelbach}
%         and \textsc{A.~Samarin}.
%         \newblock The \LaTeX{} Companion.
%         \newblock
%         Addison-Wesley, Reading, Massachusetts, 1994.
%       \bibitem{latex:source} \textsc{Lamport, Leslie.} \newblock
%                               \texttt{latex.tex}, version 2.09, date
%                               Feb. 1990.
%    \end{thebibliography}
%    \PrintIndex
%    \PrintChanges}
%
%
% \section{The documentation driver file}
%
% The next bit of code contains the documentation driver file for
% \TeX{}, i.e., the file that will produce the documentation you are
% currently reading. It will be extracted from this file by the
% \texttt{docstrip} program. Since it is the first code in the file
% one can alternatively process this file directly with \LaTeXe{} to
% obtain the documentation.
%
%    \begin{macrocode}
%<*driver>
\documentclass{ltxdoc}

\usepackage{theorem}

%<+driver>% The next few lines define theorem sets which are used
%<+driver>% in the example section of the documentation.

\theoremstyle{break}       \newtheorem{Cor}{Corollary}
\theoremstyle{plain}       \newtheorem{Exa}{Example}[section]
{\theorembodyfont{\upshape}\newtheorem{Rem}{Remark}}
\theoremstyle{marginbreak} \newtheorem{Lem}[Cor]{Lemma}
\theoremstyle{change}
\theorembodyfont{\itshape} \newtheorem{Def}[Cor]{Definition}

\theoremheaderfont{\scshape}

\RecordChanges

\begin{document}
   \DocInput{theorem.dtx}
\end{document}
%</driver>
%    \end{macrocode}
%
% \section{Definition of the Macros}
%
% If the file has been loaded before, we abort immediately. If not the
% package announces itself (this is actually done at the very top of
% the file---the way it is done isn't good style so don't copy it).
% \changes{v2.0e}{89/07/19}{Spaces removed from `typeout'.}
% \changes{v2.2b}{95/11/19}{Announce moved to top of file}
%    \begin{macrocode}
%<*package>
%\@ifundefined{theorem@style}{}{\endinput}
%\def\FMithmInfo{1995/11/19 v2.2b Theorem extension package (FMi)}
%\ProvidesPackage{theorem}[\FMithmInfo]
%    \end{macrocode}
%
%
% \subsection{Definition of theorem styles and fonts}
%
% All the definitions in this file are done globally to allow
% inputting this file inside a group.
%
% \begin{macro}{\theoremstyle}
%    Before a theorem style can be installed, the chosen style must be
%    known. For that reason, we must test to see that
%    |\th@|\meta{style} is known or, more precisely, that it is
%    different from |\relax|. If the style is not known then
%    |\th@plain| is used.
%    \begin{macrocode}
\gdef\theoremstyle#1{%
   \@ifundefined{th@#1}{\@warning
          {Unknown theoremstyle `#1'. Using `plain'}%
          \theorem@style{plain}}%
%    \end{macrocode}
%    We save the theorem style to be used in the token register
%    |\theorem@style|.
%    \begin{macrocode}
      {\theorem@style{#1}}%
%    \end{macrocode}
%    Now we ``evaluate'' the theorem style: this means, we call the
%    macro |\th@|\meta{style} which will activate the relevant
%    definitions which are contained in a separate file.  This is done
%    in a group to suppress changes to the current font.  This could
%    otherwise pose problems together with the new font selection
%    scheme\footnote{When I printed the original article using the new
%    font selection scheme I ended with a document with slanted
%    typefaces (text headings and all) simply because one of the
%    theorem styles used \texttt{\bslash sl} at toplevel.} if the
%    |\th@|\meta{style} is evaluated a second time.
% \changes{v2.1a}{90/02/15}{Group added around `th@..}
% \changes{v2.1a}{90/02/15}{Reseting of `..@begintheorem removed.}
%    \begin{macrocode}
      \begingroup
        \csname th@\the\theorem@style \endcsname
      \endgroup}
%    \end{macrocode}
% \end{macro}
%
%
% \begin{macro}{\@begintheorem}
% \changes{v2.1a}{90/02/15}{Relaxed at toplevel.}
% \begin{macro}{\@opargbegintheorem}
% \changes{v2.1a}{90/02/15}{Relaxed at toplevel.}
%    We reset |\@begintheorem| and |\@opargbegintheorem| to
%    |\relax| since these commands are no longer necessary at
%    toplevel. This will save a few tokens.
%    \begin{macrocode}
\global\let\@begintheorem\relax
\global\let\@opargbegintheorem\relax
%    \end{macrocode}
% \end{macro}
% \end{macro}
%
% \begin{macro}{\theorem@style}
%    Obviously the \textsf{token} register used above has to be
%    allocated. To assure the utmost compatibility with the
%    original \LaTeX{} definition, we set the default theorem style to
%    \texttt{plain}, which implements the usual \LaTeX{} convention.
% \changes{v2.0f}{89/07/20}{Unnecessary `global' removed.}
%    \begin{macrocode}
\newtoks\theorem@style
\global\theorem@style{plain}
%    \end{macrocode}
% \end{macro}
%
% \begin{macro}{\theorembodyfont}
% \begin{macro}{\theorem@bodyfont}
%    If the |\theorembodyfont| is set by the user then it should not
%    interact with the default font set in the theorem style. When the
%    new font selection is in force this may happen if, for example,
%    the default is |\itshape| and the new |\theorembodyfont| is
%    |\sffamily|. So we add a |\reset@font| command in front of the
%    user definition.
% \changes{v2.0f}{89/07/20}{Unnecessary `global' removed.}
% \changes{v2.0g}{89/09/19}{`theoremfont to `theorembodyfont.}
% \changes{v2.1c}{92/06/22}{`reset@font added and logic changed.}
% \changes{v2.1d}{92/10/28}{Reset to empty if arg is empty}
%    \begin{macrocode}
\gdef\theorembodyfont#1{%
%    \end{macrocode}
%    We check if the argument supplied is empty and if so put nothing
%    into the |\theorem@bodyfont| token register to allow for
%    |\theorembodyfont{}| as a mean of using the default of the
%    current |\theoremstyle|.
%    \begin{macrocode}
   \def\@tempa{#1}%
   \ifx\@tempa\@empty
    \theorem@bodyfont{}%
   \else
    \theorem@bodyfont{\reset@font#1}%
   \fi
   }
\newtoks\theorem@bodyfont
\global\theorem@bodyfont{}
%    \end{macrocode}
% \end{macro}
% \end{macro}
%
%
% \begin{macro}{\theoremheaderfont}
% \changes{v2.0g}{89/09/19}{Macro added.}
%    The font for the theorem headers is handled differently because
%    this definition applies to all theorem styles.
%    \begin{macrocode}
\gdef\theoremheaderfont#1{\gdef\theorem@headerfont{#1}%
%    \end{macrocode}
%    After using the macro once it is redefined to produce an error
%    message.
%    \begin{macrocode}
       \gdef\theoremheaderfont##1{%
        \typeout{\string\theoremheaderfont\space should be used
                 only once.}}}
%    \end{macrocode}
% \end{macro}
%
% \begin{macro}{\theorem@headerfont}
% \changes{v2.0g}{89/09/19}{Macro added.}
%    To set the |\theorem@headerfont| default we first test if
%    the new fontselection scheme is in force.
% \changes{v2.1a}{90/02/15}{Support for new font selection.}
%    \begin{macrocode}
\ifx\upshape\undefined
%    \end{macrocode}
%    If not we define it to expand into |\bfseries|. We don't use
%    |\let| just in case a following style option redefines this
%    macro.
%    \begin{macrocode}
\gdef\theorem@headerfont{\bfseries}
%    \end{macrocode}
%    Otherwise we reset the current shape before calling |\bfseries|.
%    \begin{macrocode}
\else \gdef\theorem@headerfont{\normalfont\bfseries}\fi
%    \end{macrocode}
% \end{macro}
%
% \begin{macro}{\th@plain}
% \begin{macro}{\th@break}
% \begin{macro}{\th@marginbreak}
% \begin{macro}{\th@changebreak}
% \begin{macro}{\th@change}
% \begin{macro}{\th@margin}
%    The different styles are defined in macros such as
%    |\th@plain|.  Since memory space is precious in
%    ``non-Big-versions'', we have to avoid offering too many unused
%    definitions.  Therefore we define these styles in separate files
%    that can be loaded on demand.  Thus the commands themselves only
%    load these files. We use |\@input@|, a \LaTeXe{} internal command
%    that ensures that the file will be listed with |\listfiles|.
% \changes{v2.2a}{94/02/02}{Use `@input@ to load theorem layout files}
%    \begin{macrocode}
\gdef\th@plain{\@input@{thp.sty}}
\gdef\th@break{\@input@{thb.sty}}
\gdef\th@marginbreak{\@input@{thmb.sty}}
\gdef\th@changebreak{\@input@{thcb.sty}}
\gdef\th@change{\@input@{thc.sty}}
\gdef\th@margin{\@input@{thm.sty}}
%    \end{macrocode}
%    This list will be expanded when new styles become
%    available.  For testing,  just append new theorem substyles
%    as document options.
% \end{macro}
% \end{macro}
% \end{macro}
% \end{macro}
% \end{macro}
% \end{macro}
%
%
%
% \subsection{Definition of a new theorem set}
%
% As already pointed out, a new theorem environment can be defined in
% three different ways:
% \begin{verbatim}
%    \newtheorem{Lem}{Lemma}
%    \newtheorem{Lem}{Lemma}[section]
%    \newtheorem{Lem}[Theorem]{Lemma}
%\end{verbatim}
% The function of the macro |\newtheorem| is to recognize these cases
% and then to branch into one of the three macros |\@ynthm|, |\@xnthm|
% or |\@othm|.  This mechanism is adopted unchanged from
% \cite{latex:source}; the essential point here is that, for example,
% in the second case, the arguments \texttt{Lem}, \texttt{Lemma} and
% \texttt{section} are passed over to the macro |\@xnthm|.
%
% We inspect this case first because the others present fewer problems,
% and thus are easily derived from this one.
%
% \begin{macro}{\@xnthm}
%    For our example arguments, the macro |\@xnthm| must fulfill
%    the following:
%    \begin{itemize}
%       \item
%          Define a new \LaTeX-counter `Lem'
%       \item
%          reset this counter within a |\section|
%       \item
%          define the macro |\theLem|
%       \item
%          define the environment macros |\Lem| and
%          |\endLem| using the current
%          |\theoremstyle| and |\theorem@bodyfont|.
%    \end{itemize}
%    Obviously, all this should happen only if the first argument
%    of |\@xnthm| (i.e.\ \texttt{Lem} in our example) is chosen so
%    as not to conflict with any previously defined commands or
%    environments.  This test is performed by the \LaTeX{} macro
%    |\@ifdefinable|.
%    \begin{macrocode}
\gdef\@xnthm#1#2[#3]{\expandafter\@ifdefinable\csname #1\endcsname
%    \end{macrocode}
%    Therefore, the first argument of |\@ifdefinable| is the
%    expansion (in the example, |\Lem|) of
%    |\csname||#1\endcsname|.  The second argument is
%    executed only if the test has been completed successfully.
%    \begin{macrocode}
   {%
%    \end{macrocode}
%    Now we define the new counter. The names of the \LaTeX{}
%    macros employed should speak for themselves:
% \changes{v2.2b}{1995/10/07}{Use \cs{@newctr}}
% \changes{v2.2c}{1995/11/23}{Use \cs{@newctr} properly.}
%    \begin{macrocode}
    \@definecounter{#1}\@newctr{#1}[#3]%
%    \end{macrocode}
%    Using |\@newctr| will give a proper error message if the counter
%    in |#3| is not defined.
%    In defining `|\theLem|' we must generate the desired macro
%    name by use of |\expandafter| and |\csname|.
%    \begin{macrocode}
    \expandafter\xdef\csname the#1\endcsname
%    \end{macrocode}
%    An |\xdef| is used in order to make the definition global,
%    and to ensure that it contains the replacement texts of
%    |\@thmcountersep| and |\@thmcounter|.\footnote{These
%    two macros can be defined by the document style.  Their default
%    values produce a `\texttt{.}' as separation and an arabic
%    representation of the number.} However, not everything should be
%    expanded. For example, it saves space to use |\thesection|
%    instead of its---at times---lengthy expansion.
%    \begin{macrocode}
      {\expandafter \noexpand \csname the#3\endcsname
       \@thmcountersep \@thmcounter{#1}}%
%    \end{macrocode}
%    Thus with the defaults of \LaTeX{}, |\theLem| would be
%    replaced by the command sequence |\thesection.\arabic{Lem}|.
%
%    We will now look at the definition of the macro which is executed
%    at the beginning of the actual environment (in our example this
%    macro is |\Lem|). It should be noted that we use an
%    ``|\expandafter| trick'' to expand only certain parts of the
%    replacement text at the time of the definition.
%    \begin{macrocode}
    \def\@tempa{\global\@namedef{#1}}%
    \expandafter \@tempa \expandafter{%
%    \end{macrocode}
%    First, the macro that contains the current definitions of
%    |\@begintheorem| and |\@opargtheorem| should be called
%    up.  The name of this macro---as is already known---has the form
%    |\th@|\meta{theorem\ style}; therefore, it must be called by
%    \begin{macrocode}
      \csname th@\the \theorem@style
%    \end{macrocode}
%    In addition the default theorem font should be changeable, i.e.\
%    we have to insert the contents of |\theorem@bodyfont|. For
%    that reason, we expand even further, beyond |\endcsname|,
%    and thus insert the contents of the \textsf{token} register
%    |\theorem@bodyfont| in the replacement text.
%    \begin{macrocode}
            \expandafter \endcsname \the \theorem@bodyfont
%    \end{macrocode}
%    Now it is time to call the macro |\@thm| which takes over
%    the further processing.  It has two arguments: the current
%    counter name (in our example, \texttt{Lem}), and the text of the
%    label (in our example, \texttt{Lemma}).
%    \begin{macrocode}
     \@thm{#1}{#2}}%
%    \end{macrocode}
%    With this, the `sub-definition' is complete.  The macro
%    |\@endtheorem| ends a theorem environment and is, so far, nothing
%    but an |\endtrivlist|.  (Hence it is defined globally, and not
%    within the theorem styles.\footnote{This has to be changed as
%    soon as theorem styles that change \texttt{\bslash @endtheorem}
%    exist.  In such a case, all existing styles must be changed as
%    well since they will have to reset the macro.}) Therefore, we can
%    set it equivalent to the macro that ends the theorem set (in our
%    example, |\endLem|).  However, if some day theorem styles exist
%    that do change |\@endtheorem|, we would have to use the
%    commented-out line instead.
%    \begin{macrocode}
    \global \expandafter \let \csname end#1\endcsname \@endtheorem
%   \global\@namedef{end#1}{\@endtheorem}%
%    \end{macrocode}
%    With these commands all the required definitions are employed,
%    unless the test |\@ifdefinable| has failed.  Therefore, we
%    end the second argument of this macro and with it the definition
%    of |\@xnthm|.
%    \begin{macrocode}
   }}
%    \end{macrocode}
% \end{macro}
%
% \begin{macro}{\@ynthm}
%    The definition of |\@ynthm| is completely analogous. In
%    this case the new counter that is defined is not reset within
%    another counter; thus the definition of |\the...|
%    is simplified:
%    \begin{macrocode}
\gdef\@ynthm#1#2{\expandafter\@ifdefinable\csname #1\endcsname
   {\@definecounter{#1}%
    \expandafter\xdef\csname the#1\endcsname{\@thmcounter{#1}}%
%    \end{macrocode}
%    The rest of the definition corresponds literally to that of
%    |\@xnthm|:
%    \begin{macrocode}
    \def\@tempa{\global\@namedef{#1}}\expandafter \@tempa
     \expandafter{\csname th@\the \theorem@style \expandafter
     \endcsname \the\theorem@bodyfont \@thm{#1}{#2}}%
    \global \expandafter \let \csname end#1\endcsname \@endtheorem}}
%    \end{macrocode}
% \end{macro}
%
% \begin{macro}{\@othm}
%    The definition of |\@othm| does not contain anything new.
%    \begin{macrocode}
\gdef\@othm#1[#2]#3{%
%    \end{macrocode}
% \changes{v2.2b}{95/10/07}{Build in check for counter defined}
%    We do not define a new counter but instead use one that has
%    already been defined.  Thus the only definition we need is that
%    of this pseudo-counter (i.e.\ |\the|\meta{env.\ name}).
%    First we check if |#2| corresponds to a known counter name.
%    \begin{macrocode}
  \expandafter\ifx\csname c@#2\endcsname\relax
   \@nocounterr{#2}%
  \else
   \expandafter\@ifdefinable\csname #1\endcsname
   {\expandafter \xdef \csname the#1\endcsname
     {\expandafter \noexpand \csname the#2\endcsname}%
%    \end{macrocode}
%    All other parts of the definition can be adopted from
%    |\@xnthm|. We have to remember, though, that in this case
%    the name of the current counter and the theorem label have moved
%    to the second and third arguments.
%    \begin{macrocode}
    \def\@tempa{\global\@namedef{#1}}\expandafter \@tempa
     \expandafter{\csname th@\the \theorem@style \expandafter
     \endcsname \the\theorem@bodyfont \@thm{#2}{#3}}%
    \global \expandafter \let \csname end#1\endcsname \@endtheorem}%
  \fi}
%    \end{macrocode}
% \end{macro}
%
% \subsection{Macros that are employed in a {\envfont theorem}
%    environment}
%
% \begin{macro}{\@thm}
%    The macro |\@thm| has to increase the current counter.  Then,
%    depending on whether the environment has (or does not have) an
%    optional argument, it has to branch into either |\@begintheorem|
%    or |\@opargtheorem|.
%    \begin{macrocode}
\gdef\@thm#1#2{\refstepcounter{#1}%
%    \end{macrocode}
%    Now we start a {\envfont trivlist} environment, and give
%    |\@topsep| and |\@topsepadd| the values of the \textsf{skip}
%    registers |\theorempreskipamount| and |\theorempostskipamount|.
%    The value in |\@topsep| is the vertical space that is inserted by
%    the first (and only) |\item| in our |\trivlist| whilst
%    |\@topsepadd| is inserted by |\@endparenv| at the end of that
%    {\envfont trivlist} environment.  By using these registers, we
%    obtain the desired space around a {\envfont theorem} environment.
%    \begin{macrocode}
   \trivlist
   \@topsep \theorempreskipamount               % used by first \item
   \@topsepadd \theorempostskipamount           % used by \@endparenv
%    \end{macrocode}
%    Now we have to test whether an optional argument has been given.
% \changes{v2.0d}{89/07/18}{Forgotten percent added.}
%    \begin{macrocode}
   \@ifnextchar [%
%    \end{macrocode}
%    If there is an optional argument, we will call |\@ythm|, and
%    move the arguments read back into the input stream.
%    \begin{macrocode}
   {\@ythm{#1}{#2}}%
%    \end{macrocode}
%    If not, we call |\@begintheorem|. Its first argument is the
%    name of the theorem set (hence the second argument of
%    |\@thm|). Its second argument is the macro that produces the
%    current number.
%    \begin{macrocode}
   {\@begintheorem{#2}{\csname the#1\endcsname}\ignorespaces}}
%    \end{macrocode}
% \end{macro}
%
% \begin{macro}{\@xthm}
% \begin{macro}{\@ythm}
%    Both these macros were originally called by |\@thm|.  We do
%    not need |\@xthm| anymore, hence we reset it to
%    |\relax|. The definition of |\@ythm| has not changed at
%    all from its definition in \LaTeX. In order to make the macros
%    easier to understand, we will
%    nevertheless present it (commented out).
%    \begin{macrocode}
\global\let\@xthm\relax
% \def\@ythm#1#2[#3]{\@opargbegintheorem{#2}{\csname
%           the#1\endcsname}{#3}\ignorespaces}
%</package>
%    \end{macrocode}
%    The primitive |\ignorespaces| in |\@ythm| and
%    |\@thm| is needed to remove the spaces between the
%    |\begin{...}| and the actual text.
% \end{macro}
% \end{macro}
%
%
% \subsection{Definition of the theorem substyles}
%
% As already pointed out, the theorem substyles, defined below, are
% only loaded when necessary.  Note that all these substyles, except
% \texttt{plain}, have |\slshape| as the default body font.
%
% \subsubsection{The \texttt{plain} style}
%
% \label{dok:th@plain} As the following macros use |@|, we have
% to locally set the |\catcode| of this symbol to ``letter''.
% This happens within a group, so that we do not have to worry about
% which |\catcode| that symbol had before.
%    \begin{macrocode}
%<*thp>
\begingroup \makeatletter
%    \end{macrocode}
% Since we are now within a group, we must make all definitions
% globally.  First we make sure that \texttt{theorem.sty} is loaded.
% This will allow us to use this file as a document style option
% without having to call \texttt{theorem} itself as an option.  At the
% same time, we assure that at least version 2 is loaded, since
% |\theorem@style| was not defined in earlier versions.
%    \begin{macrocode}
\@ifundefined{theorem@style}{\input{theorem.sty}}{}
\ProvidesFile{thp.sty}
             [\FMithmInfo]
%    \end{macrocode}
%
%
% \begin{macro}{\th@plain}
% \begin{macro}{\@begintheorem}
% \begin{macro}{\@opargbegintheorem}
%    |\theoremstyle{plain}| corresponds to the original
%    definition, except that the distances to the surrounding text are
%    determined by the parameters |\theorempreskipamount| and
%    |\theorempostskipamount|.
%    First we set the default body font.
%    \begin{macrocode}
\gdef\th@plain{\normalfont\itshape
%    \end{macrocode}
%    Then we define |\@begintheorem| and
%    |\@opargbegintheorem|. These two macros define how the
%    header of a theorem is typeset.  |\@opargbegintheorem| will
%    be called if a \textsf{theorem} environment with an optional
%    argument is encountered; otherwise, the header is constructed by
%    calling |\@begintheorem|.  If one of these macros is
%    executed, we are within a \textsf{trivlist} environment started by
%    |\@thm|.  So the theorem header is produced with an
%    |\item| command.
%
%    Instead of specifying the header font directly, all standard
%    theorem styles use the |\theorem@headerfont| macro to allow
%    customization.  The extra space (|\labelsep|) is necessary
%    because of problems in the \textsf{trivlist} environment.
%    \begin{macrocode}
  \def\@begintheorem##1##2{%
        \item[\hskip\labelsep \theorem@headerfont ##1\ ##2]}%
%    \end{macrocode}
%    The definition of |\@opargbegintheorem| is completely
%    analogous. The only difference is the fact that there exists a
%    third argument (which is the optional parameter of the
%    environment and contains additional information about the
%    theorem).  Customarily we enclose it in parentheses.
%    \begin{macrocode}
\def\@opargbegintheorem##1##2##3{%
   \item[\hskip\labelsep \theorem@headerfont ##1\ ##2\ (##3)]}}
%    \end{macrocode}
% \end{macro}
% \end{macro}
% \end{macro}
%
% \noindent
% We conclude with an |\endgroup| to restore the
% |\catcode| of |@|.
%    \begin{macrocode}
\endgroup
%</thp>
%    \end{macrocode}
%
%
% \subsubsection{The \texttt{break} style}
%
% This style option is stored in the file |thb.sty|.
% For the next two lines see the documentation for |\th@plain| on
% page~\pageref{dok:th@plain}.
%    \begin{macrocode}
%<*thb>
\begingroup \makeatletter
\@ifundefined{theorem@style}{\input{theorem.sty}}{}
\ProvidesFile{thb.sty}
             [\FMithmInfo]
%    \end{macrocode}
%
% \begin{macro}{\th@break}
%    |\theoremstyle{break}| produces a line break after the name
%    of the theorem.  The font is |\slshape|.  Hence, we define
%    |\th@break| as follows:
%    \begin{macrocode}
\gdef\th@break{\normalfont\slshape
  \def\@begintheorem##1##2{\item[%
%    \end{macrocode}
%    We run into the following problem: it is not possible to create
%    the header with |\item[|\meta{title}|]| and then start
%    a new line by, for example, |\mbox{}\\|. Such a definition
%    will fail whenever a list environment follows immediately.  With
%    the above construction, the |\mbox{}| causes the switch
%    |@inlabel| (cf.\ definition of |\list| and
%    |\trivlist| in \cite{latex:source}) to be set to \texttt{false}
%    and so the following list will insert additional vertical
%    space (|\topskip|). This is quite annoying. Therefore, we
%    create the line break within the |\item|. In order to ensure
%    that the text will begin at the proper position in the following
%    line, we simply pretend that the label does not take any
%    room.\footnote{This will lead to problems whenever very high
%    symbols occurring in the line tower into the
%    heading. So, something else has to be done here sometime.}
%    \begin{macrocode}
       \rlap{\vbox{\hbox{\hskip \labelsep\theorem@headerfont ##1\ ##2}%
                   \hbox{\strut}}}]}%
%    \end{macrocode}
%    Again, the definition of |\@opargbegintheorem| is
%    completely analogous.
%    \begin{macrocode}
\def\@opargbegintheorem##1##2##3{%
  \item[\rlap{\vbox{\hbox{\hskip \labelsep \theorem@headerfont
                     ##1\ ##2\ (##3)}%
                    \hbox{\strut}}}]}}
\endgroup
%</thb>
%    \end{macrocode}
% \end{macro}
%
%
% \subsubsection{The \texttt{changebreak} style}
%
%    \begin{macrocode}
% This style option is stored in the file |thcb.sty|.
%    \begin{macrocode}
%<*thcb>
\begingroup \makeatletter
\@ifundefined{theorem@style}{\input{theorem.sty}}{}
\ProvidesFile{thcb.sty}
             [\FMithmInfo]
%    \end{macrocode}
%
% \begin{macro}{\th@changebreak}
%    The \texttt{change-break} theorem style is like \texttt{break}
%    but with interchange of theorem name and theorem number.  Thus we
%    define |\th@changebreak| as follows:
%    \begin{macrocode}
\gdef\th@changebreak{\normalfont\slshape
  \def\@begintheorem##1##2{\item
       [\rlap{\vbox{\hbox{\hskip\labelsep \theorem@headerfont ##2\ ##1}%
                    \hbox{\strut}}}]}%
\def\@opargbegintheorem##1##2##3{%
  \item[\rlap{\vbox{\hbox{\hskip\labelsep \theorem@headerfont
                     ##2\ ##1\ (##3)}%
                    \hbox{\strut}}}]}}
\endgroup
%</thcb>
%    \end{macrocode}
% \end{macro}
%
%
% \subsubsection{The \texttt{change} style}
%
% This style option is stored in the file |thc.sty|.
%    \begin{macrocode}
%<*thc>
\begingroup \makeatletter
\@ifundefined{theorem@style}{\input{theorem.sty}}{}
\ProvidesFile{thc.sty}
             [\FMithmInfo]
%    \end{macrocode}
%
% \begin{macro}{\th@change}
%    The \texttt{change} theorem style corresponds to the
%    \texttt{change break} style without a linebreak after the header.
%    To say it in another way, it's the same as the \texttt{plain}
%    style but with number and name interchanged and |\slshape| as the
%    default font.
%    \begin{macrocode}
\gdef\th@change{\normalfont\slshape
  \def\@begintheorem##1##2{\item
       [\hskip\labelsep \theorem@headerfont ##2\ ##1]}%
\def\@opargbegintheorem##1##2##3{%
  \item[\hskip\labelsep \theorem@headerfont ##2\ ##1\ (##3)]}}
\endgroup
%</thc>
%    \end{macrocode}
% \end{macro}
%
%
% \subsubsection{The \texttt{marginbreak} style}
%
% This style option is the one used most often at Mainz.  It is saved
% in the file |thmb.sty|.
%    \begin{macrocode}
%<*thmb>
\begingroup \makeatletter
\@ifundefined{theorem@style}{\input{theorem.sty}}{}
\ProvidesFile{thmb.sty}
             [\FMithmInfo]
%    \end{macrocode}
%
% \begin{macro}{\th@marginbreak}
%    The \texttt{margin break} style is nearly the same as the
%    \texttt{change break} style. The only difference is the placement
%    of the theorem number.  We use |\llap| to place it in the left
%    margin.
%
%    In this style |\labelsep| denotes the separation between the
%    number and the text.
%    \begin{macrocode}
\gdef\th@marginbreak{\normalfont\slshape
  \def\@begintheorem##1##2{\item
       [\rlap{\vbox{\theorem@headerfont
                    \hbox{\llap{##2}\hskip\labelsep ##1}%
                    \hbox{\strut}}}]}%
\def\@opargbegintheorem##1##2##3{%
  \item[\rlap{\vbox{\theorem@headerfont
                    \hbox{\llap{##2}\hskip\labelsep ##1\ (##3)}%
                    \hbox{\strut}}}]}}
\endgroup
%</thmb>
%    \end{macrocode}
% \changes{v2.1c}{90/02/17}{`endgroup outside macrocode env.}
% \end{macro}
%
%
%
% \subsubsection{The \texttt{margin} style}
%
% This style option is stored in the file |thm.sty|.
%    \begin{macrocode}
%<*thm>
\begingroup \makeatletter
\@ifundefined{theorem@style}{\input{theorem.sty}}{}
\ProvidesFile{thm.sty}
             [\FMithmInfo]
%    \end{macrocode}
%
% \begin{macro}{\th@margin}
%    Again this is only a variant of the theorem styles described
%    above without any new ideas.
%    \begin{macrocode}
\gdef\th@margin{\normalfont\slshape
  \def\@begintheorem##1##2{\item
       [\theorem@headerfont \llap{##2}\hskip\labelsep ##1]}%
\def\@opargbegintheorem##1##2##3{%
  \item[\theorem@headerfont \llap{##2}\hskip\labelsep ##1\ (##3)]}}
\endgroup
%</thm>
%    \end{macrocode}
% \end{macro}
%
%
%
%
% \subsection{Final Definitions}
%
% \begin{macro}{\theorempreskipamount}
% \begin{macro}{\theorempostskipamount}
%    The \textsf{skip} parameters that regulate the vertical empty space
%    before and after the {\envfont theorem} environment have
%    to be allocated as well.
% \changes{v2.0f}{89/07/20}{Unnecessary `global' removed.}
%    \begin{macrocode}
%<*package>
\newskip\theorempreskipamount
\newskip\theorempostskipamount
%    \end{macrocode}
%    Since we have used the same values for all theorem sets, we now can
%    assign them.
%    \begin{macrocode}
\global\setlength\theorempreskipamount{12pt plus 5pt minus 3pt}
\global\setlength\theorempostskipamount{8pt plus 3pt minus 1.5pt}
%    \end{macrocode}
% \end{macro}
% \end{macro}
%
%
% \begin{macro}{\@endtheorem}
%    The same holds for the macro |\@endtheorem|, which ends a
%    {\envfont theorem} environment. Since it is the same for all
%    theorem sets, it is removed from the macros |\th@|\meta{style}.
%    It simply ends the {\envfont trivlist} environment, which was
%    begun in |\@thm|.
%    \begin{macrocode}
\global\let\@endtheorem=\endtrivlist
%    \end{macrocode}
% \end{macro}
%
%
%
% \begin{macro}{\@preamblecmds}
%    All macros defined above are to be used only in the preamble.
%    Therefore, we insert them in |\@preamblecmds| which will disable
%    them at begin document. This is done by the internal \LaTeXe{}
%    command |\@onlypreamble|.
% \changes{v2.0g}{89/09/19}{`theoremheaderfont and `theorembodyfont
%    added.}
% \changes{v2.2a}{94/02/03}{Use \LaTeXe{} interface}
%    \begin{macrocode}
\@onlypreamble\@xnthm
\@onlypreamble\@ynthm
\@onlypreamble\@othm
\@onlypreamble\newtheorem
\@onlypreamble\theoremstyle
\@onlypreamble\theorembodyfont
\@onlypreamble\theoremheaderfont
%    \end{macrocode}
% \end{macro}
%
% Finally we declare the \texttt{plain} theorem style to be the default.
% \changes{v2.1b}{90/06/04}{Fixed missing `item by declaring default
%    style.}
%    \begin{macrocode}
\theoremstyle{plain}
%</package>
%    \end{macrocode}
%
% \Finale

\endinput
}{}
\ProvidesFile{thc.sty}
             [\FMithmInfo]
%    \end{macrocode}
%
% \begin{macro}{\th@change}
%    The \texttt{change} theorem style corresponds to the
%    \texttt{change break} style without a linebreak after the header.
%    To say it in another way, it's the same as the \texttt{plain}
%    style but with number and name interchanged and |\slshape| as the
%    default font.
%    \begin{macrocode}
\gdef\th@change{\normalfont\slshape
  \def\@begintheorem##1##2{\item
       [\hskip\labelsep \theorem@headerfont ##2\ ##1]}%
\def\@opargbegintheorem##1##2##3{%
  \item[\hskip\labelsep \theorem@headerfont ##2\ ##1\ (##3)]}}
\endgroup
%</thc>
%    \end{macrocode}
% \end{macro}
%
%
% \subsubsection{The \texttt{marginbreak} style}
%
% This style option is the one used most often at Mainz.  It is saved
% in the file |thmb.sty|.
%    \begin{macrocode}
%<*thmb>
\begingroup \makeatletter
\@ifundefined{theorem@style}{% \iffalse meta-comment
%
% Copyright (C) 1993-2024
%
% The LaTeX Project and any individual authors listed elsewhere
% in this file.
%
% This file is part of the Standard LaTeX `Tools Bundle'.
% -------------------------------------------------------
%
% It may be distributed and/or modified under the
% conditions of the LaTeX Project Public License, either version 1.3c
% of this license or (at your option) any later version.
% The latest version of this license is in
%    https://www.latex-project.org/lppl.txt
% and version 1.3c or later is part of all distributions of LaTeX
% version 2005/12/01 or later.
%
% The list of all files belonging to the LaTeX `Tools Bundle' is
% given in the file `manifest.txt'.
%
% \fi
%
%%
%
%
\def\FMithmInfo{2023/07/05 v2.2c Theorem extension package (FMi)}
%
% \ProvidesFile{theorem.dtx}[\FMithmInfo]
%\iffalse   % this is a METACOMMENT !
%%
%% Package `theorem' to use with LaTeX2e
%% Copyright (C) 1989-1995 Frank Mittelbach, all rights reserved.
%<+package>\@ifundefined{theorem@style}{}{\endinput}
%<+package>\NeedsTeXFormat{LaTeX2e}[1995/06/01]
%<+package>\ProvidesPackage{theorem}
%<+package>         [\FMithmInfo]
%
% \fi
%
% \changes{v1.0}{87/05/07}{First draft.}
% \changes{v1.0a}{87/05/07}{`thistheoremstyle' added.}
% \changes{v1.1a}{87/07/29}{New concept: redefine the internal
%                          \protect\LaTeX{} commands `@xnthm' etc.}
% \changes{v1.1b}{87/07/31}{`theoremfont' used to specify font.}
% \changes{v1.2a}{88/01/08}{`@endtheorem' now globally defined.}
% \changes{v1.2b}{88/01/11}{`newtheorem' only allowed in the preamble.}
% \changes{v1.3a}{88/02/14}{`theoremfont' now a toks register.}
%
% \changes{v2.0a}{88/11/16}{New implementation and new documentation.}
% \changes{v2.0a}{88/11/16}{No layout changes but internal modifications
%                           old substyles won't work.}
% \changes{v2.0b}{89/01/30}{`global' used for all definitions.}
% \changes{v2.0c}{89/06/15}{Documentation translated to English.
%                            Thanks to LLL and Nina}
% \changes{v2.1a}{90/02/15}{Support for new font selection.}
% \changes{v2.2a}{94/02/03}{Cleanup for LaTeX2e}
%
% \DoNotIndex{\@endparenv,\@ifdefinable,\@ifnextchar,\@ifundefined}
% \DoNotIndex{\@namedef,\@nameuse,\@spaces,\@tempa,\@warning}
% \DoNotIndex{\ }
% \DoNotIndex{\begingroup,\bfseries}
% \DoNotIndex{\csname}
% \DoNotIndex{\def,\do,\docdate}
% \DoNotIndex{\endcsname,\endgroup,\endinput,\endtrivlist,\expandafter}
% \DoNotIndex{\filedate,\fileversion}
% \DoNotIndex{\gdef,\global}
% \DoNotIndex{\hbox,\hskip}
% \DoNotIndex{\ignorespaces,\input,\itshape,\item}
% \DoNotIndex{\labelsep,\let,\llap}
% \DoNotIndex{\makeatletter}
% \DoNotIndex{\noexpand}
% \DoNotIndex{\refstepcounter,\relax,\rlap}
% \DoNotIndex{\setlength,\slshape,\space,\string,\strut}
% \DoNotIndex{\the,\trivlist,\typeout}
% \DoNotIndex{\vbox}
% \DoNotIndex{\xdef}
%
% \def\deflabel#1{\ttfamily #1\hfill}
% \def\deflist#1{\begin{list}{}{\settowidth\labelwidth{\ttfamily #1}%
%                               \setlength\leftmargin\labelwidth
%                               \addtolength\leftmargin\labelsep
%                               \let\makelabel\deflabel}}
% \def\enddeflist{\end{list}}
% \def\envfont{\normalfont\ttfamily}
%
% \MakeShortVerb{\|}
%
% \newcommand{\TUB}{{\sl TUGboat\/}}
%
% \GetFileInfo{theorem.dtx}
% \title{An Extension of the
%        \LaTeX{} theorem environment\thanks{This file
%        has version number \fileversion, last
%        revised \filedate.}}
% \author{Frank Mittelbach}
%
% \MaintainedByLaTeXTeam{tools}
% \maketitle
%
% \begin{abstract}
%    The macros described in this paper yield an extension of the
%    \LaTeX{} theorem mechanism. It is designed to satisfy the
%    different requirements of various journals. Thus, the
%    layout of the ``theorems'' can be manipulated by determining a
%    ``style''. This article describes not only the use, but
%    also the definition, of the necessary macros.
% \end{abstract}
%
% \section*{Preface to version 2.2}
%
% For \LaTeXe{} this package did not need any fundamental changes. I
% only modified the messages generated so that theorem layout styles
% will show up with the |\listfiles| command and cleaned the section
% on the New Font Selection Scheme since this is now included in
% \LaTeX.
%
% \section*{Preface to version 2.1}
%
% This version is identical to 2.0g described in \TUB\ 10\#3 except
% for some internal defaults which are now set depending on the used
% font selection scheme.
%
% This was done to avoid unpleasant surprises if the new font selection
% scheme is in force. For further details see
% section~\ref{sec:newfonts} and \cite{bk:GMS94}.
%
% \section{Introduction}
%
% For our purposes here, ``theorems'' are labelled enunciations,
% often set off from the main text by extra space and a font change.
% Theorems, corollaries, conjectures, definitions, and remarks
% are all instances of ``theorems''.  The ``header'' of these structures
% is composed of a label (such as \textsc{Theorem} or \textsc{Remark})
% and a number which serializes an item in the sequence of items
% with the same label.
%
% Shortly after the introduction of \LaTeX{} at the Fachbereich
% Mathematik in Mainz, the desire to manipulate the layout of
% ``theorems'' arose.  In Mainz, the following two conventions came
% into general use:
% \begin{enumerate}
%   \item The number of the theorem is shown in the margin.
%   \item There is a line break at
%         the end of the theorem header.
% \end{enumerate}
%
% Additionally, some journals require different formats which depend on
% the ``sort of theorem'': e.g.\ often remarks and definitions are set
% in |\upshape|, while |\itshape| is employed for main theorems.
%
% Confronted with these requirements, a theorem environment was
% developed in Mainz which allows separate determination of the
% layout of the ``theorems sets'', comparable to |\pagestyle|.
%
%
% \section{The user interface}
% \subsection{Defining new theorem sets}
%
% \DescribeMacro\newtheorem
% As in the original \LaTeX{} version, the command |\newtheorem|
% defines a new ``theorem set'' or ``theorem-like structure''.
% Two required arguments name the new environment and give the text
% to be typeset with each instance of the new ``set'', while
% an optional argument determines how
% the ``set'' is enumerated:
% \begin{description}
%    \item[\ttfamily \bslash newtheorem\{foo\}\{bar\}]
%       The theorem set {\envfont foo} (whose name is \texttt{bar})
%       uses its own counter.
%    \item[\ttfamily \bslash newtheorem\{foo2\}{[foo]}\{bar2\}]
%       The theorem set {\envfont foo2} (printed name \texttt{bar2})
%       uses the same counter as the theorem set \texttt{foo}.
%    \item[\ttfamily \bslash newtheorem\{foo3\}\{bar3\}{[section]}]
%       The theorem set {\envfont foo3} (printed name \texttt{bar3}) is
%       enumerated within the counter \texttt{section}, i.e.\ with every
%       new |\section| the enumeration begins again with 1, and
%       the enumeration is composed from the section-number and the
%       theorem counter itself.
% \end{description}
%
% \DescribeMacro\theoremstyle
% Additionally, the command |\theoremstyle| can define the layout
% of various, or all, theorem sets.  It should be noted that any
% theorem set defined by |\newtheorem| is typeset in the
% |\theoremstyle| that is current at the time of the definition.
% Thus, the following
% \begin{verbatim}
%    \theoremstyle{break}        \newtheorem{Cor}{Corollary}
%    \theoremstyle{plain}        \newtheorem{Exa}{Example}[section]
%\end{verbatim}                             ^^A No blank at line start
% leads to the result that the set {\envfont Cor} is formatted in the
% style \texttt{break}, while the set {\envfont Exa} and all the
% following ones are formatted in the style \texttt{plain}, unless
% another |\theoremstyle| follows.  Since the definitions installed by
% |\newtheorem| are global, one also can limit |\theoremstyle| locally
% by grouping braces.
%
% \DescribeMacro\theorembodyfont
% The choice of the font for the theorem body is completely
% independent of the chosen |\theoremstyle|; this has proven to
% be very advantageous.  For example,
% \begin{verbatim}
%   {\theorembodyfont{\upshape}          \newtheorem{Rem}{Remark}}
%\end{verbatim}
% defines a theorem set {\envfont Rem}, which will be set in
% |\upshape| in the current layout (which in our example is
% \texttt{plain}).  As with |\theoremstyle|, the |\theorembodyfont|
% chosen is that current at the time of |\newtheorem|.  If
% |\theorembodyfont| is not specified or one defines
% |\theorembodyfont{}|, then the font used will be that defined by the
% |\theoremstyle|.
%
% \DescribeMacro\theoremheaderfont It is also possible to customize
% the font used for the theorem headers.  This is, however, a global
% declaration, and therefore there should be at most one
% |\theoremheaderfont| declaration in the preamble.\footnote{If it is
% actually necessary to have different header fonts, one has to define
% new theorem styles (substituting the desired font) or specify the
% information directly in the \texttt{\bslash newtheorem} declaration
% (the unclean variant).}
%
% \DescribeMacro\theorempreskipamount
% \DescribeMacro\theorempostskipamount Two additional parameters
% affect the vertical space around the theorem environments:
% |\theorempreskipamount| and |\theorempostskipamount| define,
% respectively, the spacing before and after such an environment.
% These parameters apply for all theorem sets and can be manipulated
% with the ordinary length macros.  They are rubber lengths
% (`\textsf{skips}'), and therefore can contain \texttt{plus} and
% \texttt{minus} parts.
%
% Since the definition of theorem sets should---most sensibly---be
% placed in the preamble, we only allow installation there.
% It is therefore possible to release the memory used here after
% |\begin{document}|, in order to make room for other
% applications.
%
%
% \subsection{Existing theorem styles}
%
% The following theorem styles exist to date:
% \begin{deflist}{marginbreak}
%    \item[plain]
%       This theorem style emulates the original \LaTeX{} definition,
%       except that additionally the parameters
%       |\theorem...skipamount| are used.
%    \item[break]
%       In this style, the theorem header is followed by a line
%       break.
%    \item[marginbreak]
%       The theorem number is set in the margin, and there is a line
%       break as in \texttt{break}.
%    \item[changebreak]
%       Like \texttt{break}, but with header number and text
%       interchanged.
%    \item[change]
%       Header number and text are interchanged, without a line break.
%    \item[margin]
%       The number is set in the left margin, without a line break.
% \end{deflist}
% All styles (except \texttt{plain}) select |\slshape| as the default
% |\theorembodyfont|.
%
%
% \subsection{Examples}
%
% Given the above theorem sets {\envfont Cor}, {\envfont Exa}
% and {\envfont Rem},
% suppose that the preamble also contains the declarations:
% \begin{verbatim}
%    \theoremstyle{marginbreak}   \newtheorem{Lem}[Cor]{Lemma}
%    \theoremstyle{change}
%    \theorembodyfont{\itshape}        \newtheorem{Def}[Cor]{Definition}
%
%    \theoremheaderfont{\scshape}
%\end{verbatim}
% Then the following are some typical examples of the typeset output
% resulting from their use.
%
% \begin{Cor}
%    This is a sentence typeset in the theorem environment {\envfont
%    Cor}.
% \end{Cor}
%
% \begin{Exa}
%    This is a sentence typeset in the theorem environment {\envfont
%    Exa}.
% \end{Exa}
%
% \begin{Rem}
%    This is a sentence typeset in the theorem environment {\envfont
%    Rem}.
% \end{Rem}
%
% \begin{Lem}[Ben User]
%    This is a sentence typeset in the theorem environment {\envfont
%    Lem}.
% \end{Lem}
%
% \begin{Def}[Very impressive Definition]
%    This is a sentence typeset in the theorem environment {\envfont
%    Def}.
% \end{Def}
%
% The last two examples show the effect of the optional argument to
% a theorem environment (it is the text typeset in parentheses).
%
%
% \section{Special Considerations}
% \label{sec:newfonts}
%
% Theoremheader and body are implemented as a unit. This means that
% the |\theoremheaderfont|  will inherit characteristics of the
% |\theorembodyfont| in \LaTeXe. Thus, if for example
% |\theorembodyfont| is |\itshape| and |\theoremheaderfont|
% is |\bfseries| the font selected for the header will have the
% characteristics `bold extended italic'. If this is not desired one
% should set the |\theoremheaderfont| to something like
% \begin{verbatim}
%      \theoremheaderfont{\normalfont\bfseries}
%\end{verbatim}
% i.e.\ supplying all necessary font information explicitly.
%
%
%
% \section{Acknowledgements}
%
% The publication of this set of macros was only possible with the
% help of Christina Busse (translating the manuscript into English),
% Joachim Pense (playing the r\^ole of typist), Chris Rowley
% (looking everything over) and many others providing useful
% suggestions.
%
% \MaybeStop{%
%    \begin{thebibliography}{9}
%       \bibitem{bk:GMS94} \textsc{M.~Goossens}, \textsc{F.~Mittelbach}
%         and \textsc{A.~Samarin}.
%         \newblock The \LaTeX{} Companion.
%         \newblock
%         Addison-Wesley, Reading, Massachusetts, 1994.
%       \bibitem{latex:source} \textsc{Lamport, Leslie.} \newblock
%                               \texttt{latex.tex}, version 2.09, date
%                               Feb. 1990.
%    \end{thebibliography}
%    \PrintIndex
%    \PrintChanges}
%
%
% \section{The documentation driver file}
%
% The next bit of code contains the documentation driver file for
% \TeX{}, i.e., the file that will produce the documentation you are
% currently reading. It will be extracted from this file by the
% \texttt{docstrip} program. Since it is the first code in the file
% one can alternatively process this file directly with \LaTeXe{} to
% obtain the documentation.
%
%    \begin{macrocode}
%<*driver>
\documentclass{ltxdoc}

\usepackage{theorem}

%<+driver>% The next few lines define theorem sets which are used
%<+driver>% in the example section of the documentation.

\theoremstyle{break}       \newtheorem{Cor}{Corollary}
\theoremstyle{plain}       \newtheorem{Exa}{Example}[section]
{\theorembodyfont{\upshape}\newtheorem{Rem}{Remark}}
\theoremstyle{marginbreak} \newtheorem{Lem}[Cor]{Lemma}
\theoremstyle{change}
\theorembodyfont{\itshape} \newtheorem{Def}[Cor]{Definition}

\theoremheaderfont{\scshape}

\RecordChanges

\begin{document}
   \DocInput{theorem.dtx}
\end{document}
%</driver>
%    \end{macrocode}
%
% \section{Definition of the Macros}
%
% If the file has been loaded before, we abort immediately. If not the
% package announces itself (this is actually done at the very top of
% the file---the way it is done isn't good style so don't copy it).
% \changes{v2.0e}{89/07/19}{Spaces removed from `typeout'.}
% \changes{v2.2b}{95/11/19}{Announce moved to top of file}
%    \begin{macrocode}
%<*package>
%\@ifundefined{theorem@style}{}{\endinput}
%\def\FMithmInfo{1995/11/19 v2.2b Theorem extension package (FMi)}
%\ProvidesPackage{theorem}[\FMithmInfo]
%    \end{macrocode}
%
%
% \subsection{Definition of theorem styles and fonts}
%
% All the definitions in this file are done globally to allow
% inputting this file inside a group.
%
% \begin{macro}{\theoremstyle}
%    Before a theorem style can be installed, the chosen style must be
%    known. For that reason, we must test to see that
%    |\th@|\meta{style} is known or, more precisely, that it is
%    different from |\relax|. If the style is not known then
%    |\th@plain| is used.
%    \begin{macrocode}
\gdef\theoremstyle#1{%
   \@ifundefined{th@#1}{\@warning
          {Unknown theoremstyle `#1'. Using `plain'}%
          \theorem@style{plain}}%
%    \end{macrocode}
%    We save the theorem style to be used in the token register
%    |\theorem@style|.
%    \begin{macrocode}
      {\theorem@style{#1}}%
%    \end{macrocode}
%    Now we ``evaluate'' the theorem style: this means, we call the
%    macro |\th@|\meta{style} which will activate the relevant
%    definitions which are contained in a separate file.  This is done
%    in a group to suppress changes to the current font.  This could
%    otherwise pose problems together with the new font selection
%    scheme\footnote{When I printed the original article using the new
%    font selection scheme I ended with a document with slanted
%    typefaces (text headings and all) simply because one of the
%    theorem styles used \texttt{\bslash sl} at toplevel.} if the
%    |\th@|\meta{style} is evaluated a second time.
% \changes{v2.1a}{90/02/15}{Group added around `th@..}
% \changes{v2.1a}{90/02/15}{Reseting of `..@begintheorem removed.}
%    \begin{macrocode}
      \begingroup
        \csname th@\the\theorem@style \endcsname
      \endgroup}
%    \end{macrocode}
% \end{macro}
%
%
% \begin{macro}{\@begintheorem}
% \changes{v2.1a}{90/02/15}{Relaxed at toplevel.}
% \begin{macro}{\@opargbegintheorem}
% \changes{v2.1a}{90/02/15}{Relaxed at toplevel.}
%    We reset |\@begintheorem| and |\@opargbegintheorem| to
%    |\relax| since these commands are no longer necessary at
%    toplevel. This will save a few tokens.
%    \begin{macrocode}
\global\let\@begintheorem\relax
\global\let\@opargbegintheorem\relax
%    \end{macrocode}
% \end{macro}
% \end{macro}
%
% \begin{macro}{\theorem@style}
%    Obviously the \textsf{token} register used above has to be
%    allocated. To assure the utmost compatibility with the
%    original \LaTeX{} definition, we set the default theorem style to
%    \texttt{plain}, which implements the usual \LaTeX{} convention.
% \changes{v2.0f}{89/07/20}{Unnecessary `global' removed.}
%    \begin{macrocode}
\newtoks\theorem@style
\global\theorem@style{plain}
%    \end{macrocode}
% \end{macro}
%
% \begin{macro}{\theorembodyfont}
% \begin{macro}{\theorem@bodyfont}
%    If the |\theorembodyfont| is set by the user then it should not
%    interact with the default font set in the theorem style. When the
%    new font selection is in force this may happen if, for example,
%    the default is |\itshape| and the new |\theorembodyfont| is
%    |\sffamily|. So we add a |\reset@font| command in front of the
%    user definition.
% \changes{v2.0f}{89/07/20}{Unnecessary `global' removed.}
% \changes{v2.0g}{89/09/19}{`theoremfont to `theorembodyfont.}
% \changes{v2.1c}{92/06/22}{`reset@font added and logic changed.}
% \changes{v2.1d}{92/10/28}{Reset to empty if arg is empty}
%    \begin{macrocode}
\gdef\theorembodyfont#1{%
%    \end{macrocode}
%    We check if the argument supplied is empty and if so put nothing
%    into the |\theorem@bodyfont| token register to allow for
%    |\theorembodyfont{}| as a mean of using the default of the
%    current |\theoremstyle|.
%    \begin{macrocode}
   \def\@tempa{#1}%
   \ifx\@tempa\@empty
    \theorem@bodyfont{}%
   \else
    \theorem@bodyfont{\reset@font#1}%
   \fi
   }
\newtoks\theorem@bodyfont
\global\theorem@bodyfont{}
%    \end{macrocode}
% \end{macro}
% \end{macro}
%
%
% \begin{macro}{\theoremheaderfont}
% \changes{v2.0g}{89/09/19}{Macro added.}
%    The font for the theorem headers is handled differently because
%    this definition applies to all theorem styles.
%    \begin{macrocode}
\gdef\theoremheaderfont#1{\gdef\theorem@headerfont{#1}%
%    \end{macrocode}
%    After using the macro once it is redefined to produce an error
%    message.
%    \begin{macrocode}
       \gdef\theoremheaderfont##1{%
        \typeout{\string\theoremheaderfont\space should be used
                 only once.}}}
%    \end{macrocode}
% \end{macro}
%
% \begin{macro}{\theorem@headerfont}
% \changes{v2.0g}{89/09/19}{Macro added.}
%    To set the |\theorem@headerfont| default we first test if
%    the new fontselection scheme is in force.
% \changes{v2.1a}{90/02/15}{Support for new font selection.}
%    \begin{macrocode}
\ifx\upshape\undefined
%    \end{macrocode}
%    If not we define it to expand into |\bfseries|. We don't use
%    |\let| just in case a following style option redefines this
%    macro.
%    \begin{macrocode}
\gdef\theorem@headerfont{\bfseries}
%    \end{macrocode}
%    Otherwise we reset the current shape before calling |\bfseries|.
%    \begin{macrocode}
\else \gdef\theorem@headerfont{\normalfont\bfseries}\fi
%    \end{macrocode}
% \end{macro}
%
% \begin{macro}{\th@plain}
% \begin{macro}{\th@break}
% \begin{macro}{\th@marginbreak}
% \begin{macro}{\th@changebreak}
% \begin{macro}{\th@change}
% \begin{macro}{\th@margin}
%    The different styles are defined in macros such as
%    |\th@plain|.  Since memory space is precious in
%    ``non-Big-versions'', we have to avoid offering too many unused
%    definitions.  Therefore we define these styles in separate files
%    that can be loaded on demand.  Thus the commands themselves only
%    load these files. We use |\@input@|, a \LaTeXe{} internal command
%    that ensures that the file will be listed with |\listfiles|.
% \changes{v2.2a}{94/02/02}{Use `@input@ to load theorem layout files}
%    \begin{macrocode}
\gdef\th@plain{\@input@{thp.sty}}
\gdef\th@break{\@input@{thb.sty}}
\gdef\th@marginbreak{\@input@{thmb.sty}}
\gdef\th@changebreak{\@input@{thcb.sty}}
\gdef\th@change{\@input@{thc.sty}}
\gdef\th@margin{\@input@{thm.sty}}
%    \end{macrocode}
%    This list will be expanded when new styles become
%    available.  For testing,  just append new theorem substyles
%    as document options.
% \end{macro}
% \end{macro}
% \end{macro}
% \end{macro}
% \end{macro}
% \end{macro}
%
%
%
% \subsection{Definition of a new theorem set}
%
% As already pointed out, a new theorem environment can be defined in
% three different ways:
% \begin{verbatim}
%    \newtheorem{Lem}{Lemma}
%    \newtheorem{Lem}{Lemma}[section]
%    \newtheorem{Lem}[Theorem]{Lemma}
%\end{verbatim}
% The function of the macro |\newtheorem| is to recognize these cases
% and then to branch into one of the three macros |\@ynthm|, |\@xnthm|
% or |\@othm|.  This mechanism is adopted unchanged from
% \cite{latex:source}; the essential point here is that, for example,
% in the second case, the arguments \texttt{Lem}, \texttt{Lemma} and
% \texttt{section} are passed over to the macro |\@xnthm|.
%
% We inspect this case first because the others present fewer problems,
% and thus are easily derived from this one.
%
% \begin{macro}{\@xnthm}
%    For our example arguments, the macro |\@xnthm| must fulfill
%    the following:
%    \begin{itemize}
%       \item
%          Define a new \LaTeX-counter `Lem'
%       \item
%          reset this counter within a |\section|
%       \item
%          define the macro |\theLem|
%       \item
%          define the environment macros |\Lem| and
%          |\endLem| using the current
%          |\theoremstyle| and |\theorem@bodyfont|.
%    \end{itemize}
%    Obviously, all this should happen only if the first argument
%    of |\@xnthm| (i.e.\ \texttt{Lem} in our example) is chosen so
%    as not to conflict with any previously defined commands or
%    environments.  This test is performed by the \LaTeX{} macro
%    |\@ifdefinable|.
%    \begin{macrocode}
\gdef\@xnthm#1#2[#3]{\expandafter\@ifdefinable\csname #1\endcsname
%    \end{macrocode}
%    Therefore, the first argument of |\@ifdefinable| is the
%    expansion (in the example, |\Lem|) of
%    |\csname||#1\endcsname|.  The second argument is
%    executed only if the test has been completed successfully.
%    \begin{macrocode}
   {%
%    \end{macrocode}
%    Now we define the new counter. The names of the \LaTeX{}
%    macros employed should speak for themselves:
% \changes{v2.2b}{1995/10/07}{Use \cs{@newctr}}
% \changes{v2.2c}{1995/11/23}{Use \cs{@newctr} properly.}
%    \begin{macrocode}
    \@definecounter{#1}\@newctr{#1}[#3]%
%    \end{macrocode}
%    Using |\@newctr| will give a proper error message if the counter
%    in |#3| is not defined.
%    In defining `|\theLem|' we must generate the desired macro
%    name by use of |\expandafter| and |\csname|.
%    \begin{macrocode}
    \expandafter\xdef\csname the#1\endcsname
%    \end{macrocode}
%    An |\xdef| is used in order to make the definition global,
%    and to ensure that it contains the replacement texts of
%    |\@thmcountersep| and |\@thmcounter|.\footnote{These
%    two macros can be defined by the document style.  Their default
%    values produce a `\texttt{.}' as separation and an arabic
%    representation of the number.} However, not everything should be
%    expanded. For example, it saves space to use |\thesection|
%    instead of its---at times---lengthy expansion.
%    \begin{macrocode}
      {\expandafter \noexpand \csname the#3\endcsname
       \@thmcountersep \@thmcounter{#1}}%
%    \end{macrocode}
%    Thus with the defaults of \LaTeX{}, |\theLem| would be
%    replaced by the command sequence |\thesection.\arabic{Lem}|.
%
%    We will now look at the definition of the macro which is executed
%    at the beginning of the actual environment (in our example this
%    macro is |\Lem|). It should be noted that we use an
%    ``|\expandafter| trick'' to expand only certain parts of the
%    replacement text at the time of the definition.
%    \begin{macrocode}
    \def\@tempa{\global\@namedef{#1}}%
    \expandafter \@tempa \expandafter{%
%    \end{macrocode}
%    First, the macro that contains the current definitions of
%    |\@begintheorem| and |\@opargtheorem| should be called
%    up.  The name of this macro---as is already known---has the form
%    |\th@|\meta{theorem\ style}; therefore, it must be called by
%    \begin{macrocode}
      \csname th@\the \theorem@style
%    \end{macrocode}
%    In addition the default theorem font should be changeable, i.e.\
%    we have to insert the contents of |\theorem@bodyfont|. For
%    that reason, we expand even further, beyond |\endcsname|,
%    and thus insert the contents of the \textsf{token} register
%    |\theorem@bodyfont| in the replacement text.
%    \begin{macrocode}
            \expandafter \endcsname \the \theorem@bodyfont
%    \end{macrocode}
%    Now it is time to call the macro |\@thm| which takes over
%    the further processing.  It has two arguments: the current
%    counter name (in our example, \texttt{Lem}), and the text of the
%    label (in our example, \texttt{Lemma}).
%    \begin{macrocode}
     \@thm{#1}{#2}}%
%    \end{macrocode}
%    With this, the `sub-definition' is complete.  The macro
%    |\@endtheorem| ends a theorem environment and is, so far, nothing
%    but an |\endtrivlist|.  (Hence it is defined globally, and not
%    within the theorem styles.\footnote{This has to be changed as
%    soon as theorem styles that change \texttt{\bslash @endtheorem}
%    exist.  In such a case, all existing styles must be changed as
%    well since they will have to reset the macro.}) Therefore, we can
%    set it equivalent to the macro that ends the theorem set (in our
%    example, |\endLem|).  However, if some day theorem styles exist
%    that do change |\@endtheorem|, we would have to use the
%    commented-out line instead.
%    \begin{macrocode}
    \global \expandafter \let \csname end#1\endcsname \@endtheorem
%   \global\@namedef{end#1}{\@endtheorem}%
%    \end{macrocode}
%    With these commands all the required definitions are employed,
%    unless the test |\@ifdefinable| has failed.  Therefore, we
%    end the second argument of this macro and with it the definition
%    of |\@xnthm|.
%    \begin{macrocode}
   }}
%    \end{macrocode}
% \end{macro}
%
% \begin{macro}{\@ynthm}
%    The definition of |\@ynthm| is completely analogous. In
%    this case the new counter that is defined is not reset within
%    another counter; thus the definition of |\the...|
%    is simplified:
%    \begin{macrocode}
\gdef\@ynthm#1#2{\expandafter\@ifdefinable\csname #1\endcsname
   {\@definecounter{#1}%
    \expandafter\xdef\csname the#1\endcsname{\@thmcounter{#1}}%
%    \end{macrocode}
%    The rest of the definition corresponds literally to that of
%    |\@xnthm|:
%    \begin{macrocode}
    \def\@tempa{\global\@namedef{#1}}\expandafter \@tempa
     \expandafter{\csname th@\the \theorem@style \expandafter
     \endcsname \the\theorem@bodyfont \@thm{#1}{#2}}%
    \global \expandafter \let \csname end#1\endcsname \@endtheorem}}
%    \end{macrocode}
% \end{macro}
%
% \begin{macro}{\@othm}
%    The definition of |\@othm| does not contain anything new.
%    \begin{macrocode}
\gdef\@othm#1[#2]#3{%
%    \end{macrocode}
% \changes{v2.2b}{95/10/07}{Build in check for counter defined}
%    We do not define a new counter but instead use one that has
%    already been defined.  Thus the only definition we need is that
%    of this pseudo-counter (i.e.\ |\the|\meta{env.\ name}).
%    First we check if |#2| corresponds to a known counter name.
%    \begin{macrocode}
  \expandafter\ifx\csname c@#2\endcsname\relax
   \@nocounterr{#2}%
  \else
   \expandafter\@ifdefinable\csname #1\endcsname
   {\expandafter \xdef \csname the#1\endcsname
     {\expandafter \noexpand \csname the#2\endcsname}%
%    \end{macrocode}
%    All other parts of the definition can be adopted from
%    |\@xnthm|. We have to remember, though, that in this case
%    the name of the current counter and the theorem label have moved
%    to the second and third arguments.
%    \begin{macrocode}
    \def\@tempa{\global\@namedef{#1}}\expandafter \@tempa
     \expandafter{\csname th@\the \theorem@style \expandafter
     \endcsname \the\theorem@bodyfont \@thm{#2}{#3}}%
    \global \expandafter \let \csname end#1\endcsname \@endtheorem}%
  \fi}
%    \end{macrocode}
% \end{macro}
%
% \subsection{Macros that are employed in a {\envfont theorem}
%    environment}
%
% \begin{macro}{\@thm}
%    The macro |\@thm| has to increase the current counter.  Then,
%    depending on whether the environment has (or does not have) an
%    optional argument, it has to branch into either |\@begintheorem|
%    or |\@opargtheorem|.
%    \begin{macrocode}
\gdef\@thm#1#2{\refstepcounter{#1}%
%    \end{macrocode}
%    Now we start a {\envfont trivlist} environment, and give
%    |\@topsep| and |\@topsepadd| the values of the \textsf{skip}
%    registers |\theorempreskipamount| and |\theorempostskipamount|.
%    The value in |\@topsep| is the vertical space that is inserted by
%    the first (and only) |\item| in our |\trivlist| whilst
%    |\@topsepadd| is inserted by |\@endparenv| at the end of that
%    {\envfont trivlist} environment.  By using these registers, we
%    obtain the desired space around a {\envfont theorem} environment.
%    \begin{macrocode}
   \trivlist
   \@topsep \theorempreskipamount               % used by first \item
   \@topsepadd \theorempostskipamount           % used by \@endparenv
%    \end{macrocode}
%    Now we have to test whether an optional argument has been given.
% \changes{v2.0d}{89/07/18}{Forgotten percent added.}
%    \begin{macrocode}
   \@ifnextchar [%
%    \end{macrocode}
%    If there is an optional argument, we will call |\@ythm|, and
%    move the arguments read back into the input stream.
%    \begin{macrocode}
   {\@ythm{#1}{#2}}%
%    \end{macrocode}
%    If not, we call |\@begintheorem|. Its first argument is the
%    name of the theorem set (hence the second argument of
%    |\@thm|). Its second argument is the macro that produces the
%    current number.
%    \begin{macrocode}
   {\@begintheorem{#2}{\csname the#1\endcsname}\ignorespaces}}
%    \end{macrocode}
% \end{macro}
%
% \begin{macro}{\@xthm}
% \begin{macro}{\@ythm}
%    Both these macros were originally called by |\@thm|.  We do
%    not need |\@xthm| anymore, hence we reset it to
%    |\relax|. The definition of |\@ythm| has not changed at
%    all from its definition in \LaTeX. In order to make the macros
%    easier to understand, we will
%    nevertheless present it (commented out).
%    \begin{macrocode}
\global\let\@xthm\relax
% \def\@ythm#1#2[#3]{\@opargbegintheorem{#2}{\csname
%           the#1\endcsname}{#3}\ignorespaces}
%</package>
%    \end{macrocode}
%    The primitive |\ignorespaces| in |\@ythm| and
%    |\@thm| is needed to remove the spaces between the
%    |\begin{...}| and the actual text.
% \end{macro}
% \end{macro}
%
%
% \subsection{Definition of the theorem substyles}
%
% As already pointed out, the theorem substyles, defined below, are
% only loaded when necessary.  Note that all these substyles, except
% \texttt{plain}, have |\slshape| as the default body font.
%
% \subsubsection{The \texttt{plain} style}
%
% \label{dok:th@plain} As the following macros use |@|, we have
% to locally set the |\catcode| of this symbol to ``letter''.
% This happens within a group, so that we do not have to worry about
% which |\catcode| that symbol had before.
%    \begin{macrocode}
%<*thp>
\begingroup \makeatletter
%    \end{macrocode}
% Since we are now within a group, we must make all definitions
% globally.  First we make sure that \texttt{theorem.sty} is loaded.
% This will allow us to use this file as a document style option
% without having to call \texttt{theorem} itself as an option.  At the
% same time, we assure that at least version 2 is loaded, since
% |\theorem@style| was not defined in earlier versions.
%    \begin{macrocode}
\@ifundefined{theorem@style}{\input{theorem.sty}}{}
\ProvidesFile{thp.sty}
             [\FMithmInfo]
%    \end{macrocode}
%
%
% \begin{macro}{\th@plain}
% \begin{macro}{\@begintheorem}
% \begin{macro}{\@opargbegintheorem}
%    |\theoremstyle{plain}| corresponds to the original
%    definition, except that the distances to the surrounding text are
%    determined by the parameters |\theorempreskipamount| and
%    |\theorempostskipamount|.
%    First we set the default body font.
%    \begin{macrocode}
\gdef\th@plain{\normalfont\itshape
%    \end{macrocode}
%    Then we define |\@begintheorem| and
%    |\@opargbegintheorem|. These two macros define how the
%    header of a theorem is typeset.  |\@opargbegintheorem| will
%    be called if a \textsf{theorem} environment with an optional
%    argument is encountered; otherwise, the header is constructed by
%    calling |\@begintheorem|.  If one of these macros is
%    executed, we are within a \textsf{trivlist} environment started by
%    |\@thm|.  So the theorem header is produced with an
%    |\item| command.
%
%    Instead of specifying the header font directly, all standard
%    theorem styles use the |\theorem@headerfont| macro to allow
%    customization.  The extra space (|\labelsep|) is necessary
%    because of problems in the \textsf{trivlist} environment.
%    \begin{macrocode}
  \def\@begintheorem##1##2{%
        \item[\hskip\labelsep \theorem@headerfont ##1\ ##2]}%
%    \end{macrocode}
%    The definition of |\@opargbegintheorem| is completely
%    analogous. The only difference is the fact that there exists a
%    third argument (which is the optional parameter of the
%    environment and contains additional information about the
%    theorem).  Customarily we enclose it in parentheses.
%    \begin{macrocode}
\def\@opargbegintheorem##1##2##3{%
   \item[\hskip\labelsep \theorem@headerfont ##1\ ##2\ (##3)]}}
%    \end{macrocode}
% \end{macro}
% \end{macro}
% \end{macro}
%
% \noindent
% We conclude with an |\endgroup| to restore the
% |\catcode| of |@|.
%    \begin{macrocode}
\endgroup
%</thp>
%    \end{macrocode}
%
%
% \subsubsection{The \texttt{break} style}
%
% This style option is stored in the file |thb.sty|.
% For the next two lines see the documentation for |\th@plain| on
% page~\pageref{dok:th@plain}.
%    \begin{macrocode}
%<*thb>
\begingroup \makeatletter
\@ifundefined{theorem@style}{\input{theorem.sty}}{}
\ProvidesFile{thb.sty}
             [\FMithmInfo]
%    \end{macrocode}
%
% \begin{macro}{\th@break}
%    |\theoremstyle{break}| produces a line break after the name
%    of the theorem.  The font is |\slshape|.  Hence, we define
%    |\th@break| as follows:
%    \begin{macrocode}
\gdef\th@break{\normalfont\slshape
  \def\@begintheorem##1##2{\item[%
%    \end{macrocode}
%    We run into the following problem: it is not possible to create
%    the header with |\item[|\meta{title}|]| and then start
%    a new line by, for example, |\mbox{}\\|. Such a definition
%    will fail whenever a list environment follows immediately.  With
%    the above construction, the |\mbox{}| causes the switch
%    |@inlabel| (cf.\ definition of |\list| and
%    |\trivlist| in \cite{latex:source}) to be set to \texttt{false}
%    and so the following list will insert additional vertical
%    space (|\topskip|). This is quite annoying. Therefore, we
%    create the line break within the |\item|. In order to ensure
%    that the text will begin at the proper position in the following
%    line, we simply pretend that the label does not take any
%    room.\footnote{This will lead to problems whenever very high
%    symbols occurring in the line tower into the
%    heading. So, something else has to be done here sometime.}
%    \begin{macrocode}
       \rlap{\vbox{\hbox{\hskip \labelsep\theorem@headerfont ##1\ ##2}%
                   \hbox{\strut}}}]}%
%    \end{macrocode}
%    Again, the definition of |\@opargbegintheorem| is
%    completely analogous.
%    \begin{macrocode}
\def\@opargbegintheorem##1##2##3{%
  \item[\rlap{\vbox{\hbox{\hskip \labelsep \theorem@headerfont
                     ##1\ ##2\ (##3)}%
                    \hbox{\strut}}}]}}
\endgroup
%</thb>
%    \end{macrocode}
% \end{macro}
%
%
% \subsubsection{The \texttt{changebreak} style}
%
%    \begin{macrocode}
% This style option is stored in the file |thcb.sty|.
%    \begin{macrocode}
%<*thcb>
\begingroup \makeatletter
\@ifundefined{theorem@style}{\input{theorem.sty}}{}
\ProvidesFile{thcb.sty}
             [\FMithmInfo]
%    \end{macrocode}
%
% \begin{macro}{\th@changebreak}
%    The \texttt{change-break} theorem style is like \texttt{break}
%    but with interchange of theorem name and theorem number.  Thus we
%    define |\th@changebreak| as follows:
%    \begin{macrocode}
\gdef\th@changebreak{\normalfont\slshape
  \def\@begintheorem##1##2{\item
       [\rlap{\vbox{\hbox{\hskip\labelsep \theorem@headerfont ##2\ ##1}%
                    \hbox{\strut}}}]}%
\def\@opargbegintheorem##1##2##3{%
  \item[\rlap{\vbox{\hbox{\hskip\labelsep \theorem@headerfont
                     ##2\ ##1\ (##3)}%
                    \hbox{\strut}}}]}}
\endgroup
%</thcb>
%    \end{macrocode}
% \end{macro}
%
%
% \subsubsection{The \texttt{change} style}
%
% This style option is stored in the file |thc.sty|.
%    \begin{macrocode}
%<*thc>
\begingroup \makeatletter
\@ifundefined{theorem@style}{\input{theorem.sty}}{}
\ProvidesFile{thc.sty}
             [\FMithmInfo]
%    \end{macrocode}
%
% \begin{macro}{\th@change}
%    The \texttt{change} theorem style corresponds to the
%    \texttt{change break} style without a linebreak after the header.
%    To say it in another way, it's the same as the \texttt{plain}
%    style but with number and name interchanged and |\slshape| as the
%    default font.
%    \begin{macrocode}
\gdef\th@change{\normalfont\slshape
  \def\@begintheorem##1##2{\item
       [\hskip\labelsep \theorem@headerfont ##2\ ##1]}%
\def\@opargbegintheorem##1##2##3{%
  \item[\hskip\labelsep \theorem@headerfont ##2\ ##1\ (##3)]}}
\endgroup
%</thc>
%    \end{macrocode}
% \end{macro}
%
%
% \subsubsection{The \texttt{marginbreak} style}
%
% This style option is the one used most often at Mainz.  It is saved
% in the file |thmb.sty|.
%    \begin{macrocode}
%<*thmb>
\begingroup \makeatletter
\@ifundefined{theorem@style}{\input{theorem.sty}}{}
\ProvidesFile{thmb.sty}
             [\FMithmInfo]
%    \end{macrocode}
%
% \begin{macro}{\th@marginbreak}
%    The \texttt{margin break} style is nearly the same as the
%    \texttt{change break} style. The only difference is the placement
%    of the theorem number.  We use |\llap| to place it in the left
%    margin.
%
%    In this style |\labelsep| denotes the separation between the
%    number and the text.
%    \begin{macrocode}
\gdef\th@marginbreak{\normalfont\slshape
  \def\@begintheorem##1##2{\item
       [\rlap{\vbox{\theorem@headerfont
                    \hbox{\llap{##2}\hskip\labelsep ##1}%
                    \hbox{\strut}}}]}%
\def\@opargbegintheorem##1##2##3{%
  \item[\rlap{\vbox{\theorem@headerfont
                    \hbox{\llap{##2}\hskip\labelsep ##1\ (##3)}%
                    \hbox{\strut}}}]}}
\endgroup
%</thmb>
%    \end{macrocode}
% \changes{v2.1c}{90/02/17}{`endgroup outside macrocode env.}
% \end{macro}
%
%
%
% \subsubsection{The \texttt{margin} style}
%
% This style option is stored in the file |thm.sty|.
%    \begin{macrocode}
%<*thm>
\begingroup \makeatletter
\@ifundefined{theorem@style}{\input{theorem.sty}}{}
\ProvidesFile{thm.sty}
             [\FMithmInfo]
%    \end{macrocode}
%
% \begin{macro}{\th@margin}
%    Again this is only a variant of the theorem styles described
%    above without any new ideas.
%    \begin{macrocode}
\gdef\th@margin{\normalfont\slshape
  \def\@begintheorem##1##2{\item
       [\theorem@headerfont \llap{##2}\hskip\labelsep ##1]}%
\def\@opargbegintheorem##1##2##3{%
  \item[\theorem@headerfont \llap{##2}\hskip\labelsep ##1\ (##3)]}}
\endgroup
%</thm>
%    \end{macrocode}
% \end{macro}
%
%
%
%
% \subsection{Final Definitions}
%
% \begin{macro}{\theorempreskipamount}
% \begin{macro}{\theorempostskipamount}
%    The \textsf{skip} parameters that regulate the vertical empty space
%    before and after the {\envfont theorem} environment have
%    to be allocated as well.
% \changes{v2.0f}{89/07/20}{Unnecessary `global' removed.}
%    \begin{macrocode}
%<*package>
\newskip\theorempreskipamount
\newskip\theorempostskipamount
%    \end{macrocode}
%    Since we have used the same values for all theorem sets, we now can
%    assign them.
%    \begin{macrocode}
\global\setlength\theorempreskipamount{12pt plus 5pt minus 3pt}
\global\setlength\theorempostskipamount{8pt plus 3pt minus 1.5pt}
%    \end{macrocode}
% \end{macro}
% \end{macro}
%
%
% \begin{macro}{\@endtheorem}
%    The same holds for the macro |\@endtheorem|, which ends a
%    {\envfont theorem} environment. Since it is the same for all
%    theorem sets, it is removed from the macros |\th@|\meta{style}.
%    It simply ends the {\envfont trivlist} environment, which was
%    begun in |\@thm|.
%    \begin{macrocode}
\global\let\@endtheorem=\endtrivlist
%    \end{macrocode}
% \end{macro}
%
%
%
% \begin{macro}{\@preamblecmds}
%    All macros defined above are to be used only in the preamble.
%    Therefore, we insert them in |\@preamblecmds| which will disable
%    them at begin document. This is done by the internal \LaTeXe{}
%    command |\@onlypreamble|.
% \changes{v2.0g}{89/09/19}{`theoremheaderfont and `theorembodyfont
%    added.}
% \changes{v2.2a}{94/02/03}{Use \LaTeXe{} interface}
%    \begin{macrocode}
\@onlypreamble\@xnthm
\@onlypreamble\@ynthm
\@onlypreamble\@othm
\@onlypreamble\newtheorem
\@onlypreamble\theoremstyle
\@onlypreamble\theorembodyfont
\@onlypreamble\theoremheaderfont
%    \end{macrocode}
% \end{macro}
%
% Finally we declare the \texttt{plain} theorem style to be the default.
% \changes{v2.1b}{90/06/04}{Fixed missing `item by declaring default
%    style.}
%    \begin{macrocode}
\theoremstyle{plain}
%</package>
%    \end{macrocode}
%
% \Finale

\endinput
}{}
\ProvidesFile{thmb.sty}
             [\FMithmInfo]
%    \end{macrocode}
%
% \begin{macro}{\th@marginbreak}
%    The \texttt{margin break} style is nearly the same as the
%    \texttt{change break} style. The only difference is the placement
%    of the theorem number.  We use |\llap| to place it in the left
%    margin.
%
%    In this style |\labelsep| denotes the separation between the
%    number and the text.
%    \begin{macrocode}
\gdef\th@marginbreak{\normalfont\slshape
  \def\@begintheorem##1##2{\item
       [\rlap{\vbox{\theorem@headerfont
                    \hbox{\llap{##2}\hskip\labelsep ##1}%
                    \hbox{\strut}}}]}%
\def\@opargbegintheorem##1##2##3{%
  \item[\rlap{\vbox{\theorem@headerfont
                    \hbox{\llap{##2}\hskip\labelsep ##1\ (##3)}%
                    \hbox{\strut}}}]}}
\endgroup
%</thmb>
%    \end{macrocode}
% \changes{v2.1c}{90/02/17}{`endgroup outside macrocode env.}
% \end{macro}
%
%
%
% \subsubsection{The \texttt{margin} style}
%
% This style option is stored in the file |thm.sty|.
%    \begin{macrocode}
%<*thm>
\begingroup \makeatletter
\@ifundefined{theorem@style}{% \iffalse meta-comment
%
% Copyright (C) 1993-2024
%
% The LaTeX Project and any individual authors listed elsewhere
% in this file.
%
% This file is part of the Standard LaTeX `Tools Bundle'.
% -------------------------------------------------------
%
% It may be distributed and/or modified under the
% conditions of the LaTeX Project Public License, either version 1.3c
% of this license or (at your option) any later version.
% The latest version of this license is in
%    https://www.latex-project.org/lppl.txt
% and version 1.3c or later is part of all distributions of LaTeX
% version 2005/12/01 or later.
%
% The list of all files belonging to the LaTeX `Tools Bundle' is
% given in the file `manifest.txt'.
%
% \fi
%
%%
%
%
\def\FMithmInfo{2023/07/05 v2.2c Theorem extension package (FMi)}
%
% \ProvidesFile{theorem.dtx}[\FMithmInfo]
%\iffalse   % this is a METACOMMENT !
%%
%% Package `theorem' to use with LaTeX2e
%% Copyright (C) 1989-1995 Frank Mittelbach, all rights reserved.
%<+package>\@ifundefined{theorem@style}{}{\endinput}
%<+package>\NeedsTeXFormat{LaTeX2e}[1995/06/01]
%<+package>\ProvidesPackage{theorem}
%<+package>         [\FMithmInfo]
%
% \fi
%
% \changes{v1.0}{87/05/07}{First draft.}
% \changes{v1.0a}{87/05/07}{`thistheoremstyle' added.}
% \changes{v1.1a}{87/07/29}{New concept: redefine the internal
%                          \protect\LaTeX{} commands `@xnthm' etc.}
% \changes{v1.1b}{87/07/31}{`theoremfont' used to specify font.}
% \changes{v1.2a}{88/01/08}{`@endtheorem' now globally defined.}
% \changes{v1.2b}{88/01/11}{`newtheorem' only allowed in the preamble.}
% \changes{v1.3a}{88/02/14}{`theoremfont' now a toks register.}
%
% \changes{v2.0a}{88/11/16}{New implementation and new documentation.}
% \changes{v2.0a}{88/11/16}{No layout changes but internal modifications
%                           old substyles won't work.}
% \changes{v2.0b}{89/01/30}{`global' used for all definitions.}
% \changes{v2.0c}{89/06/15}{Documentation translated to English.
%                            Thanks to LLL and Nina}
% \changes{v2.1a}{90/02/15}{Support for new font selection.}
% \changes{v2.2a}{94/02/03}{Cleanup for LaTeX2e}
%
% \DoNotIndex{\@endparenv,\@ifdefinable,\@ifnextchar,\@ifundefined}
% \DoNotIndex{\@namedef,\@nameuse,\@spaces,\@tempa,\@warning}
% \DoNotIndex{\ }
% \DoNotIndex{\begingroup,\bfseries}
% \DoNotIndex{\csname}
% \DoNotIndex{\def,\do,\docdate}
% \DoNotIndex{\endcsname,\endgroup,\endinput,\endtrivlist,\expandafter}
% \DoNotIndex{\filedate,\fileversion}
% \DoNotIndex{\gdef,\global}
% \DoNotIndex{\hbox,\hskip}
% \DoNotIndex{\ignorespaces,\input,\itshape,\item}
% \DoNotIndex{\labelsep,\let,\llap}
% \DoNotIndex{\makeatletter}
% \DoNotIndex{\noexpand}
% \DoNotIndex{\refstepcounter,\relax,\rlap}
% \DoNotIndex{\setlength,\slshape,\space,\string,\strut}
% \DoNotIndex{\the,\trivlist,\typeout}
% \DoNotIndex{\vbox}
% \DoNotIndex{\xdef}
%
% \def\deflabel#1{\ttfamily #1\hfill}
% \def\deflist#1{\begin{list}{}{\settowidth\labelwidth{\ttfamily #1}%
%                               \setlength\leftmargin\labelwidth
%                               \addtolength\leftmargin\labelsep
%                               \let\makelabel\deflabel}}
% \def\enddeflist{\end{list}}
% \def\envfont{\normalfont\ttfamily}
%
% \MakeShortVerb{\|}
%
% \newcommand{\TUB}{{\sl TUGboat\/}}
%
% \GetFileInfo{theorem.dtx}
% \title{An Extension of the
%        \LaTeX{} theorem environment\thanks{This file
%        has version number \fileversion, last
%        revised \filedate.}}
% \author{Frank Mittelbach}
%
% \MaintainedByLaTeXTeam{tools}
% \maketitle
%
% \begin{abstract}
%    The macros described in this paper yield an extension of the
%    \LaTeX{} theorem mechanism. It is designed to satisfy the
%    different requirements of various journals. Thus, the
%    layout of the ``theorems'' can be manipulated by determining a
%    ``style''. This article describes not only the use, but
%    also the definition, of the necessary macros.
% \end{abstract}
%
% \section*{Preface to version 2.2}
%
% For \LaTeXe{} this package did not need any fundamental changes. I
% only modified the messages generated so that theorem layout styles
% will show up with the |\listfiles| command and cleaned the section
% on the New Font Selection Scheme since this is now included in
% \LaTeX.
%
% \section*{Preface to version 2.1}
%
% This version is identical to 2.0g described in \TUB\ 10\#3 except
% for some internal defaults which are now set depending on the used
% font selection scheme.
%
% This was done to avoid unpleasant surprises if the new font selection
% scheme is in force. For further details see
% section~\ref{sec:newfonts} and \cite{bk:GMS94}.
%
% \section{Introduction}
%
% For our purposes here, ``theorems'' are labelled enunciations,
% often set off from the main text by extra space and a font change.
% Theorems, corollaries, conjectures, definitions, and remarks
% are all instances of ``theorems''.  The ``header'' of these structures
% is composed of a label (such as \textsc{Theorem} or \textsc{Remark})
% and a number which serializes an item in the sequence of items
% with the same label.
%
% Shortly after the introduction of \LaTeX{} at the Fachbereich
% Mathematik in Mainz, the desire to manipulate the layout of
% ``theorems'' arose.  In Mainz, the following two conventions came
% into general use:
% \begin{enumerate}
%   \item The number of the theorem is shown in the margin.
%   \item There is a line break at
%         the end of the theorem header.
% \end{enumerate}
%
% Additionally, some journals require different formats which depend on
% the ``sort of theorem'': e.g.\ often remarks and definitions are set
% in |\upshape|, while |\itshape| is employed for main theorems.
%
% Confronted with these requirements, a theorem environment was
% developed in Mainz which allows separate determination of the
% layout of the ``theorems sets'', comparable to |\pagestyle|.
%
%
% \section{The user interface}
% \subsection{Defining new theorem sets}
%
% \DescribeMacro\newtheorem
% As in the original \LaTeX{} version, the command |\newtheorem|
% defines a new ``theorem set'' or ``theorem-like structure''.
% Two required arguments name the new environment and give the text
% to be typeset with each instance of the new ``set'', while
% an optional argument determines how
% the ``set'' is enumerated:
% \begin{description}
%    \item[\ttfamily \bslash newtheorem\{foo\}\{bar\}]
%       The theorem set {\envfont foo} (whose name is \texttt{bar})
%       uses its own counter.
%    \item[\ttfamily \bslash newtheorem\{foo2\}{[foo]}\{bar2\}]
%       The theorem set {\envfont foo2} (printed name \texttt{bar2})
%       uses the same counter as the theorem set \texttt{foo}.
%    \item[\ttfamily \bslash newtheorem\{foo3\}\{bar3\}{[section]}]
%       The theorem set {\envfont foo3} (printed name \texttt{bar3}) is
%       enumerated within the counter \texttt{section}, i.e.\ with every
%       new |\section| the enumeration begins again with 1, and
%       the enumeration is composed from the section-number and the
%       theorem counter itself.
% \end{description}
%
% \DescribeMacro\theoremstyle
% Additionally, the command |\theoremstyle| can define the layout
% of various, or all, theorem sets.  It should be noted that any
% theorem set defined by |\newtheorem| is typeset in the
% |\theoremstyle| that is current at the time of the definition.
% Thus, the following
% \begin{verbatim}
%    \theoremstyle{break}        \newtheorem{Cor}{Corollary}
%    \theoremstyle{plain}        \newtheorem{Exa}{Example}[section]
%\end{verbatim}                             ^^A No blank at line start
% leads to the result that the set {\envfont Cor} is formatted in the
% style \texttt{break}, while the set {\envfont Exa} and all the
% following ones are formatted in the style \texttt{plain}, unless
% another |\theoremstyle| follows.  Since the definitions installed by
% |\newtheorem| are global, one also can limit |\theoremstyle| locally
% by grouping braces.
%
% \DescribeMacro\theorembodyfont
% The choice of the font for the theorem body is completely
% independent of the chosen |\theoremstyle|; this has proven to
% be very advantageous.  For example,
% \begin{verbatim}
%   {\theorembodyfont{\upshape}          \newtheorem{Rem}{Remark}}
%\end{verbatim}
% defines a theorem set {\envfont Rem}, which will be set in
% |\upshape| in the current layout (which in our example is
% \texttt{plain}).  As with |\theoremstyle|, the |\theorembodyfont|
% chosen is that current at the time of |\newtheorem|.  If
% |\theorembodyfont| is not specified or one defines
% |\theorembodyfont{}|, then the font used will be that defined by the
% |\theoremstyle|.
%
% \DescribeMacro\theoremheaderfont It is also possible to customize
% the font used for the theorem headers.  This is, however, a global
% declaration, and therefore there should be at most one
% |\theoremheaderfont| declaration in the preamble.\footnote{If it is
% actually necessary to have different header fonts, one has to define
% new theorem styles (substituting the desired font) or specify the
% information directly in the \texttt{\bslash newtheorem} declaration
% (the unclean variant).}
%
% \DescribeMacro\theorempreskipamount
% \DescribeMacro\theorempostskipamount Two additional parameters
% affect the vertical space around the theorem environments:
% |\theorempreskipamount| and |\theorempostskipamount| define,
% respectively, the spacing before and after such an environment.
% These parameters apply for all theorem sets and can be manipulated
% with the ordinary length macros.  They are rubber lengths
% (`\textsf{skips}'), and therefore can contain \texttt{plus} and
% \texttt{minus} parts.
%
% Since the definition of theorem sets should---most sensibly---be
% placed in the preamble, we only allow installation there.
% It is therefore possible to release the memory used here after
% |\begin{document}|, in order to make room for other
% applications.
%
%
% \subsection{Existing theorem styles}
%
% The following theorem styles exist to date:
% \begin{deflist}{marginbreak}
%    \item[plain]
%       This theorem style emulates the original \LaTeX{} definition,
%       except that additionally the parameters
%       |\theorem...skipamount| are used.
%    \item[break]
%       In this style, the theorem header is followed by a line
%       break.
%    \item[marginbreak]
%       The theorem number is set in the margin, and there is a line
%       break as in \texttt{break}.
%    \item[changebreak]
%       Like \texttt{break}, but with header number and text
%       interchanged.
%    \item[change]
%       Header number and text are interchanged, without a line break.
%    \item[margin]
%       The number is set in the left margin, without a line break.
% \end{deflist}
% All styles (except \texttt{plain}) select |\slshape| as the default
% |\theorembodyfont|.
%
%
% \subsection{Examples}
%
% Given the above theorem sets {\envfont Cor}, {\envfont Exa}
% and {\envfont Rem},
% suppose that the preamble also contains the declarations:
% \begin{verbatim}
%    \theoremstyle{marginbreak}   \newtheorem{Lem}[Cor]{Lemma}
%    \theoremstyle{change}
%    \theorembodyfont{\itshape}        \newtheorem{Def}[Cor]{Definition}
%
%    \theoremheaderfont{\scshape}
%\end{verbatim}
% Then the following are some typical examples of the typeset output
% resulting from their use.
%
% \begin{Cor}
%    This is a sentence typeset in the theorem environment {\envfont
%    Cor}.
% \end{Cor}
%
% \begin{Exa}
%    This is a sentence typeset in the theorem environment {\envfont
%    Exa}.
% \end{Exa}
%
% \begin{Rem}
%    This is a sentence typeset in the theorem environment {\envfont
%    Rem}.
% \end{Rem}
%
% \begin{Lem}[Ben User]
%    This is a sentence typeset in the theorem environment {\envfont
%    Lem}.
% \end{Lem}
%
% \begin{Def}[Very impressive Definition]
%    This is a sentence typeset in the theorem environment {\envfont
%    Def}.
% \end{Def}
%
% The last two examples show the effect of the optional argument to
% a theorem environment (it is the text typeset in parentheses).
%
%
% \section{Special Considerations}
% \label{sec:newfonts}
%
% Theoremheader and body are implemented as a unit. This means that
% the |\theoremheaderfont|  will inherit characteristics of the
% |\theorembodyfont| in \LaTeXe. Thus, if for example
% |\theorembodyfont| is |\itshape| and |\theoremheaderfont|
% is |\bfseries| the font selected for the header will have the
% characteristics `bold extended italic'. If this is not desired one
% should set the |\theoremheaderfont| to something like
% \begin{verbatim}
%      \theoremheaderfont{\normalfont\bfseries}
%\end{verbatim}
% i.e.\ supplying all necessary font information explicitly.
%
%
%
% \section{Acknowledgements}
%
% The publication of this set of macros was only possible with the
% help of Christina Busse (translating the manuscript into English),
% Joachim Pense (playing the r\^ole of typist), Chris Rowley
% (looking everything over) and many others providing useful
% suggestions.
%
% \MaybeStop{%
%    \begin{thebibliography}{9}
%       \bibitem{bk:GMS94} \textsc{M.~Goossens}, \textsc{F.~Mittelbach}
%         and \textsc{A.~Samarin}.
%         \newblock The \LaTeX{} Companion.
%         \newblock
%         Addison-Wesley, Reading, Massachusetts, 1994.
%       \bibitem{latex:source} \textsc{Lamport, Leslie.} \newblock
%                               \texttt{latex.tex}, version 2.09, date
%                               Feb. 1990.
%    \end{thebibliography}
%    \PrintIndex
%    \PrintChanges}
%
%
% \section{The documentation driver file}
%
% The next bit of code contains the documentation driver file for
% \TeX{}, i.e., the file that will produce the documentation you are
% currently reading. It will be extracted from this file by the
% \texttt{docstrip} program. Since it is the first code in the file
% one can alternatively process this file directly with \LaTeXe{} to
% obtain the documentation.
%
%    \begin{macrocode}
%<*driver>
\documentclass{ltxdoc}

\usepackage{theorem}

%<+driver>% The next few lines define theorem sets which are used
%<+driver>% in the example section of the documentation.

\theoremstyle{break}       \newtheorem{Cor}{Corollary}
\theoremstyle{plain}       \newtheorem{Exa}{Example}[section]
{\theorembodyfont{\upshape}\newtheorem{Rem}{Remark}}
\theoremstyle{marginbreak} \newtheorem{Lem}[Cor]{Lemma}
\theoremstyle{change}
\theorembodyfont{\itshape} \newtheorem{Def}[Cor]{Definition}

\theoremheaderfont{\scshape}

\RecordChanges

\begin{document}
   \DocInput{theorem.dtx}
\end{document}
%</driver>
%    \end{macrocode}
%
% \section{Definition of the Macros}
%
% If the file has been loaded before, we abort immediately. If not the
% package announces itself (this is actually done at the very top of
% the file---the way it is done isn't good style so don't copy it).
% \changes{v2.0e}{89/07/19}{Spaces removed from `typeout'.}
% \changes{v2.2b}{95/11/19}{Announce moved to top of file}
%    \begin{macrocode}
%<*package>
%\@ifundefined{theorem@style}{}{\endinput}
%\def\FMithmInfo{1995/11/19 v2.2b Theorem extension package (FMi)}
%\ProvidesPackage{theorem}[\FMithmInfo]
%    \end{macrocode}
%
%
% \subsection{Definition of theorem styles and fonts}
%
% All the definitions in this file are done globally to allow
% inputting this file inside a group.
%
% \begin{macro}{\theoremstyle}
%    Before a theorem style can be installed, the chosen style must be
%    known. For that reason, we must test to see that
%    |\th@|\meta{style} is known or, more precisely, that it is
%    different from |\relax|. If the style is not known then
%    |\th@plain| is used.
%    \begin{macrocode}
\gdef\theoremstyle#1{%
   \@ifundefined{th@#1}{\@warning
          {Unknown theoremstyle `#1'. Using `plain'}%
          \theorem@style{plain}}%
%    \end{macrocode}
%    We save the theorem style to be used in the token register
%    |\theorem@style|.
%    \begin{macrocode}
      {\theorem@style{#1}}%
%    \end{macrocode}
%    Now we ``evaluate'' the theorem style: this means, we call the
%    macro |\th@|\meta{style} which will activate the relevant
%    definitions which are contained in a separate file.  This is done
%    in a group to suppress changes to the current font.  This could
%    otherwise pose problems together with the new font selection
%    scheme\footnote{When I printed the original article using the new
%    font selection scheme I ended with a document with slanted
%    typefaces (text headings and all) simply because one of the
%    theorem styles used \texttt{\bslash sl} at toplevel.} if the
%    |\th@|\meta{style} is evaluated a second time.
% \changes{v2.1a}{90/02/15}{Group added around `th@..}
% \changes{v2.1a}{90/02/15}{Reseting of `..@begintheorem removed.}
%    \begin{macrocode}
      \begingroup
        \csname th@\the\theorem@style \endcsname
      \endgroup}
%    \end{macrocode}
% \end{macro}
%
%
% \begin{macro}{\@begintheorem}
% \changes{v2.1a}{90/02/15}{Relaxed at toplevel.}
% \begin{macro}{\@opargbegintheorem}
% \changes{v2.1a}{90/02/15}{Relaxed at toplevel.}
%    We reset |\@begintheorem| and |\@opargbegintheorem| to
%    |\relax| since these commands are no longer necessary at
%    toplevel. This will save a few tokens.
%    \begin{macrocode}
\global\let\@begintheorem\relax
\global\let\@opargbegintheorem\relax
%    \end{macrocode}
% \end{macro}
% \end{macro}
%
% \begin{macro}{\theorem@style}
%    Obviously the \textsf{token} register used above has to be
%    allocated. To assure the utmost compatibility with the
%    original \LaTeX{} definition, we set the default theorem style to
%    \texttt{plain}, which implements the usual \LaTeX{} convention.
% \changes{v2.0f}{89/07/20}{Unnecessary `global' removed.}
%    \begin{macrocode}
\newtoks\theorem@style
\global\theorem@style{plain}
%    \end{macrocode}
% \end{macro}
%
% \begin{macro}{\theorembodyfont}
% \begin{macro}{\theorem@bodyfont}
%    If the |\theorembodyfont| is set by the user then it should not
%    interact with the default font set in the theorem style. When the
%    new font selection is in force this may happen if, for example,
%    the default is |\itshape| and the new |\theorembodyfont| is
%    |\sffamily|. So we add a |\reset@font| command in front of the
%    user definition.
% \changes{v2.0f}{89/07/20}{Unnecessary `global' removed.}
% \changes{v2.0g}{89/09/19}{`theoremfont to `theorembodyfont.}
% \changes{v2.1c}{92/06/22}{`reset@font added and logic changed.}
% \changes{v2.1d}{92/10/28}{Reset to empty if arg is empty}
%    \begin{macrocode}
\gdef\theorembodyfont#1{%
%    \end{macrocode}
%    We check if the argument supplied is empty and if so put nothing
%    into the |\theorem@bodyfont| token register to allow for
%    |\theorembodyfont{}| as a mean of using the default of the
%    current |\theoremstyle|.
%    \begin{macrocode}
   \def\@tempa{#1}%
   \ifx\@tempa\@empty
    \theorem@bodyfont{}%
   \else
    \theorem@bodyfont{\reset@font#1}%
   \fi
   }
\newtoks\theorem@bodyfont
\global\theorem@bodyfont{}
%    \end{macrocode}
% \end{macro}
% \end{macro}
%
%
% \begin{macro}{\theoremheaderfont}
% \changes{v2.0g}{89/09/19}{Macro added.}
%    The font for the theorem headers is handled differently because
%    this definition applies to all theorem styles.
%    \begin{macrocode}
\gdef\theoremheaderfont#1{\gdef\theorem@headerfont{#1}%
%    \end{macrocode}
%    After using the macro once it is redefined to produce an error
%    message.
%    \begin{macrocode}
       \gdef\theoremheaderfont##1{%
        \typeout{\string\theoremheaderfont\space should be used
                 only once.}}}
%    \end{macrocode}
% \end{macro}
%
% \begin{macro}{\theorem@headerfont}
% \changes{v2.0g}{89/09/19}{Macro added.}
%    To set the |\theorem@headerfont| default we first test if
%    the new fontselection scheme is in force.
% \changes{v2.1a}{90/02/15}{Support for new font selection.}
%    \begin{macrocode}
\ifx\upshape\undefined
%    \end{macrocode}
%    If not we define it to expand into |\bfseries|. We don't use
%    |\let| just in case a following style option redefines this
%    macro.
%    \begin{macrocode}
\gdef\theorem@headerfont{\bfseries}
%    \end{macrocode}
%    Otherwise we reset the current shape before calling |\bfseries|.
%    \begin{macrocode}
\else \gdef\theorem@headerfont{\normalfont\bfseries}\fi
%    \end{macrocode}
% \end{macro}
%
% \begin{macro}{\th@plain}
% \begin{macro}{\th@break}
% \begin{macro}{\th@marginbreak}
% \begin{macro}{\th@changebreak}
% \begin{macro}{\th@change}
% \begin{macro}{\th@margin}
%    The different styles are defined in macros such as
%    |\th@plain|.  Since memory space is precious in
%    ``non-Big-versions'', we have to avoid offering too many unused
%    definitions.  Therefore we define these styles in separate files
%    that can be loaded on demand.  Thus the commands themselves only
%    load these files. We use |\@input@|, a \LaTeXe{} internal command
%    that ensures that the file will be listed with |\listfiles|.
% \changes{v2.2a}{94/02/02}{Use `@input@ to load theorem layout files}
%    \begin{macrocode}
\gdef\th@plain{\@input@{thp.sty}}
\gdef\th@break{\@input@{thb.sty}}
\gdef\th@marginbreak{\@input@{thmb.sty}}
\gdef\th@changebreak{\@input@{thcb.sty}}
\gdef\th@change{\@input@{thc.sty}}
\gdef\th@margin{\@input@{thm.sty}}
%    \end{macrocode}
%    This list will be expanded when new styles become
%    available.  For testing,  just append new theorem substyles
%    as document options.
% \end{macro}
% \end{macro}
% \end{macro}
% \end{macro}
% \end{macro}
% \end{macro}
%
%
%
% \subsection{Definition of a new theorem set}
%
% As already pointed out, a new theorem environment can be defined in
% three different ways:
% \begin{verbatim}
%    \newtheorem{Lem}{Lemma}
%    \newtheorem{Lem}{Lemma}[section]
%    \newtheorem{Lem}[Theorem]{Lemma}
%\end{verbatim}
% The function of the macro |\newtheorem| is to recognize these cases
% and then to branch into one of the three macros |\@ynthm|, |\@xnthm|
% or |\@othm|.  This mechanism is adopted unchanged from
% \cite{latex:source}; the essential point here is that, for example,
% in the second case, the arguments \texttt{Lem}, \texttt{Lemma} and
% \texttt{section} are passed over to the macro |\@xnthm|.
%
% We inspect this case first because the others present fewer problems,
% and thus are easily derived from this one.
%
% \begin{macro}{\@xnthm}
%    For our example arguments, the macro |\@xnthm| must fulfill
%    the following:
%    \begin{itemize}
%       \item
%          Define a new \LaTeX-counter `Lem'
%       \item
%          reset this counter within a |\section|
%       \item
%          define the macro |\theLem|
%       \item
%          define the environment macros |\Lem| and
%          |\endLem| using the current
%          |\theoremstyle| and |\theorem@bodyfont|.
%    \end{itemize}
%    Obviously, all this should happen only if the first argument
%    of |\@xnthm| (i.e.\ \texttt{Lem} in our example) is chosen so
%    as not to conflict with any previously defined commands or
%    environments.  This test is performed by the \LaTeX{} macro
%    |\@ifdefinable|.
%    \begin{macrocode}
\gdef\@xnthm#1#2[#3]{\expandafter\@ifdefinable\csname #1\endcsname
%    \end{macrocode}
%    Therefore, the first argument of |\@ifdefinable| is the
%    expansion (in the example, |\Lem|) of
%    |\csname||#1\endcsname|.  The second argument is
%    executed only if the test has been completed successfully.
%    \begin{macrocode}
   {%
%    \end{macrocode}
%    Now we define the new counter. The names of the \LaTeX{}
%    macros employed should speak for themselves:
% \changes{v2.2b}{1995/10/07}{Use \cs{@newctr}}
% \changes{v2.2c}{1995/11/23}{Use \cs{@newctr} properly.}
%    \begin{macrocode}
    \@definecounter{#1}\@newctr{#1}[#3]%
%    \end{macrocode}
%    Using |\@newctr| will give a proper error message if the counter
%    in |#3| is not defined.
%    In defining `|\theLem|' we must generate the desired macro
%    name by use of |\expandafter| and |\csname|.
%    \begin{macrocode}
    \expandafter\xdef\csname the#1\endcsname
%    \end{macrocode}
%    An |\xdef| is used in order to make the definition global,
%    and to ensure that it contains the replacement texts of
%    |\@thmcountersep| and |\@thmcounter|.\footnote{These
%    two macros can be defined by the document style.  Their default
%    values produce a `\texttt{.}' as separation and an arabic
%    representation of the number.} However, not everything should be
%    expanded. For example, it saves space to use |\thesection|
%    instead of its---at times---lengthy expansion.
%    \begin{macrocode}
      {\expandafter \noexpand \csname the#3\endcsname
       \@thmcountersep \@thmcounter{#1}}%
%    \end{macrocode}
%    Thus with the defaults of \LaTeX{}, |\theLem| would be
%    replaced by the command sequence |\thesection.\arabic{Lem}|.
%
%    We will now look at the definition of the macro which is executed
%    at the beginning of the actual environment (in our example this
%    macro is |\Lem|). It should be noted that we use an
%    ``|\expandafter| trick'' to expand only certain parts of the
%    replacement text at the time of the definition.
%    \begin{macrocode}
    \def\@tempa{\global\@namedef{#1}}%
    \expandafter \@tempa \expandafter{%
%    \end{macrocode}
%    First, the macro that contains the current definitions of
%    |\@begintheorem| and |\@opargtheorem| should be called
%    up.  The name of this macro---as is already known---has the form
%    |\th@|\meta{theorem\ style}; therefore, it must be called by
%    \begin{macrocode}
      \csname th@\the \theorem@style
%    \end{macrocode}
%    In addition the default theorem font should be changeable, i.e.\
%    we have to insert the contents of |\theorem@bodyfont|. For
%    that reason, we expand even further, beyond |\endcsname|,
%    and thus insert the contents of the \textsf{token} register
%    |\theorem@bodyfont| in the replacement text.
%    \begin{macrocode}
            \expandafter \endcsname \the \theorem@bodyfont
%    \end{macrocode}
%    Now it is time to call the macro |\@thm| which takes over
%    the further processing.  It has two arguments: the current
%    counter name (in our example, \texttt{Lem}), and the text of the
%    label (in our example, \texttt{Lemma}).
%    \begin{macrocode}
     \@thm{#1}{#2}}%
%    \end{macrocode}
%    With this, the `sub-definition' is complete.  The macro
%    |\@endtheorem| ends a theorem environment and is, so far, nothing
%    but an |\endtrivlist|.  (Hence it is defined globally, and not
%    within the theorem styles.\footnote{This has to be changed as
%    soon as theorem styles that change \texttt{\bslash @endtheorem}
%    exist.  In such a case, all existing styles must be changed as
%    well since they will have to reset the macro.}) Therefore, we can
%    set it equivalent to the macro that ends the theorem set (in our
%    example, |\endLem|).  However, if some day theorem styles exist
%    that do change |\@endtheorem|, we would have to use the
%    commented-out line instead.
%    \begin{macrocode}
    \global \expandafter \let \csname end#1\endcsname \@endtheorem
%   \global\@namedef{end#1}{\@endtheorem}%
%    \end{macrocode}
%    With these commands all the required definitions are employed,
%    unless the test |\@ifdefinable| has failed.  Therefore, we
%    end the second argument of this macro and with it the definition
%    of |\@xnthm|.
%    \begin{macrocode}
   }}
%    \end{macrocode}
% \end{macro}
%
% \begin{macro}{\@ynthm}
%    The definition of |\@ynthm| is completely analogous. In
%    this case the new counter that is defined is not reset within
%    another counter; thus the definition of |\the...|
%    is simplified:
%    \begin{macrocode}
\gdef\@ynthm#1#2{\expandafter\@ifdefinable\csname #1\endcsname
   {\@definecounter{#1}%
    \expandafter\xdef\csname the#1\endcsname{\@thmcounter{#1}}%
%    \end{macrocode}
%    The rest of the definition corresponds literally to that of
%    |\@xnthm|:
%    \begin{macrocode}
    \def\@tempa{\global\@namedef{#1}}\expandafter \@tempa
     \expandafter{\csname th@\the \theorem@style \expandafter
     \endcsname \the\theorem@bodyfont \@thm{#1}{#2}}%
    \global \expandafter \let \csname end#1\endcsname \@endtheorem}}
%    \end{macrocode}
% \end{macro}
%
% \begin{macro}{\@othm}
%    The definition of |\@othm| does not contain anything new.
%    \begin{macrocode}
\gdef\@othm#1[#2]#3{%
%    \end{macrocode}
% \changes{v2.2b}{95/10/07}{Build in check for counter defined}
%    We do not define a new counter but instead use one that has
%    already been defined.  Thus the only definition we need is that
%    of this pseudo-counter (i.e.\ |\the|\meta{env.\ name}).
%    First we check if |#2| corresponds to a known counter name.
%    \begin{macrocode}
  \expandafter\ifx\csname c@#2\endcsname\relax
   \@nocounterr{#2}%
  \else
   \expandafter\@ifdefinable\csname #1\endcsname
   {\expandafter \xdef \csname the#1\endcsname
     {\expandafter \noexpand \csname the#2\endcsname}%
%    \end{macrocode}
%    All other parts of the definition can be adopted from
%    |\@xnthm|. We have to remember, though, that in this case
%    the name of the current counter and the theorem label have moved
%    to the second and third arguments.
%    \begin{macrocode}
    \def\@tempa{\global\@namedef{#1}}\expandafter \@tempa
     \expandafter{\csname th@\the \theorem@style \expandafter
     \endcsname \the\theorem@bodyfont \@thm{#2}{#3}}%
    \global \expandafter \let \csname end#1\endcsname \@endtheorem}%
  \fi}
%    \end{macrocode}
% \end{macro}
%
% \subsection{Macros that are employed in a {\envfont theorem}
%    environment}
%
% \begin{macro}{\@thm}
%    The macro |\@thm| has to increase the current counter.  Then,
%    depending on whether the environment has (or does not have) an
%    optional argument, it has to branch into either |\@begintheorem|
%    or |\@opargtheorem|.
%    \begin{macrocode}
\gdef\@thm#1#2{\refstepcounter{#1}%
%    \end{macrocode}
%    Now we start a {\envfont trivlist} environment, and give
%    |\@topsep| and |\@topsepadd| the values of the \textsf{skip}
%    registers |\theorempreskipamount| and |\theorempostskipamount|.
%    The value in |\@topsep| is the vertical space that is inserted by
%    the first (and only) |\item| in our |\trivlist| whilst
%    |\@topsepadd| is inserted by |\@endparenv| at the end of that
%    {\envfont trivlist} environment.  By using these registers, we
%    obtain the desired space around a {\envfont theorem} environment.
%    \begin{macrocode}
   \trivlist
   \@topsep \theorempreskipamount               % used by first \item
   \@topsepadd \theorempostskipamount           % used by \@endparenv
%    \end{macrocode}
%    Now we have to test whether an optional argument has been given.
% \changes{v2.0d}{89/07/18}{Forgotten percent added.}
%    \begin{macrocode}
   \@ifnextchar [%
%    \end{macrocode}
%    If there is an optional argument, we will call |\@ythm|, and
%    move the arguments read back into the input stream.
%    \begin{macrocode}
   {\@ythm{#1}{#2}}%
%    \end{macrocode}
%    If not, we call |\@begintheorem|. Its first argument is the
%    name of the theorem set (hence the second argument of
%    |\@thm|). Its second argument is the macro that produces the
%    current number.
%    \begin{macrocode}
   {\@begintheorem{#2}{\csname the#1\endcsname}\ignorespaces}}
%    \end{macrocode}
% \end{macro}
%
% \begin{macro}{\@xthm}
% \begin{macro}{\@ythm}
%    Both these macros were originally called by |\@thm|.  We do
%    not need |\@xthm| anymore, hence we reset it to
%    |\relax|. The definition of |\@ythm| has not changed at
%    all from its definition in \LaTeX. In order to make the macros
%    easier to understand, we will
%    nevertheless present it (commented out).
%    \begin{macrocode}
\global\let\@xthm\relax
% \def\@ythm#1#2[#3]{\@opargbegintheorem{#2}{\csname
%           the#1\endcsname}{#3}\ignorespaces}
%</package>
%    \end{macrocode}
%    The primitive |\ignorespaces| in |\@ythm| and
%    |\@thm| is needed to remove the spaces between the
%    |\begin{...}| and the actual text.
% \end{macro}
% \end{macro}
%
%
% \subsection{Definition of the theorem substyles}
%
% As already pointed out, the theorem substyles, defined below, are
% only loaded when necessary.  Note that all these substyles, except
% \texttt{plain}, have |\slshape| as the default body font.
%
% \subsubsection{The \texttt{plain} style}
%
% \label{dok:th@plain} As the following macros use |@|, we have
% to locally set the |\catcode| of this symbol to ``letter''.
% This happens within a group, so that we do not have to worry about
% which |\catcode| that symbol had before.
%    \begin{macrocode}
%<*thp>
\begingroup \makeatletter
%    \end{macrocode}
% Since we are now within a group, we must make all definitions
% globally.  First we make sure that \texttt{theorem.sty} is loaded.
% This will allow us to use this file as a document style option
% without having to call \texttt{theorem} itself as an option.  At the
% same time, we assure that at least version 2 is loaded, since
% |\theorem@style| was not defined in earlier versions.
%    \begin{macrocode}
\@ifundefined{theorem@style}{\input{theorem.sty}}{}
\ProvidesFile{thp.sty}
             [\FMithmInfo]
%    \end{macrocode}
%
%
% \begin{macro}{\th@plain}
% \begin{macro}{\@begintheorem}
% \begin{macro}{\@opargbegintheorem}
%    |\theoremstyle{plain}| corresponds to the original
%    definition, except that the distances to the surrounding text are
%    determined by the parameters |\theorempreskipamount| and
%    |\theorempostskipamount|.
%    First we set the default body font.
%    \begin{macrocode}
\gdef\th@plain{\normalfont\itshape
%    \end{macrocode}
%    Then we define |\@begintheorem| and
%    |\@opargbegintheorem|. These two macros define how the
%    header of a theorem is typeset.  |\@opargbegintheorem| will
%    be called if a \textsf{theorem} environment with an optional
%    argument is encountered; otherwise, the header is constructed by
%    calling |\@begintheorem|.  If one of these macros is
%    executed, we are within a \textsf{trivlist} environment started by
%    |\@thm|.  So the theorem header is produced with an
%    |\item| command.
%
%    Instead of specifying the header font directly, all standard
%    theorem styles use the |\theorem@headerfont| macro to allow
%    customization.  The extra space (|\labelsep|) is necessary
%    because of problems in the \textsf{trivlist} environment.
%    \begin{macrocode}
  \def\@begintheorem##1##2{%
        \item[\hskip\labelsep \theorem@headerfont ##1\ ##2]}%
%    \end{macrocode}
%    The definition of |\@opargbegintheorem| is completely
%    analogous. The only difference is the fact that there exists a
%    third argument (which is the optional parameter of the
%    environment and contains additional information about the
%    theorem).  Customarily we enclose it in parentheses.
%    \begin{macrocode}
\def\@opargbegintheorem##1##2##3{%
   \item[\hskip\labelsep \theorem@headerfont ##1\ ##2\ (##3)]}}
%    \end{macrocode}
% \end{macro}
% \end{macro}
% \end{macro}
%
% \noindent
% We conclude with an |\endgroup| to restore the
% |\catcode| of |@|.
%    \begin{macrocode}
\endgroup
%</thp>
%    \end{macrocode}
%
%
% \subsubsection{The \texttt{break} style}
%
% This style option is stored in the file |thb.sty|.
% For the next two lines see the documentation for |\th@plain| on
% page~\pageref{dok:th@plain}.
%    \begin{macrocode}
%<*thb>
\begingroup \makeatletter
\@ifundefined{theorem@style}{\input{theorem.sty}}{}
\ProvidesFile{thb.sty}
             [\FMithmInfo]
%    \end{macrocode}
%
% \begin{macro}{\th@break}
%    |\theoremstyle{break}| produces a line break after the name
%    of the theorem.  The font is |\slshape|.  Hence, we define
%    |\th@break| as follows:
%    \begin{macrocode}
\gdef\th@break{\normalfont\slshape
  \def\@begintheorem##1##2{\item[%
%    \end{macrocode}
%    We run into the following problem: it is not possible to create
%    the header with |\item[|\meta{title}|]| and then start
%    a new line by, for example, |\mbox{}\\|. Such a definition
%    will fail whenever a list environment follows immediately.  With
%    the above construction, the |\mbox{}| causes the switch
%    |@inlabel| (cf.\ definition of |\list| and
%    |\trivlist| in \cite{latex:source}) to be set to \texttt{false}
%    and so the following list will insert additional vertical
%    space (|\topskip|). This is quite annoying. Therefore, we
%    create the line break within the |\item|. In order to ensure
%    that the text will begin at the proper position in the following
%    line, we simply pretend that the label does not take any
%    room.\footnote{This will lead to problems whenever very high
%    symbols occurring in the line tower into the
%    heading. So, something else has to be done here sometime.}
%    \begin{macrocode}
       \rlap{\vbox{\hbox{\hskip \labelsep\theorem@headerfont ##1\ ##2}%
                   \hbox{\strut}}}]}%
%    \end{macrocode}
%    Again, the definition of |\@opargbegintheorem| is
%    completely analogous.
%    \begin{macrocode}
\def\@opargbegintheorem##1##2##3{%
  \item[\rlap{\vbox{\hbox{\hskip \labelsep \theorem@headerfont
                     ##1\ ##2\ (##3)}%
                    \hbox{\strut}}}]}}
\endgroup
%</thb>
%    \end{macrocode}
% \end{macro}
%
%
% \subsubsection{The \texttt{changebreak} style}
%
%    \begin{macrocode}
% This style option is stored in the file |thcb.sty|.
%    \begin{macrocode}
%<*thcb>
\begingroup \makeatletter
\@ifundefined{theorem@style}{\input{theorem.sty}}{}
\ProvidesFile{thcb.sty}
             [\FMithmInfo]
%    \end{macrocode}
%
% \begin{macro}{\th@changebreak}
%    The \texttt{change-break} theorem style is like \texttt{break}
%    but with interchange of theorem name and theorem number.  Thus we
%    define |\th@changebreak| as follows:
%    \begin{macrocode}
\gdef\th@changebreak{\normalfont\slshape
  \def\@begintheorem##1##2{\item
       [\rlap{\vbox{\hbox{\hskip\labelsep \theorem@headerfont ##2\ ##1}%
                    \hbox{\strut}}}]}%
\def\@opargbegintheorem##1##2##3{%
  \item[\rlap{\vbox{\hbox{\hskip\labelsep \theorem@headerfont
                     ##2\ ##1\ (##3)}%
                    \hbox{\strut}}}]}}
\endgroup
%</thcb>
%    \end{macrocode}
% \end{macro}
%
%
% \subsubsection{The \texttt{change} style}
%
% This style option is stored in the file |thc.sty|.
%    \begin{macrocode}
%<*thc>
\begingroup \makeatletter
\@ifundefined{theorem@style}{\input{theorem.sty}}{}
\ProvidesFile{thc.sty}
             [\FMithmInfo]
%    \end{macrocode}
%
% \begin{macro}{\th@change}
%    The \texttt{change} theorem style corresponds to the
%    \texttt{change break} style without a linebreak after the header.
%    To say it in another way, it's the same as the \texttt{plain}
%    style but with number and name interchanged and |\slshape| as the
%    default font.
%    \begin{macrocode}
\gdef\th@change{\normalfont\slshape
  \def\@begintheorem##1##2{\item
       [\hskip\labelsep \theorem@headerfont ##2\ ##1]}%
\def\@opargbegintheorem##1##2##3{%
  \item[\hskip\labelsep \theorem@headerfont ##2\ ##1\ (##3)]}}
\endgroup
%</thc>
%    \end{macrocode}
% \end{macro}
%
%
% \subsubsection{The \texttt{marginbreak} style}
%
% This style option is the one used most often at Mainz.  It is saved
% in the file |thmb.sty|.
%    \begin{macrocode}
%<*thmb>
\begingroup \makeatletter
\@ifundefined{theorem@style}{\input{theorem.sty}}{}
\ProvidesFile{thmb.sty}
             [\FMithmInfo]
%    \end{macrocode}
%
% \begin{macro}{\th@marginbreak}
%    The \texttt{margin break} style is nearly the same as the
%    \texttt{change break} style. The only difference is the placement
%    of the theorem number.  We use |\llap| to place it in the left
%    margin.
%
%    In this style |\labelsep| denotes the separation between the
%    number and the text.
%    \begin{macrocode}
\gdef\th@marginbreak{\normalfont\slshape
  \def\@begintheorem##1##2{\item
       [\rlap{\vbox{\theorem@headerfont
                    \hbox{\llap{##2}\hskip\labelsep ##1}%
                    \hbox{\strut}}}]}%
\def\@opargbegintheorem##1##2##3{%
  \item[\rlap{\vbox{\theorem@headerfont
                    \hbox{\llap{##2}\hskip\labelsep ##1\ (##3)}%
                    \hbox{\strut}}}]}}
\endgroup
%</thmb>
%    \end{macrocode}
% \changes{v2.1c}{90/02/17}{`endgroup outside macrocode env.}
% \end{macro}
%
%
%
% \subsubsection{The \texttt{margin} style}
%
% This style option is stored in the file |thm.sty|.
%    \begin{macrocode}
%<*thm>
\begingroup \makeatletter
\@ifundefined{theorem@style}{\input{theorem.sty}}{}
\ProvidesFile{thm.sty}
             [\FMithmInfo]
%    \end{macrocode}
%
% \begin{macro}{\th@margin}
%    Again this is only a variant of the theorem styles described
%    above without any new ideas.
%    \begin{macrocode}
\gdef\th@margin{\normalfont\slshape
  \def\@begintheorem##1##2{\item
       [\theorem@headerfont \llap{##2}\hskip\labelsep ##1]}%
\def\@opargbegintheorem##1##2##3{%
  \item[\theorem@headerfont \llap{##2}\hskip\labelsep ##1\ (##3)]}}
\endgroup
%</thm>
%    \end{macrocode}
% \end{macro}
%
%
%
%
% \subsection{Final Definitions}
%
% \begin{macro}{\theorempreskipamount}
% \begin{macro}{\theorempostskipamount}
%    The \textsf{skip} parameters that regulate the vertical empty space
%    before and after the {\envfont theorem} environment have
%    to be allocated as well.
% \changes{v2.0f}{89/07/20}{Unnecessary `global' removed.}
%    \begin{macrocode}
%<*package>
\newskip\theorempreskipamount
\newskip\theorempostskipamount
%    \end{macrocode}
%    Since we have used the same values for all theorem sets, we now can
%    assign them.
%    \begin{macrocode}
\global\setlength\theorempreskipamount{12pt plus 5pt minus 3pt}
\global\setlength\theorempostskipamount{8pt plus 3pt minus 1.5pt}
%    \end{macrocode}
% \end{macro}
% \end{macro}
%
%
% \begin{macro}{\@endtheorem}
%    The same holds for the macro |\@endtheorem|, which ends a
%    {\envfont theorem} environment. Since it is the same for all
%    theorem sets, it is removed from the macros |\th@|\meta{style}.
%    It simply ends the {\envfont trivlist} environment, which was
%    begun in |\@thm|.
%    \begin{macrocode}
\global\let\@endtheorem=\endtrivlist
%    \end{macrocode}
% \end{macro}
%
%
%
% \begin{macro}{\@preamblecmds}
%    All macros defined above are to be used only in the preamble.
%    Therefore, we insert them in |\@preamblecmds| which will disable
%    them at begin document. This is done by the internal \LaTeXe{}
%    command |\@onlypreamble|.
% \changes{v2.0g}{89/09/19}{`theoremheaderfont and `theorembodyfont
%    added.}
% \changes{v2.2a}{94/02/03}{Use \LaTeXe{} interface}
%    \begin{macrocode}
\@onlypreamble\@xnthm
\@onlypreamble\@ynthm
\@onlypreamble\@othm
\@onlypreamble\newtheorem
\@onlypreamble\theoremstyle
\@onlypreamble\theorembodyfont
\@onlypreamble\theoremheaderfont
%    \end{macrocode}
% \end{macro}
%
% Finally we declare the \texttt{plain} theorem style to be the default.
% \changes{v2.1b}{90/06/04}{Fixed missing `item by declaring default
%    style.}
%    \begin{macrocode}
\theoremstyle{plain}
%</package>
%    \end{macrocode}
%
% \Finale

\endinput
}{}
\ProvidesFile{thm.sty}
             [\FMithmInfo]
%    \end{macrocode}
%
% \begin{macro}{\th@margin}
%    Again this is only a variant of the theorem styles described
%    above without any new ideas.
%    \begin{macrocode}
\gdef\th@margin{\normalfont\slshape
  \def\@begintheorem##1##2{\item
       [\theorem@headerfont \llap{##2}\hskip\labelsep ##1]}%
\def\@opargbegintheorem##1##2##3{%
  \item[\theorem@headerfont \llap{##2}\hskip\labelsep ##1\ (##3)]}}
\endgroup
%</thm>
%    \end{macrocode}
% \end{macro}
%
%
%
%
% \subsection{Final Definitions}
%
% \begin{macro}{\theorempreskipamount}
% \begin{macro}{\theorempostskipamount}
%    The \textsf{skip} parameters that regulate the vertical empty space
%    before and after the {\envfont theorem} environment have
%    to be allocated as well.
% \changes{v2.0f}{89/07/20}{Unnecessary `global' removed.}
%    \begin{macrocode}
%<*package>
\newskip\theorempreskipamount
\newskip\theorempostskipamount
%    \end{macrocode}
%    Since we have used the same values for all theorem sets, we now can
%    assign them.
%    \begin{macrocode}
\global\setlength\theorempreskipamount{12pt plus 5pt minus 3pt}
\global\setlength\theorempostskipamount{8pt plus 3pt minus 1.5pt}
%    \end{macrocode}
% \end{macro}
% \end{macro}
%
%
% \begin{macro}{\@endtheorem}
%    The same holds for the macro |\@endtheorem|, which ends a
%    {\envfont theorem} environment. Since it is the same for all
%    theorem sets, it is removed from the macros |\th@|\meta{style}.
%    It simply ends the {\envfont trivlist} environment, which was
%    begun in |\@thm|.
%    \begin{macrocode}
\global\let\@endtheorem=\endtrivlist
%    \end{macrocode}
% \end{macro}
%
%
%
% \begin{macro}{\@preamblecmds}
%    All macros defined above are to be used only in the preamble.
%    Therefore, we insert them in |\@preamblecmds| which will disable
%    them at begin document. This is done by the internal \LaTeXe{}
%    command |\@onlypreamble|.
% \changes{v2.0g}{89/09/19}{`theoremheaderfont and `theorembodyfont
%    added.}
% \changes{v2.2a}{94/02/03}{Use \LaTeXe{} interface}
%    \begin{macrocode}
\@onlypreamble\@xnthm
\@onlypreamble\@ynthm
\@onlypreamble\@othm
\@onlypreamble\newtheorem
\@onlypreamble\theoremstyle
\@onlypreamble\theorembodyfont
\@onlypreamble\theoremheaderfont
%    \end{macrocode}
% \end{macro}
%
% Finally we declare the \texttt{plain} theorem style to be the default.
% \changes{v2.1b}{90/06/04}{Fixed missing `item by declaring default
%    style.}
%    \begin{macrocode}
\theoremstyle{plain}
%</package>
%    \end{macrocode}
%
% \Finale

\endinput
}{}
\ProvidesFile{thc.sty}
             [\FMithmInfo]
%    \end{macrocode}
%
% \begin{macro}{\th@change}
%    The \texttt{change} theorem style corresponds to the
%    \texttt{change break} style without a linebreak after the header.
%    To say it in another way, it's the same as the \texttt{plain}
%    style but with number and name interchanged and |\slshape| as the
%    default font.
%    \begin{macrocode}
\gdef\th@change{\normalfont\slshape
  \def\@begintheorem##1##2{\item
       [\hskip\labelsep \theorem@headerfont ##2\ ##1]}%
\def\@opargbegintheorem##1##2##3{%
  \item[\hskip\labelsep \theorem@headerfont ##2\ ##1\ (##3)]}}
\endgroup
%</thc>
%    \end{macrocode}
% \end{macro}
%
%
% \subsubsection{The \texttt{marginbreak} style}
%
% This style option is the one used most often at Mainz.  It is saved
% in the file |thmb.sty|.
%    \begin{macrocode}
%<*thmb>
\begingroup \makeatletter
\@ifundefined{theorem@style}{% \iffalse meta-comment
%
% Copyright (C) 1993-2024
%
% The LaTeX Project and any individual authors listed elsewhere
% in this file.
%
% This file is part of the Standard LaTeX `Tools Bundle'.
% -------------------------------------------------------
%
% It may be distributed and/or modified under the
% conditions of the LaTeX Project Public License, either version 1.3c
% of this license or (at your option) any later version.
% The latest version of this license is in
%    https://www.latex-project.org/lppl.txt
% and version 1.3c or later is part of all distributions of LaTeX
% version 2005/12/01 or later.
%
% The list of all files belonging to the LaTeX `Tools Bundle' is
% given in the file `manifest.txt'.
%
% \fi
%
%%
%
%
\def\FMithmInfo{2023/07/05 v2.2c Theorem extension package (FMi)}
%
% \ProvidesFile{theorem.dtx}[\FMithmInfo]
%\iffalse   % this is a METACOMMENT !
%%
%% Package `theorem' to use with LaTeX2e
%% Copyright (C) 1989-1995 Frank Mittelbach, all rights reserved.
%<+package>\@ifundefined{theorem@style}{}{\endinput}
%<+package>\NeedsTeXFormat{LaTeX2e}[1995/06/01]
%<+package>\ProvidesPackage{theorem}
%<+package>         [\FMithmInfo]
%
% \fi
%
% \changes{v1.0}{87/05/07}{First draft.}
% \changes{v1.0a}{87/05/07}{`thistheoremstyle' added.}
% \changes{v1.1a}{87/07/29}{New concept: redefine the internal
%                          \protect\LaTeX{} commands `@xnthm' etc.}
% \changes{v1.1b}{87/07/31}{`theoremfont' used to specify font.}
% \changes{v1.2a}{88/01/08}{`@endtheorem' now globally defined.}
% \changes{v1.2b}{88/01/11}{`newtheorem' only allowed in the preamble.}
% \changes{v1.3a}{88/02/14}{`theoremfont' now a toks register.}
%
% \changes{v2.0a}{88/11/16}{New implementation and new documentation.}
% \changes{v2.0a}{88/11/16}{No layout changes but internal modifications
%                           old substyles won't work.}
% \changes{v2.0b}{89/01/30}{`global' used for all definitions.}
% \changes{v2.0c}{89/06/15}{Documentation translated to English.
%                            Thanks to LLL and Nina}
% \changes{v2.1a}{90/02/15}{Support for new font selection.}
% \changes{v2.2a}{94/02/03}{Cleanup for LaTeX2e}
%
% \DoNotIndex{\@endparenv,\@ifdefinable,\@ifnextchar,\@ifundefined}
% \DoNotIndex{\@namedef,\@nameuse,\@spaces,\@tempa,\@warning}
% \DoNotIndex{\ }
% \DoNotIndex{\begingroup,\bfseries}
% \DoNotIndex{\csname}
% \DoNotIndex{\def,\do,\docdate}
% \DoNotIndex{\endcsname,\endgroup,\endinput,\endtrivlist,\expandafter}
% \DoNotIndex{\filedate,\fileversion}
% \DoNotIndex{\gdef,\global}
% \DoNotIndex{\hbox,\hskip}
% \DoNotIndex{\ignorespaces,\input,\itshape,\item}
% \DoNotIndex{\labelsep,\let,\llap}
% \DoNotIndex{\makeatletter}
% \DoNotIndex{\noexpand}
% \DoNotIndex{\refstepcounter,\relax,\rlap}
% \DoNotIndex{\setlength,\slshape,\space,\string,\strut}
% \DoNotIndex{\the,\trivlist,\typeout}
% \DoNotIndex{\vbox}
% \DoNotIndex{\xdef}
%
% \def\deflabel#1{\ttfamily #1\hfill}
% \def\deflist#1{\begin{list}{}{\settowidth\labelwidth{\ttfamily #1}%
%                               \setlength\leftmargin\labelwidth
%                               \addtolength\leftmargin\labelsep
%                               \let\makelabel\deflabel}}
% \def\enddeflist{\end{list}}
% \def\envfont{\normalfont\ttfamily}
%
% \MakeShortVerb{\|}
%
% \newcommand{\TUB}{{\sl TUGboat\/}}
%
% \GetFileInfo{theorem.dtx}
% \title{An Extension of the
%        \LaTeX{} theorem environment\thanks{This file
%        has version number \fileversion, last
%        revised \filedate.}}
% \author{Frank Mittelbach}
%
% \MaintainedByLaTeXTeam{tools}
% \maketitle
%
% \begin{abstract}
%    The macros described in this paper yield an extension of the
%    \LaTeX{} theorem mechanism. It is designed to satisfy the
%    different requirements of various journals. Thus, the
%    layout of the ``theorems'' can be manipulated by determining a
%    ``style''. This article describes not only the use, but
%    also the definition, of the necessary macros.
% \end{abstract}
%
% \section*{Preface to version 2.2}
%
% For \LaTeXe{} this package did not need any fundamental changes. I
% only modified the messages generated so that theorem layout styles
% will show up with the |\listfiles| command and cleaned the section
% on the New Font Selection Scheme since this is now included in
% \LaTeX.
%
% \section*{Preface to version 2.1}
%
% This version is identical to 2.0g described in \TUB\ 10\#3 except
% for some internal defaults which are now set depending on the used
% font selection scheme.
%
% This was done to avoid unpleasant surprises if the new font selection
% scheme is in force. For further details see
% section~\ref{sec:newfonts} and \cite{bk:GMS94}.
%
% \section{Introduction}
%
% For our purposes here, ``theorems'' are labelled enunciations,
% often set off from the main text by extra space and a font change.
% Theorems, corollaries, conjectures, definitions, and remarks
% are all instances of ``theorems''.  The ``header'' of these structures
% is composed of a label (such as \textsc{Theorem} or \textsc{Remark})
% and a number which serializes an item in the sequence of items
% with the same label.
%
% Shortly after the introduction of \LaTeX{} at the Fachbereich
% Mathematik in Mainz, the desire to manipulate the layout of
% ``theorems'' arose.  In Mainz, the following two conventions came
% into general use:
% \begin{enumerate}
%   \item The number of the theorem is shown in the margin.
%   \item There is a line break at
%         the end of the theorem header.
% \end{enumerate}
%
% Additionally, some journals require different formats which depend on
% the ``sort of theorem'': e.g.\ often remarks and definitions are set
% in |\upshape|, while |\itshape| is employed for main theorems.
%
% Confronted with these requirements, a theorem environment was
% developed in Mainz which allows separate determination of the
% layout of the ``theorems sets'', comparable to |\pagestyle|.
%
%
% \section{The user interface}
% \subsection{Defining new theorem sets}
%
% \DescribeMacro\newtheorem
% As in the original \LaTeX{} version, the command |\newtheorem|
% defines a new ``theorem set'' or ``theorem-like structure''.
% Two required arguments name the new environment and give the text
% to be typeset with each instance of the new ``set'', while
% an optional argument determines how
% the ``set'' is enumerated:
% \begin{description}
%    \item[\ttfamily \bslash newtheorem\{foo\}\{bar\}]
%       The theorem set {\envfont foo} (whose name is \texttt{bar})
%       uses its own counter.
%    \item[\ttfamily \bslash newtheorem\{foo2\}{[foo]}\{bar2\}]
%       The theorem set {\envfont foo2} (printed name \texttt{bar2})
%       uses the same counter as the theorem set \texttt{foo}.
%    \item[\ttfamily \bslash newtheorem\{foo3\}\{bar3\}{[section]}]
%       The theorem set {\envfont foo3} (printed name \texttt{bar3}) is
%       enumerated within the counter \texttt{section}, i.e.\ with every
%       new |\section| the enumeration begins again with 1, and
%       the enumeration is composed from the section-number and the
%       theorem counter itself.
% \end{description}
%
% \DescribeMacro\theoremstyle
% Additionally, the command |\theoremstyle| can define the layout
% of various, or all, theorem sets.  It should be noted that any
% theorem set defined by |\newtheorem| is typeset in the
% |\theoremstyle| that is current at the time of the definition.
% Thus, the following
% \begin{verbatim}
%    \theoremstyle{break}        \newtheorem{Cor}{Corollary}
%    \theoremstyle{plain}        \newtheorem{Exa}{Example}[section]
%\end{verbatim}                             ^^A No blank at line start
% leads to the result that the set {\envfont Cor} is formatted in the
% style \texttt{break}, while the set {\envfont Exa} and all the
% following ones are formatted in the style \texttt{plain}, unless
% another |\theoremstyle| follows.  Since the definitions installed by
% |\newtheorem| are global, one also can limit |\theoremstyle| locally
% by grouping braces.
%
% \DescribeMacro\theorembodyfont
% The choice of the font for the theorem body is completely
% independent of the chosen |\theoremstyle|; this has proven to
% be very advantageous.  For example,
% \begin{verbatim}
%   {\theorembodyfont{\upshape}          \newtheorem{Rem}{Remark}}
%\end{verbatim}
% defines a theorem set {\envfont Rem}, which will be set in
% |\upshape| in the current layout (which in our example is
% \texttt{plain}).  As with |\theoremstyle|, the |\theorembodyfont|
% chosen is that current at the time of |\newtheorem|.  If
% |\theorembodyfont| is not specified or one defines
% |\theorembodyfont{}|, then the font used will be that defined by the
% |\theoremstyle|.
%
% \DescribeMacro\theoremheaderfont It is also possible to customize
% the font used for the theorem headers.  This is, however, a global
% declaration, and therefore there should be at most one
% |\theoremheaderfont| declaration in the preamble.\footnote{If it is
% actually necessary to have different header fonts, one has to define
% new theorem styles (substituting the desired font) or specify the
% information directly in the \texttt{\bslash newtheorem} declaration
% (the unclean variant).}
%
% \DescribeMacro\theorempreskipamount
% \DescribeMacro\theorempostskipamount Two additional parameters
% affect the vertical space around the theorem environments:
% |\theorempreskipamount| and |\theorempostskipamount| define,
% respectively, the spacing before and after such an environment.
% These parameters apply for all theorem sets and can be manipulated
% with the ordinary length macros.  They are rubber lengths
% (`\textsf{skips}'), and therefore can contain \texttt{plus} and
% \texttt{minus} parts.
%
% Since the definition of theorem sets should---most sensibly---be
% placed in the preamble, we only allow installation there.
% It is therefore possible to release the memory used here after
% |\begin{document}|, in order to make room for other
% applications.
%
%
% \subsection{Existing theorem styles}
%
% The following theorem styles exist to date:
% \begin{deflist}{marginbreak}
%    \item[plain]
%       This theorem style emulates the original \LaTeX{} definition,
%       except that additionally the parameters
%       |\theorem...skipamount| are used.
%    \item[break]
%       In this style, the theorem header is followed by a line
%       break.
%    \item[marginbreak]
%       The theorem number is set in the margin, and there is a line
%       break as in \texttt{break}.
%    \item[changebreak]
%       Like \texttt{break}, but with header number and text
%       interchanged.
%    \item[change]
%       Header number and text are interchanged, without a line break.
%    \item[margin]
%       The number is set in the left margin, without a line break.
% \end{deflist}
% All styles (except \texttt{plain}) select |\slshape| as the default
% |\theorembodyfont|.
%
%
% \subsection{Examples}
%
% Given the above theorem sets {\envfont Cor}, {\envfont Exa}
% and {\envfont Rem},
% suppose that the preamble also contains the declarations:
% \begin{verbatim}
%    \theoremstyle{marginbreak}   \newtheorem{Lem}[Cor]{Lemma}
%    \theoremstyle{change}
%    \theorembodyfont{\itshape}        \newtheorem{Def}[Cor]{Definition}
%
%    \theoremheaderfont{\scshape}
%\end{verbatim}
% Then the following are some typical examples of the typeset output
% resulting from their use.
%
% \begin{Cor}
%    This is a sentence typeset in the theorem environment {\envfont
%    Cor}.
% \end{Cor}
%
% \begin{Exa}
%    This is a sentence typeset in the theorem environment {\envfont
%    Exa}.
% \end{Exa}
%
% \begin{Rem}
%    This is a sentence typeset in the theorem environment {\envfont
%    Rem}.
% \end{Rem}
%
% \begin{Lem}[Ben User]
%    This is a sentence typeset in the theorem environment {\envfont
%    Lem}.
% \end{Lem}
%
% \begin{Def}[Very impressive Definition]
%    This is a sentence typeset in the theorem environment {\envfont
%    Def}.
% \end{Def}
%
% The last two examples show the effect of the optional argument to
% a theorem environment (it is the text typeset in parentheses).
%
%
% \section{Special Considerations}
% \label{sec:newfonts}
%
% Theoremheader and body are implemented as a unit. This means that
% the |\theoremheaderfont|  will inherit characteristics of the
% |\theorembodyfont| in \LaTeXe. Thus, if for example
% |\theorembodyfont| is |\itshape| and |\theoremheaderfont|
% is |\bfseries| the font selected for the header will have the
% characteristics `bold extended italic'. If this is not desired one
% should set the |\theoremheaderfont| to something like
% \begin{verbatim}
%      \theoremheaderfont{\normalfont\bfseries}
%\end{verbatim}
% i.e.\ supplying all necessary font information explicitly.
%
%
%
% \section{Acknowledgements}
%
% The publication of this set of macros was only possible with the
% help of Christina Busse (translating the manuscript into English),
% Joachim Pense (playing the r\^ole of typist), Chris Rowley
% (looking everything over) and many others providing useful
% suggestions.
%
% \MaybeStop{%
%    \begin{thebibliography}{9}
%       \bibitem{bk:GMS94} \textsc{M.~Goossens}, \textsc{F.~Mittelbach}
%         and \textsc{A.~Samarin}.
%         \newblock The \LaTeX{} Companion.
%         \newblock
%         Addison-Wesley, Reading, Massachusetts, 1994.
%       \bibitem{latex:source} \textsc{Lamport, Leslie.} \newblock
%                               \texttt{latex.tex}, version 2.09, date
%                               Feb. 1990.
%    \end{thebibliography}
%    \PrintIndex
%    \PrintChanges}
%
%
% \section{The documentation driver file}
%
% The next bit of code contains the documentation driver file for
% \TeX{}, i.e., the file that will produce the documentation you are
% currently reading. It will be extracted from this file by the
% \texttt{docstrip} program. Since it is the first code in the file
% one can alternatively process this file directly with \LaTeXe{} to
% obtain the documentation.
%
%    \begin{macrocode}
%<*driver>
\documentclass{ltxdoc}

\usepackage{theorem}

%<+driver>% The next few lines define theorem sets which are used
%<+driver>% in the example section of the documentation.

\theoremstyle{break}       \newtheorem{Cor}{Corollary}
\theoremstyle{plain}       \newtheorem{Exa}{Example}[section]
{\theorembodyfont{\upshape}\newtheorem{Rem}{Remark}}
\theoremstyle{marginbreak} \newtheorem{Lem}[Cor]{Lemma}
\theoremstyle{change}
\theorembodyfont{\itshape} \newtheorem{Def}[Cor]{Definition}

\theoremheaderfont{\scshape}

\RecordChanges

\begin{document}
   \DocInput{theorem.dtx}
\end{document}
%</driver>
%    \end{macrocode}
%
% \section{Definition of the Macros}
%
% If the file has been loaded before, we abort immediately. If not the
% package announces itself (this is actually done at the very top of
% the file---the way it is done isn't good style so don't copy it).
% \changes{v2.0e}{89/07/19}{Spaces removed from `typeout'.}
% \changes{v2.2b}{95/11/19}{Announce moved to top of file}
%    \begin{macrocode}
%<*package>
%\@ifundefined{theorem@style}{}{\endinput}
%\def\FMithmInfo{1995/11/19 v2.2b Theorem extension package (FMi)}
%\ProvidesPackage{theorem}[\FMithmInfo]
%    \end{macrocode}
%
%
% \subsection{Definition of theorem styles and fonts}
%
% All the definitions in this file are done globally to allow
% inputting this file inside a group.
%
% \begin{macro}{\theoremstyle}
%    Before a theorem style can be installed, the chosen style must be
%    known. For that reason, we must test to see that
%    |\th@|\meta{style} is known or, more precisely, that it is
%    different from |\relax|. If the style is not known then
%    |\th@plain| is used.
%    \begin{macrocode}
\gdef\theoremstyle#1{%
   \@ifundefined{th@#1}{\@warning
          {Unknown theoremstyle `#1'. Using `plain'}%
          \theorem@style{plain}}%
%    \end{macrocode}
%    We save the theorem style to be used in the token register
%    |\theorem@style|.
%    \begin{macrocode}
      {\theorem@style{#1}}%
%    \end{macrocode}
%    Now we ``evaluate'' the theorem style: this means, we call the
%    macro |\th@|\meta{style} which will activate the relevant
%    definitions which are contained in a separate file.  This is done
%    in a group to suppress changes to the current font.  This could
%    otherwise pose problems together with the new font selection
%    scheme\footnote{When I printed the original article using the new
%    font selection scheme I ended with a document with slanted
%    typefaces (text headings and all) simply because one of the
%    theorem styles used \texttt{\bslash sl} at toplevel.} if the
%    |\th@|\meta{style} is evaluated a second time.
% \changes{v2.1a}{90/02/15}{Group added around `th@..}
% \changes{v2.1a}{90/02/15}{Reseting of `..@begintheorem removed.}
%    \begin{macrocode}
      \begingroup
        \csname th@\the\theorem@style \endcsname
      \endgroup}
%    \end{macrocode}
% \end{macro}
%
%
% \begin{macro}{\@begintheorem}
% \changes{v2.1a}{90/02/15}{Relaxed at toplevel.}
% \begin{macro}{\@opargbegintheorem}
% \changes{v2.1a}{90/02/15}{Relaxed at toplevel.}
%    We reset |\@begintheorem| and |\@opargbegintheorem| to
%    |\relax| since these commands are no longer necessary at
%    toplevel. This will save a few tokens.
%    \begin{macrocode}
\global\let\@begintheorem\relax
\global\let\@opargbegintheorem\relax
%    \end{macrocode}
% \end{macro}
% \end{macro}
%
% \begin{macro}{\theorem@style}
%    Obviously the \textsf{token} register used above has to be
%    allocated. To assure the utmost compatibility with the
%    original \LaTeX{} definition, we set the default theorem style to
%    \texttt{plain}, which implements the usual \LaTeX{} convention.
% \changes{v2.0f}{89/07/20}{Unnecessary `global' removed.}
%    \begin{macrocode}
\newtoks\theorem@style
\global\theorem@style{plain}
%    \end{macrocode}
% \end{macro}
%
% \begin{macro}{\theorembodyfont}
% \begin{macro}{\theorem@bodyfont}
%    If the |\theorembodyfont| is set by the user then it should not
%    interact with the default font set in the theorem style. When the
%    new font selection is in force this may happen if, for example,
%    the default is |\itshape| and the new |\theorembodyfont| is
%    |\sffamily|. So we add a |\reset@font| command in front of the
%    user definition.
% \changes{v2.0f}{89/07/20}{Unnecessary `global' removed.}
% \changes{v2.0g}{89/09/19}{`theoremfont to `theorembodyfont.}
% \changes{v2.1c}{92/06/22}{`reset@font added and logic changed.}
% \changes{v2.1d}{92/10/28}{Reset to empty if arg is empty}
%    \begin{macrocode}
\gdef\theorembodyfont#1{%
%    \end{macrocode}
%    We check if the argument supplied is empty and if so put nothing
%    into the |\theorem@bodyfont| token register to allow for
%    |\theorembodyfont{}| as a mean of using the default of the
%    current |\theoremstyle|.
%    \begin{macrocode}
   \def\@tempa{#1}%
   \ifx\@tempa\@empty
    \theorem@bodyfont{}%
   \else
    \theorem@bodyfont{\reset@font#1}%
   \fi
   }
\newtoks\theorem@bodyfont
\global\theorem@bodyfont{}
%    \end{macrocode}
% \end{macro}
% \end{macro}
%
%
% \begin{macro}{\theoremheaderfont}
% \changes{v2.0g}{89/09/19}{Macro added.}
%    The font for the theorem headers is handled differently because
%    this definition applies to all theorem styles.
%    \begin{macrocode}
\gdef\theoremheaderfont#1{\gdef\theorem@headerfont{#1}%
%    \end{macrocode}
%    After using the macro once it is redefined to produce an error
%    message.
%    \begin{macrocode}
       \gdef\theoremheaderfont##1{%
        \typeout{\string\theoremheaderfont\space should be used
                 only once.}}}
%    \end{macrocode}
% \end{macro}
%
% \begin{macro}{\theorem@headerfont}
% \changes{v2.0g}{89/09/19}{Macro added.}
%    To set the |\theorem@headerfont| default we first test if
%    the new fontselection scheme is in force.
% \changes{v2.1a}{90/02/15}{Support for new font selection.}
%    \begin{macrocode}
\ifx\upshape\undefined
%    \end{macrocode}
%    If not we define it to expand into |\bfseries|. We don't use
%    |\let| just in case a following style option redefines this
%    macro.
%    \begin{macrocode}
\gdef\theorem@headerfont{\bfseries}
%    \end{macrocode}
%    Otherwise we reset the current shape before calling |\bfseries|.
%    \begin{macrocode}
\else \gdef\theorem@headerfont{\normalfont\bfseries}\fi
%    \end{macrocode}
% \end{macro}
%
% \begin{macro}{\th@plain}
% \begin{macro}{\th@break}
% \begin{macro}{\th@marginbreak}
% \begin{macro}{\th@changebreak}
% \begin{macro}{\th@change}
% \begin{macro}{\th@margin}
%    The different styles are defined in macros such as
%    |\th@plain|.  Since memory space is precious in
%    ``non-Big-versions'', we have to avoid offering too many unused
%    definitions.  Therefore we define these styles in separate files
%    that can be loaded on demand.  Thus the commands themselves only
%    load these files. We use |\@input@|, a \LaTeXe{} internal command
%    that ensures that the file will be listed with |\listfiles|.
% \changes{v2.2a}{94/02/02}{Use `@input@ to load theorem layout files}
%    \begin{macrocode}
\gdef\th@plain{\@input@{thp.sty}}
\gdef\th@break{\@input@{thb.sty}}
\gdef\th@marginbreak{\@input@{thmb.sty}}
\gdef\th@changebreak{\@input@{thcb.sty}}
\gdef\th@change{\@input@{thc.sty}}
\gdef\th@margin{\@input@{thm.sty}}
%    \end{macrocode}
%    This list will be expanded when new styles become
%    available.  For testing,  just append new theorem substyles
%    as document options.
% \end{macro}
% \end{macro}
% \end{macro}
% \end{macro}
% \end{macro}
% \end{macro}
%
%
%
% \subsection{Definition of a new theorem set}
%
% As already pointed out, a new theorem environment can be defined in
% three different ways:
% \begin{verbatim}
%    \newtheorem{Lem}{Lemma}
%    \newtheorem{Lem}{Lemma}[section]
%    \newtheorem{Lem}[Theorem]{Lemma}
%\end{verbatim}
% The function of the macro |\newtheorem| is to recognize these cases
% and then to branch into one of the three macros |\@ynthm|, |\@xnthm|
% or |\@othm|.  This mechanism is adopted unchanged from
% \cite{latex:source}; the essential point here is that, for example,
% in the second case, the arguments \texttt{Lem}, \texttt{Lemma} and
% \texttt{section} are passed over to the macro |\@xnthm|.
%
% We inspect this case first because the others present fewer problems,
% and thus are easily derived from this one.
%
% \begin{macro}{\@xnthm}
%    For our example arguments, the macro |\@xnthm| must fulfill
%    the following:
%    \begin{itemize}
%       \item
%          Define a new \LaTeX-counter `Lem'
%       \item
%          reset this counter within a |\section|
%       \item
%          define the macro |\theLem|
%       \item
%          define the environment macros |\Lem| and
%          |\endLem| using the current
%          |\theoremstyle| and |\theorem@bodyfont|.
%    \end{itemize}
%    Obviously, all this should happen only if the first argument
%    of |\@xnthm| (i.e.\ \texttt{Lem} in our example) is chosen so
%    as not to conflict with any previously defined commands or
%    environments.  This test is performed by the \LaTeX{} macro
%    |\@ifdefinable|.
%    \begin{macrocode}
\gdef\@xnthm#1#2[#3]{\expandafter\@ifdefinable\csname #1\endcsname
%    \end{macrocode}
%    Therefore, the first argument of |\@ifdefinable| is the
%    expansion (in the example, |\Lem|) of
%    |\csname||#1\endcsname|.  The second argument is
%    executed only if the test has been completed successfully.
%    \begin{macrocode}
   {%
%    \end{macrocode}
%    Now we define the new counter. The names of the \LaTeX{}
%    macros employed should speak for themselves:
% \changes{v2.2b}{1995/10/07}{Use \cs{@newctr}}
% \changes{v2.2c}{1995/11/23}{Use \cs{@newctr} properly.}
%    \begin{macrocode}
    \@definecounter{#1}\@newctr{#1}[#3]%
%    \end{macrocode}
%    Using |\@newctr| will give a proper error message if the counter
%    in |#3| is not defined.
%    In defining `|\theLem|' we must generate the desired macro
%    name by use of |\expandafter| and |\csname|.
%    \begin{macrocode}
    \expandafter\xdef\csname the#1\endcsname
%    \end{macrocode}
%    An |\xdef| is used in order to make the definition global,
%    and to ensure that it contains the replacement texts of
%    |\@thmcountersep| and |\@thmcounter|.\footnote{These
%    two macros can be defined by the document style.  Their default
%    values produce a `\texttt{.}' as separation and an arabic
%    representation of the number.} However, not everything should be
%    expanded. For example, it saves space to use |\thesection|
%    instead of its---at times---lengthy expansion.
%    \begin{macrocode}
      {\expandafter \noexpand \csname the#3\endcsname
       \@thmcountersep \@thmcounter{#1}}%
%    \end{macrocode}
%    Thus with the defaults of \LaTeX{}, |\theLem| would be
%    replaced by the command sequence |\thesection.\arabic{Lem}|.
%
%    We will now look at the definition of the macro which is executed
%    at the beginning of the actual environment (in our example this
%    macro is |\Lem|). It should be noted that we use an
%    ``|\expandafter| trick'' to expand only certain parts of the
%    replacement text at the time of the definition.
%    \begin{macrocode}
    \def\@tempa{\global\@namedef{#1}}%
    \expandafter \@tempa \expandafter{%
%    \end{macrocode}
%    First, the macro that contains the current definitions of
%    |\@begintheorem| and |\@opargtheorem| should be called
%    up.  The name of this macro---as is already known---has the form
%    |\th@|\meta{theorem\ style}; therefore, it must be called by
%    \begin{macrocode}
      \csname th@\the \theorem@style
%    \end{macrocode}
%    In addition the default theorem font should be changeable, i.e.\
%    we have to insert the contents of |\theorem@bodyfont|. For
%    that reason, we expand even further, beyond |\endcsname|,
%    and thus insert the contents of the \textsf{token} register
%    |\theorem@bodyfont| in the replacement text.
%    \begin{macrocode}
            \expandafter \endcsname \the \theorem@bodyfont
%    \end{macrocode}
%    Now it is time to call the macro |\@thm| which takes over
%    the further processing.  It has two arguments: the current
%    counter name (in our example, \texttt{Lem}), and the text of the
%    label (in our example, \texttt{Lemma}).
%    \begin{macrocode}
     \@thm{#1}{#2}}%
%    \end{macrocode}
%    With this, the `sub-definition' is complete.  The macro
%    |\@endtheorem| ends a theorem environment and is, so far, nothing
%    but an |\endtrivlist|.  (Hence it is defined globally, and not
%    within the theorem styles.\footnote{This has to be changed as
%    soon as theorem styles that change \texttt{\bslash @endtheorem}
%    exist.  In such a case, all existing styles must be changed as
%    well since they will have to reset the macro.}) Therefore, we can
%    set it equivalent to the macro that ends the theorem set (in our
%    example, |\endLem|).  However, if some day theorem styles exist
%    that do change |\@endtheorem|, we would have to use the
%    commented-out line instead.
%    \begin{macrocode}
    \global \expandafter \let \csname end#1\endcsname \@endtheorem
%   \global\@namedef{end#1}{\@endtheorem}%
%    \end{macrocode}
%    With these commands all the required definitions are employed,
%    unless the test |\@ifdefinable| has failed.  Therefore, we
%    end the second argument of this macro and with it the definition
%    of |\@xnthm|.
%    \begin{macrocode}
   }}
%    \end{macrocode}
% \end{macro}
%
% \begin{macro}{\@ynthm}
%    The definition of |\@ynthm| is completely analogous. In
%    this case the new counter that is defined is not reset within
%    another counter; thus the definition of |\the...|
%    is simplified:
%    \begin{macrocode}
\gdef\@ynthm#1#2{\expandafter\@ifdefinable\csname #1\endcsname
   {\@definecounter{#1}%
    \expandafter\xdef\csname the#1\endcsname{\@thmcounter{#1}}%
%    \end{macrocode}
%    The rest of the definition corresponds literally to that of
%    |\@xnthm|:
%    \begin{macrocode}
    \def\@tempa{\global\@namedef{#1}}\expandafter \@tempa
     \expandafter{\csname th@\the \theorem@style \expandafter
     \endcsname \the\theorem@bodyfont \@thm{#1}{#2}}%
    \global \expandafter \let \csname end#1\endcsname \@endtheorem}}
%    \end{macrocode}
% \end{macro}
%
% \begin{macro}{\@othm}
%    The definition of |\@othm| does not contain anything new.
%    \begin{macrocode}
\gdef\@othm#1[#2]#3{%
%    \end{macrocode}
% \changes{v2.2b}{95/10/07}{Build in check for counter defined}
%    We do not define a new counter but instead use one that has
%    already been defined.  Thus the only definition we need is that
%    of this pseudo-counter (i.e.\ |\the|\meta{env.\ name}).
%    First we check if |#2| corresponds to a known counter name.
%    \begin{macrocode}
  \expandafter\ifx\csname c@#2\endcsname\relax
   \@nocounterr{#2}%
  \else
   \expandafter\@ifdefinable\csname #1\endcsname
   {\expandafter \xdef \csname the#1\endcsname
     {\expandafter \noexpand \csname the#2\endcsname}%
%    \end{macrocode}
%    All other parts of the definition can be adopted from
%    |\@xnthm|. We have to remember, though, that in this case
%    the name of the current counter and the theorem label have moved
%    to the second and third arguments.
%    \begin{macrocode}
    \def\@tempa{\global\@namedef{#1}}\expandafter \@tempa
     \expandafter{\csname th@\the \theorem@style \expandafter
     \endcsname \the\theorem@bodyfont \@thm{#2}{#3}}%
    \global \expandafter \let \csname end#1\endcsname \@endtheorem}%
  \fi}
%    \end{macrocode}
% \end{macro}
%
% \subsection{Macros that are employed in a {\envfont theorem}
%    environment}
%
% \begin{macro}{\@thm}
%    The macro |\@thm| has to increase the current counter.  Then,
%    depending on whether the environment has (or does not have) an
%    optional argument, it has to branch into either |\@begintheorem|
%    or |\@opargtheorem|.
%    \begin{macrocode}
\gdef\@thm#1#2{\refstepcounter{#1}%
%    \end{macrocode}
%    Now we start a {\envfont trivlist} environment, and give
%    |\@topsep| and |\@topsepadd| the values of the \textsf{skip}
%    registers |\theorempreskipamount| and |\theorempostskipamount|.
%    The value in |\@topsep| is the vertical space that is inserted by
%    the first (and only) |\item| in our |\trivlist| whilst
%    |\@topsepadd| is inserted by |\@endparenv| at the end of that
%    {\envfont trivlist} environment.  By using these registers, we
%    obtain the desired space around a {\envfont theorem} environment.
%    \begin{macrocode}
   \trivlist
   \@topsep \theorempreskipamount               % used by first \item
   \@topsepadd \theorempostskipamount           % used by \@endparenv
%    \end{macrocode}
%    Now we have to test whether an optional argument has been given.
% \changes{v2.0d}{89/07/18}{Forgotten percent added.}
%    \begin{macrocode}
   \@ifnextchar [%
%    \end{macrocode}
%    If there is an optional argument, we will call |\@ythm|, and
%    move the arguments read back into the input stream.
%    \begin{macrocode}
   {\@ythm{#1}{#2}}%
%    \end{macrocode}
%    If not, we call |\@begintheorem|. Its first argument is the
%    name of the theorem set (hence the second argument of
%    |\@thm|). Its second argument is the macro that produces the
%    current number.
%    \begin{macrocode}
   {\@begintheorem{#2}{\csname the#1\endcsname}\ignorespaces}}
%    \end{macrocode}
% \end{macro}
%
% \begin{macro}{\@xthm}
% \begin{macro}{\@ythm}
%    Both these macros were originally called by |\@thm|.  We do
%    not need |\@xthm| anymore, hence we reset it to
%    |\relax|. The definition of |\@ythm| has not changed at
%    all from its definition in \LaTeX. In order to make the macros
%    easier to understand, we will
%    nevertheless present it (commented out).
%    \begin{macrocode}
\global\let\@xthm\relax
% \def\@ythm#1#2[#3]{\@opargbegintheorem{#2}{\csname
%           the#1\endcsname}{#3}\ignorespaces}
%</package>
%    \end{macrocode}
%    The primitive |\ignorespaces| in |\@ythm| and
%    |\@thm| is needed to remove the spaces between the
%    |\begin{...}| and the actual text.
% \end{macro}
% \end{macro}
%
%
% \subsection{Definition of the theorem substyles}
%
% As already pointed out, the theorem substyles, defined below, are
% only loaded when necessary.  Note that all these substyles, except
% \texttt{plain}, have |\slshape| as the default body font.
%
% \subsubsection{The \texttt{plain} style}
%
% \label{dok:th@plain} As the following macros use |@|, we have
% to locally set the |\catcode| of this symbol to ``letter''.
% This happens within a group, so that we do not have to worry about
% which |\catcode| that symbol had before.
%    \begin{macrocode}
%<*thp>
\begingroup \makeatletter
%    \end{macrocode}
% Since we are now within a group, we must make all definitions
% globally.  First we make sure that \texttt{theorem.sty} is loaded.
% This will allow us to use this file as a document style option
% without having to call \texttt{theorem} itself as an option.  At the
% same time, we assure that at least version 2 is loaded, since
% |\theorem@style| was not defined in earlier versions.
%    \begin{macrocode}
\@ifundefined{theorem@style}{% \iffalse meta-comment
%
% Copyright (C) 1993-2024
%
% The LaTeX Project and any individual authors listed elsewhere
% in this file.
%
% This file is part of the Standard LaTeX `Tools Bundle'.
% -------------------------------------------------------
%
% It may be distributed and/or modified under the
% conditions of the LaTeX Project Public License, either version 1.3c
% of this license or (at your option) any later version.
% The latest version of this license is in
%    https://www.latex-project.org/lppl.txt
% and version 1.3c or later is part of all distributions of LaTeX
% version 2005/12/01 or later.
%
% The list of all files belonging to the LaTeX `Tools Bundle' is
% given in the file `manifest.txt'.
%
% \fi
%
%%
%
%
\def\FMithmInfo{2023/07/05 v2.2c Theorem extension package (FMi)}
%
% \ProvidesFile{theorem.dtx}[\FMithmInfo]
%\iffalse   % this is a METACOMMENT !
%%
%% Package `theorem' to use with LaTeX2e
%% Copyright (C) 1989-1995 Frank Mittelbach, all rights reserved.
%<+package>\@ifundefined{theorem@style}{}{\endinput}
%<+package>\NeedsTeXFormat{LaTeX2e}[1995/06/01]
%<+package>\ProvidesPackage{theorem}
%<+package>         [\FMithmInfo]
%
% \fi
%
% \changes{v1.0}{87/05/07}{First draft.}
% \changes{v1.0a}{87/05/07}{`thistheoremstyle' added.}
% \changes{v1.1a}{87/07/29}{New concept: redefine the internal
%                          \protect\LaTeX{} commands `@xnthm' etc.}
% \changes{v1.1b}{87/07/31}{`theoremfont' used to specify font.}
% \changes{v1.2a}{88/01/08}{`@endtheorem' now globally defined.}
% \changes{v1.2b}{88/01/11}{`newtheorem' only allowed in the preamble.}
% \changes{v1.3a}{88/02/14}{`theoremfont' now a toks register.}
%
% \changes{v2.0a}{88/11/16}{New implementation and new documentation.}
% \changes{v2.0a}{88/11/16}{No layout changes but internal modifications
%                           old substyles won't work.}
% \changes{v2.0b}{89/01/30}{`global' used for all definitions.}
% \changes{v2.0c}{89/06/15}{Documentation translated to English.
%                            Thanks to LLL and Nina}
% \changes{v2.1a}{90/02/15}{Support for new font selection.}
% \changes{v2.2a}{94/02/03}{Cleanup for LaTeX2e}
%
% \DoNotIndex{\@endparenv,\@ifdefinable,\@ifnextchar,\@ifundefined}
% \DoNotIndex{\@namedef,\@nameuse,\@spaces,\@tempa,\@warning}
% \DoNotIndex{\ }
% \DoNotIndex{\begingroup,\bfseries}
% \DoNotIndex{\csname}
% \DoNotIndex{\def,\do,\docdate}
% \DoNotIndex{\endcsname,\endgroup,\endinput,\endtrivlist,\expandafter}
% \DoNotIndex{\filedate,\fileversion}
% \DoNotIndex{\gdef,\global}
% \DoNotIndex{\hbox,\hskip}
% \DoNotIndex{\ignorespaces,\input,\itshape,\item}
% \DoNotIndex{\labelsep,\let,\llap}
% \DoNotIndex{\makeatletter}
% \DoNotIndex{\noexpand}
% \DoNotIndex{\refstepcounter,\relax,\rlap}
% \DoNotIndex{\setlength,\slshape,\space,\string,\strut}
% \DoNotIndex{\the,\trivlist,\typeout}
% \DoNotIndex{\vbox}
% \DoNotIndex{\xdef}
%
% \def\deflabel#1{\ttfamily #1\hfill}
% \def\deflist#1{\begin{list}{}{\settowidth\labelwidth{\ttfamily #1}%
%                               \setlength\leftmargin\labelwidth
%                               \addtolength\leftmargin\labelsep
%                               \let\makelabel\deflabel}}
% \def\enddeflist{\end{list}}
% \def\envfont{\normalfont\ttfamily}
%
% \MakeShortVerb{\|}
%
% \newcommand{\TUB}{{\sl TUGboat\/}}
%
% \GetFileInfo{theorem.dtx}
% \title{An Extension of the
%        \LaTeX{} theorem environment\thanks{This file
%        has version number \fileversion, last
%        revised \filedate.}}
% \author{Frank Mittelbach}
%
% \MaintainedByLaTeXTeam{tools}
% \maketitle
%
% \begin{abstract}
%    The macros described in this paper yield an extension of the
%    \LaTeX{} theorem mechanism. It is designed to satisfy the
%    different requirements of various journals. Thus, the
%    layout of the ``theorems'' can be manipulated by determining a
%    ``style''. This article describes not only the use, but
%    also the definition, of the necessary macros.
% \end{abstract}
%
% \section*{Preface to version 2.2}
%
% For \LaTeXe{} this package did not need any fundamental changes. I
% only modified the messages generated so that theorem layout styles
% will show up with the |\listfiles| command and cleaned the section
% on the New Font Selection Scheme since this is now included in
% \LaTeX.
%
% \section*{Preface to version 2.1}
%
% This version is identical to 2.0g described in \TUB\ 10\#3 except
% for some internal defaults which are now set depending on the used
% font selection scheme.
%
% This was done to avoid unpleasant surprises if the new font selection
% scheme is in force. For further details see
% section~\ref{sec:newfonts} and \cite{bk:GMS94}.
%
% \section{Introduction}
%
% For our purposes here, ``theorems'' are labelled enunciations,
% often set off from the main text by extra space and a font change.
% Theorems, corollaries, conjectures, definitions, and remarks
% are all instances of ``theorems''.  The ``header'' of these structures
% is composed of a label (such as \textsc{Theorem} or \textsc{Remark})
% and a number which serializes an item in the sequence of items
% with the same label.
%
% Shortly after the introduction of \LaTeX{} at the Fachbereich
% Mathematik in Mainz, the desire to manipulate the layout of
% ``theorems'' arose.  In Mainz, the following two conventions came
% into general use:
% \begin{enumerate}
%   \item The number of the theorem is shown in the margin.
%   \item There is a line break at
%         the end of the theorem header.
% \end{enumerate}
%
% Additionally, some journals require different formats which depend on
% the ``sort of theorem'': e.g.\ often remarks and definitions are set
% in |\upshape|, while |\itshape| is employed for main theorems.
%
% Confronted with these requirements, a theorem environment was
% developed in Mainz which allows separate determination of the
% layout of the ``theorems sets'', comparable to |\pagestyle|.
%
%
% \section{The user interface}
% \subsection{Defining new theorem sets}
%
% \DescribeMacro\newtheorem
% As in the original \LaTeX{} version, the command |\newtheorem|
% defines a new ``theorem set'' or ``theorem-like structure''.
% Two required arguments name the new environment and give the text
% to be typeset with each instance of the new ``set'', while
% an optional argument determines how
% the ``set'' is enumerated:
% \begin{description}
%    \item[\ttfamily \bslash newtheorem\{foo\}\{bar\}]
%       The theorem set {\envfont foo} (whose name is \texttt{bar})
%       uses its own counter.
%    \item[\ttfamily \bslash newtheorem\{foo2\}{[foo]}\{bar2\}]
%       The theorem set {\envfont foo2} (printed name \texttt{bar2})
%       uses the same counter as the theorem set \texttt{foo}.
%    \item[\ttfamily \bslash newtheorem\{foo3\}\{bar3\}{[section]}]
%       The theorem set {\envfont foo3} (printed name \texttt{bar3}) is
%       enumerated within the counter \texttt{section}, i.e.\ with every
%       new |\section| the enumeration begins again with 1, and
%       the enumeration is composed from the section-number and the
%       theorem counter itself.
% \end{description}
%
% \DescribeMacro\theoremstyle
% Additionally, the command |\theoremstyle| can define the layout
% of various, or all, theorem sets.  It should be noted that any
% theorem set defined by |\newtheorem| is typeset in the
% |\theoremstyle| that is current at the time of the definition.
% Thus, the following
% \begin{verbatim}
%    \theoremstyle{break}        \newtheorem{Cor}{Corollary}
%    \theoremstyle{plain}        \newtheorem{Exa}{Example}[section]
%\end{verbatim}                             ^^A No blank at line start
% leads to the result that the set {\envfont Cor} is formatted in the
% style \texttt{break}, while the set {\envfont Exa} and all the
% following ones are formatted in the style \texttt{plain}, unless
% another |\theoremstyle| follows.  Since the definitions installed by
% |\newtheorem| are global, one also can limit |\theoremstyle| locally
% by grouping braces.
%
% \DescribeMacro\theorembodyfont
% The choice of the font for the theorem body is completely
% independent of the chosen |\theoremstyle|; this has proven to
% be very advantageous.  For example,
% \begin{verbatim}
%   {\theorembodyfont{\upshape}          \newtheorem{Rem}{Remark}}
%\end{verbatim}
% defines a theorem set {\envfont Rem}, which will be set in
% |\upshape| in the current layout (which in our example is
% \texttt{plain}).  As with |\theoremstyle|, the |\theorembodyfont|
% chosen is that current at the time of |\newtheorem|.  If
% |\theorembodyfont| is not specified or one defines
% |\theorembodyfont{}|, then the font used will be that defined by the
% |\theoremstyle|.
%
% \DescribeMacro\theoremheaderfont It is also possible to customize
% the font used for the theorem headers.  This is, however, a global
% declaration, and therefore there should be at most one
% |\theoremheaderfont| declaration in the preamble.\footnote{If it is
% actually necessary to have different header fonts, one has to define
% new theorem styles (substituting the desired font) or specify the
% information directly in the \texttt{\bslash newtheorem} declaration
% (the unclean variant).}
%
% \DescribeMacro\theorempreskipamount
% \DescribeMacro\theorempostskipamount Two additional parameters
% affect the vertical space around the theorem environments:
% |\theorempreskipamount| and |\theorempostskipamount| define,
% respectively, the spacing before and after such an environment.
% These parameters apply for all theorem sets and can be manipulated
% with the ordinary length macros.  They are rubber lengths
% (`\textsf{skips}'), and therefore can contain \texttt{plus} and
% \texttt{minus} parts.
%
% Since the definition of theorem sets should---most sensibly---be
% placed in the preamble, we only allow installation there.
% It is therefore possible to release the memory used here after
% |\begin{document}|, in order to make room for other
% applications.
%
%
% \subsection{Existing theorem styles}
%
% The following theorem styles exist to date:
% \begin{deflist}{marginbreak}
%    \item[plain]
%       This theorem style emulates the original \LaTeX{} definition,
%       except that additionally the parameters
%       |\theorem...skipamount| are used.
%    \item[break]
%       In this style, the theorem header is followed by a line
%       break.
%    \item[marginbreak]
%       The theorem number is set in the margin, and there is a line
%       break as in \texttt{break}.
%    \item[changebreak]
%       Like \texttt{break}, but with header number and text
%       interchanged.
%    \item[change]
%       Header number and text are interchanged, without a line break.
%    \item[margin]
%       The number is set in the left margin, without a line break.
% \end{deflist}
% All styles (except \texttt{plain}) select |\slshape| as the default
% |\theorembodyfont|.
%
%
% \subsection{Examples}
%
% Given the above theorem sets {\envfont Cor}, {\envfont Exa}
% and {\envfont Rem},
% suppose that the preamble also contains the declarations:
% \begin{verbatim}
%    \theoremstyle{marginbreak}   \newtheorem{Lem}[Cor]{Lemma}
%    \theoremstyle{change}
%    \theorembodyfont{\itshape}        \newtheorem{Def}[Cor]{Definition}
%
%    \theoremheaderfont{\scshape}
%\end{verbatim}
% Then the following are some typical examples of the typeset output
% resulting from their use.
%
% \begin{Cor}
%    This is a sentence typeset in the theorem environment {\envfont
%    Cor}.
% \end{Cor}
%
% \begin{Exa}
%    This is a sentence typeset in the theorem environment {\envfont
%    Exa}.
% \end{Exa}
%
% \begin{Rem}
%    This is a sentence typeset in the theorem environment {\envfont
%    Rem}.
% \end{Rem}
%
% \begin{Lem}[Ben User]
%    This is a sentence typeset in the theorem environment {\envfont
%    Lem}.
% \end{Lem}
%
% \begin{Def}[Very impressive Definition]
%    This is a sentence typeset in the theorem environment {\envfont
%    Def}.
% \end{Def}
%
% The last two examples show the effect of the optional argument to
% a theorem environment (it is the text typeset in parentheses).
%
%
% \section{Special Considerations}
% \label{sec:newfonts}
%
% Theoremheader and body are implemented as a unit. This means that
% the |\theoremheaderfont|  will inherit characteristics of the
% |\theorembodyfont| in \LaTeXe. Thus, if for example
% |\theorembodyfont| is |\itshape| and |\theoremheaderfont|
% is |\bfseries| the font selected for the header will have the
% characteristics `bold extended italic'. If this is not desired one
% should set the |\theoremheaderfont| to something like
% \begin{verbatim}
%      \theoremheaderfont{\normalfont\bfseries}
%\end{verbatim}
% i.e.\ supplying all necessary font information explicitly.
%
%
%
% \section{Acknowledgements}
%
% The publication of this set of macros was only possible with the
% help of Christina Busse (translating the manuscript into English),
% Joachim Pense (playing the r\^ole of typist), Chris Rowley
% (looking everything over) and many others providing useful
% suggestions.
%
% \MaybeStop{%
%    \begin{thebibliography}{9}
%       \bibitem{bk:GMS94} \textsc{M.~Goossens}, \textsc{F.~Mittelbach}
%         and \textsc{A.~Samarin}.
%         \newblock The \LaTeX{} Companion.
%         \newblock
%         Addison-Wesley, Reading, Massachusetts, 1994.
%       \bibitem{latex:source} \textsc{Lamport, Leslie.} \newblock
%                               \texttt{latex.tex}, version 2.09, date
%                               Feb. 1990.
%    \end{thebibliography}
%    \PrintIndex
%    \PrintChanges}
%
%
% \section{The documentation driver file}
%
% The next bit of code contains the documentation driver file for
% \TeX{}, i.e., the file that will produce the documentation you are
% currently reading. It will be extracted from this file by the
% \texttt{docstrip} program. Since it is the first code in the file
% one can alternatively process this file directly with \LaTeXe{} to
% obtain the documentation.
%
%    \begin{macrocode}
%<*driver>
\documentclass{ltxdoc}

\usepackage{theorem}

%<+driver>% The next few lines define theorem sets which are used
%<+driver>% in the example section of the documentation.

\theoremstyle{break}       \newtheorem{Cor}{Corollary}
\theoremstyle{plain}       \newtheorem{Exa}{Example}[section]
{\theorembodyfont{\upshape}\newtheorem{Rem}{Remark}}
\theoremstyle{marginbreak} \newtheorem{Lem}[Cor]{Lemma}
\theoremstyle{change}
\theorembodyfont{\itshape} \newtheorem{Def}[Cor]{Definition}

\theoremheaderfont{\scshape}

\RecordChanges

\begin{document}
   \DocInput{theorem.dtx}
\end{document}
%</driver>
%    \end{macrocode}
%
% \section{Definition of the Macros}
%
% If the file has been loaded before, we abort immediately. If not the
% package announces itself (this is actually done at the very top of
% the file---the way it is done isn't good style so don't copy it).
% \changes{v2.0e}{89/07/19}{Spaces removed from `typeout'.}
% \changes{v2.2b}{95/11/19}{Announce moved to top of file}
%    \begin{macrocode}
%<*package>
%\@ifundefined{theorem@style}{}{\endinput}
%\def\FMithmInfo{1995/11/19 v2.2b Theorem extension package (FMi)}
%\ProvidesPackage{theorem}[\FMithmInfo]
%    \end{macrocode}
%
%
% \subsection{Definition of theorem styles and fonts}
%
% All the definitions in this file are done globally to allow
% inputting this file inside a group.
%
% \begin{macro}{\theoremstyle}
%    Before a theorem style can be installed, the chosen style must be
%    known. For that reason, we must test to see that
%    |\th@|\meta{style} is known or, more precisely, that it is
%    different from |\relax|. If the style is not known then
%    |\th@plain| is used.
%    \begin{macrocode}
\gdef\theoremstyle#1{%
   \@ifundefined{th@#1}{\@warning
          {Unknown theoremstyle `#1'. Using `plain'}%
          \theorem@style{plain}}%
%    \end{macrocode}
%    We save the theorem style to be used in the token register
%    |\theorem@style|.
%    \begin{macrocode}
      {\theorem@style{#1}}%
%    \end{macrocode}
%    Now we ``evaluate'' the theorem style: this means, we call the
%    macro |\th@|\meta{style} which will activate the relevant
%    definitions which are contained in a separate file.  This is done
%    in a group to suppress changes to the current font.  This could
%    otherwise pose problems together with the new font selection
%    scheme\footnote{When I printed the original article using the new
%    font selection scheme I ended with a document with slanted
%    typefaces (text headings and all) simply because one of the
%    theorem styles used \texttt{\bslash sl} at toplevel.} if the
%    |\th@|\meta{style} is evaluated a second time.
% \changes{v2.1a}{90/02/15}{Group added around `th@..}
% \changes{v2.1a}{90/02/15}{Reseting of `..@begintheorem removed.}
%    \begin{macrocode}
      \begingroup
        \csname th@\the\theorem@style \endcsname
      \endgroup}
%    \end{macrocode}
% \end{macro}
%
%
% \begin{macro}{\@begintheorem}
% \changes{v2.1a}{90/02/15}{Relaxed at toplevel.}
% \begin{macro}{\@opargbegintheorem}
% \changes{v2.1a}{90/02/15}{Relaxed at toplevel.}
%    We reset |\@begintheorem| and |\@opargbegintheorem| to
%    |\relax| since these commands are no longer necessary at
%    toplevel. This will save a few tokens.
%    \begin{macrocode}
\global\let\@begintheorem\relax
\global\let\@opargbegintheorem\relax
%    \end{macrocode}
% \end{macro}
% \end{macro}
%
% \begin{macro}{\theorem@style}
%    Obviously the \textsf{token} register used above has to be
%    allocated. To assure the utmost compatibility with the
%    original \LaTeX{} definition, we set the default theorem style to
%    \texttt{plain}, which implements the usual \LaTeX{} convention.
% \changes{v2.0f}{89/07/20}{Unnecessary `global' removed.}
%    \begin{macrocode}
\newtoks\theorem@style
\global\theorem@style{plain}
%    \end{macrocode}
% \end{macro}
%
% \begin{macro}{\theorembodyfont}
% \begin{macro}{\theorem@bodyfont}
%    If the |\theorembodyfont| is set by the user then it should not
%    interact with the default font set in the theorem style. When the
%    new font selection is in force this may happen if, for example,
%    the default is |\itshape| and the new |\theorembodyfont| is
%    |\sffamily|. So we add a |\reset@font| command in front of the
%    user definition.
% \changes{v2.0f}{89/07/20}{Unnecessary `global' removed.}
% \changes{v2.0g}{89/09/19}{`theoremfont to `theorembodyfont.}
% \changes{v2.1c}{92/06/22}{`reset@font added and logic changed.}
% \changes{v2.1d}{92/10/28}{Reset to empty if arg is empty}
%    \begin{macrocode}
\gdef\theorembodyfont#1{%
%    \end{macrocode}
%    We check if the argument supplied is empty and if so put nothing
%    into the |\theorem@bodyfont| token register to allow for
%    |\theorembodyfont{}| as a mean of using the default of the
%    current |\theoremstyle|.
%    \begin{macrocode}
   \def\@tempa{#1}%
   \ifx\@tempa\@empty
    \theorem@bodyfont{}%
   \else
    \theorem@bodyfont{\reset@font#1}%
   \fi
   }
\newtoks\theorem@bodyfont
\global\theorem@bodyfont{}
%    \end{macrocode}
% \end{macro}
% \end{macro}
%
%
% \begin{macro}{\theoremheaderfont}
% \changes{v2.0g}{89/09/19}{Macro added.}
%    The font for the theorem headers is handled differently because
%    this definition applies to all theorem styles.
%    \begin{macrocode}
\gdef\theoremheaderfont#1{\gdef\theorem@headerfont{#1}%
%    \end{macrocode}
%    After using the macro once it is redefined to produce an error
%    message.
%    \begin{macrocode}
       \gdef\theoremheaderfont##1{%
        \typeout{\string\theoremheaderfont\space should be used
                 only once.}}}
%    \end{macrocode}
% \end{macro}
%
% \begin{macro}{\theorem@headerfont}
% \changes{v2.0g}{89/09/19}{Macro added.}
%    To set the |\theorem@headerfont| default we first test if
%    the new fontselection scheme is in force.
% \changes{v2.1a}{90/02/15}{Support for new font selection.}
%    \begin{macrocode}
\ifx\upshape\undefined
%    \end{macrocode}
%    If not we define it to expand into |\bfseries|. We don't use
%    |\let| just in case a following style option redefines this
%    macro.
%    \begin{macrocode}
\gdef\theorem@headerfont{\bfseries}
%    \end{macrocode}
%    Otherwise we reset the current shape before calling |\bfseries|.
%    \begin{macrocode}
\else \gdef\theorem@headerfont{\normalfont\bfseries}\fi
%    \end{macrocode}
% \end{macro}
%
% \begin{macro}{\th@plain}
% \begin{macro}{\th@break}
% \begin{macro}{\th@marginbreak}
% \begin{macro}{\th@changebreak}
% \begin{macro}{\th@change}
% \begin{macro}{\th@margin}
%    The different styles are defined in macros such as
%    |\th@plain|.  Since memory space is precious in
%    ``non-Big-versions'', we have to avoid offering too many unused
%    definitions.  Therefore we define these styles in separate files
%    that can be loaded on demand.  Thus the commands themselves only
%    load these files. We use |\@input@|, a \LaTeXe{} internal command
%    that ensures that the file will be listed with |\listfiles|.
% \changes{v2.2a}{94/02/02}{Use `@input@ to load theorem layout files}
%    \begin{macrocode}
\gdef\th@plain{\@input@{thp.sty}}
\gdef\th@break{\@input@{thb.sty}}
\gdef\th@marginbreak{\@input@{thmb.sty}}
\gdef\th@changebreak{\@input@{thcb.sty}}
\gdef\th@change{\@input@{thc.sty}}
\gdef\th@margin{\@input@{thm.sty}}
%    \end{macrocode}
%    This list will be expanded when new styles become
%    available.  For testing,  just append new theorem substyles
%    as document options.
% \end{macro}
% \end{macro}
% \end{macro}
% \end{macro}
% \end{macro}
% \end{macro}
%
%
%
% \subsection{Definition of a new theorem set}
%
% As already pointed out, a new theorem environment can be defined in
% three different ways:
% \begin{verbatim}
%    \newtheorem{Lem}{Lemma}
%    \newtheorem{Lem}{Lemma}[section]
%    \newtheorem{Lem}[Theorem]{Lemma}
%\end{verbatim}
% The function of the macro |\newtheorem| is to recognize these cases
% and then to branch into one of the three macros |\@ynthm|, |\@xnthm|
% or |\@othm|.  This mechanism is adopted unchanged from
% \cite{latex:source}; the essential point here is that, for example,
% in the second case, the arguments \texttt{Lem}, \texttt{Lemma} and
% \texttt{section} are passed over to the macro |\@xnthm|.
%
% We inspect this case first because the others present fewer problems,
% and thus are easily derived from this one.
%
% \begin{macro}{\@xnthm}
%    For our example arguments, the macro |\@xnthm| must fulfill
%    the following:
%    \begin{itemize}
%       \item
%          Define a new \LaTeX-counter `Lem'
%       \item
%          reset this counter within a |\section|
%       \item
%          define the macro |\theLem|
%       \item
%          define the environment macros |\Lem| and
%          |\endLem| using the current
%          |\theoremstyle| and |\theorem@bodyfont|.
%    \end{itemize}
%    Obviously, all this should happen only if the first argument
%    of |\@xnthm| (i.e.\ \texttt{Lem} in our example) is chosen so
%    as not to conflict with any previously defined commands or
%    environments.  This test is performed by the \LaTeX{} macro
%    |\@ifdefinable|.
%    \begin{macrocode}
\gdef\@xnthm#1#2[#3]{\expandafter\@ifdefinable\csname #1\endcsname
%    \end{macrocode}
%    Therefore, the first argument of |\@ifdefinable| is the
%    expansion (in the example, |\Lem|) of
%    |\csname||#1\endcsname|.  The second argument is
%    executed only if the test has been completed successfully.
%    \begin{macrocode}
   {%
%    \end{macrocode}
%    Now we define the new counter. The names of the \LaTeX{}
%    macros employed should speak for themselves:
% \changes{v2.2b}{1995/10/07}{Use \cs{@newctr}}
% \changes{v2.2c}{1995/11/23}{Use \cs{@newctr} properly.}
%    \begin{macrocode}
    \@definecounter{#1}\@newctr{#1}[#3]%
%    \end{macrocode}
%    Using |\@newctr| will give a proper error message if the counter
%    in |#3| is not defined.
%    In defining `|\theLem|' we must generate the desired macro
%    name by use of |\expandafter| and |\csname|.
%    \begin{macrocode}
    \expandafter\xdef\csname the#1\endcsname
%    \end{macrocode}
%    An |\xdef| is used in order to make the definition global,
%    and to ensure that it contains the replacement texts of
%    |\@thmcountersep| and |\@thmcounter|.\footnote{These
%    two macros can be defined by the document style.  Their default
%    values produce a `\texttt{.}' as separation and an arabic
%    representation of the number.} However, not everything should be
%    expanded. For example, it saves space to use |\thesection|
%    instead of its---at times---lengthy expansion.
%    \begin{macrocode}
      {\expandafter \noexpand \csname the#3\endcsname
       \@thmcountersep \@thmcounter{#1}}%
%    \end{macrocode}
%    Thus with the defaults of \LaTeX{}, |\theLem| would be
%    replaced by the command sequence |\thesection.\arabic{Lem}|.
%
%    We will now look at the definition of the macro which is executed
%    at the beginning of the actual environment (in our example this
%    macro is |\Lem|). It should be noted that we use an
%    ``|\expandafter| trick'' to expand only certain parts of the
%    replacement text at the time of the definition.
%    \begin{macrocode}
    \def\@tempa{\global\@namedef{#1}}%
    \expandafter \@tempa \expandafter{%
%    \end{macrocode}
%    First, the macro that contains the current definitions of
%    |\@begintheorem| and |\@opargtheorem| should be called
%    up.  The name of this macro---as is already known---has the form
%    |\th@|\meta{theorem\ style}; therefore, it must be called by
%    \begin{macrocode}
      \csname th@\the \theorem@style
%    \end{macrocode}
%    In addition the default theorem font should be changeable, i.e.\
%    we have to insert the contents of |\theorem@bodyfont|. For
%    that reason, we expand even further, beyond |\endcsname|,
%    and thus insert the contents of the \textsf{token} register
%    |\theorem@bodyfont| in the replacement text.
%    \begin{macrocode}
            \expandafter \endcsname \the \theorem@bodyfont
%    \end{macrocode}
%    Now it is time to call the macro |\@thm| which takes over
%    the further processing.  It has two arguments: the current
%    counter name (in our example, \texttt{Lem}), and the text of the
%    label (in our example, \texttt{Lemma}).
%    \begin{macrocode}
     \@thm{#1}{#2}}%
%    \end{macrocode}
%    With this, the `sub-definition' is complete.  The macro
%    |\@endtheorem| ends a theorem environment and is, so far, nothing
%    but an |\endtrivlist|.  (Hence it is defined globally, and not
%    within the theorem styles.\footnote{This has to be changed as
%    soon as theorem styles that change \texttt{\bslash @endtheorem}
%    exist.  In such a case, all existing styles must be changed as
%    well since they will have to reset the macro.}) Therefore, we can
%    set it equivalent to the macro that ends the theorem set (in our
%    example, |\endLem|).  However, if some day theorem styles exist
%    that do change |\@endtheorem|, we would have to use the
%    commented-out line instead.
%    \begin{macrocode}
    \global \expandafter \let \csname end#1\endcsname \@endtheorem
%   \global\@namedef{end#1}{\@endtheorem}%
%    \end{macrocode}
%    With these commands all the required definitions are employed,
%    unless the test |\@ifdefinable| has failed.  Therefore, we
%    end the second argument of this macro and with it the definition
%    of |\@xnthm|.
%    \begin{macrocode}
   }}
%    \end{macrocode}
% \end{macro}
%
% \begin{macro}{\@ynthm}
%    The definition of |\@ynthm| is completely analogous. In
%    this case the new counter that is defined is not reset within
%    another counter; thus the definition of |\the...|
%    is simplified:
%    \begin{macrocode}
\gdef\@ynthm#1#2{\expandafter\@ifdefinable\csname #1\endcsname
   {\@definecounter{#1}%
    \expandafter\xdef\csname the#1\endcsname{\@thmcounter{#1}}%
%    \end{macrocode}
%    The rest of the definition corresponds literally to that of
%    |\@xnthm|:
%    \begin{macrocode}
    \def\@tempa{\global\@namedef{#1}}\expandafter \@tempa
     \expandafter{\csname th@\the \theorem@style \expandafter
     \endcsname \the\theorem@bodyfont \@thm{#1}{#2}}%
    \global \expandafter \let \csname end#1\endcsname \@endtheorem}}
%    \end{macrocode}
% \end{macro}
%
% \begin{macro}{\@othm}
%    The definition of |\@othm| does not contain anything new.
%    \begin{macrocode}
\gdef\@othm#1[#2]#3{%
%    \end{macrocode}
% \changes{v2.2b}{95/10/07}{Build in check for counter defined}
%    We do not define a new counter but instead use one that has
%    already been defined.  Thus the only definition we need is that
%    of this pseudo-counter (i.e.\ |\the|\meta{env.\ name}).
%    First we check if |#2| corresponds to a known counter name.
%    \begin{macrocode}
  \expandafter\ifx\csname c@#2\endcsname\relax
   \@nocounterr{#2}%
  \else
   \expandafter\@ifdefinable\csname #1\endcsname
   {\expandafter \xdef \csname the#1\endcsname
     {\expandafter \noexpand \csname the#2\endcsname}%
%    \end{macrocode}
%    All other parts of the definition can be adopted from
%    |\@xnthm|. We have to remember, though, that in this case
%    the name of the current counter and the theorem label have moved
%    to the second and third arguments.
%    \begin{macrocode}
    \def\@tempa{\global\@namedef{#1}}\expandafter \@tempa
     \expandafter{\csname th@\the \theorem@style \expandafter
     \endcsname \the\theorem@bodyfont \@thm{#2}{#3}}%
    \global \expandafter \let \csname end#1\endcsname \@endtheorem}%
  \fi}
%    \end{macrocode}
% \end{macro}
%
% \subsection{Macros that are employed in a {\envfont theorem}
%    environment}
%
% \begin{macro}{\@thm}
%    The macro |\@thm| has to increase the current counter.  Then,
%    depending on whether the environment has (or does not have) an
%    optional argument, it has to branch into either |\@begintheorem|
%    or |\@opargtheorem|.
%    \begin{macrocode}
\gdef\@thm#1#2{\refstepcounter{#1}%
%    \end{macrocode}
%    Now we start a {\envfont trivlist} environment, and give
%    |\@topsep| and |\@topsepadd| the values of the \textsf{skip}
%    registers |\theorempreskipamount| and |\theorempostskipamount|.
%    The value in |\@topsep| is the vertical space that is inserted by
%    the first (and only) |\item| in our |\trivlist| whilst
%    |\@topsepadd| is inserted by |\@endparenv| at the end of that
%    {\envfont trivlist} environment.  By using these registers, we
%    obtain the desired space around a {\envfont theorem} environment.
%    \begin{macrocode}
   \trivlist
   \@topsep \theorempreskipamount               % used by first \item
   \@topsepadd \theorempostskipamount           % used by \@endparenv
%    \end{macrocode}
%    Now we have to test whether an optional argument has been given.
% \changes{v2.0d}{89/07/18}{Forgotten percent added.}
%    \begin{macrocode}
   \@ifnextchar [%
%    \end{macrocode}
%    If there is an optional argument, we will call |\@ythm|, and
%    move the arguments read back into the input stream.
%    \begin{macrocode}
   {\@ythm{#1}{#2}}%
%    \end{macrocode}
%    If not, we call |\@begintheorem|. Its first argument is the
%    name of the theorem set (hence the second argument of
%    |\@thm|). Its second argument is the macro that produces the
%    current number.
%    \begin{macrocode}
   {\@begintheorem{#2}{\csname the#1\endcsname}\ignorespaces}}
%    \end{macrocode}
% \end{macro}
%
% \begin{macro}{\@xthm}
% \begin{macro}{\@ythm}
%    Both these macros were originally called by |\@thm|.  We do
%    not need |\@xthm| anymore, hence we reset it to
%    |\relax|. The definition of |\@ythm| has not changed at
%    all from its definition in \LaTeX. In order to make the macros
%    easier to understand, we will
%    nevertheless present it (commented out).
%    \begin{macrocode}
\global\let\@xthm\relax
% \def\@ythm#1#2[#3]{\@opargbegintheorem{#2}{\csname
%           the#1\endcsname}{#3}\ignorespaces}
%</package>
%    \end{macrocode}
%    The primitive |\ignorespaces| in |\@ythm| and
%    |\@thm| is needed to remove the spaces between the
%    |\begin{...}| and the actual text.
% \end{macro}
% \end{macro}
%
%
% \subsection{Definition of the theorem substyles}
%
% As already pointed out, the theorem substyles, defined below, are
% only loaded when necessary.  Note that all these substyles, except
% \texttt{plain}, have |\slshape| as the default body font.
%
% \subsubsection{The \texttt{plain} style}
%
% \label{dok:th@plain} As the following macros use |@|, we have
% to locally set the |\catcode| of this symbol to ``letter''.
% This happens within a group, so that we do not have to worry about
% which |\catcode| that symbol had before.
%    \begin{macrocode}
%<*thp>
\begingroup \makeatletter
%    \end{macrocode}
% Since we are now within a group, we must make all definitions
% globally.  First we make sure that \texttt{theorem.sty} is loaded.
% This will allow us to use this file as a document style option
% without having to call \texttt{theorem} itself as an option.  At the
% same time, we assure that at least version 2 is loaded, since
% |\theorem@style| was not defined in earlier versions.
%    \begin{macrocode}
\@ifundefined{theorem@style}{\input{theorem.sty}}{}
\ProvidesFile{thp.sty}
             [\FMithmInfo]
%    \end{macrocode}
%
%
% \begin{macro}{\th@plain}
% \begin{macro}{\@begintheorem}
% \begin{macro}{\@opargbegintheorem}
%    |\theoremstyle{plain}| corresponds to the original
%    definition, except that the distances to the surrounding text are
%    determined by the parameters |\theorempreskipamount| and
%    |\theorempostskipamount|.
%    First we set the default body font.
%    \begin{macrocode}
\gdef\th@plain{\normalfont\itshape
%    \end{macrocode}
%    Then we define |\@begintheorem| and
%    |\@opargbegintheorem|. These two macros define how the
%    header of a theorem is typeset.  |\@opargbegintheorem| will
%    be called if a \textsf{theorem} environment with an optional
%    argument is encountered; otherwise, the header is constructed by
%    calling |\@begintheorem|.  If one of these macros is
%    executed, we are within a \textsf{trivlist} environment started by
%    |\@thm|.  So the theorem header is produced with an
%    |\item| command.
%
%    Instead of specifying the header font directly, all standard
%    theorem styles use the |\theorem@headerfont| macro to allow
%    customization.  The extra space (|\labelsep|) is necessary
%    because of problems in the \textsf{trivlist} environment.
%    \begin{macrocode}
  \def\@begintheorem##1##2{%
        \item[\hskip\labelsep \theorem@headerfont ##1\ ##2]}%
%    \end{macrocode}
%    The definition of |\@opargbegintheorem| is completely
%    analogous. The only difference is the fact that there exists a
%    third argument (which is the optional parameter of the
%    environment and contains additional information about the
%    theorem).  Customarily we enclose it in parentheses.
%    \begin{macrocode}
\def\@opargbegintheorem##1##2##3{%
   \item[\hskip\labelsep \theorem@headerfont ##1\ ##2\ (##3)]}}
%    \end{macrocode}
% \end{macro}
% \end{macro}
% \end{macro}
%
% \noindent
% We conclude with an |\endgroup| to restore the
% |\catcode| of |@|.
%    \begin{macrocode}
\endgroup
%</thp>
%    \end{macrocode}
%
%
% \subsubsection{The \texttt{break} style}
%
% This style option is stored in the file |thb.sty|.
% For the next two lines see the documentation for |\th@plain| on
% page~\pageref{dok:th@plain}.
%    \begin{macrocode}
%<*thb>
\begingroup \makeatletter
\@ifundefined{theorem@style}{\input{theorem.sty}}{}
\ProvidesFile{thb.sty}
             [\FMithmInfo]
%    \end{macrocode}
%
% \begin{macro}{\th@break}
%    |\theoremstyle{break}| produces a line break after the name
%    of the theorem.  The font is |\slshape|.  Hence, we define
%    |\th@break| as follows:
%    \begin{macrocode}
\gdef\th@break{\normalfont\slshape
  \def\@begintheorem##1##2{\item[%
%    \end{macrocode}
%    We run into the following problem: it is not possible to create
%    the header with |\item[|\meta{title}|]| and then start
%    a new line by, for example, |\mbox{}\\|. Such a definition
%    will fail whenever a list environment follows immediately.  With
%    the above construction, the |\mbox{}| causes the switch
%    |@inlabel| (cf.\ definition of |\list| and
%    |\trivlist| in \cite{latex:source}) to be set to \texttt{false}
%    and so the following list will insert additional vertical
%    space (|\topskip|). This is quite annoying. Therefore, we
%    create the line break within the |\item|. In order to ensure
%    that the text will begin at the proper position in the following
%    line, we simply pretend that the label does not take any
%    room.\footnote{This will lead to problems whenever very high
%    symbols occurring in the line tower into the
%    heading. So, something else has to be done here sometime.}
%    \begin{macrocode}
       \rlap{\vbox{\hbox{\hskip \labelsep\theorem@headerfont ##1\ ##2}%
                   \hbox{\strut}}}]}%
%    \end{macrocode}
%    Again, the definition of |\@opargbegintheorem| is
%    completely analogous.
%    \begin{macrocode}
\def\@opargbegintheorem##1##2##3{%
  \item[\rlap{\vbox{\hbox{\hskip \labelsep \theorem@headerfont
                     ##1\ ##2\ (##3)}%
                    \hbox{\strut}}}]}}
\endgroup
%</thb>
%    \end{macrocode}
% \end{macro}
%
%
% \subsubsection{The \texttt{changebreak} style}
%
%    \begin{macrocode}
% This style option is stored in the file |thcb.sty|.
%    \begin{macrocode}
%<*thcb>
\begingroup \makeatletter
\@ifundefined{theorem@style}{\input{theorem.sty}}{}
\ProvidesFile{thcb.sty}
             [\FMithmInfo]
%    \end{macrocode}
%
% \begin{macro}{\th@changebreak}
%    The \texttt{change-break} theorem style is like \texttt{break}
%    but with interchange of theorem name and theorem number.  Thus we
%    define |\th@changebreak| as follows:
%    \begin{macrocode}
\gdef\th@changebreak{\normalfont\slshape
  \def\@begintheorem##1##2{\item
       [\rlap{\vbox{\hbox{\hskip\labelsep \theorem@headerfont ##2\ ##1}%
                    \hbox{\strut}}}]}%
\def\@opargbegintheorem##1##2##3{%
  \item[\rlap{\vbox{\hbox{\hskip\labelsep \theorem@headerfont
                     ##2\ ##1\ (##3)}%
                    \hbox{\strut}}}]}}
\endgroup
%</thcb>
%    \end{macrocode}
% \end{macro}
%
%
% \subsubsection{The \texttt{change} style}
%
% This style option is stored in the file |thc.sty|.
%    \begin{macrocode}
%<*thc>
\begingroup \makeatletter
\@ifundefined{theorem@style}{\input{theorem.sty}}{}
\ProvidesFile{thc.sty}
             [\FMithmInfo]
%    \end{macrocode}
%
% \begin{macro}{\th@change}
%    The \texttt{change} theorem style corresponds to the
%    \texttt{change break} style without a linebreak after the header.
%    To say it in another way, it's the same as the \texttt{plain}
%    style but with number and name interchanged and |\slshape| as the
%    default font.
%    \begin{macrocode}
\gdef\th@change{\normalfont\slshape
  \def\@begintheorem##1##2{\item
       [\hskip\labelsep \theorem@headerfont ##2\ ##1]}%
\def\@opargbegintheorem##1##2##3{%
  \item[\hskip\labelsep \theorem@headerfont ##2\ ##1\ (##3)]}}
\endgroup
%</thc>
%    \end{macrocode}
% \end{macro}
%
%
% \subsubsection{The \texttt{marginbreak} style}
%
% This style option is the one used most often at Mainz.  It is saved
% in the file |thmb.sty|.
%    \begin{macrocode}
%<*thmb>
\begingroup \makeatletter
\@ifundefined{theorem@style}{\input{theorem.sty}}{}
\ProvidesFile{thmb.sty}
             [\FMithmInfo]
%    \end{macrocode}
%
% \begin{macro}{\th@marginbreak}
%    The \texttt{margin break} style is nearly the same as the
%    \texttt{change break} style. The only difference is the placement
%    of the theorem number.  We use |\llap| to place it in the left
%    margin.
%
%    In this style |\labelsep| denotes the separation between the
%    number and the text.
%    \begin{macrocode}
\gdef\th@marginbreak{\normalfont\slshape
  \def\@begintheorem##1##2{\item
       [\rlap{\vbox{\theorem@headerfont
                    \hbox{\llap{##2}\hskip\labelsep ##1}%
                    \hbox{\strut}}}]}%
\def\@opargbegintheorem##1##2##3{%
  \item[\rlap{\vbox{\theorem@headerfont
                    \hbox{\llap{##2}\hskip\labelsep ##1\ (##3)}%
                    \hbox{\strut}}}]}}
\endgroup
%</thmb>
%    \end{macrocode}
% \changes{v2.1c}{90/02/17}{`endgroup outside macrocode env.}
% \end{macro}
%
%
%
% \subsubsection{The \texttt{margin} style}
%
% This style option is stored in the file |thm.sty|.
%    \begin{macrocode}
%<*thm>
\begingroup \makeatletter
\@ifundefined{theorem@style}{\input{theorem.sty}}{}
\ProvidesFile{thm.sty}
             [\FMithmInfo]
%    \end{macrocode}
%
% \begin{macro}{\th@margin}
%    Again this is only a variant of the theorem styles described
%    above without any new ideas.
%    \begin{macrocode}
\gdef\th@margin{\normalfont\slshape
  \def\@begintheorem##1##2{\item
       [\theorem@headerfont \llap{##2}\hskip\labelsep ##1]}%
\def\@opargbegintheorem##1##2##3{%
  \item[\theorem@headerfont \llap{##2}\hskip\labelsep ##1\ (##3)]}}
\endgroup
%</thm>
%    \end{macrocode}
% \end{macro}
%
%
%
%
% \subsection{Final Definitions}
%
% \begin{macro}{\theorempreskipamount}
% \begin{macro}{\theorempostskipamount}
%    The \textsf{skip} parameters that regulate the vertical empty space
%    before and after the {\envfont theorem} environment have
%    to be allocated as well.
% \changes{v2.0f}{89/07/20}{Unnecessary `global' removed.}
%    \begin{macrocode}
%<*package>
\newskip\theorempreskipamount
\newskip\theorempostskipamount
%    \end{macrocode}
%    Since we have used the same values for all theorem sets, we now can
%    assign them.
%    \begin{macrocode}
\global\setlength\theorempreskipamount{12pt plus 5pt minus 3pt}
\global\setlength\theorempostskipamount{8pt plus 3pt minus 1.5pt}
%    \end{macrocode}
% \end{macro}
% \end{macro}
%
%
% \begin{macro}{\@endtheorem}
%    The same holds for the macro |\@endtheorem|, which ends a
%    {\envfont theorem} environment. Since it is the same for all
%    theorem sets, it is removed from the macros |\th@|\meta{style}.
%    It simply ends the {\envfont trivlist} environment, which was
%    begun in |\@thm|.
%    \begin{macrocode}
\global\let\@endtheorem=\endtrivlist
%    \end{macrocode}
% \end{macro}
%
%
%
% \begin{macro}{\@preamblecmds}
%    All macros defined above are to be used only in the preamble.
%    Therefore, we insert them in |\@preamblecmds| which will disable
%    them at begin document. This is done by the internal \LaTeXe{}
%    command |\@onlypreamble|.
% \changes{v2.0g}{89/09/19}{`theoremheaderfont and `theorembodyfont
%    added.}
% \changes{v2.2a}{94/02/03}{Use \LaTeXe{} interface}
%    \begin{macrocode}
\@onlypreamble\@xnthm
\@onlypreamble\@ynthm
\@onlypreamble\@othm
\@onlypreamble\newtheorem
\@onlypreamble\theoremstyle
\@onlypreamble\theorembodyfont
\@onlypreamble\theoremheaderfont
%    \end{macrocode}
% \end{macro}
%
% Finally we declare the \texttt{plain} theorem style to be the default.
% \changes{v2.1b}{90/06/04}{Fixed missing `item by declaring default
%    style.}
%    \begin{macrocode}
\theoremstyle{plain}
%</package>
%    \end{macrocode}
%
% \Finale

\endinput
}{}
\ProvidesFile{thp.sty}
             [\FMithmInfo]
%    \end{macrocode}
%
%
% \begin{macro}{\th@plain}
% \begin{macro}{\@begintheorem}
% \begin{macro}{\@opargbegintheorem}
%    |\theoremstyle{plain}| corresponds to the original
%    definition, except that the distances to the surrounding text are
%    determined by the parameters |\theorempreskipamount| and
%    |\theorempostskipamount|.
%    First we set the default body font.
%    \begin{macrocode}
\gdef\th@plain{\normalfont\itshape
%    \end{macrocode}
%    Then we define |\@begintheorem| and
%    |\@opargbegintheorem|. These two macros define how the
%    header of a theorem is typeset.  |\@opargbegintheorem| will
%    be called if a \textsf{theorem} environment with an optional
%    argument is encountered; otherwise, the header is constructed by
%    calling |\@begintheorem|.  If one of these macros is
%    executed, we are within a \textsf{trivlist} environment started by
%    |\@thm|.  So the theorem header is produced with an
%    |\item| command.
%
%    Instead of specifying the header font directly, all standard
%    theorem styles use the |\theorem@headerfont| macro to allow
%    customization.  The extra space (|\labelsep|) is necessary
%    because of problems in the \textsf{trivlist} environment.
%    \begin{macrocode}
  \def\@begintheorem##1##2{%
        \item[\hskip\labelsep \theorem@headerfont ##1\ ##2]}%
%    \end{macrocode}
%    The definition of |\@opargbegintheorem| is completely
%    analogous. The only difference is the fact that there exists a
%    third argument (which is the optional parameter of the
%    environment and contains additional information about the
%    theorem).  Customarily we enclose it in parentheses.
%    \begin{macrocode}
\def\@opargbegintheorem##1##2##3{%
   \item[\hskip\labelsep \theorem@headerfont ##1\ ##2\ (##3)]}}
%    \end{macrocode}
% \end{macro}
% \end{macro}
% \end{macro}
%
% \noindent
% We conclude with an |\endgroup| to restore the
% |\catcode| of |@|.
%    \begin{macrocode}
\endgroup
%</thp>
%    \end{macrocode}
%
%
% \subsubsection{The \texttt{break} style}
%
% This style option is stored in the file |thb.sty|.
% For the next two lines see the documentation for |\th@plain| on
% page~\pageref{dok:th@plain}.
%    \begin{macrocode}
%<*thb>
\begingroup \makeatletter
\@ifundefined{theorem@style}{% \iffalse meta-comment
%
% Copyright (C) 1993-2024
%
% The LaTeX Project and any individual authors listed elsewhere
% in this file.
%
% This file is part of the Standard LaTeX `Tools Bundle'.
% -------------------------------------------------------
%
% It may be distributed and/or modified under the
% conditions of the LaTeX Project Public License, either version 1.3c
% of this license or (at your option) any later version.
% The latest version of this license is in
%    https://www.latex-project.org/lppl.txt
% and version 1.3c or later is part of all distributions of LaTeX
% version 2005/12/01 or later.
%
% The list of all files belonging to the LaTeX `Tools Bundle' is
% given in the file `manifest.txt'.
%
% \fi
%
%%
%
%
\def\FMithmInfo{2023/07/05 v2.2c Theorem extension package (FMi)}
%
% \ProvidesFile{theorem.dtx}[\FMithmInfo]
%\iffalse   % this is a METACOMMENT !
%%
%% Package `theorem' to use with LaTeX2e
%% Copyright (C) 1989-1995 Frank Mittelbach, all rights reserved.
%<+package>\@ifundefined{theorem@style}{}{\endinput}
%<+package>\NeedsTeXFormat{LaTeX2e}[1995/06/01]
%<+package>\ProvidesPackage{theorem}
%<+package>         [\FMithmInfo]
%
% \fi
%
% \changes{v1.0}{87/05/07}{First draft.}
% \changes{v1.0a}{87/05/07}{`thistheoremstyle' added.}
% \changes{v1.1a}{87/07/29}{New concept: redefine the internal
%                          \protect\LaTeX{} commands `@xnthm' etc.}
% \changes{v1.1b}{87/07/31}{`theoremfont' used to specify font.}
% \changes{v1.2a}{88/01/08}{`@endtheorem' now globally defined.}
% \changes{v1.2b}{88/01/11}{`newtheorem' only allowed in the preamble.}
% \changes{v1.3a}{88/02/14}{`theoremfont' now a toks register.}
%
% \changes{v2.0a}{88/11/16}{New implementation and new documentation.}
% \changes{v2.0a}{88/11/16}{No layout changes but internal modifications
%                           old substyles won't work.}
% \changes{v2.0b}{89/01/30}{`global' used for all definitions.}
% \changes{v2.0c}{89/06/15}{Documentation translated to English.
%                            Thanks to LLL and Nina}
% \changes{v2.1a}{90/02/15}{Support for new font selection.}
% \changes{v2.2a}{94/02/03}{Cleanup for LaTeX2e}
%
% \DoNotIndex{\@endparenv,\@ifdefinable,\@ifnextchar,\@ifundefined}
% \DoNotIndex{\@namedef,\@nameuse,\@spaces,\@tempa,\@warning}
% \DoNotIndex{\ }
% \DoNotIndex{\begingroup,\bfseries}
% \DoNotIndex{\csname}
% \DoNotIndex{\def,\do,\docdate}
% \DoNotIndex{\endcsname,\endgroup,\endinput,\endtrivlist,\expandafter}
% \DoNotIndex{\filedate,\fileversion}
% \DoNotIndex{\gdef,\global}
% \DoNotIndex{\hbox,\hskip}
% \DoNotIndex{\ignorespaces,\input,\itshape,\item}
% \DoNotIndex{\labelsep,\let,\llap}
% \DoNotIndex{\makeatletter}
% \DoNotIndex{\noexpand}
% \DoNotIndex{\refstepcounter,\relax,\rlap}
% \DoNotIndex{\setlength,\slshape,\space,\string,\strut}
% \DoNotIndex{\the,\trivlist,\typeout}
% \DoNotIndex{\vbox}
% \DoNotIndex{\xdef}
%
% \def\deflabel#1{\ttfamily #1\hfill}
% \def\deflist#1{\begin{list}{}{\settowidth\labelwidth{\ttfamily #1}%
%                               \setlength\leftmargin\labelwidth
%                               \addtolength\leftmargin\labelsep
%                               \let\makelabel\deflabel}}
% \def\enddeflist{\end{list}}
% \def\envfont{\normalfont\ttfamily}
%
% \MakeShortVerb{\|}
%
% \newcommand{\TUB}{{\sl TUGboat\/}}
%
% \GetFileInfo{theorem.dtx}
% \title{An Extension of the
%        \LaTeX{} theorem environment\thanks{This file
%        has version number \fileversion, last
%        revised \filedate.}}
% \author{Frank Mittelbach}
%
% \MaintainedByLaTeXTeam{tools}
% \maketitle
%
% \begin{abstract}
%    The macros described in this paper yield an extension of the
%    \LaTeX{} theorem mechanism. It is designed to satisfy the
%    different requirements of various journals. Thus, the
%    layout of the ``theorems'' can be manipulated by determining a
%    ``style''. This article describes not only the use, but
%    also the definition, of the necessary macros.
% \end{abstract}
%
% \section*{Preface to version 2.2}
%
% For \LaTeXe{} this package did not need any fundamental changes. I
% only modified the messages generated so that theorem layout styles
% will show up with the |\listfiles| command and cleaned the section
% on the New Font Selection Scheme since this is now included in
% \LaTeX.
%
% \section*{Preface to version 2.1}
%
% This version is identical to 2.0g described in \TUB\ 10\#3 except
% for some internal defaults which are now set depending on the used
% font selection scheme.
%
% This was done to avoid unpleasant surprises if the new font selection
% scheme is in force. For further details see
% section~\ref{sec:newfonts} and \cite{bk:GMS94}.
%
% \section{Introduction}
%
% For our purposes here, ``theorems'' are labelled enunciations,
% often set off from the main text by extra space and a font change.
% Theorems, corollaries, conjectures, definitions, and remarks
% are all instances of ``theorems''.  The ``header'' of these structures
% is composed of a label (such as \textsc{Theorem} or \textsc{Remark})
% and a number which serializes an item in the sequence of items
% with the same label.
%
% Shortly after the introduction of \LaTeX{} at the Fachbereich
% Mathematik in Mainz, the desire to manipulate the layout of
% ``theorems'' arose.  In Mainz, the following two conventions came
% into general use:
% \begin{enumerate}
%   \item The number of the theorem is shown in the margin.
%   \item There is a line break at
%         the end of the theorem header.
% \end{enumerate}
%
% Additionally, some journals require different formats which depend on
% the ``sort of theorem'': e.g.\ often remarks and definitions are set
% in |\upshape|, while |\itshape| is employed for main theorems.
%
% Confronted with these requirements, a theorem environment was
% developed in Mainz which allows separate determination of the
% layout of the ``theorems sets'', comparable to |\pagestyle|.
%
%
% \section{The user interface}
% \subsection{Defining new theorem sets}
%
% \DescribeMacro\newtheorem
% As in the original \LaTeX{} version, the command |\newtheorem|
% defines a new ``theorem set'' or ``theorem-like structure''.
% Two required arguments name the new environment and give the text
% to be typeset with each instance of the new ``set'', while
% an optional argument determines how
% the ``set'' is enumerated:
% \begin{description}
%    \item[\ttfamily \bslash newtheorem\{foo\}\{bar\}]
%       The theorem set {\envfont foo} (whose name is \texttt{bar})
%       uses its own counter.
%    \item[\ttfamily \bslash newtheorem\{foo2\}{[foo]}\{bar2\}]
%       The theorem set {\envfont foo2} (printed name \texttt{bar2})
%       uses the same counter as the theorem set \texttt{foo}.
%    \item[\ttfamily \bslash newtheorem\{foo3\}\{bar3\}{[section]}]
%       The theorem set {\envfont foo3} (printed name \texttt{bar3}) is
%       enumerated within the counter \texttt{section}, i.e.\ with every
%       new |\section| the enumeration begins again with 1, and
%       the enumeration is composed from the section-number and the
%       theorem counter itself.
% \end{description}
%
% \DescribeMacro\theoremstyle
% Additionally, the command |\theoremstyle| can define the layout
% of various, or all, theorem sets.  It should be noted that any
% theorem set defined by |\newtheorem| is typeset in the
% |\theoremstyle| that is current at the time of the definition.
% Thus, the following
% \begin{verbatim}
%    \theoremstyle{break}        \newtheorem{Cor}{Corollary}
%    \theoremstyle{plain}        \newtheorem{Exa}{Example}[section]
%\end{verbatim}                             ^^A No blank at line start
% leads to the result that the set {\envfont Cor} is formatted in the
% style \texttt{break}, while the set {\envfont Exa} and all the
% following ones are formatted in the style \texttt{plain}, unless
% another |\theoremstyle| follows.  Since the definitions installed by
% |\newtheorem| are global, one also can limit |\theoremstyle| locally
% by grouping braces.
%
% \DescribeMacro\theorembodyfont
% The choice of the font for the theorem body is completely
% independent of the chosen |\theoremstyle|; this has proven to
% be very advantageous.  For example,
% \begin{verbatim}
%   {\theorembodyfont{\upshape}          \newtheorem{Rem}{Remark}}
%\end{verbatim}
% defines a theorem set {\envfont Rem}, which will be set in
% |\upshape| in the current layout (which in our example is
% \texttt{plain}).  As with |\theoremstyle|, the |\theorembodyfont|
% chosen is that current at the time of |\newtheorem|.  If
% |\theorembodyfont| is not specified or one defines
% |\theorembodyfont{}|, then the font used will be that defined by the
% |\theoremstyle|.
%
% \DescribeMacro\theoremheaderfont It is also possible to customize
% the font used for the theorem headers.  This is, however, a global
% declaration, and therefore there should be at most one
% |\theoremheaderfont| declaration in the preamble.\footnote{If it is
% actually necessary to have different header fonts, one has to define
% new theorem styles (substituting the desired font) or specify the
% information directly in the \texttt{\bslash newtheorem} declaration
% (the unclean variant).}
%
% \DescribeMacro\theorempreskipamount
% \DescribeMacro\theorempostskipamount Two additional parameters
% affect the vertical space around the theorem environments:
% |\theorempreskipamount| and |\theorempostskipamount| define,
% respectively, the spacing before and after such an environment.
% These parameters apply for all theorem sets and can be manipulated
% with the ordinary length macros.  They are rubber lengths
% (`\textsf{skips}'), and therefore can contain \texttt{plus} and
% \texttt{minus} parts.
%
% Since the definition of theorem sets should---most sensibly---be
% placed in the preamble, we only allow installation there.
% It is therefore possible to release the memory used here after
% |\begin{document}|, in order to make room for other
% applications.
%
%
% \subsection{Existing theorem styles}
%
% The following theorem styles exist to date:
% \begin{deflist}{marginbreak}
%    \item[plain]
%       This theorem style emulates the original \LaTeX{} definition,
%       except that additionally the parameters
%       |\theorem...skipamount| are used.
%    \item[break]
%       In this style, the theorem header is followed by a line
%       break.
%    \item[marginbreak]
%       The theorem number is set in the margin, and there is a line
%       break as in \texttt{break}.
%    \item[changebreak]
%       Like \texttt{break}, but with header number and text
%       interchanged.
%    \item[change]
%       Header number and text are interchanged, without a line break.
%    \item[margin]
%       The number is set in the left margin, without a line break.
% \end{deflist}
% All styles (except \texttt{plain}) select |\slshape| as the default
% |\theorembodyfont|.
%
%
% \subsection{Examples}
%
% Given the above theorem sets {\envfont Cor}, {\envfont Exa}
% and {\envfont Rem},
% suppose that the preamble also contains the declarations:
% \begin{verbatim}
%    \theoremstyle{marginbreak}   \newtheorem{Lem}[Cor]{Lemma}
%    \theoremstyle{change}
%    \theorembodyfont{\itshape}        \newtheorem{Def}[Cor]{Definition}
%
%    \theoremheaderfont{\scshape}
%\end{verbatim}
% Then the following are some typical examples of the typeset output
% resulting from their use.
%
% \begin{Cor}
%    This is a sentence typeset in the theorem environment {\envfont
%    Cor}.
% \end{Cor}
%
% \begin{Exa}
%    This is a sentence typeset in the theorem environment {\envfont
%    Exa}.
% \end{Exa}
%
% \begin{Rem}
%    This is a sentence typeset in the theorem environment {\envfont
%    Rem}.
% \end{Rem}
%
% \begin{Lem}[Ben User]
%    This is a sentence typeset in the theorem environment {\envfont
%    Lem}.
% \end{Lem}
%
% \begin{Def}[Very impressive Definition]
%    This is a sentence typeset in the theorem environment {\envfont
%    Def}.
% \end{Def}
%
% The last two examples show the effect of the optional argument to
% a theorem environment (it is the text typeset in parentheses).
%
%
% \section{Special Considerations}
% \label{sec:newfonts}
%
% Theoremheader and body are implemented as a unit. This means that
% the |\theoremheaderfont|  will inherit characteristics of the
% |\theorembodyfont| in \LaTeXe. Thus, if for example
% |\theorembodyfont| is |\itshape| and |\theoremheaderfont|
% is |\bfseries| the font selected for the header will have the
% characteristics `bold extended italic'. If this is not desired one
% should set the |\theoremheaderfont| to something like
% \begin{verbatim}
%      \theoremheaderfont{\normalfont\bfseries}
%\end{verbatim}
% i.e.\ supplying all necessary font information explicitly.
%
%
%
% \section{Acknowledgements}
%
% The publication of this set of macros was only possible with the
% help of Christina Busse (translating the manuscript into English),
% Joachim Pense (playing the r\^ole of typist), Chris Rowley
% (looking everything over) and many others providing useful
% suggestions.
%
% \MaybeStop{%
%    \begin{thebibliography}{9}
%       \bibitem{bk:GMS94} \textsc{M.~Goossens}, \textsc{F.~Mittelbach}
%         and \textsc{A.~Samarin}.
%         \newblock The \LaTeX{} Companion.
%         \newblock
%         Addison-Wesley, Reading, Massachusetts, 1994.
%       \bibitem{latex:source} \textsc{Lamport, Leslie.} \newblock
%                               \texttt{latex.tex}, version 2.09, date
%                               Feb. 1990.
%    \end{thebibliography}
%    \PrintIndex
%    \PrintChanges}
%
%
% \section{The documentation driver file}
%
% The next bit of code contains the documentation driver file for
% \TeX{}, i.e., the file that will produce the documentation you are
% currently reading. It will be extracted from this file by the
% \texttt{docstrip} program. Since it is the first code in the file
% one can alternatively process this file directly with \LaTeXe{} to
% obtain the documentation.
%
%    \begin{macrocode}
%<*driver>
\documentclass{ltxdoc}

\usepackage{theorem}

%<+driver>% The next few lines define theorem sets which are used
%<+driver>% in the example section of the documentation.

\theoremstyle{break}       \newtheorem{Cor}{Corollary}
\theoremstyle{plain}       \newtheorem{Exa}{Example}[section]
{\theorembodyfont{\upshape}\newtheorem{Rem}{Remark}}
\theoremstyle{marginbreak} \newtheorem{Lem}[Cor]{Lemma}
\theoremstyle{change}
\theorembodyfont{\itshape} \newtheorem{Def}[Cor]{Definition}

\theoremheaderfont{\scshape}

\RecordChanges

\begin{document}
   \DocInput{theorem.dtx}
\end{document}
%</driver>
%    \end{macrocode}
%
% \section{Definition of the Macros}
%
% If the file has been loaded before, we abort immediately. If not the
% package announces itself (this is actually done at the very top of
% the file---the way it is done isn't good style so don't copy it).
% \changes{v2.0e}{89/07/19}{Spaces removed from `typeout'.}
% \changes{v2.2b}{95/11/19}{Announce moved to top of file}
%    \begin{macrocode}
%<*package>
%\@ifundefined{theorem@style}{}{\endinput}
%\def\FMithmInfo{1995/11/19 v2.2b Theorem extension package (FMi)}
%\ProvidesPackage{theorem}[\FMithmInfo]
%    \end{macrocode}
%
%
% \subsection{Definition of theorem styles and fonts}
%
% All the definitions in this file are done globally to allow
% inputting this file inside a group.
%
% \begin{macro}{\theoremstyle}
%    Before a theorem style can be installed, the chosen style must be
%    known. For that reason, we must test to see that
%    |\th@|\meta{style} is known or, more precisely, that it is
%    different from |\relax|. If the style is not known then
%    |\th@plain| is used.
%    \begin{macrocode}
\gdef\theoremstyle#1{%
   \@ifundefined{th@#1}{\@warning
          {Unknown theoremstyle `#1'. Using `plain'}%
          \theorem@style{plain}}%
%    \end{macrocode}
%    We save the theorem style to be used in the token register
%    |\theorem@style|.
%    \begin{macrocode}
      {\theorem@style{#1}}%
%    \end{macrocode}
%    Now we ``evaluate'' the theorem style: this means, we call the
%    macro |\th@|\meta{style} which will activate the relevant
%    definitions which are contained in a separate file.  This is done
%    in a group to suppress changes to the current font.  This could
%    otherwise pose problems together with the new font selection
%    scheme\footnote{When I printed the original article using the new
%    font selection scheme I ended with a document with slanted
%    typefaces (text headings and all) simply because one of the
%    theorem styles used \texttt{\bslash sl} at toplevel.} if the
%    |\th@|\meta{style} is evaluated a second time.
% \changes{v2.1a}{90/02/15}{Group added around `th@..}
% \changes{v2.1a}{90/02/15}{Reseting of `..@begintheorem removed.}
%    \begin{macrocode}
      \begingroup
        \csname th@\the\theorem@style \endcsname
      \endgroup}
%    \end{macrocode}
% \end{macro}
%
%
% \begin{macro}{\@begintheorem}
% \changes{v2.1a}{90/02/15}{Relaxed at toplevel.}
% \begin{macro}{\@opargbegintheorem}
% \changes{v2.1a}{90/02/15}{Relaxed at toplevel.}
%    We reset |\@begintheorem| and |\@opargbegintheorem| to
%    |\relax| since these commands are no longer necessary at
%    toplevel. This will save a few tokens.
%    \begin{macrocode}
\global\let\@begintheorem\relax
\global\let\@opargbegintheorem\relax
%    \end{macrocode}
% \end{macro}
% \end{macro}
%
% \begin{macro}{\theorem@style}
%    Obviously the \textsf{token} register used above has to be
%    allocated. To assure the utmost compatibility with the
%    original \LaTeX{} definition, we set the default theorem style to
%    \texttt{plain}, which implements the usual \LaTeX{} convention.
% \changes{v2.0f}{89/07/20}{Unnecessary `global' removed.}
%    \begin{macrocode}
\newtoks\theorem@style
\global\theorem@style{plain}
%    \end{macrocode}
% \end{macro}
%
% \begin{macro}{\theorembodyfont}
% \begin{macro}{\theorem@bodyfont}
%    If the |\theorembodyfont| is set by the user then it should not
%    interact with the default font set in the theorem style. When the
%    new font selection is in force this may happen if, for example,
%    the default is |\itshape| and the new |\theorembodyfont| is
%    |\sffamily|. So we add a |\reset@font| command in front of the
%    user definition.
% \changes{v2.0f}{89/07/20}{Unnecessary `global' removed.}
% \changes{v2.0g}{89/09/19}{`theoremfont to `theorembodyfont.}
% \changes{v2.1c}{92/06/22}{`reset@font added and logic changed.}
% \changes{v2.1d}{92/10/28}{Reset to empty if arg is empty}
%    \begin{macrocode}
\gdef\theorembodyfont#1{%
%    \end{macrocode}
%    We check if the argument supplied is empty and if so put nothing
%    into the |\theorem@bodyfont| token register to allow for
%    |\theorembodyfont{}| as a mean of using the default of the
%    current |\theoremstyle|.
%    \begin{macrocode}
   \def\@tempa{#1}%
   \ifx\@tempa\@empty
    \theorem@bodyfont{}%
   \else
    \theorem@bodyfont{\reset@font#1}%
   \fi
   }
\newtoks\theorem@bodyfont
\global\theorem@bodyfont{}
%    \end{macrocode}
% \end{macro}
% \end{macro}
%
%
% \begin{macro}{\theoremheaderfont}
% \changes{v2.0g}{89/09/19}{Macro added.}
%    The font for the theorem headers is handled differently because
%    this definition applies to all theorem styles.
%    \begin{macrocode}
\gdef\theoremheaderfont#1{\gdef\theorem@headerfont{#1}%
%    \end{macrocode}
%    After using the macro once it is redefined to produce an error
%    message.
%    \begin{macrocode}
       \gdef\theoremheaderfont##1{%
        \typeout{\string\theoremheaderfont\space should be used
                 only once.}}}
%    \end{macrocode}
% \end{macro}
%
% \begin{macro}{\theorem@headerfont}
% \changes{v2.0g}{89/09/19}{Macro added.}
%    To set the |\theorem@headerfont| default we first test if
%    the new fontselection scheme is in force.
% \changes{v2.1a}{90/02/15}{Support for new font selection.}
%    \begin{macrocode}
\ifx\upshape\undefined
%    \end{macrocode}
%    If not we define it to expand into |\bfseries|. We don't use
%    |\let| just in case a following style option redefines this
%    macro.
%    \begin{macrocode}
\gdef\theorem@headerfont{\bfseries}
%    \end{macrocode}
%    Otherwise we reset the current shape before calling |\bfseries|.
%    \begin{macrocode}
\else \gdef\theorem@headerfont{\normalfont\bfseries}\fi
%    \end{macrocode}
% \end{macro}
%
% \begin{macro}{\th@plain}
% \begin{macro}{\th@break}
% \begin{macro}{\th@marginbreak}
% \begin{macro}{\th@changebreak}
% \begin{macro}{\th@change}
% \begin{macro}{\th@margin}
%    The different styles are defined in macros such as
%    |\th@plain|.  Since memory space is precious in
%    ``non-Big-versions'', we have to avoid offering too many unused
%    definitions.  Therefore we define these styles in separate files
%    that can be loaded on demand.  Thus the commands themselves only
%    load these files. We use |\@input@|, a \LaTeXe{} internal command
%    that ensures that the file will be listed with |\listfiles|.
% \changes{v2.2a}{94/02/02}{Use `@input@ to load theorem layout files}
%    \begin{macrocode}
\gdef\th@plain{\@input@{thp.sty}}
\gdef\th@break{\@input@{thb.sty}}
\gdef\th@marginbreak{\@input@{thmb.sty}}
\gdef\th@changebreak{\@input@{thcb.sty}}
\gdef\th@change{\@input@{thc.sty}}
\gdef\th@margin{\@input@{thm.sty}}
%    \end{macrocode}
%    This list will be expanded when new styles become
%    available.  For testing,  just append new theorem substyles
%    as document options.
% \end{macro}
% \end{macro}
% \end{macro}
% \end{macro}
% \end{macro}
% \end{macro}
%
%
%
% \subsection{Definition of a new theorem set}
%
% As already pointed out, a new theorem environment can be defined in
% three different ways:
% \begin{verbatim}
%    \newtheorem{Lem}{Lemma}
%    \newtheorem{Lem}{Lemma}[section]
%    \newtheorem{Lem}[Theorem]{Lemma}
%\end{verbatim}
% The function of the macro |\newtheorem| is to recognize these cases
% and then to branch into one of the three macros |\@ynthm|, |\@xnthm|
% or |\@othm|.  This mechanism is adopted unchanged from
% \cite{latex:source}; the essential point here is that, for example,
% in the second case, the arguments \texttt{Lem}, \texttt{Lemma} and
% \texttt{section} are passed over to the macro |\@xnthm|.
%
% We inspect this case first because the others present fewer problems,
% and thus are easily derived from this one.
%
% \begin{macro}{\@xnthm}
%    For our example arguments, the macro |\@xnthm| must fulfill
%    the following:
%    \begin{itemize}
%       \item
%          Define a new \LaTeX-counter `Lem'
%       \item
%          reset this counter within a |\section|
%       \item
%          define the macro |\theLem|
%       \item
%          define the environment macros |\Lem| and
%          |\endLem| using the current
%          |\theoremstyle| and |\theorem@bodyfont|.
%    \end{itemize}
%    Obviously, all this should happen only if the first argument
%    of |\@xnthm| (i.e.\ \texttt{Lem} in our example) is chosen so
%    as not to conflict with any previously defined commands or
%    environments.  This test is performed by the \LaTeX{} macro
%    |\@ifdefinable|.
%    \begin{macrocode}
\gdef\@xnthm#1#2[#3]{\expandafter\@ifdefinable\csname #1\endcsname
%    \end{macrocode}
%    Therefore, the first argument of |\@ifdefinable| is the
%    expansion (in the example, |\Lem|) of
%    |\csname||#1\endcsname|.  The second argument is
%    executed only if the test has been completed successfully.
%    \begin{macrocode}
   {%
%    \end{macrocode}
%    Now we define the new counter. The names of the \LaTeX{}
%    macros employed should speak for themselves:
% \changes{v2.2b}{1995/10/07}{Use \cs{@newctr}}
% \changes{v2.2c}{1995/11/23}{Use \cs{@newctr} properly.}
%    \begin{macrocode}
    \@definecounter{#1}\@newctr{#1}[#3]%
%    \end{macrocode}
%    Using |\@newctr| will give a proper error message if the counter
%    in |#3| is not defined.
%    In defining `|\theLem|' we must generate the desired macro
%    name by use of |\expandafter| and |\csname|.
%    \begin{macrocode}
    \expandafter\xdef\csname the#1\endcsname
%    \end{macrocode}
%    An |\xdef| is used in order to make the definition global,
%    and to ensure that it contains the replacement texts of
%    |\@thmcountersep| and |\@thmcounter|.\footnote{These
%    two macros can be defined by the document style.  Their default
%    values produce a `\texttt{.}' as separation and an arabic
%    representation of the number.} However, not everything should be
%    expanded. For example, it saves space to use |\thesection|
%    instead of its---at times---lengthy expansion.
%    \begin{macrocode}
      {\expandafter \noexpand \csname the#3\endcsname
       \@thmcountersep \@thmcounter{#1}}%
%    \end{macrocode}
%    Thus with the defaults of \LaTeX{}, |\theLem| would be
%    replaced by the command sequence |\thesection.\arabic{Lem}|.
%
%    We will now look at the definition of the macro which is executed
%    at the beginning of the actual environment (in our example this
%    macro is |\Lem|). It should be noted that we use an
%    ``|\expandafter| trick'' to expand only certain parts of the
%    replacement text at the time of the definition.
%    \begin{macrocode}
    \def\@tempa{\global\@namedef{#1}}%
    \expandafter \@tempa \expandafter{%
%    \end{macrocode}
%    First, the macro that contains the current definitions of
%    |\@begintheorem| and |\@opargtheorem| should be called
%    up.  The name of this macro---as is already known---has the form
%    |\th@|\meta{theorem\ style}; therefore, it must be called by
%    \begin{macrocode}
      \csname th@\the \theorem@style
%    \end{macrocode}
%    In addition the default theorem font should be changeable, i.e.\
%    we have to insert the contents of |\theorem@bodyfont|. For
%    that reason, we expand even further, beyond |\endcsname|,
%    and thus insert the contents of the \textsf{token} register
%    |\theorem@bodyfont| in the replacement text.
%    \begin{macrocode}
            \expandafter \endcsname \the \theorem@bodyfont
%    \end{macrocode}
%    Now it is time to call the macro |\@thm| which takes over
%    the further processing.  It has two arguments: the current
%    counter name (in our example, \texttt{Lem}), and the text of the
%    label (in our example, \texttt{Lemma}).
%    \begin{macrocode}
     \@thm{#1}{#2}}%
%    \end{macrocode}
%    With this, the `sub-definition' is complete.  The macro
%    |\@endtheorem| ends a theorem environment and is, so far, nothing
%    but an |\endtrivlist|.  (Hence it is defined globally, and not
%    within the theorem styles.\footnote{This has to be changed as
%    soon as theorem styles that change \texttt{\bslash @endtheorem}
%    exist.  In such a case, all existing styles must be changed as
%    well since they will have to reset the macro.}) Therefore, we can
%    set it equivalent to the macro that ends the theorem set (in our
%    example, |\endLem|).  However, if some day theorem styles exist
%    that do change |\@endtheorem|, we would have to use the
%    commented-out line instead.
%    \begin{macrocode}
    \global \expandafter \let \csname end#1\endcsname \@endtheorem
%   \global\@namedef{end#1}{\@endtheorem}%
%    \end{macrocode}
%    With these commands all the required definitions are employed,
%    unless the test |\@ifdefinable| has failed.  Therefore, we
%    end the second argument of this macro and with it the definition
%    of |\@xnthm|.
%    \begin{macrocode}
   }}
%    \end{macrocode}
% \end{macro}
%
% \begin{macro}{\@ynthm}
%    The definition of |\@ynthm| is completely analogous. In
%    this case the new counter that is defined is not reset within
%    another counter; thus the definition of |\the...|
%    is simplified:
%    \begin{macrocode}
\gdef\@ynthm#1#2{\expandafter\@ifdefinable\csname #1\endcsname
   {\@definecounter{#1}%
    \expandafter\xdef\csname the#1\endcsname{\@thmcounter{#1}}%
%    \end{macrocode}
%    The rest of the definition corresponds literally to that of
%    |\@xnthm|:
%    \begin{macrocode}
    \def\@tempa{\global\@namedef{#1}}\expandafter \@tempa
     \expandafter{\csname th@\the \theorem@style \expandafter
     \endcsname \the\theorem@bodyfont \@thm{#1}{#2}}%
    \global \expandafter \let \csname end#1\endcsname \@endtheorem}}
%    \end{macrocode}
% \end{macro}
%
% \begin{macro}{\@othm}
%    The definition of |\@othm| does not contain anything new.
%    \begin{macrocode}
\gdef\@othm#1[#2]#3{%
%    \end{macrocode}
% \changes{v2.2b}{95/10/07}{Build in check for counter defined}
%    We do not define a new counter but instead use one that has
%    already been defined.  Thus the only definition we need is that
%    of this pseudo-counter (i.e.\ |\the|\meta{env.\ name}).
%    First we check if |#2| corresponds to a known counter name.
%    \begin{macrocode}
  \expandafter\ifx\csname c@#2\endcsname\relax
   \@nocounterr{#2}%
  \else
   \expandafter\@ifdefinable\csname #1\endcsname
   {\expandafter \xdef \csname the#1\endcsname
     {\expandafter \noexpand \csname the#2\endcsname}%
%    \end{macrocode}
%    All other parts of the definition can be adopted from
%    |\@xnthm|. We have to remember, though, that in this case
%    the name of the current counter and the theorem label have moved
%    to the second and third arguments.
%    \begin{macrocode}
    \def\@tempa{\global\@namedef{#1}}\expandafter \@tempa
     \expandafter{\csname th@\the \theorem@style \expandafter
     \endcsname \the\theorem@bodyfont \@thm{#2}{#3}}%
    \global \expandafter \let \csname end#1\endcsname \@endtheorem}%
  \fi}
%    \end{macrocode}
% \end{macro}
%
% \subsection{Macros that are employed in a {\envfont theorem}
%    environment}
%
% \begin{macro}{\@thm}
%    The macro |\@thm| has to increase the current counter.  Then,
%    depending on whether the environment has (or does not have) an
%    optional argument, it has to branch into either |\@begintheorem|
%    or |\@opargtheorem|.
%    \begin{macrocode}
\gdef\@thm#1#2{\refstepcounter{#1}%
%    \end{macrocode}
%    Now we start a {\envfont trivlist} environment, and give
%    |\@topsep| and |\@topsepadd| the values of the \textsf{skip}
%    registers |\theorempreskipamount| and |\theorempostskipamount|.
%    The value in |\@topsep| is the vertical space that is inserted by
%    the first (and only) |\item| in our |\trivlist| whilst
%    |\@topsepadd| is inserted by |\@endparenv| at the end of that
%    {\envfont trivlist} environment.  By using these registers, we
%    obtain the desired space around a {\envfont theorem} environment.
%    \begin{macrocode}
   \trivlist
   \@topsep \theorempreskipamount               % used by first \item
   \@topsepadd \theorempostskipamount           % used by \@endparenv
%    \end{macrocode}
%    Now we have to test whether an optional argument has been given.
% \changes{v2.0d}{89/07/18}{Forgotten percent added.}
%    \begin{macrocode}
   \@ifnextchar [%
%    \end{macrocode}
%    If there is an optional argument, we will call |\@ythm|, and
%    move the arguments read back into the input stream.
%    \begin{macrocode}
   {\@ythm{#1}{#2}}%
%    \end{macrocode}
%    If not, we call |\@begintheorem|. Its first argument is the
%    name of the theorem set (hence the second argument of
%    |\@thm|). Its second argument is the macro that produces the
%    current number.
%    \begin{macrocode}
   {\@begintheorem{#2}{\csname the#1\endcsname}\ignorespaces}}
%    \end{macrocode}
% \end{macro}
%
% \begin{macro}{\@xthm}
% \begin{macro}{\@ythm}
%    Both these macros were originally called by |\@thm|.  We do
%    not need |\@xthm| anymore, hence we reset it to
%    |\relax|. The definition of |\@ythm| has not changed at
%    all from its definition in \LaTeX. In order to make the macros
%    easier to understand, we will
%    nevertheless present it (commented out).
%    \begin{macrocode}
\global\let\@xthm\relax
% \def\@ythm#1#2[#3]{\@opargbegintheorem{#2}{\csname
%           the#1\endcsname}{#3}\ignorespaces}
%</package>
%    \end{macrocode}
%    The primitive |\ignorespaces| in |\@ythm| and
%    |\@thm| is needed to remove the spaces between the
%    |\begin{...}| and the actual text.
% \end{macro}
% \end{macro}
%
%
% \subsection{Definition of the theorem substyles}
%
% As already pointed out, the theorem substyles, defined below, are
% only loaded when necessary.  Note that all these substyles, except
% \texttt{plain}, have |\slshape| as the default body font.
%
% \subsubsection{The \texttt{plain} style}
%
% \label{dok:th@plain} As the following macros use |@|, we have
% to locally set the |\catcode| of this symbol to ``letter''.
% This happens within a group, so that we do not have to worry about
% which |\catcode| that symbol had before.
%    \begin{macrocode}
%<*thp>
\begingroup \makeatletter
%    \end{macrocode}
% Since we are now within a group, we must make all definitions
% globally.  First we make sure that \texttt{theorem.sty} is loaded.
% This will allow us to use this file as a document style option
% without having to call \texttt{theorem} itself as an option.  At the
% same time, we assure that at least version 2 is loaded, since
% |\theorem@style| was not defined in earlier versions.
%    \begin{macrocode}
\@ifundefined{theorem@style}{\input{theorem.sty}}{}
\ProvidesFile{thp.sty}
             [\FMithmInfo]
%    \end{macrocode}
%
%
% \begin{macro}{\th@plain}
% \begin{macro}{\@begintheorem}
% \begin{macro}{\@opargbegintheorem}
%    |\theoremstyle{plain}| corresponds to the original
%    definition, except that the distances to the surrounding text are
%    determined by the parameters |\theorempreskipamount| and
%    |\theorempostskipamount|.
%    First we set the default body font.
%    \begin{macrocode}
\gdef\th@plain{\normalfont\itshape
%    \end{macrocode}
%    Then we define |\@begintheorem| and
%    |\@opargbegintheorem|. These two macros define how the
%    header of a theorem is typeset.  |\@opargbegintheorem| will
%    be called if a \textsf{theorem} environment with an optional
%    argument is encountered; otherwise, the header is constructed by
%    calling |\@begintheorem|.  If one of these macros is
%    executed, we are within a \textsf{trivlist} environment started by
%    |\@thm|.  So the theorem header is produced with an
%    |\item| command.
%
%    Instead of specifying the header font directly, all standard
%    theorem styles use the |\theorem@headerfont| macro to allow
%    customization.  The extra space (|\labelsep|) is necessary
%    because of problems in the \textsf{trivlist} environment.
%    \begin{macrocode}
  \def\@begintheorem##1##2{%
        \item[\hskip\labelsep \theorem@headerfont ##1\ ##2]}%
%    \end{macrocode}
%    The definition of |\@opargbegintheorem| is completely
%    analogous. The only difference is the fact that there exists a
%    third argument (which is the optional parameter of the
%    environment and contains additional information about the
%    theorem).  Customarily we enclose it in parentheses.
%    \begin{macrocode}
\def\@opargbegintheorem##1##2##3{%
   \item[\hskip\labelsep \theorem@headerfont ##1\ ##2\ (##3)]}}
%    \end{macrocode}
% \end{macro}
% \end{macro}
% \end{macro}
%
% \noindent
% We conclude with an |\endgroup| to restore the
% |\catcode| of |@|.
%    \begin{macrocode}
\endgroup
%</thp>
%    \end{macrocode}
%
%
% \subsubsection{The \texttt{break} style}
%
% This style option is stored in the file |thb.sty|.
% For the next two lines see the documentation for |\th@plain| on
% page~\pageref{dok:th@plain}.
%    \begin{macrocode}
%<*thb>
\begingroup \makeatletter
\@ifundefined{theorem@style}{\input{theorem.sty}}{}
\ProvidesFile{thb.sty}
             [\FMithmInfo]
%    \end{macrocode}
%
% \begin{macro}{\th@break}
%    |\theoremstyle{break}| produces a line break after the name
%    of the theorem.  The font is |\slshape|.  Hence, we define
%    |\th@break| as follows:
%    \begin{macrocode}
\gdef\th@break{\normalfont\slshape
  \def\@begintheorem##1##2{\item[%
%    \end{macrocode}
%    We run into the following problem: it is not possible to create
%    the header with |\item[|\meta{title}|]| and then start
%    a new line by, for example, |\mbox{}\\|. Such a definition
%    will fail whenever a list environment follows immediately.  With
%    the above construction, the |\mbox{}| causes the switch
%    |@inlabel| (cf.\ definition of |\list| and
%    |\trivlist| in \cite{latex:source}) to be set to \texttt{false}
%    and so the following list will insert additional vertical
%    space (|\topskip|). This is quite annoying. Therefore, we
%    create the line break within the |\item|. In order to ensure
%    that the text will begin at the proper position in the following
%    line, we simply pretend that the label does not take any
%    room.\footnote{This will lead to problems whenever very high
%    symbols occurring in the line tower into the
%    heading. So, something else has to be done here sometime.}
%    \begin{macrocode}
       \rlap{\vbox{\hbox{\hskip \labelsep\theorem@headerfont ##1\ ##2}%
                   \hbox{\strut}}}]}%
%    \end{macrocode}
%    Again, the definition of |\@opargbegintheorem| is
%    completely analogous.
%    \begin{macrocode}
\def\@opargbegintheorem##1##2##3{%
  \item[\rlap{\vbox{\hbox{\hskip \labelsep \theorem@headerfont
                     ##1\ ##2\ (##3)}%
                    \hbox{\strut}}}]}}
\endgroup
%</thb>
%    \end{macrocode}
% \end{macro}
%
%
% \subsubsection{The \texttt{changebreak} style}
%
%    \begin{macrocode}
% This style option is stored in the file |thcb.sty|.
%    \begin{macrocode}
%<*thcb>
\begingroup \makeatletter
\@ifundefined{theorem@style}{\input{theorem.sty}}{}
\ProvidesFile{thcb.sty}
             [\FMithmInfo]
%    \end{macrocode}
%
% \begin{macro}{\th@changebreak}
%    The \texttt{change-break} theorem style is like \texttt{break}
%    but with interchange of theorem name and theorem number.  Thus we
%    define |\th@changebreak| as follows:
%    \begin{macrocode}
\gdef\th@changebreak{\normalfont\slshape
  \def\@begintheorem##1##2{\item
       [\rlap{\vbox{\hbox{\hskip\labelsep \theorem@headerfont ##2\ ##1}%
                    \hbox{\strut}}}]}%
\def\@opargbegintheorem##1##2##3{%
  \item[\rlap{\vbox{\hbox{\hskip\labelsep \theorem@headerfont
                     ##2\ ##1\ (##3)}%
                    \hbox{\strut}}}]}}
\endgroup
%</thcb>
%    \end{macrocode}
% \end{macro}
%
%
% \subsubsection{The \texttt{change} style}
%
% This style option is stored in the file |thc.sty|.
%    \begin{macrocode}
%<*thc>
\begingroup \makeatletter
\@ifundefined{theorem@style}{\input{theorem.sty}}{}
\ProvidesFile{thc.sty}
             [\FMithmInfo]
%    \end{macrocode}
%
% \begin{macro}{\th@change}
%    The \texttt{change} theorem style corresponds to the
%    \texttt{change break} style without a linebreak after the header.
%    To say it in another way, it's the same as the \texttt{plain}
%    style but with number and name interchanged and |\slshape| as the
%    default font.
%    \begin{macrocode}
\gdef\th@change{\normalfont\slshape
  \def\@begintheorem##1##2{\item
       [\hskip\labelsep \theorem@headerfont ##2\ ##1]}%
\def\@opargbegintheorem##1##2##3{%
  \item[\hskip\labelsep \theorem@headerfont ##2\ ##1\ (##3)]}}
\endgroup
%</thc>
%    \end{macrocode}
% \end{macro}
%
%
% \subsubsection{The \texttt{marginbreak} style}
%
% This style option is the one used most often at Mainz.  It is saved
% in the file |thmb.sty|.
%    \begin{macrocode}
%<*thmb>
\begingroup \makeatletter
\@ifundefined{theorem@style}{\input{theorem.sty}}{}
\ProvidesFile{thmb.sty}
             [\FMithmInfo]
%    \end{macrocode}
%
% \begin{macro}{\th@marginbreak}
%    The \texttt{margin break} style is nearly the same as the
%    \texttt{change break} style. The only difference is the placement
%    of the theorem number.  We use |\llap| to place it in the left
%    margin.
%
%    In this style |\labelsep| denotes the separation between the
%    number and the text.
%    \begin{macrocode}
\gdef\th@marginbreak{\normalfont\slshape
  \def\@begintheorem##1##2{\item
       [\rlap{\vbox{\theorem@headerfont
                    \hbox{\llap{##2}\hskip\labelsep ##1}%
                    \hbox{\strut}}}]}%
\def\@opargbegintheorem##1##2##3{%
  \item[\rlap{\vbox{\theorem@headerfont
                    \hbox{\llap{##2}\hskip\labelsep ##1\ (##3)}%
                    \hbox{\strut}}}]}}
\endgroup
%</thmb>
%    \end{macrocode}
% \changes{v2.1c}{90/02/17}{`endgroup outside macrocode env.}
% \end{macro}
%
%
%
% \subsubsection{The \texttt{margin} style}
%
% This style option is stored in the file |thm.sty|.
%    \begin{macrocode}
%<*thm>
\begingroup \makeatletter
\@ifundefined{theorem@style}{\input{theorem.sty}}{}
\ProvidesFile{thm.sty}
             [\FMithmInfo]
%    \end{macrocode}
%
% \begin{macro}{\th@margin}
%    Again this is only a variant of the theorem styles described
%    above without any new ideas.
%    \begin{macrocode}
\gdef\th@margin{\normalfont\slshape
  \def\@begintheorem##1##2{\item
       [\theorem@headerfont \llap{##2}\hskip\labelsep ##1]}%
\def\@opargbegintheorem##1##2##3{%
  \item[\theorem@headerfont \llap{##2}\hskip\labelsep ##1\ (##3)]}}
\endgroup
%</thm>
%    \end{macrocode}
% \end{macro}
%
%
%
%
% \subsection{Final Definitions}
%
% \begin{macro}{\theorempreskipamount}
% \begin{macro}{\theorempostskipamount}
%    The \textsf{skip} parameters that regulate the vertical empty space
%    before and after the {\envfont theorem} environment have
%    to be allocated as well.
% \changes{v2.0f}{89/07/20}{Unnecessary `global' removed.}
%    \begin{macrocode}
%<*package>
\newskip\theorempreskipamount
\newskip\theorempostskipamount
%    \end{macrocode}
%    Since we have used the same values for all theorem sets, we now can
%    assign them.
%    \begin{macrocode}
\global\setlength\theorempreskipamount{12pt plus 5pt minus 3pt}
\global\setlength\theorempostskipamount{8pt plus 3pt minus 1.5pt}
%    \end{macrocode}
% \end{macro}
% \end{macro}
%
%
% \begin{macro}{\@endtheorem}
%    The same holds for the macro |\@endtheorem|, which ends a
%    {\envfont theorem} environment. Since it is the same for all
%    theorem sets, it is removed from the macros |\th@|\meta{style}.
%    It simply ends the {\envfont trivlist} environment, which was
%    begun in |\@thm|.
%    \begin{macrocode}
\global\let\@endtheorem=\endtrivlist
%    \end{macrocode}
% \end{macro}
%
%
%
% \begin{macro}{\@preamblecmds}
%    All macros defined above are to be used only in the preamble.
%    Therefore, we insert them in |\@preamblecmds| which will disable
%    them at begin document. This is done by the internal \LaTeXe{}
%    command |\@onlypreamble|.
% \changes{v2.0g}{89/09/19}{`theoremheaderfont and `theorembodyfont
%    added.}
% \changes{v2.2a}{94/02/03}{Use \LaTeXe{} interface}
%    \begin{macrocode}
\@onlypreamble\@xnthm
\@onlypreamble\@ynthm
\@onlypreamble\@othm
\@onlypreamble\newtheorem
\@onlypreamble\theoremstyle
\@onlypreamble\theorembodyfont
\@onlypreamble\theoremheaderfont
%    \end{macrocode}
% \end{macro}
%
% Finally we declare the \texttt{plain} theorem style to be the default.
% \changes{v2.1b}{90/06/04}{Fixed missing `item by declaring default
%    style.}
%    \begin{macrocode}
\theoremstyle{plain}
%</package>
%    \end{macrocode}
%
% \Finale

\endinput
}{}
\ProvidesFile{thb.sty}
             [\FMithmInfo]
%    \end{macrocode}
%
% \begin{macro}{\th@break}
%    |\theoremstyle{break}| produces a line break after the name
%    of the theorem.  The font is |\slshape|.  Hence, we define
%    |\th@break| as follows:
%    \begin{macrocode}
\gdef\th@break{\normalfont\slshape
  \def\@begintheorem##1##2{\item[%
%    \end{macrocode}
%    We run into the following problem: it is not possible to create
%    the header with |\item[|\meta{title}|]| and then start
%    a new line by, for example, |\mbox{}\\|. Such a definition
%    will fail whenever a list environment follows immediately.  With
%    the above construction, the |\mbox{}| causes the switch
%    |@inlabel| (cf.\ definition of |\list| and
%    |\trivlist| in \cite{latex:source}) to be set to \texttt{false}
%    and so the following list will insert additional vertical
%    space (|\topskip|). This is quite annoying. Therefore, we
%    create the line break within the |\item|. In order to ensure
%    that the text will begin at the proper position in the following
%    line, we simply pretend that the label does not take any
%    room.\footnote{This will lead to problems whenever very high
%    symbols occurring in the line tower into the
%    heading. So, something else has to be done here sometime.}
%    \begin{macrocode}
       \rlap{\vbox{\hbox{\hskip \labelsep\theorem@headerfont ##1\ ##2}%
                   \hbox{\strut}}}]}%
%    \end{macrocode}
%    Again, the definition of |\@opargbegintheorem| is
%    completely analogous.
%    \begin{macrocode}
\def\@opargbegintheorem##1##2##3{%
  \item[\rlap{\vbox{\hbox{\hskip \labelsep \theorem@headerfont
                     ##1\ ##2\ (##3)}%
                    \hbox{\strut}}}]}}
\endgroup
%</thb>
%    \end{macrocode}
% \end{macro}
%
%
% \subsubsection{The \texttt{changebreak} style}
%
%    \begin{macrocode}
% This style option is stored in the file |thcb.sty|.
%    \begin{macrocode}
%<*thcb>
\begingroup \makeatletter
\@ifundefined{theorem@style}{% \iffalse meta-comment
%
% Copyright (C) 1993-2024
%
% The LaTeX Project and any individual authors listed elsewhere
% in this file.
%
% This file is part of the Standard LaTeX `Tools Bundle'.
% -------------------------------------------------------
%
% It may be distributed and/or modified under the
% conditions of the LaTeX Project Public License, either version 1.3c
% of this license or (at your option) any later version.
% The latest version of this license is in
%    https://www.latex-project.org/lppl.txt
% and version 1.3c or later is part of all distributions of LaTeX
% version 2005/12/01 or later.
%
% The list of all files belonging to the LaTeX `Tools Bundle' is
% given in the file `manifest.txt'.
%
% \fi
%
%%
%
%
\def\FMithmInfo{2023/07/05 v2.2c Theorem extension package (FMi)}
%
% \ProvidesFile{theorem.dtx}[\FMithmInfo]
%\iffalse   % this is a METACOMMENT !
%%
%% Package `theorem' to use with LaTeX2e
%% Copyright (C) 1989-1995 Frank Mittelbach, all rights reserved.
%<+package>\@ifundefined{theorem@style}{}{\endinput}
%<+package>\NeedsTeXFormat{LaTeX2e}[1995/06/01]
%<+package>\ProvidesPackage{theorem}
%<+package>         [\FMithmInfo]
%
% \fi
%
% \changes{v1.0}{87/05/07}{First draft.}
% \changes{v1.0a}{87/05/07}{`thistheoremstyle' added.}
% \changes{v1.1a}{87/07/29}{New concept: redefine the internal
%                          \protect\LaTeX{} commands `@xnthm' etc.}
% \changes{v1.1b}{87/07/31}{`theoremfont' used to specify font.}
% \changes{v1.2a}{88/01/08}{`@endtheorem' now globally defined.}
% \changes{v1.2b}{88/01/11}{`newtheorem' only allowed in the preamble.}
% \changes{v1.3a}{88/02/14}{`theoremfont' now a toks register.}
%
% \changes{v2.0a}{88/11/16}{New implementation and new documentation.}
% \changes{v2.0a}{88/11/16}{No layout changes but internal modifications
%                           old substyles won't work.}
% \changes{v2.0b}{89/01/30}{`global' used for all definitions.}
% \changes{v2.0c}{89/06/15}{Documentation translated to English.
%                            Thanks to LLL and Nina}
% \changes{v2.1a}{90/02/15}{Support for new font selection.}
% \changes{v2.2a}{94/02/03}{Cleanup for LaTeX2e}
%
% \DoNotIndex{\@endparenv,\@ifdefinable,\@ifnextchar,\@ifundefined}
% \DoNotIndex{\@namedef,\@nameuse,\@spaces,\@tempa,\@warning}
% \DoNotIndex{\ }
% \DoNotIndex{\begingroup,\bfseries}
% \DoNotIndex{\csname}
% \DoNotIndex{\def,\do,\docdate}
% \DoNotIndex{\endcsname,\endgroup,\endinput,\endtrivlist,\expandafter}
% \DoNotIndex{\filedate,\fileversion}
% \DoNotIndex{\gdef,\global}
% \DoNotIndex{\hbox,\hskip}
% \DoNotIndex{\ignorespaces,\input,\itshape,\item}
% \DoNotIndex{\labelsep,\let,\llap}
% \DoNotIndex{\makeatletter}
% \DoNotIndex{\noexpand}
% \DoNotIndex{\refstepcounter,\relax,\rlap}
% \DoNotIndex{\setlength,\slshape,\space,\string,\strut}
% \DoNotIndex{\the,\trivlist,\typeout}
% \DoNotIndex{\vbox}
% \DoNotIndex{\xdef}
%
% \def\deflabel#1{\ttfamily #1\hfill}
% \def\deflist#1{\begin{list}{}{\settowidth\labelwidth{\ttfamily #1}%
%                               \setlength\leftmargin\labelwidth
%                               \addtolength\leftmargin\labelsep
%                               \let\makelabel\deflabel}}
% \def\enddeflist{\end{list}}
% \def\envfont{\normalfont\ttfamily}
%
% \MakeShortVerb{\|}
%
% \newcommand{\TUB}{{\sl TUGboat\/}}
%
% \GetFileInfo{theorem.dtx}
% \title{An Extension of the
%        \LaTeX{} theorem environment\thanks{This file
%        has version number \fileversion, last
%        revised \filedate.}}
% \author{Frank Mittelbach}
%
% \MaintainedByLaTeXTeam{tools}
% \maketitle
%
% \begin{abstract}
%    The macros described in this paper yield an extension of the
%    \LaTeX{} theorem mechanism. It is designed to satisfy the
%    different requirements of various journals. Thus, the
%    layout of the ``theorems'' can be manipulated by determining a
%    ``style''. This article describes not only the use, but
%    also the definition, of the necessary macros.
% \end{abstract}
%
% \section*{Preface to version 2.2}
%
% For \LaTeXe{} this package did not need any fundamental changes. I
% only modified the messages generated so that theorem layout styles
% will show up with the |\listfiles| command and cleaned the section
% on the New Font Selection Scheme since this is now included in
% \LaTeX.
%
% \section*{Preface to version 2.1}
%
% This version is identical to 2.0g described in \TUB\ 10\#3 except
% for some internal defaults which are now set depending on the used
% font selection scheme.
%
% This was done to avoid unpleasant surprises if the new font selection
% scheme is in force. For further details see
% section~\ref{sec:newfonts} and \cite{bk:GMS94}.
%
% \section{Introduction}
%
% For our purposes here, ``theorems'' are labelled enunciations,
% often set off from the main text by extra space and a font change.
% Theorems, corollaries, conjectures, definitions, and remarks
% are all instances of ``theorems''.  The ``header'' of these structures
% is composed of a label (such as \textsc{Theorem} or \textsc{Remark})
% and a number which serializes an item in the sequence of items
% with the same label.
%
% Shortly after the introduction of \LaTeX{} at the Fachbereich
% Mathematik in Mainz, the desire to manipulate the layout of
% ``theorems'' arose.  In Mainz, the following two conventions came
% into general use:
% \begin{enumerate}
%   \item The number of the theorem is shown in the margin.
%   \item There is a line break at
%         the end of the theorem header.
% \end{enumerate}
%
% Additionally, some journals require different formats which depend on
% the ``sort of theorem'': e.g.\ often remarks and definitions are set
% in |\upshape|, while |\itshape| is employed for main theorems.
%
% Confronted with these requirements, a theorem environment was
% developed in Mainz which allows separate determination of the
% layout of the ``theorems sets'', comparable to |\pagestyle|.
%
%
% \section{The user interface}
% \subsection{Defining new theorem sets}
%
% \DescribeMacro\newtheorem
% As in the original \LaTeX{} version, the command |\newtheorem|
% defines a new ``theorem set'' or ``theorem-like structure''.
% Two required arguments name the new environment and give the text
% to be typeset with each instance of the new ``set'', while
% an optional argument determines how
% the ``set'' is enumerated:
% \begin{description}
%    \item[\ttfamily \bslash newtheorem\{foo\}\{bar\}]
%       The theorem set {\envfont foo} (whose name is \texttt{bar})
%       uses its own counter.
%    \item[\ttfamily \bslash newtheorem\{foo2\}{[foo]}\{bar2\}]
%       The theorem set {\envfont foo2} (printed name \texttt{bar2})
%       uses the same counter as the theorem set \texttt{foo}.
%    \item[\ttfamily \bslash newtheorem\{foo3\}\{bar3\}{[section]}]
%       The theorem set {\envfont foo3} (printed name \texttt{bar3}) is
%       enumerated within the counter \texttt{section}, i.e.\ with every
%       new |\section| the enumeration begins again with 1, and
%       the enumeration is composed from the section-number and the
%       theorem counter itself.
% \end{description}
%
% \DescribeMacro\theoremstyle
% Additionally, the command |\theoremstyle| can define the layout
% of various, or all, theorem sets.  It should be noted that any
% theorem set defined by |\newtheorem| is typeset in the
% |\theoremstyle| that is current at the time of the definition.
% Thus, the following
% \begin{verbatim}
%    \theoremstyle{break}        \newtheorem{Cor}{Corollary}
%    \theoremstyle{plain}        \newtheorem{Exa}{Example}[section]
%\end{verbatim}                             ^^A No blank at line start
% leads to the result that the set {\envfont Cor} is formatted in the
% style \texttt{break}, while the set {\envfont Exa} and all the
% following ones are formatted in the style \texttt{plain}, unless
% another |\theoremstyle| follows.  Since the definitions installed by
% |\newtheorem| are global, one also can limit |\theoremstyle| locally
% by grouping braces.
%
% \DescribeMacro\theorembodyfont
% The choice of the font for the theorem body is completely
% independent of the chosen |\theoremstyle|; this has proven to
% be very advantageous.  For example,
% \begin{verbatim}
%   {\theorembodyfont{\upshape}          \newtheorem{Rem}{Remark}}
%\end{verbatim}
% defines a theorem set {\envfont Rem}, which will be set in
% |\upshape| in the current layout (which in our example is
% \texttt{plain}).  As with |\theoremstyle|, the |\theorembodyfont|
% chosen is that current at the time of |\newtheorem|.  If
% |\theorembodyfont| is not specified or one defines
% |\theorembodyfont{}|, then the font used will be that defined by the
% |\theoremstyle|.
%
% \DescribeMacro\theoremheaderfont It is also possible to customize
% the font used for the theorem headers.  This is, however, a global
% declaration, and therefore there should be at most one
% |\theoremheaderfont| declaration in the preamble.\footnote{If it is
% actually necessary to have different header fonts, one has to define
% new theorem styles (substituting the desired font) or specify the
% information directly in the \texttt{\bslash newtheorem} declaration
% (the unclean variant).}
%
% \DescribeMacro\theorempreskipamount
% \DescribeMacro\theorempostskipamount Two additional parameters
% affect the vertical space around the theorem environments:
% |\theorempreskipamount| and |\theorempostskipamount| define,
% respectively, the spacing before and after such an environment.
% These parameters apply for all theorem sets and can be manipulated
% with the ordinary length macros.  They are rubber lengths
% (`\textsf{skips}'), and therefore can contain \texttt{plus} and
% \texttt{minus} parts.
%
% Since the definition of theorem sets should---most sensibly---be
% placed in the preamble, we only allow installation there.
% It is therefore possible to release the memory used here after
% |\begin{document}|, in order to make room for other
% applications.
%
%
% \subsection{Existing theorem styles}
%
% The following theorem styles exist to date:
% \begin{deflist}{marginbreak}
%    \item[plain]
%       This theorem style emulates the original \LaTeX{} definition,
%       except that additionally the parameters
%       |\theorem...skipamount| are used.
%    \item[break]
%       In this style, the theorem header is followed by a line
%       break.
%    \item[marginbreak]
%       The theorem number is set in the margin, and there is a line
%       break as in \texttt{break}.
%    \item[changebreak]
%       Like \texttt{break}, but with header number and text
%       interchanged.
%    \item[change]
%       Header number and text are interchanged, without a line break.
%    \item[margin]
%       The number is set in the left margin, without a line break.
% \end{deflist}
% All styles (except \texttt{plain}) select |\slshape| as the default
% |\theorembodyfont|.
%
%
% \subsection{Examples}
%
% Given the above theorem sets {\envfont Cor}, {\envfont Exa}
% and {\envfont Rem},
% suppose that the preamble also contains the declarations:
% \begin{verbatim}
%    \theoremstyle{marginbreak}   \newtheorem{Lem}[Cor]{Lemma}
%    \theoremstyle{change}
%    \theorembodyfont{\itshape}        \newtheorem{Def}[Cor]{Definition}
%
%    \theoremheaderfont{\scshape}
%\end{verbatim}
% Then the following are some typical examples of the typeset output
% resulting from their use.
%
% \begin{Cor}
%    This is a sentence typeset in the theorem environment {\envfont
%    Cor}.
% \end{Cor}
%
% \begin{Exa}
%    This is a sentence typeset in the theorem environment {\envfont
%    Exa}.
% \end{Exa}
%
% \begin{Rem}
%    This is a sentence typeset in the theorem environment {\envfont
%    Rem}.
% \end{Rem}
%
% \begin{Lem}[Ben User]
%    This is a sentence typeset in the theorem environment {\envfont
%    Lem}.
% \end{Lem}
%
% \begin{Def}[Very impressive Definition]
%    This is a sentence typeset in the theorem environment {\envfont
%    Def}.
% \end{Def}
%
% The last two examples show the effect of the optional argument to
% a theorem environment (it is the text typeset in parentheses).
%
%
% \section{Special Considerations}
% \label{sec:newfonts}
%
% Theoremheader and body are implemented as a unit. This means that
% the |\theoremheaderfont|  will inherit characteristics of the
% |\theorembodyfont| in \LaTeXe. Thus, if for example
% |\theorembodyfont| is |\itshape| and |\theoremheaderfont|
% is |\bfseries| the font selected for the header will have the
% characteristics `bold extended italic'. If this is not desired one
% should set the |\theoremheaderfont| to something like
% \begin{verbatim}
%      \theoremheaderfont{\normalfont\bfseries}
%\end{verbatim}
% i.e.\ supplying all necessary font information explicitly.
%
%
%
% \section{Acknowledgements}
%
% The publication of this set of macros was only possible with the
% help of Christina Busse (translating the manuscript into English),
% Joachim Pense (playing the r\^ole of typist), Chris Rowley
% (looking everything over) and many others providing useful
% suggestions.
%
% \MaybeStop{%
%    \begin{thebibliography}{9}
%       \bibitem{bk:GMS94} \textsc{M.~Goossens}, \textsc{F.~Mittelbach}
%         and \textsc{A.~Samarin}.
%         \newblock The \LaTeX{} Companion.
%         \newblock
%         Addison-Wesley, Reading, Massachusetts, 1994.
%       \bibitem{latex:source} \textsc{Lamport, Leslie.} \newblock
%                               \texttt{latex.tex}, version 2.09, date
%                               Feb. 1990.
%    \end{thebibliography}
%    \PrintIndex
%    \PrintChanges}
%
%
% \section{The documentation driver file}
%
% The next bit of code contains the documentation driver file for
% \TeX{}, i.e., the file that will produce the documentation you are
% currently reading. It will be extracted from this file by the
% \texttt{docstrip} program. Since it is the first code in the file
% one can alternatively process this file directly with \LaTeXe{} to
% obtain the documentation.
%
%    \begin{macrocode}
%<*driver>
\documentclass{ltxdoc}

\usepackage{theorem}

%<+driver>% The next few lines define theorem sets which are used
%<+driver>% in the example section of the documentation.

\theoremstyle{break}       \newtheorem{Cor}{Corollary}
\theoremstyle{plain}       \newtheorem{Exa}{Example}[section]
{\theorembodyfont{\upshape}\newtheorem{Rem}{Remark}}
\theoremstyle{marginbreak} \newtheorem{Lem}[Cor]{Lemma}
\theoremstyle{change}
\theorembodyfont{\itshape} \newtheorem{Def}[Cor]{Definition}

\theoremheaderfont{\scshape}

\RecordChanges

\begin{document}
   \DocInput{theorem.dtx}
\end{document}
%</driver>
%    \end{macrocode}
%
% \section{Definition of the Macros}
%
% If the file has been loaded before, we abort immediately. If not the
% package announces itself (this is actually done at the very top of
% the file---the way it is done isn't good style so don't copy it).
% \changes{v2.0e}{89/07/19}{Spaces removed from `typeout'.}
% \changes{v2.2b}{95/11/19}{Announce moved to top of file}
%    \begin{macrocode}
%<*package>
%\@ifundefined{theorem@style}{}{\endinput}
%\def\FMithmInfo{1995/11/19 v2.2b Theorem extension package (FMi)}
%\ProvidesPackage{theorem}[\FMithmInfo]
%    \end{macrocode}
%
%
% \subsection{Definition of theorem styles and fonts}
%
% All the definitions in this file are done globally to allow
% inputting this file inside a group.
%
% \begin{macro}{\theoremstyle}
%    Before a theorem style can be installed, the chosen style must be
%    known. For that reason, we must test to see that
%    |\th@|\meta{style} is known or, more precisely, that it is
%    different from |\relax|. If the style is not known then
%    |\th@plain| is used.
%    \begin{macrocode}
\gdef\theoremstyle#1{%
   \@ifundefined{th@#1}{\@warning
          {Unknown theoremstyle `#1'. Using `plain'}%
          \theorem@style{plain}}%
%    \end{macrocode}
%    We save the theorem style to be used in the token register
%    |\theorem@style|.
%    \begin{macrocode}
      {\theorem@style{#1}}%
%    \end{macrocode}
%    Now we ``evaluate'' the theorem style: this means, we call the
%    macro |\th@|\meta{style} which will activate the relevant
%    definitions which are contained in a separate file.  This is done
%    in a group to suppress changes to the current font.  This could
%    otherwise pose problems together with the new font selection
%    scheme\footnote{When I printed the original article using the new
%    font selection scheme I ended with a document with slanted
%    typefaces (text headings and all) simply because one of the
%    theorem styles used \texttt{\bslash sl} at toplevel.} if the
%    |\th@|\meta{style} is evaluated a second time.
% \changes{v2.1a}{90/02/15}{Group added around `th@..}
% \changes{v2.1a}{90/02/15}{Reseting of `..@begintheorem removed.}
%    \begin{macrocode}
      \begingroup
        \csname th@\the\theorem@style \endcsname
      \endgroup}
%    \end{macrocode}
% \end{macro}
%
%
% \begin{macro}{\@begintheorem}
% \changes{v2.1a}{90/02/15}{Relaxed at toplevel.}
% \begin{macro}{\@opargbegintheorem}
% \changes{v2.1a}{90/02/15}{Relaxed at toplevel.}
%    We reset |\@begintheorem| and |\@opargbegintheorem| to
%    |\relax| since these commands are no longer necessary at
%    toplevel. This will save a few tokens.
%    \begin{macrocode}
\global\let\@begintheorem\relax
\global\let\@opargbegintheorem\relax
%    \end{macrocode}
% \end{macro}
% \end{macro}
%
% \begin{macro}{\theorem@style}
%    Obviously the \textsf{token} register used above has to be
%    allocated. To assure the utmost compatibility with the
%    original \LaTeX{} definition, we set the default theorem style to
%    \texttt{plain}, which implements the usual \LaTeX{} convention.
% \changes{v2.0f}{89/07/20}{Unnecessary `global' removed.}
%    \begin{macrocode}
\newtoks\theorem@style
\global\theorem@style{plain}
%    \end{macrocode}
% \end{macro}
%
% \begin{macro}{\theorembodyfont}
% \begin{macro}{\theorem@bodyfont}
%    If the |\theorembodyfont| is set by the user then it should not
%    interact with the default font set in the theorem style. When the
%    new font selection is in force this may happen if, for example,
%    the default is |\itshape| and the new |\theorembodyfont| is
%    |\sffamily|. So we add a |\reset@font| command in front of the
%    user definition.
% \changes{v2.0f}{89/07/20}{Unnecessary `global' removed.}
% \changes{v2.0g}{89/09/19}{`theoremfont to `theorembodyfont.}
% \changes{v2.1c}{92/06/22}{`reset@font added and logic changed.}
% \changes{v2.1d}{92/10/28}{Reset to empty if arg is empty}
%    \begin{macrocode}
\gdef\theorembodyfont#1{%
%    \end{macrocode}
%    We check if the argument supplied is empty and if so put nothing
%    into the |\theorem@bodyfont| token register to allow for
%    |\theorembodyfont{}| as a mean of using the default of the
%    current |\theoremstyle|.
%    \begin{macrocode}
   \def\@tempa{#1}%
   \ifx\@tempa\@empty
    \theorem@bodyfont{}%
   \else
    \theorem@bodyfont{\reset@font#1}%
   \fi
   }
\newtoks\theorem@bodyfont
\global\theorem@bodyfont{}
%    \end{macrocode}
% \end{macro}
% \end{macro}
%
%
% \begin{macro}{\theoremheaderfont}
% \changes{v2.0g}{89/09/19}{Macro added.}
%    The font for the theorem headers is handled differently because
%    this definition applies to all theorem styles.
%    \begin{macrocode}
\gdef\theoremheaderfont#1{\gdef\theorem@headerfont{#1}%
%    \end{macrocode}
%    After using the macro once it is redefined to produce an error
%    message.
%    \begin{macrocode}
       \gdef\theoremheaderfont##1{%
        \typeout{\string\theoremheaderfont\space should be used
                 only once.}}}
%    \end{macrocode}
% \end{macro}
%
% \begin{macro}{\theorem@headerfont}
% \changes{v2.0g}{89/09/19}{Macro added.}
%    To set the |\theorem@headerfont| default we first test if
%    the new fontselection scheme is in force.
% \changes{v2.1a}{90/02/15}{Support for new font selection.}
%    \begin{macrocode}
\ifx\upshape\undefined
%    \end{macrocode}
%    If not we define it to expand into |\bfseries|. We don't use
%    |\let| just in case a following style option redefines this
%    macro.
%    \begin{macrocode}
\gdef\theorem@headerfont{\bfseries}
%    \end{macrocode}
%    Otherwise we reset the current shape before calling |\bfseries|.
%    \begin{macrocode}
\else \gdef\theorem@headerfont{\normalfont\bfseries}\fi
%    \end{macrocode}
% \end{macro}
%
% \begin{macro}{\th@plain}
% \begin{macro}{\th@break}
% \begin{macro}{\th@marginbreak}
% \begin{macro}{\th@changebreak}
% \begin{macro}{\th@change}
% \begin{macro}{\th@margin}
%    The different styles are defined in macros such as
%    |\th@plain|.  Since memory space is precious in
%    ``non-Big-versions'', we have to avoid offering too many unused
%    definitions.  Therefore we define these styles in separate files
%    that can be loaded on demand.  Thus the commands themselves only
%    load these files. We use |\@input@|, a \LaTeXe{} internal command
%    that ensures that the file will be listed with |\listfiles|.
% \changes{v2.2a}{94/02/02}{Use `@input@ to load theorem layout files}
%    \begin{macrocode}
\gdef\th@plain{\@input@{thp.sty}}
\gdef\th@break{\@input@{thb.sty}}
\gdef\th@marginbreak{\@input@{thmb.sty}}
\gdef\th@changebreak{\@input@{thcb.sty}}
\gdef\th@change{\@input@{thc.sty}}
\gdef\th@margin{\@input@{thm.sty}}
%    \end{macrocode}
%    This list will be expanded when new styles become
%    available.  For testing,  just append new theorem substyles
%    as document options.
% \end{macro}
% \end{macro}
% \end{macro}
% \end{macro}
% \end{macro}
% \end{macro}
%
%
%
% \subsection{Definition of a new theorem set}
%
% As already pointed out, a new theorem environment can be defined in
% three different ways:
% \begin{verbatim}
%    \newtheorem{Lem}{Lemma}
%    \newtheorem{Lem}{Lemma}[section]
%    \newtheorem{Lem}[Theorem]{Lemma}
%\end{verbatim}
% The function of the macro |\newtheorem| is to recognize these cases
% and then to branch into one of the three macros |\@ynthm|, |\@xnthm|
% or |\@othm|.  This mechanism is adopted unchanged from
% \cite{latex:source}; the essential point here is that, for example,
% in the second case, the arguments \texttt{Lem}, \texttt{Lemma} and
% \texttt{section} are passed over to the macro |\@xnthm|.
%
% We inspect this case first because the others present fewer problems,
% and thus are easily derived from this one.
%
% \begin{macro}{\@xnthm}
%    For our example arguments, the macro |\@xnthm| must fulfill
%    the following:
%    \begin{itemize}
%       \item
%          Define a new \LaTeX-counter `Lem'
%       \item
%          reset this counter within a |\section|
%       \item
%          define the macro |\theLem|
%       \item
%          define the environment macros |\Lem| and
%          |\endLem| using the current
%          |\theoremstyle| and |\theorem@bodyfont|.
%    \end{itemize}
%    Obviously, all this should happen only if the first argument
%    of |\@xnthm| (i.e.\ \texttt{Lem} in our example) is chosen so
%    as not to conflict with any previously defined commands or
%    environments.  This test is performed by the \LaTeX{} macro
%    |\@ifdefinable|.
%    \begin{macrocode}
\gdef\@xnthm#1#2[#3]{\expandafter\@ifdefinable\csname #1\endcsname
%    \end{macrocode}
%    Therefore, the first argument of |\@ifdefinable| is the
%    expansion (in the example, |\Lem|) of
%    |\csname||#1\endcsname|.  The second argument is
%    executed only if the test has been completed successfully.
%    \begin{macrocode}
   {%
%    \end{macrocode}
%    Now we define the new counter. The names of the \LaTeX{}
%    macros employed should speak for themselves:
% \changes{v2.2b}{1995/10/07}{Use \cs{@newctr}}
% \changes{v2.2c}{1995/11/23}{Use \cs{@newctr} properly.}
%    \begin{macrocode}
    \@definecounter{#1}\@newctr{#1}[#3]%
%    \end{macrocode}
%    Using |\@newctr| will give a proper error message if the counter
%    in |#3| is not defined.
%    In defining `|\theLem|' we must generate the desired macro
%    name by use of |\expandafter| and |\csname|.
%    \begin{macrocode}
    \expandafter\xdef\csname the#1\endcsname
%    \end{macrocode}
%    An |\xdef| is used in order to make the definition global,
%    and to ensure that it contains the replacement texts of
%    |\@thmcountersep| and |\@thmcounter|.\footnote{These
%    two macros can be defined by the document style.  Their default
%    values produce a `\texttt{.}' as separation and an arabic
%    representation of the number.} However, not everything should be
%    expanded. For example, it saves space to use |\thesection|
%    instead of its---at times---lengthy expansion.
%    \begin{macrocode}
      {\expandafter \noexpand \csname the#3\endcsname
       \@thmcountersep \@thmcounter{#1}}%
%    \end{macrocode}
%    Thus with the defaults of \LaTeX{}, |\theLem| would be
%    replaced by the command sequence |\thesection.\arabic{Lem}|.
%
%    We will now look at the definition of the macro which is executed
%    at the beginning of the actual environment (in our example this
%    macro is |\Lem|). It should be noted that we use an
%    ``|\expandafter| trick'' to expand only certain parts of the
%    replacement text at the time of the definition.
%    \begin{macrocode}
    \def\@tempa{\global\@namedef{#1}}%
    \expandafter \@tempa \expandafter{%
%    \end{macrocode}
%    First, the macro that contains the current definitions of
%    |\@begintheorem| and |\@opargtheorem| should be called
%    up.  The name of this macro---as is already known---has the form
%    |\th@|\meta{theorem\ style}; therefore, it must be called by
%    \begin{macrocode}
      \csname th@\the \theorem@style
%    \end{macrocode}
%    In addition the default theorem font should be changeable, i.e.\
%    we have to insert the contents of |\theorem@bodyfont|. For
%    that reason, we expand even further, beyond |\endcsname|,
%    and thus insert the contents of the \textsf{token} register
%    |\theorem@bodyfont| in the replacement text.
%    \begin{macrocode}
            \expandafter \endcsname \the \theorem@bodyfont
%    \end{macrocode}
%    Now it is time to call the macro |\@thm| which takes over
%    the further processing.  It has two arguments: the current
%    counter name (in our example, \texttt{Lem}), and the text of the
%    label (in our example, \texttt{Lemma}).
%    \begin{macrocode}
     \@thm{#1}{#2}}%
%    \end{macrocode}
%    With this, the `sub-definition' is complete.  The macro
%    |\@endtheorem| ends a theorem environment and is, so far, nothing
%    but an |\endtrivlist|.  (Hence it is defined globally, and not
%    within the theorem styles.\footnote{This has to be changed as
%    soon as theorem styles that change \texttt{\bslash @endtheorem}
%    exist.  In such a case, all existing styles must be changed as
%    well since they will have to reset the macro.}) Therefore, we can
%    set it equivalent to the macro that ends the theorem set (in our
%    example, |\endLem|).  However, if some day theorem styles exist
%    that do change |\@endtheorem|, we would have to use the
%    commented-out line instead.
%    \begin{macrocode}
    \global \expandafter \let \csname end#1\endcsname \@endtheorem
%   \global\@namedef{end#1}{\@endtheorem}%
%    \end{macrocode}
%    With these commands all the required definitions are employed,
%    unless the test |\@ifdefinable| has failed.  Therefore, we
%    end the second argument of this macro and with it the definition
%    of |\@xnthm|.
%    \begin{macrocode}
   }}
%    \end{macrocode}
% \end{macro}
%
% \begin{macro}{\@ynthm}
%    The definition of |\@ynthm| is completely analogous. In
%    this case the new counter that is defined is not reset within
%    another counter; thus the definition of |\the...|
%    is simplified:
%    \begin{macrocode}
\gdef\@ynthm#1#2{\expandafter\@ifdefinable\csname #1\endcsname
   {\@definecounter{#1}%
    \expandafter\xdef\csname the#1\endcsname{\@thmcounter{#1}}%
%    \end{macrocode}
%    The rest of the definition corresponds literally to that of
%    |\@xnthm|:
%    \begin{macrocode}
    \def\@tempa{\global\@namedef{#1}}\expandafter \@tempa
     \expandafter{\csname th@\the \theorem@style \expandafter
     \endcsname \the\theorem@bodyfont \@thm{#1}{#2}}%
    \global \expandafter \let \csname end#1\endcsname \@endtheorem}}
%    \end{macrocode}
% \end{macro}
%
% \begin{macro}{\@othm}
%    The definition of |\@othm| does not contain anything new.
%    \begin{macrocode}
\gdef\@othm#1[#2]#3{%
%    \end{macrocode}
% \changes{v2.2b}{95/10/07}{Build in check for counter defined}
%    We do not define a new counter but instead use one that has
%    already been defined.  Thus the only definition we need is that
%    of this pseudo-counter (i.e.\ |\the|\meta{env.\ name}).
%    First we check if |#2| corresponds to a known counter name.
%    \begin{macrocode}
  \expandafter\ifx\csname c@#2\endcsname\relax
   \@nocounterr{#2}%
  \else
   \expandafter\@ifdefinable\csname #1\endcsname
   {\expandafter \xdef \csname the#1\endcsname
     {\expandafter \noexpand \csname the#2\endcsname}%
%    \end{macrocode}
%    All other parts of the definition can be adopted from
%    |\@xnthm|. We have to remember, though, that in this case
%    the name of the current counter and the theorem label have moved
%    to the second and third arguments.
%    \begin{macrocode}
    \def\@tempa{\global\@namedef{#1}}\expandafter \@tempa
     \expandafter{\csname th@\the \theorem@style \expandafter
     \endcsname \the\theorem@bodyfont \@thm{#2}{#3}}%
    \global \expandafter \let \csname end#1\endcsname \@endtheorem}%
  \fi}
%    \end{macrocode}
% \end{macro}
%
% \subsection{Macros that are employed in a {\envfont theorem}
%    environment}
%
% \begin{macro}{\@thm}
%    The macro |\@thm| has to increase the current counter.  Then,
%    depending on whether the environment has (or does not have) an
%    optional argument, it has to branch into either |\@begintheorem|
%    or |\@opargtheorem|.
%    \begin{macrocode}
\gdef\@thm#1#2{\refstepcounter{#1}%
%    \end{macrocode}
%    Now we start a {\envfont trivlist} environment, and give
%    |\@topsep| and |\@topsepadd| the values of the \textsf{skip}
%    registers |\theorempreskipamount| and |\theorempostskipamount|.
%    The value in |\@topsep| is the vertical space that is inserted by
%    the first (and only) |\item| in our |\trivlist| whilst
%    |\@topsepadd| is inserted by |\@endparenv| at the end of that
%    {\envfont trivlist} environment.  By using these registers, we
%    obtain the desired space around a {\envfont theorem} environment.
%    \begin{macrocode}
   \trivlist
   \@topsep \theorempreskipamount               % used by first \item
   \@topsepadd \theorempostskipamount           % used by \@endparenv
%    \end{macrocode}
%    Now we have to test whether an optional argument has been given.
% \changes{v2.0d}{89/07/18}{Forgotten percent added.}
%    \begin{macrocode}
   \@ifnextchar [%
%    \end{macrocode}
%    If there is an optional argument, we will call |\@ythm|, and
%    move the arguments read back into the input stream.
%    \begin{macrocode}
   {\@ythm{#1}{#2}}%
%    \end{macrocode}
%    If not, we call |\@begintheorem|. Its first argument is the
%    name of the theorem set (hence the second argument of
%    |\@thm|). Its second argument is the macro that produces the
%    current number.
%    \begin{macrocode}
   {\@begintheorem{#2}{\csname the#1\endcsname}\ignorespaces}}
%    \end{macrocode}
% \end{macro}
%
% \begin{macro}{\@xthm}
% \begin{macro}{\@ythm}
%    Both these macros were originally called by |\@thm|.  We do
%    not need |\@xthm| anymore, hence we reset it to
%    |\relax|. The definition of |\@ythm| has not changed at
%    all from its definition in \LaTeX. In order to make the macros
%    easier to understand, we will
%    nevertheless present it (commented out).
%    \begin{macrocode}
\global\let\@xthm\relax
% \def\@ythm#1#2[#3]{\@opargbegintheorem{#2}{\csname
%           the#1\endcsname}{#3}\ignorespaces}
%</package>
%    \end{macrocode}
%    The primitive |\ignorespaces| in |\@ythm| and
%    |\@thm| is needed to remove the spaces between the
%    |\begin{...}| and the actual text.
% \end{macro}
% \end{macro}
%
%
% \subsection{Definition of the theorem substyles}
%
% As already pointed out, the theorem substyles, defined below, are
% only loaded when necessary.  Note that all these substyles, except
% \texttt{plain}, have |\slshape| as the default body font.
%
% \subsubsection{The \texttt{plain} style}
%
% \label{dok:th@plain} As the following macros use |@|, we have
% to locally set the |\catcode| of this symbol to ``letter''.
% This happens within a group, so that we do not have to worry about
% which |\catcode| that symbol had before.
%    \begin{macrocode}
%<*thp>
\begingroup \makeatletter
%    \end{macrocode}
% Since we are now within a group, we must make all definitions
% globally.  First we make sure that \texttt{theorem.sty} is loaded.
% This will allow us to use this file as a document style option
% without having to call \texttt{theorem} itself as an option.  At the
% same time, we assure that at least version 2 is loaded, since
% |\theorem@style| was not defined in earlier versions.
%    \begin{macrocode}
\@ifundefined{theorem@style}{\input{theorem.sty}}{}
\ProvidesFile{thp.sty}
             [\FMithmInfo]
%    \end{macrocode}
%
%
% \begin{macro}{\th@plain}
% \begin{macro}{\@begintheorem}
% \begin{macro}{\@opargbegintheorem}
%    |\theoremstyle{plain}| corresponds to the original
%    definition, except that the distances to the surrounding text are
%    determined by the parameters |\theorempreskipamount| and
%    |\theorempostskipamount|.
%    First we set the default body font.
%    \begin{macrocode}
\gdef\th@plain{\normalfont\itshape
%    \end{macrocode}
%    Then we define |\@begintheorem| and
%    |\@opargbegintheorem|. These two macros define how the
%    header of a theorem is typeset.  |\@opargbegintheorem| will
%    be called if a \textsf{theorem} environment with an optional
%    argument is encountered; otherwise, the header is constructed by
%    calling |\@begintheorem|.  If one of these macros is
%    executed, we are within a \textsf{trivlist} environment started by
%    |\@thm|.  So the theorem header is produced with an
%    |\item| command.
%
%    Instead of specifying the header font directly, all standard
%    theorem styles use the |\theorem@headerfont| macro to allow
%    customization.  The extra space (|\labelsep|) is necessary
%    because of problems in the \textsf{trivlist} environment.
%    \begin{macrocode}
  \def\@begintheorem##1##2{%
        \item[\hskip\labelsep \theorem@headerfont ##1\ ##2]}%
%    \end{macrocode}
%    The definition of |\@opargbegintheorem| is completely
%    analogous. The only difference is the fact that there exists a
%    third argument (which is the optional parameter of the
%    environment and contains additional information about the
%    theorem).  Customarily we enclose it in parentheses.
%    \begin{macrocode}
\def\@opargbegintheorem##1##2##3{%
   \item[\hskip\labelsep \theorem@headerfont ##1\ ##2\ (##3)]}}
%    \end{macrocode}
% \end{macro}
% \end{macro}
% \end{macro}
%
% \noindent
% We conclude with an |\endgroup| to restore the
% |\catcode| of |@|.
%    \begin{macrocode}
\endgroup
%</thp>
%    \end{macrocode}
%
%
% \subsubsection{The \texttt{break} style}
%
% This style option is stored in the file |thb.sty|.
% For the next two lines see the documentation for |\th@plain| on
% page~\pageref{dok:th@plain}.
%    \begin{macrocode}
%<*thb>
\begingroup \makeatletter
\@ifundefined{theorem@style}{\input{theorem.sty}}{}
\ProvidesFile{thb.sty}
             [\FMithmInfo]
%    \end{macrocode}
%
% \begin{macro}{\th@break}
%    |\theoremstyle{break}| produces a line break after the name
%    of the theorem.  The font is |\slshape|.  Hence, we define
%    |\th@break| as follows:
%    \begin{macrocode}
\gdef\th@break{\normalfont\slshape
  \def\@begintheorem##1##2{\item[%
%    \end{macrocode}
%    We run into the following problem: it is not possible to create
%    the header with |\item[|\meta{title}|]| and then start
%    a new line by, for example, |\mbox{}\\|. Such a definition
%    will fail whenever a list environment follows immediately.  With
%    the above construction, the |\mbox{}| causes the switch
%    |@inlabel| (cf.\ definition of |\list| and
%    |\trivlist| in \cite{latex:source}) to be set to \texttt{false}
%    and so the following list will insert additional vertical
%    space (|\topskip|). This is quite annoying. Therefore, we
%    create the line break within the |\item|. In order to ensure
%    that the text will begin at the proper position in the following
%    line, we simply pretend that the label does not take any
%    room.\footnote{This will lead to problems whenever very high
%    symbols occurring in the line tower into the
%    heading. So, something else has to be done here sometime.}
%    \begin{macrocode}
       \rlap{\vbox{\hbox{\hskip \labelsep\theorem@headerfont ##1\ ##2}%
                   \hbox{\strut}}}]}%
%    \end{macrocode}
%    Again, the definition of |\@opargbegintheorem| is
%    completely analogous.
%    \begin{macrocode}
\def\@opargbegintheorem##1##2##3{%
  \item[\rlap{\vbox{\hbox{\hskip \labelsep \theorem@headerfont
                     ##1\ ##2\ (##3)}%
                    \hbox{\strut}}}]}}
\endgroup
%</thb>
%    \end{macrocode}
% \end{macro}
%
%
% \subsubsection{The \texttt{changebreak} style}
%
%    \begin{macrocode}
% This style option is stored in the file |thcb.sty|.
%    \begin{macrocode}
%<*thcb>
\begingroup \makeatletter
\@ifundefined{theorem@style}{\input{theorem.sty}}{}
\ProvidesFile{thcb.sty}
             [\FMithmInfo]
%    \end{macrocode}
%
% \begin{macro}{\th@changebreak}
%    The \texttt{change-break} theorem style is like \texttt{break}
%    but with interchange of theorem name and theorem number.  Thus we
%    define |\th@changebreak| as follows:
%    \begin{macrocode}
\gdef\th@changebreak{\normalfont\slshape
  \def\@begintheorem##1##2{\item
       [\rlap{\vbox{\hbox{\hskip\labelsep \theorem@headerfont ##2\ ##1}%
                    \hbox{\strut}}}]}%
\def\@opargbegintheorem##1##2##3{%
  \item[\rlap{\vbox{\hbox{\hskip\labelsep \theorem@headerfont
                     ##2\ ##1\ (##3)}%
                    \hbox{\strut}}}]}}
\endgroup
%</thcb>
%    \end{macrocode}
% \end{macro}
%
%
% \subsubsection{The \texttt{change} style}
%
% This style option is stored in the file |thc.sty|.
%    \begin{macrocode}
%<*thc>
\begingroup \makeatletter
\@ifundefined{theorem@style}{\input{theorem.sty}}{}
\ProvidesFile{thc.sty}
             [\FMithmInfo]
%    \end{macrocode}
%
% \begin{macro}{\th@change}
%    The \texttt{change} theorem style corresponds to the
%    \texttt{change break} style without a linebreak after the header.
%    To say it in another way, it's the same as the \texttt{plain}
%    style but with number and name interchanged and |\slshape| as the
%    default font.
%    \begin{macrocode}
\gdef\th@change{\normalfont\slshape
  \def\@begintheorem##1##2{\item
       [\hskip\labelsep \theorem@headerfont ##2\ ##1]}%
\def\@opargbegintheorem##1##2##3{%
  \item[\hskip\labelsep \theorem@headerfont ##2\ ##1\ (##3)]}}
\endgroup
%</thc>
%    \end{macrocode}
% \end{macro}
%
%
% \subsubsection{The \texttt{marginbreak} style}
%
% This style option is the one used most often at Mainz.  It is saved
% in the file |thmb.sty|.
%    \begin{macrocode}
%<*thmb>
\begingroup \makeatletter
\@ifundefined{theorem@style}{\input{theorem.sty}}{}
\ProvidesFile{thmb.sty}
             [\FMithmInfo]
%    \end{macrocode}
%
% \begin{macro}{\th@marginbreak}
%    The \texttt{margin break} style is nearly the same as the
%    \texttt{change break} style. The only difference is the placement
%    of the theorem number.  We use |\llap| to place it in the left
%    margin.
%
%    In this style |\labelsep| denotes the separation between the
%    number and the text.
%    \begin{macrocode}
\gdef\th@marginbreak{\normalfont\slshape
  \def\@begintheorem##1##2{\item
       [\rlap{\vbox{\theorem@headerfont
                    \hbox{\llap{##2}\hskip\labelsep ##1}%
                    \hbox{\strut}}}]}%
\def\@opargbegintheorem##1##2##3{%
  \item[\rlap{\vbox{\theorem@headerfont
                    \hbox{\llap{##2}\hskip\labelsep ##1\ (##3)}%
                    \hbox{\strut}}}]}}
\endgroup
%</thmb>
%    \end{macrocode}
% \changes{v2.1c}{90/02/17}{`endgroup outside macrocode env.}
% \end{macro}
%
%
%
% \subsubsection{The \texttt{margin} style}
%
% This style option is stored in the file |thm.sty|.
%    \begin{macrocode}
%<*thm>
\begingroup \makeatletter
\@ifundefined{theorem@style}{\input{theorem.sty}}{}
\ProvidesFile{thm.sty}
             [\FMithmInfo]
%    \end{macrocode}
%
% \begin{macro}{\th@margin}
%    Again this is only a variant of the theorem styles described
%    above without any new ideas.
%    \begin{macrocode}
\gdef\th@margin{\normalfont\slshape
  \def\@begintheorem##1##2{\item
       [\theorem@headerfont \llap{##2}\hskip\labelsep ##1]}%
\def\@opargbegintheorem##1##2##3{%
  \item[\theorem@headerfont \llap{##2}\hskip\labelsep ##1\ (##3)]}}
\endgroup
%</thm>
%    \end{macrocode}
% \end{macro}
%
%
%
%
% \subsection{Final Definitions}
%
% \begin{macro}{\theorempreskipamount}
% \begin{macro}{\theorempostskipamount}
%    The \textsf{skip} parameters that regulate the vertical empty space
%    before and after the {\envfont theorem} environment have
%    to be allocated as well.
% \changes{v2.0f}{89/07/20}{Unnecessary `global' removed.}
%    \begin{macrocode}
%<*package>
\newskip\theorempreskipamount
\newskip\theorempostskipamount
%    \end{macrocode}
%    Since we have used the same values for all theorem sets, we now can
%    assign them.
%    \begin{macrocode}
\global\setlength\theorempreskipamount{12pt plus 5pt minus 3pt}
\global\setlength\theorempostskipamount{8pt plus 3pt minus 1.5pt}
%    \end{macrocode}
% \end{macro}
% \end{macro}
%
%
% \begin{macro}{\@endtheorem}
%    The same holds for the macro |\@endtheorem|, which ends a
%    {\envfont theorem} environment. Since it is the same for all
%    theorem sets, it is removed from the macros |\th@|\meta{style}.
%    It simply ends the {\envfont trivlist} environment, which was
%    begun in |\@thm|.
%    \begin{macrocode}
\global\let\@endtheorem=\endtrivlist
%    \end{macrocode}
% \end{macro}
%
%
%
% \begin{macro}{\@preamblecmds}
%    All macros defined above are to be used only in the preamble.
%    Therefore, we insert them in |\@preamblecmds| which will disable
%    them at begin document. This is done by the internal \LaTeXe{}
%    command |\@onlypreamble|.
% \changes{v2.0g}{89/09/19}{`theoremheaderfont and `theorembodyfont
%    added.}
% \changes{v2.2a}{94/02/03}{Use \LaTeXe{} interface}
%    \begin{macrocode}
\@onlypreamble\@xnthm
\@onlypreamble\@ynthm
\@onlypreamble\@othm
\@onlypreamble\newtheorem
\@onlypreamble\theoremstyle
\@onlypreamble\theorembodyfont
\@onlypreamble\theoremheaderfont
%    \end{macrocode}
% \end{macro}
%
% Finally we declare the \texttt{plain} theorem style to be the default.
% \changes{v2.1b}{90/06/04}{Fixed missing `item by declaring default
%    style.}
%    \begin{macrocode}
\theoremstyle{plain}
%</package>
%    \end{macrocode}
%
% \Finale

\endinput
}{}
\ProvidesFile{thcb.sty}
             [\FMithmInfo]
%    \end{macrocode}
%
% \begin{macro}{\th@changebreak}
%    The \texttt{change-break} theorem style is like \texttt{break}
%    but with interchange of theorem name and theorem number.  Thus we
%    define |\th@changebreak| as follows:
%    \begin{macrocode}
\gdef\th@changebreak{\normalfont\slshape
  \def\@begintheorem##1##2{\item
       [\rlap{\vbox{\hbox{\hskip\labelsep \theorem@headerfont ##2\ ##1}%
                    \hbox{\strut}}}]}%
\def\@opargbegintheorem##1##2##3{%
  \item[\rlap{\vbox{\hbox{\hskip\labelsep \theorem@headerfont
                     ##2\ ##1\ (##3)}%
                    \hbox{\strut}}}]}}
\endgroup
%</thcb>
%    \end{macrocode}
% \end{macro}
%
%
% \subsubsection{The \texttt{change} style}
%
% This style option is stored in the file |thc.sty|.
%    \begin{macrocode}
%<*thc>
\begingroup \makeatletter
\@ifundefined{theorem@style}{% \iffalse meta-comment
%
% Copyright (C) 1993-2024
%
% The LaTeX Project and any individual authors listed elsewhere
% in this file.
%
% This file is part of the Standard LaTeX `Tools Bundle'.
% -------------------------------------------------------
%
% It may be distributed and/or modified under the
% conditions of the LaTeX Project Public License, either version 1.3c
% of this license or (at your option) any later version.
% The latest version of this license is in
%    https://www.latex-project.org/lppl.txt
% and version 1.3c or later is part of all distributions of LaTeX
% version 2005/12/01 or later.
%
% The list of all files belonging to the LaTeX `Tools Bundle' is
% given in the file `manifest.txt'.
%
% \fi
%
%%
%
%
\def\FMithmInfo{2023/07/05 v2.2c Theorem extension package (FMi)}
%
% \ProvidesFile{theorem.dtx}[\FMithmInfo]
%\iffalse   % this is a METACOMMENT !
%%
%% Package `theorem' to use with LaTeX2e
%% Copyright (C) 1989-1995 Frank Mittelbach, all rights reserved.
%<+package>\@ifundefined{theorem@style}{}{\endinput}
%<+package>\NeedsTeXFormat{LaTeX2e}[1995/06/01]
%<+package>\ProvidesPackage{theorem}
%<+package>         [\FMithmInfo]
%
% \fi
%
% \changes{v1.0}{87/05/07}{First draft.}
% \changes{v1.0a}{87/05/07}{`thistheoremstyle' added.}
% \changes{v1.1a}{87/07/29}{New concept: redefine the internal
%                          \protect\LaTeX{} commands `@xnthm' etc.}
% \changes{v1.1b}{87/07/31}{`theoremfont' used to specify font.}
% \changes{v1.2a}{88/01/08}{`@endtheorem' now globally defined.}
% \changes{v1.2b}{88/01/11}{`newtheorem' only allowed in the preamble.}
% \changes{v1.3a}{88/02/14}{`theoremfont' now a toks register.}
%
% \changes{v2.0a}{88/11/16}{New implementation and new documentation.}
% \changes{v2.0a}{88/11/16}{No layout changes but internal modifications
%                           old substyles won't work.}
% \changes{v2.0b}{89/01/30}{`global' used for all definitions.}
% \changes{v2.0c}{89/06/15}{Documentation translated to English.
%                            Thanks to LLL and Nina}
% \changes{v2.1a}{90/02/15}{Support for new font selection.}
% \changes{v2.2a}{94/02/03}{Cleanup for LaTeX2e}
%
% \DoNotIndex{\@endparenv,\@ifdefinable,\@ifnextchar,\@ifundefined}
% \DoNotIndex{\@namedef,\@nameuse,\@spaces,\@tempa,\@warning}
% \DoNotIndex{\ }
% \DoNotIndex{\begingroup,\bfseries}
% \DoNotIndex{\csname}
% \DoNotIndex{\def,\do,\docdate}
% \DoNotIndex{\endcsname,\endgroup,\endinput,\endtrivlist,\expandafter}
% \DoNotIndex{\filedate,\fileversion}
% \DoNotIndex{\gdef,\global}
% \DoNotIndex{\hbox,\hskip}
% \DoNotIndex{\ignorespaces,\input,\itshape,\item}
% \DoNotIndex{\labelsep,\let,\llap}
% \DoNotIndex{\makeatletter}
% \DoNotIndex{\noexpand}
% \DoNotIndex{\refstepcounter,\relax,\rlap}
% \DoNotIndex{\setlength,\slshape,\space,\string,\strut}
% \DoNotIndex{\the,\trivlist,\typeout}
% \DoNotIndex{\vbox}
% \DoNotIndex{\xdef}
%
% \def\deflabel#1{\ttfamily #1\hfill}
% \def\deflist#1{\begin{list}{}{\settowidth\labelwidth{\ttfamily #1}%
%                               \setlength\leftmargin\labelwidth
%                               \addtolength\leftmargin\labelsep
%                               \let\makelabel\deflabel}}
% \def\enddeflist{\end{list}}
% \def\envfont{\normalfont\ttfamily}
%
% \MakeShortVerb{\|}
%
% \newcommand{\TUB}{{\sl TUGboat\/}}
%
% \GetFileInfo{theorem.dtx}
% \title{An Extension of the
%        \LaTeX{} theorem environment\thanks{This file
%        has version number \fileversion, last
%        revised \filedate.}}
% \author{Frank Mittelbach}
%
% \MaintainedByLaTeXTeam{tools}
% \maketitle
%
% \begin{abstract}
%    The macros described in this paper yield an extension of the
%    \LaTeX{} theorem mechanism. It is designed to satisfy the
%    different requirements of various journals. Thus, the
%    layout of the ``theorems'' can be manipulated by determining a
%    ``style''. This article describes not only the use, but
%    also the definition, of the necessary macros.
% \end{abstract}
%
% \section*{Preface to version 2.2}
%
% For \LaTeXe{} this package did not need any fundamental changes. I
% only modified the messages generated so that theorem layout styles
% will show up with the |\listfiles| command and cleaned the section
% on the New Font Selection Scheme since this is now included in
% \LaTeX.
%
% \section*{Preface to version 2.1}
%
% This version is identical to 2.0g described in \TUB\ 10\#3 except
% for some internal defaults which are now set depending on the used
% font selection scheme.
%
% This was done to avoid unpleasant surprises if the new font selection
% scheme is in force. For further details see
% section~\ref{sec:newfonts} and \cite{bk:GMS94}.
%
% \section{Introduction}
%
% For our purposes here, ``theorems'' are labelled enunciations,
% often set off from the main text by extra space and a font change.
% Theorems, corollaries, conjectures, definitions, and remarks
% are all instances of ``theorems''.  The ``header'' of these structures
% is composed of a label (such as \textsc{Theorem} or \textsc{Remark})
% and a number which serializes an item in the sequence of items
% with the same label.
%
% Shortly after the introduction of \LaTeX{} at the Fachbereich
% Mathematik in Mainz, the desire to manipulate the layout of
% ``theorems'' arose.  In Mainz, the following two conventions came
% into general use:
% \begin{enumerate}
%   \item The number of the theorem is shown in the margin.
%   \item There is a line break at
%         the end of the theorem header.
% \end{enumerate}
%
% Additionally, some journals require different formats which depend on
% the ``sort of theorem'': e.g.\ often remarks and definitions are set
% in |\upshape|, while |\itshape| is employed for main theorems.
%
% Confronted with these requirements, a theorem environment was
% developed in Mainz which allows separate determination of the
% layout of the ``theorems sets'', comparable to |\pagestyle|.
%
%
% \section{The user interface}
% \subsection{Defining new theorem sets}
%
% \DescribeMacro\newtheorem
% As in the original \LaTeX{} version, the command |\newtheorem|
% defines a new ``theorem set'' or ``theorem-like structure''.
% Two required arguments name the new environment and give the text
% to be typeset with each instance of the new ``set'', while
% an optional argument determines how
% the ``set'' is enumerated:
% \begin{description}
%    \item[\ttfamily \bslash newtheorem\{foo\}\{bar\}]
%       The theorem set {\envfont foo} (whose name is \texttt{bar})
%       uses its own counter.
%    \item[\ttfamily \bslash newtheorem\{foo2\}{[foo]}\{bar2\}]
%       The theorem set {\envfont foo2} (printed name \texttt{bar2})
%       uses the same counter as the theorem set \texttt{foo}.
%    \item[\ttfamily \bslash newtheorem\{foo3\}\{bar3\}{[section]}]
%       The theorem set {\envfont foo3} (printed name \texttt{bar3}) is
%       enumerated within the counter \texttt{section}, i.e.\ with every
%       new |\section| the enumeration begins again with 1, and
%       the enumeration is composed from the section-number and the
%       theorem counter itself.
% \end{description}
%
% \DescribeMacro\theoremstyle
% Additionally, the command |\theoremstyle| can define the layout
% of various, or all, theorem sets.  It should be noted that any
% theorem set defined by |\newtheorem| is typeset in the
% |\theoremstyle| that is current at the time of the definition.
% Thus, the following
% \begin{verbatim}
%    \theoremstyle{break}        \newtheorem{Cor}{Corollary}
%    \theoremstyle{plain}        \newtheorem{Exa}{Example}[section]
%\end{verbatim}                             ^^A No blank at line start
% leads to the result that the set {\envfont Cor} is formatted in the
% style \texttt{break}, while the set {\envfont Exa} and all the
% following ones are formatted in the style \texttt{plain}, unless
% another |\theoremstyle| follows.  Since the definitions installed by
% |\newtheorem| are global, one also can limit |\theoremstyle| locally
% by grouping braces.
%
% \DescribeMacro\theorembodyfont
% The choice of the font for the theorem body is completely
% independent of the chosen |\theoremstyle|; this has proven to
% be very advantageous.  For example,
% \begin{verbatim}
%   {\theorembodyfont{\upshape}          \newtheorem{Rem}{Remark}}
%\end{verbatim}
% defines a theorem set {\envfont Rem}, which will be set in
% |\upshape| in the current layout (which in our example is
% \texttt{plain}).  As with |\theoremstyle|, the |\theorembodyfont|
% chosen is that current at the time of |\newtheorem|.  If
% |\theorembodyfont| is not specified or one defines
% |\theorembodyfont{}|, then the font used will be that defined by the
% |\theoremstyle|.
%
% \DescribeMacro\theoremheaderfont It is also possible to customize
% the font used for the theorem headers.  This is, however, a global
% declaration, and therefore there should be at most one
% |\theoremheaderfont| declaration in the preamble.\footnote{If it is
% actually necessary to have different header fonts, one has to define
% new theorem styles (substituting the desired font) or specify the
% information directly in the \texttt{\bslash newtheorem} declaration
% (the unclean variant).}
%
% \DescribeMacro\theorempreskipamount
% \DescribeMacro\theorempostskipamount Two additional parameters
% affect the vertical space around the theorem environments:
% |\theorempreskipamount| and |\theorempostskipamount| define,
% respectively, the spacing before and after such an environment.
% These parameters apply for all theorem sets and can be manipulated
% with the ordinary length macros.  They are rubber lengths
% (`\textsf{skips}'), and therefore can contain \texttt{plus} and
% \texttt{minus} parts.
%
% Since the definition of theorem sets should---most sensibly---be
% placed in the preamble, we only allow installation there.
% It is therefore possible to release the memory used here after
% |\begin{document}|, in order to make room for other
% applications.
%
%
% \subsection{Existing theorem styles}
%
% The following theorem styles exist to date:
% \begin{deflist}{marginbreak}
%    \item[plain]
%       This theorem style emulates the original \LaTeX{} definition,
%       except that additionally the parameters
%       |\theorem...skipamount| are used.
%    \item[break]
%       In this style, the theorem header is followed by a line
%       break.
%    \item[marginbreak]
%       The theorem number is set in the margin, and there is a line
%       break as in \texttt{break}.
%    \item[changebreak]
%       Like \texttt{break}, but with header number and text
%       interchanged.
%    \item[change]
%       Header number and text are interchanged, without a line break.
%    \item[margin]
%       The number is set in the left margin, without a line break.
% \end{deflist}
% All styles (except \texttt{plain}) select |\slshape| as the default
% |\theorembodyfont|.
%
%
% \subsection{Examples}
%
% Given the above theorem sets {\envfont Cor}, {\envfont Exa}
% and {\envfont Rem},
% suppose that the preamble also contains the declarations:
% \begin{verbatim}
%    \theoremstyle{marginbreak}   \newtheorem{Lem}[Cor]{Lemma}
%    \theoremstyle{change}
%    \theorembodyfont{\itshape}        \newtheorem{Def}[Cor]{Definition}
%
%    \theoremheaderfont{\scshape}
%\end{verbatim}
% Then the following are some typical examples of the typeset output
% resulting from their use.
%
% \begin{Cor}
%    This is a sentence typeset in the theorem environment {\envfont
%    Cor}.
% \end{Cor}
%
% \begin{Exa}
%    This is a sentence typeset in the theorem environment {\envfont
%    Exa}.
% \end{Exa}
%
% \begin{Rem}
%    This is a sentence typeset in the theorem environment {\envfont
%    Rem}.
% \end{Rem}
%
% \begin{Lem}[Ben User]
%    This is a sentence typeset in the theorem environment {\envfont
%    Lem}.
% \end{Lem}
%
% \begin{Def}[Very impressive Definition]
%    This is a sentence typeset in the theorem environment {\envfont
%    Def}.
% \end{Def}
%
% The last two examples show the effect of the optional argument to
% a theorem environment (it is the text typeset in parentheses).
%
%
% \section{Special Considerations}
% \label{sec:newfonts}
%
% Theoremheader and body are implemented as a unit. This means that
% the |\theoremheaderfont|  will inherit characteristics of the
% |\theorembodyfont| in \LaTeXe. Thus, if for example
% |\theorembodyfont| is |\itshape| and |\theoremheaderfont|
% is |\bfseries| the font selected for the header will have the
% characteristics `bold extended italic'. If this is not desired one
% should set the |\theoremheaderfont| to something like
% \begin{verbatim}
%      \theoremheaderfont{\normalfont\bfseries}
%\end{verbatim}
% i.e.\ supplying all necessary font information explicitly.
%
%
%
% \section{Acknowledgements}
%
% The publication of this set of macros was only possible with the
% help of Christina Busse (translating the manuscript into English),
% Joachim Pense (playing the r\^ole of typist), Chris Rowley
% (looking everything over) and many others providing useful
% suggestions.
%
% \MaybeStop{%
%    \begin{thebibliography}{9}
%       \bibitem{bk:GMS94} \textsc{M.~Goossens}, \textsc{F.~Mittelbach}
%         and \textsc{A.~Samarin}.
%         \newblock The \LaTeX{} Companion.
%         \newblock
%         Addison-Wesley, Reading, Massachusetts, 1994.
%       \bibitem{latex:source} \textsc{Lamport, Leslie.} \newblock
%                               \texttt{latex.tex}, version 2.09, date
%                               Feb. 1990.
%    \end{thebibliography}
%    \PrintIndex
%    \PrintChanges}
%
%
% \section{The documentation driver file}
%
% The next bit of code contains the documentation driver file for
% \TeX{}, i.e., the file that will produce the documentation you are
% currently reading. It will be extracted from this file by the
% \texttt{docstrip} program. Since it is the first code in the file
% one can alternatively process this file directly with \LaTeXe{} to
% obtain the documentation.
%
%    \begin{macrocode}
%<*driver>
\documentclass{ltxdoc}

\usepackage{theorem}

%<+driver>% The next few lines define theorem sets which are used
%<+driver>% in the example section of the documentation.

\theoremstyle{break}       \newtheorem{Cor}{Corollary}
\theoremstyle{plain}       \newtheorem{Exa}{Example}[section]
{\theorembodyfont{\upshape}\newtheorem{Rem}{Remark}}
\theoremstyle{marginbreak} \newtheorem{Lem}[Cor]{Lemma}
\theoremstyle{change}
\theorembodyfont{\itshape} \newtheorem{Def}[Cor]{Definition}

\theoremheaderfont{\scshape}

\RecordChanges

\begin{document}
   \DocInput{theorem.dtx}
\end{document}
%</driver>
%    \end{macrocode}
%
% \section{Definition of the Macros}
%
% If the file has been loaded before, we abort immediately. If not the
% package announces itself (this is actually done at the very top of
% the file---the way it is done isn't good style so don't copy it).
% \changes{v2.0e}{89/07/19}{Spaces removed from `typeout'.}
% \changes{v2.2b}{95/11/19}{Announce moved to top of file}
%    \begin{macrocode}
%<*package>
%\@ifundefined{theorem@style}{}{\endinput}
%\def\FMithmInfo{1995/11/19 v2.2b Theorem extension package (FMi)}
%\ProvidesPackage{theorem}[\FMithmInfo]
%    \end{macrocode}
%
%
% \subsection{Definition of theorem styles and fonts}
%
% All the definitions in this file are done globally to allow
% inputting this file inside a group.
%
% \begin{macro}{\theoremstyle}
%    Before a theorem style can be installed, the chosen style must be
%    known. For that reason, we must test to see that
%    |\th@|\meta{style} is known or, more precisely, that it is
%    different from |\relax|. If the style is not known then
%    |\th@plain| is used.
%    \begin{macrocode}
\gdef\theoremstyle#1{%
   \@ifundefined{th@#1}{\@warning
          {Unknown theoremstyle `#1'. Using `plain'}%
          \theorem@style{plain}}%
%    \end{macrocode}
%    We save the theorem style to be used in the token register
%    |\theorem@style|.
%    \begin{macrocode}
      {\theorem@style{#1}}%
%    \end{macrocode}
%    Now we ``evaluate'' the theorem style: this means, we call the
%    macro |\th@|\meta{style} which will activate the relevant
%    definitions which are contained in a separate file.  This is done
%    in a group to suppress changes to the current font.  This could
%    otherwise pose problems together with the new font selection
%    scheme\footnote{When I printed the original article using the new
%    font selection scheme I ended with a document with slanted
%    typefaces (text headings and all) simply because one of the
%    theorem styles used \texttt{\bslash sl} at toplevel.} if the
%    |\th@|\meta{style} is evaluated a second time.
% \changes{v2.1a}{90/02/15}{Group added around `th@..}
% \changes{v2.1a}{90/02/15}{Reseting of `..@begintheorem removed.}
%    \begin{macrocode}
      \begingroup
        \csname th@\the\theorem@style \endcsname
      \endgroup}
%    \end{macrocode}
% \end{macro}
%
%
% \begin{macro}{\@begintheorem}
% \changes{v2.1a}{90/02/15}{Relaxed at toplevel.}
% \begin{macro}{\@opargbegintheorem}
% \changes{v2.1a}{90/02/15}{Relaxed at toplevel.}
%    We reset |\@begintheorem| and |\@opargbegintheorem| to
%    |\relax| since these commands are no longer necessary at
%    toplevel. This will save a few tokens.
%    \begin{macrocode}
\global\let\@begintheorem\relax
\global\let\@opargbegintheorem\relax
%    \end{macrocode}
% \end{macro}
% \end{macro}
%
% \begin{macro}{\theorem@style}
%    Obviously the \textsf{token} register used above has to be
%    allocated. To assure the utmost compatibility with the
%    original \LaTeX{} definition, we set the default theorem style to
%    \texttt{plain}, which implements the usual \LaTeX{} convention.
% \changes{v2.0f}{89/07/20}{Unnecessary `global' removed.}
%    \begin{macrocode}
\newtoks\theorem@style
\global\theorem@style{plain}
%    \end{macrocode}
% \end{macro}
%
% \begin{macro}{\theorembodyfont}
% \begin{macro}{\theorem@bodyfont}
%    If the |\theorembodyfont| is set by the user then it should not
%    interact with the default font set in the theorem style. When the
%    new font selection is in force this may happen if, for example,
%    the default is |\itshape| and the new |\theorembodyfont| is
%    |\sffamily|. So we add a |\reset@font| command in front of the
%    user definition.
% \changes{v2.0f}{89/07/20}{Unnecessary `global' removed.}
% \changes{v2.0g}{89/09/19}{`theoremfont to `theorembodyfont.}
% \changes{v2.1c}{92/06/22}{`reset@font added and logic changed.}
% \changes{v2.1d}{92/10/28}{Reset to empty if arg is empty}
%    \begin{macrocode}
\gdef\theorembodyfont#1{%
%    \end{macrocode}
%    We check if the argument supplied is empty and if so put nothing
%    into the |\theorem@bodyfont| token register to allow for
%    |\theorembodyfont{}| as a mean of using the default of the
%    current |\theoremstyle|.
%    \begin{macrocode}
   \def\@tempa{#1}%
   \ifx\@tempa\@empty
    \theorem@bodyfont{}%
   \else
    \theorem@bodyfont{\reset@font#1}%
   \fi
   }
\newtoks\theorem@bodyfont
\global\theorem@bodyfont{}
%    \end{macrocode}
% \end{macro}
% \end{macro}
%
%
% \begin{macro}{\theoremheaderfont}
% \changes{v2.0g}{89/09/19}{Macro added.}
%    The font for the theorem headers is handled differently because
%    this definition applies to all theorem styles.
%    \begin{macrocode}
\gdef\theoremheaderfont#1{\gdef\theorem@headerfont{#1}%
%    \end{macrocode}
%    After using the macro once it is redefined to produce an error
%    message.
%    \begin{macrocode}
       \gdef\theoremheaderfont##1{%
        \typeout{\string\theoremheaderfont\space should be used
                 only once.}}}
%    \end{macrocode}
% \end{macro}
%
% \begin{macro}{\theorem@headerfont}
% \changes{v2.0g}{89/09/19}{Macro added.}
%    To set the |\theorem@headerfont| default we first test if
%    the new fontselection scheme is in force.
% \changes{v2.1a}{90/02/15}{Support for new font selection.}
%    \begin{macrocode}
\ifx\upshape\undefined
%    \end{macrocode}
%    If not we define it to expand into |\bfseries|. We don't use
%    |\let| just in case a following style option redefines this
%    macro.
%    \begin{macrocode}
\gdef\theorem@headerfont{\bfseries}
%    \end{macrocode}
%    Otherwise we reset the current shape before calling |\bfseries|.
%    \begin{macrocode}
\else \gdef\theorem@headerfont{\normalfont\bfseries}\fi
%    \end{macrocode}
% \end{macro}
%
% \begin{macro}{\th@plain}
% \begin{macro}{\th@break}
% \begin{macro}{\th@marginbreak}
% \begin{macro}{\th@changebreak}
% \begin{macro}{\th@change}
% \begin{macro}{\th@margin}
%    The different styles are defined in macros such as
%    |\th@plain|.  Since memory space is precious in
%    ``non-Big-versions'', we have to avoid offering too many unused
%    definitions.  Therefore we define these styles in separate files
%    that can be loaded on demand.  Thus the commands themselves only
%    load these files. We use |\@input@|, a \LaTeXe{} internal command
%    that ensures that the file will be listed with |\listfiles|.
% \changes{v2.2a}{94/02/02}{Use `@input@ to load theorem layout files}
%    \begin{macrocode}
\gdef\th@plain{\@input@{thp.sty}}
\gdef\th@break{\@input@{thb.sty}}
\gdef\th@marginbreak{\@input@{thmb.sty}}
\gdef\th@changebreak{\@input@{thcb.sty}}
\gdef\th@change{\@input@{thc.sty}}
\gdef\th@margin{\@input@{thm.sty}}
%    \end{macrocode}
%    This list will be expanded when new styles become
%    available.  For testing,  just append new theorem substyles
%    as document options.
% \end{macro}
% \end{macro}
% \end{macro}
% \end{macro}
% \end{macro}
% \end{macro}
%
%
%
% \subsection{Definition of a new theorem set}
%
% As already pointed out, a new theorem environment can be defined in
% three different ways:
% \begin{verbatim}
%    \newtheorem{Lem}{Lemma}
%    \newtheorem{Lem}{Lemma}[section]
%    \newtheorem{Lem}[Theorem]{Lemma}
%\end{verbatim}
% The function of the macro |\newtheorem| is to recognize these cases
% and then to branch into one of the three macros |\@ynthm|, |\@xnthm|
% or |\@othm|.  This mechanism is adopted unchanged from
% \cite{latex:source}; the essential point here is that, for example,
% in the second case, the arguments \texttt{Lem}, \texttt{Lemma} and
% \texttt{section} are passed over to the macro |\@xnthm|.
%
% We inspect this case first because the others present fewer problems,
% and thus are easily derived from this one.
%
% \begin{macro}{\@xnthm}
%    For our example arguments, the macro |\@xnthm| must fulfill
%    the following:
%    \begin{itemize}
%       \item
%          Define a new \LaTeX-counter `Lem'
%       \item
%          reset this counter within a |\section|
%       \item
%          define the macro |\theLem|
%       \item
%          define the environment macros |\Lem| and
%          |\endLem| using the current
%          |\theoremstyle| and |\theorem@bodyfont|.
%    \end{itemize}
%    Obviously, all this should happen only if the first argument
%    of |\@xnthm| (i.e.\ \texttt{Lem} in our example) is chosen so
%    as not to conflict with any previously defined commands or
%    environments.  This test is performed by the \LaTeX{} macro
%    |\@ifdefinable|.
%    \begin{macrocode}
\gdef\@xnthm#1#2[#3]{\expandafter\@ifdefinable\csname #1\endcsname
%    \end{macrocode}
%    Therefore, the first argument of |\@ifdefinable| is the
%    expansion (in the example, |\Lem|) of
%    |\csname||#1\endcsname|.  The second argument is
%    executed only if the test has been completed successfully.
%    \begin{macrocode}
   {%
%    \end{macrocode}
%    Now we define the new counter. The names of the \LaTeX{}
%    macros employed should speak for themselves:
% \changes{v2.2b}{1995/10/07}{Use \cs{@newctr}}
% \changes{v2.2c}{1995/11/23}{Use \cs{@newctr} properly.}
%    \begin{macrocode}
    \@definecounter{#1}\@newctr{#1}[#3]%
%    \end{macrocode}
%    Using |\@newctr| will give a proper error message if the counter
%    in |#3| is not defined.
%    In defining `|\theLem|' we must generate the desired macro
%    name by use of |\expandafter| and |\csname|.
%    \begin{macrocode}
    \expandafter\xdef\csname the#1\endcsname
%    \end{macrocode}
%    An |\xdef| is used in order to make the definition global,
%    and to ensure that it contains the replacement texts of
%    |\@thmcountersep| and |\@thmcounter|.\footnote{These
%    two macros can be defined by the document style.  Their default
%    values produce a `\texttt{.}' as separation and an arabic
%    representation of the number.} However, not everything should be
%    expanded. For example, it saves space to use |\thesection|
%    instead of its---at times---lengthy expansion.
%    \begin{macrocode}
      {\expandafter \noexpand \csname the#3\endcsname
       \@thmcountersep \@thmcounter{#1}}%
%    \end{macrocode}
%    Thus with the defaults of \LaTeX{}, |\theLem| would be
%    replaced by the command sequence |\thesection.\arabic{Lem}|.
%
%    We will now look at the definition of the macro which is executed
%    at the beginning of the actual environment (in our example this
%    macro is |\Lem|). It should be noted that we use an
%    ``|\expandafter| trick'' to expand only certain parts of the
%    replacement text at the time of the definition.
%    \begin{macrocode}
    \def\@tempa{\global\@namedef{#1}}%
    \expandafter \@tempa \expandafter{%
%    \end{macrocode}
%    First, the macro that contains the current definitions of
%    |\@begintheorem| and |\@opargtheorem| should be called
%    up.  The name of this macro---as is already known---has the form
%    |\th@|\meta{theorem\ style}; therefore, it must be called by
%    \begin{macrocode}
      \csname th@\the \theorem@style
%    \end{macrocode}
%    In addition the default theorem font should be changeable, i.e.\
%    we have to insert the contents of |\theorem@bodyfont|. For
%    that reason, we expand even further, beyond |\endcsname|,
%    and thus insert the contents of the \textsf{token} register
%    |\theorem@bodyfont| in the replacement text.
%    \begin{macrocode}
            \expandafter \endcsname \the \theorem@bodyfont
%    \end{macrocode}
%    Now it is time to call the macro |\@thm| which takes over
%    the further processing.  It has two arguments: the current
%    counter name (in our example, \texttt{Lem}), and the text of the
%    label (in our example, \texttt{Lemma}).
%    \begin{macrocode}
     \@thm{#1}{#2}}%
%    \end{macrocode}
%    With this, the `sub-definition' is complete.  The macro
%    |\@endtheorem| ends a theorem environment and is, so far, nothing
%    but an |\endtrivlist|.  (Hence it is defined globally, and not
%    within the theorem styles.\footnote{This has to be changed as
%    soon as theorem styles that change \texttt{\bslash @endtheorem}
%    exist.  In such a case, all existing styles must be changed as
%    well since they will have to reset the macro.}) Therefore, we can
%    set it equivalent to the macro that ends the theorem set (in our
%    example, |\endLem|).  However, if some day theorem styles exist
%    that do change |\@endtheorem|, we would have to use the
%    commented-out line instead.
%    \begin{macrocode}
    \global \expandafter \let \csname end#1\endcsname \@endtheorem
%   \global\@namedef{end#1}{\@endtheorem}%
%    \end{macrocode}
%    With these commands all the required definitions are employed,
%    unless the test |\@ifdefinable| has failed.  Therefore, we
%    end the second argument of this macro and with it the definition
%    of |\@xnthm|.
%    \begin{macrocode}
   }}
%    \end{macrocode}
% \end{macro}
%
% \begin{macro}{\@ynthm}
%    The definition of |\@ynthm| is completely analogous. In
%    this case the new counter that is defined is not reset within
%    another counter; thus the definition of |\the...|
%    is simplified:
%    \begin{macrocode}
\gdef\@ynthm#1#2{\expandafter\@ifdefinable\csname #1\endcsname
   {\@definecounter{#1}%
    \expandafter\xdef\csname the#1\endcsname{\@thmcounter{#1}}%
%    \end{macrocode}
%    The rest of the definition corresponds literally to that of
%    |\@xnthm|:
%    \begin{macrocode}
    \def\@tempa{\global\@namedef{#1}}\expandafter \@tempa
     \expandafter{\csname th@\the \theorem@style \expandafter
     \endcsname \the\theorem@bodyfont \@thm{#1}{#2}}%
    \global \expandafter \let \csname end#1\endcsname \@endtheorem}}
%    \end{macrocode}
% \end{macro}
%
% \begin{macro}{\@othm}
%    The definition of |\@othm| does not contain anything new.
%    \begin{macrocode}
\gdef\@othm#1[#2]#3{%
%    \end{macrocode}
% \changes{v2.2b}{95/10/07}{Build in check for counter defined}
%    We do not define a new counter but instead use one that has
%    already been defined.  Thus the only definition we need is that
%    of this pseudo-counter (i.e.\ |\the|\meta{env.\ name}).
%    First we check if |#2| corresponds to a known counter name.
%    \begin{macrocode}
  \expandafter\ifx\csname c@#2\endcsname\relax
   \@nocounterr{#2}%
  \else
   \expandafter\@ifdefinable\csname #1\endcsname
   {\expandafter \xdef \csname the#1\endcsname
     {\expandafter \noexpand \csname the#2\endcsname}%
%    \end{macrocode}
%    All other parts of the definition can be adopted from
%    |\@xnthm|. We have to remember, though, that in this case
%    the name of the current counter and the theorem label have moved
%    to the second and third arguments.
%    \begin{macrocode}
    \def\@tempa{\global\@namedef{#1}}\expandafter \@tempa
     \expandafter{\csname th@\the \theorem@style \expandafter
     \endcsname \the\theorem@bodyfont \@thm{#2}{#3}}%
    \global \expandafter \let \csname end#1\endcsname \@endtheorem}%
  \fi}
%    \end{macrocode}
% \end{macro}
%
% \subsection{Macros that are employed in a {\envfont theorem}
%    environment}
%
% \begin{macro}{\@thm}
%    The macro |\@thm| has to increase the current counter.  Then,
%    depending on whether the environment has (or does not have) an
%    optional argument, it has to branch into either |\@begintheorem|
%    or |\@opargtheorem|.
%    \begin{macrocode}
\gdef\@thm#1#2{\refstepcounter{#1}%
%    \end{macrocode}
%    Now we start a {\envfont trivlist} environment, and give
%    |\@topsep| and |\@topsepadd| the values of the \textsf{skip}
%    registers |\theorempreskipamount| and |\theorempostskipamount|.
%    The value in |\@topsep| is the vertical space that is inserted by
%    the first (and only) |\item| in our |\trivlist| whilst
%    |\@topsepadd| is inserted by |\@endparenv| at the end of that
%    {\envfont trivlist} environment.  By using these registers, we
%    obtain the desired space around a {\envfont theorem} environment.
%    \begin{macrocode}
   \trivlist
   \@topsep \theorempreskipamount               % used by first \item
   \@topsepadd \theorempostskipamount           % used by \@endparenv
%    \end{macrocode}
%    Now we have to test whether an optional argument has been given.
% \changes{v2.0d}{89/07/18}{Forgotten percent added.}
%    \begin{macrocode}
   \@ifnextchar [%
%    \end{macrocode}
%    If there is an optional argument, we will call |\@ythm|, and
%    move the arguments read back into the input stream.
%    \begin{macrocode}
   {\@ythm{#1}{#2}}%
%    \end{macrocode}
%    If not, we call |\@begintheorem|. Its first argument is the
%    name of the theorem set (hence the second argument of
%    |\@thm|). Its second argument is the macro that produces the
%    current number.
%    \begin{macrocode}
   {\@begintheorem{#2}{\csname the#1\endcsname}\ignorespaces}}
%    \end{macrocode}
% \end{macro}
%
% \begin{macro}{\@xthm}
% \begin{macro}{\@ythm}
%    Both these macros were originally called by |\@thm|.  We do
%    not need |\@xthm| anymore, hence we reset it to
%    |\relax|. The definition of |\@ythm| has not changed at
%    all from its definition in \LaTeX. In order to make the macros
%    easier to understand, we will
%    nevertheless present it (commented out).
%    \begin{macrocode}
\global\let\@xthm\relax
% \def\@ythm#1#2[#3]{\@opargbegintheorem{#2}{\csname
%           the#1\endcsname}{#3}\ignorespaces}
%</package>
%    \end{macrocode}
%    The primitive |\ignorespaces| in |\@ythm| and
%    |\@thm| is needed to remove the spaces between the
%    |\begin{...}| and the actual text.
% \end{macro}
% \end{macro}
%
%
% \subsection{Definition of the theorem substyles}
%
% As already pointed out, the theorem substyles, defined below, are
% only loaded when necessary.  Note that all these substyles, except
% \texttt{plain}, have |\slshape| as the default body font.
%
% \subsubsection{The \texttt{plain} style}
%
% \label{dok:th@plain} As the following macros use |@|, we have
% to locally set the |\catcode| of this symbol to ``letter''.
% This happens within a group, so that we do not have to worry about
% which |\catcode| that symbol had before.
%    \begin{macrocode}
%<*thp>
\begingroup \makeatletter
%    \end{macrocode}
% Since we are now within a group, we must make all definitions
% globally.  First we make sure that \texttt{theorem.sty} is loaded.
% This will allow us to use this file as a document style option
% without having to call \texttt{theorem} itself as an option.  At the
% same time, we assure that at least version 2 is loaded, since
% |\theorem@style| was not defined in earlier versions.
%    \begin{macrocode}
\@ifundefined{theorem@style}{\input{theorem.sty}}{}
\ProvidesFile{thp.sty}
             [\FMithmInfo]
%    \end{macrocode}
%
%
% \begin{macro}{\th@plain}
% \begin{macro}{\@begintheorem}
% \begin{macro}{\@opargbegintheorem}
%    |\theoremstyle{plain}| corresponds to the original
%    definition, except that the distances to the surrounding text are
%    determined by the parameters |\theorempreskipamount| and
%    |\theorempostskipamount|.
%    First we set the default body font.
%    \begin{macrocode}
\gdef\th@plain{\normalfont\itshape
%    \end{macrocode}
%    Then we define |\@begintheorem| and
%    |\@opargbegintheorem|. These two macros define how the
%    header of a theorem is typeset.  |\@opargbegintheorem| will
%    be called if a \textsf{theorem} environment with an optional
%    argument is encountered; otherwise, the header is constructed by
%    calling |\@begintheorem|.  If one of these macros is
%    executed, we are within a \textsf{trivlist} environment started by
%    |\@thm|.  So the theorem header is produced with an
%    |\item| command.
%
%    Instead of specifying the header font directly, all standard
%    theorem styles use the |\theorem@headerfont| macro to allow
%    customization.  The extra space (|\labelsep|) is necessary
%    because of problems in the \textsf{trivlist} environment.
%    \begin{macrocode}
  \def\@begintheorem##1##2{%
        \item[\hskip\labelsep \theorem@headerfont ##1\ ##2]}%
%    \end{macrocode}
%    The definition of |\@opargbegintheorem| is completely
%    analogous. The only difference is the fact that there exists a
%    third argument (which is the optional parameter of the
%    environment and contains additional information about the
%    theorem).  Customarily we enclose it in parentheses.
%    \begin{macrocode}
\def\@opargbegintheorem##1##2##3{%
   \item[\hskip\labelsep \theorem@headerfont ##1\ ##2\ (##3)]}}
%    \end{macrocode}
% \end{macro}
% \end{macro}
% \end{macro}
%
% \noindent
% We conclude with an |\endgroup| to restore the
% |\catcode| of |@|.
%    \begin{macrocode}
\endgroup
%</thp>
%    \end{macrocode}
%
%
% \subsubsection{The \texttt{break} style}
%
% This style option is stored in the file |thb.sty|.
% For the next two lines see the documentation for |\th@plain| on
% page~\pageref{dok:th@plain}.
%    \begin{macrocode}
%<*thb>
\begingroup \makeatletter
\@ifundefined{theorem@style}{\input{theorem.sty}}{}
\ProvidesFile{thb.sty}
             [\FMithmInfo]
%    \end{macrocode}
%
% \begin{macro}{\th@break}
%    |\theoremstyle{break}| produces a line break after the name
%    of the theorem.  The font is |\slshape|.  Hence, we define
%    |\th@break| as follows:
%    \begin{macrocode}
\gdef\th@break{\normalfont\slshape
  \def\@begintheorem##1##2{\item[%
%    \end{macrocode}
%    We run into the following problem: it is not possible to create
%    the header with |\item[|\meta{title}|]| and then start
%    a new line by, for example, |\mbox{}\\|. Such a definition
%    will fail whenever a list environment follows immediately.  With
%    the above construction, the |\mbox{}| causes the switch
%    |@inlabel| (cf.\ definition of |\list| and
%    |\trivlist| in \cite{latex:source}) to be set to \texttt{false}
%    and so the following list will insert additional vertical
%    space (|\topskip|). This is quite annoying. Therefore, we
%    create the line break within the |\item|. In order to ensure
%    that the text will begin at the proper position in the following
%    line, we simply pretend that the label does not take any
%    room.\footnote{This will lead to problems whenever very high
%    symbols occurring in the line tower into the
%    heading. So, something else has to be done here sometime.}
%    \begin{macrocode}
       \rlap{\vbox{\hbox{\hskip \labelsep\theorem@headerfont ##1\ ##2}%
                   \hbox{\strut}}}]}%
%    \end{macrocode}
%    Again, the definition of |\@opargbegintheorem| is
%    completely analogous.
%    \begin{macrocode}
\def\@opargbegintheorem##1##2##3{%
  \item[\rlap{\vbox{\hbox{\hskip \labelsep \theorem@headerfont
                     ##1\ ##2\ (##3)}%
                    \hbox{\strut}}}]}}
\endgroup
%</thb>
%    \end{macrocode}
% \end{macro}
%
%
% \subsubsection{The \texttt{changebreak} style}
%
%    \begin{macrocode}
% This style option is stored in the file |thcb.sty|.
%    \begin{macrocode}
%<*thcb>
\begingroup \makeatletter
\@ifundefined{theorem@style}{\input{theorem.sty}}{}
\ProvidesFile{thcb.sty}
             [\FMithmInfo]
%    \end{macrocode}
%
% \begin{macro}{\th@changebreak}
%    The \texttt{change-break} theorem style is like \texttt{break}
%    but with interchange of theorem name and theorem number.  Thus we
%    define |\th@changebreak| as follows:
%    \begin{macrocode}
\gdef\th@changebreak{\normalfont\slshape
  \def\@begintheorem##1##2{\item
       [\rlap{\vbox{\hbox{\hskip\labelsep \theorem@headerfont ##2\ ##1}%
                    \hbox{\strut}}}]}%
\def\@opargbegintheorem##1##2##3{%
  \item[\rlap{\vbox{\hbox{\hskip\labelsep \theorem@headerfont
                     ##2\ ##1\ (##3)}%
                    \hbox{\strut}}}]}}
\endgroup
%</thcb>
%    \end{macrocode}
% \end{macro}
%
%
% \subsubsection{The \texttt{change} style}
%
% This style option is stored in the file |thc.sty|.
%    \begin{macrocode}
%<*thc>
\begingroup \makeatletter
\@ifundefined{theorem@style}{\input{theorem.sty}}{}
\ProvidesFile{thc.sty}
             [\FMithmInfo]
%    \end{macrocode}
%
% \begin{macro}{\th@change}
%    The \texttt{change} theorem style corresponds to the
%    \texttt{change break} style without a linebreak after the header.
%    To say it in another way, it's the same as the \texttt{plain}
%    style but with number and name interchanged and |\slshape| as the
%    default font.
%    \begin{macrocode}
\gdef\th@change{\normalfont\slshape
  \def\@begintheorem##1##2{\item
       [\hskip\labelsep \theorem@headerfont ##2\ ##1]}%
\def\@opargbegintheorem##1##2##3{%
  \item[\hskip\labelsep \theorem@headerfont ##2\ ##1\ (##3)]}}
\endgroup
%</thc>
%    \end{macrocode}
% \end{macro}
%
%
% \subsubsection{The \texttt{marginbreak} style}
%
% This style option is the one used most often at Mainz.  It is saved
% in the file |thmb.sty|.
%    \begin{macrocode}
%<*thmb>
\begingroup \makeatletter
\@ifundefined{theorem@style}{\input{theorem.sty}}{}
\ProvidesFile{thmb.sty}
             [\FMithmInfo]
%    \end{macrocode}
%
% \begin{macro}{\th@marginbreak}
%    The \texttt{margin break} style is nearly the same as the
%    \texttt{change break} style. The only difference is the placement
%    of the theorem number.  We use |\llap| to place it in the left
%    margin.
%
%    In this style |\labelsep| denotes the separation between the
%    number and the text.
%    \begin{macrocode}
\gdef\th@marginbreak{\normalfont\slshape
  \def\@begintheorem##1##2{\item
       [\rlap{\vbox{\theorem@headerfont
                    \hbox{\llap{##2}\hskip\labelsep ##1}%
                    \hbox{\strut}}}]}%
\def\@opargbegintheorem##1##2##3{%
  \item[\rlap{\vbox{\theorem@headerfont
                    \hbox{\llap{##2}\hskip\labelsep ##1\ (##3)}%
                    \hbox{\strut}}}]}}
\endgroup
%</thmb>
%    \end{macrocode}
% \changes{v2.1c}{90/02/17}{`endgroup outside macrocode env.}
% \end{macro}
%
%
%
% \subsubsection{The \texttt{margin} style}
%
% This style option is stored in the file |thm.sty|.
%    \begin{macrocode}
%<*thm>
\begingroup \makeatletter
\@ifundefined{theorem@style}{\input{theorem.sty}}{}
\ProvidesFile{thm.sty}
             [\FMithmInfo]
%    \end{macrocode}
%
% \begin{macro}{\th@margin}
%    Again this is only a variant of the theorem styles described
%    above without any new ideas.
%    \begin{macrocode}
\gdef\th@margin{\normalfont\slshape
  \def\@begintheorem##1##2{\item
       [\theorem@headerfont \llap{##2}\hskip\labelsep ##1]}%
\def\@opargbegintheorem##1##2##3{%
  \item[\theorem@headerfont \llap{##2}\hskip\labelsep ##1\ (##3)]}}
\endgroup
%</thm>
%    \end{macrocode}
% \end{macro}
%
%
%
%
% \subsection{Final Definitions}
%
% \begin{macro}{\theorempreskipamount}
% \begin{macro}{\theorempostskipamount}
%    The \textsf{skip} parameters that regulate the vertical empty space
%    before and after the {\envfont theorem} environment have
%    to be allocated as well.
% \changes{v2.0f}{89/07/20}{Unnecessary `global' removed.}
%    \begin{macrocode}
%<*package>
\newskip\theorempreskipamount
\newskip\theorempostskipamount
%    \end{macrocode}
%    Since we have used the same values for all theorem sets, we now can
%    assign them.
%    \begin{macrocode}
\global\setlength\theorempreskipamount{12pt plus 5pt minus 3pt}
\global\setlength\theorempostskipamount{8pt plus 3pt minus 1.5pt}
%    \end{macrocode}
% \end{macro}
% \end{macro}
%
%
% \begin{macro}{\@endtheorem}
%    The same holds for the macro |\@endtheorem|, which ends a
%    {\envfont theorem} environment. Since it is the same for all
%    theorem sets, it is removed from the macros |\th@|\meta{style}.
%    It simply ends the {\envfont trivlist} environment, which was
%    begun in |\@thm|.
%    \begin{macrocode}
\global\let\@endtheorem=\endtrivlist
%    \end{macrocode}
% \end{macro}
%
%
%
% \begin{macro}{\@preamblecmds}
%    All macros defined above are to be used only in the preamble.
%    Therefore, we insert them in |\@preamblecmds| which will disable
%    them at begin document. This is done by the internal \LaTeXe{}
%    command |\@onlypreamble|.
% \changes{v2.0g}{89/09/19}{`theoremheaderfont and `theorembodyfont
%    added.}
% \changes{v2.2a}{94/02/03}{Use \LaTeXe{} interface}
%    \begin{macrocode}
\@onlypreamble\@xnthm
\@onlypreamble\@ynthm
\@onlypreamble\@othm
\@onlypreamble\newtheorem
\@onlypreamble\theoremstyle
\@onlypreamble\theorembodyfont
\@onlypreamble\theoremheaderfont
%    \end{macrocode}
% \end{macro}
%
% Finally we declare the \texttt{plain} theorem style to be the default.
% \changes{v2.1b}{90/06/04}{Fixed missing `item by declaring default
%    style.}
%    \begin{macrocode}
\theoremstyle{plain}
%</package>
%    \end{macrocode}
%
% \Finale

\endinput
}{}
\ProvidesFile{thc.sty}
             [\FMithmInfo]
%    \end{macrocode}
%
% \begin{macro}{\th@change}
%    The \texttt{change} theorem style corresponds to the
%    \texttt{change break} style without a linebreak after the header.
%    To say it in another way, it's the same as the \texttt{plain}
%    style but with number and name interchanged and |\slshape| as the
%    default font.
%    \begin{macrocode}
\gdef\th@change{\normalfont\slshape
  \def\@begintheorem##1##2{\item
       [\hskip\labelsep \theorem@headerfont ##2\ ##1]}%
\def\@opargbegintheorem##1##2##3{%
  \item[\hskip\labelsep \theorem@headerfont ##2\ ##1\ (##3)]}}
\endgroup
%</thc>
%    \end{macrocode}
% \end{macro}
%
%
% \subsubsection{The \texttt{marginbreak} style}
%
% This style option is the one used most often at Mainz.  It is saved
% in the file |thmb.sty|.
%    \begin{macrocode}
%<*thmb>
\begingroup \makeatletter
\@ifundefined{theorem@style}{% \iffalse meta-comment
%
% Copyright (C) 1993-2024
%
% The LaTeX Project and any individual authors listed elsewhere
% in this file.
%
% This file is part of the Standard LaTeX `Tools Bundle'.
% -------------------------------------------------------
%
% It may be distributed and/or modified under the
% conditions of the LaTeX Project Public License, either version 1.3c
% of this license or (at your option) any later version.
% The latest version of this license is in
%    https://www.latex-project.org/lppl.txt
% and version 1.3c or later is part of all distributions of LaTeX
% version 2005/12/01 or later.
%
% The list of all files belonging to the LaTeX `Tools Bundle' is
% given in the file `manifest.txt'.
%
% \fi
%
%%
%
%
\def\FMithmInfo{2023/07/05 v2.2c Theorem extension package (FMi)}
%
% \ProvidesFile{theorem.dtx}[\FMithmInfo]
%\iffalse   % this is a METACOMMENT !
%%
%% Package `theorem' to use with LaTeX2e
%% Copyright (C) 1989-1995 Frank Mittelbach, all rights reserved.
%<+package>\@ifundefined{theorem@style}{}{\endinput}
%<+package>\NeedsTeXFormat{LaTeX2e}[1995/06/01]
%<+package>\ProvidesPackage{theorem}
%<+package>         [\FMithmInfo]
%
% \fi
%
% \changes{v1.0}{87/05/07}{First draft.}
% \changes{v1.0a}{87/05/07}{`thistheoremstyle' added.}
% \changes{v1.1a}{87/07/29}{New concept: redefine the internal
%                          \protect\LaTeX{} commands `@xnthm' etc.}
% \changes{v1.1b}{87/07/31}{`theoremfont' used to specify font.}
% \changes{v1.2a}{88/01/08}{`@endtheorem' now globally defined.}
% \changes{v1.2b}{88/01/11}{`newtheorem' only allowed in the preamble.}
% \changes{v1.3a}{88/02/14}{`theoremfont' now a toks register.}
%
% \changes{v2.0a}{88/11/16}{New implementation and new documentation.}
% \changes{v2.0a}{88/11/16}{No layout changes but internal modifications
%                           old substyles won't work.}
% \changes{v2.0b}{89/01/30}{`global' used for all definitions.}
% \changes{v2.0c}{89/06/15}{Documentation translated to English.
%                            Thanks to LLL and Nina}
% \changes{v2.1a}{90/02/15}{Support for new font selection.}
% \changes{v2.2a}{94/02/03}{Cleanup for LaTeX2e}
%
% \DoNotIndex{\@endparenv,\@ifdefinable,\@ifnextchar,\@ifundefined}
% \DoNotIndex{\@namedef,\@nameuse,\@spaces,\@tempa,\@warning}
% \DoNotIndex{\ }
% \DoNotIndex{\begingroup,\bfseries}
% \DoNotIndex{\csname}
% \DoNotIndex{\def,\do,\docdate}
% \DoNotIndex{\endcsname,\endgroup,\endinput,\endtrivlist,\expandafter}
% \DoNotIndex{\filedate,\fileversion}
% \DoNotIndex{\gdef,\global}
% \DoNotIndex{\hbox,\hskip}
% \DoNotIndex{\ignorespaces,\input,\itshape,\item}
% \DoNotIndex{\labelsep,\let,\llap}
% \DoNotIndex{\makeatletter}
% \DoNotIndex{\noexpand}
% \DoNotIndex{\refstepcounter,\relax,\rlap}
% \DoNotIndex{\setlength,\slshape,\space,\string,\strut}
% \DoNotIndex{\the,\trivlist,\typeout}
% \DoNotIndex{\vbox}
% \DoNotIndex{\xdef}
%
% \def\deflabel#1{\ttfamily #1\hfill}
% \def\deflist#1{\begin{list}{}{\settowidth\labelwidth{\ttfamily #1}%
%                               \setlength\leftmargin\labelwidth
%                               \addtolength\leftmargin\labelsep
%                               \let\makelabel\deflabel}}
% \def\enddeflist{\end{list}}
% \def\envfont{\normalfont\ttfamily}
%
% \MakeShortVerb{\|}
%
% \newcommand{\TUB}{{\sl TUGboat\/}}
%
% \GetFileInfo{theorem.dtx}
% \title{An Extension of the
%        \LaTeX{} theorem environment\thanks{This file
%        has version number \fileversion, last
%        revised \filedate.}}
% \author{Frank Mittelbach}
%
% \MaintainedByLaTeXTeam{tools}
% \maketitle
%
% \begin{abstract}
%    The macros described in this paper yield an extension of the
%    \LaTeX{} theorem mechanism. It is designed to satisfy the
%    different requirements of various journals. Thus, the
%    layout of the ``theorems'' can be manipulated by determining a
%    ``style''. This article describes not only the use, but
%    also the definition, of the necessary macros.
% \end{abstract}
%
% \section*{Preface to version 2.2}
%
% For \LaTeXe{} this package did not need any fundamental changes. I
% only modified the messages generated so that theorem layout styles
% will show up with the |\listfiles| command and cleaned the section
% on the New Font Selection Scheme since this is now included in
% \LaTeX.
%
% \section*{Preface to version 2.1}
%
% This version is identical to 2.0g described in \TUB\ 10\#3 except
% for some internal defaults which are now set depending on the used
% font selection scheme.
%
% This was done to avoid unpleasant surprises if the new font selection
% scheme is in force. For further details see
% section~\ref{sec:newfonts} and \cite{bk:GMS94}.
%
% \section{Introduction}
%
% For our purposes here, ``theorems'' are labelled enunciations,
% often set off from the main text by extra space and a font change.
% Theorems, corollaries, conjectures, definitions, and remarks
% are all instances of ``theorems''.  The ``header'' of these structures
% is composed of a label (such as \textsc{Theorem} or \textsc{Remark})
% and a number which serializes an item in the sequence of items
% with the same label.
%
% Shortly after the introduction of \LaTeX{} at the Fachbereich
% Mathematik in Mainz, the desire to manipulate the layout of
% ``theorems'' arose.  In Mainz, the following two conventions came
% into general use:
% \begin{enumerate}
%   \item The number of the theorem is shown in the margin.
%   \item There is a line break at
%         the end of the theorem header.
% \end{enumerate}
%
% Additionally, some journals require different formats which depend on
% the ``sort of theorem'': e.g.\ often remarks and definitions are set
% in |\upshape|, while |\itshape| is employed for main theorems.
%
% Confronted with these requirements, a theorem environment was
% developed in Mainz which allows separate determination of the
% layout of the ``theorems sets'', comparable to |\pagestyle|.
%
%
% \section{The user interface}
% \subsection{Defining new theorem sets}
%
% \DescribeMacro\newtheorem
% As in the original \LaTeX{} version, the command |\newtheorem|
% defines a new ``theorem set'' or ``theorem-like structure''.
% Two required arguments name the new environment and give the text
% to be typeset with each instance of the new ``set'', while
% an optional argument determines how
% the ``set'' is enumerated:
% \begin{description}
%    \item[\ttfamily \bslash newtheorem\{foo\}\{bar\}]
%       The theorem set {\envfont foo} (whose name is \texttt{bar})
%       uses its own counter.
%    \item[\ttfamily \bslash newtheorem\{foo2\}{[foo]}\{bar2\}]
%       The theorem set {\envfont foo2} (printed name \texttt{bar2})
%       uses the same counter as the theorem set \texttt{foo}.
%    \item[\ttfamily \bslash newtheorem\{foo3\}\{bar3\}{[section]}]
%       The theorem set {\envfont foo3} (printed name \texttt{bar3}) is
%       enumerated within the counter \texttt{section}, i.e.\ with every
%       new |\section| the enumeration begins again with 1, and
%       the enumeration is composed from the section-number and the
%       theorem counter itself.
% \end{description}
%
% \DescribeMacro\theoremstyle
% Additionally, the command |\theoremstyle| can define the layout
% of various, or all, theorem sets.  It should be noted that any
% theorem set defined by |\newtheorem| is typeset in the
% |\theoremstyle| that is current at the time of the definition.
% Thus, the following
% \begin{verbatim}
%    \theoremstyle{break}        \newtheorem{Cor}{Corollary}
%    \theoremstyle{plain}        \newtheorem{Exa}{Example}[section]
%\end{verbatim}                             ^^A No blank at line start
% leads to the result that the set {\envfont Cor} is formatted in the
% style \texttt{break}, while the set {\envfont Exa} and all the
% following ones are formatted in the style \texttt{plain}, unless
% another |\theoremstyle| follows.  Since the definitions installed by
% |\newtheorem| are global, one also can limit |\theoremstyle| locally
% by grouping braces.
%
% \DescribeMacro\theorembodyfont
% The choice of the font for the theorem body is completely
% independent of the chosen |\theoremstyle|; this has proven to
% be very advantageous.  For example,
% \begin{verbatim}
%   {\theorembodyfont{\upshape}          \newtheorem{Rem}{Remark}}
%\end{verbatim}
% defines a theorem set {\envfont Rem}, which will be set in
% |\upshape| in the current layout (which in our example is
% \texttt{plain}).  As with |\theoremstyle|, the |\theorembodyfont|
% chosen is that current at the time of |\newtheorem|.  If
% |\theorembodyfont| is not specified or one defines
% |\theorembodyfont{}|, then the font used will be that defined by the
% |\theoremstyle|.
%
% \DescribeMacro\theoremheaderfont It is also possible to customize
% the font used for the theorem headers.  This is, however, a global
% declaration, and therefore there should be at most one
% |\theoremheaderfont| declaration in the preamble.\footnote{If it is
% actually necessary to have different header fonts, one has to define
% new theorem styles (substituting the desired font) or specify the
% information directly in the \texttt{\bslash newtheorem} declaration
% (the unclean variant).}
%
% \DescribeMacro\theorempreskipamount
% \DescribeMacro\theorempostskipamount Two additional parameters
% affect the vertical space around the theorem environments:
% |\theorempreskipamount| and |\theorempostskipamount| define,
% respectively, the spacing before and after such an environment.
% These parameters apply for all theorem sets and can be manipulated
% with the ordinary length macros.  They are rubber lengths
% (`\textsf{skips}'), and therefore can contain \texttt{plus} and
% \texttt{minus} parts.
%
% Since the definition of theorem sets should---most sensibly---be
% placed in the preamble, we only allow installation there.
% It is therefore possible to release the memory used here after
% |\begin{document}|, in order to make room for other
% applications.
%
%
% \subsection{Existing theorem styles}
%
% The following theorem styles exist to date:
% \begin{deflist}{marginbreak}
%    \item[plain]
%       This theorem style emulates the original \LaTeX{} definition,
%       except that additionally the parameters
%       |\theorem...skipamount| are used.
%    \item[break]
%       In this style, the theorem header is followed by a line
%       break.
%    \item[marginbreak]
%       The theorem number is set in the margin, and there is a line
%       break as in \texttt{break}.
%    \item[changebreak]
%       Like \texttt{break}, but with header number and text
%       interchanged.
%    \item[change]
%       Header number and text are interchanged, without a line break.
%    \item[margin]
%       The number is set in the left margin, without a line break.
% \end{deflist}
% All styles (except \texttt{plain}) select |\slshape| as the default
% |\theorembodyfont|.
%
%
% \subsection{Examples}
%
% Given the above theorem sets {\envfont Cor}, {\envfont Exa}
% and {\envfont Rem},
% suppose that the preamble also contains the declarations:
% \begin{verbatim}
%    \theoremstyle{marginbreak}   \newtheorem{Lem}[Cor]{Lemma}
%    \theoremstyle{change}
%    \theorembodyfont{\itshape}        \newtheorem{Def}[Cor]{Definition}
%
%    \theoremheaderfont{\scshape}
%\end{verbatim}
% Then the following are some typical examples of the typeset output
% resulting from their use.
%
% \begin{Cor}
%    This is a sentence typeset in the theorem environment {\envfont
%    Cor}.
% \end{Cor}
%
% \begin{Exa}
%    This is a sentence typeset in the theorem environment {\envfont
%    Exa}.
% \end{Exa}
%
% \begin{Rem}
%    This is a sentence typeset in the theorem environment {\envfont
%    Rem}.
% \end{Rem}
%
% \begin{Lem}[Ben User]
%    This is a sentence typeset in the theorem environment {\envfont
%    Lem}.
% \end{Lem}
%
% \begin{Def}[Very impressive Definition]
%    This is a sentence typeset in the theorem environment {\envfont
%    Def}.
% \end{Def}
%
% The last two examples show the effect of the optional argument to
% a theorem environment (it is the text typeset in parentheses).
%
%
% \section{Special Considerations}
% \label{sec:newfonts}
%
% Theoremheader and body are implemented as a unit. This means that
% the |\theoremheaderfont|  will inherit characteristics of the
% |\theorembodyfont| in \LaTeXe. Thus, if for example
% |\theorembodyfont| is |\itshape| and |\theoremheaderfont|
% is |\bfseries| the font selected for the header will have the
% characteristics `bold extended italic'. If this is not desired one
% should set the |\theoremheaderfont| to something like
% \begin{verbatim}
%      \theoremheaderfont{\normalfont\bfseries}
%\end{verbatim}
% i.e.\ supplying all necessary font information explicitly.
%
%
%
% \section{Acknowledgements}
%
% The publication of this set of macros was only possible with the
% help of Christina Busse (translating the manuscript into English),
% Joachim Pense (playing the r\^ole of typist), Chris Rowley
% (looking everything over) and many others providing useful
% suggestions.
%
% \MaybeStop{%
%    \begin{thebibliography}{9}
%       \bibitem{bk:GMS94} \textsc{M.~Goossens}, \textsc{F.~Mittelbach}
%         and \textsc{A.~Samarin}.
%         \newblock The \LaTeX{} Companion.
%         \newblock
%         Addison-Wesley, Reading, Massachusetts, 1994.
%       \bibitem{latex:source} \textsc{Lamport, Leslie.} \newblock
%                               \texttt{latex.tex}, version 2.09, date
%                               Feb. 1990.
%    \end{thebibliography}
%    \PrintIndex
%    \PrintChanges}
%
%
% \section{The documentation driver file}
%
% The next bit of code contains the documentation driver file for
% \TeX{}, i.e., the file that will produce the documentation you are
% currently reading. It will be extracted from this file by the
% \texttt{docstrip} program. Since it is the first code in the file
% one can alternatively process this file directly with \LaTeXe{} to
% obtain the documentation.
%
%    \begin{macrocode}
%<*driver>
\documentclass{ltxdoc}

\usepackage{theorem}

%<+driver>% The next few lines define theorem sets which are used
%<+driver>% in the example section of the documentation.

\theoremstyle{break}       \newtheorem{Cor}{Corollary}
\theoremstyle{plain}       \newtheorem{Exa}{Example}[section]
{\theorembodyfont{\upshape}\newtheorem{Rem}{Remark}}
\theoremstyle{marginbreak} \newtheorem{Lem}[Cor]{Lemma}
\theoremstyle{change}
\theorembodyfont{\itshape} \newtheorem{Def}[Cor]{Definition}

\theoremheaderfont{\scshape}

\RecordChanges

\begin{document}
   \DocInput{theorem.dtx}
\end{document}
%</driver>
%    \end{macrocode}
%
% \section{Definition of the Macros}
%
% If the file has been loaded before, we abort immediately. If not the
% package announces itself (this is actually done at the very top of
% the file---the way it is done isn't good style so don't copy it).
% \changes{v2.0e}{89/07/19}{Spaces removed from `typeout'.}
% \changes{v2.2b}{95/11/19}{Announce moved to top of file}
%    \begin{macrocode}
%<*package>
%\@ifundefined{theorem@style}{}{\endinput}
%\def\FMithmInfo{1995/11/19 v2.2b Theorem extension package (FMi)}
%\ProvidesPackage{theorem}[\FMithmInfo]
%    \end{macrocode}
%
%
% \subsection{Definition of theorem styles and fonts}
%
% All the definitions in this file are done globally to allow
% inputting this file inside a group.
%
% \begin{macro}{\theoremstyle}
%    Before a theorem style can be installed, the chosen style must be
%    known. For that reason, we must test to see that
%    |\th@|\meta{style} is known or, more precisely, that it is
%    different from |\relax|. If the style is not known then
%    |\th@plain| is used.
%    \begin{macrocode}
\gdef\theoremstyle#1{%
   \@ifundefined{th@#1}{\@warning
          {Unknown theoremstyle `#1'. Using `plain'}%
          \theorem@style{plain}}%
%    \end{macrocode}
%    We save the theorem style to be used in the token register
%    |\theorem@style|.
%    \begin{macrocode}
      {\theorem@style{#1}}%
%    \end{macrocode}
%    Now we ``evaluate'' the theorem style: this means, we call the
%    macro |\th@|\meta{style} which will activate the relevant
%    definitions which are contained in a separate file.  This is done
%    in a group to suppress changes to the current font.  This could
%    otherwise pose problems together with the new font selection
%    scheme\footnote{When I printed the original article using the new
%    font selection scheme I ended with a document with slanted
%    typefaces (text headings and all) simply because one of the
%    theorem styles used \texttt{\bslash sl} at toplevel.} if the
%    |\th@|\meta{style} is evaluated a second time.
% \changes{v2.1a}{90/02/15}{Group added around `th@..}
% \changes{v2.1a}{90/02/15}{Reseting of `..@begintheorem removed.}
%    \begin{macrocode}
      \begingroup
        \csname th@\the\theorem@style \endcsname
      \endgroup}
%    \end{macrocode}
% \end{macro}
%
%
% \begin{macro}{\@begintheorem}
% \changes{v2.1a}{90/02/15}{Relaxed at toplevel.}
% \begin{macro}{\@opargbegintheorem}
% \changes{v2.1a}{90/02/15}{Relaxed at toplevel.}
%    We reset |\@begintheorem| and |\@opargbegintheorem| to
%    |\relax| since these commands are no longer necessary at
%    toplevel. This will save a few tokens.
%    \begin{macrocode}
\global\let\@begintheorem\relax
\global\let\@opargbegintheorem\relax
%    \end{macrocode}
% \end{macro}
% \end{macro}
%
% \begin{macro}{\theorem@style}
%    Obviously the \textsf{token} register used above has to be
%    allocated. To assure the utmost compatibility with the
%    original \LaTeX{} definition, we set the default theorem style to
%    \texttt{plain}, which implements the usual \LaTeX{} convention.
% \changes{v2.0f}{89/07/20}{Unnecessary `global' removed.}
%    \begin{macrocode}
\newtoks\theorem@style
\global\theorem@style{plain}
%    \end{macrocode}
% \end{macro}
%
% \begin{macro}{\theorembodyfont}
% \begin{macro}{\theorem@bodyfont}
%    If the |\theorembodyfont| is set by the user then it should not
%    interact with the default font set in the theorem style. When the
%    new font selection is in force this may happen if, for example,
%    the default is |\itshape| and the new |\theorembodyfont| is
%    |\sffamily|. So we add a |\reset@font| command in front of the
%    user definition.
% \changes{v2.0f}{89/07/20}{Unnecessary `global' removed.}
% \changes{v2.0g}{89/09/19}{`theoremfont to `theorembodyfont.}
% \changes{v2.1c}{92/06/22}{`reset@font added and logic changed.}
% \changes{v2.1d}{92/10/28}{Reset to empty if arg is empty}
%    \begin{macrocode}
\gdef\theorembodyfont#1{%
%    \end{macrocode}
%    We check if the argument supplied is empty and if so put nothing
%    into the |\theorem@bodyfont| token register to allow for
%    |\theorembodyfont{}| as a mean of using the default of the
%    current |\theoremstyle|.
%    \begin{macrocode}
   \def\@tempa{#1}%
   \ifx\@tempa\@empty
    \theorem@bodyfont{}%
   \else
    \theorem@bodyfont{\reset@font#1}%
   \fi
   }
\newtoks\theorem@bodyfont
\global\theorem@bodyfont{}
%    \end{macrocode}
% \end{macro}
% \end{macro}
%
%
% \begin{macro}{\theoremheaderfont}
% \changes{v2.0g}{89/09/19}{Macro added.}
%    The font for the theorem headers is handled differently because
%    this definition applies to all theorem styles.
%    \begin{macrocode}
\gdef\theoremheaderfont#1{\gdef\theorem@headerfont{#1}%
%    \end{macrocode}
%    After using the macro once it is redefined to produce an error
%    message.
%    \begin{macrocode}
       \gdef\theoremheaderfont##1{%
        \typeout{\string\theoremheaderfont\space should be used
                 only once.}}}
%    \end{macrocode}
% \end{macro}
%
% \begin{macro}{\theorem@headerfont}
% \changes{v2.0g}{89/09/19}{Macro added.}
%    To set the |\theorem@headerfont| default we first test if
%    the new fontselection scheme is in force.
% \changes{v2.1a}{90/02/15}{Support for new font selection.}
%    \begin{macrocode}
\ifx\upshape\undefined
%    \end{macrocode}
%    If not we define it to expand into |\bfseries|. We don't use
%    |\let| just in case a following style option redefines this
%    macro.
%    \begin{macrocode}
\gdef\theorem@headerfont{\bfseries}
%    \end{macrocode}
%    Otherwise we reset the current shape before calling |\bfseries|.
%    \begin{macrocode}
\else \gdef\theorem@headerfont{\normalfont\bfseries}\fi
%    \end{macrocode}
% \end{macro}
%
% \begin{macro}{\th@plain}
% \begin{macro}{\th@break}
% \begin{macro}{\th@marginbreak}
% \begin{macro}{\th@changebreak}
% \begin{macro}{\th@change}
% \begin{macro}{\th@margin}
%    The different styles are defined in macros such as
%    |\th@plain|.  Since memory space is precious in
%    ``non-Big-versions'', we have to avoid offering too many unused
%    definitions.  Therefore we define these styles in separate files
%    that can be loaded on demand.  Thus the commands themselves only
%    load these files. We use |\@input@|, a \LaTeXe{} internal command
%    that ensures that the file will be listed with |\listfiles|.
% \changes{v2.2a}{94/02/02}{Use `@input@ to load theorem layout files}
%    \begin{macrocode}
\gdef\th@plain{\@input@{thp.sty}}
\gdef\th@break{\@input@{thb.sty}}
\gdef\th@marginbreak{\@input@{thmb.sty}}
\gdef\th@changebreak{\@input@{thcb.sty}}
\gdef\th@change{\@input@{thc.sty}}
\gdef\th@margin{\@input@{thm.sty}}
%    \end{macrocode}
%    This list will be expanded when new styles become
%    available.  For testing,  just append new theorem substyles
%    as document options.
% \end{macro}
% \end{macro}
% \end{macro}
% \end{macro}
% \end{macro}
% \end{macro}
%
%
%
% \subsection{Definition of a new theorem set}
%
% As already pointed out, a new theorem environment can be defined in
% three different ways:
% \begin{verbatim}
%    \newtheorem{Lem}{Lemma}
%    \newtheorem{Lem}{Lemma}[section]
%    \newtheorem{Lem}[Theorem]{Lemma}
%\end{verbatim}
% The function of the macro |\newtheorem| is to recognize these cases
% and then to branch into one of the three macros |\@ynthm|, |\@xnthm|
% or |\@othm|.  This mechanism is adopted unchanged from
% \cite{latex:source}; the essential point here is that, for example,
% in the second case, the arguments \texttt{Lem}, \texttt{Lemma} and
% \texttt{section} are passed over to the macro |\@xnthm|.
%
% We inspect this case first because the others present fewer problems,
% and thus are easily derived from this one.
%
% \begin{macro}{\@xnthm}
%    For our example arguments, the macro |\@xnthm| must fulfill
%    the following:
%    \begin{itemize}
%       \item
%          Define a new \LaTeX-counter `Lem'
%       \item
%          reset this counter within a |\section|
%       \item
%          define the macro |\theLem|
%       \item
%          define the environment macros |\Lem| and
%          |\endLem| using the current
%          |\theoremstyle| and |\theorem@bodyfont|.
%    \end{itemize}
%    Obviously, all this should happen only if the first argument
%    of |\@xnthm| (i.e.\ \texttt{Lem} in our example) is chosen so
%    as not to conflict with any previously defined commands or
%    environments.  This test is performed by the \LaTeX{} macro
%    |\@ifdefinable|.
%    \begin{macrocode}
\gdef\@xnthm#1#2[#3]{\expandafter\@ifdefinable\csname #1\endcsname
%    \end{macrocode}
%    Therefore, the first argument of |\@ifdefinable| is the
%    expansion (in the example, |\Lem|) of
%    |\csname||#1\endcsname|.  The second argument is
%    executed only if the test has been completed successfully.
%    \begin{macrocode}
   {%
%    \end{macrocode}
%    Now we define the new counter. The names of the \LaTeX{}
%    macros employed should speak for themselves:
% \changes{v2.2b}{1995/10/07}{Use \cs{@newctr}}
% \changes{v2.2c}{1995/11/23}{Use \cs{@newctr} properly.}
%    \begin{macrocode}
    \@definecounter{#1}\@newctr{#1}[#3]%
%    \end{macrocode}
%    Using |\@newctr| will give a proper error message if the counter
%    in |#3| is not defined.
%    In defining `|\theLem|' we must generate the desired macro
%    name by use of |\expandafter| and |\csname|.
%    \begin{macrocode}
    \expandafter\xdef\csname the#1\endcsname
%    \end{macrocode}
%    An |\xdef| is used in order to make the definition global,
%    and to ensure that it contains the replacement texts of
%    |\@thmcountersep| and |\@thmcounter|.\footnote{These
%    two macros can be defined by the document style.  Their default
%    values produce a `\texttt{.}' as separation and an arabic
%    representation of the number.} However, not everything should be
%    expanded. For example, it saves space to use |\thesection|
%    instead of its---at times---lengthy expansion.
%    \begin{macrocode}
      {\expandafter \noexpand \csname the#3\endcsname
       \@thmcountersep \@thmcounter{#1}}%
%    \end{macrocode}
%    Thus with the defaults of \LaTeX{}, |\theLem| would be
%    replaced by the command sequence |\thesection.\arabic{Lem}|.
%
%    We will now look at the definition of the macro which is executed
%    at the beginning of the actual environment (in our example this
%    macro is |\Lem|). It should be noted that we use an
%    ``|\expandafter| trick'' to expand only certain parts of the
%    replacement text at the time of the definition.
%    \begin{macrocode}
    \def\@tempa{\global\@namedef{#1}}%
    \expandafter \@tempa \expandafter{%
%    \end{macrocode}
%    First, the macro that contains the current definitions of
%    |\@begintheorem| and |\@opargtheorem| should be called
%    up.  The name of this macro---as is already known---has the form
%    |\th@|\meta{theorem\ style}; therefore, it must be called by
%    \begin{macrocode}
      \csname th@\the \theorem@style
%    \end{macrocode}
%    In addition the default theorem font should be changeable, i.e.\
%    we have to insert the contents of |\theorem@bodyfont|. For
%    that reason, we expand even further, beyond |\endcsname|,
%    and thus insert the contents of the \textsf{token} register
%    |\theorem@bodyfont| in the replacement text.
%    \begin{macrocode}
            \expandafter \endcsname \the \theorem@bodyfont
%    \end{macrocode}
%    Now it is time to call the macro |\@thm| which takes over
%    the further processing.  It has two arguments: the current
%    counter name (in our example, \texttt{Lem}), and the text of the
%    label (in our example, \texttt{Lemma}).
%    \begin{macrocode}
     \@thm{#1}{#2}}%
%    \end{macrocode}
%    With this, the `sub-definition' is complete.  The macro
%    |\@endtheorem| ends a theorem environment and is, so far, nothing
%    but an |\endtrivlist|.  (Hence it is defined globally, and not
%    within the theorem styles.\footnote{This has to be changed as
%    soon as theorem styles that change \texttt{\bslash @endtheorem}
%    exist.  In such a case, all existing styles must be changed as
%    well since they will have to reset the macro.}) Therefore, we can
%    set it equivalent to the macro that ends the theorem set (in our
%    example, |\endLem|).  However, if some day theorem styles exist
%    that do change |\@endtheorem|, we would have to use the
%    commented-out line instead.
%    \begin{macrocode}
    \global \expandafter \let \csname end#1\endcsname \@endtheorem
%   \global\@namedef{end#1}{\@endtheorem}%
%    \end{macrocode}
%    With these commands all the required definitions are employed,
%    unless the test |\@ifdefinable| has failed.  Therefore, we
%    end the second argument of this macro and with it the definition
%    of |\@xnthm|.
%    \begin{macrocode}
   }}
%    \end{macrocode}
% \end{macro}
%
% \begin{macro}{\@ynthm}
%    The definition of |\@ynthm| is completely analogous. In
%    this case the new counter that is defined is not reset within
%    another counter; thus the definition of |\the...|
%    is simplified:
%    \begin{macrocode}
\gdef\@ynthm#1#2{\expandafter\@ifdefinable\csname #1\endcsname
   {\@definecounter{#1}%
    \expandafter\xdef\csname the#1\endcsname{\@thmcounter{#1}}%
%    \end{macrocode}
%    The rest of the definition corresponds literally to that of
%    |\@xnthm|:
%    \begin{macrocode}
    \def\@tempa{\global\@namedef{#1}}\expandafter \@tempa
     \expandafter{\csname th@\the \theorem@style \expandafter
     \endcsname \the\theorem@bodyfont \@thm{#1}{#2}}%
    \global \expandafter \let \csname end#1\endcsname \@endtheorem}}
%    \end{macrocode}
% \end{macro}
%
% \begin{macro}{\@othm}
%    The definition of |\@othm| does not contain anything new.
%    \begin{macrocode}
\gdef\@othm#1[#2]#3{%
%    \end{macrocode}
% \changes{v2.2b}{95/10/07}{Build in check for counter defined}
%    We do not define a new counter but instead use one that has
%    already been defined.  Thus the only definition we need is that
%    of this pseudo-counter (i.e.\ |\the|\meta{env.\ name}).
%    First we check if |#2| corresponds to a known counter name.
%    \begin{macrocode}
  \expandafter\ifx\csname c@#2\endcsname\relax
   \@nocounterr{#2}%
  \else
   \expandafter\@ifdefinable\csname #1\endcsname
   {\expandafter \xdef \csname the#1\endcsname
     {\expandafter \noexpand \csname the#2\endcsname}%
%    \end{macrocode}
%    All other parts of the definition can be adopted from
%    |\@xnthm|. We have to remember, though, that in this case
%    the name of the current counter and the theorem label have moved
%    to the second and third arguments.
%    \begin{macrocode}
    \def\@tempa{\global\@namedef{#1}}\expandafter \@tempa
     \expandafter{\csname th@\the \theorem@style \expandafter
     \endcsname \the\theorem@bodyfont \@thm{#2}{#3}}%
    \global \expandafter \let \csname end#1\endcsname \@endtheorem}%
  \fi}
%    \end{macrocode}
% \end{macro}
%
% \subsection{Macros that are employed in a {\envfont theorem}
%    environment}
%
% \begin{macro}{\@thm}
%    The macro |\@thm| has to increase the current counter.  Then,
%    depending on whether the environment has (or does not have) an
%    optional argument, it has to branch into either |\@begintheorem|
%    or |\@opargtheorem|.
%    \begin{macrocode}
\gdef\@thm#1#2{\refstepcounter{#1}%
%    \end{macrocode}
%    Now we start a {\envfont trivlist} environment, and give
%    |\@topsep| and |\@topsepadd| the values of the \textsf{skip}
%    registers |\theorempreskipamount| and |\theorempostskipamount|.
%    The value in |\@topsep| is the vertical space that is inserted by
%    the first (and only) |\item| in our |\trivlist| whilst
%    |\@topsepadd| is inserted by |\@endparenv| at the end of that
%    {\envfont trivlist} environment.  By using these registers, we
%    obtain the desired space around a {\envfont theorem} environment.
%    \begin{macrocode}
   \trivlist
   \@topsep \theorempreskipamount               % used by first \item
   \@topsepadd \theorempostskipamount           % used by \@endparenv
%    \end{macrocode}
%    Now we have to test whether an optional argument has been given.
% \changes{v2.0d}{89/07/18}{Forgotten percent added.}
%    \begin{macrocode}
   \@ifnextchar [%
%    \end{macrocode}
%    If there is an optional argument, we will call |\@ythm|, and
%    move the arguments read back into the input stream.
%    \begin{macrocode}
   {\@ythm{#1}{#2}}%
%    \end{macrocode}
%    If not, we call |\@begintheorem|. Its first argument is the
%    name of the theorem set (hence the second argument of
%    |\@thm|). Its second argument is the macro that produces the
%    current number.
%    \begin{macrocode}
   {\@begintheorem{#2}{\csname the#1\endcsname}\ignorespaces}}
%    \end{macrocode}
% \end{macro}
%
% \begin{macro}{\@xthm}
% \begin{macro}{\@ythm}
%    Both these macros were originally called by |\@thm|.  We do
%    not need |\@xthm| anymore, hence we reset it to
%    |\relax|. The definition of |\@ythm| has not changed at
%    all from its definition in \LaTeX. In order to make the macros
%    easier to understand, we will
%    nevertheless present it (commented out).
%    \begin{macrocode}
\global\let\@xthm\relax
% \def\@ythm#1#2[#3]{\@opargbegintheorem{#2}{\csname
%           the#1\endcsname}{#3}\ignorespaces}
%</package>
%    \end{macrocode}
%    The primitive |\ignorespaces| in |\@ythm| and
%    |\@thm| is needed to remove the spaces between the
%    |\begin{...}| and the actual text.
% \end{macro}
% \end{macro}
%
%
% \subsection{Definition of the theorem substyles}
%
% As already pointed out, the theorem substyles, defined below, are
% only loaded when necessary.  Note that all these substyles, except
% \texttt{plain}, have |\slshape| as the default body font.
%
% \subsubsection{The \texttt{plain} style}
%
% \label{dok:th@plain} As the following macros use |@|, we have
% to locally set the |\catcode| of this symbol to ``letter''.
% This happens within a group, so that we do not have to worry about
% which |\catcode| that symbol had before.
%    \begin{macrocode}
%<*thp>
\begingroup \makeatletter
%    \end{macrocode}
% Since we are now within a group, we must make all definitions
% globally.  First we make sure that \texttt{theorem.sty} is loaded.
% This will allow us to use this file as a document style option
% without having to call \texttt{theorem} itself as an option.  At the
% same time, we assure that at least version 2 is loaded, since
% |\theorem@style| was not defined in earlier versions.
%    \begin{macrocode}
\@ifundefined{theorem@style}{\input{theorem.sty}}{}
\ProvidesFile{thp.sty}
             [\FMithmInfo]
%    \end{macrocode}
%
%
% \begin{macro}{\th@plain}
% \begin{macro}{\@begintheorem}
% \begin{macro}{\@opargbegintheorem}
%    |\theoremstyle{plain}| corresponds to the original
%    definition, except that the distances to the surrounding text are
%    determined by the parameters |\theorempreskipamount| and
%    |\theorempostskipamount|.
%    First we set the default body font.
%    \begin{macrocode}
\gdef\th@plain{\normalfont\itshape
%    \end{macrocode}
%    Then we define |\@begintheorem| and
%    |\@opargbegintheorem|. These two macros define how the
%    header of a theorem is typeset.  |\@opargbegintheorem| will
%    be called if a \textsf{theorem} environment with an optional
%    argument is encountered; otherwise, the header is constructed by
%    calling |\@begintheorem|.  If one of these macros is
%    executed, we are within a \textsf{trivlist} environment started by
%    |\@thm|.  So the theorem header is produced with an
%    |\item| command.
%
%    Instead of specifying the header font directly, all standard
%    theorem styles use the |\theorem@headerfont| macro to allow
%    customization.  The extra space (|\labelsep|) is necessary
%    because of problems in the \textsf{trivlist} environment.
%    \begin{macrocode}
  \def\@begintheorem##1##2{%
        \item[\hskip\labelsep \theorem@headerfont ##1\ ##2]}%
%    \end{macrocode}
%    The definition of |\@opargbegintheorem| is completely
%    analogous. The only difference is the fact that there exists a
%    third argument (which is the optional parameter of the
%    environment and contains additional information about the
%    theorem).  Customarily we enclose it in parentheses.
%    \begin{macrocode}
\def\@opargbegintheorem##1##2##3{%
   \item[\hskip\labelsep \theorem@headerfont ##1\ ##2\ (##3)]}}
%    \end{macrocode}
% \end{macro}
% \end{macro}
% \end{macro}
%
% \noindent
% We conclude with an |\endgroup| to restore the
% |\catcode| of |@|.
%    \begin{macrocode}
\endgroup
%</thp>
%    \end{macrocode}
%
%
% \subsubsection{The \texttt{break} style}
%
% This style option is stored in the file |thb.sty|.
% For the next two lines see the documentation for |\th@plain| on
% page~\pageref{dok:th@plain}.
%    \begin{macrocode}
%<*thb>
\begingroup \makeatletter
\@ifundefined{theorem@style}{\input{theorem.sty}}{}
\ProvidesFile{thb.sty}
             [\FMithmInfo]
%    \end{macrocode}
%
% \begin{macro}{\th@break}
%    |\theoremstyle{break}| produces a line break after the name
%    of the theorem.  The font is |\slshape|.  Hence, we define
%    |\th@break| as follows:
%    \begin{macrocode}
\gdef\th@break{\normalfont\slshape
  \def\@begintheorem##1##2{\item[%
%    \end{macrocode}
%    We run into the following problem: it is not possible to create
%    the header with |\item[|\meta{title}|]| and then start
%    a new line by, for example, |\mbox{}\\|. Such a definition
%    will fail whenever a list environment follows immediately.  With
%    the above construction, the |\mbox{}| causes the switch
%    |@inlabel| (cf.\ definition of |\list| and
%    |\trivlist| in \cite{latex:source}) to be set to \texttt{false}
%    and so the following list will insert additional vertical
%    space (|\topskip|). This is quite annoying. Therefore, we
%    create the line break within the |\item|. In order to ensure
%    that the text will begin at the proper position in the following
%    line, we simply pretend that the label does not take any
%    room.\footnote{This will lead to problems whenever very high
%    symbols occurring in the line tower into the
%    heading. So, something else has to be done here sometime.}
%    \begin{macrocode}
       \rlap{\vbox{\hbox{\hskip \labelsep\theorem@headerfont ##1\ ##2}%
                   \hbox{\strut}}}]}%
%    \end{macrocode}
%    Again, the definition of |\@opargbegintheorem| is
%    completely analogous.
%    \begin{macrocode}
\def\@opargbegintheorem##1##2##3{%
  \item[\rlap{\vbox{\hbox{\hskip \labelsep \theorem@headerfont
                     ##1\ ##2\ (##3)}%
                    \hbox{\strut}}}]}}
\endgroup
%</thb>
%    \end{macrocode}
% \end{macro}
%
%
% \subsubsection{The \texttt{changebreak} style}
%
%    \begin{macrocode}
% This style option is stored in the file |thcb.sty|.
%    \begin{macrocode}
%<*thcb>
\begingroup \makeatletter
\@ifundefined{theorem@style}{\input{theorem.sty}}{}
\ProvidesFile{thcb.sty}
             [\FMithmInfo]
%    \end{macrocode}
%
% \begin{macro}{\th@changebreak}
%    The \texttt{change-break} theorem style is like \texttt{break}
%    but with interchange of theorem name and theorem number.  Thus we
%    define |\th@changebreak| as follows:
%    \begin{macrocode}
\gdef\th@changebreak{\normalfont\slshape
  \def\@begintheorem##1##2{\item
       [\rlap{\vbox{\hbox{\hskip\labelsep \theorem@headerfont ##2\ ##1}%
                    \hbox{\strut}}}]}%
\def\@opargbegintheorem##1##2##3{%
  \item[\rlap{\vbox{\hbox{\hskip\labelsep \theorem@headerfont
                     ##2\ ##1\ (##3)}%
                    \hbox{\strut}}}]}}
\endgroup
%</thcb>
%    \end{macrocode}
% \end{macro}
%
%
% \subsubsection{The \texttt{change} style}
%
% This style option is stored in the file |thc.sty|.
%    \begin{macrocode}
%<*thc>
\begingroup \makeatletter
\@ifundefined{theorem@style}{\input{theorem.sty}}{}
\ProvidesFile{thc.sty}
             [\FMithmInfo]
%    \end{macrocode}
%
% \begin{macro}{\th@change}
%    The \texttt{change} theorem style corresponds to the
%    \texttt{change break} style without a linebreak after the header.
%    To say it in another way, it's the same as the \texttt{plain}
%    style but with number and name interchanged and |\slshape| as the
%    default font.
%    \begin{macrocode}
\gdef\th@change{\normalfont\slshape
  \def\@begintheorem##1##2{\item
       [\hskip\labelsep \theorem@headerfont ##2\ ##1]}%
\def\@opargbegintheorem##1##2##3{%
  \item[\hskip\labelsep \theorem@headerfont ##2\ ##1\ (##3)]}}
\endgroup
%</thc>
%    \end{macrocode}
% \end{macro}
%
%
% \subsubsection{The \texttt{marginbreak} style}
%
% This style option is the one used most often at Mainz.  It is saved
% in the file |thmb.sty|.
%    \begin{macrocode}
%<*thmb>
\begingroup \makeatletter
\@ifundefined{theorem@style}{\input{theorem.sty}}{}
\ProvidesFile{thmb.sty}
             [\FMithmInfo]
%    \end{macrocode}
%
% \begin{macro}{\th@marginbreak}
%    The \texttt{margin break} style is nearly the same as the
%    \texttt{change break} style. The only difference is the placement
%    of the theorem number.  We use |\llap| to place it in the left
%    margin.
%
%    In this style |\labelsep| denotes the separation between the
%    number and the text.
%    \begin{macrocode}
\gdef\th@marginbreak{\normalfont\slshape
  \def\@begintheorem##1##2{\item
       [\rlap{\vbox{\theorem@headerfont
                    \hbox{\llap{##2}\hskip\labelsep ##1}%
                    \hbox{\strut}}}]}%
\def\@opargbegintheorem##1##2##3{%
  \item[\rlap{\vbox{\theorem@headerfont
                    \hbox{\llap{##2}\hskip\labelsep ##1\ (##3)}%
                    \hbox{\strut}}}]}}
\endgroup
%</thmb>
%    \end{macrocode}
% \changes{v2.1c}{90/02/17}{`endgroup outside macrocode env.}
% \end{macro}
%
%
%
% \subsubsection{The \texttt{margin} style}
%
% This style option is stored in the file |thm.sty|.
%    \begin{macrocode}
%<*thm>
\begingroup \makeatletter
\@ifundefined{theorem@style}{\input{theorem.sty}}{}
\ProvidesFile{thm.sty}
             [\FMithmInfo]
%    \end{macrocode}
%
% \begin{macro}{\th@margin}
%    Again this is only a variant of the theorem styles described
%    above without any new ideas.
%    \begin{macrocode}
\gdef\th@margin{\normalfont\slshape
  \def\@begintheorem##1##2{\item
       [\theorem@headerfont \llap{##2}\hskip\labelsep ##1]}%
\def\@opargbegintheorem##1##2##3{%
  \item[\theorem@headerfont \llap{##2}\hskip\labelsep ##1\ (##3)]}}
\endgroup
%</thm>
%    \end{macrocode}
% \end{macro}
%
%
%
%
% \subsection{Final Definitions}
%
% \begin{macro}{\theorempreskipamount}
% \begin{macro}{\theorempostskipamount}
%    The \textsf{skip} parameters that regulate the vertical empty space
%    before and after the {\envfont theorem} environment have
%    to be allocated as well.
% \changes{v2.0f}{89/07/20}{Unnecessary `global' removed.}
%    \begin{macrocode}
%<*package>
\newskip\theorempreskipamount
\newskip\theorempostskipamount
%    \end{macrocode}
%    Since we have used the same values for all theorem sets, we now can
%    assign them.
%    \begin{macrocode}
\global\setlength\theorempreskipamount{12pt plus 5pt minus 3pt}
\global\setlength\theorempostskipamount{8pt plus 3pt minus 1.5pt}
%    \end{macrocode}
% \end{macro}
% \end{macro}
%
%
% \begin{macro}{\@endtheorem}
%    The same holds for the macro |\@endtheorem|, which ends a
%    {\envfont theorem} environment. Since it is the same for all
%    theorem sets, it is removed from the macros |\th@|\meta{style}.
%    It simply ends the {\envfont trivlist} environment, which was
%    begun in |\@thm|.
%    \begin{macrocode}
\global\let\@endtheorem=\endtrivlist
%    \end{macrocode}
% \end{macro}
%
%
%
% \begin{macro}{\@preamblecmds}
%    All macros defined above are to be used only in the preamble.
%    Therefore, we insert them in |\@preamblecmds| which will disable
%    them at begin document. This is done by the internal \LaTeXe{}
%    command |\@onlypreamble|.
% \changes{v2.0g}{89/09/19}{`theoremheaderfont and `theorembodyfont
%    added.}
% \changes{v2.2a}{94/02/03}{Use \LaTeXe{} interface}
%    \begin{macrocode}
\@onlypreamble\@xnthm
\@onlypreamble\@ynthm
\@onlypreamble\@othm
\@onlypreamble\newtheorem
\@onlypreamble\theoremstyle
\@onlypreamble\theorembodyfont
\@onlypreamble\theoremheaderfont
%    \end{macrocode}
% \end{macro}
%
% Finally we declare the \texttt{plain} theorem style to be the default.
% \changes{v2.1b}{90/06/04}{Fixed missing `item by declaring default
%    style.}
%    \begin{macrocode}
\theoremstyle{plain}
%</package>
%    \end{macrocode}
%
% \Finale

\endinput
}{}
\ProvidesFile{thmb.sty}
             [\FMithmInfo]
%    \end{macrocode}
%
% \begin{macro}{\th@marginbreak}
%    The \texttt{margin break} style is nearly the same as the
%    \texttt{change break} style. The only difference is the placement
%    of the theorem number.  We use |\llap| to place it in the left
%    margin.
%
%    In this style |\labelsep| denotes the separation between the
%    number and the text.
%    \begin{macrocode}
\gdef\th@marginbreak{\normalfont\slshape
  \def\@begintheorem##1##2{\item
       [\rlap{\vbox{\theorem@headerfont
                    \hbox{\llap{##2}\hskip\labelsep ##1}%
                    \hbox{\strut}}}]}%
\def\@opargbegintheorem##1##2##3{%
  \item[\rlap{\vbox{\theorem@headerfont
                    \hbox{\llap{##2}\hskip\labelsep ##1\ (##3)}%
                    \hbox{\strut}}}]}}
\endgroup
%</thmb>
%    \end{macrocode}
% \changes{v2.1c}{90/02/17}{`endgroup outside macrocode env.}
% \end{macro}
%
%
%
% \subsubsection{The \texttt{margin} style}
%
% This style option is stored in the file |thm.sty|.
%    \begin{macrocode}
%<*thm>
\begingroup \makeatletter
\@ifundefined{theorem@style}{% \iffalse meta-comment
%
% Copyright (C) 1993-2024
%
% The LaTeX Project and any individual authors listed elsewhere
% in this file.
%
% This file is part of the Standard LaTeX `Tools Bundle'.
% -------------------------------------------------------
%
% It may be distributed and/or modified under the
% conditions of the LaTeX Project Public License, either version 1.3c
% of this license or (at your option) any later version.
% The latest version of this license is in
%    https://www.latex-project.org/lppl.txt
% and version 1.3c or later is part of all distributions of LaTeX
% version 2005/12/01 or later.
%
% The list of all files belonging to the LaTeX `Tools Bundle' is
% given in the file `manifest.txt'.
%
% \fi
%
%%
%
%
\def\FMithmInfo{2023/07/05 v2.2c Theorem extension package (FMi)}
%
% \ProvidesFile{theorem.dtx}[\FMithmInfo]
%\iffalse   % this is a METACOMMENT !
%%
%% Package `theorem' to use with LaTeX2e
%% Copyright (C) 1989-1995 Frank Mittelbach, all rights reserved.
%<+package>\@ifundefined{theorem@style}{}{\endinput}
%<+package>\NeedsTeXFormat{LaTeX2e}[1995/06/01]
%<+package>\ProvidesPackage{theorem}
%<+package>         [\FMithmInfo]
%
% \fi
%
% \changes{v1.0}{87/05/07}{First draft.}
% \changes{v1.0a}{87/05/07}{`thistheoremstyle' added.}
% \changes{v1.1a}{87/07/29}{New concept: redefine the internal
%                          \protect\LaTeX{} commands `@xnthm' etc.}
% \changes{v1.1b}{87/07/31}{`theoremfont' used to specify font.}
% \changes{v1.2a}{88/01/08}{`@endtheorem' now globally defined.}
% \changes{v1.2b}{88/01/11}{`newtheorem' only allowed in the preamble.}
% \changes{v1.3a}{88/02/14}{`theoremfont' now a toks register.}
%
% \changes{v2.0a}{88/11/16}{New implementation and new documentation.}
% \changes{v2.0a}{88/11/16}{No layout changes but internal modifications
%                           old substyles won't work.}
% \changes{v2.0b}{89/01/30}{`global' used for all definitions.}
% \changes{v2.0c}{89/06/15}{Documentation translated to English.
%                            Thanks to LLL and Nina}
% \changes{v2.1a}{90/02/15}{Support for new font selection.}
% \changes{v2.2a}{94/02/03}{Cleanup for LaTeX2e}
%
% \DoNotIndex{\@endparenv,\@ifdefinable,\@ifnextchar,\@ifundefined}
% \DoNotIndex{\@namedef,\@nameuse,\@spaces,\@tempa,\@warning}
% \DoNotIndex{\ }
% \DoNotIndex{\begingroup,\bfseries}
% \DoNotIndex{\csname}
% \DoNotIndex{\def,\do,\docdate}
% \DoNotIndex{\endcsname,\endgroup,\endinput,\endtrivlist,\expandafter}
% \DoNotIndex{\filedate,\fileversion}
% \DoNotIndex{\gdef,\global}
% \DoNotIndex{\hbox,\hskip}
% \DoNotIndex{\ignorespaces,\input,\itshape,\item}
% \DoNotIndex{\labelsep,\let,\llap}
% \DoNotIndex{\makeatletter}
% \DoNotIndex{\noexpand}
% \DoNotIndex{\refstepcounter,\relax,\rlap}
% \DoNotIndex{\setlength,\slshape,\space,\string,\strut}
% \DoNotIndex{\the,\trivlist,\typeout}
% \DoNotIndex{\vbox}
% \DoNotIndex{\xdef}
%
% \def\deflabel#1{\ttfamily #1\hfill}
% \def\deflist#1{\begin{list}{}{\settowidth\labelwidth{\ttfamily #1}%
%                               \setlength\leftmargin\labelwidth
%                               \addtolength\leftmargin\labelsep
%                               \let\makelabel\deflabel}}
% \def\enddeflist{\end{list}}
% \def\envfont{\normalfont\ttfamily}
%
% \MakeShortVerb{\|}
%
% \newcommand{\TUB}{{\sl TUGboat\/}}
%
% \GetFileInfo{theorem.dtx}
% \title{An Extension of the
%        \LaTeX{} theorem environment\thanks{This file
%        has version number \fileversion, last
%        revised \filedate.}}
% \author{Frank Mittelbach}
%
% \MaintainedByLaTeXTeam{tools}
% \maketitle
%
% \begin{abstract}
%    The macros described in this paper yield an extension of the
%    \LaTeX{} theorem mechanism. It is designed to satisfy the
%    different requirements of various journals. Thus, the
%    layout of the ``theorems'' can be manipulated by determining a
%    ``style''. This article describes not only the use, but
%    also the definition, of the necessary macros.
% \end{abstract}
%
% \section*{Preface to version 2.2}
%
% For \LaTeXe{} this package did not need any fundamental changes. I
% only modified the messages generated so that theorem layout styles
% will show up with the |\listfiles| command and cleaned the section
% on the New Font Selection Scheme since this is now included in
% \LaTeX.
%
% \section*{Preface to version 2.1}
%
% This version is identical to 2.0g described in \TUB\ 10\#3 except
% for some internal defaults which are now set depending on the used
% font selection scheme.
%
% This was done to avoid unpleasant surprises if the new font selection
% scheme is in force. For further details see
% section~\ref{sec:newfonts} and \cite{bk:GMS94}.
%
% \section{Introduction}
%
% For our purposes here, ``theorems'' are labelled enunciations,
% often set off from the main text by extra space and a font change.
% Theorems, corollaries, conjectures, definitions, and remarks
% are all instances of ``theorems''.  The ``header'' of these structures
% is composed of a label (such as \textsc{Theorem} or \textsc{Remark})
% and a number which serializes an item in the sequence of items
% with the same label.
%
% Shortly after the introduction of \LaTeX{} at the Fachbereich
% Mathematik in Mainz, the desire to manipulate the layout of
% ``theorems'' arose.  In Mainz, the following two conventions came
% into general use:
% \begin{enumerate}
%   \item The number of the theorem is shown in the margin.
%   \item There is a line break at
%         the end of the theorem header.
% \end{enumerate}
%
% Additionally, some journals require different formats which depend on
% the ``sort of theorem'': e.g.\ often remarks and definitions are set
% in |\upshape|, while |\itshape| is employed for main theorems.
%
% Confronted with these requirements, a theorem environment was
% developed in Mainz which allows separate determination of the
% layout of the ``theorems sets'', comparable to |\pagestyle|.
%
%
% \section{The user interface}
% \subsection{Defining new theorem sets}
%
% \DescribeMacro\newtheorem
% As in the original \LaTeX{} version, the command |\newtheorem|
% defines a new ``theorem set'' or ``theorem-like structure''.
% Two required arguments name the new environment and give the text
% to be typeset with each instance of the new ``set'', while
% an optional argument determines how
% the ``set'' is enumerated:
% \begin{description}
%    \item[\ttfamily \bslash newtheorem\{foo\}\{bar\}]
%       The theorem set {\envfont foo} (whose name is \texttt{bar})
%       uses its own counter.
%    \item[\ttfamily \bslash newtheorem\{foo2\}{[foo]}\{bar2\}]
%       The theorem set {\envfont foo2} (printed name \texttt{bar2})
%       uses the same counter as the theorem set \texttt{foo}.
%    \item[\ttfamily \bslash newtheorem\{foo3\}\{bar3\}{[section]}]
%       The theorem set {\envfont foo3} (printed name \texttt{bar3}) is
%       enumerated within the counter \texttt{section}, i.e.\ with every
%       new |\section| the enumeration begins again with 1, and
%       the enumeration is composed from the section-number and the
%       theorem counter itself.
% \end{description}
%
% \DescribeMacro\theoremstyle
% Additionally, the command |\theoremstyle| can define the layout
% of various, or all, theorem sets.  It should be noted that any
% theorem set defined by |\newtheorem| is typeset in the
% |\theoremstyle| that is current at the time of the definition.
% Thus, the following
% \begin{verbatim}
%    \theoremstyle{break}        \newtheorem{Cor}{Corollary}
%    \theoremstyle{plain}        \newtheorem{Exa}{Example}[section]
%\end{verbatim}                             ^^A No blank at line start
% leads to the result that the set {\envfont Cor} is formatted in the
% style \texttt{break}, while the set {\envfont Exa} and all the
% following ones are formatted in the style \texttt{plain}, unless
% another |\theoremstyle| follows.  Since the definitions installed by
% |\newtheorem| are global, one also can limit |\theoremstyle| locally
% by grouping braces.
%
% \DescribeMacro\theorembodyfont
% The choice of the font for the theorem body is completely
% independent of the chosen |\theoremstyle|; this has proven to
% be very advantageous.  For example,
% \begin{verbatim}
%   {\theorembodyfont{\upshape}          \newtheorem{Rem}{Remark}}
%\end{verbatim}
% defines a theorem set {\envfont Rem}, which will be set in
% |\upshape| in the current layout (which in our example is
% \texttt{plain}).  As with |\theoremstyle|, the |\theorembodyfont|
% chosen is that current at the time of |\newtheorem|.  If
% |\theorembodyfont| is not specified or one defines
% |\theorembodyfont{}|, then the font used will be that defined by the
% |\theoremstyle|.
%
% \DescribeMacro\theoremheaderfont It is also possible to customize
% the font used for the theorem headers.  This is, however, a global
% declaration, and therefore there should be at most one
% |\theoremheaderfont| declaration in the preamble.\footnote{If it is
% actually necessary to have different header fonts, one has to define
% new theorem styles (substituting the desired font) or specify the
% information directly in the \texttt{\bslash newtheorem} declaration
% (the unclean variant).}
%
% \DescribeMacro\theorempreskipamount
% \DescribeMacro\theorempostskipamount Two additional parameters
% affect the vertical space around the theorem environments:
% |\theorempreskipamount| and |\theorempostskipamount| define,
% respectively, the spacing before and after such an environment.
% These parameters apply for all theorem sets and can be manipulated
% with the ordinary length macros.  They are rubber lengths
% (`\textsf{skips}'), and therefore can contain \texttt{plus} and
% \texttt{minus} parts.
%
% Since the definition of theorem sets should---most sensibly---be
% placed in the preamble, we only allow installation there.
% It is therefore possible to release the memory used here after
% |\begin{document}|, in order to make room for other
% applications.
%
%
% \subsection{Existing theorem styles}
%
% The following theorem styles exist to date:
% \begin{deflist}{marginbreak}
%    \item[plain]
%       This theorem style emulates the original \LaTeX{} definition,
%       except that additionally the parameters
%       |\theorem...skipamount| are used.
%    \item[break]
%       In this style, the theorem header is followed by a line
%       break.
%    \item[marginbreak]
%       The theorem number is set in the margin, and there is a line
%       break as in \texttt{break}.
%    \item[changebreak]
%       Like \texttt{break}, but with header number and text
%       interchanged.
%    \item[change]
%       Header number and text are interchanged, without a line break.
%    \item[margin]
%       The number is set in the left margin, without a line break.
% \end{deflist}
% All styles (except \texttt{plain}) select |\slshape| as the default
% |\theorembodyfont|.
%
%
% \subsection{Examples}
%
% Given the above theorem sets {\envfont Cor}, {\envfont Exa}
% and {\envfont Rem},
% suppose that the preamble also contains the declarations:
% \begin{verbatim}
%    \theoremstyle{marginbreak}   \newtheorem{Lem}[Cor]{Lemma}
%    \theoremstyle{change}
%    \theorembodyfont{\itshape}        \newtheorem{Def}[Cor]{Definition}
%
%    \theoremheaderfont{\scshape}
%\end{verbatim}
% Then the following are some typical examples of the typeset output
% resulting from their use.
%
% \begin{Cor}
%    This is a sentence typeset in the theorem environment {\envfont
%    Cor}.
% \end{Cor}
%
% \begin{Exa}
%    This is a sentence typeset in the theorem environment {\envfont
%    Exa}.
% \end{Exa}
%
% \begin{Rem}
%    This is a sentence typeset in the theorem environment {\envfont
%    Rem}.
% \end{Rem}
%
% \begin{Lem}[Ben User]
%    This is a sentence typeset in the theorem environment {\envfont
%    Lem}.
% \end{Lem}
%
% \begin{Def}[Very impressive Definition]
%    This is a sentence typeset in the theorem environment {\envfont
%    Def}.
% \end{Def}
%
% The last two examples show the effect of the optional argument to
% a theorem environment (it is the text typeset in parentheses).
%
%
% \section{Special Considerations}
% \label{sec:newfonts}
%
% Theoremheader and body are implemented as a unit. This means that
% the |\theoremheaderfont|  will inherit characteristics of the
% |\theorembodyfont| in \LaTeXe. Thus, if for example
% |\theorembodyfont| is |\itshape| and |\theoremheaderfont|
% is |\bfseries| the font selected for the header will have the
% characteristics `bold extended italic'. If this is not desired one
% should set the |\theoremheaderfont| to something like
% \begin{verbatim}
%      \theoremheaderfont{\normalfont\bfseries}
%\end{verbatim}
% i.e.\ supplying all necessary font information explicitly.
%
%
%
% \section{Acknowledgements}
%
% The publication of this set of macros was only possible with the
% help of Christina Busse (translating the manuscript into English),
% Joachim Pense (playing the r\^ole of typist), Chris Rowley
% (looking everything over) and many others providing useful
% suggestions.
%
% \MaybeStop{%
%    \begin{thebibliography}{9}
%       \bibitem{bk:GMS94} \textsc{M.~Goossens}, \textsc{F.~Mittelbach}
%         and \textsc{A.~Samarin}.
%         \newblock The \LaTeX{} Companion.
%         \newblock
%         Addison-Wesley, Reading, Massachusetts, 1994.
%       \bibitem{latex:source} \textsc{Lamport, Leslie.} \newblock
%                               \texttt{latex.tex}, version 2.09, date
%                               Feb. 1990.
%    \end{thebibliography}
%    \PrintIndex
%    \PrintChanges}
%
%
% \section{The documentation driver file}
%
% The next bit of code contains the documentation driver file for
% \TeX{}, i.e., the file that will produce the documentation you are
% currently reading. It will be extracted from this file by the
% \texttt{docstrip} program. Since it is the first code in the file
% one can alternatively process this file directly with \LaTeXe{} to
% obtain the documentation.
%
%    \begin{macrocode}
%<*driver>
\documentclass{ltxdoc}

\usepackage{theorem}

%<+driver>% The next few lines define theorem sets which are used
%<+driver>% in the example section of the documentation.

\theoremstyle{break}       \newtheorem{Cor}{Corollary}
\theoremstyle{plain}       \newtheorem{Exa}{Example}[section]
{\theorembodyfont{\upshape}\newtheorem{Rem}{Remark}}
\theoremstyle{marginbreak} \newtheorem{Lem}[Cor]{Lemma}
\theoremstyle{change}
\theorembodyfont{\itshape} \newtheorem{Def}[Cor]{Definition}

\theoremheaderfont{\scshape}

\RecordChanges

\begin{document}
   \DocInput{theorem.dtx}
\end{document}
%</driver>
%    \end{macrocode}
%
% \section{Definition of the Macros}
%
% If the file has been loaded before, we abort immediately. If not the
% package announces itself (this is actually done at the very top of
% the file---the way it is done isn't good style so don't copy it).
% \changes{v2.0e}{89/07/19}{Spaces removed from `typeout'.}
% \changes{v2.2b}{95/11/19}{Announce moved to top of file}
%    \begin{macrocode}
%<*package>
%\@ifundefined{theorem@style}{}{\endinput}
%\def\FMithmInfo{1995/11/19 v2.2b Theorem extension package (FMi)}
%\ProvidesPackage{theorem}[\FMithmInfo]
%    \end{macrocode}
%
%
% \subsection{Definition of theorem styles and fonts}
%
% All the definitions in this file are done globally to allow
% inputting this file inside a group.
%
% \begin{macro}{\theoremstyle}
%    Before a theorem style can be installed, the chosen style must be
%    known. For that reason, we must test to see that
%    |\th@|\meta{style} is known or, more precisely, that it is
%    different from |\relax|. If the style is not known then
%    |\th@plain| is used.
%    \begin{macrocode}
\gdef\theoremstyle#1{%
   \@ifundefined{th@#1}{\@warning
          {Unknown theoremstyle `#1'. Using `plain'}%
          \theorem@style{plain}}%
%    \end{macrocode}
%    We save the theorem style to be used in the token register
%    |\theorem@style|.
%    \begin{macrocode}
      {\theorem@style{#1}}%
%    \end{macrocode}
%    Now we ``evaluate'' the theorem style: this means, we call the
%    macro |\th@|\meta{style} which will activate the relevant
%    definitions which are contained in a separate file.  This is done
%    in a group to suppress changes to the current font.  This could
%    otherwise pose problems together with the new font selection
%    scheme\footnote{When I printed the original article using the new
%    font selection scheme I ended with a document with slanted
%    typefaces (text headings and all) simply because one of the
%    theorem styles used \texttt{\bslash sl} at toplevel.} if the
%    |\th@|\meta{style} is evaluated a second time.
% \changes{v2.1a}{90/02/15}{Group added around `th@..}
% \changes{v2.1a}{90/02/15}{Reseting of `..@begintheorem removed.}
%    \begin{macrocode}
      \begingroup
        \csname th@\the\theorem@style \endcsname
      \endgroup}
%    \end{macrocode}
% \end{macro}
%
%
% \begin{macro}{\@begintheorem}
% \changes{v2.1a}{90/02/15}{Relaxed at toplevel.}
% \begin{macro}{\@opargbegintheorem}
% \changes{v2.1a}{90/02/15}{Relaxed at toplevel.}
%    We reset |\@begintheorem| and |\@opargbegintheorem| to
%    |\relax| since these commands are no longer necessary at
%    toplevel. This will save a few tokens.
%    \begin{macrocode}
\global\let\@begintheorem\relax
\global\let\@opargbegintheorem\relax
%    \end{macrocode}
% \end{macro}
% \end{macro}
%
% \begin{macro}{\theorem@style}
%    Obviously the \textsf{token} register used above has to be
%    allocated. To assure the utmost compatibility with the
%    original \LaTeX{} definition, we set the default theorem style to
%    \texttt{plain}, which implements the usual \LaTeX{} convention.
% \changes{v2.0f}{89/07/20}{Unnecessary `global' removed.}
%    \begin{macrocode}
\newtoks\theorem@style
\global\theorem@style{plain}
%    \end{macrocode}
% \end{macro}
%
% \begin{macro}{\theorembodyfont}
% \begin{macro}{\theorem@bodyfont}
%    If the |\theorembodyfont| is set by the user then it should not
%    interact with the default font set in the theorem style. When the
%    new font selection is in force this may happen if, for example,
%    the default is |\itshape| and the new |\theorembodyfont| is
%    |\sffamily|. So we add a |\reset@font| command in front of the
%    user definition.
% \changes{v2.0f}{89/07/20}{Unnecessary `global' removed.}
% \changes{v2.0g}{89/09/19}{`theoremfont to `theorembodyfont.}
% \changes{v2.1c}{92/06/22}{`reset@font added and logic changed.}
% \changes{v2.1d}{92/10/28}{Reset to empty if arg is empty}
%    \begin{macrocode}
\gdef\theorembodyfont#1{%
%    \end{macrocode}
%    We check if the argument supplied is empty and if so put nothing
%    into the |\theorem@bodyfont| token register to allow for
%    |\theorembodyfont{}| as a mean of using the default of the
%    current |\theoremstyle|.
%    \begin{macrocode}
   \def\@tempa{#1}%
   \ifx\@tempa\@empty
    \theorem@bodyfont{}%
   \else
    \theorem@bodyfont{\reset@font#1}%
   \fi
   }
\newtoks\theorem@bodyfont
\global\theorem@bodyfont{}
%    \end{macrocode}
% \end{macro}
% \end{macro}
%
%
% \begin{macro}{\theoremheaderfont}
% \changes{v2.0g}{89/09/19}{Macro added.}
%    The font for the theorem headers is handled differently because
%    this definition applies to all theorem styles.
%    \begin{macrocode}
\gdef\theoremheaderfont#1{\gdef\theorem@headerfont{#1}%
%    \end{macrocode}
%    After using the macro once it is redefined to produce an error
%    message.
%    \begin{macrocode}
       \gdef\theoremheaderfont##1{%
        \typeout{\string\theoremheaderfont\space should be used
                 only once.}}}
%    \end{macrocode}
% \end{macro}
%
% \begin{macro}{\theorem@headerfont}
% \changes{v2.0g}{89/09/19}{Macro added.}
%    To set the |\theorem@headerfont| default we first test if
%    the new fontselection scheme is in force.
% \changes{v2.1a}{90/02/15}{Support for new font selection.}
%    \begin{macrocode}
\ifx\upshape\undefined
%    \end{macrocode}
%    If not we define it to expand into |\bfseries|. We don't use
%    |\let| just in case a following style option redefines this
%    macro.
%    \begin{macrocode}
\gdef\theorem@headerfont{\bfseries}
%    \end{macrocode}
%    Otherwise we reset the current shape before calling |\bfseries|.
%    \begin{macrocode}
\else \gdef\theorem@headerfont{\normalfont\bfseries}\fi
%    \end{macrocode}
% \end{macro}
%
% \begin{macro}{\th@plain}
% \begin{macro}{\th@break}
% \begin{macro}{\th@marginbreak}
% \begin{macro}{\th@changebreak}
% \begin{macro}{\th@change}
% \begin{macro}{\th@margin}
%    The different styles are defined in macros such as
%    |\th@plain|.  Since memory space is precious in
%    ``non-Big-versions'', we have to avoid offering too many unused
%    definitions.  Therefore we define these styles in separate files
%    that can be loaded on demand.  Thus the commands themselves only
%    load these files. We use |\@input@|, a \LaTeXe{} internal command
%    that ensures that the file will be listed with |\listfiles|.
% \changes{v2.2a}{94/02/02}{Use `@input@ to load theorem layout files}
%    \begin{macrocode}
\gdef\th@plain{\@input@{thp.sty}}
\gdef\th@break{\@input@{thb.sty}}
\gdef\th@marginbreak{\@input@{thmb.sty}}
\gdef\th@changebreak{\@input@{thcb.sty}}
\gdef\th@change{\@input@{thc.sty}}
\gdef\th@margin{\@input@{thm.sty}}
%    \end{macrocode}
%    This list will be expanded when new styles become
%    available.  For testing,  just append new theorem substyles
%    as document options.
% \end{macro}
% \end{macro}
% \end{macro}
% \end{macro}
% \end{macro}
% \end{macro}
%
%
%
% \subsection{Definition of a new theorem set}
%
% As already pointed out, a new theorem environment can be defined in
% three different ways:
% \begin{verbatim}
%    \newtheorem{Lem}{Lemma}
%    \newtheorem{Lem}{Lemma}[section]
%    \newtheorem{Lem}[Theorem]{Lemma}
%\end{verbatim}
% The function of the macro |\newtheorem| is to recognize these cases
% and then to branch into one of the three macros |\@ynthm|, |\@xnthm|
% or |\@othm|.  This mechanism is adopted unchanged from
% \cite{latex:source}; the essential point here is that, for example,
% in the second case, the arguments \texttt{Lem}, \texttt{Lemma} and
% \texttt{section} are passed over to the macro |\@xnthm|.
%
% We inspect this case first because the others present fewer problems,
% and thus are easily derived from this one.
%
% \begin{macro}{\@xnthm}
%    For our example arguments, the macro |\@xnthm| must fulfill
%    the following:
%    \begin{itemize}
%       \item
%          Define a new \LaTeX-counter `Lem'
%       \item
%          reset this counter within a |\section|
%       \item
%          define the macro |\theLem|
%       \item
%          define the environment macros |\Lem| and
%          |\endLem| using the current
%          |\theoremstyle| and |\theorem@bodyfont|.
%    \end{itemize}
%    Obviously, all this should happen only if the first argument
%    of |\@xnthm| (i.e.\ \texttt{Lem} in our example) is chosen so
%    as not to conflict with any previously defined commands or
%    environments.  This test is performed by the \LaTeX{} macro
%    |\@ifdefinable|.
%    \begin{macrocode}
\gdef\@xnthm#1#2[#3]{\expandafter\@ifdefinable\csname #1\endcsname
%    \end{macrocode}
%    Therefore, the first argument of |\@ifdefinable| is the
%    expansion (in the example, |\Lem|) of
%    |\csname||#1\endcsname|.  The second argument is
%    executed only if the test has been completed successfully.
%    \begin{macrocode}
   {%
%    \end{macrocode}
%    Now we define the new counter. The names of the \LaTeX{}
%    macros employed should speak for themselves:
% \changes{v2.2b}{1995/10/07}{Use \cs{@newctr}}
% \changes{v2.2c}{1995/11/23}{Use \cs{@newctr} properly.}
%    \begin{macrocode}
    \@definecounter{#1}\@newctr{#1}[#3]%
%    \end{macrocode}
%    Using |\@newctr| will give a proper error message if the counter
%    in |#3| is not defined.
%    In defining `|\theLem|' we must generate the desired macro
%    name by use of |\expandafter| and |\csname|.
%    \begin{macrocode}
    \expandafter\xdef\csname the#1\endcsname
%    \end{macrocode}
%    An |\xdef| is used in order to make the definition global,
%    and to ensure that it contains the replacement texts of
%    |\@thmcountersep| and |\@thmcounter|.\footnote{These
%    two macros can be defined by the document style.  Their default
%    values produce a `\texttt{.}' as separation and an arabic
%    representation of the number.} However, not everything should be
%    expanded. For example, it saves space to use |\thesection|
%    instead of its---at times---lengthy expansion.
%    \begin{macrocode}
      {\expandafter \noexpand \csname the#3\endcsname
       \@thmcountersep \@thmcounter{#1}}%
%    \end{macrocode}
%    Thus with the defaults of \LaTeX{}, |\theLem| would be
%    replaced by the command sequence |\thesection.\arabic{Lem}|.
%
%    We will now look at the definition of the macro which is executed
%    at the beginning of the actual environment (in our example this
%    macro is |\Lem|). It should be noted that we use an
%    ``|\expandafter| trick'' to expand only certain parts of the
%    replacement text at the time of the definition.
%    \begin{macrocode}
    \def\@tempa{\global\@namedef{#1}}%
    \expandafter \@tempa \expandafter{%
%    \end{macrocode}
%    First, the macro that contains the current definitions of
%    |\@begintheorem| and |\@opargtheorem| should be called
%    up.  The name of this macro---as is already known---has the form
%    |\th@|\meta{theorem\ style}; therefore, it must be called by
%    \begin{macrocode}
      \csname th@\the \theorem@style
%    \end{macrocode}
%    In addition the default theorem font should be changeable, i.e.\
%    we have to insert the contents of |\theorem@bodyfont|. For
%    that reason, we expand even further, beyond |\endcsname|,
%    and thus insert the contents of the \textsf{token} register
%    |\theorem@bodyfont| in the replacement text.
%    \begin{macrocode}
            \expandafter \endcsname \the \theorem@bodyfont
%    \end{macrocode}
%    Now it is time to call the macro |\@thm| which takes over
%    the further processing.  It has two arguments: the current
%    counter name (in our example, \texttt{Lem}), and the text of the
%    label (in our example, \texttt{Lemma}).
%    \begin{macrocode}
     \@thm{#1}{#2}}%
%    \end{macrocode}
%    With this, the `sub-definition' is complete.  The macro
%    |\@endtheorem| ends a theorem environment and is, so far, nothing
%    but an |\endtrivlist|.  (Hence it is defined globally, and not
%    within the theorem styles.\footnote{This has to be changed as
%    soon as theorem styles that change \texttt{\bslash @endtheorem}
%    exist.  In such a case, all existing styles must be changed as
%    well since they will have to reset the macro.}) Therefore, we can
%    set it equivalent to the macro that ends the theorem set (in our
%    example, |\endLem|).  However, if some day theorem styles exist
%    that do change |\@endtheorem|, we would have to use the
%    commented-out line instead.
%    \begin{macrocode}
    \global \expandafter \let \csname end#1\endcsname \@endtheorem
%   \global\@namedef{end#1}{\@endtheorem}%
%    \end{macrocode}
%    With these commands all the required definitions are employed,
%    unless the test |\@ifdefinable| has failed.  Therefore, we
%    end the second argument of this macro and with it the definition
%    of |\@xnthm|.
%    \begin{macrocode}
   }}
%    \end{macrocode}
% \end{macro}
%
% \begin{macro}{\@ynthm}
%    The definition of |\@ynthm| is completely analogous. In
%    this case the new counter that is defined is not reset within
%    another counter; thus the definition of |\the...|
%    is simplified:
%    \begin{macrocode}
\gdef\@ynthm#1#2{\expandafter\@ifdefinable\csname #1\endcsname
   {\@definecounter{#1}%
    \expandafter\xdef\csname the#1\endcsname{\@thmcounter{#1}}%
%    \end{macrocode}
%    The rest of the definition corresponds literally to that of
%    |\@xnthm|:
%    \begin{macrocode}
    \def\@tempa{\global\@namedef{#1}}\expandafter \@tempa
     \expandafter{\csname th@\the \theorem@style \expandafter
     \endcsname \the\theorem@bodyfont \@thm{#1}{#2}}%
    \global \expandafter \let \csname end#1\endcsname \@endtheorem}}
%    \end{macrocode}
% \end{macro}
%
% \begin{macro}{\@othm}
%    The definition of |\@othm| does not contain anything new.
%    \begin{macrocode}
\gdef\@othm#1[#2]#3{%
%    \end{macrocode}
% \changes{v2.2b}{95/10/07}{Build in check for counter defined}
%    We do not define a new counter but instead use one that has
%    already been defined.  Thus the only definition we need is that
%    of this pseudo-counter (i.e.\ |\the|\meta{env.\ name}).
%    First we check if |#2| corresponds to a known counter name.
%    \begin{macrocode}
  \expandafter\ifx\csname c@#2\endcsname\relax
   \@nocounterr{#2}%
  \else
   \expandafter\@ifdefinable\csname #1\endcsname
   {\expandafter \xdef \csname the#1\endcsname
     {\expandafter \noexpand \csname the#2\endcsname}%
%    \end{macrocode}
%    All other parts of the definition can be adopted from
%    |\@xnthm|. We have to remember, though, that in this case
%    the name of the current counter and the theorem label have moved
%    to the second and third arguments.
%    \begin{macrocode}
    \def\@tempa{\global\@namedef{#1}}\expandafter \@tempa
     \expandafter{\csname th@\the \theorem@style \expandafter
     \endcsname \the\theorem@bodyfont \@thm{#2}{#3}}%
    \global \expandafter \let \csname end#1\endcsname \@endtheorem}%
  \fi}
%    \end{macrocode}
% \end{macro}
%
% \subsection{Macros that are employed in a {\envfont theorem}
%    environment}
%
% \begin{macro}{\@thm}
%    The macro |\@thm| has to increase the current counter.  Then,
%    depending on whether the environment has (or does not have) an
%    optional argument, it has to branch into either |\@begintheorem|
%    or |\@opargtheorem|.
%    \begin{macrocode}
\gdef\@thm#1#2{\refstepcounter{#1}%
%    \end{macrocode}
%    Now we start a {\envfont trivlist} environment, and give
%    |\@topsep| and |\@topsepadd| the values of the \textsf{skip}
%    registers |\theorempreskipamount| and |\theorempostskipamount|.
%    The value in |\@topsep| is the vertical space that is inserted by
%    the first (and only) |\item| in our |\trivlist| whilst
%    |\@topsepadd| is inserted by |\@endparenv| at the end of that
%    {\envfont trivlist} environment.  By using these registers, we
%    obtain the desired space around a {\envfont theorem} environment.
%    \begin{macrocode}
   \trivlist
   \@topsep \theorempreskipamount               % used by first \item
   \@topsepadd \theorempostskipamount           % used by \@endparenv
%    \end{macrocode}
%    Now we have to test whether an optional argument has been given.
% \changes{v2.0d}{89/07/18}{Forgotten percent added.}
%    \begin{macrocode}
   \@ifnextchar [%
%    \end{macrocode}
%    If there is an optional argument, we will call |\@ythm|, and
%    move the arguments read back into the input stream.
%    \begin{macrocode}
   {\@ythm{#1}{#2}}%
%    \end{macrocode}
%    If not, we call |\@begintheorem|. Its first argument is the
%    name of the theorem set (hence the second argument of
%    |\@thm|). Its second argument is the macro that produces the
%    current number.
%    \begin{macrocode}
   {\@begintheorem{#2}{\csname the#1\endcsname}\ignorespaces}}
%    \end{macrocode}
% \end{macro}
%
% \begin{macro}{\@xthm}
% \begin{macro}{\@ythm}
%    Both these macros were originally called by |\@thm|.  We do
%    not need |\@xthm| anymore, hence we reset it to
%    |\relax|. The definition of |\@ythm| has not changed at
%    all from its definition in \LaTeX. In order to make the macros
%    easier to understand, we will
%    nevertheless present it (commented out).
%    \begin{macrocode}
\global\let\@xthm\relax
% \def\@ythm#1#2[#3]{\@opargbegintheorem{#2}{\csname
%           the#1\endcsname}{#3}\ignorespaces}
%</package>
%    \end{macrocode}
%    The primitive |\ignorespaces| in |\@ythm| and
%    |\@thm| is needed to remove the spaces between the
%    |\begin{...}| and the actual text.
% \end{macro}
% \end{macro}
%
%
% \subsection{Definition of the theorem substyles}
%
% As already pointed out, the theorem substyles, defined below, are
% only loaded when necessary.  Note that all these substyles, except
% \texttt{plain}, have |\slshape| as the default body font.
%
% \subsubsection{The \texttt{plain} style}
%
% \label{dok:th@plain} As the following macros use |@|, we have
% to locally set the |\catcode| of this symbol to ``letter''.
% This happens within a group, so that we do not have to worry about
% which |\catcode| that symbol had before.
%    \begin{macrocode}
%<*thp>
\begingroup \makeatletter
%    \end{macrocode}
% Since we are now within a group, we must make all definitions
% globally.  First we make sure that \texttt{theorem.sty} is loaded.
% This will allow us to use this file as a document style option
% without having to call \texttt{theorem} itself as an option.  At the
% same time, we assure that at least version 2 is loaded, since
% |\theorem@style| was not defined in earlier versions.
%    \begin{macrocode}
\@ifundefined{theorem@style}{\input{theorem.sty}}{}
\ProvidesFile{thp.sty}
             [\FMithmInfo]
%    \end{macrocode}
%
%
% \begin{macro}{\th@plain}
% \begin{macro}{\@begintheorem}
% \begin{macro}{\@opargbegintheorem}
%    |\theoremstyle{plain}| corresponds to the original
%    definition, except that the distances to the surrounding text are
%    determined by the parameters |\theorempreskipamount| and
%    |\theorempostskipamount|.
%    First we set the default body font.
%    \begin{macrocode}
\gdef\th@plain{\normalfont\itshape
%    \end{macrocode}
%    Then we define |\@begintheorem| and
%    |\@opargbegintheorem|. These two macros define how the
%    header of a theorem is typeset.  |\@opargbegintheorem| will
%    be called if a \textsf{theorem} environment with an optional
%    argument is encountered; otherwise, the header is constructed by
%    calling |\@begintheorem|.  If one of these macros is
%    executed, we are within a \textsf{trivlist} environment started by
%    |\@thm|.  So the theorem header is produced with an
%    |\item| command.
%
%    Instead of specifying the header font directly, all standard
%    theorem styles use the |\theorem@headerfont| macro to allow
%    customization.  The extra space (|\labelsep|) is necessary
%    because of problems in the \textsf{trivlist} environment.
%    \begin{macrocode}
  \def\@begintheorem##1##2{%
        \item[\hskip\labelsep \theorem@headerfont ##1\ ##2]}%
%    \end{macrocode}
%    The definition of |\@opargbegintheorem| is completely
%    analogous. The only difference is the fact that there exists a
%    third argument (which is the optional parameter of the
%    environment and contains additional information about the
%    theorem).  Customarily we enclose it in parentheses.
%    \begin{macrocode}
\def\@opargbegintheorem##1##2##3{%
   \item[\hskip\labelsep \theorem@headerfont ##1\ ##2\ (##3)]}}
%    \end{macrocode}
% \end{macro}
% \end{macro}
% \end{macro}
%
% \noindent
% We conclude with an |\endgroup| to restore the
% |\catcode| of |@|.
%    \begin{macrocode}
\endgroup
%</thp>
%    \end{macrocode}
%
%
% \subsubsection{The \texttt{break} style}
%
% This style option is stored in the file |thb.sty|.
% For the next two lines see the documentation for |\th@plain| on
% page~\pageref{dok:th@plain}.
%    \begin{macrocode}
%<*thb>
\begingroup \makeatletter
\@ifundefined{theorem@style}{\input{theorem.sty}}{}
\ProvidesFile{thb.sty}
             [\FMithmInfo]
%    \end{macrocode}
%
% \begin{macro}{\th@break}
%    |\theoremstyle{break}| produces a line break after the name
%    of the theorem.  The font is |\slshape|.  Hence, we define
%    |\th@break| as follows:
%    \begin{macrocode}
\gdef\th@break{\normalfont\slshape
  \def\@begintheorem##1##2{\item[%
%    \end{macrocode}
%    We run into the following problem: it is not possible to create
%    the header with |\item[|\meta{title}|]| and then start
%    a new line by, for example, |\mbox{}\\|. Such a definition
%    will fail whenever a list environment follows immediately.  With
%    the above construction, the |\mbox{}| causes the switch
%    |@inlabel| (cf.\ definition of |\list| and
%    |\trivlist| in \cite{latex:source}) to be set to \texttt{false}
%    and so the following list will insert additional vertical
%    space (|\topskip|). This is quite annoying. Therefore, we
%    create the line break within the |\item|. In order to ensure
%    that the text will begin at the proper position in the following
%    line, we simply pretend that the label does not take any
%    room.\footnote{This will lead to problems whenever very high
%    symbols occurring in the line tower into the
%    heading. So, something else has to be done here sometime.}
%    \begin{macrocode}
       \rlap{\vbox{\hbox{\hskip \labelsep\theorem@headerfont ##1\ ##2}%
                   \hbox{\strut}}}]}%
%    \end{macrocode}
%    Again, the definition of |\@opargbegintheorem| is
%    completely analogous.
%    \begin{macrocode}
\def\@opargbegintheorem##1##2##3{%
  \item[\rlap{\vbox{\hbox{\hskip \labelsep \theorem@headerfont
                     ##1\ ##2\ (##3)}%
                    \hbox{\strut}}}]}}
\endgroup
%</thb>
%    \end{macrocode}
% \end{macro}
%
%
% \subsubsection{The \texttt{changebreak} style}
%
%    \begin{macrocode}
% This style option is stored in the file |thcb.sty|.
%    \begin{macrocode}
%<*thcb>
\begingroup \makeatletter
\@ifundefined{theorem@style}{\input{theorem.sty}}{}
\ProvidesFile{thcb.sty}
             [\FMithmInfo]
%    \end{macrocode}
%
% \begin{macro}{\th@changebreak}
%    The \texttt{change-break} theorem style is like \texttt{break}
%    but with interchange of theorem name and theorem number.  Thus we
%    define |\th@changebreak| as follows:
%    \begin{macrocode}
\gdef\th@changebreak{\normalfont\slshape
  \def\@begintheorem##1##2{\item
       [\rlap{\vbox{\hbox{\hskip\labelsep \theorem@headerfont ##2\ ##1}%
                    \hbox{\strut}}}]}%
\def\@opargbegintheorem##1##2##3{%
  \item[\rlap{\vbox{\hbox{\hskip\labelsep \theorem@headerfont
                     ##2\ ##1\ (##3)}%
                    \hbox{\strut}}}]}}
\endgroup
%</thcb>
%    \end{macrocode}
% \end{macro}
%
%
% \subsubsection{The \texttt{change} style}
%
% This style option is stored in the file |thc.sty|.
%    \begin{macrocode}
%<*thc>
\begingroup \makeatletter
\@ifundefined{theorem@style}{\input{theorem.sty}}{}
\ProvidesFile{thc.sty}
             [\FMithmInfo]
%    \end{macrocode}
%
% \begin{macro}{\th@change}
%    The \texttt{change} theorem style corresponds to the
%    \texttt{change break} style without a linebreak after the header.
%    To say it in another way, it's the same as the \texttt{plain}
%    style but with number and name interchanged and |\slshape| as the
%    default font.
%    \begin{macrocode}
\gdef\th@change{\normalfont\slshape
  \def\@begintheorem##1##2{\item
       [\hskip\labelsep \theorem@headerfont ##2\ ##1]}%
\def\@opargbegintheorem##1##2##3{%
  \item[\hskip\labelsep \theorem@headerfont ##2\ ##1\ (##3)]}}
\endgroup
%</thc>
%    \end{macrocode}
% \end{macro}
%
%
% \subsubsection{The \texttt{marginbreak} style}
%
% This style option is the one used most often at Mainz.  It is saved
% in the file |thmb.sty|.
%    \begin{macrocode}
%<*thmb>
\begingroup \makeatletter
\@ifundefined{theorem@style}{\input{theorem.sty}}{}
\ProvidesFile{thmb.sty}
             [\FMithmInfo]
%    \end{macrocode}
%
% \begin{macro}{\th@marginbreak}
%    The \texttt{margin break} style is nearly the same as the
%    \texttt{change break} style. The only difference is the placement
%    of the theorem number.  We use |\llap| to place it in the left
%    margin.
%
%    In this style |\labelsep| denotes the separation between the
%    number and the text.
%    \begin{macrocode}
\gdef\th@marginbreak{\normalfont\slshape
  \def\@begintheorem##1##2{\item
       [\rlap{\vbox{\theorem@headerfont
                    \hbox{\llap{##2}\hskip\labelsep ##1}%
                    \hbox{\strut}}}]}%
\def\@opargbegintheorem##1##2##3{%
  \item[\rlap{\vbox{\theorem@headerfont
                    \hbox{\llap{##2}\hskip\labelsep ##1\ (##3)}%
                    \hbox{\strut}}}]}}
\endgroup
%</thmb>
%    \end{macrocode}
% \changes{v2.1c}{90/02/17}{`endgroup outside macrocode env.}
% \end{macro}
%
%
%
% \subsubsection{The \texttt{margin} style}
%
% This style option is stored in the file |thm.sty|.
%    \begin{macrocode}
%<*thm>
\begingroup \makeatletter
\@ifundefined{theorem@style}{\input{theorem.sty}}{}
\ProvidesFile{thm.sty}
             [\FMithmInfo]
%    \end{macrocode}
%
% \begin{macro}{\th@margin}
%    Again this is only a variant of the theorem styles described
%    above without any new ideas.
%    \begin{macrocode}
\gdef\th@margin{\normalfont\slshape
  \def\@begintheorem##1##2{\item
       [\theorem@headerfont \llap{##2}\hskip\labelsep ##1]}%
\def\@opargbegintheorem##1##2##3{%
  \item[\theorem@headerfont \llap{##2}\hskip\labelsep ##1\ (##3)]}}
\endgroup
%</thm>
%    \end{macrocode}
% \end{macro}
%
%
%
%
% \subsection{Final Definitions}
%
% \begin{macro}{\theorempreskipamount}
% \begin{macro}{\theorempostskipamount}
%    The \textsf{skip} parameters that regulate the vertical empty space
%    before and after the {\envfont theorem} environment have
%    to be allocated as well.
% \changes{v2.0f}{89/07/20}{Unnecessary `global' removed.}
%    \begin{macrocode}
%<*package>
\newskip\theorempreskipamount
\newskip\theorempostskipamount
%    \end{macrocode}
%    Since we have used the same values for all theorem sets, we now can
%    assign them.
%    \begin{macrocode}
\global\setlength\theorempreskipamount{12pt plus 5pt minus 3pt}
\global\setlength\theorempostskipamount{8pt plus 3pt minus 1.5pt}
%    \end{macrocode}
% \end{macro}
% \end{macro}
%
%
% \begin{macro}{\@endtheorem}
%    The same holds for the macro |\@endtheorem|, which ends a
%    {\envfont theorem} environment. Since it is the same for all
%    theorem sets, it is removed from the macros |\th@|\meta{style}.
%    It simply ends the {\envfont trivlist} environment, which was
%    begun in |\@thm|.
%    \begin{macrocode}
\global\let\@endtheorem=\endtrivlist
%    \end{macrocode}
% \end{macro}
%
%
%
% \begin{macro}{\@preamblecmds}
%    All macros defined above are to be used only in the preamble.
%    Therefore, we insert them in |\@preamblecmds| which will disable
%    them at begin document. This is done by the internal \LaTeXe{}
%    command |\@onlypreamble|.
% \changes{v2.0g}{89/09/19}{`theoremheaderfont and `theorembodyfont
%    added.}
% \changes{v2.2a}{94/02/03}{Use \LaTeXe{} interface}
%    \begin{macrocode}
\@onlypreamble\@xnthm
\@onlypreamble\@ynthm
\@onlypreamble\@othm
\@onlypreamble\newtheorem
\@onlypreamble\theoremstyle
\@onlypreamble\theorembodyfont
\@onlypreamble\theoremheaderfont
%    \end{macrocode}
% \end{macro}
%
% Finally we declare the \texttt{plain} theorem style to be the default.
% \changes{v2.1b}{90/06/04}{Fixed missing `item by declaring default
%    style.}
%    \begin{macrocode}
\theoremstyle{plain}
%</package>
%    \end{macrocode}
%
% \Finale

\endinput
}{}
\ProvidesFile{thm.sty}
             [\FMithmInfo]
%    \end{macrocode}
%
% \begin{macro}{\th@margin}
%    Again this is only a variant of the theorem styles described
%    above without any new ideas.
%    \begin{macrocode}
\gdef\th@margin{\normalfont\slshape
  \def\@begintheorem##1##2{\item
       [\theorem@headerfont \llap{##2}\hskip\labelsep ##1]}%
\def\@opargbegintheorem##1##2##3{%
  \item[\theorem@headerfont \llap{##2}\hskip\labelsep ##1\ (##3)]}}
\endgroup
%</thm>
%    \end{macrocode}
% \end{macro}
%
%
%
%
% \subsection{Final Definitions}
%
% \begin{macro}{\theorempreskipamount}
% \begin{macro}{\theorempostskipamount}
%    The \textsf{skip} parameters that regulate the vertical empty space
%    before and after the {\envfont theorem} environment have
%    to be allocated as well.
% \changes{v2.0f}{89/07/20}{Unnecessary `global' removed.}
%    \begin{macrocode}
%<*package>
\newskip\theorempreskipamount
\newskip\theorempostskipamount
%    \end{macrocode}
%    Since we have used the same values for all theorem sets, we now can
%    assign them.
%    \begin{macrocode}
\global\setlength\theorempreskipamount{12pt plus 5pt minus 3pt}
\global\setlength\theorempostskipamount{8pt plus 3pt minus 1.5pt}
%    \end{macrocode}
% \end{macro}
% \end{macro}
%
%
% \begin{macro}{\@endtheorem}
%    The same holds for the macro |\@endtheorem|, which ends a
%    {\envfont theorem} environment. Since it is the same for all
%    theorem sets, it is removed from the macros |\th@|\meta{style}.
%    It simply ends the {\envfont trivlist} environment, which was
%    begun in |\@thm|.
%    \begin{macrocode}
\global\let\@endtheorem=\endtrivlist
%    \end{macrocode}
% \end{macro}
%
%
%
% \begin{macro}{\@preamblecmds}
%    All macros defined above are to be used only in the preamble.
%    Therefore, we insert them in |\@preamblecmds| which will disable
%    them at begin document. This is done by the internal \LaTeXe{}
%    command |\@onlypreamble|.
% \changes{v2.0g}{89/09/19}{`theoremheaderfont and `theorembodyfont
%    added.}
% \changes{v2.2a}{94/02/03}{Use \LaTeXe{} interface}
%    \begin{macrocode}
\@onlypreamble\@xnthm
\@onlypreamble\@ynthm
\@onlypreamble\@othm
\@onlypreamble\newtheorem
\@onlypreamble\theoremstyle
\@onlypreamble\theorembodyfont
\@onlypreamble\theoremheaderfont
%    \end{macrocode}
% \end{macro}
%
% Finally we declare the \texttt{plain} theorem style to be the default.
% \changes{v2.1b}{90/06/04}{Fixed missing `item by declaring default
%    style.}
%    \begin{macrocode}
\theoremstyle{plain}
%</package>
%    \end{macrocode}
%
% \Finale

\endinput
}{}
\ProvidesFile{thmb.sty}
             [\FMithmInfo]
%    \end{macrocode}
%
% \begin{macro}{\th@marginbreak}
%    The \texttt{margin break} style is nearly the same as the
%    \texttt{change break} style. The only difference is the placement
%    of the theorem number.  We use |\llap| to place it in the left
%    margin.
%
%    In this style |\labelsep| denotes the separation between the
%    number and the text.
%    \begin{macrocode}
\gdef\th@marginbreak{\normalfont\slshape
  \def\@begintheorem##1##2{\item
       [\rlap{\vbox{\theorem@headerfont
                    \hbox{\llap{##2}\hskip\labelsep ##1}%
                    \hbox{\strut}}}]}%
\def\@opargbegintheorem##1##2##3{%
  \item[\rlap{\vbox{\theorem@headerfont
                    \hbox{\llap{##2}\hskip\labelsep ##1\ (##3)}%
                    \hbox{\strut}}}]}}
\endgroup
%</thmb>
%    \end{macrocode}
% \changes{v2.1c}{90/02/17}{`endgroup outside macrocode env.}
% \end{macro}
%
%
%
% \subsubsection{The \texttt{margin} style}
%
% This style option is stored in the file |thm.sty|.
%    \begin{macrocode}
%<*thm>
\begingroup \makeatletter
\@ifundefined{theorem@style}{% \iffalse meta-comment
%
% Copyright (C) 1993-2024
%
% The LaTeX Project and any individual authors listed elsewhere
% in this file.
%
% This file is part of the Standard LaTeX `Tools Bundle'.
% -------------------------------------------------------
%
% It may be distributed and/or modified under the
% conditions of the LaTeX Project Public License, either version 1.3c
% of this license or (at your option) any later version.
% The latest version of this license is in
%    https://www.latex-project.org/lppl.txt
% and version 1.3c or later is part of all distributions of LaTeX
% version 2005/12/01 or later.
%
% The list of all files belonging to the LaTeX `Tools Bundle' is
% given in the file `manifest.txt'.
%
% \fi
%
%%
%
%
\def\FMithmInfo{2023/07/05 v2.2c Theorem extension package (FMi)}
%
% \ProvidesFile{theorem.dtx}[\FMithmInfo]
%\iffalse   % this is a METACOMMENT !
%%
%% Package `theorem' to use with LaTeX2e
%% Copyright (C) 1989-1995 Frank Mittelbach, all rights reserved.
%<+package>\@ifundefined{theorem@style}{}{\endinput}
%<+package>\NeedsTeXFormat{LaTeX2e}[1995/06/01]
%<+package>\ProvidesPackage{theorem}
%<+package>         [\FMithmInfo]
%
% \fi
%
% \changes{v1.0}{87/05/07}{First draft.}
% \changes{v1.0a}{87/05/07}{`thistheoremstyle' added.}
% \changes{v1.1a}{87/07/29}{New concept: redefine the internal
%                          \protect\LaTeX{} commands `@xnthm' etc.}
% \changes{v1.1b}{87/07/31}{`theoremfont' used to specify font.}
% \changes{v1.2a}{88/01/08}{`@endtheorem' now globally defined.}
% \changes{v1.2b}{88/01/11}{`newtheorem' only allowed in the preamble.}
% \changes{v1.3a}{88/02/14}{`theoremfont' now a toks register.}
%
% \changes{v2.0a}{88/11/16}{New implementation and new documentation.}
% \changes{v2.0a}{88/11/16}{No layout changes but internal modifications
%                           old substyles won't work.}
% \changes{v2.0b}{89/01/30}{`global' used for all definitions.}
% \changes{v2.0c}{89/06/15}{Documentation translated to English.
%                            Thanks to LLL and Nina}
% \changes{v2.1a}{90/02/15}{Support for new font selection.}
% \changes{v2.2a}{94/02/03}{Cleanup for LaTeX2e}
%
% \DoNotIndex{\@endparenv,\@ifdefinable,\@ifnextchar,\@ifundefined}
% \DoNotIndex{\@namedef,\@nameuse,\@spaces,\@tempa,\@warning}
% \DoNotIndex{\ }
% \DoNotIndex{\begingroup,\bfseries}
% \DoNotIndex{\csname}
% \DoNotIndex{\def,\do,\docdate}
% \DoNotIndex{\endcsname,\endgroup,\endinput,\endtrivlist,\expandafter}
% \DoNotIndex{\filedate,\fileversion}
% \DoNotIndex{\gdef,\global}
% \DoNotIndex{\hbox,\hskip}
% \DoNotIndex{\ignorespaces,\input,\itshape,\item}
% \DoNotIndex{\labelsep,\let,\llap}
% \DoNotIndex{\makeatletter}
% \DoNotIndex{\noexpand}
% \DoNotIndex{\refstepcounter,\relax,\rlap}
% \DoNotIndex{\setlength,\slshape,\space,\string,\strut}
% \DoNotIndex{\the,\trivlist,\typeout}
% \DoNotIndex{\vbox}
% \DoNotIndex{\xdef}
%
% \def\deflabel#1{\ttfamily #1\hfill}
% \def\deflist#1{\begin{list}{}{\settowidth\labelwidth{\ttfamily #1}%
%                               \setlength\leftmargin\labelwidth
%                               \addtolength\leftmargin\labelsep
%                               \let\makelabel\deflabel}}
% \def\enddeflist{\end{list}}
% \def\envfont{\normalfont\ttfamily}
%
% \MakeShortVerb{\|}
%
% \newcommand{\TUB}{{\sl TUGboat\/}}
%
% \GetFileInfo{theorem.dtx}
% \title{An Extension of the
%        \LaTeX{} theorem environment\thanks{This file
%        has version number \fileversion, last
%        revised \filedate.}}
% \author{Frank Mittelbach}
%
% \MaintainedByLaTeXTeam{tools}
% \maketitle
%
% \begin{abstract}
%    The macros described in this paper yield an extension of the
%    \LaTeX{} theorem mechanism. It is designed to satisfy the
%    different requirements of various journals. Thus, the
%    layout of the ``theorems'' can be manipulated by determining a
%    ``style''. This article describes not only the use, but
%    also the definition, of the necessary macros.
% \end{abstract}
%
% \section*{Preface to version 2.2}
%
% For \LaTeXe{} this package did not need any fundamental changes. I
% only modified the messages generated so that theorem layout styles
% will show up with the |\listfiles| command and cleaned the section
% on the New Font Selection Scheme since this is now included in
% \LaTeX.
%
% \section*{Preface to version 2.1}
%
% This version is identical to 2.0g described in \TUB\ 10\#3 except
% for some internal defaults which are now set depending on the used
% font selection scheme.
%
% This was done to avoid unpleasant surprises if the new font selection
% scheme is in force. For further details see
% section~\ref{sec:newfonts} and \cite{bk:GMS94}.
%
% \section{Introduction}
%
% For our purposes here, ``theorems'' are labelled enunciations,
% often set off from the main text by extra space and a font change.
% Theorems, corollaries, conjectures, definitions, and remarks
% are all instances of ``theorems''.  The ``header'' of these structures
% is composed of a label (such as \textsc{Theorem} or \textsc{Remark})
% and a number which serializes an item in the sequence of items
% with the same label.
%
% Shortly after the introduction of \LaTeX{} at the Fachbereich
% Mathematik in Mainz, the desire to manipulate the layout of
% ``theorems'' arose.  In Mainz, the following two conventions came
% into general use:
% \begin{enumerate}
%   \item The number of the theorem is shown in the margin.
%   \item There is a line break at
%         the end of the theorem header.
% \end{enumerate}
%
% Additionally, some journals require different formats which depend on
% the ``sort of theorem'': e.g.\ often remarks and definitions are set
% in |\upshape|, while |\itshape| is employed for main theorems.
%
% Confronted with these requirements, a theorem environment was
% developed in Mainz which allows separate determination of the
% layout of the ``theorems sets'', comparable to |\pagestyle|.
%
%
% \section{The user interface}
% \subsection{Defining new theorem sets}
%
% \DescribeMacro\newtheorem
% As in the original \LaTeX{} version, the command |\newtheorem|
% defines a new ``theorem set'' or ``theorem-like structure''.
% Two required arguments name the new environment and give the text
% to be typeset with each instance of the new ``set'', while
% an optional argument determines how
% the ``set'' is enumerated:
% \begin{description}
%    \item[\ttfamily \bslash newtheorem\{foo\}\{bar\}]
%       The theorem set {\envfont foo} (whose name is \texttt{bar})
%       uses its own counter.
%    \item[\ttfamily \bslash newtheorem\{foo2\}{[foo]}\{bar2\}]
%       The theorem set {\envfont foo2} (printed name \texttt{bar2})
%       uses the same counter as the theorem set \texttt{foo}.
%    \item[\ttfamily \bslash newtheorem\{foo3\}\{bar3\}{[section]}]
%       The theorem set {\envfont foo3} (printed name \texttt{bar3}) is
%       enumerated within the counter \texttt{section}, i.e.\ with every
%       new |\section| the enumeration begins again with 1, and
%       the enumeration is composed from the section-number and the
%       theorem counter itself.
% \end{description}
%
% \DescribeMacro\theoremstyle
% Additionally, the command |\theoremstyle| can define the layout
% of various, or all, theorem sets.  It should be noted that any
% theorem set defined by |\newtheorem| is typeset in the
% |\theoremstyle| that is current at the time of the definition.
% Thus, the following
% \begin{verbatim}
%    \theoremstyle{break}        \newtheorem{Cor}{Corollary}
%    \theoremstyle{plain}        \newtheorem{Exa}{Example}[section]
%\end{verbatim}                             ^^A No blank at line start
% leads to the result that the set {\envfont Cor} is formatted in the
% style \texttt{break}, while the set {\envfont Exa} and all the
% following ones are formatted in the style \texttt{plain}, unless
% another |\theoremstyle| follows.  Since the definitions installed by
% |\newtheorem| are global, one also can limit |\theoremstyle| locally
% by grouping braces.
%
% \DescribeMacro\theorembodyfont
% The choice of the font for the theorem body is completely
% independent of the chosen |\theoremstyle|; this has proven to
% be very advantageous.  For example,
% \begin{verbatim}
%   {\theorembodyfont{\upshape}          \newtheorem{Rem}{Remark}}
%\end{verbatim}
% defines a theorem set {\envfont Rem}, which will be set in
% |\upshape| in the current layout (which in our example is
% \texttt{plain}).  As with |\theoremstyle|, the |\theorembodyfont|
% chosen is that current at the time of |\newtheorem|.  If
% |\theorembodyfont| is not specified or one defines
% |\theorembodyfont{}|, then the font used will be that defined by the
% |\theoremstyle|.
%
% \DescribeMacro\theoremheaderfont It is also possible to customize
% the font used for the theorem headers.  This is, however, a global
% declaration, and therefore there should be at most one
% |\theoremheaderfont| declaration in the preamble.\footnote{If it is
% actually necessary to have different header fonts, one has to define
% new theorem styles (substituting the desired font) or specify the
% information directly in the \texttt{\bslash newtheorem} declaration
% (the unclean variant).}
%
% \DescribeMacro\theorempreskipamount
% \DescribeMacro\theorempostskipamount Two additional parameters
% affect the vertical space around the theorem environments:
% |\theorempreskipamount| and |\theorempostskipamount| define,
% respectively, the spacing before and after such an environment.
% These parameters apply for all theorem sets and can be manipulated
% with the ordinary length macros.  They are rubber lengths
% (`\textsf{skips}'), and therefore can contain \texttt{plus} and
% \texttt{minus} parts.
%
% Since the definition of theorem sets should---most sensibly---be
% placed in the preamble, we only allow installation there.
% It is therefore possible to release the memory used here after
% |\begin{document}|, in order to make room for other
% applications.
%
%
% \subsection{Existing theorem styles}
%
% The following theorem styles exist to date:
% \begin{deflist}{marginbreak}
%    \item[plain]
%       This theorem style emulates the original \LaTeX{} definition,
%       except that additionally the parameters
%       |\theorem...skipamount| are used.
%    \item[break]
%       In this style, the theorem header is followed by a line
%       break.
%    \item[marginbreak]
%       The theorem number is set in the margin, and there is a line
%       break as in \texttt{break}.
%    \item[changebreak]
%       Like \texttt{break}, but with header number and text
%       interchanged.
%    \item[change]
%       Header number and text are interchanged, without a line break.
%    \item[margin]
%       The number is set in the left margin, without a line break.
% \end{deflist}
% All styles (except \texttt{plain}) select |\slshape| as the default
% |\theorembodyfont|.
%
%
% \subsection{Examples}
%
% Given the above theorem sets {\envfont Cor}, {\envfont Exa}
% and {\envfont Rem},
% suppose that the preamble also contains the declarations:
% \begin{verbatim}
%    \theoremstyle{marginbreak}   \newtheorem{Lem}[Cor]{Lemma}
%    \theoremstyle{change}
%    \theorembodyfont{\itshape}        \newtheorem{Def}[Cor]{Definition}
%
%    \theoremheaderfont{\scshape}
%\end{verbatim}
% Then the following are some typical examples of the typeset output
% resulting from their use.
%
% \begin{Cor}
%    This is a sentence typeset in the theorem environment {\envfont
%    Cor}.
% \end{Cor}
%
% \begin{Exa}
%    This is a sentence typeset in the theorem environment {\envfont
%    Exa}.
% \end{Exa}
%
% \begin{Rem}
%    This is a sentence typeset in the theorem environment {\envfont
%    Rem}.
% \end{Rem}
%
% \begin{Lem}[Ben User]
%    This is a sentence typeset in the theorem environment {\envfont
%    Lem}.
% \end{Lem}
%
% \begin{Def}[Very impressive Definition]
%    This is a sentence typeset in the theorem environment {\envfont
%    Def}.
% \end{Def}
%
% The last two examples show the effect of the optional argument to
% a theorem environment (it is the text typeset in parentheses).
%
%
% \section{Special Considerations}
% \label{sec:newfonts}
%
% Theoremheader and body are implemented as a unit. This means that
% the |\theoremheaderfont|  will inherit characteristics of the
% |\theorembodyfont| in \LaTeXe. Thus, if for example
% |\theorembodyfont| is |\itshape| and |\theoremheaderfont|
% is |\bfseries| the font selected for the header will have the
% characteristics `bold extended italic'. If this is not desired one
% should set the |\theoremheaderfont| to something like
% \begin{verbatim}
%      \theoremheaderfont{\normalfont\bfseries}
%\end{verbatim}
% i.e.\ supplying all necessary font information explicitly.
%
%
%
% \section{Acknowledgements}
%
% The publication of this set of macros was only possible with the
% help of Christina Busse (translating the manuscript into English),
% Joachim Pense (playing the r\^ole of typist), Chris Rowley
% (looking everything over) and many others providing useful
% suggestions.
%
% \MaybeStop{%
%    \begin{thebibliography}{9}
%       \bibitem{bk:GMS94} \textsc{M.~Goossens}, \textsc{F.~Mittelbach}
%         and \textsc{A.~Samarin}.
%         \newblock The \LaTeX{} Companion.
%         \newblock
%         Addison-Wesley, Reading, Massachusetts, 1994.
%       \bibitem{latex:source} \textsc{Lamport, Leslie.} \newblock
%                               \texttt{latex.tex}, version 2.09, date
%                               Feb. 1990.
%    \end{thebibliography}
%    \PrintIndex
%    \PrintChanges}
%
%
% \section{The documentation driver file}
%
% The next bit of code contains the documentation driver file for
% \TeX{}, i.e., the file that will produce the documentation you are
% currently reading. It will be extracted from this file by the
% \texttt{docstrip} program. Since it is the first code in the file
% one can alternatively process this file directly with \LaTeXe{} to
% obtain the documentation.
%
%    \begin{macrocode}
%<*driver>
\documentclass{ltxdoc}

\usepackage{theorem}

%<+driver>% The next few lines define theorem sets which are used
%<+driver>% in the example section of the documentation.

\theoremstyle{break}       \newtheorem{Cor}{Corollary}
\theoremstyle{plain}       \newtheorem{Exa}{Example}[section]
{\theorembodyfont{\upshape}\newtheorem{Rem}{Remark}}
\theoremstyle{marginbreak} \newtheorem{Lem}[Cor]{Lemma}
\theoremstyle{change}
\theorembodyfont{\itshape} \newtheorem{Def}[Cor]{Definition}

\theoremheaderfont{\scshape}

\RecordChanges

\begin{document}
   \DocInput{theorem.dtx}
\end{document}
%</driver>
%    \end{macrocode}
%
% \section{Definition of the Macros}
%
% If the file has been loaded before, we abort immediately. If not the
% package announces itself (this is actually done at the very top of
% the file---the way it is done isn't good style so don't copy it).
% \changes{v2.0e}{89/07/19}{Spaces removed from `typeout'.}
% \changes{v2.2b}{95/11/19}{Announce moved to top of file}
%    \begin{macrocode}
%<*package>
%\@ifundefined{theorem@style}{}{\endinput}
%\def\FMithmInfo{1995/11/19 v2.2b Theorem extension package (FMi)}
%\ProvidesPackage{theorem}[\FMithmInfo]
%    \end{macrocode}
%
%
% \subsection{Definition of theorem styles and fonts}
%
% All the definitions in this file are done globally to allow
% inputting this file inside a group.
%
% \begin{macro}{\theoremstyle}
%    Before a theorem style can be installed, the chosen style must be
%    known. For that reason, we must test to see that
%    |\th@|\meta{style} is known or, more precisely, that it is
%    different from |\relax|. If the style is not known then
%    |\th@plain| is used.
%    \begin{macrocode}
\gdef\theoremstyle#1{%
   \@ifundefined{th@#1}{\@warning
          {Unknown theoremstyle `#1'. Using `plain'}%
          \theorem@style{plain}}%
%    \end{macrocode}
%    We save the theorem style to be used in the token register
%    |\theorem@style|.
%    \begin{macrocode}
      {\theorem@style{#1}}%
%    \end{macrocode}
%    Now we ``evaluate'' the theorem style: this means, we call the
%    macro |\th@|\meta{style} which will activate the relevant
%    definitions which are contained in a separate file.  This is done
%    in a group to suppress changes to the current font.  This could
%    otherwise pose problems together with the new font selection
%    scheme\footnote{When I printed the original article using the new
%    font selection scheme I ended with a document with slanted
%    typefaces (text headings and all) simply because one of the
%    theorem styles used \texttt{\bslash sl} at toplevel.} if the
%    |\th@|\meta{style} is evaluated a second time.
% \changes{v2.1a}{90/02/15}{Group added around `th@..}
% \changes{v2.1a}{90/02/15}{Reseting of `..@begintheorem removed.}
%    \begin{macrocode}
      \begingroup
        \csname th@\the\theorem@style \endcsname
      \endgroup}
%    \end{macrocode}
% \end{macro}
%
%
% \begin{macro}{\@begintheorem}
% \changes{v2.1a}{90/02/15}{Relaxed at toplevel.}
% \begin{macro}{\@opargbegintheorem}
% \changes{v2.1a}{90/02/15}{Relaxed at toplevel.}
%    We reset |\@begintheorem| and |\@opargbegintheorem| to
%    |\relax| since these commands are no longer necessary at
%    toplevel. This will save a few tokens.
%    \begin{macrocode}
\global\let\@begintheorem\relax
\global\let\@opargbegintheorem\relax
%    \end{macrocode}
% \end{macro}
% \end{macro}
%
% \begin{macro}{\theorem@style}
%    Obviously the \textsf{token} register used above has to be
%    allocated. To assure the utmost compatibility with the
%    original \LaTeX{} definition, we set the default theorem style to
%    \texttt{plain}, which implements the usual \LaTeX{} convention.
% \changes{v2.0f}{89/07/20}{Unnecessary `global' removed.}
%    \begin{macrocode}
\newtoks\theorem@style
\global\theorem@style{plain}
%    \end{macrocode}
% \end{macro}
%
% \begin{macro}{\theorembodyfont}
% \begin{macro}{\theorem@bodyfont}
%    If the |\theorembodyfont| is set by the user then it should not
%    interact with the default font set in the theorem style. When the
%    new font selection is in force this may happen if, for example,
%    the default is |\itshape| and the new |\theorembodyfont| is
%    |\sffamily|. So we add a |\reset@font| command in front of the
%    user definition.
% \changes{v2.0f}{89/07/20}{Unnecessary `global' removed.}
% \changes{v2.0g}{89/09/19}{`theoremfont to `theorembodyfont.}
% \changes{v2.1c}{92/06/22}{`reset@font added and logic changed.}
% \changes{v2.1d}{92/10/28}{Reset to empty if arg is empty}
%    \begin{macrocode}
\gdef\theorembodyfont#1{%
%    \end{macrocode}
%    We check if the argument supplied is empty and if so put nothing
%    into the |\theorem@bodyfont| token register to allow for
%    |\theorembodyfont{}| as a mean of using the default of the
%    current |\theoremstyle|.
%    \begin{macrocode}
   \def\@tempa{#1}%
   \ifx\@tempa\@empty
    \theorem@bodyfont{}%
   \else
    \theorem@bodyfont{\reset@font#1}%
   \fi
   }
\newtoks\theorem@bodyfont
\global\theorem@bodyfont{}
%    \end{macrocode}
% \end{macro}
% \end{macro}
%
%
% \begin{macro}{\theoremheaderfont}
% \changes{v2.0g}{89/09/19}{Macro added.}
%    The font for the theorem headers is handled differently because
%    this definition applies to all theorem styles.
%    \begin{macrocode}
\gdef\theoremheaderfont#1{\gdef\theorem@headerfont{#1}%
%    \end{macrocode}
%    After using the macro once it is redefined to produce an error
%    message.
%    \begin{macrocode}
       \gdef\theoremheaderfont##1{%
        \typeout{\string\theoremheaderfont\space should be used
                 only once.}}}
%    \end{macrocode}
% \end{macro}
%
% \begin{macro}{\theorem@headerfont}
% \changes{v2.0g}{89/09/19}{Macro added.}
%    To set the |\theorem@headerfont| default we first test if
%    the new fontselection scheme is in force.
% \changes{v2.1a}{90/02/15}{Support for new font selection.}
%    \begin{macrocode}
\ifx\upshape\undefined
%    \end{macrocode}
%    If not we define it to expand into |\bfseries|. We don't use
%    |\let| just in case a following style option redefines this
%    macro.
%    \begin{macrocode}
\gdef\theorem@headerfont{\bfseries}
%    \end{macrocode}
%    Otherwise we reset the current shape before calling |\bfseries|.
%    \begin{macrocode}
\else \gdef\theorem@headerfont{\normalfont\bfseries}\fi
%    \end{macrocode}
% \end{macro}
%
% \begin{macro}{\th@plain}
% \begin{macro}{\th@break}
% \begin{macro}{\th@marginbreak}
% \begin{macro}{\th@changebreak}
% \begin{macro}{\th@change}
% \begin{macro}{\th@margin}
%    The different styles are defined in macros such as
%    |\th@plain|.  Since memory space is precious in
%    ``non-Big-versions'', we have to avoid offering too many unused
%    definitions.  Therefore we define these styles in separate files
%    that can be loaded on demand.  Thus the commands themselves only
%    load these files. We use |\@input@|, a \LaTeXe{} internal command
%    that ensures that the file will be listed with |\listfiles|.
% \changes{v2.2a}{94/02/02}{Use `@input@ to load theorem layout files}
%    \begin{macrocode}
\gdef\th@plain{\@input@{thp.sty}}
\gdef\th@break{\@input@{thb.sty}}
\gdef\th@marginbreak{\@input@{thmb.sty}}
\gdef\th@changebreak{\@input@{thcb.sty}}
\gdef\th@change{\@input@{thc.sty}}
\gdef\th@margin{\@input@{thm.sty}}
%    \end{macrocode}
%    This list will be expanded when new styles become
%    available.  For testing,  just append new theorem substyles
%    as document options.
% \end{macro}
% \end{macro}
% \end{macro}
% \end{macro}
% \end{macro}
% \end{macro}
%
%
%
% \subsection{Definition of a new theorem set}
%
% As already pointed out, a new theorem environment can be defined in
% three different ways:
% \begin{verbatim}
%    \newtheorem{Lem}{Lemma}
%    \newtheorem{Lem}{Lemma}[section]
%    \newtheorem{Lem}[Theorem]{Lemma}
%\end{verbatim}
% The function of the macro |\newtheorem| is to recognize these cases
% and then to branch into one of the three macros |\@ynthm|, |\@xnthm|
% or |\@othm|.  This mechanism is adopted unchanged from
% \cite{latex:source}; the essential point here is that, for example,
% in the second case, the arguments \texttt{Lem}, \texttt{Lemma} and
% \texttt{section} are passed over to the macro |\@xnthm|.
%
% We inspect this case first because the others present fewer problems,
% and thus are easily derived from this one.
%
% \begin{macro}{\@xnthm}
%    For our example arguments, the macro |\@xnthm| must fulfill
%    the following:
%    \begin{itemize}
%       \item
%          Define a new \LaTeX-counter `Lem'
%       \item
%          reset this counter within a |\section|
%       \item
%          define the macro |\theLem|
%       \item
%          define the environment macros |\Lem| and
%          |\endLem| using the current
%          |\theoremstyle| and |\theorem@bodyfont|.
%    \end{itemize}
%    Obviously, all this should happen only if the first argument
%    of |\@xnthm| (i.e.\ \texttt{Lem} in our example) is chosen so
%    as not to conflict with any previously defined commands or
%    environments.  This test is performed by the \LaTeX{} macro
%    |\@ifdefinable|.
%    \begin{macrocode}
\gdef\@xnthm#1#2[#3]{\expandafter\@ifdefinable\csname #1\endcsname
%    \end{macrocode}
%    Therefore, the first argument of |\@ifdefinable| is the
%    expansion (in the example, |\Lem|) of
%    |\csname||#1\endcsname|.  The second argument is
%    executed only if the test has been completed successfully.
%    \begin{macrocode}
   {%
%    \end{macrocode}
%    Now we define the new counter. The names of the \LaTeX{}
%    macros employed should speak for themselves:
% \changes{v2.2b}{1995/10/07}{Use \cs{@newctr}}
% \changes{v2.2c}{1995/11/23}{Use \cs{@newctr} properly.}
%    \begin{macrocode}
    \@definecounter{#1}\@newctr{#1}[#3]%
%    \end{macrocode}
%    Using |\@newctr| will give a proper error message if the counter
%    in |#3| is not defined.
%    In defining `|\theLem|' we must generate the desired macro
%    name by use of |\expandafter| and |\csname|.
%    \begin{macrocode}
    \expandafter\xdef\csname the#1\endcsname
%    \end{macrocode}
%    An |\xdef| is used in order to make the definition global,
%    and to ensure that it contains the replacement texts of
%    |\@thmcountersep| and |\@thmcounter|.\footnote{These
%    two macros can be defined by the document style.  Their default
%    values produce a `\texttt{.}' as separation and an arabic
%    representation of the number.} However, not everything should be
%    expanded. For example, it saves space to use |\thesection|
%    instead of its---at times---lengthy expansion.
%    \begin{macrocode}
      {\expandafter \noexpand \csname the#3\endcsname
       \@thmcountersep \@thmcounter{#1}}%
%    \end{macrocode}
%    Thus with the defaults of \LaTeX{}, |\theLem| would be
%    replaced by the command sequence |\thesection.\arabic{Lem}|.
%
%    We will now look at the definition of the macro which is executed
%    at the beginning of the actual environment (in our example this
%    macro is |\Lem|). It should be noted that we use an
%    ``|\expandafter| trick'' to expand only certain parts of the
%    replacement text at the time of the definition.
%    \begin{macrocode}
    \def\@tempa{\global\@namedef{#1}}%
    \expandafter \@tempa \expandafter{%
%    \end{macrocode}
%    First, the macro that contains the current definitions of
%    |\@begintheorem| and |\@opargtheorem| should be called
%    up.  The name of this macro---as is already known---has the form
%    |\th@|\meta{theorem\ style}; therefore, it must be called by
%    \begin{macrocode}
      \csname th@\the \theorem@style
%    \end{macrocode}
%    In addition the default theorem font should be changeable, i.e.\
%    we have to insert the contents of |\theorem@bodyfont|. For
%    that reason, we expand even further, beyond |\endcsname|,
%    and thus insert the contents of the \textsf{token} register
%    |\theorem@bodyfont| in the replacement text.
%    \begin{macrocode}
            \expandafter \endcsname \the \theorem@bodyfont
%    \end{macrocode}
%    Now it is time to call the macro |\@thm| which takes over
%    the further processing.  It has two arguments: the current
%    counter name (in our example, \texttt{Lem}), and the text of the
%    label (in our example, \texttt{Lemma}).
%    \begin{macrocode}
     \@thm{#1}{#2}}%
%    \end{macrocode}
%    With this, the `sub-definition' is complete.  The macro
%    |\@endtheorem| ends a theorem environment and is, so far, nothing
%    but an |\endtrivlist|.  (Hence it is defined globally, and not
%    within the theorem styles.\footnote{This has to be changed as
%    soon as theorem styles that change \texttt{\bslash @endtheorem}
%    exist.  In such a case, all existing styles must be changed as
%    well since they will have to reset the macro.}) Therefore, we can
%    set it equivalent to the macro that ends the theorem set (in our
%    example, |\endLem|).  However, if some day theorem styles exist
%    that do change |\@endtheorem|, we would have to use the
%    commented-out line instead.
%    \begin{macrocode}
    \global \expandafter \let \csname end#1\endcsname \@endtheorem
%   \global\@namedef{end#1}{\@endtheorem}%
%    \end{macrocode}
%    With these commands all the required definitions are employed,
%    unless the test |\@ifdefinable| has failed.  Therefore, we
%    end the second argument of this macro and with it the definition
%    of |\@xnthm|.
%    \begin{macrocode}
   }}
%    \end{macrocode}
% \end{macro}
%
% \begin{macro}{\@ynthm}
%    The definition of |\@ynthm| is completely analogous. In
%    this case the new counter that is defined is not reset within
%    another counter; thus the definition of |\the...|
%    is simplified:
%    \begin{macrocode}
\gdef\@ynthm#1#2{\expandafter\@ifdefinable\csname #1\endcsname
   {\@definecounter{#1}%
    \expandafter\xdef\csname the#1\endcsname{\@thmcounter{#1}}%
%    \end{macrocode}
%    The rest of the definition corresponds literally to that of
%    |\@xnthm|:
%    \begin{macrocode}
    \def\@tempa{\global\@namedef{#1}}\expandafter \@tempa
     \expandafter{\csname th@\the \theorem@style \expandafter
     \endcsname \the\theorem@bodyfont \@thm{#1}{#2}}%
    \global \expandafter \let \csname end#1\endcsname \@endtheorem}}
%    \end{macrocode}
% \end{macro}
%
% \begin{macro}{\@othm}
%    The definition of |\@othm| does not contain anything new.
%    \begin{macrocode}
\gdef\@othm#1[#2]#3{%
%    \end{macrocode}
% \changes{v2.2b}{95/10/07}{Build in check for counter defined}
%    We do not define a new counter but instead use one that has
%    already been defined.  Thus the only definition we need is that
%    of this pseudo-counter (i.e.\ |\the|\meta{env.\ name}).
%    First we check if |#2| corresponds to a known counter name.
%    \begin{macrocode}
  \expandafter\ifx\csname c@#2\endcsname\relax
   \@nocounterr{#2}%
  \else
   \expandafter\@ifdefinable\csname #1\endcsname
   {\expandafter \xdef \csname the#1\endcsname
     {\expandafter \noexpand \csname the#2\endcsname}%
%    \end{macrocode}
%    All other parts of the definition can be adopted from
%    |\@xnthm|. We have to remember, though, that in this case
%    the name of the current counter and the theorem label have moved
%    to the second and third arguments.
%    \begin{macrocode}
    \def\@tempa{\global\@namedef{#1}}\expandafter \@tempa
     \expandafter{\csname th@\the \theorem@style \expandafter
     \endcsname \the\theorem@bodyfont \@thm{#2}{#3}}%
    \global \expandafter \let \csname end#1\endcsname \@endtheorem}%
  \fi}
%    \end{macrocode}
% \end{macro}
%
% \subsection{Macros that are employed in a {\envfont theorem}
%    environment}
%
% \begin{macro}{\@thm}
%    The macro |\@thm| has to increase the current counter.  Then,
%    depending on whether the environment has (or does not have) an
%    optional argument, it has to branch into either |\@begintheorem|
%    or |\@opargtheorem|.
%    \begin{macrocode}
\gdef\@thm#1#2{\refstepcounter{#1}%
%    \end{macrocode}
%    Now we start a {\envfont trivlist} environment, and give
%    |\@topsep| and |\@topsepadd| the values of the \textsf{skip}
%    registers |\theorempreskipamount| and |\theorempostskipamount|.
%    The value in |\@topsep| is the vertical space that is inserted by
%    the first (and only) |\item| in our |\trivlist| whilst
%    |\@topsepadd| is inserted by |\@endparenv| at the end of that
%    {\envfont trivlist} environment.  By using these registers, we
%    obtain the desired space around a {\envfont theorem} environment.
%    \begin{macrocode}
   \trivlist
   \@topsep \theorempreskipamount               % used by first \item
   \@topsepadd \theorempostskipamount           % used by \@endparenv
%    \end{macrocode}
%    Now we have to test whether an optional argument has been given.
% \changes{v2.0d}{89/07/18}{Forgotten percent added.}
%    \begin{macrocode}
   \@ifnextchar [%
%    \end{macrocode}
%    If there is an optional argument, we will call |\@ythm|, and
%    move the arguments read back into the input stream.
%    \begin{macrocode}
   {\@ythm{#1}{#2}}%
%    \end{macrocode}
%    If not, we call |\@begintheorem|. Its first argument is the
%    name of the theorem set (hence the second argument of
%    |\@thm|). Its second argument is the macro that produces the
%    current number.
%    \begin{macrocode}
   {\@begintheorem{#2}{\csname the#1\endcsname}\ignorespaces}}
%    \end{macrocode}
% \end{macro}
%
% \begin{macro}{\@xthm}
% \begin{macro}{\@ythm}
%    Both these macros were originally called by |\@thm|.  We do
%    not need |\@xthm| anymore, hence we reset it to
%    |\relax|. The definition of |\@ythm| has not changed at
%    all from its definition in \LaTeX. In order to make the macros
%    easier to understand, we will
%    nevertheless present it (commented out).
%    \begin{macrocode}
\global\let\@xthm\relax
% \def\@ythm#1#2[#3]{\@opargbegintheorem{#2}{\csname
%           the#1\endcsname}{#3}\ignorespaces}
%</package>
%    \end{macrocode}
%    The primitive |\ignorespaces| in |\@ythm| and
%    |\@thm| is needed to remove the spaces between the
%    |\begin{...}| and the actual text.
% \end{macro}
% \end{macro}
%
%
% \subsection{Definition of the theorem substyles}
%
% As already pointed out, the theorem substyles, defined below, are
% only loaded when necessary.  Note that all these substyles, except
% \texttt{plain}, have |\slshape| as the default body font.
%
% \subsubsection{The \texttt{plain} style}
%
% \label{dok:th@plain} As the following macros use |@|, we have
% to locally set the |\catcode| of this symbol to ``letter''.
% This happens within a group, so that we do not have to worry about
% which |\catcode| that symbol had before.
%    \begin{macrocode}
%<*thp>
\begingroup \makeatletter
%    \end{macrocode}
% Since we are now within a group, we must make all definitions
% globally.  First we make sure that \texttt{theorem.sty} is loaded.
% This will allow us to use this file as a document style option
% without having to call \texttt{theorem} itself as an option.  At the
% same time, we assure that at least version 2 is loaded, since
% |\theorem@style| was not defined in earlier versions.
%    \begin{macrocode}
\@ifundefined{theorem@style}{% \iffalse meta-comment
%
% Copyright (C) 1993-2024
%
% The LaTeX Project and any individual authors listed elsewhere
% in this file.
%
% This file is part of the Standard LaTeX `Tools Bundle'.
% -------------------------------------------------------
%
% It may be distributed and/or modified under the
% conditions of the LaTeX Project Public License, either version 1.3c
% of this license or (at your option) any later version.
% The latest version of this license is in
%    https://www.latex-project.org/lppl.txt
% and version 1.3c or later is part of all distributions of LaTeX
% version 2005/12/01 or later.
%
% The list of all files belonging to the LaTeX `Tools Bundle' is
% given in the file `manifest.txt'.
%
% \fi
%
%%
%
%
\def\FMithmInfo{2023/07/05 v2.2c Theorem extension package (FMi)}
%
% \ProvidesFile{theorem.dtx}[\FMithmInfo]
%\iffalse   % this is a METACOMMENT !
%%
%% Package `theorem' to use with LaTeX2e
%% Copyright (C) 1989-1995 Frank Mittelbach, all rights reserved.
%<+package>\@ifundefined{theorem@style}{}{\endinput}
%<+package>\NeedsTeXFormat{LaTeX2e}[1995/06/01]
%<+package>\ProvidesPackage{theorem}
%<+package>         [\FMithmInfo]
%
% \fi
%
% \changes{v1.0}{87/05/07}{First draft.}
% \changes{v1.0a}{87/05/07}{`thistheoremstyle' added.}
% \changes{v1.1a}{87/07/29}{New concept: redefine the internal
%                          \protect\LaTeX{} commands `@xnthm' etc.}
% \changes{v1.1b}{87/07/31}{`theoremfont' used to specify font.}
% \changes{v1.2a}{88/01/08}{`@endtheorem' now globally defined.}
% \changes{v1.2b}{88/01/11}{`newtheorem' only allowed in the preamble.}
% \changes{v1.3a}{88/02/14}{`theoremfont' now a toks register.}
%
% \changes{v2.0a}{88/11/16}{New implementation and new documentation.}
% \changes{v2.0a}{88/11/16}{No layout changes but internal modifications
%                           old substyles won't work.}
% \changes{v2.0b}{89/01/30}{`global' used for all definitions.}
% \changes{v2.0c}{89/06/15}{Documentation translated to English.
%                            Thanks to LLL and Nina}
% \changes{v2.1a}{90/02/15}{Support for new font selection.}
% \changes{v2.2a}{94/02/03}{Cleanup for LaTeX2e}
%
% \DoNotIndex{\@endparenv,\@ifdefinable,\@ifnextchar,\@ifundefined}
% \DoNotIndex{\@namedef,\@nameuse,\@spaces,\@tempa,\@warning}
% \DoNotIndex{\ }
% \DoNotIndex{\begingroup,\bfseries}
% \DoNotIndex{\csname}
% \DoNotIndex{\def,\do,\docdate}
% \DoNotIndex{\endcsname,\endgroup,\endinput,\endtrivlist,\expandafter}
% \DoNotIndex{\filedate,\fileversion}
% \DoNotIndex{\gdef,\global}
% \DoNotIndex{\hbox,\hskip}
% \DoNotIndex{\ignorespaces,\input,\itshape,\item}
% \DoNotIndex{\labelsep,\let,\llap}
% \DoNotIndex{\makeatletter}
% \DoNotIndex{\noexpand}
% \DoNotIndex{\refstepcounter,\relax,\rlap}
% \DoNotIndex{\setlength,\slshape,\space,\string,\strut}
% \DoNotIndex{\the,\trivlist,\typeout}
% \DoNotIndex{\vbox}
% \DoNotIndex{\xdef}
%
% \def\deflabel#1{\ttfamily #1\hfill}
% \def\deflist#1{\begin{list}{}{\settowidth\labelwidth{\ttfamily #1}%
%                               \setlength\leftmargin\labelwidth
%                               \addtolength\leftmargin\labelsep
%                               \let\makelabel\deflabel}}
% \def\enddeflist{\end{list}}
% \def\envfont{\normalfont\ttfamily}
%
% \MakeShortVerb{\|}
%
% \newcommand{\TUB}{{\sl TUGboat\/}}
%
% \GetFileInfo{theorem.dtx}
% \title{An Extension of the
%        \LaTeX{} theorem environment\thanks{This file
%        has version number \fileversion, last
%        revised \filedate.}}
% \author{Frank Mittelbach}
%
% \MaintainedByLaTeXTeam{tools}
% \maketitle
%
% \begin{abstract}
%    The macros described in this paper yield an extension of the
%    \LaTeX{} theorem mechanism. It is designed to satisfy the
%    different requirements of various journals. Thus, the
%    layout of the ``theorems'' can be manipulated by determining a
%    ``style''. This article describes not only the use, but
%    also the definition, of the necessary macros.
% \end{abstract}
%
% \section*{Preface to version 2.2}
%
% For \LaTeXe{} this package did not need any fundamental changes. I
% only modified the messages generated so that theorem layout styles
% will show up with the |\listfiles| command and cleaned the section
% on the New Font Selection Scheme since this is now included in
% \LaTeX.
%
% \section*{Preface to version 2.1}
%
% This version is identical to 2.0g described in \TUB\ 10\#3 except
% for some internal defaults which are now set depending on the used
% font selection scheme.
%
% This was done to avoid unpleasant surprises if the new font selection
% scheme is in force. For further details see
% section~\ref{sec:newfonts} and \cite{bk:GMS94}.
%
% \section{Introduction}
%
% For our purposes here, ``theorems'' are labelled enunciations,
% often set off from the main text by extra space and a font change.
% Theorems, corollaries, conjectures, definitions, and remarks
% are all instances of ``theorems''.  The ``header'' of these structures
% is composed of a label (such as \textsc{Theorem} or \textsc{Remark})
% and a number which serializes an item in the sequence of items
% with the same label.
%
% Shortly after the introduction of \LaTeX{} at the Fachbereich
% Mathematik in Mainz, the desire to manipulate the layout of
% ``theorems'' arose.  In Mainz, the following two conventions came
% into general use:
% \begin{enumerate}
%   \item The number of the theorem is shown in the margin.
%   \item There is a line break at
%         the end of the theorem header.
% \end{enumerate}
%
% Additionally, some journals require different formats which depend on
% the ``sort of theorem'': e.g.\ often remarks and definitions are set
% in |\upshape|, while |\itshape| is employed for main theorems.
%
% Confronted with these requirements, a theorem environment was
% developed in Mainz which allows separate determination of the
% layout of the ``theorems sets'', comparable to |\pagestyle|.
%
%
% \section{The user interface}
% \subsection{Defining new theorem sets}
%
% \DescribeMacro\newtheorem
% As in the original \LaTeX{} version, the command |\newtheorem|
% defines a new ``theorem set'' or ``theorem-like structure''.
% Two required arguments name the new environment and give the text
% to be typeset with each instance of the new ``set'', while
% an optional argument determines how
% the ``set'' is enumerated:
% \begin{description}
%    \item[\ttfamily \bslash newtheorem\{foo\}\{bar\}]
%       The theorem set {\envfont foo} (whose name is \texttt{bar})
%       uses its own counter.
%    \item[\ttfamily \bslash newtheorem\{foo2\}{[foo]}\{bar2\}]
%       The theorem set {\envfont foo2} (printed name \texttt{bar2})
%       uses the same counter as the theorem set \texttt{foo}.
%    \item[\ttfamily \bslash newtheorem\{foo3\}\{bar3\}{[section]}]
%       The theorem set {\envfont foo3} (printed name \texttt{bar3}) is
%       enumerated within the counter \texttt{section}, i.e.\ with every
%       new |\section| the enumeration begins again with 1, and
%       the enumeration is composed from the section-number and the
%       theorem counter itself.
% \end{description}
%
% \DescribeMacro\theoremstyle
% Additionally, the command |\theoremstyle| can define the layout
% of various, or all, theorem sets.  It should be noted that any
% theorem set defined by |\newtheorem| is typeset in the
% |\theoremstyle| that is current at the time of the definition.
% Thus, the following
% \begin{verbatim}
%    \theoremstyle{break}        \newtheorem{Cor}{Corollary}
%    \theoremstyle{plain}        \newtheorem{Exa}{Example}[section]
%\end{verbatim}                             ^^A No blank at line start
% leads to the result that the set {\envfont Cor} is formatted in the
% style \texttt{break}, while the set {\envfont Exa} and all the
% following ones are formatted in the style \texttt{plain}, unless
% another |\theoremstyle| follows.  Since the definitions installed by
% |\newtheorem| are global, one also can limit |\theoremstyle| locally
% by grouping braces.
%
% \DescribeMacro\theorembodyfont
% The choice of the font for the theorem body is completely
% independent of the chosen |\theoremstyle|; this has proven to
% be very advantageous.  For example,
% \begin{verbatim}
%   {\theorembodyfont{\upshape}          \newtheorem{Rem}{Remark}}
%\end{verbatim}
% defines a theorem set {\envfont Rem}, which will be set in
% |\upshape| in the current layout (which in our example is
% \texttt{plain}).  As with |\theoremstyle|, the |\theorembodyfont|
% chosen is that current at the time of |\newtheorem|.  If
% |\theorembodyfont| is not specified or one defines
% |\theorembodyfont{}|, then the font used will be that defined by the
% |\theoremstyle|.
%
% \DescribeMacro\theoremheaderfont It is also possible to customize
% the font used for the theorem headers.  This is, however, a global
% declaration, and therefore there should be at most one
% |\theoremheaderfont| declaration in the preamble.\footnote{If it is
% actually necessary to have different header fonts, one has to define
% new theorem styles (substituting the desired font) or specify the
% information directly in the \texttt{\bslash newtheorem} declaration
% (the unclean variant).}
%
% \DescribeMacro\theorempreskipamount
% \DescribeMacro\theorempostskipamount Two additional parameters
% affect the vertical space around the theorem environments:
% |\theorempreskipamount| and |\theorempostskipamount| define,
% respectively, the spacing before and after such an environment.
% These parameters apply for all theorem sets and can be manipulated
% with the ordinary length macros.  They are rubber lengths
% (`\textsf{skips}'), and therefore can contain \texttt{plus} and
% \texttt{minus} parts.
%
% Since the definition of theorem sets should---most sensibly---be
% placed in the preamble, we only allow installation there.
% It is therefore possible to release the memory used here after
% |\begin{document}|, in order to make room for other
% applications.
%
%
% \subsection{Existing theorem styles}
%
% The following theorem styles exist to date:
% \begin{deflist}{marginbreak}
%    \item[plain]
%       This theorem style emulates the original \LaTeX{} definition,
%       except that additionally the parameters
%       |\theorem...skipamount| are used.
%    \item[break]
%       In this style, the theorem header is followed by a line
%       break.
%    \item[marginbreak]
%       The theorem number is set in the margin, and there is a line
%       break as in \texttt{break}.
%    \item[changebreak]
%       Like \texttt{break}, but with header number and text
%       interchanged.
%    \item[change]
%       Header number and text are interchanged, without a line break.
%    \item[margin]
%       The number is set in the left margin, without a line break.
% \end{deflist}
% All styles (except \texttt{plain}) select |\slshape| as the default
% |\theorembodyfont|.
%
%
% \subsection{Examples}
%
% Given the above theorem sets {\envfont Cor}, {\envfont Exa}
% and {\envfont Rem},
% suppose that the preamble also contains the declarations:
% \begin{verbatim}
%    \theoremstyle{marginbreak}   \newtheorem{Lem}[Cor]{Lemma}
%    \theoremstyle{change}
%    \theorembodyfont{\itshape}        \newtheorem{Def}[Cor]{Definition}
%
%    \theoremheaderfont{\scshape}
%\end{verbatim}
% Then the following are some typical examples of the typeset output
% resulting from their use.
%
% \begin{Cor}
%    This is a sentence typeset in the theorem environment {\envfont
%    Cor}.
% \end{Cor}
%
% \begin{Exa}
%    This is a sentence typeset in the theorem environment {\envfont
%    Exa}.
% \end{Exa}
%
% \begin{Rem}
%    This is a sentence typeset in the theorem environment {\envfont
%    Rem}.
% \end{Rem}
%
% \begin{Lem}[Ben User]
%    This is a sentence typeset in the theorem environment {\envfont
%    Lem}.
% \end{Lem}
%
% \begin{Def}[Very impressive Definition]
%    This is a sentence typeset in the theorem environment {\envfont
%    Def}.
% \end{Def}
%
% The last two examples show the effect of the optional argument to
% a theorem environment (it is the text typeset in parentheses).
%
%
% \section{Special Considerations}
% \label{sec:newfonts}
%
% Theoremheader and body are implemented as a unit. This means that
% the |\theoremheaderfont|  will inherit characteristics of the
% |\theorembodyfont| in \LaTeXe. Thus, if for example
% |\theorembodyfont| is |\itshape| and |\theoremheaderfont|
% is |\bfseries| the font selected for the header will have the
% characteristics `bold extended italic'. If this is not desired one
% should set the |\theoremheaderfont| to something like
% \begin{verbatim}
%      \theoremheaderfont{\normalfont\bfseries}
%\end{verbatim}
% i.e.\ supplying all necessary font information explicitly.
%
%
%
% \section{Acknowledgements}
%
% The publication of this set of macros was only possible with the
% help of Christina Busse (translating the manuscript into English),
% Joachim Pense (playing the r\^ole of typist), Chris Rowley
% (looking everything over) and many others providing useful
% suggestions.
%
% \MaybeStop{%
%    \begin{thebibliography}{9}
%       \bibitem{bk:GMS94} \textsc{M.~Goossens}, \textsc{F.~Mittelbach}
%         and \textsc{A.~Samarin}.
%         \newblock The \LaTeX{} Companion.
%         \newblock
%         Addison-Wesley, Reading, Massachusetts, 1994.
%       \bibitem{latex:source} \textsc{Lamport, Leslie.} \newblock
%                               \texttt{latex.tex}, version 2.09, date
%                               Feb. 1990.
%    \end{thebibliography}
%    \PrintIndex
%    \PrintChanges}
%
%
% \section{The documentation driver file}
%
% The next bit of code contains the documentation driver file for
% \TeX{}, i.e., the file that will produce the documentation you are
% currently reading. It will be extracted from this file by the
% \texttt{docstrip} program. Since it is the first code in the file
% one can alternatively process this file directly with \LaTeXe{} to
% obtain the documentation.
%
%    \begin{macrocode}
%<*driver>
\documentclass{ltxdoc}

\usepackage{theorem}

%<+driver>% The next few lines define theorem sets which are used
%<+driver>% in the example section of the documentation.

\theoremstyle{break}       \newtheorem{Cor}{Corollary}
\theoremstyle{plain}       \newtheorem{Exa}{Example}[section]
{\theorembodyfont{\upshape}\newtheorem{Rem}{Remark}}
\theoremstyle{marginbreak} \newtheorem{Lem}[Cor]{Lemma}
\theoremstyle{change}
\theorembodyfont{\itshape} \newtheorem{Def}[Cor]{Definition}

\theoremheaderfont{\scshape}

\RecordChanges

\begin{document}
   \DocInput{theorem.dtx}
\end{document}
%</driver>
%    \end{macrocode}
%
% \section{Definition of the Macros}
%
% If the file has been loaded before, we abort immediately. If not the
% package announces itself (this is actually done at the very top of
% the file---the way it is done isn't good style so don't copy it).
% \changes{v2.0e}{89/07/19}{Spaces removed from `typeout'.}
% \changes{v2.2b}{95/11/19}{Announce moved to top of file}
%    \begin{macrocode}
%<*package>
%\@ifundefined{theorem@style}{}{\endinput}
%\def\FMithmInfo{1995/11/19 v2.2b Theorem extension package (FMi)}
%\ProvidesPackage{theorem}[\FMithmInfo]
%    \end{macrocode}
%
%
% \subsection{Definition of theorem styles and fonts}
%
% All the definitions in this file are done globally to allow
% inputting this file inside a group.
%
% \begin{macro}{\theoremstyle}
%    Before a theorem style can be installed, the chosen style must be
%    known. For that reason, we must test to see that
%    |\th@|\meta{style} is known or, more precisely, that it is
%    different from |\relax|. If the style is not known then
%    |\th@plain| is used.
%    \begin{macrocode}
\gdef\theoremstyle#1{%
   \@ifundefined{th@#1}{\@warning
          {Unknown theoremstyle `#1'. Using `plain'}%
          \theorem@style{plain}}%
%    \end{macrocode}
%    We save the theorem style to be used in the token register
%    |\theorem@style|.
%    \begin{macrocode}
      {\theorem@style{#1}}%
%    \end{macrocode}
%    Now we ``evaluate'' the theorem style: this means, we call the
%    macro |\th@|\meta{style} which will activate the relevant
%    definitions which are contained in a separate file.  This is done
%    in a group to suppress changes to the current font.  This could
%    otherwise pose problems together with the new font selection
%    scheme\footnote{When I printed the original article using the new
%    font selection scheme I ended with a document with slanted
%    typefaces (text headings and all) simply because one of the
%    theorem styles used \texttt{\bslash sl} at toplevel.} if the
%    |\th@|\meta{style} is evaluated a second time.
% \changes{v2.1a}{90/02/15}{Group added around `th@..}
% \changes{v2.1a}{90/02/15}{Reseting of `..@begintheorem removed.}
%    \begin{macrocode}
      \begingroup
        \csname th@\the\theorem@style \endcsname
      \endgroup}
%    \end{macrocode}
% \end{macro}
%
%
% \begin{macro}{\@begintheorem}
% \changes{v2.1a}{90/02/15}{Relaxed at toplevel.}
% \begin{macro}{\@opargbegintheorem}
% \changes{v2.1a}{90/02/15}{Relaxed at toplevel.}
%    We reset |\@begintheorem| and |\@opargbegintheorem| to
%    |\relax| since these commands are no longer necessary at
%    toplevel. This will save a few tokens.
%    \begin{macrocode}
\global\let\@begintheorem\relax
\global\let\@opargbegintheorem\relax
%    \end{macrocode}
% \end{macro}
% \end{macro}
%
% \begin{macro}{\theorem@style}
%    Obviously the \textsf{token} register used above has to be
%    allocated. To assure the utmost compatibility with the
%    original \LaTeX{} definition, we set the default theorem style to
%    \texttt{plain}, which implements the usual \LaTeX{} convention.
% \changes{v2.0f}{89/07/20}{Unnecessary `global' removed.}
%    \begin{macrocode}
\newtoks\theorem@style
\global\theorem@style{plain}
%    \end{macrocode}
% \end{macro}
%
% \begin{macro}{\theorembodyfont}
% \begin{macro}{\theorem@bodyfont}
%    If the |\theorembodyfont| is set by the user then it should not
%    interact with the default font set in the theorem style. When the
%    new font selection is in force this may happen if, for example,
%    the default is |\itshape| and the new |\theorembodyfont| is
%    |\sffamily|. So we add a |\reset@font| command in front of the
%    user definition.
% \changes{v2.0f}{89/07/20}{Unnecessary `global' removed.}
% \changes{v2.0g}{89/09/19}{`theoremfont to `theorembodyfont.}
% \changes{v2.1c}{92/06/22}{`reset@font added and logic changed.}
% \changes{v2.1d}{92/10/28}{Reset to empty if arg is empty}
%    \begin{macrocode}
\gdef\theorembodyfont#1{%
%    \end{macrocode}
%    We check if the argument supplied is empty and if so put nothing
%    into the |\theorem@bodyfont| token register to allow for
%    |\theorembodyfont{}| as a mean of using the default of the
%    current |\theoremstyle|.
%    \begin{macrocode}
   \def\@tempa{#1}%
   \ifx\@tempa\@empty
    \theorem@bodyfont{}%
   \else
    \theorem@bodyfont{\reset@font#1}%
   \fi
   }
\newtoks\theorem@bodyfont
\global\theorem@bodyfont{}
%    \end{macrocode}
% \end{macro}
% \end{macro}
%
%
% \begin{macro}{\theoremheaderfont}
% \changes{v2.0g}{89/09/19}{Macro added.}
%    The font for the theorem headers is handled differently because
%    this definition applies to all theorem styles.
%    \begin{macrocode}
\gdef\theoremheaderfont#1{\gdef\theorem@headerfont{#1}%
%    \end{macrocode}
%    After using the macro once it is redefined to produce an error
%    message.
%    \begin{macrocode}
       \gdef\theoremheaderfont##1{%
        \typeout{\string\theoremheaderfont\space should be used
                 only once.}}}
%    \end{macrocode}
% \end{macro}
%
% \begin{macro}{\theorem@headerfont}
% \changes{v2.0g}{89/09/19}{Macro added.}
%    To set the |\theorem@headerfont| default we first test if
%    the new fontselection scheme is in force.
% \changes{v2.1a}{90/02/15}{Support for new font selection.}
%    \begin{macrocode}
\ifx\upshape\undefined
%    \end{macrocode}
%    If not we define it to expand into |\bfseries|. We don't use
%    |\let| just in case a following style option redefines this
%    macro.
%    \begin{macrocode}
\gdef\theorem@headerfont{\bfseries}
%    \end{macrocode}
%    Otherwise we reset the current shape before calling |\bfseries|.
%    \begin{macrocode}
\else \gdef\theorem@headerfont{\normalfont\bfseries}\fi
%    \end{macrocode}
% \end{macro}
%
% \begin{macro}{\th@plain}
% \begin{macro}{\th@break}
% \begin{macro}{\th@marginbreak}
% \begin{macro}{\th@changebreak}
% \begin{macro}{\th@change}
% \begin{macro}{\th@margin}
%    The different styles are defined in macros such as
%    |\th@plain|.  Since memory space is precious in
%    ``non-Big-versions'', we have to avoid offering too many unused
%    definitions.  Therefore we define these styles in separate files
%    that can be loaded on demand.  Thus the commands themselves only
%    load these files. We use |\@input@|, a \LaTeXe{} internal command
%    that ensures that the file will be listed with |\listfiles|.
% \changes{v2.2a}{94/02/02}{Use `@input@ to load theorem layout files}
%    \begin{macrocode}
\gdef\th@plain{\@input@{thp.sty}}
\gdef\th@break{\@input@{thb.sty}}
\gdef\th@marginbreak{\@input@{thmb.sty}}
\gdef\th@changebreak{\@input@{thcb.sty}}
\gdef\th@change{\@input@{thc.sty}}
\gdef\th@margin{\@input@{thm.sty}}
%    \end{macrocode}
%    This list will be expanded when new styles become
%    available.  For testing,  just append new theorem substyles
%    as document options.
% \end{macro}
% \end{macro}
% \end{macro}
% \end{macro}
% \end{macro}
% \end{macro}
%
%
%
% \subsection{Definition of a new theorem set}
%
% As already pointed out, a new theorem environment can be defined in
% three different ways:
% \begin{verbatim}
%    \newtheorem{Lem}{Lemma}
%    \newtheorem{Lem}{Lemma}[section]
%    \newtheorem{Lem}[Theorem]{Lemma}
%\end{verbatim}
% The function of the macro |\newtheorem| is to recognize these cases
% and then to branch into one of the three macros |\@ynthm|, |\@xnthm|
% or |\@othm|.  This mechanism is adopted unchanged from
% \cite{latex:source}; the essential point here is that, for example,
% in the second case, the arguments \texttt{Lem}, \texttt{Lemma} and
% \texttt{section} are passed over to the macro |\@xnthm|.
%
% We inspect this case first because the others present fewer problems,
% and thus are easily derived from this one.
%
% \begin{macro}{\@xnthm}
%    For our example arguments, the macro |\@xnthm| must fulfill
%    the following:
%    \begin{itemize}
%       \item
%          Define a new \LaTeX-counter `Lem'
%       \item
%          reset this counter within a |\section|
%       \item
%          define the macro |\theLem|
%       \item
%          define the environment macros |\Lem| and
%          |\endLem| using the current
%          |\theoremstyle| and |\theorem@bodyfont|.
%    \end{itemize}
%    Obviously, all this should happen only if the first argument
%    of |\@xnthm| (i.e.\ \texttt{Lem} in our example) is chosen so
%    as not to conflict with any previously defined commands or
%    environments.  This test is performed by the \LaTeX{} macro
%    |\@ifdefinable|.
%    \begin{macrocode}
\gdef\@xnthm#1#2[#3]{\expandafter\@ifdefinable\csname #1\endcsname
%    \end{macrocode}
%    Therefore, the first argument of |\@ifdefinable| is the
%    expansion (in the example, |\Lem|) of
%    |\csname||#1\endcsname|.  The second argument is
%    executed only if the test has been completed successfully.
%    \begin{macrocode}
   {%
%    \end{macrocode}
%    Now we define the new counter. The names of the \LaTeX{}
%    macros employed should speak for themselves:
% \changes{v2.2b}{1995/10/07}{Use \cs{@newctr}}
% \changes{v2.2c}{1995/11/23}{Use \cs{@newctr} properly.}
%    \begin{macrocode}
    \@definecounter{#1}\@newctr{#1}[#3]%
%    \end{macrocode}
%    Using |\@newctr| will give a proper error message if the counter
%    in |#3| is not defined.
%    In defining `|\theLem|' we must generate the desired macro
%    name by use of |\expandafter| and |\csname|.
%    \begin{macrocode}
    \expandafter\xdef\csname the#1\endcsname
%    \end{macrocode}
%    An |\xdef| is used in order to make the definition global,
%    and to ensure that it contains the replacement texts of
%    |\@thmcountersep| and |\@thmcounter|.\footnote{These
%    two macros can be defined by the document style.  Their default
%    values produce a `\texttt{.}' as separation and an arabic
%    representation of the number.} However, not everything should be
%    expanded. For example, it saves space to use |\thesection|
%    instead of its---at times---lengthy expansion.
%    \begin{macrocode}
      {\expandafter \noexpand \csname the#3\endcsname
       \@thmcountersep \@thmcounter{#1}}%
%    \end{macrocode}
%    Thus with the defaults of \LaTeX{}, |\theLem| would be
%    replaced by the command sequence |\thesection.\arabic{Lem}|.
%
%    We will now look at the definition of the macro which is executed
%    at the beginning of the actual environment (in our example this
%    macro is |\Lem|). It should be noted that we use an
%    ``|\expandafter| trick'' to expand only certain parts of the
%    replacement text at the time of the definition.
%    \begin{macrocode}
    \def\@tempa{\global\@namedef{#1}}%
    \expandafter \@tempa \expandafter{%
%    \end{macrocode}
%    First, the macro that contains the current definitions of
%    |\@begintheorem| and |\@opargtheorem| should be called
%    up.  The name of this macro---as is already known---has the form
%    |\th@|\meta{theorem\ style}; therefore, it must be called by
%    \begin{macrocode}
      \csname th@\the \theorem@style
%    \end{macrocode}
%    In addition the default theorem font should be changeable, i.e.\
%    we have to insert the contents of |\theorem@bodyfont|. For
%    that reason, we expand even further, beyond |\endcsname|,
%    and thus insert the contents of the \textsf{token} register
%    |\theorem@bodyfont| in the replacement text.
%    \begin{macrocode}
            \expandafter \endcsname \the \theorem@bodyfont
%    \end{macrocode}
%    Now it is time to call the macro |\@thm| which takes over
%    the further processing.  It has two arguments: the current
%    counter name (in our example, \texttt{Lem}), and the text of the
%    label (in our example, \texttt{Lemma}).
%    \begin{macrocode}
     \@thm{#1}{#2}}%
%    \end{macrocode}
%    With this, the `sub-definition' is complete.  The macro
%    |\@endtheorem| ends a theorem environment and is, so far, nothing
%    but an |\endtrivlist|.  (Hence it is defined globally, and not
%    within the theorem styles.\footnote{This has to be changed as
%    soon as theorem styles that change \texttt{\bslash @endtheorem}
%    exist.  In such a case, all existing styles must be changed as
%    well since they will have to reset the macro.}) Therefore, we can
%    set it equivalent to the macro that ends the theorem set (in our
%    example, |\endLem|).  However, if some day theorem styles exist
%    that do change |\@endtheorem|, we would have to use the
%    commented-out line instead.
%    \begin{macrocode}
    \global \expandafter \let \csname end#1\endcsname \@endtheorem
%   \global\@namedef{end#1}{\@endtheorem}%
%    \end{macrocode}
%    With these commands all the required definitions are employed,
%    unless the test |\@ifdefinable| has failed.  Therefore, we
%    end the second argument of this macro and with it the definition
%    of |\@xnthm|.
%    \begin{macrocode}
   }}
%    \end{macrocode}
% \end{macro}
%
% \begin{macro}{\@ynthm}
%    The definition of |\@ynthm| is completely analogous. In
%    this case the new counter that is defined is not reset within
%    another counter; thus the definition of |\the...|
%    is simplified:
%    \begin{macrocode}
\gdef\@ynthm#1#2{\expandafter\@ifdefinable\csname #1\endcsname
   {\@definecounter{#1}%
    \expandafter\xdef\csname the#1\endcsname{\@thmcounter{#1}}%
%    \end{macrocode}
%    The rest of the definition corresponds literally to that of
%    |\@xnthm|:
%    \begin{macrocode}
    \def\@tempa{\global\@namedef{#1}}\expandafter \@tempa
     \expandafter{\csname th@\the \theorem@style \expandafter
     \endcsname \the\theorem@bodyfont \@thm{#1}{#2}}%
    \global \expandafter \let \csname end#1\endcsname \@endtheorem}}
%    \end{macrocode}
% \end{macro}
%
% \begin{macro}{\@othm}
%    The definition of |\@othm| does not contain anything new.
%    \begin{macrocode}
\gdef\@othm#1[#2]#3{%
%    \end{macrocode}
% \changes{v2.2b}{95/10/07}{Build in check for counter defined}
%    We do not define a new counter but instead use one that has
%    already been defined.  Thus the only definition we need is that
%    of this pseudo-counter (i.e.\ |\the|\meta{env.\ name}).
%    First we check if |#2| corresponds to a known counter name.
%    \begin{macrocode}
  \expandafter\ifx\csname c@#2\endcsname\relax
   \@nocounterr{#2}%
  \else
   \expandafter\@ifdefinable\csname #1\endcsname
   {\expandafter \xdef \csname the#1\endcsname
     {\expandafter \noexpand \csname the#2\endcsname}%
%    \end{macrocode}
%    All other parts of the definition can be adopted from
%    |\@xnthm|. We have to remember, though, that in this case
%    the name of the current counter and the theorem label have moved
%    to the second and third arguments.
%    \begin{macrocode}
    \def\@tempa{\global\@namedef{#1}}\expandafter \@tempa
     \expandafter{\csname th@\the \theorem@style \expandafter
     \endcsname \the\theorem@bodyfont \@thm{#2}{#3}}%
    \global \expandafter \let \csname end#1\endcsname \@endtheorem}%
  \fi}
%    \end{macrocode}
% \end{macro}
%
% \subsection{Macros that are employed in a {\envfont theorem}
%    environment}
%
% \begin{macro}{\@thm}
%    The macro |\@thm| has to increase the current counter.  Then,
%    depending on whether the environment has (or does not have) an
%    optional argument, it has to branch into either |\@begintheorem|
%    or |\@opargtheorem|.
%    \begin{macrocode}
\gdef\@thm#1#2{\refstepcounter{#1}%
%    \end{macrocode}
%    Now we start a {\envfont trivlist} environment, and give
%    |\@topsep| and |\@topsepadd| the values of the \textsf{skip}
%    registers |\theorempreskipamount| and |\theorempostskipamount|.
%    The value in |\@topsep| is the vertical space that is inserted by
%    the first (and only) |\item| in our |\trivlist| whilst
%    |\@topsepadd| is inserted by |\@endparenv| at the end of that
%    {\envfont trivlist} environment.  By using these registers, we
%    obtain the desired space around a {\envfont theorem} environment.
%    \begin{macrocode}
   \trivlist
   \@topsep \theorempreskipamount               % used by first \item
   \@topsepadd \theorempostskipamount           % used by \@endparenv
%    \end{macrocode}
%    Now we have to test whether an optional argument has been given.
% \changes{v2.0d}{89/07/18}{Forgotten percent added.}
%    \begin{macrocode}
   \@ifnextchar [%
%    \end{macrocode}
%    If there is an optional argument, we will call |\@ythm|, and
%    move the arguments read back into the input stream.
%    \begin{macrocode}
   {\@ythm{#1}{#2}}%
%    \end{macrocode}
%    If not, we call |\@begintheorem|. Its first argument is the
%    name of the theorem set (hence the second argument of
%    |\@thm|). Its second argument is the macro that produces the
%    current number.
%    \begin{macrocode}
   {\@begintheorem{#2}{\csname the#1\endcsname}\ignorespaces}}
%    \end{macrocode}
% \end{macro}
%
% \begin{macro}{\@xthm}
% \begin{macro}{\@ythm}
%    Both these macros were originally called by |\@thm|.  We do
%    not need |\@xthm| anymore, hence we reset it to
%    |\relax|. The definition of |\@ythm| has not changed at
%    all from its definition in \LaTeX. In order to make the macros
%    easier to understand, we will
%    nevertheless present it (commented out).
%    \begin{macrocode}
\global\let\@xthm\relax
% \def\@ythm#1#2[#3]{\@opargbegintheorem{#2}{\csname
%           the#1\endcsname}{#3}\ignorespaces}
%</package>
%    \end{macrocode}
%    The primitive |\ignorespaces| in |\@ythm| and
%    |\@thm| is needed to remove the spaces between the
%    |\begin{...}| and the actual text.
% \end{macro}
% \end{macro}
%
%
% \subsection{Definition of the theorem substyles}
%
% As already pointed out, the theorem substyles, defined below, are
% only loaded when necessary.  Note that all these substyles, except
% \texttt{plain}, have |\slshape| as the default body font.
%
% \subsubsection{The \texttt{plain} style}
%
% \label{dok:th@plain} As the following macros use |@|, we have
% to locally set the |\catcode| of this symbol to ``letter''.
% This happens within a group, so that we do not have to worry about
% which |\catcode| that symbol had before.
%    \begin{macrocode}
%<*thp>
\begingroup \makeatletter
%    \end{macrocode}
% Since we are now within a group, we must make all definitions
% globally.  First we make sure that \texttt{theorem.sty} is loaded.
% This will allow us to use this file as a document style option
% without having to call \texttt{theorem} itself as an option.  At the
% same time, we assure that at least version 2 is loaded, since
% |\theorem@style| was not defined in earlier versions.
%    \begin{macrocode}
\@ifundefined{theorem@style}{\input{theorem.sty}}{}
\ProvidesFile{thp.sty}
             [\FMithmInfo]
%    \end{macrocode}
%
%
% \begin{macro}{\th@plain}
% \begin{macro}{\@begintheorem}
% \begin{macro}{\@opargbegintheorem}
%    |\theoremstyle{plain}| corresponds to the original
%    definition, except that the distances to the surrounding text are
%    determined by the parameters |\theorempreskipamount| and
%    |\theorempostskipamount|.
%    First we set the default body font.
%    \begin{macrocode}
\gdef\th@plain{\normalfont\itshape
%    \end{macrocode}
%    Then we define |\@begintheorem| and
%    |\@opargbegintheorem|. These two macros define how the
%    header of a theorem is typeset.  |\@opargbegintheorem| will
%    be called if a \textsf{theorem} environment with an optional
%    argument is encountered; otherwise, the header is constructed by
%    calling |\@begintheorem|.  If one of these macros is
%    executed, we are within a \textsf{trivlist} environment started by
%    |\@thm|.  So the theorem header is produced with an
%    |\item| command.
%
%    Instead of specifying the header font directly, all standard
%    theorem styles use the |\theorem@headerfont| macro to allow
%    customization.  The extra space (|\labelsep|) is necessary
%    because of problems in the \textsf{trivlist} environment.
%    \begin{macrocode}
  \def\@begintheorem##1##2{%
        \item[\hskip\labelsep \theorem@headerfont ##1\ ##2]}%
%    \end{macrocode}
%    The definition of |\@opargbegintheorem| is completely
%    analogous. The only difference is the fact that there exists a
%    third argument (which is the optional parameter of the
%    environment and contains additional information about the
%    theorem).  Customarily we enclose it in parentheses.
%    \begin{macrocode}
\def\@opargbegintheorem##1##2##3{%
   \item[\hskip\labelsep \theorem@headerfont ##1\ ##2\ (##3)]}}
%    \end{macrocode}
% \end{macro}
% \end{macro}
% \end{macro}
%
% \noindent
% We conclude with an |\endgroup| to restore the
% |\catcode| of |@|.
%    \begin{macrocode}
\endgroup
%</thp>
%    \end{macrocode}
%
%
% \subsubsection{The \texttt{break} style}
%
% This style option is stored in the file |thb.sty|.
% For the next two lines see the documentation for |\th@plain| on
% page~\pageref{dok:th@plain}.
%    \begin{macrocode}
%<*thb>
\begingroup \makeatletter
\@ifundefined{theorem@style}{\input{theorem.sty}}{}
\ProvidesFile{thb.sty}
             [\FMithmInfo]
%    \end{macrocode}
%
% \begin{macro}{\th@break}
%    |\theoremstyle{break}| produces a line break after the name
%    of the theorem.  The font is |\slshape|.  Hence, we define
%    |\th@break| as follows:
%    \begin{macrocode}
\gdef\th@break{\normalfont\slshape
  \def\@begintheorem##1##2{\item[%
%    \end{macrocode}
%    We run into the following problem: it is not possible to create
%    the header with |\item[|\meta{title}|]| and then start
%    a new line by, for example, |\mbox{}\\|. Such a definition
%    will fail whenever a list environment follows immediately.  With
%    the above construction, the |\mbox{}| causes the switch
%    |@inlabel| (cf.\ definition of |\list| and
%    |\trivlist| in \cite{latex:source}) to be set to \texttt{false}
%    and so the following list will insert additional vertical
%    space (|\topskip|). This is quite annoying. Therefore, we
%    create the line break within the |\item|. In order to ensure
%    that the text will begin at the proper position in the following
%    line, we simply pretend that the label does not take any
%    room.\footnote{This will lead to problems whenever very high
%    symbols occurring in the line tower into the
%    heading. So, something else has to be done here sometime.}
%    \begin{macrocode}
       \rlap{\vbox{\hbox{\hskip \labelsep\theorem@headerfont ##1\ ##2}%
                   \hbox{\strut}}}]}%
%    \end{macrocode}
%    Again, the definition of |\@opargbegintheorem| is
%    completely analogous.
%    \begin{macrocode}
\def\@opargbegintheorem##1##2##3{%
  \item[\rlap{\vbox{\hbox{\hskip \labelsep \theorem@headerfont
                     ##1\ ##2\ (##3)}%
                    \hbox{\strut}}}]}}
\endgroup
%</thb>
%    \end{macrocode}
% \end{macro}
%
%
% \subsubsection{The \texttt{changebreak} style}
%
%    \begin{macrocode}
% This style option is stored in the file |thcb.sty|.
%    \begin{macrocode}
%<*thcb>
\begingroup \makeatletter
\@ifundefined{theorem@style}{\input{theorem.sty}}{}
\ProvidesFile{thcb.sty}
             [\FMithmInfo]
%    \end{macrocode}
%
% \begin{macro}{\th@changebreak}
%    The \texttt{change-break} theorem style is like \texttt{break}
%    but with interchange of theorem name and theorem number.  Thus we
%    define |\th@changebreak| as follows:
%    \begin{macrocode}
\gdef\th@changebreak{\normalfont\slshape
  \def\@begintheorem##1##2{\item
       [\rlap{\vbox{\hbox{\hskip\labelsep \theorem@headerfont ##2\ ##1}%
                    \hbox{\strut}}}]}%
\def\@opargbegintheorem##1##2##3{%
  \item[\rlap{\vbox{\hbox{\hskip\labelsep \theorem@headerfont
                     ##2\ ##1\ (##3)}%
                    \hbox{\strut}}}]}}
\endgroup
%</thcb>
%    \end{macrocode}
% \end{macro}
%
%
% \subsubsection{The \texttt{change} style}
%
% This style option is stored in the file |thc.sty|.
%    \begin{macrocode}
%<*thc>
\begingroup \makeatletter
\@ifundefined{theorem@style}{\input{theorem.sty}}{}
\ProvidesFile{thc.sty}
             [\FMithmInfo]
%    \end{macrocode}
%
% \begin{macro}{\th@change}
%    The \texttt{change} theorem style corresponds to the
%    \texttt{change break} style without a linebreak after the header.
%    To say it in another way, it's the same as the \texttt{plain}
%    style but with number and name interchanged and |\slshape| as the
%    default font.
%    \begin{macrocode}
\gdef\th@change{\normalfont\slshape
  \def\@begintheorem##1##2{\item
       [\hskip\labelsep \theorem@headerfont ##2\ ##1]}%
\def\@opargbegintheorem##1##2##3{%
  \item[\hskip\labelsep \theorem@headerfont ##2\ ##1\ (##3)]}}
\endgroup
%</thc>
%    \end{macrocode}
% \end{macro}
%
%
% \subsubsection{The \texttt{marginbreak} style}
%
% This style option is the one used most often at Mainz.  It is saved
% in the file |thmb.sty|.
%    \begin{macrocode}
%<*thmb>
\begingroup \makeatletter
\@ifundefined{theorem@style}{\input{theorem.sty}}{}
\ProvidesFile{thmb.sty}
             [\FMithmInfo]
%    \end{macrocode}
%
% \begin{macro}{\th@marginbreak}
%    The \texttt{margin break} style is nearly the same as the
%    \texttt{change break} style. The only difference is the placement
%    of the theorem number.  We use |\llap| to place it in the left
%    margin.
%
%    In this style |\labelsep| denotes the separation between the
%    number and the text.
%    \begin{macrocode}
\gdef\th@marginbreak{\normalfont\slshape
  \def\@begintheorem##1##2{\item
       [\rlap{\vbox{\theorem@headerfont
                    \hbox{\llap{##2}\hskip\labelsep ##1}%
                    \hbox{\strut}}}]}%
\def\@opargbegintheorem##1##2##3{%
  \item[\rlap{\vbox{\theorem@headerfont
                    \hbox{\llap{##2}\hskip\labelsep ##1\ (##3)}%
                    \hbox{\strut}}}]}}
\endgroup
%</thmb>
%    \end{macrocode}
% \changes{v2.1c}{90/02/17}{`endgroup outside macrocode env.}
% \end{macro}
%
%
%
% \subsubsection{The \texttt{margin} style}
%
% This style option is stored in the file |thm.sty|.
%    \begin{macrocode}
%<*thm>
\begingroup \makeatletter
\@ifundefined{theorem@style}{\input{theorem.sty}}{}
\ProvidesFile{thm.sty}
             [\FMithmInfo]
%    \end{macrocode}
%
% \begin{macro}{\th@margin}
%    Again this is only a variant of the theorem styles described
%    above without any new ideas.
%    \begin{macrocode}
\gdef\th@margin{\normalfont\slshape
  \def\@begintheorem##1##2{\item
       [\theorem@headerfont \llap{##2}\hskip\labelsep ##1]}%
\def\@opargbegintheorem##1##2##3{%
  \item[\theorem@headerfont \llap{##2}\hskip\labelsep ##1\ (##3)]}}
\endgroup
%</thm>
%    \end{macrocode}
% \end{macro}
%
%
%
%
% \subsection{Final Definitions}
%
% \begin{macro}{\theorempreskipamount}
% \begin{macro}{\theorempostskipamount}
%    The \textsf{skip} parameters that regulate the vertical empty space
%    before and after the {\envfont theorem} environment have
%    to be allocated as well.
% \changes{v2.0f}{89/07/20}{Unnecessary `global' removed.}
%    \begin{macrocode}
%<*package>
\newskip\theorempreskipamount
\newskip\theorempostskipamount
%    \end{macrocode}
%    Since we have used the same values for all theorem sets, we now can
%    assign them.
%    \begin{macrocode}
\global\setlength\theorempreskipamount{12pt plus 5pt minus 3pt}
\global\setlength\theorempostskipamount{8pt plus 3pt minus 1.5pt}
%    \end{macrocode}
% \end{macro}
% \end{macro}
%
%
% \begin{macro}{\@endtheorem}
%    The same holds for the macro |\@endtheorem|, which ends a
%    {\envfont theorem} environment. Since it is the same for all
%    theorem sets, it is removed from the macros |\th@|\meta{style}.
%    It simply ends the {\envfont trivlist} environment, which was
%    begun in |\@thm|.
%    \begin{macrocode}
\global\let\@endtheorem=\endtrivlist
%    \end{macrocode}
% \end{macro}
%
%
%
% \begin{macro}{\@preamblecmds}
%    All macros defined above are to be used only in the preamble.
%    Therefore, we insert them in |\@preamblecmds| which will disable
%    them at begin document. This is done by the internal \LaTeXe{}
%    command |\@onlypreamble|.
% \changes{v2.0g}{89/09/19}{`theoremheaderfont and `theorembodyfont
%    added.}
% \changes{v2.2a}{94/02/03}{Use \LaTeXe{} interface}
%    \begin{macrocode}
\@onlypreamble\@xnthm
\@onlypreamble\@ynthm
\@onlypreamble\@othm
\@onlypreamble\newtheorem
\@onlypreamble\theoremstyle
\@onlypreamble\theorembodyfont
\@onlypreamble\theoremheaderfont
%    \end{macrocode}
% \end{macro}
%
% Finally we declare the \texttt{plain} theorem style to be the default.
% \changes{v2.1b}{90/06/04}{Fixed missing `item by declaring default
%    style.}
%    \begin{macrocode}
\theoremstyle{plain}
%</package>
%    \end{macrocode}
%
% \Finale

\endinput
}{}
\ProvidesFile{thp.sty}
             [\FMithmInfo]
%    \end{macrocode}
%
%
% \begin{macro}{\th@plain}
% \begin{macro}{\@begintheorem}
% \begin{macro}{\@opargbegintheorem}
%    |\theoremstyle{plain}| corresponds to the original
%    definition, except that the distances to the surrounding text are
%    determined by the parameters |\theorempreskipamount| and
%    |\theorempostskipamount|.
%    First we set the default body font.
%    \begin{macrocode}
\gdef\th@plain{\normalfont\itshape
%    \end{macrocode}
%    Then we define |\@begintheorem| and
%    |\@opargbegintheorem|. These two macros define how the
%    header of a theorem is typeset.  |\@opargbegintheorem| will
%    be called if a \textsf{theorem} environment with an optional
%    argument is encountered; otherwise, the header is constructed by
%    calling |\@begintheorem|.  If one of these macros is
%    executed, we are within a \textsf{trivlist} environment started by
%    |\@thm|.  So the theorem header is produced with an
%    |\item| command.
%
%    Instead of specifying the header font directly, all standard
%    theorem styles use the |\theorem@headerfont| macro to allow
%    customization.  The extra space (|\labelsep|) is necessary
%    because of problems in the \textsf{trivlist} environment.
%    \begin{macrocode}
  \def\@begintheorem##1##2{%
        \item[\hskip\labelsep \theorem@headerfont ##1\ ##2]}%
%    \end{macrocode}
%    The definition of |\@opargbegintheorem| is completely
%    analogous. The only difference is the fact that there exists a
%    third argument (which is the optional parameter of the
%    environment and contains additional information about the
%    theorem).  Customarily we enclose it in parentheses.
%    \begin{macrocode}
\def\@opargbegintheorem##1##2##3{%
   \item[\hskip\labelsep \theorem@headerfont ##1\ ##2\ (##3)]}}
%    \end{macrocode}
% \end{macro}
% \end{macro}
% \end{macro}
%
% \noindent
% We conclude with an |\endgroup| to restore the
% |\catcode| of |@|.
%    \begin{macrocode}
\endgroup
%</thp>
%    \end{macrocode}
%
%
% \subsubsection{The \texttt{break} style}
%
% This style option is stored in the file |thb.sty|.
% For the next two lines see the documentation for |\th@plain| on
% page~\pageref{dok:th@plain}.
%    \begin{macrocode}
%<*thb>
\begingroup \makeatletter
\@ifundefined{theorem@style}{% \iffalse meta-comment
%
% Copyright (C) 1993-2024
%
% The LaTeX Project and any individual authors listed elsewhere
% in this file.
%
% This file is part of the Standard LaTeX `Tools Bundle'.
% -------------------------------------------------------
%
% It may be distributed and/or modified under the
% conditions of the LaTeX Project Public License, either version 1.3c
% of this license or (at your option) any later version.
% The latest version of this license is in
%    https://www.latex-project.org/lppl.txt
% and version 1.3c or later is part of all distributions of LaTeX
% version 2005/12/01 or later.
%
% The list of all files belonging to the LaTeX `Tools Bundle' is
% given in the file `manifest.txt'.
%
% \fi
%
%%
%
%
\def\FMithmInfo{2023/07/05 v2.2c Theorem extension package (FMi)}
%
% \ProvidesFile{theorem.dtx}[\FMithmInfo]
%\iffalse   % this is a METACOMMENT !
%%
%% Package `theorem' to use with LaTeX2e
%% Copyright (C) 1989-1995 Frank Mittelbach, all rights reserved.
%<+package>\@ifundefined{theorem@style}{}{\endinput}
%<+package>\NeedsTeXFormat{LaTeX2e}[1995/06/01]
%<+package>\ProvidesPackage{theorem}
%<+package>         [\FMithmInfo]
%
% \fi
%
% \changes{v1.0}{87/05/07}{First draft.}
% \changes{v1.0a}{87/05/07}{`thistheoremstyle' added.}
% \changes{v1.1a}{87/07/29}{New concept: redefine the internal
%                          \protect\LaTeX{} commands `@xnthm' etc.}
% \changes{v1.1b}{87/07/31}{`theoremfont' used to specify font.}
% \changes{v1.2a}{88/01/08}{`@endtheorem' now globally defined.}
% \changes{v1.2b}{88/01/11}{`newtheorem' only allowed in the preamble.}
% \changes{v1.3a}{88/02/14}{`theoremfont' now a toks register.}
%
% \changes{v2.0a}{88/11/16}{New implementation and new documentation.}
% \changes{v2.0a}{88/11/16}{No layout changes but internal modifications
%                           old substyles won't work.}
% \changes{v2.0b}{89/01/30}{`global' used for all definitions.}
% \changes{v2.0c}{89/06/15}{Documentation translated to English.
%                            Thanks to LLL and Nina}
% \changes{v2.1a}{90/02/15}{Support for new font selection.}
% \changes{v2.2a}{94/02/03}{Cleanup for LaTeX2e}
%
% \DoNotIndex{\@endparenv,\@ifdefinable,\@ifnextchar,\@ifundefined}
% \DoNotIndex{\@namedef,\@nameuse,\@spaces,\@tempa,\@warning}
% \DoNotIndex{\ }
% \DoNotIndex{\begingroup,\bfseries}
% \DoNotIndex{\csname}
% \DoNotIndex{\def,\do,\docdate}
% \DoNotIndex{\endcsname,\endgroup,\endinput,\endtrivlist,\expandafter}
% \DoNotIndex{\filedate,\fileversion}
% \DoNotIndex{\gdef,\global}
% \DoNotIndex{\hbox,\hskip}
% \DoNotIndex{\ignorespaces,\input,\itshape,\item}
% \DoNotIndex{\labelsep,\let,\llap}
% \DoNotIndex{\makeatletter}
% \DoNotIndex{\noexpand}
% \DoNotIndex{\refstepcounter,\relax,\rlap}
% \DoNotIndex{\setlength,\slshape,\space,\string,\strut}
% \DoNotIndex{\the,\trivlist,\typeout}
% \DoNotIndex{\vbox}
% \DoNotIndex{\xdef}
%
% \def\deflabel#1{\ttfamily #1\hfill}
% \def\deflist#1{\begin{list}{}{\settowidth\labelwidth{\ttfamily #1}%
%                               \setlength\leftmargin\labelwidth
%                               \addtolength\leftmargin\labelsep
%                               \let\makelabel\deflabel}}
% \def\enddeflist{\end{list}}
% \def\envfont{\normalfont\ttfamily}
%
% \MakeShortVerb{\|}
%
% \newcommand{\TUB}{{\sl TUGboat\/}}
%
% \GetFileInfo{theorem.dtx}
% \title{An Extension of the
%        \LaTeX{} theorem environment\thanks{This file
%        has version number \fileversion, last
%        revised \filedate.}}
% \author{Frank Mittelbach}
%
% \MaintainedByLaTeXTeam{tools}
% \maketitle
%
% \begin{abstract}
%    The macros described in this paper yield an extension of the
%    \LaTeX{} theorem mechanism. It is designed to satisfy the
%    different requirements of various journals. Thus, the
%    layout of the ``theorems'' can be manipulated by determining a
%    ``style''. This article describes not only the use, but
%    also the definition, of the necessary macros.
% \end{abstract}
%
% \section*{Preface to version 2.2}
%
% For \LaTeXe{} this package did not need any fundamental changes. I
% only modified the messages generated so that theorem layout styles
% will show up with the |\listfiles| command and cleaned the section
% on the New Font Selection Scheme since this is now included in
% \LaTeX.
%
% \section*{Preface to version 2.1}
%
% This version is identical to 2.0g described in \TUB\ 10\#3 except
% for some internal defaults which are now set depending on the used
% font selection scheme.
%
% This was done to avoid unpleasant surprises if the new font selection
% scheme is in force. For further details see
% section~\ref{sec:newfonts} and \cite{bk:GMS94}.
%
% \section{Introduction}
%
% For our purposes here, ``theorems'' are labelled enunciations,
% often set off from the main text by extra space and a font change.
% Theorems, corollaries, conjectures, definitions, and remarks
% are all instances of ``theorems''.  The ``header'' of these structures
% is composed of a label (such as \textsc{Theorem} or \textsc{Remark})
% and a number which serializes an item in the sequence of items
% with the same label.
%
% Shortly after the introduction of \LaTeX{} at the Fachbereich
% Mathematik in Mainz, the desire to manipulate the layout of
% ``theorems'' arose.  In Mainz, the following two conventions came
% into general use:
% \begin{enumerate}
%   \item The number of the theorem is shown in the margin.
%   \item There is a line break at
%         the end of the theorem header.
% \end{enumerate}
%
% Additionally, some journals require different formats which depend on
% the ``sort of theorem'': e.g.\ often remarks and definitions are set
% in |\upshape|, while |\itshape| is employed for main theorems.
%
% Confronted with these requirements, a theorem environment was
% developed in Mainz which allows separate determination of the
% layout of the ``theorems sets'', comparable to |\pagestyle|.
%
%
% \section{The user interface}
% \subsection{Defining new theorem sets}
%
% \DescribeMacro\newtheorem
% As in the original \LaTeX{} version, the command |\newtheorem|
% defines a new ``theorem set'' or ``theorem-like structure''.
% Two required arguments name the new environment and give the text
% to be typeset with each instance of the new ``set'', while
% an optional argument determines how
% the ``set'' is enumerated:
% \begin{description}
%    \item[\ttfamily \bslash newtheorem\{foo\}\{bar\}]
%       The theorem set {\envfont foo} (whose name is \texttt{bar})
%       uses its own counter.
%    \item[\ttfamily \bslash newtheorem\{foo2\}{[foo]}\{bar2\}]
%       The theorem set {\envfont foo2} (printed name \texttt{bar2})
%       uses the same counter as the theorem set \texttt{foo}.
%    \item[\ttfamily \bslash newtheorem\{foo3\}\{bar3\}{[section]}]
%       The theorem set {\envfont foo3} (printed name \texttt{bar3}) is
%       enumerated within the counter \texttt{section}, i.e.\ with every
%       new |\section| the enumeration begins again with 1, and
%       the enumeration is composed from the section-number and the
%       theorem counter itself.
% \end{description}
%
% \DescribeMacro\theoremstyle
% Additionally, the command |\theoremstyle| can define the layout
% of various, or all, theorem sets.  It should be noted that any
% theorem set defined by |\newtheorem| is typeset in the
% |\theoremstyle| that is current at the time of the definition.
% Thus, the following
% \begin{verbatim}
%    \theoremstyle{break}        \newtheorem{Cor}{Corollary}
%    \theoremstyle{plain}        \newtheorem{Exa}{Example}[section]
%\end{verbatim}                             ^^A No blank at line start
% leads to the result that the set {\envfont Cor} is formatted in the
% style \texttt{break}, while the set {\envfont Exa} and all the
% following ones are formatted in the style \texttt{plain}, unless
% another |\theoremstyle| follows.  Since the definitions installed by
% |\newtheorem| are global, one also can limit |\theoremstyle| locally
% by grouping braces.
%
% \DescribeMacro\theorembodyfont
% The choice of the font for the theorem body is completely
% independent of the chosen |\theoremstyle|; this has proven to
% be very advantageous.  For example,
% \begin{verbatim}
%   {\theorembodyfont{\upshape}          \newtheorem{Rem}{Remark}}
%\end{verbatim}
% defines a theorem set {\envfont Rem}, which will be set in
% |\upshape| in the current layout (which in our example is
% \texttt{plain}).  As with |\theoremstyle|, the |\theorembodyfont|
% chosen is that current at the time of |\newtheorem|.  If
% |\theorembodyfont| is not specified or one defines
% |\theorembodyfont{}|, then the font used will be that defined by the
% |\theoremstyle|.
%
% \DescribeMacro\theoremheaderfont It is also possible to customize
% the font used for the theorem headers.  This is, however, a global
% declaration, and therefore there should be at most one
% |\theoremheaderfont| declaration in the preamble.\footnote{If it is
% actually necessary to have different header fonts, one has to define
% new theorem styles (substituting the desired font) or specify the
% information directly in the \texttt{\bslash newtheorem} declaration
% (the unclean variant).}
%
% \DescribeMacro\theorempreskipamount
% \DescribeMacro\theorempostskipamount Two additional parameters
% affect the vertical space around the theorem environments:
% |\theorempreskipamount| and |\theorempostskipamount| define,
% respectively, the spacing before and after such an environment.
% These parameters apply for all theorem sets and can be manipulated
% with the ordinary length macros.  They are rubber lengths
% (`\textsf{skips}'), and therefore can contain \texttt{plus} and
% \texttt{minus} parts.
%
% Since the definition of theorem sets should---most sensibly---be
% placed in the preamble, we only allow installation there.
% It is therefore possible to release the memory used here after
% |\begin{document}|, in order to make room for other
% applications.
%
%
% \subsection{Existing theorem styles}
%
% The following theorem styles exist to date:
% \begin{deflist}{marginbreak}
%    \item[plain]
%       This theorem style emulates the original \LaTeX{} definition,
%       except that additionally the parameters
%       |\theorem...skipamount| are used.
%    \item[break]
%       In this style, the theorem header is followed by a line
%       break.
%    \item[marginbreak]
%       The theorem number is set in the margin, and there is a line
%       break as in \texttt{break}.
%    \item[changebreak]
%       Like \texttt{break}, but with header number and text
%       interchanged.
%    \item[change]
%       Header number and text are interchanged, without a line break.
%    \item[margin]
%       The number is set in the left margin, without a line break.
% \end{deflist}
% All styles (except \texttt{plain}) select |\slshape| as the default
% |\theorembodyfont|.
%
%
% \subsection{Examples}
%
% Given the above theorem sets {\envfont Cor}, {\envfont Exa}
% and {\envfont Rem},
% suppose that the preamble also contains the declarations:
% \begin{verbatim}
%    \theoremstyle{marginbreak}   \newtheorem{Lem}[Cor]{Lemma}
%    \theoremstyle{change}
%    \theorembodyfont{\itshape}        \newtheorem{Def}[Cor]{Definition}
%
%    \theoremheaderfont{\scshape}
%\end{verbatim}
% Then the following are some typical examples of the typeset output
% resulting from their use.
%
% \begin{Cor}
%    This is a sentence typeset in the theorem environment {\envfont
%    Cor}.
% \end{Cor}
%
% \begin{Exa}
%    This is a sentence typeset in the theorem environment {\envfont
%    Exa}.
% \end{Exa}
%
% \begin{Rem}
%    This is a sentence typeset in the theorem environment {\envfont
%    Rem}.
% \end{Rem}
%
% \begin{Lem}[Ben User]
%    This is a sentence typeset in the theorem environment {\envfont
%    Lem}.
% \end{Lem}
%
% \begin{Def}[Very impressive Definition]
%    This is a sentence typeset in the theorem environment {\envfont
%    Def}.
% \end{Def}
%
% The last two examples show the effect of the optional argument to
% a theorem environment (it is the text typeset in parentheses).
%
%
% \section{Special Considerations}
% \label{sec:newfonts}
%
% Theoremheader and body are implemented as a unit. This means that
% the |\theoremheaderfont|  will inherit characteristics of the
% |\theorembodyfont| in \LaTeXe. Thus, if for example
% |\theorembodyfont| is |\itshape| and |\theoremheaderfont|
% is |\bfseries| the font selected for the header will have the
% characteristics `bold extended italic'. If this is not desired one
% should set the |\theoremheaderfont| to something like
% \begin{verbatim}
%      \theoremheaderfont{\normalfont\bfseries}
%\end{verbatim}
% i.e.\ supplying all necessary font information explicitly.
%
%
%
% \section{Acknowledgements}
%
% The publication of this set of macros was only possible with the
% help of Christina Busse (translating the manuscript into English),
% Joachim Pense (playing the r\^ole of typist), Chris Rowley
% (looking everything over) and many others providing useful
% suggestions.
%
% \MaybeStop{%
%    \begin{thebibliography}{9}
%       \bibitem{bk:GMS94} \textsc{M.~Goossens}, \textsc{F.~Mittelbach}
%         and \textsc{A.~Samarin}.
%         \newblock The \LaTeX{} Companion.
%         \newblock
%         Addison-Wesley, Reading, Massachusetts, 1994.
%       \bibitem{latex:source} \textsc{Lamport, Leslie.} \newblock
%                               \texttt{latex.tex}, version 2.09, date
%                               Feb. 1990.
%    \end{thebibliography}
%    \PrintIndex
%    \PrintChanges}
%
%
% \section{The documentation driver file}
%
% The next bit of code contains the documentation driver file for
% \TeX{}, i.e., the file that will produce the documentation you are
% currently reading. It will be extracted from this file by the
% \texttt{docstrip} program. Since it is the first code in the file
% one can alternatively process this file directly with \LaTeXe{} to
% obtain the documentation.
%
%    \begin{macrocode}
%<*driver>
\documentclass{ltxdoc}

\usepackage{theorem}

%<+driver>% The next few lines define theorem sets which are used
%<+driver>% in the example section of the documentation.

\theoremstyle{break}       \newtheorem{Cor}{Corollary}
\theoremstyle{plain}       \newtheorem{Exa}{Example}[section]
{\theorembodyfont{\upshape}\newtheorem{Rem}{Remark}}
\theoremstyle{marginbreak} \newtheorem{Lem}[Cor]{Lemma}
\theoremstyle{change}
\theorembodyfont{\itshape} \newtheorem{Def}[Cor]{Definition}

\theoremheaderfont{\scshape}

\RecordChanges

\begin{document}
   \DocInput{theorem.dtx}
\end{document}
%</driver>
%    \end{macrocode}
%
% \section{Definition of the Macros}
%
% If the file has been loaded before, we abort immediately. If not the
% package announces itself (this is actually done at the very top of
% the file---the way it is done isn't good style so don't copy it).
% \changes{v2.0e}{89/07/19}{Spaces removed from `typeout'.}
% \changes{v2.2b}{95/11/19}{Announce moved to top of file}
%    \begin{macrocode}
%<*package>
%\@ifundefined{theorem@style}{}{\endinput}
%\def\FMithmInfo{1995/11/19 v2.2b Theorem extension package (FMi)}
%\ProvidesPackage{theorem}[\FMithmInfo]
%    \end{macrocode}
%
%
% \subsection{Definition of theorem styles and fonts}
%
% All the definitions in this file are done globally to allow
% inputting this file inside a group.
%
% \begin{macro}{\theoremstyle}
%    Before a theorem style can be installed, the chosen style must be
%    known. For that reason, we must test to see that
%    |\th@|\meta{style} is known or, more precisely, that it is
%    different from |\relax|. If the style is not known then
%    |\th@plain| is used.
%    \begin{macrocode}
\gdef\theoremstyle#1{%
   \@ifundefined{th@#1}{\@warning
          {Unknown theoremstyle `#1'. Using `plain'}%
          \theorem@style{plain}}%
%    \end{macrocode}
%    We save the theorem style to be used in the token register
%    |\theorem@style|.
%    \begin{macrocode}
      {\theorem@style{#1}}%
%    \end{macrocode}
%    Now we ``evaluate'' the theorem style: this means, we call the
%    macro |\th@|\meta{style} which will activate the relevant
%    definitions which are contained in a separate file.  This is done
%    in a group to suppress changes to the current font.  This could
%    otherwise pose problems together with the new font selection
%    scheme\footnote{When I printed the original article using the new
%    font selection scheme I ended with a document with slanted
%    typefaces (text headings and all) simply because one of the
%    theorem styles used \texttt{\bslash sl} at toplevel.} if the
%    |\th@|\meta{style} is evaluated a second time.
% \changes{v2.1a}{90/02/15}{Group added around `th@..}
% \changes{v2.1a}{90/02/15}{Reseting of `..@begintheorem removed.}
%    \begin{macrocode}
      \begingroup
        \csname th@\the\theorem@style \endcsname
      \endgroup}
%    \end{macrocode}
% \end{macro}
%
%
% \begin{macro}{\@begintheorem}
% \changes{v2.1a}{90/02/15}{Relaxed at toplevel.}
% \begin{macro}{\@opargbegintheorem}
% \changes{v2.1a}{90/02/15}{Relaxed at toplevel.}
%    We reset |\@begintheorem| and |\@opargbegintheorem| to
%    |\relax| since these commands are no longer necessary at
%    toplevel. This will save a few tokens.
%    \begin{macrocode}
\global\let\@begintheorem\relax
\global\let\@opargbegintheorem\relax
%    \end{macrocode}
% \end{macro}
% \end{macro}
%
% \begin{macro}{\theorem@style}
%    Obviously the \textsf{token} register used above has to be
%    allocated. To assure the utmost compatibility with the
%    original \LaTeX{} definition, we set the default theorem style to
%    \texttt{plain}, which implements the usual \LaTeX{} convention.
% \changes{v2.0f}{89/07/20}{Unnecessary `global' removed.}
%    \begin{macrocode}
\newtoks\theorem@style
\global\theorem@style{plain}
%    \end{macrocode}
% \end{macro}
%
% \begin{macro}{\theorembodyfont}
% \begin{macro}{\theorem@bodyfont}
%    If the |\theorembodyfont| is set by the user then it should not
%    interact with the default font set in the theorem style. When the
%    new font selection is in force this may happen if, for example,
%    the default is |\itshape| and the new |\theorembodyfont| is
%    |\sffamily|. So we add a |\reset@font| command in front of the
%    user definition.
% \changes{v2.0f}{89/07/20}{Unnecessary `global' removed.}
% \changes{v2.0g}{89/09/19}{`theoremfont to `theorembodyfont.}
% \changes{v2.1c}{92/06/22}{`reset@font added and logic changed.}
% \changes{v2.1d}{92/10/28}{Reset to empty if arg is empty}
%    \begin{macrocode}
\gdef\theorembodyfont#1{%
%    \end{macrocode}
%    We check if the argument supplied is empty and if so put nothing
%    into the |\theorem@bodyfont| token register to allow for
%    |\theorembodyfont{}| as a mean of using the default of the
%    current |\theoremstyle|.
%    \begin{macrocode}
   \def\@tempa{#1}%
   \ifx\@tempa\@empty
    \theorem@bodyfont{}%
   \else
    \theorem@bodyfont{\reset@font#1}%
   \fi
   }
\newtoks\theorem@bodyfont
\global\theorem@bodyfont{}
%    \end{macrocode}
% \end{macro}
% \end{macro}
%
%
% \begin{macro}{\theoremheaderfont}
% \changes{v2.0g}{89/09/19}{Macro added.}
%    The font for the theorem headers is handled differently because
%    this definition applies to all theorem styles.
%    \begin{macrocode}
\gdef\theoremheaderfont#1{\gdef\theorem@headerfont{#1}%
%    \end{macrocode}
%    After using the macro once it is redefined to produce an error
%    message.
%    \begin{macrocode}
       \gdef\theoremheaderfont##1{%
        \typeout{\string\theoremheaderfont\space should be used
                 only once.}}}
%    \end{macrocode}
% \end{macro}
%
% \begin{macro}{\theorem@headerfont}
% \changes{v2.0g}{89/09/19}{Macro added.}
%    To set the |\theorem@headerfont| default we first test if
%    the new fontselection scheme is in force.
% \changes{v2.1a}{90/02/15}{Support for new font selection.}
%    \begin{macrocode}
\ifx\upshape\undefined
%    \end{macrocode}
%    If not we define it to expand into |\bfseries|. We don't use
%    |\let| just in case a following style option redefines this
%    macro.
%    \begin{macrocode}
\gdef\theorem@headerfont{\bfseries}
%    \end{macrocode}
%    Otherwise we reset the current shape before calling |\bfseries|.
%    \begin{macrocode}
\else \gdef\theorem@headerfont{\normalfont\bfseries}\fi
%    \end{macrocode}
% \end{macro}
%
% \begin{macro}{\th@plain}
% \begin{macro}{\th@break}
% \begin{macro}{\th@marginbreak}
% \begin{macro}{\th@changebreak}
% \begin{macro}{\th@change}
% \begin{macro}{\th@margin}
%    The different styles are defined in macros such as
%    |\th@plain|.  Since memory space is precious in
%    ``non-Big-versions'', we have to avoid offering too many unused
%    definitions.  Therefore we define these styles in separate files
%    that can be loaded on demand.  Thus the commands themselves only
%    load these files. We use |\@input@|, a \LaTeXe{} internal command
%    that ensures that the file will be listed with |\listfiles|.
% \changes{v2.2a}{94/02/02}{Use `@input@ to load theorem layout files}
%    \begin{macrocode}
\gdef\th@plain{\@input@{thp.sty}}
\gdef\th@break{\@input@{thb.sty}}
\gdef\th@marginbreak{\@input@{thmb.sty}}
\gdef\th@changebreak{\@input@{thcb.sty}}
\gdef\th@change{\@input@{thc.sty}}
\gdef\th@margin{\@input@{thm.sty}}
%    \end{macrocode}
%    This list will be expanded when new styles become
%    available.  For testing,  just append new theorem substyles
%    as document options.
% \end{macro}
% \end{macro}
% \end{macro}
% \end{macro}
% \end{macro}
% \end{macro}
%
%
%
% \subsection{Definition of a new theorem set}
%
% As already pointed out, a new theorem environment can be defined in
% three different ways:
% \begin{verbatim}
%    \newtheorem{Lem}{Lemma}
%    \newtheorem{Lem}{Lemma}[section]
%    \newtheorem{Lem}[Theorem]{Lemma}
%\end{verbatim}
% The function of the macro |\newtheorem| is to recognize these cases
% and then to branch into one of the three macros |\@ynthm|, |\@xnthm|
% or |\@othm|.  This mechanism is adopted unchanged from
% \cite{latex:source}; the essential point here is that, for example,
% in the second case, the arguments \texttt{Lem}, \texttt{Lemma} and
% \texttt{section} are passed over to the macro |\@xnthm|.
%
% We inspect this case first because the others present fewer problems,
% and thus are easily derived from this one.
%
% \begin{macro}{\@xnthm}
%    For our example arguments, the macro |\@xnthm| must fulfill
%    the following:
%    \begin{itemize}
%       \item
%          Define a new \LaTeX-counter `Lem'
%       \item
%          reset this counter within a |\section|
%       \item
%          define the macro |\theLem|
%       \item
%          define the environment macros |\Lem| and
%          |\endLem| using the current
%          |\theoremstyle| and |\theorem@bodyfont|.
%    \end{itemize}
%    Obviously, all this should happen only if the first argument
%    of |\@xnthm| (i.e.\ \texttt{Lem} in our example) is chosen so
%    as not to conflict with any previously defined commands or
%    environments.  This test is performed by the \LaTeX{} macro
%    |\@ifdefinable|.
%    \begin{macrocode}
\gdef\@xnthm#1#2[#3]{\expandafter\@ifdefinable\csname #1\endcsname
%    \end{macrocode}
%    Therefore, the first argument of |\@ifdefinable| is the
%    expansion (in the example, |\Lem|) of
%    |\csname||#1\endcsname|.  The second argument is
%    executed only if the test has been completed successfully.
%    \begin{macrocode}
   {%
%    \end{macrocode}
%    Now we define the new counter. The names of the \LaTeX{}
%    macros employed should speak for themselves:
% \changes{v2.2b}{1995/10/07}{Use \cs{@newctr}}
% \changes{v2.2c}{1995/11/23}{Use \cs{@newctr} properly.}
%    \begin{macrocode}
    \@definecounter{#1}\@newctr{#1}[#3]%
%    \end{macrocode}
%    Using |\@newctr| will give a proper error message if the counter
%    in |#3| is not defined.
%    In defining `|\theLem|' we must generate the desired macro
%    name by use of |\expandafter| and |\csname|.
%    \begin{macrocode}
    \expandafter\xdef\csname the#1\endcsname
%    \end{macrocode}
%    An |\xdef| is used in order to make the definition global,
%    and to ensure that it contains the replacement texts of
%    |\@thmcountersep| and |\@thmcounter|.\footnote{These
%    two macros can be defined by the document style.  Their default
%    values produce a `\texttt{.}' as separation and an arabic
%    representation of the number.} However, not everything should be
%    expanded. For example, it saves space to use |\thesection|
%    instead of its---at times---lengthy expansion.
%    \begin{macrocode}
      {\expandafter \noexpand \csname the#3\endcsname
       \@thmcountersep \@thmcounter{#1}}%
%    \end{macrocode}
%    Thus with the defaults of \LaTeX{}, |\theLem| would be
%    replaced by the command sequence |\thesection.\arabic{Lem}|.
%
%    We will now look at the definition of the macro which is executed
%    at the beginning of the actual environment (in our example this
%    macro is |\Lem|). It should be noted that we use an
%    ``|\expandafter| trick'' to expand only certain parts of the
%    replacement text at the time of the definition.
%    \begin{macrocode}
    \def\@tempa{\global\@namedef{#1}}%
    \expandafter \@tempa \expandafter{%
%    \end{macrocode}
%    First, the macro that contains the current definitions of
%    |\@begintheorem| and |\@opargtheorem| should be called
%    up.  The name of this macro---as is already known---has the form
%    |\th@|\meta{theorem\ style}; therefore, it must be called by
%    \begin{macrocode}
      \csname th@\the \theorem@style
%    \end{macrocode}
%    In addition the default theorem font should be changeable, i.e.\
%    we have to insert the contents of |\theorem@bodyfont|. For
%    that reason, we expand even further, beyond |\endcsname|,
%    and thus insert the contents of the \textsf{token} register
%    |\theorem@bodyfont| in the replacement text.
%    \begin{macrocode}
            \expandafter \endcsname \the \theorem@bodyfont
%    \end{macrocode}
%    Now it is time to call the macro |\@thm| which takes over
%    the further processing.  It has two arguments: the current
%    counter name (in our example, \texttt{Lem}), and the text of the
%    label (in our example, \texttt{Lemma}).
%    \begin{macrocode}
     \@thm{#1}{#2}}%
%    \end{macrocode}
%    With this, the `sub-definition' is complete.  The macro
%    |\@endtheorem| ends a theorem environment and is, so far, nothing
%    but an |\endtrivlist|.  (Hence it is defined globally, and not
%    within the theorem styles.\footnote{This has to be changed as
%    soon as theorem styles that change \texttt{\bslash @endtheorem}
%    exist.  In such a case, all existing styles must be changed as
%    well since they will have to reset the macro.}) Therefore, we can
%    set it equivalent to the macro that ends the theorem set (in our
%    example, |\endLem|).  However, if some day theorem styles exist
%    that do change |\@endtheorem|, we would have to use the
%    commented-out line instead.
%    \begin{macrocode}
    \global \expandafter \let \csname end#1\endcsname \@endtheorem
%   \global\@namedef{end#1}{\@endtheorem}%
%    \end{macrocode}
%    With these commands all the required definitions are employed,
%    unless the test |\@ifdefinable| has failed.  Therefore, we
%    end the second argument of this macro and with it the definition
%    of |\@xnthm|.
%    \begin{macrocode}
   }}
%    \end{macrocode}
% \end{macro}
%
% \begin{macro}{\@ynthm}
%    The definition of |\@ynthm| is completely analogous. In
%    this case the new counter that is defined is not reset within
%    another counter; thus the definition of |\the...|
%    is simplified:
%    \begin{macrocode}
\gdef\@ynthm#1#2{\expandafter\@ifdefinable\csname #1\endcsname
   {\@definecounter{#1}%
    \expandafter\xdef\csname the#1\endcsname{\@thmcounter{#1}}%
%    \end{macrocode}
%    The rest of the definition corresponds literally to that of
%    |\@xnthm|:
%    \begin{macrocode}
    \def\@tempa{\global\@namedef{#1}}\expandafter \@tempa
     \expandafter{\csname th@\the \theorem@style \expandafter
     \endcsname \the\theorem@bodyfont \@thm{#1}{#2}}%
    \global \expandafter \let \csname end#1\endcsname \@endtheorem}}
%    \end{macrocode}
% \end{macro}
%
% \begin{macro}{\@othm}
%    The definition of |\@othm| does not contain anything new.
%    \begin{macrocode}
\gdef\@othm#1[#2]#3{%
%    \end{macrocode}
% \changes{v2.2b}{95/10/07}{Build in check for counter defined}
%    We do not define a new counter but instead use one that has
%    already been defined.  Thus the only definition we need is that
%    of this pseudo-counter (i.e.\ |\the|\meta{env.\ name}).
%    First we check if |#2| corresponds to a known counter name.
%    \begin{macrocode}
  \expandafter\ifx\csname c@#2\endcsname\relax
   \@nocounterr{#2}%
  \else
   \expandafter\@ifdefinable\csname #1\endcsname
   {\expandafter \xdef \csname the#1\endcsname
     {\expandafter \noexpand \csname the#2\endcsname}%
%    \end{macrocode}
%    All other parts of the definition can be adopted from
%    |\@xnthm|. We have to remember, though, that in this case
%    the name of the current counter and the theorem label have moved
%    to the second and third arguments.
%    \begin{macrocode}
    \def\@tempa{\global\@namedef{#1}}\expandafter \@tempa
     \expandafter{\csname th@\the \theorem@style \expandafter
     \endcsname \the\theorem@bodyfont \@thm{#2}{#3}}%
    \global \expandafter \let \csname end#1\endcsname \@endtheorem}%
  \fi}
%    \end{macrocode}
% \end{macro}
%
% \subsection{Macros that are employed in a {\envfont theorem}
%    environment}
%
% \begin{macro}{\@thm}
%    The macro |\@thm| has to increase the current counter.  Then,
%    depending on whether the environment has (or does not have) an
%    optional argument, it has to branch into either |\@begintheorem|
%    or |\@opargtheorem|.
%    \begin{macrocode}
\gdef\@thm#1#2{\refstepcounter{#1}%
%    \end{macrocode}
%    Now we start a {\envfont trivlist} environment, and give
%    |\@topsep| and |\@topsepadd| the values of the \textsf{skip}
%    registers |\theorempreskipamount| and |\theorempostskipamount|.
%    The value in |\@topsep| is the vertical space that is inserted by
%    the first (and only) |\item| in our |\trivlist| whilst
%    |\@topsepadd| is inserted by |\@endparenv| at the end of that
%    {\envfont trivlist} environment.  By using these registers, we
%    obtain the desired space around a {\envfont theorem} environment.
%    \begin{macrocode}
   \trivlist
   \@topsep \theorempreskipamount               % used by first \item
   \@topsepadd \theorempostskipamount           % used by \@endparenv
%    \end{macrocode}
%    Now we have to test whether an optional argument has been given.
% \changes{v2.0d}{89/07/18}{Forgotten percent added.}
%    \begin{macrocode}
   \@ifnextchar [%
%    \end{macrocode}
%    If there is an optional argument, we will call |\@ythm|, and
%    move the arguments read back into the input stream.
%    \begin{macrocode}
   {\@ythm{#1}{#2}}%
%    \end{macrocode}
%    If not, we call |\@begintheorem|. Its first argument is the
%    name of the theorem set (hence the second argument of
%    |\@thm|). Its second argument is the macro that produces the
%    current number.
%    \begin{macrocode}
   {\@begintheorem{#2}{\csname the#1\endcsname}\ignorespaces}}
%    \end{macrocode}
% \end{macro}
%
% \begin{macro}{\@xthm}
% \begin{macro}{\@ythm}
%    Both these macros were originally called by |\@thm|.  We do
%    not need |\@xthm| anymore, hence we reset it to
%    |\relax|. The definition of |\@ythm| has not changed at
%    all from its definition in \LaTeX. In order to make the macros
%    easier to understand, we will
%    nevertheless present it (commented out).
%    \begin{macrocode}
\global\let\@xthm\relax
% \def\@ythm#1#2[#3]{\@opargbegintheorem{#2}{\csname
%           the#1\endcsname}{#3}\ignorespaces}
%</package>
%    \end{macrocode}
%    The primitive |\ignorespaces| in |\@ythm| and
%    |\@thm| is needed to remove the spaces between the
%    |\begin{...}| and the actual text.
% \end{macro}
% \end{macro}
%
%
% \subsection{Definition of the theorem substyles}
%
% As already pointed out, the theorem substyles, defined below, are
% only loaded when necessary.  Note that all these substyles, except
% \texttt{plain}, have |\slshape| as the default body font.
%
% \subsubsection{The \texttt{plain} style}
%
% \label{dok:th@plain} As the following macros use |@|, we have
% to locally set the |\catcode| of this symbol to ``letter''.
% This happens within a group, so that we do not have to worry about
% which |\catcode| that symbol had before.
%    \begin{macrocode}
%<*thp>
\begingroup \makeatletter
%    \end{macrocode}
% Since we are now within a group, we must make all definitions
% globally.  First we make sure that \texttt{theorem.sty} is loaded.
% This will allow us to use this file as a document style option
% without having to call \texttt{theorem} itself as an option.  At the
% same time, we assure that at least version 2 is loaded, since
% |\theorem@style| was not defined in earlier versions.
%    \begin{macrocode}
\@ifundefined{theorem@style}{\input{theorem.sty}}{}
\ProvidesFile{thp.sty}
             [\FMithmInfo]
%    \end{macrocode}
%
%
% \begin{macro}{\th@plain}
% \begin{macro}{\@begintheorem}
% \begin{macro}{\@opargbegintheorem}
%    |\theoremstyle{plain}| corresponds to the original
%    definition, except that the distances to the surrounding text are
%    determined by the parameters |\theorempreskipamount| and
%    |\theorempostskipamount|.
%    First we set the default body font.
%    \begin{macrocode}
\gdef\th@plain{\normalfont\itshape
%    \end{macrocode}
%    Then we define |\@begintheorem| and
%    |\@opargbegintheorem|. These two macros define how the
%    header of a theorem is typeset.  |\@opargbegintheorem| will
%    be called if a \textsf{theorem} environment with an optional
%    argument is encountered; otherwise, the header is constructed by
%    calling |\@begintheorem|.  If one of these macros is
%    executed, we are within a \textsf{trivlist} environment started by
%    |\@thm|.  So the theorem header is produced with an
%    |\item| command.
%
%    Instead of specifying the header font directly, all standard
%    theorem styles use the |\theorem@headerfont| macro to allow
%    customization.  The extra space (|\labelsep|) is necessary
%    because of problems in the \textsf{trivlist} environment.
%    \begin{macrocode}
  \def\@begintheorem##1##2{%
        \item[\hskip\labelsep \theorem@headerfont ##1\ ##2]}%
%    \end{macrocode}
%    The definition of |\@opargbegintheorem| is completely
%    analogous. The only difference is the fact that there exists a
%    third argument (which is the optional parameter of the
%    environment and contains additional information about the
%    theorem).  Customarily we enclose it in parentheses.
%    \begin{macrocode}
\def\@opargbegintheorem##1##2##3{%
   \item[\hskip\labelsep \theorem@headerfont ##1\ ##2\ (##3)]}}
%    \end{macrocode}
% \end{macro}
% \end{macro}
% \end{macro}
%
% \noindent
% We conclude with an |\endgroup| to restore the
% |\catcode| of |@|.
%    \begin{macrocode}
\endgroup
%</thp>
%    \end{macrocode}
%
%
% \subsubsection{The \texttt{break} style}
%
% This style option is stored in the file |thb.sty|.
% For the next two lines see the documentation for |\th@plain| on
% page~\pageref{dok:th@plain}.
%    \begin{macrocode}
%<*thb>
\begingroup \makeatletter
\@ifundefined{theorem@style}{\input{theorem.sty}}{}
\ProvidesFile{thb.sty}
             [\FMithmInfo]
%    \end{macrocode}
%
% \begin{macro}{\th@break}
%    |\theoremstyle{break}| produces a line break after the name
%    of the theorem.  The font is |\slshape|.  Hence, we define
%    |\th@break| as follows:
%    \begin{macrocode}
\gdef\th@break{\normalfont\slshape
  \def\@begintheorem##1##2{\item[%
%    \end{macrocode}
%    We run into the following problem: it is not possible to create
%    the header with |\item[|\meta{title}|]| and then start
%    a new line by, for example, |\mbox{}\\|. Such a definition
%    will fail whenever a list environment follows immediately.  With
%    the above construction, the |\mbox{}| causes the switch
%    |@inlabel| (cf.\ definition of |\list| and
%    |\trivlist| in \cite{latex:source}) to be set to \texttt{false}
%    and so the following list will insert additional vertical
%    space (|\topskip|). This is quite annoying. Therefore, we
%    create the line break within the |\item|. In order to ensure
%    that the text will begin at the proper position in the following
%    line, we simply pretend that the label does not take any
%    room.\footnote{This will lead to problems whenever very high
%    symbols occurring in the line tower into the
%    heading. So, something else has to be done here sometime.}
%    \begin{macrocode}
       \rlap{\vbox{\hbox{\hskip \labelsep\theorem@headerfont ##1\ ##2}%
                   \hbox{\strut}}}]}%
%    \end{macrocode}
%    Again, the definition of |\@opargbegintheorem| is
%    completely analogous.
%    \begin{macrocode}
\def\@opargbegintheorem##1##2##3{%
  \item[\rlap{\vbox{\hbox{\hskip \labelsep \theorem@headerfont
                     ##1\ ##2\ (##3)}%
                    \hbox{\strut}}}]}}
\endgroup
%</thb>
%    \end{macrocode}
% \end{macro}
%
%
% \subsubsection{The \texttt{changebreak} style}
%
%    \begin{macrocode}
% This style option is stored in the file |thcb.sty|.
%    \begin{macrocode}
%<*thcb>
\begingroup \makeatletter
\@ifundefined{theorem@style}{\input{theorem.sty}}{}
\ProvidesFile{thcb.sty}
             [\FMithmInfo]
%    \end{macrocode}
%
% \begin{macro}{\th@changebreak}
%    The \texttt{change-break} theorem style is like \texttt{break}
%    but with interchange of theorem name and theorem number.  Thus we
%    define |\th@changebreak| as follows:
%    \begin{macrocode}
\gdef\th@changebreak{\normalfont\slshape
  \def\@begintheorem##1##2{\item
       [\rlap{\vbox{\hbox{\hskip\labelsep \theorem@headerfont ##2\ ##1}%
                    \hbox{\strut}}}]}%
\def\@opargbegintheorem##1##2##3{%
  \item[\rlap{\vbox{\hbox{\hskip\labelsep \theorem@headerfont
                     ##2\ ##1\ (##3)}%
                    \hbox{\strut}}}]}}
\endgroup
%</thcb>
%    \end{macrocode}
% \end{macro}
%
%
% \subsubsection{The \texttt{change} style}
%
% This style option is stored in the file |thc.sty|.
%    \begin{macrocode}
%<*thc>
\begingroup \makeatletter
\@ifundefined{theorem@style}{\input{theorem.sty}}{}
\ProvidesFile{thc.sty}
             [\FMithmInfo]
%    \end{macrocode}
%
% \begin{macro}{\th@change}
%    The \texttt{change} theorem style corresponds to the
%    \texttt{change break} style without a linebreak after the header.
%    To say it in another way, it's the same as the \texttt{plain}
%    style but with number and name interchanged and |\slshape| as the
%    default font.
%    \begin{macrocode}
\gdef\th@change{\normalfont\slshape
  \def\@begintheorem##1##2{\item
       [\hskip\labelsep \theorem@headerfont ##2\ ##1]}%
\def\@opargbegintheorem##1##2##3{%
  \item[\hskip\labelsep \theorem@headerfont ##2\ ##1\ (##3)]}}
\endgroup
%</thc>
%    \end{macrocode}
% \end{macro}
%
%
% \subsubsection{The \texttt{marginbreak} style}
%
% This style option is the one used most often at Mainz.  It is saved
% in the file |thmb.sty|.
%    \begin{macrocode}
%<*thmb>
\begingroup \makeatletter
\@ifundefined{theorem@style}{\input{theorem.sty}}{}
\ProvidesFile{thmb.sty}
             [\FMithmInfo]
%    \end{macrocode}
%
% \begin{macro}{\th@marginbreak}
%    The \texttt{margin break} style is nearly the same as the
%    \texttt{change break} style. The only difference is the placement
%    of the theorem number.  We use |\llap| to place it in the left
%    margin.
%
%    In this style |\labelsep| denotes the separation between the
%    number and the text.
%    \begin{macrocode}
\gdef\th@marginbreak{\normalfont\slshape
  \def\@begintheorem##1##2{\item
       [\rlap{\vbox{\theorem@headerfont
                    \hbox{\llap{##2}\hskip\labelsep ##1}%
                    \hbox{\strut}}}]}%
\def\@opargbegintheorem##1##2##3{%
  \item[\rlap{\vbox{\theorem@headerfont
                    \hbox{\llap{##2}\hskip\labelsep ##1\ (##3)}%
                    \hbox{\strut}}}]}}
\endgroup
%</thmb>
%    \end{macrocode}
% \changes{v2.1c}{90/02/17}{`endgroup outside macrocode env.}
% \end{macro}
%
%
%
% \subsubsection{The \texttt{margin} style}
%
% This style option is stored in the file |thm.sty|.
%    \begin{macrocode}
%<*thm>
\begingroup \makeatletter
\@ifundefined{theorem@style}{\input{theorem.sty}}{}
\ProvidesFile{thm.sty}
             [\FMithmInfo]
%    \end{macrocode}
%
% \begin{macro}{\th@margin}
%    Again this is only a variant of the theorem styles described
%    above without any new ideas.
%    \begin{macrocode}
\gdef\th@margin{\normalfont\slshape
  \def\@begintheorem##1##2{\item
       [\theorem@headerfont \llap{##2}\hskip\labelsep ##1]}%
\def\@opargbegintheorem##1##2##3{%
  \item[\theorem@headerfont \llap{##2}\hskip\labelsep ##1\ (##3)]}}
\endgroup
%</thm>
%    \end{macrocode}
% \end{macro}
%
%
%
%
% \subsection{Final Definitions}
%
% \begin{macro}{\theorempreskipamount}
% \begin{macro}{\theorempostskipamount}
%    The \textsf{skip} parameters that regulate the vertical empty space
%    before and after the {\envfont theorem} environment have
%    to be allocated as well.
% \changes{v2.0f}{89/07/20}{Unnecessary `global' removed.}
%    \begin{macrocode}
%<*package>
\newskip\theorempreskipamount
\newskip\theorempostskipamount
%    \end{macrocode}
%    Since we have used the same values for all theorem sets, we now can
%    assign them.
%    \begin{macrocode}
\global\setlength\theorempreskipamount{12pt plus 5pt minus 3pt}
\global\setlength\theorempostskipamount{8pt plus 3pt minus 1.5pt}
%    \end{macrocode}
% \end{macro}
% \end{macro}
%
%
% \begin{macro}{\@endtheorem}
%    The same holds for the macro |\@endtheorem|, which ends a
%    {\envfont theorem} environment. Since it is the same for all
%    theorem sets, it is removed from the macros |\th@|\meta{style}.
%    It simply ends the {\envfont trivlist} environment, which was
%    begun in |\@thm|.
%    \begin{macrocode}
\global\let\@endtheorem=\endtrivlist
%    \end{macrocode}
% \end{macro}
%
%
%
% \begin{macro}{\@preamblecmds}
%    All macros defined above are to be used only in the preamble.
%    Therefore, we insert them in |\@preamblecmds| which will disable
%    them at begin document. This is done by the internal \LaTeXe{}
%    command |\@onlypreamble|.
% \changes{v2.0g}{89/09/19}{`theoremheaderfont and `theorembodyfont
%    added.}
% \changes{v2.2a}{94/02/03}{Use \LaTeXe{} interface}
%    \begin{macrocode}
\@onlypreamble\@xnthm
\@onlypreamble\@ynthm
\@onlypreamble\@othm
\@onlypreamble\newtheorem
\@onlypreamble\theoremstyle
\@onlypreamble\theorembodyfont
\@onlypreamble\theoremheaderfont
%    \end{macrocode}
% \end{macro}
%
% Finally we declare the \texttt{plain} theorem style to be the default.
% \changes{v2.1b}{90/06/04}{Fixed missing `item by declaring default
%    style.}
%    \begin{macrocode}
\theoremstyle{plain}
%</package>
%    \end{macrocode}
%
% \Finale

\endinput
}{}
\ProvidesFile{thb.sty}
             [\FMithmInfo]
%    \end{macrocode}
%
% \begin{macro}{\th@break}
%    |\theoremstyle{break}| produces a line break after the name
%    of the theorem.  The font is |\slshape|.  Hence, we define
%    |\th@break| as follows:
%    \begin{macrocode}
\gdef\th@break{\normalfont\slshape
  \def\@begintheorem##1##2{\item[%
%    \end{macrocode}
%    We run into the following problem: it is not possible to create
%    the header with |\item[|\meta{title}|]| and then start
%    a new line by, for example, |\mbox{}\\|. Such a definition
%    will fail whenever a list environment follows immediately.  With
%    the above construction, the |\mbox{}| causes the switch
%    |@inlabel| (cf.\ definition of |\list| and
%    |\trivlist| in \cite{latex:source}) to be set to \texttt{false}
%    and so the following list will insert additional vertical
%    space (|\topskip|). This is quite annoying. Therefore, we
%    create the line break within the |\item|. In order to ensure
%    that the text will begin at the proper position in the following
%    line, we simply pretend that the label does not take any
%    room.\footnote{This will lead to problems whenever very high
%    symbols occurring in the line tower into the
%    heading. So, something else has to be done here sometime.}
%    \begin{macrocode}
       \rlap{\vbox{\hbox{\hskip \labelsep\theorem@headerfont ##1\ ##2}%
                   \hbox{\strut}}}]}%
%    \end{macrocode}
%    Again, the definition of |\@opargbegintheorem| is
%    completely analogous.
%    \begin{macrocode}
\def\@opargbegintheorem##1##2##3{%
  \item[\rlap{\vbox{\hbox{\hskip \labelsep \theorem@headerfont
                     ##1\ ##2\ (##3)}%
                    \hbox{\strut}}}]}}
\endgroup
%</thb>
%    \end{macrocode}
% \end{macro}
%
%
% \subsubsection{The \texttt{changebreak} style}
%
%    \begin{macrocode}
% This style option is stored in the file |thcb.sty|.
%    \begin{macrocode}
%<*thcb>
\begingroup \makeatletter
\@ifundefined{theorem@style}{% \iffalse meta-comment
%
% Copyright (C) 1993-2024
%
% The LaTeX Project and any individual authors listed elsewhere
% in this file.
%
% This file is part of the Standard LaTeX `Tools Bundle'.
% -------------------------------------------------------
%
% It may be distributed and/or modified under the
% conditions of the LaTeX Project Public License, either version 1.3c
% of this license or (at your option) any later version.
% The latest version of this license is in
%    https://www.latex-project.org/lppl.txt
% and version 1.3c or later is part of all distributions of LaTeX
% version 2005/12/01 or later.
%
% The list of all files belonging to the LaTeX `Tools Bundle' is
% given in the file `manifest.txt'.
%
% \fi
%
%%
%
%
\def\FMithmInfo{2023/07/05 v2.2c Theorem extension package (FMi)}
%
% \ProvidesFile{theorem.dtx}[\FMithmInfo]
%\iffalse   % this is a METACOMMENT !
%%
%% Package `theorem' to use with LaTeX2e
%% Copyright (C) 1989-1995 Frank Mittelbach, all rights reserved.
%<+package>\@ifundefined{theorem@style}{}{\endinput}
%<+package>\NeedsTeXFormat{LaTeX2e}[1995/06/01]
%<+package>\ProvidesPackage{theorem}
%<+package>         [\FMithmInfo]
%
% \fi
%
% \changes{v1.0}{87/05/07}{First draft.}
% \changes{v1.0a}{87/05/07}{`thistheoremstyle' added.}
% \changes{v1.1a}{87/07/29}{New concept: redefine the internal
%                          \protect\LaTeX{} commands `@xnthm' etc.}
% \changes{v1.1b}{87/07/31}{`theoremfont' used to specify font.}
% \changes{v1.2a}{88/01/08}{`@endtheorem' now globally defined.}
% \changes{v1.2b}{88/01/11}{`newtheorem' only allowed in the preamble.}
% \changes{v1.3a}{88/02/14}{`theoremfont' now a toks register.}
%
% \changes{v2.0a}{88/11/16}{New implementation and new documentation.}
% \changes{v2.0a}{88/11/16}{No layout changes but internal modifications
%                           old substyles won't work.}
% \changes{v2.0b}{89/01/30}{`global' used for all definitions.}
% \changes{v2.0c}{89/06/15}{Documentation translated to English.
%                            Thanks to LLL and Nina}
% \changes{v2.1a}{90/02/15}{Support for new font selection.}
% \changes{v2.2a}{94/02/03}{Cleanup for LaTeX2e}
%
% \DoNotIndex{\@endparenv,\@ifdefinable,\@ifnextchar,\@ifundefined}
% \DoNotIndex{\@namedef,\@nameuse,\@spaces,\@tempa,\@warning}
% \DoNotIndex{\ }
% \DoNotIndex{\begingroup,\bfseries}
% \DoNotIndex{\csname}
% \DoNotIndex{\def,\do,\docdate}
% \DoNotIndex{\endcsname,\endgroup,\endinput,\endtrivlist,\expandafter}
% \DoNotIndex{\filedate,\fileversion}
% \DoNotIndex{\gdef,\global}
% \DoNotIndex{\hbox,\hskip}
% \DoNotIndex{\ignorespaces,\input,\itshape,\item}
% \DoNotIndex{\labelsep,\let,\llap}
% \DoNotIndex{\makeatletter}
% \DoNotIndex{\noexpand}
% \DoNotIndex{\refstepcounter,\relax,\rlap}
% \DoNotIndex{\setlength,\slshape,\space,\string,\strut}
% \DoNotIndex{\the,\trivlist,\typeout}
% \DoNotIndex{\vbox}
% \DoNotIndex{\xdef}
%
% \def\deflabel#1{\ttfamily #1\hfill}
% \def\deflist#1{\begin{list}{}{\settowidth\labelwidth{\ttfamily #1}%
%                               \setlength\leftmargin\labelwidth
%                               \addtolength\leftmargin\labelsep
%                               \let\makelabel\deflabel}}
% \def\enddeflist{\end{list}}
% \def\envfont{\normalfont\ttfamily}
%
% \MakeShortVerb{\|}
%
% \newcommand{\TUB}{{\sl TUGboat\/}}
%
% \GetFileInfo{theorem.dtx}
% \title{An Extension of the
%        \LaTeX{} theorem environment\thanks{This file
%        has version number \fileversion, last
%        revised \filedate.}}
% \author{Frank Mittelbach}
%
% \MaintainedByLaTeXTeam{tools}
% \maketitle
%
% \begin{abstract}
%    The macros described in this paper yield an extension of the
%    \LaTeX{} theorem mechanism. It is designed to satisfy the
%    different requirements of various journals. Thus, the
%    layout of the ``theorems'' can be manipulated by determining a
%    ``style''. This article describes not only the use, but
%    also the definition, of the necessary macros.
% \end{abstract}
%
% \section*{Preface to version 2.2}
%
% For \LaTeXe{} this package did not need any fundamental changes. I
% only modified the messages generated so that theorem layout styles
% will show up with the |\listfiles| command and cleaned the section
% on the New Font Selection Scheme since this is now included in
% \LaTeX.
%
% \section*{Preface to version 2.1}
%
% This version is identical to 2.0g described in \TUB\ 10\#3 except
% for some internal defaults which are now set depending on the used
% font selection scheme.
%
% This was done to avoid unpleasant surprises if the new font selection
% scheme is in force. For further details see
% section~\ref{sec:newfonts} and \cite{bk:GMS94}.
%
% \section{Introduction}
%
% For our purposes here, ``theorems'' are labelled enunciations,
% often set off from the main text by extra space and a font change.
% Theorems, corollaries, conjectures, definitions, and remarks
% are all instances of ``theorems''.  The ``header'' of these structures
% is composed of a label (such as \textsc{Theorem} or \textsc{Remark})
% and a number which serializes an item in the sequence of items
% with the same label.
%
% Shortly after the introduction of \LaTeX{} at the Fachbereich
% Mathematik in Mainz, the desire to manipulate the layout of
% ``theorems'' arose.  In Mainz, the following two conventions came
% into general use:
% \begin{enumerate}
%   \item The number of the theorem is shown in the margin.
%   \item There is a line break at
%         the end of the theorem header.
% \end{enumerate}
%
% Additionally, some journals require different formats which depend on
% the ``sort of theorem'': e.g.\ often remarks and definitions are set
% in |\upshape|, while |\itshape| is employed for main theorems.
%
% Confronted with these requirements, a theorem environment was
% developed in Mainz which allows separate determination of the
% layout of the ``theorems sets'', comparable to |\pagestyle|.
%
%
% \section{The user interface}
% \subsection{Defining new theorem sets}
%
% \DescribeMacro\newtheorem
% As in the original \LaTeX{} version, the command |\newtheorem|
% defines a new ``theorem set'' or ``theorem-like structure''.
% Two required arguments name the new environment and give the text
% to be typeset with each instance of the new ``set'', while
% an optional argument determines how
% the ``set'' is enumerated:
% \begin{description}
%    \item[\ttfamily \bslash newtheorem\{foo\}\{bar\}]
%       The theorem set {\envfont foo} (whose name is \texttt{bar})
%       uses its own counter.
%    \item[\ttfamily \bslash newtheorem\{foo2\}{[foo]}\{bar2\}]
%       The theorem set {\envfont foo2} (printed name \texttt{bar2})
%       uses the same counter as the theorem set \texttt{foo}.
%    \item[\ttfamily \bslash newtheorem\{foo3\}\{bar3\}{[section]}]
%       The theorem set {\envfont foo3} (printed name \texttt{bar3}) is
%       enumerated within the counter \texttt{section}, i.e.\ with every
%       new |\section| the enumeration begins again with 1, and
%       the enumeration is composed from the section-number and the
%       theorem counter itself.
% \end{description}
%
% \DescribeMacro\theoremstyle
% Additionally, the command |\theoremstyle| can define the layout
% of various, or all, theorem sets.  It should be noted that any
% theorem set defined by |\newtheorem| is typeset in the
% |\theoremstyle| that is current at the time of the definition.
% Thus, the following
% \begin{verbatim}
%    \theoremstyle{break}        \newtheorem{Cor}{Corollary}
%    \theoremstyle{plain}        \newtheorem{Exa}{Example}[section]
%\end{verbatim}                             ^^A No blank at line start
% leads to the result that the set {\envfont Cor} is formatted in the
% style \texttt{break}, while the set {\envfont Exa} and all the
% following ones are formatted in the style \texttt{plain}, unless
% another |\theoremstyle| follows.  Since the definitions installed by
% |\newtheorem| are global, one also can limit |\theoremstyle| locally
% by grouping braces.
%
% \DescribeMacro\theorembodyfont
% The choice of the font for the theorem body is completely
% independent of the chosen |\theoremstyle|; this has proven to
% be very advantageous.  For example,
% \begin{verbatim}
%   {\theorembodyfont{\upshape}          \newtheorem{Rem}{Remark}}
%\end{verbatim}
% defines a theorem set {\envfont Rem}, which will be set in
% |\upshape| in the current layout (which in our example is
% \texttt{plain}).  As with |\theoremstyle|, the |\theorembodyfont|
% chosen is that current at the time of |\newtheorem|.  If
% |\theorembodyfont| is not specified or one defines
% |\theorembodyfont{}|, then the font used will be that defined by the
% |\theoremstyle|.
%
% \DescribeMacro\theoremheaderfont It is also possible to customize
% the font used for the theorem headers.  This is, however, a global
% declaration, and therefore there should be at most one
% |\theoremheaderfont| declaration in the preamble.\footnote{If it is
% actually necessary to have different header fonts, one has to define
% new theorem styles (substituting the desired font) or specify the
% information directly in the \texttt{\bslash newtheorem} declaration
% (the unclean variant).}
%
% \DescribeMacro\theorempreskipamount
% \DescribeMacro\theorempostskipamount Two additional parameters
% affect the vertical space around the theorem environments:
% |\theorempreskipamount| and |\theorempostskipamount| define,
% respectively, the spacing before and after such an environment.
% These parameters apply for all theorem sets and can be manipulated
% with the ordinary length macros.  They are rubber lengths
% (`\textsf{skips}'), and therefore can contain \texttt{plus} and
% \texttt{minus} parts.
%
% Since the definition of theorem sets should---most sensibly---be
% placed in the preamble, we only allow installation there.
% It is therefore possible to release the memory used here after
% |\begin{document}|, in order to make room for other
% applications.
%
%
% \subsection{Existing theorem styles}
%
% The following theorem styles exist to date:
% \begin{deflist}{marginbreak}
%    \item[plain]
%       This theorem style emulates the original \LaTeX{} definition,
%       except that additionally the parameters
%       |\theorem...skipamount| are used.
%    \item[break]
%       In this style, the theorem header is followed by a line
%       break.
%    \item[marginbreak]
%       The theorem number is set in the margin, and there is a line
%       break as in \texttt{break}.
%    \item[changebreak]
%       Like \texttt{break}, but with header number and text
%       interchanged.
%    \item[change]
%       Header number and text are interchanged, without a line break.
%    \item[margin]
%       The number is set in the left margin, without a line break.
% \end{deflist}
% All styles (except \texttt{plain}) select |\slshape| as the default
% |\theorembodyfont|.
%
%
% \subsection{Examples}
%
% Given the above theorem sets {\envfont Cor}, {\envfont Exa}
% and {\envfont Rem},
% suppose that the preamble also contains the declarations:
% \begin{verbatim}
%    \theoremstyle{marginbreak}   \newtheorem{Lem}[Cor]{Lemma}
%    \theoremstyle{change}
%    \theorembodyfont{\itshape}        \newtheorem{Def}[Cor]{Definition}
%
%    \theoremheaderfont{\scshape}
%\end{verbatim}
% Then the following are some typical examples of the typeset output
% resulting from their use.
%
% \begin{Cor}
%    This is a sentence typeset in the theorem environment {\envfont
%    Cor}.
% \end{Cor}
%
% \begin{Exa}
%    This is a sentence typeset in the theorem environment {\envfont
%    Exa}.
% \end{Exa}
%
% \begin{Rem}
%    This is a sentence typeset in the theorem environment {\envfont
%    Rem}.
% \end{Rem}
%
% \begin{Lem}[Ben User]
%    This is a sentence typeset in the theorem environment {\envfont
%    Lem}.
% \end{Lem}
%
% \begin{Def}[Very impressive Definition]
%    This is a sentence typeset in the theorem environment {\envfont
%    Def}.
% \end{Def}
%
% The last two examples show the effect of the optional argument to
% a theorem environment (it is the text typeset in parentheses).
%
%
% \section{Special Considerations}
% \label{sec:newfonts}
%
% Theoremheader and body are implemented as a unit. This means that
% the |\theoremheaderfont|  will inherit characteristics of the
% |\theorembodyfont| in \LaTeXe. Thus, if for example
% |\theorembodyfont| is |\itshape| and |\theoremheaderfont|
% is |\bfseries| the font selected for the header will have the
% characteristics `bold extended italic'. If this is not desired one
% should set the |\theoremheaderfont| to something like
% \begin{verbatim}
%      \theoremheaderfont{\normalfont\bfseries}
%\end{verbatim}
% i.e.\ supplying all necessary font information explicitly.
%
%
%
% \section{Acknowledgements}
%
% The publication of this set of macros was only possible with the
% help of Christina Busse (translating the manuscript into English),
% Joachim Pense (playing the r\^ole of typist), Chris Rowley
% (looking everything over) and many others providing useful
% suggestions.
%
% \MaybeStop{%
%    \begin{thebibliography}{9}
%       \bibitem{bk:GMS94} \textsc{M.~Goossens}, \textsc{F.~Mittelbach}
%         and \textsc{A.~Samarin}.
%         \newblock The \LaTeX{} Companion.
%         \newblock
%         Addison-Wesley, Reading, Massachusetts, 1994.
%       \bibitem{latex:source} \textsc{Lamport, Leslie.} \newblock
%                               \texttt{latex.tex}, version 2.09, date
%                               Feb. 1990.
%    \end{thebibliography}
%    \PrintIndex
%    \PrintChanges}
%
%
% \section{The documentation driver file}
%
% The next bit of code contains the documentation driver file for
% \TeX{}, i.e., the file that will produce the documentation you are
% currently reading. It will be extracted from this file by the
% \texttt{docstrip} program. Since it is the first code in the file
% one can alternatively process this file directly with \LaTeXe{} to
% obtain the documentation.
%
%    \begin{macrocode}
%<*driver>
\documentclass{ltxdoc}

\usepackage{theorem}

%<+driver>% The next few lines define theorem sets which are used
%<+driver>% in the example section of the documentation.

\theoremstyle{break}       \newtheorem{Cor}{Corollary}
\theoremstyle{plain}       \newtheorem{Exa}{Example}[section]
{\theorembodyfont{\upshape}\newtheorem{Rem}{Remark}}
\theoremstyle{marginbreak} \newtheorem{Lem}[Cor]{Lemma}
\theoremstyle{change}
\theorembodyfont{\itshape} \newtheorem{Def}[Cor]{Definition}

\theoremheaderfont{\scshape}

\RecordChanges

\begin{document}
   \DocInput{theorem.dtx}
\end{document}
%</driver>
%    \end{macrocode}
%
% \section{Definition of the Macros}
%
% If the file has been loaded before, we abort immediately. If not the
% package announces itself (this is actually done at the very top of
% the file---the way it is done isn't good style so don't copy it).
% \changes{v2.0e}{89/07/19}{Spaces removed from `typeout'.}
% \changes{v2.2b}{95/11/19}{Announce moved to top of file}
%    \begin{macrocode}
%<*package>
%\@ifundefined{theorem@style}{}{\endinput}
%\def\FMithmInfo{1995/11/19 v2.2b Theorem extension package (FMi)}
%\ProvidesPackage{theorem}[\FMithmInfo]
%    \end{macrocode}
%
%
% \subsection{Definition of theorem styles and fonts}
%
% All the definitions in this file are done globally to allow
% inputting this file inside a group.
%
% \begin{macro}{\theoremstyle}
%    Before a theorem style can be installed, the chosen style must be
%    known. For that reason, we must test to see that
%    |\th@|\meta{style} is known or, more precisely, that it is
%    different from |\relax|. If the style is not known then
%    |\th@plain| is used.
%    \begin{macrocode}
\gdef\theoremstyle#1{%
   \@ifundefined{th@#1}{\@warning
          {Unknown theoremstyle `#1'. Using `plain'}%
          \theorem@style{plain}}%
%    \end{macrocode}
%    We save the theorem style to be used in the token register
%    |\theorem@style|.
%    \begin{macrocode}
      {\theorem@style{#1}}%
%    \end{macrocode}
%    Now we ``evaluate'' the theorem style: this means, we call the
%    macro |\th@|\meta{style} which will activate the relevant
%    definitions which are contained in a separate file.  This is done
%    in a group to suppress changes to the current font.  This could
%    otherwise pose problems together with the new font selection
%    scheme\footnote{When I printed the original article using the new
%    font selection scheme I ended with a document with slanted
%    typefaces (text headings and all) simply because one of the
%    theorem styles used \texttt{\bslash sl} at toplevel.} if the
%    |\th@|\meta{style} is evaluated a second time.
% \changes{v2.1a}{90/02/15}{Group added around `th@..}
% \changes{v2.1a}{90/02/15}{Reseting of `..@begintheorem removed.}
%    \begin{macrocode}
      \begingroup
        \csname th@\the\theorem@style \endcsname
      \endgroup}
%    \end{macrocode}
% \end{macro}
%
%
% \begin{macro}{\@begintheorem}
% \changes{v2.1a}{90/02/15}{Relaxed at toplevel.}
% \begin{macro}{\@opargbegintheorem}
% \changes{v2.1a}{90/02/15}{Relaxed at toplevel.}
%    We reset |\@begintheorem| and |\@opargbegintheorem| to
%    |\relax| since these commands are no longer necessary at
%    toplevel. This will save a few tokens.
%    \begin{macrocode}
\global\let\@begintheorem\relax
\global\let\@opargbegintheorem\relax
%    \end{macrocode}
% \end{macro}
% \end{macro}
%
% \begin{macro}{\theorem@style}
%    Obviously the \textsf{token} register used above has to be
%    allocated. To assure the utmost compatibility with the
%    original \LaTeX{} definition, we set the default theorem style to
%    \texttt{plain}, which implements the usual \LaTeX{} convention.
% \changes{v2.0f}{89/07/20}{Unnecessary `global' removed.}
%    \begin{macrocode}
\newtoks\theorem@style
\global\theorem@style{plain}
%    \end{macrocode}
% \end{macro}
%
% \begin{macro}{\theorembodyfont}
% \begin{macro}{\theorem@bodyfont}
%    If the |\theorembodyfont| is set by the user then it should not
%    interact with the default font set in the theorem style. When the
%    new font selection is in force this may happen if, for example,
%    the default is |\itshape| and the new |\theorembodyfont| is
%    |\sffamily|. So we add a |\reset@font| command in front of the
%    user definition.
% \changes{v2.0f}{89/07/20}{Unnecessary `global' removed.}
% \changes{v2.0g}{89/09/19}{`theoremfont to `theorembodyfont.}
% \changes{v2.1c}{92/06/22}{`reset@font added and logic changed.}
% \changes{v2.1d}{92/10/28}{Reset to empty if arg is empty}
%    \begin{macrocode}
\gdef\theorembodyfont#1{%
%    \end{macrocode}
%    We check if the argument supplied is empty and if so put nothing
%    into the |\theorem@bodyfont| token register to allow for
%    |\theorembodyfont{}| as a mean of using the default of the
%    current |\theoremstyle|.
%    \begin{macrocode}
   \def\@tempa{#1}%
   \ifx\@tempa\@empty
    \theorem@bodyfont{}%
   \else
    \theorem@bodyfont{\reset@font#1}%
   \fi
   }
\newtoks\theorem@bodyfont
\global\theorem@bodyfont{}
%    \end{macrocode}
% \end{macro}
% \end{macro}
%
%
% \begin{macro}{\theoremheaderfont}
% \changes{v2.0g}{89/09/19}{Macro added.}
%    The font for the theorem headers is handled differently because
%    this definition applies to all theorem styles.
%    \begin{macrocode}
\gdef\theoremheaderfont#1{\gdef\theorem@headerfont{#1}%
%    \end{macrocode}
%    After using the macro once it is redefined to produce an error
%    message.
%    \begin{macrocode}
       \gdef\theoremheaderfont##1{%
        \typeout{\string\theoremheaderfont\space should be used
                 only once.}}}
%    \end{macrocode}
% \end{macro}
%
% \begin{macro}{\theorem@headerfont}
% \changes{v2.0g}{89/09/19}{Macro added.}
%    To set the |\theorem@headerfont| default we first test if
%    the new fontselection scheme is in force.
% \changes{v2.1a}{90/02/15}{Support for new font selection.}
%    \begin{macrocode}
\ifx\upshape\undefined
%    \end{macrocode}
%    If not we define it to expand into |\bfseries|. We don't use
%    |\let| just in case a following style option redefines this
%    macro.
%    \begin{macrocode}
\gdef\theorem@headerfont{\bfseries}
%    \end{macrocode}
%    Otherwise we reset the current shape before calling |\bfseries|.
%    \begin{macrocode}
\else \gdef\theorem@headerfont{\normalfont\bfseries}\fi
%    \end{macrocode}
% \end{macro}
%
% \begin{macro}{\th@plain}
% \begin{macro}{\th@break}
% \begin{macro}{\th@marginbreak}
% \begin{macro}{\th@changebreak}
% \begin{macro}{\th@change}
% \begin{macro}{\th@margin}
%    The different styles are defined in macros such as
%    |\th@plain|.  Since memory space is precious in
%    ``non-Big-versions'', we have to avoid offering too many unused
%    definitions.  Therefore we define these styles in separate files
%    that can be loaded on demand.  Thus the commands themselves only
%    load these files. We use |\@input@|, a \LaTeXe{} internal command
%    that ensures that the file will be listed with |\listfiles|.
% \changes{v2.2a}{94/02/02}{Use `@input@ to load theorem layout files}
%    \begin{macrocode}
\gdef\th@plain{\@input@{thp.sty}}
\gdef\th@break{\@input@{thb.sty}}
\gdef\th@marginbreak{\@input@{thmb.sty}}
\gdef\th@changebreak{\@input@{thcb.sty}}
\gdef\th@change{\@input@{thc.sty}}
\gdef\th@margin{\@input@{thm.sty}}
%    \end{macrocode}
%    This list will be expanded when new styles become
%    available.  For testing,  just append new theorem substyles
%    as document options.
% \end{macro}
% \end{macro}
% \end{macro}
% \end{macro}
% \end{macro}
% \end{macro}
%
%
%
% \subsection{Definition of a new theorem set}
%
% As already pointed out, a new theorem environment can be defined in
% three different ways:
% \begin{verbatim}
%    \newtheorem{Lem}{Lemma}
%    \newtheorem{Lem}{Lemma}[section]
%    \newtheorem{Lem}[Theorem]{Lemma}
%\end{verbatim}
% The function of the macro |\newtheorem| is to recognize these cases
% and then to branch into one of the three macros |\@ynthm|, |\@xnthm|
% or |\@othm|.  This mechanism is adopted unchanged from
% \cite{latex:source}; the essential point here is that, for example,
% in the second case, the arguments \texttt{Lem}, \texttt{Lemma} and
% \texttt{section} are passed over to the macro |\@xnthm|.
%
% We inspect this case first because the others present fewer problems,
% and thus are easily derived from this one.
%
% \begin{macro}{\@xnthm}
%    For our example arguments, the macro |\@xnthm| must fulfill
%    the following:
%    \begin{itemize}
%       \item
%          Define a new \LaTeX-counter `Lem'
%       \item
%          reset this counter within a |\section|
%       \item
%          define the macro |\theLem|
%       \item
%          define the environment macros |\Lem| and
%          |\endLem| using the current
%          |\theoremstyle| and |\theorem@bodyfont|.
%    \end{itemize}
%    Obviously, all this should happen only if the first argument
%    of |\@xnthm| (i.e.\ \texttt{Lem} in our example) is chosen so
%    as not to conflict with any previously defined commands or
%    environments.  This test is performed by the \LaTeX{} macro
%    |\@ifdefinable|.
%    \begin{macrocode}
\gdef\@xnthm#1#2[#3]{\expandafter\@ifdefinable\csname #1\endcsname
%    \end{macrocode}
%    Therefore, the first argument of |\@ifdefinable| is the
%    expansion (in the example, |\Lem|) of
%    |\csname||#1\endcsname|.  The second argument is
%    executed only if the test has been completed successfully.
%    \begin{macrocode}
   {%
%    \end{macrocode}
%    Now we define the new counter. The names of the \LaTeX{}
%    macros employed should speak for themselves:
% \changes{v2.2b}{1995/10/07}{Use \cs{@newctr}}
% \changes{v2.2c}{1995/11/23}{Use \cs{@newctr} properly.}
%    \begin{macrocode}
    \@definecounter{#1}\@newctr{#1}[#3]%
%    \end{macrocode}
%    Using |\@newctr| will give a proper error message if the counter
%    in |#3| is not defined.
%    In defining `|\theLem|' we must generate the desired macro
%    name by use of |\expandafter| and |\csname|.
%    \begin{macrocode}
    \expandafter\xdef\csname the#1\endcsname
%    \end{macrocode}
%    An |\xdef| is used in order to make the definition global,
%    and to ensure that it contains the replacement texts of
%    |\@thmcountersep| and |\@thmcounter|.\footnote{These
%    two macros can be defined by the document style.  Their default
%    values produce a `\texttt{.}' as separation and an arabic
%    representation of the number.} However, not everything should be
%    expanded. For example, it saves space to use |\thesection|
%    instead of its---at times---lengthy expansion.
%    \begin{macrocode}
      {\expandafter \noexpand \csname the#3\endcsname
       \@thmcountersep \@thmcounter{#1}}%
%    \end{macrocode}
%    Thus with the defaults of \LaTeX{}, |\theLem| would be
%    replaced by the command sequence |\thesection.\arabic{Lem}|.
%
%    We will now look at the definition of the macro which is executed
%    at the beginning of the actual environment (in our example this
%    macro is |\Lem|). It should be noted that we use an
%    ``|\expandafter| trick'' to expand only certain parts of the
%    replacement text at the time of the definition.
%    \begin{macrocode}
    \def\@tempa{\global\@namedef{#1}}%
    \expandafter \@tempa \expandafter{%
%    \end{macrocode}
%    First, the macro that contains the current definitions of
%    |\@begintheorem| and |\@opargtheorem| should be called
%    up.  The name of this macro---as is already known---has the form
%    |\th@|\meta{theorem\ style}; therefore, it must be called by
%    \begin{macrocode}
      \csname th@\the \theorem@style
%    \end{macrocode}
%    In addition the default theorem font should be changeable, i.e.\
%    we have to insert the contents of |\theorem@bodyfont|. For
%    that reason, we expand even further, beyond |\endcsname|,
%    and thus insert the contents of the \textsf{token} register
%    |\theorem@bodyfont| in the replacement text.
%    \begin{macrocode}
            \expandafter \endcsname \the \theorem@bodyfont
%    \end{macrocode}
%    Now it is time to call the macro |\@thm| which takes over
%    the further processing.  It has two arguments: the current
%    counter name (in our example, \texttt{Lem}), and the text of the
%    label (in our example, \texttt{Lemma}).
%    \begin{macrocode}
     \@thm{#1}{#2}}%
%    \end{macrocode}
%    With this, the `sub-definition' is complete.  The macro
%    |\@endtheorem| ends a theorem environment and is, so far, nothing
%    but an |\endtrivlist|.  (Hence it is defined globally, and not
%    within the theorem styles.\footnote{This has to be changed as
%    soon as theorem styles that change \texttt{\bslash @endtheorem}
%    exist.  In such a case, all existing styles must be changed as
%    well since they will have to reset the macro.}) Therefore, we can
%    set it equivalent to the macro that ends the theorem set (in our
%    example, |\endLem|).  However, if some day theorem styles exist
%    that do change |\@endtheorem|, we would have to use the
%    commented-out line instead.
%    \begin{macrocode}
    \global \expandafter \let \csname end#1\endcsname \@endtheorem
%   \global\@namedef{end#1}{\@endtheorem}%
%    \end{macrocode}
%    With these commands all the required definitions are employed,
%    unless the test |\@ifdefinable| has failed.  Therefore, we
%    end the second argument of this macro and with it the definition
%    of |\@xnthm|.
%    \begin{macrocode}
   }}
%    \end{macrocode}
% \end{macro}
%
% \begin{macro}{\@ynthm}
%    The definition of |\@ynthm| is completely analogous. In
%    this case the new counter that is defined is not reset within
%    another counter; thus the definition of |\the...|
%    is simplified:
%    \begin{macrocode}
\gdef\@ynthm#1#2{\expandafter\@ifdefinable\csname #1\endcsname
   {\@definecounter{#1}%
    \expandafter\xdef\csname the#1\endcsname{\@thmcounter{#1}}%
%    \end{macrocode}
%    The rest of the definition corresponds literally to that of
%    |\@xnthm|:
%    \begin{macrocode}
    \def\@tempa{\global\@namedef{#1}}\expandafter \@tempa
     \expandafter{\csname th@\the \theorem@style \expandafter
     \endcsname \the\theorem@bodyfont \@thm{#1}{#2}}%
    \global \expandafter \let \csname end#1\endcsname \@endtheorem}}
%    \end{macrocode}
% \end{macro}
%
% \begin{macro}{\@othm}
%    The definition of |\@othm| does not contain anything new.
%    \begin{macrocode}
\gdef\@othm#1[#2]#3{%
%    \end{macrocode}
% \changes{v2.2b}{95/10/07}{Build in check for counter defined}
%    We do not define a new counter but instead use one that has
%    already been defined.  Thus the only definition we need is that
%    of this pseudo-counter (i.e.\ |\the|\meta{env.\ name}).
%    First we check if |#2| corresponds to a known counter name.
%    \begin{macrocode}
  \expandafter\ifx\csname c@#2\endcsname\relax
   \@nocounterr{#2}%
  \else
   \expandafter\@ifdefinable\csname #1\endcsname
   {\expandafter \xdef \csname the#1\endcsname
     {\expandafter \noexpand \csname the#2\endcsname}%
%    \end{macrocode}
%    All other parts of the definition can be adopted from
%    |\@xnthm|. We have to remember, though, that in this case
%    the name of the current counter and the theorem label have moved
%    to the second and third arguments.
%    \begin{macrocode}
    \def\@tempa{\global\@namedef{#1}}\expandafter \@tempa
     \expandafter{\csname th@\the \theorem@style \expandafter
     \endcsname \the\theorem@bodyfont \@thm{#2}{#3}}%
    \global \expandafter \let \csname end#1\endcsname \@endtheorem}%
  \fi}
%    \end{macrocode}
% \end{macro}
%
% \subsection{Macros that are employed in a {\envfont theorem}
%    environment}
%
% \begin{macro}{\@thm}
%    The macro |\@thm| has to increase the current counter.  Then,
%    depending on whether the environment has (or does not have) an
%    optional argument, it has to branch into either |\@begintheorem|
%    or |\@opargtheorem|.
%    \begin{macrocode}
\gdef\@thm#1#2{\refstepcounter{#1}%
%    \end{macrocode}
%    Now we start a {\envfont trivlist} environment, and give
%    |\@topsep| and |\@topsepadd| the values of the \textsf{skip}
%    registers |\theorempreskipamount| and |\theorempostskipamount|.
%    The value in |\@topsep| is the vertical space that is inserted by
%    the first (and only) |\item| in our |\trivlist| whilst
%    |\@topsepadd| is inserted by |\@endparenv| at the end of that
%    {\envfont trivlist} environment.  By using these registers, we
%    obtain the desired space around a {\envfont theorem} environment.
%    \begin{macrocode}
   \trivlist
   \@topsep \theorempreskipamount               % used by first \item
   \@topsepadd \theorempostskipamount           % used by \@endparenv
%    \end{macrocode}
%    Now we have to test whether an optional argument has been given.
% \changes{v2.0d}{89/07/18}{Forgotten percent added.}
%    \begin{macrocode}
   \@ifnextchar [%
%    \end{macrocode}
%    If there is an optional argument, we will call |\@ythm|, and
%    move the arguments read back into the input stream.
%    \begin{macrocode}
   {\@ythm{#1}{#2}}%
%    \end{macrocode}
%    If not, we call |\@begintheorem|. Its first argument is the
%    name of the theorem set (hence the second argument of
%    |\@thm|). Its second argument is the macro that produces the
%    current number.
%    \begin{macrocode}
   {\@begintheorem{#2}{\csname the#1\endcsname}\ignorespaces}}
%    \end{macrocode}
% \end{macro}
%
% \begin{macro}{\@xthm}
% \begin{macro}{\@ythm}
%    Both these macros were originally called by |\@thm|.  We do
%    not need |\@xthm| anymore, hence we reset it to
%    |\relax|. The definition of |\@ythm| has not changed at
%    all from its definition in \LaTeX. In order to make the macros
%    easier to understand, we will
%    nevertheless present it (commented out).
%    \begin{macrocode}
\global\let\@xthm\relax
% \def\@ythm#1#2[#3]{\@opargbegintheorem{#2}{\csname
%           the#1\endcsname}{#3}\ignorespaces}
%</package>
%    \end{macrocode}
%    The primitive |\ignorespaces| in |\@ythm| and
%    |\@thm| is needed to remove the spaces between the
%    |\begin{...}| and the actual text.
% \end{macro}
% \end{macro}
%
%
% \subsection{Definition of the theorem substyles}
%
% As already pointed out, the theorem substyles, defined below, are
% only loaded when necessary.  Note that all these substyles, except
% \texttt{plain}, have |\slshape| as the default body font.
%
% \subsubsection{The \texttt{plain} style}
%
% \label{dok:th@plain} As the following macros use |@|, we have
% to locally set the |\catcode| of this symbol to ``letter''.
% This happens within a group, so that we do not have to worry about
% which |\catcode| that symbol had before.
%    \begin{macrocode}
%<*thp>
\begingroup \makeatletter
%    \end{macrocode}
% Since we are now within a group, we must make all definitions
% globally.  First we make sure that \texttt{theorem.sty} is loaded.
% This will allow us to use this file as a document style option
% without having to call \texttt{theorem} itself as an option.  At the
% same time, we assure that at least version 2 is loaded, since
% |\theorem@style| was not defined in earlier versions.
%    \begin{macrocode}
\@ifundefined{theorem@style}{\input{theorem.sty}}{}
\ProvidesFile{thp.sty}
             [\FMithmInfo]
%    \end{macrocode}
%
%
% \begin{macro}{\th@plain}
% \begin{macro}{\@begintheorem}
% \begin{macro}{\@opargbegintheorem}
%    |\theoremstyle{plain}| corresponds to the original
%    definition, except that the distances to the surrounding text are
%    determined by the parameters |\theorempreskipamount| and
%    |\theorempostskipamount|.
%    First we set the default body font.
%    \begin{macrocode}
\gdef\th@plain{\normalfont\itshape
%    \end{macrocode}
%    Then we define |\@begintheorem| and
%    |\@opargbegintheorem|. These two macros define how the
%    header of a theorem is typeset.  |\@opargbegintheorem| will
%    be called if a \textsf{theorem} environment with an optional
%    argument is encountered; otherwise, the header is constructed by
%    calling |\@begintheorem|.  If one of these macros is
%    executed, we are within a \textsf{trivlist} environment started by
%    |\@thm|.  So the theorem header is produced with an
%    |\item| command.
%
%    Instead of specifying the header font directly, all standard
%    theorem styles use the |\theorem@headerfont| macro to allow
%    customization.  The extra space (|\labelsep|) is necessary
%    because of problems in the \textsf{trivlist} environment.
%    \begin{macrocode}
  \def\@begintheorem##1##2{%
        \item[\hskip\labelsep \theorem@headerfont ##1\ ##2]}%
%    \end{macrocode}
%    The definition of |\@opargbegintheorem| is completely
%    analogous. The only difference is the fact that there exists a
%    third argument (which is the optional parameter of the
%    environment and contains additional information about the
%    theorem).  Customarily we enclose it in parentheses.
%    \begin{macrocode}
\def\@opargbegintheorem##1##2##3{%
   \item[\hskip\labelsep \theorem@headerfont ##1\ ##2\ (##3)]}}
%    \end{macrocode}
% \end{macro}
% \end{macro}
% \end{macro}
%
% \noindent
% We conclude with an |\endgroup| to restore the
% |\catcode| of |@|.
%    \begin{macrocode}
\endgroup
%</thp>
%    \end{macrocode}
%
%
% \subsubsection{The \texttt{break} style}
%
% This style option is stored in the file |thb.sty|.
% For the next two lines see the documentation for |\th@plain| on
% page~\pageref{dok:th@plain}.
%    \begin{macrocode}
%<*thb>
\begingroup \makeatletter
\@ifundefined{theorem@style}{\input{theorem.sty}}{}
\ProvidesFile{thb.sty}
             [\FMithmInfo]
%    \end{macrocode}
%
% \begin{macro}{\th@break}
%    |\theoremstyle{break}| produces a line break after the name
%    of the theorem.  The font is |\slshape|.  Hence, we define
%    |\th@break| as follows:
%    \begin{macrocode}
\gdef\th@break{\normalfont\slshape
  \def\@begintheorem##1##2{\item[%
%    \end{macrocode}
%    We run into the following problem: it is not possible to create
%    the header with |\item[|\meta{title}|]| and then start
%    a new line by, for example, |\mbox{}\\|. Such a definition
%    will fail whenever a list environment follows immediately.  With
%    the above construction, the |\mbox{}| causes the switch
%    |@inlabel| (cf.\ definition of |\list| and
%    |\trivlist| in \cite{latex:source}) to be set to \texttt{false}
%    and so the following list will insert additional vertical
%    space (|\topskip|). This is quite annoying. Therefore, we
%    create the line break within the |\item|. In order to ensure
%    that the text will begin at the proper position in the following
%    line, we simply pretend that the label does not take any
%    room.\footnote{This will lead to problems whenever very high
%    symbols occurring in the line tower into the
%    heading. So, something else has to be done here sometime.}
%    \begin{macrocode}
       \rlap{\vbox{\hbox{\hskip \labelsep\theorem@headerfont ##1\ ##2}%
                   \hbox{\strut}}}]}%
%    \end{macrocode}
%    Again, the definition of |\@opargbegintheorem| is
%    completely analogous.
%    \begin{macrocode}
\def\@opargbegintheorem##1##2##3{%
  \item[\rlap{\vbox{\hbox{\hskip \labelsep \theorem@headerfont
                     ##1\ ##2\ (##3)}%
                    \hbox{\strut}}}]}}
\endgroup
%</thb>
%    \end{macrocode}
% \end{macro}
%
%
% \subsubsection{The \texttt{changebreak} style}
%
%    \begin{macrocode}
% This style option is stored in the file |thcb.sty|.
%    \begin{macrocode}
%<*thcb>
\begingroup \makeatletter
\@ifundefined{theorem@style}{\input{theorem.sty}}{}
\ProvidesFile{thcb.sty}
             [\FMithmInfo]
%    \end{macrocode}
%
% \begin{macro}{\th@changebreak}
%    The \texttt{change-break} theorem style is like \texttt{break}
%    but with interchange of theorem name and theorem number.  Thus we
%    define |\th@changebreak| as follows:
%    \begin{macrocode}
\gdef\th@changebreak{\normalfont\slshape
  \def\@begintheorem##1##2{\item
       [\rlap{\vbox{\hbox{\hskip\labelsep \theorem@headerfont ##2\ ##1}%
                    \hbox{\strut}}}]}%
\def\@opargbegintheorem##1##2##3{%
  \item[\rlap{\vbox{\hbox{\hskip\labelsep \theorem@headerfont
                     ##2\ ##1\ (##3)}%
                    \hbox{\strut}}}]}}
\endgroup
%</thcb>
%    \end{macrocode}
% \end{macro}
%
%
% \subsubsection{The \texttt{change} style}
%
% This style option is stored in the file |thc.sty|.
%    \begin{macrocode}
%<*thc>
\begingroup \makeatletter
\@ifundefined{theorem@style}{\input{theorem.sty}}{}
\ProvidesFile{thc.sty}
             [\FMithmInfo]
%    \end{macrocode}
%
% \begin{macro}{\th@change}
%    The \texttt{change} theorem style corresponds to the
%    \texttt{change break} style without a linebreak after the header.
%    To say it in another way, it's the same as the \texttt{plain}
%    style but with number and name interchanged and |\slshape| as the
%    default font.
%    \begin{macrocode}
\gdef\th@change{\normalfont\slshape
  \def\@begintheorem##1##2{\item
       [\hskip\labelsep \theorem@headerfont ##2\ ##1]}%
\def\@opargbegintheorem##1##2##3{%
  \item[\hskip\labelsep \theorem@headerfont ##2\ ##1\ (##3)]}}
\endgroup
%</thc>
%    \end{macrocode}
% \end{macro}
%
%
% \subsubsection{The \texttt{marginbreak} style}
%
% This style option is the one used most often at Mainz.  It is saved
% in the file |thmb.sty|.
%    \begin{macrocode}
%<*thmb>
\begingroup \makeatletter
\@ifundefined{theorem@style}{\input{theorem.sty}}{}
\ProvidesFile{thmb.sty}
             [\FMithmInfo]
%    \end{macrocode}
%
% \begin{macro}{\th@marginbreak}
%    The \texttt{margin break} style is nearly the same as the
%    \texttt{change break} style. The only difference is the placement
%    of the theorem number.  We use |\llap| to place it in the left
%    margin.
%
%    In this style |\labelsep| denotes the separation between the
%    number and the text.
%    \begin{macrocode}
\gdef\th@marginbreak{\normalfont\slshape
  \def\@begintheorem##1##2{\item
       [\rlap{\vbox{\theorem@headerfont
                    \hbox{\llap{##2}\hskip\labelsep ##1}%
                    \hbox{\strut}}}]}%
\def\@opargbegintheorem##1##2##3{%
  \item[\rlap{\vbox{\theorem@headerfont
                    \hbox{\llap{##2}\hskip\labelsep ##1\ (##3)}%
                    \hbox{\strut}}}]}}
\endgroup
%</thmb>
%    \end{macrocode}
% \changes{v2.1c}{90/02/17}{`endgroup outside macrocode env.}
% \end{macro}
%
%
%
% \subsubsection{The \texttt{margin} style}
%
% This style option is stored in the file |thm.sty|.
%    \begin{macrocode}
%<*thm>
\begingroup \makeatletter
\@ifundefined{theorem@style}{\input{theorem.sty}}{}
\ProvidesFile{thm.sty}
             [\FMithmInfo]
%    \end{macrocode}
%
% \begin{macro}{\th@margin}
%    Again this is only a variant of the theorem styles described
%    above without any new ideas.
%    \begin{macrocode}
\gdef\th@margin{\normalfont\slshape
  \def\@begintheorem##1##2{\item
       [\theorem@headerfont \llap{##2}\hskip\labelsep ##1]}%
\def\@opargbegintheorem##1##2##3{%
  \item[\theorem@headerfont \llap{##2}\hskip\labelsep ##1\ (##3)]}}
\endgroup
%</thm>
%    \end{macrocode}
% \end{macro}
%
%
%
%
% \subsection{Final Definitions}
%
% \begin{macro}{\theorempreskipamount}
% \begin{macro}{\theorempostskipamount}
%    The \textsf{skip} parameters that regulate the vertical empty space
%    before and after the {\envfont theorem} environment have
%    to be allocated as well.
% \changes{v2.0f}{89/07/20}{Unnecessary `global' removed.}
%    \begin{macrocode}
%<*package>
\newskip\theorempreskipamount
\newskip\theorempostskipamount
%    \end{macrocode}
%    Since we have used the same values for all theorem sets, we now can
%    assign them.
%    \begin{macrocode}
\global\setlength\theorempreskipamount{12pt plus 5pt minus 3pt}
\global\setlength\theorempostskipamount{8pt plus 3pt minus 1.5pt}
%    \end{macrocode}
% \end{macro}
% \end{macro}
%
%
% \begin{macro}{\@endtheorem}
%    The same holds for the macro |\@endtheorem|, which ends a
%    {\envfont theorem} environment. Since it is the same for all
%    theorem sets, it is removed from the macros |\th@|\meta{style}.
%    It simply ends the {\envfont trivlist} environment, which was
%    begun in |\@thm|.
%    \begin{macrocode}
\global\let\@endtheorem=\endtrivlist
%    \end{macrocode}
% \end{macro}
%
%
%
% \begin{macro}{\@preamblecmds}
%    All macros defined above are to be used only in the preamble.
%    Therefore, we insert them in |\@preamblecmds| which will disable
%    them at begin document. This is done by the internal \LaTeXe{}
%    command |\@onlypreamble|.
% \changes{v2.0g}{89/09/19}{`theoremheaderfont and `theorembodyfont
%    added.}
% \changes{v2.2a}{94/02/03}{Use \LaTeXe{} interface}
%    \begin{macrocode}
\@onlypreamble\@xnthm
\@onlypreamble\@ynthm
\@onlypreamble\@othm
\@onlypreamble\newtheorem
\@onlypreamble\theoremstyle
\@onlypreamble\theorembodyfont
\@onlypreamble\theoremheaderfont
%    \end{macrocode}
% \end{macro}
%
% Finally we declare the \texttt{plain} theorem style to be the default.
% \changes{v2.1b}{90/06/04}{Fixed missing `item by declaring default
%    style.}
%    \begin{macrocode}
\theoremstyle{plain}
%</package>
%    \end{macrocode}
%
% \Finale

\endinput
}{}
\ProvidesFile{thcb.sty}
             [\FMithmInfo]
%    \end{macrocode}
%
% \begin{macro}{\th@changebreak}
%    The \texttt{change-break} theorem style is like \texttt{break}
%    but with interchange of theorem name and theorem number.  Thus we
%    define |\th@changebreak| as follows:
%    \begin{macrocode}
\gdef\th@changebreak{\normalfont\slshape
  \def\@begintheorem##1##2{\item
       [\rlap{\vbox{\hbox{\hskip\labelsep \theorem@headerfont ##2\ ##1}%
                    \hbox{\strut}}}]}%
\def\@opargbegintheorem##1##2##3{%
  \item[\rlap{\vbox{\hbox{\hskip\labelsep \theorem@headerfont
                     ##2\ ##1\ (##3)}%
                    \hbox{\strut}}}]}}
\endgroup
%</thcb>
%    \end{macrocode}
% \end{macro}
%
%
% \subsubsection{The \texttt{change} style}
%
% This style option is stored in the file |thc.sty|.
%    \begin{macrocode}
%<*thc>
\begingroup \makeatletter
\@ifundefined{theorem@style}{% \iffalse meta-comment
%
% Copyright (C) 1993-2024
%
% The LaTeX Project and any individual authors listed elsewhere
% in this file.
%
% This file is part of the Standard LaTeX `Tools Bundle'.
% -------------------------------------------------------
%
% It may be distributed and/or modified under the
% conditions of the LaTeX Project Public License, either version 1.3c
% of this license or (at your option) any later version.
% The latest version of this license is in
%    https://www.latex-project.org/lppl.txt
% and version 1.3c or later is part of all distributions of LaTeX
% version 2005/12/01 or later.
%
% The list of all files belonging to the LaTeX `Tools Bundle' is
% given in the file `manifest.txt'.
%
% \fi
%
%%
%
%
\def\FMithmInfo{2023/07/05 v2.2c Theorem extension package (FMi)}
%
% \ProvidesFile{theorem.dtx}[\FMithmInfo]
%\iffalse   % this is a METACOMMENT !
%%
%% Package `theorem' to use with LaTeX2e
%% Copyright (C) 1989-1995 Frank Mittelbach, all rights reserved.
%<+package>\@ifundefined{theorem@style}{}{\endinput}
%<+package>\NeedsTeXFormat{LaTeX2e}[1995/06/01]
%<+package>\ProvidesPackage{theorem}
%<+package>         [\FMithmInfo]
%
% \fi
%
% \changes{v1.0}{87/05/07}{First draft.}
% \changes{v1.0a}{87/05/07}{`thistheoremstyle' added.}
% \changes{v1.1a}{87/07/29}{New concept: redefine the internal
%                          \protect\LaTeX{} commands `@xnthm' etc.}
% \changes{v1.1b}{87/07/31}{`theoremfont' used to specify font.}
% \changes{v1.2a}{88/01/08}{`@endtheorem' now globally defined.}
% \changes{v1.2b}{88/01/11}{`newtheorem' only allowed in the preamble.}
% \changes{v1.3a}{88/02/14}{`theoremfont' now a toks register.}
%
% \changes{v2.0a}{88/11/16}{New implementation and new documentation.}
% \changes{v2.0a}{88/11/16}{No layout changes but internal modifications
%                           old substyles won't work.}
% \changes{v2.0b}{89/01/30}{`global' used for all definitions.}
% \changes{v2.0c}{89/06/15}{Documentation translated to English.
%                            Thanks to LLL and Nina}
% \changes{v2.1a}{90/02/15}{Support for new font selection.}
% \changes{v2.2a}{94/02/03}{Cleanup for LaTeX2e}
%
% \DoNotIndex{\@endparenv,\@ifdefinable,\@ifnextchar,\@ifundefined}
% \DoNotIndex{\@namedef,\@nameuse,\@spaces,\@tempa,\@warning}
% \DoNotIndex{\ }
% \DoNotIndex{\begingroup,\bfseries}
% \DoNotIndex{\csname}
% \DoNotIndex{\def,\do,\docdate}
% \DoNotIndex{\endcsname,\endgroup,\endinput,\endtrivlist,\expandafter}
% \DoNotIndex{\filedate,\fileversion}
% \DoNotIndex{\gdef,\global}
% \DoNotIndex{\hbox,\hskip}
% \DoNotIndex{\ignorespaces,\input,\itshape,\item}
% \DoNotIndex{\labelsep,\let,\llap}
% \DoNotIndex{\makeatletter}
% \DoNotIndex{\noexpand}
% \DoNotIndex{\refstepcounter,\relax,\rlap}
% \DoNotIndex{\setlength,\slshape,\space,\string,\strut}
% \DoNotIndex{\the,\trivlist,\typeout}
% \DoNotIndex{\vbox}
% \DoNotIndex{\xdef}
%
% \def\deflabel#1{\ttfamily #1\hfill}
% \def\deflist#1{\begin{list}{}{\settowidth\labelwidth{\ttfamily #1}%
%                               \setlength\leftmargin\labelwidth
%                               \addtolength\leftmargin\labelsep
%                               \let\makelabel\deflabel}}
% \def\enddeflist{\end{list}}
% \def\envfont{\normalfont\ttfamily}
%
% \MakeShortVerb{\|}
%
% \newcommand{\TUB}{{\sl TUGboat\/}}
%
% \GetFileInfo{theorem.dtx}
% \title{An Extension of the
%        \LaTeX{} theorem environment\thanks{This file
%        has version number \fileversion, last
%        revised \filedate.}}
% \author{Frank Mittelbach}
%
% \MaintainedByLaTeXTeam{tools}
% \maketitle
%
% \begin{abstract}
%    The macros described in this paper yield an extension of the
%    \LaTeX{} theorem mechanism. It is designed to satisfy the
%    different requirements of various journals. Thus, the
%    layout of the ``theorems'' can be manipulated by determining a
%    ``style''. This article describes not only the use, but
%    also the definition, of the necessary macros.
% \end{abstract}
%
% \section*{Preface to version 2.2}
%
% For \LaTeXe{} this package did not need any fundamental changes. I
% only modified the messages generated so that theorem layout styles
% will show up with the |\listfiles| command and cleaned the section
% on the New Font Selection Scheme since this is now included in
% \LaTeX.
%
% \section*{Preface to version 2.1}
%
% This version is identical to 2.0g described in \TUB\ 10\#3 except
% for some internal defaults which are now set depending on the used
% font selection scheme.
%
% This was done to avoid unpleasant surprises if the new font selection
% scheme is in force. For further details see
% section~\ref{sec:newfonts} and \cite{bk:GMS94}.
%
% \section{Introduction}
%
% For our purposes here, ``theorems'' are labelled enunciations,
% often set off from the main text by extra space and a font change.
% Theorems, corollaries, conjectures, definitions, and remarks
% are all instances of ``theorems''.  The ``header'' of these structures
% is composed of a label (such as \textsc{Theorem} or \textsc{Remark})
% and a number which serializes an item in the sequence of items
% with the same label.
%
% Shortly after the introduction of \LaTeX{} at the Fachbereich
% Mathematik in Mainz, the desire to manipulate the layout of
% ``theorems'' arose.  In Mainz, the following two conventions came
% into general use:
% \begin{enumerate}
%   \item The number of the theorem is shown in the margin.
%   \item There is a line break at
%         the end of the theorem header.
% \end{enumerate}
%
% Additionally, some journals require different formats which depend on
% the ``sort of theorem'': e.g.\ often remarks and definitions are set
% in |\upshape|, while |\itshape| is employed for main theorems.
%
% Confronted with these requirements, a theorem environment was
% developed in Mainz which allows separate determination of the
% layout of the ``theorems sets'', comparable to |\pagestyle|.
%
%
% \section{The user interface}
% \subsection{Defining new theorem sets}
%
% \DescribeMacro\newtheorem
% As in the original \LaTeX{} version, the command |\newtheorem|
% defines a new ``theorem set'' or ``theorem-like structure''.
% Two required arguments name the new environment and give the text
% to be typeset with each instance of the new ``set'', while
% an optional argument determines how
% the ``set'' is enumerated:
% \begin{description}
%    \item[\ttfamily \bslash newtheorem\{foo\}\{bar\}]
%       The theorem set {\envfont foo} (whose name is \texttt{bar})
%       uses its own counter.
%    \item[\ttfamily \bslash newtheorem\{foo2\}{[foo]}\{bar2\}]
%       The theorem set {\envfont foo2} (printed name \texttt{bar2})
%       uses the same counter as the theorem set \texttt{foo}.
%    \item[\ttfamily \bslash newtheorem\{foo3\}\{bar3\}{[section]}]
%       The theorem set {\envfont foo3} (printed name \texttt{bar3}) is
%       enumerated within the counter \texttt{section}, i.e.\ with every
%       new |\section| the enumeration begins again with 1, and
%       the enumeration is composed from the section-number and the
%       theorem counter itself.
% \end{description}
%
% \DescribeMacro\theoremstyle
% Additionally, the command |\theoremstyle| can define the layout
% of various, or all, theorem sets.  It should be noted that any
% theorem set defined by |\newtheorem| is typeset in the
% |\theoremstyle| that is current at the time of the definition.
% Thus, the following
% \begin{verbatim}
%    \theoremstyle{break}        \newtheorem{Cor}{Corollary}
%    \theoremstyle{plain}        \newtheorem{Exa}{Example}[section]
%\end{verbatim}                             ^^A No blank at line start
% leads to the result that the set {\envfont Cor} is formatted in the
% style \texttt{break}, while the set {\envfont Exa} and all the
% following ones are formatted in the style \texttt{plain}, unless
% another |\theoremstyle| follows.  Since the definitions installed by
% |\newtheorem| are global, one also can limit |\theoremstyle| locally
% by grouping braces.
%
% \DescribeMacro\theorembodyfont
% The choice of the font for the theorem body is completely
% independent of the chosen |\theoremstyle|; this has proven to
% be very advantageous.  For example,
% \begin{verbatim}
%   {\theorembodyfont{\upshape}          \newtheorem{Rem}{Remark}}
%\end{verbatim}
% defines a theorem set {\envfont Rem}, which will be set in
% |\upshape| in the current layout (which in our example is
% \texttt{plain}).  As with |\theoremstyle|, the |\theorembodyfont|
% chosen is that current at the time of |\newtheorem|.  If
% |\theorembodyfont| is not specified or one defines
% |\theorembodyfont{}|, then the font used will be that defined by the
% |\theoremstyle|.
%
% \DescribeMacro\theoremheaderfont It is also possible to customize
% the font used for the theorem headers.  This is, however, a global
% declaration, and therefore there should be at most one
% |\theoremheaderfont| declaration in the preamble.\footnote{If it is
% actually necessary to have different header fonts, one has to define
% new theorem styles (substituting the desired font) or specify the
% information directly in the \texttt{\bslash newtheorem} declaration
% (the unclean variant).}
%
% \DescribeMacro\theorempreskipamount
% \DescribeMacro\theorempostskipamount Two additional parameters
% affect the vertical space around the theorem environments:
% |\theorempreskipamount| and |\theorempostskipamount| define,
% respectively, the spacing before and after such an environment.
% These parameters apply for all theorem sets and can be manipulated
% with the ordinary length macros.  They are rubber lengths
% (`\textsf{skips}'), and therefore can contain \texttt{plus} and
% \texttt{minus} parts.
%
% Since the definition of theorem sets should---most sensibly---be
% placed in the preamble, we only allow installation there.
% It is therefore possible to release the memory used here after
% |\begin{document}|, in order to make room for other
% applications.
%
%
% \subsection{Existing theorem styles}
%
% The following theorem styles exist to date:
% \begin{deflist}{marginbreak}
%    \item[plain]
%       This theorem style emulates the original \LaTeX{} definition,
%       except that additionally the parameters
%       |\theorem...skipamount| are used.
%    \item[break]
%       In this style, the theorem header is followed by a line
%       break.
%    \item[marginbreak]
%       The theorem number is set in the margin, and there is a line
%       break as in \texttt{break}.
%    \item[changebreak]
%       Like \texttt{break}, but with header number and text
%       interchanged.
%    \item[change]
%       Header number and text are interchanged, without a line break.
%    \item[margin]
%       The number is set in the left margin, without a line break.
% \end{deflist}
% All styles (except \texttt{plain}) select |\slshape| as the default
% |\theorembodyfont|.
%
%
% \subsection{Examples}
%
% Given the above theorem sets {\envfont Cor}, {\envfont Exa}
% and {\envfont Rem},
% suppose that the preamble also contains the declarations:
% \begin{verbatim}
%    \theoremstyle{marginbreak}   \newtheorem{Lem}[Cor]{Lemma}
%    \theoremstyle{change}
%    \theorembodyfont{\itshape}        \newtheorem{Def}[Cor]{Definition}
%
%    \theoremheaderfont{\scshape}
%\end{verbatim}
% Then the following are some typical examples of the typeset output
% resulting from their use.
%
% \begin{Cor}
%    This is a sentence typeset in the theorem environment {\envfont
%    Cor}.
% \end{Cor}
%
% \begin{Exa}
%    This is a sentence typeset in the theorem environment {\envfont
%    Exa}.
% \end{Exa}
%
% \begin{Rem}
%    This is a sentence typeset in the theorem environment {\envfont
%    Rem}.
% \end{Rem}
%
% \begin{Lem}[Ben User]
%    This is a sentence typeset in the theorem environment {\envfont
%    Lem}.
% \end{Lem}
%
% \begin{Def}[Very impressive Definition]
%    This is a sentence typeset in the theorem environment {\envfont
%    Def}.
% \end{Def}
%
% The last two examples show the effect of the optional argument to
% a theorem environment (it is the text typeset in parentheses).
%
%
% \section{Special Considerations}
% \label{sec:newfonts}
%
% Theoremheader and body are implemented as a unit. This means that
% the |\theoremheaderfont|  will inherit characteristics of the
% |\theorembodyfont| in \LaTeXe. Thus, if for example
% |\theorembodyfont| is |\itshape| and |\theoremheaderfont|
% is |\bfseries| the font selected for the header will have the
% characteristics `bold extended italic'. If this is not desired one
% should set the |\theoremheaderfont| to something like
% \begin{verbatim}
%      \theoremheaderfont{\normalfont\bfseries}
%\end{verbatim}
% i.e.\ supplying all necessary font information explicitly.
%
%
%
% \section{Acknowledgements}
%
% The publication of this set of macros was only possible with the
% help of Christina Busse (translating the manuscript into English),
% Joachim Pense (playing the r\^ole of typist), Chris Rowley
% (looking everything over) and many others providing useful
% suggestions.
%
% \MaybeStop{%
%    \begin{thebibliography}{9}
%       \bibitem{bk:GMS94} \textsc{M.~Goossens}, \textsc{F.~Mittelbach}
%         and \textsc{A.~Samarin}.
%         \newblock The \LaTeX{} Companion.
%         \newblock
%         Addison-Wesley, Reading, Massachusetts, 1994.
%       \bibitem{latex:source} \textsc{Lamport, Leslie.} \newblock
%                               \texttt{latex.tex}, version 2.09, date
%                               Feb. 1990.
%    \end{thebibliography}
%    \PrintIndex
%    \PrintChanges}
%
%
% \section{The documentation driver file}
%
% The next bit of code contains the documentation driver file for
% \TeX{}, i.e., the file that will produce the documentation you are
% currently reading. It will be extracted from this file by the
% \texttt{docstrip} program. Since it is the first code in the file
% one can alternatively process this file directly with \LaTeXe{} to
% obtain the documentation.
%
%    \begin{macrocode}
%<*driver>
\documentclass{ltxdoc}

\usepackage{theorem}

%<+driver>% The next few lines define theorem sets which are used
%<+driver>% in the example section of the documentation.

\theoremstyle{break}       \newtheorem{Cor}{Corollary}
\theoremstyle{plain}       \newtheorem{Exa}{Example}[section]
{\theorembodyfont{\upshape}\newtheorem{Rem}{Remark}}
\theoremstyle{marginbreak} \newtheorem{Lem}[Cor]{Lemma}
\theoremstyle{change}
\theorembodyfont{\itshape} \newtheorem{Def}[Cor]{Definition}

\theoremheaderfont{\scshape}

\RecordChanges

\begin{document}
   \DocInput{theorem.dtx}
\end{document}
%</driver>
%    \end{macrocode}
%
% \section{Definition of the Macros}
%
% If the file has been loaded before, we abort immediately. If not the
% package announces itself (this is actually done at the very top of
% the file---the way it is done isn't good style so don't copy it).
% \changes{v2.0e}{89/07/19}{Spaces removed from `typeout'.}
% \changes{v2.2b}{95/11/19}{Announce moved to top of file}
%    \begin{macrocode}
%<*package>
%\@ifundefined{theorem@style}{}{\endinput}
%\def\FMithmInfo{1995/11/19 v2.2b Theorem extension package (FMi)}
%\ProvidesPackage{theorem}[\FMithmInfo]
%    \end{macrocode}
%
%
% \subsection{Definition of theorem styles and fonts}
%
% All the definitions in this file are done globally to allow
% inputting this file inside a group.
%
% \begin{macro}{\theoremstyle}
%    Before a theorem style can be installed, the chosen style must be
%    known. For that reason, we must test to see that
%    |\th@|\meta{style} is known or, more precisely, that it is
%    different from |\relax|. If the style is not known then
%    |\th@plain| is used.
%    \begin{macrocode}
\gdef\theoremstyle#1{%
   \@ifundefined{th@#1}{\@warning
          {Unknown theoremstyle `#1'. Using `plain'}%
          \theorem@style{plain}}%
%    \end{macrocode}
%    We save the theorem style to be used in the token register
%    |\theorem@style|.
%    \begin{macrocode}
      {\theorem@style{#1}}%
%    \end{macrocode}
%    Now we ``evaluate'' the theorem style: this means, we call the
%    macro |\th@|\meta{style} which will activate the relevant
%    definitions which are contained in a separate file.  This is done
%    in a group to suppress changes to the current font.  This could
%    otherwise pose problems together with the new font selection
%    scheme\footnote{When I printed the original article using the new
%    font selection scheme I ended with a document with slanted
%    typefaces (text headings and all) simply because one of the
%    theorem styles used \texttt{\bslash sl} at toplevel.} if the
%    |\th@|\meta{style} is evaluated a second time.
% \changes{v2.1a}{90/02/15}{Group added around `th@..}
% \changes{v2.1a}{90/02/15}{Reseting of `..@begintheorem removed.}
%    \begin{macrocode}
      \begingroup
        \csname th@\the\theorem@style \endcsname
      \endgroup}
%    \end{macrocode}
% \end{macro}
%
%
% \begin{macro}{\@begintheorem}
% \changes{v2.1a}{90/02/15}{Relaxed at toplevel.}
% \begin{macro}{\@opargbegintheorem}
% \changes{v2.1a}{90/02/15}{Relaxed at toplevel.}
%    We reset |\@begintheorem| and |\@opargbegintheorem| to
%    |\relax| since these commands are no longer necessary at
%    toplevel. This will save a few tokens.
%    \begin{macrocode}
\global\let\@begintheorem\relax
\global\let\@opargbegintheorem\relax
%    \end{macrocode}
% \end{macro}
% \end{macro}
%
% \begin{macro}{\theorem@style}
%    Obviously the \textsf{token} register used above has to be
%    allocated. To assure the utmost compatibility with the
%    original \LaTeX{} definition, we set the default theorem style to
%    \texttt{plain}, which implements the usual \LaTeX{} convention.
% \changes{v2.0f}{89/07/20}{Unnecessary `global' removed.}
%    \begin{macrocode}
\newtoks\theorem@style
\global\theorem@style{plain}
%    \end{macrocode}
% \end{macro}
%
% \begin{macro}{\theorembodyfont}
% \begin{macro}{\theorem@bodyfont}
%    If the |\theorembodyfont| is set by the user then it should not
%    interact with the default font set in the theorem style. When the
%    new font selection is in force this may happen if, for example,
%    the default is |\itshape| and the new |\theorembodyfont| is
%    |\sffamily|. So we add a |\reset@font| command in front of the
%    user definition.
% \changes{v2.0f}{89/07/20}{Unnecessary `global' removed.}
% \changes{v2.0g}{89/09/19}{`theoremfont to `theorembodyfont.}
% \changes{v2.1c}{92/06/22}{`reset@font added and logic changed.}
% \changes{v2.1d}{92/10/28}{Reset to empty if arg is empty}
%    \begin{macrocode}
\gdef\theorembodyfont#1{%
%    \end{macrocode}
%    We check if the argument supplied is empty and if so put nothing
%    into the |\theorem@bodyfont| token register to allow for
%    |\theorembodyfont{}| as a mean of using the default of the
%    current |\theoremstyle|.
%    \begin{macrocode}
   \def\@tempa{#1}%
   \ifx\@tempa\@empty
    \theorem@bodyfont{}%
   \else
    \theorem@bodyfont{\reset@font#1}%
   \fi
   }
\newtoks\theorem@bodyfont
\global\theorem@bodyfont{}
%    \end{macrocode}
% \end{macro}
% \end{macro}
%
%
% \begin{macro}{\theoremheaderfont}
% \changes{v2.0g}{89/09/19}{Macro added.}
%    The font for the theorem headers is handled differently because
%    this definition applies to all theorem styles.
%    \begin{macrocode}
\gdef\theoremheaderfont#1{\gdef\theorem@headerfont{#1}%
%    \end{macrocode}
%    After using the macro once it is redefined to produce an error
%    message.
%    \begin{macrocode}
       \gdef\theoremheaderfont##1{%
        \typeout{\string\theoremheaderfont\space should be used
                 only once.}}}
%    \end{macrocode}
% \end{macro}
%
% \begin{macro}{\theorem@headerfont}
% \changes{v2.0g}{89/09/19}{Macro added.}
%    To set the |\theorem@headerfont| default we first test if
%    the new fontselection scheme is in force.
% \changes{v2.1a}{90/02/15}{Support for new font selection.}
%    \begin{macrocode}
\ifx\upshape\undefined
%    \end{macrocode}
%    If not we define it to expand into |\bfseries|. We don't use
%    |\let| just in case a following style option redefines this
%    macro.
%    \begin{macrocode}
\gdef\theorem@headerfont{\bfseries}
%    \end{macrocode}
%    Otherwise we reset the current shape before calling |\bfseries|.
%    \begin{macrocode}
\else \gdef\theorem@headerfont{\normalfont\bfseries}\fi
%    \end{macrocode}
% \end{macro}
%
% \begin{macro}{\th@plain}
% \begin{macro}{\th@break}
% \begin{macro}{\th@marginbreak}
% \begin{macro}{\th@changebreak}
% \begin{macro}{\th@change}
% \begin{macro}{\th@margin}
%    The different styles are defined in macros such as
%    |\th@plain|.  Since memory space is precious in
%    ``non-Big-versions'', we have to avoid offering too many unused
%    definitions.  Therefore we define these styles in separate files
%    that can be loaded on demand.  Thus the commands themselves only
%    load these files. We use |\@input@|, a \LaTeXe{} internal command
%    that ensures that the file will be listed with |\listfiles|.
% \changes{v2.2a}{94/02/02}{Use `@input@ to load theorem layout files}
%    \begin{macrocode}
\gdef\th@plain{\@input@{thp.sty}}
\gdef\th@break{\@input@{thb.sty}}
\gdef\th@marginbreak{\@input@{thmb.sty}}
\gdef\th@changebreak{\@input@{thcb.sty}}
\gdef\th@change{\@input@{thc.sty}}
\gdef\th@margin{\@input@{thm.sty}}
%    \end{macrocode}
%    This list will be expanded when new styles become
%    available.  For testing,  just append new theorem substyles
%    as document options.
% \end{macro}
% \end{macro}
% \end{macro}
% \end{macro}
% \end{macro}
% \end{macro}
%
%
%
% \subsection{Definition of a new theorem set}
%
% As already pointed out, a new theorem environment can be defined in
% three different ways:
% \begin{verbatim}
%    \newtheorem{Lem}{Lemma}
%    \newtheorem{Lem}{Lemma}[section]
%    \newtheorem{Lem}[Theorem]{Lemma}
%\end{verbatim}
% The function of the macro |\newtheorem| is to recognize these cases
% and then to branch into one of the three macros |\@ynthm|, |\@xnthm|
% or |\@othm|.  This mechanism is adopted unchanged from
% \cite{latex:source}; the essential point here is that, for example,
% in the second case, the arguments \texttt{Lem}, \texttt{Lemma} and
% \texttt{section} are passed over to the macro |\@xnthm|.
%
% We inspect this case first because the others present fewer problems,
% and thus are easily derived from this one.
%
% \begin{macro}{\@xnthm}
%    For our example arguments, the macro |\@xnthm| must fulfill
%    the following:
%    \begin{itemize}
%       \item
%          Define a new \LaTeX-counter `Lem'
%       \item
%          reset this counter within a |\section|
%       \item
%          define the macro |\theLem|
%       \item
%          define the environment macros |\Lem| and
%          |\endLem| using the current
%          |\theoremstyle| and |\theorem@bodyfont|.
%    \end{itemize}
%    Obviously, all this should happen only if the first argument
%    of |\@xnthm| (i.e.\ \texttt{Lem} in our example) is chosen so
%    as not to conflict with any previously defined commands or
%    environments.  This test is performed by the \LaTeX{} macro
%    |\@ifdefinable|.
%    \begin{macrocode}
\gdef\@xnthm#1#2[#3]{\expandafter\@ifdefinable\csname #1\endcsname
%    \end{macrocode}
%    Therefore, the first argument of |\@ifdefinable| is the
%    expansion (in the example, |\Lem|) of
%    |\csname||#1\endcsname|.  The second argument is
%    executed only if the test has been completed successfully.
%    \begin{macrocode}
   {%
%    \end{macrocode}
%    Now we define the new counter. The names of the \LaTeX{}
%    macros employed should speak for themselves:
% \changes{v2.2b}{1995/10/07}{Use \cs{@newctr}}
% \changes{v2.2c}{1995/11/23}{Use \cs{@newctr} properly.}
%    \begin{macrocode}
    \@definecounter{#1}\@newctr{#1}[#3]%
%    \end{macrocode}
%    Using |\@newctr| will give a proper error message if the counter
%    in |#3| is not defined.
%    In defining `|\theLem|' we must generate the desired macro
%    name by use of |\expandafter| and |\csname|.
%    \begin{macrocode}
    \expandafter\xdef\csname the#1\endcsname
%    \end{macrocode}
%    An |\xdef| is used in order to make the definition global,
%    and to ensure that it contains the replacement texts of
%    |\@thmcountersep| and |\@thmcounter|.\footnote{These
%    two macros can be defined by the document style.  Their default
%    values produce a `\texttt{.}' as separation and an arabic
%    representation of the number.} However, not everything should be
%    expanded. For example, it saves space to use |\thesection|
%    instead of its---at times---lengthy expansion.
%    \begin{macrocode}
      {\expandafter \noexpand \csname the#3\endcsname
       \@thmcountersep \@thmcounter{#1}}%
%    \end{macrocode}
%    Thus with the defaults of \LaTeX{}, |\theLem| would be
%    replaced by the command sequence |\thesection.\arabic{Lem}|.
%
%    We will now look at the definition of the macro which is executed
%    at the beginning of the actual environment (in our example this
%    macro is |\Lem|). It should be noted that we use an
%    ``|\expandafter| trick'' to expand only certain parts of the
%    replacement text at the time of the definition.
%    \begin{macrocode}
    \def\@tempa{\global\@namedef{#1}}%
    \expandafter \@tempa \expandafter{%
%    \end{macrocode}
%    First, the macro that contains the current definitions of
%    |\@begintheorem| and |\@opargtheorem| should be called
%    up.  The name of this macro---as is already known---has the form
%    |\th@|\meta{theorem\ style}; therefore, it must be called by
%    \begin{macrocode}
      \csname th@\the \theorem@style
%    \end{macrocode}
%    In addition the default theorem font should be changeable, i.e.\
%    we have to insert the contents of |\theorem@bodyfont|. For
%    that reason, we expand even further, beyond |\endcsname|,
%    and thus insert the contents of the \textsf{token} register
%    |\theorem@bodyfont| in the replacement text.
%    \begin{macrocode}
            \expandafter \endcsname \the \theorem@bodyfont
%    \end{macrocode}
%    Now it is time to call the macro |\@thm| which takes over
%    the further processing.  It has two arguments: the current
%    counter name (in our example, \texttt{Lem}), and the text of the
%    label (in our example, \texttt{Lemma}).
%    \begin{macrocode}
     \@thm{#1}{#2}}%
%    \end{macrocode}
%    With this, the `sub-definition' is complete.  The macro
%    |\@endtheorem| ends a theorem environment and is, so far, nothing
%    but an |\endtrivlist|.  (Hence it is defined globally, and not
%    within the theorem styles.\footnote{This has to be changed as
%    soon as theorem styles that change \texttt{\bslash @endtheorem}
%    exist.  In such a case, all existing styles must be changed as
%    well since they will have to reset the macro.}) Therefore, we can
%    set it equivalent to the macro that ends the theorem set (in our
%    example, |\endLem|).  However, if some day theorem styles exist
%    that do change |\@endtheorem|, we would have to use the
%    commented-out line instead.
%    \begin{macrocode}
    \global \expandafter \let \csname end#1\endcsname \@endtheorem
%   \global\@namedef{end#1}{\@endtheorem}%
%    \end{macrocode}
%    With these commands all the required definitions are employed,
%    unless the test |\@ifdefinable| has failed.  Therefore, we
%    end the second argument of this macro and with it the definition
%    of |\@xnthm|.
%    \begin{macrocode}
   }}
%    \end{macrocode}
% \end{macro}
%
% \begin{macro}{\@ynthm}
%    The definition of |\@ynthm| is completely analogous. In
%    this case the new counter that is defined is not reset within
%    another counter; thus the definition of |\the...|
%    is simplified:
%    \begin{macrocode}
\gdef\@ynthm#1#2{\expandafter\@ifdefinable\csname #1\endcsname
   {\@definecounter{#1}%
    \expandafter\xdef\csname the#1\endcsname{\@thmcounter{#1}}%
%    \end{macrocode}
%    The rest of the definition corresponds literally to that of
%    |\@xnthm|:
%    \begin{macrocode}
    \def\@tempa{\global\@namedef{#1}}\expandafter \@tempa
     \expandafter{\csname th@\the \theorem@style \expandafter
     \endcsname \the\theorem@bodyfont \@thm{#1}{#2}}%
    \global \expandafter \let \csname end#1\endcsname \@endtheorem}}
%    \end{macrocode}
% \end{macro}
%
% \begin{macro}{\@othm}
%    The definition of |\@othm| does not contain anything new.
%    \begin{macrocode}
\gdef\@othm#1[#2]#3{%
%    \end{macrocode}
% \changes{v2.2b}{95/10/07}{Build in check for counter defined}
%    We do not define a new counter but instead use one that has
%    already been defined.  Thus the only definition we need is that
%    of this pseudo-counter (i.e.\ |\the|\meta{env.\ name}).
%    First we check if |#2| corresponds to a known counter name.
%    \begin{macrocode}
  \expandafter\ifx\csname c@#2\endcsname\relax
   \@nocounterr{#2}%
  \else
   \expandafter\@ifdefinable\csname #1\endcsname
   {\expandafter \xdef \csname the#1\endcsname
     {\expandafter \noexpand \csname the#2\endcsname}%
%    \end{macrocode}
%    All other parts of the definition can be adopted from
%    |\@xnthm|. We have to remember, though, that in this case
%    the name of the current counter and the theorem label have moved
%    to the second and third arguments.
%    \begin{macrocode}
    \def\@tempa{\global\@namedef{#1}}\expandafter \@tempa
     \expandafter{\csname th@\the \theorem@style \expandafter
     \endcsname \the\theorem@bodyfont \@thm{#2}{#3}}%
    \global \expandafter \let \csname end#1\endcsname \@endtheorem}%
  \fi}
%    \end{macrocode}
% \end{macro}
%
% \subsection{Macros that are employed in a {\envfont theorem}
%    environment}
%
% \begin{macro}{\@thm}
%    The macro |\@thm| has to increase the current counter.  Then,
%    depending on whether the environment has (or does not have) an
%    optional argument, it has to branch into either |\@begintheorem|
%    or |\@opargtheorem|.
%    \begin{macrocode}
\gdef\@thm#1#2{\refstepcounter{#1}%
%    \end{macrocode}
%    Now we start a {\envfont trivlist} environment, and give
%    |\@topsep| and |\@topsepadd| the values of the \textsf{skip}
%    registers |\theorempreskipamount| and |\theorempostskipamount|.
%    The value in |\@topsep| is the vertical space that is inserted by
%    the first (and only) |\item| in our |\trivlist| whilst
%    |\@topsepadd| is inserted by |\@endparenv| at the end of that
%    {\envfont trivlist} environment.  By using these registers, we
%    obtain the desired space around a {\envfont theorem} environment.
%    \begin{macrocode}
   \trivlist
   \@topsep \theorempreskipamount               % used by first \item
   \@topsepadd \theorempostskipamount           % used by \@endparenv
%    \end{macrocode}
%    Now we have to test whether an optional argument has been given.
% \changes{v2.0d}{89/07/18}{Forgotten percent added.}
%    \begin{macrocode}
   \@ifnextchar [%
%    \end{macrocode}
%    If there is an optional argument, we will call |\@ythm|, and
%    move the arguments read back into the input stream.
%    \begin{macrocode}
   {\@ythm{#1}{#2}}%
%    \end{macrocode}
%    If not, we call |\@begintheorem|. Its first argument is the
%    name of the theorem set (hence the second argument of
%    |\@thm|). Its second argument is the macro that produces the
%    current number.
%    \begin{macrocode}
   {\@begintheorem{#2}{\csname the#1\endcsname}\ignorespaces}}
%    \end{macrocode}
% \end{macro}
%
% \begin{macro}{\@xthm}
% \begin{macro}{\@ythm}
%    Both these macros were originally called by |\@thm|.  We do
%    not need |\@xthm| anymore, hence we reset it to
%    |\relax|. The definition of |\@ythm| has not changed at
%    all from its definition in \LaTeX. In order to make the macros
%    easier to understand, we will
%    nevertheless present it (commented out).
%    \begin{macrocode}
\global\let\@xthm\relax
% \def\@ythm#1#2[#3]{\@opargbegintheorem{#2}{\csname
%           the#1\endcsname}{#3}\ignorespaces}
%</package>
%    \end{macrocode}
%    The primitive |\ignorespaces| in |\@ythm| and
%    |\@thm| is needed to remove the spaces between the
%    |\begin{...}| and the actual text.
% \end{macro}
% \end{macro}
%
%
% \subsection{Definition of the theorem substyles}
%
% As already pointed out, the theorem substyles, defined below, are
% only loaded when necessary.  Note that all these substyles, except
% \texttt{plain}, have |\slshape| as the default body font.
%
% \subsubsection{The \texttt{plain} style}
%
% \label{dok:th@plain} As the following macros use |@|, we have
% to locally set the |\catcode| of this symbol to ``letter''.
% This happens within a group, so that we do not have to worry about
% which |\catcode| that symbol had before.
%    \begin{macrocode}
%<*thp>
\begingroup \makeatletter
%    \end{macrocode}
% Since we are now within a group, we must make all definitions
% globally.  First we make sure that \texttt{theorem.sty} is loaded.
% This will allow us to use this file as a document style option
% without having to call \texttt{theorem} itself as an option.  At the
% same time, we assure that at least version 2 is loaded, since
% |\theorem@style| was not defined in earlier versions.
%    \begin{macrocode}
\@ifundefined{theorem@style}{\input{theorem.sty}}{}
\ProvidesFile{thp.sty}
             [\FMithmInfo]
%    \end{macrocode}
%
%
% \begin{macro}{\th@plain}
% \begin{macro}{\@begintheorem}
% \begin{macro}{\@opargbegintheorem}
%    |\theoremstyle{plain}| corresponds to the original
%    definition, except that the distances to the surrounding text are
%    determined by the parameters |\theorempreskipamount| and
%    |\theorempostskipamount|.
%    First we set the default body font.
%    \begin{macrocode}
\gdef\th@plain{\normalfont\itshape
%    \end{macrocode}
%    Then we define |\@begintheorem| and
%    |\@opargbegintheorem|. These two macros define how the
%    header of a theorem is typeset.  |\@opargbegintheorem| will
%    be called if a \textsf{theorem} environment with an optional
%    argument is encountered; otherwise, the header is constructed by
%    calling |\@begintheorem|.  If one of these macros is
%    executed, we are within a \textsf{trivlist} environment started by
%    |\@thm|.  So the theorem header is produced with an
%    |\item| command.
%
%    Instead of specifying the header font directly, all standard
%    theorem styles use the |\theorem@headerfont| macro to allow
%    customization.  The extra space (|\labelsep|) is necessary
%    because of problems in the \textsf{trivlist} environment.
%    \begin{macrocode}
  \def\@begintheorem##1##2{%
        \item[\hskip\labelsep \theorem@headerfont ##1\ ##2]}%
%    \end{macrocode}
%    The definition of |\@opargbegintheorem| is completely
%    analogous. The only difference is the fact that there exists a
%    third argument (which is the optional parameter of the
%    environment and contains additional information about the
%    theorem).  Customarily we enclose it in parentheses.
%    \begin{macrocode}
\def\@opargbegintheorem##1##2##3{%
   \item[\hskip\labelsep \theorem@headerfont ##1\ ##2\ (##3)]}}
%    \end{macrocode}
% \end{macro}
% \end{macro}
% \end{macro}
%
% \noindent
% We conclude with an |\endgroup| to restore the
% |\catcode| of |@|.
%    \begin{macrocode}
\endgroup
%</thp>
%    \end{macrocode}
%
%
% \subsubsection{The \texttt{break} style}
%
% This style option is stored in the file |thb.sty|.
% For the next two lines see the documentation for |\th@plain| on
% page~\pageref{dok:th@plain}.
%    \begin{macrocode}
%<*thb>
\begingroup \makeatletter
\@ifundefined{theorem@style}{\input{theorem.sty}}{}
\ProvidesFile{thb.sty}
             [\FMithmInfo]
%    \end{macrocode}
%
% \begin{macro}{\th@break}
%    |\theoremstyle{break}| produces a line break after the name
%    of the theorem.  The font is |\slshape|.  Hence, we define
%    |\th@break| as follows:
%    \begin{macrocode}
\gdef\th@break{\normalfont\slshape
  \def\@begintheorem##1##2{\item[%
%    \end{macrocode}
%    We run into the following problem: it is not possible to create
%    the header with |\item[|\meta{title}|]| and then start
%    a new line by, for example, |\mbox{}\\|. Such a definition
%    will fail whenever a list environment follows immediately.  With
%    the above construction, the |\mbox{}| causes the switch
%    |@inlabel| (cf.\ definition of |\list| and
%    |\trivlist| in \cite{latex:source}) to be set to \texttt{false}
%    and so the following list will insert additional vertical
%    space (|\topskip|). This is quite annoying. Therefore, we
%    create the line break within the |\item|. In order to ensure
%    that the text will begin at the proper position in the following
%    line, we simply pretend that the label does not take any
%    room.\footnote{This will lead to problems whenever very high
%    symbols occurring in the line tower into the
%    heading. So, something else has to be done here sometime.}
%    \begin{macrocode}
       \rlap{\vbox{\hbox{\hskip \labelsep\theorem@headerfont ##1\ ##2}%
                   \hbox{\strut}}}]}%
%    \end{macrocode}
%    Again, the definition of |\@opargbegintheorem| is
%    completely analogous.
%    \begin{macrocode}
\def\@opargbegintheorem##1##2##3{%
  \item[\rlap{\vbox{\hbox{\hskip \labelsep \theorem@headerfont
                     ##1\ ##2\ (##3)}%
                    \hbox{\strut}}}]}}
\endgroup
%</thb>
%    \end{macrocode}
% \end{macro}
%
%
% \subsubsection{The \texttt{changebreak} style}
%
%    \begin{macrocode}
% This style option is stored in the file |thcb.sty|.
%    \begin{macrocode}
%<*thcb>
\begingroup \makeatletter
\@ifundefined{theorem@style}{\input{theorem.sty}}{}
\ProvidesFile{thcb.sty}
             [\FMithmInfo]
%    \end{macrocode}
%
% \begin{macro}{\th@changebreak}
%    The \texttt{change-break} theorem style is like \texttt{break}
%    but with interchange of theorem name and theorem number.  Thus we
%    define |\th@changebreak| as follows:
%    \begin{macrocode}
\gdef\th@changebreak{\normalfont\slshape
  \def\@begintheorem##1##2{\item
       [\rlap{\vbox{\hbox{\hskip\labelsep \theorem@headerfont ##2\ ##1}%
                    \hbox{\strut}}}]}%
\def\@opargbegintheorem##1##2##3{%
  \item[\rlap{\vbox{\hbox{\hskip\labelsep \theorem@headerfont
                     ##2\ ##1\ (##3)}%
                    \hbox{\strut}}}]}}
\endgroup
%</thcb>
%    \end{macrocode}
% \end{macro}
%
%
% \subsubsection{The \texttt{change} style}
%
% This style option is stored in the file |thc.sty|.
%    \begin{macrocode}
%<*thc>
\begingroup \makeatletter
\@ifundefined{theorem@style}{\input{theorem.sty}}{}
\ProvidesFile{thc.sty}
             [\FMithmInfo]
%    \end{macrocode}
%
% \begin{macro}{\th@change}
%    The \texttt{change} theorem style corresponds to the
%    \texttt{change break} style without a linebreak after the header.
%    To say it in another way, it's the same as the \texttt{plain}
%    style but with number and name interchanged and |\slshape| as the
%    default font.
%    \begin{macrocode}
\gdef\th@change{\normalfont\slshape
  \def\@begintheorem##1##2{\item
       [\hskip\labelsep \theorem@headerfont ##2\ ##1]}%
\def\@opargbegintheorem##1##2##3{%
  \item[\hskip\labelsep \theorem@headerfont ##2\ ##1\ (##3)]}}
\endgroup
%</thc>
%    \end{macrocode}
% \end{macro}
%
%
% \subsubsection{The \texttt{marginbreak} style}
%
% This style option is the one used most often at Mainz.  It is saved
% in the file |thmb.sty|.
%    \begin{macrocode}
%<*thmb>
\begingroup \makeatletter
\@ifundefined{theorem@style}{\input{theorem.sty}}{}
\ProvidesFile{thmb.sty}
             [\FMithmInfo]
%    \end{macrocode}
%
% \begin{macro}{\th@marginbreak}
%    The \texttt{margin break} style is nearly the same as the
%    \texttt{change break} style. The only difference is the placement
%    of the theorem number.  We use |\llap| to place it in the left
%    margin.
%
%    In this style |\labelsep| denotes the separation between the
%    number and the text.
%    \begin{macrocode}
\gdef\th@marginbreak{\normalfont\slshape
  \def\@begintheorem##1##2{\item
       [\rlap{\vbox{\theorem@headerfont
                    \hbox{\llap{##2}\hskip\labelsep ##1}%
                    \hbox{\strut}}}]}%
\def\@opargbegintheorem##1##2##3{%
  \item[\rlap{\vbox{\theorem@headerfont
                    \hbox{\llap{##2}\hskip\labelsep ##1\ (##3)}%
                    \hbox{\strut}}}]}}
\endgroup
%</thmb>
%    \end{macrocode}
% \changes{v2.1c}{90/02/17}{`endgroup outside macrocode env.}
% \end{macro}
%
%
%
% \subsubsection{The \texttt{margin} style}
%
% This style option is stored in the file |thm.sty|.
%    \begin{macrocode}
%<*thm>
\begingroup \makeatletter
\@ifundefined{theorem@style}{\input{theorem.sty}}{}
\ProvidesFile{thm.sty}
             [\FMithmInfo]
%    \end{macrocode}
%
% \begin{macro}{\th@margin}
%    Again this is only a variant of the theorem styles described
%    above without any new ideas.
%    \begin{macrocode}
\gdef\th@margin{\normalfont\slshape
  \def\@begintheorem##1##2{\item
       [\theorem@headerfont \llap{##2}\hskip\labelsep ##1]}%
\def\@opargbegintheorem##1##2##3{%
  \item[\theorem@headerfont \llap{##2}\hskip\labelsep ##1\ (##3)]}}
\endgroup
%</thm>
%    \end{macrocode}
% \end{macro}
%
%
%
%
% \subsection{Final Definitions}
%
% \begin{macro}{\theorempreskipamount}
% \begin{macro}{\theorempostskipamount}
%    The \textsf{skip} parameters that regulate the vertical empty space
%    before and after the {\envfont theorem} environment have
%    to be allocated as well.
% \changes{v2.0f}{89/07/20}{Unnecessary `global' removed.}
%    \begin{macrocode}
%<*package>
\newskip\theorempreskipamount
\newskip\theorempostskipamount
%    \end{macrocode}
%    Since we have used the same values for all theorem sets, we now can
%    assign them.
%    \begin{macrocode}
\global\setlength\theorempreskipamount{12pt plus 5pt minus 3pt}
\global\setlength\theorempostskipamount{8pt plus 3pt minus 1.5pt}
%    \end{macrocode}
% \end{macro}
% \end{macro}
%
%
% \begin{macro}{\@endtheorem}
%    The same holds for the macro |\@endtheorem|, which ends a
%    {\envfont theorem} environment. Since it is the same for all
%    theorem sets, it is removed from the macros |\th@|\meta{style}.
%    It simply ends the {\envfont trivlist} environment, which was
%    begun in |\@thm|.
%    \begin{macrocode}
\global\let\@endtheorem=\endtrivlist
%    \end{macrocode}
% \end{macro}
%
%
%
% \begin{macro}{\@preamblecmds}
%    All macros defined above are to be used only in the preamble.
%    Therefore, we insert them in |\@preamblecmds| which will disable
%    them at begin document. This is done by the internal \LaTeXe{}
%    command |\@onlypreamble|.
% \changes{v2.0g}{89/09/19}{`theoremheaderfont and `theorembodyfont
%    added.}
% \changes{v2.2a}{94/02/03}{Use \LaTeXe{} interface}
%    \begin{macrocode}
\@onlypreamble\@xnthm
\@onlypreamble\@ynthm
\@onlypreamble\@othm
\@onlypreamble\newtheorem
\@onlypreamble\theoremstyle
\@onlypreamble\theorembodyfont
\@onlypreamble\theoremheaderfont
%    \end{macrocode}
% \end{macro}
%
% Finally we declare the \texttt{plain} theorem style to be the default.
% \changes{v2.1b}{90/06/04}{Fixed missing `item by declaring default
%    style.}
%    \begin{macrocode}
\theoremstyle{plain}
%</package>
%    \end{macrocode}
%
% \Finale

\endinput
}{}
\ProvidesFile{thc.sty}
             [\FMithmInfo]
%    \end{macrocode}
%
% \begin{macro}{\th@change}
%    The \texttt{change} theorem style corresponds to the
%    \texttt{change break} style without a linebreak after the header.
%    To say it in another way, it's the same as the \texttt{plain}
%    style but with number and name interchanged and |\slshape| as the
%    default font.
%    \begin{macrocode}
\gdef\th@change{\normalfont\slshape
  \def\@begintheorem##1##2{\item
       [\hskip\labelsep \theorem@headerfont ##2\ ##1]}%
\def\@opargbegintheorem##1##2##3{%
  \item[\hskip\labelsep \theorem@headerfont ##2\ ##1\ (##3)]}}
\endgroup
%</thc>
%    \end{macrocode}
% \end{macro}
%
%
% \subsubsection{The \texttt{marginbreak} style}
%
% This style option is the one used most often at Mainz.  It is saved
% in the file |thmb.sty|.
%    \begin{macrocode}
%<*thmb>
\begingroup \makeatletter
\@ifundefined{theorem@style}{% \iffalse meta-comment
%
% Copyright (C) 1993-2024
%
% The LaTeX Project and any individual authors listed elsewhere
% in this file.
%
% This file is part of the Standard LaTeX `Tools Bundle'.
% -------------------------------------------------------
%
% It may be distributed and/or modified under the
% conditions of the LaTeX Project Public License, either version 1.3c
% of this license or (at your option) any later version.
% The latest version of this license is in
%    https://www.latex-project.org/lppl.txt
% and version 1.3c or later is part of all distributions of LaTeX
% version 2005/12/01 or later.
%
% The list of all files belonging to the LaTeX `Tools Bundle' is
% given in the file `manifest.txt'.
%
% \fi
%
%%
%
%
\def\FMithmInfo{2023/07/05 v2.2c Theorem extension package (FMi)}
%
% \ProvidesFile{theorem.dtx}[\FMithmInfo]
%\iffalse   % this is a METACOMMENT !
%%
%% Package `theorem' to use with LaTeX2e
%% Copyright (C) 1989-1995 Frank Mittelbach, all rights reserved.
%<+package>\@ifundefined{theorem@style}{}{\endinput}
%<+package>\NeedsTeXFormat{LaTeX2e}[1995/06/01]
%<+package>\ProvidesPackage{theorem}
%<+package>         [\FMithmInfo]
%
% \fi
%
% \changes{v1.0}{87/05/07}{First draft.}
% \changes{v1.0a}{87/05/07}{`thistheoremstyle' added.}
% \changes{v1.1a}{87/07/29}{New concept: redefine the internal
%                          \protect\LaTeX{} commands `@xnthm' etc.}
% \changes{v1.1b}{87/07/31}{`theoremfont' used to specify font.}
% \changes{v1.2a}{88/01/08}{`@endtheorem' now globally defined.}
% \changes{v1.2b}{88/01/11}{`newtheorem' only allowed in the preamble.}
% \changes{v1.3a}{88/02/14}{`theoremfont' now a toks register.}
%
% \changes{v2.0a}{88/11/16}{New implementation and new documentation.}
% \changes{v2.0a}{88/11/16}{No layout changes but internal modifications
%                           old substyles won't work.}
% \changes{v2.0b}{89/01/30}{`global' used for all definitions.}
% \changes{v2.0c}{89/06/15}{Documentation translated to English.
%                            Thanks to LLL and Nina}
% \changes{v2.1a}{90/02/15}{Support for new font selection.}
% \changes{v2.2a}{94/02/03}{Cleanup for LaTeX2e}
%
% \DoNotIndex{\@endparenv,\@ifdefinable,\@ifnextchar,\@ifundefined}
% \DoNotIndex{\@namedef,\@nameuse,\@spaces,\@tempa,\@warning}
% \DoNotIndex{\ }
% \DoNotIndex{\begingroup,\bfseries}
% \DoNotIndex{\csname}
% \DoNotIndex{\def,\do,\docdate}
% \DoNotIndex{\endcsname,\endgroup,\endinput,\endtrivlist,\expandafter}
% \DoNotIndex{\filedate,\fileversion}
% \DoNotIndex{\gdef,\global}
% \DoNotIndex{\hbox,\hskip}
% \DoNotIndex{\ignorespaces,\input,\itshape,\item}
% \DoNotIndex{\labelsep,\let,\llap}
% \DoNotIndex{\makeatletter}
% \DoNotIndex{\noexpand}
% \DoNotIndex{\refstepcounter,\relax,\rlap}
% \DoNotIndex{\setlength,\slshape,\space,\string,\strut}
% \DoNotIndex{\the,\trivlist,\typeout}
% \DoNotIndex{\vbox}
% \DoNotIndex{\xdef}
%
% \def\deflabel#1{\ttfamily #1\hfill}
% \def\deflist#1{\begin{list}{}{\settowidth\labelwidth{\ttfamily #1}%
%                               \setlength\leftmargin\labelwidth
%                               \addtolength\leftmargin\labelsep
%                               \let\makelabel\deflabel}}
% \def\enddeflist{\end{list}}
% \def\envfont{\normalfont\ttfamily}
%
% \MakeShortVerb{\|}
%
% \newcommand{\TUB}{{\sl TUGboat\/}}
%
% \GetFileInfo{theorem.dtx}
% \title{An Extension of the
%        \LaTeX{} theorem environment\thanks{This file
%        has version number \fileversion, last
%        revised \filedate.}}
% \author{Frank Mittelbach}
%
% \MaintainedByLaTeXTeam{tools}
% \maketitle
%
% \begin{abstract}
%    The macros described in this paper yield an extension of the
%    \LaTeX{} theorem mechanism. It is designed to satisfy the
%    different requirements of various journals. Thus, the
%    layout of the ``theorems'' can be manipulated by determining a
%    ``style''. This article describes not only the use, but
%    also the definition, of the necessary macros.
% \end{abstract}
%
% \section*{Preface to version 2.2}
%
% For \LaTeXe{} this package did not need any fundamental changes. I
% only modified the messages generated so that theorem layout styles
% will show up with the |\listfiles| command and cleaned the section
% on the New Font Selection Scheme since this is now included in
% \LaTeX.
%
% \section*{Preface to version 2.1}
%
% This version is identical to 2.0g described in \TUB\ 10\#3 except
% for some internal defaults which are now set depending on the used
% font selection scheme.
%
% This was done to avoid unpleasant surprises if the new font selection
% scheme is in force. For further details see
% section~\ref{sec:newfonts} and \cite{bk:GMS94}.
%
% \section{Introduction}
%
% For our purposes here, ``theorems'' are labelled enunciations,
% often set off from the main text by extra space and a font change.
% Theorems, corollaries, conjectures, definitions, and remarks
% are all instances of ``theorems''.  The ``header'' of these structures
% is composed of a label (such as \textsc{Theorem} or \textsc{Remark})
% and a number which serializes an item in the sequence of items
% with the same label.
%
% Shortly after the introduction of \LaTeX{} at the Fachbereich
% Mathematik in Mainz, the desire to manipulate the layout of
% ``theorems'' arose.  In Mainz, the following two conventions came
% into general use:
% \begin{enumerate}
%   \item The number of the theorem is shown in the margin.
%   \item There is a line break at
%         the end of the theorem header.
% \end{enumerate}
%
% Additionally, some journals require different formats which depend on
% the ``sort of theorem'': e.g.\ often remarks and definitions are set
% in |\upshape|, while |\itshape| is employed for main theorems.
%
% Confronted with these requirements, a theorem environment was
% developed in Mainz which allows separate determination of the
% layout of the ``theorems sets'', comparable to |\pagestyle|.
%
%
% \section{The user interface}
% \subsection{Defining new theorem sets}
%
% \DescribeMacro\newtheorem
% As in the original \LaTeX{} version, the command |\newtheorem|
% defines a new ``theorem set'' or ``theorem-like structure''.
% Two required arguments name the new environment and give the text
% to be typeset with each instance of the new ``set'', while
% an optional argument determines how
% the ``set'' is enumerated:
% \begin{description}
%    \item[\ttfamily \bslash newtheorem\{foo\}\{bar\}]
%       The theorem set {\envfont foo} (whose name is \texttt{bar})
%       uses its own counter.
%    \item[\ttfamily \bslash newtheorem\{foo2\}{[foo]}\{bar2\}]
%       The theorem set {\envfont foo2} (printed name \texttt{bar2})
%       uses the same counter as the theorem set \texttt{foo}.
%    \item[\ttfamily \bslash newtheorem\{foo3\}\{bar3\}{[section]}]
%       The theorem set {\envfont foo3} (printed name \texttt{bar3}) is
%       enumerated within the counter \texttt{section}, i.e.\ with every
%       new |\section| the enumeration begins again with 1, and
%       the enumeration is composed from the section-number and the
%       theorem counter itself.
% \end{description}
%
% \DescribeMacro\theoremstyle
% Additionally, the command |\theoremstyle| can define the layout
% of various, or all, theorem sets.  It should be noted that any
% theorem set defined by |\newtheorem| is typeset in the
% |\theoremstyle| that is current at the time of the definition.
% Thus, the following
% \begin{verbatim}
%    \theoremstyle{break}        \newtheorem{Cor}{Corollary}
%    \theoremstyle{plain}        \newtheorem{Exa}{Example}[section]
%\end{verbatim}                             ^^A No blank at line start
% leads to the result that the set {\envfont Cor} is formatted in the
% style \texttt{break}, while the set {\envfont Exa} and all the
% following ones are formatted in the style \texttt{plain}, unless
% another |\theoremstyle| follows.  Since the definitions installed by
% |\newtheorem| are global, one also can limit |\theoremstyle| locally
% by grouping braces.
%
% \DescribeMacro\theorembodyfont
% The choice of the font for the theorem body is completely
% independent of the chosen |\theoremstyle|; this has proven to
% be very advantageous.  For example,
% \begin{verbatim}
%   {\theorembodyfont{\upshape}          \newtheorem{Rem}{Remark}}
%\end{verbatim}
% defines a theorem set {\envfont Rem}, which will be set in
% |\upshape| in the current layout (which in our example is
% \texttt{plain}).  As with |\theoremstyle|, the |\theorembodyfont|
% chosen is that current at the time of |\newtheorem|.  If
% |\theorembodyfont| is not specified or one defines
% |\theorembodyfont{}|, then the font used will be that defined by the
% |\theoremstyle|.
%
% \DescribeMacro\theoremheaderfont It is also possible to customize
% the font used for the theorem headers.  This is, however, a global
% declaration, and therefore there should be at most one
% |\theoremheaderfont| declaration in the preamble.\footnote{If it is
% actually necessary to have different header fonts, one has to define
% new theorem styles (substituting the desired font) or specify the
% information directly in the \texttt{\bslash newtheorem} declaration
% (the unclean variant).}
%
% \DescribeMacro\theorempreskipamount
% \DescribeMacro\theorempostskipamount Two additional parameters
% affect the vertical space around the theorem environments:
% |\theorempreskipamount| and |\theorempostskipamount| define,
% respectively, the spacing before and after such an environment.
% These parameters apply for all theorem sets and can be manipulated
% with the ordinary length macros.  They are rubber lengths
% (`\textsf{skips}'), and therefore can contain \texttt{plus} and
% \texttt{minus} parts.
%
% Since the definition of theorem sets should---most sensibly---be
% placed in the preamble, we only allow installation there.
% It is therefore possible to release the memory used here after
% |\begin{document}|, in order to make room for other
% applications.
%
%
% \subsection{Existing theorem styles}
%
% The following theorem styles exist to date:
% \begin{deflist}{marginbreak}
%    \item[plain]
%       This theorem style emulates the original \LaTeX{} definition,
%       except that additionally the parameters
%       |\theorem...skipamount| are used.
%    \item[break]
%       In this style, the theorem header is followed by a line
%       break.
%    \item[marginbreak]
%       The theorem number is set in the margin, and there is a line
%       break as in \texttt{break}.
%    \item[changebreak]
%       Like \texttt{break}, but with header number and text
%       interchanged.
%    \item[change]
%       Header number and text are interchanged, without a line break.
%    \item[margin]
%       The number is set in the left margin, without a line break.
% \end{deflist}
% All styles (except \texttt{plain}) select |\slshape| as the default
% |\theorembodyfont|.
%
%
% \subsection{Examples}
%
% Given the above theorem sets {\envfont Cor}, {\envfont Exa}
% and {\envfont Rem},
% suppose that the preamble also contains the declarations:
% \begin{verbatim}
%    \theoremstyle{marginbreak}   \newtheorem{Lem}[Cor]{Lemma}
%    \theoremstyle{change}
%    \theorembodyfont{\itshape}        \newtheorem{Def}[Cor]{Definition}
%
%    \theoremheaderfont{\scshape}
%\end{verbatim}
% Then the following are some typical examples of the typeset output
% resulting from their use.
%
% \begin{Cor}
%    This is a sentence typeset in the theorem environment {\envfont
%    Cor}.
% \end{Cor}
%
% \begin{Exa}
%    This is a sentence typeset in the theorem environment {\envfont
%    Exa}.
% \end{Exa}
%
% \begin{Rem}
%    This is a sentence typeset in the theorem environment {\envfont
%    Rem}.
% \end{Rem}
%
% \begin{Lem}[Ben User]
%    This is a sentence typeset in the theorem environment {\envfont
%    Lem}.
% \end{Lem}
%
% \begin{Def}[Very impressive Definition]
%    This is a sentence typeset in the theorem environment {\envfont
%    Def}.
% \end{Def}
%
% The last two examples show the effect of the optional argument to
% a theorem environment (it is the text typeset in parentheses).
%
%
% \section{Special Considerations}
% \label{sec:newfonts}
%
% Theoremheader and body are implemented as a unit. This means that
% the |\theoremheaderfont|  will inherit characteristics of the
% |\theorembodyfont| in \LaTeXe. Thus, if for example
% |\theorembodyfont| is |\itshape| and |\theoremheaderfont|
% is |\bfseries| the font selected for the header will have the
% characteristics `bold extended italic'. If this is not desired one
% should set the |\theoremheaderfont| to something like
% \begin{verbatim}
%      \theoremheaderfont{\normalfont\bfseries}
%\end{verbatim}
% i.e.\ supplying all necessary font information explicitly.
%
%
%
% \section{Acknowledgements}
%
% The publication of this set of macros was only possible with the
% help of Christina Busse (translating the manuscript into English),
% Joachim Pense (playing the r\^ole of typist), Chris Rowley
% (looking everything over) and many others providing useful
% suggestions.
%
% \MaybeStop{%
%    \begin{thebibliography}{9}
%       \bibitem{bk:GMS94} \textsc{M.~Goossens}, \textsc{F.~Mittelbach}
%         and \textsc{A.~Samarin}.
%         \newblock The \LaTeX{} Companion.
%         \newblock
%         Addison-Wesley, Reading, Massachusetts, 1994.
%       \bibitem{latex:source} \textsc{Lamport, Leslie.} \newblock
%                               \texttt{latex.tex}, version 2.09, date
%                               Feb. 1990.
%    \end{thebibliography}
%    \PrintIndex
%    \PrintChanges}
%
%
% \section{The documentation driver file}
%
% The next bit of code contains the documentation driver file for
% \TeX{}, i.e., the file that will produce the documentation you are
% currently reading. It will be extracted from this file by the
% \texttt{docstrip} program. Since it is the first code in the file
% one can alternatively process this file directly with \LaTeXe{} to
% obtain the documentation.
%
%    \begin{macrocode}
%<*driver>
\documentclass{ltxdoc}

\usepackage{theorem}

%<+driver>% The next few lines define theorem sets which are used
%<+driver>% in the example section of the documentation.

\theoremstyle{break}       \newtheorem{Cor}{Corollary}
\theoremstyle{plain}       \newtheorem{Exa}{Example}[section]
{\theorembodyfont{\upshape}\newtheorem{Rem}{Remark}}
\theoremstyle{marginbreak} \newtheorem{Lem}[Cor]{Lemma}
\theoremstyle{change}
\theorembodyfont{\itshape} \newtheorem{Def}[Cor]{Definition}

\theoremheaderfont{\scshape}

\RecordChanges

\begin{document}
   \DocInput{theorem.dtx}
\end{document}
%</driver>
%    \end{macrocode}
%
% \section{Definition of the Macros}
%
% If the file has been loaded before, we abort immediately. If not the
% package announces itself (this is actually done at the very top of
% the file---the way it is done isn't good style so don't copy it).
% \changes{v2.0e}{89/07/19}{Spaces removed from `typeout'.}
% \changes{v2.2b}{95/11/19}{Announce moved to top of file}
%    \begin{macrocode}
%<*package>
%\@ifundefined{theorem@style}{}{\endinput}
%\def\FMithmInfo{1995/11/19 v2.2b Theorem extension package (FMi)}
%\ProvidesPackage{theorem}[\FMithmInfo]
%    \end{macrocode}
%
%
% \subsection{Definition of theorem styles and fonts}
%
% All the definitions in this file are done globally to allow
% inputting this file inside a group.
%
% \begin{macro}{\theoremstyle}
%    Before a theorem style can be installed, the chosen style must be
%    known. For that reason, we must test to see that
%    |\th@|\meta{style} is known or, more precisely, that it is
%    different from |\relax|. If the style is not known then
%    |\th@plain| is used.
%    \begin{macrocode}
\gdef\theoremstyle#1{%
   \@ifundefined{th@#1}{\@warning
          {Unknown theoremstyle `#1'. Using `plain'}%
          \theorem@style{plain}}%
%    \end{macrocode}
%    We save the theorem style to be used in the token register
%    |\theorem@style|.
%    \begin{macrocode}
      {\theorem@style{#1}}%
%    \end{macrocode}
%    Now we ``evaluate'' the theorem style: this means, we call the
%    macro |\th@|\meta{style} which will activate the relevant
%    definitions which are contained in a separate file.  This is done
%    in a group to suppress changes to the current font.  This could
%    otherwise pose problems together with the new font selection
%    scheme\footnote{When I printed the original article using the new
%    font selection scheme I ended with a document with slanted
%    typefaces (text headings and all) simply because one of the
%    theorem styles used \texttt{\bslash sl} at toplevel.} if the
%    |\th@|\meta{style} is evaluated a second time.
% \changes{v2.1a}{90/02/15}{Group added around `th@..}
% \changes{v2.1a}{90/02/15}{Reseting of `..@begintheorem removed.}
%    \begin{macrocode}
      \begingroup
        \csname th@\the\theorem@style \endcsname
      \endgroup}
%    \end{macrocode}
% \end{macro}
%
%
% \begin{macro}{\@begintheorem}
% \changes{v2.1a}{90/02/15}{Relaxed at toplevel.}
% \begin{macro}{\@opargbegintheorem}
% \changes{v2.1a}{90/02/15}{Relaxed at toplevel.}
%    We reset |\@begintheorem| and |\@opargbegintheorem| to
%    |\relax| since these commands are no longer necessary at
%    toplevel. This will save a few tokens.
%    \begin{macrocode}
\global\let\@begintheorem\relax
\global\let\@opargbegintheorem\relax
%    \end{macrocode}
% \end{macro}
% \end{macro}
%
% \begin{macro}{\theorem@style}
%    Obviously the \textsf{token} register used above has to be
%    allocated. To assure the utmost compatibility with the
%    original \LaTeX{} definition, we set the default theorem style to
%    \texttt{plain}, which implements the usual \LaTeX{} convention.
% \changes{v2.0f}{89/07/20}{Unnecessary `global' removed.}
%    \begin{macrocode}
\newtoks\theorem@style
\global\theorem@style{plain}
%    \end{macrocode}
% \end{macro}
%
% \begin{macro}{\theorembodyfont}
% \begin{macro}{\theorem@bodyfont}
%    If the |\theorembodyfont| is set by the user then it should not
%    interact with the default font set in the theorem style. When the
%    new font selection is in force this may happen if, for example,
%    the default is |\itshape| and the new |\theorembodyfont| is
%    |\sffamily|. So we add a |\reset@font| command in front of the
%    user definition.
% \changes{v2.0f}{89/07/20}{Unnecessary `global' removed.}
% \changes{v2.0g}{89/09/19}{`theoremfont to `theorembodyfont.}
% \changes{v2.1c}{92/06/22}{`reset@font added and logic changed.}
% \changes{v2.1d}{92/10/28}{Reset to empty if arg is empty}
%    \begin{macrocode}
\gdef\theorembodyfont#1{%
%    \end{macrocode}
%    We check if the argument supplied is empty and if so put nothing
%    into the |\theorem@bodyfont| token register to allow for
%    |\theorembodyfont{}| as a mean of using the default of the
%    current |\theoremstyle|.
%    \begin{macrocode}
   \def\@tempa{#1}%
   \ifx\@tempa\@empty
    \theorem@bodyfont{}%
   \else
    \theorem@bodyfont{\reset@font#1}%
   \fi
   }
\newtoks\theorem@bodyfont
\global\theorem@bodyfont{}
%    \end{macrocode}
% \end{macro}
% \end{macro}
%
%
% \begin{macro}{\theoremheaderfont}
% \changes{v2.0g}{89/09/19}{Macro added.}
%    The font for the theorem headers is handled differently because
%    this definition applies to all theorem styles.
%    \begin{macrocode}
\gdef\theoremheaderfont#1{\gdef\theorem@headerfont{#1}%
%    \end{macrocode}
%    After using the macro once it is redefined to produce an error
%    message.
%    \begin{macrocode}
       \gdef\theoremheaderfont##1{%
        \typeout{\string\theoremheaderfont\space should be used
                 only once.}}}
%    \end{macrocode}
% \end{macro}
%
% \begin{macro}{\theorem@headerfont}
% \changes{v2.0g}{89/09/19}{Macro added.}
%    To set the |\theorem@headerfont| default we first test if
%    the new fontselection scheme is in force.
% \changes{v2.1a}{90/02/15}{Support for new font selection.}
%    \begin{macrocode}
\ifx\upshape\undefined
%    \end{macrocode}
%    If not we define it to expand into |\bfseries|. We don't use
%    |\let| just in case a following style option redefines this
%    macro.
%    \begin{macrocode}
\gdef\theorem@headerfont{\bfseries}
%    \end{macrocode}
%    Otherwise we reset the current shape before calling |\bfseries|.
%    \begin{macrocode}
\else \gdef\theorem@headerfont{\normalfont\bfseries}\fi
%    \end{macrocode}
% \end{macro}
%
% \begin{macro}{\th@plain}
% \begin{macro}{\th@break}
% \begin{macro}{\th@marginbreak}
% \begin{macro}{\th@changebreak}
% \begin{macro}{\th@change}
% \begin{macro}{\th@margin}
%    The different styles are defined in macros such as
%    |\th@plain|.  Since memory space is precious in
%    ``non-Big-versions'', we have to avoid offering too many unused
%    definitions.  Therefore we define these styles in separate files
%    that can be loaded on demand.  Thus the commands themselves only
%    load these files. We use |\@input@|, a \LaTeXe{} internal command
%    that ensures that the file will be listed with |\listfiles|.
% \changes{v2.2a}{94/02/02}{Use `@input@ to load theorem layout files}
%    \begin{macrocode}
\gdef\th@plain{\@input@{thp.sty}}
\gdef\th@break{\@input@{thb.sty}}
\gdef\th@marginbreak{\@input@{thmb.sty}}
\gdef\th@changebreak{\@input@{thcb.sty}}
\gdef\th@change{\@input@{thc.sty}}
\gdef\th@margin{\@input@{thm.sty}}
%    \end{macrocode}
%    This list will be expanded when new styles become
%    available.  For testing,  just append new theorem substyles
%    as document options.
% \end{macro}
% \end{macro}
% \end{macro}
% \end{macro}
% \end{macro}
% \end{macro}
%
%
%
% \subsection{Definition of a new theorem set}
%
% As already pointed out, a new theorem environment can be defined in
% three different ways:
% \begin{verbatim}
%    \newtheorem{Lem}{Lemma}
%    \newtheorem{Lem}{Lemma}[section]
%    \newtheorem{Lem}[Theorem]{Lemma}
%\end{verbatim}
% The function of the macro |\newtheorem| is to recognize these cases
% and then to branch into one of the three macros |\@ynthm|, |\@xnthm|
% or |\@othm|.  This mechanism is adopted unchanged from
% \cite{latex:source}; the essential point here is that, for example,
% in the second case, the arguments \texttt{Lem}, \texttt{Lemma} and
% \texttt{section} are passed over to the macro |\@xnthm|.
%
% We inspect this case first because the others present fewer problems,
% and thus are easily derived from this one.
%
% \begin{macro}{\@xnthm}
%    For our example arguments, the macro |\@xnthm| must fulfill
%    the following:
%    \begin{itemize}
%       \item
%          Define a new \LaTeX-counter `Lem'
%       \item
%          reset this counter within a |\section|
%       \item
%          define the macro |\theLem|
%       \item
%          define the environment macros |\Lem| and
%          |\endLem| using the current
%          |\theoremstyle| and |\theorem@bodyfont|.
%    \end{itemize}
%    Obviously, all this should happen only if the first argument
%    of |\@xnthm| (i.e.\ \texttt{Lem} in our example) is chosen so
%    as not to conflict with any previously defined commands or
%    environments.  This test is performed by the \LaTeX{} macro
%    |\@ifdefinable|.
%    \begin{macrocode}
\gdef\@xnthm#1#2[#3]{\expandafter\@ifdefinable\csname #1\endcsname
%    \end{macrocode}
%    Therefore, the first argument of |\@ifdefinable| is the
%    expansion (in the example, |\Lem|) of
%    |\csname||#1\endcsname|.  The second argument is
%    executed only if the test has been completed successfully.
%    \begin{macrocode}
   {%
%    \end{macrocode}
%    Now we define the new counter. The names of the \LaTeX{}
%    macros employed should speak for themselves:
% \changes{v2.2b}{1995/10/07}{Use \cs{@newctr}}
% \changes{v2.2c}{1995/11/23}{Use \cs{@newctr} properly.}
%    \begin{macrocode}
    \@definecounter{#1}\@newctr{#1}[#3]%
%    \end{macrocode}
%    Using |\@newctr| will give a proper error message if the counter
%    in |#3| is not defined.
%    In defining `|\theLem|' we must generate the desired macro
%    name by use of |\expandafter| and |\csname|.
%    \begin{macrocode}
    \expandafter\xdef\csname the#1\endcsname
%    \end{macrocode}
%    An |\xdef| is used in order to make the definition global,
%    and to ensure that it contains the replacement texts of
%    |\@thmcountersep| and |\@thmcounter|.\footnote{These
%    two macros can be defined by the document style.  Their default
%    values produce a `\texttt{.}' as separation and an arabic
%    representation of the number.} However, not everything should be
%    expanded. For example, it saves space to use |\thesection|
%    instead of its---at times---lengthy expansion.
%    \begin{macrocode}
      {\expandafter \noexpand \csname the#3\endcsname
       \@thmcountersep \@thmcounter{#1}}%
%    \end{macrocode}
%    Thus with the defaults of \LaTeX{}, |\theLem| would be
%    replaced by the command sequence |\thesection.\arabic{Lem}|.
%
%    We will now look at the definition of the macro which is executed
%    at the beginning of the actual environment (in our example this
%    macro is |\Lem|). It should be noted that we use an
%    ``|\expandafter| trick'' to expand only certain parts of the
%    replacement text at the time of the definition.
%    \begin{macrocode}
    \def\@tempa{\global\@namedef{#1}}%
    \expandafter \@tempa \expandafter{%
%    \end{macrocode}
%    First, the macro that contains the current definitions of
%    |\@begintheorem| and |\@opargtheorem| should be called
%    up.  The name of this macro---as is already known---has the form
%    |\th@|\meta{theorem\ style}; therefore, it must be called by
%    \begin{macrocode}
      \csname th@\the \theorem@style
%    \end{macrocode}
%    In addition the default theorem font should be changeable, i.e.\
%    we have to insert the contents of |\theorem@bodyfont|. For
%    that reason, we expand even further, beyond |\endcsname|,
%    and thus insert the contents of the \textsf{token} register
%    |\theorem@bodyfont| in the replacement text.
%    \begin{macrocode}
            \expandafter \endcsname \the \theorem@bodyfont
%    \end{macrocode}
%    Now it is time to call the macro |\@thm| which takes over
%    the further processing.  It has two arguments: the current
%    counter name (in our example, \texttt{Lem}), and the text of the
%    label (in our example, \texttt{Lemma}).
%    \begin{macrocode}
     \@thm{#1}{#2}}%
%    \end{macrocode}
%    With this, the `sub-definition' is complete.  The macro
%    |\@endtheorem| ends a theorem environment and is, so far, nothing
%    but an |\endtrivlist|.  (Hence it is defined globally, and not
%    within the theorem styles.\footnote{This has to be changed as
%    soon as theorem styles that change \texttt{\bslash @endtheorem}
%    exist.  In such a case, all existing styles must be changed as
%    well since they will have to reset the macro.}) Therefore, we can
%    set it equivalent to the macro that ends the theorem set (in our
%    example, |\endLem|).  However, if some day theorem styles exist
%    that do change |\@endtheorem|, we would have to use the
%    commented-out line instead.
%    \begin{macrocode}
    \global \expandafter \let \csname end#1\endcsname \@endtheorem
%   \global\@namedef{end#1}{\@endtheorem}%
%    \end{macrocode}
%    With these commands all the required definitions are employed,
%    unless the test |\@ifdefinable| has failed.  Therefore, we
%    end the second argument of this macro and with it the definition
%    of |\@xnthm|.
%    \begin{macrocode}
   }}
%    \end{macrocode}
% \end{macro}
%
% \begin{macro}{\@ynthm}
%    The definition of |\@ynthm| is completely analogous. In
%    this case the new counter that is defined is not reset within
%    another counter; thus the definition of |\the...|
%    is simplified:
%    \begin{macrocode}
\gdef\@ynthm#1#2{\expandafter\@ifdefinable\csname #1\endcsname
   {\@definecounter{#1}%
    \expandafter\xdef\csname the#1\endcsname{\@thmcounter{#1}}%
%    \end{macrocode}
%    The rest of the definition corresponds literally to that of
%    |\@xnthm|:
%    \begin{macrocode}
    \def\@tempa{\global\@namedef{#1}}\expandafter \@tempa
     \expandafter{\csname th@\the \theorem@style \expandafter
     \endcsname \the\theorem@bodyfont \@thm{#1}{#2}}%
    \global \expandafter \let \csname end#1\endcsname \@endtheorem}}
%    \end{macrocode}
% \end{macro}
%
% \begin{macro}{\@othm}
%    The definition of |\@othm| does not contain anything new.
%    \begin{macrocode}
\gdef\@othm#1[#2]#3{%
%    \end{macrocode}
% \changes{v2.2b}{95/10/07}{Build in check for counter defined}
%    We do not define a new counter but instead use one that has
%    already been defined.  Thus the only definition we need is that
%    of this pseudo-counter (i.e.\ |\the|\meta{env.\ name}).
%    First we check if |#2| corresponds to a known counter name.
%    \begin{macrocode}
  \expandafter\ifx\csname c@#2\endcsname\relax
   \@nocounterr{#2}%
  \else
   \expandafter\@ifdefinable\csname #1\endcsname
   {\expandafter \xdef \csname the#1\endcsname
     {\expandafter \noexpand \csname the#2\endcsname}%
%    \end{macrocode}
%    All other parts of the definition can be adopted from
%    |\@xnthm|. We have to remember, though, that in this case
%    the name of the current counter and the theorem label have moved
%    to the second and third arguments.
%    \begin{macrocode}
    \def\@tempa{\global\@namedef{#1}}\expandafter \@tempa
     \expandafter{\csname th@\the \theorem@style \expandafter
     \endcsname \the\theorem@bodyfont \@thm{#2}{#3}}%
    \global \expandafter \let \csname end#1\endcsname \@endtheorem}%
  \fi}
%    \end{macrocode}
% \end{macro}
%
% \subsection{Macros that are employed in a {\envfont theorem}
%    environment}
%
% \begin{macro}{\@thm}
%    The macro |\@thm| has to increase the current counter.  Then,
%    depending on whether the environment has (or does not have) an
%    optional argument, it has to branch into either |\@begintheorem|
%    or |\@opargtheorem|.
%    \begin{macrocode}
\gdef\@thm#1#2{\refstepcounter{#1}%
%    \end{macrocode}
%    Now we start a {\envfont trivlist} environment, and give
%    |\@topsep| and |\@topsepadd| the values of the \textsf{skip}
%    registers |\theorempreskipamount| and |\theorempostskipamount|.
%    The value in |\@topsep| is the vertical space that is inserted by
%    the first (and only) |\item| in our |\trivlist| whilst
%    |\@topsepadd| is inserted by |\@endparenv| at the end of that
%    {\envfont trivlist} environment.  By using these registers, we
%    obtain the desired space around a {\envfont theorem} environment.
%    \begin{macrocode}
   \trivlist
   \@topsep \theorempreskipamount               % used by first \item
   \@topsepadd \theorempostskipamount           % used by \@endparenv
%    \end{macrocode}
%    Now we have to test whether an optional argument has been given.
% \changes{v2.0d}{89/07/18}{Forgotten percent added.}
%    \begin{macrocode}
   \@ifnextchar [%
%    \end{macrocode}
%    If there is an optional argument, we will call |\@ythm|, and
%    move the arguments read back into the input stream.
%    \begin{macrocode}
   {\@ythm{#1}{#2}}%
%    \end{macrocode}
%    If not, we call |\@begintheorem|. Its first argument is the
%    name of the theorem set (hence the second argument of
%    |\@thm|). Its second argument is the macro that produces the
%    current number.
%    \begin{macrocode}
   {\@begintheorem{#2}{\csname the#1\endcsname}\ignorespaces}}
%    \end{macrocode}
% \end{macro}
%
% \begin{macro}{\@xthm}
% \begin{macro}{\@ythm}
%    Both these macros were originally called by |\@thm|.  We do
%    not need |\@xthm| anymore, hence we reset it to
%    |\relax|. The definition of |\@ythm| has not changed at
%    all from its definition in \LaTeX. In order to make the macros
%    easier to understand, we will
%    nevertheless present it (commented out).
%    \begin{macrocode}
\global\let\@xthm\relax
% \def\@ythm#1#2[#3]{\@opargbegintheorem{#2}{\csname
%           the#1\endcsname}{#3}\ignorespaces}
%</package>
%    \end{macrocode}
%    The primitive |\ignorespaces| in |\@ythm| and
%    |\@thm| is needed to remove the spaces between the
%    |\begin{...}| and the actual text.
% \end{macro}
% \end{macro}
%
%
% \subsection{Definition of the theorem substyles}
%
% As already pointed out, the theorem substyles, defined below, are
% only loaded when necessary.  Note that all these substyles, except
% \texttt{plain}, have |\slshape| as the default body font.
%
% \subsubsection{The \texttt{plain} style}
%
% \label{dok:th@plain} As the following macros use |@|, we have
% to locally set the |\catcode| of this symbol to ``letter''.
% This happens within a group, so that we do not have to worry about
% which |\catcode| that symbol had before.
%    \begin{macrocode}
%<*thp>
\begingroup \makeatletter
%    \end{macrocode}
% Since we are now within a group, we must make all definitions
% globally.  First we make sure that \texttt{theorem.sty} is loaded.
% This will allow us to use this file as a document style option
% without having to call \texttt{theorem} itself as an option.  At the
% same time, we assure that at least version 2 is loaded, since
% |\theorem@style| was not defined in earlier versions.
%    \begin{macrocode}
\@ifundefined{theorem@style}{\input{theorem.sty}}{}
\ProvidesFile{thp.sty}
             [\FMithmInfo]
%    \end{macrocode}
%
%
% \begin{macro}{\th@plain}
% \begin{macro}{\@begintheorem}
% \begin{macro}{\@opargbegintheorem}
%    |\theoremstyle{plain}| corresponds to the original
%    definition, except that the distances to the surrounding text are
%    determined by the parameters |\theorempreskipamount| and
%    |\theorempostskipamount|.
%    First we set the default body font.
%    \begin{macrocode}
\gdef\th@plain{\normalfont\itshape
%    \end{macrocode}
%    Then we define |\@begintheorem| and
%    |\@opargbegintheorem|. These two macros define how the
%    header of a theorem is typeset.  |\@opargbegintheorem| will
%    be called if a \textsf{theorem} environment with an optional
%    argument is encountered; otherwise, the header is constructed by
%    calling |\@begintheorem|.  If one of these macros is
%    executed, we are within a \textsf{trivlist} environment started by
%    |\@thm|.  So the theorem header is produced with an
%    |\item| command.
%
%    Instead of specifying the header font directly, all standard
%    theorem styles use the |\theorem@headerfont| macro to allow
%    customization.  The extra space (|\labelsep|) is necessary
%    because of problems in the \textsf{trivlist} environment.
%    \begin{macrocode}
  \def\@begintheorem##1##2{%
        \item[\hskip\labelsep \theorem@headerfont ##1\ ##2]}%
%    \end{macrocode}
%    The definition of |\@opargbegintheorem| is completely
%    analogous. The only difference is the fact that there exists a
%    third argument (which is the optional parameter of the
%    environment and contains additional information about the
%    theorem).  Customarily we enclose it in parentheses.
%    \begin{macrocode}
\def\@opargbegintheorem##1##2##3{%
   \item[\hskip\labelsep \theorem@headerfont ##1\ ##2\ (##3)]}}
%    \end{macrocode}
% \end{macro}
% \end{macro}
% \end{macro}
%
% \noindent
% We conclude with an |\endgroup| to restore the
% |\catcode| of |@|.
%    \begin{macrocode}
\endgroup
%</thp>
%    \end{macrocode}
%
%
% \subsubsection{The \texttt{break} style}
%
% This style option is stored in the file |thb.sty|.
% For the next two lines see the documentation for |\th@plain| on
% page~\pageref{dok:th@plain}.
%    \begin{macrocode}
%<*thb>
\begingroup \makeatletter
\@ifundefined{theorem@style}{\input{theorem.sty}}{}
\ProvidesFile{thb.sty}
             [\FMithmInfo]
%    \end{macrocode}
%
% \begin{macro}{\th@break}
%    |\theoremstyle{break}| produces a line break after the name
%    of the theorem.  The font is |\slshape|.  Hence, we define
%    |\th@break| as follows:
%    \begin{macrocode}
\gdef\th@break{\normalfont\slshape
  \def\@begintheorem##1##2{\item[%
%    \end{macrocode}
%    We run into the following problem: it is not possible to create
%    the header with |\item[|\meta{title}|]| and then start
%    a new line by, for example, |\mbox{}\\|. Such a definition
%    will fail whenever a list environment follows immediately.  With
%    the above construction, the |\mbox{}| causes the switch
%    |@inlabel| (cf.\ definition of |\list| and
%    |\trivlist| in \cite{latex:source}) to be set to \texttt{false}
%    and so the following list will insert additional vertical
%    space (|\topskip|). This is quite annoying. Therefore, we
%    create the line break within the |\item|. In order to ensure
%    that the text will begin at the proper position in the following
%    line, we simply pretend that the label does not take any
%    room.\footnote{This will lead to problems whenever very high
%    symbols occurring in the line tower into the
%    heading. So, something else has to be done here sometime.}
%    \begin{macrocode}
       \rlap{\vbox{\hbox{\hskip \labelsep\theorem@headerfont ##1\ ##2}%
                   \hbox{\strut}}}]}%
%    \end{macrocode}
%    Again, the definition of |\@opargbegintheorem| is
%    completely analogous.
%    \begin{macrocode}
\def\@opargbegintheorem##1##2##3{%
  \item[\rlap{\vbox{\hbox{\hskip \labelsep \theorem@headerfont
                     ##1\ ##2\ (##3)}%
                    \hbox{\strut}}}]}}
\endgroup
%</thb>
%    \end{macrocode}
% \end{macro}
%
%
% \subsubsection{The \texttt{changebreak} style}
%
%    \begin{macrocode}
% This style option is stored in the file |thcb.sty|.
%    \begin{macrocode}
%<*thcb>
\begingroup \makeatletter
\@ifundefined{theorem@style}{\input{theorem.sty}}{}
\ProvidesFile{thcb.sty}
             [\FMithmInfo]
%    \end{macrocode}
%
% \begin{macro}{\th@changebreak}
%    The \texttt{change-break} theorem style is like \texttt{break}
%    but with interchange of theorem name and theorem number.  Thus we
%    define |\th@changebreak| as follows:
%    \begin{macrocode}
\gdef\th@changebreak{\normalfont\slshape
  \def\@begintheorem##1##2{\item
       [\rlap{\vbox{\hbox{\hskip\labelsep \theorem@headerfont ##2\ ##1}%
                    \hbox{\strut}}}]}%
\def\@opargbegintheorem##1##2##3{%
  \item[\rlap{\vbox{\hbox{\hskip\labelsep \theorem@headerfont
                     ##2\ ##1\ (##3)}%
                    \hbox{\strut}}}]}}
\endgroup
%</thcb>
%    \end{macrocode}
% \end{macro}
%
%
% \subsubsection{The \texttt{change} style}
%
% This style option is stored in the file |thc.sty|.
%    \begin{macrocode}
%<*thc>
\begingroup \makeatletter
\@ifundefined{theorem@style}{\input{theorem.sty}}{}
\ProvidesFile{thc.sty}
             [\FMithmInfo]
%    \end{macrocode}
%
% \begin{macro}{\th@change}
%    The \texttt{change} theorem style corresponds to the
%    \texttt{change break} style without a linebreak after the header.
%    To say it in another way, it's the same as the \texttt{plain}
%    style but with number and name interchanged and |\slshape| as the
%    default font.
%    \begin{macrocode}
\gdef\th@change{\normalfont\slshape
  \def\@begintheorem##1##2{\item
       [\hskip\labelsep \theorem@headerfont ##2\ ##1]}%
\def\@opargbegintheorem##1##2##3{%
  \item[\hskip\labelsep \theorem@headerfont ##2\ ##1\ (##3)]}}
\endgroup
%</thc>
%    \end{macrocode}
% \end{macro}
%
%
% \subsubsection{The \texttt{marginbreak} style}
%
% This style option is the one used most often at Mainz.  It is saved
% in the file |thmb.sty|.
%    \begin{macrocode}
%<*thmb>
\begingroup \makeatletter
\@ifundefined{theorem@style}{\input{theorem.sty}}{}
\ProvidesFile{thmb.sty}
             [\FMithmInfo]
%    \end{macrocode}
%
% \begin{macro}{\th@marginbreak}
%    The \texttt{margin break} style is nearly the same as the
%    \texttt{change break} style. The only difference is the placement
%    of the theorem number.  We use |\llap| to place it in the left
%    margin.
%
%    In this style |\labelsep| denotes the separation between the
%    number and the text.
%    \begin{macrocode}
\gdef\th@marginbreak{\normalfont\slshape
  \def\@begintheorem##1##2{\item
       [\rlap{\vbox{\theorem@headerfont
                    \hbox{\llap{##2}\hskip\labelsep ##1}%
                    \hbox{\strut}}}]}%
\def\@opargbegintheorem##1##2##3{%
  \item[\rlap{\vbox{\theorem@headerfont
                    \hbox{\llap{##2}\hskip\labelsep ##1\ (##3)}%
                    \hbox{\strut}}}]}}
\endgroup
%</thmb>
%    \end{macrocode}
% \changes{v2.1c}{90/02/17}{`endgroup outside macrocode env.}
% \end{macro}
%
%
%
% \subsubsection{The \texttt{margin} style}
%
% This style option is stored in the file |thm.sty|.
%    \begin{macrocode}
%<*thm>
\begingroup \makeatletter
\@ifundefined{theorem@style}{\input{theorem.sty}}{}
\ProvidesFile{thm.sty}
             [\FMithmInfo]
%    \end{macrocode}
%
% \begin{macro}{\th@margin}
%    Again this is only a variant of the theorem styles described
%    above without any new ideas.
%    \begin{macrocode}
\gdef\th@margin{\normalfont\slshape
  \def\@begintheorem##1##2{\item
       [\theorem@headerfont \llap{##2}\hskip\labelsep ##1]}%
\def\@opargbegintheorem##1##2##3{%
  \item[\theorem@headerfont \llap{##2}\hskip\labelsep ##1\ (##3)]}}
\endgroup
%</thm>
%    \end{macrocode}
% \end{macro}
%
%
%
%
% \subsection{Final Definitions}
%
% \begin{macro}{\theorempreskipamount}
% \begin{macro}{\theorempostskipamount}
%    The \textsf{skip} parameters that regulate the vertical empty space
%    before and after the {\envfont theorem} environment have
%    to be allocated as well.
% \changes{v2.0f}{89/07/20}{Unnecessary `global' removed.}
%    \begin{macrocode}
%<*package>
\newskip\theorempreskipamount
\newskip\theorempostskipamount
%    \end{macrocode}
%    Since we have used the same values for all theorem sets, we now can
%    assign them.
%    \begin{macrocode}
\global\setlength\theorempreskipamount{12pt plus 5pt minus 3pt}
\global\setlength\theorempostskipamount{8pt plus 3pt minus 1.5pt}
%    \end{macrocode}
% \end{macro}
% \end{macro}
%
%
% \begin{macro}{\@endtheorem}
%    The same holds for the macro |\@endtheorem|, which ends a
%    {\envfont theorem} environment. Since it is the same for all
%    theorem sets, it is removed from the macros |\th@|\meta{style}.
%    It simply ends the {\envfont trivlist} environment, which was
%    begun in |\@thm|.
%    \begin{macrocode}
\global\let\@endtheorem=\endtrivlist
%    \end{macrocode}
% \end{macro}
%
%
%
% \begin{macro}{\@preamblecmds}
%    All macros defined above are to be used only in the preamble.
%    Therefore, we insert them in |\@preamblecmds| which will disable
%    them at begin document. This is done by the internal \LaTeXe{}
%    command |\@onlypreamble|.
% \changes{v2.0g}{89/09/19}{`theoremheaderfont and `theorembodyfont
%    added.}
% \changes{v2.2a}{94/02/03}{Use \LaTeXe{} interface}
%    \begin{macrocode}
\@onlypreamble\@xnthm
\@onlypreamble\@ynthm
\@onlypreamble\@othm
\@onlypreamble\newtheorem
\@onlypreamble\theoremstyle
\@onlypreamble\theorembodyfont
\@onlypreamble\theoremheaderfont
%    \end{macrocode}
% \end{macro}
%
% Finally we declare the \texttt{plain} theorem style to be the default.
% \changes{v2.1b}{90/06/04}{Fixed missing `item by declaring default
%    style.}
%    \begin{macrocode}
\theoremstyle{plain}
%</package>
%    \end{macrocode}
%
% \Finale

\endinput
}{}
\ProvidesFile{thmb.sty}
             [\FMithmInfo]
%    \end{macrocode}
%
% \begin{macro}{\th@marginbreak}
%    The \texttt{margin break} style is nearly the same as the
%    \texttt{change break} style. The only difference is the placement
%    of the theorem number.  We use |\llap| to place it in the left
%    margin.
%
%    In this style |\labelsep| denotes the separation between the
%    number and the text.
%    \begin{macrocode}
\gdef\th@marginbreak{\normalfont\slshape
  \def\@begintheorem##1##2{\item
       [\rlap{\vbox{\theorem@headerfont
                    \hbox{\llap{##2}\hskip\labelsep ##1}%
                    \hbox{\strut}}}]}%
\def\@opargbegintheorem##1##2##3{%
  \item[\rlap{\vbox{\theorem@headerfont
                    \hbox{\llap{##2}\hskip\labelsep ##1\ (##3)}%
                    \hbox{\strut}}}]}}
\endgroup
%</thmb>
%    \end{macrocode}
% \changes{v2.1c}{90/02/17}{`endgroup outside macrocode env.}
% \end{macro}
%
%
%
% \subsubsection{The \texttt{margin} style}
%
% This style option is stored in the file |thm.sty|.
%    \begin{macrocode}
%<*thm>
\begingroup \makeatletter
\@ifundefined{theorem@style}{% \iffalse meta-comment
%
% Copyright (C) 1993-2024
%
% The LaTeX Project and any individual authors listed elsewhere
% in this file.
%
% This file is part of the Standard LaTeX `Tools Bundle'.
% -------------------------------------------------------
%
% It may be distributed and/or modified under the
% conditions of the LaTeX Project Public License, either version 1.3c
% of this license or (at your option) any later version.
% The latest version of this license is in
%    https://www.latex-project.org/lppl.txt
% and version 1.3c or later is part of all distributions of LaTeX
% version 2005/12/01 or later.
%
% The list of all files belonging to the LaTeX `Tools Bundle' is
% given in the file `manifest.txt'.
%
% \fi
%
%%
%
%
\def\FMithmInfo{2023/07/05 v2.2c Theorem extension package (FMi)}
%
% \ProvidesFile{theorem.dtx}[\FMithmInfo]
%\iffalse   % this is a METACOMMENT !
%%
%% Package `theorem' to use with LaTeX2e
%% Copyright (C) 1989-1995 Frank Mittelbach, all rights reserved.
%<+package>\@ifundefined{theorem@style}{}{\endinput}
%<+package>\NeedsTeXFormat{LaTeX2e}[1995/06/01]
%<+package>\ProvidesPackage{theorem}
%<+package>         [\FMithmInfo]
%
% \fi
%
% \changes{v1.0}{87/05/07}{First draft.}
% \changes{v1.0a}{87/05/07}{`thistheoremstyle' added.}
% \changes{v1.1a}{87/07/29}{New concept: redefine the internal
%                          \protect\LaTeX{} commands `@xnthm' etc.}
% \changes{v1.1b}{87/07/31}{`theoremfont' used to specify font.}
% \changes{v1.2a}{88/01/08}{`@endtheorem' now globally defined.}
% \changes{v1.2b}{88/01/11}{`newtheorem' only allowed in the preamble.}
% \changes{v1.3a}{88/02/14}{`theoremfont' now a toks register.}
%
% \changes{v2.0a}{88/11/16}{New implementation and new documentation.}
% \changes{v2.0a}{88/11/16}{No layout changes but internal modifications
%                           old substyles won't work.}
% \changes{v2.0b}{89/01/30}{`global' used for all definitions.}
% \changes{v2.0c}{89/06/15}{Documentation translated to English.
%                            Thanks to LLL and Nina}
% \changes{v2.1a}{90/02/15}{Support for new font selection.}
% \changes{v2.2a}{94/02/03}{Cleanup for LaTeX2e}
%
% \DoNotIndex{\@endparenv,\@ifdefinable,\@ifnextchar,\@ifundefined}
% \DoNotIndex{\@namedef,\@nameuse,\@spaces,\@tempa,\@warning}
% \DoNotIndex{\ }
% \DoNotIndex{\begingroup,\bfseries}
% \DoNotIndex{\csname}
% \DoNotIndex{\def,\do,\docdate}
% \DoNotIndex{\endcsname,\endgroup,\endinput,\endtrivlist,\expandafter}
% \DoNotIndex{\filedate,\fileversion}
% \DoNotIndex{\gdef,\global}
% \DoNotIndex{\hbox,\hskip}
% \DoNotIndex{\ignorespaces,\input,\itshape,\item}
% \DoNotIndex{\labelsep,\let,\llap}
% \DoNotIndex{\makeatletter}
% \DoNotIndex{\noexpand}
% \DoNotIndex{\refstepcounter,\relax,\rlap}
% \DoNotIndex{\setlength,\slshape,\space,\string,\strut}
% \DoNotIndex{\the,\trivlist,\typeout}
% \DoNotIndex{\vbox}
% \DoNotIndex{\xdef}
%
% \def\deflabel#1{\ttfamily #1\hfill}
% \def\deflist#1{\begin{list}{}{\settowidth\labelwidth{\ttfamily #1}%
%                               \setlength\leftmargin\labelwidth
%                               \addtolength\leftmargin\labelsep
%                               \let\makelabel\deflabel}}
% \def\enddeflist{\end{list}}
% \def\envfont{\normalfont\ttfamily}
%
% \MakeShortVerb{\|}
%
% \newcommand{\TUB}{{\sl TUGboat\/}}
%
% \GetFileInfo{theorem.dtx}
% \title{An Extension of the
%        \LaTeX{} theorem environment\thanks{This file
%        has version number \fileversion, last
%        revised \filedate.}}
% \author{Frank Mittelbach}
%
% \MaintainedByLaTeXTeam{tools}
% \maketitle
%
% \begin{abstract}
%    The macros described in this paper yield an extension of the
%    \LaTeX{} theorem mechanism. It is designed to satisfy the
%    different requirements of various journals. Thus, the
%    layout of the ``theorems'' can be manipulated by determining a
%    ``style''. This article describes not only the use, but
%    also the definition, of the necessary macros.
% \end{abstract}
%
% \section*{Preface to version 2.2}
%
% For \LaTeXe{} this package did not need any fundamental changes. I
% only modified the messages generated so that theorem layout styles
% will show up with the |\listfiles| command and cleaned the section
% on the New Font Selection Scheme since this is now included in
% \LaTeX.
%
% \section*{Preface to version 2.1}
%
% This version is identical to 2.0g described in \TUB\ 10\#3 except
% for some internal defaults which are now set depending on the used
% font selection scheme.
%
% This was done to avoid unpleasant surprises if the new font selection
% scheme is in force. For further details see
% section~\ref{sec:newfonts} and \cite{bk:GMS94}.
%
% \section{Introduction}
%
% For our purposes here, ``theorems'' are labelled enunciations,
% often set off from the main text by extra space and a font change.
% Theorems, corollaries, conjectures, definitions, and remarks
% are all instances of ``theorems''.  The ``header'' of these structures
% is composed of a label (such as \textsc{Theorem} or \textsc{Remark})
% and a number which serializes an item in the sequence of items
% with the same label.
%
% Shortly after the introduction of \LaTeX{} at the Fachbereich
% Mathematik in Mainz, the desire to manipulate the layout of
% ``theorems'' arose.  In Mainz, the following two conventions came
% into general use:
% \begin{enumerate}
%   \item The number of the theorem is shown in the margin.
%   \item There is a line break at
%         the end of the theorem header.
% \end{enumerate}
%
% Additionally, some journals require different formats which depend on
% the ``sort of theorem'': e.g.\ often remarks and definitions are set
% in |\upshape|, while |\itshape| is employed for main theorems.
%
% Confronted with these requirements, a theorem environment was
% developed in Mainz which allows separate determination of the
% layout of the ``theorems sets'', comparable to |\pagestyle|.
%
%
% \section{The user interface}
% \subsection{Defining new theorem sets}
%
% \DescribeMacro\newtheorem
% As in the original \LaTeX{} version, the command |\newtheorem|
% defines a new ``theorem set'' or ``theorem-like structure''.
% Two required arguments name the new environment and give the text
% to be typeset with each instance of the new ``set'', while
% an optional argument determines how
% the ``set'' is enumerated:
% \begin{description}
%    \item[\ttfamily \bslash newtheorem\{foo\}\{bar\}]
%       The theorem set {\envfont foo} (whose name is \texttt{bar})
%       uses its own counter.
%    \item[\ttfamily \bslash newtheorem\{foo2\}{[foo]}\{bar2\}]
%       The theorem set {\envfont foo2} (printed name \texttt{bar2})
%       uses the same counter as the theorem set \texttt{foo}.
%    \item[\ttfamily \bslash newtheorem\{foo3\}\{bar3\}{[section]}]
%       The theorem set {\envfont foo3} (printed name \texttt{bar3}) is
%       enumerated within the counter \texttt{section}, i.e.\ with every
%       new |\section| the enumeration begins again with 1, and
%       the enumeration is composed from the section-number and the
%       theorem counter itself.
% \end{description}
%
% \DescribeMacro\theoremstyle
% Additionally, the command |\theoremstyle| can define the layout
% of various, or all, theorem sets.  It should be noted that any
% theorem set defined by |\newtheorem| is typeset in the
% |\theoremstyle| that is current at the time of the definition.
% Thus, the following
% \begin{verbatim}
%    \theoremstyle{break}        \newtheorem{Cor}{Corollary}
%    \theoremstyle{plain}        \newtheorem{Exa}{Example}[section]
%\end{verbatim}                             ^^A No blank at line start
% leads to the result that the set {\envfont Cor} is formatted in the
% style \texttt{break}, while the set {\envfont Exa} and all the
% following ones are formatted in the style \texttt{plain}, unless
% another |\theoremstyle| follows.  Since the definitions installed by
% |\newtheorem| are global, one also can limit |\theoremstyle| locally
% by grouping braces.
%
% \DescribeMacro\theorembodyfont
% The choice of the font for the theorem body is completely
% independent of the chosen |\theoremstyle|; this has proven to
% be very advantageous.  For example,
% \begin{verbatim}
%   {\theorembodyfont{\upshape}          \newtheorem{Rem}{Remark}}
%\end{verbatim}
% defines a theorem set {\envfont Rem}, which will be set in
% |\upshape| in the current layout (which in our example is
% \texttt{plain}).  As with |\theoremstyle|, the |\theorembodyfont|
% chosen is that current at the time of |\newtheorem|.  If
% |\theorembodyfont| is not specified or one defines
% |\theorembodyfont{}|, then the font used will be that defined by the
% |\theoremstyle|.
%
% \DescribeMacro\theoremheaderfont It is also possible to customize
% the font used for the theorem headers.  This is, however, a global
% declaration, and therefore there should be at most one
% |\theoremheaderfont| declaration in the preamble.\footnote{If it is
% actually necessary to have different header fonts, one has to define
% new theorem styles (substituting the desired font) or specify the
% information directly in the \texttt{\bslash newtheorem} declaration
% (the unclean variant).}
%
% \DescribeMacro\theorempreskipamount
% \DescribeMacro\theorempostskipamount Two additional parameters
% affect the vertical space around the theorem environments:
% |\theorempreskipamount| and |\theorempostskipamount| define,
% respectively, the spacing before and after such an environment.
% These parameters apply for all theorem sets and can be manipulated
% with the ordinary length macros.  They are rubber lengths
% (`\textsf{skips}'), and therefore can contain \texttt{plus} and
% \texttt{minus} parts.
%
% Since the definition of theorem sets should---most sensibly---be
% placed in the preamble, we only allow installation there.
% It is therefore possible to release the memory used here after
% |\begin{document}|, in order to make room for other
% applications.
%
%
% \subsection{Existing theorem styles}
%
% The following theorem styles exist to date:
% \begin{deflist}{marginbreak}
%    \item[plain]
%       This theorem style emulates the original \LaTeX{} definition,
%       except that additionally the parameters
%       |\theorem...skipamount| are used.
%    \item[break]
%       In this style, the theorem header is followed by a line
%       break.
%    \item[marginbreak]
%       The theorem number is set in the margin, and there is a line
%       break as in \texttt{break}.
%    \item[changebreak]
%       Like \texttt{break}, but with header number and text
%       interchanged.
%    \item[change]
%       Header number and text are interchanged, without a line break.
%    \item[margin]
%       The number is set in the left margin, without a line break.
% \end{deflist}
% All styles (except \texttt{plain}) select |\slshape| as the default
% |\theorembodyfont|.
%
%
% \subsection{Examples}
%
% Given the above theorem sets {\envfont Cor}, {\envfont Exa}
% and {\envfont Rem},
% suppose that the preamble also contains the declarations:
% \begin{verbatim}
%    \theoremstyle{marginbreak}   \newtheorem{Lem}[Cor]{Lemma}
%    \theoremstyle{change}
%    \theorembodyfont{\itshape}        \newtheorem{Def}[Cor]{Definition}
%
%    \theoremheaderfont{\scshape}
%\end{verbatim}
% Then the following are some typical examples of the typeset output
% resulting from their use.
%
% \begin{Cor}
%    This is a sentence typeset in the theorem environment {\envfont
%    Cor}.
% \end{Cor}
%
% \begin{Exa}
%    This is a sentence typeset in the theorem environment {\envfont
%    Exa}.
% \end{Exa}
%
% \begin{Rem}
%    This is a sentence typeset in the theorem environment {\envfont
%    Rem}.
% \end{Rem}
%
% \begin{Lem}[Ben User]
%    This is a sentence typeset in the theorem environment {\envfont
%    Lem}.
% \end{Lem}
%
% \begin{Def}[Very impressive Definition]
%    This is a sentence typeset in the theorem environment {\envfont
%    Def}.
% \end{Def}
%
% The last two examples show the effect of the optional argument to
% a theorem environment (it is the text typeset in parentheses).
%
%
% \section{Special Considerations}
% \label{sec:newfonts}
%
% Theoremheader and body are implemented as a unit. This means that
% the |\theoremheaderfont|  will inherit characteristics of the
% |\theorembodyfont| in \LaTeXe. Thus, if for example
% |\theorembodyfont| is |\itshape| and |\theoremheaderfont|
% is |\bfseries| the font selected for the header will have the
% characteristics `bold extended italic'. If this is not desired one
% should set the |\theoremheaderfont| to something like
% \begin{verbatim}
%      \theoremheaderfont{\normalfont\bfseries}
%\end{verbatim}
% i.e.\ supplying all necessary font information explicitly.
%
%
%
% \section{Acknowledgements}
%
% The publication of this set of macros was only possible with the
% help of Christina Busse (translating the manuscript into English),
% Joachim Pense (playing the r\^ole of typist), Chris Rowley
% (looking everything over) and many others providing useful
% suggestions.
%
% \MaybeStop{%
%    \begin{thebibliography}{9}
%       \bibitem{bk:GMS94} \textsc{M.~Goossens}, \textsc{F.~Mittelbach}
%         and \textsc{A.~Samarin}.
%         \newblock The \LaTeX{} Companion.
%         \newblock
%         Addison-Wesley, Reading, Massachusetts, 1994.
%       \bibitem{latex:source} \textsc{Lamport, Leslie.} \newblock
%                               \texttt{latex.tex}, version 2.09, date
%                               Feb. 1990.
%    \end{thebibliography}
%    \PrintIndex
%    \PrintChanges}
%
%
% \section{The documentation driver file}
%
% The next bit of code contains the documentation driver file for
% \TeX{}, i.e., the file that will produce the documentation you are
% currently reading. It will be extracted from this file by the
% \texttt{docstrip} program. Since it is the first code in the file
% one can alternatively process this file directly with \LaTeXe{} to
% obtain the documentation.
%
%    \begin{macrocode}
%<*driver>
\documentclass{ltxdoc}

\usepackage{theorem}

%<+driver>% The next few lines define theorem sets which are used
%<+driver>% in the example section of the documentation.

\theoremstyle{break}       \newtheorem{Cor}{Corollary}
\theoremstyle{plain}       \newtheorem{Exa}{Example}[section]
{\theorembodyfont{\upshape}\newtheorem{Rem}{Remark}}
\theoremstyle{marginbreak} \newtheorem{Lem}[Cor]{Lemma}
\theoremstyle{change}
\theorembodyfont{\itshape} \newtheorem{Def}[Cor]{Definition}

\theoremheaderfont{\scshape}

\RecordChanges

\begin{document}
   \DocInput{theorem.dtx}
\end{document}
%</driver>
%    \end{macrocode}
%
% \section{Definition of the Macros}
%
% If the file has been loaded before, we abort immediately. If not the
% package announces itself (this is actually done at the very top of
% the file---the way it is done isn't good style so don't copy it).
% \changes{v2.0e}{89/07/19}{Spaces removed from `typeout'.}
% \changes{v2.2b}{95/11/19}{Announce moved to top of file}
%    \begin{macrocode}
%<*package>
%\@ifundefined{theorem@style}{}{\endinput}
%\def\FMithmInfo{1995/11/19 v2.2b Theorem extension package (FMi)}
%\ProvidesPackage{theorem}[\FMithmInfo]
%    \end{macrocode}
%
%
% \subsection{Definition of theorem styles and fonts}
%
% All the definitions in this file are done globally to allow
% inputting this file inside a group.
%
% \begin{macro}{\theoremstyle}
%    Before a theorem style can be installed, the chosen style must be
%    known. For that reason, we must test to see that
%    |\th@|\meta{style} is known or, more precisely, that it is
%    different from |\relax|. If the style is not known then
%    |\th@plain| is used.
%    \begin{macrocode}
\gdef\theoremstyle#1{%
   \@ifundefined{th@#1}{\@warning
          {Unknown theoremstyle `#1'. Using `plain'}%
          \theorem@style{plain}}%
%    \end{macrocode}
%    We save the theorem style to be used in the token register
%    |\theorem@style|.
%    \begin{macrocode}
      {\theorem@style{#1}}%
%    \end{macrocode}
%    Now we ``evaluate'' the theorem style: this means, we call the
%    macro |\th@|\meta{style} which will activate the relevant
%    definitions which are contained in a separate file.  This is done
%    in a group to suppress changes to the current font.  This could
%    otherwise pose problems together with the new font selection
%    scheme\footnote{When I printed the original article using the new
%    font selection scheme I ended with a document with slanted
%    typefaces (text headings and all) simply because one of the
%    theorem styles used \texttt{\bslash sl} at toplevel.} if the
%    |\th@|\meta{style} is evaluated a second time.
% \changes{v2.1a}{90/02/15}{Group added around `th@..}
% \changes{v2.1a}{90/02/15}{Reseting of `..@begintheorem removed.}
%    \begin{macrocode}
      \begingroup
        \csname th@\the\theorem@style \endcsname
      \endgroup}
%    \end{macrocode}
% \end{macro}
%
%
% \begin{macro}{\@begintheorem}
% \changes{v2.1a}{90/02/15}{Relaxed at toplevel.}
% \begin{macro}{\@opargbegintheorem}
% \changes{v2.1a}{90/02/15}{Relaxed at toplevel.}
%    We reset |\@begintheorem| and |\@opargbegintheorem| to
%    |\relax| since these commands are no longer necessary at
%    toplevel. This will save a few tokens.
%    \begin{macrocode}
\global\let\@begintheorem\relax
\global\let\@opargbegintheorem\relax
%    \end{macrocode}
% \end{macro}
% \end{macro}
%
% \begin{macro}{\theorem@style}
%    Obviously the \textsf{token} register used above has to be
%    allocated. To assure the utmost compatibility with the
%    original \LaTeX{} definition, we set the default theorem style to
%    \texttt{plain}, which implements the usual \LaTeX{} convention.
% \changes{v2.0f}{89/07/20}{Unnecessary `global' removed.}
%    \begin{macrocode}
\newtoks\theorem@style
\global\theorem@style{plain}
%    \end{macrocode}
% \end{macro}
%
% \begin{macro}{\theorembodyfont}
% \begin{macro}{\theorem@bodyfont}
%    If the |\theorembodyfont| is set by the user then it should not
%    interact with the default font set in the theorem style. When the
%    new font selection is in force this may happen if, for example,
%    the default is |\itshape| and the new |\theorembodyfont| is
%    |\sffamily|. So we add a |\reset@font| command in front of the
%    user definition.
% \changes{v2.0f}{89/07/20}{Unnecessary `global' removed.}
% \changes{v2.0g}{89/09/19}{`theoremfont to `theorembodyfont.}
% \changes{v2.1c}{92/06/22}{`reset@font added and logic changed.}
% \changes{v2.1d}{92/10/28}{Reset to empty if arg is empty}
%    \begin{macrocode}
\gdef\theorembodyfont#1{%
%    \end{macrocode}
%    We check if the argument supplied is empty and if so put nothing
%    into the |\theorem@bodyfont| token register to allow for
%    |\theorembodyfont{}| as a mean of using the default of the
%    current |\theoremstyle|.
%    \begin{macrocode}
   \def\@tempa{#1}%
   \ifx\@tempa\@empty
    \theorem@bodyfont{}%
   \else
    \theorem@bodyfont{\reset@font#1}%
   \fi
   }
\newtoks\theorem@bodyfont
\global\theorem@bodyfont{}
%    \end{macrocode}
% \end{macro}
% \end{macro}
%
%
% \begin{macro}{\theoremheaderfont}
% \changes{v2.0g}{89/09/19}{Macro added.}
%    The font for the theorem headers is handled differently because
%    this definition applies to all theorem styles.
%    \begin{macrocode}
\gdef\theoremheaderfont#1{\gdef\theorem@headerfont{#1}%
%    \end{macrocode}
%    After using the macro once it is redefined to produce an error
%    message.
%    \begin{macrocode}
       \gdef\theoremheaderfont##1{%
        \typeout{\string\theoremheaderfont\space should be used
                 only once.}}}
%    \end{macrocode}
% \end{macro}
%
% \begin{macro}{\theorem@headerfont}
% \changes{v2.0g}{89/09/19}{Macro added.}
%    To set the |\theorem@headerfont| default we first test if
%    the new fontselection scheme is in force.
% \changes{v2.1a}{90/02/15}{Support for new font selection.}
%    \begin{macrocode}
\ifx\upshape\undefined
%    \end{macrocode}
%    If not we define it to expand into |\bfseries|. We don't use
%    |\let| just in case a following style option redefines this
%    macro.
%    \begin{macrocode}
\gdef\theorem@headerfont{\bfseries}
%    \end{macrocode}
%    Otherwise we reset the current shape before calling |\bfseries|.
%    \begin{macrocode}
\else \gdef\theorem@headerfont{\normalfont\bfseries}\fi
%    \end{macrocode}
% \end{macro}
%
% \begin{macro}{\th@plain}
% \begin{macro}{\th@break}
% \begin{macro}{\th@marginbreak}
% \begin{macro}{\th@changebreak}
% \begin{macro}{\th@change}
% \begin{macro}{\th@margin}
%    The different styles are defined in macros such as
%    |\th@plain|.  Since memory space is precious in
%    ``non-Big-versions'', we have to avoid offering too many unused
%    definitions.  Therefore we define these styles in separate files
%    that can be loaded on demand.  Thus the commands themselves only
%    load these files. We use |\@input@|, a \LaTeXe{} internal command
%    that ensures that the file will be listed with |\listfiles|.
% \changes{v2.2a}{94/02/02}{Use `@input@ to load theorem layout files}
%    \begin{macrocode}
\gdef\th@plain{\@input@{thp.sty}}
\gdef\th@break{\@input@{thb.sty}}
\gdef\th@marginbreak{\@input@{thmb.sty}}
\gdef\th@changebreak{\@input@{thcb.sty}}
\gdef\th@change{\@input@{thc.sty}}
\gdef\th@margin{\@input@{thm.sty}}
%    \end{macrocode}
%    This list will be expanded when new styles become
%    available.  For testing,  just append new theorem substyles
%    as document options.
% \end{macro}
% \end{macro}
% \end{macro}
% \end{macro}
% \end{macro}
% \end{macro}
%
%
%
% \subsection{Definition of a new theorem set}
%
% As already pointed out, a new theorem environment can be defined in
% three different ways:
% \begin{verbatim}
%    \newtheorem{Lem}{Lemma}
%    \newtheorem{Lem}{Lemma}[section]
%    \newtheorem{Lem}[Theorem]{Lemma}
%\end{verbatim}
% The function of the macro |\newtheorem| is to recognize these cases
% and then to branch into one of the three macros |\@ynthm|, |\@xnthm|
% or |\@othm|.  This mechanism is adopted unchanged from
% \cite{latex:source}; the essential point here is that, for example,
% in the second case, the arguments \texttt{Lem}, \texttt{Lemma} and
% \texttt{section} are passed over to the macro |\@xnthm|.
%
% We inspect this case first because the others present fewer problems,
% and thus are easily derived from this one.
%
% \begin{macro}{\@xnthm}
%    For our example arguments, the macro |\@xnthm| must fulfill
%    the following:
%    \begin{itemize}
%       \item
%          Define a new \LaTeX-counter `Lem'
%       \item
%          reset this counter within a |\section|
%       \item
%          define the macro |\theLem|
%       \item
%          define the environment macros |\Lem| and
%          |\endLem| using the current
%          |\theoremstyle| and |\theorem@bodyfont|.
%    \end{itemize}
%    Obviously, all this should happen only if the first argument
%    of |\@xnthm| (i.e.\ \texttt{Lem} in our example) is chosen so
%    as not to conflict with any previously defined commands or
%    environments.  This test is performed by the \LaTeX{} macro
%    |\@ifdefinable|.
%    \begin{macrocode}
\gdef\@xnthm#1#2[#3]{\expandafter\@ifdefinable\csname #1\endcsname
%    \end{macrocode}
%    Therefore, the first argument of |\@ifdefinable| is the
%    expansion (in the example, |\Lem|) of
%    |\csname||#1\endcsname|.  The second argument is
%    executed only if the test has been completed successfully.
%    \begin{macrocode}
   {%
%    \end{macrocode}
%    Now we define the new counter. The names of the \LaTeX{}
%    macros employed should speak for themselves:
% \changes{v2.2b}{1995/10/07}{Use \cs{@newctr}}
% \changes{v2.2c}{1995/11/23}{Use \cs{@newctr} properly.}
%    \begin{macrocode}
    \@definecounter{#1}\@newctr{#1}[#3]%
%    \end{macrocode}
%    Using |\@newctr| will give a proper error message if the counter
%    in |#3| is not defined.
%    In defining `|\theLem|' we must generate the desired macro
%    name by use of |\expandafter| and |\csname|.
%    \begin{macrocode}
    \expandafter\xdef\csname the#1\endcsname
%    \end{macrocode}
%    An |\xdef| is used in order to make the definition global,
%    and to ensure that it contains the replacement texts of
%    |\@thmcountersep| and |\@thmcounter|.\footnote{These
%    two macros can be defined by the document style.  Their default
%    values produce a `\texttt{.}' as separation and an arabic
%    representation of the number.} However, not everything should be
%    expanded. For example, it saves space to use |\thesection|
%    instead of its---at times---lengthy expansion.
%    \begin{macrocode}
      {\expandafter \noexpand \csname the#3\endcsname
       \@thmcountersep \@thmcounter{#1}}%
%    \end{macrocode}
%    Thus with the defaults of \LaTeX{}, |\theLem| would be
%    replaced by the command sequence |\thesection.\arabic{Lem}|.
%
%    We will now look at the definition of the macro which is executed
%    at the beginning of the actual environment (in our example this
%    macro is |\Lem|). It should be noted that we use an
%    ``|\expandafter| trick'' to expand only certain parts of the
%    replacement text at the time of the definition.
%    \begin{macrocode}
    \def\@tempa{\global\@namedef{#1}}%
    \expandafter \@tempa \expandafter{%
%    \end{macrocode}
%    First, the macro that contains the current definitions of
%    |\@begintheorem| and |\@opargtheorem| should be called
%    up.  The name of this macro---as is already known---has the form
%    |\th@|\meta{theorem\ style}; therefore, it must be called by
%    \begin{macrocode}
      \csname th@\the \theorem@style
%    \end{macrocode}
%    In addition the default theorem font should be changeable, i.e.\
%    we have to insert the contents of |\theorem@bodyfont|. For
%    that reason, we expand even further, beyond |\endcsname|,
%    and thus insert the contents of the \textsf{token} register
%    |\theorem@bodyfont| in the replacement text.
%    \begin{macrocode}
            \expandafter \endcsname \the \theorem@bodyfont
%    \end{macrocode}
%    Now it is time to call the macro |\@thm| which takes over
%    the further processing.  It has two arguments: the current
%    counter name (in our example, \texttt{Lem}), and the text of the
%    label (in our example, \texttt{Lemma}).
%    \begin{macrocode}
     \@thm{#1}{#2}}%
%    \end{macrocode}
%    With this, the `sub-definition' is complete.  The macro
%    |\@endtheorem| ends a theorem environment and is, so far, nothing
%    but an |\endtrivlist|.  (Hence it is defined globally, and not
%    within the theorem styles.\footnote{This has to be changed as
%    soon as theorem styles that change \texttt{\bslash @endtheorem}
%    exist.  In such a case, all existing styles must be changed as
%    well since they will have to reset the macro.}) Therefore, we can
%    set it equivalent to the macro that ends the theorem set (in our
%    example, |\endLem|).  However, if some day theorem styles exist
%    that do change |\@endtheorem|, we would have to use the
%    commented-out line instead.
%    \begin{macrocode}
    \global \expandafter \let \csname end#1\endcsname \@endtheorem
%   \global\@namedef{end#1}{\@endtheorem}%
%    \end{macrocode}
%    With these commands all the required definitions are employed,
%    unless the test |\@ifdefinable| has failed.  Therefore, we
%    end the second argument of this macro and with it the definition
%    of |\@xnthm|.
%    \begin{macrocode}
   }}
%    \end{macrocode}
% \end{macro}
%
% \begin{macro}{\@ynthm}
%    The definition of |\@ynthm| is completely analogous. In
%    this case the new counter that is defined is not reset within
%    another counter; thus the definition of |\the...|
%    is simplified:
%    \begin{macrocode}
\gdef\@ynthm#1#2{\expandafter\@ifdefinable\csname #1\endcsname
   {\@definecounter{#1}%
    \expandafter\xdef\csname the#1\endcsname{\@thmcounter{#1}}%
%    \end{macrocode}
%    The rest of the definition corresponds literally to that of
%    |\@xnthm|:
%    \begin{macrocode}
    \def\@tempa{\global\@namedef{#1}}\expandafter \@tempa
     \expandafter{\csname th@\the \theorem@style \expandafter
     \endcsname \the\theorem@bodyfont \@thm{#1}{#2}}%
    \global \expandafter \let \csname end#1\endcsname \@endtheorem}}
%    \end{macrocode}
% \end{macro}
%
% \begin{macro}{\@othm}
%    The definition of |\@othm| does not contain anything new.
%    \begin{macrocode}
\gdef\@othm#1[#2]#3{%
%    \end{macrocode}
% \changes{v2.2b}{95/10/07}{Build in check for counter defined}
%    We do not define a new counter but instead use one that has
%    already been defined.  Thus the only definition we need is that
%    of this pseudo-counter (i.e.\ |\the|\meta{env.\ name}).
%    First we check if |#2| corresponds to a known counter name.
%    \begin{macrocode}
  \expandafter\ifx\csname c@#2\endcsname\relax
   \@nocounterr{#2}%
  \else
   \expandafter\@ifdefinable\csname #1\endcsname
   {\expandafter \xdef \csname the#1\endcsname
     {\expandafter \noexpand \csname the#2\endcsname}%
%    \end{macrocode}
%    All other parts of the definition can be adopted from
%    |\@xnthm|. We have to remember, though, that in this case
%    the name of the current counter and the theorem label have moved
%    to the second and third arguments.
%    \begin{macrocode}
    \def\@tempa{\global\@namedef{#1}}\expandafter \@tempa
     \expandafter{\csname th@\the \theorem@style \expandafter
     \endcsname \the\theorem@bodyfont \@thm{#2}{#3}}%
    \global \expandafter \let \csname end#1\endcsname \@endtheorem}%
  \fi}
%    \end{macrocode}
% \end{macro}
%
% \subsection{Macros that are employed in a {\envfont theorem}
%    environment}
%
% \begin{macro}{\@thm}
%    The macro |\@thm| has to increase the current counter.  Then,
%    depending on whether the environment has (or does not have) an
%    optional argument, it has to branch into either |\@begintheorem|
%    or |\@opargtheorem|.
%    \begin{macrocode}
\gdef\@thm#1#2{\refstepcounter{#1}%
%    \end{macrocode}
%    Now we start a {\envfont trivlist} environment, and give
%    |\@topsep| and |\@topsepadd| the values of the \textsf{skip}
%    registers |\theorempreskipamount| and |\theorempostskipamount|.
%    The value in |\@topsep| is the vertical space that is inserted by
%    the first (and only) |\item| in our |\trivlist| whilst
%    |\@topsepadd| is inserted by |\@endparenv| at the end of that
%    {\envfont trivlist} environment.  By using these registers, we
%    obtain the desired space around a {\envfont theorem} environment.
%    \begin{macrocode}
   \trivlist
   \@topsep \theorempreskipamount               % used by first \item
   \@topsepadd \theorempostskipamount           % used by \@endparenv
%    \end{macrocode}
%    Now we have to test whether an optional argument has been given.
% \changes{v2.0d}{89/07/18}{Forgotten percent added.}
%    \begin{macrocode}
   \@ifnextchar [%
%    \end{macrocode}
%    If there is an optional argument, we will call |\@ythm|, and
%    move the arguments read back into the input stream.
%    \begin{macrocode}
   {\@ythm{#1}{#2}}%
%    \end{macrocode}
%    If not, we call |\@begintheorem|. Its first argument is the
%    name of the theorem set (hence the second argument of
%    |\@thm|). Its second argument is the macro that produces the
%    current number.
%    \begin{macrocode}
   {\@begintheorem{#2}{\csname the#1\endcsname}\ignorespaces}}
%    \end{macrocode}
% \end{macro}
%
% \begin{macro}{\@xthm}
% \begin{macro}{\@ythm}
%    Both these macros were originally called by |\@thm|.  We do
%    not need |\@xthm| anymore, hence we reset it to
%    |\relax|. The definition of |\@ythm| has not changed at
%    all from its definition in \LaTeX. In order to make the macros
%    easier to understand, we will
%    nevertheless present it (commented out).
%    \begin{macrocode}
\global\let\@xthm\relax
% \def\@ythm#1#2[#3]{\@opargbegintheorem{#2}{\csname
%           the#1\endcsname}{#3}\ignorespaces}
%</package>
%    \end{macrocode}
%    The primitive |\ignorespaces| in |\@ythm| and
%    |\@thm| is needed to remove the spaces between the
%    |\begin{...}| and the actual text.
% \end{macro}
% \end{macro}
%
%
% \subsection{Definition of the theorem substyles}
%
% As already pointed out, the theorem substyles, defined below, are
% only loaded when necessary.  Note that all these substyles, except
% \texttt{plain}, have |\slshape| as the default body font.
%
% \subsubsection{The \texttt{plain} style}
%
% \label{dok:th@plain} As the following macros use |@|, we have
% to locally set the |\catcode| of this symbol to ``letter''.
% This happens within a group, so that we do not have to worry about
% which |\catcode| that symbol had before.
%    \begin{macrocode}
%<*thp>
\begingroup \makeatletter
%    \end{macrocode}
% Since we are now within a group, we must make all definitions
% globally.  First we make sure that \texttt{theorem.sty} is loaded.
% This will allow us to use this file as a document style option
% without having to call \texttt{theorem} itself as an option.  At the
% same time, we assure that at least version 2 is loaded, since
% |\theorem@style| was not defined in earlier versions.
%    \begin{macrocode}
\@ifundefined{theorem@style}{\input{theorem.sty}}{}
\ProvidesFile{thp.sty}
             [\FMithmInfo]
%    \end{macrocode}
%
%
% \begin{macro}{\th@plain}
% \begin{macro}{\@begintheorem}
% \begin{macro}{\@opargbegintheorem}
%    |\theoremstyle{plain}| corresponds to the original
%    definition, except that the distances to the surrounding text are
%    determined by the parameters |\theorempreskipamount| and
%    |\theorempostskipamount|.
%    First we set the default body font.
%    \begin{macrocode}
\gdef\th@plain{\normalfont\itshape
%    \end{macrocode}
%    Then we define |\@begintheorem| and
%    |\@opargbegintheorem|. These two macros define how the
%    header of a theorem is typeset.  |\@opargbegintheorem| will
%    be called if a \textsf{theorem} environment with an optional
%    argument is encountered; otherwise, the header is constructed by
%    calling |\@begintheorem|.  If one of these macros is
%    executed, we are within a \textsf{trivlist} environment started by
%    |\@thm|.  So the theorem header is produced with an
%    |\item| command.
%
%    Instead of specifying the header font directly, all standard
%    theorem styles use the |\theorem@headerfont| macro to allow
%    customization.  The extra space (|\labelsep|) is necessary
%    because of problems in the \textsf{trivlist} environment.
%    \begin{macrocode}
  \def\@begintheorem##1##2{%
        \item[\hskip\labelsep \theorem@headerfont ##1\ ##2]}%
%    \end{macrocode}
%    The definition of |\@opargbegintheorem| is completely
%    analogous. The only difference is the fact that there exists a
%    third argument (which is the optional parameter of the
%    environment and contains additional information about the
%    theorem).  Customarily we enclose it in parentheses.
%    \begin{macrocode}
\def\@opargbegintheorem##1##2##3{%
   \item[\hskip\labelsep \theorem@headerfont ##1\ ##2\ (##3)]}}
%    \end{macrocode}
% \end{macro}
% \end{macro}
% \end{macro}
%
% \noindent
% We conclude with an |\endgroup| to restore the
% |\catcode| of |@|.
%    \begin{macrocode}
\endgroup
%</thp>
%    \end{macrocode}
%
%
% \subsubsection{The \texttt{break} style}
%
% This style option is stored in the file |thb.sty|.
% For the next two lines see the documentation for |\th@plain| on
% page~\pageref{dok:th@plain}.
%    \begin{macrocode}
%<*thb>
\begingroup \makeatletter
\@ifundefined{theorem@style}{\input{theorem.sty}}{}
\ProvidesFile{thb.sty}
             [\FMithmInfo]
%    \end{macrocode}
%
% \begin{macro}{\th@break}
%    |\theoremstyle{break}| produces a line break after the name
%    of the theorem.  The font is |\slshape|.  Hence, we define
%    |\th@break| as follows:
%    \begin{macrocode}
\gdef\th@break{\normalfont\slshape
  \def\@begintheorem##1##2{\item[%
%    \end{macrocode}
%    We run into the following problem: it is not possible to create
%    the header with |\item[|\meta{title}|]| and then start
%    a new line by, for example, |\mbox{}\\|. Such a definition
%    will fail whenever a list environment follows immediately.  With
%    the above construction, the |\mbox{}| causes the switch
%    |@inlabel| (cf.\ definition of |\list| and
%    |\trivlist| in \cite{latex:source}) to be set to \texttt{false}
%    and so the following list will insert additional vertical
%    space (|\topskip|). This is quite annoying. Therefore, we
%    create the line break within the |\item|. In order to ensure
%    that the text will begin at the proper position in the following
%    line, we simply pretend that the label does not take any
%    room.\footnote{This will lead to problems whenever very high
%    symbols occurring in the line tower into the
%    heading. So, something else has to be done here sometime.}
%    \begin{macrocode}
       \rlap{\vbox{\hbox{\hskip \labelsep\theorem@headerfont ##1\ ##2}%
                   \hbox{\strut}}}]}%
%    \end{macrocode}
%    Again, the definition of |\@opargbegintheorem| is
%    completely analogous.
%    \begin{macrocode}
\def\@opargbegintheorem##1##2##3{%
  \item[\rlap{\vbox{\hbox{\hskip \labelsep \theorem@headerfont
                     ##1\ ##2\ (##3)}%
                    \hbox{\strut}}}]}}
\endgroup
%</thb>
%    \end{macrocode}
% \end{macro}
%
%
% \subsubsection{The \texttt{changebreak} style}
%
%    \begin{macrocode}
% This style option is stored in the file |thcb.sty|.
%    \begin{macrocode}
%<*thcb>
\begingroup \makeatletter
\@ifundefined{theorem@style}{\input{theorem.sty}}{}
\ProvidesFile{thcb.sty}
             [\FMithmInfo]
%    \end{macrocode}
%
% \begin{macro}{\th@changebreak}
%    The \texttt{change-break} theorem style is like \texttt{break}
%    but with interchange of theorem name and theorem number.  Thus we
%    define |\th@changebreak| as follows:
%    \begin{macrocode}
\gdef\th@changebreak{\normalfont\slshape
  \def\@begintheorem##1##2{\item
       [\rlap{\vbox{\hbox{\hskip\labelsep \theorem@headerfont ##2\ ##1}%
                    \hbox{\strut}}}]}%
\def\@opargbegintheorem##1##2##3{%
  \item[\rlap{\vbox{\hbox{\hskip\labelsep \theorem@headerfont
                     ##2\ ##1\ (##3)}%
                    \hbox{\strut}}}]}}
\endgroup
%</thcb>
%    \end{macrocode}
% \end{macro}
%
%
% \subsubsection{The \texttt{change} style}
%
% This style option is stored in the file |thc.sty|.
%    \begin{macrocode}
%<*thc>
\begingroup \makeatletter
\@ifundefined{theorem@style}{\input{theorem.sty}}{}
\ProvidesFile{thc.sty}
             [\FMithmInfo]
%    \end{macrocode}
%
% \begin{macro}{\th@change}
%    The \texttt{change} theorem style corresponds to the
%    \texttt{change break} style without a linebreak after the header.
%    To say it in another way, it's the same as the \texttt{plain}
%    style but with number and name interchanged and |\slshape| as the
%    default font.
%    \begin{macrocode}
\gdef\th@change{\normalfont\slshape
  \def\@begintheorem##1##2{\item
       [\hskip\labelsep \theorem@headerfont ##2\ ##1]}%
\def\@opargbegintheorem##1##2##3{%
  \item[\hskip\labelsep \theorem@headerfont ##2\ ##1\ (##3)]}}
\endgroup
%</thc>
%    \end{macrocode}
% \end{macro}
%
%
% \subsubsection{The \texttt{marginbreak} style}
%
% This style option is the one used most often at Mainz.  It is saved
% in the file |thmb.sty|.
%    \begin{macrocode}
%<*thmb>
\begingroup \makeatletter
\@ifundefined{theorem@style}{\input{theorem.sty}}{}
\ProvidesFile{thmb.sty}
             [\FMithmInfo]
%    \end{macrocode}
%
% \begin{macro}{\th@marginbreak}
%    The \texttt{margin break} style is nearly the same as the
%    \texttt{change break} style. The only difference is the placement
%    of the theorem number.  We use |\llap| to place it in the left
%    margin.
%
%    In this style |\labelsep| denotes the separation between the
%    number and the text.
%    \begin{macrocode}
\gdef\th@marginbreak{\normalfont\slshape
  \def\@begintheorem##1##2{\item
       [\rlap{\vbox{\theorem@headerfont
                    \hbox{\llap{##2}\hskip\labelsep ##1}%
                    \hbox{\strut}}}]}%
\def\@opargbegintheorem##1##2##3{%
  \item[\rlap{\vbox{\theorem@headerfont
                    \hbox{\llap{##2}\hskip\labelsep ##1\ (##3)}%
                    \hbox{\strut}}}]}}
\endgroup
%</thmb>
%    \end{macrocode}
% \changes{v2.1c}{90/02/17}{`endgroup outside macrocode env.}
% \end{macro}
%
%
%
% \subsubsection{The \texttt{margin} style}
%
% This style option is stored in the file |thm.sty|.
%    \begin{macrocode}
%<*thm>
\begingroup \makeatletter
\@ifundefined{theorem@style}{\input{theorem.sty}}{}
\ProvidesFile{thm.sty}
             [\FMithmInfo]
%    \end{macrocode}
%
% \begin{macro}{\th@margin}
%    Again this is only a variant of the theorem styles described
%    above without any new ideas.
%    \begin{macrocode}
\gdef\th@margin{\normalfont\slshape
  \def\@begintheorem##1##2{\item
       [\theorem@headerfont \llap{##2}\hskip\labelsep ##1]}%
\def\@opargbegintheorem##1##2##3{%
  \item[\theorem@headerfont \llap{##2}\hskip\labelsep ##1\ (##3)]}}
\endgroup
%</thm>
%    \end{macrocode}
% \end{macro}
%
%
%
%
% \subsection{Final Definitions}
%
% \begin{macro}{\theorempreskipamount}
% \begin{macro}{\theorempostskipamount}
%    The \textsf{skip} parameters that regulate the vertical empty space
%    before and after the {\envfont theorem} environment have
%    to be allocated as well.
% \changes{v2.0f}{89/07/20}{Unnecessary `global' removed.}
%    \begin{macrocode}
%<*package>
\newskip\theorempreskipamount
\newskip\theorempostskipamount
%    \end{macrocode}
%    Since we have used the same values for all theorem sets, we now can
%    assign them.
%    \begin{macrocode}
\global\setlength\theorempreskipamount{12pt plus 5pt minus 3pt}
\global\setlength\theorempostskipamount{8pt plus 3pt minus 1.5pt}
%    \end{macrocode}
% \end{macro}
% \end{macro}
%
%
% \begin{macro}{\@endtheorem}
%    The same holds for the macro |\@endtheorem|, which ends a
%    {\envfont theorem} environment. Since it is the same for all
%    theorem sets, it is removed from the macros |\th@|\meta{style}.
%    It simply ends the {\envfont trivlist} environment, which was
%    begun in |\@thm|.
%    \begin{macrocode}
\global\let\@endtheorem=\endtrivlist
%    \end{macrocode}
% \end{macro}
%
%
%
% \begin{macro}{\@preamblecmds}
%    All macros defined above are to be used only in the preamble.
%    Therefore, we insert them in |\@preamblecmds| which will disable
%    them at begin document. This is done by the internal \LaTeXe{}
%    command |\@onlypreamble|.
% \changes{v2.0g}{89/09/19}{`theoremheaderfont and `theorembodyfont
%    added.}
% \changes{v2.2a}{94/02/03}{Use \LaTeXe{} interface}
%    \begin{macrocode}
\@onlypreamble\@xnthm
\@onlypreamble\@ynthm
\@onlypreamble\@othm
\@onlypreamble\newtheorem
\@onlypreamble\theoremstyle
\@onlypreamble\theorembodyfont
\@onlypreamble\theoremheaderfont
%    \end{macrocode}
% \end{macro}
%
% Finally we declare the \texttt{plain} theorem style to be the default.
% \changes{v2.1b}{90/06/04}{Fixed missing `item by declaring default
%    style.}
%    \begin{macrocode}
\theoremstyle{plain}
%</package>
%    \end{macrocode}
%
% \Finale

\endinput
}{}
\ProvidesFile{thm.sty}
             [\FMithmInfo]
%    \end{macrocode}
%
% \begin{macro}{\th@margin}
%    Again this is only a variant of the theorem styles described
%    above without any new ideas.
%    \begin{macrocode}
\gdef\th@margin{\normalfont\slshape
  \def\@begintheorem##1##2{\item
       [\theorem@headerfont \llap{##2}\hskip\labelsep ##1]}%
\def\@opargbegintheorem##1##2##3{%
  \item[\theorem@headerfont \llap{##2}\hskip\labelsep ##1\ (##3)]}}
\endgroup
%</thm>
%    \end{macrocode}
% \end{macro}
%
%
%
%
% \subsection{Final Definitions}
%
% \begin{macro}{\theorempreskipamount}
% \begin{macro}{\theorempostskipamount}
%    The \textsf{skip} parameters that regulate the vertical empty space
%    before and after the {\envfont theorem} environment have
%    to be allocated as well.
% \changes{v2.0f}{89/07/20}{Unnecessary `global' removed.}
%    \begin{macrocode}
%<*package>
\newskip\theorempreskipamount
\newskip\theorempostskipamount
%    \end{macrocode}
%    Since we have used the same values for all theorem sets, we now can
%    assign them.
%    \begin{macrocode}
\global\setlength\theorempreskipamount{12pt plus 5pt minus 3pt}
\global\setlength\theorempostskipamount{8pt plus 3pt minus 1.5pt}
%    \end{macrocode}
% \end{macro}
% \end{macro}
%
%
% \begin{macro}{\@endtheorem}
%    The same holds for the macro |\@endtheorem|, which ends a
%    {\envfont theorem} environment. Since it is the same for all
%    theorem sets, it is removed from the macros |\th@|\meta{style}.
%    It simply ends the {\envfont trivlist} environment, which was
%    begun in |\@thm|.
%    \begin{macrocode}
\global\let\@endtheorem=\endtrivlist
%    \end{macrocode}
% \end{macro}
%
%
%
% \begin{macro}{\@preamblecmds}
%    All macros defined above are to be used only in the preamble.
%    Therefore, we insert them in |\@preamblecmds| which will disable
%    them at begin document. This is done by the internal \LaTeXe{}
%    command |\@onlypreamble|.
% \changes{v2.0g}{89/09/19}{`theoremheaderfont and `theorembodyfont
%    added.}
% \changes{v2.2a}{94/02/03}{Use \LaTeXe{} interface}
%    \begin{macrocode}
\@onlypreamble\@xnthm
\@onlypreamble\@ynthm
\@onlypreamble\@othm
\@onlypreamble\newtheorem
\@onlypreamble\theoremstyle
\@onlypreamble\theorembodyfont
\@onlypreamble\theoremheaderfont
%    \end{macrocode}
% \end{macro}
%
% Finally we declare the \texttt{plain} theorem style to be the default.
% \changes{v2.1b}{90/06/04}{Fixed missing `item by declaring default
%    style.}
%    \begin{macrocode}
\theoremstyle{plain}
%</package>
%    \end{macrocode}
%
% \Finale

\endinput
}{}
\ProvidesFile{thm.sty}
             [\FMithmInfo]
%    \end{macrocode}
%
% \begin{macro}{\th@margin}
%    Again this is only a variant of the theorem styles described
%    above without any new ideas.
%    \begin{macrocode}
\gdef\th@margin{\normalfont\slshape
  \def\@begintheorem##1##2{\item
       [\theorem@headerfont \llap{##2}\hskip\labelsep ##1]}%
\def\@opargbegintheorem##1##2##3{%
  \item[\theorem@headerfont \llap{##2}\hskip\labelsep ##1\ (##3)]}}
\endgroup
%</thm>
%    \end{macrocode}
% \end{macro}
%
%
%
%
% \subsection{Final Definitions}
%
% \begin{macro}{\theorempreskipamount}
% \begin{macro}{\theorempostskipamount}
%    The \textsf{skip} parameters that regulate the vertical empty space
%    before and after the {\envfont theorem} environment have
%    to be allocated as well.
% \changes{v2.0f}{89/07/20}{Unnecessary `global' removed.}
%    \begin{macrocode}
%<*package>
\newskip\theorempreskipamount
\newskip\theorempostskipamount
%    \end{macrocode}
%    Since we have used the same values for all theorem sets, we now can
%    assign them.
%    \begin{macrocode}
\global\setlength\theorempreskipamount{12pt plus 5pt minus 3pt}
\global\setlength\theorempostskipamount{8pt plus 3pt minus 1.5pt}
%    \end{macrocode}
% \end{macro}
% \end{macro}
%
%
% \begin{macro}{\@endtheorem}
%    The same holds for the macro |\@endtheorem|, which ends a
%    {\envfont theorem} environment. Since it is the same for all
%    theorem sets, it is removed from the macros |\th@|\meta{style}.
%    It simply ends the {\envfont trivlist} environment, which was
%    begun in |\@thm|.
%    \begin{macrocode}
\global\let\@endtheorem=\endtrivlist
%    \end{macrocode}
% \end{macro}
%
%
%
% \begin{macro}{\@preamblecmds}
%    All macros defined above are to be used only in the preamble.
%    Therefore, we insert them in |\@preamblecmds| which will disable
%    them at begin document. This is done by the internal \LaTeXe{}
%    command |\@onlypreamble|.
% \changes{v2.0g}{89/09/19}{`theoremheaderfont and `theorembodyfont
%    added.}
% \changes{v2.2a}{94/02/03}{Use \LaTeXe{} interface}
%    \begin{macrocode}
\@onlypreamble\@xnthm
\@onlypreamble\@ynthm
\@onlypreamble\@othm
\@onlypreamble\newtheorem
\@onlypreamble\theoremstyle
\@onlypreamble\theorembodyfont
\@onlypreamble\theoremheaderfont
%    \end{macrocode}
% \end{macro}
%
% Finally we declare the \texttt{plain} theorem style to be the default.
% \changes{v2.1b}{90/06/04}{Fixed missing `item by declaring default
%    style.}
%    \begin{macrocode}
\theoremstyle{plain}
%</package>
%    \end{macrocode}
%
% \Finale

\endinput
}{}
\ProvidesFile{thc.sty}
             [\FMithmInfo]
%    \end{macrocode}
%
% \begin{macro}{\th@change}
%    The \texttt{change} theorem style corresponds to the
%    \texttt{change break} style without a linebreak after the header.
%    To say it in another way, it's the same as the \texttt{plain}
%    style but with number and name interchanged and |\slshape| as the
%    default font.
%    \begin{macrocode}
\gdef\th@change{\normalfont\slshape
  \def\@begintheorem##1##2{\item
       [\hskip\labelsep \theorem@headerfont ##2\ ##1]}%
\def\@opargbegintheorem##1##2##3{%
  \item[\hskip\labelsep \theorem@headerfont ##2\ ##1\ (##3)]}}
\endgroup
%</thc>
%    \end{macrocode}
% \end{macro}
%
%
% \subsubsection{The \texttt{marginbreak} style}
%
% This style option is the one used most often at Mainz.  It is saved
% in the file |thmb.sty|.
%    \begin{macrocode}
%<*thmb>
\begingroup \makeatletter
\@ifundefined{theorem@style}{% \iffalse meta-comment
%
% Copyright (C) 1993-2024
%
% The LaTeX Project and any individual authors listed elsewhere
% in this file.
%
% This file is part of the Standard LaTeX `Tools Bundle'.
% -------------------------------------------------------
%
% It may be distributed and/or modified under the
% conditions of the LaTeX Project Public License, either version 1.3c
% of this license or (at your option) any later version.
% The latest version of this license is in
%    https://www.latex-project.org/lppl.txt
% and version 1.3c or later is part of all distributions of LaTeX
% version 2005/12/01 or later.
%
% The list of all files belonging to the LaTeX `Tools Bundle' is
% given in the file `manifest.txt'.
%
% \fi
%
%%
%
%
\def\FMithmInfo{2023/07/05 v2.2c Theorem extension package (FMi)}
%
% \ProvidesFile{theorem.dtx}[\FMithmInfo]
%\iffalse   % this is a METACOMMENT !
%%
%% Package `theorem' to use with LaTeX2e
%% Copyright (C) 1989-1995 Frank Mittelbach, all rights reserved.
%<+package>\@ifundefined{theorem@style}{}{\endinput}
%<+package>\NeedsTeXFormat{LaTeX2e}[1995/06/01]
%<+package>\ProvidesPackage{theorem}
%<+package>         [\FMithmInfo]
%
% \fi
%
% \changes{v1.0}{87/05/07}{First draft.}
% \changes{v1.0a}{87/05/07}{`thistheoremstyle' added.}
% \changes{v1.1a}{87/07/29}{New concept: redefine the internal
%                          \protect\LaTeX{} commands `@xnthm' etc.}
% \changes{v1.1b}{87/07/31}{`theoremfont' used to specify font.}
% \changes{v1.2a}{88/01/08}{`@endtheorem' now globally defined.}
% \changes{v1.2b}{88/01/11}{`newtheorem' only allowed in the preamble.}
% \changes{v1.3a}{88/02/14}{`theoremfont' now a toks register.}
%
% \changes{v2.0a}{88/11/16}{New implementation and new documentation.}
% \changes{v2.0a}{88/11/16}{No layout changes but internal modifications
%                           old substyles won't work.}
% \changes{v2.0b}{89/01/30}{`global' used for all definitions.}
% \changes{v2.0c}{89/06/15}{Documentation translated to English.
%                            Thanks to LLL and Nina}
% \changes{v2.1a}{90/02/15}{Support for new font selection.}
% \changes{v2.2a}{94/02/03}{Cleanup for LaTeX2e}
%
% \DoNotIndex{\@endparenv,\@ifdefinable,\@ifnextchar,\@ifundefined}
% \DoNotIndex{\@namedef,\@nameuse,\@spaces,\@tempa,\@warning}
% \DoNotIndex{\ }
% \DoNotIndex{\begingroup,\bfseries}
% \DoNotIndex{\csname}
% \DoNotIndex{\def,\do,\docdate}
% \DoNotIndex{\endcsname,\endgroup,\endinput,\endtrivlist,\expandafter}
% \DoNotIndex{\filedate,\fileversion}
% \DoNotIndex{\gdef,\global}
% \DoNotIndex{\hbox,\hskip}
% \DoNotIndex{\ignorespaces,\input,\itshape,\item}
% \DoNotIndex{\labelsep,\let,\llap}
% \DoNotIndex{\makeatletter}
% \DoNotIndex{\noexpand}
% \DoNotIndex{\refstepcounter,\relax,\rlap}
% \DoNotIndex{\setlength,\slshape,\space,\string,\strut}
% \DoNotIndex{\the,\trivlist,\typeout}
% \DoNotIndex{\vbox}
% \DoNotIndex{\xdef}
%
% \def\deflabel#1{\ttfamily #1\hfill}
% \def\deflist#1{\begin{list}{}{\settowidth\labelwidth{\ttfamily #1}%
%                               \setlength\leftmargin\labelwidth
%                               \addtolength\leftmargin\labelsep
%                               \let\makelabel\deflabel}}
% \def\enddeflist{\end{list}}
% \def\envfont{\normalfont\ttfamily}
%
% \MakeShortVerb{\|}
%
% \newcommand{\TUB}{{\sl TUGboat\/}}
%
% \GetFileInfo{theorem.dtx}
% \title{An Extension of the
%        \LaTeX{} theorem environment\thanks{This file
%        has version number \fileversion, last
%        revised \filedate.}}
% \author{Frank Mittelbach}
%
% \MaintainedByLaTeXTeam{tools}
% \maketitle
%
% \begin{abstract}
%    The macros described in this paper yield an extension of the
%    \LaTeX{} theorem mechanism. It is designed to satisfy the
%    different requirements of various journals. Thus, the
%    layout of the ``theorems'' can be manipulated by determining a
%    ``style''. This article describes not only the use, but
%    also the definition, of the necessary macros.
% \end{abstract}
%
% \section*{Preface to version 2.2}
%
% For \LaTeXe{} this package did not need any fundamental changes. I
% only modified the messages generated so that theorem layout styles
% will show up with the |\listfiles| command and cleaned the section
% on the New Font Selection Scheme since this is now included in
% \LaTeX.
%
% \section*{Preface to version 2.1}
%
% This version is identical to 2.0g described in \TUB\ 10\#3 except
% for some internal defaults which are now set depending on the used
% font selection scheme.
%
% This was done to avoid unpleasant surprises if the new font selection
% scheme is in force. For further details see
% section~\ref{sec:newfonts} and \cite{bk:GMS94}.
%
% \section{Introduction}
%
% For our purposes here, ``theorems'' are labelled enunciations,
% often set off from the main text by extra space and a font change.
% Theorems, corollaries, conjectures, definitions, and remarks
% are all instances of ``theorems''.  The ``header'' of these structures
% is composed of a label (such as \textsc{Theorem} or \textsc{Remark})
% and a number which serializes an item in the sequence of items
% with the same label.
%
% Shortly after the introduction of \LaTeX{} at the Fachbereich
% Mathematik in Mainz, the desire to manipulate the layout of
% ``theorems'' arose.  In Mainz, the following two conventions came
% into general use:
% \begin{enumerate}
%   \item The number of the theorem is shown in the margin.
%   \item There is a line break at
%         the end of the theorem header.
% \end{enumerate}
%
% Additionally, some journals require different formats which depend on
% the ``sort of theorem'': e.g.\ often remarks and definitions are set
% in |\upshape|, while |\itshape| is employed for main theorems.
%
% Confronted with these requirements, a theorem environment was
% developed in Mainz which allows separate determination of the
% layout of the ``theorems sets'', comparable to |\pagestyle|.
%
%
% \section{The user interface}
% \subsection{Defining new theorem sets}
%
% \DescribeMacro\newtheorem
% As in the original \LaTeX{} version, the command |\newtheorem|
% defines a new ``theorem set'' or ``theorem-like structure''.
% Two required arguments name the new environment and give the text
% to be typeset with each instance of the new ``set'', while
% an optional argument determines how
% the ``set'' is enumerated:
% \begin{description}
%    \item[\ttfamily \bslash newtheorem\{foo\}\{bar\}]
%       The theorem set {\envfont foo} (whose name is \texttt{bar})
%       uses its own counter.
%    \item[\ttfamily \bslash newtheorem\{foo2\}{[foo]}\{bar2\}]
%       The theorem set {\envfont foo2} (printed name \texttt{bar2})
%       uses the same counter as the theorem set \texttt{foo}.
%    \item[\ttfamily \bslash newtheorem\{foo3\}\{bar3\}{[section]}]
%       The theorem set {\envfont foo3} (printed name \texttt{bar3}) is
%       enumerated within the counter \texttt{section}, i.e.\ with every
%       new |\section| the enumeration begins again with 1, and
%       the enumeration is composed from the section-number and the
%       theorem counter itself.
% \end{description}
%
% \DescribeMacro\theoremstyle
% Additionally, the command |\theoremstyle| can define the layout
% of various, or all, theorem sets.  It should be noted that any
% theorem set defined by |\newtheorem| is typeset in the
% |\theoremstyle| that is current at the time of the definition.
% Thus, the following
% \begin{verbatim}
%    \theoremstyle{break}        \newtheorem{Cor}{Corollary}
%    \theoremstyle{plain}        \newtheorem{Exa}{Example}[section]
%\end{verbatim}                             ^^A No blank at line start
% leads to the result that the set {\envfont Cor} is formatted in the
% style \texttt{break}, while the set {\envfont Exa} and all the
% following ones are formatted in the style \texttt{plain}, unless
% another |\theoremstyle| follows.  Since the definitions installed by
% |\newtheorem| are global, one also can limit |\theoremstyle| locally
% by grouping braces.
%
% \DescribeMacro\theorembodyfont
% The choice of the font for the theorem body is completely
% independent of the chosen |\theoremstyle|; this has proven to
% be very advantageous.  For example,
% \begin{verbatim}
%   {\theorembodyfont{\upshape}          \newtheorem{Rem}{Remark}}
%\end{verbatim}
% defines a theorem set {\envfont Rem}, which will be set in
% |\upshape| in the current layout (which in our example is
% \texttt{plain}).  As with |\theoremstyle|, the |\theorembodyfont|
% chosen is that current at the time of |\newtheorem|.  If
% |\theorembodyfont| is not specified or one defines
% |\theorembodyfont{}|, then the font used will be that defined by the
% |\theoremstyle|.
%
% \DescribeMacro\theoremheaderfont It is also possible to customize
% the font used for the theorem headers.  This is, however, a global
% declaration, and therefore there should be at most one
% |\theoremheaderfont| declaration in the preamble.\footnote{If it is
% actually necessary to have different header fonts, one has to define
% new theorem styles (substituting the desired font) or specify the
% information directly in the \texttt{\bslash newtheorem} declaration
% (the unclean variant).}
%
% \DescribeMacro\theorempreskipamount
% \DescribeMacro\theorempostskipamount Two additional parameters
% affect the vertical space around the theorem environments:
% |\theorempreskipamount| and |\theorempostskipamount| define,
% respectively, the spacing before and after such an environment.
% These parameters apply for all theorem sets and can be manipulated
% with the ordinary length macros.  They are rubber lengths
% (`\textsf{skips}'), and therefore can contain \texttt{plus} and
% \texttt{minus} parts.
%
% Since the definition of theorem sets should---most sensibly---be
% placed in the preamble, we only allow installation there.
% It is therefore possible to release the memory used here after
% |\begin{document}|, in order to make room for other
% applications.
%
%
% \subsection{Existing theorem styles}
%
% The following theorem styles exist to date:
% \begin{deflist}{marginbreak}
%    \item[plain]
%       This theorem style emulates the original \LaTeX{} definition,
%       except that additionally the parameters
%       |\theorem...skipamount| are used.
%    \item[break]
%       In this style, the theorem header is followed by a line
%       break.
%    \item[marginbreak]
%       The theorem number is set in the margin, and there is a line
%       break as in \texttt{break}.
%    \item[changebreak]
%       Like \texttt{break}, but with header number and text
%       interchanged.
%    \item[change]
%       Header number and text are interchanged, without a line break.
%    \item[margin]
%       The number is set in the left margin, without a line break.
% \end{deflist}
% All styles (except \texttt{plain}) select |\slshape| as the default
% |\theorembodyfont|.
%
%
% \subsection{Examples}
%
% Given the above theorem sets {\envfont Cor}, {\envfont Exa}
% and {\envfont Rem},
% suppose that the preamble also contains the declarations:
% \begin{verbatim}
%    \theoremstyle{marginbreak}   \newtheorem{Lem}[Cor]{Lemma}
%    \theoremstyle{change}
%    \theorembodyfont{\itshape}        \newtheorem{Def}[Cor]{Definition}
%
%    \theoremheaderfont{\scshape}
%\end{verbatim}
% Then the following are some typical examples of the typeset output
% resulting from their use.
%
% \begin{Cor}
%    This is a sentence typeset in the theorem environment {\envfont
%    Cor}.
% \end{Cor}
%
% \begin{Exa}
%    This is a sentence typeset in the theorem environment {\envfont
%    Exa}.
% \end{Exa}
%
% \begin{Rem}
%    This is a sentence typeset in the theorem environment {\envfont
%    Rem}.
% \end{Rem}
%
% \begin{Lem}[Ben User]
%    This is a sentence typeset in the theorem environment {\envfont
%    Lem}.
% \end{Lem}
%
% \begin{Def}[Very impressive Definition]
%    This is a sentence typeset in the theorem environment {\envfont
%    Def}.
% \end{Def}
%
% The last two examples show the effect of the optional argument to
% a theorem environment (it is the text typeset in parentheses).
%
%
% \section{Special Considerations}
% \label{sec:newfonts}
%
% Theoremheader and body are implemented as a unit. This means that
% the |\theoremheaderfont|  will inherit characteristics of the
% |\theorembodyfont| in \LaTeXe. Thus, if for example
% |\theorembodyfont| is |\itshape| and |\theoremheaderfont|
% is |\bfseries| the font selected for the header will have the
% characteristics `bold extended italic'. If this is not desired one
% should set the |\theoremheaderfont| to something like
% \begin{verbatim}
%      \theoremheaderfont{\normalfont\bfseries}
%\end{verbatim}
% i.e.\ supplying all necessary font information explicitly.
%
%
%
% \section{Acknowledgements}
%
% The publication of this set of macros was only possible with the
% help of Christina Busse (translating the manuscript into English),
% Joachim Pense (playing the r\^ole of typist), Chris Rowley
% (looking everything over) and many others providing useful
% suggestions.
%
% \MaybeStop{%
%    \begin{thebibliography}{9}
%       \bibitem{bk:GMS94} \textsc{M.~Goossens}, \textsc{F.~Mittelbach}
%         and \textsc{A.~Samarin}.
%         \newblock The \LaTeX{} Companion.
%         \newblock
%         Addison-Wesley, Reading, Massachusetts, 1994.
%       \bibitem{latex:source} \textsc{Lamport, Leslie.} \newblock
%                               \texttt{latex.tex}, version 2.09, date
%                               Feb. 1990.
%    \end{thebibliography}
%    \PrintIndex
%    \PrintChanges}
%
%
% \section{The documentation driver file}
%
% The next bit of code contains the documentation driver file for
% \TeX{}, i.e., the file that will produce the documentation you are
% currently reading. It will be extracted from this file by the
% \texttt{docstrip} program. Since it is the first code in the file
% one can alternatively process this file directly with \LaTeXe{} to
% obtain the documentation.
%
%    \begin{macrocode}
%<*driver>
\documentclass{ltxdoc}

\usepackage{theorem}

%<+driver>% The next few lines define theorem sets which are used
%<+driver>% in the example section of the documentation.

\theoremstyle{break}       \newtheorem{Cor}{Corollary}
\theoremstyle{plain}       \newtheorem{Exa}{Example}[section]
{\theorembodyfont{\upshape}\newtheorem{Rem}{Remark}}
\theoremstyle{marginbreak} \newtheorem{Lem}[Cor]{Lemma}
\theoremstyle{change}
\theorembodyfont{\itshape} \newtheorem{Def}[Cor]{Definition}

\theoremheaderfont{\scshape}

\RecordChanges

\begin{document}
   \DocInput{theorem.dtx}
\end{document}
%</driver>
%    \end{macrocode}
%
% \section{Definition of the Macros}
%
% If the file has been loaded before, we abort immediately. If not the
% package announces itself (this is actually done at the very top of
% the file---the way it is done isn't good style so don't copy it).
% \changes{v2.0e}{89/07/19}{Spaces removed from `typeout'.}
% \changes{v2.2b}{95/11/19}{Announce moved to top of file}
%    \begin{macrocode}
%<*package>
%\@ifundefined{theorem@style}{}{\endinput}
%\def\FMithmInfo{1995/11/19 v2.2b Theorem extension package (FMi)}
%\ProvidesPackage{theorem}[\FMithmInfo]
%    \end{macrocode}
%
%
% \subsection{Definition of theorem styles and fonts}
%
% All the definitions in this file are done globally to allow
% inputting this file inside a group.
%
% \begin{macro}{\theoremstyle}
%    Before a theorem style can be installed, the chosen style must be
%    known. For that reason, we must test to see that
%    |\th@|\meta{style} is known or, more precisely, that it is
%    different from |\relax|. If the style is not known then
%    |\th@plain| is used.
%    \begin{macrocode}
\gdef\theoremstyle#1{%
   \@ifundefined{th@#1}{\@warning
          {Unknown theoremstyle `#1'. Using `plain'}%
          \theorem@style{plain}}%
%    \end{macrocode}
%    We save the theorem style to be used in the token register
%    |\theorem@style|.
%    \begin{macrocode}
      {\theorem@style{#1}}%
%    \end{macrocode}
%    Now we ``evaluate'' the theorem style: this means, we call the
%    macro |\th@|\meta{style} which will activate the relevant
%    definitions which are contained in a separate file.  This is done
%    in a group to suppress changes to the current font.  This could
%    otherwise pose problems together with the new font selection
%    scheme\footnote{When I printed the original article using the new
%    font selection scheme I ended with a document with slanted
%    typefaces (text headings and all) simply because one of the
%    theorem styles used \texttt{\bslash sl} at toplevel.} if the
%    |\th@|\meta{style} is evaluated a second time.
% \changes{v2.1a}{90/02/15}{Group added around `th@..}
% \changes{v2.1a}{90/02/15}{Reseting of `..@begintheorem removed.}
%    \begin{macrocode}
      \begingroup
        \csname th@\the\theorem@style \endcsname
      \endgroup}
%    \end{macrocode}
% \end{macro}
%
%
% \begin{macro}{\@begintheorem}
% \changes{v2.1a}{90/02/15}{Relaxed at toplevel.}
% \begin{macro}{\@opargbegintheorem}
% \changes{v2.1a}{90/02/15}{Relaxed at toplevel.}
%    We reset |\@begintheorem| and |\@opargbegintheorem| to
%    |\relax| since these commands are no longer necessary at
%    toplevel. This will save a few tokens.
%    \begin{macrocode}
\global\let\@begintheorem\relax
\global\let\@opargbegintheorem\relax
%    \end{macrocode}
% \end{macro}
% \end{macro}
%
% \begin{macro}{\theorem@style}
%    Obviously the \textsf{token} register used above has to be
%    allocated. To assure the utmost compatibility with the
%    original \LaTeX{} definition, we set the default theorem style to
%    \texttt{plain}, which implements the usual \LaTeX{} convention.
% \changes{v2.0f}{89/07/20}{Unnecessary `global' removed.}
%    \begin{macrocode}
\newtoks\theorem@style
\global\theorem@style{plain}
%    \end{macrocode}
% \end{macro}
%
% \begin{macro}{\theorembodyfont}
% \begin{macro}{\theorem@bodyfont}
%    If the |\theorembodyfont| is set by the user then it should not
%    interact with the default font set in the theorem style. When the
%    new font selection is in force this may happen if, for example,
%    the default is |\itshape| and the new |\theorembodyfont| is
%    |\sffamily|. So we add a |\reset@font| command in front of the
%    user definition.
% \changes{v2.0f}{89/07/20}{Unnecessary `global' removed.}
% \changes{v2.0g}{89/09/19}{`theoremfont to `theorembodyfont.}
% \changes{v2.1c}{92/06/22}{`reset@font added and logic changed.}
% \changes{v2.1d}{92/10/28}{Reset to empty if arg is empty}
%    \begin{macrocode}
\gdef\theorembodyfont#1{%
%    \end{macrocode}
%    We check if the argument supplied is empty and if so put nothing
%    into the |\theorem@bodyfont| token register to allow for
%    |\theorembodyfont{}| as a mean of using the default of the
%    current |\theoremstyle|.
%    \begin{macrocode}
   \def\@tempa{#1}%
   \ifx\@tempa\@empty
    \theorem@bodyfont{}%
   \else
    \theorem@bodyfont{\reset@font#1}%
   \fi
   }
\newtoks\theorem@bodyfont
\global\theorem@bodyfont{}
%    \end{macrocode}
% \end{macro}
% \end{macro}
%
%
% \begin{macro}{\theoremheaderfont}
% \changes{v2.0g}{89/09/19}{Macro added.}
%    The font for the theorem headers is handled differently because
%    this definition applies to all theorem styles.
%    \begin{macrocode}
\gdef\theoremheaderfont#1{\gdef\theorem@headerfont{#1}%
%    \end{macrocode}
%    After using the macro once it is redefined to produce an error
%    message.
%    \begin{macrocode}
       \gdef\theoremheaderfont##1{%
        \typeout{\string\theoremheaderfont\space should be used
                 only once.}}}
%    \end{macrocode}
% \end{macro}
%
% \begin{macro}{\theorem@headerfont}
% \changes{v2.0g}{89/09/19}{Macro added.}
%    To set the |\theorem@headerfont| default we first test if
%    the new fontselection scheme is in force.
% \changes{v2.1a}{90/02/15}{Support for new font selection.}
%    \begin{macrocode}
\ifx\upshape\undefined
%    \end{macrocode}
%    If not we define it to expand into |\bfseries|. We don't use
%    |\let| just in case a following style option redefines this
%    macro.
%    \begin{macrocode}
\gdef\theorem@headerfont{\bfseries}
%    \end{macrocode}
%    Otherwise we reset the current shape before calling |\bfseries|.
%    \begin{macrocode}
\else \gdef\theorem@headerfont{\normalfont\bfseries}\fi
%    \end{macrocode}
% \end{macro}
%
% \begin{macro}{\th@plain}
% \begin{macro}{\th@break}
% \begin{macro}{\th@marginbreak}
% \begin{macro}{\th@changebreak}
% \begin{macro}{\th@change}
% \begin{macro}{\th@margin}
%    The different styles are defined in macros such as
%    |\th@plain|.  Since memory space is precious in
%    ``non-Big-versions'', we have to avoid offering too many unused
%    definitions.  Therefore we define these styles in separate files
%    that can be loaded on demand.  Thus the commands themselves only
%    load these files. We use |\@input@|, a \LaTeXe{} internal command
%    that ensures that the file will be listed with |\listfiles|.
% \changes{v2.2a}{94/02/02}{Use `@input@ to load theorem layout files}
%    \begin{macrocode}
\gdef\th@plain{\@input@{thp.sty}}
\gdef\th@break{\@input@{thb.sty}}
\gdef\th@marginbreak{\@input@{thmb.sty}}
\gdef\th@changebreak{\@input@{thcb.sty}}
\gdef\th@change{\@input@{thc.sty}}
\gdef\th@margin{\@input@{thm.sty}}
%    \end{macrocode}
%    This list will be expanded when new styles become
%    available.  For testing,  just append new theorem substyles
%    as document options.
% \end{macro}
% \end{macro}
% \end{macro}
% \end{macro}
% \end{macro}
% \end{macro}
%
%
%
% \subsection{Definition of a new theorem set}
%
% As already pointed out, a new theorem environment can be defined in
% three different ways:
% \begin{verbatim}
%    \newtheorem{Lem}{Lemma}
%    \newtheorem{Lem}{Lemma}[section]
%    \newtheorem{Lem}[Theorem]{Lemma}
%\end{verbatim}
% The function of the macro |\newtheorem| is to recognize these cases
% and then to branch into one of the three macros |\@ynthm|, |\@xnthm|
% or |\@othm|.  This mechanism is adopted unchanged from
% \cite{latex:source}; the essential point here is that, for example,
% in the second case, the arguments \texttt{Lem}, \texttt{Lemma} and
% \texttt{section} are passed over to the macro |\@xnthm|.
%
% We inspect this case first because the others present fewer problems,
% and thus are easily derived from this one.
%
% \begin{macro}{\@xnthm}
%    For our example arguments, the macro |\@xnthm| must fulfill
%    the following:
%    \begin{itemize}
%       \item
%          Define a new \LaTeX-counter `Lem'
%       \item
%          reset this counter within a |\section|
%       \item
%          define the macro |\theLem|
%       \item
%          define the environment macros |\Lem| and
%          |\endLem| using the current
%          |\theoremstyle| and |\theorem@bodyfont|.
%    \end{itemize}
%    Obviously, all this should happen only if the first argument
%    of |\@xnthm| (i.e.\ \texttt{Lem} in our example) is chosen so
%    as not to conflict with any previously defined commands or
%    environments.  This test is performed by the \LaTeX{} macro
%    |\@ifdefinable|.
%    \begin{macrocode}
\gdef\@xnthm#1#2[#3]{\expandafter\@ifdefinable\csname #1\endcsname
%    \end{macrocode}
%    Therefore, the first argument of |\@ifdefinable| is the
%    expansion (in the example, |\Lem|) of
%    |\csname||#1\endcsname|.  The second argument is
%    executed only if the test has been completed successfully.
%    \begin{macrocode}
   {%
%    \end{macrocode}
%    Now we define the new counter. The names of the \LaTeX{}
%    macros employed should speak for themselves:
% \changes{v2.2b}{1995/10/07}{Use \cs{@newctr}}
% \changes{v2.2c}{1995/11/23}{Use \cs{@newctr} properly.}
%    \begin{macrocode}
    \@definecounter{#1}\@newctr{#1}[#3]%
%    \end{macrocode}
%    Using |\@newctr| will give a proper error message if the counter
%    in |#3| is not defined.
%    In defining `|\theLem|' we must generate the desired macro
%    name by use of |\expandafter| and |\csname|.
%    \begin{macrocode}
    \expandafter\xdef\csname the#1\endcsname
%    \end{macrocode}
%    An |\xdef| is used in order to make the definition global,
%    and to ensure that it contains the replacement texts of
%    |\@thmcountersep| and |\@thmcounter|.\footnote{These
%    two macros can be defined by the document style.  Their default
%    values produce a `\texttt{.}' as separation and an arabic
%    representation of the number.} However, not everything should be
%    expanded. For example, it saves space to use |\thesection|
%    instead of its---at times---lengthy expansion.
%    \begin{macrocode}
      {\expandafter \noexpand \csname the#3\endcsname
       \@thmcountersep \@thmcounter{#1}}%
%    \end{macrocode}
%    Thus with the defaults of \LaTeX{}, |\theLem| would be
%    replaced by the command sequence |\thesection.\arabic{Lem}|.
%
%    We will now look at the definition of the macro which is executed
%    at the beginning of the actual environment (in our example this
%    macro is |\Lem|). It should be noted that we use an
%    ``|\expandafter| trick'' to expand only certain parts of the
%    replacement text at the time of the definition.
%    \begin{macrocode}
    \def\@tempa{\global\@namedef{#1}}%
    \expandafter \@tempa \expandafter{%
%    \end{macrocode}
%    First, the macro that contains the current definitions of
%    |\@begintheorem| and |\@opargtheorem| should be called
%    up.  The name of this macro---as is already known---has the form
%    |\th@|\meta{theorem\ style}; therefore, it must be called by
%    \begin{macrocode}
      \csname th@\the \theorem@style
%    \end{macrocode}
%    In addition the default theorem font should be changeable, i.e.\
%    we have to insert the contents of |\theorem@bodyfont|. For
%    that reason, we expand even further, beyond |\endcsname|,
%    and thus insert the contents of the \textsf{token} register
%    |\theorem@bodyfont| in the replacement text.
%    \begin{macrocode}
            \expandafter \endcsname \the \theorem@bodyfont
%    \end{macrocode}
%    Now it is time to call the macro |\@thm| which takes over
%    the further processing.  It has two arguments: the current
%    counter name (in our example, \texttt{Lem}), and the text of the
%    label (in our example, \texttt{Lemma}).
%    \begin{macrocode}
     \@thm{#1}{#2}}%
%    \end{macrocode}
%    With this, the `sub-definition' is complete.  The macro
%    |\@endtheorem| ends a theorem environment and is, so far, nothing
%    but an |\endtrivlist|.  (Hence it is defined globally, and not
%    within the theorem styles.\footnote{This has to be changed as
%    soon as theorem styles that change \texttt{\bslash @endtheorem}
%    exist.  In such a case, all existing styles must be changed as
%    well since they will have to reset the macro.}) Therefore, we can
%    set it equivalent to the macro that ends the theorem set (in our
%    example, |\endLem|).  However, if some day theorem styles exist
%    that do change |\@endtheorem|, we would have to use the
%    commented-out line instead.
%    \begin{macrocode}
    \global \expandafter \let \csname end#1\endcsname \@endtheorem
%   \global\@namedef{end#1}{\@endtheorem}%
%    \end{macrocode}
%    With these commands all the required definitions are employed,
%    unless the test |\@ifdefinable| has failed.  Therefore, we
%    end the second argument of this macro and with it the definition
%    of |\@xnthm|.
%    \begin{macrocode}
   }}
%    \end{macrocode}
% \end{macro}
%
% \begin{macro}{\@ynthm}
%    The definition of |\@ynthm| is completely analogous. In
%    this case the new counter that is defined is not reset within
%    another counter; thus the definition of |\the...|
%    is simplified:
%    \begin{macrocode}
\gdef\@ynthm#1#2{\expandafter\@ifdefinable\csname #1\endcsname
   {\@definecounter{#1}%
    \expandafter\xdef\csname the#1\endcsname{\@thmcounter{#1}}%
%    \end{macrocode}
%    The rest of the definition corresponds literally to that of
%    |\@xnthm|:
%    \begin{macrocode}
    \def\@tempa{\global\@namedef{#1}}\expandafter \@tempa
     \expandafter{\csname th@\the \theorem@style \expandafter
     \endcsname \the\theorem@bodyfont \@thm{#1}{#2}}%
    \global \expandafter \let \csname end#1\endcsname \@endtheorem}}
%    \end{macrocode}
% \end{macro}
%
% \begin{macro}{\@othm}
%    The definition of |\@othm| does not contain anything new.
%    \begin{macrocode}
\gdef\@othm#1[#2]#3{%
%    \end{macrocode}
% \changes{v2.2b}{95/10/07}{Build in check for counter defined}
%    We do not define a new counter but instead use one that has
%    already been defined.  Thus the only definition we need is that
%    of this pseudo-counter (i.e.\ |\the|\meta{env.\ name}).
%    First we check if |#2| corresponds to a known counter name.
%    \begin{macrocode}
  \expandafter\ifx\csname c@#2\endcsname\relax
   \@nocounterr{#2}%
  \else
   \expandafter\@ifdefinable\csname #1\endcsname
   {\expandafter \xdef \csname the#1\endcsname
     {\expandafter \noexpand \csname the#2\endcsname}%
%    \end{macrocode}
%    All other parts of the definition can be adopted from
%    |\@xnthm|. We have to remember, though, that in this case
%    the name of the current counter and the theorem label have moved
%    to the second and third arguments.
%    \begin{macrocode}
    \def\@tempa{\global\@namedef{#1}}\expandafter \@tempa
     \expandafter{\csname th@\the \theorem@style \expandafter
     \endcsname \the\theorem@bodyfont \@thm{#2}{#3}}%
    \global \expandafter \let \csname end#1\endcsname \@endtheorem}%
  \fi}
%    \end{macrocode}
% \end{macro}
%
% \subsection{Macros that are employed in a {\envfont theorem}
%    environment}
%
% \begin{macro}{\@thm}
%    The macro |\@thm| has to increase the current counter.  Then,
%    depending on whether the environment has (or does not have) an
%    optional argument, it has to branch into either |\@begintheorem|
%    or |\@opargtheorem|.
%    \begin{macrocode}
\gdef\@thm#1#2{\refstepcounter{#1}%
%    \end{macrocode}
%    Now we start a {\envfont trivlist} environment, and give
%    |\@topsep| and |\@topsepadd| the values of the \textsf{skip}
%    registers |\theorempreskipamount| and |\theorempostskipamount|.
%    The value in |\@topsep| is the vertical space that is inserted by
%    the first (and only) |\item| in our |\trivlist| whilst
%    |\@topsepadd| is inserted by |\@endparenv| at the end of that
%    {\envfont trivlist} environment.  By using these registers, we
%    obtain the desired space around a {\envfont theorem} environment.
%    \begin{macrocode}
   \trivlist
   \@topsep \theorempreskipamount               % used by first \item
   \@topsepadd \theorempostskipamount           % used by \@endparenv
%    \end{macrocode}
%    Now we have to test whether an optional argument has been given.
% \changes{v2.0d}{89/07/18}{Forgotten percent added.}
%    \begin{macrocode}
   \@ifnextchar [%
%    \end{macrocode}
%    If there is an optional argument, we will call |\@ythm|, and
%    move the arguments read back into the input stream.
%    \begin{macrocode}
   {\@ythm{#1}{#2}}%
%    \end{macrocode}
%    If not, we call |\@begintheorem|. Its first argument is the
%    name of the theorem set (hence the second argument of
%    |\@thm|). Its second argument is the macro that produces the
%    current number.
%    \begin{macrocode}
   {\@begintheorem{#2}{\csname the#1\endcsname}\ignorespaces}}
%    \end{macrocode}
% \end{macro}
%
% \begin{macro}{\@xthm}
% \begin{macro}{\@ythm}
%    Both these macros were originally called by |\@thm|.  We do
%    not need |\@xthm| anymore, hence we reset it to
%    |\relax|. The definition of |\@ythm| has not changed at
%    all from its definition in \LaTeX. In order to make the macros
%    easier to understand, we will
%    nevertheless present it (commented out).
%    \begin{macrocode}
\global\let\@xthm\relax
% \def\@ythm#1#2[#3]{\@opargbegintheorem{#2}{\csname
%           the#1\endcsname}{#3}\ignorespaces}
%</package>
%    \end{macrocode}
%    The primitive |\ignorespaces| in |\@ythm| and
%    |\@thm| is needed to remove the spaces between the
%    |\begin{...}| and the actual text.
% \end{macro}
% \end{macro}
%
%
% \subsection{Definition of the theorem substyles}
%
% As already pointed out, the theorem substyles, defined below, are
% only loaded when necessary.  Note that all these substyles, except
% \texttt{plain}, have |\slshape| as the default body font.
%
% \subsubsection{The \texttt{plain} style}
%
% \label{dok:th@plain} As the following macros use |@|, we have
% to locally set the |\catcode| of this symbol to ``letter''.
% This happens within a group, so that we do not have to worry about
% which |\catcode| that symbol had before.
%    \begin{macrocode}
%<*thp>
\begingroup \makeatletter
%    \end{macrocode}
% Since we are now within a group, we must make all definitions
% globally.  First we make sure that \texttt{theorem.sty} is loaded.
% This will allow us to use this file as a document style option
% without having to call \texttt{theorem} itself as an option.  At the
% same time, we assure that at least version 2 is loaded, since
% |\theorem@style| was not defined in earlier versions.
%    \begin{macrocode}
\@ifundefined{theorem@style}{% \iffalse meta-comment
%
% Copyright (C) 1993-2024
%
% The LaTeX Project and any individual authors listed elsewhere
% in this file.
%
% This file is part of the Standard LaTeX `Tools Bundle'.
% -------------------------------------------------------
%
% It may be distributed and/or modified under the
% conditions of the LaTeX Project Public License, either version 1.3c
% of this license or (at your option) any later version.
% The latest version of this license is in
%    https://www.latex-project.org/lppl.txt
% and version 1.3c or later is part of all distributions of LaTeX
% version 2005/12/01 or later.
%
% The list of all files belonging to the LaTeX `Tools Bundle' is
% given in the file `manifest.txt'.
%
% \fi
%
%%
%
%
\def\FMithmInfo{2023/07/05 v2.2c Theorem extension package (FMi)}
%
% \ProvidesFile{theorem.dtx}[\FMithmInfo]
%\iffalse   % this is a METACOMMENT !
%%
%% Package `theorem' to use with LaTeX2e
%% Copyright (C) 1989-1995 Frank Mittelbach, all rights reserved.
%<+package>\@ifundefined{theorem@style}{}{\endinput}
%<+package>\NeedsTeXFormat{LaTeX2e}[1995/06/01]
%<+package>\ProvidesPackage{theorem}
%<+package>         [\FMithmInfo]
%
% \fi
%
% \changes{v1.0}{87/05/07}{First draft.}
% \changes{v1.0a}{87/05/07}{`thistheoremstyle' added.}
% \changes{v1.1a}{87/07/29}{New concept: redefine the internal
%                          \protect\LaTeX{} commands `@xnthm' etc.}
% \changes{v1.1b}{87/07/31}{`theoremfont' used to specify font.}
% \changes{v1.2a}{88/01/08}{`@endtheorem' now globally defined.}
% \changes{v1.2b}{88/01/11}{`newtheorem' only allowed in the preamble.}
% \changes{v1.3a}{88/02/14}{`theoremfont' now a toks register.}
%
% \changes{v2.0a}{88/11/16}{New implementation and new documentation.}
% \changes{v2.0a}{88/11/16}{No layout changes but internal modifications
%                           old substyles won't work.}
% \changes{v2.0b}{89/01/30}{`global' used for all definitions.}
% \changes{v2.0c}{89/06/15}{Documentation translated to English.
%                            Thanks to LLL and Nina}
% \changes{v2.1a}{90/02/15}{Support for new font selection.}
% \changes{v2.2a}{94/02/03}{Cleanup for LaTeX2e}
%
% \DoNotIndex{\@endparenv,\@ifdefinable,\@ifnextchar,\@ifundefined}
% \DoNotIndex{\@namedef,\@nameuse,\@spaces,\@tempa,\@warning}
% \DoNotIndex{\ }
% \DoNotIndex{\begingroup,\bfseries}
% \DoNotIndex{\csname}
% \DoNotIndex{\def,\do,\docdate}
% \DoNotIndex{\endcsname,\endgroup,\endinput,\endtrivlist,\expandafter}
% \DoNotIndex{\filedate,\fileversion}
% \DoNotIndex{\gdef,\global}
% \DoNotIndex{\hbox,\hskip}
% \DoNotIndex{\ignorespaces,\input,\itshape,\item}
% \DoNotIndex{\labelsep,\let,\llap}
% \DoNotIndex{\makeatletter}
% \DoNotIndex{\noexpand}
% \DoNotIndex{\refstepcounter,\relax,\rlap}
% \DoNotIndex{\setlength,\slshape,\space,\string,\strut}
% \DoNotIndex{\the,\trivlist,\typeout}
% \DoNotIndex{\vbox}
% \DoNotIndex{\xdef}
%
% \def\deflabel#1{\ttfamily #1\hfill}
% \def\deflist#1{\begin{list}{}{\settowidth\labelwidth{\ttfamily #1}%
%                               \setlength\leftmargin\labelwidth
%                               \addtolength\leftmargin\labelsep
%                               \let\makelabel\deflabel}}
% \def\enddeflist{\end{list}}
% \def\envfont{\normalfont\ttfamily}
%
% \MakeShortVerb{\|}
%
% \newcommand{\TUB}{{\sl TUGboat\/}}
%
% \GetFileInfo{theorem.dtx}
% \title{An Extension of the
%        \LaTeX{} theorem environment\thanks{This file
%        has version number \fileversion, last
%        revised \filedate.}}
% \author{Frank Mittelbach}
%
% \MaintainedByLaTeXTeam{tools}
% \maketitle
%
% \begin{abstract}
%    The macros described in this paper yield an extension of the
%    \LaTeX{} theorem mechanism. It is designed to satisfy the
%    different requirements of various journals. Thus, the
%    layout of the ``theorems'' can be manipulated by determining a
%    ``style''. This article describes not only the use, but
%    also the definition, of the necessary macros.
% \end{abstract}
%
% \section*{Preface to version 2.2}
%
% For \LaTeXe{} this package did not need any fundamental changes. I
% only modified the messages generated so that theorem layout styles
% will show up with the |\listfiles| command and cleaned the section
% on the New Font Selection Scheme since this is now included in
% \LaTeX.
%
% \section*{Preface to version 2.1}
%
% This version is identical to 2.0g described in \TUB\ 10\#3 except
% for some internal defaults which are now set depending on the used
% font selection scheme.
%
% This was done to avoid unpleasant surprises if the new font selection
% scheme is in force. For further details see
% section~\ref{sec:newfonts} and \cite{bk:GMS94}.
%
% \section{Introduction}
%
% For our purposes here, ``theorems'' are labelled enunciations,
% often set off from the main text by extra space and a font change.
% Theorems, corollaries, conjectures, definitions, and remarks
% are all instances of ``theorems''.  The ``header'' of these structures
% is composed of a label (such as \textsc{Theorem} or \textsc{Remark})
% and a number which serializes an item in the sequence of items
% with the same label.
%
% Shortly after the introduction of \LaTeX{} at the Fachbereich
% Mathematik in Mainz, the desire to manipulate the layout of
% ``theorems'' arose.  In Mainz, the following two conventions came
% into general use:
% \begin{enumerate}
%   \item The number of the theorem is shown in the margin.
%   \item There is a line break at
%         the end of the theorem header.
% \end{enumerate}
%
% Additionally, some journals require different formats which depend on
% the ``sort of theorem'': e.g.\ often remarks and definitions are set
% in |\upshape|, while |\itshape| is employed for main theorems.
%
% Confronted with these requirements, a theorem environment was
% developed in Mainz which allows separate determination of the
% layout of the ``theorems sets'', comparable to |\pagestyle|.
%
%
% \section{The user interface}
% \subsection{Defining new theorem sets}
%
% \DescribeMacro\newtheorem
% As in the original \LaTeX{} version, the command |\newtheorem|
% defines a new ``theorem set'' or ``theorem-like structure''.
% Two required arguments name the new environment and give the text
% to be typeset with each instance of the new ``set'', while
% an optional argument determines how
% the ``set'' is enumerated:
% \begin{description}
%    \item[\ttfamily \bslash newtheorem\{foo\}\{bar\}]
%       The theorem set {\envfont foo} (whose name is \texttt{bar})
%       uses its own counter.
%    \item[\ttfamily \bslash newtheorem\{foo2\}{[foo]}\{bar2\}]
%       The theorem set {\envfont foo2} (printed name \texttt{bar2})
%       uses the same counter as the theorem set \texttt{foo}.
%    \item[\ttfamily \bslash newtheorem\{foo3\}\{bar3\}{[section]}]
%       The theorem set {\envfont foo3} (printed name \texttt{bar3}) is
%       enumerated within the counter \texttt{section}, i.e.\ with every
%       new |\section| the enumeration begins again with 1, and
%       the enumeration is composed from the section-number and the
%       theorem counter itself.
% \end{description}
%
% \DescribeMacro\theoremstyle
% Additionally, the command |\theoremstyle| can define the layout
% of various, or all, theorem sets.  It should be noted that any
% theorem set defined by |\newtheorem| is typeset in the
% |\theoremstyle| that is current at the time of the definition.
% Thus, the following
% \begin{verbatim}
%    \theoremstyle{break}        \newtheorem{Cor}{Corollary}
%    \theoremstyle{plain}        \newtheorem{Exa}{Example}[section]
%\end{verbatim}                             ^^A No blank at line start
% leads to the result that the set {\envfont Cor} is formatted in the
% style \texttt{break}, while the set {\envfont Exa} and all the
% following ones are formatted in the style \texttt{plain}, unless
% another |\theoremstyle| follows.  Since the definitions installed by
% |\newtheorem| are global, one also can limit |\theoremstyle| locally
% by grouping braces.
%
% \DescribeMacro\theorembodyfont
% The choice of the font for the theorem body is completely
% independent of the chosen |\theoremstyle|; this has proven to
% be very advantageous.  For example,
% \begin{verbatim}
%   {\theorembodyfont{\upshape}          \newtheorem{Rem}{Remark}}
%\end{verbatim}
% defines a theorem set {\envfont Rem}, which will be set in
% |\upshape| in the current layout (which in our example is
% \texttt{plain}).  As with |\theoremstyle|, the |\theorembodyfont|
% chosen is that current at the time of |\newtheorem|.  If
% |\theorembodyfont| is not specified or one defines
% |\theorembodyfont{}|, then the font used will be that defined by the
% |\theoremstyle|.
%
% \DescribeMacro\theoremheaderfont It is also possible to customize
% the font used for the theorem headers.  This is, however, a global
% declaration, and therefore there should be at most one
% |\theoremheaderfont| declaration in the preamble.\footnote{If it is
% actually necessary to have different header fonts, one has to define
% new theorem styles (substituting the desired font) or specify the
% information directly in the \texttt{\bslash newtheorem} declaration
% (the unclean variant).}
%
% \DescribeMacro\theorempreskipamount
% \DescribeMacro\theorempostskipamount Two additional parameters
% affect the vertical space around the theorem environments:
% |\theorempreskipamount| and |\theorempostskipamount| define,
% respectively, the spacing before and after such an environment.
% These parameters apply for all theorem sets and can be manipulated
% with the ordinary length macros.  They are rubber lengths
% (`\textsf{skips}'), and therefore can contain \texttt{plus} and
% \texttt{minus} parts.
%
% Since the definition of theorem sets should---most sensibly---be
% placed in the preamble, we only allow installation there.
% It is therefore possible to release the memory used here after
% |\begin{document}|, in order to make room for other
% applications.
%
%
% \subsection{Existing theorem styles}
%
% The following theorem styles exist to date:
% \begin{deflist}{marginbreak}
%    \item[plain]
%       This theorem style emulates the original \LaTeX{} definition,
%       except that additionally the parameters
%       |\theorem...skipamount| are used.
%    \item[break]
%       In this style, the theorem header is followed by a line
%       break.
%    \item[marginbreak]
%       The theorem number is set in the margin, and there is a line
%       break as in \texttt{break}.
%    \item[changebreak]
%       Like \texttt{break}, but with header number and text
%       interchanged.
%    \item[change]
%       Header number and text are interchanged, without a line break.
%    \item[margin]
%       The number is set in the left margin, without a line break.
% \end{deflist}
% All styles (except \texttt{plain}) select |\slshape| as the default
% |\theorembodyfont|.
%
%
% \subsection{Examples}
%
% Given the above theorem sets {\envfont Cor}, {\envfont Exa}
% and {\envfont Rem},
% suppose that the preamble also contains the declarations:
% \begin{verbatim}
%    \theoremstyle{marginbreak}   \newtheorem{Lem}[Cor]{Lemma}
%    \theoremstyle{change}
%    \theorembodyfont{\itshape}        \newtheorem{Def}[Cor]{Definition}
%
%    \theoremheaderfont{\scshape}
%\end{verbatim}
% Then the following are some typical examples of the typeset output
% resulting from their use.
%
% \begin{Cor}
%    This is a sentence typeset in the theorem environment {\envfont
%    Cor}.
% \end{Cor}
%
% \begin{Exa}
%    This is a sentence typeset in the theorem environment {\envfont
%    Exa}.
% \end{Exa}
%
% \begin{Rem}
%    This is a sentence typeset in the theorem environment {\envfont
%    Rem}.
% \end{Rem}
%
% \begin{Lem}[Ben User]
%    This is a sentence typeset in the theorem environment {\envfont
%    Lem}.
% \end{Lem}
%
% \begin{Def}[Very impressive Definition]
%    This is a sentence typeset in the theorem environment {\envfont
%    Def}.
% \end{Def}
%
% The last two examples show the effect of the optional argument to
% a theorem environment (it is the text typeset in parentheses).
%
%
% \section{Special Considerations}
% \label{sec:newfonts}
%
% Theoremheader and body are implemented as a unit. This means that
% the |\theoremheaderfont|  will inherit characteristics of the
% |\theorembodyfont| in \LaTeXe. Thus, if for example
% |\theorembodyfont| is |\itshape| and |\theoremheaderfont|
% is |\bfseries| the font selected for the header will have the
% characteristics `bold extended italic'. If this is not desired one
% should set the |\theoremheaderfont| to something like
% \begin{verbatim}
%      \theoremheaderfont{\normalfont\bfseries}
%\end{verbatim}
% i.e.\ supplying all necessary font information explicitly.
%
%
%
% \section{Acknowledgements}
%
% The publication of this set of macros was only possible with the
% help of Christina Busse (translating the manuscript into English),
% Joachim Pense (playing the r\^ole of typist), Chris Rowley
% (looking everything over) and many others providing useful
% suggestions.
%
% \MaybeStop{%
%    \begin{thebibliography}{9}
%       \bibitem{bk:GMS94} \textsc{M.~Goossens}, \textsc{F.~Mittelbach}
%         and \textsc{A.~Samarin}.
%         \newblock The \LaTeX{} Companion.
%         \newblock
%         Addison-Wesley, Reading, Massachusetts, 1994.
%       \bibitem{latex:source} \textsc{Lamport, Leslie.} \newblock
%                               \texttt{latex.tex}, version 2.09, date
%                               Feb. 1990.
%    \end{thebibliography}
%    \PrintIndex
%    \PrintChanges}
%
%
% \section{The documentation driver file}
%
% The next bit of code contains the documentation driver file for
% \TeX{}, i.e., the file that will produce the documentation you are
% currently reading. It will be extracted from this file by the
% \texttt{docstrip} program. Since it is the first code in the file
% one can alternatively process this file directly with \LaTeXe{} to
% obtain the documentation.
%
%    \begin{macrocode}
%<*driver>
\documentclass{ltxdoc}

\usepackage{theorem}

%<+driver>% The next few lines define theorem sets which are used
%<+driver>% in the example section of the documentation.

\theoremstyle{break}       \newtheorem{Cor}{Corollary}
\theoremstyle{plain}       \newtheorem{Exa}{Example}[section]
{\theorembodyfont{\upshape}\newtheorem{Rem}{Remark}}
\theoremstyle{marginbreak} \newtheorem{Lem}[Cor]{Lemma}
\theoremstyle{change}
\theorembodyfont{\itshape} \newtheorem{Def}[Cor]{Definition}

\theoremheaderfont{\scshape}

\RecordChanges

\begin{document}
   \DocInput{theorem.dtx}
\end{document}
%</driver>
%    \end{macrocode}
%
% \section{Definition of the Macros}
%
% If the file has been loaded before, we abort immediately. If not the
% package announces itself (this is actually done at the very top of
% the file---the way it is done isn't good style so don't copy it).
% \changes{v2.0e}{89/07/19}{Spaces removed from `typeout'.}
% \changes{v2.2b}{95/11/19}{Announce moved to top of file}
%    \begin{macrocode}
%<*package>
%\@ifundefined{theorem@style}{}{\endinput}
%\def\FMithmInfo{1995/11/19 v2.2b Theorem extension package (FMi)}
%\ProvidesPackage{theorem}[\FMithmInfo]
%    \end{macrocode}
%
%
% \subsection{Definition of theorem styles and fonts}
%
% All the definitions in this file are done globally to allow
% inputting this file inside a group.
%
% \begin{macro}{\theoremstyle}
%    Before a theorem style can be installed, the chosen style must be
%    known. For that reason, we must test to see that
%    |\th@|\meta{style} is known or, more precisely, that it is
%    different from |\relax|. If the style is not known then
%    |\th@plain| is used.
%    \begin{macrocode}
\gdef\theoremstyle#1{%
   \@ifundefined{th@#1}{\@warning
          {Unknown theoremstyle `#1'. Using `plain'}%
          \theorem@style{plain}}%
%    \end{macrocode}
%    We save the theorem style to be used in the token register
%    |\theorem@style|.
%    \begin{macrocode}
      {\theorem@style{#1}}%
%    \end{macrocode}
%    Now we ``evaluate'' the theorem style: this means, we call the
%    macro |\th@|\meta{style} which will activate the relevant
%    definitions which are contained in a separate file.  This is done
%    in a group to suppress changes to the current font.  This could
%    otherwise pose problems together with the new font selection
%    scheme\footnote{When I printed the original article using the new
%    font selection scheme I ended with a document with slanted
%    typefaces (text headings and all) simply because one of the
%    theorem styles used \texttt{\bslash sl} at toplevel.} if the
%    |\th@|\meta{style} is evaluated a second time.
% \changes{v2.1a}{90/02/15}{Group added around `th@..}
% \changes{v2.1a}{90/02/15}{Reseting of `..@begintheorem removed.}
%    \begin{macrocode}
      \begingroup
        \csname th@\the\theorem@style \endcsname
      \endgroup}
%    \end{macrocode}
% \end{macro}
%
%
% \begin{macro}{\@begintheorem}
% \changes{v2.1a}{90/02/15}{Relaxed at toplevel.}
% \begin{macro}{\@opargbegintheorem}
% \changes{v2.1a}{90/02/15}{Relaxed at toplevel.}
%    We reset |\@begintheorem| and |\@opargbegintheorem| to
%    |\relax| since these commands are no longer necessary at
%    toplevel. This will save a few tokens.
%    \begin{macrocode}
\global\let\@begintheorem\relax
\global\let\@opargbegintheorem\relax
%    \end{macrocode}
% \end{macro}
% \end{macro}
%
% \begin{macro}{\theorem@style}
%    Obviously the \textsf{token} register used above has to be
%    allocated. To assure the utmost compatibility with the
%    original \LaTeX{} definition, we set the default theorem style to
%    \texttt{plain}, which implements the usual \LaTeX{} convention.
% \changes{v2.0f}{89/07/20}{Unnecessary `global' removed.}
%    \begin{macrocode}
\newtoks\theorem@style
\global\theorem@style{plain}
%    \end{macrocode}
% \end{macro}
%
% \begin{macro}{\theorembodyfont}
% \begin{macro}{\theorem@bodyfont}
%    If the |\theorembodyfont| is set by the user then it should not
%    interact with the default font set in the theorem style. When the
%    new font selection is in force this may happen if, for example,
%    the default is |\itshape| and the new |\theorembodyfont| is
%    |\sffamily|. So we add a |\reset@font| command in front of the
%    user definition.
% \changes{v2.0f}{89/07/20}{Unnecessary `global' removed.}
% \changes{v2.0g}{89/09/19}{`theoremfont to `theorembodyfont.}
% \changes{v2.1c}{92/06/22}{`reset@font added and logic changed.}
% \changes{v2.1d}{92/10/28}{Reset to empty if arg is empty}
%    \begin{macrocode}
\gdef\theorembodyfont#1{%
%    \end{macrocode}
%    We check if the argument supplied is empty and if so put nothing
%    into the |\theorem@bodyfont| token register to allow for
%    |\theorembodyfont{}| as a mean of using the default of the
%    current |\theoremstyle|.
%    \begin{macrocode}
   \def\@tempa{#1}%
   \ifx\@tempa\@empty
    \theorem@bodyfont{}%
   \else
    \theorem@bodyfont{\reset@font#1}%
   \fi
   }
\newtoks\theorem@bodyfont
\global\theorem@bodyfont{}
%    \end{macrocode}
% \end{macro}
% \end{macro}
%
%
% \begin{macro}{\theoremheaderfont}
% \changes{v2.0g}{89/09/19}{Macro added.}
%    The font for the theorem headers is handled differently because
%    this definition applies to all theorem styles.
%    \begin{macrocode}
\gdef\theoremheaderfont#1{\gdef\theorem@headerfont{#1}%
%    \end{macrocode}
%    After using the macro once it is redefined to produce an error
%    message.
%    \begin{macrocode}
       \gdef\theoremheaderfont##1{%
        \typeout{\string\theoremheaderfont\space should be used
                 only once.}}}
%    \end{macrocode}
% \end{macro}
%
% \begin{macro}{\theorem@headerfont}
% \changes{v2.0g}{89/09/19}{Macro added.}
%    To set the |\theorem@headerfont| default we first test if
%    the new fontselection scheme is in force.
% \changes{v2.1a}{90/02/15}{Support for new font selection.}
%    \begin{macrocode}
\ifx\upshape\undefined
%    \end{macrocode}
%    If not we define it to expand into |\bfseries|. We don't use
%    |\let| just in case a following style option redefines this
%    macro.
%    \begin{macrocode}
\gdef\theorem@headerfont{\bfseries}
%    \end{macrocode}
%    Otherwise we reset the current shape before calling |\bfseries|.
%    \begin{macrocode}
\else \gdef\theorem@headerfont{\normalfont\bfseries}\fi
%    \end{macrocode}
% \end{macro}
%
% \begin{macro}{\th@plain}
% \begin{macro}{\th@break}
% \begin{macro}{\th@marginbreak}
% \begin{macro}{\th@changebreak}
% \begin{macro}{\th@change}
% \begin{macro}{\th@margin}
%    The different styles are defined in macros such as
%    |\th@plain|.  Since memory space is precious in
%    ``non-Big-versions'', we have to avoid offering too many unused
%    definitions.  Therefore we define these styles in separate files
%    that can be loaded on demand.  Thus the commands themselves only
%    load these files. We use |\@input@|, a \LaTeXe{} internal command
%    that ensures that the file will be listed with |\listfiles|.
% \changes{v2.2a}{94/02/02}{Use `@input@ to load theorem layout files}
%    \begin{macrocode}
\gdef\th@plain{\@input@{thp.sty}}
\gdef\th@break{\@input@{thb.sty}}
\gdef\th@marginbreak{\@input@{thmb.sty}}
\gdef\th@changebreak{\@input@{thcb.sty}}
\gdef\th@change{\@input@{thc.sty}}
\gdef\th@margin{\@input@{thm.sty}}
%    \end{macrocode}
%    This list will be expanded when new styles become
%    available.  For testing,  just append new theorem substyles
%    as document options.
% \end{macro}
% \end{macro}
% \end{macro}
% \end{macro}
% \end{macro}
% \end{macro}
%
%
%
% \subsection{Definition of a new theorem set}
%
% As already pointed out, a new theorem environment can be defined in
% three different ways:
% \begin{verbatim}
%    \newtheorem{Lem}{Lemma}
%    \newtheorem{Lem}{Lemma}[section]
%    \newtheorem{Lem}[Theorem]{Lemma}
%\end{verbatim}
% The function of the macro |\newtheorem| is to recognize these cases
% and then to branch into one of the three macros |\@ynthm|, |\@xnthm|
% or |\@othm|.  This mechanism is adopted unchanged from
% \cite{latex:source}; the essential point here is that, for example,
% in the second case, the arguments \texttt{Lem}, \texttt{Lemma} and
% \texttt{section} are passed over to the macro |\@xnthm|.
%
% We inspect this case first because the others present fewer problems,
% and thus are easily derived from this one.
%
% \begin{macro}{\@xnthm}
%    For our example arguments, the macro |\@xnthm| must fulfill
%    the following:
%    \begin{itemize}
%       \item
%          Define a new \LaTeX-counter `Lem'
%       \item
%          reset this counter within a |\section|
%       \item
%          define the macro |\theLem|
%       \item
%          define the environment macros |\Lem| and
%          |\endLem| using the current
%          |\theoremstyle| and |\theorem@bodyfont|.
%    \end{itemize}
%    Obviously, all this should happen only if the first argument
%    of |\@xnthm| (i.e.\ \texttt{Lem} in our example) is chosen so
%    as not to conflict with any previously defined commands or
%    environments.  This test is performed by the \LaTeX{} macro
%    |\@ifdefinable|.
%    \begin{macrocode}
\gdef\@xnthm#1#2[#3]{\expandafter\@ifdefinable\csname #1\endcsname
%    \end{macrocode}
%    Therefore, the first argument of |\@ifdefinable| is the
%    expansion (in the example, |\Lem|) of
%    |\csname||#1\endcsname|.  The second argument is
%    executed only if the test has been completed successfully.
%    \begin{macrocode}
   {%
%    \end{macrocode}
%    Now we define the new counter. The names of the \LaTeX{}
%    macros employed should speak for themselves:
% \changes{v2.2b}{1995/10/07}{Use \cs{@newctr}}
% \changes{v2.2c}{1995/11/23}{Use \cs{@newctr} properly.}
%    \begin{macrocode}
    \@definecounter{#1}\@newctr{#1}[#3]%
%    \end{macrocode}
%    Using |\@newctr| will give a proper error message if the counter
%    in |#3| is not defined.
%    In defining `|\theLem|' we must generate the desired macro
%    name by use of |\expandafter| and |\csname|.
%    \begin{macrocode}
    \expandafter\xdef\csname the#1\endcsname
%    \end{macrocode}
%    An |\xdef| is used in order to make the definition global,
%    and to ensure that it contains the replacement texts of
%    |\@thmcountersep| and |\@thmcounter|.\footnote{These
%    two macros can be defined by the document style.  Their default
%    values produce a `\texttt{.}' as separation and an arabic
%    representation of the number.} However, not everything should be
%    expanded. For example, it saves space to use |\thesection|
%    instead of its---at times---lengthy expansion.
%    \begin{macrocode}
      {\expandafter \noexpand \csname the#3\endcsname
       \@thmcountersep \@thmcounter{#1}}%
%    \end{macrocode}
%    Thus with the defaults of \LaTeX{}, |\theLem| would be
%    replaced by the command sequence |\thesection.\arabic{Lem}|.
%
%    We will now look at the definition of the macro which is executed
%    at the beginning of the actual environment (in our example this
%    macro is |\Lem|). It should be noted that we use an
%    ``|\expandafter| trick'' to expand only certain parts of the
%    replacement text at the time of the definition.
%    \begin{macrocode}
    \def\@tempa{\global\@namedef{#1}}%
    \expandafter \@tempa \expandafter{%
%    \end{macrocode}
%    First, the macro that contains the current definitions of
%    |\@begintheorem| and |\@opargtheorem| should be called
%    up.  The name of this macro---as is already known---has the form
%    |\th@|\meta{theorem\ style}; therefore, it must be called by
%    \begin{macrocode}
      \csname th@\the \theorem@style
%    \end{macrocode}
%    In addition the default theorem font should be changeable, i.e.\
%    we have to insert the contents of |\theorem@bodyfont|. For
%    that reason, we expand even further, beyond |\endcsname|,
%    and thus insert the contents of the \textsf{token} register
%    |\theorem@bodyfont| in the replacement text.
%    \begin{macrocode}
            \expandafter \endcsname \the \theorem@bodyfont
%    \end{macrocode}
%    Now it is time to call the macro |\@thm| which takes over
%    the further processing.  It has two arguments: the current
%    counter name (in our example, \texttt{Lem}), and the text of the
%    label (in our example, \texttt{Lemma}).
%    \begin{macrocode}
     \@thm{#1}{#2}}%
%    \end{macrocode}
%    With this, the `sub-definition' is complete.  The macro
%    |\@endtheorem| ends a theorem environment and is, so far, nothing
%    but an |\endtrivlist|.  (Hence it is defined globally, and not
%    within the theorem styles.\footnote{This has to be changed as
%    soon as theorem styles that change \texttt{\bslash @endtheorem}
%    exist.  In such a case, all existing styles must be changed as
%    well since they will have to reset the macro.}) Therefore, we can
%    set it equivalent to the macro that ends the theorem set (in our
%    example, |\endLem|).  However, if some day theorem styles exist
%    that do change |\@endtheorem|, we would have to use the
%    commented-out line instead.
%    \begin{macrocode}
    \global \expandafter \let \csname end#1\endcsname \@endtheorem
%   \global\@namedef{end#1}{\@endtheorem}%
%    \end{macrocode}
%    With these commands all the required definitions are employed,
%    unless the test |\@ifdefinable| has failed.  Therefore, we
%    end the second argument of this macro and with it the definition
%    of |\@xnthm|.
%    \begin{macrocode}
   }}
%    \end{macrocode}
% \end{macro}
%
% \begin{macro}{\@ynthm}
%    The definition of |\@ynthm| is completely analogous. In
%    this case the new counter that is defined is not reset within
%    another counter; thus the definition of |\the...|
%    is simplified:
%    \begin{macrocode}
\gdef\@ynthm#1#2{\expandafter\@ifdefinable\csname #1\endcsname
   {\@definecounter{#1}%
    \expandafter\xdef\csname the#1\endcsname{\@thmcounter{#1}}%
%    \end{macrocode}
%    The rest of the definition corresponds literally to that of
%    |\@xnthm|:
%    \begin{macrocode}
    \def\@tempa{\global\@namedef{#1}}\expandafter \@tempa
     \expandafter{\csname th@\the \theorem@style \expandafter
     \endcsname \the\theorem@bodyfont \@thm{#1}{#2}}%
    \global \expandafter \let \csname end#1\endcsname \@endtheorem}}
%    \end{macrocode}
% \end{macro}
%
% \begin{macro}{\@othm}
%    The definition of |\@othm| does not contain anything new.
%    \begin{macrocode}
\gdef\@othm#1[#2]#3{%
%    \end{macrocode}
% \changes{v2.2b}{95/10/07}{Build in check for counter defined}
%    We do not define a new counter but instead use one that has
%    already been defined.  Thus the only definition we need is that
%    of this pseudo-counter (i.e.\ |\the|\meta{env.\ name}).
%    First we check if |#2| corresponds to a known counter name.
%    \begin{macrocode}
  \expandafter\ifx\csname c@#2\endcsname\relax
   \@nocounterr{#2}%
  \else
   \expandafter\@ifdefinable\csname #1\endcsname
   {\expandafter \xdef \csname the#1\endcsname
     {\expandafter \noexpand \csname the#2\endcsname}%
%    \end{macrocode}
%    All other parts of the definition can be adopted from
%    |\@xnthm|. We have to remember, though, that in this case
%    the name of the current counter and the theorem label have moved
%    to the second and third arguments.
%    \begin{macrocode}
    \def\@tempa{\global\@namedef{#1}}\expandafter \@tempa
     \expandafter{\csname th@\the \theorem@style \expandafter
     \endcsname \the\theorem@bodyfont \@thm{#2}{#3}}%
    \global \expandafter \let \csname end#1\endcsname \@endtheorem}%
  \fi}
%    \end{macrocode}
% \end{macro}
%
% \subsection{Macros that are employed in a {\envfont theorem}
%    environment}
%
% \begin{macro}{\@thm}
%    The macro |\@thm| has to increase the current counter.  Then,
%    depending on whether the environment has (or does not have) an
%    optional argument, it has to branch into either |\@begintheorem|
%    or |\@opargtheorem|.
%    \begin{macrocode}
\gdef\@thm#1#2{\refstepcounter{#1}%
%    \end{macrocode}
%    Now we start a {\envfont trivlist} environment, and give
%    |\@topsep| and |\@topsepadd| the values of the \textsf{skip}
%    registers |\theorempreskipamount| and |\theorempostskipamount|.
%    The value in |\@topsep| is the vertical space that is inserted by
%    the first (and only) |\item| in our |\trivlist| whilst
%    |\@topsepadd| is inserted by |\@endparenv| at the end of that
%    {\envfont trivlist} environment.  By using these registers, we
%    obtain the desired space around a {\envfont theorem} environment.
%    \begin{macrocode}
   \trivlist
   \@topsep \theorempreskipamount               % used by first \item
   \@topsepadd \theorempostskipamount           % used by \@endparenv
%    \end{macrocode}
%    Now we have to test whether an optional argument has been given.
% \changes{v2.0d}{89/07/18}{Forgotten percent added.}
%    \begin{macrocode}
   \@ifnextchar [%
%    \end{macrocode}
%    If there is an optional argument, we will call |\@ythm|, and
%    move the arguments read back into the input stream.
%    \begin{macrocode}
   {\@ythm{#1}{#2}}%
%    \end{macrocode}
%    If not, we call |\@begintheorem|. Its first argument is the
%    name of the theorem set (hence the second argument of
%    |\@thm|). Its second argument is the macro that produces the
%    current number.
%    \begin{macrocode}
   {\@begintheorem{#2}{\csname the#1\endcsname}\ignorespaces}}
%    \end{macrocode}
% \end{macro}
%
% \begin{macro}{\@xthm}
% \begin{macro}{\@ythm}
%    Both these macros were originally called by |\@thm|.  We do
%    not need |\@xthm| anymore, hence we reset it to
%    |\relax|. The definition of |\@ythm| has not changed at
%    all from its definition in \LaTeX. In order to make the macros
%    easier to understand, we will
%    nevertheless present it (commented out).
%    \begin{macrocode}
\global\let\@xthm\relax
% \def\@ythm#1#2[#3]{\@opargbegintheorem{#2}{\csname
%           the#1\endcsname}{#3}\ignorespaces}
%</package>
%    \end{macrocode}
%    The primitive |\ignorespaces| in |\@ythm| and
%    |\@thm| is needed to remove the spaces between the
%    |\begin{...}| and the actual text.
% \end{macro}
% \end{macro}
%
%
% \subsection{Definition of the theorem substyles}
%
% As already pointed out, the theorem substyles, defined below, are
% only loaded when necessary.  Note that all these substyles, except
% \texttt{plain}, have |\slshape| as the default body font.
%
% \subsubsection{The \texttt{plain} style}
%
% \label{dok:th@plain} As the following macros use |@|, we have
% to locally set the |\catcode| of this symbol to ``letter''.
% This happens within a group, so that we do not have to worry about
% which |\catcode| that symbol had before.
%    \begin{macrocode}
%<*thp>
\begingroup \makeatletter
%    \end{macrocode}
% Since we are now within a group, we must make all definitions
% globally.  First we make sure that \texttt{theorem.sty} is loaded.
% This will allow us to use this file as a document style option
% without having to call \texttt{theorem} itself as an option.  At the
% same time, we assure that at least version 2 is loaded, since
% |\theorem@style| was not defined in earlier versions.
%    \begin{macrocode}
\@ifundefined{theorem@style}{% \iffalse meta-comment
%
% Copyright (C) 1993-2024
%
% The LaTeX Project and any individual authors listed elsewhere
% in this file.
%
% This file is part of the Standard LaTeX `Tools Bundle'.
% -------------------------------------------------------
%
% It may be distributed and/or modified under the
% conditions of the LaTeX Project Public License, either version 1.3c
% of this license or (at your option) any later version.
% The latest version of this license is in
%    https://www.latex-project.org/lppl.txt
% and version 1.3c or later is part of all distributions of LaTeX
% version 2005/12/01 or later.
%
% The list of all files belonging to the LaTeX `Tools Bundle' is
% given in the file `manifest.txt'.
%
% \fi
%
%%
%
%
\def\FMithmInfo{2023/07/05 v2.2c Theorem extension package (FMi)}
%
% \ProvidesFile{theorem.dtx}[\FMithmInfo]
%\iffalse   % this is a METACOMMENT !
%%
%% Package `theorem' to use with LaTeX2e
%% Copyright (C) 1989-1995 Frank Mittelbach, all rights reserved.
%<+package>\@ifundefined{theorem@style}{}{\endinput}
%<+package>\NeedsTeXFormat{LaTeX2e}[1995/06/01]
%<+package>\ProvidesPackage{theorem}
%<+package>         [\FMithmInfo]
%
% \fi
%
% \changes{v1.0}{87/05/07}{First draft.}
% \changes{v1.0a}{87/05/07}{`thistheoremstyle' added.}
% \changes{v1.1a}{87/07/29}{New concept: redefine the internal
%                          \protect\LaTeX{} commands `@xnthm' etc.}
% \changes{v1.1b}{87/07/31}{`theoremfont' used to specify font.}
% \changes{v1.2a}{88/01/08}{`@endtheorem' now globally defined.}
% \changes{v1.2b}{88/01/11}{`newtheorem' only allowed in the preamble.}
% \changes{v1.3a}{88/02/14}{`theoremfont' now a toks register.}
%
% \changes{v2.0a}{88/11/16}{New implementation and new documentation.}
% \changes{v2.0a}{88/11/16}{No layout changes but internal modifications
%                           old substyles won't work.}
% \changes{v2.0b}{89/01/30}{`global' used for all definitions.}
% \changes{v2.0c}{89/06/15}{Documentation translated to English.
%                            Thanks to LLL and Nina}
% \changes{v2.1a}{90/02/15}{Support for new font selection.}
% \changes{v2.2a}{94/02/03}{Cleanup for LaTeX2e}
%
% \DoNotIndex{\@endparenv,\@ifdefinable,\@ifnextchar,\@ifundefined}
% \DoNotIndex{\@namedef,\@nameuse,\@spaces,\@tempa,\@warning}
% \DoNotIndex{\ }
% \DoNotIndex{\begingroup,\bfseries}
% \DoNotIndex{\csname}
% \DoNotIndex{\def,\do,\docdate}
% \DoNotIndex{\endcsname,\endgroup,\endinput,\endtrivlist,\expandafter}
% \DoNotIndex{\filedate,\fileversion}
% \DoNotIndex{\gdef,\global}
% \DoNotIndex{\hbox,\hskip}
% \DoNotIndex{\ignorespaces,\input,\itshape,\item}
% \DoNotIndex{\labelsep,\let,\llap}
% \DoNotIndex{\makeatletter}
% \DoNotIndex{\noexpand}
% \DoNotIndex{\refstepcounter,\relax,\rlap}
% \DoNotIndex{\setlength,\slshape,\space,\string,\strut}
% \DoNotIndex{\the,\trivlist,\typeout}
% \DoNotIndex{\vbox}
% \DoNotIndex{\xdef}
%
% \def\deflabel#1{\ttfamily #1\hfill}
% \def\deflist#1{\begin{list}{}{\settowidth\labelwidth{\ttfamily #1}%
%                               \setlength\leftmargin\labelwidth
%                               \addtolength\leftmargin\labelsep
%                               \let\makelabel\deflabel}}
% \def\enddeflist{\end{list}}
% \def\envfont{\normalfont\ttfamily}
%
% \MakeShortVerb{\|}
%
% \newcommand{\TUB}{{\sl TUGboat\/}}
%
% \GetFileInfo{theorem.dtx}
% \title{An Extension of the
%        \LaTeX{} theorem environment\thanks{This file
%        has version number \fileversion, last
%        revised \filedate.}}
% \author{Frank Mittelbach}
%
% \MaintainedByLaTeXTeam{tools}
% \maketitle
%
% \begin{abstract}
%    The macros described in this paper yield an extension of the
%    \LaTeX{} theorem mechanism. It is designed to satisfy the
%    different requirements of various journals. Thus, the
%    layout of the ``theorems'' can be manipulated by determining a
%    ``style''. This article describes not only the use, but
%    also the definition, of the necessary macros.
% \end{abstract}
%
% \section*{Preface to version 2.2}
%
% For \LaTeXe{} this package did not need any fundamental changes. I
% only modified the messages generated so that theorem layout styles
% will show up with the |\listfiles| command and cleaned the section
% on the New Font Selection Scheme since this is now included in
% \LaTeX.
%
% \section*{Preface to version 2.1}
%
% This version is identical to 2.0g described in \TUB\ 10\#3 except
% for some internal defaults which are now set depending on the used
% font selection scheme.
%
% This was done to avoid unpleasant surprises if the new font selection
% scheme is in force. For further details see
% section~\ref{sec:newfonts} and \cite{bk:GMS94}.
%
% \section{Introduction}
%
% For our purposes here, ``theorems'' are labelled enunciations,
% often set off from the main text by extra space and a font change.
% Theorems, corollaries, conjectures, definitions, and remarks
% are all instances of ``theorems''.  The ``header'' of these structures
% is composed of a label (such as \textsc{Theorem} or \textsc{Remark})
% and a number which serializes an item in the sequence of items
% with the same label.
%
% Shortly after the introduction of \LaTeX{} at the Fachbereich
% Mathematik in Mainz, the desire to manipulate the layout of
% ``theorems'' arose.  In Mainz, the following two conventions came
% into general use:
% \begin{enumerate}
%   \item The number of the theorem is shown in the margin.
%   \item There is a line break at
%         the end of the theorem header.
% \end{enumerate}
%
% Additionally, some journals require different formats which depend on
% the ``sort of theorem'': e.g.\ often remarks and definitions are set
% in |\upshape|, while |\itshape| is employed for main theorems.
%
% Confronted with these requirements, a theorem environment was
% developed in Mainz which allows separate determination of the
% layout of the ``theorems sets'', comparable to |\pagestyle|.
%
%
% \section{The user interface}
% \subsection{Defining new theorem sets}
%
% \DescribeMacro\newtheorem
% As in the original \LaTeX{} version, the command |\newtheorem|
% defines a new ``theorem set'' or ``theorem-like structure''.
% Two required arguments name the new environment and give the text
% to be typeset with each instance of the new ``set'', while
% an optional argument determines how
% the ``set'' is enumerated:
% \begin{description}
%    \item[\ttfamily \bslash newtheorem\{foo\}\{bar\}]
%       The theorem set {\envfont foo} (whose name is \texttt{bar})
%       uses its own counter.
%    \item[\ttfamily \bslash newtheorem\{foo2\}{[foo]}\{bar2\}]
%       The theorem set {\envfont foo2} (printed name \texttt{bar2})
%       uses the same counter as the theorem set \texttt{foo}.
%    \item[\ttfamily \bslash newtheorem\{foo3\}\{bar3\}{[section]}]
%       The theorem set {\envfont foo3} (printed name \texttt{bar3}) is
%       enumerated within the counter \texttt{section}, i.e.\ with every
%       new |\section| the enumeration begins again with 1, and
%       the enumeration is composed from the section-number and the
%       theorem counter itself.
% \end{description}
%
% \DescribeMacro\theoremstyle
% Additionally, the command |\theoremstyle| can define the layout
% of various, or all, theorem sets.  It should be noted that any
% theorem set defined by |\newtheorem| is typeset in the
% |\theoremstyle| that is current at the time of the definition.
% Thus, the following
% \begin{verbatim}
%    \theoremstyle{break}        \newtheorem{Cor}{Corollary}
%    \theoremstyle{plain}        \newtheorem{Exa}{Example}[section]
%\end{verbatim}                             ^^A No blank at line start
% leads to the result that the set {\envfont Cor} is formatted in the
% style \texttt{break}, while the set {\envfont Exa} and all the
% following ones are formatted in the style \texttt{plain}, unless
% another |\theoremstyle| follows.  Since the definitions installed by
% |\newtheorem| are global, one also can limit |\theoremstyle| locally
% by grouping braces.
%
% \DescribeMacro\theorembodyfont
% The choice of the font for the theorem body is completely
% independent of the chosen |\theoremstyle|; this has proven to
% be very advantageous.  For example,
% \begin{verbatim}
%   {\theorembodyfont{\upshape}          \newtheorem{Rem}{Remark}}
%\end{verbatim}
% defines a theorem set {\envfont Rem}, which will be set in
% |\upshape| in the current layout (which in our example is
% \texttt{plain}).  As with |\theoremstyle|, the |\theorembodyfont|
% chosen is that current at the time of |\newtheorem|.  If
% |\theorembodyfont| is not specified or one defines
% |\theorembodyfont{}|, then the font used will be that defined by the
% |\theoremstyle|.
%
% \DescribeMacro\theoremheaderfont It is also possible to customize
% the font used for the theorem headers.  This is, however, a global
% declaration, and therefore there should be at most one
% |\theoremheaderfont| declaration in the preamble.\footnote{If it is
% actually necessary to have different header fonts, one has to define
% new theorem styles (substituting the desired font) or specify the
% information directly in the \texttt{\bslash newtheorem} declaration
% (the unclean variant).}
%
% \DescribeMacro\theorempreskipamount
% \DescribeMacro\theorempostskipamount Two additional parameters
% affect the vertical space around the theorem environments:
% |\theorempreskipamount| and |\theorempostskipamount| define,
% respectively, the spacing before and after such an environment.
% These parameters apply for all theorem sets and can be manipulated
% with the ordinary length macros.  They are rubber lengths
% (`\textsf{skips}'), and therefore can contain \texttt{plus} and
% \texttt{minus} parts.
%
% Since the definition of theorem sets should---most sensibly---be
% placed in the preamble, we only allow installation there.
% It is therefore possible to release the memory used here after
% |\begin{document}|, in order to make room for other
% applications.
%
%
% \subsection{Existing theorem styles}
%
% The following theorem styles exist to date:
% \begin{deflist}{marginbreak}
%    \item[plain]
%       This theorem style emulates the original \LaTeX{} definition,
%       except that additionally the parameters
%       |\theorem...skipamount| are used.
%    \item[break]
%       In this style, the theorem header is followed by a line
%       break.
%    \item[marginbreak]
%       The theorem number is set in the margin, and there is a line
%       break as in \texttt{break}.
%    \item[changebreak]
%       Like \texttt{break}, but with header number and text
%       interchanged.
%    \item[change]
%       Header number and text are interchanged, without a line break.
%    \item[margin]
%       The number is set in the left margin, without a line break.
% \end{deflist}
% All styles (except \texttt{plain}) select |\slshape| as the default
% |\theorembodyfont|.
%
%
% \subsection{Examples}
%
% Given the above theorem sets {\envfont Cor}, {\envfont Exa}
% and {\envfont Rem},
% suppose that the preamble also contains the declarations:
% \begin{verbatim}
%    \theoremstyle{marginbreak}   \newtheorem{Lem}[Cor]{Lemma}
%    \theoremstyle{change}
%    \theorembodyfont{\itshape}        \newtheorem{Def}[Cor]{Definition}
%
%    \theoremheaderfont{\scshape}
%\end{verbatim}
% Then the following are some typical examples of the typeset output
% resulting from their use.
%
% \begin{Cor}
%    This is a sentence typeset in the theorem environment {\envfont
%    Cor}.
% \end{Cor}
%
% \begin{Exa}
%    This is a sentence typeset in the theorem environment {\envfont
%    Exa}.
% \end{Exa}
%
% \begin{Rem}
%    This is a sentence typeset in the theorem environment {\envfont
%    Rem}.
% \end{Rem}
%
% \begin{Lem}[Ben User]
%    This is a sentence typeset in the theorem environment {\envfont
%    Lem}.
% \end{Lem}
%
% \begin{Def}[Very impressive Definition]
%    This is a sentence typeset in the theorem environment {\envfont
%    Def}.
% \end{Def}
%
% The last two examples show the effect of the optional argument to
% a theorem environment (it is the text typeset in parentheses).
%
%
% \section{Special Considerations}
% \label{sec:newfonts}
%
% Theoremheader and body are implemented as a unit. This means that
% the |\theoremheaderfont|  will inherit characteristics of the
% |\theorembodyfont| in \LaTeXe. Thus, if for example
% |\theorembodyfont| is |\itshape| and |\theoremheaderfont|
% is |\bfseries| the font selected for the header will have the
% characteristics `bold extended italic'. If this is not desired one
% should set the |\theoremheaderfont| to something like
% \begin{verbatim}
%      \theoremheaderfont{\normalfont\bfseries}
%\end{verbatim}
% i.e.\ supplying all necessary font information explicitly.
%
%
%
% \section{Acknowledgements}
%
% The publication of this set of macros was only possible with the
% help of Christina Busse (translating the manuscript into English),
% Joachim Pense (playing the r\^ole of typist), Chris Rowley
% (looking everything over) and many others providing useful
% suggestions.
%
% \MaybeStop{%
%    \begin{thebibliography}{9}
%       \bibitem{bk:GMS94} \textsc{M.~Goossens}, \textsc{F.~Mittelbach}
%         and \textsc{A.~Samarin}.
%         \newblock The \LaTeX{} Companion.
%         \newblock
%         Addison-Wesley, Reading, Massachusetts, 1994.
%       \bibitem{latex:source} \textsc{Lamport, Leslie.} \newblock
%                               \texttt{latex.tex}, version 2.09, date
%                               Feb. 1990.
%    \end{thebibliography}
%    \PrintIndex
%    \PrintChanges}
%
%
% \section{The documentation driver file}
%
% The next bit of code contains the documentation driver file for
% \TeX{}, i.e., the file that will produce the documentation you are
% currently reading. It will be extracted from this file by the
% \texttt{docstrip} program. Since it is the first code in the file
% one can alternatively process this file directly with \LaTeXe{} to
% obtain the documentation.
%
%    \begin{macrocode}
%<*driver>
\documentclass{ltxdoc}

\usepackage{theorem}

%<+driver>% The next few lines define theorem sets which are used
%<+driver>% in the example section of the documentation.

\theoremstyle{break}       \newtheorem{Cor}{Corollary}
\theoremstyle{plain}       \newtheorem{Exa}{Example}[section]
{\theorembodyfont{\upshape}\newtheorem{Rem}{Remark}}
\theoremstyle{marginbreak} \newtheorem{Lem}[Cor]{Lemma}
\theoremstyle{change}
\theorembodyfont{\itshape} \newtheorem{Def}[Cor]{Definition}

\theoremheaderfont{\scshape}

\RecordChanges

\begin{document}
   \DocInput{theorem.dtx}
\end{document}
%</driver>
%    \end{macrocode}
%
% \section{Definition of the Macros}
%
% If the file has been loaded before, we abort immediately. If not the
% package announces itself (this is actually done at the very top of
% the file---the way it is done isn't good style so don't copy it).
% \changes{v2.0e}{89/07/19}{Spaces removed from `typeout'.}
% \changes{v2.2b}{95/11/19}{Announce moved to top of file}
%    \begin{macrocode}
%<*package>
%\@ifundefined{theorem@style}{}{\endinput}
%\def\FMithmInfo{1995/11/19 v2.2b Theorem extension package (FMi)}
%\ProvidesPackage{theorem}[\FMithmInfo]
%    \end{macrocode}
%
%
% \subsection{Definition of theorem styles and fonts}
%
% All the definitions in this file are done globally to allow
% inputting this file inside a group.
%
% \begin{macro}{\theoremstyle}
%    Before a theorem style can be installed, the chosen style must be
%    known. For that reason, we must test to see that
%    |\th@|\meta{style} is known or, more precisely, that it is
%    different from |\relax|. If the style is not known then
%    |\th@plain| is used.
%    \begin{macrocode}
\gdef\theoremstyle#1{%
   \@ifundefined{th@#1}{\@warning
          {Unknown theoremstyle `#1'. Using `plain'}%
          \theorem@style{plain}}%
%    \end{macrocode}
%    We save the theorem style to be used in the token register
%    |\theorem@style|.
%    \begin{macrocode}
      {\theorem@style{#1}}%
%    \end{macrocode}
%    Now we ``evaluate'' the theorem style: this means, we call the
%    macro |\th@|\meta{style} which will activate the relevant
%    definitions which are contained in a separate file.  This is done
%    in a group to suppress changes to the current font.  This could
%    otherwise pose problems together with the new font selection
%    scheme\footnote{When I printed the original article using the new
%    font selection scheme I ended with a document with slanted
%    typefaces (text headings and all) simply because one of the
%    theorem styles used \texttt{\bslash sl} at toplevel.} if the
%    |\th@|\meta{style} is evaluated a second time.
% \changes{v2.1a}{90/02/15}{Group added around `th@..}
% \changes{v2.1a}{90/02/15}{Reseting of `..@begintheorem removed.}
%    \begin{macrocode}
      \begingroup
        \csname th@\the\theorem@style \endcsname
      \endgroup}
%    \end{macrocode}
% \end{macro}
%
%
% \begin{macro}{\@begintheorem}
% \changes{v2.1a}{90/02/15}{Relaxed at toplevel.}
% \begin{macro}{\@opargbegintheorem}
% \changes{v2.1a}{90/02/15}{Relaxed at toplevel.}
%    We reset |\@begintheorem| and |\@opargbegintheorem| to
%    |\relax| since these commands are no longer necessary at
%    toplevel. This will save a few tokens.
%    \begin{macrocode}
\global\let\@begintheorem\relax
\global\let\@opargbegintheorem\relax
%    \end{macrocode}
% \end{macro}
% \end{macro}
%
% \begin{macro}{\theorem@style}
%    Obviously the \textsf{token} register used above has to be
%    allocated. To assure the utmost compatibility with the
%    original \LaTeX{} definition, we set the default theorem style to
%    \texttt{plain}, which implements the usual \LaTeX{} convention.
% \changes{v2.0f}{89/07/20}{Unnecessary `global' removed.}
%    \begin{macrocode}
\newtoks\theorem@style
\global\theorem@style{plain}
%    \end{macrocode}
% \end{macro}
%
% \begin{macro}{\theorembodyfont}
% \begin{macro}{\theorem@bodyfont}
%    If the |\theorembodyfont| is set by the user then it should not
%    interact with the default font set in the theorem style. When the
%    new font selection is in force this may happen if, for example,
%    the default is |\itshape| and the new |\theorembodyfont| is
%    |\sffamily|. So we add a |\reset@font| command in front of the
%    user definition.
% \changes{v2.0f}{89/07/20}{Unnecessary `global' removed.}
% \changes{v2.0g}{89/09/19}{`theoremfont to `theorembodyfont.}
% \changes{v2.1c}{92/06/22}{`reset@font added and logic changed.}
% \changes{v2.1d}{92/10/28}{Reset to empty if arg is empty}
%    \begin{macrocode}
\gdef\theorembodyfont#1{%
%    \end{macrocode}
%    We check if the argument supplied is empty and if so put nothing
%    into the |\theorem@bodyfont| token register to allow for
%    |\theorembodyfont{}| as a mean of using the default of the
%    current |\theoremstyle|.
%    \begin{macrocode}
   \def\@tempa{#1}%
   \ifx\@tempa\@empty
    \theorem@bodyfont{}%
   \else
    \theorem@bodyfont{\reset@font#1}%
   \fi
   }
\newtoks\theorem@bodyfont
\global\theorem@bodyfont{}
%    \end{macrocode}
% \end{macro}
% \end{macro}
%
%
% \begin{macro}{\theoremheaderfont}
% \changes{v2.0g}{89/09/19}{Macro added.}
%    The font for the theorem headers is handled differently because
%    this definition applies to all theorem styles.
%    \begin{macrocode}
\gdef\theoremheaderfont#1{\gdef\theorem@headerfont{#1}%
%    \end{macrocode}
%    After using the macro once it is redefined to produce an error
%    message.
%    \begin{macrocode}
       \gdef\theoremheaderfont##1{%
        \typeout{\string\theoremheaderfont\space should be used
                 only once.}}}
%    \end{macrocode}
% \end{macro}
%
% \begin{macro}{\theorem@headerfont}
% \changes{v2.0g}{89/09/19}{Macro added.}
%    To set the |\theorem@headerfont| default we first test if
%    the new fontselection scheme is in force.
% \changes{v2.1a}{90/02/15}{Support for new font selection.}
%    \begin{macrocode}
\ifx\upshape\undefined
%    \end{macrocode}
%    If not we define it to expand into |\bfseries|. We don't use
%    |\let| just in case a following style option redefines this
%    macro.
%    \begin{macrocode}
\gdef\theorem@headerfont{\bfseries}
%    \end{macrocode}
%    Otherwise we reset the current shape before calling |\bfseries|.
%    \begin{macrocode}
\else \gdef\theorem@headerfont{\normalfont\bfseries}\fi
%    \end{macrocode}
% \end{macro}
%
% \begin{macro}{\th@plain}
% \begin{macro}{\th@break}
% \begin{macro}{\th@marginbreak}
% \begin{macro}{\th@changebreak}
% \begin{macro}{\th@change}
% \begin{macro}{\th@margin}
%    The different styles are defined in macros such as
%    |\th@plain|.  Since memory space is precious in
%    ``non-Big-versions'', we have to avoid offering too many unused
%    definitions.  Therefore we define these styles in separate files
%    that can be loaded on demand.  Thus the commands themselves only
%    load these files. We use |\@input@|, a \LaTeXe{} internal command
%    that ensures that the file will be listed with |\listfiles|.
% \changes{v2.2a}{94/02/02}{Use `@input@ to load theorem layout files}
%    \begin{macrocode}
\gdef\th@plain{\@input@{thp.sty}}
\gdef\th@break{\@input@{thb.sty}}
\gdef\th@marginbreak{\@input@{thmb.sty}}
\gdef\th@changebreak{\@input@{thcb.sty}}
\gdef\th@change{\@input@{thc.sty}}
\gdef\th@margin{\@input@{thm.sty}}
%    \end{macrocode}
%    This list will be expanded when new styles become
%    available.  For testing,  just append new theorem substyles
%    as document options.
% \end{macro}
% \end{macro}
% \end{macro}
% \end{macro}
% \end{macro}
% \end{macro}
%
%
%
% \subsection{Definition of a new theorem set}
%
% As already pointed out, a new theorem environment can be defined in
% three different ways:
% \begin{verbatim}
%    \newtheorem{Lem}{Lemma}
%    \newtheorem{Lem}{Lemma}[section]
%    \newtheorem{Lem}[Theorem]{Lemma}
%\end{verbatim}
% The function of the macro |\newtheorem| is to recognize these cases
% and then to branch into one of the three macros |\@ynthm|, |\@xnthm|
% or |\@othm|.  This mechanism is adopted unchanged from
% \cite{latex:source}; the essential point here is that, for example,
% in the second case, the arguments \texttt{Lem}, \texttt{Lemma} and
% \texttt{section} are passed over to the macro |\@xnthm|.
%
% We inspect this case first because the others present fewer problems,
% and thus are easily derived from this one.
%
% \begin{macro}{\@xnthm}
%    For our example arguments, the macro |\@xnthm| must fulfill
%    the following:
%    \begin{itemize}
%       \item
%          Define a new \LaTeX-counter `Lem'
%       \item
%          reset this counter within a |\section|
%       \item
%          define the macro |\theLem|
%       \item
%          define the environment macros |\Lem| and
%          |\endLem| using the current
%          |\theoremstyle| and |\theorem@bodyfont|.
%    \end{itemize}
%    Obviously, all this should happen only if the first argument
%    of |\@xnthm| (i.e.\ \texttt{Lem} in our example) is chosen so
%    as not to conflict with any previously defined commands or
%    environments.  This test is performed by the \LaTeX{} macro
%    |\@ifdefinable|.
%    \begin{macrocode}
\gdef\@xnthm#1#2[#3]{\expandafter\@ifdefinable\csname #1\endcsname
%    \end{macrocode}
%    Therefore, the first argument of |\@ifdefinable| is the
%    expansion (in the example, |\Lem|) of
%    |\csname||#1\endcsname|.  The second argument is
%    executed only if the test has been completed successfully.
%    \begin{macrocode}
   {%
%    \end{macrocode}
%    Now we define the new counter. The names of the \LaTeX{}
%    macros employed should speak for themselves:
% \changes{v2.2b}{1995/10/07}{Use \cs{@newctr}}
% \changes{v2.2c}{1995/11/23}{Use \cs{@newctr} properly.}
%    \begin{macrocode}
    \@definecounter{#1}\@newctr{#1}[#3]%
%    \end{macrocode}
%    Using |\@newctr| will give a proper error message if the counter
%    in |#3| is not defined.
%    In defining `|\theLem|' we must generate the desired macro
%    name by use of |\expandafter| and |\csname|.
%    \begin{macrocode}
    \expandafter\xdef\csname the#1\endcsname
%    \end{macrocode}
%    An |\xdef| is used in order to make the definition global,
%    and to ensure that it contains the replacement texts of
%    |\@thmcountersep| and |\@thmcounter|.\footnote{These
%    two macros can be defined by the document style.  Their default
%    values produce a `\texttt{.}' as separation and an arabic
%    representation of the number.} However, not everything should be
%    expanded. For example, it saves space to use |\thesection|
%    instead of its---at times---lengthy expansion.
%    \begin{macrocode}
      {\expandafter \noexpand \csname the#3\endcsname
       \@thmcountersep \@thmcounter{#1}}%
%    \end{macrocode}
%    Thus with the defaults of \LaTeX{}, |\theLem| would be
%    replaced by the command sequence |\thesection.\arabic{Lem}|.
%
%    We will now look at the definition of the macro which is executed
%    at the beginning of the actual environment (in our example this
%    macro is |\Lem|). It should be noted that we use an
%    ``|\expandafter| trick'' to expand only certain parts of the
%    replacement text at the time of the definition.
%    \begin{macrocode}
    \def\@tempa{\global\@namedef{#1}}%
    \expandafter \@tempa \expandafter{%
%    \end{macrocode}
%    First, the macro that contains the current definitions of
%    |\@begintheorem| and |\@opargtheorem| should be called
%    up.  The name of this macro---as is already known---has the form
%    |\th@|\meta{theorem\ style}; therefore, it must be called by
%    \begin{macrocode}
      \csname th@\the \theorem@style
%    \end{macrocode}
%    In addition the default theorem font should be changeable, i.e.\
%    we have to insert the contents of |\theorem@bodyfont|. For
%    that reason, we expand even further, beyond |\endcsname|,
%    and thus insert the contents of the \textsf{token} register
%    |\theorem@bodyfont| in the replacement text.
%    \begin{macrocode}
            \expandafter \endcsname \the \theorem@bodyfont
%    \end{macrocode}
%    Now it is time to call the macro |\@thm| which takes over
%    the further processing.  It has two arguments: the current
%    counter name (in our example, \texttt{Lem}), and the text of the
%    label (in our example, \texttt{Lemma}).
%    \begin{macrocode}
     \@thm{#1}{#2}}%
%    \end{macrocode}
%    With this, the `sub-definition' is complete.  The macro
%    |\@endtheorem| ends a theorem environment and is, so far, nothing
%    but an |\endtrivlist|.  (Hence it is defined globally, and not
%    within the theorem styles.\footnote{This has to be changed as
%    soon as theorem styles that change \texttt{\bslash @endtheorem}
%    exist.  In such a case, all existing styles must be changed as
%    well since they will have to reset the macro.}) Therefore, we can
%    set it equivalent to the macro that ends the theorem set (in our
%    example, |\endLem|).  However, if some day theorem styles exist
%    that do change |\@endtheorem|, we would have to use the
%    commented-out line instead.
%    \begin{macrocode}
    \global \expandafter \let \csname end#1\endcsname \@endtheorem
%   \global\@namedef{end#1}{\@endtheorem}%
%    \end{macrocode}
%    With these commands all the required definitions are employed,
%    unless the test |\@ifdefinable| has failed.  Therefore, we
%    end the second argument of this macro and with it the definition
%    of |\@xnthm|.
%    \begin{macrocode}
   }}
%    \end{macrocode}
% \end{macro}
%
% \begin{macro}{\@ynthm}
%    The definition of |\@ynthm| is completely analogous. In
%    this case the new counter that is defined is not reset within
%    another counter; thus the definition of |\the...|
%    is simplified:
%    \begin{macrocode}
\gdef\@ynthm#1#2{\expandafter\@ifdefinable\csname #1\endcsname
   {\@definecounter{#1}%
    \expandafter\xdef\csname the#1\endcsname{\@thmcounter{#1}}%
%    \end{macrocode}
%    The rest of the definition corresponds literally to that of
%    |\@xnthm|:
%    \begin{macrocode}
    \def\@tempa{\global\@namedef{#1}}\expandafter \@tempa
     \expandafter{\csname th@\the \theorem@style \expandafter
     \endcsname \the\theorem@bodyfont \@thm{#1}{#2}}%
    \global \expandafter \let \csname end#1\endcsname \@endtheorem}}
%    \end{macrocode}
% \end{macro}
%
% \begin{macro}{\@othm}
%    The definition of |\@othm| does not contain anything new.
%    \begin{macrocode}
\gdef\@othm#1[#2]#3{%
%    \end{macrocode}
% \changes{v2.2b}{95/10/07}{Build in check for counter defined}
%    We do not define a new counter but instead use one that has
%    already been defined.  Thus the only definition we need is that
%    of this pseudo-counter (i.e.\ |\the|\meta{env.\ name}).
%    First we check if |#2| corresponds to a known counter name.
%    \begin{macrocode}
  \expandafter\ifx\csname c@#2\endcsname\relax
   \@nocounterr{#2}%
  \else
   \expandafter\@ifdefinable\csname #1\endcsname
   {\expandafter \xdef \csname the#1\endcsname
     {\expandafter \noexpand \csname the#2\endcsname}%
%    \end{macrocode}
%    All other parts of the definition can be adopted from
%    |\@xnthm|. We have to remember, though, that in this case
%    the name of the current counter and the theorem label have moved
%    to the second and third arguments.
%    \begin{macrocode}
    \def\@tempa{\global\@namedef{#1}}\expandafter \@tempa
     \expandafter{\csname th@\the \theorem@style \expandafter
     \endcsname \the\theorem@bodyfont \@thm{#2}{#3}}%
    \global \expandafter \let \csname end#1\endcsname \@endtheorem}%
  \fi}
%    \end{macrocode}
% \end{macro}
%
% \subsection{Macros that are employed in a {\envfont theorem}
%    environment}
%
% \begin{macro}{\@thm}
%    The macro |\@thm| has to increase the current counter.  Then,
%    depending on whether the environment has (or does not have) an
%    optional argument, it has to branch into either |\@begintheorem|
%    or |\@opargtheorem|.
%    \begin{macrocode}
\gdef\@thm#1#2{\refstepcounter{#1}%
%    \end{macrocode}
%    Now we start a {\envfont trivlist} environment, and give
%    |\@topsep| and |\@topsepadd| the values of the \textsf{skip}
%    registers |\theorempreskipamount| and |\theorempostskipamount|.
%    The value in |\@topsep| is the vertical space that is inserted by
%    the first (and only) |\item| in our |\trivlist| whilst
%    |\@topsepadd| is inserted by |\@endparenv| at the end of that
%    {\envfont trivlist} environment.  By using these registers, we
%    obtain the desired space around a {\envfont theorem} environment.
%    \begin{macrocode}
   \trivlist
   \@topsep \theorempreskipamount               % used by first \item
   \@topsepadd \theorempostskipamount           % used by \@endparenv
%    \end{macrocode}
%    Now we have to test whether an optional argument has been given.
% \changes{v2.0d}{89/07/18}{Forgotten percent added.}
%    \begin{macrocode}
   \@ifnextchar [%
%    \end{macrocode}
%    If there is an optional argument, we will call |\@ythm|, and
%    move the arguments read back into the input stream.
%    \begin{macrocode}
   {\@ythm{#1}{#2}}%
%    \end{macrocode}
%    If not, we call |\@begintheorem|. Its first argument is the
%    name of the theorem set (hence the second argument of
%    |\@thm|). Its second argument is the macro that produces the
%    current number.
%    \begin{macrocode}
   {\@begintheorem{#2}{\csname the#1\endcsname}\ignorespaces}}
%    \end{macrocode}
% \end{macro}
%
% \begin{macro}{\@xthm}
% \begin{macro}{\@ythm}
%    Both these macros were originally called by |\@thm|.  We do
%    not need |\@xthm| anymore, hence we reset it to
%    |\relax|. The definition of |\@ythm| has not changed at
%    all from its definition in \LaTeX. In order to make the macros
%    easier to understand, we will
%    nevertheless present it (commented out).
%    \begin{macrocode}
\global\let\@xthm\relax
% \def\@ythm#1#2[#3]{\@opargbegintheorem{#2}{\csname
%           the#1\endcsname}{#3}\ignorespaces}
%</package>
%    \end{macrocode}
%    The primitive |\ignorespaces| in |\@ythm| and
%    |\@thm| is needed to remove the spaces between the
%    |\begin{...}| and the actual text.
% \end{macro}
% \end{macro}
%
%
% \subsection{Definition of the theorem substyles}
%
% As already pointed out, the theorem substyles, defined below, are
% only loaded when necessary.  Note that all these substyles, except
% \texttt{plain}, have |\slshape| as the default body font.
%
% \subsubsection{The \texttt{plain} style}
%
% \label{dok:th@plain} As the following macros use |@|, we have
% to locally set the |\catcode| of this symbol to ``letter''.
% This happens within a group, so that we do not have to worry about
% which |\catcode| that symbol had before.
%    \begin{macrocode}
%<*thp>
\begingroup \makeatletter
%    \end{macrocode}
% Since we are now within a group, we must make all definitions
% globally.  First we make sure that \texttt{theorem.sty} is loaded.
% This will allow us to use this file as a document style option
% without having to call \texttt{theorem} itself as an option.  At the
% same time, we assure that at least version 2 is loaded, since
% |\theorem@style| was not defined in earlier versions.
%    \begin{macrocode}
\@ifundefined{theorem@style}{\input{theorem.sty}}{}
\ProvidesFile{thp.sty}
             [\FMithmInfo]
%    \end{macrocode}
%
%
% \begin{macro}{\th@plain}
% \begin{macro}{\@begintheorem}
% \begin{macro}{\@opargbegintheorem}
%    |\theoremstyle{plain}| corresponds to the original
%    definition, except that the distances to the surrounding text are
%    determined by the parameters |\theorempreskipamount| and
%    |\theorempostskipamount|.
%    First we set the default body font.
%    \begin{macrocode}
\gdef\th@plain{\normalfont\itshape
%    \end{macrocode}
%    Then we define |\@begintheorem| and
%    |\@opargbegintheorem|. These two macros define how the
%    header of a theorem is typeset.  |\@opargbegintheorem| will
%    be called if a \textsf{theorem} environment with an optional
%    argument is encountered; otherwise, the header is constructed by
%    calling |\@begintheorem|.  If one of these macros is
%    executed, we are within a \textsf{trivlist} environment started by
%    |\@thm|.  So the theorem header is produced with an
%    |\item| command.
%
%    Instead of specifying the header font directly, all standard
%    theorem styles use the |\theorem@headerfont| macro to allow
%    customization.  The extra space (|\labelsep|) is necessary
%    because of problems in the \textsf{trivlist} environment.
%    \begin{macrocode}
  \def\@begintheorem##1##2{%
        \item[\hskip\labelsep \theorem@headerfont ##1\ ##2]}%
%    \end{macrocode}
%    The definition of |\@opargbegintheorem| is completely
%    analogous. The only difference is the fact that there exists a
%    third argument (which is the optional parameter of the
%    environment and contains additional information about the
%    theorem).  Customarily we enclose it in parentheses.
%    \begin{macrocode}
\def\@opargbegintheorem##1##2##3{%
   \item[\hskip\labelsep \theorem@headerfont ##1\ ##2\ (##3)]}}
%    \end{macrocode}
% \end{macro}
% \end{macro}
% \end{macro}
%
% \noindent
% We conclude with an |\endgroup| to restore the
% |\catcode| of |@|.
%    \begin{macrocode}
\endgroup
%</thp>
%    \end{macrocode}
%
%
% \subsubsection{The \texttt{break} style}
%
% This style option is stored in the file |thb.sty|.
% For the next two lines see the documentation for |\th@plain| on
% page~\pageref{dok:th@plain}.
%    \begin{macrocode}
%<*thb>
\begingroup \makeatletter
\@ifundefined{theorem@style}{\input{theorem.sty}}{}
\ProvidesFile{thb.sty}
             [\FMithmInfo]
%    \end{macrocode}
%
% \begin{macro}{\th@break}
%    |\theoremstyle{break}| produces a line break after the name
%    of the theorem.  The font is |\slshape|.  Hence, we define
%    |\th@break| as follows:
%    \begin{macrocode}
\gdef\th@break{\normalfont\slshape
  \def\@begintheorem##1##2{\item[%
%    \end{macrocode}
%    We run into the following problem: it is not possible to create
%    the header with |\item[|\meta{title}|]| and then start
%    a new line by, for example, |\mbox{}\\|. Such a definition
%    will fail whenever a list environment follows immediately.  With
%    the above construction, the |\mbox{}| causes the switch
%    |@inlabel| (cf.\ definition of |\list| and
%    |\trivlist| in \cite{latex:source}) to be set to \texttt{false}
%    and so the following list will insert additional vertical
%    space (|\topskip|). This is quite annoying. Therefore, we
%    create the line break within the |\item|. In order to ensure
%    that the text will begin at the proper position in the following
%    line, we simply pretend that the label does not take any
%    room.\footnote{This will lead to problems whenever very high
%    symbols occurring in the line tower into the
%    heading. So, something else has to be done here sometime.}
%    \begin{macrocode}
       \rlap{\vbox{\hbox{\hskip \labelsep\theorem@headerfont ##1\ ##2}%
                   \hbox{\strut}}}]}%
%    \end{macrocode}
%    Again, the definition of |\@opargbegintheorem| is
%    completely analogous.
%    \begin{macrocode}
\def\@opargbegintheorem##1##2##3{%
  \item[\rlap{\vbox{\hbox{\hskip \labelsep \theorem@headerfont
                     ##1\ ##2\ (##3)}%
                    \hbox{\strut}}}]}}
\endgroup
%</thb>
%    \end{macrocode}
% \end{macro}
%
%
% \subsubsection{The \texttt{changebreak} style}
%
%    \begin{macrocode}
% This style option is stored in the file |thcb.sty|.
%    \begin{macrocode}
%<*thcb>
\begingroup \makeatletter
\@ifundefined{theorem@style}{\input{theorem.sty}}{}
\ProvidesFile{thcb.sty}
             [\FMithmInfo]
%    \end{macrocode}
%
% \begin{macro}{\th@changebreak}
%    The \texttt{change-break} theorem style is like \texttt{break}
%    but with interchange of theorem name and theorem number.  Thus we
%    define |\th@changebreak| as follows:
%    \begin{macrocode}
\gdef\th@changebreak{\normalfont\slshape
  \def\@begintheorem##1##2{\item
       [\rlap{\vbox{\hbox{\hskip\labelsep \theorem@headerfont ##2\ ##1}%
                    \hbox{\strut}}}]}%
\def\@opargbegintheorem##1##2##3{%
  \item[\rlap{\vbox{\hbox{\hskip\labelsep \theorem@headerfont
                     ##2\ ##1\ (##3)}%
                    \hbox{\strut}}}]}}
\endgroup
%</thcb>
%    \end{macrocode}
% \end{macro}
%
%
% \subsubsection{The \texttt{change} style}
%
% This style option is stored in the file |thc.sty|.
%    \begin{macrocode}
%<*thc>
\begingroup \makeatletter
\@ifundefined{theorem@style}{\input{theorem.sty}}{}
\ProvidesFile{thc.sty}
             [\FMithmInfo]
%    \end{macrocode}
%
% \begin{macro}{\th@change}
%    The \texttt{change} theorem style corresponds to the
%    \texttt{change break} style without a linebreak after the header.
%    To say it in another way, it's the same as the \texttt{plain}
%    style but with number and name interchanged and |\slshape| as the
%    default font.
%    \begin{macrocode}
\gdef\th@change{\normalfont\slshape
  \def\@begintheorem##1##2{\item
       [\hskip\labelsep \theorem@headerfont ##2\ ##1]}%
\def\@opargbegintheorem##1##2##3{%
  \item[\hskip\labelsep \theorem@headerfont ##2\ ##1\ (##3)]}}
\endgroup
%</thc>
%    \end{macrocode}
% \end{macro}
%
%
% \subsubsection{The \texttt{marginbreak} style}
%
% This style option is the one used most often at Mainz.  It is saved
% in the file |thmb.sty|.
%    \begin{macrocode}
%<*thmb>
\begingroup \makeatletter
\@ifundefined{theorem@style}{\input{theorem.sty}}{}
\ProvidesFile{thmb.sty}
             [\FMithmInfo]
%    \end{macrocode}
%
% \begin{macro}{\th@marginbreak}
%    The \texttt{margin break} style is nearly the same as the
%    \texttt{change break} style. The only difference is the placement
%    of the theorem number.  We use |\llap| to place it in the left
%    margin.
%
%    In this style |\labelsep| denotes the separation between the
%    number and the text.
%    \begin{macrocode}
\gdef\th@marginbreak{\normalfont\slshape
  \def\@begintheorem##1##2{\item
       [\rlap{\vbox{\theorem@headerfont
                    \hbox{\llap{##2}\hskip\labelsep ##1}%
                    \hbox{\strut}}}]}%
\def\@opargbegintheorem##1##2##3{%
  \item[\rlap{\vbox{\theorem@headerfont
                    \hbox{\llap{##2}\hskip\labelsep ##1\ (##3)}%
                    \hbox{\strut}}}]}}
\endgroup
%</thmb>
%    \end{macrocode}
% \changes{v2.1c}{90/02/17}{`endgroup outside macrocode env.}
% \end{macro}
%
%
%
% \subsubsection{The \texttt{margin} style}
%
% This style option is stored in the file |thm.sty|.
%    \begin{macrocode}
%<*thm>
\begingroup \makeatletter
\@ifundefined{theorem@style}{\input{theorem.sty}}{}
\ProvidesFile{thm.sty}
             [\FMithmInfo]
%    \end{macrocode}
%
% \begin{macro}{\th@margin}
%    Again this is only a variant of the theorem styles described
%    above without any new ideas.
%    \begin{macrocode}
\gdef\th@margin{\normalfont\slshape
  \def\@begintheorem##1##2{\item
       [\theorem@headerfont \llap{##2}\hskip\labelsep ##1]}%
\def\@opargbegintheorem##1##2##3{%
  \item[\theorem@headerfont \llap{##2}\hskip\labelsep ##1\ (##3)]}}
\endgroup
%</thm>
%    \end{macrocode}
% \end{macro}
%
%
%
%
% \subsection{Final Definitions}
%
% \begin{macro}{\theorempreskipamount}
% \begin{macro}{\theorempostskipamount}
%    The \textsf{skip} parameters that regulate the vertical empty space
%    before and after the {\envfont theorem} environment have
%    to be allocated as well.
% \changes{v2.0f}{89/07/20}{Unnecessary `global' removed.}
%    \begin{macrocode}
%<*package>
\newskip\theorempreskipamount
\newskip\theorempostskipamount
%    \end{macrocode}
%    Since we have used the same values for all theorem sets, we now can
%    assign them.
%    \begin{macrocode}
\global\setlength\theorempreskipamount{12pt plus 5pt minus 3pt}
\global\setlength\theorempostskipamount{8pt plus 3pt minus 1.5pt}
%    \end{macrocode}
% \end{macro}
% \end{macro}
%
%
% \begin{macro}{\@endtheorem}
%    The same holds for the macro |\@endtheorem|, which ends a
%    {\envfont theorem} environment. Since it is the same for all
%    theorem sets, it is removed from the macros |\th@|\meta{style}.
%    It simply ends the {\envfont trivlist} environment, which was
%    begun in |\@thm|.
%    \begin{macrocode}
\global\let\@endtheorem=\endtrivlist
%    \end{macrocode}
% \end{macro}
%
%
%
% \begin{macro}{\@preamblecmds}
%    All macros defined above are to be used only in the preamble.
%    Therefore, we insert them in |\@preamblecmds| which will disable
%    them at begin document. This is done by the internal \LaTeXe{}
%    command |\@onlypreamble|.
% \changes{v2.0g}{89/09/19}{`theoremheaderfont and `theorembodyfont
%    added.}
% \changes{v2.2a}{94/02/03}{Use \LaTeXe{} interface}
%    \begin{macrocode}
\@onlypreamble\@xnthm
\@onlypreamble\@ynthm
\@onlypreamble\@othm
\@onlypreamble\newtheorem
\@onlypreamble\theoremstyle
\@onlypreamble\theorembodyfont
\@onlypreamble\theoremheaderfont
%    \end{macrocode}
% \end{macro}
%
% Finally we declare the \texttt{plain} theorem style to be the default.
% \changes{v2.1b}{90/06/04}{Fixed missing `item by declaring default
%    style.}
%    \begin{macrocode}
\theoremstyle{plain}
%</package>
%    \end{macrocode}
%
% \Finale

\endinput
}{}
\ProvidesFile{thp.sty}
             [\FMithmInfo]
%    \end{macrocode}
%
%
% \begin{macro}{\th@plain}
% \begin{macro}{\@begintheorem}
% \begin{macro}{\@opargbegintheorem}
%    |\theoremstyle{plain}| corresponds to the original
%    definition, except that the distances to the surrounding text are
%    determined by the parameters |\theorempreskipamount| and
%    |\theorempostskipamount|.
%    First we set the default body font.
%    \begin{macrocode}
\gdef\th@plain{\normalfont\itshape
%    \end{macrocode}
%    Then we define |\@begintheorem| and
%    |\@opargbegintheorem|. These two macros define how the
%    header of a theorem is typeset.  |\@opargbegintheorem| will
%    be called if a \textsf{theorem} environment with an optional
%    argument is encountered; otherwise, the header is constructed by
%    calling |\@begintheorem|.  If one of these macros is
%    executed, we are within a \textsf{trivlist} environment started by
%    |\@thm|.  So the theorem header is produced with an
%    |\item| command.
%
%    Instead of specifying the header font directly, all standard
%    theorem styles use the |\theorem@headerfont| macro to allow
%    customization.  The extra space (|\labelsep|) is necessary
%    because of problems in the \textsf{trivlist} environment.
%    \begin{macrocode}
  \def\@begintheorem##1##2{%
        \item[\hskip\labelsep \theorem@headerfont ##1\ ##2]}%
%    \end{macrocode}
%    The definition of |\@opargbegintheorem| is completely
%    analogous. The only difference is the fact that there exists a
%    third argument (which is the optional parameter of the
%    environment and contains additional information about the
%    theorem).  Customarily we enclose it in parentheses.
%    \begin{macrocode}
\def\@opargbegintheorem##1##2##3{%
   \item[\hskip\labelsep \theorem@headerfont ##1\ ##2\ (##3)]}}
%    \end{macrocode}
% \end{macro}
% \end{macro}
% \end{macro}
%
% \noindent
% We conclude with an |\endgroup| to restore the
% |\catcode| of |@|.
%    \begin{macrocode}
\endgroup
%</thp>
%    \end{macrocode}
%
%
% \subsubsection{The \texttt{break} style}
%
% This style option is stored in the file |thb.sty|.
% For the next two lines see the documentation for |\th@plain| on
% page~\pageref{dok:th@plain}.
%    \begin{macrocode}
%<*thb>
\begingroup \makeatletter
\@ifundefined{theorem@style}{% \iffalse meta-comment
%
% Copyright (C) 1993-2024
%
% The LaTeX Project and any individual authors listed elsewhere
% in this file.
%
% This file is part of the Standard LaTeX `Tools Bundle'.
% -------------------------------------------------------
%
% It may be distributed and/or modified under the
% conditions of the LaTeX Project Public License, either version 1.3c
% of this license or (at your option) any later version.
% The latest version of this license is in
%    https://www.latex-project.org/lppl.txt
% and version 1.3c or later is part of all distributions of LaTeX
% version 2005/12/01 or later.
%
% The list of all files belonging to the LaTeX `Tools Bundle' is
% given in the file `manifest.txt'.
%
% \fi
%
%%
%
%
\def\FMithmInfo{2023/07/05 v2.2c Theorem extension package (FMi)}
%
% \ProvidesFile{theorem.dtx}[\FMithmInfo]
%\iffalse   % this is a METACOMMENT !
%%
%% Package `theorem' to use with LaTeX2e
%% Copyright (C) 1989-1995 Frank Mittelbach, all rights reserved.
%<+package>\@ifundefined{theorem@style}{}{\endinput}
%<+package>\NeedsTeXFormat{LaTeX2e}[1995/06/01]
%<+package>\ProvidesPackage{theorem}
%<+package>         [\FMithmInfo]
%
% \fi
%
% \changes{v1.0}{87/05/07}{First draft.}
% \changes{v1.0a}{87/05/07}{`thistheoremstyle' added.}
% \changes{v1.1a}{87/07/29}{New concept: redefine the internal
%                          \protect\LaTeX{} commands `@xnthm' etc.}
% \changes{v1.1b}{87/07/31}{`theoremfont' used to specify font.}
% \changes{v1.2a}{88/01/08}{`@endtheorem' now globally defined.}
% \changes{v1.2b}{88/01/11}{`newtheorem' only allowed in the preamble.}
% \changes{v1.3a}{88/02/14}{`theoremfont' now a toks register.}
%
% \changes{v2.0a}{88/11/16}{New implementation and new documentation.}
% \changes{v2.0a}{88/11/16}{No layout changes but internal modifications
%                           old substyles won't work.}
% \changes{v2.0b}{89/01/30}{`global' used for all definitions.}
% \changes{v2.0c}{89/06/15}{Documentation translated to English.
%                            Thanks to LLL and Nina}
% \changes{v2.1a}{90/02/15}{Support for new font selection.}
% \changes{v2.2a}{94/02/03}{Cleanup for LaTeX2e}
%
% \DoNotIndex{\@endparenv,\@ifdefinable,\@ifnextchar,\@ifundefined}
% \DoNotIndex{\@namedef,\@nameuse,\@spaces,\@tempa,\@warning}
% \DoNotIndex{\ }
% \DoNotIndex{\begingroup,\bfseries}
% \DoNotIndex{\csname}
% \DoNotIndex{\def,\do,\docdate}
% \DoNotIndex{\endcsname,\endgroup,\endinput,\endtrivlist,\expandafter}
% \DoNotIndex{\filedate,\fileversion}
% \DoNotIndex{\gdef,\global}
% \DoNotIndex{\hbox,\hskip}
% \DoNotIndex{\ignorespaces,\input,\itshape,\item}
% \DoNotIndex{\labelsep,\let,\llap}
% \DoNotIndex{\makeatletter}
% \DoNotIndex{\noexpand}
% \DoNotIndex{\refstepcounter,\relax,\rlap}
% \DoNotIndex{\setlength,\slshape,\space,\string,\strut}
% \DoNotIndex{\the,\trivlist,\typeout}
% \DoNotIndex{\vbox}
% \DoNotIndex{\xdef}
%
% \def\deflabel#1{\ttfamily #1\hfill}
% \def\deflist#1{\begin{list}{}{\settowidth\labelwidth{\ttfamily #1}%
%                               \setlength\leftmargin\labelwidth
%                               \addtolength\leftmargin\labelsep
%                               \let\makelabel\deflabel}}
% \def\enddeflist{\end{list}}
% \def\envfont{\normalfont\ttfamily}
%
% \MakeShortVerb{\|}
%
% \newcommand{\TUB}{{\sl TUGboat\/}}
%
% \GetFileInfo{theorem.dtx}
% \title{An Extension of the
%        \LaTeX{} theorem environment\thanks{This file
%        has version number \fileversion, last
%        revised \filedate.}}
% \author{Frank Mittelbach}
%
% \MaintainedByLaTeXTeam{tools}
% \maketitle
%
% \begin{abstract}
%    The macros described in this paper yield an extension of the
%    \LaTeX{} theorem mechanism. It is designed to satisfy the
%    different requirements of various journals. Thus, the
%    layout of the ``theorems'' can be manipulated by determining a
%    ``style''. This article describes not only the use, but
%    also the definition, of the necessary macros.
% \end{abstract}
%
% \section*{Preface to version 2.2}
%
% For \LaTeXe{} this package did not need any fundamental changes. I
% only modified the messages generated so that theorem layout styles
% will show up with the |\listfiles| command and cleaned the section
% on the New Font Selection Scheme since this is now included in
% \LaTeX.
%
% \section*{Preface to version 2.1}
%
% This version is identical to 2.0g described in \TUB\ 10\#3 except
% for some internal defaults which are now set depending on the used
% font selection scheme.
%
% This was done to avoid unpleasant surprises if the new font selection
% scheme is in force. For further details see
% section~\ref{sec:newfonts} and \cite{bk:GMS94}.
%
% \section{Introduction}
%
% For our purposes here, ``theorems'' are labelled enunciations,
% often set off from the main text by extra space and a font change.
% Theorems, corollaries, conjectures, definitions, and remarks
% are all instances of ``theorems''.  The ``header'' of these structures
% is composed of a label (such as \textsc{Theorem} or \textsc{Remark})
% and a number which serializes an item in the sequence of items
% with the same label.
%
% Shortly after the introduction of \LaTeX{} at the Fachbereich
% Mathematik in Mainz, the desire to manipulate the layout of
% ``theorems'' arose.  In Mainz, the following two conventions came
% into general use:
% \begin{enumerate}
%   \item The number of the theorem is shown in the margin.
%   \item There is a line break at
%         the end of the theorem header.
% \end{enumerate}
%
% Additionally, some journals require different formats which depend on
% the ``sort of theorem'': e.g.\ often remarks and definitions are set
% in |\upshape|, while |\itshape| is employed for main theorems.
%
% Confronted with these requirements, a theorem environment was
% developed in Mainz which allows separate determination of the
% layout of the ``theorems sets'', comparable to |\pagestyle|.
%
%
% \section{The user interface}
% \subsection{Defining new theorem sets}
%
% \DescribeMacro\newtheorem
% As in the original \LaTeX{} version, the command |\newtheorem|
% defines a new ``theorem set'' or ``theorem-like structure''.
% Two required arguments name the new environment and give the text
% to be typeset with each instance of the new ``set'', while
% an optional argument determines how
% the ``set'' is enumerated:
% \begin{description}
%    \item[\ttfamily \bslash newtheorem\{foo\}\{bar\}]
%       The theorem set {\envfont foo} (whose name is \texttt{bar})
%       uses its own counter.
%    \item[\ttfamily \bslash newtheorem\{foo2\}{[foo]}\{bar2\}]
%       The theorem set {\envfont foo2} (printed name \texttt{bar2})
%       uses the same counter as the theorem set \texttt{foo}.
%    \item[\ttfamily \bslash newtheorem\{foo3\}\{bar3\}{[section]}]
%       The theorem set {\envfont foo3} (printed name \texttt{bar3}) is
%       enumerated within the counter \texttt{section}, i.e.\ with every
%       new |\section| the enumeration begins again with 1, and
%       the enumeration is composed from the section-number and the
%       theorem counter itself.
% \end{description}
%
% \DescribeMacro\theoremstyle
% Additionally, the command |\theoremstyle| can define the layout
% of various, or all, theorem sets.  It should be noted that any
% theorem set defined by |\newtheorem| is typeset in the
% |\theoremstyle| that is current at the time of the definition.
% Thus, the following
% \begin{verbatim}
%    \theoremstyle{break}        \newtheorem{Cor}{Corollary}
%    \theoremstyle{plain}        \newtheorem{Exa}{Example}[section]
%\end{verbatim}                             ^^A No blank at line start
% leads to the result that the set {\envfont Cor} is formatted in the
% style \texttt{break}, while the set {\envfont Exa} and all the
% following ones are formatted in the style \texttt{plain}, unless
% another |\theoremstyle| follows.  Since the definitions installed by
% |\newtheorem| are global, one also can limit |\theoremstyle| locally
% by grouping braces.
%
% \DescribeMacro\theorembodyfont
% The choice of the font for the theorem body is completely
% independent of the chosen |\theoremstyle|; this has proven to
% be very advantageous.  For example,
% \begin{verbatim}
%   {\theorembodyfont{\upshape}          \newtheorem{Rem}{Remark}}
%\end{verbatim}
% defines a theorem set {\envfont Rem}, which will be set in
% |\upshape| in the current layout (which in our example is
% \texttt{plain}).  As with |\theoremstyle|, the |\theorembodyfont|
% chosen is that current at the time of |\newtheorem|.  If
% |\theorembodyfont| is not specified or one defines
% |\theorembodyfont{}|, then the font used will be that defined by the
% |\theoremstyle|.
%
% \DescribeMacro\theoremheaderfont It is also possible to customize
% the font used for the theorem headers.  This is, however, a global
% declaration, and therefore there should be at most one
% |\theoremheaderfont| declaration in the preamble.\footnote{If it is
% actually necessary to have different header fonts, one has to define
% new theorem styles (substituting the desired font) or specify the
% information directly in the \texttt{\bslash newtheorem} declaration
% (the unclean variant).}
%
% \DescribeMacro\theorempreskipamount
% \DescribeMacro\theorempostskipamount Two additional parameters
% affect the vertical space around the theorem environments:
% |\theorempreskipamount| and |\theorempostskipamount| define,
% respectively, the spacing before and after such an environment.
% These parameters apply for all theorem sets and can be manipulated
% with the ordinary length macros.  They are rubber lengths
% (`\textsf{skips}'), and therefore can contain \texttt{plus} and
% \texttt{minus} parts.
%
% Since the definition of theorem sets should---most sensibly---be
% placed in the preamble, we only allow installation there.
% It is therefore possible to release the memory used here after
% |\begin{document}|, in order to make room for other
% applications.
%
%
% \subsection{Existing theorem styles}
%
% The following theorem styles exist to date:
% \begin{deflist}{marginbreak}
%    \item[plain]
%       This theorem style emulates the original \LaTeX{} definition,
%       except that additionally the parameters
%       |\theorem...skipamount| are used.
%    \item[break]
%       In this style, the theorem header is followed by a line
%       break.
%    \item[marginbreak]
%       The theorem number is set in the margin, and there is a line
%       break as in \texttt{break}.
%    \item[changebreak]
%       Like \texttt{break}, but with header number and text
%       interchanged.
%    \item[change]
%       Header number and text are interchanged, without a line break.
%    \item[margin]
%       The number is set in the left margin, without a line break.
% \end{deflist}
% All styles (except \texttt{plain}) select |\slshape| as the default
% |\theorembodyfont|.
%
%
% \subsection{Examples}
%
% Given the above theorem sets {\envfont Cor}, {\envfont Exa}
% and {\envfont Rem},
% suppose that the preamble also contains the declarations:
% \begin{verbatim}
%    \theoremstyle{marginbreak}   \newtheorem{Lem}[Cor]{Lemma}
%    \theoremstyle{change}
%    \theorembodyfont{\itshape}        \newtheorem{Def}[Cor]{Definition}
%
%    \theoremheaderfont{\scshape}
%\end{verbatim}
% Then the following are some typical examples of the typeset output
% resulting from their use.
%
% \begin{Cor}
%    This is a sentence typeset in the theorem environment {\envfont
%    Cor}.
% \end{Cor}
%
% \begin{Exa}
%    This is a sentence typeset in the theorem environment {\envfont
%    Exa}.
% \end{Exa}
%
% \begin{Rem}
%    This is a sentence typeset in the theorem environment {\envfont
%    Rem}.
% \end{Rem}
%
% \begin{Lem}[Ben User]
%    This is a sentence typeset in the theorem environment {\envfont
%    Lem}.
% \end{Lem}
%
% \begin{Def}[Very impressive Definition]
%    This is a sentence typeset in the theorem environment {\envfont
%    Def}.
% \end{Def}
%
% The last two examples show the effect of the optional argument to
% a theorem environment (it is the text typeset in parentheses).
%
%
% \section{Special Considerations}
% \label{sec:newfonts}
%
% Theoremheader and body are implemented as a unit. This means that
% the |\theoremheaderfont|  will inherit characteristics of the
% |\theorembodyfont| in \LaTeXe. Thus, if for example
% |\theorembodyfont| is |\itshape| and |\theoremheaderfont|
% is |\bfseries| the font selected for the header will have the
% characteristics `bold extended italic'. If this is not desired one
% should set the |\theoremheaderfont| to something like
% \begin{verbatim}
%      \theoremheaderfont{\normalfont\bfseries}
%\end{verbatim}
% i.e.\ supplying all necessary font information explicitly.
%
%
%
% \section{Acknowledgements}
%
% The publication of this set of macros was only possible with the
% help of Christina Busse (translating the manuscript into English),
% Joachim Pense (playing the r\^ole of typist), Chris Rowley
% (looking everything over) and many others providing useful
% suggestions.
%
% \MaybeStop{%
%    \begin{thebibliography}{9}
%       \bibitem{bk:GMS94} \textsc{M.~Goossens}, \textsc{F.~Mittelbach}
%         and \textsc{A.~Samarin}.
%         \newblock The \LaTeX{} Companion.
%         \newblock
%         Addison-Wesley, Reading, Massachusetts, 1994.
%       \bibitem{latex:source} \textsc{Lamport, Leslie.} \newblock
%                               \texttt{latex.tex}, version 2.09, date
%                               Feb. 1990.
%    \end{thebibliography}
%    \PrintIndex
%    \PrintChanges}
%
%
% \section{The documentation driver file}
%
% The next bit of code contains the documentation driver file for
% \TeX{}, i.e., the file that will produce the documentation you are
% currently reading. It will be extracted from this file by the
% \texttt{docstrip} program. Since it is the first code in the file
% one can alternatively process this file directly with \LaTeXe{} to
% obtain the documentation.
%
%    \begin{macrocode}
%<*driver>
\documentclass{ltxdoc}

\usepackage{theorem}

%<+driver>% The next few lines define theorem sets which are used
%<+driver>% in the example section of the documentation.

\theoremstyle{break}       \newtheorem{Cor}{Corollary}
\theoremstyle{plain}       \newtheorem{Exa}{Example}[section]
{\theorembodyfont{\upshape}\newtheorem{Rem}{Remark}}
\theoremstyle{marginbreak} \newtheorem{Lem}[Cor]{Lemma}
\theoremstyle{change}
\theorembodyfont{\itshape} \newtheorem{Def}[Cor]{Definition}

\theoremheaderfont{\scshape}

\RecordChanges

\begin{document}
   \DocInput{theorem.dtx}
\end{document}
%</driver>
%    \end{macrocode}
%
% \section{Definition of the Macros}
%
% If the file has been loaded before, we abort immediately. If not the
% package announces itself (this is actually done at the very top of
% the file---the way it is done isn't good style so don't copy it).
% \changes{v2.0e}{89/07/19}{Spaces removed from `typeout'.}
% \changes{v2.2b}{95/11/19}{Announce moved to top of file}
%    \begin{macrocode}
%<*package>
%\@ifundefined{theorem@style}{}{\endinput}
%\def\FMithmInfo{1995/11/19 v2.2b Theorem extension package (FMi)}
%\ProvidesPackage{theorem}[\FMithmInfo]
%    \end{macrocode}
%
%
% \subsection{Definition of theorem styles and fonts}
%
% All the definitions in this file are done globally to allow
% inputting this file inside a group.
%
% \begin{macro}{\theoremstyle}
%    Before a theorem style can be installed, the chosen style must be
%    known. For that reason, we must test to see that
%    |\th@|\meta{style} is known or, more precisely, that it is
%    different from |\relax|. If the style is not known then
%    |\th@plain| is used.
%    \begin{macrocode}
\gdef\theoremstyle#1{%
   \@ifundefined{th@#1}{\@warning
          {Unknown theoremstyle `#1'. Using `plain'}%
          \theorem@style{plain}}%
%    \end{macrocode}
%    We save the theorem style to be used in the token register
%    |\theorem@style|.
%    \begin{macrocode}
      {\theorem@style{#1}}%
%    \end{macrocode}
%    Now we ``evaluate'' the theorem style: this means, we call the
%    macro |\th@|\meta{style} which will activate the relevant
%    definitions which are contained in a separate file.  This is done
%    in a group to suppress changes to the current font.  This could
%    otherwise pose problems together with the new font selection
%    scheme\footnote{When I printed the original article using the new
%    font selection scheme I ended with a document with slanted
%    typefaces (text headings and all) simply because one of the
%    theorem styles used \texttt{\bslash sl} at toplevel.} if the
%    |\th@|\meta{style} is evaluated a second time.
% \changes{v2.1a}{90/02/15}{Group added around `th@..}
% \changes{v2.1a}{90/02/15}{Reseting of `..@begintheorem removed.}
%    \begin{macrocode}
      \begingroup
        \csname th@\the\theorem@style \endcsname
      \endgroup}
%    \end{macrocode}
% \end{macro}
%
%
% \begin{macro}{\@begintheorem}
% \changes{v2.1a}{90/02/15}{Relaxed at toplevel.}
% \begin{macro}{\@opargbegintheorem}
% \changes{v2.1a}{90/02/15}{Relaxed at toplevel.}
%    We reset |\@begintheorem| and |\@opargbegintheorem| to
%    |\relax| since these commands are no longer necessary at
%    toplevel. This will save a few tokens.
%    \begin{macrocode}
\global\let\@begintheorem\relax
\global\let\@opargbegintheorem\relax
%    \end{macrocode}
% \end{macro}
% \end{macro}
%
% \begin{macro}{\theorem@style}
%    Obviously the \textsf{token} register used above has to be
%    allocated. To assure the utmost compatibility with the
%    original \LaTeX{} definition, we set the default theorem style to
%    \texttt{plain}, which implements the usual \LaTeX{} convention.
% \changes{v2.0f}{89/07/20}{Unnecessary `global' removed.}
%    \begin{macrocode}
\newtoks\theorem@style
\global\theorem@style{plain}
%    \end{macrocode}
% \end{macro}
%
% \begin{macro}{\theorembodyfont}
% \begin{macro}{\theorem@bodyfont}
%    If the |\theorembodyfont| is set by the user then it should not
%    interact with the default font set in the theorem style. When the
%    new font selection is in force this may happen if, for example,
%    the default is |\itshape| and the new |\theorembodyfont| is
%    |\sffamily|. So we add a |\reset@font| command in front of the
%    user definition.
% \changes{v2.0f}{89/07/20}{Unnecessary `global' removed.}
% \changes{v2.0g}{89/09/19}{`theoremfont to `theorembodyfont.}
% \changes{v2.1c}{92/06/22}{`reset@font added and logic changed.}
% \changes{v2.1d}{92/10/28}{Reset to empty if arg is empty}
%    \begin{macrocode}
\gdef\theorembodyfont#1{%
%    \end{macrocode}
%    We check if the argument supplied is empty and if so put nothing
%    into the |\theorem@bodyfont| token register to allow for
%    |\theorembodyfont{}| as a mean of using the default of the
%    current |\theoremstyle|.
%    \begin{macrocode}
   \def\@tempa{#1}%
   \ifx\@tempa\@empty
    \theorem@bodyfont{}%
   \else
    \theorem@bodyfont{\reset@font#1}%
   \fi
   }
\newtoks\theorem@bodyfont
\global\theorem@bodyfont{}
%    \end{macrocode}
% \end{macro}
% \end{macro}
%
%
% \begin{macro}{\theoremheaderfont}
% \changes{v2.0g}{89/09/19}{Macro added.}
%    The font for the theorem headers is handled differently because
%    this definition applies to all theorem styles.
%    \begin{macrocode}
\gdef\theoremheaderfont#1{\gdef\theorem@headerfont{#1}%
%    \end{macrocode}
%    After using the macro once it is redefined to produce an error
%    message.
%    \begin{macrocode}
       \gdef\theoremheaderfont##1{%
        \typeout{\string\theoremheaderfont\space should be used
                 only once.}}}
%    \end{macrocode}
% \end{macro}
%
% \begin{macro}{\theorem@headerfont}
% \changes{v2.0g}{89/09/19}{Macro added.}
%    To set the |\theorem@headerfont| default we first test if
%    the new fontselection scheme is in force.
% \changes{v2.1a}{90/02/15}{Support for new font selection.}
%    \begin{macrocode}
\ifx\upshape\undefined
%    \end{macrocode}
%    If not we define it to expand into |\bfseries|. We don't use
%    |\let| just in case a following style option redefines this
%    macro.
%    \begin{macrocode}
\gdef\theorem@headerfont{\bfseries}
%    \end{macrocode}
%    Otherwise we reset the current shape before calling |\bfseries|.
%    \begin{macrocode}
\else \gdef\theorem@headerfont{\normalfont\bfseries}\fi
%    \end{macrocode}
% \end{macro}
%
% \begin{macro}{\th@plain}
% \begin{macro}{\th@break}
% \begin{macro}{\th@marginbreak}
% \begin{macro}{\th@changebreak}
% \begin{macro}{\th@change}
% \begin{macro}{\th@margin}
%    The different styles are defined in macros such as
%    |\th@plain|.  Since memory space is precious in
%    ``non-Big-versions'', we have to avoid offering too many unused
%    definitions.  Therefore we define these styles in separate files
%    that can be loaded on demand.  Thus the commands themselves only
%    load these files. We use |\@input@|, a \LaTeXe{} internal command
%    that ensures that the file will be listed with |\listfiles|.
% \changes{v2.2a}{94/02/02}{Use `@input@ to load theorem layout files}
%    \begin{macrocode}
\gdef\th@plain{\@input@{thp.sty}}
\gdef\th@break{\@input@{thb.sty}}
\gdef\th@marginbreak{\@input@{thmb.sty}}
\gdef\th@changebreak{\@input@{thcb.sty}}
\gdef\th@change{\@input@{thc.sty}}
\gdef\th@margin{\@input@{thm.sty}}
%    \end{macrocode}
%    This list will be expanded when new styles become
%    available.  For testing,  just append new theorem substyles
%    as document options.
% \end{macro}
% \end{macro}
% \end{macro}
% \end{macro}
% \end{macro}
% \end{macro}
%
%
%
% \subsection{Definition of a new theorem set}
%
% As already pointed out, a new theorem environment can be defined in
% three different ways:
% \begin{verbatim}
%    \newtheorem{Lem}{Lemma}
%    \newtheorem{Lem}{Lemma}[section]
%    \newtheorem{Lem}[Theorem]{Lemma}
%\end{verbatim}
% The function of the macro |\newtheorem| is to recognize these cases
% and then to branch into one of the three macros |\@ynthm|, |\@xnthm|
% or |\@othm|.  This mechanism is adopted unchanged from
% \cite{latex:source}; the essential point here is that, for example,
% in the second case, the arguments \texttt{Lem}, \texttt{Lemma} and
% \texttt{section} are passed over to the macro |\@xnthm|.
%
% We inspect this case first because the others present fewer problems,
% and thus are easily derived from this one.
%
% \begin{macro}{\@xnthm}
%    For our example arguments, the macro |\@xnthm| must fulfill
%    the following:
%    \begin{itemize}
%       \item
%          Define a new \LaTeX-counter `Lem'
%       \item
%          reset this counter within a |\section|
%       \item
%          define the macro |\theLem|
%       \item
%          define the environment macros |\Lem| and
%          |\endLem| using the current
%          |\theoremstyle| and |\theorem@bodyfont|.
%    \end{itemize}
%    Obviously, all this should happen only if the first argument
%    of |\@xnthm| (i.e.\ \texttt{Lem} in our example) is chosen so
%    as not to conflict with any previously defined commands or
%    environments.  This test is performed by the \LaTeX{} macro
%    |\@ifdefinable|.
%    \begin{macrocode}
\gdef\@xnthm#1#2[#3]{\expandafter\@ifdefinable\csname #1\endcsname
%    \end{macrocode}
%    Therefore, the first argument of |\@ifdefinable| is the
%    expansion (in the example, |\Lem|) of
%    |\csname||#1\endcsname|.  The second argument is
%    executed only if the test has been completed successfully.
%    \begin{macrocode}
   {%
%    \end{macrocode}
%    Now we define the new counter. The names of the \LaTeX{}
%    macros employed should speak for themselves:
% \changes{v2.2b}{1995/10/07}{Use \cs{@newctr}}
% \changes{v2.2c}{1995/11/23}{Use \cs{@newctr} properly.}
%    \begin{macrocode}
    \@definecounter{#1}\@newctr{#1}[#3]%
%    \end{macrocode}
%    Using |\@newctr| will give a proper error message if the counter
%    in |#3| is not defined.
%    In defining `|\theLem|' we must generate the desired macro
%    name by use of |\expandafter| and |\csname|.
%    \begin{macrocode}
    \expandafter\xdef\csname the#1\endcsname
%    \end{macrocode}
%    An |\xdef| is used in order to make the definition global,
%    and to ensure that it contains the replacement texts of
%    |\@thmcountersep| and |\@thmcounter|.\footnote{These
%    two macros can be defined by the document style.  Their default
%    values produce a `\texttt{.}' as separation and an arabic
%    representation of the number.} However, not everything should be
%    expanded. For example, it saves space to use |\thesection|
%    instead of its---at times---lengthy expansion.
%    \begin{macrocode}
      {\expandafter \noexpand \csname the#3\endcsname
       \@thmcountersep \@thmcounter{#1}}%
%    \end{macrocode}
%    Thus with the defaults of \LaTeX{}, |\theLem| would be
%    replaced by the command sequence |\thesection.\arabic{Lem}|.
%
%    We will now look at the definition of the macro which is executed
%    at the beginning of the actual environment (in our example this
%    macro is |\Lem|). It should be noted that we use an
%    ``|\expandafter| trick'' to expand only certain parts of the
%    replacement text at the time of the definition.
%    \begin{macrocode}
    \def\@tempa{\global\@namedef{#1}}%
    \expandafter \@tempa \expandafter{%
%    \end{macrocode}
%    First, the macro that contains the current definitions of
%    |\@begintheorem| and |\@opargtheorem| should be called
%    up.  The name of this macro---as is already known---has the form
%    |\th@|\meta{theorem\ style}; therefore, it must be called by
%    \begin{macrocode}
      \csname th@\the \theorem@style
%    \end{macrocode}
%    In addition the default theorem font should be changeable, i.e.\
%    we have to insert the contents of |\theorem@bodyfont|. For
%    that reason, we expand even further, beyond |\endcsname|,
%    and thus insert the contents of the \textsf{token} register
%    |\theorem@bodyfont| in the replacement text.
%    \begin{macrocode}
            \expandafter \endcsname \the \theorem@bodyfont
%    \end{macrocode}
%    Now it is time to call the macro |\@thm| which takes over
%    the further processing.  It has two arguments: the current
%    counter name (in our example, \texttt{Lem}), and the text of the
%    label (in our example, \texttt{Lemma}).
%    \begin{macrocode}
     \@thm{#1}{#2}}%
%    \end{macrocode}
%    With this, the `sub-definition' is complete.  The macro
%    |\@endtheorem| ends a theorem environment and is, so far, nothing
%    but an |\endtrivlist|.  (Hence it is defined globally, and not
%    within the theorem styles.\footnote{This has to be changed as
%    soon as theorem styles that change \texttt{\bslash @endtheorem}
%    exist.  In such a case, all existing styles must be changed as
%    well since they will have to reset the macro.}) Therefore, we can
%    set it equivalent to the macro that ends the theorem set (in our
%    example, |\endLem|).  However, if some day theorem styles exist
%    that do change |\@endtheorem|, we would have to use the
%    commented-out line instead.
%    \begin{macrocode}
    \global \expandafter \let \csname end#1\endcsname \@endtheorem
%   \global\@namedef{end#1}{\@endtheorem}%
%    \end{macrocode}
%    With these commands all the required definitions are employed,
%    unless the test |\@ifdefinable| has failed.  Therefore, we
%    end the second argument of this macro and with it the definition
%    of |\@xnthm|.
%    \begin{macrocode}
   }}
%    \end{macrocode}
% \end{macro}
%
% \begin{macro}{\@ynthm}
%    The definition of |\@ynthm| is completely analogous. In
%    this case the new counter that is defined is not reset within
%    another counter; thus the definition of |\the...|
%    is simplified:
%    \begin{macrocode}
\gdef\@ynthm#1#2{\expandafter\@ifdefinable\csname #1\endcsname
   {\@definecounter{#1}%
    \expandafter\xdef\csname the#1\endcsname{\@thmcounter{#1}}%
%    \end{macrocode}
%    The rest of the definition corresponds literally to that of
%    |\@xnthm|:
%    \begin{macrocode}
    \def\@tempa{\global\@namedef{#1}}\expandafter \@tempa
     \expandafter{\csname th@\the \theorem@style \expandafter
     \endcsname \the\theorem@bodyfont \@thm{#1}{#2}}%
    \global \expandafter \let \csname end#1\endcsname \@endtheorem}}
%    \end{macrocode}
% \end{macro}
%
% \begin{macro}{\@othm}
%    The definition of |\@othm| does not contain anything new.
%    \begin{macrocode}
\gdef\@othm#1[#2]#3{%
%    \end{macrocode}
% \changes{v2.2b}{95/10/07}{Build in check for counter defined}
%    We do not define a new counter but instead use one that has
%    already been defined.  Thus the only definition we need is that
%    of this pseudo-counter (i.e.\ |\the|\meta{env.\ name}).
%    First we check if |#2| corresponds to a known counter name.
%    \begin{macrocode}
  \expandafter\ifx\csname c@#2\endcsname\relax
   \@nocounterr{#2}%
  \else
   \expandafter\@ifdefinable\csname #1\endcsname
   {\expandafter \xdef \csname the#1\endcsname
     {\expandafter \noexpand \csname the#2\endcsname}%
%    \end{macrocode}
%    All other parts of the definition can be adopted from
%    |\@xnthm|. We have to remember, though, that in this case
%    the name of the current counter and the theorem label have moved
%    to the second and third arguments.
%    \begin{macrocode}
    \def\@tempa{\global\@namedef{#1}}\expandafter \@tempa
     \expandafter{\csname th@\the \theorem@style \expandafter
     \endcsname \the\theorem@bodyfont \@thm{#2}{#3}}%
    \global \expandafter \let \csname end#1\endcsname \@endtheorem}%
  \fi}
%    \end{macrocode}
% \end{macro}
%
% \subsection{Macros that are employed in a {\envfont theorem}
%    environment}
%
% \begin{macro}{\@thm}
%    The macro |\@thm| has to increase the current counter.  Then,
%    depending on whether the environment has (or does not have) an
%    optional argument, it has to branch into either |\@begintheorem|
%    or |\@opargtheorem|.
%    \begin{macrocode}
\gdef\@thm#1#2{\refstepcounter{#1}%
%    \end{macrocode}
%    Now we start a {\envfont trivlist} environment, and give
%    |\@topsep| and |\@topsepadd| the values of the \textsf{skip}
%    registers |\theorempreskipamount| and |\theorempostskipamount|.
%    The value in |\@topsep| is the vertical space that is inserted by
%    the first (and only) |\item| in our |\trivlist| whilst
%    |\@topsepadd| is inserted by |\@endparenv| at the end of that
%    {\envfont trivlist} environment.  By using these registers, we
%    obtain the desired space around a {\envfont theorem} environment.
%    \begin{macrocode}
   \trivlist
   \@topsep \theorempreskipamount               % used by first \item
   \@topsepadd \theorempostskipamount           % used by \@endparenv
%    \end{macrocode}
%    Now we have to test whether an optional argument has been given.
% \changes{v2.0d}{89/07/18}{Forgotten percent added.}
%    \begin{macrocode}
   \@ifnextchar [%
%    \end{macrocode}
%    If there is an optional argument, we will call |\@ythm|, and
%    move the arguments read back into the input stream.
%    \begin{macrocode}
   {\@ythm{#1}{#2}}%
%    \end{macrocode}
%    If not, we call |\@begintheorem|. Its first argument is the
%    name of the theorem set (hence the second argument of
%    |\@thm|). Its second argument is the macro that produces the
%    current number.
%    \begin{macrocode}
   {\@begintheorem{#2}{\csname the#1\endcsname}\ignorespaces}}
%    \end{macrocode}
% \end{macro}
%
% \begin{macro}{\@xthm}
% \begin{macro}{\@ythm}
%    Both these macros were originally called by |\@thm|.  We do
%    not need |\@xthm| anymore, hence we reset it to
%    |\relax|. The definition of |\@ythm| has not changed at
%    all from its definition in \LaTeX. In order to make the macros
%    easier to understand, we will
%    nevertheless present it (commented out).
%    \begin{macrocode}
\global\let\@xthm\relax
% \def\@ythm#1#2[#3]{\@opargbegintheorem{#2}{\csname
%           the#1\endcsname}{#3}\ignorespaces}
%</package>
%    \end{macrocode}
%    The primitive |\ignorespaces| in |\@ythm| and
%    |\@thm| is needed to remove the spaces between the
%    |\begin{...}| and the actual text.
% \end{macro}
% \end{macro}
%
%
% \subsection{Definition of the theorem substyles}
%
% As already pointed out, the theorem substyles, defined below, are
% only loaded when necessary.  Note that all these substyles, except
% \texttt{plain}, have |\slshape| as the default body font.
%
% \subsubsection{The \texttt{plain} style}
%
% \label{dok:th@plain} As the following macros use |@|, we have
% to locally set the |\catcode| of this symbol to ``letter''.
% This happens within a group, so that we do not have to worry about
% which |\catcode| that symbol had before.
%    \begin{macrocode}
%<*thp>
\begingroup \makeatletter
%    \end{macrocode}
% Since we are now within a group, we must make all definitions
% globally.  First we make sure that \texttt{theorem.sty} is loaded.
% This will allow us to use this file as a document style option
% without having to call \texttt{theorem} itself as an option.  At the
% same time, we assure that at least version 2 is loaded, since
% |\theorem@style| was not defined in earlier versions.
%    \begin{macrocode}
\@ifundefined{theorem@style}{\input{theorem.sty}}{}
\ProvidesFile{thp.sty}
             [\FMithmInfo]
%    \end{macrocode}
%
%
% \begin{macro}{\th@plain}
% \begin{macro}{\@begintheorem}
% \begin{macro}{\@opargbegintheorem}
%    |\theoremstyle{plain}| corresponds to the original
%    definition, except that the distances to the surrounding text are
%    determined by the parameters |\theorempreskipamount| and
%    |\theorempostskipamount|.
%    First we set the default body font.
%    \begin{macrocode}
\gdef\th@plain{\normalfont\itshape
%    \end{macrocode}
%    Then we define |\@begintheorem| and
%    |\@opargbegintheorem|. These two macros define how the
%    header of a theorem is typeset.  |\@opargbegintheorem| will
%    be called if a \textsf{theorem} environment with an optional
%    argument is encountered; otherwise, the header is constructed by
%    calling |\@begintheorem|.  If one of these macros is
%    executed, we are within a \textsf{trivlist} environment started by
%    |\@thm|.  So the theorem header is produced with an
%    |\item| command.
%
%    Instead of specifying the header font directly, all standard
%    theorem styles use the |\theorem@headerfont| macro to allow
%    customization.  The extra space (|\labelsep|) is necessary
%    because of problems in the \textsf{trivlist} environment.
%    \begin{macrocode}
  \def\@begintheorem##1##2{%
        \item[\hskip\labelsep \theorem@headerfont ##1\ ##2]}%
%    \end{macrocode}
%    The definition of |\@opargbegintheorem| is completely
%    analogous. The only difference is the fact that there exists a
%    third argument (which is the optional parameter of the
%    environment and contains additional information about the
%    theorem).  Customarily we enclose it in parentheses.
%    \begin{macrocode}
\def\@opargbegintheorem##1##2##3{%
   \item[\hskip\labelsep \theorem@headerfont ##1\ ##2\ (##3)]}}
%    \end{macrocode}
% \end{macro}
% \end{macro}
% \end{macro}
%
% \noindent
% We conclude with an |\endgroup| to restore the
% |\catcode| of |@|.
%    \begin{macrocode}
\endgroup
%</thp>
%    \end{macrocode}
%
%
% \subsubsection{The \texttt{break} style}
%
% This style option is stored in the file |thb.sty|.
% For the next two lines see the documentation for |\th@plain| on
% page~\pageref{dok:th@plain}.
%    \begin{macrocode}
%<*thb>
\begingroup \makeatletter
\@ifundefined{theorem@style}{\input{theorem.sty}}{}
\ProvidesFile{thb.sty}
             [\FMithmInfo]
%    \end{macrocode}
%
% \begin{macro}{\th@break}
%    |\theoremstyle{break}| produces a line break after the name
%    of the theorem.  The font is |\slshape|.  Hence, we define
%    |\th@break| as follows:
%    \begin{macrocode}
\gdef\th@break{\normalfont\slshape
  \def\@begintheorem##1##2{\item[%
%    \end{macrocode}
%    We run into the following problem: it is not possible to create
%    the header with |\item[|\meta{title}|]| and then start
%    a new line by, for example, |\mbox{}\\|. Such a definition
%    will fail whenever a list environment follows immediately.  With
%    the above construction, the |\mbox{}| causes the switch
%    |@inlabel| (cf.\ definition of |\list| and
%    |\trivlist| in \cite{latex:source}) to be set to \texttt{false}
%    and so the following list will insert additional vertical
%    space (|\topskip|). This is quite annoying. Therefore, we
%    create the line break within the |\item|. In order to ensure
%    that the text will begin at the proper position in the following
%    line, we simply pretend that the label does not take any
%    room.\footnote{This will lead to problems whenever very high
%    symbols occurring in the line tower into the
%    heading. So, something else has to be done here sometime.}
%    \begin{macrocode}
       \rlap{\vbox{\hbox{\hskip \labelsep\theorem@headerfont ##1\ ##2}%
                   \hbox{\strut}}}]}%
%    \end{macrocode}
%    Again, the definition of |\@opargbegintheorem| is
%    completely analogous.
%    \begin{macrocode}
\def\@opargbegintheorem##1##2##3{%
  \item[\rlap{\vbox{\hbox{\hskip \labelsep \theorem@headerfont
                     ##1\ ##2\ (##3)}%
                    \hbox{\strut}}}]}}
\endgroup
%</thb>
%    \end{macrocode}
% \end{macro}
%
%
% \subsubsection{The \texttt{changebreak} style}
%
%    \begin{macrocode}
% This style option is stored in the file |thcb.sty|.
%    \begin{macrocode}
%<*thcb>
\begingroup \makeatletter
\@ifundefined{theorem@style}{\input{theorem.sty}}{}
\ProvidesFile{thcb.sty}
             [\FMithmInfo]
%    \end{macrocode}
%
% \begin{macro}{\th@changebreak}
%    The \texttt{change-break} theorem style is like \texttt{break}
%    but with interchange of theorem name and theorem number.  Thus we
%    define |\th@changebreak| as follows:
%    \begin{macrocode}
\gdef\th@changebreak{\normalfont\slshape
  \def\@begintheorem##1##2{\item
       [\rlap{\vbox{\hbox{\hskip\labelsep \theorem@headerfont ##2\ ##1}%
                    \hbox{\strut}}}]}%
\def\@opargbegintheorem##1##2##3{%
  \item[\rlap{\vbox{\hbox{\hskip\labelsep \theorem@headerfont
                     ##2\ ##1\ (##3)}%
                    \hbox{\strut}}}]}}
\endgroup
%</thcb>
%    \end{macrocode}
% \end{macro}
%
%
% \subsubsection{The \texttt{change} style}
%
% This style option is stored in the file |thc.sty|.
%    \begin{macrocode}
%<*thc>
\begingroup \makeatletter
\@ifundefined{theorem@style}{\input{theorem.sty}}{}
\ProvidesFile{thc.sty}
             [\FMithmInfo]
%    \end{macrocode}
%
% \begin{macro}{\th@change}
%    The \texttt{change} theorem style corresponds to the
%    \texttt{change break} style without a linebreak after the header.
%    To say it in another way, it's the same as the \texttt{plain}
%    style but with number and name interchanged and |\slshape| as the
%    default font.
%    \begin{macrocode}
\gdef\th@change{\normalfont\slshape
  \def\@begintheorem##1##2{\item
       [\hskip\labelsep \theorem@headerfont ##2\ ##1]}%
\def\@opargbegintheorem##1##2##3{%
  \item[\hskip\labelsep \theorem@headerfont ##2\ ##1\ (##3)]}}
\endgroup
%</thc>
%    \end{macrocode}
% \end{macro}
%
%
% \subsubsection{The \texttt{marginbreak} style}
%
% This style option is the one used most often at Mainz.  It is saved
% in the file |thmb.sty|.
%    \begin{macrocode}
%<*thmb>
\begingroup \makeatletter
\@ifundefined{theorem@style}{\input{theorem.sty}}{}
\ProvidesFile{thmb.sty}
             [\FMithmInfo]
%    \end{macrocode}
%
% \begin{macro}{\th@marginbreak}
%    The \texttt{margin break} style is nearly the same as the
%    \texttt{change break} style. The only difference is the placement
%    of the theorem number.  We use |\llap| to place it in the left
%    margin.
%
%    In this style |\labelsep| denotes the separation between the
%    number and the text.
%    \begin{macrocode}
\gdef\th@marginbreak{\normalfont\slshape
  \def\@begintheorem##1##2{\item
       [\rlap{\vbox{\theorem@headerfont
                    \hbox{\llap{##2}\hskip\labelsep ##1}%
                    \hbox{\strut}}}]}%
\def\@opargbegintheorem##1##2##3{%
  \item[\rlap{\vbox{\theorem@headerfont
                    \hbox{\llap{##2}\hskip\labelsep ##1\ (##3)}%
                    \hbox{\strut}}}]}}
\endgroup
%</thmb>
%    \end{macrocode}
% \changes{v2.1c}{90/02/17}{`endgroup outside macrocode env.}
% \end{macro}
%
%
%
% \subsubsection{The \texttt{margin} style}
%
% This style option is stored in the file |thm.sty|.
%    \begin{macrocode}
%<*thm>
\begingroup \makeatletter
\@ifundefined{theorem@style}{\input{theorem.sty}}{}
\ProvidesFile{thm.sty}
             [\FMithmInfo]
%    \end{macrocode}
%
% \begin{macro}{\th@margin}
%    Again this is only a variant of the theorem styles described
%    above without any new ideas.
%    \begin{macrocode}
\gdef\th@margin{\normalfont\slshape
  \def\@begintheorem##1##2{\item
       [\theorem@headerfont \llap{##2}\hskip\labelsep ##1]}%
\def\@opargbegintheorem##1##2##3{%
  \item[\theorem@headerfont \llap{##2}\hskip\labelsep ##1\ (##3)]}}
\endgroup
%</thm>
%    \end{macrocode}
% \end{macro}
%
%
%
%
% \subsection{Final Definitions}
%
% \begin{macro}{\theorempreskipamount}
% \begin{macro}{\theorempostskipamount}
%    The \textsf{skip} parameters that regulate the vertical empty space
%    before and after the {\envfont theorem} environment have
%    to be allocated as well.
% \changes{v2.0f}{89/07/20}{Unnecessary `global' removed.}
%    \begin{macrocode}
%<*package>
\newskip\theorempreskipamount
\newskip\theorempostskipamount
%    \end{macrocode}
%    Since we have used the same values for all theorem sets, we now can
%    assign them.
%    \begin{macrocode}
\global\setlength\theorempreskipamount{12pt plus 5pt minus 3pt}
\global\setlength\theorempostskipamount{8pt plus 3pt minus 1.5pt}
%    \end{macrocode}
% \end{macro}
% \end{macro}
%
%
% \begin{macro}{\@endtheorem}
%    The same holds for the macro |\@endtheorem|, which ends a
%    {\envfont theorem} environment. Since it is the same for all
%    theorem sets, it is removed from the macros |\th@|\meta{style}.
%    It simply ends the {\envfont trivlist} environment, which was
%    begun in |\@thm|.
%    \begin{macrocode}
\global\let\@endtheorem=\endtrivlist
%    \end{macrocode}
% \end{macro}
%
%
%
% \begin{macro}{\@preamblecmds}
%    All macros defined above are to be used only in the preamble.
%    Therefore, we insert them in |\@preamblecmds| which will disable
%    them at begin document. This is done by the internal \LaTeXe{}
%    command |\@onlypreamble|.
% \changes{v2.0g}{89/09/19}{`theoremheaderfont and `theorembodyfont
%    added.}
% \changes{v2.2a}{94/02/03}{Use \LaTeXe{} interface}
%    \begin{macrocode}
\@onlypreamble\@xnthm
\@onlypreamble\@ynthm
\@onlypreamble\@othm
\@onlypreamble\newtheorem
\@onlypreamble\theoremstyle
\@onlypreamble\theorembodyfont
\@onlypreamble\theoremheaderfont
%    \end{macrocode}
% \end{macro}
%
% Finally we declare the \texttt{plain} theorem style to be the default.
% \changes{v2.1b}{90/06/04}{Fixed missing `item by declaring default
%    style.}
%    \begin{macrocode}
\theoremstyle{plain}
%</package>
%    \end{macrocode}
%
% \Finale

\endinput
}{}
\ProvidesFile{thb.sty}
             [\FMithmInfo]
%    \end{macrocode}
%
% \begin{macro}{\th@break}
%    |\theoremstyle{break}| produces a line break after the name
%    of the theorem.  The font is |\slshape|.  Hence, we define
%    |\th@break| as follows:
%    \begin{macrocode}
\gdef\th@break{\normalfont\slshape
  \def\@begintheorem##1##2{\item[%
%    \end{macrocode}
%    We run into the following problem: it is not possible to create
%    the header with |\item[|\meta{title}|]| and then start
%    a new line by, for example, |\mbox{}\\|. Such a definition
%    will fail whenever a list environment follows immediately.  With
%    the above construction, the |\mbox{}| causes the switch
%    |@inlabel| (cf.\ definition of |\list| and
%    |\trivlist| in \cite{latex:source}) to be set to \texttt{false}
%    and so the following list will insert additional vertical
%    space (|\topskip|). This is quite annoying. Therefore, we
%    create the line break within the |\item|. In order to ensure
%    that the text will begin at the proper position in the following
%    line, we simply pretend that the label does not take any
%    room.\footnote{This will lead to problems whenever very high
%    symbols occurring in the line tower into the
%    heading. So, something else has to be done here sometime.}
%    \begin{macrocode}
       \rlap{\vbox{\hbox{\hskip \labelsep\theorem@headerfont ##1\ ##2}%
                   \hbox{\strut}}}]}%
%    \end{macrocode}
%    Again, the definition of |\@opargbegintheorem| is
%    completely analogous.
%    \begin{macrocode}
\def\@opargbegintheorem##1##2##3{%
  \item[\rlap{\vbox{\hbox{\hskip \labelsep \theorem@headerfont
                     ##1\ ##2\ (##3)}%
                    \hbox{\strut}}}]}}
\endgroup
%</thb>
%    \end{macrocode}
% \end{macro}
%
%
% \subsubsection{The \texttt{changebreak} style}
%
%    \begin{macrocode}
% This style option is stored in the file |thcb.sty|.
%    \begin{macrocode}
%<*thcb>
\begingroup \makeatletter
\@ifundefined{theorem@style}{% \iffalse meta-comment
%
% Copyright (C) 1993-2024
%
% The LaTeX Project and any individual authors listed elsewhere
% in this file.
%
% This file is part of the Standard LaTeX `Tools Bundle'.
% -------------------------------------------------------
%
% It may be distributed and/or modified under the
% conditions of the LaTeX Project Public License, either version 1.3c
% of this license or (at your option) any later version.
% The latest version of this license is in
%    https://www.latex-project.org/lppl.txt
% and version 1.3c or later is part of all distributions of LaTeX
% version 2005/12/01 or later.
%
% The list of all files belonging to the LaTeX `Tools Bundle' is
% given in the file `manifest.txt'.
%
% \fi
%
%%
%
%
\def\FMithmInfo{2023/07/05 v2.2c Theorem extension package (FMi)}
%
% \ProvidesFile{theorem.dtx}[\FMithmInfo]
%\iffalse   % this is a METACOMMENT !
%%
%% Package `theorem' to use with LaTeX2e
%% Copyright (C) 1989-1995 Frank Mittelbach, all rights reserved.
%<+package>\@ifundefined{theorem@style}{}{\endinput}
%<+package>\NeedsTeXFormat{LaTeX2e}[1995/06/01]
%<+package>\ProvidesPackage{theorem}
%<+package>         [\FMithmInfo]
%
% \fi
%
% \changes{v1.0}{87/05/07}{First draft.}
% \changes{v1.0a}{87/05/07}{`thistheoremstyle' added.}
% \changes{v1.1a}{87/07/29}{New concept: redefine the internal
%                          \protect\LaTeX{} commands `@xnthm' etc.}
% \changes{v1.1b}{87/07/31}{`theoremfont' used to specify font.}
% \changes{v1.2a}{88/01/08}{`@endtheorem' now globally defined.}
% \changes{v1.2b}{88/01/11}{`newtheorem' only allowed in the preamble.}
% \changes{v1.3a}{88/02/14}{`theoremfont' now a toks register.}
%
% \changes{v2.0a}{88/11/16}{New implementation and new documentation.}
% \changes{v2.0a}{88/11/16}{No layout changes but internal modifications
%                           old substyles won't work.}
% \changes{v2.0b}{89/01/30}{`global' used for all definitions.}
% \changes{v2.0c}{89/06/15}{Documentation translated to English.
%                            Thanks to LLL and Nina}
% \changes{v2.1a}{90/02/15}{Support for new font selection.}
% \changes{v2.2a}{94/02/03}{Cleanup for LaTeX2e}
%
% \DoNotIndex{\@endparenv,\@ifdefinable,\@ifnextchar,\@ifundefined}
% \DoNotIndex{\@namedef,\@nameuse,\@spaces,\@tempa,\@warning}
% \DoNotIndex{\ }
% \DoNotIndex{\begingroup,\bfseries}
% \DoNotIndex{\csname}
% \DoNotIndex{\def,\do,\docdate}
% \DoNotIndex{\endcsname,\endgroup,\endinput,\endtrivlist,\expandafter}
% \DoNotIndex{\filedate,\fileversion}
% \DoNotIndex{\gdef,\global}
% \DoNotIndex{\hbox,\hskip}
% \DoNotIndex{\ignorespaces,\input,\itshape,\item}
% \DoNotIndex{\labelsep,\let,\llap}
% \DoNotIndex{\makeatletter}
% \DoNotIndex{\noexpand}
% \DoNotIndex{\refstepcounter,\relax,\rlap}
% \DoNotIndex{\setlength,\slshape,\space,\string,\strut}
% \DoNotIndex{\the,\trivlist,\typeout}
% \DoNotIndex{\vbox}
% \DoNotIndex{\xdef}
%
% \def\deflabel#1{\ttfamily #1\hfill}
% \def\deflist#1{\begin{list}{}{\settowidth\labelwidth{\ttfamily #1}%
%                               \setlength\leftmargin\labelwidth
%                               \addtolength\leftmargin\labelsep
%                               \let\makelabel\deflabel}}
% \def\enddeflist{\end{list}}
% \def\envfont{\normalfont\ttfamily}
%
% \MakeShortVerb{\|}
%
% \newcommand{\TUB}{{\sl TUGboat\/}}
%
% \GetFileInfo{theorem.dtx}
% \title{An Extension of the
%        \LaTeX{} theorem environment\thanks{This file
%        has version number \fileversion, last
%        revised \filedate.}}
% \author{Frank Mittelbach}
%
% \MaintainedByLaTeXTeam{tools}
% \maketitle
%
% \begin{abstract}
%    The macros described in this paper yield an extension of the
%    \LaTeX{} theorem mechanism. It is designed to satisfy the
%    different requirements of various journals. Thus, the
%    layout of the ``theorems'' can be manipulated by determining a
%    ``style''. This article describes not only the use, but
%    also the definition, of the necessary macros.
% \end{abstract}
%
% \section*{Preface to version 2.2}
%
% For \LaTeXe{} this package did not need any fundamental changes. I
% only modified the messages generated so that theorem layout styles
% will show up with the |\listfiles| command and cleaned the section
% on the New Font Selection Scheme since this is now included in
% \LaTeX.
%
% \section*{Preface to version 2.1}
%
% This version is identical to 2.0g described in \TUB\ 10\#3 except
% for some internal defaults which are now set depending on the used
% font selection scheme.
%
% This was done to avoid unpleasant surprises if the new font selection
% scheme is in force. For further details see
% section~\ref{sec:newfonts} and \cite{bk:GMS94}.
%
% \section{Introduction}
%
% For our purposes here, ``theorems'' are labelled enunciations,
% often set off from the main text by extra space and a font change.
% Theorems, corollaries, conjectures, definitions, and remarks
% are all instances of ``theorems''.  The ``header'' of these structures
% is composed of a label (such as \textsc{Theorem} or \textsc{Remark})
% and a number which serializes an item in the sequence of items
% with the same label.
%
% Shortly after the introduction of \LaTeX{} at the Fachbereich
% Mathematik in Mainz, the desire to manipulate the layout of
% ``theorems'' arose.  In Mainz, the following two conventions came
% into general use:
% \begin{enumerate}
%   \item The number of the theorem is shown in the margin.
%   \item There is a line break at
%         the end of the theorem header.
% \end{enumerate}
%
% Additionally, some journals require different formats which depend on
% the ``sort of theorem'': e.g.\ often remarks and definitions are set
% in |\upshape|, while |\itshape| is employed for main theorems.
%
% Confronted with these requirements, a theorem environment was
% developed in Mainz which allows separate determination of the
% layout of the ``theorems sets'', comparable to |\pagestyle|.
%
%
% \section{The user interface}
% \subsection{Defining new theorem sets}
%
% \DescribeMacro\newtheorem
% As in the original \LaTeX{} version, the command |\newtheorem|
% defines a new ``theorem set'' or ``theorem-like structure''.
% Two required arguments name the new environment and give the text
% to be typeset with each instance of the new ``set'', while
% an optional argument determines how
% the ``set'' is enumerated:
% \begin{description}
%    \item[\ttfamily \bslash newtheorem\{foo\}\{bar\}]
%       The theorem set {\envfont foo} (whose name is \texttt{bar})
%       uses its own counter.
%    \item[\ttfamily \bslash newtheorem\{foo2\}{[foo]}\{bar2\}]
%       The theorem set {\envfont foo2} (printed name \texttt{bar2})
%       uses the same counter as the theorem set \texttt{foo}.
%    \item[\ttfamily \bslash newtheorem\{foo3\}\{bar3\}{[section]}]
%       The theorem set {\envfont foo3} (printed name \texttt{bar3}) is
%       enumerated within the counter \texttt{section}, i.e.\ with every
%       new |\section| the enumeration begins again with 1, and
%       the enumeration is composed from the section-number and the
%       theorem counter itself.
% \end{description}
%
% \DescribeMacro\theoremstyle
% Additionally, the command |\theoremstyle| can define the layout
% of various, or all, theorem sets.  It should be noted that any
% theorem set defined by |\newtheorem| is typeset in the
% |\theoremstyle| that is current at the time of the definition.
% Thus, the following
% \begin{verbatim}
%    \theoremstyle{break}        \newtheorem{Cor}{Corollary}
%    \theoremstyle{plain}        \newtheorem{Exa}{Example}[section]
%\end{verbatim}                             ^^A No blank at line start
% leads to the result that the set {\envfont Cor} is formatted in the
% style \texttt{break}, while the set {\envfont Exa} and all the
% following ones are formatted in the style \texttt{plain}, unless
% another |\theoremstyle| follows.  Since the definitions installed by
% |\newtheorem| are global, one also can limit |\theoremstyle| locally
% by grouping braces.
%
% \DescribeMacro\theorembodyfont
% The choice of the font for the theorem body is completely
% independent of the chosen |\theoremstyle|; this has proven to
% be very advantageous.  For example,
% \begin{verbatim}
%   {\theorembodyfont{\upshape}          \newtheorem{Rem}{Remark}}
%\end{verbatim}
% defines a theorem set {\envfont Rem}, which will be set in
% |\upshape| in the current layout (which in our example is
% \texttt{plain}).  As with |\theoremstyle|, the |\theorembodyfont|
% chosen is that current at the time of |\newtheorem|.  If
% |\theorembodyfont| is not specified or one defines
% |\theorembodyfont{}|, then the font used will be that defined by the
% |\theoremstyle|.
%
% \DescribeMacro\theoremheaderfont It is also possible to customize
% the font used for the theorem headers.  This is, however, a global
% declaration, and therefore there should be at most one
% |\theoremheaderfont| declaration in the preamble.\footnote{If it is
% actually necessary to have different header fonts, one has to define
% new theorem styles (substituting the desired font) or specify the
% information directly in the \texttt{\bslash newtheorem} declaration
% (the unclean variant).}
%
% \DescribeMacro\theorempreskipamount
% \DescribeMacro\theorempostskipamount Two additional parameters
% affect the vertical space around the theorem environments:
% |\theorempreskipamount| and |\theorempostskipamount| define,
% respectively, the spacing before and after such an environment.
% These parameters apply for all theorem sets and can be manipulated
% with the ordinary length macros.  They are rubber lengths
% (`\textsf{skips}'), and therefore can contain \texttt{plus} and
% \texttt{minus} parts.
%
% Since the definition of theorem sets should---most sensibly---be
% placed in the preamble, we only allow installation there.
% It is therefore possible to release the memory used here after
% |\begin{document}|, in order to make room for other
% applications.
%
%
% \subsection{Existing theorem styles}
%
% The following theorem styles exist to date:
% \begin{deflist}{marginbreak}
%    \item[plain]
%       This theorem style emulates the original \LaTeX{} definition,
%       except that additionally the parameters
%       |\theorem...skipamount| are used.
%    \item[break]
%       In this style, the theorem header is followed by a line
%       break.
%    \item[marginbreak]
%       The theorem number is set in the margin, and there is a line
%       break as in \texttt{break}.
%    \item[changebreak]
%       Like \texttt{break}, but with header number and text
%       interchanged.
%    \item[change]
%       Header number and text are interchanged, without a line break.
%    \item[margin]
%       The number is set in the left margin, without a line break.
% \end{deflist}
% All styles (except \texttt{plain}) select |\slshape| as the default
% |\theorembodyfont|.
%
%
% \subsection{Examples}
%
% Given the above theorem sets {\envfont Cor}, {\envfont Exa}
% and {\envfont Rem},
% suppose that the preamble also contains the declarations:
% \begin{verbatim}
%    \theoremstyle{marginbreak}   \newtheorem{Lem}[Cor]{Lemma}
%    \theoremstyle{change}
%    \theorembodyfont{\itshape}        \newtheorem{Def}[Cor]{Definition}
%
%    \theoremheaderfont{\scshape}
%\end{verbatim}
% Then the following are some typical examples of the typeset output
% resulting from their use.
%
% \begin{Cor}
%    This is a sentence typeset in the theorem environment {\envfont
%    Cor}.
% \end{Cor}
%
% \begin{Exa}
%    This is a sentence typeset in the theorem environment {\envfont
%    Exa}.
% \end{Exa}
%
% \begin{Rem}
%    This is a sentence typeset in the theorem environment {\envfont
%    Rem}.
% \end{Rem}
%
% \begin{Lem}[Ben User]
%    This is a sentence typeset in the theorem environment {\envfont
%    Lem}.
% \end{Lem}
%
% \begin{Def}[Very impressive Definition]
%    This is a sentence typeset in the theorem environment {\envfont
%    Def}.
% \end{Def}
%
% The last two examples show the effect of the optional argument to
% a theorem environment (it is the text typeset in parentheses).
%
%
% \section{Special Considerations}
% \label{sec:newfonts}
%
% Theoremheader and body are implemented as a unit. This means that
% the |\theoremheaderfont|  will inherit characteristics of the
% |\theorembodyfont| in \LaTeXe. Thus, if for example
% |\theorembodyfont| is |\itshape| and |\theoremheaderfont|
% is |\bfseries| the font selected for the header will have the
% characteristics `bold extended italic'. If this is not desired one
% should set the |\theoremheaderfont| to something like
% \begin{verbatim}
%      \theoremheaderfont{\normalfont\bfseries}
%\end{verbatim}
% i.e.\ supplying all necessary font information explicitly.
%
%
%
% \section{Acknowledgements}
%
% The publication of this set of macros was only possible with the
% help of Christina Busse (translating the manuscript into English),
% Joachim Pense (playing the r\^ole of typist), Chris Rowley
% (looking everything over) and many others providing useful
% suggestions.
%
% \MaybeStop{%
%    \begin{thebibliography}{9}
%       \bibitem{bk:GMS94} \textsc{M.~Goossens}, \textsc{F.~Mittelbach}
%         and \textsc{A.~Samarin}.
%         \newblock The \LaTeX{} Companion.
%         \newblock
%         Addison-Wesley, Reading, Massachusetts, 1994.
%       \bibitem{latex:source} \textsc{Lamport, Leslie.} \newblock
%                               \texttt{latex.tex}, version 2.09, date
%                               Feb. 1990.
%    \end{thebibliography}
%    \PrintIndex
%    \PrintChanges}
%
%
% \section{The documentation driver file}
%
% The next bit of code contains the documentation driver file for
% \TeX{}, i.e., the file that will produce the documentation you are
% currently reading. It will be extracted from this file by the
% \texttt{docstrip} program. Since it is the first code in the file
% one can alternatively process this file directly with \LaTeXe{} to
% obtain the documentation.
%
%    \begin{macrocode}
%<*driver>
\documentclass{ltxdoc}

\usepackage{theorem}

%<+driver>% The next few lines define theorem sets which are used
%<+driver>% in the example section of the documentation.

\theoremstyle{break}       \newtheorem{Cor}{Corollary}
\theoremstyle{plain}       \newtheorem{Exa}{Example}[section]
{\theorembodyfont{\upshape}\newtheorem{Rem}{Remark}}
\theoremstyle{marginbreak} \newtheorem{Lem}[Cor]{Lemma}
\theoremstyle{change}
\theorembodyfont{\itshape} \newtheorem{Def}[Cor]{Definition}

\theoremheaderfont{\scshape}

\RecordChanges

\begin{document}
   \DocInput{theorem.dtx}
\end{document}
%</driver>
%    \end{macrocode}
%
% \section{Definition of the Macros}
%
% If the file has been loaded before, we abort immediately. If not the
% package announces itself (this is actually done at the very top of
% the file---the way it is done isn't good style so don't copy it).
% \changes{v2.0e}{89/07/19}{Spaces removed from `typeout'.}
% \changes{v2.2b}{95/11/19}{Announce moved to top of file}
%    \begin{macrocode}
%<*package>
%\@ifundefined{theorem@style}{}{\endinput}
%\def\FMithmInfo{1995/11/19 v2.2b Theorem extension package (FMi)}
%\ProvidesPackage{theorem}[\FMithmInfo]
%    \end{macrocode}
%
%
% \subsection{Definition of theorem styles and fonts}
%
% All the definitions in this file are done globally to allow
% inputting this file inside a group.
%
% \begin{macro}{\theoremstyle}
%    Before a theorem style can be installed, the chosen style must be
%    known. For that reason, we must test to see that
%    |\th@|\meta{style} is known or, more precisely, that it is
%    different from |\relax|. If the style is not known then
%    |\th@plain| is used.
%    \begin{macrocode}
\gdef\theoremstyle#1{%
   \@ifundefined{th@#1}{\@warning
          {Unknown theoremstyle `#1'. Using `plain'}%
          \theorem@style{plain}}%
%    \end{macrocode}
%    We save the theorem style to be used in the token register
%    |\theorem@style|.
%    \begin{macrocode}
      {\theorem@style{#1}}%
%    \end{macrocode}
%    Now we ``evaluate'' the theorem style: this means, we call the
%    macro |\th@|\meta{style} which will activate the relevant
%    definitions which are contained in a separate file.  This is done
%    in a group to suppress changes to the current font.  This could
%    otherwise pose problems together with the new font selection
%    scheme\footnote{When I printed the original article using the new
%    font selection scheme I ended with a document with slanted
%    typefaces (text headings and all) simply because one of the
%    theorem styles used \texttt{\bslash sl} at toplevel.} if the
%    |\th@|\meta{style} is evaluated a second time.
% \changes{v2.1a}{90/02/15}{Group added around `th@..}
% \changes{v2.1a}{90/02/15}{Reseting of `..@begintheorem removed.}
%    \begin{macrocode}
      \begingroup
        \csname th@\the\theorem@style \endcsname
      \endgroup}
%    \end{macrocode}
% \end{macro}
%
%
% \begin{macro}{\@begintheorem}
% \changes{v2.1a}{90/02/15}{Relaxed at toplevel.}
% \begin{macro}{\@opargbegintheorem}
% \changes{v2.1a}{90/02/15}{Relaxed at toplevel.}
%    We reset |\@begintheorem| and |\@opargbegintheorem| to
%    |\relax| since these commands are no longer necessary at
%    toplevel. This will save a few tokens.
%    \begin{macrocode}
\global\let\@begintheorem\relax
\global\let\@opargbegintheorem\relax
%    \end{macrocode}
% \end{macro}
% \end{macro}
%
% \begin{macro}{\theorem@style}
%    Obviously the \textsf{token} register used above has to be
%    allocated. To assure the utmost compatibility with the
%    original \LaTeX{} definition, we set the default theorem style to
%    \texttt{plain}, which implements the usual \LaTeX{} convention.
% \changes{v2.0f}{89/07/20}{Unnecessary `global' removed.}
%    \begin{macrocode}
\newtoks\theorem@style
\global\theorem@style{plain}
%    \end{macrocode}
% \end{macro}
%
% \begin{macro}{\theorembodyfont}
% \begin{macro}{\theorem@bodyfont}
%    If the |\theorembodyfont| is set by the user then it should not
%    interact with the default font set in the theorem style. When the
%    new font selection is in force this may happen if, for example,
%    the default is |\itshape| and the new |\theorembodyfont| is
%    |\sffamily|. So we add a |\reset@font| command in front of the
%    user definition.
% \changes{v2.0f}{89/07/20}{Unnecessary `global' removed.}
% \changes{v2.0g}{89/09/19}{`theoremfont to `theorembodyfont.}
% \changes{v2.1c}{92/06/22}{`reset@font added and logic changed.}
% \changes{v2.1d}{92/10/28}{Reset to empty if arg is empty}
%    \begin{macrocode}
\gdef\theorembodyfont#1{%
%    \end{macrocode}
%    We check if the argument supplied is empty and if so put nothing
%    into the |\theorem@bodyfont| token register to allow for
%    |\theorembodyfont{}| as a mean of using the default of the
%    current |\theoremstyle|.
%    \begin{macrocode}
   \def\@tempa{#1}%
   \ifx\@tempa\@empty
    \theorem@bodyfont{}%
   \else
    \theorem@bodyfont{\reset@font#1}%
   \fi
   }
\newtoks\theorem@bodyfont
\global\theorem@bodyfont{}
%    \end{macrocode}
% \end{macro}
% \end{macro}
%
%
% \begin{macro}{\theoremheaderfont}
% \changes{v2.0g}{89/09/19}{Macro added.}
%    The font for the theorem headers is handled differently because
%    this definition applies to all theorem styles.
%    \begin{macrocode}
\gdef\theoremheaderfont#1{\gdef\theorem@headerfont{#1}%
%    \end{macrocode}
%    After using the macro once it is redefined to produce an error
%    message.
%    \begin{macrocode}
       \gdef\theoremheaderfont##1{%
        \typeout{\string\theoremheaderfont\space should be used
                 only once.}}}
%    \end{macrocode}
% \end{macro}
%
% \begin{macro}{\theorem@headerfont}
% \changes{v2.0g}{89/09/19}{Macro added.}
%    To set the |\theorem@headerfont| default we first test if
%    the new fontselection scheme is in force.
% \changes{v2.1a}{90/02/15}{Support for new font selection.}
%    \begin{macrocode}
\ifx\upshape\undefined
%    \end{macrocode}
%    If not we define it to expand into |\bfseries|. We don't use
%    |\let| just in case a following style option redefines this
%    macro.
%    \begin{macrocode}
\gdef\theorem@headerfont{\bfseries}
%    \end{macrocode}
%    Otherwise we reset the current shape before calling |\bfseries|.
%    \begin{macrocode}
\else \gdef\theorem@headerfont{\normalfont\bfseries}\fi
%    \end{macrocode}
% \end{macro}
%
% \begin{macro}{\th@plain}
% \begin{macro}{\th@break}
% \begin{macro}{\th@marginbreak}
% \begin{macro}{\th@changebreak}
% \begin{macro}{\th@change}
% \begin{macro}{\th@margin}
%    The different styles are defined in macros such as
%    |\th@plain|.  Since memory space is precious in
%    ``non-Big-versions'', we have to avoid offering too many unused
%    definitions.  Therefore we define these styles in separate files
%    that can be loaded on demand.  Thus the commands themselves only
%    load these files. We use |\@input@|, a \LaTeXe{} internal command
%    that ensures that the file will be listed with |\listfiles|.
% \changes{v2.2a}{94/02/02}{Use `@input@ to load theorem layout files}
%    \begin{macrocode}
\gdef\th@plain{\@input@{thp.sty}}
\gdef\th@break{\@input@{thb.sty}}
\gdef\th@marginbreak{\@input@{thmb.sty}}
\gdef\th@changebreak{\@input@{thcb.sty}}
\gdef\th@change{\@input@{thc.sty}}
\gdef\th@margin{\@input@{thm.sty}}
%    \end{macrocode}
%    This list will be expanded when new styles become
%    available.  For testing,  just append new theorem substyles
%    as document options.
% \end{macro}
% \end{macro}
% \end{macro}
% \end{macro}
% \end{macro}
% \end{macro}
%
%
%
% \subsection{Definition of a new theorem set}
%
% As already pointed out, a new theorem environment can be defined in
% three different ways:
% \begin{verbatim}
%    \newtheorem{Lem}{Lemma}
%    \newtheorem{Lem}{Lemma}[section]
%    \newtheorem{Lem}[Theorem]{Lemma}
%\end{verbatim}
% The function of the macro |\newtheorem| is to recognize these cases
% and then to branch into one of the three macros |\@ynthm|, |\@xnthm|
% or |\@othm|.  This mechanism is adopted unchanged from
% \cite{latex:source}; the essential point here is that, for example,
% in the second case, the arguments \texttt{Lem}, \texttt{Lemma} and
% \texttt{section} are passed over to the macro |\@xnthm|.
%
% We inspect this case first because the others present fewer problems,
% and thus are easily derived from this one.
%
% \begin{macro}{\@xnthm}
%    For our example arguments, the macro |\@xnthm| must fulfill
%    the following:
%    \begin{itemize}
%       \item
%          Define a new \LaTeX-counter `Lem'
%       \item
%          reset this counter within a |\section|
%       \item
%          define the macro |\theLem|
%       \item
%          define the environment macros |\Lem| and
%          |\endLem| using the current
%          |\theoremstyle| and |\theorem@bodyfont|.
%    \end{itemize}
%    Obviously, all this should happen only if the first argument
%    of |\@xnthm| (i.e.\ \texttt{Lem} in our example) is chosen so
%    as not to conflict with any previously defined commands or
%    environments.  This test is performed by the \LaTeX{} macro
%    |\@ifdefinable|.
%    \begin{macrocode}
\gdef\@xnthm#1#2[#3]{\expandafter\@ifdefinable\csname #1\endcsname
%    \end{macrocode}
%    Therefore, the first argument of |\@ifdefinable| is the
%    expansion (in the example, |\Lem|) of
%    |\csname||#1\endcsname|.  The second argument is
%    executed only if the test has been completed successfully.
%    \begin{macrocode}
   {%
%    \end{macrocode}
%    Now we define the new counter. The names of the \LaTeX{}
%    macros employed should speak for themselves:
% \changes{v2.2b}{1995/10/07}{Use \cs{@newctr}}
% \changes{v2.2c}{1995/11/23}{Use \cs{@newctr} properly.}
%    \begin{macrocode}
    \@definecounter{#1}\@newctr{#1}[#3]%
%    \end{macrocode}
%    Using |\@newctr| will give a proper error message if the counter
%    in |#3| is not defined.
%    In defining `|\theLem|' we must generate the desired macro
%    name by use of |\expandafter| and |\csname|.
%    \begin{macrocode}
    \expandafter\xdef\csname the#1\endcsname
%    \end{macrocode}
%    An |\xdef| is used in order to make the definition global,
%    and to ensure that it contains the replacement texts of
%    |\@thmcountersep| and |\@thmcounter|.\footnote{These
%    two macros can be defined by the document style.  Their default
%    values produce a `\texttt{.}' as separation and an arabic
%    representation of the number.} However, not everything should be
%    expanded. For example, it saves space to use |\thesection|
%    instead of its---at times---lengthy expansion.
%    \begin{macrocode}
      {\expandafter \noexpand \csname the#3\endcsname
       \@thmcountersep \@thmcounter{#1}}%
%    \end{macrocode}
%    Thus with the defaults of \LaTeX{}, |\theLem| would be
%    replaced by the command sequence |\thesection.\arabic{Lem}|.
%
%    We will now look at the definition of the macro which is executed
%    at the beginning of the actual environment (in our example this
%    macro is |\Lem|). It should be noted that we use an
%    ``|\expandafter| trick'' to expand only certain parts of the
%    replacement text at the time of the definition.
%    \begin{macrocode}
    \def\@tempa{\global\@namedef{#1}}%
    \expandafter \@tempa \expandafter{%
%    \end{macrocode}
%    First, the macro that contains the current definitions of
%    |\@begintheorem| and |\@opargtheorem| should be called
%    up.  The name of this macro---as is already known---has the form
%    |\th@|\meta{theorem\ style}; therefore, it must be called by
%    \begin{macrocode}
      \csname th@\the \theorem@style
%    \end{macrocode}
%    In addition the default theorem font should be changeable, i.e.\
%    we have to insert the contents of |\theorem@bodyfont|. For
%    that reason, we expand even further, beyond |\endcsname|,
%    and thus insert the contents of the \textsf{token} register
%    |\theorem@bodyfont| in the replacement text.
%    \begin{macrocode}
            \expandafter \endcsname \the \theorem@bodyfont
%    \end{macrocode}
%    Now it is time to call the macro |\@thm| which takes over
%    the further processing.  It has two arguments: the current
%    counter name (in our example, \texttt{Lem}), and the text of the
%    label (in our example, \texttt{Lemma}).
%    \begin{macrocode}
     \@thm{#1}{#2}}%
%    \end{macrocode}
%    With this, the `sub-definition' is complete.  The macro
%    |\@endtheorem| ends a theorem environment and is, so far, nothing
%    but an |\endtrivlist|.  (Hence it is defined globally, and not
%    within the theorem styles.\footnote{This has to be changed as
%    soon as theorem styles that change \texttt{\bslash @endtheorem}
%    exist.  In such a case, all existing styles must be changed as
%    well since they will have to reset the macro.}) Therefore, we can
%    set it equivalent to the macro that ends the theorem set (in our
%    example, |\endLem|).  However, if some day theorem styles exist
%    that do change |\@endtheorem|, we would have to use the
%    commented-out line instead.
%    \begin{macrocode}
    \global \expandafter \let \csname end#1\endcsname \@endtheorem
%   \global\@namedef{end#1}{\@endtheorem}%
%    \end{macrocode}
%    With these commands all the required definitions are employed,
%    unless the test |\@ifdefinable| has failed.  Therefore, we
%    end the second argument of this macro and with it the definition
%    of |\@xnthm|.
%    \begin{macrocode}
   }}
%    \end{macrocode}
% \end{macro}
%
% \begin{macro}{\@ynthm}
%    The definition of |\@ynthm| is completely analogous. In
%    this case the new counter that is defined is not reset within
%    another counter; thus the definition of |\the...|
%    is simplified:
%    \begin{macrocode}
\gdef\@ynthm#1#2{\expandafter\@ifdefinable\csname #1\endcsname
   {\@definecounter{#1}%
    \expandafter\xdef\csname the#1\endcsname{\@thmcounter{#1}}%
%    \end{macrocode}
%    The rest of the definition corresponds literally to that of
%    |\@xnthm|:
%    \begin{macrocode}
    \def\@tempa{\global\@namedef{#1}}\expandafter \@tempa
     \expandafter{\csname th@\the \theorem@style \expandafter
     \endcsname \the\theorem@bodyfont \@thm{#1}{#2}}%
    \global \expandafter \let \csname end#1\endcsname \@endtheorem}}
%    \end{macrocode}
% \end{macro}
%
% \begin{macro}{\@othm}
%    The definition of |\@othm| does not contain anything new.
%    \begin{macrocode}
\gdef\@othm#1[#2]#3{%
%    \end{macrocode}
% \changes{v2.2b}{95/10/07}{Build in check for counter defined}
%    We do not define a new counter but instead use one that has
%    already been defined.  Thus the only definition we need is that
%    of this pseudo-counter (i.e.\ |\the|\meta{env.\ name}).
%    First we check if |#2| corresponds to a known counter name.
%    \begin{macrocode}
  \expandafter\ifx\csname c@#2\endcsname\relax
   \@nocounterr{#2}%
  \else
   \expandafter\@ifdefinable\csname #1\endcsname
   {\expandafter \xdef \csname the#1\endcsname
     {\expandafter \noexpand \csname the#2\endcsname}%
%    \end{macrocode}
%    All other parts of the definition can be adopted from
%    |\@xnthm|. We have to remember, though, that in this case
%    the name of the current counter and the theorem label have moved
%    to the second and third arguments.
%    \begin{macrocode}
    \def\@tempa{\global\@namedef{#1}}\expandafter \@tempa
     \expandafter{\csname th@\the \theorem@style \expandafter
     \endcsname \the\theorem@bodyfont \@thm{#2}{#3}}%
    \global \expandafter \let \csname end#1\endcsname \@endtheorem}%
  \fi}
%    \end{macrocode}
% \end{macro}
%
% \subsection{Macros that are employed in a {\envfont theorem}
%    environment}
%
% \begin{macro}{\@thm}
%    The macro |\@thm| has to increase the current counter.  Then,
%    depending on whether the environment has (or does not have) an
%    optional argument, it has to branch into either |\@begintheorem|
%    or |\@opargtheorem|.
%    \begin{macrocode}
\gdef\@thm#1#2{\refstepcounter{#1}%
%    \end{macrocode}
%    Now we start a {\envfont trivlist} environment, and give
%    |\@topsep| and |\@topsepadd| the values of the \textsf{skip}
%    registers |\theorempreskipamount| and |\theorempostskipamount|.
%    The value in |\@topsep| is the vertical space that is inserted by
%    the first (and only) |\item| in our |\trivlist| whilst
%    |\@topsepadd| is inserted by |\@endparenv| at the end of that
%    {\envfont trivlist} environment.  By using these registers, we
%    obtain the desired space around a {\envfont theorem} environment.
%    \begin{macrocode}
   \trivlist
   \@topsep \theorempreskipamount               % used by first \item
   \@topsepadd \theorempostskipamount           % used by \@endparenv
%    \end{macrocode}
%    Now we have to test whether an optional argument has been given.
% \changes{v2.0d}{89/07/18}{Forgotten percent added.}
%    \begin{macrocode}
   \@ifnextchar [%
%    \end{macrocode}
%    If there is an optional argument, we will call |\@ythm|, and
%    move the arguments read back into the input stream.
%    \begin{macrocode}
   {\@ythm{#1}{#2}}%
%    \end{macrocode}
%    If not, we call |\@begintheorem|. Its first argument is the
%    name of the theorem set (hence the second argument of
%    |\@thm|). Its second argument is the macro that produces the
%    current number.
%    \begin{macrocode}
   {\@begintheorem{#2}{\csname the#1\endcsname}\ignorespaces}}
%    \end{macrocode}
% \end{macro}
%
% \begin{macro}{\@xthm}
% \begin{macro}{\@ythm}
%    Both these macros were originally called by |\@thm|.  We do
%    not need |\@xthm| anymore, hence we reset it to
%    |\relax|. The definition of |\@ythm| has not changed at
%    all from its definition in \LaTeX. In order to make the macros
%    easier to understand, we will
%    nevertheless present it (commented out).
%    \begin{macrocode}
\global\let\@xthm\relax
% \def\@ythm#1#2[#3]{\@opargbegintheorem{#2}{\csname
%           the#1\endcsname}{#3}\ignorespaces}
%</package>
%    \end{macrocode}
%    The primitive |\ignorespaces| in |\@ythm| and
%    |\@thm| is needed to remove the spaces between the
%    |\begin{...}| and the actual text.
% \end{macro}
% \end{macro}
%
%
% \subsection{Definition of the theorem substyles}
%
% As already pointed out, the theorem substyles, defined below, are
% only loaded when necessary.  Note that all these substyles, except
% \texttt{plain}, have |\slshape| as the default body font.
%
% \subsubsection{The \texttt{plain} style}
%
% \label{dok:th@plain} As the following macros use |@|, we have
% to locally set the |\catcode| of this symbol to ``letter''.
% This happens within a group, so that we do not have to worry about
% which |\catcode| that symbol had before.
%    \begin{macrocode}
%<*thp>
\begingroup \makeatletter
%    \end{macrocode}
% Since we are now within a group, we must make all definitions
% globally.  First we make sure that \texttt{theorem.sty} is loaded.
% This will allow us to use this file as a document style option
% without having to call \texttt{theorem} itself as an option.  At the
% same time, we assure that at least version 2 is loaded, since
% |\theorem@style| was not defined in earlier versions.
%    \begin{macrocode}
\@ifundefined{theorem@style}{\input{theorem.sty}}{}
\ProvidesFile{thp.sty}
             [\FMithmInfo]
%    \end{macrocode}
%
%
% \begin{macro}{\th@plain}
% \begin{macro}{\@begintheorem}
% \begin{macro}{\@opargbegintheorem}
%    |\theoremstyle{plain}| corresponds to the original
%    definition, except that the distances to the surrounding text are
%    determined by the parameters |\theorempreskipamount| and
%    |\theorempostskipamount|.
%    First we set the default body font.
%    \begin{macrocode}
\gdef\th@plain{\normalfont\itshape
%    \end{macrocode}
%    Then we define |\@begintheorem| and
%    |\@opargbegintheorem|. These two macros define how the
%    header of a theorem is typeset.  |\@opargbegintheorem| will
%    be called if a \textsf{theorem} environment with an optional
%    argument is encountered; otherwise, the header is constructed by
%    calling |\@begintheorem|.  If one of these macros is
%    executed, we are within a \textsf{trivlist} environment started by
%    |\@thm|.  So the theorem header is produced with an
%    |\item| command.
%
%    Instead of specifying the header font directly, all standard
%    theorem styles use the |\theorem@headerfont| macro to allow
%    customization.  The extra space (|\labelsep|) is necessary
%    because of problems in the \textsf{trivlist} environment.
%    \begin{macrocode}
  \def\@begintheorem##1##2{%
        \item[\hskip\labelsep \theorem@headerfont ##1\ ##2]}%
%    \end{macrocode}
%    The definition of |\@opargbegintheorem| is completely
%    analogous. The only difference is the fact that there exists a
%    third argument (which is the optional parameter of the
%    environment and contains additional information about the
%    theorem).  Customarily we enclose it in parentheses.
%    \begin{macrocode}
\def\@opargbegintheorem##1##2##3{%
   \item[\hskip\labelsep \theorem@headerfont ##1\ ##2\ (##3)]}}
%    \end{macrocode}
% \end{macro}
% \end{macro}
% \end{macro}
%
% \noindent
% We conclude with an |\endgroup| to restore the
% |\catcode| of |@|.
%    \begin{macrocode}
\endgroup
%</thp>
%    \end{macrocode}
%
%
% \subsubsection{The \texttt{break} style}
%
% This style option is stored in the file |thb.sty|.
% For the next two lines see the documentation for |\th@plain| on
% page~\pageref{dok:th@plain}.
%    \begin{macrocode}
%<*thb>
\begingroup \makeatletter
\@ifundefined{theorem@style}{\input{theorem.sty}}{}
\ProvidesFile{thb.sty}
             [\FMithmInfo]
%    \end{macrocode}
%
% \begin{macro}{\th@break}
%    |\theoremstyle{break}| produces a line break after the name
%    of the theorem.  The font is |\slshape|.  Hence, we define
%    |\th@break| as follows:
%    \begin{macrocode}
\gdef\th@break{\normalfont\slshape
  \def\@begintheorem##1##2{\item[%
%    \end{macrocode}
%    We run into the following problem: it is not possible to create
%    the header with |\item[|\meta{title}|]| and then start
%    a new line by, for example, |\mbox{}\\|. Such a definition
%    will fail whenever a list environment follows immediately.  With
%    the above construction, the |\mbox{}| causes the switch
%    |@inlabel| (cf.\ definition of |\list| and
%    |\trivlist| in \cite{latex:source}) to be set to \texttt{false}
%    and so the following list will insert additional vertical
%    space (|\topskip|). This is quite annoying. Therefore, we
%    create the line break within the |\item|. In order to ensure
%    that the text will begin at the proper position in the following
%    line, we simply pretend that the label does not take any
%    room.\footnote{This will lead to problems whenever very high
%    symbols occurring in the line tower into the
%    heading. So, something else has to be done here sometime.}
%    \begin{macrocode}
       \rlap{\vbox{\hbox{\hskip \labelsep\theorem@headerfont ##1\ ##2}%
                   \hbox{\strut}}}]}%
%    \end{macrocode}
%    Again, the definition of |\@opargbegintheorem| is
%    completely analogous.
%    \begin{macrocode}
\def\@opargbegintheorem##1##2##3{%
  \item[\rlap{\vbox{\hbox{\hskip \labelsep \theorem@headerfont
                     ##1\ ##2\ (##3)}%
                    \hbox{\strut}}}]}}
\endgroup
%</thb>
%    \end{macrocode}
% \end{macro}
%
%
% \subsubsection{The \texttt{changebreak} style}
%
%    \begin{macrocode}
% This style option is stored in the file |thcb.sty|.
%    \begin{macrocode}
%<*thcb>
\begingroup \makeatletter
\@ifundefined{theorem@style}{\input{theorem.sty}}{}
\ProvidesFile{thcb.sty}
             [\FMithmInfo]
%    \end{macrocode}
%
% \begin{macro}{\th@changebreak}
%    The \texttt{change-break} theorem style is like \texttt{break}
%    but with interchange of theorem name and theorem number.  Thus we
%    define |\th@changebreak| as follows:
%    \begin{macrocode}
\gdef\th@changebreak{\normalfont\slshape
  \def\@begintheorem##1##2{\item
       [\rlap{\vbox{\hbox{\hskip\labelsep \theorem@headerfont ##2\ ##1}%
                    \hbox{\strut}}}]}%
\def\@opargbegintheorem##1##2##3{%
  \item[\rlap{\vbox{\hbox{\hskip\labelsep \theorem@headerfont
                     ##2\ ##1\ (##3)}%
                    \hbox{\strut}}}]}}
\endgroup
%</thcb>
%    \end{macrocode}
% \end{macro}
%
%
% \subsubsection{The \texttt{change} style}
%
% This style option is stored in the file |thc.sty|.
%    \begin{macrocode}
%<*thc>
\begingroup \makeatletter
\@ifundefined{theorem@style}{\input{theorem.sty}}{}
\ProvidesFile{thc.sty}
             [\FMithmInfo]
%    \end{macrocode}
%
% \begin{macro}{\th@change}
%    The \texttt{change} theorem style corresponds to the
%    \texttt{change break} style without a linebreak after the header.
%    To say it in another way, it's the same as the \texttt{plain}
%    style but with number and name interchanged and |\slshape| as the
%    default font.
%    \begin{macrocode}
\gdef\th@change{\normalfont\slshape
  \def\@begintheorem##1##2{\item
       [\hskip\labelsep \theorem@headerfont ##2\ ##1]}%
\def\@opargbegintheorem##1##2##3{%
  \item[\hskip\labelsep \theorem@headerfont ##2\ ##1\ (##3)]}}
\endgroup
%</thc>
%    \end{macrocode}
% \end{macro}
%
%
% \subsubsection{The \texttt{marginbreak} style}
%
% This style option is the one used most often at Mainz.  It is saved
% in the file |thmb.sty|.
%    \begin{macrocode}
%<*thmb>
\begingroup \makeatletter
\@ifundefined{theorem@style}{\input{theorem.sty}}{}
\ProvidesFile{thmb.sty}
             [\FMithmInfo]
%    \end{macrocode}
%
% \begin{macro}{\th@marginbreak}
%    The \texttt{margin break} style is nearly the same as the
%    \texttt{change break} style. The only difference is the placement
%    of the theorem number.  We use |\llap| to place it in the left
%    margin.
%
%    In this style |\labelsep| denotes the separation between the
%    number and the text.
%    \begin{macrocode}
\gdef\th@marginbreak{\normalfont\slshape
  \def\@begintheorem##1##2{\item
       [\rlap{\vbox{\theorem@headerfont
                    \hbox{\llap{##2}\hskip\labelsep ##1}%
                    \hbox{\strut}}}]}%
\def\@opargbegintheorem##1##2##3{%
  \item[\rlap{\vbox{\theorem@headerfont
                    \hbox{\llap{##2}\hskip\labelsep ##1\ (##3)}%
                    \hbox{\strut}}}]}}
\endgroup
%</thmb>
%    \end{macrocode}
% \changes{v2.1c}{90/02/17}{`endgroup outside macrocode env.}
% \end{macro}
%
%
%
% \subsubsection{The \texttt{margin} style}
%
% This style option is stored in the file |thm.sty|.
%    \begin{macrocode}
%<*thm>
\begingroup \makeatletter
\@ifundefined{theorem@style}{\input{theorem.sty}}{}
\ProvidesFile{thm.sty}
             [\FMithmInfo]
%    \end{macrocode}
%
% \begin{macro}{\th@margin}
%    Again this is only a variant of the theorem styles described
%    above without any new ideas.
%    \begin{macrocode}
\gdef\th@margin{\normalfont\slshape
  \def\@begintheorem##1##2{\item
       [\theorem@headerfont \llap{##2}\hskip\labelsep ##1]}%
\def\@opargbegintheorem##1##2##3{%
  \item[\theorem@headerfont \llap{##2}\hskip\labelsep ##1\ (##3)]}}
\endgroup
%</thm>
%    \end{macrocode}
% \end{macro}
%
%
%
%
% \subsection{Final Definitions}
%
% \begin{macro}{\theorempreskipamount}
% \begin{macro}{\theorempostskipamount}
%    The \textsf{skip} parameters that regulate the vertical empty space
%    before and after the {\envfont theorem} environment have
%    to be allocated as well.
% \changes{v2.0f}{89/07/20}{Unnecessary `global' removed.}
%    \begin{macrocode}
%<*package>
\newskip\theorempreskipamount
\newskip\theorempostskipamount
%    \end{macrocode}
%    Since we have used the same values for all theorem sets, we now can
%    assign them.
%    \begin{macrocode}
\global\setlength\theorempreskipamount{12pt plus 5pt minus 3pt}
\global\setlength\theorempostskipamount{8pt plus 3pt minus 1.5pt}
%    \end{macrocode}
% \end{macro}
% \end{macro}
%
%
% \begin{macro}{\@endtheorem}
%    The same holds for the macro |\@endtheorem|, which ends a
%    {\envfont theorem} environment. Since it is the same for all
%    theorem sets, it is removed from the macros |\th@|\meta{style}.
%    It simply ends the {\envfont trivlist} environment, which was
%    begun in |\@thm|.
%    \begin{macrocode}
\global\let\@endtheorem=\endtrivlist
%    \end{macrocode}
% \end{macro}
%
%
%
% \begin{macro}{\@preamblecmds}
%    All macros defined above are to be used only in the preamble.
%    Therefore, we insert them in |\@preamblecmds| which will disable
%    them at begin document. This is done by the internal \LaTeXe{}
%    command |\@onlypreamble|.
% \changes{v2.0g}{89/09/19}{`theoremheaderfont and `theorembodyfont
%    added.}
% \changes{v2.2a}{94/02/03}{Use \LaTeXe{} interface}
%    \begin{macrocode}
\@onlypreamble\@xnthm
\@onlypreamble\@ynthm
\@onlypreamble\@othm
\@onlypreamble\newtheorem
\@onlypreamble\theoremstyle
\@onlypreamble\theorembodyfont
\@onlypreamble\theoremheaderfont
%    \end{macrocode}
% \end{macro}
%
% Finally we declare the \texttt{plain} theorem style to be the default.
% \changes{v2.1b}{90/06/04}{Fixed missing `item by declaring default
%    style.}
%    \begin{macrocode}
\theoremstyle{plain}
%</package>
%    \end{macrocode}
%
% \Finale

\endinput
}{}
\ProvidesFile{thcb.sty}
             [\FMithmInfo]
%    \end{macrocode}
%
% \begin{macro}{\th@changebreak}
%    The \texttt{change-break} theorem style is like \texttt{break}
%    but with interchange of theorem name and theorem number.  Thus we
%    define |\th@changebreak| as follows:
%    \begin{macrocode}
\gdef\th@changebreak{\normalfont\slshape
  \def\@begintheorem##1##2{\item
       [\rlap{\vbox{\hbox{\hskip\labelsep \theorem@headerfont ##2\ ##1}%
                    \hbox{\strut}}}]}%
\def\@opargbegintheorem##1##2##3{%
  \item[\rlap{\vbox{\hbox{\hskip\labelsep \theorem@headerfont
                     ##2\ ##1\ (##3)}%
                    \hbox{\strut}}}]}}
\endgroup
%</thcb>
%    \end{macrocode}
% \end{macro}
%
%
% \subsubsection{The \texttt{change} style}
%
% This style option is stored in the file |thc.sty|.
%    \begin{macrocode}
%<*thc>
\begingroup \makeatletter
\@ifundefined{theorem@style}{% \iffalse meta-comment
%
% Copyright (C) 1993-2024
%
% The LaTeX Project and any individual authors listed elsewhere
% in this file.
%
% This file is part of the Standard LaTeX `Tools Bundle'.
% -------------------------------------------------------
%
% It may be distributed and/or modified under the
% conditions of the LaTeX Project Public License, either version 1.3c
% of this license or (at your option) any later version.
% The latest version of this license is in
%    https://www.latex-project.org/lppl.txt
% and version 1.3c or later is part of all distributions of LaTeX
% version 2005/12/01 or later.
%
% The list of all files belonging to the LaTeX `Tools Bundle' is
% given in the file `manifest.txt'.
%
% \fi
%
%%
%
%
\def\FMithmInfo{2023/07/05 v2.2c Theorem extension package (FMi)}
%
% \ProvidesFile{theorem.dtx}[\FMithmInfo]
%\iffalse   % this is a METACOMMENT !
%%
%% Package `theorem' to use with LaTeX2e
%% Copyright (C) 1989-1995 Frank Mittelbach, all rights reserved.
%<+package>\@ifundefined{theorem@style}{}{\endinput}
%<+package>\NeedsTeXFormat{LaTeX2e}[1995/06/01]
%<+package>\ProvidesPackage{theorem}
%<+package>         [\FMithmInfo]
%
% \fi
%
% \changes{v1.0}{87/05/07}{First draft.}
% \changes{v1.0a}{87/05/07}{`thistheoremstyle' added.}
% \changes{v1.1a}{87/07/29}{New concept: redefine the internal
%                          \protect\LaTeX{} commands `@xnthm' etc.}
% \changes{v1.1b}{87/07/31}{`theoremfont' used to specify font.}
% \changes{v1.2a}{88/01/08}{`@endtheorem' now globally defined.}
% \changes{v1.2b}{88/01/11}{`newtheorem' only allowed in the preamble.}
% \changes{v1.3a}{88/02/14}{`theoremfont' now a toks register.}
%
% \changes{v2.0a}{88/11/16}{New implementation and new documentation.}
% \changes{v2.0a}{88/11/16}{No layout changes but internal modifications
%                           old substyles won't work.}
% \changes{v2.0b}{89/01/30}{`global' used for all definitions.}
% \changes{v2.0c}{89/06/15}{Documentation translated to English.
%                            Thanks to LLL and Nina}
% \changes{v2.1a}{90/02/15}{Support for new font selection.}
% \changes{v2.2a}{94/02/03}{Cleanup for LaTeX2e}
%
% \DoNotIndex{\@endparenv,\@ifdefinable,\@ifnextchar,\@ifundefined}
% \DoNotIndex{\@namedef,\@nameuse,\@spaces,\@tempa,\@warning}
% \DoNotIndex{\ }
% \DoNotIndex{\begingroup,\bfseries}
% \DoNotIndex{\csname}
% \DoNotIndex{\def,\do,\docdate}
% \DoNotIndex{\endcsname,\endgroup,\endinput,\endtrivlist,\expandafter}
% \DoNotIndex{\filedate,\fileversion}
% \DoNotIndex{\gdef,\global}
% \DoNotIndex{\hbox,\hskip}
% \DoNotIndex{\ignorespaces,\input,\itshape,\item}
% \DoNotIndex{\labelsep,\let,\llap}
% \DoNotIndex{\makeatletter}
% \DoNotIndex{\noexpand}
% \DoNotIndex{\refstepcounter,\relax,\rlap}
% \DoNotIndex{\setlength,\slshape,\space,\string,\strut}
% \DoNotIndex{\the,\trivlist,\typeout}
% \DoNotIndex{\vbox}
% \DoNotIndex{\xdef}
%
% \def\deflabel#1{\ttfamily #1\hfill}
% \def\deflist#1{\begin{list}{}{\settowidth\labelwidth{\ttfamily #1}%
%                               \setlength\leftmargin\labelwidth
%                               \addtolength\leftmargin\labelsep
%                               \let\makelabel\deflabel}}
% \def\enddeflist{\end{list}}
% \def\envfont{\normalfont\ttfamily}
%
% \MakeShortVerb{\|}
%
% \newcommand{\TUB}{{\sl TUGboat\/}}
%
% \GetFileInfo{theorem.dtx}
% \title{An Extension of the
%        \LaTeX{} theorem environment\thanks{This file
%        has version number \fileversion, last
%        revised \filedate.}}
% \author{Frank Mittelbach}
%
% \MaintainedByLaTeXTeam{tools}
% \maketitle
%
% \begin{abstract}
%    The macros described in this paper yield an extension of the
%    \LaTeX{} theorem mechanism. It is designed to satisfy the
%    different requirements of various journals. Thus, the
%    layout of the ``theorems'' can be manipulated by determining a
%    ``style''. This article describes not only the use, but
%    also the definition, of the necessary macros.
% \end{abstract}
%
% \section*{Preface to version 2.2}
%
% For \LaTeXe{} this package did not need any fundamental changes. I
% only modified the messages generated so that theorem layout styles
% will show up with the |\listfiles| command and cleaned the section
% on the New Font Selection Scheme since this is now included in
% \LaTeX.
%
% \section*{Preface to version 2.1}
%
% This version is identical to 2.0g described in \TUB\ 10\#3 except
% for some internal defaults which are now set depending on the used
% font selection scheme.
%
% This was done to avoid unpleasant surprises if the new font selection
% scheme is in force. For further details see
% section~\ref{sec:newfonts} and \cite{bk:GMS94}.
%
% \section{Introduction}
%
% For our purposes here, ``theorems'' are labelled enunciations,
% often set off from the main text by extra space and a font change.
% Theorems, corollaries, conjectures, definitions, and remarks
% are all instances of ``theorems''.  The ``header'' of these structures
% is composed of a label (such as \textsc{Theorem} or \textsc{Remark})
% and a number which serializes an item in the sequence of items
% with the same label.
%
% Shortly after the introduction of \LaTeX{} at the Fachbereich
% Mathematik in Mainz, the desire to manipulate the layout of
% ``theorems'' arose.  In Mainz, the following two conventions came
% into general use:
% \begin{enumerate}
%   \item The number of the theorem is shown in the margin.
%   \item There is a line break at
%         the end of the theorem header.
% \end{enumerate}
%
% Additionally, some journals require different formats which depend on
% the ``sort of theorem'': e.g.\ often remarks and definitions are set
% in |\upshape|, while |\itshape| is employed for main theorems.
%
% Confronted with these requirements, a theorem environment was
% developed in Mainz which allows separate determination of the
% layout of the ``theorems sets'', comparable to |\pagestyle|.
%
%
% \section{The user interface}
% \subsection{Defining new theorem sets}
%
% \DescribeMacro\newtheorem
% As in the original \LaTeX{} version, the command |\newtheorem|
% defines a new ``theorem set'' or ``theorem-like structure''.
% Two required arguments name the new environment and give the text
% to be typeset with each instance of the new ``set'', while
% an optional argument determines how
% the ``set'' is enumerated:
% \begin{description}
%    \item[\ttfamily \bslash newtheorem\{foo\}\{bar\}]
%       The theorem set {\envfont foo} (whose name is \texttt{bar})
%       uses its own counter.
%    \item[\ttfamily \bslash newtheorem\{foo2\}{[foo]}\{bar2\}]
%       The theorem set {\envfont foo2} (printed name \texttt{bar2})
%       uses the same counter as the theorem set \texttt{foo}.
%    \item[\ttfamily \bslash newtheorem\{foo3\}\{bar3\}{[section]}]
%       The theorem set {\envfont foo3} (printed name \texttt{bar3}) is
%       enumerated within the counter \texttt{section}, i.e.\ with every
%       new |\section| the enumeration begins again with 1, and
%       the enumeration is composed from the section-number and the
%       theorem counter itself.
% \end{description}
%
% \DescribeMacro\theoremstyle
% Additionally, the command |\theoremstyle| can define the layout
% of various, or all, theorem sets.  It should be noted that any
% theorem set defined by |\newtheorem| is typeset in the
% |\theoremstyle| that is current at the time of the definition.
% Thus, the following
% \begin{verbatim}
%    \theoremstyle{break}        \newtheorem{Cor}{Corollary}
%    \theoremstyle{plain}        \newtheorem{Exa}{Example}[section]
%\end{verbatim}                             ^^A No blank at line start
% leads to the result that the set {\envfont Cor} is formatted in the
% style \texttt{break}, while the set {\envfont Exa} and all the
% following ones are formatted in the style \texttt{plain}, unless
% another |\theoremstyle| follows.  Since the definitions installed by
% |\newtheorem| are global, one also can limit |\theoremstyle| locally
% by grouping braces.
%
% \DescribeMacro\theorembodyfont
% The choice of the font for the theorem body is completely
% independent of the chosen |\theoremstyle|; this has proven to
% be very advantageous.  For example,
% \begin{verbatim}
%   {\theorembodyfont{\upshape}          \newtheorem{Rem}{Remark}}
%\end{verbatim}
% defines a theorem set {\envfont Rem}, which will be set in
% |\upshape| in the current layout (which in our example is
% \texttt{plain}).  As with |\theoremstyle|, the |\theorembodyfont|
% chosen is that current at the time of |\newtheorem|.  If
% |\theorembodyfont| is not specified or one defines
% |\theorembodyfont{}|, then the font used will be that defined by the
% |\theoremstyle|.
%
% \DescribeMacro\theoremheaderfont It is also possible to customize
% the font used for the theorem headers.  This is, however, a global
% declaration, and therefore there should be at most one
% |\theoremheaderfont| declaration in the preamble.\footnote{If it is
% actually necessary to have different header fonts, one has to define
% new theorem styles (substituting the desired font) or specify the
% information directly in the \texttt{\bslash newtheorem} declaration
% (the unclean variant).}
%
% \DescribeMacro\theorempreskipamount
% \DescribeMacro\theorempostskipamount Two additional parameters
% affect the vertical space around the theorem environments:
% |\theorempreskipamount| and |\theorempostskipamount| define,
% respectively, the spacing before and after such an environment.
% These parameters apply for all theorem sets and can be manipulated
% with the ordinary length macros.  They are rubber lengths
% (`\textsf{skips}'), and therefore can contain \texttt{plus} and
% \texttt{minus} parts.
%
% Since the definition of theorem sets should---most sensibly---be
% placed in the preamble, we only allow installation there.
% It is therefore possible to release the memory used here after
% |\begin{document}|, in order to make room for other
% applications.
%
%
% \subsection{Existing theorem styles}
%
% The following theorem styles exist to date:
% \begin{deflist}{marginbreak}
%    \item[plain]
%       This theorem style emulates the original \LaTeX{} definition,
%       except that additionally the parameters
%       |\theorem...skipamount| are used.
%    \item[break]
%       In this style, the theorem header is followed by a line
%       break.
%    \item[marginbreak]
%       The theorem number is set in the margin, and there is a line
%       break as in \texttt{break}.
%    \item[changebreak]
%       Like \texttt{break}, but with header number and text
%       interchanged.
%    \item[change]
%       Header number and text are interchanged, without a line break.
%    \item[margin]
%       The number is set in the left margin, without a line break.
% \end{deflist}
% All styles (except \texttt{plain}) select |\slshape| as the default
% |\theorembodyfont|.
%
%
% \subsection{Examples}
%
% Given the above theorem sets {\envfont Cor}, {\envfont Exa}
% and {\envfont Rem},
% suppose that the preamble also contains the declarations:
% \begin{verbatim}
%    \theoremstyle{marginbreak}   \newtheorem{Lem}[Cor]{Lemma}
%    \theoremstyle{change}
%    \theorembodyfont{\itshape}        \newtheorem{Def}[Cor]{Definition}
%
%    \theoremheaderfont{\scshape}
%\end{verbatim}
% Then the following are some typical examples of the typeset output
% resulting from their use.
%
% \begin{Cor}
%    This is a sentence typeset in the theorem environment {\envfont
%    Cor}.
% \end{Cor}
%
% \begin{Exa}
%    This is a sentence typeset in the theorem environment {\envfont
%    Exa}.
% \end{Exa}
%
% \begin{Rem}
%    This is a sentence typeset in the theorem environment {\envfont
%    Rem}.
% \end{Rem}
%
% \begin{Lem}[Ben User]
%    This is a sentence typeset in the theorem environment {\envfont
%    Lem}.
% \end{Lem}
%
% \begin{Def}[Very impressive Definition]
%    This is a sentence typeset in the theorem environment {\envfont
%    Def}.
% \end{Def}
%
% The last two examples show the effect of the optional argument to
% a theorem environment (it is the text typeset in parentheses).
%
%
% \section{Special Considerations}
% \label{sec:newfonts}
%
% Theoremheader and body are implemented as a unit. This means that
% the |\theoremheaderfont|  will inherit characteristics of the
% |\theorembodyfont| in \LaTeXe. Thus, if for example
% |\theorembodyfont| is |\itshape| and |\theoremheaderfont|
% is |\bfseries| the font selected for the header will have the
% characteristics `bold extended italic'. If this is not desired one
% should set the |\theoremheaderfont| to something like
% \begin{verbatim}
%      \theoremheaderfont{\normalfont\bfseries}
%\end{verbatim}
% i.e.\ supplying all necessary font information explicitly.
%
%
%
% \section{Acknowledgements}
%
% The publication of this set of macros was only possible with the
% help of Christina Busse (translating the manuscript into English),
% Joachim Pense (playing the r\^ole of typist), Chris Rowley
% (looking everything over) and many others providing useful
% suggestions.
%
% \MaybeStop{%
%    \begin{thebibliography}{9}
%       \bibitem{bk:GMS94} \textsc{M.~Goossens}, \textsc{F.~Mittelbach}
%         and \textsc{A.~Samarin}.
%         \newblock The \LaTeX{} Companion.
%         \newblock
%         Addison-Wesley, Reading, Massachusetts, 1994.
%       \bibitem{latex:source} \textsc{Lamport, Leslie.} \newblock
%                               \texttt{latex.tex}, version 2.09, date
%                               Feb. 1990.
%    \end{thebibliography}
%    \PrintIndex
%    \PrintChanges}
%
%
% \section{The documentation driver file}
%
% The next bit of code contains the documentation driver file for
% \TeX{}, i.e., the file that will produce the documentation you are
% currently reading. It will be extracted from this file by the
% \texttt{docstrip} program. Since it is the first code in the file
% one can alternatively process this file directly with \LaTeXe{} to
% obtain the documentation.
%
%    \begin{macrocode}
%<*driver>
\documentclass{ltxdoc}

\usepackage{theorem}

%<+driver>% The next few lines define theorem sets which are used
%<+driver>% in the example section of the documentation.

\theoremstyle{break}       \newtheorem{Cor}{Corollary}
\theoremstyle{plain}       \newtheorem{Exa}{Example}[section]
{\theorembodyfont{\upshape}\newtheorem{Rem}{Remark}}
\theoremstyle{marginbreak} \newtheorem{Lem}[Cor]{Lemma}
\theoremstyle{change}
\theorembodyfont{\itshape} \newtheorem{Def}[Cor]{Definition}

\theoremheaderfont{\scshape}

\RecordChanges

\begin{document}
   \DocInput{theorem.dtx}
\end{document}
%</driver>
%    \end{macrocode}
%
% \section{Definition of the Macros}
%
% If the file has been loaded before, we abort immediately. If not the
% package announces itself (this is actually done at the very top of
% the file---the way it is done isn't good style so don't copy it).
% \changes{v2.0e}{89/07/19}{Spaces removed from `typeout'.}
% \changes{v2.2b}{95/11/19}{Announce moved to top of file}
%    \begin{macrocode}
%<*package>
%\@ifundefined{theorem@style}{}{\endinput}
%\def\FMithmInfo{1995/11/19 v2.2b Theorem extension package (FMi)}
%\ProvidesPackage{theorem}[\FMithmInfo]
%    \end{macrocode}
%
%
% \subsection{Definition of theorem styles and fonts}
%
% All the definitions in this file are done globally to allow
% inputting this file inside a group.
%
% \begin{macro}{\theoremstyle}
%    Before a theorem style can be installed, the chosen style must be
%    known. For that reason, we must test to see that
%    |\th@|\meta{style} is known or, more precisely, that it is
%    different from |\relax|. If the style is not known then
%    |\th@plain| is used.
%    \begin{macrocode}
\gdef\theoremstyle#1{%
   \@ifundefined{th@#1}{\@warning
          {Unknown theoremstyle `#1'. Using `plain'}%
          \theorem@style{plain}}%
%    \end{macrocode}
%    We save the theorem style to be used in the token register
%    |\theorem@style|.
%    \begin{macrocode}
      {\theorem@style{#1}}%
%    \end{macrocode}
%    Now we ``evaluate'' the theorem style: this means, we call the
%    macro |\th@|\meta{style} which will activate the relevant
%    definitions which are contained in a separate file.  This is done
%    in a group to suppress changes to the current font.  This could
%    otherwise pose problems together with the new font selection
%    scheme\footnote{When I printed the original article using the new
%    font selection scheme I ended with a document with slanted
%    typefaces (text headings and all) simply because one of the
%    theorem styles used \texttt{\bslash sl} at toplevel.} if the
%    |\th@|\meta{style} is evaluated a second time.
% \changes{v2.1a}{90/02/15}{Group added around `th@..}
% \changes{v2.1a}{90/02/15}{Reseting of `..@begintheorem removed.}
%    \begin{macrocode}
      \begingroup
        \csname th@\the\theorem@style \endcsname
      \endgroup}
%    \end{macrocode}
% \end{macro}
%
%
% \begin{macro}{\@begintheorem}
% \changes{v2.1a}{90/02/15}{Relaxed at toplevel.}
% \begin{macro}{\@opargbegintheorem}
% \changes{v2.1a}{90/02/15}{Relaxed at toplevel.}
%    We reset |\@begintheorem| and |\@opargbegintheorem| to
%    |\relax| since these commands are no longer necessary at
%    toplevel. This will save a few tokens.
%    \begin{macrocode}
\global\let\@begintheorem\relax
\global\let\@opargbegintheorem\relax
%    \end{macrocode}
% \end{macro}
% \end{macro}
%
% \begin{macro}{\theorem@style}
%    Obviously the \textsf{token} register used above has to be
%    allocated. To assure the utmost compatibility with the
%    original \LaTeX{} definition, we set the default theorem style to
%    \texttt{plain}, which implements the usual \LaTeX{} convention.
% \changes{v2.0f}{89/07/20}{Unnecessary `global' removed.}
%    \begin{macrocode}
\newtoks\theorem@style
\global\theorem@style{plain}
%    \end{macrocode}
% \end{macro}
%
% \begin{macro}{\theorembodyfont}
% \begin{macro}{\theorem@bodyfont}
%    If the |\theorembodyfont| is set by the user then it should not
%    interact with the default font set in the theorem style. When the
%    new font selection is in force this may happen if, for example,
%    the default is |\itshape| and the new |\theorembodyfont| is
%    |\sffamily|. So we add a |\reset@font| command in front of the
%    user definition.
% \changes{v2.0f}{89/07/20}{Unnecessary `global' removed.}
% \changes{v2.0g}{89/09/19}{`theoremfont to `theorembodyfont.}
% \changes{v2.1c}{92/06/22}{`reset@font added and logic changed.}
% \changes{v2.1d}{92/10/28}{Reset to empty if arg is empty}
%    \begin{macrocode}
\gdef\theorembodyfont#1{%
%    \end{macrocode}
%    We check if the argument supplied is empty and if so put nothing
%    into the |\theorem@bodyfont| token register to allow for
%    |\theorembodyfont{}| as a mean of using the default of the
%    current |\theoremstyle|.
%    \begin{macrocode}
   \def\@tempa{#1}%
   \ifx\@tempa\@empty
    \theorem@bodyfont{}%
   \else
    \theorem@bodyfont{\reset@font#1}%
   \fi
   }
\newtoks\theorem@bodyfont
\global\theorem@bodyfont{}
%    \end{macrocode}
% \end{macro}
% \end{macro}
%
%
% \begin{macro}{\theoremheaderfont}
% \changes{v2.0g}{89/09/19}{Macro added.}
%    The font for the theorem headers is handled differently because
%    this definition applies to all theorem styles.
%    \begin{macrocode}
\gdef\theoremheaderfont#1{\gdef\theorem@headerfont{#1}%
%    \end{macrocode}
%    After using the macro once it is redefined to produce an error
%    message.
%    \begin{macrocode}
       \gdef\theoremheaderfont##1{%
        \typeout{\string\theoremheaderfont\space should be used
                 only once.}}}
%    \end{macrocode}
% \end{macro}
%
% \begin{macro}{\theorem@headerfont}
% \changes{v2.0g}{89/09/19}{Macro added.}
%    To set the |\theorem@headerfont| default we first test if
%    the new fontselection scheme is in force.
% \changes{v2.1a}{90/02/15}{Support for new font selection.}
%    \begin{macrocode}
\ifx\upshape\undefined
%    \end{macrocode}
%    If not we define it to expand into |\bfseries|. We don't use
%    |\let| just in case a following style option redefines this
%    macro.
%    \begin{macrocode}
\gdef\theorem@headerfont{\bfseries}
%    \end{macrocode}
%    Otherwise we reset the current shape before calling |\bfseries|.
%    \begin{macrocode}
\else \gdef\theorem@headerfont{\normalfont\bfseries}\fi
%    \end{macrocode}
% \end{macro}
%
% \begin{macro}{\th@plain}
% \begin{macro}{\th@break}
% \begin{macro}{\th@marginbreak}
% \begin{macro}{\th@changebreak}
% \begin{macro}{\th@change}
% \begin{macro}{\th@margin}
%    The different styles are defined in macros such as
%    |\th@plain|.  Since memory space is precious in
%    ``non-Big-versions'', we have to avoid offering too many unused
%    definitions.  Therefore we define these styles in separate files
%    that can be loaded on demand.  Thus the commands themselves only
%    load these files. We use |\@input@|, a \LaTeXe{} internal command
%    that ensures that the file will be listed with |\listfiles|.
% \changes{v2.2a}{94/02/02}{Use `@input@ to load theorem layout files}
%    \begin{macrocode}
\gdef\th@plain{\@input@{thp.sty}}
\gdef\th@break{\@input@{thb.sty}}
\gdef\th@marginbreak{\@input@{thmb.sty}}
\gdef\th@changebreak{\@input@{thcb.sty}}
\gdef\th@change{\@input@{thc.sty}}
\gdef\th@margin{\@input@{thm.sty}}
%    \end{macrocode}
%    This list will be expanded when new styles become
%    available.  For testing,  just append new theorem substyles
%    as document options.
% \end{macro}
% \end{macro}
% \end{macro}
% \end{macro}
% \end{macro}
% \end{macro}
%
%
%
% \subsection{Definition of a new theorem set}
%
% As already pointed out, a new theorem environment can be defined in
% three different ways:
% \begin{verbatim}
%    \newtheorem{Lem}{Lemma}
%    \newtheorem{Lem}{Lemma}[section]
%    \newtheorem{Lem}[Theorem]{Lemma}
%\end{verbatim}
% The function of the macro |\newtheorem| is to recognize these cases
% and then to branch into one of the three macros |\@ynthm|, |\@xnthm|
% or |\@othm|.  This mechanism is adopted unchanged from
% \cite{latex:source}; the essential point here is that, for example,
% in the second case, the arguments \texttt{Lem}, \texttt{Lemma} and
% \texttt{section} are passed over to the macro |\@xnthm|.
%
% We inspect this case first because the others present fewer problems,
% and thus are easily derived from this one.
%
% \begin{macro}{\@xnthm}
%    For our example arguments, the macro |\@xnthm| must fulfill
%    the following:
%    \begin{itemize}
%       \item
%          Define a new \LaTeX-counter `Lem'
%       \item
%          reset this counter within a |\section|
%       \item
%          define the macro |\theLem|
%       \item
%          define the environment macros |\Lem| and
%          |\endLem| using the current
%          |\theoremstyle| and |\theorem@bodyfont|.
%    \end{itemize}
%    Obviously, all this should happen only if the first argument
%    of |\@xnthm| (i.e.\ \texttt{Lem} in our example) is chosen so
%    as not to conflict with any previously defined commands or
%    environments.  This test is performed by the \LaTeX{} macro
%    |\@ifdefinable|.
%    \begin{macrocode}
\gdef\@xnthm#1#2[#3]{\expandafter\@ifdefinable\csname #1\endcsname
%    \end{macrocode}
%    Therefore, the first argument of |\@ifdefinable| is the
%    expansion (in the example, |\Lem|) of
%    |\csname||#1\endcsname|.  The second argument is
%    executed only if the test has been completed successfully.
%    \begin{macrocode}
   {%
%    \end{macrocode}
%    Now we define the new counter. The names of the \LaTeX{}
%    macros employed should speak for themselves:
% \changes{v2.2b}{1995/10/07}{Use \cs{@newctr}}
% \changes{v2.2c}{1995/11/23}{Use \cs{@newctr} properly.}
%    \begin{macrocode}
    \@definecounter{#1}\@newctr{#1}[#3]%
%    \end{macrocode}
%    Using |\@newctr| will give a proper error message if the counter
%    in |#3| is not defined.
%    In defining `|\theLem|' we must generate the desired macro
%    name by use of |\expandafter| and |\csname|.
%    \begin{macrocode}
    \expandafter\xdef\csname the#1\endcsname
%    \end{macrocode}
%    An |\xdef| is used in order to make the definition global,
%    and to ensure that it contains the replacement texts of
%    |\@thmcountersep| and |\@thmcounter|.\footnote{These
%    two macros can be defined by the document style.  Their default
%    values produce a `\texttt{.}' as separation and an arabic
%    representation of the number.} However, not everything should be
%    expanded. For example, it saves space to use |\thesection|
%    instead of its---at times---lengthy expansion.
%    \begin{macrocode}
      {\expandafter \noexpand \csname the#3\endcsname
       \@thmcountersep \@thmcounter{#1}}%
%    \end{macrocode}
%    Thus with the defaults of \LaTeX{}, |\theLem| would be
%    replaced by the command sequence |\thesection.\arabic{Lem}|.
%
%    We will now look at the definition of the macro which is executed
%    at the beginning of the actual environment (in our example this
%    macro is |\Lem|). It should be noted that we use an
%    ``|\expandafter| trick'' to expand only certain parts of the
%    replacement text at the time of the definition.
%    \begin{macrocode}
    \def\@tempa{\global\@namedef{#1}}%
    \expandafter \@tempa \expandafter{%
%    \end{macrocode}
%    First, the macro that contains the current definitions of
%    |\@begintheorem| and |\@opargtheorem| should be called
%    up.  The name of this macro---as is already known---has the form
%    |\th@|\meta{theorem\ style}; therefore, it must be called by
%    \begin{macrocode}
      \csname th@\the \theorem@style
%    \end{macrocode}
%    In addition the default theorem font should be changeable, i.e.\
%    we have to insert the contents of |\theorem@bodyfont|. For
%    that reason, we expand even further, beyond |\endcsname|,
%    and thus insert the contents of the \textsf{token} register
%    |\theorem@bodyfont| in the replacement text.
%    \begin{macrocode}
            \expandafter \endcsname \the \theorem@bodyfont
%    \end{macrocode}
%    Now it is time to call the macro |\@thm| which takes over
%    the further processing.  It has two arguments: the current
%    counter name (in our example, \texttt{Lem}), and the text of the
%    label (in our example, \texttt{Lemma}).
%    \begin{macrocode}
     \@thm{#1}{#2}}%
%    \end{macrocode}
%    With this, the `sub-definition' is complete.  The macro
%    |\@endtheorem| ends a theorem environment and is, so far, nothing
%    but an |\endtrivlist|.  (Hence it is defined globally, and not
%    within the theorem styles.\footnote{This has to be changed as
%    soon as theorem styles that change \texttt{\bslash @endtheorem}
%    exist.  In such a case, all existing styles must be changed as
%    well since they will have to reset the macro.}) Therefore, we can
%    set it equivalent to the macro that ends the theorem set (in our
%    example, |\endLem|).  However, if some day theorem styles exist
%    that do change |\@endtheorem|, we would have to use the
%    commented-out line instead.
%    \begin{macrocode}
    \global \expandafter \let \csname end#1\endcsname \@endtheorem
%   \global\@namedef{end#1}{\@endtheorem}%
%    \end{macrocode}
%    With these commands all the required definitions are employed,
%    unless the test |\@ifdefinable| has failed.  Therefore, we
%    end the second argument of this macro and with it the definition
%    of |\@xnthm|.
%    \begin{macrocode}
   }}
%    \end{macrocode}
% \end{macro}
%
% \begin{macro}{\@ynthm}
%    The definition of |\@ynthm| is completely analogous. In
%    this case the new counter that is defined is not reset within
%    another counter; thus the definition of |\the...|
%    is simplified:
%    \begin{macrocode}
\gdef\@ynthm#1#2{\expandafter\@ifdefinable\csname #1\endcsname
   {\@definecounter{#1}%
    \expandafter\xdef\csname the#1\endcsname{\@thmcounter{#1}}%
%    \end{macrocode}
%    The rest of the definition corresponds literally to that of
%    |\@xnthm|:
%    \begin{macrocode}
    \def\@tempa{\global\@namedef{#1}}\expandafter \@tempa
     \expandafter{\csname th@\the \theorem@style \expandafter
     \endcsname \the\theorem@bodyfont \@thm{#1}{#2}}%
    \global \expandafter \let \csname end#1\endcsname \@endtheorem}}
%    \end{macrocode}
% \end{macro}
%
% \begin{macro}{\@othm}
%    The definition of |\@othm| does not contain anything new.
%    \begin{macrocode}
\gdef\@othm#1[#2]#3{%
%    \end{macrocode}
% \changes{v2.2b}{95/10/07}{Build in check for counter defined}
%    We do not define a new counter but instead use one that has
%    already been defined.  Thus the only definition we need is that
%    of this pseudo-counter (i.e.\ |\the|\meta{env.\ name}).
%    First we check if |#2| corresponds to a known counter name.
%    \begin{macrocode}
  \expandafter\ifx\csname c@#2\endcsname\relax
   \@nocounterr{#2}%
  \else
   \expandafter\@ifdefinable\csname #1\endcsname
   {\expandafter \xdef \csname the#1\endcsname
     {\expandafter \noexpand \csname the#2\endcsname}%
%    \end{macrocode}
%    All other parts of the definition can be adopted from
%    |\@xnthm|. We have to remember, though, that in this case
%    the name of the current counter and the theorem label have moved
%    to the second and third arguments.
%    \begin{macrocode}
    \def\@tempa{\global\@namedef{#1}}\expandafter \@tempa
     \expandafter{\csname th@\the \theorem@style \expandafter
     \endcsname \the\theorem@bodyfont \@thm{#2}{#3}}%
    \global \expandafter \let \csname end#1\endcsname \@endtheorem}%
  \fi}
%    \end{macrocode}
% \end{macro}
%
% \subsection{Macros that are employed in a {\envfont theorem}
%    environment}
%
% \begin{macro}{\@thm}
%    The macro |\@thm| has to increase the current counter.  Then,
%    depending on whether the environment has (or does not have) an
%    optional argument, it has to branch into either |\@begintheorem|
%    or |\@opargtheorem|.
%    \begin{macrocode}
\gdef\@thm#1#2{\refstepcounter{#1}%
%    \end{macrocode}
%    Now we start a {\envfont trivlist} environment, and give
%    |\@topsep| and |\@topsepadd| the values of the \textsf{skip}
%    registers |\theorempreskipamount| and |\theorempostskipamount|.
%    The value in |\@topsep| is the vertical space that is inserted by
%    the first (and only) |\item| in our |\trivlist| whilst
%    |\@topsepadd| is inserted by |\@endparenv| at the end of that
%    {\envfont trivlist} environment.  By using these registers, we
%    obtain the desired space around a {\envfont theorem} environment.
%    \begin{macrocode}
   \trivlist
   \@topsep \theorempreskipamount               % used by first \item
   \@topsepadd \theorempostskipamount           % used by \@endparenv
%    \end{macrocode}
%    Now we have to test whether an optional argument has been given.
% \changes{v2.0d}{89/07/18}{Forgotten percent added.}
%    \begin{macrocode}
   \@ifnextchar [%
%    \end{macrocode}
%    If there is an optional argument, we will call |\@ythm|, and
%    move the arguments read back into the input stream.
%    \begin{macrocode}
   {\@ythm{#1}{#2}}%
%    \end{macrocode}
%    If not, we call |\@begintheorem|. Its first argument is the
%    name of the theorem set (hence the second argument of
%    |\@thm|). Its second argument is the macro that produces the
%    current number.
%    \begin{macrocode}
   {\@begintheorem{#2}{\csname the#1\endcsname}\ignorespaces}}
%    \end{macrocode}
% \end{macro}
%
% \begin{macro}{\@xthm}
% \begin{macro}{\@ythm}
%    Both these macros were originally called by |\@thm|.  We do
%    not need |\@xthm| anymore, hence we reset it to
%    |\relax|. The definition of |\@ythm| has not changed at
%    all from its definition in \LaTeX. In order to make the macros
%    easier to understand, we will
%    nevertheless present it (commented out).
%    \begin{macrocode}
\global\let\@xthm\relax
% \def\@ythm#1#2[#3]{\@opargbegintheorem{#2}{\csname
%           the#1\endcsname}{#3}\ignorespaces}
%</package>
%    \end{macrocode}
%    The primitive |\ignorespaces| in |\@ythm| and
%    |\@thm| is needed to remove the spaces between the
%    |\begin{...}| and the actual text.
% \end{macro}
% \end{macro}
%
%
% \subsection{Definition of the theorem substyles}
%
% As already pointed out, the theorem substyles, defined below, are
% only loaded when necessary.  Note that all these substyles, except
% \texttt{plain}, have |\slshape| as the default body font.
%
% \subsubsection{The \texttt{plain} style}
%
% \label{dok:th@plain} As the following macros use |@|, we have
% to locally set the |\catcode| of this symbol to ``letter''.
% This happens within a group, so that we do not have to worry about
% which |\catcode| that symbol had before.
%    \begin{macrocode}
%<*thp>
\begingroup \makeatletter
%    \end{macrocode}
% Since we are now within a group, we must make all definitions
% globally.  First we make sure that \texttt{theorem.sty} is loaded.
% This will allow us to use this file as a document style option
% without having to call \texttt{theorem} itself as an option.  At the
% same time, we assure that at least version 2 is loaded, since
% |\theorem@style| was not defined in earlier versions.
%    \begin{macrocode}
\@ifundefined{theorem@style}{\input{theorem.sty}}{}
\ProvidesFile{thp.sty}
             [\FMithmInfo]
%    \end{macrocode}
%
%
% \begin{macro}{\th@plain}
% \begin{macro}{\@begintheorem}
% \begin{macro}{\@opargbegintheorem}
%    |\theoremstyle{plain}| corresponds to the original
%    definition, except that the distances to the surrounding text are
%    determined by the parameters |\theorempreskipamount| and
%    |\theorempostskipamount|.
%    First we set the default body font.
%    \begin{macrocode}
\gdef\th@plain{\normalfont\itshape
%    \end{macrocode}
%    Then we define |\@begintheorem| and
%    |\@opargbegintheorem|. These two macros define how the
%    header of a theorem is typeset.  |\@opargbegintheorem| will
%    be called if a \textsf{theorem} environment with an optional
%    argument is encountered; otherwise, the header is constructed by
%    calling |\@begintheorem|.  If one of these macros is
%    executed, we are within a \textsf{trivlist} environment started by
%    |\@thm|.  So the theorem header is produced with an
%    |\item| command.
%
%    Instead of specifying the header font directly, all standard
%    theorem styles use the |\theorem@headerfont| macro to allow
%    customization.  The extra space (|\labelsep|) is necessary
%    because of problems in the \textsf{trivlist} environment.
%    \begin{macrocode}
  \def\@begintheorem##1##2{%
        \item[\hskip\labelsep \theorem@headerfont ##1\ ##2]}%
%    \end{macrocode}
%    The definition of |\@opargbegintheorem| is completely
%    analogous. The only difference is the fact that there exists a
%    third argument (which is the optional parameter of the
%    environment and contains additional information about the
%    theorem).  Customarily we enclose it in parentheses.
%    \begin{macrocode}
\def\@opargbegintheorem##1##2##3{%
   \item[\hskip\labelsep \theorem@headerfont ##1\ ##2\ (##3)]}}
%    \end{macrocode}
% \end{macro}
% \end{macro}
% \end{macro}
%
% \noindent
% We conclude with an |\endgroup| to restore the
% |\catcode| of |@|.
%    \begin{macrocode}
\endgroup
%</thp>
%    \end{macrocode}
%
%
% \subsubsection{The \texttt{break} style}
%
% This style option is stored in the file |thb.sty|.
% For the next two lines see the documentation for |\th@plain| on
% page~\pageref{dok:th@plain}.
%    \begin{macrocode}
%<*thb>
\begingroup \makeatletter
\@ifundefined{theorem@style}{\input{theorem.sty}}{}
\ProvidesFile{thb.sty}
             [\FMithmInfo]
%    \end{macrocode}
%
% \begin{macro}{\th@break}
%    |\theoremstyle{break}| produces a line break after the name
%    of the theorem.  The font is |\slshape|.  Hence, we define
%    |\th@break| as follows:
%    \begin{macrocode}
\gdef\th@break{\normalfont\slshape
  \def\@begintheorem##1##2{\item[%
%    \end{macrocode}
%    We run into the following problem: it is not possible to create
%    the header with |\item[|\meta{title}|]| and then start
%    a new line by, for example, |\mbox{}\\|. Such a definition
%    will fail whenever a list environment follows immediately.  With
%    the above construction, the |\mbox{}| causes the switch
%    |@inlabel| (cf.\ definition of |\list| and
%    |\trivlist| in \cite{latex:source}) to be set to \texttt{false}
%    and so the following list will insert additional vertical
%    space (|\topskip|). This is quite annoying. Therefore, we
%    create the line break within the |\item|. In order to ensure
%    that the text will begin at the proper position in the following
%    line, we simply pretend that the label does not take any
%    room.\footnote{This will lead to problems whenever very high
%    symbols occurring in the line tower into the
%    heading. So, something else has to be done here sometime.}
%    \begin{macrocode}
       \rlap{\vbox{\hbox{\hskip \labelsep\theorem@headerfont ##1\ ##2}%
                   \hbox{\strut}}}]}%
%    \end{macrocode}
%    Again, the definition of |\@opargbegintheorem| is
%    completely analogous.
%    \begin{macrocode}
\def\@opargbegintheorem##1##2##3{%
  \item[\rlap{\vbox{\hbox{\hskip \labelsep \theorem@headerfont
                     ##1\ ##2\ (##3)}%
                    \hbox{\strut}}}]}}
\endgroup
%</thb>
%    \end{macrocode}
% \end{macro}
%
%
% \subsubsection{The \texttt{changebreak} style}
%
%    \begin{macrocode}
% This style option is stored in the file |thcb.sty|.
%    \begin{macrocode}
%<*thcb>
\begingroup \makeatletter
\@ifundefined{theorem@style}{\input{theorem.sty}}{}
\ProvidesFile{thcb.sty}
             [\FMithmInfo]
%    \end{macrocode}
%
% \begin{macro}{\th@changebreak}
%    The \texttt{change-break} theorem style is like \texttt{break}
%    but with interchange of theorem name and theorem number.  Thus we
%    define |\th@changebreak| as follows:
%    \begin{macrocode}
\gdef\th@changebreak{\normalfont\slshape
  \def\@begintheorem##1##2{\item
       [\rlap{\vbox{\hbox{\hskip\labelsep \theorem@headerfont ##2\ ##1}%
                    \hbox{\strut}}}]}%
\def\@opargbegintheorem##1##2##3{%
  \item[\rlap{\vbox{\hbox{\hskip\labelsep \theorem@headerfont
                     ##2\ ##1\ (##3)}%
                    \hbox{\strut}}}]}}
\endgroup
%</thcb>
%    \end{macrocode}
% \end{macro}
%
%
% \subsubsection{The \texttt{change} style}
%
% This style option is stored in the file |thc.sty|.
%    \begin{macrocode}
%<*thc>
\begingroup \makeatletter
\@ifundefined{theorem@style}{\input{theorem.sty}}{}
\ProvidesFile{thc.sty}
             [\FMithmInfo]
%    \end{macrocode}
%
% \begin{macro}{\th@change}
%    The \texttt{change} theorem style corresponds to the
%    \texttt{change break} style without a linebreak after the header.
%    To say it in another way, it's the same as the \texttt{plain}
%    style but with number and name interchanged and |\slshape| as the
%    default font.
%    \begin{macrocode}
\gdef\th@change{\normalfont\slshape
  \def\@begintheorem##1##2{\item
       [\hskip\labelsep \theorem@headerfont ##2\ ##1]}%
\def\@opargbegintheorem##1##2##3{%
  \item[\hskip\labelsep \theorem@headerfont ##2\ ##1\ (##3)]}}
\endgroup
%</thc>
%    \end{macrocode}
% \end{macro}
%
%
% \subsubsection{The \texttt{marginbreak} style}
%
% This style option is the one used most often at Mainz.  It is saved
% in the file |thmb.sty|.
%    \begin{macrocode}
%<*thmb>
\begingroup \makeatletter
\@ifundefined{theorem@style}{\input{theorem.sty}}{}
\ProvidesFile{thmb.sty}
             [\FMithmInfo]
%    \end{macrocode}
%
% \begin{macro}{\th@marginbreak}
%    The \texttt{margin break} style is nearly the same as the
%    \texttt{change break} style. The only difference is the placement
%    of the theorem number.  We use |\llap| to place it in the left
%    margin.
%
%    In this style |\labelsep| denotes the separation between the
%    number and the text.
%    \begin{macrocode}
\gdef\th@marginbreak{\normalfont\slshape
  \def\@begintheorem##1##2{\item
       [\rlap{\vbox{\theorem@headerfont
                    \hbox{\llap{##2}\hskip\labelsep ##1}%
                    \hbox{\strut}}}]}%
\def\@opargbegintheorem##1##2##3{%
  \item[\rlap{\vbox{\theorem@headerfont
                    \hbox{\llap{##2}\hskip\labelsep ##1\ (##3)}%
                    \hbox{\strut}}}]}}
\endgroup
%</thmb>
%    \end{macrocode}
% \changes{v2.1c}{90/02/17}{`endgroup outside macrocode env.}
% \end{macro}
%
%
%
% \subsubsection{The \texttt{margin} style}
%
% This style option is stored in the file |thm.sty|.
%    \begin{macrocode}
%<*thm>
\begingroup \makeatletter
\@ifundefined{theorem@style}{\input{theorem.sty}}{}
\ProvidesFile{thm.sty}
             [\FMithmInfo]
%    \end{macrocode}
%
% \begin{macro}{\th@margin}
%    Again this is only a variant of the theorem styles described
%    above without any new ideas.
%    \begin{macrocode}
\gdef\th@margin{\normalfont\slshape
  \def\@begintheorem##1##2{\item
       [\theorem@headerfont \llap{##2}\hskip\labelsep ##1]}%
\def\@opargbegintheorem##1##2##3{%
  \item[\theorem@headerfont \llap{##2}\hskip\labelsep ##1\ (##3)]}}
\endgroup
%</thm>
%    \end{macrocode}
% \end{macro}
%
%
%
%
% \subsection{Final Definitions}
%
% \begin{macro}{\theorempreskipamount}
% \begin{macro}{\theorempostskipamount}
%    The \textsf{skip} parameters that regulate the vertical empty space
%    before and after the {\envfont theorem} environment have
%    to be allocated as well.
% \changes{v2.0f}{89/07/20}{Unnecessary `global' removed.}
%    \begin{macrocode}
%<*package>
\newskip\theorempreskipamount
\newskip\theorempostskipamount
%    \end{macrocode}
%    Since we have used the same values for all theorem sets, we now can
%    assign them.
%    \begin{macrocode}
\global\setlength\theorempreskipamount{12pt plus 5pt minus 3pt}
\global\setlength\theorempostskipamount{8pt plus 3pt minus 1.5pt}
%    \end{macrocode}
% \end{macro}
% \end{macro}
%
%
% \begin{macro}{\@endtheorem}
%    The same holds for the macro |\@endtheorem|, which ends a
%    {\envfont theorem} environment. Since it is the same for all
%    theorem sets, it is removed from the macros |\th@|\meta{style}.
%    It simply ends the {\envfont trivlist} environment, which was
%    begun in |\@thm|.
%    \begin{macrocode}
\global\let\@endtheorem=\endtrivlist
%    \end{macrocode}
% \end{macro}
%
%
%
% \begin{macro}{\@preamblecmds}
%    All macros defined above are to be used only in the preamble.
%    Therefore, we insert them in |\@preamblecmds| which will disable
%    them at begin document. This is done by the internal \LaTeXe{}
%    command |\@onlypreamble|.
% \changes{v2.0g}{89/09/19}{`theoremheaderfont and `theorembodyfont
%    added.}
% \changes{v2.2a}{94/02/03}{Use \LaTeXe{} interface}
%    \begin{macrocode}
\@onlypreamble\@xnthm
\@onlypreamble\@ynthm
\@onlypreamble\@othm
\@onlypreamble\newtheorem
\@onlypreamble\theoremstyle
\@onlypreamble\theorembodyfont
\@onlypreamble\theoremheaderfont
%    \end{macrocode}
% \end{macro}
%
% Finally we declare the \texttt{plain} theorem style to be the default.
% \changes{v2.1b}{90/06/04}{Fixed missing `item by declaring default
%    style.}
%    \begin{macrocode}
\theoremstyle{plain}
%</package>
%    \end{macrocode}
%
% \Finale

\endinput
}{}
\ProvidesFile{thc.sty}
             [\FMithmInfo]
%    \end{macrocode}
%
% \begin{macro}{\th@change}
%    The \texttt{change} theorem style corresponds to the
%    \texttt{change break} style without a linebreak after the header.
%    To say it in another way, it's the same as the \texttt{plain}
%    style but with number and name interchanged and |\slshape| as the
%    default font.
%    \begin{macrocode}
\gdef\th@change{\normalfont\slshape
  \def\@begintheorem##1##2{\item
       [\hskip\labelsep \theorem@headerfont ##2\ ##1]}%
\def\@opargbegintheorem##1##2##3{%
  \item[\hskip\labelsep \theorem@headerfont ##2\ ##1\ (##3)]}}
\endgroup
%</thc>
%    \end{macrocode}
% \end{macro}
%
%
% \subsubsection{The \texttt{marginbreak} style}
%
% This style option is the one used most often at Mainz.  It is saved
% in the file |thmb.sty|.
%    \begin{macrocode}
%<*thmb>
\begingroup \makeatletter
\@ifundefined{theorem@style}{% \iffalse meta-comment
%
% Copyright (C) 1993-2024
%
% The LaTeX Project and any individual authors listed elsewhere
% in this file.
%
% This file is part of the Standard LaTeX `Tools Bundle'.
% -------------------------------------------------------
%
% It may be distributed and/or modified under the
% conditions of the LaTeX Project Public License, either version 1.3c
% of this license or (at your option) any later version.
% The latest version of this license is in
%    https://www.latex-project.org/lppl.txt
% and version 1.3c or later is part of all distributions of LaTeX
% version 2005/12/01 or later.
%
% The list of all files belonging to the LaTeX `Tools Bundle' is
% given in the file `manifest.txt'.
%
% \fi
%
%%
%
%
\def\FMithmInfo{2023/07/05 v2.2c Theorem extension package (FMi)}
%
% \ProvidesFile{theorem.dtx}[\FMithmInfo]
%\iffalse   % this is a METACOMMENT !
%%
%% Package `theorem' to use with LaTeX2e
%% Copyright (C) 1989-1995 Frank Mittelbach, all rights reserved.
%<+package>\@ifundefined{theorem@style}{}{\endinput}
%<+package>\NeedsTeXFormat{LaTeX2e}[1995/06/01]
%<+package>\ProvidesPackage{theorem}
%<+package>         [\FMithmInfo]
%
% \fi
%
% \changes{v1.0}{87/05/07}{First draft.}
% \changes{v1.0a}{87/05/07}{`thistheoremstyle' added.}
% \changes{v1.1a}{87/07/29}{New concept: redefine the internal
%                          \protect\LaTeX{} commands `@xnthm' etc.}
% \changes{v1.1b}{87/07/31}{`theoremfont' used to specify font.}
% \changes{v1.2a}{88/01/08}{`@endtheorem' now globally defined.}
% \changes{v1.2b}{88/01/11}{`newtheorem' only allowed in the preamble.}
% \changes{v1.3a}{88/02/14}{`theoremfont' now a toks register.}
%
% \changes{v2.0a}{88/11/16}{New implementation and new documentation.}
% \changes{v2.0a}{88/11/16}{No layout changes but internal modifications
%                           old substyles won't work.}
% \changes{v2.0b}{89/01/30}{`global' used for all definitions.}
% \changes{v2.0c}{89/06/15}{Documentation translated to English.
%                            Thanks to LLL and Nina}
% \changes{v2.1a}{90/02/15}{Support for new font selection.}
% \changes{v2.2a}{94/02/03}{Cleanup for LaTeX2e}
%
% \DoNotIndex{\@endparenv,\@ifdefinable,\@ifnextchar,\@ifundefined}
% \DoNotIndex{\@namedef,\@nameuse,\@spaces,\@tempa,\@warning}
% \DoNotIndex{\ }
% \DoNotIndex{\begingroup,\bfseries}
% \DoNotIndex{\csname}
% \DoNotIndex{\def,\do,\docdate}
% \DoNotIndex{\endcsname,\endgroup,\endinput,\endtrivlist,\expandafter}
% \DoNotIndex{\filedate,\fileversion}
% \DoNotIndex{\gdef,\global}
% \DoNotIndex{\hbox,\hskip}
% \DoNotIndex{\ignorespaces,\input,\itshape,\item}
% \DoNotIndex{\labelsep,\let,\llap}
% \DoNotIndex{\makeatletter}
% \DoNotIndex{\noexpand}
% \DoNotIndex{\refstepcounter,\relax,\rlap}
% \DoNotIndex{\setlength,\slshape,\space,\string,\strut}
% \DoNotIndex{\the,\trivlist,\typeout}
% \DoNotIndex{\vbox}
% \DoNotIndex{\xdef}
%
% \def\deflabel#1{\ttfamily #1\hfill}
% \def\deflist#1{\begin{list}{}{\settowidth\labelwidth{\ttfamily #1}%
%                               \setlength\leftmargin\labelwidth
%                               \addtolength\leftmargin\labelsep
%                               \let\makelabel\deflabel}}
% \def\enddeflist{\end{list}}
% \def\envfont{\normalfont\ttfamily}
%
% \MakeShortVerb{\|}
%
% \newcommand{\TUB}{{\sl TUGboat\/}}
%
% \GetFileInfo{theorem.dtx}
% \title{An Extension of the
%        \LaTeX{} theorem environment\thanks{This file
%        has version number \fileversion, last
%        revised \filedate.}}
% \author{Frank Mittelbach}
%
% \MaintainedByLaTeXTeam{tools}
% \maketitle
%
% \begin{abstract}
%    The macros described in this paper yield an extension of the
%    \LaTeX{} theorem mechanism. It is designed to satisfy the
%    different requirements of various journals. Thus, the
%    layout of the ``theorems'' can be manipulated by determining a
%    ``style''. This article describes not only the use, but
%    also the definition, of the necessary macros.
% \end{abstract}
%
% \section*{Preface to version 2.2}
%
% For \LaTeXe{} this package did not need any fundamental changes. I
% only modified the messages generated so that theorem layout styles
% will show up with the |\listfiles| command and cleaned the section
% on the New Font Selection Scheme since this is now included in
% \LaTeX.
%
% \section*{Preface to version 2.1}
%
% This version is identical to 2.0g described in \TUB\ 10\#3 except
% for some internal defaults which are now set depending on the used
% font selection scheme.
%
% This was done to avoid unpleasant surprises if the new font selection
% scheme is in force. For further details see
% section~\ref{sec:newfonts} and \cite{bk:GMS94}.
%
% \section{Introduction}
%
% For our purposes here, ``theorems'' are labelled enunciations,
% often set off from the main text by extra space and a font change.
% Theorems, corollaries, conjectures, definitions, and remarks
% are all instances of ``theorems''.  The ``header'' of these structures
% is composed of a label (such as \textsc{Theorem} or \textsc{Remark})
% and a number which serializes an item in the sequence of items
% with the same label.
%
% Shortly after the introduction of \LaTeX{} at the Fachbereich
% Mathematik in Mainz, the desire to manipulate the layout of
% ``theorems'' arose.  In Mainz, the following two conventions came
% into general use:
% \begin{enumerate}
%   \item The number of the theorem is shown in the margin.
%   \item There is a line break at
%         the end of the theorem header.
% \end{enumerate}
%
% Additionally, some journals require different formats which depend on
% the ``sort of theorem'': e.g.\ often remarks and definitions are set
% in |\upshape|, while |\itshape| is employed for main theorems.
%
% Confronted with these requirements, a theorem environment was
% developed in Mainz which allows separate determination of the
% layout of the ``theorems sets'', comparable to |\pagestyle|.
%
%
% \section{The user interface}
% \subsection{Defining new theorem sets}
%
% \DescribeMacro\newtheorem
% As in the original \LaTeX{} version, the command |\newtheorem|
% defines a new ``theorem set'' or ``theorem-like structure''.
% Two required arguments name the new environment and give the text
% to be typeset with each instance of the new ``set'', while
% an optional argument determines how
% the ``set'' is enumerated:
% \begin{description}
%    \item[\ttfamily \bslash newtheorem\{foo\}\{bar\}]
%       The theorem set {\envfont foo} (whose name is \texttt{bar})
%       uses its own counter.
%    \item[\ttfamily \bslash newtheorem\{foo2\}{[foo]}\{bar2\}]
%       The theorem set {\envfont foo2} (printed name \texttt{bar2})
%       uses the same counter as the theorem set \texttt{foo}.
%    \item[\ttfamily \bslash newtheorem\{foo3\}\{bar3\}{[section]}]
%       The theorem set {\envfont foo3} (printed name \texttt{bar3}) is
%       enumerated within the counter \texttt{section}, i.e.\ with every
%       new |\section| the enumeration begins again with 1, and
%       the enumeration is composed from the section-number and the
%       theorem counter itself.
% \end{description}
%
% \DescribeMacro\theoremstyle
% Additionally, the command |\theoremstyle| can define the layout
% of various, or all, theorem sets.  It should be noted that any
% theorem set defined by |\newtheorem| is typeset in the
% |\theoremstyle| that is current at the time of the definition.
% Thus, the following
% \begin{verbatim}
%    \theoremstyle{break}        \newtheorem{Cor}{Corollary}
%    \theoremstyle{plain}        \newtheorem{Exa}{Example}[section]
%\end{verbatim}                             ^^A No blank at line start
% leads to the result that the set {\envfont Cor} is formatted in the
% style \texttt{break}, while the set {\envfont Exa} and all the
% following ones are formatted in the style \texttt{plain}, unless
% another |\theoremstyle| follows.  Since the definitions installed by
% |\newtheorem| are global, one also can limit |\theoremstyle| locally
% by grouping braces.
%
% \DescribeMacro\theorembodyfont
% The choice of the font for the theorem body is completely
% independent of the chosen |\theoremstyle|; this has proven to
% be very advantageous.  For example,
% \begin{verbatim}
%   {\theorembodyfont{\upshape}          \newtheorem{Rem}{Remark}}
%\end{verbatim}
% defines a theorem set {\envfont Rem}, which will be set in
% |\upshape| in the current layout (which in our example is
% \texttt{plain}).  As with |\theoremstyle|, the |\theorembodyfont|
% chosen is that current at the time of |\newtheorem|.  If
% |\theorembodyfont| is not specified or one defines
% |\theorembodyfont{}|, then the font used will be that defined by the
% |\theoremstyle|.
%
% \DescribeMacro\theoremheaderfont It is also possible to customize
% the font used for the theorem headers.  This is, however, a global
% declaration, and therefore there should be at most one
% |\theoremheaderfont| declaration in the preamble.\footnote{If it is
% actually necessary to have different header fonts, one has to define
% new theorem styles (substituting the desired font) or specify the
% information directly in the \texttt{\bslash newtheorem} declaration
% (the unclean variant).}
%
% \DescribeMacro\theorempreskipamount
% \DescribeMacro\theorempostskipamount Two additional parameters
% affect the vertical space around the theorem environments:
% |\theorempreskipamount| and |\theorempostskipamount| define,
% respectively, the spacing before and after such an environment.
% These parameters apply for all theorem sets and can be manipulated
% with the ordinary length macros.  They are rubber lengths
% (`\textsf{skips}'), and therefore can contain \texttt{plus} and
% \texttt{minus} parts.
%
% Since the definition of theorem sets should---most sensibly---be
% placed in the preamble, we only allow installation there.
% It is therefore possible to release the memory used here after
% |\begin{document}|, in order to make room for other
% applications.
%
%
% \subsection{Existing theorem styles}
%
% The following theorem styles exist to date:
% \begin{deflist}{marginbreak}
%    \item[plain]
%       This theorem style emulates the original \LaTeX{} definition,
%       except that additionally the parameters
%       |\theorem...skipamount| are used.
%    \item[break]
%       In this style, the theorem header is followed by a line
%       break.
%    \item[marginbreak]
%       The theorem number is set in the margin, and there is a line
%       break as in \texttt{break}.
%    \item[changebreak]
%       Like \texttt{break}, but with header number and text
%       interchanged.
%    \item[change]
%       Header number and text are interchanged, without a line break.
%    \item[margin]
%       The number is set in the left margin, without a line break.
% \end{deflist}
% All styles (except \texttt{plain}) select |\slshape| as the default
% |\theorembodyfont|.
%
%
% \subsection{Examples}
%
% Given the above theorem sets {\envfont Cor}, {\envfont Exa}
% and {\envfont Rem},
% suppose that the preamble also contains the declarations:
% \begin{verbatim}
%    \theoremstyle{marginbreak}   \newtheorem{Lem}[Cor]{Lemma}
%    \theoremstyle{change}
%    \theorembodyfont{\itshape}        \newtheorem{Def}[Cor]{Definition}
%
%    \theoremheaderfont{\scshape}
%\end{verbatim}
% Then the following are some typical examples of the typeset output
% resulting from their use.
%
% \begin{Cor}
%    This is a sentence typeset in the theorem environment {\envfont
%    Cor}.
% \end{Cor}
%
% \begin{Exa}
%    This is a sentence typeset in the theorem environment {\envfont
%    Exa}.
% \end{Exa}
%
% \begin{Rem}
%    This is a sentence typeset in the theorem environment {\envfont
%    Rem}.
% \end{Rem}
%
% \begin{Lem}[Ben User]
%    This is a sentence typeset in the theorem environment {\envfont
%    Lem}.
% \end{Lem}
%
% \begin{Def}[Very impressive Definition]
%    This is a sentence typeset in the theorem environment {\envfont
%    Def}.
% \end{Def}
%
% The last two examples show the effect of the optional argument to
% a theorem environment (it is the text typeset in parentheses).
%
%
% \section{Special Considerations}
% \label{sec:newfonts}
%
% Theoremheader and body are implemented as a unit. This means that
% the |\theoremheaderfont|  will inherit characteristics of the
% |\theorembodyfont| in \LaTeXe. Thus, if for example
% |\theorembodyfont| is |\itshape| and |\theoremheaderfont|
% is |\bfseries| the font selected for the header will have the
% characteristics `bold extended italic'. If this is not desired one
% should set the |\theoremheaderfont| to something like
% \begin{verbatim}
%      \theoremheaderfont{\normalfont\bfseries}
%\end{verbatim}
% i.e.\ supplying all necessary font information explicitly.
%
%
%
% \section{Acknowledgements}
%
% The publication of this set of macros was only possible with the
% help of Christina Busse (translating the manuscript into English),
% Joachim Pense (playing the r\^ole of typist), Chris Rowley
% (looking everything over) and many others providing useful
% suggestions.
%
% \MaybeStop{%
%    \begin{thebibliography}{9}
%       \bibitem{bk:GMS94} \textsc{M.~Goossens}, \textsc{F.~Mittelbach}
%         and \textsc{A.~Samarin}.
%         \newblock The \LaTeX{} Companion.
%         \newblock
%         Addison-Wesley, Reading, Massachusetts, 1994.
%       \bibitem{latex:source} \textsc{Lamport, Leslie.} \newblock
%                               \texttt{latex.tex}, version 2.09, date
%                               Feb. 1990.
%    \end{thebibliography}
%    \PrintIndex
%    \PrintChanges}
%
%
% \section{The documentation driver file}
%
% The next bit of code contains the documentation driver file for
% \TeX{}, i.e., the file that will produce the documentation you are
% currently reading. It will be extracted from this file by the
% \texttt{docstrip} program. Since it is the first code in the file
% one can alternatively process this file directly with \LaTeXe{} to
% obtain the documentation.
%
%    \begin{macrocode}
%<*driver>
\documentclass{ltxdoc}

\usepackage{theorem}

%<+driver>% The next few lines define theorem sets which are used
%<+driver>% in the example section of the documentation.

\theoremstyle{break}       \newtheorem{Cor}{Corollary}
\theoremstyle{plain}       \newtheorem{Exa}{Example}[section]
{\theorembodyfont{\upshape}\newtheorem{Rem}{Remark}}
\theoremstyle{marginbreak} \newtheorem{Lem}[Cor]{Lemma}
\theoremstyle{change}
\theorembodyfont{\itshape} \newtheorem{Def}[Cor]{Definition}

\theoremheaderfont{\scshape}

\RecordChanges

\begin{document}
   \DocInput{theorem.dtx}
\end{document}
%</driver>
%    \end{macrocode}
%
% \section{Definition of the Macros}
%
% If the file has been loaded before, we abort immediately. If not the
% package announces itself (this is actually done at the very top of
% the file---the way it is done isn't good style so don't copy it).
% \changes{v2.0e}{89/07/19}{Spaces removed from `typeout'.}
% \changes{v2.2b}{95/11/19}{Announce moved to top of file}
%    \begin{macrocode}
%<*package>
%\@ifundefined{theorem@style}{}{\endinput}
%\def\FMithmInfo{1995/11/19 v2.2b Theorem extension package (FMi)}
%\ProvidesPackage{theorem}[\FMithmInfo]
%    \end{macrocode}
%
%
% \subsection{Definition of theorem styles and fonts}
%
% All the definitions in this file are done globally to allow
% inputting this file inside a group.
%
% \begin{macro}{\theoremstyle}
%    Before a theorem style can be installed, the chosen style must be
%    known. For that reason, we must test to see that
%    |\th@|\meta{style} is known or, more precisely, that it is
%    different from |\relax|. If the style is not known then
%    |\th@plain| is used.
%    \begin{macrocode}
\gdef\theoremstyle#1{%
   \@ifundefined{th@#1}{\@warning
          {Unknown theoremstyle `#1'. Using `plain'}%
          \theorem@style{plain}}%
%    \end{macrocode}
%    We save the theorem style to be used in the token register
%    |\theorem@style|.
%    \begin{macrocode}
      {\theorem@style{#1}}%
%    \end{macrocode}
%    Now we ``evaluate'' the theorem style: this means, we call the
%    macro |\th@|\meta{style} which will activate the relevant
%    definitions which are contained in a separate file.  This is done
%    in a group to suppress changes to the current font.  This could
%    otherwise pose problems together with the new font selection
%    scheme\footnote{When I printed the original article using the new
%    font selection scheme I ended with a document with slanted
%    typefaces (text headings and all) simply because one of the
%    theorem styles used \texttt{\bslash sl} at toplevel.} if the
%    |\th@|\meta{style} is evaluated a second time.
% \changes{v2.1a}{90/02/15}{Group added around `th@..}
% \changes{v2.1a}{90/02/15}{Reseting of `..@begintheorem removed.}
%    \begin{macrocode}
      \begingroup
        \csname th@\the\theorem@style \endcsname
      \endgroup}
%    \end{macrocode}
% \end{macro}
%
%
% \begin{macro}{\@begintheorem}
% \changes{v2.1a}{90/02/15}{Relaxed at toplevel.}
% \begin{macro}{\@opargbegintheorem}
% \changes{v2.1a}{90/02/15}{Relaxed at toplevel.}
%    We reset |\@begintheorem| and |\@opargbegintheorem| to
%    |\relax| since these commands are no longer necessary at
%    toplevel. This will save a few tokens.
%    \begin{macrocode}
\global\let\@begintheorem\relax
\global\let\@opargbegintheorem\relax
%    \end{macrocode}
% \end{macro}
% \end{macro}
%
% \begin{macro}{\theorem@style}
%    Obviously the \textsf{token} register used above has to be
%    allocated. To assure the utmost compatibility with the
%    original \LaTeX{} definition, we set the default theorem style to
%    \texttt{plain}, which implements the usual \LaTeX{} convention.
% \changes{v2.0f}{89/07/20}{Unnecessary `global' removed.}
%    \begin{macrocode}
\newtoks\theorem@style
\global\theorem@style{plain}
%    \end{macrocode}
% \end{macro}
%
% \begin{macro}{\theorembodyfont}
% \begin{macro}{\theorem@bodyfont}
%    If the |\theorembodyfont| is set by the user then it should not
%    interact with the default font set in the theorem style. When the
%    new font selection is in force this may happen if, for example,
%    the default is |\itshape| and the new |\theorembodyfont| is
%    |\sffamily|. So we add a |\reset@font| command in front of the
%    user definition.
% \changes{v2.0f}{89/07/20}{Unnecessary `global' removed.}
% \changes{v2.0g}{89/09/19}{`theoremfont to `theorembodyfont.}
% \changes{v2.1c}{92/06/22}{`reset@font added and logic changed.}
% \changes{v2.1d}{92/10/28}{Reset to empty if arg is empty}
%    \begin{macrocode}
\gdef\theorembodyfont#1{%
%    \end{macrocode}
%    We check if the argument supplied is empty and if so put nothing
%    into the |\theorem@bodyfont| token register to allow for
%    |\theorembodyfont{}| as a mean of using the default of the
%    current |\theoremstyle|.
%    \begin{macrocode}
   \def\@tempa{#1}%
   \ifx\@tempa\@empty
    \theorem@bodyfont{}%
   \else
    \theorem@bodyfont{\reset@font#1}%
   \fi
   }
\newtoks\theorem@bodyfont
\global\theorem@bodyfont{}
%    \end{macrocode}
% \end{macro}
% \end{macro}
%
%
% \begin{macro}{\theoremheaderfont}
% \changes{v2.0g}{89/09/19}{Macro added.}
%    The font for the theorem headers is handled differently because
%    this definition applies to all theorem styles.
%    \begin{macrocode}
\gdef\theoremheaderfont#1{\gdef\theorem@headerfont{#1}%
%    \end{macrocode}
%    After using the macro once it is redefined to produce an error
%    message.
%    \begin{macrocode}
       \gdef\theoremheaderfont##1{%
        \typeout{\string\theoremheaderfont\space should be used
                 only once.}}}
%    \end{macrocode}
% \end{macro}
%
% \begin{macro}{\theorem@headerfont}
% \changes{v2.0g}{89/09/19}{Macro added.}
%    To set the |\theorem@headerfont| default we first test if
%    the new fontselection scheme is in force.
% \changes{v2.1a}{90/02/15}{Support for new font selection.}
%    \begin{macrocode}
\ifx\upshape\undefined
%    \end{macrocode}
%    If not we define it to expand into |\bfseries|. We don't use
%    |\let| just in case a following style option redefines this
%    macro.
%    \begin{macrocode}
\gdef\theorem@headerfont{\bfseries}
%    \end{macrocode}
%    Otherwise we reset the current shape before calling |\bfseries|.
%    \begin{macrocode}
\else \gdef\theorem@headerfont{\normalfont\bfseries}\fi
%    \end{macrocode}
% \end{macro}
%
% \begin{macro}{\th@plain}
% \begin{macro}{\th@break}
% \begin{macro}{\th@marginbreak}
% \begin{macro}{\th@changebreak}
% \begin{macro}{\th@change}
% \begin{macro}{\th@margin}
%    The different styles are defined in macros such as
%    |\th@plain|.  Since memory space is precious in
%    ``non-Big-versions'', we have to avoid offering too many unused
%    definitions.  Therefore we define these styles in separate files
%    that can be loaded on demand.  Thus the commands themselves only
%    load these files. We use |\@input@|, a \LaTeXe{} internal command
%    that ensures that the file will be listed with |\listfiles|.
% \changes{v2.2a}{94/02/02}{Use `@input@ to load theorem layout files}
%    \begin{macrocode}
\gdef\th@plain{\@input@{thp.sty}}
\gdef\th@break{\@input@{thb.sty}}
\gdef\th@marginbreak{\@input@{thmb.sty}}
\gdef\th@changebreak{\@input@{thcb.sty}}
\gdef\th@change{\@input@{thc.sty}}
\gdef\th@margin{\@input@{thm.sty}}
%    \end{macrocode}
%    This list will be expanded when new styles become
%    available.  For testing,  just append new theorem substyles
%    as document options.
% \end{macro}
% \end{macro}
% \end{macro}
% \end{macro}
% \end{macro}
% \end{macro}
%
%
%
% \subsection{Definition of a new theorem set}
%
% As already pointed out, a new theorem environment can be defined in
% three different ways:
% \begin{verbatim}
%    \newtheorem{Lem}{Lemma}
%    \newtheorem{Lem}{Lemma}[section]
%    \newtheorem{Lem}[Theorem]{Lemma}
%\end{verbatim}
% The function of the macro |\newtheorem| is to recognize these cases
% and then to branch into one of the three macros |\@ynthm|, |\@xnthm|
% or |\@othm|.  This mechanism is adopted unchanged from
% \cite{latex:source}; the essential point here is that, for example,
% in the second case, the arguments \texttt{Lem}, \texttt{Lemma} and
% \texttt{section} are passed over to the macro |\@xnthm|.
%
% We inspect this case first because the others present fewer problems,
% and thus are easily derived from this one.
%
% \begin{macro}{\@xnthm}
%    For our example arguments, the macro |\@xnthm| must fulfill
%    the following:
%    \begin{itemize}
%       \item
%          Define a new \LaTeX-counter `Lem'
%       \item
%          reset this counter within a |\section|
%       \item
%          define the macro |\theLem|
%       \item
%          define the environment macros |\Lem| and
%          |\endLem| using the current
%          |\theoremstyle| and |\theorem@bodyfont|.
%    \end{itemize}
%    Obviously, all this should happen only if the first argument
%    of |\@xnthm| (i.e.\ \texttt{Lem} in our example) is chosen so
%    as not to conflict with any previously defined commands or
%    environments.  This test is performed by the \LaTeX{} macro
%    |\@ifdefinable|.
%    \begin{macrocode}
\gdef\@xnthm#1#2[#3]{\expandafter\@ifdefinable\csname #1\endcsname
%    \end{macrocode}
%    Therefore, the first argument of |\@ifdefinable| is the
%    expansion (in the example, |\Lem|) of
%    |\csname||#1\endcsname|.  The second argument is
%    executed only if the test has been completed successfully.
%    \begin{macrocode}
   {%
%    \end{macrocode}
%    Now we define the new counter. The names of the \LaTeX{}
%    macros employed should speak for themselves:
% \changes{v2.2b}{1995/10/07}{Use \cs{@newctr}}
% \changes{v2.2c}{1995/11/23}{Use \cs{@newctr} properly.}
%    \begin{macrocode}
    \@definecounter{#1}\@newctr{#1}[#3]%
%    \end{macrocode}
%    Using |\@newctr| will give a proper error message if the counter
%    in |#3| is not defined.
%    In defining `|\theLem|' we must generate the desired macro
%    name by use of |\expandafter| and |\csname|.
%    \begin{macrocode}
    \expandafter\xdef\csname the#1\endcsname
%    \end{macrocode}
%    An |\xdef| is used in order to make the definition global,
%    and to ensure that it contains the replacement texts of
%    |\@thmcountersep| and |\@thmcounter|.\footnote{These
%    two macros can be defined by the document style.  Their default
%    values produce a `\texttt{.}' as separation and an arabic
%    representation of the number.} However, not everything should be
%    expanded. For example, it saves space to use |\thesection|
%    instead of its---at times---lengthy expansion.
%    \begin{macrocode}
      {\expandafter \noexpand \csname the#3\endcsname
       \@thmcountersep \@thmcounter{#1}}%
%    \end{macrocode}
%    Thus with the defaults of \LaTeX{}, |\theLem| would be
%    replaced by the command sequence |\thesection.\arabic{Lem}|.
%
%    We will now look at the definition of the macro which is executed
%    at the beginning of the actual environment (in our example this
%    macro is |\Lem|). It should be noted that we use an
%    ``|\expandafter| trick'' to expand only certain parts of the
%    replacement text at the time of the definition.
%    \begin{macrocode}
    \def\@tempa{\global\@namedef{#1}}%
    \expandafter \@tempa \expandafter{%
%    \end{macrocode}
%    First, the macro that contains the current definitions of
%    |\@begintheorem| and |\@opargtheorem| should be called
%    up.  The name of this macro---as is already known---has the form
%    |\th@|\meta{theorem\ style}; therefore, it must be called by
%    \begin{macrocode}
      \csname th@\the \theorem@style
%    \end{macrocode}
%    In addition the default theorem font should be changeable, i.e.\
%    we have to insert the contents of |\theorem@bodyfont|. For
%    that reason, we expand even further, beyond |\endcsname|,
%    and thus insert the contents of the \textsf{token} register
%    |\theorem@bodyfont| in the replacement text.
%    \begin{macrocode}
            \expandafter \endcsname \the \theorem@bodyfont
%    \end{macrocode}
%    Now it is time to call the macro |\@thm| which takes over
%    the further processing.  It has two arguments: the current
%    counter name (in our example, \texttt{Lem}), and the text of the
%    label (in our example, \texttt{Lemma}).
%    \begin{macrocode}
     \@thm{#1}{#2}}%
%    \end{macrocode}
%    With this, the `sub-definition' is complete.  The macro
%    |\@endtheorem| ends a theorem environment and is, so far, nothing
%    but an |\endtrivlist|.  (Hence it is defined globally, and not
%    within the theorem styles.\footnote{This has to be changed as
%    soon as theorem styles that change \texttt{\bslash @endtheorem}
%    exist.  In such a case, all existing styles must be changed as
%    well since they will have to reset the macro.}) Therefore, we can
%    set it equivalent to the macro that ends the theorem set (in our
%    example, |\endLem|).  However, if some day theorem styles exist
%    that do change |\@endtheorem|, we would have to use the
%    commented-out line instead.
%    \begin{macrocode}
    \global \expandafter \let \csname end#1\endcsname \@endtheorem
%   \global\@namedef{end#1}{\@endtheorem}%
%    \end{macrocode}
%    With these commands all the required definitions are employed,
%    unless the test |\@ifdefinable| has failed.  Therefore, we
%    end the second argument of this macro and with it the definition
%    of |\@xnthm|.
%    \begin{macrocode}
   }}
%    \end{macrocode}
% \end{macro}
%
% \begin{macro}{\@ynthm}
%    The definition of |\@ynthm| is completely analogous. In
%    this case the new counter that is defined is not reset within
%    another counter; thus the definition of |\the...|
%    is simplified:
%    \begin{macrocode}
\gdef\@ynthm#1#2{\expandafter\@ifdefinable\csname #1\endcsname
   {\@definecounter{#1}%
    \expandafter\xdef\csname the#1\endcsname{\@thmcounter{#1}}%
%    \end{macrocode}
%    The rest of the definition corresponds literally to that of
%    |\@xnthm|:
%    \begin{macrocode}
    \def\@tempa{\global\@namedef{#1}}\expandafter \@tempa
     \expandafter{\csname th@\the \theorem@style \expandafter
     \endcsname \the\theorem@bodyfont \@thm{#1}{#2}}%
    \global \expandafter \let \csname end#1\endcsname \@endtheorem}}
%    \end{macrocode}
% \end{macro}
%
% \begin{macro}{\@othm}
%    The definition of |\@othm| does not contain anything new.
%    \begin{macrocode}
\gdef\@othm#1[#2]#3{%
%    \end{macrocode}
% \changes{v2.2b}{95/10/07}{Build in check for counter defined}
%    We do not define a new counter but instead use one that has
%    already been defined.  Thus the only definition we need is that
%    of this pseudo-counter (i.e.\ |\the|\meta{env.\ name}).
%    First we check if |#2| corresponds to a known counter name.
%    \begin{macrocode}
  \expandafter\ifx\csname c@#2\endcsname\relax
   \@nocounterr{#2}%
  \else
   \expandafter\@ifdefinable\csname #1\endcsname
   {\expandafter \xdef \csname the#1\endcsname
     {\expandafter \noexpand \csname the#2\endcsname}%
%    \end{macrocode}
%    All other parts of the definition can be adopted from
%    |\@xnthm|. We have to remember, though, that in this case
%    the name of the current counter and the theorem label have moved
%    to the second and third arguments.
%    \begin{macrocode}
    \def\@tempa{\global\@namedef{#1}}\expandafter \@tempa
     \expandafter{\csname th@\the \theorem@style \expandafter
     \endcsname \the\theorem@bodyfont \@thm{#2}{#3}}%
    \global \expandafter \let \csname end#1\endcsname \@endtheorem}%
  \fi}
%    \end{macrocode}
% \end{macro}
%
% \subsection{Macros that are employed in a {\envfont theorem}
%    environment}
%
% \begin{macro}{\@thm}
%    The macro |\@thm| has to increase the current counter.  Then,
%    depending on whether the environment has (or does not have) an
%    optional argument, it has to branch into either |\@begintheorem|
%    or |\@opargtheorem|.
%    \begin{macrocode}
\gdef\@thm#1#2{\refstepcounter{#1}%
%    \end{macrocode}
%    Now we start a {\envfont trivlist} environment, and give
%    |\@topsep| and |\@topsepadd| the values of the \textsf{skip}
%    registers |\theorempreskipamount| and |\theorempostskipamount|.
%    The value in |\@topsep| is the vertical space that is inserted by
%    the first (and only) |\item| in our |\trivlist| whilst
%    |\@topsepadd| is inserted by |\@endparenv| at the end of that
%    {\envfont trivlist} environment.  By using these registers, we
%    obtain the desired space around a {\envfont theorem} environment.
%    \begin{macrocode}
   \trivlist
   \@topsep \theorempreskipamount               % used by first \item
   \@topsepadd \theorempostskipamount           % used by \@endparenv
%    \end{macrocode}
%    Now we have to test whether an optional argument has been given.
% \changes{v2.0d}{89/07/18}{Forgotten percent added.}
%    \begin{macrocode}
   \@ifnextchar [%
%    \end{macrocode}
%    If there is an optional argument, we will call |\@ythm|, and
%    move the arguments read back into the input stream.
%    \begin{macrocode}
   {\@ythm{#1}{#2}}%
%    \end{macrocode}
%    If not, we call |\@begintheorem|. Its first argument is the
%    name of the theorem set (hence the second argument of
%    |\@thm|). Its second argument is the macro that produces the
%    current number.
%    \begin{macrocode}
   {\@begintheorem{#2}{\csname the#1\endcsname}\ignorespaces}}
%    \end{macrocode}
% \end{macro}
%
% \begin{macro}{\@xthm}
% \begin{macro}{\@ythm}
%    Both these macros were originally called by |\@thm|.  We do
%    not need |\@xthm| anymore, hence we reset it to
%    |\relax|. The definition of |\@ythm| has not changed at
%    all from its definition in \LaTeX. In order to make the macros
%    easier to understand, we will
%    nevertheless present it (commented out).
%    \begin{macrocode}
\global\let\@xthm\relax
% \def\@ythm#1#2[#3]{\@opargbegintheorem{#2}{\csname
%           the#1\endcsname}{#3}\ignorespaces}
%</package>
%    \end{macrocode}
%    The primitive |\ignorespaces| in |\@ythm| and
%    |\@thm| is needed to remove the spaces between the
%    |\begin{...}| and the actual text.
% \end{macro}
% \end{macro}
%
%
% \subsection{Definition of the theorem substyles}
%
% As already pointed out, the theorem substyles, defined below, are
% only loaded when necessary.  Note that all these substyles, except
% \texttt{plain}, have |\slshape| as the default body font.
%
% \subsubsection{The \texttt{plain} style}
%
% \label{dok:th@plain} As the following macros use |@|, we have
% to locally set the |\catcode| of this symbol to ``letter''.
% This happens within a group, so that we do not have to worry about
% which |\catcode| that symbol had before.
%    \begin{macrocode}
%<*thp>
\begingroup \makeatletter
%    \end{macrocode}
% Since we are now within a group, we must make all definitions
% globally.  First we make sure that \texttt{theorem.sty} is loaded.
% This will allow us to use this file as a document style option
% without having to call \texttt{theorem} itself as an option.  At the
% same time, we assure that at least version 2 is loaded, since
% |\theorem@style| was not defined in earlier versions.
%    \begin{macrocode}
\@ifundefined{theorem@style}{\input{theorem.sty}}{}
\ProvidesFile{thp.sty}
             [\FMithmInfo]
%    \end{macrocode}
%
%
% \begin{macro}{\th@plain}
% \begin{macro}{\@begintheorem}
% \begin{macro}{\@opargbegintheorem}
%    |\theoremstyle{plain}| corresponds to the original
%    definition, except that the distances to the surrounding text are
%    determined by the parameters |\theorempreskipamount| and
%    |\theorempostskipamount|.
%    First we set the default body font.
%    \begin{macrocode}
\gdef\th@plain{\normalfont\itshape
%    \end{macrocode}
%    Then we define |\@begintheorem| and
%    |\@opargbegintheorem|. These two macros define how the
%    header of a theorem is typeset.  |\@opargbegintheorem| will
%    be called if a \textsf{theorem} environment with an optional
%    argument is encountered; otherwise, the header is constructed by
%    calling |\@begintheorem|.  If one of these macros is
%    executed, we are within a \textsf{trivlist} environment started by
%    |\@thm|.  So the theorem header is produced with an
%    |\item| command.
%
%    Instead of specifying the header font directly, all standard
%    theorem styles use the |\theorem@headerfont| macro to allow
%    customization.  The extra space (|\labelsep|) is necessary
%    because of problems in the \textsf{trivlist} environment.
%    \begin{macrocode}
  \def\@begintheorem##1##2{%
        \item[\hskip\labelsep \theorem@headerfont ##1\ ##2]}%
%    \end{macrocode}
%    The definition of |\@opargbegintheorem| is completely
%    analogous. The only difference is the fact that there exists a
%    third argument (which is the optional parameter of the
%    environment and contains additional information about the
%    theorem).  Customarily we enclose it in parentheses.
%    \begin{macrocode}
\def\@opargbegintheorem##1##2##3{%
   \item[\hskip\labelsep \theorem@headerfont ##1\ ##2\ (##3)]}}
%    \end{macrocode}
% \end{macro}
% \end{macro}
% \end{macro}
%
% \noindent
% We conclude with an |\endgroup| to restore the
% |\catcode| of |@|.
%    \begin{macrocode}
\endgroup
%</thp>
%    \end{macrocode}
%
%
% \subsubsection{The \texttt{break} style}
%
% This style option is stored in the file |thb.sty|.
% For the next two lines see the documentation for |\th@plain| on
% page~\pageref{dok:th@plain}.
%    \begin{macrocode}
%<*thb>
\begingroup \makeatletter
\@ifundefined{theorem@style}{\input{theorem.sty}}{}
\ProvidesFile{thb.sty}
             [\FMithmInfo]
%    \end{macrocode}
%
% \begin{macro}{\th@break}
%    |\theoremstyle{break}| produces a line break after the name
%    of the theorem.  The font is |\slshape|.  Hence, we define
%    |\th@break| as follows:
%    \begin{macrocode}
\gdef\th@break{\normalfont\slshape
  \def\@begintheorem##1##2{\item[%
%    \end{macrocode}
%    We run into the following problem: it is not possible to create
%    the header with |\item[|\meta{title}|]| and then start
%    a new line by, for example, |\mbox{}\\|. Such a definition
%    will fail whenever a list environment follows immediately.  With
%    the above construction, the |\mbox{}| causes the switch
%    |@inlabel| (cf.\ definition of |\list| and
%    |\trivlist| in \cite{latex:source}) to be set to \texttt{false}
%    and so the following list will insert additional vertical
%    space (|\topskip|). This is quite annoying. Therefore, we
%    create the line break within the |\item|. In order to ensure
%    that the text will begin at the proper position in the following
%    line, we simply pretend that the label does not take any
%    room.\footnote{This will lead to problems whenever very high
%    symbols occurring in the line tower into the
%    heading. So, something else has to be done here sometime.}
%    \begin{macrocode}
       \rlap{\vbox{\hbox{\hskip \labelsep\theorem@headerfont ##1\ ##2}%
                   \hbox{\strut}}}]}%
%    \end{macrocode}
%    Again, the definition of |\@opargbegintheorem| is
%    completely analogous.
%    \begin{macrocode}
\def\@opargbegintheorem##1##2##3{%
  \item[\rlap{\vbox{\hbox{\hskip \labelsep \theorem@headerfont
                     ##1\ ##2\ (##3)}%
                    \hbox{\strut}}}]}}
\endgroup
%</thb>
%    \end{macrocode}
% \end{macro}
%
%
% \subsubsection{The \texttt{changebreak} style}
%
%    \begin{macrocode}
% This style option is stored in the file |thcb.sty|.
%    \begin{macrocode}
%<*thcb>
\begingroup \makeatletter
\@ifundefined{theorem@style}{\input{theorem.sty}}{}
\ProvidesFile{thcb.sty}
             [\FMithmInfo]
%    \end{macrocode}
%
% \begin{macro}{\th@changebreak}
%    The \texttt{change-break} theorem style is like \texttt{break}
%    but with interchange of theorem name and theorem number.  Thus we
%    define |\th@changebreak| as follows:
%    \begin{macrocode}
\gdef\th@changebreak{\normalfont\slshape
  \def\@begintheorem##1##2{\item
       [\rlap{\vbox{\hbox{\hskip\labelsep \theorem@headerfont ##2\ ##1}%
                    \hbox{\strut}}}]}%
\def\@opargbegintheorem##1##2##3{%
  \item[\rlap{\vbox{\hbox{\hskip\labelsep \theorem@headerfont
                     ##2\ ##1\ (##3)}%
                    \hbox{\strut}}}]}}
\endgroup
%</thcb>
%    \end{macrocode}
% \end{macro}
%
%
% \subsubsection{The \texttt{change} style}
%
% This style option is stored in the file |thc.sty|.
%    \begin{macrocode}
%<*thc>
\begingroup \makeatletter
\@ifundefined{theorem@style}{\input{theorem.sty}}{}
\ProvidesFile{thc.sty}
             [\FMithmInfo]
%    \end{macrocode}
%
% \begin{macro}{\th@change}
%    The \texttt{change} theorem style corresponds to the
%    \texttt{change break} style without a linebreak after the header.
%    To say it in another way, it's the same as the \texttt{plain}
%    style but with number and name interchanged and |\slshape| as the
%    default font.
%    \begin{macrocode}
\gdef\th@change{\normalfont\slshape
  \def\@begintheorem##1##2{\item
       [\hskip\labelsep \theorem@headerfont ##2\ ##1]}%
\def\@opargbegintheorem##1##2##3{%
  \item[\hskip\labelsep \theorem@headerfont ##2\ ##1\ (##3)]}}
\endgroup
%</thc>
%    \end{macrocode}
% \end{macro}
%
%
% \subsubsection{The \texttt{marginbreak} style}
%
% This style option is the one used most often at Mainz.  It is saved
% in the file |thmb.sty|.
%    \begin{macrocode}
%<*thmb>
\begingroup \makeatletter
\@ifundefined{theorem@style}{\input{theorem.sty}}{}
\ProvidesFile{thmb.sty}
             [\FMithmInfo]
%    \end{macrocode}
%
% \begin{macro}{\th@marginbreak}
%    The \texttt{margin break} style is nearly the same as the
%    \texttt{change break} style. The only difference is the placement
%    of the theorem number.  We use |\llap| to place it in the left
%    margin.
%
%    In this style |\labelsep| denotes the separation between the
%    number and the text.
%    \begin{macrocode}
\gdef\th@marginbreak{\normalfont\slshape
  \def\@begintheorem##1##2{\item
       [\rlap{\vbox{\theorem@headerfont
                    \hbox{\llap{##2}\hskip\labelsep ##1}%
                    \hbox{\strut}}}]}%
\def\@opargbegintheorem##1##2##3{%
  \item[\rlap{\vbox{\theorem@headerfont
                    \hbox{\llap{##2}\hskip\labelsep ##1\ (##3)}%
                    \hbox{\strut}}}]}}
\endgroup
%</thmb>
%    \end{macrocode}
% \changes{v2.1c}{90/02/17}{`endgroup outside macrocode env.}
% \end{macro}
%
%
%
% \subsubsection{The \texttt{margin} style}
%
% This style option is stored in the file |thm.sty|.
%    \begin{macrocode}
%<*thm>
\begingroup \makeatletter
\@ifundefined{theorem@style}{\input{theorem.sty}}{}
\ProvidesFile{thm.sty}
             [\FMithmInfo]
%    \end{macrocode}
%
% \begin{macro}{\th@margin}
%    Again this is only a variant of the theorem styles described
%    above without any new ideas.
%    \begin{macrocode}
\gdef\th@margin{\normalfont\slshape
  \def\@begintheorem##1##2{\item
       [\theorem@headerfont \llap{##2}\hskip\labelsep ##1]}%
\def\@opargbegintheorem##1##2##3{%
  \item[\theorem@headerfont \llap{##2}\hskip\labelsep ##1\ (##3)]}}
\endgroup
%</thm>
%    \end{macrocode}
% \end{macro}
%
%
%
%
% \subsection{Final Definitions}
%
% \begin{macro}{\theorempreskipamount}
% \begin{macro}{\theorempostskipamount}
%    The \textsf{skip} parameters that regulate the vertical empty space
%    before and after the {\envfont theorem} environment have
%    to be allocated as well.
% \changes{v2.0f}{89/07/20}{Unnecessary `global' removed.}
%    \begin{macrocode}
%<*package>
\newskip\theorempreskipamount
\newskip\theorempostskipamount
%    \end{macrocode}
%    Since we have used the same values for all theorem sets, we now can
%    assign them.
%    \begin{macrocode}
\global\setlength\theorempreskipamount{12pt plus 5pt minus 3pt}
\global\setlength\theorempostskipamount{8pt plus 3pt minus 1.5pt}
%    \end{macrocode}
% \end{macro}
% \end{macro}
%
%
% \begin{macro}{\@endtheorem}
%    The same holds for the macro |\@endtheorem|, which ends a
%    {\envfont theorem} environment. Since it is the same for all
%    theorem sets, it is removed from the macros |\th@|\meta{style}.
%    It simply ends the {\envfont trivlist} environment, which was
%    begun in |\@thm|.
%    \begin{macrocode}
\global\let\@endtheorem=\endtrivlist
%    \end{macrocode}
% \end{macro}
%
%
%
% \begin{macro}{\@preamblecmds}
%    All macros defined above are to be used only in the preamble.
%    Therefore, we insert them in |\@preamblecmds| which will disable
%    them at begin document. This is done by the internal \LaTeXe{}
%    command |\@onlypreamble|.
% \changes{v2.0g}{89/09/19}{`theoremheaderfont and `theorembodyfont
%    added.}
% \changes{v2.2a}{94/02/03}{Use \LaTeXe{} interface}
%    \begin{macrocode}
\@onlypreamble\@xnthm
\@onlypreamble\@ynthm
\@onlypreamble\@othm
\@onlypreamble\newtheorem
\@onlypreamble\theoremstyle
\@onlypreamble\theorembodyfont
\@onlypreamble\theoremheaderfont
%    \end{macrocode}
% \end{macro}
%
% Finally we declare the \texttt{plain} theorem style to be the default.
% \changes{v2.1b}{90/06/04}{Fixed missing `item by declaring default
%    style.}
%    \begin{macrocode}
\theoremstyle{plain}
%</package>
%    \end{macrocode}
%
% \Finale

\endinput
}{}
\ProvidesFile{thmb.sty}
             [\FMithmInfo]
%    \end{macrocode}
%
% \begin{macro}{\th@marginbreak}
%    The \texttt{margin break} style is nearly the same as the
%    \texttt{change break} style. The only difference is the placement
%    of the theorem number.  We use |\llap| to place it in the left
%    margin.
%
%    In this style |\labelsep| denotes the separation between the
%    number and the text.
%    \begin{macrocode}
\gdef\th@marginbreak{\normalfont\slshape
  \def\@begintheorem##1##2{\item
       [\rlap{\vbox{\theorem@headerfont
                    \hbox{\llap{##2}\hskip\labelsep ##1}%
                    \hbox{\strut}}}]}%
\def\@opargbegintheorem##1##2##3{%
  \item[\rlap{\vbox{\theorem@headerfont
                    \hbox{\llap{##2}\hskip\labelsep ##1\ (##3)}%
                    \hbox{\strut}}}]}}
\endgroup
%</thmb>
%    \end{macrocode}
% \changes{v2.1c}{90/02/17}{`endgroup outside macrocode env.}
% \end{macro}
%
%
%
% \subsubsection{The \texttt{margin} style}
%
% This style option is stored in the file |thm.sty|.
%    \begin{macrocode}
%<*thm>
\begingroup \makeatletter
\@ifundefined{theorem@style}{% \iffalse meta-comment
%
% Copyright (C) 1993-2024
%
% The LaTeX Project and any individual authors listed elsewhere
% in this file.
%
% This file is part of the Standard LaTeX `Tools Bundle'.
% -------------------------------------------------------
%
% It may be distributed and/or modified under the
% conditions of the LaTeX Project Public License, either version 1.3c
% of this license or (at your option) any later version.
% The latest version of this license is in
%    https://www.latex-project.org/lppl.txt
% and version 1.3c or later is part of all distributions of LaTeX
% version 2005/12/01 or later.
%
% The list of all files belonging to the LaTeX `Tools Bundle' is
% given in the file `manifest.txt'.
%
% \fi
%
%%
%
%
\def\FMithmInfo{2023/07/05 v2.2c Theorem extension package (FMi)}
%
% \ProvidesFile{theorem.dtx}[\FMithmInfo]
%\iffalse   % this is a METACOMMENT !
%%
%% Package `theorem' to use with LaTeX2e
%% Copyright (C) 1989-1995 Frank Mittelbach, all rights reserved.
%<+package>\@ifundefined{theorem@style}{}{\endinput}
%<+package>\NeedsTeXFormat{LaTeX2e}[1995/06/01]
%<+package>\ProvidesPackage{theorem}
%<+package>         [\FMithmInfo]
%
% \fi
%
% \changes{v1.0}{87/05/07}{First draft.}
% \changes{v1.0a}{87/05/07}{`thistheoremstyle' added.}
% \changes{v1.1a}{87/07/29}{New concept: redefine the internal
%                          \protect\LaTeX{} commands `@xnthm' etc.}
% \changes{v1.1b}{87/07/31}{`theoremfont' used to specify font.}
% \changes{v1.2a}{88/01/08}{`@endtheorem' now globally defined.}
% \changes{v1.2b}{88/01/11}{`newtheorem' only allowed in the preamble.}
% \changes{v1.3a}{88/02/14}{`theoremfont' now a toks register.}
%
% \changes{v2.0a}{88/11/16}{New implementation and new documentation.}
% \changes{v2.0a}{88/11/16}{No layout changes but internal modifications
%                           old substyles won't work.}
% \changes{v2.0b}{89/01/30}{`global' used for all definitions.}
% \changes{v2.0c}{89/06/15}{Documentation translated to English.
%                            Thanks to LLL and Nina}
% \changes{v2.1a}{90/02/15}{Support for new font selection.}
% \changes{v2.2a}{94/02/03}{Cleanup for LaTeX2e}
%
% \DoNotIndex{\@endparenv,\@ifdefinable,\@ifnextchar,\@ifundefined}
% \DoNotIndex{\@namedef,\@nameuse,\@spaces,\@tempa,\@warning}
% \DoNotIndex{\ }
% \DoNotIndex{\begingroup,\bfseries}
% \DoNotIndex{\csname}
% \DoNotIndex{\def,\do,\docdate}
% \DoNotIndex{\endcsname,\endgroup,\endinput,\endtrivlist,\expandafter}
% \DoNotIndex{\filedate,\fileversion}
% \DoNotIndex{\gdef,\global}
% \DoNotIndex{\hbox,\hskip}
% \DoNotIndex{\ignorespaces,\input,\itshape,\item}
% \DoNotIndex{\labelsep,\let,\llap}
% \DoNotIndex{\makeatletter}
% \DoNotIndex{\noexpand}
% \DoNotIndex{\refstepcounter,\relax,\rlap}
% \DoNotIndex{\setlength,\slshape,\space,\string,\strut}
% \DoNotIndex{\the,\trivlist,\typeout}
% \DoNotIndex{\vbox}
% \DoNotIndex{\xdef}
%
% \def\deflabel#1{\ttfamily #1\hfill}
% \def\deflist#1{\begin{list}{}{\settowidth\labelwidth{\ttfamily #1}%
%                               \setlength\leftmargin\labelwidth
%                               \addtolength\leftmargin\labelsep
%                               \let\makelabel\deflabel}}
% \def\enddeflist{\end{list}}
% \def\envfont{\normalfont\ttfamily}
%
% \MakeShortVerb{\|}
%
% \newcommand{\TUB}{{\sl TUGboat\/}}
%
% \GetFileInfo{theorem.dtx}
% \title{An Extension of the
%        \LaTeX{} theorem environment\thanks{This file
%        has version number \fileversion, last
%        revised \filedate.}}
% \author{Frank Mittelbach}
%
% \MaintainedByLaTeXTeam{tools}
% \maketitle
%
% \begin{abstract}
%    The macros described in this paper yield an extension of the
%    \LaTeX{} theorem mechanism. It is designed to satisfy the
%    different requirements of various journals. Thus, the
%    layout of the ``theorems'' can be manipulated by determining a
%    ``style''. This article describes not only the use, but
%    also the definition, of the necessary macros.
% \end{abstract}
%
% \section*{Preface to version 2.2}
%
% For \LaTeXe{} this package did not need any fundamental changes. I
% only modified the messages generated so that theorem layout styles
% will show up with the |\listfiles| command and cleaned the section
% on the New Font Selection Scheme since this is now included in
% \LaTeX.
%
% \section*{Preface to version 2.1}
%
% This version is identical to 2.0g described in \TUB\ 10\#3 except
% for some internal defaults which are now set depending on the used
% font selection scheme.
%
% This was done to avoid unpleasant surprises if the new font selection
% scheme is in force. For further details see
% section~\ref{sec:newfonts} and \cite{bk:GMS94}.
%
% \section{Introduction}
%
% For our purposes here, ``theorems'' are labelled enunciations,
% often set off from the main text by extra space and a font change.
% Theorems, corollaries, conjectures, definitions, and remarks
% are all instances of ``theorems''.  The ``header'' of these structures
% is composed of a label (such as \textsc{Theorem} or \textsc{Remark})
% and a number which serializes an item in the sequence of items
% with the same label.
%
% Shortly after the introduction of \LaTeX{} at the Fachbereich
% Mathematik in Mainz, the desire to manipulate the layout of
% ``theorems'' arose.  In Mainz, the following two conventions came
% into general use:
% \begin{enumerate}
%   \item The number of the theorem is shown in the margin.
%   \item There is a line break at
%         the end of the theorem header.
% \end{enumerate}
%
% Additionally, some journals require different formats which depend on
% the ``sort of theorem'': e.g.\ often remarks and definitions are set
% in |\upshape|, while |\itshape| is employed for main theorems.
%
% Confronted with these requirements, a theorem environment was
% developed in Mainz which allows separate determination of the
% layout of the ``theorems sets'', comparable to |\pagestyle|.
%
%
% \section{The user interface}
% \subsection{Defining new theorem sets}
%
% \DescribeMacro\newtheorem
% As in the original \LaTeX{} version, the command |\newtheorem|
% defines a new ``theorem set'' or ``theorem-like structure''.
% Two required arguments name the new environment and give the text
% to be typeset with each instance of the new ``set'', while
% an optional argument determines how
% the ``set'' is enumerated:
% \begin{description}
%    \item[\ttfamily \bslash newtheorem\{foo\}\{bar\}]
%       The theorem set {\envfont foo} (whose name is \texttt{bar})
%       uses its own counter.
%    \item[\ttfamily \bslash newtheorem\{foo2\}{[foo]}\{bar2\}]
%       The theorem set {\envfont foo2} (printed name \texttt{bar2})
%       uses the same counter as the theorem set \texttt{foo}.
%    \item[\ttfamily \bslash newtheorem\{foo3\}\{bar3\}{[section]}]
%       The theorem set {\envfont foo3} (printed name \texttt{bar3}) is
%       enumerated within the counter \texttt{section}, i.e.\ with every
%       new |\section| the enumeration begins again with 1, and
%       the enumeration is composed from the section-number and the
%       theorem counter itself.
% \end{description}
%
% \DescribeMacro\theoremstyle
% Additionally, the command |\theoremstyle| can define the layout
% of various, or all, theorem sets.  It should be noted that any
% theorem set defined by |\newtheorem| is typeset in the
% |\theoremstyle| that is current at the time of the definition.
% Thus, the following
% \begin{verbatim}
%    \theoremstyle{break}        \newtheorem{Cor}{Corollary}
%    \theoremstyle{plain}        \newtheorem{Exa}{Example}[section]
%\end{verbatim}                             ^^A No blank at line start
% leads to the result that the set {\envfont Cor} is formatted in the
% style \texttt{break}, while the set {\envfont Exa} and all the
% following ones are formatted in the style \texttt{plain}, unless
% another |\theoremstyle| follows.  Since the definitions installed by
% |\newtheorem| are global, one also can limit |\theoremstyle| locally
% by grouping braces.
%
% \DescribeMacro\theorembodyfont
% The choice of the font for the theorem body is completely
% independent of the chosen |\theoremstyle|; this has proven to
% be very advantageous.  For example,
% \begin{verbatim}
%   {\theorembodyfont{\upshape}          \newtheorem{Rem}{Remark}}
%\end{verbatim}
% defines a theorem set {\envfont Rem}, which will be set in
% |\upshape| in the current layout (which in our example is
% \texttt{plain}).  As with |\theoremstyle|, the |\theorembodyfont|
% chosen is that current at the time of |\newtheorem|.  If
% |\theorembodyfont| is not specified or one defines
% |\theorembodyfont{}|, then the font used will be that defined by the
% |\theoremstyle|.
%
% \DescribeMacro\theoremheaderfont It is also possible to customize
% the font used for the theorem headers.  This is, however, a global
% declaration, and therefore there should be at most one
% |\theoremheaderfont| declaration in the preamble.\footnote{If it is
% actually necessary to have different header fonts, one has to define
% new theorem styles (substituting the desired font) or specify the
% information directly in the \texttt{\bslash newtheorem} declaration
% (the unclean variant).}
%
% \DescribeMacro\theorempreskipamount
% \DescribeMacro\theorempostskipamount Two additional parameters
% affect the vertical space around the theorem environments:
% |\theorempreskipamount| and |\theorempostskipamount| define,
% respectively, the spacing before and after such an environment.
% These parameters apply for all theorem sets and can be manipulated
% with the ordinary length macros.  They are rubber lengths
% (`\textsf{skips}'), and therefore can contain \texttt{plus} and
% \texttt{minus} parts.
%
% Since the definition of theorem sets should---most sensibly---be
% placed in the preamble, we only allow installation there.
% It is therefore possible to release the memory used here after
% |\begin{document}|, in order to make room for other
% applications.
%
%
% \subsection{Existing theorem styles}
%
% The following theorem styles exist to date:
% \begin{deflist}{marginbreak}
%    \item[plain]
%       This theorem style emulates the original \LaTeX{} definition,
%       except that additionally the parameters
%       |\theorem...skipamount| are used.
%    \item[break]
%       In this style, the theorem header is followed by a line
%       break.
%    \item[marginbreak]
%       The theorem number is set in the margin, and there is a line
%       break as in \texttt{break}.
%    \item[changebreak]
%       Like \texttt{break}, but with header number and text
%       interchanged.
%    \item[change]
%       Header number and text are interchanged, without a line break.
%    \item[margin]
%       The number is set in the left margin, without a line break.
% \end{deflist}
% All styles (except \texttt{plain}) select |\slshape| as the default
% |\theorembodyfont|.
%
%
% \subsection{Examples}
%
% Given the above theorem sets {\envfont Cor}, {\envfont Exa}
% and {\envfont Rem},
% suppose that the preamble also contains the declarations:
% \begin{verbatim}
%    \theoremstyle{marginbreak}   \newtheorem{Lem}[Cor]{Lemma}
%    \theoremstyle{change}
%    \theorembodyfont{\itshape}        \newtheorem{Def}[Cor]{Definition}
%
%    \theoremheaderfont{\scshape}
%\end{verbatim}
% Then the following are some typical examples of the typeset output
% resulting from their use.
%
% \begin{Cor}
%    This is a sentence typeset in the theorem environment {\envfont
%    Cor}.
% \end{Cor}
%
% \begin{Exa}
%    This is a sentence typeset in the theorem environment {\envfont
%    Exa}.
% \end{Exa}
%
% \begin{Rem}
%    This is a sentence typeset in the theorem environment {\envfont
%    Rem}.
% \end{Rem}
%
% \begin{Lem}[Ben User]
%    This is a sentence typeset in the theorem environment {\envfont
%    Lem}.
% \end{Lem}
%
% \begin{Def}[Very impressive Definition]
%    This is a sentence typeset in the theorem environment {\envfont
%    Def}.
% \end{Def}
%
% The last two examples show the effect of the optional argument to
% a theorem environment (it is the text typeset in parentheses).
%
%
% \section{Special Considerations}
% \label{sec:newfonts}
%
% Theoremheader and body are implemented as a unit. This means that
% the |\theoremheaderfont|  will inherit characteristics of the
% |\theorembodyfont| in \LaTeXe. Thus, if for example
% |\theorembodyfont| is |\itshape| and |\theoremheaderfont|
% is |\bfseries| the font selected for the header will have the
% characteristics `bold extended italic'. If this is not desired one
% should set the |\theoremheaderfont| to something like
% \begin{verbatim}
%      \theoremheaderfont{\normalfont\bfseries}
%\end{verbatim}
% i.e.\ supplying all necessary font information explicitly.
%
%
%
% \section{Acknowledgements}
%
% The publication of this set of macros was only possible with the
% help of Christina Busse (translating the manuscript into English),
% Joachim Pense (playing the r\^ole of typist), Chris Rowley
% (looking everything over) and many others providing useful
% suggestions.
%
% \MaybeStop{%
%    \begin{thebibliography}{9}
%       \bibitem{bk:GMS94} \textsc{M.~Goossens}, \textsc{F.~Mittelbach}
%         and \textsc{A.~Samarin}.
%         \newblock The \LaTeX{} Companion.
%         \newblock
%         Addison-Wesley, Reading, Massachusetts, 1994.
%       \bibitem{latex:source} \textsc{Lamport, Leslie.} \newblock
%                               \texttt{latex.tex}, version 2.09, date
%                               Feb. 1990.
%    \end{thebibliography}
%    \PrintIndex
%    \PrintChanges}
%
%
% \section{The documentation driver file}
%
% The next bit of code contains the documentation driver file for
% \TeX{}, i.e., the file that will produce the documentation you are
% currently reading. It will be extracted from this file by the
% \texttt{docstrip} program. Since it is the first code in the file
% one can alternatively process this file directly with \LaTeXe{} to
% obtain the documentation.
%
%    \begin{macrocode}
%<*driver>
\documentclass{ltxdoc}

\usepackage{theorem}

%<+driver>% The next few lines define theorem sets which are used
%<+driver>% in the example section of the documentation.

\theoremstyle{break}       \newtheorem{Cor}{Corollary}
\theoremstyle{plain}       \newtheorem{Exa}{Example}[section]
{\theorembodyfont{\upshape}\newtheorem{Rem}{Remark}}
\theoremstyle{marginbreak} \newtheorem{Lem}[Cor]{Lemma}
\theoremstyle{change}
\theorembodyfont{\itshape} \newtheorem{Def}[Cor]{Definition}

\theoremheaderfont{\scshape}

\RecordChanges

\begin{document}
   \DocInput{theorem.dtx}
\end{document}
%</driver>
%    \end{macrocode}
%
% \section{Definition of the Macros}
%
% If the file has been loaded before, we abort immediately. If not the
% package announces itself (this is actually done at the very top of
% the file---the way it is done isn't good style so don't copy it).
% \changes{v2.0e}{89/07/19}{Spaces removed from `typeout'.}
% \changes{v2.2b}{95/11/19}{Announce moved to top of file}
%    \begin{macrocode}
%<*package>
%\@ifundefined{theorem@style}{}{\endinput}
%\def\FMithmInfo{1995/11/19 v2.2b Theorem extension package (FMi)}
%\ProvidesPackage{theorem}[\FMithmInfo]
%    \end{macrocode}
%
%
% \subsection{Definition of theorem styles and fonts}
%
% All the definitions in this file are done globally to allow
% inputting this file inside a group.
%
% \begin{macro}{\theoremstyle}
%    Before a theorem style can be installed, the chosen style must be
%    known. For that reason, we must test to see that
%    |\th@|\meta{style} is known or, more precisely, that it is
%    different from |\relax|. If the style is not known then
%    |\th@plain| is used.
%    \begin{macrocode}
\gdef\theoremstyle#1{%
   \@ifundefined{th@#1}{\@warning
          {Unknown theoremstyle `#1'. Using `plain'}%
          \theorem@style{plain}}%
%    \end{macrocode}
%    We save the theorem style to be used in the token register
%    |\theorem@style|.
%    \begin{macrocode}
      {\theorem@style{#1}}%
%    \end{macrocode}
%    Now we ``evaluate'' the theorem style: this means, we call the
%    macro |\th@|\meta{style} which will activate the relevant
%    definitions which are contained in a separate file.  This is done
%    in a group to suppress changes to the current font.  This could
%    otherwise pose problems together with the new font selection
%    scheme\footnote{When I printed the original article using the new
%    font selection scheme I ended with a document with slanted
%    typefaces (text headings and all) simply because one of the
%    theorem styles used \texttt{\bslash sl} at toplevel.} if the
%    |\th@|\meta{style} is evaluated a second time.
% \changes{v2.1a}{90/02/15}{Group added around `th@..}
% \changes{v2.1a}{90/02/15}{Reseting of `..@begintheorem removed.}
%    \begin{macrocode}
      \begingroup
        \csname th@\the\theorem@style \endcsname
      \endgroup}
%    \end{macrocode}
% \end{macro}
%
%
% \begin{macro}{\@begintheorem}
% \changes{v2.1a}{90/02/15}{Relaxed at toplevel.}
% \begin{macro}{\@opargbegintheorem}
% \changes{v2.1a}{90/02/15}{Relaxed at toplevel.}
%    We reset |\@begintheorem| and |\@opargbegintheorem| to
%    |\relax| since these commands are no longer necessary at
%    toplevel. This will save a few tokens.
%    \begin{macrocode}
\global\let\@begintheorem\relax
\global\let\@opargbegintheorem\relax
%    \end{macrocode}
% \end{macro}
% \end{macro}
%
% \begin{macro}{\theorem@style}
%    Obviously the \textsf{token} register used above has to be
%    allocated. To assure the utmost compatibility with the
%    original \LaTeX{} definition, we set the default theorem style to
%    \texttt{plain}, which implements the usual \LaTeX{} convention.
% \changes{v2.0f}{89/07/20}{Unnecessary `global' removed.}
%    \begin{macrocode}
\newtoks\theorem@style
\global\theorem@style{plain}
%    \end{macrocode}
% \end{macro}
%
% \begin{macro}{\theorembodyfont}
% \begin{macro}{\theorem@bodyfont}
%    If the |\theorembodyfont| is set by the user then it should not
%    interact with the default font set in the theorem style. When the
%    new font selection is in force this may happen if, for example,
%    the default is |\itshape| and the new |\theorembodyfont| is
%    |\sffamily|. So we add a |\reset@font| command in front of the
%    user definition.
% \changes{v2.0f}{89/07/20}{Unnecessary `global' removed.}
% \changes{v2.0g}{89/09/19}{`theoremfont to `theorembodyfont.}
% \changes{v2.1c}{92/06/22}{`reset@font added and logic changed.}
% \changes{v2.1d}{92/10/28}{Reset to empty if arg is empty}
%    \begin{macrocode}
\gdef\theorembodyfont#1{%
%    \end{macrocode}
%    We check if the argument supplied is empty and if so put nothing
%    into the |\theorem@bodyfont| token register to allow for
%    |\theorembodyfont{}| as a mean of using the default of the
%    current |\theoremstyle|.
%    \begin{macrocode}
   \def\@tempa{#1}%
   \ifx\@tempa\@empty
    \theorem@bodyfont{}%
   \else
    \theorem@bodyfont{\reset@font#1}%
   \fi
   }
\newtoks\theorem@bodyfont
\global\theorem@bodyfont{}
%    \end{macrocode}
% \end{macro}
% \end{macro}
%
%
% \begin{macro}{\theoremheaderfont}
% \changes{v2.0g}{89/09/19}{Macro added.}
%    The font for the theorem headers is handled differently because
%    this definition applies to all theorem styles.
%    \begin{macrocode}
\gdef\theoremheaderfont#1{\gdef\theorem@headerfont{#1}%
%    \end{macrocode}
%    After using the macro once it is redefined to produce an error
%    message.
%    \begin{macrocode}
       \gdef\theoremheaderfont##1{%
        \typeout{\string\theoremheaderfont\space should be used
                 only once.}}}
%    \end{macrocode}
% \end{macro}
%
% \begin{macro}{\theorem@headerfont}
% \changes{v2.0g}{89/09/19}{Macro added.}
%    To set the |\theorem@headerfont| default we first test if
%    the new fontselection scheme is in force.
% \changes{v2.1a}{90/02/15}{Support for new font selection.}
%    \begin{macrocode}
\ifx\upshape\undefined
%    \end{macrocode}
%    If not we define it to expand into |\bfseries|. We don't use
%    |\let| just in case a following style option redefines this
%    macro.
%    \begin{macrocode}
\gdef\theorem@headerfont{\bfseries}
%    \end{macrocode}
%    Otherwise we reset the current shape before calling |\bfseries|.
%    \begin{macrocode}
\else \gdef\theorem@headerfont{\normalfont\bfseries}\fi
%    \end{macrocode}
% \end{macro}
%
% \begin{macro}{\th@plain}
% \begin{macro}{\th@break}
% \begin{macro}{\th@marginbreak}
% \begin{macro}{\th@changebreak}
% \begin{macro}{\th@change}
% \begin{macro}{\th@margin}
%    The different styles are defined in macros such as
%    |\th@plain|.  Since memory space is precious in
%    ``non-Big-versions'', we have to avoid offering too many unused
%    definitions.  Therefore we define these styles in separate files
%    that can be loaded on demand.  Thus the commands themselves only
%    load these files. We use |\@input@|, a \LaTeXe{} internal command
%    that ensures that the file will be listed with |\listfiles|.
% \changes{v2.2a}{94/02/02}{Use `@input@ to load theorem layout files}
%    \begin{macrocode}
\gdef\th@plain{\@input@{thp.sty}}
\gdef\th@break{\@input@{thb.sty}}
\gdef\th@marginbreak{\@input@{thmb.sty}}
\gdef\th@changebreak{\@input@{thcb.sty}}
\gdef\th@change{\@input@{thc.sty}}
\gdef\th@margin{\@input@{thm.sty}}
%    \end{macrocode}
%    This list will be expanded when new styles become
%    available.  For testing,  just append new theorem substyles
%    as document options.
% \end{macro}
% \end{macro}
% \end{macro}
% \end{macro}
% \end{macro}
% \end{macro}
%
%
%
% \subsection{Definition of a new theorem set}
%
% As already pointed out, a new theorem environment can be defined in
% three different ways:
% \begin{verbatim}
%    \newtheorem{Lem}{Lemma}
%    \newtheorem{Lem}{Lemma}[section]
%    \newtheorem{Lem}[Theorem]{Lemma}
%\end{verbatim}
% The function of the macro |\newtheorem| is to recognize these cases
% and then to branch into one of the three macros |\@ynthm|, |\@xnthm|
% or |\@othm|.  This mechanism is adopted unchanged from
% \cite{latex:source}; the essential point here is that, for example,
% in the second case, the arguments \texttt{Lem}, \texttt{Lemma} and
% \texttt{section} are passed over to the macro |\@xnthm|.
%
% We inspect this case first because the others present fewer problems,
% and thus are easily derived from this one.
%
% \begin{macro}{\@xnthm}
%    For our example arguments, the macro |\@xnthm| must fulfill
%    the following:
%    \begin{itemize}
%       \item
%          Define a new \LaTeX-counter `Lem'
%       \item
%          reset this counter within a |\section|
%       \item
%          define the macro |\theLem|
%       \item
%          define the environment macros |\Lem| and
%          |\endLem| using the current
%          |\theoremstyle| and |\theorem@bodyfont|.
%    \end{itemize}
%    Obviously, all this should happen only if the first argument
%    of |\@xnthm| (i.e.\ \texttt{Lem} in our example) is chosen so
%    as not to conflict with any previously defined commands or
%    environments.  This test is performed by the \LaTeX{} macro
%    |\@ifdefinable|.
%    \begin{macrocode}
\gdef\@xnthm#1#2[#3]{\expandafter\@ifdefinable\csname #1\endcsname
%    \end{macrocode}
%    Therefore, the first argument of |\@ifdefinable| is the
%    expansion (in the example, |\Lem|) of
%    |\csname||#1\endcsname|.  The second argument is
%    executed only if the test has been completed successfully.
%    \begin{macrocode}
   {%
%    \end{macrocode}
%    Now we define the new counter. The names of the \LaTeX{}
%    macros employed should speak for themselves:
% \changes{v2.2b}{1995/10/07}{Use \cs{@newctr}}
% \changes{v2.2c}{1995/11/23}{Use \cs{@newctr} properly.}
%    \begin{macrocode}
    \@definecounter{#1}\@newctr{#1}[#3]%
%    \end{macrocode}
%    Using |\@newctr| will give a proper error message if the counter
%    in |#3| is not defined.
%    In defining `|\theLem|' we must generate the desired macro
%    name by use of |\expandafter| and |\csname|.
%    \begin{macrocode}
    \expandafter\xdef\csname the#1\endcsname
%    \end{macrocode}
%    An |\xdef| is used in order to make the definition global,
%    and to ensure that it contains the replacement texts of
%    |\@thmcountersep| and |\@thmcounter|.\footnote{These
%    two macros can be defined by the document style.  Their default
%    values produce a `\texttt{.}' as separation and an arabic
%    representation of the number.} However, not everything should be
%    expanded. For example, it saves space to use |\thesection|
%    instead of its---at times---lengthy expansion.
%    \begin{macrocode}
      {\expandafter \noexpand \csname the#3\endcsname
       \@thmcountersep \@thmcounter{#1}}%
%    \end{macrocode}
%    Thus with the defaults of \LaTeX{}, |\theLem| would be
%    replaced by the command sequence |\thesection.\arabic{Lem}|.
%
%    We will now look at the definition of the macro which is executed
%    at the beginning of the actual environment (in our example this
%    macro is |\Lem|). It should be noted that we use an
%    ``|\expandafter| trick'' to expand only certain parts of the
%    replacement text at the time of the definition.
%    \begin{macrocode}
    \def\@tempa{\global\@namedef{#1}}%
    \expandafter \@tempa \expandafter{%
%    \end{macrocode}
%    First, the macro that contains the current definitions of
%    |\@begintheorem| and |\@opargtheorem| should be called
%    up.  The name of this macro---as is already known---has the form
%    |\th@|\meta{theorem\ style}; therefore, it must be called by
%    \begin{macrocode}
      \csname th@\the \theorem@style
%    \end{macrocode}
%    In addition the default theorem font should be changeable, i.e.\
%    we have to insert the contents of |\theorem@bodyfont|. For
%    that reason, we expand even further, beyond |\endcsname|,
%    and thus insert the contents of the \textsf{token} register
%    |\theorem@bodyfont| in the replacement text.
%    \begin{macrocode}
            \expandafter \endcsname \the \theorem@bodyfont
%    \end{macrocode}
%    Now it is time to call the macro |\@thm| which takes over
%    the further processing.  It has two arguments: the current
%    counter name (in our example, \texttt{Lem}), and the text of the
%    label (in our example, \texttt{Lemma}).
%    \begin{macrocode}
     \@thm{#1}{#2}}%
%    \end{macrocode}
%    With this, the `sub-definition' is complete.  The macro
%    |\@endtheorem| ends a theorem environment and is, so far, nothing
%    but an |\endtrivlist|.  (Hence it is defined globally, and not
%    within the theorem styles.\footnote{This has to be changed as
%    soon as theorem styles that change \texttt{\bslash @endtheorem}
%    exist.  In such a case, all existing styles must be changed as
%    well since they will have to reset the macro.}) Therefore, we can
%    set it equivalent to the macro that ends the theorem set (in our
%    example, |\endLem|).  However, if some day theorem styles exist
%    that do change |\@endtheorem|, we would have to use the
%    commented-out line instead.
%    \begin{macrocode}
    \global \expandafter \let \csname end#1\endcsname \@endtheorem
%   \global\@namedef{end#1}{\@endtheorem}%
%    \end{macrocode}
%    With these commands all the required definitions are employed,
%    unless the test |\@ifdefinable| has failed.  Therefore, we
%    end the second argument of this macro and with it the definition
%    of |\@xnthm|.
%    \begin{macrocode}
   }}
%    \end{macrocode}
% \end{macro}
%
% \begin{macro}{\@ynthm}
%    The definition of |\@ynthm| is completely analogous. In
%    this case the new counter that is defined is not reset within
%    another counter; thus the definition of |\the...|
%    is simplified:
%    \begin{macrocode}
\gdef\@ynthm#1#2{\expandafter\@ifdefinable\csname #1\endcsname
   {\@definecounter{#1}%
    \expandafter\xdef\csname the#1\endcsname{\@thmcounter{#1}}%
%    \end{macrocode}
%    The rest of the definition corresponds literally to that of
%    |\@xnthm|:
%    \begin{macrocode}
    \def\@tempa{\global\@namedef{#1}}\expandafter \@tempa
     \expandafter{\csname th@\the \theorem@style \expandafter
     \endcsname \the\theorem@bodyfont \@thm{#1}{#2}}%
    \global \expandafter \let \csname end#1\endcsname \@endtheorem}}
%    \end{macrocode}
% \end{macro}
%
% \begin{macro}{\@othm}
%    The definition of |\@othm| does not contain anything new.
%    \begin{macrocode}
\gdef\@othm#1[#2]#3{%
%    \end{macrocode}
% \changes{v2.2b}{95/10/07}{Build in check for counter defined}
%    We do not define a new counter but instead use one that has
%    already been defined.  Thus the only definition we need is that
%    of this pseudo-counter (i.e.\ |\the|\meta{env.\ name}).
%    First we check if |#2| corresponds to a known counter name.
%    \begin{macrocode}
  \expandafter\ifx\csname c@#2\endcsname\relax
   \@nocounterr{#2}%
  \else
   \expandafter\@ifdefinable\csname #1\endcsname
   {\expandafter \xdef \csname the#1\endcsname
     {\expandafter \noexpand \csname the#2\endcsname}%
%    \end{macrocode}
%    All other parts of the definition can be adopted from
%    |\@xnthm|. We have to remember, though, that in this case
%    the name of the current counter and the theorem label have moved
%    to the second and third arguments.
%    \begin{macrocode}
    \def\@tempa{\global\@namedef{#1}}\expandafter \@tempa
     \expandafter{\csname th@\the \theorem@style \expandafter
     \endcsname \the\theorem@bodyfont \@thm{#2}{#3}}%
    \global \expandafter \let \csname end#1\endcsname \@endtheorem}%
  \fi}
%    \end{macrocode}
% \end{macro}
%
% \subsection{Macros that are employed in a {\envfont theorem}
%    environment}
%
% \begin{macro}{\@thm}
%    The macro |\@thm| has to increase the current counter.  Then,
%    depending on whether the environment has (or does not have) an
%    optional argument, it has to branch into either |\@begintheorem|
%    or |\@opargtheorem|.
%    \begin{macrocode}
\gdef\@thm#1#2{\refstepcounter{#1}%
%    \end{macrocode}
%    Now we start a {\envfont trivlist} environment, and give
%    |\@topsep| and |\@topsepadd| the values of the \textsf{skip}
%    registers |\theorempreskipamount| and |\theorempostskipamount|.
%    The value in |\@topsep| is the vertical space that is inserted by
%    the first (and only) |\item| in our |\trivlist| whilst
%    |\@topsepadd| is inserted by |\@endparenv| at the end of that
%    {\envfont trivlist} environment.  By using these registers, we
%    obtain the desired space around a {\envfont theorem} environment.
%    \begin{macrocode}
   \trivlist
   \@topsep \theorempreskipamount               % used by first \item
   \@topsepadd \theorempostskipamount           % used by \@endparenv
%    \end{macrocode}
%    Now we have to test whether an optional argument has been given.
% \changes{v2.0d}{89/07/18}{Forgotten percent added.}
%    \begin{macrocode}
   \@ifnextchar [%
%    \end{macrocode}
%    If there is an optional argument, we will call |\@ythm|, and
%    move the arguments read back into the input stream.
%    \begin{macrocode}
   {\@ythm{#1}{#2}}%
%    \end{macrocode}
%    If not, we call |\@begintheorem|. Its first argument is the
%    name of the theorem set (hence the second argument of
%    |\@thm|). Its second argument is the macro that produces the
%    current number.
%    \begin{macrocode}
   {\@begintheorem{#2}{\csname the#1\endcsname}\ignorespaces}}
%    \end{macrocode}
% \end{macro}
%
% \begin{macro}{\@xthm}
% \begin{macro}{\@ythm}
%    Both these macros were originally called by |\@thm|.  We do
%    not need |\@xthm| anymore, hence we reset it to
%    |\relax|. The definition of |\@ythm| has not changed at
%    all from its definition in \LaTeX. In order to make the macros
%    easier to understand, we will
%    nevertheless present it (commented out).
%    \begin{macrocode}
\global\let\@xthm\relax
% \def\@ythm#1#2[#3]{\@opargbegintheorem{#2}{\csname
%           the#1\endcsname}{#3}\ignorespaces}
%</package>
%    \end{macrocode}
%    The primitive |\ignorespaces| in |\@ythm| and
%    |\@thm| is needed to remove the spaces between the
%    |\begin{...}| and the actual text.
% \end{macro}
% \end{macro}
%
%
% \subsection{Definition of the theorem substyles}
%
% As already pointed out, the theorem substyles, defined below, are
% only loaded when necessary.  Note that all these substyles, except
% \texttt{plain}, have |\slshape| as the default body font.
%
% \subsubsection{The \texttt{plain} style}
%
% \label{dok:th@plain} As the following macros use |@|, we have
% to locally set the |\catcode| of this symbol to ``letter''.
% This happens within a group, so that we do not have to worry about
% which |\catcode| that symbol had before.
%    \begin{macrocode}
%<*thp>
\begingroup \makeatletter
%    \end{macrocode}
% Since we are now within a group, we must make all definitions
% globally.  First we make sure that \texttt{theorem.sty} is loaded.
% This will allow us to use this file as a document style option
% without having to call \texttt{theorem} itself as an option.  At the
% same time, we assure that at least version 2 is loaded, since
% |\theorem@style| was not defined in earlier versions.
%    \begin{macrocode}
\@ifundefined{theorem@style}{\input{theorem.sty}}{}
\ProvidesFile{thp.sty}
             [\FMithmInfo]
%    \end{macrocode}
%
%
% \begin{macro}{\th@plain}
% \begin{macro}{\@begintheorem}
% \begin{macro}{\@opargbegintheorem}
%    |\theoremstyle{plain}| corresponds to the original
%    definition, except that the distances to the surrounding text are
%    determined by the parameters |\theorempreskipamount| and
%    |\theorempostskipamount|.
%    First we set the default body font.
%    \begin{macrocode}
\gdef\th@plain{\normalfont\itshape
%    \end{macrocode}
%    Then we define |\@begintheorem| and
%    |\@opargbegintheorem|. These two macros define how the
%    header of a theorem is typeset.  |\@opargbegintheorem| will
%    be called if a \textsf{theorem} environment with an optional
%    argument is encountered; otherwise, the header is constructed by
%    calling |\@begintheorem|.  If one of these macros is
%    executed, we are within a \textsf{trivlist} environment started by
%    |\@thm|.  So the theorem header is produced with an
%    |\item| command.
%
%    Instead of specifying the header font directly, all standard
%    theorem styles use the |\theorem@headerfont| macro to allow
%    customization.  The extra space (|\labelsep|) is necessary
%    because of problems in the \textsf{trivlist} environment.
%    \begin{macrocode}
  \def\@begintheorem##1##2{%
        \item[\hskip\labelsep \theorem@headerfont ##1\ ##2]}%
%    \end{macrocode}
%    The definition of |\@opargbegintheorem| is completely
%    analogous. The only difference is the fact that there exists a
%    third argument (which is the optional parameter of the
%    environment and contains additional information about the
%    theorem).  Customarily we enclose it in parentheses.
%    \begin{macrocode}
\def\@opargbegintheorem##1##2##3{%
   \item[\hskip\labelsep \theorem@headerfont ##1\ ##2\ (##3)]}}
%    \end{macrocode}
% \end{macro}
% \end{macro}
% \end{macro}
%
% \noindent
% We conclude with an |\endgroup| to restore the
% |\catcode| of |@|.
%    \begin{macrocode}
\endgroup
%</thp>
%    \end{macrocode}
%
%
% \subsubsection{The \texttt{break} style}
%
% This style option is stored in the file |thb.sty|.
% For the next two lines see the documentation for |\th@plain| on
% page~\pageref{dok:th@plain}.
%    \begin{macrocode}
%<*thb>
\begingroup \makeatletter
\@ifundefined{theorem@style}{\input{theorem.sty}}{}
\ProvidesFile{thb.sty}
             [\FMithmInfo]
%    \end{macrocode}
%
% \begin{macro}{\th@break}
%    |\theoremstyle{break}| produces a line break after the name
%    of the theorem.  The font is |\slshape|.  Hence, we define
%    |\th@break| as follows:
%    \begin{macrocode}
\gdef\th@break{\normalfont\slshape
  \def\@begintheorem##1##2{\item[%
%    \end{macrocode}
%    We run into the following problem: it is not possible to create
%    the header with |\item[|\meta{title}|]| and then start
%    a new line by, for example, |\mbox{}\\|. Such a definition
%    will fail whenever a list environment follows immediately.  With
%    the above construction, the |\mbox{}| causes the switch
%    |@inlabel| (cf.\ definition of |\list| and
%    |\trivlist| in \cite{latex:source}) to be set to \texttt{false}
%    and so the following list will insert additional vertical
%    space (|\topskip|). This is quite annoying. Therefore, we
%    create the line break within the |\item|. In order to ensure
%    that the text will begin at the proper position in the following
%    line, we simply pretend that the label does not take any
%    room.\footnote{This will lead to problems whenever very high
%    symbols occurring in the line tower into the
%    heading. So, something else has to be done here sometime.}
%    \begin{macrocode}
       \rlap{\vbox{\hbox{\hskip \labelsep\theorem@headerfont ##1\ ##2}%
                   \hbox{\strut}}}]}%
%    \end{macrocode}
%    Again, the definition of |\@opargbegintheorem| is
%    completely analogous.
%    \begin{macrocode}
\def\@opargbegintheorem##1##2##3{%
  \item[\rlap{\vbox{\hbox{\hskip \labelsep \theorem@headerfont
                     ##1\ ##2\ (##3)}%
                    \hbox{\strut}}}]}}
\endgroup
%</thb>
%    \end{macrocode}
% \end{macro}
%
%
% \subsubsection{The \texttt{changebreak} style}
%
%    \begin{macrocode}
% This style option is stored in the file |thcb.sty|.
%    \begin{macrocode}
%<*thcb>
\begingroup \makeatletter
\@ifundefined{theorem@style}{\input{theorem.sty}}{}
\ProvidesFile{thcb.sty}
             [\FMithmInfo]
%    \end{macrocode}
%
% \begin{macro}{\th@changebreak}
%    The \texttt{change-break} theorem style is like \texttt{break}
%    but with interchange of theorem name and theorem number.  Thus we
%    define |\th@changebreak| as follows:
%    \begin{macrocode}
\gdef\th@changebreak{\normalfont\slshape
  \def\@begintheorem##1##2{\item
       [\rlap{\vbox{\hbox{\hskip\labelsep \theorem@headerfont ##2\ ##1}%
                    \hbox{\strut}}}]}%
\def\@opargbegintheorem##1##2##3{%
  \item[\rlap{\vbox{\hbox{\hskip\labelsep \theorem@headerfont
                     ##2\ ##1\ (##3)}%
                    \hbox{\strut}}}]}}
\endgroup
%</thcb>
%    \end{macrocode}
% \end{macro}
%
%
% \subsubsection{The \texttt{change} style}
%
% This style option is stored in the file |thc.sty|.
%    \begin{macrocode}
%<*thc>
\begingroup \makeatletter
\@ifundefined{theorem@style}{\input{theorem.sty}}{}
\ProvidesFile{thc.sty}
             [\FMithmInfo]
%    \end{macrocode}
%
% \begin{macro}{\th@change}
%    The \texttt{change} theorem style corresponds to the
%    \texttt{change break} style without a linebreak after the header.
%    To say it in another way, it's the same as the \texttt{plain}
%    style but with number and name interchanged and |\slshape| as the
%    default font.
%    \begin{macrocode}
\gdef\th@change{\normalfont\slshape
  \def\@begintheorem##1##2{\item
       [\hskip\labelsep \theorem@headerfont ##2\ ##1]}%
\def\@opargbegintheorem##1##2##3{%
  \item[\hskip\labelsep \theorem@headerfont ##2\ ##1\ (##3)]}}
\endgroup
%</thc>
%    \end{macrocode}
% \end{macro}
%
%
% \subsubsection{The \texttt{marginbreak} style}
%
% This style option is the one used most often at Mainz.  It is saved
% in the file |thmb.sty|.
%    \begin{macrocode}
%<*thmb>
\begingroup \makeatletter
\@ifundefined{theorem@style}{\input{theorem.sty}}{}
\ProvidesFile{thmb.sty}
             [\FMithmInfo]
%    \end{macrocode}
%
% \begin{macro}{\th@marginbreak}
%    The \texttt{margin break} style is nearly the same as the
%    \texttt{change break} style. The only difference is the placement
%    of the theorem number.  We use |\llap| to place it in the left
%    margin.
%
%    In this style |\labelsep| denotes the separation between the
%    number and the text.
%    \begin{macrocode}
\gdef\th@marginbreak{\normalfont\slshape
  \def\@begintheorem##1##2{\item
       [\rlap{\vbox{\theorem@headerfont
                    \hbox{\llap{##2}\hskip\labelsep ##1}%
                    \hbox{\strut}}}]}%
\def\@opargbegintheorem##1##2##3{%
  \item[\rlap{\vbox{\theorem@headerfont
                    \hbox{\llap{##2}\hskip\labelsep ##1\ (##3)}%
                    \hbox{\strut}}}]}}
\endgroup
%</thmb>
%    \end{macrocode}
% \changes{v2.1c}{90/02/17}{`endgroup outside macrocode env.}
% \end{macro}
%
%
%
% \subsubsection{The \texttt{margin} style}
%
% This style option is stored in the file |thm.sty|.
%    \begin{macrocode}
%<*thm>
\begingroup \makeatletter
\@ifundefined{theorem@style}{\input{theorem.sty}}{}
\ProvidesFile{thm.sty}
             [\FMithmInfo]
%    \end{macrocode}
%
% \begin{macro}{\th@margin}
%    Again this is only a variant of the theorem styles described
%    above without any new ideas.
%    \begin{macrocode}
\gdef\th@margin{\normalfont\slshape
  \def\@begintheorem##1##2{\item
       [\theorem@headerfont \llap{##2}\hskip\labelsep ##1]}%
\def\@opargbegintheorem##1##2##3{%
  \item[\theorem@headerfont \llap{##2}\hskip\labelsep ##1\ (##3)]}}
\endgroup
%</thm>
%    \end{macrocode}
% \end{macro}
%
%
%
%
% \subsection{Final Definitions}
%
% \begin{macro}{\theorempreskipamount}
% \begin{macro}{\theorempostskipamount}
%    The \textsf{skip} parameters that regulate the vertical empty space
%    before and after the {\envfont theorem} environment have
%    to be allocated as well.
% \changes{v2.0f}{89/07/20}{Unnecessary `global' removed.}
%    \begin{macrocode}
%<*package>
\newskip\theorempreskipamount
\newskip\theorempostskipamount
%    \end{macrocode}
%    Since we have used the same values for all theorem sets, we now can
%    assign them.
%    \begin{macrocode}
\global\setlength\theorempreskipamount{12pt plus 5pt minus 3pt}
\global\setlength\theorempostskipamount{8pt plus 3pt minus 1.5pt}
%    \end{macrocode}
% \end{macro}
% \end{macro}
%
%
% \begin{macro}{\@endtheorem}
%    The same holds for the macro |\@endtheorem|, which ends a
%    {\envfont theorem} environment. Since it is the same for all
%    theorem sets, it is removed from the macros |\th@|\meta{style}.
%    It simply ends the {\envfont trivlist} environment, which was
%    begun in |\@thm|.
%    \begin{macrocode}
\global\let\@endtheorem=\endtrivlist
%    \end{macrocode}
% \end{macro}
%
%
%
% \begin{macro}{\@preamblecmds}
%    All macros defined above are to be used only in the preamble.
%    Therefore, we insert them in |\@preamblecmds| which will disable
%    them at begin document. This is done by the internal \LaTeXe{}
%    command |\@onlypreamble|.
% \changes{v2.0g}{89/09/19}{`theoremheaderfont and `theorembodyfont
%    added.}
% \changes{v2.2a}{94/02/03}{Use \LaTeXe{} interface}
%    \begin{macrocode}
\@onlypreamble\@xnthm
\@onlypreamble\@ynthm
\@onlypreamble\@othm
\@onlypreamble\newtheorem
\@onlypreamble\theoremstyle
\@onlypreamble\theorembodyfont
\@onlypreamble\theoremheaderfont
%    \end{macrocode}
% \end{macro}
%
% Finally we declare the \texttt{plain} theorem style to be the default.
% \changes{v2.1b}{90/06/04}{Fixed missing `item by declaring default
%    style.}
%    \begin{macrocode}
\theoremstyle{plain}
%</package>
%    \end{macrocode}
%
% \Finale

\endinput
}{}
\ProvidesFile{thm.sty}
             [\FMithmInfo]
%    \end{macrocode}
%
% \begin{macro}{\th@margin}
%    Again this is only a variant of the theorem styles described
%    above without any new ideas.
%    \begin{macrocode}
\gdef\th@margin{\normalfont\slshape
  \def\@begintheorem##1##2{\item
       [\theorem@headerfont \llap{##2}\hskip\labelsep ##1]}%
\def\@opargbegintheorem##1##2##3{%
  \item[\theorem@headerfont \llap{##2}\hskip\labelsep ##1\ (##3)]}}
\endgroup
%</thm>
%    \end{macrocode}
% \end{macro}
%
%
%
%
% \subsection{Final Definitions}
%
% \begin{macro}{\theorempreskipamount}
% \begin{macro}{\theorempostskipamount}
%    The \textsf{skip} parameters that regulate the vertical empty space
%    before and after the {\envfont theorem} environment have
%    to be allocated as well.
% \changes{v2.0f}{89/07/20}{Unnecessary `global' removed.}
%    \begin{macrocode}
%<*package>
\newskip\theorempreskipamount
\newskip\theorempostskipamount
%    \end{macrocode}
%    Since we have used the same values for all theorem sets, we now can
%    assign them.
%    \begin{macrocode}
\global\setlength\theorempreskipamount{12pt plus 5pt minus 3pt}
\global\setlength\theorempostskipamount{8pt plus 3pt minus 1.5pt}
%    \end{macrocode}
% \end{macro}
% \end{macro}
%
%
% \begin{macro}{\@endtheorem}
%    The same holds for the macro |\@endtheorem|, which ends a
%    {\envfont theorem} environment. Since it is the same for all
%    theorem sets, it is removed from the macros |\th@|\meta{style}.
%    It simply ends the {\envfont trivlist} environment, which was
%    begun in |\@thm|.
%    \begin{macrocode}
\global\let\@endtheorem=\endtrivlist
%    \end{macrocode}
% \end{macro}
%
%
%
% \begin{macro}{\@preamblecmds}
%    All macros defined above are to be used only in the preamble.
%    Therefore, we insert them in |\@preamblecmds| which will disable
%    them at begin document. This is done by the internal \LaTeXe{}
%    command |\@onlypreamble|.
% \changes{v2.0g}{89/09/19}{`theoremheaderfont and `theorembodyfont
%    added.}
% \changes{v2.2a}{94/02/03}{Use \LaTeXe{} interface}
%    \begin{macrocode}
\@onlypreamble\@xnthm
\@onlypreamble\@ynthm
\@onlypreamble\@othm
\@onlypreamble\newtheorem
\@onlypreamble\theoremstyle
\@onlypreamble\theorembodyfont
\@onlypreamble\theoremheaderfont
%    \end{macrocode}
% \end{macro}
%
% Finally we declare the \texttt{plain} theorem style to be the default.
% \changes{v2.1b}{90/06/04}{Fixed missing `item by declaring default
%    style.}
%    \begin{macrocode}
\theoremstyle{plain}
%</package>
%    \end{macrocode}
%
% \Finale

\endinput
}{}
\ProvidesFile{thp.sty}
             [\FMithmInfo]
%    \end{macrocode}
%
%
% \begin{macro}{\th@plain}
% \begin{macro}{\@begintheorem}
% \begin{macro}{\@opargbegintheorem}
%    |\theoremstyle{plain}| corresponds to the original
%    definition, except that the distances to the surrounding text are
%    determined by the parameters |\theorempreskipamount| and
%    |\theorempostskipamount|.
%    First we set the default body font.
%    \begin{macrocode}
\gdef\th@plain{\normalfont\itshape
%    \end{macrocode}
%    Then we define |\@begintheorem| and
%    |\@opargbegintheorem|. These two macros define how the
%    header of a theorem is typeset.  |\@opargbegintheorem| will
%    be called if a \textsf{theorem} environment with an optional
%    argument is encountered; otherwise, the header is constructed by
%    calling |\@begintheorem|.  If one of these macros is
%    executed, we are within a \textsf{trivlist} environment started by
%    |\@thm|.  So the theorem header is produced with an
%    |\item| command.
%
%    Instead of specifying the header font directly, all standard
%    theorem styles use the |\theorem@headerfont| macro to allow
%    customization.  The extra space (|\labelsep|) is necessary
%    because of problems in the \textsf{trivlist} environment.
%    \begin{macrocode}
  \def\@begintheorem##1##2{%
        \item[\hskip\labelsep \theorem@headerfont ##1\ ##2]}%
%    \end{macrocode}
%    The definition of |\@opargbegintheorem| is completely
%    analogous. The only difference is the fact that there exists a
%    third argument (which is the optional parameter of the
%    environment and contains additional information about the
%    theorem).  Customarily we enclose it in parentheses.
%    \begin{macrocode}
\def\@opargbegintheorem##1##2##3{%
   \item[\hskip\labelsep \theorem@headerfont ##1\ ##2\ (##3)]}}
%    \end{macrocode}
% \end{macro}
% \end{macro}
% \end{macro}
%
% \noindent
% We conclude with an |\endgroup| to restore the
% |\catcode| of |@|.
%    \begin{macrocode}
\endgroup
%</thp>
%    \end{macrocode}
%
%
% \subsubsection{The \texttt{break} style}
%
% This style option is stored in the file |thb.sty|.
% For the next two lines see the documentation for |\th@plain| on
% page~\pageref{dok:th@plain}.
%    \begin{macrocode}
%<*thb>
\begingroup \makeatletter
\@ifundefined{theorem@style}{% \iffalse meta-comment
%
% Copyright (C) 1993-2024
%
% The LaTeX Project and any individual authors listed elsewhere
% in this file.
%
% This file is part of the Standard LaTeX `Tools Bundle'.
% -------------------------------------------------------
%
% It may be distributed and/or modified under the
% conditions of the LaTeX Project Public License, either version 1.3c
% of this license or (at your option) any later version.
% The latest version of this license is in
%    https://www.latex-project.org/lppl.txt
% and version 1.3c or later is part of all distributions of LaTeX
% version 2005/12/01 or later.
%
% The list of all files belonging to the LaTeX `Tools Bundle' is
% given in the file `manifest.txt'.
%
% \fi
%
%%
%
%
\def\FMithmInfo{2023/07/05 v2.2c Theorem extension package (FMi)}
%
% \ProvidesFile{theorem.dtx}[\FMithmInfo]
%\iffalse   % this is a METACOMMENT !
%%
%% Package `theorem' to use with LaTeX2e
%% Copyright (C) 1989-1995 Frank Mittelbach, all rights reserved.
%<+package>\@ifundefined{theorem@style}{}{\endinput}
%<+package>\NeedsTeXFormat{LaTeX2e}[1995/06/01]
%<+package>\ProvidesPackage{theorem}
%<+package>         [\FMithmInfo]
%
% \fi
%
% \changes{v1.0}{87/05/07}{First draft.}
% \changes{v1.0a}{87/05/07}{`thistheoremstyle' added.}
% \changes{v1.1a}{87/07/29}{New concept: redefine the internal
%                          \protect\LaTeX{} commands `@xnthm' etc.}
% \changes{v1.1b}{87/07/31}{`theoremfont' used to specify font.}
% \changes{v1.2a}{88/01/08}{`@endtheorem' now globally defined.}
% \changes{v1.2b}{88/01/11}{`newtheorem' only allowed in the preamble.}
% \changes{v1.3a}{88/02/14}{`theoremfont' now a toks register.}
%
% \changes{v2.0a}{88/11/16}{New implementation and new documentation.}
% \changes{v2.0a}{88/11/16}{No layout changes but internal modifications
%                           old substyles won't work.}
% \changes{v2.0b}{89/01/30}{`global' used for all definitions.}
% \changes{v2.0c}{89/06/15}{Documentation translated to English.
%                            Thanks to LLL and Nina}
% \changes{v2.1a}{90/02/15}{Support for new font selection.}
% \changes{v2.2a}{94/02/03}{Cleanup for LaTeX2e}
%
% \DoNotIndex{\@endparenv,\@ifdefinable,\@ifnextchar,\@ifundefined}
% \DoNotIndex{\@namedef,\@nameuse,\@spaces,\@tempa,\@warning}
% \DoNotIndex{\ }
% \DoNotIndex{\begingroup,\bfseries}
% \DoNotIndex{\csname}
% \DoNotIndex{\def,\do,\docdate}
% \DoNotIndex{\endcsname,\endgroup,\endinput,\endtrivlist,\expandafter}
% \DoNotIndex{\filedate,\fileversion}
% \DoNotIndex{\gdef,\global}
% \DoNotIndex{\hbox,\hskip}
% \DoNotIndex{\ignorespaces,\input,\itshape,\item}
% \DoNotIndex{\labelsep,\let,\llap}
% \DoNotIndex{\makeatletter}
% \DoNotIndex{\noexpand}
% \DoNotIndex{\refstepcounter,\relax,\rlap}
% \DoNotIndex{\setlength,\slshape,\space,\string,\strut}
% \DoNotIndex{\the,\trivlist,\typeout}
% \DoNotIndex{\vbox}
% \DoNotIndex{\xdef}
%
% \def\deflabel#1{\ttfamily #1\hfill}
% \def\deflist#1{\begin{list}{}{\settowidth\labelwidth{\ttfamily #1}%
%                               \setlength\leftmargin\labelwidth
%                               \addtolength\leftmargin\labelsep
%                               \let\makelabel\deflabel}}
% \def\enddeflist{\end{list}}
% \def\envfont{\normalfont\ttfamily}
%
% \MakeShortVerb{\|}
%
% \newcommand{\TUB}{{\sl TUGboat\/}}
%
% \GetFileInfo{theorem.dtx}
% \title{An Extension of the
%        \LaTeX{} theorem environment\thanks{This file
%        has version number \fileversion, last
%        revised \filedate.}}
% \author{Frank Mittelbach}
%
% \MaintainedByLaTeXTeam{tools}
% \maketitle
%
% \begin{abstract}
%    The macros described in this paper yield an extension of the
%    \LaTeX{} theorem mechanism. It is designed to satisfy the
%    different requirements of various journals. Thus, the
%    layout of the ``theorems'' can be manipulated by determining a
%    ``style''. This article describes not only the use, but
%    also the definition, of the necessary macros.
% \end{abstract}
%
% \section*{Preface to version 2.2}
%
% For \LaTeXe{} this package did not need any fundamental changes. I
% only modified the messages generated so that theorem layout styles
% will show up with the |\listfiles| command and cleaned the section
% on the New Font Selection Scheme since this is now included in
% \LaTeX.
%
% \section*{Preface to version 2.1}
%
% This version is identical to 2.0g described in \TUB\ 10\#3 except
% for some internal defaults which are now set depending on the used
% font selection scheme.
%
% This was done to avoid unpleasant surprises if the new font selection
% scheme is in force. For further details see
% section~\ref{sec:newfonts} and \cite{bk:GMS94}.
%
% \section{Introduction}
%
% For our purposes here, ``theorems'' are labelled enunciations,
% often set off from the main text by extra space and a font change.
% Theorems, corollaries, conjectures, definitions, and remarks
% are all instances of ``theorems''.  The ``header'' of these structures
% is composed of a label (such as \textsc{Theorem} or \textsc{Remark})
% and a number which serializes an item in the sequence of items
% with the same label.
%
% Shortly after the introduction of \LaTeX{} at the Fachbereich
% Mathematik in Mainz, the desire to manipulate the layout of
% ``theorems'' arose.  In Mainz, the following two conventions came
% into general use:
% \begin{enumerate}
%   \item The number of the theorem is shown in the margin.
%   \item There is a line break at
%         the end of the theorem header.
% \end{enumerate}
%
% Additionally, some journals require different formats which depend on
% the ``sort of theorem'': e.g.\ often remarks and definitions are set
% in |\upshape|, while |\itshape| is employed for main theorems.
%
% Confronted with these requirements, a theorem environment was
% developed in Mainz which allows separate determination of the
% layout of the ``theorems sets'', comparable to |\pagestyle|.
%
%
% \section{The user interface}
% \subsection{Defining new theorem sets}
%
% \DescribeMacro\newtheorem
% As in the original \LaTeX{} version, the command |\newtheorem|
% defines a new ``theorem set'' or ``theorem-like structure''.
% Two required arguments name the new environment and give the text
% to be typeset with each instance of the new ``set'', while
% an optional argument determines how
% the ``set'' is enumerated:
% \begin{description}
%    \item[\ttfamily \bslash newtheorem\{foo\}\{bar\}]
%       The theorem set {\envfont foo} (whose name is \texttt{bar})
%       uses its own counter.
%    \item[\ttfamily \bslash newtheorem\{foo2\}{[foo]}\{bar2\}]
%       The theorem set {\envfont foo2} (printed name \texttt{bar2})
%       uses the same counter as the theorem set \texttt{foo}.
%    \item[\ttfamily \bslash newtheorem\{foo3\}\{bar3\}{[section]}]
%       The theorem set {\envfont foo3} (printed name \texttt{bar3}) is
%       enumerated within the counter \texttt{section}, i.e.\ with every
%       new |\section| the enumeration begins again with 1, and
%       the enumeration is composed from the section-number and the
%       theorem counter itself.
% \end{description}
%
% \DescribeMacro\theoremstyle
% Additionally, the command |\theoremstyle| can define the layout
% of various, or all, theorem sets.  It should be noted that any
% theorem set defined by |\newtheorem| is typeset in the
% |\theoremstyle| that is current at the time of the definition.
% Thus, the following
% \begin{verbatim}
%    \theoremstyle{break}        \newtheorem{Cor}{Corollary}
%    \theoremstyle{plain}        \newtheorem{Exa}{Example}[section]
%\end{verbatim}                             ^^A No blank at line start
% leads to the result that the set {\envfont Cor} is formatted in the
% style \texttt{break}, while the set {\envfont Exa} and all the
% following ones are formatted in the style \texttt{plain}, unless
% another |\theoremstyle| follows.  Since the definitions installed by
% |\newtheorem| are global, one also can limit |\theoremstyle| locally
% by grouping braces.
%
% \DescribeMacro\theorembodyfont
% The choice of the font for the theorem body is completely
% independent of the chosen |\theoremstyle|; this has proven to
% be very advantageous.  For example,
% \begin{verbatim}
%   {\theorembodyfont{\upshape}          \newtheorem{Rem}{Remark}}
%\end{verbatim}
% defines a theorem set {\envfont Rem}, which will be set in
% |\upshape| in the current layout (which in our example is
% \texttt{plain}).  As with |\theoremstyle|, the |\theorembodyfont|
% chosen is that current at the time of |\newtheorem|.  If
% |\theorembodyfont| is not specified or one defines
% |\theorembodyfont{}|, then the font used will be that defined by the
% |\theoremstyle|.
%
% \DescribeMacro\theoremheaderfont It is also possible to customize
% the font used for the theorem headers.  This is, however, a global
% declaration, and therefore there should be at most one
% |\theoremheaderfont| declaration in the preamble.\footnote{If it is
% actually necessary to have different header fonts, one has to define
% new theorem styles (substituting the desired font) or specify the
% information directly in the \texttt{\bslash newtheorem} declaration
% (the unclean variant).}
%
% \DescribeMacro\theorempreskipamount
% \DescribeMacro\theorempostskipamount Two additional parameters
% affect the vertical space around the theorem environments:
% |\theorempreskipamount| and |\theorempostskipamount| define,
% respectively, the spacing before and after such an environment.
% These parameters apply for all theorem sets and can be manipulated
% with the ordinary length macros.  They are rubber lengths
% (`\textsf{skips}'), and therefore can contain \texttt{plus} and
% \texttt{minus} parts.
%
% Since the definition of theorem sets should---most sensibly---be
% placed in the preamble, we only allow installation there.
% It is therefore possible to release the memory used here after
% |\begin{document}|, in order to make room for other
% applications.
%
%
% \subsection{Existing theorem styles}
%
% The following theorem styles exist to date:
% \begin{deflist}{marginbreak}
%    \item[plain]
%       This theorem style emulates the original \LaTeX{} definition,
%       except that additionally the parameters
%       |\theorem...skipamount| are used.
%    \item[break]
%       In this style, the theorem header is followed by a line
%       break.
%    \item[marginbreak]
%       The theorem number is set in the margin, and there is a line
%       break as in \texttt{break}.
%    \item[changebreak]
%       Like \texttt{break}, but with header number and text
%       interchanged.
%    \item[change]
%       Header number and text are interchanged, without a line break.
%    \item[margin]
%       The number is set in the left margin, without a line break.
% \end{deflist}
% All styles (except \texttt{plain}) select |\slshape| as the default
% |\theorembodyfont|.
%
%
% \subsection{Examples}
%
% Given the above theorem sets {\envfont Cor}, {\envfont Exa}
% and {\envfont Rem},
% suppose that the preamble also contains the declarations:
% \begin{verbatim}
%    \theoremstyle{marginbreak}   \newtheorem{Lem}[Cor]{Lemma}
%    \theoremstyle{change}
%    \theorembodyfont{\itshape}        \newtheorem{Def}[Cor]{Definition}
%
%    \theoremheaderfont{\scshape}
%\end{verbatim}
% Then the following are some typical examples of the typeset output
% resulting from their use.
%
% \begin{Cor}
%    This is a sentence typeset in the theorem environment {\envfont
%    Cor}.
% \end{Cor}
%
% \begin{Exa}
%    This is a sentence typeset in the theorem environment {\envfont
%    Exa}.
% \end{Exa}
%
% \begin{Rem}
%    This is a sentence typeset in the theorem environment {\envfont
%    Rem}.
% \end{Rem}
%
% \begin{Lem}[Ben User]
%    This is a sentence typeset in the theorem environment {\envfont
%    Lem}.
% \end{Lem}
%
% \begin{Def}[Very impressive Definition]
%    This is a sentence typeset in the theorem environment {\envfont
%    Def}.
% \end{Def}
%
% The last two examples show the effect of the optional argument to
% a theorem environment (it is the text typeset in parentheses).
%
%
% \section{Special Considerations}
% \label{sec:newfonts}
%
% Theoremheader and body are implemented as a unit. This means that
% the |\theoremheaderfont|  will inherit characteristics of the
% |\theorembodyfont| in \LaTeXe. Thus, if for example
% |\theorembodyfont| is |\itshape| and |\theoremheaderfont|
% is |\bfseries| the font selected for the header will have the
% characteristics `bold extended italic'. If this is not desired one
% should set the |\theoremheaderfont| to something like
% \begin{verbatim}
%      \theoremheaderfont{\normalfont\bfseries}
%\end{verbatim}
% i.e.\ supplying all necessary font information explicitly.
%
%
%
% \section{Acknowledgements}
%
% The publication of this set of macros was only possible with the
% help of Christina Busse (translating the manuscript into English),
% Joachim Pense (playing the r\^ole of typist), Chris Rowley
% (looking everything over) and many others providing useful
% suggestions.
%
% \MaybeStop{%
%    \begin{thebibliography}{9}
%       \bibitem{bk:GMS94} \textsc{M.~Goossens}, \textsc{F.~Mittelbach}
%         and \textsc{A.~Samarin}.
%         \newblock The \LaTeX{} Companion.
%         \newblock
%         Addison-Wesley, Reading, Massachusetts, 1994.
%       \bibitem{latex:source} \textsc{Lamport, Leslie.} \newblock
%                               \texttt{latex.tex}, version 2.09, date
%                               Feb. 1990.
%    \end{thebibliography}
%    \PrintIndex
%    \PrintChanges}
%
%
% \section{The documentation driver file}
%
% The next bit of code contains the documentation driver file for
% \TeX{}, i.e., the file that will produce the documentation you are
% currently reading. It will be extracted from this file by the
% \texttt{docstrip} program. Since it is the first code in the file
% one can alternatively process this file directly with \LaTeXe{} to
% obtain the documentation.
%
%    \begin{macrocode}
%<*driver>
\documentclass{ltxdoc}

\usepackage{theorem}

%<+driver>% The next few lines define theorem sets which are used
%<+driver>% in the example section of the documentation.

\theoremstyle{break}       \newtheorem{Cor}{Corollary}
\theoremstyle{plain}       \newtheorem{Exa}{Example}[section]
{\theorembodyfont{\upshape}\newtheorem{Rem}{Remark}}
\theoremstyle{marginbreak} \newtheorem{Lem}[Cor]{Lemma}
\theoremstyle{change}
\theorembodyfont{\itshape} \newtheorem{Def}[Cor]{Definition}

\theoremheaderfont{\scshape}

\RecordChanges

\begin{document}
   \DocInput{theorem.dtx}
\end{document}
%</driver>
%    \end{macrocode}
%
% \section{Definition of the Macros}
%
% If the file has been loaded before, we abort immediately. If not the
% package announces itself (this is actually done at the very top of
% the file---the way it is done isn't good style so don't copy it).
% \changes{v2.0e}{89/07/19}{Spaces removed from `typeout'.}
% \changes{v2.2b}{95/11/19}{Announce moved to top of file}
%    \begin{macrocode}
%<*package>
%\@ifundefined{theorem@style}{}{\endinput}
%\def\FMithmInfo{1995/11/19 v2.2b Theorem extension package (FMi)}
%\ProvidesPackage{theorem}[\FMithmInfo]
%    \end{macrocode}
%
%
% \subsection{Definition of theorem styles and fonts}
%
% All the definitions in this file are done globally to allow
% inputting this file inside a group.
%
% \begin{macro}{\theoremstyle}
%    Before a theorem style can be installed, the chosen style must be
%    known. For that reason, we must test to see that
%    |\th@|\meta{style} is known or, more precisely, that it is
%    different from |\relax|. If the style is not known then
%    |\th@plain| is used.
%    \begin{macrocode}
\gdef\theoremstyle#1{%
   \@ifundefined{th@#1}{\@warning
          {Unknown theoremstyle `#1'. Using `plain'}%
          \theorem@style{plain}}%
%    \end{macrocode}
%    We save the theorem style to be used in the token register
%    |\theorem@style|.
%    \begin{macrocode}
      {\theorem@style{#1}}%
%    \end{macrocode}
%    Now we ``evaluate'' the theorem style: this means, we call the
%    macro |\th@|\meta{style} which will activate the relevant
%    definitions which are contained in a separate file.  This is done
%    in a group to suppress changes to the current font.  This could
%    otherwise pose problems together with the new font selection
%    scheme\footnote{When I printed the original article using the new
%    font selection scheme I ended with a document with slanted
%    typefaces (text headings and all) simply because one of the
%    theorem styles used \texttt{\bslash sl} at toplevel.} if the
%    |\th@|\meta{style} is evaluated a second time.
% \changes{v2.1a}{90/02/15}{Group added around `th@..}
% \changes{v2.1a}{90/02/15}{Reseting of `..@begintheorem removed.}
%    \begin{macrocode}
      \begingroup
        \csname th@\the\theorem@style \endcsname
      \endgroup}
%    \end{macrocode}
% \end{macro}
%
%
% \begin{macro}{\@begintheorem}
% \changes{v2.1a}{90/02/15}{Relaxed at toplevel.}
% \begin{macro}{\@opargbegintheorem}
% \changes{v2.1a}{90/02/15}{Relaxed at toplevel.}
%    We reset |\@begintheorem| and |\@opargbegintheorem| to
%    |\relax| since these commands are no longer necessary at
%    toplevel. This will save a few tokens.
%    \begin{macrocode}
\global\let\@begintheorem\relax
\global\let\@opargbegintheorem\relax
%    \end{macrocode}
% \end{macro}
% \end{macro}
%
% \begin{macro}{\theorem@style}
%    Obviously the \textsf{token} register used above has to be
%    allocated. To assure the utmost compatibility with the
%    original \LaTeX{} definition, we set the default theorem style to
%    \texttt{plain}, which implements the usual \LaTeX{} convention.
% \changes{v2.0f}{89/07/20}{Unnecessary `global' removed.}
%    \begin{macrocode}
\newtoks\theorem@style
\global\theorem@style{plain}
%    \end{macrocode}
% \end{macro}
%
% \begin{macro}{\theorembodyfont}
% \begin{macro}{\theorem@bodyfont}
%    If the |\theorembodyfont| is set by the user then it should not
%    interact with the default font set in the theorem style. When the
%    new font selection is in force this may happen if, for example,
%    the default is |\itshape| and the new |\theorembodyfont| is
%    |\sffamily|. So we add a |\reset@font| command in front of the
%    user definition.
% \changes{v2.0f}{89/07/20}{Unnecessary `global' removed.}
% \changes{v2.0g}{89/09/19}{`theoremfont to `theorembodyfont.}
% \changes{v2.1c}{92/06/22}{`reset@font added and logic changed.}
% \changes{v2.1d}{92/10/28}{Reset to empty if arg is empty}
%    \begin{macrocode}
\gdef\theorembodyfont#1{%
%    \end{macrocode}
%    We check if the argument supplied is empty and if so put nothing
%    into the |\theorem@bodyfont| token register to allow for
%    |\theorembodyfont{}| as a mean of using the default of the
%    current |\theoremstyle|.
%    \begin{macrocode}
   \def\@tempa{#1}%
   \ifx\@tempa\@empty
    \theorem@bodyfont{}%
   \else
    \theorem@bodyfont{\reset@font#1}%
   \fi
   }
\newtoks\theorem@bodyfont
\global\theorem@bodyfont{}
%    \end{macrocode}
% \end{macro}
% \end{macro}
%
%
% \begin{macro}{\theoremheaderfont}
% \changes{v2.0g}{89/09/19}{Macro added.}
%    The font for the theorem headers is handled differently because
%    this definition applies to all theorem styles.
%    \begin{macrocode}
\gdef\theoremheaderfont#1{\gdef\theorem@headerfont{#1}%
%    \end{macrocode}
%    After using the macro once it is redefined to produce an error
%    message.
%    \begin{macrocode}
       \gdef\theoremheaderfont##1{%
        \typeout{\string\theoremheaderfont\space should be used
                 only once.}}}
%    \end{macrocode}
% \end{macro}
%
% \begin{macro}{\theorem@headerfont}
% \changes{v2.0g}{89/09/19}{Macro added.}
%    To set the |\theorem@headerfont| default we first test if
%    the new fontselection scheme is in force.
% \changes{v2.1a}{90/02/15}{Support for new font selection.}
%    \begin{macrocode}
\ifx\upshape\undefined
%    \end{macrocode}
%    If not we define it to expand into |\bfseries|. We don't use
%    |\let| just in case a following style option redefines this
%    macro.
%    \begin{macrocode}
\gdef\theorem@headerfont{\bfseries}
%    \end{macrocode}
%    Otherwise we reset the current shape before calling |\bfseries|.
%    \begin{macrocode}
\else \gdef\theorem@headerfont{\normalfont\bfseries}\fi
%    \end{macrocode}
% \end{macro}
%
% \begin{macro}{\th@plain}
% \begin{macro}{\th@break}
% \begin{macro}{\th@marginbreak}
% \begin{macro}{\th@changebreak}
% \begin{macro}{\th@change}
% \begin{macro}{\th@margin}
%    The different styles are defined in macros such as
%    |\th@plain|.  Since memory space is precious in
%    ``non-Big-versions'', we have to avoid offering too many unused
%    definitions.  Therefore we define these styles in separate files
%    that can be loaded on demand.  Thus the commands themselves only
%    load these files. We use |\@input@|, a \LaTeXe{} internal command
%    that ensures that the file will be listed with |\listfiles|.
% \changes{v2.2a}{94/02/02}{Use `@input@ to load theorem layout files}
%    \begin{macrocode}
\gdef\th@plain{\@input@{thp.sty}}
\gdef\th@break{\@input@{thb.sty}}
\gdef\th@marginbreak{\@input@{thmb.sty}}
\gdef\th@changebreak{\@input@{thcb.sty}}
\gdef\th@change{\@input@{thc.sty}}
\gdef\th@margin{\@input@{thm.sty}}
%    \end{macrocode}
%    This list will be expanded when new styles become
%    available.  For testing,  just append new theorem substyles
%    as document options.
% \end{macro}
% \end{macro}
% \end{macro}
% \end{macro}
% \end{macro}
% \end{macro}
%
%
%
% \subsection{Definition of a new theorem set}
%
% As already pointed out, a new theorem environment can be defined in
% three different ways:
% \begin{verbatim}
%    \newtheorem{Lem}{Lemma}
%    \newtheorem{Lem}{Lemma}[section]
%    \newtheorem{Lem}[Theorem]{Lemma}
%\end{verbatim}
% The function of the macro |\newtheorem| is to recognize these cases
% and then to branch into one of the three macros |\@ynthm|, |\@xnthm|
% or |\@othm|.  This mechanism is adopted unchanged from
% \cite{latex:source}; the essential point here is that, for example,
% in the second case, the arguments \texttt{Lem}, \texttt{Lemma} and
% \texttt{section} are passed over to the macro |\@xnthm|.
%
% We inspect this case first because the others present fewer problems,
% and thus are easily derived from this one.
%
% \begin{macro}{\@xnthm}
%    For our example arguments, the macro |\@xnthm| must fulfill
%    the following:
%    \begin{itemize}
%       \item
%          Define a new \LaTeX-counter `Lem'
%       \item
%          reset this counter within a |\section|
%       \item
%          define the macro |\theLem|
%       \item
%          define the environment macros |\Lem| and
%          |\endLem| using the current
%          |\theoremstyle| and |\theorem@bodyfont|.
%    \end{itemize}
%    Obviously, all this should happen only if the first argument
%    of |\@xnthm| (i.e.\ \texttt{Lem} in our example) is chosen so
%    as not to conflict with any previously defined commands or
%    environments.  This test is performed by the \LaTeX{} macro
%    |\@ifdefinable|.
%    \begin{macrocode}
\gdef\@xnthm#1#2[#3]{\expandafter\@ifdefinable\csname #1\endcsname
%    \end{macrocode}
%    Therefore, the first argument of |\@ifdefinable| is the
%    expansion (in the example, |\Lem|) of
%    |\csname||#1\endcsname|.  The second argument is
%    executed only if the test has been completed successfully.
%    \begin{macrocode}
   {%
%    \end{macrocode}
%    Now we define the new counter. The names of the \LaTeX{}
%    macros employed should speak for themselves:
% \changes{v2.2b}{1995/10/07}{Use \cs{@newctr}}
% \changes{v2.2c}{1995/11/23}{Use \cs{@newctr} properly.}
%    \begin{macrocode}
    \@definecounter{#1}\@newctr{#1}[#3]%
%    \end{macrocode}
%    Using |\@newctr| will give a proper error message if the counter
%    in |#3| is not defined.
%    In defining `|\theLem|' we must generate the desired macro
%    name by use of |\expandafter| and |\csname|.
%    \begin{macrocode}
    \expandafter\xdef\csname the#1\endcsname
%    \end{macrocode}
%    An |\xdef| is used in order to make the definition global,
%    and to ensure that it contains the replacement texts of
%    |\@thmcountersep| and |\@thmcounter|.\footnote{These
%    two macros can be defined by the document style.  Their default
%    values produce a `\texttt{.}' as separation and an arabic
%    representation of the number.} However, not everything should be
%    expanded. For example, it saves space to use |\thesection|
%    instead of its---at times---lengthy expansion.
%    \begin{macrocode}
      {\expandafter \noexpand \csname the#3\endcsname
       \@thmcountersep \@thmcounter{#1}}%
%    \end{macrocode}
%    Thus with the defaults of \LaTeX{}, |\theLem| would be
%    replaced by the command sequence |\thesection.\arabic{Lem}|.
%
%    We will now look at the definition of the macro which is executed
%    at the beginning of the actual environment (in our example this
%    macro is |\Lem|). It should be noted that we use an
%    ``|\expandafter| trick'' to expand only certain parts of the
%    replacement text at the time of the definition.
%    \begin{macrocode}
    \def\@tempa{\global\@namedef{#1}}%
    \expandafter \@tempa \expandafter{%
%    \end{macrocode}
%    First, the macro that contains the current definitions of
%    |\@begintheorem| and |\@opargtheorem| should be called
%    up.  The name of this macro---as is already known---has the form
%    |\th@|\meta{theorem\ style}; therefore, it must be called by
%    \begin{macrocode}
      \csname th@\the \theorem@style
%    \end{macrocode}
%    In addition the default theorem font should be changeable, i.e.\
%    we have to insert the contents of |\theorem@bodyfont|. For
%    that reason, we expand even further, beyond |\endcsname|,
%    and thus insert the contents of the \textsf{token} register
%    |\theorem@bodyfont| in the replacement text.
%    \begin{macrocode}
            \expandafter \endcsname \the \theorem@bodyfont
%    \end{macrocode}
%    Now it is time to call the macro |\@thm| which takes over
%    the further processing.  It has two arguments: the current
%    counter name (in our example, \texttt{Lem}), and the text of the
%    label (in our example, \texttt{Lemma}).
%    \begin{macrocode}
     \@thm{#1}{#2}}%
%    \end{macrocode}
%    With this, the `sub-definition' is complete.  The macro
%    |\@endtheorem| ends a theorem environment and is, so far, nothing
%    but an |\endtrivlist|.  (Hence it is defined globally, and not
%    within the theorem styles.\footnote{This has to be changed as
%    soon as theorem styles that change \texttt{\bslash @endtheorem}
%    exist.  In such a case, all existing styles must be changed as
%    well since they will have to reset the macro.}) Therefore, we can
%    set it equivalent to the macro that ends the theorem set (in our
%    example, |\endLem|).  However, if some day theorem styles exist
%    that do change |\@endtheorem|, we would have to use the
%    commented-out line instead.
%    \begin{macrocode}
    \global \expandafter \let \csname end#1\endcsname \@endtheorem
%   \global\@namedef{end#1}{\@endtheorem}%
%    \end{macrocode}
%    With these commands all the required definitions are employed,
%    unless the test |\@ifdefinable| has failed.  Therefore, we
%    end the second argument of this macro and with it the definition
%    of |\@xnthm|.
%    \begin{macrocode}
   }}
%    \end{macrocode}
% \end{macro}
%
% \begin{macro}{\@ynthm}
%    The definition of |\@ynthm| is completely analogous. In
%    this case the new counter that is defined is not reset within
%    another counter; thus the definition of |\the...|
%    is simplified:
%    \begin{macrocode}
\gdef\@ynthm#1#2{\expandafter\@ifdefinable\csname #1\endcsname
   {\@definecounter{#1}%
    \expandafter\xdef\csname the#1\endcsname{\@thmcounter{#1}}%
%    \end{macrocode}
%    The rest of the definition corresponds literally to that of
%    |\@xnthm|:
%    \begin{macrocode}
    \def\@tempa{\global\@namedef{#1}}\expandafter \@tempa
     \expandafter{\csname th@\the \theorem@style \expandafter
     \endcsname \the\theorem@bodyfont \@thm{#1}{#2}}%
    \global \expandafter \let \csname end#1\endcsname \@endtheorem}}
%    \end{macrocode}
% \end{macro}
%
% \begin{macro}{\@othm}
%    The definition of |\@othm| does not contain anything new.
%    \begin{macrocode}
\gdef\@othm#1[#2]#3{%
%    \end{macrocode}
% \changes{v2.2b}{95/10/07}{Build in check for counter defined}
%    We do not define a new counter but instead use one that has
%    already been defined.  Thus the only definition we need is that
%    of this pseudo-counter (i.e.\ |\the|\meta{env.\ name}).
%    First we check if |#2| corresponds to a known counter name.
%    \begin{macrocode}
  \expandafter\ifx\csname c@#2\endcsname\relax
   \@nocounterr{#2}%
  \else
   \expandafter\@ifdefinable\csname #1\endcsname
   {\expandafter \xdef \csname the#1\endcsname
     {\expandafter \noexpand \csname the#2\endcsname}%
%    \end{macrocode}
%    All other parts of the definition can be adopted from
%    |\@xnthm|. We have to remember, though, that in this case
%    the name of the current counter and the theorem label have moved
%    to the second and third arguments.
%    \begin{macrocode}
    \def\@tempa{\global\@namedef{#1}}\expandafter \@tempa
     \expandafter{\csname th@\the \theorem@style \expandafter
     \endcsname \the\theorem@bodyfont \@thm{#2}{#3}}%
    \global \expandafter \let \csname end#1\endcsname \@endtheorem}%
  \fi}
%    \end{macrocode}
% \end{macro}
%
% \subsection{Macros that are employed in a {\envfont theorem}
%    environment}
%
% \begin{macro}{\@thm}
%    The macro |\@thm| has to increase the current counter.  Then,
%    depending on whether the environment has (or does not have) an
%    optional argument, it has to branch into either |\@begintheorem|
%    or |\@opargtheorem|.
%    \begin{macrocode}
\gdef\@thm#1#2{\refstepcounter{#1}%
%    \end{macrocode}
%    Now we start a {\envfont trivlist} environment, and give
%    |\@topsep| and |\@topsepadd| the values of the \textsf{skip}
%    registers |\theorempreskipamount| and |\theorempostskipamount|.
%    The value in |\@topsep| is the vertical space that is inserted by
%    the first (and only) |\item| in our |\trivlist| whilst
%    |\@topsepadd| is inserted by |\@endparenv| at the end of that
%    {\envfont trivlist} environment.  By using these registers, we
%    obtain the desired space around a {\envfont theorem} environment.
%    \begin{macrocode}
   \trivlist
   \@topsep \theorempreskipamount               % used by first \item
   \@topsepadd \theorempostskipamount           % used by \@endparenv
%    \end{macrocode}
%    Now we have to test whether an optional argument has been given.
% \changes{v2.0d}{89/07/18}{Forgotten percent added.}
%    \begin{macrocode}
   \@ifnextchar [%
%    \end{macrocode}
%    If there is an optional argument, we will call |\@ythm|, and
%    move the arguments read back into the input stream.
%    \begin{macrocode}
   {\@ythm{#1}{#2}}%
%    \end{macrocode}
%    If not, we call |\@begintheorem|. Its first argument is the
%    name of the theorem set (hence the second argument of
%    |\@thm|). Its second argument is the macro that produces the
%    current number.
%    \begin{macrocode}
   {\@begintheorem{#2}{\csname the#1\endcsname}\ignorespaces}}
%    \end{macrocode}
% \end{macro}
%
% \begin{macro}{\@xthm}
% \begin{macro}{\@ythm}
%    Both these macros were originally called by |\@thm|.  We do
%    not need |\@xthm| anymore, hence we reset it to
%    |\relax|. The definition of |\@ythm| has not changed at
%    all from its definition in \LaTeX. In order to make the macros
%    easier to understand, we will
%    nevertheless present it (commented out).
%    \begin{macrocode}
\global\let\@xthm\relax
% \def\@ythm#1#2[#3]{\@opargbegintheorem{#2}{\csname
%           the#1\endcsname}{#3}\ignorespaces}
%</package>
%    \end{macrocode}
%    The primitive |\ignorespaces| in |\@ythm| and
%    |\@thm| is needed to remove the spaces between the
%    |\begin{...}| and the actual text.
% \end{macro}
% \end{macro}
%
%
% \subsection{Definition of the theorem substyles}
%
% As already pointed out, the theorem substyles, defined below, are
% only loaded when necessary.  Note that all these substyles, except
% \texttt{plain}, have |\slshape| as the default body font.
%
% \subsubsection{The \texttt{plain} style}
%
% \label{dok:th@plain} As the following macros use |@|, we have
% to locally set the |\catcode| of this symbol to ``letter''.
% This happens within a group, so that we do not have to worry about
% which |\catcode| that symbol had before.
%    \begin{macrocode}
%<*thp>
\begingroup \makeatletter
%    \end{macrocode}
% Since we are now within a group, we must make all definitions
% globally.  First we make sure that \texttt{theorem.sty} is loaded.
% This will allow us to use this file as a document style option
% without having to call \texttt{theorem} itself as an option.  At the
% same time, we assure that at least version 2 is loaded, since
% |\theorem@style| was not defined in earlier versions.
%    \begin{macrocode}
\@ifundefined{theorem@style}{% \iffalse meta-comment
%
% Copyright (C) 1993-2024
%
% The LaTeX Project and any individual authors listed elsewhere
% in this file.
%
% This file is part of the Standard LaTeX `Tools Bundle'.
% -------------------------------------------------------
%
% It may be distributed and/or modified under the
% conditions of the LaTeX Project Public License, either version 1.3c
% of this license or (at your option) any later version.
% The latest version of this license is in
%    https://www.latex-project.org/lppl.txt
% and version 1.3c or later is part of all distributions of LaTeX
% version 2005/12/01 or later.
%
% The list of all files belonging to the LaTeX `Tools Bundle' is
% given in the file `manifest.txt'.
%
% \fi
%
%%
%
%
\def\FMithmInfo{2023/07/05 v2.2c Theorem extension package (FMi)}
%
% \ProvidesFile{theorem.dtx}[\FMithmInfo]
%\iffalse   % this is a METACOMMENT !
%%
%% Package `theorem' to use with LaTeX2e
%% Copyright (C) 1989-1995 Frank Mittelbach, all rights reserved.
%<+package>\@ifundefined{theorem@style}{}{\endinput}
%<+package>\NeedsTeXFormat{LaTeX2e}[1995/06/01]
%<+package>\ProvidesPackage{theorem}
%<+package>         [\FMithmInfo]
%
% \fi
%
% \changes{v1.0}{87/05/07}{First draft.}
% \changes{v1.0a}{87/05/07}{`thistheoremstyle' added.}
% \changes{v1.1a}{87/07/29}{New concept: redefine the internal
%                          \protect\LaTeX{} commands `@xnthm' etc.}
% \changes{v1.1b}{87/07/31}{`theoremfont' used to specify font.}
% \changes{v1.2a}{88/01/08}{`@endtheorem' now globally defined.}
% \changes{v1.2b}{88/01/11}{`newtheorem' only allowed in the preamble.}
% \changes{v1.3a}{88/02/14}{`theoremfont' now a toks register.}
%
% \changes{v2.0a}{88/11/16}{New implementation and new documentation.}
% \changes{v2.0a}{88/11/16}{No layout changes but internal modifications
%                           old substyles won't work.}
% \changes{v2.0b}{89/01/30}{`global' used for all definitions.}
% \changes{v2.0c}{89/06/15}{Documentation translated to English.
%                            Thanks to LLL and Nina}
% \changes{v2.1a}{90/02/15}{Support for new font selection.}
% \changes{v2.2a}{94/02/03}{Cleanup for LaTeX2e}
%
% \DoNotIndex{\@endparenv,\@ifdefinable,\@ifnextchar,\@ifundefined}
% \DoNotIndex{\@namedef,\@nameuse,\@spaces,\@tempa,\@warning}
% \DoNotIndex{\ }
% \DoNotIndex{\begingroup,\bfseries}
% \DoNotIndex{\csname}
% \DoNotIndex{\def,\do,\docdate}
% \DoNotIndex{\endcsname,\endgroup,\endinput,\endtrivlist,\expandafter}
% \DoNotIndex{\filedate,\fileversion}
% \DoNotIndex{\gdef,\global}
% \DoNotIndex{\hbox,\hskip}
% \DoNotIndex{\ignorespaces,\input,\itshape,\item}
% \DoNotIndex{\labelsep,\let,\llap}
% \DoNotIndex{\makeatletter}
% \DoNotIndex{\noexpand}
% \DoNotIndex{\refstepcounter,\relax,\rlap}
% \DoNotIndex{\setlength,\slshape,\space,\string,\strut}
% \DoNotIndex{\the,\trivlist,\typeout}
% \DoNotIndex{\vbox}
% \DoNotIndex{\xdef}
%
% \def\deflabel#1{\ttfamily #1\hfill}
% \def\deflist#1{\begin{list}{}{\settowidth\labelwidth{\ttfamily #1}%
%                               \setlength\leftmargin\labelwidth
%                               \addtolength\leftmargin\labelsep
%                               \let\makelabel\deflabel}}
% \def\enddeflist{\end{list}}
% \def\envfont{\normalfont\ttfamily}
%
% \MakeShortVerb{\|}
%
% \newcommand{\TUB}{{\sl TUGboat\/}}
%
% \GetFileInfo{theorem.dtx}
% \title{An Extension of the
%        \LaTeX{} theorem environment\thanks{This file
%        has version number \fileversion, last
%        revised \filedate.}}
% \author{Frank Mittelbach}
%
% \MaintainedByLaTeXTeam{tools}
% \maketitle
%
% \begin{abstract}
%    The macros described in this paper yield an extension of the
%    \LaTeX{} theorem mechanism. It is designed to satisfy the
%    different requirements of various journals. Thus, the
%    layout of the ``theorems'' can be manipulated by determining a
%    ``style''. This article describes not only the use, but
%    also the definition, of the necessary macros.
% \end{abstract}
%
% \section*{Preface to version 2.2}
%
% For \LaTeXe{} this package did not need any fundamental changes. I
% only modified the messages generated so that theorem layout styles
% will show up with the |\listfiles| command and cleaned the section
% on the New Font Selection Scheme since this is now included in
% \LaTeX.
%
% \section*{Preface to version 2.1}
%
% This version is identical to 2.0g described in \TUB\ 10\#3 except
% for some internal defaults which are now set depending on the used
% font selection scheme.
%
% This was done to avoid unpleasant surprises if the new font selection
% scheme is in force. For further details see
% section~\ref{sec:newfonts} and \cite{bk:GMS94}.
%
% \section{Introduction}
%
% For our purposes here, ``theorems'' are labelled enunciations,
% often set off from the main text by extra space and a font change.
% Theorems, corollaries, conjectures, definitions, and remarks
% are all instances of ``theorems''.  The ``header'' of these structures
% is composed of a label (such as \textsc{Theorem} or \textsc{Remark})
% and a number which serializes an item in the sequence of items
% with the same label.
%
% Shortly after the introduction of \LaTeX{} at the Fachbereich
% Mathematik in Mainz, the desire to manipulate the layout of
% ``theorems'' arose.  In Mainz, the following two conventions came
% into general use:
% \begin{enumerate}
%   \item The number of the theorem is shown in the margin.
%   \item There is a line break at
%         the end of the theorem header.
% \end{enumerate}
%
% Additionally, some journals require different formats which depend on
% the ``sort of theorem'': e.g.\ often remarks and definitions are set
% in |\upshape|, while |\itshape| is employed for main theorems.
%
% Confronted with these requirements, a theorem environment was
% developed in Mainz which allows separate determination of the
% layout of the ``theorems sets'', comparable to |\pagestyle|.
%
%
% \section{The user interface}
% \subsection{Defining new theorem sets}
%
% \DescribeMacro\newtheorem
% As in the original \LaTeX{} version, the command |\newtheorem|
% defines a new ``theorem set'' or ``theorem-like structure''.
% Two required arguments name the new environment and give the text
% to be typeset with each instance of the new ``set'', while
% an optional argument determines how
% the ``set'' is enumerated:
% \begin{description}
%    \item[\ttfamily \bslash newtheorem\{foo\}\{bar\}]
%       The theorem set {\envfont foo} (whose name is \texttt{bar})
%       uses its own counter.
%    \item[\ttfamily \bslash newtheorem\{foo2\}{[foo]}\{bar2\}]
%       The theorem set {\envfont foo2} (printed name \texttt{bar2})
%       uses the same counter as the theorem set \texttt{foo}.
%    \item[\ttfamily \bslash newtheorem\{foo3\}\{bar3\}{[section]}]
%       The theorem set {\envfont foo3} (printed name \texttt{bar3}) is
%       enumerated within the counter \texttt{section}, i.e.\ with every
%       new |\section| the enumeration begins again with 1, and
%       the enumeration is composed from the section-number and the
%       theorem counter itself.
% \end{description}
%
% \DescribeMacro\theoremstyle
% Additionally, the command |\theoremstyle| can define the layout
% of various, or all, theorem sets.  It should be noted that any
% theorem set defined by |\newtheorem| is typeset in the
% |\theoremstyle| that is current at the time of the definition.
% Thus, the following
% \begin{verbatim}
%    \theoremstyle{break}        \newtheorem{Cor}{Corollary}
%    \theoremstyle{plain}        \newtheorem{Exa}{Example}[section]
%\end{verbatim}                             ^^A No blank at line start
% leads to the result that the set {\envfont Cor} is formatted in the
% style \texttt{break}, while the set {\envfont Exa} and all the
% following ones are formatted in the style \texttt{plain}, unless
% another |\theoremstyle| follows.  Since the definitions installed by
% |\newtheorem| are global, one also can limit |\theoremstyle| locally
% by grouping braces.
%
% \DescribeMacro\theorembodyfont
% The choice of the font for the theorem body is completely
% independent of the chosen |\theoremstyle|; this has proven to
% be very advantageous.  For example,
% \begin{verbatim}
%   {\theorembodyfont{\upshape}          \newtheorem{Rem}{Remark}}
%\end{verbatim}
% defines a theorem set {\envfont Rem}, which will be set in
% |\upshape| in the current layout (which in our example is
% \texttt{plain}).  As with |\theoremstyle|, the |\theorembodyfont|
% chosen is that current at the time of |\newtheorem|.  If
% |\theorembodyfont| is not specified or one defines
% |\theorembodyfont{}|, then the font used will be that defined by the
% |\theoremstyle|.
%
% \DescribeMacro\theoremheaderfont It is also possible to customize
% the font used for the theorem headers.  This is, however, a global
% declaration, and therefore there should be at most one
% |\theoremheaderfont| declaration in the preamble.\footnote{If it is
% actually necessary to have different header fonts, one has to define
% new theorem styles (substituting the desired font) or specify the
% information directly in the \texttt{\bslash newtheorem} declaration
% (the unclean variant).}
%
% \DescribeMacro\theorempreskipamount
% \DescribeMacro\theorempostskipamount Two additional parameters
% affect the vertical space around the theorem environments:
% |\theorempreskipamount| and |\theorempostskipamount| define,
% respectively, the spacing before and after such an environment.
% These parameters apply for all theorem sets and can be manipulated
% with the ordinary length macros.  They are rubber lengths
% (`\textsf{skips}'), and therefore can contain \texttt{plus} and
% \texttt{minus} parts.
%
% Since the definition of theorem sets should---most sensibly---be
% placed in the preamble, we only allow installation there.
% It is therefore possible to release the memory used here after
% |\begin{document}|, in order to make room for other
% applications.
%
%
% \subsection{Existing theorem styles}
%
% The following theorem styles exist to date:
% \begin{deflist}{marginbreak}
%    \item[plain]
%       This theorem style emulates the original \LaTeX{} definition,
%       except that additionally the parameters
%       |\theorem...skipamount| are used.
%    \item[break]
%       In this style, the theorem header is followed by a line
%       break.
%    \item[marginbreak]
%       The theorem number is set in the margin, and there is a line
%       break as in \texttt{break}.
%    \item[changebreak]
%       Like \texttt{break}, but with header number and text
%       interchanged.
%    \item[change]
%       Header number and text are interchanged, without a line break.
%    \item[margin]
%       The number is set in the left margin, without a line break.
% \end{deflist}
% All styles (except \texttt{plain}) select |\slshape| as the default
% |\theorembodyfont|.
%
%
% \subsection{Examples}
%
% Given the above theorem sets {\envfont Cor}, {\envfont Exa}
% and {\envfont Rem},
% suppose that the preamble also contains the declarations:
% \begin{verbatim}
%    \theoremstyle{marginbreak}   \newtheorem{Lem}[Cor]{Lemma}
%    \theoremstyle{change}
%    \theorembodyfont{\itshape}        \newtheorem{Def}[Cor]{Definition}
%
%    \theoremheaderfont{\scshape}
%\end{verbatim}
% Then the following are some typical examples of the typeset output
% resulting from their use.
%
% \begin{Cor}
%    This is a sentence typeset in the theorem environment {\envfont
%    Cor}.
% \end{Cor}
%
% \begin{Exa}
%    This is a sentence typeset in the theorem environment {\envfont
%    Exa}.
% \end{Exa}
%
% \begin{Rem}
%    This is a sentence typeset in the theorem environment {\envfont
%    Rem}.
% \end{Rem}
%
% \begin{Lem}[Ben User]
%    This is a sentence typeset in the theorem environment {\envfont
%    Lem}.
% \end{Lem}
%
% \begin{Def}[Very impressive Definition]
%    This is a sentence typeset in the theorem environment {\envfont
%    Def}.
% \end{Def}
%
% The last two examples show the effect of the optional argument to
% a theorem environment (it is the text typeset in parentheses).
%
%
% \section{Special Considerations}
% \label{sec:newfonts}
%
% Theoremheader and body are implemented as a unit. This means that
% the |\theoremheaderfont|  will inherit characteristics of the
% |\theorembodyfont| in \LaTeXe. Thus, if for example
% |\theorembodyfont| is |\itshape| and |\theoremheaderfont|
% is |\bfseries| the font selected for the header will have the
% characteristics `bold extended italic'. If this is not desired one
% should set the |\theoremheaderfont| to something like
% \begin{verbatim}
%      \theoremheaderfont{\normalfont\bfseries}
%\end{verbatim}
% i.e.\ supplying all necessary font information explicitly.
%
%
%
% \section{Acknowledgements}
%
% The publication of this set of macros was only possible with the
% help of Christina Busse (translating the manuscript into English),
% Joachim Pense (playing the r\^ole of typist), Chris Rowley
% (looking everything over) and many others providing useful
% suggestions.
%
% \MaybeStop{%
%    \begin{thebibliography}{9}
%       \bibitem{bk:GMS94} \textsc{M.~Goossens}, \textsc{F.~Mittelbach}
%         and \textsc{A.~Samarin}.
%         \newblock The \LaTeX{} Companion.
%         \newblock
%         Addison-Wesley, Reading, Massachusetts, 1994.
%       \bibitem{latex:source} \textsc{Lamport, Leslie.} \newblock
%                               \texttt{latex.tex}, version 2.09, date
%                               Feb. 1990.
%    \end{thebibliography}
%    \PrintIndex
%    \PrintChanges}
%
%
% \section{The documentation driver file}
%
% The next bit of code contains the documentation driver file for
% \TeX{}, i.e., the file that will produce the documentation you are
% currently reading. It will be extracted from this file by the
% \texttt{docstrip} program. Since it is the first code in the file
% one can alternatively process this file directly with \LaTeXe{} to
% obtain the documentation.
%
%    \begin{macrocode}
%<*driver>
\documentclass{ltxdoc}

\usepackage{theorem}

%<+driver>% The next few lines define theorem sets which are used
%<+driver>% in the example section of the documentation.

\theoremstyle{break}       \newtheorem{Cor}{Corollary}
\theoremstyle{plain}       \newtheorem{Exa}{Example}[section]
{\theorembodyfont{\upshape}\newtheorem{Rem}{Remark}}
\theoremstyle{marginbreak} \newtheorem{Lem}[Cor]{Lemma}
\theoremstyle{change}
\theorembodyfont{\itshape} \newtheorem{Def}[Cor]{Definition}

\theoremheaderfont{\scshape}

\RecordChanges

\begin{document}
   \DocInput{theorem.dtx}
\end{document}
%</driver>
%    \end{macrocode}
%
% \section{Definition of the Macros}
%
% If the file has been loaded before, we abort immediately. If not the
% package announces itself (this is actually done at the very top of
% the file---the way it is done isn't good style so don't copy it).
% \changes{v2.0e}{89/07/19}{Spaces removed from `typeout'.}
% \changes{v2.2b}{95/11/19}{Announce moved to top of file}
%    \begin{macrocode}
%<*package>
%\@ifundefined{theorem@style}{}{\endinput}
%\def\FMithmInfo{1995/11/19 v2.2b Theorem extension package (FMi)}
%\ProvidesPackage{theorem}[\FMithmInfo]
%    \end{macrocode}
%
%
% \subsection{Definition of theorem styles and fonts}
%
% All the definitions in this file are done globally to allow
% inputting this file inside a group.
%
% \begin{macro}{\theoremstyle}
%    Before a theorem style can be installed, the chosen style must be
%    known. For that reason, we must test to see that
%    |\th@|\meta{style} is known or, more precisely, that it is
%    different from |\relax|. If the style is not known then
%    |\th@plain| is used.
%    \begin{macrocode}
\gdef\theoremstyle#1{%
   \@ifundefined{th@#1}{\@warning
          {Unknown theoremstyle `#1'. Using `plain'}%
          \theorem@style{plain}}%
%    \end{macrocode}
%    We save the theorem style to be used in the token register
%    |\theorem@style|.
%    \begin{macrocode}
      {\theorem@style{#1}}%
%    \end{macrocode}
%    Now we ``evaluate'' the theorem style: this means, we call the
%    macro |\th@|\meta{style} which will activate the relevant
%    definitions which are contained in a separate file.  This is done
%    in a group to suppress changes to the current font.  This could
%    otherwise pose problems together with the new font selection
%    scheme\footnote{When I printed the original article using the new
%    font selection scheme I ended with a document with slanted
%    typefaces (text headings and all) simply because one of the
%    theorem styles used \texttt{\bslash sl} at toplevel.} if the
%    |\th@|\meta{style} is evaluated a second time.
% \changes{v2.1a}{90/02/15}{Group added around `th@..}
% \changes{v2.1a}{90/02/15}{Reseting of `..@begintheorem removed.}
%    \begin{macrocode}
      \begingroup
        \csname th@\the\theorem@style \endcsname
      \endgroup}
%    \end{macrocode}
% \end{macro}
%
%
% \begin{macro}{\@begintheorem}
% \changes{v2.1a}{90/02/15}{Relaxed at toplevel.}
% \begin{macro}{\@opargbegintheorem}
% \changes{v2.1a}{90/02/15}{Relaxed at toplevel.}
%    We reset |\@begintheorem| and |\@opargbegintheorem| to
%    |\relax| since these commands are no longer necessary at
%    toplevel. This will save a few tokens.
%    \begin{macrocode}
\global\let\@begintheorem\relax
\global\let\@opargbegintheorem\relax
%    \end{macrocode}
% \end{macro}
% \end{macro}
%
% \begin{macro}{\theorem@style}
%    Obviously the \textsf{token} register used above has to be
%    allocated. To assure the utmost compatibility with the
%    original \LaTeX{} definition, we set the default theorem style to
%    \texttt{plain}, which implements the usual \LaTeX{} convention.
% \changes{v2.0f}{89/07/20}{Unnecessary `global' removed.}
%    \begin{macrocode}
\newtoks\theorem@style
\global\theorem@style{plain}
%    \end{macrocode}
% \end{macro}
%
% \begin{macro}{\theorembodyfont}
% \begin{macro}{\theorem@bodyfont}
%    If the |\theorembodyfont| is set by the user then it should not
%    interact with the default font set in the theorem style. When the
%    new font selection is in force this may happen if, for example,
%    the default is |\itshape| and the new |\theorembodyfont| is
%    |\sffamily|. So we add a |\reset@font| command in front of the
%    user definition.
% \changes{v2.0f}{89/07/20}{Unnecessary `global' removed.}
% \changes{v2.0g}{89/09/19}{`theoremfont to `theorembodyfont.}
% \changes{v2.1c}{92/06/22}{`reset@font added and logic changed.}
% \changes{v2.1d}{92/10/28}{Reset to empty if arg is empty}
%    \begin{macrocode}
\gdef\theorembodyfont#1{%
%    \end{macrocode}
%    We check if the argument supplied is empty and if so put nothing
%    into the |\theorem@bodyfont| token register to allow for
%    |\theorembodyfont{}| as a mean of using the default of the
%    current |\theoremstyle|.
%    \begin{macrocode}
   \def\@tempa{#1}%
   \ifx\@tempa\@empty
    \theorem@bodyfont{}%
   \else
    \theorem@bodyfont{\reset@font#1}%
   \fi
   }
\newtoks\theorem@bodyfont
\global\theorem@bodyfont{}
%    \end{macrocode}
% \end{macro}
% \end{macro}
%
%
% \begin{macro}{\theoremheaderfont}
% \changes{v2.0g}{89/09/19}{Macro added.}
%    The font for the theorem headers is handled differently because
%    this definition applies to all theorem styles.
%    \begin{macrocode}
\gdef\theoremheaderfont#1{\gdef\theorem@headerfont{#1}%
%    \end{macrocode}
%    After using the macro once it is redefined to produce an error
%    message.
%    \begin{macrocode}
       \gdef\theoremheaderfont##1{%
        \typeout{\string\theoremheaderfont\space should be used
                 only once.}}}
%    \end{macrocode}
% \end{macro}
%
% \begin{macro}{\theorem@headerfont}
% \changes{v2.0g}{89/09/19}{Macro added.}
%    To set the |\theorem@headerfont| default we first test if
%    the new fontselection scheme is in force.
% \changes{v2.1a}{90/02/15}{Support for new font selection.}
%    \begin{macrocode}
\ifx\upshape\undefined
%    \end{macrocode}
%    If not we define it to expand into |\bfseries|. We don't use
%    |\let| just in case a following style option redefines this
%    macro.
%    \begin{macrocode}
\gdef\theorem@headerfont{\bfseries}
%    \end{macrocode}
%    Otherwise we reset the current shape before calling |\bfseries|.
%    \begin{macrocode}
\else \gdef\theorem@headerfont{\normalfont\bfseries}\fi
%    \end{macrocode}
% \end{macro}
%
% \begin{macro}{\th@plain}
% \begin{macro}{\th@break}
% \begin{macro}{\th@marginbreak}
% \begin{macro}{\th@changebreak}
% \begin{macro}{\th@change}
% \begin{macro}{\th@margin}
%    The different styles are defined in macros such as
%    |\th@plain|.  Since memory space is precious in
%    ``non-Big-versions'', we have to avoid offering too many unused
%    definitions.  Therefore we define these styles in separate files
%    that can be loaded on demand.  Thus the commands themselves only
%    load these files. We use |\@input@|, a \LaTeXe{} internal command
%    that ensures that the file will be listed with |\listfiles|.
% \changes{v2.2a}{94/02/02}{Use `@input@ to load theorem layout files}
%    \begin{macrocode}
\gdef\th@plain{\@input@{thp.sty}}
\gdef\th@break{\@input@{thb.sty}}
\gdef\th@marginbreak{\@input@{thmb.sty}}
\gdef\th@changebreak{\@input@{thcb.sty}}
\gdef\th@change{\@input@{thc.sty}}
\gdef\th@margin{\@input@{thm.sty}}
%    \end{macrocode}
%    This list will be expanded when new styles become
%    available.  For testing,  just append new theorem substyles
%    as document options.
% \end{macro}
% \end{macro}
% \end{macro}
% \end{macro}
% \end{macro}
% \end{macro}
%
%
%
% \subsection{Definition of a new theorem set}
%
% As already pointed out, a new theorem environment can be defined in
% three different ways:
% \begin{verbatim}
%    \newtheorem{Lem}{Lemma}
%    \newtheorem{Lem}{Lemma}[section]
%    \newtheorem{Lem}[Theorem]{Lemma}
%\end{verbatim}
% The function of the macro |\newtheorem| is to recognize these cases
% and then to branch into one of the three macros |\@ynthm|, |\@xnthm|
% or |\@othm|.  This mechanism is adopted unchanged from
% \cite{latex:source}; the essential point here is that, for example,
% in the second case, the arguments \texttt{Lem}, \texttt{Lemma} and
% \texttt{section} are passed over to the macro |\@xnthm|.
%
% We inspect this case first because the others present fewer problems,
% and thus are easily derived from this one.
%
% \begin{macro}{\@xnthm}
%    For our example arguments, the macro |\@xnthm| must fulfill
%    the following:
%    \begin{itemize}
%       \item
%          Define a new \LaTeX-counter `Lem'
%       \item
%          reset this counter within a |\section|
%       \item
%          define the macro |\theLem|
%       \item
%          define the environment macros |\Lem| and
%          |\endLem| using the current
%          |\theoremstyle| and |\theorem@bodyfont|.
%    \end{itemize}
%    Obviously, all this should happen only if the first argument
%    of |\@xnthm| (i.e.\ \texttt{Lem} in our example) is chosen so
%    as not to conflict with any previously defined commands or
%    environments.  This test is performed by the \LaTeX{} macro
%    |\@ifdefinable|.
%    \begin{macrocode}
\gdef\@xnthm#1#2[#3]{\expandafter\@ifdefinable\csname #1\endcsname
%    \end{macrocode}
%    Therefore, the first argument of |\@ifdefinable| is the
%    expansion (in the example, |\Lem|) of
%    |\csname||#1\endcsname|.  The second argument is
%    executed only if the test has been completed successfully.
%    \begin{macrocode}
   {%
%    \end{macrocode}
%    Now we define the new counter. The names of the \LaTeX{}
%    macros employed should speak for themselves:
% \changes{v2.2b}{1995/10/07}{Use \cs{@newctr}}
% \changes{v2.2c}{1995/11/23}{Use \cs{@newctr} properly.}
%    \begin{macrocode}
    \@definecounter{#1}\@newctr{#1}[#3]%
%    \end{macrocode}
%    Using |\@newctr| will give a proper error message if the counter
%    in |#3| is not defined.
%    In defining `|\theLem|' we must generate the desired macro
%    name by use of |\expandafter| and |\csname|.
%    \begin{macrocode}
    \expandafter\xdef\csname the#1\endcsname
%    \end{macrocode}
%    An |\xdef| is used in order to make the definition global,
%    and to ensure that it contains the replacement texts of
%    |\@thmcountersep| and |\@thmcounter|.\footnote{These
%    two macros can be defined by the document style.  Their default
%    values produce a `\texttt{.}' as separation and an arabic
%    representation of the number.} However, not everything should be
%    expanded. For example, it saves space to use |\thesection|
%    instead of its---at times---lengthy expansion.
%    \begin{macrocode}
      {\expandafter \noexpand \csname the#3\endcsname
       \@thmcountersep \@thmcounter{#1}}%
%    \end{macrocode}
%    Thus with the defaults of \LaTeX{}, |\theLem| would be
%    replaced by the command sequence |\thesection.\arabic{Lem}|.
%
%    We will now look at the definition of the macro which is executed
%    at the beginning of the actual environment (in our example this
%    macro is |\Lem|). It should be noted that we use an
%    ``|\expandafter| trick'' to expand only certain parts of the
%    replacement text at the time of the definition.
%    \begin{macrocode}
    \def\@tempa{\global\@namedef{#1}}%
    \expandafter \@tempa \expandafter{%
%    \end{macrocode}
%    First, the macro that contains the current definitions of
%    |\@begintheorem| and |\@opargtheorem| should be called
%    up.  The name of this macro---as is already known---has the form
%    |\th@|\meta{theorem\ style}; therefore, it must be called by
%    \begin{macrocode}
      \csname th@\the \theorem@style
%    \end{macrocode}
%    In addition the default theorem font should be changeable, i.e.\
%    we have to insert the contents of |\theorem@bodyfont|. For
%    that reason, we expand even further, beyond |\endcsname|,
%    and thus insert the contents of the \textsf{token} register
%    |\theorem@bodyfont| in the replacement text.
%    \begin{macrocode}
            \expandafter \endcsname \the \theorem@bodyfont
%    \end{macrocode}
%    Now it is time to call the macro |\@thm| which takes over
%    the further processing.  It has two arguments: the current
%    counter name (in our example, \texttt{Lem}), and the text of the
%    label (in our example, \texttt{Lemma}).
%    \begin{macrocode}
     \@thm{#1}{#2}}%
%    \end{macrocode}
%    With this, the `sub-definition' is complete.  The macro
%    |\@endtheorem| ends a theorem environment and is, so far, nothing
%    but an |\endtrivlist|.  (Hence it is defined globally, and not
%    within the theorem styles.\footnote{This has to be changed as
%    soon as theorem styles that change \texttt{\bslash @endtheorem}
%    exist.  In such a case, all existing styles must be changed as
%    well since they will have to reset the macro.}) Therefore, we can
%    set it equivalent to the macro that ends the theorem set (in our
%    example, |\endLem|).  However, if some day theorem styles exist
%    that do change |\@endtheorem|, we would have to use the
%    commented-out line instead.
%    \begin{macrocode}
    \global \expandafter \let \csname end#1\endcsname \@endtheorem
%   \global\@namedef{end#1}{\@endtheorem}%
%    \end{macrocode}
%    With these commands all the required definitions are employed,
%    unless the test |\@ifdefinable| has failed.  Therefore, we
%    end the second argument of this macro and with it the definition
%    of |\@xnthm|.
%    \begin{macrocode}
   }}
%    \end{macrocode}
% \end{macro}
%
% \begin{macro}{\@ynthm}
%    The definition of |\@ynthm| is completely analogous. In
%    this case the new counter that is defined is not reset within
%    another counter; thus the definition of |\the...|
%    is simplified:
%    \begin{macrocode}
\gdef\@ynthm#1#2{\expandafter\@ifdefinable\csname #1\endcsname
   {\@definecounter{#1}%
    \expandafter\xdef\csname the#1\endcsname{\@thmcounter{#1}}%
%    \end{macrocode}
%    The rest of the definition corresponds literally to that of
%    |\@xnthm|:
%    \begin{macrocode}
    \def\@tempa{\global\@namedef{#1}}\expandafter \@tempa
     \expandafter{\csname th@\the \theorem@style \expandafter
     \endcsname \the\theorem@bodyfont \@thm{#1}{#2}}%
    \global \expandafter \let \csname end#1\endcsname \@endtheorem}}
%    \end{macrocode}
% \end{macro}
%
% \begin{macro}{\@othm}
%    The definition of |\@othm| does not contain anything new.
%    \begin{macrocode}
\gdef\@othm#1[#2]#3{%
%    \end{macrocode}
% \changes{v2.2b}{95/10/07}{Build in check for counter defined}
%    We do not define a new counter but instead use one that has
%    already been defined.  Thus the only definition we need is that
%    of this pseudo-counter (i.e.\ |\the|\meta{env.\ name}).
%    First we check if |#2| corresponds to a known counter name.
%    \begin{macrocode}
  \expandafter\ifx\csname c@#2\endcsname\relax
   \@nocounterr{#2}%
  \else
   \expandafter\@ifdefinable\csname #1\endcsname
   {\expandafter \xdef \csname the#1\endcsname
     {\expandafter \noexpand \csname the#2\endcsname}%
%    \end{macrocode}
%    All other parts of the definition can be adopted from
%    |\@xnthm|. We have to remember, though, that in this case
%    the name of the current counter and the theorem label have moved
%    to the second and third arguments.
%    \begin{macrocode}
    \def\@tempa{\global\@namedef{#1}}\expandafter \@tempa
     \expandafter{\csname th@\the \theorem@style \expandafter
     \endcsname \the\theorem@bodyfont \@thm{#2}{#3}}%
    \global \expandafter \let \csname end#1\endcsname \@endtheorem}%
  \fi}
%    \end{macrocode}
% \end{macro}
%
% \subsection{Macros that are employed in a {\envfont theorem}
%    environment}
%
% \begin{macro}{\@thm}
%    The macro |\@thm| has to increase the current counter.  Then,
%    depending on whether the environment has (or does not have) an
%    optional argument, it has to branch into either |\@begintheorem|
%    or |\@opargtheorem|.
%    \begin{macrocode}
\gdef\@thm#1#2{\refstepcounter{#1}%
%    \end{macrocode}
%    Now we start a {\envfont trivlist} environment, and give
%    |\@topsep| and |\@topsepadd| the values of the \textsf{skip}
%    registers |\theorempreskipamount| and |\theorempostskipamount|.
%    The value in |\@topsep| is the vertical space that is inserted by
%    the first (and only) |\item| in our |\trivlist| whilst
%    |\@topsepadd| is inserted by |\@endparenv| at the end of that
%    {\envfont trivlist} environment.  By using these registers, we
%    obtain the desired space around a {\envfont theorem} environment.
%    \begin{macrocode}
   \trivlist
   \@topsep \theorempreskipamount               % used by first \item
   \@topsepadd \theorempostskipamount           % used by \@endparenv
%    \end{macrocode}
%    Now we have to test whether an optional argument has been given.
% \changes{v2.0d}{89/07/18}{Forgotten percent added.}
%    \begin{macrocode}
   \@ifnextchar [%
%    \end{macrocode}
%    If there is an optional argument, we will call |\@ythm|, and
%    move the arguments read back into the input stream.
%    \begin{macrocode}
   {\@ythm{#1}{#2}}%
%    \end{macrocode}
%    If not, we call |\@begintheorem|. Its first argument is the
%    name of the theorem set (hence the second argument of
%    |\@thm|). Its second argument is the macro that produces the
%    current number.
%    \begin{macrocode}
   {\@begintheorem{#2}{\csname the#1\endcsname}\ignorespaces}}
%    \end{macrocode}
% \end{macro}
%
% \begin{macro}{\@xthm}
% \begin{macro}{\@ythm}
%    Both these macros were originally called by |\@thm|.  We do
%    not need |\@xthm| anymore, hence we reset it to
%    |\relax|. The definition of |\@ythm| has not changed at
%    all from its definition in \LaTeX. In order to make the macros
%    easier to understand, we will
%    nevertheless present it (commented out).
%    \begin{macrocode}
\global\let\@xthm\relax
% \def\@ythm#1#2[#3]{\@opargbegintheorem{#2}{\csname
%           the#1\endcsname}{#3}\ignorespaces}
%</package>
%    \end{macrocode}
%    The primitive |\ignorespaces| in |\@ythm| and
%    |\@thm| is needed to remove the spaces between the
%    |\begin{...}| and the actual text.
% \end{macro}
% \end{macro}
%
%
% \subsection{Definition of the theorem substyles}
%
% As already pointed out, the theorem substyles, defined below, are
% only loaded when necessary.  Note that all these substyles, except
% \texttt{plain}, have |\slshape| as the default body font.
%
% \subsubsection{The \texttt{plain} style}
%
% \label{dok:th@plain} As the following macros use |@|, we have
% to locally set the |\catcode| of this symbol to ``letter''.
% This happens within a group, so that we do not have to worry about
% which |\catcode| that symbol had before.
%    \begin{macrocode}
%<*thp>
\begingroup \makeatletter
%    \end{macrocode}
% Since we are now within a group, we must make all definitions
% globally.  First we make sure that \texttt{theorem.sty} is loaded.
% This will allow us to use this file as a document style option
% without having to call \texttt{theorem} itself as an option.  At the
% same time, we assure that at least version 2 is loaded, since
% |\theorem@style| was not defined in earlier versions.
%    \begin{macrocode}
\@ifundefined{theorem@style}{\input{theorem.sty}}{}
\ProvidesFile{thp.sty}
             [\FMithmInfo]
%    \end{macrocode}
%
%
% \begin{macro}{\th@plain}
% \begin{macro}{\@begintheorem}
% \begin{macro}{\@opargbegintheorem}
%    |\theoremstyle{plain}| corresponds to the original
%    definition, except that the distances to the surrounding text are
%    determined by the parameters |\theorempreskipamount| and
%    |\theorempostskipamount|.
%    First we set the default body font.
%    \begin{macrocode}
\gdef\th@plain{\normalfont\itshape
%    \end{macrocode}
%    Then we define |\@begintheorem| and
%    |\@opargbegintheorem|. These two macros define how the
%    header of a theorem is typeset.  |\@opargbegintheorem| will
%    be called if a \textsf{theorem} environment with an optional
%    argument is encountered; otherwise, the header is constructed by
%    calling |\@begintheorem|.  If one of these macros is
%    executed, we are within a \textsf{trivlist} environment started by
%    |\@thm|.  So the theorem header is produced with an
%    |\item| command.
%
%    Instead of specifying the header font directly, all standard
%    theorem styles use the |\theorem@headerfont| macro to allow
%    customization.  The extra space (|\labelsep|) is necessary
%    because of problems in the \textsf{trivlist} environment.
%    \begin{macrocode}
  \def\@begintheorem##1##2{%
        \item[\hskip\labelsep \theorem@headerfont ##1\ ##2]}%
%    \end{macrocode}
%    The definition of |\@opargbegintheorem| is completely
%    analogous. The only difference is the fact that there exists a
%    third argument (which is the optional parameter of the
%    environment and contains additional information about the
%    theorem).  Customarily we enclose it in parentheses.
%    \begin{macrocode}
\def\@opargbegintheorem##1##2##3{%
   \item[\hskip\labelsep \theorem@headerfont ##1\ ##2\ (##3)]}}
%    \end{macrocode}
% \end{macro}
% \end{macro}
% \end{macro}
%
% \noindent
% We conclude with an |\endgroup| to restore the
% |\catcode| of |@|.
%    \begin{macrocode}
\endgroup
%</thp>
%    \end{macrocode}
%
%
% \subsubsection{The \texttt{break} style}
%
% This style option is stored in the file |thb.sty|.
% For the next two lines see the documentation for |\th@plain| on
% page~\pageref{dok:th@plain}.
%    \begin{macrocode}
%<*thb>
\begingroup \makeatletter
\@ifundefined{theorem@style}{\input{theorem.sty}}{}
\ProvidesFile{thb.sty}
             [\FMithmInfo]
%    \end{macrocode}
%
% \begin{macro}{\th@break}
%    |\theoremstyle{break}| produces a line break after the name
%    of the theorem.  The font is |\slshape|.  Hence, we define
%    |\th@break| as follows:
%    \begin{macrocode}
\gdef\th@break{\normalfont\slshape
  \def\@begintheorem##1##2{\item[%
%    \end{macrocode}
%    We run into the following problem: it is not possible to create
%    the header with |\item[|\meta{title}|]| and then start
%    a new line by, for example, |\mbox{}\\|. Such a definition
%    will fail whenever a list environment follows immediately.  With
%    the above construction, the |\mbox{}| causes the switch
%    |@inlabel| (cf.\ definition of |\list| and
%    |\trivlist| in \cite{latex:source}) to be set to \texttt{false}
%    and so the following list will insert additional vertical
%    space (|\topskip|). This is quite annoying. Therefore, we
%    create the line break within the |\item|. In order to ensure
%    that the text will begin at the proper position in the following
%    line, we simply pretend that the label does not take any
%    room.\footnote{This will lead to problems whenever very high
%    symbols occurring in the line tower into the
%    heading. So, something else has to be done here sometime.}
%    \begin{macrocode}
       \rlap{\vbox{\hbox{\hskip \labelsep\theorem@headerfont ##1\ ##2}%
                   \hbox{\strut}}}]}%
%    \end{macrocode}
%    Again, the definition of |\@opargbegintheorem| is
%    completely analogous.
%    \begin{macrocode}
\def\@opargbegintheorem##1##2##3{%
  \item[\rlap{\vbox{\hbox{\hskip \labelsep \theorem@headerfont
                     ##1\ ##2\ (##3)}%
                    \hbox{\strut}}}]}}
\endgroup
%</thb>
%    \end{macrocode}
% \end{macro}
%
%
% \subsubsection{The \texttt{changebreak} style}
%
%    \begin{macrocode}
% This style option is stored in the file |thcb.sty|.
%    \begin{macrocode}
%<*thcb>
\begingroup \makeatletter
\@ifundefined{theorem@style}{\input{theorem.sty}}{}
\ProvidesFile{thcb.sty}
             [\FMithmInfo]
%    \end{macrocode}
%
% \begin{macro}{\th@changebreak}
%    The \texttt{change-break} theorem style is like \texttt{break}
%    but with interchange of theorem name and theorem number.  Thus we
%    define |\th@changebreak| as follows:
%    \begin{macrocode}
\gdef\th@changebreak{\normalfont\slshape
  \def\@begintheorem##1##2{\item
       [\rlap{\vbox{\hbox{\hskip\labelsep \theorem@headerfont ##2\ ##1}%
                    \hbox{\strut}}}]}%
\def\@opargbegintheorem##1##2##3{%
  \item[\rlap{\vbox{\hbox{\hskip\labelsep \theorem@headerfont
                     ##2\ ##1\ (##3)}%
                    \hbox{\strut}}}]}}
\endgroup
%</thcb>
%    \end{macrocode}
% \end{macro}
%
%
% \subsubsection{The \texttt{change} style}
%
% This style option is stored in the file |thc.sty|.
%    \begin{macrocode}
%<*thc>
\begingroup \makeatletter
\@ifundefined{theorem@style}{\input{theorem.sty}}{}
\ProvidesFile{thc.sty}
             [\FMithmInfo]
%    \end{macrocode}
%
% \begin{macro}{\th@change}
%    The \texttt{change} theorem style corresponds to the
%    \texttt{change break} style without a linebreak after the header.
%    To say it in another way, it's the same as the \texttt{plain}
%    style but with number and name interchanged and |\slshape| as the
%    default font.
%    \begin{macrocode}
\gdef\th@change{\normalfont\slshape
  \def\@begintheorem##1##2{\item
       [\hskip\labelsep \theorem@headerfont ##2\ ##1]}%
\def\@opargbegintheorem##1##2##3{%
  \item[\hskip\labelsep \theorem@headerfont ##2\ ##1\ (##3)]}}
\endgroup
%</thc>
%    \end{macrocode}
% \end{macro}
%
%
% \subsubsection{The \texttt{marginbreak} style}
%
% This style option is the one used most often at Mainz.  It is saved
% in the file |thmb.sty|.
%    \begin{macrocode}
%<*thmb>
\begingroup \makeatletter
\@ifundefined{theorem@style}{\input{theorem.sty}}{}
\ProvidesFile{thmb.sty}
             [\FMithmInfo]
%    \end{macrocode}
%
% \begin{macro}{\th@marginbreak}
%    The \texttt{margin break} style is nearly the same as the
%    \texttt{change break} style. The only difference is the placement
%    of the theorem number.  We use |\llap| to place it in the left
%    margin.
%
%    In this style |\labelsep| denotes the separation between the
%    number and the text.
%    \begin{macrocode}
\gdef\th@marginbreak{\normalfont\slshape
  \def\@begintheorem##1##2{\item
       [\rlap{\vbox{\theorem@headerfont
                    \hbox{\llap{##2}\hskip\labelsep ##1}%
                    \hbox{\strut}}}]}%
\def\@opargbegintheorem##1##2##3{%
  \item[\rlap{\vbox{\theorem@headerfont
                    \hbox{\llap{##2}\hskip\labelsep ##1\ (##3)}%
                    \hbox{\strut}}}]}}
\endgroup
%</thmb>
%    \end{macrocode}
% \changes{v2.1c}{90/02/17}{`endgroup outside macrocode env.}
% \end{macro}
%
%
%
% \subsubsection{The \texttt{margin} style}
%
% This style option is stored in the file |thm.sty|.
%    \begin{macrocode}
%<*thm>
\begingroup \makeatletter
\@ifundefined{theorem@style}{\input{theorem.sty}}{}
\ProvidesFile{thm.sty}
             [\FMithmInfo]
%    \end{macrocode}
%
% \begin{macro}{\th@margin}
%    Again this is only a variant of the theorem styles described
%    above without any new ideas.
%    \begin{macrocode}
\gdef\th@margin{\normalfont\slshape
  \def\@begintheorem##1##2{\item
       [\theorem@headerfont \llap{##2}\hskip\labelsep ##1]}%
\def\@opargbegintheorem##1##2##3{%
  \item[\theorem@headerfont \llap{##2}\hskip\labelsep ##1\ (##3)]}}
\endgroup
%</thm>
%    \end{macrocode}
% \end{macro}
%
%
%
%
% \subsection{Final Definitions}
%
% \begin{macro}{\theorempreskipamount}
% \begin{macro}{\theorempostskipamount}
%    The \textsf{skip} parameters that regulate the vertical empty space
%    before and after the {\envfont theorem} environment have
%    to be allocated as well.
% \changes{v2.0f}{89/07/20}{Unnecessary `global' removed.}
%    \begin{macrocode}
%<*package>
\newskip\theorempreskipamount
\newskip\theorempostskipamount
%    \end{macrocode}
%    Since we have used the same values for all theorem sets, we now can
%    assign them.
%    \begin{macrocode}
\global\setlength\theorempreskipamount{12pt plus 5pt minus 3pt}
\global\setlength\theorempostskipamount{8pt plus 3pt minus 1.5pt}
%    \end{macrocode}
% \end{macro}
% \end{macro}
%
%
% \begin{macro}{\@endtheorem}
%    The same holds for the macro |\@endtheorem|, which ends a
%    {\envfont theorem} environment. Since it is the same for all
%    theorem sets, it is removed from the macros |\th@|\meta{style}.
%    It simply ends the {\envfont trivlist} environment, which was
%    begun in |\@thm|.
%    \begin{macrocode}
\global\let\@endtheorem=\endtrivlist
%    \end{macrocode}
% \end{macro}
%
%
%
% \begin{macro}{\@preamblecmds}
%    All macros defined above are to be used only in the preamble.
%    Therefore, we insert them in |\@preamblecmds| which will disable
%    them at begin document. This is done by the internal \LaTeXe{}
%    command |\@onlypreamble|.
% \changes{v2.0g}{89/09/19}{`theoremheaderfont and `theorembodyfont
%    added.}
% \changes{v2.2a}{94/02/03}{Use \LaTeXe{} interface}
%    \begin{macrocode}
\@onlypreamble\@xnthm
\@onlypreamble\@ynthm
\@onlypreamble\@othm
\@onlypreamble\newtheorem
\@onlypreamble\theoremstyle
\@onlypreamble\theorembodyfont
\@onlypreamble\theoremheaderfont
%    \end{macrocode}
% \end{macro}
%
% Finally we declare the \texttt{plain} theorem style to be the default.
% \changes{v2.1b}{90/06/04}{Fixed missing `item by declaring default
%    style.}
%    \begin{macrocode}
\theoremstyle{plain}
%</package>
%    \end{macrocode}
%
% \Finale

\endinput
}{}
\ProvidesFile{thp.sty}
             [\FMithmInfo]
%    \end{macrocode}
%
%
% \begin{macro}{\th@plain}
% \begin{macro}{\@begintheorem}
% \begin{macro}{\@opargbegintheorem}
%    |\theoremstyle{plain}| corresponds to the original
%    definition, except that the distances to the surrounding text are
%    determined by the parameters |\theorempreskipamount| and
%    |\theorempostskipamount|.
%    First we set the default body font.
%    \begin{macrocode}
\gdef\th@plain{\normalfont\itshape
%    \end{macrocode}
%    Then we define |\@begintheorem| and
%    |\@opargbegintheorem|. These two macros define how the
%    header of a theorem is typeset.  |\@opargbegintheorem| will
%    be called if a \textsf{theorem} environment with an optional
%    argument is encountered; otherwise, the header is constructed by
%    calling |\@begintheorem|.  If one of these macros is
%    executed, we are within a \textsf{trivlist} environment started by
%    |\@thm|.  So the theorem header is produced with an
%    |\item| command.
%
%    Instead of specifying the header font directly, all standard
%    theorem styles use the |\theorem@headerfont| macro to allow
%    customization.  The extra space (|\labelsep|) is necessary
%    because of problems in the \textsf{trivlist} environment.
%    \begin{macrocode}
  \def\@begintheorem##1##2{%
        \item[\hskip\labelsep \theorem@headerfont ##1\ ##2]}%
%    \end{macrocode}
%    The definition of |\@opargbegintheorem| is completely
%    analogous. The only difference is the fact that there exists a
%    third argument (which is the optional parameter of the
%    environment and contains additional information about the
%    theorem).  Customarily we enclose it in parentheses.
%    \begin{macrocode}
\def\@opargbegintheorem##1##2##3{%
   \item[\hskip\labelsep \theorem@headerfont ##1\ ##2\ (##3)]}}
%    \end{macrocode}
% \end{macro}
% \end{macro}
% \end{macro}
%
% \noindent
% We conclude with an |\endgroup| to restore the
% |\catcode| of |@|.
%    \begin{macrocode}
\endgroup
%</thp>
%    \end{macrocode}
%
%
% \subsubsection{The \texttt{break} style}
%
% This style option is stored in the file |thb.sty|.
% For the next two lines see the documentation for |\th@plain| on
% page~\pageref{dok:th@plain}.
%    \begin{macrocode}
%<*thb>
\begingroup \makeatletter
\@ifundefined{theorem@style}{% \iffalse meta-comment
%
% Copyright (C) 1993-2024
%
% The LaTeX Project and any individual authors listed elsewhere
% in this file.
%
% This file is part of the Standard LaTeX `Tools Bundle'.
% -------------------------------------------------------
%
% It may be distributed and/or modified under the
% conditions of the LaTeX Project Public License, either version 1.3c
% of this license or (at your option) any later version.
% The latest version of this license is in
%    https://www.latex-project.org/lppl.txt
% and version 1.3c or later is part of all distributions of LaTeX
% version 2005/12/01 or later.
%
% The list of all files belonging to the LaTeX `Tools Bundle' is
% given in the file `manifest.txt'.
%
% \fi
%
%%
%
%
\def\FMithmInfo{2023/07/05 v2.2c Theorem extension package (FMi)}
%
% \ProvidesFile{theorem.dtx}[\FMithmInfo]
%\iffalse   % this is a METACOMMENT !
%%
%% Package `theorem' to use with LaTeX2e
%% Copyright (C) 1989-1995 Frank Mittelbach, all rights reserved.
%<+package>\@ifundefined{theorem@style}{}{\endinput}
%<+package>\NeedsTeXFormat{LaTeX2e}[1995/06/01]
%<+package>\ProvidesPackage{theorem}
%<+package>         [\FMithmInfo]
%
% \fi
%
% \changes{v1.0}{87/05/07}{First draft.}
% \changes{v1.0a}{87/05/07}{`thistheoremstyle' added.}
% \changes{v1.1a}{87/07/29}{New concept: redefine the internal
%                          \protect\LaTeX{} commands `@xnthm' etc.}
% \changes{v1.1b}{87/07/31}{`theoremfont' used to specify font.}
% \changes{v1.2a}{88/01/08}{`@endtheorem' now globally defined.}
% \changes{v1.2b}{88/01/11}{`newtheorem' only allowed in the preamble.}
% \changes{v1.3a}{88/02/14}{`theoremfont' now a toks register.}
%
% \changes{v2.0a}{88/11/16}{New implementation and new documentation.}
% \changes{v2.0a}{88/11/16}{No layout changes but internal modifications
%                           old substyles won't work.}
% \changes{v2.0b}{89/01/30}{`global' used for all definitions.}
% \changes{v2.0c}{89/06/15}{Documentation translated to English.
%                            Thanks to LLL and Nina}
% \changes{v2.1a}{90/02/15}{Support for new font selection.}
% \changes{v2.2a}{94/02/03}{Cleanup for LaTeX2e}
%
% \DoNotIndex{\@endparenv,\@ifdefinable,\@ifnextchar,\@ifundefined}
% \DoNotIndex{\@namedef,\@nameuse,\@spaces,\@tempa,\@warning}
% \DoNotIndex{\ }
% \DoNotIndex{\begingroup,\bfseries}
% \DoNotIndex{\csname}
% \DoNotIndex{\def,\do,\docdate}
% \DoNotIndex{\endcsname,\endgroup,\endinput,\endtrivlist,\expandafter}
% \DoNotIndex{\filedate,\fileversion}
% \DoNotIndex{\gdef,\global}
% \DoNotIndex{\hbox,\hskip}
% \DoNotIndex{\ignorespaces,\input,\itshape,\item}
% \DoNotIndex{\labelsep,\let,\llap}
% \DoNotIndex{\makeatletter}
% \DoNotIndex{\noexpand}
% \DoNotIndex{\refstepcounter,\relax,\rlap}
% \DoNotIndex{\setlength,\slshape,\space,\string,\strut}
% \DoNotIndex{\the,\trivlist,\typeout}
% \DoNotIndex{\vbox}
% \DoNotIndex{\xdef}
%
% \def\deflabel#1{\ttfamily #1\hfill}
% \def\deflist#1{\begin{list}{}{\settowidth\labelwidth{\ttfamily #1}%
%                               \setlength\leftmargin\labelwidth
%                               \addtolength\leftmargin\labelsep
%                               \let\makelabel\deflabel}}
% \def\enddeflist{\end{list}}
% \def\envfont{\normalfont\ttfamily}
%
% \MakeShortVerb{\|}
%
% \newcommand{\TUB}{{\sl TUGboat\/}}
%
% \GetFileInfo{theorem.dtx}
% \title{An Extension of the
%        \LaTeX{} theorem environment\thanks{This file
%        has version number \fileversion, last
%        revised \filedate.}}
% \author{Frank Mittelbach}
%
% \MaintainedByLaTeXTeam{tools}
% \maketitle
%
% \begin{abstract}
%    The macros described in this paper yield an extension of the
%    \LaTeX{} theorem mechanism. It is designed to satisfy the
%    different requirements of various journals. Thus, the
%    layout of the ``theorems'' can be manipulated by determining a
%    ``style''. This article describes not only the use, but
%    also the definition, of the necessary macros.
% \end{abstract}
%
% \section*{Preface to version 2.2}
%
% For \LaTeXe{} this package did not need any fundamental changes. I
% only modified the messages generated so that theorem layout styles
% will show up with the |\listfiles| command and cleaned the section
% on the New Font Selection Scheme since this is now included in
% \LaTeX.
%
% \section*{Preface to version 2.1}
%
% This version is identical to 2.0g described in \TUB\ 10\#3 except
% for some internal defaults which are now set depending on the used
% font selection scheme.
%
% This was done to avoid unpleasant surprises if the new font selection
% scheme is in force. For further details see
% section~\ref{sec:newfonts} and \cite{bk:GMS94}.
%
% \section{Introduction}
%
% For our purposes here, ``theorems'' are labelled enunciations,
% often set off from the main text by extra space and a font change.
% Theorems, corollaries, conjectures, definitions, and remarks
% are all instances of ``theorems''.  The ``header'' of these structures
% is composed of a label (such as \textsc{Theorem} or \textsc{Remark})
% and a number which serializes an item in the sequence of items
% with the same label.
%
% Shortly after the introduction of \LaTeX{} at the Fachbereich
% Mathematik in Mainz, the desire to manipulate the layout of
% ``theorems'' arose.  In Mainz, the following two conventions came
% into general use:
% \begin{enumerate}
%   \item The number of the theorem is shown in the margin.
%   \item There is a line break at
%         the end of the theorem header.
% \end{enumerate}
%
% Additionally, some journals require different formats which depend on
% the ``sort of theorem'': e.g.\ often remarks and definitions are set
% in |\upshape|, while |\itshape| is employed for main theorems.
%
% Confronted with these requirements, a theorem environment was
% developed in Mainz which allows separate determination of the
% layout of the ``theorems sets'', comparable to |\pagestyle|.
%
%
% \section{The user interface}
% \subsection{Defining new theorem sets}
%
% \DescribeMacro\newtheorem
% As in the original \LaTeX{} version, the command |\newtheorem|
% defines a new ``theorem set'' or ``theorem-like structure''.
% Two required arguments name the new environment and give the text
% to be typeset with each instance of the new ``set'', while
% an optional argument determines how
% the ``set'' is enumerated:
% \begin{description}
%    \item[\ttfamily \bslash newtheorem\{foo\}\{bar\}]
%       The theorem set {\envfont foo} (whose name is \texttt{bar})
%       uses its own counter.
%    \item[\ttfamily \bslash newtheorem\{foo2\}{[foo]}\{bar2\}]
%       The theorem set {\envfont foo2} (printed name \texttt{bar2})
%       uses the same counter as the theorem set \texttt{foo}.
%    \item[\ttfamily \bslash newtheorem\{foo3\}\{bar3\}{[section]}]
%       The theorem set {\envfont foo3} (printed name \texttt{bar3}) is
%       enumerated within the counter \texttt{section}, i.e.\ with every
%       new |\section| the enumeration begins again with 1, and
%       the enumeration is composed from the section-number and the
%       theorem counter itself.
% \end{description}
%
% \DescribeMacro\theoremstyle
% Additionally, the command |\theoremstyle| can define the layout
% of various, or all, theorem sets.  It should be noted that any
% theorem set defined by |\newtheorem| is typeset in the
% |\theoremstyle| that is current at the time of the definition.
% Thus, the following
% \begin{verbatim}
%    \theoremstyle{break}        \newtheorem{Cor}{Corollary}
%    \theoremstyle{plain}        \newtheorem{Exa}{Example}[section]
%\end{verbatim}                             ^^A No blank at line start
% leads to the result that the set {\envfont Cor} is formatted in the
% style \texttt{break}, while the set {\envfont Exa} and all the
% following ones are formatted in the style \texttt{plain}, unless
% another |\theoremstyle| follows.  Since the definitions installed by
% |\newtheorem| are global, one also can limit |\theoremstyle| locally
% by grouping braces.
%
% \DescribeMacro\theorembodyfont
% The choice of the font for the theorem body is completely
% independent of the chosen |\theoremstyle|; this has proven to
% be very advantageous.  For example,
% \begin{verbatim}
%   {\theorembodyfont{\upshape}          \newtheorem{Rem}{Remark}}
%\end{verbatim}
% defines a theorem set {\envfont Rem}, which will be set in
% |\upshape| in the current layout (which in our example is
% \texttt{plain}).  As with |\theoremstyle|, the |\theorembodyfont|
% chosen is that current at the time of |\newtheorem|.  If
% |\theorembodyfont| is not specified or one defines
% |\theorembodyfont{}|, then the font used will be that defined by the
% |\theoremstyle|.
%
% \DescribeMacro\theoremheaderfont It is also possible to customize
% the font used for the theorem headers.  This is, however, a global
% declaration, and therefore there should be at most one
% |\theoremheaderfont| declaration in the preamble.\footnote{If it is
% actually necessary to have different header fonts, one has to define
% new theorem styles (substituting the desired font) or specify the
% information directly in the \texttt{\bslash newtheorem} declaration
% (the unclean variant).}
%
% \DescribeMacro\theorempreskipamount
% \DescribeMacro\theorempostskipamount Two additional parameters
% affect the vertical space around the theorem environments:
% |\theorempreskipamount| and |\theorempostskipamount| define,
% respectively, the spacing before and after such an environment.
% These parameters apply for all theorem sets and can be manipulated
% with the ordinary length macros.  They are rubber lengths
% (`\textsf{skips}'), and therefore can contain \texttt{plus} and
% \texttt{minus} parts.
%
% Since the definition of theorem sets should---most sensibly---be
% placed in the preamble, we only allow installation there.
% It is therefore possible to release the memory used here after
% |\begin{document}|, in order to make room for other
% applications.
%
%
% \subsection{Existing theorem styles}
%
% The following theorem styles exist to date:
% \begin{deflist}{marginbreak}
%    \item[plain]
%       This theorem style emulates the original \LaTeX{} definition,
%       except that additionally the parameters
%       |\theorem...skipamount| are used.
%    \item[break]
%       In this style, the theorem header is followed by a line
%       break.
%    \item[marginbreak]
%       The theorem number is set in the margin, and there is a line
%       break as in \texttt{break}.
%    \item[changebreak]
%       Like \texttt{break}, but with header number and text
%       interchanged.
%    \item[change]
%       Header number and text are interchanged, without a line break.
%    \item[margin]
%       The number is set in the left margin, without a line break.
% \end{deflist}
% All styles (except \texttt{plain}) select |\slshape| as the default
% |\theorembodyfont|.
%
%
% \subsection{Examples}
%
% Given the above theorem sets {\envfont Cor}, {\envfont Exa}
% and {\envfont Rem},
% suppose that the preamble also contains the declarations:
% \begin{verbatim}
%    \theoremstyle{marginbreak}   \newtheorem{Lem}[Cor]{Lemma}
%    \theoremstyle{change}
%    \theorembodyfont{\itshape}        \newtheorem{Def}[Cor]{Definition}
%
%    \theoremheaderfont{\scshape}
%\end{verbatim}
% Then the following are some typical examples of the typeset output
% resulting from their use.
%
% \begin{Cor}
%    This is a sentence typeset in the theorem environment {\envfont
%    Cor}.
% \end{Cor}
%
% \begin{Exa}
%    This is a sentence typeset in the theorem environment {\envfont
%    Exa}.
% \end{Exa}
%
% \begin{Rem}
%    This is a sentence typeset in the theorem environment {\envfont
%    Rem}.
% \end{Rem}
%
% \begin{Lem}[Ben User]
%    This is a sentence typeset in the theorem environment {\envfont
%    Lem}.
% \end{Lem}
%
% \begin{Def}[Very impressive Definition]
%    This is a sentence typeset in the theorem environment {\envfont
%    Def}.
% \end{Def}
%
% The last two examples show the effect of the optional argument to
% a theorem environment (it is the text typeset in parentheses).
%
%
% \section{Special Considerations}
% \label{sec:newfonts}
%
% Theoremheader and body are implemented as a unit. This means that
% the |\theoremheaderfont|  will inherit characteristics of the
% |\theorembodyfont| in \LaTeXe. Thus, if for example
% |\theorembodyfont| is |\itshape| and |\theoremheaderfont|
% is |\bfseries| the font selected for the header will have the
% characteristics `bold extended italic'. If this is not desired one
% should set the |\theoremheaderfont| to something like
% \begin{verbatim}
%      \theoremheaderfont{\normalfont\bfseries}
%\end{verbatim}
% i.e.\ supplying all necessary font information explicitly.
%
%
%
% \section{Acknowledgements}
%
% The publication of this set of macros was only possible with the
% help of Christina Busse (translating the manuscript into English),
% Joachim Pense (playing the r\^ole of typist), Chris Rowley
% (looking everything over) and many others providing useful
% suggestions.
%
% \MaybeStop{%
%    \begin{thebibliography}{9}
%       \bibitem{bk:GMS94} \textsc{M.~Goossens}, \textsc{F.~Mittelbach}
%         and \textsc{A.~Samarin}.
%         \newblock The \LaTeX{} Companion.
%         \newblock
%         Addison-Wesley, Reading, Massachusetts, 1994.
%       \bibitem{latex:source} \textsc{Lamport, Leslie.} \newblock
%                               \texttt{latex.tex}, version 2.09, date
%                               Feb. 1990.
%    \end{thebibliography}
%    \PrintIndex
%    \PrintChanges}
%
%
% \section{The documentation driver file}
%
% The next bit of code contains the documentation driver file for
% \TeX{}, i.e., the file that will produce the documentation you are
% currently reading. It will be extracted from this file by the
% \texttt{docstrip} program. Since it is the first code in the file
% one can alternatively process this file directly with \LaTeXe{} to
% obtain the documentation.
%
%    \begin{macrocode}
%<*driver>
\documentclass{ltxdoc}

\usepackage{theorem}

%<+driver>% The next few lines define theorem sets which are used
%<+driver>% in the example section of the documentation.

\theoremstyle{break}       \newtheorem{Cor}{Corollary}
\theoremstyle{plain}       \newtheorem{Exa}{Example}[section]
{\theorembodyfont{\upshape}\newtheorem{Rem}{Remark}}
\theoremstyle{marginbreak} \newtheorem{Lem}[Cor]{Lemma}
\theoremstyle{change}
\theorembodyfont{\itshape} \newtheorem{Def}[Cor]{Definition}

\theoremheaderfont{\scshape}

\RecordChanges

\begin{document}
   \DocInput{theorem.dtx}
\end{document}
%</driver>
%    \end{macrocode}
%
% \section{Definition of the Macros}
%
% If the file has been loaded before, we abort immediately. If not the
% package announces itself (this is actually done at the very top of
% the file---the way it is done isn't good style so don't copy it).
% \changes{v2.0e}{89/07/19}{Spaces removed from `typeout'.}
% \changes{v2.2b}{95/11/19}{Announce moved to top of file}
%    \begin{macrocode}
%<*package>
%\@ifundefined{theorem@style}{}{\endinput}
%\def\FMithmInfo{1995/11/19 v2.2b Theorem extension package (FMi)}
%\ProvidesPackage{theorem}[\FMithmInfo]
%    \end{macrocode}
%
%
% \subsection{Definition of theorem styles and fonts}
%
% All the definitions in this file are done globally to allow
% inputting this file inside a group.
%
% \begin{macro}{\theoremstyle}
%    Before a theorem style can be installed, the chosen style must be
%    known. For that reason, we must test to see that
%    |\th@|\meta{style} is known or, more precisely, that it is
%    different from |\relax|. If the style is not known then
%    |\th@plain| is used.
%    \begin{macrocode}
\gdef\theoremstyle#1{%
   \@ifundefined{th@#1}{\@warning
          {Unknown theoremstyle `#1'. Using `plain'}%
          \theorem@style{plain}}%
%    \end{macrocode}
%    We save the theorem style to be used in the token register
%    |\theorem@style|.
%    \begin{macrocode}
      {\theorem@style{#1}}%
%    \end{macrocode}
%    Now we ``evaluate'' the theorem style: this means, we call the
%    macro |\th@|\meta{style} which will activate the relevant
%    definitions which are contained in a separate file.  This is done
%    in a group to suppress changes to the current font.  This could
%    otherwise pose problems together with the new font selection
%    scheme\footnote{When I printed the original article using the new
%    font selection scheme I ended with a document with slanted
%    typefaces (text headings and all) simply because one of the
%    theorem styles used \texttt{\bslash sl} at toplevel.} if the
%    |\th@|\meta{style} is evaluated a second time.
% \changes{v2.1a}{90/02/15}{Group added around `th@..}
% \changes{v2.1a}{90/02/15}{Reseting of `..@begintheorem removed.}
%    \begin{macrocode}
      \begingroup
        \csname th@\the\theorem@style \endcsname
      \endgroup}
%    \end{macrocode}
% \end{macro}
%
%
% \begin{macro}{\@begintheorem}
% \changes{v2.1a}{90/02/15}{Relaxed at toplevel.}
% \begin{macro}{\@opargbegintheorem}
% \changes{v2.1a}{90/02/15}{Relaxed at toplevel.}
%    We reset |\@begintheorem| and |\@opargbegintheorem| to
%    |\relax| since these commands are no longer necessary at
%    toplevel. This will save a few tokens.
%    \begin{macrocode}
\global\let\@begintheorem\relax
\global\let\@opargbegintheorem\relax
%    \end{macrocode}
% \end{macro}
% \end{macro}
%
% \begin{macro}{\theorem@style}
%    Obviously the \textsf{token} register used above has to be
%    allocated. To assure the utmost compatibility with the
%    original \LaTeX{} definition, we set the default theorem style to
%    \texttt{plain}, which implements the usual \LaTeX{} convention.
% \changes{v2.0f}{89/07/20}{Unnecessary `global' removed.}
%    \begin{macrocode}
\newtoks\theorem@style
\global\theorem@style{plain}
%    \end{macrocode}
% \end{macro}
%
% \begin{macro}{\theorembodyfont}
% \begin{macro}{\theorem@bodyfont}
%    If the |\theorembodyfont| is set by the user then it should not
%    interact with the default font set in the theorem style. When the
%    new font selection is in force this may happen if, for example,
%    the default is |\itshape| and the new |\theorembodyfont| is
%    |\sffamily|. So we add a |\reset@font| command in front of the
%    user definition.
% \changes{v2.0f}{89/07/20}{Unnecessary `global' removed.}
% \changes{v2.0g}{89/09/19}{`theoremfont to `theorembodyfont.}
% \changes{v2.1c}{92/06/22}{`reset@font added and logic changed.}
% \changes{v2.1d}{92/10/28}{Reset to empty if arg is empty}
%    \begin{macrocode}
\gdef\theorembodyfont#1{%
%    \end{macrocode}
%    We check if the argument supplied is empty and if so put nothing
%    into the |\theorem@bodyfont| token register to allow for
%    |\theorembodyfont{}| as a mean of using the default of the
%    current |\theoremstyle|.
%    \begin{macrocode}
   \def\@tempa{#1}%
   \ifx\@tempa\@empty
    \theorem@bodyfont{}%
   \else
    \theorem@bodyfont{\reset@font#1}%
   \fi
   }
\newtoks\theorem@bodyfont
\global\theorem@bodyfont{}
%    \end{macrocode}
% \end{macro}
% \end{macro}
%
%
% \begin{macro}{\theoremheaderfont}
% \changes{v2.0g}{89/09/19}{Macro added.}
%    The font for the theorem headers is handled differently because
%    this definition applies to all theorem styles.
%    \begin{macrocode}
\gdef\theoremheaderfont#1{\gdef\theorem@headerfont{#1}%
%    \end{macrocode}
%    After using the macro once it is redefined to produce an error
%    message.
%    \begin{macrocode}
       \gdef\theoremheaderfont##1{%
        \typeout{\string\theoremheaderfont\space should be used
                 only once.}}}
%    \end{macrocode}
% \end{macro}
%
% \begin{macro}{\theorem@headerfont}
% \changes{v2.0g}{89/09/19}{Macro added.}
%    To set the |\theorem@headerfont| default we first test if
%    the new fontselection scheme is in force.
% \changes{v2.1a}{90/02/15}{Support for new font selection.}
%    \begin{macrocode}
\ifx\upshape\undefined
%    \end{macrocode}
%    If not we define it to expand into |\bfseries|. We don't use
%    |\let| just in case a following style option redefines this
%    macro.
%    \begin{macrocode}
\gdef\theorem@headerfont{\bfseries}
%    \end{macrocode}
%    Otherwise we reset the current shape before calling |\bfseries|.
%    \begin{macrocode}
\else \gdef\theorem@headerfont{\normalfont\bfseries}\fi
%    \end{macrocode}
% \end{macro}
%
% \begin{macro}{\th@plain}
% \begin{macro}{\th@break}
% \begin{macro}{\th@marginbreak}
% \begin{macro}{\th@changebreak}
% \begin{macro}{\th@change}
% \begin{macro}{\th@margin}
%    The different styles are defined in macros such as
%    |\th@plain|.  Since memory space is precious in
%    ``non-Big-versions'', we have to avoid offering too many unused
%    definitions.  Therefore we define these styles in separate files
%    that can be loaded on demand.  Thus the commands themselves only
%    load these files. We use |\@input@|, a \LaTeXe{} internal command
%    that ensures that the file will be listed with |\listfiles|.
% \changes{v2.2a}{94/02/02}{Use `@input@ to load theorem layout files}
%    \begin{macrocode}
\gdef\th@plain{\@input@{thp.sty}}
\gdef\th@break{\@input@{thb.sty}}
\gdef\th@marginbreak{\@input@{thmb.sty}}
\gdef\th@changebreak{\@input@{thcb.sty}}
\gdef\th@change{\@input@{thc.sty}}
\gdef\th@margin{\@input@{thm.sty}}
%    \end{macrocode}
%    This list will be expanded when new styles become
%    available.  For testing,  just append new theorem substyles
%    as document options.
% \end{macro}
% \end{macro}
% \end{macro}
% \end{macro}
% \end{macro}
% \end{macro}
%
%
%
% \subsection{Definition of a new theorem set}
%
% As already pointed out, a new theorem environment can be defined in
% three different ways:
% \begin{verbatim}
%    \newtheorem{Lem}{Lemma}
%    \newtheorem{Lem}{Lemma}[section]
%    \newtheorem{Lem}[Theorem]{Lemma}
%\end{verbatim}
% The function of the macro |\newtheorem| is to recognize these cases
% and then to branch into one of the three macros |\@ynthm|, |\@xnthm|
% or |\@othm|.  This mechanism is adopted unchanged from
% \cite{latex:source}; the essential point here is that, for example,
% in the second case, the arguments \texttt{Lem}, \texttt{Lemma} and
% \texttt{section} are passed over to the macro |\@xnthm|.
%
% We inspect this case first because the others present fewer problems,
% and thus are easily derived from this one.
%
% \begin{macro}{\@xnthm}
%    For our example arguments, the macro |\@xnthm| must fulfill
%    the following:
%    \begin{itemize}
%       \item
%          Define a new \LaTeX-counter `Lem'
%       \item
%          reset this counter within a |\section|
%       \item
%          define the macro |\theLem|
%       \item
%          define the environment macros |\Lem| and
%          |\endLem| using the current
%          |\theoremstyle| and |\theorem@bodyfont|.
%    \end{itemize}
%    Obviously, all this should happen only if the first argument
%    of |\@xnthm| (i.e.\ \texttt{Lem} in our example) is chosen so
%    as not to conflict with any previously defined commands or
%    environments.  This test is performed by the \LaTeX{} macro
%    |\@ifdefinable|.
%    \begin{macrocode}
\gdef\@xnthm#1#2[#3]{\expandafter\@ifdefinable\csname #1\endcsname
%    \end{macrocode}
%    Therefore, the first argument of |\@ifdefinable| is the
%    expansion (in the example, |\Lem|) of
%    |\csname||#1\endcsname|.  The second argument is
%    executed only if the test has been completed successfully.
%    \begin{macrocode}
   {%
%    \end{macrocode}
%    Now we define the new counter. The names of the \LaTeX{}
%    macros employed should speak for themselves:
% \changes{v2.2b}{1995/10/07}{Use \cs{@newctr}}
% \changes{v2.2c}{1995/11/23}{Use \cs{@newctr} properly.}
%    \begin{macrocode}
    \@definecounter{#1}\@newctr{#1}[#3]%
%    \end{macrocode}
%    Using |\@newctr| will give a proper error message if the counter
%    in |#3| is not defined.
%    In defining `|\theLem|' we must generate the desired macro
%    name by use of |\expandafter| and |\csname|.
%    \begin{macrocode}
    \expandafter\xdef\csname the#1\endcsname
%    \end{macrocode}
%    An |\xdef| is used in order to make the definition global,
%    and to ensure that it contains the replacement texts of
%    |\@thmcountersep| and |\@thmcounter|.\footnote{These
%    two macros can be defined by the document style.  Their default
%    values produce a `\texttt{.}' as separation and an arabic
%    representation of the number.} However, not everything should be
%    expanded. For example, it saves space to use |\thesection|
%    instead of its---at times---lengthy expansion.
%    \begin{macrocode}
      {\expandafter \noexpand \csname the#3\endcsname
       \@thmcountersep \@thmcounter{#1}}%
%    \end{macrocode}
%    Thus with the defaults of \LaTeX{}, |\theLem| would be
%    replaced by the command sequence |\thesection.\arabic{Lem}|.
%
%    We will now look at the definition of the macro which is executed
%    at the beginning of the actual environment (in our example this
%    macro is |\Lem|). It should be noted that we use an
%    ``|\expandafter| trick'' to expand only certain parts of the
%    replacement text at the time of the definition.
%    \begin{macrocode}
    \def\@tempa{\global\@namedef{#1}}%
    \expandafter \@tempa \expandafter{%
%    \end{macrocode}
%    First, the macro that contains the current definitions of
%    |\@begintheorem| and |\@opargtheorem| should be called
%    up.  The name of this macro---as is already known---has the form
%    |\th@|\meta{theorem\ style}; therefore, it must be called by
%    \begin{macrocode}
      \csname th@\the \theorem@style
%    \end{macrocode}
%    In addition the default theorem font should be changeable, i.e.\
%    we have to insert the contents of |\theorem@bodyfont|. For
%    that reason, we expand even further, beyond |\endcsname|,
%    and thus insert the contents of the \textsf{token} register
%    |\theorem@bodyfont| in the replacement text.
%    \begin{macrocode}
            \expandafter \endcsname \the \theorem@bodyfont
%    \end{macrocode}
%    Now it is time to call the macro |\@thm| which takes over
%    the further processing.  It has two arguments: the current
%    counter name (in our example, \texttt{Lem}), and the text of the
%    label (in our example, \texttt{Lemma}).
%    \begin{macrocode}
     \@thm{#1}{#2}}%
%    \end{macrocode}
%    With this, the `sub-definition' is complete.  The macro
%    |\@endtheorem| ends a theorem environment and is, so far, nothing
%    but an |\endtrivlist|.  (Hence it is defined globally, and not
%    within the theorem styles.\footnote{This has to be changed as
%    soon as theorem styles that change \texttt{\bslash @endtheorem}
%    exist.  In such a case, all existing styles must be changed as
%    well since they will have to reset the macro.}) Therefore, we can
%    set it equivalent to the macro that ends the theorem set (in our
%    example, |\endLem|).  However, if some day theorem styles exist
%    that do change |\@endtheorem|, we would have to use the
%    commented-out line instead.
%    \begin{macrocode}
    \global \expandafter \let \csname end#1\endcsname \@endtheorem
%   \global\@namedef{end#1}{\@endtheorem}%
%    \end{macrocode}
%    With these commands all the required definitions are employed,
%    unless the test |\@ifdefinable| has failed.  Therefore, we
%    end the second argument of this macro and with it the definition
%    of |\@xnthm|.
%    \begin{macrocode}
   }}
%    \end{macrocode}
% \end{macro}
%
% \begin{macro}{\@ynthm}
%    The definition of |\@ynthm| is completely analogous. In
%    this case the new counter that is defined is not reset within
%    another counter; thus the definition of |\the...|
%    is simplified:
%    \begin{macrocode}
\gdef\@ynthm#1#2{\expandafter\@ifdefinable\csname #1\endcsname
   {\@definecounter{#1}%
    \expandafter\xdef\csname the#1\endcsname{\@thmcounter{#1}}%
%    \end{macrocode}
%    The rest of the definition corresponds literally to that of
%    |\@xnthm|:
%    \begin{macrocode}
    \def\@tempa{\global\@namedef{#1}}\expandafter \@tempa
     \expandafter{\csname th@\the \theorem@style \expandafter
     \endcsname \the\theorem@bodyfont \@thm{#1}{#2}}%
    \global \expandafter \let \csname end#1\endcsname \@endtheorem}}
%    \end{macrocode}
% \end{macro}
%
% \begin{macro}{\@othm}
%    The definition of |\@othm| does not contain anything new.
%    \begin{macrocode}
\gdef\@othm#1[#2]#3{%
%    \end{macrocode}
% \changes{v2.2b}{95/10/07}{Build in check for counter defined}
%    We do not define a new counter but instead use one that has
%    already been defined.  Thus the only definition we need is that
%    of this pseudo-counter (i.e.\ |\the|\meta{env.\ name}).
%    First we check if |#2| corresponds to a known counter name.
%    \begin{macrocode}
  \expandafter\ifx\csname c@#2\endcsname\relax
   \@nocounterr{#2}%
  \else
   \expandafter\@ifdefinable\csname #1\endcsname
   {\expandafter \xdef \csname the#1\endcsname
     {\expandafter \noexpand \csname the#2\endcsname}%
%    \end{macrocode}
%    All other parts of the definition can be adopted from
%    |\@xnthm|. We have to remember, though, that in this case
%    the name of the current counter and the theorem label have moved
%    to the second and third arguments.
%    \begin{macrocode}
    \def\@tempa{\global\@namedef{#1}}\expandafter \@tempa
     \expandafter{\csname th@\the \theorem@style \expandafter
     \endcsname \the\theorem@bodyfont \@thm{#2}{#3}}%
    \global \expandafter \let \csname end#1\endcsname \@endtheorem}%
  \fi}
%    \end{macrocode}
% \end{macro}
%
% \subsection{Macros that are employed in a {\envfont theorem}
%    environment}
%
% \begin{macro}{\@thm}
%    The macro |\@thm| has to increase the current counter.  Then,
%    depending on whether the environment has (or does not have) an
%    optional argument, it has to branch into either |\@begintheorem|
%    or |\@opargtheorem|.
%    \begin{macrocode}
\gdef\@thm#1#2{\refstepcounter{#1}%
%    \end{macrocode}
%    Now we start a {\envfont trivlist} environment, and give
%    |\@topsep| and |\@topsepadd| the values of the \textsf{skip}
%    registers |\theorempreskipamount| and |\theorempostskipamount|.
%    The value in |\@topsep| is the vertical space that is inserted by
%    the first (and only) |\item| in our |\trivlist| whilst
%    |\@topsepadd| is inserted by |\@endparenv| at the end of that
%    {\envfont trivlist} environment.  By using these registers, we
%    obtain the desired space around a {\envfont theorem} environment.
%    \begin{macrocode}
   \trivlist
   \@topsep \theorempreskipamount               % used by first \item
   \@topsepadd \theorempostskipamount           % used by \@endparenv
%    \end{macrocode}
%    Now we have to test whether an optional argument has been given.
% \changes{v2.0d}{89/07/18}{Forgotten percent added.}
%    \begin{macrocode}
   \@ifnextchar [%
%    \end{macrocode}
%    If there is an optional argument, we will call |\@ythm|, and
%    move the arguments read back into the input stream.
%    \begin{macrocode}
   {\@ythm{#1}{#2}}%
%    \end{macrocode}
%    If not, we call |\@begintheorem|. Its first argument is the
%    name of the theorem set (hence the second argument of
%    |\@thm|). Its second argument is the macro that produces the
%    current number.
%    \begin{macrocode}
   {\@begintheorem{#2}{\csname the#1\endcsname}\ignorespaces}}
%    \end{macrocode}
% \end{macro}
%
% \begin{macro}{\@xthm}
% \begin{macro}{\@ythm}
%    Both these macros were originally called by |\@thm|.  We do
%    not need |\@xthm| anymore, hence we reset it to
%    |\relax|. The definition of |\@ythm| has not changed at
%    all from its definition in \LaTeX. In order to make the macros
%    easier to understand, we will
%    nevertheless present it (commented out).
%    \begin{macrocode}
\global\let\@xthm\relax
% \def\@ythm#1#2[#3]{\@opargbegintheorem{#2}{\csname
%           the#1\endcsname}{#3}\ignorespaces}
%</package>
%    \end{macrocode}
%    The primitive |\ignorespaces| in |\@ythm| and
%    |\@thm| is needed to remove the spaces between the
%    |\begin{...}| and the actual text.
% \end{macro}
% \end{macro}
%
%
% \subsection{Definition of the theorem substyles}
%
% As already pointed out, the theorem substyles, defined below, are
% only loaded when necessary.  Note that all these substyles, except
% \texttt{plain}, have |\slshape| as the default body font.
%
% \subsubsection{The \texttt{plain} style}
%
% \label{dok:th@plain} As the following macros use |@|, we have
% to locally set the |\catcode| of this symbol to ``letter''.
% This happens within a group, so that we do not have to worry about
% which |\catcode| that symbol had before.
%    \begin{macrocode}
%<*thp>
\begingroup \makeatletter
%    \end{macrocode}
% Since we are now within a group, we must make all definitions
% globally.  First we make sure that \texttt{theorem.sty} is loaded.
% This will allow us to use this file as a document style option
% without having to call \texttt{theorem} itself as an option.  At the
% same time, we assure that at least version 2 is loaded, since
% |\theorem@style| was not defined in earlier versions.
%    \begin{macrocode}
\@ifundefined{theorem@style}{\input{theorem.sty}}{}
\ProvidesFile{thp.sty}
             [\FMithmInfo]
%    \end{macrocode}
%
%
% \begin{macro}{\th@plain}
% \begin{macro}{\@begintheorem}
% \begin{macro}{\@opargbegintheorem}
%    |\theoremstyle{plain}| corresponds to the original
%    definition, except that the distances to the surrounding text are
%    determined by the parameters |\theorempreskipamount| and
%    |\theorempostskipamount|.
%    First we set the default body font.
%    \begin{macrocode}
\gdef\th@plain{\normalfont\itshape
%    \end{macrocode}
%    Then we define |\@begintheorem| and
%    |\@opargbegintheorem|. These two macros define how the
%    header of a theorem is typeset.  |\@opargbegintheorem| will
%    be called if a \textsf{theorem} environment with an optional
%    argument is encountered; otherwise, the header is constructed by
%    calling |\@begintheorem|.  If one of these macros is
%    executed, we are within a \textsf{trivlist} environment started by
%    |\@thm|.  So the theorem header is produced with an
%    |\item| command.
%
%    Instead of specifying the header font directly, all standard
%    theorem styles use the |\theorem@headerfont| macro to allow
%    customization.  The extra space (|\labelsep|) is necessary
%    because of problems in the \textsf{trivlist} environment.
%    \begin{macrocode}
  \def\@begintheorem##1##2{%
        \item[\hskip\labelsep \theorem@headerfont ##1\ ##2]}%
%    \end{macrocode}
%    The definition of |\@opargbegintheorem| is completely
%    analogous. The only difference is the fact that there exists a
%    third argument (which is the optional parameter of the
%    environment and contains additional information about the
%    theorem).  Customarily we enclose it in parentheses.
%    \begin{macrocode}
\def\@opargbegintheorem##1##2##3{%
   \item[\hskip\labelsep \theorem@headerfont ##1\ ##2\ (##3)]}}
%    \end{macrocode}
% \end{macro}
% \end{macro}
% \end{macro}
%
% \noindent
% We conclude with an |\endgroup| to restore the
% |\catcode| of |@|.
%    \begin{macrocode}
\endgroup
%</thp>
%    \end{macrocode}
%
%
% \subsubsection{The \texttt{break} style}
%
% This style option is stored in the file |thb.sty|.
% For the next two lines see the documentation for |\th@plain| on
% page~\pageref{dok:th@plain}.
%    \begin{macrocode}
%<*thb>
\begingroup \makeatletter
\@ifundefined{theorem@style}{\input{theorem.sty}}{}
\ProvidesFile{thb.sty}
             [\FMithmInfo]
%    \end{macrocode}
%
% \begin{macro}{\th@break}
%    |\theoremstyle{break}| produces a line break after the name
%    of the theorem.  The font is |\slshape|.  Hence, we define
%    |\th@break| as follows:
%    \begin{macrocode}
\gdef\th@break{\normalfont\slshape
  \def\@begintheorem##1##2{\item[%
%    \end{macrocode}
%    We run into the following problem: it is not possible to create
%    the header with |\item[|\meta{title}|]| and then start
%    a new line by, for example, |\mbox{}\\|. Such a definition
%    will fail whenever a list environment follows immediately.  With
%    the above construction, the |\mbox{}| causes the switch
%    |@inlabel| (cf.\ definition of |\list| and
%    |\trivlist| in \cite{latex:source}) to be set to \texttt{false}
%    and so the following list will insert additional vertical
%    space (|\topskip|). This is quite annoying. Therefore, we
%    create the line break within the |\item|. In order to ensure
%    that the text will begin at the proper position in the following
%    line, we simply pretend that the label does not take any
%    room.\footnote{This will lead to problems whenever very high
%    symbols occurring in the line tower into the
%    heading. So, something else has to be done here sometime.}
%    \begin{macrocode}
       \rlap{\vbox{\hbox{\hskip \labelsep\theorem@headerfont ##1\ ##2}%
                   \hbox{\strut}}}]}%
%    \end{macrocode}
%    Again, the definition of |\@opargbegintheorem| is
%    completely analogous.
%    \begin{macrocode}
\def\@opargbegintheorem##1##2##3{%
  \item[\rlap{\vbox{\hbox{\hskip \labelsep \theorem@headerfont
                     ##1\ ##2\ (##3)}%
                    \hbox{\strut}}}]}}
\endgroup
%</thb>
%    \end{macrocode}
% \end{macro}
%
%
% \subsubsection{The \texttt{changebreak} style}
%
%    \begin{macrocode}
% This style option is stored in the file |thcb.sty|.
%    \begin{macrocode}
%<*thcb>
\begingroup \makeatletter
\@ifundefined{theorem@style}{\input{theorem.sty}}{}
\ProvidesFile{thcb.sty}
             [\FMithmInfo]
%    \end{macrocode}
%
% \begin{macro}{\th@changebreak}
%    The \texttt{change-break} theorem style is like \texttt{break}
%    but with interchange of theorem name and theorem number.  Thus we
%    define |\th@changebreak| as follows:
%    \begin{macrocode}
\gdef\th@changebreak{\normalfont\slshape
  \def\@begintheorem##1##2{\item
       [\rlap{\vbox{\hbox{\hskip\labelsep \theorem@headerfont ##2\ ##1}%
                    \hbox{\strut}}}]}%
\def\@opargbegintheorem##1##2##3{%
  \item[\rlap{\vbox{\hbox{\hskip\labelsep \theorem@headerfont
                     ##2\ ##1\ (##3)}%
                    \hbox{\strut}}}]}}
\endgroup
%</thcb>
%    \end{macrocode}
% \end{macro}
%
%
% \subsubsection{The \texttt{change} style}
%
% This style option is stored in the file |thc.sty|.
%    \begin{macrocode}
%<*thc>
\begingroup \makeatletter
\@ifundefined{theorem@style}{\input{theorem.sty}}{}
\ProvidesFile{thc.sty}
             [\FMithmInfo]
%    \end{macrocode}
%
% \begin{macro}{\th@change}
%    The \texttt{change} theorem style corresponds to the
%    \texttt{change break} style without a linebreak after the header.
%    To say it in another way, it's the same as the \texttt{plain}
%    style but with number and name interchanged and |\slshape| as the
%    default font.
%    \begin{macrocode}
\gdef\th@change{\normalfont\slshape
  \def\@begintheorem##1##2{\item
       [\hskip\labelsep \theorem@headerfont ##2\ ##1]}%
\def\@opargbegintheorem##1##2##3{%
  \item[\hskip\labelsep \theorem@headerfont ##2\ ##1\ (##3)]}}
\endgroup
%</thc>
%    \end{macrocode}
% \end{macro}
%
%
% \subsubsection{The \texttt{marginbreak} style}
%
% This style option is the one used most often at Mainz.  It is saved
% in the file |thmb.sty|.
%    \begin{macrocode}
%<*thmb>
\begingroup \makeatletter
\@ifundefined{theorem@style}{\input{theorem.sty}}{}
\ProvidesFile{thmb.sty}
             [\FMithmInfo]
%    \end{macrocode}
%
% \begin{macro}{\th@marginbreak}
%    The \texttt{margin break} style is nearly the same as the
%    \texttt{change break} style. The only difference is the placement
%    of the theorem number.  We use |\llap| to place it in the left
%    margin.
%
%    In this style |\labelsep| denotes the separation between the
%    number and the text.
%    \begin{macrocode}
\gdef\th@marginbreak{\normalfont\slshape
  \def\@begintheorem##1##2{\item
       [\rlap{\vbox{\theorem@headerfont
                    \hbox{\llap{##2}\hskip\labelsep ##1}%
                    \hbox{\strut}}}]}%
\def\@opargbegintheorem##1##2##3{%
  \item[\rlap{\vbox{\theorem@headerfont
                    \hbox{\llap{##2}\hskip\labelsep ##1\ (##3)}%
                    \hbox{\strut}}}]}}
\endgroup
%</thmb>
%    \end{macrocode}
% \changes{v2.1c}{90/02/17}{`endgroup outside macrocode env.}
% \end{macro}
%
%
%
% \subsubsection{The \texttt{margin} style}
%
% This style option is stored in the file |thm.sty|.
%    \begin{macrocode}
%<*thm>
\begingroup \makeatletter
\@ifundefined{theorem@style}{\input{theorem.sty}}{}
\ProvidesFile{thm.sty}
             [\FMithmInfo]
%    \end{macrocode}
%
% \begin{macro}{\th@margin}
%    Again this is only a variant of the theorem styles described
%    above without any new ideas.
%    \begin{macrocode}
\gdef\th@margin{\normalfont\slshape
  \def\@begintheorem##1##2{\item
       [\theorem@headerfont \llap{##2}\hskip\labelsep ##1]}%
\def\@opargbegintheorem##1##2##3{%
  \item[\theorem@headerfont \llap{##2}\hskip\labelsep ##1\ (##3)]}}
\endgroup
%</thm>
%    \end{macrocode}
% \end{macro}
%
%
%
%
% \subsection{Final Definitions}
%
% \begin{macro}{\theorempreskipamount}
% \begin{macro}{\theorempostskipamount}
%    The \textsf{skip} parameters that regulate the vertical empty space
%    before and after the {\envfont theorem} environment have
%    to be allocated as well.
% \changes{v2.0f}{89/07/20}{Unnecessary `global' removed.}
%    \begin{macrocode}
%<*package>
\newskip\theorempreskipamount
\newskip\theorempostskipamount
%    \end{macrocode}
%    Since we have used the same values for all theorem sets, we now can
%    assign them.
%    \begin{macrocode}
\global\setlength\theorempreskipamount{12pt plus 5pt minus 3pt}
\global\setlength\theorempostskipamount{8pt plus 3pt minus 1.5pt}
%    \end{macrocode}
% \end{macro}
% \end{macro}
%
%
% \begin{macro}{\@endtheorem}
%    The same holds for the macro |\@endtheorem|, which ends a
%    {\envfont theorem} environment. Since it is the same for all
%    theorem sets, it is removed from the macros |\th@|\meta{style}.
%    It simply ends the {\envfont trivlist} environment, which was
%    begun in |\@thm|.
%    \begin{macrocode}
\global\let\@endtheorem=\endtrivlist
%    \end{macrocode}
% \end{macro}
%
%
%
% \begin{macro}{\@preamblecmds}
%    All macros defined above are to be used only in the preamble.
%    Therefore, we insert them in |\@preamblecmds| which will disable
%    them at begin document. This is done by the internal \LaTeXe{}
%    command |\@onlypreamble|.
% \changes{v2.0g}{89/09/19}{`theoremheaderfont and `theorembodyfont
%    added.}
% \changes{v2.2a}{94/02/03}{Use \LaTeXe{} interface}
%    \begin{macrocode}
\@onlypreamble\@xnthm
\@onlypreamble\@ynthm
\@onlypreamble\@othm
\@onlypreamble\newtheorem
\@onlypreamble\theoremstyle
\@onlypreamble\theorembodyfont
\@onlypreamble\theoremheaderfont
%    \end{macrocode}
% \end{macro}
%
% Finally we declare the \texttt{plain} theorem style to be the default.
% \changes{v2.1b}{90/06/04}{Fixed missing `item by declaring default
%    style.}
%    \begin{macrocode}
\theoremstyle{plain}
%</package>
%    \end{macrocode}
%
% \Finale

\endinput
}{}
\ProvidesFile{thb.sty}
             [\FMithmInfo]
%    \end{macrocode}
%
% \begin{macro}{\th@break}
%    |\theoremstyle{break}| produces a line break after the name
%    of the theorem.  The font is |\slshape|.  Hence, we define
%    |\th@break| as follows:
%    \begin{macrocode}
\gdef\th@break{\normalfont\slshape
  \def\@begintheorem##1##2{\item[%
%    \end{macrocode}
%    We run into the following problem: it is not possible to create
%    the header with |\item[|\meta{title}|]| and then start
%    a new line by, for example, |\mbox{}\\|. Such a definition
%    will fail whenever a list environment follows immediately.  With
%    the above construction, the |\mbox{}| causes the switch
%    |@inlabel| (cf.\ definition of |\list| and
%    |\trivlist| in \cite{latex:source}) to be set to \texttt{false}
%    and so the following list will insert additional vertical
%    space (|\topskip|). This is quite annoying. Therefore, we
%    create the line break within the |\item|. In order to ensure
%    that the text will begin at the proper position in the following
%    line, we simply pretend that the label does not take any
%    room.\footnote{This will lead to problems whenever very high
%    symbols occurring in the line tower into the
%    heading. So, something else has to be done here sometime.}
%    \begin{macrocode}
       \rlap{\vbox{\hbox{\hskip \labelsep\theorem@headerfont ##1\ ##2}%
                   \hbox{\strut}}}]}%
%    \end{macrocode}
%    Again, the definition of |\@opargbegintheorem| is
%    completely analogous.
%    \begin{macrocode}
\def\@opargbegintheorem##1##2##3{%
  \item[\rlap{\vbox{\hbox{\hskip \labelsep \theorem@headerfont
                     ##1\ ##2\ (##3)}%
                    \hbox{\strut}}}]}}
\endgroup
%</thb>
%    \end{macrocode}
% \end{macro}
%
%
% \subsubsection{The \texttt{changebreak} style}
%
%    \begin{macrocode}
% This style option is stored in the file |thcb.sty|.
%    \begin{macrocode}
%<*thcb>
\begingroup \makeatletter
\@ifundefined{theorem@style}{% \iffalse meta-comment
%
% Copyright (C) 1993-2024
%
% The LaTeX Project and any individual authors listed elsewhere
% in this file.
%
% This file is part of the Standard LaTeX `Tools Bundle'.
% -------------------------------------------------------
%
% It may be distributed and/or modified under the
% conditions of the LaTeX Project Public License, either version 1.3c
% of this license or (at your option) any later version.
% The latest version of this license is in
%    https://www.latex-project.org/lppl.txt
% and version 1.3c or later is part of all distributions of LaTeX
% version 2005/12/01 or later.
%
% The list of all files belonging to the LaTeX `Tools Bundle' is
% given in the file `manifest.txt'.
%
% \fi
%
%%
%
%
\def\FMithmInfo{2023/07/05 v2.2c Theorem extension package (FMi)}
%
% \ProvidesFile{theorem.dtx}[\FMithmInfo]
%\iffalse   % this is a METACOMMENT !
%%
%% Package `theorem' to use with LaTeX2e
%% Copyright (C) 1989-1995 Frank Mittelbach, all rights reserved.
%<+package>\@ifundefined{theorem@style}{}{\endinput}
%<+package>\NeedsTeXFormat{LaTeX2e}[1995/06/01]
%<+package>\ProvidesPackage{theorem}
%<+package>         [\FMithmInfo]
%
% \fi
%
% \changes{v1.0}{87/05/07}{First draft.}
% \changes{v1.0a}{87/05/07}{`thistheoremstyle' added.}
% \changes{v1.1a}{87/07/29}{New concept: redefine the internal
%                          \protect\LaTeX{} commands `@xnthm' etc.}
% \changes{v1.1b}{87/07/31}{`theoremfont' used to specify font.}
% \changes{v1.2a}{88/01/08}{`@endtheorem' now globally defined.}
% \changes{v1.2b}{88/01/11}{`newtheorem' only allowed in the preamble.}
% \changes{v1.3a}{88/02/14}{`theoremfont' now a toks register.}
%
% \changes{v2.0a}{88/11/16}{New implementation and new documentation.}
% \changes{v2.0a}{88/11/16}{No layout changes but internal modifications
%                           old substyles won't work.}
% \changes{v2.0b}{89/01/30}{`global' used for all definitions.}
% \changes{v2.0c}{89/06/15}{Documentation translated to English.
%                            Thanks to LLL and Nina}
% \changes{v2.1a}{90/02/15}{Support for new font selection.}
% \changes{v2.2a}{94/02/03}{Cleanup for LaTeX2e}
%
% \DoNotIndex{\@endparenv,\@ifdefinable,\@ifnextchar,\@ifundefined}
% \DoNotIndex{\@namedef,\@nameuse,\@spaces,\@tempa,\@warning}
% \DoNotIndex{\ }
% \DoNotIndex{\begingroup,\bfseries}
% \DoNotIndex{\csname}
% \DoNotIndex{\def,\do,\docdate}
% \DoNotIndex{\endcsname,\endgroup,\endinput,\endtrivlist,\expandafter}
% \DoNotIndex{\filedate,\fileversion}
% \DoNotIndex{\gdef,\global}
% \DoNotIndex{\hbox,\hskip}
% \DoNotIndex{\ignorespaces,\input,\itshape,\item}
% \DoNotIndex{\labelsep,\let,\llap}
% \DoNotIndex{\makeatletter}
% \DoNotIndex{\noexpand}
% \DoNotIndex{\refstepcounter,\relax,\rlap}
% \DoNotIndex{\setlength,\slshape,\space,\string,\strut}
% \DoNotIndex{\the,\trivlist,\typeout}
% \DoNotIndex{\vbox}
% \DoNotIndex{\xdef}
%
% \def\deflabel#1{\ttfamily #1\hfill}
% \def\deflist#1{\begin{list}{}{\settowidth\labelwidth{\ttfamily #1}%
%                               \setlength\leftmargin\labelwidth
%                               \addtolength\leftmargin\labelsep
%                               \let\makelabel\deflabel}}
% \def\enddeflist{\end{list}}
% \def\envfont{\normalfont\ttfamily}
%
% \MakeShortVerb{\|}
%
% \newcommand{\TUB}{{\sl TUGboat\/}}
%
% \GetFileInfo{theorem.dtx}
% \title{An Extension of the
%        \LaTeX{} theorem environment\thanks{This file
%        has version number \fileversion, last
%        revised \filedate.}}
% \author{Frank Mittelbach}
%
% \MaintainedByLaTeXTeam{tools}
% \maketitle
%
% \begin{abstract}
%    The macros described in this paper yield an extension of the
%    \LaTeX{} theorem mechanism. It is designed to satisfy the
%    different requirements of various journals. Thus, the
%    layout of the ``theorems'' can be manipulated by determining a
%    ``style''. This article describes not only the use, but
%    also the definition, of the necessary macros.
% \end{abstract}
%
% \section*{Preface to version 2.2}
%
% For \LaTeXe{} this package did not need any fundamental changes. I
% only modified the messages generated so that theorem layout styles
% will show up with the |\listfiles| command and cleaned the section
% on the New Font Selection Scheme since this is now included in
% \LaTeX.
%
% \section*{Preface to version 2.1}
%
% This version is identical to 2.0g described in \TUB\ 10\#3 except
% for some internal defaults which are now set depending on the used
% font selection scheme.
%
% This was done to avoid unpleasant surprises if the new font selection
% scheme is in force. For further details see
% section~\ref{sec:newfonts} and \cite{bk:GMS94}.
%
% \section{Introduction}
%
% For our purposes here, ``theorems'' are labelled enunciations,
% often set off from the main text by extra space and a font change.
% Theorems, corollaries, conjectures, definitions, and remarks
% are all instances of ``theorems''.  The ``header'' of these structures
% is composed of a label (such as \textsc{Theorem} or \textsc{Remark})
% and a number which serializes an item in the sequence of items
% with the same label.
%
% Shortly after the introduction of \LaTeX{} at the Fachbereich
% Mathematik in Mainz, the desire to manipulate the layout of
% ``theorems'' arose.  In Mainz, the following two conventions came
% into general use:
% \begin{enumerate}
%   \item The number of the theorem is shown in the margin.
%   \item There is a line break at
%         the end of the theorem header.
% \end{enumerate}
%
% Additionally, some journals require different formats which depend on
% the ``sort of theorem'': e.g.\ often remarks and definitions are set
% in |\upshape|, while |\itshape| is employed for main theorems.
%
% Confronted with these requirements, a theorem environment was
% developed in Mainz which allows separate determination of the
% layout of the ``theorems sets'', comparable to |\pagestyle|.
%
%
% \section{The user interface}
% \subsection{Defining new theorem sets}
%
% \DescribeMacro\newtheorem
% As in the original \LaTeX{} version, the command |\newtheorem|
% defines a new ``theorem set'' or ``theorem-like structure''.
% Two required arguments name the new environment and give the text
% to be typeset with each instance of the new ``set'', while
% an optional argument determines how
% the ``set'' is enumerated:
% \begin{description}
%    \item[\ttfamily \bslash newtheorem\{foo\}\{bar\}]
%       The theorem set {\envfont foo} (whose name is \texttt{bar})
%       uses its own counter.
%    \item[\ttfamily \bslash newtheorem\{foo2\}{[foo]}\{bar2\}]
%       The theorem set {\envfont foo2} (printed name \texttt{bar2})
%       uses the same counter as the theorem set \texttt{foo}.
%    \item[\ttfamily \bslash newtheorem\{foo3\}\{bar3\}{[section]}]
%       The theorem set {\envfont foo3} (printed name \texttt{bar3}) is
%       enumerated within the counter \texttt{section}, i.e.\ with every
%       new |\section| the enumeration begins again with 1, and
%       the enumeration is composed from the section-number and the
%       theorem counter itself.
% \end{description}
%
% \DescribeMacro\theoremstyle
% Additionally, the command |\theoremstyle| can define the layout
% of various, or all, theorem sets.  It should be noted that any
% theorem set defined by |\newtheorem| is typeset in the
% |\theoremstyle| that is current at the time of the definition.
% Thus, the following
% \begin{verbatim}
%    \theoremstyle{break}        \newtheorem{Cor}{Corollary}
%    \theoremstyle{plain}        \newtheorem{Exa}{Example}[section]
%\end{verbatim}                             ^^A No blank at line start
% leads to the result that the set {\envfont Cor} is formatted in the
% style \texttt{break}, while the set {\envfont Exa} and all the
% following ones are formatted in the style \texttt{plain}, unless
% another |\theoremstyle| follows.  Since the definitions installed by
% |\newtheorem| are global, one also can limit |\theoremstyle| locally
% by grouping braces.
%
% \DescribeMacro\theorembodyfont
% The choice of the font for the theorem body is completely
% independent of the chosen |\theoremstyle|; this has proven to
% be very advantageous.  For example,
% \begin{verbatim}
%   {\theorembodyfont{\upshape}          \newtheorem{Rem}{Remark}}
%\end{verbatim}
% defines a theorem set {\envfont Rem}, which will be set in
% |\upshape| in the current layout (which in our example is
% \texttt{plain}).  As with |\theoremstyle|, the |\theorembodyfont|
% chosen is that current at the time of |\newtheorem|.  If
% |\theorembodyfont| is not specified or one defines
% |\theorembodyfont{}|, then the font used will be that defined by the
% |\theoremstyle|.
%
% \DescribeMacro\theoremheaderfont It is also possible to customize
% the font used for the theorem headers.  This is, however, a global
% declaration, and therefore there should be at most one
% |\theoremheaderfont| declaration in the preamble.\footnote{If it is
% actually necessary to have different header fonts, one has to define
% new theorem styles (substituting the desired font) or specify the
% information directly in the \texttt{\bslash newtheorem} declaration
% (the unclean variant).}
%
% \DescribeMacro\theorempreskipamount
% \DescribeMacro\theorempostskipamount Two additional parameters
% affect the vertical space around the theorem environments:
% |\theorempreskipamount| and |\theorempostskipamount| define,
% respectively, the spacing before and after such an environment.
% These parameters apply for all theorem sets and can be manipulated
% with the ordinary length macros.  They are rubber lengths
% (`\textsf{skips}'), and therefore can contain \texttt{plus} and
% \texttt{minus} parts.
%
% Since the definition of theorem sets should---most sensibly---be
% placed in the preamble, we only allow installation there.
% It is therefore possible to release the memory used here after
% |\begin{document}|, in order to make room for other
% applications.
%
%
% \subsection{Existing theorem styles}
%
% The following theorem styles exist to date:
% \begin{deflist}{marginbreak}
%    \item[plain]
%       This theorem style emulates the original \LaTeX{} definition,
%       except that additionally the parameters
%       |\theorem...skipamount| are used.
%    \item[break]
%       In this style, the theorem header is followed by a line
%       break.
%    \item[marginbreak]
%       The theorem number is set in the margin, and there is a line
%       break as in \texttt{break}.
%    \item[changebreak]
%       Like \texttt{break}, but with header number and text
%       interchanged.
%    \item[change]
%       Header number and text are interchanged, without a line break.
%    \item[margin]
%       The number is set in the left margin, without a line break.
% \end{deflist}
% All styles (except \texttt{plain}) select |\slshape| as the default
% |\theorembodyfont|.
%
%
% \subsection{Examples}
%
% Given the above theorem sets {\envfont Cor}, {\envfont Exa}
% and {\envfont Rem},
% suppose that the preamble also contains the declarations:
% \begin{verbatim}
%    \theoremstyle{marginbreak}   \newtheorem{Lem}[Cor]{Lemma}
%    \theoremstyle{change}
%    \theorembodyfont{\itshape}        \newtheorem{Def}[Cor]{Definition}
%
%    \theoremheaderfont{\scshape}
%\end{verbatim}
% Then the following are some typical examples of the typeset output
% resulting from their use.
%
% \begin{Cor}
%    This is a sentence typeset in the theorem environment {\envfont
%    Cor}.
% \end{Cor}
%
% \begin{Exa}
%    This is a sentence typeset in the theorem environment {\envfont
%    Exa}.
% \end{Exa}
%
% \begin{Rem}
%    This is a sentence typeset in the theorem environment {\envfont
%    Rem}.
% \end{Rem}
%
% \begin{Lem}[Ben User]
%    This is a sentence typeset in the theorem environment {\envfont
%    Lem}.
% \end{Lem}
%
% \begin{Def}[Very impressive Definition]
%    This is a sentence typeset in the theorem environment {\envfont
%    Def}.
% \end{Def}
%
% The last two examples show the effect of the optional argument to
% a theorem environment (it is the text typeset in parentheses).
%
%
% \section{Special Considerations}
% \label{sec:newfonts}
%
% Theoremheader and body are implemented as a unit. This means that
% the |\theoremheaderfont|  will inherit characteristics of the
% |\theorembodyfont| in \LaTeXe. Thus, if for example
% |\theorembodyfont| is |\itshape| and |\theoremheaderfont|
% is |\bfseries| the font selected for the header will have the
% characteristics `bold extended italic'. If this is not desired one
% should set the |\theoremheaderfont| to something like
% \begin{verbatim}
%      \theoremheaderfont{\normalfont\bfseries}
%\end{verbatim}
% i.e.\ supplying all necessary font information explicitly.
%
%
%
% \section{Acknowledgements}
%
% The publication of this set of macros was only possible with the
% help of Christina Busse (translating the manuscript into English),
% Joachim Pense (playing the r\^ole of typist), Chris Rowley
% (looking everything over) and many others providing useful
% suggestions.
%
% \MaybeStop{%
%    \begin{thebibliography}{9}
%       \bibitem{bk:GMS94} \textsc{M.~Goossens}, \textsc{F.~Mittelbach}
%         and \textsc{A.~Samarin}.
%         \newblock The \LaTeX{} Companion.
%         \newblock
%         Addison-Wesley, Reading, Massachusetts, 1994.
%       \bibitem{latex:source} \textsc{Lamport, Leslie.} \newblock
%                               \texttt{latex.tex}, version 2.09, date
%                               Feb. 1990.
%    \end{thebibliography}
%    \PrintIndex
%    \PrintChanges}
%
%
% \section{The documentation driver file}
%
% The next bit of code contains the documentation driver file for
% \TeX{}, i.e., the file that will produce the documentation you are
% currently reading. It will be extracted from this file by the
% \texttt{docstrip} program. Since it is the first code in the file
% one can alternatively process this file directly with \LaTeXe{} to
% obtain the documentation.
%
%    \begin{macrocode}
%<*driver>
\documentclass{ltxdoc}

\usepackage{theorem}

%<+driver>% The next few lines define theorem sets which are used
%<+driver>% in the example section of the documentation.

\theoremstyle{break}       \newtheorem{Cor}{Corollary}
\theoremstyle{plain}       \newtheorem{Exa}{Example}[section]
{\theorembodyfont{\upshape}\newtheorem{Rem}{Remark}}
\theoremstyle{marginbreak} \newtheorem{Lem}[Cor]{Lemma}
\theoremstyle{change}
\theorembodyfont{\itshape} \newtheorem{Def}[Cor]{Definition}

\theoremheaderfont{\scshape}

\RecordChanges

\begin{document}
   \DocInput{theorem.dtx}
\end{document}
%</driver>
%    \end{macrocode}
%
% \section{Definition of the Macros}
%
% If the file has been loaded before, we abort immediately. If not the
% package announces itself (this is actually done at the very top of
% the file---the way it is done isn't good style so don't copy it).
% \changes{v2.0e}{89/07/19}{Spaces removed from `typeout'.}
% \changes{v2.2b}{95/11/19}{Announce moved to top of file}
%    \begin{macrocode}
%<*package>
%\@ifundefined{theorem@style}{}{\endinput}
%\def\FMithmInfo{1995/11/19 v2.2b Theorem extension package (FMi)}
%\ProvidesPackage{theorem}[\FMithmInfo]
%    \end{macrocode}
%
%
% \subsection{Definition of theorem styles and fonts}
%
% All the definitions in this file are done globally to allow
% inputting this file inside a group.
%
% \begin{macro}{\theoremstyle}
%    Before a theorem style can be installed, the chosen style must be
%    known. For that reason, we must test to see that
%    |\th@|\meta{style} is known or, more precisely, that it is
%    different from |\relax|. If the style is not known then
%    |\th@plain| is used.
%    \begin{macrocode}
\gdef\theoremstyle#1{%
   \@ifundefined{th@#1}{\@warning
          {Unknown theoremstyle `#1'. Using `plain'}%
          \theorem@style{plain}}%
%    \end{macrocode}
%    We save the theorem style to be used in the token register
%    |\theorem@style|.
%    \begin{macrocode}
      {\theorem@style{#1}}%
%    \end{macrocode}
%    Now we ``evaluate'' the theorem style: this means, we call the
%    macro |\th@|\meta{style} which will activate the relevant
%    definitions which are contained in a separate file.  This is done
%    in a group to suppress changes to the current font.  This could
%    otherwise pose problems together with the new font selection
%    scheme\footnote{When I printed the original article using the new
%    font selection scheme I ended with a document with slanted
%    typefaces (text headings and all) simply because one of the
%    theorem styles used \texttt{\bslash sl} at toplevel.} if the
%    |\th@|\meta{style} is evaluated a second time.
% \changes{v2.1a}{90/02/15}{Group added around `th@..}
% \changes{v2.1a}{90/02/15}{Reseting of `..@begintheorem removed.}
%    \begin{macrocode}
      \begingroup
        \csname th@\the\theorem@style \endcsname
      \endgroup}
%    \end{macrocode}
% \end{macro}
%
%
% \begin{macro}{\@begintheorem}
% \changes{v2.1a}{90/02/15}{Relaxed at toplevel.}
% \begin{macro}{\@opargbegintheorem}
% \changes{v2.1a}{90/02/15}{Relaxed at toplevel.}
%    We reset |\@begintheorem| and |\@opargbegintheorem| to
%    |\relax| since these commands are no longer necessary at
%    toplevel. This will save a few tokens.
%    \begin{macrocode}
\global\let\@begintheorem\relax
\global\let\@opargbegintheorem\relax
%    \end{macrocode}
% \end{macro}
% \end{macro}
%
% \begin{macro}{\theorem@style}
%    Obviously the \textsf{token} register used above has to be
%    allocated. To assure the utmost compatibility with the
%    original \LaTeX{} definition, we set the default theorem style to
%    \texttt{plain}, which implements the usual \LaTeX{} convention.
% \changes{v2.0f}{89/07/20}{Unnecessary `global' removed.}
%    \begin{macrocode}
\newtoks\theorem@style
\global\theorem@style{plain}
%    \end{macrocode}
% \end{macro}
%
% \begin{macro}{\theorembodyfont}
% \begin{macro}{\theorem@bodyfont}
%    If the |\theorembodyfont| is set by the user then it should not
%    interact with the default font set in the theorem style. When the
%    new font selection is in force this may happen if, for example,
%    the default is |\itshape| and the new |\theorembodyfont| is
%    |\sffamily|. So we add a |\reset@font| command in front of the
%    user definition.
% \changes{v2.0f}{89/07/20}{Unnecessary `global' removed.}
% \changes{v2.0g}{89/09/19}{`theoremfont to `theorembodyfont.}
% \changes{v2.1c}{92/06/22}{`reset@font added and logic changed.}
% \changes{v2.1d}{92/10/28}{Reset to empty if arg is empty}
%    \begin{macrocode}
\gdef\theorembodyfont#1{%
%    \end{macrocode}
%    We check if the argument supplied is empty and if so put nothing
%    into the |\theorem@bodyfont| token register to allow for
%    |\theorembodyfont{}| as a mean of using the default of the
%    current |\theoremstyle|.
%    \begin{macrocode}
   \def\@tempa{#1}%
   \ifx\@tempa\@empty
    \theorem@bodyfont{}%
   \else
    \theorem@bodyfont{\reset@font#1}%
   \fi
   }
\newtoks\theorem@bodyfont
\global\theorem@bodyfont{}
%    \end{macrocode}
% \end{macro}
% \end{macro}
%
%
% \begin{macro}{\theoremheaderfont}
% \changes{v2.0g}{89/09/19}{Macro added.}
%    The font for the theorem headers is handled differently because
%    this definition applies to all theorem styles.
%    \begin{macrocode}
\gdef\theoremheaderfont#1{\gdef\theorem@headerfont{#1}%
%    \end{macrocode}
%    After using the macro once it is redefined to produce an error
%    message.
%    \begin{macrocode}
       \gdef\theoremheaderfont##1{%
        \typeout{\string\theoremheaderfont\space should be used
                 only once.}}}
%    \end{macrocode}
% \end{macro}
%
% \begin{macro}{\theorem@headerfont}
% \changes{v2.0g}{89/09/19}{Macro added.}
%    To set the |\theorem@headerfont| default we first test if
%    the new fontselection scheme is in force.
% \changes{v2.1a}{90/02/15}{Support for new font selection.}
%    \begin{macrocode}
\ifx\upshape\undefined
%    \end{macrocode}
%    If not we define it to expand into |\bfseries|. We don't use
%    |\let| just in case a following style option redefines this
%    macro.
%    \begin{macrocode}
\gdef\theorem@headerfont{\bfseries}
%    \end{macrocode}
%    Otherwise we reset the current shape before calling |\bfseries|.
%    \begin{macrocode}
\else \gdef\theorem@headerfont{\normalfont\bfseries}\fi
%    \end{macrocode}
% \end{macro}
%
% \begin{macro}{\th@plain}
% \begin{macro}{\th@break}
% \begin{macro}{\th@marginbreak}
% \begin{macro}{\th@changebreak}
% \begin{macro}{\th@change}
% \begin{macro}{\th@margin}
%    The different styles are defined in macros such as
%    |\th@plain|.  Since memory space is precious in
%    ``non-Big-versions'', we have to avoid offering too many unused
%    definitions.  Therefore we define these styles in separate files
%    that can be loaded on demand.  Thus the commands themselves only
%    load these files. We use |\@input@|, a \LaTeXe{} internal command
%    that ensures that the file will be listed with |\listfiles|.
% \changes{v2.2a}{94/02/02}{Use `@input@ to load theorem layout files}
%    \begin{macrocode}
\gdef\th@plain{\@input@{thp.sty}}
\gdef\th@break{\@input@{thb.sty}}
\gdef\th@marginbreak{\@input@{thmb.sty}}
\gdef\th@changebreak{\@input@{thcb.sty}}
\gdef\th@change{\@input@{thc.sty}}
\gdef\th@margin{\@input@{thm.sty}}
%    \end{macrocode}
%    This list will be expanded when new styles become
%    available.  For testing,  just append new theorem substyles
%    as document options.
% \end{macro}
% \end{macro}
% \end{macro}
% \end{macro}
% \end{macro}
% \end{macro}
%
%
%
% \subsection{Definition of a new theorem set}
%
% As already pointed out, a new theorem environment can be defined in
% three different ways:
% \begin{verbatim}
%    \newtheorem{Lem}{Lemma}
%    \newtheorem{Lem}{Lemma}[section]
%    \newtheorem{Lem}[Theorem]{Lemma}
%\end{verbatim}
% The function of the macro |\newtheorem| is to recognize these cases
% and then to branch into one of the three macros |\@ynthm|, |\@xnthm|
% or |\@othm|.  This mechanism is adopted unchanged from
% \cite{latex:source}; the essential point here is that, for example,
% in the second case, the arguments \texttt{Lem}, \texttt{Lemma} and
% \texttt{section} are passed over to the macro |\@xnthm|.
%
% We inspect this case first because the others present fewer problems,
% and thus are easily derived from this one.
%
% \begin{macro}{\@xnthm}
%    For our example arguments, the macro |\@xnthm| must fulfill
%    the following:
%    \begin{itemize}
%       \item
%          Define a new \LaTeX-counter `Lem'
%       \item
%          reset this counter within a |\section|
%       \item
%          define the macro |\theLem|
%       \item
%          define the environment macros |\Lem| and
%          |\endLem| using the current
%          |\theoremstyle| and |\theorem@bodyfont|.
%    \end{itemize}
%    Obviously, all this should happen only if the first argument
%    of |\@xnthm| (i.e.\ \texttt{Lem} in our example) is chosen so
%    as not to conflict with any previously defined commands or
%    environments.  This test is performed by the \LaTeX{} macro
%    |\@ifdefinable|.
%    \begin{macrocode}
\gdef\@xnthm#1#2[#3]{\expandafter\@ifdefinable\csname #1\endcsname
%    \end{macrocode}
%    Therefore, the first argument of |\@ifdefinable| is the
%    expansion (in the example, |\Lem|) of
%    |\csname||#1\endcsname|.  The second argument is
%    executed only if the test has been completed successfully.
%    \begin{macrocode}
   {%
%    \end{macrocode}
%    Now we define the new counter. The names of the \LaTeX{}
%    macros employed should speak for themselves:
% \changes{v2.2b}{1995/10/07}{Use \cs{@newctr}}
% \changes{v2.2c}{1995/11/23}{Use \cs{@newctr} properly.}
%    \begin{macrocode}
    \@definecounter{#1}\@newctr{#1}[#3]%
%    \end{macrocode}
%    Using |\@newctr| will give a proper error message if the counter
%    in |#3| is not defined.
%    In defining `|\theLem|' we must generate the desired macro
%    name by use of |\expandafter| and |\csname|.
%    \begin{macrocode}
    \expandafter\xdef\csname the#1\endcsname
%    \end{macrocode}
%    An |\xdef| is used in order to make the definition global,
%    and to ensure that it contains the replacement texts of
%    |\@thmcountersep| and |\@thmcounter|.\footnote{These
%    two macros can be defined by the document style.  Their default
%    values produce a `\texttt{.}' as separation and an arabic
%    representation of the number.} However, not everything should be
%    expanded. For example, it saves space to use |\thesection|
%    instead of its---at times---lengthy expansion.
%    \begin{macrocode}
      {\expandafter \noexpand \csname the#3\endcsname
       \@thmcountersep \@thmcounter{#1}}%
%    \end{macrocode}
%    Thus with the defaults of \LaTeX{}, |\theLem| would be
%    replaced by the command sequence |\thesection.\arabic{Lem}|.
%
%    We will now look at the definition of the macro which is executed
%    at the beginning of the actual environment (in our example this
%    macro is |\Lem|). It should be noted that we use an
%    ``|\expandafter| trick'' to expand only certain parts of the
%    replacement text at the time of the definition.
%    \begin{macrocode}
    \def\@tempa{\global\@namedef{#1}}%
    \expandafter \@tempa \expandafter{%
%    \end{macrocode}
%    First, the macro that contains the current definitions of
%    |\@begintheorem| and |\@opargtheorem| should be called
%    up.  The name of this macro---as is already known---has the form
%    |\th@|\meta{theorem\ style}; therefore, it must be called by
%    \begin{macrocode}
      \csname th@\the \theorem@style
%    \end{macrocode}
%    In addition the default theorem font should be changeable, i.e.\
%    we have to insert the contents of |\theorem@bodyfont|. For
%    that reason, we expand even further, beyond |\endcsname|,
%    and thus insert the contents of the \textsf{token} register
%    |\theorem@bodyfont| in the replacement text.
%    \begin{macrocode}
            \expandafter \endcsname \the \theorem@bodyfont
%    \end{macrocode}
%    Now it is time to call the macro |\@thm| which takes over
%    the further processing.  It has two arguments: the current
%    counter name (in our example, \texttt{Lem}), and the text of the
%    label (in our example, \texttt{Lemma}).
%    \begin{macrocode}
     \@thm{#1}{#2}}%
%    \end{macrocode}
%    With this, the `sub-definition' is complete.  The macro
%    |\@endtheorem| ends a theorem environment and is, so far, nothing
%    but an |\endtrivlist|.  (Hence it is defined globally, and not
%    within the theorem styles.\footnote{This has to be changed as
%    soon as theorem styles that change \texttt{\bslash @endtheorem}
%    exist.  In such a case, all existing styles must be changed as
%    well since they will have to reset the macro.}) Therefore, we can
%    set it equivalent to the macro that ends the theorem set (in our
%    example, |\endLem|).  However, if some day theorem styles exist
%    that do change |\@endtheorem|, we would have to use the
%    commented-out line instead.
%    \begin{macrocode}
    \global \expandafter \let \csname end#1\endcsname \@endtheorem
%   \global\@namedef{end#1}{\@endtheorem}%
%    \end{macrocode}
%    With these commands all the required definitions are employed,
%    unless the test |\@ifdefinable| has failed.  Therefore, we
%    end the second argument of this macro and with it the definition
%    of |\@xnthm|.
%    \begin{macrocode}
   }}
%    \end{macrocode}
% \end{macro}
%
% \begin{macro}{\@ynthm}
%    The definition of |\@ynthm| is completely analogous. In
%    this case the new counter that is defined is not reset within
%    another counter; thus the definition of |\the...|
%    is simplified:
%    \begin{macrocode}
\gdef\@ynthm#1#2{\expandafter\@ifdefinable\csname #1\endcsname
   {\@definecounter{#1}%
    \expandafter\xdef\csname the#1\endcsname{\@thmcounter{#1}}%
%    \end{macrocode}
%    The rest of the definition corresponds literally to that of
%    |\@xnthm|:
%    \begin{macrocode}
    \def\@tempa{\global\@namedef{#1}}\expandafter \@tempa
     \expandafter{\csname th@\the \theorem@style \expandafter
     \endcsname \the\theorem@bodyfont \@thm{#1}{#2}}%
    \global \expandafter \let \csname end#1\endcsname \@endtheorem}}
%    \end{macrocode}
% \end{macro}
%
% \begin{macro}{\@othm}
%    The definition of |\@othm| does not contain anything new.
%    \begin{macrocode}
\gdef\@othm#1[#2]#3{%
%    \end{macrocode}
% \changes{v2.2b}{95/10/07}{Build in check for counter defined}
%    We do not define a new counter but instead use one that has
%    already been defined.  Thus the only definition we need is that
%    of this pseudo-counter (i.e.\ |\the|\meta{env.\ name}).
%    First we check if |#2| corresponds to a known counter name.
%    \begin{macrocode}
  \expandafter\ifx\csname c@#2\endcsname\relax
   \@nocounterr{#2}%
  \else
   \expandafter\@ifdefinable\csname #1\endcsname
   {\expandafter \xdef \csname the#1\endcsname
     {\expandafter \noexpand \csname the#2\endcsname}%
%    \end{macrocode}
%    All other parts of the definition can be adopted from
%    |\@xnthm|. We have to remember, though, that in this case
%    the name of the current counter and the theorem label have moved
%    to the second and third arguments.
%    \begin{macrocode}
    \def\@tempa{\global\@namedef{#1}}\expandafter \@tempa
     \expandafter{\csname th@\the \theorem@style \expandafter
     \endcsname \the\theorem@bodyfont \@thm{#2}{#3}}%
    \global \expandafter \let \csname end#1\endcsname \@endtheorem}%
  \fi}
%    \end{macrocode}
% \end{macro}
%
% \subsection{Macros that are employed in a {\envfont theorem}
%    environment}
%
% \begin{macro}{\@thm}
%    The macro |\@thm| has to increase the current counter.  Then,
%    depending on whether the environment has (or does not have) an
%    optional argument, it has to branch into either |\@begintheorem|
%    or |\@opargtheorem|.
%    \begin{macrocode}
\gdef\@thm#1#2{\refstepcounter{#1}%
%    \end{macrocode}
%    Now we start a {\envfont trivlist} environment, and give
%    |\@topsep| and |\@topsepadd| the values of the \textsf{skip}
%    registers |\theorempreskipamount| and |\theorempostskipamount|.
%    The value in |\@topsep| is the vertical space that is inserted by
%    the first (and only) |\item| in our |\trivlist| whilst
%    |\@topsepadd| is inserted by |\@endparenv| at the end of that
%    {\envfont trivlist} environment.  By using these registers, we
%    obtain the desired space around a {\envfont theorem} environment.
%    \begin{macrocode}
   \trivlist
   \@topsep \theorempreskipamount               % used by first \item
   \@topsepadd \theorempostskipamount           % used by \@endparenv
%    \end{macrocode}
%    Now we have to test whether an optional argument has been given.
% \changes{v2.0d}{89/07/18}{Forgotten percent added.}
%    \begin{macrocode}
   \@ifnextchar [%
%    \end{macrocode}
%    If there is an optional argument, we will call |\@ythm|, and
%    move the arguments read back into the input stream.
%    \begin{macrocode}
   {\@ythm{#1}{#2}}%
%    \end{macrocode}
%    If not, we call |\@begintheorem|. Its first argument is the
%    name of the theorem set (hence the second argument of
%    |\@thm|). Its second argument is the macro that produces the
%    current number.
%    \begin{macrocode}
   {\@begintheorem{#2}{\csname the#1\endcsname}\ignorespaces}}
%    \end{macrocode}
% \end{macro}
%
% \begin{macro}{\@xthm}
% \begin{macro}{\@ythm}
%    Both these macros were originally called by |\@thm|.  We do
%    not need |\@xthm| anymore, hence we reset it to
%    |\relax|. The definition of |\@ythm| has not changed at
%    all from its definition in \LaTeX. In order to make the macros
%    easier to understand, we will
%    nevertheless present it (commented out).
%    \begin{macrocode}
\global\let\@xthm\relax
% \def\@ythm#1#2[#3]{\@opargbegintheorem{#2}{\csname
%           the#1\endcsname}{#3}\ignorespaces}
%</package>
%    \end{macrocode}
%    The primitive |\ignorespaces| in |\@ythm| and
%    |\@thm| is needed to remove the spaces between the
%    |\begin{...}| and the actual text.
% \end{macro}
% \end{macro}
%
%
% \subsection{Definition of the theorem substyles}
%
% As already pointed out, the theorem substyles, defined below, are
% only loaded when necessary.  Note that all these substyles, except
% \texttt{plain}, have |\slshape| as the default body font.
%
% \subsubsection{The \texttt{plain} style}
%
% \label{dok:th@plain} As the following macros use |@|, we have
% to locally set the |\catcode| of this symbol to ``letter''.
% This happens within a group, so that we do not have to worry about
% which |\catcode| that symbol had before.
%    \begin{macrocode}
%<*thp>
\begingroup \makeatletter
%    \end{macrocode}
% Since we are now within a group, we must make all definitions
% globally.  First we make sure that \texttt{theorem.sty} is loaded.
% This will allow us to use this file as a document style option
% without having to call \texttt{theorem} itself as an option.  At the
% same time, we assure that at least version 2 is loaded, since
% |\theorem@style| was not defined in earlier versions.
%    \begin{macrocode}
\@ifundefined{theorem@style}{\input{theorem.sty}}{}
\ProvidesFile{thp.sty}
             [\FMithmInfo]
%    \end{macrocode}
%
%
% \begin{macro}{\th@plain}
% \begin{macro}{\@begintheorem}
% \begin{macro}{\@opargbegintheorem}
%    |\theoremstyle{plain}| corresponds to the original
%    definition, except that the distances to the surrounding text are
%    determined by the parameters |\theorempreskipamount| and
%    |\theorempostskipamount|.
%    First we set the default body font.
%    \begin{macrocode}
\gdef\th@plain{\normalfont\itshape
%    \end{macrocode}
%    Then we define |\@begintheorem| and
%    |\@opargbegintheorem|. These two macros define how the
%    header of a theorem is typeset.  |\@opargbegintheorem| will
%    be called if a \textsf{theorem} environment with an optional
%    argument is encountered; otherwise, the header is constructed by
%    calling |\@begintheorem|.  If one of these macros is
%    executed, we are within a \textsf{trivlist} environment started by
%    |\@thm|.  So the theorem header is produced with an
%    |\item| command.
%
%    Instead of specifying the header font directly, all standard
%    theorem styles use the |\theorem@headerfont| macro to allow
%    customization.  The extra space (|\labelsep|) is necessary
%    because of problems in the \textsf{trivlist} environment.
%    \begin{macrocode}
  \def\@begintheorem##1##2{%
        \item[\hskip\labelsep \theorem@headerfont ##1\ ##2]}%
%    \end{macrocode}
%    The definition of |\@opargbegintheorem| is completely
%    analogous. The only difference is the fact that there exists a
%    third argument (which is the optional parameter of the
%    environment and contains additional information about the
%    theorem).  Customarily we enclose it in parentheses.
%    \begin{macrocode}
\def\@opargbegintheorem##1##2##3{%
   \item[\hskip\labelsep \theorem@headerfont ##1\ ##2\ (##3)]}}
%    \end{macrocode}
% \end{macro}
% \end{macro}
% \end{macro}
%
% \noindent
% We conclude with an |\endgroup| to restore the
% |\catcode| of |@|.
%    \begin{macrocode}
\endgroup
%</thp>
%    \end{macrocode}
%
%
% \subsubsection{The \texttt{break} style}
%
% This style option is stored in the file |thb.sty|.
% For the next two lines see the documentation for |\th@plain| on
% page~\pageref{dok:th@plain}.
%    \begin{macrocode}
%<*thb>
\begingroup \makeatletter
\@ifundefined{theorem@style}{\input{theorem.sty}}{}
\ProvidesFile{thb.sty}
             [\FMithmInfo]
%    \end{macrocode}
%
% \begin{macro}{\th@break}
%    |\theoremstyle{break}| produces a line break after the name
%    of the theorem.  The font is |\slshape|.  Hence, we define
%    |\th@break| as follows:
%    \begin{macrocode}
\gdef\th@break{\normalfont\slshape
  \def\@begintheorem##1##2{\item[%
%    \end{macrocode}
%    We run into the following problem: it is not possible to create
%    the header with |\item[|\meta{title}|]| and then start
%    a new line by, for example, |\mbox{}\\|. Such a definition
%    will fail whenever a list environment follows immediately.  With
%    the above construction, the |\mbox{}| causes the switch
%    |@inlabel| (cf.\ definition of |\list| and
%    |\trivlist| in \cite{latex:source}) to be set to \texttt{false}
%    and so the following list will insert additional vertical
%    space (|\topskip|). This is quite annoying. Therefore, we
%    create the line break within the |\item|. In order to ensure
%    that the text will begin at the proper position in the following
%    line, we simply pretend that the label does not take any
%    room.\footnote{This will lead to problems whenever very high
%    symbols occurring in the line tower into the
%    heading. So, something else has to be done here sometime.}
%    \begin{macrocode}
       \rlap{\vbox{\hbox{\hskip \labelsep\theorem@headerfont ##1\ ##2}%
                   \hbox{\strut}}}]}%
%    \end{macrocode}
%    Again, the definition of |\@opargbegintheorem| is
%    completely analogous.
%    \begin{macrocode}
\def\@opargbegintheorem##1##2##3{%
  \item[\rlap{\vbox{\hbox{\hskip \labelsep \theorem@headerfont
                     ##1\ ##2\ (##3)}%
                    \hbox{\strut}}}]}}
\endgroup
%</thb>
%    \end{macrocode}
% \end{macro}
%
%
% \subsubsection{The \texttt{changebreak} style}
%
%    \begin{macrocode}
% This style option is stored in the file |thcb.sty|.
%    \begin{macrocode}
%<*thcb>
\begingroup \makeatletter
\@ifundefined{theorem@style}{\input{theorem.sty}}{}
\ProvidesFile{thcb.sty}
             [\FMithmInfo]
%    \end{macrocode}
%
% \begin{macro}{\th@changebreak}
%    The \texttt{change-break} theorem style is like \texttt{break}
%    but with interchange of theorem name and theorem number.  Thus we
%    define |\th@changebreak| as follows:
%    \begin{macrocode}
\gdef\th@changebreak{\normalfont\slshape
  \def\@begintheorem##1##2{\item
       [\rlap{\vbox{\hbox{\hskip\labelsep \theorem@headerfont ##2\ ##1}%
                    \hbox{\strut}}}]}%
\def\@opargbegintheorem##1##2##3{%
  \item[\rlap{\vbox{\hbox{\hskip\labelsep \theorem@headerfont
                     ##2\ ##1\ (##3)}%
                    \hbox{\strut}}}]}}
\endgroup
%</thcb>
%    \end{macrocode}
% \end{macro}
%
%
% \subsubsection{The \texttt{change} style}
%
% This style option is stored in the file |thc.sty|.
%    \begin{macrocode}
%<*thc>
\begingroup \makeatletter
\@ifundefined{theorem@style}{\input{theorem.sty}}{}
\ProvidesFile{thc.sty}
             [\FMithmInfo]
%    \end{macrocode}
%
% \begin{macro}{\th@change}
%    The \texttt{change} theorem style corresponds to the
%    \texttt{change break} style without a linebreak after the header.
%    To say it in another way, it's the same as the \texttt{plain}
%    style but with number and name interchanged and |\slshape| as the
%    default font.
%    \begin{macrocode}
\gdef\th@change{\normalfont\slshape
  \def\@begintheorem##1##2{\item
       [\hskip\labelsep \theorem@headerfont ##2\ ##1]}%
\def\@opargbegintheorem##1##2##3{%
  \item[\hskip\labelsep \theorem@headerfont ##2\ ##1\ (##3)]}}
\endgroup
%</thc>
%    \end{macrocode}
% \end{macro}
%
%
% \subsubsection{The \texttt{marginbreak} style}
%
% This style option is the one used most often at Mainz.  It is saved
% in the file |thmb.sty|.
%    \begin{macrocode}
%<*thmb>
\begingroup \makeatletter
\@ifundefined{theorem@style}{\input{theorem.sty}}{}
\ProvidesFile{thmb.sty}
             [\FMithmInfo]
%    \end{macrocode}
%
% \begin{macro}{\th@marginbreak}
%    The \texttt{margin break} style is nearly the same as the
%    \texttt{change break} style. The only difference is the placement
%    of the theorem number.  We use |\llap| to place it in the left
%    margin.
%
%    In this style |\labelsep| denotes the separation between the
%    number and the text.
%    \begin{macrocode}
\gdef\th@marginbreak{\normalfont\slshape
  \def\@begintheorem##1##2{\item
       [\rlap{\vbox{\theorem@headerfont
                    \hbox{\llap{##2}\hskip\labelsep ##1}%
                    \hbox{\strut}}}]}%
\def\@opargbegintheorem##1##2##3{%
  \item[\rlap{\vbox{\theorem@headerfont
                    \hbox{\llap{##2}\hskip\labelsep ##1\ (##3)}%
                    \hbox{\strut}}}]}}
\endgroup
%</thmb>
%    \end{macrocode}
% \changes{v2.1c}{90/02/17}{`endgroup outside macrocode env.}
% \end{macro}
%
%
%
% \subsubsection{The \texttt{margin} style}
%
% This style option is stored in the file |thm.sty|.
%    \begin{macrocode}
%<*thm>
\begingroup \makeatletter
\@ifundefined{theorem@style}{\input{theorem.sty}}{}
\ProvidesFile{thm.sty}
             [\FMithmInfo]
%    \end{macrocode}
%
% \begin{macro}{\th@margin}
%    Again this is only a variant of the theorem styles described
%    above without any new ideas.
%    \begin{macrocode}
\gdef\th@margin{\normalfont\slshape
  \def\@begintheorem##1##2{\item
       [\theorem@headerfont \llap{##2}\hskip\labelsep ##1]}%
\def\@opargbegintheorem##1##2##3{%
  \item[\theorem@headerfont \llap{##2}\hskip\labelsep ##1\ (##3)]}}
\endgroup
%</thm>
%    \end{macrocode}
% \end{macro}
%
%
%
%
% \subsection{Final Definitions}
%
% \begin{macro}{\theorempreskipamount}
% \begin{macro}{\theorempostskipamount}
%    The \textsf{skip} parameters that regulate the vertical empty space
%    before and after the {\envfont theorem} environment have
%    to be allocated as well.
% \changes{v2.0f}{89/07/20}{Unnecessary `global' removed.}
%    \begin{macrocode}
%<*package>
\newskip\theorempreskipamount
\newskip\theorempostskipamount
%    \end{macrocode}
%    Since we have used the same values for all theorem sets, we now can
%    assign them.
%    \begin{macrocode}
\global\setlength\theorempreskipamount{12pt plus 5pt minus 3pt}
\global\setlength\theorempostskipamount{8pt plus 3pt minus 1.5pt}
%    \end{macrocode}
% \end{macro}
% \end{macro}
%
%
% \begin{macro}{\@endtheorem}
%    The same holds for the macro |\@endtheorem|, which ends a
%    {\envfont theorem} environment. Since it is the same for all
%    theorem sets, it is removed from the macros |\th@|\meta{style}.
%    It simply ends the {\envfont trivlist} environment, which was
%    begun in |\@thm|.
%    \begin{macrocode}
\global\let\@endtheorem=\endtrivlist
%    \end{macrocode}
% \end{macro}
%
%
%
% \begin{macro}{\@preamblecmds}
%    All macros defined above are to be used only in the preamble.
%    Therefore, we insert them in |\@preamblecmds| which will disable
%    them at begin document. This is done by the internal \LaTeXe{}
%    command |\@onlypreamble|.
% \changes{v2.0g}{89/09/19}{`theoremheaderfont and `theorembodyfont
%    added.}
% \changes{v2.2a}{94/02/03}{Use \LaTeXe{} interface}
%    \begin{macrocode}
\@onlypreamble\@xnthm
\@onlypreamble\@ynthm
\@onlypreamble\@othm
\@onlypreamble\newtheorem
\@onlypreamble\theoremstyle
\@onlypreamble\theorembodyfont
\@onlypreamble\theoremheaderfont
%    \end{macrocode}
% \end{macro}
%
% Finally we declare the \texttt{plain} theorem style to be the default.
% \changes{v2.1b}{90/06/04}{Fixed missing `item by declaring default
%    style.}
%    \begin{macrocode}
\theoremstyle{plain}
%</package>
%    \end{macrocode}
%
% \Finale

\endinput
}{}
\ProvidesFile{thcb.sty}
             [\FMithmInfo]
%    \end{macrocode}
%
% \begin{macro}{\th@changebreak}
%    The \texttt{change-break} theorem style is like \texttt{break}
%    but with interchange of theorem name and theorem number.  Thus we
%    define |\th@changebreak| as follows:
%    \begin{macrocode}
\gdef\th@changebreak{\normalfont\slshape
  \def\@begintheorem##1##2{\item
       [\rlap{\vbox{\hbox{\hskip\labelsep \theorem@headerfont ##2\ ##1}%
                    \hbox{\strut}}}]}%
\def\@opargbegintheorem##1##2##3{%
  \item[\rlap{\vbox{\hbox{\hskip\labelsep \theorem@headerfont
                     ##2\ ##1\ (##3)}%
                    \hbox{\strut}}}]}}
\endgroup
%</thcb>
%    \end{macrocode}
% \end{macro}
%
%
% \subsubsection{The \texttt{change} style}
%
% This style option is stored in the file |thc.sty|.
%    \begin{macrocode}
%<*thc>
\begingroup \makeatletter
\@ifundefined{theorem@style}{% \iffalse meta-comment
%
% Copyright (C) 1993-2024
%
% The LaTeX Project and any individual authors listed elsewhere
% in this file.
%
% This file is part of the Standard LaTeX `Tools Bundle'.
% -------------------------------------------------------
%
% It may be distributed and/or modified under the
% conditions of the LaTeX Project Public License, either version 1.3c
% of this license or (at your option) any later version.
% The latest version of this license is in
%    https://www.latex-project.org/lppl.txt
% and version 1.3c or later is part of all distributions of LaTeX
% version 2005/12/01 or later.
%
% The list of all files belonging to the LaTeX `Tools Bundle' is
% given in the file `manifest.txt'.
%
% \fi
%
%%
%
%
\def\FMithmInfo{2023/07/05 v2.2c Theorem extension package (FMi)}
%
% \ProvidesFile{theorem.dtx}[\FMithmInfo]
%\iffalse   % this is a METACOMMENT !
%%
%% Package `theorem' to use with LaTeX2e
%% Copyright (C) 1989-1995 Frank Mittelbach, all rights reserved.
%<+package>\@ifundefined{theorem@style}{}{\endinput}
%<+package>\NeedsTeXFormat{LaTeX2e}[1995/06/01]
%<+package>\ProvidesPackage{theorem}
%<+package>         [\FMithmInfo]
%
% \fi
%
% \changes{v1.0}{87/05/07}{First draft.}
% \changes{v1.0a}{87/05/07}{`thistheoremstyle' added.}
% \changes{v1.1a}{87/07/29}{New concept: redefine the internal
%                          \protect\LaTeX{} commands `@xnthm' etc.}
% \changes{v1.1b}{87/07/31}{`theoremfont' used to specify font.}
% \changes{v1.2a}{88/01/08}{`@endtheorem' now globally defined.}
% \changes{v1.2b}{88/01/11}{`newtheorem' only allowed in the preamble.}
% \changes{v1.3a}{88/02/14}{`theoremfont' now a toks register.}
%
% \changes{v2.0a}{88/11/16}{New implementation and new documentation.}
% \changes{v2.0a}{88/11/16}{No layout changes but internal modifications
%                           old substyles won't work.}
% \changes{v2.0b}{89/01/30}{`global' used for all definitions.}
% \changes{v2.0c}{89/06/15}{Documentation translated to English.
%                            Thanks to LLL and Nina}
% \changes{v2.1a}{90/02/15}{Support for new font selection.}
% \changes{v2.2a}{94/02/03}{Cleanup for LaTeX2e}
%
% \DoNotIndex{\@endparenv,\@ifdefinable,\@ifnextchar,\@ifundefined}
% \DoNotIndex{\@namedef,\@nameuse,\@spaces,\@tempa,\@warning}
% \DoNotIndex{\ }
% \DoNotIndex{\begingroup,\bfseries}
% \DoNotIndex{\csname}
% \DoNotIndex{\def,\do,\docdate}
% \DoNotIndex{\endcsname,\endgroup,\endinput,\endtrivlist,\expandafter}
% \DoNotIndex{\filedate,\fileversion}
% \DoNotIndex{\gdef,\global}
% \DoNotIndex{\hbox,\hskip}
% \DoNotIndex{\ignorespaces,\input,\itshape,\item}
% \DoNotIndex{\labelsep,\let,\llap}
% \DoNotIndex{\makeatletter}
% \DoNotIndex{\noexpand}
% \DoNotIndex{\refstepcounter,\relax,\rlap}
% \DoNotIndex{\setlength,\slshape,\space,\string,\strut}
% \DoNotIndex{\the,\trivlist,\typeout}
% \DoNotIndex{\vbox}
% \DoNotIndex{\xdef}
%
% \def\deflabel#1{\ttfamily #1\hfill}
% \def\deflist#1{\begin{list}{}{\settowidth\labelwidth{\ttfamily #1}%
%                               \setlength\leftmargin\labelwidth
%                               \addtolength\leftmargin\labelsep
%                               \let\makelabel\deflabel}}
% \def\enddeflist{\end{list}}
% \def\envfont{\normalfont\ttfamily}
%
% \MakeShortVerb{\|}
%
% \newcommand{\TUB}{{\sl TUGboat\/}}
%
% \GetFileInfo{theorem.dtx}
% \title{An Extension of the
%        \LaTeX{} theorem environment\thanks{This file
%        has version number \fileversion, last
%        revised \filedate.}}
% \author{Frank Mittelbach}
%
% \MaintainedByLaTeXTeam{tools}
% \maketitle
%
% \begin{abstract}
%    The macros described in this paper yield an extension of the
%    \LaTeX{} theorem mechanism. It is designed to satisfy the
%    different requirements of various journals. Thus, the
%    layout of the ``theorems'' can be manipulated by determining a
%    ``style''. This article describes not only the use, but
%    also the definition, of the necessary macros.
% \end{abstract}
%
% \section*{Preface to version 2.2}
%
% For \LaTeXe{} this package did not need any fundamental changes. I
% only modified the messages generated so that theorem layout styles
% will show up with the |\listfiles| command and cleaned the section
% on the New Font Selection Scheme since this is now included in
% \LaTeX.
%
% \section*{Preface to version 2.1}
%
% This version is identical to 2.0g described in \TUB\ 10\#3 except
% for some internal defaults which are now set depending on the used
% font selection scheme.
%
% This was done to avoid unpleasant surprises if the new font selection
% scheme is in force. For further details see
% section~\ref{sec:newfonts} and \cite{bk:GMS94}.
%
% \section{Introduction}
%
% For our purposes here, ``theorems'' are labelled enunciations,
% often set off from the main text by extra space and a font change.
% Theorems, corollaries, conjectures, definitions, and remarks
% are all instances of ``theorems''.  The ``header'' of these structures
% is composed of a label (such as \textsc{Theorem} or \textsc{Remark})
% and a number which serializes an item in the sequence of items
% with the same label.
%
% Shortly after the introduction of \LaTeX{} at the Fachbereich
% Mathematik in Mainz, the desire to manipulate the layout of
% ``theorems'' arose.  In Mainz, the following two conventions came
% into general use:
% \begin{enumerate}
%   \item The number of the theorem is shown in the margin.
%   \item There is a line break at
%         the end of the theorem header.
% \end{enumerate}
%
% Additionally, some journals require different formats which depend on
% the ``sort of theorem'': e.g.\ often remarks and definitions are set
% in |\upshape|, while |\itshape| is employed for main theorems.
%
% Confronted with these requirements, a theorem environment was
% developed in Mainz which allows separate determination of the
% layout of the ``theorems sets'', comparable to |\pagestyle|.
%
%
% \section{The user interface}
% \subsection{Defining new theorem sets}
%
% \DescribeMacro\newtheorem
% As in the original \LaTeX{} version, the command |\newtheorem|
% defines a new ``theorem set'' or ``theorem-like structure''.
% Two required arguments name the new environment and give the text
% to be typeset with each instance of the new ``set'', while
% an optional argument determines how
% the ``set'' is enumerated:
% \begin{description}
%    \item[\ttfamily \bslash newtheorem\{foo\}\{bar\}]
%       The theorem set {\envfont foo} (whose name is \texttt{bar})
%       uses its own counter.
%    \item[\ttfamily \bslash newtheorem\{foo2\}{[foo]}\{bar2\}]
%       The theorem set {\envfont foo2} (printed name \texttt{bar2})
%       uses the same counter as the theorem set \texttt{foo}.
%    \item[\ttfamily \bslash newtheorem\{foo3\}\{bar3\}{[section]}]
%       The theorem set {\envfont foo3} (printed name \texttt{bar3}) is
%       enumerated within the counter \texttt{section}, i.e.\ with every
%       new |\section| the enumeration begins again with 1, and
%       the enumeration is composed from the section-number and the
%       theorem counter itself.
% \end{description}
%
% \DescribeMacro\theoremstyle
% Additionally, the command |\theoremstyle| can define the layout
% of various, or all, theorem sets.  It should be noted that any
% theorem set defined by |\newtheorem| is typeset in the
% |\theoremstyle| that is current at the time of the definition.
% Thus, the following
% \begin{verbatim}
%    \theoremstyle{break}        \newtheorem{Cor}{Corollary}
%    \theoremstyle{plain}        \newtheorem{Exa}{Example}[section]
%\end{verbatim}                             ^^A No blank at line start
% leads to the result that the set {\envfont Cor} is formatted in the
% style \texttt{break}, while the set {\envfont Exa} and all the
% following ones are formatted in the style \texttt{plain}, unless
% another |\theoremstyle| follows.  Since the definitions installed by
% |\newtheorem| are global, one also can limit |\theoremstyle| locally
% by grouping braces.
%
% \DescribeMacro\theorembodyfont
% The choice of the font for the theorem body is completely
% independent of the chosen |\theoremstyle|; this has proven to
% be very advantageous.  For example,
% \begin{verbatim}
%   {\theorembodyfont{\upshape}          \newtheorem{Rem}{Remark}}
%\end{verbatim}
% defines a theorem set {\envfont Rem}, which will be set in
% |\upshape| in the current layout (which in our example is
% \texttt{plain}).  As with |\theoremstyle|, the |\theorembodyfont|
% chosen is that current at the time of |\newtheorem|.  If
% |\theorembodyfont| is not specified or one defines
% |\theorembodyfont{}|, then the font used will be that defined by the
% |\theoremstyle|.
%
% \DescribeMacro\theoremheaderfont It is also possible to customize
% the font used for the theorem headers.  This is, however, a global
% declaration, and therefore there should be at most one
% |\theoremheaderfont| declaration in the preamble.\footnote{If it is
% actually necessary to have different header fonts, one has to define
% new theorem styles (substituting the desired font) or specify the
% information directly in the \texttt{\bslash newtheorem} declaration
% (the unclean variant).}
%
% \DescribeMacro\theorempreskipamount
% \DescribeMacro\theorempostskipamount Two additional parameters
% affect the vertical space around the theorem environments:
% |\theorempreskipamount| and |\theorempostskipamount| define,
% respectively, the spacing before and after such an environment.
% These parameters apply for all theorem sets and can be manipulated
% with the ordinary length macros.  They are rubber lengths
% (`\textsf{skips}'), and therefore can contain \texttt{plus} and
% \texttt{minus} parts.
%
% Since the definition of theorem sets should---most sensibly---be
% placed in the preamble, we only allow installation there.
% It is therefore possible to release the memory used here after
% |\begin{document}|, in order to make room for other
% applications.
%
%
% \subsection{Existing theorem styles}
%
% The following theorem styles exist to date:
% \begin{deflist}{marginbreak}
%    \item[plain]
%       This theorem style emulates the original \LaTeX{} definition,
%       except that additionally the parameters
%       |\theorem...skipamount| are used.
%    \item[break]
%       In this style, the theorem header is followed by a line
%       break.
%    \item[marginbreak]
%       The theorem number is set in the margin, and there is a line
%       break as in \texttt{break}.
%    \item[changebreak]
%       Like \texttt{break}, but with header number and text
%       interchanged.
%    \item[change]
%       Header number and text are interchanged, without a line break.
%    \item[margin]
%       The number is set in the left margin, without a line break.
% \end{deflist}
% All styles (except \texttt{plain}) select |\slshape| as the default
% |\theorembodyfont|.
%
%
% \subsection{Examples}
%
% Given the above theorem sets {\envfont Cor}, {\envfont Exa}
% and {\envfont Rem},
% suppose that the preamble also contains the declarations:
% \begin{verbatim}
%    \theoremstyle{marginbreak}   \newtheorem{Lem}[Cor]{Lemma}
%    \theoremstyle{change}
%    \theorembodyfont{\itshape}        \newtheorem{Def}[Cor]{Definition}
%
%    \theoremheaderfont{\scshape}
%\end{verbatim}
% Then the following are some typical examples of the typeset output
% resulting from their use.
%
% \begin{Cor}
%    This is a sentence typeset in the theorem environment {\envfont
%    Cor}.
% \end{Cor}
%
% \begin{Exa}
%    This is a sentence typeset in the theorem environment {\envfont
%    Exa}.
% \end{Exa}
%
% \begin{Rem}
%    This is a sentence typeset in the theorem environment {\envfont
%    Rem}.
% \end{Rem}
%
% \begin{Lem}[Ben User]
%    This is a sentence typeset in the theorem environment {\envfont
%    Lem}.
% \end{Lem}
%
% \begin{Def}[Very impressive Definition]
%    This is a sentence typeset in the theorem environment {\envfont
%    Def}.
% \end{Def}
%
% The last two examples show the effect of the optional argument to
% a theorem environment (it is the text typeset in parentheses).
%
%
% \section{Special Considerations}
% \label{sec:newfonts}
%
% Theoremheader and body are implemented as a unit. This means that
% the |\theoremheaderfont|  will inherit characteristics of the
% |\theorembodyfont| in \LaTeXe. Thus, if for example
% |\theorembodyfont| is |\itshape| and |\theoremheaderfont|
% is |\bfseries| the font selected for the header will have the
% characteristics `bold extended italic'. If this is not desired one
% should set the |\theoremheaderfont| to something like
% \begin{verbatim}
%      \theoremheaderfont{\normalfont\bfseries}
%\end{verbatim}
% i.e.\ supplying all necessary font information explicitly.
%
%
%
% \section{Acknowledgements}
%
% The publication of this set of macros was only possible with the
% help of Christina Busse (translating the manuscript into English),
% Joachim Pense (playing the r\^ole of typist), Chris Rowley
% (looking everything over) and many others providing useful
% suggestions.
%
% \MaybeStop{%
%    \begin{thebibliography}{9}
%       \bibitem{bk:GMS94} \textsc{M.~Goossens}, \textsc{F.~Mittelbach}
%         and \textsc{A.~Samarin}.
%         \newblock The \LaTeX{} Companion.
%         \newblock
%         Addison-Wesley, Reading, Massachusetts, 1994.
%       \bibitem{latex:source} \textsc{Lamport, Leslie.} \newblock
%                               \texttt{latex.tex}, version 2.09, date
%                               Feb. 1990.
%    \end{thebibliography}
%    \PrintIndex
%    \PrintChanges}
%
%
% \section{The documentation driver file}
%
% The next bit of code contains the documentation driver file for
% \TeX{}, i.e., the file that will produce the documentation you are
% currently reading. It will be extracted from this file by the
% \texttt{docstrip} program. Since it is the first code in the file
% one can alternatively process this file directly with \LaTeXe{} to
% obtain the documentation.
%
%    \begin{macrocode}
%<*driver>
\documentclass{ltxdoc}

\usepackage{theorem}

%<+driver>% The next few lines define theorem sets which are used
%<+driver>% in the example section of the documentation.

\theoremstyle{break}       \newtheorem{Cor}{Corollary}
\theoremstyle{plain}       \newtheorem{Exa}{Example}[section]
{\theorembodyfont{\upshape}\newtheorem{Rem}{Remark}}
\theoremstyle{marginbreak} \newtheorem{Lem}[Cor]{Lemma}
\theoremstyle{change}
\theorembodyfont{\itshape} \newtheorem{Def}[Cor]{Definition}

\theoremheaderfont{\scshape}

\RecordChanges

\begin{document}
   \DocInput{theorem.dtx}
\end{document}
%</driver>
%    \end{macrocode}
%
% \section{Definition of the Macros}
%
% If the file has been loaded before, we abort immediately. If not the
% package announces itself (this is actually done at the very top of
% the file---the way it is done isn't good style so don't copy it).
% \changes{v2.0e}{89/07/19}{Spaces removed from `typeout'.}
% \changes{v2.2b}{95/11/19}{Announce moved to top of file}
%    \begin{macrocode}
%<*package>
%\@ifundefined{theorem@style}{}{\endinput}
%\def\FMithmInfo{1995/11/19 v2.2b Theorem extension package (FMi)}
%\ProvidesPackage{theorem}[\FMithmInfo]
%    \end{macrocode}
%
%
% \subsection{Definition of theorem styles and fonts}
%
% All the definitions in this file are done globally to allow
% inputting this file inside a group.
%
% \begin{macro}{\theoremstyle}
%    Before a theorem style can be installed, the chosen style must be
%    known. For that reason, we must test to see that
%    |\th@|\meta{style} is known or, more precisely, that it is
%    different from |\relax|. If the style is not known then
%    |\th@plain| is used.
%    \begin{macrocode}
\gdef\theoremstyle#1{%
   \@ifundefined{th@#1}{\@warning
          {Unknown theoremstyle `#1'. Using `plain'}%
          \theorem@style{plain}}%
%    \end{macrocode}
%    We save the theorem style to be used in the token register
%    |\theorem@style|.
%    \begin{macrocode}
      {\theorem@style{#1}}%
%    \end{macrocode}
%    Now we ``evaluate'' the theorem style: this means, we call the
%    macro |\th@|\meta{style} which will activate the relevant
%    definitions which are contained in a separate file.  This is done
%    in a group to suppress changes to the current font.  This could
%    otherwise pose problems together with the new font selection
%    scheme\footnote{When I printed the original article using the new
%    font selection scheme I ended with a document with slanted
%    typefaces (text headings and all) simply because one of the
%    theorem styles used \texttt{\bslash sl} at toplevel.} if the
%    |\th@|\meta{style} is evaluated a second time.
% \changes{v2.1a}{90/02/15}{Group added around `th@..}
% \changes{v2.1a}{90/02/15}{Reseting of `..@begintheorem removed.}
%    \begin{macrocode}
      \begingroup
        \csname th@\the\theorem@style \endcsname
      \endgroup}
%    \end{macrocode}
% \end{macro}
%
%
% \begin{macro}{\@begintheorem}
% \changes{v2.1a}{90/02/15}{Relaxed at toplevel.}
% \begin{macro}{\@opargbegintheorem}
% \changes{v2.1a}{90/02/15}{Relaxed at toplevel.}
%    We reset |\@begintheorem| and |\@opargbegintheorem| to
%    |\relax| since these commands are no longer necessary at
%    toplevel. This will save a few tokens.
%    \begin{macrocode}
\global\let\@begintheorem\relax
\global\let\@opargbegintheorem\relax
%    \end{macrocode}
% \end{macro}
% \end{macro}
%
% \begin{macro}{\theorem@style}
%    Obviously the \textsf{token} register used above has to be
%    allocated. To assure the utmost compatibility with the
%    original \LaTeX{} definition, we set the default theorem style to
%    \texttt{plain}, which implements the usual \LaTeX{} convention.
% \changes{v2.0f}{89/07/20}{Unnecessary `global' removed.}
%    \begin{macrocode}
\newtoks\theorem@style
\global\theorem@style{plain}
%    \end{macrocode}
% \end{macro}
%
% \begin{macro}{\theorembodyfont}
% \begin{macro}{\theorem@bodyfont}
%    If the |\theorembodyfont| is set by the user then it should not
%    interact with the default font set in the theorem style. When the
%    new font selection is in force this may happen if, for example,
%    the default is |\itshape| and the new |\theorembodyfont| is
%    |\sffamily|. So we add a |\reset@font| command in front of the
%    user definition.
% \changes{v2.0f}{89/07/20}{Unnecessary `global' removed.}
% \changes{v2.0g}{89/09/19}{`theoremfont to `theorembodyfont.}
% \changes{v2.1c}{92/06/22}{`reset@font added and logic changed.}
% \changes{v2.1d}{92/10/28}{Reset to empty if arg is empty}
%    \begin{macrocode}
\gdef\theorembodyfont#1{%
%    \end{macrocode}
%    We check if the argument supplied is empty and if so put nothing
%    into the |\theorem@bodyfont| token register to allow for
%    |\theorembodyfont{}| as a mean of using the default of the
%    current |\theoremstyle|.
%    \begin{macrocode}
   \def\@tempa{#1}%
   \ifx\@tempa\@empty
    \theorem@bodyfont{}%
   \else
    \theorem@bodyfont{\reset@font#1}%
   \fi
   }
\newtoks\theorem@bodyfont
\global\theorem@bodyfont{}
%    \end{macrocode}
% \end{macro}
% \end{macro}
%
%
% \begin{macro}{\theoremheaderfont}
% \changes{v2.0g}{89/09/19}{Macro added.}
%    The font for the theorem headers is handled differently because
%    this definition applies to all theorem styles.
%    \begin{macrocode}
\gdef\theoremheaderfont#1{\gdef\theorem@headerfont{#1}%
%    \end{macrocode}
%    After using the macro once it is redefined to produce an error
%    message.
%    \begin{macrocode}
       \gdef\theoremheaderfont##1{%
        \typeout{\string\theoremheaderfont\space should be used
                 only once.}}}
%    \end{macrocode}
% \end{macro}
%
% \begin{macro}{\theorem@headerfont}
% \changes{v2.0g}{89/09/19}{Macro added.}
%    To set the |\theorem@headerfont| default we first test if
%    the new fontselection scheme is in force.
% \changes{v2.1a}{90/02/15}{Support for new font selection.}
%    \begin{macrocode}
\ifx\upshape\undefined
%    \end{macrocode}
%    If not we define it to expand into |\bfseries|. We don't use
%    |\let| just in case a following style option redefines this
%    macro.
%    \begin{macrocode}
\gdef\theorem@headerfont{\bfseries}
%    \end{macrocode}
%    Otherwise we reset the current shape before calling |\bfseries|.
%    \begin{macrocode}
\else \gdef\theorem@headerfont{\normalfont\bfseries}\fi
%    \end{macrocode}
% \end{macro}
%
% \begin{macro}{\th@plain}
% \begin{macro}{\th@break}
% \begin{macro}{\th@marginbreak}
% \begin{macro}{\th@changebreak}
% \begin{macro}{\th@change}
% \begin{macro}{\th@margin}
%    The different styles are defined in macros such as
%    |\th@plain|.  Since memory space is precious in
%    ``non-Big-versions'', we have to avoid offering too many unused
%    definitions.  Therefore we define these styles in separate files
%    that can be loaded on demand.  Thus the commands themselves only
%    load these files. We use |\@input@|, a \LaTeXe{} internal command
%    that ensures that the file will be listed with |\listfiles|.
% \changes{v2.2a}{94/02/02}{Use `@input@ to load theorem layout files}
%    \begin{macrocode}
\gdef\th@plain{\@input@{thp.sty}}
\gdef\th@break{\@input@{thb.sty}}
\gdef\th@marginbreak{\@input@{thmb.sty}}
\gdef\th@changebreak{\@input@{thcb.sty}}
\gdef\th@change{\@input@{thc.sty}}
\gdef\th@margin{\@input@{thm.sty}}
%    \end{macrocode}
%    This list will be expanded when new styles become
%    available.  For testing,  just append new theorem substyles
%    as document options.
% \end{macro}
% \end{macro}
% \end{macro}
% \end{macro}
% \end{macro}
% \end{macro}
%
%
%
% \subsection{Definition of a new theorem set}
%
% As already pointed out, a new theorem environment can be defined in
% three different ways:
% \begin{verbatim}
%    \newtheorem{Lem}{Lemma}
%    \newtheorem{Lem}{Lemma}[section]
%    \newtheorem{Lem}[Theorem]{Lemma}
%\end{verbatim}
% The function of the macro |\newtheorem| is to recognize these cases
% and then to branch into one of the three macros |\@ynthm|, |\@xnthm|
% or |\@othm|.  This mechanism is adopted unchanged from
% \cite{latex:source}; the essential point here is that, for example,
% in the second case, the arguments \texttt{Lem}, \texttt{Lemma} and
% \texttt{section} are passed over to the macro |\@xnthm|.
%
% We inspect this case first because the others present fewer problems,
% and thus are easily derived from this one.
%
% \begin{macro}{\@xnthm}
%    For our example arguments, the macro |\@xnthm| must fulfill
%    the following:
%    \begin{itemize}
%       \item
%          Define a new \LaTeX-counter `Lem'
%       \item
%          reset this counter within a |\section|
%       \item
%          define the macro |\theLem|
%       \item
%          define the environment macros |\Lem| and
%          |\endLem| using the current
%          |\theoremstyle| and |\theorem@bodyfont|.
%    \end{itemize}
%    Obviously, all this should happen only if the first argument
%    of |\@xnthm| (i.e.\ \texttt{Lem} in our example) is chosen so
%    as not to conflict with any previously defined commands or
%    environments.  This test is performed by the \LaTeX{} macro
%    |\@ifdefinable|.
%    \begin{macrocode}
\gdef\@xnthm#1#2[#3]{\expandafter\@ifdefinable\csname #1\endcsname
%    \end{macrocode}
%    Therefore, the first argument of |\@ifdefinable| is the
%    expansion (in the example, |\Lem|) of
%    |\csname||#1\endcsname|.  The second argument is
%    executed only if the test has been completed successfully.
%    \begin{macrocode}
   {%
%    \end{macrocode}
%    Now we define the new counter. The names of the \LaTeX{}
%    macros employed should speak for themselves:
% \changes{v2.2b}{1995/10/07}{Use \cs{@newctr}}
% \changes{v2.2c}{1995/11/23}{Use \cs{@newctr} properly.}
%    \begin{macrocode}
    \@definecounter{#1}\@newctr{#1}[#3]%
%    \end{macrocode}
%    Using |\@newctr| will give a proper error message if the counter
%    in |#3| is not defined.
%    In defining `|\theLem|' we must generate the desired macro
%    name by use of |\expandafter| and |\csname|.
%    \begin{macrocode}
    \expandafter\xdef\csname the#1\endcsname
%    \end{macrocode}
%    An |\xdef| is used in order to make the definition global,
%    and to ensure that it contains the replacement texts of
%    |\@thmcountersep| and |\@thmcounter|.\footnote{These
%    two macros can be defined by the document style.  Their default
%    values produce a `\texttt{.}' as separation and an arabic
%    representation of the number.} However, not everything should be
%    expanded. For example, it saves space to use |\thesection|
%    instead of its---at times---lengthy expansion.
%    \begin{macrocode}
      {\expandafter \noexpand \csname the#3\endcsname
       \@thmcountersep \@thmcounter{#1}}%
%    \end{macrocode}
%    Thus with the defaults of \LaTeX{}, |\theLem| would be
%    replaced by the command sequence |\thesection.\arabic{Lem}|.
%
%    We will now look at the definition of the macro which is executed
%    at the beginning of the actual environment (in our example this
%    macro is |\Lem|). It should be noted that we use an
%    ``|\expandafter| trick'' to expand only certain parts of the
%    replacement text at the time of the definition.
%    \begin{macrocode}
    \def\@tempa{\global\@namedef{#1}}%
    \expandafter \@tempa \expandafter{%
%    \end{macrocode}
%    First, the macro that contains the current definitions of
%    |\@begintheorem| and |\@opargtheorem| should be called
%    up.  The name of this macro---as is already known---has the form
%    |\th@|\meta{theorem\ style}; therefore, it must be called by
%    \begin{macrocode}
      \csname th@\the \theorem@style
%    \end{macrocode}
%    In addition the default theorem font should be changeable, i.e.\
%    we have to insert the contents of |\theorem@bodyfont|. For
%    that reason, we expand even further, beyond |\endcsname|,
%    and thus insert the contents of the \textsf{token} register
%    |\theorem@bodyfont| in the replacement text.
%    \begin{macrocode}
            \expandafter \endcsname \the \theorem@bodyfont
%    \end{macrocode}
%    Now it is time to call the macro |\@thm| which takes over
%    the further processing.  It has two arguments: the current
%    counter name (in our example, \texttt{Lem}), and the text of the
%    label (in our example, \texttt{Lemma}).
%    \begin{macrocode}
     \@thm{#1}{#2}}%
%    \end{macrocode}
%    With this, the `sub-definition' is complete.  The macro
%    |\@endtheorem| ends a theorem environment and is, so far, nothing
%    but an |\endtrivlist|.  (Hence it is defined globally, and not
%    within the theorem styles.\footnote{This has to be changed as
%    soon as theorem styles that change \texttt{\bslash @endtheorem}
%    exist.  In such a case, all existing styles must be changed as
%    well since they will have to reset the macro.}) Therefore, we can
%    set it equivalent to the macro that ends the theorem set (in our
%    example, |\endLem|).  However, if some day theorem styles exist
%    that do change |\@endtheorem|, we would have to use the
%    commented-out line instead.
%    \begin{macrocode}
    \global \expandafter \let \csname end#1\endcsname \@endtheorem
%   \global\@namedef{end#1}{\@endtheorem}%
%    \end{macrocode}
%    With these commands all the required definitions are employed,
%    unless the test |\@ifdefinable| has failed.  Therefore, we
%    end the second argument of this macro and with it the definition
%    of |\@xnthm|.
%    \begin{macrocode}
   }}
%    \end{macrocode}
% \end{macro}
%
% \begin{macro}{\@ynthm}
%    The definition of |\@ynthm| is completely analogous. In
%    this case the new counter that is defined is not reset within
%    another counter; thus the definition of |\the...|
%    is simplified:
%    \begin{macrocode}
\gdef\@ynthm#1#2{\expandafter\@ifdefinable\csname #1\endcsname
   {\@definecounter{#1}%
    \expandafter\xdef\csname the#1\endcsname{\@thmcounter{#1}}%
%    \end{macrocode}
%    The rest of the definition corresponds literally to that of
%    |\@xnthm|:
%    \begin{macrocode}
    \def\@tempa{\global\@namedef{#1}}\expandafter \@tempa
     \expandafter{\csname th@\the \theorem@style \expandafter
     \endcsname \the\theorem@bodyfont \@thm{#1}{#2}}%
    \global \expandafter \let \csname end#1\endcsname \@endtheorem}}
%    \end{macrocode}
% \end{macro}
%
% \begin{macro}{\@othm}
%    The definition of |\@othm| does not contain anything new.
%    \begin{macrocode}
\gdef\@othm#1[#2]#3{%
%    \end{macrocode}
% \changes{v2.2b}{95/10/07}{Build in check for counter defined}
%    We do not define a new counter but instead use one that has
%    already been defined.  Thus the only definition we need is that
%    of this pseudo-counter (i.e.\ |\the|\meta{env.\ name}).
%    First we check if |#2| corresponds to a known counter name.
%    \begin{macrocode}
  \expandafter\ifx\csname c@#2\endcsname\relax
   \@nocounterr{#2}%
  \else
   \expandafter\@ifdefinable\csname #1\endcsname
   {\expandafter \xdef \csname the#1\endcsname
     {\expandafter \noexpand \csname the#2\endcsname}%
%    \end{macrocode}
%    All other parts of the definition can be adopted from
%    |\@xnthm|. We have to remember, though, that in this case
%    the name of the current counter and the theorem label have moved
%    to the second and third arguments.
%    \begin{macrocode}
    \def\@tempa{\global\@namedef{#1}}\expandafter \@tempa
     \expandafter{\csname th@\the \theorem@style \expandafter
     \endcsname \the\theorem@bodyfont \@thm{#2}{#3}}%
    \global \expandafter \let \csname end#1\endcsname \@endtheorem}%
  \fi}
%    \end{macrocode}
% \end{macro}
%
% \subsection{Macros that are employed in a {\envfont theorem}
%    environment}
%
% \begin{macro}{\@thm}
%    The macro |\@thm| has to increase the current counter.  Then,
%    depending on whether the environment has (or does not have) an
%    optional argument, it has to branch into either |\@begintheorem|
%    or |\@opargtheorem|.
%    \begin{macrocode}
\gdef\@thm#1#2{\refstepcounter{#1}%
%    \end{macrocode}
%    Now we start a {\envfont trivlist} environment, and give
%    |\@topsep| and |\@topsepadd| the values of the \textsf{skip}
%    registers |\theorempreskipamount| and |\theorempostskipamount|.
%    The value in |\@topsep| is the vertical space that is inserted by
%    the first (and only) |\item| in our |\trivlist| whilst
%    |\@topsepadd| is inserted by |\@endparenv| at the end of that
%    {\envfont trivlist} environment.  By using these registers, we
%    obtain the desired space around a {\envfont theorem} environment.
%    \begin{macrocode}
   \trivlist
   \@topsep \theorempreskipamount               % used by first \item
   \@topsepadd \theorempostskipamount           % used by \@endparenv
%    \end{macrocode}
%    Now we have to test whether an optional argument has been given.
% \changes{v2.0d}{89/07/18}{Forgotten percent added.}
%    \begin{macrocode}
   \@ifnextchar [%
%    \end{macrocode}
%    If there is an optional argument, we will call |\@ythm|, and
%    move the arguments read back into the input stream.
%    \begin{macrocode}
   {\@ythm{#1}{#2}}%
%    \end{macrocode}
%    If not, we call |\@begintheorem|. Its first argument is the
%    name of the theorem set (hence the second argument of
%    |\@thm|). Its second argument is the macro that produces the
%    current number.
%    \begin{macrocode}
   {\@begintheorem{#2}{\csname the#1\endcsname}\ignorespaces}}
%    \end{macrocode}
% \end{macro}
%
% \begin{macro}{\@xthm}
% \begin{macro}{\@ythm}
%    Both these macros were originally called by |\@thm|.  We do
%    not need |\@xthm| anymore, hence we reset it to
%    |\relax|. The definition of |\@ythm| has not changed at
%    all from its definition in \LaTeX. In order to make the macros
%    easier to understand, we will
%    nevertheless present it (commented out).
%    \begin{macrocode}
\global\let\@xthm\relax
% \def\@ythm#1#2[#3]{\@opargbegintheorem{#2}{\csname
%           the#1\endcsname}{#3}\ignorespaces}
%</package>
%    \end{macrocode}
%    The primitive |\ignorespaces| in |\@ythm| and
%    |\@thm| is needed to remove the spaces between the
%    |\begin{...}| and the actual text.
% \end{macro}
% \end{macro}
%
%
% \subsection{Definition of the theorem substyles}
%
% As already pointed out, the theorem substyles, defined below, are
% only loaded when necessary.  Note that all these substyles, except
% \texttt{plain}, have |\slshape| as the default body font.
%
% \subsubsection{The \texttt{plain} style}
%
% \label{dok:th@plain} As the following macros use |@|, we have
% to locally set the |\catcode| of this symbol to ``letter''.
% This happens within a group, so that we do not have to worry about
% which |\catcode| that symbol had before.
%    \begin{macrocode}
%<*thp>
\begingroup \makeatletter
%    \end{macrocode}
% Since we are now within a group, we must make all definitions
% globally.  First we make sure that \texttt{theorem.sty} is loaded.
% This will allow us to use this file as a document style option
% without having to call \texttt{theorem} itself as an option.  At the
% same time, we assure that at least version 2 is loaded, since
% |\theorem@style| was not defined in earlier versions.
%    \begin{macrocode}
\@ifundefined{theorem@style}{\input{theorem.sty}}{}
\ProvidesFile{thp.sty}
             [\FMithmInfo]
%    \end{macrocode}
%
%
% \begin{macro}{\th@plain}
% \begin{macro}{\@begintheorem}
% \begin{macro}{\@opargbegintheorem}
%    |\theoremstyle{plain}| corresponds to the original
%    definition, except that the distances to the surrounding text are
%    determined by the parameters |\theorempreskipamount| and
%    |\theorempostskipamount|.
%    First we set the default body font.
%    \begin{macrocode}
\gdef\th@plain{\normalfont\itshape
%    \end{macrocode}
%    Then we define |\@begintheorem| and
%    |\@opargbegintheorem|. These two macros define how the
%    header of a theorem is typeset.  |\@opargbegintheorem| will
%    be called if a \textsf{theorem} environment with an optional
%    argument is encountered; otherwise, the header is constructed by
%    calling |\@begintheorem|.  If one of these macros is
%    executed, we are within a \textsf{trivlist} environment started by
%    |\@thm|.  So the theorem header is produced with an
%    |\item| command.
%
%    Instead of specifying the header font directly, all standard
%    theorem styles use the |\theorem@headerfont| macro to allow
%    customization.  The extra space (|\labelsep|) is necessary
%    because of problems in the \textsf{trivlist} environment.
%    \begin{macrocode}
  \def\@begintheorem##1##2{%
        \item[\hskip\labelsep \theorem@headerfont ##1\ ##2]}%
%    \end{macrocode}
%    The definition of |\@opargbegintheorem| is completely
%    analogous. The only difference is the fact that there exists a
%    third argument (which is the optional parameter of the
%    environment and contains additional information about the
%    theorem).  Customarily we enclose it in parentheses.
%    \begin{macrocode}
\def\@opargbegintheorem##1##2##3{%
   \item[\hskip\labelsep \theorem@headerfont ##1\ ##2\ (##3)]}}
%    \end{macrocode}
% \end{macro}
% \end{macro}
% \end{macro}
%
% \noindent
% We conclude with an |\endgroup| to restore the
% |\catcode| of |@|.
%    \begin{macrocode}
\endgroup
%</thp>
%    \end{macrocode}
%
%
% \subsubsection{The \texttt{break} style}
%
% This style option is stored in the file |thb.sty|.
% For the next two lines see the documentation for |\th@plain| on
% page~\pageref{dok:th@plain}.
%    \begin{macrocode}
%<*thb>
\begingroup \makeatletter
\@ifundefined{theorem@style}{\input{theorem.sty}}{}
\ProvidesFile{thb.sty}
             [\FMithmInfo]
%    \end{macrocode}
%
% \begin{macro}{\th@break}
%    |\theoremstyle{break}| produces a line break after the name
%    of the theorem.  The font is |\slshape|.  Hence, we define
%    |\th@break| as follows:
%    \begin{macrocode}
\gdef\th@break{\normalfont\slshape
  \def\@begintheorem##1##2{\item[%
%    \end{macrocode}
%    We run into the following problem: it is not possible to create
%    the header with |\item[|\meta{title}|]| and then start
%    a new line by, for example, |\mbox{}\\|. Such a definition
%    will fail whenever a list environment follows immediately.  With
%    the above construction, the |\mbox{}| causes the switch
%    |@inlabel| (cf.\ definition of |\list| and
%    |\trivlist| in \cite{latex:source}) to be set to \texttt{false}
%    and so the following list will insert additional vertical
%    space (|\topskip|). This is quite annoying. Therefore, we
%    create the line break within the |\item|. In order to ensure
%    that the text will begin at the proper position in the following
%    line, we simply pretend that the label does not take any
%    room.\footnote{This will lead to problems whenever very high
%    symbols occurring in the line tower into the
%    heading. So, something else has to be done here sometime.}
%    \begin{macrocode}
       \rlap{\vbox{\hbox{\hskip \labelsep\theorem@headerfont ##1\ ##2}%
                   \hbox{\strut}}}]}%
%    \end{macrocode}
%    Again, the definition of |\@opargbegintheorem| is
%    completely analogous.
%    \begin{macrocode}
\def\@opargbegintheorem##1##2##3{%
  \item[\rlap{\vbox{\hbox{\hskip \labelsep \theorem@headerfont
                     ##1\ ##2\ (##3)}%
                    \hbox{\strut}}}]}}
\endgroup
%</thb>
%    \end{macrocode}
% \end{macro}
%
%
% \subsubsection{The \texttt{changebreak} style}
%
%    \begin{macrocode}
% This style option is stored in the file |thcb.sty|.
%    \begin{macrocode}
%<*thcb>
\begingroup \makeatletter
\@ifundefined{theorem@style}{\input{theorem.sty}}{}
\ProvidesFile{thcb.sty}
             [\FMithmInfo]
%    \end{macrocode}
%
% \begin{macro}{\th@changebreak}
%    The \texttt{change-break} theorem style is like \texttt{break}
%    but with interchange of theorem name and theorem number.  Thus we
%    define |\th@changebreak| as follows:
%    \begin{macrocode}
\gdef\th@changebreak{\normalfont\slshape
  \def\@begintheorem##1##2{\item
       [\rlap{\vbox{\hbox{\hskip\labelsep \theorem@headerfont ##2\ ##1}%
                    \hbox{\strut}}}]}%
\def\@opargbegintheorem##1##2##3{%
  \item[\rlap{\vbox{\hbox{\hskip\labelsep \theorem@headerfont
                     ##2\ ##1\ (##3)}%
                    \hbox{\strut}}}]}}
\endgroup
%</thcb>
%    \end{macrocode}
% \end{macro}
%
%
% \subsubsection{The \texttt{change} style}
%
% This style option is stored in the file |thc.sty|.
%    \begin{macrocode}
%<*thc>
\begingroup \makeatletter
\@ifundefined{theorem@style}{\input{theorem.sty}}{}
\ProvidesFile{thc.sty}
             [\FMithmInfo]
%    \end{macrocode}
%
% \begin{macro}{\th@change}
%    The \texttt{change} theorem style corresponds to the
%    \texttt{change break} style without a linebreak after the header.
%    To say it in another way, it's the same as the \texttt{plain}
%    style but with number and name interchanged and |\slshape| as the
%    default font.
%    \begin{macrocode}
\gdef\th@change{\normalfont\slshape
  \def\@begintheorem##1##2{\item
       [\hskip\labelsep \theorem@headerfont ##2\ ##1]}%
\def\@opargbegintheorem##1##2##3{%
  \item[\hskip\labelsep \theorem@headerfont ##2\ ##1\ (##3)]}}
\endgroup
%</thc>
%    \end{macrocode}
% \end{macro}
%
%
% \subsubsection{The \texttt{marginbreak} style}
%
% This style option is the one used most often at Mainz.  It is saved
% in the file |thmb.sty|.
%    \begin{macrocode}
%<*thmb>
\begingroup \makeatletter
\@ifundefined{theorem@style}{\input{theorem.sty}}{}
\ProvidesFile{thmb.sty}
             [\FMithmInfo]
%    \end{macrocode}
%
% \begin{macro}{\th@marginbreak}
%    The \texttt{margin break} style is nearly the same as the
%    \texttt{change break} style. The only difference is the placement
%    of the theorem number.  We use |\llap| to place it in the left
%    margin.
%
%    In this style |\labelsep| denotes the separation between the
%    number and the text.
%    \begin{macrocode}
\gdef\th@marginbreak{\normalfont\slshape
  \def\@begintheorem##1##2{\item
       [\rlap{\vbox{\theorem@headerfont
                    \hbox{\llap{##2}\hskip\labelsep ##1}%
                    \hbox{\strut}}}]}%
\def\@opargbegintheorem##1##2##3{%
  \item[\rlap{\vbox{\theorem@headerfont
                    \hbox{\llap{##2}\hskip\labelsep ##1\ (##3)}%
                    \hbox{\strut}}}]}}
\endgroup
%</thmb>
%    \end{macrocode}
% \changes{v2.1c}{90/02/17}{`endgroup outside macrocode env.}
% \end{macro}
%
%
%
% \subsubsection{The \texttt{margin} style}
%
% This style option is stored in the file |thm.sty|.
%    \begin{macrocode}
%<*thm>
\begingroup \makeatletter
\@ifundefined{theorem@style}{\input{theorem.sty}}{}
\ProvidesFile{thm.sty}
             [\FMithmInfo]
%    \end{macrocode}
%
% \begin{macro}{\th@margin}
%    Again this is only a variant of the theorem styles described
%    above without any new ideas.
%    \begin{macrocode}
\gdef\th@margin{\normalfont\slshape
  \def\@begintheorem##1##2{\item
       [\theorem@headerfont \llap{##2}\hskip\labelsep ##1]}%
\def\@opargbegintheorem##1##2##3{%
  \item[\theorem@headerfont \llap{##2}\hskip\labelsep ##1\ (##3)]}}
\endgroup
%</thm>
%    \end{macrocode}
% \end{macro}
%
%
%
%
% \subsection{Final Definitions}
%
% \begin{macro}{\theorempreskipamount}
% \begin{macro}{\theorempostskipamount}
%    The \textsf{skip} parameters that regulate the vertical empty space
%    before and after the {\envfont theorem} environment have
%    to be allocated as well.
% \changes{v2.0f}{89/07/20}{Unnecessary `global' removed.}
%    \begin{macrocode}
%<*package>
\newskip\theorempreskipamount
\newskip\theorempostskipamount
%    \end{macrocode}
%    Since we have used the same values for all theorem sets, we now can
%    assign them.
%    \begin{macrocode}
\global\setlength\theorempreskipamount{12pt plus 5pt minus 3pt}
\global\setlength\theorempostskipamount{8pt plus 3pt minus 1.5pt}
%    \end{macrocode}
% \end{macro}
% \end{macro}
%
%
% \begin{macro}{\@endtheorem}
%    The same holds for the macro |\@endtheorem|, which ends a
%    {\envfont theorem} environment. Since it is the same for all
%    theorem sets, it is removed from the macros |\th@|\meta{style}.
%    It simply ends the {\envfont trivlist} environment, which was
%    begun in |\@thm|.
%    \begin{macrocode}
\global\let\@endtheorem=\endtrivlist
%    \end{macrocode}
% \end{macro}
%
%
%
% \begin{macro}{\@preamblecmds}
%    All macros defined above are to be used only in the preamble.
%    Therefore, we insert them in |\@preamblecmds| which will disable
%    them at begin document. This is done by the internal \LaTeXe{}
%    command |\@onlypreamble|.
% \changes{v2.0g}{89/09/19}{`theoremheaderfont and `theorembodyfont
%    added.}
% \changes{v2.2a}{94/02/03}{Use \LaTeXe{} interface}
%    \begin{macrocode}
\@onlypreamble\@xnthm
\@onlypreamble\@ynthm
\@onlypreamble\@othm
\@onlypreamble\newtheorem
\@onlypreamble\theoremstyle
\@onlypreamble\theorembodyfont
\@onlypreamble\theoremheaderfont
%    \end{macrocode}
% \end{macro}
%
% Finally we declare the \texttt{plain} theorem style to be the default.
% \changes{v2.1b}{90/06/04}{Fixed missing `item by declaring default
%    style.}
%    \begin{macrocode}
\theoremstyle{plain}
%</package>
%    \end{macrocode}
%
% \Finale

\endinput
}{}
\ProvidesFile{thc.sty}
             [\FMithmInfo]
%    \end{macrocode}
%
% \begin{macro}{\th@change}
%    The \texttt{change} theorem style corresponds to the
%    \texttt{change break} style without a linebreak after the header.
%    To say it in another way, it's the same as the \texttt{plain}
%    style but with number and name interchanged and |\slshape| as the
%    default font.
%    \begin{macrocode}
\gdef\th@change{\normalfont\slshape
  \def\@begintheorem##1##2{\item
       [\hskip\labelsep \theorem@headerfont ##2\ ##1]}%
\def\@opargbegintheorem##1##2##3{%
  \item[\hskip\labelsep \theorem@headerfont ##2\ ##1\ (##3)]}}
\endgroup
%</thc>
%    \end{macrocode}
% \end{macro}
%
%
% \subsubsection{The \texttt{marginbreak} style}
%
% This style option is the one used most often at Mainz.  It is saved
% in the file |thmb.sty|.
%    \begin{macrocode}
%<*thmb>
\begingroup \makeatletter
\@ifundefined{theorem@style}{% \iffalse meta-comment
%
% Copyright (C) 1993-2024
%
% The LaTeX Project and any individual authors listed elsewhere
% in this file.
%
% This file is part of the Standard LaTeX `Tools Bundle'.
% -------------------------------------------------------
%
% It may be distributed and/or modified under the
% conditions of the LaTeX Project Public License, either version 1.3c
% of this license or (at your option) any later version.
% The latest version of this license is in
%    https://www.latex-project.org/lppl.txt
% and version 1.3c or later is part of all distributions of LaTeX
% version 2005/12/01 or later.
%
% The list of all files belonging to the LaTeX `Tools Bundle' is
% given in the file `manifest.txt'.
%
% \fi
%
%%
%
%
\def\FMithmInfo{2023/07/05 v2.2c Theorem extension package (FMi)}
%
% \ProvidesFile{theorem.dtx}[\FMithmInfo]
%\iffalse   % this is a METACOMMENT !
%%
%% Package `theorem' to use with LaTeX2e
%% Copyright (C) 1989-1995 Frank Mittelbach, all rights reserved.
%<+package>\@ifundefined{theorem@style}{}{\endinput}
%<+package>\NeedsTeXFormat{LaTeX2e}[1995/06/01]
%<+package>\ProvidesPackage{theorem}
%<+package>         [\FMithmInfo]
%
% \fi
%
% \changes{v1.0}{87/05/07}{First draft.}
% \changes{v1.0a}{87/05/07}{`thistheoremstyle' added.}
% \changes{v1.1a}{87/07/29}{New concept: redefine the internal
%                          \protect\LaTeX{} commands `@xnthm' etc.}
% \changes{v1.1b}{87/07/31}{`theoremfont' used to specify font.}
% \changes{v1.2a}{88/01/08}{`@endtheorem' now globally defined.}
% \changes{v1.2b}{88/01/11}{`newtheorem' only allowed in the preamble.}
% \changes{v1.3a}{88/02/14}{`theoremfont' now a toks register.}
%
% \changes{v2.0a}{88/11/16}{New implementation and new documentation.}
% \changes{v2.0a}{88/11/16}{No layout changes but internal modifications
%                           old substyles won't work.}
% \changes{v2.0b}{89/01/30}{`global' used for all definitions.}
% \changes{v2.0c}{89/06/15}{Documentation translated to English.
%                            Thanks to LLL and Nina}
% \changes{v2.1a}{90/02/15}{Support for new font selection.}
% \changes{v2.2a}{94/02/03}{Cleanup for LaTeX2e}
%
% \DoNotIndex{\@endparenv,\@ifdefinable,\@ifnextchar,\@ifundefined}
% \DoNotIndex{\@namedef,\@nameuse,\@spaces,\@tempa,\@warning}
% \DoNotIndex{\ }
% \DoNotIndex{\begingroup,\bfseries}
% \DoNotIndex{\csname}
% \DoNotIndex{\def,\do,\docdate}
% \DoNotIndex{\endcsname,\endgroup,\endinput,\endtrivlist,\expandafter}
% \DoNotIndex{\filedate,\fileversion}
% \DoNotIndex{\gdef,\global}
% \DoNotIndex{\hbox,\hskip}
% \DoNotIndex{\ignorespaces,\input,\itshape,\item}
% \DoNotIndex{\labelsep,\let,\llap}
% \DoNotIndex{\makeatletter}
% \DoNotIndex{\noexpand}
% \DoNotIndex{\refstepcounter,\relax,\rlap}
% \DoNotIndex{\setlength,\slshape,\space,\string,\strut}
% \DoNotIndex{\the,\trivlist,\typeout}
% \DoNotIndex{\vbox}
% \DoNotIndex{\xdef}
%
% \def\deflabel#1{\ttfamily #1\hfill}
% \def\deflist#1{\begin{list}{}{\settowidth\labelwidth{\ttfamily #1}%
%                               \setlength\leftmargin\labelwidth
%                               \addtolength\leftmargin\labelsep
%                               \let\makelabel\deflabel}}
% \def\enddeflist{\end{list}}
% \def\envfont{\normalfont\ttfamily}
%
% \MakeShortVerb{\|}
%
% \newcommand{\TUB}{{\sl TUGboat\/}}
%
% \GetFileInfo{theorem.dtx}
% \title{An Extension of the
%        \LaTeX{} theorem environment\thanks{This file
%        has version number \fileversion, last
%        revised \filedate.}}
% \author{Frank Mittelbach}
%
% \MaintainedByLaTeXTeam{tools}
% \maketitle
%
% \begin{abstract}
%    The macros described in this paper yield an extension of the
%    \LaTeX{} theorem mechanism. It is designed to satisfy the
%    different requirements of various journals. Thus, the
%    layout of the ``theorems'' can be manipulated by determining a
%    ``style''. This article describes not only the use, but
%    also the definition, of the necessary macros.
% \end{abstract}
%
% \section*{Preface to version 2.2}
%
% For \LaTeXe{} this package did not need any fundamental changes. I
% only modified the messages generated so that theorem layout styles
% will show up with the |\listfiles| command and cleaned the section
% on the New Font Selection Scheme since this is now included in
% \LaTeX.
%
% \section*{Preface to version 2.1}
%
% This version is identical to 2.0g described in \TUB\ 10\#3 except
% for some internal defaults which are now set depending on the used
% font selection scheme.
%
% This was done to avoid unpleasant surprises if the new font selection
% scheme is in force. For further details see
% section~\ref{sec:newfonts} and \cite{bk:GMS94}.
%
% \section{Introduction}
%
% For our purposes here, ``theorems'' are labelled enunciations,
% often set off from the main text by extra space and a font change.
% Theorems, corollaries, conjectures, definitions, and remarks
% are all instances of ``theorems''.  The ``header'' of these structures
% is composed of a label (such as \textsc{Theorem} or \textsc{Remark})
% and a number which serializes an item in the sequence of items
% with the same label.
%
% Shortly after the introduction of \LaTeX{} at the Fachbereich
% Mathematik in Mainz, the desire to manipulate the layout of
% ``theorems'' arose.  In Mainz, the following two conventions came
% into general use:
% \begin{enumerate}
%   \item The number of the theorem is shown in the margin.
%   \item There is a line break at
%         the end of the theorem header.
% \end{enumerate}
%
% Additionally, some journals require different formats which depend on
% the ``sort of theorem'': e.g.\ often remarks and definitions are set
% in |\upshape|, while |\itshape| is employed for main theorems.
%
% Confronted with these requirements, a theorem environment was
% developed in Mainz which allows separate determination of the
% layout of the ``theorems sets'', comparable to |\pagestyle|.
%
%
% \section{The user interface}
% \subsection{Defining new theorem sets}
%
% \DescribeMacro\newtheorem
% As in the original \LaTeX{} version, the command |\newtheorem|
% defines a new ``theorem set'' or ``theorem-like structure''.
% Two required arguments name the new environment and give the text
% to be typeset with each instance of the new ``set'', while
% an optional argument determines how
% the ``set'' is enumerated:
% \begin{description}
%    \item[\ttfamily \bslash newtheorem\{foo\}\{bar\}]
%       The theorem set {\envfont foo} (whose name is \texttt{bar})
%       uses its own counter.
%    \item[\ttfamily \bslash newtheorem\{foo2\}{[foo]}\{bar2\}]
%       The theorem set {\envfont foo2} (printed name \texttt{bar2})
%       uses the same counter as the theorem set \texttt{foo}.
%    \item[\ttfamily \bslash newtheorem\{foo3\}\{bar3\}{[section]}]
%       The theorem set {\envfont foo3} (printed name \texttt{bar3}) is
%       enumerated within the counter \texttt{section}, i.e.\ with every
%       new |\section| the enumeration begins again with 1, and
%       the enumeration is composed from the section-number and the
%       theorem counter itself.
% \end{description}
%
% \DescribeMacro\theoremstyle
% Additionally, the command |\theoremstyle| can define the layout
% of various, or all, theorem sets.  It should be noted that any
% theorem set defined by |\newtheorem| is typeset in the
% |\theoremstyle| that is current at the time of the definition.
% Thus, the following
% \begin{verbatim}
%    \theoremstyle{break}        \newtheorem{Cor}{Corollary}
%    \theoremstyle{plain}        \newtheorem{Exa}{Example}[section]
%\end{verbatim}                             ^^A No blank at line start
% leads to the result that the set {\envfont Cor} is formatted in the
% style \texttt{break}, while the set {\envfont Exa} and all the
% following ones are formatted in the style \texttt{plain}, unless
% another |\theoremstyle| follows.  Since the definitions installed by
% |\newtheorem| are global, one also can limit |\theoremstyle| locally
% by grouping braces.
%
% \DescribeMacro\theorembodyfont
% The choice of the font for the theorem body is completely
% independent of the chosen |\theoremstyle|; this has proven to
% be very advantageous.  For example,
% \begin{verbatim}
%   {\theorembodyfont{\upshape}          \newtheorem{Rem}{Remark}}
%\end{verbatim}
% defines a theorem set {\envfont Rem}, which will be set in
% |\upshape| in the current layout (which in our example is
% \texttt{plain}).  As with |\theoremstyle|, the |\theorembodyfont|
% chosen is that current at the time of |\newtheorem|.  If
% |\theorembodyfont| is not specified or one defines
% |\theorembodyfont{}|, then the font used will be that defined by the
% |\theoremstyle|.
%
% \DescribeMacro\theoremheaderfont It is also possible to customize
% the font used for the theorem headers.  This is, however, a global
% declaration, and therefore there should be at most one
% |\theoremheaderfont| declaration in the preamble.\footnote{If it is
% actually necessary to have different header fonts, one has to define
% new theorem styles (substituting the desired font) or specify the
% information directly in the \texttt{\bslash newtheorem} declaration
% (the unclean variant).}
%
% \DescribeMacro\theorempreskipamount
% \DescribeMacro\theorempostskipamount Two additional parameters
% affect the vertical space around the theorem environments:
% |\theorempreskipamount| and |\theorempostskipamount| define,
% respectively, the spacing before and after such an environment.
% These parameters apply for all theorem sets and can be manipulated
% with the ordinary length macros.  They are rubber lengths
% (`\textsf{skips}'), and therefore can contain \texttt{plus} and
% \texttt{minus} parts.
%
% Since the definition of theorem sets should---most sensibly---be
% placed in the preamble, we only allow installation there.
% It is therefore possible to release the memory used here after
% |\begin{document}|, in order to make room for other
% applications.
%
%
% \subsection{Existing theorem styles}
%
% The following theorem styles exist to date:
% \begin{deflist}{marginbreak}
%    \item[plain]
%       This theorem style emulates the original \LaTeX{} definition,
%       except that additionally the parameters
%       |\theorem...skipamount| are used.
%    \item[break]
%       In this style, the theorem header is followed by a line
%       break.
%    \item[marginbreak]
%       The theorem number is set in the margin, and there is a line
%       break as in \texttt{break}.
%    \item[changebreak]
%       Like \texttt{break}, but with header number and text
%       interchanged.
%    \item[change]
%       Header number and text are interchanged, without a line break.
%    \item[margin]
%       The number is set in the left margin, without a line break.
% \end{deflist}
% All styles (except \texttt{plain}) select |\slshape| as the default
% |\theorembodyfont|.
%
%
% \subsection{Examples}
%
% Given the above theorem sets {\envfont Cor}, {\envfont Exa}
% and {\envfont Rem},
% suppose that the preamble also contains the declarations:
% \begin{verbatim}
%    \theoremstyle{marginbreak}   \newtheorem{Lem}[Cor]{Lemma}
%    \theoremstyle{change}
%    \theorembodyfont{\itshape}        \newtheorem{Def}[Cor]{Definition}
%
%    \theoremheaderfont{\scshape}
%\end{verbatim}
% Then the following are some typical examples of the typeset output
% resulting from their use.
%
% \begin{Cor}
%    This is a sentence typeset in the theorem environment {\envfont
%    Cor}.
% \end{Cor}
%
% \begin{Exa}
%    This is a sentence typeset in the theorem environment {\envfont
%    Exa}.
% \end{Exa}
%
% \begin{Rem}
%    This is a sentence typeset in the theorem environment {\envfont
%    Rem}.
% \end{Rem}
%
% \begin{Lem}[Ben User]
%    This is a sentence typeset in the theorem environment {\envfont
%    Lem}.
% \end{Lem}
%
% \begin{Def}[Very impressive Definition]
%    This is a sentence typeset in the theorem environment {\envfont
%    Def}.
% \end{Def}
%
% The last two examples show the effect of the optional argument to
% a theorem environment (it is the text typeset in parentheses).
%
%
% \section{Special Considerations}
% \label{sec:newfonts}
%
% Theoremheader and body are implemented as a unit. This means that
% the |\theoremheaderfont|  will inherit characteristics of the
% |\theorembodyfont| in \LaTeXe. Thus, if for example
% |\theorembodyfont| is |\itshape| and |\theoremheaderfont|
% is |\bfseries| the font selected for the header will have the
% characteristics `bold extended italic'. If this is not desired one
% should set the |\theoremheaderfont| to something like
% \begin{verbatim}
%      \theoremheaderfont{\normalfont\bfseries}
%\end{verbatim}
% i.e.\ supplying all necessary font information explicitly.
%
%
%
% \section{Acknowledgements}
%
% The publication of this set of macros was only possible with the
% help of Christina Busse (translating the manuscript into English),
% Joachim Pense (playing the r\^ole of typist), Chris Rowley
% (looking everything over) and many others providing useful
% suggestions.
%
% \MaybeStop{%
%    \begin{thebibliography}{9}
%       \bibitem{bk:GMS94} \textsc{M.~Goossens}, \textsc{F.~Mittelbach}
%         and \textsc{A.~Samarin}.
%         \newblock The \LaTeX{} Companion.
%         \newblock
%         Addison-Wesley, Reading, Massachusetts, 1994.
%       \bibitem{latex:source} \textsc{Lamport, Leslie.} \newblock
%                               \texttt{latex.tex}, version 2.09, date
%                               Feb. 1990.
%    \end{thebibliography}
%    \PrintIndex
%    \PrintChanges}
%
%
% \section{The documentation driver file}
%
% The next bit of code contains the documentation driver file for
% \TeX{}, i.e., the file that will produce the documentation you are
% currently reading. It will be extracted from this file by the
% \texttt{docstrip} program. Since it is the first code in the file
% one can alternatively process this file directly with \LaTeXe{} to
% obtain the documentation.
%
%    \begin{macrocode}
%<*driver>
\documentclass{ltxdoc}

\usepackage{theorem}

%<+driver>% The next few lines define theorem sets which are used
%<+driver>% in the example section of the documentation.

\theoremstyle{break}       \newtheorem{Cor}{Corollary}
\theoremstyle{plain}       \newtheorem{Exa}{Example}[section]
{\theorembodyfont{\upshape}\newtheorem{Rem}{Remark}}
\theoremstyle{marginbreak} \newtheorem{Lem}[Cor]{Lemma}
\theoremstyle{change}
\theorembodyfont{\itshape} \newtheorem{Def}[Cor]{Definition}

\theoremheaderfont{\scshape}

\RecordChanges

\begin{document}
   \DocInput{theorem.dtx}
\end{document}
%</driver>
%    \end{macrocode}
%
% \section{Definition of the Macros}
%
% If the file has been loaded before, we abort immediately. If not the
% package announces itself (this is actually done at the very top of
% the file---the way it is done isn't good style so don't copy it).
% \changes{v2.0e}{89/07/19}{Spaces removed from `typeout'.}
% \changes{v2.2b}{95/11/19}{Announce moved to top of file}
%    \begin{macrocode}
%<*package>
%\@ifundefined{theorem@style}{}{\endinput}
%\def\FMithmInfo{1995/11/19 v2.2b Theorem extension package (FMi)}
%\ProvidesPackage{theorem}[\FMithmInfo]
%    \end{macrocode}
%
%
% \subsection{Definition of theorem styles and fonts}
%
% All the definitions in this file are done globally to allow
% inputting this file inside a group.
%
% \begin{macro}{\theoremstyle}
%    Before a theorem style can be installed, the chosen style must be
%    known. For that reason, we must test to see that
%    |\th@|\meta{style} is known or, more precisely, that it is
%    different from |\relax|. If the style is not known then
%    |\th@plain| is used.
%    \begin{macrocode}
\gdef\theoremstyle#1{%
   \@ifundefined{th@#1}{\@warning
          {Unknown theoremstyle `#1'. Using `plain'}%
          \theorem@style{plain}}%
%    \end{macrocode}
%    We save the theorem style to be used in the token register
%    |\theorem@style|.
%    \begin{macrocode}
      {\theorem@style{#1}}%
%    \end{macrocode}
%    Now we ``evaluate'' the theorem style: this means, we call the
%    macro |\th@|\meta{style} which will activate the relevant
%    definitions which are contained in a separate file.  This is done
%    in a group to suppress changes to the current font.  This could
%    otherwise pose problems together with the new font selection
%    scheme\footnote{When I printed the original article using the new
%    font selection scheme I ended with a document with slanted
%    typefaces (text headings and all) simply because one of the
%    theorem styles used \texttt{\bslash sl} at toplevel.} if the
%    |\th@|\meta{style} is evaluated a second time.
% \changes{v2.1a}{90/02/15}{Group added around `th@..}
% \changes{v2.1a}{90/02/15}{Reseting of `..@begintheorem removed.}
%    \begin{macrocode}
      \begingroup
        \csname th@\the\theorem@style \endcsname
      \endgroup}
%    \end{macrocode}
% \end{macro}
%
%
% \begin{macro}{\@begintheorem}
% \changes{v2.1a}{90/02/15}{Relaxed at toplevel.}
% \begin{macro}{\@opargbegintheorem}
% \changes{v2.1a}{90/02/15}{Relaxed at toplevel.}
%    We reset |\@begintheorem| and |\@opargbegintheorem| to
%    |\relax| since these commands are no longer necessary at
%    toplevel. This will save a few tokens.
%    \begin{macrocode}
\global\let\@begintheorem\relax
\global\let\@opargbegintheorem\relax
%    \end{macrocode}
% \end{macro}
% \end{macro}
%
% \begin{macro}{\theorem@style}
%    Obviously the \textsf{token} register used above has to be
%    allocated. To assure the utmost compatibility with the
%    original \LaTeX{} definition, we set the default theorem style to
%    \texttt{plain}, which implements the usual \LaTeX{} convention.
% \changes{v2.0f}{89/07/20}{Unnecessary `global' removed.}
%    \begin{macrocode}
\newtoks\theorem@style
\global\theorem@style{plain}
%    \end{macrocode}
% \end{macro}
%
% \begin{macro}{\theorembodyfont}
% \begin{macro}{\theorem@bodyfont}
%    If the |\theorembodyfont| is set by the user then it should not
%    interact with the default font set in the theorem style. When the
%    new font selection is in force this may happen if, for example,
%    the default is |\itshape| and the new |\theorembodyfont| is
%    |\sffamily|. So we add a |\reset@font| command in front of the
%    user definition.
% \changes{v2.0f}{89/07/20}{Unnecessary `global' removed.}
% \changes{v2.0g}{89/09/19}{`theoremfont to `theorembodyfont.}
% \changes{v2.1c}{92/06/22}{`reset@font added and logic changed.}
% \changes{v2.1d}{92/10/28}{Reset to empty if arg is empty}
%    \begin{macrocode}
\gdef\theorembodyfont#1{%
%    \end{macrocode}
%    We check if the argument supplied is empty and if so put nothing
%    into the |\theorem@bodyfont| token register to allow for
%    |\theorembodyfont{}| as a mean of using the default of the
%    current |\theoremstyle|.
%    \begin{macrocode}
   \def\@tempa{#1}%
   \ifx\@tempa\@empty
    \theorem@bodyfont{}%
   \else
    \theorem@bodyfont{\reset@font#1}%
   \fi
   }
\newtoks\theorem@bodyfont
\global\theorem@bodyfont{}
%    \end{macrocode}
% \end{macro}
% \end{macro}
%
%
% \begin{macro}{\theoremheaderfont}
% \changes{v2.0g}{89/09/19}{Macro added.}
%    The font for the theorem headers is handled differently because
%    this definition applies to all theorem styles.
%    \begin{macrocode}
\gdef\theoremheaderfont#1{\gdef\theorem@headerfont{#1}%
%    \end{macrocode}
%    After using the macro once it is redefined to produce an error
%    message.
%    \begin{macrocode}
       \gdef\theoremheaderfont##1{%
        \typeout{\string\theoremheaderfont\space should be used
                 only once.}}}
%    \end{macrocode}
% \end{macro}
%
% \begin{macro}{\theorem@headerfont}
% \changes{v2.0g}{89/09/19}{Macro added.}
%    To set the |\theorem@headerfont| default we first test if
%    the new fontselection scheme is in force.
% \changes{v2.1a}{90/02/15}{Support for new font selection.}
%    \begin{macrocode}
\ifx\upshape\undefined
%    \end{macrocode}
%    If not we define it to expand into |\bfseries|. We don't use
%    |\let| just in case a following style option redefines this
%    macro.
%    \begin{macrocode}
\gdef\theorem@headerfont{\bfseries}
%    \end{macrocode}
%    Otherwise we reset the current shape before calling |\bfseries|.
%    \begin{macrocode}
\else \gdef\theorem@headerfont{\normalfont\bfseries}\fi
%    \end{macrocode}
% \end{macro}
%
% \begin{macro}{\th@plain}
% \begin{macro}{\th@break}
% \begin{macro}{\th@marginbreak}
% \begin{macro}{\th@changebreak}
% \begin{macro}{\th@change}
% \begin{macro}{\th@margin}
%    The different styles are defined in macros such as
%    |\th@plain|.  Since memory space is precious in
%    ``non-Big-versions'', we have to avoid offering too many unused
%    definitions.  Therefore we define these styles in separate files
%    that can be loaded on demand.  Thus the commands themselves only
%    load these files. We use |\@input@|, a \LaTeXe{} internal command
%    that ensures that the file will be listed with |\listfiles|.
% \changes{v2.2a}{94/02/02}{Use `@input@ to load theorem layout files}
%    \begin{macrocode}
\gdef\th@plain{\@input@{thp.sty}}
\gdef\th@break{\@input@{thb.sty}}
\gdef\th@marginbreak{\@input@{thmb.sty}}
\gdef\th@changebreak{\@input@{thcb.sty}}
\gdef\th@change{\@input@{thc.sty}}
\gdef\th@margin{\@input@{thm.sty}}
%    \end{macrocode}
%    This list will be expanded when new styles become
%    available.  For testing,  just append new theorem substyles
%    as document options.
% \end{macro}
% \end{macro}
% \end{macro}
% \end{macro}
% \end{macro}
% \end{macro}
%
%
%
% \subsection{Definition of a new theorem set}
%
% As already pointed out, a new theorem environment can be defined in
% three different ways:
% \begin{verbatim}
%    \newtheorem{Lem}{Lemma}
%    \newtheorem{Lem}{Lemma}[section]
%    \newtheorem{Lem}[Theorem]{Lemma}
%\end{verbatim}
% The function of the macro |\newtheorem| is to recognize these cases
% and then to branch into one of the three macros |\@ynthm|, |\@xnthm|
% or |\@othm|.  This mechanism is adopted unchanged from
% \cite{latex:source}; the essential point here is that, for example,
% in the second case, the arguments \texttt{Lem}, \texttt{Lemma} and
% \texttt{section} are passed over to the macro |\@xnthm|.
%
% We inspect this case first because the others present fewer problems,
% and thus are easily derived from this one.
%
% \begin{macro}{\@xnthm}
%    For our example arguments, the macro |\@xnthm| must fulfill
%    the following:
%    \begin{itemize}
%       \item
%          Define a new \LaTeX-counter `Lem'
%       \item
%          reset this counter within a |\section|
%       \item
%          define the macro |\theLem|
%       \item
%          define the environment macros |\Lem| and
%          |\endLem| using the current
%          |\theoremstyle| and |\theorem@bodyfont|.
%    \end{itemize}
%    Obviously, all this should happen only if the first argument
%    of |\@xnthm| (i.e.\ \texttt{Lem} in our example) is chosen so
%    as not to conflict with any previously defined commands or
%    environments.  This test is performed by the \LaTeX{} macro
%    |\@ifdefinable|.
%    \begin{macrocode}
\gdef\@xnthm#1#2[#3]{\expandafter\@ifdefinable\csname #1\endcsname
%    \end{macrocode}
%    Therefore, the first argument of |\@ifdefinable| is the
%    expansion (in the example, |\Lem|) of
%    |\csname||#1\endcsname|.  The second argument is
%    executed only if the test has been completed successfully.
%    \begin{macrocode}
   {%
%    \end{macrocode}
%    Now we define the new counter. The names of the \LaTeX{}
%    macros employed should speak for themselves:
% \changes{v2.2b}{1995/10/07}{Use \cs{@newctr}}
% \changes{v2.2c}{1995/11/23}{Use \cs{@newctr} properly.}
%    \begin{macrocode}
    \@definecounter{#1}\@newctr{#1}[#3]%
%    \end{macrocode}
%    Using |\@newctr| will give a proper error message if the counter
%    in |#3| is not defined.
%    In defining `|\theLem|' we must generate the desired macro
%    name by use of |\expandafter| and |\csname|.
%    \begin{macrocode}
    \expandafter\xdef\csname the#1\endcsname
%    \end{macrocode}
%    An |\xdef| is used in order to make the definition global,
%    and to ensure that it contains the replacement texts of
%    |\@thmcountersep| and |\@thmcounter|.\footnote{These
%    two macros can be defined by the document style.  Their default
%    values produce a `\texttt{.}' as separation and an arabic
%    representation of the number.} However, not everything should be
%    expanded. For example, it saves space to use |\thesection|
%    instead of its---at times---lengthy expansion.
%    \begin{macrocode}
      {\expandafter \noexpand \csname the#3\endcsname
       \@thmcountersep \@thmcounter{#1}}%
%    \end{macrocode}
%    Thus with the defaults of \LaTeX{}, |\theLem| would be
%    replaced by the command sequence |\thesection.\arabic{Lem}|.
%
%    We will now look at the definition of the macro which is executed
%    at the beginning of the actual environment (in our example this
%    macro is |\Lem|). It should be noted that we use an
%    ``|\expandafter| trick'' to expand only certain parts of the
%    replacement text at the time of the definition.
%    \begin{macrocode}
    \def\@tempa{\global\@namedef{#1}}%
    \expandafter \@tempa \expandafter{%
%    \end{macrocode}
%    First, the macro that contains the current definitions of
%    |\@begintheorem| and |\@opargtheorem| should be called
%    up.  The name of this macro---as is already known---has the form
%    |\th@|\meta{theorem\ style}; therefore, it must be called by
%    \begin{macrocode}
      \csname th@\the \theorem@style
%    \end{macrocode}
%    In addition the default theorem font should be changeable, i.e.\
%    we have to insert the contents of |\theorem@bodyfont|. For
%    that reason, we expand even further, beyond |\endcsname|,
%    and thus insert the contents of the \textsf{token} register
%    |\theorem@bodyfont| in the replacement text.
%    \begin{macrocode}
            \expandafter \endcsname \the \theorem@bodyfont
%    \end{macrocode}
%    Now it is time to call the macro |\@thm| which takes over
%    the further processing.  It has two arguments: the current
%    counter name (in our example, \texttt{Lem}), and the text of the
%    label (in our example, \texttt{Lemma}).
%    \begin{macrocode}
     \@thm{#1}{#2}}%
%    \end{macrocode}
%    With this, the `sub-definition' is complete.  The macro
%    |\@endtheorem| ends a theorem environment and is, so far, nothing
%    but an |\endtrivlist|.  (Hence it is defined globally, and not
%    within the theorem styles.\footnote{This has to be changed as
%    soon as theorem styles that change \texttt{\bslash @endtheorem}
%    exist.  In such a case, all existing styles must be changed as
%    well since they will have to reset the macro.}) Therefore, we can
%    set it equivalent to the macro that ends the theorem set (in our
%    example, |\endLem|).  However, if some day theorem styles exist
%    that do change |\@endtheorem|, we would have to use the
%    commented-out line instead.
%    \begin{macrocode}
    \global \expandafter \let \csname end#1\endcsname \@endtheorem
%   \global\@namedef{end#1}{\@endtheorem}%
%    \end{macrocode}
%    With these commands all the required definitions are employed,
%    unless the test |\@ifdefinable| has failed.  Therefore, we
%    end the second argument of this macro and with it the definition
%    of |\@xnthm|.
%    \begin{macrocode}
   }}
%    \end{macrocode}
% \end{macro}
%
% \begin{macro}{\@ynthm}
%    The definition of |\@ynthm| is completely analogous. In
%    this case the new counter that is defined is not reset within
%    another counter; thus the definition of |\the...|
%    is simplified:
%    \begin{macrocode}
\gdef\@ynthm#1#2{\expandafter\@ifdefinable\csname #1\endcsname
   {\@definecounter{#1}%
    \expandafter\xdef\csname the#1\endcsname{\@thmcounter{#1}}%
%    \end{macrocode}
%    The rest of the definition corresponds literally to that of
%    |\@xnthm|:
%    \begin{macrocode}
    \def\@tempa{\global\@namedef{#1}}\expandafter \@tempa
     \expandafter{\csname th@\the \theorem@style \expandafter
     \endcsname \the\theorem@bodyfont \@thm{#1}{#2}}%
    \global \expandafter \let \csname end#1\endcsname \@endtheorem}}
%    \end{macrocode}
% \end{macro}
%
% \begin{macro}{\@othm}
%    The definition of |\@othm| does not contain anything new.
%    \begin{macrocode}
\gdef\@othm#1[#2]#3{%
%    \end{macrocode}
% \changes{v2.2b}{95/10/07}{Build in check for counter defined}
%    We do not define a new counter but instead use one that has
%    already been defined.  Thus the only definition we need is that
%    of this pseudo-counter (i.e.\ |\the|\meta{env.\ name}).
%    First we check if |#2| corresponds to a known counter name.
%    \begin{macrocode}
  \expandafter\ifx\csname c@#2\endcsname\relax
   \@nocounterr{#2}%
  \else
   \expandafter\@ifdefinable\csname #1\endcsname
   {\expandafter \xdef \csname the#1\endcsname
     {\expandafter \noexpand \csname the#2\endcsname}%
%    \end{macrocode}
%    All other parts of the definition can be adopted from
%    |\@xnthm|. We have to remember, though, that in this case
%    the name of the current counter and the theorem label have moved
%    to the second and third arguments.
%    \begin{macrocode}
    \def\@tempa{\global\@namedef{#1}}\expandafter \@tempa
     \expandafter{\csname th@\the \theorem@style \expandafter
     \endcsname \the\theorem@bodyfont \@thm{#2}{#3}}%
    \global \expandafter \let \csname end#1\endcsname \@endtheorem}%
  \fi}
%    \end{macrocode}
% \end{macro}
%
% \subsection{Macros that are employed in a {\envfont theorem}
%    environment}
%
% \begin{macro}{\@thm}
%    The macro |\@thm| has to increase the current counter.  Then,
%    depending on whether the environment has (or does not have) an
%    optional argument, it has to branch into either |\@begintheorem|
%    or |\@opargtheorem|.
%    \begin{macrocode}
\gdef\@thm#1#2{\refstepcounter{#1}%
%    \end{macrocode}
%    Now we start a {\envfont trivlist} environment, and give
%    |\@topsep| and |\@topsepadd| the values of the \textsf{skip}
%    registers |\theorempreskipamount| and |\theorempostskipamount|.
%    The value in |\@topsep| is the vertical space that is inserted by
%    the first (and only) |\item| in our |\trivlist| whilst
%    |\@topsepadd| is inserted by |\@endparenv| at the end of that
%    {\envfont trivlist} environment.  By using these registers, we
%    obtain the desired space around a {\envfont theorem} environment.
%    \begin{macrocode}
   \trivlist
   \@topsep \theorempreskipamount               % used by first \item
   \@topsepadd \theorempostskipamount           % used by \@endparenv
%    \end{macrocode}
%    Now we have to test whether an optional argument has been given.
% \changes{v2.0d}{89/07/18}{Forgotten percent added.}
%    \begin{macrocode}
   \@ifnextchar [%
%    \end{macrocode}
%    If there is an optional argument, we will call |\@ythm|, and
%    move the arguments read back into the input stream.
%    \begin{macrocode}
   {\@ythm{#1}{#2}}%
%    \end{macrocode}
%    If not, we call |\@begintheorem|. Its first argument is the
%    name of the theorem set (hence the second argument of
%    |\@thm|). Its second argument is the macro that produces the
%    current number.
%    \begin{macrocode}
   {\@begintheorem{#2}{\csname the#1\endcsname}\ignorespaces}}
%    \end{macrocode}
% \end{macro}
%
% \begin{macro}{\@xthm}
% \begin{macro}{\@ythm}
%    Both these macros were originally called by |\@thm|.  We do
%    not need |\@xthm| anymore, hence we reset it to
%    |\relax|. The definition of |\@ythm| has not changed at
%    all from its definition in \LaTeX. In order to make the macros
%    easier to understand, we will
%    nevertheless present it (commented out).
%    \begin{macrocode}
\global\let\@xthm\relax
% \def\@ythm#1#2[#3]{\@opargbegintheorem{#2}{\csname
%           the#1\endcsname}{#3}\ignorespaces}
%</package>
%    \end{macrocode}
%    The primitive |\ignorespaces| in |\@ythm| and
%    |\@thm| is needed to remove the spaces between the
%    |\begin{...}| and the actual text.
% \end{macro}
% \end{macro}
%
%
% \subsection{Definition of the theorem substyles}
%
% As already pointed out, the theorem substyles, defined below, are
% only loaded when necessary.  Note that all these substyles, except
% \texttt{plain}, have |\slshape| as the default body font.
%
% \subsubsection{The \texttt{plain} style}
%
% \label{dok:th@plain} As the following macros use |@|, we have
% to locally set the |\catcode| of this symbol to ``letter''.
% This happens within a group, so that we do not have to worry about
% which |\catcode| that symbol had before.
%    \begin{macrocode}
%<*thp>
\begingroup \makeatletter
%    \end{macrocode}
% Since we are now within a group, we must make all definitions
% globally.  First we make sure that \texttt{theorem.sty} is loaded.
% This will allow us to use this file as a document style option
% without having to call \texttt{theorem} itself as an option.  At the
% same time, we assure that at least version 2 is loaded, since
% |\theorem@style| was not defined in earlier versions.
%    \begin{macrocode}
\@ifundefined{theorem@style}{\input{theorem.sty}}{}
\ProvidesFile{thp.sty}
             [\FMithmInfo]
%    \end{macrocode}
%
%
% \begin{macro}{\th@plain}
% \begin{macro}{\@begintheorem}
% \begin{macro}{\@opargbegintheorem}
%    |\theoremstyle{plain}| corresponds to the original
%    definition, except that the distances to the surrounding text are
%    determined by the parameters |\theorempreskipamount| and
%    |\theorempostskipamount|.
%    First we set the default body font.
%    \begin{macrocode}
\gdef\th@plain{\normalfont\itshape
%    \end{macrocode}
%    Then we define |\@begintheorem| and
%    |\@opargbegintheorem|. These two macros define how the
%    header of a theorem is typeset.  |\@opargbegintheorem| will
%    be called if a \textsf{theorem} environment with an optional
%    argument is encountered; otherwise, the header is constructed by
%    calling |\@begintheorem|.  If one of these macros is
%    executed, we are within a \textsf{trivlist} environment started by
%    |\@thm|.  So the theorem header is produced with an
%    |\item| command.
%
%    Instead of specifying the header font directly, all standard
%    theorem styles use the |\theorem@headerfont| macro to allow
%    customization.  The extra space (|\labelsep|) is necessary
%    because of problems in the \textsf{trivlist} environment.
%    \begin{macrocode}
  \def\@begintheorem##1##2{%
        \item[\hskip\labelsep \theorem@headerfont ##1\ ##2]}%
%    \end{macrocode}
%    The definition of |\@opargbegintheorem| is completely
%    analogous. The only difference is the fact that there exists a
%    third argument (which is the optional parameter of the
%    environment and contains additional information about the
%    theorem).  Customarily we enclose it in parentheses.
%    \begin{macrocode}
\def\@opargbegintheorem##1##2##3{%
   \item[\hskip\labelsep \theorem@headerfont ##1\ ##2\ (##3)]}}
%    \end{macrocode}
% \end{macro}
% \end{macro}
% \end{macro}
%
% \noindent
% We conclude with an |\endgroup| to restore the
% |\catcode| of |@|.
%    \begin{macrocode}
\endgroup
%</thp>
%    \end{macrocode}
%
%
% \subsubsection{The \texttt{break} style}
%
% This style option is stored in the file |thb.sty|.
% For the next two lines see the documentation for |\th@plain| on
% page~\pageref{dok:th@plain}.
%    \begin{macrocode}
%<*thb>
\begingroup \makeatletter
\@ifundefined{theorem@style}{\input{theorem.sty}}{}
\ProvidesFile{thb.sty}
             [\FMithmInfo]
%    \end{macrocode}
%
% \begin{macro}{\th@break}
%    |\theoremstyle{break}| produces a line break after the name
%    of the theorem.  The font is |\slshape|.  Hence, we define
%    |\th@break| as follows:
%    \begin{macrocode}
\gdef\th@break{\normalfont\slshape
  \def\@begintheorem##1##2{\item[%
%    \end{macrocode}
%    We run into the following problem: it is not possible to create
%    the header with |\item[|\meta{title}|]| and then start
%    a new line by, for example, |\mbox{}\\|. Such a definition
%    will fail whenever a list environment follows immediately.  With
%    the above construction, the |\mbox{}| causes the switch
%    |@inlabel| (cf.\ definition of |\list| and
%    |\trivlist| in \cite{latex:source}) to be set to \texttt{false}
%    and so the following list will insert additional vertical
%    space (|\topskip|). This is quite annoying. Therefore, we
%    create the line break within the |\item|. In order to ensure
%    that the text will begin at the proper position in the following
%    line, we simply pretend that the label does not take any
%    room.\footnote{This will lead to problems whenever very high
%    symbols occurring in the line tower into the
%    heading. So, something else has to be done here sometime.}
%    \begin{macrocode}
       \rlap{\vbox{\hbox{\hskip \labelsep\theorem@headerfont ##1\ ##2}%
                   \hbox{\strut}}}]}%
%    \end{macrocode}
%    Again, the definition of |\@opargbegintheorem| is
%    completely analogous.
%    \begin{macrocode}
\def\@opargbegintheorem##1##2##3{%
  \item[\rlap{\vbox{\hbox{\hskip \labelsep \theorem@headerfont
                     ##1\ ##2\ (##3)}%
                    \hbox{\strut}}}]}}
\endgroup
%</thb>
%    \end{macrocode}
% \end{macro}
%
%
% \subsubsection{The \texttt{changebreak} style}
%
%    \begin{macrocode}
% This style option is stored in the file |thcb.sty|.
%    \begin{macrocode}
%<*thcb>
\begingroup \makeatletter
\@ifundefined{theorem@style}{\input{theorem.sty}}{}
\ProvidesFile{thcb.sty}
             [\FMithmInfo]
%    \end{macrocode}
%
% \begin{macro}{\th@changebreak}
%    The \texttt{change-break} theorem style is like \texttt{break}
%    but with interchange of theorem name and theorem number.  Thus we
%    define |\th@changebreak| as follows:
%    \begin{macrocode}
\gdef\th@changebreak{\normalfont\slshape
  \def\@begintheorem##1##2{\item
       [\rlap{\vbox{\hbox{\hskip\labelsep \theorem@headerfont ##2\ ##1}%
                    \hbox{\strut}}}]}%
\def\@opargbegintheorem##1##2##3{%
  \item[\rlap{\vbox{\hbox{\hskip\labelsep \theorem@headerfont
                     ##2\ ##1\ (##3)}%
                    \hbox{\strut}}}]}}
\endgroup
%</thcb>
%    \end{macrocode}
% \end{macro}
%
%
% \subsubsection{The \texttt{change} style}
%
% This style option is stored in the file |thc.sty|.
%    \begin{macrocode}
%<*thc>
\begingroup \makeatletter
\@ifundefined{theorem@style}{\input{theorem.sty}}{}
\ProvidesFile{thc.sty}
             [\FMithmInfo]
%    \end{macrocode}
%
% \begin{macro}{\th@change}
%    The \texttt{change} theorem style corresponds to the
%    \texttt{change break} style without a linebreak after the header.
%    To say it in another way, it's the same as the \texttt{plain}
%    style but with number and name interchanged and |\slshape| as the
%    default font.
%    \begin{macrocode}
\gdef\th@change{\normalfont\slshape
  \def\@begintheorem##1##2{\item
       [\hskip\labelsep \theorem@headerfont ##2\ ##1]}%
\def\@opargbegintheorem##1##2##3{%
  \item[\hskip\labelsep \theorem@headerfont ##2\ ##1\ (##3)]}}
\endgroup
%</thc>
%    \end{macrocode}
% \end{macro}
%
%
% \subsubsection{The \texttt{marginbreak} style}
%
% This style option is the one used most often at Mainz.  It is saved
% in the file |thmb.sty|.
%    \begin{macrocode}
%<*thmb>
\begingroup \makeatletter
\@ifundefined{theorem@style}{\input{theorem.sty}}{}
\ProvidesFile{thmb.sty}
             [\FMithmInfo]
%    \end{macrocode}
%
% \begin{macro}{\th@marginbreak}
%    The \texttt{margin break} style is nearly the same as the
%    \texttt{change break} style. The only difference is the placement
%    of the theorem number.  We use |\llap| to place it in the left
%    margin.
%
%    In this style |\labelsep| denotes the separation between the
%    number and the text.
%    \begin{macrocode}
\gdef\th@marginbreak{\normalfont\slshape
  \def\@begintheorem##1##2{\item
       [\rlap{\vbox{\theorem@headerfont
                    \hbox{\llap{##2}\hskip\labelsep ##1}%
                    \hbox{\strut}}}]}%
\def\@opargbegintheorem##1##2##3{%
  \item[\rlap{\vbox{\theorem@headerfont
                    \hbox{\llap{##2}\hskip\labelsep ##1\ (##3)}%
                    \hbox{\strut}}}]}}
\endgroup
%</thmb>
%    \end{macrocode}
% \changes{v2.1c}{90/02/17}{`endgroup outside macrocode env.}
% \end{macro}
%
%
%
% \subsubsection{The \texttt{margin} style}
%
% This style option is stored in the file |thm.sty|.
%    \begin{macrocode}
%<*thm>
\begingroup \makeatletter
\@ifundefined{theorem@style}{\input{theorem.sty}}{}
\ProvidesFile{thm.sty}
             [\FMithmInfo]
%    \end{macrocode}
%
% \begin{macro}{\th@margin}
%    Again this is only a variant of the theorem styles described
%    above without any new ideas.
%    \begin{macrocode}
\gdef\th@margin{\normalfont\slshape
  \def\@begintheorem##1##2{\item
       [\theorem@headerfont \llap{##2}\hskip\labelsep ##1]}%
\def\@opargbegintheorem##1##2##3{%
  \item[\theorem@headerfont \llap{##2}\hskip\labelsep ##1\ (##3)]}}
\endgroup
%</thm>
%    \end{macrocode}
% \end{macro}
%
%
%
%
% \subsection{Final Definitions}
%
% \begin{macro}{\theorempreskipamount}
% \begin{macro}{\theorempostskipamount}
%    The \textsf{skip} parameters that regulate the vertical empty space
%    before and after the {\envfont theorem} environment have
%    to be allocated as well.
% \changes{v2.0f}{89/07/20}{Unnecessary `global' removed.}
%    \begin{macrocode}
%<*package>
\newskip\theorempreskipamount
\newskip\theorempostskipamount
%    \end{macrocode}
%    Since we have used the same values for all theorem sets, we now can
%    assign them.
%    \begin{macrocode}
\global\setlength\theorempreskipamount{12pt plus 5pt minus 3pt}
\global\setlength\theorempostskipamount{8pt plus 3pt minus 1.5pt}
%    \end{macrocode}
% \end{macro}
% \end{macro}
%
%
% \begin{macro}{\@endtheorem}
%    The same holds for the macro |\@endtheorem|, which ends a
%    {\envfont theorem} environment. Since it is the same for all
%    theorem sets, it is removed from the macros |\th@|\meta{style}.
%    It simply ends the {\envfont trivlist} environment, which was
%    begun in |\@thm|.
%    \begin{macrocode}
\global\let\@endtheorem=\endtrivlist
%    \end{macrocode}
% \end{macro}
%
%
%
% \begin{macro}{\@preamblecmds}
%    All macros defined above are to be used only in the preamble.
%    Therefore, we insert them in |\@preamblecmds| which will disable
%    them at begin document. This is done by the internal \LaTeXe{}
%    command |\@onlypreamble|.
% \changes{v2.0g}{89/09/19}{`theoremheaderfont and `theorembodyfont
%    added.}
% \changes{v2.2a}{94/02/03}{Use \LaTeXe{} interface}
%    \begin{macrocode}
\@onlypreamble\@xnthm
\@onlypreamble\@ynthm
\@onlypreamble\@othm
\@onlypreamble\newtheorem
\@onlypreamble\theoremstyle
\@onlypreamble\theorembodyfont
\@onlypreamble\theoremheaderfont
%    \end{macrocode}
% \end{macro}
%
% Finally we declare the \texttt{plain} theorem style to be the default.
% \changes{v2.1b}{90/06/04}{Fixed missing `item by declaring default
%    style.}
%    \begin{macrocode}
\theoremstyle{plain}
%</package>
%    \end{macrocode}
%
% \Finale

\endinput
}{}
\ProvidesFile{thmb.sty}
             [\FMithmInfo]
%    \end{macrocode}
%
% \begin{macro}{\th@marginbreak}
%    The \texttt{margin break} style is nearly the same as the
%    \texttt{change break} style. The only difference is the placement
%    of the theorem number.  We use |\llap| to place it in the left
%    margin.
%
%    In this style |\labelsep| denotes the separation between the
%    number and the text.
%    \begin{macrocode}
\gdef\th@marginbreak{\normalfont\slshape
  \def\@begintheorem##1##2{\item
       [\rlap{\vbox{\theorem@headerfont
                    \hbox{\llap{##2}\hskip\labelsep ##1}%
                    \hbox{\strut}}}]}%
\def\@opargbegintheorem##1##2##3{%
  \item[\rlap{\vbox{\theorem@headerfont
                    \hbox{\llap{##2}\hskip\labelsep ##1\ (##3)}%
                    \hbox{\strut}}}]}}
\endgroup
%</thmb>
%    \end{macrocode}
% \changes{v2.1c}{90/02/17}{`endgroup outside macrocode env.}
% \end{macro}
%
%
%
% \subsubsection{The \texttt{margin} style}
%
% This style option is stored in the file |thm.sty|.
%    \begin{macrocode}
%<*thm>
\begingroup \makeatletter
\@ifundefined{theorem@style}{% \iffalse meta-comment
%
% Copyright (C) 1993-2024
%
% The LaTeX Project and any individual authors listed elsewhere
% in this file.
%
% This file is part of the Standard LaTeX `Tools Bundle'.
% -------------------------------------------------------
%
% It may be distributed and/or modified under the
% conditions of the LaTeX Project Public License, either version 1.3c
% of this license or (at your option) any later version.
% The latest version of this license is in
%    https://www.latex-project.org/lppl.txt
% and version 1.3c or later is part of all distributions of LaTeX
% version 2005/12/01 or later.
%
% The list of all files belonging to the LaTeX `Tools Bundle' is
% given in the file `manifest.txt'.
%
% \fi
%
%%
%
%
\def\FMithmInfo{2023/07/05 v2.2c Theorem extension package (FMi)}
%
% \ProvidesFile{theorem.dtx}[\FMithmInfo]
%\iffalse   % this is a METACOMMENT !
%%
%% Package `theorem' to use with LaTeX2e
%% Copyright (C) 1989-1995 Frank Mittelbach, all rights reserved.
%<+package>\@ifundefined{theorem@style}{}{\endinput}
%<+package>\NeedsTeXFormat{LaTeX2e}[1995/06/01]
%<+package>\ProvidesPackage{theorem}
%<+package>         [\FMithmInfo]
%
% \fi
%
% \changes{v1.0}{87/05/07}{First draft.}
% \changes{v1.0a}{87/05/07}{`thistheoremstyle' added.}
% \changes{v1.1a}{87/07/29}{New concept: redefine the internal
%                          \protect\LaTeX{} commands `@xnthm' etc.}
% \changes{v1.1b}{87/07/31}{`theoremfont' used to specify font.}
% \changes{v1.2a}{88/01/08}{`@endtheorem' now globally defined.}
% \changes{v1.2b}{88/01/11}{`newtheorem' only allowed in the preamble.}
% \changes{v1.3a}{88/02/14}{`theoremfont' now a toks register.}
%
% \changes{v2.0a}{88/11/16}{New implementation and new documentation.}
% \changes{v2.0a}{88/11/16}{No layout changes but internal modifications
%                           old substyles won't work.}
% \changes{v2.0b}{89/01/30}{`global' used for all definitions.}
% \changes{v2.0c}{89/06/15}{Documentation translated to English.
%                            Thanks to LLL and Nina}
% \changes{v2.1a}{90/02/15}{Support for new font selection.}
% \changes{v2.2a}{94/02/03}{Cleanup for LaTeX2e}
%
% \DoNotIndex{\@endparenv,\@ifdefinable,\@ifnextchar,\@ifundefined}
% \DoNotIndex{\@namedef,\@nameuse,\@spaces,\@tempa,\@warning}
% \DoNotIndex{\ }
% \DoNotIndex{\begingroup,\bfseries}
% \DoNotIndex{\csname}
% \DoNotIndex{\def,\do,\docdate}
% \DoNotIndex{\endcsname,\endgroup,\endinput,\endtrivlist,\expandafter}
% \DoNotIndex{\filedate,\fileversion}
% \DoNotIndex{\gdef,\global}
% \DoNotIndex{\hbox,\hskip}
% \DoNotIndex{\ignorespaces,\input,\itshape,\item}
% \DoNotIndex{\labelsep,\let,\llap}
% \DoNotIndex{\makeatletter}
% \DoNotIndex{\noexpand}
% \DoNotIndex{\refstepcounter,\relax,\rlap}
% \DoNotIndex{\setlength,\slshape,\space,\string,\strut}
% \DoNotIndex{\the,\trivlist,\typeout}
% \DoNotIndex{\vbox}
% \DoNotIndex{\xdef}
%
% \def\deflabel#1{\ttfamily #1\hfill}
% \def\deflist#1{\begin{list}{}{\settowidth\labelwidth{\ttfamily #1}%
%                               \setlength\leftmargin\labelwidth
%                               \addtolength\leftmargin\labelsep
%                               \let\makelabel\deflabel}}
% \def\enddeflist{\end{list}}
% \def\envfont{\normalfont\ttfamily}
%
% \MakeShortVerb{\|}
%
% \newcommand{\TUB}{{\sl TUGboat\/}}
%
% \GetFileInfo{theorem.dtx}
% \title{An Extension of the
%        \LaTeX{} theorem environment\thanks{This file
%        has version number \fileversion, last
%        revised \filedate.}}
% \author{Frank Mittelbach}
%
% \MaintainedByLaTeXTeam{tools}
% \maketitle
%
% \begin{abstract}
%    The macros described in this paper yield an extension of the
%    \LaTeX{} theorem mechanism. It is designed to satisfy the
%    different requirements of various journals. Thus, the
%    layout of the ``theorems'' can be manipulated by determining a
%    ``style''. This article describes not only the use, but
%    also the definition, of the necessary macros.
% \end{abstract}
%
% \section*{Preface to version 2.2}
%
% For \LaTeXe{} this package did not need any fundamental changes. I
% only modified the messages generated so that theorem layout styles
% will show up with the |\listfiles| command and cleaned the section
% on the New Font Selection Scheme since this is now included in
% \LaTeX.
%
% \section*{Preface to version 2.1}
%
% This version is identical to 2.0g described in \TUB\ 10\#3 except
% for some internal defaults which are now set depending on the used
% font selection scheme.
%
% This was done to avoid unpleasant surprises if the new font selection
% scheme is in force. For further details see
% section~\ref{sec:newfonts} and \cite{bk:GMS94}.
%
% \section{Introduction}
%
% For our purposes here, ``theorems'' are labelled enunciations,
% often set off from the main text by extra space and a font change.
% Theorems, corollaries, conjectures, definitions, and remarks
% are all instances of ``theorems''.  The ``header'' of these structures
% is composed of a label (such as \textsc{Theorem} or \textsc{Remark})
% and a number which serializes an item in the sequence of items
% with the same label.
%
% Shortly after the introduction of \LaTeX{} at the Fachbereich
% Mathematik in Mainz, the desire to manipulate the layout of
% ``theorems'' arose.  In Mainz, the following two conventions came
% into general use:
% \begin{enumerate}
%   \item The number of the theorem is shown in the margin.
%   \item There is a line break at
%         the end of the theorem header.
% \end{enumerate}
%
% Additionally, some journals require different formats which depend on
% the ``sort of theorem'': e.g.\ often remarks and definitions are set
% in |\upshape|, while |\itshape| is employed for main theorems.
%
% Confronted with these requirements, a theorem environment was
% developed in Mainz which allows separate determination of the
% layout of the ``theorems sets'', comparable to |\pagestyle|.
%
%
% \section{The user interface}
% \subsection{Defining new theorem sets}
%
% \DescribeMacro\newtheorem
% As in the original \LaTeX{} version, the command |\newtheorem|
% defines a new ``theorem set'' or ``theorem-like structure''.
% Two required arguments name the new environment and give the text
% to be typeset with each instance of the new ``set'', while
% an optional argument determines how
% the ``set'' is enumerated:
% \begin{description}
%    \item[\ttfamily \bslash newtheorem\{foo\}\{bar\}]
%       The theorem set {\envfont foo} (whose name is \texttt{bar})
%       uses its own counter.
%    \item[\ttfamily \bslash newtheorem\{foo2\}{[foo]}\{bar2\}]
%       The theorem set {\envfont foo2} (printed name \texttt{bar2})
%       uses the same counter as the theorem set \texttt{foo}.
%    \item[\ttfamily \bslash newtheorem\{foo3\}\{bar3\}{[section]}]
%       The theorem set {\envfont foo3} (printed name \texttt{bar3}) is
%       enumerated within the counter \texttt{section}, i.e.\ with every
%       new |\section| the enumeration begins again with 1, and
%       the enumeration is composed from the section-number and the
%       theorem counter itself.
% \end{description}
%
% \DescribeMacro\theoremstyle
% Additionally, the command |\theoremstyle| can define the layout
% of various, or all, theorem sets.  It should be noted that any
% theorem set defined by |\newtheorem| is typeset in the
% |\theoremstyle| that is current at the time of the definition.
% Thus, the following
% \begin{verbatim}
%    \theoremstyle{break}        \newtheorem{Cor}{Corollary}
%    \theoremstyle{plain}        \newtheorem{Exa}{Example}[section]
%\end{verbatim}                             ^^A No blank at line start
% leads to the result that the set {\envfont Cor} is formatted in the
% style \texttt{break}, while the set {\envfont Exa} and all the
% following ones are formatted in the style \texttt{plain}, unless
% another |\theoremstyle| follows.  Since the definitions installed by
% |\newtheorem| are global, one also can limit |\theoremstyle| locally
% by grouping braces.
%
% \DescribeMacro\theorembodyfont
% The choice of the font for the theorem body is completely
% independent of the chosen |\theoremstyle|; this has proven to
% be very advantageous.  For example,
% \begin{verbatim}
%   {\theorembodyfont{\upshape}          \newtheorem{Rem}{Remark}}
%\end{verbatim}
% defines a theorem set {\envfont Rem}, which will be set in
% |\upshape| in the current layout (which in our example is
% \texttt{plain}).  As with |\theoremstyle|, the |\theorembodyfont|
% chosen is that current at the time of |\newtheorem|.  If
% |\theorembodyfont| is not specified or one defines
% |\theorembodyfont{}|, then the font used will be that defined by the
% |\theoremstyle|.
%
% \DescribeMacro\theoremheaderfont It is also possible to customize
% the font used for the theorem headers.  This is, however, a global
% declaration, and therefore there should be at most one
% |\theoremheaderfont| declaration in the preamble.\footnote{If it is
% actually necessary to have different header fonts, one has to define
% new theorem styles (substituting the desired font) or specify the
% information directly in the \texttt{\bslash newtheorem} declaration
% (the unclean variant).}
%
% \DescribeMacro\theorempreskipamount
% \DescribeMacro\theorempostskipamount Two additional parameters
% affect the vertical space around the theorem environments:
% |\theorempreskipamount| and |\theorempostskipamount| define,
% respectively, the spacing before and after such an environment.
% These parameters apply for all theorem sets and can be manipulated
% with the ordinary length macros.  They are rubber lengths
% (`\textsf{skips}'), and therefore can contain \texttt{plus} and
% \texttt{minus} parts.
%
% Since the definition of theorem sets should---most sensibly---be
% placed in the preamble, we only allow installation there.
% It is therefore possible to release the memory used here after
% |\begin{document}|, in order to make room for other
% applications.
%
%
% \subsection{Existing theorem styles}
%
% The following theorem styles exist to date:
% \begin{deflist}{marginbreak}
%    \item[plain]
%       This theorem style emulates the original \LaTeX{} definition,
%       except that additionally the parameters
%       |\theorem...skipamount| are used.
%    \item[break]
%       In this style, the theorem header is followed by a line
%       break.
%    \item[marginbreak]
%       The theorem number is set in the margin, and there is a line
%       break as in \texttt{break}.
%    \item[changebreak]
%       Like \texttt{break}, but with header number and text
%       interchanged.
%    \item[change]
%       Header number and text are interchanged, without a line break.
%    \item[margin]
%       The number is set in the left margin, without a line break.
% \end{deflist}
% All styles (except \texttt{plain}) select |\slshape| as the default
% |\theorembodyfont|.
%
%
% \subsection{Examples}
%
% Given the above theorem sets {\envfont Cor}, {\envfont Exa}
% and {\envfont Rem},
% suppose that the preamble also contains the declarations:
% \begin{verbatim}
%    \theoremstyle{marginbreak}   \newtheorem{Lem}[Cor]{Lemma}
%    \theoremstyle{change}
%    \theorembodyfont{\itshape}        \newtheorem{Def}[Cor]{Definition}
%
%    \theoremheaderfont{\scshape}
%\end{verbatim}
% Then the following are some typical examples of the typeset output
% resulting from their use.
%
% \begin{Cor}
%    This is a sentence typeset in the theorem environment {\envfont
%    Cor}.
% \end{Cor}
%
% \begin{Exa}
%    This is a sentence typeset in the theorem environment {\envfont
%    Exa}.
% \end{Exa}
%
% \begin{Rem}
%    This is a sentence typeset in the theorem environment {\envfont
%    Rem}.
% \end{Rem}
%
% \begin{Lem}[Ben User]
%    This is a sentence typeset in the theorem environment {\envfont
%    Lem}.
% \end{Lem}
%
% \begin{Def}[Very impressive Definition]
%    This is a sentence typeset in the theorem environment {\envfont
%    Def}.
% \end{Def}
%
% The last two examples show the effect of the optional argument to
% a theorem environment (it is the text typeset in parentheses).
%
%
% \section{Special Considerations}
% \label{sec:newfonts}
%
% Theoremheader and body are implemented as a unit. This means that
% the |\theoremheaderfont|  will inherit characteristics of the
% |\theorembodyfont| in \LaTeXe. Thus, if for example
% |\theorembodyfont| is |\itshape| and |\theoremheaderfont|
% is |\bfseries| the font selected for the header will have the
% characteristics `bold extended italic'. If this is not desired one
% should set the |\theoremheaderfont| to something like
% \begin{verbatim}
%      \theoremheaderfont{\normalfont\bfseries}
%\end{verbatim}
% i.e.\ supplying all necessary font information explicitly.
%
%
%
% \section{Acknowledgements}
%
% The publication of this set of macros was only possible with the
% help of Christina Busse (translating the manuscript into English),
% Joachim Pense (playing the r\^ole of typist), Chris Rowley
% (looking everything over) and many others providing useful
% suggestions.
%
% \MaybeStop{%
%    \begin{thebibliography}{9}
%       \bibitem{bk:GMS94} \textsc{M.~Goossens}, \textsc{F.~Mittelbach}
%         and \textsc{A.~Samarin}.
%         \newblock The \LaTeX{} Companion.
%         \newblock
%         Addison-Wesley, Reading, Massachusetts, 1994.
%       \bibitem{latex:source} \textsc{Lamport, Leslie.} \newblock
%                               \texttt{latex.tex}, version 2.09, date
%                               Feb. 1990.
%    \end{thebibliography}
%    \PrintIndex
%    \PrintChanges}
%
%
% \section{The documentation driver file}
%
% The next bit of code contains the documentation driver file for
% \TeX{}, i.e., the file that will produce the documentation you are
% currently reading. It will be extracted from this file by the
% \texttt{docstrip} program. Since it is the first code in the file
% one can alternatively process this file directly with \LaTeXe{} to
% obtain the documentation.
%
%    \begin{macrocode}
%<*driver>
\documentclass{ltxdoc}

\usepackage{theorem}

%<+driver>% The next few lines define theorem sets which are used
%<+driver>% in the example section of the documentation.

\theoremstyle{break}       \newtheorem{Cor}{Corollary}
\theoremstyle{plain}       \newtheorem{Exa}{Example}[section]
{\theorembodyfont{\upshape}\newtheorem{Rem}{Remark}}
\theoremstyle{marginbreak} \newtheorem{Lem}[Cor]{Lemma}
\theoremstyle{change}
\theorembodyfont{\itshape} \newtheorem{Def}[Cor]{Definition}

\theoremheaderfont{\scshape}

\RecordChanges

\begin{document}
   \DocInput{theorem.dtx}
\end{document}
%</driver>
%    \end{macrocode}
%
% \section{Definition of the Macros}
%
% If the file has been loaded before, we abort immediately. If not the
% package announces itself (this is actually done at the very top of
% the file---the way it is done isn't good style so don't copy it).
% \changes{v2.0e}{89/07/19}{Spaces removed from `typeout'.}
% \changes{v2.2b}{95/11/19}{Announce moved to top of file}
%    \begin{macrocode}
%<*package>
%\@ifundefined{theorem@style}{}{\endinput}
%\def\FMithmInfo{1995/11/19 v2.2b Theorem extension package (FMi)}
%\ProvidesPackage{theorem}[\FMithmInfo]
%    \end{macrocode}
%
%
% \subsection{Definition of theorem styles and fonts}
%
% All the definitions in this file are done globally to allow
% inputting this file inside a group.
%
% \begin{macro}{\theoremstyle}
%    Before a theorem style can be installed, the chosen style must be
%    known. For that reason, we must test to see that
%    |\th@|\meta{style} is known or, more precisely, that it is
%    different from |\relax|. If the style is not known then
%    |\th@plain| is used.
%    \begin{macrocode}
\gdef\theoremstyle#1{%
   \@ifundefined{th@#1}{\@warning
          {Unknown theoremstyle `#1'. Using `plain'}%
          \theorem@style{plain}}%
%    \end{macrocode}
%    We save the theorem style to be used in the token register
%    |\theorem@style|.
%    \begin{macrocode}
      {\theorem@style{#1}}%
%    \end{macrocode}
%    Now we ``evaluate'' the theorem style: this means, we call the
%    macro |\th@|\meta{style} which will activate the relevant
%    definitions which are contained in a separate file.  This is done
%    in a group to suppress changes to the current font.  This could
%    otherwise pose problems together with the new font selection
%    scheme\footnote{When I printed the original article using the new
%    font selection scheme I ended with a document with slanted
%    typefaces (text headings and all) simply because one of the
%    theorem styles used \texttt{\bslash sl} at toplevel.} if the
%    |\th@|\meta{style} is evaluated a second time.
% \changes{v2.1a}{90/02/15}{Group added around `th@..}
% \changes{v2.1a}{90/02/15}{Reseting of `..@begintheorem removed.}
%    \begin{macrocode}
      \begingroup
        \csname th@\the\theorem@style \endcsname
      \endgroup}
%    \end{macrocode}
% \end{macro}
%
%
% \begin{macro}{\@begintheorem}
% \changes{v2.1a}{90/02/15}{Relaxed at toplevel.}
% \begin{macro}{\@opargbegintheorem}
% \changes{v2.1a}{90/02/15}{Relaxed at toplevel.}
%    We reset |\@begintheorem| and |\@opargbegintheorem| to
%    |\relax| since these commands are no longer necessary at
%    toplevel. This will save a few tokens.
%    \begin{macrocode}
\global\let\@begintheorem\relax
\global\let\@opargbegintheorem\relax
%    \end{macrocode}
% \end{macro}
% \end{macro}
%
% \begin{macro}{\theorem@style}
%    Obviously the \textsf{token} register used above has to be
%    allocated. To assure the utmost compatibility with the
%    original \LaTeX{} definition, we set the default theorem style to
%    \texttt{plain}, which implements the usual \LaTeX{} convention.
% \changes{v2.0f}{89/07/20}{Unnecessary `global' removed.}
%    \begin{macrocode}
\newtoks\theorem@style
\global\theorem@style{plain}
%    \end{macrocode}
% \end{macro}
%
% \begin{macro}{\theorembodyfont}
% \begin{macro}{\theorem@bodyfont}
%    If the |\theorembodyfont| is set by the user then it should not
%    interact with the default font set in the theorem style. When the
%    new font selection is in force this may happen if, for example,
%    the default is |\itshape| and the new |\theorembodyfont| is
%    |\sffamily|. So we add a |\reset@font| command in front of the
%    user definition.
% \changes{v2.0f}{89/07/20}{Unnecessary `global' removed.}
% \changes{v2.0g}{89/09/19}{`theoremfont to `theorembodyfont.}
% \changes{v2.1c}{92/06/22}{`reset@font added and logic changed.}
% \changes{v2.1d}{92/10/28}{Reset to empty if arg is empty}
%    \begin{macrocode}
\gdef\theorembodyfont#1{%
%    \end{macrocode}
%    We check if the argument supplied is empty and if so put nothing
%    into the |\theorem@bodyfont| token register to allow for
%    |\theorembodyfont{}| as a mean of using the default of the
%    current |\theoremstyle|.
%    \begin{macrocode}
   \def\@tempa{#1}%
   \ifx\@tempa\@empty
    \theorem@bodyfont{}%
   \else
    \theorem@bodyfont{\reset@font#1}%
   \fi
   }
\newtoks\theorem@bodyfont
\global\theorem@bodyfont{}
%    \end{macrocode}
% \end{macro}
% \end{macro}
%
%
% \begin{macro}{\theoremheaderfont}
% \changes{v2.0g}{89/09/19}{Macro added.}
%    The font for the theorem headers is handled differently because
%    this definition applies to all theorem styles.
%    \begin{macrocode}
\gdef\theoremheaderfont#1{\gdef\theorem@headerfont{#1}%
%    \end{macrocode}
%    After using the macro once it is redefined to produce an error
%    message.
%    \begin{macrocode}
       \gdef\theoremheaderfont##1{%
        \typeout{\string\theoremheaderfont\space should be used
                 only once.}}}
%    \end{macrocode}
% \end{macro}
%
% \begin{macro}{\theorem@headerfont}
% \changes{v2.0g}{89/09/19}{Macro added.}
%    To set the |\theorem@headerfont| default we first test if
%    the new fontselection scheme is in force.
% \changes{v2.1a}{90/02/15}{Support for new font selection.}
%    \begin{macrocode}
\ifx\upshape\undefined
%    \end{macrocode}
%    If not we define it to expand into |\bfseries|. We don't use
%    |\let| just in case a following style option redefines this
%    macro.
%    \begin{macrocode}
\gdef\theorem@headerfont{\bfseries}
%    \end{macrocode}
%    Otherwise we reset the current shape before calling |\bfseries|.
%    \begin{macrocode}
\else \gdef\theorem@headerfont{\normalfont\bfseries}\fi
%    \end{macrocode}
% \end{macro}
%
% \begin{macro}{\th@plain}
% \begin{macro}{\th@break}
% \begin{macro}{\th@marginbreak}
% \begin{macro}{\th@changebreak}
% \begin{macro}{\th@change}
% \begin{macro}{\th@margin}
%    The different styles are defined in macros such as
%    |\th@plain|.  Since memory space is precious in
%    ``non-Big-versions'', we have to avoid offering too many unused
%    definitions.  Therefore we define these styles in separate files
%    that can be loaded on demand.  Thus the commands themselves only
%    load these files. We use |\@input@|, a \LaTeXe{} internal command
%    that ensures that the file will be listed with |\listfiles|.
% \changes{v2.2a}{94/02/02}{Use `@input@ to load theorem layout files}
%    \begin{macrocode}
\gdef\th@plain{\@input@{thp.sty}}
\gdef\th@break{\@input@{thb.sty}}
\gdef\th@marginbreak{\@input@{thmb.sty}}
\gdef\th@changebreak{\@input@{thcb.sty}}
\gdef\th@change{\@input@{thc.sty}}
\gdef\th@margin{\@input@{thm.sty}}
%    \end{macrocode}
%    This list will be expanded when new styles become
%    available.  For testing,  just append new theorem substyles
%    as document options.
% \end{macro}
% \end{macro}
% \end{macro}
% \end{macro}
% \end{macro}
% \end{macro}
%
%
%
% \subsection{Definition of a new theorem set}
%
% As already pointed out, a new theorem environment can be defined in
% three different ways:
% \begin{verbatim}
%    \newtheorem{Lem}{Lemma}
%    \newtheorem{Lem}{Lemma}[section]
%    \newtheorem{Lem}[Theorem]{Lemma}
%\end{verbatim}
% The function of the macro |\newtheorem| is to recognize these cases
% and then to branch into one of the three macros |\@ynthm|, |\@xnthm|
% or |\@othm|.  This mechanism is adopted unchanged from
% \cite{latex:source}; the essential point here is that, for example,
% in the second case, the arguments \texttt{Lem}, \texttt{Lemma} and
% \texttt{section} are passed over to the macro |\@xnthm|.
%
% We inspect this case first because the others present fewer problems,
% and thus are easily derived from this one.
%
% \begin{macro}{\@xnthm}
%    For our example arguments, the macro |\@xnthm| must fulfill
%    the following:
%    \begin{itemize}
%       \item
%          Define a new \LaTeX-counter `Lem'
%       \item
%          reset this counter within a |\section|
%       \item
%          define the macro |\theLem|
%       \item
%          define the environment macros |\Lem| and
%          |\endLem| using the current
%          |\theoremstyle| and |\theorem@bodyfont|.
%    \end{itemize}
%    Obviously, all this should happen only if the first argument
%    of |\@xnthm| (i.e.\ \texttt{Lem} in our example) is chosen so
%    as not to conflict with any previously defined commands or
%    environments.  This test is performed by the \LaTeX{} macro
%    |\@ifdefinable|.
%    \begin{macrocode}
\gdef\@xnthm#1#2[#3]{\expandafter\@ifdefinable\csname #1\endcsname
%    \end{macrocode}
%    Therefore, the first argument of |\@ifdefinable| is the
%    expansion (in the example, |\Lem|) of
%    |\csname||#1\endcsname|.  The second argument is
%    executed only if the test has been completed successfully.
%    \begin{macrocode}
   {%
%    \end{macrocode}
%    Now we define the new counter. The names of the \LaTeX{}
%    macros employed should speak for themselves:
% \changes{v2.2b}{1995/10/07}{Use \cs{@newctr}}
% \changes{v2.2c}{1995/11/23}{Use \cs{@newctr} properly.}
%    \begin{macrocode}
    \@definecounter{#1}\@newctr{#1}[#3]%
%    \end{macrocode}
%    Using |\@newctr| will give a proper error message if the counter
%    in |#3| is not defined.
%    In defining `|\theLem|' we must generate the desired macro
%    name by use of |\expandafter| and |\csname|.
%    \begin{macrocode}
    \expandafter\xdef\csname the#1\endcsname
%    \end{macrocode}
%    An |\xdef| is used in order to make the definition global,
%    and to ensure that it contains the replacement texts of
%    |\@thmcountersep| and |\@thmcounter|.\footnote{These
%    two macros can be defined by the document style.  Their default
%    values produce a `\texttt{.}' as separation and an arabic
%    representation of the number.} However, not everything should be
%    expanded. For example, it saves space to use |\thesection|
%    instead of its---at times---lengthy expansion.
%    \begin{macrocode}
      {\expandafter \noexpand \csname the#3\endcsname
       \@thmcountersep \@thmcounter{#1}}%
%    \end{macrocode}
%    Thus with the defaults of \LaTeX{}, |\theLem| would be
%    replaced by the command sequence |\thesection.\arabic{Lem}|.
%
%    We will now look at the definition of the macro which is executed
%    at the beginning of the actual environment (in our example this
%    macro is |\Lem|). It should be noted that we use an
%    ``|\expandafter| trick'' to expand only certain parts of the
%    replacement text at the time of the definition.
%    \begin{macrocode}
    \def\@tempa{\global\@namedef{#1}}%
    \expandafter \@tempa \expandafter{%
%    \end{macrocode}
%    First, the macro that contains the current definitions of
%    |\@begintheorem| and |\@opargtheorem| should be called
%    up.  The name of this macro---as is already known---has the form
%    |\th@|\meta{theorem\ style}; therefore, it must be called by
%    \begin{macrocode}
      \csname th@\the \theorem@style
%    \end{macrocode}
%    In addition the default theorem font should be changeable, i.e.\
%    we have to insert the contents of |\theorem@bodyfont|. For
%    that reason, we expand even further, beyond |\endcsname|,
%    and thus insert the contents of the \textsf{token} register
%    |\theorem@bodyfont| in the replacement text.
%    \begin{macrocode}
            \expandafter \endcsname \the \theorem@bodyfont
%    \end{macrocode}
%    Now it is time to call the macro |\@thm| which takes over
%    the further processing.  It has two arguments: the current
%    counter name (in our example, \texttt{Lem}), and the text of the
%    label (in our example, \texttt{Lemma}).
%    \begin{macrocode}
     \@thm{#1}{#2}}%
%    \end{macrocode}
%    With this, the `sub-definition' is complete.  The macro
%    |\@endtheorem| ends a theorem environment and is, so far, nothing
%    but an |\endtrivlist|.  (Hence it is defined globally, and not
%    within the theorem styles.\footnote{This has to be changed as
%    soon as theorem styles that change \texttt{\bslash @endtheorem}
%    exist.  In such a case, all existing styles must be changed as
%    well since they will have to reset the macro.}) Therefore, we can
%    set it equivalent to the macro that ends the theorem set (in our
%    example, |\endLem|).  However, if some day theorem styles exist
%    that do change |\@endtheorem|, we would have to use the
%    commented-out line instead.
%    \begin{macrocode}
    \global \expandafter \let \csname end#1\endcsname \@endtheorem
%   \global\@namedef{end#1}{\@endtheorem}%
%    \end{macrocode}
%    With these commands all the required definitions are employed,
%    unless the test |\@ifdefinable| has failed.  Therefore, we
%    end the second argument of this macro and with it the definition
%    of |\@xnthm|.
%    \begin{macrocode}
   }}
%    \end{macrocode}
% \end{macro}
%
% \begin{macro}{\@ynthm}
%    The definition of |\@ynthm| is completely analogous. In
%    this case the new counter that is defined is not reset within
%    another counter; thus the definition of |\the...|
%    is simplified:
%    \begin{macrocode}
\gdef\@ynthm#1#2{\expandafter\@ifdefinable\csname #1\endcsname
   {\@definecounter{#1}%
    \expandafter\xdef\csname the#1\endcsname{\@thmcounter{#1}}%
%    \end{macrocode}
%    The rest of the definition corresponds literally to that of
%    |\@xnthm|:
%    \begin{macrocode}
    \def\@tempa{\global\@namedef{#1}}\expandafter \@tempa
     \expandafter{\csname th@\the \theorem@style \expandafter
     \endcsname \the\theorem@bodyfont \@thm{#1}{#2}}%
    \global \expandafter \let \csname end#1\endcsname \@endtheorem}}
%    \end{macrocode}
% \end{macro}
%
% \begin{macro}{\@othm}
%    The definition of |\@othm| does not contain anything new.
%    \begin{macrocode}
\gdef\@othm#1[#2]#3{%
%    \end{macrocode}
% \changes{v2.2b}{95/10/07}{Build in check for counter defined}
%    We do not define a new counter but instead use one that has
%    already been defined.  Thus the only definition we need is that
%    of this pseudo-counter (i.e.\ |\the|\meta{env.\ name}).
%    First we check if |#2| corresponds to a known counter name.
%    \begin{macrocode}
  \expandafter\ifx\csname c@#2\endcsname\relax
   \@nocounterr{#2}%
  \else
   \expandafter\@ifdefinable\csname #1\endcsname
   {\expandafter \xdef \csname the#1\endcsname
     {\expandafter \noexpand \csname the#2\endcsname}%
%    \end{macrocode}
%    All other parts of the definition can be adopted from
%    |\@xnthm|. We have to remember, though, that in this case
%    the name of the current counter and the theorem label have moved
%    to the second and third arguments.
%    \begin{macrocode}
    \def\@tempa{\global\@namedef{#1}}\expandafter \@tempa
     \expandafter{\csname th@\the \theorem@style \expandafter
     \endcsname \the\theorem@bodyfont \@thm{#2}{#3}}%
    \global \expandafter \let \csname end#1\endcsname \@endtheorem}%
  \fi}
%    \end{macrocode}
% \end{macro}
%
% \subsection{Macros that are employed in a {\envfont theorem}
%    environment}
%
% \begin{macro}{\@thm}
%    The macro |\@thm| has to increase the current counter.  Then,
%    depending on whether the environment has (or does not have) an
%    optional argument, it has to branch into either |\@begintheorem|
%    or |\@opargtheorem|.
%    \begin{macrocode}
\gdef\@thm#1#2{\refstepcounter{#1}%
%    \end{macrocode}
%    Now we start a {\envfont trivlist} environment, and give
%    |\@topsep| and |\@topsepadd| the values of the \textsf{skip}
%    registers |\theorempreskipamount| and |\theorempostskipamount|.
%    The value in |\@topsep| is the vertical space that is inserted by
%    the first (and only) |\item| in our |\trivlist| whilst
%    |\@topsepadd| is inserted by |\@endparenv| at the end of that
%    {\envfont trivlist} environment.  By using these registers, we
%    obtain the desired space around a {\envfont theorem} environment.
%    \begin{macrocode}
   \trivlist
   \@topsep \theorempreskipamount               % used by first \item
   \@topsepadd \theorempostskipamount           % used by \@endparenv
%    \end{macrocode}
%    Now we have to test whether an optional argument has been given.
% \changes{v2.0d}{89/07/18}{Forgotten percent added.}
%    \begin{macrocode}
   \@ifnextchar [%
%    \end{macrocode}
%    If there is an optional argument, we will call |\@ythm|, and
%    move the arguments read back into the input stream.
%    \begin{macrocode}
   {\@ythm{#1}{#2}}%
%    \end{macrocode}
%    If not, we call |\@begintheorem|. Its first argument is the
%    name of the theorem set (hence the second argument of
%    |\@thm|). Its second argument is the macro that produces the
%    current number.
%    \begin{macrocode}
   {\@begintheorem{#2}{\csname the#1\endcsname}\ignorespaces}}
%    \end{macrocode}
% \end{macro}
%
% \begin{macro}{\@xthm}
% \begin{macro}{\@ythm}
%    Both these macros were originally called by |\@thm|.  We do
%    not need |\@xthm| anymore, hence we reset it to
%    |\relax|. The definition of |\@ythm| has not changed at
%    all from its definition in \LaTeX. In order to make the macros
%    easier to understand, we will
%    nevertheless present it (commented out).
%    \begin{macrocode}
\global\let\@xthm\relax
% \def\@ythm#1#2[#3]{\@opargbegintheorem{#2}{\csname
%           the#1\endcsname}{#3}\ignorespaces}
%</package>
%    \end{macrocode}
%    The primitive |\ignorespaces| in |\@ythm| and
%    |\@thm| is needed to remove the spaces between the
%    |\begin{...}| and the actual text.
% \end{macro}
% \end{macro}
%
%
% \subsection{Definition of the theorem substyles}
%
% As already pointed out, the theorem substyles, defined below, are
% only loaded when necessary.  Note that all these substyles, except
% \texttt{plain}, have |\slshape| as the default body font.
%
% \subsubsection{The \texttt{plain} style}
%
% \label{dok:th@plain} As the following macros use |@|, we have
% to locally set the |\catcode| of this symbol to ``letter''.
% This happens within a group, so that we do not have to worry about
% which |\catcode| that symbol had before.
%    \begin{macrocode}
%<*thp>
\begingroup \makeatletter
%    \end{macrocode}
% Since we are now within a group, we must make all definitions
% globally.  First we make sure that \texttt{theorem.sty} is loaded.
% This will allow us to use this file as a document style option
% without having to call \texttt{theorem} itself as an option.  At the
% same time, we assure that at least version 2 is loaded, since
% |\theorem@style| was not defined in earlier versions.
%    \begin{macrocode}
\@ifundefined{theorem@style}{\input{theorem.sty}}{}
\ProvidesFile{thp.sty}
             [\FMithmInfo]
%    \end{macrocode}
%
%
% \begin{macro}{\th@plain}
% \begin{macro}{\@begintheorem}
% \begin{macro}{\@opargbegintheorem}
%    |\theoremstyle{plain}| corresponds to the original
%    definition, except that the distances to the surrounding text are
%    determined by the parameters |\theorempreskipamount| and
%    |\theorempostskipamount|.
%    First we set the default body font.
%    \begin{macrocode}
\gdef\th@plain{\normalfont\itshape
%    \end{macrocode}
%    Then we define |\@begintheorem| and
%    |\@opargbegintheorem|. These two macros define how the
%    header of a theorem is typeset.  |\@opargbegintheorem| will
%    be called if a \textsf{theorem} environment with an optional
%    argument is encountered; otherwise, the header is constructed by
%    calling |\@begintheorem|.  If one of these macros is
%    executed, we are within a \textsf{trivlist} environment started by
%    |\@thm|.  So the theorem header is produced with an
%    |\item| command.
%
%    Instead of specifying the header font directly, all standard
%    theorem styles use the |\theorem@headerfont| macro to allow
%    customization.  The extra space (|\labelsep|) is necessary
%    because of problems in the \textsf{trivlist} environment.
%    \begin{macrocode}
  \def\@begintheorem##1##2{%
        \item[\hskip\labelsep \theorem@headerfont ##1\ ##2]}%
%    \end{macrocode}
%    The definition of |\@opargbegintheorem| is completely
%    analogous. The only difference is the fact that there exists a
%    third argument (which is the optional parameter of the
%    environment and contains additional information about the
%    theorem).  Customarily we enclose it in parentheses.
%    \begin{macrocode}
\def\@opargbegintheorem##1##2##3{%
   \item[\hskip\labelsep \theorem@headerfont ##1\ ##2\ (##3)]}}
%    \end{macrocode}
% \end{macro}
% \end{macro}
% \end{macro}
%
% \noindent
% We conclude with an |\endgroup| to restore the
% |\catcode| of |@|.
%    \begin{macrocode}
\endgroup
%</thp>
%    \end{macrocode}
%
%
% \subsubsection{The \texttt{break} style}
%
% This style option is stored in the file |thb.sty|.
% For the next two lines see the documentation for |\th@plain| on
% page~\pageref{dok:th@plain}.
%    \begin{macrocode}
%<*thb>
\begingroup \makeatletter
\@ifundefined{theorem@style}{\input{theorem.sty}}{}
\ProvidesFile{thb.sty}
             [\FMithmInfo]
%    \end{macrocode}
%
% \begin{macro}{\th@break}
%    |\theoremstyle{break}| produces a line break after the name
%    of the theorem.  The font is |\slshape|.  Hence, we define
%    |\th@break| as follows:
%    \begin{macrocode}
\gdef\th@break{\normalfont\slshape
  \def\@begintheorem##1##2{\item[%
%    \end{macrocode}
%    We run into the following problem: it is not possible to create
%    the header with |\item[|\meta{title}|]| and then start
%    a new line by, for example, |\mbox{}\\|. Such a definition
%    will fail whenever a list environment follows immediately.  With
%    the above construction, the |\mbox{}| causes the switch
%    |@inlabel| (cf.\ definition of |\list| and
%    |\trivlist| in \cite{latex:source}) to be set to \texttt{false}
%    and so the following list will insert additional vertical
%    space (|\topskip|). This is quite annoying. Therefore, we
%    create the line break within the |\item|. In order to ensure
%    that the text will begin at the proper position in the following
%    line, we simply pretend that the label does not take any
%    room.\footnote{This will lead to problems whenever very high
%    symbols occurring in the line tower into the
%    heading. So, something else has to be done here sometime.}
%    \begin{macrocode}
       \rlap{\vbox{\hbox{\hskip \labelsep\theorem@headerfont ##1\ ##2}%
                   \hbox{\strut}}}]}%
%    \end{macrocode}
%    Again, the definition of |\@opargbegintheorem| is
%    completely analogous.
%    \begin{macrocode}
\def\@opargbegintheorem##1##2##3{%
  \item[\rlap{\vbox{\hbox{\hskip \labelsep \theorem@headerfont
                     ##1\ ##2\ (##3)}%
                    \hbox{\strut}}}]}}
\endgroup
%</thb>
%    \end{macrocode}
% \end{macro}
%
%
% \subsubsection{The \texttt{changebreak} style}
%
%    \begin{macrocode}
% This style option is stored in the file |thcb.sty|.
%    \begin{macrocode}
%<*thcb>
\begingroup \makeatletter
\@ifundefined{theorem@style}{\input{theorem.sty}}{}
\ProvidesFile{thcb.sty}
             [\FMithmInfo]
%    \end{macrocode}
%
% \begin{macro}{\th@changebreak}
%    The \texttt{change-break} theorem style is like \texttt{break}
%    but with interchange of theorem name and theorem number.  Thus we
%    define |\th@changebreak| as follows:
%    \begin{macrocode}
\gdef\th@changebreak{\normalfont\slshape
  \def\@begintheorem##1##2{\item
       [\rlap{\vbox{\hbox{\hskip\labelsep \theorem@headerfont ##2\ ##1}%
                    \hbox{\strut}}}]}%
\def\@opargbegintheorem##1##2##3{%
  \item[\rlap{\vbox{\hbox{\hskip\labelsep \theorem@headerfont
                     ##2\ ##1\ (##3)}%
                    \hbox{\strut}}}]}}
\endgroup
%</thcb>
%    \end{macrocode}
% \end{macro}
%
%
% \subsubsection{The \texttt{change} style}
%
% This style option is stored in the file |thc.sty|.
%    \begin{macrocode}
%<*thc>
\begingroup \makeatletter
\@ifundefined{theorem@style}{\input{theorem.sty}}{}
\ProvidesFile{thc.sty}
             [\FMithmInfo]
%    \end{macrocode}
%
% \begin{macro}{\th@change}
%    The \texttt{change} theorem style corresponds to the
%    \texttt{change break} style without a linebreak after the header.
%    To say it in another way, it's the same as the \texttt{plain}
%    style but with number and name interchanged and |\slshape| as the
%    default font.
%    \begin{macrocode}
\gdef\th@change{\normalfont\slshape
  \def\@begintheorem##1##2{\item
       [\hskip\labelsep \theorem@headerfont ##2\ ##1]}%
\def\@opargbegintheorem##1##2##3{%
  \item[\hskip\labelsep \theorem@headerfont ##2\ ##1\ (##3)]}}
\endgroup
%</thc>
%    \end{macrocode}
% \end{macro}
%
%
% \subsubsection{The \texttt{marginbreak} style}
%
% This style option is the one used most often at Mainz.  It is saved
% in the file |thmb.sty|.
%    \begin{macrocode}
%<*thmb>
\begingroup \makeatletter
\@ifundefined{theorem@style}{\input{theorem.sty}}{}
\ProvidesFile{thmb.sty}
             [\FMithmInfo]
%    \end{macrocode}
%
% \begin{macro}{\th@marginbreak}
%    The \texttt{margin break} style is nearly the same as the
%    \texttt{change break} style. The only difference is the placement
%    of the theorem number.  We use |\llap| to place it in the left
%    margin.
%
%    In this style |\labelsep| denotes the separation between the
%    number and the text.
%    \begin{macrocode}
\gdef\th@marginbreak{\normalfont\slshape
  \def\@begintheorem##1##2{\item
       [\rlap{\vbox{\theorem@headerfont
                    \hbox{\llap{##2}\hskip\labelsep ##1}%
                    \hbox{\strut}}}]}%
\def\@opargbegintheorem##1##2##3{%
  \item[\rlap{\vbox{\theorem@headerfont
                    \hbox{\llap{##2}\hskip\labelsep ##1\ (##3)}%
                    \hbox{\strut}}}]}}
\endgroup
%</thmb>
%    \end{macrocode}
% \changes{v2.1c}{90/02/17}{`endgroup outside macrocode env.}
% \end{macro}
%
%
%
% \subsubsection{The \texttt{margin} style}
%
% This style option is stored in the file |thm.sty|.
%    \begin{macrocode}
%<*thm>
\begingroup \makeatletter
\@ifundefined{theorem@style}{\input{theorem.sty}}{}
\ProvidesFile{thm.sty}
             [\FMithmInfo]
%    \end{macrocode}
%
% \begin{macro}{\th@margin}
%    Again this is only a variant of the theorem styles described
%    above without any new ideas.
%    \begin{macrocode}
\gdef\th@margin{\normalfont\slshape
  \def\@begintheorem##1##2{\item
       [\theorem@headerfont \llap{##2}\hskip\labelsep ##1]}%
\def\@opargbegintheorem##1##2##3{%
  \item[\theorem@headerfont \llap{##2}\hskip\labelsep ##1\ (##3)]}}
\endgroup
%</thm>
%    \end{macrocode}
% \end{macro}
%
%
%
%
% \subsection{Final Definitions}
%
% \begin{macro}{\theorempreskipamount}
% \begin{macro}{\theorempostskipamount}
%    The \textsf{skip} parameters that regulate the vertical empty space
%    before and after the {\envfont theorem} environment have
%    to be allocated as well.
% \changes{v2.0f}{89/07/20}{Unnecessary `global' removed.}
%    \begin{macrocode}
%<*package>
\newskip\theorempreskipamount
\newskip\theorempostskipamount
%    \end{macrocode}
%    Since we have used the same values for all theorem sets, we now can
%    assign them.
%    \begin{macrocode}
\global\setlength\theorempreskipamount{12pt plus 5pt minus 3pt}
\global\setlength\theorempostskipamount{8pt plus 3pt minus 1.5pt}
%    \end{macrocode}
% \end{macro}
% \end{macro}
%
%
% \begin{macro}{\@endtheorem}
%    The same holds for the macro |\@endtheorem|, which ends a
%    {\envfont theorem} environment. Since it is the same for all
%    theorem sets, it is removed from the macros |\th@|\meta{style}.
%    It simply ends the {\envfont trivlist} environment, which was
%    begun in |\@thm|.
%    \begin{macrocode}
\global\let\@endtheorem=\endtrivlist
%    \end{macrocode}
% \end{macro}
%
%
%
% \begin{macro}{\@preamblecmds}
%    All macros defined above are to be used only in the preamble.
%    Therefore, we insert them in |\@preamblecmds| which will disable
%    them at begin document. This is done by the internal \LaTeXe{}
%    command |\@onlypreamble|.
% \changes{v2.0g}{89/09/19}{`theoremheaderfont and `theorembodyfont
%    added.}
% \changes{v2.2a}{94/02/03}{Use \LaTeXe{} interface}
%    \begin{macrocode}
\@onlypreamble\@xnthm
\@onlypreamble\@ynthm
\@onlypreamble\@othm
\@onlypreamble\newtheorem
\@onlypreamble\theoremstyle
\@onlypreamble\theorembodyfont
\@onlypreamble\theoremheaderfont
%    \end{macrocode}
% \end{macro}
%
% Finally we declare the \texttt{plain} theorem style to be the default.
% \changes{v2.1b}{90/06/04}{Fixed missing `item by declaring default
%    style.}
%    \begin{macrocode}
\theoremstyle{plain}
%</package>
%    \end{macrocode}
%
% \Finale

\endinput
}{}
\ProvidesFile{thm.sty}
             [\FMithmInfo]
%    \end{macrocode}
%
% \begin{macro}{\th@margin}
%    Again this is only a variant of the theorem styles described
%    above without any new ideas.
%    \begin{macrocode}
\gdef\th@margin{\normalfont\slshape
  \def\@begintheorem##1##2{\item
       [\theorem@headerfont \llap{##2}\hskip\labelsep ##1]}%
\def\@opargbegintheorem##1##2##3{%
  \item[\theorem@headerfont \llap{##2}\hskip\labelsep ##1\ (##3)]}}
\endgroup
%</thm>
%    \end{macrocode}
% \end{macro}
%
%
%
%
% \subsection{Final Definitions}
%
% \begin{macro}{\theorempreskipamount}
% \begin{macro}{\theorempostskipamount}
%    The \textsf{skip} parameters that regulate the vertical empty space
%    before and after the {\envfont theorem} environment have
%    to be allocated as well.
% \changes{v2.0f}{89/07/20}{Unnecessary `global' removed.}
%    \begin{macrocode}
%<*package>
\newskip\theorempreskipamount
\newskip\theorempostskipamount
%    \end{macrocode}
%    Since we have used the same values for all theorem sets, we now can
%    assign them.
%    \begin{macrocode}
\global\setlength\theorempreskipamount{12pt plus 5pt minus 3pt}
\global\setlength\theorempostskipamount{8pt plus 3pt minus 1.5pt}
%    \end{macrocode}
% \end{macro}
% \end{macro}
%
%
% \begin{macro}{\@endtheorem}
%    The same holds for the macro |\@endtheorem|, which ends a
%    {\envfont theorem} environment. Since it is the same for all
%    theorem sets, it is removed from the macros |\th@|\meta{style}.
%    It simply ends the {\envfont trivlist} environment, which was
%    begun in |\@thm|.
%    \begin{macrocode}
\global\let\@endtheorem=\endtrivlist
%    \end{macrocode}
% \end{macro}
%
%
%
% \begin{macro}{\@preamblecmds}
%    All macros defined above are to be used only in the preamble.
%    Therefore, we insert them in |\@preamblecmds| which will disable
%    them at begin document. This is done by the internal \LaTeXe{}
%    command |\@onlypreamble|.
% \changes{v2.0g}{89/09/19}{`theoremheaderfont and `theorembodyfont
%    added.}
% \changes{v2.2a}{94/02/03}{Use \LaTeXe{} interface}
%    \begin{macrocode}
\@onlypreamble\@xnthm
\@onlypreamble\@ynthm
\@onlypreamble\@othm
\@onlypreamble\newtheorem
\@onlypreamble\theoremstyle
\@onlypreamble\theorembodyfont
\@onlypreamble\theoremheaderfont
%    \end{macrocode}
% \end{macro}
%
% Finally we declare the \texttt{plain} theorem style to be the default.
% \changes{v2.1b}{90/06/04}{Fixed missing `item by declaring default
%    style.}
%    \begin{macrocode}
\theoremstyle{plain}
%</package>
%    \end{macrocode}
%
% \Finale

\endinput
}{}
\ProvidesFile{thb.sty}
             [\FMithmInfo]
%    \end{macrocode}
%
% \begin{macro}{\th@break}
%    |\theoremstyle{break}| produces a line break after the name
%    of the theorem.  The font is |\slshape|.  Hence, we define
%    |\th@break| as follows:
%    \begin{macrocode}
\gdef\th@break{\normalfont\slshape
  \def\@begintheorem##1##2{\item[%
%    \end{macrocode}
%    We run into the following problem: it is not possible to create
%    the header with |\item[|\meta{title}|]| and then start
%    a new line by, for example, |\mbox{}\\|. Such a definition
%    will fail whenever a list environment follows immediately.  With
%    the above construction, the |\mbox{}| causes the switch
%    |@inlabel| (cf.\ definition of |\list| and
%    |\trivlist| in \cite{latex:source}) to be set to \texttt{false}
%    and so the following list will insert additional vertical
%    space (|\topskip|). This is quite annoying. Therefore, we
%    create the line break within the |\item|. In order to ensure
%    that the text will begin at the proper position in the following
%    line, we simply pretend that the label does not take any
%    room.\footnote{This will lead to problems whenever very high
%    symbols occurring in the line tower into the
%    heading. So, something else has to be done here sometime.}
%    \begin{macrocode}
       \rlap{\vbox{\hbox{\hskip \labelsep\theorem@headerfont ##1\ ##2}%
                   \hbox{\strut}}}]}%
%    \end{macrocode}
%    Again, the definition of |\@opargbegintheorem| is
%    completely analogous.
%    \begin{macrocode}
\def\@opargbegintheorem##1##2##3{%
  \item[\rlap{\vbox{\hbox{\hskip \labelsep \theorem@headerfont
                     ##1\ ##2\ (##3)}%
                    \hbox{\strut}}}]}}
\endgroup
%</thb>
%    \end{macrocode}
% \end{macro}
%
%
% \subsubsection{The \texttt{changebreak} style}
%
%    \begin{macrocode}
% This style option is stored in the file |thcb.sty|.
%    \begin{macrocode}
%<*thcb>
\begingroup \makeatletter
\@ifundefined{theorem@style}{% \iffalse meta-comment
%
% Copyright (C) 1993-2024
%
% The LaTeX Project and any individual authors listed elsewhere
% in this file.
%
% This file is part of the Standard LaTeX `Tools Bundle'.
% -------------------------------------------------------
%
% It may be distributed and/or modified under the
% conditions of the LaTeX Project Public License, either version 1.3c
% of this license or (at your option) any later version.
% The latest version of this license is in
%    https://www.latex-project.org/lppl.txt
% and version 1.3c or later is part of all distributions of LaTeX
% version 2005/12/01 or later.
%
% The list of all files belonging to the LaTeX `Tools Bundle' is
% given in the file `manifest.txt'.
%
% \fi
%
%%
%
%
\def\FMithmInfo{2023/07/05 v2.2c Theorem extension package (FMi)}
%
% \ProvidesFile{theorem.dtx}[\FMithmInfo]
%\iffalse   % this is a METACOMMENT !
%%
%% Package `theorem' to use with LaTeX2e
%% Copyright (C) 1989-1995 Frank Mittelbach, all rights reserved.
%<+package>\@ifundefined{theorem@style}{}{\endinput}
%<+package>\NeedsTeXFormat{LaTeX2e}[1995/06/01]
%<+package>\ProvidesPackage{theorem}
%<+package>         [\FMithmInfo]
%
% \fi
%
% \changes{v1.0}{87/05/07}{First draft.}
% \changes{v1.0a}{87/05/07}{`thistheoremstyle' added.}
% \changes{v1.1a}{87/07/29}{New concept: redefine the internal
%                          \protect\LaTeX{} commands `@xnthm' etc.}
% \changes{v1.1b}{87/07/31}{`theoremfont' used to specify font.}
% \changes{v1.2a}{88/01/08}{`@endtheorem' now globally defined.}
% \changes{v1.2b}{88/01/11}{`newtheorem' only allowed in the preamble.}
% \changes{v1.3a}{88/02/14}{`theoremfont' now a toks register.}
%
% \changes{v2.0a}{88/11/16}{New implementation and new documentation.}
% \changes{v2.0a}{88/11/16}{No layout changes but internal modifications
%                           old substyles won't work.}
% \changes{v2.0b}{89/01/30}{`global' used for all definitions.}
% \changes{v2.0c}{89/06/15}{Documentation translated to English.
%                            Thanks to LLL and Nina}
% \changes{v2.1a}{90/02/15}{Support for new font selection.}
% \changes{v2.2a}{94/02/03}{Cleanup for LaTeX2e}
%
% \DoNotIndex{\@endparenv,\@ifdefinable,\@ifnextchar,\@ifundefined}
% \DoNotIndex{\@namedef,\@nameuse,\@spaces,\@tempa,\@warning}
% \DoNotIndex{\ }
% \DoNotIndex{\begingroup,\bfseries}
% \DoNotIndex{\csname}
% \DoNotIndex{\def,\do,\docdate}
% \DoNotIndex{\endcsname,\endgroup,\endinput,\endtrivlist,\expandafter}
% \DoNotIndex{\filedate,\fileversion}
% \DoNotIndex{\gdef,\global}
% \DoNotIndex{\hbox,\hskip}
% \DoNotIndex{\ignorespaces,\input,\itshape,\item}
% \DoNotIndex{\labelsep,\let,\llap}
% \DoNotIndex{\makeatletter}
% \DoNotIndex{\noexpand}
% \DoNotIndex{\refstepcounter,\relax,\rlap}
% \DoNotIndex{\setlength,\slshape,\space,\string,\strut}
% \DoNotIndex{\the,\trivlist,\typeout}
% \DoNotIndex{\vbox}
% \DoNotIndex{\xdef}
%
% \def\deflabel#1{\ttfamily #1\hfill}
% \def\deflist#1{\begin{list}{}{\settowidth\labelwidth{\ttfamily #1}%
%                               \setlength\leftmargin\labelwidth
%                               \addtolength\leftmargin\labelsep
%                               \let\makelabel\deflabel}}
% \def\enddeflist{\end{list}}
% \def\envfont{\normalfont\ttfamily}
%
% \MakeShortVerb{\|}
%
% \newcommand{\TUB}{{\sl TUGboat\/}}
%
% \GetFileInfo{theorem.dtx}
% \title{An Extension of the
%        \LaTeX{} theorem environment\thanks{This file
%        has version number \fileversion, last
%        revised \filedate.}}
% \author{Frank Mittelbach}
%
% \MaintainedByLaTeXTeam{tools}
% \maketitle
%
% \begin{abstract}
%    The macros described in this paper yield an extension of the
%    \LaTeX{} theorem mechanism. It is designed to satisfy the
%    different requirements of various journals. Thus, the
%    layout of the ``theorems'' can be manipulated by determining a
%    ``style''. This article describes not only the use, but
%    also the definition, of the necessary macros.
% \end{abstract}
%
% \section*{Preface to version 2.2}
%
% For \LaTeXe{} this package did not need any fundamental changes. I
% only modified the messages generated so that theorem layout styles
% will show up with the |\listfiles| command and cleaned the section
% on the New Font Selection Scheme since this is now included in
% \LaTeX.
%
% \section*{Preface to version 2.1}
%
% This version is identical to 2.0g described in \TUB\ 10\#3 except
% for some internal defaults which are now set depending on the used
% font selection scheme.
%
% This was done to avoid unpleasant surprises if the new font selection
% scheme is in force. For further details see
% section~\ref{sec:newfonts} and \cite{bk:GMS94}.
%
% \section{Introduction}
%
% For our purposes here, ``theorems'' are labelled enunciations,
% often set off from the main text by extra space and a font change.
% Theorems, corollaries, conjectures, definitions, and remarks
% are all instances of ``theorems''.  The ``header'' of these structures
% is composed of a label (such as \textsc{Theorem} or \textsc{Remark})
% and a number which serializes an item in the sequence of items
% with the same label.
%
% Shortly after the introduction of \LaTeX{} at the Fachbereich
% Mathematik in Mainz, the desire to manipulate the layout of
% ``theorems'' arose.  In Mainz, the following two conventions came
% into general use:
% \begin{enumerate}
%   \item The number of the theorem is shown in the margin.
%   \item There is a line break at
%         the end of the theorem header.
% \end{enumerate}
%
% Additionally, some journals require different formats which depend on
% the ``sort of theorem'': e.g.\ often remarks and definitions are set
% in |\upshape|, while |\itshape| is employed for main theorems.
%
% Confronted with these requirements, a theorem environment was
% developed in Mainz which allows separate determination of the
% layout of the ``theorems sets'', comparable to |\pagestyle|.
%
%
% \section{The user interface}
% \subsection{Defining new theorem sets}
%
% \DescribeMacro\newtheorem
% As in the original \LaTeX{} version, the command |\newtheorem|
% defines a new ``theorem set'' or ``theorem-like structure''.
% Two required arguments name the new environment and give the text
% to be typeset with each instance of the new ``set'', while
% an optional argument determines how
% the ``set'' is enumerated:
% \begin{description}
%    \item[\ttfamily \bslash newtheorem\{foo\}\{bar\}]
%       The theorem set {\envfont foo} (whose name is \texttt{bar})
%       uses its own counter.
%    \item[\ttfamily \bslash newtheorem\{foo2\}{[foo]}\{bar2\}]
%       The theorem set {\envfont foo2} (printed name \texttt{bar2})
%       uses the same counter as the theorem set \texttt{foo}.
%    \item[\ttfamily \bslash newtheorem\{foo3\}\{bar3\}{[section]}]
%       The theorem set {\envfont foo3} (printed name \texttt{bar3}) is
%       enumerated within the counter \texttt{section}, i.e.\ with every
%       new |\section| the enumeration begins again with 1, and
%       the enumeration is composed from the section-number and the
%       theorem counter itself.
% \end{description}
%
% \DescribeMacro\theoremstyle
% Additionally, the command |\theoremstyle| can define the layout
% of various, or all, theorem sets.  It should be noted that any
% theorem set defined by |\newtheorem| is typeset in the
% |\theoremstyle| that is current at the time of the definition.
% Thus, the following
% \begin{verbatim}
%    \theoremstyle{break}        \newtheorem{Cor}{Corollary}
%    \theoremstyle{plain}        \newtheorem{Exa}{Example}[section]
%\end{verbatim}                             ^^A No blank at line start
% leads to the result that the set {\envfont Cor} is formatted in the
% style \texttt{break}, while the set {\envfont Exa} and all the
% following ones are formatted in the style \texttt{plain}, unless
% another |\theoremstyle| follows.  Since the definitions installed by
% |\newtheorem| are global, one also can limit |\theoremstyle| locally
% by grouping braces.
%
% \DescribeMacro\theorembodyfont
% The choice of the font for the theorem body is completely
% independent of the chosen |\theoremstyle|; this has proven to
% be very advantageous.  For example,
% \begin{verbatim}
%   {\theorembodyfont{\upshape}          \newtheorem{Rem}{Remark}}
%\end{verbatim}
% defines a theorem set {\envfont Rem}, which will be set in
% |\upshape| in the current layout (which in our example is
% \texttt{plain}).  As with |\theoremstyle|, the |\theorembodyfont|
% chosen is that current at the time of |\newtheorem|.  If
% |\theorembodyfont| is not specified or one defines
% |\theorembodyfont{}|, then the font used will be that defined by the
% |\theoremstyle|.
%
% \DescribeMacro\theoremheaderfont It is also possible to customize
% the font used for the theorem headers.  This is, however, a global
% declaration, and therefore there should be at most one
% |\theoremheaderfont| declaration in the preamble.\footnote{If it is
% actually necessary to have different header fonts, one has to define
% new theorem styles (substituting the desired font) or specify the
% information directly in the \texttt{\bslash newtheorem} declaration
% (the unclean variant).}
%
% \DescribeMacro\theorempreskipamount
% \DescribeMacro\theorempostskipamount Two additional parameters
% affect the vertical space around the theorem environments:
% |\theorempreskipamount| and |\theorempostskipamount| define,
% respectively, the spacing before and after such an environment.
% These parameters apply for all theorem sets and can be manipulated
% with the ordinary length macros.  They are rubber lengths
% (`\textsf{skips}'), and therefore can contain \texttt{plus} and
% \texttt{minus} parts.
%
% Since the definition of theorem sets should---most sensibly---be
% placed in the preamble, we only allow installation there.
% It is therefore possible to release the memory used here after
% |\begin{document}|, in order to make room for other
% applications.
%
%
% \subsection{Existing theorem styles}
%
% The following theorem styles exist to date:
% \begin{deflist}{marginbreak}
%    \item[plain]
%       This theorem style emulates the original \LaTeX{} definition,
%       except that additionally the parameters
%       |\theorem...skipamount| are used.
%    \item[break]
%       In this style, the theorem header is followed by a line
%       break.
%    \item[marginbreak]
%       The theorem number is set in the margin, and there is a line
%       break as in \texttt{break}.
%    \item[changebreak]
%       Like \texttt{break}, but with header number and text
%       interchanged.
%    \item[change]
%       Header number and text are interchanged, without a line break.
%    \item[margin]
%       The number is set in the left margin, without a line break.
% \end{deflist}
% All styles (except \texttt{plain}) select |\slshape| as the default
% |\theorembodyfont|.
%
%
% \subsection{Examples}
%
% Given the above theorem sets {\envfont Cor}, {\envfont Exa}
% and {\envfont Rem},
% suppose that the preamble also contains the declarations:
% \begin{verbatim}
%    \theoremstyle{marginbreak}   \newtheorem{Lem}[Cor]{Lemma}
%    \theoremstyle{change}
%    \theorembodyfont{\itshape}        \newtheorem{Def}[Cor]{Definition}
%
%    \theoremheaderfont{\scshape}
%\end{verbatim}
% Then the following are some typical examples of the typeset output
% resulting from their use.
%
% \begin{Cor}
%    This is a sentence typeset in the theorem environment {\envfont
%    Cor}.
% \end{Cor}
%
% \begin{Exa}
%    This is a sentence typeset in the theorem environment {\envfont
%    Exa}.
% \end{Exa}
%
% \begin{Rem}
%    This is a sentence typeset in the theorem environment {\envfont
%    Rem}.
% \end{Rem}
%
% \begin{Lem}[Ben User]
%    This is a sentence typeset in the theorem environment {\envfont
%    Lem}.
% \end{Lem}
%
% \begin{Def}[Very impressive Definition]
%    This is a sentence typeset in the theorem environment {\envfont
%    Def}.
% \end{Def}
%
% The last two examples show the effect of the optional argument to
% a theorem environment (it is the text typeset in parentheses).
%
%
% \section{Special Considerations}
% \label{sec:newfonts}
%
% Theoremheader and body are implemented as a unit. This means that
% the |\theoremheaderfont|  will inherit characteristics of the
% |\theorembodyfont| in \LaTeXe. Thus, if for example
% |\theorembodyfont| is |\itshape| and |\theoremheaderfont|
% is |\bfseries| the font selected for the header will have the
% characteristics `bold extended italic'. If this is not desired one
% should set the |\theoremheaderfont| to something like
% \begin{verbatim}
%      \theoremheaderfont{\normalfont\bfseries}
%\end{verbatim}
% i.e.\ supplying all necessary font information explicitly.
%
%
%
% \section{Acknowledgements}
%
% The publication of this set of macros was only possible with the
% help of Christina Busse (translating the manuscript into English),
% Joachim Pense (playing the r\^ole of typist), Chris Rowley
% (looking everything over) and many others providing useful
% suggestions.
%
% \MaybeStop{%
%    \begin{thebibliography}{9}
%       \bibitem{bk:GMS94} \textsc{M.~Goossens}, \textsc{F.~Mittelbach}
%         and \textsc{A.~Samarin}.
%         \newblock The \LaTeX{} Companion.
%         \newblock
%         Addison-Wesley, Reading, Massachusetts, 1994.
%       \bibitem{latex:source} \textsc{Lamport, Leslie.} \newblock
%                               \texttt{latex.tex}, version 2.09, date
%                               Feb. 1990.
%    \end{thebibliography}
%    \PrintIndex
%    \PrintChanges}
%
%
% \section{The documentation driver file}
%
% The next bit of code contains the documentation driver file for
% \TeX{}, i.e., the file that will produce the documentation you are
% currently reading. It will be extracted from this file by the
% \texttt{docstrip} program. Since it is the first code in the file
% one can alternatively process this file directly with \LaTeXe{} to
% obtain the documentation.
%
%    \begin{macrocode}
%<*driver>
\documentclass{ltxdoc}

\usepackage{theorem}

%<+driver>% The next few lines define theorem sets which are used
%<+driver>% in the example section of the documentation.

\theoremstyle{break}       \newtheorem{Cor}{Corollary}
\theoremstyle{plain}       \newtheorem{Exa}{Example}[section]
{\theorembodyfont{\upshape}\newtheorem{Rem}{Remark}}
\theoremstyle{marginbreak} \newtheorem{Lem}[Cor]{Lemma}
\theoremstyle{change}
\theorembodyfont{\itshape} \newtheorem{Def}[Cor]{Definition}

\theoremheaderfont{\scshape}

\RecordChanges

\begin{document}
   \DocInput{theorem.dtx}
\end{document}
%</driver>
%    \end{macrocode}
%
% \section{Definition of the Macros}
%
% If the file has been loaded before, we abort immediately. If not the
% package announces itself (this is actually done at the very top of
% the file---the way it is done isn't good style so don't copy it).
% \changes{v2.0e}{89/07/19}{Spaces removed from `typeout'.}
% \changes{v2.2b}{95/11/19}{Announce moved to top of file}
%    \begin{macrocode}
%<*package>
%\@ifundefined{theorem@style}{}{\endinput}
%\def\FMithmInfo{1995/11/19 v2.2b Theorem extension package (FMi)}
%\ProvidesPackage{theorem}[\FMithmInfo]
%    \end{macrocode}
%
%
% \subsection{Definition of theorem styles and fonts}
%
% All the definitions in this file are done globally to allow
% inputting this file inside a group.
%
% \begin{macro}{\theoremstyle}
%    Before a theorem style can be installed, the chosen style must be
%    known. For that reason, we must test to see that
%    |\th@|\meta{style} is known or, more precisely, that it is
%    different from |\relax|. If the style is not known then
%    |\th@plain| is used.
%    \begin{macrocode}
\gdef\theoremstyle#1{%
   \@ifundefined{th@#1}{\@warning
          {Unknown theoremstyle `#1'. Using `plain'}%
          \theorem@style{plain}}%
%    \end{macrocode}
%    We save the theorem style to be used in the token register
%    |\theorem@style|.
%    \begin{macrocode}
      {\theorem@style{#1}}%
%    \end{macrocode}
%    Now we ``evaluate'' the theorem style: this means, we call the
%    macro |\th@|\meta{style} which will activate the relevant
%    definitions which are contained in a separate file.  This is done
%    in a group to suppress changes to the current font.  This could
%    otherwise pose problems together with the new font selection
%    scheme\footnote{When I printed the original article using the new
%    font selection scheme I ended with a document with slanted
%    typefaces (text headings and all) simply because one of the
%    theorem styles used \texttt{\bslash sl} at toplevel.} if the
%    |\th@|\meta{style} is evaluated a second time.
% \changes{v2.1a}{90/02/15}{Group added around `th@..}
% \changes{v2.1a}{90/02/15}{Reseting of `..@begintheorem removed.}
%    \begin{macrocode}
      \begingroup
        \csname th@\the\theorem@style \endcsname
      \endgroup}
%    \end{macrocode}
% \end{macro}
%
%
% \begin{macro}{\@begintheorem}
% \changes{v2.1a}{90/02/15}{Relaxed at toplevel.}
% \begin{macro}{\@opargbegintheorem}
% \changes{v2.1a}{90/02/15}{Relaxed at toplevel.}
%    We reset |\@begintheorem| and |\@opargbegintheorem| to
%    |\relax| since these commands are no longer necessary at
%    toplevel. This will save a few tokens.
%    \begin{macrocode}
\global\let\@begintheorem\relax
\global\let\@opargbegintheorem\relax
%    \end{macrocode}
% \end{macro}
% \end{macro}
%
% \begin{macro}{\theorem@style}
%    Obviously the \textsf{token} register used above has to be
%    allocated. To assure the utmost compatibility with the
%    original \LaTeX{} definition, we set the default theorem style to
%    \texttt{plain}, which implements the usual \LaTeX{} convention.
% \changes{v2.0f}{89/07/20}{Unnecessary `global' removed.}
%    \begin{macrocode}
\newtoks\theorem@style
\global\theorem@style{plain}
%    \end{macrocode}
% \end{macro}
%
% \begin{macro}{\theorembodyfont}
% \begin{macro}{\theorem@bodyfont}
%    If the |\theorembodyfont| is set by the user then it should not
%    interact with the default font set in the theorem style. When the
%    new font selection is in force this may happen if, for example,
%    the default is |\itshape| and the new |\theorembodyfont| is
%    |\sffamily|. So we add a |\reset@font| command in front of the
%    user definition.
% \changes{v2.0f}{89/07/20}{Unnecessary `global' removed.}
% \changes{v2.0g}{89/09/19}{`theoremfont to `theorembodyfont.}
% \changes{v2.1c}{92/06/22}{`reset@font added and logic changed.}
% \changes{v2.1d}{92/10/28}{Reset to empty if arg is empty}
%    \begin{macrocode}
\gdef\theorembodyfont#1{%
%    \end{macrocode}
%    We check if the argument supplied is empty and if so put nothing
%    into the |\theorem@bodyfont| token register to allow for
%    |\theorembodyfont{}| as a mean of using the default of the
%    current |\theoremstyle|.
%    \begin{macrocode}
   \def\@tempa{#1}%
   \ifx\@tempa\@empty
    \theorem@bodyfont{}%
   \else
    \theorem@bodyfont{\reset@font#1}%
   \fi
   }
\newtoks\theorem@bodyfont
\global\theorem@bodyfont{}
%    \end{macrocode}
% \end{macro}
% \end{macro}
%
%
% \begin{macro}{\theoremheaderfont}
% \changes{v2.0g}{89/09/19}{Macro added.}
%    The font for the theorem headers is handled differently because
%    this definition applies to all theorem styles.
%    \begin{macrocode}
\gdef\theoremheaderfont#1{\gdef\theorem@headerfont{#1}%
%    \end{macrocode}
%    After using the macro once it is redefined to produce an error
%    message.
%    \begin{macrocode}
       \gdef\theoremheaderfont##1{%
        \typeout{\string\theoremheaderfont\space should be used
                 only once.}}}
%    \end{macrocode}
% \end{macro}
%
% \begin{macro}{\theorem@headerfont}
% \changes{v2.0g}{89/09/19}{Macro added.}
%    To set the |\theorem@headerfont| default we first test if
%    the new fontselection scheme is in force.
% \changes{v2.1a}{90/02/15}{Support for new font selection.}
%    \begin{macrocode}
\ifx\upshape\undefined
%    \end{macrocode}
%    If not we define it to expand into |\bfseries|. We don't use
%    |\let| just in case a following style option redefines this
%    macro.
%    \begin{macrocode}
\gdef\theorem@headerfont{\bfseries}
%    \end{macrocode}
%    Otherwise we reset the current shape before calling |\bfseries|.
%    \begin{macrocode}
\else \gdef\theorem@headerfont{\normalfont\bfseries}\fi
%    \end{macrocode}
% \end{macro}
%
% \begin{macro}{\th@plain}
% \begin{macro}{\th@break}
% \begin{macro}{\th@marginbreak}
% \begin{macro}{\th@changebreak}
% \begin{macro}{\th@change}
% \begin{macro}{\th@margin}
%    The different styles are defined in macros such as
%    |\th@plain|.  Since memory space is precious in
%    ``non-Big-versions'', we have to avoid offering too many unused
%    definitions.  Therefore we define these styles in separate files
%    that can be loaded on demand.  Thus the commands themselves only
%    load these files. We use |\@input@|, a \LaTeXe{} internal command
%    that ensures that the file will be listed with |\listfiles|.
% \changes{v2.2a}{94/02/02}{Use `@input@ to load theorem layout files}
%    \begin{macrocode}
\gdef\th@plain{\@input@{thp.sty}}
\gdef\th@break{\@input@{thb.sty}}
\gdef\th@marginbreak{\@input@{thmb.sty}}
\gdef\th@changebreak{\@input@{thcb.sty}}
\gdef\th@change{\@input@{thc.sty}}
\gdef\th@margin{\@input@{thm.sty}}
%    \end{macrocode}
%    This list will be expanded when new styles become
%    available.  For testing,  just append new theorem substyles
%    as document options.
% \end{macro}
% \end{macro}
% \end{macro}
% \end{macro}
% \end{macro}
% \end{macro}
%
%
%
% \subsection{Definition of a new theorem set}
%
% As already pointed out, a new theorem environment can be defined in
% three different ways:
% \begin{verbatim}
%    \newtheorem{Lem}{Lemma}
%    \newtheorem{Lem}{Lemma}[section]
%    \newtheorem{Lem}[Theorem]{Lemma}
%\end{verbatim}
% The function of the macro |\newtheorem| is to recognize these cases
% and then to branch into one of the three macros |\@ynthm|, |\@xnthm|
% or |\@othm|.  This mechanism is adopted unchanged from
% \cite{latex:source}; the essential point here is that, for example,
% in the second case, the arguments \texttt{Lem}, \texttt{Lemma} and
% \texttt{section} are passed over to the macro |\@xnthm|.
%
% We inspect this case first because the others present fewer problems,
% and thus are easily derived from this one.
%
% \begin{macro}{\@xnthm}
%    For our example arguments, the macro |\@xnthm| must fulfill
%    the following:
%    \begin{itemize}
%       \item
%          Define a new \LaTeX-counter `Lem'
%       \item
%          reset this counter within a |\section|
%       \item
%          define the macro |\theLem|
%       \item
%          define the environment macros |\Lem| and
%          |\endLem| using the current
%          |\theoremstyle| and |\theorem@bodyfont|.
%    \end{itemize}
%    Obviously, all this should happen only if the first argument
%    of |\@xnthm| (i.e.\ \texttt{Lem} in our example) is chosen so
%    as not to conflict with any previously defined commands or
%    environments.  This test is performed by the \LaTeX{} macro
%    |\@ifdefinable|.
%    \begin{macrocode}
\gdef\@xnthm#1#2[#3]{\expandafter\@ifdefinable\csname #1\endcsname
%    \end{macrocode}
%    Therefore, the first argument of |\@ifdefinable| is the
%    expansion (in the example, |\Lem|) of
%    |\csname||#1\endcsname|.  The second argument is
%    executed only if the test has been completed successfully.
%    \begin{macrocode}
   {%
%    \end{macrocode}
%    Now we define the new counter. The names of the \LaTeX{}
%    macros employed should speak for themselves:
% \changes{v2.2b}{1995/10/07}{Use \cs{@newctr}}
% \changes{v2.2c}{1995/11/23}{Use \cs{@newctr} properly.}
%    \begin{macrocode}
    \@definecounter{#1}\@newctr{#1}[#3]%
%    \end{macrocode}
%    Using |\@newctr| will give a proper error message if the counter
%    in |#3| is not defined.
%    In defining `|\theLem|' we must generate the desired macro
%    name by use of |\expandafter| and |\csname|.
%    \begin{macrocode}
    \expandafter\xdef\csname the#1\endcsname
%    \end{macrocode}
%    An |\xdef| is used in order to make the definition global,
%    and to ensure that it contains the replacement texts of
%    |\@thmcountersep| and |\@thmcounter|.\footnote{These
%    two macros can be defined by the document style.  Their default
%    values produce a `\texttt{.}' as separation and an arabic
%    representation of the number.} However, not everything should be
%    expanded. For example, it saves space to use |\thesection|
%    instead of its---at times---lengthy expansion.
%    \begin{macrocode}
      {\expandafter \noexpand \csname the#3\endcsname
       \@thmcountersep \@thmcounter{#1}}%
%    \end{macrocode}
%    Thus with the defaults of \LaTeX{}, |\theLem| would be
%    replaced by the command sequence |\thesection.\arabic{Lem}|.
%
%    We will now look at the definition of the macro which is executed
%    at the beginning of the actual environment (in our example this
%    macro is |\Lem|). It should be noted that we use an
%    ``|\expandafter| trick'' to expand only certain parts of the
%    replacement text at the time of the definition.
%    \begin{macrocode}
    \def\@tempa{\global\@namedef{#1}}%
    \expandafter \@tempa \expandafter{%
%    \end{macrocode}
%    First, the macro that contains the current definitions of
%    |\@begintheorem| and |\@opargtheorem| should be called
%    up.  The name of this macro---as is already known---has the form
%    |\th@|\meta{theorem\ style}; therefore, it must be called by
%    \begin{macrocode}
      \csname th@\the \theorem@style
%    \end{macrocode}
%    In addition the default theorem font should be changeable, i.e.\
%    we have to insert the contents of |\theorem@bodyfont|. For
%    that reason, we expand even further, beyond |\endcsname|,
%    and thus insert the contents of the \textsf{token} register
%    |\theorem@bodyfont| in the replacement text.
%    \begin{macrocode}
            \expandafter \endcsname \the \theorem@bodyfont
%    \end{macrocode}
%    Now it is time to call the macro |\@thm| which takes over
%    the further processing.  It has two arguments: the current
%    counter name (in our example, \texttt{Lem}), and the text of the
%    label (in our example, \texttt{Lemma}).
%    \begin{macrocode}
     \@thm{#1}{#2}}%
%    \end{macrocode}
%    With this, the `sub-definition' is complete.  The macro
%    |\@endtheorem| ends a theorem environment and is, so far, nothing
%    but an |\endtrivlist|.  (Hence it is defined globally, and not
%    within the theorem styles.\footnote{This has to be changed as
%    soon as theorem styles that change \texttt{\bslash @endtheorem}
%    exist.  In such a case, all existing styles must be changed as
%    well since they will have to reset the macro.}) Therefore, we can
%    set it equivalent to the macro that ends the theorem set (in our
%    example, |\endLem|).  However, if some day theorem styles exist
%    that do change |\@endtheorem|, we would have to use the
%    commented-out line instead.
%    \begin{macrocode}
    \global \expandafter \let \csname end#1\endcsname \@endtheorem
%   \global\@namedef{end#1}{\@endtheorem}%
%    \end{macrocode}
%    With these commands all the required definitions are employed,
%    unless the test |\@ifdefinable| has failed.  Therefore, we
%    end the second argument of this macro and with it the definition
%    of |\@xnthm|.
%    \begin{macrocode}
   }}
%    \end{macrocode}
% \end{macro}
%
% \begin{macro}{\@ynthm}
%    The definition of |\@ynthm| is completely analogous. In
%    this case the new counter that is defined is not reset within
%    another counter; thus the definition of |\the...|
%    is simplified:
%    \begin{macrocode}
\gdef\@ynthm#1#2{\expandafter\@ifdefinable\csname #1\endcsname
   {\@definecounter{#1}%
    \expandafter\xdef\csname the#1\endcsname{\@thmcounter{#1}}%
%    \end{macrocode}
%    The rest of the definition corresponds literally to that of
%    |\@xnthm|:
%    \begin{macrocode}
    \def\@tempa{\global\@namedef{#1}}\expandafter \@tempa
     \expandafter{\csname th@\the \theorem@style \expandafter
     \endcsname \the\theorem@bodyfont \@thm{#1}{#2}}%
    \global \expandafter \let \csname end#1\endcsname \@endtheorem}}
%    \end{macrocode}
% \end{macro}
%
% \begin{macro}{\@othm}
%    The definition of |\@othm| does not contain anything new.
%    \begin{macrocode}
\gdef\@othm#1[#2]#3{%
%    \end{macrocode}
% \changes{v2.2b}{95/10/07}{Build in check for counter defined}
%    We do not define a new counter but instead use one that has
%    already been defined.  Thus the only definition we need is that
%    of this pseudo-counter (i.e.\ |\the|\meta{env.\ name}).
%    First we check if |#2| corresponds to a known counter name.
%    \begin{macrocode}
  \expandafter\ifx\csname c@#2\endcsname\relax
   \@nocounterr{#2}%
  \else
   \expandafter\@ifdefinable\csname #1\endcsname
   {\expandafter \xdef \csname the#1\endcsname
     {\expandafter \noexpand \csname the#2\endcsname}%
%    \end{macrocode}
%    All other parts of the definition can be adopted from
%    |\@xnthm|. We have to remember, though, that in this case
%    the name of the current counter and the theorem label have moved
%    to the second and third arguments.
%    \begin{macrocode}
    \def\@tempa{\global\@namedef{#1}}\expandafter \@tempa
     \expandafter{\csname th@\the \theorem@style \expandafter
     \endcsname \the\theorem@bodyfont \@thm{#2}{#3}}%
    \global \expandafter \let \csname end#1\endcsname \@endtheorem}%
  \fi}
%    \end{macrocode}
% \end{macro}
%
% \subsection{Macros that are employed in a {\envfont theorem}
%    environment}
%
% \begin{macro}{\@thm}
%    The macro |\@thm| has to increase the current counter.  Then,
%    depending on whether the environment has (or does not have) an
%    optional argument, it has to branch into either |\@begintheorem|
%    or |\@opargtheorem|.
%    \begin{macrocode}
\gdef\@thm#1#2{\refstepcounter{#1}%
%    \end{macrocode}
%    Now we start a {\envfont trivlist} environment, and give
%    |\@topsep| and |\@topsepadd| the values of the \textsf{skip}
%    registers |\theorempreskipamount| and |\theorempostskipamount|.
%    The value in |\@topsep| is the vertical space that is inserted by
%    the first (and only) |\item| in our |\trivlist| whilst
%    |\@topsepadd| is inserted by |\@endparenv| at the end of that
%    {\envfont trivlist} environment.  By using these registers, we
%    obtain the desired space around a {\envfont theorem} environment.
%    \begin{macrocode}
   \trivlist
   \@topsep \theorempreskipamount               % used by first \item
   \@topsepadd \theorempostskipamount           % used by \@endparenv
%    \end{macrocode}
%    Now we have to test whether an optional argument has been given.
% \changes{v2.0d}{89/07/18}{Forgotten percent added.}
%    \begin{macrocode}
   \@ifnextchar [%
%    \end{macrocode}
%    If there is an optional argument, we will call |\@ythm|, and
%    move the arguments read back into the input stream.
%    \begin{macrocode}
   {\@ythm{#1}{#2}}%
%    \end{macrocode}
%    If not, we call |\@begintheorem|. Its first argument is the
%    name of the theorem set (hence the second argument of
%    |\@thm|). Its second argument is the macro that produces the
%    current number.
%    \begin{macrocode}
   {\@begintheorem{#2}{\csname the#1\endcsname}\ignorespaces}}
%    \end{macrocode}
% \end{macro}
%
% \begin{macro}{\@xthm}
% \begin{macro}{\@ythm}
%    Both these macros were originally called by |\@thm|.  We do
%    not need |\@xthm| anymore, hence we reset it to
%    |\relax|. The definition of |\@ythm| has not changed at
%    all from its definition in \LaTeX. In order to make the macros
%    easier to understand, we will
%    nevertheless present it (commented out).
%    \begin{macrocode}
\global\let\@xthm\relax
% \def\@ythm#1#2[#3]{\@opargbegintheorem{#2}{\csname
%           the#1\endcsname}{#3}\ignorespaces}
%</package>
%    \end{macrocode}
%    The primitive |\ignorespaces| in |\@ythm| and
%    |\@thm| is needed to remove the spaces between the
%    |\begin{...}| and the actual text.
% \end{macro}
% \end{macro}
%
%
% \subsection{Definition of the theorem substyles}
%
% As already pointed out, the theorem substyles, defined below, are
% only loaded when necessary.  Note that all these substyles, except
% \texttt{plain}, have |\slshape| as the default body font.
%
% \subsubsection{The \texttt{plain} style}
%
% \label{dok:th@plain} As the following macros use |@|, we have
% to locally set the |\catcode| of this symbol to ``letter''.
% This happens within a group, so that we do not have to worry about
% which |\catcode| that symbol had before.
%    \begin{macrocode}
%<*thp>
\begingroup \makeatletter
%    \end{macrocode}
% Since we are now within a group, we must make all definitions
% globally.  First we make sure that \texttt{theorem.sty} is loaded.
% This will allow us to use this file as a document style option
% without having to call \texttt{theorem} itself as an option.  At the
% same time, we assure that at least version 2 is loaded, since
% |\theorem@style| was not defined in earlier versions.
%    \begin{macrocode}
\@ifundefined{theorem@style}{% \iffalse meta-comment
%
% Copyright (C) 1993-2024
%
% The LaTeX Project and any individual authors listed elsewhere
% in this file.
%
% This file is part of the Standard LaTeX `Tools Bundle'.
% -------------------------------------------------------
%
% It may be distributed and/or modified under the
% conditions of the LaTeX Project Public License, either version 1.3c
% of this license or (at your option) any later version.
% The latest version of this license is in
%    https://www.latex-project.org/lppl.txt
% and version 1.3c or later is part of all distributions of LaTeX
% version 2005/12/01 or later.
%
% The list of all files belonging to the LaTeX `Tools Bundle' is
% given in the file `manifest.txt'.
%
% \fi
%
%%
%
%
\def\FMithmInfo{2023/07/05 v2.2c Theorem extension package (FMi)}
%
% \ProvidesFile{theorem.dtx}[\FMithmInfo]
%\iffalse   % this is a METACOMMENT !
%%
%% Package `theorem' to use with LaTeX2e
%% Copyright (C) 1989-1995 Frank Mittelbach, all rights reserved.
%<+package>\@ifundefined{theorem@style}{}{\endinput}
%<+package>\NeedsTeXFormat{LaTeX2e}[1995/06/01]
%<+package>\ProvidesPackage{theorem}
%<+package>         [\FMithmInfo]
%
% \fi
%
% \changes{v1.0}{87/05/07}{First draft.}
% \changes{v1.0a}{87/05/07}{`thistheoremstyle' added.}
% \changes{v1.1a}{87/07/29}{New concept: redefine the internal
%                          \protect\LaTeX{} commands `@xnthm' etc.}
% \changes{v1.1b}{87/07/31}{`theoremfont' used to specify font.}
% \changes{v1.2a}{88/01/08}{`@endtheorem' now globally defined.}
% \changes{v1.2b}{88/01/11}{`newtheorem' only allowed in the preamble.}
% \changes{v1.3a}{88/02/14}{`theoremfont' now a toks register.}
%
% \changes{v2.0a}{88/11/16}{New implementation and new documentation.}
% \changes{v2.0a}{88/11/16}{No layout changes but internal modifications
%                           old substyles won't work.}
% \changes{v2.0b}{89/01/30}{`global' used for all definitions.}
% \changes{v2.0c}{89/06/15}{Documentation translated to English.
%                            Thanks to LLL and Nina}
% \changes{v2.1a}{90/02/15}{Support for new font selection.}
% \changes{v2.2a}{94/02/03}{Cleanup for LaTeX2e}
%
% \DoNotIndex{\@endparenv,\@ifdefinable,\@ifnextchar,\@ifundefined}
% \DoNotIndex{\@namedef,\@nameuse,\@spaces,\@tempa,\@warning}
% \DoNotIndex{\ }
% \DoNotIndex{\begingroup,\bfseries}
% \DoNotIndex{\csname}
% \DoNotIndex{\def,\do,\docdate}
% \DoNotIndex{\endcsname,\endgroup,\endinput,\endtrivlist,\expandafter}
% \DoNotIndex{\filedate,\fileversion}
% \DoNotIndex{\gdef,\global}
% \DoNotIndex{\hbox,\hskip}
% \DoNotIndex{\ignorespaces,\input,\itshape,\item}
% \DoNotIndex{\labelsep,\let,\llap}
% \DoNotIndex{\makeatletter}
% \DoNotIndex{\noexpand}
% \DoNotIndex{\refstepcounter,\relax,\rlap}
% \DoNotIndex{\setlength,\slshape,\space,\string,\strut}
% \DoNotIndex{\the,\trivlist,\typeout}
% \DoNotIndex{\vbox}
% \DoNotIndex{\xdef}
%
% \def\deflabel#1{\ttfamily #1\hfill}
% \def\deflist#1{\begin{list}{}{\settowidth\labelwidth{\ttfamily #1}%
%                               \setlength\leftmargin\labelwidth
%                               \addtolength\leftmargin\labelsep
%                               \let\makelabel\deflabel}}
% \def\enddeflist{\end{list}}
% \def\envfont{\normalfont\ttfamily}
%
% \MakeShortVerb{\|}
%
% \newcommand{\TUB}{{\sl TUGboat\/}}
%
% \GetFileInfo{theorem.dtx}
% \title{An Extension of the
%        \LaTeX{} theorem environment\thanks{This file
%        has version number \fileversion, last
%        revised \filedate.}}
% \author{Frank Mittelbach}
%
% \MaintainedByLaTeXTeam{tools}
% \maketitle
%
% \begin{abstract}
%    The macros described in this paper yield an extension of the
%    \LaTeX{} theorem mechanism. It is designed to satisfy the
%    different requirements of various journals. Thus, the
%    layout of the ``theorems'' can be manipulated by determining a
%    ``style''. This article describes not only the use, but
%    also the definition, of the necessary macros.
% \end{abstract}
%
% \section*{Preface to version 2.2}
%
% For \LaTeXe{} this package did not need any fundamental changes. I
% only modified the messages generated so that theorem layout styles
% will show up with the |\listfiles| command and cleaned the section
% on the New Font Selection Scheme since this is now included in
% \LaTeX.
%
% \section*{Preface to version 2.1}
%
% This version is identical to 2.0g described in \TUB\ 10\#3 except
% for some internal defaults which are now set depending on the used
% font selection scheme.
%
% This was done to avoid unpleasant surprises if the new font selection
% scheme is in force. For further details see
% section~\ref{sec:newfonts} and \cite{bk:GMS94}.
%
% \section{Introduction}
%
% For our purposes here, ``theorems'' are labelled enunciations,
% often set off from the main text by extra space and a font change.
% Theorems, corollaries, conjectures, definitions, and remarks
% are all instances of ``theorems''.  The ``header'' of these structures
% is composed of a label (such as \textsc{Theorem} or \textsc{Remark})
% and a number which serializes an item in the sequence of items
% with the same label.
%
% Shortly after the introduction of \LaTeX{} at the Fachbereich
% Mathematik in Mainz, the desire to manipulate the layout of
% ``theorems'' arose.  In Mainz, the following two conventions came
% into general use:
% \begin{enumerate}
%   \item The number of the theorem is shown in the margin.
%   \item There is a line break at
%         the end of the theorem header.
% \end{enumerate}
%
% Additionally, some journals require different formats which depend on
% the ``sort of theorem'': e.g.\ often remarks and definitions are set
% in |\upshape|, while |\itshape| is employed for main theorems.
%
% Confronted with these requirements, a theorem environment was
% developed in Mainz which allows separate determination of the
% layout of the ``theorems sets'', comparable to |\pagestyle|.
%
%
% \section{The user interface}
% \subsection{Defining new theorem sets}
%
% \DescribeMacro\newtheorem
% As in the original \LaTeX{} version, the command |\newtheorem|
% defines a new ``theorem set'' or ``theorem-like structure''.
% Two required arguments name the new environment and give the text
% to be typeset with each instance of the new ``set'', while
% an optional argument determines how
% the ``set'' is enumerated:
% \begin{description}
%    \item[\ttfamily \bslash newtheorem\{foo\}\{bar\}]
%       The theorem set {\envfont foo} (whose name is \texttt{bar})
%       uses its own counter.
%    \item[\ttfamily \bslash newtheorem\{foo2\}{[foo]}\{bar2\}]
%       The theorem set {\envfont foo2} (printed name \texttt{bar2})
%       uses the same counter as the theorem set \texttt{foo}.
%    \item[\ttfamily \bslash newtheorem\{foo3\}\{bar3\}{[section]}]
%       The theorem set {\envfont foo3} (printed name \texttt{bar3}) is
%       enumerated within the counter \texttt{section}, i.e.\ with every
%       new |\section| the enumeration begins again with 1, and
%       the enumeration is composed from the section-number and the
%       theorem counter itself.
% \end{description}
%
% \DescribeMacro\theoremstyle
% Additionally, the command |\theoremstyle| can define the layout
% of various, or all, theorem sets.  It should be noted that any
% theorem set defined by |\newtheorem| is typeset in the
% |\theoremstyle| that is current at the time of the definition.
% Thus, the following
% \begin{verbatim}
%    \theoremstyle{break}        \newtheorem{Cor}{Corollary}
%    \theoremstyle{plain}        \newtheorem{Exa}{Example}[section]
%\end{verbatim}                             ^^A No blank at line start
% leads to the result that the set {\envfont Cor} is formatted in the
% style \texttt{break}, while the set {\envfont Exa} and all the
% following ones are formatted in the style \texttt{plain}, unless
% another |\theoremstyle| follows.  Since the definitions installed by
% |\newtheorem| are global, one also can limit |\theoremstyle| locally
% by grouping braces.
%
% \DescribeMacro\theorembodyfont
% The choice of the font for the theorem body is completely
% independent of the chosen |\theoremstyle|; this has proven to
% be very advantageous.  For example,
% \begin{verbatim}
%   {\theorembodyfont{\upshape}          \newtheorem{Rem}{Remark}}
%\end{verbatim}
% defines a theorem set {\envfont Rem}, which will be set in
% |\upshape| in the current layout (which in our example is
% \texttt{plain}).  As with |\theoremstyle|, the |\theorembodyfont|
% chosen is that current at the time of |\newtheorem|.  If
% |\theorembodyfont| is not specified or one defines
% |\theorembodyfont{}|, then the font used will be that defined by the
% |\theoremstyle|.
%
% \DescribeMacro\theoremheaderfont It is also possible to customize
% the font used for the theorem headers.  This is, however, a global
% declaration, and therefore there should be at most one
% |\theoremheaderfont| declaration in the preamble.\footnote{If it is
% actually necessary to have different header fonts, one has to define
% new theorem styles (substituting the desired font) or specify the
% information directly in the \texttt{\bslash newtheorem} declaration
% (the unclean variant).}
%
% \DescribeMacro\theorempreskipamount
% \DescribeMacro\theorempostskipamount Two additional parameters
% affect the vertical space around the theorem environments:
% |\theorempreskipamount| and |\theorempostskipamount| define,
% respectively, the spacing before and after such an environment.
% These parameters apply for all theorem sets and can be manipulated
% with the ordinary length macros.  They are rubber lengths
% (`\textsf{skips}'), and therefore can contain \texttt{plus} and
% \texttt{minus} parts.
%
% Since the definition of theorem sets should---most sensibly---be
% placed in the preamble, we only allow installation there.
% It is therefore possible to release the memory used here after
% |\begin{document}|, in order to make room for other
% applications.
%
%
% \subsection{Existing theorem styles}
%
% The following theorem styles exist to date:
% \begin{deflist}{marginbreak}
%    \item[plain]
%       This theorem style emulates the original \LaTeX{} definition,
%       except that additionally the parameters
%       |\theorem...skipamount| are used.
%    \item[break]
%       In this style, the theorem header is followed by a line
%       break.
%    \item[marginbreak]
%       The theorem number is set in the margin, and there is a line
%       break as in \texttt{break}.
%    \item[changebreak]
%       Like \texttt{break}, but with header number and text
%       interchanged.
%    \item[change]
%       Header number and text are interchanged, without a line break.
%    \item[margin]
%       The number is set in the left margin, without a line break.
% \end{deflist}
% All styles (except \texttt{plain}) select |\slshape| as the default
% |\theorembodyfont|.
%
%
% \subsection{Examples}
%
% Given the above theorem sets {\envfont Cor}, {\envfont Exa}
% and {\envfont Rem},
% suppose that the preamble also contains the declarations:
% \begin{verbatim}
%    \theoremstyle{marginbreak}   \newtheorem{Lem}[Cor]{Lemma}
%    \theoremstyle{change}
%    \theorembodyfont{\itshape}        \newtheorem{Def}[Cor]{Definition}
%
%    \theoremheaderfont{\scshape}
%\end{verbatim}
% Then the following are some typical examples of the typeset output
% resulting from their use.
%
% \begin{Cor}
%    This is a sentence typeset in the theorem environment {\envfont
%    Cor}.
% \end{Cor}
%
% \begin{Exa}
%    This is a sentence typeset in the theorem environment {\envfont
%    Exa}.
% \end{Exa}
%
% \begin{Rem}
%    This is a sentence typeset in the theorem environment {\envfont
%    Rem}.
% \end{Rem}
%
% \begin{Lem}[Ben User]
%    This is a sentence typeset in the theorem environment {\envfont
%    Lem}.
% \end{Lem}
%
% \begin{Def}[Very impressive Definition]
%    This is a sentence typeset in the theorem environment {\envfont
%    Def}.
% \end{Def}
%
% The last two examples show the effect of the optional argument to
% a theorem environment (it is the text typeset in parentheses).
%
%
% \section{Special Considerations}
% \label{sec:newfonts}
%
% Theoremheader and body are implemented as a unit. This means that
% the |\theoremheaderfont|  will inherit characteristics of the
% |\theorembodyfont| in \LaTeXe. Thus, if for example
% |\theorembodyfont| is |\itshape| and |\theoremheaderfont|
% is |\bfseries| the font selected for the header will have the
% characteristics `bold extended italic'. If this is not desired one
% should set the |\theoremheaderfont| to something like
% \begin{verbatim}
%      \theoremheaderfont{\normalfont\bfseries}
%\end{verbatim}
% i.e.\ supplying all necessary font information explicitly.
%
%
%
% \section{Acknowledgements}
%
% The publication of this set of macros was only possible with the
% help of Christina Busse (translating the manuscript into English),
% Joachim Pense (playing the r\^ole of typist), Chris Rowley
% (looking everything over) and many others providing useful
% suggestions.
%
% \MaybeStop{%
%    \begin{thebibliography}{9}
%       \bibitem{bk:GMS94} \textsc{M.~Goossens}, \textsc{F.~Mittelbach}
%         and \textsc{A.~Samarin}.
%         \newblock The \LaTeX{} Companion.
%         \newblock
%         Addison-Wesley, Reading, Massachusetts, 1994.
%       \bibitem{latex:source} \textsc{Lamport, Leslie.} \newblock
%                               \texttt{latex.tex}, version 2.09, date
%                               Feb. 1990.
%    \end{thebibliography}
%    \PrintIndex
%    \PrintChanges}
%
%
% \section{The documentation driver file}
%
% The next bit of code contains the documentation driver file for
% \TeX{}, i.e., the file that will produce the documentation you are
% currently reading. It will be extracted from this file by the
% \texttt{docstrip} program. Since it is the first code in the file
% one can alternatively process this file directly with \LaTeXe{} to
% obtain the documentation.
%
%    \begin{macrocode}
%<*driver>
\documentclass{ltxdoc}

\usepackage{theorem}

%<+driver>% The next few lines define theorem sets which are used
%<+driver>% in the example section of the documentation.

\theoremstyle{break}       \newtheorem{Cor}{Corollary}
\theoremstyle{plain}       \newtheorem{Exa}{Example}[section]
{\theorembodyfont{\upshape}\newtheorem{Rem}{Remark}}
\theoremstyle{marginbreak} \newtheorem{Lem}[Cor]{Lemma}
\theoremstyle{change}
\theorembodyfont{\itshape} \newtheorem{Def}[Cor]{Definition}

\theoremheaderfont{\scshape}

\RecordChanges

\begin{document}
   \DocInput{theorem.dtx}
\end{document}
%</driver>
%    \end{macrocode}
%
% \section{Definition of the Macros}
%
% If the file has been loaded before, we abort immediately. If not the
% package announces itself (this is actually done at the very top of
% the file---the way it is done isn't good style so don't copy it).
% \changes{v2.0e}{89/07/19}{Spaces removed from `typeout'.}
% \changes{v2.2b}{95/11/19}{Announce moved to top of file}
%    \begin{macrocode}
%<*package>
%\@ifundefined{theorem@style}{}{\endinput}
%\def\FMithmInfo{1995/11/19 v2.2b Theorem extension package (FMi)}
%\ProvidesPackage{theorem}[\FMithmInfo]
%    \end{macrocode}
%
%
% \subsection{Definition of theorem styles and fonts}
%
% All the definitions in this file are done globally to allow
% inputting this file inside a group.
%
% \begin{macro}{\theoremstyle}
%    Before a theorem style can be installed, the chosen style must be
%    known. For that reason, we must test to see that
%    |\th@|\meta{style} is known or, more precisely, that it is
%    different from |\relax|. If the style is not known then
%    |\th@plain| is used.
%    \begin{macrocode}
\gdef\theoremstyle#1{%
   \@ifundefined{th@#1}{\@warning
          {Unknown theoremstyle `#1'. Using `plain'}%
          \theorem@style{plain}}%
%    \end{macrocode}
%    We save the theorem style to be used in the token register
%    |\theorem@style|.
%    \begin{macrocode}
      {\theorem@style{#1}}%
%    \end{macrocode}
%    Now we ``evaluate'' the theorem style: this means, we call the
%    macro |\th@|\meta{style} which will activate the relevant
%    definitions which are contained in a separate file.  This is done
%    in a group to suppress changes to the current font.  This could
%    otherwise pose problems together with the new font selection
%    scheme\footnote{When I printed the original article using the new
%    font selection scheme I ended with a document with slanted
%    typefaces (text headings and all) simply because one of the
%    theorem styles used \texttt{\bslash sl} at toplevel.} if the
%    |\th@|\meta{style} is evaluated a second time.
% \changes{v2.1a}{90/02/15}{Group added around `th@..}
% \changes{v2.1a}{90/02/15}{Reseting of `..@begintheorem removed.}
%    \begin{macrocode}
      \begingroup
        \csname th@\the\theorem@style \endcsname
      \endgroup}
%    \end{macrocode}
% \end{macro}
%
%
% \begin{macro}{\@begintheorem}
% \changes{v2.1a}{90/02/15}{Relaxed at toplevel.}
% \begin{macro}{\@opargbegintheorem}
% \changes{v2.1a}{90/02/15}{Relaxed at toplevel.}
%    We reset |\@begintheorem| and |\@opargbegintheorem| to
%    |\relax| since these commands are no longer necessary at
%    toplevel. This will save a few tokens.
%    \begin{macrocode}
\global\let\@begintheorem\relax
\global\let\@opargbegintheorem\relax
%    \end{macrocode}
% \end{macro}
% \end{macro}
%
% \begin{macro}{\theorem@style}
%    Obviously the \textsf{token} register used above has to be
%    allocated. To assure the utmost compatibility with the
%    original \LaTeX{} definition, we set the default theorem style to
%    \texttt{plain}, which implements the usual \LaTeX{} convention.
% \changes{v2.0f}{89/07/20}{Unnecessary `global' removed.}
%    \begin{macrocode}
\newtoks\theorem@style
\global\theorem@style{plain}
%    \end{macrocode}
% \end{macro}
%
% \begin{macro}{\theorembodyfont}
% \begin{macro}{\theorem@bodyfont}
%    If the |\theorembodyfont| is set by the user then it should not
%    interact with the default font set in the theorem style. When the
%    new font selection is in force this may happen if, for example,
%    the default is |\itshape| and the new |\theorembodyfont| is
%    |\sffamily|. So we add a |\reset@font| command in front of the
%    user definition.
% \changes{v2.0f}{89/07/20}{Unnecessary `global' removed.}
% \changes{v2.0g}{89/09/19}{`theoremfont to `theorembodyfont.}
% \changes{v2.1c}{92/06/22}{`reset@font added and logic changed.}
% \changes{v2.1d}{92/10/28}{Reset to empty if arg is empty}
%    \begin{macrocode}
\gdef\theorembodyfont#1{%
%    \end{macrocode}
%    We check if the argument supplied is empty and if so put nothing
%    into the |\theorem@bodyfont| token register to allow for
%    |\theorembodyfont{}| as a mean of using the default of the
%    current |\theoremstyle|.
%    \begin{macrocode}
   \def\@tempa{#1}%
   \ifx\@tempa\@empty
    \theorem@bodyfont{}%
   \else
    \theorem@bodyfont{\reset@font#1}%
   \fi
   }
\newtoks\theorem@bodyfont
\global\theorem@bodyfont{}
%    \end{macrocode}
% \end{macro}
% \end{macro}
%
%
% \begin{macro}{\theoremheaderfont}
% \changes{v2.0g}{89/09/19}{Macro added.}
%    The font for the theorem headers is handled differently because
%    this definition applies to all theorem styles.
%    \begin{macrocode}
\gdef\theoremheaderfont#1{\gdef\theorem@headerfont{#1}%
%    \end{macrocode}
%    After using the macro once it is redefined to produce an error
%    message.
%    \begin{macrocode}
       \gdef\theoremheaderfont##1{%
        \typeout{\string\theoremheaderfont\space should be used
                 only once.}}}
%    \end{macrocode}
% \end{macro}
%
% \begin{macro}{\theorem@headerfont}
% \changes{v2.0g}{89/09/19}{Macro added.}
%    To set the |\theorem@headerfont| default we first test if
%    the new fontselection scheme is in force.
% \changes{v2.1a}{90/02/15}{Support for new font selection.}
%    \begin{macrocode}
\ifx\upshape\undefined
%    \end{macrocode}
%    If not we define it to expand into |\bfseries|. We don't use
%    |\let| just in case a following style option redefines this
%    macro.
%    \begin{macrocode}
\gdef\theorem@headerfont{\bfseries}
%    \end{macrocode}
%    Otherwise we reset the current shape before calling |\bfseries|.
%    \begin{macrocode}
\else \gdef\theorem@headerfont{\normalfont\bfseries}\fi
%    \end{macrocode}
% \end{macro}
%
% \begin{macro}{\th@plain}
% \begin{macro}{\th@break}
% \begin{macro}{\th@marginbreak}
% \begin{macro}{\th@changebreak}
% \begin{macro}{\th@change}
% \begin{macro}{\th@margin}
%    The different styles are defined in macros such as
%    |\th@plain|.  Since memory space is precious in
%    ``non-Big-versions'', we have to avoid offering too many unused
%    definitions.  Therefore we define these styles in separate files
%    that can be loaded on demand.  Thus the commands themselves only
%    load these files. We use |\@input@|, a \LaTeXe{} internal command
%    that ensures that the file will be listed with |\listfiles|.
% \changes{v2.2a}{94/02/02}{Use `@input@ to load theorem layout files}
%    \begin{macrocode}
\gdef\th@plain{\@input@{thp.sty}}
\gdef\th@break{\@input@{thb.sty}}
\gdef\th@marginbreak{\@input@{thmb.sty}}
\gdef\th@changebreak{\@input@{thcb.sty}}
\gdef\th@change{\@input@{thc.sty}}
\gdef\th@margin{\@input@{thm.sty}}
%    \end{macrocode}
%    This list will be expanded when new styles become
%    available.  For testing,  just append new theorem substyles
%    as document options.
% \end{macro}
% \end{macro}
% \end{macro}
% \end{macro}
% \end{macro}
% \end{macro}
%
%
%
% \subsection{Definition of a new theorem set}
%
% As already pointed out, a new theorem environment can be defined in
% three different ways:
% \begin{verbatim}
%    \newtheorem{Lem}{Lemma}
%    \newtheorem{Lem}{Lemma}[section]
%    \newtheorem{Lem}[Theorem]{Lemma}
%\end{verbatim}
% The function of the macro |\newtheorem| is to recognize these cases
% and then to branch into one of the three macros |\@ynthm|, |\@xnthm|
% or |\@othm|.  This mechanism is adopted unchanged from
% \cite{latex:source}; the essential point here is that, for example,
% in the second case, the arguments \texttt{Lem}, \texttt{Lemma} and
% \texttt{section} are passed over to the macro |\@xnthm|.
%
% We inspect this case first because the others present fewer problems,
% and thus are easily derived from this one.
%
% \begin{macro}{\@xnthm}
%    For our example arguments, the macro |\@xnthm| must fulfill
%    the following:
%    \begin{itemize}
%       \item
%          Define a new \LaTeX-counter `Lem'
%       \item
%          reset this counter within a |\section|
%       \item
%          define the macro |\theLem|
%       \item
%          define the environment macros |\Lem| and
%          |\endLem| using the current
%          |\theoremstyle| and |\theorem@bodyfont|.
%    \end{itemize}
%    Obviously, all this should happen only if the first argument
%    of |\@xnthm| (i.e.\ \texttt{Lem} in our example) is chosen so
%    as not to conflict with any previously defined commands or
%    environments.  This test is performed by the \LaTeX{} macro
%    |\@ifdefinable|.
%    \begin{macrocode}
\gdef\@xnthm#1#2[#3]{\expandafter\@ifdefinable\csname #1\endcsname
%    \end{macrocode}
%    Therefore, the first argument of |\@ifdefinable| is the
%    expansion (in the example, |\Lem|) of
%    |\csname||#1\endcsname|.  The second argument is
%    executed only if the test has been completed successfully.
%    \begin{macrocode}
   {%
%    \end{macrocode}
%    Now we define the new counter. The names of the \LaTeX{}
%    macros employed should speak for themselves:
% \changes{v2.2b}{1995/10/07}{Use \cs{@newctr}}
% \changes{v2.2c}{1995/11/23}{Use \cs{@newctr} properly.}
%    \begin{macrocode}
    \@definecounter{#1}\@newctr{#1}[#3]%
%    \end{macrocode}
%    Using |\@newctr| will give a proper error message if the counter
%    in |#3| is not defined.
%    In defining `|\theLem|' we must generate the desired macro
%    name by use of |\expandafter| and |\csname|.
%    \begin{macrocode}
    \expandafter\xdef\csname the#1\endcsname
%    \end{macrocode}
%    An |\xdef| is used in order to make the definition global,
%    and to ensure that it contains the replacement texts of
%    |\@thmcountersep| and |\@thmcounter|.\footnote{These
%    two macros can be defined by the document style.  Their default
%    values produce a `\texttt{.}' as separation and an arabic
%    representation of the number.} However, not everything should be
%    expanded. For example, it saves space to use |\thesection|
%    instead of its---at times---lengthy expansion.
%    \begin{macrocode}
      {\expandafter \noexpand \csname the#3\endcsname
       \@thmcountersep \@thmcounter{#1}}%
%    \end{macrocode}
%    Thus with the defaults of \LaTeX{}, |\theLem| would be
%    replaced by the command sequence |\thesection.\arabic{Lem}|.
%
%    We will now look at the definition of the macro which is executed
%    at the beginning of the actual environment (in our example this
%    macro is |\Lem|). It should be noted that we use an
%    ``|\expandafter| trick'' to expand only certain parts of the
%    replacement text at the time of the definition.
%    \begin{macrocode}
    \def\@tempa{\global\@namedef{#1}}%
    \expandafter \@tempa \expandafter{%
%    \end{macrocode}
%    First, the macro that contains the current definitions of
%    |\@begintheorem| and |\@opargtheorem| should be called
%    up.  The name of this macro---as is already known---has the form
%    |\th@|\meta{theorem\ style}; therefore, it must be called by
%    \begin{macrocode}
      \csname th@\the \theorem@style
%    \end{macrocode}
%    In addition the default theorem font should be changeable, i.e.\
%    we have to insert the contents of |\theorem@bodyfont|. For
%    that reason, we expand even further, beyond |\endcsname|,
%    and thus insert the contents of the \textsf{token} register
%    |\theorem@bodyfont| in the replacement text.
%    \begin{macrocode}
            \expandafter \endcsname \the \theorem@bodyfont
%    \end{macrocode}
%    Now it is time to call the macro |\@thm| which takes over
%    the further processing.  It has two arguments: the current
%    counter name (in our example, \texttt{Lem}), and the text of the
%    label (in our example, \texttt{Lemma}).
%    \begin{macrocode}
     \@thm{#1}{#2}}%
%    \end{macrocode}
%    With this, the `sub-definition' is complete.  The macro
%    |\@endtheorem| ends a theorem environment and is, so far, nothing
%    but an |\endtrivlist|.  (Hence it is defined globally, and not
%    within the theorem styles.\footnote{This has to be changed as
%    soon as theorem styles that change \texttt{\bslash @endtheorem}
%    exist.  In such a case, all existing styles must be changed as
%    well since they will have to reset the macro.}) Therefore, we can
%    set it equivalent to the macro that ends the theorem set (in our
%    example, |\endLem|).  However, if some day theorem styles exist
%    that do change |\@endtheorem|, we would have to use the
%    commented-out line instead.
%    \begin{macrocode}
    \global \expandafter \let \csname end#1\endcsname \@endtheorem
%   \global\@namedef{end#1}{\@endtheorem}%
%    \end{macrocode}
%    With these commands all the required definitions are employed,
%    unless the test |\@ifdefinable| has failed.  Therefore, we
%    end the second argument of this macro and with it the definition
%    of |\@xnthm|.
%    \begin{macrocode}
   }}
%    \end{macrocode}
% \end{macro}
%
% \begin{macro}{\@ynthm}
%    The definition of |\@ynthm| is completely analogous. In
%    this case the new counter that is defined is not reset within
%    another counter; thus the definition of |\the...|
%    is simplified:
%    \begin{macrocode}
\gdef\@ynthm#1#2{\expandafter\@ifdefinable\csname #1\endcsname
   {\@definecounter{#1}%
    \expandafter\xdef\csname the#1\endcsname{\@thmcounter{#1}}%
%    \end{macrocode}
%    The rest of the definition corresponds literally to that of
%    |\@xnthm|:
%    \begin{macrocode}
    \def\@tempa{\global\@namedef{#1}}\expandafter \@tempa
     \expandafter{\csname th@\the \theorem@style \expandafter
     \endcsname \the\theorem@bodyfont \@thm{#1}{#2}}%
    \global \expandafter \let \csname end#1\endcsname \@endtheorem}}
%    \end{macrocode}
% \end{macro}
%
% \begin{macro}{\@othm}
%    The definition of |\@othm| does not contain anything new.
%    \begin{macrocode}
\gdef\@othm#1[#2]#3{%
%    \end{macrocode}
% \changes{v2.2b}{95/10/07}{Build in check for counter defined}
%    We do not define a new counter but instead use one that has
%    already been defined.  Thus the only definition we need is that
%    of this pseudo-counter (i.e.\ |\the|\meta{env.\ name}).
%    First we check if |#2| corresponds to a known counter name.
%    \begin{macrocode}
  \expandafter\ifx\csname c@#2\endcsname\relax
   \@nocounterr{#2}%
  \else
   \expandafter\@ifdefinable\csname #1\endcsname
   {\expandafter \xdef \csname the#1\endcsname
     {\expandafter \noexpand \csname the#2\endcsname}%
%    \end{macrocode}
%    All other parts of the definition can be adopted from
%    |\@xnthm|. We have to remember, though, that in this case
%    the name of the current counter and the theorem label have moved
%    to the second and third arguments.
%    \begin{macrocode}
    \def\@tempa{\global\@namedef{#1}}\expandafter \@tempa
     \expandafter{\csname th@\the \theorem@style \expandafter
     \endcsname \the\theorem@bodyfont \@thm{#2}{#3}}%
    \global \expandafter \let \csname end#1\endcsname \@endtheorem}%
  \fi}
%    \end{macrocode}
% \end{macro}
%
% \subsection{Macros that are employed in a {\envfont theorem}
%    environment}
%
% \begin{macro}{\@thm}
%    The macro |\@thm| has to increase the current counter.  Then,
%    depending on whether the environment has (or does not have) an
%    optional argument, it has to branch into either |\@begintheorem|
%    or |\@opargtheorem|.
%    \begin{macrocode}
\gdef\@thm#1#2{\refstepcounter{#1}%
%    \end{macrocode}
%    Now we start a {\envfont trivlist} environment, and give
%    |\@topsep| and |\@topsepadd| the values of the \textsf{skip}
%    registers |\theorempreskipamount| and |\theorempostskipamount|.
%    The value in |\@topsep| is the vertical space that is inserted by
%    the first (and only) |\item| in our |\trivlist| whilst
%    |\@topsepadd| is inserted by |\@endparenv| at the end of that
%    {\envfont trivlist} environment.  By using these registers, we
%    obtain the desired space around a {\envfont theorem} environment.
%    \begin{macrocode}
   \trivlist
   \@topsep \theorempreskipamount               % used by first \item
   \@topsepadd \theorempostskipamount           % used by \@endparenv
%    \end{macrocode}
%    Now we have to test whether an optional argument has been given.
% \changes{v2.0d}{89/07/18}{Forgotten percent added.}
%    \begin{macrocode}
   \@ifnextchar [%
%    \end{macrocode}
%    If there is an optional argument, we will call |\@ythm|, and
%    move the arguments read back into the input stream.
%    \begin{macrocode}
   {\@ythm{#1}{#2}}%
%    \end{macrocode}
%    If not, we call |\@begintheorem|. Its first argument is the
%    name of the theorem set (hence the second argument of
%    |\@thm|). Its second argument is the macro that produces the
%    current number.
%    \begin{macrocode}
   {\@begintheorem{#2}{\csname the#1\endcsname}\ignorespaces}}
%    \end{macrocode}
% \end{macro}
%
% \begin{macro}{\@xthm}
% \begin{macro}{\@ythm}
%    Both these macros were originally called by |\@thm|.  We do
%    not need |\@xthm| anymore, hence we reset it to
%    |\relax|. The definition of |\@ythm| has not changed at
%    all from its definition in \LaTeX. In order to make the macros
%    easier to understand, we will
%    nevertheless present it (commented out).
%    \begin{macrocode}
\global\let\@xthm\relax
% \def\@ythm#1#2[#3]{\@opargbegintheorem{#2}{\csname
%           the#1\endcsname}{#3}\ignorespaces}
%</package>
%    \end{macrocode}
%    The primitive |\ignorespaces| in |\@ythm| and
%    |\@thm| is needed to remove the spaces between the
%    |\begin{...}| and the actual text.
% \end{macro}
% \end{macro}
%
%
% \subsection{Definition of the theorem substyles}
%
% As already pointed out, the theorem substyles, defined below, are
% only loaded when necessary.  Note that all these substyles, except
% \texttt{plain}, have |\slshape| as the default body font.
%
% \subsubsection{The \texttt{plain} style}
%
% \label{dok:th@plain} As the following macros use |@|, we have
% to locally set the |\catcode| of this symbol to ``letter''.
% This happens within a group, so that we do not have to worry about
% which |\catcode| that symbol had before.
%    \begin{macrocode}
%<*thp>
\begingroup \makeatletter
%    \end{macrocode}
% Since we are now within a group, we must make all definitions
% globally.  First we make sure that \texttt{theorem.sty} is loaded.
% This will allow us to use this file as a document style option
% without having to call \texttt{theorem} itself as an option.  At the
% same time, we assure that at least version 2 is loaded, since
% |\theorem@style| was not defined in earlier versions.
%    \begin{macrocode}
\@ifundefined{theorem@style}{\input{theorem.sty}}{}
\ProvidesFile{thp.sty}
             [\FMithmInfo]
%    \end{macrocode}
%
%
% \begin{macro}{\th@plain}
% \begin{macro}{\@begintheorem}
% \begin{macro}{\@opargbegintheorem}
%    |\theoremstyle{plain}| corresponds to the original
%    definition, except that the distances to the surrounding text are
%    determined by the parameters |\theorempreskipamount| and
%    |\theorempostskipamount|.
%    First we set the default body font.
%    \begin{macrocode}
\gdef\th@plain{\normalfont\itshape
%    \end{macrocode}
%    Then we define |\@begintheorem| and
%    |\@opargbegintheorem|. These two macros define how the
%    header of a theorem is typeset.  |\@opargbegintheorem| will
%    be called if a \textsf{theorem} environment with an optional
%    argument is encountered; otherwise, the header is constructed by
%    calling |\@begintheorem|.  If one of these macros is
%    executed, we are within a \textsf{trivlist} environment started by
%    |\@thm|.  So the theorem header is produced with an
%    |\item| command.
%
%    Instead of specifying the header font directly, all standard
%    theorem styles use the |\theorem@headerfont| macro to allow
%    customization.  The extra space (|\labelsep|) is necessary
%    because of problems in the \textsf{trivlist} environment.
%    \begin{macrocode}
  \def\@begintheorem##1##2{%
        \item[\hskip\labelsep \theorem@headerfont ##1\ ##2]}%
%    \end{macrocode}
%    The definition of |\@opargbegintheorem| is completely
%    analogous. The only difference is the fact that there exists a
%    third argument (which is the optional parameter of the
%    environment and contains additional information about the
%    theorem).  Customarily we enclose it in parentheses.
%    \begin{macrocode}
\def\@opargbegintheorem##1##2##3{%
   \item[\hskip\labelsep \theorem@headerfont ##1\ ##2\ (##3)]}}
%    \end{macrocode}
% \end{macro}
% \end{macro}
% \end{macro}
%
% \noindent
% We conclude with an |\endgroup| to restore the
% |\catcode| of |@|.
%    \begin{macrocode}
\endgroup
%</thp>
%    \end{macrocode}
%
%
% \subsubsection{The \texttt{break} style}
%
% This style option is stored in the file |thb.sty|.
% For the next two lines see the documentation for |\th@plain| on
% page~\pageref{dok:th@plain}.
%    \begin{macrocode}
%<*thb>
\begingroup \makeatletter
\@ifundefined{theorem@style}{\input{theorem.sty}}{}
\ProvidesFile{thb.sty}
             [\FMithmInfo]
%    \end{macrocode}
%
% \begin{macro}{\th@break}
%    |\theoremstyle{break}| produces a line break after the name
%    of the theorem.  The font is |\slshape|.  Hence, we define
%    |\th@break| as follows:
%    \begin{macrocode}
\gdef\th@break{\normalfont\slshape
  \def\@begintheorem##1##2{\item[%
%    \end{macrocode}
%    We run into the following problem: it is not possible to create
%    the header with |\item[|\meta{title}|]| and then start
%    a new line by, for example, |\mbox{}\\|. Such a definition
%    will fail whenever a list environment follows immediately.  With
%    the above construction, the |\mbox{}| causes the switch
%    |@inlabel| (cf.\ definition of |\list| and
%    |\trivlist| in \cite{latex:source}) to be set to \texttt{false}
%    and so the following list will insert additional vertical
%    space (|\topskip|). This is quite annoying. Therefore, we
%    create the line break within the |\item|. In order to ensure
%    that the text will begin at the proper position in the following
%    line, we simply pretend that the label does not take any
%    room.\footnote{This will lead to problems whenever very high
%    symbols occurring in the line tower into the
%    heading. So, something else has to be done here sometime.}
%    \begin{macrocode}
       \rlap{\vbox{\hbox{\hskip \labelsep\theorem@headerfont ##1\ ##2}%
                   \hbox{\strut}}}]}%
%    \end{macrocode}
%    Again, the definition of |\@opargbegintheorem| is
%    completely analogous.
%    \begin{macrocode}
\def\@opargbegintheorem##1##2##3{%
  \item[\rlap{\vbox{\hbox{\hskip \labelsep \theorem@headerfont
                     ##1\ ##2\ (##3)}%
                    \hbox{\strut}}}]}}
\endgroup
%</thb>
%    \end{macrocode}
% \end{macro}
%
%
% \subsubsection{The \texttt{changebreak} style}
%
%    \begin{macrocode}
% This style option is stored in the file |thcb.sty|.
%    \begin{macrocode}
%<*thcb>
\begingroup \makeatletter
\@ifundefined{theorem@style}{\input{theorem.sty}}{}
\ProvidesFile{thcb.sty}
             [\FMithmInfo]
%    \end{macrocode}
%
% \begin{macro}{\th@changebreak}
%    The \texttt{change-break} theorem style is like \texttt{break}
%    but with interchange of theorem name and theorem number.  Thus we
%    define |\th@changebreak| as follows:
%    \begin{macrocode}
\gdef\th@changebreak{\normalfont\slshape
  \def\@begintheorem##1##2{\item
       [\rlap{\vbox{\hbox{\hskip\labelsep \theorem@headerfont ##2\ ##1}%
                    \hbox{\strut}}}]}%
\def\@opargbegintheorem##1##2##3{%
  \item[\rlap{\vbox{\hbox{\hskip\labelsep \theorem@headerfont
                     ##2\ ##1\ (##3)}%
                    \hbox{\strut}}}]}}
\endgroup
%</thcb>
%    \end{macrocode}
% \end{macro}
%
%
% \subsubsection{The \texttt{change} style}
%
% This style option is stored in the file |thc.sty|.
%    \begin{macrocode}
%<*thc>
\begingroup \makeatletter
\@ifundefined{theorem@style}{\input{theorem.sty}}{}
\ProvidesFile{thc.sty}
             [\FMithmInfo]
%    \end{macrocode}
%
% \begin{macro}{\th@change}
%    The \texttt{change} theorem style corresponds to the
%    \texttt{change break} style without a linebreak after the header.
%    To say it in another way, it's the same as the \texttt{plain}
%    style but with number and name interchanged and |\slshape| as the
%    default font.
%    \begin{macrocode}
\gdef\th@change{\normalfont\slshape
  \def\@begintheorem##1##2{\item
       [\hskip\labelsep \theorem@headerfont ##2\ ##1]}%
\def\@opargbegintheorem##1##2##3{%
  \item[\hskip\labelsep \theorem@headerfont ##2\ ##1\ (##3)]}}
\endgroup
%</thc>
%    \end{macrocode}
% \end{macro}
%
%
% \subsubsection{The \texttt{marginbreak} style}
%
% This style option is the one used most often at Mainz.  It is saved
% in the file |thmb.sty|.
%    \begin{macrocode}
%<*thmb>
\begingroup \makeatletter
\@ifundefined{theorem@style}{\input{theorem.sty}}{}
\ProvidesFile{thmb.sty}
             [\FMithmInfo]
%    \end{macrocode}
%
% \begin{macro}{\th@marginbreak}
%    The \texttt{margin break} style is nearly the same as the
%    \texttt{change break} style. The only difference is the placement
%    of the theorem number.  We use |\llap| to place it in the left
%    margin.
%
%    In this style |\labelsep| denotes the separation between the
%    number and the text.
%    \begin{macrocode}
\gdef\th@marginbreak{\normalfont\slshape
  \def\@begintheorem##1##2{\item
       [\rlap{\vbox{\theorem@headerfont
                    \hbox{\llap{##2}\hskip\labelsep ##1}%
                    \hbox{\strut}}}]}%
\def\@opargbegintheorem##1##2##3{%
  \item[\rlap{\vbox{\theorem@headerfont
                    \hbox{\llap{##2}\hskip\labelsep ##1\ (##3)}%
                    \hbox{\strut}}}]}}
\endgroup
%</thmb>
%    \end{macrocode}
% \changes{v2.1c}{90/02/17}{`endgroup outside macrocode env.}
% \end{macro}
%
%
%
% \subsubsection{The \texttt{margin} style}
%
% This style option is stored in the file |thm.sty|.
%    \begin{macrocode}
%<*thm>
\begingroup \makeatletter
\@ifundefined{theorem@style}{\input{theorem.sty}}{}
\ProvidesFile{thm.sty}
             [\FMithmInfo]
%    \end{macrocode}
%
% \begin{macro}{\th@margin}
%    Again this is only a variant of the theorem styles described
%    above without any new ideas.
%    \begin{macrocode}
\gdef\th@margin{\normalfont\slshape
  \def\@begintheorem##1##2{\item
       [\theorem@headerfont \llap{##2}\hskip\labelsep ##1]}%
\def\@opargbegintheorem##1##2##3{%
  \item[\theorem@headerfont \llap{##2}\hskip\labelsep ##1\ (##3)]}}
\endgroup
%</thm>
%    \end{macrocode}
% \end{macro}
%
%
%
%
% \subsection{Final Definitions}
%
% \begin{macro}{\theorempreskipamount}
% \begin{macro}{\theorempostskipamount}
%    The \textsf{skip} parameters that regulate the vertical empty space
%    before and after the {\envfont theorem} environment have
%    to be allocated as well.
% \changes{v2.0f}{89/07/20}{Unnecessary `global' removed.}
%    \begin{macrocode}
%<*package>
\newskip\theorempreskipamount
\newskip\theorempostskipamount
%    \end{macrocode}
%    Since we have used the same values for all theorem sets, we now can
%    assign them.
%    \begin{macrocode}
\global\setlength\theorempreskipamount{12pt plus 5pt minus 3pt}
\global\setlength\theorempostskipamount{8pt plus 3pt minus 1.5pt}
%    \end{macrocode}
% \end{macro}
% \end{macro}
%
%
% \begin{macro}{\@endtheorem}
%    The same holds for the macro |\@endtheorem|, which ends a
%    {\envfont theorem} environment. Since it is the same for all
%    theorem sets, it is removed from the macros |\th@|\meta{style}.
%    It simply ends the {\envfont trivlist} environment, which was
%    begun in |\@thm|.
%    \begin{macrocode}
\global\let\@endtheorem=\endtrivlist
%    \end{macrocode}
% \end{macro}
%
%
%
% \begin{macro}{\@preamblecmds}
%    All macros defined above are to be used only in the preamble.
%    Therefore, we insert them in |\@preamblecmds| which will disable
%    them at begin document. This is done by the internal \LaTeXe{}
%    command |\@onlypreamble|.
% \changes{v2.0g}{89/09/19}{`theoremheaderfont and `theorembodyfont
%    added.}
% \changes{v2.2a}{94/02/03}{Use \LaTeXe{} interface}
%    \begin{macrocode}
\@onlypreamble\@xnthm
\@onlypreamble\@ynthm
\@onlypreamble\@othm
\@onlypreamble\newtheorem
\@onlypreamble\theoremstyle
\@onlypreamble\theorembodyfont
\@onlypreamble\theoremheaderfont
%    \end{macrocode}
% \end{macro}
%
% Finally we declare the \texttt{plain} theorem style to be the default.
% \changes{v2.1b}{90/06/04}{Fixed missing `item by declaring default
%    style.}
%    \begin{macrocode}
\theoremstyle{plain}
%</package>
%    \end{macrocode}
%
% \Finale

\endinput
}{}
\ProvidesFile{thp.sty}
             [\FMithmInfo]
%    \end{macrocode}
%
%
% \begin{macro}{\th@plain}
% \begin{macro}{\@begintheorem}
% \begin{macro}{\@opargbegintheorem}
%    |\theoremstyle{plain}| corresponds to the original
%    definition, except that the distances to the surrounding text are
%    determined by the parameters |\theorempreskipamount| and
%    |\theorempostskipamount|.
%    First we set the default body font.
%    \begin{macrocode}
\gdef\th@plain{\normalfont\itshape
%    \end{macrocode}
%    Then we define |\@begintheorem| and
%    |\@opargbegintheorem|. These two macros define how the
%    header of a theorem is typeset.  |\@opargbegintheorem| will
%    be called if a \textsf{theorem} environment with an optional
%    argument is encountered; otherwise, the header is constructed by
%    calling |\@begintheorem|.  If one of these macros is
%    executed, we are within a \textsf{trivlist} environment started by
%    |\@thm|.  So the theorem header is produced with an
%    |\item| command.
%
%    Instead of specifying the header font directly, all standard
%    theorem styles use the |\theorem@headerfont| macro to allow
%    customization.  The extra space (|\labelsep|) is necessary
%    because of problems in the \textsf{trivlist} environment.
%    \begin{macrocode}
  \def\@begintheorem##1##2{%
        \item[\hskip\labelsep \theorem@headerfont ##1\ ##2]}%
%    \end{macrocode}
%    The definition of |\@opargbegintheorem| is completely
%    analogous. The only difference is the fact that there exists a
%    third argument (which is the optional parameter of the
%    environment and contains additional information about the
%    theorem).  Customarily we enclose it in parentheses.
%    \begin{macrocode}
\def\@opargbegintheorem##1##2##3{%
   \item[\hskip\labelsep \theorem@headerfont ##1\ ##2\ (##3)]}}
%    \end{macrocode}
% \end{macro}
% \end{macro}
% \end{macro}
%
% \noindent
% We conclude with an |\endgroup| to restore the
% |\catcode| of |@|.
%    \begin{macrocode}
\endgroup
%</thp>
%    \end{macrocode}
%
%
% \subsubsection{The \texttt{break} style}
%
% This style option is stored in the file |thb.sty|.
% For the next two lines see the documentation for |\th@plain| on
% page~\pageref{dok:th@plain}.
%    \begin{macrocode}
%<*thb>
\begingroup \makeatletter
\@ifundefined{theorem@style}{% \iffalse meta-comment
%
% Copyright (C) 1993-2024
%
% The LaTeX Project and any individual authors listed elsewhere
% in this file.
%
% This file is part of the Standard LaTeX `Tools Bundle'.
% -------------------------------------------------------
%
% It may be distributed and/or modified under the
% conditions of the LaTeX Project Public License, either version 1.3c
% of this license or (at your option) any later version.
% The latest version of this license is in
%    https://www.latex-project.org/lppl.txt
% and version 1.3c or later is part of all distributions of LaTeX
% version 2005/12/01 or later.
%
% The list of all files belonging to the LaTeX `Tools Bundle' is
% given in the file `manifest.txt'.
%
% \fi
%
%%
%
%
\def\FMithmInfo{2023/07/05 v2.2c Theorem extension package (FMi)}
%
% \ProvidesFile{theorem.dtx}[\FMithmInfo]
%\iffalse   % this is a METACOMMENT !
%%
%% Package `theorem' to use with LaTeX2e
%% Copyright (C) 1989-1995 Frank Mittelbach, all rights reserved.
%<+package>\@ifundefined{theorem@style}{}{\endinput}
%<+package>\NeedsTeXFormat{LaTeX2e}[1995/06/01]
%<+package>\ProvidesPackage{theorem}
%<+package>         [\FMithmInfo]
%
% \fi
%
% \changes{v1.0}{87/05/07}{First draft.}
% \changes{v1.0a}{87/05/07}{`thistheoremstyle' added.}
% \changes{v1.1a}{87/07/29}{New concept: redefine the internal
%                          \protect\LaTeX{} commands `@xnthm' etc.}
% \changes{v1.1b}{87/07/31}{`theoremfont' used to specify font.}
% \changes{v1.2a}{88/01/08}{`@endtheorem' now globally defined.}
% \changes{v1.2b}{88/01/11}{`newtheorem' only allowed in the preamble.}
% \changes{v1.3a}{88/02/14}{`theoremfont' now a toks register.}
%
% \changes{v2.0a}{88/11/16}{New implementation and new documentation.}
% \changes{v2.0a}{88/11/16}{No layout changes but internal modifications
%                           old substyles won't work.}
% \changes{v2.0b}{89/01/30}{`global' used for all definitions.}
% \changes{v2.0c}{89/06/15}{Documentation translated to English.
%                            Thanks to LLL and Nina}
% \changes{v2.1a}{90/02/15}{Support for new font selection.}
% \changes{v2.2a}{94/02/03}{Cleanup for LaTeX2e}
%
% \DoNotIndex{\@endparenv,\@ifdefinable,\@ifnextchar,\@ifundefined}
% \DoNotIndex{\@namedef,\@nameuse,\@spaces,\@tempa,\@warning}
% \DoNotIndex{\ }
% \DoNotIndex{\begingroup,\bfseries}
% \DoNotIndex{\csname}
% \DoNotIndex{\def,\do,\docdate}
% \DoNotIndex{\endcsname,\endgroup,\endinput,\endtrivlist,\expandafter}
% \DoNotIndex{\filedate,\fileversion}
% \DoNotIndex{\gdef,\global}
% \DoNotIndex{\hbox,\hskip}
% \DoNotIndex{\ignorespaces,\input,\itshape,\item}
% \DoNotIndex{\labelsep,\let,\llap}
% \DoNotIndex{\makeatletter}
% \DoNotIndex{\noexpand}
% \DoNotIndex{\refstepcounter,\relax,\rlap}
% \DoNotIndex{\setlength,\slshape,\space,\string,\strut}
% \DoNotIndex{\the,\trivlist,\typeout}
% \DoNotIndex{\vbox}
% \DoNotIndex{\xdef}
%
% \def\deflabel#1{\ttfamily #1\hfill}
% \def\deflist#1{\begin{list}{}{\settowidth\labelwidth{\ttfamily #1}%
%                               \setlength\leftmargin\labelwidth
%                               \addtolength\leftmargin\labelsep
%                               \let\makelabel\deflabel}}
% \def\enddeflist{\end{list}}
% \def\envfont{\normalfont\ttfamily}
%
% \MakeShortVerb{\|}
%
% \newcommand{\TUB}{{\sl TUGboat\/}}
%
% \GetFileInfo{theorem.dtx}
% \title{An Extension of the
%        \LaTeX{} theorem environment\thanks{This file
%        has version number \fileversion, last
%        revised \filedate.}}
% \author{Frank Mittelbach}
%
% \MaintainedByLaTeXTeam{tools}
% \maketitle
%
% \begin{abstract}
%    The macros described in this paper yield an extension of the
%    \LaTeX{} theorem mechanism. It is designed to satisfy the
%    different requirements of various journals. Thus, the
%    layout of the ``theorems'' can be manipulated by determining a
%    ``style''. This article describes not only the use, but
%    also the definition, of the necessary macros.
% \end{abstract}
%
% \section*{Preface to version 2.2}
%
% For \LaTeXe{} this package did not need any fundamental changes. I
% only modified the messages generated so that theorem layout styles
% will show up with the |\listfiles| command and cleaned the section
% on the New Font Selection Scheme since this is now included in
% \LaTeX.
%
% \section*{Preface to version 2.1}
%
% This version is identical to 2.0g described in \TUB\ 10\#3 except
% for some internal defaults which are now set depending on the used
% font selection scheme.
%
% This was done to avoid unpleasant surprises if the new font selection
% scheme is in force. For further details see
% section~\ref{sec:newfonts} and \cite{bk:GMS94}.
%
% \section{Introduction}
%
% For our purposes here, ``theorems'' are labelled enunciations,
% often set off from the main text by extra space and a font change.
% Theorems, corollaries, conjectures, definitions, and remarks
% are all instances of ``theorems''.  The ``header'' of these structures
% is composed of a label (such as \textsc{Theorem} or \textsc{Remark})
% and a number which serializes an item in the sequence of items
% with the same label.
%
% Shortly after the introduction of \LaTeX{} at the Fachbereich
% Mathematik in Mainz, the desire to manipulate the layout of
% ``theorems'' arose.  In Mainz, the following two conventions came
% into general use:
% \begin{enumerate}
%   \item The number of the theorem is shown in the margin.
%   \item There is a line break at
%         the end of the theorem header.
% \end{enumerate}
%
% Additionally, some journals require different formats which depend on
% the ``sort of theorem'': e.g.\ often remarks and definitions are set
% in |\upshape|, while |\itshape| is employed for main theorems.
%
% Confronted with these requirements, a theorem environment was
% developed in Mainz which allows separate determination of the
% layout of the ``theorems sets'', comparable to |\pagestyle|.
%
%
% \section{The user interface}
% \subsection{Defining new theorem sets}
%
% \DescribeMacro\newtheorem
% As in the original \LaTeX{} version, the command |\newtheorem|
% defines a new ``theorem set'' or ``theorem-like structure''.
% Two required arguments name the new environment and give the text
% to be typeset with each instance of the new ``set'', while
% an optional argument determines how
% the ``set'' is enumerated:
% \begin{description}
%    \item[\ttfamily \bslash newtheorem\{foo\}\{bar\}]
%       The theorem set {\envfont foo} (whose name is \texttt{bar})
%       uses its own counter.
%    \item[\ttfamily \bslash newtheorem\{foo2\}{[foo]}\{bar2\}]
%       The theorem set {\envfont foo2} (printed name \texttt{bar2})
%       uses the same counter as the theorem set \texttt{foo}.
%    \item[\ttfamily \bslash newtheorem\{foo3\}\{bar3\}{[section]}]
%       The theorem set {\envfont foo3} (printed name \texttt{bar3}) is
%       enumerated within the counter \texttt{section}, i.e.\ with every
%       new |\section| the enumeration begins again with 1, and
%       the enumeration is composed from the section-number and the
%       theorem counter itself.
% \end{description}
%
% \DescribeMacro\theoremstyle
% Additionally, the command |\theoremstyle| can define the layout
% of various, or all, theorem sets.  It should be noted that any
% theorem set defined by |\newtheorem| is typeset in the
% |\theoremstyle| that is current at the time of the definition.
% Thus, the following
% \begin{verbatim}
%    \theoremstyle{break}        \newtheorem{Cor}{Corollary}
%    \theoremstyle{plain}        \newtheorem{Exa}{Example}[section]
%\end{verbatim}                             ^^A No blank at line start
% leads to the result that the set {\envfont Cor} is formatted in the
% style \texttt{break}, while the set {\envfont Exa} and all the
% following ones are formatted in the style \texttt{plain}, unless
% another |\theoremstyle| follows.  Since the definitions installed by
% |\newtheorem| are global, one also can limit |\theoremstyle| locally
% by grouping braces.
%
% \DescribeMacro\theorembodyfont
% The choice of the font for the theorem body is completely
% independent of the chosen |\theoremstyle|; this has proven to
% be very advantageous.  For example,
% \begin{verbatim}
%   {\theorembodyfont{\upshape}          \newtheorem{Rem}{Remark}}
%\end{verbatim}
% defines a theorem set {\envfont Rem}, which will be set in
% |\upshape| in the current layout (which in our example is
% \texttt{plain}).  As with |\theoremstyle|, the |\theorembodyfont|
% chosen is that current at the time of |\newtheorem|.  If
% |\theorembodyfont| is not specified or one defines
% |\theorembodyfont{}|, then the font used will be that defined by the
% |\theoremstyle|.
%
% \DescribeMacro\theoremheaderfont It is also possible to customize
% the font used for the theorem headers.  This is, however, a global
% declaration, and therefore there should be at most one
% |\theoremheaderfont| declaration in the preamble.\footnote{If it is
% actually necessary to have different header fonts, one has to define
% new theorem styles (substituting the desired font) or specify the
% information directly in the \texttt{\bslash newtheorem} declaration
% (the unclean variant).}
%
% \DescribeMacro\theorempreskipamount
% \DescribeMacro\theorempostskipamount Two additional parameters
% affect the vertical space around the theorem environments:
% |\theorempreskipamount| and |\theorempostskipamount| define,
% respectively, the spacing before and after such an environment.
% These parameters apply for all theorem sets and can be manipulated
% with the ordinary length macros.  They are rubber lengths
% (`\textsf{skips}'), and therefore can contain \texttt{plus} and
% \texttt{minus} parts.
%
% Since the definition of theorem sets should---most sensibly---be
% placed in the preamble, we only allow installation there.
% It is therefore possible to release the memory used here after
% |\begin{document}|, in order to make room for other
% applications.
%
%
% \subsection{Existing theorem styles}
%
% The following theorem styles exist to date:
% \begin{deflist}{marginbreak}
%    \item[plain]
%       This theorem style emulates the original \LaTeX{} definition,
%       except that additionally the parameters
%       |\theorem...skipamount| are used.
%    \item[break]
%       In this style, the theorem header is followed by a line
%       break.
%    \item[marginbreak]
%       The theorem number is set in the margin, and there is a line
%       break as in \texttt{break}.
%    \item[changebreak]
%       Like \texttt{break}, but with header number and text
%       interchanged.
%    \item[change]
%       Header number and text are interchanged, without a line break.
%    \item[margin]
%       The number is set in the left margin, without a line break.
% \end{deflist}
% All styles (except \texttt{plain}) select |\slshape| as the default
% |\theorembodyfont|.
%
%
% \subsection{Examples}
%
% Given the above theorem sets {\envfont Cor}, {\envfont Exa}
% and {\envfont Rem},
% suppose that the preamble also contains the declarations:
% \begin{verbatim}
%    \theoremstyle{marginbreak}   \newtheorem{Lem}[Cor]{Lemma}
%    \theoremstyle{change}
%    \theorembodyfont{\itshape}        \newtheorem{Def}[Cor]{Definition}
%
%    \theoremheaderfont{\scshape}
%\end{verbatim}
% Then the following are some typical examples of the typeset output
% resulting from their use.
%
% \begin{Cor}
%    This is a sentence typeset in the theorem environment {\envfont
%    Cor}.
% \end{Cor}
%
% \begin{Exa}
%    This is a sentence typeset in the theorem environment {\envfont
%    Exa}.
% \end{Exa}
%
% \begin{Rem}
%    This is a sentence typeset in the theorem environment {\envfont
%    Rem}.
% \end{Rem}
%
% \begin{Lem}[Ben User]
%    This is a sentence typeset in the theorem environment {\envfont
%    Lem}.
% \end{Lem}
%
% \begin{Def}[Very impressive Definition]
%    This is a sentence typeset in the theorem environment {\envfont
%    Def}.
% \end{Def}
%
% The last two examples show the effect of the optional argument to
% a theorem environment (it is the text typeset in parentheses).
%
%
% \section{Special Considerations}
% \label{sec:newfonts}
%
% Theoremheader and body are implemented as a unit. This means that
% the |\theoremheaderfont|  will inherit characteristics of the
% |\theorembodyfont| in \LaTeXe. Thus, if for example
% |\theorembodyfont| is |\itshape| and |\theoremheaderfont|
% is |\bfseries| the font selected for the header will have the
% characteristics `bold extended italic'. If this is not desired one
% should set the |\theoremheaderfont| to something like
% \begin{verbatim}
%      \theoremheaderfont{\normalfont\bfseries}
%\end{verbatim}
% i.e.\ supplying all necessary font information explicitly.
%
%
%
% \section{Acknowledgements}
%
% The publication of this set of macros was only possible with the
% help of Christina Busse (translating the manuscript into English),
% Joachim Pense (playing the r\^ole of typist), Chris Rowley
% (looking everything over) and many others providing useful
% suggestions.
%
% \MaybeStop{%
%    \begin{thebibliography}{9}
%       \bibitem{bk:GMS94} \textsc{M.~Goossens}, \textsc{F.~Mittelbach}
%         and \textsc{A.~Samarin}.
%         \newblock The \LaTeX{} Companion.
%         \newblock
%         Addison-Wesley, Reading, Massachusetts, 1994.
%       \bibitem{latex:source} \textsc{Lamport, Leslie.} \newblock
%                               \texttt{latex.tex}, version 2.09, date
%                               Feb. 1990.
%    \end{thebibliography}
%    \PrintIndex
%    \PrintChanges}
%
%
% \section{The documentation driver file}
%
% The next bit of code contains the documentation driver file for
% \TeX{}, i.e., the file that will produce the documentation you are
% currently reading. It will be extracted from this file by the
% \texttt{docstrip} program. Since it is the first code in the file
% one can alternatively process this file directly with \LaTeXe{} to
% obtain the documentation.
%
%    \begin{macrocode}
%<*driver>
\documentclass{ltxdoc}

\usepackage{theorem}

%<+driver>% The next few lines define theorem sets which are used
%<+driver>% in the example section of the documentation.

\theoremstyle{break}       \newtheorem{Cor}{Corollary}
\theoremstyle{plain}       \newtheorem{Exa}{Example}[section]
{\theorembodyfont{\upshape}\newtheorem{Rem}{Remark}}
\theoremstyle{marginbreak} \newtheorem{Lem}[Cor]{Lemma}
\theoremstyle{change}
\theorembodyfont{\itshape} \newtheorem{Def}[Cor]{Definition}

\theoremheaderfont{\scshape}

\RecordChanges

\begin{document}
   \DocInput{theorem.dtx}
\end{document}
%</driver>
%    \end{macrocode}
%
% \section{Definition of the Macros}
%
% If the file has been loaded before, we abort immediately. If not the
% package announces itself (this is actually done at the very top of
% the file---the way it is done isn't good style so don't copy it).
% \changes{v2.0e}{89/07/19}{Spaces removed from `typeout'.}
% \changes{v2.2b}{95/11/19}{Announce moved to top of file}
%    \begin{macrocode}
%<*package>
%\@ifundefined{theorem@style}{}{\endinput}
%\def\FMithmInfo{1995/11/19 v2.2b Theorem extension package (FMi)}
%\ProvidesPackage{theorem}[\FMithmInfo]
%    \end{macrocode}
%
%
% \subsection{Definition of theorem styles and fonts}
%
% All the definitions in this file are done globally to allow
% inputting this file inside a group.
%
% \begin{macro}{\theoremstyle}
%    Before a theorem style can be installed, the chosen style must be
%    known. For that reason, we must test to see that
%    |\th@|\meta{style} is known or, more precisely, that it is
%    different from |\relax|. If the style is not known then
%    |\th@plain| is used.
%    \begin{macrocode}
\gdef\theoremstyle#1{%
   \@ifundefined{th@#1}{\@warning
          {Unknown theoremstyle `#1'. Using `plain'}%
          \theorem@style{plain}}%
%    \end{macrocode}
%    We save the theorem style to be used in the token register
%    |\theorem@style|.
%    \begin{macrocode}
      {\theorem@style{#1}}%
%    \end{macrocode}
%    Now we ``evaluate'' the theorem style: this means, we call the
%    macro |\th@|\meta{style} which will activate the relevant
%    definitions which are contained in a separate file.  This is done
%    in a group to suppress changes to the current font.  This could
%    otherwise pose problems together with the new font selection
%    scheme\footnote{When I printed the original article using the new
%    font selection scheme I ended with a document with slanted
%    typefaces (text headings and all) simply because one of the
%    theorem styles used \texttt{\bslash sl} at toplevel.} if the
%    |\th@|\meta{style} is evaluated a second time.
% \changes{v2.1a}{90/02/15}{Group added around `th@..}
% \changes{v2.1a}{90/02/15}{Reseting of `..@begintheorem removed.}
%    \begin{macrocode}
      \begingroup
        \csname th@\the\theorem@style \endcsname
      \endgroup}
%    \end{macrocode}
% \end{macro}
%
%
% \begin{macro}{\@begintheorem}
% \changes{v2.1a}{90/02/15}{Relaxed at toplevel.}
% \begin{macro}{\@opargbegintheorem}
% \changes{v2.1a}{90/02/15}{Relaxed at toplevel.}
%    We reset |\@begintheorem| and |\@opargbegintheorem| to
%    |\relax| since these commands are no longer necessary at
%    toplevel. This will save a few tokens.
%    \begin{macrocode}
\global\let\@begintheorem\relax
\global\let\@opargbegintheorem\relax
%    \end{macrocode}
% \end{macro}
% \end{macro}
%
% \begin{macro}{\theorem@style}
%    Obviously the \textsf{token} register used above has to be
%    allocated. To assure the utmost compatibility with the
%    original \LaTeX{} definition, we set the default theorem style to
%    \texttt{plain}, which implements the usual \LaTeX{} convention.
% \changes{v2.0f}{89/07/20}{Unnecessary `global' removed.}
%    \begin{macrocode}
\newtoks\theorem@style
\global\theorem@style{plain}
%    \end{macrocode}
% \end{macro}
%
% \begin{macro}{\theorembodyfont}
% \begin{macro}{\theorem@bodyfont}
%    If the |\theorembodyfont| is set by the user then it should not
%    interact with the default font set in the theorem style. When the
%    new font selection is in force this may happen if, for example,
%    the default is |\itshape| and the new |\theorembodyfont| is
%    |\sffamily|. So we add a |\reset@font| command in front of the
%    user definition.
% \changes{v2.0f}{89/07/20}{Unnecessary `global' removed.}
% \changes{v2.0g}{89/09/19}{`theoremfont to `theorembodyfont.}
% \changes{v2.1c}{92/06/22}{`reset@font added and logic changed.}
% \changes{v2.1d}{92/10/28}{Reset to empty if arg is empty}
%    \begin{macrocode}
\gdef\theorembodyfont#1{%
%    \end{macrocode}
%    We check if the argument supplied is empty and if so put nothing
%    into the |\theorem@bodyfont| token register to allow for
%    |\theorembodyfont{}| as a mean of using the default of the
%    current |\theoremstyle|.
%    \begin{macrocode}
   \def\@tempa{#1}%
   \ifx\@tempa\@empty
    \theorem@bodyfont{}%
   \else
    \theorem@bodyfont{\reset@font#1}%
   \fi
   }
\newtoks\theorem@bodyfont
\global\theorem@bodyfont{}
%    \end{macrocode}
% \end{macro}
% \end{macro}
%
%
% \begin{macro}{\theoremheaderfont}
% \changes{v2.0g}{89/09/19}{Macro added.}
%    The font for the theorem headers is handled differently because
%    this definition applies to all theorem styles.
%    \begin{macrocode}
\gdef\theoremheaderfont#1{\gdef\theorem@headerfont{#1}%
%    \end{macrocode}
%    After using the macro once it is redefined to produce an error
%    message.
%    \begin{macrocode}
       \gdef\theoremheaderfont##1{%
        \typeout{\string\theoremheaderfont\space should be used
                 only once.}}}
%    \end{macrocode}
% \end{macro}
%
% \begin{macro}{\theorem@headerfont}
% \changes{v2.0g}{89/09/19}{Macro added.}
%    To set the |\theorem@headerfont| default we first test if
%    the new fontselection scheme is in force.
% \changes{v2.1a}{90/02/15}{Support for new font selection.}
%    \begin{macrocode}
\ifx\upshape\undefined
%    \end{macrocode}
%    If not we define it to expand into |\bfseries|. We don't use
%    |\let| just in case a following style option redefines this
%    macro.
%    \begin{macrocode}
\gdef\theorem@headerfont{\bfseries}
%    \end{macrocode}
%    Otherwise we reset the current shape before calling |\bfseries|.
%    \begin{macrocode}
\else \gdef\theorem@headerfont{\normalfont\bfseries}\fi
%    \end{macrocode}
% \end{macro}
%
% \begin{macro}{\th@plain}
% \begin{macro}{\th@break}
% \begin{macro}{\th@marginbreak}
% \begin{macro}{\th@changebreak}
% \begin{macro}{\th@change}
% \begin{macro}{\th@margin}
%    The different styles are defined in macros such as
%    |\th@plain|.  Since memory space is precious in
%    ``non-Big-versions'', we have to avoid offering too many unused
%    definitions.  Therefore we define these styles in separate files
%    that can be loaded on demand.  Thus the commands themselves only
%    load these files. We use |\@input@|, a \LaTeXe{} internal command
%    that ensures that the file will be listed with |\listfiles|.
% \changes{v2.2a}{94/02/02}{Use `@input@ to load theorem layout files}
%    \begin{macrocode}
\gdef\th@plain{\@input@{thp.sty}}
\gdef\th@break{\@input@{thb.sty}}
\gdef\th@marginbreak{\@input@{thmb.sty}}
\gdef\th@changebreak{\@input@{thcb.sty}}
\gdef\th@change{\@input@{thc.sty}}
\gdef\th@margin{\@input@{thm.sty}}
%    \end{macrocode}
%    This list will be expanded when new styles become
%    available.  For testing,  just append new theorem substyles
%    as document options.
% \end{macro}
% \end{macro}
% \end{macro}
% \end{macro}
% \end{macro}
% \end{macro}
%
%
%
% \subsection{Definition of a new theorem set}
%
% As already pointed out, a new theorem environment can be defined in
% three different ways:
% \begin{verbatim}
%    \newtheorem{Lem}{Lemma}
%    \newtheorem{Lem}{Lemma}[section]
%    \newtheorem{Lem}[Theorem]{Lemma}
%\end{verbatim}
% The function of the macro |\newtheorem| is to recognize these cases
% and then to branch into one of the three macros |\@ynthm|, |\@xnthm|
% or |\@othm|.  This mechanism is adopted unchanged from
% \cite{latex:source}; the essential point here is that, for example,
% in the second case, the arguments \texttt{Lem}, \texttt{Lemma} and
% \texttt{section} are passed over to the macro |\@xnthm|.
%
% We inspect this case first because the others present fewer problems,
% and thus are easily derived from this one.
%
% \begin{macro}{\@xnthm}
%    For our example arguments, the macro |\@xnthm| must fulfill
%    the following:
%    \begin{itemize}
%       \item
%          Define a new \LaTeX-counter `Lem'
%       \item
%          reset this counter within a |\section|
%       \item
%          define the macro |\theLem|
%       \item
%          define the environment macros |\Lem| and
%          |\endLem| using the current
%          |\theoremstyle| and |\theorem@bodyfont|.
%    \end{itemize}
%    Obviously, all this should happen only if the first argument
%    of |\@xnthm| (i.e.\ \texttt{Lem} in our example) is chosen so
%    as not to conflict with any previously defined commands or
%    environments.  This test is performed by the \LaTeX{} macro
%    |\@ifdefinable|.
%    \begin{macrocode}
\gdef\@xnthm#1#2[#3]{\expandafter\@ifdefinable\csname #1\endcsname
%    \end{macrocode}
%    Therefore, the first argument of |\@ifdefinable| is the
%    expansion (in the example, |\Lem|) of
%    |\csname||#1\endcsname|.  The second argument is
%    executed only if the test has been completed successfully.
%    \begin{macrocode}
   {%
%    \end{macrocode}
%    Now we define the new counter. The names of the \LaTeX{}
%    macros employed should speak for themselves:
% \changes{v2.2b}{1995/10/07}{Use \cs{@newctr}}
% \changes{v2.2c}{1995/11/23}{Use \cs{@newctr} properly.}
%    \begin{macrocode}
    \@definecounter{#1}\@newctr{#1}[#3]%
%    \end{macrocode}
%    Using |\@newctr| will give a proper error message if the counter
%    in |#3| is not defined.
%    In defining `|\theLem|' we must generate the desired macro
%    name by use of |\expandafter| and |\csname|.
%    \begin{macrocode}
    \expandafter\xdef\csname the#1\endcsname
%    \end{macrocode}
%    An |\xdef| is used in order to make the definition global,
%    and to ensure that it contains the replacement texts of
%    |\@thmcountersep| and |\@thmcounter|.\footnote{These
%    two macros can be defined by the document style.  Their default
%    values produce a `\texttt{.}' as separation and an arabic
%    representation of the number.} However, not everything should be
%    expanded. For example, it saves space to use |\thesection|
%    instead of its---at times---lengthy expansion.
%    \begin{macrocode}
      {\expandafter \noexpand \csname the#3\endcsname
       \@thmcountersep \@thmcounter{#1}}%
%    \end{macrocode}
%    Thus with the defaults of \LaTeX{}, |\theLem| would be
%    replaced by the command sequence |\thesection.\arabic{Lem}|.
%
%    We will now look at the definition of the macro which is executed
%    at the beginning of the actual environment (in our example this
%    macro is |\Lem|). It should be noted that we use an
%    ``|\expandafter| trick'' to expand only certain parts of the
%    replacement text at the time of the definition.
%    \begin{macrocode}
    \def\@tempa{\global\@namedef{#1}}%
    \expandafter \@tempa \expandafter{%
%    \end{macrocode}
%    First, the macro that contains the current definitions of
%    |\@begintheorem| and |\@opargtheorem| should be called
%    up.  The name of this macro---as is already known---has the form
%    |\th@|\meta{theorem\ style}; therefore, it must be called by
%    \begin{macrocode}
      \csname th@\the \theorem@style
%    \end{macrocode}
%    In addition the default theorem font should be changeable, i.e.\
%    we have to insert the contents of |\theorem@bodyfont|. For
%    that reason, we expand even further, beyond |\endcsname|,
%    and thus insert the contents of the \textsf{token} register
%    |\theorem@bodyfont| in the replacement text.
%    \begin{macrocode}
            \expandafter \endcsname \the \theorem@bodyfont
%    \end{macrocode}
%    Now it is time to call the macro |\@thm| which takes over
%    the further processing.  It has two arguments: the current
%    counter name (in our example, \texttt{Lem}), and the text of the
%    label (in our example, \texttt{Lemma}).
%    \begin{macrocode}
     \@thm{#1}{#2}}%
%    \end{macrocode}
%    With this, the `sub-definition' is complete.  The macro
%    |\@endtheorem| ends a theorem environment and is, so far, nothing
%    but an |\endtrivlist|.  (Hence it is defined globally, and not
%    within the theorem styles.\footnote{This has to be changed as
%    soon as theorem styles that change \texttt{\bslash @endtheorem}
%    exist.  In such a case, all existing styles must be changed as
%    well since they will have to reset the macro.}) Therefore, we can
%    set it equivalent to the macro that ends the theorem set (in our
%    example, |\endLem|).  However, if some day theorem styles exist
%    that do change |\@endtheorem|, we would have to use the
%    commented-out line instead.
%    \begin{macrocode}
    \global \expandafter \let \csname end#1\endcsname \@endtheorem
%   \global\@namedef{end#1}{\@endtheorem}%
%    \end{macrocode}
%    With these commands all the required definitions are employed,
%    unless the test |\@ifdefinable| has failed.  Therefore, we
%    end the second argument of this macro and with it the definition
%    of |\@xnthm|.
%    \begin{macrocode}
   }}
%    \end{macrocode}
% \end{macro}
%
% \begin{macro}{\@ynthm}
%    The definition of |\@ynthm| is completely analogous. In
%    this case the new counter that is defined is not reset within
%    another counter; thus the definition of |\the...|
%    is simplified:
%    \begin{macrocode}
\gdef\@ynthm#1#2{\expandafter\@ifdefinable\csname #1\endcsname
   {\@definecounter{#1}%
    \expandafter\xdef\csname the#1\endcsname{\@thmcounter{#1}}%
%    \end{macrocode}
%    The rest of the definition corresponds literally to that of
%    |\@xnthm|:
%    \begin{macrocode}
    \def\@tempa{\global\@namedef{#1}}\expandafter \@tempa
     \expandafter{\csname th@\the \theorem@style \expandafter
     \endcsname \the\theorem@bodyfont \@thm{#1}{#2}}%
    \global \expandafter \let \csname end#1\endcsname \@endtheorem}}
%    \end{macrocode}
% \end{macro}
%
% \begin{macro}{\@othm}
%    The definition of |\@othm| does not contain anything new.
%    \begin{macrocode}
\gdef\@othm#1[#2]#3{%
%    \end{macrocode}
% \changes{v2.2b}{95/10/07}{Build in check for counter defined}
%    We do not define a new counter but instead use one that has
%    already been defined.  Thus the only definition we need is that
%    of this pseudo-counter (i.e.\ |\the|\meta{env.\ name}).
%    First we check if |#2| corresponds to a known counter name.
%    \begin{macrocode}
  \expandafter\ifx\csname c@#2\endcsname\relax
   \@nocounterr{#2}%
  \else
   \expandafter\@ifdefinable\csname #1\endcsname
   {\expandafter \xdef \csname the#1\endcsname
     {\expandafter \noexpand \csname the#2\endcsname}%
%    \end{macrocode}
%    All other parts of the definition can be adopted from
%    |\@xnthm|. We have to remember, though, that in this case
%    the name of the current counter and the theorem label have moved
%    to the second and third arguments.
%    \begin{macrocode}
    \def\@tempa{\global\@namedef{#1}}\expandafter \@tempa
     \expandafter{\csname th@\the \theorem@style \expandafter
     \endcsname \the\theorem@bodyfont \@thm{#2}{#3}}%
    \global \expandafter \let \csname end#1\endcsname \@endtheorem}%
  \fi}
%    \end{macrocode}
% \end{macro}
%
% \subsection{Macros that are employed in a {\envfont theorem}
%    environment}
%
% \begin{macro}{\@thm}
%    The macro |\@thm| has to increase the current counter.  Then,
%    depending on whether the environment has (or does not have) an
%    optional argument, it has to branch into either |\@begintheorem|
%    or |\@opargtheorem|.
%    \begin{macrocode}
\gdef\@thm#1#2{\refstepcounter{#1}%
%    \end{macrocode}
%    Now we start a {\envfont trivlist} environment, and give
%    |\@topsep| and |\@topsepadd| the values of the \textsf{skip}
%    registers |\theorempreskipamount| and |\theorempostskipamount|.
%    The value in |\@topsep| is the vertical space that is inserted by
%    the first (and only) |\item| in our |\trivlist| whilst
%    |\@topsepadd| is inserted by |\@endparenv| at the end of that
%    {\envfont trivlist} environment.  By using these registers, we
%    obtain the desired space around a {\envfont theorem} environment.
%    \begin{macrocode}
   \trivlist
   \@topsep \theorempreskipamount               % used by first \item
   \@topsepadd \theorempostskipamount           % used by \@endparenv
%    \end{macrocode}
%    Now we have to test whether an optional argument has been given.
% \changes{v2.0d}{89/07/18}{Forgotten percent added.}
%    \begin{macrocode}
   \@ifnextchar [%
%    \end{macrocode}
%    If there is an optional argument, we will call |\@ythm|, and
%    move the arguments read back into the input stream.
%    \begin{macrocode}
   {\@ythm{#1}{#2}}%
%    \end{macrocode}
%    If not, we call |\@begintheorem|. Its first argument is the
%    name of the theorem set (hence the second argument of
%    |\@thm|). Its second argument is the macro that produces the
%    current number.
%    \begin{macrocode}
   {\@begintheorem{#2}{\csname the#1\endcsname}\ignorespaces}}
%    \end{macrocode}
% \end{macro}
%
% \begin{macro}{\@xthm}
% \begin{macro}{\@ythm}
%    Both these macros were originally called by |\@thm|.  We do
%    not need |\@xthm| anymore, hence we reset it to
%    |\relax|. The definition of |\@ythm| has not changed at
%    all from its definition in \LaTeX. In order to make the macros
%    easier to understand, we will
%    nevertheless present it (commented out).
%    \begin{macrocode}
\global\let\@xthm\relax
% \def\@ythm#1#2[#3]{\@opargbegintheorem{#2}{\csname
%           the#1\endcsname}{#3}\ignorespaces}
%</package>
%    \end{macrocode}
%    The primitive |\ignorespaces| in |\@ythm| and
%    |\@thm| is needed to remove the spaces between the
%    |\begin{...}| and the actual text.
% \end{macro}
% \end{macro}
%
%
% \subsection{Definition of the theorem substyles}
%
% As already pointed out, the theorem substyles, defined below, are
% only loaded when necessary.  Note that all these substyles, except
% \texttt{plain}, have |\slshape| as the default body font.
%
% \subsubsection{The \texttt{plain} style}
%
% \label{dok:th@plain} As the following macros use |@|, we have
% to locally set the |\catcode| of this symbol to ``letter''.
% This happens within a group, so that we do not have to worry about
% which |\catcode| that symbol had before.
%    \begin{macrocode}
%<*thp>
\begingroup \makeatletter
%    \end{macrocode}
% Since we are now within a group, we must make all definitions
% globally.  First we make sure that \texttt{theorem.sty} is loaded.
% This will allow us to use this file as a document style option
% without having to call \texttt{theorem} itself as an option.  At the
% same time, we assure that at least version 2 is loaded, since
% |\theorem@style| was not defined in earlier versions.
%    \begin{macrocode}
\@ifundefined{theorem@style}{\input{theorem.sty}}{}
\ProvidesFile{thp.sty}
             [\FMithmInfo]
%    \end{macrocode}
%
%
% \begin{macro}{\th@plain}
% \begin{macro}{\@begintheorem}
% \begin{macro}{\@opargbegintheorem}
%    |\theoremstyle{plain}| corresponds to the original
%    definition, except that the distances to the surrounding text are
%    determined by the parameters |\theorempreskipamount| and
%    |\theorempostskipamount|.
%    First we set the default body font.
%    \begin{macrocode}
\gdef\th@plain{\normalfont\itshape
%    \end{macrocode}
%    Then we define |\@begintheorem| and
%    |\@opargbegintheorem|. These two macros define how the
%    header of a theorem is typeset.  |\@opargbegintheorem| will
%    be called if a \textsf{theorem} environment with an optional
%    argument is encountered; otherwise, the header is constructed by
%    calling |\@begintheorem|.  If one of these macros is
%    executed, we are within a \textsf{trivlist} environment started by
%    |\@thm|.  So the theorem header is produced with an
%    |\item| command.
%
%    Instead of specifying the header font directly, all standard
%    theorem styles use the |\theorem@headerfont| macro to allow
%    customization.  The extra space (|\labelsep|) is necessary
%    because of problems in the \textsf{trivlist} environment.
%    \begin{macrocode}
  \def\@begintheorem##1##2{%
        \item[\hskip\labelsep \theorem@headerfont ##1\ ##2]}%
%    \end{macrocode}
%    The definition of |\@opargbegintheorem| is completely
%    analogous. The only difference is the fact that there exists a
%    third argument (which is the optional parameter of the
%    environment and contains additional information about the
%    theorem).  Customarily we enclose it in parentheses.
%    \begin{macrocode}
\def\@opargbegintheorem##1##2##3{%
   \item[\hskip\labelsep \theorem@headerfont ##1\ ##2\ (##3)]}}
%    \end{macrocode}
% \end{macro}
% \end{macro}
% \end{macro}
%
% \noindent
% We conclude with an |\endgroup| to restore the
% |\catcode| of |@|.
%    \begin{macrocode}
\endgroup
%</thp>
%    \end{macrocode}
%
%
% \subsubsection{The \texttt{break} style}
%
% This style option is stored in the file |thb.sty|.
% For the next two lines see the documentation for |\th@plain| on
% page~\pageref{dok:th@plain}.
%    \begin{macrocode}
%<*thb>
\begingroup \makeatletter
\@ifundefined{theorem@style}{\input{theorem.sty}}{}
\ProvidesFile{thb.sty}
             [\FMithmInfo]
%    \end{macrocode}
%
% \begin{macro}{\th@break}
%    |\theoremstyle{break}| produces a line break after the name
%    of the theorem.  The font is |\slshape|.  Hence, we define
%    |\th@break| as follows:
%    \begin{macrocode}
\gdef\th@break{\normalfont\slshape
  \def\@begintheorem##1##2{\item[%
%    \end{macrocode}
%    We run into the following problem: it is not possible to create
%    the header with |\item[|\meta{title}|]| and then start
%    a new line by, for example, |\mbox{}\\|. Such a definition
%    will fail whenever a list environment follows immediately.  With
%    the above construction, the |\mbox{}| causes the switch
%    |@inlabel| (cf.\ definition of |\list| and
%    |\trivlist| in \cite{latex:source}) to be set to \texttt{false}
%    and so the following list will insert additional vertical
%    space (|\topskip|). This is quite annoying. Therefore, we
%    create the line break within the |\item|. In order to ensure
%    that the text will begin at the proper position in the following
%    line, we simply pretend that the label does not take any
%    room.\footnote{This will lead to problems whenever very high
%    symbols occurring in the line tower into the
%    heading. So, something else has to be done here sometime.}
%    \begin{macrocode}
       \rlap{\vbox{\hbox{\hskip \labelsep\theorem@headerfont ##1\ ##2}%
                   \hbox{\strut}}}]}%
%    \end{macrocode}
%    Again, the definition of |\@opargbegintheorem| is
%    completely analogous.
%    \begin{macrocode}
\def\@opargbegintheorem##1##2##3{%
  \item[\rlap{\vbox{\hbox{\hskip \labelsep \theorem@headerfont
                     ##1\ ##2\ (##3)}%
                    \hbox{\strut}}}]}}
\endgroup
%</thb>
%    \end{macrocode}
% \end{macro}
%
%
% \subsubsection{The \texttt{changebreak} style}
%
%    \begin{macrocode}
% This style option is stored in the file |thcb.sty|.
%    \begin{macrocode}
%<*thcb>
\begingroup \makeatletter
\@ifundefined{theorem@style}{\input{theorem.sty}}{}
\ProvidesFile{thcb.sty}
             [\FMithmInfo]
%    \end{macrocode}
%
% \begin{macro}{\th@changebreak}
%    The \texttt{change-break} theorem style is like \texttt{break}
%    but with interchange of theorem name and theorem number.  Thus we
%    define |\th@changebreak| as follows:
%    \begin{macrocode}
\gdef\th@changebreak{\normalfont\slshape
  \def\@begintheorem##1##2{\item
       [\rlap{\vbox{\hbox{\hskip\labelsep \theorem@headerfont ##2\ ##1}%
                    \hbox{\strut}}}]}%
\def\@opargbegintheorem##1##2##3{%
  \item[\rlap{\vbox{\hbox{\hskip\labelsep \theorem@headerfont
                     ##2\ ##1\ (##3)}%
                    \hbox{\strut}}}]}}
\endgroup
%</thcb>
%    \end{macrocode}
% \end{macro}
%
%
% \subsubsection{The \texttt{change} style}
%
% This style option is stored in the file |thc.sty|.
%    \begin{macrocode}
%<*thc>
\begingroup \makeatletter
\@ifundefined{theorem@style}{\input{theorem.sty}}{}
\ProvidesFile{thc.sty}
             [\FMithmInfo]
%    \end{macrocode}
%
% \begin{macro}{\th@change}
%    The \texttt{change} theorem style corresponds to the
%    \texttt{change break} style without a linebreak after the header.
%    To say it in another way, it's the same as the \texttt{plain}
%    style but with number and name interchanged and |\slshape| as the
%    default font.
%    \begin{macrocode}
\gdef\th@change{\normalfont\slshape
  \def\@begintheorem##1##2{\item
       [\hskip\labelsep \theorem@headerfont ##2\ ##1]}%
\def\@opargbegintheorem##1##2##3{%
  \item[\hskip\labelsep \theorem@headerfont ##2\ ##1\ (##3)]}}
\endgroup
%</thc>
%    \end{macrocode}
% \end{macro}
%
%
% \subsubsection{The \texttt{marginbreak} style}
%
% This style option is the one used most often at Mainz.  It is saved
% in the file |thmb.sty|.
%    \begin{macrocode}
%<*thmb>
\begingroup \makeatletter
\@ifundefined{theorem@style}{\input{theorem.sty}}{}
\ProvidesFile{thmb.sty}
             [\FMithmInfo]
%    \end{macrocode}
%
% \begin{macro}{\th@marginbreak}
%    The \texttt{margin break} style is nearly the same as the
%    \texttt{change break} style. The only difference is the placement
%    of the theorem number.  We use |\llap| to place it in the left
%    margin.
%
%    In this style |\labelsep| denotes the separation between the
%    number and the text.
%    \begin{macrocode}
\gdef\th@marginbreak{\normalfont\slshape
  \def\@begintheorem##1##2{\item
       [\rlap{\vbox{\theorem@headerfont
                    \hbox{\llap{##2}\hskip\labelsep ##1}%
                    \hbox{\strut}}}]}%
\def\@opargbegintheorem##1##2##3{%
  \item[\rlap{\vbox{\theorem@headerfont
                    \hbox{\llap{##2}\hskip\labelsep ##1\ (##3)}%
                    \hbox{\strut}}}]}}
\endgroup
%</thmb>
%    \end{macrocode}
% \changes{v2.1c}{90/02/17}{`endgroup outside macrocode env.}
% \end{macro}
%
%
%
% \subsubsection{The \texttt{margin} style}
%
% This style option is stored in the file |thm.sty|.
%    \begin{macrocode}
%<*thm>
\begingroup \makeatletter
\@ifundefined{theorem@style}{\input{theorem.sty}}{}
\ProvidesFile{thm.sty}
             [\FMithmInfo]
%    \end{macrocode}
%
% \begin{macro}{\th@margin}
%    Again this is only a variant of the theorem styles described
%    above without any new ideas.
%    \begin{macrocode}
\gdef\th@margin{\normalfont\slshape
  \def\@begintheorem##1##2{\item
       [\theorem@headerfont \llap{##2}\hskip\labelsep ##1]}%
\def\@opargbegintheorem##1##2##3{%
  \item[\theorem@headerfont \llap{##2}\hskip\labelsep ##1\ (##3)]}}
\endgroup
%</thm>
%    \end{macrocode}
% \end{macro}
%
%
%
%
% \subsection{Final Definitions}
%
% \begin{macro}{\theorempreskipamount}
% \begin{macro}{\theorempostskipamount}
%    The \textsf{skip} parameters that regulate the vertical empty space
%    before and after the {\envfont theorem} environment have
%    to be allocated as well.
% \changes{v2.0f}{89/07/20}{Unnecessary `global' removed.}
%    \begin{macrocode}
%<*package>
\newskip\theorempreskipamount
\newskip\theorempostskipamount
%    \end{macrocode}
%    Since we have used the same values for all theorem sets, we now can
%    assign them.
%    \begin{macrocode}
\global\setlength\theorempreskipamount{12pt plus 5pt minus 3pt}
\global\setlength\theorempostskipamount{8pt plus 3pt minus 1.5pt}
%    \end{macrocode}
% \end{macro}
% \end{macro}
%
%
% \begin{macro}{\@endtheorem}
%    The same holds for the macro |\@endtheorem|, which ends a
%    {\envfont theorem} environment. Since it is the same for all
%    theorem sets, it is removed from the macros |\th@|\meta{style}.
%    It simply ends the {\envfont trivlist} environment, which was
%    begun in |\@thm|.
%    \begin{macrocode}
\global\let\@endtheorem=\endtrivlist
%    \end{macrocode}
% \end{macro}
%
%
%
% \begin{macro}{\@preamblecmds}
%    All macros defined above are to be used only in the preamble.
%    Therefore, we insert them in |\@preamblecmds| which will disable
%    them at begin document. This is done by the internal \LaTeXe{}
%    command |\@onlypreamble|.
% \changes{v2.0g}{89/09/19}{`theoremheaderfont and `theorembodyfont
%    added.}
% \changes{v2.2a}{94/02/03}{Use \LaTeXe{} interface}
%    \begin{macrocode}
\@onlypreamble\@xnthm
\@onlypreamble\@ynthm
\@onlypreamble\@othm
\@onlypreamble\newtheorem
\@onlypreamble\theoremstyle
\@onlypreamble\theorembodyfont
\@onlypreamble\theoremheaderfont
%    \end{macrocode}
% \end{macro}
%
% Finally we declare the \texttt{plain} theorem style to be the default.
% \changes{v2.1b}{90/06/04}{Fixed missing `item by declaring default
%    style.}
%    \begin{macrocode}
\theoremstyle{plain}
%</package>
%    \end{macrocode}
%
% \Finale

\endinput
}{}
\ProvidesFile{thb.sty}
             [\FMithmInfo]
%    \end{macrocode}
%
% \begin{macro}{\th@break}
%    |\theoremstyle{break}| produces a line break after the name
%    of the theorem.  The font is |\slshape|.  Hence, we define
%    |\th@break| as follows:
%    \begin{macrocode}
\gdef\th@break{\normalfont\slshape
  \def\@begintheorem##1##2{\item[%
%    \end{macrocode}
%    We run into the following problem: it is not possible to create
%    the header with |\item[|\meta{title}|]| and then start
%    a new line by, for example, |\mbox{}\\|. Such a definition
%    will fail whenever a list environment follows immediately.  With
%    the above construction, the |\mbox{}| causes the switch
%    |@inlabel| (cf.\ definition of |\list| and
%    |\trivlist| in \cite{latex:source}) to be set to \texttt{false}
%    and so the following list will insert additional vertical
%    space (|\topskip|). This is quite annoying. Therefore, we
%    create the line break within the |\item|. In order to ensure
%    that the text will begin at the proper position in the following
%    line, we simply pretend that the label does not take any
%    room.\footnote{This will lead to problems whenever very high
%    symbols occurring in the line tower into the
%    heading. So, something else has to be done here sometime.}
%    \begin{macrocode}
       \rlap{\vbox{\hbox{\hskip \labelsep\theorem@headerfont ##1\ ##2}%
                   \hbox{\strut}}}]}%
%    \end{macrocode}
%    Again, the definition of |\@opargbegintheorem| is
%    completely analogous.
%    \begin{macrocode}
\def\@opargbegintheorem##1##2##3{%
  \item[\rlap{\vbox{\hbox{\hskip \labelsep \theorem@headerfont
                     ##1\ ##2\ (##3)}%
                    \hbox{\strut}}}]}}
\endgroup
%</thb>
%    \end{macrocode}
% \end{macro}
%
%
% \subsubsection{The \texttt{changebreak} style}
%
%    \begin{macrocode}
% This style option is stored in the file |thcb.sty|.
%    \begin{macrocode}
%<*thcb>
\begingroup \makeatletter
\@ifundefined{theorem@style}{% \iffalse meta-comment
%
% Copyright (C) 1993-2024
%
% The LaTeX Project and any individual authors listed elsewhere
% in this file.
%
% This file is part of the Standard LaTeX `Tools Bundle'.
% -------------------------------------------------------
%
% It may be distributed and/or modified under the
% conditions of the LaTeX Project Public License, either version 1.3c
% of this license or (at your option) any later version.
% The latest version of this license is in
%    https://www.latex-project.org/lppl.txt
% and version 1.3c or later is part of all distributions of LaTeX
% version 2005/12/01 or later.
%
% The list of all files belonging to the LaTeX `Tools Bundle' is
% given in the file `manifest.txt'.
%
% \fi
%
%%
%
%
\def\FMithmInfo{2023/07/05 v2.2c Theorem extension package (FMi)}
%
% \ProvidesFile{theorem.dtx}[\FMithmInfo]
%\iffalse   % this is a METACOMMENT !
%%
%% Package `theorem' to use with LaTeX2e
%% Copyright (C) 1989-1995 Frank Mittelbach, all rights reserved.
%<+package>\@ifundefined{theorem@style}{}{\endinput}
%<+package>\NeedsTeXFormat{LaTeX2e}[1995/06/01]
%<+package>\ProvidesPackage{theorem}
%<+package>         [\FMithmInfo]
%
% \fi
%
% \changes{v1.0}{87/05/07}{First draft.}
% \changes{v1.0a}{87/05/07}{`thistheoremstyle' added.}
% \changes{v1.1a}{87/07/29}{New concept: redefine the internal
%                          \protect\LaTeX{} commands `@xnthm' etc.}
% \changes{v1.1b}{87/07/31}{`theoremfont' used to specify font.}
% \changes{v1.2a}{88/01/08}{`@endtheorem' now globally defined.}
% \changes{v1.2b}{88/01/11}{`newtheorem' only allowed in the preamble.}
% \changes{v1.3a}{88/02/14}{`theoremfont' now a toks register.}
%
% \changes{v2.0a}{88/11/16}{New implementation and new documentation.}
% \changes{v2.0a}{88/11/16}{No layout changes but internal modifications
%                           old substyles won't work.}
% \changes{v2.0b}{89/01/30}{`global' used for all definitions.}
% \changes{v2.0c}{89/06/15}{Documentation translated to English.
%                            Thanks to LLL and Nina}
% \changes{v2.1a}{90/02/15}{Support for new font selection.}
% \changes{v2.2a}{94/02/03}{Cleanup for LaTeX2e}
%
% \DoNotIndex{\@endparenv,\@ifdefinable,\@ifnextchar,\@ifundefined}
% \DoNotIndex{\@namedef,\@nameuse,\@spaces,\@tempa,\@warning}
% \DoNotIndex{\ }
% \DoNotIndex{\begingroup,\bfseries}
% \DoNotIndex{\csname}
% \DoNotIndex{\def,\do,\docdate}
% \DoNotIndex{\endcsname,\endgroup,\endinput,\endtrivlist,\expandafter}
% \DoNotIndex{\filedate,\fileversion}
% \DoNotIndex{\gdef,\global}
% \DoNotIndex{\hbox,\hskip}
% \DoNotIndex{\ignorespaces,\input,\itshape,\item}
% \DoNotIndex{\labelsep,\let,\llap}
% \DoNotIndex{\makeatletter}
% \DoNotIndex{\noexpand}
% \DoNotIndex{\refstepcounter,\relax,\rlap}
% \DoNotIndex{\setlength,\slshape,\space,\string,\strut}
% \DoNotIndex{\the,\trivlist,\typeout}
% \DoNotIndex{\vbox}
% \DoNotIndex{\xdef}
%
% \def\deflabel#1{\ttfamily #1\hfill}
% \def\deflist#1{\begin{list}{}{\settowidth\labelwidth{\ttfamily #1}%
%                               \setlength\leftmargin\labelwidth
%                               \addtolength\leftmargin\labelsep
%                               \let\makelabel\deflabel}}
% \def\enddeflist{\end{list}}
% \def\envfont{\normalfont\ttfamily}
%
% \MakeShortVerb{\|}
%
% \newcommand{\TUB}{{\sl TUGboat\/}}
%
% \GetFileInfo{theorem.dtx}
% \title{An Extension of the
%        \LaTeX{} theorem environment\thanks{This file
%        has version number \fileversion, last
%        revised \filedate.}}
% \author{Frank Mittelbach}
%
% \MaintainedByLaTeXTeam{tools}
% \maketitle
%
% \begin{abstract}
%    The macros described in this paper yield an extension of the
%    \LaTeX{} theorem mechanism. It is designed to satisfy the
%    different requirements of various journals. Thus, the
%    layout of the ``theorems'' can be manipulated by determining a
%    ``style''. This article describes not only the use, but
%    also the definition, of the necessary macros.
% \end{abstract}
%
% \section*{Preface to version 2.2}
%
% For \LaTeXe{} this package did not need any fundamental changes. I
% only modified the messages generated so that theorem layout styles
% will show up with the |\listfiles| command and cleaned the section
% on the New Font Selection Scheme since this is now included in
% \LaTeX.
%
% \section*{Preface to version 2.1}
%
% This version is identical to 2.0g described in \TUB\ 10\#3 except
% for some internal defaults which are now set depending on the used
% font selection scheme.
%
% This was done to avoid unpleasant surprises if the new font selection
% scheme is in force. For further details see
% section~\ref{sec:newfonts} and \cite{bk:GMS94}.
%
% \section{Introduction}
%
% For our purposes here, ``theorems'' are labelled enunciations,
% often set off from the main text by extra space and a font change.
% Theorems, corollaries, conjectures, definitions, and remarks
% are all instances of ``theorems''.  The ``header'' of these structures
% is composed of a label (such as \textsc{Theorem} or \textsc{Remark})
% and a number which serializes an item in the sequence of items
% with the same label.
%
% Shortly after the introduction of \LaTeX{} at the Fachbereich
% Mathematik in Mainz, the desire to manipulate the layout of
% ``theorems'' arose.  In Mainz, the following two conventions came
% into general use:
% \begin{enumerate}
%   \item The number of the theorem is shown in the margin.
%   \item There is a line break at
%         the end of the theorem header.
% \end{enumerate}
%
% Additionally, some journals require different formats which depend on
% the ``sort of theorem'': e.g.\ often remarks and definitions are set
% in |\upshape|, while |\itshape| is employed for main theorems.
%
% Confronted with these requirements, a theorem environment was
% developed in Mainz which allows separate determination of the
% layout of the ``theorems sets'', comparable to |\pagestyle|.
%
%
% \section{The user interface}
% \subsection{Defining new theorem sets}
%
% \DescribeMacro\newtheorem
% As in the original \LaTeX{} version, the command |\newtheorem|
% defines a new ``theorem set'' or ``theorem-like structure''.
% Two required arguments name the new environment and give the text
% to be typeset with each instance of the new ``set'', while
% an optional argument determines how
% the ``set'' is enumerated:
% \begin{description}
%    \item[\ttfamily \bslash newtheorem\{foo\}\{bar\}]
%       The theorem set {\envfont foo} (whose name is \texttt{bar})
%       uses its own counter.
%    \item[\ttfamily \bslash newtheorem\{foo2\}{[foo]}\{bar2\}]
%       The theorem set {\envfont foo2} (printed name \texttt{bar2})
%       uses the same counter as the theorem set \texttt{foo}.
%    \item[\ttfamily \bslash newtheorem\{foo3\}\{bar3\}{[section]}]
%       The theorem set {\envfont foo3} (printed name \texttt{bar3}) is
%       enumerated within the counter \texttt{section}, i.e.\ with every
%       new |\section| the enumeration begins again with 1, and
%       the enumeration is composed from the section-number and the
%       theorem counter itself.
% \end{description}
%
% \DescribeMacro\theoremstyle
% Additionally, the command |\theoremstyle| can define the layout
% of various, or all, theorem sets.  It should be noted that any
% theorem set defined by |\newtheorem| is typeset in the
% |\theoremstyle| that is current at the time of the definition.
% Thus, the following
% \begin{verbatim}
%    \theoremstyle{break}        \newtheorem{Cor}{Corollary}
%    \theoremstyle{plain}        \newtheorem{Exa}{Example}[section]
%\end{verbatim}                             ^^A No blank at line start
% leads to the result that the set {\envfont Cor} is formatted in the
% style \texttt{break}, while the set {\envfont Exa} and all the
% following ones are formatted in the style \texttt{plain}, unless
% another |\theoremstyle| follows.  Since the definitions installed by
% |\newtheorem| are global, one also can limit |\theoremstyle| locally
% by grouping braces.
%
% \DescribeMacro\theorembodyfont
% The choice of the font for the theorem body is completely
% independent of the chosen |\theoremstyle|; this has proven to
% be very advantageous.  For example,
% \begin{verbatim}
%   {\theorembodyfont{\upshape}          \newtheorem{Rem}{Remark}}
%\end{verbatim}
% defines a theorem set {\envfont Rem}, which will be set in
% |\upshape| in the current layout (which in our example is
% \texttt{plain}).  As with |\theoremstyle|, the |\theorembodyfont|
% chosen is that current at the time of |\newtheorem|.  If
% |\theorembodyfont| is not specified or one defines
% |\theorembodyfont{}|, then the font used will be that defined by the
% |\theoremstyle|.
%
% \DescribeMacro\theoremheaderfont It is also possible to customize
% the font used for the theorem headers.  This is, however, a global
% declaration, and therefore there should be at most one
% |\theoremheaderfont| declaration in the preamble.\footnote{If it is
% actually necessary to have different header fonts, one has to define
% new theorem styles (substituting the desired font) or specify the
% information directly in the \texttt{\bslash newtheorem} declaration
% (the unclean variant).}
%
% \DescribeMacro\theorempreskipamount
% \DescribeMacro\theorempostskipamount Two additional parameters
% affect the vertical space around the theorem environments:
% |\theorempreskipamount| and |\theorempostskipamount| define,
% respectively, the spacing before and after such an environment.
% These parameters apply for all theorem sets and can be manipulated
% with the ordinary length macros.  They are rubber lengths
% (`\textsf{skips}'), and therefore can contain \texttt{plus} and
% \texttt{minus} parts.
%
% Since the definition of theorem sets should---most sensibly---be
% placed in the preamble, we only allow installation there.
% It is therefore possible to release the memory used here after
% |\begin{document}|, in order to make room for other
% applications.
%
%
% \subsection{Existing theorem styles}
%
% The following theorem styles exist to date:
% \begin{deflist}{marginbreak}
%    \item[plain]
%       This theorem style emulates the original \LaTeX{} definition,
%       except that additionally the parameters
%       |\theorem...skipamount| are used.
%    \item[break]
%       In this style, the theorem header is followed by a line
%       break.
%    \item[marginbreak]
%       The theorem number is set in the margin, and there is a line
%       break as in \texttt{break}.
%    \item[changebreak]
%       Like \texttt{break}, but with header number and text
%       interchanged.
%    \item[change]
%       Header number and text are interchanged, without a line break.
%    \item[margin]
%       The number is set in the left margin, without a line break.
% \end{deflist}
% All styles (except \texttt{plain}) select |\slshape| as the default
% |\theorembodyfont|.
%
%
% \subsection{Examples}
%
% Given the above theorem sets {\envfont Cor}, {\envfont Exa}
% and {\envfont Rem},
% suppose that the preamble also contains the declarations:
% \begin{verbatim}
%    \theoremstyle{marginbreak}   \newtheorem{Lem}[Cor]{Lemma}
%    \theoremstyle{change}
%    \theorembodyfont{\itshape}        \newtheorem{Def}[Cor]{Definition}
%
%    \theoremheaderfont{\scshape}
%\end{verbatim}
% Then the following are some typical examples of the typeset output
% resulting from their use.
%
% \begin{Cor}
%    This is a sentence typeset in the theorem environment {\envfont
%    Cor}.
% \end{Cor}
%
% \begin{Exa}
%    This is a sentence typeset in the theorem environment {\envfont
%    Exa}.
% \end{Exa}
%
% \begin{Rem}
%    This is a sentence typeset in the theorem environment {\envfont
%    Rem}.
% \end{Rem}
%
% \begin{Lem}[Ben User]
%    This is a sentence typeset in the theorem environment {\envfont
%    Lem}.
% \end{Lem}
%
% \begin{Def}[Very impressive Definition]
%    This is a sentence typeset in the theorem environment {\envfont
%    Def}.
% \end{Def}
%
% The last two examples show the effect of the optional argument to
% a theorem environment (it is the text typeset in parentheses).
%
%
% \section{Special Considerations}
% \label{sec:newfonts}
%
% Theoremheader and body are implemented as a unit. This means that
% the |\theoremheaderfont|  will inherit characteristics of the
% |\theorembodyfont| in \LaTeXe. Thus, if for example
% |\theorembodyfont| is |\itshape| and |\theoremheaderfont|
% is |\bfseries| the font selected for the header will have the
% characteristics `bold extended italic'. If this is not desired one
% should set the |\theoremheaderfont| to something like
% \begin{verbatim}
%      \theoremheaderfont{\normalfont\bfseries}
%\end{verbatim}
% i.e.\ supplying all necessary font information explicitly.
%
%
%
% \section{Acknowledgements}
%
% The publication of this set of macros was only possible with the
% help of Christina Busse (translating the manuscript into English),
% Joachim Pense (playing the r\^ole of typist), Chris Rowley
% (looking everything over) and many others providing useful
% suggestions.
%
% \MaybeStop{%
%    \begin{thebibliography}{9}
%       \bibitem{bk:GMS94} \textsc{M.~Goossens}, \textsc{F.~Mittelbach}
%         and \textsc{A.~Samarin}.
%         \newblock The \LaTeX{} Companion.
%         \newblock
%         Addison-Wesley, Reading, Massachusetts, 1994.
%       \bibitem{latex:source} \textsc{Lamport, Leslie.} \newblock
%                               \texttt{latex.tex}, version 2.09, date
%                               Feb. 1990.
%    \end{thebibliography}
%    \PrintIndex
%    \PrintChanges}
%
%
% \section{The documentation driver file}
%
% The next bit of code contains the documentation driver file for
% \TeX{}, i.e., the file that will produce the documentation you are
% currently reading. It will be extracted from this file by the
% \texttt{docstrip} program. Since it is the first code in the file
% one can alternatively process this file directly with \LaTeXe{} to
% obtain the documentation.
%
%    \begin{macrocode}
%<*driver>
\documentclass{ltxdoc}

\usepackage{theorem}

%<+driver>% The next few lines define theorem sets which are used
%<+driver>% in the example section of the documentation.

\theoremstyle{break}       \newtheorem{Cor}{Corollary}
\theoremstyle{plain}       \newtheorem{Exa}{Example}[section]
{\theorembodyfont{\upshape}\newtheorem{Rem}{Remark}}
\theoremstyle{marginbreak} \newtheorem{Lem}[Cor]{Lemma}
\theoremstyle{change}
\theorembodyfont{\itshape} \newtheorem{Def}[Cor]{Definition}

\theoremheaderfont{\scshape}

\RecordChanges

\begin{document}
   \DocInput{theorem.dtx}
\end{document}
%</driver>
%    \end{macrocode}
%
% \section{Definition of the Macros}
%
% If the file has been loaded before, we abort immediately. If not the
% package announces itself (this is actually done at the very top of
% the file---the way it is done isn't good style so don't copy it).
% \changes{v2.0e}{89/07/19}{Spaces removed from `typeout'.}
% \changes{v2.2b}{95/11/19}{Announce moved to top of file}
%    \begin{macrocode}
%<*package>
%\@ifundefined{theorem@style}{}{\endinput}
%\def\FMithmInfo{1995/11/19 v2.2b Theorem extension package (FMi)}
%\ProvidesPackage{theorem}[\FMithmInfo]
%    \end{macrocode}
%
%
% \subsection{Definition of theorem styles and fonts}
%
% All the definitions in this file are done globally to allow
% inputting this file inside a group.
%
% \begin{macro}{\theoremstyle}
%    Before a theorem style can be installed, the chosen style must be
%    known. For that reason, we must test to see that
%    |\th@|\meta{style} is known or, more precisely, that it is
%    different from |\relax|. If the style is not known then
%    |\th@plain| is used.
%    \begin{macrocode}
\gdef\theoremstyle#1{%
   \@ifundefined{th@#1}{\@warning
          {Unknown theoremstyle `#1'. Using `plain'}%
          \theorem@style{plain}}%
%    \end{macrocode}
%    We save the theorem style to be used in the token register
%    |\theorem@style|.
%    \begin{macrocode}
      {\theorem@style{#1}}%
%    \end{macrocode}
%    Now we ``evaluate'' the theorem style: this means, we call the
%    macro |\th@|\meta{style} which will activate the relevant
%    definitions which are contained in a separate file.  This is done
%    in a group to suppress changes to the current font.  This could
%    otherwise pose problems together with the new font selection
%    scheme\footnote{When I printed the original article using the new
%    font selection scheme I ended with a document with slanted
%    typefaces (text headings and all) simply because one of the
%    theorem styles used \texttt{\bslash sl} at toplevel.} if the
%    |\th@|\meta{style} is evaluated a second time.
% \changes{v2.1a}{90/02/15}{Group added around `th@..}
% \changes{v2.1a}{90/02/15}{Reseting of `..@begintheorem removed.}
%    \begin{macrocode}
      \begingroup
        \csname th@\the\theorem@style \endcsname
      \endgroup}
%    \end{macrocode}
% \end{macro}
%
%
% \begin{macro}{\@begintheorem}
% \changes{v2.1a}{90/02/15}{Relaxed at toplevel.}
% \begin{macro}{\@opargbegintheorem}
% \changes{v2.1a}{90/02/15}{Relaxed at toplevel.}
%    We reset |\@begintheorem| and |\@opargbegintheorem| to
%    |\relax| since these commands are no longer necessary at
%    toplevel. This will save a few tokens.
%    \begin{macrocode}
\global\let\@begintheorem\relax
\global\let\@opargbegintheorem\relax
%    \end{macrocode}
% \end{macro}
% \end{macro}
%
% \begin{macro}{\theorem@style}
%    Obviously the \textsf{token} register used above has to be
%    allocated. To assure the utmost compatibility with the
%    original \LaTeX{} definition, we set the default theorem style to
%    \texttt{plain}, which implements the usual \LaTeX{} convention.
% \changes{v2.0f}{89/07/20}{Unnecessary `global' removed.}
%    \begin{macrocode}
\newtoks\theorem@style
\global\theorem@style{plain}
%    \end{macrocode}
% \end{macro}
%
% \begin{macro}{\theorembodyfont}
% \begin{macro}{\theorem@bodyfont}
%    If the |\theorembodyfont| is set by the user then it should not
%    interact with the default font set in the theorem style. When the
%    new font selection is in force this may happen if, for example,
%    the default is |\itshape| and the new |\theorembodyfont| is
%    |\sffamily|. So we add a |\reset@font| command in front of the
%    user definition.
% \changes{v2.0f}{89/07/20}{Unnecessary `global' removed.}
% \changes{v2.0g}{89/09/19}{`theoremfont to `theorembodyfont.}
% \changes{v2.1c}{92/06/22}{`reset@font added and logic changed.}
% \changes{v2.1d}{92/10/28}{Reset to empty if arg is empty}
%    \begin{macrocode}
\gdef\theorembodyfont#1{%
%    \end{macrocode}
%    We check if the argument supplied is empty and if so put nothing
%    into the |\theorem@bodyfont| token register to allow for
%    |\theorembodyfont{}| as a mean of using the default of the
%    current |\theoremstyle|.
%    \begin{macrocode}
   \def\@tempa{#1}%
   \ifx\@tempa\@empty
    \theorem@bodyfont{}%
   \else
    \theorem@bodyfont{\reset@font#1}%
   \fi
   }
\newtoks\theorem@bodyfont
\global\theorem@bodyfont{}
%    \end{macrocode}
% \end{macro}
% \end{macro}
%
%
% \begin{macro}{\theoremheaderfont}
% \changes{v2.0g}{89/09/19}{Macro added.}
%    The font for the theorem headers is handled differently because
%    this definition applies to all theorem styles.
%    \begin{macrocode}
\gdef\theoremheaderfont#1{\gdef\theorem@headerfont{#1}%
%    \end{macrocode}
%    After using the macro once it is redefined to produce an error
%    message.
%    \begin{macrocode}
       \gdef\theoremheaderfont##1{%
        \typeout{\string\theoremheaderfont\space should be used
                 only once.}}}
%    \end{macrocode}
% \end{macro}
%
% \begin{macro}{\theorem@headerfont}
% \changes{v2.0g}{89/09/19}{Macro added.}
%    To set the |\theorem@headerfont| default we first test if
%    the new fontselection scheme is in force.
% \changes{v2.1a}{90/02/15}{Support for new font selection.}
%    \begin{macrocode}
\ifx\upshape\undefined
%    \end{macrocode}
%    If not we define it to expand into |\bfseries|. We don't use
%    |\let| just in case a following style option redefines this
%    macro.
%    \begin{macrocode}
\gdef\theorem@headerfont{\bfseries}
%    \end{macrocode}
%    Otherwise we reset the current shape before calling |\bfseries|.
%    \begin{macrocode}
\else \gdef\theorem@headerfont{\normalfont\bfseries}\fi
%    \end{macrocode}
% \end{macro}
%
% \begin{macro}{\th@plain}
% \begin{macro}{\th@break}
% \begin{macro}{\th@marginbreak}
% \begin{macro}{\th@changebreak}
% \begin{macro}{\th@change}
% \begin{macro}{\th@margin}
%    The different styles are defined in macros such as
%    |\th@plain|.  Since memory space is precious in
%    ``non-Big-versions'', we have to avoid offering too many unused
%    definitions.  Therefore we define these styles in separate files
%    that can be loaded on demand.  Thus the commands themselves only
%    load these files. We use |\@input@|, a \LaTeXe{} internal command
%    that ensures that the file will be listed with |\listfiles|.
% \changes{v2.2a}{94/02/02}{Use `@input@ to load theorem layout files}
%    \begin{macrocode}
\gdef\th@plain{\@input@{thp.sty}}
\gdef\th@break{\@input@{thb.sty}}
\gdef\th@marginbreak{\@input@{thmb.sty}}
\gdef\th@changebreak{\@input@{thcb.sty}}
\gdef\th@change{\@input@{thc.sty}}
\gdef\th@margin{\@input@{thm.sty}}
%    \end{macrocode}
%    This list will be expanded when new styles become
%    available.  For testing,  just append new theorem substyles
%    as document options.
% \end{macro}
% \end{macro}
% \end{macro}
% \end{macro}
% \end{macro}
% \end{macro}
%
%
%
% \subsection{Definition of a new theorem set}
%
% As already pointed out, a new theorem environment can be defined in
% three different ways:
% \begin{verbatim}
%    \newtheorem{Lem}{Lemma}
%    \newtheorem{Lem}{Lemma}[section]
%    \newtheorem{Lem}[Theorem]{Lemma}
%\end{verbatim}
% The function of the macro |\newtheorem| is to recognize these cases
% and then to branch into one of the three macros |\@ynthm|, |\@xnthm|
% or |\@othm|.  This mechanism is adopted unchanged from
% \cite{latex:source}; the essential point here is that, for example,
% in the second case, the arguments \texttt{Lem}, \texttt{Lemma} and
% \texttt{section} are passed over to the macro |\@xnthm|.
%
% We inspect this case first because the others present fewer problems,
% and thus are easily derived from this one.
%
% \begin{macro}{\@xnthm}
%    For our example arguments, the macro |\@xnthm| must fulfill
%    the following:
%    \begin{itemize}
%       \item
%          Define a new \LaTeX-counter `Lem'
%       \item
%          reset this counter within a |\section|
%       \item
%          define the macro |\theLem|
%       \item
%          define the environment macros |\Lem| and
%          |\endLem| using the current
%          |\theoremstyle| and |\theorem@bodyfont|.
%    \end{itemize}
%    Obviously, all this should happen only if the first argument
%    of |\@xnthm| (i.e.\ \texttt{Lem} in our example) is chosen so
%    as not to conflict with any previously defined commands or
%    environments.  This test is performed by the \LaTeX{} macro
%    |\@ifdefinable|.
%    \begin{macrocode}
\gdef\@xnthm#1#2[#3]{\expandafter\@ifdefinable\csname #1\endcsname
%    \end{macrocode}
%    Therefore, the first argument of |\@ifdefinable| is the
%    expansion (in the example, |\Lem|) of
%    |\csname||#1\endcsname|.  The second argument is
%    executed only if the test has been completed successfully.
%    \begin{macrocode}
   {%
%    \end{macrocode}
%    Now we define the new counter. The names of the \LaTeX{}
%    macros employed should speak for themselves:
% \changes{v2.2b}{1995/10/07}{Use \cs{@newctr}}
% \changes{v2.2c}{1995/11/23}{Use \cs{@newctr} properly.}
%    \begin{macrocode}
    \@definecounter{#1}\@newctr{#1}[#3]%
%    \end{macrocode}
%    Using |\@newctr| will give a proper error message if the counter
%    in |#3| is not defined.
%    In defining `|\theLem|' we must generate the desired macro
%    name by use of |\expandafter| and |\csname|.
%    \begin{macrocode}
    \expandafter\xdef\csname the#1\endcsname
%    \end{macrocode}
%    An |\xdef| is used in order to make the definition global,
%    and to ensure that it contains the replacement texts of
%    |\@thmcountersep| and |\@thmcounter|.\footnote{These
%    two macros can be defined by the document style.  Their default
%    values produce a `\texttt{.}' as separation and an arabic
%    representation of the number.} However, not everything should be
%    expanded. For example, it saves space to use |\thesection|
%    instead of its---at times---lengthy expansion.
%    \begin{macrocode}
      {\expandafter \noexpand \csname the#3\endcsname
       \@thmcountersep \@thmcounter{#1}}%
%    \end{macrocode}
%    Thus with the defaults of \LaTeX{}, |\theLem| would be
%    replaced by the command sequence |\thesection.\arabic{Lem}|.
%
%    We will now look at the definition of the macro which is executed
%    at the beginning of the actual environment (in our example this
%    macro is |\Lem|). It should be noted that we use an
%    ``|\expandafter| trick'' to expand only certain parts of the
%    replacement text at the time of the definition.
%    \begin{macrocode}
    \def\@tempa{\global\@namedef{#1}}%
    \expandafter \@tempa \expandafter{%
%    \end{macrocode}
%    First, the macro that contains the current definitions of
%    |\@begintheorem| and |\@opargtheorem| should be called
%    up.  The name of this macro---as is already known---has the form
%    |\th@|\meta{theorem\ style}; therefore, it must be called by
%    \begin{macrocode}
      \csname th@\the \theorem@style
%    \end{macrocode}
%    In addition the default theorem font should be changeable, i.e.\
%    we have to insert the contents of |\theorem@bodyfont|. For
%    that reason, we expand even further, beyond |\endcsname|,
%    and thus insert the contents of the \textsf{token} register
%    |\theorem@bodyfont| in the replacement text.
%    \begin{macrocode}
            \expandafter \endcsname \the \theorem@bodyfont
%    \end{macrocode}
%    Now it is time to call the macro |\@thm| which takes over
%    the further processing.  It has two arguments: the current
%    counter name (in our example, \texttt{Lem}), and the text of the
%    label (in our example, \texttt{Lemma}).
%    \begin{macrocode}
     \@thm{#1}{#2}}%
%    \end{macrocode}
%    With this, the `sub-definition' is complete.  The macro
%    |\@endtheorem| ends a theorem environment and is, so far, nothing
%    but an |\endtrivlist|.  (Hence it is defined globally, and not
%    within the theorem styles.\footnote{This has to be changed as
%    soon as theorem styles that change \texttt{\bslash @endtheorem}
%    exist.  In such a case, all existing styles must be changed as
%    well since they will have to reset the macro.}) Therefore, we can
%    set it equivalent to the macro that ends the theorem set (in our
%    example, |\endLem|).  However, if some day theorem styles exist
%    that do change |\@endtheorem|, we would have to use the
%    commented-out line instead.
%    \begin{macrocode}
    \global \expandafter \let \csname end#1\endcsname \@endtheorem
%   \global\@namedef{end#1}{\@endtheorem}%
%    \end{macrocode}
%    With these commands all the required definitions are employed,
%    unless the test |\@ifdefinable| has failed.  Therefore, we
%    end the second argument of this macro and with it the definition
%    of |\@xnthm|.
%    \begin{macrocode}
   }}
%    \end{macrocode}
% \end{macro}
%
% \begin{macro}{\@ynthm}
%    The definition of |\@ynthm| is completely analogous. In
%    this case the new counter that is defined is not reset within
%    another counter; thus the definition of |\the...|
%    is simplified:
%    \begin{macrocode}
\gdef\@ynthm#1#2{\expandafter\@ifdefinable\csname #1\endcsname
   {\@definecounter{#1}%
    \expandafter\xdef\csname the#1\endcsname{\@thmcounter{#1}}%
%    \end{macrocode}
%    The rest of the definition corresponds literally to that of
%    |\@xnthm|:
%    \begin{macrocode}
    \def\@tempa{\global\@namedef{#1}}\expandafter \@tempa
     \expandafter{\csname th@\the \theorem@style \expandafter
     \endcsname \the\theorem@bodyfont \@thm{#1}{#2}}%
    \global \expandafter \let \csname end#1\endcsname \@endtheorem}}
%    \end{macrocode}
% \end{macro}
%
% \begin{macro}{\@othm}
%    The definition of |\@othm| does not contain anything new.
%    \begin{macrocode}
\gdef\@othm#1[#2]#3{%
%    \end{macrocode}
% \changes{v2.2b}{95/10/07}{Build in check for counter defined}
%    We do not define a new counter but instead use one that has
%    already been defined.  Thus the only definition we need is that
%    of this pseudo-counter (i.e.\ |\the|\meta{env.\ name}).
%    First we check if |#2| corresponds to a known counter name.
%    \begin{macrocode}
  \expandafter\ifx\csname c@#2\endcsname\relax
   \@nocounterr{#2}%
  \else
   \expandafter\@ifdefinable\csname #1\endcsname
   {\expandafter \xdef \csname the#1\endcsname
     {\expandafter \noexpand \csname the#2\endcsname}%
%    \end{macrocode}
%    All other parts of the definition can be adopted from
%    |\@xnthm|. We have to remember, though, that in this case
%    the name of the current counter and the theorem label have moved
%    to the second and third arguments.
%    \begin{macrocode}
    \def\@tempa{\global\@namedef{#1}}\expandafter \@tempa
     \expandafter{\csname th@\the \theorem@style \expandafter
     \endcsname \the\theorem@bodyfont \@thm{#2}{#3}}%
    \global \expandafter \let \csname end#1\endcsname \@endtheorem}%
  \fi}
%    \end{macrocode}
% \end{macro}
%
% \subsection{Macros that are employed in a {\envfont theorem}
%    environment}
%
% \begin{macro}{\@thm}
%    The macro |\@thm| has to increase the current counter.  Then,
%    depending on whether the environment has (or does not have) an
%    optional argument, it has to branch into either |\@begintheorem|
%    or |\@opargtheorem|.
%    \begin{macrocode}
\gdef\@thm#1#2{\refstepcounter{#1}%
%    \end{macrocode}
%    Now we start a {\envfont trivlist} environment, and give
%    |\@topsep| and |\@topsepadd| the values of the \textsf{skip}
%    registers |\theorempreskipamount| and |\theorempostskipamount|.
%    The value in |\@topsep| is the vertical space that is inserted by
%    the first (and only) |\item| in our |\trivlist| whilst
%    |\@topsepadd| is inserted by |\@endparenv| at the end of that
%    {\envfont trivlist} environment.  By using these registers, we
%    obtain the desired space around a {\envfont theorem} environment.
%    \begin{macrocode}
   \trivlist
   \@topsep \theorempreskipamount               % used by first \item
   \@topsepadd \theorempostskipamount           % used by \@endparenv
%    \end{macrocode}
%    Now we have to test whether an optional argument has been given.
% \changes{v2.0d}{89/07/18}{Forgotten percent added.}
%    \begin{macrocode}
   \@ifnextchar [%
%    \end{macrocode}
%    If there is an optional argument, we will call |\@ythm|, and
%    move the arguments read back into the input stream.
%    \begin{macrocode}
   {\@ythm{#1}{#2}}%
%    \end{macrocode}
%    If not, we call |\@begintheorem|. Its first argument is the
%    name of the theorem set (hence the second argument of
%    |\@thm|). Its second argument is the macro that produces the
%    current number.
%    \begin{macrocode}
   {\@begintheorem{#2}{\csname the#1\endcsname}\ignorespaces}}
%    \end{macrocode}
% \end{macro}
%
% \begin{macro}{\@xthm}
% \begin{macro}{\@ythm}
%    Both these macros were originally called by |\@thm|.  We do
%    not need |\@xthm| anymore, hence we reset it to
%    |\relax|. The definition of |\@ythm| has not changed at
%    all from its definition in \LaTeX. In order to make the macros
%    easier to understand, we will
%    nevertheless present it (commented out).
%    \begin{macrocode}
\global\let\@xthm\relax
% \def\@ythm#1#2[#3]{\@opargbegintheorem{#2}{\csname
%           the#1\endcsname}{#3}\ignorespaces}
%</package>
%    \end{macrocode}
%    The primitive |\ignorespaces| in |\@ythm| and
%    |\@thm| is needed to remove the spaces between the
%    |\begin{...}| and the actual text.
% \end{macro}
% \end{macro}
%
%
% \subsection{Definition of the theorem substyles}
%
% As already pointed out, the theorem substyles, defined below, are
% only loaded when necessary.  Note that all these substyles, except
% \texttt{plain}, have |\slshape| as the default body font.
%
% \subsubsection{The \texttt{plain} style}
%
% \label{dok:th@plain} As the following macros use |@|, we have
% to locally set the |\catcode| of this symbol to ``letter''.
% This happens within a group, so that we do not have to worry about
% which |\catcode| that symbol had before.
%    \begin{macrocode}
%<*thp>
\begingroup \makeatletter
%    \end{macrocode}
% Since we are now within a group, we must make all definitions
% globally.  First we make sure that \texttt{theorem.sty} is loaded.
% This will allow us to use this file as a document style option
% without having to call \texttt{theorem} itself as an option.  At the
% same time, we assure that at least version 2 is loaded, since
% |\theorem@style| was not defined in earlier versions.
%    \begin{macrocode}
\@ifundefined{theorem@style}{\input{theorem.sty}}{}
\ProvidesFile{thp.sty}
             [\FMithmInfo]
%    \end{macrocode}
%
%
% \begin{macro}{\th@plain}
% \begin{macro}{\@begintheorem}
% \begin{macro}{\@opargbegintheorem}
%    |\theoremstyle{plain}| corresponds to the original
%    definition, except that the distances to the surrounding text are
%    determined by the parameters |\theorempreskipamount| and
%    |\theorempostskipamount|.
%    First we set the default body font.
%    \begin{macrocode}
\gdef\th@plain{\normalfont\itshape
%    \end{macrocode}
%    Then we define |\@begintheorem| and
%    |\@opargbegintheorem|. These two macros define how the
%    header of a theorem is typeset.  |\@opargbegintheorem| will
%    be called if a \textsf{theorem} environment with an optional
%    argument is encountered; otherwise, the header is constructed by
%    calling |\@begintheorem|.  If one of these macros is
%    executed, we are within a \textsf{trivlist} environment started by
%    |\@thm|.  So the theorem header is produced with an
%    |\item| command.
%
%    Instead of specifying the header font directly, all standard
%    theorem styles use the |\theorem@headerfont| macro to allow
%    customization.  The extra space (|\labelsep|) is necessary
%    because of problems in the \textsf{trivlist} environment.
%    \begin{macrocode}
  \def\@begintheorem##1##2{%
        \item[\hskip\labelsep \theorem@headerfont ##1\ ##2]}%
%    \end{macrocode}
%    The definition of |\@opargbegintheorem| is completely
%    analogous. The only difference is the fact that there exists a
%    third argument (which is the optional parameter of the
%    environment and contains additional information about the
%    theorem).  Customarily we enclose it in parentheses.
%    \begin{macrocode}
\def\@opargbegintheorem##1##2##3{%
   \item[\hskip\labelsep \theorem@headerfont ##1\ ##2\ (##3)]}}
%    \end{macrocode}
% \end{macro}
% \end{macro}
% \end{macro}
%
% \noindent
% We conclude with an |\endgroup| to restore the
% |\catcode| of |@|.
%    \begin{macrocode}
\endgroup
%</thp>
%    \end{macrocode}
%
%
% \subsubsection{The \texttt{break} style}
%
% This style option is stored in the file |thb.sty|.
% For the next two lines see the documentation for |\th@plain| on
% page~\pageref{dok:th@plain}.
%    \begin{macrocode}
%<*thb>
\begingroup \makeatletter
\@ifundefined{theorem@style}{\input{theorem.sty}}{}
\ProvidesFile{thb.sty}
             [\FMithmInfo]
%    \end{macrocode}
%
% \begin{macro}{\th@break}
%    |\theoremstyle{break}| produces a line break after the name
%    of the theorem.  The font is |\slshape|.  Hence, we define
%    |\th@break| as follows:
%    \begin{macrocode}
\gdef\th@break{\normalfont\slshape
  \def\@begintheorem##1##2{\item[%
%    \end{macrocode}
%    We run into the following problem: it is not possible to create
%    the header with |\item[|\meta{title}|]| and then start
%    a new line by, for example, |\mbox{}\\|. Such a definition
%    will fail whenever a list environment follows immediately.  With
%    the above construction, the |\mbox{}| causes the switch
%    |@inlabel| (cf.\ definition of |\list| and
%    |\trivlist| in \cite{latex:source}) to be set to \texttt{false}
%    and so the following list will insert additional vertical
%    space (|\topskip|). This is quite annoying. Therefore, we
%    create the line break within the |\item|. In order to ensure
%    that the text will begin at the proper position in the following
%    line, we simply pretend that the label does not take any
%    room.\footnote{This will lead to problems whenever very high
%    symbols occurring in the line tower into the
%    heading. So, something else has to be done here sometime.}
%    \begin{macrocode}
       \rlap{\vbox{\hbox{\hskip \labelsep\theorem@headerfont ##1\ ##2}%
                   \hbox{\strut}}}]}%
%    \end{macrocode}
%    Again, the definition of |\@opargbegintheorem| is
%    completely analogous.
%    \begin{macrocode}
\def\@opargbegintheorem##1##2##3{%
  \item[\rlap{\vbox{\hbox{\hskip \labelsep \theorem@headerfont
                     ##1\ ##2\ (##3)}%
                    \hbox{\strut}}}]}}
\endgroup
%</thb>
%    \end{macrocode}
% \end{macro}
%
%
% \subsubsection{The \texttt{changebreak} style}
%
%    \begin{macrocode}
% This style option is stored in the file |thcb.sty|.
%    \begin{macrocode}
%<*thcb>
\begingroup \makeatletter
\@ifundefined{theorem@style}{\input{theorem.sty}}{}
\ProvidesFile{thcb.sty}
             [\FMithmInfo]
%    \end{macrocode}
%
% \begin{macro}{\th@changebreak}
%    The \texttt{change-break} theorem style is like \texttt{break}
%    but with interchange of theorem name and theorem number.  Thus we
%    define |\th@changebreak| as follows:
%    \begin{macrocode}
\gdef\th@changebreak{\normalfont\slshape
  \def\@begintheorem##1##2{\item
       [\rlap{\vbox{\hbox{\hskip\labelsep \theorem@headerfont ##2\ ##1}%
                    \hbox{\strut}}}]}%
\def\@opargbegintheorem##1##2##3{%
  \item[\rlap{\vbox{\hbox{\hskip\labelsep \theorem@headerfont
                     ##2\ ##1\ (##3)}%
                    \hbox{\strut}}}]}}
\endgroup
%</thcb>
%    \end{macrocode}
% \end{macro}
%
%
% \subsubsection{The \texttt{change} style}
%
% This style option is stored in the file |thc.sty|.
%    \begin{macrocode}
%<*thc>
\begingroup \makeatletter
\@ifundefined{theorem@style}{\input{theorem.sty}}{}
\ProvidesFile{thc.sty}
             [\FMithmInfo]
%    \end{macrocode}
%
% \begin{macro}{\th@change}
%    The \texttt{change} theorem style corresponds to the
%    \texttt{change break} style without a linebreak after the header.
%    To say it in another way, it's the same as the \texttt{plain}
%    style but with number and name interchanged and |\slshape| as the
%    default font.
%    \begin{macrocode}
\gdef\th@change{\normalfont\slshape
  \def\@begintheorem##1##2{\item
       [\hskip\labelsep \theorem@headerfont ##2\ ##1]}%
\def\@opargbegintheorem##1##2##3{%
  \item[\hskip\labelsep \theorem@headerfont ##2\ ##1\ (##3)]}}
\endgroup
%</thc>
%    \end{macrocode}
% \end{macro}
%
%
% \subsubsection{The \texttt{marginbreak} style}
%
% This style option is the one used most often at Mainz.  It is saved
% in the file |thmb.sty|.
%    \begin{macrocode}
%<*thmb>
\begingroup \makeatletter
\@ifundefined{theorem@style}{\input{theorem.sty}}{}
\ProvidesFile{thmb.sty}
             [\FMithmInfo]
%    \end{macrocode}
%
% \begin{macro}{\th@marginbreak}
%    The \texttt{margin break} style is nearly the same as the
%    \texttt{change break} style. The only difference is the placement
%    of the theorem number.  We use |\llap| to place it in the left
%    margin.
%
%    In this style |\labelsep| denotes the separation between the
%    number and the text.
%    \begin{macrocode}
\gdef\th@marginbreak{\normalfont\slshape
  \def\@begintheorem##1##2{\item
       [\rlap{\vbox{\theorem@headerfont
                    \hbox{\llap{##2}\hskip\labelsep ##1}%
                    \hbox{\strut}}}]}%
\def\@opargbegintheorem##1##2##3{%
  \item[\rlap{\vbox{\theorem@headerfont
                    \hbox{\llap{##2}\hskip\labelsep ##1\ (##3)}%
                    \hbox{\strut}}}]}}
\endgroup
%</thmb>
%    \end{macrocode}
% \changes{v2.1c}{90/02/17}{`endgroup outside macrocode env.}
% \end{macro}
%
%
%
% \subsubsection{The \texttt{margin} style}
%
% This style option is stored in the file |thm.sty|.
%    \begin{macrocode}
%<*thm>
\begingroup \makeatletter
\@ifundefined{theorem@style}{\input{theorem.sty}}{}
\ProvidesFile{thm.sty}
             [\FMithmInfo]
%    \end{macrocode}
%
% \begin{macro}{\th@margin}
%    Again this is only a variant of the theorem styles described
%    above without any new ideas.
%    \begin{macrocode}
\gdef\th@margin{\normalfont\slshape
  \def\@begintheorem##1##2{\item
       [\theorem@headerfont \llap{##2}\hskip\labelsep ##1]}%
\def\@opargbegintheorem##1##2##3{%
  \item[\theorem@headerfont \llap{##2}\hskip\labelsep ##1\ (##3)]}}
\endgroup
%</thm>
%    \end{macrocode}
% \end{macro}
%
%
%
%
% \subsection{Final Definitions}
%
% \begin{macro}{\theorempreskipamount}
% \begin{macro}{\theorempostskipamount}
%    The \textsf{skip} parameters that regulate the vertical empty space
%    before and after the {\envfont theorem} environment have
%    to be allocated as well.
% \changes{v2.0f}{89/07/20}{Unnecessary `global' removed.}
%    \begin{macrocode}
%<*package>
\newskip\theorempreskipamount
\newskip\theorempostskipamount
%    \end{macrocode}
%    Since we have used the same values for all theorem sets, we now can
%    assign them.
%    \begin{macrocode}
\global\setlength\theorempreskipamount{12pt plus 5pt minus 3pt}
\global\setlength\theorempostskipamount{8pt plus 3pt minus 1.5pt}
%    \end{macrocode}
% \end{macro}
% \end{macro}
%
%
% \begin{macro}{\@endtheorem}
%    The same holds for the macro |\@endtheorem|, which ends a
%    {\envfont theorem} environment. Since it is the same for all
%    theorem sets, it is removed from the macros |\th@|\meta{style}.
%    It simply ends the {\envfont trivlist} environment, which was
%    begun in |\@thm|.
%    \begin{macrocode}
\global\let\@endtheorem=\endtrivlist
%    \end{macrocode}
% \end{macro}
%
%
%
% \begin{macro}{\@preamblecmds}
%    All macros defined above are to be used only in the preamble.
%    Therefore, we insert them in |\@preamblecmds| which will disable
%    them at begin document. This is done by the internal \LaTeXe{}
%    command |\@onlypreamble|.
% \changes{v2.0g}{89/09/19}{`theoremheaderfont and `theorembodyfont
%    added.}
% \changes{v2.2a}{94/02/03}{Use \LaTeXe{} interface}
%    \begin{macrocode}
\@onlypreamble\@xnthm
\@onlypreamble\@ynthm
\@onlypreamble\@othm
\@onlypreamble\newtheorem
\@onlypreamble\theoremstyle
\@onlypreamble\theorembodyfont
\@onlypreamble\theoremheaderfont
%    \end{macrocode}
% \end{macro}
%
% Finally we declare the \texttt{plain} theorem style to be the default.
% \changes{v2.1b}{90/06/04}{Fixed missing `item by declaring default
%    style.}
%    \begin{macrocode}
\theoremstyle{plain}
%</package>
%    \end{macrocode}
%
% \Finale

\endinput
}{}
\ProvidesFile{thcb.sty}
             [\FMithmInfo]
%    \end{macrocode}
%
% \begin{macro}{\th@changebreak}
%    The \texttt{change-break} theorem style is like \texttt{break}
%    but with interchange of theorem name and theorem number.  Thus we
%    define |\th@changebreak| as follows:
%    \begin{macrocode}
\gdef\th@changebreak{\normalfont\slshape
  \def\@begintheorem##1##2{\item
       [\rlap{\vbox{\hbox{\hskip\labelsep \theorem@headerfont ##2\ ##1}%
                    \hbox{\strut}}}]}%
\def\@opargbegintheorem##1##2##3{%
  \item[\rlap{\vbox{\hbox{\hskip\labelsep \theorem@headerfont
                     ##2\ ##1\ (##3)}%
                    \hbox{\strut}}}]}}
\endgroup
%</thcb>
%    \end{macrocode}
% \end{macro}
%
%
% \subsubsection{The \texttt{change} style}
%
% This style option is stored in the file |thc.sty|.
%    \begin{macrocode}
%<*thc>
\begingroup \makeatletter
\@ifundefined{theorem@style}{% \iffalse meta-comment
%
% Copyright (C) 1993-2024
%
% The LaTeX Project and any individual authors listed elsewhere
% in this file.
%
% This file is part of the Standard LaTeX `Tools Bundle'.
% -------------------------------------------------------
%
% It may be distributed and/or modified under the
% conditions of the LaTeX Project Public License, either version 1.3c
% of this license or (at your option) any later version.
% The latest version of this license is in
%    https://www.latex-project.org/lppl.txt
% and version 1.3c or later is part of all distributions of LaTeX
% version 2005/12/01 or later.
%
% The list of all files belonging to the LaTeX `Tools Bundle' is
% given in the file `manifest.txt'.
%
% \fi
%
%%
%
%
\def\FMithmInfo{2023/07/05 v2.2c Theorem extension package (FMi)}
%
% \ProvidesFile{theorem.dtx}[\FMithmInfo]
%\iffalse   % this is a METACOMMENT !
%%
%% Package `theorem' to use with LaTeX2e
%% Copyright (C) 1989-1995 Frank Mittelbach, all rights reserved.
%<+package>\@ifundefined{theorem@style}{}{\endinput}
%<+package>\NeedsTeXFormat{LaTeX2e}[1995/06/01]
%<+package>\ProvidesPackage{theorem}
%<+package>         [\FMithmInfo]
%
% \fi
%
% \changes{v1.0}{87/05/07}{First draft.}
% \changes{v1.0a}{87/05/07}{`thistheoremstyle' added.}
% \changes{v1.1a}{87/07/29}{New concept: redefine the internal
%                          \protect\LaTeX{} commands `@xnthm' etc.}
% \changes{v1.1b}{87/07/31}{`theoremfont' used to specify font.}
% \changes{v1.2a}{88/01/08}{`@endtheorem' now globally defined.}
% \changes{v1.2b}{88/01/11}{`newtheorem' only allowed in the preamble.}
% \changes{v1.3a}{88/02/14}{`theoremfont' now a toks register.}
%
% \changes{v2.0a}{88/11/16}{New implementation and new documentation.}
% \changes{v2.0a}{88/11/16}{No layout changes but internal modifications
%                           old substyles won't work.}
% \changes{v2.0b}{89/01/30}{`global' used for all definitions.}
% \changes{v2.0c}{89/06/15}{Documentation translated to English.
%                            Thanks to LLL and Nina}
% \changes{v2.1a}{90/02/15}{Support for new font selection.}
% \changes{v2.2a}{94/02/03}{Cleanup for LaTeX2e}
%
% \DoNotIndex{\@endparenv,\@ifdefinable,\@ifnextchar,\@ifundefined}
% \DoNotIndex{\@namedef,\@nameuse,\@spaces,\@tempa,\@warning}
% \DoNotIndex{\ }
% \DoNotIndex{\begingroup,\bfseries}
% \DoNotIndex{\csname}
% \DoNotIndex{\def,\do,\docdate}
% \DoNotIndex{\endcsname,\endgroup,\endinput,\endtrivlist,\expandafter}
% \DoNotIndex{\filedate,\fileversion}
% \DoNotIndex{\gdef,\global}
% \DoNotIndex{\hbox,\hskip}
% \DoNotIndex{\ignorespaces,\input,\itshape,\item}
% \DoNotIndex{\labelsep,\let,\llap}
% \DoNotIndex{\makeatletter}
% \DoNotIndex{\noexpand}
% \DoNotIndex{\refstepcounter,\relax,\rlap}
% \DoNotIndex{\setlength,\slshape,\space,\string,\strut}
% \DoNotIndex{\the,\trivlist,\typeout}
% \DoNotIndex{\vbox}
% \DoNotIndex{\xdef}
%
% \def\deflabel#1{\ttfamily #1\hfill}
% \def\deflist#1{\begin{list}{}{\settowidth\labelwidth{\ttfamily #1}%
%                               \setlength\leftmargin\labelwidth
%                               \addtolength\leftmargin\labelsep
%                               \let\makelabel\deflabel}}
% \def\enddeflist{\end{list}}
% \def\envfont{\normalfont\ttfamily}
%
% \MakeShortVerb{\|}
%
% \newcommand{\TUB}{{\sl TUGboat\/}}
%
% \GetFileInfo{theorem.dtx}
% \title{An Extension of the
%        \LaTeX{} theorem environment\thanks{This file
%        has version number \fileversion, last
%        revised \filedate.}}
% \author{Frank Mittelbach}
%
% \MaintainedByLaTeXTeam{tools}
% \maketitle
%
% \begin{abstract}
%    The macros described in this paper yield an extension of the
%    \LaTeX{} theorem mechanism. It is designed to satisfy the
%    different requirements of various journals. Thus, the
%    layout of the ``theorems'' can be manipulated by determining a
%    ``style''. This article describes not only the use, but
%    also the definition, of the necessary macros.
% \end{abstract}
%
% \section*{Preface to version 2.2}
%
% For \LaTeXe{} this package did not need any fundamental changes. I
% only modified the messages generated so that theorem layout styles
% will show up with the |\listfiles| command and cleaned the section
% on the New Font Selection Scheme since this is now included in
% \LaTeX.
%
% \section*{Preface to version 2.1}
%
% This version is identical to 2.0g described in \TUB\ 10\#3 except
% for some internal defaults which are now set depending on the used
% font selection scheme.
%
% This was done to avoid unpleasant surprises if the new font selection
% scheme is in force. For further details see
% section~\ref{sec:newfonts} and \cite{bk:GMS94}.
%
% \section{Introduction}
%
% For our purposes here, ``theorems'' are labelled enunciations,
% often set off from the main text by extra space and a font change.
% Theorems, corollaries, conjectures, definitions, and remarks
% are all instances of ``theorems''.  The ``header'' of these structures
% is composed of a label (such as \textsc{Theorem} or \textsc{Remark})
% and a number which serializes an item in the sequence of items
% with the same label.
%
% Shortly after the introduction of \LaTeX{} at the Fachbereich
% Mathematik in Mainz, the desire to manipulate the layout of
% ``theorems'' arose.  In Mainz, the following two conventions came
% into general use:
% \begin{enumerate}
%   \item The number of the theorem is shown in the margin.
%   \item There is a line break at
%         the end of the theorem header.
% \end{enumerate}
%
% Additionally, some journals require different formats which depend on
% the ``sort of theorem'': e.g.\ often remarks and definitions are set
% in |\upshape|, while |\itshape| is employed for main theorems.
%
% Confronted with these requirements, a theorem environment was
% developed in Mainz which allows separate determination of the
% layout of the ``theorems sets'', comparable to |\pagestyle|.
%
%
% \section{The user interface}
% \subsection{Defining new theorem sets}
%
% \DescribeMacro\newtheorem
% As in the original \LaTeX{} version, the command |\newtheorem|
% defines a new ``theorem set'' or ``theorem-like structure''.
% Two required arguments name the new environment and give the text
% to be typeset with each instance of the new ``set'', while
% an optional argument determines how
% the ``set'' is enumerated:
% \begin{description}
%    \item[\ttfamily \bslash newtheorem\{foo\}\{bar\}]
%       The theorem set {\envfont foo} (whose name is \texttt{bar})
%       uses its own counter.
%    \item[\ttfamily \bslash newtheorem\{foo2\}{[foo]}\{bar2\}]
%       The theorem set {\envfont foo2} (printed name \texttt{bar2})
%       uses the same counter as the theorem set \texttt{foo}.
%    \item[\ttfamily \bslash newtheorem\{foo3\}\{bar3\}{[section]}]
%       The theorem set {\envfont foo3} (printed name \texttt{bar3}) is
%       enumerated within the counter \texttt{section}, i.e.\ with every
%       new |\section| the enumeration begins again with 1, and
%       the enumeration is composed from the section-number and the
%       theorem counter itself.
% \end{description}
%
% \DescribeMacro\theoremstyle
% Additionally, the command |\theoremstyle| can define the layout
% of various, or all, theorem sets.  It should be noted that any
% theorem set defined by |\newtheorem| is typeset in the
% |\theoremstyle| that is current at the time of the definition.
% Thus, the following
% \begin{verbatim}
%    \theoremstyle{break}        \newtheorem{Cor}{Corollary}
%    \theoremstyle{plain}        \newtheorem{Exa}{Example}[section]
%\end{verbatim}                             ^^A No blank at line start
% leads to the result that the set {\envfont Cor} is formatted in the
% style \texttt{break}, while the set {\envfont Exa} and all the
% following ones are formatted in the style \texttt{plain}, unless
% another |\theoremstyle| follows.  Since the definitions installed by
% |\newtheorem| are global, one also can limit |\theoremstyle| locally
% by grouping braces.
%
% \DescribeMacro\theorembodyfont
% The choice of the font for the theorem body is completely
% independent of the chosen |\theoremstyle|; this has proven to
% be very advantageous.  For example,
% \begin{verbatim}
%   {\theorembodyfont{\upshape}          \newtheorem{Rem}{Remark}}
%\end{verbatim}
% defines a theorem set {\envfont Rem}, which will be set in
% |\upshape| in the current layout (which in our example is
% \texttt{plain}).  As with |\theoremstyle|, the |\theorembodyfont|
% chosen is that current at the time of |\newtheorem|.  If
% |\theorembodyfont| is not specified or one defines
% |\theorembodyfont{}|, then the font used will be that defined by the
% |\theoremstyle|.
%
% \DescribeMacro\theoremheaderfont It is also possible to customize
% the font used for the theorem headers.  This is, however, a global
% declaration, and therefore there should be at most one
% |\theoremheaderfont| declaration in the preamble.\footnote{If it is
% actually necessary to have different header fonts, one has to define
% new theorem styles (substituting the desired font) or specify the
% information directly in the \texttt{\bslash newtheorem} declaration
% (the unclean variant).}
%
% \DescribeMacro\theorempreskipamount
% \DescribeMacro\theorempostskipamount Two additional parameters
% affect the vertical space around the theorem environments:
% |\theorempreskipamount| and |\theorempostskipamount| define,
% respectively, the spacing before and after such an environment.
% These parameters apply for all theorem sets and can be manipulated
% with the ordinary length macros.  They are rubber lengths
% (`\textsf{skips}'), and therefore can contain \texttt{plus} and
% \texttt{minus} parts.
%
% Since the definition of theorem sets should---most sensibly---be
% placed in the preamble, we only allow installation there.
% It is therefore possible to release the memory used here after
% |\begin{document}|, in order to make room for other
% applications.
%
%
% \subsection{Existing theorem styles}
%
% The following theorem styles exist to date:
% \begin{deflist}{marginbreak}
%    \item[plain]
%       This theorem style emulates the original \LaTeX{} definition,
%       except that additionally the parameters
%       |\theorem...skipamount| are used.
%    \item[break]
%       In this style, the theorem header is followed by a line
%       break.
%    \item[marginbreak]
%       The theorem number is set in the margin, and there is a line
%       break as in \texttt{break}.
%    \item[changebreak]
%       Like \texttt{break}, but with header number and text
%       interchanged.
%    \item[change]
%       Header number and text are interchanged, without a line break.
%    \item[margin]
%       The number is set in the left margin, without a line break.
% \end{deflist}
% All styles (except \texttt{plain}) select |\slshape| as the default
% |\theorembodyfont|.
%
%
% \subsection{Examples}
%
% Given the above theorem sets {\envfont Cor}, {\envfont Exa}
% and {\envfont Rem},
% suppose that the preamble also contains the declarations:
% \begin{verbatim}
%    \theoremstyle{marginbreak}   \newtheorem{Lem}[Cor]{Lemma}
%    \theoremstyle{change}
%    \theorembodyfont{\itshape}        \newtheorem{Def}[Cor]{Definition}
%
%    \theoremheaderfont{\scshape}
%\end{verbatim}
% Then the following are some typical examples of the typeset output
% resulting from their use.
%
% \begin{Cor}
%    This is a sentence typeset in the theorem environment {\envfont
%    Cor}.
% \end{Cor}
%
% \begin{Exa}
%    This is a sentence typeset in the theorem environment {\envfont
%    Exa}.
% \end{Exa}
%
% \begin{Rem}
%    This is a sentence typeset in the theorem environment {\envfont
%    Rem}.
% \end{Rem}
%
% \begin{Lem}[Ben User]
%    This is a sentence typeset in the theorem environment {\envfont
%    Lem}.
% \end{Lem}
%
% \begin{Def}[Very impressive Definition]
%    This is a sentence typeset in the theorem environment {\envfont
%    Def}.
% \end{Def}
%
% The last two examples show the effect of the optional argument to
% a theorem environment (it is the text typeset in parentheses).
%
%
% \section{Special Considerations}
% \label{sec:newfonts}
%
% Theoremheader and body are implemented as a unit. This means that
% the |\theoremheaderfont|  will inherit characteristics of the
% |\theorembodyfont| in \LaTeXe. Thus, if for example
% |\theorembodyfont| is |\itshape| and |\theoremheaderfont|
% is |\bfseries| the font selected for the header will have the
% characteristics `bold extended italic'. If this is not desired one
% should set the |\theoremheaderfont| to something like
% \begin{verbatim}
%      \theoremheaderfont{\normalfont\bfseries}
%\end{verbatim}
% i.e.\ supplying all necessary font information explicitly.
%
%
%
% \section{Acknowledgements}
%
% The publication of this set of macros was only possible with the
% help of Christina Busse (translating the manuscript into English),
% Joachim Pense (playing the r\^ole of typist), Chris Rowley
% (looking everything over) and many others providing useful
% suggestions.
%
% \MaybeStop{%
%    \begin{thebibliography}{9}
%       \bibitem{bk:GMS94} \textsc{M.~Goossens}, \textsc{F.~Mittelbach}
%         and \textsc{A.~Samarin}.
%         \newblock The \LaTeX{} Companion.
%         \newblock
%         Addison-Wesley, Reading, Massachusetts, 1994.
%       \bibitem{latex:source} \textsc{Lamport, Leslie.} \newblock
%                               \texttt{latex.tex}, version 2.09, date
%                               Feb. 1990.
%    \end{thebibliography}
%    \PrintIndex
%    \PrintChanges}
%
%
% \section{The documentation driver file}
%
% The next bit of code contains the documentation driver file for
% \TeX{}, i.e., the file that will produce the documentation you are
% currently reading. It will be extracted from this file by the
% \texttt{docstrip} program. Since it is the first code in the file
% one can alternatively process this file directly with \LaTeXe{} to
% obtain the documentation.
%
%    \begin{macrocode}
%<*driver>
\documentclass{ltxdoc}

\usepackage{theorem}

%<+driver>% The next few lines define theorem sets which are used
%<+driver>% in the example section of the documentation.

\theoremstyle{break}       \newtheorem{Cor}{Corollary}
\theoremstyle{plain}       \newtheorem{Exa}{Example}[section]
{\theorembodyfont{\upshape}\newtheorem{Rem}{Remark}}
\theoremstyle{marginbreak} \newtheorem{Lem}[Cor]{Lemma}
\theoremstyle{change}
\theorembodyfont{\itshape} \newtheorem{Def}[Cor]{Definition}

\theoremheaderfont{\scshape}

\RecordChanges

\begin{document}
   \DocInput{theorem.dtx}
\end{document}
%</driver>
%    \end{macrocode}
%
% \section{Definition of the Macros}
%
% If the file has been loaded before, we abort immediately. If not the
% package announces itself (this is actually done at the very top of
% the file---the way it is done isn't good style so don't copy it).
% \changes{v2.0e}{89/07/19}{Spaces removed from `typeout'.}
% \changes{v2.2b}{95/11/19}{Announce moved to top of file}
%    \begin{macrocode}
%<*package>
%\@ifundefined{theorem@style}{}{\endinput}
%\def\FMithmInfo{1995/11/19 v2.2b Theorem extension package (FMi)}
%\ProvidesPackage{theorem}[\FMithmInfo]
%    \end{macrocode}
%
%
% \subsection{Definition of theorem styles and fonts}
%
% All the definitions in this file are done globally to allow
% inputting this file inside a group.
%
% \begin{macro}{\theoremstyle}
%    Before a theorem style can be installed, the chosen style must be
%    known. For that reason, we must test to see that
%    |\th@|\meta{style} is known or, more precisely, that it is
%    different from |\relax|. If the style is not known then
%    |\th@plain| is used.
%    \begin{macrocode}
\gdef\theoremstyle#1{%
   \@ifundefined{th@#1}{\@warning
          {Unknown theoremstyle `#1'. Using `plain'}%
          \theorem@style{plain}}%
%    \end{macrocode}
%    We save the theorem style to be used in the token register
%    |\theorem@style|.
%    \begin{macrocode}
      {\theorem@style{#1}}%
%    \end{macrocode}
%    Now we ``evaluate'' the theorem style: this means, we call the
%    macro |\th@|\meta{style} which will activate the relevant
%    definitions which are contained in a separate file.  This is done
%    in a group to suppress changes to the current font.  This could
%    otherwise pose problems together with the new font selection
%    scheme\footnote{When I printed the original article using the new
%    font selection scheme I ended with a document with slanted
%    typefaces (text headings and all) simply because one of the
%    theorem styles used \texttt{\bslash sl} at toplevel.} if the
%    |\th@|\meta{style} is evaluated a second time.
% \changes{v2.1a}{90/02/15}{Group added around `th@..}
% \changes{v2.1a}{90/02/15}{Reseting of `..@begintheorem removed.}
%    \begin{macrocode}
      \begingroup
        \csname th@\the\theorem@style \endcsname
      \endgroup}
%    \end{macrocode}
% \end{macro}
%
%
% \begin{macro}{\@begintheorem}
% \changes{v2.1a}{90/02/15}{Relaxed at toplevel.}
% \begin{macro}{\@opargbegintheorem}
% \changes{v2.1a}{90/02/15}{Relaxed at toplevel.}
%    We reset |\@begintheorem| and |\@opargbegintheorem| to
%    |\relax| since these commands are no longer necessary at
%    toplevel. This will save a few tokens.
%    \begin{macrocode}
\global\let\@begintheorem\relax
\global\let\@opargbegintheorem\relax
%    \end{macrocode}
% \end{macro}
% \end{macro}
%
% \begin{macro}{\theorem@style}
%    Obviously the \textsf{token} register used above has to be
%    allocated. To assure the utmost compatibility with the
%    original \LaTeX{} definition, we set the default theorem style to
%    \texttt{plain}, which implements the usual \LaTeX{} convention.
% \changes{v2.0f}{89/07/20}{Unnecessary `global' removed.}
%    \begin{macrocode}
\newtoks\theorem@style
\global\theorem@style{plain}
%    \end{macrocode}
% \end{macro}
%
% \begin{macro}{\theorembodyfont}
% \begin{macro}{\theorem@bodyfont}
%    If the |\theorembodyfont| is set by the user then it should not
%    interact with the default font set in the theorem style. When the
%    new font selection is in force this may happen if, for example,
%    the default is |\itshape| and the new |\theorembodyfont| is
%    |\sffamily|. So we add a |\reset@font| command in front of the
%    user definition.
% \changes{v2.0f}{89/07/20}{Unnecessary `global' removed.}
% \changes{v2.0g}{89/09/19}{`theoremfont to `theorembodyfont.}
% \changes{v2.1c}{92/06/22}{`reset@font added and logic changed.}
% \changes{v2.1d}{92/10/28}{Reset to empty if arg is empty}
%    \begin{macrocode}
\gdef\theorembodyfont#1{%
%    \end{macrocode}
%    We check if the argument supplied is empty and if so put nothing
%    into the |\theorem@bodyfont| token register to allow for
%    |\theorembodyfont{}| as a mean of using the default of the
%    current |\theoremstyle|.
%    \begin{macrocode}
   \def\@tempa{#1}%
   \ifx\@tempa\@empty
    \theorem@bodyfont{}%
   \else
    \theorem@bodyfont{\reset@font#1}%
   \fi
   }
\newtoks\theorem@bodyfont
\global\theorem@bodyfont{}
%    \end{macrocode}
% \end{macro}
% \end{macro}
%
%
% \begin{macro}{\theoremheaderfont}
% \changes{v2.0g}{89/09/19}{Macro added.}
%    The font for the theorem headers is handled differently because
%    this definition applies to all theorem styles.
%    \begin{macrocode}
\gdef\theoremheaderfont#1{\gdef\theorem@headerfont{#1}%
%    \end{macrocode}
%    After using the macro once it is redefined to produce an error
%    message.
%    \begin{macrocode}
       \gdef\theoremheaderfont##1{%
        \typeout{\string\theoremheaderfont\space should be used
                 only once.}}}
%    \end{macrocode}
% \end{macro}
%
% \begin{macro}{\theorem@headerfont}
% \changes{v2.0g}{89/09/19}{Macro added.}
%    To set the |\theorem@headerfont| default we first test if
%    the new fontselection scheme is in force.
% \changes{v2.1a}{90/02/15}{Support for new font selection.}
%    \begin{macrocode}
\ifx\upshape\undefined
%    \end{macrocode}
%    If not we define it to expand into |\bfseries|. We don't use
%    |\let| just in case a following style option redefines this
%    macro.
%    \begin{macrocode}
\gdef\theorem@headerfont{\bfseries}
%    \end{macrocode}
%    Otherwise we reset the current shape before calling |\bfseries|.
%    \begin{macrocode}
\else \gdef\theorem@headerfont{\normalfont\bfseries}\fi
%    \end{macrocode}
% \end{macro}
%
% \begin{macro}{\th@plain}
% \begin{macro}{\th@break}
% \begin{macro}{\th@marginbreak}
% \begin{macro}{\th@changebreak}
% \begin{macro}{\th@change}
% \begin{macro}{\th@margin}
%    The different styles are defined in macros such as
%    |\th@plain|.  Since memory space is precious in
%    ``non-Big-versions'', we have to avoid offering too many unused
%    definitions.  Therefore we define these styles in separate files
%    that can be loaded on demand.  Thus the commands themselves only
%    load these files. We use |\@input@|, a \LaTeXe{} internal command
%    that ensures that the file will be listed with |\listfiles|.
% \changes{v2.2a}{94/02/02}{Use `@input@ to load theorem layout files}
%    \begin{macrocode}
\gdef\th@plain{\@input@{thp.sty}}
\gdef\th@break{\@input@{thb.sty}}
\gdef\th@marginbreak{\@input@{thmb.sty}}
\gdef\th@changebreak{\@input@{thcb.sty}}
\gdef\th@change{\@input@{thc.sty}}
\gdef\th@margin{\@input@{thm.sty}}
%    \end{macrocode}
%    This list will be expanded when new styles become
%    available.  For testing,  just append new theorem substyles
%    as document options.
% \end{macro}
% \end{macro}
% \end{macro}
% \end{macro}
% \end{macro}
% \end{macro}
%
%
%
% \subsection{Definition of a new theorem set}
%
% As already pointed out, a new theorem environment can be defined in
% three different ways:
% \begin{verbatim}
%    \newtheorem{Lem}{Lemma}
%    \newtheorem{Lem}{Lemma}[section]
%    \newtheorem{Lem}[Theorem]{Lemma}
%\end{verbatim}
% The function of the macro |\newtheorem| is to recognize these cases
% and then to branch into one of the three macros |\@ynthm|, |\@xnthm|
% or |\@othm|.  This mechanism is adopted unchanged from
% \cite{latex:source}; the essential point here is that, for example,
% in the second case, the arguments \texttt{Lem}, \texttt{Lemma} and
% \texttt{section} are passed over to the macro |\@xnthm|.
%
% We inspect this case first because the others present fewer problems,
% and thus are easily derived from this one.
%
% \begin{macro}{\@xnthm}
%    For our example arguments, the macro |\@xnthm| must fulfill
%    the following:
%    \begin{itemize}
%       \item
%          Define a new \LaTeX-counter `Lem'
%       \item
%          reset this counter within a |\section|
%       \item
%          define the macro |\theLem|
%       \item
%          define the environment macros |\Lem| and
%          |\endLem| using the current
%          |\theoremstyle| and |\theorem@bodyfont|.
%    \end{itemize}
%    Obviously, all this should happen only if the first argument
%    of |\@xnthm| (i.e.\ \texttt{Lem} in our example) is chosen so
%    as not to conflict with any previously defined commands or
%    environments.  This test is performed by the \LaTeX{} macro
%    |\@ifdefinable|.
%    \begin{macrocode}
\gdef\@xnthm#1#2[#3]{\expandafter\@ifdefinable\csname #1\endcsname
%    \end{macrocode}
%    Therefore, the first argument of |\@ifdefinable| is the
%    expansion (in the example, |\Lem|) of
%    |\csname||#1\endcsname|.  The second argument is
%    executed only if the test has been completed successfully.
%    \begin{macrocode}
   {%
%    \end{macrocode}
%    Now we define the new counter. The names of the \LaTeX{}
%    macros employed should speak for themselves:
% \changes{v2.2b}{1995/10/07}{Use \cs{@newctr}}
% \changes{v2.2c}{1995/11/23}{Use \cs{@newctr} properly.}
%    \begin{macrocode}
    \@definecounter{#1}\@newctr{#1}[#3]%
%    \end{macrocode}
%    Using |\@newctr| will give a proper error message if the counter
%    in |#3| is not defined.
%    In defining `|\theLem|' we must generate the desired macro
%    name by use of |\expandafter| and |\csname|.
%    \begin{macrocode}
    \expandafter\xdef\csname the#1\endcsname
%    \end{macrocode}
%    An |\xdef| is used in order to make the definition global,
%    and to ensure that it contains the replacement texts of
%    |\@thmcountersep| and |\@thmcounter|.\footnote{These
%    two macros can be defined by the document style.  Their default
%    values produce a `\texttt{.}' as separation and an arabic
%    representation of the number.} However, not everything should be
%    expanded. For example, it saves space to use |\thesection|
%    instead of its---at times---lengthy expansion.
%    \begin{macrocode}
      {\expandafter \noexpand \csname the#3\endcsname
       \@thmcountersep \@thmcounter{#1}}%
%    \end{macrocode}
%    Thus with the defaults of \LaTeX{}, |\theLem| would be
%    replaced by the command sequence |\thesection.\arabic{Lem}|.
%
%    We will now look at the definition of the macro which is executed
%    at the beginning of the actual environment (in our example this
%    macro is |\Lem|). It should be noted that we use an
%    ``|\expandafter| trick'' to expand only certain parts of the
%    replacement text at the time of the definition.
%    \begin{macrocode}
    \def\@tempa{\global\@namedef{#1}}%
    \expandafter \@tempa \expandafter{%
%    \end{macrocode}
%    First, the macro that contains the current definitions of
%    |\@begintheorem| and |\@opargtheorem| should be called
%    up.  The name of this macro---as is already known---has the form
%    |\th@|\meta{theorem\ style}; therefore, it must be called by
%    \begin{macrocode}
      \csname th@\the \theorem@style
%    \end{macrocode}
%    In addition the default theorem font should be changeable, i.e.\
%    we have to insert the contents of |\theorem@bodyfont|. For
%    that reason, we expand even further, beyond |\endcsname|,
%    and thus insert the contents of the \textsf{token} register
%    |\theorem@bodyfont| in the replacement text.
%    \begin{macrocode}
            \expandafter \endcsname \the \theorem@bodyfont
%    \end{macrocode}
%    Now it is time to call the macro |\@thm| which takes over
%    the further processing.  It has two arguments: the current
%    counter name (in our example, \texttt{Lem}), and the text of the
%    label (in our example, \texttt{Lemma}).
%    \begin{macrocode}
     \@thm{#1}{#2}}%
%    \end{macrocode}
%    With this, the `sub-definition' is complete.  The macro
%    |\@endtheorem| ends a theorem environment and is, so far, nothing
%    but an |\endtrivlist|.  (Hence it is defined globally, and not
%    within the theorem styles.\footnote{This has to be changed as
%    soon as theorem styles that change \texttt{\bslash @endtheorem}
%    exist.  In such a case, all existing styles must be changed as
%    well since they will have to reset the macro.}) Therefore, we can
%    set it equivalent to the macro that ends the theorem set (in our
%    example, |\endLem|).  However, if some day theorem styles exist
%    that do change |\@endtheorem|, we would have to use the
%    commented-out line instead.
%    \begin{macrocode}
    \global \expandafter \let \csname end#1\endcsname \@endtheorem
%   \global\@namedef{end#1}{\@endtheorem}%
%    \end{macrocode}
%    With these commands all the required definitions are employed,
%    unless the test |\@ifdefinable| has failed.  Therefore, we
%    end the second argument of this macro and with it the definition
%    of |\@xnthm|.
%    \begin{macrocode}
   }}
%    \end{macrocode}
% \end{macro}
%
% \begin{macro}{\@ynthm}
%    The definition of |\@ynthm| is completely analogous. In
%    this case the new counter that is defined is not reset within
%    another counter; thus the definition of |\the...|
%    is simplified:
%    \begin{macrocode}
\gdef\@ynthm#1#2{\expandafter\@ifdefinable\csname #1\endcsname
   {\@definecounter{#1}%
    \expandafter\xdef\csname the#1\endcsname{\@thmcounter{#1}}%
%    \end{macrocode}
%    The rest of the definition corresponds literally to that of
%    |\@xnthm|:
%    \begin{macrocode}
    \def\@tempa{\global\@namedef{#1}}\expandafter \@tempa
     \expandafter{\csname th@\the \theorem@style \expandafter
     \endcsname \the\theorem@bodyfont \@thm{#1}{#2}}%
    \global \expandafter \let \csname end#1\endcsname \@endtheorem}}
%    \end{macrocode}
% \end{macro}
%
% \begin{macro}{\@othm}
%    The definition of |\@othm| does not contain anything new.
%    \begin{macrocode}
\gdef\@othm#1[#2]#3{%
%    \end{macrocode}
% \changes{v2.2b}{95/10/07}{Build in check for counter defined}
%    We do not define a new counter but instead use one that has
%    already been defined.  Thus the only definition we need is that
%    of this pseudo-counter (i.e.\ |\the|\meta{env.\ name}).
%    First we check if |#2| corresponds to a known counter name.
%    \begin{macrocode}
  \expandafter\ifx\csname c@#2\endcsname\relax
   \@nocounterr{#2}%
  \else
   \expandafter\@ifdefinable\csname #1\endcsname
   {\expandafter \xdef \csname the#1\endcsname
     {\expandafter \noexpand \csname the#2\endcsname}%
%    \end{macrocode}
%    All other parts of the definition can be adopted from
%    |\@xnthm|. We have to remember, though, that in this case
%    the name of the current counter and the theorem label have moved
%    to the second and third arguments.
%    \begin{macrocode}
    \def\@tempa{\global\@namedef{#1}}\expandafter \@tempa
     \expandafter{\csname th@\the \theorem@style \expandafter
     \endcsname \the\theorem@bodyfont \@thm{#2}{#3}}%
    \global \expandafter \let \csname end#1\endcsname \@endtheorem}%
  \fi}
%    \end{macrocode}
% \end{macro}
%
% \subsection{Macros that are employed in a {\envfont theorem}
%    environment}
%
% \begin{macro}{\@thm}
%    The macro |\@thm| has to increase the current counter.  Then,
%    depending on whether the environment has (or does not have) an
%    optional argument, it has to branch into either |\@begintheorem|
%    or |\@opargtheorem|.
%    \begin{macrocode}
\gdef\@thm#1#2{\refstepcounter{#1}%
%    \end{macrocode}
%    Now we start a {\envfont trivlist} environment, and give
%    |\@topsep| and |\@topsepadd| the values of the \textsf{skip}
%    registers |\theorempreskipamount| and |\theorempostskipamount|.
%    The value in |\@topsep| is the vertical space that is inserted by
%    the first (and only) |\item| in our |\trivlist| whilst
%    |\@topsepadd| is inserted by |\@endparenv| at the end of that
%    {\envfont trivlist} environment.  By using these registers, we
%    obtain the desired space around a {\envfont theorem} environment.
%    \begin{macrocode}
   \trivlist
   \@topsep \theorempreskipamount               % used by first \item
   \@topsepadd \theorempostskipamount           % used by \@endparenv
%    \end{macrocode}
%    Now we have to test whether an optional argument has been given.
% \changes{v2.0d}{89/07/18}{Forgotten percent added.}
%    \begin{macrocode}
   \@ifnextchar [%
%    \end{macrocode}
%    If there is an optional argument, we will call |\@ythm|, and
%    move the arguments read back into the input stream.
%    \begin{macrocode}
   {\@ythm{#1}{#2}}%
%    \end{macrocode}
%    If not, we call |\@begintheorem|. Its first argument is the
%    name of the theorem set (hence the second argument of
%    |\@thm|). Its second argument is the macro that produces the
%    current number.
%    \begin{macrocode}
   {\@begintheorem{#2}{\csname the#1\endcsname}\ignorespaces}}
%    \end{macrocode}
% \end{macro}
%
% \begin{macro}{\@xthm}
% \begin{macro}{\@ythm}
%    Both these macros were originally called by |\@thm|.  We do
%    not need |\@xthm| anymore, hence we reset it to
%    |\relax|. The definition of |\@ythm| has not changed at
%    all from its definition in \LaTeX. In order to make the macros
%    easier to understand, we will
%    nevertheless present it (commented out).
%    \begin{macrocode}
\global\let\@xthm\relax
% \def\@ythm#1#2[#3]{\@opargbegintheorem{#2}{\csname
%           the#1\endcsname}{#3}\ignorespaces}
%</package>
%    \end{macrocode}
%    The primitive |\ignorespaces| in |\@ythm| and
%    |\@thm| is needed to remove the spaces between the
%    |\begin{...}| and the actual text.
% \end{macro}
% \end{macro}
%
%
% \subsection{Definition of the theorem substyles}
%
% As already pointed out, the theorem substyles, defined below, are
% only loaded when necessary.  Note that all these substyles, except
% \texttt{plain}, have |\slshape| as the default body font.
%
% \subsubsection{The \texttt{plain} style}
%
% \label{dok:th@plain} As the following macros use |@|, we have
% to locally set the |\catcode| of this symbol to ``letter''.
% This happens within a group, so that we do not have to worry about
% which |\catcode| that symbol had before.
%    \begin{macrocode}
%<*thp>
\begingroup \makeatletter
%    \end{macrocode}
% Since we are now within a group, we must make all definitions
% globally.  First we make sure that \texttt{theorem.sty} is loaded.
% This will allow us to use this file as a document style option
% without having to call \texttt{theorem} itself as an option.  At the
% same time, we assure that at least version 2 is loaded, since
% |\theorem@style| was not defined in earlier versions.
%    \begin{macrocode}
\@ifundefined{theorem@style}{\input{theorem.sty}}{}
\ProvidesFile{thp.sty}
             [\FMithmInfo]
%    \end{macrocode}
%
%
% \begin{macro}{\th@plain}
% \begin{macro}{\@begintheorem}
% \begin{macro}{\@opargbegintheorem}
%    |\theoremstyle{plain}| corresponds to the original
%    definition, except that the distances to the surrounding text are
%    determined by the parameters |\theorempreskipamount| and
%    |\theorempostskipamount|.
%    First we set the default body font.
%    \begin{macrocode}
\gdef\th@plain{\normalfont\itshape
%    \end{macrocode}
%    Then we define |\@begintheorem| and
%    |\@opargbegintheorem|. These two macros define how the
%    header of a theorem is typeset.  |\@opargbegintheorem| will
%    be called if a \textsf{theorem} environment with an optional
%    argument is encountered; otherwise, the header is constructed by
%    calling |\@begintheorem|.  If one of these macros is
%    executed, we are within a \textsf{trivlist} environment started by
%    |\@thm|.  So the theorem header is produced with an
%    |\item| command.
%
%    Instead of specifying the header font directly, all standard
%    theorem styles use the |\theorem@headerfont| macro to allow
%    customization.  The extra space (|\labelsep|) is necessary
%    because of problems in the \textsf{trivlist} environment.
%    \begin{macrocode}
  \def\@begintheorem##1##2{%
        \item[\hskip\labelsep \theorem@headerfont ##1\ ##2]}%
%    \end{macrocode}
%    The definition of |\@opargbegintheorem| is completely
%    analogous. The only difference is the fact that there exists a
%    third argument (which is the optional parameter of the
%    environment and contains additional information about the
%    theorem).  Customarily we enclose it in parentheses.
%    \begin{macrocode}
\def\@opargbegintheorem##1##2##3{%
   \item[\hskip\labelsep \theorem@headerfont ##1\ ##2\ (##3)]}}
%    \end{macrocode}
% \end{macro}
% \end{macro}
% \end{macro}
%
% \noindent
% We conclude with an |\endgroup| to restore the
% |\catcode| of |@|.
%    \begin{macrocode}
\endgroup
%</thp>
%    \end{macrocode}
%
%
% \subsubsection{The \texttt{break} style}
%
% This style option is stored in the file |thb.sty|.
% For the next two lines see the documentation for |\th@plain| on
% page~\pageref{dok:th@plain}.
%    \begin{macrocode}
%<*thb>
\begingroup \makeatletter
\@ifundefined{theorem@style}{\input{theorem.sty}}{}
\ProvidesFile{thb.sty}
             [\FMithmInfo]
%    \end{macrocode}
%
% \begin{macro}{\th@break}
%    |\theoremstyle{break}| produces a line break after the name
%    of the theorem.  The font is |\slshape|.  Hence, we define
%    |\th@break| as follows:
%    \begin{macrocode}
\gdef\th@break{\normalfont\slshape
  \def\@begintheorem##1##2{\item[%
%    \end{macrocode}
%    We run into the following problem: it is not possible to create
%    the header with |\item[|\meta{title}|]| and then start
%    a new line by, for example, |\mbox{}\\|. Such a definition
%    will fail whenever a list environment follows immediately.  With
%    the above construction, the |\mbox{}| causes the switch
%    |@inlabel| (cf.\ definition of |\list| and
%    |\trivlist| in \cite{latex:source}) to be set to \texttt{false}
%    and so the following list will insert additional vertical
%    space (|\topskip|). This is quite annoying. Therefore, we
%    create the line break within the |\item|. In order to ensure
%    that the text will begin at the proper position in the following
%    line, we simply pretend that the label does not take any
%    room.\footnote{This will lead to problems whenever very high
%    symbols occurring in the line tower into the
%    heading. So, something else has to be done here sometime.}
%    \begin{macrocode}
       \rlap{\vbox{\hbox{\hskip \labelsep\theorem@headerfont ##1\ ##2}%
                   \hbox{\strut}}}]}%
%    \end{macrocode}
%    Again, the definition of |\@opargbegintheorem| is
%    completely analogous.
%    \begin{macrocode}
\def\@opargbegintheorem##1##2##3{%
  \item[\rlap{\vbox{\hbox{\hskip \labelsep \theorem@headerfont
                     ##1\ ##2\ (##3)}%
                    \hbox{\strut}}}]}}
\endgroup
%</thb>
%    \end{macrocode}
% \end{macro}
%
%
% \subsubsection{The \texttt{changebreak} style}
%
%    \begin{macrocode}
% This style option is stored in the file |thcb.sty|.
%    \begin{macrocode}
%<*thcb>
\begingroup \makeatletter
\@ifundefined{theorem@style}{\input{theorem.sty}}{}
\ProvidesFile{thcb.sty}
             [\FMithmInfo]
%    \end{macrocode}
%
% \begin{macro}{\th@changebreak}
%    The \texttt{change-break} theorem style is like \texttt{break}
%    but with interchange of theorem name and theorem number.  Thus we
%    define |\th@changebreak| as follows:
%    \begin{macrocode}
\gdef\th@changebreak{\normalfont\slshape
  \def\@begintheorem##1##2{\item
       [\rlap{\vbox{\hbox{\hskip\labelsep \theorem@headerfont ##2\ ##1}%
                    \hbox{\strut}}}]}%
\def\@opargbegintheorem##1##2##3{%
  \item[\rlap{\vbox{\hbox{\hskip\labelsep \theorem@headerfont
                     ##2\ ##1\ (##3)}%
                    \hbox{\strut}}}]}}
\endgroup
%</thcb>
%    \end{macrocode}
% \end{macro}
%
%
% \subsubsection{The \texttt{change} style}
%
% This style option is stored in the file |thc.sty|.
%    \begin{macrocode}
%<*thc>
\begingroup \makeatletter
\@ifundefined{theorem@style}{\input{theorem.sty}}{}
\ProvidesFile{thc.sty}
             [\FMithmInfo]
%    \end{macrocode}
%
% \begin{macro}{\th@change}
%    The \texttt{change} theorem style corresponds to the
%    \texttt{change break} style without a linebreak after the header.
%    To say it in another way, it's the same as the \texttt{plain}
%    style but with number and name interchanged and |\slshape| as the
%    default font.
%    \begin{macrocode}
\gdef\th@change{\normalfont\slshape
  \def\@begintheorem##1##2{\item
       [\hskip\labelsep \theorem@headerfont ##2\ ##1]}%
\def\@opargbegintheorem##1##2##3{%
  \item[\hskip\labelsep \theorem@headerfont ##2\ ##1\ (##3)]}}
\endgroup
%</thc>
%    \end{macrocode}
% \end{macro}
%
%
% \subsubsection{The \texttt{marginbreak} style}
%
% This style option is the one used most often at Mainz.  It is saved
% in the file |thmb.sty|.
%    \begin{macrocode}
%<*thmb>
\begingroup \makeatletter
\@ifundefined{theorem@style}{\input{theorem.sty}}{}
\ProvidesFile{thmb.sty}
             [\FMithmInfo]
%    \end{macrocode}
%
% \begin{macro}{\th@marginbreak}
%    The \texttt{margin break} style is nearly the same as the
%    \texttt{change break} style. The only difference is the placement
%    of the theorem number.  We use |\llap| to place it in the left
%    margin.
%
%    In this style |\labelsep| denotes the separation between the
%    number and the text.
%    \begin{macrocode}
\gdef\th@marginbreak{\normalfont\slshape
  \def\@begintheorem##1##2{\item
       [\rlap{\vbox{\theorem@headerfont
                    \hbox{\llap{##2}\hskip\labelsep ##1}%
                    \hbox{\strut}}}]}%
\def\@opargbegintheorem##1##2##3{%
  \item[\rlap{\vbox{\theorem@headerfont
                    \hbox{\llap{##2}\hskip\labelsep ##1\ (##3)}%
                    \hbox{\strut}}}]}}
\endgroup
%</thmb>
%    \end{macrocode}
% \changes{v2.1c}{90/02/17}{`endgroup outside macrocode env.}
% \end{macro}
%
%
%
% \subsubsection{The \texttt{margin} style}
%
% This style option is stored in the file |thm.sty|.
%    \begin{macrocode}
%<*thm>
\begingroup \makeatletter
\@ifundefined{theorem@style}{\input{theorem.sty}}{}
\ProvidesFile{thm.sty}
             [\FMithmInfo]
%    \end{macrocode}
%
% \begin{macro}{\th@margin}
%    Again this is only a variant of the theorem styles described
%    above without any new ideas.
%    \begin{macrocode}
\gdef\th@margin{\normalfont\slshape
  \def\@begintheorem##1##2{\item
       [\theorem@headerfont \llap{##2}\hskip\labelsep ##1]}%
\def\@opargbegintheorem##1##2##3{%
  \item[\theorem@headerfont \llap{##2}\hskip\labelsep ##1\ (##3)]}}
\endgroup
%</thm>
%    \end{macrocode}
% \end{macro}
%
%
%
%
% \subsection{Final Definitions}
%
% \begin{macro}{\theorempreskipamount}
% \begin{macro}{\theorempostskipamount}
%    The \textsf{skip} parameters that regulate the vertical empty space
%    before and after the {\envfont theorem} environment have
%    to be allocated as well.
% \changes{v2.0f}{89/07/20}{Unnecessary `global' removed.}
%    \begin{macrocode}
%<*package>
\newskip\theorempreskipamount
\newskip\theorempostskipamount
%    \end{macrocode}
%    Since we have used the same values for all theorem sets, we now can
%    assign them.
%    \begin{macrocode}
\global\setlength\theorempreskipamount{12pt plus 5pt minus 3pt}
\global\setlength\theorempostskipamount{8pt plus 3pt minus 1.5pt}
%    \end{macrocode}
% \end{macro}
% \end{macro}
%
%
% \begin{macro}{\@endtheorem}
%    The same holds for the macro |\@endtheorem|, which ends a
%    {\envfont theorem} environment. Since it is the same for all
%    theorem sets, it is removed from the macros |\th@|\meta{style}.
%    It simply ends the {\envfont trivlist} environment, which was
%    begun in |\@thm|.
%    \begin{macrocode}
\global\let\@endtheorem=\endtrivlist
%    \end{macrocode}
% \end{macro}
%
%
%
% \begin{macro}{\@preamblecmds}
%    All macros defined above are to be used only in the preamble.
%    Therefore, we insert them in |\@preamblecmds| which will disable
%    them at begin document. This is done by the internal \LaTeXe{}
%    command |\@onlypreamble|.
% \changes{v2.0g}{89/09/19}{`theoremheaderfont and `theorembodyfont
%    added.}
% \changes{v2.2a}{94/02/03}{Use \LaTeXe{} interface}
%    \begin{macrocode}
\@onlypreamble\@xnthm
\@onlypreamble\@ynthm
\@onlypreamble\@othm
\@onlypreamble\newtheorem
\@onlypreamble\theoremstyle
\@onlypreamble\theorembodyfont
\@onlypreamble\theoremheaderfont
%    \end{macrocode}
% \end{macro}
%
% Finally we declare the \texttt{plain} theorem style to be the default.
% \changes{v2.1b}{90/06/04}{Fixed missing `item by declaring default
%    style.}
%    \begin{macrocode}
\theoremstyle{plain}
%</package>
%    \end{macrocode}
%
% \Finale

\endinput
}{}
\ProvidesFile{thc.sty}
             [\FMithmInfo]
%    \end{macrocode}
%
% \begin{macro}{\th@change}
%    The \texttt{change} theorem style corresponds to the
%    \texttt{change break} style without a linebreak after the header.
%    To say it in another way, it's the same as the \texttt{plain}
%    style but with number and name interchanged and |\slshape| as the
%    default font.
%    \begin{macrocode}
\gdef\th@change{\normalfont\slshape
  \def\@begintheorem##1##2{\item
       [\hskip\labelsep \theorem@headerfont ##2\ ##1]}%
\def\@opargbegintheorem##1##2##3{%
  \item[\hskip\labelsep \theorem@headerfont ##2\ ##1\ (##3)]}}
\endgroup
%</thc>
%    \end{macrocode}
% \end{macro}
%
%
% \subsubsection{The \texttt{marginbreak} style}
%
% This style option is the one used most often at Mainz.  It is saved
% in the file |thmb.sty|.
%    \begin{macrocode}
%<*thmb>
\begingroup \makeatletter
\@ifundefined{theorem@style}{% \iffalse meta-comment
%
% Copyright (C) 1993-2024
%
% The LaTeX Project and any individual authors listed elsewhere
% in this file.
%
% This file is part of the Standard LaTeX `Tools Bundle'.
% -------------------------------------------------------
%
% It may be distributed and/or modified under the
% conditions of the LaTeX Project Public License, either version 1.3c
% of this license or (at your option) any later version.
% The latest version of this license is in
%    https://www.latex-project.org/lppl.txt
% and version 1.3c or later is part of all distributions of LaTeX
% version 2005/12/01 or later.
%
% The list of all files belonging to the LaTeX `Tools Bundle' is
% given in the file `manifest.txt'.
%
% \fi
%
%%
%
%
\def\FMithmInfo{2023/07/05 v2.2c Theorem extension package (FMi)}
%
% \ProvidesFile{theorem.dtx}[\FMithmInfo]
%\iffalse   % this is a METACOMMENT !
%%
%% Package `theorem' to use with LaTeX2e
%% Copyright (C) 1989-1995 Frank Mittelbach, all rights reserved.
%<+package>\@ifundefined{theorem@style}{}{\endinput}
%<+package>\NeedsTeXFormat{LaTeX2e}[1995/06/01]
%<+package>\ProvidesPackage{theorem}
%<+package>         [\FMithmInfo]
%
% \fi
%
% \changes{v1.0}{87/05/07}{First draft.}
% \changes{v1.0a}{87/05/07}{`thistheoremstyle' added.}
% \changes{v1.1a}{87/07/29}{New concept: redefine the internal
%                          \protect\LaTeX{} commands `@xnthm' etc.}
% \changes{v1.1b}{87/07/31}{`theoremfont' used to specify font.}
% \changes{v1.2a}{88/01/08}{`@endtheorem' now globally defined.}
% \changes{v1.2b}{88/01/11}{`newtheorem' only allowed in the preamble.}
% \changes{v1.3a}{88/02/14}{`theoremfont' now a toks register.}
%
% \changes{v2.0a}{88/11/16}{New implementation and new documentation.}
% \changes{v2.0a}{88/11/16}{No layout changes but internal modifications
%                           old substyles won't work.}
% \changes{v2.0b}{89/01/30}{`global' used for all definitions.}
% \changes{v2.0c}{89/06/15}{Documentation translated to English.
%                            Thanks to LLL and Nina}
% \changes{v2.1a}{90/02/15}{Support for new font selection.}
% \changes{v2.2a}{94/02/03}{Cleanup for LaTeX2e}
%
% \DoNotIndex{\@endparenv,\@ifdefinable,\@ifnextchar,\@ifundefined}
% \DoNotIndex{\@namedef,\@nameuse,\@spaces,\@tempa,\@warning}
% \DoNotIndex{\ }
% \DoNotIndex{\begingroup,\bfseries}
% \DoNotIndex{\csname}
% \DoNotIndex{\def,\do,\docdate}
% \DoNotIndex{\endcsname,\endgroup,\endinput,\endtrivlist,\expandafter}
% \DoNotIndex{\filedate,\fileversion}
% \DoNotIndex{\gdef,\global}
% \DoNotIndex{\hbox,\hskip}
% \DoNotIndex{\ignorespaces,\input,\itshape,\item}
% \DoNotIndex{\labelsep,\let,\llap}
% \DoNotIndex{\makeatletter}
% \DoNotIndex{\noexpand}
% \DoNotIndex{\refstepcounter,\relax,\rlap}
% \DoNotIndex{\setlength,\slshape,\space,\string,\strut}
% \DoNotIndex{\the,\trivlist,\typeout}
% \DoNotIndex{\vbox}
% \DoNotIndex{\xdef}
%
% \def\deflabel#1{\ttfamily #1\hfill}
% \def\deflist#1{\begin{list}{}{\settowidth\labelwidth{\ttfamily #1}%
%                               \setlength\leftmargin\labelwidth
%                               \addtolength\leftmargin\labelsep
%                               \let\makelabel\deflabel}}
% \def\enddeflist{\end{list}}
% \def\envfont{\normalfont\ttfamily}
%
% \MakeShortVerb{\|}
%
% \newcommand{\TUB}{{\sl TUGboat\/}}
%
% \GetFileInfo{theorem.dtx}
% \title{An Extension of the
%        \LaTeX{} theorem environment\thanks{This file
%        has version number \fileversion, last
%        revised \filedate.}}
% \author{Frank Mittelbach}
%
% \MaintainedByLaTeXTeam{tools}
% \maketitle
%
% \begin{abstract}
%    The macros described in this paper yield an extension of the
%    \LaTeX{} theorem mechanism. It is designed to satisfy the
%    different requirements of various journals. Thus, the
%    layout of the ``theorems'' can be manipulated by determining a
%    ``style''. This article describes not only the use, but
%    also the definition, of the necessary macros.
% \end{abstract}
%
% \section*{Preface to version 2.2}
%
% For \LaTeXe{} this package did not need any fundamental changes. I
% only modified the messages generated so that theorem layout styles
% will show up with the |\listfiles| command and cleaned the section
% on the New Font Selection Scheme since this is now included in
% \LaTeX.
%
% \section*{Preface to version 2.1}
%
% This version is identical to 2.0g described in \TUB\ 10\#3 except
% for some internal defaults which are now set depending on the used
% font selection scheme.
%
% This was done to avoid unpleasant surprises if the new font selection
% scheme is in force. For further details see
% section~\ref{sec:newfonts} and \cite{bk:GMS94}.
%
% \section{Introduction}
%
% For our purposes here, ``theorems'' are labelled enunciations,
% often set off from the main text by extra space and a font change.
% Theorems, corollaries, conjectures, definitions, and remarks
% are all instances of ``theorems''.  The ``header'' of these structures
% is composed of a label (such as \textsc{Theorem} or \textsc{Remark})
% and a number which serializes an item in the sequence of items
% with the same label.
%
% Shortly after the introduction of \LaTeX{} at the Fachbereich
% Mathematik in Mainz, the desire to manipulate the layout of
% ``theorems'' arose.  In Mainz, the following two conventions came
% into general use:
% \begin{enumerate}
%   \item The number of the theorem is shown in the margin.
%   \item There is a line break at
%         the end of the theorem header.
% \end{enumerate}
%
% Additionally, some journals require different formats which depend on
% the ``sort of theorem'': e.g.\ often remarks and definitions are set
% in |\upshape|, while |\itshape| is employed for main theorems.
%
% Confronted with these requirements, a theorem environment was
% developed in Mainz which allows separate determination of the
% layout of the ``theorems sets'', comparable to |\pagestyle|.
%
%
% \section{The user interface}
% \subsection{Defining new theorem sets}
%
% \DescribeMacro\newtheorem
% As in the original \LaTeX{} version, the command |\newtheorem|
% defines a new ``theorem set'' or ``theorem-like structure''.
% Two required arguments name the new environment and give the text
% to be typeset with each instance of the new ``set'', while
% an optional argument determines how
% the ``set'' is enumerated:
% \begin{description}
%    \item[\ttfamily \bslash newtheorem\{foo\}\{bar\}]
%       The theorem set {\envfont foo} (whose name is \texttt{bar})
%       uses its own counter.
%    \item[\ttfamily \bslash newtheorem\{foo2\}{[foo]}\{bar2\}]
%       The theorem set {\envfont foo2} (printed name \texttt{bar2})
%       uses the same counter as the theorem set \texttt{foo}.
%    \item[\ttfamily \bslash newtheorem\{foo3\}\{bar3\}{[section]}]
%       The theorem set {\envfont foo3} (printed name \texttt{bar3}) is
%       enumerated within the counter \texttt{section}, i.e.\ with every
%       new |\section| the enumeration begins again with 1, and
%       the enumeration is composed from the section-number and the
%       theorem counter itself.
% \end{description}
%
% \DescribeMacro\theoremstyle
% Additionally, the command |\theoremstyle| can define the layout
% of various, or all, theorem sets.  It should be noted that any
% theorem set defined by |\newtheorem| is typeset in the
% |\theoremstyle| that is current at the time of the definition.
% Thus, the following
% \begin{verbatim}
%    \theoremstyle{break}        \newtheorem{Cor}{Corollary}
%    \theoremstyle{plain}        \newtheorem{Exa}{Example}[section]
%\end{verbatim}                             ^^A No blank at line start
% leads to the result that the set {\envfont Cor} is formatted in the
% style \texttt{break}, while the set {\envfont Exa} and all the
% following ones are formatted in the style \texttt{plain}, unless
% another |\theoremstyle| follows.  Since the definitions installed by
% |\newtheorem| are global, one also can limit |\theoremstyle| locally
% by grouping braces.
%
% \DescribeMacro\theorembodyfont
% The choice of the font for the theorem body is completely
% independent of the chosen |\theoremstyle|; this has proven to
% be very advantageous.  For example,
% \begin{verbatim}
%   {\theorembodyfont{\upshape}          \newtheorem{Rem}{Remark}}
%\end{verbatim}
% defines a theorem set {\envfont Rem}, which will be set in
% |\upshape| in the current layout (which in our example is
% \texttt{plain}).  As with |\theoremstyle|, the |\theorembodyfont|
% chosen is that current at the time of |\newtheorem|.  If
% |\theorembodyfont| is not specified or one defines
% |\theorembodyfont{}|, then the font used will be that defined by the
% |\theoremstyle|.
%
% \DescribeMacro\theoremheaderfont It is also possible to customize
% the font used for the theorem headers.  This is, however, a global
% declaration, and therefore there should be at most one
% |\theoremheaderfont| declaration in the preamble.\footnote{If it is
% actually necessary to have different header fonts, one has to define
% new theorem styles (substituting the desired font) or specify the
% information directly in the \texttt{\bslash newtheorem} declaration
% (the unclean variant).}
%
% \DescribeMacro\theorempreskipamount
% \DescribeMacro\theorempostskipamount Two additional parameters
% affect the vertical space around the theorem environments:
% |\theorempreskipamount| and |\theorempostskipamount| define,
% respectively, the spacing before and after such an environment.
% These parameters apply for all theorem sets and can be manipulated
% with the ordinary length macros.  They are rubber lengths
% (`\textsf{skips}'), and therefore can contain \texttt{plus} and
% \texttt{minus} parts.
%
% Since the definition of theorem sets should---most sensibly---be
% placed in the preamble, we only allow installation there.
% It is therefore possible to release the memory used here after
% |\begin{document}|, in order to make room for other
% applications.
%
%
% \subsection{Existing theorem styles}
%
% The following theorem styles exist to date:
% \begin{deflist}{marginbreak}
%    \item[plain]
%       This theorem style emulates the original \LaTeX{} definition,
%       except that additionally the parameters
%       |\theorem...skipamount| are used.
%    \item[break]
%       In this style, the theorem header is followed by a line
%       break.
%    \item[marginbreak]
%       The theorem number is set in the margin, and there is a line
%       break as in \texttt{break}.
%    \item[changebreak]
%       Like \texttt{break}, but with header number and text
%       interchanged.
%    \item[change]
%       Header number and text are interchanged, without a line break.
%    \item[margin]
%       The number is set in the left margin, without a line break.
% \end{deflist}
% All styles (except \texttt{plain}) select |\slshape| as the default
% |\theorembodyfont|.
%
%
% \subsection{Examples}
%
% Given the above theorem sets {\envfont Cor}, {\envfont Exa}
% and {\envfont Rem},
% suppose that the preamble also contains the declarations:
% \begin{verbatim}
%    \theoremstyle{marginbreak}   \newtheorem{Lem}[Cor]{Lemma}
%    \theoremstyle{change}
%    \theorembodyfont{\itshape}        \newtheorem{Def}[Cor]{Definition}
%
%    \theoremheaderfont{\scshape}
%\end{verbatim}
% Then the following are some typical examples of the typeset output
% resulting from their use.
%
% \begin{Cor}
%    This is a sentence typeset in the theorem environment {\envfont
%    Cor}.
% \end{Cor}
%
% \begin{Exa}
%    This is a sentence typeset in the theorem environment {\envfont
%    Exa}.
% \end{Exa}
%
% \begin{Rem}
%    This is a sentence typeset in the theorem environment {\envfont
%    Rem}.
% \end{Rem}
%
% \begin{Lem}[Ben User]
%    This is a sentence typeset in the theorem environment {\envfont
%    Lem}.
% \end{Lem}
%
% \begin{Def}[Very impressive Definition]
%    This is a sentence typeset in the theorem environment {\envfont
%    Def}.
% \end{Def}
%
% The last two examples show the effect of the optional argument to
% a theorem environment (it is the text typeset in parentheses).
%
%
% \section{Special Considerations}
% \label{sec:newfonts}
%
% Theoremheader and body are implemented as a unit. This means that
% the |\theoremheaderfont|  will inherit characteristics of the
% |\theorembodyfont| in \LaTeXe. Thus, if for example
% |\theorembodyfont| is |\itshape| and |\theoremheaderfont|
% is |\bfseries| the font selected for the header will have the
% characteristics `bold extended italic'. If this is not desired one
% should set the |\theoremheaderfont| to something like
% \begin{verbatim}
%      \theoremheaderfont{\normalfont\bfseries}
%\end{verbatim}
% i.e.\ supplying all necessary font information explicitly.
%
%
%
% \section{Acknowledgements}
%
% The publication of this set of macros was only possible with the
% help of Christina Busse (translating the manuscript into English),
% Joachim Pense (playing the r\^ole of typist), Chris Rowley
% (looking everything over) and many others providing useful
% suggestions.
%
% \MaybeStop{%
%    \begin{thebibliography}{9}
%       \bibitem{bk:GMS94} \textsc{M.~Goossens}, \textsc{F.~Mittelbach}
%         and \textsc{A.~Samarin}.
%         \newblock The \LaTeX{} Companion.
%         \newblock
%         Addison-Wesley, Reading, Massachusetts, 1994.
%       \bibitem{latex:source} \textsc{Lamport, Leslie.} \newblock
%                               \texttt{latex.tex}, version 2.09, date
%                               Feb. 1990.
%    \end{thebibliography}
%    \PrintIndex
%    \PrintChanges}
%
%
% \section{The documentation driver file}
%
% The next bit of code contains the documentation driver file for
% \TeX{}, i.e., the file that will produce the documentation you are
% currently reading. It will be extracted from this file by the
% \texttt{docstrip} program. Since it is the first code in the file
% one can alternatively process this file directly with \LaTeXe{} to
% obtain the documentation.
%
%    \begin{macrocode}
%<*driver>
\documentclass{ltxdoc}

\usepackage{theorem}

%<+driver>% The next few lines define theorem sets which are used
%<+driver>% in the example section of the documentation.

\theoremstyle{break}       \newtheorem{Cor}{Corollary}
\theoremstyle{plain}       \newtheorem{Exa}{Example}[section]
{\theorembodyfont{\upshape}\newtheorem{Rem}{Remark}}
\theoremstyle{marginbreak} \newtheorem{Lem}[Cor]{Lemma}
\theoremstyle{change}
\theorembodyfont{\itshape} \newtheorem{Def}[Cor]{Definition}

\theoremheaderfont{\scshape}

\RecordChanges

\begin{document}
   \DocInput{theorem.dtx}
\end{document}
%</driver>
%    \end{macrocode}
%
% \section{Definition of the Macros}
%
% If the file has been loaded before, we abort immediately. If not the
% package announces itself (this is actually done at the very top of
% the file---the way it is done isn't good style so don't copy it).
% \changes{v2.0e}{89/07/19}{Spaces removed from `typeout'.}
% \changes{v2.2b}{95/11/19}{Announce moved to top of file}
%    \begin{macrocode}
%<*package>
%\@ifundefined{theorem@style}{}{\endinput}
%\def\FMithmInfo{1995/11/19 v2.2b Theorem extension package (FMi)}
%\ProvidesPackage{theorem}[\FMithmInfo]
%    \end{macrocode}
%
%
% \subsection{Definition of theorem styles and fonts}
%
% All the definitions in this file are done globally to allow
% inputting this file inside a group.
%
% \begin{macro}{\theoremstyle}
%    Before a theorem style can be installed, the chosen style must be
%    known. For that reason, we must test to see that
%    |\th@|\meta{style} is known or, more precisely, that it is
%    different from |\relax|. If the style is not known then
%    |\th@plain| is used.
%    \begin{macrocode}
\gdef\theoremstyle#1{%
   \@ifundefined{th@#1}{\@warning
          {Unknown theoremstyle `#1'. Using `plain'}%
          \theorem@style{plain}}%
%    \end{macrocode}
%    We save the theorem style to be used in the token register
%    |\theorem@style|.
%    \begin{macrocode}
      {\theorem@style{#1}}%
%    \end{macrocode}
%    Now we ``evaluate'' the theorem style: this means, we call the
%    macro |\th@|\meta{style} which will activate the relevant
%    definitions which are contained in a separate file.  This is done
%    in a group to suppress changes to the current font.  This could
%    otherwise pose problems together with the new font selection
%    scheme\footnote{When I printed the original article using the new
%    font selection scheme I ended with a document with slanted
%    typefaces (text headings and all) simply because one of the
%    theorem styles used \texttt{\bslash sl} at toplevel.} if the
%    |\th@|\meta{style} is evaluated a second time.
% \changes{v2.1a}{90/02/15}{Group added around `th@..}
% \changes{v2.1a}{90/02/15}{Reseting of `..@begintheorem removed.}
%    \begin{macrocode}
      \begingroup
        \csname th@\the\theorem@style \endcsname
      \endgroup}
%    \end{macrocode}
% \end{macro}
%
%
% \begin{macro}{\@begintheorem}
% \changes{v2.1a}{90/02/15}{Relaxed at toplevel.}
% \begin{macro}{\@opargbegintheorem}
% \changes{v2.1a}{90/02/15}{Relaxed at toplevel.}
%    We reset |\@begintheorem| and |\@opargbegintheorem| to
%    |\relax| since these commands are no longer necessary at
%    toplevel. This will save a few tokens.
%    \begin{macrocode}
\global\let\@begintheorem\relax
\global\let\@opargbegintheorem\relax
%    \end{macrocode}
% \end{macro}
% \end{macro}
%
% \begin{macro}{\theorem@style}
%    Obviously the \textsf{token} register used above has to be
%    allocated. To assure the utmost compatibility with the
%    original \LaTeX{} definition, we set the default theorem style to
%    \texttt{plain}, which implements the usual \LaTeX{} convention.
% \changes{v2.0f}{89/07/20}{Unnecessary `global' removed.}
%    \begin{macrocode}
\newtoks\theorem@style
\global\theorem@style{plain}
%    \end{macrocode}
% \end{macro}
%
% \begin{macro}{\theorembodyfont}
% \begin{macro}{\theorem@bodyfont}
%    If the |\theorembodyfont| is set by the user then it should not
%    interact with the default font set in the theorem style. When the
%    new font selection is in force this may happen if, for example,
%    the default is |\itshape| and the new |\theorembodyfont| is
%    |\sffamily|. So we add a |\reset@font| command in front of the
%    user definition.
% \changes{v2.0f}{89/07/20}{Unnecessary `global' removed.}
% \changes{v2.0g}{89/09/19}{`theoremfont to `theorembodyfont.}
% \changes{v2.1c}{92/06/22}{`reset@font added and logic changed.}
% \changes{v2.1d}{92/10/28}{Reset to empty if arg is empty}
%    \begin{macrocode}
\gdef\theorembodyfont#1{%
%    \end{macrocode}
%    We check if the argument supplied is empty and if so put nothing
%    into the |\theorem@bodyfont| token register to allow for
%    |\theorembodyfont{}| as a mean of using the default of the
%    current |\theoremstyle|.
%    \begin{macrocode}
   \def\@tempa{#1}%
   \ifx\@tempa\@empty
    \theorem@bodyfont{}%
   \else
    \theorem@bodyfont{\reset@font#1}%
   \fi
   }
\newtoks\theorem@bodyfont
\global\theorem@bodyfont{}
%    \end{macrocode}
% \end{macro}
% \end{macro}
%
%
% \begin{macro}{\theoremheaderfont}
% \changes{v2.0g}{89/09/19}{Macro added.}
%    The font for the theorem headers is handled differently because
%    this definition applies to all theorem styles.
%    \begin{macrocode}
\gdef\theoremheaderfont#1{\gdef\theorem@headerfont{#1}%
%    \end{macrocode}
%    After using the macro once it is redefined to produce an error
%    message.
%    \begin{macrocode}
       \gdef\theoremheaderfont##1{%
        \typeout{\string\theoremheaderfont\space should be used
                 only once.}}}
%    \end{macrocode}
% \end{macro}
%
% \begin{macro}{\theorem@headerfont}
% \changes{v2.0g}{89/09/19}{Macro added.}
%    To set the |\theorem@headerfont| default we first test if
%    the new fontselection scheme is in force.
% \changes{v2.1a}{90/02/15}{Support for new font selection.}
%    \begin{macrocode}
\ifx\upshape\undefined
%    \end{macrocode}
%    If not we define it to expand into |\bfseries|. We don't use
%    |\let| just in case a following style option redefines this
%    macro.
%    \begin{macrocode}
\gdef\theorem@headerfont{\bfseries}
%    \end{macrocode}
%    Otherwise we reset the current shape before calling |\bfseries|.
%    \begin{macrocode}
\else \gdef\theorem@headerfont{\normalfont\bfseries}\fi
%    \end{macrocode}
% \end{macro}
%
% \begin{macro}{\th@plain}
% \begin{macro}{\th@break}
% \begin{macro}{\th@marginbreak}
% \begin{macro}{\th@changebreak}
% \begin{macro}{\th@change}
% \begin{macro}{\th@margin}
%    The different styles are defined in macros such as
%    |\th@plain|.  Since memory space is precious in
%    ``non-Big-versions'', we have to avoid offering too many unused
%    definitions.  Therefore we define these styles in separate files
%    that can be loaded on demand.  Thus the commands themselves only
%    load these files. We use |\@input@|, a \LaTeXe{} internal command
%    that ensures that the file will be listed with |\listfiles|.
% \changes{v2.2a}{94/02/02}{Use `@input@ to load theorem layout files}
%    \begin{macrocode}
\gdef\th@plain{\@input@{thp.sty}}
\gdef\th@break{\@input@{thb.sty}}
\gdef\th@marginbreak{\@input@{thmb.sty}}
\gdef\th@changebreak{\@input@{thcb.sty}}
\gdef\th@change{\@input@{thc.sty}}
\gdef\th@margin{\@input@{thm.sty}}
%    \end{macrocode}
%    This list will be expanded when new styles become
%    available.  For testing,  just append new theorem substyles
%    as document options.
% \end{macro}
% \end{macro}
% \end{macro}
% \end{macro}
% \end{macro}
% \end{macro}
%
%
%
% \subsection{Definition of a new theorem set}
%
% As already pointed out, a new theorem environment can be defined in
% three different ways:
% \begin{verbatim}
%    \newtheorem{Lem}{Lemma}
%    \newtheorem{Lem}{Lemma}[section]
%    \newtheorem{Lem}[Theorem]{Lemma}
%\end{verbatim}
% The function of the macro |\newtheorem| is to recognize these cases
% and then to branch into one of the three macros |\@ynthm|, |\@xnthm|
% or |\@othm|.  This mechanism is adopted unchanged from
% \cite{latex:source}; the essential point here is that, for example,
% in the second case, the arguments \texttt{Lem}, \texttt{Lemma} and
% \texttt{section} are passed over to the macro |\@xnthm|.
%
% We inspect this case first because the others present fewer problems,
% and thus are easily derived from this one.
%
% \begin{macro}{\@xnthm}
%    For our example arguments, the macro |\@xnthm| must fulfill
%    the following:
%    \begin{itemize}
%       \item
%          Define a new \LaTeX-counter `Lem'
%       \item
%          reset this counter within a |\section|
%       \item
%          define the macro |\theLem|
%       \item
%          define the environment macros |\Lem| and
%          |\endLem| using the current
%          |\theoremstyle| and |\theorem@bodyfont|.
%    \end{itemize}
%    Obviously, all this should happen only if the first argument
%    of |\@xnthm| (i.e.\ \texttt{Lem} in our example) is chosen so
%    as not to conflict with any previously defined commands or
%    environments.  This test is performed by the \LaTeX{} macro
%    |\@ifdefinable|.
%    \begin{macrocode}
\gdef\@xnthm#1#2[#3]{\expandafter\@ifdefinable\csname #1\endcsname
%    \end{macrocode}
%    Therefore, the first argument of |\@ifdefinable| is the
%    expansion (in the example, |\Lem|) of
%    |\csname||#1\endcsname|.  The second argument is
%    executed only if the test has been completed successfully.
%    \begin{macrocode}
   {%
%    \end{macrocode}
%    Now we define the new counter. The names of the \LaTeX{}
%    macros employed should speak for themselves:
% \changes{v2.2b}{1995/10/07}{Use \cs{@newctr}}
% \changes{v2.2c}{1995/11/23}{Use \cs{@newctr} properly.}
%    \begin{macrocode}
    \@definecounter{#1}\@newctr{#1}[#3]%
%    \end{macrocode}
%    Using |\@newctr| will give a proper error message if the counter
%    in |#3| is not defined.
%    In defining `|\theLem|' we must generate the desired macro
%    name by use of |\expandafter| and |\csname|.
%    \begin{macrocode}
    \expandafter\xdef\csname the#1\endcsname
%    \end{macrocode}
%    An |\xdef| is used in order to make the definition global,
%    and to ensure that it contains the replacement texts of
%    |\@thmcountersep| and |\@thmcounter|.\footnote{These
%    two macros can be defined by the document style.  Their default
%    values produce a `\texttt{.}' as separation and an arabic
%    representation of the number.} However, not everything should be
%    expanded. For example, it saves space to use |\thesection|
%    instead of its---at times---lengthy expansion.
%    \begin{macrocode}
      {\expandafter \noexpand \csname the#3\endcsname
       \@thmcountersep \@thmcounter{#1}}%
%    \end{macrocode}
%    Thus with the defaults of \LaTeX{}, |\theLem| would be
%    replaced by the command sequence |\thesection.\arabic{Lem}|.
%
%    We will now look at the definition of the macro which is executed
%    at the beginning of the actual environment (in our example this
%    macro is |\Lem|). It should be noted that we use an
%    ``|\expandafter| trick'' to expand only certain parts of the
%    replacement text at the time of the definition.
%    \begin{macrocode}
    \def\@tempa{\global\@namedef{#1}}%
    \expandafter \@tempa \expandafter{%
%    \end{macrocode}
%    First, the macro that contains the current definitions of
%    |\@begintheorem| and |\@opargtheorem| should be called
%    up.  The name of this macro---as is already known---has the form
%    |\th@|\meta{theorem\ style}; therefore, it must be called by
%    \begin{macrocode}
      \csname th@\the \theorem@style
%    \end{macrocode}
%    In addition the default theorem font should be changeable, i.e.\
%    we have to insert the contents of |\theorem@bodyfont|. For
%    that reason, we expand even further, beyond |\endcsname|,
%    and thus insert the contents of the \textsf{token} register
%    |\theorem@bodyfont| in the replacement text.
%    \begin{macrocode}
            \expandafter \endcsname \the \theorem@bodyfont
%    \end{macrocode}
%    Now it is time to call the macro |\@thm| which takes over
%    the further processing.  It has two arguments: the current
%    counter name (in our example, \texttt{Lem}), and the text of the
%    label (in our example, \texttt{Lemma}).
%    \begin{macrocode}
     \@thm{#1}{#2}}%
%    \end{macrocode}
%    With this, the `sub-definition' is complete.  The macro
%    |\@endtheorem| ends a theorem environment and is, so far, nothing
%    but an |\endtrivlist|.  (Hence it is defined globally, and not
%    within the theorem styles.\footnote{This has to be changed as
%    soon as theorem styles that change \texttt{\bslash @endtheorem}
%    exist.  In such a case, all existing styles must be changed as
%    well since they will have to reset the macro.}) Therefore, we can
%    set it equivalent to the macro that ends the theorem set (in our
%    example, |\endLem|).  However, if some day theorem styles exist
%    that do change |\@endtheorem|, we would have to use the
%    commented-out line instead.
%    \begin{macrocode}
    \global \expandafter \let \csname end#1\endcsname \@endtheorem
%   \global\@namedef{end#1}{\@endtheorem}%
%    \end{macrocode}
%    With these commands all the required definitions are employed,
%    unless the test |\@ifdefinable| has failed.  Therefore, we
%    end the second argument of this macro and with it the definition
%    of |\@xnthm|.
%    \begin{macrocode}
   }}
%    \end{macrocode}
% \end{macro}
%
% \begin{macro}{\@ynthm}
%    The definition of |\@ynthm| is completely analogous. In
%    this case the new counter that is defined is not reset within
%    another counter; thus the definition of |\the...|
%    is simplified:
%    \begin{macrocode}
\gdef\@ynthm#1#2{\expandafter\@ifdefinable\csname #1\endcsname
   {\@definecounter{#1}%
    \expandafter\xdef\csname the#1\endcsname{\@thmcounter{#1}}%
%    \end{macrocode}
%    The rest of the definition corresponds literally to that of
%    |\@xnthm|:
%    \begin{macrocode}
    \def\@tempa{\global\@namedef{#1}}\expandafter \@tempa
     \expandafter{\csname th@\the \theorem@style \expandafter
     \endcsname \the\theorem@bodyfont \@thm{#1}{#2}}%
    \global \expandafter \let \csname end#1\endcsname \@endtheorem}}
%    \end{macrocode}
% \end{macro}
%
% \begin{macro}{\@othm}
%    The definition of |\@othm| does not contain anything new.
%    \begin{macrocode}
\gdef\@othm#1[#2]#3{%
%    \end{macrocode}
% \changes{v2.2b}{95/10/07}{Build in check for counter defined}
%    We do not define a new counter but instead use one that has
%    already been defined.  Thus the only definition we need is that
%    of this pseudo-counter (i.e.\ |\the|\meta{env.\ name}).
%    First we check if |#2| corresponds to a known counter name.
%    \begin{macrocode}
  \expandafter\ifx\csname c@#2\endcsname\relax
   \@nocounterr{#2}%
  \else
   \expandafter\@ifdefinable\csname #1\endcsname
   {\expandafter \xdef \csname the#1\endcsname
     {\expandafter \noexpand \csname the#2\endcsname}%
%    \end{macrocode}
%    All other parts of the definition can be adopted from
%    |\@xnthm|. We have to remember, though, that in this case
%    the name of the current counter and the theorem label have moved
%    to the second and third arguments.
%    \begin{macrocode}
    \def\@tempa{\global\@namedef{#1}}\expandafter \@tempa
     \expandafter{\csname th@\the \theorem@style \expandafter
     \endcsname \the\theorem@bodyfont \@thm{#2}{#3}}%
    \global \expandafter \let \csname end#1\endcsname \@endtheorem}%
  \fi}
%    \end{macrocode}
% \end{macro}
%
% \subsection{Macros that are employed in a {\envfont theorem}
%    environment}
%
% \begin{macro}{\@thm}
%    The macro |\@thm| has to increase the current counter.  Then,
%    depending on whether the environment has (or does not have) an
%    optional argument, it has to branch into either |\@begintheorem|
%    or |\@opargtheorem|.
%    \begin{macrocode}
\gdef\@thm#1#2{\refstepcounter{#1}%
%    \end{macrocode}
%    Now we start a {\envfont trivlist} environment, and give
%    |\@topsep| and |\@topsepadd| the values of the \textsf{skip}
%    registers |\theorempreskipamount| and |\theorempostskipamount|.
%    The value in |\@topsep| is the vertical space that is inserted by
%    the first (and only) |\item| in our |\trivlist| whilst
%    |\@topsepadd| is inserted by |\@endparenv| at the end of that
%    {\envfont trivlist} environment.  By using these registers, we
%    obtain the desired space around a {\envfont theorem} environment.
%    \begin{macrocode}
   \trivlist
   \@topsep \theorempreskipamount               % used by first \item
   \@topsepadd \theorempostskipamount           % used by \@endparenv
%    \end{macrocode}
%    Now we have to test whether an optional argument has been given.
% \changes{v2.0d}{89/07/18}{Forgotten percent added.}
%    \begin{macrocode}
   \@ifnextchar [%
%    \end{macrocode}
%    If there is an optional argument, we will call |\@ythm|, and
%    move the arguments read back into the input stream.
%    \begin{macrocode}
   {\@ythm{#1}{#2}}%
%    \end{macrocode}
%    If not, we call |\@begintheorem|. Its first argument is the
%    name of the theorem set (hence the second argument of
%    |\@thm|). Its second argument is the macro that produces the
%    current number.
%    \begin{macrocode}
   {\@begintheorem{#2}{\csname the#1\endcsname}\ignorespaces}}
%    \end{macrocode}
% \end{macro}
%
% \begin{macro}{\@xthm}
% \begin{macro}{\@ythm}
%    Both these macros were originally called by |\@thm|.  We do
%    not need |\@xthm| anymore, hence we reset it to
%    |\relax|. The definition of |\@ythm| has not changed at
%    all from its definition in \LaTeX. In order to make the macros
%    easier to understand, we will
%    nevertheless present it (commented out).
%    \begin{macrocode}
\global\let\@xthm\relax
% \def\@ythm#1#2[#3]{\@opargbegintheorem{#2}{\csname
%           the#1\endcsname}{#3}\ignorespaces}
%</package>
%    \end{macrocode}
%    The primitive |\ignorespaces| in |\@ythm| and
%    |\@thm| is needed to remove the spaces between the
%    |\begin{...}| and the actual text.
% \end{macro}
% \end{macro}
%
%
% \subsection{Definition of the theorem substyles}
%
% As already pointed out, the theorem substyles, defined below, are
% only loaded when necessary.  Note that all these substyles, except
% \texttt{plain}, have |\slshape| as the default body font.
%
% \subsubsection{The \texttt{plain} style}
%
% \label{dok:th@plain} As the following macros use |@|, we have
% to locally set the |\catcode| of this symbol to ``letter''.
% This happens within a group, so that we do not have to worry about
% which |\catcode| that symbol had before.
%    \begin{macrocode}
%<*thp>
\begingroup \makeatletter
%    \end{macrocode}
% Since we are now within a group, we must make all definitions
% globally.  First we make sure that \texttt{theorem.sty} is loaded.
% This will allow us to use this file as a document style option
% without having to call \texttt{theorem} itself as an option.  At the
% same time, we assure that at least version 2 is loaded, since
% |\theorem@style| was not defined in earlier versions.
%    \begin{macrocode}
\@ifundefined{theorem@style}{\input{theorem.sty}}{}
\ProvidesFile{thp.sty}
             [\FMithmInfo]
%    \end{macrocode}
%
%
% \begin{macro}{\th@plain}
% \begin{macro}{\@begintheorem}
% \begin{macro}{\@opargbegintheorem}
%    |\theoremstyle{plain}| corresponds to the original
%    definition, except that the distances to the surrounding text are
%    determined by the parameters |\theorempreskipamount| and
%    |\theorempostskipamount|.
%    First we set the default body font.
%    \begin{macrocode}
\gdef\th@plain{\normalfont\itshape
%    \end{macrocode}
%    Then we define |\@begintheorem| and
%    |\@opargbegintheorem|. These two macros define how the
%    header of a theorem is typeset.  |\@opargbegintheorem| will
%    be called if a \textsf{theorem} environment with an optional
%    argument is encountered; otherwise, the header is constructed by
%    calling |\@begintheorem|.  If one of these macros is
%    executed, we are within a \textsf{trivlist} environment started by
%    |\@thm|.  So the theorem header is produced with an
%    |\item| command.
%
%    Instead of specifying the header font directly, all standard
%    theorem styles use the |\theorem@headerfont| macro to allow
%    customization.  The extra space (|\labelsep|) is necessary
%    because of problems in the \textsf{trivlist} environment.
%    \begin{macrocode}
  \def\@begintheorem##1##2{%
        \item[\hskip\labelsep \theorem@headerfont ##1\ ##2]}%
%    \end{macrocode}
%    The definition of |\@opargbegintheorem| is completely
%    analogous. The only difference is the fact that there exists a
%    third argument (which is the optional parameter of the
%    environment and contains additional information about the
%    theorem).  Customarily we enclose it in parentheses.
%    \begin{macrocode}
\def\@opargbegintheorem##1##2##3{%
   \item[\hskip\labelsep \theorem@headerfont ##1\ ##2\ (##3)]}}
%    \end{macrocode}
% \end{macro}
% \end{macro}
% \end{macro}
%
% \noindent
% We conclude with an |\endgroup| to restore the
% |\catcode| of |@|.
%    \begin{macrocode}
\endgroup
%</thp>
%    \end{macrocode}
%
%
% \subsubsection{The \texttt{break} style}
%
% This style option is stored in the file |thb.sty|.
% For the next two lines see the documentation for |\th@plain| on
% page~\pageref{dok:th@plain}.
%    \begin{macrocode}
%<*thb>
\begingroup \makeatletter
\@ifundefined{theorem@style}{\input{theorem.sty}}{}
\ProvidesFile{thb.sty}
             [\FMithmInfo]
%    \end{macrocode}
%
% \begin{macro}{\th@break}
%    |\theoremstyle{break}| produces a line break after the name
%    of the theorem.  The font is |\slshape|.  Hence, we define
%    |\th@break| as follows:
%    \begin{macrocode}
\gdef\th@break{\normalfont\slshape
  \def\@begintheorem##1##2{\item[%
%    \end{macrocode}
%    We run into the following problem: it is not possible to create
%    the header with |\item[|\meta{title}|]| and then start
%    a new line by, for example, |\mbox{}\\|. Such a definition
%    will fail whenever a list environment follows immediately.  With
%    the above construction, the |\mbox{}| causes the switch
%    |@inlabel| (cf.\ definition of |\list| and
%    |\trivlist| in \cite{latex:source}) to be set to \texttt{false}
%    and so the following list will insert additional vertical
%    space (|\topskip|). This is quite annoying. Therefore, we
%    create the line break within the |\item|. In order to ensure
%    that the text will begin at the proper position in the following
%    line, we simply pretend that the label does not take any
%    room.\footnote{This will lead to problems whenever very high
%    symbols occurring in the line tower into the
%    heading. So, something else has to be done here sometime.}
%    \begin{macrocode}
       \rlap{\vbox{\hbox{\hskip \labelsep\theorem@headerfont ##1\ ##2}%
                   \hbox{\strut}}}]}%
%    \end{macrocode}
%    Again, the definition of |\@opargbegintheorem| is
%    completely analogous.
%    \begin{macrocode}
\def\@opargbegintheorem##1##2##3{%
  \item[\rlap{\vbox{\hbox{\hskip \labelsep \theorem@headerfont
                     ##1\ ##2\ (##3)}%
                    \hbox{\strut}}}]}}
\endgroup
%</thb>
%    \end{macrocode}
% \end{macro}
%
%
% \subsubsection{The \texttt{changebreak} style}
%
%    \begin{macrocode}
% This style option is stored in the file |thcb.sty|.
%    \begin{macrocode}
%<*thcb>
\begingroup \makeatletter
\@ifundefined{theorem@style}{\input{theorem.sty}}{}
\ProvidesFile{thcb.sty}
             [\FMithmInfo]
%    \end{macrocode}
%
% \begin{macro}{\th@changebreak}
%    The \texttt{change-break} theorem style is like \texttt{break}
%    but with interchange of theorem name and theorem number.  Thus we
%    define |\th@changebreak| as follows:
%    \begin{macrocode}
\gdef\th@changebreak{\normalfont\slshape
  \def\@begintheorem##1##2{\item
       [\rlap{\vbox{\hbox{\hskip\labelsep \theorem@headerfont ##2\ ##1}%
                    \hbox{\strut}}}]}%
\def\@opargbegintheorem##1##2##3{%
  \item[\rlap{\vbox{\hbox{\hskip\labelsep \theorem@headerfont
                     ##2\ ##1\ (##3)}%
                    \hbox{\strut}}}]}}
\endgroup
%</thcb>
%    \end{macrocode}
% \end{macro}
%
%
% \subsubsection{The \texttt{change} style}
%
% This style option is stored in the file |thc.sty|.
%    \begin{macrocode}
%<*thc>
\begingroup \makeatletter
\@ifundefined{theorem@style}{\input{theorem.sty}}{}
\ProvidesFile{thc.sty}
             [\FMithmInfo]
%    \end{macrocode}
%
% \begin{macro}{\th@change}
%    The \texttt{change} theorem style corresponds to the
%    \texttt{change break} style without a linebreak after the header.
%    To say it in another way, it's the same as the \texttt{plain}
%    style but with number and name interchanged and |\slshape| as the
%    default font.
%    \begin{macrocode}
\gdef\th@change{\normalfont\slshape
  \def\@begintheorem##1##2{\item
       [\hskip\labelsep \theorem@headerfont ##2\ ##1]}%
\def\@opargbegintheorem##1##2##3{%
  \item[\hskip\labelsep \theorem@headerfont ##2\ ##1\ (##3)]}}
\endgroup
%</thc>
%    \end{macrocode}
% \end{macro}
%
%
% \subsubsection{The \texttt{marginbreak} style}
%
% This style option is the one used most often at Mainz.  It is saved
% in the file |thmb.sty|.
%    \begin{macrocode}
%<*thmb>
\begingroup \makeatletter
\@ifundefined{theorem@style}{\input{theorem.sty}}{}
\ProvidesFile{thmb.sty}
             [\FMithmInfo]
%    \end{macrocode}
%
% \begin{macro}{\th@marginbreak}
%    The \texttt{margin break} style is nearly the same as the
%    \texttt{change break} style. The only difference is the placement
%    of the theorem number.  We use |\llap| to place it in the left
%    margin.
%
%    In this style |\labelsep| denotes the separation between the
%    number and the text.
%    \begin{macrocode}
\gdef\th@marginbreak{\normalfont\slshape
  \def\@begintheorem##1##2{\item
       [\rlap{\vbox{\theorem@headerfont
                    \hbox{\llap{##2}\hskip\labelsep ##1}%
                    \hbox{\strut}}}]}%
\def\@opargbegintheorem##1##2##3{%
  \item[\rlap{\vbox{\theorem@headerfont
                    \hbox{\llap{##2}\hskip\labelsep ##1\ (##3)}%
                    \hbox{\strut}}}]}}
\endgroup
%</thmb>
%    \end{macrocode}
% \changes{v2.1c}{90/02/17}{`endgroup outside macrocode env.}
% \end{macro}
%
%
%
% \subsubsection{The \texttt{margin} style}
%
% This style option is stored in the file |thm.sty|.
%    \begin{macrocode}
%<*thm>
\begingroup \makeatletter
\@ifundefined{theorem@style}{\input{theorem.sty}}{}
\ProvidesFile{thm.sty}
             [\FMithmInfo]
%    \end{macrocode}
%
% \begin{macro}{\th@margin}
%    Again this is only a variant of the theorem styles described
%    above without any new ideas.
%    \begin{macrocode}
\gdef\th@margin{\normalfont\slshape
  \def\@begintheorem##1##2{\item
       [\theorem@headerfont \llap{##2}\hskip\labelsep ##1]}%
\def\@opargbegintheorem##1##2##3{%
  \item[\theorem@headerfont \llap{##2}\hskip\labelsep ##1\ (##3)]}}
\endgroup
%</thm>
%    \end{macrocode}
% \end{macro}
%
%
%
%
% \subsection{Final Definitions}
%
% \begin{macro}{\theorempreskipamount}
% \begin{macro}{\theorempostskipamount}
%    The \textsf{skip} parameters that regulate the vertical empty space
%    before and after the {\envfont theorem} environment have
%    to be allocated as well.
% \changes{v2.0f}{89/07/20}{Unnecessary `global' removed.}
%    \begin{macrocode}
%<*package>
\newskip\theorempreskipamount
\newskip\theorempostskipamount
%    \end{macrocode}
%    Since we have used the same values for all theorem sets, we now can
%    assign them.
%    \begin{macrocode}
\global\setlength\theorempreskipamount{12pt plus 5pt minus 3pt}
\global\setlength\theorempostskipamount{8pt plus 3pt minus 1.5pt}
%    \end{macrocode}
% \end{macro}
% \end{macro}
%
%
% \begin{macro}{\@endtheorem}
%    The same holds for the macro |\@endtheorem|, which ends a
%    {\envfont theorem} environment. Since it is the same for all
%    theorem sets, it is removed from the macros |\th@|\meta{style}.
%    It simply ends the {\envfont trivlist} environment, which was
%    begun in |\@thm|.
%    \begin{macrocode}
\global\let\@endtheorem=\endtrivlist
%    \end{macrocode}
% \end{macro}
%
%
%
% \begin{macro}{\@preamblecmds}
%    All macros defined above are to be used only in the preamble.
%    Therefore, we insert them in |\@preamblecmds| which will disable
%    them at begin document. This is done by the internal \LaTeXe{}
%    command |\@onlypreamble|.
% \changes{v2.0g}{89/09/19}{`theoremheaderfont and `theorembodyfont
%    added.}
% \changes{v2.2a}{94/02/03}{Use \LaTeXe{} interface}
%    \begin{macrocode}
\@onlypreamble\@xnthm
\@onlypreamble\@ynthm
\@onlypreamble\@othm
\@onlypreamble\newtheorem
\@onlypreamble\theoremstyle
\@onlypreamble\theorembodyfont
\@onlypreamble\theoremheaderfont
%    \end{macrocode}
% \end{macro}
%
% Finally we declare the \texttt{plain} theorem style to be the default.
% \changes{v2.1b}{90/06/04}{Fixed missing `item by declaring default
%    style.}
%    \begin{macrocode}
\theoremstyle{plain}
%</package>
%    \end{macrocode}
%
% \Finale

\endinput
}{}
\ProvidesFile{thmb.sty}
             [\FMithmInfo]
%    \end{macrocode}
%
% \begin{macro}{\th@marginbreak}
%    The \texttt{margin break} style is nearly the same as the
%    \texttt{change break} style. The only difference is the placement
%    of the theorem number.  We use |\llap| to place it in the left
%    margin.
%
%    In this style |\labelsep| denotes the separation between the
%    number and the text.
%    \begin{macrocode}
\gdef\th@marginbreak{\normalfont\slshape
  \def\@begintheorem##1##2{\item
       [\rlap{\vbox{\theorem@headerfont
                    \hbox{\llap{##2}\hskip\labelsep ##1}%
                    \hbox{\strut}}}]}%
\def\@opargbegintheorem##1##2##3{%
  \item[\rlap{\vbox{\theorem@headerfont
                    \hbox{\llap{##2}\hskip\labelsep ##1\ (##3)}%
                    \hbox{\strut}}}]}}
\endgroup
%</thmb>
%    \end{macrocode}
% \changes{v2.1c}{90/02/17}{`endgroup outside macrocode env.}
% \end{macro}
%
%
%
% \subsubsection{The \texttt{margin} style}
%
% This style option is stored in the file |thm.sty|.
%    \begin{macrocode}
%<*thm>
\begingroup \makeatletter
\@ifundefined{theorem@style}{% \iffalse meta-comment
%
% Copyright (C) 1993-2024
%
% The LaTeX Project and any individual authors listed elsewhere
% in this file.
%
% This file is part of the Standard LaTeX `Tools Bundle'.
% -------------------------------------------------------
%
% It may be distributed and/or modified under the
% conditions of the LaTeX Project Public License, either version 1.3c
% of this license or (at your option) any later version.
% The latest version of this license is in
%    https://www.latex-project.org/lppl.txt
% and version 1.3c or later is part of all distributions of LaTeX
% version 2005/12/01 or later.
%
% The list of all files belonging to the LaTeX `Tools Bundle' is
% given in the file `manifest.txt'.
%
% \fi
%
%%
%
%
\def\FMithmInfo{2023/07/05 v2.2c Theorem extension package (FMi)}
%
% \ProvidesFile{theorem.dtx}[\FMithmInfo]
%\iffalse   % this is a METACOMMENT !
%%
%% Package `theorem' to use with LaTeX2e
%% Copyright (C) 1989-1995 Frank Mittelbach, all rights reserved.
%<+package>\@ifundefined{theorem@style}{}{\endinput}
%<+package>\NeedsTeXFormat{LaTeX2e}[1995/06/01]
%<+package>\ProvidesPackage{theorem}
%<+package>         [\FMithmInfo]
%
% \fi
%
% \changes{v1.0}{87/05/07}{First draft.}
% \changes{v1.0a}{87/05/07}{`thistheoremstyle' added.}
% \changes{v1.1a}{87/07/29}{New concept: redefine the internal
%                          \protect\LaTeX{} commands `@xnthm' etc.}
% \changes{v1.1b}{87/07/31}{`theoremfont' used to specify font.}
% \changes{v1.2a}{88/01/08}{`@endtheorem' now globally defined.}
% \changes{v1.2b}{88/01/11}{`newtheorem' only allowed in the preamble.}
% \changes{v1.3a}{88/02/14}{`theoremfont' now a toks register.}
%
% \changes{v2.0a}{88/11/16}{New implementation and new documentation.}
% \changes{v2.0a}{88/11/16}{No layout changes but internal modifications
%                           old substyles won't work.}
% \changes{v2.0b}{89/01/30}{`global' used for all definitions.}
% \changes{v2.0c}{89/06/15}{Documentation translated to English.
%                            Thanks to LLL and Nina}
% \changes{v2.1a}{90/02/15}{Support for new font selection.}
% \changes{v2.2a}{94/02/03}{Cleanup for LaTeX2e}
%
% \DoNotIndex{\@endparenv,\@ifdefinable,\@ifnextchar,\@ifundefined}
% \DoNotIndex{\@namedef,\@nameuse,\@spaces,\@tempa,\@warning}
% \DoNotIndex{\ }
% \DoNotIndex{\begingroup,\bfseries}
% \DoNotIndex{\csname}
% \DoNotIndex{\def,\do,\docdate}
% \DoNotIndex{\endcsname,\endgroup,\endinput,\endtrivlist,\expandafter}
% \DoNotIndex{\filedate,\fileversion}
% \DoNotIndex{\gdef,\global}
% \DoNotIndex{\hbox,\hskip}
% \DoNotIndex{\ignorespaces,\input,\itshape,\item}
% \DoNotIndex{\labelsep,\let,\llap}
% \DoNotIndex{\makeatletter}
% \DoNotIndex{\noexpand}
% \DoNotIndex{\refstepcounter,\relax,\rlap}
% \DoNotIndex{\setlength,\slshape,\space,\string,\strut}
% \DoNotIndex{\the,\trivlist,\typeout}
% \DoNotIndex{\vbox}
% \DoNotIndex{\xdef}
%
% \def\deflabel#1{\ttfamily #1\hfill}
% \def\deflist#1{\begin{list}{}{\settowidth\labelwidth{\ttfamily #1}%
%                               \setlength\leftmargin\labelwidth
%                               \addtolength\leftmargin\labelsep
%                               \let\makelabel\deflabel}}
% \def\enddeflist{\end{list}}
% \def\envfont{\normalfont\ttfamily}
%
% \MakeShortVerb{\|}
%
% \newcommand{\TUB}{{\sl TUGboat\/}}
%
% \GetFileInfo{theorem.dtx}
% \title{An Extension of the
%        \LaTeX{} theorem environment\thanks{This file
%        has version number \fileversion, last
%        revised \filedate.}}
% \author{Frank Mittelbach}
%
% \MaintainedByLaTeXTeam{tools}
% \maketitle
%
% \begin{abstract}
%    The macros described in this paper yield an extension of the
%    \LaTeX{} theorem mechanism. It is designed to satisfy the
%    different requirements of various journals. Thus, the
%    layout of the ``theorems'' can be manipulated by determining a
%    ``style''. This article describes not only the use, but
%    also the definition, of the necessary macros.
% \end{abstract}
%
% \section*{Preface to version 2.2}
%
% For \LaTeXe{} this package did not need any fundamental changes. I
% only modified the messages generated so that theorem layout styles
% will show up with the |\listfiles| command and cleaned the section
% on the New Font Selection Scheme since this is now included in
% \LaTeX.
%
% \section*{Preface to version 2.1}
%
% This version is identical to 2.0g described in \TUB\ 10\#3 except
% for some internal defaults which are now set depending on the used
% font selection scheme.
%
% This was done to avoid unpleasant surprises if the new font selection
% scheme is in force. For further details see
% section~\ref{sec:newfonts} and \cite{bk:GMS94}.
%
% \section{Introduction}
%
% For our purposes here, ``theorems'' are labelled enunciations,
% often set off from the main text by extra space and a font change.
% Theorems, corollaries, conjectures, definitions, and remarks
% are all instances of ``theorems''.  The ``header'' of these structures
% is composed of a label (such as \textsc{Theorem} or \textsc{Remark})
% and a number which serializes an item in the sequence of items
% with the same label.
%
% Shortly after the introduction of \LaTeX{} at the Fachbereich
% Mathematik in Mainz, the desire to manipulate the layout of
% ``theorems'' arose.  In Mainz, the following two conventions came
% into general use:
% \begin{enumerate}
%   \item The number of the theorem is shown in the margin.
%   \item There is a line break at
%         the end of the theorem header.
% \end{enumerate}
%
% Additionally, some journals require different formats which depend on
% the ``sort of theorem'': e.g.\ often remarks and definitions are set
% in |\upshape|, while |\itshape| is employed for main theorems.
%
% Confronted with these requirements, a theorem environment was
% developed in Mainz which allows separate determination of the
% layout of the ``theorems sets'', comparable to |\pagestyle|.
%
%
% \section{The user interface}
% \subsection{Defining new theorem sets}
%
% \DescribeMacro\newtheorem
% As in the original \LaTeX{} version, the command |\newtheorem|
% defines a new ``theorem set'' or ``theorem-like structure''.
% Two required arguments name the new environment and give the text
% to be typeset with each instance of the new ``set'', while
% an optional argument determines how
% the ``set'' is enumerated:
% \begin{description}
%    \item[\ttfamily \bslash newtheorem\{foo\}\{bar\}]
%       The theorem set {\envfont foo} (whose name is \texttt{bar})
%       uses its own counter.
%    \item[\ttfamily \bslash newtheorem\{foo2\}{[foo]}\{bar2\}]
%       The theorem set {\envfont foo2} (printed name \texttt{bar2})
%       uses the same counter as the theorem set \texttt{foo}.
%    \item[\ttfamily \bslash newtheorem\{foo3\}\{bar3\}{[section]}]
%       The theorem set {\envfont foo3} (printed name \texttt{bar3}) is
%       enumerated within the counter \texttt{section}, i.e.\ with every
%       new |\section| the enumeration begins again with 1, and
%       the enumeration is composed from the section-number and the
%       theorem counter itself.
% \end{description}
%
% \DescribeMacro\theoremstyle
% Additionally, the command |\theoremstyle| can define the layout
% of various, or all, theorem sets.  It should be noted that any
% theorem set defined by |\newtheorem| is typeset in the
% |\theoremstyle| that is current at the time of the definition.
% Thus, the following
% \begin{verbatim}
%    \theoremstyle{break}        \newtheorem{Cor}{Corollary}
%    \theoremstyle{plain}        \newtheorem{Exa}{Example}[section]
%\end{verbatim}                             ^^A No blank at line start
% leads to the result that the set {\envfont Cor} is formatted in the
% style \texttt{break}, while the set {\envfont Exa} and all the
% following ones are formatted in the style \texttt{plain}, unless
% another |\theoremstyle| follows.  Since the definitions installed by
% |\newtheorem| are global, one also can limit |\theoremstyle| locally
% by grouping braces.
%
% \DescribeMacro\theorembodyfont
% The choice of the font for the theorem body is completely
% independent of the chosen |\theoremstyle|; this has proven to
% be very advantageous.  For example,
% \begin{verbatim}
%   {\theorembodyfont{\upshape}          \newtheorem{Rem}{Remark}}
%\end{verbatim}
% defines a theorem set {\envfont Rem}, which will be set in
% |\upshape| in the current layout (which in our example is
% \texttt{plain}).  As with |\theoremstyle|, the |\theorembodyfont|
% chosen is that current at the time of |\newtheorem|.  If
% |\theorembodyfont| is not specified or one defines
% |\theorembodyfont{}|, then the font used will be that defined by the
% |\theoremstyle|.
%
% \DescribeMacro\theoremheaderfont It is also possible to customize
% the font used for the theorem headers.  This is, however, a global
% declaration, and therefore there should be at most one
% |\theoremheaderfont| declaration in the preamble.\footnote{If it is
% actually necessary to have different header fonts, one has to define
% new theorem styles (substituting the desired font) or specify the
% information directly in the \texttt{\bslash newtheorem} declaration
% (the unclean variant).}
%
% \DescribeMacro\theorempreskipamount
% \DescribeMacro\theorempostskipamount Two additional parameters
% affect the vertical space around the theorem environments:
% |\theorempreskipamount| and |\theorempostskipamount| define,
% respectively, the spacing before and after such an environment.
% These parameters apply for all theorem sets and can be manipulated
% with the ordinary length macros.  They are rubber lengths
% (`\textsf{skips}'), and therefore can contain \texttt{plus} and
% \texttt{minus} parts.
%
% Since the definition of theorem sets should---most sensibly---be
% placed in the preamble, we only allow installation there.
% It is therefore possible to release the memory used here after
% |\begin{document}|, in order to make room for other
% applications.
%
%
% \subsection{Existing theorem styles}
%
% The following theorem styles exist to date:
% \begin{deflist}{marginbreak}
%    \item[plain]
%       This theorem style emulates the original \LaTeX{} definition,
%       except that additionally the parameters
%       |\theorem...skipamount| are used.
%    \item[break]
%       In this style, the theorem header is followed by a line
%       break.
%    \item[marginbreak]
%       The theorem number is set in the margin, and there is a line
%       break as in \texttt{break}.
%    \item[changebreak]
%       Like \texttt{break}, but with header number and text
%       interchanged.
%    \item[change]
%       Header number and text are interchanged, without a line break.
%    \item[margin]
%       The number is set in the left margin, without a line break.
% \end{deflist}
% All styles (except \texttt{plain}) select |\slshape| as the default
% |\theorembodyfont|.
%
%
% \subsection{Examples}
%
% Given the above theorem sets {\envfont Cor}, {\envfont Exa}
% and {\envfont Rem},
% suppose that the preamble also contains the declarations:
% \begin{verbatim}
%    \theoremstyle{marginbreak}   \newtheorem{Lem}[Cor]{Lemma}
%    \theoremstyle{change}
%    \theorembodyfont{\itshape}        \newtheorem{Def}[Cor]{Definition}
%
%    \theoremheaderfont{\scshape}
%\end{verbatim}
% Then the following are some typical examples of the typeset output
% resulting from their use.
%
% \begin{Cor}
%    This is a sentence typeset in the theorem environment {\envfont
%    Cor}.
% \end{Cor}
%
% \begin{Exa}
%    This is a sentence typeset in the theorem environment {\envfont
%    Exa}.
% \end{Exa}
%
% \begin{Rem}
%    This is a sentence typeset in the theorem environment {\envfont
%    Rem}.
% \end{Rem}
%
% \begin{Lem}[Ben User]
%    This is a sentence typeset in the theorem environment {\envfont
%    Lem}.
% \end{Lem}
%
% \begin{Def}[Very impressive Definition]
%    This is a sentence typeset in the theorem environment {\envfont
%    Def}.
% \end{Def}
%
% The last two examples show the effect of the optional argument to
% a theorem environment (it is the text typeset in parentheses).
%
%
% \section{Special Considerations}
% \label{sec:newfonts}
%
% Theoremheader and body are implemented as a unit. This means that
% the |\theoremheaderfont|  will inherit characteristics of the
% |\theorembodyfont| in \LaTeXe. Thus, if for example
% |\theorembodyfont| is |\itshape| and |\theoremheaderfont|
% is |\bfseries| the font selected for the header will have the
% characteristics `bold extended italic'. If this is not desired one
% should set the |\theoremheaderfont| to something like
% \begin{verbatim}
%      \theoremheaderfont{\normalfont\bfseries}
%\end{verbatim}
% i.e.\ supplying all necessary font information explicitly.
%
%
%
% \section{Acknowledgements}
%
% The publication of this set of macros was only possible with the
% help of Christina Busse (translating the manuscript into English),
% Joachim Pense (playing the r\^ole of typist), Chris Rowley
% (looking everything over) and many others providing useful
% suggestions.
%
% \MaybeStop{%
%    \begin{thebibliography}{9}
%       \bibitem{bk:GMS94} \textsc{M.~Goossens}, \textsc{F.~Mittelbach}
%         and \textsc{A.~Samarin}.
%         \newblock The \LaTeX{} Companion.
%         \newblock
%         Addison-Wesley, Reading, Massachusetts, 1994.
%       \bibitem{latex:source} \textsc{Lamport, Leslie.} \newblock
%                               \texttt{latex.tex}, version 2.09, date
%                               Feb. 1990.
%    \end{thebibliography}
%    \PrintIndex
%    \PrintChanges}
%
%
% \section{The documentation driver file}
%
% The next bit of code contains the documentation driver file for
% \TeX{}, i.e., the file that will produce the documentation you are
% currently reading. It will be extracted from this file by the
% \texttt{docstrip} program. Since it is the first code in the file
% one can alternatively process this file directly with \LaTeXe{} to
% obtain the documentation.
%
%    \begin{macrocode}
%<*driver>
\documentclass{ltxdoc}

\usepackage{theorem}

%<+driver>% The next few lines define theorem sets which are used
%<+driver>% in the example section of the documentation.

\theoremstyle{break}       \newtheorem{Cor}{Corollary}
\theoremstyle{plain}       \newtheorem{Exa}{Example}[section]
{\theorembodyfont{\upshape}\newtheorem{Rem}{Remark}}
\theoremstyle{marginbreak} \newtheorem{Lem}[Cor]{Lemma}
\theoremstyle{change}
\theorembodyfont{\itshape} \newtheorem{Def}[Cor]{Definition}

\theoremheaderfont{\scshape}

\RecordChanges

\begin{document}
   \DocInput{theorem.dtx}
\end{document}
%</driver>
%    \end{macrocode}
%
% \section{Definition of the Macros}
%
% If the file has been loaded before, we abort immediately. If not the
% package announces itself (this is actually done at the very top of
% the file---the way it is done isn't good style so don't copy it).
% \changes{v2.0e}{89/07/19}{Spaces removed from `typeout'.}
% \changes{v2.2b}{95/11/19}{Announce moved to top of file}
%    \begin{macrocode}
%<*package>
%\@ifundefined{theorem@style}{}{\endinput}
%\def\FMithmInfo{1995/11/19 v2.2b Theorem extension package (FMi)}
%\ProvidesPackage{theorem}[\FMithmInfo]
%    \end{macrocode}
%
%
% \subsection{Definition of theorem styles and fonts}
%
% All the definitions in this file are done globally to allow
% inputting this file inside a group.
%
% \begin{macro}{\theoremstyle}
%    Before a theorem style can be installed, the chosen style must be
%    known. For that reason, we must test to see that
%    |\th@|\meta{style} is known or, more precisely, that it is
%    different from |\relax|. If the style is not known then
%    |\th@plain| is used.
%    \begin{macrocode}
\gdef\theoremstyle#1{%
   \@ifundefined{th@#1}{\@warning
          {Unknown theoremstyle `#1'. Using `plain'}%
          \theorem@style{plain}}%
%    \end{macrocode}
%    We save the theorem style to be used in the token register
%    |\theorem@style|.
%    \begin{macrocode}
      {\theorem@style{#1}}%
%    \end{macrocode}
%    Now we ``evaluate'' the theorem style: this means, we call the
%    macro |\th@|\meta{style} which will activate the relevant
%    definitions which are contained in a separate file.  This is done
%    in a group to suppress changes to the current font.  This could
%    otherwise pose problems together with the new font selection
%    scheme\footnote{When I printed the original article using the new
%    font selection scheme I ended with a document with slanted
%    typefaces (text headings and all) simply because one of the
%    theorem styles used \texttt{\bslash sl} at toplevel.} if the
%    |\th@|\meta{style} is evaluated a second time.
% \changes{v2.1a}{90/02/15}{Group added around `th@..}
% \changes{v2.1a}{90/02/15}{Reseting of `..@begintheorem removed.}
%    \begin{macrocode}
      \begingroup
        \csname th@\the\theorem@style \endcsname
      \endgroup}
%    \end{macrocode}
% \end{macro}
%
%
% \begin{macro}{\@begintheorem}
% \changes{v2.1a}{90/02/15}{Relaxed at toplevel.}
% \begin{macro}{\@opargbegintheorem}
% \changes{v2.1a}{90/02/15}{Relaxed at toplevel.}
%    We reset |\@begintheorem| and |\@opargbegintheorem| to
%    |\relax| since these commands are no longer necessary at
%    toplevel. This will save a few tokens.
%    \begin{macrocode}
\global\let\@begintheorem\relax
\global\let\@opargbegintheorem\relax
%    \end{macrocode}
% \end{macro}
% \end{macro}
%
% \begin{macro}{\theorem@style}
%    Obviously the \textsf{token} register used above has to be
%    allocated. To assure the utmost compatibility with the
%    original \LaTeX{} definition, we set the default theorem style to
%    \texttt{plain}, which implements the usual \LaTeX{} convention.
% \changes{v2.0f}{89/07/20}{Unnecessary `global' removed.}
%    \begin{macrocode}
\newtoks\theorem@style
\global\theorem@style{plain}
%    \end{macrocode}
% \end{macro}
%
% \begin{macro}{\theorembodyfont}
% \begin{macro}{\theorem@bodyfont}
%    If the |\theorembodyfont| is set by the user then it should not
%    interact with the default font set in the theorem style. When the
%    new font selection is in force this may happen if, for example,
%    the default is |\itshape| and the new |\theorembodyfont| is
%    |\sffamily|. So we add a |\reset@font| command in front of the
%    user definition.
% \changes{v2.0f}{89/07/20}{Unnecessary `global' removed.}
% \changes{v2.0g}{89/09/19}{`theoremfont to `theorembodyfont.}
% \changes{v2.1c}{92/06/22}{`reset@font added and logic changed.}
% \changes{v2.1d}{92/10/28}{Reset to empty if arg is empty}
%    \begin{macrocode}
\gdef\theorembodyfont#1{%
%    \end{macrocode}
%    We check if the argument supplied is empty and if so put nothing
%    into the |\theorem@bodyfont| token register to allow for
%    |\theorembodyfont{}| as a mean of using the default of the
%    current |\theoremstyle|.
%    \begin{macrocode}
   \def\@tempa{#1}%
   \ifx\@tempa\@empty
    \theorem@bodyfont{}%
   \else
    \theorem@bodyfont{\reset@font#1}%
   \fi
   }
\newtoks\theorem@bodyfont
\global\theorem@bodyfont{}
%    \end{macrocode}
% \end{macro}
% \end{macro}
%
%
% \begin{macro}{\theoremheaderfont}
% \changes{v2.0g}{89/09/19}{Macro added.}
%    The font for the theorem headers is handled differently because
%    this definition applies to all theorem styles.
%    \begin{macrocode}
\gdef\theoremheaderfont#1{\gdef\theorem@headerfont{#1}%
%    \end{macrocode}
%    After using the macro once it is redefined to produce an error
%    message.
%    \begin{macrocode}
       \gdef\theoremheaderfont##1{%
        \typeout{\string\theoremheaderfont\space should be used
                 only once.}}}
%    \end{macrocode}
% \end{macro}
%
% \begin{macro}{\theorem@headerfont}
% \changes{v2.0g}{89/09/19}{Macro added.}
%    To set the |\theorem@headerfont| default we first test if
%    the new fontselection scheme is in force.
% \changes{v2.1a}{90/02/15}{Support for new font selection.}
%    \begin{macrocode}
\ifx\upshape\undefined
%    \end{macrocode}
%    If not we define it to expand into |\bfseries|. We don't use
%    |\let| just in case a following style option redefines this
%    macro.
%    \begin{macrocode}
\gdef\theorem@headerfont{\bfseries}
%    \end{macrocode}
%    Otherwise we reset the current shape before calling |\bfseries|.
%    \begin{macrocode}
\else \gdef\theorem@headerfont{\normalfont\bfseries}\fi
%    \end{macrocode}
% \end{macro}
%
% \begin{macro}{\th@plain}
% \begin{macro}{\th@break}
% \begin{macro}{\th@marginbreak}
% \begin{macro}{\th@changebreak}
% \begin{macro}{\th@change}
% \begin{macro}{\th@margin}
%    The different styles are defined in macros such as
%    |\th@plain|.  Since memory space is precious in
%    ``non-Big-versions'', we have to avoid offering too many unused
%    definitions.  Therefore we define these styles in separate files
%    that can be loaded on demand.  Thus the commands themselves only
%    load these files. We use |\@input@|, a \LaTeXe{} internal command
%    that ensures that the file will be listed with |\listfiles|.
% \changes{v2.2a}{94/02/02}{Use `@input@ to load theorem layout files}
%    \begin{macrocode}
\gdef\th@plain{\@input@{thp.sty}}
\gdef\th@break{\@input@{thb.sty}}
\gdef\th@marginbreak{\@input@{thmb.sty}}
\gdef\th@changebreak{\@input@{thcb.sty}}
\gdef\th@change{\@input@{thc.sty}}
\gdef\th@margin{\@input@{thm.sty}}
%    \end{macrocode}
%    This list will be expanded when new styles become
%    available.  For testing,  just append new theorem substyles
%    as document options.
% \end{macro}
% \end{macro}
% \end{macro}
% \end{macro}
% \end{macro}
% \end{macro}
%
%
%
% \subsection{Definition of a new theorem set}
%
% As already pointed out, a new theorem environment can be defined in
% three different ways:
% \begin{verbatim}
%    \newtheorem{Lem}{Lemma}
%    \newtheorem{Lem}{Lemma}[section]
%    \newtheorem{Lem}[Theorem]{Lemma}
%\end{verbatim}
% The function of the macro |\newtheorem| is to recognize these cases
% and then to branch into one of the three macros |\@ynthm|, |\@xnthm|
% or |\@othm|.  This mechanism is adopted unchanged from
% \cite{latex:source}; the essential point here is that, for example,
% in the second case, the arguments \texttt{Lem}, \texttt{Lemma} and
% \texttt{section} are passed over to the macro |\@xnthm|.
%
% We inspect this case first because the others present fewer problems,
% and thus are easily derived from this one.
%
% \begin{macro}{\@xnthm}
%    For our example arguments, the macro |\@xnthm| must fulfill
%    the following:
%    \begin{itemize}
%       \item
%          Define a new \LaTeX-counter `Lem'
%       \item
%          reset this counter within a |\section|
%       \item
%          define the macro |\theLem|
%       \item
%          define the environment macros |\Lem| and
%          |\endLem| using the current
%          |\theoremstyle| and |\theorem@bodyfont|.
%    \end{itemize}
%    Obviously, all this should happen only if the first argument
%    of |\@xnthm| (i.e.\ \texttt{Lem} in our example) is chosen so
%    as not to conflict with any previously defined commands or
%    environments.  This test is performed by the \LaTeX{} macro
%    |\@ifdefinable|.
%    \begin{macrocode}
\gdef\@xnthm#1#2[#3]{\expandafter\@ifdefinable\csname #1\endcsname
%    \end{macrocode}
%    Therefore, the first argument of |\@ifdefinable| is the
%    expansion (in the example, |\Lem|) of
%    |\csname||#1\endcsname|.  The second argument is
%    executed only if the test has been completed successfully.
%    \begin{macrocode}
   {%
%    \end{macrocode}
%    Now we define the new counter. The names of the \LaTeX{}
%    macros employed should speak for themselves:
% \changes{v2.2b}{1995/10/07}{Use \cs{@newctr}}
% \changes{v2.2c}{1995/11/23}{Use \cs{@newctr} properly.}
%    \begin{macrocode}
    \@definecounter{#1}\@newctr{#1}[#3]%
%    \end{macrocode}
%    Using |\@newctr| will give a proper error message if the counter
%    in |#3| is not defined.
%    In defining `|\theLem|' we must generate the desired macro
%    name by use of |\expandafter| and |\csname|.
%    \begin{macrocode}
    \expandafter\xdef\csname the#1\endcsname
%    \end{macrocode}
%    An |\xdef| is used in order to make the definition global,
%    and to ensure that it contains the replacement texts of
%    |\@thmcountersep| and |\@thmcounter|.\footnote{These
%    two macros can be defined by the document style.  Their default
%    values produce a `\texttt{.}' as separation and an arabic
%    representation of the number.} However, not everything should be
%    expanded. For example, it saves space to use |\thesection|
%    instead of its---at times---lengthy expansion.
%    \begin{macrocode}
      {\expandafter \noexpand \csname the#3\endcsname
       \@thmcountersep \@thmcounter{#1}}%
%    \end{macrocode}
%    Thus with the defaults of \LaTeX{}, |\theLem| would be
%    replaced by the command sequence |\thesection.\arabic{Lem}|.
%
%    We will now look at the definition of the macro which is executed
%    at the beginning of the actual environment (in our example this
%    macro is |\Lem|). It should be noted that we use an
%    ``|\expandafter| trick'' to expand only certain parts of the
%    replacement text at the time of the definition.
%    \begin{macrocode}
    \def\@tempa{\global\@namedef{#1}}%
    \expandafter \@tempa \expandafter{%
%    \end{macrocode}
%    First, the macro that contains the current definitions of
%    |\@begintheorem| and |\@opargtheorem| should be called
%    up.  The name of this macro---as is already known---has the form
%    |\th@|\meta{theorem\ style}; therefore, it must be called by
%    \begin{macrocode}
      \csname th@\the \theorem@style
%    \end{macrocode}
%    In addition the default theorem font should be changeable, i.e.\
%    we have to insert the contents of |\theorem@bodyfont|. For
%    that reason, we expand even further, beyond |\endcsname|,
%    and thus insert the contents of the \textsf{token} register
%    |\theorem@bodyfont| in the replacement text.
%    \begin{macrocode}
            \expandafter \endcsname \the \theorem@bodyfont
%    \end{macrocode}
%    Now it is time to call the macro |\@thm| which takes over
%    the further processing.  It has two arguments: the current
%    counter name (in our example, \texttt{Lem}), and the text of the
%    label (in our example, \texttt{Lemma}).
%    \begin{macrocode}
     \@thm{#1}{#2}}%
%    \end{macrocode}
%    With this, the `sub-definition' is complete.  The macro
%    |\@endtheorem| ends a theorem environment and is, so far, nothing
%    but an |\endtrivlist|.  (Hence it is defined globally, and not
%    within the theorem styles.\footnote{This has to be changed as
%    soon as theorem styles that change \texttt{\bslash @endtheorem}
%    exist.  In such a case, all existing styles must be changed as
%    well since they will have to reset the macro.}) Therefore, we can
%    set it equivalent to the macro that ends the theorem set (in our
%    example, |\endLem|).  However, if some day theorem styles exist
%    that do change |\@endtheorem|, we would have to use the
%    commented-out line instead.
%    \begin{macrocode}
    \global \expandafter \let \csname end#1\endcsname \@endtheorem
%   \global\@namedef{end#1}{\@endtheorem}%
%    \end{macrocode}
%    With these commands all the required definitions are employed,
%    unless the test |\@ifdefinable| has failed.  Therefore, we
%    end the second argument of this macro and with it the definition
%    of |\@xnthm|.
%    \begin{macrocode}
   }}
%    \end{macrocode}
% \end{macro}
%
% \begin{macro}{\@ynthm}
%    The definition of |\@ynthm| is completely analogous. In
%    this case the new counter that is defined is not reset within
%    another counter; thus the definition of |\the...|
%    is simplified:
%    \begin{macrocode}
\gdef\@ynthm#1#2{\expandafter\@ifdefinable\csname #1\endcsname
   {\@definecounter{#1}%
    \expandafter\xdef\csname the#1\endcsname{\@thmcounter{#1}}%
%    \end{macrocode}
%    The rest of the definition corresponds literally to that of
%    |\@xnthm|:
%    \begin{macrocode}
    \def\@tempa{\global\@namedef{#1}}\expandafter \@tempa
     \expandafter{\csname th@\the \theorem@style \expandafter
     \endcsname \the\theorem@bodyfont \@thm{#1}{#2}}%
    \global \expandafter \let \csname end#1\endcsname \@endtheorem}}
%    \end{macrocode}
% \end{macro}
%
% \begin{macro}{\@othm}
%    The definition of |\@othm| does not contain anything new.
%    \begin{macrocode}
\gdef\@othm#1[#2]#3{%
%    \end{macrocode}
% \changes{v2.2b}{95/10/07}{Build in check for counter defined}
%    We do not define a new counter but instead use one that has
%    already been defined.  Thus the only definition we need is that
%    of this pseudo-counter (i.e.\ |\the|\meta{env.\ name}).
%    First we check if |#2| corresponds to a known counter name.
%    \begin{macrocode}
  \expandafter\ifx\csname c@#2\endcsname\relax
   \@nocounterr{#2}%
  \else
   \expandafter\@ifdefinable\csname #1\endcsname
   {\expandafter \xdef \csname the#1\endcsname
     {\expandafter \noexpand \csname the#2\endcsname}%
%    \end{macrocode}
%    All other parts of the definition can be adopted from
%    |\@xnthm|. We have to remember, though, that in this case
%    the name of the current counter and the theorem label have moved
%    to the second and third arguments.
%    \begin{macrocode}
    \def\@tempa{\global\@namedef{#1}}\expandafter \@tempa
     \expandafter{\csname th@\the \theorem@style \expandafter
     \endcsname \the\theorem@bodyfont \@thm{#2}{#3}}%
    \global \expandafter \let \csname end#1\endcsname \@endtheorem}%
  \fi}
%    \end{macrocode}
% \end{macro}
%
% \subsection{Macros that are employed in a {\envfont theorem}
%    environment}
%
% \begin{macro}{\@thm}
%    The macro |\@thm| has to increase the current counter.  Then,
%    depending on whether the environment has (or does not have) an
%    optional argument, it has to branch into either |\@begintheorem|
%    or |\@opargtheorem|.
%    \begin{macrocode}
\gdef\@thm#1#2{\refstepcounter{#1}%
%    \end{macrocode}
%    Now we start a {\envfont trivlist} environment, and give
%    |\@topsep| and |\@topsepadd| the values of the \textsf{skip}
%    registers |\theorempreskipamount| and |\theorempostskipamount|.
%    The value in |\@topsep| is the vertical space that is inserted by
%    the first (and only) |\item| in our |\trivlist| whilst
%    |\@topsepadd| is inserted by |\@endparenv| at the end of that
%    {\envfont trivlist} environment.  By using these registers, we
%    obtain the desired space around a {\envfont theorem} environment.
%    \begin{macrocode}
   \trivlist
   \@topsep \theorempreskipamount               % used by first \item
   \@topsepadd \theorempostskipamount           % used by \@endparenv
%    \end{macrocode}
%    Now we have to test whether an optional argument has been given.
% \changes{v2.0d}{89/07/18}{Forgotten percent added.}
%    \begin{macrocode}
   \@ifnextchar [%
%    \end{macrocode}
%    If there is an optional argument, we will call |\@ythm|, and
%    move the arguments read back into the input stream.
%    \begin{macrocode}
   {\@ythm{#1}{#2}}%
%    \end{macrocode}
%    If not, we call |\@begintheorem|. Its first argument is the
%    name of the theorem set (hence the second argument of
%    |\@thm|). Its second argument is the macro that produces the
%    current number.
%    \begin{macrocode}
   {\@begintheorem{#2}{\csname the#1\endcsname}\ignorespaces}}
%    \end{macrocode}
% \end{macro}
%
% \begin{macro}{\@xthm}
% \begin{macro}{\@ythm}
%    Both these macros were originally called by |\@thm|.  We do
%    not need |\@xthm| anymore, hence we reset it to
%    |\relax|. The definition of |\@ythm| has not changed at
%    all from its definition in \LaTeX. In order to make the macros
%    easier to understand, we will
%    nevertheless present it (commented out).
%    \begin{macrocode}
\global\let\@xthm\relax
% \def\@ythm#1#2[#3]{\@opargbegintheorem{#2}{\csname
%           the#1\endcsname}{#3}\ignorespaces}
%</package>
%    \end{macrocode}
%    The primitive |\ignorespaces| in |\@ythm| and
%    |\@thm| is needed to remove the spaces between the
%    |\begin{...}| and the actual text.
% \end{macro}
% \end{macro}
%
%
% \subsection{Definition of the theorem substyles}
%
% As already pointed out, the theorem substyles, defined below, are
% only loaded when necessary.  Note that all these substyles, except
% \texttt{plain}, have |\slshape| as the default body font.
%
% \subsubsection{The \texttt{plain} style}
%
% \label{dok:th@plain} As the following macros use |@|, we have
% to locally set the |\catcode| of this symbol to ``letter''.
% This happens within a group, so that we do not have to worry about
% which |\catcode| that symbol had before.
%    \begin{macrocode}
%<*thp>
\begingroup \makeatletter
%    \end{macrocode}
% Since we are now within a group, we must make all definitions
% globally.  First we make sure that \texttt{theorem.sty} is loaded.
% This will allow us to use this file as a document style option
% without having to call \texttt{theorem} itself as an option.  At the
% same time, we assure that at least version 2 is loaded, since
% |\theorem@style| was not defined in earlier versions.
%    \begin{macrocode}
\@ifundefined{theorem@style}{\input{theorem.sty}}{}
\ProvidesFile{thp.sty}
             [\FMithmInfo]
%    \end{macrocode}
%
%
% \begin{macro}{\th@plain}
% \begin{macro}{\@begintheorem}
% \begin{macro}{\@opargbegintheorem}
%    |\theoremstyle{plain}| corresponds to the original
%    definition, except that the distances to the surrounding text are
%    determined by the parameters |\theorempreskipamount| and
%    |\theorempostskipamount|.
%    First we set the default body font.
%    \begin{macrocode}
\gdef\th@plain{\normalfont\itshape
%    \end{macrocode}
%    Then we define |\@begintheorem| and
%    |\@opargbegintheorem|. These two macros define how the
%    header of a theorem is typeset.  |\@opargbegintheorem| will
%    be called if a \textsf{theorem} environment with an optional
%    argument is encountered; otherwise, the header is constructed by
%    calling |\@begintheorem|.  If one of these macros is
%    executed, we are within a \textsf{trivlist} environment started by
%    |\@thm|.  So the theorem header is produced with an
%    |\item| command.
%
%    Instead of specifying the header font directly, all standard
%    theorem styles use the |\theorem@headerfont| macro to allow
%    customization.  The extra space (|\labelsep|) is necessary
%    because of problems in the \textsf{trivlist} environment.
%    \begin{macrocode}
  \def\@begintheorem##1##2{%
        \item[\hskip\labelsep \theorem@headerfont ##1\ ##2]}%
%    \end{macrocode}
%    The definition of |\@opargbegintheorem| is completely
%    analogous. The only difference is the fact that there exists a
%    third argument (which is the optional parameter of the
%    environment and contains additional information about the
%    theorem).  Customarily we enclose it in parentheses.
%    \begin{macrocode}
\def\@opargbegintheorem##1##2##3{%
   \item[\hskip\labelsep \theorem@headerfont ##1\ ##2\ (##3)]}}
%    \end{macrocode}
% \end{macro}
% \end{macro}
% \end{macro}
%
% \noindent
% We conclude with an |\endgroup| to restore the
% |\catcode| of |@|.
%    \begin{macrocode}
\endgroup
%</thp>
%    \end{macrocode}
%
%
% \subsubsection{The \texttt{break} style}
%
% This style option is stored in the file |thb.sty|.
% For the next two lines see the documentation for |\th@plain| on
% page~\pageref{dok:th@plain}.
%    \begin{macrocode}
%<*thb>
\begingroup \makeatletter
\@ifundefined{theorem@style}{\input{theorem.sty}}{}
\ProvidesFile{thb.sty}
             [\FMithmInfo]
%    \end{macrocode}
%
% \begin{macro}{\th@break}
%    |\theoremstyle{break}| produces a line break after the name
%    of the theorem.  The font is |\slshape|.  Hence, we define
%    |\th@break| as follows:
%    \begin{macrocode}
\gdef\th@break{\normalfont\slshape
  \def\@begintheorem##1##2{\item[%
%    \end{macrocode}
%    We run into the following problem: it is not possible to create
%    the header with |\item[|\meta{title}|]| and then start
%    a new line by, for example, |\mbox{}\\|. Such a definition
%    will fail whenever a list environment follows immediately.  With
%    the above construction, the |\mbox{}| causes the switch
%    |@inlabel| (cf.\ definition of |\list| and
%    |\trivlist| in \cite{latex:source}) to be set to \texttt{false}
%    and so the following list will insert additional vertical
%    space (|\topskip|). This is quite annoying. Therefore, we
%    create the line break within the |\item|. In order to ensure
%    that the text will begin at the proper position in the following
%    line, we simply pretend that the label does not take any
%    room.\footnote{This will lead to problems whenever very high
%    symbols occurring in the line tower into the
%    heading. So, something else has to be done here sometime.}
%    \begin{macrocode}
       \rlap{\vbox{\hbox{\hskip \labelsep\theorem@headerfont ##1\ ##2}%
                   \hbox{\strut}}}]}%
%    \end{macrocode}
%    Again, the definition of |\@opargbegintheorem| is
%    completely analogous.
%    \begin{macrocode}
\def\@opargbegintheorem##1##2##3{%
  \item[\rlap{\vbox{\hbox{\hskip \labelsep \theorem@headerfont
                     ##1\ ##2\ (##3)}%
                    \hbox{\strut}}}]}}
\endgroup
%</thb>
%    \end{macrocode}
% \end{macro}
%
%
% \subsubsection{The \texttt{changebreak} style}
%
%    \begin{macrocode}
% This style option is stored in the file |thcb.sty|.
%    \begin{macrocode}
%<*thcb>
\begingroup \makeatletter
\@ifundefined{theorem@style}{\input{theorem.sty}}{}
\ProvidesFile{thcb.sty}
             [\FMithmInfo]
%    \end{macrocode}
%
% \begin{macro}{\th@changebreak}
%    The \texttt{change-break} theorem style is like \texttt{break}
%    but with interchange of theorem name and theorem number.  Thus we
%    define |\th@changebreak| as follows:
%    \begin{macrocode}
\gdef\th@changebreak{\normalfont\slshape
  \def\@begintheorem##1##2{\item
       [\rlap{\vbox{\hbox{\hskip\labelsep \theorem@headerfont ##2\ ##1}%
                    \hbox{\strut}}}]}%
\def\@opargbegintheorem##1##2##3{%
  \item[\rlap{\vbox{\hbox{\hskip\labelsep \theorem@headerfont
                     ##2\ ##1\ (##3)}%
                    \hbox{\strut}}}]}}
\endgroup
%</thcb>
%    \end{macrocode}
% \end{macro}
%
%
% \subsubsection{The \texttt{change} style}
%
% This style option is stored in the file |thc.sty|.
%    \begin{macrocode}
%<*thc>
\begingroup \makeatletter
\@ifundefined{theorem@style}{\input{theorem.sty}}{}
\ProvidesFile{thc.sty}
             [\FMithmInfo]
%    \end{macrocode}
%
% \begin{macro}{\th@change}
%    The \texttt{change} theorem style corresponds to the
%    \texttt{change break} style without a linebreak after the header.
%    To say it in another way, it's the same as the \texttt{plain}
%    style but with number and name interchanged and |\slshape| as the
%    default font.
%    \begin{macrocode}
\gdef\th@change{\normalfont\slshape
  \def\@begintheorem##1##2{\item
       [\hskip\labelsep \theorem@headerfont ##2\ ##1]}%
\def\@opargbegintheorem##1##2##3{%
  \item[\hskip\labelsep \theorem@headerfont ##2\ ##1\ (##3)]}}
\endgroup
%</thc>
%    \end{macrocode}
% \end{macro}
%
%
% \subsubsection{The \texttt{marginbreak} style}
%
% This style option is the one used most often at Mainz.  It is saved
% in the file |thmb.sty|.
%    \begin{macrocode}
%<*thmb>
\begingroup \makeatletter
\@ifundefined{theorem@style}{\input{theorem.sty}}{}
\ProvidesFile{thmb.sty}
             [\FMithmInfo]
%    \end{macrocode}
%
% \begin{macro}{\th@marginbreak}
%    The \texttt{margin break} style is nearly the same as the
%    \texttt{change break} style. The only difference is the placement
%    of the theorem number.  We use |\llap| to place it in the left
%    margin.
%
%    In this style |\labelsep| denotes the separation between the
%    number and the text.
%    \begin{macrocode}
\gdef\th@marginbreak{\normalfont\slshape
  \def\@begintheorem##1##2{\item
       [\rlap{\vbox{\theorem@headerfont
                    \hbox{\llap{##2}\hskip\labelsep ##1}%
                    \hbox{\strut}}}]}%
\def\@opargbegintheorem##1##2##3{%
  \item[\rlap{\vbox{\theorem@headerfont
                    \hbox{\llap{##2}\hskip\labelsep ##1\ (##3)}%
                    \hbox{\strut}}}]}}
\endgroup
%</thmb>
%    \end{macrocode}
% \changes{v2.1c}{90/02/17}{`endgroup outside macrocode env.}
% \end{macro}
%
%
%
% \subsubsection{The \texttt{margin} style}
%
% This style option is stored in the file |thm.sty|.
%    \begin{macrocode}
%<*thm>
\begingroup \makeatletter
\@ifundefined{theorem@style}{\input{theorem.sty}}{}
\ProvidesFile{thm.sty}
             [\FMithmInfo]
%    \end{macrocode}
%
% \begin{macro}{\th@margin}
%    Again this is only a variant of the theorem styles described
%    above without any new ideas.
%    \begin{macrocode}
\gdef\th@margin{\normalfont\slshape
  \def\@begintheorem##1##2{\item
       [\theorem@headerfont \llap{##2}\hskip\labelsep ##1]}%
\def\@opargbegintheorem##1##2##3{%
  \item[\theorem@headerfont \llap{##2}\hskip\labelsep ##1\ (##3)]}}
\endgroup
%</thm>
%    \end{macrocode}
% \end{macro}
%
%
%
%
% \subsection{Final Definitions}
%
% \begin{macro}{\theorempreskipamount}
% \begin{macro}{\theorempostskipamount}
%    The \textsf{skip} parameters that regulate the vertical empty space
%    before and after the {\envfont theorem} environment have
%    to be allocated as well.
% \changes{v2.0f}{89/07/20}{Unnecessary `global' removed.}
%    \begin{macrocode}
%<*package>
\newskip\theorempreskipamount
\newskip\theorempostskipamount
%    \end{macrocode}
%    Since we have used the same values for all theorem sets, we now can
%    assign them.
%    \begin{macrocode}
\global\setlength\theorempreskipamount{12pt plus 5pt minus 3pt}
\global\setlength\theorempostskipamount{8pt plus 3pt minus 1.5pt}
%    \end{macrocode}
% \end{macro}
% \end{macro}
%
%
% \begin{macro}{\@endtheorem}
%    The same holds for the macro |\@endtheorem|, which ends a
%    {\envfont theorem} environment. Since it is the same for all
%    theorem sets, it is removed from the macros |\th@|\meta{style}.
%    It simply ends the {\envfont trivlist} environment, which was
%    begun in |\@thm|.
%    \begin{macrocode}
\global\let\@endtheorem=\endtrivlist
%    \end{macrocode}
% \end{macro}
%
%
%
% \begin{macro}{\@preamblecmds}
%    All macros defined above are to be used only in the preamble.
%    Therefore, we insert them in |\@preamblecmds| which will disable
%    them at begin document. This is done by the internal \LaTeXe{}
%    command |\@onlypreamble|.
% \changes{v2.0g}{89/09/19}{`theoremheaderfont and `theorembodyfont
%    added.}
% \changes{v2.2a}{94/02/03}{Use \LaTeXe{} interface}
%    \begin{macrocode}
\@onlypreamble\@xnthm
\@onlypreamble\@ynthm
\@onlypreamble\@othm
\@onlypreamble\newtheorem
\@onlypreamble\theoremstyle
\@onlypreamble\theorembodyfont
\@onlypreamble\theoremheaderfont
%    \end{macrocode}
% \end{macro}
%
% Finally we declare the \texttt{plain} theorem style to be the default.
% \changes{v2.1b}{90/06/04}{Fixed missing `item by declaring default
%    style.}
%    \begin{macrocode}
\theoremstyle{plain}
%</package>
%    \end{macrocode}
%
% \Finale

\endinput
}{}
\ProvidesFile{thm.sty}
             [\FMithmInfo]
%    \end{macrocode}
%
% \begin{macro}{\th@margin}
%    Again this is only a variant of the theorem styles described
%    above without any new ideas.
%    \begin{macrocode}
\gdef\th@margin{\normalfont\slshape
  \def\@begintheorem##1##2{\item
       [\theorem@headerfont \llap{##2}\hskip\labelsep ##1]}%
\def\@opargbegintheorem##1##2##3{%
  \item[\theorem@headerfont \llap{##2}\hskip\labelsep ##1\ (##3)]}}
\endgroup
%</thm>
%    \end{macrocode}
% \end{macro}
%
%
%
%
% \subsection{Final Definitions}
%
% \begin{macro}{\theorempreskipamount}
% \begin{macro}{\theorempostskipamount}
%    The \textsf{skip} parameters that regulate the vertical empty space
%    before and after the {\envfont theorem} environment have
%    to be allocated as well.
% \changes{v2.0f}{89/07/20}{Unnecessary `global' removed.}
%    \begin{macrocode}
%<*package>
\newskip\theorempreskipamount
\newskip\theorempostskipamount
%    \end{macrocode}
%    Since we have used the same values for all theorem sets, we now can
%    assign them.
%    \begin{macrocode}
\global\setlength\theorempreskipamount{12pt plus 5pt minus 3pt}
\global\setlength\theorempostskipamount{8pt plus 3pt minus 1.5pt}
%    \end{macrocode}
% \end{macro}
% \end{macro}
%
%
% \begin{macro}{\@endtheorem}
%    The same holds for the macro |\@endtheorem|, which ends a
%    {\envfont theorem} environment. Since it is the same for all
%    theorem sets, it is removed from the macros |\th@|\meta{style}.
%    It simply ends the {\envfont trivlist} environment, which was
%    begun in |\@thm|.
%    \begin{macrocode}
\global\let\@endtheorem=\endtrivlist
%    \end{macrocode}
% \end{macro}
%
%
%
% \begin{macro}{\@preamblecmds}
%    All macros defined above are to be used only in the preamble.
%    Therefore, we insert them in |\@preamblecmds| which will disable
%    them at begin document. This is done by the internal \LaTeXe{}
%    command |\@onlypreamble|.
% \changes{v2.0g}{89/09/19}{`theoremheaderfont and `theorembodyfont
%    added.}
% \changes{v2.2a}{94/02/03}{Use \LaTeXe{} interface}
%    \begin{macrocode}
\@onlypreamble\@xnthm
\@onlypreamble\@ynthm
\@onlypreamble\@othm
\@onlypreamble\newtheorem
\@onlypreamble\theoremstyle
\@onlypreamble\theorembodyfont
\@onlypreamble\theoremheaderfont
%    \end{macrocode}
% \end{macro}
%
% Finally we declare the \texttt{plain} theorem style to be the default.
% \changes{v2.1b}{90/06/04}{Fixed missing `item by declaring default
%    style.}
%    \begin{macrocode}
\theoremstyle{plain}
%</package>
%    \end{macrocode}
%
% \Finale

\endinput
}{}
\ProvidesFile{thcb.sty}
             [\FMithmInfo]
%    \end{macrocode}
%
% \begin{macro}{\th@changebreak}
%    The \texttt{change-break} theorem style is like \texttt{break}
%    but with interchange of theorem name and theorem number.  Thus we
%    define |\th@changebreak| as follows:
%    \begin{macrocode}
\gdef\th@changebreak{\normalfont\slshape
  \def\@begintheorem##1##2{\item
       [\rlap{\vbox{\hbox{\hskip\labelsep \theorem@headerfont ##2\ ##1}%
                    \hbox{\strut}}}]}%
\def\@opargbegintheorem##1##2##3{%
  \item[\rlap{\vbox{\hbox{\hskip\labelsep \theorem@headerfont
                     ##2\ ##1\ (##3)}%
                    \hbox{\strut}}}]}}
\endgroup
%</thcb>
%    \end{macrocode}
% \end{macro}
%
%
% \subsubsection{The \texttt{change} style}
%
% This style option is stored in the file |thc.sty|.
%    \begin{macrocode}
%<*thc>
\begingroup \makeatletter
\@ifundefined{theorem@style}{% \iffalse meta-comment
%
% Copyright (C) 1993-2024
%
% The LaTeX Project and any individual authors listed elsewhere
% in this file.
%
% This file is part of the Standard LaTeX `Tools Bundle'.
% -------------------------------------------------------
%
% It may be distributed and/or modified under the
% conditions of the LaTeX Project Public License, either version 1.3c
% of this license or (at your option) any later version.
% The latest version of this license is in
%    https://www.latex-project.org/lppl.txt
% and version 1.3c or later is part of all distributions of LaTeX
% version 2005/12/01 or later.
%
% The list of all files belonging to the LaTeX `Tools Bundle' is
% given in the file `manifest.txt'.
%
% \fi
%
%%
%
%
\def\FMithmInfo{2023/07/05 v2.2c Theorem extension package (FMi)}
%
% \ProvidesFile{theorem.dtx}[\FMithmInfo]
%\iffalse   % this is a METACOMMENT !
%%
%% Package `theorem' to use with LaTeX2e
%% Copyright (C) 1989-1995 Frank Mittelbach, all rights reserved.
%<+package>\@ifundefined{theorem@style}{}{\endinput}
%<+package>\NeedsTeXFormat{LaTeX2e}[1995/06/01]
%<+package>\ProvidesPackage{theorem}
%<+package>         [\FMithmInfo]
%
% \fi
%
% \changes{v1.0}{87/05/07}{First draft.}
% \changes{v1.0a}{87/05/07}{`thistheoremstyle' added.}
% \changes{v1.1a}{87/07/29}{New concept: redefine the internal
%                          \protect\LaTeX{} commands `@xnthm' etc.}
% \changes{v1.1b}{87/07/31}{`theoremfont' used to specify font.}
% \changes{v1.2a}{88/01/08}{`@endtheorem' now globally defined.}
% \changes{v1.2b}{88/01/11}{`newtheorem' only allowed in the preamble.}
% \changes{v1.3a}{88/02/14}{`theoremfont' now a toks register.}
%
% \changes{v2.0a}{88/11/16}{New implementation and new documentation.}
% \changes{v2.0a}{88/11/16}{No layout changes but internal modifications
%                           old substyles won't work.}
% \changes{v2.0b}{89/01/30}{`global' used for all definitions.}
% \changes{v2.0c}{89/06/15}{Documentation translated to English.
%                            Thanks to LLL and Nina}
% \changes{v2.1a}{90/02/15}{Support for new font selection.}
% \changes{v2.2a}{94/02/03}{Cleanup for LaTeX2e}
%
% \DoNotIndex{\@endparenv,\@ifdefinable,\@ifnextchar,\@ifundefined}
% \DoNotIndex{\@namedef,\@nameuse,\@spaces,\@tempa,\@warning}
% \DoNotIndex{\ }
% \DoNotIndex{\begingroup,\bfseries}
% \DoNotIndex{\csname}
% \DoNotIndex{\def,\do,\docdate}
% \DoNotIndex{\endcsname,\endgroup,\endinput,\endtrivlist,\expandafter}
% \DoNotIndex{\filedate,\fileversion}
% \DoNotIndex{\gdef,\global}
% \DoNotIndex{\hbox,\hskip}
% \DoNotIndex{\ignorespaces,\input,\itshape,\item}
% \DoNotIndex{\labelsep,\let,\llap}
% \DoNotIndex{\makeatletter}
% \DoNotIndex{\noexpand}
% \DoNotIndex{\refstepcounter,\relax,\rlap}
% \DoNotIndex{\setlength,\slshape,\space,\string,\strut}
% \DoNotIndex{\the,\trivlist,\typeout}
% \DoNotIndex{\vbox}
% \DoNotIndex{\xdef}
%
% \def\deflabel#1{\ttfamily #1\hfill}
% \def\deflist#1{\begin{list}{}{\settowidth\labelwidth{\ttfamily #1}%
%                               \setlength\leftmargin\labelwidth
%                               \addtolength\leftmargin\labelsep
%                               \let\makelabel\deflabel}}
% \def\enddeflist{\end{list}}
% \def\envfont{\normalfont\ttfamily}
%
% \MakeShortVerb{\|}
%
% \newcommand{\TUB}{{\sl TUGboat\/}}
%
% \GetFileInfo{theorem.dtx}
% \title{An Extension of the
%        \LaTeX{} theorem environment\thanks{This file
%        has version number \fileversion, last
%        revised \filedate.}}
% \author{Frank Mittelbach}
%
% \MaintainedByLaTeXTeam{tools}
% \maketitle
%
% \begin{abstract}
%    The macros described in this paper yield an extension of the
%    \LaTeX{} theorem mechanism. It is designed to satisfy the
%    different requirements of various journals. Thus, the
%    layout of the ``theorems'' can be manipulated by determining a
%    ``style''. This article describes not only the use, but
%    also the definition, of the necessary macros.
% \end{abstract}
%
% \section*{Preface to version 2.2}
%
% For \LaTeXe{} this package did not need any fundamental changes. I
% only modified the messages generated so that theorem layout styles
% will show up with the |\listfiles| command and cleaned the section
% on the New Font Selection Scheme since this is now included in
% \LaTeX.
%
% \section*{Preface to version 2.1}
%
% This version is identical to 2.0g described in \TUB\ 10\#3 except
% for some internal defaults which are now set depending on the used
% font selection scheme.
%
% This was done to avoid unpleasant surprises if the new font selection
% scheme is in force. For further details see
% section~\ref{sec:newfonts} and \cite{bk:GMS94}.
%
% \section{Introduction}
%
% For our purposes here, ``theorems'' are labelled enunciations,
% often set off from the main text by extra space and a font change.
% Theorems, corollaries, conjectures, definitions, and remarks
% are all instances of ``theorems''.  The ``header'' of these structures
% is composed of a label (such as \textsc{Theorem} or \textsc{Remark})
% and a number which serializes an item in the sequence of items
% with the same label.
%
% Shortly after the introduction of \LaTeX{} at the Fachbereich
% Mathematik in Mainz, the desire to manipulate the layout of
% ``theorems'' arose.  In Mainz, the following two conventions came
% into general use:
% \begin{enumerate}
%   \item The number of the theorem is shown in the margin.
%   \item There is a line break at
%         the end of the theorem header.
% \end{enumerate}
%
% Additionally, some journals require different formats which depend on
% the ``sort of theorem'': e.g.\ often remarks and definitions are set
% in |\upshape|, while |\itshape| is employed for main theorems.
%
% Confronted with these requirements, a theorem environment was
% developed in Mainz which allows separate determination of the
% layout of the ``theorems sets'', comparable to |\pagestyle|.
%
%
% \section{The user interface}
% \subsection{Defining new theorem sets}
%
% \DescribeMacro\newtheorem
% As in the original \LaTeX{} version, the command |\newtheorem|
% defines a new ``theorem set'' or ``theorem-like structure''.
% Two required arguments name the new environment and give the text
% to be typeset with each instance of the new ``set'', while
% an optional argument determines how
% the ``set'' is enumerated:
% \begin{description}
%    \item[\ttfamily \bslash newtheorem\{foo\}\{bar\}]
%       The theorem set {\envfont foo} (whose name is \texttt{bar})
%       uses its own counter.
%    \item[\ttfamily \bslash newtheorem\{foo2\}{[foo]}\{bar2\}]
%       The theorem set {\envfont foo2} (printed name \texttt{bar2})
%       uses the same counter as the theorem set \texttt{foo}.
%    \item[\ttfamily \bslash newtheorem\{foo3\}\{bar3\}{[section]}]
%       The theorem set {\envfont foo3} (printed name \texttt{bar3}) is
%       enumerated within the counter \texttt{section}, i.e.\ with every
%       new |\section| the enumeration begins again with 1, and
%       the enumeration is composed from the section-number and the
%       theorem counter itself.
% \end{description}
%
% \DescribeMacro\theoremstyle
% Additionally, the command |\theoremstyle| can define the layout
% of various, or all, theorem sets.  It should be noted that any
% theorem set defined by |\newtheorem| is typeset in the
% |\theoremstyle| that is current at the time of the definition.
% Thus, the following
% \begin{verbatim}
%    \theoremstyle{break}        \newtheorem{Cor}{Corollary}
%    \theoremstyle{plain}        \newtheorem{Exa}{Example}[section]
%\end{verbatim}                             ^^A No blank at line start
% leads to the result that the set {\envfont Cor} is formatted in the
% style \texttt{break}, while the set {\envfont Exa} and all the
% following ones are formatted in the style \texttt{plain}, unless
% another |\theoremstyle| follows.  Since the definitions installed by
% |\newtheorem| are global, one also can limit |\theoremstyle| locally
% by grouping braces.
%
% \DescribeMacro\theorembodyfont
% The choice of the font for the theorem body is completely
% independent of the chosen |\theoremstyle|; this has proven to
% be very advantageous.  For example,
% \begin{verbatim}
%   {\theorembodyfont{\upshape}          \newtheorem{Rem}{Remark}}
%\end{verbatim}
% defines a theorem set {\envfont Rem}, which will be set in
% |\upshape| in the current layout (which in our example is
% \texttt{plain}).  As with |\theoremstyle|, the |\theorembodyfont|
% chosen is that current at the time of |\newtheorem|.  If
% |\theorembodyfont| is not specified or one defines
% |\theorembodyfont{}|, then the font used will be that defined by the
% |\theoremstyle|.
%
% \DescribeMacro\theoremheaderfont It is also possible to customize
% the font used for the theorem headers.  This is, however, a global
% declaration, and therefore there should be at most one
% |\theoremheaderfont| declaration in the preamble.\footnote{If it is
% actually necessary to have different header fonts, one has to define
% new theorem styles (substituting the desired font) or specify the
% information directly in the \texttt{\bslash newtheorem} declaration
% (the unclean variant).}
%
% \DescribeMacro\theorempreskipamount
% \DescribeMacro\theorempostskipamount Two additional parameters
% affect the vertical space around the theorem environments:
% |\theorempreskipamount| and |\theorempostskipamount| define,
% respectively, the spacing before and after such an environment.
% These parameters apply for all theorem sets and can be manipulated
% with the ordinary length macros.  They are rubber lengths
% (`\textsf{skips}'), and therefore can contain \texttt{plus} and
% \texttt{minus} parts.
%
% Since the definition of theorem sets should---most sensibly---be
% placed in the preamble, we only allow installation there.
% It is therefore possible to release the memory used here after
% |\begin{document}|, in order to make room for other
% applications.
%
%
% \subsection{Existing theorem styles}
%
% The following theorem styles exist to date:
% \begin{deflist}{marginbreak}
%    \item[plain]
%       This theorem style emulates the original \LaTeX{} definition,
%       except that additionally the parameters
%       |\theorem...skipamount| are used.
%    \item[break]
%       In this style, the theorem header is followed by a line
%       break.
%    \item[marginbreak]
%       The theorem number is set in the margin, and there is a line
%       break as in \texttt{break}.
%    \item[changebreak]
%       Like \texttt{break}, but with header number and text
%       interchanged.
%    \item[change]
%       Header number and text are interchanged, without a line break.
%    \item[margin]
%       The number is set in the left margin, without a line break.
% \end{deflist}
% All styles (except \texttt{plain}) select |\slshape| as the default
% |\theorembodyfont|.
%
%
% \subsection{Examples}
%
% Given the above theorem sets {\envfont Cor}, {\envfont Exa}
% and {\envfont Rem},
% suppose that the preamble also contains the declarations:
% \begin{verbatim}
%    \theoremstyle{marginbreak}   \newtheorem{Lem}[Cor]{Lemma}
%    \theoremstyle{change}
%    \theorembodyfont{\itshape}        \newtheorem{Def}[Cor]{Definition}
%
%    \theoremheaderfont{\scshape}
%\end{verbatim}
% Then the following are some typical examples of the typeset output
% resulting from their use.
%
% \begin{Cor}
%    This is a sentence typeset in the theorem environment {\envfont
%    Cor}.
% \end{Cor}
%
% \begin{Exa}
%    This is a sentence typeset in the theorem environment {\envfont
%    Exa}.
% \end{Exa}
%
% \begin{Rem}
%    This is a sentence typeset in the theorem environment {\envfont
%    Rem}.
% \end{Rem}
%
% \begin{Lem}[Ben User]
%    This is a sentence typeset in the theorem environment {\envfont
%    Lem}.
% \end{Lem}
%
% \begin{Def}[Very impressive Definition]
%    This is a sentence typeset in the theorem environment {\envfont
%    Def}.
% \end{Def}
%
% The last two examples show the effect of the optional argument to
% a theorem environment (it is the text typeset in parentheses).
%
%
% \section{Special Considerations}
% \label{sec:newfonts}
%
% Theoremheader and body are implemented as a unit. This means that
% the |\theoremheaderfont|  will inherit characteristics of the
% |\theorembodyfont| in \LaTeXe. Thus, if for example
% |\theorembodyfont| is |\itshape| and |\theoremheaderfont|
% is |\bfseries| the font selected for the header will have the
% characteristics `bold extended italic'. If this is not desired one
% should set the |\theoremheaderfont| to something like
% \begin{verbatim}
%      \theoremheaderfont{\normalfont\bfseries}
%\end{verbatim}
% i.e.\ supplying all necessary font information explicitly.
%
%
%
% \section{Acknowledgements}
%
% The publication of this set of macros was only possible with the
% help of Christina Busse (translating the manuscript into English),
% Joachim Pense (playing the r\^ole of typist), Chris Rowley
% (looking everything over) and many others providing useful
% suggestions.
%
% \MaybeStop{%
%    \begin{thebibliography}{9}
%       \bibitem{bk:GMS94} \textsc{M.~Goossens}, \textsc{F.~Mittelbach}
%         and \textsc{A.~Samarin}.
%         \newblock The \LaTeX{} Companion.
%         \newblock
%         Addison-Wesley, Reading, Massachusetts, 1994.
%       \bibitem{latex:source} \textsc{Lamport, Leslie.} \newblock
%                               \texttt{latex.tex}, version 2.09, date
%                               Feb. 1990.
%    \end{thebibliography}
%    \PrintIndex
%    \PrintChanges}
%
%
% \section{The documentation driver file}
%
% The next bit of code contains the documentation driver file for
% \TeX{}, i.e., the file that will produce the documentation you are
% currently reading. It will be extracted from this file by the
% \texttt{docstrip} program. Since it is the first code in the file
% one can alternatively process this file directly with \LaTeXe{} to
% obtain the documentation.
%
%    \begin{macrocode}
%<*driver>
\documentclass{ltxdoc}

\usepackage{theorem}

%<+driver>% The next few lines define theorem sets which are used
%<+driver>% in the example section of the documentation.

\theoremstyle{break}       \newtheorem{Cor}{Corollary}
\theoremstyle{plain}       \newtheorem{Exa}{Example}[section]
{\theorembodyfont{\upshape}\newtheorem{Rem}{Remark}}
\theoremstyle{marginbreak} \newtheorem{Lem}[Cor]{Lemma}
\theoremstyle{change}
\theorembodyfont{\itshape} \newtheorem{Def}[Cor]{Definition}

\theoremheaderfont{\scshape}

\RecordChanges

\begin{document}
   \DocInput{theorem.dtx}
\end{document}
%</driver>
%    \end{macrocode}
%
% \section{Definition of the Macros}
%
% If the file has been loaded before, we abort immediately. If not the
% package announces itself (this is actually done at the very top of
% the file---the way it is done isn't good style so don't copy it).
% \changes{v2.0e}{89/07/19}{Spaces removed from `typeout'.}
% \changes{v2.2b}{95/11/19}{Announce moved to top of file}
%    \begin{macrocode}
%<*package>
%\@ifundefined{theorem@style}{}{\endinput}
%\def\FMithmInfo{1995/11/19 v2.2b Theorem extension package (FMi)}
%\ProvidesPackage{theorem}[\FMithmInfo]
%    \end{macrocode}
%
%
% \subsection{Definition of theorem styles and fonts}
%
% All the definitions in this file are done globally to allow
% inputting this file inside a group.
%
% \begin{macro}{\theoremstyle}
%    Before a theorem style can be installed, the chosen style must be
%    known. For that reason, we must test to see that
%    |\th@|\meta{style} is known or, more precisely, that it is
%    different from |\relax|. If the style is not known then
%    |\th@plain| is used.
%    \begin{macrocode}
\gdef\theoremstyle#1{%
   \@ifundefined{th@#1}{\@warning
          {Unknown theoremstyle `#1'. Using `plain'}%
          \theorem@style{plain}}%
%    \end{macrocode}
%    We save the theorem style to be used in the token register
%    |\theorem@style|.
%    \begin{macrocode}
      {\theorem@style{#1}}%
%    \end{macrocode}
%    Now we ``evaluate'' the theorem style: this means, we call the
%    macro |\th@|\meta{style} which will activate the relevant
%    definitions which are contained in a separate file.  This is done
%    in a group to suppress changes to the current font.  This could
%    otherwise pose problems together with the new font selection
%    scheme\footnote{When I printed the original article using the new
%    font selection scheme I ended with a document with slanted
%    typefaces (text headings and all) simply because one of the
%    theorem styles used \texttt{\bslash sl} at toplevel.} if the
%    |\th@|\meta{style} is evaluated a second time.
% \changes{v2.1a}{90/02/15}{Group added around `th@..}
% \changes{v2.1a}{90/02/15}{Reseting of `..@begintheorem removed.}
%    \begin{macrocode}
      \begingroup
        \csname th@\the\theorem@style \endcsname
      \endgroup}
%    \end{macrocode}
% \end{macro}
%
%
% \begin{macro}{\@begintheorem}
% \changes{v2.1a}{90/02/15}{Relaxed at toplevel.}
% \begin{macro}{\@opargbegintheorem}
% \changes{v2.1a}{90/02/15}{Relaxed at toplevel.}
%    We reset |\@begintheorem| and |\@opargbegintheorem| to
%    |\relax| since these commands are no longer necessary at
%    toplevel. This will save a few tokens.
%    \begin{macrocode}
\global\let\@begintheorem\relax
\global\let\@opargbegintheorem\relax
%    \end{macrocode}
% \end{macro}
% \end{macro}
%
% \begin{macro}{\theorem@style}
%    Obviously the \textsf{token} register used above has to be
%    allocated. To assure the utmost compatibility with the
%    original \LaTeX{} definition, we set the default theorem style to
%    \texttt{plain}, which implements the usual \LaTeX{} convention.
% \changes{v2.0f}{89/07/20}{Unnecessary `global' removed.}
%    \begin{macrocode}
\newtoks\theorem@style
\global\theorem@style{plain}
%    \end{macrocode}
% \end{macro}
%
% \begin{macro}{\theorembodyfont}
% \begin{macro}{\theorem@bodyfont}
%    If the |\theorembodyfont| is set by the user then it should not
%    interact with the default font set in the theorem style. When the
%    new font selection is in force this may happen if, for example,
%    the default is |\itshape| and the new |\theorembodyfont| is
%    |\sffamily|. So we add a |\reset@font| command in front of the
%    user definition.
% \changes{v2.0f}{89/07/20}{Unnecessary `global' removed.}
% \changes{v2.0g}{89/09/19}{`theoremfont to `theorembodyfont.}
% \changes{v2.1c}{92/06/22}{`reset@font added and logic changed.}
% \changes{v2.1d}{92/10/28}{Reset to empty if arg is empty}
%    \begin{macrocode}
\gdef\theorembodyfont#1{%
%    \end{macrocode}
%    We check if the argument supplied is empty and if so put nothing
%    into the |\theorem@bodyfont| token register to allow for
%    |\theorembodyfont{}| as a mean of using the default of the
%    current |\theoremstyle|.
%    \begin{macrocode}
   \def\@tempa{#1}%
   \ifx\@tempa\@empty
    \theorem@bodyfont{}%
   \else
    \theorem@bodyfont{\reset@font#1}%
   \fi
   }
\newtoks\theorem@bodyfont
\global\theorem@bodyfont{}
%    \end{macrocode}
% \end{macro}
% \end{macro}
%
%
% \begin{macro}{\theoremheaderfont}
% \changes{v2.0g}{89/09/19}{Macro added.}
%    The font for the theorem headers is handled differently because
%    this definition applies to all theorem styles.
%    \begin{macrocode}
\gdef\theoremheaderfont#1{\gdef\theorem@headerfont{#1}%
%    \end{macrocode}
%    After using the macro once it is redefined to produce an error
%    message.
%    \begin{macrocode}
       \gdef\theoremheaderfont##1{%
        \typeout{\string\theoremheaderfont\space should be used
                 only once.}}}
%    \end{macrocode}
% \end{macro}
%
% \begin{macro}{\theorem@headerfont}
% \changes{v2.0g}{89/09/19}{Macro added.}
%    To set the |\theorem@headerfont| default we first test if
%    the new fontselection scheme is in force.
% \changes{v2.1a}{90/02/15}{Support for new font selection.}
%    \begin{macrocode}
\ifx\upshape\undefined
%    \end{macrocode}
%    If not we define it to expand into |\bfseries|. We don't use
%    |\let| just in case a following style option redefines this
%    macro.
%    \begin{macrocode}
\gdef\theorem@headerfont{\bfseries}
%    \end{macrocode}
%    Otherwise we reset the current shape before calling |\bfseries|.
%    \begin{macrocode}
\else \gdef\theorem@headerfont{\normalfont\bfseries}\fi
%    \end{macrocode}
% \end{macro}
%
% \begin{macro}{\th@plain}
% \begin{macro}{\th@break}
% \begin{macro}{\th@marginbreak}
% \begin{macro}{\th@changebreak}
% \begin{macro}{\th@change}
% \begin{macro}{\th@margin}
%    The different styles are defined in macros such as
%    |\th@plain|.  Since memory space is precious in
%    ``non-Big-versions'', we have to avoid offering too many unused
%    definitions.  Therefore we define these styles in separate files
%    that can be loaded on demand.  Thus the commands themselves only
%    load these files. We use |\@input@|, a \LaTeXe{} internal command
%    that ensures that the file will be listed with |\listfiles|.
% \changes{v2.2a}{94/02/02}{Use `@input@ to load theorem layout files}
%    \begin{macrocode}
\gdef\th@plain{\@input@{thp.sty}}
\gdef\th@break{\@input@{thb.sty}}
\gdef\th@marginbreak{\@input@{thmb.sty}}
\gdef\th@changebreak{\@input@{thcb.sty}}
\gdef\th@change{\@input@{thc.sty}}
\gdef\th@margin{\@input@{thm.sty}}
%    \end{macrocode}
%    This list will be expanded when new styles become
%    available.  For testing,  just append new theorem substyles
%    as document options.
% \end{macro}
% \end{macro}
% \end{macro}
% \end{macro}
% \end{macro}
% \end{macro}
%
%
%
% \subsection{Definition of a new theorem set}
%
% As already pointed out, a new theorem environment can be defined in
% three different ways:
% \begin{verbatim}
%    \newtheorem{Lem}{Lemma}
%    \newtheorem{Lem}{Lemma}[section]
%    \newtheorem{Lem}[Theorem]{Lemma}
%\end{verbatim}
% The function of the macro |\newtheorem| is to recognize these cases
% and then to branch into one of the three macros |\@ynthm|, |\@xnthm|
% or |\@othm|.  This mechanism is adopted unchanged from
% \cite{latex:source}; the essential point here is that, for example,
% in the second case, the arguments \texttt{Lem}, \texttt{Lemma} and
% \texttt{section} are passed over to the macro |\@xnthm|.
%
% We inspect this case first because the others present fewer problems,
% and thus are easily derived from this one.
%
% \begin{macro}{\@xnthm}
%    For our example arguments, the macro |\@xnthm| must fulfill
%    the following:
%    \begin{itemize}
%       \item
%          Define a new \LaTeX-counter `Lem'
%       \item
%          reset this counter within a |\section|
%       \item
%          define the macro |\theLem|
%       \item
%          define the environment macros |\Lem| and
%          |\endLem| using the current
%          |\theoremstyle| and |\theorem@bodyfont|.
%    \end{itemize}
%    Obviously, all this should happen only if the first argument
%    of |\@xnthm| (i.e.\ \texttt{Lem} in our example) is chosen so
%    as not to conflict with any previously defined commands or
%    environments.  This test is performed by the \LaTeX{} macro
%    |\@ifdefinable|.
%    \begin{macrocode}
\gdef\@xnthm#1#2[#3]{\expandafter\@ifdefinable\csname #1\endcsname
%    \end{macrocode}
%    Therefore, the first argument of |\@ifdefinable| is the
%    expansion (in the example, |\Lem|) of
%    |\csname||#1\endcsname|.  The second argument is
%    executed only if the test has been completed successfully.
%    \begin{macrocode}
   {%
%    \end{macrocode}
%    Now we define the new counter. The names of the \LaTeX{}
%    macros employed should speak for themselves:
% \changes{v2.2b}{1995/10/07}{Use \cs{@newctr}}
% \changes{v2.2c}{1995/11/23}{Use \cs{@newctr} properly.}
%    \begin{macrocode}
    \@definecounter{#1}\@newctr{#1}[#3]%
%    \end{macrocode}
%    Using |\@newctr| will give a proper error message if the counter
%    in |#3| is not defined.
%    In defining `|\theLem|' we must generate the desired macro
%    name by use of |\expandafter| and |\csname|.
%    \begin{macrocode}
    \expandafter\xdef\csname the#1\endcsname
%    \end{macrocode}
%    An |\xdef| is used in order to make the definition global,
%    and to ensure that it contains the replacement texts of
%    |\@thmcountersep| and |\@thmcounter|.\footnote{These
%    two macros can be defined by the document style.  Their default
%    values produce a `\texttt{.}' as separation and an arabic
%    representation of the number.} However, not everything should be
%    expanded. For example, it saves space to use |\thesection|
%    instead of its---at times---lengthy expansion.
%    \begin{macrocode}
      {\expandafter \noexpand \csname the#3\endcsname
       \@thmcountersep \@thmcounter{#1}}%
%    \end{macrocode}
%    Thus with the defaults of \LaTeX{}, |\theLem| would be
%    replaced by the command sequence |\thesection.\arabic{Lem}|.
%
%    We will now look at the definition of the macro which is executed
%    at the beginning of the actual environment (in our example this
%    macro is |\Lem|). It should be noted that we use an
%    ``|\expandafter| trick'' to expand only certain parts of the
%    replacement text at the time of the definition.
%    \begin{macrocode}
    \def\@tempa{\global\@namedef{#1}}%
    \expandafter \@tempa \expandafter{%
%    \end{macrocode}
%    First, the macro that contains the current definitions of
%    |\@begintheorem| and |\@opargtheorem| should be called
%    up.  The name of this macro---as is already known---has the form
%    |\th@|\meta{theorem\ style}; therefore, it must be called by
%    \begin{macrocode}
      \csname th@\the \theorem@style
%    \end{macrocode}
%    In addition the default theorem font should be changeable, i.e.\
%    we have to insert the contents of |\theorem@bodyfont|. For
%    that reason, we expand even further, beyond |\endcsname|,
%    and thus insert the contents of the \textsf{token} register
%    |\theorem@bodyfont| in the replacement text.
%    \begin{macrocode}
            \expandafter \endcsname \the \theorem@bodyfont
%    \end{macrocode}
%    Now it is time to call the macro |\@thm| which takes over
%    the further processing.  It has two arguments: the current
%    counter name (in our example, \texttt{Lem}), and the text of the
%    label (in our example, \texttt{Lemma}).
%    \begin{macrocode}
     \@thm{#1}{#2}}%
%    \end{macrocode}
%    With this, the `sub-definition' is complete.  The macro
%    |\@endtheorem| ends a theorem environment and is, so far, nothing
%    but an |\endtrivlist|.  (Hence it is defined globally, and not
%    within the theorem styles.\footnote{This has to be changed as
%    soon as theorem styles that change \texttt{\bslash @endtheorem}
%    exist.  In such a case, all existing styles must be changed as
%    well since they will have to reset the macro.}) Therefore, we can
%    set it equivalent to the macro that ends the theorem set (in our
%    example, |\endLem|).  However, if some day theorem styles exist
%    that do change |\@endtheorem|, we would have to use the
%    commented-out line instead.
%    \begin{macrocode}
    \global \expandafter \let \csname end#1\endcsname \@endtheorem
%   \global\@namedef{end#1}{\@endtheorem}%
%    \end{macrocode}
%    With these commands all the required definitions are employed,
%    unless the test |\@ifdefinable| has failed.  Therefore, we
%    end the second argument of this macro and with it the definition
%    of |\@xnthm|.
%    \begin{macrocode}
   }}
%    \end{macrocode}
% \end{macro}
%
% \begin{macro}{\@ynthm}
%    The definition of |\@ynthm| is completely analogous. In
%    this case the new counter that is defined is not reset within
%    another counter; thus the definition of |\the...|
%    is simplified:
%    \begin{macrocode}
\gdef\@ynthm#1#2{\expandafter\@ifdefinable\csname #1\endcsname
   {\@definecounter{#1}%
    \expandafter\xdef\csname the#1\endcsname{\@thmcounter{#1}}%
%    \end{macrocode}
%    The rest of the definition corresponds literally to that of
%    |\@xnthm|:
%    \begin{macrocode}
    \def\@tempa{\global\@namedef{#1}}\expandafter \@tempa
     \expandafter{\csname th@\the \theorem@style \expandafter
     \endcsname \the\theorem@bodyfont \@thm{#1}{#2}}%
    \global \expandafter \let \csname end#1\endcsname \@endtheorem}}
%    \end{macrocode}
% \end{macro}
%
% \begin{macro}{\@othm}
%    The definition of |\@othm| does not contain anything new.
%    \begin{macrocode}
\gdef\@othm#1[#2]#3{%
%    \end{macrocode}
% \changes{v2.2b}{95/10/07}{Build in check for counter defined}
%    We do not define a new counter but instead use one that has
%    already been defined.  Thus the only definition we need is that
%    of this pseudo-counter (i.e.\ |\the|\meta{env.\ name}).
%    First we check if |#2| corresponds to a known counter name.
%    \begin{macrocode}
  \expandafter\ifx\csname c@#2\endcsname\relax
   \@nocounterr{#2}%
  \else
   \expandafter\@ifdefinable\csname #1\endcsname
   {\expandafter \xdef \csname the#1\endcsname
     {\expandafter \noexpand \csname the#2\endcsname}%
%    \end{macrocode}
%    All other parts of the definition can be adopted from
%    |\@xnthm|. We have to remember, though, that in this case
%    the name of the current counter and the theorem label have moved
%    to the second and third arguments.
%    \begin{macrocode}
    \def\@tempa{\global\@namedef{#1}}\expandafter \@tempa
     \expandafter{\csname th@\the \theorem@style \expandafter
     \endcsname \the\theorem@bodyfont \@thm{#2}{#3}}%
    \global \expandafter \let \csname end#1\endcsname \@endtheorem}%
  \fi}
%    \end{macrocode}
% \end{macro}
%
% \subsection{Macros that are employed in a {\envfont theorem}
%    environment}
%
% \begin{macro}{\@thm}
%    The macro |\@thm| has to increase the current counter.  Then,
%    depending on whether the environment has (or does not have) an
%    optional argument, it has to branch into either |\@begintheorem|
%    or |\@opargtheorem|.
%    \begin{macrocode}
\gdef\@thm#1#2{\refstepcounter{#1}%
%    \end{macrocode}
%    Now we start a {\envfont trivlist} environment, and give
%    |\@topsep| and |\@topsepadd| the values of the \textsf{skip}
%    registers |\theorempreskipamount| and |\theorempostskipamount|.
%    The value in |\@topsep| is the vertical space that is inserted by
%    the first (and only) |\item| in our |\trivlist| whilst
%    |\@topsepadd| is inserted by |\@endparenv| at the end of that
%    {\envfont trivlist} environment.  By using these registers, we
%    obtain the desired space around a {\envfont theorem} environment.
%    \begin{macrocode}
   \trivlist
   \@topsep \theorempreskipamount               % used by first \item
   \@topsepadd \theorempostskipamount           % used by \@endparenv
%    \end{macrocode}
%    Now we have to test whether an optional argument has been given.
% \changes{v2.0d}{89/07/18}{Forgotten percent added.}
%    \begin{macrocode}
   \@ifnextchar [%
%    \end{macrocode}
%    If there is an optional argument, we will call |\@ythm|, and
%    move the arguments read back into the input stream.
%    \begin{macrocode}
   {\@ythm{#1}{#2}}%
%    \end{macrocode}
%    If not, we call |\@begintheorem|. Its first argument is the
%    name of the theorem set (hence the second argument of
%    |\@thm|). Its second argument is the macro that produces the
%    current number.
%    \begin{macrocode}
   {\@begintheorem{#2}{\csname the#1\endcsname}\ignorespaces}}
%    \end{macrocode}
% \end{macro}
%
% \begin{macro}{\@xthm}
% \begin{macro}{\@ythm}
%    Both these macros were originally called by |\@thm|.  We do
%    not need |\@xthm| anymore, hence we reset it to
%    |\relax|. The definition of |\@ythm| has not changed at
%    all from its definition in \LaTeX. In order to make the macros
%    easier to understand, we will
%    nevertheless present it (commented out).
%    \begin{macrocode}
\global\let\@xthm\relax
% \def\@ythm#1#2[#3]{\@opargbegintheorem{#2}{\csname
%           the#1\endcsname}{#3}\ignorespaces}
%</package>
%    \end{macrocode}
%    The primitive |\ignorespaces| in |\@ythm| and
%    |\@thm| is needed to remove the spaces between the
%    |\begin{...}| and the actual text.
% \end{macro}
% \end{macro}
%
%
% \subsection{Definition of the theorem substyles}
%
% As already pointed out, the theorem substyles, defined below, are
% only loaded when necessary.  Note that all these substyles, except
% \texttt{plain}, have |\slshape| as the default body font.
%
% \subsubsection{The \texttt{plain} style}
%
% \label{dok:th@plain} As the following macros use |@|, we have
% to locally set the |\catcode| of this symbol to ``letter''.
% This happens within a group, so that we do not have to worry about
% which |\catcode| that symbol had before.
%    \begin{macrocode}
%<*thp>
\begingroup \makeatletter
%    \end{macrocode}
% Since we are now within a group, we must make all definitions
% globally.  First we make sure that \texttt{theorem.sty} is loaded.
% This will allow us to use this file as a document style option
% without having to call \texttt{theorem} itself as an option.  At the
% same time, we assure that at least version 2 is loaded, since
% |\theorem@style| was not defined in earlier versions.
%    \begin{macrocode}
\@ifundefined{theorem@style}{% \iffalse meta-comment
%
% Copyright (C) 1993-2024
%
% The LaTeX Project and any individual authors listed elsewhere
% in this file.
%
% This file is part of the Standard LaTeX `Tools Bundle'.
% -------------------------------------------------------
%
% It may be distributed and/or modified under the
% conditions of the LaTeX Project Public License, either version 1.3c
% of this license or (at your option) any later version.
% The latest version of this license is in
%    https://www.latex-project.org/lppl.txt
% and version 1.3c or later is part of all distributions of LaTeX
% version 2005/12/01 or later.
%
% The list of all files belonging to the LaTeX `Tools Bundle' is
% given in the file `manifest.txt'.
%
% \fi
%
%%
%
%
\def\FMithmInfo{2023/07/05 v2.2c Theorem extension package (FMi)}
%
% \ProvidesFile{theorem.dtx}[\FMithmInfo]
%\iffalse   % this is a METACOMMENT !
%%
%% Package `theorem' to use with LaTeX2e
%% Copyright (C) 1989-1995 Frank Mittelbach, all rights reserved.
%<+package>\@ifundefined{theorem@style}{}{\endinput}
%<+package>\NeedsTeXFormat{LaTeX2e}[1995/06/01]
%<+package>\ProvidesPackage{theorem}
%<+package>         [\FMithmInfo]
%
% \fi
%
% \changes{v1.0}{87/05/07}{First draft.}
% \changes{v1.0a}{87/05/07}{`thistheoremstyle' added.}
% \changes{v1.1a}{87/07/29}{New concept: redefine the internal
%                          \protect\LaTeX{} commands `@xnthm' etc.}
% \changes{v1.1b}{87/07/31}{`theoremfont' used to specify font.}
% \changes{v1.2a}{88/01/08}{`@endtheorem' now globally defined.}
% \changes{v1.2b}{88/01/11}{`newtheorem' only allowed in the preamble.}
% \changes{v1.3a}{88/02/14}{`theoremfont' now a toks register.}
%
% \changes{v2.0a}{88/11/16}{New implementation and new documentation.}
% \changes{v2.0a}{88/11/16}{No layout changes but internal modifications
%                           old substyles won't work.}
% \changes{v2.0b}{89/01/30}{`global' used for all definitions.}
% \changes{v2.0c}{89/06/15}{Documentation translated to English.
%                            Thanks to LLL and Nina}
% \changes{v2.1a}{90/02/15}{Support for new font selection.}
% \changes{v2.2a}{94/02/03}{Cleanup for LaTeX2e}
%
% \DoNotIndex{\@endparenv,\@ifdefinable,\@ifnextchar,\@ifundefined}
% \DoNotIndex{\@namedef,\@nameuse,\@spaces,\@tempa,\@warning}
% \DoNotIndex{\ }
% \DoNotIndex{\begingroup,\bfseries}
% \DoNotIndex{\csname}
% \DoNotIndex{\def,\do,\docdate}
% \DoNotIndex{\endcsname,\endgroup,\endinput,\endtrivlist,\expandafter}
% \DoNotIndex{\filedate,\fileversion}
% \DoNotIndex{\gdef,\global}
% \DoNotIndex{\hbox,\hskip}
% \DoNotIndex{\ignorespaces,\input,\itshape,\item}
% \DoNotIndex{\labelsep,\let,\llap}
% \DoNotIndex{\makeatletter}
% \DoNotIndex{\noexpand}
% \DoNotIndex{\refstepcounter,\relax,\rlap}
% \DoNotIndex{\setlength,\slshape,\space,\string,\strut}
% \DoNotIndex{\the,\trivlist,\typeout}
% \DoNotIndex{\vbox}
% \DoNotIndex{\xdef}
%
% \def\deflabel#1{\ttfamily #1\hfill}
% \def\deflist#1{\begin{list}{}{\settowidth\labelwidth{\ttfamily #1}%
%                               \setlength\leftmargin\labelwidth
%                               \addtolength\leftmargin\labelsep
%                               \let\makelabel\deflabel}}
% \def\enddeflist{\end{list}}
% \def\envfont{\normalfont\ttfamily}
%
% \MakeShortVerb{\|}
%
% \newcommand{\TUB}{{\sl TUGboat\/}}
%
% \GetFileInfo{theorem.dtx}
% \title{An Extension of the
%        \LaTeX{} theorem environment\thanks{This file
%        has version number \fileversion, last
%        revised \filedate.}}
% \author{Frank Mittelbach}
%
% \MaintainedByLaTeXTeam{tools}
% \maketitle
%
% \begin{abstract}
%    The macros described in this paper yield an extension of the
%    \LaTeX{} theorem mechanism. It is designed to satisfy the
%    different requirements of various journals. Thus, the
%    layout of the ``theorems'' can be manipulated by determining a
%    ``style''. This article describes not only the use, but
%    also the definition, of the necessary macros.
% \end{abstract}
%
% \section*{Preface to version 2.2}
%
% For \LaTeXe{} this package did not need any fundamental changes. I
% only modified the messages generated so that theorem layout styles
% will show up with the |\listfiles| command and cleaned the section
% on the New Font Selection Scheme since this is now included in
% \LaTeX.
%
% \section*{Preface to version 2.1}
%
% This version is identical to 2.0g described in \TUB\ 10\#3 except
% for some internal defaults which are now set depending on the used
% font selection scheme.
%
% This was done to avoid unpleasant surprises if the new font selection
% scheme is in force. For further details see
% section~\ref{sec:newfonts} and \cite{bk:GMS94}.
%
% \section{Introduction}
%
% For our purposes here, ``theorems'' are labelled enunciations,
% often set off from the main text by extra space and a font change.
% Theorems, corollaries, conjectures, definitions, and remarks
% are all instances of ``theorems''.  The ``header'' of these structures
% is composed of a label (such as \textsc{Theorem} or \textsc{Remark})
% and a number which serializes an item in the sequence of items
% with the same label.
%
% Shortly after the introduction of \LaTeX{} at the Fachbereich
% Mathematik in Mainz, the desire to manipulate the layout of
% ``theorems'' arose.  In Mainz, the following two conventions came
% into general use:
% \begin{enumerate}
%   \item The number of the theorem is shown in the margin.
%   \item There is a line break at
%         the end of the theorem header.
% \end{enumerate}
%
% Additionally, some journals require different formats which depend on
% the ``sort of theorem'': e.g.\ often remarks and definitions are set
% in |\upshape|, while |\itshape| is employed for main theorems.
%
% Confronted with these requirements, a theorem environment was
% developed in Mainz which allows separate determination of the
% layout of the ``theorems sets'', comparable to |\pagestyle|.
%
%
% \section{The user interface}
% \subsection{Defining new theorem sets}
%
% \DescribeMacro\newtheorem
% As in the original \LaTeX{} version, the command |\newtheorem|
% defines a new ``theorem set'' or ``theorem-like structure''.
% Two required arguments name the new environment and give the text
% to be typeset with each instance of the new ``set'', while
% an optional argument determines how
% the ``set'' is enumerated:
% \begin{description}
%    \item[\ttfamily \bslash newtheorem\{foo\}\{bar\}]
%       The theorem set {\envfont foo} (whose name is \texttt{bar})
%       uses its own counter.
%    \item[\ttfamily \bslash newtheorem\{foo2\}{[foo]}\{bar2\}]
%       The theorem set {\envfont foo2} (printed name \texttt{bar2})
%       uses the same counter as the theorem set \texttt{foo}.
%    \item[\ttfamily \bslash newtheorem\{foo3\}\{bar3\}{[section]}]
%       The theorem set {\envfont foo3} (printed name \texttt{bar3}) is
%       enumerated within the counter \texttt{section}, i.e.\ with every
%       new |\section| the enumeration begins again with 1, and
%       the enumeration is composed from the section-number and the
%       theorem counter itself.
% \end{description}
%
% \DescribeMacro\theoremstyle
% Additionally, the command |\theoremstyle| can define the layout
% of various, or all, theorem sets.  It should be noted that any
% theorem set defined by |\newtheorem| is typeset in the
% |\theoremstyle| that is current at the time of the definition.
% Thus, the following
% \begin{verbatim}
%    \theoremstyle{break}        \newtheorem{Cor}{Corollary}
%    \theoremstyle{plain}        \newtheorem{Exa}{Example}[section]
%\end{verbatim}                             ^^A No blank at line start
% leads to the result that the set {\envfont Cor} is formatted in the
% style \texttt{break}, while the set {\envfont Exa} and all the
% following ones are formatted in the style \texttt{plain}, unless
% another |\theoremstyle| follows.  Since the definitions installed by
% |\newtheorem| are global, one also can limit |\theoremstyle| locally
% by grouping braces.
%
% \DescribeMacro\theorembodyfont
% The choice of the font for the theorem body is completely
% independent of the chosen |\theoremstyle|; this has proven to
% be very advantageous.  For example,
% \begin{verbatim}
%   {\theorembodyfont{\upshape}          \newtheorem{Rem}{Remark}}
%\end{verbatim}
% defines a theorem set {\envfont Rem}, which will be set in
% |\upshape| in the current layout (which in our example is
% \texttt{plain}).  As with |\theoremstyle|, the |\theorembodyfont|
% chosen is that current at the time of |\newtheorem|.  If
% |\theorembodyfont| is not specified or one defines
% |\theorembodyfont{}|, then the font used will be that defined by the
% |\theoremstyle|.
%
% \DescribeMacro\theoremheaderfont It is also possible to customize
% the font used for the theorem headers.  This is, however, a global
% declaration, and therefore there should be at most one
% |\theoremheaderfont| declaration in the preamble.\footnote{If it is
% actually necessary to have different header fonts, one has to define
% new theorem styles (substituting the desired font) or specify the
% information directly in the \texttt{\bslash newtheorem} declaration
% (the unclean variant).}
%
% \DescribeMacro\theorempreskipamount
% \DescribeMacro\theorempostskipamount Two additional parameters
% affect the vertical space around the theorem environments:
% |\theorempreskipamount| and |\theorempostskipamount| define,
% respectively, the spacing before and after such an environment.
% These parameters apply for all theorem sets and can be manipulated
% with the ordinary length macros.  They are rubber lengths
% (`\textsf{skips}'), and therefore can contain \texttt{plus} and
% \texttt{minus} parts.
%
% Since the definition of theorem sets should---most sensibly---be
% placed in the preamble, we only allow installation there.
% It is therefore possible to release the memory used here after
% |\begin{document}|, in order to make room for other
% applications.
%
%
% \subsection{Existing theorem styles}
%
% The following theorem styles exist to date:
% \begin{deflist}{marginbreak}
%    \item[plain]
%       This theorem style emulates the original \LaTeX{} definition,
%       except that additionally the parameters
%       |\theorem...skipamount| are used.
%    \item[break]
%       In this style, the theorem header is followed by a line
%       break.
%    \item[marginbreak]
%       The theorem number is set in the margin, and there is a line
%       break as in \texttt{break}.
%    \item[changebreak]
%       Like \texttt{break}, but with header number and text
%       interchanged.
%    \item[change]
%       Header number and text are interchanged, without a line break.
%    \item[margin]
%       The number is set in the left margin, without a line break.
% \end{deflist}
% All styles (except \texttt{plain}) select |\slshape| as the default
% |\theorembodyfont|.
%
%
% \subsection{Examples}
%
% Given the above theorem sets {\envfont Cor}, {\envfont Exa}
% and {\envfont Rem},
% suppose that the preamble also contains the declarations:
% \begin{verbatim}
%    \theoremstyle{marginbreak}   \newtheorem{Lem}[Cor]{Lemma}
%    \theoremstyle{change}
%    \theorembodyfont{\itshape}        \newtheorem{Def}[Cor]{Definition}
%
%    \theoremheaderfont{\scshape}
%\end{verbatim}
% Then the following are some typical examples of the typeset output
% resulting from their use.
%
% \begin{Cor}
%    This is a sentence typeset in the theorem environment {\envfont
%    Cor}.
% \end{Cor}
%
% \begin{Exa}
%    This is a sentence typeset in the theorem environment {\envfont
%    Exa}.
% \end{Exa}
%
% \begin{Rem}
%    This is a sentence typeset in the theorem environment {\envfont
%    Rem}.
% \end{Rem}
%
% \begin{Lem}[Ben User]
%    This is a sentence typeset in the theorem environment {\envfont
%    Lem}.
% \end{Lem}
%
% \begin{Def}[Very impressive Definition]
%    This is a sentence typeset in the theorem environment {\envfont
%    Def}.
% \end{Def}
%
% The last two examples show the effect of the optional argument to
% a theorem environment (it is the text typeset in parentheses).
%
%
% \section{Special Considerations}
% \label{sec:newfonts}
%
% Theoremheader and body are implemented as a unit. This means that
% the |\theoremheaderfont|  will inherit characteristics of the
% |\theorembodyfont| in \LaTeXe. Thus, if for example
% |\theorembodyfont| is |\itshape| and |\theoremheaderfont|
% is |\bfseries| the font selected for the header will have the
% characteristics `bold extended italic'. If this is not desired one
% should set the |\theoremheaderfont| to something like
% \begin{verbatim}
%      \theoremheaderfont{\normalfont\bfseries}
%\end{verbatim}
% i.e.\ supplying all necessary font information explicitly.
%
%
%
% \section{Acknowledgements}
%
% The publication of this set of macros was only possible with the
% help of Christina Busse (translating the manuscript into English),
% Joachim Pense (playing the r\^ole of typist), Chris Rowley
% (looking everything over) and many others providing useful
% suggestions.
%
% \MaybeStop{%
%    \begin{thebibliography}{9}
%       \bibitem{bk:GMS94} \textsc{M.~Goossens}, \textsc{F.~Mittelbach}
%         and \textsc{A.~Samarin}.
%         \newblock The \LaTeX{} Companion.
%         \newblock
%         Addison-Wesley, Reading, Massachusetts, 1994.
%       \bibitem{latex:source} \textsc{Lamport, Leslie.} \newblock
%                               \texttt{latex.tex}, version 2.09, date
%                               Feb. 1990.
%    \end{thebibliography}
%    \PrintIndex
%    \PrintChanges}
%
%
% \section{The documentation driver file}
%
% The next bit of code contains the documentation driver file for
% \TeX{}, i.e., the file that will produce the documentation you are
% currently reading. It will be extracted from this file by the
% \texttt{docstrip} program. Since it is the first code in the file
% one can alternatively process this file directly with \LaTeXe{} to
% obtain the documentation.
%
%    \begin{macrocode}
%<*driver>
\documentclass{ltxdoc}

\usepackage{theorem}

%<+driver>% The next few lines define theorem sets which are used
%<+driver>% in the example section of the documentation.

\theoremstyle{break}       \newtheorem{Cor}{Corollary}
\theoremstyle{plain}       \newtheorem{Exa}{Example}[section]
{\theorembodyfont{\upshape}\newtheorem{Rem}{Remark}}
\theoremstyle{marginbreak} \newtheorem{Lem}[Cor]{Lemma}
\theoremstyle{change}
\theorembodyfont{\itshape} \newtheorem{Def}[Cor]{Definition}

\theoremheaderfont{\scshape}

\RecordChanges

\begin{document}
   \DocInput{theorem.dtx}
\end{document}
%</driver>
%    \end{macrocode}
%
% \section{Definition of the Macros}
%
% If the file has been loaded before, we abort immediately. If not the
% package announces itself (this is actually done at the very top of
% the file---the way it is done isn't good style so don't copy it).
% \changes{v2.0e}{89/07/19}{Spaces removed from `typeout'.}
% \changes{v2.2b}{95/11/19}{Announce moved to top of file}
%    \begin{macrocode}
%<*package>
%\@ifundefined{theorem@style}{}{\endinput}
%\def\FMithmInfo{1995/11/19 v2.2b Theorem extension package (FMi)}
%\ProvidesPackage{theorem}[\FMithmInfo]
%    \end{macrocode}
%
%
% \subsection{Definition of theorem styles and fonts}
%
% All the definitions in this file are done globally to allow
% inputting this file inside a group.
%
% \begin{macro}{\theoremstyle}
%    Before a theorem style can be installed, the chosen style must be
%    known. For that reason, we must test to see that
%    |\th@|\meta{style} is known or, more precisely, that it is
%    different from |\relax|. If the style is not known then
%    |\th@plain| is used.
%    \begin{macrocode}
\gdef\theoremstyle#1{%
   \@ifundefined{th@#1}{\@warning
          {Unknown theoremstyle `#1'. Using `plain'}%
          \theorem@style{plain}}%
%    \end{macrocode}
%    We save the theorem style to be used in the token register
%    |\theorem@style|.
%    \begin{macrocode}
      {\theorem@style{#1}}%
%    \end{macrocode}
%    Now we ``evaluate'' the theorem style: this means, we call the
%    macro |\th@|\meta{style} which will activate the relevant
%    definitions which are contained in a separate file.  This is done
%    in a group to suppress changes to the current font.  This could
%    otherwise pose problems together with the new font selection
%    scheme\footnote{When I printed the original article using the new
%    font selection scheme I ended with a document with slanted
%    typefaces (text headings and all) simply because one of the
%    theorem styles used \texttt{\bslash sl} at toplevel.} if the
%    |\th@|\meta{style} is evaluated a second time.
% \changes{v2.1a}{90/02/15}{Group added around `th@..}
% \changes{v2.1a}{90/02/15}{Reseting of `..@begintheorem removed.}
%    \begin{macrocode}
      \begingroup
        \csname th@\the\theorem@style \endcsname
      \endgroup}
%    \end{macrocode}
% \end{macro}
%
%
% \begin{macro}{\@begintheorem}
% \changes{v2.1a}{90/02/15}{Relaxed at toplevel.}
% \begin{macro}{\@opargbegintheorem}
% \changes{v2.1a}{90/02/15}{Relaxed at toplevel.}
%    We reset |\@begintheorem| and |\@opargbegintheorem| to
%    |\relax| since these commands are no longer necessary at
%    toplevel. This will save a few tokens.
%    \begin{macrocode}
\global\let\@begintheorem\relax
\global\let\@opargbegintheorem\relax
%    \end{macrocode}
% \end{macro}
% \end{macro}
%
% \begin{macro}{\theorem@style}
%    Obviously the \textsf{token} register used above has to be
%    allocated. To assure the utmost compatibility with the
%    original \LaTeX{} definition, we set the default theorem style to
%    \texttt{plain}, which implements the usual \LaTeX{} convention.
% \changes{v2.0f}{89/07/20}{Unnecessary `global' removed.}
%    \begin{macrocode}
\newtoks\theorem@style
\global\theorem@style{plain}
%    \end{macrocode}
% \end{macro}
%
% \begin{macro}{\theorembodyfont}
% \begin{macro}{\theorem@bodyfont}
%    If the |\theorembodyfont| is set by the user then it should not
%    interact with the default font set in the theorem style. When the
%    new font selection is in force this may happen if, for example,
%    the default is |\itshape| and the new |\theorembodyfont| is
%    |\sffamily|. So we add a |\reset@font| command in front of the
%    user definition.
% \changes{v2.0f}{89/07/20}{Unnecessary `global' removed.}
% \changes{v2.0g}{89/09/19}{`theoremfont to `theorembodyfont.}
% \changes{v2.1c}{92/06/22}{`reset@font added and logic changed.}
% \changes{v2.1d}{92/10/28}{Reset to empty if arg is empty}
%    \begin{macrocode}
\gdef\theorembodyfont#1{%
%    \end{macrocode}
%    We check if the argument supplied is empty and if so put nothing
%    into the |\theorem@bodyfont| token register to allow for
%    |\theorembodyfont{}| as a mean of using the default of the
%    current |\theoremstyle|.
%    \begin{macrocode}
   \def\@tempa{#1}%
   \ifx\@tempa\@empty
    \theorem@bodyfont{}%
   \else
    \theorem@bodyfont{\reset@font#1}%
   \fi
   }
\newtoks\theorem@bodyfont
\global\theorem@bodyfont{}
%    \end{macrocode}
% \end{macro}
% \end{macro}
%
%
% \begin{macro}{\theoremheaderfont}
% \changes{v2.0g}{89/09/19}{Macro added.}
%    The font for the theorem headers is handled differently because
%    this definition applies to all theorem styles.
%    \begin{macrocode}
\gdef\theoremheaderfont#1{\gdef\theorem@headerfont{#1}%
%    \end{macrocode}
%    After using the macro once it is redefined to produce an error
%    message.
%    \begin{macrocode}
       \gdef\theoremheaderfont##1{%
        \typeout{\string\theoremheaderfont\space should be used
                 only once.}}}
%    \end{macrocode}
% \end{macro}
%
% \begin{macro}{\theorem@headerfont}
% \changes{v2.0g}{89/09/19}{Macro added.}
%    To set the |\theorem@headerfont| default we first test if
%    the new fontselection scheme is in force.
% \changes{v2.1a}{90/02/15}{Support for new font selection.}
%    \begin{macrocode}
\ifx\upshape\undefined
%    \end{macrocode}
%    If not we define it to expand into |\bfseries|. We don't use
%    |\let| just in case a following style option redefines this
%    macro.
%    \begin{macrocode}
\gdef\theorem@headerfont{\bfseries}
%    \end{macrocode}
%    Otherwise we reset the current shape before calling |\bfseries|.
%    \begin{macrocode}
\else \gdef\theorem@headerfont{\normalfont\bfseries}\fi
%    \end{macrocode}
% \end{macro}
%
% \begin{macro}{\th@plain}
% \begin{macro}{\th@break}
% \begin{macro}{\th@marginbreak}
% \begin{macro}{\th@changebreak}
% \begin{macro}{\th@change}
% \begin{macro}{\th@margin}
%    The different styles are defined in macros such as
%    |\th@plain|.  Since memory space is precious in
%    ``non-Big-versions'', we have to avoid offering too many unused
%    definitions.  Therefore we define these styles in separate files
%    that can be loaded on demand.  Thus the commands themselves only
%    load these files. We use |\@input@|, a \LaTeXe{} internal command
%    that ensures that the file will be listed with |\listfiles|.
% \changes{v2.2a}{94/02/02}{Use `@input@ to load theorem layout files}
%    \begin{macrocode}
\gdef\th@plain{\@input@{thp.sty}}
\gdef\th@break{\@input@{thb.sty}}
\gdef\th@marginbreak{\@input@{thmb.sty}}
\gdef\th@changebreak{\@input@{thcb.sty}}
\gdef\th@change{\@input@{thc.sty}}
\gdef\th@margin{\@input@{thm.sty}}
%    \end{macrocode}
%    This list will be expanded when new styles become
%    available.  For testing,  just append new theorem substyles
%    as document options.
% \end{macro}
% \end{macro}
% \end{macro}
% \end{macro}
% \end{macro}
% \end{macro}
%
%
%
% \subsection{Definition of a new theorem set}
%
% As already pointed out, a new theorem environment can be defined in
% three different ways:
% \begin{verbatim}
%    \newtheorem{Lem}{Lemma}
%    \newtheorem{Lem}{Lemma}[section]
%    \newtheorem{Lem}[Theorem]{Lemma}
%\end{verbatim}
% The function of the macro |\newtheorem| is to recognize these cases
% and then to branch into one of the three macros |\@ynthm|, |\@xnthm|
% or |\@othm|.  This mechanism is adopted unchanged from
% \cite{latex:source}; the essential point here is that, for example,
% in the second case, the arguments \texttt{Lem}, \texttt{Lemma} and
% \texttt{section} are passed over to the macro |\@xnthm|.
%
% We inspect this case first because the others present fewer problems,
% and thus are easily derived from this one.
%
% \begin{macro}{\@xnthm}
%    For our example arguments, the macro |\@xnthm| must fulfill
%    the following:
%    \begin{itemize}
%       \item
%          Define a new \LaTeX-counter `Lem'
%       \item
%          reset this counter within a |\section|
%       \item
%          define the macro |\theLem|
%       \item
%          define the environment macros |\Lem| and
%          |\endLem| using the current
%          |\theoremstyle| and |\theorem@bodyfont|.
%    \end{itemize}
%    Obviously, all this should happen only if the first argument
%    of |\@xnthm| (i.e.\ \texttt{Lem} in our example) is chosen so
%    as not to conflict with any previously defined commands or
%    environments.  This test is performed by the \LaTeX{} macro
%    |\@ifdefinable|.
%    \begin{macrocode}
\gdef\@xnthm#1#2[#3]{\expandafter\@ifdefinable\csname #1\endcsname
%    \end{macrocode}
%    Therefore, the first argument of |\@ifdefinable| is the
%    expansion (in the example, |\Lem|) of
%    |\csname||#1\endcsname|.  The second argument is
%    executed only if the test has been completed successfully.
%    \begin{macrocode}
   {%
%    \end{macrocode}
%    Now we define the new counter. The names of the \LaTeX{}
%    macros employed should speak for themselves:
% \changes{v2.2b}{1995/10/07}{Use \cs{@newctr}}
% \changes{v2.2c}{1995/11/23}{Use \cs{@newctr} properly.}
%    \begin{macrocode}
    \@definecounter{#1}\@newctr{#1}[#3]%
%    \end{macrocode}
%    Using |\@newctr| will give a proper error message if the counter
%    in |#3| is not defined.
%    In defining `|\theLem|' we must generate the desired macro
%    name by use of |\expandafter| and |\csname|.
%    \begin{macrocode}
    \expandafter\xdef\csname the#1\endcsname
%    \end{macrocode}
%    An |\xdef| is used in order to make the definition global,
%    and to ensure that it contains the replacement texts of
%    |\@thmcountersep| and |\@thmcounter|.\footnote{These
%    two macros can be defined by the document style.  Their default
%    values produce a `\texttt{.}' as separation and an arabic
%    representation of the number.} However, not everything should be
%    expanded. For example, it saves space to use |\thesection|
%    instead of its---at times---lengthy expansion.
%    \begin{macrocode}
      {\expandafter \noexpand \csname the#3\endcsname
       \@thmcountersep \@thmcounter{#1}}%
%    \end{macrocode}
%    Thus with the defaults of \LaTeX{}, |\theLem| would be
%    replaced by the command sequence |\thesection.\arabic{Lem}|.
%
%    We will now look at the definition of the macro which is executed
%    at the beginning of the actual environment (in our example this
%    macro is |\Lem|). It should be noted that we use an
%    ``|\expandafter| trick'' to expand only certain parts of the
%    replacement text at the time of the definition.
%    \begin{macrocode}
    \def\@tempa{\global\@namedef{#1}}%
    \expandafter \@tempa \expandafter{%
%    \end{macrocode}
%    First, the macro that contains the current definitions of
%    |\@begintheorem| and |\@opargtheorem| should be called
%    up.  The name of this macro---as is already known---has the form
%    |\th@|\meta{theorem\ style}; therefore, it must be called by
%    \begin{macrocode}
      \csname th@\the \theorem@style
%    \end{macrocode}
%    In addition the default theorem font should be changeable, i.e.\
%    we have to insert the contents of |\theorem@bodyfont|. For
%    that reason, we expand even further, beyond |\endcsname|,
%    and thus insert the contents of the \textsf{token} register
%    |\theorem@bodyfont| in the replacement text.
%    \begin{macrocode}
            \expandafter \endcsname \the \theorem@bodyfont
%    \end{macrocode}
%    Now it is time to call the macro |\@thm| which takes over
%    the further processing.  It has two arguments: the current
%    counter name (in our example, \texttt{Lem}), and the text of the
%    label (in our example, \texttt{Lemma}).
%    \begin{macrocode}
     \@thm{#1}{#2}}%
%    \end{macrocode}
%    With this, the `sub-definition' is complete.  The macro
%    |\@endtheorem| ends a theorem environment and is, so far, nothing
%    but an |\endtrivlist|.  (Hence it is defined globally, and not
%    within the theorem styles.\footnote{This has to be changed as
%    soon as theorem styles that change \texttt{\bslash @endtheorem}
%    exist.  In such a case, all existing styles must be changed as
%    well since they will have to reset the macro.}) Therefore, we can
%    set it equivalent to the macro that ends the theorem set (in our
%    example, |\endLem|).  However, if some day theorem styles exist
%    that do change |\@endtheorem|, we would have to use the
%    commented-out line instead.
%    \begin{macrocode}
    \global \expandafter \let \csname end#1\endcsname \@endtheorem
%   \global\@namedef{end#1}{\@endtheorem}%
%    \end{macrocode}
%    With these commands all the required definitions are employed,
%    unless the test |\@ifdefinable| has failed.  Therefore, we
%    end the second argument of this macro and with it the definition
%    of |\@xnthm|.
%    \begin{macrocode}
   }}
%    \end{macrocode}
% \end{macro}
%
% \begin{macro}{\@ynthm}
%    The definition of |\@ynthm| is completely analogous. In
%    this case the new counter that is defined is not reset within
%    another counter; thus the definition of |\the...|
%    is simplified:
%    \begin{macrocode}
\gdef\@ynthm#1#2{\expandafter\@ifdefinable\csname #1\endcsname
   {\@definecounter{#1}%
    \expandafter\xdef\csname the#1\endcsname{\@thmcounter{#1}}%
%    \end{macrocode}
%    The rest of the definition corresponds literally to that of
%    |\@xnthm|:
%    \begin{macrocode}
    \def\@tempa{\global\@namedef{#1}}\expandafter \@tempa
     \expandafter{\csname th@\the \theorem@style \expandafter
     \endcsname \the\theorem@bodyfont \@thm{#1}{#2}}%
    \global \expandafter \let \csname end#1\endcsname \@endtheorem}}
%    \end{macrocode}
% \end{macro}
%
% \begin{macro}{\@othm}
%    The definition of |\@othm| does not contain anything new.
%    \begin{macrocode}
\gdef\@othm#1[#2]#3{%
%    \end{macrocode}
% \changes{v2.2b}{95/10/07}{Build in check for counter defined}
%    We do not define a new counter but instead use one that has
%    already been defined.  Thus the only definition we need is that
%    of this pseudo-counter (i.e.\ |\the|\meta{env.\ name}).
%    First we check if |#2| corresponds to a known counter name.
%    \begin{macrocode}
  \expandafter\ifx\csname c@#2\endcsname\relax
   \@nocounterr{#2}%
  \else
   \expandafter\@ifdefinable\csname #1\endcsname
   {\expandafter \xdef \csname the#1\endcsname
     {\expandafter \noexpand \csname the#2\endcsname}%
%    \end{macrocode}
%    All other parts of the definition can be adopted from
%    |\@xnthm|. We have to remember, though, that in this case
%    the name of the current counter and the theorem label have moved
%    to the second and third arguments.
%    \begin{macrocode}
    \def\@tempa{\global\@namedef{#1}}\expandafter \@tempa
     \expandafter{\csname th@\the \theorem@style \expandafter
     \endcsname \the\theorem@bodyfont \@thm{#2}{#3}}%
    \global \expandafter \let \csname end#1\endcsname \@endtheorem}%
  \fi}
%    \end{macrocode}
% \end{macro}
%
% \subsection{Macros that are employed in a {\envfont theorem}
%    environment}
%
% \begin{macro}{\@thm}
%    The macro |\@thm| has to increase the current counter.  Then,
%    depending on whether the environment has (or does not have) an
%    optional argument, it has to branch into either |\@begintheorem|
%    or |\@opargtheorem|.
%    \begin{macrocode}
\gdef\@thm#1#2{\refstepcounter{#1}%
%    \end{macrocode}
%    Now we start a {\envfont trivlist} environment, and give
%    |\@topsep| and |\@topsepadd| the values of the \textsf{skip}
%    registers |\theorempreskipamount| and |\theorempostskipamount|.
%    The value in |\@topsep| is the vertical space that is inserted by
%    the first (and only) |\item| in our |\trivlist| whilst
%    |\@topsepadd| is inserted by |\@endparenv| at the end of that
%    {\envfont trivlist} environment.  By using these registers, we
%    obtain the desired space around a {\envfont theorem} environment.
%    \begin{macrocode}
   \trivlist
   \@topsep \theorempreskipamount               % used by first \item
   \@topsepadd \theorempostskipamount           % used by \@endparenv
%    \end{macrocode}
%    Now we have to test whether an optional argument has been given.
% \changes{v2.0d}{89/07/18}{Forgotten percent added.}
%    \begin{macrocode}
   \@ifnextchar [%
%    \end{macrocode}
%    If there is an optional argument, we will call |\@ythm|, and
%    move the arguments read back into the input stream.
%    \begin{macrocode}
   {\@ythm{#1}{#2}}%
%    \end{macrocode}
%    If not, we call |\@begintheorem|. Its first argument is the
%    name of the theorem set (hence the second argument of
%    |\@thm|). Its second argument is the macro that produces the
%    current number.
%    \begin{macrocode}
   {\@begintheorem{#2}{\csname the#1\endcsname}\ignorespaces}}
%    \end{macrocode}
% \end{macro}
%
% \begin{macro}{\@xthm}
% \begin{macro}{\@ythm}
%    Both these macros were originally called by |\@thm|.  We do
%    not need |\@xthm| anymore, hence we reset it to
%    |\relax|. The definition of |\@ythm| has not changed at
%    all from its definition in \LaTeX. In order to make the macros
%    easier to understand, we will
%    nevertheless present it (commented out).
%    \begin{macrocode}
\global\let\@xthm\relax
% \def\@ythm#1#2[#3]{\@opargbegintheorem{#2}{\csname
%           the#1\endcsname}{#3}\ignorespaces}
%</package>
%    \end{macrocode}
%    The primitive |\ignorespaces| in |\@ythm| and
%    |\@thm| is needed to remove the spaces between the
%    |\begin{...}| and the actual text.
% \end{macro}
% \end{macro}
%
%
% \subsection{Definition of the theorem substyles}
%
% As already pointed out, the theorem substyles, defined below, are
% only loaded when necessary.  Note that all these substyles, except
% \texttt{plain}, have |\slshape| as the default body font.
%
% \subsubsection{The \texttt{plain} style}
%
% \label{dok:th@plain} As the following macros use |@|, we have
% to locally set the |\catcode| of this symbol to ``letter''.
% This happens within a group, so that we do not have to worry about
% which |\catcode| that symbol had before.
%    \begin{macrocode}
%<*thp>
\begingroup \makeatletter
%    \end{macrocode}
% Since we are now within a group, we must make all definitions
% globally.  First we make sure that \texttt{theorem.sty} is loaded.
% This will allow us to use this file as a document style option
% without having to call \texttt{theorem} itself as an option.  At the
% same time, we assure that at least version 2 is loaded, since
% |\theorem@style| was not defined in earlier versions.
%    \begin{macrocode}
\@ifundefined{theorem@style}{\input{theorem.sty}}{}
\ProvidesFile{thp.sty}
             [\FMithmInfo]
%    \end{macrocode}
%
%
% \begin{macro}{\th@plain}
% \begin{macro}{\@begintheorem}
% \begin{macro}{\@opargbegintheorem}
%    |\theoremstyle{plain}| corresponds to the original
%    definition, except that the distances to the surrounding text are
%    determined by the parameters |\theorempreskipamount| and
%    |\theorempostskipamount|.
%    First we set the default body font.
%    \begin{macrocode}
\gdef\th@plain{\normalfont\itshape
%    \end{macrocode}
%    Then we define |\@begintheorem| and
%    |\@opargbegintheorem|. These two macros define how the
%    header of a theorem is typeset.  |\@opargbegintheorem| will
%    be called if a \textsf{theorem} environment with an optional
%    argument is encountered; otherwise, the header is constructed by
%    calling |\@begintheorem|.  If one of these macros is
%    executed, we are within a \textsf{trivlist} environment started by
%    |\@thm|.  So the theorem header is produced with an
%    |\item| command.
%
%    Instead of specifying the header font directly, all standard
%    theorem styles use the |\theorem@headerfont| macro to allow
%    customization.  The extra space (|\labelsep|) is necessary
%    because of problems in the \textsf{trivlist} environment.
%    \begin{macrocode}
  \def\@begintheorem##1##2{%
        \item[\hskip\labelsep \theorem@headerfont ##1\ ##2]}%
%    \end{macrocode}
%    The definition of |\@opargbegintheorem| is completely
%    analogous. The only difference is the fact that there exists a
%    third argument (which is the optional parameter of the
%    environment and contains additional information about the
%    theorem).  Customarily we enclose it in parentheses.
%    \begin{macrocode}
\def\@opargbegintheorem##1##2##3{%
   \item[\hskip\labelsep \theorem@headerfont ##1\ ##2\ (##3)]}}
%    \end{macrocode}
% \end{macro}
% \end{macro}
% \end{macro}
%
% \noindent
% We conclude with an |\endgroup| to restore the
% |\catcode| of |@|.
%    \begin{macrocode}
\endgroup
%</thp>
%    \end{macrocode}
%
%
% \subsubsection{The \texttt{break} style}
%
% This style option is stored in the file |thb.sty|.
% For the next two lines see the documentation for |\th@plain| on
% page~\pageref{dok:th@plain}.
%    \begin{macrocode}
%<*thb>
\begingroup \makeatletter
\@ifundefined{theorem@style}{\input{theorem.sty}}{}
\ProvidesFile{thb.sty}
             [\FMithmInfo]
%    \end{macrocode}
%
% \begin{macro}{\th@break}
%    |\theoremstyle{break}| produces a line break after the name
%    of the theorem.  The font is |\slshape|.  Hence, we define
%    |\th@break| as follows:
%    \begin{macrocode}
\gdef\th@break{\normalfont\slshape
  \def\@begintheorem##1##2{\item[%
%    \end{macrocode}
%    We run into the following problem: it is not possible to create
%    the header with |\item[|\meta{title}|]| and then start
%    a new line by, for example, |\mbox{}\\|. Such a definition
%    will fail whenever a list environment follows immediately.  With
%    the above construction, the |\mbox{}| causes the switch
%    |@inlabel| (cf.\ definition of |\list| and
%    |\trivlist| in \cite{latex:source}) to be set to \texttt{false}
%    and so the following list will insert additional vertical
%    space (|\topskip|). This is quite annoying. Therefore, we
%    create the line break within the |\item|. In order to ensure
%    that the text will begin at the proper position in the following
%    line, we simply pretend that the label does not take any
%    room.\footnote{This will lead to problems whenever very high
%    symbols occurring in the line tower into the
%    heading. So, something else has to be done here sometime.}
%    \begin{macrocode}
       \rlap{\vbox{\hbox{\hskip \labelsep\theorem@headerfont ##1\ ##2}%
                   \hbox{\strut}}}]}%
%    \end{macrocode}
%    Again, the definition of |\@opargbegintheorem| is
%    completely analogous.
%    \begin{macrocode}
\def\@opargbegintheorem##1##2##3{%
  \item[\rlap{\vbox{\hbox{\hskip \labelsep \theorem@headerfont
                     ##1\ ##2\ (##3)}%
                    \hbox{\strut}}}]}}
\endgroup
%</thb>
%    \end{macrocode}
% \end{macro}
%
%
% \subsubsection{The \texttt{changebreak} style}
%
%    \begin{macrocode}
% This style option is stored in the file |thcb.sty|.
%    \begin{macrocode}
%<*thcb>
\begingroup \makeatletter
\@ifundefined{theorem@style}{\input{theorem.sty}}{}
\ProvidesFile{thcb.sty}
             [\FMithmInfo]
%    \end{macrocode}
%
% \begin{macro}{\th@changebreak}
%    The \texttt{change-break} theorem style is like \texttt{break}
%    but with interchange of theorem name and theorem number.  Thus we
%    define |\th@changebreak| as follows:
%    \begin{macrocode}
\gdef\th@changebreak{\normalfont\slshape
  \def\@begintheorem##1##2{\item
       [\rlap{\vbox{\hbox{\hskip\labelsep \theorem@headerfont ##2\ ##1}%
                    \hbox{\strut}}}]}%
\def\@opargbegintheorem##1##2##3{%
  \item[\rlap{\vbox{\hbox{\hskip\labelsep \theorem@headerfont
                     ##2\ ##1\ (##3)}%
                    \hbox{\strut}}}]}}
\endgroup
%</thcb>
%    \end{macrocode}
% \end{macro}
%
%
% \subsubsection{The \texttt{change} style}
%
% This style option is stored in the file |thc.sty|.
%    \begin{macrocode}
%<*thc>
\begingroup \makeatletter
\@ifundefined{theorem@style}{\input{theorem.sty}}{}
\ProvidesFile{thc.sty}
             [\FMithmInfo]
%    \end{macrocode}
%
% \begin{macro}{\th@change}
%    The \texttt{change} theorem style corresponds to the
%    \texttt{change break} style without a linebreak after the header.
%    To say it in another way, it's the same as the \texttt{plain}
%    style but with number and name interchanged and |\slshape| as the
%    default font.
%    \begin{macrocode}
\gdef\th@change{\normalfont\slshape
  \def\@begintheorem##1##2{\item
       [\hskip\labelsep \theorem@headerfont ##2\ ##1]}%
\def\@opargbegintheorem##1##2##3{%
  \item[\hskip\labelsep \theorem@headerfont ##2\ ##1\ (##3)]}}
\endgroup
%</thc>
%    \end{macrocode}
% \end{macro}
%
%
% \subsubsection{The \texttt{marginbreak} style}
%
% This style option is the one used most often at Mainz.  It is saved
% in the file |thmb.sty|.
%    \begin{macrocode}
%<*thmb>
\begingroup \makeatletter
\@ifundefined{theorem@style}{\input{theorem.sty}}{}
\ProvidesFile{thmb.sty}
             [\FMithmInfo]
%    \end{macrocode}
%
% \begin{macro}{\th@marginbreak}
%    The \texttt{margin break} style is nearly the same as the
%    \texttt{change break} style. The only difference is the placement
%    of the theorem number.  We use |\llap| to place it in the left
%    margin.
%
%    In this style |\labelsep| denotes the separation between the
%    number and the text.
%    \begin{macrocode}
\gdef\th@marginbreak{\normalfont\slshape
  \def\@begintheorem##1##2{\item
       [\rlap{\vbox{\theorem@headerfont
                    \hbox{\llap{##2}\hskip\labelsep ##1}%
                    \hbox{\strut}}}]}%
\def\@opargbegintheorem##1##2##3{%
  \item[\rlap{\vbox{\theorem@headerfont
                    \hbox{\llap{##2}\hskip\labelsep ##1\ (##3)}%
                    \hbox{\strut}}}]}}
\endgroup
%</thmb>
%    \end{macrocode}
% \changes{v2.1c}{90/02/17}{`endgroup outside macrocode env.}
% \end{macro}
%
%
%
% \subsubsection{The \texttt{margin} style}
%
% This style option is stored in the file |thm.sty|.
%    \begin{macrocode}
%<*thm>
\begingroup \makeatletter
\@ifundefined{theorem@style}{\input{theorem.sty}}{}
\ProvidesFile{thm.sty}
             [\FMithmInfo]
%    \end{macrocode}
%
% \begin{macro}{\th@margin}
%    Again this is only a variant of the theorem styles described
%    above without any new ideas.
%    \begin{macrocode}
\gdef\th@margin{\normalfont\slshape
  \def\@begintheorem##1##2{\item
       [\theorem@headerfont \llap{##2}\hskip\labelsep ##1]}%
\def\@opargbegintheorem##1##2##3{%
  \item[\theorem@headerfont \llap{##2}\hskip\labelsep ##1\ (##3)]}}
\endgroup
%</thm>
%    \end{macrocode}
% \end{macro}
%
%
%
%
% \subsection{Final Definitions}
%
% \begin{macro}{\theorempreskipamount}
% \begin{macro}{\theorempostskipamount}
%    The \textsf{skip} parameters that regulate the vertical empty space
%    before and after the {\envfont theorem} environment have
%    to be allocated as well.
% \changes{v2.0f}{89/07/20}{Unnecessary `global' removed.}
%    \begin{macrocode}
%<*package>
\newskip\theorempreskipamount
\newskip\theorempostskipamount
%    \end{macrocode}
%    Since we have used the same values for all theorem sets, we now can
%    assign them.
%    \begin{macrocode}
\global\setlength\theorempreskipamount{12pt plus 5pt minus 3pt}
\global\setlength\theorempostskipamount{8pt plus 3pt minus 1.5pt}
%    \end{macrocode}
% \end{macro}
% \end{macro}
%
%
% \begin{macro}{\@endtheorem}
%    The same holds for the macro |\@endtheorem|, which ends a
%    {\envfont theorem} environment. Since it is the same for all
%    theorem sets, it is removed from the macros |\th@|\meta{style}.
%    It simply ends the {\envfont trivlist} environment, which was
%    begun in |\@thm|.
%    \begin{macrocode}
\global\let\@endtheorem=\endtrivlist
%    \end{macrocode}
% \end{macro}
%
%
%
% \begin{macro}{\@preamblecmds}
%    All macros defined above are to be used only in the preamble.
%    Therefore, we insert them in |\@preamblecmds| which will disable
%    them at begin document. This is done by the internal \LaTeXe{}
%    command |\@onlypreamble|.
% \changes{v2.0g}{89/09/19}{`theoremheaderfont and `theorembodyfont
%    added.}
% \changes{v2.2a}{94/02/03}{Use \LaTeXe{} interface}
%    \begin{macrocode}
\@onlypreamble\@xnthm
\@onlypreamble\@ynthm
\@onlypreamble\@othm
\@onlypreamble\newtheorem
\@onlypreamble\theoremstyle
\@onlypreamble\theorembodyfont
\@onlypreamble\theoremheaderfont
%    \end{macrocode}
% \end{macro}
%
% Finally we declare the \texttt{plain} theorem style to be the default.
% \changes{v2.1b}{90/06/04}{Fixed missing `item by declaring default
%    style.}
%    \begin{macrocode}
\theoremstyle{plain}
%</package>
%    \end{macrocode}
%
% \Finale

\endinput
}{}
\ProvidesFile{thp.sty}
             [\FMithmInfo]
%    \end{macrocode}
%
%
% \begin{macro}{\th@plain}
% \begin{macro}{\@begintheorem}
% \begin{macro}{\@opargbegintheorem}
%    |\theoremstyle{plain}| corresponds to the original
%    definition, except that the distances to the surrounding text are
%    determined by the parameters |\theorempreskipamount| and
%    |\theorempostskipamount|.
%    First we set the default body font.
%    \begin{macrocode}
\gdef\th@plain{\normalfont\itshape
%    \end{macrocode}
%    Then we define |\@begintheorem| and
%    |\@opargbegintheorem|. These two macros define how the
%    header of a theorem is typeset.  |\@opargbegintheorem| will
%    be called if a \textsf{theorem} environment with an optional
%    argument is encountered; otherwise, the header is constructed by
%    calling |\@begintheorem|.  If one of these macros is
%    executed, we are within a \textsf{trivlist} environment started by
%    |\@thm|.  So the theorem header is produced with an
%    |\item| command.
%
%    Instead of specifying the header font directly, all standard
%    theorem styles use the |\theorem@headerfont| macro to allow
%    customization.  The extra space (|\labelsep|) is necessary
%    because of problems in the \textsf{trivlist} environment.
%    \begin{macrocode}
  \def\@begintheorem##1##2{%
        \item[\hskip\labelsep \theorem@headerfont ##1\ ##2]}%
%    \end{macrocode}
%    The definition of |\@opargbegintheorem| is completely
%    analogous. The only difference is the fact that there exists a
%    third argument (which is the optional parameter of the
%    environment and contains additional information about the
%    theorem).  Customarily we enclose it in parentheses.
%    \begin{macrocode}
\def\@opargbegintheorem##1##2##3{%
   \item[\hskip\labelsep \theorem@headerfont ##1\ ##2\ (##3)]}}
%    \end{macrocode}
% \end{macro}
% \end{macro}
% \end{macro}
%
% \noindent
% We conclude with an |\endgroup| to restore the
% |\catcode| of |@|.
%    \begin{macrocode}
\endgroup
%</thp>
%    \end{macrocode}
%
%
% \subsubsection{The \texttt{break} style}
%
% This style option is stored in the file |thb.sty|.
% For the next two lines see the documentation for |\th@plain| on
% page~\pageref{dok:th@plain}.
%    \begin{macrocode}
%<*thb>
\begingroup \makeatletter
\@ifundefined{theorem@style}{% \iffalse meta-comment
%
% Copyright (C) 1993-2024
%
% The LaTeX Project and any individual authors listed elsewhere
% in this file.
%
% This file is part of the Standard LaTeX `Tools Bundle'.
% -------------------------------------------------------
%
% It may be distributed and/or modified under the
% conditions of the LaTeX Project Public License, either version 1.3c
% of this license or (at your option) any later version.
% The latest version of this license is in
%    https://www.latex-project.org/lppl.txt
% and version 1.3c or later is part of all distributions of LaTeX
% version 2005/12/01 or later.
%
% The list of all files belonging to the LaTeX `Tools Bundle' is
% given in the file `manifest.txt'.
%
% \fi
%
%%
%
%
\def\FMithmInfo{2023/07/05 v2.2c Theorem extension package (FMi)}
%
% \ProvidesFile{theorem.dtx}[\FMithmInfo]
%\iffalse   % this is a METACOMMENT !
%%
%% Package `theorem' to use with LaTeX2e
%% Copyright (C) 1989-1995 Frank Mittelbach, all rights reserved.
%<+package>\@ifundefined{theorem@style}{}{\endinput}
%<+package>\NeedsTeXFormat{LaTeX2e}[1995/06/01]
%<+package>\ProvidesPackage{theorem}
%<+package>         [\FMithmInfo]
%
% \fi
%
% \changes{v1.0}{87/05/07}{First draft.}
% \changes{v1.0a}{87/05/07}{`thistheoremstyle' added.}
% \changes{v1.1a}{87/07/29}{New concept: redefine the internal
%                          \protect\LaTeX{} commands `@xnthm' etc.}
% \changes{v1.1b}{87/07/31}{`theoremfont' used to specify font.}
% \changes{v1.2a}{88/01/08}{`@endtheorem' now globally defined.}
% \changes{v1.2b}{88/01/11}{`newtheorem' only allowed in the preamble.}
% \changes{v1.3a}{88/02/14}{`theoremfont' now a toks register.}
%
% \changes{v2.0a}{88/11/16}{New implementation and new documentation.}
% \changes{v2.0a}{88/11/16}{No layout changes but internal modifications
%                           old substyles won't work.}
% \changes{v2.0b}{89/01/30}{`global' used for all definitions.}
% \changes{v2.0c}{89/06/15}{Documentation translated to English.
%                            Thanks to LLL and Nina}
% \changes{v2.1a}{90/02/15}{Support for new font selection.}
% \changes{v2.2a}{94/02/03}{Cleanup for LaTeX2e}
%
% \DoNotIndex{\@endparenv,\@ifdefinable,\@ifnextchar,\@ifundefined}
% \DoNotIndex{\@namedef,\@nameuse,\@spaces,\@tempa,\@warning}
% \DoNotIndex{\ }
% \DoNotIndex{\begingroup,\bfseries}
% \DoNotIndex{\csname}
% \DoNotIndex{\def,\do,\docdate}
% \DoNotIndex{\endcsname,\endgroup,\endinput,\endtrivlist,\expandafter}
% \DoNotIndex{\filedate,\fileversion}
% \DoNotIndex{\gdef,\global}
% \DoNotIndex{\hbox,\hskip}
% \DoNotIndex{\ignorespaces,\input,\itshape,\item}
% \DoNotIndex{\labelsep,\let,\llap}
% \DoNotIndex{\makeatletter}
% \DoNotIndex{\noexpand}
% \DoNotIndex{\refstepcounter,\relax,\rlap}
% \DoNotIndex{\setlength,\slshape,\space,\string,\strut}
% \DoNotIndex{\the,\trivlist,\typeout}
% \DoNotIndex{\vbox}
% \DoNotIndex{\xdef}
%
% \def\deflabel#1{\ttfamily #1\hfill}
% \def\deflist#1{\begin{list}{}{\settowidth\labelwidth{\ttfamily #1}%
%                               \setlength\leftmargin\labelwidth
%                               \addtolength\leftmargin\labelsep
%                               \let\makelabel\deflabel}}
% \def\enddeflist{\end{list}}
% \def\envfont{\normalfont\ttfamily}
%
% \MakeShortVerb{\|}
%
% \newcommand{\TUB}{{\sl TUGboat\/}}
%
% \GetFileInfo{theorem.dtx}
% \title{An Extension of the
%        \LaTeX{} theorem environment\thanks{This file
%        has version number \fileversion, last
%        revised \filedate.}}
% \author{Frank Mittelbach}
%
% \MaintainedByLaTeXTeam{tools}
% \maketitle
%
% \begin{abstract}
%    The macros described in this paper yield an extension of the
%    \LaTeX{} theorem mechanism. It is designed to satisfy the
%    different requirements of various journals. Thus, the
%    layout of the ``theorems'' can be manipulated by determining a
%    ``style''. This article describes not only the use, but
%    also the definition, of the necessary macros.
% \end{abstract}
%
% \section*{Preface to version 2.2}
%
% For \LaTeXe{} this package did not need any fundamental changes. I
% only modified the messages generated so that theorem layout styles
% will show up with the |\listfiles| command and cleaned the section
% on the New Font Selection Scheme since this is now included in
% \LaTeX.
%
% \section*{Preface to version 2.1}
%
% This version is identical to 2.0g described in \TUB\ 10\#3 except
% for some internal defaults which are now set depending on the used
% font selection scheme.
%
% This was done to avoid unpleasant surprises if the new font selection
% scheme is in force. For further details see
% section~\ref{sec:newfonts} and \cite{bk:GMS94}.
%
% \section{Introduction}
%
% For our purposes here, ``theorems'' are labelled enunciations,
% often set off from the main text by extra space and a font change.
% Theorems, corollaries, conjectures, definitions, and remarks
% are all instances of ``theorems''.  The ``header'' of these structures
% is composed of a label (such as \textsc{Theorem} or \textsc{Remark})
% and a number which serializes an item in the sequence of items
% with the same label.
%
% Shortly after the introduction of \LaTeX{} at the Fachbereich
% Mathematik in Mainz, the desire to manipulate the layout of
% ``theorems'' arose.  In Mainz, the following two conventions came
% into general use:
% \begin{enumerate}
%   \item The number of the theorem is shown in the margin.
%   \item There is a line break at
%         the end of the theorem header.
% \end{enumerate}
%
% Additionally, some journals require different formats which depend on
% the ``sort of theorem'': e.g.\ often remarks and definitions are set
% in |\upshape|, while |\itshape| is employed for main theorems.
%
% Confronted with these requirements, a theorem environment was
% developed in Mainz which allows separate determination of the
% layout of the ``theorems sets'', comparable to |\pagestyle|.
%
%
% \section{The user interface}
% \subsection{Defining new theorem sets}
%
% \DescribeMacro\newtheorem
% As in the original \LaTeX{} version, the command |\newtheorem|
% defines a new ``theorem set'' or ``theorem-like structure''.
% Two required arguments name the new environment and give the text
% to be typeset with each instance of the new ``set'', while
% an optional argument determines how
% the ``set'' is enumerated:
% \begin{description}
%    \item[\ttfamily \bslash newtheorem\{foo\}\{bar\}]
%       The theorem set {\envfont foo} (whose name is \texttt{bar})
%       uses its own counter.
%    \item[\ttfamily \bslash newtheorem\{foo2\}{[foo]}\{bar2\}]
%       The theorem set {\envfont foo2} (printed name \texttt{bar2})
%       uses the same counter as the theorem set \texttt{foo}.
%    \item[\ttfamily \bslash newtheorem\{foo3\}\{bar3\}{[section]}]
%       The theorem set {\envfont foo3} (printed name \texttt{bar3}) is
%       enumerated within the counter \texttt{section}, i.e.\ with every
%       new |\section| the enumeration begins again with 1, and
%       the enumeration is composed from the section-number and the
%       theorem counter itself.
% \end{description}
%
% \DescribeMacro\theoremstyle
% Additionally, the command |\theoremstyle| can define the layout
% of various, or all, theorem sets.  It should be noted that any
% theorem set defined by |\newtheorem| is typeset in the
% |\theoremstyle| that is current at the time of the definition.
% Thus, the following
% \begin{verbatim}
%    \theoremstyle{break}        \newtheorem{Cor}{Corollary}
%    \theoremstyle{plain}        \newtheorem{Exa}{Example}[section]
%\end{verbatim}                             ^^A No blank at line start
% leads to the result that the set {\envfont Cor} is formatted in the
% style \texttt{break}, while the set {\envfont Exa} and all the
% following ones are formatted in the style \texttt{plain}, unless
% another |\theoremstyle| follows.  Since the definitions installed by
% |\newtheorem| are global, one also can limit |\theoremstyle| locally
% by grouping braces.
%
% \DescribeMacro\theorembodyfont
% The choice of the font for the theorem body is completely
% independent of the chosen |\theoremstyle|; this has proven to
% be very advantageous.  For example,
% \begin{verbatim}
%   {\theorembodyfont{\upshape}          \newtheorem{Rem}{Remark}}
%\end{verbatim}
% defines a theorem set {\envfont Rem}, which will be set in
% |\upshape| in the current layout (which in our example is
% \texttt{plain}).  As with |\theoremstyle|, the |\theorembodyfont|
% chosen is that current at the time of |\newtheorem|.  If
% |\theorembodyfont| is not specified or one defines
% |\theorembodyfont{}|, then the font used will be that defined by the
% |\theoremstyle|.
%
% \DescribeMacro\theoremheaderfont It is also possible to customize
% the font used for the theorem headers.  This is, however, a global
% declaration, and therefore there should be at most one
% |\theoremheaderfont| declaration in the preamble.\footnote{If it is
% actually necessary to have different header fonts, one has to define
% new theorem styles (substituting the desired font) or specify the
% information directly in the \texttt{\bslash newtheorem} declaration
% (the unclean variant).}
%
% \DescribeMacro\theorempreskipamount
% \DescribeMacro\theorempostskipamount Two additional parameters
% affect the vertical space around the theorem environments:
% |\theorempreskipamount| and |\theorempostskipamount| define,
% respectively, the spacing before and after such an environment.
% These parameters apply for all theorem sets and can be manipulated
% with the ordinary length macros.  They are rubber lengths
% (`\textsf{skips}'), and therefore can contain \texttt{plus} and
% \texttt{minus} parts.
%
% Since the definition of theorem sets should---most sensibly---be
% placed in the preamble, we only allow installation there.
% It is therefore possible to release the memory used here after
% |\begin{document}|, in order to make room for other
% applications.
%
%
% \subsection{Existing theorem styles}
%
% The following theorem styles exist to date:
% \begin{deflist}{marginbreak}
%    \item[plain]
%       This theorem style emulates the original \LaTeX{} definition,
%       except that additionally the parameters
%       |\theorem...skipamount| are used.
%    \item[break]
%       In this style, the theorem header is followed by a line
%       break.
%    \item[marginbreak]
%       The theorem number is set in the margin, and there is a line
%       break as in \texttt{break}.
%    \item[changebreak]
%       Like \texttt{break}, but with header number and text
%       interchanged.
%    \item[change]
%       Header number and text are interchanged, without a line break.
%    \item[margin]
%       The number is set in the left margin, without a line break.
% \end{deflist}
% All styles (except \texttt{plain}) select |\slshape| as the default
% |\theorembodyfont|.
%
%
% \subsection{Examples}
%
% Given the above theorem sets {\envfont Cor}, {\envfont Exa}
% and {\envfont Rem},
% suppose that the preamble also contains the declarations:
% \begin{verbatim}
%    \theoremstyle{marginbreak}   \newtheorem{Lem}[Cor]{Lemma}
%    \theoremstyle{change}
%    \theorembodyfont{\itshape}        \newtheorem{Def}[Cor]{Definition}
%
%    \theoremheaderfont{\scshape}
%\end{verbatim}
% Then the following are some typical examples of the typeset output
% resulting from their use.
%
% \begin{Cor}
%    This is a sentence typeset in the theorem environment {\envfont
%    Cor}.
% \end{Cor}
%
% \begin{Exa}
%    This is a sentence typeset in the theorem environment {\envfont
%    Exa}.
% \end{Exa}
%
% \begin{Rem}
%    This is a sentence typeset in the theorem environment {\envfont
%    Rem}.
% \end{Rem}
%
% \begin{Lem}[Ben User]
%    This is a sentence typeset in the theorem environment {\envfont
%    Lem}.
% \end{Lem}
%
% \begin{Def}[Very impressive Definition]
%    This is a sentence typeset in the theorem environment {\envfont
%    Def}.
% \end{Def}
%
% The last two examples show the effect of the optional argument to
% a theorem environment (it is the text typeset in parentheses).
%
%
% \section{Special Considerations}
% \label{sec:newfonts}
%
% Theoremheader and body are implemented as a unit. This means that
% the |\theoremheaderfont|  will inherit characteristics of the
% |\theorembodyfont| in \LaTeXe. Thus, if for example
% |\theorembodyfont| is |\itshape| and |\theoremheaderfont|
% is |\bfseries| the font selected for the header will have the
% characteristics `bold extended italic'. If this is not desired one
% should set the |\theoremheaderfont| to something like
% \begin{verbatim}
%      \theoremheaderfont{\normalfont\bfseries}
%\end{verbatim}
% i.e.\ supplying all necessary font information explicitly.
%
%
%
% \section{Acknowledgements}
%
% The publication of this set of macros was only possible with the
% help of Christina Busse (translating the manuscript into English),
% Joachim Pense (playing the r\^ole of typist), Chris Rowley
% (looking everything over) and many others providing useful
% suggestions.
%
% \MaybeStop{%
%    \begin{thebibliography}{9}
%       \bibitem{bk:GMS94} \textsc{M.~Goossens}, \textsc{F.~Mittelbach}
%         and \textsc{A.~Samarin}.
%         \newblock The \LaTeX{} Companion.
%         \newblock
%         Addison-Wesley, Reading, Massachusetts, 1994.
%       \bibitem{latex:source} \textsc{Lamport, Leslie.} \newblock
%                               \texttt{latex.tex}, version 2.09, date
%                               Feb. 1990.
%    \end{thebibliography}
%    \PrintIndex
%    \PrintChanges}
%
%
% \section{The documentation driver file}
%
% The next bit of code contains the documentation driver file for
% \TeX{}, i.e., the file that will produce the documentation you are
% currently reading. It will be extracted from this file by the
% \texttt{docstrip} program. Since it is the first code in the file
% one can alternatively process this file directly with \LaTeXe{} to
% obtain the documentation.
%
%    \begin{macrocode}
%<*driver>
\documentclass{ltxdoc}

\usepackage{theorem}

%<+driver>% The next few lines define theorem sets which are used
%<+driver>% in the example section of the documentation.

\theoremstyle{break}       \newtheorem{Cor}{Corollary}
\theoremstyle{plain}       \newtheorem{Exa}{Example}[section]
{\theorembodyfont{\upshape}\newtheorem{Rem}{Remark}}
\theoremstyle{marginbreak} \newtheorem{Lem}[Cor]{Lemma}
\theoremstyle{change}
\theorembodyfont{\itshape} \newtheorem{Def}[Cor]{Definition}

\theoremheaderfont{\scshape}

\RecordChanges

\begin{document}
   \DocInput{theorem.dtx}
\end{document}
%</driver>
%    \end{macrocode}
%
% \section{Definition of the Macros}
%
% If the file has been loaded before, we abort immediately. If not the
% package announces itself (this is actually done at the very top of
% the file---the way it is done isn't good style so don't copy it).
% \changes{v2.0e}{89/07/19}{Spaces removed from `typeout'.}
% \changes{v2.2b}{95/11/19}{Announce moved to top of file}
%    \begin{macrocode}
%<*package>
%\@ifundefined{theorem@style}{}{\endinput}
%\def\FMithmInfo{1995/11/19 v2.2b Theorem extension package (FMi)}
%\ProvidesPackage{theorem}[\FMithmInfo]
%    \end{macrocode}
%
%
% \subsection{Definition of theorem styles and fonts}
%
% All the definitions in this file are done globally to allow
% inputting this file inside a group.
%
% \begin{macro}{\theoremstyle}
%    Before a theorem style can be installed, the chosen style must be
%    known. For that reason, we must test to see that
%    |\th@|\meta{style} is known or, more precisely, that it is
%    different from |\relax|. If the style is not known then
%    |\th@plain| is used.
%    \begin{macrocode}
\gdef\theoremstyle#1{%
   \@ifundefined{th@#1}{\@warning
          {Unknown theoremstyle `#1'. Using `plain'}%
          \theorem@style{plain}}%
%    \end{macrocode}
%    We save the theorem style to be used in the token register
%    |\theorem@style|.
%    \begin{macrocode}
      {\theorem@style{#1}}%
%    \end{macrocode}
%    Now we ``evaluate'' the theorem style: this means, we call the
%    macro |\th@|\meta{style} which will activate the relevant
%    definitions which are contained in a separate file.  This is done
%    in a group to suppress changes to the current font.  This could
%    otherwise pose problems together with the new font selection
%    scheme\footnote{When I printed the original article using the new
%    font selection scheme I ended with a document with slanted
%    typefaces (text headings and all) simply because one of the
%    theorem styles used \texttt{\bslash sl} at toplevel.} if the
%    |\th@|\meta{style} is evaluated a second time.
% \changes{v2.1a}{90/02/15}{Group added around `th@..}
% \changes{v2.1a}{90/02/15}{Reseting of `..@begintheorem removed.}
%    \begin{macrocode}
      \begingroup
        \csname th@\the\theorem@style \endcsname
      \endgroup}
%    \end{macrocode}
% \end{macro}
%
%
% \begin{macro}{\@begintheorem}
% \changes{v2.1a}{90/02/15}{Relaxed at toplevel.}
% \begin{macro}{\@opargbegintheorem}
% \changes{v2.1a}{90/02/15}{Relaxed at toplevel.}
%    We reset |\@begintheorem| and |\@opargbegintheorem| to
%    |\relax| since these commands are no longer necessary at
%    toplevel. This will save a few tokens.
%    \begin{macrocode}
\global\let\@begintheorem\relax
\global\let\@opargbegintheorem\relax
%    \end{macrocode}
% \end{macro}
% \end{macro}
%
% \begin{macro}{\theorem@style}
%    Obviously the \textsf{token} register used above has to be
%    allocated. To assure the utmost compatibility with the
%    original \LaTeX{} definition, we set the default theorem style to
%    \texttt{plain}, which implements the usual \LaTeX{} convention.
% \changes{v2.0f}{89/07/20}{Unnecessary `global' removed.}
%    \begin{macrocode}
\newtoks\theorem@style
\global\theorem@style{plain}
%    \end{macrocode}
% \end{macro}
%
% \begin{macro}{\theorembodyfont}
% \begin{macro}{\theorem@bodyfont}
%    If the |\theorembodyfont| is set by the user then it should not
%    interact with the default font set in the theorem style. When the
%    new font selection is in force this may happen if, for example,
%    the default is |\itshape| and the new |\theorembodyfont| is
%    |\sffamily|. So we add a |\reset@font| command in front of the
%    user definition.
% \changes{v2.0f}{89/07/20}{Unnecessary `global' removed.}
% \changes{v2.0g}{89/09/19}{`theoremfont to `theorembodyfont.}
% \changes{v2.1c}{92/06/22}{`reset@font added and logic changed.}
% \changes{v2.1d}{92/10/28}{Reset to empty if arg is empty}
%    \begin{macrocode}
\gdef\theorembodyfont#1{%
%    \end{macrocode}
%    We check if the argument supplied is empty and if so put nothing
%    into the |\theorem@bodyfont| token register to allow for
%    |\theorembodyfont{}| as a mean of using the default of the
%    current |\theoremstyle|.
%    \begin{macrocode}
   \def\@tempa{#1}%
   \ifx\@tempa\@empty
    \theorem@bodyfont{}%
   \else
    \theorem@bodyfont{\reset@font#1}%
   \fi
   }
\newtoks\theorem@bodyfont
\global\theorem@bodyfont{}
%    \end{macrocode}
% \end{macro}
% \end{macro}
%
%
% \begin{macro}{\theoremheaderfont}
% \changes{v2.0g}{89/09/19}{Macro added.}
%    The font for the theorem headers is handled differently because
%    this definition applies to all theorem styles.
%    \begin{macrocode}
\gdef\theoremheaderfont#1{\gdef\theorem@headerfont{#1}%
%    \end{macrocode}
%    After using the macro once it is redefined to produce an error
%    message.
%    \begin{macrocode}
       \gdef\theoremheaderfont##1{%
        \typeout{\string\theoremheaderfont\space should be used
                 only once.}}}
%    \end{macrocode}
% \end{macro}
%
% \begin{macro}{\theorem@headerfont}
% \changes{v2.0g}{89/09/19}{Macro added.}
%    To set the |\theorem@headerfont| default we first test if
%    the new fontselection scheme is in force.
% \changes{v2.1a}{90/02/15}{Support for new font selection.}
%    \begin{macrocode}
\ifx\upshape\undefined
%    \end{macrocode}
%    If not we define it to expand into |\bfseries|. We don't use
%    |\let| just in case a following style option redefines this
%    macro.
%    \begin{macrocode}
\gdef\theorem@headerfont{\bfseries}
%    \end{macrocode}
%    Otherwise we reset the current shape before calling |\bfseries|.
%    \begin{macrocode}
\else \gdef\theorem@headerfont{\normalfont\bfseries}\fi
%    \end{macrocode}
% \end{macro}
%
% \begin{macro}{\th@plain}
% \begin{macro}{\th@break}
% \begin{macro}{\th@marginbreak}
% \begin{macro}{\th@changebreak}
% \begin{macro}{\th@change}
% \begin{macro}{\th@margin}
%    The different styles are defined in macros such as
%    |\th@plain|.  Since memory space is precious in
%    ``non-Big-versions'', we have to avoid offering too many unused
%    definitions.  Therefore we define these styles in separate files
%    that can be loaded on demand.  Thus the commands themselves only
%    load these files. We use |\@input@|, a \LaTeXe{} internal command
%    that ensures that the file will be listed with |\listfiles|.
% \changes{v2.2a}{94/02/02}{Use `@input@ to load theorem layout files}
%    \begin{macrocode}
\gdef\th@plain{\@input@{thp.sty}}
\gdef\th@break{\@input@{thb.sty}}
\gdef\th@marginbreak{\@input@{thmb.sty}}
\gdef\th@changebreak{\@input@{thcb.sty}}
\gdef\th@change{\@input@{thc.sty}}
\gdef\th@margin{\@input@{thm.sty}}
%    \end{macrocode}
%    This list will be expanded when new styles become
%    available.  For testing,  just append new theorem substyles
%    as document options.
% \end{macro}
% \end{macro}
% \end{macro}
% \end{macro}
% \end{macro}
% \end{macro}
%
%
%
% \subsection{Definition of a new theorem set}
%
% As already pointed out, a new theorem environment can be defined in
% three different ways:
% \begin{verbatim}
%    \newtheorem{Lem}{Lemma}
%    \newtheorem{Lem}{Lemma}[section]
%    \newtheorem{Lem}[Theorem]{Lemma}
%\end{verbatim}
% The function of the macro |\newtheorem| is to recognize these cases
% and then to branch into one of the three macros |\@ynthm|, |\@xnthm|
% or |\@othm|.  This mechanism is adopted unchanged from
% \cite{latex:source}; the essential point here is that, for example,
% in the second case, the arguments \texttt{Lem}, \texttt{Lemma} and
% \texttt{section} are passed over to the macro |\@xnthm|.
%
% We inspect this case first because the others present fewer problems,
% and thus are easily derived from this one.
%
% \begin{macro}{\@xnthm}
%    For our example arguments, the macro |\@xnthm| must fulfill
%    the following:
%    \begin{itemize}
%       \item
%          Define a new \LaTeX-counter `Lem'
%       \item
%          reset this counter within a |\section|
%       \item
%          define the macro |\theLem|
%       \item
%          define the environment macros |\Lem| and
%          |\endLem| using the current
%          |\theoremstyle| and |\theorem@bodyfont|.
%    \end{itemize}
%    Obviously, all this should happen only if the first argument
%    of |\@xnthm| (i.e.\ \texttt{Lem} in our example) is chosen so
%    as not to conflict with any previously defined commands or
%    environments.  This test is performed by the \LaTeX{} macro
%    |\@ifdefinable|.
%    \begin{macrocode}
\gdef\@xnthm#1#2[#3]{\expandafter\@ifdefinable\csname #1\endcsname
%    \end{macrocode}
%    Therefore, the first argument of |\@ifdefinable| is the
%    expansion (in the example, |\Lem|) of
%    |\csname||#1\endcsname|.  The second argument is
%    executed only if the test has been completed successfully.
%    \begin{macrocode}
   {%
%    \end{macrocode}
%    Now we define the new counter. The names of the \LaTeX{}
%    macros employed should speak for themselves:
% \changes{v2.2b}{1995/10/07}{Use \cs{@newctr}}
% \changes{v2.2c}{1995/11/23}{Use \cs{@newctr} properly.}
%    \begin{macrocode}
    \@definecounter{#1}\@newctr{#1}[#3]%
%    \end{macrocode}
%    Using |\@newctr| will give a proper error message if the counter
%    in |#3| is not defined.
%    In defining `|\theLem|' we must generate the desired macro
%    name by use of |\expandafter| and |\csname|.
%    \begin{macrocode}
    \expandafter\xdef\csname the#1\endcsname
%    \end{macrocode}
%    An |\xdef| is used in order to make the definition global,
%    and to ensure that it contains the replacement texts of
%    |\@thmcountersep| and |\@thmcounter|.\footnote{These
%    two macros can be defined by the document style.  Their default
%    values produce a `\texttt{.}' as separation and an arabic
%    representation of the number.} However, not everything should be
%    expanded. For example, it saves space to use |\thesection|
%    instead of its---at times---lengthy expansion.
%    \begin{macrocode}
      {\expandafter \noexpand \csname the#3\endcsname
       \@thmcountersep \@thmcounter{#1}}%
%    \end{macrocode}
%    Thus with the defaults of \LaTeX{}, |\theLem| would be
%    replaced by the command sequence |\thesection.\arabic{Lem}|.
%
%    We will now look at the definition of the macro which is executed
%    at the beginning of the actual environment (in our example this
%    macro is |\Lem|). It should be noted that we use an
%    ``|\expandafter| trick'' to expand only certain parts of the
%    replacement text at the time of the definition.
%    \begin{macrocode}
    \def\@tempa{\global\@namedef{#1}}%
    \expandafter \@tempa \expandafter{%
%    \end{macrocode}
%    First, the macro that contains the current definitions of
%    |\@begintheorem| and |\@opargtheorem| should be called
%    up.  The name of this macro---as is already known---has the form
%    |\th@|\meta{theorem\ style}; therefore, it must be called by
%    \begin{macrocode}
      \csname th@\the \theorem@style
%    \end{macrocode}
%    In addition the default theorem font should be changeable, i.e.\
%    we have to insert the contents of |\theorem@bodyfont|. For
%    that reason, we expand even further, beyond |\endcsname|,
%    and thus insert the contents of the \textsf{token} register
%    |\theorem@bodyfont| in the replacement text.
%    \begin{macrocode}
            \expandafter \endcsname \the \theorem@bodyfont
%    \end{macrocode}
%    Now it is time to call the macro |\@thm| which takes over
%    the further processing.  It has two arguments: the current
%    counter name (in our example, \texttt{Lem}), and the text of the
%    label (in our example, \texttt{Lemma}).
%    \begin{macrocode}
     \@thm{#1}{#2}}%
%    \end{macrocode}
%    With this, the `sub-definition' is complete.  The macro
%    |\@endtheorem| ends a theorem environment and is, so far, nothing
%    but an |\endtrivlist|.  (Hence it is defined globally, and not
%    within the theorem styles.\footnote{This has to be changed as
%    soon as theorem styles that change \texttt{\bslash @endtheorem}
%    exist.  In such a case, all existing styles must be changed as
%    well since they will have to reset the macro.}) Therefore, we can
%    set it equivalent to the macro that ends the theorem set (in our
%    example, |\endLem|).  However, if some day theorem styles exist
%    that do change |\@endtheorem|, we would have to use the
%    commented-out line instead.
%    \begin{macrocode}
    \global \expandafter \let \csname end#1\endcsname \@endtheorem
%   \global\@namedef{end#1}{\@endtheorem}%
%    \end{macrocode}
%    With these commands all the required definitions are employed,
%    unless the test |\@ifdefinable| has failed.  Therefore, we
%    end the second argument of this macro and with it the definition
%    of |\@xnthm|.
%    \begin{macrocode}
   }}
%    \end{macrocode}
% \end{macro}
%
% \begin{macro}{\@ynthm}
%    The definition of |\@ynthm| is completely analogous. In
%    this case the new counter that is defined is not reset within
%    another counter; thus the definition of |\the...|
%    is simplified:
%    \begin{macrocode}
\gdef\@ynthm#1#2{\expandafter\@ifdefinable\csname #1\endcsname
   {\@definecounter{#1}%
    \expandafter\xdef\csname the#1\endcsname{\@thmcounter{#1}}%
%    \end{macrocode}
%    The rest of the definition corresponds literally to that of
%    |\@xnthm|:
%    \begin{macrocode}
    \def\@tempa{\global\@namedef{#1}}\expandafter \@tempa
     \expandafter{\csname th@\the \theorem@style \expandafter
     \endcsname \the\theorem@bodyfont \@thm{#1}{#2}}%
    \global \expandafter \let \csname end#1\endcsname \@endtheorem}}
%    \end{macrocode}
% \end{macro}
%
% \begin{macro}{\@othm}
%    The definition of |\@othm| does not contain anything new.
%    \begin{macrocode}
\gdef\@othm#1[#2]#3{%
%    \end{macrocode}
% \changes{v2.2b}{95/10/07}{Build in check for counter defined}
%    We do not define a new counter but instead use one that has
%    already been defined.  Thus the only definition we need is that
%    of this pseudo-counter (i.e.\ |\the|\meta{env.\ name}).
%    First we check if |#2| corresponds to a known counter name.
%    \begin{macrocode}
  \expandafter\ifx\csname c@#2\endcsname\relax
   \@nocounterr{#2}%
  \else
   \expandafter\@ifdefinable\csname #1\endcsname
   {\expandafter \xdef \csname the#1\endcsname
     {\expandafter \noexpand \csname the#2\endcsname}%
%    \end{macrocode}
%    All other parts of the definition can be adopted from
%    |\@xnthm|. We have to remember, though, that in this case
%    the name of the current counter and the theorem label have moved
%    to the second and third arguments.
%    \begin{macrocode}
    \def\@tempa{\global\@namedef{#1}}\expandafter \@tempa
     \expandafter{\csname th@\the \theorem@style \expandafter
     \endcsname \the\theorem@bodyfont \@thm{#2}{#3}}%
    \global \expandafter \let \csname end#1\endcsname \@endtheorem}%
  \fi}
%    \end{macrocode}
% \end{macro}
%
% \subsection{Macros that are employed in a {\envfont theorem}
%    environment}
%
% \begin{macro}{\@thm}
%    The macro |\@thm| has to increase the current counter.  Then,
%    depending on whether the environment has (or does not have) an
%    optional argument, it has to branch into either |\@begintheorem|
%    or |\@opargtheorem|.
%    \begin{macrocode}
\gdef\@thm#1#2{\refstepcounter{#1}%
%    \end{macrocode}
%    Now we start a {\envfont trivlist} environment, and give
%    |\@topsep| and |\@topsepadd| the values of the \textsf{skip}
%    registers |\theorempreskipamount| and |\theorempostskipamount|.
%    The value in |\@topsep| is the vertical space that is inserted by
%    the first (and only) |\item| in our |\trivlist| whilst
%    |\@topsepadd| is inserted by |\@endparenv| at the end of that
%    {\envfont trivlist} environment.  By using these registers, we
%    obtain the desired space around a {\envfont theorem} environment.
%    \begin{macrocode}
   \trivlist
   \@topsep \theorempreskipamount               % used by first \item
   \@topsepadd \theorempostskipamount           % used by \@endparenv
%    \end{macrocode}
%    Now we have to test whether an optional argument has been given.
% \changes{v2.0d}{89/07/18}{Forgotten percent added.}
%    \begin{macrocode}
   \@ifnextchar [%
%    \end{macrocode}
%    If there is an optional argument, we will call |\@ythm|, and
%    move the arguments read back into the input stream.
%    \begin{macrocode}
   {\@ythm{#1}{#2}}%
%    \end{macrocode}
%    If not, we call |\@begintheorem|. Its first argument is the
%    name of the theorem set (hence the second argument of
%    |\@thm|). Its second argument is the macro that produces the
%    current number.
%    \begin{macrocode}
   {\@begintheorem{#2}{\csname the#1\endcsname}\ignorespaces}}
%    \end{macrocode}
% \end{macro}
%
% \begin{macro}{\@xthm}
% \begin{macro}{\@ythm}
%    Both these macros were originally called by |\@thm|.  We do
%    not need |\@xthm| anymore, hence we reset it to
%    |\relax|. The definition of |\@ythm| has not changed at
%    all from its definition in \LaTeX. In order to make the macros
%    easier to understand, we will
%    nevertheless present it (commented out).
%    \begin{macrocode}
\global\let\@xthm\relax
% \def\@ythm#1#2[#3]{\@opargbegintheorem{#2}{\csname
%           the#1\endcsname}{#3}\ignorespaces}
%</package>
%    \end{macrocode}
%    The primitive |\ignorespaces| in |\@ythm| and
%    |\@thm| is needed to remove the spaces between the
%    |\begin{...}| and the actual text.
% \end{macro}
% \end{macro}
%
%
% \subsection{Definition of the theorem substyles}
%
% As already pointed out, the theorem substyles, defined below, are
% only loaded when necessary.  Note that all these substyles, except
% \texttt{plain}, have |\slshape| as the default body font.
%
% \subsubsection{The \texttt{plain} style}
%
% \label{dok:th@plain} As the following macros use |@|, we have
% to locally set the |\catcode| of this symbol to ``letter''.
% This happens within a group, so that we do not have to worry about
% which |\catcode| that symbol had before.
%    \begin{macrocode}
%<*thp>
\begingroup \makeatletter
%    \end{macrocode}
% Since we are now within a group, we must make all definitions
% globally.  First we make sure that \texttt{theorem.sty} is loaded.
% This will allow us to use this file as a document style option
% without having to call \texttt{theorem} itself as an option.  At the
% same time, we assure that at least version 2 is loaded, since
% |\theorem@style| was not defined in earlier versions.
%    \begin{macrocode}
\@ifundefined{theorem@style}{\input{theorem.sty}}{}
\ProvidesFile{thp.sty}
             [\FMithmInfo]
%    \end{macrocode}
%
%
% \begin{macro}{\th@plain}
% \begin{macro}{\@begintheorem}
% \begin{macro}{\@opargbegintheorem}
%    |\theoremstyle{plain}| corresponds to the original
%    definition, except that the distances to the surrounding text are
%    determined by the parameters |\theorempreskipamount| and
%    |\theorempostskipamount|.
%    First we set the default body font.
%    \begin{macrocode}
\gdef\th@plain{\normalfont\itshape
%    \end{macrocode}
%    Then we define |\@begintheorem| and
%    |\@opargbegintheorem|. These two macros define how the
%    header of a theorem is typeset.  |\@opargbegintheorem| will
%    be called if a \textsf{theorem} environment with an optional
%    argument is encountered; otherwise, the header is constructed by
%    calling |\@begintheorem|.  If one of these macros is
%    executed, we are within a \textsf{trivlist} environment started by
%    |\@thm|.  So the theorem header is produced with an
%    |\item| command.
%
%    Instead of specifying the header font directly, all standard
%    theorem styles use the |\theorem@headerfont| macro to allow
%    customization.  The extra space (|\labelsep|) is necessary
%    because of problems in the \textsf{trivlist} environment.
%    \begin{macrocode}
  \def\@begintheorem##1##2{%
        \item[\hskip\labelsep \theorem@headerfont ##1\ ##2]}%
%    \end{macrocode}
%    The definition of |\@opargbegintheorem| is completely
%    analogous. The only difference is the fact that there exists a
%    third argument (which is the optional parameter of the
%    environment and contains additional information about the
%    theorem).  Customarily we enclose it in parentheses.
%    \begin{macrocode}
\def\@opargbegintheorem##1##2##3{%
   \item[\hskip\labelsep \theorem@headerfont ##1\ ##2\ (##3)]}}
%    \end{macrocode}
% \end{macro}
% \end{macro}
% \end{macro}
%
% \noindent
% We conclude with an |\endgroup| to restore the
% |\catcode| of |@|.
%    \begin{macrocode}
\endgroup
%</thp>
%    \end{macrocode}
%
%
% \subsubsection{The \texttt{break} style}
%
% This style option is stored in the file |thb.sty|.
% For the next two lines see the documentation for |\th@plain| on
% page~\pageref{dok:th@plain}.
%    \begin{macrocode}
%<*thb>
\begingroup \makeatletter
\@ifundefined{theorem@style}{\input{theorem.sty}}{}
\ProvidesFile{thb.sty}
             [\FMithmInfo]
%    \end{macrocode}
%
% \begin{macro}{\th@break}
%    |\theoremstyle{break}| produces a line break after the name
%    of the theorem.  The font is |\slshape|.  Hence, we define
%    |\th@break| as follows:
%    \begin{macrocode}
\gdef\th@break{\normalfont\slshape
  \def\@begintheorem##1##2{\item[%
%    \end{macrocode}
%    We run into the following problem: it is not possible to create
%    the header with |\item[|\meta{title}|]| and then start
%    a new line by, for example, |\mbox{}\\|. Such a definition
%    will fail whenever a list environment follows immediately.  With
%    the above construction, the |\mbox{}| causes the switch
%    |@inlabel| (cf.\ definition of |\list| and
%    |\trivlist| in \cite{latex:source}) to be set to \texttt{false}
%    and so the following list will insert additional vertical
%    space (|\topskip|). This is quite annoying. Therefore, we
%    create the line break within the |\item|. In order to ensure
%    that the text will begin at the proper position in the following
%    line, we simply pretend that the label does not take any
%    room.\footnote{This will lead to problems whenever very high
%    symbols occurring in the line tower into the
%    heading. So, something else has to be done here sometime.}
%    \begin{macrocode}
       \rlap{\vbox{\hbox{\hskip \labelsep\theorem@headerfont ##1\ ##2}%
                   \hbox{\strut}}}]}%
%    \end{macrocode}
%    Again, the definition of |\@opargbegintheorem| is
%    completely analogous.
%    \begin{macrocode}
\def\@opargbegintheorem##1##2##3{%
  \item[\rlap{\vbox{\hbox{\hskip \labelsep \theorem@headerfont
                     ##1\ ##2\ (##3)}%
                    \hbox{\strut}}}]}}
\endgroup
%</thb>
%    \end{macrocode}
% \end{macro}
%
%
% \subsubsection{The \texttt{changebreak} style}
%
%    \begin{macrocode}
% This style option is stored in the file |thcb.sty|.
%    \begin{macrocode}
%<*thcb>
\begingroup \makeatletter
\@ifundefined{theorem@style}{\input{theorem.sty}}{}
\ProvidesFile{thcb.sty}
             [\FMithmInfo]
%    \end{macrocode}
%
% \begin{macro}{\th@changebreak}
%    The \texttt{change-break} theorem style is like \texttt{break}
%    but with interchange of theorem name and theorem number.  Thus we
%    define |\th@changebreak| as follows:
%    \begin{macrocode}
\gdef\th@changebreak{\normalfont\slshape
  \def\@begintheorem##1##2{\item
       [\rlap{\vbox{\hbox{\hskip\labelsep \theorem@headerfont ##2\ ##1}%
                    \hbox{\strut}}}]}%
\def\@opargbegintheorem##1##2##3{%
  \item[\rlap{\vbox{\hbox{\hskip\labelsep \theorem@headerfont
                     ##2\ ##1\ (##3)}%
                    \hbox{\strut}}}]}}
\endgroup
%</thcb>
%    \end{macrocode}
% \end{macro}
%
%
% \subsubsection{The \texttt{change} style}
%
% This style option is stored in the file |thc.sty|.
%    \begin{macrocode}
%<*thc>
\begingroup \makeatletter
\@ifundefined{theorem@style}{\input{theorem.sty}}{}
\ProvidesFile{thc.sty}
             [\FMithmInfo]
%    \end{macrocode}
%
% \begin{macro}{\th@change}
%    The \texttt{change} theorem style corresponds to the
%    \texttt{change break} style without a linebreak after the header.
%    To say it in another way, it's the same as the \texttt{plain}
%    style but with number and name interchanged and |\slshape| as the
%    default font.
%    \begin{macrocode}
\gdef\th@change{\normalfont\slshape
  \def\@begintheorem##1##2{\item
       [\hskip\labelsep \theorem@headerfont ##2\ ##1]}%
\def\@opargbegintheorem##1##2##3{%
  \item[\hskip\labelsep \theorem@headerfont ##2\ ##1\ (##3)]}}
\endgroup
%</thc>
%    \end{macrocode}
% \end{macro}
%
%
% \subsubsection{The \texttt{marginbreak} style}
%
% This style option is the one used most often at Mainz.  It is saved
% in the file |thmb.sty|.
%    \begin{macrocode}
%<*thmb>
\begingroup \makeatletter
\@ifundefined{theorem@style}{\input{theorem.sty}}{}
\ProvidesFile{thmb.sty}
             [\FMithmInfo]
%    \end{macrocode}
%
% \begin{macro}{\th@marginbreak}
%    The \texttt{margin break} style is nearly the same as the
%    \texttt{change break} style. The only difference is the placement
%    of the theorem number.  We use |\llap| to place it in the left
%    margin.
%
%    In this style |\labelsep| denotes the separation between the
%    number and the text.
%    \begin{macrocode}
\gdef\th@marginbreak{\normalfont\slshape
  \def\@begintheorem##1##2{\item
       [\rlap{\vbox{\theorem@headerfont
                    \hbox{\llap{##2}\hskip\labelsep ##1}%
                    \hbox{\strut}}}]}%
\def\@opargbegintheorem##1##2##3{%
  \item[\rlap{\vbox{\theorem@headerfont
                    \hbox{\llap{##2}\hskip\labelsep ##1\ (##3)}%
                    \hbox{\strut}}}]}}
\endgroup
%</thmb>
%    \end{macrocode}
% \changes{v2.1c}{90/02/17}{`endgroup outside macrocode env.}
% \end{macro}
%
%
%
% \subsubsection{The \texttt{margin} style}
%
% This style option is stored in the file |thm.sty|.
%    \begin{macrocode}
%<*thm>
\begingroup \makeatletter
\@ifundefined{theorem@style}{\input{theorem.sty}}{}
\ProvidesFile{thm.sty}
             [\FMithmInfo]
%    \end{macrocode}
%
% \begin{macro}{\th@margin}
%    Again this is only a variant of the theorem styles described
%    above without any new ideas.
%    \begin{macrocode}
\gdef\th@margin{\normalfont\slshape
  \def\@begintheorem##1##2{\item
       [\theorem@headerfont \llap{##2}\hskip\labelsep ##1]}%
\def\@opargbegintheorem##1##2##3{%
  \item[\theorem@headerfont \llap{##2}\hskip\labelsep ##1\ (##3)]}}
\endgroup
%</thm>
%    \end{macrocode}
% \end{macro}
%
%
%
%
% \subsection{Final Definitions}
%
% \begin{macro}{\theorempreskipamount}
% \begin{macro}{\theorempostskipamount}
%    The \textsf{skip} parameters that regulate the vertical empty space
%    before and after the {\envfont theorem} environment have
%    to be allocated as well.
% \changes{v2.0f}{89/07/20}{Unnecessary `global' removed.}
%    \begin{macrocode}
%<*package>
\newskip\theorempreskipamount
\newskip\theorempostskipamount
%    \end{macrocode}
%    Since we have used the same values for all theorem sets, we now can
%    assign them.
%    \begin{macrocode}
\global\setlength\theorempreskipamount{12pt plus 5pt minus 3pt}
\global\setlength\theorempostskipamount{8pt plus 3pt minus 1.5pt}
%    \end{macrocode}
% \end{macro}
% \end{macro}
%
%
% \begin{macro}{\@endtheorem}
%    The same holds for the macro |\@endtheorem|, which ends a
%    {\envfont theorem} environment. Since it is the same for all
%    theorem sets, it is removed from the macros |\th@|\meta{style}.
%    It simply ends the {\envfont trivlist} environment, which was
%    begun in |\@thm|.
%    \begin{macrocode}
\global\let\@endtheorem=\endtrivlist
%    \end{macrocode}
% \end{macro}
%
%
%
% \begin{macro}{\@preamblecmds}
%    All macros defined above are to be used only in the preamble.
%    Therefore, we insert them in |\@preamblecmds| which will disable
%    them at begin document. This is done by the internal \LaTeXe{}
%    command |\@onlypreamble|.
% \changes{v2.0g}{89/09/19}{`theoremheaderfont and `theorembodyfont
%    added.}
% \changes{v2.2a}{94/02/03}{Use \LaTeXe{} interface}
%    \begin{macrocode}
\@onlypreamble\@xnthm
\@onlypreamble\@ynthm
\@onlypreamble\@othm
\@onlypreamble\newtheorem
\@onlypreamble\theoremstyle
\@onlypreamble\theorembodyfont
\@onlypreamble\theoremheaderfont
%    \end{macrocode}
% \end{macro}
%
% Finally we declare the \texttt{plain} theorem style to be the default.
% \changes{v2.1b}{90/06/04}{Fixed missing `item by declaring default
%    style.}
%    \begin{macrocode}
\theoremstyle{plain}
%</package>
%    \end{macrocode}
%
% \Finale

\endinput
}{}
\ProvidesFile{thb.sty}
             [\FMithmInfo]
%    \end{macrocode}
%
% \begin{macro}{\th@break}
%    |\theoremstyle{break}| produces a line break after the name
%    of the theorem.  The font is |\slshape|.  Hence, we define
%    |\th@break| as follows:
%    \begin{macrocode}
\gdef\th@break{\normalfont\slshape
  \def\@begintheorem##1##2{\item[%
%    \end{macrocode}
%    We run into the following problem: it is not possible to create
%    the header with |\item[|\meta{title}|]| and then start
%    a new line by, for example, |\mbox{}\\|. Such a definition
%    will fail whenever a list environment follows immediately.  With
%    the above construction, the |\mbox{}| causes the switch
%    |@inlabel| (cf.\ definition of |\list| and
%    |\trivlist| in \cite{latex:source}) to be set to \texttt{false}
%    and so the following list will insert additional vertical
%    space (|\topskip|). This is quite annoying. Therefore, we
%    create the line break within the |\item|. In order to ensure
%    that the text will begin at the proper position in the following
%    line, we simply pretend that the label does not take any
%    room.\footnote{This will lead to problems whenever very high
%    symbols occurring in the line tower into the
%    heading. So, something else has to be done here sometime.}
%    \begin{macrocode}
       \rlap{\vbox{\hbox{\hskip \labelsep\theorem@headerfont ##1\ ##2}%
                   \hbox{\strut}}}]}%
%    \end{macrocode}
%    Again, the definition of |\@opargbegintheorem| is
%    completely analogous.
%    \begin{macrocode}
\def\@opargbegintheorem##1##2##3{%
  \item[\rlap{\vbox{\hbox{\hskip \labelsep \theorem@headerfont
                     ##1\ ##2\ (##3)}%
                    \hbox{\strut}}}]}}
\endgroup
%</thb>
%    \end{macrocode}
% \end{macro}
%
%
% \subsubsection{The \texttt{changebreak} style}
%
%    \begin{macrocode}
% This style option is stored in the file |thcb.sty|.
%    \begin{macrocode}
%<*thcb>
\begingroup \makeatletter
\@ifundefined{theorem@style}{% \iffalse meta-comment
%
% Copyright (C) 1993-2024
%
% The LaTeX Project and any individual authors listed elsewhere
% in this file.
%
% This file is part of the Standard LaTeX `Tools Bundle'.
% -------------------------------------------------------
%
% It may be distributed and/or modified under the
% conditions of the LaTeX Project Public License, either version 1.3c
% of this license or (at your option) any later version.
% The latest version of this license is in
%    https://www.latex-project.org/lppl.txt
% and version 1.3c or later is part of all distributions of LaTeX
% version 2005/12/01 or later.
%
% The list of all files belonging to the LaTeX `Tools Bundle' is
% given in the file `manifest.txt'.
%
% \fi
%
%%
%
%
\def\FMithmInfo{2023/07/05 v2.2c Theorem extension package (FMi)}
%
% \ProvidesFile{theorem.dtx}[\FMithmInfo]
%\iffalse   % this is a METACOMMENT !
%%
%% Package `theorem' to use with LaTeX2e
%% Copyright (C) 1989-1995 Frank Mittelbach, all rights reserved.
%<+package>\@ifundefined{theorem@style}{}{\endinput}
%<+package>\NeedsTeXFormat{LaTeX2e}[1995/06/01]
%<+package>\ProvidesPackage{theorem}
%<+package>         [\FMithmInfo]
%
% \fi
%
% \changes{v1.0}{87/05/07}{First draft.}
% \changes{v1.0a}{87/05/07}{`thistheoremstyle' added.}
% \changes{v1.1a}{87/07/29}{New concept: redefine the internal
%                          \protect\LaTeX{} commands `@xnthm' etc.}
% \changes{v1.1b}{87/07/31}{`theoremfont' used to specify font.}
% \changes{v1.2a}{88/01/08}{`@endtheorem' now globally defined.}
% \changes{v1.2b}{88/01/11}{`newtheorem' only allowed in the preamble.}
% \changes{v1.3a}{88/02/14}{`theoremfont' now a toks register.}
%
% \changes{v2.0a}{88/11/16}{New implementation and new documentation.}
% \changes{v2.0a}{88/11/16}{No layout changes but internal modifications
%                           old substyles won't work.}
% \changes{v2.0b}{89/01/30}{`global' used for all definitions.}
% \changes{v2.0c}{89/06/15}{Documentation translated to English.
%                            Thanks to LLL and Nina}
% \changes{v2.1a}{90/02/15}{Support for new font selection.}
% \changes{v2.2a}{94/02/03}{Cleanup for LaTeX2e}
%
% \DoNotIndex{\@endparenv,\@ifdefinable,\@ifnextchar,\@ifundefined}
% \DoNotIndex{\@namedef,\@nameuse,\@spaces,\@tempa,\@warning}
% \DoNotIndex{\ }
% \DoNotIndex{\begingroup,\bfseries}
% \DoNotIndex{\csname}
% \DoNotIndex{\def,\do,\docdate}
% \DoNotIndex{\endcsname,\endgroup,\endinput,\endtrivlist,\expandafter}
% \DoNotIndex{\filedate,\fileversion}
% \DoNotIndex{\gdef,\global}
% \DoNotIndex{\hbox,\hskip}
% \DoNotIndex{\ignorespaces,\input,\itshape,\item}
% \DoNotIndex{\labelsep,\let,\llap}
% \DoNotIndex{\makeatletter}
% \DoNotIndex{\noexpand}
% \DoNotIndex{\refstepcounter,\relax,\rlap}
% \DoNotIndex{\setlength,\slshape,\space,\string,\strut}
% \DoNotIndex{\the,\trivlist,\typeout}
% \DoNotIndex{\vbox}
% \DoNotIndex{\xdef}
%
% \def\deflabel#1{\ttfamily #1\hfill}
% \def\deflist#1{\begin{list}{}{\settowidth\labelwidth{\ttfamily #1}%
%                               \setlength\leftmargin\labelwidth
%                               \addtolength\leftmargin\labelsep
%                               \let\makelabel\deflabel}}
% \def\enddeflist{\end{list}}
% \def\envfont{\normalfont\ttfamily}
%
% \MakeShortVerb{\|}
%
% \newcommand{\TUB}{{\sl TUGboat\/}}
%
% \GetFileInfo{theorem.dtx}
% \title{An Extension of the
%        \LaTeX{} theorem environment\thanks{This file
%        has version number \fileversion, last
%        revised \filedate.}}
% \author{Frank Mittelbach}
%
% \MaintainedByLaTeXTeam{tools}
% \maketitle
%
% \begin{abstract}
%    The macros described in this paper yield an extension of the
%    \LaTeX{} theorem mechanism. It is designed to satisfy the
%    different requirements of various journals. Thus, the
%    layout of the ``theorems'' can be manipulated by determining a
%    ``style''. This article describes not only the use, but
%    also the definition, of the necessary macros.
% \end{abstract}
%
% \section*{Preface to version 2.2}
%
% For \LaTeXe{} this package did not need any fundamental changes. I
% only modified the messages generated so that theorem layout styles
% will show up with the |\listfiles| command and cleaned the section
% on the New Font Selection Scheme since this is now included in
% \LaTeX.
%
% \section*{Preface to version 2.1}
%
% This version is identical to 2.0g described in \TUB\ 10\#3 except
% for some internal defaults which are now set depending on the used
% font selection scheme.
%
% This was done to avoid unpleasant surprises if the new font selection
% scheme is in force. For further details see
% section~\ref{sec:newfonts} and \cite{bk:GMS94}.
%
% \section{Introduction}
%
% For our purposes here, ``theorems'' are labelled enunciations,
% often set off from the main text by extra space and a font change.
% Theorems, corollaries, conjectures, definitions, and remarks
% are all instances of ``theorems''.  The ``header'' of these structures
% is composed of a label (such as \textsc{Theorem} or \textsc{Remark})
% and a number which serializes an item in the sequence of items
% with the same label.
%
% Shortly after the introduction of \LaTeX{} at the Fachbereich
% Mathematik in Mainz, the desire to manipulate the layout of
% ``theorems'' arose.  In Mainz, the following two conventions came
% into general use:
% \begin{enumerate}
%   \item The number of the theorem is shown in the margin.
%   \item There is a line break at
%         the end of the theorem header.
% \end{enumerate}
%
% Additionally, some journals require different formats which depend on
% the ``sort of theorem'': e.g.\ often remarks and definitions are set
% in |\upshape|, while |\itshape| is employed for main theorems.
%
% Confronted with these requirements, a theorem environment was
% developed in Mainz which allows separate determination of the
% layout of the ``theorems sets'', comparable to |\pagestyle|.
%
%
% \section{The user interface}
% \subsection{Defining new theorem sets}
%
% \DescribeMacro\newtheorem
% As in the original \LaTeX{} version, the command |\newtheorem|
% defines a new ``theorem set'' or ``theorem-like structure''.
% Two required arguments name the new environment and give the text
% to be typeset with each instance of the new ``set'', while
% an optional argument determines how
% the ``set'' is enumerated:
% \begin{description}
%    \item[\ttfamily \bslash newtheorem\{foo\}\{bar\}]
%       The theorem set {\envfont foo} (whose name is \texttt{bar})
%       uses its own counter.
%    \item[\ttfamily \bslash newtheorem\{foo2\}{[foo]}\{bar2\}]
%       The theorem set {\envfont foo2} (printed name \texttt{bar2})
%       uses the same counter as the theorem set \texttt{foo}.
%    \item[\ttfamily \bslash newtheorem\{foo3\}\{bar3\}{[section]}]
%       The theorem set {\envfont foo3} (printed name \texttt{bar3}) is
%       enumerated within the counter \texttt{section}, i.e.\ with every
%       new |\section| the enumeration begins again with 1, and
%       the enumeration is composed from the section-number and the
%       theorem counter itself.
% \end{description}
%
% \DescribeMacro\theoremstyle
% Additionally, the command |\theoremstyle| can define the layout
% of various, or all, theorem sets.  It should be noted that any
% theorem set defined by |\newtheorem| is typeset in the
% |\theoremstyle| that is current at the time of the definition.
% Thus, the following
% \begin{verbatim}
%    \theoremstyle{break}        \newtheorem{Cor}{Corollary}
%    \theoremstyle{plain}        \newtheorem{Exa}{Example}[section]
%\end{verbatim}                             ^^A No blank at line start
% leads to the result that the set {\envfont Cor} is formatted in the
% style \texttt{break}, while the set {\envfont Exa} and all the
% following ones are formatted in the style \texttt{plain}, unless
% another |\theoremstyle| follows.  Since the definitions installed by
% |\newtheorem| are global, one also can limit |\theoremstyle| locally
% by grouping braces.
%
% \DescribeMacro\theorembodyfont
% The choice of the font for the theorem body is completely
% independent of the chosen |\theoremstyle|; this has proven to
% be very advantageous.  For example,
% \begin{verbatim}
%   {\theorembodyfont{\upshape}          \newtheorem{Rem}{Remark}}
%\end{verbatim}
% defines a theorem set {\envfont Rem}, which will be set in
% |\upshape| in the current layout (which in our example is
% \texttt{plain}).  As with |\theoremstyle|, the |\theorembodyfont|
% chosen is that current at the time of |\newtheorem|.  If
% |\theorembodyfont| is not specified or one defines
% |\theorembodyfont{}|, then the font used will be that defined by the
% |\theoremstyle|.
%
% \DescribeMacro\theoremheaderfont It is also possible to customize
% the font used for the theorem headers.  This is, however, a global
% declaration, and therefore there should be at most one
% |\theoremheaderfont| declaration in the preamble.\footnote{If it is
% actually necessary to have different header fonts, one has to define
% new theorem styles (substituting the desired font) or specify the
% information directly in the \texttt{\bslash newtheorem} declaration
% (the unclean variant).}
%
% \DescribeMacro\theorempreskipamount
% \DescribeMacro\theorempostskipamount Two additional parameters
% affect the vertical space around the theorem environments:
% |\theorempreskipamount| and |\theorempostskipamount| define,
% respectively, the spacing before and after such an environment.
% These parameters apply for all theorem sets and can be manipulated
% with the ordinary length macros.  They are rubber lengths
% (`\textsf{skips}'), and therefore can contain \texttt{plus} and
% \texttt{minus} parts.
%
% Since the definition of theorem sets should---most sensibly---be
% placed in the preamble, we only allow installation there.
% It is therefore possible to release the memory used here after
% |\begin{document}|, in order to make room for other
% applications.
%
%
% \subsection{Existing theorem styles}
%
% The following theorem styles exist to date:
% \begin{deflist}{marginbreak}
%    \item[plain]
%       This theorem style emulates the original \LaTeX{} definition,
%       except that additionally the parameters
%       |\theorem...skipamount| are used.
%    \item[break]
%       In this style, the theorem header is followed by a line
%       break.
%    \item[marginbreak]
%       The theorem number is set in the margin, and there is a line
%       break as in \texttt{break}.
%    \item[changebreak]
%       Like \texttt{break}, but with header number and text
%       interchanged.
%    \item[change]
%       Header number and text are interchanged, without a line break.
%    \item[margin]
%       The number is set in the left margin, without a line break.
% \end{deflist}
% All styles (except \texttt{plain}) select |\slshape| as the default
% |\theorembodyfont|.
%
%
% \subsection{Examples}
%
% Given the above theorem sets {\envfont Cor}, {\envfont Exa}
% and {\envfont Rem},
% suppose that the preamble also contains the declarations:
% \begin{verbatim}
%    \theoremstyle{marginbreak}   \newtheorem{Lem}[Cor]{Lemma}
%    \theoremstyle{change}
%    \theorembodyfont{\itshape}        \newtheorem{Def}[Cor]{Definition}
%
%    \theoremheaderfont{\scshape}
%\end{verbatim}
% Then the following are some typical examples of the typeset output
% resulting from their use.
%
% \begin{Cor}
%    This is a sentence typeset in the theorem environment {\envfont
%    Cor}.
% \end{Cor}
%
% \begin{Exa}
%    This is a sentence typeset in the theorem environment {\envfont
%    Exa}.
% \end{Exa}
%
% \begin{Rem}
%    This is a sentence typeset in the theorem environment {\envfont
%    Rem}.
% \end{Rem}
%
% \begin{Lem}[Ben User]
%    This is a sentence typeset in the theorem environment {\envfont
%    Lem}.
% \end{Lem}
%
% \begin{Def}[Very impressive Definition]
%    This is a sentence typeset in the theorem environment {\envfont
%    Def}.
% \end{Def}
%
% The last two examples show the effect of the optional argument to
% a theorem environment (it is the text typeset in parentheses).
%
%
% \section{Special Considerations}
% \label{sec:newfonts}
%
% Theoremheader and body are implemented as a unit. This means that
% the |\theoremheaderfont|  will inherit characteristics of the
% |\theorembodyfont| in \LaTeXe. Thus, if for example
% |\theorembodyfont| is |\itshape| and |\theoremheaderfont|
% is |\bfseries| the font selected for the header will have the
% characteristics `bold extended italic'. If this is not desired one
% should set the |\theoremheaderfont| to something like
% \begin{verbatim}
%      \theoremheaderfont{\normalfont\bfseries}
%\end{verbatim}
% i.e.\ supplying all necessary font information explicitly.
%
%
%
% \section{Acknowledgements}
%
% The publication of this set of macros was only possible with the
% help of Christina Busse (translating the manuscript into English),
% Joachim Pense (playing the r\^ole of typist), Chris Rowley
% (looking everything over) and many others providing useful
% suggestions.
%
% \MaybeStop{%
%    \begin{thebibliography}{9}
%       \bibitem{bk:GMS94} \textsc{M.~Goossens}, \textsc{F.~Mittelbach}
%         and \textsc{A.~Samarin}.
%         \newblock The \LaTeX{} Companion.
%         \newblock
%         Addison-Wesley, Reading, Massachusetts, 1994.
%       \bibitem{latex:source} \textsc{Lamport, Leslie.} \newblock
%                               \texttt{latex.tex}, version 2.09, date
%                               Feb. 1990.
%    \end{thebibliography}
%    \PrintIndex
%    \PrintChanges}
%
%
% \section{The documentation driver file}
%
% The next bit of code contains the documentation driver file for
% \TeX{}, i.e., the file that will produce the documentation you are
% currently reading. It will be extracted from this file by the
% \texttt{docstrip} program. Since it is the first code in the file
% one can alternatively process this file directly with \LaTeXe{} to
% obtain the documentation.
%
%    \begin{macrocode}
%<*driver>
\documentclass{ltxdoc}

\usepackage{theorem}

%<+driver>% The next few lines define theorem sets which are used
%<+driver>% in the example section of the documentation.

\theoremstyle{break}       \newtheorem{Cor}{Corollary}
\theoremstyle{plain}       \newtheorem{Exa}{Example}[section]
{\theorembodyfont{\upshape}\newtheorem{Rem}{Remark}}
\theoremstyle{marginbreak} \newtheorem{Lem}[Cor]{Lemma}
\theoremstyle{change}
\theorembodyfont{\itshape} \newtheorem{Def}[Cor]{Definition}

\theoremheaderfont{\scshape}

\RecordChanges

\begin{document}
   \DocInput{theorem.dtx}
\end{document}
%</driver>
%    \end{macrocode}
%
% \section{Definition of the Macros}
%
% If the file has been loaded before, we abort immediately. If not the
% package announces itself (this is actually done at the very top of
% the file---the way it is done isn't good style so don't copy it).
% \changes{v2.0e}{89/07/19}{Spaces removed from `typeout'.}
% \changes{v2.2b}{95/11/19}{Announce moved to top of file}
%    \begin{macrocode}
%<*package>
%\@ifundefined{theorem@style}{}{\endinput}
%\def\FMithmInfo{1995/11/19 v2.2b Theorem extension package (FMi)}
%\ProvidesPackage{theorem}[\FMithmInfo]
%    \end{macrocode}
%
%
% \subsection{Definition of theorem styles and fonts}
%
% All the definitions in this file are done globally to allow
% inputting this file inside a group.
%
% \begin{macro}{\theoremstyle}
%    Before a theorem style can be installed, the chosen style must be
%    known. For that reason, we must test to see that
%    |\th@|\meta{style} is known or, more precisely, that it is
%    different from |\relax|. If the style is not known then
%    |\th@plain| is used.
%    \begin{macrocode}
\gdef\theoremstyle#1{%
   \@ifundefined{th@#1}{\@warning
          {Unknown theoremstyle `#1'. Using `plain'}%
          \theorem@style{plain}}%
%    \end{macrocode}
%    We save the theorem style to be used in the token register
%    |\theorem@style|.
%    \begin{macrocode}
      {\theorem@style{#1}}%
%    \end{macrocode}
%    Now we ``evaluate'' the theorem style: this means, we call the
%    macro |\th@|\meta{style} which will activate the relevant
%    definitions which are contained in a separate file.  This is done
%    in a group to suppress changes to the current font.  This could
%    otherwise pose problems together with the new font selection
%    scheme\footnote{When I printed the original article using the new
%    font selection scheme I ended with a document with slanted
%    typefaces (text headings and all) simply because one of the
%    theorem styles used \texttt{\bslash sl} at toplevel.} if the
%    |\th@|\meta{style} is evaluated a second time.
% \changes{v2.1a}{90/02/15}{Group added around `th@..}
% \changes{v2.1a}{90/02/15}{Reseting of `..@begintheorem removed.}
%    \begin{macrocode}
      \begingroup
        \csname th@\the\theorem@style \endcsname
      \endgroup}
%    \end{macrocode}
% \end{macro}
%
%
% \begin{macro}{\@begintheorem}
% \changes{v2.1a}{90/02/15}{Relaxed at toplevel.}
% \begin{macro}{\@opargbegintheorem}
% \changes{v2.1a}{90/02/15}{Relaxed at toplevel.}
%    We reset |\@begintheorem| and |\@opargbegintheorem| to
%    |\relax| since these commands are no longer necessary at
%    toplevel. This will save a few tokens.
%    \begin{macrocode}
\global\let\@begintheorem\relax
\global\let\@opargbegintheorem\relax
%    \end{macrocode}
% \end{macro}
% \end{macro}
%
% \begin{macro}{\theorem@style}
%    Obviously the \textsf{token} register used above has to be
%    allocated. To assure the utmost compatibility with the
%    original \LaTeX{} definition, we set the default theorem style to
%    \texttt{plain}, which implements the usual \LaTeX{} convention.
% \changes{v2.0f}{89/07/20}{Unnecessary `global' removed.}
%    \begin{macrocode}
\newtoks\theorem@style
\global\theorem@style{plain}
%    \end{macrocode}
% \end{macro}
%
% \begin{macro}{\theorembodyfont}
% \begin{macro}{\theorem@bodyfont}
%    If the |\theorembodyfont| is set by the user then it should not
%    interact with the default font set in the theorem style. When the
%    new font selection is in force this may happen if, for example,
%    the default is |\itshape| and the new |\theorembodyfont| is
%    |\sffamily|. So we add a |\reset@font| command in front of the
%    user definition.
% \changes{v2.0f}{89/07/20}{Unnecessary `global' removed.}
% \changes{v2.0g}{89/09/19}{`theoremfont to `theorembodyfont.}
% \changes{v2.1c}{92/06/22}{`reset@font added and logic changed.}
% \changes{v2.1d}{92/10/28}{Reset to empty if arg is empty}
%    \begin{macrocode}
\gdef\theorembodyfont#1{%
%    \end{macrocode}
%    We check if the argument supplied is empty and if so put nothing
%    into the |\theorem@bodyfont| token register to allow for
%    |\theorembodyfont{}| as a mean of using the default of the
%    current |\theoremstyle|.
%    \begin{macrocode}
   \def\@tempa{#1}%
   \ifx\@tempa\@empty
    \theorem@bodyfont{}%
   \else
    \theorem@bodyfont{\reset@font#1}%
   \fi
   }
\newtoks\theorem@bodyfont
\global\theorem@bodyfont{}
%    \end{macrocode}
% \end{macro}
% \end{macro}
%
%
% \begin{macro}{\theoremheaderfont}
% \changes{v2.0g}{89/09/19}{Macro added.}
%    The font for the theorem headers is handled differently because
%    this definition applies to all theorem styles.
%    \begin{macrocode}
\gdef\theoremheaderfont#1{\gdef\theorem@headerfont{#1}%
%    \end{macrocode}
%    After using the macro once it is redefined to produce an error
%    message.
%    \begin{macrocode}
       \gdef\theoremheaderfont##1{%
        \typeout{\string\theoremheaderfont\space should be used
                 only once.}}}
%    \end{macrocode}
% \end{macro}
%
% \begin{macro}{\theorem@headerfont}
% \changes{v2.0g}{89/09/19}{Macro added.}
%    To set the |\theorem@headerfont| default we first test if
%    the new fontselection scheme is in force.
% \changes{v2.1a}{90/02/15}{Support for new font selection.}
%    \begin{macrocode}
\ifx\upshape\undefined
%    \end{macrocode}
%    If not we define it to expand into |\bfseries|. We don't use
%    |\let| just in case a following style option redefines this
%    macro.
%    \begin{macrocode}
\gdef\theorem@headerfont{\bfseries}
%    \end{macrocode}
%    Otherwise we reset the current shape before calling |\bfseries|.
%    \begin{macrocode}
\else \gdef\theorem@headerfont{\normalfont\bfseries}\fi
%    \end{macrocode}
% \end{macro}
%
% \begin{macro}{\th@plain}
% \begin{macro}{\th@break}
% \begin{macro}{\th@marginbreak}
% \begin{macro}{\th@changebreak}
% \begin{macro}{\th@change}
% \begin{macro}{\th@margin}
%    The different styles are defined in macros such as
%    |\th@plain|.  Since memory space is precious in
%    ``non-Big-versions'', we have to avoid offering too many unused
%    definitions.  Therefore we define these styles in separate files
%    that can be loaded on demand.  Thus the commands themselves only
%    load these files. We use |\@input@|, a \LaTeXe{} internal command
%    that ensures that the file will be listed with |\listfiles|.
% \changes{v2.2a}{94/02/02}{Use `@input@ to load theorem layout files}
%    \begin{macrocode}
\gdef\th@plain{\@input@{thp.sty}}
\gdef\th@break{\@input@{thb.sty}}
\gdef\th@marginbreak{\@input@{thmb.sty}}
\gdef\th@changebreak{\@input@{thcb.sty}}
\gdef\th@change{\@input@{thc.sty}}
\gdef\th@margin{\@input@{thm.sty}}
%    \end{macrocode}
%    This list will be expanded when new styles become
%    available.  For testing,  just append new theorem substyles
%    as document options.
% \end{macro}
% \end{macro}
% \end{macro}
% \end{macro}
% \end{macro}
% \end{macro}
%
%
%
% \subsection{Definition of a new theorem set}
%
% As already pointed out, a new theorem environment can be defined in
% three different ways:
% \begin{verbatim}
%    \newtheorem{Lem}{Lemma}
%    \newtheorem{Lem}{Lemma}[section]
%    \newtheorem{Lem}[Theorem]{Lemma}
%\end{verbatim}
% The function of the macro |\newtheorem| is to recognize these cases
% and then to branch into one of the three macros |\@ynthm|, |\@xnthm|
% or |\@othm|.  This mechanism is adopted unchanged from
% \cite{latex:source}; the essential point here is that, for example,
% in the second case, the arguments \texttt{Lem}, \texttt{Lemma} and
% \texttt{section} are passed over to the macro |\@xnthm|.
%
% We inspect this case first because the others present fewer problems,
% and thus are easily derived from this one.
%
% \begin{macro}{\@xnthm}
%    For our example arguments, the macro |\@xnthm| must fulfill
%    the following:
%    \begin{itemize}
%       \item
%          Define a new \LaTeX-counter `Lem'
%       \item
%          reset this counter within a |\section|
%       \item
%          define the macro |\theLem|
%       \item
%          define the environment macros |\Lem| and
%          |\endLem| using the current
%          |\theoremstyle| and |\theorem@bodyfont|.
%    \end{itemize}
%    Obviously, all this should happen only if the first argument
%    of |\@xnthm| (i.e.\ \texttt{Lem} in our example) is chosen so
%    as not to conflict with any previously defined commands or
%    environments.  This test is performed by the \LaTeX{} macro
%    |\@ifdefinable|.
%    \begin{macrocode}
\gdef\@xnthm#1#2[#3]{\expandafter\@ifdefinable\csname #1\endcsname
%    \end{macrocode}
%    Therefore, the first argument of |\@ifdefinable| is the
%    expansion (in the example, |\Lem|) of
%    |\csname||#1\endcsname|.  The second argument is
%    executed only if the test has been completed successfully.
%    \begin{macrocode}
   {%
%    \end{macrocode}
%    Now we define the new counter. The names of the \LaTeX{}
%    macros employed should speak for themselves:
% \changes{v2.2b}{1995/10/07}{Use \cs{@newctr}}
% \changes{v2.2c}{1995/11/23}{Use \cs{@newctr} properly.}
%    \begin{macrocode}
    \@definecounter{#1}\@newctr{#1}[#3]%
%    \end{macrocode}
%    Using |\@newctr| will give a proper error message if the counter
%    in |#3| is not defined.
%    In defining `|\theLem|' we must generate the desired macro
%    name by use of |\expandafter| and |\csname|.
%    \begin{macrocode}
    \expandafter\xdef\csname the#1\endcsname
%    \end{macrocode}
%    An |\xdef| is used in order to make the definition global,
%    and to ensure that it contains the replacement texts of
%    |\@thmcountersep| and |\@thmcounter|.\footnote{These
%    two macros can be defined by the document style.  Their default
%    values produce a `\texttt{.}' as separation and an arabic
%    representation of the number.} However, not everything should be
%    expanded. For example, it saves space to use |\thesection|
%    instead of its---at times---lengthy expansion.
%    \begin{macrocode}
      {\expandafter \noexpand \csname the#3\endcsname
       \@thmcountersep \@thmcounter{#1}}%
%    \end{macrocode}
%    Thus with the defaults of \LaTeX{}, |\theLem| would be
%    replaced by the command sequence |\thesection.\arabic{Lem}|.
%
%    We will now look at the definition of the macro which is executed
%    at the beginning of the actual environment (in our example this
%    macro is |\Lem|). It should be noted that we use an
%    ``|\expandafter| trick'' to expand only certain parts of the
%    replacement text at the time of the definition.
%    \begin{macrocode}
    \def\@tempa{\global\@namedef{#1}}%
    \expandafter \@tempa \expandafter{%
%    \end{macrocode}
%    First, the macro that contains the current definitions of
%    |\@begintheorem| and |\@opargtheorem| should be called
%    up.  The name of this macro---as is already known---has the form
%    |\th@|\meta{theorem\ style}; therefore, it must be called by
%    \begin{macrocode}
      \csname th@\the \theorem@style
%    \end{macrocode}
%    In addition the default theorem font should be changeable, i.e.\
%    we have to insert the contents of |\theorem@bodyfont|. For
%    that reason, we expand even further, beyond |\endcsname|,
%    and thus insert the contents of the \textsf{token} register
%    |\theorem@bodyfont| in the replacement text.
%    \begin{macrocode}
            \expandafter \endcsname \the \theorem@bodyfont
%    \end{macrocode}
%    Now it is time to call the macro |\@thm| which takes over
%    the further processing.  It has two arguments: the current
%    counter name (in our example, \texttt{Lem}), and the text of the
%    label (in our example, \texttt{Lemma}).
%    \begin{macrocode}
     \@thm{#1}{#2}}%
%    \end{macrocode}
%    With this, the `sub-definition' is complete.  The macro
%    |\@endtheorem| ends a theorem environment and is, so far, nothing
%    but an |\endtrivlist|.  (Hence it is defined globally, and not
%    within the theorem styles.\footnote{This has to be changed as
%    soon as theorem styles that change \texttt{\bslash @endtheorem}
%    exist.  In such a case, all existing styles must be changed as
%    well since they will have to reset the macro.}) Therefore, we can
%    set it equivalent to the macro that ends the theorem set (in our
%    example, |\endLem|).  However, if some day theorem styles exist
%    that do change |\@endtheorem|, we would have to use the
%    commented-out line instead.
%    \begin{macrocode}
    \global \expandafter \let \csname end#1\endcsname \@endtheorem
%   \global\@namedef{end#1}{\@endtheorem}%
%    \end{macrocode}
%    With these commands all the required definitions are employed,
%    unless the test |\@ifdefinable| has failed.  Therefore, we
%    end the second argument of this macro and with it the definition
%    of |\@xnthm|.
%    \begin{macrocode}
   }}
%    \end{macrocode}
% \end{macro}
%
% \begin{macro}{\@ynthm}
%    The definition of |\@ynthm| is completely analogous. In
%    this case the new counter that is defined is not reset within
%    another counter; thus the definition of |\the...|
%    is simplified:
%    \begin{macrocode}
\gdef\@ynthm#1#2{\expandafter\@ifdefinable\csname #1\endcsname
   {\@definecounter{#1}%
    \expandafter\xdef\csname the#1\endcsname{\@thmcounter{#1}}%
%    \end{macrocode}
%    The rest of the definition corresponds literally to that of
%    |\@xnthm|:
%    \begin{macrocode}
    \def\@tempa{\global\@namedef{#1}}\expandafter \@tempa
     \expandafter{\csname th@\the \theorem@style \expandafter
     \endcsname \the\theorem@bodyfont \@thm{#1}{#2}}%
    \global \expandafter \let \csname end#1\endcsname \@endtheorem}}
%    \end{macrocode}
% \end{macro}
%
% \begin{macro}{\@othm}
%    The definition of |\@othm| does not contain anything new.
%    \begin{macrocode}
\gdef\@othm#1[#2]#3{%
%    \end{macrocode}
% \changes{v2.2b}{95/10/07}{Build in check for counter defined}
%    We do not define a new counter but instead use one that has
%    already been defined.  Thus the only definition we need is that
%    of this pseudo-counter (i.e.\ |\the|\meta{env.\ name}).
%    First we check if |#2| corresponds to a known counter name.
%    \begin{macrocode}
  \expandafter\ifx\csname c@#2\endcsname\relax
   \@nocounterr{#2}%
  \else
   \expandafter\@ifdefinable\csname #1\endcsname
   {\expandafter \xdef \csname the#1\endcsname
     {\expandafter \noexpand \csname the#2\endcsname}%
%    \end{macrocode}
%    All other parts of the definition can be adopted from
%    |\@xnthm|. We have to remember, though, that in this case
%    the name of the current counter and the theorem label have moved
%    to the second and third arguments.
%    \begin{macrocode}
    \def\@tempa{\global\@namedef{#1}}\expandafter \@tempa
     \expandafter{\csname th@\the \theorem@style \expandafter
     \endcsname \the\theorem@bodyfont \@thm{#2}{#3}}%
    \global \expandafter \let \csname end#1\endcsname \@endtheorem}%
  \fi}
%    \end{macrocode}
% \end{macro}
%
% \subsection{Macros that are employed in a {\envfont theorem}
%    environment}
%
% \begin{macro}{\@thm}
%    The macro |\@thm| has to increase the current counter.  Then,
%    depending on whether the environment has (or does not have) an
%    optional argument, it has to branch into either |\@begintheorem|
%    or |\@opargtheorem|.
%    \begin{macrocode}
\gdef\@thm#1#2{\refstepcounter{#1}%
%    \end{macrocode}
%    Now we start a {\envfont trivlist} environment, and give
%    |\@topsep| and |\@topsepadd| the values of the \textsf{skip}
%    registers |\theorempreskipamount| and |\theorempostskipamount|.
%    The value in |\@topsep| is the vertical space that is inserted by
%    the first (and only) |\item| in our |\trivlist| whilst
%    |\@topsepadd| is inserted by |\@endparenv| at the end of that
%    {\envfont trivlist} environment.  By using these registers, we
%    obtain the desired space around a {\envfont theorem} environment.
%    \begin{macrocode}
   \trivlist
   \@topsep \theorempreskipamount               % used by first \item
   \@topsepadd \theorempostskipamount           % used by \@endparenv
%    \end{macrocode}
%    Now we have to test whether an optional argument has been given.
% \changes{v2.0d}{89/07/18}{Forgotten percent added.}
%    \begin{macrocode}
   \@ifnextchar [%
%    \end{macrocode}
%    If there is an optional argument, we will call |\@ythm|, and
%    move the arguments read back into the input stream.
%    \begin{macrocode}
   {\@ythm{#1}{#2}}%
%    \end{macrocode}
%    If not, we call |\@begintheorem|. Its first argument is the
%    name of the theorem set (hence the second argument of
%    |\@thm|). Its second argument is the macro that produces the
%    current number.
%    \begin{macrocode}
   {\@begintheorem{#2}{\csname the#1\endcsname}\ignorespaces}}
%    \end{macrocode}
% \end{macro}
%
% \begin{macro}{\@xthm}
% \begin{macro}{\@ythm}
%    Both these macros were originally called by |\@thm|.  We do
%    not need |\@xthm| anymore, hence we reset it to
%    |\relax|. The definition of |\@ythm| has not changed at
%    all from its definition in \LaTeX. In order to make the macros
%    easier to understand, we will
%    nevertheless present it (commented out).
%    \begin{macrocode}
\global\let\@xthm\relax
% \def\@ythm#1#2[#3]{\@opargbegintheorem{#2}{\csname
%           the#1\endcsname}{#3}\ignorespaces}
%</package>
%    \end{macrocode}
%    The primitive |\ignorespaces| in |\@ythm| and
%    |\@thm| is needed to remove the spaces between the
%    |\begin{...}| and the actual text.
% \end{macro}
% \end{macro}
%
%
% \subsection{Definition of the theorem substyles}
%
% As already pointed out, the theorem substyles, defined below, are
% only loaded when necessary.  Note that all these substyles, except
% \texttt{plain}, have |\slshape| as the default body font.
%
% \subsubsection{The \texttt{plain} style}
%
% \label{dok:th@plain} As the following macros use |@|, we have
% to locally set the |\catcode| of this symbol to ``letter''.
% This happens within a group, so that we do not have to worry about
% which |\catcode| that symbol had before.
%    \begin{macrocode}
%<*thp>
\begingroup \makeatletter
%    \end{macrocode}
% Since we are now within a group, we must make all definitions
% globally.  First we make sure that \texttt{theorem.sty} is loaded.
% This will allow us to use this file as a document style option
% without having to call \texttt{theorem} itself as an option.  At the
% same time, we assure that at least version 2 is loaded, since
% |\theorem@style| was not defined in earlier versions.
%    \begin{macrocode}
\@ifundefined{theorem@style}{\input{theorem.sty}}{}
\ProvidesFile{thp.sty}
             [\FMithmInfo]
%    \end{macrocode}
%
%
% \begin{macro}{\th@plain}
% \begin{macro}{\@begintheorem}
% \begin{macro}{\@opargbegintheorem}
%    |\theoremstyle{plain}| corresponds to the original
%    definition, except that the distances to the surrounding text are
%    determined by the parameters |\theorempreskipamount| and
%    |\theorempostskipamount|.
%    First we set the default body font.
%    \begin{macrocode}
\gdef\th@plain{\normalfont\itshape
%    \end{macrocode}
%    Then we define |\@begintheorem| and
%    |\@opargbegintheorem|. These two macros define how the
%    header of a theorem is typeset.  |\@opargbegintheorem| will
%    be called if a \textsf{theorem} environment with an optional
%    argument is encountered; otherwise, the header is constructed by
%    calling |\@begintheorem|.  If one of these macros is
%    executed, we are within a \textsf{trivlist} environment started by
%    |\@thm|.  So the theorem header is produced with an
%    |\item| command.
%
%    Instead of specifying the header font directly, all standard
%    theorem styles use the |\theorem@headerfont| macro to allow
%    customization.  The extra space (|\labelsep|) is necessary
%    because of problems in the \textsf{trivlist} environment.
%    \begin{macrocode}
  \def\@begintheorem##1##2{%
        \item[\hskip\labelsep \theorem@headerfont ##1\ ##2]}%
%    \end{macrocode}
%    The definition of |\@opargbegintheorem| is completely
%    analogous. The only difference is the fact that there exists a
%    third argument (which is the optional parameter of the
%    environment and contains additional information about the
%    theorem).  Customarily we enclose it in parentheses.
%    \begin{macrocode}
\def\@opargbegintheorem##1##2##3{%
   \item[\hskip\labelsep \theorem@headerfont ##1\ ##2\ (##3)]}}
%    \end{macrocode}
% \end{macro}
% \end{macro}
% \end{macro}
%
% \noindent
% We conclude with an |\endgroup| to restore the
% |\catcode| of |@|.
%    \begin{macrocode}
\endgroup
%</thp>
%    \end{macrocode}
%
%
% \subsubsection{The \texttt{break} style}
%
% This style option is stored in the file |thb.sty|.
% For the next two lines see the documentation for |\th@plain| on
% page~\pageref{dok:th@plain}.
%    \begin{macrocode}
%<*thb>
\begingroup \makeatletter
\@ifundefined{theorem@style}{\input{theorem.sty}}{}
\ProvidesFile{thb.sty}
             [\FMithmInfo]
%    \end{macrocode}
%
% \begin{macro}{\th@break}
%    |\theoremstyle{break}| produces a line break after the name
%    of the theorem.  The font is |\slshape|.  Hence, we define
%    |\th@break| as follows:
%    \begin{macrocode}
\gdef\th@break{\normalfont\slshape
  \def\@begintheorem##1##2{\item[%
%    \end{macrocode}
%    We run into the following problem: it is not possible to create
%    the header with |\item[|\meta{title}|]| and then start
%    a new line by, for example, |\mbox{}\\|. Such a definition
%    will fail whenever a list environment follows immediately.  With
%    the above construction, the |\mbox{}| causes the switch
%    |@inlabel| (cf.\ definition of |\list| and
%    |\trivlist| in \cite{latex:source}) to be set to \texttt{false}
%    and so the following list will insert additional vertical
%    space (|\topskip|). This is quite annoying. Therefore, we
%    create the line break within the |\item|. In order to ensure
%    that the text will begin at the proper position in the following
%    line, we simply pretend that the label does not take any
%    room.\footnote{This will lead to problems whenever very high
%    symbols occurring in the line tower into the
%    heading. So, something else has to be done here sometime.}
%    \begin{macrocode}
       \rlap{\vbox{\hbox{\hskip \labelsep\theorem@headerfont ##1\ ##2}%
                   \hbox{\strut}}}]}%
%    \end{macrocode}
%    Again, the definition of |\@opargbegintheorem| is
%    completely analogous.
%    \begin{macrocode}
\def\@opargbegintheorem##1##2##3{%
  \item[\rlap{\vbox{\hbox{\hskip \labelsep \theorem@headerfont
                     ##1\ ##2\ (##3)}%
                    \hbox{\strut}}}]}}
\endgroup
%</thb>
%    \end{macrocode}
% \end{macro}
%
%
% \subsubsection{The \texttt{changebreak} style}
%
%    \begin{macrocode}
% This style option is stored in the file |thcb.sty|.
%    \begin{macrocode}
%<*thcb>
\begingroup \makeatletter
\@ifundefined{theorem@style}{\input{theorem.sty}}{}
\ProvidesFile{thcb.sty}
             [\FMithmInfo]
%    \end{macrocode}
%
% \begin{macro}{\th@changebreak}
%    The \texttt{change-break} theorem style is like \texttt{break}
%    but with interchange of theorem name and theorem number.  Thus we
%    define |\th@changebreak| as follows:
%    \begin{macrocode}
\gdef\th@changebreak{\normalfont\slshape
  \def\@begintheorem##1##2{\item
       [\rlap{\vbox{\hbox{\hskip\labelsep \theorem@headerfont ##2\ ##1}%
                    \hbox{\strut}}}]}%
\def\@opargbegintheorem##1##2##3{%
  \item[\rlap{\vbox{\hbox{\hskip\labelsep \theorem@headerfont
                     ##2\ ##1\ (##3)}%
                    \hbox{\strut}}}]}}
\endgroup
%</thcb>
%    \end{macrocode}
% \end{macro}
%
%
% \subsubsection{The \texttt{change} style}
%
% This style option is stored in the file |thc.sty|.
%    \begin{macrocode}
%<*thc>
\begingroup \makeatletter
\@ifundefined{theorem@style}{\input{theorem.sty}}{}
\ProvidesFile{thc.sty}
             [\FMithmInfo]
%    \end{macrocode}
%
% \begin{macro}{\th@change}
%    The \texttt{change} theorem style corresponds to the
%    \texttt{change break} style without a linebreak after the header.
%    To say it in another way, it's the same as the \texttt{plain}
%    style but with number and name interchanged and |\slshape| as the
%    default font.
%    \begin{macrocode}
\gdef\th@change{\normalfont\slshape
  \def\@begintheorem##1##2{\item
       [\hskip\labelsep \theorem@headerfont ##2\ ##1]}%
\def\@opargbegintheorem##1##2##3{%
  \item[\hskip\labelsep \theorem@headerfont ##2\ ##1\ (##3)]}}
\endgroup
%</thc>
%    \end{macrocode}
% \end{macro}
%
%
% \subsubsection{The \texttt{marginbreak} style}
%
% This style option is the one used most often at Mainz.  It is saved
% in the file |thmb.sty|.
%    \begin{macrocode}
%<*thmb>
\begingroup \makeatletter
\@ifundefined{theorem@style}{\input{theorem.sty}}{}
\ProvidesFile{thmb.sty}
             [\FMithmInfo]
%    \end{macrocode}
%
% \begin{macro}{\th@marginbreak}
%    The \texttt{margin break} style is nearly the same as the
%    \texttt{change break} style. The only difference is the placement
%    of the theorem number.  We use |\llap| to place it in the left
%    margin.
%
%    In this style |\labelsep| denotes the separation between the
%    number and the text.
%    \begin{macrocode}
\gdef\th@marginbreak{\normalfont\slshape
  \def\@begintheorem##1##2{\item
       [\rlap{\vbox{\theorem@headerfont
                    \hbox{\llap{##2}\hskip\labelsep ##1}%
                    \hbox{\strut}}}]}%
\def\@opargbegintheorem##1##2##3{%
  \item[\rlap{\vbox{\theorem@headerfont
                    \hbox{\llap{##2}\hskip\labelsep ##1\ (##3)}%
                    \hbox{\strut}}}]}}
\endgroup
%</thmb>
%    \end{macrocode}
% \changes{v2.1c}{90/02/17}{`endgroup outside macrocode env.}
% \end{macro}
%
%
%
% \subsubsection{The \texttt{margin} style}
%
% This style option is stored in the file |thm.sty|.
%    \begin{macrocode}
%<*thm>
\begingroup \makeatletter
\@ifundefined{theorem@style}{\input{theorem.sty}}{}
\ProvidesFile{thm.sty}
             [\FMithmInfo]
%    \end{macrocode}
%
% \begin{macro}{\th@margin}
%    Again this is only a variant of the theorem styles described
%    above without any new ideas.
%    \begin{macrocode}
\gdef\th@margin{\normalfont\slshape
  \def\@begintheorem##1##2{\item
       [\theorem@headerfont \llap{##2}\hskip\labelsep ##1]}%
\def\@opargbegintheorem##1##2##3{%
  \item[\theorem@headerfont \llap{##2}\hskip\labelsep ##1\ (##3)]}}
\endgroup
%</thm>
%    \end{macrocode}
% \end{macro}
%
%
%
%
% \subsection{Final Definitions}
%
% \begin{macro}{\theorempreskipamount}
% \begin{macro}{\theorempostskipamount}
%    The \textsf{skip} parameters that regulate the vertical empty space
%    before and after the {\envfont theorem} environment have
%    to be allocated as well.
% \changes{v2.0f}{89/07/20}{Unnecessary `global' removed.}
%    \begin{macrocode}
%<*package>
\newskip\theorempreskipamount
\newskip\theorempostskipamount
%    \end{macrocode}
%    Since we have used the same values for all theorem sets, we now can
%    assign them.
%    \begin{macrocode}
\global\setlength\theorempreskipamount{12pt plus 5pt minus 3pt}
\global\setlength\theorempostskipamount{8pt plus 3pt minus 1.5pt}
%    \end{macrocode}
% \end{macro}
% \end{macro}
%
%
% \begin{macro}{\@endtheorem}
%    The same holds for the macro |\@endtheorem|, which ends a
%    {\envfont theorem} environment. Since it is the same for all
%    theorem sets, it is removed from the macros |\th@|\meta{style}.
%    It simply ends the {\envfont trivlist} environment, which was
%    begun in |\@thm|.
%    \begin{macrocode}
\global\let\@endtheorem=\endtrivlist
%    \end{macrocode}
% \end{macro}
%
%
%
% \begin{macro}{\@preamblecmds}
%    All macros defined above are to be used only in the preamble.
%    Therefore, we insert them in |\@preamblecmds| which will disable
%    them at begin document. This is done by the internal \LaTeXe{}
%    command |\@onlypreamble|.
% \changes{v2.0g}{89/09/19}{`theoremheaderfont and `theorembodyfont
%    added.}
% \changes{v2.2a}{94/02/03}{Use \LaTeXe{} interface}
%    \begin{macrocode}
\@onlypreamble\@xnthm
\@onlypreamble\@ynthm
\@onlypreamble\@othm
\@onlypreamble\newtheorem
\@onlypreamble\theoremstyle
\@onlypreamble\theorembodyfont
\@onlypreamble\theoremheaderfont
%    \end{macrocode}
% \end{macro}
%
% Finally we declare the \texttt{plain} theorem style to be the default.
% \changes{v2.1b}{90/06/04}{Fixed missing `item by declaring default
%    style.}
%    \begin{macrocode}
\theoremstyle{plain}
%</package>
%    \end{macrocode}
%
% \Finale

\endinput
}{}
\ProvidesFile{thcb.sty}
             [\FMithmInfo]
%    \end{macrocode}
%
% \begin{macro}{\th@changebreak}
%    The \texttt{change-break} theorem style is like \texttt{break}
%    but with interchange of theorem name and theorem number.  Thus we
%    define |\th@changebreak| as follows:
%    \begin{macrocode}
\gdef\th@changebreak{\normalfont\slshape
  \def\@begintheorem##1##2{\item
       [\rlap{\vbox{\hbox{\hskip\labelsep \theorem@headerfont ##2\ ##1}%
                    \hbox{\strut}}}]}%
\def\@opargbegintheorem##1##2##3{%
  \item[\rlap{\vbox{\hbox{\hskip\labelsep \theorem@headerfont
                     ##2\ ##1\ (##3)}%
                    \hbox{\strut}}}]}}
\endgroup
%</thcb>
%    \end{macrocode}
% \end{macro}
%
%
% \subsubsection{The \texttt{change} style}
%
% This style option is stored in the file |thc.sty|.
%    \begin{macrocode}
%<*thc>
\begingroup \makeatletter
\@ifundefined{theorem@style}{% \iffalse meta-comment
%
% Copyright (C) 1993-2024
%
% The LaTeX Project and any individual authors listed elsewhere
% in this file.
%
% This file is part of the Standard LaTeX `Tools Bundle'.
% -------------------------------------------------------
%
% It may be distributed and/or modified under the
% conditions of the LaTeX Project Public License, either version 1.3c
% of this license or (at your option) any later version.
% The latest version of this license is in
%    https://www.latex-project.org/lppl.txt
% and version 1.3c or later is part of all distributions of LaTeX
% version 2005/12/01 or later.
%
% The list of all files belonging to the LaTeX `Tools Bundle' is
% given in the file `manifest.txt'.
%
% \fi
%
%%
%
%
\def\FMithmInfo{2023/07/05 v2.2c Theorem extension package (FMi)}
%
% \ProvidesFile{theorem.dtx}[\FMithmInfo]
%\iffalse   % this is a METACOMMENT !
%%
%% Package `theorem' to use with LaTeX2e
%% Copyright (C) 1989-1995 Frank Mittelbach, all rights reserved.
%<+package>\@ifundefined{theorem@style}{}{\endinput}
%<+package>\NeedsTeXFormat{LaTeX2e}[1995/06/01]
%<+package>\ProvidesPackage{theorem}
%<+package>         [\FMithmInfo]
%
% \fi
%
% \changes{v1.0}{87/05/07}{First draft.}
% \changes{v1.0a}{87/05/07}{`thistheoremstyle' added.}
% \changes{v1.1a}{87/07/29}{New concept: redefine the internal
%                          \protect\LaTeX{} commands `@xnthm' etc.}
% \changes{v1.1b}{87/07/31}{`theoremfont' used to specify font.}
% \changes{v1.2a}{88/01/08}{`@endtheorem' now globally defined.}
% \changes{v1.2b}{88/01/11}{`newtheorem' only allowed in the preamble.}
% \changes{v1.3a}{88/02/14}{`theoremfont' now a toks register.}
%
% \changes{v2.0a}{88/11/16}{New implementation and new documentation.}
% \changes{v2.0a}{88/11/16}{No layout changes but internal modifications
%                           old substyles won't work.}
% \changes{v2.0b}{89/01/30}{`global' used for all definitions.}
% \changes{v2.0c}{89/06/15}{Documentation translated to English.
%                            Thanks to LLL and Nina}
% \changes{v2.1a}{90/02/15}{Support for new font selection.}
% \changes{v2.2a}{94/02/03}{Cleanup for LaTeX2e}
%
% \DoNotIndex{\@endparenv,\@ifdefinable,\@ifnextchar,\@ifundefined}
% \DoNotIndex{\@namedef,\@nameuse,\@spaces,\@tempa,\@warning}
% \DoNotIndex{\ }
% \DoNotIndex{\begingroup,\bfseries}
% \DoNotIndex{\csname}
% \DoNotIndex{\def,\do,\docdate}
% \DoNotIndex{\endcsname,\endgroup,\endinput,\endtrivlist,\expandafter}
% \DoNotIndex{\filedate,\fileversion}
% \DoNotIndex{\gdef,\global}
% \DoNotIndex{\hbox,\hskip}
% \DoNotIndex{\ignorespaces,\input,\itshape,\item}
% \DoNotIndex{\labelsep,\let,\llap}
% \DoNotIndex{\makeatletter}
% \DoNotIndex{\noexpand}
% \DoNotIndex{\refstepcounter,\relax,\rlap}
% \DoNotIndex{\setlength,\slshape,\space,\string,\strut}
% \DoNotIndex{\the,\trivlist,\typeout}
% \DoNotIndex{\vbox}
% \DoNotIndex{\xdef}
%
% \def\deflabel#1{\ttfamily #1\hfill}
% \def\deflist#1{\begin{list}{}{\settowidth\labelwidth{\ttfamily #1}%
%                               \setlength\leftmargin\labelwidth
%                               \addtolength\leftmargin\labelsep
%                               \let\makelabel\deflabel}}
% \def\enddeflist{\end{list}}
% \def\envfont{\normalfont\ttfamily}
%
% \MakeShortVerb{\|}
%
% \newcommand{\TUB}{{\sl TUGboat\/}}
%
% \GetFileInfo{theorem.dtx}
% \title{An Extension of the
%        \LaTeX{} theorem environment\thanks{This file
%        has version number \fileversion, last
%        revised \filedate.}}
% \author{Frank Mittelbach}
%
% \MaintainedByLaTeXTeam{tools}
% \maketitle
%
% \begin{abstract}
%    The macros described in this paper yield an extension of the
%    \LaTeX{} theorem mechanism. It is designed to satisfy the
%    different requirements of various journals. Thus, the
%    layout of the ``theorems'' can be manipulated by determining a
%    ``style''. This article describes not only the use, but
%    also the definition, of the necessary macros.
% \end{abstract}
%
% \section*{Preface to version 2.2}
%
% For \LaTeXe{} this package did not need any fundamental changes. I
% only modified the messages generated so that theorem layout styles
% will show up with the |\listfiles| command and cleaned the section
% on the New Font Selection Scheme since this is now included in
% \LaTeX.
%
% \section*{Preface to version 2.1}
%
% This version is identical to 2.0g described in \TUB\ 10\#3 except
% for some internal defaults which are now set depending on the used
% font selection scheme.
%
% This was done to avoid unpleasant surprises if the new font selection
% scheme is in force. For further details see
% section~\ref{sec:newfonts} and \cite{bk:GMS94}.
%
% \section{Introduction}
%
% For our purposes here, ``theorems'' are labelled enunciations,
% often set off from the main text by extra space and a font change.
% Theorems, corollaries, conjectures, definitions, and remarks
% are all instances of ``theorems''.  The ``header'' of these structures
% is composed of a label (such as \textsc{Theorem} or \textsc{Remark})
% and a number which serializes an item in the sequence of items
% with the same label.
%
% Shortly after the introduction of \LaTeX{} at the Fachbereich
% Mathematik in Mainz, the desire to manipulate the layout of
% ``theorems'' arose.  In Mainz, the following two conventions came
% into general use:
% \begin{enumerate}
%   \item The number of the theorem is shown in the margin.
%   \item There is a line break at
%         the end of the theorem header.
% \end{enumerate}
%
% Additionally, some journals require different formats which depend on
% the ``sort of theorem'': e.g.\ often remarks and definitions are set
% in |\upshape|, while |\itshape| is employed for main theorems.
%
% Confronted with these requirements, a theorem environment was
% developed in Mainz which allows separate determination of the
% layout of the ``theorems sets'', comparable to |\pagestyle|.
%
%
% \section{The user interface}
% \subsection{Defining new theorem sets}
%
% \DescribeMacro\newtheorem
% As in the original \LaTeX{} version, the command |\newtheorem|
% defines a new ``theorem set'' or ``theorem-like structure''.
% Two required arguments name the new environment and give the text
% to be typeset with each instance of the new ``set'', while
% an optional argument determines how
% the ``set'' is enumerated:
% \begin{description}
%    \item[\ttfamily \bslash newtheorem\{foo\}\{bar\}]
%       The theorem set {\envfont foo} (whose name is \texttt{bar})
%       uses its own counter.
%    \item[\ttfamily \bslash newtheorem\{foo2\}{[foo]}\{bar2\}]
%       The theorem set {\envfont foo2} (printed name \texttt{bar2})
%       uses the same counter as the theorem set \texttt{foo}.
%    \item[\ttfamily \bslash newtheorem\{foo3\}\{bar3\}{[section]}]
%       The theorem set {\envfont foo3} (printed name \texttt{bar3}) is
%       enumerated within the counter \texttt{section}, i.e.\ with every
%       new |\section| the enumeration begins again with 1, and
%       the enumeration is composed from the section-number and the
%       theorem counter itself.
% \end{description}
%
% \DescribeMacro\theoremstyle
% Additionally, the command |\theoremstyle| can define the layout
% of various, or all, theorem sets.  It should be noted that any
% theorem set defined by |\newtheorem| is typeset in the
% |\theoremstyle| that is current at the time of the definition.
% Thus, the following
% \begin{verbatim}
%    \theoremstyle{break}        \newtheorem{Cor}{Corollary}
%    \theoremstyle{plain}        \newtheorem{Exa}{Example}[section]
%\end{verbatim}                             ^^A No blank at line start
% leads to the result that the set {\envfont Cor} is formatted in the
% style \texttt{break}, while the set {\envfont Exa} and all the
% following ones are formatted in the style \texttt{plain}, unless
% another |\theoremstyle| follows.  Since the definitions installed by
% |\newtheorem| are global, one also can limit |\theoremstyle| locally
% by grouping braces.
%
% \DescribeMacro\theorembodyfont
% The choice of the font for the theorem body is completely
% independent of the chosen |\theoremstyle|; this has proven to
% be very advantageous.  For example,
% \begin{verbatim}
%   {\theorembodyfont{\upshape}          \newtheorem{Rem}{Remark}}
%\end{verbatim}
% defines a theorem set {\envfont Rem}, which will be set in
% |\upshape| in the current layout (which in our example is
% \texttt{plain}).  As with |\theoremstyle|, the |\theorembodyfont|
% chosen is that current at the time of |\newtheorem|.  If
% |\theorembodyfont| is not specified or one defines
% |\theorembodyfont{}|, then the font used will be that defined by the
% |\theoremstyle|.
%
% \DescribeMacro\theoremheaderfont It is also possible to customize
% the font used for the theorem headers.  This is, however, a global
% declaration, and therefore there should be at most one
% |\theoremheaderfont| declaration in the preamble.\footnote{If it is
% actually necessary to have different header fonts, one has to define
% new theorem styles (substituting the desired font) or specify the
% information directly in the \texttt{\bslash newtheorem} declaration
% (the unclean variant).}
%
% \DescribeMacro\theorempreskipamount
% \DescribeMacro\theorempostskipamount Two additional parameters
% affect the vertical space around the theorem environments:
% |\theorempreskipamount| and |\theorempostskipamount| define,
% respectively, the spacing before and after such an environment.
% These parameters apply for all theorem sets and can be manipulated
% with the ordinary length macros.  They are rubber lengths
% (`\textsf{skips}'), and therefore can contain \texttt{plus} and
% \texttt{minus} parts.
%
% Since the definition of theorem sets should---most sensibly---be
% placed in the preamble, we only allow installation there.
% It is therefore possible to release the memory used here after
% |\begin{document}|, in order to make room for other
% applications.
%
%
% \subsection{Existing theorem styles}
%
% The following theorem styles exist to date:
% \begin{deflist}{marginbreak}
%    \item[plain]
%       This theorem style emulates the original \LaTeX{} definition,
%       except that additionally the parameters
%       |\theorem...skipamount| are used.
%    \item[break]
%       In this style, the theorem header is followed by a line
%       break.
%    \item[marginbreak]
%       The theorem number is set in the margin, and there is a line
%       break as in \texttt{break}.
%    \item[changebreak]
%       Like \texttt{break}, but with header number and text
%       interchanged.
%    \item[change]
%       Header number and text are interchanged, without a line break.
%    \item[margin]
%       The number is set in the left margin, without a line break.
% \end{deflist}
% All styles (except \texttt{plain}) select |\slshape| as the default
% |\theorembodyfont|.
%
%
% \subsection{Examples}
%
% Given the above theorem sets {\envfont Cor}, {\envfont Exa}
% and {\envfont Rem},
% suppose that the preamble also contains the declarations:
% \begin{verbatim}
%    \theoremstyle{marginbreak}   \newtheorem{Lem}[Cor]{Lemma}
%    \theoremstyle{change}
%    \theorembodyfont{\itshape}        \newtheorem{Def}[Cor]{Definition}
%
%    \theoremheaderfont{\scshape}
%\end{verbatim}
% Then the following are some typical examples of the typeset output
% resulting from their use.
%
% \begin{Cor}
%    This is a sentence typeset in the theorem environment {\envfont
%    Cor}.
% \end{Cor}
%
% \begin{Exa}
%    This is a sentence typeset in the theorem environment {\envfont
%    Exa}.
% \end{Exa}
%
% \begin{Rem}
%    This is a sentence typeset in the theorem environment {\envfont
%    Rem}.
% \end{Rem}
%
% \begin{Lem}[Ben User]
%    This is a sentence typeset in the theorem environment {\envfont
%    Lem}.
% \end{Lem}
%
% \begin{Def}[Very impressive Definition]
%    This is a sentence typeset in the theorem environment {\envfont
%    Def}.
% \end{Def}
%
% The last two examples show the effect of the optional argument to
% a theorem environment (it is the text typeset in parentheses).
%
%
% \section{Special Considerations}
% \label{sec:newfonts}
%
% Theoremheader and body are implemented as a unit. This means that
% the |\theoremheaderfont|  will inherit characteristics of the
% |\theorembodyfont| in \LaTeXe. Thus, if for example
% |\theorembodyfont| is |\itshape| and |\theoremheaderfont|
% is |\bfseries| the font selected for the header will have the
% characteristics `bold extended italic'. If this is not desired one
% should set the |\theoremheaderfont| to something like
% \begin{verbatim}
%      \theoremheaderfont{\normalfont\bfseries}
%\end{verbatim}
% i.e.\ supplying all necessary font information explicitly.
%
%
%
% \section{Acknowledgements}
%
% The publication of this set of macros was only possible with the
% help of Christina Busse (translating the manuscript into English),
% Joachim Pense (playing the r\^ole of typist), Chris Rowley
% (looking everything over) and many others providing useful
% suggestions.
%
% \MaybeStop{%
%    \begin{thebibliography}{9}
%       \bibitem{bk:GMS94} \textsc{M.~Goossens}, \textsc{F.~Mittelbach}
%         and \textsc{A.~Samarin}.
%         \newblock The \LaTeX{} Companion.
%         \newblock
%         Addison-Wesley, Reading, Massachusetts, 1994.
%       \bibitem{latex:source} \textsc{Lamport, Leslie.} \newblock
%                               \texttt{latex.tex}, version 2.09, date
%                               Feb. 1990.
%    \end{thebibliography}
%    \PrintIndex
%    \PrintChanges}
%
%
% \section{The documentation driver file}
%
% The next bit of code contains the documentation driver file for
% \TeX{}, i.e., the file that will produce the documentation you are
% currently reading. It will be extracted from this file by the
% \texttt{docstrip} program. Since it is the first code in the file
% one can alternatively process this file directly with \LaTeXe{} to
% obtain the documentation.
%
%    \begin{macrocode}
%<*driver>
\documentclass{ltxdoc}

\usepackage{theorem}

%<+driver>% The next few lines define theorem sets which are used
%<+driver>% in the example section of the documentation.

\theoremstyle{break}       \newtheorem{Cor}{Corollary}
\theoremstyle{plain}       \newtheorem{Exa}{Example}[section]
{\theorembodyfont{\upshape}\newtheorem{Rem}{Remark}}
\theoremstyle{marginbreak} \newtheorem{Lem}[Cor]{Lemma}
\theoremstyle{change}
\theorembodyfont{\itshape} \newtheorem{Def}[Cor]{Definition}

\theoremheaderfont{\scshape}

\RecordChanges

\begin{document}
   \DocInput{theorem.dtx}
\end{document}
%</driver>
%    \end{macrocode}
%
% \section{Definition of the Macros}
%
% If the file has been loaded before, we abort immediately. If not the
% package announces itself (this is actually done at the very top of
% the file---the way it is done isn't good style so don't copy it).
% \changes{v2.0e}{89/07/19}{Spaces removed from `typeout'.}
% \changes{v2.2b}{95/11/19}{Announce moved to top of file}
%    \begin{macrocode}
%<*package>
%\@ifundefined{theorem@style}{}{\endinput}
%\def\FMithmInfo{1995/11/19 v2.2b Theorem extension package (FMi)}
%\ProvidesPackage{theorem}[\FMithmInfo]
%    \end{macrocode}
%
%
% \subsection{Definition of theorem styles and fonts}
%
% All the definitions in this file are done globally to allow
% inputting this file inside a group.
%
% \begin{macro}{\theoremstyle}
%    Before a theorem style can be installed, the chosen style must be
%    known. For that reason, we must test to see that
%    |\th@|\meta{style} is known or, more precisely, that it is
%    different from |\relax|. If the style is not known then
%    |\th@plain| is used.
%    \begin{macrocode}
\gdef\theoremstyle#1{%
   \@ifundefined{th@#1}{\@warning
          {Unknown theoremstyle `#1'. Using `plain'}%
          \theorem@style{plain}}%
%    \end{macrocode}
%    We save the theorem style to be used in the token register
%    |\theorem@style|.
%    \begin{macrocode}
      {\theorem@style{#1}}%
%    \end{macrocode}
%    Now we ``evaluate'' the theorem style: this means, we call the
%    macro |\th@|\meta{style} which will activate the relevant
%    definitions which are contained in a separate file.  This is done
%    in a group to suppress changes to the current font.  This could
%    otherwise pose problems together with the new font selection
%    scheme\footnote{When I printed the original article using the new
%    font selection scheme I ended with a document with slanted
%    typefaces (text headings and all) simply because one of the
%    theorem styles used \texttt{\bslash sl} at toplevel.} if the
%    |\th@|\meta{style} is evaluated a second time.
% \changes{v2.1a}{90/02/15}{Group added around `th@..}
% \changes{v2.1a}{90/02/15}{Reseting of `..@begintheorem removed.}
%    \begin{macrocode}
      \begingroup
        \csname th@\the\theorem@style \endcsname
      \endgroup}
%    \end{macrocode}
% \end{macro}
%
%
% \begin{macro}{\@begintheorem}
% \changes{v2.1a}{90/02/15}{Relaxed at toplevel.}
% \begin{macro}{\@opargbegintheorem}
% \changes{v2.1a}{90/02/15}{Relaxed at toplevel.}
%    We reset |\@begintheorem| and |\@opargbegintheorem| to
%    |\relax| since these commands are no longer necessary at
%    toplevel. This will save a few tokens.
%    \begin{macrocode}
\global\let\@begintheorem\relax
\global\let\@opargbegintheorem\relax
%    \end{macrocode}
% \end{macro}
% \end{macro}
%
% \begin{macro}{\theorem@style}
%    Obviously the \textsf{token} register used above has to be
%    allocated. To assure the utmost compatibility with the
%    original \LaTeX{} definition, we set the default theorem style to
%    \texttt{plain}, which implements the usual \LaTeX{} convention.
% \changes{v2.0f}{89/07/20}{Unnecessary `global' removed.}
%    \begin{macrocode}
\newtoks\theorem@style
\global\theorem@style{plain}
%    \end{macrocode}
% \end{macro}
%
% \begin{macro}{\theorembodyfont}
% \begin{macro}{\theorem@bodyfont}
%    If the |\theorembodyfont| is set by the user then it should not
%    interact with the default font set in the theorem style. When the
%    new font selection is in force this may happen if, for example,
%    the default is |\itshape| and the new |\theorembodyfont| is
%    |\sffamily|. So we add a |\reset@font| command in front of the
%    user definition.
% \changes{v2.0f}{89/07/20}{Unnecessary `global' removed.}
% \changes{v2.0g}{89/09/19}{`theoremfont to `theorembodyfont.}
% \changes{v2.1c}{92/06/22}{`reset@font added and logic changed.}
% \changes{v2.1d}{92/10/28}{Reset to empty if arg is empty}
%    \begin{macrocode}
\gdef\theorembodyfont#1{%
%    \end{macrocode}
%    We check if the argument supplied is empty and if so put nothing
%    into the |\theorem@bodyfont| token register to allow for
%    |\theorembodyfont{}| as a mean of using the default of the
%    current |\theoremstyle|.
%    \begin{macrocode}
   \def\@tempa{#1}%
   \ifx\@tempa\@empty
    \theorem@bodyfont{}%
   \else
    \theorem@bodyfont{\reset@font#1}%
   \fi
   }
\newtoks\theorem@bodyfont
\global\theorem@bodyfont{}
%    \end{macrocode}
% \end{macro}
% \end{macro}
%
%
% \begin{macro}{\theoremheaderfont}
% \changes{v2.0g}{89/09/19}{Macro added.}
%    The font for the theorem headers is handled differently because
%    this definition applies to all theorem styles.
%    \begin{macrocode}
\gdef\theoremheaderfont#1{\gdef\theorem@headerfont{#1}%
%    \end{macrocode}
%    After using the macro once it is redefined to produce an error
%    message.
%    \begin{macrocode}
       \gdef\theoremheaderfont##1{%
        \typeout{\string\theoremheaderfont\space should be used
                 only once.}}}
%    \end{macrocode}
% \end{macro}
%
% \begin{macro}{\theorem@headerfont}
% \changes{v2.0g}{89/09/19}{Macro added.}
%    To set the |\theorem@headerfont| default we first test if
%    the new fontselection scheme is in force.
% \changes{v2.1a}{90/02/15}{Support for new font selection.}
%    \begin{macrocode}
\ifx\upshape\undefined
%    \end{macrocode}
%    If not we define it to expand into |\bfseries|. We don't use
%    |\let| just in case a following style option redefines this
%    macro.
%    \begin{macrocode}
\gdef\theorem@headerfont{\bfseries}
%    \end{macrocode}
%    Otherwise we reset the current shape before calling |\bfseries|.
%    \begin{macrocode}
\else \gdef\theorem@headerfont{\normalfont\bfseries}\fi
%    \end{macrocode}
% \end{macro}
%
% \begin{macro}{\th@plain}
% \begin{macro}{\th@break}
% \begin{macro}{\th@marginbreak}
% \begin{macro}{\th@changebreak}
% \begin{macro}{\th@change}
% \begin{macro}{\th@margin}
%    The different styles are defined in macros such as
%    |\th@plain|.  Since memory space is precious in
%    ``non-Big-versions'', we have to avoid offering too many unused
%    definitions.  Therefore we define these styles in separate files
%    that can be loaded on demand.  Thus the commands themselves only
%    load these files. We use |\@input@|, a \LaTeXe{} internal command
%    that ensures that the file will be listed with |\listfiles|.
% \changes{v2.2a}{94/02/02}{Use `@input@ to load theorem layout files}
%    \begin{macrocode}
\gdef\th@plain{\@input@{thp.sty}}
\gdef\th@break{\@input@{thb.sty}}
\gdef\th@marginbreak{\@input@{thmb.sty}}
\gdef\th@changebreak{\@input@{thcb.sty}}
\gdef\th@change{\@input@{thc.sty}}
\gdef\th@margin{\@input@{thm.sty}}
%    \end{macrocode}
%    This list will be expanded when new styles become
%    available.  For testing,  just append new theorem substyles
%    as document options.
% \end{macro}
% \end{macro}
% \end{macro}
% \end{macro}
% \end{macro}
% \end{macro}
%
%
%
% \subsection{Definition of a new theorem set}
%
% As already pointed out, a new theorem environment can be defined in
% three different ways:
% \begin{verbatim}
%    \newtheorem{Lem}{Lemma}
%    \newtheorem{Lem}{Lemma}[section]
%    \newtheorem{Lem}[Theorem]{Lemma}
%\end{verbatim}
% The function of the macro |\newtheorem| is to recognize these cases
% and then to branch into one of the three macros |\@ynthm|, |\@xnthm|
% or |\@othm|.  This mechanism is adopted unchanged from
% \cite{latex:source}; the essential point here is that, for example,
% in the second case, the arguments \texttt{Lem}, \texttt{Lemma} and
% \texttt{section} are passed over to the macro |\@xnthm|.
%
% We inspect this case first because the others present fewer problems,
% and thus are easily derived from this one.
%
% \begin{macro}{\@xnthm}
%    For our example arguments, the macro |\@xnthm| must fulfill
%    the following:
%    \begin{itemize}
%       \item
%          Define a new \LaTeX-counter `Lem'
%       \item
%          reset this counter within a |\section|
%       \item
%          define the macro |\theLem|
%       \item
%          define the environment macros |\Lem| and
%          |\endLem| using the current
%          |\theoremstyle| and |\theorem@bodyfont|.
%    \end{itemize}
%    Obviously, all this should happen only if the first argument
%    of |\@xnthm| (i.e.\ \texttt{Lem} in our example) is chosen so
%    as not to conflict with any previously defined commands or
%    environments.  This test is performed by the \LaTeX{} macro
%    |\@ifdefinable|.
%    \begin{macrocode}
\gdef\@xnthm#1#2[#3]{\expandafter\@ifdefinable\csname #1\endcsname
%    \end{macrocode}
%    Therefore, the first argument of |\@ifdefinable| is the
%    expansion (in the example, |\Lem|) of
%    |\csname||#1\endcsname|.  The second argument is
%    executed only if the test has been completed successfully.
%    \begin{macrocode}
   {%
%    \end{macrocode}
%    Now we define the new counter. The names of the \LaTeX{}
%    macros employed should speak for themselves:
% \changes{v2.2b}{1995/10/07}{Use \cs{@newctr}}
% \changes{v2.2c}{1995/11/23}{Use \cs{@newctr} properly.}
%    \begin{macrocode}
    \@definecounter{#1}\@newctr{#1}[#3]%
%    \end{macrocode}
%    Using |\@newctr| will give a proper error message if the counter
%    in |#3| is not defined.
%    In defining `|\theLem|' we must generate the desired macro
%    name by use of |\expandafter| and |\csname|.
%    \begin{macrocode}
    \expandafter\xdef\csname the#1\endcsname
%    \end{macrocode}
%    An |\xdef| is used in order to make the definition global,
%    and to ensure that it contains the replacement texts of
%    |\@thmcountersep| and |\@thmcounter|.\footnote{These
%    two macros can be defined by the document style.  Their default
%    values produce a `\texttt{.}' as separation and an arabic
%    representation of the number.} However, not everything should be
%    expanded. For example, it saves space to use |\thesection|
%    instead of its---at times---lengthy expansion.
%    \begin{macrocode}
      {\expandafter \noexpand \csname the#3\endcsname
       \@thmcountersep \@thmcounter{#1}}%
%    \end{macrocode}
%    Thus with the defaults of \LaTeX{}, |\theLem| would be
%    replaced by the command sequence |\thesection.\arabic{Lem}|.
%
%    We will now look at the definition of the macro which is executed
%    at the beginning of the actual environment (in our example this
%    macro is |\Lem|). It should be noted that we use an
%    ``|\expandafter| trick'' to expand only certain parts of the
%    replacement text at the time of the definition.
%    \begin{macrocode}
    \def\@tempa{\global\@namedef{#1}}%
    \expandafter \@tempa \expandafter{%
%    \end{macrocode}
%    First, the macro that contains the current definitions of
%    |\@begintheorem| and |\@opargtheorem| should be called
%    up.  The name of this macro---as is already known---has the form
%    |\th@|\meta{theorem\ style}; therefore, it must be called by
%    \begin{macrocode}
      \csname th@\the \theorem@style
%    \end{macrocode}
%    In addition the default theorem font should be changeable, i.e.\
%    we have to insert the contents of |\theorem@bodyfont|. For
%    that reason, we expand even further, beyond |\endcsname|,
%    and thus insert the contents of the \textsf{token} register
%    |\theorem@bodyfont| in the replacement text.
%    \begin{macrocode}
            \expandafter \endcsname \the \theorem@bodyfont
%    \end{macrocode}
%    Now it is time to call the macro |\@thm| which takes over
%    the further processing.  It has two arguments: the current
%    counter name (in our example, \texttt{Lem}), and the text of the
%    label (in our example, \texttt{Lemma}).
%    \begin{macrocode}
     \@thm{#1}{#2}}%
%    \end{macrocode}
%    With this, the `sub-definition' is complete.  The macro
%    |\@endtheorem| ends a theorem environment and is, so far, nothing
%    but an |\endtrivlist|.  (Hence it is defined globally, and not
%    within the theorem styles.\footnote{This has to be changed as
%    soon as theorem styles that change \texttt{\bslash @endtheorem}
%    exist.  In such a case, all existing styles must be changed as
%    well since they will have to reset the macro.}) Therefore, we can
%    set it equivalent to the macro that ends the theorem set (in our
%    example, |\endLem|).  However, if some day theorem styles exist
%    that do change |\@endtheorem|, we would have to use the
%    commented-out line instead.
%    \begin{macrocode}
    \global \expandafter \let \csname end#1\endcsname \@endtheorem
%   \global\@namedef{end#1}{\@endtheorem}%
%    \end{macrocode}
%    With these commands all the required definitions are employed,
%    unless the test |\@ifdefinable| has failed.  Therefore, we
%    end the second argument of this macro and with it the definition
%    of |\@xnthm|.
%    \begin{macrocode}
   }}
%    \end{macrocode}
% \end{macro}
%
% \begin{macro}{\@ynthm}
%    The definition of |\@ynthm| is completely analogous. In
%    this case the new counter that is defined is not reset within
%    another counter; thus the definition of |\the...|
%    is simplified:
%    \begin{macrocode}
\gdef\@ynthm#1#2{\expandafter\@ifdefinable\csname #1\endcsname
   {\@definecounter{#1}%
    \expandafter\xdef\csname the#1\endcsname{\@thmcounter{#1}}%
%    \end{macrocode}
%    The rest of the definition corresponds literally to that of
%    |\@xnthm|:
%    \begin{macrocode}
    \def\@tempa{\global\@namedef{#1}}\expandafter \@tempa
     \expandafter{\csname th@\the \theorem@style \expandafter
     \endcsname \the\theorem@bodyfont \@thm{#1}{#2}}%
    \global \expandafter \let \csname end#1\endcsname \@endtheorem}}
%    \end{macrocode}
% \end{macro}
%
% \begin{macro}{\@othm}
%    The definition of |\@othm| does not contain anything new.
%    \begin{macrocode}
\gdef\@othm#1[#2]#3{%
%    \end{macrocode}
% \changes{v2.2b}{95/10/07}{Build in check for counter defined}
%    We do not define a new counter but instead use one that has
%    already been defined.  Thus the only definition we need is that
%    of this pseudo-counter (i.e.\ |\the|\meta{env.\ name}).
%    First we check if |#2| corresponds to a known counter name.
%    \begin{macrocode}
  \expandafter\ifx\csname c@#2\endcsname\relax
   \@nocounterr{#2}%
  \else
   \expandafter\@ifdefinable\csname #1\endcsname
   {\expandafter \xdef \csname the#1\endcsname
     {\expandafter \noexpand \csname the#2\endcsname}%
%    \end{macrocode}
%    All other parts of the definition can be adopted from
%    |\@xnthm|. We have to remember, though, that in this case
%    the name of the current counter and the theorem label have moved
%    to the second and third arguments.
%    \begin{macrocode}
    \def\@tempa{\global\@namedef{#1}}\expandafter \@tempa
     \expandafter{\csname th@\the \theorem@style \expandafter
     \endcsname \the\theorem@bodyfont \@thm{#2}{#3}}%
    \global \expandafter \let \csname end#1\endcsname \@endtheorem}%
  \fi}
%    \end{macrocode}
% \end{macro}
%
% \subsection{Macros that are employed in a {\envfont theorem}
%    environment}
%
% \begin{macro}{\@thm}
%    The macro |\@thm| has to increase the current counter.  Then,
%    depending on whether the environment has (or does not have) an
%    optional argument, it has to branch into either |\@begintheorem|
%    or |\@opargtheorem|.
%    \begin{macrocode}
\gdef\@thm#1#2{\refstepcounter{#1}%
%    \end{macrocode}
%    Now we start a {\envfont trivlist} environment, and give
%    |\@topsep| and |\@topsepadd| the values of the \textsf{skip}
%    registers |\theorempreskipamount| and |\theorempostskipamount|.
%    The value in |\@topsep| is the vertical space that is inserted by
%    the first (and only) |\item| in our |\trivlist| whilst
%    |\@topsepadd| is inserted by |\@endparenv| at the end of that
%    {\envfont trivlist} environment.  By using these registers, we
%    obtain the desired space around a {\envfont theorem} environment.
%    \begin{macrocode}
   \trivlist
   \@topsep \theorempreskipamount               % used by first \item
   \@topsepadd \theorempostskipamount           % used by \@endparenv
%    \end{macrocode}
%    Now we have to test whether an optional argument has been given.
% \changes{v2.0d}{89/07/18}{Forgotten percent added.}
%    \begin{macrocode}
   \@ifnextchar [%
%    \end{macrocode}
%    If there is an optional argument, we will call |\@ythm|, and
%    move the arguments read back into the input stream.
%    \begin{macrocode}
   {\@ythm{#1}{#2}}%
%    \end{macrocode}
%    If not, we call |\@begintheorem|. Its first argument is the
%    name of the theorem set (hence the second argument of
%    |\@thm|). Its second argument is the macro that produces the
%    current number.
%    \begin{macrocode}
   {\@begintheorem{#2}{\csname the#1\endcsname}\ignorespaces}}
%    \end{macrocode}
% \end{macro}
%
% \begin{macro}{\@xthm}
% \begin{macro}{\@ythm}
%    Both these macros were originally called by |\@thm|.  We do
%    not need |\@xthm| anymore, hence we reset it to
%    |\relax|. The definition of |\@ythm| has not changed at
%    all from its definition in \LaTeX. In order to make the macros
%    easier to understand, we will
%    nevertheless present it (commented out).
%    \begin{macrocode}
\global\let\@xthm\relax
% \def\@ythm#1#2[#3]{\@opargbegintheorem{#2}{\csname
%           the#1\endcsname}{#3}\ignorespaces}
%</package>
%    \end{macrocode}
%    The primitive |\ignorespaces| in |\@ythm| and
%    |\@thm| is needed to remove the spaces between the
%    |\begin{...}| and the actual text.
% \end{macro}
% \end{macro}
%
%
% \subsection{Definition of the theorem substyles}
%
% As already pointed out, the theorem substyles, defined below, are
% only loaded when necessary.  Note that all these substyles, except
% \texttt{plain}, have |\slshape| as the default body font.
%
% \subsubsection{The \texttt{plain} style}
%
% \label{dok:th@plain} As the following macros use |@|, we have
% to locally set the |\catcode| of this symbol to ``letter''.
% This happens within a group, so that we do not have to worry about
% which |\catcode| that symbol had before.
%    \begin{macrocode}
%<*thp>
\begingroup \makeatletter
%    \end{macrocode}
% Since we are now within a group, we must make all definitions
% globally.  First we make sure that \texttt{theorem.sty} is loaded.
% This will allow us to use this file as a document style option
% without having to call \texttt{theorem} itself as an option.  At the
% same time, we assure that at least version 2 is loaded, since
% |\theorem@style| was not defined in earlier versions.
%    \begin{macrocode}
\@ifundefined{theorem@style}{\input{theorem.sty}}{}
\ProvidesFile{thp.sty}
             [\FMithmInfo]
%    \end{macrocode}
%
%
% \begin{macro}{\th@plain}
% \begin{macro}{\@begintheorem}
% \begin{macro}{\@opargbegintheorem}
%    |\theoremstyle{plain}| corresponds to the original
%    definition, except that the distances to the surrounding text are
%    determined by the parameters |\theorempreskipamount| and
%    |\theorempostskipamount|.
%    First we set the default body font.
%    \begin{macrocode}
\gdef\th@plain{\normalfont\itshape
%    \end{macrocode}
%    Then we define |\@begintheorem| and
%    |\@opargbegintheorem|. These two macros define how the
%    header of a theorem is typeset.  |\@opargbegintheorem| will
%    be called if a \textsf{theorem} environment with an optional
%    argument is encountered; otherwise, the header is constructed by
%    calling |\@begintheorem|.  If one of these macros is
%    executed, we are within a \textsf{trivlist} environment started by
%    |\@thm|.  So the theorem header is produced with an
%    |\item| command.
%
%    Instead of specifying the header font directly, all standard
%    theorem styles use the |\theorem@headerfont| macro to allow
%    customization.  The extra space (|\labelsep|) is necessary
%    because of problems in the \textsf{trivlist} environment.
%    \begin{macrocode}
  \def\@begintheorem##1##2{%
        \item[\hskip\labelsep \theorem@headerfont ##1\ ##2]}%
%    \end{macrocode}
%    The definition of |\@opargbegintheorem| is completely
%    analogous. The only difference is the fact that there exists a
%    third argument (which is the optional parameter of the
%    environment and contains additional information about the
%    theorem).  Customarily we enclose it in parentheses.
%    \begin{macrocode}
\def\@opargbegintheorem##1##2##3{%
   \item[\hskip\labelsep \theorem@headerfont ##1\ ##2\ (##3)]}}
%    \end{macrocode}
% \end{macro}
% \end{macro}
% \end{macro}
%
% \noindent
% We conclude with an |\endgroup| to restore the
% |\catcode| of |@|.
%    \begin{macrocode}
\endgroup
%</thp>
%    \end{macrocode}
%
%
% \subsubsection{The \texttt{break} style}
%
% This style option is stored in the file |thb.sty|.
% For the next two lines see the documentation for |\th@plain| on
% page~\pageref{dok:th@plain}.
%    \begin{macrocode}
%<*thb>
\begingroup \makeatletter
\@ifundefined{theorem@style}{\input{theorem.sty}}{}
\ProvidesFile{thb.sty}
             [\FMithmInfo]
%    \end{macrocode}
%
% \begin{macro}{\th@break}
%    |\theoremstyle{break}| produces a line break after the name
%    of the theorem.  The font is |\slshape|.  Hence, we define
%    |\th@break| as follows:
%    \begin{macrocode}
\gdef\th@break{\normalfont\slshape
  \def\@begintheorem##1##2{\item[%
%    \end{macrocode}
%    We run into the following problem: it is not possible to create
%    the header with |\item[|\meta{title}|]| and then start
%    a new line by, for example, |\mbox{}\\|. Such a definition
%    will fail whenever a list environment follows immediately.  With
%    the above construction, the |\mbox{}| causes the switch
%    |@inlabel| (cf.\ definition of |\list| and
%    |\trivlist| in \cite{latex:source}) to be set to \texttt{false}
%    and so the following list will insert additional vertical
%    space (|\topskip|). This is quite annoying. Therefore, we
%    create the line break within the |\item|. In order to ensure
%    that the text will begin at the proper position in the following
%    line, we simply pretend that the label does not take any
%    room.\footnote{This will lead to problems whenever very high
%    symbols occurring in the line tower into the
%    heading. So, something else has to be done here sometime.}
%    \begin{macrocode}
       \rlap{\vbox{\hbox{\hskip \labelsep\theorem@headerfont ##1\ ##2}%
                   \hbox{\strut}}}]}%
%    \end{macrocode}
%    Again, the definition of |\@opargbegintheorem| is
%    completely analogous.
%    \begin{macrocode}
\def\@opargbegintheorem##1##2##3{%
  \item[\rlap{\vbox{\hbox{\hskip \labelsep \theorem@headerfont
                     ##1\ ##2\ (##3)}%
                    \hbox{\strut}}}]}}
\endgroup
%</thb>
%    \end{macrocode}
% \end{macro}
%
%
% \subsubsection{The \texttt{changebreak} style}
%
%    \begin{macrocode}
% This style option is stored in the file |thcb.sty|.
%    \begin{macrocode}
%<*thcb>
\begingroup \makeatletter
\@ifundefined{theorem@style}{\input{theorem.sty}}{}
\ProvidesFile{thcb.sty}
             [\FMithmInfo]
%    \end{macrocode}
%
% \begin{macro}{\th@changebreak}
%    The \texttt{change-break} theorem style is like \texttt{break}
%    but with interchange of theorem name and theorem number.  Thus we
%    define |\th@changebreak| as follows:
%    \begin{macrocode}
\gdef\th@changebreak{\normalfont\slshape
  \def\@begintheorem##1##2{\item
       [\rlap{\vbox{\hbox{\hskip\labelsep \theorem@headerfont ##2\ ##1}%
                    \hbox{\strut}}}]}%
\def\@opargbegintheorem##1##2##3{%
  \item[\rlap{\vbox{\hbox{\hskip\labelsep \theorem@headerfont
                     ##2\ ##1\ (##3)}%
                    \hbox{\strut}}}]}}
\endgroup
%</thcb>
%    \end{macrocode}
% \end{macro}
%
%
% \subsubsection{The \texttt{change} style}
%
% This style option is stored in the file |thc.sty|.
%    \begin{macrocode}
%<*thc>
\begingroup \makeatletter
\@ifundefined{theorem@style}{\input{theorem.sty}}{}
\ProvidesFile{thc.sty}
             [\FMithmInfo]
%    \end{macrocode}
%
% \begin{macro}{\th@change}
%    The \texttt{change} theorem style corresponds to the
%    \texttt{change break} style without a linebreak after the header.
%    To say it in another way, it's the same as the \texttt{plain}
%    style but with number and name interchanged and |\slshape| as the
%    default font.
%    \begin{macrocode}
\gdef\th@change{\normalfont\slshape
  \def\@begintheorem##1##2{\item
       [\hskip\labelsep \theorem@headerfont ##2\ ##1]}%
\def\@opargbegintheorem##1##2##3{%
  \item[\hskip\labelsep \theorem@headerfont ##2\ ##1\ (##3)]}}
\endgroup
%</thc>
%    \end{macrocode}
% \end{macro}
%
%
% \subsubsection{The \texttt{marginbreak} style}
%
% This style option is the one used most often at Mainz.  It is saved
% in the file |thmb.sty|.
%    \begin{macrocode}
%<*thmb>
\begingroup \makeatletter
\@ifundefined{theorem@style}{\input{theorem.sty}}{}
\ProvidesFile{thmb.sty}
             [\FMithmInfo]
%    \end{macrocode}
%
% \begin{macro}{\th@marginbreak}
%    The \texttt{margin break} style is nearly the same as the
%    \texttt{change break} style. The only difference is the placement
%    of the theorem number.  We use |\llap| to place it in the left
%    margin.
%
%    In this style |\labelsep| denotes the separation between the
%    number and the text.
%    \begin{macrocode}
\gdef\th@marginbreak{\normalfont\slshape
  \def\@begintheorem##1##2{\item
       [\rlap{\vbox{\theorem@headerfont
                    \hbox{\llap{##2}\hskip\labelsep ##1}%
                    \hbox{\strut}}}]}%
\def\@opargbegintheorem##1##2##3{%
  \item[\rlap{\vbox{\theorem@headerfont
                    \hbox{\llap{##2}\hskip\labelsep ##1\ (##3)}%
                    \hbox{\strut}}}]}}
\endgroup
%</thmb>
%    \end{macrocode}
% \changes{v2.1c}{90/02/17}{`endgroup outside macrocode env.}
% \end{macro}
%
%
%
% \subsubsection{The \texttt{margin} style}
%
% This style option is stored in the file |thm.sty|.
%    \begin{macrocode}
%<*thm>
\begingroup \makeatletter
\@ifundefined{theorem@style}{\input{theorem.sty}}{}
\ProvidesFile{thm.sty}
             [\FMithmInfo]
%    \end{macrocode}
%
% \begin{macro}{\th@margin}
%    Again this is only a variant of the theorem styles described
%    above without any new ideas.
%    \begin{macrocode}
\gdef\th@margin{\normalfont\slshape
  \def\@begintheorem##1##2{\item
       [\theorem@headerfont \llap{##2}\hskip\labelsep ##1]}%
\def\@opargbegintheorem##1##2##3{%
  \item[\theorem@headerfont \llap{##2}\hskip\labelsep ##1\ (##3)]}}
\endgroup
%</thm>
%    \end{macrocode}
% \end{macro}
%
%
%
%
% \subsection{Final Definitions}
%
% \begin{macro}{\theorempreskipamount}
% \begin{macro}{\theorempostskipamount}
%    The \textsf{skip} parameters that regulate the vertical empty space
%    before and after the {\envfont theorem} environment have
%    to be allocated as well.
% \changes{v2.0f}{89/07/20}{Unnecessary `global' removed.}
%    \begin{macrocode}
%<*package>
\newskip\theorempreskipamount
\newskip\theorempostskipamount
%    \end{macrocode}
%    Since we have used the same values for all theorem sets, we now can
%    assign them.
%    \begin{macrocode}
\global\setlength\theorempreskipamount{12pt plus 5pt minus 3pt}
\global\setlength\theorempostskipamount{8pt plus 3pt minus 1.5pt}
%    \end{macrocode}
% \end{macro}
% \end{macro}
%
%
% \begin{macro}{\@endtheorem}
%    The same holds for the macro |\@endtheorem|, which ends a
%    {\envfont theorem} environment. Since it is the same for all
%    theorem sets, it is removed from the macros |\th@|\meta{style}.
%    It simply ends the {\envfont trivlist} environment, which was
%    begun in |\@thm|.
%    \begin{macrocode}
\global\let\@endtheorem=\endtrivlist
%    \end{macrocode}
% \end{macro}
%
%
%
% \begin{macro}{\@preamblecmds}
%    All macros defined above are to be used only in the preamble.
%    Therefore, we insert them in |\@preamblecmds| which will disable
%    them at begin document. This is done by the internal \LaTeXe{}
%    command |\@onlypreamble|.
% \changes{v2.0g}{89/09/19}{`theoremheaderfont and `theorembodyfont
%    added.}
% \changes{v2.2a}{94/02/03}{Use \LaTeXe{} interface}
%    \begin{macrocode}
\@onlypreamble\@xnthm
\@onlypreamble\@ynthm
\@onlypreamble\@othm
\@onlypreamble\newtheorem
\@onlypreamble\theoremstyle
\@onlypreamble\theorembodyfont
\@onlypreamble\theoremheaderfont
%    \end{macrocode}
% \end{macro}
%
% Finally we declare the \texttt{plain} theorem style to be the default.
% \changes{v2.1b}{90/06/04}{Fixed missing `item by declaring default
%    style.}
%    \begin{macrocode}
\theoremstyle{plain}
%</package>
%    \end{macrocode}
%
% \Finale

\endinput
}{}
\ProvidesFile{thc.sty}
             [\FMithmInfo]
%    \end{macrocode}
%
% \begin{macro}{\th@change}
%    The \texttt{change} theorem style corresponds to the
%    \texttt{change break} style without a linebreak after the header.
%    To say it in another way, it's the same as the \texttt{plain}
%    style but with number and name interchanged and |\slshape| as the
%    default font.
%    \begin{macrocode}
\gdef\th@change{\normalfont\slshape
  \def\@begintheorem##1##2{\item
       [\hskip\labelsep \theorem@headerfont ##2\ ##1]}%
\def\@opargbegintheorem##1##2##3{%
  \item[\hskip\labelsep \theorem@headerfont ##2\ ##1\ (##3)]}}
\endgroup
%</thc>
%    \end{macrocode}
% \end{macro}
%
%
% \subsubsection{The \texttt{marginbreak} style}
%
% This style option is the one used most often at Mainz.  It is saved
% in the file |thmb.sty|.
%    \begin{macrocode}
%<*thmb>
\begingroup \makeatletter
\@ifundefined{theorem@style}{% \iffalse meta-comment
%
% Copyright (C) 1993-2024
%
% The LaTeX Project and any individual authors listed elsewhere
% in this file.
%
% This file is part of the Standard LaTeX `Tools Bundle'.
% -------------------------------------------------------
%
% It may be distributed and/or modified under the
% conditions of the LaTeX Project Public License, either version 1.3c
% of this license or (at your option) any later version.
% The latest version of this license is in
%    https://www.latex-project.org/lppl.txt
% and version 1.3c or later is part of all distributions of LaTeX
% version 2005/12/01 or later.
%
% The list of all files belonging to the LaTeX `Tools Bundle' is
% given in the file `manifest.txt'.
%
% \fi
%
%%
%
%
\def\FMithmInfo{2023/07/05 v2.2c Theorem extension package (FMi)}
%
% \ProvidesFile{theorem.dtx}[\FMithmInfo]
%\iffalse   % this is a METACOMMENT !
%%
%% Package `theorem' to use with LaTeX2e
%% Copyright (C) 1989-1995 Frank Mittelbach, all rights reserved.
%<+package>\@ifundefined{theorem@style}{}{\endinput}
%<+package>\NeedsTeXFormat{LaTeX2e}[1995/06/01]
%<+package>\ProvidesPackage{theorem}
%<+package>         [\FMithmInfo]
%
% \fi
%
% \changes{v1.0}{87/05/07}{First draft.}
% \changes{v1.0a}{87/05/07}{`thistheoremstyle' added.}
% \changes{v1.1a}{87/07/29}{New concept: redefine the internal
%                          \protect\LaTeX{} commands `@xnthm' etc.}
% \changes{v1.1b}{87/07/31}{`theoremfont' used to specify font.}
% \changes{v1.2a}{88/01/08}{`@endtheorem' now globally defined.}
% \changes{v1.2b}{88/01/11}{`newtheorem' only allowed in the preamble.}
% \changes{v1.3a}{88/02/14}{`theoremfont' now a toks register.}
%
% \changes{v2.0a}{88/11/16}{New implementation and new documentation.}
% \changes{v2.0a}{88/11/16}{No layout changes but internal modifications
%                           old substyles won't work.}
% \changes{v2.0b}{89/01/30}{`global' used for all definitions.}
% \changes{v2.0c}{89/06/15}{Documentation translated to English.
%                            Thanks to LLL and Nina}
% \changes{v2.1a}{90/02/15}{Support for new font selection.}
% \changes{v2.2a}{94/02/03}{Cleanup for LaTeX2e}
%
% \DoNotIndex{\@endparenv,\@ifdefinable,\@ifnextchar,\@ifundefined}
% \DoNotIndex{\@namedef,\@nameuse,\@spaces,\@tempa,\@warning}
% \DoNotIndex{\ }
% \DoNotIndex{\begingroup,\bfseries}
% \DoNotIndex{\csname}
% \DoNotIndex{\def,\do,\docdate}
% \DoNotIndex{\endcsname,\endgroup,\endinput,\endtrivlist,\expandafter}
% \DoNotIndex{\filedate,\fileversion}
% \DoNotIndex{\gdef,\global}
% \DoNotIndex{\hbox,\hskip}
% \DoNotIndex{\ignorespaces,\input,\itshape,\item}
% \DoNotIndex{\labelsep,\let,\llap}
% \DoNotIndex{\makeatletter}
% \DoNotIndex{\noexpand}
% \DoNotIndex{\refstepcounter,\relax,\rlap}
% \DoNotIndex{\setlength,\slshape,\space,\string,\strut}
% \DoNotIndex{\the,\trivlist,\typeout}
% \DoNotIndex{\vbox}
% \DoNotIndex{\xdef}
%
% \def\deflabel#1{\ttfamily #1\hfill}
% \def\deflist#1{\begin{list}{}{\settowidth\labelwidth{\ttfamily #1}%
%                               \setlength\leftmargin\labelwidth
%                               \addtolength\leftmargin\labelsep
%                               \let\makelabel\deflabel}}
% \def\enddeflist{\end{list}}
% \def\envfont{\normalfont\ttfamily}
%
% \MakeShortVerb{\|}
%
% \newcommand{\TUB}{{\sl TUGboat\/}}
%
% \GetFileInfo{theorem.dtx}
% \title{An Extension of the
%        \LaTeX{} theorem environment\thanks{This file
%        has version number \fileversion, last
%        revised \filedate.}}
% \author{Frank Mittelbach}
%
% \MaintainedByLaTeXTeam{tools}
% \maketitle
%
% \begin{abstract}
%    The macros described in this paper yield an extension of the
%    \LaTeX{} theorem mechanism. It is designed to satisfy the
%    different requirements of various journals. Thus, the
%    layout of the ``theorems'' can be manipulated by determining a
%    ``style''. This article describes not only the use, but
%    also the definition, of the necessary macros.
% \end{abstract}
%
% \section*{Preface to version 2.2}
%
% For \LaTeXe{} this package did not need any fundamental changes. I
% only modified the messages generated so that theorem layout styles
% will show up with the |\listfiles| command and cleaned the section
% on the New Font Selection Scheme since this is now included in
% \LaTeX.
%
% \section*{Preface to version 2.1}
%
% This version is identical to 2.0g described in \TUB\ 10\#3 except
% for some internal defaults which are now set depending on the used
% font selection scheme.
%
% This was done to avoid unpleasant surprises if the new font selection
% scheme is in force. For further details see
% section~\ref{sec:newfonts} and \cite{bk:GMS94}.
%
% \section{Introduction}
%
% For our purposes here, ``theorems'' are labelled enunciations,
% often set off from the main text by extra space and a font change.
% Theorems, corollaries, conjectures, definitions, and remarks
% are all instances of ``theorems''.  The ``header'' of these structures
% is composed of a label (such as \textsc{Theorem} or \textsc{Remark})
% and a number which serializes an item in the sequence of items
% with the same label.
%
% Shortly after the introduction of \LaTeX{} at the Fachbereich
% Mathematik in Mainz, the desire to manipulate the layout of
% ``theorems'' arose.  In Mainz, the following two conventions came
% into general use:
% \begin{enumerate}
%   \item The number of the theorem is shown in the margin.
%   \item There is a line break at
%         the end of the theorem header.
% \end{enumerate}
%
% Additionally, some journals require different formats which depend on
% the ``sort of theorem'': e.g.\ often remarks and definitions are set
% in |\upshape|, while |\itshape| is employed for main theorems.
%
% Confronted with these requirements, a theorem environment was
% developed in Mainz which allows separate determination of the
% layout of the ``theorems sets'', comparable to |\pagestyle|.
%
%
% \section{The user interface}
% \subsection{Defining new theorem sets}
%
% \DescribeMacro\newtheorem
% As in the original \LaTeX{} version, the command |\newtheorem|
% defines a new ``theorem set'' or ``theorem-like structure''.
% Two required arguments name the new environment and give the text
% to be typeset with each instance of the new ``set'', while
% an optional argument determines how
% the ``set'' is enumerated:
% \begin{description}
%    \item[\ttfamily \bslash newtheorem\{foo\}\{bar\}]
%       The theorem set {\envfont foo} (whose name is \texttt{bar})
%       uses its own counter.
%    \item[\ttfamily \bslash newtheorem\{foo2\}{[foo]}\{bar2\}]
%       The theorem set {\envfont foo2} (printed name \texttt{bar2})
%       uses the same counter as the theorem set \texttt{foo}.
%    \item[\ttfamily \bslash newtheorem\{foo3\}\{bar3\}{[section]}]
%       The theorem set {\envfont foo3} (printed name \texttt{bar3}) is
%       enumerated within the counter \texttt{section}, i.e.\ with every
%       new |\section| the enumeration begins again with 1, and
%       the enumeration is composed from the section-number and the
%       theorem counter itself.
% \end{description}
%
% \DescribeMacro\theoremstyle
% Additionally, the command |\theoremstyle| can define the layout
% of various, or all, theorem sets.  It should be noted that any
% theorem set defined by |\newtheorem| is typeset in the
% |\theoremstyle| that is current at the time of the definition.
% Thus, the following
% \begin{verbatim}
%    \theoremstyle{break}        \newtheorem{Cor}{Corollary}
%    \theoremstyle{plain}        \newtheorem{Exa}{Example}[section]
%\end{verbatim}                             ^^A No blank at line start
% leads to the result that the set {\envfont Cor} is formatted in the
% style \texttt{break}, while the set {\envfont Exa} and all the
% following ones are formatted in the style \texttt{plain}, unless
% another |\theoremstyle| follows.  Since the definitions installed by
% |\newtheorem| are global, one also can limit |\theoremstyle| locally
% by grouping braces.
%
% \DescribeMacro\theorembodyfont
% The choice of the font for the theorem body is completely
% independent of the chosen |\theoremstyle|; this has proven to
% be very advantageous.  For example,
% \begin{verbatim}
%   {\theorembodyfont{\upshape}          \newtheorem{Rem}{Remark}}
%\end{verbatim}
% defines a theorem set {\envfont Rem}, which will be set in
% |\upshape| in the current layout (which in our example is
% \texttt{plain}).  As with |\theoremstyle|, the |\theorembodyfont|
% chosen is that current at the time of |\newtheorem|.  If
% |\theorembodyfont| is not specified or one defines
% |\theorembodyfont{}|, then the font used will be that defined by the
% |\theoremstyle|.
%
% \DescribeMacro\theoremheaderfont It is also possible to customize
% the font used for the theorem headers.  This is, however, a global
% declaration, and therefore there should be at most one
% |\theoremheaderfont| declaration in the preamble.\footnote{If it is
% actually necessary to have different header fonts, one has to define
% new theorem styles (substituting the desired font) or specify the
% information directly in the \texttt{\bslash newtheorem} declaration
% (the unclean variant).}
%
% \DescribeMacro\theorempreskipamount
% \DescribeMacro\theorempostskipamount Two additional parameters
% affect the vertical space around the theorem environments:
% |\theorempreskipamount| and |\theorempostskipamount| define,
% respectively, the spacing before and after such an environment.
% These parameters apply for all theorem sets and can be manipulated
% with the ordinary length macros.  They are rubber lengths
% (`\textsf{skips}'), and therefore can contain \texttt{plus} and
% \texttt{minus} parts.
%
% Since the definition of theorem sets should---most sensibly---be
% placed in the preamble, we only allow installation there.
% It is therefore possible to release the memory used here after
% |\begin{document}|, in order to make room for other
% applications.
%
%
% \subsection{Existing theorem styles}
%
% The following theorem styles exist to date:
% \begin{deflist}{marginbreak}
%    \item[plain]
%       This theorem style emulates the original \LaTeX{} definition,
%       except that additionally the parameters
%       |\theorem...skipamount| are used.
%    \item[break]
%       In this style, the theorem header is followed by a line
%       break.
%    \item[marginbreak]
%       The theorem number is set in the margin, and there is a line
%       break as in \texttt{break}.
%    \item[changebreak]
%       Like \texttt{break}, but with header number and text
%       interchanged.
%    \item[change]
%       Header number and text are interchanged, without a line break.
%    \item[margin]
%       The number is set in the left margin, without a line break.
% \end{deflist}
% All styles (except \texttt{plain}) select |\slshape| as the default
% |\theorembodyfont|.
%
%
% \subsection{Examples}
%
% Given the above theorem sets {\envfont Cor}, {\envfont Exa}
% and {\envfont Rem},
% suppose that the preamble also contains the declarations:
% \begin{verbatim}
%    \theoremstyle{marginbreak}   \newtheorem{Lem}[Cor]{Lemma}
%    \theoremstyle{change}
%    \theorembodyfont{\itshape}        \newtheorem{Def}[Cor]{Definition}
%
%    \theoremheaderfont{\scshape}
%\end{verbatim}
% Then the following are some typical examples of the typeset output
% resulting from their use.
%
% \begin{Cor}
%    This is a sentence typeset in the theorem environment {\envfont
%    Cor}.
% \end{Cor}
%
% \begin{Exa}
%    This is a sentence typeset in the theorem environment {\envfont
%    Exa}.
% \end{Exa}
%
% \begin{Rem}
%    This is a sentence typeset in the theorem environment {\envfont
%    Rem}.
% \end{Rem}
%
% \begin{Lem}[Ben User]
%    This is a sentence typeset in the theorem environment {\envfont
%    Lem}.
% \end{Lem}
%
% \begin{Def}[Very impressive Definition]
%    This is a sentence typeset in the theorem environment {\envfont
%    Def}.
% \end{Def}
%
% The last two examples show the effect of the optional argument to
% a theorem environment (it is the text typeset in parentheses).
%
%
% \section{Special Considerations}
% \label{sec:newfonts}
%
% Theoremheader and body are implemented as a unit. This means that
% the |\theoremheaderfont|  will inherit characteristics of the
% |\theorembodyfont| in \LaTeXe. Thus, if for example
% |\theorembodyfont| is |\itshape| and |\theoremheaderfont|
% is |\bfseries| the font selected for the header will have the
% characteristics `bold extended italic'. If this is not desired one
% should set the |\theoremheaderfont| to something like
% \begin{verbatim}
%      \theoremheaderfont{\normalfont\bfseries}
%\end{verbatim}
% i.e.\ supplying all necessary font information explicitly.
%
%
%
% \section{Acknowledgements}
%
% The publication of this set of macros was only possible with the
% help of Christina Busse (translating the manuscript into English),
% Joachim Pense (playing the r\^ole of typist), Chris Rowley
% (looking everything over) and many others providing useful
% suggestions.
%
% \MaybeStop{%
%    \begin{thebibliography}{9}
%       \bibitem{bk:GMS94} \textsc{M.~Goossens}, \textsc{F.~Mittelbach}
%         and \textsc{A.~Samarin}.
%         \newblock The \LaTeX{} Companion.
%         \newblock
%         Addison-Wesley, Reading, Massachusetts, 1994.
%       \bibitem{latex:source} \textsc{Lamport, Leslie.} \newblock
%                               \texttt{latex.tex}, version 2.09, date
%                               Feb. 1990.
%    \end{thebibliography}
%    \PrintIndex
%    \PrintChanges}
%
%
% \section{The documentation driver file}
%
% The next bit of code contains the documentation driver file for
% \TeX{}, i.e., the file that will produce the documentation you are
% currently reading. It will be extracted from this file by the
% \texttt{docstrip} program. Since it is the first code in the file
% one can alternatively process this file directly with \LaTeXe{} to
% obtain the documentation.
%
%    \begin{macrocode}
%<*driver>
\documentclass{ltxdoc}

\usepackage{theorem}

%<+driver>% The next few lines define theorem sets which are used
%<+driver>% in the example section of the documentation.

\theoremstyle{break}       \newtheorem{Cor}{Corollary}
\theoremstyle{plain}       \newtheorem{Exa}{Example}[section]
{\theorembodyfont{\upshape}\newtheorem{Rem}{Remark}}
\theoremstyle{marginbreak} \newtheorem{Lem}[Cor]{Lemma}
\theoremstyle{change}
\theorembodyfont{\itshape} \newtheorem{Def}[Cor]{Definition}

\theoremheaderfont{\scshape}

\RecordChanges

\begin{document}
   \DocInput{theorem.dtx}
\end{document}
%</driver>
%    \end{macrocode}
%
% \section{Definition of the Macros}
%
% If the file has been loaded before, we abort immediately. If not the
% package announces itself (this is actually done at the very top of
% the file---the way it is done isn't good style so don't copy it).
% \changes{v2.0e}{89/07/19}{Spaces removed from `typeout'.}
% \changes{v2.2b}{95/11/19}{Announce moved to top of file}
%    \begin{macrocode}
%<*package>
%\@ifundefined{theorem@style}{}{\endinput}
%\def\FMithmInfo{1995/11/19 v2.2b Theorem extension package (FMi)}
%\ProvidesPackage{theorem}[\FMithmInfo]
%    \end{macrocode}
%
%
% \subsection{Definition of theorem styles and fonts}
%
% All the definitions in this file are done globally to allow
% inputting this file inside a group.
%
% \begin{macro}{\theoremstyle}
%    Before a theorem style can be installed, the chosen style must be
%    known. For that reason, we must test to see that
%    |\th@|\meta{style} is known or, more precisely, that it is
%    different from |\relax|. If the style is not known then
%    |\th@plain| is used.
%    \begin{macrocode}
\gdef\theoremstyle#1{%
   \@ifundefined{th@#1}{\@warning
          {Unknown theoremstyle `#1'. Using `plain'}%
          \theorem@style{plain}}%
%    \end{macrocode}
%    We save the theorem style to be used in the token register
%    |\theorem@style|.
%    \begin{macrocode}
      {\theorem@style{#1}}%
%    \end{macrocode}
%    Now we ``evaluate'' the theorem style: this means, we call the
%    macro |\th@|\meta{style} which will activate the relevant
%    definitions which are contained in a separate file.  This is done
%    in a group to suppress changes to the current font.  This could
%    otherwise pose problems together with the new font selection
%    scheme\footnote{When I printed the original article using the new
%    font selection scheme I ended with a document with slanted
%    typefaces (text headings and all) simply because one of the
%    theorem styles used \texttt{\bslash sl} at toplevel.} if the
%    |\th@|\meta{style} is evaluated a second time.
% \changes{v2.1a}{90/02/15}{Group added around `th@..}
% \changes{v2.1a}{90/02/15}{Reseting of `..@begintheorem removed.}
%    \begin{macrocode}
      \begingroup
        \csname th@\the\theorem@style \endcsname
      \endgroup}
%    \end{macrocode}
% \end{macro}
%
%
% \begin{macro}{\@begintheorem}
% \changes{v2.1a}{90/02/15}{Relaxed at toplevel.}
% \begin{macro}{\@opargbegintheorem}
% \changes{v2.1a}{90/02/15}{Relaxed at toplevel.}
%    We reset |\@begintheorem| and |\@opargbegintheorem| to
%    |\relax| since these commands are no longer necessary at
%    toplevel. This will save a few tokens.
%    \begin{macrocode}
\global\let\@begintheorem\relax
\global\let\@opargbegintheorem\relax
%    \end{macrocode}
% \end{macro}
% \end{macro}
%
% \begin{macro}{\theorem@style}
%    Obviously the \textsf{token} register used above has to be
%    allocated. To assure the utmost compatibility with the
%    original \LaTeX{} definition, we set the default theorem style to
%    \texttt{plain}, which implements the usual \LaTeX{} convention.
% \changes{v2.0f}{89/07/20}{Unnecessary `global' removed.}
%    \begin{macrocode}
\newtoks\theorem@style
\global\theorem@style{plain}
%    \end{macrocode}
% \end{macro}
%
% \begin{macro}{\theorembodyfont}
% \begin{macro}{\theorem@bodyfont}
%    If the |\theorembodyfont| is set by the user then it should not
%    interact with the default font set in the theorem style. When the
%    new font selection is in force this may happen if, for example,
%    the default is |\itshape| and the new |\theorembodyfont| is
%    |\sffamily|. So we add a |\reset@font| command in front of the
%    user definition.
% \changes{v2.0f}{89/07/20}{Unnecessary `global' removed.}
% \changes{v2.0g}{89/09/19}{`theoremfont to `theorembodyfont.}
% \changes{v2.1c}{92/06/22}{`reset@font added and logic changed.}
% \changes{v2.1d}{92/10/28}{Reset to empty if arg is empty}
%    \begin{macrocode}
\gdef\theorembodyfont#1{%
%    \end{macrocode}
%    We check if the argument supplied is empty and if so put nothing
%    into the |\theorem@bodyfont| token register to allow for
%    |\theorembodyfont{}| as a mean of using the default of the
%    current |\theoremstyle|.
%    \begin{macrocode}
   \def\@tempa{#1}%
   \ifx\@tempa\@empty
    \theorem@bodyfont{}%
   \else
    \theorem@bodyfont{\reset@font#1}%
   \fi
   }
\newtoks\theorem@bodyfont
\global\theorem@bodyfont{}
%    \end{macrocode}
% \end{macro}
% \end{macro}
%
%
% \begin{macro}{\theoremheaderfont}
% \changes{v2.0g}{89/09/19}{Macro added.}
%    The font for the theorem headers is handled differently because
%    this definition applies to all theorem styles.
%    \begin{macrocode}
\gdef\theoremheaderfont#1{\gdef\theorem@headerfont{#1}%
%    \end{macrocode}
%    After using the macro once it is redefined to produce an error
%    message.
%    \begin{macrocode}
       \gdef\theoremheaderfont##1{%
        \typeout{\string\theoremheaderfont\space should be used
                 only once.}}}
%    \end{macrocode}
% \end{macro}
%
% \begin{macro}{\theorem@headerfont}
% \changes{v2.0g}{89/09/19}{Macro added.}
%    To set the |\theorem@headerfont| default we first test if
%    the new fontselection scheme is in force.
% \changes{v2.1a}{90/02/15}{Support for new font selection.}
%    \begin{macrocode}
\ifx\upshape\undefined
%    \end{macrocode}
%    If not we define it to expand into |\bfseries|. We don't use
%    |\let| just in case a following style option redefines this
%    macro.
%    \begin{macrocode}
\gdef\theorem@headerfont{\bfseries}
%    \end{macrocode}
%    Otherwise we reset the current shape before calling |\bfseries|.
%    \begin{macrocode}
\else \gdef\theorem@headerfont{\normalfont\bfseries}\fi
%    \end{macrocode}
% \end{macro}
%
% \begin{macro}{\th@plain}
% \begin{macro}{\th@break}
% \begin{macro}{\th@marginbreak}
% \begin{macro}{\th@changebreak}
% \begin{macro}{\th@change}
% \begin{macro}{\th@margin}
%    The different styles are defined in macros such as
%    |\th@plain|.  Since memory space is precious in
%    ``non-Big-versions'', we have to avoid offering too many unused
%    definitions.  Therefore we define these styles in separate files
%    that can be loaded on demand.  Thus the commands themselves only
%    load these files. We use |\@input@|, a \LaTeXe{} internal command
%    that ensures that the file will be listed with |\listfiles|.
% \changes{v2.2a}{94/02/02}{Use `@input@ to load theorem layout files}
%    \begin{macrocode}
\gdef\th@plain{\@input@{thp.sty}}
\gdef\th@break{\@input@{thb.sty}}
\gdef\th@marginbreak{\@input@{thmb.sty}}
\gdef\th@changebreak{\@input@{thcb.sty}}
\gdef\th@change{\@input@{thc.sty}}
\gdef\th@margin{\@input@{thm.sty}}
%    \end{macrocode}
%    This list will be expanded when new styles become
%    available.  For testing,  just append new theorem substyles
%    as document options.
% \end{macro}
% \end{macro}
% \end{macro}
% \end{macro}
% \end{macro}
% \end{macro}
%
%
%
% \subsection{Definition of a new theorem set}
%
% As already pointed out, a new theorem environment can be defined in
% three different ways:
% \begin{verbatim}
%    \newtheorem{Lem}{Lemma}
%    \newtheorem{Lem}{Lemma}[section]
%    \newtheorem{Lem}[Theorem]{Lemma}
%\end{verbatim}
% The function of the macro |\newtheorem| is to recognize these cases
% and then to branch into one of the three macros |\@ynthm|, |\@xnthm|
% or |\@othm|.  This mechanism is adopted unchanged from
% \cite{latex:source}; the essential point here is that, for example,
% in the second case, the arguments \texttt{Lem}, \texttt{Lemma} and
% \texttt{section} are passed over to the macro |\@xnthm|.
%
% We inspect this case first because the others present fewer problems,
% and thus are easily derived from this one.
%
% \begin{macro}{\@xnthm}
%    For our example arguments, the macro |\@xnthm| must fulfill
%    the following:
%    \begin{itemize}
%       \item
%          Define a new \LaTeX-counter `Lem'
%       \item
%          reset this counter within a |\section|
%       \item
%          define the macro |\theLem|
%       \item
%          define the environment macros |\Lem| and
%          |\endLem| using the current
%          |\theoremstyle| and |\theorem@bodyfont|.
%    \end{itemize}
%    Obviously, all this should happen only if the first argument
%    of |\@xnthm| (i.e.\ \texttt{Lem} in our example) is chosen so
%    as not to conflict with any previously defined commands or
%    environments.  This test is performed by the \LaTeX{} macro
%    |\@ifdefinable|.
%    \begin{macrocode}
\gdef\@xnthm#1#2[#3]{\expandafter\@ifdefinable\csname #1\endcsname
%    \end{macrocode}
%    Therefore, the first argument of |\@ifdefinable| is the
%    expansion (in the example, |\Lem|) of
%    |\csname||#1\endcsname|.  The second argument is
%    executed only if the test has been completed successfully.
%    \begin{macrocode}
   {%
%    \end{macrocode}
%    Now we define the new counter. The names of the \LaTeX{}
%    macros employed should speak for themselves:
% \changes{v2.2b}{1995/10/07}{Use \cs{@newctr}}
% \changes{v2.2c}{1995/11/23}{Use \cs{@newctr} properly.}
%    \begin{macrocode}
    \@definecounter{#1}\@newctr{#1}[#3]%
%    \end{macrocode}
%    Using |\@newctr| will give a proper error message if the counter
%    in |#3| is not defined.
%    In defining `|\theLem|' we must generate the desired macro
%    name by use of |\expandafter| and |\csname|.
%    \begin{macrocode}
    \expandafter\xdef\csname the#1\endcsname
%    \end{macrocode}
%    An |\xdef| is used in order to make the definition global,
%    and to ensure that it contains the replacement texts of
%    |\@thmcountersep| and |\@thmcounter|.\footnote{These
%    two macros can be defined by the document style.  Their default
%    values produce a `\texttt{.}' as separation and an arabic
%    representation of the number.} However, not everything should be
%    expanded. For example, it saves space to use |\thesection|
%    instead of its---at times---lengthy expansion.
%    \begin{macrocode}
      {\expandafter \noexpand \csname the#3\endcsname
       \@thmcountersep \@thmcounter{#1}}%
%    \end{macrocode}
%    Thus with the defaults of \LaTeX{}, |\theLem| would be
%    replaced by the command sequence |\thesection.\arabic{Lem}|.
%
%    We will now look at the definition of the macro which is executed
%    at the beginning of the actual environment (in our example this
%    macro is |\Lem|). It should be noted that we use an
%    ``|\expandafter| trick'' to expand only certain parts of the
%    replacement text at the time of the definition.
%    \begin{macrocode}
    \def\@tempa{\global\@namedef{#1}}%
    \expandafter \@tempa \expandafter{%
%    \end{macrocode}
%    First, the macro that contains the current definitions of
%    |\@begintheorem| and |\@opargtheorem| should be called
%    up.  The name of this macro---as is already known---has the form
%    |\th@|\meta{theorem\ style}; therefore, it must be called by
%    \begin{macrocode}
      \csname th@\the \theorem@style
%    \end{macrocode}
%    In addition the default theorem font should be changeable, i.e.\
%    we have to insert the contents of |\theorem@bodyfont|. For
%    that reason, we expand even further, beyond |\endcsname|,
%    and thus insert the contents of the \textsf{token} register
%    |\theorem@bodyfont| in the replacement text.
%    \begin{macrocode}
            \expandafter \endcsname \the \theorem@bodyfont
%    \end{macrocode}
%    Now it is time to call the macro |\@thm| which takes over
%    the further processing.  It has two arguments: the current
%    counter name (in our example, \texttt{Lem}), and the text of the
%    label (in our example, \texttt{Lemma}).
%    \begin{macrocode}
     \@thm{#1}{#2}}%
%    \end{macrocode}
%    With this, the `sub-definition' is complete.  The macro
%    |\@endtheorem| ends a theorem environment and is, so far, nothing
%    but an |\endtrivlist|.  (Hence it is defined globally, and not
%    within the theorem styles.\footnote{This has to be changed as
%    soon as theorem styles that change \texttt{\bslash @endtheorem}
%    exist.  In such a case, all existing styles must be changed as
%    well since they will have to reset the macro.}) Therefore, we can
%    set it equivalent to the macro that ends the theorem set (in our
%    example, |\endLem|).  However, if some day theorem styles exist
%    that do change |\@endtheorem|, we would have to use the
%    commented-out line instead.
%    \begin{macrocode}
    \global \expandafter \let \csname end#1\endcsname \@endtheorem
%   \global\@namedef{end#1}{\@endtheorem}%
%    \end{macrocode}
%    With these commands all the required definitions are employed,
%    unless the test |\@ifdefinable| has failed.  Therefore, we
%    end the second argument of this macro and with it the definition
%    of |\@xnthm|.
%    \begin{macrocode}
   }}
%    \end{macrocode}
% \end{macro}
%
% \begin{macro}{\@ynthm}
%    The definition of |\@ynthm| is completely analogous. In
%    this case the new counter that is defined is not reset within
%    another counter; thus the definition of |\the...|
%    is simplified:
%    \begin{macrocode}
\gdef\@ynthm#1#2{\expandafter\@ifdefinable\csname #1\endcsname
   {\@definecounter{#1}%
    \expandafter\xdef\csname the#1\endcsname{\@thmcounter{#1}}%
%    \end{macrocode}
%    The rest of the definition corresponds literally to that of
%    |\@xnthm|:
%    \begin{macrocode}
    \def\@tempa{\global\@namedef{#1}}\expandafter \@tempa
     \expandafter{\csname th@\the \theorem@style \expandafter
     \endcsname \the\theorem@bodyfont \@thm{#1}{#2}}%
    \global \expandafter \let \csname end#1\endcsname \@endtheorem}}
%    \end{macrocode}
% \end{macro}
%
% \begin{macro}{\@othm}
%    The definition of |\@othm| does not contain anything new.
%    \begin{macrocode}
\gdef\@othm#1[#2]#3{%
%    \end{macrocode}
% \changes{v2.2b}{95/10/07}{Build in check for counter defined}
%    We do not define a new counter but instead use one that has
%    already been defined.  Thus the only definition we need is that
%    of this pseudo-counter (i.e.\ |\the|\meta{env.\ name}).
%    First we check if |#2| corresponds to a known counter name.
%    \begin{macrocode}
  \expandafter\ifx\csname c@#2\endcsname\relax
   \@nocounterr{#2}%
  \else
   \expandafter\@ifdefinable\csname #1\endcsname
   {\expandafter \xdef \csname the#1\endcsname
     {\expandafter \noexpand \csname the#2\endcsname}%
%    \end{macrocode}
%    All other parts of the definition can be adopted from
%    |\@xnthm|. We have to remember, though, that in this case
%    the name of the current counter and the theorem label have moved
%    to the second and third arguments.
%    \begin{macrocode}
    \def\@tempa{\global\@namedef{#1}}\expandafter \@tempa
     \expandafter{\csname th@\the \theorem@style \expandafter
     \endcsname \the\theorem@bodyfont \@thm{#2}{#3}}%
    \global \expandafter \let \csname end#1\endcsname \@endtheorem}%
  \fi}
%    \end{macrocode}
% \end{macro}
%
% \subsection{Macros that are employed in a {\envfont theorem}
%    environment}
%
% \begin{macro}{\@thm}
%    The macro |\@thm| has to increase the current counter.  Then,
%    depending on whether the environment has (or does not have) an
%    optional argument, it has to branch into either |\@begintheorem|
%    or |\@opargtheorem|.
%    \begin{macrocode}
\gdef\@thm#1#2{\refstepcounter{#1}%
%    \end{macrocode}
%    Now we start a {\envfont trivlist} environment, and give
%    |\@topsep| and |\@topsepadd| the values of the \textsf{skip}
%    registers |\theorempreskipamount| and |\theorempostskipamount|.
%    The value in |\@topsep| is the vertical space that is inserted by
%    the first (and only) |\item| in our |\trivlist| whilst
%    |\@topsepadd| is inserted by |\@endparenv| at the end of that
%    {\envfont trivlist} environment.  By using these registers, we
%    obtain the desired space around a {\envfont theorem} environment.
%    \begin{macrocode}
   \trivlist
   \@topsep \theorempreskipamount               % used by first \item
   \@topsepadd \theorempostskipamount           % used by \@endparenv
%    \end{macrocode}
%    Now we have to test whether an optional argument has been given.
% \changes{v2.0d}{89/07/18}{Forgotten percent added.}
%    \begin{macrocode}
   \@ifnextchar [%
%    \end{macrocode}
%    If there is an optional argument, we will call |\@ythm|, and
%    move the arguments read back into the input stream.
%    \begin{macrocode}
   {\@ythm{#1}{#2}}%
%    \end{macrocode}
%    If not, we call |\@begintheorem|. Its first argument is the
%    name of the theorem set (hence the second argument of
%    |\@thm|). Its second argument is the macro that produces the
%    current number.
%    \begin{macrocode}
   {\@begintheorem{#2}{\csname the#1\endcsname}\ignorespaces}}
%    \end{macrocode}
% \end{macro}
%
% \begin{macro}{\@xthm}
% \begin{macro}{\@ythm}
%    Both these macros were originally called by |\@thm|.  We do
%    not need |\@xthm| anymore, hence we reset it to
%    |\relax|. The definition of |\@ythm| has not changed at
%    all from its definition in \LaTeX. In order to make the macros
%    easier to understand, we will
%    nevertheless present it (commented out).
%    \begin{macrocode}
\global\let\@xthm\relax
% \def\@ythm#1#2[#3]{\@opargbegintheorem{#2}{\csname
%           the#1\endcsname}{#3}\ignorespaces}
%</package>
%    \end{macrocode}
%    The primitive |\ignorespaces| in |\@ythm| and
%    |\@thm| is needed to remove the spaces between the
%    |\begin{...}| and the actual text.
% \end{macro}
% \end{macro}
%
%
% \subsection{Definition of the theorem substyles}
%
% As already pointed out, the theorem substyles, defined below, are
% only loaded when necessary.  Note that all these substyles, except
% \texttt{plain}, have |\slshape| as the default body font.
%
% \subsubsection{The \texttt{plain} style}
%
% \label{dok:th@plain} As the following macros use |@|, we have
% to locally set the |\catcode| of this symbol to ``letter''.
% This happens within a group, so that we do not have to worry about
% which |\catcode| that symbol had before.
%    \begin{macrocode}
%<*thp>
\begingroup \makeatletter
%    \end{macrocode}
% Since we are now within a group, we must make all definitions
% globally.  First we make sure that \texttt{theorem.sty} is loaded.
% This will allow us to use this file as a document style option
% without having to call \texttt{theorem} itself as an option.  At the
% same time, we assure that at least version 2 is loaded, since
% |\theorem@style| was not defined in earlier versions.
%    \begin{macrocode}
\@ifundefined{theorem@style}{\input{theorem.sty}}{}
\ProvidesFile{thp.sty}
             [\FMithmInfo]
%    \end{macrocode}
%
%
% \begin{macro}{\th@plain}
% \begin{macro}{\@begintheorem}
% \begin{macro}{\@opargbegintheorem}
%    |\theoremstyle{plain}| corresponds to the original
%    definition, except that the distances to the surrounding text are
%    determined by the parameters |\theorempreskipamount| and
%    |\theorempostskipamount|.
%    First we set the default body font.
%    \begin{macrocode}
\gdef\th@plain{\normalfont\itshape
%    \end{macrocode}
%    Then we define |\@begintheorem| and
%    |\@opargbegintheorem|. These two macros define how the
%    header of a theorem is typeset.  |\@opargbegintheorem| will
%    be called if a \textsf{theorem} environment with an optional
%    argument is encountered; otherwise, the header is constructed by
%    calling |\@begintheorem|.  If one of these macros is
%    executed, we are within a \textsf{trivlist} environment started by
%    |\@thm|.  So the theorem header is produced with an
%    |\item| command.
%
%    Instead of specifying the header font directly, all standard
%    theorem styles use the |\theorem@headerfont| macro to allow
%    customization.  The extra space (|\labelsep|) is necessary
%    because of problems in the \textsf{trivlist} environment.
%    \begin{macrocode}
  \def\@begintheorem##1##2{%
        \item[\hskip\labelsep \theorem@headerfont ##1\ ##2]}%
%    \end{macrocode}
%    The definition of |\@opargbegintheorem| is completely
%    analogous. The only difference is the fact that there exists a
%    third argument (which is the optional parameter of the
%    environment and contains additional information about the
%    theorem).  Customarily we enclose it in parentheses.
%    \begin{macrocode}
\def\@opargbegintheorem##1##2##3{%
   \item[\hskip\labelsep \theorem@headerfont ##1\ ##2\ (##3)]}}
%    \end{macrocode}
% \end{macro}
% \end{macro}
% \end{macro}
%
% \noindent
% We conclude with an |\endgroup| to restore the
% |\catcode| of |@|.
%    \begin{macrocode}
\endgroup
%</thp>
%    \end{macrocode}
%
%
% \subsubsection{The \texttt{break} style}
%
% This style option is stored in the file |thb.sty|.
% For the next two lines see the documentation for |\th@plain| on
% page~\pageref{dok:th@plain}.
%    \begin{macrocode}
%<*thb>
\begingroup \makeatletter
\@ifundefined{theorem@style}{\input{theorem.sty}}{}
\ProvidesFile{thb.sty}
             [\FMithmInfo]
%    \end{macrocode}
%
% \begin{macro}{\th@break}
%    |\theoremstyle{break}| produces a line break after the name
%    of the theorem.  The font is |\slshape|.  Hence, we define
%    |\th@break| as follows:
%    \begin{macrocode}
\gdef\th@break{\normalfont\slshape
  \def\@begintheorem##1##2{\item[%
%    \end{macrocode}
%    We run into the following problem: it is not possible to create
%    the header with |\item[|\meta{title}|]| and then start
%    a new line by, for example, |\mbox{}\\|. Such a definition
%    will fail whenever a list environment follows immediately.  With
%    the above construction, the |\mbox{}| causes the switch
%    |@inlabel| (cf.\ definition of |\list| and
%    |\trivlist| in \cite{latex:source}) to be set to \texttt{false}
%    and so the following list will insert additional vertical
%    space (|\topskip|). This is quite annoying. Therefore, we
%    create the line break within the |\item|. In order to ensure
%    that the text will begin at the proper position in the following
%    line, we simply pretend that the label does not take any
%    room.\footnote{This will lead to problems whenever very high
%    symbols occurring in the line tower into the
%    heading. So, something else has to be done here sometime.}
%    \begin{macrocode}
       \rlap{\vbox{\hbox{\hskip \labelsep\theorem@headerfont ##1\ ##2}%
                   \hbox{\strut}}}]}%
%    \end{macrocode}
%    Again, the definition of |\@opargbegintheorem| is
%    completely analogous.
%    \begin{macrocode}
\def\@opargbegintheorem##1##2##3{%
  \item[\rlap{\vbox{\hbox{\hskip \labelsep \theorem@headerfont
                     ##1\ ##2\ (##3)}%
                    \hbox{\strut}}}]}}
\endgroup
%</thb>
%    \end{macrocode}
% \end{macro}
%
%
% \subsubsection{The \texttt{changebreak} style}
%
%    \begin{macrocode}
% This style option is stored in the file |thcb.sty|.
%    \begin{macrocode}
%<*thcb>
\begingroup \makeatletter
\@ifundefined{theorem@style}{\input{theorem.sty}}{}
\ProvidesFile{thcb.sty}
             [\FMithmInfo]
%    \end{macrocode}
%
% \begin{macro}{\th@changebreak}
%    The \texttt{change-break} theorem style is like \texttt{break}
%    but with interchange of theorem name and theorem number.  Thus we
%    define |\th@changebreak| as follows:
%    \begin{macrocode}
\gdef\th@changebreak{\normalfont\slshape
  \def\@begintheorem##1##2{\item
       [\rlap{\vbox{\hbox{\hskip\labelsep \theorem@headerfont ##2\ ##1}%
                    \hbox{\strut}}}]}%
\def\@opargbegintheorem##1##2##3{%
  \item[\rlap{\vbox{\hbox{\hskip\labelsep \theorem@headerfont
                     ##2\ ##1\ (##3)}%
                    \hbox{\strut}}}]}}
\endgroup
%</thcb>
%    \end{macrocode}
% \end{macro}
%
%
% \subsubsection{The \texttt{change} style}
%
% This style option is stored in the file |thc.sty|.
%    \begin{macrocode}
%<*thc>
\begingroup \makeatletter
\@ifundefined{theorem@style}{\input{theorem.sty}}{}
\ProvidesFile{thc.sty}
             [\FMithmInfo]
%    \end{macrocode}
%
% \begin{macro}{\th@change}
%    The \texttt{change} theorem style corresponds to the
%    \texttt{change break} style without a linebreak after the header.
%    To say it in another way, it's the same as the \texttt{plain}
%    style but with number and name interchanged and |\slshape| as the
%    default font.
%    \begin{macrocode}
\gdef\th@change{\normalfont\slshape
  \def\@begintheorem##1##2{\item
       [\hskip\labelsep \theorem@headerfont ##2\ ##1]}%
\def\@opargbegintheorem##1##2##3{%
  \item[\hskip\labelsep \theorem@headerfont ##2\ ##1\ (##3)]}}
\endgroup
%</thc>
%    \end{macrocode}
% \end{macro}
%
%
% \subsubsection{The \texttt{marginbreak} style}
%
% This style option is the one used most often at Mainz.  It is saved
% in the file |thmb.sty|.
%    \begin{macrocode}
%<*thmb>
\begingroup \makeatletter
\@ifundefined{theorem@style}{\input{theorem.sty}}{}
\ProvidesFile{thmb.sty}
             [\FMithmInfo]
%    \end{macrocode}
%
% \begin{macro}{\th@marginbreak}
%    The \texttt{margin break} style is nearly the same as the
%    \texttt{change break} style. The only difference is the placement
%    of the theorem number.  We use |\llap| to place it in the left
%    margin.
%
%    In this style |\labelsep| denotes the separation between the
%    number and the text.
%    \begin{macrocode}
\gdef\th@marginbreak{\normalfont\slshape
  \def\@begintheorem##1##2{\item
       [\rlap{\vbox{\theorem@headerfont
                    \hbox{\llap{##2}\hskip\labelsep ##1}%
                    \hbox{\strut}}}]}%
\def\@opargbegintheorem##1##2##3{%
  \item[\rlap{\vbox{\theorem@headerfont
                    \hbox{\llap{##2}\hskip\labelsep ##1\ (##3)}%
                    \hbox{\strut}}}]}}
\endgroup
%</thmb>
%    \end{macrocode}
% \changes{v2.1c}{90/02/17}{`endgroup outside macrocode env.}
% \end{macro}
%
%
%
% \subsubsection{The \texttt{margin} style}
%
% This style option is stored in the file |thm.sty|.
%    \begin{macrocode}
%<*thm>
\begingroup \makeatletter
\@ifundefined{theorem@style}{\input{theorem.sty}}{}
\ProvidesFile{thm.sty}
             [\FMithmInfo]
%    \end{macrocode}
%
% \begin{macro}{\th@margin}
%    Again this is only a variant of the theorem styles described
%    above without any new ideas.
%    \begin{macrocode}
\gdef\th@margin{\normalfont\slshape
  \def\@begintheorem##1##2{\item
       [\theorem@headerfont \llap{##2}\hskip\labelsep ##1]}%
\def\@opargbegintheorem##1##2##3{%
  \item[\theorem@headerfont \llap{##2}\hskip\labelsep ##1\ (##3)]}}
\endgroup
%</thm>
%    \end{macrocode}
% \end{macro}
%
%
%
%
% \subsection{Final Definitions}
%
% \begin{macro}{\theorempreskipamount}
% \begin{macro}{\theorempostskipamount}
%    The \textsf{skip} parameters that regulate the vertical empty space
%    before and after the {\envfont theorem} environment have
%    to be allocated as well.
% \changes{v2.0f}{89/07/20}{Unnecessary `global' removed.}
%    \begin{macrocode}
%<*package>
\newskip\theorempreskipamount
\newskip\theorempostskipamount
%    \end{macrocode}
%    Since we have used the same values for all theorem sets, we now can
%    assign them.
%    \begin{macrocode}
\global\setlength\theorempreskipamount{12pt plus 5pt minus 3pt}
\global\setlength\theorempostskipamount{8pt plus 3pt minus 1.5pt}
%    \end{macrocode}
% \end{macro}
% \end{macro}
%
%
% \begin{macro}{\@endtheorem}
%    The same holds for the macro |\@endtheorem|, which ends a
%    {\envfont theorem} environment. Since it is the same for all
%    theorem sets, it is removed from the macros |\th@|\meta{style}.
%    It simply ends the {\envfont trivlist} environment, which was
%    begun in |\@thm|.
%    \begin{macrocode}
\global\let\@endtheorem=\endtrivlist
%    \end{macrocode}
% \end{macro}
%
%
%
% \begin{macro}{\@preamblecmds}
%    All macros defined above are to be used only in the preamble.
%    Therefore, we insert them in |\@preamblecmds| which will disable
%    them at begin document. This is done by the internal \LaTeXe{}
%    command |\@onlypreamble|.
% \changes{v2.0g}{89/09/19}{`theoremheaderfont and `theorembodyfont
%    added.}
% \changes{v2.2a}{94/02/03}{Use \LaTeXe{} interface}
%    \begin{macrocode}
\@onlypreamble\@xnthm
\@onlypreamble\@ynthm
\@onlypreamble\@othm
\@onlypreamble\newtheorem
\@onlypreamble\theoremstyle
\@onlypreamble\theorembodyfont
\@onlypreamble\theoremheaderfont
%    \end{macrocode}
% \end{macro}
%
% Finally we declare the \texttt{plain} theorem style to be the default.
% \changes{v2.1b}{90/06/04}{Fixed missing `item by declaring default
%    style.}
%    \begin{macrocode}
\theoremstyle{plain}
%</package>
%    \end{macrocode}
%
% \Finale

\endinput
}{}
\ProvidesFile{thmb.sty}
             [\FMithmInfo]
%    \end{macrocode}
%
% \begin{macro}{\th@marginbreak}
%    The \texttt{margin break} style is nearly the same as the
%    \texttt{change break} style. The only difference is the placement
%    of the theorem number.  We use |\llap| to place it in the left
%    margin.
%
%    In this style |\labelsep| denotes the separation between the
%    number and the text.
%    \begin{macrocode}
\gdef\th@marginbreak{\normalfont\slshape
  \def\@begintheorem##1##2{\item
       [\rlap{\vbox{\theorem@headerfont
                    \hbox{\llap{##2}\hskip\labelsep ##1}%
                    \hbox{\strut}}}]}%
\def\@opargbegintheorem##1##2##3{%
  \item[\rlap{\vbox{\theorem@headerfont
                    \hbox{\llap{##2}\hskip\labelsep ##1\ (##3)}%
                    \hbox{\strut}}}]}}
\endgroup
%</thmb>
%    \end{macrocode}
% \changes{v2.1c}{90/02/17}{`endgroup outside macrocode env.}
% \end{macro}
%
%
%
% \subsubsection{The \texttt{margin} style}
%
% This style option is stored in the file |thm.sty|.
%    \begin{macrocode}
%<*thm>
\begingroup \makeatletter
\@ifundefined{theorem@style}{% \iffalse meta-comment
%
% Copyright (C) 1993-2024
%
% The LaTeX Project and any individual authors listed elsewhere
% in this file.
%
% This file is part of the Standard LaTeX `Tools Bundle'.
% -------------------------------------------------------
%
% It may be distributed and/or modified under the
% conditions of the LaTeX Project Public License, either version 1.3c
% of this license or (at your option) any later version.
% The latest version of this license is in
%    https://www.latex-project.org/lppl.txt
% and version 1.3c or later is part of all distributions of LaTeX
% version 2005/12/01 or later.
%
% The list of all files belonging to the LaTeX `Tools Bundle' is
% given in the file `manifest.txt'.
%
% \fi
%
%%
%
%
\def\FMithmInfo{2023/07/05 v2.2c Theorem extension package (FMi)}
%
% \ProvidesFile{theorem.dtx}[\FMithmInfo]
%\iffalse   % this is a METACOMMENT !
%%
%% Package `theorem' to use with LaTeX2e
%% Copyright (C) 1989-1995 Frank Mittelbach, all rights reserved.
%<+package>\@ifundefined{theorem@style}{}{\endinput}
%<+package>\NeedsTeXFormat{LaTeX2e}[1995/06/01]
%<+package>\ProvidesPackage{theorem}
%<+package>         [\FMithmInfo]
%
% \fi
%
% \changes{v1.0}{87/05/07}{First draft.}
% \changes{v1.0a}{87/05/07}{`thistheoremstyle' added.}
% \changes{v1.1a}{87/07/29}{New concept: redefine the internal
%                          \protect\LaTeX{} commands `@xnthm' etc.}
% \changes{v1.1b}{87/07/31}{`theoremfont' used to specify font.}
% \changes{v1.2a}{88/01/08}{`@endtheorem' now globally defined.}
% \changes{v1.2b}{88/01/11}{`newtheorem' only allowed in the preamble.}
% \changes{v1.3a}{88/02/14}{`theoremfont' now a toks register.}
%
% \changes{v2.0a}{88/11/16}{New implementation and new documentation.}
% \changes{v2.0a}{88/11/16}{No layout changes but internal modifications
%                           old substyles won't work.}
% \changes{v2.0b}{89/01/30}{`global' used for all definitions.}
% \changes{v2.0c}{89/06/15}{Documentation translated to English.
%                            Thanks to LLL and Nina}
% \changes{v2.1a}{90/02/15}{Support for new font selection.}
% \changes{v2.2a}{94/02/03}{Cleanup for LaTeX2e}
%
% \DoNotIndex{\@endparenv,\@ifdefinable,\@ifnextchar,\@ifundefined}
% \DoNotIndex{\@namedef,\@nameuse,\@spaces,\@tempa,\@warning}
% \DoNotIndex{\ }
% \DoNotIndex{\begingroup,\bfseries}
% \DoNotIndex{\csname}
% \DoNotIndex{\def,\do,\docdate}
% \DoNotIndex{\endcsname,\endgroup,\endinput,\endtrivlist,\expandafter}
% \DoNotIndex{\filedate,\fileversion}
% \DoNotIndex{\gdef,\global}
% \DoNotIndex{\hbox,\hskip}
% \DoNotIndex{\ignorespaces,\input,\itshape,\item}
% \DoNotIndex{\labelsep,\let,\llap}
% \DoNotIndex{\makeatletter}
% \DoNotIndex{\noexpand}
% \DoNotIndex{\refstepcounter,\relax,\rlap}
% \DoNotIndex{\setlength,\slshape,\space,\string,\strut}
% \DoNotIndex{\the,\trivlist,\typeout}
% \DoNotIndex{\vbox}
% \DoNotIndex{\xdef}
%
% \def\deflabel#1{\ttfamily #1\hfill}
% \def\deflist#1{\begin{list}{}{\settowidth\labelwidth{\ttfamily #1}%
%                               \setlength\leftmargin\labelwidth
%                               \addtolength\leftmargin\labelsep
%                               \let\makelabel\deflabel}}
% \def\enddeflist{\end{list}}
% \def\envfont{\normalfont\ttfamily}
%
% \MakeShortVerb{\|}
%
% \newcommand{\TUB}{{\sl TUGboat\/}}
%
% \GetFileInfo{theorem.dtx}
% \title{An Extension of the
%        \LaTeX{} theorem environment\thanks{This file
%        has version number \fileversion, last
%        revised \filedate.}}
% \author{Frank Mittelbach}
%
% \MaintainedByLaTeXTeam{tools}
% \maketitle
%
% \begin{abstract}
%    The macros described in this paper yield an extension of the
%    \LaTeX{} theorem mechanism. It is designed to satisfy the
%    different requirements of various journals. Thus, the
%    layout of the ``theorems'' can be manipulated by determining a
%    ``style''. This article describes not only the use, but
%    also the definition, of the necessary macros.
% \end{abstract}
%
% \section*{Preface to version 2.2}
%
% For \LaTeXe{} this package did not need any fundamental changes. I
% only modified the messages generated so that theorem layout styles
% will show up with the |\listfiles| command and cleaned the section
% on the New Font Selection Scheme since this is now included in
% \LaTeX.
%
% \section*{Preface to version 2.1}
%
% This version is identical to 2.0g described in \TUB\ 10\#3 except
% for some internal defaults which are now set depending on the used
% font selection scheme.
%
% This was done to avoid unpleasant surprises if the new font selection
% scheme is in force. For further details see
% section~\ref{sec:newfonts} and \cite{bk:GMS94}.
%
% \section{Introduction}
%
% For our purposes here, ``theorems'' are labelled enunciations,
% often set off from the main text by extra space and a font change.
% Theorems, corollaries, conjectures, definitions, and remarks
% are all instances of ``theorems''.  The ``header'' of these structures
% is composed of a label (such as \textsc{Theorem} or \textsc{Remark})
% and a number which serializes an item in the sequence of items
% with the same label.
%
% Shortly after the introduction of \LaTeX{} at the Fachbereich
% Mathematik in Mainz, the desire to manipulate the layout of
% ``theorems'' arose.  In Mainz, the following two conventions came
% into general use:
% \begin{enumerate}
%   \item The number of the theorem is shown in the margin.
%   \item There is a line break at
%         the end of the theorem header.
% \end{enumerate}
%
% Additionally, some journals require different formats which depend on
% the ``sort of theorem'': e.g.\ often remarks and definitions are set
% in |\upshape|, while |\itshape| is employed for main theorems.
%
% Confronted with these requirements, a theorem environment was
% developed in Mainz which allows separate determination of the
% layout of the ``theorems sets'', comparable to |\pagestyle|.
%
%
% \section{The user interface}
% \subsection{Defining new theorem sets}
%
% \DescribeMacro\newtheorem
% As in the original \LaTeX{} version, the command |\newtheorem|
% defines a new ``theorem set'' or ``theorem-like structure''.
% Two required arguments name the new environment and give the text
% to be typeset with each instance of the new ``set'', while
% an optional argument determines how
% the ``set'' is enumerated:
% \begin{description}
%    \item[\ttfamily \bslash newtheorem\{foo\}\{bar\}]
%       The theorem set {\envfont foo} (whose name is \texttt{bar})
%       uses its own counter.
%    \item[\ttfamily \bslash newtheorem\{foo2\}{[foo]}\{bar2\}]
%       The theorem set {\envfont foo2} (printed name \texttt{bar2})
%       uses the same counter as the theorem set \texttt{foo}.
%    \item[\ttfamily \bslash newtheorem\{foo3\}\{bar3\}{[section]}]
%       The theorem set {\envfont foo3} (printed name \texttt{bar3}) is
%       enumerated within the counter \texttt{section}, i.e.\ with every
%       new |\section| the enumeration begins again with 1, and
%       the enumeration is composed from the section-number and the
%       theorem counter itself.
% \end{description}
%
% \DescribeMacro\theoremstyle
% Additionally, the command |\theoremstyle| can define the layout
% of various, or all, theorem sets.  It should be noted that any
% theorem set defined by |\newtheorem| is typeset in the
% |\theoremstyle| that is current at the time of the definition.
% Thus, the following
% \begin{verbatim}
%    \theoremstyle{break}        \newtheorem{Cor}{Corollary}
%    \theoremstyle{plain}        \newtheorem{Exa}{Example}[section]
%\end{verbatim}                             ^^A No blank at line start
% leads to the result that the set {\envfont Cor} is formatted in the
% style \texttt{break}, while the set {\envfont Exa} and all the
% following ones are formatted in the style \texttt{plain}, unless
% another |\theoremstyle| follows.  Since the definitions installed by
% |\newtheorem| are global, one also can limit |\theoremstyle| locally
% by grouping braces.
%
% \DescribeMacro\theorembodyfont
% The choice of the font for the theorem body is completely
% independent of the chosen |\theoremstyle|; this has proven to
% be very advantageous.  For example,
% \begin{verbatim}
%   {\theorembodyfont{\upshape}          \newtheorem{Rem}{Remark}}
%\end{verbatim}
% defines a theorem set {\envfont Rem}, which will be set in
% |\upshape| in the current layout (which in our example is
% \texttt{plain}).  As with |\theoremstyle|, the |\theorembodyfont|
% chosen is that current at the time of |\newtheorem|.  If
% |\theorembodyfont| is not specified or one defines
% |\theorembodyfont{}|, then the font used will be that defined by the
% |\theoremstyle|.
%
% \DescribeMacro\theoremheaderfont It is also possible to customize
% the font used for the theorem headers.  This is, however, a global
% declaration, and therefore there should be at most one
% |\theoremheaderfont| declaration in the preamble.\footnote{If it is
% actually necessary to have different header fonts, one has to define
% new theorem styles (substituting the desired font) or specify the
% information directly in the \texttt{\bslash newtheorem} declaration
% (the unclean variant).}
%
% \DescribeMacro\theorempreskipamount
% \DescribeMacro\theorempostskipamount Two additional parameters
% affect the vertical space around the theorem environments:
% |\theorempreskipamount| and |\theorempostskipamount| define,
% respectively, the spacing before and after such an environment.
% These parameters apply for all theorem sets and can be manipulated
% with the ordinary length macros.  They are rubber lengths
% (`\textsf{skips}'), and therefore can contain \texttt{plus} and
% \texttt{minus} parts.
%
% Since the definition of theorem sets should---most sensibly---be
% placed in the preamble, we only allow installation there.
% It is therefore possible to release the memory used here after
% |\begin{document}|, in order to make room for other
% applications.
%
%
% \subsection{Existing theorem styles}
%
% The following theorem styles exist to date:
% \begin{deflist}{marginbreak}
%    \item[plain]
%       This theorem style emulates the original \LaTeX{} definition,
%       except that additionally the parameters
%       |\theorem...skipamount| are used.
%    \item[break]
%       In this style, the theorem header is followed by a line
%       break.
%    \item[marginbreak]
%       The theorem number is set in the margin, and there is a line
%       break as in \texttt{break}.
%    \item[changebreak]
%       Like \texttt{break}, but with header number and text
%       interchanged.
%    \item[change]
%       Header number and text are interchanged, without a line break.
%    \item[margin]
%       The number is set in the left margin, without a line break.
% \end{deflist}
% All styles (except \texttt{plain}) select |\slshape| as the default
% |\theorembodyfont|.
%
%
% \subsection{Examples}
%
% Given the above theorem sets {\envfont Cor}, {\envfont Exa}
% and {\envfont Rem},
% suppose that the preamble also contains the declarations:
% \begin{verbatim}
%    \theoremstyle{marginbreak}   \newtheorem{Lem}[Cor]{Lemma}
%    \theoremstyle{change}
%    \theorembodyfont{\itshape}        \newtheorem{Def}[Cor]{Definition}
%
%    \theoremheaderfont{\scshape}
%\end{verbatim}
% Then the following are some typical examples of the typeset output
% resulting from their use.
%
% \begin{Cor}
%    This is a sentence typeset in the theorem environment {\envfont
%    Cor}.
% \end{Cor}
%
% \begin{Exa}
%    This is a sentence typeset in the theorem environment {\envfont
%    Exa}.
% \end{Exa}
%
% \begin{Rem}
%    This is a sentence typeset in the theorem environment {\envfont
%    Rem}.
% \end{Rem}
%
% \begin{Lem}[Ben User]
%    This is a sentence typeset in the theorem environment {\envfont
%    Lem}.
% \end{Lem}
%
% \begin{Def}[Very impressive Definition]
%    This is a sentence typeset in the theorem environment {\envfont
%    Def}.
% \end{Def}
%
% The last two examples show the effect of the optional argument to
% a theorem environment (it is the text typeset in parentheses).
%
%
% \section{Special Considerations}
% \label{sec:newfonts}
%
% Theoremheader and body are implemented as a unit. This means that
% the |\theoremheaderfont|  will inherit characteristics of the
% |\theorembodyfont| in \LaTeXe. Thus, if for example
% |\theorembodyfont| is |\itshape| and |\theoremheaderfont|
% is |\bfseries| the font selected for the header will have the
% characteristics `bold extended italic'. If this is not desired one
% should set the |\theoremheaderfont| to something like
% \begin{verbatim}
%      \theoremheaderfont{\normalfont\bfseries}
%\end{verbatim}
% i.e.\ supplying all necessary font information explicitly.
%
%
%
% \section{Acknowledgements}
%
% The publication of this set of macros was only possible with the
% help of Christina Busse (translating the manuscript into English),
% Joachim Pense (playing the r\^ole of typist), Chris Rowley
% (looking everything over) and many others providing useful
% suggestions.
%
% \MaybeStop{%
%    \begin{thebibliography}{9}
%       \bibitem{bk:GMS94} \textsc{M.~Goossens}, \textsc{F.~Mittelbach}
%         and \textsc{A.~Samarin}.
%         \newblock The \LaTeX{} Companion.
%         \newblock
%         Addison-Wesley, Reading, Massachusetts, 1994.
%       \bibitem{latex:source} \textsc{Lamport, Leslie.} \newblock
%                               \texttt{latex.tex}, version 2.09, date
%                               Feb. 1990.
%    \end{thebibliography}
%    \PrintIndex
%    \PrintChanges}
%
%
% \section{The documentation driver file}
%
% The next bit of code contains the documentation driver file for
% \TeX{}, i.e., the file that will produce the documentation you are
% currently reading. It will be extracted from this file by the
% \texttt{docstrip} program. Since it is the first code in the file
% one can alternatively process this file directly with \LaTeXe{} to
% obtain the documentation.
%
%    \begin{macrocode}
%<*driver>
\documentclass{ltxdoc}

\usepackage{theorem}

%<+driver>% The next few lines define theorem sets which are used
%<+driver>% in the example section of the documentation.

\theoremstyle{break}       \newtheorem{Cor}{Corollary}
\theoremstyle{plain}       \newtheorem{Exa}{Example}[section]
{\theorembodyfont{\upshape}\newtheorem{Rem}{Remark}}
\theoremstyle{marginbreak} \newtheorem{Lem}[Cor]{Lemma}
\theoremstyle{change}
\theorembodyfont{\itshape} \newtheorem{Def}[Cor]{Definition}

\theoremheaderfont{\scshape}

\RecordChanges

\begin{document}
   \DocInput{theorem.dtx}
\end{document}
%</driver>
%    \end{macrocode}
%
% \section{Definition of the Macros}
%
% If the file has been loaded before, we abort immediately. If not the
% package announces itself (this is actually done at the very top of
% the file---the way it is done isn't good style so don't copy it).
% \changes{v2.0e}{89/07/19}{Spaces removed from `typeout'.}
% \changes{v2.2b}{95/11/19}{Announce moved to top of file}
%    \begin{macrocode}
%<*package>
%\@ifundefined{theorem@style}{}{\endinput}
%\def\FMithmInfo{1995/11/19 v2.2b Theorem extension package (FMi)}
%\ProvidesPackage{theorem}[\FMithmInfo]
%    \end{macrocode}
%
%
% \subsection{Definition of theorem styles and fonts}
%
% All the definitions in this file are done globally to allow
% inputting this file inside a group.
%
% \begin{macro}{\theoremstyle}
%    Before a theorem style can be installed, the chosen style must be
%    known. For that reason, we must test to see that
%    |\th@|\meta{style} is known or, more precisely, that it is
%    different from |\relax|. If the style is not known then
%    |\th@plain| is used.
%    \begin{macrocode}
\gdef\theoremstyle#1{%
   \@ifundefined{th@#1}{\@warning
          {Unknown theoremstyle `#1'. Using `plain'}%
          \theorem@style{plain}}%
%    \end{macrocode}
%    We save the theorem style to be used in the token register
%    |\theorem@style|.
%    \begin{macrocode}
      {\theorem@style{#1}}%
%    \end{macrocode}
%    Now we ``evaluate'' the theorem style: this means, we call the
%    macro |\th@|\meta{style} which will activate the relevant
%    definitions which are contained in a separate file.  This is done
%    in a group to suppress changes to the current font.  This could
%    otherwise pose problems together with the new font selection
%    scheme\footnote{When I printed the original article using the new
%    font selection scheme I ended with a document with slanted
%    typefaces (text headings and all) simply because one of the
%    theorem styles used \texttt{\bslash sl} at toplevel.} if the
%    |\th@|\meta{style} is evaluated a second time.
% \changes{v2.1a}{90/02/15}{Group added around `th@..}
% \changes{v2.1a}{90/02/15}{Reseting of `..@begintheorem removed.}
%    \begin{macrocode}
      \begingroup
        \csname th@\the\theorem@style \endcsname
      \endgroup}
%    \end{macrocode}
% \end{macro}
%
%
% \begin{macro}{\@begintheorem}
% \changes{v2.1a}{90/02/15}{Relaxed at toplevel.}
% \begin{macro}{\@opargbegintheorem}
% \changes{v2.1a}{90/02/15}{Relaxed at toplevel.}
%    We reset |\@begintheorem| and |\@opargbegintheorem| to
%    |\relax| since these commands are no longer necessary at
%    toplevel. This will save a few tokens.
%    \begin{macrocode}
\global\let\@begintheorem\relax
\global\let\@opargbegintheorem\relax
%    \end{macrocode}
% \end{macro}
% \end{macro}
%
% \begin{macro}{\theorem@style}
%    Obviously the \textsf{token} register used above has to be
%    allocated. To assure the utmost compatibility with the
%    original \LaTeX{} definition, we set the default theorem style to
%    \texttt{plain}, which implements the usual \LaTeX{} convention.
% \changes{v2.0f}{89/07/20}{Unnecessary `global' removed.}
%    \begin{macrocode}
\newtoks\theorem@style
\global\theorem@style{plain}
%    \end{macrocode}
% \end{macro}
%
% \begin{macro}{\theorembodyfont}
% \begin{macro}{\theorem@bodyfont}
%    If the |\theorembodyfont| is set by the user then it should not
%    interact with the default font set in the theorem style. When the
%    new font selection is in force this may happen if, for example,
%    the default is |\itshape| and the new |\theorembodyfont| is
%    |\sffamily|. So we add a |\reset@font| command in front of the
%    user definition.
% \changes{v2.0f}{89/07/20}{Unnecessary `global' removed.}
% \changes{v2.0g}{89/09/19}{`theoremfont to `theorembodyfont.}
% \changes{v2.1c}{92/06/22}{`reset@font added and logic changed.}
% \changes{v2.1d}{92/10/28}{Reset to empty if arg is empty}
%    \begin{macrocode}
\gdef\theorembodyfont#1{%
%    \end{macrocode}
%    We check if the argument supplied is empty and if so put nothing
%    into the |\theorem@bodyfont| token register to allow for
%    |\theorembodyfont{}| as a mean of using the default of the
%    current |\theoremstyle|.
%    \begin{macrocode}
   \def\@tempa{#1}%
   \ifx\@tempa\@empty
    \theorem@bodyfont{}%
   \else
    \theorem@bodyfont{\reset@font#1}%
   \fi
   }
\newtoks\theorem@bodyfont
\global\theorem@bodyfont{}
%    \end{macrocode}
% \end{macro}
% \end{macro}
%
%
% \begin{macro}{\theoremheaderfont}
% \changes{v2.0g}{89/09/19}{Macro added.}
%    The font for the theorem headers is handled differently because
%    this definition applies to all theorem styles.
%    \begin{macrocode}
\gdef\theoremheaderfont#1{\gdef\theorem@headerfont{#1}%
%    \end{macrocode}
%    After using the macro once it is redefined to produce an error
%    message.
%    \begin{macrocode}
       \gdef\theoremheaderfont##1{%
        \typeout{\string\theoremheaderfont\space should be used
                 only once.}}}
%    \end{macrocode}
% \end{macro}
%
% \begin{macro}{\theorem@headerfont}
% \changes{v2.0g}{89/09/19}{Macro added.}
%    To set the |\theorem@headerfont| default we first test if
%    the new fontselection scheme is in force.
% \changes{v2.1a}{90/02/15}{Support for new font selection.}
%    \begin{macrocode}
\ifx\upshape\undefined
%    \end{macrocode}
%    If not we define it to expand into |\bfseries|. We don't use
%    |\let| just in case a following style option redefines this
%    macro.
%    \begin{macrocode}
\gdef\theorem@headerfont{\bfseries}
%    \end{macrocode}
%    Otherwise we reset the current shape before calling |\bfseries|.
%    \begin{macrocode}
\else \gdef\theorem@headerfont{\normalfont\bfseries}\fi
%    \end{macrocode}
% \end{macro}
%
% \begin{macro}{\th@plain}
% \begin{macro}{\th@break}
% \begin{macro}{\th@marginbreak}
% \begin{macro}{\th@changebreak}
% \begin{macro}{\th@change}
% \begin{macro}{\th@margin}
%    The different styles are defined in macros such as
%    |\th@plain|.  Since memory space is precious in
%    ``non-Big-versions'', we have to avoid offering too many unused
%    definitions.  Therefore we define these styles in separate files
%    that can be loaded on demand.  Thus the commands themselves only
%    load these files. We use |\@input@|, a \LaTeXe{} internal command
%    that ensures that the file will be listed with |\listfiles|.
% \changes{v2.2a}{94/02/02}{Use `@input@ to load theorem layout files}
%    \begin{macrocode}
\gdef\th@plain{\@input@{thp.sty}}
\gdef\th@break{\@input@{thb.sty}}
\gdef\th@marginbreak{\@input@{thmb.sty}}
\gdef\th@changebreak{\@input@{thcb.sty}}
\gdef\th@change{\@input@{thc.sty}}
\gdef\th@margin{\@input@{thm.sty}}
%    \end{macrocode}
%    This list will be expanded when new styles become
%    available.  For testing,  just append new theorem substyles
%    as document options.
% \end{macro}
% \end{macro}
% \end{macro}
% \end{macro}
% \end{macro}
% \end{macro}
%
%
%
% \subsection{Definition of a new theorem set}
%
% As already pointed out, a new theorem environment can be defined in
% three different ways:
% \begin{verbatim}
%    \newtheorem{Lem}{Lemma}
%    \newtheorem{Lem}{Lemma}[section]
%    \newtheorem{Lem}[Theorem]{Lemma}
%\end{verbatim}
% The function of the macro |\newtheorem| is to recognize these cases
% and then to branch into one of the three macros |\@ynthm|, |\@xnthm|
% or |\@othm|.  This mechanism is adopted unchanged from
% \cite{latex:source}; the essential point here is that, for example,
% in the second case, the arguments \texttt{Lem}, \texttt{Lemma} and
% \texttt{section} are passed over to the macro |\@xnthm|.
%
% We inspect this case first because the others present fewer problems,
% and thus are easily derived from this one.
%
% \begin{macro}{\@xnthm}
%    For our example arguments, the macro |\@xnthm| must fulfill
%    the following:
%    \begin{itemize}
%       \item
%          Define a new \LaTeX-counter `Lem'
%       \item
%          reset this counter within a |\section|
%       \item
%          define the macro |\theLem|
%       \item
%          define the environment macros |\Lem| and
%          |\endLem| using the current
%          |\theoremstyle| and |\theorem@bodyfont|.
%    \end{itemize}
%    Obviously, all this should happen only if the first argument
%    of |\@xnthm| (i.e.\ \texttt{Lem} in our example) is chosen so
%    as not to conflict with any previously defined commands or
%    environments.  This test is performed by the \LaTeX{} macro
%    |\@ifdefinable|.
%    \begin{macrocode}
\gdef\@xnthm#1#2[#3]{\expandafter\@ifdefinable\csname #1\endcsname
%    \end{macrocode}
%    Therefore, the first argument of |\@ifdefinable| is the
%    expansion (in the example, |\Lem|) of
%    |\csname||#1\endcsname|.  The second argument is
%    executed only if the test has been completed successfully.
%    \begin{macrocode}
   {%
%    \end{macrocode}
%    Now we define the new counter. The names of the \LaTeX{}
%    macros employed should speak for themselves:
% \changes{v2.2b}{1995/10/07}{Use \cs{@newctr}}
% \changes{v2.2c}{1995/11/23}{Use \cs{@newctr} properly.}
%    \begin{macrocode}
    \@definecounter{#1}\@newctr{#1}[#3]%
%    \end{macrocode}
%    Using |\@newctr| will give a proper error message if the counter
%    in |#3| is not defined.
%    In defining `|\theLem|' we must generate the desired macro
%    name by use of |\expandafter| and |\csname|.
%    \begin{macrocode}
    \expandafter\xdef\csname the#1\endcsname
%    \end{macrocode}
%    An |\xdef| is used in order to make the definition global,
%    and to ensure that it contains the replacement texts of
%    |\@thmcountersep| and |\@thmcounter|.\footnote{These
%    two macros can be defined by the document style.  Their default
%    values produce a `\texttt{.}' as separation and an arabic
%    representation of the number.} However, not everything should be
%    expanded. For example, it saves space to use |\thesection|
%    instead of its---at times---lengthy expansion.
%    \begin{macrocode}
      {\expandafter \noexpand \csname the#3\endcsname
       \@thmcountersep \@thmcounter{#1}}%
%    \end{macrocode}
%    Thus with the defaults of \LaTeX{}, |\theLem| would be
%    replaced by the command sequence |\thesection.\arabic{Lem}|.
%
%    We will now look at the definition of the macro which is executed
%    at the beginning of the actual environment (in our example this
%    macro is |\Lem|). It should be noted that we use an
%    ``|\expandafter| trick'' to expand only certain parts of the
%    replacement text at the time of the definition.
%    \begin{macrocode}
    \def\@tempa{\global\@namedef{#1}}%
    \expandafter \@tempa \expandafter{%
%    \end{macrocode}
%    First, the macro that contains the current definitions of
%    |\@begintheorem| and |\@opargtheorem| should be called
%    up.  The name of this macro---as is already known---has the form
%    |\th@|\meta{theorem\ style}; therefore, it must be called by
%    \begin{macrocode}
      \csname th@\the \theorem@style
%    \end{macrocode}
%    In addition the default theorem font should be changeable, i.e.\
%    we have to insert the contents of |\theorem@bodyfont|. For
%    that reason, we expand even further, beyond |\endcsname|,
%    and thus insert the contents of the \textsf{token} register
%    |\theorem@bodyfont| in the replacement text.
%    \begin{macrocode}
            \expandafter \endcsname \the \theorem@bodyfont
%    \end{macrocode}
%    Now it is time to call the macro |\@thm| which takes over
%    the further processing.  It has two arguments: the current
%    counter name (in our example, \texttt{Lem}), and the text of the
%    label (in our example, \texttt{Lemma}).
%    \begin{macrocode}
     \@thm{#1}{#2}}%
%    \end{macrocode}
%    With this, the `sub-definition' is complete.  The macro
%    |\@endtheorem| ends a theorem environment and is, so far, nothing
%    but an |\endtrivlist|.  (Hence it is defined globally, and not
%    within the theorem styles.\footnote{This has to be changed as
%    soon as theorem styles that change \texttt{\bslash @endtheorem}
%    exist.  In such a case, all existing styles must be changed as
%    well since they will have to reset the macro.}) Therefore, we can
%    set it equivalent to the macro that ends the theorem set (in our
%    example, |\endLem|).  However, if some day theorem styles exist
%    that do change |\@endtheorem|, we would have to use the
%    commented-out line instead.
%    \begin{macrocode}
    \global \expandafter \let \csname end#1\endcsname \@endtheorem
%   \global\@namedef{end#1}{\@endtheorem}%
%    \end{macrocode}
%    With these commands all the required definitions are employed,
%    unless the test |\@ifdefinable| has failed.  Therefore, we
%    end the second argument of this macro and with it the definition
%    of |\@xnthm|.
%    \begin{macrocode}
   }}
%    \end{macrocode}
% \end{macro}
%
% \begin{macro}{\@ynthm}
%    The definition of |\@ynthm| is completely analogous. In
%    this case the new counter that is defined is not reset within
%    another counter; thus the definition of |\the...|
%    is simplified:
%    \begin{macrocode}
\gdef\@ynthm#1#2{\expandafter\@ifdefinable\csname #1\endcsname
   {\@definecounter{#1}%
    \expandafter\xdef\csname the#1\endcsname{\@thmcounter{#1}}%
%    \end{macrocode}
%    The rest of the definition corresponds literally to that of
%    |\@xnthm|:
%    \begin{macrocode}
    \def\@tempa{\global\@namedef{#1}}\expandafter \@tempa
     \expandafter{\csname th@\the \theorem@style \expandafter
     \endcsname \the\theorem@bodyfont \@thm{#1}{#2}}%
    \global \expandafter \let \csname end#1\endcsname \@endtheorem}}
%    \end{macrocode}
% \end{macro}
%
% \begin{macro}{\@othm}
%    The definition of |\@othm| does not contain anything new.
%    \begin{macrocode}
\gdef\@othm#1[#2]#3{%
%    \end{macrocode}
% \changes{v2.2b}{95/10/07}{Build in check for counter defined}
%    We do not define a new counter but instead use one that has
%    already been defined.  Thus the only definition we need is that
%    of this pseudo-counter (i.e.\ |\the|\meta{env.\ name}).
%    First we check if |#2| corresponds to a known counter name.
%    \begin{macrocode}
  \expandafter\ifx\csname c@#2\endcsname\relax
   \@nocounterr{#2}%
  \else
   \expandafter\@ifdefinable\csname #1\endcsname
   {\expandafter \xdef \csname the#1\endcsname
     {\expandafter \noexpand \csname the#2\endcsname}%
%    \end{macrocode}
%    All other parts of the definition can be adopted from
%    |\@xnthm|. We have to remember, though, that in this case
%    the name of the current counter and the theorem label have moved
%    to the second and third arguments.
%    \begin{macrocode}
    \def\@tempa{\global\@namedef{#1}}\expandafter \@tempa
     \expandafter{\csname th@\the \theorem@style \expandafter
     \endcsname \the\theorem@bodyfont \@thm{#2}{#3}}%
    \global \expandafter \let \csname end#1\endcsname \@endtheorem}%
  \fi}
%    \end{macrocode}
% \end{macro}
%
% \subsection{Macros that are employed in a {\envfont theorem}
%    environment}
%
% \begin{macro}{\@thm}
%    The macro |\@thm| has to increase the current counter.  Then,
%    depending on whether the environment has (or does not have) an
%    optional argument, it has to branch into either |\@begintheorem|
%    or |\@opargtheorem|.
%    \begin{macrocode}
\gdef\@thm#1#2{\refstepcounter{#1}%
%    \end{macrocode}
%    Now we start a {\envfont trivlist} environment, and give
%    |\@topsep| and |\@topsepadd| the values of the \textsf{skip}
%    registers |\theorempreskipamount| and |\theorempostskipamount|.
%    The value in |\@topsep| is the vertical space that is inserted by
%    the first (and only) |\item| in our |\trivlist| whilst
%    |\@topsepadd| is inserted by |\@endparenv| at the end of that
%    {\envfont trivlist} environment.  By using these registers, we
%    obtain the desired space around a {\envfont theorem} environment.
%    \begin{macrocode}
   \trivlist
   \@topsep \theorempreskipamount               % used by first \item
   \@topsepadd \theorempostskipamount           % used by \@endparenv
%    \end{macrocode}
%    Now we have to test whether an optional argument has been given.
% \changes{v2.0d}{89/07/18}{Forgotten percent added.}
%    \begin{macrocode}
   \@ifnextchar [%
%    \end{macrocode}
%    If there is an optional argument, we will call |\@ythm|, and
%    move the arguments read back into the input stream.
%    \begin{macrocode}
   {\@ythm{#1}{#2}}%
%    \end{macrocode}
%    If not, we call |\@begintheorem|. Its first argument is the
%    name of the theorem set (hence the second argument of
%    |\@thm|). Its second argument is the macro that produces the
%    current number.
%    \begin{macrocode}
   {\@begintheorem{#2}{\csname the#1\endcsname}\ignorespaces}}
%    \end{macrocode}
% \end{macro}
%
% \begin{macro}{\@xthm}
% \begin{macro}{\@ythm}
%    Both these macros were originally called by |\@thm|.  We do
%    not need |\@xthm| anymore, hence we reset it to
%    |\relax|. The definition of |\@ythm| has not changed at
%    all from its definition in \LaTeX. In order to make the macros
%    easier to understand, we will
%    nevertheless present it (commented out).
%    \begin{macrocode}
\global\let\@xthm\relax
% \def\@ythm#1#2[#3]{\@opargbegintheorem{#2}{\csname
%           the#1\endcsname}{#3}\ignorespaces}
%</package>
%    \end{macrocode}
%    The primitive |\ignorespaces| in |\@ythm| and
%    |\@thm| is needed to remove the spaces between the
%    |\begin{...}| and the actual text.
% \end{macro}
% \end{macro}
%
%
% \subsection{Definition of the theorem substyles}
%
% As already pointed out, the theorem substyles, defined below, are
% only loaded when necessary.  Note that all these substyles, except
% \texttt{plain}, have |\slshape| as the default body font.
%
% \subsubsection{The \texttt{plain} style}
%
% \label{dok:th@plain} As the following macros use |@|, we have
% to locally set the |\catcode| of this symbol to ``letter''.
% This happens within a group, so that we do not have to worry about
% which |\catcode| that symbol had before.
%    \begin{macrocode}
%<*thp>
\begingroup \makeatletter
%    \end{macrocode}
% Since we are now within a group, we must make all definitions
% globally.  First we make sure that \texttt{theorem.sty} is loaded.
% This will allow us to use this file as a document style option
% without having to call \texttt{theorem} itself as an option.  At the
% same time, we assure that at least version 2 is loaded, since
% |\theorem@style| was not defined in earlier versions.
%    \begin{macrocode}
\@ifundefined{theorem@style}{\input{theorem.sty}}{}
\ProvidesFile{thp.sty}
             [\FMithmInfo]
%    \end{macrocode}
%
%
% \begin{macro}{\th@plain}
% \begin{macro}{\@begintheorem}
% \begin{macro}{\@opargbegintheorem}
%    |\theoremstyle{plain}| corresponds to the original
%    definition, except that the distances to the surrounding text are
%    determined by the parameters |\theorempreskipamount| and
%    |\theorempostskipamount|.
%    First we set the default body font.
%    \begin{macrocode}
\gdef\th@plain{\normalfont\itshape
%    \end{macrocode}
%    Then we define |\@begintheorem| and
%    |\@opargbegintheorem|. These two macros define how the
%    header of a theorem is typeset.  |\@opargbegintheorem| will
%    be called if a \textsf{theorem} environment with an optional
%    argument is encountered; otherwise, the header is constructed by
%    calling |\@begintheorem|.  If one of these macros is
%    executed, we are within a \textsf{trivlist} environment started by
%    |\@thm|.  So the theorem header is produced with an
%    |\item| command.
%
%    Instead of specifying the header font directly, all standard
%    theorem styles use the |\theorem@headerfont| macro to allow
%    customization.  The extra space (|\labelsep|) is necessary
%    because of problems in the \textsf{trivlist} environment.
%    \begin{macrocode}
  \def\@begintheorem##1##2{%
        \item[\hskip\labelsep \theorem@headerfont ##1\ ##2]}%
%    \end{macrocode}
%    The definition of |\@opargbegintheorem| is completely
%    analogous. The only difference is the fact that there exists a
%    third argument (which is the optional parameter of the
%    environment and contains additional information about the
%    theorem).  Customarily we enclose it in parentheses.
%    \begin{macrocode}
\def\@opargbegintheorem##1##2##3{%
   \item[\hskip\labelsep \theorem@headerfont ##1\ ##2\ (##3)]}}
%    \end{macrocode}
% \end{macro}
% \end{macro}
% \end{macro}
%
% \noindent
% We conclude with an |\endgroup| to restore the
% |\catcode| of |@|.
%    \begin{macrocode}
\endgroup
%</thp>
%    \end{macrocode}
%
%
% \subsubsection{The \texttt{break} style}
%
% This style option is stored in the file |thb.sty|.
% For the next two lines see the documentation for |\th@plain| on
% page~\pageref{dok:th@plain}.
%    \begin{macrocode}
%<*thb>
\begingroup \makeatletter
\@ifundefined{theorem@style}{\input{theorem.sty}}{}
\ProvidesFile{thb.sty}
             [\FMithmInfo]
%    \end{macrocode}
%
% \begin{macro}{\th@break}
%    |\theoremstyle{break}| produces a line break after the name
%    of the theorem.  The font is |\slshape|.  Hence, we define
%    |\th@break| as follows:
%    \begin{macrocode}
\gdef\th@break{\normalfont\slshape
  \def\@begintheorem##1##2{\item[%
%    \end{macrocode}
%    We run into the following problem: it is not possible to create
%    the header with |\item[|\meta{title}|]| and then start
%    a new line by, for example, |\mbox{}\\|. Such a definition
%    will fail whenever a list environment follows immediately.  With
%    the above construction, the |\mbox{}| causes the switch
%    |@inlabel| (cf.\ definition of |\list| and
%    |\trivlist| in \cite{latex:source}) to be set to \texttt{false}
%    and so the following list will insert additional vertical
%    space (|\topskip|). This is quite annoying. Therefore, we
%    create the line break within the |\item|. In order to ensure
%    that the text will begin at the proper position in the following
%    line, we simply pretend that the label does not take any
%    room.\footnote{This will lead to problems whenever very high
%    symbols occurring in the line tower into the
%    heading. So, something else has to be done here sometime.}
%    \begin{macrocode}
       \rlap{\vbox{\hbox{\hskip \labelsep\theorem@headerfont ##1\ ##2}%
                   \hbox{\strut}}}]}%
%    \end{macrocode}
%    Again, the definition of |\@opargbegintheorem| is
%    completely analogous.
%    \begin{macrocode}
\def\@opargbegintheorem##1##2##3{%
  \item[\rlap{\vbox{\hbox{\hskip \labelsep \theorem@headerfont
                     ##1\ ##2\ (##3)}%
                    \hbox{\strut}}}]}}
\endgroup
%</thb>
%    \end{macrocode}
% \end{macro}
%
%
% \subsubsection{The \texttt{changebreak} style}
%
%    \begin{macrocode}
% This style option is stored in the file |thcb.sty|.
%    \begin{macrocode}
%<*thcb>
\begingroup \makeatletter
\@ifundefined{theorem@style}{\input{theorem.sty}}{}
\ProvidesFile{thcb.sty}
             [\FMithmInfo]
%    \end{macrocode}
%
% \begin{macro}{\th@changebreak}
%    The \texttt{change-break} theorem style is like \texttt{break}
%    but with interchange of theorem name and theorem number.  Thus we
%    define |\th@changebreak| as follows:
%    \begin{macrocode}
\gdef\th@changebreak{\normalfont\slshape
  \def\@begintheorem##1##2{\item
       [\rlap{\vbox{\hbox{\hskip\labelsep \theorem@headerfont ##2\ ##1}%
                    \hbox{\strut}}}]}%
\def\@opargbegintheorem##1##2##3{%
  \item[\rlap{\vbox{\hbox{\hskip\labelsep \theorem@headerfont
                     ##2\ ##1\ (##3)}%
                    \hbox{\strut}}}]}}
\endgroup
%</thcb>
%    \end{macrocode}
% \end{macro}
%
%
% \subsubsection{The \texttt{change} style}
%
% This style option is stored in the file |thc.sty|.
%    \begin{macrocode}
%<*thc>
\begingroup \makeatletter
\@ifundefined{theorem@style}{\input{theorem.sty}}{}
\ProvidesFile{thc.sty}
             [\FMithmInfo]
%    \end{macrocode}
%
% \begin{macro}{\th@change}
%    The \texttt{change} theorem style corresponds to the
%    \texttt{change break} style without a linebreak after the header.
%    To say it in another way, it's the same as the \texttt{plain}
%    style but with number and name interchanged and |\slshape| as the
%    default font.
%    \begin{macrocode}
\gdef\th@change{\normalfont\slshape
  \def\@begintheorem##1##2{\item
       [\hskip\labelsep \theorem@headerfont ##2\ ##1]}%
\def\@opargbegintheorem##1##2##3{%
  \item[\hskip\labelsep \theorem@headerfont ##2\ ##1\ (##3)]}}
\endgroup
%</thc>
%    \end{macrocode}
% \end{macro}
%
%
% \subsubsection{The \texttt{marginbreak} style}
%
% This style option is the one used most often at Mainz.  It is saved
% in the file |thmb.sty|.
%    \begin{macrocode}
%<*thmb>
\begingroup \makeatletter
\@ifundefined{theorem@style}{\input{theorem.sty}}{}
\ProvidesFile{thmb.sty}
             [\FMithmInfo]
%    \end{macrocode}
%
% \begin{macro}{\th@marginbreak}
%    The \texttt{margin break} style is nearly the same as the
%    \texttt{change break} style. The only difference is the placement
%    of the theorem number.  We use |\llap| to place it in the left
%    margin.
%
%    In this style |\labelsep| denotes the separation between the
%    number and the text.
%    \begin{macrocode}
\gdef\th@marginbreak{\normalfont\slshape
  \def\@begintheorem##1##2{\item
       [\rlap{\vbox{\theorem@headerfont
                    \hbox{\llap{##2}\hskip\labelsep ##1}%
                    \hbox{\strut}}}]}%
\def\@opargbegintheorem##1##2##3{%
  \item[\rlap{\vbox{\theorem@headerfont
                    \hbox{\llap{##2}\hskip\labelsep ##1\ (##3)}%
                    \hbox{\strut}}}]}}
\endgroup
%</thmb>
%    \end{macrocode}
% \changes{v2.1c}{90/02/17}{`endgroup outside macrocode env.}
% \end{macro}
%
%
%
% \subsubsection{The \texttt{margin} style}
%
% This style option is stored in the file |thm.sty|.
%    \begin{macrocode}
%<*thm>
\begingroup \makeatletter
\@ifundefined{theorem@style}{\input{theorem.sty}}{}
\ProvidesFile{thm.sty}
             [\FMithmInfo]
%    \end{macrocode}
%
% \begin{macro}{\th@margin}
%    Again this is only a variant of the theorem styles described
%    above without any new ideas.
%    \begin{macrocode}
\gdef\th@margin{\normalfont\slshape
  \def\@begintheorem##1##2{\item
       [\theorem@headerfont \llap{##2}\hskip\labelsep ##1]}%
\def\@opargbegintheorem##1##2##3{%
  \item[\theorem@headerfont \llap{##2}\hskip\labelsep ##1\ (##3)]}}
\endgroup
%</thm>
%    \end{macrocode}
% \end{macro}
%
%
%
%
% \subsection{Final Definitions}
%
% \begin{macro}{\theorempreskipamount}
% \begin{macro}{\theorempostskipamount}
%    The \textsf{skip} parameters that regulate the vertical empty space
%    before and after the {\envfont theorem} environment have
%    to be allocated as well.
% \changes{v2.0f}{89/07/20}{Unnecessary `global' removed.}
%    \begin{macrocode}
%<*package>
\newskip\theorempreskipamount
\newskip\theorempostskipamount
%    \end{macrocode}
%    Since we have used the same values for all theorem sets, we now can
%    assign them.
%    \begin{macrocode}
\global\setlength\theorempreskipamount{12pt plus 5pt minus 3pt}
\global\setlength\theorempostskipamount{8pt plus 3pt minus 1.5pt}
%    \end{macrocode}
% \end{macro}
% \end{macro}
%
%
% \begin{macro}{\@endtheorem}
%    The same holds for the macro |\@endtheorem|, which ends a
%    {\envfont theorem} environment. Since it is the same for all
%    theorem sets, it is removed from the macros |\th@|\meta{style}.
%    It simply ends the {\envfont trivlist} environment, which was
%    begun in |\@thm|.
%    \begin{macrocode}
\global\let\@endtheorem=\endtrivlist
%    \end{macrocode}
% \end{macro}
%
%
%
% \begin{macro}{\@preamblecmds}
%    All macros defined above are to be used only in the preamble.
%    Therefore, we insert them in |\@preamblecmds| which will disable
%    them at begin document. This is done by the internal \LaTeXe{}
%    command |\@onlypreamble|.
% \changes{v2.0g}{89/09/19}{`theoremheaderfont and `theorembodyfont
%    added.}
% \changes{v2.2a}{94/02/03}{Use \LaTeXe{} interface}
%    \begin{macrocode}
\@onlypreamble\@xnthm
\@onlypreamble\@ynthm
\@onlypreamble\@othm
\@onlypreamble\newtheorem
\@onlypreamble\theoremstyle
\@onlypreamble\theorembodyfont
\@onlypreamble\theoremheaderfont
%    \end{macrocode}
% \end{macro}
%
% Finally we declare the \texttt{plain} theorem style to be the default.
% \changes{v2.1b}{90/06/04}{Fixed missing `item by declaring default
%    style.}
%    \begin{macrocode}
\theoremstyle{plain}
%</package>
%    \end{macrocode}
%
% \Finale

\endinput
}{}
\ProvidesFile{thm.sty}
             [\FMithmInfo]
%    \end{macrocode}
%
% \begin{macro}{\th@margin}
%    Again this is only a variant of the theorem styles described
%    above without any new ideas.
%    \begin{macrocode}
\gdef\th@margin{\normalfont\slshape
  \def\@begintheorem##1##2{\item
       [\theorem@headerfont \llap{##2}\hskip\labelsep ##1]}%
\def\@opargbegintheorem##1##2##3{%
  \item[\theorem@headerfont \llap{##2}\hskip\labelsep ##1\ (##3)]}}
\endgroup
%</thm>
%    \end{macrocode}
% \end{macro}
%
%
%
%
% \subsection{Final Definitions}
%
% \begin{macro}{\theorempreskipamount}
% \begin{macro}{\theorempostskipamount}
%    The \textsf{skip} parameters that regulate the vertical empty space
%    before and after the {\envfont theorem} environment have
%    to be allocated as well.
% \changes{v2.0f}{89/07/20}{Unnecessary `global' removed.}
%    \begin{macrocode}
%<*package>
\newskip\theorempreskipamount
\newskip\theorempostskipamount
%    \end{macrocode}
%    Since we have used the same values for all theorem sets, we now can
%    assign them.
%    \begin{macrocode}
\global\setlength\theorempreskipamount{12pt plus 5pt minus 3pt}
\global\setlength\theorempostskipamount{8pt plus 3pt minus 1.5pt}
%    \end{macrocode}
% \end{macro}
% \end{macro}
%
%
% \begin{macro}{\@endtheorem}
%    The same holds for the macro |\@endtheorem|, which ends a
%    {\envfont theorem} environment. Since it is the same for all
%    theorem sets, it is removed from the macros |\th@|\meta{style}.
%    It simply ends the {\envfont trivlist} environment, which was
%    begun in |\@thm|.
%    \begin{macrocode}
\global\let\@endtheorem=\endtrivlist
%    \end{macrocode}
% \end{macro}
%
%
%
% \begin{macro}{\@preamblecmds}
%    All macros defined above are to be used only in the preamble.
%    Therefore, we insert them in |\@preamblecmds| which will disable
%    them at begin document. This is done by the internal \LaTeXe{}
%    command |\@onlypreamble|.
% \changes{v2.0g}{89/09/19}{`theoremheaderfont and `theorembodyfont
%    added.}
% \changes{v2.2a}{94/02/03}{Use \LaTeXe{} interface}
%    \begin{macrocode}
\@onlypreamble\@xnthm
\@onlypreamble\@ynthm
\@onlypreamble\@othm
\@onlypreamble\newtheorem
\@onlypreamble\theoremstyle
\@onlypreamble\theorembodyfont
\@onlypreamble\theoremheaderfont
%    \end{macrocode}
% \end{macro}
%
% Finally we declare the \texttt{plain} theorem style to be the default.
% \changes{v2.1b}{90/06/04}{Fixed missing `item by declaring default
%    style.}
%    \begin{macrocode}
\theoremstyle{plain}
%</package>
%    \end{macrocode}
%
% \Finale

\endinput
}{}
\ProvidesFile{thc.sty}
             [\FMithmInfo]
%    \end{macrocode}
%
% \begin{macro}{\th@change}
%    The \texttt{change} theorem style corresponds to the
%    \texttt{change break} style without a linebreak after the header.
%    To say it in another way, it's the same as the \texttt{plain}
%    style but with number and name interchanged and |\slshape| as the
%    default font.
%    \begin{macrocode}
\gdef\th@change{\normalfont\slshape
  \def\@begintheorem##1##2{\item
       [\hskip\labelsep \theorem@headerfont ##2\ ##1]}%
\def\@opargbegintheorem##1##2##3{%
  \item[\hskip\labelsep \theorem@headerfont ##2\ ##1\ (##3)]}}
\endgroup
%</thc>
%    \end{macrocode}
% \end{macro}
%
%
% \subsubsection{The \texttt{marginbreak} style}
%
% This style option is the one used most often at Mainz.  It is saved
% in the file |thmb.sty|.
%    \begin{macrocode}
%<*thmb>
\begingroup \makeatletter
\@ifundefined{theorem@style}{% \iffalse meta-comment
%
% Copyright (C) 1993-2024
%
% The LaTeX Project and any individual authors listed elsewhere
% in this file.
%
% This file is part of the Standard LaTeX `Tools Bundle'.
% -------------------------------------------------------
%
% It may be distributed and/or modified under the
% conditions of the LaTeX Project Public License, either version 1.3c
% of this license or (at your option) any later version.
% The latest version of this license is in
%    https://www.latex-project.org/lppl.txt
% and version 1.3c or later is part of all distributions of LaTeX
% version 2005/12/01 or later.
%
% The list of all files belonging to the LaTeX `Tools Bundle' is
% given in the file `manifest.txt'.
%
% \fi
%
%%
%
%
\def\FMithmInfo{2023/07/05 v2.2c Theorem extension package (FMi)}
%
% \ProvidesFile{theorem.dtx}[\FMithmInfo]
%\iffalse   % this is a METACOMMENT !
%%
%% Package `theorem' to use with LaTeX2e
%% Copyright (C) 1989-1995 Frank Mittelbach, all rights reserved.
%<+package>\@ifundefined{theorem@style}{}{\endinput}
%<+package>\NeedsTeXFormat{LaTeX2e}[1995/06/01]
%<+package>\ProvidesPackage{theorem}
%<+package>         [\FMithmInfo]
%
% \fi
%
% \changes{v1.0}{87/05/07}{First draft.}
% \changes{v1.0a}{87/05/07}{`thistheoremstyle' added.}
% \changes{v1.1a}{87/07/29}{New concept: redefine the internal
%                          \protect\LaTeX{} commands `@xnthm' etc.}
% \changes{v1.1b}{87/07/31}{`theoremfont' used to specify font.}
% \changes{v1.2a}{88/01/08}{`@endtheorem' now globally defined.}
% \changes{v1.2b}{88/01/11}{`newtheorem' only allowed in the preamble.}
% \changes{v1.3a}{88/02/14}{`theoremfont' now a toks register.}
%
% \changes{v2.0a}{88/11/16}{New implementation and new documentation.}
% \changes{v2.0a}{88/11/16}{No layout changes but internal modifications
%                           old substyles won't work.}
% \changes{v2.0b}{89/01/30}{`global' used for all definitions.}
% \changes{v2.0c}{89/06/15}{Documentation translated to English.
%                            Thanks to LLL and Nina}
% \changes{v2.1a}{90/02/15}{Support for new font selection.}
% \changes{v2.2a}{94/02/03}{Cleanup for LaTeX2e}
%
% \DoNotIndex{\@endparenv,\@ifdefinable,\@ifnextchar,\@ifundefined}
% \DoNotIndex{\@namedef,\@nameuse,\@spaces,\@tempa,\@warning}
% \DoNotIndex{\ }
% \DoNotIndex{\begingroup,\bfseries}
% \DoNotIndex{\csname}
% \DoNotIndex{\def,\do,\docdate}
% \DoNotIndex{\endcsname,\endgroup,\endinput,\endtrivlist,\expandafter}
% \DoNotIndex{\filedate,\fileversion}
% \DoNotIndex{\gdef,\global}
% \DoNotIndex{\hbox,\hskip}
% \DoNotIndex{\ignorespaces,\input,\itshape,\item}
% \DoNotIndex{\labelsep,\let,\llap}
% \DoNotIndex{\makeatletter}
% \DoNotIndex{\noexpand}
% \DoNotIndex{\refstepcounter,\relax,\rlap}
% \DoNotIndex{\setlength,\slshape,\space,\string,\strut}
% \DoNotIndex{\the,\trivlist,\typeout}
% \DoNotIndex{\vbox}
% \DoNotIndex{\xdef}
%
% \def\deflabel#1{\ttfamily #1\hfill}
% \def\deflist#1{\begin{list}{}{\settowidth\labelwidth{\ttfamily #1}%
%                               \setlength\leftmargin\labelwidth
%                               \addtolength\leftmargin\labelsep
%                               \let\makelabel\deflabel}}
% \def\enddeflist{\end{list}}
% \def\envfont{\normalfont\ttfamily}
%
% \MakeShortVerb{\|}
%
% \newcommand{\TUB}{{\sl TUGboat\/}}
%
% \GetFileInfo{theorem.dtx}
% \title{An Extension of the
%        \LaTeX{} theorem environment\thanks{This file
%        has version number \fileversion, last
%        revised \filedate.}}
% \author{Frank Mittelbach}
%
% \MaintainedByLaTeXTeam{tools}
% \maketitle
%
% \begin{abstract}
%    The macros described in this paper yield an extension of the
%    \LaTeX{} theorem mechanism. It is designed to satisfy the
%    different requirements of various journals. Thus, the
%    layout of the ``theorems'' can be manipulated by determining a
%    ``style''. This article describes not only the use, but
%    also the definition, of the necessary macros.
% \end{abstract}
%
% \section*{Preface to version 2.2}
%
% For \LaTeXe{} this package did not need any fundamental changes. I
% only modified the messages generated so that theorem layout styles
% will show up with the |\listfiles| command and cleaned the section
% on the New Font Selection Scheme since this is now included in
% \LaTeX.
%
% \section*{Preface to version 2.1}
%
% This version is identical to 2.0g described in \TUB\ 10\#3 except
% for some internal defaults which are now set depending on the used
% font selection scheme.
%
% This was done to avoid unpleasant surprises if the new font selection
% scheme is in force. For further details see
% section~\ref{sec:newfonts} and \cite{bk:GMS94}.
%
% \section{Introduction}
%
% For our purposes here, ``theorems'' are labelled enunciations,
% often set off from the main text by extra space and a font change.
% Theorems, corollaries, conjectures, definitions, and remarks
% are all instances of ``theorems''.  The ``header'' of these structures
% is composed of a label (such as \textsc{Theorem} or \textsc{Remark})
% and a number which serializes an item in the sequence of items
% with the same label.
%
% Shortly after the introduction of \LaTeX{} at the Fachbereich
% Mathematik in Mainz, the desire to manipulate the layout of
% ``theorems'' arose.  In Mainz, the following two conventions came
% into general use:
% \begin{enumerate}
%   \item The number of the theorem is shown in the margin.
%   \item There is a line break at
%         the end of the theorem header.
% \end{enumerate}
%
% Additionally, some journals require different formats which depend on
% the ``sort of theorem'': e.g.\ often remarks and definitions are set
% in |\upshape|, while |\itshape| is employed for main theorems.
%
% Confronted with these requirements, a theorem environment was
% developed in Mainz which allows separate determination of the
% layout of the ``theorems sets'', comparable to |\pagestyle|.
%
%
% \section{The user interface}
% \subsection{Defining new theorem sets}
%
% \DescribeMacro\newtheorem
% As in the original \LaTeX{} version, the command |\newtheorem|
% defines a new ``theorem set'' or ``theorem-like structure''.
% Two required arguments name the new environment and give the text
% to be typeset with each instance of the new ``set'', while
% an optional argument determines how
% the ``set'' is enumerated:
% \begin{description}
%    \item[\ttfamily \bslash newtheorem\{foo\}\{bar\}]
%       The theorem set {\envfont foo} (whose name is \texttt{bar})
%       uses its own counter.
%    \item[\ttfamily \bslash newtheorem\{foo2\}{[foo]}\{bar2\}]
%       The theorem set {\envfont foo2} (printed name \texttt{bar2})
%       uses the same counter as the theorem set \texttt{foo}.
%    \item[\ttfamily \bslash newtheorem\{foo3\}\{bar3\}{[section]}]
%       The theorem set {\envfont foo3} (printed name \texttt{bar3}) is
%       enumerated within the counter \texttt{section}, i.e.\ with every
%       new |\section| the enumeration begins again with 1, and
%       the enumeration is composed from the section-number and the
%       theorem counter itself.
% \end{description}
%
% \DescribeMacro\theoremstyle
% Additionally, the command |\theoremstyle| can define the layout
% of various, or all, theorem sets.  It should be noted that any
% theorem set defined by |\newtheorem| is typeset in the
% |\theoremstyle| that is current at the time of the definition.
% Thus, the following
% \begin{verbatim}
%    \theoremstyle{break}        \newtheorem{Cor}{Corollary}
%    \theoremstyle{plain}        \newtheorem{Exa}{Example}[section]
%\end{verbatim}                             ^^A No blank at line start
% leads to the result that the set {\envfont Cor} is formatted in the
% style \texttt{break}, while the set {\envfont Exa} and all the
% following ones are formatted in the style \texttt{plain}, unless
% another |\theoremstyle| follows.  Since the definitions installed by
% |\newtheorem| are global, one also can limit |\theoremstyle| locally
% by grouping braces.
%
% \DescribeMacro\theorembodyfont
% The choice of the font for the theorem body is completely
% independent of the chosen |\theoremstyle|; this has proven to
% be very advantageous.  For example,
% \begin{verbatim}
%   {\theorembodyfont{\upshape}          \newtheorem{Rem}{Remark}}
%\end{verbatim}
% defines a theorem set {\envfont Rem}, which will be set in
% |\upshape| in the current layout (which in our example is
% \texttt{plain}).  As with |\theoremstyle|, the |\theorembodyfont|
% chosen is that current at the time of |\newtheorem|.  If
% |\theorembodyfont| is not specified or one defines
% |\theorembodyfont{}|, then the font used will be that defined by the
% |\theoremstyle|.
%
% \DescribeMacro\theoremheaderfont It is also possible to customize
% the font used for the theorem headers.  This is, however, a global
% declaration, and therefore there should be at most one
% |\theoremheaderfont| declaration in the preamble.\footnote{If it is
% actually necessary to have different header fonts, one has to define
% new theorem styles (substituting the desired font) or specify the
% information directly in the \texttt{\bslash newtheorem} declaration
% (the unclean variant).}
%
% \DescribeMacro\theorempreskipamount
% \DescribeMacro\theorempostskipamount Two additional parameters
% affect the vertical space around the theorem environments:
% |\theorempreskipamount| and |\theorempostskipamount| define,
% respectively, the spacing before and after such an environment.
% These parameters apply for all theorem sets and can be manipulated
% with the ordinary length macros.  They are rubber lengths
% (`\textsf{skips}'), and therefore can contain \texttt{plus} and
% \texttt{minus} parts.
%
% Since the definition of theorem sets should---most sensibly---be
% placed in the preamble, we only allow installation there.
% It is therefore possible to release the memory used here after
% |\begin{document}|, in order to make room for other
% applications.
%
%
% \subsection{Existing theorem styles}
%
% The following theorem styles exist to date:
% \begin{deflist}{marginbreak}
%    \item[plain]
%       This theorem style emulates the original \LaTeX{} definition,
%       except that additionally the parameters
%       |\theorem...skipamount| are used.
%    \item[break]
%       In this style, the theorem header is followed by a line
%       break.
%    \item[marginbreak]
%       The theorem number is set in the margin, and there is a line
%       break as in \texttt{break}.
%    \item[changebreak]
%       Like \texttt{break}, but with header number and text
%       interchanged.
%    \item[change]
%       Header number and text are interchanged, without a line break.
%    \item[margin]
%       The number is set in the left margin, without a line break.
% \end{deflist}
% All styles (except \texttt{plain}) select |\slshape| as the default
% |\theorembodyfont|.
%
%
% \subsection{Examples}
%
% Given the above theorem sets {\envfont Cor}, {\envfont Exa}
% and {\envfont Rem},
% suppose that the preamble also contains the declarations:
% \begin{verbatim}
%    \theoremstyle{marginbreak}   \newtheorem{Lem}[Cor]{Lemma}
%    \theoremstyle{change}
%    \theorembodyfont{\itshape}        \newtheorem{Def}[Cor]{Definition}
%
%    \theoremheaderfont{\scshape}
%\end{verbatim}
% Then the following are some typical examples of the typeset output
% resulting from their use.
%
% \begin{Cor}
%    This is a sentence typeset in the theorem environment {\envfont
%    Cor}.
% \end{Cor}
%
% \begin{Exa}
%    This is a sentence typeset in the theorem environment {\envfont
%    Exa}.
% \end{Exa}
%
% \begin{Rem}
%    This is a sentence typeset in the theorem environment {\envfont
%    Rem}.
% \end{Rem}
%
% \begin{Lem}[Ben User]
%    This is a sentence typeset in the theorem environment {\envfont
%    Lem}.
% \end{Lem}
%
% \begin{Def}[Very impressive Definition]
%    This is a sentence typeset in the theorem environment {\envfont
%    Def}.
% \end{Def}
%
% The last two examples show the effect of the optional argument to
% a theorem environment (it is the text typeset in parentheses).
%
%
% \section{Special Considerations}
% \label{sec:newfonts}
%
% Theoremheader and body are implemented as a unit. This means that
% the |\theoremheaderfont|  will inherit characteristics of the
% |\theorembodyfont| in \LaTeXe. Thus, if for example
% |\theorembodyfont| is |\itshape| and |\theoremheaderfont|
% is |\bfseries| the font selected for the header will have the
% characteristics `bold extended italic'. If this is not desired one
% should set the |\theoremheaderfont| to something like
% \begin{verbatim}
%      \theoremheaderfont{\normalfont\bfseries}
%\end{verbatim}
% i.e.\ supplying all necessary font information explicitly.
%
%
%
% \section{Acknowledgements}
%
% The publication of this set of macros was only possible with the
% help of Christina Busse (translating the manuscript into English),
% Joachim Pense (playing the r\^ole of typist), Chris Rowley
% (looking everything over) and many others providing useful
% suggestions.
%
% \MaybeStop{%
%    \begin{thebibliography}{9}
%       \bibitem{bk:GMS94} \textsc{M.~Goossens}, \textsc{F.~Mittelbach}
%         and \textsc{A.~Samarin}.
%         \newblock The \LaTeX{} Companion.
%         \newblock
%         Addison-Wesley, Reading, Massachusetts, 1994.
%       \bibitem{latex:source} \textsc{Lamport, Leslie.} \newblock
%                               \texttt{latex.tex}, version 2.09, date
%                               Feb. 1990.
%    \end{thebibliography}
%    \PrintIndex
%    \PrintChanges}
%
%
% \section{The documentation driver file}
%
% The next bit of code contains the documentation driver file for
% \TeX{}, i.e., the file that will produce the documentation you are
% currently reading. It will be extracted from this file by the
% \texttt{docstrip} program. Since it is the first code in the file
% one can alternatively process this file directly with \LaTeXe{} to
% obtain the documentation.
%
%    \begin{macrocode}
%<*driver>
\documentclass{ltxdoc}

\usepackage{theorem}

%<+driver>% The next few lines define theorem sets which are used
%<+driver>% in the example section of the documentation.

\theoremstyle{break}       \newtheorem{Cor}{Corollary}
\theoremstyle{plain}       \newtheorem{Exa}{Example}[section]
{\theorembodyfont{\upshape}\newtheorem{Rem}{Remark}}
\theoremstyle{marginbreak} \newtheorem{Lem}[Cor]{Lemma}
\theoremstyle{change}
\theorembodyfont{\itshape} \newtheorem{Def}[Cor]{Definition}

\theoremheaderfont{\scshape}

\RecordChanges

\begin{document}
   \DocInput{theorem.dtx}
\end{document}
%</driver>
%    \end{macrocode}
%
% \section{Definition of the Macros}
%
% If the file has been loaded before, we abort immediately. If not the
% package announces itself (this is actually done at the very top of
% the file---the way it is done isn't good style so don't copy it).
% \changes{v2.0e}{89/07/19}{Spaces removed from `typeout'.}
% \changes{v2.2b}{95/11/19}{Announce moved to top of file}
%    \begin{macrocode}
%<*package>
%\@ifundefined{theorem@style}{}{\endinput}
%\def\FMithmInfo{1995/11/19 v2.2b Theorem extension package (FMi)}
%\ProvidesPackage{theorem}[\FMithmInfo]
%    \end{macrocode}
%
%
% \subsection{Definition of theorem styles and fonts}
%
% All the definitions in this file are done globally to allow
% inputting this file inside a group.
%
% \begin{macro}{\theoremstyle}
%    Before a theorem style can be installed, the chosen style must be
%    known. For that reason, we must test to see that
%    |\th@|\meta{style} is known or, more precisely, that it is
%    different from |\relax|. If the style is not known then
%    |\th@plain| is used.
%    \begin{macrocode}
\gdef\theoremstyle#1{%
   \@ifundefined{th@#1}{\@warning
          {Unknown theoremstyle `#1'. Using `plain'}%
          \theorem@style{plain}}%
%    \end{macrocode}
%    We save the theorem style to be used in the token register
%    |\theorem@style|.
%    \begin{macrocode}
      {\theorem@style{#1}}%
%    \end{macrocode}
%    Now we ``evaluate'' the theorem style: this means, we call the
%    macro |\th@|\meta{style} which will activate the relevant
%    definitions which are contained in a separate file.  This is done
%    in a group to suppress changes to the current font.  This could
%    otherwise pose problems together with the new font selection
%    scheme\footnote{When I printed the original article using the new
%    font selection scheme I ended with a document with slanted
%    typefaces (text headings and all) simply because one of the
%    theorem styles used \texttt{\bslash sl} at toplevel.} if the
%    |\th@|\meta{style} is evaluated a second time.
% \changes{v2.1a}{90/02/15}{Group added around `th@..}
% \changes{v2.1a}{90/02/15}{Reseting of `..@begintheorem removed.}
%    \begin{macrocode}
      \begingroup
        \csname th@\the\theorem@style \endcsname
      \endgroup}
%    \end{macrocode}
% \end{macro}
%
%
% \begin{macro}{\@begintheorem}
% \changes{v2.1a}{90/02/15}{Relaxed at toplevel.}
% \begin{macro}{\@opargbegintheorem}
% \changes{v2.1a}{90/02/15}{Relaxed at toplevel.}
%    We reset |\@begintheorem| and |\@opargbegintheorem| to
%    |\relax| since these commands are no longer necessary at
%    toplevel. This will save a few tokens.
%    \begin{macrocode}
\global\let\@begintheorem\relax
\global\let\@opargbegintheorem\relax
%    \end{macrocode}
% \end{macro}
% \end{macro}
%
% \begin{macro}{\theorem@style}
%    Obviously the \textsf{token} register used above has to be
%    allocated. To assure the utmost compatibility with the
%    original \LaTeX{} definition, we set the default theorem style to
%    \texttt{plain}, which implements the usual \LaTeX{} convention.
% \changes{v2.0f}{89/07/20}{Unnecessary `global' removed.}
%    \begin{macrocode}
\newtoks\theorem@style
\global\theorem@style{plain}
%    \end{macrocode}
% \end{macro}
%
% \begin{macro}{\theorembodyfont}
% \begin{macro}{\theorem@bodyfont}
%    If the |\theorembodyfont| is set by the user then it should not
%    interact with the default font set in the theorem style. When the
%    new font selection is in force this may happen if, for example,
%    the default is |\itshape| and the new |\theorembodyfont| is
%    |\sffamily|. So we add a |\reset@font| command in front of the
%    user definition.
% \changes{v2.0f}{89/07/20}{Unnecessary `global' removed.}
% \changes{v2.0g}{89/09/19}{`theoremfont to `theorembodyfont.}
% \changes{v2.1c}{92/06/22}{`reset@font added and logic changed.}
% \changes{v2.1d}{92/10/28}{Reset to empty if arg is empty}
%    \begin{macrocode}
\gdef\theorembodyfont#1{%
%    \end{macrocode}
%    We check if the argument supplied is empty and if so put nothing
%    into the |\theorem@bodyfont| token register to allow for
%    |\theorembodyfont{}| as a mean of using the default of the
%    current |\theoremstyle|.
%    \begin{macrocode}
   \def\@tempa{#1}%
   \ifx\@tempa\@empty
    \theorem@bodyfont{}%
   \else
    \theorem@bodyfont{\reset@font#1}%
   \fi
   }
\newtoks\theorem@bodyfont
\global\theorem@bodyfont{}
%    \end{macrocode}
% \end{macro}
% \end{macro}
%
%
% \begin{macro}{\theoremheaderfont}
% \changes{v2.0g}{89/09/19}{Macro added.}
%    The font for the theorem headers is handled differently because
%    this definition applies to all theorem styles.
%    \begin{macrocode}
\gdef\theoremheaderfont#1{\gdef\theorem@headerfont{#1}%
%    \end{macrocode}
%    After using the macro once it is redefined to produce an error
%    message.
%    \begin{macrocode}
       \gdef\theoremheaderfont##1{%
        \typeout{\string\theoremheaderfont\space should be used
                 only once.}}}
%    \end{macrocode}
% \end{macro}
%
% \begin{macro}{\theorem@headerfont}
% \changes{v2.0g}{89/09/19}{Macro added.}
%    To set the |\theorem@headerfont| default we first test if
%    the new fontselection scheme is in force.
% \changes{v2.1a}{90/02/15}{Support for new font selection.}
%    \begin{macrocode}
\ifx\upshape\undefined
%    \end{macrocode}
%    If not we define it to expand into |\bfseries|. We don't use
%    |\let| just in case a following style option redefines this
%    macro.
%    \begin{macrocode}
\gdef\theorem@headerfont{\bfseries}
%    \end{macrocode}
%    Otherwise we reset the current shape before calling |\bfseries|.
%    \begin{macrocode}
\else \gdef\theorem@headerfont{\normalfont\bfseries}\fi
%    \end{macrocode}
% \end{macro}
%
% \begin{macro}{\th@plain}
% \begin{macro}{\th@break}
% \begin{macro}{\th@marginbreak}
% \begin{macro}{\th@changebreak}
% \begin{macro}{\th@change}
% \begin{macro}{\th@margin}
%    The different styles are defined in macros such as
%    |\th@plain|.  Since memory space is precious in
%    ``non-Big-versions'', we have to avoid offering too many unused
%    definitions.  Therefore we define these styles in separate files
%    that can be loaded on demand.  Thus the commands themselves only
%    load these files. We use |\@input@|, a \LaTeXe{} internal command
%    that ensures that the file will be listed with |\listfiles|.
% \changes{v2.2a}{94/02/02}{Use `@input@ to load theorem layout files}
%    \begin{macrocode}
\gdef\th@plain{\@input@{thp.sty}}
\gdef\th@break{\@input@{thb.sty}}
\gdef\th@marginbreak{\@input@{thmb.sty}}
\gdef\th@changebreak{\@input@{thcb.sty}}
\gdef\th@change{\@input@{thc.sty}}
\gdef\th@margin{\@input@{thm.sty}}
%    \end{macrocode}
%    This list will be expanded when new styles become
%    available.  For testing,  just append new theorem substyles
%    as document options.
% \end{macro}
% \end{macro}
% \end{macro}
% \end{macro}
% \end{macro}
% \end{macro}
%
%
%
% \subsection{Definition of a new theorem set}
%
% As already pointed out, a new theorem environment can be defined in
% three different ways:
% \begin{verbatim}
%    \newtheorem{Lem}{Lemma}
%    \newtheorem{Lem}{Lemma}[section]
%    \newtheorem{Lem}[Theorem]{Lemma}
%\end{verbatim}
% The function of the macro |\newtheorem| is to recognize these cases
% and then to branch into one of the three macros |\@ynthm|, |\@xnthm|
% or |\@othm|.  This mechanism is adopted unchanged from
% \cite{latex:source}; the essential point here is that, for example,
% in the second case, the arguments \texttt{Lem}, \texttt{Lemma} and
% \texttt{section} are passed over to the macro |\@xnthm|.
%
% We inspect this case first because the others present fewer problems,
% and thus are easily derived from this one.
%
% \begin{macro}{\@xnthm}
%    For our example arguments, the macro |\@xnthm| must fulfill
%    the following:
%    \begin{itemize}
%       \item
%          Define a new \LaTeX-counter `Lem'
%       \item
%          reset this counter within a |\section|
%       \item
%          define the macro |\theLem|
%       \item
%          define the environment macros |\Lem| and
%          |\endLem| using the current
%          |\theoremstyle| and |\theorem@bodyfont|.
%    \end{itemize}
%    Obviously, all this should happen only if the first argument
%    of |\@xnthm| (i.e.\ \texttt{Lem} in our example) is chosen so
%    as not to conflict with any previously defined commands or
%    environments.  This test is performed by the \LaTeX{} macro
%    |\@ifdefinable|.
%    \begin{macrocode}
\gdef\@xnthm#1#2[#3]{\expandafter\@ifdefinable\csname #1\endcsname
%    \end{macrocode}
%    Therefore, the first argument of |\@ifdefinable| is the
%    expansion (in the example, |\Lem|) of
%    |\csname||#1\endcsname|.  The second argument is
%    executed only if the test has been completed successfully.
%    \begin{macrocode}
   {%
%    \end{macrocode}
%    Now we define the new counter. The names of the \LaTeX{}
%    macros employed should speak for themselves:
% \changes{v2.2b}{1995/10/07}{Use \cs{@newctr}}
% \changes{v2.2c}{1995/11/23}{Use \cs{@newctr} properly.}
%    \begin{macrocode}
    \@definecounter{#1}\@newctr{#1}[#3]%
%    \end{macrocode}
%    Using |\@newctr| will give a proper error message if the counter
%    in |#3| is not defined.
%    In defining `|\theLem|' we must generate the desired macro
%    name by use of |\expandafter| and |\csname|.
%    \begin{macrocode}
    \expandafter\xdef\csname the#1\endcsname
%    \end{macrocode}
%    An |\xdef| is used in order to make the definition global,
%    and to ensure that it contains the replacement texts of
%    |\@thmcountersep| and |\@thmcounter|.\footnote{These
%    two macros can be defined by the document style.  Their default
%    values produce a `\texttt{.}' as separation and an arabic
%    representation of the number.} However, not everything should be
%    expanded. For example, it saves space to use |\thesection|
%    instead of its---at times---lengthy expansion.
%    \begin{macrocode}
      {\expandafter \noexpand \csname the#3\endcsname
       \@thmcountersep \@thmcounter{#1}}%
%    \end{macrocode}
%    Thus with the defaults of \LaTeX{}, |\theLem| would be
%    replaced by the command sequence |\thesection.\arabic{Lem}|.
%
%    We will now look at the definition of the macro which is executed
%    at the beginning of the actual environment (in our example this
%    macro is |\Lem|). It should be noted that we use an
%    ``|\expandafter| trick'' to expand only certain parts of the
%    replacement text at the time of the definition.
%    \begin{macrocode}
    \def\@tempa{\global\@namedef{#1}}%
    \expandafter \@tempa \expandafter{%
%    \end{macrocode}
%    First, the macro that contains the current definitions of
%    |\@begintheorem| and |\@opargtheorem| should be called
%    up.  The name of this macro---as is already known---has the form
%    |\th@|\meta{theorem\ style}; therefore, it must be called by
%    \begin{macrocode}
      \csname th@\the \theorem@style
%    \end{macrocode}
%    In addition the default theorem font should be changeable, i.e.\
%    we have to insert the contents of |\theorem@bodyfont|. For
%    that reason, we expand even further, beyond |\endcsname|,
%    and thus insert the contents of the \textsf{token} register
%    |\theorem@bodyfont| in the replacement text.
%    \begin{macrocode}
            \expandafter \endcsname \the \theorem@bodyfont
%    \end{macrocode}
%    Now it is time to call the macro |\@thm| which takes over
%    the further processing.  It has two arguments: the current
%    counter name (in our example, \texttt{Lem}), and the text of the
%    label (in our example, \texttt{Lemma}).
%    \begin{macrocode}
     \@thm{#1}{#2}}%
%    \end{macrocode}
%    With this, the `sub-definition' is complete.  The macro
%    |\@endtheorem| ends a theorem environment and is, so far, nothing
%    but an |\endtrivlist|.  (Hence it is defined globally, and not
%    within the theorem styles.\footnote{This has to be changed as
%    soon as theorem styles that change \texttt{\bslash @endtheorem}
%    exist.  In such a case, all existing styles must be changed as
%    well since they will have to reset the macro.}) Therefore, we can
%    set it equivalent to the macro that ends the theorem set (in our
%    example, |\endLem|).  However, if some day theorem styles exist
%    that do change |\@endtheorem|, we would have to use the
%    commented-out line instead.
%    \begin{macrocode}
    \global \expandafter \let \csname end#1\endcsname \@endtheorem
%   \global\@namedef{end#1}{\@endtheorem}%
%    \end{macrocode}
%    With these commands all the required definitions are employed,
%    unless the test |\@ifdefinable| has failed.  Therefore, we
%    end the second argument of this macro and with it the definition
%    of |\@xnthm|.
%    \begin{macrocode}
   }}
%    \end{macrocode}
% \end{macro}
%
% \begin{macro}{\@ynthm}
%    The definition of |\@ynthm| is completely analogous. In
%    this case the new counter that is defined is not reset within
%    another counter; thus the definition of |\the...|
%    is simplified:
%    \begin{macrocode}
\gdef\@ynthm#1#2{\expandafter\@ifdefinable\csname #1\endcsname
   {\@definecounter{#1}%
    \expandafter\xdef\csname the#1\endcsname{\@thmcounter{#1}}%
%    \end{macrocode}
%    The rest of the definition corresponds literally to that of
%    |\@xnthm|:
%    \begin{macrocode}
    \def\@tempa{\global\@namedef{#1}}\expandafter \@tempa
     \expandafter{\csname th@\the \theorem@style \expandafter
     \endcsname \the\theorem@bodyfont \@thm{#1}{#2}}%
    \global \expandafter \let \csname end#1\endcsname \@endtheorem}}
%    \end{macrocode}
% \end{macro}
%
% \begin{macro}{\@othm}
%    The definition of |\@othm| does not contain anything new.
%    \begin{macrocode}
\gdef\@othm#1[#2]#3{%
%    \end{macrocode}
% \changes{v2.2b}{95/10/07}{Build in check for counter defined}
%    We do not define a new counter but instead use one that has
%    already been defined.  Thus the only definition we need is that
%    of this pseudo-counter (i.e.\ |\the|\meta{env.\ name}).
%    First we check if |#2| corresponds to a known counter name.
%    \begin{macrocode}
  \expandafter\ifx\csname c@#2\endcsname\relax
   \@nocounterr{#2}%
  \else
   \expandafter\@ifdefinable\csname #1\endcsname
   {\expandafter \xdef \csname the#1\endcsname
     {\expandafter \noexpand \csname the#2\endcsname}%
%    \end{macrocode}
%    All other parts of the definition can be adopted from
%    |\@xnthm|. We have to remember, though, that in this case
%    the name of the current counter and the theorem label have moved
%    to the second and third arguments.
%    \begin{macrocode}
    \def\@tempa{\global\@namedef{#1}}\expandafter \@tempa
     \expandafter{\csname th@\the \theorem@style \expandafter
     \endcsname \the\theorem@bodyfont \@thm{#2}{#3}}%
    \global \expandafter \let \csname end#1\endcsname \@endtheorem}%
  \fi}
%    \end{macrocode}
% \end{macro}
%
% \subsection{Macros that are employed in a {\envfont theorem}
%    environment}
%
% \begin{macro}{\@thm}
%    The macro |\@thm| has to increase the current counter.  Then,
%    depending on whether the environment has (or does not have) an
%    optional argument, it has to branch into either |\@begintheorem|
%    or |\@opargtheorem|.
%    \begin{macrocode}
\gdef\@thm#1#2{\refstepcounter{#1}%
%    \end{macrocode}
%    Now we start a {\envfont trivlist} environment, and give
%    |\@topsep| and |\@topsepadd| the values of the \textsf{skip}
%    registers |\theorempreskipamount| and |\theorempostskipamount|.
%    The value in |\@topsep| is the vertical space that is inserted by
%    the first (and only) |\item| in our |\trivlist| whilst
%    |\@topsepadd| is inserted by |\@endparenv| at the end of that
%    {\envfont trivlist} environment.  By using these registers, we
%    obtain the desired space around a {\envfont theorem} environment.
%    \begin{macrocode}
   \trivlist
   \@topsep \theorempreskipamount               % used by first \item
   \@topsepadd \theorempostskipamount           % used by \@endparenv
%    \end{macrocode}
%    Now we have to test whether an optional argument has been given.
% \changes{v2.0d}{89/07/18}{Forgotten percent added.}
%    \begin{macrocode}
   \@ifnextchar [%
%    \end{macrocode}
%    If there is an optional argument, we will call |\@ythm|, and
%    move the arguments read back into the input stream.
%    \begin{macrocode}
   {\@ythm{#1}{#2}}%
%    \end{macrocode}
%    If not, we call |\@begintheorem|. Its first argument is the
%    name of the theorem set (hence the second argument of
%    |\@thm|). Its second argument is the macro that produces the
%    current number.
%    \begin{macrocode}
   {\@begintheorem{#2}{\csname the#1\endcsname}\ignorespaces}}
%    \end{macrocode}
% \end{macro}
%
% \begin{macro}{\@xthm}
% \begin{macro}{\@ythm}
%    Both these macros were originally called by |\@thm|.  We do
%    not need |\@xthm| anymore, hence we reset it to
%    |\relax|. The definition of |\@ythm| has not changed at
%    all from its definition in \LaTeX. In order to make the macros
%    easier to understand, we will
%    nevertheless present it (commented out).
%    \begin{macrocode}
\global\let\@xthm\relax
% \def\@ythm#1#2[#3]{\@opargbegintheorem{#2}{\csname
%           the#1\endcsname}{#3}\ignorespaces}
%</package>
%    \end{macrocode}
%    The primitive |\ignorespaces| in |\@ythm| and
%    |\@thm| is needed to remove the spaces between the
%    |\begin{...}| and the actual text.
% \end{macro}
% \end{macro}
%
%
% \subsection{Definition of the theorem substyles}
%
% As already pointed out, the theorem substyles, defined below, are
% only loaded when necessary.  Note that all these substyles, except
% \texttt{plain}, have |\slshape| as the default body font.
%
% \subsubsection{The \texttt{plain} style}
%
% \label{dok:th@plain} As the following macros use |@|, we have
% to locally set the |\catcode| of this symbol to ``letter''.
% This happens within a group, so that we do not have to worry about
% which |\catcode| that symbol had before.
%    \begin{macrocode}
%<*thp>
\begingroup \makeatletter
%    \end{macrocode}
% Since we are now within a group, we must make all definitions
% globally.  First we make sure that \texttt{theorem.sty} is loaded.
% This will allow us to use this file as a document style option
% without having to call \texttt{theorem} itself as an option.  At the
% same time, we assure that at least version 2 is loaded, since
% |\theorem@style| was not defined in earlier versions.
%    \begin{macrocode}
\@ifundefined{theorem@style}{% \iffalse meta-comment
%
% Copyright (C) 1993-2024
%
% The LaTeX Project and any individual authors listed elsewhere
% in this file.
%
% This file is part of the Standard LaTeX `Tools Bundle'.
% -------------------------------------------------------
%
% It may be distributed and/or modified under the
% conditions of the LaTeX Project Public License, either version 1.3c
% of this license or (at your option) any later version.
% The latest version of this license is in
%    https://www.latex-project.org/lppl.txt
% and version 1.3c or later is part of all distributions of LaTeX
% version 2005/12/01 or later.
%
% The list of all files belonging to the LaTeX `Tools Bundle' is
% given in the file `manifest.txt'.
%
% \fi
%
%%
%
%
\def\FMithmInfo{2023/07/05 v2.2c Theorem extension package (FMi)}
%
% \ProvidesFile{theorem.dtx}[\FMithmInfo]
%\iffalse   % this is a METACOMMENT !
%%
%% Package `theorem' to use with LaTeX2e
%% Copyright (C) 1989-1995 Frank Mittelbach, all rights reserved.
%<+package>\@ifundefined{theorem@style}{}{\endinput}
%<+package>\NeedsTeXFormat{LaTeX2e}[1995/06/01]
%<+package>\ProvidesPackage{theorem}
%<+package>         [\FMithmInfo]
%
% \fi
%
% \changes{v1.0}{87/05/07}{First draft.}
% \changes{v1.0a}{87/05/07}{`thistheoremstyle' added.}
% \changes{v1.1a}{87/07/29}{New concept: redefine the internal
%                          \protect\LaTeX{} commands `@xnthm' etc.}
% \changes{v1.1b}{87/07/31}{`theoremfont' used to specify font.}
% \changes{v1.2a}{88/01/08}{`@endtheorem' now globally defined.}
% \changes{v1.2b}{88/01/11}{`newtheorem' only allowed in the preamble.}
% \changes{v1.3a}{88/02/14}{`theoremfont' now a toks register.}
%
% \changes{v2.0a}{88/11/16}{New implementation and new documentation.}
% \changes{v2.0a}{88/11/16}{No layout changes but internal modifications
%                           old substyles won't work.}
% \changes{v2.0b}{89/01/30}{`global' used for all definitions.}
% \changes{v2.0c}{89/06/15}{Documentation translated to English.
%                            Thanks to LLL and Nina}
% \changes{v2.1a}{90/02/15}{Support for new font selection.}
% \changes{v2.2a}{94/02/03}{Cleanup for LaTeX2e}
%
% \DoNotIndex{\@endparenv,\@ifdefinable,\@ifnextchar,\@ifundefined}
% \DoNotIndex{\@namedef,\@nameuse,\@spaces,\@tempa,\@warning}
% \DoNotIndex{\ }
% \DoNotIndex{\begingroup,\bfseries}
% \DoNotIndex{\csname}
% \DoNotIndex{\def,\do,\docdate}
% \DoNotIndex{\endcsname,\endgroup,\endinput,\endtrivlist,\expandafter}
% \DoNotIndex{\filedate,\fileversion}
% \DoNotIndex{\gdef,\global}
% \DoNotIndex{\hbox,\hskip}
% \DoNotIndex{\ignorespaces,\input,\itshape,\item}
% \DoNotIndex{\labelsep,\let,\llap}
% \DoNotIndex{\makeatletter}
% \DoNotIndex{\noexpand}
% \DoNotIndex{\refstepcounter,\relax,\rlap}
% \DoNotIndex{\setlength,\slshape,\space,\string,\strut}
% \DoNotIndex{\the,\trivlist,\typeout}
% \DoNotIndex{\vbox}
% \DoNotIndex{\xdef}
%
% \def\deflabel#1{\ttfamily #1\hfill}
% \def\deflist#1{\begin{list}{}{\settowidth\labelwidth{\ttfamily #1}%
%                               \setlength\leftmargin\labelwidth
%                               \addtolength\leftmargin\labelsep
%                               \let\makelabel\deflabel}}
% \def\enddeflist{\end{list}}
% \def\envfont{\normalfont\ttfamily}
%
% \MakeShortVerb{\|}
%
% \newcommand{\TUB}{{\sl TUGboat\/}}
%
% \GetFileInfo{theorem.dtx}
% \title{An Extension of the
%        \LaTeX{} theorem environment\thanks{This file
%        has version number \fileversion, last
%        revised \filedate.}}
% \author{Frank Mittelbach}
%
% \MaintainedByLaTeXTeam{tools}
% \maketitle
%
% \begin{abstract}
%    The macros described in this paper yield an extension of the
%    \LaTeX{} theorem mechanism. It is designed to satisfy the
%    different requirements of various journals. Thus, the
%    layout of the ``theorems'' can be manipulated by determining a
%    ``style''. This article describes not only the use, but
%    also the definition, of the necessary macros.
% \end{abstract}
%
% \section*{Preface to version 2.2}
%
% For \LaTeXe{} this package did not need any fundamental changes. I
% only modified the messages generated so that theorem layout styles
% will show up with the |\listfiles| command and cleaned the section
% on the New Font Selection Scheme since this is now included in
% \LaTeX.
%
% \section*{Preface to version 2.1}
%
% This version is identical to 2.0g described in \TUB\ 10\#3 except
% for some internal defaults which are now set depending on the used
% font selection scheme.
%
% This was done to avoid unpleasant surprises if the new font selection
% scheme is in force. For further details see
% section~\ref{sec:newfonts} and \cite{bk:GMS94}.
%
% \section{Introduction}
%
% For our purposes here, ``theorems'' are labelled enunciations,
% often set off from the main text by extra space and a font change.
% Theorems, corollaries, conjectures, definitions, and remarks
% are all instances of ``theorems''.  The ``header'' of these structures
% is composed of a label (such as \textsc{Theorem} or \textsc{Remark})
% and a number which serializes an item in the sequence of items
% with the same label.
%
% Shortly after the introduction of \LaTeX{} at the Fachbereich
% Mathematik in Mainz, the desire to manipulate the layout of
% ``theorems'' arose.  In Mainz, the following two conventions came
% into general use:
% \begin{enumerate}
%   \item The number of the theorem is shown in the margin.
%   \item There is a line break at
%         the end of the theorem header.
% \end{enumerate}
%
% Additionally, some journals require different formats which depend on
% the ``sort of theorem'': e.g.\ often remarks and definitions are set
% in |\upshape|, while |\itshape| is employed for main theorems.
%
% Confronted with these requirements, a theorem environment was
% developed in Mainz which allows separate determination of the
% layout of the ``theorems sets'', comparable to |\pagestyle|.
%
%
% \section{The user interface}
% \subsection{Defining new theorem sets}
%
% \DescribeMacro\newtheorem
% As in the original \LaTeX{} version, the command |\newtheorem|
% defines a new ``theorem set'' or ``theorem-like structure''.
% Two required arguments name the new environment and give the text
% to be typeset with each instance of the new ``set'', while
% an optional argument determines how
% the ``set'' is enumerated:
% \begin{description}
%    \item[\ttfamily \bslash newtheorem\{foo\}\{bar\}]
%       The theorem set {\envfont foo} (whose name is \texttt{bar})
%       uses its own counter.
%    \item[\ttfamily \bslash newtheorem\{foo2\}{[foo]}\{bar2\}]
%       The theorem set {\envfont foo2} (printed name \texttt{bar2})
%       uses the same counter as the theorem set \texttt{foo}.
%    \item[\ttfamily \bslash newtheorem\{foo3\}\{bar3\}{[section]}]
%       The theorem set {\envfont foo3} (printed name \texttt{bar3}) is
%       enumerated within the counter \texttt{section}, i.e.\ with every
%       new |\section| the enumeration begins again with 1, and
%       the enumeration is composed from the section-number and the
%       theorem counter itself.
% \end{description}
%
% \DescribeMacro\theoremstyle
% Additionally, the command |\theoremstyle| can define the layout
% of various, or all, theorem sets.  It should be noted that any
% theorem set defined by |\newtheorem| is typeset in the
% |\theoremstyle| that is current at the time of the definition.
% Thus, the following
% \begin{verbatim}
%    \theoremstyle{break}        \newtheorem{Cor}{Corollary}
%    \theoremstyle{plain}        \newtheorem{Exa}{Example}[section]
%\end{verbatim}                             ^^A No blank at line start
% leads to the result that the set {\envfont Cor} is formatted in the
% style \texttt{break}, while the set {\envfont Exa} and all the
% following ones are formatted in the style \texttt{plain}, unless
% another |\theoremstyle| follows.  Since the definitions installed by
% |\newtheorem| are global, one also can limit |\theoremstyle| locally
% by grouping braces.
%
% \DescribeMacro\theorembodyfont
% The choice of the font for the theorem body is completely
% independent of the chosen |\theoremstyle|; this has proven to
% be very advantageous.  For example,
% \begin{verbatim}
%   {\theorembodyfont{\upshape}          \newtheorem{Rem}{Remark}}
%\end{verbatim}
% defines a theorem set {\envfont Rem}, which will be set in
% |\upshape| in the current layout (which in our example is
% \texttt{plain}).  As with |\theoremstyle|, the |\theorembodyfont|
% chosen is that current at the time of |\newtheorem|.  If
% |\theorembodyfont| is not specified or one defines
% |\theorembodyfont{}|, then the font used will be that defined by the
% |\theoremstyle|.
%
% \DescribeMacro\theoremheaderfont It is also possible to customize
% the font used for the theorem headers.  This is, however, a global
% declaration, and therefore there should be at most one
% |\theoremheaderfont| declaration in the preamble.\footnote{If it is
% actually necessary to have different header fonts, one has to define
% new theorem styles (substituting the desired font) or specify the
% information directly in the \texttt{\bslash newtheorem} declaration
% (the unclean variant).}
%
% \DescribeMacro\theorempreskipamount
% \DescribeMacro\theorempostskipamount Two additional parameters
% affect the vertical space around the theorem environments:
% |\theorempreskipamount| and |\theorempostskipamount| define,
% respectively, the spacing before and after such an environment.
% These parameters apply for all theorem sets and can be manipulated
% with the ordinary length macros.  They are rubber lengths
% (`\textsf{skips}'), and therefore can contain \texttt{plus} and
% \texttt{minus} parts.
%
% Since the definition of theorem sets should---most sensibly---be
% placed in the preamble, we only allow installation there.
% It is therefore possible to release the memory used here after
% |\begin{document}|, in order to make room for other
% applications.
%
%
% \subsection{Existing theorem styles}
%
% The following theorem styles exist to date:
% \begin{deflist}{marginbreak}
%    \item[plain]
%       This theorem style emulates the original \LaTeX{} definition,
%       except that additionally the parameters
%       |\theorem...skipamount| are used.
%    \item[break]
%       In this style, the theorem header is followed by a line
%       break.
%    \item[marginbreak]
%       The theorem number is set in the margin, and there is a line
%       break as in \texttt{break}.
%    \item[changebreak]
%       Like \texttt{break}, but with header number and text
%       interchanged.
%    \item[change]
%       Header number and text are interchanged, without a line break.
%    \item[margin]
%       The number is set in the left margin, without a line break.
% \end{deflist}
% All styles (except \texttt{plain}) select |\slshape| as the default
% |\theorembodyfont|.
%
%
% \subsection{Examples}
%
% Given the above theorem sets {\envfont Cor}, {\envfont Exa}
% and {\envfont Rem},
% suppose that the preamble also contains the declarations:
% \begin{verbatim}
%    \theoremstyle{marginbreak}   \newtheorem{Lem}[Cor]{Lemma}
%    \theoremstyle{change}
%    \theorembodyfont{\itshape}        \newtheorem{Def}[Cor]{Definition}
%
%    \theoremheaderfont{\scshape}
%\end{verbatim}
% Then the following are some typical examples of the typeset output
% resulting from their use.
%
% \begin{Cor}
%    This is a sentence typeset in the theorem environment {\envfont
%    Cor}.
% \end{Cor}
%
% \begin{Exa}
%    This is a sentence typeset in the theorem environment {\envfont
%    Exa}.
% \end{Exa}
%
% \begin{Rem}
%    This is a sentence typeset in the theorem environment {\envfont
%    Rem}.
% \end{Rem}
%
% \begin{Lem}[Ben User]
%    This is a sentence typeset in the theorem environment {\envfont
%    Lem}.
% \end{Lem}
%
% \begin{Def}[Very impressive Definition]
%    This is a sentence typeset in the theorem environment {\envfont
%    Def}.
% \end{Def}
%
% The last two examples show the effect of the optional argument to
% a theorem environment (it is the text typeset in parentheses).
%
%
% \section{Special Considerations}
% \label{sec:newfonts}
%
% Theoremheader and body are implemented as a unit. This means that
% the |\theoremheaderfont|  will inherit characteristics of the
% |\theorembodyfont| in \LaTeXe. Thus, if for example
% |\theorembodyfont| is |\itshape| and |\theoremheaderfont|
% is |\bfseries| the font selected for the header will have the
% characteristics `bold extended italic'. If this is not desired one
% should set the |\theoremheaderfont| to something like
% \begin{verbatim}
%      \theoremheaderfont{\normalfont\bfseries}
%\end{verbatim}
% i.e.\ supplying all necessary font information explicitly.
%
%
%
% \section{Acknowledgements}
%
% The publication of this set of macros was only possible with the
% help of Christina Busse (translating the manuscript into English),
% Joachim Pense (playing the r\^ole of typist), Chris Rowley
% (looking everything over) and many others providing useful
% suggestions.
%
% \MaybeStop{%
%    \begin{thebibliography}{9}
%       \bibitem{bk:GMS94} \textsc{M.~Goossens}, \textsc{F.~Mittelbach}
%         and \textsc{A.~Samarin}.
%         \newblock The \LaTeX{} Companion.
%         \newblock
%         Addison-Wesley, Reading, Massachusetts, 1994.
%       \bibitem{latex:source} \textsc{Lamport, Leslie.} \newblock
%                               \texttt{latex.tex}, version 2.09, date
%                               Feb. 1990.
%    \end{thebibliography}
%    \PrintIndex
%    \PrintChanges}
%
%
% \section{The documentation driver file}
%
% The next bit of code contains the documentation driver file for
% \TeX{}, i.e., the file that will produce the documentation you are
% currently reading. It will be extracted from this file by the
% \texttt{docstrip} program. Since it is the first code in the file
% one can alternatively process this file directly with \LaTeXe{} to
% obtain the documentation.
%
%    \begin{macrocode}
%<*driver>
\documentclass{ltxdoc}

\usepackage{theorem}

%<+driver>% The next few lines define theorem sets which are used
%<+driver>% in the example section of the documentation.

\theoremstyle{break}       \newtheorem{Cor}{Corollary}
\theoremstyle{plain}       \newtheorem{Exa}{Example}[section]
{\theorembodyfont{\upshape}\newtheorem{Rem}{Remark}}
\theoremstyle{marginbreak} \newtheorem{Lem}[Cor]{Lemma}
\theoremstyle{change}
\theorembodyfont{\itshape} \newtheorem{Def}[Cor]{Definition}

\theoremheaderfont{\scshape}

\RecordChanges

\begin{document}
   \DocInput{theorem.dtx}
\end{document}
%</driver>
%    \end{macrocode}
%
% \section{Definition of the Macros}
%
% If the file has been loaded before, we abort immediately. If not the
% package announces itself (this is actually done at the very top of
% the file---the way it is done isn't good style so don't copy it).
% \changes{v2.0e}{89/07/19}{Spaces removed from `typeout'.}
% \changes{v2.2b}{95/11/19}{Announce moved to top of file}
%    \begin{macrocode}
%<*package>
%\@ifundefined{theorem@style}{}{\endinput}
%\def\FMithmInfo{1995/11/19 v2.2b Theorem extension package (FMi)}
%\ProvidesPackage{theorem}[\FMithmInfo]
%    \end{macrocode}
%
%
% \subsection{Definition of theorem styles and fonts}
%
% All the definitions in this file are done globally to allow
% inputting this file inside a group.
%
% \begin{macro}{\theoremstyle}
%    Before a theorem style can be installed, the chosen style must be
%    known. For that reason, we must test to see that
%    |\th@|\meta{style} is known or, more precisely, that it is
%    different from |\relax|. If the style is not known then
%    |\th@plain| is used.
%    \begin{macrocode}
\gdef\theoremstyle#1{%
   \@ifundefined{th@#1}{\@warning
          {Unknown theoremstyle `#1'. Using `plain'}%
          \theorem@style{plain}}%
%    \end{macrocode}
%    We save the theorem style to be used in the token register
%    |\theorem@style|.
%    \begin{macrocode}
      {\theorem@style{#1}}%
%    \end{macrocode}
%    Now we ``evaluate'' the theorem style: this means, we call the
%    macro |\th@|\meta{style} which will activate the relevant
%    definitions which are contained in a separate file.  This is done
%    in a group to suppress changes to the current font.  This could
%    otherwise pose problems together with the new font selection
%    scheme\footnote{When I printed the original article using the new
%    font selection scheme I ended with a document with slanted
%    typefaces (text headings and all) simply because one of the
%    theorem styles used \texttt{\bslash sl} at toplevel.} if the
%    |\th@|\meta{style} is evaluated a second time.
% \changes{v2.1a}{90/02/15}{Group added around `th@..}
% \changes{v2.1a}{90/02/15}{Reseting of `..@begintheorem removed.}
%    \begin{macrocode}
      \begingroup
        \csname th@\the\theorem@style \endcsname
      \endgroup}
%    \end{macrocode}
% \end{macro}
%
%
% \begin{macro}{\@begintheorem}
% \changes{v2.1a}{90/02/15}{Relaxed at toplevel.}
% \begin{macro}{\@opargbegintheorem}
% \changes{v2.1a}{90/02/15}{Relaxed at toplevel.}
%    We reset |\@begintheorem| and |\@opargbegintheorem| to
%    |\relax| since these commands are no longer necessary at
%    toplevel. This will save a few tokens.
%    \begin{macrocode}
\global\let\@begintheorem\relax
\global\let\@opargbegintheorem\relax
%    \end{macrocode}
% \end{macro}
% \end{macro}
%
% \begin{macro}{\theorem@style}
%    Obviously the \textsf{token} register used above has to be
%    allocated. To assure the utmost compatibility with the
%    original \LaTeX{} definition, we set the default theorem style to
%    \texttt{plain}, which implements the usual \LaTeX{} convention.
% \changes{v2.0f}{89/07/20}{Unnecessary `global' removed.}
%    \begin{macrocode}
\newtoks\theorem@style
\global\theorem@style{plain}
%    \end{macrocode}
% \end{macro}
%
% \begin{macro}{\theorembodyfont}
% \begin{macro}{\theorem@bodyfont}
%    If the |\theorembodyfont| is set by the user then it should not
%    interact with the default font set in the theorem style. When the
%    new font selection is in force this may happen if, for example,
%    the default is |\itshape| and the new |\theorembodyfont| is
%    |\sffamily|. So we add a |\reset@font| command in front of the
%    user definition.
% \changes{v2.0f}{89/07/20}{Unnecessary `global' removed.}
% \changes{v2.0g}{89/09/19}{`theoremfont to `theorembodyfont.}
% \changes{v2.1c}{92/06/22}{`reset@font added and logic changed.}
% \changes{v2.1d}{92/10/28}{Reset to empty if arg is empty}
%    \begin{macrocode}
\gdef\theorembodyfont#1{%
%    \end{macrocode}
%    We check if the argument supplied is empty and if so put nothing
%    into the |\theorem@bodyfont| token register to allow for
%    |\theorembodyfont{}| as a mean of using the default of the
%    current |\theoremstyle|.
%    \begin{macrocode}
   \def\@tempa{#1}%
   \ifx\@tempa\@empty
    \theorem@bodyfont{}%
   \else
    \theorem@bodyfont{\reset@font#1}%
   \fi
   }
\newtoks\theorem@bodyfont
\global\theorem@bodyfont{}
%    \end{macrocode}
% \end{macro}
% \end{macro}
%
%
% \begin{macro}{\theoremheaderfont}
% \changes{v2.0g}{89/09/19}{Macro added.}
%    The font for the theorem headers is handled differently because
%    this definition applies to all theorem styles.
%    \begin{macrocode}
\gdef\theoremheaderfont#1{\gdef\theorem@headerfont{#1}%
%    \end{macrocode}
%    After using the macro once it is redefined to produce an error
%    message.
%    \begin{macrocode}
       \gdef\theoremheaderfont##1{%
        \typeout{\string\theoremheaderfont\space should be used
                 only once.}}}
%    \end{macrocode}
% \end{macro}
%
% \begin{macro}{\theorem@headerfont}
% \changes{v2.0g}{89/09/19}{Macro added.}
%    To set the |\theorem@headerfont| default we first test if
%    the new fontselection scheme is in force.
% \changes{v2.1a}{90/02/15}{Support for new font selection.}
%    \begin{macrocode}
\ifx\upshape\undefined
%    \end{macrocode}
%    If not we define it to expand into |\bfseries|. We don't use
%    |\let| just in case a following style option redefines this
%    macro.
%    \begin{macrocode}
\gdef\theorem@headerfont{\bfseries}
%    \end{macrocode}
%    Otherwise we reset the current shape before calling |\bfseries|.
%    \begin{macrocode}
\else \gdef\theorem@headerfont{\normalfont\bfseries}\fi
%    \end{macrocode}
% \end{macro}
%
% \begin{macro}{\th@plain}
% \begin{macro}{\th@break}
% \begin{macro}{\th@marginbreak}
% \begin{macro}{\th@changebreak}
% \begin{macro}{\th@change}
% \begin{macro}{\th@margin}
%    The different styles are defined in macros such as
%    |\th@plain|.  Since memory space is precious in
%    ``non-Big-versions'', we have to avoid offering too many unused
%    definitions.  Therefore we define these styles in separate files
%    that can be loaded on demand.  Thus the commands themselves only
%    load these files. We use |\@input@|, a \LaTeXe{} internal command
%    that ensures that the file will be listed with |\listfiles|.
% \changes{v2.2a}{94/02/02}{Use `@input@ to load theorem layout files}
%    \begin{macrocode}
\gdef\th@plain{\@input@{thp.sty}}
\gdef\th@break{\@input@{thb.sty}}
\gdef\th@marginbreak{\@input@{thmb.sty}}
\gdef\th@changebreak{\@input@{thcb.sty}}
\gdef\th@change{\@input@{thc.sty}}
\gdef\th@margin{\@input@{thm.sty}}
%    \end{macrocode}
%    This list will be expanded when new styles become
%    available.  For testing,  just append new theorem substyles
%    as document options.
% \end{macro}
% \end{macro}
% \end{macro}
% \end{macro}
% \end{macro}
% \end{macro}
%
%
%
% \subsection{Definition of a new theorem set}
%
% As already pointed out, a new theorem environment can be defined in
% three different ways:
% \begin{verbatim}
%    \newtheorem{Lem}{Lemma}
%    \newtheorem{Lem}{Lemma}[section]
%    \newtheorem{Lem}[Theorem]{Lemma}
%\end{verbatim}
% The function of the macro |\newtheorem| is to recognize these cases
% and then to branch into one of the three macros |\@ynthm|, |\@xnthm|
% or |\@othm|.  This mechanism is adopted unchanged from
% \cite{latex:source}; the essential point here is that, for example,
% in the second case, the arguments \texttt{Lem}, \texttt{Lemma} and
% \texttt{section} are passed over to the macro |\@xnthm|.
%
% We inspect this case first because the others present fewer problems,
% and thus are easily derived from this one.
%
% \begin{macro}{\@xnthm}
%    For our example arguments, the macro |\@xnthm| must fulfill
%    the following:
%    \begin{itemize}
%       \item
%          Define a new \LaTeX-counter `Lem'
%       \item
%          reset this counter within a |\section|
%       \item
%          define the macro |\theLem|
%       \item
%          define the environment macros |\Lem| and
%          |\endLem| using the current
%          |\theoremstyle| and |\theorem@bodyfont|.
%    \end{itemize}
%    Obviously, all this should happen only if the first argument
%    of |\@xnthm| (i.e.\ \texttt{Lem} in our example) is chosen so
%    as not to conflict with any previously defined commands or
%    environments.  This test is performed by the \LaTeX{} macro
%    |\@ifdefinable|.
%    \begin{macrocode}
\gdef\@xnthm#1#2[#3]{\expandafter\@ifdefinable\csname #1\endcsname
%    \end{macrocode}
%    Therefore, the first argument of |\@ifdefinable| is the
%    expansion (in the example, |\Lem|) of
%    |\csname||#1\endcsname|.  The second argument is
%    executed only if the test has been completed successfully.
%    \begin{macrocode}
   {%
%    \end{macrocode}
%    Now we define the new counter. The names of the \LaTeX{}
%    macros employed should speak for themselves:
% \changes{v2.2b}{1995/10/07}{Use \cs{@newctr}}
% \changes{v2.2c}{1995/11/23}{Use \cs{@newctr} properly.}
%    \begin{macrocode}
    \@definecounter{#1}\@newctr{#1}[#3]%
%    \end{macrocode}
%    Using |\@newctr| will give a proper error message if the counter
%    in |#3| is not defined.
%    In defining `|\theLem|' we must generate the desired macro
%    name by use of |\expandafter| and |\csname|.
%    \begin{macrocode}
    \expandafter\xdef\csname the#1\endcsname
%    \end{macrocode}
%    An |\xdef| is used in order to make the definition global,
%    and to ensure that it contains the replacement texts of
%    |\@thmcountersep| and |\@thmcounter|.\footnote{These
%    two macros can be defined by the document style.  Their default
%    values produce a `\texttt{.}' as separation and an arabic
%    representation of the number.} However, not everything should be
%    expanded. For example, it saves space to use |\thesection|
%    instead of its---at times---lengthy expansion.
%    \begin{macrocode}
      {\expandafter \noexpand \csname the#3\endcsname
       \@thmcountersep \@thmcounter{#1}}%
%    \end{macrocode}
%    Thus with the defaults of \LaTeX{}, |\theLem| would be
%    replaced by the command sequence |\thesection.\arabic{Lem}|.
%
%    We will now look at the definition of the macro which is executed
%    at the beginning of the actual environment (in our example this
%    macro is |\Lem|). It should be noted that we use an
%    ``|\expandafter| trick'' to expand only certain parts of the
%    replacement text at the time of the definition.
%    \begin{macrocode}
    \def\@tempa{\global\@namedef{#1}}%
    \expandafter \@tempa \expandafter{%
%    \end{macrocode}
%    First, the macro that contains the current definitions of
%    |\@begintheorem| and |\@opargtheorem| should be called
%    up.  The name of this macro---as is already known---has the form
%    |\th@|\meta{theorem\ style}; therefore, it must be called by
%    \begin{macrocode}
      \csname th@\the \theorem@style
%    \end{macrocode}
%    In addition the default theorem font should be changeable, i.e.\
%    we have to insert the contents of |\theorem@bodyfont|. For
%    that reason, we expand even further, beyond |\endcsname|,
%    and thus insert the contents of the \textsf{token} register
%    |\theorem@bodyfont| in the replacement text.
%    \begin{macrocode}
            \expandafter \endcsname \the \theorem@bodyfont
%    \end{macrocode}
%    Now it is time to call the macro |\@thm| which takes over
%    the further processing.  It has two arguments: the current
%    counter name (in our example, \texttt{Lem}), and the text of the
%    label (in our example, \texttt{Lemma}).
%    \begin{macrocode}
     \@thm{#1}{#2}}%
%    \end{macrocode}
%    With this, the `sub-definition' is complete.  The macro
%    |\@endtheorem| ends a theorem environment and is, so far, nothing
%    but an |\endtrivlist|.  (Hence it is defined globally, and not
%    within the theorem styles.\footnote{This has to be changed as
%    soon as theorem styles that change \texttt{\bslash @endtheorem}
%    exist.  In such a case, all existing styles must be changed as
%    well since they will have to reset the macro.}) Therefore, we can
%    set it equivalent to the macro that ends the theorem set (in our
%    example, |\endLem|).  However, if some day theorem styles exist
%    that do change |\@endtheorem|, we would have to use the
%    commented-out line instead.
%    \begin{macrocode}
    \global \expandafter \let \csname end#1\endcsname \@endtheorem
%   \global\@namedef{end#1}{\@endtheorem}%
%    \end{macrocode}
%    With these commands all the required definitions are employed,
%    unless the test |\@ifdefinable| has failed.  Therefore, we
%    end the second argument of this macro and with it the definition
%    of |\@xnthm|.
%    \begin{macrocode}
   }}
%    \end{macrocode}
% \end{macro}
%
% \begin{macro}{\@ynthm}
%    The definition of |\@ynthm| is completely analogous. In
%    this case the new counter that is defined is not reset within
%    another counter; thus the definition of |\the...|
%    is simplified:
%    \begin{macrocode}
\gdef\@ynthm#1#2{\expandafter\@ifdefinable\csname #1\endcsname
   {\@definecounter{#1}%
    \expandafter\xdef\csname the#1\endcsname{\@thmcounter{#1}}%
%    \end{macrocode}
%    The rest of the definition corresponds literally to that of
%    |\@xnthm|:
%    \begin{macrocode}
    \def\@tempa{\global\@namedef{#1}}\expandafter \@tempa
     \expandafter{\csname th@\the \theorem@style \expandafter
     \endcsname \the\theorem@bodyfont \@thm{#1}{#2}}%
    \global \expandafter \let \csname end#1\endcsname \@endtheorem}}
%    \end{macrocode}
% \end{macro}
%
% \begin{macro}{\@othm}
%    The definition of |\@othm| does not contain anything new.
%    \begin{macrocode}
\gdef\@othm#1[#2]#3{%
%    \end{macrocode}
% \changes{v2.2b}{95/10/07}{Build in check for counter defined}
%    We do not define a new counter but instead use one that has
%    already been defined.  Thus the only definition we need is that
%    of this pseudo-counter (i.e.\ |\the|\meta{env.\ name}).
%    First we check if |#2| corresponds to a known counter name.
%    \begin{macrocode}
  \expandafter\ifx\csname c@#2\endcsname\relax
   \@nocounterr{#2}%
  \else
   \expandafter\@ifdefinable\csname #1\endcsname
   {\expandafter \xdef \csname the#1\endcsname
     {\expandafter \noexpand \csname the#2\endcsname}%
%    \end{macrocode}
%    All other parts of the definition can be adopted from
%    |\@xnthm|. We have to remember, though, that in this case
%    the name of the current counter and the theorem label have moved
%    to the second and third arguments.
%    \begin{macrocode}
    \def\@tempa{\global\@namedef{#1}}\expandafter \@tempa
     \expandafter{\csname th@\the \theorem@style \expandafter
     \endcsname \the\theorem@bodyfont \@thm{#2}{#3}}%
    \global \expandafter \let \csname end#1\endcsname \@endtheorem}%
  \fi}
%    \end{macrocode}
% \end{macro}
%
% \subsection{Macros that are employed in a {\envfont theorem}
%    environment}
%
% \begin{macro}{\@thm}
%    The macro |\@thm| has to increase the current counter.  Then,
%    depending on whether the environment has (or does not have) an
%    optional argument, it has to branch into either |\@begintheorem|
%    or |\@opargtheorem|.
%    \begin{macrocode}
\gdef\@thm#1#2{\refstepcounter{#1}%
%    \end{macrocode}
%    Now we start a {\envfont trivlist} environment, and give
%    |\@topsep| and |\@topsepadd| the values of the \textsf{skip}
%    registers |\theorempreskipamount| and |\theorempostskipamount|.
%    The value in |\@topsep| is the vertical space that is inserted by
%    the first (and only) |\item| in our |\trivlist| whilst
%    |\@topsepadd| is inserted by |\@endparenv| at the end of that
%    {\envfont trivlist} environment.  By using these registers, we
%    obtain the desired space around a {\envfont theorem} environment.
%    \begin{macrocode}
   \trivlist
   \@topsep \theorempreskipamount               % used by first \item
   \@topsepadd \theorempostskipamount           % used by \@endparenv
%    \end{macrocode}
%    Now we have to test whether an optional argument has been given.
% \changes{v2.0d}{89/07/18}{Forgotten percent added.}
%    \begin{macrocode}
   \@ifnextchar [%
%    \end{macrocode}
%    If there is an optional argument, we will call |\@ythm|, and
%    move the arguments read back into the input stream.
%    \begin{macrocode}
   {\@ythm{#1}{#2}}%
%    \end{macrocode}
%    If not, we call |\@begintheorem|. Its first argument is the
%    name of the theorem set (hence the second argument of
%    |\@thm|). Its second argument is the macro that produces the
%    current number.
%    \begin{macrocode}
   {\@begintheorem{#2}{\csname the#1\endcsname}\ignorespaces}}
%    \end{macrocode}
% \end{macro}
%
% \begin{macro}{\@xthm}
% \begin{macro}{\@ythm}
%    Both these macros were originally called by |\@thm|.  We do
%    not need |\@xthm| anymore, hence we reset it to
%    |\relax|. The definition of |\@ythm| has not changed at
%    all from its definition in \LaTeX. In order to make the macros
%    easier to understand, we will
%    nevertheless present it (commented out).
%    \begin{macrocode}
\global\let\@xthm\relax
% \def\@ythm#1#2[#3]{\@opargbegintheorem{#2}{\csname
%           the#1\endcsname}{#3}\ignorespaces}
%</package>
%    \end{macrocode}
%    The primitive |\ignorespaces| in |\@ythm| and
%    |\@thm| is needed to remove the spaces between the
%    |\begin{...}| and the actual text.
% \end{macro}
% \end{macro}
%
%
% \subsection{Definition of the theorem substyles}
%
% As already pointed out, the theorem substyles, defined below, are
% only loaded when necessary.  Note that all these substyles, except
% \texttt{plain}, have |\slshape| as the default body font.
%
% \subsubsection{The \texttt{plain} style}
%
% \label{dok:th@plain} As the following macros use |@|, we have
% to locally set the |\catcode| of this symbol to ``letter''.
% This happens within a group, so that we do not have to worry about
% which |\catcode| that symbol had before.
%    \begin{macrocode}
%<*thp>
\begingroup \makeatletter
%    \end{macrocode}
% Since we are now within a group, we must make all definitions
% globally.  First we make sure that \texttt{theorem.sty} is loaded.
% This will allow us to use this file as a document style option
% without having to call \texttt{theorem} itself as an option.  At the
% same time, we assure that at least version 2 is loaded, since
% |\theorem@style| was not defined in earlier versions.
%    \begin{macrocode}
\@ifundefined{theorem@style}{\input{theorem.sty}}{}
\ProvidesFile{thp.sty}
             [\FMithmInfo]
%    \end{macrocode}
%
%
% \begin{macro}{\th@plain}
% \begin{macro}{\@begintheorem}
% \begin{macro}{\@opargbegintheorem}
%    |\theoremstyle{plain}| corresponds to the original
%    definition, except that the distances to the surrounding text are
%    determined by the parameters |\theorempreskipamount| and
%    |\theorempostskipamount|.
%    First we set the default body font.
%    \begin{macrocode}
\gdef\th@plain{\normalfont\itshape
%    \end{macrocode}
%    Then we define |\@begintheorem| and
%    |\@opargbegintheorem|. These two macros define how the
%    header of a theorem is typeset.  |\@opargbegintheorem| will
%    be called if a \textsf{theorem} environment with an optional
%    argument is encountered; otherwise, the header is constructed by
%    calling |\@begintheorem|.  If one of these macros is
%    executed, we are within a \textsf{trivlist} environment started by
%    |\@thm|.  So the theorem header is produced with an
%    |\item| command.
%
%    Instead of specifying the header font directly, all standard
%    theorem styles use the |\theorem@headerfont| macro to allow
%    customization.  The extra space (|\labelsep|) is necessary
%    because of problems in the \textsf{trivlist} environment.
%    \begin{macrocode}
  \def\@begintheorem##1##2{%
        \item[\hskip\labelsep \theorem@headerfont ##1\ ##2]}%
%    \end{macrocode}
%    The definition of |\@opargbegintheorem| is completely
%    analogous. The only difference is the fact that there exists a
%    third argument (which is the optional parameter of the
%    environment and contains additional information about the
%    theorem).  Customarily we enclose it in parentheses.
%    \begin{macrocode}
\def\@opargbegintheorem##1##2##3{%
   \item[\hskip\labelsep \theorem@headerfont ##1\ ##2\ (##3)]}}
%    \end{macrocode}
% \end{macro}
% \end{macro}
% \end{macro}
%
% \noindent
% We conclude with an |\endgroup| to restore the
% |\catcode| of |@|.
%    \begin{macrocode}
\endgroup
%</thp>
%    \end{macrocode}
%
%
% \subsubsection{The \texttt{break} style}
%
% This style option is stored in the file |thb.sty|.
% For the next two lines see the documentation for |\th@plain| on
% page~\pageref{dok:th@plain}.
%    \begin{macrocode}
%<*thb>
\begingroup \makeatletter
\@ifundefined{theorem@style}{\input{theorem.sty}}{}
\ProvidesFile{thb.sty}
             [\FMithmInfo]
%    \end{macrocode}
%
% \begin{macro}{\th@break}
%    |\theoremstyle{break}| produces a line break after the name
%    of the theorem.  The font is |\slshape|.  Hence, we define
%    |\th@break| as follows:
%    \begin{macrocode}
\gdef\th@break{\normalfont\slshape
  \def\@begintheorem##1##2{\item[%
%    \end{macrocode}
%    We run into the following problem: it is not possible to create
%    the header with |\item[|\meta{title}|]| and then start
%    a new line by, for example, |\mbox{}\\|. Such a definition
%    will fail whenever a list environment follows immediately.  With
%    the above construction, the |\mbox{}| causes the switch
%    |@inlabel| (cf.\ definition of |\list| and
%    |\trivlist| in \cite{latex:source}) to be set to \texttt{false}
%    and so the following list will insert additional vertical
%    space (|\topskip|). This is quite annoying. Therefore, we
%    create the line break within the |\item|. In order to ensure
%    that the text will begin at the proper position in the following
%    line, we simply pretend that the label does not take any
%    room.\footnote{This will lead to problems whenever very high
%    symbols occurring in the line tower into the
%    heading. So, something else has to be done here sometime.}
%    \begin{macrocode}
       \rlap{\vbox{\hbox{\hskip \labelsep\theorem@headerfont ##1\ ##2}%
                   \hbox{\strut}}}]}%
%    \end{macrocode}
%    Again, the definition of |\@opargbegintheorem| is
%    completely analogous.
%    \begin{macrocode}
\def\@opargbegintheorem##1##2##3{%
  \item[\rlap{\vbox{\hbox{\hskip \labelsep \theorem@headerfont
                     ##1\ ##2\ (##3)}%
                    \hbox{\strut}}}]}}
\endgroup
%</thb>
%    \end{macrocode}
% \end{macro}
%
%
% \subsubsection{The \texttt{changebreak} style}
%
%    \begin{macrocode}
% This style option is stored in the file |thcb.sty|.
%    \begin{macrocode}
%<*thcb>
\begingroup \makeatletter
\@ifundefined{theorem@style}{\input{theorem.sty}}{}
\ProvidesFile{thcb.sty}
             [\FMithmInfo]
%    \end{macrocode}
%
% \begin{macro}{\th@changebreak}
%    The \texttt{change-break} theorem style is like \texttt{break}
%    but with interchange of theorem name and theorem number.  Thus we
%    define |\th@changebreak| as follows:
%    \begin{macrocode}
\gdef\th@changebreak{\normalfont\slshape
  \def\@begintheorem##1##2{\item
       [\rlap{\vbox{\hbox{\hskip\labelsep \theorem@headerfont ##2\ ##1}%
                    \hbox{\strut}}}]}%
\def\@opargbegintheorem##1##2##3{%
  \item[\rlap{\vbox{\hbox{\hskip\labelsep \theorem@headerfont
                     ##2\ ##1\ (##3)}%
                    \hbox{\strut}}}]}}
\endgroup
%</thcb>
%    \end{macrocode}
% \end{macro}
%
%
% \subsubsection{The \texttt{change} style}
%
% This style option is stored in the file |thc.sty|.
%    \begin{macrocode}
%<*thc>
\begingroup \makeatletter
\@ifundefined{theorem@style}{\input{theorem.sty}}{}
\ProvidesFile{thc.sty}
             [\FMithmInfo]
%    \end{macrocode}
%
% \begin{macro}{\th@change}
%    The \texttt{change} theorem style corresponds to the
%    \texttt{change break} style without a linebreak after the header.
%    To say it in another way, it's the same as the \texttt{plain}
%    style but with number and name interchanged and |\slshape| as the
%    default font.
%    \begin{macrocode}
\gdef\th@change{\normalfont\slshape
  \def\@begintheorem##1##2{\item
       [\hskip\labelsep \theorem@headerfont ##2\ ##1]}%
\def\@opargbegintheorem##1##2##3{%
  \item[\hskip\labelsep \theorem@headerfont ##2\ ##1\ (##3)]}}
\endgroup
%</thc>
%    \end{macrocode}
% \end{macro}
%
%
% \subsubsection{The \texttt{marginbreak} style}
%
% This style option is the one used most often at Mainz.  It is saved
% in the file |thmb.sty|.
%    \begin{macrocode}
%<*thmb>
\begingroup \makeatletter
\@ifundefined{theorem@style}{\input{theorem.sty}}{}
\ProvidesFile{thmb.sty}
             [\FMithmInfo]
%    \end{macrocode}
%
% \begin{macro}{\th@marginbreak}
%    The \texttt{margin break} style is nearly the same as the
%    \texttt{change break} style. The only difference is the placement
%    of the theorem number.  We use |\llap| to place it in the left
%    margin.
%
%    In this style |\labelsep| denotes the separation between the
%    number and the text.
%    \begin{macrocode}
\gdef\th@marginbreak{\normalfont\slshape
  \def\@begintheorem##1##2{\item
       [\rlap{\vbox{\theorem@headerfont
                    \hbox{\llap{##2}\hskip\labelsep ##1}%
                    \hbox{\strut}}}]}%
\def\@opargbegintheorem##1##2##3{%
  \item[\rlap{\vbox{\theorem@headerfont
                    \hbox{\llap{##2}\hskip\labelsep ##1\ (##3)}%
                    \hbox{\strut}}}]}}
\endgroup
%</thmb>
%    \end{macrocode}
% \changes{v2.1c}{90/02/17}{`endgroup outside macrocode env.}
% \end{macro}
%
%
%
% \subsubsection{The \texttt{margin} style}
%
% This style option is stored in the file |thm.sty|.
%    \begin{macrocode}
%<*thm>
\begingroup \makeatletter
\@ifundefined{theorem@style}{\input{theorem.sty}}{}
\ProvidesFile{thm.sty}
             [\FMithmInfo]
%    \end{macrocode}
%
% \begin{macro}{\th@margin}
%    Again this is only a variant of the theorem styles described
%    above without any new ideas.
%    \begin{macrocode}
\gdef\th@margin{\normalfont\slshape
  \def\@begintheorem##1##2{\item
       [\theorem@headerfont \llap{##2}\hskip\labelsep ##1]}%
\def\@opargbegintheorem##1##2##3{%
  \item[\theorem@headerfont \llap{##2}\hskip\labelsep ##1\ (##3)]}}
\endgroup
%</thm>
%    \end{macrocode}
% \end{macro}
%
%
%
%
% \subsection{Final Definitions}
%
% \begin{macro}{\theorempreskipamount}
% \begin{macro}{\theorempostskipamount}
%    The \textsf{skip} parameters that regulate the vertical empty space
%    before and after the {\envfont theorem} environment have
%    to be allocated as well.
% \changes{v2.0f}{89/07/20}{Unnecessary `global' removed.}
%    \begin{macrocode}
%<*package>
\newskip\theorempreskipamount
\newskip\theorempostskipamount
%    \end{macrocode}
%    Since we have used the same values for all theorem sets, we now can
%    assign them.
%    \begin{macrocode}
\global\setlength\theorempreskipamount{12pt plus 5pt minus 3pt}
\global\setlength\theorempostskipamount{8pt plus 3pt minus 1.5pt}
%    \end{macrocode}
% \end{macro}
% \end{macro}
%
%
% \begin{macro}{\@endtheorem}
%    The same holds for the macro |\@endtheorem|, which ends a
%    {\envfont theorem} environment. Since it is the same for all
%    theorem sets, it is removed from the macros |\th@|\meta{style}.
%    It simply ends the {\envfont trivlist} environment, which was
%    begun in |\@thm|.
%    \begin{macrocode}
\global\let\@endtheorem=\endtrivlist
%    \end{macrocode}
% \end{macro}
%
%
%
% \begin{macro}{\@preamblecmds}
%    All macros defined above are to be used only in the preamble.
%    Therefore, we insert them in |\@preamblecmds| which will disable
%    them at begin document. This is done by the internal \LaTeXe{}
%    command |\@onlypreamble|.
% \changes{v2.0g}{89/09/19}{`theoremheaderfont and `theorembodyfont
%    added.}
% \changes{v2.2a}{94/02/03}{Use \LaTeXe{} interface}
%    \begin{macrocode}
\@onlypreamble\@xnthm
\@onlypreamble\@ynthm
\@onlypreamble\@othm
\@onlypreamble\newtheorem
\@onlypreamble\theoremstyle
\@onlypreamble\theorembodyfont
\@onlypreamble\theoremheaderfont
%    \end{macrocode}
% \end{macro}
%
% Finally we declare the \texttt{plain} theorem style to be the default.
% \changes{v2.1b}{90/06/04}{Fixed missing `item by declaring default
%    style.}
%    \begin{macrocode}
\theoremstyle{plain}
%</package>
%    \end{macrocode}
%
% \Finale

\endinput
}{}
\ProvidesFile{thp.sty}
             [\FMithmInfo]
%    \end{macrocode}
%
%
% \begin{macro}{\th@plain}
% \begin{macro}{\@begintheorem}
% \begin{macro}{\@opargbegintheorem}
%    |\theoremstyle{plain}| corresponds to the original
%    definition, except that the distances to the surrounding text are
%    determined by the parameters |\theorempreskipamount| and
%    |\theorempostskipamount|.
%    First we set the default body font.
%    \begin{macrocode}
\gdef\th@plain{\normalfont\itshape
%    \end{macrocode}
%    Then we define |\@begintheorem| and
%    |\@opargbegintheorem|. These two macros define how the
%    header of a theorem is typeset.  |\@opargbegintheorem| will
%    be called if a \textsf{theorem} environment with an optional
%    argument is encountered; otherwise, the header is constructed by
%    calling |\@begintheorem|.  If one of these macros is
%    executed, we are within a \textsf{trivlist} environment started by
%    |\@thm|.  So the theorem header is produced with an
%    |\item| command.
%
%    Instead of specifying the header font directly, all standard
%    theorem styles use the |\theorem@headerfont| macro to allow
%    customization.  The extra space (|\labelsep|) is necessary
%    because of problems in the \textsf{trivlist} environment.
%    \begin{macrocode}
  \def\@begintheorem##1##2{%
        \item[\hskip\labelsep \theorem@headerfont ##1\ ##2]}%
%    \end{macrocode}
%    The definition of |\@opargbegintheorem| is completely
%    analogous. The only difference is the fact that there exists a
%    third argument (which is the optional parameter of the
%    environment and contains additional information about the
%    theorem).  Customarily we enclose it in parentheses.
%    \begin{macrocode}
\def\@opargbegintheorem##1##2##3{%
   \item[\hskip\labelsep \theorem@headerfont ##1\ ##2\ (##3)]}}
%    \end{macrocode}
% \end{macro}
% \end{macro}
% \end{macro}
%
% \noindent
% We conclude with an |\endgroup| to restore the
% |\catcode| of |@|.
%    \begin{macrocode}
\endgroup
%</thp>
%    \end{macrocode}
%
%
% \subsubsection{The \texttt{break} style}
%
% This style option is stored in the file |thb.sty|.
% For the next two lines see the documentation for |\th@plain| on
% page~\pageref{dok:th@plain}.
%    \begin{macrocode}
%<*thb>
\begingroup \makeatletter
\@ifundefined{theorem@style}{% \iffalse meta-comment
%
% Copyright (C) 1993-2024
%
% The LaTeX Project and any individual authors listed elsewhere
% in this file.
%
% This file is part of the Standard LaTeX `Tools Bundle'.
% -------------------------------------------------------
%
% It may be distributed and/or modified under the
% conditions of the LaTeX Project Public License, either version 1.3c
% of this license or (at your option) any later version.
% The latest version of this license is in
%    https://www.latex-project.org/lppl.txt
% and version 1.3c or later is part of all distributions of LaTeX
% version 2005/12/01 or later.
%
% The list of all files belonging to the LaTeX `Tools Bundle' is
% given in the file `manifest.txt'.
%
% \fi
%
%%
%
%
\def\FMithmInfo{2023/07/05 v2.2c Theorem extension package (FMi)}
%
% \ProvidesFile{theorem.dtx}[\FMithmInfo]
%\iffalse   % this is a METACOMMENT !
%%
%% Package `theorem' to use with LaTeX2e
%% Copyright (C) 1989-1995 Frank Mittelbach, all rights reserved.
%<+package>\@ifundefined{theorem@style}{}{\endinput}
%<+package>\NeedsTeXFormat{LaTeX2e}[1995/06/01]
%<+package>\ProvidesPackage{theorem}
%<+package>         [\FMithmInfo]
%
% \fi
%
% \changes{v1.0}{87/05/07}{First draft.}
% \changes{v1.0a}{87/05/07}{`thistheoremstyle' added.}
% \changes{v1.1a}{87/07/29}{New concept: redefine the internal
%                          \protect\LaTeX{} commands `@xnthm' etc.}
% \changes{v1.1b}{87/07/31}{`theoremfont' used to specify font.}
% \changes{v1.2a}{88/01/08}{`@endtheorem' now globally defined.}
% \changes{v1.2b}{88/01/11}{`newtheorem' only allowed in the preamble.}
% \changes{v1.3a}{88/02/14}{`theoremfont' now a toks register.}
%
% \changes{v2.0a}{88/11/16}{New implementation and new documentation.}
% \changes{v2.0a}{88/11/16}{No layout changes but internal modifications
%                           old substyles won't work.}
% \changes{v2.0b}{89/01/30}{`global' used for all definitions.}
% \changes{v2.0c}{89/06/15}{Documentation translated to English.
%                            Thanks to LLL and Nina}
% \changes{v2.1a}{90/02/15}{Support for new font selection.}
% \changes{v2.2a}{94/02/03}{Cleanup for LaTeX2e}
%
% \DoNotIndex{\@endparenv,\@ifdefinable,\@ifnextchar,\@ifundefined}
% \DoNotIndex{\@namedef,\@nameuse,\@spaces,\@tempa,\@warning}
% \DoNotIndex{\ }
% \DoNotIndex{\begingroup,\bfseries}
% \DoNotIndex{\csname}
% \DoNotIndex{\def,\do,\docdate}
% \DoNotIndex{\endcsname,\endgroup,\endinput,\endtrivlist,\expandafter}
% \DoNotIndex{\filedate,\fileversion}
% \DoNotIndex{\gdef,\global}
% \DoNotIndex{\hbox,\hskip}
% \DoNotIndex{\ignorespaces,\input,\itshape,\item}
% \DoNotIndex{\labelsep,\let,\llap}
% \DoNotIndex{\makeatletter}
% \DoNotIndex{\noexpand}
% \DoNotIndex{\refstepcounter,\relax,\rlap}
% \DoNotIndex{\setlength,\slshape,\space,\string,\strut}
% \DoNotIndex{\the,\trivlist,\typeout}
% \DoNotIndex{\vbox}
% \DoNotIndex{\xdef}
%
% \def\deflabel#1{\ttfamily #1\hfill}
% \def\deflist#1{\begin{list}{}{\settowidth\labelwidth{\ttfamily #1}%
%                               \setlength\leftmargin\labelwidth
%                               \addtolength\leftmargin\labelsep
%                               \let\makelabel\deflabel}}
% \def\enddeflist{\end{list}}
% \def\envfont{\normalfont\ttfamily}
%
% \MakeShortVerb{\|}
%
% \newcommand{\TUB}{{\sl TUGboat\/}}
%
% \GetFileInfo{theorem.dtx}
% \title{An Extension of the
%        \LaTeX{} theorem environment\thanks{This file
%        has version number \fileversion, last
%        revised \filedate.}}
% \author{Frank Mittelbach}
%
% \MaintainedByLaTeXTeam{tools}
% \maketitle
%
% \begin{abstract}
%    The macros described in this paper yield an extension of the
%    \LaTeX{} theorem mechanism. It is designed to satisfy the
%    different requirements of various journals. Thus, the
%    layout of the ``theorems'' can be manipulated by determining a
%    ``style''. This article describes not only the use, but
%    also the definition, of the necessary macros.
% \end{abstract}
%
% \section*{Preface to version 2.2}
%
% For \LaTeXe{} this package did not need any fundamental changes. I
% only modified the messages generated so that theorem layout styles
% will show up with the |\listfiles| command and cleaned the section
% on the New Font Selection Scheme since this is now included in
% \LaTeX.
%
% \section*{Preface to version 2.1}
%
% This version is identical to 2.0g described in \TUB\ 10\#3 except
% for some internal defaults which are now set depending on the used
% font selection scheme.
%
% This was done to avoid unpleasant surprises if the new font selection
% scheme is in force. For further details see
% section~\ref{sec:newfonts} and \cite{bk:GMS94}.
%
% \section{Introduction}
%
% For our purposes here, ``theorems'' are labelled enunciations,
% often set off from the main text by extra space and a font change.
% Theorems, corollaries, conjectures, definitions, and remarks
% are all instances of ``theorems''.  The ``header'' of these structures
% is composed of a label (such as \textsc{Theorem} or \textsc{Remark})
% and a number which serializes an item in the sequence of items
% with the same label.
%
% Shortly after the introduction of \LaTeX{} at the Fachbereich
% Mathematik in Mainz, the desire to manipulate the layout of
% ``theorems'' arose.  In Mainz, the following two conventions came
% into general use:
% \begin{enumerate}
%   \item The number of the theorem is shown in the margin.
%   \item There is a line break at
%         the end of the theorem header.
% \end{enumerate}
%
% Additionally, some journals require different formats which depend on
% the ``sort of theorem'': e.g.\ often remarks and definitions are set
% in |\upshape|, while |\itshape| is employed for main theorems.
%
% Confronted with these requirements, a theorem environment was
% developed in Mainz which allows separate determination of the
% layout of the ``theorems sets'', comparable to |\pagestyle|.
%
%
% \section{The user interface}
% \subsection{Defining new theorem sets}
%
% \DescribeMacro\newtheorem
% As in the original \LaTeX{} version, the command |\newtheorem|
% defines a new ``theorem set'' or ``theorem-like structure''.
% Two required arguments name the new environment and give the text
% to be typeset with each instance of the new ``set'', while
% an optional argument determines how
% the ``set'' is enumerated:
% \begin{description}
%    \item[\ttfamily \bslash newtheorem\{foo\}\{bar\}]
%       The theorem set {\envfont foo} (whose name is \texttt{bar})
%       uses its own counter.
%    \item[\ttfamily \bslash newtheorem\{foo2\}{[foo]}\{bar2\}]
%       The theorem set {\envfont foo2} (printed name \texttt{bar2})
%       uses the same counter as the theorem set \texttt{foo}.
%    \item[\ttfamily \bslash newtheorem\{foo3\}\{bar3\}{[section]}]
%       The theorem set {\envfont foo3} (printed name \texttt{bar3}) is
%       enumerated within the counter \texttt{section}, i.e.\ with every
%       new |\section| the enumeration begins again with 1, and
%       the enumeration is composed from the section-number and the
%       theorem counter itself.
% \end{description}
%
% \DescribeMacro\theoremstyle
% Additionally, the command |\theoremstyle| can define the layout
% of various, or all, theorem sets.  It should be noted that any
% theorem set defined by |\newtheorem| is typeset in the
% |\theoremstyle| that is current at the time of the definition.
% Thus, the following
% \begin{verbatim}
%    \theoremstyle{break}        \newtheorem{Cor}{Corollary}
%    \theoremstyle{plain}        \newtheorem{Exa}{Example}[section]
%\end{verbatim}                             ^^A No blank at line start
% leads to the result that the set {\envfont Cor} is formatted in the
% style \texttt{break}, while the set {\envfont Exa} and all the
% following ones are formatted in the style \texttt{plain}, unless
% another |\theoremstyle| follows.  Since the definitions installed by
% |\newtheorem| are global, one also can limit |\theoremstyle| locally
% by grouping braces.
%
% \DescribeMacro\theorembodyfont
% The choice of the font for the theorem body is completely
% independent of the chosen |\theoremstyle|; this has proven to
% be very advantageous.  For example,
% \begin{verbatim}
%   {\theorembodyfont{\upshape}          \newtheorem{Rem}{Remark}}
%\end{verbatim}
% defines a theorem set {\envfont Rem}, which will be set in
% |\upshape| in the current layout (which in our example is
% \texttt{plain}).  As with |\theoremstyle|, the |\theorembodyfont|
% chosen is that current at the time of |\newtheorem|.  If
% |\theorembodyfont| is not specified or one defines
% |\theorembodyfont{}|, then the font used will be that defined by the
% |\theoremstyle|.
%
% \DescribeMacro\theoremheaderfont It is also possible to customize
% the font used for the theorem headers.  This is, however, a global
% declaration, and therefore there should be at most one
% |\theoremheaderfont| declaration in the preamble.\footnote{If it is
% actually necessary to have different header fonts, one has to define
% new theorem styles (substituting the desired font) or specify the
% information directly in the \texttt{\bslash newtheorem} declaration
% (the unclean variant).}
%
% \DescribeMacro\theorempreskipamount
% \DescribeMacro\theorempostskipamount Two additional parameters
% affect the vertical space around the theorem environments:
% |\theorempreskipamount| and |\theorempostskipamount| define,
% respectively, the spacing before and after such an environment.
% These parameters apply for all theorem sets and can be manipulated
% with the ordinary length macros.  They are rubber lengths
% (`\textsf{skips}'), and therefore can contain \texttt{plus} and
% \texttt{minus} parts.
%
% Since the definition of theorem sets should---most sensibly---be
% placed in the preamble, we only allow installation there.
% It is therefore possible to release the memory used here after
% |\begin{document}|, in order to make room for other
% applications.
%
%
% \subsection{Existing theorem styles}
%
% The following theorem styles exist to date:
% \begin{deflist}{marginbreak}
%    \item[plain]
%       This theorem style emulates the original \LaTeX{} definition,
%       except that additionally the parameters
%       |\theorem...skipamount| are used.
%    \item[break]
%       In this style, the theorem header is followed by a line
%       break.
%    \item[marginbreak]
%       The theorem number is set in the margin, and there is a line
%       break as in \texttt{break}.
%    \item[changebreak]
%       Like \texttt{break}, but with header number and text
%       interchanged.
%    \item[change]
%       Header number and text are interchanged, without a line break.
%    \item[margin]
%       The number is set in the left margin, without a line break.
% \end{deflist}
% All styles (except \texttt{plain}) select |\slshape| as the default
% |\theorembodyfont|.
%
%
% \subsection{Examples}
%
% Given the above theorem sets {\envfont Cor}, {\envfont Exa}
% and {\envfont Rem},
% suppose that the preamble also contains the declarations:
% \begin{verbatim}
%    \theoremstyle{marginbreak}   \newtheorem{Lem}[Cor]{Lemma}
%    \theoremstyle{change}
%    \theorembodyfont{\itshape}        \newtheorem{Def}[Cor]{Definition}
%
%    \theoremheaderfont{\scshape}
%\end{verbatim}
% Then the following are some typical examples of the typeset output
% resulting from their use.
%
% \begin{Cor}
%    This is a sentence typeset in the theorem environment {\envfont
%    Cor}.
% \end{Cor}
%
% \begin{Exa}
%    This is a sentence typeset in the theorem environment {\envfont
%    Exa}.
% \end{Exa}
%
% \begin{Rem}
%    This is a sentence typeset in the theorem environment {\envfont
%    Rem}.
% \end{Rem}
%
% \begin{Lem}[Ben User]
%    This is a sentence typeset in the theorem environment {\envfont
%    Lem}.
% \end{Lem}
%
% \begin{Def}[Very impressive Definition]
%    This is a sentence typeset in the theorem environment {\envfont
%    Def}.
% \end{Def}
%
% The last two examples show the effect of the optional argument to
% a theorem environment (it is the text typeset in parentheses).
%
%
% \section{Special Considerations}
% \label{sec:newfonts}
%
% Theoremheader and body are implemented as a unit. This means that
% the |\theoremheaderfont|  will inherit characteristics of the
% |\theorembodyfont| in \LaTeXe. Thus, if for example
% |\theorembodyfont| is |\itshape| and |\theoremheaderfont|
% is |\bfseries| the font selected for the header will have the
% characteristics `bold extended italic'. If this is not desired one
% should set the |\theoremheaderfont| to something like
% \begin{verbatim}
%      \theoremheaderfont{\normalfont\bfseries}
%\end{verbatim}
% i.e.\ supplying all necessary font information explicitly.
%
%
%
% \section{Acknowledgements}
%
% The publication of this set of macros was only possible with the
% help of Christina Busse (translating the manuscript into English),
% Joachim Pense (playing the r\^ole of typist), Chris Rowley
% (looking everything over) and many others providing useful
% suggestions.
%
% \MaybeStop{%
%    \begin{thebibliography}{9}
%       \bibitem{bk:GMS94} \textsc{M.~Goossens}, \textsc{F.~Mittelbach}
%         and \textsc{A.~Samarin}.
%         \newblock The \LaTeX{} Companion.
%         \newblock
%         Addison-Wesley, Reading, Massachusetts, 1994.
%       \bibitem{latex:source} \textsc{Lamport, Leslie.} \newblock
%                               \texttt{latex.tex}, version 2.09, date
%                               Feb. 1990.
%    \end{thebibliography}
%    \PrintIndex
%    \PrintChanges}
%
%
% \section{The documentation driver file}
%
% The next bit of code contains the documentation driver file for
% \TeX{}, i.e., the file that will produce the documentation you are
% currently reading. It will be extracted from this file by the
% \texttt{docstrip} program. Since it is the first code in the file
% one can alternatively process this file directly with \LaTeXe{} to
% obtain the documentation.
%
%    \begin{macrocode}
%<*driver>
\documentclass{ltxdoc}

\usepackage{theorem}

%<+driver>% The next few lines define theorem sets which are used
%<+driver>% in the example section of the documentation.

\theoremstyle{break}       \newtheorem{Cor}{Corollary}
\theoremstyle{plain}       \newtheorem{Exa}{Example}[section]
{\theorembodyfont{\upshape}\newtheorem{Rem}{Remark}}
\theoremstyle{marginbreak} \newtheorem{Lem}[Cor]{Lemma}
\theoremstyle{change}
\theorembodyfont{\itshape} \newtheorem{Def}[Cor]{Definition}

\theoremheaderfont{\scshape}

\RecordChanges

\begin{document}
   \DocInput{theorem.dtx}
\end{document}
%</driver>
%    \end{macrocode}
%
% \section{Definition of the Macros}
%
% If the file has been loaded before, we abort immediately. If not the
% package announces itself (this is actually done at the very top of
% the file---the way it is done isn't good style so don't copy it).
% \changes{v2.0e}{89/07/19}{Spaces removed from `typeout'.}
% \changes{v2.2b}{95/11/19}{Announce moved to top of file}
%    \begin{macrocode}
%<*package>
%\@ifundefined{theorem@style}{}{\endinput}
%\def\FMithmInfo{1995/11/19 v2.2b Theorem extension package (FMi)}
%\ProvidesPackage{theorem}[\FMithmInfo]
%    \end{macrocode}
%
%
% \subsection{Definition of theorem styles and fonts}
%
% All the definitions in this file are done globally to allow
% inputting this file inside a group.
%
% \begin{macro}{\theoremstyle}
%    Before a theorem style can be installed, the chosen style must be
%    known. For that reason, we must test to see that
%    |\th@|\meta{style} is known or, more precisely, that it is
%    different from |\relax|. If the style is not known then
%    |\th@plain| is used.
%    \begin{macrocode}
\gdef\theoremstyle#1{%
   \@ifundefined{th@#1}{\@warning
          {Unknown theoremstyle `#1'. Using `plain'}%
          \theorem@style{plain}}%
%    \end{macrocode}
%    We save the theorem style to be used in the token register
%    |\theorem@style|.
%    \begin{macrocode}
      {\theorem@style{#1}}%
%    \end{macrocode}
%    Now we ``evaluate'' the theorem style: this means, we call the
%    macro |\th@|\meta{style} which will activate the relevant
%    definitions which are contained in a separate file.  This is done
%    in a group to suppress changes to the current font.  This could
%    otherwise pose problems together with the new font selection
%    scheme\footnote{When I printed the original article using the new
%    font selection scheme I ended with a document with slanted
%    typefaces (text headings and all) simply because one of the
%    theorem styles used \texttt{\bslash sl} at toplevel.} if the
%    |\th@|\meta{style} is evaluated a second time.
% \changes{v2.1a}{90/02/15}{Group added around `th@..}
% \changes{v2.1a}{90/02/15}{Reseting of `..@begintheorem removed.}
%    \begin{macrocode}
      \begingroup
        \csname th@\the\theorem@style \endcsname
      \endgroup}
%    \end{macrocode}
% \end{macro}
%
%
% \begin{macro}{\@begintheorem}
% \changes{v2.1a}{90/02/15}{Relaxed at toplevel.}
% \begin{macro}{\@opargbegintheorem}
% \changes{v2.1a}{90/02/15}{Relaxed at toplevel.}
%    We reset |\@begintheorem| and |\@opargbegintheorem| to
%    |\relax| since these commands are no longer necessary at
%    toplevel. This will save a few tokens.
%    \begin{macrocode}
\global\let\@begintheorem\relax
\global\let\@opargbegintheorem\relax
%    \end{macrocode}
% \end{macro}
% \end{macro}
%
% \begin{macro}{\theorem@style}
%    Obviously the \textsf{token} register used above has to be
%    allocated. To assure the utmost compatibility with the
%    original \LaTeX{} definition, we set the default theorem style to
%    \texttt{plain}, which implements the usual \LaTeX{} convention.
% \changes{v2.0f}{89/07/20}{Unnecessary `global' removed.}
%    \begin{macrocode}
\newtoks\theorem@style
\global\theorem@style{plain}
%    \end{macrocode}
% \end{macro}
%
% \begin{macro}{\theorembodyfont}
% \begin{macro}{\theorem@bodyfont}
%    If the |\theorembodyfont| is set by the user then it should not
%    interact with the default font set in the theorem style. When the
%    new font selection is in force this may happen if, for example,
%    the default is |\itshape| and the new |\theorembodyfont| is
%    |\sffamily|. So we add a |\reset@font| command in front of the
%    user definition.
% \changes{v2.0f}{89/07/20}{Unnecessary `global' removed.}
% \changes{v2.0g}{89/09/19}{`theoremfont to `theorembodyfont.}
% \changes{v2.1c}{92/06/22}{`reset@font added and logic changed.}
% \changes{v2.1d}{92/10/28}{Reset to empty if arg is empty}
%    \begin{macrocode}
\gdef\theorembodyfont#1{%
%    \end{macrocode}
%    We check if the argument supplied is empty and if so put nothing
%    into the |\theorem@bodyfont| token register to allow for
%    |\theorembodyfont{}| as a mean of using the default of the
%    current |\theoremstyle|.
%    \begin{macrocode}
   \def\@tempa{#1}%
   \ifx\@tempa\@empty
    \theorem@bodyfont{}%
   \else
    \theorem@bodyfont{\reset@font#1}%
   \fi
   }
\newtoks\theorem@bodyfont
\global\theorem@bodyfont{}
%    \end{macrocode}
% \end{macro}
% \end{macro}
%
%
% \begin{macro}{\theoremheaderfont}
% \changes{v2.0g}{89/09/19}{Macro added.}
%    The font for the theorem headers is handled differently because
%    this definition applies to all theorem styles.
%    \begin{macrocode}
\gdef\theoremheaderfont#1{\gdef\theorem@headerfont{#1}%
%    \end{macrocode}
%    After using the macro once it is redefined to produce an error
%    message.
%    \begin{macrocode}
       \gdef\theoremheaderfont##1{%
        \typeout{\string\theoremheaderfont\space should be used
                 only once.}}}
%    \end{macrocode}
% \end{macro}
%
% \begin{macro}{\theorem@headerfont}
% \changes{v2.0g}{89/09/19}{Macro added.}
%    To set the |\theorem@headerfont| default we first test if
%    the new fontselection scheme is in force.
% \changes{v2.1a}{90/02/15}{Support for new font selection.}
%    \begin{macrocode}
\ifx\upshape\undefined
%    \end{macrocode}
%    If not we define it to expand into |\bfseries|. We don't use
%    |\let| just in case a following style option redefines this
%    macro.
%    \begin{macrocode}
\gdef\theorem@headerfont{\bfseries}
%    \end{macrocode}
%    Otherwise we reset the current shape before calling |\bfseries|.
%    \begin{macrocode}
\else \gdef\theorem@headerfont{\normalfont\bfseries}\fi
%    \end{macrocode}
% \end{macro}
%
% \begin{macro}{\th@plain}
% \begin{macro}{\th@break}
% \begin{macro}{\th@marginbreak}
% \begin{macro}{\th@changebreak}
% \begin{macro}{\th@change}
% \begin{macro}{\th@margin}
%    The different styles are defined in macros such as
%    |\th@plain|.  Since memory space is precious in
%    ``non-Big-versions'', we have to avoid offering too many unused
%    definitions.  Therefore we define these styles in separate files
%    that can be loaded on demand.  Thus the commands themselves only
%    load these files. We use |\@input@|, a \LaTeXe{} internal command
%    that ensures that the file will be listed with |\listfiles|.
% \changes{v2.2a}{94/02/02}{Use `@input@ to load theorem layout files}
%    \begin{macrocode}
\gdef\th@plain{\@input@{thp.sty}}
\gdef\th@break{\@input@{thb.sty}}
\gdef\th@marginbreak{\@input@{thmb.sty}}
\gdef\th@changebreak{\@input@{thcb.sty}}
\gdef\th@change{\@input@{thc.sty}}
\gdef\th@margin{\@input@{thm.sty}}
%    \end{macrocode}
%    This list will be expanded when new styles become
%    available.  For testing,  just append new theorem substyles
%    as document options.
% \end{macro}
% \end{macro}
% \end{macro}
% \end{macro}
% \end{macro}
% \end{macro}
%
%
%
% \subsection{Definition of a new theorem set}
%
% As already pointed out, a new theorem environment can be defined in
% three different ways:
% \begin{verbatim}
%    \newtheorem{Lem}{Lemma}
%    \newtheorem{Lem}{Lemma}[section]
%    \newtheorem{Lem}[Theorem]{Lemma}
%\end{verbatim}
% The function of the macro |\newtheorem| is to recognize these cases
% and then to branch into one of the three macros |\@ynthm|, |\@xnthm|
% or |\@othm|.  This mechanism is adopted unchanged from
% \cite{latex:source}; the essential point here is that, for example,
% in the second case, the arguments \texttt{Lem}, \texttt{Lemma} and
% \texttt{section} are passed over to the macro |\@xnthm|.
%
% We inspect this case first because the others present fewer problems,
% and thus are easily derived from this one.
%
% \begin{macro}{\@xnthm}
%    For our example arguments, the macro |\@xnthm| must fulfill
%    the following:
%    \begin{itemize}
%       \item
%          Define a new \LaTeX-counter `Lem'
%       \item
%          reset this counter within a |\section|
%       \item
%          define the macro |\theLem|
%       \item
%          define the environment macros |\Lem| and
%          |\endLem| using the current
%          |\theoremstyle| and |\theorem@bodyfont|.
%    \end{itemize}
%    Obviously, all this should happen only if the first argument
%    of |\@xnthm| (i.e.\ \texttt{Lem} in our example) is chosen so
%    as not to conflict with any previously defined commands or
%    environments.  This test is performed by the \LaTeX{} macro
%    |\@ifdefinable|.
%    \begin{macrocode}
\gdef\@xnthm#1#2[#3]{\expandafter\@ifdefinable\csname #1\endcsname
%    \end{macrocode}
%    Therefore, the first argument of |\@ifdefinable| is the
%    expansion (in the example, |\Lem|) of
%    |\csname||#1\endcsname|.  The second argument is
%    executed only if the test has been completed successfully.
%    \begin{macrocode}
   {%
%    \end{macrocode}
%    Now we define the new counter. The names of the \LaTeX{}
%    macros employed should speak for themselves:
% \changes{v2.2b}{1995/10/07}{Use \cs{@newctr}}
% \changes{v2.2c}{1995/11/23}{Use \cs{@newctr} properly.}
%    \begin{macrocode}
    \@definecounter{#1}\@newctr{#1}[#3]%
%    \end{macrocode}
%    Using |\@newctr| will give a proper error message if the counter
%    in |#3| is not defined.
%    In defining `|\theLem|' we must generate the desired macro
%    name by use of |\expandafter| and |\csname|.
%    \begin{macrocode}
    \expandafter\xdef\csname the#1\endcsname
%    \end{macrocode}
%    An |\xdef| is used in order to make the definition global,
%    and to ensure that it contains the replacement texts of
%    |\@thmcountersep| and |\@thmcounter|.\footnote{These
%    two macros can be defined by the document style.  Their default
%    values produce a `\texttt{.}' as separation and an arabic
%    representation of the number.} However, not everything should be
%    expanded. For example, it saves space to use |\thesection|
%    instead of its---at times---lengthy expansion.
%    \begin{macrocode}
      {\expandafter \noexpand \csname the#3\endcsname
       \@thmcountersep \@thmcounter{#1}}%
%    \end{macrocode}
%    Thus with the defaults of \LaTeX{}, |\theLem| would be
%    replaced by the command sequence |\thesection.\arabic{Lem}|.
%
%    We will now look at the definition of the macro which is executed
%    at the beginning of the actual environment (in our example this
%    macro is |\Lem|). It should be noted that we use an
%    ``|\expandafter| trick'' to expand only certain parts of the
%    replacement text at the time of the definition.
%    \begin{macrocode}
    \def\@tempa{\global\@namedef{#1}}%
    \expandafter \@tempa \expandafter{%
%    \end{macrocode}
%    First, the macro that contains the current definitions of
%    |\@begintheorem| and |\@opargtheorem| should be called
%    up.  The name of this macro---as is already known---has the form
%    |\th@|\meta{theorem\ style}; therefore, it must be called by
%    \begin{macrocode}
      \csname th@\the \theorem@style
%    \end{macrocode}
%    In addition the default theorem font should be changeable, i.e.\
%    we have to insert the contents of |\theorem@bodyfont|. For
%    that reason, we expand even further, beyond |\endcsname|,
%    and thus insert the contents of the \textsf{token} register
%    |\theorem@bodyfont| in the replacement text.
%    \begin{macrocode}
            \expandafter \endcsname \the \theorem@bodyfont
%    \end{macrocode}
%    Now it is time to call the macro |\@thm| which takes over
%    the further processing.  It has two arguments: the current
%    counter name (in our example, \texttt{Lem}), and the text of the
%    label (in our example, \texttt{Lemma}).
%    \begin{macrocode}
     \@thm{#1}{#2}}%
%    \end{macrocode}
%    With this, the `sub-definition' is complete.  The macro
%    |\@endtheorem| ends a theorem environment and is, so far, nothing
%    but an |\endtrivlist|.  (Hence it is defined globally, and not
%    within the theorem styles.\footnote{This has to be changed as
%    soon as theorem styles that change \texttt{\bslash @endtheorem}
%    exist.  In such a case, all existing styles must be changed as
%    well since they will have to reset the macro.}) Therefore, we can
%    set it equivalent to the macro that ends the theorem set (in our
%    example, |\endLem|).  However, if some day theorem styles exist
%    that do change |\@endtheorem|, we would have to use the
%    commented-out line instead.
%    \begin{macrocode}
    \global \expandafter \let \csname end#1\endcsname \@endtheorem
%   \global\@namedef{end#1}{\@endtheorem}%
%    \end{macrocode}
%    With these commands all the required definitions are employed,
%    unless the test |\@ifdefinable| has failed.  Therefore, we
%    end the second argument of this macro and with it the definition
%    of |\@xnthm|.
%    \begin{macrocode}
   }}
%    \end{macrocode}
% \end{macro}
%
% \begin{macro}{\@ynthm}
%    The definition of |\@ynthm| is completely analogous. In
%    this case the new counter that is defined is not reset within
%    another counter; thus the definition of |\the...|
%    is simplified:
%    \begin{macrocode}
\gdef\@ynthm#1#2{\expandafter\@ifdefinable\csname #1\endcsname
   {\@definecounter{#1}%
    \expandafter\xdef\csname the#1\endcsname{\@thmcounter{#1}}%
%    \end{macrocode}
%    The rest of the definition corresponds literally to that of
%    |\@xnthm|:
%    \begin{macrocode}
    \def\@tempa{\global\@namedef{#1}}\expandafter \@tempa
     \expandafter{\csname th@\the \theorem@style \expandafter
     \endcsname \the\theorem@bodyfont \@thm{#1}{#2}}%
    \global \expandafter \let \csname end#1\endcsname \@endtheorem}}
%    \end{macrocode}
% \end{macro}
%
% \begin{macro}{\@othm}
%    The definition of |\@othm| does not contain anything new.
%    \begin{macrocode}
\gdef\@othm#1[#2]#3{%
%    \end{macrocode}
% \changes{v2.2b}{95/10/07}{Build in check for counter defined}
%    We do not define a new counter but instead use one that has
%    already been defined.  Thus the only definition we need is that
%    of this pseudo-counter (i.e.\ |\the|\meta{env.\ name}).
%    First we check if |#2| corresponds to a known counter name.
%    \begin{macrocode}
  \expandafter\ifx\csname c@#2\endcsname\relax
   \@nocounterr{#2}%
  \else
   \expandafter\@ifdefinable\csname #1\endcsname
   {\expandafter \xdef \csname the#1\endcsname
     {\expandafter \noexpand \csname the#2\endcsname}%
%    \end{macrocode}
%    All other parts of the definition can be adopted from
%    |\@xnthm|. We have to remember, though, that in this case
%    the name of the current counter and the theorem label have moved
%    to the second and third arguments.
%    \begin{macrocode}
    \def\@tempa{\global\@namedef{#1}}\expandafter \@tempa
     \expandafter{\csname th@\the \theorem@style \expandafter
     \endcsname \the\theorem@bodyfont \@thm{#2}{#3}}%
    \global \expandafter \let \csname end#1\endcsname \@endtheorem}%
  \fi}
%    \end{macrocode}
% \end{macro}
%
% \subsection{Macros that are employed in a {\envfont theorem}
%    environment}
%
% \begin{macro}{\@thm}
%    The macro |\@thm| has to increase the current counter.  Then,
%    depending on whether the environment has (or does not have) an
%    optional argument, it has to branch into either |\@begintheorem|
%    or |\@opargtheorem|.
%    \begin{macrocode}
\gdef\@thm#1#2{\refstepcounter{#1}%
%    \end{macrocode}
%    Now we start a {\envfont trivlist} environment, and give
%    |\@topsep| and |\@topsepadd| the values of the \textsf{skip}
%    registers |\theorempreskipamount| and |\theorempostskipamount|.
%    The value in |\@topsep| is the vertical space that is inserted by
%    the first (and only) |\item| in our |\trivlist| whilst
%    |\@topsepadd| is inserted by |\@endparenv| at the end of that
%    {\envfont trivlist} environment.  By using these registers, we
%    obtain the desired space around a {\envfont theorem} environment.
%    \begin{macrocode}
   \trivlist
   \@topsep \theorempreskipamount               % used by first \item
   \@topsepadd \theorempostskipamount           % used by \@endparenv
%    \end{macrocode}
%    Now we have to test whether an optional argument has been given.
% \changes{v2.0d}{89/07/18}{Forgotten percent added.}
%    \begin{macrocode}
   \@ifnextchar [%
%    \end{macrocode}
%    If there is an optional argument, we will call |\@ythm|, and
%    move the arguments read back into the input stream.
%    \begin{macrocode}
   {\@ythm{#1}{#2}}%
%    \end{macrocode}
%    If not, we call |\@begintheorem|. Its first argument is the
%    name of the theorem set (hence the second argument of
%    |\@thm|). Its second argument is the macro that produces the
%    current number.
%    \begin{macrocode}
   {\@begintheorem{#2}{\csname the#1\endcsname}\ignorespaces}}
%    \end{macrocode}
% \end{macro}
%
% \begin{macro}{\@xthm}
% \begin{macro}{\@ythm}
%    Both these macros were originally called by |\@thm|.  We do
%    not need |\@xthm| anymore, hence we reset it to
%    |\relax|. The definition of |\@ythm| has not changed at
%    all from its definition in \LaTeX. In order to make the macros
%    easier to understand, we will
%    nevertheless present it (commented out).
%    \begin{macrocode}
\global\let\@xthm\relax
% \def\@ythm#1#2[#3]{\@opargbegintheorem{#2}{\csname
%           the#1\endcsname}{#3}\ignorespaces}
%</package>
%    \end{macrocode}
%    The primitive |\ignorespaces| in |\@ythm| and
%    |\@thm| is needed to remove the spaces between the
%    |\begin{...}| and the actual text.
% \end{macro}
% \end{macro}
%
%
% \subsection{Definition of the theorem substyles}
%
% As already pointed out, the theorem substyles, defined below, are
% only loaded when necessary.  Note that all these substyles, except
% \texttt{plain}, have |\slshape| as the default body font.
%
% \subsubsection{The \texttt{plain} style}
%
% \label{dok:th@plain} As the following macros use |@|, we have
% to locally set the |\catcode| of this symbol to ``letter''.
% This happens within a group, so that we do not have to worry about
% which |\catcode| that symbol had before.
%    \begin{macrocode}
%<*thp>
\begingroup \makeatletter
%    \end{macrocode}
% Since we are now within a group, we must make all definitions
% globally.  First we make sure that \texttt{theorem.sty} is loaded.
% This will allow us to use this file as a document style option
% without having to call \texttt{theorem} itself as an option.  At the
% same time, we assure that at least version 2 is loaded, since
% |\theorem@style| was not defined in earlier versions.
%    \begin{macrocode}
\@ifundefined{theorem@style}{\input{theorem.sty}}{}
\ProvidesFile{thp.sty}
             [\FMithmInfo]
%    \end{macrocode}
%
%
% \begin{macro}{\th@plain}
% \begin{macro}{\@begintheorem}
% \begin{macro}{\@opargbegintheorem}
%    |\theoremstyle{plain}| corresponds to the original
%    definition, except that the distances to the surrounding text are
%    determined by the parameters |\theorempreskipamount| and
%    |\theorempostskipamount|.
%    First we set the default body font.
%    \begin{macrocode}
\gdef\th@plain{\normalfont\itshape
%    \end{macrocode}
%    Then we define |\@begintheorem| and
%    |\@opargbegintheorem|. These two macros define how the
%    header of a theorem is typeset.  |\@opargbegintheorem| will
%    be called if a \textsf{theorem} environment with an optional
%    argument is encountered; otherwise, the header is constructed by
%    calling |\@begintheorem|.  If one of these macros is
%    executed, we are within a \textsf{trivlist} environment started by
%    |\@thm|.  So the theorem header is produced with an
%    |\item| command.
%
%    Instead of specifying the header font directly, all standard
%    theorem styles use the |\theorem@headerfont| macro to allow
%    customization.  The extra space (|\labelsep|) is necessary
%    because of problems in the \textsf{trivlist} environment.
%    \begin{macrocode}
  \def\@begintheorem##1##2{%
        \item[\hskip\labelsep \theorem@headerfont ##1\ ##2]}%
%    \end{macrocode}
%    The definition of |\@opargbegintheorem| is completely
%    analogous. The only difference is the fact that there exists a
%    third argument (which is the optional parameter of the
%    environment and contains additional information about the
%    theorem).  Customarily we enclose it in parentheses.
%    \begin{macrocode}
\def\@opargbegintheorem##1##2##3{%
   \item[\hskip\labelsep \theorem@headerfont ##1\ ##2\ (##3)]}}
%    \end{macrocode}
% \end{macro}
% \end{macro}
% \end{macro}
%
% \noindent
% We conclude with an |\endgroup| to restore the
% |\catcode| of |@|.
%    \begin{macrocode}
\endgroup
%</thp>
%    \end{macrocode}
%
%
% \subsubsection{The \texttt{break} style}
%
% This style option is stored in the file |thb.sty|.
% For the next two lines see the documentation for |\th@plain| on
% page~\pageref{dok:th@plain}.
%    \begin{macrocode}
%<*thb>
\begingroup \makeatletter
\@ifundefined{theorem@style}{\input{theorem.sty}}{}
\ProvidesFile{thb.sty}
             [\FMithmInfo]
%    \end{macrocode}
%
% \begin{macro}{\th@break}
%    |\theoremstyle{break}| produces a line break after the name
%    of the theorem.  The font is |\slshape|.  Hence, we define
%    |\th@break| as follows:
%    \begin{macrocode}
\gdef\th@break{\normalfont\slshape
  \def\@begintheorem##1##2{\item[%
%    \end{macrocode}
%    We run into the following problem: it is not possible to create
%    the header with |\item[|\meta{title}|]| and then start
%    a new line by, for example, |\mbox{}\\|. Such a definition
%    will fail whenever a list environment follows immediately.  With
%    the above construction, the |\mbox{}| causes the switch
%    |@inlabel| (cf.\ definition of |\list| and
%    |\trivlist| in \cite{latex:source}) to be set to \texttt{false}
%    and so the following list will insert additional vertical
%    space (|\topskip|). This is quite annoying. Therefore, we
%    create the line break within the |\item|. In order to ensure
%    that the text will begin at the proper position in the following
%    line, we simply pretend that the label does not take any
%    room.\footnote{This will lead to problems whenever very high
%    symbols occurring in the line tower into the
%    heading. So, something else has to be done here sometime.}
%    \begin{macrocode}
       \rlap{\vbox{\hbox{\hskip \labelsep\theorem@headerfont ##1\ ##2}%
                   \hbox{\strut}}}]}%
%    \end{macrocode}
%    Again, the definition of |\@opargbegintheorem| is
%    completely analogous.
%    \begin{macrocode}
\def\@opargbegintheorem##1##2##3{%
  \item[\rlap{\vbox{\hbox{\hskip \labelsep \theorem@headerfont
                     ##1\ ##2\ (##3)}%
                    \hbox{\strut}}}]}}
\endgroup
%</thb>
%    \end{macrocode}
% \end{macro}
%
%
% \subsubsection{The \texttt{changebreak} style}
%
%    \begin{macrocode}
% This style option is stored in the file |thcb.sty|.
%    \begin{macrocode}
%<*thcb>
\begingroup \makeatletter
\@ifundefined{theorem@style}{\input{theorem.sty}}{}
\ProvidesFile{thcb.sty}
             [\FMithmInfo]
%    \end{macrocode}
%
% \begin{macro}{\th@changebreak}
%    The \texttt{change-break} theorem style is like \texttt{break}
%    but with interchange of theorem name and theorem number.  Thus we
%    define |\th@changebreak| as follows:
%    \begin{macrocode}
\gdef\th@changebreak{\normalfont\slshape
  \def\@begintheorem##1##2{\item
       [\rlap{\vbox{\hbox{\hskip\labelsep \theorem@headerfont ##2\ ##1}%
                    \hbox{\strut}}}]}%
\def\@opargbegintheorem##1##2##3{%
  \item[\rlap{\vbox{\hbox{\hskip\labelsep \theorem@headerfont
                     ##2\ ##1\ (##3)}%
                    \hbox{\strut}}}]}}
\endgroup
%</thcb>
%    \end{macrocode}
% \end{macro}
%
%
% \subsubsection{The \texttt{change} style}
%
% This style option is stored in the file |thc.sty|.
%    \begin{macrocode}
%<*thc>
\begingroup \makeatletter
\@ifundefined{theorem@style}{\input{theorem.sty}}{}
\ProvidesFile{thc.sty}
             [\FMithmInfo]
%    \end{macrocode}
%
% \begin{macro}{\th@change}
%    The \texttt{change} theorem style corresponds to the
%    \texttt{change break} style without a linebreak after the header.
%    To say it in another way, it's the same as the \texttt{plain}
%    style but with number and name interchanged and |\slshape| as the
%    default font.
%    \begin{macrocode}
\gdef\th@change{\normalfont\slshape
  \def\@begintheorem##1##2{\item
       [\hskip\labelsep \theorem@headerfont ##2\ ##1]}%
\def\@opargbegintheorem##1##2##3{%
  \item[\hskip\labelsep \theorem@headerfont ##2\ ##1\ (##3)]}}
\endgroup
%</thc>
%    \end{macrocode}
% \end{macro}
%
%
% \subsubsection{The \texttt{marginbreak} style}
%
% This style option is the one used most often at Mainz.  It is saved
% in the file |thmb.sty|.
%    \begin{macrocode}
%<*thmb>
\begingroup \makeatletter
\@ifundefined{theorem@style}{\input{theorem.sty}}{}
\ProvidesFile{thmb.sty}
             [\FMithmInfo]
%    \end{macrocode}
%
% \begin{macro}{\th@marginbreak}
%    The \texttt{margin break} style is nearly the same as the
%    \texttt{change break} style. The only difference is the placement
%    of the theorem number.  We use |\llap| to place it in the left
%    margin.
%
%    In this style |\labelsep| denotes the separation between the
%    number and the text.
%    \begin{macrocode}
\gdef\th@marginbreak{\normalfont\slshape
  \def\@begintheorem##1##2{\item
       [\rlap{\vbox{\theorem@headerfont
                    \hbox{\llap{##2}\hskip\labelsep ##1}%
                    \hbox{\strut}}}]}%
\def\@opargbegintheorem##1##2##3{%
  \item[\rlap{\vbox{\theorem@headerfont
                    \hbox{\llap{##2}\hskip\labelsep ##1\ (##3)}%
                    \hbox{\strut}}}]}}
\endgroup
%</thmb>
%    \end{macrocode}
% \changes{v2.1c}{90/02/17}{`endgroup outside macrocode env.}
% \end{macro}
%
%
%
% \subsubsection{The \texttt{margin} style}
%
% This style option is stored in the file |thm.sty|.
%    \begin{macrocode}
%<*thm>
\begingroup \makeatletter
\@ifundefined{theorem@style}{\input{theorem.sty}}{}
\ProvidesFile{thm.sty}
             [\FMithmInfo]
%    \end{macrocode}
%
% \begin{macro}{\th@margin}
%    Again this is only a variant of the theorem styles described
%    above without any new ideas.
%    \begin{macrocode}
\gdef\th@margin{\normalfont\slshape
  \def\@begintheorem##1##2{\item
       [\theorem@headerfont \llap{##2}\hskip\labelsep ##1]}%
\def\@opargbegintheorem##1##2##3{%
  \item[\theorem@headerfont \llap{##2}\hskip\labelsep ##1\ (##3)]}}
\endgroup
%</thm>
%    \end{macrocode}
% \end{macro}
%
%
%
%
% \subsection{Final Definitions}
%
% \begin{macro}{\theorempreskipamount}
% \begin{macro}{\theorempostskipamount}
%    The \textsf{skip} parameters that regulate the vertical empty space
%    before and after the {\envfont theorem} environment have
%    to be allocated as well.
% \changes{v2.0f}{89/07/20}{Unnecessary `global' removed.}
%    \begin{macrocode}
%<*package>
\newskip\theorempreskipamount
\newskip\theorempostskipamount
%    \end{macrocode}
%    Since we have used the same values for all theorem sets, we now can
%    assign them.
%    \begin{macrocode}
\global\setlength\theorempreskipamount{12pt plus 5pt minus 3pt}
\global\setlength\theorempostskipamount{8pt plus 3pt minus 1.5pt}
%    \end{macrocode}
% \end{macro}
% \end{macro}
%
%
% \begin{macro}{\@endtheorem}
%    The same holds for the macro |\@endtheorem|, which ends a
%    {\envfont theorem} environment. Since it is the same for all
%    theorem sets, it is removed from the macros |\th@|\meta{style}.
%    It simply ends the {\envfont trivlist} environment, which was
%    begun in |\@thm|.
%    \begin{macrocode}
\global\let\@endtheorem=\endtrivlist
%    \end{macrocode}
% \end{macro}
%
%
%
% \begin{macro}{\@preamblecmds}
%    All macros defined above are to be used only in the preamble.
%    Therefore, we insert them in |\@preamblecmds| which will disable
%    them at begin document. This is done by the internal \LaTeXe{}
%    command |\@onlypreamble|.
% \changes{v2.0g}{89/09/19}{`theoremheaderfont and `theorembodyfont
%    added.}
% \changes{v2.2a}{94/02/03}{Use \LaTeXe{} interface}
%    \begin{macrocode}
\@onlypreamble\@xnthm
\@onlypreamble\@ynthm
\@onlypreamble\@othm
\@onlypreamble\newtheorem
\@onlypreamble\theoremstyle
\@onlypreamble\theorembodyfont
\@onlypreamble\theoremheaderfont
%    \end{macrocode}
% \end{macro}
%
% Finally we declare the \texttt{plain} theorem style to be the default.
% \changes{v2.1b}{90/06/04}{Fixed missing `item by declaring default
%    style.}
%    \begin{macrocode}
\theoremstyle{plain}
%</package>
%    \end{macrocode}
%
% \Finale

\endinput
}{}
\ProvidesFile{thb.sty}
             [\FMithmInfo]
%    \end{macrocode}
%
% \begin{macro}{\th@break}
%    |\theoremstyle{break}| produces a line break after the name
%    of the theorem.  The font is |\slshape|.  Hence, we define
%    |\th@break| as follows:
%    \begin{macrocode}
\gdef\th@break{\normalfont\slshape
  \def\@begintheorem##1##2{\item[%
%    \end{macrocode}
%    We run into the following problem: it is not possible to create
%    the header with |\item[|\meta{title}|]| and then start
%    a new line by, for example, |\mbox{}\\|. Such a definition
%    will fail whenever a list environment follows immediately.  With
%    the above construction, the |\mbox{}| causes the switch
%    |@inlabel| (cf.\ definition of |\list| and
%    |\trivlist| in \cite{latex:source}) to be set to \texttt{false}
%    and so the following list will insert additional vertical
%    space (|\topskip|). This is quite annoying. Therefore, we
%    create the line break within the |\item|. In order to ensure
%    that the text will begin at the proper position in the following
%    line, we simply pretend that the label does not take any
%    room.\footnote{This will lead to problems whenever very high
%    symbols occurring in the line tower into the
%    heading. So, something else has to be done here sometime.}
%    \begin{macrocode}
       \rlap{\vbox{\hbox{\hskip \labelsep\theorem@headerfont ##1\ ##2}%
                   \hbox{\strut}}}]}%
%    \end{macrocode}
%    Again, the definition of |\@opargbegintheorem| is
%    completely analogous.
%    \begin{macrocode}
\def\@opargbegintheorem##1##2##3{%
  \item[\rlap{\vbox{\hbox{\hskip \labelsep \theorem@headerfont
                     ##1\ ##2\ (##3)}%
                    \hbox{\strut}}}]}}
\endgroup
%</thb>
%    \end{macrocode}
% \end{macro}
%
%
% \subsubsection{The \texttt{changebreak} style}
%
%    \begin{macrocode}
% This style option is stored in the file |thcb.sty|.
%    \begin{macrocode}
%<*thcb>
\begingroup \makeatletter
\@ifundefined{theorem@style}{% \iffalse meta-comment
%
% Copyright (C) 1993-2024
%
% The LaTeX Project and any individual authors listed elsewhere
% in this file.
%
% This file is part of the Standard LaTeX `Tools Bundle'.
% -------------------------------------------------------
%
% It may be distributed and/or modified under the
% conditions of the LaTeX Project Public License, either version 1.3c
% of this license or (at your option) any later version.
% The latest version of this license is in
%    https://www.latex-project.org/lppl.txt
% and version 1.3c or later is part of all distributions of LaTeX
% version 2005/12/01 or later.
%
% The list of all files belonging to the LaTeX `Tools Bundle' is
% given in the file `manifest.txt'.
%
% \fi
%
%%
%
%
\def\FMithmInfo{2023/07/05 v2.2c Theorem extension package (FMi)}
%
% \ProvidesFile{theorem.dtx}[\FMithmInfo]
%\iffalse   % this is a METACOMMENT !
%%
%% Package `theorem' to use with LaTeX2e
%% Copyright (C) 1989-1995 Frank Mittelbach, all rights reserved.
%<+package>\@ifundefined{theorem@style}{}{\endinput}
%<+package>\NeedsTeXFormat{LaTeX2e}[1995/06/01]
%<+package>\ProvidesPackage{theorem}
%<+package>         [\FMithmInfo]
%
% \fi
%
% \changes{v1.0}{87/05/07}{First draft.}
% \changes{v1.0a}{87/05/07}{`thistheoremstyle' added.}
% \changes{v1.1a}{87/07/29}{New concept: redefine the internal
%                          \protect\LaTeX{} commands `@xnthm' etc.}
% \changes{v1.1b}{87/07/31}{`theoremfont' used to specify font.}
% \changes{v1.2a}{88/01/08}{`@endtheorem' now globally defined.}
% \changes{v1.2b}{88/01/11}{`newtheorem' only allowed in the preamble.}
% \changes{v1.3a}{88/02/14}{`theoremfont' now a toks register.}
%
% \changes{v2.0a}{88/11/16}{New implementation and new documentation.}
% \changes{v2.0a}{88/11/16}{No layout changes but internal modifications
%                           old substyles won't work.}
% \changes{v2.0b}{89/01/30}{`global' used for all definitions.}
% \changes{v2.0c}{89/06/15}{Documentation translated to English.
%                            Thanks to LLL and Nina}
% \changes{v2.1a}{90/02/15}{Support for new font selection.}
% \changes{v2.2a}{94/02/03}{Cleanup for LaTeX2e}
%
% \DoNotIndex{\@endparenv,\@ifdefinable,\@ifnextchar,\@ifundefined}
% \DoNotIndex{\@namedef,\@nameuse,\@spaces,\@tempa,\@warning}
% \DoNotIndex{\ }
% \DoNotIndex{\begingroup,\bfseries}
% \DoNotIndex{\csname}
% \DoNotIndex{\def,\do,\docdate}
% \DoNotIndex{\endcsname,\endgroup,\endinput,\endtrivlist,\expandafter}
% \DoNotIndex{\filedate,\fileversion}
% \DoNotIndex{\gdef,\global}
% \DoNotIndex{\hbox,\hskip}
% \DoNotIndex{\ignorespaces,\input,\itshape,\item}
% \DoNotIndex{\labelsep,\let,\llap}
% \DoNotIndex{\makeatletter}
% \DoNotIndex{\noexpand}
% \DoNotIndex{\refstepcounter,\relax,\rlap}
% \DoNotIndex{\setlength,\slshape,\space,\string,\strut}
% \DoNotIndex{\the,\trivlist,\typeout}
% \DoNotIndex{\vbox}
% \DoNotIndex{\xdef}
%
% \def\deflabel#1{\ttfamily #1\hfill}
% \def\deflist#1{\begin{list}{}{\settowidth\labelwidth{\ttfamily #1}%
%                               \setlength\leftmargin\labelwidth
%                               \addtolength\leftmargin\labelsep
%                               \let\makelabel\deflabel}}
% \def\enddeflist{\end{list}}
% \def\envfont{\normalfont\ttfamily}
%
% \MakeShortVerb{\|}
%
% \newcommand{\TUB}{{\sl TUGboat\/}}
%
% \GetFileInfo{theorem.dtx}
% \title{An Extension of the
%        \LaTeX{} theorem environment\thanks{This file
%        has version number \fileversion, last
%        revised \filedate.}}
% \author{Frank Mittelbach}
%
% \MaintainedByLaTeXTeam{tools}
% \maketitle
%
% \begin{abstract}
%    The macros described in this paper yield an extension of the
%    \LaTeX{} theorem mechanism. It is designed to satisfy the
%    different requirements of various journals. Thus, the
%    layout of the ``theorems'' can be manipulated by determining a
%    ``style''. This article describes not only the use, but
%    also the definition, of the necessary macros.
% \end{abstract}
%
% \section*{Preface to version 2.2}
%
% For \LaTeXe{} this package did not need any fundamental changes. I
% only modified the messages generated so that theorem layout styles
% will show up with the |\listfiles| command and cleaned the section
% on the New Font Selection Scheme since this is now included in
% \LaTeX.
%
% \section*{Preface to version 2.1}
%
% This version is identical to 2.0g described in \TUB\ 10\#3 except
% for some internal defaults which are now set depending on the used
% font selection scheme.
%
% This was done to avoid unpleasant surprises if the new font selection
% scheme is in force. For further details see
% section~\ref{sec:newfonts} and \cite{bk:GMS94}.
%
% \section{Introduction}
%
% For our purposes here, ``theorems'' are labelled enunciations,
% often set off from the main text by extra space and a font change.
% Theorems, corollaries, conjectures, definitions, and remarks
% are all instances of ``theorems''.  The ``header'' of these structures
% is composed of a label (such as \textsc{Theorem} or \textsc{Remark})
% and a number which serializes an item in the sequence of items
% with the same label.
%
% Shortly after the introduction of \LaTeX{} at the Fachbereich
% Mathematik in Mainz, the desire to manipulate the layout of
% ``theorems'' arose.  In Mainz, the following two conventions came
% into general use:
% \begin{enumerate}
%   \item The number of the theorem is shown in the margin.
%   \item There is a line break at
%         the end of the theorem header.
% \end{enumerate}
%
% Additionally, some journals require different formats which depend on
% the ``sort of theorem'': e.g.\ often remarks and definitions are set
% in |\upshape|, while |\itshape| is employed for main theorems.
%
% Confronted with these requirements, a theorem environment was
% developed in Mainz which allows separate determination of the
% layout of the ``theorems sets'', comparable to |\pagestyle|.
%
%
% \section{The user interface}
% \subsection{Defining new theorem sets}
%
% \DescribeMacro\newtheorem
% As in the original \LaTeX{} version, the command |\newtheorem|
% defines a new ``theorem set'' or ``theorem-like structure''.
% Two required arguments name the new environment and give the text
% to be typeset with each instance of the new ``set'', while
% an optional argument determines how
% the ``set'' is enumerated:
% \begin{description}
%    \item[\ttfamily \bslash newtheorem\{foo\}\{bar\}]
%       The theorem set {\envfont foo} (whose name is \texttt{bar})
%       uses its own counter.
%    \item[\ttfamily \bslash newtheorem\{foo2\}{[foo]}\{bar2\}]
%       The theorem set {\envfont foo2} (printed name \texttt{bar2})
%       uses the same counter as the theorem set \texttt{foo}.
%    \item[\ttfamily \bslash newtheorem\{foo3\}\{bar3\}{[section]}]
%       The theorem set {\envfont foo3} (printed name \texttt{bar3}) is
%       enumerated within the counter \texttt{section}, i.e.\ with every
%       new |\section| the enumeration begins again with 1, and
%       the enumeration is composed from the section-number and the
%       theorem counter itself.
% \end{description}
%
% \DescribeMacro\theoremstyle
% Additionally, the command |\theoremstyle| can define the layout
% of various, or all, theorem sets.  It should be noted that any
% theorem set defined by |\newtheorem| is typeset in the
% |\theoremstyle| that is current at the time of the definition.
% Thus, the following
% \begin{verbatim}
%    \theoremstyle{break}        \newtheorem{Cor}{Corollary}
%    \theoremstyle{plain}        \newtheorem{Exa}{Example}[section]
%\end{verbatim}                             ^^A No blank at line start
% leads to the result that the set {\envfont Cor} is formatted in the
% style \texttt{break}, while the set {\envfont Exa} and all the
% following ones are formatted in the style \texttt{plain}, unless
% another |\theoremstyle| follows.  Since the definitions installed by
% |\newtheorem| are global, one also can limit |\theoremstyle| locally
% by grouping braces.
%
% \DescribeMacro\theorembodyfont
% The choice of the font for the theorem body is completely
% independent of the chosen |\theoremstyle|; this has proven to
% be very advantageous.  For example,
% \begin{verbatim}
%   {\theorembodyfont{\upshape}          \newtheorem{Rem}{Remark}}
%\end{verbatim}
% defines a theorem set {\envfont Rem}, which will be set in
% |\upshape| in the current layout (which in our example is
% \texttt{plain}).  As with |\theoremstyle|, the |\theorembodyfont|
% chosen is that current at the time of |\newtheorem|.  If
% |\theorembodyfont| is not specified or one defines
% |\theorembodyfont{}|, then the font used will be that defined by the
% |\theoremstyle|.
%
% \DescribeMacro\theoremheaderfont It is also possible to customize
% the font used for the theorem headers.  This is, however, a global
% declaration, and therefore there should be at most one
% |\theoremheaderfont| declaration in the preamble.\footnote{If it is
% actually necessary to have different header fonts, one has to define
% new theorem styles (substituting the desired font) or specify the
% information directly in the \texttt{\bslash newtheorem} declaration
% (the unclean variant).}
%
% \DescribeMacro\theorempreskipamount
% \DescribeMacro\theorempostskipamount Two additional parameters
% affect the vertical space around the theorem environments:
% |\theorempreskipamount| and |\theorempostskipamount| define,
% respectively, the spacing before and after such an environment.
% These parameters apply for all theorem sets and can be manipulated
% with the ordinary length macros.  They are rubber lengths
% (`\textsf{skips}'), and therefore can contain \texttt{plus} and
% \texttt{minus} parts.
%
% Since the definition of theorem sets should---most sensibly---be
% placed in the preamble, we only allow installation there.
% It is therefore possible to release the memory used here after
% |\begin{document}|, in order to make room for other
% applications.
%
%
% \subsection{Existing theorem styles}
%
% The following theorem styles exist to date:
% \begin{deflist}{marginbreak}
%    \item[plain]
%       This theorem style emulates the original \LaTeX{} definition,
%       except that additionally the parameters
%       |\theorem...skipamount| are used.
%    \item[break]
%       In this style, the theorem header is followed by a line
%       break.
%    \item[marginbreak]
%       The theorem number is set in the margin, and there is a line
%       break as in \texttt{break}.
%    \item[changebreak]
%       Like \texttt{break}, but with header number and text
%       interchanged.
%    \item[change]
%       Header number and text are interchanged, without a line break.
%    \item[margin]
%       The number is set in the left margin, without a line break.
% \end{deflist}
% All styles (except \texttt{plain}) select |\slshape| as the default
% |\theorembodyfont|.
%
%
% \subsection{Examples}
%
% Given the above theorem sets {\envfont Cor}, {\envfont Exa}
% and {\envfont Rem},
% suppose that the preamble also contains the declarations:
% \begin{verbatim}
%    \theoremstyle{marginbreak}   \newtheorem{Lem}[Cor]{Lemma}
%    \theoremstyle{change}
%    \theorembodyfont{\itshape}        \newtheorem{Def}[Cor]{Definition}
%
%    \theoremheaderfont{\scshape}
%\end{verbatim}
% Then the following are some typical examples of the typeset output
% resulting from their use.
%
% \begin{Cor}
%    This is a sentence typeset in the theorem environment {\envfont
%    Cor}.
% \end{Cor}
%
% \begin{Exa}
%    This is a sentence typeset in the theorem environment {\envfont
%    Exa}.
% \end{Exa}
%
% \begin{Rem}
%    This is a sentence typeset in the theorem environment {\envfont
%    Rem}.
% \end{Rem}
%
% \begin{Lem}[Ben User]
%    This is a sentence typeset in the theorem environment {\envfont
%    Lem}.
% \end{Lem}
%
% \begin{Def}[Very impressive Definition]
%    This is a sentence typeset in the theorem environment {\envfont
%    Def}.
% \end{Def}
%
% The last two examples show the effect of the optional argument to
% a theorem environment (it is the text typeset in parentheses).
%
%
% \section{Special Considerations}
% \label{sec:newfonts}
%
% Theoremheader and body are implemented as a unit. This means that
% the |\theoremheaderfont|  will inherit characteristics of the
% |\theorembodyfont| in \LaTeXe. Thus, if for example
% |\theorembodyfont| is |\itshape| and |\theoremheaderfont|
% is |\bfseries| the font selected for the header will have the
% characteristics `bold extended italic'. If this is not desired one
% should set the |\theoremheaderfont| to something like
% \begin{verbatim}
%      \theoremheaderfont{\normalfont\bfseries}
%\end{verbatim}
% i.e.\ supplying all necessary font information explicitly.
%
%
%
% \section{Acknowledgements}
%
% The publication of this set of macros was only possible with the
% help of Christina Busse (translating the manuscript into English),
% Joachim Pense (playing the r\^ole of typist), Chris Rowley
% (looking everything over) and many others providing useful
% suggestions.
%
% \MaybeStop{%
%    \begin{thebibliography}{9}
%       \bibitem{bk:GMS94} \textsc{M.~Goossens}, \textsc{F.~Mittelbach}
%         and \textsc{A.~Samarin}.
%         \newblock The \LaTeX{} Companion.
%         \newblock
%         Addison-Wesley, Reading, Massachusetts, 1994.
%       \bibitem{latex:source} \textsc{Lamport, Leslie.} \newblock
%                               \texttt{latex.tex}, version 2.09, date
%                               Feb. 1990.
%    \end{thebibliography}
%    \PrintIndex
%    \PrintChanges}
%
%
% \section{The documentation driver file}
%
% The next bit of code contains the documentation driver file for
% \TeX{}, i.e., the file that will produce the documentation you are
% currently reading. It will be extracted from this file by the
% \texttt{docstrip} program. Since it is the first code in the file
% one can alternatively process this file directly with \LaTeXe{} to
% obtain the documentation.
%
%    \begin{macrocode}
%<*driver>
\documentclass{ltxdoc}

\usepackage{theorem}

%<+driver>% The next few lines define theorem sets which are used
%<+driver>% in the example section of the documentation.

\theoremstyle{break}       \newtheorem{Cor}{Corollary}
\theoremstyle{plain}       \newtheorem{Exa}{Example}[section]
{\theorembodyfont{\upshape}\newtheorem{Rem}{Remark}}
\theoremstyle{marginbreak} \newtheorem{Lem}[Cor]{Lemma}
\theoremstyle{change}
\theorembodyfont{\itshape} \newtheorem{Def}[Cor]{Definition}

\theoremheaderfont{\scshape}

\RecordChanges

\begin{document}
   \DocInput{theorem.dtx}
\end{document}
%</driver>
%    \end{macrocode}
%
% \section{Definition of the Macros}
%
% If the file has been loaded before, we abort immediately. If not the
% package announces itself (this is actually done at the very top of
% the file---the way it is done isn't good style so don't copy it).
% \changes{v2.0e}{89/07/19}{Spaces removed from `typeout'.}
% \changes{v2.2b}{95/11/19}{Announce moved to top of file}
%    \begin{macrocode}
%<*package>
%\@ifundefined{theorem@style}{}{\endinput}
%\def\FMithmInfo{1995/11/19 v2.2b Theorem extension package (FMi)}
%\ProvidesPackage{theorem}[\FMithmInfo]
%    \end{macrocode}
%
%
% \subsection{Definition of theorem styles and fonts}
%
% All the definitions in this file are done globally to allow
% inputting this file inside a group.
%
% \begin{macro}{\theoremstyle}
%    Before a theorem style can be installed, the chosen style must be
%    known. For that reason, we must test to see that
%    |\th@|\meta{style} is known or, more precisely, that it is
%    different from |\relax|. If the style is not known then
%    |\th@plain| is used.
%    \begin{macrocode}
\gdef\theoremstyle#1{%
   \@ifundefined{th@#1}{\@warning
          {Unknown theoremstyle `#1'. Using `plain'}%
          \theorem@style{plain}}%
%    \end{macrocode}
%    We save the theorem style to be used in the token register
%    |\theorem@style|.
%    \begin{macrocode}
      {\theorem@style{#1}}%
%    \end{macrocode}
%    Now we ``evaluate'' the theorem style: this means, we call the
%    macro |\th@|\meta{style} which will activate the relevant
%    definitions which are contained in a separate file.  This is done
%    in a group to suppress changes to the current font.  This could
%    otherwise pose problems together with the new font selection
%    scheme\footnote{When I printed the original article using the new
%    font selection scheme I ended with a document with slanted
%    typefaces (text headings and all) simply because one of the
%    theorem styles used \texttt{\bslash sl} at toplevel.} if the
%    |\th@|\meta{style} is evaluated a second time.
% \changes{v2.1a}{90/02/15}{Group added around `th@..}
% \changes{v2.1a}{90/02/15}{Reseting of `..@begintheorem removed.}
%    \begin{macrocode}
      \begingroup
        \csname th@\the\theorem@style \endcsname
      \endgroup}
%    \end{macrocode}
% \end{macro}
%
%
% \begin{macro}{\@begintheorem}
% \changes{v2.1a}{90/02/15}{Relaxed at toplevel.}
% \begin{macro}{\@opargbegintheorem}
% \changes{v2.1a}{90/02/15}{Relaxed at toplevel.}
%    We reset |\@begintheorem| and |\@opargbegintheorem| to
%    |\relax| since these commands are no longer necessary at
%    toplevel. This will save a few tokens.
%    \begin{macrocode}
\global\let\@begintheorem\relax
\global\let\@opargbegintheorem\relax
%    \end{macrocode}
% \end{macro}
% \end{macro}
%
% \begin{macro}{\theorem@style}
%    Obviously the \textsf{token} register used above has to be
%    allocated. To assure the utmost compatibility with the
%    original \LaTeX{} definition, we set the default theorem style to
%    \texttt{plain}, which implements the usual \LaTeX{} convention.
% \changes{v2.0f}{89/07/20}{Unnecessary `global' removed.}
%    \begin{macrocode}
\newtoks\theorem@style
\global\theorem@style{plain}
%    \end{macrocode}
% \end{macro}
%
% \begin{macro}{\theorembodyfont}
% \begin{macro}{\theorem@bodyfont}
%    If the |\theorembodyfont| is set by the user then it should not
%    interact with the default font set in the theorem style. When the
%    new font selection is in force this may happen if, for example,
%    the default is |\itshape| and the new |\theorembodyfont| is
%    |\sffamily|. So we add a |\reset@font| command in front of the
%    user definition.
% \changes{v2.0f}{89/07/20}{Unnecessary `global' removed.}
% \changes{v2.0g}{89/09/19}{`theoremfont to `theorembodyfont.}
% \changes{v2.1c}{92/06/22}{`reset@font added and logic changed.}
% \changes{v2.1d}{92/10/28}{Reset to empty if arg is empty}
%    \begin{macrocode}
\gdef\theorembodyfont#1{%
%    \end{macrocode}
%    We check if the argument supplied is empty and if so put nothing
%    into the |\theorem@bodyfont| token register to allow for
%    |\theorembodyfont{}| as a mean of using the default of the
%    current |\theoremstyle|.
%    \begin{macrocode}
   \def\@tempa{#1}%
   \ifx\@tempa\@empty
    \theorem@bodyfont{}%
   \else
    \theorem@bodyfont{\reset@font#1}%
   \fi
   }
\newtoks\theorem@bodyfont
\global\theorem@bodyfont{}
%    \end{macrocode}
% \end{macro}
% \end{macro}
%
%
% \begin{macro}{\theoremheaderfont}
% \changes{v2.0g}{89/09/19}{Macro added.}
%    The font for the theorem headers is handled differently because
%    this definition applies to all theorem styles.
%    \begin{macrocode}
\gdef\theoremheaderfont#1{\gdef\theorem@headerfont{#1}%
%    \end{macrocode}
%    After using the macro once it is redefined to produce an error
%    message.
%    \begin{macrocode}
       \gdef\theoremheaderfont##1{%
        \typeout{\string\theoremheaderfont\space should be used
                 only once.}}}
%    \end{macrocode}
% \end{macro}
%
% \begin{macro}{\theorem@headerfont}
% \changes{v2.0g}{89/09/19}{Macro added.}
%    To set the |\theorem@headerfont| default we first test if
%    the new fontselection scheme is in force.
% \changes{v2.1a}{90/02/15}{Support for new font selection.}
%    \begin{macrocode}
\ifx\upshape\undefined
%    \end{macrocode}
%    If not we define it to expand into |\bfseries|. We don't use
%    |\let| just in case a following style option redefines this
%    macro.
%    \begin{macrocode}
\gdef\theorem@headerfont{\bfseries}
%    \end{macrocode}
%    Otherwise we reset the current shape before calling |\bfseries|.
%    \begin{macrocode}
\else \gdef\theorem@headerfont{\normalfont\bfseries}\fi
%    \end{macrocode}
% \end{macro}
%
% \begin{macro}{\th@plain}
% \begin{macro}{\th@break}
% \begin{macro}{\th@marginbreak}
% \begin{macro}{\th@changebreak}
% \begin{macro}{\th@change}
% \begin{macro}{\th@margin}
%    The different styles are defined in macros such as
%    |\th@plain|.  Since memory space is precious in
%    ``non-Big-versions'', we have to avoid offering too many unused
%    definitions.  Therefore we define these styles in separate files
%    that can be loaded on demand.  Thus the commands themselves only
%    load these files. We use |\@input@|, a \LaTeXe{} internal command
%    that ensures that the file will be listed with |\listfiles|.
% \changes{v2.2a}{94/02/02}{Use `@input@ to load theorem layout files}
%    \begin{macrocode}
\gdef\th@plain{\@input@{thp.sty}}
\gdef\th@break{\@input@{thb.sty}}
\gdef\th@marginbreak{\@input@{thmb.sty}}
\gdef\th@changebreak{\@input@{thcb.sty}}
\gdef\th@change{\@input@{thc.sty}}
\gdef\th@margin{\@input@{thm.sty}}
%    \end{macrocode}
%    This list will be expanded when new styles become
%    available.  For testing,  just append new theorem substyles
%    as document options.
% \end{macro}
% \end{macro}
% \end{macro}
% \end{macro}
% \end{macro}
% \end{macro}
%
%
%
% \subsection{Definition of a new theorem set}
%
% As already pointed out, a new theorem environment can be defined in
% three different ways:
% \begin{verbatim}
%    \newtheorem{Lem}{Lemma}
%    \newtheorem{Lem}{Lemma}[section]
%    \newtheorem{Lem}[Theorem]{Lemma}
%\end{verbatim}
% The function of the macro |\newtheorem| is to recognize these cases
% and then to branch into one of the three macros |\@ynthm|, |\@xnthm|
% or |\@othm|.  This mechanism is adopted unchanged from
% \cite{latex:source}; the essential point here is that, for example,
% in the second case, the arguments \texttt{Lem}, \texttt{Lemma} and
% \texttt{section} are passed over to the macro |\@xnthm|.
%
% We inspect this case first because the others present fewer problems,
% and thus are easily derived from this one.
%
% \begin{macro}{\@xnthm}
%    For our example arguments, the macro |\@xnthm| must fulfill
%    the following:
%    \begin{itemize}
%       \item
%          Define a new \LaTeX-counter `Lem'
%       \item
%          reset this counter within a |\section|
%       \item
%          define the macro |\theLem|
%       \item
%          define the environment macros |\Lem| and
%          |\endLem| using the current
%          |\theoremstyle| and |\theorem@bodyfont|.
%    \end{itemize}
%    Obviously, all this should happen only if the first argument
%    of |\@xnthm| (i.e.\ \texttt{Lem} in our example) is chosen so
%    as not to conflict with any previously defined commands or
%    environments.  This test is performed by the \LaTeX{} macro
%    |\@ifdefinable|.
%    \begin{macrocode}
\gdef\@xnthm#1#2[#3]{\expandafter\@ifdefinable\csname #1\endcsname
%    \end{macrocode}
%    Therefore, the first argument of |\@ifdefinable| is the
%    expansion (in the example, |\Lem|) of
%    |\csname||#1\endcsname|.  The second argument is
%    executed only if the test has been completed successfully.
%    \begin{macrocode}
   {%
%    \end{macrocode}
%    Now we define the new counter. The names of the \LaTeX{}
%    macros employed should speak for themselves:
% \changes{v2.2b}{1995/10/07}{Use \cs{@newctr}}
% \changes{v2.2c}{1995/11/23}{Use \cs{@newctr} properly.}
%    \begin{macrocode}
    \@definecounter{#1}\@newctr{#1}[#3]%
%    \end{macrocode}
%    Using |\@newctr| will give a proper error message if the counter
%    in |#3| is not defined.
%    In defining `|\theLem|' we must generate the desired macro
%    name by use of |\expandafter| and |\csname|.
%    \begin{macrocode}
    \expandafter\xdef\csname the#1\endcsname
%    \end{macrocode}
%    An |\xdef| is used in order to make the definition global,
%    and to ensure that it contains the replacement texts of
%    |\@thmcountersep| and |\@thmcounter|.\footnote{These
%    two macros can be defined by the document style.  Their default
%    values produce a `\texttt{.}' as separation and an arabic
%    representation of the number.} However, not everything should be
%    expanded. For example, it saves space to use |\thesection|
%    instead of its---at times---lengthy expansion.
%    \begin{macrocode}
      {\expandafter \noexpand \csname the#3\endcsname
       \@thmcountersep \@thmcounter{#1}}%
%    \end{macrocode}
%    Thus with the defaults of \LaTeX{}, |\theLem| would be
%    replaced by the command sequence |\thesection.\arabic{Lem}|.
%
%    We will now look at the definition of the macro which is executed
%    at the beginning of the actual environment (in our example this
%    macro is |\Lem|). It should be noted that we use an
%    ``|\expandafter| trick'' to expand only certain parts of the
%    replacement text at the time of the definition.
%    \begin{macrocode}
    \def\@tempa{\global\@namedef{#1}}%
    \expandafter \@tempa \expandafter{%
%    \end{macrocode}
%    First, the macro that contains the current definitions of
%    |\@begintheorem| and |\@opargtheorem| should be called
%    up.  The name of this macro---as is already known---has the form
%    |\th@|\meta{theorem\ style}; therefore, it must be called by
%    \begin{macrocode}
      \csname th@\the \theorem@style
%    \end{macrocode}
%    In addition the default theorem font should be changeable, i.e.\
%    we have to insert the contents of |\theorem@bodyfont|. For
%    that reason, we expand even further, beyond |\endcsname|,
%    and thus insert the contents of the \textsf{token} register
%    |\theorem@bodyfont| in the replacement text.
%    \begin{macrocode}
            \expandafter \endcsname \the \theorem@bodyfont
%    \end{macrocode}
%    Now it is time to call the macro |\@thm| which takes over
%    the further processing.  It has two arguments: the current
%    counter name (in our example, \texttt{Lem}), and the text of the
%    label (in our example, \texttt{Lemma}).
%    \begin{macrocode}
     \@thm{#1}{#2}}%
%    \end{macrocode}
%    With this, the `sub-definition' is complete.  The macro
%    |\@endtheorem| ends a theorem environment and is, so far, nothing
%    but an |\endtrivlist|.  (Hence it is defined globally, and not
%    within the theorem styles.\footnote{This has to be changed as
%    soon as theorem styles that change \texttt{\bslash @endtheorem}
%    exist.  In such a case, all existing styles must be changed as
%    well since they will have to reset the macro.}) Therefore, we can
%    set it equivalent to the macro that ends the theorem set (in our
%    example, |\endLem|).  However, if some day theorem styles exist
%    that do change |\@endtheorem|, we would have to use the
%    commented-out line instead.
%    \begin{macrocode}
    \global \expandafter \let \csname end#1\endcsname \@endtheorem
%   \global\@namedef{end#1}{\@endtheorem}%
%    \end{macrocode}
%    With these commands all the required definitions are employed,
%    unless the test |\@ifdefinable| has failed.  Therefore, we
%    end the second argument of this macro and with it the definition
%    of |\@xnthm|.
%    \begin{macrocode}
   }}
%    \end{macrocode}
% \end{macro}
%
% \begin{macro}{\@ynthm}
%    The definition of |\@ynthm| is completely analogous. In
%    this case the new counter that is defined is not reset within
%    another counter; thus the definition of |\the...|
%    is simplified:
%    \begin{macrocode}
\gdef\@ynthm#1#2{\expandafter\@ifdefinable\csname #1\endcsname
   {\@definecounter{#1}%
    \expandafter\xdef\csname the#1\endcsname{\@thmcounter{#1}}%
%    \end{macrocode}
%    The rest of the definition corresponds literally to that of
%    |\@xnthm|:
%    \begin{macrocode}
    \def\@tempa{\global\@namedef{#1}}\expandafter \@tempa
     \expandafter{\csname th@\the \theorem@style \expandafter
     \endcsname \the\theorem@bodyfont \@thm{#1}{#2}}%
    \global \expandafter \let \csname end#1\endcsname \@endtheorem}}
%    \end{macrocode}
% \end{macro}
%
% \begin{macro}{\@othm}
%    The definition of |\@othm| does not contain anything new.
%    \begin{macrocode}
\gdef\@othm#1[#2]#3{%
%    \end{macrocode}
% \changes{v2.2b}{95/10/07}{Build in check for counter defined}
%    We do not define a new counter but instead use one that has
%    already been defined.  Thus the only definition we need is that
%    of this pseudo-counter (i.e.\ |\the|\meta{env.\ name}).
%    First we check if |#2| corresponds to a known counter name.
%    \begin{macrocode}
  \expandafter\ifx\csname c@#2\endcsname\relax
   \@nocounterr{#2}%
  \else
   \expandafter\@ifdefinable\csname #1\endcsname
   {\expandafter \xdef \csname the#1\endcsname
     {\expandafter \noexpand \csname the#2\endcsname}%
%    \end{macrocode}
%    All other parts of the definition can be adopted from
%    |\@xnthm|. We have to remember, though, that in this case
%    the name of the current counter and the theorem label have moved
%    to the second and third arguments.
%    \begin{macrocode}
    \def\@tempa{\global\@namedef{#1}}\expandafter \@tempa
     \expandafter{\csname th@\the \theorem@style \expandafter
     \endcsname \the\theorem@bodyfont \@thm{#2}{#3}}%
    \global \expandafter \let \csname end#1\endcsname \@endtheorem}%
  \fi}
%    \end{macrocode}
% \end{macro}
%
% \subsection{Macros that are employed in a {\envfont theorem}
%    environment}
%
% \begin{macro}{\@thm}
%    The macro |\@thm| has to increase the current counter.  Then,
%    depending on whether the environment has (or does not have) an
%    optional argument, it has to branch into either |\@begintheorem|
%    or |\@opargtheorem|.
%    \begin{macrocode}
\gdef\@thm#1#2{\refstepcounter{#1}%
%    \end{macrocode}
%    Now we start a {\envfont trivlist} environment, and give
%    |\@topsep| and |\@topsepadd| the values of the \textsf{skip}
%    registers |\theorempreskipamount| and |\theorempostskipamount|.
%    The value in |\@topsep| is the vertical space that is inserted by
%    the first (and only) |\item| in our |\trivlist| whilst
%    |\@topsepadd| is inserted by |\@endparenv| at the end of that
%    {\envfont trivlist} environment.  By using these registers, we
%    obtain the desired space around a {\envfont theorem} environment.
%    \begin{macrocode}
   \trivlist
   \@topsep \theorempreskipamount               % used by first \item
   \@topsepadd \theorempostskipamount           % used by \@endparenv
%    \end{macrocode}
%    Now we have to test whether an optional argument has been given.
% \changes{v2.0d}{89/07/18}{Forgotten percent added.}
%    \begin{macrocode}
   \@ifnextchar [%
%    \end{macrocode}
%    If there is an optional argument, we will call |\@ythm|, and
%    move the arguments read back into the input stream.
%    \begin{macrocode}
   {\@ythm{#1}{#2}}%
%    \end{macrocode}
%    If not, we call |\@begintheorem|. Its first argument is the
%    name of the theorem set (hence the second argument of
%    |\@thm|). Its second argument is the macro that produces the
%    current number.
%    \begin{macrocode}
   {\@begintheorem{#2}{\csname the#1\endcsname}\ignorespaces}}
%    \end{macrocode}
% \end{macro}
%
% \begin{macro}{\@xthm}
% \begin{macro}{\@ythm}
%    Both these macros were originally called by |\@thm|.  We do
%    not need |\@xthm| anymore, hence we reset it to
%    |\relax|. The definition of |\@ythm| has not changed at
%    all from its definition in \LaTeX. In order to make the macros
%    easier to understand, we will
%    nevertheless present it (commented out).
%    \begin{macrocode}
\global\let\@xthm\relax
% \def\@ythm#1#2[#3]{\@opargbegintheorem{#2}{\csname
%           the#1\endcsname}{#3}\ignorespaces}
%</package>
%    \end{macrocode}
%    The primitive |\ignorespaces| in |\@ythm| and
%    |\@thm| is needed to remove the spaces between the
%    |\begin{...}| and the actual text.
% \end{macro}
% \end{macro}
%
%
% \subsection{Definition of the theorem substyles}
%
% As already pointed out, the theorem substyles, defined below, are
% only loaded when necessary.  Note that all these substyles, except
% \texttt{plain}, have |\slshape| as the default body font.
%
% \subsubsection{The \texttt{plain} style}
%
% \label{dok:th@plain} As the following macros use |@|, we have
% to locally set the |\catcode| of this symbol to ``letter''.
% This happens within a group, so that we do not have to worry about
% which |\catcode| that symbol had before.
%    \begin{macrocode}
%<*thp>
\begingroup \makeatletter
%    \end{macrocode}
% Since we are now within a group, we must make all definitions
% globally.  First we make sure that \texttt{theorem.sty} is loaded.
% This will allow us to use this file as a document style option
% without having to call \texttt{theorem} itself as an option.  At the
% same time, we assure that at least version 2 is loaded, since
% |\theorem@style| was not defined in earlier versions.
%    \begin{macrocode}
\@ifundefined{theorem@style}{\input{theorem.sty}}{}
\ProvidesFile{thp.sty}
             [\FMithmInfo]
%    \end{macrocode}
%
%
% \begin{macro}{\th@plain}
% \begin{macro}{\@begintheorem}
% \begin{macro}{\@opargbegintheorem}
%    |\theoremstyle{plain}| corresponds to the original
%    definition, except that the distances to the surrounding text are
%    determined by the parameters |\theorempreskipamount| and
%    |\theorempostskipamount|.
%    First we set the default body font.
%    \begin{macrocode}
\gdef\th@plain{\normalfont\itshape
%    \end{macrocode}
%    Then we define |\@begintheorem| and
%    |\@opargbegintheorem|. These two macros define how the
%    header of a theorem is typeset.  |\@opargbegintheorem| will
%    be called if a \textsf{theorem} environment with an optional
%    argument is encountered; otherwise, the header is constructed by
%    calling |\@begintheorem|.  If one of these macros is
%    executed, we are within a \textsf{trivlist} environment started by
%    |\@thm|.  So the theorem header is produced with an
%    |\item| command.
%
%    Instead of specifying the header font directly, all standard
%    theorem styles use the |\theorem@headerfont| macro to allow
%    customization.  The extra space (|\labelsep|) is necessary
%    because of problems in the \textsf{trivlist} environment.
%    \begin{macrocode}
  \def\@begintheorem##1##2{%
        \item[\hskip\labelsep \theorem@headerfont ##1\ ##2]}%
%    \end{macrocode}
%    The definition of |\@opargbegintheorem| is completely
%    analogous. The only difference is the fact that there exists a
%    third argument (which is the optional parameter of the
%    environment and contains additional information about the
%    theorem).  Customarily we enclose it in parentheses.
%    \begin{macrocode}
\def\@opargbegintheorem##1##2##3{%
   \item[\hskip\labelsep \theorem@headerfont ##1\ ##2\ (##3)]}}
%    \end{macrocode}
% \end{macro}
% \end{macro}
% \end{macro}
%
% \noindent
% We conclude with an |\endgroup| to restore the
% |\catcode| of |@|.
%    \begin{macrocode}
\endgroup
%</thp>
%    \end{macrocode}
%
%
% \subsubsection{The \texttt{break} style}
%
% This style option is stored in the file |thb.sty|.
% For the next two lines see the documentation for |\th@plain| on
% page~\pageref{dok:th@plain}.
%    \begin{macrocode}
%<*thb>
\begingroup \makeatletter
\@ifundefined{theorem@style}{\input{theorem.sty}}{}
\ProvidesFile{thb.sty}
             [\FMithmInfo]
%    \end{macrocode}
%
% \begin{macro}{\th@break}
%    |\theoremstyle{break}| produces a line break after the name
%    of the theorem.  The font is |\slshape|.  Hence, we define
%    |\th@break| as follows:
%    \begin{macrocode}
\gdef\th@break{\normalfont\slshape
  \def\@begintheorem##1##2{\item[%
%    \end{macrocode}
%    We run into the following problem: it is not possible to create
%    the header with |\item[|\meta{title}|]| and then start
%    a new line by, for example, |\mbox{}\\|. Such a definition
%    will fail whenever a list environment follows immediately.  With
%    the above construction, the |\mbox{}| causes the switch
%    |@inlabel| (cf.\ definition of |\list| and
%    |\trivlist| in \cite{latex:source}) to be set to \texttt{false}
%    and so the following list will insert additional vertical
%    space (|\topskip|). This is quite annoying. Therefore, we
%    create the line break within the |\item|. In order to ensure
%    that the text will begin at the proper position in the following
%    line, we simply pretend that the label does not take any
%    room.\footnote{This will lead to problems whenever very high
%    symbols occurring in the line tower into the
%    heading. So, something else has to be done here sometime.}
%    \begin{macrocode}
       \rlap{\vbox{\hbox{\hskip \labelsep\theorem@headerfont ##1\ ##2}%
                   \hbox{\strut}}}]}%
%    \end{macrocode}
%    Again, the definition of |\@opargbegintheorem| is
%    completely analogous.
%    \begin{macrocode}
\def\@opargbegintheorem##1##2##3{%
  \item[\rlap{\vbox{\hbox{\hskip \labelsep \theorem@headerfont
                     ##1\ ##2\ (##3)}%
                    \hbox{\strut}}}]}}
\endgroup
%</thb>
%    \end{macrocode}
% \end{macro}
%
%
% \subsubsection{The \texttt{changebreak} style}
%
%    \begin{macrocode}
% This style option is stored in the file |thcb.sty|.
%    \begin{macrocode}
%<*thcb>
\begingroup \makeatletter
\@ifundefined{theorem@style}{\input{theorem.sty}}{}
\ProvidesFile{thcb.sty}
             [\FMithmInfo]
%    \end{macrocode}
%
% \begin{macro}{\th@changebreak}
%    The \texttt{change-break} theorem style is like \texttt{break}
%    but with interchange of theorem name and theorem number.  Thus we
%    define |\th@changebreak| as follows:
%    \begin{macrocode}
\gdef\th@changebreak{\normalfont\slshape
  \def\@begintheorem##1##2{\item
       [\rlap{\vbox{\hbox{\hskip\labelsep \theorem@headerfont ##2\ ##1}%
                    \hbox{\strut}}}]}%
\def\@opargbegintheorem##1##2##3{%
  \item[\rlap{\vbox{\hbox{\hskip\labelsep \theorem@headerfont
                     ##2\ ##1\ (##3)}%
                    \hbox{\strut}}}]}}
\endgroup
%</thcb>
%    \end{macrocode}
% \end{macro}
%
%
% \subsubsection{The \texttt{change} style}
%
% This style option is stored in the file |thc.sty|.
%    \begin{macrocode}
%<*thc>
\begingroup \makeatletter
\@ifundefined{theorem@style}{\input{theorem.sty}}{}
\ProvidesFile{thc.sty}
             [\FMithmInfo]
%    \end{macrocode}
%
% \begin{macro}{\th@change}
%    The \texttt{change} theorem style corresponds to the
%    \texttt{change break} style without a linebreak after the header.
%    To say it in another way, it's the same as the \texttt{plain}
%    style but with number and name interchanged and |\slshape| as the
%    default font.
%    \begin{macrocode}
\gdef\th@change{\normalfont\slshape
  \def\@begintheorem##1##2{\item
       [\hskip\labelsep \theorem@headerfont ##2\ ##1]}%
\def\@opargbegintheorem##1##2##3{%
  \item[\hskip\labelsep \theorem@headerfont ##2\ ##1\ (##3)]}}
\endgroup
%</thc>
%    \end{macrocode}
% \end{macro}
%
%
% \subsubsection{The \texttt{marginbreak} style}
%
% This style option is the one used most often at Mainz.  It is saved
% in the file |thmb.sty|.
%    \begin{macrocode}
%<*thmb>
\begingroup \makeatletter
\@ifundefined{theorem@style}{\input{theorem.sty}}{}
\ProvidesFile{thmb.sty}
             [\FMithmInfo]
%    \end{macrocode}
%
% \begin{macro}{\th@marginbreak}
%    The \texttt{margin break} style is nearly the same as the
%    \texttt{change break} style. The only difference is the placement
%    of the theorem number.  We use |\llap| to place it in the left
%    margin.
%
%    In this style |\labelsep| denotes the separation between the
%    number and the text.
%    \begin{macrocode}
\gdef\th@marginbreak{\normalfont\slshape
  \def\@begintheorem##1##2{\item
       [\rlap{\vbox{\theorem@headerfont
                    \hbox{\llap{##2}\hskip\labelsep ##1}%
                    \hbox{\strut}}}]}%
\def\@opargbegintheorem##1##2##3{%
  \item[\rlap{\vbox{\theorem@headerfont
                    \hbox{\llap{##2}\hskip\labelsep ##1\ (##3)}%
                    \hbox{\strut}}}]}}
\endgroup
%</thmb>
%    \end{macrocode}
% \changes{v2.1c}{90/02/17}{`endgroup outside macrocode env.}
% \end{macro}
%
%
%
% \subsubsection{The \texttt{margin} style}
%
% This style option is stored in the file |thm.sty|.
%    \begin{macrocode}
%<*thm>
\begingroup \makeatletter
\@ifundefined{theorem@style}{\input{theorem.sty}}{}
\ProvidesFile{thm.sty}
             [\FMithmInfo]
%    \end{macrocode}
%
% \begin{macro}{\th@margin}
%    Again this is only a variant of the theorem styles described
%    above without any new ideas.
%    \begin{macrocode}
\gdef\th@margin{\normalfont\slshape
  \def\@begintheorem##1##2{\item
       [\theorem@headerfont \llap{##2}\hskip\labelsep ##1]}%
\def\@opargbegintheorem##1##2##3{%
  \item[\theorem@headerfont \llap{##2}\hskip\labelsep ##1\ (##3)]}}
\endgroup
%</thm>
%    \end{macrocode}
% \end{macro}
%
%
%
%
% \subsection{Final Definitions}
%
% \begin{macro}{\theorempreskipamount}
% \begin{macro}{\theorempostskipamount}
%    The \textsf{skip} parameters that regulate the vertical empty space
%    before and after the {\envfont theorem} environment have
%    to be allocated as well.
% \changes{v2.0f}{89/07/20}{Unnecessary `global' removed.}
%    \begin{macrocode}
%<*package>
\newskip\theorempreskipamount
\newskip\theorempostskipamount
%    \end{macrocode}
%    Since we have used the same values for all theorem sets, we now can
%    assign them.
%    \begin{macrocode}
\global\setlength\theorempreskipamount{12pt plus 5pt minus 3pt}
\global\setlength\theorempostskipamount{8pt plus 3pt minus 1.5pt}
%    \end{macrocode}
% \end{macro}
% \end{macro}
%
%
% \begin{macro}{\@endtheorem}
%    The same holds for the macro |\@endtheorem|, which ends a
%    {\envfont theorem} environment. Since it is the same for all
%    theorem sets, it is removed from the macros |\th@|\meta{style}.
%    It simply ends the {\envfont trivlist} environment, which was
%    begun in |\@thm|.
%    \begin{macrocode}
\global\let\@endtheorem=\endtrivlist
%    \end{macrocode}
% \end{macro}
%
%
%
% \begin{macro}{\@preamblecmds}
%    All macros defined above are to be used only in the preamble.
%    Therefore, we insert them in |\@preamblecmds| which will disable
%    them at begin document. This is done by the internal \LaTeXe{}
%    command |\@onlypreamble|.
% \changes{v2.0g}{89/09/19}{`theoremheaderfont and `theorembodyfont
%    added.}
% \changes{v2.2a}{94/02/03}{Use \LaTeXe{} interface}
%    \begin{macrocode}
\@onlypreamble\@xnthm
\@onlypreamble\@ynthm
\@onlypreamble\@othm
\@onlypreamble\newtheorem
\@onlypreamble\theoremstyle
\@onlypreamble\theorembodyfont
\@onlypreamble\theoremheaderfont
%    \end{macrocode}
% \end{macro}
%
% Finally we declare the \texttt{plain} theorem style to be the default.
% \changes{v2.1b}{90/06/04}{Fixed missing `item by declaring default
%    style.}
%    \begin{macrocode}
\theoremstyle{plain}
%</package>
%    \end{macrocode}
%
% \Finale

\endinput
}{}
\ProvidesFile{thcb.sty}
             [\FMithmInfo]
%    \end{macrocode}
%
% \begin{macro}{\th@changebreak}
%    The \texttt{change-break} theorem style is like \texttt{break}
%    but with interchange of theorem name and theorem number.  Thus we
%    define |\th@changebreak| as follows:
%    \begin{macrocode}
\gdef\th@changebreak{\normalfont\slshape
  \def\@begintheorem##1##2{\item
       [\rlap{\vbox{\hbox{\hskip\labelsep \theorem@headerfont ##2\ ##1}%
                    \hbox{\strut}}}]}%
\def\@opargbegintheorem##1##2##3{%
  \item[\rlap{\vbox{\hbox{\hskip\labelsep \theorem@headerfont
                     ##2\ ##1\ (##3)}%
                    \hbox{\strut}}}]}}
\endgroup
%</thcb>
%    \end{macrocode}
% \end{macro}
%
%
% \subsubsection{The \texttt{change} style}
%
% This style option is stored in the file |thc.sty|.
%    \begin{macrocode}
%<*thc>
\begingroup \makeatletter
\@ifundefined{theorem@style}{% \iffalse meta-comment
%
% Copyright (C) 1993-2024
%
% The LaTeX Project and any individual authors listed elsewhere
% in this file.
%
% This file is part of the Standard LaTeX `Tools Bundle'.
% -------------------------------------------------------
%
% It may be distributed and/or modified under the
% conditions of the LaTeX Project Public License, either version 1.3c
% of this license or (at your option) any later version.
% The latest version of this license is in
%    https://www.latex-project.org/lppl.txt
% and version 1.3c or later is part of all distributions of LaTeX
% version 2005/12/01 or later.
%
% The list of all files belonging to the LaTeX `Tools Bundle' is
% given in the file `manifest.txt'.
%
% \fi
%
%%
%
%
\def\FMithmInfo{2023/07/05 v2.2c Theorem extension package (FMi)}
%
% \ProvidesFile{theorem.dtx}[\FMithmInfo]
%\iffalse   % this is a METACOMMENT !
%%
%% Package `theorem' to use with LaTeX2e
%% Copyright (C) 1989-1995 Frank Mittelbach, all rights reserved.
%<+package>\@ifundefined{theorem@style}{}{\endinput}
%<+package>\NeedsTeXFormat{LaTeX2e}[1995/06/01]
%<+package>\ProvidesPackage{theorem}
%<+package>         [\FMithmInfo]
%
% \fi
%
% \changes{v1.0}{87/05/07}{First draft.}
% \changes{v1.0a}{87/05/07}{`thistheoremstyle' added.}
% \changes{v1.1a}{87/07/29}{New concept: redefine the internal
%                          \protect\LaTeX{} commands `@xnthm' etc.}
% \changes{v1.1b}{87/07/31}{`theoremfont' used to specify font.}
% \changes{v1.2a}{88/01/08}{`@endtheorem' now globally defined.}
% \changes{v1.2b}{88/01/11}{`newtheorem' only allowed in the preamble.}
% \changes{v1.3a}{88/02/14}{`theoremfont' now a toks register.}
%
% \changes{v2.0a}{88/11/16}{New implementation and new documentation.}
% \changes{v2.0a}{88/11/16}{No layout changes but internal modifications
%                           old substyles won't work.}
% \changes{v2.0b}{89/01/30}{`global' used for all definitions.}
% \changes{v2.0c}{89/06/15}{Documentation translated to English.
%                            Thanks to LLL and Nina}
% \changes{v2.1a}{90/02/15}{Support for new font selection.}
% \changes{v2.2a}{94/02/03}{Cleanup for LaTeX2e}
%
% \DoNotIndex{\@endparenv,\@ifdefinable,\@ifnextchar,\@ifundefined}
% \DoNotIndex{\@namedef,\@nameuse,\@spaces,\@tempa,\@warning}
% \DoNotIndex{\ }
% \DoNotIndex{\begingroup,\bfseries}
% \DoNotIndex{\csname}
% \DoNotIndex{\def,\do,\docdate}
% \DoNotIndex{\endcsname,\endgroup,\endinput,\endtrivlist,\expandafter}
% \DoNotIndex{\filedate,\fileversion}
% \DoNotIndex{\gdef,\global}
% \DoNotIndex{\hbox,\hskip}
% \DoNotIndex{\ignorespaces,\input,\itshape,\item}
% \DoNotIndex{\labelsep,\let,\llap}
% \DoNotIndex{\makeatletter}
% \DoNotIndex{\noexpand}
% \DoNotIndex{\refstepcounter,\relax,\rlap}
% \DoNotIndex{\setlength,\slshape,\space,\string,\strut}
% \DoNotIndex{\the,\trivlist,\typeout}
% \DoNotIndex{\vbox}
% \DoNotIndex{\xdef}
%
% \def\deflabel#1{\ttfamily #1\hfill}
% \def\deflist#1{\begin{list}{}{\settowidth\labelwidth{\ttfamily #1}%
%                               \setlength\leftmargin\labelwidth
%                               \addtolength\leftmargin\labelsep
%                               \let\makelabel\deflabel}}
% \def\enddeflist{\end{list}}
% \def\envfont{\normalfont\ttfamily}
%
% \MakeShortVerb{\|}
%
% \newcommand{\TUB}{{\sl TUGboat\/}}
%
% \GetFileInfo{theorem.dtx}
% \title{An Extension of the
%        \LaTeX{} theorem environment\thanks{This file
%        has version number \fileversion, last
%        revised \filedate.}}
% \author{Frank Mittelbach}
%
% \MaintainedByLaTeXTeam{tools}
% \maketitle
%
% \begin{abstract}
%    The macros described in this paper yield an extension of the
%    \LaTeX{} theorem mechanism. It is designed to satisfy the
%    different requirements of various journals. Thus, the
%    layout of the ``theorems'' can be manipulated by determining a
%    ``style''. This article describes not only the use, but
%    also the definition, of the necessary macros.
% \end{abstract}
%
% \section*{Preface to version 2.2}
%
% For \LaTeXe{} this package did not need any fundamental changes. I
% only modified the messages generated so that theorem layout styles
% will show up with the |\listfiles| command and cleaned the section
% on the New Font Selection Scheme since this is now included in
% \LaTeX.
%
% \section*{Preface to version 2.1}
%
% This version is identical to 2.0g described in \TUB\ 10\#3 except
% for some internal defaults which are now set depending on the used
% font selection scheme.
%
% This was done to avoid unpleasant surprises if the new font selection
% scheme is in force. For further details see
% section~\ref{sec:newfonts} and \cite{bk:GMS94}.
%
% \section{Introduction}
%
% For our purposes here, ``theorems'' are labelled enunciations,
% often set off from the main text by extra space and a font change.
% Theorems, corollaries, conjectures, definitions, and remarks
% are all instances of ``theorems''.  The ``header'' of these structures
% is composed of a label (such as \textsc{Theorem} or \textsc{Remark})
% and a number which serializes an item in the sequence of items
% with the same label.
%
% Shortly after the introduction of \LaTeX{} at the Fachbereich
% Mathematik in Mainz, the desire to manipulate the layout of
% ``theorems'' arose.  In Mainz, the following two conventions came
% into general use:
% \begin{enumerate}
%   \item The number of the theorem is shown in the margin.
%   \item There is a line break at
%         the end of the theorem header.
% \end{enumerate}
%
% Additionally, some journals require different formats which depend on
% the ``sort of theorem'': e.g.\ often remarks and definitions are set
% in |\upshape|, while |\itshape| is employed for main theorems.
%
% Confronted with these requirements, a theorem environment was
% developed in Mainz which allows separate determination of the
% layout of the ``theorems sets'', comparable to |\pagestyle|.
%
%
% \section{The user interface}
% \subsection{Defining new theorem sets}
%
% \DescribeMacro\newtheorem
% As in the original \LaTeX{} version, the command |\newtheorem|
% defines a new ``theorem set'' or ``theorem-like structure''.
% Two required arguments name the new environment and give the text
% to be typeset with each instance of the new ``set'', while
% an optional argument determines how
% the ``set'' is enumerated:
% \begin{description}
%    \item[\ttfamily \bslash newtheorem\{foo\}\{bar\}]
%       The theorem set {\envfont foo} (whose name is \texttt{bar})
%       uses its own counter.
%    \item[\ttfamily \bslash newtheorem\{foo2\}{[foo]}\{bar2\}]
%       The theorem set {\envfont foo2} (printed name \texttt{bar2})
%       uses the same counter as the theorem set \texttt{foo}.
%    \item[\ttfamily \bslash newtheorem\{foo3\}\{bar3\}{[section]}]
%       The theorem set {\envfont foo3} (printed name \texttt{bar3}) is
%       enumerated within the counter \texttt{section}, i.e.\ with every
%       new |\section| the enumeration begins again with 1, and
%       the enumeration is composed from the section-number and the
%       theorem counter itself.
% \end{description}
%
% \DescribeMacro\theoremstyle
% Additionally, the command |\theoremstyle| can define the layout
% of various, or all, theorem sets.  It should be noted that any
% theorem set defined by |\newtheorem| is typeset in the
% |\theoremstyle| that is current at the time of the definition.
% Thus, the following
% \begin{verbatim}
%    \theoremstyle{break}        \newtheorem{Cor}{Corollary}
%    \theoremstyle{plain}        \newtheorem{Exa}{Example}[section]
%\end{verbatim}                             ^^A No blank at line start
% leads to the result that the set {\envfont Cor} is formatted in the
% style \texttt{break}, while the set {\envfont Exa} and all the
% following ones are formatted in the style \texttt{plain}, unless
% another |\theoremstyle| follows.  Since the definitions installed by
% |\newtheorem| are global, one also can limit |\theoremstyle| locally
% by grouping braces.
%
% \DescribeMacro\theorembodyfont
% The choice of the font for the theorem body is completely
% independent of the chosen |\theoremstyle|; this has proven to
% be very advantageous.  For example,
% \begin{verbatim}
%   {\theorembodyfont{\upshape}          \newtheorem{Rem}{Remark}}
%\end{verbatim}
% defines a theorem set {\envfont Rem}, which will be set in
% |\upshape| in the current layout (which in our example is
% \texttt{plain}).  As with |\theoremstyle|, the |\theorembodyfont|
% chosen is that current at the time of |\newtheorem|.  If
% |\theorembodyfont| is not specified or one defines
% |\theorembodyfont{}|, then the font used will be that defined by the
% |\theoremstyle|.
%
% \DescribeMacro\theoremheaderfont It is also possible to customize
% the font used for the theorem headers.  This is, however, a global
% declaration, and therefore there should be at most one
% |\theoremheaderfont| declaration in the preamble.\footnote{If it is
% actually necessary to have different header fonts, one has to define
% new theorem styles (substituting the desired font) or specify the
% information directly in the \texttt{\bslash newtheorem} declaration
% (the unclean variant).}
%
% \DescribeMacro\theorempreskipamount
% \DescribeMacro\theorempostskipamount Two additional parameters
% affect the vertical space around the theorem environments:
% |\theorempreskipamount| and |\theorempostskipamount| define,
% respectively, the spacing before and after such an environment.
% These parameters apply for all theorem sets and can be manipulated
% with the ordinary length macros.  They are rubber lengths
% (`\textsf{skips}'), and therefore can contain \texttt{plus} and
% \texttt{minus} parts.
%
% Since the definition of theorem sets should---most sensibly---be
% placed in the preamble, we only allow installation there.
% It is therefore possible to release the memory used here after
% |\begin{document}|, in order to make room for other
% applications.
%
%
% \subsection{Existing theorem styles}
%
% The following theorem styles exist to date:
% \begin{deflist}{marginbreak}
%    \item[plain]
%       This theorem style emulates the original \LaTeX{} definition,
%       except that additionally the parameters
%       |\theorem...skipamount| are used.
%    \item[break]
%       In this style, the theorem header is followed by a line
%       break.
%    \item[marginbreak]
%       The theorem number is set in the margin, and there is a line
%       break as in \texttt{break}.
%    \item[changebreak]
%       Like \texttt{break}, but with header number and text
%       interchanged.
%    \item[change]
%       Header number and text are interchanged, without a line break.
%    \item[margin]
%       The number is set in the left margin, without a line break.
% \end{deflist}
% All styles (except \texttt{plain}) select |\slshape| as the default
% |\theorembodyfont|.
%
%
% \subsection{Examples}
%
% Given the above theorem sets {\envfont Cor}, {\envfont Exa}
% and {\envfont Rem},
% suppose that the preamble also contains the declarations:
% \begin{verbatim}
%    \theoremstyle{marginbreak}   \newtheorem{Lem}[Cor]{Lemma}
%    \theoremstyle{change}
%    \theorembodyfont{\itshape}        \newtheorem{Def}[Cor]{Definition}
%
%    \theoremheaderfont{\scshape}
%\end{verbatim}
% Then the following are some typical examples of the typeset output
% resulting from their use.
%
% \begin{Cor}
%    This is a sentence typeset in the theorem environment {\envfont
%    Cor}.
% \end{Cor}
%
% \begin{Exa}
%    This is a sentence typeset in the theorem environment {\envfont
%    Exa}.
% \end{Exa}
%
% \begin{Rem}
%    This is a sentence typeset in the theorem environment {\envfont
%    Rem}.
% \end{Rem}
%
% \begin{Lem}[Ben User]
%    This is a sentence typeset in the theorem environment {\envfont
%    Lem}.
% \end{Lem}
%
% \begin{Def}[Very impressive Definition]
%    This is a sentence typeset in the theorem environment {\envfont
%    Def}.
% \end{Def}
%
% The last two examples show the effect of the optional argument to
% a theorem environment (it is the text typeset in parentheses).
%
%
% \section{Special Considerations}
% \label{sec:newfonts}
%
% Theoremheader and body are implemented as a unit. This means that
% the |\theoremheaderfont|  will inherit characteristics of the
% |\theorembodyfont| in \LaTeXe. Thus, if for example
% |\theorembodyfont| is |\itshape| and |\theoremheaderfont|
% is |\bfseries| the font selected for the header will have the
% characteristics `bold extended italic'. If this is not desired one
% should set the |\theoremheaderfont| to something like
% \begin{verbatim}
%      \theoremheaderfont{\normalfont\bfseries}
%\end{verbatim}
% i.e.\ supplying all necessary font information explicitly.
%
%
%
% \section{Acknowledgements}
%
% The publication of this set of macros was only possible with the
% help of Christina Busse (translating the manuscript into English),
% Joachim Pense (playing the r\^ole of typist), Chris Rowley
% (looking everything over) and many others providing useful
% suggestions.
%
% \MaybeStop{%
%    \begin{thebibliography}{9}
%       \bibitem{bk:GMS94} \textsc{M.~Goossens}, \textsc{F.~Mittelbach}
%         and \textsc{A.~Samarin}.
%         \newblock The \LaTeX{} Companion.
%         \newblock
%         Addison-Wesley, Reading, Massachusetts, 1994.
%       \bibitem{latex:source} \textsc{Lamport, Leslie.} \newblock
%                               \texttt{latex.tex}, version 2.09, date
%                               Feb. 1990.
%    \end{thebibliography}
%    \PrintIndex
%    \PrintChanges}
%
%
% \section{The documentation driver file}
%
% The next bit of code contains the documentation driver file for
% \TeX{}, i.e., the file that will produce the documentation you are
% currently reading. It will be extracted from this file by the
% \texttt{docstrip} program. Since it is the first code in the file
% one can alternatively process this file directly with \LaTeXe{} to
% obtain the documentation.
%
%    \begin{macrocode}
%<*driver>
\documentclass{ltxdoc}

\usepackage{theorem}

%<+driver>% The next few lines define theorem sets which are used
%<+driver>% in the example section of the documentation.

\theoremstyle{break}       \newtheorem{Cor}{Corollary}
\theoremstyle{plain}       \newtheorem{Exa}{Example}[section]
{\theorembodyfont{\upshape}\newtheorem{Rem}{Remark}}
\theoremstyle{marginbreak} \newtheorem{Lem}[Cor]{Lemma}
\theoremstyle{change}
\theorembodyfont{\itshape} \newtheorem{Def}[Cor]{Definition}

\theoremheaderfont{\scshape}

\RecordChanges

\begin{document}
   \DocInput{theorem.dtx}
\end{document}
%</driver>
%    \end{macrocode}
%
% \section{Definition of the Macros}
%
% If the file has been loaded before, we abort immediately. If not the
% package announces itself (this is actually done at the very top of
% the file---the way it is done isn't good style so don't copy it).
% \changes{v2.0e}{89/07/19}{Spaces removed from `typeout'.}
% \changes{v2.2b}{95/11/19}{Announce moved to top of file}
%    \begin{macrocode}
%<*package>
%\@ifundefined{theorem@style}{}{\endinput}
%\def\FMithmInfo{1995/11/19 v2.2b Theorem extension package (FMi)}
%\ProvidesPackage{theorem}[\FMithmInfo]
%    \end{macrocode}
%
%
% \subsection{Definition of theorem styles and fonts}
%
% All the definitions in this file are done globally to allow
% inputting this file inside a group.
%
% \begin{macro}{\theoremstyle}
%    Before a theorem style can be installed, the chosen style must be
%    known. For that reason, we must test to see that
%    |\th@|\meta{style} is known or, more precisely, that it is
%    different from |\relax|. If the style is not known then
%    |\th@plain| is used.
%    \begin{macrocode}
\gdef\theoremstyle#1{%
   \@ifundefined{th@#1}{\@warning
          {Unknown theoremstyle `#1'. Using `plain'}%
          \theorem@style{plain}}%
%    \end{macrocode}
%    We save the theorem style to be used in the token register
%    |\theorem@style|.
%    \begin{macrocode}
      {\theorem@style{#1}}%
%    \end{macrocode}
%    Now we ``evaluate'' the theorem style: this means, we call the
%    macro |\th@|\meta{style} which will activate the relevant
%    definitions which are contained in a separate file.  This is done
%    in a group to suppress changes to the current font.  This could
%    otherwise pose problems together with the new font selection
%    scheme\footnote{When I printed the original article using the new
%    font selection scheme I ended with a document with slanted
%    typefaces (text headings and all) simply because one of the
%    theorem styles used \texttt{\bslash sl} at toplevel.} if the
%    |\th@|\meta{style} is evaluated a second time.
% \changes{v2.1a}{90/02/15}{Group added around `th@..}
% \changes{v2.1a}{90/02/15}{Reseting of `..@begintheorem removed.}
%    \begin{macrocode}
      \begingroup
        \csname th@\the\theorem@style \endcsname
      \endgroup}
%    \end{macrocode}
% \end{macro}
%
%
% \begin{macro}{\@begintheorem}
% \changes{v2.1a}{90/02/15}{Relaxed at toplevel.}
% \begin{macro}{\@opargbegintheorem}
% \changes{v2.1a}{90/02/15}{Relaxed at toplevel.}
%    We reset |\@begintheorem| and |\@opargbegintheorem| to
%    |\relax| since these commands are no longer necessary at
%    toplevel. This will save a few tokens.
%    \begin{macrocode}
\global\let\@begintheorem\relax
\global\let\@opargbegintheorem\relax
%    \end{macrocode}
% \end{macro}
% \end{macro}
%
% \begin{macro}{\theorem@style}
%    Obviously the \textsf{token} register used above has to be
%    allocated. To assure the utmost compatibility with the
%    original \LaTeX{} definition, we set the default theorem style to
%    \texttt{plain}, which implements the usual \LaTeX{} convention.
% \changes{v2.0f}{89/07/20}{Unnecessary `global' removed.}
%    \begin{macrocode}
\newtoks\theorem@style
\global\theorem@style{plain}
%    \end{macrocode}
% \end{macro}
%
% \begin{macro}{\theorembodyfont}
% \begin{macro}{\theorem@bodyfont}
%    If the |\theorembodyfont| is set by the user then it should not
%    interact with the default font set in the theorem style. When the
%    new font selection is in force this may happen if, for example,
%    the default is |\itshape| and the new |\theorembodyfont| is
%    |\sffamily|. So we add a |\reset@font| command in front of the
%    user definition.
% \changes{v2.0f}{89/07/20}{Unnecessary `global' removed.}
% \changes{v2.0g}{89/09/19}{`theoremfont to `theorembodyfont.}
% \changes{v2.1c}{92/06/22}{`reset@font added and logic changed.}
% \changes{v2.1d}{92/10/28}{Reset to empty if arg is empty}
%    \begin{macrocode}
\gdef\theorembodyfont#1{%
%    \end{macrocode}
%    We check if the argument supplied is empty and if so put nothing
%    into the |\theorem@bodyfont| token register to allow for
%    |\theorembodyfont{}| as a mean of using the default of the
%    current |\theoremstyle|.
%    \begin{macrocode}
   \def\@tempa{#1}%
   \ifx\@tempa\@empty
    \theorem@bodyfont{}%
   \else
    \theorem@bodyfont{\reset@font#1}%
   \fi
   }
\newtoks\theorem@bodyfont
\global\theorem@bodyfont{}
%    \end{macrocode}
% \end{macro}
% \end{macro}
%
%
% \begin{macro}{\theoremheaderfont}
% \changes{v2.0g}{89/09/19}{Macro added.}
%    The font for the theorem headers is handled differently because
%    this definition applies to all theorem styles.
%    \begin{macrocode}
\gdef\theoremheaderfont#1{\gdef\theorem@headerfont{#1}%
%    \end{macrocode}
%    After using the macro once it is redefined to produce an error
%    message.
%    \begin{macrocode}
       \gdef\theoremheaderfont##1{%
        \typeout{\string\theoremheaderfont\space should be used
                 only once.}}}
%    \end{macrocode}
% \end{macro}
%
% \begin{macro}{\theorem@headerfont}
% \changes{v2.0g}{89/09/19}{Macro added.}
%    To set the |\theorem@headerfont| default we first test if
%    the new fontselection scheme is in force.
% \changes{v2.1a}{90/02/15}{Support for new font selection.}
%    \begin{macrocode}
\ifx\upshape\undefined
%    \end{macrocode}
%    If not we define it to expand into |\bfseries|. We don't use
%    |\let| just in case a following style option redefines this
%    macro.
%    \begin{macrocode}
\gdef\theorem@headerfont{\bfseries}
%    \end{macrocode}
%    Otherwise we reset the current shape before calling |\bfseries|.
%    \begin{macrocode}
\else \gdef\theorem@headerfont{\normalfont\bfseries}\fi
%    \end{macrocode}
% \end{macro}
%
% \begin{macro}{\th@plain}
% \begin{macro}{\th@break}
% \begin{macro}{\th@marginbreak}
% \begin{macro}{\th@changebreak}
% \begin{macro}{\th@change}
% \begin{macro}{\th@margin}
%    The different styles are defined in macros such as
%    |\th@plain|.  Since memory space is precious in
%    ``non-Big-versions'', we have to avoid offering too many unused
%    definitions.  Therefore we define these styles in separate files
%    that can be loaded on demand.  Thus the commands themselves only
%    load these files. We use |\@input@|, a \LaTeXe{} internal command
%    that ensures that the file will be listed with |\listfiles|.
% \changes{v2.2a}{94/02/02}{Use `@input@ to load theorem layout files}
%    \begin{macrocode}
\gdef\th@plain{\@input@{thp.sty}}
\gdef\th@break{\@input@{thb.sty}}
\gdef\th@marginbreak{\@input@{thmb.sty}}
\gdef\th@changebreak{\@input@{thcb.sty}}
\gdef\th@change{\@input@{thc.sty}}
\gdef\th@margin{\@input@{thm.sty}}
%    \end{macrocode}
%    This list will be expanded when new styles become
%    available.  For testing,  just append new theorem substyles
%    as document options.
% \end{macro}
% \end{macro}
% \end{macro}
% \end{macro}
% \end{macro}
% \end{macro}
%
%
%
% \subsection{Definition of a new theorem set}
%
% As already pointed out, a new theorem environment can be defined in
% three different ways:
% \begin{verbatim}
%    \newtheorem{Lem}{Lemma}
%    \newtheorem{Lem}{Lemma}[section]
%    \newtheorem{Lem}[Theorem]{Lemma}
%\end{verbatim}
% The function of the macro |\newtheorem| is to recognize these cases
% and then to branch into one of the three macros |\@ynthm|, |\@xnthm|
% or |\@othm|.  This mechanism is adopted unchanged from
% \cite{latex:source}; the essential point here is that, for example,
% in the second case, the arguments \texttt{Lem}, \texttt{Lemma} and
% \texttt{section} are passed over to the macro |\@xnthm|.
%
% We inspect this case first because the others present fewer problems,
% and thus are easily derived from this one.
%
% \begin{macro}{\@xnthm}
%    For our example arguments, the macro |\@xnthm| must fulfill
%    the following:
%    \begin{itemize}
%       \item
%          Define a new \LaTeX-counter `Lem'
%       \item
%          reset this counter within a |\section|
%       \item
%          define the macro |\theLem|
%       \item
%          define the environment macros |\Lem| and
%          |\endLem| using the current
%          |\theoremstyle| and |\theorem@bodyfont|.
%    \end{itemize}
%    Obviously, all this should happen only if the first argument
%    of |\@xnthm| (i.e.\ \texttt{Lem} in our example) is chosen so
%    as not to conflict with any previously defined commands or
%    environments.  This test is performed by the \LaTeX{} macro
%    |\@ifdefinable|.
%    \begin{macrocode}
\gdef\@xnthm#1#2[#3]{\expandafter\@ifdefinable\csname #1\endcsname
%    \end{macrocode}
%    Therefore, the first argument of |\@ifdefinable| is the
%    expansion (in the example, |\Lem|) of
%    |\csname||#1\endcsname|.  The second argument is
%    executed only if the test has been completed successfully.
%    \begin{macrocode}
   {%
%    \end{macrocode}
%    Now we define the new counter. The names of the \LaTeX{}
%    macros employed should speak for themselves:
% \changes{v2.2b}{1995/10/07}{Use \cs{@newctr}}
% \changes{v2.2c}{1995/11/23}{Use \cs{@newctr} properly.}
%    \begin{macrocode}
    \@definecounter{#1}\@newctr{#1}[#3]%
%    \end{macrocode}
%    Using |\@newctr| will give a proper error message if the counter
%    in |#3| is not defined.
%    In defining `|\theLem|' we must generate the desired macro
%    name by use of |\expandafter| and |\csname|.
%    \begin{macrocode}
    \expandafter\xdef\csname the#1\endcsname
%    \end{macrocode}
%    An |\xdef| is used in order to make the definition global,
%    and to ensure that it contains the replacement texts of
%    |\@thmcountersep| and |\@thmcounter|.\footnote{These
%    two macros can be defined by the document style.  Their default
%    values produce a `\texttt{.}' as separation and an arabic
%    representation of the number.} However, not everything should be
%    expanded. For example, it saves space to use |\thesection|
%    instead of its---at times---lengthy expansion.
%    \begin{macrocode}
      {\expandafter \noexpand \csname the#3\endcsname
       \@thmcountersep \@thmcounter{#1}}%
%    \end{macrocode}
%    Thus with the defaults of \LaTeX{}, |\theLem| would be
%    replaced by the command sequence |\thesection.\arabic{Lem}|.
%
%    We will now look at the definition of the macro which is executed
%    at the beginning of the actual environment (in our example this
%    macro is |\Lem|). It should be noted that we use an
%    ``|\expandafter| trick'' to expand only certain parts of the
%    replacement text at the time of the definition.
%    \begin{macrocode}
    \def\@tempa{\global\@namedef{#1}}%
    \expandafter \@tempa \expandafter{%
%    \end{macrocode}
%    First, the macro that contains the current definitions of
%    |\@begintheorem| and |\@opargtheorem| should be called
%    up.  The name of this macro---as is already known---has the form
%    |\th@|\meta{theorem\ style}; therefore, it must be called by
%    \begin{macrocode}
      \csname th@\the \theorem@style
%    \end{macrocode}
%    In addition the default theorem font should be changeable, i.e.\
%    we have to insert the contents of |\theorem@bodyfont|. For
%    that reason, we expand even further, beyond |\endcsname|,
%    and thus insert the contents of the \textsf{token} register
%    |\theorem@bodyfont| in the replacement text.
%    \begin{macrocode}
            \expandafter \endcsname \the \theorem@bodyfont
%    \end{macrocode}
%    Now it is time to call the macro |\@thm| which takes over
%    the further processing.  It has two arguments: the current
%    counter name (in our example, \texttt{Lem}), and the text of the
%    label (in our example, \texttt{Lemma}).
%    \begin{macrocode}
     \@thm{#1}{#2}}%
%    \end{macrocode}
%    With this, the `sub-definition' is complete.  The macro
%    |\@endtheorem| ends a theorem environment and is, so far, nothing
%    but an |\endtrivlist|.  (Hence it is defined globally, and not
%    within the theorem styles.\footnote{This has to be changed as
%    soon as theorem styles that change \texttt{\bslash @endtheorem}
%    exist.  In such a case, all existing styles must be changed as
%    well since they will have to reset the macro.}) Therefore, we can
%    set it equivalent to the macro that ends the theorem set (in our
%    example, |\endLem|).  However, if some day theorem styles exist
%    that do change |\@endtheorem|, we would have to use the
%    commented-out line instead.
%    \begin{macrocode}
    \global \expandafter \let \csname end#1\endcsname \@endtheorem
%   \global\@namedef{end#1}{\@endtheorem}%
%    \end{macrocode}
%    With these commands all the required definitions are employed,
%    unless the test |\@ifdefinable| has failed.  Therefore, we
%    end the second argument of this macro and with it the definition
%    of |\@xnthm|.
%    \begin{macrocode}
   }}
%    \end{macrocode}
% \end{macro}
%
% \begin{macro}{\@ynthm}
%    The definition of |\@ynthm| is completely analogous. In
%    this case the new counter that is defined is not reset within
%    another counter; thus the definition of |\the...|
%    is simplified:
%    \begin{macrocode}
\gdef\@ynthm#1#2{\expandafter\@ifdefinable\csname #1\endcsname
   {\@definecounter{#1}%
    \expandafter\xdef\csname the#1\endcsname{\@thmcounter{#1}}%
%    \end{macrocode}
%    The rest of the definition corresponds literally to that of
%    |\@xnthm|:
%    \begin{macrocode}
    \def\@tempa{\global\@namedef{#1}}\expandafter \@tempa
     \expandafter{\csname th@\the \theorem@style \expandafter
     \endcsname \the\theorem@bodyfont \@thm{#1}{#2}}%
    \global \expandafter \let \csname end#1\endcsname \@endtheorem}}
%    \end{macrocode}
% \end{macro}
%
% \begin{macro}{\@othm}
%    The definition of |\@othm| does not contain anything new.
%    \begin{macrocode}
\gdef\@othm#1[#2]#3{%
%    \end{macrocode}
% \changes{v2.2b}{95/10/07}{Build in check for counter defined}
%    We do not define a new counter but instead use one that has
%    already been defined.  Thus the only definition we need is that
%    of this pseudo-counter (i.e.\ |\the|\meta{env.\ name}).
%    First we check if |#2| corresponds to a known counter name.
%    \begin{macrocode}
  \expandafter\ifx\csname c@#2\endcsname\relax
   \@nocounterr{#2}%
  \else
   \expandafter\@ifdefinable\csname #1\endcsname
   {\expandafter \xdef \csname the#1\endcsname
     {\expandafter \noexpand \csname the#2\endcsname}%
%    \end{macrocode}
%    All other parts of the definition can be adopted from
%    |\@xnthm|. We have to remember, though, that in this case
%    the name of the current counter and the theorem label have moved
%    to the second and third arguments.
%    \begin{macrocode}
    \def\@tempa{\global\@namedef{#1}}\expandafter \@tempa
     \expandafter{\csname th@\the \theorem@style \expandafter
     \endcsname \the\theorem@bodyfont \@thm{#2}{#3}}%
    \global \expandafter \let \csname end#1\endcsname \@endtheorem}%
  \fi}
%    \end{macrocode}
% \end{macro}
%
% \subsection{Macros that are employed in a {\envfont theorem}
%    environment}
%
% \begin{macro}{\@thm}
%    The macro |\@thm| has to increase the current counter.  Then,
%    depending on whether the environment has (or does not have) an
%    optional argument, it has to branch into either |\@begintheorem|
%    or |\@opargtheorem|.
%    \begin{macrocode}
\gdef\@thm#1#2{\refstepcounter{#1}%
%    \end{macrocode}
%    Now we start a {\envfont trivlist} environment, and give
%    |\@topsep| and |\@topsepadd| the values of the \textsf{skip}
%    registers |\theorempreskipamount| and |\theorempostskipamount|.
%    The value in |\@topsep| is the vertical space that is inserted by
%    the first (and only) |\item| in our |\trivlist| whilst
%    |\@topsepadd| is inserted by |\@endparenv| at the end of that
%    {\envfont trivlist} environment.  By using these registers, we
%    obtain the desired space around a {\envfont theorem} environment.
%    \begin{macrocode}
   \trivlist
   \@topsep \theorempreskipamount               % used by first \item
   \@topsepadd \theorempostskipamount           % used by \@endparenv
%    \end{macrocode}
%    Now we have to test whether an optional argument has been given.
% \changes{v2.0d}{89/07/18}{Forgotten percent added.}
%    \begin{macrocode}
   \@ifnextchar [%
%    \end{macrocode}
%    If there is an optional argument, we will call |\@ythm|, and
%    move the arguments read back into the input stream.
%    \begin{macrocode}
   {\@ythm{#1}{#2}}%
%    \end{macrocode}
%    If not, we call |\@begintheorem|. Its first argument is the
%    name of the theorem set (hence the second argument of
%    |\@thm|). Its second argument is the macro that produces the
%    current number.
%    \begin{macrocode}
   {\@begintheorem{#2}{\csname the#1\endcsname}\ignorespaces}}
%    \end{macrocode}
% \end{macro}
%
% \begin{macro}{\@xthm}
% \begin{macro}{\@ythm}
%    Both these macros were originally called by |\@thm|.  We do
%    not need |\@xthm| anymore, hence we reset it to
%    |\relax|. The definition of |\@ythm| has not changed at
%    all from its definition in \LaTeX. In order to make the macros
%    easier to understand, we will
%    nevertheless present it (commented out).
%    \begin{macrocode}
\global\let\@xthm\relax
% \def\@ythm#1#2[#3]{\@opargbegintheorem{#2}{\csname
%           the#1\endcsname}{#3}\ignorespaces}
%</package>
%    \end{macrocode}
%    The primitive |\ignorespaces| in |\@ythm| and
%    |\@thm| is needed to remove the spaces between the
%    |\begin{...}| and the actual text.
% \end{macro}
% \end{macro}
%
%
% \subsection{Definition of the theorem substyles}
%
% As already pointed out, the theorem substyles, defined below, are
% only loaded when necessary.  Note that all these substyles, except
% \texttt{plain}, have |\slshape| as the default body font.
%
% \subsubsection{The \texttt{plain} style}
%
% \label{dok:th@plain} As the following macros use |@|, we have
% to locally set the |\catcode| of this symbol to ``letter''.
% This happens within a group, so that we do not have to worry about
% which |\catcode| that symbol had before.
%    \begin{macrocode}
%<*thp>
\begingroup \makeatletter
%    \end{macrocode}
% Since we are now within a group, we must make all definitions
% globally.  First we make sure that \texttt{theorem.sty} is loaded.
% This will allow us to use this file as a document style option
% without having to call \texttt{theorem} itself as an option.  At the
% same time, we assure that at least version 2 is loaded, since
% |\theorem@style| was not defined in earlier versions.
%    \begin{macrocode}
\@ifundefined{theorem@style}{\input{theorem.sty}}{}
\ProvidesFile{thp.sty}
             [\FMithmInfo]
%    \end{macrocode}
%
%
% \begin{macro}{\th@plain}
% \begin{macro}{\@begintheorem}
% \begin{macro}{\@opargbegintheorem}
%    |\theoremstyle{plain}| corresponds to the original
%    definition, except that the distances to the surrounding text are
%    determined by the parameters |\theorempreskipamount| and
%    |\theorempostskipamount|.
%    First we set the default body font.
%    \begin{macrocode}
\gdef\th@plain{\normalfont\itshape
%    \end{macrocode}
%    Then we define |\@begintheorem| and
%    |\@opargbegintheorem|. These two macros define how the
%    header of a theorem is typeset.  |\@opargbegintheorem| will
%    be called if a \textsf{theorem} environment with an optional
%    argument is encountered; otherwise, the header is constructed by
%    calling |\@begintheorem|.  If one of these macros is
%    executed, we are within a \textsf{trivlist} environment started by
%    |\@thm|.  So the theorem header is produced with an
%    |\item| command.
%
%    Instead of specifying the header font directly, all standard
%    theorem styles use the |\theorem@headerfont| macro to allow
%    customization.  The extra space (|\labelsep|) is necessary
%    because of problems in the \textsf{trivlist} environment.
%    \begin{macrocode}
  \def\@begintheorem##1##2{%
        \item[\hskip\labelsep \theorem@headerfont ##1\ ##2]}%
%    \end{macrocode}
%    The definition of |\@opargbegintheorem| is completely
%    analogous. The only difference is the fact that there exists a
%    third argument (which is the optional parameter of the
%    environment and contains additional information about the
%    theorem).  Customarily we enclose it in parentheses.
%    \begin{macrocode}
\def\@opargbegintheorem##1##2##3{%
   \item[\hskip\labelsep \theorem@headerfont ##1\ ##2\ (##3)]}}
%    \end{macrocode}
% \end{macro}
% \end{macro}
% \end{macro}
%
% \noindent
% We conclude with an |\endgroup| to restore the
% |\catcode| of |@|.
%    \begin{macrocode}
\endgroup
%</thp>
%    \end{macrocode}
%
%
% \subsubsection{The \texttt{break} style}
%
% This style option is stored in the file |thb.sty|.
% For the next two lines see the documentation for |\th@plain| on
% page~\pageref{dok:th@plain}.
%    \begin{macrocode}
%<*thb>
\begingroup \makeatletter
\@ifundefined{theorem@style}{\input{theorem.sty}}{}
\ProvidesFile{thb.sty}
             [\FMithmInfo]
%    \end{macrocode}
%
% \begin{macro}{\th@break}
%    |\theoremstyle{break}| produces a line break after the name
%    of the theorem.  The font is |\slshape|.  Hence, we define
%    |\th@break| as follows:
%    \begin{macrocode}
\gdef\th@break{\normalfont\slshape
  \def\@begintheorem##1##2{\item[%
%    \end{macrocode}
%    We run into the following problem: it is not possible to create
%    the header with |\item[|\meta{title}|]| and then start
%    a new line by, for example, |\mbox{}\\|. Such a definition
%    will fail whenever a list environment follows immediately.  With
%    the above construction, the |\mbox{}| causes the switch
%    |@inlabel| (cf.\ definition of |\list| and
%    |\trivlist| in \cite{latex:source}) to be set to \texttt{false}
%    and so the following list will insert additional vertical
%    space (|\topskip|). This is quite annoying. Therefore, we
%    create the line break within the |\item|. In order to ensure
%    that the text will begin at the proper position in the following
%    line, we simply pretend that the label does not take any
%    room.\footnote{This will lead to problems whenever very high
%    symbols occurring in the line tower into the
%    heading. So, something else has to be done here sometime.}
%    \begin{macrocode}
       \rlap{\vbox{\hbox{\hskip \labelsep\theorem@headerfont ##1\ ##2}%
                   \hbox{\strut}}}]}%
%    \end{macrocode}
%    Again, the definition of |\@opargbegintheorem| is
%    completely analogous.
%    \begin{macrocode}
\def\@opargbegintheorem##1##2##3{%
  \item[\rlap{\vbox{\hbox{\hskip \labelsep \theorem@headerfont
                     ##1\ ##2\ (##3)}%
                    \hbox{\strut}}}]}}
\endgroup
%</thb>
%    \end{macrocode}
% \end{macro}
%
%
% \subsubsection{The \texttt{changebreak} style}
%
%    \begin{macrocode}
% This style option is stored in the file |thcb.sty|.
%    \begin{macrocode}
%<*thcb>
\begingroup \makeatletter
\@ifundefined{theorem@style}{\input{theorem.sty}}{}
\ProvidesFile{thcb.sty}
             [\FMithmInfo]
%    \end{macrocode}
%
% \begin{macro}{\th@changebreak}
%    The \texttt{change-break} theorem style is like \texttt{break}
%    but with interchange of theorem name and theorem number.  Thus we
%    define |\th@changebreak| as follows:
%    \begin{macrocode}
\gdef\th@changebreak{\normalfont\slshape
  \def\@begintheorem##1##2{\item
       [\rlap{\vbox{\hbox{\hskip\labelsep \theorem@headerfont ##2\ ##1}%
                    \hbox{\strut}}}]}%
\def\@opargbegintheorem##1##2##3{%
  \item[\rlap{\vbox{\hbox{\hskip\labelsep \theorem@headerfont
                     ##2\ ##1\ (##3)}%
                    \hbox{\strut}}}]}}
\endgroup
%</thcb>
%    \end{macrocode}
% \end{macro}
%
%
% \subsubsection{The \texttt{change} style}
%
% This style option is stored in the file |thc.sty|.
%    \begin{macrocode}
%<*thc>
\begingroup \makeatletter
\@ifundefined{theorem@style}{\input{theorem.sty}}{}
\ProvidesFile{thc.sty}
             [\FMithmInfo]
%    \end{macrocode}
%
% \begin{macro}{\th@change}
%    The \texttt{change} theorem style corresponds to the
%    \texttt{change break} style without a linebreak after the header.
%    To say it in another way, it's the same as the \texttt{plain}
%    style but with number and name interchanged and |\slshape| as the
%    default font.
%    \begin{macrocode}
\gdef\th@change{\normalfont\slshape
  \def\@begintheorem##1##2{\item
       [\hskip\labelsep \theorem@headerfont ##2\ ##1]}%
\def\@opargbegintheorem##1##2##3{%
  \item[\hskip\labelsep \theorem@headerfont ##2\ ##1\ (##3)]}}
\endgroup
%</thc>
%    \end{macrocode}
% \end{macro}
%
%
% \subsubsection{The \texttt{marginbreak} style}
%
% This style option is the one used most often at Mainz.  It is saved
% in the file |thmb.sty|.
%    \begin{macrocode}
%<*thmb>
\begingroup \makeatletter
\@ifundefined{theorem@style}{\input{theorem.sty}}{}
\ProvidesFile{thmb.sty}
             [\FMithmInfo]
%    \end{macrocode}
%
% \begin{macro}{\th@marginbreak}
%    The \texttt{margin break} style is nearly the same as the
%    \texttt{change break} style. The only difference is the placement
%    of the theorem number.  We use |\llap| to place it in the left
%    margin.
%
%    In this style |\labelsep| denotes the separation between the
%    number and the text.
%    \begin{macrocode}
\gdef\th@marginbreak{\normalfont\slshape
  \def\@begintheorem##1##2{\item
       [\rlap{\vbox{\theorem@headerfont
                    \hbox{\llap{##2}\hskip\labelsep ##1}%
                    \hbox{\strut}}}]}%
\def\@opargbegintheorem##1##2##3{%
  \item[\rlap{\vbox{\theorem@headerfont
                    \hbox{\llap{##2}\hskip\labelsep ##1\ (##3)}%
                    \hbox{\strut}}}]}}
\endgroup
%</thmb>
%    \end{macrocode}
% \changes{v2.1c}{90/02/17}{`endgroup outside macrocode env.}
% \end{macro}
%
%
%
% \subsubsection{The \texttt{margin} style}
%
% This style option is stored in the file |thm.sty|.
%    \begin{macrocode}
%<*thm>
\begingroup \makeatletter
\@ifundefined{theorem@style}{\input{theorem.sty}}{}
\ProvidesFile{thm.sty}
             [\FMithmInfo]
%    \end{macrocode}
%
% \begin{macro}{\th@margin}
%    Again this is only a variant of the theorem styles described
%    above without any new ideas.
%    \begin{macrocode}
\gdef\th@margin{\normalfont\slshape
  \def\@begintheorem##1##2{\item
       [\theorem@headerfont \llap{##2}\hskip\labelsep ##1]}%
\def\@opargbegintheorem##1##2##3{%
  \item[\theorem@headerfont \llap{##2}\hskip\labelsep ##1\ (##3)]}}
\endgroup
%</thm>
%    \end{macrocode}
% \end{macro}
%
%
%
%
% \subsection{Final Definitions}
%
% \begin{macro}{\theorempreskipamount}
% \begin{macro}{\theorempostskipamount}
%    The \textsf{skip} parameters that regulate the vertical empty space
%    before and after the {\envfont theorem} environment have
%    to be allocated as well.
% \changes{v2.0f}{89/07/20}{Unnecessary `global' removed.}
%    \begin{macrocode}
%<*package>
\newskip\theorempreskipamount
\newskip\theorempostskipamount
%    \end{macrocode}
%    Since we have used the same values for all theorem sets, we now can
%    assign them.
%    \begin{macrocode}
\global\setlength\theorempreskipamount{12pt plus 5pt minus 3pt}
\global\setlength\theorempostskipamount{8pt plus 3pt minus 1.5pt}
%    \end{macrocode}
% \end{macro}
% \end{macro}
%
%
% \begin{macro}{\@endtheorem}
%    The same holds for the macro |\@endtheorem|, which ends a
%    {\envfont theorem} environment. Since it is the same for all
%    theorem sets, it is removed from the macros |\th@|\meta{style}.
%    It simply ends the {\envfont trivlist} environment, which was
%    begun in |\@thm|.
%    \begin{macrocode}
\global\let\@endtheorem=\endtrivlist
%    \end{macrocode}
% \end{macro}
%
%
%
% \begin{macro}{\@preamblecmds}
%    All macros defined above are to be used only in the preamble.
%    Therefore, we insert them in |\@preamblecmds| which will disable
%    them at begin document. This is done by the internal \LaTeXe{}
%    command |\@onlypreamble|.
% \changes{v2.0g}{89/09/19}{`theoremheaderfont and `theorembodyfont
%    added.}
% \changes{v2.2a}{94/02/03}{Use \LaTeXe{} interface}
%    \begin{macrocode}
\@onlypreamble\@xnthm
\@onlypreamble\@ynthm
\@onlypreamble\@othm
\@onlypreamble\newtheorem
\@onlypreamble\theoremstyle
\@onlypreamble\theorembodyfont
\@onlypreamble\theoremheaderfont
%    \end{macrocode}
% \end{macro}
%
% Finally we declare the \texttt{plain} theorem style to be the default.
% \changes{v2.1b}{90/06/04}{Fixed missing `item by declaring default
%    style.}
%    \begin{macrocode}
\theoremstyle{plain}
%</package>
%    \end{macrocode}
%
% \Finale

\endinput
}{}
\ProvidesFile{thc.sty}
             [\FMithmInfo]
%    \end{macrocode}
%
% \begin{macro}{\th@change}
%    The \texttt{change} theorem style corresponds to the
%    \texttt{change break} style without a linebreak after the header.
%    To say it in another way, it's the same as the \texttt{plain}
%    style but with number and name interchanged and |\slshape| as the
%    default font.
%    \begin{macrocode}
\gdef\th@change{\normalfont\slshape
  \def\@begintheorem##1##2{\item
       [\hskip\labelsep \theorem@headerfont ##2\ ##1]}%
\def\@opargbegintheorem##1##2##3{%
  \item[\hskip\labelsep \theorem@headerfont ##2\ ##1\ (##3)]}}
\endgroup
%</thc>
%    \end{macrocode}
% \end{macro}
%
%
% \subsubsection{The \texttt{marginbreak} style}
%
% This style option is the one used most often at Mainz.  It is saved
% in the file |thmb.sty|.
%    \begin{macrocode}
%<*thmb>
\begingroup \makeatletter
\@ifundefined{theorem@style}{% \iffalse meta-comment
%
% Copyright (C) 1993-2024
%
% The LaTeX Project and any individual authors listed elsewhere
% in this file.
%
% This file is part of the Standard LaTeX `Tools Bundle'.
% -------------------------------------------------------
%
% It may be distributed and/or modified under the
% conditions of the LaTeX Project Public License, either version 1.3c
% of this license or (at your option) any later version.
% The latest version of this license is in
%    https://www.latex-project.org/lppl.txt
% and version 1.3c or later is part of all distributions of LaTeX
% version 2005/12/01 or later.
%
% The list of all files belonging to the LaTeX `Tools Bundle' is
% given in the file `manifest.txt'.
%
% \fi
%
%%
%
%
\def\FMithmInfo{2023/07/05 v2.2c Theorem extension package (FMi)}
%
% \ProvidesFile{theorem.dtx}[\FMithmInfo]
%\iffalse   % this is a METACOMMENT !
%%
%% Package `theorem' to use with LaTeX2e
%% Copyright (C) 1989-1995 Frank Mittelbach, all rights reserved.
%<+package>\@ifundefined{theorem@style}{}{\endinput}
%<+package>\NeedsTeXFormat{LaTeX2e}[1995/06/01]
%<+package>\ProvidesPackage{theorem}
%<+package>         [\FMithmInfo]
%
% \fi
%
% \changes{v1.0}{87/05/07}{First draft.}
% \changes{v1.0a}{87/05/07}{`thistheoremstyle' added.}
% \changes{v1.1a}{87/07/29}{New concept: redefine the internal
%                          \protect\LaTeX{} commands `@xnthm' etc.}
% \changes{v1.1b}{87/07/31}{`theoremfont' used to specify font.}
% \changes{v1.2a}{88/01/08}{`@endtheorem' now globally defined.}
% \changes{v1.2b}{88/01/11}{`newtheorem' only allowed in the preamble.}
% \changes{v1.3a}{88/02/14}{`theoremfont' now a toks register.}
%
% \changes{v2.0a}{88/11/16}{New implementation and new documentation.}
% \changes{v2.0a}{88/11/16}{No layout changes but internal modifications
%                           old substyles won't work.}
% \changes{v2.0b}{89/01/30}{`global' used for all definitions.}
% \changes{v2.0c}{89/06/15}{Documentation translated to English.
%                            Thanks to LLL and Nina}
% \changes{v2.1a}{90/02/15}{Support for new font selection.}
% \changes{v2.2a}{94/02/03}{Cleanup for LaTeX2e}
%
% \DoNotIndex{\@endparenv,\@ifdefinable,\@ifnextchar,\@ifundefined}
% \DoNotIndex{\@namedef,\@nameuse,\@spaces,\@tempa,\@warning}
% \DoNotIndex{\ }
% \DoNotIndex{\begingroup,\bfseries}
% \DoNotIndex{\csname}
% \DoNotIndex{\def,\do,\docdate}
% \DoNotIndex{\endcsname,\endgroup,\endinput,\endtrivlist,\expandafter}
% \DoNotIndex{\filedate,\fileversion}
% \DoNotIndex{\gdef,\global}
% \DoNotIndex{\hbox,\hskip}
% \DoNotIndex{\ignorespaces,\input,\itshape,\item}
% \DoNotIndex{\labelsep,\let,\llap}
% \DoNotIndex{\makeatletter}
% \DoNotIndex{\noexpand}
% \DoNotIndex{\refstepcounter,\relax,\rlap}
% \DoNotIndex{\setlength,\slshape,\space,\string,\strut}
% \DoNotIndex{\the,\trivlist,\typeout}
% \DoNotIndex{\vbox}
% \DoNotIndex{\xdef}
%
% \def\deflabel#1{\ttfamily #1\hfill}
% \def\deflist#1{\begin{list}{}{\settowidth\labelwidth{\ttfamily #1}%
%                               \setlength\leftmargin\labelwidth
%                               \addtolength\leftmargin\labelsep
%                               \let\makelabel\deflabel}}
% \def\enddeflist{\end{list}}
% \def\envfont{\normalfont\ttfamily}
%
% \MakeShortVerb{\|}
%
% \newcommand{\TUB}{{\sl TUGboat\/}}
%
% \GetFileInfo{theorem.dtx}
% \title{An Extension of the
%        \LaTeX{} theorem environment\thanks{This file
%        has version number \fileversion, last
%        revised \filedate.}}
% \author{Frank Mittelbach}
%
% \MaintainedByLaTeXTeam{tools}
% \maketitle
%
% \begin{abstract}
%    The macros described in this paper yield an extension of the
%    \LaTeX{} theorem mechanism. It is designed to satisfy the
%    different requirements of various journals. Thus, the
%    layout of the ``theorems'' can be manipulated by determining a
%    ``style''. This article describes not only the use, but
%    also the definition, of the necessary macros.
% \end{abstract}
%
% \section*{Preface to version 2.2}
%
% For \LaTeXe{} this package did not need any fundamental changes. I
% only modified the messages generated so that theorem layout styles
% will show up with the |\listfiles| command and cleaned the section
% on the New Font Selection Scheme since this is now included in
% \LaTeX.
%
% \section*{Preface to version 2.1}
%
% This version is identical to 2.0g described in \TUB\ 10\#3 except
% for some internal defaults which are now set depending on the used
% font selection scheme.
%
% This was done to avoid unpleasant surprises if the new font selection
% scheme is in force. For further details see
% section~\ref{sec:newfonts} and \cite{bk:GMS94}.
%
% \section{Introduction}
%
% For our purposes here, ``theorems'' are labelled enunciations,
% often set off from the main text by extra space and a font change.
% Theorems, corollaries, conjectures, definitions, and remarks
% are all instances of ``theorems''.  The ``header'' of these structures
% is composed of a label (such as \textsc{Theorem} or \textsc{Remark})
% and a number which serializes an item in the sequence of items
% with the same label.
%
% Shortly after the introduction of \LaTeX{} at the Fachbereich
% Mathematik in Mainz, the desire to manipulate the layout of
% ``theorems'' arose.  In Mainz, the following two conventions came
% into general use:
% \begin{enumerate}
%   \item The number of the theorem is shown in the margin.
%   \item There is a line break at
%         the end of the theorem header.
% \end{enumerate}
%
% Additionally, some journals require different formats which depend on
% the ``sort of theorem'': e.g.\ often remarks and definitions are set
% in |\upshape|, while |\itshape| is employed for main theorems.
%
% Confronted with these requirements, a theorem environment was
% developed in Mainz which allows separate determination of the
% layout of the ``theorems sets'', comparable to |\pagestyle|.
%
%
% \section{The user interface}
% \subsection{Defining new theorem sets}
%
% \DescribeMacro\newtheorem
% As in the original \LaTeX{} version, the command |\newtheorem|
% defines a new ``theorem set'' or ``theorem-like structure''.
% Two required arguments name the new environment and give the text
% to be typeset with each instance of the new ``set'', while
% an optional argument determines how
% the ``set'' is enumerated:
% \begin{description}
%    \item[\ttfamily \bslash newtheorem\{foo\}\{bar\}]
%       The theorem set {\envfont foo} (whose name is \texttt{bar})
%       uses its own counter.
%    \item[\ttfamily \bslash newtheorem\{foo2\}{[foo]}\{bar2\}]
%       The theorem set {\envfont foo2} (printed name \texttt{bar2})
%       uses the same counter as the theorem set \texttt{foo}.
%    \item[\ttfamily \bslash newtheorem\{foo3\}\{bar3\}{[section]}]
%       The theorem set {\envfont foo3} (printed name \texttt{bar3}) is
%       enumerated within the counter \texttt{section}, i.e.\ with every
%       new |\section| the enumeration begins again with 1, and
%       the enumeration is composed from the section-number and the
%       theorem counter itself.
% \end{description}
%
% \DescribeMacro\theoremstyle
% Additionally, the command |\theoremstyle| can define the layout
% of various, or all, theorem sets.  It should be noted that any
% theorem set defined by |\newtheorem| is typeset in the
% |\theoremstyle| that is current at the time of the definition.
% Thus, the following
% \begin{verbatim}
%    \theoremstyle{break}        \newtheorem{Cor}{Corollary}
%    \theoremstyle{plain}        \newtheorem{Exa}{Example}[section]
%\end{verbatim}                             ^^A No blank at line start
% leads to the result that the set {\envfont Cor} is formatted in the
% style \texttt{break}, while the set {\envfont Exa} and all the
% following ones are formatted in the style \texttt{plain}, unless
% another |\theoremstyle| follows.  Since the definitions installed by
% |\newtheorem| are global, one also can limit |\theoremstyle| locally
% by grouping braces.
%
% \DescribeMacro\theorembodyfont
% The choice of the font for the theorem body is completely
% independent of the chosen |\theoremstyle|; this has proven to
% be very advantageous.  For example,
% \begin{verbatim}
%   {\theorembodyfont{\upshape}          \newtheorem{Rem}{Remark}}
%\end{verbatim}
% defines a theorem set {\envfont Rem}, which will be set in
% |\upshape| in the current layout (which in our example is
% \texttt{plain}).  As with |\theoremstyle|, the |\theorembodyfont|
% chosen is that current at the time of |\newtheorem|.  If
% |\theorembodyfont| is not specified or one defines
% |\theorembodyfont{}|, then the font used will be that defined by the
% |\theoremstyle|.
%
% \DescribeMacro\theoremheaderfont It is also possible to customize
% the font used for the theorem headers.  This is, however, a global
% declaration, and therefore there should be at most one
% |\theoremheaderfont| declaration in the preamble.\footnote{If it is
% actually necessary to have different header fonts, one has to define
% new theorem styles (substituting the desired font) or specify the
% information directly in the \texttt{\bslash newtheorem} declaration
% (the unclean variant).}
%
% \DescribeMacro\theorempreskipamount
% \DescribeMacro\theorempostskipamount Two additional parameters
% affect the vertical space around the theorem environments:
% |\theorempreskipamount| and |\theorempostskipamount| define,
% respectively, the spacing before and after such an environment.
% These parameters apply for all theorem sets and can be manipulated
% with the ordinary length macros.  They are rubber lengths
% (`\textsf{skips}'), and therefore can contain \texttt{plus} and
% \texttt{minus} parts.
%
% Since the definition of theorem sets should---most sensibly---be
% placed in the preamble, we only allow installation there.
% It is therefore possible to release the memory used here after
% |\begin{document}|, in order to make room for other
% applications.
%
%
% \subsection{Existing theorem styles}
%
% The following theorem styles exist to date:
% \begin{deflist}{marginbreak}
%    \item[plain]
%       This theorem style emulates the original \LaTeX{} definition,
%       except that additionally the parameters
%       |\theorem...skipamount| are used.
%    \item[break]
%       In this style, the theorem header is followed by a line
%       break.
%    \item[marginbreak]
%       The theorem number is set in the margin, and there is a line
%       break as in \texttt{break}.
%    \item[changebreak]
%       Like \texttt{break}, but with header number and text
%       interchanged.
%    \item[change]
%       Header number and text are interchanged, without a line break.
%    \item[margin]
%       The number is set in the left margin, without a line break.
% \end{deflist}
% All styles (except \texttt{plain}) select |\slshape| as the default
% |\theorembodyfont|.
%
%
% \subsection{Examples}
%
% Given the above theorem sets {\envfont Cor}, {\envfont Exa}
% and {\envfont Rem},
% suppose that the preamble also contains the declarations:
% \begin{verbatim}
%    \theoremstyle{marginbreak}   \newtheorem{Lem}[Cor]{Lemma}
%    \theoremstyle{change}
%    \theorembodyfont{\itshape}        \newtheorem{Def}[Cor]{Definition}
%
%    \theoremheaderfont{\scshape}
%\end{verbatim}
% Then the following are some typical examples of the typeset output
% resulting from their use.
%
% \begin{Cor}
%    This is a sentence typeset in the theorem environment {\envfont
%    Cor}.
% \end{Cor}
%
% \begin{Exa}
%    This is a sentence typeset in the theorem environment {\envfont
%    Exa}.
% \end{Exa}
%
% \begin{Rem}
%    This is a sentence typeset in the theorem environment {\envfont
%    Rem}.
% \end{Rem}
%
% \begin{Lem}[Ben User]
%    This is a sentence typeset in the theorem environment {\envfont
%    Lem}.
% \end{Lem}
%
% \begin{Def}[Very impressive Definition]
%    This is a sentence typeset in the theorem environment {\envfont
%    Def}.
% \end{Def}
%
% The last two examples show the effect of the optional argument to
% a theorem environment (it is the text typeset in parentheses).
%
%
% \section{Special Considerations}
% \label{sec:newfonts}
%
% Theoremheader and body are implemented as a unit. This means that
% the |\theoremheaderfont|  will inherit characteristics of the
% |\theorembodyfont| in \LaTeXe. Thus, if for example
% |\theorembodyfont| is |\itshape| and |\theoremheaderfont|
% is |\bfseries| the font selected for the header will have the
% characteristics `bold extended italic'. If this is not desired one
% should set the |\theoremheaderfont| to something like
% \begin{verbatim}
%      \theoremheaderfont{\normalfont\bfseries}
%\end{verbatim}
% i.e.\ supplying all necessary font information explicitly.
%
%
%
% \section{Acknowledgements}
%
% The publication of this set of macros was only possible with the
% help of Christina Busse (translating the manuscript into English),
% Joachim Pense (playing the r\^ole of typist), Chris Rowley
% (looking everything over) and many others providing useful
% suggestions.
%
% \MaybeStop{%
%    \begin{thebibliography}{9}
%       \bibitem{bk:GMS94} \textsc{M.~Goossens}, \textsc{F.~Mittelbach}
%         and \textsc{A.~Samarin}.
%         \newblock The \LaTeX{} Companion.
%         \newblock
%         Addison-Wesley, Reading, Massachusetts, 1994.
%       \bibitem{latex:source} \textsc{Lamport, Leslie.} \newblock
%                               \texttt{latex.tex}, version 2.09, date
%                               Feb. 1990.
%    \end{thebibliography}
%    \PrintIndex
%    \PrintChanges}
%
%
% \section{The documentation driver file}
%
% The next bit of code contains the documentation driver file for
% \TeX{}, i.e., the file that will produce the documentation you are
% currently reading. It will be extracted from this file by the
% \texttt{docstrip} program. Since it is the first code in the file
% one can alternatively process this file directly with \LaTeXe{} to
% obtain the documentation.
%
%    \begin{macrocode}
%<*driver>
\documentclass{ltxdoc}

\usepackage{theorem}

%<+driver>% The next few lines define theorem sets which are used
%<+driver>% in the example section of the documentation.

\theoremstyle{break}       \newtheorem{Cor}{Corollary}
\theoremstyle{plain}       \newtheorem{Exa}{Example}[section]
{\theorembodyfont{\upshape}\newtheorem{Rem}{Remark}}
\theoremstyle{marginbreak} \newtheorem{Lem}[Cor]{Lemma}
\theoremstyle{change}
\theorembodyfont{\itshape} \newtheorem{Def}[Cor]{Definition}

\theoremheaderfont{\scshape}

\RecordChanges

\begin{document}
   \DocInput{theorem.dtx}
\end{document}
%</driver>
%    \end{macrocode}
%
% \section{Definition of the Macros}
%
% If the file has been loaded before, we abort immediately. If not the
% package announces itself (this is actually done at the very top of
% the file---the way it is done isn't good style so don't copy it).
% \changes{v2.0e}{89/07/19}{Spaces removed from `typeout'.}
% \changes{v2.2b}{95/11/19}{Announce moved to top of file}
%    \begin{macrocode}
%<*package>
%\@ifundefined{theorem@style}{}{\endinput}
%\def\FMithmInfo{1995/11/19 v2.2b Theorem extension package (FMi)}
%\ProvidesPackage{theorem}[\FMithmInfo]
%    \end{macrocode}
%
%
% \subsection{Definition of theorem styles and fonts}
%
% All the definitions in this file are done globally to allow
% inputting this file inside a group.
%
% \begin{macro}{\theoremstyle}
%    Before a theorem style can be installed, the chosen style must be
%    known. For that reason, we must test to see that
%    |\th@|\meta{style} is known or, more precisely, that it is
%    different from |\relax|. If the style is not known then
%    |\th@plain| is used.
%    \begin{macrocode}
\gdef\theoremstyle#1{%
   \@ifundefined{th@#1}{\@warning
          {Unknown theoremstyle `#1'. Using `plain'}%
          \theorem@style{plain}}%
%    \end{macrocode}
%    We save the theorem style to be used in the token register
%    |\theorem@style|.
%    \begin{macrocode}
      {\theorem@style{#1}}%
%    \end{macrocode}
%    Now we ``evaluate'' the theorem style: this means, we call the
%    macro |\th@|\meta{style} which will activate the relevant
%    definitions which are contained in a separate file.  This is done
%    in a group to suppress changes to the current font.  This could
%    otherwise pose problems together with the new font selection
%    scheme\footnote{When I printed the original article using the new
%    font selection scheme I ended with a document with slanted
%    typefaces (text headings and all) simply because one of the
%    theorem styles used \texttt{\bslash sl} at toplevel.} if the
%    |\th@|\meta{style} is evaluated a second time.
% \changes{v2.1a}{90/02/15}{Group added around `th@..}
% \changes{v2.1a}{90/02/15}{Reseting of `..@begintheorem removed.}
%    \begin{macrocode}
      \begingroup
        \csname th@\the\theorem@style \endcsname
      \endgroup}
%    \end{macrocode}
% \end{macro}
%
%
% \begin{macro}{\@begintheorem}
% \changes{v2.1a}{90/02/15}{Relaxed at toplevel.}
% \begin{macro}{\@opargbegintheorem}
% \changes{v2.1a}{90/02/15}{Relaxed at toplevel.}
%    We reset |\@begintheorem| and |\@opargbegintheorem| to
%    |\relax| since these commands are no longer necessary at
%    toplevel. This will save a few tokens.
%    \begin{macrocode}
\global\let\@begintheorem\relax
\global\let\@opargbegintheorem\relax
%    \end{macrocode}
% \end{macro}
% \end{macro}
%
% \begin{macro}{\theorem@style}
%    Obviously the \textsf{token} register used above has to be
%    allocated. To assure the utmost compatibility with the
%    original \LaTeX{} definition, we set the default theorem style to
%    \texttt{plain}, which implements the usual \LaTeX{} convention.
% \changes{v2.0f}{89/07/20}{Unnecessary `global' removed.}
%    \begin{macrocode}
\newtoks\theorem@style
\global\theorem@style{plain}
%    \end{macrocode}
% \end{macro}
%
% \begin{macro}{\theorembodyfont}
% \begin{macro}{\theorem@bodyfont}
%    If the |\theorembodyfont| is set by the user then it should not
%    interact with the default font set in the theorem style. When the
%    new font selection is in force this may happen if, for example,
%    the default is |\itshape| and the new |\theorembodyfont| is
%    |\sffamily|. So we add a |\reset@font| command in front of the
%    user definition.
% \changes{v2.0f}{89/07/20}{Unnecessary `global' removed.}
% \changes{v2.0g}{89/09/19}{`theoremfont to `theorembodyfont.}
% \changes{v2.1c}{92/06/22}{`reset@font added and logic changed.}
% \changes{v2.1d}{92/10/28}{Reset to empty if arg is empty}
%    \begin{macrocode}
\gdef\theorembodyfont#1{%
%    \end{macrocode}
%    We check if the argument supplied is empty and if so put nothing
%    into the |\theorem@bodyfont| token register to allow for
%    |\theorembodyfont{}| as a mean of using the default of the
%    current |\theoremstyle|.
%    \begin{macrocode}
   \def\@tempa{#1}%
   \ifx\@tempa\@empty
    \theorem@bodyfont{}%
   \else
    \theorem@bodyfont{\reset@font#1}%
   \fi
   }
\newtoks\theorem@bodyfont
\global\theorem@bodyfont{}
%    \end{macrocode}
% \end{macro}
% \end{macro}
%
%
% \begin{macro}{\theoremheaderfont}
% \changes{v2.0g}{89/09/19}{Macro added.}
%    The font for the theorem headers is handled differently because
%    this definition applies to all theorem styles.
%    \begin{macrocode}
\gdef\theoremheaderfont#1{\gdef\theorem@headerfont{#1}%
%    \end{macrocode}
%    After using the macro once it is redefined to produce an error
%    message.
%    \begin{macrocode}
       \gdef\theoremheaderfont##1{%
        \typeout{\string\theoremheaderfont\space should be used
                 only once.}}}
%    \end{macrocode}
% \end{macro}
%
% \begin{macro}{\theorem@headerfont}
% \changes{v2.0g}{89/09/19}{Macro added.}
%    To set the |\theorem@headerfont| default we first test if
%    the new fontselection scheme is in force.
% \changes{v2.1a}{90/02/15}{Support for new font selection.}
%    \begin{macrocode}
\ifx\upshape\undefined
%    \end{macrocode}
%    If not we define it to expand into |\bfseries|. We don't use
%    |\let| just in case a following style option redefines this
%    macro.
%    \begin{macrocode}
\gdef\theorem@headerfont{\bfseries}
%    \end{macrocode}
%    Otherwise we reset the current shape before calling |\bfseries|.
%    \begin{macrocode}
\else \gdef\theorem@headerfont{\normalfont\bfseries}\fi
%    \end{macrocode}
% \end{macro}
%
% \begin{macro}{\th@plain}
% \begin{macro}{\th@break}
% \begin{macro}{\th@marginbreak}
% \begin{macro}{\th@changebreak}
% \begin{macro}{\th@change}
% \begin{macro}{\th@margin}
%    The different styles are defined in macros such as
%    |\th@plain|.  Since memory space is precious in
%    ``non-Big-versions'', we have to avoid offering too many unused
%    definitions.  Therefore we define these styles in separate files
%    that can be loaded on demand.  Thus the commands themselves only
%    load these files. We use |\@input@|, a \LaTeXe{} internal command
%    that ensures that the file will be listed with |\listfiles|.
% \changes{v2.2a}{94/02/02}{Use `@input@ to load theorem layout files}
%    \begin{macrocode}
\gdef\th@plain{\@input@{thp.sty}}
\gdef\th@break{\@input@{thb.sty}}
\gdef\th@marginbreak{\@input@{thmb.sty}}
\gdef\th@changebreak{\@input@{thcb.sty}}
\gdef\th@change{\@input@{thc.sty}}
\gdef\th@margin{\@input@{thm.sty}}
%    \end{macrocode}
%    This list will be expanded when new styles become
%    available.  For testing,  just append new theorem substyles
%    as document options.
% \end{macro}
% \end{macro}
% \end{macro}
% \end{macro}
% \end{macro}
% \end{macro}
%
%
%
% \subsection{Definition of a new theorem set}
%
% As already pointed out, a new theorem environment can be defined in
% three different ways:
% \begin{verbatim}
%    \newtheorem{Lem}{Lemma}
%    \newtheorem{Lem}{Lemma}[section]
%    \newtheorem{Lem}[Theorem]{Lemma}
%\end{verbatim}
% The function of the macro |\newtheorem| is to recognize these cases
% and then to branch into one of the three macros |\@ynthm|, |\@xnthm|
% or |\@othm|.  This mechanism is adopted unchanged from
% \cite{latex:source}; the essential point here is that, for example,
% in the second case, the arguments \texttt{Lem}, \texttt{Lemma} and
% \texttt{section} are passed over to the macro |\@xnthm|.
%
% We inspect this case first because the others present fewer problems,
% and thus are easily derived from this one.
%
% \begin{macro}{\@xnthm}
%    For our example arguments, the macro |\@xnthm| must fulfill
%    the following:
%    \begin{itemize}
%       \item
%          Define a new \LaTeX-counter `Lem'
%       \item
%          reset this counter within a |\section|
%       \item
%          define the macro |\theLem|
%       \item
%          define the environment macros |\Lem| and
%          |\endLem| using the current
%          |\theoremstyle| and |\theorem@bodyfont|.
%    \end{itemize}
%    Obviously, all this should happen only if the first argument
%    of |\@xnthm| (i.e.\ \texttt{Lem} in our example) is chosen so
%    as not to conflict with any previously defined commands or
%    environments.  This test is performed by the \LaTeX{} macro
%    |\@ifdefinable|.
%    \begin{macrocode}
\gdef\@xnthm#1#2[#3]{\expandafter\@ifdefinable\csname #1\endcsname
%    \end{macrocode}
%    Therefore, the first argument of |\@ifdefinable| is the
%    expansion (in the example, |\Lem|) of
%    |\csname||#1\endcsname|.  The second argument is
%    executed only if the test has been completed successfully.
%    \begin{macrocode}
   {%
%    \end{macrocode}
%    Now we define the new counter. The names of the \LaTeX{}
%    macros employed should speak for themselves:
% \changes{v2.2b}{1995/10/07}{Use \cs{@newctr}}
% \changes{v2.2c}{1995/11/23}{Use \cs{@newctr} properly.}
%    \begin{macrocode}
    \@definecounter{#1}\@newctr{#1}[#3]%
%    \end{macrocode}
%    Using |\@newctr| will give a proper error message if the counter
%    in |#3| is not defined.
%    In defining `|\theLem|' we must generate the desired macro
%    name by use of |\expandafter| and |\csname|.
%    \begin{macrocode}
    \expandafter\xdef\csname the#1\endcsname
%    \end{macrocode}
%    An |\xdef| is used in order to make the definition global,
%    and to ensure that it contains the replacement texts of
%    |\@thmcountersep| and |\@thmcounter|.\footnote{These
%    two macros can be defined by the document style.  Their default
%    values produce a `\texttt{.}' as separation and an arabic
%    representation of the number.} However, not everything should be
%    expanded. For example, it saves space to use |\thesection|
%    instead of its---at times---lengthy expansion.
%    \begin{macrocode}
      {\expandafter \noexpand \csname the#3\endcsname
       \@thmcountersep \@thmcounter{#1}}%
%    \end{macrocode}
%    Thus with the defaults of \LaTeX{}, |\theLem| would be
%    replaced by the command sequence |\thesection.\arabic{Lem}|.
%
%    We will now look at the definition of the macro which is executed
%    at the beginning of the actual environment (in our example this
%    macro is |\Lem|). It should be noted that we use an
%    ``|\expandafter| trick'' to expand only certain parts of the
%    replacement text at the time of the definition.
%    \begin{macrocode}
    \def\@tempa{\global\@namedef{#1}}%
    \expandafter \@tempa \expandafter{%
%    \end{macrocode}
%    First, the macro that contains the current definitions of
%    |\@begintheorem| and |\@opargtheorem| should be called
%    up.  The name of this macro---as is already known---has the form
%    |\th@|\meta{theorem\ style}; therefore, it must be called by
%    \begin{macrocode}
      \csname th@\the \theorem@style
%    \end{macrocode}
%    In addition the default theorem font should be changeable, i.e.\
%    we have to insert the contents of |\theorem@bodyfont|. For
%    that reason, we expand even further, beyond |\endcsname|,
%    and thus insert the contents of the \textsf{token} register
%    |\theorem@bodyfont| in the replacement text.
%    \begin{macrocode}
            \expandafter \endcsname \the \theorem@bodyfont
%    \end{macrocode}
%    Now it is time to call the macro |\@thm| which takes over
%    the further processing.  It has two arguments: the current
%    counter name (in our example, \texttt{Lem}), and the text of the
%    label (in our example, \texttt{Lemma}).
%    \begin{macrocode}
     \@thm{#1}{#2}}%
%    \end{macrocode}
%    With this, the `sub-definition' is complete.  The macro
%    |\@endtheorem| ends a theorem environment and is, so far, nothing
%    but an |\endtrivlist|.  (Hence it is defined globally, and not
%    within the theorem styles.\footnote{This has to be changed as
%    soon as theorem styles that change \texttt{\bslash @endtheorem}
%    exist.  In such a case, all existing styles must be changed as
%    well since they will have to reset the macro.}) Therefore, we can
%    set it equivalent to the macro that ends the theorem set (in our
%    example, |\endLem|).  However, if some day theorem styles exist
%    that do change |\@endtheorem|, we would have to use the
%    commented-out line instead.
%    \begin{macrocode}
    \global \expandafter \let \csname end#1\endcsname \@endtheorem
%   \global\@namedef{end#1}{\@endtheorem}%
%    \end{macrocode}
%    With these commands all the required definitions are employed,
%    unless the test |\@ifdefinable| has failed.  Therefore, we
%    end the second argument of this macro and with it the definition
%    of |\@xnthm|.
%    \begin{macrocode}
   }}
%    \end{macrocode}
% \end{macro}
%
% \begin{macro}{\@ynthm}
%    The definition of |\@ynthm| is completely analogous. In
%    this case the new counter that is defined is not reset within
%    another counter; thus the definition of |\the...|
%    is simplified:
%    \begin{macrocode}
\gdef\@ynthm#1#2{\expandafter\@ifdefinable\csname #1\endcsname
   {\@definecounter{#1}%
    \expandafter\xdef\csname the#1\endcsname{\@thmcounter{#1}}%
%    \end{macrocode}
%    The rest of the definition corresponds literally to that of
%    |\@xnthm|:
%    \begin{macrocode}
    \def\@tempa{\global\@namedef{#1}}\expandafter \@tempa
     \expandafter{\csname th@\the \theorem@style \expandafter
     \endcsname \the\theorem@bodyfont \@thm{#1}{#2}}%
    \global \expandafter \let \csname end#1\endcsname \@endtheorem}}
%    \end{macrocode}
% \end{macro}
%
% \begin{macro}{\@othm}
%    The definition of |\@othm| does not contain anything new.
%    \begin{macrocode}
\gdef\@othm#1[#2]#3{%
%    \end{macrocode}
% \changes{v2.2b}{95/10/07}{Build in check for counter defined}
%    We do not define a new counter but instead use one that has
%    already been defined.  Thus the only definition we need is that
%    of this pseudo-counter (i.e.\ |\the|\meta{env.\ name}).
%    First we check if |#2| corresponds to a known counter name.
%    \begin{macrocode}
  \expandafter\ifx\csname c@#2\endcsname\relax
   \@nocounterr{#2}%
  \else
   \expandafter\@ifdefinable\csname #1\endcsname
   {\expandafter \xdef \csname the#1\endcsname
     {\expandafter \noexpand \csname the#2\endcsname}%
%    \end{macrocode}
%    All other parts of the definition can be adopted from
%    |\@xnthm|. We have to remember, though, that in this case
%    the name of the current counter and the theorem label have moved
%    to the second and third arguments.
%    \begin{macrocode}
    \def\@tempa{\global\@namedef{#1}}\expandafter \@tempa
     \expandafter{\csname th@\the \theorem@style \expandafter
     \endcsname \the\theorem@bodyfont \@thm{#2}{#3}}%
    \global \expandafter \let \csname end#1\endcsname \@endtheorem}%
  \fi}
%    \end{macrocode}
% \end{macro}
%
% \subsection{Macros that are employed in a {\envfont theorem}
%    environment}
%
% \begin{macro}{\@thm}
%    The macro |\@thm| has to increase the current counter.  Then,
%    depending on whether the environment has (or does not have) an
%    optional argument, it has to branch into either |\@begintheorem|
%    or |\@opargtheorem|.
%    \begin{macrocode}
\gdef\@thm#1#2{\refstepcounter{#1}%
%    \end{macrocode}
%    Now we start a {\envfont trivlist} environment, and give
%    |\@topsep| and |\@topsepadd| the values of the \textsf{skip}
%    registers |\theorempreskipamount| and |\theorempostskipamount|.
%    The value in |\@topsep| is the vertical space that is inserted by
%    the first (and only) |\item| in our |\trivlist| whilst
%    |\@topsepadd| is inserted by |\@endparenv| at the end of that
%    {\envfont trivlist} environment.  By using these registers, we
%    obtain the desired space around a {\envfont theorem} environment.
%    \begin{macrocode}
   \trivlist
   \@topsep \theorempreskipamount               % used by first \item
   \@topsepadd \theorempostskipamount           % used by \@endparenv
%    \end{macrocode}
%    Now we have to test whether an optional argument has been given.
% \changes{v2.0d}{89/07/18}{Forgotten percent added.}
%    \begin{macrocode}
   \@ifnextchar [%
%    \end{macrocode}
%    If there is an optional argument, we will call |\@ythm|, and
%    move the arguments read back into the input stream.
%    \begin{macrocode}
   {\@ythm{#1}{#2}}%
%    \end{macrocode}
%    If not, we call |\@begintheorem|. Its first argument is the
%    name of the theorem set (hence the second argument of
%    |\@thm|). Its second argument is the macro that produces the
%    current number.
%    \begin{macrocode}
   {\@begintheorem{#2}{\csname the#1\endcsname}\ignorespaces}}
%    \end{macrocode}
% \end{macro}
%
% \begin{macro}{\@xthm}
% \begin{macro}{\@ythm}
%    Both these macros were originally called by |\@thm|.  We do
%    not need |\@xthm| anymore, hence we reset it to
%    |\relax|. The definition of |\@ythm| has not changed at
%    all from its definition in \LaTeX. In order to make the macros
%    easier to understand, we will
%    nevertheless present it (commented out).
%    \begin{macrocode}
\global\let\@xthm\relax
% \def\@ythm#1#2[#3]{\@opargbegintheorem{#2}{\csname
%           the#1\endcsname}{#3}\ignorespaces}
%</package>
%    \end{macrocode}
%    The primitive |\ignorespaces| in |\@ythm| and
%    |\@thm| is needed to remove the spaces between the
%    |\begin{...}| and the actual text.
% \end{macro}
% \end{macro}
%
%
% \subsection{Definition of the theorem substyles}
%
% As already pointed out, the theorem substyles, defined below, are
% only loaded when necessary.  Note that all these substyles, except
% \texttt{plain}, have |\slshape| as the default body font.
%
% \subsubsection{The \texttt{plain} style}
%
% \label{dok:th@plain} As the following macros use |@|, we have
% to locally set the |\catcode| of this symbol to ``letter''.
% This happens within a group, so that we do not have to worry about
% which |\catcode| that symbol had before.
%    \begin{macrocode}
%<*thp>
\begingroup \makeatletter
%    \end{macrocode}
% Since we are now within a group, we must make all definitions
% globally.  First we make sure that \texttt{theorem.sty} is loaded.
% This will allow us to use this file as a document style option
% without having to call \texttt{theorem} itself as an option.  At the
% same time, we assure that at least version 2 is loaded, since
% |\theorem@style| was not defined in earlier versions.
%    \begin{macrocode}
\@ifundefined{theorem@style}{\input{theorem.sty}}{}
\ProvidesFile{thp.sty}
             [\FMithmInfo]
%    \end{macrocode}
%
%
% \begin{macro}{\th@plain}
% \begin{macro}{\@begintheorem}
% \begin{macro}{\@opargbegintheorem}
%    |\theoremstyle{plain}| corresponds to the original
%    definition, except that the distances to the surrounding text are
%    determined by the parameters |\theorempreskipamount| and
%    |\theorempostskipamount|.
%    First we set the default body font.
%    \begin{macrocode}
\gdef\th@plain{\normalfont\itshape
%    \end{macrocode}
%    Then we define |\@begintheorem| and
%    |\@opargbegintheorem|. These two macros define how the
%    header of a theorem is typeset.  |\@opargbegintheorem| will
%    be called if a \textsf{theorem} environment with an optional
%    argument is encountered; otherwise, the header is constructed by
%    calling |\@begintheorem|.  If one of these macros is
%    executed, we are within a \textsf{trivlist} environment started by
%    |\@thm|.  So the theorem header is produced with an
%    |\item| command.
%
%    Instead of specifying the header font directly, all standard
%    theorem styles use the |\theorem@headerfont| macro to allow
%    customization.  The extra space (|\labelsep|) is necessary
%    because of problems in the \textsf{trivlist} environment.
%    \begin{macrocode}
  \def\@begintheorem##1##2{%
        \item[\hskip\labelsep \theorem@headerfont ##1\ ##2]}%
%    \end{macrocode}
%    The definition of |\@opargbegintheorem| is completely
%    analogous. The only difference is the fact that there exists a
%    third argument (which is the optional parameter of the
%    environment and contains additional information about the
%    theorem).  Customarily we enclose it in parentheses.
%    \begin{macrocode}
\def\@opargbegintheorem##1##2##3{%
   \item[\hskip\labelsep \theorem@headerfont ##1\ ##2\ (##3)]}}
%    \end{macrocode}
% \end{macro}
% \end{macro}
% \end{macro}
%
% \noindent
% We conclude with an |\endgroup| to restore the
% |\catcode| of |@|.
%    \begin{macrocode}
\endgroup
%</thp>
%    \end{macrocode}
%
%
% \subsubsection{The \texttt{break} style}
%
% This style option is stored in the file |thb.sty|.
% For the next two lines see the documentation for |\th@plain| on
% page~\pageref{dok:th@plain}.
%    \begin{macrocode}
%<*thb>
\begingroup \makeatletter
\@ifundefined{theorem@style}{\input{theorem.sty}}{}
\ProvidesFile{thb.sty}
             [\FMithmInfo]
%    \end{macrocode}
%
% \begin{macro}{\th@break}
%    |\theoremstyle{break}| produces a line break after the name
%    of the theorem.  The font is |\slshape|.  Hence, we define
%    |\th@break| as follows:
%    \begin{macrocode}
\gdef\th@break{\normalfont\slshape
  \def\@begintheorem##1##2{\item[%
%    \end{macrocode}
%    We run into the following problem: it is not possible to create
%    the header with |\item[|\meta{title}|]| and then start
%    a new line by, for example, |\mbox{}\\|. Such a definition
%    will fail whenever a list environment follows immediately.  With
%    the above construction, the |\mbox{}| causes the switch
%    |@inlabel| (cf.\ definition of |\list| and
%    |\trivlist| in \cite{latex:source}) to be set to \texttt{false}
%    and so the following list will insert additional vertical
%    space (|\topskip|). This is quite annoying. Therefore, we
%    create the line break within the |\item|. In order to ensure
%    that the text will begin at the proper position in the following
%    line, we simply pretend that the label does not take any
%    room.\footnote{This will lead to problems whenever very high
%    symbols occurring in the line tower into the
%    heading. So, something else has to be done here sometime.}
%    \begin{macrocode}
       \rlap{\vbox{\hbox{\hskip \labelsep\theorem@headerfont ##1\ ##2}%
                   \hbox{\strut}}}]}%
%    \end{macrocode}
%    Again, the definition of |\@opargbegintheorem| is
%    completely analogous.
%    \begin{macrocode}
\def\@opargbegintheorem##1##2##3{%
  \item[\rlap{\vbox{\hbox{\hskip \labelsep \theorem@headerfont
                     ##1\ ##2\ (##3)}%
                    \hbox{\strut}}}]}}
\endgroup
%</thb>
%    \end{macrocode}
% \end{macro}
%
%
% \subsubsection{The \texttt{changebreak} style}
%
%    \begin{macrocode}
% This style option is stored in the file |thcb.sty|.
%    \begin{macrocode}
%<*thcb>
\begingroup \makeatletter
\@ifundefined{theorem@style}{\input{theorem.sty}}{}
\ProvidesFile{thcb.sty}
             [\FMithmInfo]
%    \end{macrocode}
%
% \begin{macro}{\th@changebreak}
%    The \texttt{change-break} theorem style is like \texttt{break}
%    but with interchange of theorem name and theorem number.  Thus we
%    define |\th@changebreak| as follows:
%    \begin{macrocode}
\gdef\th@changebreak{\normalfont\slshape
  \def\@begintheorem##1##2{\item
       [\rlap{\vbox{\hbox{\hskip\labelsep \theorem@headerfont ##2\ ##1}%
                    \hbox{\strut}}}]}%
\def\@opargbegintheorem##1##2##3{%
  \item[\rlap{\vbox{\hbox{\hskip\labelsep \theorem@headerfont
                     ##2\ ##1\ (##3)}%
                    \hbox{\strut}}}]}}
\endgroup
%</thcb>
%    \end{macrocode}
% \end{macro}
%
%
% \subsubsection{The \texttt{change} style}
%
% This style option is stored in the file |thc.sty|.
%    \begin{macrocode}
%<*thc>
\begingroup \makeatletter
\@ifundefined{theorem@style}{\input{theorem.sty}}{}
\ProvidesFile{thc.sty}
             [\FMithmInfo]
%    \end{macrocode}
%
% \begin{macro}{\th@change}
%    The \texttt{change} theorem style corresponds to the
%    \texttt{change break} style without a linebreak after the header.
%    To say it in another way, it's the same as the \texttt{plain}
%    style but with number and name interchanged and |\slshape| as the
%    default font.
%    \begin{macrocode}
\gdef\th@change{\normalfont\slshape
  \def\@begintheorem##1##2{\item
       [\hskip\labelsep \theorem@headerfont ##2\ ##1]}%
\def\@opargbegintheorem##1##2##3{%
  \item[\hskip\labelsep \theorem@headerfont ##2\ ##1\ (##3)]}}
\endgroup
%</thc>
%    \end{macrocode}
% \end{macro}
%
%
% \subsubsection{The \texttt{marginbreak} style}
%
% This style option is the one used most often at Mainz.  It is saved
% in the file |thmb.sty|.
%    \begin{macrocode}
%<*thmb>
\begingroup \makeatletter
\@ifundefined{theorem@style}{\input{theorem.sty}}{}
\ProvidesFile{thmb.sty}
             [\FMithmInfo]
%    \end{macrocode}
%
% \begin{macro}{\th@marginbreak}
%    The \texttt{margin break} style is nearly the same as the
%    \texttt{change break} style. The only difference is the placement
%    of the theorem number.  We use |\llap| to place it in the left
%    margin.
%
%    In this style |\labelsep| denotes the separation between the
%    number and the text.
%    \begin{macrocode}
\gdef\th@marginbreak{\normalfont\slshape
  \def\@begintheorem##1##2{\item
       [\rlap{\vbox{\theorem@headerfont
                    \hbox{\llap{##2}\hskip\labelsep ##1}%
                    \hbox{\strut}}}]}%
\def\@opargbegintheorem##1##2##3{%
  \item[\rlap{\vbox{\theorem@headerfont
                    \hbox{\llap{##2}\hskip\labelsep ##1\ (##3)}%
                    \hbox{\strut}}}]}}
\endgroup
%</thmb>
%    \end{macrocode}
% \changes{v2.1c}{90/02/17}{`endgroup outside macrocode env.}
% \end{macro}
%
%
%
% \subsubsection{The \texttt{margin} style}
%
% This style option is stored in the file |thm.sty|.
%    \begin{macrocode}
%<*thm>
\begingroup \makeatletter
\@ifundefined{theorem@style}{\input{theorem.sty}}{}
\ProvidesFile{thm.sty}
             [\FMithmInfo]
%    \end{macrocode}
%
% \begin{macro}{\th@margin}
%    Again this is only a variant of the theorem styles described
%    above without any new ideas.
%    \begin{macrocode}
\gdef\th@margin{\normalfont\slshape
  \def\@begintheorem##1##2{\item
       [\theorem@headerfont \llap{##2}\hskip\labelsep ##1]}%
\def\@opargbegintheorem##1##2##3{%
  \item[\theorem@headerfont \llap{##2}\hskip\labelsep ##1\ (##3)]}}
\endgroup
%</thm>
%    \end{macrocode}
% \end{macro}
%
%
%
%
% \subsection{Final Definitions}
%
% \begin{macro}{\theorempreskipamount}
% \begin{macro}{\theorempostskipamount}
%    The \textsf{skip} parameters that regulate the vertical empty space
%    before and after the {\envfont theorem} environment have
%    to be allocated as well.
% \changes{v2.0f}{89/07/20}{Unnecessary `global' removed.}
%    \begin{macrocode}
%<*package>
\newskip\theorempreskipamount
\newskip\theorempostskipamount
%    \end{macrocode}
%    Since we have used the same values for all theorem sets, we now can
%    assign them.
%    \begin{macrocode}
\global\setlength\theorempreskipamount{12pt plus 5pt minus 3pt}
\global\setlength\theorempostskipamount{8pt plus 3pt minus 1.5pt}
%    \end{macrocode}
% \end{macro}
% \end{macro}
%
%
% \begin{macro}{\@endtheorem}
%    The same holds for the macro |\@endtheorem|, which ends a
%    {\envfont theorem} environment. Since it is the same for all
%    theorem sets, it is removed from the macros |\th@|\meta{style}.
%    It simply ends the {\envfont trivlist} environment, which was
%    begun in |\@thm|.
%    \begin{macrocode}
\global\let\@endtheorem=\endtrivlist
%    \end{macrocode}
% \end{macro}
%
%
%
% \begin{macro}{\@preamblecmds}
%    All macros defined above are to be used only in the preamble.
%    Therefore, we insert them in |\@preamblecmds| which will disable
%    them at begin document. This is done by the internal \LaTeXe{}
%    command |\@onlypreamble|.
% \changes{v2.0g}{89/09/19}{`theoremheaderfont and `theorembodyfont
%    added.}
% \changes{v2.2a}{94/02/03}{Use \LaTeXe{} interface}
%    \begin{macrocode}
\@onlypreamble\@xnthm
\@onlypreamble\@ynthm
\@onlypreamble\@othm
\@onlypreamble\newtheorem
\@onlypreamble\theoremstyle
\@onlypreamble\theorembodyfont
\@onlypreamble\theoremheaderfont
%    \end{macrocode}
% \end{macro}
%
% Finally we declare the \texttt{plain} theorem style to be the default.
% \changes{v2.1b}{90/06/04}{Fixed missing `item by declaring default
%    style.}
%    \begin{macrocode}
\theoremstyle{plain}
%</package>
%    \end{macrocode}
%
% \Finale

\endinput
}{}
\ProvidesFile{thmb.sty}
             [\FMithmInfo]
%    \end{macrocode}
%
% \begin{macro}{\th@marginbreak}
%    The \texttt{margin break} style is nearly the same as the
%    \texttt{change break} style. The only difference is the placement
%    of the theorem number.  We use |\llap| to place it in the left
%    margin.
%
%    In this style |\labelsep| denotes the separation between the
%    number and the text.
%    \begin{macrocode}
\gdef\th@marginbreak{\normalfont\slshape
  \def\@begintheorem##1##2{\item
       [\rlap{\vbox{\theorem@headerfont
                    \hbox{\llap{##2}\hskip\labelsep ##1}%
                    \hbox{\strut}}}]}%
\def\@opargbegintheorem##1##2##3{%
  \item[\rlap{\vbox{\theorem@headerfont
                    \hbox{\llap{##2}\hskip\labelsep ##1\ (##3)}%
                    \hbox{\strut}}}]}}
\endgroup
%</thmb>
%    \end{macrocode}
% \changes{v2.1c}{90/02/17}{`endgroup outside macrocode env.}
% \end{macro}
%
%
%
% \subsubsection{The \texttt{margin} style}
%
% This style option is stored in the file |thm.sty|.
%    \begin{macrocode}
%<*thm>
\begingroup \makeatletter
\@ifundefined{theorem@style}{% \iffalse meta-comment
%
% Copyright (C) 1993-2024
%
% The LaTeX Project and any individual authors listed elsewhere
% in this file.
%
% This file is part of the Standard LaTeX `Tools Bundle'.
% -------------------------------------------------------
%
% It may be distributed and/or modified under the
% conditions of the LaTeX Project Public License, either version 1.3c
% of this license or (at your option) any later version.
% The latest version of this license is in
%    https://www.latex-project.org/lppl.txt
% and version 1.3c or later is part of all distributions of LaTeX
% version 2005/12/01 or later.
%
% The list of all files belonging to the LaTeX `Tools Bundle' is
% given in the file `manifest.txt'.
%
% \fi
%
%%
%
%
\def\FMithmInfo{2023/07/05 v2.2c Theorem extension package (FMi)}
%
% \ProvidesFile{theorem.dtx}[\FMithmInfo]
%\iffalse   % this is a METACOMMENT !
%%
%% Package `theorem' to use with LaTeX2e
%% Copyright (C) 1989-1995 Frank Mittelbach, all rights reserved.
%<+package>\@ifundefined{theorem@style}{}{\endinput}
%<+package>\NeedsTeXFormat{LaTeX2e}[1995/06/01]
%<+package>\ProvidesPackage{theorem}
%<+package>         [\FMithmInfo]
%
% \fi
%
% \changes{v1.0}{87/05/07}{First draft.}
% \changes{v1.0a}{87/05/07}{`thistheoremstyle' added.}
% \changes{v1.1a}{87/07/29}{New concept: redefine the internal
%                          \protect\LaTeX{} commands `@xnthm' etc.}
% \changes{v1.1b}{87/07/31}{`theoremfont' used to specify font.}
% \changes{v1.2a}{88/01/08}{`@endtheorem' now globally defined.}
% \changes{v1.2b}{88/01/11}{`newtheorem' only allowed in the preamble.}
% \changes{v1.3a}{88/02/14}{`theoremfont' now a toks register.}
%
% \changes{v2.0a}{88/11/16}{New implementation and new documentation.}
% \changes{v2.0a}{88/11/16}{No layout changes but internal modifications
%                           old substyles won't work.}
% \changes{v2.0b}{89/01/30}{`global' used for all definitions.}
% \changes{v2.0c}{89/06/15}{Documentation translated to English.
%                            Thanks to LLL and Nina}
% \changes{v2.1a}{90/02/15}{Support for new font selection.}
% \changes{v2.2a}{94/02/03}{Cleanup for LaTeX2e}
%
% \DoNotIndex{\@endparenv,\@ifdefinable,\@ifnextchar,\@ifundefined}
% \DoNotIndex{\@namedef,\@nameuse,\@spaces,\@tempa,\@warning}
% \DoNotIndex{\ }
% \DoNotIndex{\begingroup,\bfseries}
% \DoNotIndex{\csname}
% \DoNotIndex{\def,\do,\docdate}
% \DoNotIndex{\endcsname,\endgroup,\endinput,\endtrivlist,\expandafter}
% \DoNotIndex{\filedate,\fileversion}
% \DoNotIndex{\gdef,\global}
% \DoNotIndex{\hbox,\hskip}
% \DoNotIndex{\ignorespaces,\input,\itshape,\item}
% \DoNotIndex{\labelsep,\let,\llap}
% \DoNotIndex{\makeatletter}
% \DoNotIndex{\noexpand}
% \DoNotIndex{\refstepcounter,\relax,\rlap}
% \DoNotIndex{\setlength,\slshape,\space,\string,\strut}
% \DoNotIndex{\the,\trivlist,\typeout}
% \DoNotIndex{\vbox}
% \DoNotIndex{\xdef}
%
% \def\deflabel#1{\ttfamily #1\hfill}
% \def\deflist#1{\begin{list}{}{\settowidth\labelwidth{\ttfamily #1}%
%                               \setlength\leftmargin\labelwidth
%                               \addtolength\leftmargin\labelsep
%                               \let\makelabel\deflabel}}
% \def\enddeflist{\end{list}}
% \def\envfont{\normalfont\ttfamily}
%
% \MakeShortVerb{\|}
%
% \newcommand{\TUB}{{\sl TUGboat\/}}
%
% \GetFileInfo{theorem.dtx}
% \title{An Extension of the
%        \LaTeX{} theorem environment\thanks{This file
%        has version number \fileversion, last
%        revised \filedate.}}
% \author{Frank Mittelbach}
%
% \MaintainedByLaTeXTeam{tools}
% \maketitle
%
% \begin{abstract}
%    The macros described in this paper yield an extension of the
%    \LaTeX{} theorem mechanism. It is designed to satisfy the
%    different requirements of various journals. Thus, the
%    layout of the ``theorems'' can be manipulated by determining a
%    ``style''. This article describes not only the use, but
%    also the definition, of the necessary macros.
% \end{abstract}
%
% \section*{Preface to version 2.2}
%
% For \LaTeXe{} this package did not need any fundamental changes. I
% only modified the messages generated so that theorem layout styles
% will show up with the |\listfiles| command and cleaned the section
% on the New Font Selection Scheme since this is now included in
% \LaTeX.
%
% \section*{Preface to version 2.1}
%
% This version is identical to 2.0g described in \TUB\ 10\#3 except
% for some internal defaults which are now set depending on the used
% font selection scheme.
%
% This was done to avoid unpleasant surprises if the new font selection
% scheme is in force. For further details see
% section~\ref{sec:newfonts} and \cite{bk:GMS94}.
%
% \section{Introduction}
%
% For our purposes here, ``theorems'' are labelled enunciations,
% often set off from the main text by extra space and a font change.
% Theorems, corollaries, conjectures, definitions, and remarks
% are all instances of ``theorems''.  The ``header'' of these structures
% is composed of a label (such as \textsc{Theorem} or \textsc{Remark})
% and a number which serializes an item in the sequence of items
% with the same label.
%
% Shortly after the introduction of \LaTeX{} at the Fachbereich
% Mathematik in Mainz, the desire to manipulate the layout of
% ``theorems'' arose.  In Mainz, the following two conventions came
% into general use:
% \begin{enumerate}
%   \item The number of the theorem is shown in the margin.
%   \item There is a line break at
%         the end of the theorem header.
% \end{enumerate}
%
% Additionally, some journals require different formats which depend on
% the ``sort of theorem'': e.g.\ often remarks and definitions are set
% in |\upshape|, while |\itshape| is employed for main theorems.
%
% Confronted with these requirements, a theorem environment was
% developed in Mainz which allows separate determination of the
% layout of the ``theorems sets'', comparable to |\pagestyle|.
%
%
% \section{The user interface}
% \subsection{Defining new theorem sets}
%
% \DescribeMacro\newtheorem
% As in the original \LaTeX{} version, the command |\newtheorem|
% defines a new ``theorem set'' or ``theorem-like structure''.
% Two required arguments name the new environment and give the text
% to be typeset with each instance of the new ``set'', while
% an optional argument determines how
% the ``set'' is enumerated:
% \begin{description}
%    \item[\ttfamily \bslash newtheorem\{foo\}\{bar\}]
%       The theorem set {\envfont foo} (whose name is \texttt{bar})
%       uses its own counter.
%    \item[\ttfamily \bslash newtheorem\{foo2\}{[foo]}\{bar2\}]
%       The theorem set {\envfont foo2} (printed name \texttt{bar2})
%       uses the same counter as the theorem set \texttt{foo}.
%    \item[\ttfamily \bslash newtheorem\{foo3\}\{bar3\}{[section]}]
%       The theorem set {\envfont foo3} (printed name \texttt{bar3}) is
%       enumerated within the counter \texttt{section}, i.e.\ with every
%       new |\section| the enumeration begins again with 1, and
%       the enumeration is composed from the section-number and the
%       theorem counter itself.
% \end{description}
%
% \DescribeMacro\theoremstyle
% Additionally, the command |\theoremstyle| can define the layout
% of various, or all, theorem sets.  It should be noted that any
% theorem set defined by |\newtheorem| is typeset in the
% |\theoremstyle| that is current at the time of the definition.
% Thus, the following
% \begin{verbatim}
%    \theoremstyle{break}        \newtheorem{Cor}{Corollary}
%    \theoremstyle{plain}        \newtheorem{Exa}{Example}[section]
%\end{verbatim}                             ^^A No blank at line start
% leads to the result that the set {\envfont Cor} is formatted in the
% style \texttt{break}, while the set {\envfont Exa} and all the
% following ones are formatted in the style \texttt{plain}, unless
% another |\theoremstyle| follows.  Since the definitions installed by
% |\newtheorem| are global, one also can limit |\theoremstyle| locally
% by grouping braces.
%
% \DescribeMacro\theorembodyfont
% The choice of the font for the theorem body is completely
% independent of the chosen |\theoremstyle|; this has proven to
% be very advantageous.  For example,
% \begin{verbatim}
%   {\theorembodyfont{\upshape}          \newtheorem{Rem}{Remark}}
%\end{verbatim}
% defines a theorem set {\envfont Rem}, which will be set in
% |\upshape| in the current layout (which in our example is
% \texttt{plain}).  As with |\theoremstyle|, the |\theorembodyfont|
% chosen is that current at the time of |\newtheorem|.  If
% |\theorembodyfont| is not specified or one defines
% |\theorembodyfont{}|, then the font used will be that defined by the
% |\theoremstyle|.
%
% \DescribeMacro\theoremheaderfont It is also possible to customize
% the font used for the theorem headers.  This is, however, a global
% declaration, and therefore there should be at most one
% |\theoremheaderfont| declaration in the preamble.\footnote{If it is
% actually necessary to have different header fonts, one has to define
% new theorem styles (substituting the desired font) or specify the
% information directly in the \texttt{\bslash newtheorem} declaration
% (the unclean variant).}
%
% \DescribeMacro\theorempreskipamount
% \DescribeMacro\theorempostskipamount Two additional parameters
% affect the vertical space around the theorem environments:
% |\theorempreskipamount| and |\theorempostskipamount| define,
% respectively, the spacing before and after such an environment.
% These parameters apply for all theorem sets and can be manipulated
% with the ordinary length macros.  They are rubber lengths
% (`\textsf{skips}'), and therefore can contain \texttt{plus} and
% \texttt{minus} parts.
%
% Since the definition of theorem sets should---most sensibly---be
% placed in the preamble, we only allow installation there.
% It is therefore possible to release the memory used here after
% |\begin{document}|, in order to make room for other
% applications.
%
%
% \subsection{Existing theorem styles}
%
% The following theorem styles exist to date:
% \begin{deflist}{marginbreak}
%    \item[plain]
%       This theorem style emulates the original \LaTeX{} definition,
%       except that additionally the parameters
%       |\theorem...skipamount| are used.
%    \item[break]
%       In this style, the theorem header is followed by a line
%       break.
%    \item[marginbreak]
%       The theorem number is set in the margin, and there is a line
%       break as in \texttt{break}.
%    \item[changebreak]
%       Like \texttt{break}, but with header number and text
%       interchanged.
%    \item[change]
%       Header number and text are interchanged, without a line break.
%    \item[margin]
%       The number is set in the left margin, without a line break.
% \end{deflist}
% All styles (except \texttt{plain}) select |\slshape| as the default
% |\theorembodyfont|.
%
%
% \subsection{Examples}
%
% Given the above theorem sets {\envfont Cor}, {\envfont Exa}
% and {\envfont Rem},
% suppose that the preamble also contains the declarations:
% \begin{verbatim}
%    \theoremstyle{marginbreak}   \newtheorem{Lem}[Cor]{Lemma}
%    \theoremstyle{change}
%    \theorembodyfont{\itshape}        \newtheorem{Def}[Cor]{Definition}
%
%    \theoremheaderfont{\scshape}
%\end{verbatim}
% Then the following are some typical examples of the typeset output
% resulting from their use.
%
% \begin{Cor}
%    This is a sentence typeset in the theorem environment {\envfont
%    Cor}.
% \end{Cor}
%
% \begin{Exa}
%    This is a sentence typeset in the theorem environment {\envfont
%    Exa}.
% \end{Exa}
%
% \begin{Rem}
%    This is a sentence typeset in the theorem environment {\envfont
%    Rem}.
% \end{Rem}
%
% \begin{Lem}[Ben User]
%    This is a sentence typeset in the theorem environment {\envfont
%    Lem}.
% \end{Lem}
%
% \begin{Def}[Very impressive Definition]
%    This is a sentence typeset in the theorem environment {\envfont
%    Def}.
% \end{Def}
%
% The last two examples show the effect of the optional argument to
% a theorem environment (it is the text typeset in parentheses).
%
%
% \section{Special Considerations}
% \label{sec:newfonts}
%
% Theoremheader and body are implemented as a unit. This means that
% the |\theoremheaderfont|  will inherit characteristics of the
% |\theorembodyfont| in \LaTeXe. Thus, if for example
% |\theorembodyfont| is |\itshape| and |\theoremheaderfont|
% is |\bfseries| the font selected for the header will have the
% characteristics `bold extended italic'. If this is not desired one
% should set the |\theoremheaderfont| to something like
% \begin{verbatim}
%      \theoremheaderfont{\normalfont\bfseries}
%\end{verbatim}
% i.e.\ supplying all necessary font information explicitly.
%
%
%
% \section{Acknowledgements}
%
% The publication of this set of macros was only possible with the
% help of Christina Busse (translating the manuscript into English),
% Joachim Pense (playing the r\^ole of typist), Chris Rowley
% (looking everything over) and many others providing useful
% suggestions.
%
% \MaybeStop{%
%    \begin{thebibliography}{9}
%       \bibitem{bk:GMS94} \textsc{M.~Goossens}, \textsc{F.~Mittelbach}
%         and \textsc{A.~Samarin}.
%         \newblock The \LaTeX{} Companion.
%         \newblock
%         Addison-Wesley, Reading, Massachusetts, 1994.
%       \bibitem{latex:source} \textsc{Lamport, Leslie.} \newblock
%                               \texttt{latex.tex}, version 2.09, date
%                               Feb. 1990.
%    \end{thebibliography}
%    \PrintIndex
%    \PrintChanges}
%
%
% \section{The documentation driver file}
%
% The next bit of code contains the documentation driver file for
% \TeX{}, i.e., the file that will produce the documentation you are
% currently reading. It will be extracted from this file by the
% \texttt{docstrip} program. Since it is the first code in the file
% one can alternatively process this file directly with \LaTeXe{} to
% obtain the documentation.
%
%    \begin{macrocode}
%<*driver>
\documentclass{ltxdoc}

\usepackage{theorem}

%<+driver>% The next few lines define theorem sets which are used
%<+driver>% in the example section of the documentation.

\theoremstyle{break}       \newtheorem{Cor}{Corollary}
\theoremstyle{plain}       \newtheorem{Exa}{Example}[section]
{\theorembodyfont{\upshape}\newtheorem{Rem}{Remark}}
\theoremstyle{marginbreak} \newtheorem{Lem}[Cor]{Lemma}
\theoremstyle{change}
\theorembodyfont{\itshape} \newtheorem{Def}[Cor]{Definition}

\theoremheaderfont{\scshape}

\RecordChanges

\begin{document}
   \DocInput{theorem.dtx}
\end{document}
%</driver>
%    \end{macrocode}
%
% \section{Definition of the Macros}
%
% If the file has been loaded before, we abort immediately. If not the
% package announces itself (this is actually done at the very top of
% the file---the way it is done isn't good style so don't copy it).
% \changes{v2.0e}{89/07/19}{Spaces removed from `typeout'.}
% \changes{v2.2b}{95/11/19}{Announce moved to top of file}
%    \begin{macrocode}
%<*package>
%\@ifundefined{theorem@style}{}{\endinput}
%\def\FMithmInfo{1995/11/19 v2.2b Theorem extension package (FMi)}
%\ProvidesPackage{theorem}[\FMithmInfo]
%    \end{macrocode}
%
%
% \subsection{Definition of theorem styles and fonts}
%
% All the definitions in this file are done globally to allow
% inputting this file inside a group.
%
% \begin{macro}{\theoremstyle}
%    Before a theorem style can be installed, the chosen style must be
%    known. For that reason, we must test to see that
%    |\th@|\meta{style} is known or, more precisely, that it is
%    different from |\relax|. If the style is not known then
%    |\th@plain| is used.
%    \begin{macrocode}
\gdef\theoremstyle#1{%
   \@ifundefined{th@#1}{\@warning
          {Unknown theoremstyle `#1'. Using `plain'}%
          \theorem@style{plain}}%
%    \end{macrocode}
%    We save the theorem style to be used in the token register
%    |\theorem@style|.
%    \begin{macrocode}
      {\theorem@style{#1}}%
%    \end{macrocode}
%    Now we ``evaluate'' the theorem style: this means, we call the
%    macro |\th@|\meta{style} which will activate the relevant
%    definitions which are contained in a separate file.  This is done
%    in a group to suppress changes to the current font.  This could
%    otherwise pose problems together with the new font selection
%    scheme\footnote{When I printed the original article using the new
%    font selection scheme I ended with a document with slanted
%    typefaces (text headings and all) simply because one of the
%    theorem styles used \texttt{\bslash sl} at toplevel.} if the
%    |\th@|\meta{style} is evaluated a second time.
% \changes{v2.1a}{90/02/15}{Group added around `th@..}
% \changes{v2.1a}{90/02/15}{Reseting of `..@begintheorem removed.}
%    \begin{macrocode}
      \begingroup
        \csname th@\the\theorem@style \endcsname
      \endgroup}
%    \end{macrocode}
% \end{macro}
%
%
% \begin{macro}{\@begintheorem}
% \changes{v2.1a}{90/02/15}{Relaxed at toplevel.}
% \begin{macro}{\@opargbegintheorem}
% \changes{v2.1a}{90/02/15}{Relaxed at toplevel.}
%    We reset |\@begintheorem| and |\@opargbegintheorem| to
%    |\relax| since these commands are no longer necessary at
%    toplevel. This will save a few tokens.
%    \begin{macrocode}
\global\let\@begintheorem\relax
\global\let\@opargbegintheorem\relax
%    \end{macrocode}
% \end{macro}
% \end{macro}
%
% \begin{macro}{\theorem@style}
%    Obviously the \textsf{token} register used above has to be
%    allocated. To assure the utmost compatibility with the
%    original \LaTeX{} definition, we set the default theorem style to
%    \texttt{plain}, which implements the usual \LaTeX{} convention.
% \changes{v2.0f}{89/07/20}{Unnecessary `global' removed.}
%    \begin{macrocode}
\newtoks\theorem@style
\global\theorem@style{plain}
%    \end{macrocode}
% \end{macro}
%
% \begin{macro}{\theorembodyfont}
% \begin{macro}{\theorem@bodyfont}
%    If the |\theorembodyfont| is set by the user then it should not
%    interact with the default font set in the theorem style. When the
%    new font selection is in force this may happen if, for example,
%    the default is |\itshape| and the new |\theorembodyfont| is
%    |\sffamily|. So we add a |\reset@font| command in front of the
%    user definition.
% \changes{v2.0f}{89/07/20}{Unnecessary `global' removed.}
% \changes{v2.0g}{89/09/19}{`theoremfont to `theorembodyfont.}
% \changes{v2.1c}{92/06/22}{`reset@font added and logic changed.}
% \changes{v2.1d}{92/10/28}{Reset to empty if arg is empty}
%    \begin{macrocode}
\gdef\theorembodyfont#1{%
%    \end{macrocode}
%    We check if the argument supplied is empty and if so put nothing
%    into the |\theorem@bodyfont| token register to allow for
%    |\theorembodyfont{}| as a mean of using the default of the
%    current |\theoremstyle|.
%    \begin{macrocode}
   \def\@tempa{#1}%
   \ifx\@tempa\@empty
    \theorem@bodyfont{}%
   \else
    \theorem@bodyfont{\reset@font#1}%
   \fi
   }
\newtoks\theorem@bodyfont
\global\theorem@bodyfont{}
%    \end{macrocode}
% \end{macro}
% \end{macro}
%
%
% \begin{macro}{\theoremheaderfont}
% \changes{v2.0g}{89/09/19}{Macro added.}
%    The font for the theorem headers is handled differently because
%    this definition applies to all theorem styles.
%    \begin{macrocode}
\gdef\theoremheaderfont#1{\gdef\theorem@headerfont{#1}%
%    \end{macrocode}
%    After using the macro once it is redefined to produce an error
%    message.
%    \begin{macrocode}
       \gdef\theoremheaderfont##1{%
        \typeout{\string\theoremheaderfont\space should be used
                 only once.}}}
%    \end{macrocode}
% \end{macro}
%
% \begin{macro}{\theorem@headerfont}
% \changes{v2.0g}{89/09/19}{Macro added.}
%    To set the |\theorem@headerfont| default we first test if
%    the new fontselection scheme is in force.
% \changes{v2.1a}{90/02/15}{Support for new font selection.}
%    \begin{macrocode}
\ifx\upshape\undefined
%    \end{macrocode}
%    If not we define it to expand into |\bfseries|. We don't use
%    |\let| just in case a following style option redefines this
%    macro.
%    \begin{macrocode}
\gdef\theorem@headerfont{\bfseries}
%    \end{macrocode}
%    Otherwise we reset the current shape before calling |\bfseries|.
%    \begin{macrocode}
\else \gdef\theorem@headerfont{\normalfont\bfseries}\fi
%    \end{macrocode}
% \end{macro}
%
% \begin{macro}{\th@plain}
% \begin{macro}{\th@break}
% \begin{macro}{\th@marginbreak}
% \begin{macro}{\th@changebreak}
% \begin{macro}{\th@change}
% \begin{macro}{\th@margin}
%    The different styles are defined in macros such as
%    |\th@plain|.  Since memory space is precious in
%    ``non-Big-versions'', we have to avoid offering too many unused
%    definitions.  Therefore we define these styles in separate files
%    that can be loaded on demand.  Thus the commands themselves only
%    load these files. We use |\@input@|, a \LaTeXe{} internal command
%    that ensures that the file will be listed with |\listfiles|.
% \changes{v2.2a}{94/02/02}{Use `@input@ to load theorem layout files}
%    \begin{macrocode}
\gdef\th@plain{\@input@{thp.sty}}
\gdef\th@break{\@input@{thb.sty}}
\gdef\th@marginbreak{\@input@{thmb.sty}}
\gdef\th@changebreak{\@input@{thcb.sty}}
\gdef\th@change{\@input@{thc.sty}}
\gdef\th@margin{\@input@{thm.sty}}
%    \end{macrocode}
%    This list will be expanded when new styles become
%    available.  For testing,  just append new theorem substyles
%    as document options.
% \end{macro}
% \end{macro}
% \end{macro}
% \end{macro}
% \end{macro}
% \end{macro}
%
%
%
% \subsection{Definition of a new theorem set}
%
% As already pointed out, a new theorem environment can be defined in
% three different ways:
% \begin{verbatim}
%    \newtheorem{Lem}{Lemma}
%    \newtheorem{Lem}{Lemma}[section]
%    \newtheorem{Lem}[Theorem]{Lemma}
%\end{verbatim}
% The function of the macro |\newtheorem| is to recognize these cases
% and then to branch into one of the three macros |\@ynthm|, |\@xnthm|
% or |\@othm|.  This mechanism is adopted unchanged from
% \cite{latex:source}; the essential point here is that, for example,
% in the second case, the arguments \texttt{Lem}, \texttt{Lemma} and
% \texttt{section} are passed over to the macro |\@xnthm|.
%
% We inspect this case first because the others present fewer problems,
% and thus are easily derived from this one.
%
% \begin{macro}{\@xnthm}
%    For our example arguments, the macro |\@xnthm| must fulfill
%    the following:
%    \begin{itemize}
%       \item
%          Define a new \LaTeX-counter `Lem'
%       \item
%          reset this counter within a |\section|
%       \item
%          define the macro |\theLem|
%       \item
%          define the environment macros |\Lem| and
%          |\endLem| using the current
%          |\theoremstyle| and |\theorem@bodyfont|.
%    \end{itemize}
%    Obviously, all this should happen only if the first argument
%    of |\@xnthm| (i.e.\ \texttt{Lem} in our example) is chosen so
%    as not to conflict with any previously defined commands or
%    environments.  This test is performed by the \LaTeX{} macro
%    |\@ifdefinable|.
%    \begin{macrocode}
\gdef\@xnthm#1#2[#3]{\expandafter\@ifdefinable\csname #1\endcsname
%    \end{macrocode}
%    Therefore, the first argument of |\@ifdefinable| is the
%    expansion (in the example, |\Lem|) of
%    |\csname||#1\endcsname|.  The second argument is
%    executed only if the test has been completed successfully.
%    \begin{macrocode}
   {%
%    \end{macrocode}
%    Now we define the new counter. The names of the \LaTeX{}
%    macros employed should speak for themselves:
% \changes{v2.2b}{1995/10/07}{Use \cs{@newctr}}
% \changes{v2.2c}{1995/11/23}{Use \cs{@newctr} properly.}
%    \begin{macrocode}
    \@definecounter{#1}\@newctr{#1}[#3]%
%    \end{macrocode}
%    Using |\@newctr| will give a proper error message if the counter
%    in |#3| is not defined.
%    In defining `|\theLem|' we must generate the desired macro
%    name by use of |\expandafter| and |\csname|.
%    \begin{macrocode}
    \expandafter\xdef\csname the#1\endcsname
%    \end{macrocode}
%    An |\xdef| is used in order to make the definition global,
%    and to ensure that it contains the replacement texts of
%    |\@thmcountersep| and |\@thmcounter|.\footnote{These
%    two macros can be defined by the document style.  Their default
%    values produce a `\texttt{.}' as separation and an arabic
%    representation of the number.} However, not everything should be
%    expanded. For example, it saves space to use |\thesection|
%    instead of its---at times---lengthy expansion.
%    \begin{macrocode}
      {\expandafter \noexpand \csname the#3\endcsname
       \@thmcountersep \@thmcounter{#1}}%
%    \end{macrocode}
%    Thus with the defaults of \LaTeX{}, |\theLem| would be
%    replaced by the command sequence |\thesection.\arabic{Lem}|.
%
%    We will now look at the definition of the macro which is executed
%    at the beginning of the actual environment (in our example this
%    macro is |\Lem|). It should be noted that we use an
%    ``|\expandafter| trick'' to expand only certain parts of the
%    replacement text at the time of the definition.
%    \begin{macrocode}
    \def\@tempa{\global\@namedef{#1}}%
    \expandafter \@tempa \expandafter{%
%    \end{macrocode}
%    First, the macro that contains the current definitions of
%    |\@begintheorem| and |\@opargtheorem| should be called
%    up.  The name of this macro---as is already known---has the form
%    |\th@|\meta{theorem\ style}; therefore, it must be called by
%    \begin{macrocode}
      \csname th@\the \theorem@style
%    \end{macrocode}
%    In addition the default theorem font should be changeable, i.e.\
%    we have to insert the contents of |\theorem@bodyfont|. For
%    that reason, we expand even further, beyond |\endcsname|,
%    and thus insert the contents of the \textsf{token} register
%    |\theorem@bodyfont| in the replacement text.
%    \begin{macrocode}
            \expandafter \endcsname \the \theorem@bodyfont
%    \end{macrocode}
%    Now it is time to call the macro |\@thm| which takes over
%    the further processing.  It has two arguments: the current
%    counter name (in our example, \texttt{Lem}), and the text of the
%    label (in our example, \texttt{Lemma}).
%    \begin{macrocode}
     \@thm{#1}{#2}}%
%    \end{macrocode}
%    With this, the `sub-definition' is complete.  The macro
%    |\@endtheorem| ends a theorem environment and is, so far, nothing
%    but an |\endtrivlist|.  (Hence it is defined globally, and not
%    within the theorem styles.\footnote{This has to be changed as
%    soon as theorem styles that change \texttt{\bslash @endtheorem}
%    exist.  In such a case, all existing styles must be changed as
%    well since they will have to reset the macro.}) Therefore, we can
%    set it equivalent to the macro that ends the theorem set (in our
%    example, |\endLem|).  However, if some day theorem styles exist
%    that do change |\@endtheorem|, we would have to use the
%    commented-out line instead.
%    \begin{macrocode}
    \global \expandafter \let \csname end#1\endcsname \@endtheorem
%   \global\@namedef{end#1}{\@endtheorem}%
%    \end{macrocode}
%    With these commands all the required definitions are employed,
%    unless the test |\@ifdefinable| has failed.  Therefore, we
%    end the second argument of this macro and with it the definition
%    of |\@xnthm|.
%    \begin{macrocode}
   }}
%    \end{macrocode}
% \end{macro}
%
% \begin{macro}{\@ynthm}
%    The definition of |\@ynthm| is completely analogous. In
%    this case the new counter that is defined is not reset within
%    another counter; thus the definition of |\the...|
%    is simplified:
%    \begin{macrocode}
\gdef\@ynthm#1#2{\expandafter\@ifdefinable\csname #1\endcsname
   {\@definecounter{#1}%
    \expandafter\xdef\csname the#1\endcsname{\@thmcounter{#1}}%
%    \end{macrocode}
%    The rest of the definition corresponds literally to that of
%    |\@xnthm|:
%    \begin{macrocode}
    \def\@tempa{\global\@namedef{#1}}\expandafter \@tempa
     \expandafter{\csname th@\the \theorem@style \expandafter
     \endcsname \the\theorem@bodyfont \@thm{#1}{#2}}%
    \global \expandafter \let \csname end#1\endcsname \@endtheorem}}
%    \end{macrocode}
% \end{macro}
%
% \begin{macro}{\@othm}
%    The definition of |\@othm| does not contain anything new.
%    \begin{macrocode}
\gdef\@othm#1[#2]#3{%
%    \end{macrocode}
% \changes{v2.2b}{95/10/07}{Build in check for counter defined}
%    We do not define a new counter but instead use one that has
%    already been defined.  Thus the only definition we need is that
%    of this pseudo-counter (i.e.\ |\the|\meta{env.\ name}).
%    First we check if |#2| corresponds to a known counter name.
%    \begin{macrocode}
  \expandafter\ifx\csname c@#2\endcsname\relax
   \@nocounterr{#2}%
  \else
   \expandafter\@ifdefinable\csname #1\endcsname
   {\expandafter \xdef \csname the#1\endcsname
     {\expandafter \noexpand \csname the#2\endcsname}%
%    \end{macrocode}
%    All other parts of the definition can be adopted from
%    |\@xnthm|. We have to remember, though, that in this case
%    the name of the current counter and the theorem label have moved
%    to the second and third arguments.
%    \begin{macrocode}
    \def\@tempa{\global\@namedef{#1}}\expandafter \@tempa
     \expandafter{\csname th@\the \theorem@style \expandafter
     \endcsname \the\theorem@bodyfont \@thm{#2}{#3}}%
    \global \expandafter \let \csname end#1\endcsname \@endtheorem}%
  \fi}
%    \end{macrocode}
% \end{macro}
%
% \subsection{Macros that are employed in a {\envfont theorem}
%    environment}
%
% \begin{macro}{\@thm}
%    The macro |\@thm| has to increase the current counter.  Then,
%    depending on whether the environment has (or does not have) an
%    optional argument, it has to branch into either |\@begintheorem|
%    or |\@opargtheorem|.
%    \begin{macrocode}
\gdef\@thm#1#2{\refstepcounter{#1}%
%    \end{macrocode}
%    Now we start a {\envfont trivlist} environment, and give
%    |\@topsep| and |\@topsepadd| the values of the \textsf{skip}
%    registers |\theorempreskipamount| and |\theorempostskipamount|.
%    The value in |\@topsep| is the vertical space that is inserted by
%    the first (and only) |\item| in our |\trivlist| whilst
%    |\@topsepadd| is inserted by |\@endparenv| at the end of that
%    {\envfont trivlist} environment.  By using these registers, we
%    obtain the desired space around a {\envfont theorem} environment.
%    \begin{macrocode}
   \trivlist
   \@topsep \theorempreskipamount               % used by first \item
   \@topsepadd \theorempostskipamount           % used by \@endparenv
%    \end{macrocode}
%    Now we have to test whether an optional argument has been given.
% \changes{v2.0d}{89/07/18}{Forgotten percent added.}
%    \begin{macrocode}
   \@ifnextchar [%
%    \end{macrocode}
%    If there is an optional argument, we will call |\@ythm|, and
%    move the arguments read back into the input stream.
%    \begin{macrocode}
   {\@ythm{#1}{#2}}%
%    \end{macrocode}
%    If not, we call |\@begintheorem|. Its first argument is the
%    name of the theorem set (hence the second argument of
%    |\@thm|). Its second argument is the macro that produces the
%    current number.
%    \begin{macrocode}
   {\@begintheorem{#2}{\csname the#1\endcsname}\ignorespaces}}
%    \end{macrocode}
% \end{macro}
%
% \begin{macro}{\@xthm}
% \begin{macro}{\@ythm}
%    Both these macros were originally called by |\@thm|.  We do
%    not need |\@xthm| anymore, hence we reset it to
%    |\relax|. The definition of |\@ythm| has not changed at
%    all from its definition in \LaTeX. In order to make the macros
%    easier to understand, we will
%    nevertheless present it (commented out).
%    \begin{macrocode}
\global\let\@xthm\relax
% \def\@ythm#1#2[#3]{\@opargbegintheorem{#2}{\csname
%           the#1\endcsname}{#3}\ignorespaces}
%</package>
%    \end{macrocode}
%    The primitive |\ignorespaces| in |\@ythm| and
%    |\@thm| is needed to remove the spaces between the
%    |\begin{...}| and the actual text.
% \end{macro}
% \end{macro}
%
%
% \subsection{Definition of the theorem substyles}
%
% As already pointed out, the theorem substyles, defined below, are
% only loaded when necessary.  Note that all these substyles, except
% \texttt{plain}, have |\slshape| as the default body font.
%
% \subsubsection{The \texttt{plain} style}
%
% \label{dok:th@plain} As the following macros use |@|, we have
% to locally set the |\catcode| of this symbol to ``letter''.
% This happens within a group, so that we do not have to worry about
% which |\catcode| that symbol had before.
%    \begin{macrocode}
%<*thp>
\begingroup \makeatletter
%    \end{macrocode}
% Since we are now within a group, we must make all definitions
% globally.  First we make sure that \texttt{theorem.sty} is loaded.
% This will allow us to use this file as a document style option
% without having to call \texttt{theorem} itself as an option.  At the
% same time, we assure that at least version 2 is loaded, since
% |\theorem@style| was not defined in earlier versions.
%    \begin{macrocode}
\@ifundefined{theorem@style}{\input{theorem.sty}}{}
\ProvidesFile{thp.sty}
             [\FMithmInfo]
%    \end{macrocode}
%
%
% \begin{macro}{\th@plain}
% \begin{macro}{\@begintheorem}
% \begin{macro}{\@opargbegintheorem}
%    |\theoremstyle{plain}| corresponds to the original
%    definition, except that the distances to the surrounding text are
%    determined by the parameters |\theorempreskipamount| and
%    |\theorempostskipamount|.
%    First we set the default body font.
%    \begin{macrocode}
\gdef\th@plain{\normalfont\itshape
%    \end{macrocode}
%    Then we define |\@begintheorem| and
%    |\@opargbegintheorem|. These two macros define how the
%    header of a theorem is typeset.  |\@opargbegintheorem| will
%    be called if a \textsf{theorem} environment with an optional
%    argument is encountered; otherwise, the header is constructed by
%    calling |\@begintheorem|.  If one of these macros is
%    executed, we are within a \textsf{trivlist} environment started by
%    |\@thm|.  So the theorem header is produced with an
%    |\item| command.
%
%    Instead of specifying the header font directly, all standard
%    theorem styles use the |\theorem@headerfont| macro to allow
%    customization.  The extra space (|\labelsep|) is necessary
%    because of problems in the \textsf{trivlist} environment.
%    \begin{macrocode}
  \def\@begintheorem##1##2{%
        \item[\hskip\labelsep \theorem@headerfont ##1\ ##2]}%
%    \end{macrocode}
%    The definition of |\@opargbegintheorem| is completely
%    analogous. The only difference is the fact that there exists a
%    third argument (which is the optional parameter of the
%    environment and contains additional information about the
%    theorem).  Customarily we enclose it in parentheses.
%    \begin{macrocode}
\def\@opargbegintheorem##1##2##3{%
   \item[\hskip\labelsep \theorem@headerfont ##1\ ##2\ (##3)]}}
%    \end{macrocode}
% \end{macro}
% \end{macro}
% \end{macro}
%
% \noindent
% We conclude with an |\endgroup| to restore the
% |\catcode| of |@|.
%    \begin{macrocode}
\endgroup
%</thp>
%    \end{macrocode}
%
%
% \subsubsection{The \texttt{break} style}
%
% This style option is stored in the file |thb.sty|.
% For the next two lines see the documentation for |\th@plain| on
% page~\pageref{dok:th@plain}.
%    \begin{macrocode}
%<*thb>
\begingroup \makeatletter
\@ifundefined{theorem@style}{\input{theorem.sty}}{}
\ProvidesFile{thb.sty}
             [\FMithmInfo]
%    \end{macrocode}
%
% \begin{macro}{\th@break}
%    |\theoremstyle{break}| produces a line break after the name
%    of the theorem.  The font is |\slshape|.  Hence, we define
%    |\th@break| as follows:
%    \begin{macrocode}
\gdef\th@break{\normalfont\slshape
  \def\@begintheorem##1##2{\item[%
%    \end{macrocode}
%    We run into the following problem: it is not possible to create
%    the header with |\item[|\meta{title}|]| and then start
%    a new line by, for example, |\mbox{}\\|. Such a definition
%    will fail whenever a list environment follows immediately.  With
%    the above construction, the |\mbox{}| causes the switch
%    |@inlabel| (cf.\ definition of |\list| and
%    |\trivlist| in \cite{latex:source}) to be set to \texttt{false}
%    and so the following list will insert additional vertical
%    space (|\topskip|). This is quite annoying. Therefore, we
%    create the line break within the |\item|. In order to ensure
%    that the text will begin at the proper position in the following
%    line, we simply pretend that the label does not take any
%    room.\footnote{This will lead to problems whenever very high
%    symbols occurring in the line tower into the
%    heading. So, something else has to be done here sometime.}
%    \begin{macrocode}
       \rlap{\vbox{\hbox{\hskip \labelsep\theorem@headerfont ##1\ ##2}%
                   \hbox{\strut}}}]}%
%    \end{macrocode}
%    Again, the definition of |\@opargbegintheorem| is
%    completely analogous.
%    \begin{macrocode}
\def\@opargbegintheorem##1##2##3{%
  \item[\rlap{\vbox{\hbox{\hskip \labelsep \theorem@headerfont
                     ##1\ ##2\ (##3)}%
                    \hbox{\strut}}}]}}
\endgroup
%</thb>
%    \end{macrocode}
% \end{macro}
%
%
% \subsubsection{The \texttt{changebreak} style}
%
%    \begin{macrocode}
% This style option is stored in the file |thcb.sty|.
%    \begin{macrocode}
%<*thcb>
\begingroup \makeatletter
\@ifundefined{theorem@style}{\input{theorem.sty}}{}
\ProvidesFile{thcb.sty}
             [\FMithmInfo]
%    \end{macrocode}
%
% \begin{macro}{\th@changebreak}
%    The \texttt{change-break} theorem style is like \texttt{break}
%    but with interchange of theorem name and theorem number.  Thus we
%    define |\th@changebreak| as follows:
%    \begin{macrocode}
\gdef\th@changebreak{\normalfont\slshape
  \def\@begintheorem##1##2{\item
       [\rlap{\vbox{\hbox{\hskip\labelsep \theorem@headerfont ##2\ ##1}%
                    \hbox{\strut}}}]}%
\def\@opargbegintheorem##1##2##3{%
  \item[\rlap{\vbox{\hbox{\hskip\labelsep \theorem@headerfont
                     ##2\ ##1\ (##3)}%
                    \hbox{\strut}}}]}}
\endgroup
%</thcb>
%    \end{macrocode}
% \end{macro}
%
%
% \subsubsection{The \texttt{change} style}
%
% This style option is stored in the file |thc.sty|.
%    \begin{macrocode}
%<*thc>
\begingroup \makeatletter
\@ifundefined{theorem@style}{\input{theorem.sty}}{}
\ProvidesFile{thc.sty}
             [\FMithmInfo]
%    \end{macrocode}
%
% \begin{macro}{\th@change}
%    The \texttt{change} theorem style corresponds to the
%    \texttt{change break} style without a linebreak after the header.
%    To say it in another way, it's the same as the \texttt{plain}
%    style but with number and name interchanged and |\slshape| as the
%    default font.
%    \begin{macrocode}
\gdef\th@change{\normalfont\slshape
  \def\@begintheorem##1##2{\item
       [\hskip\labelsep \theorem@headerfont ##2\ ##1]}%
\def\@opargbegintheorem##1##2##3{%
  \item[\hskip\labelsep \theorem@headerfont ##2\ ##1\ (##3)]}}
\endgroup
%</thc>
%    \end{macrocode}
% \end{macro}
%
%
% \subsubsection{The \texttt{marginbreak} style}
%
% This style option is the one used most often at Mainz.  It is saved
% in the file |thmb.sty|.
%    \begin{macrocode}
%<*thmb>
\begingroup \makeatletter
\@ifundefined{theorem@style}{\input{theorem.sty}}{}
\ProvidesFile{thmb.sty}
             [\FMithmInfo]
%    \end{macrocode}
%
% \begin{macro}{\th@marginbreak}
%    The \texttt{margin break} style is nearly the same as the
%    \texttt{change break} style. The only difference is the placement
%    of the theorem number.  We use |\llap| to place it in the left
%    margin.
%
%    In this style |\labelsep| denotes the separation between the
%    number and the text.
%    \begin{macrocode}
\gdef\th@marginbreak{\normalfont\slshape
  \def\@begintheorem##1##2{\item
       [\rlap{\vbox{\theorem@headerfont
                    \hbox{\llap{##2}\hskip\labelsep ##1}%
                    \hbox{\strut}}}]}%
\def\@opargbegintheorem##1##2##3{%
  \item[\rlap{\vbox{\theorem@headerfont
                    \hbox{\llap{##2}\hskip\labelsep ##1\ (##3)}%
                    \hbox{\strut}}}]}}
\endgroup
%</thmb>
%    \end{macrocode}
% \changes{v2.1c}{90/02/17}{`endgroup outside macrocode env.}
% \end{macro}
%
%
%
% \subsubsection{The \texttt{margin} style}
%
% This style option is stored in the file |thm.sty|.
%    \begin{macrocode}
%<*thm>
\begingroup \makeatletter
\@ifundefined{theorem@style}{\input{theorem.sty}}{}
\ProvidesFile{thm.sty}
             [\FMithmInfo]
%    \end{macrocode}
%
% \begin{macro}{\th@margin}
%    Again this is only a variant of the theorem styles described
%    above without any new ideas.
%    \begin{macrocode}
\gdef\th@margin{\normalfont\slshape
  \def\@begintheorem##1##2{\item
       [\theorem@headerfont \llap{##2}\hskip\labelsep ##1]}%
\def\@opargbegintheorem##1##2##3{%
  \item[\theorem@headerfont \llap{##2}\hskip\labelsep ##1\ (##3)]}}
\endgroup
%</thm>
%    \end{macrocode}
% \end{macro}
%
%
%
%
% \subsection{Final Definitions}
%
% \begin{macro}{\theorempreskipamount}
% \begin{macro}{\theorempostskipamount}
%    The \textsf{skip} parameters that regulate the vertical empty space
%    before and after the {\envfont theorem} environment have
%    to be allocated as well.
% \changes{v2.0f}{89/07/20}{Unnecessary `global' removed.}
%    \begin{macrocode}
%<*package>
\newskip\theorempreskipamount
\newskip\theorempostskipamount
%    \end{macrocode}
%    Since we have used the same values for all theorem sets, we now can
%    assign them.
%    \begin{macrocode}
\global\setlength\theorempreskipamount{12pt plus 5pt minus 3pt}
\global\setlength\theorempostskipamount{8pt plus 3pt minus 1.5pt}
%    \end{macrocode}
% \end{macro}
% \end{macro}
%
%
% \begin{macro}{\@endtheorem}
%    The same holds for the macro |\@endtheorem|, which ends a
%    {\envfont theorem} environment. Since it is the same for all
%    theorem sets, it is removed from the macros |\th@|\meta{style}.
%    It simply ends the {\envfont trivlist} environment, which was
%    begun in |\@thm|.
%    \begin{macrocode}
\global\let\@endtheorem=\endtrivlist
%    \end{macrocode}
% \end{macro}
%
%
%
% \begin{macro}{\@preamblecmds}
%    All macros defined above are to be used only in the preamble.
%    Therefore, we insert them in |\@preamblecmds| which will disable
%    them at begin document. This is done by the internal \LaTeXe{}
%    command |\@onlypreamble|.
% \changes{v2.0g}{89/09/19}{`theoremheaderfont and `theorembodyfont
%    added.}
% \changes{v2.2a}{94/02/03}{Use \LaTeXe{} interface}
%    \begin{macrocode}
\@onlypreamble\@xnthm
\@onlypreamble\@ynthm
\@onlypreamble\@othm
\@onlypreamble\newtheorem
\@onlypreamble\theoremstyle
\@onlypreamble\theorembodyfont
\@onlypreamble\theoremheaderfont
%    \end{macrocode}
% \end{macro}
%
% Finally we declare the \texttt{plain} theorem style to be the default.
% \changes{v2.1b}{90/06/04}{Fixed missing `item by declaring default
%    style.}
%    \begin{macrocode}
\theoremstyle{plain}
%</package>
%    \end{macrocode}
%
% \Finale

\endinput
}{}
\ProvidesFile{thm.sty}
             [\FMithmInfo]
%    \end{macrocode}
%
% \begin{macro}{\th@margin}
%    Again this is only a variant of the theorem styles described
%    above without any new ideas.
%    \begin{macrocode}
\gdef\th@margin{\normalfont\slshape
  \def\@begintheorem##1##2{\item
       [\theorem@headerfont \llap{##2}\hskip\labelsep ##1]}%
\def\@opargbegintheorem##1##2##3{%
  \item[\theorem@headerfont \llap{##2}\hskip\labelsep ##1\ (##3)]}}
\endgroup
%</thm>
%    \end{macrocode}
% \end{macro}
%
%
%
%
% \subsection{Final Definitions}
%
% \begin{macro}{\theorempreskipamount}
% \begin{macro}{\theorempostskipamount}
%    The \textsf{skip} parameters that regulate the vertical empty space
%    before and after the {\envfont theorem} environment have
%    to be allocated as well.
% \changes{v2.0f}{89/07/20}{Unnecessary `global' removed.}
%    \begin{macrocode}
%<*package>
\newskip\theorempreskipamount
\newskip\theorempostskipamount
%    \end{macrocode}
%    Since we have used the same values for all theorem sets, we now can
%    assign them.
%    \begin{macrocode}
\global\setlength\theorempreskipamount{12pt plus 5pt minus 3pt}
\global\setlength\theorempostskipamount{8pt plus 3pt minus 1.5pt}
%    \end{macrocode}
% \end{macro}
% \end{macro}
%
%
% \begin{macro}{\@endtheorem}
%    The same holds for the macro |\@endtheorem|, which ends a
%    {\envfont theorem} environment. Since it is the same for all
%    theorem sets, it is removed from the macros |\th@|\meta{style}.
%    It simply ends the {\envfont trivlist} environment, which was
%    begun in |\@thm|.
%    \begin{macrocode}
\global\let\@endtheorem=\endtrivlist
%    \end{macrocode}
% \end{macro}
%
%
%
% \begin{macro}{\@preamblecmds}
%    All macros defined above are to be used only in the preamble.
%    Therefore, we insert them in |\@preamblecmds| which will disable
%    them at begin document. This is done by the internal \LaTeXe{}
%    command |\@onlypreamble|.
% \changes{v2.0g}{89/09/19}{`theoremheaderfont and `theorembodyfont
%    added.}
% \changes{v2.2a}{94/02/03}{Use \LaTeXe{} interface}
%    \begin{macrocode}
\@onlypreamble\@xnthm
\@onlypreamble\@ynthm
\@onlypreamble\@othm
\@onlypreamble\newtheorem
\@onlypreamble\theoremstyle
\@onlypreamble\theorembodyfont
\@onlypreamble\theoremheaderfont
%    \end{macrocode}
% \end{macro}
%
% Finally we declare the \texttt{plain} theorem style to be the default.
% \changes{v2.1b}{90/06/04}{Fixed missing `item by declaring default
%    style.}
%    \begin{macrocode}
\theoremstyle{plain}
%</package>
%    \end{macrocode}
%
% \Finale

\endinput
}{}
\ProvidesFile{thmb.sty}
             [\FMithmInfo]
%    \end{macrocode}
%
% \begin{macro}{\th@marginbreak}
%    The \texttt{margin break} style is nearly the same as the
%    \texttt{change break} style. The only difference is the placement
%    of the theorem number.  We use |\llap| to place it in the left
%    margin.
%
%    In this style |\labelsep| denotes the separation between the
%    number and the text.
%    \begin{macrocode}
\gdef\th@marginbreak{\normalfont\slshape
  \def\@begintheorem##1##2{\item
       [\rlap{\vbox{\theorem@headerfont
                    \hbox{\llap{##2}\hskip\labelsep ##1}%
                    \hbox{\strut}}}]}%
\def\@opargbegintheorem##1##2##3{%
  \item[\rlap{\vbox{\theorem@headerfont
                    \hbox{\llap{##2}\hskip\labelsep ##1\ (##3)}%
                    \hbox{\strut}}}]}}
\endgroup
%</thmb>
%    \end{macrocode}
% \changes{v2.1c}{90/02/17}{`endgroup outside macrocode env.}
% \end{macro}
%
%
%
% \subsubsection{The \texttt{margin} style}
%
% This style option is stored in the file |thm.sty|.
%    \begin{macrocode}
%<*thm>
\begingroup \makeatletter
\@ifundefined{theorem@style}{% \iffalse meta-comment
%
% Copyright (C) 1993-2024
%
% The LaTeX Project and any individual authors listed elsewhere
% in this file.
%
% This file is part of the Standard LaTeX `Tools Bundle'.
% -------------------------------------------------------
%
% It may be distributed and/or modified under the
% conditions of the LaTeX Project Public License, either version 1.3c
% of this license or (at your option) any later version.
% The latest version of this license is in
%    https://www.latex-project.org/lppl.txt
% and version 1.3c or later is part of all distributions of LaTeX
% version 2005/12/01 or later.
%
% The list of all files belonging to the LaTeX `Tools Bundle' is
% given in the file `manifest.txt'.
%
% \fi
%
%%
%
%
\def\FMithmInfo{2023/07/05 v2.2c Theorem extension package (FMi)}
%
% \ProvidesFile{theorem.dtx}[\FMithmInfo]
%\iffalse   % this is a METACOMMENT !
%%
%% Package `theorem' to use with LaTeX2e
%% Copyright (C) 1989-1995 Frank Mittelbach, all rights reserved.
%<+package>\@ifundefined{theorem@style}{}{\endinput}
%<+package>\NeedsTeXFormat{LaTeX2e}[1995/06/01]
%<+package>\ProvidesPackage{theorem}
%<+package>         [\FMithmInfo]
%
% \fi
%
% \changes{v1.0}{87/05/07}{First draft.}
% \changes{v1.0a}{87/05/07}{`thistheoremstyle' added.}
% \changes{v1.1a}{87/07/29}{New concept: redefine the internal
%                          \protect\LaTeX{} commands `@xnthm' etc.}
% \changes{v1.1b}{87/07/31}{`theoremfont' used to specify font.}
% \changes{v1.2a}{88/01/08}{`@endtheorem' now globally defined.}
% \changes{v1.2b}{88/01/11}{`newtheorem' only allowed in the preamble.}
% \changes{v1.3a}{88/02/14}{`theoremfont' now a toks register.}
%
% \changes{v2.0a}{88/11/16}{New implementation and new documentation.}
% \changes{v2.0a}{88/11/16}{No layout changes but internal modifications
%                           old substyles won't work.}
% \changes{v2.0b}{89/01/30}{`global' used for all definitions.}
% \changes{v2.0c}{89/06/15}{Documentation translated to English.
%                            Thanks to LLL and Nina}
% \changes{v2.1a}{90/02/15}{Support for new font selection.}
% \changes{v2.2a}{94/02/03}{Cleanup for LaTeX2e}
%
% \DoNotIndex{\@endparenv,\@ifdefinable,\@ifnextchar,\@ifundefined}
% \DoNotIndex{\@namedef,\@nameuse,\@spaces,\@tempa,\@warning}
% \DoNotIndex{\ }
% \DoNotIndex{\begingroup,\bfseries}
% \DoNotIndex{\csname}
% \DoNotIndex{\def,\do,\docdate}
% \DoNotIndex{\endcsname,\endgroup,\endinput,\endtrivlist,\expandafter}
% \DoNotIndex{\filedate,\fileversion}
% \DoNotIndex{\gdef,\global}
% \DoNotIndex{\hbox,\hskip}
% \DoNotIndex{\ignorespaces,\input,\itshape,\item}
% \DoNotIndex{\labelsep,\let,\llap}
% \DoNotIndex{\makeatletter}
% \DoNotIndex{\noexpand}
% \DoNotIndex{\refstepcounter,\relax,\rlap}
% \DoNotIndex{\setlength,\slshape,\space,\string,\strut}
% \DoNotIndex{\the,\trivlist,\typeout}
% \DoNotIndex{\vbox}
% \DoNotIndex{\xdef}
%
% \def\deflabel#1{\ttfamily #1\hfill}
% \def\deflist#1{\begin{list}{}{\settowidth\labelwidth{\ttfamily #1}%
%                               \setlength\leftmargin\labelwidth
%                               \addtolength\leftmargin\labelsep
%                               \let\makelabel\deflabel}}
% \def\enddeflist{\end{list}}
% \def\envfont{\normalfont\ttfamily}
%
% \MakeShortVerb{\|}
%
% \newcommand{\TUB}{{\sl TUGboat\/}}
%
% \GetFileInfo{theorem.dtx}
% \title{An Extension of the
%        \LaTeX{} theorem environment\thanks{This file
%        has version number \fileversion, last
%        revised \filedate.}}
% \author{Frank Mittelbach}
%
% \MaintainedByLaTeXTeam{tools}
% \maketitle
%
% \begin{abstract}
%    The macros described in this paper yield an extension of the
%    \LaTeX{} theorem mechanism. It is designed to satisfy the
%    different requirements of various journals. Thus, the
%    layout of the ``theorems'' can be manipulated by determining a
%    ``style''. This article describes not only the use, but
%    also the definition, of the necessary macros.
% \end{abstract}
%
% \section*{Preface to version 2.2}
%
% For \LaTeXe{} this package did not need any fundamental changes. I
% only modified the messages generated so that theorem layout styles
% will show up with the |\listfiles| command and cleaned the section
% on the New Font Selection Scheme since this is now included in
% \LaTeX.
%
% \section*{Preface to version 2.1}
%
% This version is identical to 2.0g described in \TUB\ 10\#3 except
% for some internal defaults which are now set depending on the used
% font selection scheme.
%
% This was done to avoid unpleasant surprises if the new font selection
% scheme is in force. For further details see
% section~\ref{sec:newfonts} and \cite{bk:GMS94}.
%
% \section{Introduction}
%
% For our purposes here, ``theorems'' are labelled enunciations,
% often set off from the main text by extra space and a font change.
% Theorems, corollaries, conjectures, definitions, and remarks
% are all instances of ``theorems''.  The ``header'' of these structures
% is composed of a label (such as \textsc{Theorem} or \textsc{Remark})
% and a number which serializes an item in the sequence of items
% with the same label.
%
% Shortly after the introduction of \LaTeX{} at the Fachbereich
% Mathematik in Mainz, the desire to manipulate the layout of
% ``theorems'' arose.  In Mainz, the following two conventions came
% into general use:
% \begin{enumerate}
%   \item The number of the theorem is shown in the margin.
%   \item There is a line break at
%         the end of the theorem header.
% \end{enumerate}
%
% Additionally, some journals require different formats which depend on
% the ``sort of theorem'': e.g.\ often remarks and definitions are set
% in |\upshape|, while |\itshape| is employed for main theorems.
%
% Confronted with these requirements, a theorem environment was
% developed in Mainz which allows separate determination of the
% layout of the ``theorems sets'', comparable to |\pagestyle|.
%
%
% \section{The user interface}
% \subsection{Defining new theorem sets}
%
% \DescribeMacro\newtheorem
% As in the original \LaTeX{} version, the command |\newtheorem|
% defines a new ``theorem set'' or ``theorem-like structure''.
% Two required arguments name the new environment and give the text
% to be typeset with each instance of the new ``set'', while
% an optional argument determines how
% the ``set'' is enumerated:
% \begin{description}
%    \item[\ttfamily \bslash newtheorem\{foo\}\{bar\}]
%       The theorem set {\envfont foo} (whose name is \texttt{bar})
%       uses its own counter.
%    \item[\ttfamily \bslash newtheorem\{foo2\}{[foo]}\{bar2\}]
%       The theorem set {\envfont foo2} (printed name \texttt{bar2})
%       uses the same counter as the theorem set \texttt{foo}.
%    \item[\ttfamily \bslash newtheorem\{foo3\}\{bar3\}{[section]}]
%       The theorem set {\envfont foo3} (printed name \texttt{bar3}) is
%       enumerated within the counter \texttt{section}, i.e.\ with every
%       new |\section| the enumeration begins again with 1, and
%       the enumeration is composed from the section-number and the
%       theorem counter itself.
% \end{description}
%
% \DescribeMacro\theoremstyle
% Additionally, the command |\theoremstyle| can define the layout
% of various, or all, theorem sets.  It should be noted that any
% theorem set defined by |\newtheorem| is typeset in the
% |\theoremstyle| that is current at the time of the definition.
% Thus, the following
% \begin{verbatim}
%    \theoremstyle{break}        \newtheorem{Cor}{Corollary}
%    \theoremstyle{plain}        \newtheorem{Exa}{Example}[section]
%\end{verbatim}                             ^^A No blank at line start
% leads to the result that the set {\envfont Cor} is formatted in the
% style \texttt{break}, while the set {\envfont Exa} and all the
% following ones are formatted in the style \texttt{plain}, unless
% another |\theoremstyle| follows.  Since the definitions installed by
% |\newtheorem| are global, one also can limit |\theoremstyle| locally
% by grouping braces.
%
% \DescribeMacro\theorembodyfont
% The choice of the font for the theorem body is completely
% independent of the chosen |\theoremstyle|; this has proven to
% be very advantageous.  For example,
% \begin{verbatim}
%   {\theorembodyfont{\upshape}          \newtheorem{Rem}{Remark}}
%\end{verbatim}
% defines a theorem set {\envfont Rem}, which will be set in
% |\upshape| in the current layout (which in our example is
% \texttt{plain}).  As with |\theoremstyle|, the |\theorembodyfont|
% chosen is that current at the time of |\newtheorem|.  If
% |\theorembodyfont| is not specified or one defines
% |\theorembodyfont{}|, then the font used will be that defined by the
% |\theoremstyle|.
%
% \DescribeMacro\theoremheaderfont It is also possible to customize
% the font used for the theorem headers.  This is, however, a global
% declaration, and therefore there should be at most one
% |\theoremheaderfont| declaration in the preamble.\footnote{If it is
% actually necessary to have different header fonts, one has to define
% new theorem styles (substituting the desired font) or specify the
% information directly in the \texttt{\bslash newtheorem} declaration
% (the unclean variant).}
%
% \DescribeMacro\theorempreskipamount
% \DescribeMacro\theorempostskipamount Two additional parameters
% affect the vertical space around the theorem environments:
% |\theorempreskipamount| and |\theorempostskipamount| define,
% respectively, the spacing before and after such an environment.
% These parameters apply for all theorem sets and can be manipulated
% with the ordinary length macros.  They are rubber lengths
% (`\textsf{skips}'), and therefore can contain \texttt{plus} and
% \texttt{minus} parts.
%
% Since the definition of theorem sets should---most sensibly---be
% placed in the preamble, we only allow installation there.
% It is therefore possible to release the memory used here after
% |\begin{document}|, in order to make room for other
% applications.
%
%
% \subsection{Existing theorem styles}
%
% The following theorem styles exist to date:
% \begin{deflist}{marginbreak}
%    \item[plain]
%       This theorem style emulates the original \LaTeX{} definition,
%       except that additionally the parameters
%       |\theorem...skipamount| are used.
%    \item[break]
%       In this style, the theorem header is followed by a line
%       break.
%    \item[marginbreak]
%       The theorem number is set in the margin, and there is a line
%       break as in \texttt{break}.
%    \item[changebreak]
%       Like \texttt{break}, but with header number and text
%       interchanged.
%    \item[change]
%       Header number and text are interchanged, without a line break.
%    \item[margin]
%       The number is set in the left margin, without a line break.
% \end{deflist}
% All styles (except \texttt{plain}) select |\slshape| as the default
% |\theorembodyfont|.
%
%
% \subsection{Examples}
%
% Given the above theorem sets {\envfont Cor}, {\envfont Exa}
% and {\envfont Rem},
% suppose that the preamble also contains the declarations:
% \begin{verbatim}
%    \theoremstyle{marginbreak}   \newtheorem{Lem}[Cor]{Lemma}
%    \theoremstyle{change}
%    \theorembodyfont{\itshape}        \newtheorem{Def}[Cor]{Definition}
%
%    \theoremheaderfont{\scshape}
%\end{verbatim}
% Then the following are some typical examples of the typeset output
% resulting from their use.
%
% \begin{Cor}
%    This is a sentence typeset in the theorem environment {\envfont
%    Cor}.
% \end{Cor}
%
% \begin{Exa}
%    This is a sentence typeset in the theorem environment {\envfont
%    Exa}.
% \end{Exa}
%
% \begin{Rem}
%    This is a sentence typeset in the theorem environment {\envfont
%    Rem}.
% \end{Rem}
%
% \begin{Lem}[Ben User]
%    This is a sentence typeset in the theorem environment {\envfont
%    Lem}.
% \end{Lem}
%
% \begin{Def}[Very impressive Definition]
%    This is a sentence typeset in the theorem environment {\envfont
%    Def}.
% \end{Def}
%
% The last two examples show the effect of the optional argument to
% a theorem environment (it is the text typeset in parentheses).
%
%
% \section{Special Considerations}
% \label{sec:newfonts}
%
% Theoremheader and body are implemented as a unit. This means that
% the |\theoremheaderfont|  will inherit characteristics of the
% |\theorembodyfont| in \LaTeXe. Thus, if for example
% |\theorembodyfont| is |\itshape| and |\theoremheaderfont|
% is |\bfseries| the font selected for the header will have the
% characteristics `bold extended italic'. If this is not desired one
% should set the |\theoremheaderfont| to something like
% \begin{verbatim}
%      \theoremheaderfont{\normalfont\bfseries}
%\end{verbatim}
% i.e.\ supplying all necessary font information explicitly.
%
%
%
% \section{Acknowledgements}
%
% The publication of this set of macros was only possible with the
% help of Christina Busse (translating the manuscript into English),
% Joachim Pense (playing the r\^ole of typist), Chris Rowley
% (looking everything over) and many others providing useful
% suggestions.
%
% \MaybeStop{%
%    \begin{thebibliography}{9}
%       \bibitem{bk:GMS94} \textsc{M.~Goossens}, \textsc{F.~Mittelbach}
%         and \textsc{A.~Samarin}.
%         \newblock The \LaTeX{} Companion.
%         \newblock
%         Addison-Wesley, Reading, Massachusetts, 1994.
%       \bibitem{latex:source} \textsc{Lamport, Leslie.} \newblock
%                               \texttt{latex.tex}, version 2.09, date
%                               Feb. 1990.
%    \end{thebibliography}
%    \PrintIndex
%    \PrintChanges}
%
%
% \section{The documentation driver file}
%
% The next bit of code contains the documentation driver file for
% \TeX{}, i.e., the file that will produce the documentation you are
% currently reading. It will be extracted from this file by the
% \texttt{docstrip} program. Since it is the first code in the file
% one can alternatively process this file directly with \LaTeXe{} to
% obtain the documentation.
%
%    \begin{macrocode}
%<*driver>
\documentclass{ltxdoc}

\usepackage{theorem}

%<+driver>% The next few lines define theorem sets which are used
%<+driver>% in the example section of the documentation.

\theoremstyle{break}       \newtheorem{Cor}{Corollary}
\theoremstyle{plain}       \newtheorem{Exa}{Example}[section]
{\theorembodyfont{\upshape}\newtheorem{Rem}{Remark}}
\theoremstyle{marginbreak} \newtheorem{Lem}[Cor]{Lemma}
\theoremstyle{change}
\theorembodyfont{\itshape} \newtheorem{Def}[Cor]{Definition}

\theoremheaderfont{\scshape}

\RecordChanges

\begin{document}
   \DocInput{theorem.dtx}
\end{document}
%</driver>
%    \end{macrocode}
%
% \section{Definition of the Macros}
%
% If the file has been loaded before, we abort immediately. If not the
% package announces itself (this is actually done at the very top of
% the file---the way it is done isn't good style so don't copy it).
% \changes{v2.0e}{89/07/19}{Spaces removed from `typeout'.}
% \changes{v2.2b}{95/11/19}{Announce moved to top of file}
%    \begin{macrocode}
%<*package>
%\@ifundefined{theorem@style}{}{\endinput}
%\def\FMithmInfo{1995/11/19 v2.2b Theorem extension package (FMi)}
%\ProvidesPackage{theorem}[\FMithmInfo]
%    \end{macrocode}
%
%
% \subsection{Definition of theorem styles and fonts}
%
% All the definitions in this file are done globally to allow
% inputting this file inside a group.
%
% \begin{macro}{\theoremstyle}
%    Before a theorem style can be installed, the chosen style must be
%    known. For that reason, we must test to see that
%    |\th@|\meta{style} is known or, more precisely, that it is
%    different from |\relax|. If the style is not known then
%    |\th@plain| is used.
%    \begin{macrocode}
\gdef\theoremstyle#1{%
   \@ifundefined{th@#1}{\@warning
          {Unknown theoremstyle `#1'. Using `plain'}%
          \theorem@style{plain}}%
%    \end{macrocode}
%    We save the theorem style to be used in the token register
%    |\theorem@style|.
%    \begin{macrocode}
      {\theorem@style{#1}}%
%    \end{macrocode}
%    Now we ``evaluate'' the theorem style: this means, we call the
%    macro |\th@|\meta{style} which will activate the relevant
%    definitions which are contained in a separate file.  This is done
%    in a group to suppress changes to the current font.  This could
%    otherwise pose problems together with the new font selection
%    scheme\footnote{When I printed the original article using the new
%    font selection scheme I ended with a document with slanted
%    typefaces (text headings and all) simply because one of the
%    theorem styles used \texttt{\bslash sl} at toplevel.} if the
%    |\th@|\meta{style} is evaluated a second time.
% \changes{v2.1a}{90/02/15}{Group added around `th@..}
% \changes{v2.1a}{90/02/15}{Reseting of `..@begintheorem removed.}
%    \begin{macrocode}
      \begingroup
        \csname th@\the\theorem@style \endcsname
      \endgroup}
%    \end{macrocode}
% \end{macro}
%
%
% \begin{macro}{\@begintheorem}
% \changes{v2.1a}{90/02/15}{Relaxed at toplevel.}
% \begin{macro}{\@opargbegintheorem}
% \changes{v2.1a}{90/02/15}{Relaxed at toplevel.}
%    We reset |\@begintheorem| and |\@opargbegintheorem| to
%    |\relax| since these commands are no longer necessary at
%    toplevel. This will save a few tokens.
%    \begin{macrocode}
\global\let\@begintheorem\relax
\global\let\@opargbegintheorem\relax
%    \end{macrocode}
% \end{macro}
% \end{macro}
%
% \begin{macro}{\theorem@style}
%    Obviously the \textsf{token} register used above has to be
%    allocated. To assure the utmost compatibility with the
%    original \LaTeX{} definition, we set the default theorem style to
%    \texttt{plain}, which implements the usual \LaTeX{} convention.
% \changes{v2.0f}{89/07/20}{Unnecessary `global' removed.}
%    \begin{macrocode}
\newtoks\theorem@style
\global\theorem@style{plain}
%    \end{macrocode}
% \end{macro}
%
% \begin{macro}{\theorembodyfont}
% \begin{macro}{\theorem@bodyfont}
%    If the |\theorembodyfont| is set by the user then it should not
%    interact with the default font set in the theorem style. When the
%    new font selection is in force this may happen if, for example,
%    the default is |\itshape| and the new |\theorembodyfont| is
%    |\sffamily|. So we add a |\reset@font| command in front of the
%    user definition.
% \changes{v2.0f}{89/07/20}{Unnecessary `global' removed.}
% \changes{v2.0g}{89/09/19}{`theoremfont to `theorembodyfont.}
% \changes{v2.1c}{92/06/22}{`reset@font added and logic changed.}
% \changes{v2.1d}{92/10/28}{Reset to empty if arg is empty}
%    \begin{macrocode}
\gdef\theorembodyfont#1{%
%    \end{macrocode}
%    We check if the argument supplied is empty and if so put nothing
%    into the |\theorem@bodyfont| token register to allow for
%    |\theorembodyfont{}| as a mean of using the default of the
%    current |\theoremstyle|.
%    \begin{macrocode}
   \def\@tempa{#1}%
   \ifx\@tempa\@empty
    \theorem@bodyfont{}%
   \else
    \theorem@bodyfont{\reset@font#1}%
   \fi
   }
\newtoks\theorem@bodyfont
\global\theorem@bodyfont{}
%    \end{macrocode}
% \end{macro}
% \end{macro}
%
%
% \begin{macro}{\theoremheaderfont}
% \changes{v2.0g}{89/09/19}{Macro added.}
%    The font for the theorem headers is handled differently because
%    this definition applies to all theorem styles.
%    \begin{macrocode}
\gdef\theoremheaderfont#1{\gdef\theorem@headerfont{#1}%
%    \end{macrocode}
%    After using the macro once it is redefined to produce an error
%    message.
%    \begin{macrocode}
       \gdef\theoremheaderfont##1{%
        \typeout{\string\theoremheaderfont\space should be used
                 only once.}}}
%    \end{macrocode}
% \end{macro}
%
% \begin{macro}{\theorem@headerfont}
% \changes{v2.0g}{89/09/19}{Macro added.}
%    To set the |\theorem@headerfont| default we first test if
%    the new fontselection scheme is in force.
% \changes{v2.1a}{90/02/15}{Support for new font selection.}
%    \begin{macrocode}
\ifx\upshape\undefined
%    \end{macrocode}
%    If not we define it to expand into |\bfseries|. We don't use
%    |\let| just in case a following style option redefines this
%    macro.
%    \begin{macrocode}
\gdef\theorem@headerfont{\bfseries}
%    \end{macrocode}
%    Otherwise we reset the current shape before calling |\bfseries|.
%    \begin{macrocode}
\else \gdef\theorem@headerfont{\normalfont\bfseries}\fi
%    \end{macrocode}
% \end{macro}
%
% \begin{macro}{\th@plain}
% \begin{macro}{\th@break}
% \begin{macro}{\th@marginbreak}
% \begin{macro}{\th@changebreak}
% \begin{macro}{\th@change}
% \begin{macro}{\th@margin}
%    The different styles are defined in macros such as
%    |\th@plain|.  Since memory space is precious in
%    ``non-Big-versions'', we have to avoid offering too many unused
%    definitions.  Therefore we define these styles in separate files
%    that can be loaded on demand.  Thus the commands themselves only
%    load these files. We use |\@input@|, a \LaTeXe{} internal command
%    that ensures that the file will be listed with |\listfiles|.
% \changes{v2.2a}{94/02/02}{Use `@input@ to load theorem layout files}
%    \begin{macrocode}
\gdef\th@plain{\@input@{thp.sty}}
\gdef\th@break{\@input@{thb.sty}}
\gdef\th@marginbreak{\@input@{thmb.sty}}
\gdef\th@changebreak{\@input@{thcb.sty}}
\gdef\th@change{\@input@{thc.sty}}
\gdef\th@margin{\@input@{thm.sty}}
%    \end{macrocode}
%    This list will be expanded when new styles become
%    available.  For testing,  just append new theorem substyles
%    as document options.
% \end{macro}
% \end{macro}
% \end{macro}
% \end{macro}
% \end{macro}
% \end{macro}
%
%
%
% \subsection{Definition of a new theorem set}
%
% As already pointed out, a new theorem environment can be defined in
% three different ways:
% \begin{verbatim}
%    \newtheorem{Lem}{Lemma}
%    \newtheorem{Lem}{Lemma}[section]
%    \newtheorem{Lem}[Theorem]{Lemma}
%\end{verbatim}
% The function of the macro |\newtheorem| is to recognize these cases
% and then to branch into one of the three macros |\@ynthm|, |\@xnthm|
% or |\@othm|.  This mechanism is adopted unchanged from
% \cite{latex:source}; the essential point here is that, for example,
% in the second case, the arguments \texttt{Lem}, \texttt{Lemma} and
% \texttt{section} are passed over to the macro |\@xnthm|.
%
% We inspect this case first because the others present fewer problems,
% and thus are easily derived from this one.
%
% \begin{macro}{\@xnthm}
%    For our example arguments, the macro |\@xnthm| must fulfill
%    the following:
%    \begin{itemize}
%       \item
%          Define a new \LaTeX-counter `Lem'
%       \item
%          reset this counter within a |\section|
%       \item
%          define the macro |\theLem|
%       \item
%          define the environment macros |\Lem| and
%          |\endLem| using the current
%          |\theoremstyle| and |\theorem@bodyfont|.
%    \end{itemize}
%    Obviously, all this should happen only if the first argument
%    of |\@xnthm| (i.e.\ \texttt{Lem} in our example) is chosen so
%    as not to conflict with any previously defined commands or
%    environments.  This test is performed by the \LaTeX{} macro
%    |\@ifdefinable|.
%    \begin{macrocode}
\gdef\@xnthm#1#2[#3]{\expandafter\@ifdefinable\csname #1\endcsname
%    \end{macrocode}
%    Therefore, the first argument of |\@ifdefinable| is the
%    expansion (in the example, |\Lem|) of
%    |\csname||#1\endcsname|.  The second argument is
%    executed only if the test has been completed successfully.
%    \begin{macrocode}
   {%
%    \end{macrocode}
%    Now we define the new counter. The names of the \LaTeX{}
%    macros employed should speak for themselves:
% \changes{v2.2b}{1995/10/07}{Use \cs{@newctr}}
% \changes{v2.2c}{1995/11/23}{Use \cs{@newctr} properly.}
%    \begin{macrocode}
    \@definecounter{#1}\@newctr{#1}[#3]%
%    \end{macrocode}
%    Using |\@newctr| will give a proper error message if the counter
%    in |#3| is not defined.
%    In defining `|\theLem|' we must generate the desired macro
%    name by use of |\expandafter| and |\csname|.
%    \begin{macrocode}
    \expandafter\xdef\csname the#1\endcsname
%    \end{macrocode}
%    An |\xdef| is used in order to make the definition global,
%    and to ensure that it contains the replacement texts of
%    |\@thmcountersep| and |\@thmcounter|.\footnote{These
%    two macros can be defined by the document style.  Their default
%    values produce a `\texttt{.}' as separation and an arabic
%    representation of the number.} However, not everything should be
%    expanded. For example, it saves space to use |\thesection|
%    instead of its---at times---lengthy expansion.
%    \begin{macrocode}
      {\expandafter \noexpand \csname the#3\endcsname
       \@thmcountersep \@thmcounter{#1}}%
%    \end{macrocode}
%    Thus with the defaults of \LaTeX{}, |\theLem| would be
%    replaced by the command sequence |\thesection.\arabic{Lem}|.
%
%    We will now look at the definition of the macro which is executed
%    at the beginning of the actual environment (in our example this
%    macro is |\Lem|). It should be noted that we use an
%    ``|\expandafter| trick'' to expand only certain parts of the
%    replacement text at the time of the definition.
%    \begin{macrocode}
    \def\@tempa{\global\@namedef{#1}}%
    \expandafter \@tempa \expandafter{%
%    \end{macrocode}
%    First, the macro that contains the current definitions of
%    |\@begintheorem| and |\@opargtheorem| should be called
%    up.  The name of this macro---as is already known---has the form
%    |\th@|\meta{theorem\ style}; therefore, it must be called by
%    \begin{macrocode}
      \csname th@\the \theorem@style
%    \end{macrocode}
%    In addition the default theorem font should be changeable, i.e.\
%    we have to insert the contents of |\theorem@bodyfont|. For
%    that reason, we expand even further, beyond |\endcsname|,
%    and thus insert the contents of the \textsf{token} register
%    |\theorem@bodyfont| in the replacement text.
%    \begin{macrocode}
            \expandafter \endcsname \the \theorem@bodyfont
%    \end{macrocode}
%    Now it is time to call the macro |\@thm| which takes over
%    the further processing.  It has two arguments: the current
%    counter name (in our example, \texttt{Lem}), and the text of the
%    label (in our example, \texttt{Lemma}).
%    \begin{macrocode}
     \@thm{#1}{#2}}%
%    \end{macrocode}
%    With this, the `sub-definition' is complete.  The macro
%    |\@endtheorem| ends a theorem environment and is, so far, nothing
%    but an |\endtrivlist|.  (Hence it is defined globally, and not
%    within the theorem styles.\footnote{This has to be changed as
%    soon as theorem styles that change \texttt{\bslash @endtheorem}
%    exist.  In such a case, all existing styles must be changed as
%    well since they will have to reset the macro.}) Therefore, we can
%    set it equivalent to the macro that ends the theorem set (in our
%    example, |\endLem|).  However, if some day theorem styles exist
%    that do change |\@endtheorem|, we would have to use the
%    commented-out line instead.
%    \begin{macrocode}
    \global \expandafter \let \csname end#1\endcsname \@endtheorem
%   \global\@namedef{end#1}{\@endtheorem}%
%    \end{macrocode}
%    With these commands all the required definitions are employed,
%    unless the test |\@ifdefinable| has failed.  Therefore, we
%    end the second argument of this macro and with it the definition
%    of |\@xnthm|.
%    \begin{macrocode}
   }}
%    \end{macrocode}
% \end{macro}
%
% \begin{macro}{\@ynthm}
%    The definition of |\@ynthm| is completely analogous. In
%    this case the new counter that is defined is not reset within
%    another counter; thus the definition of |\the...|
%    is simplified:
%    \begin{macrocode}
\gdef\@ynthm#1#2{\expandafter\@ifdefinable\csname #1\endcsname
   {\@definecounter{#1}%
    \expandafter\xdef\csname the#1\endcsname{\@thmcounter{#1}}%
%    \end{macrocode}
%    The rest of the definition corresponds literally to that of
%    |\@xnthm|:
%    \begin{macrocode}
    \def\@tempa{\global\@namedef{#1}}\expandafter \@tempa
     \expandafter{\csname th@\the \theorem@style \expandafter
     \endcsname \the\theorem@bodyfont \@thm{#1}{#2}}%
    \global \expandafter \let \csname end#1\endcsname \@endtheorem}}
%    \end{macrocode}
% \end{macro}
%
% \begin{macro}{\@othm}
%    The definition of |\@othm| does not contain anything new.
%    \begin{macrocode}
\gdef\@othm#1[#2]#3{%
%    \end{macrocode}
% \changes{v2.2b}{95/10/07}{Build in check for counter defined}
%    We do not define a new counter but instead use one that has
%    already been defined.  Thus the only definition we need is that
%    of this pseudo-counter (i.e.\ |\the|\meta{env.\ name}).
%    First we check if |#2| corresponds to a known counter name.
%    \begin{macrocode}
  \expandafter\ifx\csname c@#2\endcsname\relax
   \@nocounterr{#2}%
  \else
   \expandafter\@ifdefinable\csname #1\endcsname
   {\expandafter \xdef \csname the#1\endcsname
     {\expandafter \noexpand \csname the#2\endcsname}%
%    \end{macrocode}
%    All other parts of the definition can be adopted from
%    |\@xnthm|. We have to remember, though, that in this case
%    the name of the current counter and the theorem label have moved
%    to the second and third arguments.
%    \begin{macrocode}
    \def\@tempa{\global\@namedef{#1}}\expandafter \@tempa
     \expandafter{\csname th@\the \theorem@style \expandafter
     \endcsname \the\theorem@bodyfont \@thm{#2}{#3}}%
    \global \expandafter \let \csname end#1\endcsname \@endtheorem}%
  \fi}
%    \end{macrocode}
% \end{macro}
%
% \subsection{Macros that are employed in a {\envfont theorem}
%    environment}
%
% \begin{macro}{\@thm}
%    The macro |\@thm| has to increase the current counter.  Then,
%    depending on whether the environment has (or does not have) an
%    optional argument, it has to branch into either |\@begintheorem|
%    or |\@opargtheorem|.
%    \begin{macrocode}
\gdef\@thm#1#2{\refstepcounter{#1}%
%    \end{macrocode}
%    Now we start a {\envfont trivlist} environment, and give
%    |\@topsep| and |\@topsepadd| the values of the \textsf{skip}
%    registers |\theorempreskipamount| and |\theorempostskipamount|.
%    The value in |\@topsep| is the vertical space that is inserted by
%    the first (and only) |\item| in our |\trivlist| whilst
%    |\@topsepadd| is inserted by |\@endparenv| at the end of that
%    {\envfont trivlist} environment.  By using these registers, we
%    obtain the desired space around a {\envfont theorem} environment.
%    \begin{macrocode}
   \trivlist
   \@topsep \theorempreskipamount               % used by first \item
   \@topsepadd \theorempostskipamount           % used by \@endparenv
%    \end{macrocode}
%    Now we have to test whether an optional argument has been given.
% \changes{v2.0d}{89/07/18}{Forgotten percent added.}
%    \begin{macrocode}
   \@ifnextchar [%
%    \end{macrocode}
%    If there is an optional argument, we will call |\@ythm|, and
%    move the arguments read back into the input stream.
%    \begin{macrocode}
   {\@ythm{#1}{#2}}%
%    \end{macrocode}
%    If not, we call |\@begintheorem|. Its first argument is the
%    name of the theorem set (hence the second argument of
%    |\@thm|). Its second argument is the macro that produces the
%    current number.
%    \begin{macrocode}
   {\@begintheorem{#2}{\csname the#1\endcsname}\ignorespaces}}
%    \end{macrocode}
% \end{macro}
%
% \begin{macro}{\@xthm}
% \begin{macro}{\@ythm}
%    Both these macros were originally called by |\@thm|.  We do
%    not need |\@xthm| anymore, hence we reset it to
%    |\relax|. The definition of |\@ythm| has not changed at
%    all from its definition in \LaTeX. In order to make the macros
%    easier to understand, we will
%    nevertheless present it (commented out).
%    \begin{macrocode}
\global\let\@xthm\relax
% \def\@ythm#1#2[#3]{\@opargbegintheorem{#2}{\csname
%           the#1\endcsname}{#3}\ignorespaces}
%</package>
%    \end{macrocode}
%    The primitive |\ignorespaces| in |\@ythm| and
%    |\@thm| is needed to remove the spaces between the
%    |\begin{...}| and the actual text.
% \end{macro}
% \end{macro}
%
%
% \subsection{Definition of the theorem substyles}
%
% As already pointed out, the theorem substyles, defined below, are
% only loaded when necessary.  Note that all these substyles, except
% \texttt{plain}, have |\slshape| as the default body font.
%
% \subsubsection{The \texttt{plain} style}
%
% \label{dok:th@plain} As the following macros use |@|, we have
% to locally set the |\catcode| of this symbol to ``letter''.
% This happens within a group, so that we do not have to worry about
% which |\catcode| that symbol had before.
%    \begin{macrocode}
%<*thp>
\begingroup \makeatletter
%    \end{macrocode}
% Since we are now within a group, we must make all definitions
% globally.  First we make sure that \texttt{theorem.sty} is loaded.
% This will allow us to use this file as a document style option
% without having to call \texttt{theorem} itself as an option.  At the
% same time, we assure that at least version 2 is loaded, since
% |\theorem@style| was not defined in earlier versions.
%    \begin{macrocode}
\@ifundefined{theorem@style}{% \iffalse meta-comment
%
% Copyright (C) 1993-2024
%
% The LaTeX Project and any individual authors listed elsewhere
% in this file.
%
% This file is part of the Standard LaTeX `Tools Bundle'.
% -------------------------------------------------------
%
% It may be distributed and/or modified under the
% conditions of the LaTeX Project Public License, either version 1.3c
% of this license or (at your option) any later version.
% The latest version of this license is in
%    https://www.latex-project.org/lppl.txt
% and version 1.3c or later is part of all distributions of LaTeX
% version 2005/12/01 or later.
%
% The list of all files belonging to the LaTeX `Tools Bundle' is
% given in the file `manifest.txt'.
%
% \fi
%
%%
%
%
\def\FMithmInfo{2023/07/05 v2.2c Theorem extension package (FMi)}
%
% \ProvidesFile{theorem.dtx}[\FMithmInfo]
%\iffalse   % this is a METACOMMENT !
%%
%% Package `theorem' to use with LaTeX2e
%% Copyright (C) 1989-1995 Frank Mittelbach, all rights reserved.
%<+package>\@ifundefined{theorem@style}{}{\endinput}
%<+package>\NeedsTeXFormat{LaTeX2e}[1995/06/01]
%<+package>\ProvidesPackage{theorem}
%<+package>         [\FMithmInfo]
%
% \fi
%
% \changes{v1.0}{87/05/07}{First draft.}
% \changes{v1.0a}{87/05/07}{`thistheoremstyle' added.}
% \changes{v1.1a}{87/07/29}{New concept: redefine the internal
%                          \protect\LaTeX{} commands `@xnthm' etc.}
% \changes{v1.1b}{87/07/31}{`theoremfont' used to specify font.}
% \changes{v1.2a}{88/01/08}{`@endtheorem' now globally defined.}
% \changes{v1.2b}{88/01/11}{`newtheorem' only allowed in the preamble.}
% \changes{v1.3a}{88/02/14}{`theoremfont' now a toks register.}
%
% \changes{v2.0a}{88/11/16}{New implementation and new documentation.}
% \changes{v2.0a}{88/11/16}{No layout changes but internal modifications
%                           old substyles won't work.}
% \changes{v2.0b}{89/01/30}{`global' used for all definitions.}
% \changes{v2.0c}{89/06/15}{Documentation translated to English.
%                            Thanks to LLL and Nina}
% \changes{v2.1a}{90/02/15}{Support for new font selection.}
% \changes{v2.2a}{94/02/03}{Cleanup for LaTeX2e}
%
% \DoNotIndex{\@endparenv,\@ifdefinable,\@ifnextchar,\@ifundefined}
% \DoNotIndex{\@namedef,\@nameuse,\@spaces,\@tempa,\@warning}
% \DoNotIndex{\ }
% \DoNotIndex{\begingroup,\bfseries}
% \DoNotIndex{\csname}
% \DoNotIndex{\def,\do,\docdate}
% \DoNotIndex{\endcsname,\endgroup,\endinput,\endtrivlist,\expandafter}
% \DoNotIndex{\filedate,\fileversion}
% \DoNotIndex{\gdef,\global}
% \DoNotIndex{\hbox,\hskip}
% \DoNotIndex{\ignorespaces,\input,\itshape,\item}
% \DoNotIndex{\labelsep,\let,\llap}
% \DoNotIndex{\makeatletter}
% \DoNotIndex{\noexpand}
% \DoNotIndex{\refstepcounter,\relax,\rlap}
% \DoNotIndex{\setlength,\slshape,\space,\string,\strut}
% \DoNotIndex{\the,\trivlist,\typeout}
% \DoNotIndex{\vbox}
% \DoNotIndex{\xdef}
%
% \def\deflabel#1{\ttfamily #1\hfill}
% \def\deflist#1{\begin{list}{}{\settowidth\labelwidth{\ttfamily #1}%
%                               \setlength\leftmargin\labelwidth
%                               \addtolength\leftmargin\labelsep
%                               \let\makelabel\deflabel}}
% \def\enddeflist{\end{list}}
% \def\envfont{\normalfont\ttfamily}
%
% \MakeShortVerb{\|}
%
% \newcommand{\TUB}{{\sl TUGboat\/}}
%
% \GetFileInfo{theorem.dtx}
% \title{An Extension of the
%        \LaTeX{} theorem environment\thanks{This file
%        has version number \fileversion, last
%        revised \filedate.}}
% \author{Frank Mittelbach}
%
% \MaintainedByLaTeXTeam{tools}
% \maketitle
%
% \begin{abstract}
%    The macros described in this paper yield an extension of the
%    \LaTeX{} theorem mechanism. It is designed to satisfy the
%    different requirements of various journals. Thus, the
%    layout of the ``theorems'' can be manipulated by determining a
%    ``style''. This article describes not only the use, but
%    also the definition, of the necessary macros.
% \end{abstract}
%
% \section*{Preface to version 2.2}
%
% For \LaTeXe{} this package did not need any fundamental changes. I
% only modified the messages generated so that theorem layout styles
% will show up with the |\listfiles| command and cleaned the section
% on the New Font Selection Scheme since this is now included in
% \LaTeX.
%
% \section*{Preface to version 2.1}
%
% This version is identical to 2.0g described in \TUB\ 10\#3 except
% for some internal defaults which are now set depending on the used
% font selection scheme.
%
% This was done to avoid unpleasant surprises if the new font selection
% scheme is in force. For further details see
% section~\ref{sec:newfonts} and \cite{bk:GMS94}.
%
% \section{Introduction}
%
% For our purposes here, ``theorems'' are labelled enunciations,
% often set off from the main text by extra space and a font change.
% Theorems, corollaries, conjectures, definitions, and remarks
% are all instances of ``theorems''.  The ``header'' of these structures
% is composed of a label (such as \textsc{Theorem} or \textsc{Remark})
% and a number which serializes an item in the sequence of items
% with the same label.
%
% Shortly after the introduction of \LaTeX{} at the Fachbereich
% Mathematik in Mainz, the desire to manipulate the layout of
% ``theorems'' arose.  In Mainz, the following two conventions came
% into general use:
% \begin{enumerate}
%   \item The number of the theorem is shown in the margin.
%   \item There is a line break at
%         the end of the theorem header.
% \end{enumerate}
%
% Additionally, some journals require different formats which depend on
% the ``sort of theorem'': e.g.\ often remarks and definitions are set
% in |\upshape|, while |\itshape| is employed for main theorems.
%
% Confronted with these requirements, a theorem environment was
% developed in Mainz which allows separate determination of the
% layout of the ``theorems sets'', comparable to |\pagestyle|.
%
%
% \section{The user interface}
% \subsection{Defining new theorem sets}
%
% \DescribeMacro\newtheorem
% As in the original \LaTeX{} version, the command |\newtheorem|
% defines a new ``theorem set'' or ``theorem-like structure''.
% Two required arguments name the new environment and give the text
% to be typeset with each instance of the new ``set'', while
% an optional argument determines how
% the ``set'' is enumerated:
% \begin{description}
%    \item[\ttfamily \bslash newtheorem\{foo\}\{bar\}]
%       The theorem set {\envfont foo} (whose name is \texttt{bar})
%       uses its own counter.
%    \item[\ttfamily \bslash newtheorem\{foo2\}{[foo]}\{bar2\}]
%       The theorem set {\envfont foo2} (printed name \texttt{bar2})
%       uses the same counter as the theorem set \texttt{foo}.
%    \item[\ttfamily \bslash newtheorem\{foo3\}\{bar3\}{[section]}]
%       The theorem set {\envfont foo3} (printed name \texttt{bar3}) is
%       enumerated within the counter \texttt{section}, i.e.\ with every
%       new |\section| the enumeration begins again with 1, and
%       the enumeration is composed from the section-number and the
%       theorem counter itself.
% \end{description}
%
% \DescribeMacro\theoremstyle
% Additionally, the command |\theoremstyle| can define the layout
% of various, or all, theorem sets.  It should be noted that any
% theorem set defined by |\newtheorem| is typeset in the
% |\theoremstyle| that is current at the time of the definition.
% Thus, the following
% \begin{verbatim}
%    \theoremstyle{break}        \newtheorem{Cor}{Corollary}
%    \theoremstyle{plain}        \newtheorem{Exa}{Example}[section]
%\end{verbatim}                             ^^A No blank at line start
% leads to the result that the set {\envfont Cor} is formatted in the
% style \texttt{break}, while the set {\envfont Exa} and all the
% following ones are formatted in the style \texttt{plain}, unless
% another |\theoremstyle| follows.  Since the definitions installed by
% |\newtheorem| are global, one also can limit |\theoremstyle| locally
% by grouping braces.
%
% \DescribeMacro\theorembodyfont
% The choice of the font for the theorem body is completely
% independent of the chosen |\theoremstyle|; this has proven to
% be very advantageous.  For example,
% \begin{verbatim}
%   {\theorembodyfont{\upshape}          \newtheorem{Rem}{Remark}}
%\end{verbatim}
% defines a theorem set {\envfont Rem}, which will be set in
% |\upshape| in the current layout (which in our example is
% \texttt{plain}).  As with |\theoremstyle|, the |\theorembodyfont|
% chosen is that current at the time of |\newtheorem|.  If
% |\theorembodyfont| is not specified or one defines
% |\theorembodyfont{}|, then the font used will be that defined by the
% |\theoremstyle|.
%
% \DescribeMacro\theoremheaderfont It is also possible to customize
% the font used for the theorem headers.  This is, however, a global
% declaration, and therefore there should be at most one
% |\theoremheaderfont| declaration in the preamble.\footnote{If it is
% actually necessary to have different header fonts, one has to define
% new theorem styles (substituting the desired font) or specify the
% information directly in the \texttt{\bslash newtheorem} declaration
% (the unclean variant).}
%
% \DescribeMacro\theorempreskipamount
% \DescribeMacro\theorempostskipamount Two additional parameters
% affect the vertical space around the theorem environments:
% |\theorempreskipamount| and |\theorempostskipamount| define,
% respectively, the spacing before and after such an environment.
% These parameters apply for all theorem sets and can be manipulated
% with the ordinary length macros.  They are rubber lengths
% (`\textsf{skips}'), and therefore can contain \texttt{plus} and
% \texttt{minus} parts.
%
% Since the definition of theorem sets should---most sensibly---be
% placed in the preamble, we only allow installation there.
% It is therefore possible to release the memory used here after
% |\begin{document}|, in order to make room for other
% applications.
%
%
% \subsection{Existing theorem styles}
%
% The following theorem styles exist to date:
% \begin{deflist}{marginbreak}
%    \item[plain]
%       This theorem style emulates the original \LaTeX{} definition,
%       except that additionally the parameters
%       |\theorem...skipamount| are used.
%    \item[break]
%       In this style, the theorem header is followed by a line
%       break.
%    \item[marginbreak]
%       The theorem number is set in the margin, and there is a line
%       break as in \texttt{break}.
%    \item[changebreak]
%       Like \texttt{break}, but with header number and text
%       interchanged.
%    \item[change]
%       Header number and text are interchanged, without a line break.
%    \item[margin]
%       The number is set in the left margin, without a line break.
% \end{deflist}
% All styles (except \texttt{plain}) select |\slshape| as the default
% |\theorembodyfont|.
%
%
% \subsection{Examples}
%
% Given the above theorem sets {\envfont Cor}, {\envfont Exa}
% and {\envfont Rem},
% suppose that the preamble also contains the declarations:
% \begin{verbatim}
%    \theoremstyle{marginbreak}   \newtheorem{Lem}[Cor]{Lemma}
%    \theoremstyle{change}
%    \theorembodyfont{\itshape}        \newtheorem{Def}[Cor]{Definition}
%
%    \theoremheaderfont{\scshape}
%\end{verbatim}
% Then the following are some typical examples of the typeset output
% resulting from their use.
%
% \begin{Cor}
%    This is a sentence typeset in the theorem environment {\envfont
%    Cor}.
% \end{Cor}
%
% \begin{Exa}
%    This is a sentence typeset in the theorem environment {\envfont
%    Exa}.
% \end{Exa}
%
% \begin{Rem}
%    This is a sentence typeset in the theorem environment {\envfont
%    Rem}.
% \end{Rem}
%
% \begin{Lem}[Ben User]
%    This is a sentence typeset in the theorem environment {\envfont
%    Lem}.
% \end{Lem}
%
% \begin{Def}[Very impressive Definition]
%    This is a sentence typeset in the theorem environment {\envfont
%    Def}.
% \end{Def}
%
% The last two examples show the effect of the optional argument to
% a theorem environment (it is the text typeset in parentheses).
%
%
% \section{Special Considerations}
% \label{sec:newfonts}
%
% Theoremheader and body are implemented as a unit. This means that
% the |\theoremheaderfont|  will inherit characteristics of the
% |\theorembodyfont| in \LaTeXe. Thus, if for example
% |\theorembodyfont| is |\itshape| and |\theoremheaderfont|
% is |\bfseries| the font selected for the header will have the
% characteristics `bold extended italic'. If this is not desired one
% should set the |\theoremheaderfont| to something like
% \begin{verbatim}
%      \theoremheaderfont{\normalfont\bfseries}
%\end{verbatim}
% i.e.\ supplying all necessary font information explicitly.
%
%
%
% \section{Acknowledgements}
%
% The publication of this set of macros was only possible with the
% help of Christina Busse (translating the manuscript into English),
% Joachim Pense (playing the r\^ole of typist), Chris Rowley
% (looking everything over) and many others providing useful
% suggestions.
%
% \MaybeStop{%
%    \begin{thebibliography}{9}
%       \bibitem{bk:GMS94} \textsc{M.~Goossens}, \textsc{F.~Mittelbach}
%         and \textsc{A.~Samarin}.
%         \newblock The \LaTeX{} Companion.
%         \newblock
%         Addison-Wesley, Reading, Massachusetts, 1994.
%       \bibitem{latex:source} \textsc{Lamport, Leslie.} \newblock
%                               \texttt{latex.tex}, version 2.09, date
%                               Feb. 1990.
%    \end{thebibliography}
%    \PrintIndex
%    \PrintChanges}
%
%
% \section{The documentation driver file}
%
% The next bit of code contains the documentation driver file for
% \TeX{}, i.e., the file that will produce the documentation you are
% currently reading. It will be extracted from this file by the
% \texttt{docstrip} program. Since it is the first code in the file
% one can alternatively process this file directly with \LaTeXe{} to
% obtain the documentation.
%
%    \begin{macrocode}
%<*driver>
\documentclass{ltxdoc}

\usepackage{theorem}

%<+driver>% The next few lines define theorem sets which are used
%<+driver>% in the example section of the documentation.

\theoremstyle{break}       \newtheorem{Cor}{Corollary}
\theoremstyle{plain}       \newtheorem{Exa}{Example}[section]
{\theorembodyfont{\upshape}\newtheorem{Rem}{Remark}}
\theoremstyle{marginbreak} \newtheorem{Lem}[Cor]{Lemma}
\theoremstyle{change}
\theorembodyfont{\itshape} \newtheorem{Def}[Cor]{Definition}

\theoremheaderfont{\scshape}

\RecordChanges

\begin{document}
   \DocInput{theorem.dtx}
\end{document}
%</driver>
%    \end{macrocode}
%
% \section{Definition of the Macros}
%
% If the file has been loaded before, we abort immediately. If not the
% package announces itself (this is actually done at the very top of
% the file---the way it is done isn't good style so don't copy it).
% \changes{v2.0e}{89/07/19}{Spaces removed from `typeout'.}
% \changes{v2.2b}{95/11/19}{Announce moved to top of file}
%    \begin{macrocode}
%<*package>
%\@ifundefined{theorem@style}{}{\endinput}
%\def\FMithmInfo{1995/11/19 v2.2b Theorem extension package (FMi)}
%\ProvidesPackage{theorem}[\FMithmInfo]
%    \end{macrocode}
%
%
% \subsection{Definition of theorem styles and fonts}
%
% All the definitions in this file are done globally to allow
% inputting this file inside a group.
%
% \begin{macro}{\theoremstyle}
%    Before a theorem style can be installed, the chosen style must be
%    known. For that reason, we must test to see that
%    |\th@|\meta{style} is known or, more precisely, that it is
%    different from |\relax|. If the style is not known then
%    |\th@plain| is used.
%    \begin{macrocode}
\gdef\theoremstyle#1{%
   \@ifundefined{th@#1}{\@warning
          {Unknown theoremstyle `#1'. Using `plain'}%
          \theorem@style{plain}}%
%    \end{macrocode}
%    We save the theorem style to be used in the token register
%    |\theorem@style|.
%    \begin{macrocode}
      {\theorem@style{#1}}%
%    \end{macrocode}
%    Now we ``evaluate'' the theorem style: this means, we call the
%    macro |\th@|\meta{style} which will activate the relevant
%    definitions which are contained in a separate file.  This is done
%    in a group to suppress changes to the current font.  This could
%    otherwise pose problems together with the new font selection
%    scheme\footnote{When I printed the original article using the new
%    font selection scheme I ended with a document with slanted
%    typefaces (text headings and all) simply because one of the
%    theorem styles used \texttt{\bslash sl} at toplevel.} if the
%    |\th@|\meta{style} is evaluated a second time.
% \changes{v2.1a}{90/02/15}{Group added around `th@..}
% \changes{v2.1a}{90/02/15}{Reseting of `..@begintheorem removed.}
%    \begin{macrocode}
      \begingroup
        \csname th@\the\theorem@style \endcsname
      \endgroup}
%    \end{macrocode}
% \end{macro}
%
%
% \begin{macro}{\@begintheorem}
% \changes{v2.1a}{90/02/15}{Relaxed at toplevel.}
% \begin{macro}{\@opargbegintheorem}
% \changes{v2.1a}{90/02/15}{Relaxed at toplevel.}
%    We reset |\@begintheorem| and |\@opargbegintheorem| to
%    |\relax| since these commands are no longer necessary at
%    toplevel. This will save a few tokens.
%    \begin{macrocode}
\global\let\@begintheorem\relax
\global\let\@opargbegintheorem\relax
%    \end{macrocode}
% \end{macro}
% \end{macro}
%
% \begin{macro}{\theorem@style}
%    Obviously the \textsf{token} register used above has to be
%    allocated. To assure the utmost compatibility with the
%    original \LaTeX{} definition, we set the default theorem style to
%    \texttt{plain}, which implements the usual \LaTeX{} convention.
% \changes{v2.0f}{89/07/20}{Unnecessary `global' removed.}
%    \begin{macrocode}
\newtoks\theorem@style
\global\theorem@style{plain}
%    \end{macrocode}
% \end{macro}
%
% \begin{macro}{\theorembodyfont}
% \begin{macro}{\theorem@bodyfont}
%    If the |\theorembodyfont| is set by the user then it should not
%    interact with the default font set in the theorem style. When the
%    new font selection is in force this may happen if, for example,
%    the default is |\itshape| and the new |\theorembodyfont| is
%    |\sffamily|. So we add a |\reset@font| command in front of the
%    user definition.
% \changes{v2.0f}{89/07/20}{Unnecessary `global' removed.}
% \changes{v2.0g}{89/09/19}{`theoremfont to `theorembodyfont.}
% \changes{v2.1c}{92/06/22}{`reset@font added and logic changed.}
% \changes{v2.1d}{92/10/28}{Reset to empty if arg is empty}
%    \begin{macrocode}
\gdef\theorembodyfont#1{%
%    \end{macrocode}
%    We check if the argument supplied is empty and if so put nothing
%    into the |\theorem@bodyfont| token register to allow for
%    |\theorembodyfont{}| as a mean of using the default of the
%    current |\theoremstyle|.
%    \begin{macrocode}
   \def\@tempa{#1}%
   \ifx\@tempa\@empty
    \theorem@bodyfont{}%
   \else
    \theorem@bodyfont{\reset@font#1}%
   \fi
   }
\newtoks\theorem@bodyfont
\global\theorem@bodyfont{}
%    \end{macrocode}
% \end{macro}
% \end{macro}
%
%
% \begin{macro}{\theoremheaderfont}
% \changes{v2.0g}{89/09/19}{Macro added.}
%    The font for the theorem headers is handled differently because
%    this definition applies to all theorem styles.
%    \begin{macrocode}
\gdef\theoremheaderfont#1{\gdef\theorem@headerfont{#1}%
%    \end{macrocode}
%    After using the macro once it is redefined to produce an error
%    message.
%    \begin{macrocode}
       \gdef\theoremheaderfont##1{%
        \typeout{\string\theoremheaderfont\space should be used
                 only once.}}}
%    \end{macrocode}
% \end{macro}
%
% \begin{macro}{\theorem@headerfont}
% \changes{v2.0g}{89/09/19}{Macro added.}
%    To set the |\theorem@headerfont| default we first test if
%    the new fontselection scheme is in force.
% \changes{v2.1a}{90/02/15}{Support for new font selection.}
%    \begin{macrocode}
\ifx\upshape\undefined
%    \end{macrocode}
%    If not we define it to expand into |\bfseries|. We don't use
%    |\let| just in case a following style option redefines this
%    macro.
%    \begin{macrocode}
\gdef\theorem@headerfont{\bfseries}
%    \end{macrocode}
%    Otherwise we reset the current shape before calling |\bfseries|.
%    \begin{macrocode}
\else \gdef\theorem@headerfont{\normalfont\bfseries}\fi
%    \end{macrocode}
% \end{macro}
%
% \begin{macro}{\th@plain}
% \begin{macro}{\th@break}
% \begin{macro}{\th@marginbreak}
% \begin{macro}{\th@changebreak}
% \begin{macro}{\th@change}
% \begin{macro}{\th@margin}
%    The different styles are defined in macros such as
%    |\th@plain|.  Since memory space is precious in
%    ``non-Big-versions'', we have to avoid offering too many unused
%    definitions.  Therefore we define these styles in separate files
%    that can be loaded on demand.  Thus the commands themselves only
%    load these files. We use |\@input@|, a \LaTeXe{} internal command
%    that ensures that the file will be listed with |\listfiles|.
% \changes{v2.2a}{94/02/02}{Use `@input@ to load theorem layout files}
%    \begin{macrocode}
\gdef\th@plain{\@input@{thp.sty}}
\gdef\th@break{\@input@{thb.sty}}
\gdef\th@marginbreak{\@input@{thmb.sty}}
\gdef\th@changebreak{\@input@{thcb.sty}}
\gdef\th@change{\@input@{thc.sty}}
\gdef\th@margin{\@input@{thm.sty}}
%    \end{macrocode}
%    This list will be expanded when new styles become
%    available.  For testing,  just append new theorem substyles
%    as document options.
% \end{macro}
% \end{macro}
% \end{macro}
% \end{macro}
% \end{macro}
% \end{macro}
%
%
%
% \subsection{Definition of a new theorem set}
%
% As already pointed out, a new theorem environment can be defined in
% three different ways:
% \begin{verbatim}
%    \newtheorem{Lem}{Lemma}
%    \newtheorem{Lem}{Lemma}[section]
%    \newtheorem{Lem}[Theorem]{Lemma}
%\end{verbatim}
% The function of the macro |\newtheorem| is to recognize these cases
% and then to branch into one of the three macros |\@ynthm|, |\@xnthm|
% or |\@othm|.  This mechanism is adopted unchanged from
% \cite{latex:source}; the essential point here is that, for example,
% in the second case, the arguments \texttt{Lem}, \texttt{Lemma} and
% \texttt{section} are passed over to the macro |\@xnthm|.
%
% We inspect this case first because the others present fewer problems,
% and thus are easily derived from this one.
%
% \begin{macro}{\@xnthm}
%    For our example arguments, the macro |\@xnthm| must fulfill
%    the following:
%    \begin{itemize}
%       \item
%          Define a new \LaTeX-counter `Lem'
%       \item
%          reset this counter within a |\section|
%       \item
%          define the macro |\theLem|
%       \item
%          define the environment macros |\Lem| and
%          |\endLem| using the current
%          |\theoremstyle| and |\theorem@bodyfont|.
%    \end{itemize}
%    Obviously, all this should happen only if the first argument
%    of |\@xnthm| (i.e.\ \texttt{Lem} in our example) is chosen so
%    as not to conflict with any previously defined commands or
%    environments.  This test is performed by the \LaTeX{} macro
%    |\@ifdefinable|.
%    \begin{macrocode}
\gdef\@xnthm#1#2[#3]{\expandafter\@ifdefinable\csname #1\endcsname
%    \end{macrocode}
%    Therefore, the first argument of |\@ifdefinable| is the
%    expansion (in the example, |\Lem|) of
%    |\csname||#1\endcsname|.  The second argument is
%    executed only if the test has been completed successfully.
%    \begin{macrocode}
   {%
%    \end{macrocode}
%    Now we define the new counter. The names of the \LaTeX{}
%    macros employed should speak for themselves:
% \changes{v2.2b}{1995/10/07}{Use \cs{@newctr}}
% \changes{v2.2c}{1995/11/23}{Use \cs{@newctr} properly.}
%    \begin{macrocode}
    \@definecounter{#1}\@newctr{#1}[#3]%
%    \end{macrocode}
%    Using |\@newctr| will give a proper error message if the counter
%    in |#3| is not defined.
%    In defining `|\theLem|' we must generate the desired macro
%    name by use of |\expandafter| and |\csname|.
%    \begin{macrocode}
    \expandafter\xdef\csname the#1\endcsname
%    \end{macrocode}
%    An |\xdef| is used in order to make the definition global,
%    and to ensure that it contains the replacement texts of
%    |\@thmcountersep| and |\@thmcounter|.\footnote{These
%    two macros can be defined by the document style.  Their default
%    values produce a `\texttt{.}' as separation and an arabic
%    representation of the number.} However, not everything should be
%    expanded. For example, it saves space to use |\thesection|
%    instead of its---at times---lengthy expansion.
%    \begin{macrocode}
      {\expandafter \noexpand \csname the#3\endcsname
       \@thmcountersep \@thmcounter{#1}}%
%    \end{macrocode}
%    Thus with the defaults of \LaTeX{}, |\theLem| would be
%    replaced by the command sequence |\thesection.\arabic{Lem}|.
%
%    We will now look at the definition of the macro which is executed
%    at the beginning of the actual environment (in our example this
%    macro is |\Lem|). It should be noted that we use an
%    ``|\expandafter| trick'' to expand only certain parts of the
%    replacement text at the time of the definition.
%    \begin{macrocode}
    \def\@tempa{\global\@namedef{#1}}%
    \expandafter \@tempa \expandafter{%
%    \end{macrocode}
%    First, the macro that contains the current definitions of
%    |\@begintheorem| and |\@opargtheorem| should be called
%    up.  The name of this macro---as is already known---has the form
%    |\th@|\meta{theorem\ style}; therefore, it must be called by
%    \begin{macrocode}
      \csname th@\the \theorem@style
%    \end{macrocode}
%    In addition the default theorem font should be changeable, i.e.\
%    we have to insert the contents of |\theorem@bodyfont|. For
%    that reason, we expand even further, beyond |\endcsname|,
%    and thus insert the contents of the \textsf{token} register
%    |\theorem@bodyfont| in the replacement text.
%    \begin{macrocode}
            \expandafter \endcsname \the \theorem@bodyfont
%    \end{macrocode}
%    Now it is time to call the macro |\@thm| which takes over
%    the further processing.  It has two arguments: the current
%    counter name (in our example, \texttt{Lem}), and the text of the
%    label (in our example, \texttt{Lemma}).
%    \begin{macrocode}
     \@thm{#1}{#2}}%
%    \end{macrocode}
%    With this, the `sub-definition' is complete.  The macro
%    |\@endtheorem| ends a theorem environment and is, so far, nothing
%    but an |\endtrivlist|.  (Hence it is defined globally, and not
%    within the theorem styles.\footnote{This has to be changed as
%    soon as theorem styles that change \texttt{\bslash @endtheorem}
%    exist.  In such a case, all existing styles must be changed as
%    well since they will have to reset the macro.}) Therefore, we can
%    set it equivalent to the macro that ends the theorem set (in our
%    example, |\endLem|).  However, if some day theorem styles exist
%    that do change |\@endtheorem|, we would have to use the
%    commented-out line instead.
%    \begin{macrocode}
    \global \expandafter \let \csname end#1\endcsname \@endtheorem
%   \global\@namedef{end#1}{\@endtheorem}%
%    \end{macrocode}
%    With these commands all the required definitions are employed,
%    unless the test |\@ifdefinable| has failed.  Therefore, we
%    end the second argument of this macro and with it the definition
%    of |\@xnthm|.
%    \begin{macrocode}
   }}
%    \end{macrocode}
% \end{macro}
%
% \begin{macro}{\@ynthm}
%    The definition of |\@ynthm| is completely analogous. In
%    this case the new counter that is defined is not reset within
%    another counter; thus the definition of |\the...|
%    is simplified:
%    \begin{macrocode}
\gdef\@ynthm#1#2{\expandafter\@ifdefinable\csname #1\endcsname
   {\@definecounter{#1}%
    \expandafter\xdef\csname the#1\endcsname{\@thmcounter{#1}}%
%    \end{macrocode}
%    The rest of the definition corresponds literally to that of
%    |\@xnthm|:
%    \begin{macrocode}
    \def\@tempa{\global\@namedef{#1}}\expandafter \@tempa
     \expandafter{\csname th@\the \theorem@style \expandafter
     \endcsname \the\theorem@bodyfont \@thm{#1}{#2}}%
    \global \expandafter \let \csname end#1\endcsname \@endtheorem}}
%    \end{macrocode}
% \end{macro}
%
% \begin{macro}{\@othm}
%    The definition of |\@othm| does not contain anything new.
%    \begin{macrocode}
\gdef\@othm#1[#2]#3{%
%    \end{macrocode}
% \changes{v2.2b}{95/10/07}{Build in check for counter defined}
%    We do not define a new counter but instead use one that has
%    already been defined.  Thus the only definition we need is that
%    of this pseudo-counter (i.e.\ |\the|\meta{env.\ name}).
%    First we check if |#2| corresponds to a known counter name.
%    \begin{macrocode}
  \expandafter\ifx\csname c@#2\endcsname\relax
   \@nocounterr{#2}%
  \else
   \expandafter\@ifdefinable\csname #1\endcsname
   {\expandafter \xdef \csname the#1\endcsname
     {\expandafter \noexpand \csname the#2\endcsname}%
%    \end{macrocode}
%    All other parts of the definition can be adopted from
%    |\@xnthm|. We have to remember, though, that in this case
%    the name of the current counter and the theorem label have moved
%    to the second and third arguments.
%    \begin{macrocode}
    \def\@tempa{\global\@namedef{#1}}\expandafter \@tempa
     \expandafter{\csname th@\the \theorem@style \expandafter
     \endcsname \the\theorem@bodyfont \@thm{#2}{#3}}%
    \global \expandafter \let \csname end#1\endcsname \@endtheorem}%
  \fi}
%    \end{macrocode}
% \end{macro}
%
% \subsection{Macros that are employed in a {\envfont theorem}
%    environment}
%
% \begin{macro}{\@thm}
%    The macro |\@thm| has to increase the current counter.  Then,
%    depending on whether the environment has (or does not have) an
%    optional argument, it has to branch into either |\@begintheorem|
%    or |\@opargtheorem|.
%    \begin{macrocode}
\gdef\@thm#1#2{\refstepcounter{#1}%
%    \end{macrocode}
%    Now we start a {\envfont trivlist} environment, and give
%    |\@topsep| and |\@topsepadd| the values of the \textsf{skip}
%    registers |\theorempreskipamount| and |\theorempostskipamount|.
%    The value in |\@topsep| is the vertical space that is inserted by
%    the first (and only) |\item| in our |\trivlist| whilst
%    |\@topsepadd| is inserted by |\@endparenv| at the end of that
%    {\envfont trivlist} environment.  By using these registers, we
%    obtain the desired space around a {\envfont theorem} environment.
%    \begin{macrocode}
   \trivlist
   \@topsep \theorempreskipamount               % used by first \item
   \@topsepadd \theorempostskipamount           % used by \@endparenv
%    \end{macrocode}
%    Now we have to test whether an optional argument has been given.
% \changes{v2.0d}{89/07/18}{Forgotten percent added.}
%    \begin{macrocode}
   \@ifnextchar [%
%    \end{macrocode}
%    If there is an optional argument, we will call |\@ythm|, and
%    move the arguments read back into the input stream.
%    \begin{macrocode}
   {\@ythm{#1}{#2}}%
%    \end{macrocode}
%    If not, we call |\@begintheorem|. Its first argument is the
%    name of the theorem set (hence the second argument of
%    |\@thm|). Its second argument is the macro that produces the
%    current number.
%    \begin{macrocode}
   {\@begintheorem{#2}{\csname the#1\endcsname}\ignorespaces}}
%    \end{macrocode}
% \end{macro}
%
% \begin{macro}{\@xthm}
% \begin{macro}{\@ythm}
%    Both these macros were originally called by |\@thm|.  We do
%    not need |\@xthm| anymore, hence we reset it to
%    |\relax|. The definition of |\@ythm| has not changed at
%    all from its definition in \LaTeX. In order to make the macros
%    easier to understand, we will
%    nevertheless present it (commented out).
%    \begin{macrocode}
\global\let\@xthm\relax
% \def\@ythm#1#2[#3]{\@opargbegintheorem{#2}{\csname
%           the#1\endcsname}{#3}\ignorespaces}
%</package>
%    \end{macrocode}
%    The primitive |\ignorespaces| in |\@ythm| and
%    |\@thm| is needed to remove the spaces between the
%    |\begin{...}| and the actual text.
% \end{macro}
% \end{macro}
%
%
% \subsection{Definition of the theorem substyles}
%
% As already pointed out, the theorem substyles, defined below, are
% only loaded when necessary.  Note that all these substyles, except
% \texttt{plain}, have |\slshape| as the default body font.
%
% \subsubsection{The \texttt{plain} style}
%
% \label{dok:th@plain} As the following macros use |@|, we have
% to locally set the |\catcode| of this symbol to ``letter''.
% This happens within a group, so that we do not have to worry about
% which |\catcode| that symbol had before.
%    \begin{macrocode}
%<*thp>
\begingroup \makeatletter
%    \end{macrocode}
% Since we are now within a group, we must make all definitions
% globally.  First we make sure that \texttt{theorem.sty} is loaded.
% This will allow us to use this file as a document style option
% without having to call \texttt{theorem} itself as an option.  At the
% same time, we assure that at least version 2 is loaded, since
% |\theorem@style| was not defined in earlier versions.
%    \begin{macrocode}
\@ifundefined{theorem@style}{\input{theorem.sty}}{}
\ProvidesFile{thp.sty}
             [\FMithmInfo]
%    \end{macrocode}
%
%
% \begin{macro}{\th@plain}
% \begin{macro}{\@begintheorem}
% \begin{macro}{\@opargbegintheorem}
%    |\theoremstyle{plain}| corresponds to the original
%    definition, except that the distances to the surrounding text are
%    determined by the parameters |\theorempreskipamount| and
%    |\theorempostskipamount|.
%    First we set the default body font.
%    \begin{macrocode}
\gdef\th@plain{\normalfont\itshape
%    \end{macrocode}
%    Then we define |\@begintheorem| and
%    |\@opargbegintheorem|. These two macros define how the
%    header of a theorem is typeset.  |\@opargbegintheorem| will
%    be called if a \textsf{theorem} environment with an optional
%    argument is encountered; otherwise, the header is constructed by
%    calling |\@begintheorem|.  If one of these macros is
%    executed, we are within a \textsf{trivlist} environment started by
%    |\@thm|.  So the theorem header is produced with an
%    |\item| command.
%
%    Instead of specifying the header font directly, all standard
%    theorem styles use the |\theorem@headerfont| macro to allow
%    customization.  The extra space (|\labelsep|) is necessary
%    because of problems in the \textsf{trivlist} environment.
%    \begin{macrocode}
  \def\@begintheorem##1##2{%
        \item[\hskip\labelsep \theorem@headerfont ##1\ ##2]}%
%    \end{macrocode}
%    The definition of |\@opargbegintheorem| is completely
%    analogous. The only difference is the fact that there exists a
%    third argument (which is the optional parameter of the
%    environment and contains additional information about the
%    theorem).  Customarily we enclose it in parentheses.
%    \begin{macrocode}
\def\@opargbegintheorem##1##2##3{%
   \item[\hskip\labelsep \theorem@headerfont ##1\ ##2\ (##3)]}}
%    \end{macrocode}
% \end{macro}
% \end{macro}
% \end{macro}
%
% \noindent
% We conclude with an |\endgroup| to restore the
% |\catcode| of |@|.
%    \begin{macrocode}
\endgroup
%</thp>
%    \end{macrocode}
%
%
% \subsubsection{The \texttt{break} style}
%
% This style option is stored in the file |thb.sty|.
% For the next two lines see the documentation for |\th@plain| on
% page~\pageref{dok:th@plain}.
%    \begin{macrocode}
%<*thb>
\begingroup \makeatletter
\@ifundefined{theorem@style}{\input{theorem.sty}}{}
\ProvidesFile{thb.sty}
             [\FMithmInfo]
%    \end{macrocode}
%
% \begin{macro}{\th@break}
%    |\theoremstyle{break}| produces a line break after the name
%    of the theorem.  The font is |\slshape|.  Hence, we define
%    |\th@break| as follows:
%    \begin{macrocode}
\gdef\th@break{\normalfont\slshape
  \def\@begintheorem##1##2{\item[%
%    \end{macrocode}
%    We run into the following problem: it is not possible to create
%    the header with |\item[|\meta{title}|]| and then start
%    a new line by, for example, |\mbox{}\\|. Such a definition
%    will fail whenever a list environment follows immediately.  With
%    the above construction, the |\mbox{}| causes the switch
%    |@inlabel| (cf.\ definition of |\list| and
%    |\trivlist| in \cite{latex:source}) to be set to \texttt{false}
%    and so the following list will insert additional vertical
%    space (|\topskip|). This is quite annoying. Therefore, we
%    create the line break within the |\item|. In order to ensure
%    that the text will begin at the proper position in the following
%    line, we simply pretend that the label does not take any
%    room.\footnote{This will lead to problems whenever very high
%    symbols occurring in the line tower into the
%    heading. So, something else has to be done here sometime.}
%    \begin{macrocode}
       \rlap{\vbox{\hbox{\hskip \labelsep\theorem@headerfont ##1\ ##2}%
                   \hbox{\strut}}}]}%
%    \end{macrocode}
%    Again, the definition of |\@opargbegintheorem| is
%    completely analogous.
%    \begin{macrocode}
\def\@opargbegintheorem##1##2##3{%
  \item[\rlap{\vbox{\hbox{\hskip \labelsep \theorem@headerfont
                     ##1\ ##2\ (##3)}%
                    \hbox{\strut}}}]}}
\endgroup
%</thb>
%    \end{macrocode}
% \end{macro}
%
%
% \subsubsection{The \texttt{changebreak} style}
%
%    \begin{macrocode}
% This style option is stored in the file |thcb.sty|.
%    \begin{macrocode}
%<*thcb>
\begingroup \makeatletter
\@ifundefined{theorem@style}{\input{theorem.sty}}{}
\ProvidesFile{thcb.sty}
             [\FMithmInfo]
%    \end{macrocode}
%
% \begin{macro}{\th@changebreak}
%    The \texttt{change-break} theorem style is like \texttt{break}
%    but with interchange of theorem name and theorem number.  Thus we
%    define |\th@changebreak| as follows:
%    \begin{macrocode}
\gdef\th@changebreak{\normalfont\slshape
  \def\@begintheorem##1##2{\item
       [\rlap{\vbox{\hbox{\hskip\labelsep \theorem@headerfont ##2\ ##1}%
                    \hbox{\strut}}}]}%
\def\@opargbegintheorem##1##2##3{%
  \item[\rlap{\vbox{\hbox{\hskip\labelsep \theorem@headerfont
                     ##2\ ##1\ (##3)}%
                    \hbox{\strut}}}]}}
\endgroup
%</thcb>
%    \end{macrocode}
% \end{macro}
%
%
% \subsubsection{The \texttt{change} style}
%
% This style option is stored in the file |thc.sty|.
%    \begin{macrocode}
%<*thc>
\begingroup \makeatletter
\@ifundefined{theorem@style}{\input{theorem.sty}}{}
\ProvidesFile{thc.sty}
             [\FMithmInfo]
%    \end{macrocode}
%
% \begin{macro}{\th@change}
%    The \texttt{change} theorem style corresponds to the
%    \texttt{change break} style without a linebreak after the header.
%    To say it in another way, it's the same as the \texttt{plain}
%    style but with number and name interchanged and |\slshape| as the
%    default font.
%    \begin{macrocode}
\gdef\th@change{\normalfont\slshape
  \def\@begintheorem##1##2{\item
       [\hskip\labelsep \theorem@headerfont ##2\ ##1]}%
\def\@opargbegintheorem##1##2##3{%
  \item[\hskip\labelsep \theorem@headerfont ##2\ ##1\ (##3)]}}
\endgroup
%</thc>
%    \end{macrocode}
% \end{macro}
%
%
% \subsubsection{The \texttt{marginbreak} style}
%
% This style option is the one used most often at Mainz.  It is saved
% in the file |thmb.sty|.
%    \begin{macrocode}
%<*thmb>
\begingroup \makeatletter
\@ifundefined{theorem@style}{\input{theorem.sty}}{}
\ProvidesFile{thmb.sty}
             [\FMithmInfo]
%    \end{macrocode}
%
% \begin{macro}{\th@marginbreak}
%    The \texttt{margin break} style is nearly the same as the
%    \texttt{change break} style. The only difference is the placement
%    of the theorem number.  We use |\llap| to place it in the left
%    margin.
%
%    In this style |\labelsep| denotes the separation between the
%    number and the text.
%    \begin{macrocode}
\gdef\th@marginbreak{\normalfont\slshape
  \def\@begintheorem##1##2{\item
       [\rlap{\vbox{\theorem@headerfont
                    \hbox{\llap{##2}\hskip\labelsep ##1}%
                    \hbox{\strut}}}]}%
\def\@opargbegintheorem##1##2##3{%
  \item[\rlap{\vbox{\theorem@headerfont
                    \hbox{\llap{##2}\hskip\labelsep ##1\ (##3)}%
                    \hbox{\strut}}}]}}
\endgroup
%</thmb>
%    \end{macrocode}
% \changes{v2.1c}{90/02/17}{`endgroup outside macrocode env.}
% \end{macro}
%
%
%
% \subsubsection{The \texttt{margin} style}
%
% This style option is stored in the file |thm.sty|.
%    \begin{macrocode}
%<*thm>
\begingroup \makeatletter
\@ifundefined{theorem@style}{\input{theorem.sty}}{}
\ProvidesFile{thm.sty}
             [\FMithmInfo]
%    \end{macrocode}
%
% \begin{macro}{\th@margin}
%    Again this is only a variant of the theorem styles described
%    above without any new ideas.
%    \begin{macrocode}
\gdef\th@margin{\normalfont\slshape
  \def\@begintheorem##1##2{\item
       [\theorem@headerfont \llap{##2}\hskip\labelsep ##1]}%
\def\@opargbegintheorem##1##2##3{%
  \item[\theorem@headerfont \llap{##2}\hskip\labelsep ##1\ (##3)]}}
\endgroup
%</thm>
%    \end{macrocode}
% \end{macro}
%
%
%
%
% \subsection{Final Definitions}
%
% \begin{macro}{\theorempreskipamount}
% \begin{macro}{\theorempostskipamount}
%    The \textsf{skip} parameters that regulate the vertical empty space
%    before and after the {\envfont theorem} environment have
%    to be allocated as well.
% \changes{v2.0f}{89/07/20}{Unnecessary `global' removed.}
%    \begin{macrocode}
%<*package>
\newskip\theorempreskipamount
\newskip\theorempostskipamount
%    \end{macrocode}
%    Since we have used the same values for all theorem sets, we now can
%    assign them.
%    \begin{macrocode}
\global\setlength\theorempreskipamount{12pt plus 5pt minus 3pt}
\global\setlength\theorempostskipamount{8pt plus 3pt minus 1.5pt}
%    \end{macrocode}
% \end{macro}
% \end{macro}
%
%
% \begin{macro}{\@endtheorem}
%    The same holds for the macro |\@endtheorem|, which ends a
%    {\envfont theorem} environment. Since it is the same for all
%    theorem sets, it is removed from the macros |\th@|\meta{style}.
%    It simply ends the {\envfont trivlist} environment, which was
%    begun in |\@thm|.
%    \begin{macrocode}
\global\let\@endtheorem=\endtrivlist
%    \end{macrocode}
% \end{macro}
%
%
%
% \begin{macro}{\@preamblecmds}
%    All macros defined above are to be used only in the preamble.
%    Therefore, we insert them in |\@preamblecmds| which will disable
%    them at begin document. This is done by the internal \LaTeXe{}
%    command |\@onlypreamble|.
% \changes{v2.0g}{89/09/19}{`theoremheaderfont and `theorembodyfont
%    added.}
% \changes{v2.2a}{94/02/03}{Use \LaTeXe{} interface}
%    \begin{macrocode}
\@onlypreamble\@xnthm
\@onlypreamble\@ynthm
\@onlypreamble\@othm
\@onlypreamble\newtheorem
\@onlypreamble\theoremstyle
\@onlypreamble\theorembodyfont
\@onlypreamble\theoremheaderfont
%    \end{macrocode}
% \end{macro}
%
% Finally we declare the \texttt{plain} theorem style to be the default.
% \changes{v2.1b}{90/06/04}{Fixed missing `item by declaring default
%    style.}
%    \begin{macrocode}
\theoremstyle{plain}
%</package>
%    \end{macrocode}
%
% \Finale

\endinput
}{}
\ProvidesFile{thp.sty}
             [\FMithmInfo]
%    \end{macrocode}
%
%
% \begin{macro}{\th@plain}
% \begin{macro}{\@begintheorem}
% \begin{macro}{\@opargbegintheorem}
%    |\theoremstyle{plain}| corresponds to the original
%    definition, except that the distances to the surrounding text are
%    determined by the parameters |\theorempreskipamount| and
%    |\theorempostskipamount|.
%    First we set the default body font.
%    \begin{macrocode}
\gdef\th@plain{\normalfont\itshape
%    \end{macrocode}
%    Then we define |\@begintheorem| and
%    |\@opargbegintheorem|. These two macros define how the
%    header of a theorem is typeset.  |\@opargbegintheorem| will
%    be called if a \textsf{theorem} environment with an optional
%    argument is encountered; otherwise, the header is constructed by
%    calling |\@begintheorem|.  If one of these macros is
%    executed, we are within a \textsf{trivlist} environment started by
%    |\@thm|.  So the theorem header is produced with an
%    |\item| command.
%
%    Instead of specifying the header font directly, all standard
%    theorem styles use the |\theorem@headerfont| macro to allow
%    customization.  The extra space (|\labelsep|) is necessary
%    because of problems in the \textsf{trivlist} environment.
%    \begin{macrocode}
  \def\@begintheorem##1##2{%
        \item[\hskip\labelsep \theorem@headerfont ##1\ ##2]}%
%    \end{macrocode}
%    The definition of |\@opargbegintheorem| is completely
%    analogous. The only difference is the fact that there exists a
%    third argument (which is the optional parameter of the
%    environment and contains additional information about the
%    theorem).  Customarily we enclose it in parentheses.
%    \begin{macrocode}
\def\@opargbegintheorem##1##2##3{%
   \item[\hskip\labelsep \theorem@headerfont ##1\ ##2\ (##3)]}}
%    \end{macrocode}
% \end{macro}
% \end{macro}
% \end{macro}
%
% \noindent
% We conclude with an |\endgroup| to restore the
% |\catcode| of |@|.
%    \begin{macrocode}
\endgroup
%</thp>
%    \end{macrocode}
%
%
% \subsubsection{The \texttt{break} style}
%
% This style option is stored in the file |thb.sty|.
% For the next two lines see the documentation for |\th@plain| on
% page~\pageref{dok:th@plain}.
%    \begin{macrocode}
%<*thb>
\begingroup \makeatletter
\@ifundefined{theorem@style}{% \iffalse meta-comment
%
% Copyright (C) 1993-2024
%
% The LaTeX Project and any individual authors listed elsewhere
% in this file.
%
% This file is part of the Standard LaTeX `Tools Bundle'.
% -------------------------------------------------------
%
% It may be distributed and/or modified under the
% conditions of the LaTeX Project Public License, either version 1.3c
% of this license or (at your option) any later version.
% The latest version of this license is in
%    https://www.latex-project.org/lppl.txt
% and version 1.3c or later is part of all distributions of LaTeX
% version 2005/12/01 or later.
%
% The list of all files belonging to the LaTeX `Tools Bundle' is
% given in the file `manifest.txt'.
%
% \fi
%
%%
%
%
\def\FMithmInfo{2023/07/05 v2.2c Theorem extension package (FMi)}
%
% \ProvidesFile{theorem.dtx}[\FMithmInfo]
%\iffalse   % this is a METACOMMENT !
%%
%% Package `theorem' to use with LaTeX2e
%% Copyright (C) 1989-1995 Frank Mittelbach, all rights reserved.
%<+package>\@ifundefined{theorem@style}{}{\endinput}
%<+package>\NeedsTeXFormat{LaTeX2e}[1995/06/01]
%<+package>\ProvidesPackage{theorem}
%<+package>         [\FMithmInfo]
%
% \fi
%
% \changes{v1.0}{87/05/07}{First draft.}
% \changes{v1.0a}{87/05/07}{`thistheoremstyle' added.}
% \changes{v1.1a}{87/07/29}{New concept: redefine the internal
%                          \protect\LaTeX{} commands `@xnthm' etc.}
% \changes{v1.1b}{87/07/31}{`theoremfont' used to specify font.}
% \changes{v1.2a}{88/01/08}{`@endtheorem' now globally defined.}
% \changes{v1.2b}{88/01/11}{`newtheorem' only allowed in the preamble.}
% \changes{v1.3a}{88/02/14}{`theoremfont' now a toks register.}
%
% \changes{v2.0a}{88/11/16}{New implementation and new documentation.}
% \changes{v2.0a}{88/11/16}{No layout changes but internal modifications
%                           old substyles won't work.}
% \changes{v2.0b}{89/01/30}{`global' used for all definitions.}
% \changes{v2.0c}{89/06/15}{Documentation translated to English.
%                            Thanks to LLL and Nina}
% \changes{v2.1a}{90/02/15}{Support for new font selection.}
% \changes{v2.2a}{94/02/03}{Cleanup for LaTeX2e}
%
% \DoNotIndex{\@endparenv,\@ifdefinable,\@ifnextchar,\@ifundefined}
% \DoNotIndex{\@namedef,\@nameuse,\@spaces,\@tempa,\@warning}
% \DoNotIndex{\ }
% \DoNotIndex{\begingroup,\bfseries}
% \DoNotIndex{\csname}
% \DoNotIndex{\def,\do,\docdate}
% \DoNotIndex{\endcsname,\endgroup,\endinput,\endtrivlist,\expandafter}
% \DoNotIndex{\filedate,\fileversion}
% \DoNotIndex{\gdef,\global}
% \DoNotIndex{\hbox,\hskip}
% \DoNotIndex{\ignorespaces,\input,\itshape,\item}
% \DoNotIndex{\labelsep,\let,\llap}
% \DoNotIndex{\makeatletter}
% \DoNotIndex{\noexpand}
% \DoNotIndex{\refstepcounter,\relax,\rlap}
% \DoNotIndex{\setlength,\slshape,\space,\string,\strut}
% \DoNotIndex{\the,\trivlist,\typeout}
% \DoNotIndex{\vbox}
% \DoNotIndex{\xdef}
%
% \def\deflabel#1{\ttfamily #1\hfill}
% \def\deflist#1{\begin{list}{}{\settowidth\labelwidth{\ttfamily #1}%
%                               \setlength\leftmargin\labelwidth
%                               \addtolength\leftmargin\labelsep
%                               \let\makelabel\deflabel}}
% \def\enddeflist{\end{list}}
% \def\envfont{\normalfont\ttfamily}
%
% \MakeShortVerb{\|}
%
% \newcommand{\TUB}{{\sl TUGboat\/}}
%
% \GetFileInfo{theorem.dtx}
% \title{An Extension of the
%        \LaTeX{} theorem environment\thanks{This file
%        has version number \fileversion, last
%        revised \filedate.}}
% \author{Frank Mittelbach}
%
% \MaintainedByLaTeXTeam{tools}
% \maketitle
%
% \begin{abstract}
%    The macros described in this paper yield an extension of the
%    \LaTeX{} theorem mechanism. It is designed to satisfy the
%    different requirements of various journals. Thus, the
%    layout of the ``theorems'' can be manipulated by determining a
%    ``style''. This article describes not only the use, but
%    also the definition, of the necessary macros.
% \end{abstract}
%
% \section*{Preface to version 2.2}
%
% For \LaTeXe{} this package did not need any fundamental changes. I
% only modified the messages generated so that theorem layout styles
% will show up with the |\listfiles| command and cleaned the section
% on the New Font Selection Scheme since this is now included in
% \LaTeX.
%
% \section*{Preface to version 2.1}
%
% This version is identical to 2.0g described in \TUB\ 10\#3 except
% for some internal defaults which are now set depending on the used
% font selection scheme.
%
% This was done to avoid unpleasant surprises if the new font selection
% scheme is in force. For further details see
% section~\ref{sec:newfonts} and \cite{bk:GMS94}.
%
% \section{Introduction}
%
% For our purposes here, ``theorems'' are labelled enunciations,
% often set off from the main text by extra space and a font change.
% Theorems, corollaries, conjectures, definitions, and remarks
% are all instances of ``theorems''.  The ``header'' of these structures
% is composed of a label (such as \textsc{Theorem} or \textsc{Remark})
% and a number which serializes an item in the sequence of items
% with the same label.
%
% Shortly after the introduction of \LaTeX{} at the Fachbereich
% Mathematik in Mainz, the desire to manipulate the layout of
% ``theorems'' arose.  In Mainz, the following two conventions came
% into general use:
% \begin{enumerate}
%   \item The number of the theorem is shown in the margin.
%   \item There is a line break at
%         the end of the theorem header.
% \end{enumerate}
%
% Additionally, some journals require different formats which depend on
% the ``sort of theorem'': e.g.\ often remarks and definitions are set
% in |\upshape|, while |\itshape| is employed for main theorems.
%
% Confronted with these requirements, a theorem environment was
% developed in Mainz which allows separate determination of the
% layout of the ``theorems sets'', comparable to |\pagestyle|.
%
%
% \section{The user interface}
% \subsection{Defining new theorem sets}
%
% \DescribeMacro\newtheorem
% As in the original \LaTeX{} version, the command |\newtheorem|
% defines a new ``theorem set'' or ``theorem-like structure''.
% Two required arguments name the new environment and give the text
% to be typeset with each instance of the new ``set'', while
% an optional argument determines how
% the ``set'' is enumerated:
% \begin{description}
%    \item[\ttfamily \bslash newtheorem\{foo\}\{bar\}]
%       The theorem set {\envfont foo} (whose name is \texttt{bar})
%       uses its own counter.
%    \item[\ttfamily \bslash newtheorem\{foo2\}{[foo]}\{bar2\}]
%       The theorem set {\envfont foo2} (printed name \texttt{bar2})
%       uses the same counter as the theorem set \texttt{foo}.
%    \item[\ttfamily \bslash newtheorem\{foo3\}\{bar3\}{[section]}]
%       The theorem set {\envfont foo3} (printed name \texttt{bar3}) is
%       enumerated within the counter \texttt{section}, i.e.\ with every
%       new |\section| the enumeration begins again with 1, and
%       the enumeration is composed from the section-number and the
%       theorem counter itself.
% \end{description}
%
% \DescribeMacro\theoremstyle
% Additionally, the command |\theoremstyle| can define the layout
% of various, or all, theorem sets.  It should be noted that any
% theorem set defined by |\newtheorem| is typeset in the
% |\theoremstyle| that is current at the time of the definition.
% Thus, the following
% \begin{verbatim}
%    \theoremstyle{break}        \newtheorem{Cor}{Corollary}
%    \theoremstyle{plain}        \newtheorem{Exa}{Example}[section]
%\end{verbatim}                             ^^A No blank at line start
% leads to the result that the set {\envfont Cor} is formatted in the
% style \texttt{break}, while the set {\envfont Exa} and all the
% following ones are formatted in the style \texttt{plain}, unless
% another |\theoremstyle| follows.  Since the definitions installed by
% |\newtheorem| are global, one also can limit |\theoremstyle| locally
% by grouping braces.
%
% \DescribeMacro\theorembodyfont
% The choice of the font for the theorem body is completely
% independent of the chosen |\theoremstyle|; this has proven to
% be very advantageous.  For example,
% \begin{verbatim}
%   {\theorembodyfont{\upshape}          \newtheorem{Rem}{Remark}}
%\end{verbatim}
% defines a theorem set {\envfont Rem}, which will be set in
% |\upshape| in the current layout (which in our example is
% \texttt{plain}).  As with |\theoremstyle|, the |\theorembodyfont|
% chosen is that current at the time of |\newtheorem|.  If
% |\theorembodyfont| is not specified or one defines
% |\theorembodyfont{}|, then the font used will be that defined by the
% |\theoremstyle|.
%
% \DescribeMacro\theoremheaderfont It is also possible to customize
% the font used for the theorem headers.  This is, however, a global
% declaration, and therefore there should be at most one
% |\theoremheaderfont| declaration in the preamble.\footnote{If it is
% actually necessary to have different header fonts, one has to define
% new theorem styles (substituting the desired font) or specify the
% information directly in the \texttt{\bslash newtheorem} declaration
% (the unclean variant).}
%
% \DescribeMacro\theorempreskipamount
% \DescribeMacro\theorempostskipamount Two additional parameters
% affect the vertical space around the theorem environments:
% |\theorempreskipamount| and |\theorempostskipamount| define,
% respectively, the spacing before and after such an environment.
% These parameters apply for all theorem sets and can be manipulated
% with the ordinary length macros.  They are rubber lengths
% (`\textsf{skips}'), and therefore can contain \texttt{plus} and
% \texttt{minus} parts.
%
% Since the definition of theorem sets should---most sensibly---be
% placed in the preamble, we only allow installation there.
% It is therefore possible to release the memory used here after
% |\begin{document}|, in order to make room for other
% applications.
%
%
% \subsection{Existing theorem styles}
%
% The following theorem styles exist to date:
% \begin{deflist}{marginbreak}
%    \item[plain]
%       This theorem style emulates the original \LaTeX{} definition,
%       except that additionally the parameters
%       |\theorem...skipamount| are used.
%    \item[break]
%       In this style, the theorem header is followed by a line
%       break.
%    \item[marginbreak]
%       The theorem number is set in the margin, and there is a line
%       break as in \texttt{break}.
%    \item[changebreak]
%       Like \texttt{break}, but with header number and text
%       interchanged.
%    \item[change]
%       Header number and text are interchanged, without a line break.
%    \item[margin]
%       The number is set in the left margin, without a line break.
% \end{deflist}
% All styles (except \texttt{plain}) select |\slshape| as the default
% |\theorembodyfont|.
%
%
% \subsection{Examples}
%
% Given the above theorem sets {\envfont Cor}, {\envfont Exa}
% and {\envfont Rem},
% suppose that the preamble also contains the declarations:
% \begin{verbatim}
%    \theoremstyle{marginbreak}   \newtheorem{Lem}[Cor]{Lemma}
%    \theoremstyle{change}
%    \theorembodyfont{\itshape}        \newtheorem{Def}[Cor]{Definition}
%
%    \theoremheaderfont{\scshape}
%\end{verbatim}
% Then the following are some typical examples of the typeset output
% resulting from their use.
%
% \begin{Cor}
%    This is a sentence typeset in the theorem environment {\envfont
%    Cor}.
% \end{Cor}
%
% \begin{Exa}
%    This is a sentence typeset in the theorem environment {\envfont
%    Exa}.
% \end{Exa}
%
% \begin{Rem}
%    This is a sentence typeset in the theorem environment {\envfont
%    Rem}.
% \end{Rem}
%
% \begin{Lem}[Ben User]
%    This is a sentence typeset in the theorem environment {\envfont
%    Lem}.
% \end{Lem}
%
% \begin{Def}[Very impressive Definition]
%    This is a sentence typeset in the theorem environment {\envfont
%    Def}.
% \end{Def}
%
% The last two examples show the effect of the optional argument to
% a theorem environment (it is the text typeset in parentheses).
%
%
% \section{Special Considerations}
% \label{sec:newfonts}
%
% Theoremheader and body are implemented as a unit. This means that
% the |\theoremheaderfont|  will inherit characteristics of the
% |\theorembodyfont| in \LaTeXe. Thus, if for example
% |\theorembodyfont| is |\itshape| and |\theoremheaderfont|
% is |\bfseries| the font selected for the header will have the
% characteristics `bold extended italic'. If this is not desired one
% should set the |\theoremheaderfont| to something like
% \begin{verbatim}
%      \theoremheaderfont{\normalfont\bfseries}
%\end{verbatim}
% i.e.\ supplying all necessary font information explicitly.
%
%
%
% \section{Acknowledgements}
%
% The publication of this set of macros was only possible with the
% help of Christina Busse (translating the manuscript into English),
% Joachim Pense (playing the r\^ole of typist), Chris Rowley
% (looking everything over) and many others providing useful
% suggestions.
%
% \MaybeStop{%
%    \begin{thebibliography}{9}
%       \bibitem{bk:GMS94} \textsc{M.~Goossens}, \textsc{F.~Mittelbach}
%         and \textsc{A.~Samarin}.
%         \newblock The \LaTeX{} Companion.
%         \newblock
%         Addison-Wesley, Reading, Massachusetts, 1994.
%       \bibitem{latex:source} \textsc{Lamport, Leslie.} \newblock
%                               \texttt{latex.tex}, version 2.09, date
%                               Feb. 1990.
%    \end{thebibliography}
%    \PrintIndex
%    \PrintChanges}
%
%
% \section{The documentation driver file}
%
% The next bit of code contains the documentation driver file for
% \TeX{}, i.e., the file that will produce the documentation you are
% currently reading. It will be extracted from this file by the
% \texttt{docstrip} program. Since it is the first code in the file
% one can alternatively process this file directly with \LaTeXe{} to
% obtain the documentation.
%
%    \begin{macrocode}
%<*driver>
\documentclass{ltxdoc}

\usepackage{theorem}

%<+driver>% The next few lines define theorem sets which are used
%<+driver>% in the example section of the documentation.

\theoremstyle{break}       \newtheorem{Cor}{Corollary}
\theoremstyle{plain}       \newtheorem{Exa}{Example}[section]
{\theorembodyfont{\upshape}\newtheorem{Rem}{Remark}}
\theoremstyle{marginbreak} \newtheorem{Lem}[Cor]{Lemma}
\theoremstyle{change}
\theorembodyfont{\itshape} \newtheorem{Def}[Cor]{Definition}

\theoremheaderfont{\scshape}

\RecordChanges

\begin{document}
   \DocInput{theorem.dtx}
\end{document}
%</driver>
%    \end{macrocode}
%
% \section{Definition of the Macros}
%
% If the file has been loaded before, we abort immediately. If not the
% package announces itself (this is actually done at the very top of
% the file---the way it is done isn't good style so don't copy it).
% \changes{v2.0e}{89/07/19}{Spaces removed from `typeout'.}
% \changes{v2.2b}{95/11/19}{Announce moved to top of file}
%    \begin{macrocode}
%<*package>
%\@ifundefined{theorem@style}{}{\endinput}
%\def\FMithmInfo{1995/11/19 v2.2b Theorem extension package (FMi)}
%\ProvidesPackage{theorem}[\FMithmInfo]
%    \end{macrocode}
%
%
% \subsection{Definition of theorem styles and fonts}
%
% All the definitions in this file are done globally to allow
% inputting this file inside a group.
%
% \begin{macro}{\theoremstyle}
%    Before a theorem style can be installed, the chosen style must be
%    known. For that reason, we must test to see that
%    |\th@|\meta{style} is known or, more precisely, that it is
%    different from |\relax|. If the style is not known then
%    |\th@plain| is used.
%    \begin{macrocode}
\gdef\theoremstyle#1{%
   \@ifundefined{th@#1}{\@warning
          {Unknown theoremstyle `#1'. Using `plain'}%
          \theorem@style{plain}}%
%    \end{macrocode}
%    We save the theorem style to be used in the token register
%    |\theorem@style|.
%    \begin{macrocode}
      {\theorem@style{#1}}%
%    \end{macrocode}
%    Now we ``evaluate'' the theorem style: this means, we call the
%    macro |\th@|\meta{style} which will activate the relevant
%    definitions which are contained in a separate file.  This is done
%    in a group to suppress changes to the current font.  This could
%    otherwise pose problems together with the new font selection
%    scheme\footnote{When I printed the original article using the new
%    font selection scheme I ended with a document with slanted
%    typefaces (text headings and all) simply because one of the
%    theorem styles used \texttt{\bslash sl} at toplevel.} if the
%    |\th@|\meta{style} is evaluated a second time.
% \changes{v2.1a}{90/02/15}{Group added around `th@..}
% \changes{v2.1a}{90/02/15}{Reseting of `..@begintheorem removed.}
%    \begin{macrocode}
      \begingroup
        \csname th@\the\theorem@style \endcsname
      \endgroup}
%    \end{macrocode}
% \end{macro}
%
%
% \begin{macro}{\@begintheorem}
% \changes{v2.1a}{90/02/15}{Relaxed at toplevel.}
% \begin{macro}{\@opargbegintheorem}
% \changes{v2.1a}{90/02/15}{Relaxed at toplevel.}
%    We reset |\@begintheorem| and |\@opargbegintheorem| to
%    |\relax| since these commands are no longer necessary at
%    toplevel. This will save a few tokens.
%    \begin{macrocode}
\global\let\@begintheorem\relax
\global\let\@opargbegintheorem\relax
%    \end{macrocode}
% \end{macro}
% \end{macro}
%
% \begin{macro}{\theorem@style}
%    Obviously the \textsf{token} register used above has to be
%    allocated. To assure the utmost compatibility with the
%    original \LaTeX{} definition, we set the default theorem style to
%    \texttt{plain}, which implements the usual \LaTeX{} convention.
% \changes{v2.0f}{89/07/20}{Unnecessary `global' removed.}
%    \begin{macrocode}
\newtoks\theorem@style
\global\theorem@style{plain}
%    \end{macrocode}
% \end{macro}
%
% \begin{macro}{\theorembodyfont}
% \begin{macro}{\theorem@bodyfont}
%    If the |\theorembodyfont| is set by the user then it should not
%    interact with the default font set in the theorem style. When the
%    new font selection is in force this may happen if, for example,
%    the default is |\itshape| and the new |\theorembodyfont| is
%    |\sffamily|. So we add a |\reset@font| command in front of the
%    user definition.
% \changes{v2.0f}{89/07/20}{Unnecessary `global' removed.}
% \changes{v2.0g}{89/09/19}{`theoremfont to `theorembodyfont.}
% \changes{v2.1c}{92/06/22}{`reset@font added and logic changed.}
% \changes{v2.1d}{92/10/28}{Reset to empty if arg is empty}
%    \begin{macrocode}
\gdef\theorembodyfont#1{%
%    \end{macrocode}
%    We check if the argument supplied is empty and if so put nothing
%    into the |\theorem@bodyfont| token register to allow for
%    |\theorembodyfont{}| as a mean of using the default of the
%    current |\theoremstyle|.
%    \begin{macrocode}
   \def\@tempa{#1}%
   \ifx\@tempa\@empty
    \theorem@bodyfont{}%
   \else
    \theorem@bodyfont{\reset@font#1}%
   \fi
   }
\newtoks\theorem@bodyfont
\global\theorem@bodyfont{}
%    \end{macrocode}
% \end{macro}
% \end{macro}
%
%
% \begin{macro}{\theoremheaderfont}
% \changes{v2.0g}{89/09/19}{Macro added.}
%    The font for the theorem headers is handled differently because
%    this definition applies to all theorem styles.
%    \begin{macrocode}
\gdef\theoremheaderfont#1{\gdef\theorem@headerfont{#1}%
%    \end{macrocode}
%    After using the macro once it is redefined to produce an error
%    message.
%    \begin{macrocode}
       \gdef\theoremheaderfont##1{%
        \typeout{\string\theoremheaderfont\space should be used
                 only once.}}}
%    \end{macrocode}
% \end{macro}
%
% \begin{macro}{\theorem@headerfont}
% \changes{v2.0g}{89/09/19}{Macro added.}
%    To set the |\theorem@headerfont| default we first test if
%    the new fontselection scheme is in force.
% \changes{v2.1a}{90/02/15}{Support for new font selection.}
%    \begin{macrocode}
\ifx\upshape\undefined
%    \end{macrocode}
%    If not we define it to expand into |\bfseries|. We don't use
%    |\let| just in case a following style option redefines this
%    macro.
%    \begin{macrocode}
\gdef\theorem@headerfont{\bfseries}
%    \end{macrocode}
%    Otherwise we reset the current shape before calling |\bfseries|.
%    \begin{macrocode}
\else \gdef\theorem@headerfont{\normalfont\bfseries}\fi
%    \end{macrocode}
% \end{macro}
%
% \begin{macro}{\th@plain}
% \begin{macro}{\th@break}
% \begin{macro}{\th@marginbreak}
% \begin{macro}{\th@changebreak}
% \begin{macro}{\th@change}
% \begin{macro}{\th@margin}
%    The different styles are defined in macros such as
%    |\th@plain|.  Since memory space is precious in
%    ``non-Big-versions'', we have to avoid offering too many unused
%    definitions.  Therefore we define these styles in separate files
%    that can be loaded on demand.  Thus the commands themselves only
%    load these files. We use |\@input@|, a \LaTeXe{} internal command
%    that ensures that the file will be listed with |\listfiles|.
% \changes{v2.2a}{94/02/02}{Use `@input@ to load theorem layout files}
%    \begin{macrocode}
\gdef\th@plain{\@input@{thp.sty}}
\gdef\th@break{\@input@{thb.sty}}
\gdef\th@marginbreak{\@input@{thmb.sty}}
\gdef\th@changebreak{\@input@{thcb.sty}}
\gdef\th@change{\@input@{thc.sty}}
\gdef\th@margin{\@input@{thm.sty}}
%    \end{macrocode}
%    This list will be expanded when new styles become
%    available.  For testing,  just append new theorem substyles
%    as document options.
% \end{macro}
% \end{macro}
% \end{macro}
% \end{macro}
% \end{macro}
% \end{macro}
%
%
%
% \subsection{Definition of a new theorem set}
%
% As already pointed out, a new theorem environment can be defined in
% three different ways:
% \begin{verbatim}
%    \newtheorem{Lem}{Lemma}
%    \newtheorem{Lem}{Lemma}[section]
%    \newtheorem{Lem}[Theorem]{Lemma}
%\end{verbatim}
% The function of the macro |\newtheorem| is to recognize these cases
% and then to branch into one of the three macros |\@ynthm|, |\@xnthm|
% or |\@othm|.  This mechanism is adopted unchanged from
% \cite{latex:source}; the essential point here is that, for example,
% in the second case, the arguments \texttt{Lem}, \texttt{Lemma} and
% \texttt{section} are passed over to the macro |\@xnthm|.
%
% We inspect this case first because the others present fewer problems,
% and thus are easily derived from this one.
%
% \begin{macro}{\@xnthm}
%    For our example arguments, the macro |\@xnthm| must fulfill
%    the following:
%    \begin{itemize}
%       \item
%          Define a new \LaTeX-counter `Lem'
%       \item
%          reset this counter within a |\section|
%       \item
%          define the macro |\theLem|
%       \item
%          define the environment macros |\Lem| and
%          |\endLem| using the current
%          |\theoremstyle| and |\theorem@bodyfont|.
%    \end{itemize}
%    Obviously, all this should happen only if the first argument
%    of |\@xnthm| (i.e.\ \texttt{Lem} in our example) is chosen so
%    as not to conflict with any previously defined commands or
%    environments.  This test is performed by the \LaTeX{} macro
%    |\@ifdefinable|.
%    \begin{macrocode}
\gdef\@xnthm#1#2[#3]{\expandafter\@ifdefinable\csname #1\endcsname
%    \end{macrocode}
%    Therefore, the first argument of |\@ifdefinable| is the
%    expansion (in the example, |\Lem|) of
%    |\csname||#1\endcsname|.  The second argument is
%    executed only if the test has been completed successfully.
%    \begin{macrocode}
   {%
%    \end{macrocode}
%    Now we define the new counter. The names of the \LaTeX{}
%    macros employed should speak for themselves:
% \changes{v2.2b}{1995/10/07}{Use \cs{@newctr}}
% \changes{v2.2c}{1995/11/23}{Use \cs{@newctr} properly.}
%    \begin{macrocode}
    \@definecounter{#1}\@newctr{#1}[#3]%
%    \end{macrocode}
%    Using |\@newctr| will give a proper error message if the counter
%    in |#3| is not defined.
%    In defining `|\theLem|' we must generate the desired macro
%    name by use of |\expandafter| and |\csname|.
%    \begin{macrocode}
    \expandafter\xdef\csname the#1\endcsname
%    \end{macrocode}
%    An |\xdef| is used in order to make the definition global,
%    and to ensure that it contains the replacement texts of
%    |\@thmcountersep| and |\@thmcounter|.\footnote{These
%    two macros can be defined by the document style.  Their default
%    values produce a `\texttt{.}' as separation and an arabic
%    representation of the number.} However, not everything should be
%    expanded. For example, it saves space to use |\thesection|
%    instead of its---at times---lengthy expansion.
%    \begin{macrocode}
      {\expandafter \noexpand \csname the#3\endcsname
       \@thmcountersep \@thmcounter{#1}}%
%    \end{macrocode}
%    Thus with the defaults of \LaTeX{}, |\theLem| would be
%    replaced by the command sequence |\thesection.\arabic{Lem}|.
%
%    We will now look at the definition of the macro which is executed
%    at the beginning of the actual environment (in our example this
%    macro is |\Lem|). It should be noted that we use an
%    ``|\expandafter| trick'' to expand only certain parts of the
%    replacement text at the time of the definition.
%    \begin{macrocode}
    \def\@tempa{\global\@namedef{#1}}%
    \expandafter \@tempa \expandafter{%
%    \end{macrocode}
%    First, the macro that contains the current definitions of
%    |\@begintheorem| and |\@opargtheorem| should be called
%    up.  The name of this macro---as is already known---has the form
%    |\th@|\meta{theorem\ style}; therefore, it must be called by
%    \begin{macrocode}
      \csname th@\the \theorem@style
%    \end{macrocode}
%    In addition the default theorem font should be changeable, i.e.\
%    we have to insert the contents of |\theorem@bodyfont|. For
%    that reason, we expand even further, beyond |\endcsname|,
%    and thus insert the contents of the \textsf{token} register
%    |\theorem@bodyfont| in the replacement text.
%    \begin{macrocode}
            \expandafter \endcsname \the \theorem@bodyfont
%    \end{macrocode}
%    Now it is time to call the macro |\@thm| which takes over
%    the further processing.  It has two arguments: the current
%    counter name (in our example, \texttt{Lem}), and the text of the
%    label (in our example, \texttt{Lemma}).
%    \begin{macrocode}
     \@thm{#1}{#2}}%
%    \end{macrocode}
%    With this, the `sub-definition' is complete.  The macro
%    |\@endtheorem| ends a theorem environment and is, so far, nothing
%    but an |\endtrivlist|.  (Hence it is defined globally, and not
%    within the theorem styles.\footnote{This has to be changed as
%    soon as theorem styles that change \texttt{\bslash @endtheorem}
%    exist.  In such a case, all existing styles must be changed as
%    well since they will have to reset the macro.}) Therefore, we can
%    set it equivalent to the macro that ends the theorem set (in our
%    example, |\endLem|).  However, if some day theorem styles exist
%    that do change |\@endtheorem|, we would have to use the
%    commented-out line instead.
%    \begin{macrocode}
    \global \expandafter \let \csname end#1\endcsname \@endtheorem
%   \global\@namedef{end#1}{\@endtheorem}%
%    \end{macrocode}
%    With these commands all the required definitions are employed,
%    unless the test |\@ifdefinable| has failed.  Therefore, we
%    end the second argument of this macro and with it the definition
%    of |\@xnthm|.
%    \begin{macrocode}
   }}
%    \end{macrocode}
% \end{macro}
%
% \begin{macro}{\@ynthm}
%    The definition of |\@ynthm| is completely analogous. In
%    this case the new counter that is defined is not reset within
%    another counter; thus the definition of |\the...|
%    is simplified:
%    \begin{macrocode}
\gdef\@ynthm#1#2{\expandafter\@ifdefinable\csname #1\endcsname
   {\@definecounter{#1}%
    \expandafter\xdef\csname the#1\endcsname{\@thmcounter{#1}}%
%    \end{macrocode}
%    The rest of the definition corresponds literally to that of
%    |\@xnthm|:
%    \begin{macrocode}
    \def\@tempa{\global\@namedef{#1}}\expandafter \@tempa
     \expandafter{\csname th@\the \theorem@style \expandafter
     \endcsname \the\theorem@bodyfont \@thm{#1}{#2}}%
    \global \expandafter \let \csname end#1\endcsname \@endtheorem}}
%    \end{macrocode}
% \end{macro}
%
% \begin{macro}{\@othm}
%    The definition of |\@othm| does not contain anything new.
%    \begin{macrocode}
\gdef\@othm#1[#2]#3{%
%    \end{macrocode}
% \changes{v2.2b}{95/10/07}{Build in check for counter defined}
%    We do not define a new counter but instead use one that has
%    already been defined.  Thus the only definition we need is that
%    of this pseudo-counter (i.e.\ |\the|\meta{env.\ name}).
%    First we check if |#2| corresponds to a known counter name.
%    \begin{macrocode}
  \expandafter\ifx\csname c@#2\endcsname\relax
   \@nocounterr{#2}%
  \else
   \expandafter\@ifdefinable\csname #1\endcsname
   {\expandafter \xdef \csname the#1\endcsname
     {\expandafter \noexpand \csname the#2\endcsname}%
%    \end{macrocode}
%    All other parts of the definition can be adopted from
%    |\@xnthm|. We have to remember, though, that in this case
%    the name of the current counter and the theorem label have moved
%    to the second and third arguments.
%    \begin{macrocode}
    \def\@tempa{\global\@namedef{#1}}\expandafter \@tempa
     \expandafter{\csname th@\the \theorem@style \expandafter
     \endcsname \the\theorem@bodyfont \@thm{#2}{#3}}%
    \global \expandafter \let \csname end#1\endcsname \@endtheorem}%
  \fi}
%    \end{macrocode}
% \end{macro}
%
% \subsection{Macros that are employed in a {\envfont theorem}
%    environment}
%
% \begin{macro}{\@thm}
%    The macro |\@thm| has to increase the current counter.  Then,
%    depending on whether the environment has (or does not have) an
%    optional argument, it has to branch into either |\@begintheorem|
%    or |\@opargtheorem|.
%    \begin{macrocode}
\gdef\@thm#1#2{\refstepcounter{#1}%
%    \end{macrocode}
%    Now we start a {\envfont trivlist} environment, and give
%    |\@topsep| and |\@topsepadd| the values of the \textsf{skip}
%    registers |\theorempreskipamount| and |\theorempostskipamount|.
%    The value in |\@topsep| is the vertical space that is inserted by
%    the first (and only) |\item| in our |\trivlist| whilst
%    |\@topsepadd| is inserted by |\@endparenv| at the end of that
%    {\envfont trivlist} environment.  By using these registers, we
%    obtain the desired space around a {\envfont theorem} environment.
%    \begin{macrocode}
   \trivlist
   \@topsep \theorempreskipamount               % used by first \item
   \@topsepadd \theorempostskipamount           % used by \@endparenv
%    \end{macrocode}
%    Now we have to test whether an optional argument has been given.
% \changes{v2.0d}{89/07/18}{Forgotten percent added.}
%    \begin{macrocode}
   \@ifnextchar [%
%    \end{macrocode}
%    If there is an optional argument, we will call |\@ythm|, and
%    move the arguments read back into the input stream.
%    \begin{macrocode}
   {\@ythm{#1}{#2}}%
%    \end{macrocode}
%    If not, we call |\@begintheorem|. Its first argument is the
%    name of the theorem set (hence the second argument of
%    |\@thm|). Its second argument is the macro that produces the
%    current number.
%    \begin{macrocode}
   {\@begintheorem{#2}{\csname the#1\endcsname}\ignorespaces}}
%    \end{macrocode}
% \end{macro}
%
% \begin{macro}{\@xthm}
% \begin{macro}{\@ythm}
%    Both these macros were originally called by |\@thm|.  We do
%    not need |\@xthm| anymore, hence we reset it to
%    |\relax|. The definition of |\@ythm| has not changed at
%    all from its definition in \LaTeX. In order to make the macros
%    easier to understand, we will
%    nevertheless present it (commented out).
%    \begin{macrocode}
\global\let\@xthm\relax
% \def\@ythm#1#2[#3]{\@opargbegintheorem{#2}{\csname
%           the#1\endcsname}{#3}\ignorespaces}
%</package>
%    \end{macrocode}
%    The primitive |\ignorespaces| in |\@ythm| and
%    |\@thm| is needed to remove the spaces between the
%    |\begin{...}| and the actual text.
% \end{macro}
% \end{macro}
%
%
% \subsection{Definition of the theorem substyles}
%
% As already pointed out, the theorem substyles, defined below, are
% only loaded when necessary.  Note that all these substyles, except
% \texttt{plain}, have |\slshape| as the default body font.
%
% \subsubsection{The \texttt{plain} style}
%
% \label{dok:th@plain} As the following macros use |@|, we have
% to locally set the |\catcode| of this symbol to ``letter''.
% This happens within a group, so that we do not have to worry about
% which |\catcode| that symbol had before.
%    \begin{macrocode}
%<*thp>
\begingroup \makeatletter
%    \end{macrocode}
% Since we are now within a group, we must make all definitions
% globally.  First we make sure that \texttt{theorem.sty} is loaded.
% This will allow us to use this file as a document style option
% without having to call \texttt{theorem} itself as an option.  At the
% same time, we assure that at least version 2 is loaded, since
% |\theorem@style| was not defined in earlier versions.
%    \begin{macrocode}
\@ifundefined{theorem@style}{\input{theorem.sty}}{}
\ProvidesFile{thp.sty}
             [\FMithmInfo]
%    \end{macrocode}
%
%
% \begin{macro}{\th@plain}
% \begin{macro}{\@begintheorem}
% \begin{macro}{\@opargbegintheorem}
%    |\theoremstyle{plain}| corresponds to the original
%    definition, except that the distances to the surrounding text are
%    determined by the parameters |\theorempreskipamount| and
%    |\theorempostskipamount|.
%    First we set the default body font.
%    \begin{macrocode}
\gdef\th@plain{\normalfont\itshape
%    \end{macrocode}
%    Then we define |\@begintheorem| and
%    |\@opargbegintheorem|. These two macros define how the
%    header of a theorem is typeset.  |\@opargbegintheorem| will
%    be called if a \textsf{theorem} environment with an optional
%    argument is encountered; otherwise, the header is constructed by
%    calling |\@begintheorem|.  If one of these macros is
%    executed, we are within a \textsf{trivlist} environment started by
%    |\@thm|.  So the theorem header is produced with an
%    |\item| command.
%
%    Instead of specifying the header font directly, all standard
%    theorem styles use the |\theorem@headerfont| macro to allow
%    customization.  The extra space (|\labelsep|) is necessary
%    because of problems in the \textsf{trivlist} environment.
%    \begin{macrocode}
  \def\@begintheorem##1##2{%
        \item[\hskip\labelsep \theorem@headerfont ##1\ ##2]}%
%    \end{macrocode}
%    The definition of |\@opargbegintheorem| is completely
%    analogous. The only difference is the fact that there exists a
%    third argument (which is the optional parameter of the
%    environment and contains additional information about the
%    theorem).  Customarily we enclose it in parentheses.
%    \begin{macrocode}
\def\@opargbegintheorem##1##2##3{%
   \item[\hskip\labelsep \theorem@headerfont ##1\ ##2\ (##3)]}}
%    \end{macrocode}
% \end{macro}
% \end{macro}
% \end{macro}
%
% \noindent
% We conclude with an |\endgroup| to restore the
% |\catcode| of |@|.
%    \begin{macrocode}
\endgroup
%</thp>
%    \end{macrocode}
%
%
% \subsubsection{The \texttt{break} style}
%
% This style option is stored in the file |thb.sty|.
% For the next two lines see the documentation for |\th@plain| on
% page~\pageref{dok:th@plain}.
%    \begin{macrocode}
%<*thb>
\begingroup \makeatletter
\@ifundefined{theorem@style}{\input{theorem.sty}}{}
\ProvidesFile{thb.sty}
             [\FMithmInfo]
%    \end{macrocode}
%
% \begin{macro}{\th@break}
%    |\theoremstyle{break}| produces a line break after the name
%    of the theorem.  The font is |\slshape|.  Hence, we define
%    |\th@break| as follows:
%    \begin{macrocode}
\gdef\th@break{\normalfont\slshape
  \def\@begintheorem##1##2{\item[%
%    \end{macrocode}
%    We run into the following problem: it is not possible to create
%    the header with |\item[|\meta{title}|]| and then start
%    a new line by, for example, |\mbox{}\\|. Such a definition
%    will fail whenever a list environment follows immediately.  With
%    the above construction, the |\mbox{}| causes the switch
%    |@inlabel| (cf.\ definition of |\list| and
%    |\trivlist| in \cite{latex:source}) to be set to \texttt{false}
%    and so the following list will insert additional vertical
%    space (|\topskip|). This is quite annoying. Therefore, we
%    create the line break within the |\item|. In order to ensure
%    that the text will begin at the proper position in the following
%    line, we simply pretend that the label does not take any
%    room.\footnote{This will lead to problems whenever very high
%    symbols occurring in the line tower into the
%    heading. So, something else has to be done here sometime.}
%    \begin{macrocode}
       \rlap{\vbox{\hbox{\hskip \labelsep\theorem@headerfont ##1\ ##2}%
                   \hbox{\strut}}}]}%
%    \end{macrocode}
%    Again, the definition of |\@opargbegintheorem| is
%    completely analogous.
%    \begin{macrocode}
\def\@opargbegintheorem##1##2##3{%
  \item[\rlap{\vbox{\hbox{\hskip \labelsep \theorem@headerfont
                     ##1\ ##2\ (##3)}%
                    \hbox{\strut}}}]}}
\endgroup
%</thb>
%    \end{macrocode}
% \end{macro}
%
%
% \subsubsection{The \texttt{changebreak} style}
%
%    \begin{macrocode}
% This style option is stored in the file |thcb.sty|.
%    \begin{macrocode}
%<*thcb>
\begingroup \makeatletter
\@ifundefined{theorem@style}{\input{theorem.sty}}{}
\ProvidesFile{thcb.sty}
             [\FMithmInfo]
%    \end{macrocode}
%
% \begin{macro}{\th@changebreak}
%    The \texttt{change-break} theorem style is like \texttt{break}
%    but with interchange of theorem name and theorem number.  Thus we
%    define |\th@changebreak| as follows:
%    \begin{macrocode}
\gdef\th@changebreak{\normalfont\slshape
  \def\@begintheorem##1##2{\item
       [\rlap{\vbox{\hbox{\hskip\labelsep \theorem@headerfont ##2\ ##1}%
                    \hbox{\strut}}}]}%
\def\@opargbegintheorem##1##2##3{%
  \item[\rlap{\vbox{\hbox{\hskip\labelsep \theorem@headerfont
                     ##2\ ##1\ (##3)}%
                    \hbox{\strut}}}]}}
\endgroup
%</thcb>
%    \end{macrocode}
% \end{macro}
%
%
% \subsubsection{The \texttt{change} style}
%
% This style option is stored in the file |thc.sty|.
%    \begin{macrocode}
%<*thc>
\begingroup \makeatletter
\@ifundefined{theorem@style}{\input{theorem.sty}}{}
\ProvidesFile{thc.sty}
             [\FMithmInfo]
%    \end{macrocode}
%
% \begin{macro}{\th@change}
%    The \texttt{change} theorem style corresponds to the
%    \texttt{change break} style without a linebreak after the header.
%    To say it in another way, it's the same as the \texttt{plain}
%    style but with number and name interchanged and |\slshape| as the
%    default font.
%    \begin{macrocode}
\gdef\th@change{\normalfont\slshape
  \def\@begintheorem##1##2{\item
       [\hskip\labelsep \theorem@headerfont ##2\ ##1]}%
\def\@opargbegintheorem##1##2##3{%
  \item[\hskip\labelsep \theorem@headerfont ##2\ ##1\ (##3)]}}
\endgroup
%</thc>
%    \end{macrocode}
% \end{macro}
%
%
% \subsubsection{The \texttt{marginbreak} style}
%
% This style option is the one used most often at Mainz.  It is saved
% in the file |thmb.sty|.
%    \begin{macrocode}
%<*thmb>
\begingroup \makeatletter
\@ifundefined{theorem@style}{\input{theorem.sty}}{}
\ProvidesFile{thmb.sty}
             [\FMithmInfo]
%    \end{macrocode}
%
% \begin{macro}{\th@marginbreak}
%    The \texttt{margin break} style is nearly the same as the
%    \texttt{change break} style. The only difference is the placement
%    of the theorem number.  We use |\llap| to place it in the left
%    margin.
%
%    In this style |\labelsep| denotes the separation between the
%    number and the text.
%    \begin{macrocode}
\gdef\th@marginbreak{\normalfont\slshape
  \def\@begintheorem##1##2{\item
       [\rlap{\vbox{\theorem@headerfont
                    \hbox{\llap{##2}\hskip\labelsep ##1}%
                    \hbox{\strut}}}]}%
\def\@opargbegintheorem##1##2##3{%
  \item[\rlap{\vbox{\theorem@headerfont
                    \hbox{\llap{##2}\hskip\labelsep ##1\ (##3)}%
                    \hbox{\strut}}}]}}
\endgroup
%</thmb>
%    \end{macrocode}
% \changes{v2.1c}{90/02/17}{`endgroup outside macrocode env.}
% \end{macro}
%
%
%
% \subsubsection{The \texttt{margin} style}
%
% This style option is stored in the file |thm.sty|.
%    \begin{macrocode}
%<*thm>
\begingroup \makeatletter
\@ifundefined{theorem@style}{\input{theorem.sty}}{}
\ProvidesFile{thm.sty}
             [\FMithmInfo]
%    \end{macrocode}
%
% \begin{macro}{\th@margin}
%    Again this is only a variant of the theorem styles described
%    above without any new ideas.
%    \begin{macrocode}
\gdef\th@margin{\normalfont\slshape
  \def\@begintheorem##1##2{\item
       [\theorem@headerfont \llap{##2}\hskip\labelsep ##1]}%
\def\@opargbegintheorem##1##2##3{%
  \item[\theorem@headerfont \llap{##2}\hskip\labelsep ##1\ (##3)]}}
\endgroup
%</thm>
%    \end{macrocode}
% \end{macro}
%
%
%
%
% \subsection{Final Definitions}
%
% \begin{macro}{\theorempreskipamount}
% \begin{macro}{\theorempostskipamount}
%    The \textsf{skip} parameters that regulate the vertical empty space
%    before and after the {\envfont theorem} environment have
%    to be allocated as well.
% \changes{v2.0f}{89/07/20}{Unnecessary `global' removed.}
%    \begin{macrocode}
%<*package>
\newskip\theorempreskipamount
\newskip\theorempostskipamount
%    \end{macrocode}
%    Since we have used the same values for all theorem sets, we now can
%    assign them.
%    \begin{macrocode}
\global\setlength\theorempreskipamount{12pt plus 5pt minus 3pt}
\global\setlength\theorempostskipamount{8pt plus 3pt minus 1.5pt}
%    \end{macrocode}
% \end{macro}
% \end{macro}
%
%
% \begin{macro}{\@endtheorem}
%    The same holds for the macro |\@endtheorem|, which ends a
%    {\envfont theorem} environment. Since it is the same for all
%    theorem sets, it is removed from the macros |\th@|\meta{style}.
%    It simply ends the {\envfont trivlist} environment, which was
%    begun in |\@thm|.
%    \begin{macrocode}
\global\let\@endtheorem=\endtrivlist
%    \end{macrocode}
% \end{macro}
%
%
%
% \begin{macro}{\@preamblecmds}
%    All macros defined above are to be used only in the preamble.
%    Therefore, we insert them in |\@preamblecmds| which will disable
%    them at begin document. This is done by the internal \LaTeXe{}
%    command |\@onlypreamble|.
% \changes{v2.0g}{89/09/19}{`theoremheaderfont and `theorembodyfont
%    added.}
% \changes{v2.2a}{94/02/03}{Use \LaTeXe{} interface}
%    \begin{macrocode}
\@onlypreamble\@xnthm
\@onlypreamble\@ynthm
\@onlypreamble\@othm
\@onlypreamble\newtheorem
\@onlypreamble\theoremstyle
\@onlypreamble\theorembodyfont
\@onlypreamble\theoremheaderfont
%    \end{macrocode}
% \end{macro}
%
% Finally we declare the \texttt{plain} theorem style to be the default.
% \changes{v2.1b}{90/06/04}{Fixed missing `item by declaring default
%    style.}
%    \begin{macrocode}
\theoremstyle{plain}
%</package>
%    \end{macrocode}
%
% \Finale

\endinput
}{}
\ProvidesFile{thb.sty}
             [\FMithmInfo]
%    \end{macrocode}
%
% \begin{macro}{\th@break}
%    |\theoremstyle{break}| produces a line break after the name
%    of the theorem.  The font is |\slshape|.  Hence, we define
%    |\th@break| as follows:
%    \begin{macrocode}
\gdef\th@break{\normalfont\slshape
  \def\@begintheorem##1##2{\item[%
%    \end{macrocode}
%    We run into the following problem: it is not possible to create
%    the header with |\item[|\meta{title}|]| and then start
%    a new line by, for example, |\mbox{}\\|. Such a definition
%    will fail whenever a list environment follows immediately.  With
%    the above construction, the |\mbox{}| causes the switch
%    |@inlabel| (cf.\ definition of |\list| and
%    |\trivlist| in \cite{latex:source}) to be set to \texttt{false}
%    and so the following list will insert additional vertical
%    space (|\topskip|). This is quite annoying. Therefore, we
%    create the line break within the |\item|. In order to ensure
%    that the text will begin at the proper position in the following
%    line, we simply pretend that the label does not take any
%    room.\footnote{This will lead to problems whenever very high
%    symbols occurring in the line tower into the
%    heading. So, something else has to be done here sometime.}
%    \begin{macrocode}
       \rlap{\vbox{\hbox{\hskip \labelsep\theorem@headerfont ##1\ ##2}%
                   \hbox{\strut}}}]}%
%    \end{macrocode}
%    Again, the definition of |\@opargbegintheorem| is
%    completely analogous.
%    \begin{macrocode}
\def\@opargbegintheorem##1##2##3{%
  \item[\rlap{\vbox{\hbox{\hskip \labelsep \theorem@headerfont
                     ##1\ ##2\ (##3)}%
                    \hbox{\strut}}}]}}
\endgroup
%</thb>
%    \end{macrocode}
% \end{macro}
%
%
% \subsubsection{The \texttt{changebreak} style}
%
%    \begin{macrocode}
% This style option is stored in the file |thcb.sty|.
%    \begin{macrocode}
%<*thcb>
\begingroup \makeatletter
\@ifundefined{theorem@style}{% \iffalse meta-comment
%
% Copyright (C) 1993-2024
%
% The LaTeX Project and any individual authors listed elsewhere
% in this file.
%
% This file is part of the Standard LaTeX `Tools Bundle'.
% -------------------------------------------------------
%
% It may be distributed and/or modified under the
% conditions of the LaTeX Project Public License, either version 1.3c
% of this license or (at your option) any later version.
% The latest version of this license is in
%    https://www.latex-project.org/lppl.txt
% and version 1.3c or later is part of all distributions of LaTeX
% version 2005/12/01 or later.
%
% The list of all files belonging to the LaTeX `Tools Bundle' is
% given in the file `manifest.txt'.
%
% \fi
%
%%
%
%
\def\FMithmInfo{2023/07/05 v2.2c Theorem extension package (FMi)}
%
% \ProvidesFile{theorem.dtx}[\FMithmInfo]
%\iffalse   % this is a METACOMMENT !
%%
%% Package `theorem' to use with LaTeX2e
%% Copyright (C) 1989-1995 Frank Mittelbach, all rights reserved.
%<+package>\@ifundefined{theorem@style}{}{\endinput}
%<+package>\NeedsTeXFormat{LaTeX2e}[1995/06/01]
%<+package>\ProvidesPackage{theorem}
%<+package>         [\FMithmInfo]
%
% \fi
%
% \changes{v1.0}{87/05/07}{First draft.}
% \changes{v1.0a}{87/05/07}{`thistheoremstyle' added.}
% \changes{v1.1a}{87/07/29}{New concept: redefine the internal
%                          \protect\LaTeX{} commands `@xnthm' etc.}
% \changes{v1.1b}{87/07/31}{`theoremfont' used to specify font.}
% \changes{v1.2a}{88/01/08}{`@endtheorem' now globally defined.}
% \changes{v1.2b}{88/01/11}{`newtheorem' only allowed in the preamble.}
% \changes{v1.3a}{88/02/14}{`theoremfont' now a toks register.}
%
% \changes{v2.0a}{88/11/16}{New implementation and new documentation.}
% \changes{v2.0a}{88/11/16}{No layout changes but internal modifications
%                           old substyles won't work.}
% \changes{v2.0b}{89/01/30}{`global' used for all definitions.}
% \changes{v2.0c}{89/06/15}{Documentation translated to English.
%                            Thanks to LLL and Nina}
% \changes{v2.1a}{90/02/15}{Support for new font selection.}
% \changes{v2.2a}{94/02/03}{Cleanup for LaTeX2e}
%
% \DoNotIndex{\@endparenv,\@ifdefinable,\@ifnextchar,\@ifundefined}
% \DoNotIndex{\@namedef,\@nameuse,\@spaces,\@tempa,\@warning}
% \DoNotIndex{\ }
% \DoNotIndex{\begingroup,\bfseries}
% \DoNotIndex{\csname}
% \DoNotIndex{\def,\do,\docdate}
% \DoNotIndex{\endcsname,\endgroup,\endinput,\endtrivlist,\expandafter}
% \DoNotIndex{\filedate,\fileversion}
% \DoNotIndex{\gdef,\global}
% \DoNotIndex{\hbox,\hskip}
% \DoNotIndex{\ignorespaces,\input,\itshape,\item}
% \DoNotIndex{\labelsep,\let,\llap}
% \DoNotIndex{\makeatletter}
% \DoNotIndex{\noexpand}
% \DoNotIndex{\refstepcounter,\relax,\rlap}
% \DoNotIndex{\setlength,\slshape,\space,\string,\strut}
% \DoNotIndex{\the,\trivlist,\typeout}
% \DoNotIndex{\vbox}
% \DoNotIndex{\xdef}
%
% \def\deflabel#1{\ttfamily #1\hfill}
% \def\deflist#1{\begin{list}{}{\settowidth\labelwidth{\ttfamily #1}%
%                               \setlength\leftmargin\labelwidth
%                               \addtolength\leftmargin\labelsep
%                               \let\makelabel\deflabel}}
% \def\enddeflist{\end{list}}
% \def\envfont{\normalfont\ttfamily}
%
% \MakeShortVerb{\|}
%
% \newcommand{\TUB}{{\sl TUGboat\/}}
%
% \GetFileInfo{theorem.dtx}
% \title{An Extension of the
%        \LaTeX{} theorem environment\thanks{This file
%        has version number \fileversion, last
%        revised \filedate.}}
% \author{Frank Mittelbach}
%
% \MaintainedByLaTeXTeam{tools}
% \maketitle
%
% \begin{abstract}
%    The macros described in this paper yield an extension of the
%    \LaTeX{} theorem mechanism. It is designed to satisfy the
%    different requirements of various journals. Thus, the
%    layout of the ``theorems'' can be manipulated by determining a
%    ``style''. This article describes not only the use, but
%    also the definition, of the necessary macros.
% \end{abstract}
%
% \section*{Preface to version 2.2}
%
% For \LaTeXe{} this package did not need any fundamental changes. I
% only modified the messages generated so that theorem layout styles
% will show up with the |\listfiles| command and cleaned the section
% on the New Font Selection Scheme since this is now included in
% \LaTeX.
%
% \section*{Preface to version 2.1}
%
% This version is identical to 2.0g described in \TUB\ 10\#3 except
% for some internal defaults which are now set depending on the used
% font selection scheme.
%
% This was done to avoid unpleasant surprises if the new font selection
% scheme is in force. For further details see
% section~\ref{sec:newfonts} and \cite{bk:GMS94}.
%
% \section{Introduction}
%
% For our purposes here, ``theorems'' are labelled enunciations,
% often set off from the main text by extra space and a font change.
% Theorems, corollaries, conjectures, definitions, and remarks
% are all instances of ``theorems''.  The ``header'' of these structures
% is composed of a label (such as \textsc{Theorem} or \textsc{Remark})
% and a number which serializes an item in the sequence of items
% with the same label.
%
% Shortly after the introduction of \LaTeX{} at the Fachbereich
% Mathematik in Mainz, the desire to manipulate the layout of
% ``theorems'' arose.  In Mainz, the following two conventions came
% into general use:
% \begin{enumerate}
%   \item The number of the theorem is shown in the margin.
%   \item There is a line break at
%         the end of the theorem header.
% \end{enumerate}
%
% Additionally, some journals require different formats which depend on
% the ``sort of theorem'': e.g.\ often remarks and definitions are set
% in |\upshape|, while |\itshape| is employed for main theorems.
%
% Confronted with these requirements, a theorem environment was
% developed in Mainz which allows separate determination of the
% layout of the ``theorems sets'', comparable to |\pagestyle|.
%
%
% \section{The user interface}
% \subsection{Defining new theorem sets}
%
% \DescribeMacro\newtheorem
% As in the original \LaTeX{} version, the command |\newtheorem|
% defines a new ``theorem set'' or ``theorem-like structure''.
% Two required arguments name the new environment and give the text
% to be typeset with each instance of the new ``set'', while
% an optional argument determines how
% the ``set'' is enumerated:
% \begin{description}
%    \item[\ttfamily \bslash newtheorem\{foo\}\{bar\}]
%       The theorem set {\envfont foo} (whose name is \texttt{bar})
%       uses its own counter.
%    \item[\ttfamily \bslash newtheorem\{foo2\}{[foo]}\{bar2\}]
%       The theorem set {\envfont foo2} (printed name \texttt{bar2})
%       uses the same counter as the theorem set \texttt{foo}.
%    \item[\ttfamily \bslash newtheorem\{foo3\}\{bar3\}{[section]}]
%       The theorem set {\envfont foo3} (printed name \texttt{bar3}) is
%       enumerated within the counter \texttt{section}, i.e.\ with every
%       new |\section| the enumeration begins again with 1, and
%       the enumeration is composed from the section-number and the
%       theorem counter itself.
% \end{description}
%
% \DescribeMacro\theoremstyle
% Additionally, the command |\theoremstyle| can define the layout
% of various, or all, theorem sets.  It should be noted that any
% theorem set defined by |\newtheorem| is typeset in the
% |\theoremstyle| that is current at the time of the definition.
% Thus, the following
% \begin{verbatim}
%    \theoremstyle{break}        \newtheorem{Cor}{Corollary}
%    \theoremstyle{plain}        \newtheorem{Exa}{Example}[section]
%\end{verbatim}                             ^^A No blank at line start
% leads to the result that the set {\envfont Cor} is formatted in the
% style \texttt{break}, while the set {\envfont Exa} and all the
% following ones are formatted in the style \texttt{plain}, unless
% another |\theoremstyle| follows.  Since the definitions installed by
% |\newtheorem| are global, one also can limit |\theoremstyle| locally
% by grouping braces.
%
% \DescribeMacro\theorembodyfont
% The choice of the font for the theorem body is completely
% independent of the chosen |\theoremstyle|; this has proven to
% be very advantageous.  For example,
% \begin{verbatim}
%   {\theorembodyfont{\upshape}          \newtheorem{Rem}{Remark}}
%\end{verbatim}
% defines a theorem set {\envfont Rem}, which will be set in
% |\upshape| in the current layout (which in our example is
% \texttt{plain}).  As with |\theoremstyle|, the |\theorembodyfont|
% chosen is that current at the time of |\newtheorem|.  If
% |\theorembodyfont| is not specified or one defines
% |\theorembodyfont{}|, then the font used will be that defined by the
% |\theoremstyle|.
%
% \DescribeMacro\theoremheaderfont It is also possible to customize
% the font used for the theorem headers.  This is, however, a global
% declaration, and therefore there should be at most one
% |\theoremheaderfont| declaration in the preamble.\footnote{If it is
% actually necessary to have different header fonts, one has to define
% new theorem styles (substituting the desired font) or specify the
% information directly in the \texttt{\bslash newtheorem} declaration
% (the unclean variant).}
%
% \DescribeMacro\theorempreskipamount
% \DescribeMacro\theorempostskipamount Two additional parameters
% affect the vertical space around the theorem environments:
% |\theorempreskipamount| and |\theorempostskipamount| define,
% respectively, the spacing before and after such an environment.
% These parameters apply for all theorem sets and can be manipulated
% with the ordinary length macros.  They are rubber lengths
% (`\textsf{skips}'), and therefore can contain \texttt{plus} and
% \texttt{minus} parts.
%
% Since the definition of theorem sets should---most sensibly---be
% placed in the preamble, we only allow installation there.
% It is therefore possible to release the memory used here after
% |\begin{document}|, in order to make room for other
% applications.
%
%
% \subsection{Existing theorem styles}
%
% The following theorem styles exist to date:
% \begin{deflist}{marginbreak}
%    \item[plain]
%       This theorem style emulates the original \LaTeX{} definition,
%       except that additionally the parameters
%       |\theorem...skipamount| are used.
%    \item[break]
%       In this style, the theorem header is followed by a line
%       break.
%    \item[marginbreak]
%       The theorem number is set in the margin, and there is a line
%       break as in \texttt{break}.
%    \item[changebreak]
%       Like \texttt{break}, but with header number and text
%       interchanged.
%    \item[change]
%       Header number and text are interchanged, without a line break.
%    \item[margin]
%       The number is set in the left margin, without a line break.
% \end{deflist}
% All styles (except \texttt{plain}) select |\slshape| as the default
% |\theorembodyfont|.
%
%
% \subsection{Examples}
%
% Given the above theorem sets {\envfont Cor}, {\envfont Exa}
% and {\envfont Rem},
% suppose that the preamble also contains the declarations:
% \begin{verbatim}
%    \theoremstyle{marginbreak}   \newtheorem{Lem}[Cor]{Lemma}
%    \theoremstyle{change}
%    \theorembodyfont{\itshape}        \newtheorem{Def}[Cor]{Definition}
%
%    \theoremheaderfont{\scshape}
%\end{verbatim}
% Then the following are some typical examples of the typeset output
% resulting from their use.
%
% \begin{Cor}
%    This is a sentence typeset in the theorem environment {\envfont
%    Cor}.
% \end{Cor}
%
% \begin{Exa}
%    This is a sentence typeset in the theorem environment {\envfont
%    Exa}.
% \end{Exa}
%
% \begin{Rem}
%    This is a sentence typeset in the theorem environment {\envfont
%    Rem}.
% \end{Rem}
%
% \begin{Lem}[Ben User]
%    This is a sentence typeset in the theorem environment {\envfont
%    Lem}.
% \end{Lem}
%
% \begin{Def}[Very impressive Definition]
%    This is a sentence typeset in the theorem environment {\envfont
%    Def}.
% \end{Def}
%
% The last two examples show the effect of the optional argument to
% a theorem environment (it is the text typeset in parentheses).
%
%
% \section{Special Considerations}
% \label{sec:newfonts}
%
% Theoremheader and body are implemented as a unit. This means that
% the |\theoremheaderfont|  will inherit characteristics of the
% |\theorembodyfont| in \LaTeXe. Thus, if for example
% |\theorembodyfont| is |\itshape| and |\theoremheaderfont|
% is |\bfseries| the font selected for the header will have the
% characteristics `bold extended italic'. If this is not desired one
% should set the |\theoremheaderfont| to something like
% \begin{verbatim}
%      \theoremheaderfont{\normalfont\bfseries}
%\end{verbatim}
% i.e.\ supplying all necessary font information explicitly.
%
%
%
% \section{Acknowledgements}
%
% The publication of this set of macros was only possible with the
% help of Christina Busse (translating the manuscript into English),
% Joachim Pense (playing the r\^ole of typist), Chris Rowley
% (looking everything over) and many others providing useful
% suggestions.
%
% \MaybeStop{%
%    \begin{thebibliography}{9}
%       \bibitem{bk:GMS94} \textsc{M.~Goossens}, \textsc{F.~Mittelbach}
%         and \textsc{A.~Samarin}.
%         \newblock The \LaTeX{} Companion.
%         \newblock
%         Addison-Wesley, Reading, Massachusetts, 1994.
%       \bibitem{latex:source} \textsc{Lamport, Leslie.} \newblock
%                               \texttt{latex.tex}, version 2.09, date
%                               Feb. 1990.
%    \end{thebibliography}
%    \PrintIndex
%    \PrintChanges}
%
%
% \section{The documentation driver file}
%
% The next bit of code contains the documentation driver file for
% \TeX{}, i.e., the file that will produce the documentation you are
% currently reading. It will be extracted from this file by the
% \texttt{docstrip} program. Since it is the first code in the file
% one can alternatively process this file directly with \LaTeXe{} to
% obtain the documentation.
%
%    \begin{macrocode}
%<*driver>
\documentclass{ltxdoc}

\usepackage{theorem}

%<+driver>% The next few lines define theorem sets which are used
%<+driver>% in the example section of the documentation.

\theoremstyle{break}       \newtheorem{Cor}{Corollary}
\theoremstyle{plain}       \newtheorem{Exa}{Example}[section]
{\theorembodyfont{\upshape}\newtheorem{Rem}{Remark}}
\theoremstyle{marginbreak} \newtheorem{Lem}[Cor]{Lemma}
\theoremstyle{change}
\theorembodyfont{\itshape} \newtheorem{Def}[Cor]{Definition}

\theoremheaderfont{\scshape}

\RecordChanges

\begin{document}
   \DocInput{theorem.dtx}
\end{document}
%</driver>
%    \end{macrocode}
%
% \section{Definition of the Macros}
%
% If the file has been loaded before, we abort immediately. If not the
% package announces itself (this is actually done at the very top of
% the file---the way it is done isn't good style so don't copy it).
% \changes{v2.0e}{89/07/19}{Spaces removed from `typeout'.}
% \changes{v2.2b}{95/11/19}{Announce moved to top of file}
%    \begin{macrocode}
%<*package>
%\@ifundefined{theorem@style}{}{\endinput}
%\def\FMithmInfo{1995/11/19 v2.2b Theorem extension package (FMi)}
%\ProvidesPackage{theorem}[\FMithmInfo]
%    \end{macrocode}
%
%
% \subsection{Definition of theorem styles and fonts}
%
% All the definitions in this file are done globally to allow
% inputting this file inside a group.
%
% \begin{macro}{\theoremstyle}
%    Before a theorem style can be installed, the chosen style must be
%    known. For that reason, we must test to see that
%    |\th@|\meta{style} is known or, more precisely, that it is
%    different from |\relax|. If the style is not known then
%    |\th@plain| is used.
%    \begin{macrocode}
\gdef\theoremstyle#1{%
   \@ifundefined{th@#1}{\@warning
          {Unknown theoremstyle `#1'. Using `plain'}%
          \theorem@style{plain}}%
%    \end{macrocode}
%    We save the theorem style to be used in the token register
%    |\theorem@style|.
%    \begin{macrocode}
      {\theorem@style{#1}}%
%    \end{macrocode}
%    Now we ``evaluate'' the theorem style: this means, we call the
%    macro |\th@|\meta{style} which will activate the relevant
%    definitions which are contained in a separate file.  This is done
%    in a group to suppress changes to the current font.  This could
%    otherwise pose problems together with the new font selection
%    scheme\footnote{When I printed the original article using the new
%    font selection scheme I ended with a document with slanted
%    typefaces (text headings and all) simply because one of the
%    theorem styles used \texttt{\bslash sl} at toplevel.} if the
%    |\th@|\meta{style} is evaluated a second time.
% \changes{v2.1a}{90/02/15}{Group added around `th@..}
% \changes{v2.1a}{90/02/15}{Reseting of `..@begintheorem removed.}
%    \begin{macrocode}
      \begingroup
        \csname th@\the\theorem@style \endcsname
      \endgroup}
%    \end{macrocode}
% \end{macro}
%
%
% \begin{macro}{\@begintheorem}
% \changes{v2.1a}{90/02/15}{Relaxed at toplevel.}
% \begin{macro}{\@opargbegintheorem}
% \changes{v2.1a}{90/02/15}{Relaxed at toplevel.}
%    We reset |\@begintheorem| and |\@opargbegintheorem| to
%    |\relax| since these commands are no longer necessary at
%    toplevel. This will save a few tokens.
%    \begin{macrocode}
\global\let\@begintheorem\relax
\global\let\@opargbegintheorem\relax
%    \end{macrocode}
% \end{macro}
% \end{macro}
%
% \begin{macro}{\theorem@style}
%    Obviously the \textsf{token} register used above has to be
%    allocated. To assure the utmost compatibility with the
%    original \LaTeX{} definition, we set the default theorem style to
%    \texttt{plain}, which implements the usual \LaTeX{} convention.
% \changes{v2.0f}{89/07/20}{Unnecessary `global' removed.}
%    \begin{macrocode}
\newtoks\theorem@style
\global\theorem@style{plain}
%    \end{macrocode}
% \end{macro}
%
% \begin{macro}{\theorembodyfont}
% \begin{macro}{\theorem@bodyfont}
%    If the |\theorembodyfont| is set by the user then it should not
%    interact with the default font set in the theorem style. When the
%    new font selection is in force this may happen if, for example,
%    the default is |\itshape| and the new |\theorembodyfont| is
%    |\sffamily|. So we add a |\reset@font| command in front of the
%    user definition.
% \changes{v2.0f}{89/07/20}{Unnecessary `global' removed.}
% \changes{v2.0g}{89/09/19}{`theoremfont to `theorembodyfont.}
% \changes{v2.1c}{92/06/22}{`reset@font added and logic changed.}
% \changes{v2.1d}{92/10/28}{Reset to empty if arg is empty}
%    \begin{macrocode}
\gdef\theorembodyfont#1{%
%    \end{macrocode}
%    We check if the argument supplied is empty and if so put nothing
%    into the |\theorem@bodyfont| token register to allow for
%    |\theorembodyfont{}| as a mean of using the default of the
%    current |\theoremstyle|.
%    \begin{macrocode}
   \def\@tempa{#1}%
   \ifx\@tempa\@empty
    \theorem@bodyfont{}%
   \else
    \theorem@bodyfont{\reset@font#1}%
   \fi
   }
\newtoks\theorem@bodyfont
\global\theorem@bodyfont{}
%    \end{macrocode}
% \end{macro}
% \end{macro}
%
%
% \begin{macro}{\theoremheaderfont}
% \changes{v2.0g}{89/09/19}{Macro added.}
%    The font for the theorem headers is handled differently because
%    this definition applies to all theorem styles.
%    \begin{macrocode}
\gdef\theoremheaderfont#1{\gdef\theorem@headerfont{#1}%
%    \end{macrocode}
%    After using the macro once it is redefined to produce an error
%    message.
%    \begin{macrocode}
       \gdef\theoremheaderfont##1{%
        \typeout{\string\theoremheaderfont\space should be used
                 only once.}}}
%    \end{macrocode}
% \end{macro}
%
% \begin{macro}{\theorem@headerfont}
% \changes{v2.0g}{89/09/19}{Macro added.}
%    To set the |\theorem@headerfont| default we first test if
%    the new fontselection scheme is in force.
% \changes{v2.1a}{90/02/15}{Support for new font selection.}
%    \begin{macrocode}
\ifx\upshape\undefined
%    \end{macrocode}
%    If not we define it to expand into |\bfseries|. We don't use
%    |\let| just in case a following style option redefines this
%    macro.
%    \begin{macrocode}
\gdef\theorem@headerfont{\bfseries}
%    \end{macrocode}
%    Otherwise we reset the current shape before calling |\bfseries|.
%    \begin{macrocode}
\else \gdef\theorem@headerfont{\normalfont\bfseries}\fi
%    \end{macrocode}
% \end{macro}
%
% \begin{macro}{\th@plain}
% \begin{macro}{\th@break}
% \begin{macro}{\th@marginbreak}
% \begin{macro}{\th@changebreak}
% \begin{macro}{\th@change}
% \begin{macro}{\th@margin}
%    The different styles are defined in macros such as
%    |\th@plain|.  Since memory space is precious in
%    ``non-Big-versions'', we have to avoid offering too many unused
%    definitions.  Therefore we define these styles in separate files
%    that can be loaded on demand.  Thus the commands themselves only
%    load these files. We use |\@input@|, a \LaTeXe{} internal command
%    that ensures that the file will be listed with |\listfiles|.
% \changes{v2.2a}{94/02/02}{Use `@input@ to load theorem layout files}
%    \begin{macrocode}
\gdef\th@plain{\@input@{thp.sty}}
\gdef\th@break{\@input@{thb.sty}}
\gdef\th@marginbreak{\@input@{thmb.sty}}
\gdef\th@changebreak{\@input@{thcb.sty}}
\gdef\th@change{\@input@{thc.sty}}
\gdef\th@margin{\@input@{thm.sty}}
%    \end{macrocode}
%    This list will be expanded when new styles become
%    available.  For testing,  just append new theorem substyles
%    as document options.
% \end{macro}
% \end{macro}
% \end{macro}
% \end{macro}
% \end{macro}
% \end{macro}
%
%
%
% \subsection{Definition of a new theorem set}
%
% As already pointed out, a new theorem environment can be defined in
% three different ways:
% \begin{verbatim}
%    \newtheorem{Lem}{Lemma}
%    \newtheorem{Lem}{Lemma}[section]
%    \newtheorem{Lem}[Theorem]{Lemma}
%\end{verbatim}
% The function of the macro |\newtheorem| is to recognize these cases
% and then to branch into one of the three macros |\@ynthm|, |\@xnthm|
% or |\@othm|.  This mechanism is adopted unchanged from
% \cite{latex:source}; the essential point here is that, for example,
% in the second case, the arguments \texttt{Lem}, \texttt{Lemma} and
% \texttt{section} are passed over to the macro |\@xnthm|.
%
% We inspect this case first because the others present fewer problems,
% and thus are easily derived from this one.
%
% \begin{macro}{\@xnthm}
%    For our example arguments, the macro |\@xnthm| must fulfill
%    the following:
%    \begin{itemize}
%       \item
%          Define a new \LaTeX-counter `Lem'
%       \item
%          reset this counter within a |\section|
%       \item
%          define the macro |\theLem|
%       \item
%          define the environment macros |\Lem| and
%          |\endLem| using the current
%          |\theoremstyle| and |\theorem@bodyfont|.
%    \end{itemize}
%    Obviously, all this should happen only if the first argument
%    of |\@xnthm| (i.e.\ \texttt{Lem} in our example) is chosen so
%    as not to conflict with any previously defined commands or
%    environments.  This test is performed by the \LaTeX{} macro
%    |\@ifdefinable|.
%    \begin{macrocode}
\gdef\@xnthm#1#2[#3]{\expandafter\@ifdefinable\csname #1\endcsname
%    \end{macrocode}
%    Therefore, the first argument of |\@ifdefinable| is the
%    expansion (in the example, |\Lem|) of
%    |\csname||#1\endcsname|.  The second argument is
%    executed only if the test has been completed successfully.
%    \begin{macrocode}
   {%
%    \end{macrocode}
%    Now we define the new counter. The names of the \LaTeX{}
%    macros employed should speak for themselves:
% \changes{v2.2b}{1995/10/07}{Use \cs{@newctr}}
% \changes{v2.2c}{1995/11/23}{Use \cs{@newctr} properly.}
%    \begin{macrocode}
    \@definecounter{#1}\@newctr{#1}[#3]%
%    \end{macrocode}
%    Using |\@newctr| will give a proper error message if the counter
%    in |#3| is not defined.
%    In defining `|\theLem|' we must generate the desired macro
%    name by use of |\expandafter| and |\csname|.
%    \begin{macrocode}
    \expandafter\xdef\csname the#1\endcsname
%    \end{macrocode}
%    An |\xdef| is used in order to make the definition global,
%    and to ensure that it contains the replacement texts of
%    |\@thmcountersep| and |\@thmcounter|.\footnote{These
%    two macros can be defined by the document style.  Their default
%    values produce a `\texttt{.}' as separation and an arabic
%    representation of the number.} However, not everything should be
%    expanded. For example, it saves space to use |\thesection|
%    instead of its---at times---lengthy expansion.
%    \begin{macrocode}
      {\expandafter \noexpand \csname the#3\endcsname
       \@thmcountersep \@thmcounter{#1}}%
%    \end{macrocode}
%    Thus with the defaults of \LaTeX{}, |\theLem| would be
%    replaced by the command sequence |\thesection.\arabic{Lem}|.
%
%    We will now look at the definition of the macro which is executed
%    at the beginning of the actual environment (in our example this
%    macro is |\Lem|). It should be noted that we use an
%    ``|\expandafter| trick'' to expand only certain parts of the
%    replacement text at the time of the definition.
%    \begin{macrocode}
    \def\@tempa{\global\@namedef{#1}}%
    \expandafter \@tempa \expandafter{%
%    \end{macrocode}
%    First, the macro that contains the current definitions of
%    |\@begintheorem| and |\@opargtheorem| should be called
%    up.  The name of this macro---as is already known---has the form
%    |\th@|\meta{theorem\ style}; therefore, it must be called by
%    \begin{macrocode}
      \csname th@\the \theorem@style
%    \end{macrocode}
%    In addition the default theorem font should be changeable, i.e.\
%    we have to insert the contents of |\theorem@bodyfont|. For
%    that reason, we expand even further, beyond |\endcsname|,
%    and thus insert the contents of the \textsf{token} register
%    |\theorem@bodyfont| in the replacement text.
%    \begin{macrocode}
            \expandafter \endcsname \the \theorem@bodyfont
%    \end{macrocode}
%    Now it is time to call the macro |\@thm| which takes over
%    the further processing.  It has two arguments: the current
%    counter name (in our example, \texttt{Lem}), and the text of the
%    label (in our example, \texttt{Lemma}).
%    \begin{macrocode}
     \@thm{#1}{#2}}%
%    \end{macrocode}
%    With this, the `sub-definition' is complete.  The macro
%    |\@endtheorem| ends a theorem environment and is, so far, nothing
%    but an |\endtrivlist|.  (Hence it is defined globally, and not
%    within the theorem styles.\footnote{This has to be changed as
%    soon as theorem styles that change \texttt{\bslash @endtheorem}
%    exist.  In such a case, all existing styles must be changed as
%    well since they will have to reset the macro.}) Therefore, we can
%    set it equivalent to the macro that ends the theorem set (in our
%    example, |\endLem|).  However, if some day theorem styles exist
%    that do change |\@endtheorem|, we would have to use the
%    commented-out line instead.
%    \begin{macrocode}
    \global \expandafter \let \csname end#1\endcsname \@endtheorem
%   \global\@namedef{end#1}{\@endtheorem}%
%    \end{macrocode}
%    With these commands all the required definitions are employed,
%    unless the test |\@ifdefinable| has failed.  Therefore, we
%    end the second argument of this macro and with it the definition
%    of |\@xnthm|.
%    \begin{macrocode}
   }}
%    \end{macrocode}
% \end{macro}
%
% \begin{macro}{\@ynthm}
%    The definition of |\@ynthm| is completely analogous. In
%    this case the new counter that is defined is not reset within
%    another counter; thus the definition of |\the...|
%    is simplified:
%    \begin{macrocode}
\gdef\@ynthm#1#2{\expandafter\@ifdefinable\csname #1\endcsname
   {\@definecounter{#1}%
    \expandafter\xdef\csname the#1\endcsname{\@thmcounter{#1}}%
%    \end{macrocode}
%    The rest of the definition corresponds literally to that of
%    |\@xnthm|:
%    \begin{macrocode}
    \def\@tempa{\global\@namedef{#1}}\expandafter \@tempa
     \expandafter{\csname th@\the \theorem@style \expandafter
     \endcsname \the\theorem@bodyfont \@thm{#1}{#2}}%
    \global \expandafter \let \csname end#1\endcsname \@endtheorem}}
%    \end{macrocode}
% \end{macro}
%
% \begin{macro}{\@othm}
%    The definition of |\@othm| does not contain anything new.
%    \begin{macrocode}
\gdef\@othm#1[#2]#3{%
%    \end{macrocode}
% \changes{v2.2b}{95/10/07}{Build in check for counter defined}
%    We do not define a new counter but instead use one that has
%    already been defined.  Thus the only definition we need is that
%    of this pseudo-counter (i.e.\ |\the|\meta{env.\ name}).
%    First we check if |#2| corresponds to a known counter name.
%    \begin{macrocode}
  \expandafter\ifx\csname c@#2\endcsname\relax
   \@nocounterr{#2}%
  \else
   \expandafter\@ifdefinable\csname #1\endcsname
   {\expandafter \xdef \csname the#1\endcsname
     {\expandafter \noexpand \csname the#2\endcsname}%
%    \end{macrocode}
%    All other parts of the definition can be adopted from
%    |\@xnthm|. We have to remember, though, that in this case
%    the name of the current counter and the theorem label have moved
%    to the second and third arguments.
%    \begin{macrocode}
    \def\@tempa{\global\@namedef{#1}}\expandafter \@tempa
     \expandafter{\csname th@\the \theorem@style \expandafter
     \endcsname \the\theorem@bodyfont \@thm{#2}{#3}}%
    \global \expandafter \let \csname end#1\endcsname \@endtheorem}%
  \fi}
%    \end{macrocode}
% \end{macro}
%
% \subsection{Macros that are employed in a {\envfont theorem}
%    environment}
%
% \begin{macro}{\@thm}
%    The macro |\@thm| has to increase the current counter.  Then,
%    depending on whether the environment has (or does not have) an
%    optional argument, it has to branch into either |\@begintheorem|
%    or |\@opargtheorem|.
%    \begin{macrocode}
\gdef\@thm#1#2{\refstepcounter{#1}%
%    \end{macrocode}
%    Now we start a {\envfont trivlist} environment, and give
%    |\@topsep| and |\@topsepadd| the values of the \textsf{skip}
%    registers |\theorempreskipamount| and |\theorempostskipamount|.
%    The value in |\@topsep| is the vertical space that is inserted by
%    the first (and only) |\item| in our |\trivlist| whilst
%    |\@topsepadd| is inserted by |\@endparenv| at the end of that
%    {\envfont trivlist} environment.  By using these registers, we
%    obtain the desired space around a {\envfont theorem} environment.
%    \begin{macrocode}
   \trivlist
   \@topsep \theorempreskipamount               % used by first \item
   \@topsepadd \theorempostskipamount           % used by \@endparenv
%    \end{macrocode}
%    Now we have to test whether an optional argument has been given.
% \changes{v2.0d}{89/07/18}{Forgotten percent added.}
%    \begin{macrocode}
   \@ifnextchar [%
%    \end{macrocode}
%    If there is an optional argument, we will call |\@ythm|, and
%    move the arguments read back into the input stream.
%    \begin{macrocode}
   {\@ythm{#1}{#2}}%
%    \end{macrocode}
%    If not, we call |\@begintheorem|. Its first argument is the
%    name of the theorem set (hence the second argument of
%    |\@thm|). Its second argument is the macro that produces the
%    current number.
%    \begin{macrocode}
   {\@begintheorem{#2}{\csname the#1\endcsname}\ignorespaces}}
%    \end{macrocode}
% \end{macro}
%
% \begin{macro}{\@xthm}
% \begin{macro}{\@ythm}
%    Both these macros were originally called by |\@thm|.  We do
%    not need |\@xthm| anymore, hence we reset it to
%    |\relax|. The definition of |\@ythm| has not changed at
%    all from its definition in \LaTeX. In order to make the macros
%    easier to understand, we will
%    nevertheless present it (commented out).
%    \begin{macrocode}
\global\let\@xthm\relax
% \def\@ythm#1#2[#3]{\@opargbegintheorem{#2}{\csname
%           the#1\endcsname}{#3}\ignorespaces}
%</package>
%    \end{macrocode}
%    The primitive |\ignorespaces| in |\@ythm| and
%    |\@thm| is needed to remove the spaces between the
%    |\begin{...}| and the actual text.
% \end{macro}
% \end{macro}
%
%
% \subsection{Definition of the theorem substyles}
%
% As already pointed out, the theorem substyles, defined below, are
% only loaded when necessary.  Note that all these substyles, except
% \texttt{plain}, have |\slshape| as the default body font.
%
% \subsubsection{The \texttt{plain} style}
%
% \label{dok:th@plain} As the following macros use |@|, we have
% to locally set the |\catcode| of this symbol to ``letter''.
% This happens within a group, so that we do not have to worry about
% which |\catcode| that symbol had before.
%    \begin{macrocode}
%<*thp>
\begingroup \makeatletter
%    \end{macrocode}
% Since we are now within a group, we must make all definitions
% globally.  First we make sure that \texttt{theorem.sty} is loaded.
% This will allow us to use this file as a document style option
% without having to call \texttt{theorem} itself as an option.  At the
% same time, we assure that at least version 2 is loaded, since
% |\theorem@style| was not defined in earlier versions.
%    \begin{macrocode}
\@ifundefined{theorem@style}{\input{theorem.sty}}{}
\ProvidesFile{thp.sty}
             [\FMithmInfo]
%    \end{macrocode}
%
%
% \begin{macro}{\th@plain}
% \begin{macro}{\@begintheorem}
% \begin{macro}{\@opargbegintheorem}
%    |\theoremstyle{plain}| corresponds to the original
%    definition, except that the distances to the surrounding text are
%    determined by the parameters |\theorempreskipamount| and
%    |\theorempostskipamount|.
%    First we set the default body font.
%    \begin{macrocode}
\gdef\th@plain{\normalfont\itshape
%    \end{macrocode}
%    Then we define |\@begintheorem| and
%    |\@opargbegintheorem|. These two macros define how the
%    header of a theorem is typeset.  |\@opargbegintheorem| will
%    be called if a \textsf{theorem} environment with an optional
%    argument is encountered; otherwise, the header is constructed by
%    calling |\@begintheorem|.  If one of these macros is
%    executed, we are within a \textsf{trivlist} environment started by
%    |\@thm|.  So the theorem header is produced with an
%    |\item| command.
%
%    Instead of specifying the header font directly, all standard
%    theorem styles use the |\theorem@headerfont| macro to allow
%    customization.  The extra space (|\labelsep|) is necessary
%    because of problems in the \textsf{trivlist} environment.
%    \begin{macrocode}
  \def\@begintheorem##1##2{%
        \item[\hskip\labelsep \theorem@headerfont ##1\ ##2]}%
%    \end{macrocode}
%    The definition of |\@opargbegintheorem| is completely
%    analogous. The only difference is the fact that there exists a
%    third argument (which is the optional parameter of the
%    environment and contains additional information about the
%    theorem).  Customarily we enclose it in parentheses.
%    \begin{macrocode}
\def\@opargbegintheorem##1##2##3{%
   \item[\hskip\labelsep \theorem@headerfont ##1\ ##2\ (##3)]}}
%    \end{macrocode}
% \end{macro}
% \end{macro}
% \end{macro}
%
% \noindent
% We conclude with an |\endgroup| to restore the
% |\catcode| of |@|.
%    \begin{macrocode}
\endgroup
%</thp>
%    \end{macrocode}
%
%
% \subsubsection{The \texttt{break} style}
%
% This style option is stored in the file |thb.sty|.
% For the next two lines see the documentation for |\th@plain| on
% page~\pageref{dok:th@plain}.
%    \begin{macrocode}
%<*thb>
\begingroup \makeatletter
\@ifundefined{theorem@style}{\input{theorem.sty}}{}
\ProvidesFile{thb.sty}
             [\FMithmInfo]
%    \end{macrocode}
%
% \begin{macro}{\th@break}
%    |\theoremstyle{break}| produces a line break after the name
%    of the theorem.  The font is |\slshape|.  Hence, we define
%    |\th@break| as follows:
%    \begin{macrocode}
\gdef\th@break{\normalfont\slshape
  \def\@begintheorem##1##2{\item[%
%    \end{macrocode}
%    We run into the following problem: it is not possible to create
%    the header with |\item[|\meta{title}|]| and then start
%    a new line by, for example, |\mbox{}\\|. Such a definition
%    will fail whenever a list environment follows immediately.  With
%    the above construction, the |\mbox{}| causes the switch
%    |@inlabel| (cf.\ definition of |\list| and
%    |\trivlist| in \cite{latex:source}) to be set to \texttt{false}
%    and so the following list will insert additional vertical
%    space (|\topskip|). This is quite annoying. Therefore, we
%    create the line break within the |\item|. In order to ensure
%    that the text will begin at the proper position in the following
%    line, we simply pretend that the label does not take any
%    room.\footnote{This will lead to problems whenever very high
%    symbols occurring in the line tower into the
%    heading. So, something else has to be done here sometime.}
%    \begin{macrocode}
       \rlap{\vbox{\hbox{\hskip \labelsep\theorem@headerfont ##1\ ##2}%
                   \hbox{\strut}}}]}%
%    \end{macrocode}
%    Again, the definition of |\@opargbegintheorem| is
%    completely analogous.
%    \begin{macrocode}
\def\@opargbegintheorem##1##2##3{%
  \item[\rlap{\vbox{\hbox{\hskip \labelsep \theorem@headerfont
                     ##1\ ##2\ (##3)}%
                    \hbox{\strut}}}]}}
\endgroup
%</thb>
%    \end{macrocode}
% \end{macro}
%
%
% \subsubsection{The \texttt{changebreak} style}
%
%    \begin{macrocode}
% This style option is stored in the file |thcb.sty|.
%    \begin{macrocode}
%<*thcb>
\begingroup \makeatletter
\@ifundefined{theorem@style}{\input{theorem.sty}}{}
\ProvidesFile{thcb.sty}
             [\FMithmInfo]
%    \end{macrocode}
%
% \begin{macro}{\th@changebreak}
%    The \texttt{change-break} theorem style is like \texttt{break}
%    but with interchange of theorem name and theorem number.  Thus we
%    define |\th@changebreak| as follows:
%    \begin{macrocode}
\gdef\th@changebreak{\normalfont\slshape
  \def\@begintheorem##1##2{\item
       [\rlap{\vbox{\hbox{\hskip\labelsep \theorem@headerfont ##2\ ##1}%
                    \hbox{\strut}}}]}%
\def\@opargbegintheorem##1##2##3{%
  \item[\rlap{\vbox{\hbox{\hskip\labelsep \theorem@headerfont
                     ##2\ ##1\ (##3)}%
                    \hbox{\strut}}}]}}
\endgroup
%</thcb>
%    \end{macrocode}
% \end{macro}
%
%
% \subsubsection{The \texttt{change} style}
%
% This style option is stored in the file |thc.sty|.
%    \begin{macrocode}
%<*thc>
\begingroup \makeatletter
\@ifundefined{theorem@style}{\input{theorem.sty}}{}
\ProvidesFile{thc.sty}
             [\FMithmInfo]
%    \end{macrocode}
%
% \begin{macro}{\th@change}
%    The \texttt{change} theorem style corresponds to the
%    \texttt{change break} style without a linebreak after the header.
%    To say it in another way, it's the same as the \texttt{plain}
%    style but with number and name interchanged and |\slshape| as the
%    default font.
%    \begin{macrocode}
\gdef\th@change{\normalfont\slshape
  \def\@begintheorem##1##2{\item
       [\hskip\labelsep \theorem@headerfont ##2\ ##1]}%
\def\@opargbegintheorem##1##2##3{%
  \item[\hskip\labelsep \theorem@headerfont ##2\ ##1\ (##3)]}}
\endgroup
%</thc>
%    \end{macrocode}
% \end{macro}
%
%
% \subsubsection{The \texttt{marginbreak} style}
%
% This style option is the one used most often at Mainz.  It is saved
% in the file |thmb.sty|.
%    \begin{macrocode}
%<*thmb>
\begingroup \makeatletter
\@ifundefined{theorem@style}{\input{theorem.sty}}{}
\ProvidesFile{thmb.sty}
             [\FMithmInfo]
%    \end{macrocode}
%
% \begin{macro}{\th@marginbreak}
%    The \texttt{margin break} style is nearly the same as the
%    \texttt{change break} style. The only difference is the placement
%    of the theorem number.  We use |\llap| to place it in the left
%    margin.
%
%    In this style |\labelsep| denotes the separation between the
%    number and the text.
%    \begin{macrocode}
\gdef\th@marginbreak{\normalfont\slshape
  \def\@begintheorem##1##2{\item
       [\rlap{\vbox{\theorem@headerfont
                    \hbox{\llap{##2}\hskip\labelsep ##1}%
                    \hbox{\strut}}}]}%
\def\@opargbegintheorem##1##2##3{%
  \item[\rlap{\vbox{\theorem@headerfont
                    \hbox{\llap{##2}\hskip\labelsep ##1\ (##3)}%
                    \hbox{\strut}}}]}}
\endgroup
%</thmb>
%    \end{macrocode}
% \changes{v2.1c}{90/02/17}{`endgroup outside macrocode env.}
% \end{macro}
%
%
%
% \subsubsection{The \texttt{margin} style}
%
% This style option is stored in the file |thm.sty|.
%    \begin{macrocode}
%<*thm>
\begingroup \makeatletter
\@ifundefined{theorem@style}{\input{theorem.sty}}{}
\ProvidesFile{thm.sty}
             [\FMithmInfo]
%    \end{macrocode}
%
% \begin{macro}{\th@margin}
%    Again this is only a variant of the theorem styles described
%    above without any new ideas.
%    \begin{macrocode}
\gdef\th@margin{\normalfont\slshape
  \def\@begintheorem##1##2{\item
       [\theorem@headerfont \llap{##2}\hskip\labelsep ##1]}%
\def\@opargbegintheorem##1##2##3{%
  \item[\theorem@headerfont \llap{##2}\hskip\labelsep ##1\ (##3)]}}
\endgroup
%</thm>
%    \end{macrocode}
% \end{macro}
%
%
%
%
% \subsection{Final Definitions}
%
% \begin{macro}{\theorempreskipamount}
% \begin{macro}{\theorempostskipamount}
%    The \textsf{skip} parameters that regulate the vertical empty space
%    before and after the {\envfont theorem} environment have
%    to be allocated as well.
% \changes{v2.0f}{89/07/20}{Unnecessary `global' removed.}
%    \begin{macrocode}
%<*package>
\newskip\theorempreskipamount
\newskip\theorempostskipamount
%    \end{macrocode}
%    Since we have used the same values for all theorem sets, we now can
%    assign them.
%    \begin{macrocode}
\global\setlength\theorempreskipamount{12pt plus 5pt minus 3pt}
\global\setlength\theorempostskipamount{8pt plus 3pt minus 1.5pt}
%    \end{macrocode}
% \end{macro}
% \end{macro}
%
%
% \begin{macro}{\@endtheorem}
%    The same holds for the macro |\@endtheorem|, which ends a
%    {\envfont theorem} environment. Since it is the same for all
%    theorem sets, it is removed from the macros |\th@|\meta{style}.
%    It simply ends the {\envfont trivlist} environment, which was
%    begun in |\@thm|.
%    \begin{macrocode}
\global\let\@endtheorem=\endtrivlist
%    \end{macrocode}
% \end{macro}
%
%
%
% \begin{macro}{\@preamblecmds}
%    All macros defined above are to be used only in the preamble.
%    Therefore, we insert them in |\@preamblecmds| which will disable
%    them at begin document. This is done by the internal \LaTeXe{}
%    command |\@onlypreamble|.
% \changes{v2.0g}{89/09/19}{`theoremheaderfont and `theorembodyfont
%    added.}
% \changes{v2.2a}{94/02/03}{Use \LaTeXe{} interface}
%    \begin{macrocode}
\@onlypreamble\@xnthm
\@onlypreamble\@ynthm
\@onlypreamble\@othm
\@onlypreamble\newtheorem
\@onlypreamble\theoremstyle
\@onlypreamble\theorembodyfont
\@onlypreamble\theoremheaderfont
%    \end{macrocode}
% \end{macro}
%
% Finally we declare the \texttt{plain} theorem style to be the default.
% \changes{v2.1b}{90/06/04}{Fixed missing `item by declaring default
%    style.}
%    \begin{macrocode}
\theoremstyle{plain}
%</package>
%    \end{macrocode}
%
% \Finale

\endinput
}{}
\ProvidesFile{thcb.sty}
             [\FMithmInfo]
%    \end{macrocode}
%
% \begin{macro}{\th@changebreak}
%    The \texttt{change-break} theorem style is like \texttt{break}
%    but with interchange of theorem name and theorem number.  Thus we
%    define |\th@changebreak| as follows:
%    \begin{macrocode}
\gdef\th@changebreak{\normalfont\slshape
  \def\@begintheorem##1##2{\item
       [\rlap{\vbox{\hbox{\hskip\labelsep \theorem@headerfont ##2\ ##1}%
                    \hbox{\strut}}}]}%
\def\@opargbegintheorem##1##2##3{%
  \item[\rlap{\vbox{\hbox{\hskip\labelsep \theorem@headerfont
                     ##2\ ##1\ (##3)}%
                    \hbox{\strut}}}]}}
\endgroup
%</thcb>
%    \end{macrocode}
% \end{macro}
%
%
% \subsubsection{The \texttt{change} style}
%
% This style option is stored in the file |thc.sty|.
%    \begin{macrocode}
%<*thc>
\begingroup \makeatletter
\@ifundefined{theorem@style}{% \iffalse meta-comment
%
% Copyright (C) 1993-2024
%
% The LaTeX Project and any individual authors listed elsewhere
% in this file.
%
% This file is part of the Standard LaTeX `Tools Bundle'.
% -------------------------------------------------------
%
% It may be distributed and/or modified under the
% conditions of the LaTeX Project Public License, either version 1.3c
% of this license or (at your option) any later version.
% The latest version of this license is in
%    https://www.latex-project.org/lppl.txt
% and version 1.3c or later is part of all distributions of LaTeX
% version 2005/12/01 or later.
%
% The list of all files belonging to the LaTeX `Tools Bundle' is
% given in the file `manifest.txt'.
%
% \fi
%
%%
%
%
\def\FMithmInfo{2023/07/05 v2.2c Theorem extension package (FMi)}
%
% \ProvidesFile{theorem.dtx}[\FMithmInfo]
%\iffalse   % this is a METACOMMENT !
%%
%% Package `theorem' to use with LaTeX2e
%% Copyright (C) 1989-1995 Frank Mittelbach, all rights reserved.
%<+package>\@ifundefined{theorem@style}{}{\endinput}
%<+package>\NeedsTeXFormat{LaTeX2e}[1995/06/01]
%<+package>\ProvidesPackage{theorem}
%<+package>         [\FMithmInfo]
%
% \fi
%
% \changes{v1.0}{87/05/07}{First draft.}
% \changes{v1.0a}{87/05/07}{`thistheoremstyle' added.}
% \changes{v1.1a}{87/07/29}{New concept: redefine the internal
%                          \protect\LaTeX{} commands `@xnthm' etc.}
% \changes{v1.1b}{87/07/31}{`theoremfont' used to specify font.}
% \changes{v1.2a}{88/01/08}{`@endtheorem' now globally defined.}
% \changes{v1.2b}{88/01/11}{`newtheorem' only allowed in the preamble.}
% \changes{v1.3a}{88/02/14}{`theoremfont' now a toks register.}
%
% \changes{v2.0a}{88/11/16}{New implementation and new documentation.}
% \changes{v2.0a}{88/11/16}{No layout changes but internal modifications
%                           old substyles won't work.}
% \changes{v2.0b}{89/01/30}{`global' used for all definitions.}
% \changes{v2.0c}{89/06/15}{Documentation translated to English.
%                            Thanks to LLL and Nina}
% \changes{v2.1a}{90/02/15}{Support for new font selection.}
% \changes{v2.2a}{94/02/03}{Cleanup for LaTeX2e}
%
% \DoNotIndex{\@endparenv,\@ifdefinable,\@ifnextchar,\@ifundefined}
% \DoNotIndex{\@namedef,\@nameuse,\@spaces,\@tempa,\@warning}
% \DoNotIndex{\ }
% \DoNotIndex{\begingroup,\bfseries}
% \DoNotIndex{\csname}
% \DoNotIndex{\def,\do,\docdate}
% \DoNotIndex{\endcsname,\endgroup,\endinput,\endtrivlist,\expandafter}
% \DoNotIndex{\filedate,\fileversion}
% \DoNotIndex{\gdef,\global}
% \DoNotIndex{\hbox,\hskip}
% \DoNotIndex{\ignorespaces,\input,\itshape,\item}
% \DoNotIndex{\labelsep,\let,\llap}
% \DoNotIndex{\makeatletter}
% \DoNotIndex{\noexpand}
% \DoNotIndex{\refstepcounter,\relax,\rlap}
% \DoNotIndex{\setlength,\slshape,\space,\string,\strut}
% \DoNotIndex{\the,\trivlist,\typeout}
% \DoNotIndex{\vbox}
% \DoNotIndex{\xdef}
%
% \def\deflabel#1{\ttfamily #1\hfill}
% \def\deflist#1{\begin{list}{}{\settowidth\labelwidth{\ttfamily #1}%
%                               \setlength\leftmargin\labelwidth
%                               \addtolength\leftmargin\labelsep
%                               \let\makelabel\deflabel}}
% \def\enddeflist{\end{list}}
% \def\envfont{\normalfont\ttfamily}
%
% \MakeShortVerb{\|}
%
% \newcommand{\TUB}{{\sl TUGboat\/}}
%
% \GetFileInfo{theorem.dtx}
% \title{An Extension of the
%        \LaTeX{} theorem environment\thanks{This file
%        has version number \fileversion, last
%        revised \filedate.}}
% \author{Frank Mittelbach}
%
% \MaintainedByLaTeXTeam{tools}
% \maketitle
%
% \begin{abstract}
%    The macros described in this paper yield an extension of the
%    \LaTeX{} theorem mechanism. It is designed to satisfy the
%    different requirements of various journals. Thus, the
%    layout of the ``theorems'' can be manipulated by determining a
%    ``style''. This article describes not only the use, but
%    also the definition, of the necessary macros.
% \end{abstract}
%
% \section*{Preface to version 2.2}
%
% For \LaTeXe{} this package did not need any fundamental changes. I
% only modified the messages generated so that theorem layout styles
% will show up with the |\listfiles| command and cleaned the section
% on the New Font Selection Scheme since this is now included in
% \LaTeX.
%
% \section*{Preface to version 2.1}
%
% This version is identical to 2.0g described in \TUB\ 10\#3 except
% for some internal defaults which are now set depending on the used
% font selection scheme.
%
% This was done to avoid unpleasant surprises if the new font selection
% scheme is in force. For further details see
% section~\ref{sec:newfonts} and \cite{bk:GMS94}.
%
% \section{Introduction}
%
% For our purposes here, ``theorems'' are labelled enunciations,
% often set off from the main text by extra space and a font change.
% Theorems, corollaries, conjectures, definitions, and remarks
% are all instances of ``theorems''.  The ``header'' of these structures
% is composed of a label (such as \textsc{Theorem} or \textsc{Remark})
% and a number which serializes an item in the sequence of items
% with the same label.
%
% Shortly after the introduction of \LaTeX{} at the Fachbereich
% Mathematik in Mainz, the desire to manipulate the layout of
% ``theorems'' arose.  In Mainz, the following two conventions came
% into general use:
% \begin{enumerate}
%   \item The number of the theorem is shown in the margin.
%   \item There is a line break at
%         the end of the theorem header.
% \end{enumerate}
%
% Additionally, some journals require different formats which depend on
% the ``sort of theorem'': e.g.\ often remarks and definitions are set
% in |\upshape|, while |\itshape| is employed for main theorems.
%
% Confronted with these requirements, a theorem environment was
% developed in Mainz which allows separate determination of the
% layout of the ``theorems sets'', comparable to |\pagestyle|.
%
%
% \section{The user interface}
% \subsection{Defining new theorem sets}
%
% \DescribeMacro\newtheorem
% As in the original \LaTeX{} version, the command |\newtheorem|
% defines a new ``theorem set'' or ``theorem-like structure''.
% Two required arguments name the new environment and give the text
% to be typeset with each instance of the new ``set'', while
% an optional argument determines how
% the ``set'' is enumerated:
% \begin{description}
%    \item[\ttfamily \bslash newtheorem\{foo\}\{bar\}]
%       The theorem set {\envfont foo} (whose name is \texttt{bar})
%       uses its own counter.
%    \item[\ttfamily \bslash newtheorem\{foo2\}{[foo]}\{bar2\}]
%       The theorem set {\envfont foo2} (printed name \texttt{bar2})
%       uses the same counter as the theorem set \texttt{foo}.
%    \item[\ttfamily \bslash newtheorem\{foo3\}\{bar3\}{[section]}]
%       The theorem set {\envfont foo3} (printed name \texttt{bar3}) is
%       enumerated within the counter \texttt{section}, i.e.\ with every
%       new |\section| the enumeration begins again with 1, and
%       the enumeration is composed from the section-number and the
%       theorem counter itself.
% \end{description}
%
% \DescribeMacro\theoremstyle
% Additionally, the command |\theoremstyle| can define the layout
% of various, or all, theorem sets.  It should be noted that any
% theorem set defined by |\newtheorem| is typeset in the
% |\theoremstyle| that is current at the time of the definition.
% Thus, the following
% \begin{verbatim}
%    \theoremstyle{break}        \newtheorem{Cor}{Corollary}
%    \theoremstyle{plain}        \newtheorem{Exa}{Example}[section]
%\end{verbatim}                             ^^A No blank at line start
% leads to the result that the set {\envfont Cor} is formatted in the
% style \texttt{break}, while the set {\envfont Exa} and all the
% following ones are formatted in the style \texttt{plain}, unless
% another |\theoremstyle| follows.  Since the definitions installed by
% |\newtheorem| are global, one also can limit |\theoremstyle| locally
% by grouping braces.
%
% \DescribeMacro\theorembodyfont
% The choice of the font for the theorem body is completely
% independent of the chosen |\theoremstyle|; this has proven to
% be very advantageous.  For example,
% \begin{verbatim}
%   {\theorembodyfont{\upshape}          \newtheorem{Rem}{Remark}}
%\end{verbatim}
% defines a theorem set {\envfont Rem}, which will be set in
% |\upshape| in the current layout (which in our example is
% \texttt{plain}).  As with |\theoremstyle|, the |\theorembodyfont|
% chosen is that current at the time of |\newtheorem|.  If
% |\theorembodyfont| is not specified or one defines
% |\theorembodyfont{}|, then the font used will be that defined by the
% |\theoremstyle|.
%
% \DescribeMacro\theoremheaderfont It is also possible to customize
% the font used for the theorem headers.  This is, however, a global
% declaration, and therefore there should be at most one
% |\theoremheaderfont| declaration in the preamble.\footnote{If it is
% actually necessary to have different header fonts, one has to define
% new theorem styles (substituting the desired font) or specify the
% information directly in the \texttt{\bslash newtheorem} declaration
% (the unclean variant).}
%
% \DescribeMacro\theorempreskipamount
% \DescribeMacro\theorempostskipamount Two additional parameters
% affect the vertical space around the theorem environments:
% |\theorempreskipamount| and |\theorempostskipamount| define,
% respectively, the spacing before and after such an environment.
% These parameters apply for all theorem sets and can be manipulated
% with the ordinary length macros.  They are rubber lengths
% (`\textsf{skips}'), and therefore can contain \texttt{plus} and
% \texttt{minus} parts.
%
% Since the definition of theorem sets should---most sensibly---be
% placed in the preamble, we only allow installation there.
% It is therefore possible to release the memory used here after
% |\begin{document}|, in order to make room for other
% applications.
%
%
% \subsection{Existing theorem styles}
%
% The following theorem styles exist to date:
% \begin{deflist}{marginbreak}
%    \item[plain]
%       This theorem style emulates the original \LaTeX{} definition,
%       except that additionally the parameters
%       |\theorem...skipamount| are used.
%    \item[break]
%       In this style, the theorem header is followed by a line
%       break.
%    \item[marginbreak]
%       The theorem number is set in the margin, and there is a line
%       break as in \texttt{break}.
%    \item[changebreak]
%       Like \texttt{break}, but with header number and text
%       interchanged.
%    \item[change]
%       Header number and text are interchanged, without a line break.
%    \item[margin]
%       The number is set in the left margin, without a line break.
% \end{deflist}
% All styles (except \texttt{plain}) select |\slshape| as the default
% |\theorembodyfont|.
%
%
% \subsection{Examples}
%
% Given the above theorem sets {\envfont Cor}, {\envfont Exa}
% and {\envfont Rem},
% suppose that the preamble also contains the declarations:
% \begin{verbatim}
%    \theoremstyle{marginbreak}   \newtheorem{Lem}[Cor]{Lemma}
%    \theoremstyle{change}
%    \theorembodyfont{\itshape}        \newtheorem{Def}[Cor]{Definition}
%
%    \theoremheaderfont{\scshape}
%\end{verbatim}
% Then the following are some typical examples of the typeset output
% resulting from their use.
%
% \begin{Cor}
%    This is a sentence typeset in the theorem environment {\envfont
%    Cor}.
% \end{Cor}
%
% \begin{Exa}
%    This is a sentence typeset in the theorem environment {\envfont
%    Exa}.
% \end{Exa}
%
% \begin{Rem}
%    This is a sentence typeset in the theorem environment {\envfont
%    Rem}.
% \end{Rem}
%
% \begin{Lem}[Ben User]
%    This is a sentence typeset in the theorem environment {\envfont
%    Lem}.
% \end{Lem}
%
% \begin{Def}[Very impressive Definition]
%    This is a sentence typeset in the theorem environment {\envfont
%    Def}.
% \end{Def}
%
% The last two examples show the effect of the optional argument to
% a theorem environment (it is the text typeset in parentheses).
%
%
% \section{Special Considerations}
% \label{sec:newfonts}
%
% Theoremheader and body are implemented as a unit. This means that
% the |\theoremheaderfont|  will inherit characteristics of the
% |\theorembodyfont| in \LaTeXe. Thus, if for example
% |\theorembodyfont| is |\itshape| and |\theoremheaderfont|
% is |\bfseries| the font selected for the header will have the
% characteristics `bold extended italic'. If this is not desired one
% should set the |\theoremheaderfont| to something like
% \begin{verbatim}
%      \theoremheaderfont{\normalfont\bfseries}
%\end{verbatim}
% i.e.\ supplying all necessary font information explicitly.
%
%
%
% \section{Acknowledgements}
%
% The publication of this set of macros was only possible with the
% help of Christina Busse (translating the manuscript into English),
% Joachim Pense (playing the r\^ole of typist), Chris Rowley
% (looking everything over) and many others providing useful
% suggestions.
%
% \MaybeStop{%
%    \begin{thebibliography}{9}
%       \bibitem{bk:GMS94} \textsc{M.~Goossens}, \textsc{F.~Mittelbach}
%         and \textsc{A.~Samarin}.
%         \newblock The \LaTeX{} Companion.
%         \newblock
%         Addison-Wesley, Reading, Massachusetts, 1994.
%       \bibitem{latex:source} \textsc{Lamport, Leslie.} \newblock
%                               \texttt{latex.tex}, version 2.09, date
%                               Feb. 1990.
%    \end{thebibliography}
%    \PrintIndex
%    \PrintChanges}
%
%
% \section{The documentation driver file}
%
% The next bit of code contains the documentation driver file for
% \TeX{}, i.e., the file that will produce the documentation you are
% currently reading. It will be extracted from this file by the
% \texttt{docstrip} program. Since it is the first code in the file
% one can alternatively process this file directly with \LaTeXe{} to
% obtain the documentation.
%
%    \begin{macrocode}
%<*driver>
\documentclass{ltxdoc}

\usepackage{theorem}

%<+driver>% The next few lines define theorem sets which are used
%<+driver>% in the example section of the documentation.

\theoremstyle{break}       \newtheorem{Cor}{Corollary}
\theoremstyle{plain}       \newtheorem{Exa}{Example}[section]
{\theorembodyfont{\upshape}\newtheorem{Rem}{Remark}}
\theoremstyle{marginbreak} \newtheorem{Lem}[Cor]{Lemma}
\theoremstyle{change}
\theorembodyfont{\itshape} \newtheorem{Def}[Cor]{Definition}

\theoremheaderfont{\scshape}

\RecordChanges

\begin{document}
   \DocInput{theorem.dtx}
\end{document}
%</driver>
%    \end{macrocode}
%
% \section{Definition of the Macros}
%
% If the file has been loaded before, we abort immediately. If not the
% package announces itself (this is actually done at the very top of
% the file---the way it is done isn't good style so don't copy it).
% \changes{v2.0e}{89/07/19}{Spaces removed from `typeout'.}
% \changes{v2.2b}{95/11/19}{Announce moved to top of file}
%    \begin{macrocode}
%<*package>
%\@ifundefined{theorem@style}{}{\endinput}
%\def\FMithmInfo{1995/11/19 v2.2b Theorem extension package (FMi)}
%\ProvidesPackage{theorem}[\FMithmInfo]
%    \end{macrocode}
%
%
% \subsection{Definition of theorem styles and fonts}
%
% All the definitions in this file are done globally to allow
% inputting this file inside a group.
%
% \begin{macro}{\theoremstyle}
%    Before a theorem style can be installed, the chosen style must be
%    known. For that reason, we must test to see that
%    |\th@|\meta{style} is known or, more precisely, that it is
%    different from |\relax|. If the style is not known then
%    |\th@plain| is used.
%    \begin{macrocode}
\gdef\theoremstyle#1{%
   \@ifundefined{th@#1}{\@warning
          {Unknown theoremstyle `#1'. Using `plain'}%
          \theorem@style{plain}}%
%    \end{macrocode}
%    We save the theorem style to be used in the token register
%    |\theorem@style|.
%    \begin{macrocode}
      {\theorem@style{#1}}%
%    \end{macrocode}
%    Now we ``evaluate'' the theorem style: this means, we call the
%    macro |\th@|\meta{style} which will activate the relevant
%    definitions which are contained in a separate file.  This is done
%    in a group to suppress changes to the current font.  This could
%    otherwise pose problems together with the new font selection
%    scheme\footnote{When I printed the original article using the new
%    font selection scheme I ended with a document with slanted
%    typefaces (text headings and all) simply because one of the
%    theorem styles used \texttt{\bslash sl} at toplevel.} if the
%    |\th@|\meta{style} is evaluated a second time.
% \changes{v2.1a}{90/02/15}{Group added around `th@..}
% \changes{v2.1a}{90/02/15}{Reseting of `..@begintheorem removed.}
%    \begin{macrocode}
      \begingroup
        \csname th@\the\theorem@style \endcsname
      \endgroup}
%    \end{macrocode}
% \end{macro}
%
%
% \begin{macro}{\@begintheorem}
% \changes{v2.1a}{90/02/15}{Relaxed at toplevel.}
% \begin{macro}{\@opargbegintheorem}
% \changes{v2.1a}{90/02/15}{Relaxed at toplevel.}
%    We reset |\@begintheorem| and |\@opargbegintheorem| to
%    |\relax| since these commands are no longer necessary at
%    toplevel. This will save a few tokens.
%    \begin{macrocode}
\global\let\@begintheorem\relax
\global\let\@opargbegintheorem\relax
%    \end{macrocode}
% \end{macro}
% \end{macro}
%
% \begin{macro}{\theorem@style}
%    Obviously the \textsf{token} register used above has to be
%    allocated. To assure the utmost compatibility with the
%    original \LaTeX{} definition, we set the default theorem style to
%    \texttt{plain}, which implements the usual \LaTeX{} convention.
% \changes{v2.0f}{89/07/20}{Unnecessary `global' removed.}
%    \begin{macrocode}
\newtoks\theorem@style
\global\theorem@style{plain}
%    \end{macrocode}
% \end{macro}
%
% \begin{macro}{\theorembodyfont}
% \begin{macro}{\theorem@bodyfont}
%    If the |\theorembodyfont| is set by the user then it should not
%    interact with the default font set in the theorem style. When the
%    new font selection is in force this may happen if, for example,
%    the default is |\itshape| and the new |\theorembodyfont| is
%    |\sffamily|. So we add a |\reset@font| command in front of the
%    user definition.
% \changes{v2.0f}{89/07/20}{Unnecessary `global' removed.}
% \changes{v2.0g}{89/09/19}{`theoremfont to `theorembodyfont.}
% \changes{v2.1c}{92/06/22}{`reset@font added and logic changed.}
% \changes{v2.1d}{92/10/28}{Reset to empty if arg is empty}
%    \begin{macrocode}
\gdef\theorembodyfont#1{%
%    \end{macrocode}
%    We check if the argument supplied is empty and if so put nothing
%    into the |\theorem@bodyfont| token register to allow for
%    |\theorembodyfont{}| as a mean of using the default of the
%    current |\theoremstyle|.
%    \begin{macrocode}
   \def\@tempa{#1}%
   \ifx\@tempa\@empty
    \theorem@bodyfont{}%
   \else
    \theorem@bodyfont{\reset@font#1}%
   \fi
   }
\newtoks\theorem@bodyfont
\global\theorem@bodyfont{}
%    \end{macrocode}
% \end{macro}
% \end{macro}
%
%
% \begin{macro}{\theoremheaderfont}
% \changes{v2.0g}{89/09/19}{Macro added.}
%    The font for the theorem headers is handled differently because
%    this definition applies to all theorem styles.
%    \begin{macrocode}
\gdef\theoremheaderfont#1{\gdef\theorem@headerfont{#1}%
%    \end{macrocode}
%    After using the macro once it is redefined to produce an error
%    message.
%    \begin{macrocode}
       \gdef\theoremheaderfont##1{%
        \typeout{\string\theoremheaderfont\space should be used
                 only once.}}}
%    \end{macrocode}
% \end{macro}
%
% \begin{macro}{\theorem@headerfont}
% \changes{v2.0g}{89/09/19}{Macro added.}
%    To set the |\theorem@headerfont| default we first test if
%    the new fontselection scheme is in force.
% \changes{v2.1a}{90/02/15}{Support for new font selection.}
%    \begin{macrocode}
\ifx\upshape\undefined
%    \end{macrocode}
%    If not we define it to expand into |\bfseries|. We don't use
%    |\let| just in case a following style option redefines this
%    macro.
%    \begin{macrocode}
\gdef\theorem@headerfont{\bfseries}
%    \end{macrocode}
%    Otherwise we reset the current shape before calling |\bfseries|.
%    \begin{macrocode}
\else \gdef\theorem@headerfont{\normalfont\bfseries}\fi
%    \end{macrocode}
% \end{macro}
%
% \begin{macro}{\th@plain}
% \begin{macro}{\th@break}
% \begin{macro}{\th@marginbreak}
% \begin{macro}{\th@changebreak}
% \begin{macro}{\th@change}
% \begin{macro}{\th@margin}
%    The different styles are defined in macros such as
%    |\th@plain|.  Since memory space is precious in
%    ``non-Big-versions'', we have to avoid offering too many unused
%    definitions.  Therefore we define these styles in separate files
%    that can be loaded on demand.  Thus the commands themselves only
%    load these files. We use |\@input@|, a \LaTeXe{} internal command
%    that ensures that the file will be listed with |\listfiles|.
% \changes{v2.2a}{94/02/02}{Use `@input@ to load theorem layout files}
%    \begin{macrocode}
\gdef\th@plain{\@input@{thp.sty}}
\gdef\th@break{\@input@{thb.sty}}
\gdef\th@marginbreak{\@input@{thmb.sty}}
\gdef\th@changebreak{\@input@{thcb.sty}}
\gdef\th@change{\@input@{thc.sty}}
\gdef\th@margin{\@input@{thm.sty}}
%    \end{macrocode}
%    This list will be expanded when new styles become
%    available.  For testing,  just append new theorem substyles
%    as document options.
% \end{macro}
% \end{macro}
% \end{macro}
% \end{macro}
% \end{macro}
% \end{macro}
%
%
%
% \subsection{Definition of a new theorem set}
%
% As already pointed out, a new theorem environment can be defined in
% three different ways:
% \begin{verbatim}
%    \newtheorem{Lem}{Lemma}
%    \newtheorem{Lem}{Lemma}[section]
%    \newtheorem{Lem}[Theorem]{Lemma}
%\end{verbatim}
% The function of the macro |\newtheorem| is to recognize these cases
% and then to branch into one of the three macros |\@ynthm|, |\@xnthm|
% or |\@othm|.  This mechanism is adopted unchanged from
% \cite{latex:source}; the essential point here is that, for example,
% in the second case, the arguments \texttt{Lem}, \texttt{Lemma} and
% \texttt{section} are passed over to the macro |\@xnthm|.
%
% We inspect this case first because the others present fewer problems,
% and thus are easily derived from this one.
%
% \begin{macro}{\@xnthm}
%    For our example arguments, the macro |\@xnthm| must fulfill
%    the following:
%    \begin{itemize}
%       \item
%          Define a new \LaTeX-counter `Lem'
%       \item
%          reset this counter within a |\section|
%       \item
%          define the macro |\theLem|
%       \item
%          define the environment macros |\Lem| and
%          |\endLem| using the current
%          |\theoremstyle| and |\theorem@bodyfont|.
%    \end{itemize}
%    Obviously, all this should happen only if the first argument
%    of |\@xnthm| (i.e.\ \texttt{Lem} in our example) is chosen so
%    as not to conflict with any previously defined commands or
%    environments.  This test is performed by the \LaTeX{} macro
%    |\@ifdefinable|.
%    \begin{macrocode}
\gdef\@xnthm#1#2[#3]{\expandafter\@ifdefinable\csname #1\endcsname
%    \end{macrocode}
%    Therefore, the first argument of |\@ifdefinable| is the
%    expansion (in the example, |\Lem|) of
%    |\csname||#1\endcsname|.  The second argument is
%    executed only if the test has been completed successfully.
%    \begin{macrocode}
   {%
%    \end{macrocode}
%    Now we define the new counter. The names of the \LaTeX{}
%    macros employed should speak for themselves:
% \changes{v2.2b}{1995/10/07}{Use \cs{@newctr}}
% \changes{v2.2c}{1995/11/23}{Use \cs{@newctr} properly.}
%    \begin{macrocode}
    \@definecounter{#1}\@newctr{#1}[#3]%
%    \end{macrocode}
%    Using |\@newctr| will give a proper error message if the counter
%    in |#3| is not defined.
%    In defining `|\theLem|' we must generate the desired macro
%    name by use of |\expandafter| and |\csname|.
%    \begin{macrocode}
    \expandafter\xdef\csname the#1\endcsname
%    \end{macrocode}
%    An |\xdef| is used in order to make the definition global,
%    and to ensure that it contains the replacement texts of
%    |\@thmcountersep| and |\@thmcounter|.\footnote{These
%    two macros can be defined by the document style.  Their default
%    values produce a `\texttt{.}' as separation and an arabic
%    representation of the number.} However, not everything should be
%    expanded. For example, it saves space to use |\thesection|
%    instead of its---at times---lengthy expansion.
%    \begin{macrocode}
      {\expandafter \noexpand \csname the#3\endcsname
       \@thmcountersep \@thmcounter{#1}}%
%    \end{macrocode}
%    Thus with the defaults of \LaTeX{}, |\theLem| would be
%    replaced by the command sequence |\thesection.\arabic{Lem}|.
%
%    We will now look at the definition of the macro which is executed
%    at the beginning of the actual environment (in our example this
%    macro is |\Lem|). It should be noted that we use an
%    ``|\expandafter| trick'' to expand only certain parts of the
%    replacement text at the time of the definition.
%    \begin{macrocode}
    \def\@tempa{\global\@namedef{#1}}%
    \expandafter \@tempa \expandafter{%
%    \end{macrocode}
%    First, the macro that contains the current definitions of
%    |\@begintheorem| and |\@opargtheorem| should be called
%    up.  The name of this macro---as is already known---has the form
%    |\th@|\meta{theorem\ style}; therefore, it must be called by
%    \begin{macrocode}
      \csname th@\the \theorem@style
%    \end{macrocode}
%    In addition the default theorem font should be changeable, i.e.\
%    we have to insert the contents of |\theorem@bodyfont|. For
%    that reason, we expand even further, beyond |\endcsname|,
%    and thus insert the contents of the \textsf{token} register
%    |\theorem@bodyfont| in the replacement text.
%    \begin{macrocode}
            \expandafter \endcsname \the \theorem@bodyfont
%    \end{macrocode}
%    Now it is time to call the macro |\@thm| which takes over
%    the further processing.  It has two arguments: the current
%    counter name (in our example, \texttt{Lem}), and the text of the
%    label (in our example, \texttt{Lemma}).
%    \begin{macrocode}
     \@thm{#1}{#2}}%
%    \end{macrocode}
%    With this, the `sub-definition' is complete.  The macro
%    |\@endtheorem| ends a theorem environment and is, so far, nothing
%    but an |\endtrivlist|.  (Hence it is defined globally, and not
%    within the theorem styles.\footnote{This has to be changed as
%    soon as theorem styles that change \texttt{\bslash @endtheorem}
%    exist.  In such a case, all existing styles must be changed as
%    well since they will have to reset the macro.}) Therefore, we can
%    set it equivalent to the macro that ends the theorem set (in our
%    example, |\endLem|).  However, if some day theorem styles exist
%    that do change |\@endtheorem|, we would have to use the
%    commented-out line instead.
%    \begin{macrocode}
    \global \expandafter \let \csname end#1\endcsname \@endtheorem
%   \global\@namedef{end#1}{\@endtheorem}%
%    \end{macrocode}
%    With these commands all the required definitions are employed,
%    unless the test |\@ifdefinable| has failed.  Therefore, we
%    end the second argument of this macro and with it the definition
%    of |\@xnthm|.
%    \begin{macrocode}
   }}
%    \end{macrocode}
% \end{macro}
%
% \begin{macro}{\@ynthm}
%    The definition of |\@ynthm| is completely analogous. In
%    this case the new counter that is defined is not reset within
%    another counter; thus the definition of |\the...|
%    is simplified:
%    \begin{macrocode}
\gdef\@ynthm#1#2{\expandafter\@ifdefinable\csname #1\endcsname
   {\@definecounter{#1}%
    \expandafter\xdef\csname the#1\endcsname{\@thmcounter{#1}}%
%    \end{macrocode}
%    The rest of the definition corresponds literally to that of
%    |\@xnthm|:
%    \begin{macrocode}
    \def\@tempa{\global\@namedef{#1}}\expandafter \@tempa
     \expandafter{\csname th@\the \theorem@style \expandafter
     \endcsname \the\theorem@bodyfont \@thm{#1}{#2}}%
    \global \expandafter \let \csname end#1\endcsname \@endtheorem}}
%    \end{macrocode}
% \end{macro}
%
% \begin{macro}{\@othm}
%    The definition of |\@othm| does not contain anything new.
%    \begin{macrocode}
\gdef\@othm#1[#2]#3{%
%    \end{macrocode}
% \changes{v2.2b}{95/10/07}{Build in check for counter defined}
%    We do not define a new counter but instead use one that has
%    already been defined.  Thus the only definition we need is that
%    of this pseudo-counter (i.e.\ |\the|\meta{env.\ name}).
%    First we check if |#2| corresponds to a known counter name.
%    \begin{macrocode}
  \expandafter\ifx\csname c@#2\endcsname\relax
   \@nocounterr{#2}%
  \else
   \expandafter\@ifdefinable\csname #1\endcsname
   {\expandafter \xdef \csname the#1\endcsname
     {\expandafter \noexpand \csname the#2\endcsname}%
%    \end{macrocode}
%    All other parts of the definition can be adopted from
%    |\@xnthm|. We have to remember, though, that in this case
%    the name of the current counter and the theorem label have moved
%    to the second and third arguments.
%    \begin{macrocode}
    \def\@tempa{\global\@namedef{#1}}\expandafter \@tempa
     \expandafter{\csname th@\the \theorem@style \expandafter
     \endcsname \the\theorem@bodyfont \@thm{#2}{#3}}%
    \global \expandafter \let \csname end#1\endcsname \@endtheorem}%
  \fi}
%    \end{macrocode}
% \end{macro}
%
% \subsection{Macros that are employed in a {\envfont theorem}
%    environment}
%
% \begin{macro}{\@thm}
%    The macro |\@thm| has to increase the current counter.  Then,
%    depending on whether the environment has (or does not have) an
%    optional argument, it has to branch into either |\@begintheorem|
%    or |\@opargtheorem|.
%    \begin{macrocode}
\gdef\@thm#1#2{\refstepcounter{#1}%
%    \end{macrocode}
%    Now we start a {\envfont trivlist} environment, and give
%    |\@topsep| and |\@topsepadd| the values of the \textsf{skip}
%    registers |\theorempreskipamount| and |\theorempostskipamount|.
%    The value in |\@topsep| is the vertical space that is inserted by
%    the first (and only) |\item| in our |\trivlist| whilst
%    |\@topsepadd| is inserted by |\@endparenv| at the end of that
%    {\envfont trivlist} environment.  By using these registers, we
%    obtain the desired space around a {\envfont theorem} environment.
%    \begin{macrocode}
   \trivlist
   \@topsep \theorempreskipamount               % used by first \item
   \@topsepadd \theorempostskipamount           % used by \@endparenv
%    \end{macrocode}
%    Now we have to test whether an optional argument has been given.
% \changes{v2.0d}{89/07/18}{Forgotten percent added.}
%    \begin{macrocode}
   \@ifnextchar [%
%    \end{macrocode}
%    If there is an optional argument, we will call |\@ythm|, and
%    move the arguments read back into the input stream.
%    \begin{macrocode}
   {\@ythm{#1}{#2}}%
%    \end{macrocode}
%    If not, we call |\@begintheorem|. Its first argument is the
%    name of the theorem set (hence the second argument of
%    |\@thm|). Its second argument is the macro that produces the
%    current number.
%    \begin{macrocode}
   {\@begintheorem{#2}{\csname the#1\endcsname}\ignorespaces}}
%    \end{macrocode}
% \end{macro}
%
% \begin{macro}{\@xthm}
% \begin{macro}{\@ythm}
%    Both these macros were originally called by |\@thm|.  We do
%    not need |\@xthm| anymore, hence we reset it to
%    |\relax|. The definition of |\@ythm| has not changed at
%    all from its definition in \LaTeX. In order to make the macros
%    easier to understand, we will
%    nevertheless present it (commented out).
%    \begin{macrocode}
\global\let\@xthm\relax
% \def\@ythm#1#2[#3]{\@opargbegintheorem{#2}{\csname
%           the#1\endcsname}{#3}\ignorespaces}
%</package>
%    \end{macrocode}
%    The primitive |\ignorespaces| in |\@ythm| and
%    |\@thm| is needed to remove the spaces between the
%    |\begin{...}| and the actual text.
% \end{macro}
% \end{macro}
%
%
% \subsection{Definition of the theorem substyles}
%
% As already pointed out, the theorem substyles, defined below, are
% only loaded when necessary.  Note that all these substyles, except
% \texttt{plain}, have |\slshape| as the default body font.
%
% \subsubsection{The \texttt{plain} style}
%
% \label{dok:th@plain} As the following macros use |@|, we have
% to locally set the |\catcode| of this symbol to ``letter''.
% This happens within a group, so that we do not have to worry about
% which |\catcode| that symbol had before.
%    \begin{macrocode}
%<*thp>
\begingroup \makeatletter
%    \end{macrocode}
% Since we are now within a group, we must make all definitions
% globally.  First we make sure that \texttt{theorem.sty} is loaded.
% This will allow us to use this file as a document style option
% without having to call \texttt{theorem} itself as an option.  At the
% same time, we assure that at least version 2 is loaded, since
% |\theorem@style| was not defined in earlier versions.
%    \begin{macrocode}
\@ifundefined{theorem@style}{\input{theorem.sty}}{}
\ProvidesFile{thp.sty}
             [\FMithmInfo]
%    \end{macrocode}
%
%
% \begin{macro}{\th@plain}
% \begin{macro}{\@begintheorem}
% \begin{macro}{\@opargbegintheorem}
%    |\theoremstyle{plain}| corresponds to the original
%    definition, except that the distances to the surrounding text are
%    determined by the parameters |\theorempreskipamount| and
%    |\theorempostskipamount|.
%    First we set the default body font.
%    \begin{macrocode}
\gdef\th@plain{\normalfont\itshape
%    \end{macrocode}
%    Then we define |\@begintheorem| and
%    |\@opargbegintheorem|. These two macros define how the
%    header of a theorem is typeset.  |\@opargbegintheorem| will
%    be called if a \textsf{theorem} environment with an optional
%    argument is encountered; otherwise, the header is constructed by
%    calling |\@begintheorem|.  If one of these macros is
%    executed, we are within a \textsf{trivlist} environment started by
%    |\@thm|.  So the theorem header is produced with an
%    |\item| command.
%
%    Instead of specifying the header font directly, all standard
%    theorem styles use the |\theorem@headerfont| macro to allow
%    customization.  The extra space (|\labelsep|) is necessary
%    because of problems in the \textsf{trivlist} environment.
%    \begin{macrocode}
  \def\@begintheorem##1##2{%
        \item[\hskip\labelsep \theorem@headerfont ##1\ ##2]}%
%    \end{macrocode}
%    The definition of |\@opargbegintheorem| is completely
%    analogous. The only difference is the fact that there exists a
%    third argument (which is the optional parameter of the
%    environment and contains additional information about the
%    theorem).  Customarily we enclose it in parentheses.
%    \begin{macrocode}
\def\@opargbegintheorem##1##2##3{%
   \item[\hskip\labelsep \theorem@headerfont ##1\ ##2\ (##3)]}}
%    \end{macrocode}
% \end{macro}
% \end{macro}
% \end{macro}
%
% \noindent
% We conclude with an |\endgroup| to restore the
% |\catcode| of |@|.
%    \begin{macrocode}
\endgroup
%</thp>
%    \end{macrocode}
%
%
% \subsubsection{The \texttt{break} style}
%
% This style option is stored in the file |thb.sty|.
% For the next two lines see the documentation for |\th@plain| on
% page~\pageref{dok:th@plain}.
%    \begin{macrocode}
%<*thb>
\begingroup \makeatletter
\@ifundefined{theorem@style}{\input{theorem.sty}}{}
\ProvidesFile{thb.sty}
             [\FMithmInfo]
%    \end{macrocode}
%
% \begin{macro}{\th@break}
%    |\theoremstyle{break}| produces a line break after the name
%    of the theorem.  The font is |\slshape|.  Hence, we define
%    |\th@break| as follows:
%    \begin{macrocode}
\gdef\th@break{\normalfont\slshape
  \def\@begintheorem##1##2{\item[%
%    \end{macrocode}
%    We run into the following problem: it is not possible to create
%    the header with |\item[|\meta{title}|]| and then start
%    a new line by, for example, |\mbox{}\\|. Such a definition
%    will fail whenever a list environment follows immediately.  With
%    the above construction, the |\mbox{}| causes the switch
%    |@inlabel| (cf.\ definition of |\list| and
%    |\trivlist| in \cite{latex:source}) to be set to \texttt{false}
%    and so the following list will insert additional vertical
%    space (|\topskip|). This is quite annoying. Therefore, we
%    create the line break within the |\item|. In order to ensure
%    that the text will begin at the proper position in the following
%    line, we simply pretend that the label does not take any
%    room.\footnote{This will lead to problems whenever very high
%    symbols occurring in the line tower into the
%    heading. So, something else has to be done here sometime.}
%    \begin{macrocode}
       \rlap{\vbox{\hbox{\hskip \labelsep\theorem@headerfont ##1\ ##2}%
                   \hbox{\strut}}}]}%
%    \end{macrocode}
%    Again, the definition of |\@opargbegintheorem| is
%    completely analogous.
%    \begin{macrocode}
\def\@opargbegintheorem##1##2##3{%
  \item[\rlap{\vbox{\hbox{\hskip \labelsep \theorem@headerfont
                     ##1\ ##2\ (##3)}%
                    \hbox{\strut}}}]}}
\endgroup
%</thb>
%    \end{macrocode}
% \end{macro}
%
%
% \subsubsection{The \texttt{changebreak} style}
%
%    \begin{macrocode}
% This style option is stored in the file |thcb.sty|.
%    \begin{macrocode}
%<*thcb>
\begingroup \makeatletter
\@ifundefined{theorem@style}{\input{theorem.sty}}{}
\ProvidesFile{thcb.sty}
             [\FMithmInfo]
%    \end{macrocode}
%
% \begin{macro}{\th@changebreak}
%    The \texttt{change-break} theorem style is like \texttt{break}
%    but with interchange of theorem name and theorem number.  Thus we
%    define |\th@changebreak| as follows:
%    \begin{macrocode}
\gdef\th@changebreak{\normalfont\slshape
  \def\@begintheorem##1##2{\item
       [\rlap{\vbox{\hbox{\hskip\labelsep \theorem@headerfont ##2\ ##1}%
                    \hbox{\strut}}}]}%
\def\@opargbegintheorem##1##2##3{%
  \item[\rlap{\vbox{\hbox{\hskip\labelsep \theorem@headerfont
                     ##2\ ##1\ (##3)}%
                    \hbox{\strut}}}]}}
\endgroup
%</thcb>
%    \end{macrocode}
% \end{macro}
%
%
% \subsubsection{The \texttt{change} style}
%
% This style option is stored in the file |thc.sty|.
%    \begin{macrocode}
%<*thc>
\begingroup \makeatletter
\@ifundefined{theorem@style}{\input{theorem.sty}}{}
\ProvidesFile{thc.sty}
             [\FMithmInfo]
%    \end{macrocode}
%
% \begin{macro}{\th@change}
%    The \texttt{change} theorem style corresponds to the
%    \texttt{change break} style without a linebreak after the header.
%    To say it in another way, it's the same as the \texttt{plain}
%    style but with number and name interchanged and |\slshape| as the
%    default font.
%    \begin{macrocode}
\gdef\th@change{\normalfont\slshape
  \def\@begintheorem##1##2{\item
       [\hskip\labelsep \theorem@headerfont ##2\ ##1]}%
\def\@opargbegintheorem##1##2##3{%
  \item[\hskip\labelsep \theorem@headerfont ##2\ ##1\ (##3)]}}
\endgroup
%</thc>
%    \end{macrocode}
% \end{macro}
%
%
% \subsubsection{The \texttt{marginbreak} style}
%
% This style option is the one used most often at Mainz.  It is saved
% in the file |thmb.sty|.
%    \begin{macrocode}
%<*thmb>
\begingroup \makeatletter
\@ifundefined{theorem@style}{\input{theorem.sty}}{}
\ProvidesFile{thmb.sty}
             [\FMithmInfo]
%    \end{macrocode}
%
% \begin{macro}{\th@marginbreak}
%    The \texttt{margin break} style is nearly the same as the
%    \texttt{change break} style. The only difference is the placement
%    of the theorem number.  We use |\llap| to place it in the left
%    margin.
%
%    In this style |\labelsep| denotes the separation between the
%    number and the text.
%    \begin{macrocode}
\gdef\th@marginbreak{\normalfont\slshape
  \def\@begintheorem##1##2{\item
       [\rlap{\vbox{\theorem@headerfont
                    \hbox{\llap{##2}\hskip\labelsep ##1}%
                    \hbox{\strut}}}]}%
\def\@opargbegintheorem##1##2##3{%
  \item[\rlap{\vbox{\theorem@headerfont
                    \hbox{\llap{##2}\hskip\labelsep ##1\ (##3)}%
                    \hbox{\strut}}}]}}
\endgroup
%</thmb>
%    \end{macrocode}
% \changes{v2.1c}{90/02/17}{`endgroup outside macrocode env.}
% \end{macro}
%
%
%
% \subsubsection{The \texttt{margin} style}
%
% This style option is stored in the file |thm.sty|.
%    \begin{macrocode}
%<*thm>
\begingroup \makeatletter
\@ifundefined{theorem@style}{\input{theorem.sty}}{}
\ProvidesFile{thm.sty}
             [\FMithmInfo]
%    \end{macrocode}
%
% \begin{macro}{\th@margin}
%    Again this is only a variant of the theorem styles described
%    above without any new ideas.
%    \begin{macrocode}
\gdef\th@margin{\normalfont\slshape
  \def\@begintheorem##1##2{\item
       [\theorem@headerfont \llap{##2}\hskip\labelsep ##1]}%
\def\@opargbegintheorem##1##2##3{%
  \item[\theorem@headerfont \llap{##2}\hskip\labelsep ##1\ (##3)]}}
\endgroup
%</thm>
%    \end{macrocode}
% \end{macro}
%
%
%
%
% \subsection{Final Definitions}
%
% \begin{macro}{\theorempreskipamount}
% \begin{macro}{\theorempostskipamount}
%    The \textsf{skip} parameters that regulate the vertical empty space
%    before and after the {\envfont theorem} environment have
%    to be allocated as well.
% \changes{v2.0f}{89/07/20}{Unnecessary `global' removed.}
%    \begin{macrocode}
%<*package>
\newskip\theorempreskipamount
\newskip\theorempostskipamount
%    \end{macrocode}
%    Since we have used the same values for all theorem sets, we now can
%    assign them.
%    \begin{macrocode}
\global\setlength\theorempreskipamount{12pt plus 5pt minus 3pt}
\global\setlength\theorempostskipamount{8pt plus 3pt minus 1.5pt}
%    \end{macrocode}
% \end{macro}
% \end{macro}
%
%
% \begin{macro}{\@endtheorem}
%    The same holds for the macro |\@endtheorem|, which ends a
%    {\envfont theorem} environment. Since it is the same for all
%    theorem sets, it is removed from the macros |\th@|\meta{style}.
%    It simply ends the {\envfont trivlist} environment, which was
%    begun in |\@thm|.
%    \begin{macrocode}
\global\let\@endtheorem=\endtrivlist
%    \end{macrocode}
% \end{macro}
%
%
%
% \begin{macro}{\@preamblecmds}
%    All macros defined above are to be used only in the preamble.
%    Therefore, we insert them in |\@preamblecmds| which will disable
%    them at begin document. This is done by the internal \LaTeXe{}
%    command |\@onlypreamble|.
% \changes{v2.0g}{89/09/19}{`theoremheaderfont and `theorembodyfont
%    added.}
% \changes{v2.2a}{94/02/03}{Use \LaTeXe{} interface}
%    \begin{macrocode}
\@onlypreamble\@xnthm
\@onlypreamble\@ynthm
\@onlypreamble\@othm
\@onlypreamble\newtheorem
\@onlypreamble\theoremstyle
\@onlypreamble\theorembodyfont
\@onlypreamble\theoremheaderfont
%    \end{macrocode}
% \end{macro}
%
% Finally we declare the \texttt{plain} theorem style to be the default.
% \changes{v2.1b}{90/06/04}{Fixed missing `item by declaring default
%    style.}
%    \begin{macrocode}
\theoremstyle{plain}
%</package>
%    \end{macrocode}
%
% \Finale

\endinput
}{}
\ProvidesFile{thc.sty}
             [\FMithmInfo]
%    \end{macrocode}
%
% \begin{macro}{\th@change}
%    The \texttt{change} theorem style corresponds to the
%    \texttt{change break} style without a linebreak after the header.
%    To say it in another way, it's the same as the \texttt{plain}
%    style but with number and name interchanged and |\slshape| as the
%    default font.
%    \begin{macrocode}
\gdef\th@change{\normalfont\slshape
  \def\@begintheorem##1##2{\item
       [\hskip\labelsep \theorem@headerfont ##2\ ##1]}%
\def\@opargbegintheorem##1##2##3{%
  \item[\hskip\labelsep \theorem@headerfont ##2\ ##1\ (##3)]}}
\endgroup
%</thc>
%    \end{macrocode}
% \end{macro}
%
%
% \subsubsection{The \texttt{marginbreak} style}
%
% This style option is the one used most often at Mainz.  It is saved
% in the file |thmb.sty|.
%    \begin{macrocode}
%<*thmb>
\begingroup \makeatletter
\@ifundefined{theorem@style}{% \iffalse meta-comment
%
% Copyright (C) 1993-2024
%
% The LaTeX Project and any individual authors listed elsewhere
% in this file.
%
% This file is part of the Standard LaTeX `Tools Bundle'.
% -------------------------------------------------------
%
% It may be distributed and/or modified under the
% conditions of the LaTeX Project Public License, either version 1.3c
% of this license or (at your option) any later version.
% The latest version of this license is in
%    https://www.latex-project.org/lppl.txt
% and version 1.3c or later is part of all distributions of LaTeX
% version 2005/12/01 or later.
%
% The list of all files belonging to the LaTeX `Tools Bundle' is
% given in the file `manifest.txt'.
%
% \fi
%
%%
%
%
\def\FMithmInfo{2023/07/05 v2.2c Theorem extension package (FMi)}
%
% \ProvidesFile{theorem.dtx}[\FMithmInfo]
%\iffalse   % this is a METACOMMENT !
%%
%% Package `theorem' to use with LaTeX2e
%% Copyright (C) 1989-1995 Frank Mittelbach, all rights reserved.
%<+package>\@ifundefined{theorem@style}{}{\endinput}
%<+package>\NeedsTeXFormat{LaTeX2e}[1995/06/01]
%<+package>\ProvidesPackage{theorem}
%<+package>         [\FMithmInfo]
%
% \fi
%
% \changes{v1.0}{87/05/07}{First draft.}
% \changes{v1.0a}{87/05/07}{`thistheoremstyle' added.}
% \changes{v1.1a}{87/07/29}{New concept: redefine the internal
%                          \protect\LaTeX{} commands `@xnthm' etc.}
% \changes{v1.1b}{87/07/31}{`theoremfont' used to specify font.}
% \changes{v1.2a}{88/01/08}{`@endtheorem' now globally defined.}
% \changes{v1.2b}{88/01/11}{`newtheorem' only allowed in the preamble.}
% \changes{v1.3a}{88/02/14}{`theoremfont' now a toks register.}
%
% \changes{v2.0a}{88/11/16}{New implementation and new documentation.}
% \changes{v2.0a}{88/11/16}{No layout changes but internal modifications
%                           old substyles won't work.}
% \changes{v2.0b}{89/01/30}{`global' used for all definitions.}
% \changes{v2.0c}{89/06/15}{Documentation translated to English.
%                            Thanks to LLL and Nina}
% \changes{v2.1a}{90/02/15}{Support for new font selection.}
% \changes{v2.2a}{94/02/03}{Cleanup for LaTeX2e}
%
% \DoNotIndex{\@endparenv,\@ifdefinable,\@ifnextchar,\@ifundefined}
% \DoNotIndex{\@namedef,\@nameuse,\@spaces,\@tempa,\@warning}
% \DoNotIndex{\ }
% \DoNotIndex{\begingroup,\bfseries}
% \DoNotIndex{\csname}
% \DoNotIndex{\def,\do,\docdate}
% \DoNotIndex{\endcsname,\endgroup,\endinput,\endtrivlist,\expandafter}
% \DoNotIndex{\filedate,\fileversion}
% \DoNotIndex{\gdef,\global}
% \DoNotIndex{\hbox,\hskip}
% \DoNotIndex{\ignorespaces,\input,\itshape,\item}
% \DoNotIndex{\labelsep,\let,\llap}
% \DoNotIndex{\makeatletter}
% \DoNotIndex{\noexpand}
% \DoNotIndex{\refstepcounter,\relax,\rlap}
% \DoNotIndex{\setlength,\slshape,\space,\string,\strut}
% \DoNotIndex{\the,\trivlist,\typeout}
% \DoNotIndex{\vbox}
% \DoNotIndex{\xdef}
%
% \def\deflabel#1{\ttfamily #1\hfill}
% \def\deflist#1{\begin{list}{}{\settowidth\labelwidth{\ttfamily #1}%
%                               \setlength\leftmargin\labelwidth
%                               \addtolength\leftmargin\labelsep
%                               \let\makelabel\deflabel}}
% \def\enddeflist{\end{list}}
% \def\envfont{\normalfont\ttfamily}
%
% \MakeShortVerb{\|}
%
% \newcommand{\TUB}{{\sl TUGboat\/}}
%
% \GetFileInfo{theorem.dtx}
% \title{An Extension of the
%        \LaTeX{} theorem environment\thanks{This file
%        has version number \fileversion, last
%        revised \filedate.}}
% \author{Frank Mittelbach}
%
% \MaintainedByLaTeXTeam{tools}
% \maketitle
%
% \begin{abstract}
%    The macros described in this paper yield an extension of the
%    \LaTeX{} theorem mechanism. It is designed to satisfy the
%    different requirements of various journals. Thus, the
%    layout of the ``theorems'' can be manipulated by determining a
%    ``style''. This article describes not only the use, but
%    also the definition, of the necessary macros.
% \end{abstract}
%
% \section*{Preface to version 2.2}
%
% For \LaTeXe{} this package did not need any fundamental changes. I
% only modified the messages generated so that theorem layout styles
% will show up with the |\listfiles| command and cleaned the section
% on the New Font Selection Scheme since this is now included in
% \LaTeX.
%
% \section*{Preface to version 2.1}
%
% This version is identical to 2.0g described in \TUB\ 10\#3 except
% for some internal defaults which are now set depending on the used
% font selection scheme.
%
% This was done to avoid unpleasant surprises if the new font selection
% scheme is in force. For further details see
% section~\ref{sec:newfonts} and \cite{bk:GMS94}.
%
% \section{Introduction}
%
% For our purposes here, ``theorems'' are labelled enunciations,
% often set off from the main text by extra space and a font change.
% Theorems, corollaries, conjectures, definitions, and remarks
% are all instances of ``theorems''.  The ``header'' of these structures
% is composed of a label (such as \textsc{Theorem} or \textsc{Remark})
% and a number which serializes an item in the sequence of items
% with the same label.
%
% Shortly after the introduction of \LaTeX{} at the Fachbereich
% Mathematik in Mainz, the desire to manipulate the layout of
% ``theorems'' arose.  In Mainz, the following two conventions came
% into general use:
% \begin{enumerate}
%   \item The number of the theorem is shown in the margin.
%   \item There is a line break at
%         the end of the theorem header.
% \end{enumerate}
%
% Additionally, some journals require different formats which depend on
% the ``sort of theorem'': e.g.\ often remarks and definitions are set
% in |\upshape|, while |\itshape| is employed for main theorems.
%
% Confronted with these requirements, a theorem environment was
% developed in Mainz which allows separate determination of the
% layout of the ``theorems sets'', comparable to |\pagestyle|.
%
%
% \section{The user interface}
% \subsection{Defining new theorem sets}
%
% \DescribeMacro\newtheorem
% As in the original \LaTeX{} version, the command |\newtheorem|
% defines a new ``theorem set'' or ``theorem-like structure''.
% Two required arguments name the new environment and give the text
% to be typeset with each instance of the new ``set'', while
% an optional argument determines how
% the ``set'' is enumerated:
% \begin{description}
%    \item[\ttfamily \bslash newtheorem\{foo\}\{bar\}]
%       The theorem set {\envfont foo} (whose name is \texttt{bar})
%       uses its own counter.
%    \item[\ttfamily \bslash newtheorem\{foo2\}{[foo]}\{bar2\}]
%       The theorem set {\envfont foo2} (printed name \texttt{bar2})
%       uses the same counter as the theorem set \texttt{foo}.
%    \item[\ttfamily \bslash newtheorem\{foo3\}\{bar3\}{[section]}]
%       The theorem set {\envfont foo3} (printed name \texttt{bar3}) is
%       enumerated within the counter \texttt{section}, i.e.\ with every
%       new |\section| the enumeration begins again with 1, and
%       the enumeration is composed from the section-number and the
%       theorem counter itself.
% \end{description}
%
% \DescribeMacro\theoremstyle
% Additionally, the command |\theoremstyle| can define the layout
% of various, or all, theorem sets.  It should be noted that any
% theorem set defined by |\newtheorem| is typeset in the
% |\theoremstyle| that is current at the time of the definition.
% Thus, the following
% \begin{verbatim}
%    \theoremstyle{break}        \newtheorem{Cor}{Corollary}
%    \theoremstyle{plain}        \newtheorem{Exa}{Example}[section]
%\end{verbatim}                             ^^A No blank at line start
% leads to the result that the set {\envfont Cor} is formatted in the
% style \texttt{break}, while the set {\envfont Exa} and all the
% following ones are formatted in the style \texttt{plain}, unless
% another |\theoremstyle| follows.  Since the definitions installed by
% |\newtheorem| are global, one also can limit |\theoremstyle| locally
% by grouping braces.
%
% \DescribeMacro\theorembodyfont
% The choice of the font for the theorem body is completely
% independent of the chosen |\theoremstyle|; this has proven to
% be very advantageous.  For example,
% \begin{verbatim}
%   {\theorembodyfont{\upshape}          \newtheorem{Rem}{Remark}}
%\end{verbatim}
% defines a theorem set {\envfont Rem}, which will be set in
% |\upshape| in the current layout (which in our example is
% \texttt{plain}).  As with |\theoremstyle|, the |\theorembodyfont|
% chosen is that current at the time of |\newtheorem|.  If
% |\theorembodyfont| is not specified or one defines
% |\theorembodyfont{}|, then the font used will be that defined by the
% |\theoremstyle|.
%
% \DescribeMacro\theoremheaderfont It is also possible to customize
% the font used for the theorem headers.  This is, however, a global
% declaration, and therefore there should be at most one
% |\theoremheaderfont| declaration in the preamble.\footnote{If it is
% actually necessary to have different header fonts, one has to define
% new theorem styles (substituting the desired font) or specify the
% information directly in the \texttt{\bslash newtheorem} declaration
% (the unclean variant).}
%
% \DescribeMacro\theorempreskipamount
% \DescribeMacro\theorempostskipamount Two additional parameters
% affect the vertical space around the theorem environments:
% |\theorempreskipamount| and |\theorempostskipamount| define,
% respectively, the spacing before and after such an environment.
% These parameters apply for all theorem sets and can be manipulated
% with the ordinary length macros.  They are rubber lengths
% (`\textsf{skips}'), and therefore can contain \texttt{plus} and
% \texttt{minus} parts.
%
% Since the definition of theorem sets should---most sensibly---be
% placed in the preamble, we only allow installation there.
% It is therefore possible to release the memory used here after
% |\begin{document}|, in order to make room for other
% applications.
%
%
% \subsection{Existing theorem styles}
%
% The following theorem styles exist to date:
% \begin{deflist}{marginbreak}
%    \item[plain]
%       This theorem style emulates the original \LaTeX{} definition,
%       except that additionally the parameters
%       |\theorem...skipamount| are used.
%    \item[break]
%       In this style, the theorem header is followed by a line
%       break.
%    \item[marginbreak]
%       The theorem number is set in the margin, and there is a line
%       break as in \texttt{break}.
%    \item[changebreak]
%       Like \texttt{break}, but with header number and text
%       interchanged.
%    \item[change]
%       Header number and text are interchanged, without a line break.
%    \item[margin]
%       The number is set in the left margin, without a line break.
% \end{deflist}
% All styles (except \texttt{plain}) select |\slshape| as the default
% |\theorembodyfont|.
%
%
% \subsection{Examples}
%
% Given the above theorem sets {\envfont Cor}, {\envfont Exa}
% and {\envfont Rem},
% suppose that the preamble also contains the declarations:
% \begin{verbatim}
%    \theoremstyle{marginbreak}   \newtheorem{Lem}[Cor]{Lemma}
%    \theoremstyle{change}
%    \theorembodyfont{\itshape}        \newtheorem{Def}[Cor]{Definition}
%
%    \theoremheaderfont{\scshape}
%\end{verbatim}
% Then the following are some typical examples of the typeset output
% resulting from their use.
%
% \begin{Cor}
%    This is a sentence typeset in the theorem environment {\envfont
%    Cor}.
% \end{Cor}
%
% \begin{Exa}
%    This is a sentence typeset in the theorem environment {\envfont
%    Exa}.
% \end{Exa}
%
% \begin{Rem}
%    This is a sentence typeset in the theorem environment {\envfont
%    Rem}.
% \end{Rem}
%
% \begin{Lem}[Ben User]
%    This is a sentence typeset in the theorem environment {\envfont
%    Lem}.
% \end{Lem}
%
% \begin{Def}[Very impressive Definition]
%    This is a sentence typeset in the theorem environment {\envfont
%    Def}.
% \end{Def}
%
% The last two examples show the effect of the optional argument to
% a theorem environment (it is the text typeset in parentheses).
%
%
% \section{Special Considerations}
% \label{sec:newfonts}
%
% Theoremheader and body are implemented as a unit. This means that
% the |\theoremheaderfont|  will inherit characteristics of the
% |\theorembodyfont| in \LaTeXe. Thus, if for example
% |\theorembodyfont| is |\itshape| and |\theoremheaderfont|
% is |\bfseries| the font selected for the header will have the
% characteristics `bold extended italic'. If this is not desired one
% should set the |\theoremheaderfont| to something like
% \begin{verbatim}
%      \theoremheaderfont{\normalfont\bfseries}
%\end{verbatim}
% i.e.\ supplying all necessary font information explicitly.
%
%
%
% \section{Acknowledgements}
%
% The publication of this set of macros was only possible with the
% help of Christina Busse (translating the manuscript into English),
% Joachim Pense (playing the r\^ole of typist), Chris Rowley
% (looking everything over) and many others providing useful
% suggestions.
%
% \MaybeStop{%
%    \begin{thebibliography}{9}
%       \bibitem{bk:GMS94} \textsc{M.~Goossens}, \textsc{F.~Mittelbach}
%         and \textsc{A.~Samarin}.
%         \newblock The \LaTeX{} Companion.
%         \newblock
%         Addison-Wesley, Reading, Massachusetts, 1994.
%       \bibitem{latex:source} \textsc{Lamport, Leslie.} \newblock
%                               \texttt{latex.tex}, version 2.09, date
%                               Feb. 1990.
%    \end{thebibliography}
%    \PrintIndex
%    \PrintChanges}
%
%
% \section{The documentation driver file}
%
% The next bit of code contains the documentation driver file for
% \TeX{}, i.e., the file that will produce the documentation you are
% currently reading. It will be extracted from this file by the
% \texttt{docstrip} program. Since it is the first code in the file
% one can alternatively process this file directly with \LaTeXe{} to
% obtain the documentation.
%
%    \begin{macrocode}
%<*driver>
\documentclass{ltxdoc}

\usepackage{theorem}

%<+driver>% The next few lines define theorem sets which are used
%<+driver>% in the example section of the documentation.

\theoremstyle{break}       \newtheorem{Cor}{Corollary}
\theoremstyle{plain}       \newtheorem{Exa}{Example}[section]
{\theorembodyfont{\upshape}\newtheorem{Rem}{Remark}}
\theoremstyle{marginbreak} \newtheorem{Lem}[Cor]{Lemma}
\theoremstyle{change}
\theorembodyfont{\itshape} \newtheorem{Def}[Cor]{Definition}

\theoremheaderfont{\scshape}

\RecordChanges

\begin{document}
   \DocInput{theorem.dtx}
\end{document}
%</driver>
%    \end{macrocode}
%
% \section{Definition of the Macros}
%
% If the file has been loaded before, we abort immediately. If not the
% package announces itself (this is actually done at the very top of
% the file---the way it is done isn't good style so don't copy it).
% \changes{v2.0e}{89/07/19}{Spaces removed from `typeout'.}
% \changes{v2.2b}{95/11/19}{Announce moved to top of file}
%    \begin{macrocode}
%<*package>
%\@ifundefined{theorem@style}{}{\endinput}
%\def\FMithmInfo{1995/11/19 v2.2b Theorem extension package (FMi)}
%\ProvidesPackage{theorem}[\FMithmInfo]
%    \end{macrocode}
%
%
% \subsection{Definition of theorem styles and fonts}
%
% All the definitions in this file are done globally to allow
% inputting this file inside a group.
%
% \begin{macro}{\theoremstyle}
%    Before a theorem style can be installed, the chosen style must be
%    known. For that reason, we must test to see that
%    |\th@|\meta{style} is known or, more precisely, that it is
%    different from |\relax|. If the style is not known then
%    |\th@plain| is used.
%    \begin{macrocode}
\gdef\theoremstyle#1{%
   \@ifundefined{th@#1}{\@warning
          {Unknown theoremstyle `#1'. Using `plain'}%
          \theorem@style{plain}}%
%    \end{macrocode}
%    We save the theorem style to be used in the token register
%    |\theorem@style|.
%    \begin{macrocode}
      {\theorem@style{#1}}%
%    \end{macrocode}
%    Now we ``evaluate'' the theorem style: this means, we call the
%    macro |\th@|\meta{style} which will activate the relevant
%    definitions which are contained in a separate file.  This is done
%    in a group to suppress changes to the current font.  This could
%    otherwise pose problems together with the new font selection
%    scheme\footnote{When I printed the original article using the new
%    font selection scheme I ended with a document with slanted
%    typefaces (text headings and all) simply because one of the
%    theorem styles used \texttt{\bslash sl} at toplevel.} if the
%    |\th@|\meta{style} is evaluated a second time.
% \changes{v2.1a}{90/02/15}{Group added around `th@..}
% \changes{v2.1a}{90/02/15}{Reseting of `..@begintheorem removed.}
%    \begin{macrocode}
      \begingroup
        \csname th@\the\theorem@style \endcsname
      \endgroup}
%    \end{macrocode}
% \end{macro}
%
%
% \begin{macro}{\@begintheorem}
% \changes{v2.1a}{90/02/15}{Relaxed at toplevel.}
% \begin{macro}{\@opargbegintheorem}
% \changes{v2.1a}{90/02/15}{Relaxed at toplevel.}
%    We reset |\@begintheorem| and |\@opargbegintheorem| to
%    |\relax| since these commands are no longer necessary at
%    toplevel. This will save a few tokens.
%    \begin{macrocode}
\global\let\@begintheorem\relax
\global\let\@opargbegintheorem\relax
%    \end{macrocode}
% \end{macro}
% \end{macro}
%
% \begin{macro}{\theorem@style}
%    Obviously the \textsf{token} register used above has to be
%    allocated. To assure the utmost compatibility with the
%    original \LaTeX{} definition, we set the default theorem style to
%    \texttt{plain}, which implements the usual \LaTeX{} convention.
% \changes{v2.0f}{89/07/20}{Unnecessary `global' removed.}
%    \begin{macrocode}
\newtoks\theorem@style
\global\theorem@style{plain}
%    \end{macrocode}
% \end{macro}
%
% \begin{macro}{\theorembodyfont}
% \begin{macro}{\theorem@bodyfont}
%    If the |\theorembodyfont| is set by the user then it should not
%    interact with the default font set in the theorem style. When the
%    new font selection is in force this may happen if, for example,
%    the default is |\itshape| and the new |\theorembodyfont| is
%    |\sffamily|. So we add a |\reset@font| command in front of the
%    user definition.
% \changes{v2.0f}{89/07/20}{Unnecessary `global' removed.}
% \changes{v2.0g}{89/09/19}{`theoremfont to `theorembodyfont.}
% \changes{v2.1c}{92/06/22}{`reset@font added and logic changed.}
% \changes{v2.1d}{92/10/28}{Reset to empty if arg is empty}
%    \begin{macrocode}
\gdef\theorembodyfont#1{%
%    \end{macrocode}
%    We check if the argument supplied is empty and if so put nothing
%    into the |\theorem@bodyfont| token register to allow for
%    |\theorembodyfont{}| as a mean of using the default of the
%    current |\theoremstyle|.
%    \begin{macrocode}
   \def\@tempa{#1}%
   \ifx\@tempa\@empty
    \theorem@bodyfont{}%
   \else
    \theorem@bodyfont{\reset@font#1}%
   \fi
   }
\newtoks\theorem@bodyfont
\global\theorem@bodyfont{}
%    \end{macrocode}
% \end{macro}
% \end{macro}
%
%
% \begin{macro}{\theoremheaderfont}
% \changes{v2.0g}{89/09/19}{Macro added.}
%    The font for the theorem headers is handled differently because
%    this definition applies to all theorem styles.
%    \begin{macrocode}
\gdef\theoremheaderfont#1{\gdef\theorem@headerfont{#1}%
%    \end{macrocode}
%    After using the macro once it is redefined to produce an error
%    message.
%    \begin{macrocode}
       \gdef\theoremheaderfont##1{%
        \typeout{\string\theoremheaderfont\space should be used
                 only once.}}}
%    \end{macrocode}
% \end{macro}
%
% \begin{macro}{\theorem@headerfont}
% \changes{v2.0g}{89/09/19}{Macro added.}
%    To set the |\theorem@headerfont| default we first test if
%    the new fontselection scheme is in force.
% \changes{v2.1a}{90/02/15}{Support for new font selection.}
%    \begin{macrocode}
\ifx\upshape\undefined
%    \end{macrocode}
%    If not we define it to expand into |\bfseries|. We don't use
%    |\let| just in case a following style option redefines this
%    macro.
%    \begin{macrocode}
\gdef\theorem@headerfont{\bfseries}
%    \end{macrocode}
%    Otherwise we reset the current shape before calling |\bfseries|.
%    \begin{macrocode}
\else \gdef\theorem@headerfont{\normalfont\bfseries}\fi
%    \end{macrocode}
% \end{macro}
%
% \begin{macro}{\th@plain}
% \begin{macro}{\th@break}
% \begin{macro}{\th@marginbreak}
% \begin{macro}{\th@changebreak}
% \begin{macro}{\th@change}
% \begin{macro}{\th@margin}
%    The different styles are defined in macros such as
%    |\th@plain|.  Since memory space is precious in
%    ``non-Big-versions'', we have to avoid offering too many unused
%    definitions.  Therefore we define these styles in separate files
%    that can be loaded on demand.  Thus the commands themselves only
%    load these files. We use |\@input@|, a \LaTeXe{} internal command
%    that ensures that the file will be listed with |\listfiles|.
% \changes{v2.2a}{94/02/02}{Use `@input@ to load theorem layout files}
%    \begin{macrocode}
\gdef\th@plain{\@input@{thp.sty}}
\gdef\th@break{\@input@{thb.sty}}
\gdef\th@marginbreak{\@input@{thmb.sty}}
\gdef\th@changebreak{\@input@{thcb.sty}}
\gdef\th@change{\@input@{thc.sty}}
\gdef\th@margin{\@input@{thm.sty}}
%    \end{macrocode}
%    This list will be expanded when new styles become
%    available.  For testing,  just append new theorem substyles
%    as document options.
% \end{macro}
% \end{macro}
% \end{macro}
% \end{macro}
% \end{macro}
% \end{macro}
%
%
%
% \subsection{Definition of a new theorem set}
%
% As already pointed out, a new theorem environment can be defined in
% three different ways:
% \begin{verbatim}
%    \newtheorem{Lem}{Lemma}
%    \newtheorem{Lem}{Lemma}[section]
%    \newtheorem{Lem}[Theorem]{Lemma}
%\end{verbatim}
% The function of the macro |\newtheorem| is to recognize these cases
% and then to branch into one of the three macros |\@ynthm|, |\@xnthm|
% or |\@othm|.  This mechanism is adopted unchanged from
% \cite{latex:source}; the essential point here is that, for example,
% in the second case, the arguments \texttt{Lem}, \texttt{Lemma} and
% \texttt{section} are passed over to the macro |\@xnthm|.
%
% We inspect this case first because the others present fewer problems,
% and thus are easily derived from this one.
%
% \begin{macro}{\@xnthm}
%    For our example arguments, the macro |\@xnthm| must fulfill
%    the following:
%    \begin{itemize}
%       \item
%          Define a new \LaTeX-counter `Lem'
%       \item
%          reset this counter within a |\section|
%       \item
%          define the macro |\theLem|
%       \item
%          define the environment macros |\Lem| and
%          |\endLem| using the current
%          |\theoremstyle| and |\theorem@bodyfont|.
%    \end{itemize}
%    Obviously, all this should happen only if the first argument
%    of |\@xnthm| (i.e.\ \texttt{Lem} in our example) is chosen so
%    as not to conflict with any previously defined commands or
%    environments.  This test is performed by the \LaTeX{} macro
%    |\@ifdefinable|.
%    \begin{macrocode}
\gdef\@xnthm#1#2[#3]{\expandafter\@ifdefinable\csname #1\endcsname
%    \end{macrocode}
%    Therefore, the first argument of |\@ifdefinable| is the
%    expansion (in the example, |\Lem|) of
%    |\csname||#1\endcsname|.  The second argument is
%    executed only if the test has been completed successfully.
%    \begin{macrocode}
   {%
%    \end{macrocode}
%    Now we define the new counter. The names of the \LaTeX{}
%    macros employed should speak for themselves:
% \changes{v2.2b}{1995/10/07}{Use \cs{@newctr}}
% \changes{v2.2c}{1995/11/23}{Use \cs{@newctr} properly.}
%    \begin{macrocode}
    \@definecounter{#1}\@newctr{#1}[#3]%
%    \end{macrocode}
%    Using |\@newctr| will give a proper error message if the counter
%    in |#3| is not defined.
%    In defining `|\theLem|' we must generate the desired macro
%    name by use of |\expandafter| and |\csname|.
%    \begin{macrocode}
    \expandafter\xdef\csname the#1\endcsname
%    \end{macrocode}
%    An |\xdef| is used in order to make the definition global,
%    and to ensure that it contains the replacement texts of
%    |\@thmcountersep| and |\@thmcounter|.\footnote{These
%    two macros can be defined by the document style.  Their default
%    values produce a `\texttt{.}' as separation and an arabic
%    representation of the number.} However, not everything should be
%    expanded. For example, it saves space to use |\thesection|
%    instead of its---at times---lengthy expansion.
%    \begin{macrocode}
      {\expandafter \noexpand \csname the#3\endcsname
       \@thmcountersep \@thmcounter{#1}}%
%    \end{macrocode}
%    Thus with the defaults of \LaTeX{}, |\theLem| would be
%    replaced by the command sequence |\thesection.\arabic{Lem}|.
%
%    We will now look at the definition of the macro which is executed
%    at the beginning of the actual environment (in our example this
%    macro is |\Lem|). It should be noted that we use an
%    ``|\expandafter| trick'' to expand only certain parts of the
%    replacement text at the time of the definition.
%    \begin{macrocode}
    \def\@tempa{\global\@namedef{#1}}%
    \expandafter \@tempa \expandafter{%
%    \end{macrocode}
%    First, the macro that contains the current definitions of
%    |\@begintheorem| and |\@opargtheorem| should be called
%    up.  The name of this macro---as is already known---has the form
%    |\th@|\meta{theorem\ style}; therefore, it must be called by
%    \begin{macrocode}
      \csname th@\the \theorem@style
%    \end{macrocode}
%    In addition the default theorem font should be changeable, i.e.\
%    we have to insert the contents of |\theorem@bodyfont|. For
%    that reason, we expand even further, beyond |\endcsname|,
%    and thus insert the contents of the \textsf{token} register
%    |\theorem@bodyfont| in the replacement text.
%    \begin{macrocode}
            \expandafter \endcsname \the \theorem@bodyfont
%    \end{macrocode}
%    Now it is time to call the macro |\@thm| which takes over
%    the further processing.  It has two arguments: the current
%    counter name (in our example, \texttt{Lem}), and the text of the
%    label (in our example, \texttt{Lemma}).
%    \begin{macrocode}
     \@thm{#1}{#2}}%
%    \end{macrocode}
%    With this, the `sub-definition' is complete.  The macro
%    |\@endtheorem| ends a theorem environment and is, so far, nothing
%    but an |\endtrivlist|.  (Hence it is defined globally, and not
%    within the theorem styles.\footnote{This has to be changed as
%    soon as theorem styles that change \texttt{\bslash @endtheorem}
%    exist.  In such a case, all existing styles must be changed as
%    well since they will have to reset the macro.}) Therefore, we can
%    set it equivalent to the macro that ends the theorem set (in our
%    example, |\endLem|).  However, if some day theorem styles exist
%    that do change |\@endtheorem|, we would have to use the
%    commented-out line instead.
%    \begin{macrocode}
    \global \expandafter \let \csname end#1\endcsname \@endtheorem
%   \global\@namedef{end#1}{\@endtheorem}%
%    \end{macrocode}
%    With these commands all the required definitions are employed,
%    unless the test |\@ifdefinable| has failed.  Therefore, we
%    end the second argument of this macro and with it the definition
%    of |\@xnthm|.
%    \begin{macrocode}
   }}
%    \end{macrocode}
% \end{macro}
%
% \begin{macro}{\@ynthm}
%    The definition of |\@ynthm| is completely analogous. In
%    this case the new counter that is defined is not reset within
%    another counter; thus the definition of |\the...|
%    is simplified:
%    \begin{macrocode}
\gdef\@ynthm#1#2{\expandafter\@ifdefinable\csname #1\endcsname
   {\@definecounter{#1}%
    \expandafter\xdef\csname the#1\endcsname{\@thmcounter{#1}}%
%    \end{macrocode}
%    The rest of the definition corresponds literally to that of
%    |\@xnthm|:
%    \begin{macrocode}
    \def\@tempa{\global\@namedef{#1}}\expandafter \@tempa
     \expandafter{\csname th@\the \theorem@style \expandafter
     \endcsname \the\theorem@bodyfont \@thm{#1}{#2}}%
    \global \expandafter \let \csname end#1\endcsname \@endtheorem}}
%    \end{macrocode}
% \end{macro}
%
% \begin{macro}{\@othm}
%    The definition of |\@othm| does not contain anything new.
%    \begin{macrocode}
\gdef\@othm#1[#2]#3{%
%    \end{macrocode}
% \changes{v2.2b}{95/10/07}{Build in check for counter defined}
%    We do not define a new counter but instead use one that has
%    already been defined.  Thus the only definition we need is that
%    of this pseudo-counter (i.e.\ |\the|\meta{env.\ name}).
%    First we check if |#2| corresponds to a known counter name.
%    \begin{macrocode}
  \expandafter\ifx\csname c@#2\endcsname\relax
   \@nocounterr{#2}%
  \else
   \expandafter\@ifdefinable\csname #1\endcsname
   {\expandafter \xdef \csname the#1\endcsname
     {\expandafter \noexpand \csname the#2\endcsname}%
%    \end{macrocode}
%    All other parts of the definition can be adopted from
%    |\@xnthm|. We have to remember, though, that in this case
%    the name of the current counter and the theorem label have moved
%    to the second and third arguments.
%    \begin{macrocode}
    \def\@tempa{\global\@namedef{#1}}\expandafter \@tempa
     \expandafter{\csname th@\the \theorem@style \expandafter
     \endcsname \the\theorem@bodyfont \@thm{#2}{#3}}%
    \global \expandafter \let \csname end#1\endcsname \@endtheorem}%
  \fi}
%    \end{macrocode}
% \end{macro}
%
% \subsection{Macros that are employed in a {\envfont theorem}
%    environment}
%
% \begin{macro}{\@thm}
%    The macro |\@thm| has to increase the current counter.  Then,
%    depending on whether the environment has (or does not have) an
%    optional argument, it has to branch into either |\@begintheorem|
%    or |\@opargtheorem|.
%    \begin{macrocode}
\gdef\@thm#1#2{\refstepcounter{#1}%
%    \end{macrocode}
%    Now we start a {\envfont trivlist} environment, and give
%    |\@topsep| and |\@topsepadd| the values of the \textsf{skip}
%    registers |\theorempreskipamount| and |\theorempostskipamount|.
%    The value in |\@topsep| is the vertical space that is inserted by
%    the first (and only) |\item| in our |\trivlist| whilst
%    |\@topsepadd| is inserted by |\@endparenv| at the end of that
%    {\envfont trivlist} environment.  By using these registers, we
%    obtain the desired space around a {\envfont theorem} environment.
%    \begin{macrocode}
   \trivlist
   \@topsep \theorempreskipamount               % used by first \item
   \@topsepadd \theorempostskipamount           % used by \@endparenv
%    \end{macrocode}
%    Now we have to test whether an optional argument has been given.
% \changes{v2.0d}{89/07/18}{Forgotten percent added.}
%    \begin{macrocode}
   \@ifnextchar [%
%    \end{macrocode}
%    If there is an optional argument, we will call |\@ythm|, and
%    move the arguments read back into the input stream.
%    \begin{macrocode}
   {\@ythm{#1}{#2}}%
%    \end{macrocode}
%    If not, we call |\@begintheorem|. Its first argument is the
%    name of the theorem set (hence the second argument of
%    |\@thm|). Its second argument is the macro that produces the
%    current number.
%    \begin{macrocode}
   {\@begintheorem{#2}{\csname the#1\endcsname}\ignorespaces}}
%    \end{macrocode}
% \end{macro}
%
% \begin{macro}{\@xthm}
% \begin{macro}{\@ythm}
%    Both these macros were originally called by |\@thm|.  We do
%    not need |\@xthm| anymore, hence we reset it to
%    |\relax|. The definition of |\@ythm| has not changed at
%    all from its definition in \LaTeX. In order to make the macros
%    easier to understand, we will
%    nevertheless present it (commented out).
%    \begin{macrocode}
\global\let\@xthm\relax
% \def\@ythm#1#2[#3]{\@opargbegintheorem{#2}{\csname
%           the#1\endcsname}{#3}\ignorespaces}
%</package>
%    \end{macrocode}
%    The primitive |\ignorespaces| in |\@ythm| and
%    |\@thm| is needed to remove the spaces between the
%    |\begin{...}| and the actual text.
% \end{macro}
% \end{macro}
%
%
% \subsection{Definition of the theorem substyles}
%
% As already pointed out, the theorem substyles, defined below, are
% only loaded when necessary.  Note that all these substyles, except
% \texttt{plain}, have |\slshape| as the default body font.
%
% \subsubsection{The \texttt{plain} style}
%
% \label{dok:th@plain} As the following macros use |@|, we have
% to locally set the |\catcode| of this symbol to ``letter''.
% This happens within a group, so that we do not have to worry about
% which |\catcode| that symbol had before.
%    \begin{macrocode}
%<*thp>
\begingroup \makeatletter
%    \end{macrocode}
% Since we are now within a group, we must make all definitions
% globally.  First we make sure that \texttt{theorem.sty} is loaded.
% This will allow us to use this file as a document style option
% without having to call \texttt{theorem} itself as an option.  At the
% same time, we assure that at least version 2 is loaded, since
% |\theorem@style| was not defined in earlier versions.
%    \begin{macrocode}
\@ifundefined{theorem@style}{\input{theorem.sty}}{}
\ProvidesFile{thp.sty}
             [\FMithmInfo]
%    \end{macrocode}
%
%
% \begin{macro}{\th@plain}
% \begin{macro}{\@begintheorem}
% \begin{macro}{\@opargbegintheorem}
%    |\theoremstyle{plain}| corresponds to the original
%    definition, except that the distances to the surrounding text are
%    determined by the parameters |\theorempreskipamount| and
%    |\theorempostskipamount|.
%    First we set the default body font.
%    \begin{macrocode}
\gdef\th@plain{\normalfont\itshape
%    \end{macrocode}
%    Then we define |\@begintheorem| and
%    |\@opargbegintheorem|. These two macros define how the
%    header of a theorem is typeset.  |\@opargbegintheorem| will
%    be called if a \textsf{theorem} environment with an optional
%    argument is encountered; otherwise, the header is constructed by
%    calling |\@begintheorem|.  If one of these macros is
%    executed, we are within a \textsf{trivlist} environment started by
%    |\@thm|.  So the theorem header is produced with an
%    |\item| command.
%
%    Instead of specifying the header font directly, all standard
%    theorem styles use the |\theorem@headerfont| macro to allow
%    customization.  The extra space (|\labelsep|) is necessary
%    because of problems in the \textsf{trivlist} environment.
%    \begin{macrocode}
  \def\@begintheorem##1##2{%
        \item[\hskip\labelsep \theorem@headerfont ##1\ ##2]}%
%    \end{macrocode}
%    The definition of |\@opargbegintheorem| is completely
%    analogous. The only difference is the fact that there exists a
%    third argument (which is the optional parameter of the
%    environment and contains additional information about the
%    theorem).  Customarily we enclose it in parentheses.
%    \begin{macrocode}
\def\@opargbegintheorem##1##2##3{%
   \item[\hskip\labelsep \theorem@headerfont ##1\ ##2\ (##3)]}}
%    \end{macrocode}
% \end{macro}
% \end{macro}
% \end{macro}
%
% \noindent
% We conclude with an |\endgroup| to restore the
% |\catcode| of |@|.
%    \begin{macrocode}
\endgroup
%</thp>
%    \end{macrocode}
%
%
% \subsubsection{The \texttt{break} style}
%
% This style option is stored in the file |thb.sty|.
% For the next two lines see the documentation for |\th@plain| on
% page~\pageref{dok:th@plain}.
%    \begin{macrocode}
%<*thb>
\begingroup \makeatletter
\@ifundefined{theorem@style}{\input{theorem.sty}}{}
\ProvidesFile{thb.sty}
             [\FMithmInfo]
%    \end{macrocode}
%
% \begin{macro}{\th@break}
%    |\theoremstyle{break}| produces a line break after the name
%    of the theorem.  The font is |\slshape|.  Hence, we define
%    |\th@break| as follows:
%    \begin{macrocode}
\gdef\th@break{\normalfont\slshape
  \def\@begintheorem##1##2{\item[%
%    \end{macrocode}
%    We run into the following problem: it is not possible to create
%    the header with |\item[|\meta{title}|]| and then start
%    a new line by, for example, |\mbox{}\\|. Such a definition
%    will fail whenever a list environment follows immediately.  With
%    the above construction, the |\mbox{}| causes the switch
%    |@inlabel| (cf.\ definition of |\list| and
%    |\trivlist| in \cite{latex:source}) to be set to \texttt{false}
%    and so the following list will insert additional vertical
%    space (|\topskip|). This is quite annoying. Therefore, we
%    create the line break within the |\item|. In order to ensure
%    that the text will begin at the proper position in the following
%    line, we simply pretend that the label does not take any
%    room.\footnote{This will lead to problems whenever very high
%    symbols occurring in the line tower into the
%    heading. So, something else has to be done here sometime.}
%    \begin{macrocode}
       \rlap{\vbox{\hbox{\hskip \labelsep\theorem@headerfont ##1\ ##2}%
                   \hbox{\strut}}}]}%
%    \end{macrocode}
%    Again, the definition of |\@opargbegintheorem| is
%    completely analogous.
%    \begin{macrocode}
\def\@opargbegintheorem##1##2##3{%
  \item[\rlap{\vbox{\hbox{\hskip \labelsep \theorem@headerfont
                     ##1\ ##2\ (##3)}%
                    \hbox{\strut}}}]}}
\endgroup
%</thb>
%    \end{macrocode}
% \end{macro}
%
%
% \subsubsection{The \texttt{changebreak} style}
%
%    \begin{macrocode}
% This style option is stored in the file |thcb.sty|.
%    \begin{macrocode}
%<*thcb>
\begingroup \makeatletter
\@ifundefined{theorem@style}{\input{theorem.sty}}{}
\ProvidesFile{thcb.sty}
             [\FMithmInfo]
%    \end{macrocode}
%
% \begin{macro}{\th@changebreak}
%    The \texttt{change-break} theorem style is like \texttt{break}
%    but with interchange of theorem name and theorem number.  Thus we
%    define |\th@changebreak| as follows:
%    \begin{macrocode}
\gdef\th@changebreak{\normalfont\slshape
  \def\@begintheorem##1##2{\item
       [\rlap{\vbox{\hbox{\hskip\labelsep \theorem@headerfont ##2\ ##1}%
                    \hbox{\strut}}}]}%
\def\@opargbegintheorem##1##2##3{%
  \item[\rlap{\vbox{\hbox{\hskip\labelsep \theorem@headerfont
                     ##2\ ##1\ (##3)}%
                    \hbox{\strut}}}]}}
\endgroup
%</thcb>
%    \end{macrocode}
% \end{macro}
%
%
% \subsubsection{The \texttt{change} style}
%
% This style option is stored in the file |thc.sty|.
%    \begin{macrocode}
%<*thc>
\begingroup \makeatletter
\@ifundefined{theorem@style}{\input{theorem.sty}}{}
\ProvidesFile{thc.sty}
             [\FMithmInfo]
%    \end{macrocode}
%
% \begin{macro}{\th@change}
%    The \texttt{change} theorem style corresponds to the
%    \texttt{change break} style without a linebreak after the header.
%    To say it in another way, it's the same as the \texttt{plain}
%    style but with number and name interchanged and |\slshape| as the
%    default font.
%    \begin{macrocode}
\gdef\th@change{\normalfont\slshape
  \def\@begintheorem##1##2{\item
       [\hskip\labelsep \theorem@headerfont ##2\ ##1]}%
\def\@opargbegintheorem##1##2##3{%
  \item[\hskip\labelsep \theorem@headerfont ##2\ ##1\ (##3)]}}
\endgroup
%</thc>
%    \end{macrocode}
% \end{macro}
%
%
% \subsubsection{The \texttt{marginbreak} style}
%
% This style option is the one used most often at Mainz.  It is saved
% in the file |thmb.sty|.
%    \begin{macrocode}
%<*thmb>
\begingroup \makeatletter
\@ifundefined{theorem@style}{\input{theorem.sty}}{}
\ProvidesFile{thmb.sty}
             [\FMithmInfo]
%    \end{macrocode}
%
% \begin{macro}{\th@marginbreak}
%    The \texttt{margin break} style is nearly the same as the
%    \texttt{change break} style. The only difference is the placement
%    of the theorem number.  We use |\llap| to place it in the left
%    margin.
%
%    In this style |\labelsep| denotes the separation between the
%    number and the text.
%    \begin{macrocode}
\gdef\th@marginbreak{\normalfont\slshape
  \def\@begintheorem##1##2{\item
       [\rlap{\vbox{\theorem@headerfont
                    \hbox{\llap{##2}\hskip\labelsep ##1}%
                    \hbox{\strut}}}]}%
\def\@opargbegintheorem##1##2##3{%
  \item[\rlap{\vbox{\theorem@headerfont
                    \hbox{\llap{##2}\hskip\labelsep ##1\ (##3)}%
                    \hbox{\strut}}}]}}
\endgroup
%</thmb>
%    \end{macrocode}
% \changes{v2.1c}{90/02/17}{`endgroup outside macrocode env.}
% \end{macro}
%
%
%
% \subsubsection{The \texttt{margin} style}
%
% This style option is stored in the file |thm.sty|.
%    \begin{macrocode}
%<*thm>
\begingroup \makeatletter
\@ifundefined{theorem@style}{\input{theorem.sty}}{}
\ProvidesFile{thm.sty}
             [\FMithmInfo]
%    \end{macrocode}
%
% \begin{macro}{\th@margin}
%    Again this is only a variant of the theorem styles described
%    above without any new ideas.
%    \begin{macrocode}
\gdef\th@margin{\normalfont\slshape
  \def\@begintheorem##1##2{\item
       [\theorem@headerfont \llap{##2}\hskip\labelsep ##1]}%
\def\@opargbegintheorem##1##2##3{%
  \item[\theorem@headerfont \llap{##2}\hskip\labelsep ##1\ (##3)]}}
\endgroup
%</thm>
%    \end{macrocode}
% \end{macro}
%
%
%
%
% \subsection{Final Definitions}
%
% \begin{macro}{\theorempreskipamount}
% \begin{macro}{\theorempostskipamount}
%    The \textsf{skip} parameters that regulate the vertical empty space
%    before and after the {\envfont theorem} environment have
%    to be allocated as well.
% \changes{v2.0f}{89/07/20}{Unnecessary `global' removed.}
%    \begin{macrocode}
%<*package>
\newskip\theorempreskipamount
\newskip\theorempostskipamount
%    \end{macrocode}
%    Since we have used the same values for all theorem sets, we now can
%    assign them.
%    \begin{macrocode}
\global\setlength\theorempreskipamount{12pt plus 5pt minus 3pt}
\global\setlength\theorempostskipamount{8pt plus 3pt minus 1.5pt}
%    \end{macrocode}
% \end{macro}
% \end{macro}
%
%
% \begin{macro}{\@endtheorem}
%    The same holds for the macro |\@endtheorem|, which ends a
%    {\envfont theorem} environment. Since it is the same for all
%    theorem sets, it is removed from the macros |\th@|\meta{style}.
%    It simply ends the {\envfont trivlist} environment, which was
%    begun in |\@thm|.
%    \begin{macrocode}
\global\let\@endtheorem=\endtrivlist
%    \end{macrocode}
% \end{macro}
%
%
%
% \begin{macro}{\@preamblecmds}
%    All macros defined above are to be used only in the preamble.
%    Therefore, we insert them in |\@preamblecmds| which will disable
%    them at begin document. This is done by the internal \LaTeXe{}
%    command |\@onlypreamble|.
% \changes{v2.0g}{89/09/19}{`theoremheaderfont and `theorembodyfont
%    added.}
% \changes{v2.2a}{94/02/03}{Use \LaTeXe{} interface}
%    \begin{macrocode}
\@onlypreamble\@xnthm
\@onlypreamble\@ynthm
\@onlypreamble\@othm
\@onlypreamble\newtheorem
\@onlypreamble\theoremstyle
\@onlypreamble\theorembodyfont
\@onlypreamble\theoremheaderfont
%    \end{macrocode}
% \end{macro}
%
% Finally we declare the \texttt{plain} theorem style to be the default.
% \changes{v2.1b}{90/06/04}{Fixed missing `item by declaring default
%    style.}
%    \begin{macrocode}
\theoremstyle{plain}
%</package>
%    \end{macrocode}
%
% \Finale

\endinput
}{}
\ProvidesFile{thmb.sty}
             [\FMithmInfo]
%    \end{macrocode}
%
% \begin{macro}{\th@marginbreak}
%    The \texttt{margin break} style is nearly the same as the
%    \texttt{change break} style. The only difference is the placement
%    of the theorem number.  We use |\llap| to place it in the left
%    margin.
%
%    In this style |\labelsep| denotes the separation between the
%    number and the text.
%    \begin{macrocode}
\gdef\th@marginbreak{\normalfont\slshape
  \def\@begintheorem##1##2{\item
       [\rlap{\vbox{\theorem@headerfont
                    \hbox{\llap{##2}\hskip\labelsep ##1}%
                    \hbox{\strut}}}]}%
\def\@opargbegintheorem##1##2##3{%
  \item[\rlap{\vbox{\theorem@headerfont
                    \hbox{\llap{##2}\hskip\labelsep ##1\ (##3)}%
                    \hbox{\strut}}}]}}
\endgroup
%</thmb>
%    \end{macrocode}
% \changes{v2.1c}{90/02/17}{`endgroup outside macrocode env.}
% \end{macro}
%
%
%
% \subsubsection{The \texttt{margin} style}
%
% This style option is stored in the file |thm.sty|.
%    \begin{macrocode}
%<*thm>
\begingroup \makeatletter
\@ifundefined{theorem@style}{% \iffalse meta-comment
%
% Copyright (C) 1993-2024
%
% The LaTeX Project and any individual authors listed elsewhere
% in this file.
%
% This file is part of the Standard LaTeX `Tools Bundle'.
% -------------------------------------------------------
%
% It may be distributed and/or modified under the
% conditions of the LaTeX Project Public License, either version 1.3c
% of this license or (at your option) any later version.
% The latest version of this license is in
%    https://www.latex-project.org/lppl.txt
% and version 1.3c or later is part of all distributions of LaTeX
% version 2005/12/01 or later.
%
% The list of all files belonging to the LaTeX `Tools Bundle' is
% given in the file `manifest.txt'.
%
% \fi
%
%%
%
%
\def\FMithmInfo{2023/07/05 v2.2c Theorem extension package (FMi)}
%
% \ProvidesFile{theorem.dtx}[\FMithmInfo]
%\iffalse   % this is a METACOMMENT !
%%
%% Package `theorem' to use with LaTeX2e
%% Copyright (C) 1989-1995 Frank Mittelbach, all rights reserved.
%<+package>\@ifundefined{theorem@style}{}{\endinput}
%<+package>\NeedsTeXFormat{LaTeX2e}[1995/06/01]
%<+package>\ProvidesPackage{theorem}
%<+package>         [\FMithmInfo]
%
% \fi
%
% \changes{v1.0}{87/05/07}{First draft.}
% \changes{v1.0a}{87/05/07}{`thistheoremstyle' added.}
% \changes{v1.1a}{87/07/29}{New concept: redefine the internal
%                          \protect\LaTeX{} commands `@xnthm' etc.}
% \changes{v1.1b}{87/07/31}{`theoremfont' used to specify font.}
% \changes{v1.2a}{88/01/08}{`@endtheorem' now globally defined.}
% \changes{v1.2b}{88/01/11}{`newtheorem' only allowed in the preamble.}
% \changes{v1.3a}{88/02/14}{`theoremfont' now a toks register.}
%
% \changes{v2.0a}{88/11/16}{New implementation and new documentation.}
% \changes{v2.0a}{88/11/16}{No layout changes but internal modifications
%                           old substyles won't work.}
% \changes{v2.0b}{89/01/30}{`global' used for all definitions.}
% \changes{v2.0c}{89/06/15}{Documentation translated to English.
%                            Thanks to LLL and Nina}
% \changes{v2.1a}{90/02/15}{Support for new font selection.}
% \changes{v2.2a}{94/02/03}{Cleanup for LaTeX2e}
%
% \DoNotIndex{\@endparenv,\@ifdefinable,\@ifnextchar,\@ifundefined}
% \DoNotIndex{\@namedef,\@nameuse,\@spaces,\@tempa,\@warning}
% \DoNotIndex{\ }
% \DoNotIndex{\begingroup,\bfseries}
% \DoNotIndex{\csname}
% \DoNotIndex{\def,\do,\docdate}
% \DoNotIndex{\endcsname,\endgroup,\endinput,\endtrivlist,\expandafter}
% \DoNotIndex{\filedate,\fileversion}
% \DoNotIndex{\gdef,\global}
% \DoNotIndex{\hbox,\hskip}
% \DoNotIndex{\ignorespaces,\input,\itshape,\item}
% \DoNotIndex{\labelsep,\let,\llap}
% \DoNotIndex{\makeatletter}
% \DoNotIndex{\noexpand}
% \DoNotIndex{\refstepcounter,\relax,\rlap}
% \DoNotIndex{\setlength,\slshape,\space,\string,\strut}
% \DoNotIndex{\the,\trivlist,\typeout}
% \DoNotIndex{\vbox}
% \DoNotIndex{\xdef}
%
% \def\deflabel#1{\ttfamily #1\hfill}
% \def\deflist#1{\begin{list}{}{\settowidth\labelwidth{\ttfamily #1}%
%                               \setlength\leftmargin\labelwidth
%                               \addtolength\leftmargin\labelsep
%                               \let\makelabel\deflabel}}
% \def\enddeflist{\end{list}}
% \def\envfont{\normalfont\ttfamily}
%
% \MakeShortVerb{\|}
%
% \newcommand{\TUB}{{\sl TUGboat\/}}
%
% \GetFileInfo{theorem.dtx}
% \title{An Extension of the
%        \LaTeX{} theorem environment\thanks{This file
%        has version number \fileversion, last
%        revised \filedate.}}
% \author{Frank Mittelbach}
%
% \MaintainedByLaTeXTeam{tools}
% \maketitle
%
% \begin{abstract}
%    The macros described in this paper yield an extension of the
%    \LaTeX{} theorem mechanism. It is designed to satisfy the
%    different requirements of various journals. Thus, the
%    layout of the ``theorems'' can be manipulated by determining a
%    ``style''. This article describes not only the use, but
%    also the definition, of the necessary macros.
% \end{abstract}
%
% \section*{Preface to version 2.2}
%
% For \LaTeXe{} this package did not need any fundamental changes. I
% only modified the messages generated so that theorem layout styles
% will show up with the |\listfiles| command and cleaned the section
% on the New Font Selection Scheme since this is now included in
% \LaTeX.
%
% \section*{Preface to version 2.1}
%
% This version is identical to 2.0g described in \TUB\ 10\#3 except
% for some internal defaults which are now set depending on the used
% font selection scheme.
%
% This was done to avoid unpleasant surprises if the new font selection
% scheme is in force. For further details see
% section~\ref{sec:newfonts} and \cite{bk:GMS94}.
%
% \section{Introduction}
%
% For our purposes here, ``theorems'' are labelled enunciations,
% often set off from the main text by extra space and a font change.
% Theorems, corollaries, conjectures, definitions, and remarks
% are all instances of ``theorems''.  The ``header'' of these structures
% is composed of a label (such as \textsc{Theorem} or \textsc{Remark})
% and a number which serializes an item in the sequence of items
% with the same label.
%
% Shortly after the introduction of \LaTeX{} at the Fachbereich
% Mathematik in Mainz, the desire to manipulate the layout of
% ``theorems'' arose.  In Mainz, the following two conventions came
% into general use:
% \begin{enumerate}
%   \item The number of the theorem is shown in the margin.
%   \item There is a line break at
%         the end of the theorem header.
% \end{enumerate}
%
% Additionally, some journals require different formats which depend on
% the ``sort of theorem'': e.g.\ often remarks and definitions are set
% in |\upshape|, while |\itshape| is employed for main theorems.
%
% Confronted with these requirements, a theorem environment was
% developed in Mainz which allows separate determination of the
% layout of the ``theorems sets'', comparable to |\pagestyle|.
%
%
% \section{The user interface}
% \subsection{Defining new theorem sets}
%
% \DescribeMacro\newtheorem
% As in the original \LaTeX{} version, the command |\newtheorem|
% defines a new ``theorem set'' or ``theorem-like structure''.
% Two required arguments name the new environment and give the text
% to be typeset with each instance of the new ``set'', while
% an optional argument determines how
% the ``set'' is enumerated:
% \begin{description}
%    \item[\ttfamily \bslash newtheorem\{foo\}\{bar\}]
%       The theorem set {\envfont foo} (whose name is \texttt{bar})
%       uses its own counter.
%    \item[\ttfamily \bslash newtheorem\{foo2\}{[foo]}\{bar2\}]
%       The theorem set {\envfont foo2} (printed name \texttt{bar2})
%       uses the same counter as the theorem set \texttt{foo}.
%    \item[\ttfamily \bslash newtheorem\{foo3\}\{bar3\}{[section]}]
%       The theorem set {\envfont foo3} (printed name \texttt{bar3}) is
%       enumerated within the counter \texttt{section}, i.e.\ with every
%       new |\section| the enumeration begins again with 1, and
%       the enumeration is composed from the section-number and the
%       theorem counter itself.
% \end{description}
%
% \DescribeMacro\theoremstyle
% Additionally, the command |\theoremstyle| can define the layout
% of various, or all, theorem sets.  It should be noted that any
% theorem set defined by |\newtheorem| is typeset in the
% |\theoremstyle| that is current at the time of the definition.
% Thus, the following
% \begin{verbatim}
%    \theoremstyle{break}        \newtheorem{Cor}{Corollary}
%    \theoremstyle{plain}        \newtheorem{Exa}{Example}[section]
%\end{verbatim}                             ^^A No blank at line start
% leads to the result that the set {\envfont Cor} is formatted in the
% style \texttt{break}, while the set {\envfont Exa} and all the
% following ones are formatted in the style \texttt{plain}, unless
% another |\theoremstyle| follows.  Since the definitions installed by
% |\newtheorem| are global, one also can limit |\theoremstyle| locally
% by grouping braces.
%
% \DescribeMacro\theorembodyfont
% The choice of the font for the theorem body is completely
% independent of the chosen |\theoremstyle|; this has proven to
% be very advantageous.  For example,
% \begin{verbatim}
%   {\theorembodyfont{\upshape}          \newtheorem{Rem}{Remark}}
%\end{verbatim}
% defines a theorem set {\envfont Rem}, which will be set in
% |\upshape| in the current layout (which in our example is
% \texttt{plain}).  As with |\theoremstyle|, the |\theorembodyfont|
% chosen is that current at the time of |\newtheorem|.  If
% |\theorembodyfont| is not specified or one defines
% |\theorembodyfont{}|, then the font used will be that defined by the
% |\theoremstyle|.
%
% \DescribeMacro\theoremheaderfont It is also possible to customize
% the font used for the theorem headers.  This is, however, a global
% declaration, and therefore there should be at most one
% |\theoremheaderfont| declaration in the preamble.\footnote{If it is
% actually necessary to have different header fonts, one has to define
% new theorem styles (substituting the desired font) or specify the
% information directly in the \texttt{\bslash newtheorem} declaration
% (the unclean variant).}
%
% \DescribeMacro\theorempreskipamount
% \DescribeMacro\theorempostskipamount Two additional parameters
% affect the vertical space around the theorem environments:
% |\theorempreskipamount| and |\theorempostskipamount| define,
% respectively, the spacing before and after such an environment.
% These parameters apply for all theorem sets and can be manipulated
% with the ordinary length macros.  They are rubber lengths
% (`\textsf{skips}'), and therefore can contain \texttt{plus} and
% \texttt{minus} parts.
%
% Since the definition of theorem sets should---most sensibly---be
% placed in the preamble, we only allow installation there.
% It is therefore possible to release the memory used here after
% |\begin{document}|, in order to make room for other
% applications.
%
%
% \subsection{Existing theorem styles}
%
% The following theorem styles exist to date:
% \begin{deflist}{marginbreak}
%    \item[plain]
%       This theorem style emulates the original \LaTeX{} definition,
%       except that additionally the parameters
%       |\theorem...skipamount| are used.
%    \item[break]
%       In this style, the theorem header is followed by a line
%       break.
%    \item[marginbreak]
%       The theorem number is set in the margin, and there is a line
%       break as in \texttt{break}.
%    \item[changebreak]
%       Like \texttt{break}, but with header number and text
%       interchanged.
%    \item[change]
%       Header number and text are interchanged, without a line break.
%    \item[margin]
%       The number is set in the left margin, without a line break.
% \end{deflist}
% All styles (except \texttt{plain}) select |\slshape| as the default
% |\theorembodyfont|.
%
%
% \subsection{Examples}
%
% Given the above theorem sets {\envfont Cor}, {\envfont Exa}
% and {\envfont Rem},
% suppose that the preamble also contains the declarations:
% \begin{verbatim}
%    \theoremstyle{marginbreak}   \newtheorem{Lem}[Cor]{Lemma}
%    \theoremstyle{change}
%    \theorembodyfont{\itshape}        \newtheorem{Def}[Cor]{Definition}
%
%    \theoremheaderfont{\scshape}
%\end{verbatim}
% Then the following are some typical examples of the typeset output
% resulting from their use.
%
% \begin{Cor}
%    This is a sentence typeset in the theorem environment {\envfont
%    Cor}.
% \end{Cor}
%
% \begin{Exa}
%    This is a sentence typeset in the theorem environment {\envfont
%    Exa}.
% \end{Exa}
%
% \begin{Rem}
%    This is a sentence typeset in the theorem environment {\envfont
%    Rem}.
% \end{Rem}
%
% \begin{Lem}[Ben User]
%    This is a sentence typeset in the theorem environment {\envfont
%    Lem}.
% \end{Lem}
%
% \begin{Def}[Very impressive Definition]
%    This is a sentence typeset in the theorem environment {\envfont
%    Def}.
% \end{Def}
%
% The last two examples show the effect of the optional argument to
% a theorem environment (it is the text typeset in parentheses).
%
%
% \section{Special Considerations}
% \label{sec:newfonts}
%
% Theoremheader and body are implemented as a unit. This means that
% the |\theoremheaderfont|  will inherit characteristics of the
% |\theorembodyfont| in \LaTeXe. Thus, if for example
% |\theorembodyfont| is |\itshape| and |\theoremheaderfont|
% is |\bfseries| the font selected for the header will have the
% characteristics `bold extended italic'. If this is not desired one
% should set the |\theoremheaderfont| to something like
% \begin{verbatim}
%      \theoremheaderfont{\normalfont\bfseries}
%\end{verbatim}
% i.e.\ supplying all necessary font information explicitly.
%
%
%
% \section{Acknowledgements}
%
% The publication of this set of macros was only possible with the
% help of Christina Busse (translating the manuscript into English),
% Joachim Pense (playing the r\^ole of typist), Chris Rowley
% (looking everything over) and many others providing useful
% suggestions.
%
% \MaybeStop{%
%    \begin{thebibliography}{9}
%       \bibitem{bk:GMS94} \textsc{M.~Goossens}, \textsc{F.~Mittelbach}
%         and \textsc{A.~Samarin}.
%         \newblock The \LaTeX{} Companion.
%         \newblock
%         Addison-Wesley, Reading, Massachusetts, 1994.
%       \bibitem{latex:source} \textsc{Lamport, Leslie.} \newblock
%                               \texttt{latex.tex}, version 2.09, date
%                               Feb. 1990.
%    \end{thebibliography}
%    \PrintIndex
%    \PrintChanges}
%
%
% \section{The documentation driver file}
%
% The next bit of code contains the documentation driver file for
% \TeX{}, i.e., the file that will produce the documentation you are
% currently reading. It will be extracted from this file by the
% \texttt{docstrip} program. Since it is the first code in the file
% one can alternatively process this file directly with \LaTeXe{} to
% obtain the documentation.
%
%    \begin{macrocode}
%<*driver>
\documentclass{ltxdoc}

\usepackage{theorem}

%<+driver>% The next few lines define theorem sets which are used
%<+driver>% in the example section of the documentation.

\theoremstyle{break}       \newtheorem{Cor}{Corollary}
\theoremstyle{plain}       \newtheorem{Exa}{Example}[section]
{\theorembodyfont{\upshape}\newtheorem{Rem}{Remark}}
\theoremstyle{marginbreak} \newtheorem{Lem}[Cor]{Lemma}
\theoremstyle{change}
\theorembodyfont{\itshape} \newtheorem{Def}[Cor]{Definition}

\theoremheaderfont{\scshape}

\RecordChanges

\begin{document}
   \DocInput{theorem.dtx}
\end{document}
%</driver>
%    \end{macrocode}
%
% \section{Definition of the Macros}
%
% If the file has been loaded before, we abort immediately. If not the
% package announces itself (this is actually done at the very top of
% the file---the way it is done isn't good style so don't copy it).
% \changes{v2.0e}{89/07/19}{Spaces removed from `typeout'.}
% \changes{v2.2b}{95/11/19}{Announce moved to top of file}
%    \begin{macrocode}
%<*package>
%\@ifundefined{theorem@style}{}{\endinput}
%\def\FMithmInfo{1995/11/19 v2.2b Theorem extension package (FMi)}
%\ProvidesPackage{theorem}[\FMithmInfo]
%    \end{macrocode}
%
%
% \subsection{Definition of theorem styles and fonts}
%
% All the definitions in this file are done globally to allow
% inputting this file inside a group.
%
% \begin{macro}{\theoremstyle}
%    Before a theorem style can be installed, the chosen style must be
%    known. For that reason, we must test to see that
%    |\th@|\meta{style} is known or, more precisely, that it is
%    different from |\relax|. If the style is not known then
%    |\th@plain| is used.
%    \begin{macrocode}
\gdef\theoremstyle#1{%
   \@ifundefined{th@#1}{\@warning
          {Unknown theoremstyle `#1'. Using `plain'}%
          \theorem@style{plain}}%
%    \end{macrocode}
%    We save the theorem style to be used in the token register
%    |\theorem@style|.
%    \begin{macrocode}
      {\theorem@style{#1}}%
%    \end{macrocode}
%    Now we ``evaluate'' the theorem style: this means, we call the
%    macro |\th@|\meta{style} which will activate the relevant
%    definitions which are contained in a separate file.  This is done
%    in a group to suppress changes to the current font.  This could
%    otherwise pose problems together with the new font selection
%    scheme\footnote{When I printed the original article using the new
%    font selection scheme I ended with a document with slanted
%    typefaces (text headings and all) simply because one of the
%    theorem styles used \texttt{\bslash sl} at toplevel.} if the
%    |\th@|\meta{style} is evaluated a second time.
% \changes{v2.1a}{90/02/15}{Group added around `th@..}
% \changes{v2.1a}{90/02/15}{Reseting of `..@begintheorem removed.}
%    \begin{macrocode}
      \begingroup
        \csname th@\the\theorem@style \endcsname
      \endgroup}
%    \end{macrocode}
% \end{macro}
%
%
% \begin{macro}{\@begintheorem}
% \changes{v2.1a}{90/02/15}{Relaxed at toplevel.}
% \begin{macro}{\@opargbegintheorem}
% \changes{v2.1a}{90/02/15}{Relaxed at toplevel.}
%    We reset |\@begintheorem| and |\@opargbegintheorem| to
%    |\relax| since these commands are no longer necessary at
%    toplevel. This will save a few tokens.
%    \begin{macrocode}
\global\let\@begintheorem\relax
\global\let\@opargbegintheorem\relax
%    \end{macrocode}
% \end{macro}
% \end{macro}
%
% \begin{macro}{\theorem@style}
%    Obviously the \textsf{token} register used above has to be
%    allocated. To assure the utmost compatibility with the
%    original \LaTeX{} definition, we set the default theorem style to
%    \texttt{plain}, which implements the usual \LaTeX{} convention.
% \changes{v2.0f}{89/07/20}{Unnecessary `global' removed.}
%    \begin{macrocode}
\newtoks\theorem@style
\global\theorem@style{plain}
%    \end{macrocode}
% \end{macro}
%
% \begin{macro}{\theorembodyfont}
% \begin{macro}{\theorem@bodyfont}
%    If the |\theorembodyfont| is set by the user then it should not
%    interact with the default font set in the theorem style. When the
%    new font selection is in force this may happen if, for example,
%    the default is |\itshape| and the new |\theorembodyfont| is
%    |\sffamily|. So we add a |\reset@font| command in front of the
%    user definition.
% \changes{v2.0f}{89/07/20}{Unnecessary `global' removed.}
% \changes{v2.0g}{89/09/19}{`theoremfont to `theorembodyfont.}
% \changes{v2.1c}{92/06/22}{`reset@font added and logic changed.}
% \changes{v2.1d}{92/10/28}{Reset to empty if arg is empty}
%    \begin{macrocode}
\gdef\theorembodyfont#1{%
%    \end{macrocode}
%    We check if the argument supplied is empty and if so put nothing
%    into the |\theorem@bodyfont| token register to allow for
%    |\theorembodyfont{}| as a mean of using the default of the
%    current |\theoremstyle|.
%    \begin{macrocode}
   \def\@tempa{#1}%
   \ifx\@tempa\@empty
    \theorem@bodyfont{}%
   \else
    \theorem@bodyfont{\reset@font#1}%
   \fi
   }
\newtoks\theorem@bodyfont
\global\theorem@bodyfont{}
%    \end{macrocode}
% \end{macro}
% \end{macro}
%
%
% \begin{macro}{\theoremheaderfont}
% \changes{v2.0g}{89/09/19}{Macro added.}
%    The font for the theorem headers is handled differently because
%    this definition applies to all theorem styles.
%    \begin{macrocode}
\gdef\theoremheaderfont#1{\gdef\theorem@headerfont{#1}%
%    \end{macrocode}
%    After using the macro once it is redefined to produce an error
%    message.
%    \begin{macrocode}
       \gdef\theoremheaderfont##1{%
        \typeout{\string\theoremheaderfont\space should be used
                 only once.}}}
%    \end{macrocode}
% \end{macro}
%
% \begin{macro}{\theorem@headerfont}
% \changes{v2.0g}{89/09/19}{Macro added.}
%    To set the |\theorem@headerfont| default we first test if
%    the new fontselection scheme is in force.
% \changes{v2.1a}{90/02/15}{Support for new font selection.}
%    \begin{macrocode}
\ifx\upshape\undefined
%    \end{macrocode}
%    If not we define it to expand into |\bfseries|. We don't use
%    |\let| just in case a following style option redefines this
%    macro.
%    \begin{macrocode}
\gdef\theorem@headerfont{\bfseries}
%    \end{macrocode}
%    Otherwise we reset the current shape before calling |\bfseries|.
%    \begin{macrocode}
\else \gdef\theorem@headerfont{\normalfont\bfseries}\fi
%    \end{macrocode}
% \end{macro}
%
% \begin{macro}{\th@plain}
% \begin{macro}{\th@break}
% \begin{macro}{\th@marginbreak}
% \begin{macro}{\th@changebreak}
% \begin{macro}{\th@change}
% \begin{macro}{\th@margin}
%    The different styles are defined in macros such as
%    |\th@plain|.  Since memory space is precious in
%    ``non-Big-versions'', we have to avoid offering too many unused
%    definitions.  Therefore we define these styles in separate files
%    that can be loaded on demand.  Thus the commands themselves only
%    load these files. We use |\@input@|, a \LaTeXe{} internal command
%    that ensures that the file will be listed with |\listfiles|.
% \changes{v2.2a}{94/02/02}{Use `@input@ to load theorem layout files}
%    \begin{macrocode}
\gdef\th@plain{\@input@{thp.sty}}
\gdef\th@break{\@input@{thb.sty}}
\gdef\th@marginbreak{\@input@{thmb.sty}}
\gdef\th@changebreak{\@input@{thcb.sty}}
\gdef\th@change{\@input@{thc.sty}}
\gdef\th@margin{\@input@{thm.sty}}
%    \end{macrocode}
%    This list will be expanded when new styles become
%    available.  For testing,  just append new theorem substyles
%    as document options.
% \end{macro}
% \end{macro}
% \end{macro}
% \end{macro}
% \end{macro}
% \end{macro}
%
%
%
% \subsection{Definition of a new theorem set}
%
% As already pointed out, a new theorem environment can be defined in
% three different ways:
% \begin{verbatim}
%    \newtheorem{Lem}{Lemma}
%    \newtheorem{Lem}{Lemma}[section]
%    \newtheorem{Lem}[Theorem]{Lemma}
%\end{verbatim}
% The function of the macro |\newtheorem| is to recognize these cases
% and then to branch into one of the three macros |\@ynthm|, |\@xnthm|
% or |\@othm|.  This mechanism is adopted unchanged from
% \cite{latex:source}; the essential point here is that, for example,
% in the second case, the arguments \texttt{Lem}, \texttt{Lemma} and
% \texttt{section} are passed over to the macro |\@xnthm|.
%
% We inspect this case first because the others present fewer problems,
% and thus are easily derived from this one.
%
% \begin{macro}{\@xnthm}
%    For our example arguments, the macro |\@xnthm| must fulfill
%    the following:
%    \begin{itemize}
%       \item
%          Define a new \LaTeX-counter `Lem'
%       \item
%          reset this counter within a |\section|
%       \item
%          define the macro |\theLem|
%       \item
%          define the environment macros |\Lem| and
%          |\endLem| using the current
%          |\theoremstyle| and |\theorem@bodyfont|.
%    \end{itemize}
%    Obviously, all this should happen only if the first argument
%    of |\@xnthm| (i.e.\ \texttt{Lem} in our example) is chosen so
%    as not to conflict with any previously defined commands or
%    environments.  This test is performed by the \LaTeX{} macro
%    |\@ifdefinable|.
%    \begin{macrocode}
\gdef\@xnthm#1#2[#3]{\expandafter\@ifdefinable\csname #1\endcsname
%    \end{macrocode}
%    Therefore, the first argument of |\@ifdefinable| is the
%    expansion (in the example, |\Lem|) of
%    |\csname||#1\endcsname|.  The second argument is
%    executed only if the test has been completed successfully.
%    \begin{macrocode}
   {%
%    \end{macrocode}
%    Now we define the new counter. The names of the \LaTeX{}
%    macros employed should speak for themselves:
% \changes{v2.2b}{1995/10/07}{Use \cs{@newctr}}
% \changes{v2.2c}{1995/11/23}{Use \cs{@newctr} properly.}
%    \begin{macrocode}
    \@definecounter{#1}\@newctr{#1}[#3]%
%    \end{macrocode}
%    Using |\@newctr| will give a proper error message if the counter
%    in |#3| is not defined.
%    In defining `|\theLem|' we must generate the desired macro
%    name by use of |\expandafter| and |\csname|.
%    \begin{macrocode}
    \expandafter\xdef\csname the#1\endcsname
%    \end{macrocode}
%    An |\xdef| is used in order to make the definition global,
%    and to ensure that it contains the replacement texts of
%    |\@thmcountersep| and |\@thmcounter|.\footnote{These
%    two macros can be defined by the document style.  Their default
%    values produce a `\texttt{.}' as separation and an arabic
%    representation of the number.} However, not everything should be
%    expanded. For example, it saves space to use |\thesection|
%    instead of its---at times---lengthy expansion.
%    \begin{macrocode}
      {\expandafter \noexpand \csname the#3\endcsname
       \@thmcountersep \@thmcounter{#1}}%
%    \end{macrocode}
%    Thus with the defaults of \LaTeX{}, |\theLem| would be
%    replaced by the command sequence |\thesection.\arabic{Lem}|.
%
%    We will now look at the definition of the macro which is executed
%    at the beginning of the actual environment (in our example this
%    macro is |\Lem|). It should be noted that we use an
%    ``|\expandafter| trick'' to expand only certain parts of the
%    replacement text at the time of the definition.
%    \begin{macrocode}
    \def\@tempa{\global\@namedef{#1}}%
    \expandafter \@tempa \expandafter{%
%    \end{macrocode}
%    First, the macro that contains the current definitions of
%    |\@begintheorem| and |\@opargtheorem| should be called
%    up.  The name of this macro---as is already known---has the form
%    |\th@|\meta{theorem\ style}; therefore, it must be called by
%    \begin{macrocode}
      \csname th@\the \theorem@style
%    \end{macrocode}
%    In addition the default theorem font should be changeable, i.e.\
%    we have to insert the contents of |\theorem@bodyfont|. For
%    that reason, we expand even further, beyond |\endcsname|,
%    and thus insert the contents of the \textsf{token} register
%    |\theorem@bodyfont| in the replacement text.
%    \begin{macrocode}
            \expandafter \endcsname \the \theorem@bodyfont
%    \end{macrocode}
%    Now it is time to call the macro |\@thm| which takes over
%    the further processing.  It has two arguments: the current
%    counter name (in our example, \texttt{Lem}), and the text of the
%    label (in our example, \texttt{Lemma}).
%    \begin{macrocode}
     \@thm{#1}{#2}}%
%    \end{macrocode}
%    With this, the `sub-definition' is complete.  The macro
%    |\@endtheorem| ends a theorem environment and is, so far, nothing
%    but an |\endtrivlist|.  (Hence it is defined globally, and not
%    within the theorem styles.\footnote{This has to be changed as
%    soon as theorem styles that change \texttt{\bslash @endtheorem}
%    exist.  In such a case, all existing styles must be changed as
%    well since they will have to reset the macro.}) Therefore, we can
%    set it equivalent to the macro that ends the theorem set (in our
%    example, |\endLem|).  However, if some day theorem styles exist
%    that do change |\@endtheorem|, we would have to use the
%    commented-out line instead.
%    \begin{macrocode}
    \global \expandafter \let \csname end#1\endcsname \@endtheorem
%   \global\@namedef{end#1}{\@endtheorem}%
%    \end{macrocode}
%    With these commands all the required definitions are employed,
%    unless the test |\@ifdefinable| has failed.  Therefore, we
%    end the second argument of this macro and with it the definition
%    of |\@xnthm|.
%    \begin{macrocode}
   }}
%    \end{macrocode}
% \end{macro}
%
% \begin{macro}{\@ynthm}
%    The definition of |\@ynthm| is completely analogous. In
%    this case the new counter that is defined is not reset within
%    another counter; thus the definition of |\the...|
%    is simplified:
%    \begin{macrocode}
\gdef\@ynthm#1#2{\expandafter\@ifdefinable\csname #1\endcsname
   {\@definecounter{#1}%
    \expandafter\xdef\csname the#1\endcsname{\@thmcounter{#1}}%
%    \end{macrocode}
%    The rest of the definition corresponds literally to that of
%    |\@xnthm|:
%    \begin{macrocode}
    \def\@tempa{\global\@namedef{#1}}\expandafter \@tempa
     \expandafter{\csname th@\the \theorem@style \expandafter
     \endcsname \the\theorem@bodyfont \@thm{#1}{#2}}%
    \global \expandafter \let \csname end#1\endcsname \@endtheorem}}
%    \end{macrocode}
% \end{macro}
%
% \begin{macro}{\@othm}
%    The definition of |\@othm| does not contain anything new.
%    \begin{macrocode}
\gdef\@othm#1[#2]#3{%
%    \end{macrocode}
% \changes{v2.2b}{95/10/07}{Build in check for counter defined}
%    We do not define a new counter but instead use one that has
%    already been defined.  Thus the only definition we need is that
%    of this pseudo-counter (i.e.\ |\the|\meta{env.\ name}).
%    First we check if |#2| corresponds to a known counter name.
%    \begin{macrocode}
  \expandafter\ifx\csname c@#2\endcsname\relax
   \@nocounterr{#2}%
  \else
   \expandafter\@ifdefinable\csname #1\endcsname
   {\expandafter \xdef \csname the#1\endcsname
     {\expandafter \noexpand \csname the#2\endcsname}%
%    \end{macrocode}
%    All other parts of the definition can be adopted from
%    |\@xnthm|. We have to remember, though, that in this case
%    the name of the current counter and the theorem label have moved
%    to the second and third arguments.
%    \begin{macrocode}
    \def\@tempa{\global\@namedef{#1}}\expandafter \@tempa
     \expandafter{\csname th@\the \theorem@style \expandafter
     \endcsname \the\theorem@bodyfont \@thm{#2}{#3}}%
    \global \expandafter \let \csname end#1\endcsname \@endtheorem}%
  \fi}
%    \end{macrocode}
% \end{macro}
%
% \subsection{Macros that are employed in a {\envfont theorem}
%    environment}
%
% \begin{macro}{\@thm}
%    The macro |\@thm| has to increase the current counter.  Then,
%    depending on whether the environment has (or does not have) an
%    optional argument, it has to branch into either |\@begintheorem|
%    or |\@opargtheorem|.
%    \begin{macrocode}
\gdef\@thm#1#2{\refstepcounter{#1}%
%    \end{macrocode}
%    Now we start a {\envfont trivlist} environment, and give
%    |\@topsep| and |\@topsepadd| the values of the \textsf{skip}
%    registers |\theorempreskipamount| and |\theorempostskipamount|.
%    The value in |\@topsep| is the vertical space that is inserted by
%    the first (and only) |\item| in our |\trivlist| whilst
%    |\@topsepadd| is inserted by |\@endparenv| at the end of that
%    {\envfont trivlist} environment.  By using these registers, we
%    obtain the desired space around a {\envfont theorem} environment.
%    \begin{macrocode}
   \trivlist
   \@topsep \theorempreskipamount               % used by first \item
   \@topsepadd \theorempostskipamount           % used by \@endparenv
%    \end{macrocode}
%    Now we have to test whether an optional argument has been given.
% \changes{v2.0d}{89/07/18}{Forgotten percent added.}
%    \begin{macrocode}
   \@ifnextchar [%
%    \end{macrocode}
%    If there is an optional argument, we will call |\@ythm|, and
%    move the arguments read back into the input stream.
%    \begin{macrocode}
   {\@ythm{#1}{#2}}%
%    \end{macrocode}
%    If not, we call |\@begintheorem|. Its first argument is the
%    name of the theorem set (hence the second argument of
%    |\@thm|). Its second argument is the macro that produces the
%    current number.
%    \begin{macrocode}
   {\@begintheorem{#2}{\csname the#1\endcsname}\ignorespaces}}
%    \end{macrocode}
% \end{macro}
%
% \begin{macro}{\@xthm}
% \begin{macro}{\@ythm}
%    Both these macros were originally called by |\@thm|.  We do
%    not need |\@xthm| anymore, hence we reset it to
%    |\relax|. The definition of |\@ythm| has not changed at
%    all from its definition in \LaTeX. In order to make the macros
%    easier to understand, we will
%    nevertheless present it (commented out).
%    \begin{macrocode}
\global\let\@xthm\relax
% \def\@ythm#1#2[#3]{\@opargbegintheorem{#2}{\csname
%           the#1\endcsname}{#3}\ignorespaces}
%</package>
%    \end{macrocode}
%    The primitive |\ignorespaces| in |\@ythm| and
%    |\@thm| is needed to remove the spaces between the
%    |\begin{...}| and the actual text.
% \end{macro}
% \end{macro}
%
%
% \subsection{Definition of the theorem substyles}
%
% As already pointed out, the theorem substyles, defined below, are
% only loaded when necessary.  Note that all these substyles, except
% \texttt{plain}, have |\slshape| as the default body font.
%
% \subsubsection{The \texttt{plain} style}
%
% \label{dok:th@plain} As the following macros use |@|, we have
% to locally set the |\catcode| of this symbol to ``letter''.
% This happens within a group, so that we do not have to worry about
% which |\catcode| that symbol had before.
%    \begin{macrocode}
%<*thp>
\begingroup \makeatletter
%    \end{macrocode}
% Since we are now within a group, we must make all definitions
% globally.  First we make sure that \texttt{theorem.sty} is loaded.
% This will allow us to use this file as a document style option
% without having to call \texttt{theorem} itself as an option.  At the
% same time, we assure that at least version 2 is loaded, since
% |\theorem@style| was not defined in earlier versions.
%    \begin{macrocode}
\@ifundefined{theorem@style}{\input{theorem.sty}}{}
\ProvidesFile{thp.sty}
             [\FMithmInfo]
%    \end{macrocode}
%
%
% \begin{macro}{\th@plain}
% \begin{macro}{\@begintheorem}
% \begin{macro}{\@opargbegintheorem}
%    |\theoremstyle{plain}| corresponds to the original
%    definition, except that the distances to the surrounding text are
%    determined by the parameters |\theorempreskipamount| and
%    |\theorempostskipamount|.
%    First we set the default body font.
%    \begin{macrocode}
\gdef\th@plain{\normalfont\itshape
%    \end{macrocode}
%    Then we define |\@begintheorem| and
%    |\@opargbegintheorem|. These two macros define how the
%    header of a theorem is typeset.  |\@opargbegintheorem| will
%    be called if a \textsf{theorem} environment with an optional
%    argument is encountered; otherwise, the header is constructed by
%    calling |\@begintheorem|.  If one of these macros is
%    executed, we are within a \textsf{trivlist} environment started by
%    |\@thm|.  So the theorem header is produced with an
%    |\item| command.
%
%    Instead of specifying the header font directly, all standard
%    theorem styles use the |\theorem@headerfont| macro to allow
%    customization.  The extra space (|\labelsep|) is necessary
%    because of problems in the \textsf{trivlist} environment.
%    \begin{macrocode}
  \def\@begintheorem##1##2{%
        \item[\hskip\labelsep \theorem@headerfont ##1\ ##2]}%
%    \end{macrocode}
%    The definition of |\@opargbegintheorem| is completely
%    analogous. The only difference is the fact that there exists a
%    third argument (which is the optional parameter of the
%    environment and contains additional information about the
%    theorem).  Customarily we enclose it in parentheses.
%    \begin{macrocode}
\def\@opargbegintheorem##1##2##3{%
   \item[\hskip\labelsep \theorem@headerfont ##1\ ##2\ (##3)]}}
%    \end{macrocode}
% \end{macro}
% \end{macro}
% \end{macro}
%
% \noindent
% We conclude with an |\endgroup| to restore the
% |\catcode| of |@|.
%    \begin{macrocode}
\endgroup
%</thp>
%    \end{macrocode}
%
%
% \subsubsection{The \texttt{break} style}
%
% This style option is stored in the file |thb.sty|.
% For the next two lines see the documentation for |\th@plain| on
% page~\pageref{dok:th@plain}.
%    \begin{macrocode}
%<*thb>
\begingroup \makeatletter
\@ifundefined{theorem@style}{\input{theorem.sty}}{}
\ProvidesFile{thb.sty}
             [\FMithmInfo]
%    \end{macrocode}
%
% \begin{macro}{\th@break}
%    |\theoremstyle{break}| produces a line break after the name
%    of the theorem.  The font is |\slshape|.  Hence, we define
%    |\th@break| as follows:
%    \begin{macrocode}
\gdef\th@break{\normalfont\slshape
  \def\@begintheorem##1##2{\item[%
%    \end{macrocode}
%    We run into the following problem: it is not possible to create
%    the header with |\item[|\meta{title}|]| and then start
%    a new line by, for example, |\mbox{}\\|. Such a definition
%    will fail whenever a list environment follows immediately.  With
%    the above construction, the |\mbox{}| causes the switch
%    |@inlabel| (cf.\ definition of |\list| and
%    |\trivlist| in \cite{latex:source}) to be set to \texttt{false}
%    and so the following list will insert additional vertical
%    space (|\topskip|). This is quite annoying. Therefore, we
%    create the line break within the |\item|. In order to ensure
%    that the text will begin at the proper position in the following
%    line, we simply pretend that the label does not take any
%    room.\footnote{This will lead to problems whenever very high
%    symbols occurring in the line tower into the
%    heading. So, something else has to be done here sometime.}
%    \begin{macrocode}
       \rlap{\vbox{\hbox{\hskip \labelsep\theorem@headerfont ##1\ ##2}%
                   \hbox{\strut}}}]}%
%    \end{macrocode}
%    Again, the definition of |\@opargbegintheorem| is
%    completely analogous.
%    \begin{macrocode}
\def\@opargbegintheorem##1##2##3{%
  \item[\rlap{\vbox{\hbox{\hskip \labelsep \theorem@headerfont
                     ##1\ ##2\ (##3)}%
                    \hbox{\strut}}}]}}
\endgroup
%</thb>
%    \end{macrocode}
% \end{macro}
%
%
% \subsubsection{The \texttt{changebreak} style}
%
%    \begin{macrocode}
% This style option is stored in the file |thcb.sty|.
%    \begin{macrocode}
%<*thcb>
\begingroup \makeatletter
\@ifundefined{theorem@style}{\input{theorem.sty}}{}
\ProvidesFile{thcb.sty}
             [\FMithmInfo]
%    \end{macrocode}
%
% \begin{macro}{\th@changebreak}
%    The \texttt{change-break} theorem style is like \texttt{break}
%    but with interchange of theorem name and theorem number.  Thus we
%    define |\th@changebreak| as follows:
%    \begin{macrocode}
\gdef\th@changebreak{\normalfont\slshape
  \def\@begintheorem##1##2{\item
       [\rlap{\vbox{\hbox{\hskip\labelsep \theorem@headerfont ##2\ ##1}%
                    \hbox{\strut}}}]}%
\def\@opargbegintheorem##1##2##3{%
  \item[\rlap{\vbox{\hbox{\hskip\labelsep \theorem@headerfont
                     ##2\ ##1\ (##3)}%
                    \hbox{\strut}}}]}}
\endgroup
%</thcb>
%    \end{macrocode}
% \end{macro}
%
%
% \subsubsection{The \texttt{change} style}
%
% This style option is stored in the file |thc.sty|.
%    \begin{macrocode}
%<*thc>
\begingroup \makeatletter
\@ifundefined{theorem@style}{\input{theorem.sty}}{}
\ProvidesFile{thc.sty}
             [\FMithmInfo]
%    \end{macrocode}
%
% \begin{macro}{\th@change}
%    The \texttt{change} theorem style corresponds to the
%    \texttt{change break} style without a linebreak after the header.
%    To say it in another way, it's the same as the \texttt{plain}
%    style but with number and name interchanged and |\slshape| as the
%    default font.
%    \begin{macrocode}
\gdef\th@change{\normalfont\slshape
  \def\@begintheorem##1##2{\item
       [\hskip\labelsep \theorem@headerfont ##2\ ##1]}%
\def\@opargbegintheorem##1##2##3{%
  \item[\hskip\labelsep \theorem@headerfont ##2\ ##1\ (##3)]}}
\endgroup
%</thc>
%    \end{macrocode}
% \end{macro}
%
%
% \subsubsection{The \texttt{marginbreak} style}
%
% This style option is the one used most often at Mainz.  It is saved
% in the file |thmb.sty|.
%    \begin{macrocode}
%<*thmb>
\begingroup \makeatletter
\@ifundefined{theorem@style}{\input{theorem.sty}}{}
\ProvidesFile{thmb.sty}
             [\FMithmInfo]
%    \end{macrocode}
%
% \begin{macro}{\th@marginbreak}
%    The \texttt{margin break} style is nearly the same as the
%    \texttt{change break} style. The only difference is the placement
%    of the theorem number.  We use |\llap| to place it in the left
%    margin.
%
%    In this style |\labelsep| denotes the separation between the
%    number and the text.
%    \begin{macrocode}
\gdef\th@marginbreak{\normalfont\slshape
  \def\@begintheorem##1##2{\item
       [\rlap{\vbox{\theorem@headerfont
                    \hbox{\llap{##2}\hskip\labelsep ##1}%
                    \hbox{\strut}}}]}%
\def\@opargbegintheorem##1##2##3{%
  \item[\rlap{\vbox{\theorem@headerfont
                    \hbox{\llap{##2}\hskip\labelsep ##1\ (##3)}%
                    \hbox{\strut}}}]}}
\endgroup
%</thmb>
%    \end{macrocode}
% \changes{v2.1c}{90/02/17}{`endgroup outside macrocode env.}
% \end{macro}
%
%
%
% \subsubsection{The \texttt{margin} style}
%
% This style option is stored in the file |thm.sty|.
%    \begin{macrocode}
%<*thm>
\begingroup \makeatletter
\@ifundefined{theorem@style}{\input{theorem.sty}}{}
\ProvidesFile{thm.sty}
             [\FMithmInfo]
%    \end{macrocode}
%
% \begin{macro}{\th@margin}
%    Again this is only a variant of the theorem styles described
%    above without any new ideas.
%    \begin{macrocode}
\gdef\th@margin{\normalfont\slshape
  \def\@begintheorem##1##2{\item
       [\theorem@headerfont \llap{##2}\hskip\labelsep ##1]}%
\def\@opargbegintheorem##1##2##3{%
  \item[\theorem@headerfont \llap{##2}\hskip\labelsep ##1\ (##3)]}}
\endgroup
%</thm>
%    \end{macrocode}
% \end{macro}
%
%
%
%
% \subsection{Final Definitions}
%
% \begin{macro}{\theorempreskipamount}
% \begin{macro}{\theorempostskipamount}
%    The \textsf{skip} parameters that regulate the vertical empty space
%    before and after the {\envfont theorem} environment have
%    to be allocated as well.
% \changes{v2.0f}{89/07/20}{Unnecessary `global' removed.}
%    \begin{macrocode}
%<*package>
\newskip\theorempreskipamount
\newskip\theorempostskipamount
%    \end{macrocode}
%    Since we have used the same values for all theorem sets, we now can
%    assign them.
%    \begin{macrocode}
\global\setlength\theorempreskipamount{12pt plus 5pt minus 3pt}
\global\setlength\theorempostskipamount{8pt plus 3pt minus 1.5pt}
%    \end{macrocode}
% \end{macro}
% \end{macro}
%
%
% \begin{macro}{\@endtheorem}
%    The same holds for the macro |\@endtheorem|, which ends a
%    {\envfont theorem} environment. Since it is the same for all
%    theorem sets, it is removed from the macros |\th@|\meta{style}.
%    It simply ends the {\envfont trivlist} environment, which was
%    begun in |\@thm|.
%    \begin{macrocode}
\global\let\@endtheorem=\endtrivlist
%    \end{macrocode}
% \end{macro}
%
%
%
% \begin{macro}{\@preamblecmds}
%    All macros defined above are to be used only in the preamble.
%    Therefore, we insert them in |\@preamblecmds| which will disable
%    them at begin document. This is done by the internal \LaTeXe{}
%    command |\@onlypreamble|.
% \changes{v2.0g}{89/09/19}{`theoremheaderfont and `theorembodyfont
%    added.}
% \changes{v2.2a}{94/02/03}{Use \LaTeXe{} interface}
%    \begin{macrocode}
\@onlypreamble\@xnthm
\@onlypreamble\@ynthm
\@onlypreamble\@othm
\@onlypreamble\newtheorem
\@onlypreamble\theoremstyle
\@onlypreamble\theorembodyfont
\@onlypreamble\theoremheaderfont
%    \end{macrocode}
% \end{macro}
%
% Finally we declare the \texttt{plain} theorem style to be the default.
% \changes{v2.1b}{90/06/04}{Fixed missing `item by declaring default
%    style.}
%    \begin{macrocode}
\theoremstyle{plain}
%</package>
%    \end{macrocode}
%
% \Finale

\endinput
}{}
\ProvidesFile{thm.sty}
             [\FMithmInfo]
%    \end{macrocode}
%
% \begin{macro}{\th@margin}
%    Again this is only a variant of the theorem styles described
%    above without any new ideas.
%    \begin{macrocode}
\gdef\th@margin{\normalfont\slshape
  \def\@begintheorem##1##2{\item
       [\theorem@headerfont \llap{##2}\hskip\labelsep ##1]}%
\def\@opargbegintheorem##1##2##3{%
  \item[\theorem@headerfont \llap{##2}\hskip\labelsep ##1\ (##3)]}}
\endgroup
%</thm>
%    \end{macrocode}
% \end{macro}
%
%
%
%
% \subsection{Final Definitions}
%
% \begin{macro}{\theorempreskipamount}
% \begin{macro}{\theorempostskipamount}
%    The \textsf{skip} parameters that regulate the vertical empty space
%    before and after the {\envfont theorem} environment have
%    to be allocated as well.
% \changes{v2.0f}{89/07/20}{Unnecessary `global' removed.}
%    \begin{macrocode}
%<*package>
\newskip\theorempreskipamount
\newskip\theorempostskipamount
%    \end{macrocode}
%    Since we have used the same values for all theorem sets, we now can
%    assign them.
%    \begin{macrocode}
\global\setlength\theorempreskipamount{12pt plus 5pt minus 3pt}
\global\setlength\theorempostskipamount{8pt plus 3pt minus 1.5pt}
%    \end{macrocode}
% \end{macro}
% \end{macro}
%
%
% \begin{macro}{\@endtheorem}
%    The same holds for the macro |\@endtheorem|, which ends a
%    {\envfont theorem} environment. Since it is the same for all
%    theorem sets, it is removed from the macros |\th@|\meta{style}.
%    It simply ends the {\envfont trivlist} environment, which was
%    begun in |\@thm|.
%    \begin{macrocode}
\global\let\@endtheorem=\endtrivlist
%    \end{macrocode}
% \end{macro}
%
%
%
% \begin{macro}{\@preamblecmds}
%    All macros defined above are to be used only in the preamble.
%    Therefore, we insert them in |\@preamblecmds| which will disable
%    them at begin document. This is done by the internal \LaTeXe{}
%    command |\@onlypreamble|.
% \changes{v2.0g}{89/09/19}{`theoremheaderfont and `theorembodyfont
%    added.}
% \changes{v2.2a}{94/02/03}{Use \LaTeXe{} interface}
%    \begin{macrocode}
\@onlypreamble\@xnthm
\@onlypreamble\@ynthm
\@onlypreamble\@othm
\@onlypreamble\newtheorem
\@onlypreamble\theoremstyle
\@onlypreamble\theorembodyfont
\@onlypreamble\theoremheaderfont
%    \end{macrocode}
% \end{macro}
%
% Finally we declare the \texttt{plain} theorem style to be the default.
% \changes{v2.1b}{90/06/04}{Fixed missing `item by declaring default
%    style.}
%    \begin{macrocode}
\theoremstyle{plain}
%</package>
%    \end{macrocode}
%
% \Finale

\endinput
}{}
\ProvidesFile{thm.sty}
             [\FMithmInfo]
%    \end{macrocode}
%
% \begin{macro}{\th@margin}
%    Again this is only a variant of the theorem styles described
%    above without any new ideas.
%    \begin{macrocode}
\gdef\th@margin{\normalfont\slshape
  \def\@begintheorem##1##2{\item
       [\theorem@headerfont \llap{##2}\hskip\labelsep ##1]}%
\def\@opargbegintheorem##1##2##3{%
  \item[\theorem@headerfont \llap{##2}\hskip\labelsep ##1\ (##3)]}}
\endgroup
%</thm>
%    \end{macrocode}
% \end{macro}
%
%
%
%
% \subsection{Final Definitions}
%
% \begin{macro}{\theorempreskipamount}
% \begin{macro}{\theorempostskipamount}
%    The \textsf{skip} parameters that regulate the vertical empty space
%    before and after the {\envfont theorem} environment have
%    to be allocated as well.
% \changes{v2.0f}{89/07/20}{Unnecessary `global' removed.}
%    \begin{macrocode}
%<*package>
\newskip\theorempreskipamount
\newskip\theorempostskipamount
%    \end{macrocode}
%    Since we have used the same values for all theorem sets, we now can
%    assign them.
%    \begin{macrocode}
\global\setlength\theorempreskipamount{12pt plus 5pt minus 3pt}
\global\setlength\theorempostskipamount{8pt plus 3pt minus 1.5pt}
%    \end{macrocode}
% \end{macro}
% \end{macro}
%
%
% \begin{macro}{\@endtheorem}
%    The same holds for the macro |\@endtheorem|, which ends a
%    {\envfont theorem} environment. Since it is the same for all
%    theorem sets, it is removed from the macros |\th@|\meta{style}.
%    It simply ends the {\envfont trivlist} environment, which was
%    begun in |\@thm|.
%    \begin{macrocode}
\global\let\@endtheorem=\endtrivlist
%    \end{macrocode}
% \end{macro}
%
%
%
% \begin{macro}{\@preamblecmds}
%    All macros defined above are to be used only in the preamble.
%    Therefore, we insert them in |\@preamblecmds| which will disable
%    them at begin document. This is done by the internal \LaTeXe{}
%    command |\@onlypreamble|.
% \changes{v2.0g}{89/09/19}{`theoremheaderfont and `theorembodyfont
%    added.}
% \changes{v2.2a}{94/02/03}{Use \LaTeXe{} interface}
%    \begin{macrocode}
\@onlypreamble\@xnthm
\@onlypreamble\@ynthm
\@onlypreamble\@othm
\@onlypreamble\newtheorem
\@onlypreamble\theoremstyle
\@onlypreamble\theorembodyfont
\@onlypreamble\theoremheaderfont
%    \end{macrocode}
% \end{macro}
%
% Finally we declare the \texttt{plain} theorem style to be the default.
% \changes{v2.1b}{90/06/04}{Fixed missing `item by declaring default
%    style.}
%    \begin{macrocode}
\theoremstyle{plain}
%</package>
%    \end{macrocode}
%
% \Finale

\endinput
}{}
\ProvidesFile{thmb.sty}
             [\FMithmInfo]
%    \end{macrocode}
%
% \begin{macro}{\th@marginbreak}
%    The \texttt{margin break} style is nearly the same as the
%    \texttt{change break} style. The only difference is the placement
%    of the theorem number.  We use |\llap| to place it in the left
%    margin.
%
%    In this style |\labelsep| denotes the separation between the
%    number and the text.
%    \begin{macrocode}
\gdef\th@marginbreak{\normalfont\slshape
  \def\@begintheorem##1##2{\item
       [\rlap{\vbox{\theorem@headerfont
                    \hbox{\llap{##2}\hskip\labelsep ##1}%
                    \hbox{\strut}}}]}%
\def\@opargbegintheorem##1##2##3{%
  \item[\rlap{\vbox{\theorem@headerfont
                    \hbox{\llap{##2}\hskip\labelsep ##1\ (##3)}%
                    \hbox{\strut}}}]}}
\endgroup
%</thmb>
%    \end{macrocode}
% \changes{v2.1c}{90/02/17}{`endgroup outside macrocode env.}
% \end{macro}
%
%
%
% \subsubsection{The \texttt{margin} style}
%
% This style option is stored in the file |thm.sty|.
%    \begin{macrocode}
%<*thm>
\begingroup \makeatletter
\@ifundefined{theorem@style}{% \iffalse meta-comment
%
% Copyright (C) 1993-2024
%
% The LaTeX Project and any individual authors listed elsewhere
% in this file.
%
% This file is part of the Standard LaTeX `Tools Bundle'.
% -------------------------------------------------------
%
% It may be distributed and/or modified under the
% conditions of the LaTeX Project Public License, either version 1.3c
% of this license or (at your option) any later version.
% The latest version of this license is in
%    https://www.latex-project.org/lppl.txt
% and version 1.3c or later is part of all distributions of LaTeX
% version 2005/12/01 or later.
%
% The list of all files belonging to the LaTeX `Tools Bundle' is
% given in the file `manifest.txt'.
%
% \fi
%
%%
%
%
\def\FMithmInfo{2023/07/05 v2.2c Theorem extension package (FMi)}
%
% \ProvidesFile{theorem.dtx}[\FMithmInfo]
%\iffalse   % this is a METACOMMENT !
%%
%% Package `theorem' to use with LaTeX2e
%% Copyright (C) 1989-1995 Frank Mittelbach, all rights reserved.
%<+package>\@ifundefined{theorem@style}{}{\endinput}
%<+package>\NeedsTeXFormat{LaTeX2e}[1995/06/01]
%<+package>\ProvidesPackage{theorem}
%<+package>         [\FMithmInfo]
%
% \fi
%
% \changes{v1.0}{87/05/07}{First draft.}
% \changes{v1.0a}{87/05/07}{`thistheoremstyle' added.}
% \changes{v1.1a}{87/07/29}{New concept: redefine the internal
%                          \protect\LaTeX{} commands `@xnthm' etc.}
% \changes{v1.1b}{87/07/31}{`theoremfont' used to specify font.}
% \changes{v1.2a}{88/01/08}{`@endtheorem' now globally defined.}
% \changes{v1.2b}{88/01/11}{`newtheorem' only allowed in the preamble.}
% \changes{v1.3a}{88/02/14}{`theoremfont' now a toks register.}
%
% \changes{v2.0a}{88/11/16}{New implementation and new documentation.}
% \changes{v2.0a}{88/11/16}{No layout changes but internal modifications
%                           old substyles won't work.}
% \changes{v2.0b}{89/01/30}{`global' used for all definitions.}
% \changes{v2.0c}{89/06/15}{Documentation translated to English.
%                            Thanks to LLL and Nina}
% \changes{v2.1a}{90/02/15}{Support for new font selection.}
% \changes{v2.2a}{94/02/03}{Cleanup for LaTeX2e}
%
% \DoNotIndex{\@endparenv,\@ifdefinable,\@ifnextchar,\@ifundefined}
% \DoNotIndex{\@namedef,\@nameuse,\@spaces,\@tempa,\@warning}
% \DoNotIndex{\ }
% \DoNotIndex{\begingroup,\bfseries}
% \DoNotIndex{\csname}
% \DoNotIndex{\def,\do,\docdate}
% \DoNotIndex{\endcsname,\endgroup,\endinput,\endtrivlist,\expandafter}
% \DoNotIndex{\filedate,\fileversion}
% \DoNotIndex{\gdef,\global}
% \DoNotIndex{\hbox,\hskip}
% \DoNotIndex{\ignorespaces,\input,\itshape,\item}
% \DoNotIndex{\labelsep,\let,\llap}
% \DoNotIndex{\makeatletter}
% \DoNotIndex{\noexpand}
% \DoNotIndex{\refstepcounter,\relax,\rlap}
% \DoNotIndex{\setlength,\slshape,\space,\string,\strut}
% \DoNotIndex{\the,\trivlist,\typeout}
% \DoNotIndex{\vbox}
% \DoNotIndex{\xdef}
%
% \def\deflabel#1{\ttfamily #1\hfill}
% \def\deflist#1{\begin{list}{}{\settowidth\labelwidth{\ttfamily #1}%
%                               \setlength\leftmargin\labelwidth
%                               \addtolength\leftmargin\labelsep
%                               \let\makelabel\deflabel}}
% \def\enddeflist{\end{list}}
% \def\envfont{\normalfont\ttfamily}
%
% \MakeShortVerb{\|}
%
% \newcommand{\TUB}{{\sl TUGboat\/}}
%
% \GetFileInfo{theorem.dtx}
% \title{An Extension of the
%        \LaTeX{} theorem environment\thanks{This file
%        has version number \fileversion, last
%        revised \filedate.}}
% \author{Frank Mittelbach}
%
% \MaintainedByLaTeXTeam{tools}
% \maketitle
%
% \begin{abstract}
%    The macros described in this paper yield an extension of the
%    \LaTeX{} theorem mechanism. It is designed to satisfy the
%    different requirements of various journals. Thus, the
%    layout of the ``theorems'' can be manipulated by determining a
%    ``style''. This article describes not only the use, but
%    also the definition, of the necessary macros.
% \end{abstract}
%
% \section*{Preface to version 2.2}
%
% For \LaTeXe{} this package did not need any fundamental changes. I
% only modified the messages generated so that theorem layout styles
% will show up with the |\listfiles| command and cleaned the section
% on the New Font Selection Scheme since this is now included in
% \LaTeX.
%
% \section*{Preface to version 2.1}
%
% This version is identical to 2.0g described in \TUB\ 10\#3 except
% for some internal defaults which are now set depending on the used
% font selection scheme.
%
% This was done to avoid unpleasant surprises if the new font selection
% scheme is in force. For further details see
% section~\ref{sec:newfonts} and \cite{bk:GMS94}.
%
% \section{Introduction}
%
% For our purposes here, ``theorems'' are labelled enunciations,
% often set off from the main text by extra space and a font change.
% Theorems, corollaries, conjectures, definitions, and remarks
% are all instances of ``theorems''.  The ``header'' of these structures
% is composed of a label (such as \textsc{Theorem} or \textsc{Remark})
% and a number which serializes an item in the sequence of items
% with the same label.
%
% Shortly after the introduction of \LaTeX{} at the Fachbereich
% Mathematik in Mainz, the desire to manipulate the layout of
% ``theorems'' arose.  In Mainz, the following two conventions came
% into general use:
% \begin{enumerate}
%   \item The number of the theorem is shown in the margin.
%   \item There is a line break at
%         the end of the theorem header.
% \end{enumerate}
%
% Additionally, some journals require different formats which depend on
% the ``sort of theorem'': e.g.\ often remarks and definitions are set
% in |\upshape|, while |\itshape| is employed for main theorems.
%
% Confronted with these requirements, a theorem environment was
% developed in Mainz which allows separate determination of the
% layout of the ``theorems sets'', comparable to |\pagestyle|.
%
%
% \section{The user interface}
% \subsection{Defining new theorem sets}
%
% \DescribeMacro\newtheorem
% As in the original \LaTeX{} version, the command |\newtheorem|
% defines a new ``theorem set'' or ``theorem-like structure''.
% Two required arguments name the new environment and give the text
% to be typeset with each instance of the new ``set'', while
% an optional argument determines how
% the ``set'' is enumerated:
% \begin{description}
%    \item[\ttfamily \bslash newtheorem\{foo\}\{bar\}]
%       The theorem set {\envfont foo} (whose name is \texttt{bar})
%       uses its own counter.
%    \item[\ttfamily \bslash newtheorem\{foo2\}{[foo]}\{bar2\}]
%       The theorem set {\envfont foo2} (printed name \texttt{bar2})
%       uses the same counter as the theorem set \texttt{foo}.
%    \item[\ttfamily \bslash newtheorem\{foo3\}\{bar3\}{[section]}]
%       The theorem set {\envfont foo3} (printed name \texttt{bar3}) is
%       enumerated within the counter \texttt{section}, i.e.\ with every
%       new |\section| the enumeration begins again with 1, and
%       the enumeration is composed from the section-number and the
%       theorem counter itself.
% \end{description}
%
% \DescribeMacro\theoremstyle
% Additionally, the command |\theoremstyle| can define the layout
% of various, or all, theorem sets.  It should be noted that any
% theorem set defined by |\newtheorem| is typeset in the
% |\theoremstyle| that is current at the time of the definition.
% Thus, the following
% \begin{verbatim}
%    \theoremstyle{break}        \newtheorem{Cor}{Corollary}
%    \theoremstyle{plain}        \newtheorem{Exa}{Example}[section]
%\end{verbatim}                             ^^A No blank at line start
% leads to the result that the set {\envfont Cor} is formatted in the
% style \texttt{break}, while the set {\envfont Exa} and all the
% following ones are formatted in the style \texttt{plain}, unless
% another |\theoremstyle| follows.  Since the definitions installed by
% |\newtheorem| are global, one also can limit |\theoremstyle| locally
% by grouping braces.
%
% \DescribeMacro\theorembodyfont
% The choice of the font for the theorem body is completely
% independent of the chosen |\theoremstyle|; this has proven to
% be very advantageous.  For example,
% \begin{verbatim}
%   {\theorembodyfont{\upshape}          \newtheorem{Rem}{Remark}}
%\end{verbatim}
% defines a theorem set {\envfont Rem}, which will be set in
% |\upshape| in the current layout (which in our example is
% \texttt{plain}).  As with |\theoremstyle|, the |\theorembodyfont|
% chosen is that current at the time of |\newtheorem|.  If
% |\theorembodyfont| is not specified or one defines
% |\theorembodyfont{}|, then the font used will be that defined by the
% |\theoremstyle|.
%
% \DescribeMacro\theoremheaderfont It is also possible to customize
% the font used for the theorem headers.  This is, however, a global
% declaration, and therefore there should be at most one
% |\theoremheaderfont| declaration in the preamble.\footnote{If it is
% actually necessary to have different header fonts, one has to define
% new theorem styles (substituting the desired font) or specify the
% information directly in the \texttt{\bslash newtheorem} declaration
% (the unclean variant).}
%
% \DescribeMacro\theorempreskipamount
% \DescribeMacro\theorempostskipamount Two additional parameters
% affect the vertical space around the theorem environments:
% |\theorempreskipamount| and |\theorempostskipamount| define,
% respectively, the spacing before and after such an environment.
% These parameters apply for all theorem sets and can be manipulated
% with the ordinary length macros.  They are rubber lengths
% (`\textsf{skips}'), and therefore can contain \texttt{plus} and
% \texttt{minus} parts.
%
% Since the definition of theorem sets should---most sensibly---be
% placed in the preamble, we only allow installation there.
% It is therefore possible to release the memory used here after
% |\begin{document}|, in order to make room for other
% applications.
%
%
% \subsection{Existing theorem styles}
%
% The following theorem styles exist to date:
% \begin{deflist}{marginbreak}
%    \item[plain]
%       This theorem style emulates the original \LaTeX{} definition,
%       except that additionally the parameters
%       |\theorem...skipamount| are used.
%    \item[break]
%       In this style, the theorem header is followed by a line
%       break.
%    \item[marginbreak]
%       The theorem number is set in the margin, and there is a line
%       break as in \texttt{break}.
%    \item[changebreak]
%       Like \texttt{break}, but with header number and text
%       interchanged.
%    \item[change]
%       Header number and text are interchanged, without a line break.
%    \item[margin]
%       The number is set in the left margin, without a line break.
% \end{deflist}
% All styles (except \texttt{plain}) select |\slshape| as the default
% |\theorembodyfont|.
%
%
% \subsection{Examples}
%
% Given the above theorem sets {\envfont Cor}, {\envfont Exa}
% and {\envfont Rem},
% suppose that the preamble also contains the declarations:
% \begin{verbatim}
%    \theoremstyle{marginbreak}   \newtheorem{Lem}[Cor]{Lemma}
%    \theoremstyle{change}
%    \theorembodyfont{\itshape}        \newtheorem{Def}[Cor]{Definition}
%
%    \theoremheaderfont{\scshape}
%\end{verbatim}
% Then the following are some typical examples of the typeset output
% resulting from their use.
%
% \begin{Cor}
%    This is a sentence typeset in the theorem environment {\envfont
%    Cor}.
% \end{Cor}
%
% \begin{Exa}
%    This is a sentence typeset in the theorem environment {\envfont
%    Exa}.
% \end{Exa}
%
% \begin{Rem}
%    This is a sentence typeset in the theorem environment {\envfont
%    Rem}.
% \end{Rem}
%
% \begin{Lem}[Ben User]
%    This is a sentence typeset in the theorem environment {\envfont
%    Lem}.
% \end{Lem}
%
% \begin{Def}[Very impressive Definition]
%    This is a sentence typeset in the theorem environment {\envfont
%    Def}.
% \end{Def}
%
% The last two examples show the effect of the optional argument to
% a theorem environment (it is the text typeset in parentheses).
%
%
% \section{Special Considerations}
% \label{sec:newfonts}
%
% Theoremheader and body are implemented as a unit. This means that
% the |\theoremheaderfont|  will inherit characteristics of the
% |\theorembodyfont| in \LaTeXe. Thus, if for example
% |\theorembodyfont| is |\itshape| and |\theoremheaderfont|
% is |\bfseries| the font selected for the header will have the
% characteristics `bold extended italic'. If this is not desired one
% should set the |\theoremheaderfont| to something like
% \begin{verbatim}
%      \theoremheaderfont{\normalfont\bfseries}
%\end{verbatim}
% i.e.\ supplying all necessary font information explicitly.
%
%
%
% \section{Acknowledgements}
%
% The publication of this set of macros was only possible with the
% help of Christina Busse (translating the manuscript into English),
% Joachim Pense (playing the r\^ole of typist), Chris Rowley
% (looking everything over) and many others providing useful
% suggestions.
%
% \MaybeStop{%
%    \begin{thebibliography}{9}
%       \bibitem{bk:GMS94} \textsc{M.~Goossens}, \textsc{F.~Mittelbach}
%         and \textsc{A.~Samarin}.
%         \newblock The \LaTeX{} Companion.
%         \newblock
%         Addison-Wesley, Reading, Massachusetts, 1994.
%       \bibitem{latex:source} \textsc{Lamport, Leslie.} \newblock
%                               \texttt{latex.tex}, version 2.09, date
%                               Feb. 1990.
%    \end{thebibliography}
%    \PrintIndex
%    \PrintChanges}
%
%
% \section{The documentation driver file}
%
% The next bit of code contains the documentation driver file for
% \TeX{}, i.e., the file that will produce the documentation you are
% currently reading. It will be extracted from this file by the
% \texttt{docstrip} program. Since it is the first code in the file
% one can alternatively process this file directly with \LaTeXe{} to
% obtain the documentation.
%
%    \begin{macrocode}
%<*driver>
\documentclass{ltxdoc}

\usepackage{theorem}

%<+driver>% The next few lines define theorem sets which are used
%<+driver>% in the example section of the documentation.

\theoremstyle{break}       \newtheorem{Cor}{Corollary}
\theoremstyle{plain}       \newtheorem{Exa}{Example}[section]
{\theorembodyfont{\upshape}\newtheorem{Rem}{Remark}}
\theoremstyle{marginbreak} \newtheorem{Lem}[Cor]{Lemma}
\theoremstyle{change}
\theorembodyfont{\itshape} \newtheorem{Def}[Cor]{Definition}

\theoremheaderfont{\scshape}

\RecordChanges

\begin{document}
   \DocInput{theorem.dtx}
\end{document}
%</driver>
%    \end{macrocode}
%
% \section{Definition of the Macros}
%
% If the file has been loaded before, we abort immediately. If not the
% package announces itself (this is actually done at the very top of
% the file---the way it is done isn't good style so don't copy it).
% \changes{v2.0e}{89/07/19}{Spaces removed from `typeout'.}
% \changes{v2.2b}{95/11/19}{Announce moved to top of file}
%    \begin{macrocode}
%<*package>
%\@ifundefined{theorem@style}{}{\endinput}
%\def\FMithmInfo{1995/11/19 v2.2b Theorem extension package (FMi)}
%\ProvidesPackage{theorem}[\FMithmInfo]
%    \end{macrocode}
%
%
% \subsection{Definition of theorem styles and fonts}
%
% All the definitions in this file are done globally to allow
% inputting this file inside a group.
%
% \begin{macro}{\theoremstyle}
%    Before a theorem style can be installed, the chosen style must be
%    known. For that reason, we must test to see that
%    |\th@|\meta{style} is known or, more precisely, that it is
%    different from |\relax|. If the style is not known then
%    |\th@plain| is used.
%    \begin{macrocode}
\gdef\theoremstyle#1{%
   \@ifundefined{th@#1}{\@warning
          {Unknown theoremstyle `#1'. Using `plain'}%
          \theorem@style{plain}}%
%    \end{macrocode}
%    We save the theorem style to be used in the token register
%    |\theorem@style|.
%    \begin{macrocode}
      {\theorem@style{#1}}%
%    \end{macrocode}
%    Now we ``evaluate'' the theorem style: this means, we call the
%    macro |\th@|\meta{style} which will activate the relevant
%    definitions which are contained in a separate file.  This is done
%    in a group to suppress changes to the current font.  This could
%    otherwise pose problems together with the new font selection
%    scheme\footnote{When I printed the original article using the new
%    font selection scheme I ended with a document with slanted
%    typefaces (text headings and all) simply because one of the
%    theorem styles used \texttt{\bslash sl} at toplevel.} if the
%    |\th@|\meta{style} is evaluated a second time.
% \changes{v2.1a}{90/02/15}{Group added around `th@..}
% \changes{v2.1a}{90/02/15}{Reseting of `..@begintheorem removed.}
%    \begin{macrocode}
      \begingroup
        \csname th@\the\theorem@style \endcsname
      \endgroup}
%    \end{macrocode}
% \end{macro}
%
%
% \begin{macro}{\@begintheorem}
% \changes{v2.1a}{90/02/15}{Relaxed at toplevel.}
% \begin{macro}{\@opargbegintheorem}
% \changes{v2.1a}{90/02/15}{Relaxed at toplevel.}
%    We reset |\@begintheorem| and |\@opargbegintheorem| to
%    |\relax| since these commands are no longer necessary at
%    toplevel. This will save a few tokens.
%    \begin{macrocode}
\global\let\@begintheorem\relax
\global\let\@opargbegintheorem\relax
%    \end{macrocode}
% \end{macro}
% \end{macro}
%
% \begin{macro}{\theorem@style}
%    Obviously the \textsf{token} register used above has to be
%    allocated. To assure the utmost compatibility with the
%    original \LaTeX{} definition, we set the default theorem style to
%    \texttt{plain}, which implements the usual \LaTeX{} convention.
% \changes{v2.0f}{89/07/20}{Unnecessary `global' removed.}
%    \begin{macrocode}
\newtoks\theorem@style
\global\theorem@style{plain}
%    \end{macrocode}
% \end{macro}
%
% \begin{macro}{\theorembodyfont}
% \begin{macro}{\theorem@bodyfont}
%    If the |\theorembodyfont| is set by the user then it should not
%    interact with the default font set in the theorem style. When the
%    new font selection is in force this may happen if, for example,
%    the default is |\itshape| and the new |\theorembodyfont| is
%    |\sffamily|. So we add a |\reset@font| command in front of the
%    user definition.
% \changes{v2.0f}{89/07/20}{Unnecessary `global' removed.}
% \changes{v2.0g}{89/09/19}{`theoremfont to `theorembodyfont.}
% \changes{v2.1c}{92/06/22}{`reset@font added and logic changed.}
% \changes{v2.1d}{92/10/28}{Reset to empty if arg is empty}
%    \begin{macrocode}
\gdef\theorembodyfont#1{%
%    \end{macrocode}
%    We check if the argument supplied is empty and if so put nothing
%    into the |\theorem@bodyfont| token register to allow for
%    |\theorembodyfont{}| as a mean of using the default of the
%    current |\theoremstyle|.
%    \begin{macrocode}
   \def\@tempa{#1}%
   \ifx\@tempa\@empty
    \theorem@bodyfont{}%
   \else
    \theorem@bodyfont{\reset@font#1}%
   \fi
   }
\newtoks\theorem@bodyfont
\global\theorem@bodyfont{}
%    \end{macrocode}
% \end{macro}
% \end{macro}
%
%
% \begin{macro}{\theoremheaderfont}
% \changes{v2.0g}{89/09/19}{Macro added.}
%    The font for the theorem headers is handled differently because
%    this definition applies to all theorem styles.
%    \begin{macrocode}
\gdef\theoremheaderfont#1{\gdef\theorem@headerfont{#1}%
%    \end{macrocode}
%    After using the macro once it is redefined to produce an error
%    message.
%    \begin{macrocode}
       \gdef\theoremheaderfont##1{%
        \typeout{\string\theoremheaderfont\space should be used
                 only once.}}}
%    \end{macrocode}
% \end{macro}
%
% \begin{macro}{\theorem@headerfont}
% \changes{v2.0g}{89/09/19}{Macro added.}
%    To set the |\theorem@headerfont| default we first test if
%    the new fontselection scheme is in force.
% \changes{v2.1a}{90/02/15}{Support for new font selection.}
%    \begin{macrocode}
\ifx\upshape\undefined
%    \end{macrocode}
%    If not we define it to expand into |\bfseries|. We don't use
%    |\let| just in case a following style option redefines this
%    macro.
%    \begin{macrocode}
\gdef\theorem@headerfont{\bfseries}
%    \end{macrocode}
%    Otherwise we reset the current shape before calling |\bfseries|.
%    \begin{macrocode}
\else \gdef\theorem@headerfont{\normalfont\bfseries}\fi
%    \end{macrocode}
% \end{macro}
%
% \begin{macro}{\th@plain}
% \begin{macro}{\th@break}
% \begin{macro}{\th@marginbreak}
% \begin{macro}{\th@changebreak}
% \begin{macro}{\th@change}
% \begin{macro}{\th@margin}
%    The different styles are defined in macros such as
%    |\th@plain|.  Since memory space is precious in
%    ``non-Big-versions'', we have to avoid offering too many unused
%    definitions.  Therefore we define these styles in separate files
%    that can be loaded on demand.  Thus the commands themselves only
%    load these files. We use |\@input@|, a \LaTeXe{} internal command
%    that ensures that the file will be listed with |\listfiles|.
% \changes{v2.2a}{94/02/02}{Use `@input@ to load theorem layout files}
%    \begin{macrocode}
\gdef\th@plain{\@input@{thp.sty}}
\gdef\th@break{\@input@{thb.sty}}
\gdef\th@marginbreak{\@input@{thmb.sty}}
\gdef\th@changebreak{\@input@{thcb.sty}}
\gdef\th@change{\@input@{thc.sty}}
\gdef\th@margin{\@input@{thm.sty}}
%    \end{macrocode}
%    This list will be expanded when new styles become
%    available.  For testing,  just append new theorem substyles
%    as document options.
% \end{macro}
% \end{macro}
% \end{macro}
% \end{macro}
% \end{macro}
% \end{macro}
%
%
%
% \subsection{Definition of a new theorem set}
%
% As already pointed out, a new theorem environment can be defined in
% three different ways:
% \begin{verbatim}
%    \newtheorem{Lem}{Lemma}
%    \newtheorem{Lem}{Lemma}[section]
%    \newtheorem{Lem}[Theorem]{Lemma}
%\end{verbatim}
% The function of the macro |\newtheorem| is to recognize these cases
% and then to branch into one of the three macros |\@ynthm|, |\@xnthm|
% or |\@othm|.  This mechanism is adopted unchanged from
% \cite{latex:source}; the essential point here is that, for example,
% in the second case, the arguments \texttt{Lem}, \texttt{Lemma} and
% \texttt{section} are passed over to the macro |\@xnthm|.
%
% We inspect this case first because the others present fewer problems,
% and thus are easily derived from this one.
%
% \begin{macro}{\@xnthm}
%    For our example arguments, the macro |\@xnthm| must fulfill
%    the following:
%    \begin{itemize}
%       \item
%          Define a new \LaTeX-counter `Lem'
%       \item
%          reset this counter within a |\section|
%       \item
%          define the macro |\theLem|
%       \item
%          define the environment macros |\Lem| and
%          |\endLem| using the current
%          |\theoremstyle| and |\theorem@bodyfont|.
%    \end{itemize}
%    Obviously, all this should happen only if the first argument
%    of |\@xnthm| (i.e.\ \texttt{Lem} in our example) is chosen so
%    as not to conflict with any previously defined commands or
%    environments.  This test is performed by the \LaTeX{} macro
%    |\@ifdefinable|.
%    \begin{macrocode}
\gdef\@xnthm#1#2[#3]{\expandafter\@ifdefinable\csname #1\endcsname
%    \end{macrocode}
%    Therefore, the first argument of |\@ifdefinable| is the
%    expansion (in the example, |\Lem|) of
%    |\csname||#1\endcsname|.  The second argument is
%    executed only if the test has been completed successfully.
%    \begin{macrocode}
   {%
%    \end{macrocode}
%    Now we define the new counter. The names of the \LaTeX{}
%    macros employed should speak for themselves:
% \changes{v2.2b}{1995/10/07}{Use \cs{@newctr}}
% \changes{v2.2c}{1995/11/23}{Use \cs{@newctr} properly.}
%    \begin{macrocode}
    \@definecounter{#1}\@newctr{#1}[#3]%
%    \end{macrocode}
%    Using |\@newctr| will give a proper error message if the counter
%    in |#3| is not defined.
%    In defining `|\theLem|' we must generate the desired macro
%    name by use of |\expandafter| and |\csname|.
%    \begin{macrocode}
    \expandafter\xdef\csname the#1\endcsname
%    \end{macrocode}
%    An |\xdef| is used in order to make the definition global,
%    and to ensure that it contains the replacement texts of
%    |\@thmcountersep| and |\@thmcounter|.\footnote{These
%    two macros can be defined by the document style.  Their default
%    values produce a `\texttt{.}' as separation and an arabic
%    representation of the number.} However, not everything should be
%    expanded. For example, it saves space to use |\thesection|
%    instead of its---at times---lengthy expansion.
%    \begin{macrocode}
      {\expandafter \noexpand \csname the#3\endcsname
       \@thmcountersep \@thmcounter{#1}}%
%    \end{macrocode}
%    Thus with the defaults of \LaTeX{}, |\theLem| would be
%    replaced by the command sequence |\thesection.\arabic{Lem}|.
%
%    We will now look at the definition of the macro which is executed
%    at the beginning of the actual environment (in our example this
%    macro is |\Lem|). It should be noted that we use an
%    ``|\expandafter| trick'' to expand only certain parts of the
%    replacement text at the time of the definition.
%    \begin{macrocode}
    \def\@tempa{\global\@namedef{#1}}%
    \expandafter \@tempa \expandafter{%
%    \end{macrocode}
%    First, the macro that contains the current definitions of
%    |\@begintheorem| and |\@opargtheorem| should be called
%    up.  The name of this macro---as is already known---has the form
%    |\th@|\meta{theorem\ style}; therefore, it must be called by
%    \begin{macrocode}
      \csname th@\the \theorem@style
%    \end{macrocode}
%    In addition the default theorem font should be changeable, i.e.\
%    we have to insert the contents of |\theorem@bodyfont|. For
%    that reason, we expand even further, beyond |\endcsname|,
%    and thus insert the contents of the \textsf{token} register
%    |\theorem@bodyfont| in the replacement text.
%    \begin{macrocode}
            \expandafter \endcsname \the \theorem@bodyfont
%    \end{macrocode}
%    Now it is time to call the macro |\@thm| which takes over
%    the further processing.  It has two arguments: the current
%    counter name (in our example, \texttt{Lem}), and the text of the
%    label (in our example, \texttt{Lemma}).
%    \begin{macrocode}
     \@thm{#1}{#2}}%
%    \end{macrocode}
%    With this, the `sub-definition' is complete.  The macro
%    |\@endtheorem| ends a theorem environment and is, so far, nothing
%    but an |\endtrivlist|.  (Hence it is defined globally, and not
%    within the theorem styles.\footnote{This has to be changed as
%    soon as theorem styles that change \texttt{\bslash @endtheorem}
%    exist.  In such a case, all existing styles must be changed as
%    well since they will have to reset the macro.}) Therefore, we can
%    set it equivalent to the macro that ends the theorem set (in our
%    example, |\endLem|).  However, if some day theorem styles exist
%    that do change |\@endtheorem|, we would have to use the
%    commented-out line instead.
%    \begin{macrocode}
    \global \expandafter \let \csname end#1\endcsname \@endtheorem
%   \global\@namedef{end#1}{\@endtheorem}%
%    \end{macrocode}
%    With these commands all the required definitions are employed,
%    unless the test |\@ifdefinable| has failed.  Therefore, we
%    end the second argument of this macro and with it the definition
%    of |\@xnthm|.
%    \begin{macrocode}
   }}
%    \end{macrocode}
% \end{macro}
%
% \begin{macro}{\@ynthm}
%    The definition of |\@ynthm| is completely analogous. In
%    this case the new counter that is defined is not reset within
%    another counter; thus the definition of |\the...|
%    is simplified:
%    \begin{macrocode}
\gdef\@ynthm#1#2{\expandafter\@ifdefinable\csname #1\endcsname
   {\@definecounter{#1}%
    \expandafter\xdef\csname the#1\endcsname{\@thmcounter{#1}}%
%    \end{macrocode}
%    The rest of the definition corresponds literally to that of
%    |\@xnthm|:
%    \begin{macrocode}
    \def\@tempa{\global\@namedef{#1}}\expandafter \@tempa
     \expandafter{\csname th@\the \theorem@style \expandafter
     \endcsname \the\theorem@bodyfont \@thm{#1}{#2}}%
    \global \expandafter \let \csname end#1\endcsname \@endtheorem}}
%    \end{macrocode}
% \end{macro}
%
% \begin{macro}{\@othm}
%    The definition of |\@othm| does not contain anything new.
%    \begin{macrocode}
\gdef\@othm#1[#2]#3{%
%    \end{macrocode}
% \changes{v2.2b}{95/10/07}{Build in check for counter defined}
%    We do not define a new counter but instead use one that has
%    already been defined.  Thus the only definition we need is that
%    of this pseudo-counter (i.e.\ |\the|\meta{env.\ name}).
%    First we check if |#2| corresponds to a known counter name.
%    \begin{macrocode}
  \expandafter\ifx\csname c@#2\endcsname\relax
   \@nocounterr{#2}%
  \else
   \expandafter\@ifdefinable\csname #1\endcsname
   {\expandafter \xdef \csname the#1\endcsname
     {\expandafter \noexpand \csname the#2\endcsname}%
%    \end{macrocode}
%    All other parts of the definition can be adopted from
%    |\@xnthm|. We have to remember, though, that in this case
%    the name of the current counter and the theorem label have moved
%    to the second and third arguments.
%    \begin{macrocode}
    \def\@tempa{\global\@namedef{#1}}\expandafter \@tempa
     \expandafter{\csname th@\the \theorem@style \expandafter
     \endcsname \the\theorem@bodyfont \@thm{#2}{#3}}%
    \global \expandafter \let \csname end#1\endcsname \@endtheorem}%
  \fi}
%    \end{macrocode}
% \end{macro}
%
% \subsection{Macros that are employed in a {\envfont theorem}
%    environment}
%
% \begin{macro}{\@thm}
%    The macro |\@thm| has to increase the current counter.  Then,
%    depending on whether the environment has (or does not have) an
%    optional argument, it has to branch into either |\@begintheorem|
%    or |\@opargtheorem|.
%    \begin{macrocode}
\gdef\@thm#1#2{\refstepcounter{#1}%
%    \end{macrocode}
%    Now we start a {\envfont trivlist} environment, and give
%    |\@topsep| and |\@topsepadd| the values of the \textsf{skip}
%    registers |\theorempreskipamount| and |\theorempostskipamount|.
%    The value in |\@topsep| is the vertical space that is inserted by
%    the first (and only) |\item| in our |\trivlist| whilst
%    |\@topsepadd| is inserted by |\@endparenv| at the end of that
%    {\envfont trivlist} environment.  By using these registers, we
%    obtain the desired space around a {\envfont theorem} environment.
%    \begin{macrocode}
   \trivlist
   \@topsep \theorempreskipamount               % used by first \item
   \@topsepadd \theorempostskipamount           % used by \@endparenv
%    \end{macrocode}
%    Now we have to test whether an optional argument has been given.
% \changes{v2.0d}{89/07/18}{Forgotten percent added.}
%    \begin{macrocode}
   \@ifnextchar [%
%    \end{macrocode}
%    If there is an optional argument, we will call |\@ythm|, and
%    move the arguments read back into the input stream.
%    \begin{macrocode}
   {\@ythm{#1}{#2}}%
%    \end{macrocode}
%    If not, we call |\@begintheorem|. Its first argument is the
%    name of the theorem set (hence the second argument of
%    |\@thm|). Its second argument is the macro that produces the
%    current number.
%    \begin{macrocode}
   {\@begintheorem{#2}{\csname the#1\endcsname}\ignorespaces}}
%    \end{macrocode}
% \end{macro}
%
% \begin{macro}{\@xthm}
% \begin{macro}{\@ythm}
%    Both these macros were originally called by |\@thm|.  We do
%    not need |\@xthm| anymore, hence we reset it to
%    |\relax|. The definition of |\@ythm| has not changed at
%    all from its definition in \LaTeX. In order to make the macros
%    easier to understand, we will
%    nevertheless present it (commented out).
%    \begin{macrocode}
\global\let\@xthm\relax
% \def\@ythm#1#2[#3]{\@opargbegintheorem{#2}{\csname
%           the#1\endcsname}{#3}\ignorespaces}
%</package>
%    \end{macrocode}
%    The primitive |\ignorespaces| in |\@ythm| and
%    |\@thm| is needed to remove the spaces between the
%    |\begin{...}| and the actual text.
% \end{macro}
% \end{macro}
%
%
% \subsection{Definition of the theorem substyles}
%
% As already pointed out, the theorem substyles, defined below, are
% only loaded when necessary.  Note that all these substyles, except
% \texttt{plain}, have |\slshape| as the default body font.
%
% \subsubsection{The \texttt{plain} style}
%
% \label{dok:th@plain} As the following macros use |@|, we have
% to locally set the |\catcode| of this symbol to ``letter''.
% This happens within a group, so that we do not have to worry about
% which |\catcode| that symbol had before.
%    \begin{macrocode}
%<*thp>
\begingroup \makeatletter
%    \end{macrocode}
% Since we are now within a group, we must make all definitions
% globally.  First we make sure that \texttt{theorem.sty} is loaded.
% This will allow us to use this file as a document style option
% without having to call \texttt{theorem} itself as an option.  At the
% same time, we assure that at least version 2 is loaded, since
% |\theorem@style| was not defined in earlier versions.
%    \begin{macrocode}
\@ifundefined{theorem@style}{% \iffalse meta-comment
%
% Copyright (C) 1993-2024
%
% The LaTeX Project and any individual authors listed elsewhere
% in this file.
%
% This file is part of the Standard LaTeX `Tools Bundle'.
% -------------------------------------------------------
%
% It may be distributed and/or modified under the
% conditions of the LaTeX Project Public License, either version 1.3c
% of this license or (at your option) any later version.
% The latest version of this license is in
%    https://www.latex-project.org/lppl.txt
% and version 1.3c or later is part of all distributions of LaTeX
% version 2005/12/01 or later.
%
% The list of all files belonging to the LaTeX `Tools Bundle' is
% given in the file `manifest.txt'.
%
% \fi
%
%%
%
%
\def\FMithmInfo{2023/07/05 v2.2c Theorem extension package (FMi)}
%
% \ProvidesFile{theorem.dtx}[\FMithmInfo]
%\iffalse   % this is a METACOMMENT !
%%
%% Package `theorem' to use with LaTeX2e
%% Copyright (C) 1989-1995 Frank Mittelbach, all rights reserved.
%<+package>\@ifundefined{theorem@style}{}{\endinput}
%<+package>\NeedsTeXFormat{LaTeX2e}[1995/06/01]
%<+package>\ProvidesPackage{theorem}
%<+package>         [\FMithmInfo]
%
% \fi
%
% \changes{v1.0}{87/05/07}{First draft.}
% \changes{v1.0a}{87/05/07}{`thistheoremstyle' added.}
% \changes{v1.1a}{87/07/29}{New concept: redefine the internal
%                          \protect\LaTeX{} commands `@xnthm' etc.}
% \changes{v1.1b}{87/07/31}{`theoremfont' used to specify font.}
% \changes{v1.2a}{88/01/08}{`@endtheorem' now globally defined.}
% \changes{v1.2b}{88/01/11}{`newtheorem' only allowed in the preamble.}
% \changes{v1.3a}{88/02/14}{`theoremfont' now a toks register.}
%
% \changes{v2.0a}{88/11/16}{New implementation and new documentation.}
% \changes{v2.0a}{88/11/16}{No layout changes but internal modifications
%                           old substyles won't work.}
% \changes{v2.0b}{89/01/30}{`global' used for all definitions.}
% \changes{v2.0c}{89/06/15}{Documentation translated to English.
%                            Thanks to LLL and Nina}
% \changes{v2.1a}{90/02/15}{Support for new font selection.}
% \changes{v2.2a}{94/02/03}{Cleanup for LaTeX2e}
%
% \DoNotIndex{\@endparenv,\@ifdefinable,\@ifnextchar,\@ifundefined}
% \DoNotIndex{\@namedef,\@nameuse,\@spaces,\@tempa,\@warning}
% \DoNotIndex{\ }
% \DoNotIndex{\begingroup,\bfseries}
% \DoNotIndex{\csname}
% \DoNotIndex{\def,\do,\docdate}
% \DoNotIndex{\endcsname,\endgroup,\endinput,\endtrivlist,\expandafter}
% \DoNotIndex{\filedate,\fileversion}
% \DoNotIndex{\gdef,\global}
% \DoNotIndex{\hbox,\hskip}
% \DoNotIndex{\ignorespaces,\input,\itshape,\item}
% \DoNotIndex{\labelsep,\let,\llap}
% \DoNotIndex{\makeatletter}
% \DoNotIndex{\noexpand}
% \DoNotIndex{\refstepcounter,\relax,\rlap}
% \DoNotIndex{\setlength,\slshape,\space,\string,\strut}
% \DoNotIndex{\the,\trivlist,\typeout}
% \DoNotIndex{\vbox}
% \DoNotIndex{\xdef}
%
% \def\deflabel#1{\ttfamily #1\hfill}
% \def\deflist#1{\begin{list}{}{\settowidth\labelwidth{\ttfamily #1}%
%                               \setlength\leftmargin\labelwidth
%                               \addtolength\leftmargin\labelsep
%                               \let\makelabel\deflabel}}
% \def\enddeflist{\end{list}}
% \def\envfont{\normalfont\ttfamily}
%
% \MakeShortVerb{\|}
%
% \newcommand{\TUB}{{\sl TUGboat\/}}
%
% \GetFileInfo{theorem.dtx}
% \title{An Extension of the
%        \LaTeX{} theorem environment\thanks{This file
%        has version number \fileversion, last
%        revised \filedate.}}
% \author{Frank Mittelbach}
%
% \MaintainedByLaTeXTeam{tools}
% \maketitle
%
% \begin{abstract}
%    The macros described in this paper yield an extension of the
%    \LaTeX{} theorem mechanism. It is designed to satisfy the
%    different requirements of various journals. Thus, the
%    layout of the ``theorems'' can be manipulated by determining a
%    ``style''. This article describes not only the use, but
%    also the definition, of the necessary macros.
% \end{abstract}
%
% \section*{Preface to version 2.2}
%
% For \LaTeXe{} this package did not need any fundamental changes. I
% only modified the messages generated so that theorem layout styles
% will show up with the |\listfiles| command and cleaned the section
% on the New Font Selection Scheme since this is now included in
% \LaTeX.
%
% \section*{Preface to version 2.1}
%
% This version is identical to 2.0g described in \TUB\ 10\#3 except
% for some internal defaults which are now set depending on the used
% font selection scheme.
%
% This was done to avoid unpleasant surprises if the new font selection
% scheme is in force. For further details see
% section~\ref{sec:newfonts} and \cite{bk:GMS94}.
%
% \section{Introduction}
%
% For our purposes here, ``theorems'' are labelled enunciations,
% often set off from the main text by extra space and a font change.
% Theorems, corollaries, conjectures, definitions, and remarks
% are all instances of ``theorems''.  The ``header'' of these structures
% is composed of a label (such as \textsc{Theorem} or \textsc{Remark})
% and a number which serializes an item in the sequence of items
% with the same label.
%
% Shortly after the introduction of \LaTeX{} at the Fachbereich
% Mathematik in Mainz, the desire to manipulate the layout of
% ``theorems'' arose.  In Mainz, the following two conventions came
% into general use:
% \begin{enumerate}
%   \item The number of the theorem is shown in the margin.
%   \item There is a line break at
%         the end of the theorem header.
% \end{enumerate}
%
% Additionally, some journals require different formats which depend on
% the ``sort of theorem'': e.g.\ often remarks and definitions are set
% in |\upshape|, while |\itshape| is employed for main theorems.
%
% Confronted with these requirements, a theorem environment was
% developed in Mainz which allows separate determination of the
% layout of the ``theorems sets'', comparable to |\pagestyle|.
%
%
% \section{The user interface}
% \subsection{Defining new theorem sets}
%
% \DescribeMacro\newtheorem
% As in the original \LaTeX{} version, the command |\newtheorem|
% defines a new ``theorem set'' or ``theorem-like structure''.
% Two required arguments name the new environment and give the text
% to be typeset with each instance of the new ``set'', while
% an optional argument determines how
% the ``set'' is enumerated:
% \begin{description}
%    \item[\ttfamily \bslash newtheorem\{foo\}\{bar\}]
%       The theorem set {\envfont foo} (whose name is \texttt{bar})
%       uses its own counter.
%    \item[\ttfamily \bslash newtheorem\{foo2\}{[foo]}\{bar2\}]
%       The theorem set {\envfont foo2} (printed name \texttt{bar2})
%       uses the same counter as the theorem set \texttt{foo}.
%    \item[\ttfamily \bslash newtheorem\{foo3\}\{bar3\}{[section]}]
%       The theorem set {\envfont foo3} (printed name \texttt{bar3}) is
%       enumerated within the counter \texttt{section}, i.e.\ with every
%       new |\section| the enumeration begins again with 1, and
%       the enumeration is composed from the section-number and the
%       theorem counter itself.
% \end{description}
%
% \DescribeMacro\theoremstyle
% Additionally, the command |\theoremstyle| can define the layout
% of various, or all, theorem sets.  It should be noted that any
% theorem set defined by |\newtheorem| is typeset in the
% |\theoremstyle| that is current at the time of the definition.
% Thus, the following
% \begin{verbatim}
%    \theoremstyle{break}        \newtheorem{Cor}{Corollary}
%    \theoremstyle{plain}        \newtheorem{Exa}{Example}[section]
%\end{verbatim}                             ^^A No blank at line start
% leads to the result that the set {\envfont Cor} is formatted in the
% style \texttt{break}, while the set {\envfont Exa} and all the
% following ones are formatted in the style \texttt{plain}, unless
% another |\theoremstyle| follows.  Since the definitions installed by
% |\newtheorem| are global, one also can limit |\theoremstyle| locally
% by grouping braces.
%
% \DescribeMacro\theorembodyfont
% The choice of the font for the theorem body is completely
% independent of the chosen |\theoremstyle|; this has proven to
% be very advantageous.  For example,
% \begin{verbatim}
%   {\theorembodyfont{\upshape}          \newtheorem{Rem}{Remark}}
%\end{verbatim}
% defines a theorem set {\envfont Rem}, which will be set in
% |\upshape| in the current layout (which in our example is
% \texttt{plain}).  As with |\theoremstyle|, the |\theorembodyfont|
% chosen is that current at the time of |\newtheorem|.  If
% |\theorembodyfont| is not specified or one defines
% |\theorembodyfont{}|, then the font used will be that defined by the
% |\theoremstyle|.
%
% \DescribeMacro\theoremheaderfont It is also possible to customize
% the font used for the theorem headers.  This is, however, a global
% declaration, and therefore there should be at most one
% |\theoremheaderfont| declaration in the preamble.\footnote{If it is
% actually necessary to have different header fonts, one has to define
% new theorem styles (substituting the desired font) or specify the
% information directly in the \texttt{\bslash newtheorem} declaration
% (the unclean variant).}
%
% \DescribeMacro\theorempreskipamount
% \DescribeMacro\theorempostskipamount Two additional parameters
% affect the vertical space around the theorem environments:
% |\theorempreskipamount| and |\theorempostskipamount| define,
% respectively, the spacing before and after such an environment.
% These parameters apply for all theorem sets and can be manipulated
% with the ordinary length macros.  They are rubber lengths
% (`\textsf{skips}'), and therefore can contain \texttt{plus} and
% \texttt{minus} parts.
%
% Since the definition of theorem sets should---most sensibly---be
% placed in the preamble, we only allow installation there.
% It is therefore possible to release the memory used here after
% |\begin{document}|, in order to make room for other
% applications.
%
%
% \subsection{Existing theorem styles}
%
% The following theorem styles exist to date:
% \begin{deflist}{marginbreak}
%    \item[plain]
%       This theorem style emulates the original \LaTeX{} definition,
%       except that additionally the parameters
%       |\theorem...skipamount| are used.
%    \item[break]
%       In this style, the theorem header is followed by a line
%       break.
%    \item[marginbreak]
%       The theorem number is set in the margin, and there is a line
%       break as in \texttt{break}.
%    \item[changebreak]
%       Like \texttt{break}, but with header number and text
%       interchanged.
%    \item[change]
%       Header number and text are interchanged, without a line break.
%    \item[margin]
%       The number is set in the left margin, without a line break.
% \end{deflist}
% All styles (except \texttt{plain}) select |\slshape| as the default
% |\theorembodyfont|.
%
%
% \subsection{Examples}
%
% Given the above theorem sets {\envfont Cor}, {\envfont Exa}
% and {\envfont Rem},
% suppose that the preamble also contains the declarations:
% \begin{verbatim}
%    \theoremstyle{marginbreak}   \newtheorem{Lem}[Cor]{Lemma}
%    \theoremstyle{change}
%    \theorembodyfont{\itshape}        \newtheorem{Def}[Cor]{Definition}
%
%    \theoremheaderfont{\scshape}
%\end{verbatim}
% Then the following are some typical examples of the typeset output
% resulting from their use.
%
% \begin{Cor}
%    This is a sentence typeset in the theorem environment {\envfont
%    Cor}.
% \end{Cor}
%
% \begin{Exa}
%    This is a sentence typeset in the theorem environment {\envfont
%    Exa}.
% \end{Exa}
%
% \begin{Rem}
%    This is a sentence typeset in the theorem environment {\envfont
%    Rem}.
% \end{Rem}
%
% \begin{Lem}[Ben User]
%    This is a sentence typeset in the theorem environment {\envfont
%    Lem}.
% \end{Lem}
%
% \begin{Def}[Very impressive Definition]
%    This is a sentence typeset in the theorem environment {\envfont
%    Def}.
% \end{Def}
%
% The last two examples show the effect of the optional argument to
% a theorem environment (it is the text typeset in parentheses).
%
%
% \section{Special Considerations}
% \label{sec:newfonts}
%
% Theoremheader and body are implemented as a unit. This means that
% the |\theoremheaderfont|  will inherit characteristics of the
% |\theorembodyfont| in \LaTeXe. Thus, if for example
% |\theorembodyfont| is |\itshape| and |\theoremheaderfont|
% is |\bfseries| the font selected for the header will have the
% characteristics `bold extended italic'. If this is not desired one
% should set the |\theoremheaderfont| to something like
% \begin{verbatim}
%      \theoremheaderfont{\normalfont\bfseries}
%\end{verbatim}
% i.e.\ supplying all necessary font information explicitly.
%
%
%
% \section{Acknowledgements}
%
% The publication of this set of macros was only possible with the
% help of Christina Busse (translating the manuscript into English),
% Joachim Pense (playing the r\^ole of typist), Chris Rowley
% (looking everything over) and many others providing useful
% suggestions.
%
% \MaybeStop{%
%    \begin{thebibliography}{9}
%       \bibitem{bk:GMS94} \textsc{M.~Goossens}, \textsc{F.~Mittelbach}
%         and \textsc{A.~Samarin}.
%         \newblock The \LaTeX{} Companion.
%         \newblock
%         Addison-Wesley, Reading, Massachusetts, 1994.
%       \bibitem{latex:source} \textsc{Lamport, Leslie.} \newblock
%                               \texttt{latex.tex}, version 2.09, date
%                               Feb. 1990.
%    \end{thebibliography}
%    \PrintIndex
%    \PrintChanges}
%
%
% \section{The documentation driver file}
%
% The next bit of code contains the documentation driver file for
% \TeX{}, i.e., the file that will produce the documentation you are
% currently reading. It will be extracted from this file by the
% \texttt{docstrip} program. Since it is the first code in the file
% one can alternatively process this file directly with \LaTeXe{} to
% obtain the documentation.
%
%    \begin{macrocode}
%<*driver>
\documentclass{ltxdoc}

\usepackage{theorem}

%<+driver>% The next few lines define theorem sets which are used
%<+driver>% in the example section of the documentation.

\theoremstyle{break}       \newtheorem{Cor}{Corollary}
\theoremstyle{plain}       \newtheorem{Exa}{Example}[section]
{\theorembodyfont{\upshape}\newtheorem{Rem}{Remark}}
\theoremstyle{marginbreak} \newtheorem{Lem}[Cor]{Lemma}
\theoremstyle{change}
\theorembodyfont{\itshape} \newtheorem{Def}[Cor]{Definition}

\theoremheaderfont{\scshape}

\RecordChanges

\begin{document}
   \DocInput{theorem.dtx}
\end{document}
%</driver>
%    \end{macrocode}
%
% \section{Definition of the Macros}
%
% If the file has been loaded before, we abort immediately. If not the
% package announces itself (this is actually done at the very top of
% the file---the way it is done isn't good style so don't copy it).
% \changes{v2.0e}{89/07/19}{Spaces removed from `typeout'.}
% \changes{v2.2b}{95/11/19}{Announce moved to top of file}
%    \begin{macrocode}
%<*package>
%\@ifundefined{theorem@style}{}{\endinput}
%\def\FMithmInfo{1995/11/19 v2.2b Theorem extension package (FMi)}
%\ProvidesPackage{theorem}[\FMithmInfo]
%    \end{macrocode}
%
%
% \subsection{Definition of theorem styles and fonts}
%
% All the definitions in this file are done globally to allow
% inputting this file inside a group.
%
% \begin{macro}{\theoremstyle}
%    Before a theorem style can be installed, the chosen style must be
%    known. For that reason, we must test to see that
%    |\th@|\meta{style} is known or, more precisely, that it is
%    different from |\relax|. If the style is not known then
%    |\th@plain| is used.
%    \begin{macrocode}
\gdef\theoremstyle#1{%
   \@ifundefined{th@#1}{\@warning
          {Unknown theoremstyle `#1'. Using `plain'}%
          \theorem@style{plain}}%
%    \end{macrocode}
%    We save the theorem style to be used in the token register
%    |\theorem@style|.
%    \begin{macrocode}
      {\theorem@style{#1}}%
%    \end{macrocode}
%    Now we ``evaluate'' the theorem style: this means, we call the
%    macro |\th@|\meta{style} which will activate the relevant
%    definitions which are contained in a separate file.  This is done
%    in a group to suppress changes to the current font.  This could
%    otherwise pose problems together with the new font selection
%    scheme\footnote{When I printed the original article using the new
%    font selection scheme I ended with a document with slanted
%    typefaces (text headings and all) simply because one of the
%    theorem styles used \texttt{\bslash sl} at toplevel.} if the
%    |\th@|\meta{style} is evaluated a second time.
% \changes{v2.1a}{90/02/15}{Group added around `th@..}
% \changes{v2.1a}{90/02/15}{Reseting of `..@begintheorem removed.}
%    \begin{macrocode}
      \begingroup
        \csname th@\the\theorem@style \endcsname
      \endgroup}
%    \end{macrocode}
% \end{macro}
%
%
% \begin{macro}{\@begintheorem}
% \changes{v2.1a}{90/02/15}{Relaxed at toplevel.}
% \begin{macro}{\@opargbegintheorem}
% \changes{v2.1a}{90/02/15}{Relaxed at toplevel.}
%    We reset |\@begintheorem| and |\@opargbegintheorem| to
%    |\relax| since these commands are no longer necessary at
%    toplevel. This will save a few tokens.
%    \begin{macrocode}
\global\let\@begintheorem\relax
\global\let\@opargbegintheorem\relax
%    \end{macrocode}
% \end{macro}
% \end{macro}
%
% \begin{macro}{\theorem@style}
%    Obviously the \textsf{token} register used above has to be
%    allocated. To assure the utmost compatibility with the
%    original \LaTeX{} definition, we set the default theorem style to
%    \texttt{plain}, which implements the usual \LaTeX{} convention.
% \changes{v2.0f}{89/07/20}{Unnecessary `global' removed.}
%    \begin{macrocode}
\newtoks\theorem@style
\global\theorem@style{plain}
%    \end{macrocode}
% \end{macro}
%
% \begin{macro}{\theorembodyfont}
% \begin{macro}{\theorem@bodyfont}
%    If the |\theorembodyfont| is set by the user then it should not
%    interact with the default font set in the theorem style. When the
%    new font selection is in force this may happen if, for example,
%    the default is |\itshape| and the new |\theorembodyfont| is
%    |\sffamily|. So we add a |\reset@font| command in front of the
%    user definition.
% \changes{v2.0f}{89/07/20}{Unnecessary `global' removed.}
% \changes{v2.0g}{89/09/19}{`theoremfont to `theorembodyfont.}
% \changes{v2.1c}{92/06/22}{`reset@font added and logic changed.}
% \changes{v2.1d}{92/10/28}{Reset to empty if arg is empty}
%    \begin{macrocode}
\gdef\theorembodyfont#1{%
%    \end{macrocode}
%    We check if the argument supplied is empty and if so put nothing
%    into the |\theorem@bodyfont| token register to allow for
%    |\theorembodyfont{}| as a mean of using the default of the
%    current |\theoremstyle|.
%    \begin{macrocode}
   \def\@tempa{#1}%
   \ifx\@tempa\@empty
    \theorem@bodyfont{}%
   \else
    \theorem@bodyfont{\reset@font#1}%
   \fi
   }
\newtoks\theorem@bodyfont
\global\theorem@bodyfont{}
%    \end{macrocode}
% \end{macro}
% \end{macro}
%
%
% \begin{macro}{\theoremheaderfont}
% \changes{v2.0g}{89/09/19}{Macro added.}
%    The font for the theorem headers is handled differently because
%    this definition applies to all theorem styles.
%    \begin{macrocode}
\gdef\theoremheaderfont#1{\gdef\theorem@headerfont{#1}%
%    \end{macrocode}
%    After using the macro once it is redefined to produce an error
%    message.
%    \begin{macrocode}
       \gdef\theoremheaderfont##1{%
        \typeout{\string\theoremheaderfont\space should be used
                 only once.}}}
%    \end{macrocode}
% \end{macro}
%
% \begin{macro}{\theorem@headerfont}
% \changes{v2.0g}{89/09/19}{Macro added.}
%    To set the |\theorem@headerfont| default we first test if
%    the new fontselection scheme is in force.
% \changes{v2.1a}{90/02/15}{Support for new font selection.}
%    \begin{macrocode}
\ifx\upshape\undefined
%    \end{macrocode}
%    If not we define it to expand into |\bfseries|. We don't use
%    |\let| just in case a following style option redefines this
%    macro.
%    \begin{macrocode}
\gdef\theorem@headerfont{\bfseries}
%    \end{macrocode}
%    Otherwise we reset the current shape before calling |\bfseries|.
%    \begin{macrocode}
\else \gdef\theorem@headerfont{\normalfont\bfseries}\fi
%    \end{macrocode}
% \end{macro}
%
% \begin{macro}{\th@plain}
% \begin{macro}{\th@break}
% \begin{macro}{\th@marginbreak}
% \begin{macro}{\th@changebreak}
% \begin{macro}{\th@change}
% \begin{macro}{\th@margin}
%    The different styles are defined in macros such as
%    |\th@plain|.  Since memory space is precious in
%    ``non-Big-versions'', we have to avoid offering too many unused
%    definitions.  Therefore we define these styles in separate files
%    that can be loaded on demand.  Thus the commands themselves only
%    load these files. We use |\@input@|, a \LaTeXe{} internal command
%    that ensures that the file will be listed with |\listfiles|.
% \changes{v2.2a}{94/02/02}{Use `@input@ to load theorem layout files}
%    \begin{macrocode}
\gdef\th@plain{\@input@{thp.sty}}
\gdef\th@break{\@input@{thb.sty}}
\gdef\th@marginbreak{\@input@{thmb.sty}}
\gdef\th@changebreak{\@input@{thcb.sty}}
\gdef\th@change{\@input@{thc.sty}}
\gdef\th@margin{\@input@{thm.sty}}
%    \end{macrocode}
%    This list will be expanded when new styles become
%    available.  For testing,  just append new theorem substyles
%    as document options.
% \end{macro}
% \end{macro}
% \end{macro}
% \end{macro}
% \end{macro}
% \end{macro}
%
%
%
% \subsection{Definition of a new theorem set}
%
% As already pointed out, a new theorem environment can be defined in
% three different ways:
% \begin{verbatim}
%    \newtheorem{Lem}{Lemma}
%    \newtheorem{Lem}{Lemma}[section]
%    \newtheorem{Lem}[Theorem]{Lemma}
%\end{verbatim}
% The function of the macro |\newtheorem| is to recognize these cases
% and then to branch into one of the three macros |\@ynthm|, |\@xnthm|
% or |\@othm|.  This mechanism is adopted unchanged from
% \cite{latex:source}; the essential point here is that, for example,
% in the second case, the arguments \texttt{Lem}, \texttt{Lemma} and
% \texttt{section} are passed over to the macro |\@xnthm|.
%
% We inspect this case first because the others present fewer problems,
% and thus are easily derived from this one.
%
% \begin{macro}{\@xnthm}
%    For our example arguments, the macro |\@xnthm| must fulfill
%    the following:
%    \begin{itemize}
%       \item
%          Define a new \LaTeX-counter `Lem'
%       \item
%          reset this counter within a |\section|
%       \item
%          define the macro |\theLem|
%       \item
%          define the environment macros |\Lem| and
%          |\endLem| using the current
%          |\theoremstyle| and |\theorem@bodyfont|.
%    \end{itemize}
%    Obviously, all this should happen only if the first argument
%    of |\@xnthm| (i.e.\ \texttt{Lem} in our example) is chosen so
%    as not to conflict with any previously defined commands or
%    environments.  This test is performed by the \LaTeX{} macro
%    |\@ifdefinable|.
%    \begin{macrocode}
\gdef\@xnthm#1#2[#3]{\expandafter\@ifdefinable\csname #1\endcsname
%    \end{macrocode}
%    Therefore, the first argument of |\@ifdefinable| is the
%    expansion (in the example, |\Lem|) of
%    |\csname||#1\endcsname|.  The second argument is
%    executed only if the test has been completed successfully.
%    \begin{macrocode}
   {%
%    \end{macrocode}
%    Now we define the new counter. The names of the \LaTeX{}
%    macros employed should speak for themselves:
% \changes{v2.2b}{1995/10/07}{Use \cs{@newctr}}
% \changes{v2.2c}{1995/11/23}{Use \cs{@newctr} properly.}
%    \begin{macrocode}
    \@definecounter{#1}\@newctr{#1}[#3]%
%    \end{macrocode}
%    Using |\@newctr| will give a proper error message if the counter
%    in |#3| is not defined.
%    In defining `|\theLem|' we must generate the desired macro
%    name by use of |\expandafter| and |\csname|.
%    \begin{macrocode}
    \expandafter\xdef\csname the#1\endcsname
%    \end{macrocode}
%    An |\xdef| is used in order to make the definition global,
%    and to ensure that it contains the replacement texts of
%    |\@thmcountersep| and |\@thmcounter|.\footnote{These
%    two macros can be defined by the document style.  Their default
%    values produce a `\texttt{.}' as separation and an arabic
%    representation of the number.} However, not everything should be
%    expanded. For example, it saves space to use |\thesection|
%    instead of its---at times---lengthy expansion.
%    \begin{macrocode}
      {\expandafter \noexpand \csname the#3\endcsname
       \@thmcountersep \@thmcounter{#1}}%
%    \end{macrocode}
%    Thus with the defaults of \LaTeX{}, |\theLem| would be
%    replaced by the command sequence |\thesection.\arabic{Lem}|.
%
%    We will now look at the definition of the macro which is executed
%    at the beginning of the actual environment (in our example this
%    macro is |\Lem|). It should be noted that we use an
%    ``|\expandafter| trick'' to expand only certain parts of the
%    replacement text at the time of the definition.
%    \begin{macrocode}
    \def\@tempa{\global\@namedef{#1}}%
    \expandafter \@tempa \expandafter{%
%    \end{macrocode}
%    First, the macro that contains the current definitions of
%    |\@begintheorem| and |\@opargtheorem| should be called
%    up.  The name of this macro---as is already known---has the form
%    |\th@|\meta{theorem\ style}; therefore, it must be called by
%    \begin{macrocode}
      \csname th@\the \theorem@style
%    \end{macrocode}
%    In addition the default theorem font should be changeable, i.e.\
%    we have to insert the contents of |\theorem@bodyfont|. For
%    that reason, we expand even further, beyond |\endcsname|,
%    and thus insert the contents of the \textsf{token} register
%    |\theorem@bodyfont| in the replacement text.
%    \begin{macrocode}
            \expandafter \endcsname \the \theorem@bodyfont
%    \end{macrocode}
%    Now it is time to call the macro |\@thm| which takes over
%    the further processing.  It has two arguments: the current
%    counter name (in our example, \texttt{Lem}), and the text of the
%    label (in our example, \texttt{Lemma}).
%    \begin{macrocode}
     \@thm{#1}{#2}}%
%    \end{macrocode}
%    With this, the `sub-definition' is complete.  The macro
%    |\@endtheorem| ends a theorem environment and is, so far, nothing
%    but an |\endtrivlist|.  (Hence it is defined globally, and not
%    within the theorem styles.\footnote{This has to be changed as
%    soon as theorem styles that change \texttt{\bslash @endtheorem}
%    exist.  In such a case, all existing styles must be changed as
%    well since they will have to reset the macro.}) Therefore, we can
%    set it equivalent to the macro that ends the theorem set (in our
%    example, |\endLem|).  However, if some day theorem styles exist
%    that do change |\@endtheorem|, we would have to use the
%    commented-out line instead.
%    \begin{macrocode}
    \global \expandafter \let \csname end#1\endcsname \@endtheorem
%   \global\@namedef{end#1}{\@endtheorem}%
%    \end{macrocode}
%    With these commands all the required definitions are employed,
%    unless the test |\@ifdefinable| has failed.  Therefore, we
%    end the second argument of this macro and with it the definition
%    of |\@xnthm|.
%    \begin{macrocode}
   }}
%    \end{macrocode}
% \end{macro}
%
% \begin{macro}{\@ynthm}
%    The definition of |\@ynthm| is completely analogous. In
%    this case the new counter that is defined is not reset within
%    another counter; thus the definition of |\the...|
%    is simplified:
%    \begin{macrocode}
\gdef\@ynthm#1#2{\expandafter\@ifdefinable\csname #1\endcsname
   {\@definecounter{#1}%
    \expandafter\xdef\csname the#1\endcsname{\@thmcounter{#1}}%
%    \end{macrocode}
%    The rest of the definition corresponds literally to that of
%    |\@xnthm|:
%    \begin{macrocode}
    \def\@tempa{\global\@namedef{#1}}\expandafter \@tempa
     \expandafter{\csname th@\the \theorem@style \expandafter
     \endcsname \the\theorem@bodyfont \@thm{#1}{#2}}%
    \global \expandafter \let \csname end#1\endcsname \@endtheorem}}
%    \end{macrocode}
% \end{macro}
%
% \begin{macro}{\@othm}
%    The definition of |\@othm| does not contain anything new.
%    \begin{macrocode}
\gdef\@othm#1[#2]#3{%
%    \end{macrocode}
% \changes{v2.2b}{95/10/07}{Build in check for counter defined}
%    We do not define a new counter but instead use one that has
%    already been defined.  Thus the only definition we need is that
%    of this pseudo-counter (i.e.\ |\the|\meta{env.\ name}).
%    First we check if |#2| corresponds to a known counter name.
%    \begin{macrocode}
  \expandafter\ifx\csname c@#2\endcsname\relax
   \@nocounterr{#2}%
  \else
   \expandafter\@ifdefinable\csname #1\endcsname
   {\expandafter \xdef \csname the#1\endcsname
     {\expandafter \noexpand \csname the#2\endcsname}%
%    \end{macrocode}
%    All other parts of the definition can be adopted from
%    |\@xnthm|. We have to remember, though, that in this case
%    the name of the current counter and the theorem label have moved
%    to the second and third arguments.
%    \begin{macrocode}
    \def\@tempa{\global\@namedef{#1}}\expandafter \@tempa
     \expandafter{\csname th@\the \theorem@style \expandafter
     \endcsname \the\theorem@bodyfont \@thm{#2}{#3}}%
    \global \expandafter \let \csname end#1\endcsname \@endtheorem}%
  \fi}
%    \end{macrocode}
% \end{macro}
%
% \subsection{Macros that are employed in a {\envfont theorem}
%    environment}
%
% \begin{macro}{\@thm}
%    The macro |\@thm| has to increase the current counter.  Then,
%    depending on whether the environment has (or does not have) an
%    optional argument, it has to branch into either |\@begintheorem|
%    or |\@opargtheorem|.
%    \begin{macrocode}
\gdef\@thm#1#2{\refstepcounter{#1}%
%    \end{macrocode}
%    Now we start a {\envfont trivlist} environment, and give
%    |\@topsep| and |\@topsepadd| the values of the \textsf{skip}
%    registers |\theorempreskipamount| and |\theorempostskipamount|.
%    The value in |\@topsep| is the vertical space that is inserted by
%    the first (and only) |\item| in our |\trivlist| whilst
%    |\@topsepadd| is inserted by |\@endparenv| at the end of that
%    {\envfont trivlist} environment.  By using these registers, we
%    obtain the desired space around a {\envfont theorem} environment.
%    \begin{macrocode}
   \trivlist
   \@topsep \theorempreskipamount               % used by first \item
   \@topsepadd \theorempostskipamount           % used by \@endparenv
%    \end{macrocode}
%    Now we have to test whether an optional argument has been given.
% \changes{v2.0d}{89/07/18}{Forgotten percent added.}
%    \begin{macrocode}
   \@ifnextchar [%
%    \end{macrocode}
%    If there is an optional argument, we will call |\@ythm|, and
%    move the arguments read back into the input stream.
%    \begin{macrocode}
   {\@ythm{#1}{#2}}%
%    \end{macrocode}
%    If not, we call |\@begintheorem|. Its first argument is the
%    name of the theorem set (hence the second argument of
%    |\@thm|). Its second argument is the macro that produces the
%    current number.
%    \begin{macrocode}
   {\@begintheorem{#2}{\csname the#1\endcsname}\ignorespaces}}
%    \end{macrocode}
% \end{macro}
%
% \begin{macro}{\@xthm}
% \begin{macro}{\@ythm}
%    Both these macros were originally called by |\@thm|.  We do
%    not need |\@xthm| anymore, hence we reset it to
%    |\relax|. The definition of |\@ythm| has not changed at
%    all from its definition in \LaTeX. In order to make the macros
%    easier to understand, we will
%    nevertheless present it (commented out).
%    \begin{macrocode}
\global\let\@xthm\relax
% \def\@ythm#1#2[#3]{\@opargbegintheorem{#2}{\csname
%           the#1\endcsname}{#3}\ignorespaces}
%</package>
%    \end{macrocode}
%    The primitive |\ignorespaces| in |\@ythm| and
%    |\@thm| is needed to remove the spaces between the
%    |\begin{...}| and the actual text.
% \end{macro}
% \end{macro}
%
%
% \subsection{Definition of the theorem substyles}
%
% As already pointed out, the theorem substyles, defined below, are
% only loaded when necessary.  Note that all these substyles, except
% \texttt{plain}, have |\slshape| as the default body font.
%
% \subsubsection{The \texttt{plain} style}
%
% \label{dok:th@plain} As the following macros use |@|, we have
% to locally set the |\catcode| of this symbol to ``letter''.
% This happens within a group, so that we do not have to worry about
% which |\catcode| that symbol had before.
%    \begin{macrocode}
%<*thp>
\begingroup \makeatletter
%    \end{macrocode}
% Since we are now within a group, we must make all definitions
% globally.  First we make sure that \texttt{theorem.sty} is loaded.
% This will allow us to use this file as a document style option
% without having to call \texttt{theorem} itself as an option.  At the
% same time, we assure that at least version 2 is loaded, since
% |\theorem@style| was not defined in earlier versions.
%    \begin{macrocode}
\@ifundefined{theorem@style}{% \iffalse meta-comment
%
% Copyright (C) 1993-2024
%
% The LaTeX Project and any individual authors listed elsewhere
% in this file.
%
% This file is part of the Standard LaTeX `Tools Bundle'.
% -------------------------------------------------------
%
% It may be distributed and/or modified under the
% conditions of the LaTeX Project Public License, either version 1.3c
% of this license or (at your option) any later version.
% The latest version of this license is in
%    https://www.latex-project.org/lppl.txt
% and version 1.3c or later is part of all distributions of LaTeX
% version 2005/12/01 or later.
%
% The list of all files belonging to the LaTeX `Tools Bundle' is
% given in the file `manifest.txt'.
%
% \fi
%
%%
%
%
\def\FMithmInfo{2023/07/05 v2.2c Theorem extension package (FMi)}
%
% \ProvidesFile{theorem.dtx}[\FMithmInfo]
%\iffalse   % this is a METACOMMENT !
%%
%% Package `theorem' to use with LaTeX2e
%% Copyright (C) 1989-1995 Frank Mittelbach, all rights reserved.
%<+package>\@ifundefined{theorem@style}{}{\endinput}
%<+package>\NeedsTeXFormat{LaTeX2e}[1995/06/01]
%<+package>\ProvidesPackage{theorem}
%<+package>         [\FMithmInfo]
%
% \fi
%
% \changes{v1.0}{87/05/07}{First draft.}
% \changes{v1.0a}{87/05/07}{`thistheoremstyle' added.}
% \changes{v1.1a}{87/07/29}{New concept: redefine the internal
%                          \protect\LaTeX{} commands `@xnthm' etc.}
% \changes{v1.1b}{87/07/31}{`theoremfont' used to specify font.}
% \changes{v1.2a}{88/01/08}{`@endtheorem' now globally defined.}
% \changes{v1.2b}{88/01/11}{`newtheorem' only allowed in the preamble.}
% \changes{v1.3a}{88/02/14}{`theoremfont' now a toks register.}
%
% \changes{v2.0a}{88/11/16}{New implementation and new documentation.}
% \changes{v2.0a}{88/11/16}{No layout changes but internal modifications
%                           old substyles won't work.}
% \changes{v2.0b}{89/01/30}{`global' used for all definitions.}
% \changes{v2.0c}{89/06/15}{Documentation translated to English.
%                            Thanks to LLL and Nina}
% \changes{v2.1a}{90/02/15}{Support for new font selection.}
% \changes{v2.2a}{94/02/03}{Cleanup for LaTeX2e}
%
% \DoNotIndex{\@endparenv,\@ifdefinable,\@ifnextchar,\@ifundefined}
% \DoNotIndex{\@namedef,\@nameuse,\@spaces,\@tempa,\@warning}
% \DoNotIndex{\ }
% \DoNotIndex{\begingroup,\bfseries}
% \DoNotIndex{\csname}
% \DoNotIndex{\def,\do,\docdate}
% \DoNotIndex{\endcsname,\endgroup,\endinput,\endtrivlist,\expandafter}
% \DoNotIndex{\filedate,\fileversion}
% \DoNotIndex{\gdef,\global}
% \DoNotIndex{\hbox,\hskip}
% \DoNotIndex{\ignorespaces,\input,\itshape,\item}
% \DoNotIndex{\labelsep,\let,\llap}
% \DoNotIndex{\makeatletter}
% \DoNotIndex{\noexpand}
% \DoNotIndex{\refstepcounter,\relax,\rlap}
% \DoNotIndex{\setlength,\slshape,\space,\string,\strut}
% \DoNotIndex{\the,\trivlist,\typeout}
% \DoNotIndex{\vbox}
% \DoNotIndex{\xdef}
%
% \def\deflabel#1{\ttfamily #1\hfill}
% \def\deflist#1{\begin{list}{}{\settowidth\labelwidth{\ttfamily #1}%
%                               \setlength\leftmargin\labelwidth
%                               \addtolength\leftmargin\labelsep
%                               \let\makelabel\deflabel}}
% \def\enddeflist{\end{list}}
% \def\envfont{\normalfont\ttfamily}
%
% \MakeShortVerb{\|}
%
% \newcommand{\TUB}{{\sl TUGboat\/}}
%
% \GetFileInfo{theorem.dtx}
% \title{An Extension of the
%        \LaTeX{} theorem environment\thanks{This file
%        has version number \fileversion, last
%        revised \filedate.}}
% \author{Frank Mittelbach}
%
% \MaintainedByLaTeXTeam{tools}
% \maketitle
%
% \begin{abstract}
%    The macros described in this paper yield an extension of the
%    \LaTeX{} theorem mechanism. It is designed to satisfy the
%    different requirements of various journals. Thus, the
%    layout of the ``theorems'' can be manipulated by determining a
%    ``style''. This article describes not only the use, but
%    also the definition, of the necessary macros.
% \end{abstract}
%
% \section*{Preface to version 2.2}
%
% For \LaTeXe{} this package did not need any fundamental changes. I
% only modified the messages generated so that theorem layout styles
% will show up with the |\listfiles| command and cleaned the section
% on the New Font Selection Scheme since this is now included in
% \LaTeX.
%
% \section*{Preface to version 2.1}
%
% This version is identical to 2.0g described in \TUB\ 10\#3 except
% for some internal defaults which are now set depending on the used
% font selection scheme.
%
% This was done to avoid unpleasant surprises if the new font selection
% scheme is in force. For further details see
% section~\ref{sec:newfonts} and \cite{bk:GMS94}.
%
% \section{Introduction}
%
% For our purposes here, ``theorems'' are labelled enunciations,
% often set off from the main text by extra space and a font change.
% Theorems, corollaries, conjectures, definitions, and remarks
% are all instances of ``theorems''.  The ``header'' of these structures
% is composed of a label (such as \textsc{Theorem} or \textsc{Remark})
% and a number which serializes an item in the sequence of items
% with the same label.
%
% Shortly after the introduction of \LaTeX{} at the Fachbereich
% Mathematik in Mainz, the desire to manipulate the layout of
% ``theorems'' arose.  In Mainz, the following two conventions came
% into general use:
% \begin{enumerate}
%   \item The number of the theorem is shown in the margin.
%   \item There is a line break at
%         the end of the theorem header.
% \end{enumerate}
%
% Additionally, some journals require different formats which depend on
% the ``sort of theorem'': e.g.\ often remarks and definitions are set
% in |\upshape|, while |\itshape| is employed for main theorems.
%
% Confronted with these requirements, a theorem environment was
% developed in Mainz which allows separate determination of the
% layout of the ``theorems sets'', comparable to |\pagestyle|.
%
%
% \section{The user interface}
% \subsection{Defining new theorem sets}
%
% \DescribeMacro\newtheorem
% As in the original \LaTeX{} version, the command |\newtheorem|
% defines a new ``theorem set'' or ``theorem-like structure''.
% Two required arguments name the new environment and give the text
% to be typeset with each instance of the new ``set'', while
% an optional argument determines how
% the ``set'' is enumerated:
% \begin{description}
%    \item[\ttfamily \bslash newtheorem\{foo\}\{bar\}]
%       The theorem set {\envfont foo} (whose name is \texttt{bar})
%       uses its own counter.
%    \item[\ttfamily \bslash newtheorem\{foo2\}{[foo]}\{bar2\}]
%       The theorem set {\envfont foo2} (printed name \texttt{bar2})
%       uses the same counter as the theorem set \texttt{foo}.
%    \item[\ttfamily \bslash newtheorem\{foo3\}\{bar3\}{[section]}]
%       The theorem set {\envfont foo3} (printed name \texttt{bar3}) is
%       enumerated within the counter \texttt{section}, i.e.\ with every
%       new |\section| the enumeration begins again with 1, and
%       the enumeration is composed from the section-number and the
%       theorem counter itself.
% \end{description}
%
% \DescribeMacro\theoremstyle
% Additionally, the command |\theoremstyle| can define the layout
% of various, or all, theorem sets.  It should be noted that any
% theorem set defined by |\newtheorem| is typeset in the
% |\theoremstyle| that is current at the time of the definition.
% Thus, the following
% \begin{verbatim}
%    \theoremstyle{break}        \newtheorem{Cor}{Corollary}
%    \theoremstyle{plain}        \newtheorem{Exa}{Example}[section]
%\end{verbatim}                             ^^A No blank at line start
% leads to the result that the set {\envfont Cor} is formatted in the
% style \texttt{break}, while the set {\envfont Exa} and all the
% following ones are formatted in the style \texttt{plain}, unless
% another |\theoremstyle| follows.  Since the definitions installed by
% |\newtheorem| are global, one also can limit |\theoremstyle| locally
% by grouping braces.
%
% \DescribeMacro\theorembodyfont
% The choice of the font for the theorem body is completely
% independent of the chosen |\theoremstyle|; this has proven to
% be very advantageous.  For example,
% \begin{verbatim}
%   {\theorembodyfont{\upshape}          \newtheorem{Rem}{Remark}}
%\end{verbatim}
% defines a theorem set {\envfont Rem}, which will be set in
% |\upshape| in the current layout (which in our example is
% \texttt{plain}).  As with |\theoremstyle|, the |\theorembodyfont|
% chosen is that current at the time of |\newtheorem|.  If
% |\theorembodyfont| is not specified or one defines
% |\theorembodyfont{}|, then the font used will be that defined by the
% |\theoremstyle|.
%
% \DescribeMacro\theoremheaderfont It is also possible to customize
% the font used for the theorem headers.  This is, however, a global
% declaration, and therefore there should be at most one
% |\theoremheaderfont| declaration in the preamble.\footnote{If it is
% actually necessary to have different header fonts, one has to define
% new theorem styles (substituting the desired font) or specify the
% information directly in the \texttt{\bslash newtheorem} declaration
% (the unclean variant).}
%
% \DescribeMacro\theorempreskipamount
% \DescribeMacro\theorempostskipamount Two additional parameters
% affect the vertical space around the theorem environments:
% |\theorempreskipamount| and |\theorempostskipamount| define,
% respectively, the spacing before and after such an environment.
% These parameters apply for all theorem sets and can be manipulated
% with the ordinary length macros.  They are rubber lengths
% (`\textsf{skips}'), and therefore can contain \texttt{plus} and
% \texttt{minus} parts.
%
% Since the definition of theorem sets should---most sensibly---be
% placed in the preamble, we only allow installation there.
% It is therefore possible to release the memory used here after
% |\begin{document}|, in order to make room for other
% applications.
%
%
% \subsection{Existing theorem styles}
%
% The following theorem styles exist to date:
% \begin{deflist}{marginbreak}
%    \item[plain]
%       This theorem style emulates the original \LaTeX{} definition,
%       except that additionally the parameters
%       |\theorem...skipamount| are used.
%    \item[break]
%       In this style, the theorem header is followed by a line
%       break.
%    \item[marginbreak]
%       The theorem number is set in the margin, and there is a line
%       break as in \texttt{break}.
%    \item[changebreak]
%       Like \texttt{break}, but with header number and text
%       interchanged.
%    \item[change]
%       Header number and text are interchanged, without a line break.
%    \item[margin]
%       The number is set in the left margin, without a line break.
% \end{deflist}
% All styles (except \texttt{plain}) select |\slshape| as the default
% |\theorembodyfont|.
%
%
% \subsection{Examples}
%
% Given the above theorem sets {\envfont Cor}, {\envfont Exa}
% and {\envfont Rem},
% suppose that the preamble also contains the declarations:
% \begin{verbatim}
%    \theoremstyle{marginbreak}   \newtheorem{Lem}[Cor]{Lemma}
%    \theoremstyle{change}
%    \theorembodyfont{\itshape}        \newtheorem{Def}[Cor]{Definition}
%
%    \theoremheaderfont{\scshape}
%\end{verbatim}
% Then the following are some typical examples of the typeset output
% resulting from their use.
%
% \begin{Cor}
%    This is a sentence typeset in the theorem environment {\envfont
%    Cor}.
% \end{Cor}
%
% \begin{Exa}
%    This is a sentence typeset in the theorem environment {\envfont
%    Exa}.
% \end{Exa}
%
% \begin{Rem}
%    This is a sentence typeset in the theorem environment {\envfont
%    Rem}.
% \end{Rem}
%
% \begin{Lem}[Ben User]
%    This is a sentence typeset in the theorem environment {\envfont
%    Lem}.
% \end{Lem}
%
% \begin{Def}[Very impressive Definition]
%    This is a sentence typeset in the theorem environment {\envfont
%    Def}.
% \end{Def}
%
% The last two examples show the effect of the optional argument to
% a theorem environment (it is the text typeset in parentheses).
%
%
% \section{Special Considerations}
% \label{sec:newfonts}
%
% Theoremheader and body are implemented as a unit. This means that
% the |\theoremheaderfont|  will inherit characteristics of the
% |\theorembodyfont| in \LaTeXe. Thus, if for example
% |\theorembodyfont| is |\itshape| and |\theoremheaderfont|
% is |\bfseries| the font selected for the header will have the
% characteristics `bold extended italic'. If this is not desired one
% should set the |\theoremheaderfont| to something like
% \begin{verbatim}
%      \theoremheaderfont{\normalfont\bfseries}
%\end{verbatim}
% i.e.\ supplying all necessary font information explicitly.
%
%
%
% \section{Acknowledgements}
%
% The publication of this set of macros was only possible with the
% help of Christina Busse (translating the manuscript into English),
% Joachim Pense (playing the r\^ole of typist), Chris Rowley
% (looking everything over) and many others providing useful
% suggestions.
%
% \MaybeStop{%
%    \begin{thebibliography}{9}
%       \bibitem{bk:GMS94} \textsc{M.~Goossens}, \textsc{F.~Mittelbach}
%         and \textsc{A.~Samarin}.
%         \newblock The \LaTeX{} Companion.
%         \newblock
%         Addison-Wesley, Reading, Massachusetts, 1994.
%       \bibitem{latex:source} \textsc{Lamport, Leslie.} \newblock
%                               \texttt{latex.tex}, version 2.09, date
%                               Feb. 1990.
%    \end{thebibliography}
%    \PrintIndex
%    \PrintChanges}
%
%
% \section{The documentation driver file}
%
% The next bit of code contains the documentation driver file for
% \TeX{}, i.e., the file that will produce the documentation you are
% currently reading. It will be extracted from this file by the
% \texttt{docstrip} program. Since it is the first code in the file
% one can alternatively process this file directly with \LaTeXe{} to
% obtain the documentation.
%
%    \begin{macrocode}
%<*driver>
\documentclass{ltxdoc}

\usepackage{theorem}

%<+driver>% The next few lines define theorem sets which are used
%<+driver>% in the example section of the documentation.

\theoremstyle{break}       \newtheorem{Cor}{Corollary}
\theoremstyle{plain}       \newtheorem{Exa}{Example}[section]
{\theorembodyfont{\upshape}\newtheorem{Rem}{Remark}}
\theoremstyle{marginbreak} \newtheorem{Lem}[Cor]{Lemma}
\theoremstyle{change}
\theorembodyfont{\itshape} \newtheorem{Def}[Cor]{Definition}

\theoremheaderfont{\scshape}

\RecordChanges

\begin{document}
   \DocInput{theorem.dtx}
\end{document}
%</driver>
%    \end{macrocode}
%
% \section{Definition of the Macros}
%
% If the file has been loaded before, we abort immediately. If not the
% package announces itself (this is actually done at the very top of
% the file---the way it is done isn't good style so don't copy it).
% \changes{v2.0e}{89/07/19}{Spaces removed from `typeout'.}
% \changes{v2.2b}{95/11/19}{Announce moved to top of file}
%    \begin{macrocode}
%<*package>
%\@ifundefined{theorem@style}{}{\endinput}
%\def\FMithmInfo{1995/11/19 v2.2b Theorem extension package (FMi)}
%\ProvidesPackage{theorem}[\FMithmInfo]
%    \end{macrocode}
%
%
% \subsection{Definition of theorem styles and fonts}
%
% All the definitions in this file are done globally to allow
% inputting this file inside a group.
%
% \begin{macro}{\theoremstyle}
%    Before a theorem style can be installed, the chosen style must be
%    known. For that reason, we must test to see that
%    |\th@|\meta{style} is known or, more precisely, that it is
%    different from |\relax|. If the style is not known then
%    |\th@plain| is used.
%    \begin{macrocode}
\gdef\theoremstyle#1{%
   \@ifundefined{th@#1}{\@warning
          {Unknown theoremstyle `#1'. Using `plain'}%
          \theorem@style{plain}}%
%    \end{macrocode}
%    We save the theorem style to be used in the token register
%    |\theorem@style|.
%    \begin{macrocode}
      {\theorem@style{#1}}%
%    \end{macrocode}
%    Now we ``evaluate'' the theorem style: this means, we call the
%    macro |\th@|\meta{style} which will activate the relevant
%    definitions which are contained in a separate file.  This is done
%    in a group to suppress changes to the current font.  This could
%    otherwise pose problems together with the new font selection
%    scheme\footnote{When I printed the original article using the new
%    font selection scheme I ended with a document with slanted
%    typefaces (text headings and all) simply because one of the
%    theorem styles used \texttt{\bslash sl} at toplevel.} if the
%    |\th@|\meta{style} is evaluated a second time.
% \changes{v2.1a}{90/02/15}{Group added around `th@..}
% \changes{v2.1a}{90/02/15}{Reseting of `..@begintheorem removed.}
%    \begin{macrocode}
      \begingroup
        \csname th@\the\theorem@style \endcsname
      \endgroup}
%    \end{macrocode}
% \end{macro}
%
%
% \begin{macro}{\@begintheorem}
% \changes{v2.1a}{90/02/15}{Relaxed at toplevel.}
% \begin{macro}{\@opargbegintheorem}
% \changes{v2.1a}{90/02/15}{Relaxed at toplevel.}
%    We reset |\@begintheorem| and |\@opargbegintheorem| to
%    |\relax| since these commands are no longer necessary at
%    toplevel. This will save a few tokens.
%    \begin{macrocode}
\global\let\@begintheorem\relax
\global\let\@opargbegintheorem\relax
%    \end{macrocode}
% \end{macro}
% \end{macro}
%
% \begin{macro}{\theorem@style}
%    Obviously the \textsf{token} register used above has to be
%    allocated. To assure the utmost compatibility with the
%    original \LaTeX{} definition, we set the default theorem style to
%    \texttt{plain}, which implements the usual \LaTeX{} convention.
% \changes{v2.0f}{89/07/20}{Unnecessary `global' removed.}
%    \begin{macrocode}
\newtoks\theorem@style
\global\theorem@style{plain}
%    \end{macrocode}
% \end{macro}
%
% \begin{macro}{\theorembodyfont}
% \begin{macro}{\theorem@bodyfont}
%    If the |\theorembodyfont| is set by the user then it should not
%    interact with the default font set in the theorem style. When the
%    new font selection is in force this may happen if, for example,
%    the default is |\itshape| and the new |\theorembodyfont| is
%    |\sffamily|. So we add a |\reset@font| command in front of the
%    user definition.
% \changes{v2.0f}{89/07/20}{Unnecessary `global' removed.}
% \changes{v2.0g}{89/09/19}{`theoremfont to `theorembodyfont.}
% \changes{v2.1c}{92/06/22}{`reset@font added and logic changed.}
% \changes{v2.1d}{92/10/28}{Reset to empty if arg is empty}
%    \begin{macrocode}
\gdef\theorembodyfont#1{%
%    \end{macrocode}
%    We check if the argument supplied is empty and if so put nothing
%    into the |\theorem@bodyfont| token register to allow for
%    |\theorembodyfont{}| as a mean of using the default of the
%    current |\theoremstyle|.
%    \begin{macrocode}
   \def\@tempa{#1}%
   \ifx\@tempa\@empty
    \theorem@bodyfont{}%
   \else
    \theorem@bodyfont{\reset@font#1}%
   \fi
   }
\newtoks\theorem@bodyfont
\global\theorem@bodyfont{}
%    \end{macrocode}
% \end{macro}
% \end{macro}
%
%
% \begin{macro}{\theoremheaderfont}
% \changes{v2.0g}{89/09/19}{Macro added.}
%    The font for the theorem headers is handled differently because
%    this definition applies to all theorem styles.
%    \begin{macrocode}
\gdef\theoremheaderfont#1{\gdef\theorem@headerfont{#1}%
%    \end{macrocode}
%    After using the macro once it is redefined to produce an error
%    message.
%    \begin{macrocode}
       \gdef\theoremheaderfont##1{%
        \typeout{\string\theoremheaderfont\space should be used
                 only once.}}}
%    \end{macrocode}
% \end{macro}
%
% \begin{macro}{\theorem@headerfont}
% \changes{v2.0g}{89/09/19}{Macro added.}
%    To set the |\theorem@headerfont| default we first test if
%    the new fontselection scheme is in force.
% \changes{v2.1a}{90/02/15}{Support for new font selection.}
%    \begin{macrocode}
\ifx\upshape\undefined
%    \end{macrocode}
%    If not we define it to expand into |\bfseries|. We don't use
%    |\let| just in case a following style option redefines this
%    macro.
%    \begin{macrocode}
\gdef\theorem@headerfont{\bfseries}
%    \end{macrocode}
%    Otherwise we reset the current shape before calling |\bfseries|.
%    \begin{macrocode}
\else \gdef\theorem@headerfont{\normalfont\bfseries}\fi
%    \end{macrocode}
% \end{macro}
%
% \begin{macro}{\th@plain}
% \begin{macro}{\th@break}
% \begin{macro}{\th@marginbreak}
% \begin{macro}{\th@changebreak}
% \begin{macro}{\th@change}
% \begin{macro}{\th@margin}
%    The different styles are defined in macros such as
%    |\th@plain|.  Since memory space is precious in
%    ``non-Big-versions'', we have to avoid offering too many unused
%    definitions.  Therefore we define these styles in separate files
%    that can be loaded on demand.  Thus the commands themselves only
%    load these files. We use |\@input@|, a \LaTeXe{} internal command
%    that ensures that the file will be listed with |\listfiles|.
% \changes{v2.2a}{94/02/02}{Use `@input@ to load theorem layout files}
%    \begin{macrocode}
\gdef\th@plain{\@input@{thp.sty}}
\gdef\th@break{\@input@{thb.sty}}
\gdef\th@marginbreak{\@input@{thmb.sty}}
\gdef\th@changebreak{\@input@{thcb.sty}}
\gdef\th@change{\@input@{thc.sty}}
\gdef\th@margin{\@input@{thm.sty}}
%    \end{macrocode}
%    This list will be expanded when new styles become
%    available.  For testing,  just append new theorem substyles
%    as document options.
% \end{macro}
% \end{macro}
% \end{macro}
% \end{macro}
% \end{macro}
% \end{macro}
%
%
%
% \subsection{Definition of a new theorem set}
%
% As already pointed out, a new theorem environment can be defined in
% three different ways:
% \begin{verbatim}
%    \newtheorem{Lem}{Lemma}
%    \newtheorem{Lem}{Lemma}[section]
%    \newtheorem{Lem}[Theorem]{Lemma}
%\end{verbatim}
% The function of the macro |\newtheorem| is to recognize these cases
% and then to branch into one of the three macros |\@ynthm|, |\@xnthm|
% or |\@othm|.  This mechanism is adopted unchanged from
% \cite{latex:source}; the essential point here is that, for example,
% in the second case, the arguments \texttt{Lem}, \texttt{Lemma} and
% \texttt{section} are passed over to the macro |\@xnthm|.
%
% We inspect this case first because the others present fewer problems,
% and thus are easily derived from this one.
%
% \begin{macro}{\@xnthm}
%    For our example arguments, the macro |\@xnthm| must fulfill
%    the following:
%    \begin{itemize}
%       \item
%          Define a new \LaTeX-counter `Lem'
%       \item
%          reset this counter within a |\section|
%       \item
%          define the macro |\theLem|
%       \item
%          define the environment macros |\Lem| and
%          |\endLem| using the current
%          |\theoremstyle| and |\theorem@bodyfont|.
%    \end{itemize}
%    Obviously, all this should happen only if the first argument
%    of |\@xnthm| (i.e.\ \texttt{Lem} in our example) is chosen so
%    as not to conflict with any previously defined commands or
%    environments.  This test is performed by the \LaTeX{} macro
%    |\@ifdefinable|.
%    \begin{macrocode}
\gdef\@xnthm#1#2[#3]{\expandafter\@ifdefinable\csname #1\endcsname
%    \end{macrocode}
%    Therefore, the first argument of |\@ifdefinable| is the
%    expansion (in the example, |\Lem|) of
%    |\csname||#1\endcsname|.  The second argument is
%    executed only if the test has been completed successfully.
%    \begin{macrocode}
   {%
%    \end{macrocode}
%    Now we define the new counter. The names of the \LaTeX{}
%    macros employed should speak for themselves:
% \changes{v2.2b}{1995/10/07}{Use \cs{@newctr}}
% \changes{v2.2c}{1995/11/23}{Use \cs{@newctr} properly.}
%    \begin{macrocode}
    \@definecounter{#1}\@newctr{#1}[#3]%
%    \end{macrocode}
%    Using |\@newctr| will give a proper error message if the counter
%    in |#3| is not defined.
%    In defining `|\theLem|' we must generate the desired macro
%    name by use of |\expandafter| and |\csname|.
%    \begin{macrocode}
    \expandafter\xdef\csname the#1\endcsname
%    \end{macrocode}
%    An |\xdef| is used in order to make the definition global,
%    and to ensure that it contains the replacement texts of
%    |\@thmcountersep| and |\@thmcounter|.\footnote{These
%    two macros can be defined by the document style.  Their default
%    values produce a `\texttt{.}' as separation and an arabic
%    representation of the number.} However, not everything should be
%    expanded. For example, it saves space to use |\thesection|
%    instead of its---at times---lengthy expansion.
%    \begin{macrocode}
      {\expandafter \noexpand \csname the#3\endcsname
       \@thmcountersep \@thmcounter{#1}}%
%    \end{macrocode}
%    Thus with the defaults of \LaTeX{}, |\theLem| would be
%    replaced by the command sequence |\thesection.\arabic{Lem}|.
%
%    We will now look at the definition of the macro which is executed
%    at the beginning of the actual environment (in our example this
%    macro is |\Lem|). It should be noted that we use an
%    ``|\expandafter| trick'' to expand only certain parts of the
%    replacement text at the time of the definition.
%    \begin{macrocode}
    \def\@tempa{\global\@namedef{#1}}%
    \expandafter \@tempa \expandafter{%
%    \end{macrocode}
%    First, the macro that contains the current definitions of
%    |\@begintheorem| and |\@opargtheorem| should be called
%    up.  The name of this macro---as is already known---has the form
%    |\th@|\meta{theorem\ style}; therefore, it must be called by
%    \begin{macrocode}
      \csname th@\the \theorem@style
%    \end{macrocode}
%    In addition the default theorem font should be changeable, i.e.\
%    we have to insert the contents of |\theorem@bodyfont|. For
%    that reason, we expand even further, beyond |\endcsname|,
%    and thus insert the contents of the \textsf{token} register
%    |\theorem@bodyfont| in the replacement text.
%    \begin{macrocode}
            \expandafter \endcsname \the \theorem@bodyfont
%    \end{macrocode}
%    Now it is time to call the macro |\@thm| which takes over
%    the further processing.  It has two arguments: the current
%    counter name (in our example, \texttt{Lem}), and the text of the
%    label (in our example, \texttt{Lemma}).
%    \begin{macrocode}
     \@thm{#1}{#2}}%
%    \end{macrocode}
%    With this, the `sub-definition' is complete.  The macro
%    |\@endtheorem| ends a theorem environment and is, so far, nothing
%    but an |\endtrivlist|.  (Hence it is defined globally, and not
%    within the theorem styles.\footnote{This has to be changed as
%    soon as theorem styles that change \texttt{\bslash @endtheorem}
%    exist.  In such a case, all existing styles must be changed as
%    well since they will have to reset the macro.}) Therefore, we can
%    set it equivalent to the macro that ends the theorem set (in our
%    example, |\endLem|).  However, if some day theorem styles exist
%    that do change |\@endtheorem|, we would have to use the
%    commented-out line instead.
%    \begin{macrocode}
    \global \expandafter \let \csname end#1\endcsname \@endtheorem
%   \global\@namedef{end#1}{\@endtheorem}%
%    \end{macrocode}
%    With these commands all the required definitions are employed,
%    unless the test |\@ifdefinable| has failed.  Therefore, we
%    end the second argument of this macro and with it the definition
%    of |\@xnthm|.
%    \begin{macrocode}
   }}
%    \end{macrocode}
% \end{macro}
%
% \begin{macro}{\@ynthm}
%    The definition of |\@ynthm| is completely analogous. In
%    this case the new counter that is defined is not reset within
%    another counter; thus the definition of |\the...|
%    is simplified:
%    \begin{macrocode}
\gdef\@ynthm#1#2{\expandafter\@ifdefinable\csname #1\endcsname
   {\@definecounter{#1}%
    \expandafter\xdef\csname the#1\endcsname{\@thmcounter{#1}}%
%    \end{macrocode}
%    The rest of the definition corresponds literally to that of
%    |\@xnthm|:
%    \begin{macrocode}
    \def\@tempa{\global\@namedef{#1}}\expandafter \@tempa
     \expandafter{\csname th@\the \theorem@style \expandafter
     \endcsname \the\theorem@bodyfont \@thm{#1}{#2}}%
    \global \expandafter \let \csname end#1\endcsname \@endtheorem}}
%    \end{macrocode}
% \end{macro}
%
% \begin{macro}{\@othm}
%    The definition of |\@othm| does not contain anything new.
%    \begin{macrocode}
\gdef\@othm#1[#2]#3{%
%    \end{macrocode}
% \changes{v2.2b}{95/10/07}{Build in check for counter defined}
%    We do not define a new counter but instead use one that has
%    already been defined.  Thus the only definition we need is that
%    of this pseudo-counter (i.e.\ |\the|\meta{env.\ name}).
%    First we check if |#2| corresponds to a known counter name.
%    \begin{macrocode}
  \expandafter\ifx\csname c@#2\endcsname\relax
   \@nocounterr{#2}%
  \else
   \expandafter\@ifdefinable\csname #1\endcsname
   {\expandafter \xdef \csname the#1\endcsname
     {\expandafter \noexpand \csname the#2\endcsname}%
%    \end{macrocode}
%    All other parts of the definition can be adopted from
%    |\@xnthm|. We have to remember, though, that in this case
%    the name of the current counter and the theorem label have moved
%    to the second and third arguments.
%    \begin{macrocode}
    \def\@tempa{\global\@namedef{#1}}\expandafter \@tempa
     \expandafter{\csname th@\the \theorem@style \expandafter
     \endcsname \the\theorem@bodyfont \@thm{#2}{#3}}%
    \global \expandafter \let \csname end#1\endcsname \@endtheorem}%
  \fi}
%    \end{macrocode}
% \end{macro}
%
% \subsection{Macros that are employed in a {\envfont theorem}
%    environment}
%
% \begin{macro}{\@thm}
%    The macro |\@thm| has to increase the current counter.  Then,
%    depending on whether the environment has (or does not have) an
%    optional argument, it has to branch into either |\@begintheorem|
%    or |\@opargtheorem|.
%    \begin{macrocode}
\gdef\@thm#1#2{\refstepcounter{#1}%
%    \end{macrocode}
%    Now we start a {\envfont trivlist} environment, and give
%    |\@topsep| and |\@topsepadd| the values of the \textsf{skip}
%    registers |\theorempreskipamount| and |\theorempostskipamount|.
%    The value in |\@topsep| is the vertical space that is inserted by
%    the first (and only) |\item| in our |\trivlist| whilst
%    |\@topsepadd| is inserted by |\@endparenv| at the end of that
%    {\envfont trivlist} environment.  By using these registers, we
%    obtain the desired space around a {\envfont theorem} environment.
%    \begin{macrocode}
   \trivlist
   \@topsep \theorempreskipamount               % used by first \item
   \@topsepadd \theorempostskipamount           % used by \@endparenv
%    \end{macrocode}
%    Now we have to test whether an optional argument has been given.
% \changes{v2.0d}{89/07/18}{Forgotten percent added.}
%    \begin{macrocode}
   \@ifnextchar [%
%    \end{macrocode}
%    If there is an optional argument, we will call |\@ythm|, and
%    move the arguments read back into the input stream.
%    \begin{macrocode}
   {\@ythm{#1}{#2}}%
%    \end{macrocode}
%    If not, we call |\@begintheorem|. Its first argument is the
%    name of the theorem set (hence the second argument of
%    |\@thm|). Its second argument is the macro that produces the
%    current number.
%    \begin{macrocode}
   {\@begintheorem{#2}{\csname the#1\endcsname}\ignorespaces}}
%    \end{macrocode}
% \end{macro}
%
% \begin{macro}{\@xthm}
% \begin{macro}{\@ythm}
%    Both these macros were originally called by |\@thm|.  We do
%    not need |\@xthm| anymore, hence we reset it to
%    |\relax|. The definition of |\@ythm| has not changed at
%    all from its definition in \LaTeX. In order to make the macros
%    easier to understand, we will
%    nevertheless present it (commented out).
%    \begin{macrocode}
\global\let\@xthm\relax
% \def\@ythm#1#2[#3]{\@opargbegintheorem{#2}{\csname
%           the#1\endcsname}{#3}\ignorespaces}
%</package>
%    \end{macrocode}
%    The primitive |\ignorespaces| in |\@ythm| and
%    |\@thm| is needed to remove the spaces between the
%    |\begin{...}| and the actual text.
% \end{macro}
% \end{macro}
%
%
% \subsection{Definition of the theorem substyles}
%
% As already pointed out, the theorem substyles, defined below, are
% only loaded when necessary.  Note that all these substyles, except
% \texttt{plain}, have |\slshape| as the default body font.
%
% \subsubsection{The \texttt{plain} style}
%
% \label{dok:th@plain} As the following macros use |@|, we have
% to locally set the |\catcode| of this symbol to ``letter''.
% This happens within a group, so that we do not have to worry about
% which |\catcode| that symbol had before.
%    \begin{macrocode}
%<*thp>
\begingroup \makeatletter
%    \end{macrocode}
% Since we are now within a group, we must make all definitions
% globally.  First we make sure that \texttt{theorem.sty} is loaded.
% This will allow us to use this file as a document style option
% without having to call \texttt{theorem} itself as an option.  At the
% same time, we assure that at least version 2 is loaded, since
% |\theorem@style| was not defined in earlier versions.
%    \begin{macrocode}
\@ifundefined{theorem@style}{\input{theorem.sty}}{}
\ProvidesFile{thp.sty}
             [\FMithmInfo]
%    \end{macrocode}
%
%
% \begin{macro}{\th@plain}
% \begin{macro}{\@begintheorem}
% \begin{macro}{\@opargbegintheorem}
%    |\theoremstyle{plain}| corresponds to the original
%    definition, except that the distances to the surrounding text are
%    determined by the parameters |\theorempreskipamount| and
%    |\theorempostskipamount|.
%    First we set the default body font.
%    \begin{macrocode}
\gdef\th@plain{\normalfont\itshape
%    \end{macrocode}
%    Then we define |\@begintheorem| and
%    |\@opargbegintheorem|. These two macros define how the
%    header of a theorem is typeset.  |\@opargbegintheorem| will
%    be called if a \textsf{theorem} environment with an optional
%    argument is encountered; otherwise, the header is constructed by
%    calling |\@begintheorem|.  If one of these macros is
%    executed, we are within a \textsf{trivlist} environment started by
%    |\@thm|.  So the theorem header is produced with an
%    |\item| command.
%
%    Instead of specifying the header font directly, all standard
%    theorem styles use the |\theorem@headerfont| macro to allow
%    customization.  The extra space (|\labelsep|) is necessary
%    because of problems in the \textsf{trivlist} environment.
%    \begin{macrocode}
  \def\@begintheorem##1##2{%
        \item[\hskip\labelsep \theorem@headerfont ##1\ ##2]}%
%    \end{macrocode}
%    The definition of |\@opargbegintheorem| is completely
%    analogous. The only difference is the fact that there exists a
%    third argument (which is the optional parameter of the
%    environment and contains additional information about the
%    theorem).  Customarily we enclose it in parentheses.
%    \begin{macrocode}
\def\@opargbegintheorem##1##2##3{%
   \item[\hskip\labelsep \theorem@headerfont ##1\ ##2\ (##3)]}}
%    \end{macrocode}
% \end{macro}
% \end{macro}
% \end{macro}
%
% \noindent
% We conclude with an |\endgroup| to restore the
% |\catcode| of |@|.
%    \begin{macrocode}
\endgroup
%</thp>
%    \end{macrocode}
%
%
% \subsubsection{The \texttt{break} style}
%
% This style option is stored in the file |thb.sty|.
% For the next two lines see the documentation for |\th@plain| on
% page~\pageref{dok:th@plain}.
%    \begin{macrocode}
%<*thb>
\begingroup \makeatletter
\@ifundefined{theorem@style}{\input{theorem.sty}}{}
\ProvidesFile{thb.sty}
             [\FMithmInfo]
%    \end{macrocode}
%
% \begin{macro}{\th@break}
%    |\theoremstyle{break}| produces a line break after the name
%    of the theorem.  The font is |\slshape|.  Hence, we define
%    |\th@break| as follows:
%    \begin{macrocode}
\gdef\th@break{\normalfont\slshape
  \def\@begintheorem##1##2{\item[%
%    \end{macrocode}
%    We run into the following problem: it is not possible to create
%    the header with |\item[|\meta{title}|]| and then start
%    a new line by, for example, |\mbox{}\\|. Such a definition
%    will fail whenever a list environment follows immediately.  With
%    the above construction, the |\mbox{}| causes the switch
%    |@inlabel| (cf.\ definition of |\list| and
%    |\trivlist| in \cite{latex:source}) to be set to \texttt{false}
%    and so the following list will insert additional vertical
%    space (|\topskip|). This is quite annoying. Therefore, we
%    create the line break within the |\item|. In order to ensure
%    that the text will begin at the proper position in the following
%    line, we simply pretend that the label does not take any
%    room.\footnote{This will lead to problems whenever very high
%    symbols occurring in the line tower into the
%    heading. So, something else has to be done here sometime.}
%    \begin{macrocode}
       \rlap{\vbox{\hbox{\hskip \labelsep\theorem@headerfont ##1\ ##2}%
                   \hbox{\strut}}}]}%
%    \end{macrocode}
%    Again, the definition of |\@opargbegintheorem| is
%    completely analogous.
%    \begin{macrocode}
\def\@opargbegintheorem##1##2##3{%
  \item[\rlap{\vbox{\hbox{\hskip \labelsep \theorem@headerfont
                     ##1\ ##2\ (##3)}%
                    \hbox{\strut}}}]}}
\endgroup
%</thb>
%    \end{macrocode}
% \end{macro}
%
%
% \subsubsection{The \texttt{changebreak} style}
%
%    \begin{macrocode}
% This style option is stored in the file |thcb.sty|.
%    \begin{macrocode}
%<*thcb>
\begingroup \makeatletter
\@ifundefined{theorem@style}{\input{theorem.sty}}{}
\ProvidesFile{thcb.sty}
             [\FMithmInfo]
%    \end{macrocode}
%
% \begin{macro}{\th@changebreak}
%    The \texttt{change-break} theorem style is like \texttt{break}
%    but with interchange of theorem name and theorem number.  Thus we
%    define |\th@changebreak| as follows:
%    \begin{macrocode}
\gdef\th@changebreak{\normalfont\slshape
  \def\@begintheorem##1##2{\item
       [\rlap{\vbox{\hbox{\hskip\labelsep \theorem@headerfont ##2\ ##1}%
                    \hbox{\strut}}}]}%
\def\@opargbegintheorem##1##2##3{%
  \item[\rlap{\vbox{\hbox{\hskip\labelsep \theorem@headerfont
                     ##2\ ##1\ (##3)}%
                    \hbox{\strut}}}]}}
\endgroup
%</thcb>
%    \end{macrocode}
% \end{macro}
%
%
% \subsubsection{The \texttt{change} style}
%
% This style option is stored in the file |thc.sty|.
%    \begin{macrocode}
%<*thc>
\begingroup \makeatletter
\@ifundefined{theorem@style}{\input{theorem.sty}}{}
\ProvidesFile{thc.sty}
             [\FMithmInfo]
%    \end{macrocode}
%
% \begin{macro}{\th@change}
%    The \texttt{change} theorem style corresponds to the
%    \texttt{change break} style without a linebreak after the header.
%    To say it in another way, it's the same as the \texttt{plain}
%    style but with number and name interchanged and |\slshape| as the
%    default font.
%    \begin{macrocode}
\gdef\th@change{\normalfont\slshape
  \def\@begintheorem##1##2{\item
       [\hskip\labelsep \theorem@headerfont ##2\ ##1]}%
\def\@opargbegintheorem##1##2##3{%
  \item[\hskip\labelsep \theorem@headerfont ##2\ ##1\ (##3)]}}
\endgroup
%</thc>
%    \end{macrocode}
% \end{macro}
%
%
% \subsubsection{The \texttt{marginbreak} style}
%
% This style option is the one used most often at Mainz.  It is saved
% in the file |thmb.sty|.
%    \begin{macrocode}
%<*thmb>
\begingroup \makeatletter
\@ifundefined{theorem@style}{\input{theorem.sty}}{}
\ProvidesFile{thmb.sty}
             [\FMithmInfo]
%    \end{macrocode}
%
% \begin{macro}{\th@marginbreak}
%    The \texttt{margin break} style is nearly the same as the
%    \texttt{change break} style. The only difference is the placement
%    of the theorem number.  We use |\llap| to place it in the left
%    margin.
%
%    In this style |\labelsep| denotes the separation between the
%    number and the text.
%    \begin{macrocode}
\gdef\th@marginbreak{\normalfont\slshape
  \def\@begintheorem##1##2{\item
       [\rlap{\vbox{\theorem@headerfont
                    \hbox{\llap{##2}\hskip\labelsep ##1}%
                    \hbox{\strut}}}]}%
\def\@opargbegintheorem##1##2##3{%
  \item[\rlap{\vbox{\theorem@headerfont
                    \hbox{\llap{##2}\hskip\labelsep ##1\ (##3)}%
                    \hbox{\strut}}}]}}
\endgroup
%</thmb>
%    \end{macrocode}
% \changes{v2.1c}{90/02/17}{`endgroup outside macrocode env.}
% \end{macro}
%
%
%
% \subsubsection{The \texttt{margin} style}
%
% This style option is stored in the file |thm.sty|.
%    \begin{macrocode}
%<*thm>
\begingroup \makeatletter
\@ifundefined{theorem@style}{\input{theorem.sty}}{}
\ProvidesFile{thm.sty}
             [\FMithmInfo]
%    \end{macrocode}
%
% \begin{macro}{\th@margin}
%    Again this is only a variant of the theorem styles described
%    above without any new ideas.
%    \begin{macrocode}
\gdef\th@margin{\normalfont\slshape
  \def\@begintheorem##1##2{\item
       [\theorem@headerfont \llap{##2}\hskip\labelsep ##1]}%
\def\@opargbegintheorem##1##2##3{%
  \item[\theorem@headerfont \llap{##2}\hskip\labelsep ##1\ (##3)]}}
\endgroup
%</thm>
%    \end{macrocode}
% \end{macro}
%
%
%
%
% \subsection{Final Definitions}
%
% \begin{macro}{\theorempreskipamount}
% \begin{macro}{\theorempostskipamount}
%    The \textsf{skip} parameters that regulate the vertical empty space
%    before and after the {\envfont theorem} environment have
%    to be allocated as well.
% \changes{v2.0f}{89/07/20}{Unnecessary `global' removed.}
%    \begin{macrocode}
%<*package>
\newskip\theorempreskipamount
\newskip\theorempostskipamount
%    \end{macrocode}
%    Since we have used the same values for all theorem sets, we now can
%    assign them.
%    \begin{macrocode}
\global\setlength\theorempreskipamount{12pt plus 5pt minus 3pt}
\global\setlength\theorempostskipamount{8pt plus 3pt minus 1.5pt}
%    \end{macrocode}
% \end{macro}
% \end{macro}
%
%
% \begin{macro}{\@endtheorem}
%    The same holds for the macro |\@endtheorem|, which ends a
%    {\envfont theorem} environment. Since it is the same for all
%    theorem sets, it is removed from the macros |\th@|\meta{style}.
%    It simply ends the {\envfont trivlist} environment, which was
%    begun in |\@thm|.
%    \begin{macrocode}
\global\let\@endtheorem=\endtrivlist
%    \end{macrocode}
% \end{macro}
%
%
%
% \begin{macro}{\@preamblecmds}
%    All macros defined above are to be used only in the preamble.
%    Therefore, we insert them in |\@preamblecmds| which will disable
%    them at begin document. This is done by the internal \LaTeXe{}
%    command |\@onlypreamble|.
% \changes{v2.0g}{89/09/19}{`theoremheaderfont and `theorembodyfont
%    added.}
% \changes{v2.2a}{94/02/03}{Use \LaTeXe{} interface}
%    \begin{macrocode}
\@onlypreamble\@xnthm
\@onlypreamble\@ynthm
\@onlypreamble\@othm
\@onlypreamble\newtheorem
\@onlypreamble\theoremstyle
\@onlypreamble\theorembodyfont
\@onlypreamble\theoremheaderfont
%    \end{macrocode}
% \end{macro}
%
% Finally we declare the \texttt{plain} theorem style to be the default.
% \changes{v2.1b}{90/06/04}{Fixed missing `item by declaring default
%    style.}
%    \begin{macrocode}
\theoremstyle{plain}
%</package>
%    \end{macrocode}
%
% \Finale

\endinput
}{}
\ProvidesFile{thp.sty}
             [\FMithmInfo]
%    \end{macrocode}
%
%
% \begin{macro}{\th@plain}
% \begin{macro}{\@begintheorem}
% \begin{macro}{\@opargbegintheorem}
%    |\theoremstyle{plain}| corresponds to the original
%    definition, except that the distances to the surrounding text are
%    determined by the parameters |\theorempreskipamount| and
%    |\theorempostskipamount|.
%    First we set the default body font.
%    \begin{macrocode}
\gdef\th@plain{\normalfont\itshape
%    \end{macrocode}
%    Then we define |\@begintheorem| and
%    |\@opargbegintheorem|. These two macros define how the
%    header of a theorem is typeset.  |\@opargbegintheorem| will
%    be called if a \textsf{theorem} environment with an optional
%    argument is encountered; otherwise, the header is constructed by
%    calling |\@begintheorem|.  If one of these macros is
%    executed, we are within a \textsf{trivlist} environment started by
%    |\@thm|.  So the theorem header is produced with an
%    |\item| command.
%
%    Instead of specifying the header font directly, all standard
%    theorem styles use the |\theorem@headerfont| macro to allow
%    customization.  The extra space (|\labelsep|) is necessary
%    because of problems in the \textsf{trivlist} environment.
%    \begin{macrocode}
  \def\@begintheorem##1##2{%
        \item[\hskip\labelsep \theorem@headerfont ##1\ ##2]}%
%    \end{macrocode}
%    The definition of |\@opargbegintheorem| is completely
%    analogous. The only difference is the fact that there exists a
%    third argument (which is the optional parameter of the
%    environment and contains additional information about the
%    theorem).  Customarily we enclose it in parentheses.
%    \begin{macrocode}
\def\@opargbegintheorem##1##2##3{%
   \item[\hskip\labelsep \theorem@headerfont ##1\ ##2\ (##3)]}}
%    \end{macrocode}
% \end{macro}
% \end{macro}
% \end{macro}
%
% \noindent
% We conclude with an |\endgroup| to restore the
% |\catcode| of |@|.
%    \begin{macrocode}
\endgroup
%</thp>
%    \end{macrocode}
%
%
% \subsubsection{The \texttt{break} style}
%
% This style option is stored in the file |thb.sty|.
% For the next two lines see the documentation for |\th@plain| on
% page~\pageref{dok:th@plain}.
%    \begin{macrocode}
%<*thb>
\begingroup \makeatletter
\@ifundefined{theorem@style}{% \iffalse meta-comment
%
% Copyright (C) 1993-2024
%
% The LaTeX Project and any individual authors listed elsewhere
% in this file.
%
% This file is part of the Standard LaTeX `Tools Bundle'.
% -------------------------------------------------------
%
% It may be distributed and/or modified under the
% conditions of the LaTeX Project Public License, either version 1.3c
% of this license or (at your option) any later version.
% The latest version of this license is in
%    https://www.latex-project.org/lppl.txt
% and version 1.3c or later is part of all distributions of LaTeX
% version 2005/12/01 or later.
%
% The list of all files belonging to the LaTeX `Tools Bundle' is
% given in the file `manifest.txt'.
%
% \fi
%
%%
%
%
\def\FMithmInfo{2023/07/05 v2.2c Theorem extension package (FMi)}
%
% \ProvidesFile{theorem.dtx}[\FMithmInfo]
%\iffalse   % this is a METACOMMENT !
%%
%% Package `theorem' to use with LaTeX2e
%% Copyright (C) 1989-1995 Frank Mittelbach, all rights reserved.
%<+package>\@ifundefined{theorem@style}{}{\endinput}
%<+package>\NeedsTeXFormat{LaTeX2e}[1995/06/01]
%<+package>\ProvidesPackage{theorem}
%<+package>         [\FMithmInfo]
%
% \fi
%
% \changes{v1.0}{87/05/07}{First draft.}
% \changes{v1.0a}{87/05/07}{`thistheoremstyle' added.}
% \changes{v1.1a}{87/07/29}{New concept: redefine the internal
%                          \protect\LaTeX{} commands `@xnthm' etc.}
% \changes{v1.1b}{87/07/31}{`theoremfont' used to specify font.}
% \changes{v1.2a}{88/01/08}{`@endtheorem' now globally defined.}
% \changes{v1.2b}{88/01/11}{`newtheorem' only allowed in the preamble.}
% \changes{v1.3a}{88/02/14}{`theoremfont' now a toks register.}
%
% \changes{v2.0a}{88/11/16}{New implementation and new documentation.}
% \changes{v2.0a}{88/11/16}{No layout changes but internal modifications
%                           old substyles won't work.}
% \changes{v2.0b}{89/01/30}{`global' used for all definitions.}
% \changes{v2.0c}{89/06/15}{Documentation translated to English.
%                            Thanks to LLL and Nina}
% \changes{v2.1a}{90/02/15}{Support for new font selection.}
% \changes{v2.2a}{94/02/03}{Cleanup for LaTeX2e}
%
% \DoNotIndex{\@endparenv,\@ifdefinable,\@ifnextchar,\@ifundefined}
% \DoNotIndex{\@namedef,\@nameuse,\@spaces,\@tempa,\@warning}
% \DoNotIndex{\ }
% \DoNotIndex{\begingroup,\bfseries}
% \DoNotIndex{\csname}
% \DoNotIndex{\def,\do,\docdate}
% \DoNotIndex{\endcsname,\endgroup,\endinput,\endtrivlist,\expandafter}
% \DoNotIndex{\filedate,\fileversion}
% \DoNotIndex{\gdef,\global}
% \DoNotIndex{\hbox,\hskip}
% \DoNotIndex{\ignorespaces,\input,\itshape,\item}
% \DoNotIndex{\labelsep,\let,\llap}
% \DoNotIndex{\makeatletter}
% \DoNotIndex{\noexpand}
% \DoNotIndex{\refstepcounter,\relax,\rlap}
% \DoNotIndex{\setlength,\slshape,\space,\string,\strut}
% \DoNotIndex{\the,\trivlist,\typeout}
% \DoNotIndex{\vbox}
% \DoNotIndex{\xdef}
%
% \def\deflabel#1{\ttfamily #1\hfill}
% \def\deflist#1{\begin{list}{}{\settowidth\labelwidth{\ttfamily #1}%
%                               \setlength\leftmargin\labelwidth
%                               \addtolength\leftmargin\labelsep
%                               \let\makelabel\deflabel}}
% \def\enddeflist{\end{list}}
% \def\envfont{\normalfont\ttfamily}
%
% \MakeShortVerb{\|}
%
% \newcommand{\TUB}{{\sl TUGboat\/}}
%
% \GetFileInfo{theorem.dtx}
% \title{An Extension of the
%        \LaTeX{} theorem environment\thanks{This file
%        has version number \fileversion, last
%        revised \filedate.}}
% \author{Frank Mittelbach}
%
% \MaintainedByLaTeXTeam{tools}
% \maketitle
%
% \begin{abstract}
%    The macros described in this paper yield an extension of the
%    \LaTeX{} theorem mechanism. It is designed to satisfy the
%    different requirements of various journals. Thus, the
%    layout of the ``theorems'' can be manipulated by determining a
%    ``style''. This article describes not only the use, but
%    also the definition, of the necessary macros.
% \end{abstract}
%
% \section*{Preface to version 2.2}
%
% For \LaTeXe{} this package did not need any fundamental changes. I
% only modified the messages generated so that theorem layout styles
% will show up with the |\listfiles| command and cleaned the section
% on the New Font Selection Scheme since this is now included in
% \LaTeX.
%
% \section*{Preface to version 2.1}
%
% This version is identical to 2.0g described in \TUB\ 10\#3 except
% for some internal defaults which are now set depending on the used
% font selection scheme.
%
% This was done to avoid unpleasant surprises if the new font selection
% scheme is in force. For further details see
% section~\ref{sec:newfonts} and \cite{bk:GMS94}.
%
% \section{Introduction}
%
% For our purposes here, ``theorems'' are labelled enunciations,
% often set off from the main text by extra space and a font change.
% Theorems, corollaries, conjectures, definitions, and remarks
% are all instances of ``theorems''.  The ``header'' of these structures
% is composed of a label (such as \textsc{Theorem} or \textsc{Remark})
% and a number which serializes an item in the sequence of items
% with the same label.
%
% Shortly after the introduction of \LaTeX{} at the Fachbereich
% Mathematik in Mainz, the desire to manipulate the layout of
% ``theorems'' arose.  In Mainz, the following two conventions came
% into general use:
% \begin{enumerate}
%   \item The number of the theorem is shown in the margin.
%   \item There is a line break at
%         the end of the theorem header.
% \end{enumerate}
%
% Additionally, some journals require different formats which depend on
% the ``sort of theorem'': e.g.\ often remarks and definitions are set
% in |\upshape|, while |\itshape| is employed for main theorems.
%
% Confronted with these requirements, a theorem environment was
% developed in Mainz which allows separate determination of the
% layout of the ``theorems sets'', comparable to |\pagestyle|.
%
%
% \section{The user interface}
% \subsection{Defining new theorem sets}
%
% \DescribeMacro\newtheorem
% As in the original \LaTeX{} version, the command |\newtheorem|
% defines a new ``theorem set'' or ``theorem-like structure''.
% Two required arguments name the new environment and give the text
% to be typeset with each instance of the new ``set'', while
% an optional argument determines how
% the ``set'' is enumerated:
% \begin{description}
%    \item[\ttfamily \bslash newtheorem\{foo\}\{bar\}]
%       The theorem set {\envfont foo} (whose name is \texttt{bar})
%       uses its own counter.
%    \item[\ttfamily \bslash newtheorem\{foo2\}{[foo]}\{bar2\}]
%       The theorem set {\envfont foo2} (printed name \texttt{bar2})
%       uses the same counter as the theorem set \texttt{foo}.
%    \item[\ttfamily \bslash newtheorem\{foo3\}\{bar3\}{[section]}]
%       The theorem set {\envfont foo3} (printed name \texttt{bar3}) is
%       enumerated within the counter \texttt{section}, i.e.\ with every
%       new |\section| the enumeration begins again with 1, and
%       the enumeration is composed from the section-number and the
%       theorem counter itself.
% \end{description}
%
% \DescribeMacro\theoremstyle
% Additionally, the command |\theoremstyle| can define the layout
% of various, or all, theorem sets.  It should be noted that any
% theorem set defined by |\newtheorem| is typeset in the
% |\theoremstyle| that is current at the time of the definition.
% Thus, the following
% \begin{verbatim}
%    \theoremstyle{break}        \newtheorem{Cor}{Corollary}
%    \theoremstyle{plain}        \newtheorem{Exa}{Example}[section]
%\end{verbatim}                             ^^A No blank at line start
% leads to the result that the set {\envfont Cor} is formatted in the
% style \texttt{break}, while the set {\envfont Exa} and all the
% following ones are formatted in the style \texttt{plain}, unless
% another |\theoremstyle| follows.  Since the definitions installed by
% |\newtheorem| are global, one also can limit |\theoremstyle| locally
% by grouping braces.
%
% \DescribeMacro\theorembodyfont
% The choice of the font for the theorem body is completely
% independent of the chosen |\theoremstyle|; this has proven to
% be very advantageous.  For example,
% \begin{verbatim}
%   {\theorembodyfont{\upshape}          \newtheorem{Rem}{Remark}}
%\end{verbatim}
% defines a theorem set {\envfont Rem}, which will be set in
% |\upshape| in the current layout (which in our example is
% \texttt{plain}).  As with |\theoremstyle|, the |\theorembodyfont|
% chosen is that current at the time of |\newtheorem|.  If
% |\theorembodyfont| is not specified or one defines
% |\theorembodyfont{}|, then the font used will be that defined by the
% |\theoremstyle|.
%
% \DescribeMacro\theoremheaderfont It is also possible to customize
% the font used for the theorem headers.  This is, however, a global
% declaration, and therefore there should be at most one
% |\theoremheaderfont| declaration in the preamble.\footnote{If it is
% actually necessary to have different header fonts, one has to define
% new theorem styles (substituting the desired font) or specify the
% information directly in the \texttt{\bslash newtheorem} declaration
% (the unclean variant).}
%
% \DescribeMacro\theorempreskipamount
% \DescribeMacro\theorempostskipamount Two additional parameters
% affect the vertical space around the theorem environments:
% |\theorempreskipamount| and |\theorempostskipamount| define,
% respectively, the spacing before and after such an environment.
% These parameters apply for all theorem sets and can be manipulated
% with the ordinary length macros.  They are rubber lengths
% (`\textsf{skips}'), and therefore can contain \texttt{plus} and
% \texttt{minus} parts.
%
% Since the definition of theorem sets should---most sensibly---be
% placed in the preamble, we only allow installation there.
% It is therefore possible to release the memory used here after
% |\begin{document}|, in order to make room for other
% applications.
%
%
% \subsection{Existing theorem styles}
%
% The following theorem styles exist to date:
% \begin{deflist}{marginbreak}
%    \item[plain]
%       This theorem style emulates the original \LaTeX{} definition,
%       except that additionally the parameters
%       |\theorem...skipamount| are used.
%    \item[break]
%       In this style, the theorem header is followed by a line
%       break.
%    \item[marginbreak]
%       The theorem number is set in the margin, and there is a line
%       break as in \texttt{break}.
%    \item[changebreak]
%       Like \texttt{break}, but with header number and text
%       interchanged.
%    \item[change]
%       Header number and text are interchanged, without a line break.
%    \item[margin]
%       The number is set in the left margin, without a line break.
% \end{deflist}
% All styles (except \texttt{plain}) select |\slshape| as the default
% |\theorembodyfont|.
%
%
% \subsection{Examples}
%
% Given the above theorem sets {\envfont Cor}, {\envfont Exa}
% and {\envfont Rem},
% suppose that the preamble also contains the declarations:
% \begin{verbatim}
%    \theoremstyle{marginbreak}   \newtheorem{Lem}[Cor]{Lemma}
%    \theoremstyle{change}
%    \theorembodyfont{\itshape}        \newtheorem{Def}[Cor]{Definition}
%
%    \theoremheaderfont{\scshape}
%\end{verbatim}
% Then the following are some typical examples of the typeset output
% resulting from their use.
%
% \begin{Cor}
%    This is a sentence typeset in the theorem environment {\envfont
%    Cor}.
% \end{Cor}
%
% \begin{Exa}
%    This is a sentence typeset in the theorem environment {\envfont
%    Exa}.
% \end{Exa}
%
% \begin{Rem}
%    This is a sentence typeset in the theorem environment {\envfont
%    Rem}.
% \end{Rem}
%
% \begin{Lem}[Ben User]
%    This is a sentence typeset in the theorem environment {\envfont
%    Lem}.
% \end{Lem}
%
% \begin{Def}[Very impressive Definition]
%    This is a sentence typeset in the theorem environment {\envfont
%    Def}.
% \end{Def}
%
% The last two examples show the effect of the optional argument to
% a theorem environment (it is the text typeset in parentheses).
%
%
% \section{Special Considerations}
% \label{sec:newfonts}
%
% Theoremheader and body are implemented as a unit. This means that
% the |\theoremheaderfont|  will inherit characteristics of the
% |\theorembodyfont| in \LaTeXe. Thus, if for example
% |\theorembodyfont| is |\itshape| and |\theoremheaderfont|
% is |\bfseries| the font selected for the header will have the
% characteristics `bold extended italic'. If this is not desired one
% should set the |\theoremheaderfont| to something like
% \begin{verbatim}
%      \theoremheaderfont{\normalfont\bfseries}
%\end{verbatim}
% i.e.\ supplying all necessary font information explicitly.
%
%
%
% \section{Acknowledgements}
%
% The publication of this set of macros was only possible with the
% help of Christina Busse (translating the manuscript into English),
% Joachim Pense (playing the r\^ole of typist), Chris Rowley
% (looking everything over) and many others providing useful
% suggestions.
%
% \MaybeStop{%
%    \begin{thebibliography}{9}
%       \bibitem{bk:GMS94} \textsc{M.~Goossens}, \textsc{F.~Mittelbach}
%         and \textsc{A.~Samarin}.
%         \newblock The \LaTeX{} Companion.
%         \newblock
%         Addison-Wesley, Reading, Massachusetts, 1994.
%       \bibitem{latex:source} \textsc{Lamport, Leslie.} \newblock
%                               \texttt{latex.tex}, version 2.09, date
%                               Feb. 1990.
%    \end{thebibliography}
%    \PrintIndex
%    \PrintChanges}
%
%
% \section{The documentation driver file}
%
% The next bit of code contains the documentation driver file for
% \TeX{}, i.e., the file that will produce the documentation you are
% currently reading. It will be extracted from this file by the
% \texttt{docstrip} program. Since it is the first code in the file
% one can alternatively process this file directly with \LaTeXe{} to
% obtain the documentation.
%
%    \begin{macrocode}
%<*driver>
\documentclass{ltxdoc}

\usepackage{theorem}

%<+driver>% The next few lines define theorem sets which are used
%<+driver>% in the example section of the documentation.

\theoremstyle{break}       \newtheorem{Cor}{Corollary}
\theoremstyle{plain}       \newtheorem{Exa}{Example}[section]
{\theorembodyfont{\upshape}\newtheorem{Rem}{Remark}}
\theoremstyle{marginbreak} \newtheorem{Lem}[Cor]{Lemma}
\theoremstyle{change}
\theorembodyfont{\itshape} \newtheorem{Def}[Cor]{Definition}

\theoremheaderfont{\scshape}

\RecordChanges

\begin{document}
   \DocInput{theorem.dtx}
\end{document}
%</driver>
%    \end{macrocode}
%
% \section{Definition of the Macros}
%
% If the file has been loaded before, we abort immediately. If not the
% package announces itself (this is actually done at the very top of
% the file---the way it is done isn't good style so don't copy it).
% \changes{v2.0e}{89/07/19}{Spaces removed from `typeout'.}
% \changes{v2.2b}{95/11/19}{Announce moved to top of file}
%    \begin{macrocode}
%<*package>
%\@ifundefined{theorem@style}{}{\endinput}
%\def\FMithmInfo{1995/11/19 v2.2b Theorem extension package (FMi)}
%\ProvidesPackage{theorem}[\FMithmInfo]
%    \end{macrocode}
%
%
% \subsection{Definition of theorem styles and fonts}
%
% All the definitions in this file are done globally to allow
% inputting this file inside a group.
%
% \begin{macro}{\theoremstyle}
%    Before a theorem style can be installed, the chosen style must be
%    known. For that reason, we must test to see that
%    |\th@|\meta{style} is known or, more precisely, that it is
%    different from |\relax|. If the style is not known then
%    |\th@plain| is used.
%    \begin{macrocode}
\gdef\theoremstyle#1{%
   \@ifundefined{th@#1}{\@warning
          {Unknown theoremstyle `#1'. Using `plain'}%
          \theorem@style{plain}}%
%    \end{macrocode}
%    We save the theorem style to be used in the token register
%    |\theorem@style|.
%    \begin{macrocode}
      {\theorem@style{#1}}%
%    \end{macrocode}
%    Now we ``evaluate'' the theorem style: this means, we call the
%    macro |\th@|\meta{style} which will activate the relevant
%    definitions which are contained in a separate file.  This is done
%    in a group to suppress changes to the current font.  This could
%    otherwise pose problems together with the new font selection
%    scheme\footnote{When I printed the original article using the new
%    font selection scheme I ended with a document with slanted
%    typefaces (text headings and all) simply because one of the
%    theorem styles used \texttt{\bslash sl} at toplevel.} if the
%    |\th@|\meta{style} is evaluated a second time.
% \changes{v2.1a}{90/02/15}{Group added around `th@..}
% \changes{v2.1a}{90/02/15}{Reseting of `..@begintheorem removed.}
%    \begin{macrocode}
      \begingroup
        \csname th@\the\theorem@style \endcsname
      \endgroup}
%    \end{macrocode}
% \end{macro}
%
%
% \begin{macro}{\@begintheorem}
% \changes{v2.1a}{90/02/15}{Relaxed at toplevel.}
% \begin{macro}{\@opargbegintheorem}
% \changes{v2.1a}{90/02/15}{Relaxed at toplevel.}
%    We reset |\@begintheorem| and |\@opargbegintheorem| to
%    |\relax| since these commands are no longer necessary at
%    toplevel. This will save a few tokens.
%    \begin{macrocode}
\global\let\@begintheorem\relax
\global\let\@opargbegintheorem\relax
%    \end{macrocode}
% \end{macro}
% \end{macro}
%
% \begin{macro}{\theorem@style}
%    Obviously the \textsf{token} register used above has to be
%    allocated. To assure the utmost compatibility with the
%    original \LaTeX{} definition, we set the default theorem style to
%    \texttt{plain}, which implements the usual \LaTeX{} convention.
% \changes{v2.0f}{89/07/20}{Unnecessary `global' removed.}
%    \begin{macrocode}
\newtoks\theorem@style
\global\theorem@style{plain}
%    \end{macrocode}
% \end{macro}
%
% \begin{macro}{\theorembodyfont}
% \begin{macro}{\theorem@bodyfont}
%    If the |\theorembodyfont| is set by the user then it should not
%    interact with the default font set in the theorem style. When the
%    new font selection is in force this may happen if, for example,
%    the default is |\itshape| and the new |\theorembodyfont| is
%    |\sffamily|. So we add a |\reset@font| command in front of the
%    user definition.
% \changes{v2.0f}{89/07/20}{Unnecessary `global' removed.}
% \changes{v2.0g}{89/09/19}{`theoremfont to `theorembodyfont.}
% \changes{v2.1c}{92/06/22}{`reset@font added and logic changed.}
% \changes{v2.1d}{92/10/28}{Reset to empty if arg is empty}
%    \begin{macrocode}
\gdef\theorembodyfont#1{%
%    \end{macrocode}
%    We check if the argument supplied is empty and if so put nothing
%    into the |\theorem@bodyfont| token register to allow for
%    |\theorembodyfont{}| as a mean of using the default of the
%    current |\theoremstyle|.
%    \begin{macrocode}
   \def\@tempa{#1}%
   \ifx\@tempa\@empty
    \theorem@bodyfont{}%
   \else
    \theorem@bodyfont{\reset@font#1}%
   \fi
   }
\newtoks\theorem@bodyfont
\global\theorem@bodyfont{}
%    \end{macrocode}
% \end{macro}
% \end{macro}
%
%
% \begin{macro}{\theoremheaderfont}
% \changes{v2.0g}{89/09/19}{Macro added.}
%    The font for the theorem headers is handled differently because
%    this definition applies to all theorem styles.
%    \begin{macrocode}
\gdef\theoremheaderfont#1{\gdef\theorem@headerfont{#1}%
%    \end{macrocode}
%    After using the macro once it is redefined to produce an error
%    message.
%    \begin{macrocode}
       \gdef\theoremheaderfont##1{%
        \typeout{\string\theoremheaderfont\space should be used
                 only once.}}}
%    \end{macrocode}
% \end{macro}
%
% \begin{macro}{\theorem@headerfont}
% \changes{v2.0g}{89/09/19}{Macro added.}
%    To set the |\theorem@headerfont| default we first test if
%    the new fontselection scheme is in force.
% \changes{v2.1a}{90/02/15}{Support for new font selection.}
%    \begin{macrocode}
\ifx\upshape\undefined
%    \end{macrocode}
%    If not we define it to expand into |\bfseries|. We don't use
%    |\let| just in case a following style option redefines this
%    macro.
%    \begin{macrocode}
\gdef\theorem@headerfont{\bfseries}
%    \end{macrocode}
%    Otherwise we reset the current shape before calling |\bfseries|.
%    \begin{macrocode}
\else \gdef\theorem@headerfont{\normalfont\bfseries}\fi
%    \end{macrocode}
% \end{macro}
%
% \begin{macro}{\th@plain}
% \begin{macro}{\th@break}
% \begin{macro}{\th@marginbreak}
% \begin{macro}{\th@changebreak}
% \begin{macro}{\th@change}
% \begin{macro}{\th@margin}
%    The different styles are defined in macros such as
%    |\th@plain|.  Since memory space is precious in
%    ``non-Big-versions'', we have to avoid offering too many unused
%    definitions.  Therefore we define these styles in separate files
%    that can be loaded on demand.  Thus the commands themselves only
%    load these files. We use |\@input@|, a \LaTeXe{} internal command
%    that ensures that the file will be listed with |\listfiles|.
% \changes{v2.2a}{94/02/02}{Use `@input@ to load theorem layout files}
%    \begin{macrocode}
\gdef\th@plain{\@input@{thp.sty}}
\gdef\th@break{\@input@{thb.sty}}
\gdef\th@marginbreak{\@input@{thmb.sty}}
\gdef\th@changebreak{\@input@{thcb.sty}}
\gdef\th@change{\@input@{thc.sty}}
\gdef\th@margin{\@input@{thm.sty}}
%    \end{macrocode}
%    This list will be expanded when new styles become
%    available.  For testing,  just append new theorem substyles
%    as document options.
% \end{macro}
% \end{macro}
% \end{macro}
% \end{macro}
% \end{macro}
% \end{macro}
%
%
%
% \subsection{Definition of a new theorem set}
%
% As already pointed out, a new theorem environment can be defined in
% three different ways:
% \begin{verbatim}
%    \newtheorem{Lem}{Lemma}
%    \newtheorem{Lem}{Lemma}[section]
%    \newtheorem{Lem}[Theorem]{Lemma}
%\end{verbatim}
% The function of the macro |\newtheorem| is to recognize these cases
% and then to branch into one of the three macros |\@ynthm|, |\@xnthm|
% or |\@othm|.  This mechanism is adopted unchanged from
% \cite{latex:source}; the essential point here is that, for example,
% in the second case, the arguments \texttt{Lem}, \texttt{Lemma} and
% \texttt{section} are passed over to the macro |\@xnthm|.
%
% We inspect this case first because the others present fewer problems,
% and thus are easily derived from this one.
%
% \begin{macro}{\@xnthm}
%    For our example arguments, the macro |\@xnthm| must fulfill
%    the following:
%    \begin{itemize}
%       \item
%          Define a new \LaTeX-counter `Lem'
%       \item
%          reset this counter within a |\section|
%       \item
%          define the macro |\theLem|
%       \item
%          define the environment macros |\Lem| and
%          |\endLem| using the current
%          |\theoremstyle| and |\theorem@bodyfont|.
%    \end{itemize}
%    Obviously, all this should happen only if the first argument
%    of |\@xnthm| (i.e.\ \texttt{Lem} in our example) is chosen so
%    as not to conflict with any previously defined commands or
%    environments.  This test is performed by the \LaTeX{} macro
%    |\@ifdefinable|.
%    \begin{macrocode}
\gdef\@xnthm#1#2[#3]{\expandafter\@ifdefinable\csname #1\endcsname
%    \end{macrocode}
%    Therefore, the first argument of |\@ifdefinable| is the
%    expansion (in the example, |\Lem|) of
%    |\csname||#1\endcsname|.  The second argument is
%    executed only if the test has been completed successfully.
%    \begin{macrocode}
   {%
%    \end{macrocode}
%    Now we define the new counter. The names of the \LaTeX{}
%    macros employed should speak for themselves:
% \changes{v2.2b}{1995/10/07}{Use \cs{@newctr}}
% \changes{v2.2c}{1995/11/23}{Use \cs{@newctr} properly.}
%    \begin{macrocode}
    \@definecounter{#1}\@newctr{#1}[#3]%
%    \end{macrocode}
%    Using |\@newctr| will give a proper error message if the counter
%    in |#3| is not defined.
%    In defining `|\theLem|' we must generate the desired macro
%    name by use of |\expandafter| and |\csname|.
%    \begin{macrocode}
    \expandafter\xdef\csname the#1\endcsname
%    \end{macrocode}
%    An |\xdef| is used in order to make the definition global,
%    and to ensure that it contains the replacement texts of
%    |\@thmcountersep| and |\@thmcounter|.\footnote{These
%    two macros can be defined by the document style.  Their default
%    values produce a `\texttt{.}' as separation and an arabic
%    representation of the number.} However, not everything should be
%    expanded. For example, it saves space to use |\thesection|
%    instead of its---at times---lengthy expansion.
%    \begin{macrocode}
      {\expandafter \noexpand \csname the#3\endcsname
       \@thmcountersep \@thmcounter{#1}}%
%    \end{macrocode}
%    Thus with the defaults of \LaTeX{}, |\theLem| would be
%    replaced by the command sequence |\thesection.\arabic{Lem}|.
%
%    We will now look at the definition of the macro which is executed
%    at the beginning of the actual environment (in our example this
%    macro is |\Lem|). It should be noted that we use an
%    ``|\expandafter| trick'' to expand only certain parts of the
%    replacement text at the time of the definition.
%    \begin{macrocode}
    \def\@tempa{\global\@namedef{#1}}%
    \expandafter \@tempa \expandafter{%
%    \end{macrocode}
%    First, the macro that contains the current definitions of
%    |\@begintheorem| and |\@opargtheorem| should be called
%    up.  The name of this macro---as is already known---has the form
%    |\th@|\meta{theorem\ style}; therefore, it must be called by
%    \begin{macrocode}
      \csname th@\the \theorem@style
%    \end{macrocode}
%    In addition the default theorem font should be changeable, i.e.\
%    we have to insert the contents of |\theorem@bodyfont|. For
%    that reason, we expand even further, beyond |\endcsname|,
%    and thus insert the contents of the \textsf{token} register
%    |\theorem@bodyfont| in the replacement text.
%    \begin{macrocode}
            \expandafter \endcsname \the \theorem@bodyfont
%    \end{macrocode}
%    Now it is time to call the macro |\@thm| which takes over
%    the further processing.  It has two arguments: the current
%    counter name (in our example, \texttt{Lem}), and the text of the
%    label (in our example, \texttt{Lemma}).
%    \begin{macrocode}
     \@thm{#1}{#2}}%
%    \end{macrocode}
%    With this, the `sub-definition' is complete.  The macro
%    |\@endtheorem| ends a theorem environment and is, so far, nothing
%    but an |\endtrivlist|.  (Hence it is defined globally, and not
%    within the theorem styles.\footnote{This has to be changed as
%    soon as theorem styles that change \texttt{\bslash @endtheorem}
%    exist.  In such a case, all existing styles must be changed as
%    well since they will have to reset the macro.}) Therefore, we can
%    set it equivalent to the macro that ends the theorem set (in our
%    example, |\endLem|).  However, if some day theorem styles exist
%    that do change |\@endtheorem|, we would have to use the
%    commented-out line instead.
%    \begin{macrocode}
    \global \expandafter \let \csname end#1\endcsname \@endtheorem
%   \global\@namedef{end#1}{\@endtheorem}%
%    \end{macrocode}
%    With these commands all the required definitions are employed,
%    unless the test |\@ifdefinable| has failed.  Therefore, we
%    end the second argument of this macro and with it the definition
%    of |\@xnthm|.
%    \begin{macrocode}
   }}
%    \end{macrocode}
% \end{macro}
%
% \begin{macro}{\@ynthm}
%    The definition of |\@ynthm| is completely analogous. In
%    this case the new counter that is defined is not reset within
%    another counter; thus the definition of |\the...|
%    is simplified:
%    \begin{macrocode}
\gdef\@ynthm#1#2{\expandafter\@ifdefinable\csname #1\endcsname
   {\@definecounter{#1}%
    \expandafter\xdef\csname the#1\endcsname{\@thmcounter{#1}}%
%    \end{macrocode}
%    The rest of the definition corresponds literally to that of
%    |\@xnthm|:
%    \begin{macrocode}
    \def\@tempa{\global\@namedef{#1}}\expandafter \@tempa
     \expandafter{\csname th@\the \theorem@style \expandafter
     \endcsname \the\theorem@bodyfont \@thm{#1}{#2}}%
    \global \expandafter \let \csname end#1\endcsname \@endtheorem}}
%    \end{macrocode}
% \end{macro}
%
% \begin{macro}{\@othm}
%    The definition of |\@othm| does not contain anything new.
%    \begin{macrocode}
\gdef\@othm#1[#2]#3{%
%    \end{macrocode}
% \changes{v2.2b}{95/10/07}{Build in check for counter defined}
%    We do not define a new counter but instead use one that has
%    already been defined.  Thus the only definition we need is that
%    of this pseudo-counter (i.e.\ |\the|\meta{env.\ name}).
%    First we check if |#2| corresponds to a known counter name.
%    \begin{macrocode}
  \expandafter\ifx\csname c@#2\endcsname\relax
   \@nocounterr{#2}%
  \else
   \expandafter\@ifdefinable\csname #1\endcsname
   {\expandafter \xdef \csname the#1\endcsname
     {\expandafter \noexpand \csname the#2\endcsname}%
%    \end{macrocode}
%    All other parts of the definition can be adopted from
%    |\@xnthm|. We have to remember, though, that in this case
%    the name of the current counter and the theorem label have moved
%    to the second and third arguments.
%    \begin{macrocode}
    \def\@tempa{\global\@namedef{#1}}\expandafter \@tempa
     \expandafter{\csname th@\the \theorem@style \expandafter
     \endcsname \the\theorem@bodyfont \@thm{#2}{#3}}%
    \global \expandafter \let \csname end#1\endcsname \@endtheorem}%
  \fi}
%    \end{macrocode}
% \end{macro}
%
% \subsection{Macros that are employed in a {\envfont theorem}
%    environment}
%
% \begin{macro}{\@thm}
%    The macro |\@thm| has to increase the current counter.  Then,
%    depending on whether the environment has (or does not have) an
%    optional argument, it has to branch into either |\@begintheorem|
%    or |\@opargtheorem|.
%    \begin{macrocode}
\gdef\@thm#1#2{\refstepcounter{#1}%
%    \end{macrocode}
%    Now we start a {\envfont trivlist} environment, and give
%    |\@topsep| and |\@topsepadd| the values of the \textsf{skip}
%    registers |\theorempreskipamount| and |\theorempostskipamount|.
%    The value in |\@topsep| is the vertical space that is inserted by
%    the first (and only) |\item| in our |\trivlist| whilst
%    |\@topsepadd| is inserted by |\@endparenv| at the end of that
%    {\envfont trivlist} environment.  By using these registers, we
%    obtain the desired space around a {\envfont theorem} environment.
%    \begin{macrocode}
   \trivlist
   \@topsep \theorempreskipamount               % used by first \item
   \@topsepadd \theorempostskipamount           % used by \@endparenv
%    \end{macrocode}
%    Now we have to test whether an optional argument has been given.
% \changes{v2.0d}{89/07/18}{Forgotten percent added.}
%    \begin{macrocode}
   \@ifnextchar [%
%    \end{macrocode}
%    If there is an optional argument, we will call |\@ythm|, and
%    move the arguments read back into the input stream.
%    \begin{macrocode}
   {\@ythm{#1}{#2}}%
%    \end{macrocode}
%    If not, we call |\@begintheorem|. Its first argument is the
%    name of the theorem set (hence the second argument of
%    |\@thm|). Its second argument is the macro that produces the
%    current number.
%    \begin{macrocode}
   {\@begintheorem{#2}{\csname the#1\endcsname}\ignorespaces}}
%    \end{macrocode}
% \end{macro}
%
% \begin{macro}{\@xthm}
% \begin{macro}{\@ythm}
%    Both these macros were originally called by |\@thm|.  We do
%    not need |\@xthm| anymore, hence we reset it to
%    |\relax|. The definition of |\@ythm| has not changed at
%    all from its definition in \LaTeX. In order to make the macros
%    easier to understand, we will
%    nevertheless present it (commented out).
%    \begin{macrocode}
\global\let\@xthm\relax
% \def\@ythm#1#2[#3]{\@opargbegintheorem{#2}{\csname
%           the#1\endcsname}{#3}\ignorespaces}
%</package>
%    \end{macrocode}
%    The primitive |\ignorespaces| in |\@ythm| and
%    |\@thm| is needed to remove the spaces between the
%    |\begin{...}| and the actual text.
% \end{macro}
% \end{macro}
%
%
% \subsection{Definition of the theorem substyles}
%
% As already pointed out, the theorem substyles, defined below, are
% only loaded when necessary.  Note that all these substyles, except
% \texttt{plain}, have |\slshape| as the default body font.
%
% \subsubsection{The \texttt{plain} style}
%
% \label{dok:th@plain} As the following macros use |@|, we have
% to locally set the |\catcode| of this symbol to ``letter''.
% This happens within a group, so that we do not have to worry about
% which |\catcode| that symbol had before.
%    \begin{macrocode}
%<*thp>
\begingroup \makeatletter
%    \end{macrocode}
% Since we are now within a group, we must make all definitions
% globally.  First we make sure that \texttt{theorem.sty} is loaded.
% This will allow us to use this file as a document style option
% without having to call \texttt{theorem} itself as an option.  At the
% same time, we assure that at least version 2 is loaded, since
% |\theorem@style| was not defined in earlier versions.
%    \begin{macrocode}
\@ifundefined{theorem@style}{\input{theorem.sty}}{}
\ProvidesFile{thp.sty}
             [\FMithmInfo]
%    \end{macrocode}
%
%
% \begin{macro}{\th@plain}
% \begin{macro}{\@begintheorem}
% \begin{macro}{\@opargbegintheorem}
%    |\theoremstyle{plain}| corresponds to the original
%    definition, except that the distances to the surrounding text are
%    determined by the parameters |\theorempreskipamount| and
%    |\theorempostskipamount|.
%    First we set the default body font.
%    \begin{macrocode}
\gdef\th@plain{\normalfont\itshape
%    \end{macrocode}
%    Then we define |\@begintheorem| and
%    |\@opargbegintheorem|. These two macros define how the
%    header of a theorem is typeset.  |\@opargbegintheorem| will
%    be called if a \textsf{theorem} environment with an optional
%    argument is encountered; otherwise, the header is constructed by
%    calling |\@begintheorem|.  If one of these macros is
%    executed, we are within a \textsf{trivlist} environment started by
%    |\@thm|.  So the theorem header is produced with an
%    |\item| command.
%
%    Instead of specifying the header font directly, all standard
%    theorem styles use the |\theorem@headerfont| macro to allow
%    customization.  The extra space (|\labelsep|) is necessary
%    because of problems in the \textsf{trivlist} environment.
%    \begin{macrocode}
  \def\@begintheorem##1##2{%
        \item[\hskip\labelsep \theorem@headerfont ##1\ ##2]}%
%    \end{macrocode}
%    The definition of |\@opargbegintheorem| is completely
%    analogous. The only difference is the fact that there exists a
%    third argument (which is the optional parameter of the
%    environment and contains additional information about the
%    theorem).  Customarily we enclose it in parentheses.
%    \begin{macrocode}
\def\@opargbegintheorem##1##2##3{%
   \item[\hskip\labelsep \theorem@headerfont ##1\ ##2\ (##3)]}}
%    \end{macrocode}
% \end{macro}
% \end{macro}
% \end{macro}
%
% \noindent
% We conclude with an |\endgroup| to restore the
% |\catcode| of |@|.
%    \begin{macrocode}
\endgroup
%</thp>
%    \end{macrocode}
%
%
% \subsubsection{The \texttt{break} style}
%
% This style option is stored in the file |thb.sty|.
% For the next two lines see the documentation for |\th@plain| on
% page~\pageref{dok:th@plain}.
%    \begin{macrocode}
%<*thb>
\begingroup \makeatletter
\@ifundefined{theorem@style}{\input{theorem.sty}}{}
\ProvidesFile{thb.sty}
             [\FMithmInfo]
%    \end{macrocode}
%
% \begin{macro}{\th@break}
%    |\theoremstyle{break}| produces a line break after the name
%    of the theorem.  The font is |\slshape|.  Hence, we define
%    |\th@break| as follows:
%    \begin{macrocode}
\gdef\th@break{\normalfont\slshape
  \def\@begintheorem##1##2{\item[%
%    \end{macrocode}
%    We run into the following problem: it is not possible to create
%    the header with |\item[|\meta{title}|]| and then start
%    a new line by, for example, |\mbox{}\\|. Such a definition
%    will fail whenever a list environment follows immediately.  With
%    the above construction, the |\mbox{}| causes the switch
%    |@inlabel| (cf.\ definition of |\list| and
%    |\trivlist| in \cite{latex:source}) to be set to \texttt{false}
%    and so the following list will insert additional vertical
%    space (|\topskip|). This is quite annoying. Therefore, we
%    create the line break within the |\item|. In order to ensure
%    that the text will begin at the proper position in the following
%    line, we simply pretend that the label does not take any
%    room.\footnote{This will lead to problems whenever very high
%    symbols occurring in the line tower into the
%    heading. So, something else has to be done here sometime.}
%    \begin{macrocode}
       \rlap{\vbox{\hbox{\hskip \labelsep\theorem@headerfont ##1\ ##2}%
                   \hbox{\strut}}}]}%
%    \end{macrocode}
%    Again, the definition of |\@opargbegintheorem| is
%    completely analogous.
%    \begin{macrocode}
\def\@opargbegintheorem##1##2##3{%
  \item[\rlap{\vbox{\hbox{\hskip \labelsep \theorem@headerfont
                     ##1\ ##2\ (##3)}%
                    \hbox{\strut}}}]}}
\endgroup
%</thb>
%    \end{macrocode}
% \end{macro}
%
%
% \subsubsection{The \texttt{changebreak} style}
%
%    \begin{macrocode}
% This style option is stored in the file |thcb.sty|.
%    \begin{macrocode}
%<*thcb>
\begingroup \makeatletter
\@ifundefined{theorem@style}{\input{theorem.sty}}{}
\ProvidesFile{thcb.sty}
             [\FMithmInfo]
%    \end{macrocode}
%
% \begin{macro}{\th@changebreak}
%    The \texttt{change-break} theorem style is like \texttt{break}
%    but with interchange of theorem name and theorem number.  Thus we
%    define |\th@changebreak| as follows:
%    \begin{macrocode}
\gdef\th@changebreak{\normalfont\slshape
  \def\@begintheorem##1##2{\item
       [\rlap{\vbox{\hbox{\hskip\labelsep \theorem@headerfont ##2\ ##1}%
                    \hbox{\strut}}}]}%
\def\@opargbegintheorem##1##2##3{%
  \item[\rlap{\vbox{\hbox{\hskip\labelsep \theorem@headerfont
                     ##2\ ##1\ (##3)}%
                    \hbox{\strut}}}]}}
\endgroup
%</thcb>
%    \end{macrocode}
% \end{macro}
%
%
% \subsubsection{The \texttt{change} style}
%
% This style option is stored in the file |thc.sty|.
%    \begin{macrocode}
%<*thc>
\begingroup \makeatletter
\@ifundefined{theorem@style}{\input{theorem.sty}}{}
\ProvidesFile{thc.sty}
             [\FMithmInfo]
%    \end{macrocode}
%
% \begin{macro}{\th@change}
%    The \texttt{change} theorem style corresponds to the
%    \texttt{change break} style without a linebreak after the header.
%    To say it in another way, it's the same as the \texttt{plain}
%    style but with number and name interchanged and |\slshape| as the
%    default font.
%    \begin{macrocode}
\gdef\th@change{\normalfont\slshape
  \def\@begintheorem##1##2{\item
       [\hskip\labelsep \theorem@headerfont ##2\ ##1]}%
\def\@opargbegintheorem##1##2##3{%
  \item[\hskip\labelsep \theorem@headerfont ##2\ ##1\ (##3)]}}
\endgroup
%</thc>
%    \end{macrocode}
% \end{macro}
%
%
% \subsubsection{The \texttt{marginbreak} style}
%
% This style option is the one used most often at Mainz.  It is saved
% in the file |thmb.sty|.
%    \begin{macrocode}
%<*thmb>
\begingroup \makeatletter
\@ifundefined{theorem@style}{\input{theorem.sty}}{}
\ProvidesFile{thmb.sty}
             [\FMithmInfo]
%    \end{macrocode}
%
% \begin{macro}{\th@marginbreak}
%    The \texttt{margin break} style is nearly the same as the
%    \texttt{change break} style. The only difference is the placement
%    of the theorem number.  We use |\llap| to place it in the left
%    margin.
%
%    In this style |\labelsep| denotes the separation between the
%    number and the text.
%    \begin{macrocode}
\gdef\th@marginbreak{\normalfont\slshape
  \def\@begintheorem##1##2{\item
       [\rlap{\vbox{\theorem@headerfont
                    \hbox{\llap{##2}\hskip\labelsep ##1}%
                    \hbox{\strut}}}]}%
\def\@opargbegintheorem##1##2##3{%
  \item[\rlap{\vbox{\theorem@headerfont
                    \hbox{\llap{##2}\hskip\labelsep ##1\ (##3)}%
                    \hbox{\strut}}}]}}
\endgroup
%</thmb>
%    \end{macrocode}
% \changes{v2.1c}{90/02/17}{`endgroup outside macrocode env.}
% \end{macro}
%
%
%
% \subsubsection{The \texttt{margin} style}
%
% This style option is stored in the file |thm.sty|.
%    \begin{macrocode}
%<*thm>
\begingroup \makeatletter
\@ifundefined{theorem@style}{\input{theorem.sty}}{}
\ProvidesFile{thm.sty}
             [\FMithmInfo]
%    \end{macrocode}
%
% \begin{macro}{\th@margin}
%    Again this is only a variant of the theorem styles described
%    above without any new ideas.
%    \begin{macrocode}
\gdef\th@margin{\normalfont\slshape
  \def\@begintheorem##1##2{\item
       [\theorem@headerfont \llap{##2}\hskip\labelsep ##1]}%
\def\@opargbegintheorem##1##2##3{%
  \item[\theorem@headerfont \llap{##2}\hskip\labelsep ##1\ (##3)]}}
\endgroup
%</thm>
%    \end{macrocode}
% \end{macro}
%
%
%
%
% \subsection{Final Definitions}
%
% \begin{macro}{\theorempreskipamount}
% \begin{macro}{\theorempostskipamount}
%    The \textsf{skip} parameters that regulate the vertical empty space
%    before and after the {\envfont theorem} environment have
%    to be allocated as well.
% \changes{v2.0f}{89/07/20}{Unnecessary `global' removed.}
%    \begin{macrocode}
%<*package>
\newskip\theorempreskipamount
\newskip\theorempostskipamount
%    \end{macrocode}
%    Since we have used the same values for all theorem sets, we now can
%    assign them.
%    \begin{macrocode}
\global\setlength\theorempreskipamount{12pt plus 5pt minus 3pt}
\global\setlength\theorempostskipamount{8pt plus 3pt minus 1.5pt}
%    \end{macrocode}
% \end{macro}
% \end{macro}
%
%
% \begin{macro}{\@endtheorem}
%    The same holds for the macro |\@endtheorem|, which ends a
%    {\envfont theorem} environment. Since it is the same for all
%    theorem sets, it is removed from the macros |\th@|\meta{style}.
%    It simply ends the {\envfont trivlist} environment, which was
%    begun in |\@thm|.
%    \begin{macrocode}
\global\let\@endtheorem=\endtrivlist
%    \end{macrocode}
% \end{macro}
%
%
%
% \begin{macro}{\@preamblecmds}
%    All macros defined above are to be used only in the preamble.
%    Therefore, we insert them in |\@preamblecmds| which will disable
%    them at begin document. This is done by the internal \LaTeXe{}
%    command |\@onlypreamble|.
% \changes{v2.0g}{89/09/19}{`theoremheaderfont and `theorembodyfont
%    added.}
% \changes{v2.2a}{94/02/03}{Use \LaTeXe{} interface}
%    \begin{macrocode}
\@onlypreamble\@xnthm
\@onlypreamble\@ynthm
\@onlypreamble\@othm
\@onlypreamble\newtheorem
\@onlypreamble\theoremstyle
\@onlypreamble\theorembodyfont
\@onlypreamble\theoremheaderfont
%    \end{macrocode}
% \end{macro}
%
% Finally we declare the \texttt{plain} theorem style to be the default.
% \changes{v2.1b}{90/06/04}{Fixed missing `item by declaring default
%    style.}
%    \begin{macrocode}
\theoremstyle{plain}
%</package>
%    \end{macrocode}
%
% \Finale

\endinput
}{}
\ProvidesFile{thb.sty}
             [\FMithmInfo]
%    \end{macrocode}
%
% \begin{macro}{\th@break}
%    |\theoremstyle{break}| produces a line break after the name
%    of the theorem.  The font is |\slshape|.  Hence, we define
%    |\th@break| as follows:
%    \begin{macrocode}
\gdef\th@break{\normalfont\slshape
  \def\@begintheorem##1##2{\item[%
%    \end{macrocode}
%    We run into the following problem: it is not possible to create
%    the header with |\item[|\meta{title}|]| and then start
%    a new line by, for example, |\mbox{}\\|. Such a definition
%    will fail whenever a list environment follows immediately.  With
%    the above construction, the |\mbox{}| causes the switch
%    |@inlabel| (cf.\ definition of |\list| and
%    |\trivlist| in \cite{latex:source}) to be set to \texttt{false}
%    and so the following list will insert additional vertical
%    space (|\topskip|). This is quite annoying. Therefore, we
%    create the line break within the |\item|. In order to ensure
%    that the text will begin at the proper position in the following
%    line, we simply pretend that the label does not take any
%    room.\footnote{This will lead to problems whenever very high
%    symbols occurring in the line tower into the
%    heading. So, something else has to be done here sometime.}
%    \begin{macrocode}
       \rlap{\vbox{\hbox{\hskip \labelsep\theorem@headerfont ##1\ ##2}%
                   \hbox{\strut}}}]}%
%    \end{macrocode}
%    Again, the definition of |\@opargbegintheorem| is
%    completely analogous.
%    \begin{macrocode}
\def\@opargbegintheorem##1##2##3{%
  \item[\rlap{\vbox{\hbox{\hskip \labelsep \theorem@headerfont
                     ##1\ ##2\ (##3)}%
                    \hbox{\strut}}}]}}
\endgroup
%</thb>
%    \end{macrocode}
% \end{macro}
%
%
% \subsubsection{The \texttt{changebreak} style}
%
%    \begin{macrocode}
% This style option is stored in the file |thcb.sty|.
%    \begin{macrocode}
%<*thcb>
\begingroup \makeatletter
\@ifundefined{theorem@style}{% \iffalse meta-comment
%
% Copyright (C) 1993-2024
%
% The LaTeX Project and any individual authors listed elsewhere
% in this file.
%
% This file is part of the Standard LaTeX `Tools Bundle'.
% -------------------------------------------------------
%
% It may be distributed and/or modified under the
% conditions of the LaTeX Project Public License, either version 1.3c
% of this license or (at your option) any later version.
% The latest version of this license is in
%    https://www.latex-project.org/lppl.txt
% and version 1.3c or later is part of all distributions of LaTeX
% version 2005/12/01 or later.
%
% The list of all files belonging to the LaTeX `Tools Bundle' is
% given in the file `manifest.txt'.
%
% \fi
%
%%
%
%
\def\FMithmInfo{2023/07/05 v2.2c Theorem extension package (FMi)}
%
% \ProvidesFile{theorem.dtx}[\FMithmInfo]
%\iffalse   % this is a METACOMMENT !
%%
%% Package `theorem' to use with LaTeX2e
%% Copyright (C) 1989-1995 Frank Mittelbach, all rights reserved.
%<+package>\@ifundefined{theorem@style}{}{\endinput}
%<+package>\NeedsTeXFormat{LaTeX2e}[1995/06/01]
%<+package>\ProvidesPackage{theorem}
%<+package>         [\FMithmInfo]
%
% \fi
%
% \changes{v1.0}{87/05/07}{First draft.}
% \changes{v1.0a}{87/05/07}{`thistheoremstyle' added.}
% \changes{v1.1a}{87/07/29}{New concept: redefine the internal
%                          \protect\LaTeX{} commands `@xnthm' etc.}
% \changes{v1.1b}{87/07/31}{`theoremfont' used to specify font.}
% \changes{v1.2a}{88/01/08}{`@endtheorem' now globally defined.}
% \changes{v1.2b}{88/01/11}{`newtheorem' only allowed in the preamble.}
% \changes{v1.3a}{88/02/14}{`theoremfont' now a toks register.}
%
% \changes{v2.0a}{88/11/16}{New implementation and new documentation.}
% \changes{v2.0a}{88/11/16}{No layout changes but internal modifications
%                           old substyles won't work.}
% \changes{v2.0b}{89/01/30}{`global' used for all definitions.}
% \changes{v2.0c}{89/06/15}{Documentation translated to English.
%                            Thanks to LLL and Nina}
% \changes{v2.1a}{90/02/15}{Support for new font selection.}
% \changes{v2.2a}{94/02/03}{Cleanup for LaTeX2e}
%
% \DoNotIndex{\@endparenv,\@ifdefinable,\@ifnextchar,\@ifundefined}
% \DoNotIndex{\@namedef,\@nameuse,\@spaces,\@tempa,\@warning}
% \DoNotIndex{\ }
% \DoNotIndex{\begingroup,\bfseries}
% \DoNotIndex{\csname}
% \DoNotIndex{\def,\do,\docdate}
% \DoNotIndex{\endcsname,\endgroup,\endinput,\endtrivlist,\expandafter}
% \DoNotIndex{\filedate,\fileversion}
% \DoNotIndex{\gdef,\global}
% \DoNotIndex{\hbox,\hskip}
% \DoNotIndex{\ignorespaces,\input,\itshape,\item}
% \DoNotIndex{\labelsep,\let,\llap}
% \DoNotIndex{\makeatletter}
% \DoNotIndex{\noexpand}
% \DoNotIndex{\refstepcounter,\relax,\rlap}
% \DoNotIndex{\setlength,\slshape,\space,\string,\strut}
% \DoNotIndex{\the,\trivlist,\typeout}
% \DoNotIndex{\vbox}
% \DoNotIndex{\xdef}
%
% \def\deflabel#1{\ttfamily #1\hfill}
% \def\deflist#1{\begin{list}{}{\settowidth\labelwidth{\ttfamily #1}%
%                               \setlength\leftmargin\labelwidth
%                               \addtolength\leftmargin\labelsep
%                               \let\makelabel\deflabel}}
% \def\enddeflist{\end{list}}
% \def\envfont{\normalfont\ttfamily}
%
% \MakeShortVerb{\|}
%
% \newcommand{\TUB}{{\sl TUGboat\/}}
%
% \GetFileInfo{theorem.dtx}
% \title{An Extension of the
%        \LaTeX{} theorem environment\thanks{This file
%        has version number \fileversion, last
%        revised \filedate.}}
% \author{Frank Mittelbach}
%
% \MaintainedByLaTeXTeam{tools}
% \maketitle
%
% \begin{abstract}
%    The macros described in this paper yield an extension of the
%    \LaTeX{} theorem mechanism. It is designed to satisfy the
%    different requirements of various journals. Thus, the
%    layout of the ``theorems'' can be manipulated by determining a
%    ``style''. This article describes not only the use, but
%    also the definition, of the necessary macros.
% \end{abstract}
%
% \section*{Preface to version 2.2}
%
% For \LaTeXe{} this package did not need any fundamental changes. I
% only modified the messages generated so that theorem layout styles
% will show up with the |\listfiles| command and cleaned the section
% on the New Font Selection Scheme since this is now included in
% \LaTeX.
%
% \section*{Preface to version 2.1}
%
% This version is identical to 2.0g described in \TUB\ 10\#3 except
% for some internal defaults which are now set depending on the used
% font selection scheme.
%
% This was done to avoid unpleasant surprises if the new font selection
% scheme is in force. For further details see
% section~\ref{sec:newfonts} and \cite{bk:GMS94}.
%
% \section{Introduction}
%
% For our purposes here, ``theorems'' are labelled enunciations,
% often set off from the main text by extra space and a font change.
% Theorems, corollaries, conjectures, definitions, and remarks
% are all instances of ``theorems''.  The ``header'' of these structures
% is composed of a label (such as \textsc{Theorem} or \textsc{Remark})
% and a number which serializes an item in the sequence of items
% with the same label.
%
% Shortly after the introduction of \LaTeX{} at the Fachbereich
% Mathematik in Mainz, the desire to manipulate the layout of
% ``theorems'' arose.  In Mainz, the following two conventions came
% into general use:
% \begin{enumerate}
%   \item The number of the theorem is shown in the margin.
%   \item There is a line break at
%         the end of the theorem header.
% \end{enumerate}
%
% Additionally, some journals require different formats which depend on
% the ``sort of theorem'': e.g.\ often remarks and definitions are set
% in |\upshape|, while |\itshape| is employed for main theorems.
%
% Confronted with these requirements, a theorem environment was
% developed in Mainz which allows separate determination of the
% layout of the ``theorems sets'', comparable to |\pagestyle|.
%
%
% \section{The user interface}
% \subsection{Defining new theorem sets}
%
% \DescribeMacro\newtheorem
% As in the original \LaTeX{} version, the command |\newtheorem|
% defines a new ``theorem set'' or ``theorem-like structure''.
% Two required arguments name the new environment and give the text
% to be typeset with each instance of the new ``set'', while
% an optional argument determines how
% the ``set'' is enumerated:
% \begin{description}
%    \item[\ttfamily \bslash newtheorem\{foo\}\{bar\}]
%       The theorem set {\envfont foo} (whose name is \texttt{bar})
%       uses its own counter.
%    \item[\ttfamily \bslash newtheorem\{foo2\}{[foo]}\{bar2\}]
%       The theorem set {\envfont foo2} (printed name \texttt{bar2})
%       uses the same counter as the theorem set \texttt{foo}.
%    \item[\ttfamily \bslash newtheorem\{foo3\}\{bar3\}{[section]}]
%       The theorem set {\envfont foo3} (printed name \texttt{bar3}) is
%       enumerated within the counter \texttt{section}, i.e.\ with every
%       new |\section| the enumeration begins again with 1, and
%       the enumeration is composed from the section-number and the
%       theorem counter itself.
% \end{description}
%
% \DescribeMacro\theoremstyle
% Additionally, the command |\theoremstyle| can define the layout
% of various, or all, theorem sets.  It should be noted that any
% theorem set defined by |\newtheorem| is typeset in the
% |\theoremstyle| that is current at the time of the definition.
% Thus, the following
% \begin{verbatim}
%    \theoremstyle{break}        \newtheorem{Cor}{Corollary}
%    \theoremstyle{plain}        \newtheorem{Exa}{Example}[section]
%\end{verbatim}                             ^^A No blank at line start
% leads to the result that the set {\envfont Cor} is formatted in the
% style \texttt{break}, while the set {\envfont Exa} and all the
% following ones are formatted in the style \texttt{plain}, unless
% another |\theoremstyle| follows.  Since the definitions installed by
% |\newtheorem| are global, one also can limit |\theoremstyle| locally
% by grouping braces.
%
% \DescribeMacro\theorembodyfont
% The choice of the font for the theorem body is completely
% independent of the chosen |\theoremstyle|; this has proven to
% be very advantageous.  For example,
% \begin{verbatim}
%   {\theorembodyfont{\upshape}          \newtheorem{Rem}{Remark}}
%\end{verbatim}
% defines a theorem set {\envfont Rem}, which will be set in
% |\upshape| in the current layout (which in our example is
% \texttt{plain}).  As with |\theoremstyle|, the |\theorembodyfont|
% chosen is that current at the time of |\newtheorem|.  If
% |\theorembodyfont| is not specified or one defines
% |\theorembodyfont{}|, then the font used will be that defined by the
% |\theoremstyle|.
%
% \DescribeMacro\theoremheaderfont It is also possible to customize
% the font used for the theorem headers.  This is, however, a global
% declaration, and therefore there should be at most one
% |\theoremheaderfont| declaration in the preamble.\footnote{If it is
% actually necessary to have different header fonts, one has to define
% new theorem styles (substituting the desired font) or specify the
% information directly in the \texttt{\bslash newtheorem} declaration
% (the unclean variant).}
%
% \DescribeMacro\theorempreskipamount
% \DescribeMacro\theorempostskipamount Two additional parameters
% affect the vertical space around the theorem environments:
% |\theorempreskipamount| and |\theorempostskipamount| define,
% respectively, the spacing before and after such an environment.
% These parameters apply for all theorem sets and can be manipulated
% with the ordinary length macros.  They are rubber lengths
% (`\textsf{skips}'), and therefore can contain \texttt{plus} and
% \texttt{minus} parts.
%
% Since the definition of theorem sets should---most sensibly---be
% placed in the preamble, we only allow installation there.
% It is therefore possible to release the memory used here after
% |\begin{document}|, in order to make room for other
% applications.
%
%
% \subsection{Existing theorem styles}
%
% The following theorem styles exist to date:
% \begin{deflist}{marginbreak}
%    \item[plain]
%       This theorem style emulates the original \LaTeX{} definition,
%       except that additionally the parameters
%       |\theorem...skipamount| are used.
%    \item[break]
%       In this style, the theorem header is followed by a line
%       break.
%    \item[marginbreak]
%       The theorem number is set in the margin, and there is a line
%       break as in \texttt{break}.
%    \item[changebreak]
%       Like \texttt{break}, but with header number and text
%       interchanged.
%    \item[change]
%       Header number and text are interchanged, without a line break.
%    \item[margin]
%       The number is set in the left margin, without a line break.
% \end{deflist}
% All styles (except \texttt{plain}) select |\slshape| as the default
% |\theorembodyfont|.
%
%
% \subsection{Examples}
%
% Given the above theorem sets {\envfont Cor}, {\envfont Exa}
% and {\envfont Rem},
% suppose that the preamble also contains the declarations:
% \begin{verbatim}
%    \theoremstyle{marginbreak}   \newtheorem{Lem}[Cor]{Lemma}
%    \theoremstyle{change}
%    \theorembodyfont{\itshape}        \newtheorem{Def}[Cor]{Definition}
%
%    \theoremheaderfont{\scshape}
%\end{verbatim}
% Then the following are some typical examples of the typeset output
% resulting from their use.
%
% \begin{Cor}
%    This is a sentence typeset in the theorem environment {\envfont
%    Cor}.
% \end{Cor}
%
% \begin{Exa}
%    This is a sentence typeset in the theorem environment {\envfont
%    Exa}.
% \end{Exa}
%
% \begin{Rem}
%    This is a sentence typeset in the theorem environment {\envfont
%    Rem}.
% \end{Rem}
%
% \begin{Lem}[Ben User]
%    This is a sentence typeset in the theorem environment {\envfont
%    Lem}.
% \end{Lem}
%
% \begin{Def}[Very impressive Definition]
%    This is a sentence typeset in the theorem environment {\envfont
%    Def}.
% \end{Def}
%
% The last two examples show the effect of the optional argument to
% a theorem environment (it is the text typeset in parentheses).
%
%
% \section{Special Considerations}
% \label{sec:newfonts}
%
% Theoremheader and body are implemented as a unit. This means that
% the |\theoremheaderfont|  will inherit characteristics of the
% |\theorembodyfont| in \LaTeXe. Thus, if for example
% |\theorembodyfont| is |\itshape| and |\theoremheaderfont|
% is |\bfseries| the font selected for the header will have the
% characteristics `bold extended italic'. If this is not desired one
% should set the |\theoremheaderfont| to something like
% \begin{verbatim}
%      \theoremheaderfont{\normalfont\bfseries}
%\end{verbatim}
% i.e.\ supplying all necessary font information explicitly.
%
%
%
% \section{Acknowledgements}
%
% The publication of this set of macros was only possible with the
% help of Christina Busse (translating the manuscript into English),
% Joachim Pense (playing the r\^ole of typist), Chris Rowley
% (looking everything over) and many others providing useful
% suggestions.
%
% \MaybeStop{%
%    \begin{thebibliography}{9}
%       \bibitem{bk:GMS94} \textsc{M.~Goossens}, \textsc{F.~Mittelbach}
%         and \textsc{A.~Samarin}.
%         \newblock The \LaTeX{} Companion.
%         \newblock
%         Addison-Wesley, Reading, Massachusetts, 1994.
%       \bibitem{latex:source} \textsc{Lamport, Leslie.} \newblock
%                               \texttt{latex.tex}, version 2.09, date
%                               Feb. 1990.
%    \end{thebibliography}
%    \PrintIndex
%    \PrintChanges}
%
%
% \section{The documentation driver file}
%
% The next bit of code contains the documentation driver file for
% \TeX{}, i.e., the file that will produce the documentation you are
% currently reading. It will be extracted from this file by the
% \texttt{docstrip} program. Since it is the first code in the file
% one can alternatively process this file directly with \LaTeXe{} to
% obtain the documentation.
%
%    \begin{macrocode}
%<*driver>
\documentclass{ltxdoc}

\usepackage{theorem}

%<+driver>% The next few lines define theorem sets which are used
%<+driver>% in the example section of the documentation.

\theoremstyle{break}       \newtheorem{Cor}{Corollary}
\theoremstyle{plain}       \newtheorem{Exa}{Example}[section]
{\theorembodyfont{\upshape}\newtheorem{Rem}{Remark}}
\theoremstyle{marginbreak} \newtheorem{Lem}[Cor]{Lemma}
\theoremstyle{change}
\theorembodyfont{\itshape} \newtheorem{Def}[Cor]{Definition}

\theoremheaderfont{\scshape}

\RecordChanges

\begin{document}
   \DocInput{theorem.dtx}
\end{document}
%</driver>
%    \end{macrocode}
%
% \section{Definition of the Macros}
%
% If the file has been loaded before, we abort immediately. If not the
% package announces itself (this is actually done at the very top of
% the file---the way it is done isn't good style so don't copy it).
% \changes{v2.0e}{89/07/19}{Spaces removed from `typeout'.}
% \changes{v2.2b}{95/11/19}{Announce moved to top of file}
%    \begin{macrocode}
%<*package>
%\@ifundefined{theorem@style}{}{\endinput}
%\def\FMithmInfo{1995/11/19 v2.2b Theorem extension package (FMi)}
%\ProvidesPackage{theorem}[\FMithmInfo]
%    \end{macrocode}
%
%
% \subsection{Definition of theorem styles and fonts}
%
% All the definitions in this file are done globally to allow
% inputting this file inside a group.
%
% \begin{macro}{\theoremstyle}
%    Before a theorem style can be installed, the chosen style must be
%    known. For that reason, we must test to see that
%    |\th@|\meta{style} is known or, more precisely, that it is
%    different from |\relax|. If the style is not known then
%    |\th@plain| is used.
%    \begin{macrocode}
\gdef\theoremstyle#1{%
   \@ifundefined{th@#1}{\@warning
          {Unknown theoremstyle `#1'. Using `plain'}%
          \theorem@style{plain}}%
%    \end{macrocode}
%    We save the theorem style to be used in the token register
%    |\theorem@style|.
%    \begin{macrocode}
      {\theorem@style{#1}}%
%    \end{macrocode}
%    Now we ``evaluate'' the theorem style: this means, we call the
%    macro |\th@|\meta{style} which will activate the relevant
%    definitions which are contained in a separate file.  This is done
%    in a group to suppress changes to the current font.  This could
%    otherwise pose problems together with the new font selection
%    scheme\footnote{When I printed the original article using the new
%    font selection scheme I ended with a document with slanted
%    typefaces (text headings and all) simply because one of the
%    theorem styles used \texttt{\bslash sl} at toplevel.} if the
%    |\th@|\meta{style} is evaluated a second time.
% \changes{v2.1a}{90/02/15}{Group added around `th@..}
% \changes{v2.1a}{90/02/15}{Reseting of `..@begintheorem removed.}
%    \begin{macrocode}
      \begingroup
        \csname th@\the\theorem@style \endcsname
      \endgroup}
%    \end{macrocode}
% \end{macro}
%
%
% \begin{macro}{\@begintheorem}
% \changes{v2.1a}{90/02/15}{Relaxed at toplevel.}
% \begin{macro}{\@opargbegintheorem}
% \changes{v2.1a}{90/02/15}{Relaxed at toplevel.}
%    We reset |\@begintheorem| and |\@opargbegintheorem| to
%    |\relax| since these commands are no longer necessary at
%    toplevel. This will save a few tokens.
%    \begin{macrocode}
\global\let\@begintheorem\relax
\global\let\@opargbegintheorem\relax
%    \end{macrocode}
% \end{macro}
% \end{macro}
%
% \begin{macro}{\theorem@style}
%    Obviously the \textsf{token} register used above has to be
%    allocated. To assure the utmost compatibility with the
%    original \LaTeX{} definition, we set the default theorem style to
%    \texttt{plain}, which implements the usual \LaTeX{} convention.
% \changes{v2.0f}{89/07/20}{Unnecessary `global' removed.}
%    \begin{macrocode}
\newtoks\theorem@style
\global\theorem@style{plain}
%    \end{macrocode}
% \end{macro}
%
% \begin{macro}{\theorembodyfont}
% \begin{macro}{\theorem@bodyfont}
%    If the |\theorembodyfont| is set by the user then it should not
%    interact with the default font set in the theorem style. When the
%    new font selection is in force this may happen if, for example,
%    the default is |\itshape| and the new |\theorembodyfont| is
%    |\sffamily|. So we add a |\reset@font| command in front of the
%    user definition.
% \changes{v2.0f}{89/07/20}{Unnecessary `global' removed.}
% \changes{v2.0g}{89/09/19}{`theoremfont to `theorembodyfont.}
% \changes{v2.1c}{92/06/22}{`reset@font added and logic changed.}
% \changes{v2.1d}{92/10/28}{Reset to empty if arg is empty}
%    \begin{macrocode}
\gdef\theorembodyfont#1{%
%    \end{macrocode}
%    We check if the argument supplied is empty and if so put nothing
%    into the |\theorem@bodyfont| token register to allow for
%    |\theorembodyfont{}| as a mean of using the default of the
%    current |\theoremstyle|.
%    \begin{macrocode}
   \def\@tempa{#1}%
   \ifx\@tempa\@empty
    \theorem@bodyfont{}%
   \else
    \theorem@bodyfont{\reset@font#1}%
   \fi
   }
\newtoks\theorem@bodyfont
\global\theorem@bodyfont{}
%    \end{macrocode}
% \end{macro}
% \end{macro}
%
%
% \begin{macro}{\theoremheaderfont}
% \changes{v2.0g}{89/09/19}{Macro added.}
%    The font for the theorem headers is handled differently because
%    this definition applies to all theorem styles.
%    \begin{macrocode}
\gdef\theoremheaderfont#1{\gdef\theorem@headerfont{#1}%
%    \end{macrocode}
%    After using the macro once it is redefined to produce an error
%    message.
%    \begin{macrocode}
       \gdef\theoremheaderfont##1{%
        \typeout{\string\theoremheaderfont\space should be used
                 only once.}}}
%    \end{macrocode}
% \end{macro}
%
% \begin{macro}{\theorem@headerfont}
% \changes{v2.0g}{89/09/19}{Macro added.}
%    To set the |\theorem@headerfont| default we first test if
%    the new fontselection scheme is in force.
% \changes{v2.1a}{90/02/15}{Support for new font selection.}
%    \begin{macrocode}
\ifx\upshape\undefined
%    \end{macrocode}
%    If not we define it to expand into |\bfseries|. We don't use
%    |\let| just in case a following style option redefines this
%    macro.
%    \begin{macrocode}
\gdef\theorem@headerfont{\bfseries}
%    \end{macrocode}
%    Otherwise we reset the current shape before calling |\bfseries|.
%    \begin{macrocode}
\else \gdef\theorem@headerfont{\normalfont\bfseries}\fi
%    \end{macrocode}
% \end{macro}
%
% \begin{macro}{\th@plain}
% \begin{macro}{\th@break}
% \begin{macro}{\th@marginbreak}
% \begin{macro}{\th@changebreak}
% \begin{macro}{\th@change}
% \begin{macro}{\th@margin}
%    The different styles are defined in macros such as
%    |\th@plain|.  Since memory space is precious in
%    ``non-Big-versions'', we have to avoid offering too many unused
%    definitions.  Therefore we define these styles in separate files
%    that can be loaded on demand.  Thus the commands themselves only
%    load these files. We use |\@input@|, a \LaTeXe{} internal command
%    that ensures that the file will be listed with |\listfiles|.
% \changes{v2.2a}{94/02/02}{Use `@input@ to load theorem layout files}
%    \begin{macrocode}
\gdef\th@plain{\@input@{thp.sty}}
\gdef\th@break{\@input@{thb.sty}}
\gdef\th@marginbreak{\@input@{thmb.sty}}
\gdef\th@changebreak{\@input@{thcb.sty}}
\gdef\th@change{\@input@{thc.sty}}
\gdef\th@margin{\@input@{thm.sty}}
%    \end{macrocode}
%    This list will be expanded when new styles become
%    available.  For testing,  just append new theorem substyles
%    as document options.
% \end{macro}
% \end{macro}
% \end{macro}
% \end{macro}
% \end{macro}
% \end{macro}
%
%
%
% \subsection{Definition of a new theorem set}
%
% As already pointed out, a new theorem environment can be defined in
% three different ways:
% \begin{verbatim}
%    \newtheorem{Lem}{Lemma}
%    \newtheorem{Lem}{Lemma}[section]
%    \newtheorem{Lem}[Theorem]{Lemma}
%\end{verbatim}
% The function of the macro |\newtheorem| is to recognize these cases
% and then to branch into one of the three macros |\@ynthm|, |\@xnthm|
% or |\@othm|.  This mechanism is adopted unchanged from
% \cite{latex:source}; the essential point here is that, for example,
% in the second case, the arguments \texttt{Lem}, \texttt{Lemma} and
% \texttt{section} are passed over to the macro |\@xnthm|.
%
% We inspect this case first because the others present fewer problems,
% and thus are easily derived from this one.
%
% \begin{macro}{\@xnthm}
%    For our example arguments, the macro |\@xnthm| must fulfill
%    the following:
%    \begin{itemize}
%       \item
%          Define a new \LaTeX-counter `Lem'
%       \item
%          reset this counter within a |\section|
%       \item
%          define the macro |\theLem|
%       \item
%          define the environment macros |\Lem| and
%          |\endLem| using the current
%          |\theoremstyle| and |\theorem@bodyfont|.
%    \end{itemize}
%    Obviously, all this should happen only if the first argument
%    of |\@xnthm| (i.e.\ \texttt{Lem} in our example) is chosen so
%    as not to conflict with any previously defined commands or
%    environments.  This test is performed by the \LaTeX{} macro
%    |\@ifdefinable|.
%    \begin{macrocode}
\gdef\@xnthm#1#2[#3]{\expandafter\@ifdefinable\csname #1\endcsname
%    \end{macrocode}
%    Therefore, the first argument of |\@ifdefinable| is the
%    expansion (in the example, |\Lem|) of
%    |\csname||#1\endcsname|.  The second argument is
%    executed only if the test has been completed successfully.
%    \begin{macrocode}
   {%
%    \end{macrocode}
%    Now we define the new counter. The names of the \LaTeX{}
%    macros employed should speak for themselves:
% \changes{v2.2b}{1995/10/07}{Use \cs{@newctr}}
% \changes{v2.2c}{1995/11/23}{Use \cs{@newctr} properly.}
%    \begin{macrocode}
    \@definecounter{#1}\@newctr{#1}[#3]%
%    \end{macrocode}
%    Using |\@newctr| will give a proper error message if the counter
%    in |#3| is not defined.
%    In defining `|\theLem|' we must generate the desired macro
%    name by use of |\expandafter| and |\csname|.
%    \begin{macrocode}
    \expandafter\xdef\csname the#1\endcsname
%    \end{macrocode}
%    An |\xdef| is used in order to make the definition global,
%    and to ensure that it contains the replacement texts of
%    |\@thmcountersep| and |\@thmcounter|.\footnote{These
%    two macros can be defined by the document style.  Their default
%    values produce a `\texttt{.}' as separation and an arabic
%    representation of the number.} However, not everything should be
%    expanded. For example, it saves space to use |\thesection|
%    instead of its---at times---lengthy expansion.
%    \begin{macrocode}
      {\expandafter \noexpand \csname the#3\endcsname
       \@thmcountersep \@thmcounter{#1}}%
%    \end{macrocode}
%    Thus with the defaults of \LaTeX{}, |\theLem| would be
%    replaced by the command sequence |\thesection.\arabic{Lem}|.
%
%    We will now look at the definition of the macro which is executed
%    at the beginning of the actual environment (in our example this
%    macro is |\Lem|). It should be noted that we use an
%    ``|\expandafter| trick'' to expand only certain parts of the
%    replacement text at the time of the definition.
%    \begin{macrocode}
    \def\@tempa{\global\@namedef{#1}}%
    \expandafter \@tempa \expandafter{%
%    \end{macrocode}
%    First, the macro that contains the current definitions of
%    |\@begintheorem| and |\@opargtheorem| should be called
%    up.  The name of this macro---as is already known---has the form
%    |\th@|\meta{theorem\ style}; therefore, it must be called by
%    \begin{macrocode}
      \csname th@\the \theorem@style
%    \end{macrocode}
%    In addition the default theorem font should be changeable, i.e.\
%    we have to insert the contents of |\theorem@bodyfont|. For
%    that reason, we expand even further, beyond |\endcsname|,
%    and thus insert the contents of the \textsf{token} register
%    |\theorem@bodyfont| in the replacement text.
%    \begin{macrocode}
            \expandafter \endcsname \the \theorem@bodyfont
%    \end{macrocode}
%    Now it is time to call the macro |\@thm| which takes over
%    the further processing.  It has two arguments: the current
%    counter name (in our example, \texttt{Lem}), and the text of the
%    label (in our example, \texttt{Lemma}).
%    \begin{macrocode}
     \@thm{#1}{#2}}%
%    \end{macrocode}
%    With this, the `sub-definition' is complete.  The macro
%    |\@endtheorem| ends a theorem environment and is, so far, nothing
%    but an |\endtrivlist|.  (Hence it is defined globally, and not
%    within the theorem styles.\footnote{This has to be changed as
%    soon as theorem styles that change \texttt{\bslash @endtheorem}
%    exist.  In such a case, all existing styles must be changed as
%    well since they will have to reset the macro.}) Therefore, we can
%    set it equivalent to the macro that ends the theorem set (in our
%    example, |\endLem|).  However, if some day theorem styles exist
%    that do change |\@endtheorem|, we would have to use the
%    commented-out line instead.
%    \begin{macrocode}
    \global \expandafter \let \csname end#1\endcsname \@endtheorem
%   \global\@namedef{end#1}{\@endtheorem}%
%    \end{macrocode}
%    With these commands all the required definitions are employed,
%    unless the test |\@ifdefinable| has failed.  Therefore, we
%    end the second argument of this macro and with it the definition
%    of |\@xnthm|.
%    \begin{macrocode}
   }}
%    \end{macrocode}
% \end{macro}
%
% \begin{macro}{\@ynthm}
%    The definition of |\@ynthm| is completely analogous. In
%    this case the new counter that is defined is not reset within
%    another counter; thus the definition of |\the...|
%    is simplified:
%    \begin{macrocode}
\gdef\@ynthm#1#2{\expandafter\@ifdefinable\csname #1\endcsname
   {\@definecounter{#1}%
    \expandafter\xdef\csname the#1\endcsname{\@thmcounter{#1}}%
%    \end{macrocode}
%    The rest of the definition corresponds literally to that of
%    |\@xnthm|:
%    \begin{macrocode}
    \def\@tempa{\global\@namedef{#1}}\expandafter \@tempa
     \expandafter{\csname th@\the \theorem@style \expandafter
     \endcsname \the\theorem@bodyfont \@thm{#1}{#2}}%
    \global \expandafter \let \csname end#1\endcsname \@endtheorem}}
%    \end{macrocode}
% \end{macro}
%
% \begin{macro}{\@othm}
%    The definition of |\@othm| does not contain anything new.
%    \begin{macrocode}
\gdef\@othm#1[#2]#3{%
%    \end{macrocode}
% \changes{v2.2b}{95/10/07}{Build in check for counter defined}
%    We do not define a new counter but instead use one that has
%    already been defined.  Thus the only definition we need is that
%    of this pseudo-counter (i.e.\ |\the|\meta{env.\ name}).
%    First we check if |#2| corresponds to a known counter name.
%    \begin{macrocode}
  \expandafter\ifx\csname c@#2\endcsname\relax
   \@nocounterr{#2}%
  \else
   \expandafter\@ifdefinable\csname #1\endcsname
   {\expandafter \xdef \csname the#1\endcsname
     {\expandafter \noexpand \csname the#2\endcsname}%
%    \end{macrocode}
%    All other parts of the definition can be adopted from
%    |\@xnthm|. We have to remember, though, that in this case
%    the name of the current counter and the theorem label have moved
%    to the second and third arguments.
%    \begin{macrocode}
    \def\@tempa{\global\@namedef{#1}}\expandafter \@tempa
     \expandafter{\csname th@\the \theorem@style \expandafter
     \endcsname \the\theorem@bodyfont \@thm{#2}{#3}}%
    \global \expandafter \let \csname end#1\endcsname \@endtheorem}%
  \fi}
%    \end{macrocode}
% \end{macro}
%
% \subsection{Macros that are employed in a {\envfont theorem}
%    environment}
%
% \begin{macro}{\@thm}
%    The macro |\@thm| has to increase the current counter.  Then,
%    depending on whether the environment has (or does not have) an
%    optional argument, it has to branch into either |\@begintheorem|
%    or |\@opargtheorem|.
%    \begin{macrocode}
\gdef\@thm#1#2{\refstepcounter{#1}%
%    \end{macrocode}
%    Now we start a {\envfont trivlist} environment, and give
%    |\@topsep| and |\@topsepadd| the values of the \textsf{skip}
%    registers |\theorempreskipamount| and |\theorempostskipamount|.
%    The value in |\@topsep| is the vertical space that is inserted by
%    the first (and only) |\item| in our |\trivlist| whilst
%    |\@topsepadd| is inserted by |\@endparenv| at the end of that
%    {\envfont trivlist} environment.  By using these registers, we
%    obtain the desired space around a {\envfont theorem} environment.
%    \begin{macrocode}
   \trivlist
   \@topsep \theorempreskipamount               % used by first \item
   \@topsepadd \theorempostskipamount           % used by \@endparenv
%    \end{macrocode}
%    Now we have to test whether an optional argument has been given.
% \changes{v2.0d}{89/07/18}{Forgotten percent added.}
%    \begin{macrocode}
   \@ifnextchar [%
%    \end{macrocode}
%    If there is an optional argument, we will call |\@ythm|, and
%    move the arguments read back into the input stream.
%    \begin{macrocode}
   {\@ythm{#1}{#2}}%
%    \end{macrocode}
%    If not, we call |\@begintheorem|. Its first argument is the
%    name of the theorem set (hence the second argument of
%    |\@thm|). Its second argument is the macro that produces the
%    current number.
%    \begin{macrocode}
   {\@begintheorem{#2}{\csname the#1\endcsname}\ignorespaces}}
%    \end{macrocode}
% \end{macro}
%
% \begin{macro}{\@xthm}
% \begin{macro}{\@ythm}
%    Both these macros were originally called by |\@thm|.  We do
%    not need |\@xthm| anymore, hence we reset it to
%    |\relax|. The definition of |\@ythm| has not changed at
%    all from its definition in \LaTeX. In order to make the macros
%    easier to understand, we will
%    nevertheless present it (commented out).
%    \begin{macrocode}
\global\let\@xthm\relax
% \def\@ythm#1#2[#3]{\@opargbegintheorem{#2}{\csname
%           the#1\endcsname}{#3}\ignorespaces}
%</package>
%    \end{macrocode}
%    The primitive |\ignorespaces| in |\@ythm| and
%    |\@thm| is needed to remove the spaces between the
%    |\begin{...}| and the actual text.
% \end{macro}
% \end{macro}
%
%
% \subsection{Definition of the theorem substyles}
%
% As already pointed out, the theorem substyles, defined below, are
% only loaded when necessary.  Note that all these substyles, except
% \texttt{plain}, have |\slshape| as the default body font.
%
% \subsubsection{The \texttt{plain} style}
%
% \label{dok:th@plain} As the following macros use |@|, we have
% to locally set the |\catcode| of this symbol to ``letter''.
% This happens within a group, so that we do not have to worry about
% which |\catcode| that symbol had before.
%    \begin{macrocode}
%<*thp>
\begingroup \makeatletter
%    \end{macrocode}
% Since we are now within a group, we must make all definitions
% globally.  First we make sure that \texttt{theorem.sty} is loaded.
% This will allow us to use this file as a document style option
% without having to call \texttt{theorem} itself as an option.  At the
% same time, we assure that at least version 2 is loaded, since
% |\theorem@style| was not defined in earlier versions.
%    \begin{macrocode}
\@ifundefined{theorem@style}{\input{theorem.sty}}{}
\ProvidesFile{thp.sty}
             [\FMithmInfo]
%    \end{macrocode}
%
%
% \begin{macro}{\th@plain}
% \begin{macro}{\@begintheorem}
% \begin{macro}{\@opargbegintheorem}
%    |\theoremstyle{plain}| corresponds to the original
%    definition, except that the distances to the surrounding text are
%    determined by the parameters |\theorempreskipamount| and
%    |\theorempostskipamount|.
%    First we set the default body font.
%    \begin{macrocode}
\gdef\th@plain{\normalfont\itshape
%    \end{macrocode}
%    Then we define |\@begintheorem| and
%    |\@opargbegintheorem|. These two macros define how the
%    header of a theorem is typeset.  |\@opargbegintheorem| will
%    be called if a \textsf{theorem} environment with an optional
%    argument is encountered; otherwise, the header is constructed by
%    calling |\@begintheorem|.  If one of these macros is
%    executed, we are within a \textsf{trivlist} environment started by
%    |\@thm|.  So the theorem header is produced with an
%    |\item| command.
%
%    Instead of specifying the header font directly, all standard
%    theorem styles use the |\theorem@headerfont| macro to allow
%    customization.  The extra space (|\labelsep|) is necessary
%    because of problems in the \textsf{trivlist} environment.
%    \begin{macrocode}
  \def\@begintheorem##1##2{%
        \item[\hskip\labelsep \theorem@headerfont ##1\ ##2]}%
%    \end{macrocode}
%    The definition of |\@opargbegintheorem| is completely
%    analogous. The only difference is the fact that there exists a
%    third argument (which is the optional parameter of the
%    environment and contains additional information about the
%    theorem).  Customarily we enclose it in parentheses.
%    \begin{macrocode}
\def\@opargbegintheorem##1##2##3{%
   \item[\hskip\labelsep \theorem@headerfont ##1\ ##2\ (##3)]}}
%    \end{macrocode}
% \end{macro}
% \end{macro}
% \end{macro}
%
% \noindent
% We conclude with an |\endgroup| to restore the
% |\catcode| of |@|.
%    \begin{macrocode}
\endgroup
%</thp>
%    \end{macrocode}
%
%
% \subsubsection{The \texttt{break} style}
%
% This style option is stored in the file |thb.sty|.
% For the next two lines see the documentation for |\th@plain| on
% page~\pageref{dok:th@plain}.
%    \begin{macrocode}
%<*thb>
\begingroup \makeatletter
\@ifundefined{theorem@style}{\input{theorem.sty}}{}
\ProvidesFile{thb.sty}
             [\FMithmInfo]
%    \end{macrocode}
%
% \begin{macro}{\th@break}
%    |\theoremstyle{break}| produces a line break after the name
%    of the theorem.  The font is |\slshape|.  Hence, we define
%    |\th@break| as follows:
%    \begin{macrocode}
\gdef\th@break{\normalfont\slshape
  \def\@begintheorem##1##2{\item[%
%    \end{macrocode}
%    We run into the following problem: it is not possible to create
%    the header with |\item[|\meta{title}|]| and then start
%    a new line by, for example, |\mbox{}\\|. Such a definition
%    will fail whenever a list environment follows immediately.  With
%    the above construction, the |\mbox{}| causes the switch
%    |@inlabel| (cf.\ definition of |\list| and
%    |\trivlist| in \cite{latex:source}) to be set to \texttt{false}
%    and so the following list will insert additional vertical
%    space (|\topskip|). This is quite annoying. Therefore, we
%    create the line break within the |\item|. In order to ensure
%    that the text will begin at the proper position in the following
%    line, we simply pretend that the label does not take any
%    room.\footnote{This will lead to problems whenever very high
%    symbols occurring in the line tower into the
%    heading. So, something else has to be done here sometime.}
%    \begin{macrocode}
       \rlap{\vbox{\hbox{\hskip \labelsep\theorem@headerfont ##1\ ##2}%
                   \hbox{\strut}}}]}%
%    \end{macrocode}
%    Again, the definition of |\@opargbegintheorem| is
%    completely analogous.
%    \begin{macrocode}
\def\@opargbegintheorem##1##2##3{%
  \item[\rlap{\vbox{\hbox{\hskip \labelsep \theorem@headerfont
                     ##1\ ##2\ (##3)}%
                    \hbox{\strut}}}]}}
\endgroup
%</thb>
%    \end{macrocode}
% \end{macro}
%
%
% \subsubsection{The \texttt{changebreak} style}
%
%    \begin{macrocode}
% This style option is stored in the file |thcb.sty|.
%    \begin{macrocode}
%<*thcb>
\begingroup \makeatletter
\@ifundefined{theorem@style}{\input{theorem.sty}}{}
\ProvidesFile{thcb.sty}
             [\FMithmInfo]
%    \end{macrocode}
%
% \begin{macro}{\th@changebreak}
%    The \texttt{change-break} theorem style is like \texttt{break}
%    but with interchange of theorem name and theorem number.  Thus we
%    define |\th@changebreak| as follows:
%    \begin{macrocode}
\gdef\th@changebreak{\normalfont\slshape
  \def\@begintheorem##1##2{\item
       [\rlap{\vbox{\hbox{\hskip\labelsep \theorem@headerfont ##2\ ##1}%
                    \hbox{\strut}}}]}%
\def\@opargbegintheorem##1##2##3{%
  \item[\rlap{\vbox{\hbox{\hskip\labelsep \theorem@headerfont
                     ##2\ ##1\ (##3)}%
                    \hbox{\strut}}}]}}
\endgroup
%</thcb>
%    \end{macrocode}
% \end{macro}
%
%
% \subsubsection{The \texttt{change} style}
%
% This style option is stored in the file |thc.sty|.
%    \begin{macrocode}
%<*thc>
\begingroup \makeatletter
\@ifundefined{theorem@style}{\input{theorem.sty}}{}
\ProvidesFile{thc.sty}
             [\FMithmInfo]
%    \end{macrocode}
%
% \begin{macro}{\th@change}
%    The \texttt{change} theorem style corresponds to the
%    \texttt{change break} style without a linebreak after the header.
%    To say it in another way, it's the same as the \texttt{plain}
%    style but with number and name interchanged and |\slshape| as the
%    default font.
%    \begin{macrocode}
\gdef\th@change{\normalfont\slshape
  \def\@begintheorem##1##2{\item
       [\hskip\labelsep \theorem@headerfont ##2\ ##1]}%
\def\@opargbegintheorem##1##2##3{%
  \item[\hskip\labelsep \theorem@headerfont ##2\ ##1\ (##3)]}}
\endgroup
%</thc>
%    \end{macrocode}
% \end{macro}
%
%
% \subsubsection{The \texttt{marginbreak} style}
%
% This style option is the one used most often at Mainz.  It is saved
% in the file |thmb.sty|.
%    \begin{macrocode}
%<*thmb>
\begingroup \makeatletter
\@ifundefined{theorem@style}{\input{theorem.sty}}{}
\ProvidesFile{thmb.sty}
             [\FMithmInfo]
%    \end{macrocode}
%
% \begin{macro}{\th@marginbreak}
%    The \texttt{margin break} style is nearly the same as the
%    \texttt{change break} style. The only difference is the placement
%    of the theorem number.  We use |\llap| to place it in the left
%    margin.
%
%    In this style |\labelsep| denotes the separation between the
%    number and the text.
%    \begin{macrocode}
\gdef\th@marginbreak{\normalfont\slshape
  \def\@begintheorem##1##2{\item
       [\rlap{\vbox{\theorem@headerfont
                    \hbox{\llap{##2}\hskip\labelsep ##1}%
                    \hbox{\strut}}}]}%
\def\@opargbegintheorem##1##2##3{%
  \item[\rlap{\vbox{\theorem@headerfont
                    \hbox{\llap{##2}\hskip\labelsep ##1\ (##3)}%
                    \hbox{\strut}}}]}}
\endgroup
%</thmb>
%    \end{macrocode}
% \changes{v2.1c}{90/02/17}{`endgroup outside macrocode env.}
% \end{macro}
%
%
%
% \subsubsection{The \texttt{margin} style}
%
% This style option is stored in the file |thm.sty|.
%    \begin{macrocode}
%<*thm>
\begingroup \makeatletter
\@ifundefined{theorem@style}{\input{theorem.sty}}{}
\ProvidesFile{thm.sty}
             [\FMithmInfo]
%    \end{macrocode}
%
% \begin{macro}{\th@margin}
%    Again this is only a variant of the theorem styles described
%    above without any new ideas.
%    \begin{macrocode}
\gdef\th@margin{\normalfont\slshape
  \def\@begintheorem##1##2{\item
       [\theorem@headerfont \llap{##2}\hskip\labelsep ##1]}%
\def\@opargbegintheorem##1##2##3{%
  \item[\theorem@headerfont \llap{##2}\hskip\labelsep ##1\ (##3)]}}
\endgroup
%</thm>
%    \end{macrocode}
% \end{macro}
%
%
%
%
% \subsection{Final Definitions}
%
% \begin{macro}{\theorempreskipamount}
% \begin{macro}{\theorempostskipamount}
%    The \textsf{skip} parameters that regulate the vertical empty space
%    before and after the {\envfont theorem} environment have
%    to be allocated as well.
% \changes{v2.0f}{89/07/20}{Unnecessary `global' removed.}
%    \begin{macrocode}
%<*package>
\newskip\theorempreskipamount
\newskip\theorempostskipamount
%    \end{macrocode}
%    Since we have used the same values for all theorem sets, we now can
%    assign them.
%    \begin{macrocode}
\global\setlength\theorempreskipamount{12pt plus 5pt minus 3pt}
\global\setlength\theorempostskipamount{8pt plus 3pt minus 1.5pt}
%    \end{macrocode}
% \end{macro}
% \end{macro}
%
%
% \begin{macro}{\@endtheorem}
%    The same holds for the macro |\@endtheorem|, which ends a
%    {\envfont theorem} environment. Since it is the same for all
%    theorem sets, it is removed from the macros |\th@|\meta{style}.
%    It simply ends the {\envfont trivlist} environment, which was
%    begun in |\@thm|.
%    \begin{macrocode}
\global\let\@endtheorem=\endtrivlist
%    \end{macrocode}
% \end{macro}
%
%
%
% \begin{macro}{\@preamblecmds}
%    All macros defined above are to be used only in the preamble.
%    Therefore, we insert them in |\@preamblecmds| which will disable
%    them at begin document. This is done by the internal \LaTeXe{}
%    command |\@onlypreamble|.
% \changes{v2.0g}{89/09/19}{`theoremheaderfont and `theorembodyfont
%    added.}
% \changes{v2.2a}{94/02/03}{Use \LaTeXe{} interface}
%    \begin{macrocode}
\@onlypreamble\@xnthm
\@onlypreamble\@ynthm
\@onlypreamble\@othm
\@onlypreamble\newtheorem
\@onlypreamble\theoremstyle
\@onlypreamble\theorembodyfont
\@onlypreamble\theoremheaderfont
%    \end{macrocode}
% \end{macro}
%
% Finally we declare the \texttt{plain} theorem style to be the default.
% \changes{v2.1b}{90/06/04}{Fixed missing `item by declaring default
%    style.}
%    \begin{macrocode}
\theoremstyle{plain}
%</package>
%    \end{macrocode}
%
% \Finale

\endinput
}{}
\ProvidesFile{thcb.sty}
             [\FMithmInfo]
%    \end{macrocode}
%
% \begin{macro}{\th@changebreak}
%    The \texttt{change-break} theorem style is like \texttt{break}
%    but with interchange of theorem name and theorem number.  Thus we
%    define |\th@changebreak| as follows:
%    \begin{macrocode}
\gdef\th@changebreak{\normalfont\slshape
  \def\@begintheorem##1##2{\item
       [\rlap{\vbox{\hbox{\hskip\labelsep \theorem@headerfont ##2\ ##1}%
                    \hbox{\strut}}}]}%
\def\@opargbegintheorem##1##2##3{%
  \item[\rlap{\vbox{\hbox{\hskip\labelsep \theorem@headerfont
                     ##2\ ##1\ (##3)}%
                    \hbox{\strut}}}]}}
\endgroup
%</thcb>
%    \end{macrocode}
% \end{macro}
%
%
% \subsubsection{The \texttt{change} style}
%
% This style option is stored in the file |thc.sty|.
%    \begin{macrocode}
%<*thc>
\begingroup \makeatletter
\@ifundefined{theorem@style}{% \iffalse meta-comment
%
% Copyright (C) 1993-2024
%
% The LaTeX Project and any individual authors listed elsewhere
% in this file.
%
% This file is part of the Standard LaTeX `Tools Bundle'.
% -------------------------------------------------------
%
% It may be distributed and/or modified under the
% conditions of the LaTeX Project Public License, either version 1.3c
% of this license or (at your option) any later version.
% The latest version of this license is in
%    https://www.latex-project.org/lppl.txt
% and version 1.3c or later is part of all distributions of LaTeX
% version 2005/12/01 or later.
%
% The list of all files belonging to the LaTeX `Tools Bundle' is
% given in the file `manifest.txt'.
%
% \fi
%
%%
%
%
\def\FMithmInfo{2023/07/05 v2.2c Theorem extension package (FMi)}
%
% \ProvidesFile{theorem.dtx}[\FMithmInfo]
%\iffalse   % this is a METACOMMENT !
%%
%% Package `theorem' to use with LaTeX2e
%% Copyright (C) 1989-1995 Frank Mittelbach, all rights reserved.
%<+package>\@ifundefined{theorem@style}{}{\endinput}
%<+package>\NeedsTeXFormat{LaTeX2e}[1995/06/01]
%<+package>\ProvidesPackage{theorem}
%<+package>         [\FMithmInfo]
%
% \fi
%
% \changes{v1.0}{87/05/07}{First draft.}
% \changes{v1.0a}{87/05/07}{`thistheoremstyle' added.}
% \changes{v1.1a}{87/07/29}{New concept: redefine the internal
%                          \protect\LaTeX{} commands `@xnthm' etc.}
% \changes{v1.1b}{87/07/31}{`theoremfont' used to specify font.}
% \changes{v1.2a}{88/01/08}{`@endtheorem' now globally defined.}
% \changes{v1.2b}{88/01/11}{`newtheorem' only allowed in the preamble.}
% \changes{v1.3a}{88/02/14}{`theoremfont' now a toks register.}
%
% \changes{v2.0a}{88/11/16}{New implementation and new documentation.}
% \changes{v2.0a}{88/11/16}{No layout changes but internal modifications
%                           old substyles won't work.}
% \changes{v2.0b}{89/01/30}{`global' used for all definitions.}
% \changes{v2.0c}{89/06/15}{Documentation translated to English.
%                            Thanks to LLL and Nina}
% \changes{v2.1a}{90/02/15}{Support for new font selection.}
% \changes{v2.2a}{94/02/03}{Cleanup for LaTeX2e}
%
% \DoNotIndex{\@endparenv,\@ifdefinable,\@ifnextchar,\@ifundefined}
% \DoNotIndex{\@namedef,\@nameuse,\@spaces,\@tempa,\@warning}
% \DoNotIndex{\ }
% \DoNotIndex{\begingroup,\bfseries}
% \DoNotIndex{\csname}
% \DoNotIndex{\def,\do,\docdate}
% \DoNotIndex{\endcsname,\endgroup,\endinput,\endtrivlist,\expandafter}
% \DoNotIndex{\filedate,\fileversion}
% \DoNotIndex{\gdef,\global}
% \DoNotIndex{\hbox,\hskip}
% \DoNotIndex{\ignorespaces,\input,\itshape,\item}
% \DoNotIndex{\labelsep,\let,\llap}
% \DoNotIndex{\makeatletter}
% \DoNotIndex{\noexpand}
% \DoNotIndex{\refstepcounter,\relax,\rlap}
% \DoNotIndex{\setlength,\slshape,\space,\string,\strut}
% \DoNotIndex{\the,\trivlist,\typeout}
% \DoNotIndex{\vbox}
% \DoNotIndex{\xdef}
%
% \def\deflabel#1{\ttfamily #1\hfill}
% \def\deflist#1{\begin{list}{}{\settowidth\labelwidth{\ttfamily #1}%
%                               \setlength\leftmargin\labelwidth
%                               \addtolength\leftmargin\labelsep
%                               \let\makelabel\deflabel}}
% \def\enddeflist{\end{list}}
% \def\envfont{\normalfont\ttfamily}
%
% \MakeShortVerb{\|}
%
% \newcommand{\TUB}{{\sl TUGboat\/}}
%
% \GetFileInfo{theorem.dtx}
% \title{An Extension of the
%        \LaTeX{} theorem environment\thanks{This file
%        has version number \fileversion, last
%        revised \filedate.}}
% \author{Frank Mittelbach}
%
% \MaintainedByLaTeXTeam{tools}
% \maketitle
%
% \begin{abstract}
%    The macros described in this paper yield an extension of the
%    \LaTeX{} theorem mechanism. It is designed to satisfy the
%    different requirements of various journals. Thus, the
%    layout of the ``theorems'' can be manipulated by determining a
%    ``style''. This article describes not only the use, but
%    also the definition, of the necessary macros.
% \end{abstract}
%
% \section*{Preface to version 2.2}
%
% For \LaTeXe{} this package did not need any fundamental changes. I
% only modified the messages generated so that theorem layout styles
% will show up with the |\listfiles| command and cleaned the section
% on the New Font Selection Scheme since this is now included in
% \LaTeX.
%
% \section*{Preface to version 2.1}
%
% This version is identical to 2.0g described in \TUB\ 10\#3 except
% for some internal defaults which are now set depending on the used
% font selection scheme.
%
% This was done to avoid unpleasant surprises if the new font selection
% scheme is in force. For further details see
% section~\ref{sec:newfonts} and \cite{bk:GMS94}.
%
% \section{Introduction}
%
% For our purposes here, ``theorems'' are labelled enunciations,
% often set off from the main text by extra space and a font change.
% Theorems, corollaries, conjectures, definitions, and remarks
% are all instances of ``theorems''.  The ``header'' of these structures
% is composed of a label (such as \textsc{Theorem} or \textsc{Remark})
% and a number which serializes an item in the sequence of items
% with the same label.
%
% Shortly after the introduction of \LaTeX{} at the Fachbereich
% Mathematik in Mainz, the desire to manipulate the layout of
% ``theorems'' arose.  In Mainz, the following two conventions came
% into general use:
% \begin{enumerate}
%   \item The number of the theorem is shown in the margin.
%   \item There is a line break at
%         the end of the theorem header.
% \end{enumerate}
%
% Additionally, some journals require different formats which depend on
% the ``sort of theorem'': e.g.\ often remarks and definitions are set
% in |\upshape|, while |\itshape| is employed for main theorems.
%
% Confronted with these requirements, a theorem environment was
% developed in Mainz which allows separate determination of the
% layout of the ``theorems sets'', comparable to |\pagestyle|.
%
%
% \section{The user interface}
% \subsection{Defining new theorem sets}
%
% \DescribeMacro\newtheorem
% As in the original \LaTeX{} version, the command |\newtheorem|
% defines a new ``theorem set'' or ``theorem-like structure''.
% Two required arguments name the new environment and give the text
% to be typeset with each instance of the new ``set'', while
% an optional argument determines how
% the ``set'' is enumerated:
% \begin{description}
%    \item[\ttfamily \bslash newtheorem\{foo\}\{bar\}]
%       The theorem set {\envfont foo} (whose name is \texttt{bar})
%       uses its own counter.
%    \item[\ttfamily \bslash newtheorem\{foo2\}{[foo]}\{bar2\}]
%       The theorem set {\envfont foo2} (printed name \texttt{bar2})
%       uses the same counter as the theorem set \texttt{foo}.
%    \item[\ttfamily \bslash newtheorem\{foo3\}\{bar3\}{[section]}]
%       The theorem set {\envfont foo3} (printed name \texttt{bar3}) is
%       enumerated within the counter \texttt{section}, i.e.\ with every
%       new |\section| the enumeration begins again with 1, and
%       the enumeration is composed from the section-number and the
%       theorem counter itself.
% \end{description}
%
% \DescribeMacro\theoremstyle
% Additionally, the command |\theoremstyle| can define the layout
% of various, or all, theorem sets.  It should be noted that any
% theorem set defined by |\newtheorem| is typeset in the
% |\theoremstyle| that is current at the time of the definition.
% Thus, the following
% \begin{verbatim}
%    \theoremstyle{break}        \newtheorem{Cor}{Corollary}
%    \theoremstyle{plain}        \newtheorem{Exa}{Example}[section]
%\end{verbatim}                             ^^A No blank at line start
% leads to the result that the set {\envfont Cor} is formatted in the
% style \texttt{break}, while the set {\envfont Exa} and all the
% following ones are formatted in the style \texttt{plain}, unless
% another |\theoremstyle| follows.  Since the definitions installed by
% |\newtheorem| are global, one also can limit |\theoremstyle| locally
% by grouping braces.
%
% \DescribeMacro\theorembodyfont
% The choice of the font for the theorem body is completely
% independent of the chosen |\theoremstyle|; this has proven to
% be very advantageous.  For example,
% \begin{verbatim}
%   {\theorembodyfont{\upshape}          \newtheorem{Rem}{Remark}}
%\end{verbatim}
% defines a theorem set {\envfont Rem}, which will be set in
% |\upshape| in the current layout (which in our example is
% \texttt{plain}).  As with |\theoremstyle|, the |\theorembodyfont|
% chosen is that current at the time of |\newtheorem|.  If
% |\theorembodyfont| is not specified or one defines
% |\theorembodyfont{}|, then the font used will be that defined by the
% |\theoremstyle|.
%
% \DescribeMacro\theoremheaderfont It is also possible to customize
% the font used for the theorem headers.  This is, however, a global
% declaration, and therefore there should be at most one
% |\theoremheaderfont| declaration in the preamble.\footnote{If it is
% actually necessary to have different header fonts, one has to define
% new theorem styles (substituting the desired font) or specify the
% information directly in the \texttt{\bslash newtheorem} declaration
% (the unclean variant).}
%
% \DescribeMacro\theorempreskipamount
% \DescribeMacro\theorempostskipamount Two additional parameters
% affect the vertical space around the theorem environments:
% |\theorempreskipamount| and |\theorempostskipamount| define,
% respectively, the spacing before and after such an environment.
% These parameters apply for all theorem sets and can be manipulated
% with the ordinary length macros.  They are rubber lengths
% (`\textsf{skips}'), and therefore can contain \texttt{plus} and
% \texttt{minus} parts.
%
% Since the definition of theorem sets should---most sensibly---be
% placed in the preamble, we only allow installation there.
% It is therefore possible to release the memory used here after
% |\begin{document}|, in order to make room for other
% applications.
%
%
% \subsection{Existing theorem styles}
%
% The following theorem styles exist to date:
% \begin{deflist}{marginbreak}
%    \item[plain]
%       This theorem style emulates the original \LaTeX{} definition,
%       except that additionally the parameters
%       |\theorem...skipamount| are used.
%    \item[break]
%       In this style, the theorem header is followed by a line
%       break.
%    \item[marginbreak]
%       The theorem number is set in the margin, and there is a line
%       break as in \texttt{break}.
%    \item[changebreak]
%       Like \texttt{break}, but with header number and text
%       interchanged.
%    \item[change]
%       Header number and text are interchanged, without a line break.
%    \item[margin]
%       The number is set in the left margin, without a line break.
% \end{deflist}
% All styles (except \texttt{plain}) select |\slshape| as the default
% |\theorembodyfont|.
%
%
% \subsection{Examples}
%
% Given the above theorem sets {\envfont Cor}, {\envfont Exa}
% and {\envfont Rem},
% suppose that the preamble also contains the declarations:
% \begin{verbatim}
%    \theoremstyle{marginbreak}   \newtheorem{Lem}[Cor]{Lemma}
%    \theoremstyle{change}
%    \theorembodyfont{\itshape}        \newtheorem{Def}[Cor]{Definition}
%
%    \theoremheaderfont{\scshape}
%\end{verbatim}
% Then the following are some typical examples of the typeset output
% resulting from their use.
%
% \begin{Cor}
%    This is a sentence typeset in the theorem environment {\envfont
%    Cor}.
% \end{Cor}
%
% \begin{Exa}
%    This is a sentence typeset in the theorem environment {\envfont
%    Exa}.
% \end{Exa}
%
% \begin{Rem}
%    This is a sentence typeset in the theorem environment {\envfont
%    Rem}.
% \end{Rem}
%
% \begin{Lem}[Ben User]
%    This is a sentence typeset in the theorem environment {\envfont
%    Lem}.
% \end{Lem}
%
% \begin{Def}[Very impressive Definition]
%    This is a sentence typeset in the theorem environment {\envfont
%    Def}.
% \end{Def}
%
% The last two examples show the effect of the optional argument to
% a theorem environment (it is the text typeset in parentheses).
%
%
% \section{Special Considerations}
% \label{sec:newfonts}
%
% Theoremheader and body are implemented as a unit. This means that
% the |\theoremheaderfont|  will inherit characteristics of the
% |\theorembodyfont| in \LaTeXe. Thus, if for example
% |\theorembodyfont| is |\itshape| and |\theoremheaderfont|
% is |\bfseries| the font selected for the header will have the
% characteristics `bold extended italic'. If this is not desired one
% should set the |\theoremheaderfont| to something like
% \begin{verbatim}
%      \theoremheaderfont{\normalfont\bfseries}
%\end{verbatim}
% i.e.\ supplying all necessary font information explicitly.
%
%
%
% \section{Acknowledgements}
%
% The publication of this set of macros was only possible with the
% help of Christina Busse (translating the manuscript into English),
% Joachim Pense (playing the r\^ole of typist), Chris Rowley
% (looking everything over) and many others providing useful
% suggestions.
%
% \MaybeStop{%
%    \begin{thebibliography}{9}
%       \bibitem{bk:GMS94} \textsc{M.~Goossens}, \textsc{F.~Mittelbach}
%         and \textsc{A.~Samarin}.
%         \newblock The \LaTeX{} Companion.
%         \newblock
%         Addison-Wesley, Reading, Massachusetts, 1994.
%       \bibitem{latex:source} \textsc{Lamport, Leslie.} \newblock
%                               \texttt{latex.tex}, version 2.09, date
%                               Feb. 1990.
%    \end{thebibliography}
%    \PrintIndex
%    \PrintChanges}
%
%
% \section{The documentation driver file}
%
% The next bit of code contains the documentation driver file for
% \TeX{}, i.e., the file that will produce the documentation you are
% currently reading. It will be extracted from this file by the
% \texttt{docstrip} program. Since it is the first code in the file
% one can alternatively process this file directly with \LaTeXe{} to
% obtain the documentation.
%
%    \begin{macrocode}
%<*driver>
\documentclass{ltxdoc}

\usepackage{theorem}

%<+driver>% The next few lines define theorem sets which are used
%<+driver>% in the example section of the documentation.

\theoremstyle{break}       \newtheorem{Cor}{Corollary}
\theoremstyle{plain}       \newtheorem{Exa}{Example}[section]
{\theorembodyfont{\upshape}\newtheorem{Rem}{Remark}}
\theoremstyle{marginbreak} \newtheorem{Lem}[Cor]{Lemma}
\theoremstyle{change}
\theorembodyfont{\itshape} \newtheorem{Def}[Cor]{Definition}

\theoremheaderfont{\scshape}

\RecordChanges

\begin{document}
   \DocInput{theorem.dtx}
\end{document}
%</driver>
%    \end{macrocode}
%
% \section{Definition of the Macros}
%
% If the file has been loaded before, we abort immediately. If not the
% package announces itself (this is actually done at the very top of
% the file---the way it is done isn't good style so don't copy it).
% \changes{v2.0e}{89/07/19}{Spaces removed from `typeout'.}
% \changes{v2.2b}{95/11/19}{Announce moved to top of file}
%    \begin{macrocode}
%<*package>
%\@ifundefined{theorem@style}{}{\endinput}
%\def\FMithmInfo{1995/11/19 v2.2b Theorem extension package (FMi)}
%\ProvidesPackage{theorem}[\FMithmInfo]
%    \end{macrocode}
%
%
% \subsection{Definition of theorem styles and fonts}
%
% All the definitions in this file are done globally to allow
% inputting this file inside a group.
%
% \begin{macro}{\theoremstyle}
%    Before a theorem style can be installed, the chosen style must be
%    known. For that reason, we must test to see that
%    |\th@|\meta{style} is known or, more precisely, that it is
%    different from |\relax|. If the style is not known then
%    |\th@plain| is used.
%    \begin{macrocode}
\gdef\theoremstyle#1{%
   \@ifundefined{th@#1}{\@warning
          {Unknown theoremstyle `#1'. Using `plain'}%
          \theorem@style{plain}}%
%    \end{macrocode}
%    We save the theorem style to be used in the token register
%    |\theorem@style|.
%    \begin{macrocode}
      {\theorem@style{#1}}%
%    \end{macrocode}
%    Now we ``evaluate'' the theorem style: this means, we call the
%    macro |\th@|\meta{style} which will activate the relevant
%    definitions which are contained in a separate file.  This is done
%    in a group to suppress changes to the current font.  This could
%    otherwise pose problems together with the new font selection
%    scheme\footnote{When I printed the original article using the new
%    font selection scheme I ended with a document with slanted
%    typefaces (text headings and all) simply because one of the
%    theorem styles used \texttt{\bslash sl} at toplevel.} if the
%    |\th@|\meta{style} is evaluated a second time.
% \changes{v2.1a}{90/02/15}{Group added around `th@..}
% \changes{v2.1a}{90/02/15}{Reseting of `..@begintheorem removed.}
%    \begin{macrocode}
      \begingroup
        \csname th@\the\theorem@style \endcsname
      \endgroup}
%    \end{macrocode}
% \end{macro}
%
%
% \begin{macro}{\@begintheorem}
% \changes{v2.1a}{90/02/15}{Relaxed at toplevel.}
% \begin{macro}{\@opargbegintheorem}
% \changes{v2.1a}{90/02/15}{Relaxed at toplevel.}
%    We reset |\@begintheorem| and |\@opargbegintheorem| to
%    |\relax| since these commands are no longer necessary at
%    toplevel. This will save a few tokens.
%    \begin{macrocode}
\global\let\@begintheorem\relax
\global\let\@opargbegintheorem\relax
%    \end{macrocode}
% \end{macro}
% \end{macro}
%
% \begin{macro}{\theorem@style}
%    Obviously the \textsf{token} register used above has to be
%    allocated. To assure the utmost compatibility with the
%    original \LaTeX{} definition, we set the default theorem style to
%    \texttt{plain}, which implements the usual \LaTeX{} convention.
% \changes{v2.0f}{89/07/20}{Unnecessary `global' removed.}
%    \begin{macrocode}
\newtoks\theorem@style
\global\theorem@style{plain}
%    \end{macrocode}
% \end{macro}
%
% \begin{macro}{\theorembodyfont}
% \begin{macro}{\theorem@bodyfont}
%    If the |\theorembodyfont| is set by the user then it should not
%    interact with the default font set in the theorem style. When the
%    new font selection is in force this may happen if, for example,
%    the default is |\itshape| and the new |\theorembodyfont| is
%    |\sffamily|. So we add a |\reset@font| command in front of the
%    user definition.
% \changes{v2.0f}{89/07/20}{Unnecessary `global' removed.}
% \changes{v2.0g}{89/09/19}{`theoremfont to `theorembodyfont.}
% \changes{v2.1c}{92/06/22}{`reset@font added and logic changed.}
% \changes{v2.1d}{92/10/28}{Reset to empty if arg is empty}
%    \begin{macrocode}
\gdef\theorembodyfont#1{%
%    \end{macrocode}
%    We check if the argument supplied is empty and if so put nothing
%    into the |\theorem@bodyfont| token register to allow for
%    |\theorembodyfont{}| as a mean of using the default of the
%    current |\theoremstyle|.
%    \begin{macrocode}
   \def\@tempa{#1}%
   \ifx\@tempa\@empty
    \theorem@bodyfont{}%
   \else
    \theorem@bodyfont{\reset@font#1}%
   \fi
   }
\newtoks\theorem@bodyfont
\global\theorem@bodyfont{}
%    \end{macrocode}
% \end{macro}
% \end{macro}
%
%
% \begin{macro}{\theoremheaderfont}
% \changes{v2.0g}{89/09/19}{Macro added.}
%    The font for the theorem headers is handled differently because
%    this definition applies to all theorem styles.
%    \begin{macrocode}
\gdef\theoremheaderfont#1{\gdef\theorem@headerfont{#1}%
%    \end{macrocode}
%    After using the macro once it is redefined to produce an error
%    message.
%    \begin{macrocode}
       \gdef\theoremheaderfont##1{%
        \typeout{\string\theoremheaderfont\space should be used
                 only once.}}}
%    \end{macrocode}
% \end{macro}
%
% \begin{macro}{\theorem@headerfont}
% \changes{v2.0g}{89/09/19}{Macro added.}
%    To set the |\theorem@headerfont| default we first test if
%    the new fontselection scheme is in force.
% \changes{v2.1a}{90/02/15}{Support for new font selection.}
%    \begin{macrocode}
\ifx\upshape\undefined
%    \end{macrocode}
%    If not we define it to expand into |\bfseries|. We don't use
%    |\let| just in case a following style option redefines this
%    macro.
%    \begin{macrocode}
\gdef\theorem@headerfont{\bfseries}
%    \end{macrocode}
%    Otherwise we reset the current shape before calling |\bfseries|.
%    \begin{macrocode}
\else \gdef\theorem@headerfont{\normalfont\bfseries}\fi
%    \end{macrocode}
% \end{macro}
%
% \begin{macro}{\th@plain}
% \begin{macro}{\th@break}
% \begin{macro}{\th@marginbreak}
% \begin{macro}{\th@changebreak}
% \begin{macro}{\th@change}
% \begin{macro}{\th@margin}
%    The different styles are defined in macros such as
%    |\th@plain|.  Since memory space is precious in
%    ``non-Big-versions'', we have to avoid offering too many unused
%    definitions.  Therefore we define these styles in separate files
%    that can be loaded on demand.  Thus the commands themselves only
%    load these files. We use |\@input@|, a \LaTeXe{} internal command
%    that ensures that the file will be listed with |\listfiles|.
% \changes{v2.2a}{94/02/02}{Use `@input@ to load theorem layout files}
%    \begin{macrocode}
\gdef\th@plain{\@input@{thp.sty}}
\gdef\th@break{\@input@{thb.sty}}
\gdef\th@marginbreak{\@input@{thmb.sty}}
\gdef\th@changebreak{\@input@{thcb.sty}}
\gdef\th@change{\@input@{thc.sty}}
\gdef\th@margin{\@input@{thm.sty}}
%    \end{macrocode}
%    This list will be expanded when new styles become
%    available.  For testing,  just append new theorem substyles
%    as document options.
% \end{macro}
% \end{macro}
% \end{macro}
% \end{macro}
% \end{macro}
% \end{macro}
%
%
%
% \subsection{Definition of a new theorem set}
%
% As already pointed out, a new theorem environment can be defined in
% three different ways:
% \begin{verbatim}
%    \newtheorem{Lem}{Lemma}
%    \newtheorem{Lem}{Lemma}[section]
%    \newtheorem{Lem}[Theorem]{Lemma}
%\end{verbatim}
% The function of the macro |\newtheorem| is to recognize these cases
% and then to branch into one of the three macros |\@ynthm|, |\@xnthm|
% or |\@othm|.  This mechanism is adopted unchanged from
% \cite{latex:source}; the essential point here is that, for example,
% in the second case, the arguments \texttt{Lem}, \texttt{Lemma} and
% \texttt{section} are passed over to the macro |\@xnthm|.
%
% We inspect this case first because the others present fewer problems,
% and thus are easily derived from this one.
%
% \begin{macro}{\@xnthm}
%    For our example arguments, the macro |\@xnthm| must fulfill
%    the following:
%    \begin{itemize}
%       \item
%          Define a new \LaTeX-counter `Lem'
%       \item
%          reset this counter within a |\section|
%       \item
%          define the macro |\theLem|
%       \item
%          define the environment macros |\Lem| and
%          |\endLem| using the current
%          |\theoremstyle| and |\theorem@bodyfont|.
%    \end{itemize}
%    Obviously, all this should happen only if the first argument
%    of |\@xnthm| (i.e.\ \texttt{Lem} in our example) is chosen so
%    as not to conflict with any previously defined commands or
%    environments.  This test is performed by the \LaTeX{} macro
%    |\@ifdefinable|.
%    \begin{macrocode}
\gdef\@xnthm#1#2[#3]{\expandafter\@ifdefinable\csname #1\endcsname
%    \end{macrocode}
%    Therefore, the first argument of |\@ifdefinable| is the
%    expansion (in the example, |\Lem|) of
%    |\csname||#1\endcsname|.  The second argument is
%    executed only if the test has been completed successfully.
%    \begin{macrocode}
   {%
%    \end{macrocode}
%    Now we define the new counter. The names of the \LaTeX{}
%    macros employed should speak for themselves:
% \changes{v2.2b}{1995/10/07}{Use \cs{@newctr}}
% \changes{v2.2c}{1995/11/23}{Use \cs{@newctr} properly.}
%    \begin{macrocode}
    \@definecounter{#1}\@newctr{#1}[#3]%
%    \end{macrocode}
%    Using |\@newctr| will give a proper error message if the counter
%    in |#3| is not defined.
%    In defining `|\theLem|' we must generate the desired macro
%    name by use of |\expandafter| and |\csname|.
%    \begin{macrocode}
    \expandafter\xdef\csname the#1\endcsname
%    \end{macrocode}
%    An |\xdef| is used in order to make the definition global,
%    and to ensure that it contains the replacement texts of
%    |\@thmcountersep| and |\@thmcounter|.\footnote{These
%    two macros can be defined by the document style.  Their default
%    values produce a `\texttt{.}' as separation and an arabic
%    representation of the number.} However, not everything should be
%    expanded. For example, it saves space to use |\thesection|
%    instead of its---at times---lengthy expansion.
%    \begin{macrocode}
      {\expandafter \noexpand \csname the#3\endcsname
       \@thmcountersep \@thmcounter{#1}}%
%    \end{macrocode}
%    Thus with the defaults of \LaTeX{}, |\theLem| would be
%    replaced by the command sequence |\thesection.\arabic{Lem}|.
%
%    We will now look at the definition of the macro which is executed
%    at the beginning of the actual environment (in our example this
%    macro is |\Lem|). It should be noted that we use an
%    ``|\expandafter| trick'' to expand only certain parts of the
%    replacement text at the time of the definition.
%    \begin{macrocode}
    \def\@tempa{\global\@namedef{#1}}%
    \expandafter \@tempa \expandafter{%
%    \end{macrocode}
%    First, the macro that contains the current definitions of
%    |\@begintheorem| and |\@opargtheorem| should be called
%    up.  The name of this macro---as is already known---has the form
%    |\th@|\meta{theorem\ style}; therefore, it must be called by
%    \begin{macrocode}
      \csname th@\the \theorem@style
%    \end{macrocode}
%    In addition the default theorem font should be changeable, i.e.\
%    we have to insert the contents of |\theorem@bodyfont|. For
%    that reason, we expand even further, beyond |\endcsname|,
%    and thus insert the contents of the \textsf{token} register
%    |\theorem@bodyfont| in the replacement text.
%    \begin{macrocode}
            \expandafter \endcsname \the \theorem@bodyfont
%    \end{macrocode}
%    Now it is time to call the macro |\@thm| which takes over
%    the further processing.  It has two arguments: the current
%    counter name (in our example, \texttt{Lem}), and the text of the
%    label (in our example, \texttt{Lemma}).
%    \begin{macrocode}
     \@thm{#1}{#2}}%
%    \end{macrocode}
%    With this, the `sub-definition' is complete.  The macro
%    |\@endtheorem| ends a theorem environment and is, so far, nothing
%    but an |\endtrivlist|.  (Hence it is defined globally, and not
%    within the theorem styles.\footnote{This has to be changed as
%    soon as theorem styles that change \texttt{\bslash @endtheorem}
%    exist.  In such a case, all existing styles must be changed as
%    well since they will have to reset the macro.}) Therefore, we can
%    set it equivalent to the macro that ends the theorem set (in our
%    example, |\endLem|).  However, if some day theorem styles exist
%    that do change |\@endtheorem|, we would have to use the
%    commented-out line instead.
%    \begin{macrocode}
    \global \expandafter \let \csname end#1\endcsname \@endtheorem
%   \global\@namedef{end#1}{\@endtheorem}%
%    \end{macrocode}
%    With these commands all the required definitions are employed,
%    unless the test |\@ifdefinable| has failed.  Therefore, we
%    end the second argument of this macro and with it the definition
%    of |\@xnthm|.
%    \begin{macrocode}
   }}
%    \end{macrocode}
% \end{macro}
%
% \begin{macro}{\@ynthm}
%    The definition of |\@ynthm| is completely analogous. In
%    this case the new counter that is defined is not reset within
%    another counter; thus the definition of |\the...|
%    is simplified:
%    \begin{macrocode}
\gdef\@ynthm#1#2{\expandafter\@ifdefinable\csname #1\endcsname
   {\@definecounter{#1}%
    \expandafter\xdef\csname the#1\endcsname{\@thmcounter{#1}}%
%    \end{macrocode}
%    The rest of the definition corresponds literally to that of
%    |\@xnthm|:
%    \begin{macrocode}
    \def\@tempa{\global\@namedef{#1}}\expandafter \@tempa
     \expandafter{\csname th@\the \theorem@style \expandafter
     \endcsname \the\theorem@bodyfont \@thm{#1}{#2}}%
    \global \expandafter \let \csname end#1\endcsname \@endtheorem}}
%    \end{macrocode}
% \end{macro}
%
% \begin{macro}{\@othm}
%    The definition of |\@othm| does not contain anything new.
%    \begin{macrocode}
\gdef\@othm#1[#2]#3{%
%    \end{macrocode}
% \changes{v2.2b}{95/10/07}{Build in check for counter defined}
%    We do not define a new counter but instead use one that has
%    already been defined.  Thus the only definition we need is that
%    of this pseudo-counter (i.e.\ |\the|\meta{env.\ name}).
%    First we check if |#2| corresponds to a known counter name.
%    \begin{macrocode}
  \expandafter\ifx\csname c@#2\endcsname\relax
   \@nocounterr{#2}%
  \else
   \expandafter\@ifdefinable\csname #1\endcsname
   {\expandafter \xdef \csname the#1\endcsname
     {\expandafter \noexpand \csname the#2\endcsname}%
%    \end{macrocode}
%    All other parts of the definition can be adopted from
%    |\@xnthm|. We have to remember, though, that in this case
%    the name of the current counter and the theorem label have moved
%    to the second and third arguments.
%    \begin{macrocode}
    \def\@tempa{\global\@namedef{#1}}\expandafter \@tempa
     \expandafter{\csname th@\the \theorem@style \expandafter
     \endcsname \the\theorem@bodyfont \@thm{#2}{#3}}%
    \global \expandafter \let \csname end#1\endcsname \@endtheorem}%
  \fi}
%    \end{macrocode}
% \end{macro}
%
% \subsection{Macros that are employed in a {\envfont theorem}
%    environment}
%
% \begin{macro}{\@thm}
%    The macro |\@thm| has to increase the current counter.  Then,
%    depending on whether the environment has (or does not have) an
%    optional argument, it has to branch into either |\@begintheorem|
%    or |\@opargtheorem|.
%    \begin{macrocode}
\gdef\@thm#1#2{\refstepcounter{#1}%
%    \end{macrocode}
%    Now we start a {\envfont trivlist} environment, and give
%    |\@topsep| and |\@topsepadd| the values of the \textsf{skip}
%    registers |\theorempreskipamount| and |\theorempostskipamount|.
%    The value in |\@topsep| is the vertical space that is inserted by
%    the first (and only) |\item| in our |\trivlist| whilst
%    |\@topsepadd| is inserted by |\@endparenv| at the end of that
%    {\envfont trivlist} environment.  By using these registers, we
%    obtain the desired space around a {\envfont theorem} environment.
%    \begin{macrocode}
   \trivlist
   \@topsep \theorempreskipamount               % used by first \item
   \@topsepadd \theorempostskipamount           % used by \@endparenv
%    \end{macrocode}
%    Now we have to test whether an optional argument has been given.
% \changes{v2.0d}{89/07/18}{Forgotten percent added.}
%    \begin{macrocode}
   \@ifnextchar [%
%    \end{macrocode}
%    If there is an optional argument, we will call |\@ythm|, and
%    move the arguments read back into the input stream.
%    \begin{macrocode}
   {\@ythm{#1}{#2}}%
%    \end{macrocode}
%    If not, we call |\@begintheorem|. Its first argument is the
%    name of the theorem set (hence the second argument of
%    |\@thm|). Its second argument is the macro that produces the
%    current number.
%    \begin{macrocode}
   {\@begintheorem{#2}{\csname the#1\endcsname}\ignorespaces}}
%    \end{macrocode}
% \end{macro}
%
% \begin{macro}{\@xthm}
% \begin{macro}{\@ythm}
%    Both these macros were originally called by |\@thm|.  We do
%    not need |\@xthm| anymore, hence we reset it to
%    |\relax|. The definition of |\@ythm| has not changed at
%    all from its definition in \LaTeX. In order to make the macros
%    easier to understand, we will
%    nevertheless present it (commented out).
%    \begin{macrocode}
\global\let\@xthm\relax
% \def\@ythm#1#2[#3]{\@opargbegintheorem{#2}{\csname
%           the#1\endcsname}{#3}\ignorespaces}
%</package>
%    \end{macrocode}
%    The primitive |\ignorespaces| in |\@ythm| and
%    |\@thm| is needed to remove the spaces between the
%    |\begin{...}| and the actual text.
% \end{macro}
% \end{macro}
%
%
% \subsection{Definition of the theorem substyles}
%
% As already pointed out, the theorem substyles, defined below, are
% only loaded when necessary.  Note that all these substyles, except
% \texttt{plain}, have |\slshape| as the default body font.
%
% \subsubsection{The \texttt{plain} style}
%
% \label{dok:th@plain} As the following macros use |@|, we have
% to locally set the |\catcode| of this symbol to ``letter''.
% This happens within a group, so that we do not have to worry about
% which |\catcode| that symbol had before.
%    \begin{macrocode}
%<*thp>
\begingroup \makeatletter
%    \end{macrocode}
% Since we are now within a group, we must make all definitions
% globally.  First we make sure that \texttt{theorem.sty} is loaded.
% This will allow us to use this file as a document style option
% without having to call \texttt{theorem} itself as an option.  At the
% same time, we assure that at least version 2 is loaded, since
% |\theorem@style| was not defined in earlier versions.
%    \begin{macrocode}
\@ifundefined{theorem@style}{\input{theorem.sty}}{}
\ProvidesFile{thp.sty}
             [\FMithmInfo]
%    \end{macrocode}
%
%
% \begin{macro}{\th@plain}
% \begin{macro}{\@begintheorem}
% \begin{macro}{\@opargbegintheorem}
%    |\theoremstyle{plain}| corresponds to the original
%    definition, except that the distances to the surrounding text are
%    determined by the parameters |\theorempreskipamount| and
%    |\theorempostskipamount|.
%    First we set the default body font.
%    \begin{macrocode}
\gdef\th@plain{\normalfont\itshape
%    \end{macrocode}
%    Then we define |\@begintheorem| and
%    |\@opargbegintheorem|. These two macros define how the
%    header of a theorem is typeset.  |\@opargbegintheorem| will
%    be called if a \textsf{theorem} environment with an optional
%    argument is encountered; otherwise, the header is constructed by
%    calling |\@begintheorem|.  If one of these macros is
%    executed, we are within a \textsf{trivlist} environment started by
%    |\@thm|.  So the theorem header is produced with an
%    |\item| command.
%
%    Instead of specifying the header font directly, all standard
%    theorem styles use the |\theorem@headerfont| macro to allow
%    customization.  The extra space (|\labelsep|) is necessary
%    because of problems in the \textsf{trivlist} environment.
%    \begin{macrocode}
  \def\@begintheorem##1##2{%
        \item[\hskip\labelsep \theorem@headerfont ##1\ ##2]}%
%    \end{macrocode}
%    The definition of |\@opargbegintheorem| is completely
%    analogous. The only difference is the fact that there exists a
%    third argument (which is the optional parameter of the
%    environment and contains additional information about the
%    theorem).  Customarily we enclose it in parentheses.
%    \begin{macrocode}
\def\@opargbegintheorem##1##2##3{%
   \item[\hskip\labelsep \theorem@headerfont ##1\ ##2\ (##3)]}}
%    \end{macrocode}
% \end{macro}
% \end{macro}
% \end{macro}
%
% \noindent
% We conclude with an |\endgroup| to restore the
% |\catcode| of |@|.
%    \begin{macrocode}
\endgroup
%</thp>
%    \end{macrocode}
%
%
% \subsubsection{The \texttt{break} style}
%
% This style option is stored in the file |thb.sty|.
% For the next two lines see the documentation for |\th@plain| on
% page~\pageref{dok:th@plain}.
%    \begin{macrocode}
%<*thb>
\begingroup \makeatletter
\@ifundefined{theorem@style}{\input{theorem.sty}}{}
\ProvidesFile{thb.sty}
             [\FMithmInfo]
%    \end{macrocode}
%
% \begin{macro}{\th@break}
%    |\theoremstyle{break}| produces a line break after the name
%    of the theorem.  The font is |\slshape|.  Hence, we define
%    |\th@break| as follows:
%    \begin{macrocode}
\gdef\th@break{\normalfont\slshape
  \def\@begintheorem##1##2{\item[%
%    \end{macrocode}
%    We run into the following problem: it is not possible to create
%    the header with |\item[|\meta{title}|]| and then start
%    a new line by, for example, |\mbox{}\\|. Such a definition
%    will fail whenever a list environment follows immediately.  With
%    the above construction, the |\mbox{}| causes the switch
%    |@inlabel| (cf.\ definition of |\list| and
%    |\trivlist| in \cite{latex:source}) to be set to \texttt{false}
%    and so the following list will insert additional vertical
%    space (|\topskip|). This is quite annoying. Therefore, we
%    create the line break within the |\item|. In order to ensure
%    that the text will begin at the proper position in the following
%    line, we simply pretend that the label does not take any
%    room.\footnote{This will lead to problems whenever very high
%    symbols occurring in the line tower into the
%    heading. So, something else has to be done here sometime.}
%    \begin{macrocode}
       \rlap{\vbox{\hbox{\hskip \labelsep\theorem@headerfont ##1\ ##2}%
                   \hbox{\strut}}}]}%
%    \end{macrocode}
%    Again, the definition of |\@opargbegintheorem| is
%    completely analogous.
%    \begin{macrocode}
\def\@opargbegintheorem##1##2##3{%
  \item[\rlap{\vbox{\hbox{\hskip \labelsep \theorem@headerfont
                     ##1\ ##2\ (##3)}%
                    \hbox{\strut}}}]}}
\endgroup
%</thb>
%    \end{macrocode}
% \end{macro}
%
%
% \subsubsection{The \texttt{changebreak} style}
%
%    \begin{macrocode}
% This style option is stored in the file |thcb.sty|.
%    \begin{macrocode}
%<*thcb>
\begingroup \makeatletter
\@ifundefined{theorem@style}{\input{theorem.sty}}{}
\ProvidesFile{thcb.sty}
             [\FMithmInfo]
%    \end{macrocode}
%
% \begin{macro}{\th@changebreak}
%    The \texttt{change-break} theorem style is like \texttt{break}
%    but with interchange of theorem name and theorem number.  Thus we
%    define |\th@changebreak| as follows:
%    \begin{macrocode}
\gdef\th@changebreak{\normalfont\slshape
  \def\@begintheorem##1##2{\item
       [\rlap{\vbox{\hbox{\hskip\labelsep \theorem@headerfont ##2\ ##1}%
                    \hbox{\strut}}}]}%
\def\@opargbegintheorem##1##2##3{%
  \item[\rlap{\vbox{\hbox{\hskip\labelsep \theorem@headerfont
                     ##2\ ##1\ (##3)}%
                    \hbox{\strut}}}]}}
\endgroup
%</thcb>
%    \end{macrocode}
% \end{macro}
%
%
% \subsubsection{The \texttt{change} style}
%
% This style option is stored in the file |thc.sty|.
%    \begin{macrocode}
%<*thc>
\begingroup \makeatletter
\@ifundefined{theorem@style}{\input{theorem.sty}}{}
\ProvidesFile{thc.sty}
             [\FMithmInfo]
%    \end{macrocode}
%
% \begin{macro}{\th@change}
%    The \texttt{change} theorem style corresponds to the
%    \texttt{change break} style without a linebreak after the header.
%    To say it in another way, it's the same as the \texttt{plain}
%    style but with number and name interchanged and |\slshape| as the
%    default font.
%    \begin{macrocode}
\gdef\th@change{\normalfont\slshape
  \def\@begintheorem##1##2{\item
       [\hskip\labelsep \theorem@headerfont ##2\ ##1]}%
\def\@opargbegintheorem##1##2##3{%
  \item[\hskip\labelsep \theorem@headerfont ##2\ ##1\ (##3)]}}
\endgroup
%</thc>
%    \end{macrocode}
% \end{macro}
%
%
% \subsubsection{The \texttt{marginbreak} style}
%
% This style option is the one used most often at Mainz.  It is saved
% in the file |thmb.sty|.
%    \begin{macrocode}
%<*thmb>
\begingroup \makeatletter
\@ifundefined{theorem@style}{\input{theorem.sty}}{}
\ProvidesFile{thmb.sty}
             [\FMithmInfo]
%    \end{macrocode}
%
% \begin{macro}{\th@marginbreak}
%    The \texttt{margin break} style is nearly the same as the
%    \texttt{change break} style. The only difference is the placement
%    of the theorem number.  We use |\llap| to place it in the left
%    margin.
%
%    In this style |\labelsep| denotes the separation between the
%    number and the text.
%    \begin{macrocode}
\gdef\th@marginbreak{\normalfont\slshape
  \def\@begintheorem##1##2{\item
       [\rlap{\vbox{\theorem@headerfont
                    \hbox{\llap{##2}\hskip\labelsep ##1}%
                    \hbox{\strut}}}]}%
\def\@opargbegintheorem##1##2##3{%
  \item[\rlap{\vbox{\theorem@headerfont
                    \hbox{\llap{##2}\hskip\labelsep ##1\ (##3)}%
                    \hbox{\strut}}}]}}
\endgroup
%</thmb>
%    \end{macrocode}
% \changes{v2.1c}{90/02/17}{`endgroup outside macrocode env.}
% \end{macro}
%
%
%
% \subsubsection{The \texttt{margin} style}
%
% This style option is stored in the file |thm.sty|.
%    \begin{macrocode}
%<*thm>
\begingroup \makeatletter
\@ifundefined{theorem@style}{\input{theorem.sty}}{}
\ProvidesFile{thm.sty}
             [\FMithmInfo]
%    \end{macrocode}
%
% \begin{macro}{\th@margin}
%    Again this is only a variant of the theorem styles described
%    above without any new ideas.
%    \begin{macrocode}
\gdef\th@margin{\normalfont\slshape
  \def\@begintheorem##1##2{\item
       [\theorem@headerfont \llap{##2}\hskip\labelsep ##1]}%
\def\@opargbegintheorem##1##2##3{%
  \item[\theorem@headerfont \llap{##2}\hskip\labelsep ##1\ (##3)]}}
\endgroup
%</thm>
%    \end{macrocode}
% \end{macro}
%
%
%
%
% \subsection{Final Definitions}
%
% \begin{macro}{\theorempreskipamount}
% \begin{macro}{\theorempostskipamount}
%    The \textsf{skip} parameters that regulate the vertical empty space
%    before and after the {\envfont theorem} environment have
%    to be allocated as well.
% \changes{v2.0f}{89/07/20}{Unnecessary `global' removed.}
%    \begin{macrocode}
%<*package>
\newskip\theorempreskipamount
\newskip\theorempostskipamount
%    \end{macrocode}
%    Since we have used the same values for all theorem sets, we now can
%    assign them.
%    \begin{macrocode}
\global\setlength\theorempreskipamount{12pt plus 5pt minus 3pt}
\global\setlength\theorempostskipamount{8pt plus 3pt minus 1.5pt}
%    \end{macrocode}
% \end{macro}
% \end{macro}
%
%
% \begin{macro}{\@endtheorem}
%    The same holds for the macro |\@endtheorem|, which ends a
%    {\envfont theorem} environment. Since it is the same for all
%    theorem sets, it is removed from the macros |\th@|\meta{style}.
%    It simply ends the {\envfont trivlist} environment, which was
%    begun in |\@thm|.
%    \begin{macrocode}
\global\let\@endtheorem=\endtrivlist
%    \end{macrocode}
% \end{macro}
%
%
%
% \begin{macro}{\@preamblecmds}
%    All macros defined above are to be used only in the preamble.
%    Therefore, we insert them in |\@preamblecmds| which will disable
%    them at begin document. This is done by the internal \LaTeXe{}
%    command |\@onlypreamble|.
% \changes{v2.0g}{89/09/19}{`theoremheaderfont and `theorembodyfont
%    added.}
% \changes{v2.2a}{94/02/03}{Use \LaTeXe{} interface}
%    \begin{macrocode}
\@onlypreamble\@xnthm
\@onlypreamble\@ynthm
\@onlypreamble\@othm
\@onlypreamble\newtheorem
\@onlypreamble\theoremstyle
\@onlypreamble\theorembodyfont
\@onlypreamble\theoremheaderfont
%    \end{macrocode}
% \end{macro}
%
% Finally we declare the \texttt{plain} theorem style to be the default.
% \changes{v2.1b}{90/06/04}{Fixed missing `item by declaring default
%    style.}
%    \begin{macrocode}
\theoremstyle{plain}
%</package>
%    \end{macrocode}
%
% \Finale

\endinput
}{}
\ProvidesFile{thc.sty}
             [\FMithmInfo]
%    \end{macrocode}
%
% \begin{macro}{\th@change}
%    The \texttt{change} theorem style corresponds to the
%    \texttt{change break} style without a linebreak after the header.
%    To say it in another way, it's the same as the \texttt{plain}
%    style but with number and name interchanged and |\slshape| as the
%    default font.
%    \begin{macrocode}
\gdef\th@change{\normalfont\slshape
  \def\@begintheorem##1##2{\item
       [\hskip\labelsep \theorem@headerfont ##2\ ##1]}%
\def\@opargbegintheorem##1##2##3{%
  \item[\hskip\labelsep \theorem@headerfont ##2\ ##1\ (##3)]}}
\endgroup
%</thc>
%    \end{macrocode}
% \end{macro}
%
%
% \subsubsection{The \texttt{marginbreak} style}
%
% This style option is the one used most often at Mainz.  It is saved
% in the file |thmb.sty|.
%    \begin{macrocode}
%<*thmb>
\begingroup \makeatletter
\@ifundefined{theorem@style}{% \iffalse meta-comment
%
% Copyright (C) 1993-2024
%
% The LaTeX Project and any individual authors listed elsewhere
% in this file.
%
% This file is part of the Standard LaTeX `Tools Bundle'.
% -------------------------------------------------------
%
% It may be distributed and/or modified under the
% conditions of the LaTeX Project Public License, either version 1.3c
% of this license or (at your option) any later version.
% The latest version of this license is in
%    https://www.latex-project.org/lppl.txt
% and version 1.3c or later is part of all distributions of LaTeX
% version 2005/12/01 or later.
%
% The list of all files belonging to the LaTeX `Tools Bundle' is
% given in the file `manifest.txt'.
%
% \fi
%
%%
%
%
\def\FMithmInfo{2023/07/05 v2.2c Theorem extension package (FMi)}
%
% \ProvidesFile{theorem.dtx}[\FMithmInfo]
%\iffalse   % this is a METACOMMENT !
%%
%% Package `theorem' to use with LaTeX2e
%% Copyright (C) 1989-1995 Frank Mittelbach, all rights reserved.
%<+package>\@ifundefined{theorem@style}{}{\endinput}
%<+package>\NeedsTeXFormat{LaTeX2e}[1995/06/01]
%<+package>\ProvidesPackage{theorem}
%<+package>         [\FMithmInfo]
%
% \fi
%
% \changes{v1.0}{87/05/07}{First draft.}
% \changes{v1.0a}{87/05/07}{`thistheoremstyle' added.}
% \changes{v1.1a}{87/07/29}{New concept: redefine the internal
%                          \protect\LaTeX{} commands `@xnthm' etc.}
% \changes{v1.1b}{87/07/31}{`theoremfont' used to specify font.}
% \changes{v1.2a}{88/01/08}{`@endtheorem' now globally defined.}
% \changes{v1.2b}{88/01/11}{`newtheorem' only allowed in the preamble.}
% \changes{v1.3a}{88/02/14}{`theoremfont' now a toks register.}
%
% \changes{v2.0a}{88/11/16}{New implementation and new documentation.}
% \changes{v2.0a}{88/11/16}{No layout changes but internal modifications
%                           old substyles won't work.}
% \changes{v2.0b}{89/01/30}{`global' used for all definitions.}
% \changes{v2.0c}{89/06/15}{Documentation translated to English.
%                            Thanks to LLL and Nina}
% \changes{v2.1a}{90/02/15}{Support for new font selection.}
% \changes{v2.2a}{94/02/03}{Cleanup for LaTeX2e}
%
% \DoNotIndex{\@endparenv,\@ifdefinable,\@ifnextchar,\@ifundefined}
% \DoNotIndex{\@namedef,\@nameuse,\@spaces,\@tempa,\@warning}
% \DoNotIndex{\ }
% \DoNotIndex{\begingroup,\bfseries}
% \DoNotIndex{\csname}
% \DoNotIndex{\def,\do,\docdate}
% \DoNotIndex{\endcsname,\endgroup,\endinput,\endtrivlist,\expandafter}
% \DoNotIndex{\filedate,\fileversion}
% \DoNotIndex{\gdef,\global}
% \DoNotIndex{\hbox,\hskip}
% \DoNotIndex{\ignorespaces,\input,\itshape,\item}
% \DoNotIndex{\labelsep,\let,\llap}
% \DoNotIndex{\makeatletter}
% \DoNotIndex{\noexpand}
% \DoNotIndex{\refstepcounter,\relax,\rlap}
% \DoNotIndex{\setlength,\slshape,\space,\string,\strut}
% \DoNotIndex{\the,\trivlist,\typeout}
% \DoNotIndex{\vbox}
% \DoNotIndex{\xdef}
%
% \def\deflabel#1{\ttfamily #1\hfill}
% \def\deflist#1{\begin{list}{}{\settowidth\labelwidth{\ttfamily #1}%
%                               \setlength\leftmargin\labelwidth
%                               \addtolength\leftmargin\labelsep
%                               \let\makelabel\deflabel}}
% \def\enddeflist{\end{list}}
% \def\envfont{\normalfont\ttfamily}
%
% \MakeShortVerb{\|}
%
% \newcommand{\TUB}{{\sl TUGboat\/}}
%
% \GetFileInfo{theorem.dtx}
% \title{An Extension of the
%        \LaTeX{} theorem environment\thanks{This file
%        has version number \fileversion, last
%        revised \filedate.}}
% \author{Frank Mittelbach}
%
% \MaintainedByLaTeXTeam{tools}
% \maketitle
%
% \begin{abstract}
%    The macros described in this paper yield an extension of the
%    \LaTeX{} theorem mechanism. It is designed to satisfy the
%    different requirements of various journals. Thus, the
%    layout of the ``theorems'' can be manipulated by determining a
%    ``style''. This article describes not only the use, but
%    also the definition, of the necessary macros.
% \end{abstract}
%
% \section*{Preface to version 2.2}
%
% For \LaTeXe{} this package did not need any fundamental changes. I
% only modified the messages generated so that theorem layout styles
% will show up with the |\listfiles| command and cleaned the section
% on the New Font Selection Scheme since this is now included in
% \LaTeX.
%
% \section*{Preface to version 2.1}
%
% This version is identical to 2.0g described in \TUB\ 10\#3 except
% for some internal defaults which are now set depending on the used
% font selection scheme.
%
% This was done to avoid unpleasant surprises if the new font selection
% scheme is in force. For further details see
% section~\ref{sec:newfonts} and \cite{bk:GMS94}.
%
% \section{Introduction}
%
% For our purposes here, ``theorems'' are labelled enunciations,
% often set off from the main text by extra space and a font change.
% Theorems, corollaries, conjectures, definitions, and remarks
% are all instances of ``theorems''.  The ``header'' of these structures
% is composed of a label (such as \textsc{Theorem} or \textsc{Remark})
% and a number which serializes an item in the sequence of items
% with the same label.
%
% Shortly after the introduction of \LaTeX{} at the Fachbereich
% Mathematik in Mainz, the desire to manipulate the layout of
% ``theorems'' arose.  In Mainz, the following two conventions came
% into general use:
% \begin{enumerate}
%   \item The number of the theorem is shown in the margin.
%   \item There is a line break at
%         the end of the theorem header.
% \end{enumerate}
%
% Additionally, some journals require different formats which depend on
% the ``sort of theorem'': e.g.\ often remarks and definitions are set
% in |\upshape|, while |\itshape| is employed for main theorems.
%
% Confronted with these requirements, a theorem environment was
% developed in Mainz which allows separate determination of the
% layout of the ``theorems sets'', comparable to |\pagestyle|.
%
%
% \section{The user interface}
% \subsection{Defining new theorem sets}
%
% \DescribeMacro\newtheorem
% As in the original \LaTeX{} version, the command |\newtheorem|
% defines a new ``theorem set'' or ``theorem-like structure''.
% Two required arguments name the new environment and give the text
% to be typeset with each instance of the new ``set'', while
% an optional argument determines how
% the ``set'' is enumerated:
% \begin{description}
%    \item[\ttfamily \bslash newtheorem\{foo\}\{bar\}]
%       The theorem set {\envfont foo} (whose name is \texttt{bar})
%       uses its own counter.
%    \item[\ttfamily \bslash newtheorem\{foo2\}{[foo]}\{bar2\}]
%       The theorem set {\envfont foo2} (printed name \texttt{bar2})
%       uses the same counter as the theorem set \texttt{foo}.
%    \item[\ttfamily \bslash newtheorem\{foo3\}\{bar3\}{[section]}]
%       The theorem set {\envfont foo3} (printed name \texttt{bar3}) is
%       enumerated within the counter \texttt{section}, i.e.\ with every
%       new |\section| the enumeration begins again with 1, and
%       the enumeration is composed from the section-number and the
%       theorem counter itself.
% \end{description}
%
% \DescribeMacro\theoremstyle
% Additionally, the command |\theoremstyle| can define the layout
% of various, or all, theorem sets.  It should be noted that any
% theorem set defined by |\newtheorem| is typeset in the
% |\theoremstyle| that is current at the time of the definition.
% Thus, the following
% \begin{verbatim}
%    \theoremstyle{break}        \newtheorem{Cor}{Corollary}
%    \theoremstyle{plain}        \newtheorem{Exa}{Example}[section]
%\end{verbatim}                             ^^A No blank at line start
% leads to the result that the set {\envfont Cor} is formatted in the
% style \texttt{break}, while the set {\envfont Exa} and all the
% following ones are formatted in the style \texttt{plain}, unless
% another |\theoremstyle| follows.  Since the definitions installed by
% |\newtheorem| are global, one also can limit |\theoremstyle| locally
% by grouping braces.
%
% \DescribeMacro\theorembodyfont
% The choice of the font for the theorem body is completely
% independent of the chosen |\theoremstyle|; this has proven to
% be very advantageous.  For example,
% \begin{verbatim}
%   {\theorembodyfont{\upshape}          \newtheorem{Rem}{Remark}}
%\end{verbatim}
% defines a theorem set {\envfont Rem}, which will be set in
% |\upshape| in the current layout (which in our example is
% \texttt{plain}).  As with |\theoremstyle|, the |\theorembodyfont|
% chosen is that current at the time of |\newtheorem|.  If
% |\theorembodyfont| is not specified or one defines
% |\theorembodyfont{}|, then the font used will be that defined by the
% |\theoremstyle|.
%
% \DescribeMacro\theoremheaderfont It is also possible to customize
% the font used for the theorem headers.  This is, however, a global
% declaration, and therefore there should be at most one
% |\theoremheaderfont| declaration in the preamble.\footnote{If it is
% actually necessary to have different header fonts, one has to define
% new theorem styles (substituting the desired font) or specify the
% information directly in the \texttt{\bslash newtheorem} declaration
% (the unclean variant).}
%
% \DescribeMacro\theorempreskipamount
% \DescribeMacro\theorempostskipamount Two additional parameters
% affect the vertical space around the theorem environments:
% |\theorempreskipamount| and |\theorempostskipamount| define,
% respectively, the spacing before and after such an environment.
% These parameters apply for all theorem sets and can be manipulated
% with the ordinary length macros.  They are rubber lengths
% (`\textsf{skips}'), and therefore can contain \texttt{plus} and
% \texttt{minus} parts.
%
% Since the definition of theorem sets should---most sensibly---be
% placed in the preamble, we only allow installation there.
% It is therefore possible to release the memory used here after
% |\begin{document}|, in order to make room for other
% applications.
%
%
% \subsection{Existing theorem styles}
%
% The following theorem styles exist to date:
% \begin{deflist}{marginbreak}
%    \item[plain]
%       This theorem style emulates the original \LaTeX{} definition,
%       except that additionally the parameters
%       |\theorem...skipamount| are used.
%    \item[break]
%       In this style, the theorem header is followed by a line
%       break.
%    \item[marginbreak]
%       The theorem number is set in the margin, and there is a line
%       break as in \texttt{break}.
%    \item[changebreak]
%       Like \texttt{break}, but with header number and text
%       interchanged.
%    \item[change]
%       Header number and text are interchanged, without a line break.
%    \item[margin]
%       The number is set in the left margin, without a line break.
% \end{deflist}
% All styles (except \texttt{plain}) select |\slshape| as the default
% |\theorembodyfont|.
%
%
% \subsection{Examples}
%
% Given the above theorem sets {\envfont Cor}, {\envfont Exa}
% and {\envfont Rem},
% suppose that the preamble also contains the declarations:
% \begin{verbatim}
%    \theoremstyle{marginbreak}   \newtheorem{Lem}[Cor]{Lemma}
%    \theoremstyle{change}
%    \theorembodyfont{\itshape}        \newtheorem{Def}[Cor]{Definition}
%
%    \theoremheaderfont{\scshape}
%\end{verbatim}
% Then the following are some typical examples of the typeset output
% resulting from their use.
%
% \begin{Cor}
%    This is a sentence typeset in the theorem environment {\envfont
%    Cor}.
% \end{Cor}
%
% \begin{Exa}
%    This is a sentence typeset in the theorem environment {\envfont
%    Exa}.
% \end{Exa}
%
% \begin{Rem}
%    This is a sentence typeset in the theorem environment {\envfont
%    Rem}.
% \end{Rem}
%
% \begin{Lem}[Ben User]
%    This is a sentence typeset in the theorem environment {\envfont
%    Lem}.
% \end{Lem}
%
% \begin{Def}[Very impressive Definition]
%    This is a sentence typeset in the theorem environment {\envfont
%    Def}.
% \end{Def}
%
% The last two examples show the effect of the optional argument to
% a theorem environment (it is the text typeset in parentheses).
%
%
% \section{Special Considerations}
% \label{sec:newfonts}
%
% Theoremheader and body are implemented as a unit. This means that
% the |\theoremheaderfont|  will inherit characteristics of the
% |\theorembodyfont| in \LaTeXe. Thus, if for example
% |\theorembodyfont| is |\itshape| and |\theoremheaderfont|
% is |\bfseries| the font selected for the header will have the
% characteristics `bold extended italic'. If this is not desired one
% should set the |\theoremheaderfont| to something like
% \begin{verbatim}
%      \theoremheaderfont{\normalfont\bfseries}
%\end{verbatim}
% i.e.\ supplying all necessary font information explicitly.
%
%
%
% \section{Acknowledgements}
%
% The publication of this set of macros was only possible with the
% help of Christina Busse (translating the manuscript into English),
% Joachim Pense (playing the r\^ole of typist), Chris Rowley
% (looking everything over) and many others providing useful
% suggestions.
%
% \MaybeStop{%
%    \begin{thebibliography}{9}
%       \bibitem{bk:GMS94} \textsc{M.~Goossens}, \textsc{F.~Mittelbach}
%         and \textsc{A.~Samarin}.
%         \newblock The \LaTeX{} Companion.
%         \newblock
%         Addison-Wesley, Reading, Massachusetts, 1994.
%       \bibitem{latex:source} \textsc{Lamport, Leslie.} \newblock
%                               \texttt{latex.tex}, version 2.09, date
%                               Feb. 1990.
%    \end{thebibliography}
%    \PrintIndex
%    \PrintChanges}
%
%
% \section{The documentation driver file}
%
% The next bit of code contains the documentation driver file for
% \TeX{}, i.e., the file that will produce the documentation you are
% currently reading. It will be extracted from this file by the
% \texttt{docstrip} program. Since it is the first code in the file
% one can alternatively process this file directly with \LaTeXe{} to
% obtain the documentation.
%
%    \begin{macrocode}
%<*driver>
\documentclass{ltxdoc}

\usepackage{theorem}

%<+driver>% The next few lines define theorem sets which are used
%<+driver>% in the example section of the documentation.

\theoremstyle{break}       \newtheorem{Cor}{Corollary}
\theoremstyle{plain}       \newtheorem{Exa}{Example}[section]
{\theorembodyfont{\upshape}\newtheorem{Rem}{Remark}}
\theoremstyle{marginbreak} \newtheorem{Lem}[Cor]{Lemma}
\theoremstyle{change}
\theorembodyfont{\itshape} \newtheorem{Def}[Cor]{Definition}

\theoremheaderfont{\scshape}

\RecordChanges

\begin{document}
   \DocInput{theorem.dtx}
\end{document}
%</driver>
%    \end{macrocode}
%
% \section{Definition of the Macros}
%
% If the file has been loaded before, we abort immediately. If not the
% package announces itself (this is actually done at the very top of
% the file---the way it is done isn't good style so don't copy it).
% \changes{v2.0e}{89/07/19}{Spaces removed from `typeout'.}
% \changes{v2.2b}{95/11/19}{Announce moved to top of file}
%    \begin{macrocode}
%<*package>
%\@ifundefined{theorem@style}{}{\endinput}
%\def\FMithmInfo{1995/11/19 v2.2b Theorem extension package (FMi)}
%\ProvidesPackage{theorem}[\FMithmInfo]
%    \end{macrocode}
%
%
% \subsection{Definition of theorem styles and fonts}
%
% All the definitions in this file are done globally to allow
% inputting this file inside a group.
%
% \begin{macro}{\theoremstyle}
%    Before a theorem style can be installed, the chosen style must be
%    known. For that reason, we must test to see that
%    |\th@|\meta{style} is known or, more precisely, that it is
%    different from |\relax|. If the style is not known then
%    |\th@plain| is used.
%    \begin{macrocode}
\gdef\theoremstyle#1{%
   \@ifundefined{th@#1}{\@warning
          {Unknown theoremstyle `#1'. Using `plain'}%
          \theorem@style{plain}}%
%    \end{macrocode}
%    We save the theorem style to be used in the token register
%    |\theorem@style|.
%    \begin{macrocode}
      {\theorem@style{#1}}%
%    \end{macrocode}
%    Now we ``evaluate'' the theorem style: this means, we call the
%    macro |\th@|\meta{style} which will activate the relevant
%    definitions which are contained in a separate file.  This is done
%    in a group to suppress changes to the current font.  This could
%    otherwise pose problems together with the new font selection
%    scheme\footnote{When I printed the original article using the new
%    font selection scheme I ended with a document with slanted
%    typefaces (text headings and all) simply because one of the
%    theorem styles used \texttt{\bslash sl} at toplevel.} if the
%    |\th@|\meta{style} is evaluated a second time.
% \changes{v2.1a}{90/02/15}{Group added around `th@..}
% \changes{v2.1a}{90/02/15}{Reseting of `..@begintheorem removed.}
%    \begin{macrocode}
      \begingroup
        \csname th@\the\theorem@style \endcsname
      \endgroup}
%    \end{macrocode}
% \end{macro}
%
%
% \begin{macro}{\@begintheorem}
% \changes{v2.1a}{90/02/15}{Relaxed at toplevel.}
% \begin{macro}{\@opargbegintheorem}
% \changes{v2.1a}{90/02/15}{Relaxed at toplevel.}
%    We reset |\@begintheorem| and |\@opargbegintheorem| to
%    |\relax| since these commands are no longer necessary at
%    toplevel. This will save a few tokens.
%    \begin{macrocode}
\global\let\@begintheorem\relax
\global\let\@opargbegintheorem\relax
%    \end{macrocode}
% \end{macro}
% \end{macro}
%
% \begin{macro}{\theorem@style}
%    Obviously the \textsf{token} register used above has to be
%    allocated. To assure the utmost compatibility with the
%    original \LaTeX{} definition, we set the default theorem style to
%    \texttt{plain}, which implements the usual \LaTeX{} convention.
% \changes{v2.0f}{89/07/20}{Unnecessary `global' removed.}
%    \begin{macrocode}
\newtoks\theorem@style
\global\theorem@style{plain}
%    \end{macrocode}
% \end{macro}
%
% \begin{macro}{\theorembodyfont}
% \begin{macro}{\theorem@bodyfont}
%    If the |\theorembodyfont| is set by the user then it should not
%    interact with the default font set in the theorem style. When the
%    new font selection is in force this may happen if, for example,
%    the default is |\itshape| and the new |\theorembodyfont| is
%    |\sffamily|. So we add a |\reset@font| command in front of the
%    user definition.
% \changes{v2.0f}{89/07/20}{Unnecessary `global' removed.}
% \changes{v2.0g}{89/09/19}{`theoremfont to `theorembodyfont.}
% \changes{v2.1c}{92/06/22}{`reset@font added and logic changed.}
% \changes{v2.1d}{92/10/28}{Reset to empty if arg is empty}
%    \begin{macrocode}
\gdef\theorembodyfont#1{%
%    \end{macrocode}
%    We check if the argument supplied is empty and if so put nothing
%    into the |\theorem@bodyfont| token register to allow for
%    |\theorembodyfont{}| as a mean of using the default of the
%    current |\theoremstyle|.
%    \begin{macrocode}
   \def\@tempa{#1}%
   \ifx\@tempa\@empty
    \theorem@bodyfont{}%
   \else
    \theorem@bodyfont{\reset@font#1}%
   \fi
   }
\newtoks\theorem@bodyfont
\global\theorem@bodyfont{}
%    \end{macrocode}
% \end{macro}
% \end{macro}
%
%
% \begin{macro}{\theoremheaderfont}
% \changes{v2.0g}{89/09/19}{Macro added.}
%    The font for the theorem headers is handled differently because
%    this definition applies to all theorem styles.
%    \begin{macrocode}
\gdef\theoremheaderfont#1{\gdef\theorem@headerfont{#1}%
%    \end{macrocode}
%    After using the macro once it is redefined to produce an error
%    message.
%    \begin{macrocode}
       \gdef\theoremheaderfont##1{%
        \typeout{\string\theoremheaderfont\space should be used
                 only once.}}}
%    \end{macrocode}
% \end{macro}
%
% \begin{macro}{\theorem@headerfont}
% \changes{v2.0g}{89/09/19}{Macro added.}
%    To set the |\theorem@headerfont| default we first test if
%    the new fontselection scheme is in force.
% \changes{v2.1a}{90/02/15}{Support for new font selection.}
%    \begin{macrocode}
\ifx\upshape\undefined
%    \end{macrocode}
%    If not we define it to expand into |\bfseries|. We don't use
%    |\let| just in case a following style option redefines this
%    macro.
%    \begin{macrocode}
\gdef\theorem@headerfont{\bfseries}
%    \end{macrocode}
%    Otherwise we reset the current shape before calling |\bfseries|.
%    \begin{macrocode}
\else \gdef\theorem@headerfont{\normalfont\bfseries}\fi
%    \end{macrocode}
% \end{macro}
%
% \begin{macro}{\th@plain}
% \begin{macro}{\th@break}
% \begin{macro}{\th@marginbreak}
% \begin{macro}{\th@changebreak}
% \begin{macro}{\th@change}
% \begin{macro}{\th@margin}
%    The different styles are defined in macros such as
%    |\th@plain|.  Since memory space is precious in
%    ``non-Big-versions'', we have to avoid offering too many unused
%    definitions.  Therefore we define these styles in separate files
%    that can be loaded on demand.  Thus the commands themselves only
%    load these files. We use |\@input@|, a \LaTeXe{} internal command
%    that ensures that the file will be listed with |\listfiles|.
% \changes{v2.2a}{94/02/02}{Use `@input@ to load theorem layout files}
%    \begin{macrocode}
\gdef\th@plain{\@input@{thp.sty}}
\gdef\th@break{\@input@{thb.sty}}
\gdef\th@marginbreak{\@input@{thmb.sty}}
\gdef\th@changebreak{\@input@{thcb.sty}}
\gdef\th@change{\@input@{thc.sty}}
\gdef\th@margin{\@input@{thm.sty}}
%    \end{macrocode}
%    This list will be expanded when new styles become
%    available.  For testing,  just append new theorem substyles
%    as document options.
% \end{macro}
% \end{macro}
% \end{macro}
% \end{macro}
% \end{macro}
% \end{macro}
%
%
%
% \subsection{Definition of a new theorem set}
%
% As already pointed out, a new theorem environment can be defined in
% three different ways:
% \begin{verbatim}
%    \newtheorem{Lem}{Lemma}
%    \newtheorem{Lem}{Lemma}[section]
%    \newtheorem{Lem}[Theorem]{Lemma}
%\end{verbatim}
% The function of the macro |\newtheorem| is to recognize these cases
% and then to branch into one of the three macros |\@ynthm|, |\@xnthm|
% or |\@othm|.  This mechanism is adopted unchanged from
% \cite{latex:source}; the essential point here is that, for example,
% in the second case, the arguments \texttt{Lem}, \texttt{Lemma} and
% \texttt{section} are passed over to the macro |\@xnthm|.
%
% We inspect this case first because the others present fewer problems,
% and thus are easily derived from this one.
%
% \begin{macro}{\@xnthm}
%    For our example arguments, the macro |\@xnthm| must fulfill
%    the following:
%    \begin{itemize}
%       \item
%          Define a new \LaTeX-counter `Lem'
%       \item
%          reset this counter within a |\section|
%       \item
%          define the macro |\theLem|
%       \item
%          define the environment macros |\Lem| and
%          |\endLem| using the current
%          |\theoremstyle| and |\theorem@bodyfont|.
%    \end{itemize}
%    Obviously, all this should happen only if the first argument
%    of |\@xnthm| (i.e.\ \texttt{Lem} in our example) is chosen so
%    as not to conflict with any previously defined commands or
%    environments.  This test is performed by the \LaTeX{} macro
%    |\@ifdefinable|.
%    \begin{macrocode}
\gdef\@xnthm#1#2[#3]{\expandafter\@ifdefinable\csname #1\endcsname
%    \end{macrocode}
%    Therefore, the first argument of |\@ifdefinable| is the
%    expansion (in the example, |\Lem|) of
%    |\csname||#1\endcsname|.  The second argument is
%    executed only if the test has been completed successfully.
%    \begin{macrocode}
   {%
%    \end{macrocode}
%    Now we define the new counter. The names of the \LaTeX{}
%    macros employed should speak for themselves:
% \changes{v2.2b}{1995/10/07}{Use \cs{@newctr}}
% \changes{v2.2c}{1995/11/23}{Use \cs{@newctr} properly.}
%    \begin{macrocode}
    \@definecounter{#1}\@newctr{#1}[#3]%
%    \end{macrocode}
%    Using |\@newctr| will give a proper error message if the counter
%    in |#3| is not defined.
%    In defining `|\theLem|' we must generate the desired macro
%    name by use of |\expandafter| and |\csname|.
%    \begin{macrocode}
    \expandafter\xdef\csname the#1\endcsname
%    \end{macrocode}
%    An |\xdef| is used in order to make the definition global,
%    and to ensure that it contains the replacement texts of
%    |\@thmcountersep| and |\@thmcounter|.\footnote{These
%    two macros can be defined by the document style.  Their default
%    values produce a `\texttt{.}' as separation and an arabic
%    representation of the number.} However, not everything should be
%    expanded. For example, it saves space to use |\thesection|
%    instead of its---at times---lengthy expansion.
%    \begin{macrocode}
      {\expandafter \noexpand \csname the#3\endcsname
       \@thmcountersep \@thmcounter{#1}}%
%    \end{macrocode}
%    Thus with the defaults of \LaTeX{}, |\theLem| would be
%    replaced by the command sequence |\thesection.\arabic{Lem}|.
%
%    We will now look at the definition of the macro which is executed
%    at the beginning of the actual environment (in our example this
%    macro is |\Lem|). It should be noted that we use an
%    ``|\expandafter| trick'' to expand only certain parts of the
%    replacement text at the time of the definition.
%    \begin{macrocode}
    \def\@tempa{\global\@namedef{#1}}%
    \expandafter \@tempa \expandafter{%
%    \end{macrocode}
%    First, the macro that contains the current definitions of
%    |\@begintheorem| and |\@opargtheorem| should be called
%    up.  The name of this macro---as is already known---has the form
%    |\th@|\meta{theorem\ style}; therefore, it must be called by
%    \begin{macrocode}
      \csname th@\the \theorem@style
%    \end{macrocode}
%    In addition the default theorem font should be changeable, i.e.\
%    we have to insert the contents of |\theorem@bodyfont|. For
%    that reason, we expand even further, beyond |\endcsname|,
%    and thus insert the contents of the \textsf{token} register
%    |\theorem@bodyfont| in the replacement text.
%    \begin{macrocode}
            \expandafter \endcsname \the \theorem@bodyfont
%    \end{macrocode}
%    Now it is time to call the macro |\@thm| which takes over
%    the further processing.  It has two arguments: the current
%    counter name (in our example, \texttt{Lem}), and the text of the
%    label (in our example, \texttt{Lemma}).
%    \begin{macrocode}
     \@thm{#1}{#2}}%
%    \end{macrocode}
%    With this, the `sub-definition' is complete.  The macro
%    |\@endtheorem| ends a theorem environment and is, so far, nothing
%    but an |\endtrivlist|.  (Hence it is defined globally, and not
%    within the theorem styles.\footnote{This has to be changed as
%    soon as theorem styles that change \texttt{\bslash @endtheorem}
%    exist.  In such a case, all existing styles must be changed as
%    well since they will have to reset the macro.}) Therefore, we can
%    set it equivalent to the macro that ends the theorem set (in our
%    example, |\endLem|).  However, if some day theorem styles exist
%    that do change |\@endtheorem|, we would have to use the
%    commented-out line instead.
%    \begin{macrocode}
    \global \expandafter \let \csname end#1\endcsname \@endtheorem
%   \global\@namedef{end#1}{\@endtheorem}%
%    \end{macrocode}
%    With these commands all the required definitions are employed,
%    unless the test |\@ifdefinable| has failed.  Therefore, we
%    end the second argument of this macro and with it the definition
%    of |\@xnthm|.
%    \begin{macrocode}
   }}
%    \end{macrocode}
% \end{macro}
%
% \begin{macro}{\@ynthm}
%    The definition of |\@ynthm| is completely analogous. In
%    this case the new counter that is defined is not reset within
%    another counter; thus the definition of |\the...|
%    is simplified:
%    \begin{macrocode}
\gdef\@ynthm#1#2{\expandafter\@ifdefinable\csname #1\endcsname
   {\@definecounter{#1}%
    \expandafter\xdef\csname the#1\endcsname{\@thmcounter{#1}}%
%    \end{macrocode}
%    The rest of the definition corresponds literally to that of
%    |\@xnthm|:
%    \begin{macrocode}
    \def\@tempa{\global\@namedef{#1}}\expandafter \@tempa
     \expandafter{\csname th@\the \theorem@style \expandafter
     \endcsname \the\theorem@bodyfont \@thm{#1}{#2}}%
    \global \expandafter \let \csname end#1\endcsname \@endtheorem}}
%    \end{macrocode}
% \end{macro}
%
% \begin{macro}{\@othm}
%    The definition of |\@othm| does not contain anything new.
%    \begin{macrocode}
\gdef\@othm#1[#2]#3{%
%    \end{macrocode}
% \changes{v2.2b}{95/10/07}{Build in check for counter defined}
%    We do not define a new counter but instead use one that has
%    already been defined.  Thus the only definition we need is that
%    of this pseudo-counter (i.e.\ |\the|\meta{env.\ name}).
%    First we check if |#2| corresponds to a known counter name.
%    \begin{macrocode}
  \expandafter\ifx\csname c@#2\endcsname\relax
   \@nocounterr{#2}%
  \else
   \expandafter\@ifdefinable\csname #1\endcsname
   {\expandafter \xdef \csname the#1\endcsname
     {\expandafter \noexpand \csname the#2\endcsname}%
%    \end{macrocode}
%    All other parts of the definition can be adopted from
%    |\@xnthm|. We have to remember, though, that in this case
%    the name of the current counter and the theorem label have moved
%    to the second and third arguments.
%    \begin{macrocode}
    \def\@tempa{\global\@namedef{#1}}\expandafter \@tempa
     \expandafter{\csname th@\the \theorem@style \expandafter
     \endcsname \the\theorem@bodyfont \@thm{#2}{#3}}%
    \global \expandafter \let \csname end#1\endcsname \@endtheorem}%
  \fi}
%    \end{macrocode}
% \end{macro}
%
% \subsection{Macros that are employed in a {\envfont theorem}
%    environment}
%
% \begin{macro}{\@thm}
%    The macro |\@thm| has to increase the current counter.  Then,
%    depending on whether the environment has (or does not have) an
%    optional argument, it has to branch into either |\@begintheorem|
%    or |\@opargtheorem|.
%    \begin{macrocode}
\gdef\@thm#1#2{\refstepcounter{#1}%
%    \end{macrocode}
%    Now we start a {\envfont trivlist} environment, and give
%    |\@topsep| and |\@topsepadd| the values of the \textsf{skip}
%    registers |\theorempreskipamount| and |\theorempostskipamount|.
%    The value in |\@topsep| is the vertical space that is inserted by
%    the first (and only) |\item| in our |\trivlist| whilst
%    |\@topsepadd| is inserted by |\@endparenv| at the end of that
%    {\envfont trivlist} environment.  By using these registers, we
%    obtain the desired space around a {\envfont theorem} environment.
%    \begin{macrocode}
   \trivlist
   \@topsep \theorempreskipamount               % used by first \item
   \@topsepadd \theorempostskipamount           % used by \@endparenv
%    \end{macrocode}
%    Now we have to test whether an optional argument has been given.
% \changes{v2.0d}{89/07/18}{Forgotten percent added.}
%    \begin{macrocode}
   \@ifnextchar [%
%    \end{macrocode}
%    If there is an optional argument, we will call |\@ythm|, and
%    move the arguments read back into the input stream.
%    \begin{macrocode}
   {\@ythm{#1}{#2}}%
%    \end{macrocode}
%    If not, we call |\@begintheorem|. Its first argument is the
%    name of the theorem set (hence the second argument of
%    |\@thm|). Its second argument is the macro that produces the
%    current number.
%    \begin{macrocode}
   {\@begintheorem{#2}{\csname the#1\endcsname}\ignorespaces}}
%    \end{macrocode}
% \end{macro}
%
% \begin{macro}{\@xthm}
% \begin{macro}{\@ythm}
%    Both these macros were originally called by |\@thm|.  We do
%    not need |\@xthm| anymore, hence we reset it to
%    |\relax|. The definition of |\@ythm| has not changed at
%    all from its definition in \LaTeX. In order to make the macros
%    easier to understand, we will
%    nevertheless present it (commented out).
%    \begin{macrocode}
\global\let\@xthm\relax
% \def\@ythm#1#2[#3]{\@opargbegintheorem{#2}{\csname
%           the#1\endcsname}{#3}\ignorespaces}
%</package>
%    \end{macrocode}
%    The primitive |\ignorespaces| in |\@ythm| and
%    |\@thm| is needed to remove the spaces between the
%    |\begin{...}| and the actual text.
% \end{macro}
% \end{macro}
%
%
% \subsection{Definition of the theorem substyles}
%
% As already pointed out, the theorem substyles, defined below, are
% only loaded when necessary.  Note that all these substyles, except
% \texttt{plain}, have |\slshape| as the default body font.
%
% \subsubsection{The \texttt{plain} style}
%
% \label{dok:th@plain} As the following macros use |@|, we have
% to locally set the |\catcode| of this symbol to ``letter''.
% This happens within a group, so that we do not have to worry about
% which |\catcode| that symbol had before.
%    \begin{macrocode}
%<*thp>
\begingroup \makeatletter
%    \end{macrocode}
% Since we are now within a group, we must make all definitions
% globally.  First we make sure that \texttt{theorem.sty} is loaded.
% This will allow us to use this file as a document style option
% without having to call \texttt{theorem} itself as an option.  At the
% same time, we assure that at least version 2 is loaded, since
% |\theorem@style| was not defined in earlier versions.
%    \begin{macrocode}
\@ifundefined{theorem@style}{\input{theorem.sty}}{}
\ProvidesFile{thp.sty}
             [\FMithmInfo]
%    \end{macrocode}
%
%
% \begin{macro}{\th@plain}
% \begin{macro}{\@begintheorem}
% \begin{macro}{\@opargbegintheorem}
%    |\theoremstyle{plain}| corresponds to the original
%    definition, except that the distances to the surrounding text are
%    determined by the parameters |\theorempreskipamount| and
%    |\theorempostskipamount|.
%    First we set the default body font.
%    \begin{macrocode}
\gdef\th@plain{\normalfont\itshape
%    \end{macrocode}
%    Then we define |\@begintheorem| and
%    |\@opargbegintheorem|. These two macros define how the
%    header of a theorem is typeset.  |\@opargbegintheorem| will
%    be called if a \textsf{theorem} environment with an optional
%    argument is encountered; otherwise, the header is constructed by
%    calling |\@begintheorem|.  If one of these macros is
%    executed, we are within a \textsf{trivlist} environment started by
%    |\@thm|.  So the theorem header is produced with an
%    |\item| command.
%
%    Instead of specifying the header font directly, all standard
%    theorem styles use the |\theorem@headerfont| macro to allow
%    customization.  The extra space (|\labelsep|) is necessary
%    because of problems in the \textsf{trivlist} environment.
%    \begin{macrocode}
  \def\@begintheorem##1##2{%
        \item[\hskip\labelsep \theorem@headerfont ##1\ ##2]}%
%    \end{macrocode}
%    The definition of |\@opargbegintheorem| is completely
%    analogous. The only difference is the fact that there exists a
%    third argument (which is the optional parameter of the
%    environment and contains additional information about the
%    theorem).  Customarily we enclose it in parentheses.
%    \begin{macrocode}
\def\@opargbegintheorem##1##2##3{%
   \item[\hskip\labelsep \theorem@headerfont ##1\ ##2\ (##3)]}}
%    \end{macrocode}
% \end{macro}
% \end{macro}
% \end{macro}
%
% \noindent
% We conclude with an |\endgroup| to restore the
% |\catcode| of |@|.
%    \begin{macrocode}
\endgroup
%</thp>
%    \end{macrocode}
%
%
% \subsubsection{The \texttt{break} style}
%
% This style option is stored in the file |thb.sty|.
% For the next two lines see the documentation for |\th@plain| on
% page~\pageref{dok:th@plain}.
%    \begin{macrocode}
%<*thb>
\begingroup \makeatletter
\@ifundefined{theorem@style}{\input{theorem.sty}}{}
\ProvidesFile{thb.sty}
             [\FMithmInfo]
%    \end{macrocode}
%
% \begin{macro}{\th@break}
%    |\theoremstyle{break}| produces a line break after the name
%    of the theorem.  The font is |\slshape|.  Hence, we define
%    |\th@break| as follows:
%    \begin{macrocode}
\gdef\th@break{\normalfont\slshape
  \def\@begintheorem##1##2{\item[%
%    \end{macrocode}
%    We run into the following problem: it is not possible to create
%    the header with |\item[|\meta{title}|]| and then start
%    a new line by, for example, |\mbox{}\\|. Such a definition
%    will fail whenever a list environment follows immediately.  With
%    the above construction, the |\mbox{}| causes the switch
%    |@inlabel| (cf.\ definition of |\list| and
%    |\trivlist| in \cite{latex:source}) to be set to \texttt{false}
%    and so the following list will insert additional vertical
%    space (|\topskip|). This is quite annoying. Therefore, we
%    create the line break within the |\item|. In order to ensure
%    that the text will begin at the proper position in the following
%    line, we simply pretend that the label does not take any
%    room.\footnote{This will lead to problems whenever very high
%    symbols occurring in the line tower into the
%    heading. So, something else has to be done here sometime.}
%    \begin{macrocode}
       \rlap{\vbox{\hbox{\hskip \labelsep\theorem@headerfont ##1\ ##2}%
                   \hbox{\strut}}}]}%
%    \end{macrocode}
%    Again, the definition of |\@opargbegintheorem| is
%    completely analogous.
%    \begin{macrocode}
\def\@opargbegintheorem##1##2##3{%
  \item[\rlap{\vbox{\hbox{\hskip \labelsep \theorem@headerfont
                     ##1\ ##2\ (##3)}%
                    \hbox{\strut}}}]}}
\endgroup
%</thb>
%    \end{macrocode}
% \end{macro}
%
%
% \subsubsection{The \texttt{changebreak} style}
%
%    \begin{macrocode}
% This style option is stored in the file |thcb.sty|.
%    \begin{macrocode}
%<*thcb>
\begingroup \makeatletter
\@ifundefined{theorem@style}{\input{theorem.sty}}{}
\ProvidesFile{thcb.sty}
             [\FMithmInfo]
%    \end{macrocode}
%
% \begin{macro}{\th@changebreak}
%    The \texttt{change-break} theorem style is like \texttt{break}
%    but with interchange of theorem name and theorem number.  Thus we
%    define |\th@changebreak| as follows:
%    \begin{macrocode}
\gdef\th@changebreak{\normalfont\slshape
  \def\@begintheorem##1##2{\item
       [\rlap{\vbox{\hbox{\hskip\labelsep \theorem@headerfont ##2\ ##1}%
                    \hbox{\strut}}}]}%
\def\@opargbegintheorem##1##2##3{%
  \item[\rlap{\vbox{\hbox{\hskip\labelsep \theorem@headerfont
                     ##2\ ##1\ (##3)}%
                    \hbox{\strut}}}]}}
\endgroup
%</thcb>
%    \end{macrocode}
% \end{macro}
%
%
% \subsubsection{The \texttt{change} style}
%
% This style option is stored in the file |thc.sty|.
%    \begin{macrocode}
%<*thc>
\begingroup \makeatletter
\@ifundefined{theorem@style}{\input{theorem.sty}}{}
\ProvidesFile{thc.sty}
             [\FMithmInfo]
%    \end{macrocode}
%
% \begin{macro}{\th@change}
%    The \texttt{change} theorem style corresponds to the
%    \texttt{change break} style without a linebreak after the header.
%    To say it in another way, it's the same as the \texttt{plain}
%    style but with number and name interchanged and |\slshape| as the
%    default font.
%    \begin{macrocode}
\gdef\th@change{\normalfont\slshape
  \def\@begintheorem##1##2{\item
       [\hskip\labelsep \theorem@headerfont ##2\ ##1]}%
\def\@opargbegintheorem##1##2##3{%
  \item[\hskip\labelsep \theorem@headerfont ##2\ ##1\ (##3)]}}
\endgroup
%</thc>
%    \end{macrocode}
% \end{macro}
%
%
% \subsubsection{The \texttt{marginbreak} style}
%
% This style option is the one used most often at Mainz.  It is saved
% in the file |thmb.sty|.
%    \begin{macrocode}
%<*thmb>
\begingroup \makeatletter
\@ifundefined{theorem@style}{\input{theorem.sty}}{}
\ProvidesFile{thmb.sty}
             [\FMithmInfo]
%    \end{macrocode}
%
% \begin{macro}{\th@marginbreak}
%    The \texttt{margin break} style is nearly the same as the
%    \texttt{change break} style. The only difference is the placement
%    of the theorem number.  We use |\llap| to place it in the left
%    margin.
%
%    In this style |\labelsep| denotes the separation between the
%    number and the text.
%    \begin{macrocode}
\gdef\th@marginbreak{\normalfont\slshape
  \def\@begintheorem##1##2{\item
       [\rlap{\vbox{\theorem@headerfont
                    \hbox{\llap{##2}\hskip\labelsep ##1}%
                    \hbox{\strut}}}]}%
\def\@opargbegintheorem##1##2##3{%
  \item[\rlap{\vbox{\theorem@headerfont
                    \hbox{\llap{##2}\hskip\labelsep ##1\ (##3)}%
                    \hbox{\strut}}}]}}
\endgroup
%</thmb>
%    \end{macrocode}
% \changes{v2.1c}{90/02/17}{`endgroup outside macrocode env.}
% \end{macro}
%
%
%
% \subsubsection{The \texttt{margin} style}
%
% This style option is stored in the file |thm.sty|.
%    \begin{macrocode}
%<*thm>
\begingroup \makeatletter
\@ifundefined{theorem@style}{\input{theorem.sty}}{}
\ProvidesFile{thm.sty}
             [\FMithmInfo]
%    \end{macrocode}
%
% \begin{macro}{\th@margin}
%    Again this is only a variant of the theorem styles described
%    above without any new ideas.
%    \begin{macrocode}
\gdef\th@margin{\normalfont\slshape
  \def\@begintheorem##1##2{\item
       [\theorem@headerfont \llap{##2}\hskip\labelsep ##1]}%
\def\@opargbegintheorem##1##2##3{%
  \item[\theorem@headerfont \llap{##2}\hskip\labelsep ##1\ (##3)]}}
\endgroup
%</thm>
%    \end{macrocode}
% \end{macro}
%
%
%
%
% \subsection{Final Definitions}
%
% \begin{macro}{\theorempreskipamount}
% \begin{macro}{\theorempostskipamount}
%    The \textsf{skip} parameters that regulate the vertical empty space
%    before and after the {\envfont theorem} environment have
%    to be allocated as well.
% \changes{v2.0f}{89/07/20}{Unnecessary `global' removed.}
%    \begin{macrocode}
%<*package>
\newskip\theorempreskipamount
\newskip\theorempostskipamount
%    \end{macrocode}
%    Since we have used the same values for all theorem sets, we now can
%    assign them.
%    \begin{macrocode}
\global\setlength\theorempreskipamount{12pt plus 5pt minus 3pt}
\global\setlength\theorempostskipamount{8pt plus 3pt minus 1.5pt}
%    \end{macrocode}
% \end{macro}
% \end{macro}
%
%
% \begin{macro}{\@endtheorem}
%    The same holds for the macro |\@endtheorem|, which ends a
%    {\envfont theorem} environment. Since it is the same for all
%    theorem sets, it is removed from the macros |\th@|\meta{style}.
%    It simply ends the {\envfont trivlist} environment, which was
%    begun in |\@thm|.
%    \begin{macrocode}
\global\let\@endtheorem=\endtrivlist
%    \end{macrocode}
% \end{macro}
%
%
%
% \begin{macro}{\@preamblecmds}
%    All macros defined above are to be used only in the preamble.
%    Therefore, we insert them in |\@preamblecmds| which will disable
%    them at begin document. This is done by the internal \LaTeXe{}
%    command |\@onlypreamble|.
% \changes{v2.0g}{89/09/19}{`theoremheaderfont and `theorembodyfont
%    added.}
% \changes{v2.2a}{94/02/03}{Use \LaTeXe{} interface}
%    \begin{macrocode}
\@onlypreamble\@xnthm
\@onlypreamble\@ynthm
\@onlypreamble\@othm
\@onlypreamble\newtheorem
\@onlypreamble\theoremstyle
\@onlypreamble\theorembodyfont
\@onlypreamble\theoremheaderfont
%    \end{macrocode}
% \end{macro}
%
% Finally we declare the \texttt{plain} theorem style to be the default.
% \changes{v2.1b}{90/06/04}{Fixed missing `item by declaring default
%    style.}
%    \begin{macrocode}
\theoremstyle{plain}
%</package>
%    \end{macrocode}
%
% \Finale

\endinput
}{}
\ProvidesFile{thmb.sty}
             [\FMithmInfo]
%    \end{macrocode}
%
% \begin{macro}{\th@marginbreak}
%    The \texttt{margin break} style is nearly the same as the
%    \texttt{change break} style. The only difference is the placement
%    of the theorem number.  We use |\llap| to place it in the left
%    margin.
%
%    In this style |\labelsep| denotes the separation between the
%    number and the text.
%    \begin{macrocode}
\gdef\th@marginbreak{\normalfont\slshape
  \def\@begintheorem##1##2{\item
       [\rlap{\vbox{\theorem@headerfont
                    \hbox{\llap{##2}\hskip\labelsep ##1}%
                    \hbox{\strut}}}]}%
\def\@opargbegintheorem##1##2##3{%
  \item[\rlap{\vbox{\theorem@headerfont
                    \hbox{\llap{##2}\hskip\labelsep ##1\ (##3)}%
                    \hbox{\strut}}}]}}
\endgroup
%</thmb>
%    \end{macrocode}
% \changes{v2.1c}{90/02/17}{`endgroup outside macrocode env.}
% \end{macro}
%
%
%
% \subsubsection{The \texttt{margin} style}
%
% This style option is stored in the file |thm.sty|.
%    \begin{macrocode}
%<*thm>
\begingroup \makeatletter
\@ifundefined{theorem@style}{% \iffalse meta-comment
%
% Copyright (C) 1993-2024
%
% The LaTeX Project and any individual authors listed elsewhere
% in this file.
%
% This file is part of the Standard LaTeX `Tools Bundle'.
% -------------------------------------------------------
%
% It may be distributed and/or modified under the
% conditions of the LaTeX Project Public License, either version 1.3c
% of this license or (at your option) any later version.
% The latest version of this license is in
%    https://www.latex-project.org/lppl.txt
% and version 1.3c or later is part of all distributions of LaTeX
% version 2005/12/01 or later.
%
% The list of all files belonging to the LaTeX `Tools Bundle' is
% given in the file `manifest.txt'.
%
% \fi
%
%%
%
%
\def\FMithmInfo{2023/07/05 v2.2c Theorem extension package (FMi)}
%
% \ProvidesFile{theorem.dtx}[\FMithmInfo]
%\iffalse   % this is a METACOMMENT !
%%
%% Package `theorem' to use with LaTeX2e
%% Copyright (C) 1989-1995 Frank Mittelbach, all rights reserved.
%<+package>\@ifundefined{theorem@style}{}{\endinput}
%<+package>\NeedsTeXFormat{LaTeX2e}[1995/06/01]
%<+package>\ProvidesPackage{theorem}
%<+package>         [\FMithmInfo]
%
% \fi
%
% \changes{v1.0}{87/05/07}{First draft.}
% \changes{v1.0a}{87/05/07}{`thistheoremstyle' added.}
% \changes{v1.1a}{87/07/29}{New concept: redefine the internal
%                          \protect\LaTeX{} commands `@xnthm' etc.}
% \changes{v1.1b}{87/07/31}{`theoremfont' used to specify font.}
% \changes{v1.2a}{88/01/08}{`@endtheorem' now globally defined.}
% \changes{v1.2b}{88/01/11}{`newtheorem' only allowed in the preamble.}
% \changes{v1.3a}{88/02/14}{`theoremfont' now a toks register.}
%
% \changes{v2.0a}{88/11/16}{New implementation and new documentation.}
% \changes{v2.0a}{88/11/16}{No layout changes but internal modifications
%                           old substyles won't work.}
% \changes{v2.0b}{89/01/30}{`global' used for all definitions.}
% \changes{v2.0c}{89/06/15}{Documentation translated to English.
%                            Thanks to LLL and Nina}
% \changes{v2.1a}{90/02/15}{Support for new font selection.}
% \changes{v2.2a}{94/02/03}{Cleanup for LaTeX2e}
%
% \DoNotIndex{\@endparenv,\@ifdefinable,\@ifnextchar,\@ifundefined}
% \DoNotIndex{\@namedef,\@nameuse,\@spaces,\@tempa,\@warning}
% \DoNotIndex{\ }
% \DoNotIndex{\begingroup,\bfseries}
% \DoNotIndex{\csname}
% \DoNotIndex{\def,\do,\docdate}
% \DoNotIndex{\endcsname,\endgroup,\endinput,\endtrivlist,\expandafter}
% \DoNotIndex{\filedate,\fileversion}
% \DoNotIndex{\gdef,\global}
% \DoNotIndex{\hbox,\hskip}
% \DoNotIndex{\ignorespaces,\input,\itshape,\item}
% \DoNotIndex{\labelsep,\let,\llap}
% \DoNotIndex{\makeatletter}
% \DoNotIndex{\noexpand}
% \DoNotIndex{\refstepcounter,\relax,\rlap}
% \DoNotIndex{\setlength,\slshape,\space,\string,\strut}
% \DoNotIndex{\the,\trivlist,\typeout}
% \DoNotIndex{\vbox}
% \DoNotIndex{\xdef}
%
% \def\deflabel#1{\ttfamily #1\hfill}
% \def\deflist#1{\begin{list}{}{\settowidth\labelwidth{\ttfamily #1}%
%                               \setlength\leftmargin\labelwidth
%                               \addtolength\leftmargin\labelsep
%                               \let\makelabel\deflabel}}
% \def\enddeflist{\end{list}}
% \def\envfont{\normalfont\ttfamily}
%
% \MakeShortVerb{\|}
%
% \newcommand{\TUB}{{\sl TUGboat\/}}
%
% \GetFileInfo{theorem.dtx}
% \title{An Extension of the
%        \LaTeX{} theorem environment\thanks{This file
%        has version number \fileversion, last
%        revised \filedate.}}
% \author{Frank Mittelbach}
%
% \MaintainedByLaTeXTeam{tools}
% \maketitle
%
% \begin{abstract}
%    The macros described in this paper yield an extension of the
%    \LaTeX{} theorem mechanism. It is designed to satisfy the
%    different requirements of various journals. Thus, the
%    layout of the ``theorems'' can be manipulated by determining a
%    ``style''. This article describes not only the use, but
%    also the definition, of the necessary macros.
% \end{abstract}
%
% \section*{Preface to version 2.2}
%
% For \LaTeXe{} this package did not need any fundamental changes. I
% only modified the messages generated so that theorem layout styles
% will show up with the |\listfiles| command and cleaned the section
% on the New Font Selection Scheme since this is now included in
% \LaTeX.
%
% \section*{Preface to version 2.1}
%
% This version is identical to 2.0g described in \TUB\ 10\#3 except
% for some internal defaults which are now set depending on the used
% font selection scheme.
%
% This was done to avoid unpleasant surprises if the new font selection
% scheme is in force. For further details see
% section~\ref{sec:newfonts} and \cite{bk:GMS94}.
%
% \section{Introduction}
%
% For our purposes here, ``theorems'' are labelled enunciations,
% often set off from the main text by extra space and a font change.
% Theorems, corollaries, conjectures, definitions, and remarks
% are all instances of ``theorems''.  The ``header'' of these structures
% is composed of a label (such as \textsc{Theorem} or \textsc{Remark})
% and a number which serializes an item in the sequence of items
% with the same label.
%
% Shortly after the introduction of \LaTeX{} at the Fachbereich
% Mathematik in Mainz, the desire to manipulate the layout of
% ``theorems'' arose.  In Mainz, the following two conventions came
% into general use:
% \begin{enumerate}
%   \item The number of the theorem is shown in the margin.
%   \item There is a line break at
%         the end of the theorem header.
% \end{enumerate}
%
% Additionally, some journals require different formats which depend on
% the ``sort of theorem'': e.g.\ often remarks and definitions are set
% in |\upshape|, while |\itshape| is employed for main theorems.
%
% Confronted with these requirements, a theorem environment was
% developed in Mainz which allows separate determination of the
% layout of the ``theorems sets'', comparable to |\pagestyle|.
%
%
% \section{The user interface}
% \subsection{Defining new theorem sets}
%
% \DescribeMacro\newtheorem
% As in the original \LaTeX{} version, the command |\newtheorem|
% defines a new ``theorem set'' or ``theorem-like structure''.
% Two required arguments name the new environment and give the text
% to be typeset with each instance of the new ``set'', while
% an optional argument determines how
% the ``set'' is enumerated:
% \begin{description}
%    \item[\ttfamily \bslash newtheorem\{foo\}\{bar\}]
%       The theorem set {\envfont foo} (whose name is \texttt{bar})
%       uses its own counter.
%    \item[\ttfamily \bslash newtheorem\{foo2\}{[foo]}\{bar2\}]
%       The theorem set {\envfont foo2} (printed name \texttt{bar2})
%       uses the same counter as the theorem set \texttt{foo}.
%    \item[\ttfamily \bslash newtheorem\{foo3\}\{bar3\}{[section]}]
%       The theorem set {\envfont foo3} (printed name \texttt{bar3}) is
%       enumerated within the counter \texttt{section}, i.e.\ with every
%       new |\section| the enumeration begins again with 1, and
%       the enumeration is composed from the section-number and the
%       theorem counter itself.
% \end{description}
%
% \DescribeMacro\theoremstyle
% Additionally, the command |\theoremstyle| can define the layout
% of various, or all, theorem sets.  It should be noted that any
% theorem set defined by |\newtheorem| is typeset in the
% |\theoremstyle| that is current at the time of the definition.
% Thus, the following
% \begin{verbatim}
%    \theoremstyle{break}        \newtheorem{Cor}{Corollary}
%    \theoremstyle{plain}        \newtheorem{Exa}{Example}[section]
%\end{verbatim}                             ^^A No blank at line start
% leads to the result that the set {\envfont Cor} is formatted in the
% style \texttt{break}, while the set {\envfont Exa} and all the
% following ones are formatted in the style \texttt{plain}, unless
% another |\theoremstyle| follows.  Since the definitions installed by
% |\newtheorem| are global, one also can limit |\theoremstyle| locally
% by grouping braces.
%
% \DescribeMacro\theorembodyfont
% The choice of the font for the theorem body is completely
% independent of the chosen |\theoremstyle|; this has proven to
% be very advantageous.  For example,
% \begin{verbatim}
%   {\theorembodyfont{\upshape}          \newtheorem{Rem}{Remark}}
%\end{verbatim}
% defines a theorem set {\envfont Rem}, which will be set in
% |\upshape| in the current layout (which in our example is
% \texttt{plain}).  As with |\theoremstyle|, the |\theorembodyfont|
% chosen is that current at the time of |\newtheorem|.  If
% |\theorembodyfont| is not specified or one defines
% |\theorembodyfont{}|, then the font used will be that defined by the
% |\theoremstyle|.
%
% \DescribeMacro\theoremheaderfont It is also possible to customize
% the font used for the theorem headers.  This is, however, a global
% declaration, and therefore there should be at most one
% |\theoremheaderfont| declaration in the preamble.\footnote{If it is
% actually necessary to have different header fonts, one has to define
% new theorem styles (substituting the desired font) or specify the
% information directly in the \texttt{\bslash newtheorem} declaration
% (the unclean variant).}
%
% \DescribeMacro\theorempreskipamount
% \DescribeMacro\theorempostskipamount Two additional parameters
% affect the vertical space around the theorem environments:
% |\theorempreskipamount| and |\theorempostskipamount| define,
% respectively, the spacing before and after such an environment.
% These parameters apply for all theorem sets and can be manipulated
% with the ordinary length macros.  They are rubber lengths
% (`\textsf{skips}'), and therefore can contain \texttt{plus} and
% \texttt{minus} parts.
%
% Since the definition of theorem sets should---most sensibly---be
% placed in the preamble, we only allow installation there.
% It is therefore possible to release the memory used here after
% |\begin{document}|, in order to make room for other
% applications.
%
%
% \subsection{Existing theorem styles}
%
% The following theorem styles exist to date:
% \begin{deflist}{marginbreak}
%    \item[plain]
%       This theorem style emulates the original \LaTeX{} definition,
%       except that additionally the parameters
%       |\theorem...skipamount| are used.
%    \item[break]
%       In this style, the theorem header is followed by a line
%       break.
%    \item[marginbreak]
%       The theorem number is set in the margin, and there is a line
%       break as in \texttt{break}.
%    \item[changebreak]
%       Like \texttt{break}, but with header number and text
%       interchanged.
%    \item[change]
%       Header number and text are interchanged, without a line break.
%    \item[margin]
%       The number is set in the left margin, without a line break.
% \end{deflist}
% All styles (except \texttt{plain}) select |\slshape| as the default
% |\theorembodyfont|.
%
%
% \subsection{Examples}
%
% Given the above theorem sets {\envfont Cor}, {\envfont Exa}
% and {\envfont Rem},
% suppose that the preamble also contains the declarations:
% \begin{verbatim}
%    \theoremstyle{marginbreak}   \newtheorem{Lem}[Cor]{Lemma}
%    \theoremstyle{change}
%    \theorembodyfont{\itshape}        \newtheorem{Def}[Cor]{Definition}
%
%    \theoremheaderfont{\scshape}
%\end{verbatim}
% Then the following are some typical examples of the typeset output
% resulting from their use.
%
% \begin{Cor}
%    This is a sentence typeset in the theorem environment {\envfont
%    Cor}.
% \end{Cor}
%
% \begin{Exa}
%    This is a sentence typeset in the theorem environment {\envfont
%    Exa}.
% \end{Exa}
%
% \begin{Rem}
%    This is a sentence typeset in the theorem environment {\envfont
%    Rem}.
% \end{Rem}
%
% \begin{Lem}[Ben User]
%    This is a sentence typeset in the theorem environment {\envfont
%    Lem}.
% \end{Lem}
%
% \begin{Def}[Very impressive Definition]
%    This is a sentence typeset in the theorem environment {\envfont
%    Def}.
% \end{Def}
%
% The last two examples show the effect of the optional argument to
% a theorem environment (it is the text typeset in parentheses).
%
%
% \section{Special Considerations}
% \label{sec:newfonts}
%
% Theoremheader and body are implemented as a unit. This means that
% the |\theoremheaderfont|  will inherit characteristics of the
% |\theorembodyfont| in \LaTeXe. Thus, if for example
% |\theorembodyfont| is |\itshape| and |\theoremheaderfont|
% is |\bfseries| the font selected for the header will have the
% characteristics `bold extended italic'. If this is not desired one
% should set the |\theoremheaderfont| to something like
% \begin{verbatim}
%      \theoremheaderfont{\normalfont\bfseries}
%\end{verbatim}
% i.e.\ supplying all necessary font information explicitly.
%
%
%
% \section{Acknowledgements}
%
% The publication of this set of macros was only possible with the
% help of Christina Busse (translating the manuscript into English),
% Joachim Pense (playing the r\^ole of typist), Chris Rowley
% (looking everything over) and many others providing useful
% suggestions.
%
% \MaybeStop{%
%    \begin{thebibliography}{9}
%       \bibitem{bk:GMS94} \textsc{M.~Goossens}, \textsc{F.~Mittelbach}
%         and \textsc{A.~Samarin}.
%         \newblock The \LaTeX{} Companion.
%         \newblock
%         Addison-Wesley, Reading, Massachusetts, 1994.
%       \bibitem{latex:source} \textsc{Lamport, Leslie.} \newblock
%                               \texttt{latex.tex}, version 2.09, date
%                               Feb. 1990.
%    \end{thebibliography}
%    \PrintIndex
%    \PrintChanges}
%
%
% \section{The documentation driver file}
%
% The next bit of code contains the documentation driver file for
% \TeX{}, i.e., the file that will produce the documentation you are
% currently reading. It will be extracted from this file by the
% \texttt{docstrip} program. Since it is the first code in the file
% one can alternatively process this file directly with \LaTeXe{} to
% obtain the documentation.
%
%    \begin{macrocode}
%<*driver>
\documentclass{ltxdoc}

\usepackage{theorem}

%<+driver>% The next few lines define theorem sets which are used
%<+driver>% in the example section of the documentation.

\theoremstyle{break}       \newtheorem{Cor}{Corollary}
\theoremstyle{plain}       \newtheorem{Exa}{Example}[section]
{\theorembodyfont{\upshape}\newtheorem{Rem}{Remark}}
\theoremstyle{marginbreak} \newtheorem{Lem}[Cor]{Lemma}
\theoremstyle{change}
\theorembodyfont{\itshape} \newtheorem{Def}[Cor]{Definition}

\theoremheaderfont{\scshape}

\RecordChanges

\begin{document}
   \DocInput{theorem.dtx}
\end{document}
%</driver>
%    \end{macrocode}
%
% \section{Definition of the Macros}
%
% If the file has been loaded before, we abort immediately. If not the
% package announces itself (this is actually done at the very top of
% the file---the way it is done isn't good style so don't copy it).
% \changes{v2.0e}{89/07/19}{Spaces removed from `typeout'.}
% \changes{v2.2b}{95/11/19}{Announce moved to top of file}
%    \begin{macrocode}
%<*package>
%\@ifundefined{theorem@style}{}{\endinput}
%\def\FMithmInfo{1995/11/19 v2.2b Theorem extension package (FMi)}
%\ProvidesPackage{theorem}[\FMithmInfo]
%    \end{macrocode}
%
%
% \subsection{Definition of theorem styles and fonts}
%
% All the definitions in this file are done globally to allow
% inputting this file inside a group.
%
% \begin{macro}{\theoremstyle}
%    Before a theorem style can be installed, the chosen style must be
%    known. For that reason, we must test to see that
%    |\th@|\meta{style} is known or, more precisely, that it is
%    different from |\relax|. If the style is not known then
%    |\th@plain| is used.
%    \begin{macrocode}
\gdef\theoremstyle#1{%
   \@ifundefined{th@#1}{\@warning
          {Unknown theoremstyle `#1'. Using `plain'}%
          \theorem@style{plain}}%
%    \end{macrocode}
%    We save the theorem style to be used in the token register
%    |\theorem@style|.
%    \begin{macrocode}
      {\theorem@style{#1}}%
%    \end{macrocode}
%    Now we ``evaluate'' the theorem style: this means, we call the
%    macro |\th@|\meta{style} which will activate the relevant
%    definitions which are contained in a separate file.  This is done
%    in a group to suppress changes to the current font.  This could
%    otherwise pose problems together with the new font selection
%    scheme\footnote{When I printed the original article using the new
%    font selection scheme I ended with a document with slanted
%    typefaces (text headings and all) simply because one of the
%    theorem styles used \texttt{\bslash sl} at toplevel.} if the
%    |\th@|\meta{style} is evaluated a second time.
% \changes{v2.1a}{90/02/15}{Group added around `th@..}
% \changes{v2.1a}{90/02/15}{Reseting of `..@begintheorem removed.}
%    \begin{macrocode}
      \begingroup
        \csname th@\the\theorem@style \endcsname
      \endgroup}
%    \end{macrocode}
% \end{macro}
%
%
% \begin{macro}{\@begintheorem}
% \changes{v2.1a}{90/02/15}{Relaxed at toplevel.}
% \begin{macro}{\@opargbegintheorem}
% \changes{v2.1a}{90/02/15}{Relaxed at toplevel.}
%    We reset |\@begintheorem| and |\@opargbegintheorem| to
%    |\relax| since these commands are no longer necessary at
%    toplevel. This will save a few tokens.
%    \begin{macrocode}
\global\let\@begintheorem\relax
\global\let\@opargbegintheorem\relax
%    \end{macrocode}
% \end{macro}
% \end{macro}
%
% \begin{macro}{\theorem@style}
%    Obviously the \textsf{token} register used above has to be
%    allocated. To assure the utmost compatibility with the
%    original \LaTeX{} definition, we set the default theorem style to
%    \texttt{plain}, which implements the usual \LaTeX{} convention.
% \changes{v2.0f}{89/07/20}{Unnecessary `global' removed.}
%    \begin{macrocode}
\newtoks\theorem@style
\global\theorem@style{plain}
%    \end{macrocode}
% \end{macro}
%
% \begin{macro}{\theorembodyfont}
% \begin{macro}{\theorem@bodyfont}
%    If the |\theorembodyfont| is set by the user then it should not
%    interact with the default font set in the theorem style. When the
%    new font selection is in force this may happen if, for example,
%    the default is |\itshape| and the new |\theorembodyfont| is
%    |\sffamily|. So we add a |\reset@font| command in front of the
%    user definition.
% \changes{v2.0f}{89/07/20}{Unnecessary `global' removed.}
% \changes{v2.0g}{89/09/19}{`theoremfont to `theorembodyfont.}
% \changes{v2.1c}{92/06/22}{`reset@font added and logic changed.}
% \changes{v2.1d}{92/10/28}{Reset to empty if arg is empty}
%    \begin{macrocode}
\gdef\theorembodyfont#1{%
%    \end{macrocode}
%    We check if the argument supplied is empty and if so put nothing
%    into the |\theorem@bodyfont| token register to allow for
%    |\theorembodyfont{}| as a mean of using the default of the
%    current |\theoremstyle|.
%    \begin{macrocode}
   \def\@tempa{#1}%
   \ifx\@tempa\@empty
    \theorem@bodyfont{}%
   \else
    \theorem@bodyfont{\reset@font#1}%
   \fi
   }
\newtoks\theorem@bodyfont
\global\theorem@bodyfont{}
%    \end{macrocode}
% \end{macro}
% \end{macro}
%
%
% \begin{macro}{\theoremheaderfont}
% \changes{v2.0g}{89/09/19}{Macro added.}
%    The font for the theorem headers is handled differently because
%    this definition applies to all theorem styles.
%    \begin{macrocode}
\gdef\theoremheaderfont#1{\gdef\theorem@headerfont{#1}%
%    \end{macrocode}
%    After using the macro once it is redefined to produce an error
%    message.
%    \begin{macrocode}
       \gdef\theoremheaderfont##1{%
        \typeout{\string\theoremheaderfont\space should be used
                 only once.}}}
%    \end{macrocode}
% \end{macro}
%
% \begin{macro}{\theorem@headerfont}
% \changes{v2.0g}{89/09/19}{Macro added.}
%    To set the |\theorem@headerfont| default we first test if
%    the new fontselection scheme is in force.
% \changes{v2.1a}{90/02/15}{Support for new font selection.}
%    \begin{macrocode}
\ifx\upshape\undefined
%    \end{macrocode}
%    If not we define it to expand into |\bfseries|. We don't use
%    |\let| just in case a following style option redefines this
%    macro.
%    \begin{macrocode}
\gdef\theorem@headerfont{\bfseries}
%    \end{macrocode}
%    Otherwise we reset the current shape before calling |\bfseries|.
%    \begin{macrocode}
\else \gdef\theorem@headerfont{\normalfont\bfseries}\fi
%    \end{macrocode}
% \end{macro}
%
% \begin{macro}{\th@plain}
% \begin{macro}{\th@break}
% \begin{macro}{\th@marginbreak}
% \begin{macro}{\th@changebreak}
% \begin{macro}{\th@change}
% \begin{macro}{\th@margin}
%    The different styles are defined in macros such as
%    |\th@plain|.  Since memory space is precious in
%    ``non-Big-versions'', we have to avoid offering too many unused
%    definitions.  Therefore we define these styles in separate files
%    that can be loaded on demand.  Thus the commands themselves only
%    load these files. We use |\@input@|, a \LaTeXe{} internal command
%    that ensures that the file will be listed with |\listfiles|.
% \changes{v2.2a}{94/02/02}{Use `@input@ to load theorem layout files}
%    \begin{macrocode}
\gdef\th@plain{\@input@{thp.sty}}
\gdef\th@break{\@input@{thb.sty}}
\gdef\th@marginbreak{\@input@{thmb.sty}}
\gdef\th@changebreak{\@input@{thcb.sty}}
\gdef\th@change{\@input@{thc.sty}}
\gdef\th@margin{\@input@{thm.sty}}
%    \end{macrocode}
%    This list will be expanded when new styles become
%    available.  For testing,  just append new theorem substyles
%    as document options.
% \end{macro}
% \end{macro}
% \end{macro}
% \end{macro}
% \end{macro}
% \end{macro}
%
%
%
% \subsection{Definition of a new theorem set}
%
% As already pointed out, a new theorem environment can be defined in
% three different ways:
% \begin{verbatim}
%    \newtheorem{Lem}{Lemma}
%    \newtheorem{Lem}{Lemma}[section]
%    \newtheorem{Lem}[Theorem]{Lemma}
%\end{verbatim}
% The function of the macro |\newtheorem| is to recognize these cases
% and then to branch into one of the three macros |\@ynthm|, |\@xnthm|
% or |\@othm|.  This mechanism is adopted unchanged from
% \cite{latex:source}; the essential point here is that, for example,
% in the second case, the arguments \texttt{Lem}, \texttt{Lemma} and
% \texttt{section} are passed over to the macro |\@xnthm|.
%
% We inspect this case first because the others present fewer problems,
% and thus are easily derived from this one.
%
% \begin{macro}{\@xnthm}
%    For our example arguments, the macro |\@xnthm| must fulfill
%    the following:
%    \begin{itemize}
%       \item
%          Define a new \LaTeX-counter `Lem'
%       \item
%          reset this counter within a |\section|
%       \item
%          define the macro |\theLem|
%       \item
%          define the environment macros |\Lem| and
%          |\endLem| using the current
%          |\theoremstyle| and |\theorem@bodyfont|.
%    \end{itemize}
%    Obviously, all this should happen only if the first argument
%    of |\@xnthm| (i.e.\ \texttt{Lem} in our example) is chosen so
%    as not to conflict with any previously defined commands or
%    environments.  This test is performed by the \LaTeX{} macro
%    |\@ifdefinable|.
%    \begin{macrocode}
\gdef\@xnthm#1#2[#3]{\expandafter\@ifdefinable\csname #1\endcsname
%    \end{macrocode}
%    Therefore, the first argument of |\@ifdefinable| is the
%    expansion (in the example, |\Lem|) of
%    |\csname||#1\endcsname|.  The second argument is
%    executed only if the test has been completed successfully.
%    \begin{macrocode}
   {%
%    \end{macrocode}
%    Now we define the new counter. The names of the \LaTeX{}
%    macros employed should speak for themselves:
% \changes{v2.2b}{1995/10/07}{Use \cs{@newctr}}
% \changes{v2.2c}{1995/11/23}{Use \cs{@newctr} properly.}
%    \begin{macrocode}
    \@definecounter{#1}\@newctr{#1}[#3]%
%    \end{macrocode}
%    Using |\@newctr| will give a proper error message if the counter
%    in |#3| is not defined.
%    In defining `|\theLem|' we must generate the desired macro
%    name by use of |\expandafter| and |\csname|.
%    \begin{macrocode}
    \expandafter\xdef\csname the#1\endcsname
%    \end{macrocode}
%    An |\xdef| is used in order to make the definition global,
%    and to ensure that it contains the replacement texts of
%    |\@thmcountersep| and |\@thmcounter|.\footnote{These
%    two macros can be defined by the document style.  Their default
%    values produce a `\texttt{.}' as separation and an arabic
%    representation of the number.} However, not everything should be
%    expanded. For example, it saves space to use |\thesection|
%    instead of its---at times---lengthy expansion.
%    \begin{macrocode}
      {\expandafter \noexpand \csname the#3\endcsname
       \@thmcountersep \@thmcounter{#1}}%
%    \end{macrocode}
%    Thus with the defaults of \LaTeX{}, |\theLem| would be
%    replaced by the command sequence |\thesection.\arabic{Lem}|.
%
%    We will now look at the definition of the macro which is executed
%    at the beginning of the actual environment (in our example this
%    macro is |\Lem|). It should be noted that we use an
%    ``|\expandafter| trick'' to expand only certain parts of the
%    replacement text at the time of the definition.
%    \begin{macrocode}
    \def\@tempa{\global\@namedef{#1}}%
    \expandafter \@tempa \expandafter{%
%    \end{macrocode}
%    First, the macro that contains the current definitions of
%    |\@begintheorem| and |\@opargtheorem| should be called
%    up.  The name of this macro---as is already known---has the form
%    |\th@|\meta{theorem\ style}; therefore, it must be called by
%    \begin{macrocode}
      \csname th@\the \theorem@style
%    \end{macrocode}
%    In addition the default theorem font should be changeable, i.e.\
%    we have to insert the contents of |\theorem@bodyfont|. For
%    that reason, we expand even further, beyond |\endcsname|,
%    and thus insert the contents of the \textsf{token} register
%    |\theorem@bodyfont| in the replacement text.
%    \begin{macrocode}
            \expandafter \endcsname \the \theorem@bodyfont
%    \end{macrocode}
%    Now it is time to call the macro |\@thm| which takes over
%    the further processing.  It has two arguments: the current
%    counter name (in our example, \texttt{Lem}), and the text of the
%    label (in our example, \texttt{Lemma}).
%    \begin{macrocode}
     \@thm{#1}{#2}}%
%    \end{macrocode}
%    With this, the `sub-definition' is complete.  The macro
%    |\@endtheorem| ends a theorem environment and is, so far, nothing
%    but an |\endtrivlist|.  (Hence it is defined globally, and not
%    within the theorem styles.\footnote{This has to be changed as
%    soon as theorem styles that change \texttt{\bslash @endtheorem}
%    exist.  In such a case, all existing styles must be changed as
%    well since they will have to reset the macro.}) Therefore, we can
%    set it equivalent to the macro that ends the theorem set (in our
%    example, |\endLem|).  However, if some day theorem styles exist
%    that do change |\@endtheorem|, we would have to use the
%    commented-out line instead.
%    \begin{macrocode}
    \global \expandafter \let \csname end#1\endcsname \@endtheorem
%   \global\@namedef{end#1}{\@endtheorem}%
%    \end{macrocode}
%    With these commands all the required definitions are employed,
%    unless the test |\@ifdefinable| has failed.  Therefore, we
%    end the second argument of this macro and with it the definition
%    of |\@xnthm|.
%    \begin{macrocode}
   }}
%    \end{macrocode}
% \end{macro}
%
% \begin{macro}{\@ynthm}
%    The definition of |\@ynthm| is completely analogous. In
%    this case the new counter that is defined is not reset within
%    another counter; thus the definition of |\the...|
%    is simplified:
%    \begin{macrocode}
\gdef\@ynthm#1#2{\expandafter\@ifdefinable\csname #1\endcsname
   {\@definecounter{#1}%
    \expandafter\xdef\csname the#1\endcsname{\@thmcounter{#1}}%
%    \end{macrocode}
%    The rest of the definition corresponds literally to that of
%    |\@xnthm|:
%    \begin{macrocode}
    \def\@tempa{\global\@namedef{#1}}\expandafter \@tempa
     \expandafter{\csname th@\the \theorem@style \expandafter
     \endcsname \the\theorem@bodyfont \@thm{#1}{#2}}%
    \global \expandafter \let \csname end#1\endcsname \@endtheorem}}
%    \end{macrocode}
% \end{macro}
%
% \begin{macro}{\@othm}
%    The definition of |\@othm| does not contain anything new.
%    \begin{macrocode}
\gdef\@othm#1[#2]#3{%
%    \end{macrocode}
% \changes{v2.2b}{95/10/07}{Build in check for counter defined}
%    We do not define a new counter but instead use one that has
%    already been defined.  Thus the only definition we need is that
%    of this pseudo-counter (i.e.\ |\the|\meta{env.\ name}).
%    First we check if |#2| corresponds to a known counter name.
%    \begin{macrocode}
  \expandafter\ifx\csname c@#2\endcsname\relax
   \@nocounterr{#2}%
  \else
   \expandafter\@ifdefinable\csname #1\endcsname
   {\expandafter \xdef \csname the#1\endcsname
     {\expandafter \noexpand \csname the#2\endcsname}%
%    \end{macrocode}
%    All other parts of the definition can be adopted from
%    |\@xnthm|. We have to remember, though, that in this case
%    the name of the current counter and the theorem label have moved
%    to the second and third arguments.
%    \begin{macrocode}
    \def\@tempa{\global\@namedef{#1}}\expandafter \@tempa
     \expandafter{\csname th@\the \theorem@style \expandafter
     \endcsname \the\theorem@bodyfont \@thm{#2}{#3}}%
    \global \expandafter \let \csname end#1\endcsname \@endtheorem}%
  \fi}
%    \end{macrocode}
% \end{macro}
%
% \subsection{Macros that are employed in a {\envfont theorem}
%    environment}
%
% \begin{macro}{\@thm}
%    The macro |\@thm| has to increase the current counter.  Then,
%    depending on whether the environment has (or does not have) an
%    optional argument, it has to branch into either |\@begintheorem|
%    or |\@opargtheorem|.
%    \begin{macrocode}
\gdef\@thm#1#2{\refstepcounter{#1}%
%    \end{macrocode}
%    Now we start a {\envfont trivlist} environment, and give
%    |\@topsep| and |\@topsepadd| the values of the \textsf{skip}
%    registers |\theorempreskipamount| and |\theorempostskipamount|.
%    The value in |\@topsep| is the vertical space that is inserted by
%    the first (and only) |\item| in our |\trivlist| whilst
%    |\@topsepadd| is inserted by |\@endparenv| at the end of that
%    {\envfont trivlist} environment.  By using these registers, we
%    obtain the desired space around a {\envfont theorem} environment.
%    \begin{macrocode}
   \trivlist
   \@topsep \theorempreskipamount               % used by first \item
   \@topsepadd \theorempostskipamount           % used by \@endparenv
%    \end{macrocode}
%    Now we have to test whether an optional argument has been given.
% \changes{v2.0d}{89/07/18}{Forgotten percent added.}
%    \begin{macrocode}
   \@ifnextchar [%
%    \end{macrocode}
%    If there is an optional argument, we will call |\@ythm|, and
%    move the arguments read back into the input stream.
%    \begin{macrocode}
   {\@ythm{#1}{#2}}%
%    \end{macrocode}
%    If not, we call |\@begintheorem|. Its first argument is the
%    name of the theorem set (hence the second argument of
%    |\@thm|). Its second argument is the macro that produces the
%    current number.
%    \begin{macrocode}
   {\@begintheorem{#2}{\csname the#1\endcsname}\ignorespaces}}
%    \end{macrocode}
% \end{macro}
%
% \begin{macro}{\@xthm}
% \begin{macro}{\@ythm}
%    Both these macros were originally called by |\@thm|.  We do
%    not need |\@xthm| anymore, hence we reset it to
%    |\relax|. The definition of |\@ythm| has not changed at
%    all from its definition in \LaTeX. In order to make the macros
%    easier to understand, we will
%    nevertheless present it (commented out).
%    \begin{macrocode}
\global\let\@xthm\relax
% \def\@ythm#1#2[#3]{\@opargbegintheorem{#2}{\csname
%           the#1\endcsname}{#3}\ignorespaces}
%</package>
%    \end{macrocode}
%    The primitive |\ignorespaces| in |\@ythm| and
%    |\@thm| is needed to remove the spaces between the
%    |\begin{...}| and the actual text.
% \end{macro}
% \end{macro}
%
%
% \subsection{Definition of the theorem substyles}
%
% As already pointed out, the theorem substyles, defined below, are
% only loaded when necessary.  Note that all these substyles, except
% \texttt{plain}, have |\slshape| as the default body font.
%
% \subsubsection{The \texttt{plain} style}
%
% \label{dok:th@plain} As the following macros use |@|, we have
% to locally set the |\catcode| of this symbol to ``letter''.
% This happens within a group, so that we do not have to worry about
% which |\catcode| that symbol had before.
%    \begin{macrocode}
%<*thp>
\begingroup \makeatletter
%    \end{macrocode}
% Since we are now within a group, we must make all definitions
% globally.  First we make sure that \texttt{theorem.sty} is loaded.
% This will allow us to use this file as a document style option
% without having to call \texttt{theorem} itself as an option.  At the
% same time, we assure that at least version 2 is loaded, since
% |\theorem@style| was not defined in earlier versions.
%    \begin{macrocode}
\@ifundefined{theorem@style}{\input{theorem.sty}}{}
\ProvidesFile{thp.sty}
             [\FMithmInfo]
%    \end{macrocode}
%
%
% \begin{macro}{\th@plain}
% \begin{macro}{\@begintheorem}
% \begin{macro}{\@opargbegintheorem}
%    |\theoremstyle{plain}| corresponds to the original
%    definition, except that the distances to the surrounding text are
%    determined by the parameters |\theorempreskipamount| and
%    |\theorempostskipamount|.
%    First we set the default body font.
%    \begin{macrocode}
\gdef\th@plain{\normalfont\itshape
%    \end{macrocode}
%    Then we define |\@begintheorem| and
%    |\@opargbegintheorem|. These two macros define how the
%    header of a theorem is typeset.  |\@opargbegintheorem| will
%    be called if a \textsf{theorem} environment with an optional
%    argument is encountered; otherwise, the header is constructed by
%    calling |\@begintheorem|.  If one of these macros is
%    executed, we are within a \textsf{trivlist} environment started by
%    |\@thm|.  So the theorem header is produced with an
%    |\item| command.
%
%    Instead of specifying the header font directly, all standard
%    theorem styles use the |\theorem@headerfont| macro to allow
%    customization.  The extra space (|\labelsep|) is necessary
%    because of problems in the \textsf{trivlist} environment.
%    \begin{macrocode}
  \def\@begintheorem##1##2{%
        \item[\hskip\labelsep \theorem@headerfont ##1\ ##2]}%
%    \end{macrocode}
%    The definition of |\@opargbegintheorem| is completely
%    analogous. The only difference is the fact that there exists a
%    third argument (which is the optional parameter of the
%    environment and contains additional information about the
%    theorem).  Customarily we enclose it in parentheses.
%    \begin{macrocode}
\def\@opargbegintheorem##1##2##3{%
   \item[\hskip\labelsep \theorem@headerfont ##1\ ##2\ (##3)]}}
%    \end{macrocode}
% \end{macro}
% \end{macro}
% \end{macro}
%
% \noindent
% We conclude with an |\endgroup| to restore the
% |\catcode| of |@|.
%    \begin{macrocode}
\endgroup
%</thp>
%    \end{macrocode}
%
%
% \subsubsection{The \texttt{break} style}
%
% This style option is stored in the file |thb.sty|.
% For the next two lines see the documentation for |\th@plain| on
% page~\pageref{dok:th@plain}.
%    \begin{macrocode}
%<*thb>
\begingroup \makeatletter
\@ifundefined{theorem@style}{\input{theorem.sty}}{}
\ProvidesFile{thb.sty}
             [\FMithmInfo]
%    \end{macrocode}
%
% \begin{macro}{\th@break}
%    |\theoremstyle{break}| produces a line break after the name
%    of the theorem.  The font is |\slshape|.  Hence, we define
%    |\th@break| as follows:
%    \begin{macrocode}
\gdef\th@break{\normalfont\slshape
  \def\@begintheorem##1##2{\item[%
%    \end{macrocode}
%    We run into the following problem: it is not possible to create
%    the header with |\item[|\meta{title}|]| and then start
%    a new line by, for example, |\mbox{}\\|. Such a definition
%    will fail whenever a list environment follows immediately.  With
%    the above construction, the |\mbox{}| causes the switch
%    |@inlabel| (cf.\ definition of |\list| and
%    |\trivlist| in \cite{latex:source}) to be set to \texttt{false}
%    and so the following list will insert additional vertical
%    space (|\topskip|). This is quite annoying. Therefore, we
%    create the line break within the |\item|. In order to ensure
%    that the text will begin at the proper position in the following
%    line, we simply pretend that the label does not take any
%    room.\footnote{This will lead to problems whenever very high
%    symbols occurring in the line tower into the
%    heading. So, something else has to be done here sometime.}
%    \begin{macrocode}
       \rlap{\vbox{\hbox{\hskip \labelsep\theorem@headerfont ##1\ ##2}%
                   \hbox{\strut}}}]}%
%    \end{macrocode}
%    Again, the definition of |\@opargbegintheorem| is
%    completely analogous.
%    \begin{macrocode}
\def\@opargbegintheorem##1##2##3{%
  \item[\rlap{\vbox{\hbox{\hskip \labelsep \theorem@headerfont
                     ##1\ ##2\ (##3)}%
                    \hbox{\strut}}}]}}
\endgroup
%</thb>
%    \end{macrocode}
% \end{macro}
%
%
% \subsubsection{The \texttt{changebreak} style}
%
%    \begin{macrocode}
% This style option is stored in the file |thcb.sty|.
%    \begin{macrocode}
%<*thcb>
\begingroup \makeatletter
\@ifundefined{theorem@style}{\input{theorem.sty}}{}
\ProvidesFile{thcb.sty}
             [\FMithmInfo]
%    \end{macrocode}
%
% \begin{macro}{\th@changebreak}
%    The \texttt{change-break} theorem style is like \texttt{break}
%    but with interchange of theorem name and theorem number.  Thus we
%    define |\th@changebreak| as follows:
%    \begin{macrocode}
\gdef\th@changebreak{\normalfont\slshape
  \def\@begintheorem##1##2{\item
       [\rlap{\vbox{\hbox{\hskip\labelsep \theorem@headerfont ##2\ ##1}%
                    \hbox{\strut}}}]}%
\def\@opargbegintheorem##1##2##3{%
  \item[\rlap{\vbox{\hbox{\hskip\labelsep \theorem@headerfont
                     ##2\ ##1\ (##3)}%
                    \hbox{\strut}}}]}}
\endgroup
%</thcb>
%    \end{macrocode}
% \end{macro}
%
%
% \subsubsection{The \texttt{change} style}
%
% This style option is stored in the file |thc.sty|.
%    \begin{macrocode}
%<*thc>
\begingroup \makeatletter
\@ifundefined{theorem@style}{\input{theorem.sty}}{}
\ProvidesFile{thc.sty}
             [\FMithmInfo]
%    \end{macrocode}
%
% \begin{macro}{\th@change}
%    The \texttt{change} theorem style corresponds to the
%    \texttt{change break} style without a linebreak after the header.
%    To say it in another way, it's the same as the \texttt{plain}
%    style but with number and name interchanged and |\slshape| as the
%    default font.
%    \begin{macrocode}
\gdef\th@change{\normalfont\slshape
  \def\@begintheorem##1##2{\item
       [\hskip\labelsep \theorem@headerfont ##2\ ##1]}%
\def\@opargbegintheorem##1##2##3{%
  \item[\hskip\labelsep \theorem@headerfont ##2\ ##1\ (##3)]}}
\endgroup
%</thc>
%    \end{macrocode}
% \end{macro}
%
%
% \subsubsection{The \texttt{marginbreak} style}
%
% This style option is the one used most often at Mainz.  It is saved
% in the file |thmb.sty|.
%    \begin{macrocode}
%<*thmb>
\begingroup \makeatletter
\@ifundefined{theorem@style}{\input{theorem.sty}}{}
\ProvidesFile{thmb.sty}
             [\FMithmInfo]
%    \end{macrocode}
%
% \begin{macro}{\th@marginbreak}
%    The \texttt{margin break} style is nearly the same as the
%    \texttt{change break} style. The only difference is the placement
%    of the theorem number.  We use |\llap| to place it in the left
%    margin.
%
%    In this style |\labelsep| denotes the separation between the
%    number and the text.
%    \begin{macrocode}
\gdef\th@marginbreak{\normalfont\slshape
  \def\@begintheorem##1##2{\item
       [\rlap{\vbox{\theorem@headerfont
                    \hbox{\llap{##2}\hskip\labelsep ##1}%
                    \hbox{\strut}}}]}%
\def\@opargbegintheorem##1##2##3{%
  \item[\rlap{\vbox{\theorem@headerfont
                    \hbox{\llap{##2}\hskip\labelsep ##1\ (##3)}%
                    \hbox{\strut}}}]}}
\endgroup
%</thmb>
%    \end{macrocode}
% \changes{v2.1c}{90/02/17}{`endgroup outside macrocode env.}
% \end{macro}
%
%
%
% \subsubsection{The \texttt{margin} style}
%
% This style option is stored in the file |thm.sty|.
%    \begin{macrocode}
%<*thm>
\begingroup \makeatletter
\@ifundefined{theorem@style}{\input{theorem.sty}}{}
\ProvidesFile{thm.sty}
             [\FMithmInfo]
%    \end{macrocode}
%
% \begin{macro}{\th@margin}
%    Again this is only a variant of the theorem styles described
%    above without any new ideas.
%    \begin{macrocode}
\gdef\th@margin{\normalfont\slshape
  \def\@begintheorem##1##2{\item
       [\theorem@headerfont \llap{##2}\hskip\labelsep ##1]}%
\def\@opargbegintheorem##1##2##3{%
  \item[\theorem@headerfont \llap{##2}\hskip\labelsep ##1\ (##3)]}}
\endgroup
%</thm>
%    \end{macrocode}
% \end{macro}
%
%
%
%
% \subsection{Final Definitions}
%
% \begin{macro}{\theorempreskipamount}
% \begin{macro}{\theorempostskipamount}
%    The \textsf{skip} parameters that regulate the vertical empty space
%    before and after the {\envfont theorem} environment have
%    to be allocated as well.
% \changes{v2.0f}{89/07/20}{Unnecessary `global' removed.}
%    \begin{macrocode}
%<*package>
\newskip\theorempreskipamount
\newskip\theorempostskipamount
%    \end{macrocode}
%    Since we have used the same values for all theorem sets, we now can
%    assign them.
%    \begin{macrocode}
\global\setlength\theorempreskipamount{12pt plus 5pt minus 3pt}
\global\setlength\theorempostskipamount{8pt plus 3pt minus 1.5pt}
%    \end{macrocode}
% \end{macro}
% \end{macro}
%
%
% \begin{macro}{\@endtheorem}
%    The same holds for the macro |\@endtheorem|, which ends a
%    {\envfont theorem} environment. Since it is the same for all
%    theorem sets, it is removed from the macros |\th@|\meta{style}.
%    It simply ends the {\envfont trivlist} environment, which was
%    begun in |\@thm|.
%    \begin{macrocode}
\global\let\@endtheorem=\endtrivlist
%    \end{macrocode}
% \end{macro}
%
%
%
% \begin{macro}{\@preamblecmds}
%    All macros defined above are to be used only in the preamble.
%    Therefore, we insert them in |\@preamblecmds| which will disable
%    them at begin document. This is done by the internal \LaTeXe{}
%    command |\@onlypreamble|.
% \changes{v2.0g}{89/09/19}{`theoremheaderfont and `theorembodyfont
%    added.}
% \changes{v2.2a}{94/02/03}{Use \LaTeXe{} interface}
%    \begin{macrocode}
\@onlypreamble\@xnthm
\@onlypreamble\@ynthm
\@onlypreamble\@othm
\@onlypreamble\newtheorem
\@onlypreamble\theoremstyle
\@onlypreamble\theorembodyfont
\@onlypreamble\theoremheaderfont
%    \end{macrocode}
% \end{macro}
%
% Finally we declare the \texttt{plain} theorem style to be the default.
% \changes{v2.1b}{90/06/04}{Fixed missing `item by declaring default
%    style.}
%    \begin{macrocode}
\theoremstyle{plain}
%</package>
%    \end{macrocode}
%
% \Finale

\endinput
}{}
\ProvidesFile{thm.sty}
             [\FMithmInfo]
%    \end{macrocode}
%
% \begin{macro}{\th@margin}
%    Again this is only a variant of the theorem styles described
%    above without any new ideas.
%    \begin{macrocode}
\gdef\th@margin{\normalfont\slshape
  \def\@begintheorem##1##2{\item
       [\theorem@headerfont \llap{##2}\hskip\labelsep ##1]}%
\def\@opargbegintheorem##1##2##3{%
  \item[\theorem@headerfont \llap{##2}\hskip\labelsep ##1\ (##3)]}}
\endgroup
%</thm>
%    \end{macrocode}
% \end{macro}
%
%
%
%
% \subsection{Final Definitions}
%
% \begin{macro}{\theorempreskipamount}
% \begin{macro}{\theorempostskipamount}
%    The \textsf{skip} parameters that regulate the vertical empty space
%    before and after the {\envfont theorem} environment have
%    to be allocated as well.
% \changes{v2.0f}{89/07/20}{Unnecessary `global' removed.}
%    \begin{macrocode}
%<*package>
\newskip\theorempreskipamount
\newskip\theorempostskipamount
%    \end{macrocode}
%    Since we have used the same values for all theorem sets, we now can
%    assign them.
%    \begin{macrocode}
\global\setlength\theorempreskipamount{12pt plus 5pt minus 3pt}
\global\setlength\theorempostskipamount{8pt plus 3pt minus 1.5pt}
%    \end{macrocode}
% \end{macro}
% \end{macro}
%
%
% \begin{macro}{\@endtheorem}
%    The same holds for the macro |\@endtheorem|, which ends a
%    {\envfont theorem} environment. Since it is the same for all
%    theorem sets, it is removed from the macros |\th@|\meta{style}.
%    It simply ends the {\envfont trivlist} environment, which was
%    begun in |\@thm|.
%    \begin{macrocode}
\global\let\@endtheorem=\endtrivlist
%    \end{macrocode}
% \end{macro}
%
%
%
% \begin{macro}{\@preamblecmds}
%    All macros defined above are to be used only in the preamble.
%    Therefore, we insert them in |\@preamblecmds| which will disable
%    them at begin document. This is done by the internal \LaTeXe{}
%    command |\@onlypreamble|.
% \changes{v2.0g}{89/09/19}{`theoremheaderfont and `theorembodyfont
%    added.}
% \changes{v2.2a}{94/02/03}{Use \LaTeXe{} interface}
%    \begin{macrocode}
\@onlypreamble\@xnthm
\@onlypreamble\@ynthm
\@onlypreamble\@othm
\@onlypreamble\newtheorem
\@onlypreamble\theoremstyle
\@onlypreamble\theorembodyfont
\@onlypreamble\theoremheaderfont
%    \end{macrocode}
% \end{macro}
%
% Finally we declare the \texttt{plain} theorem style to be the default.
% \changes{v2.1b}{90/06/04}{Fixed missing `item by declaring default
%    style.}
%    \begin{macrocode}
\theoremstyle{plain}
%</package>
%    \end{macrocode}
%
% \Finale

\endinput
}{}
\ProvidesFile{thp.sty}
             [\FMithmInfo]
%    \end{macrocode}
%
%
% \begin{macro}{\th@plain}
% \begin{macro}{\@begintheorem}
% \begin{macro}{\@opargbegintheorem}
%    |\theoremstyle{plain}| corresponds to the original
%    definition, except that the distances to the surrounding text are
%    determined by the parameters |\theorempreskipamount| and
%    |\theorempostskipamount|.
%    First we set the default body font.
%    \begin{macrocode}
\gdef\th@plain{\normalfont\itshape
%    \end{macrocode}
%    Then we define |\@begintheorem| and
%    |\@opargbegintheorem|. These two macros define how the
%    header of a theorem is typeset.  |\@opargbegintheorem| will
%    be called if a \textsf{theorem} environment with an optional
%    argument is encountered; otherwise, the header is constructed by
%    calling |\@begintheorem|.  If one of these macros is
%    executed, we are within a \textsf{trivlist} environment started by
%    |\@thm|.  So the theorem header is produced with an
%    |\item| command.
%
%    Instead of specifying the header font directly, all standard
%    theorem styles use the |\theorem@headerfont| macro to allow
%    customization.  The extra space (|\labelsep|) is necessary
%    because of problems in the \textsf{trivlist} environment.
%    \begin{macrocode}
  \def\@begintheorem##1##2{%
        \item[\hskip\labelsep \theorem@headerfont ##1\ ##2]}%
%    \end{macrocode}
%    The definition of |\@opargbegintheorem| is completely
%    analogous. The only difference is the fact that there exists a
%    third argument (which is the optional parameter of the
%    environment and contains additional information about the
%    theorem).  Customarily we enclose it in parentheses.
%    \begin{macrocode}
\def\@opargbegintheorem##1##2##3{%
   \item[\hskip\labelsep \theorem@headerfont ##1\ ##2\ (##3)]}}
%    \end{macrocode}
% \end{macro}
% \end{macro}
% \end{macro}
%
% \noindent
% We conclude with an |\endgroup| to restore the
% |\catcode| of |@|.
%    \begin{macrocode}
\endgroup
%</thp>
%    \end{macrocode}
%
%
% \subsubsection{The \texttt{break} style}
%
% This style option is stored in the file |thb.sty|.
% For the next two lines see the documentation for |\th@plain| on
% page~\pageref{dok:th@plain}.
%    \begin{macrocode}
%<*thb>
\begingroup \makeatletter
\@ifundefined{theorem@style}{% \iffalse meta-comment
%
% Copyright (C) 1993-2024
%
% The LaTeX Project and any individual authors listed elsewhere
% in this file.
%
% This file is part of the Standard LaTeX `Tools Bundle'.
% -------------------------------------------------------
%
% It may be distributed and/or modified under the
% conditions of the LaTeX Project Public License, either version 1.3c
% of this license or (at your option) any later version.
% The latest version of this license is in
%    https://www.latex-project.org/lppl.txt
% and version 1.3c or later is part of all distributions of LaTeX
% version 2005/12/01 or later.
%
% The list of all files belonging to the LaTeX `Tools Bundle' is
% given in the file `manifest.txt'.
%
% \fi
%
%%
%
%
\def\FMithmInfo{2023/07/05 v2.2c Theorem extension package (FMi)}
%
% \ProvidesFile{theorem.dtx}[\FMithmInfo]
%\iffalse   % this is a METACOMMENT !
%%
%% Package `theorem' to use with LaTeX2e
%% Copyright (C) 1989-1995 Frank Mittelbach, all rights reserved.
%<+package>\@ifundefined{theorem@style}{}{\endinput}
%<+package>\NeedsTeXFormat{LaTeX2e}[1995/06/01]
%<+package>\ProvidesPackage{theorem}
%<+package>         [\FMithmInfo]
%
% \fi
%
% \changes{v1.0}{87/05/07}{First draft.}
% \changes{v1.0a}{87/05/07}{`thistheoremstyle' added.}
% \changes{v1.1a}{87/07/29}{New concept: redefine the internal
%                          \protect\LaTeX{} commands `@xnthm' etc.}
% \changes{v1.1b}{87/07/31}{`theoremfont' used to specify font.}
% \changes{v1.2a}{88/01/08}{`@endtheorem' now globally defined.}
% \changes{v1.2b}{88/01/11}{`newtheorem' only allowed in the preamble.}
% \changes{v1.3a}{88/02/14}{`theoremfont' now a toks register.}
%
% \changes{v2.0a}{88/11/16}{New implementation and new documentation.}
% \changes{v2.0a}{88/11/16}{No layout changes but internal modifications
%                           old substyles won't work.}
% \changes{v2.0b}{89/01/30}{`global' used for all definitions.}
% \changes{v2.0c}{89/06/15}{Documentation translated to English.
%                            Thanks to LLL and Nina}
% \changes{v2.1a}{90/02/15}{Support for new font selection.}
% \changes{v2.2a}{94/02/03}{Cleanup for LaTeX2e}
%
% \DoNotIndex{\@endparenv,\@ifdefinable,\@ifnextchar,\@ifundefined}
% \DoNotIndex{\@namedef,\@nameuse,\@spaces,\@tempa,\@warning}
% \DoNotIndex{\ }
% \DoNotIndex{\begingroup,\bfseries}
% \DoNotIndex{\csname}
% \DoNotIndex{\def,\do,\docdate}
% \DoNotIndex{\endcsname,\endgroup,\endinput,\endtrivlist,\expandafter}
% \DoNotIndex{\filedate,\fileversion}
% \DoNotIndex{\gdef,\global}
% \DoNotIndex{\hbox,\hskip}
% \DoNotIndex{\ignorespaces,\input,\itshape,\item}
% \DoNotIndex{\labelsep,\let,\llap}
% \DoNotIndex{\makeatletter}
% \DoNotIndex{\noexpand}
% \DoNotIndex{\refstepcounter,\relax,\rlap}
% \DoNotIndex{\setlength,\slshape,\space,\string,\strut}
% \DoNotIndex{\the,\trivlist,\typeout}
% \DoNotIndex{\vbox}
% \DoNotIndex{\xdef}
%
% \def\deflabel#1{\ttfamily #1\hfill}
% \def\deflist#1{\begin{list}{}{\settowidth\labelwidth{\ttfamily #1}%
%                               \setlength\leftmargin\labelwidth
%                               \addtolength\leftmargin\labelsep
%                               \let\makelabel\deflabel}}
% \def\enddeflist{\end{list}}
% \def\envfont{\normalfont\ttfamily}
%
% \MakeShortVerb{\|}
%
% \newcommand{\TUB}{{\sl TUGboat\/}}
%
% \GetFileInfo{theorem.dtx}
% \title{An Extension of the
%        \LaTeX{} theorem environment\thanks{This file
%        has version number \fileversion, last
%        revised \filedate.}}
% \author{Frank Mittelbach}
%
% \MaintainedByLaTeXTeam{tools}
% \maketitle
%
% \begin{abstract}
%    The macros described in this paper yield an extension of the
%    \LaTeX{} theorem mechanism. It is designed to satisfy the
%    different requirements of various journals. Thus, the
%    layout of the ``theorems'' can be manipulated by determining a
%    ``style''. This article describes not only the use, but
%    also the definition, of the necessary macros.
% \end{abstract}
%
% \section*{Preface to version 2.2}
%
% For \LaTeXe{} this package did not need any fundamental changes. I
% only modified the messages generated so that theorem layout styles
% will show up with the |\listfiles| command and cleaned the section
% on the New Font Selection Scheme since this is now included in
% \LaTeX.
%
% \section*{Preface to version 2.1}
%
% This version is identical to 2.0g described in \TUB\ 10\#3 except
% for some internal defaults which are now set depending on the used
% font selection scheme.
%
% This was done to avoid unpleasant surprises if the new font selection
% scheme is in force. For further details see
% section~\ref{sec:newfonts} and \cite{bk:GMS94}.
%
% \section{Introduction}
%
% For our purposes here, ``theorems'' are labelled enunciations,
% often set off from the main text by extra space and a font change.
% Theorems, corollaries, conjectures, definitions, and remarks
% are all instances of ``theorems''.  The ``header'' of these structures
% is composed of a label (such as \textsc{Theorem} or \textsc{Remark})
% and a number which serializes an item in the sequence of items
% with the same label.
%
% Shortly after the introduction of \LaTeX{} at the Fachbereich
% Mathematik in Mainz, the desire to manipulate the layout of
% ``theorems'' arose.  In Mainz, the following two conventions came
% into general use:
% \begin{enumerate}
%   \item The number of the theorem is shown in the margin.
%   \item There is a line break at
%         the end of the theorem header.
% \end{enumerate}
%
% Additionally, some journals require different formats which depend on
% the ``sort of theorem'': e.g.\ often remarks and definitions are set
% in |\upshape|, while |\itshape| is employed for main theorems.
%
% Confronted with these requirements, a theorem environment was
% developed in Mainz which allows separate determination of the
% layout of the ``theorems sets'', comparable to |\pagestyle|.
%
%
% \section{The user interface}
% \subsection{Defining new theorem sets}
%
% \DescribeMacro\newtheorem
% As in the original \LaTeX{} version, the command |\newtheorem|
% defines a new ``theorem set'' or ``theorem-like structure''.
% Two required arguments name the new environment and give the text
% to be typeset with each instance of the new ``set'', while
% an optional argument determines how
% the ``set'' is enumerated:
% \begin{description}
%    \item[\ttfamily \bslash newtheorem\{foo\}\{bar\}]
%       The theorem set {\envfont foo} (whose name is \texttt{bar})
%       uses its own counter.
%    \item[\ttfamily \bslash newtheorem\{foo2\}{[foo]}\{bar2\}]
%       The theorem set {\envfont foo2} (printed name \texttt{bar2})
%       uses the same counter as the theorem set \texttt{foo}.
%    \item[\ttfamily \bslash newtheorem\{foo3\}\{bar3\}{[section]}]
%       The theorem set {\envfont foo3} (printed name \texttt{bar3}) is
%       enumerated within the counter \texttt{section}, i.e.\ with every
%       new |\section| the enumeration begins again with 1, and
%       the enumeration is composed from the section-number and the
%       theorem counter itself.
% \end{description}
%
% \DescribeMacro\theoremstyle
% Additionally, the command |\theoremstyle| can define the layout
% of various, or all, theorem sets.  It should be noted that any
% theorem set defined by |\newtheorem| is typeset in the
% |\theoremstyle| that is current at the time of the definition.
% Thus, the following
% \begin{verbatim}
%    \theoremstyle{break}        \newtheorem{Cor}{Corollary}
%    \theoremstyle{plain}        \newtheorem{Exa}{Example}[section]
%\end{verbatim}                             ^^A No blank at line start
% leads to the result that the set {\envfont Cor} is formatted in the
% style \texttt{break}, while the set {\envfont Exa} and all the
% following ones are formatted in the style \texttt{plain}, unless
% another |\theoremstyle| follows.  Since the definitions installed by
% |\newtheorem| are global, one also can limit |\theoremstyle| locally
% by grouping braces.
%
% \DescribeMacro\theorembodyfont
% The choice of the font for the theorem body is completely
% independent of the chosen |\theoremstyle|; this has proven to
% be very advantageous.  For example,
% \begin{verbatim}
%   {\theorembodyfont{\upshape}          \newtheorem{Rem}{Remark}}
%\end{verbatim}
% defines a theorem set {\envfont Rem}, which will be set in
% |\upshape| in the current layout (which in our example is
% \texttt{plain}).  As with |\theoremstyle|, the |\theorembodyfont|
% chosen is that current at the time of |\newtheorem|.  If
% |\theorembodyfont| is not specified or one defines
% |\theorembodyfont{}|, then the font used will be that defined by the
% |\theoremstyle|.
%
% \DescribeMacro\theoremheaderfont It is also possible to customize
% the font used for the theorem headers.  This is, however, a global
% declaration, and therefore there should be at most one
% |\theoremheaderfont| declaration in the preamble.\footnote{If it is
% actually necessary to have different header fonts, one has to define
% new theorem styles (substituting the desired font) or specify the
% information directly in the \texttt{\bslash newtheorem} declaration
% (the unclean variant).}
%
% \DescribeMacro\theorempreskipamount
% \DescribeMacro\theorempostskipamount Two additional parameters
% affect the vertical space around the theorem environments:
% |\theorempreskipamount| and |\theorempostskipamount| define,
% respectively, the spacing before and after such an environment.
% These parameters apply for all theorem sets and can be manipulated
% with the ordinary length macros.  They are rubber lengths
% (`\textsf{skips}'), and therefore can contain \texttt{plus} and
% \texttt{minus} parts.
%
% Since the definition of theorem sets should---most sensibly---be
% placed in the preamble, we only allow installation there.
% It is therefore possible to release the memory used here after
% |\begin{document}|, in order to make room for other
% applications.
%
%
% \subsection{Existing theorem styles}
%
% The following theorem styles exist to date:
% \begin{deflist}{marginbreak}
%    \item[plain]
%       This theorem style emulates the original \LaTeX{} definition,
%       except that additionally the parameters
%       |\theorem...skipamount| are used.
%    \item[break]
%       In this style, the theorem header is followed by a line
%       break.
%    \item[marginbreak]
%       The theorem number is set in the margin, and there is a line
%       break as in \texttt{break}.
%    \item[changebreak]
%       Like \texttt{break}, but with header number and text
%       interchanged.
%    \item[change]
%       Header number and text are interchanged, without a line break.
%    \item[margin]
%       The number is set in the left margin, without a line break.
% \end{deflist}
% All styles (except \texttt{plain}) select |\slshape| as the default
% |\theorembodyfont|.
%
%
% \subsection{Examples}
%
% Given the above theorem sets {\envfont Cor}, {\envfont Exa}
% and {\envfont Rem},
% suppose that the preamble also contains the declarations:
% \begin{verbatim}
%    \theoremstyle{marginbreak}   \newtheorem{Lem}[Cor]{Lemma}
%    \theoremstyle{change}
%    \theorembodyfont{\itshape}        \newtheorem{Def}[Cor]{Definition}
%
%    \theoremheaderfont{\scshape}
%\end{verbatim}
% Then the following are some typical examples of the typeset output
% resulting from their use.
%
% \begin{Cor}
%    This is a sentence typeset in the theorem environment {\envfont
%    Cor}.
% \end{Cor}
%
% \begin{Exa}
%    This is a sentence typeset in the theorem environment {\envfont
%    Exa}.
% \end{Exa}
%
% \begin{Rem}
%    This is a sentence typeset in the theorem environment {\envfont
%    Rem}.
% \end{Rem}
%
% \begin{Lem}[Ben User]
%    This is a sentence typeset in the theorem environment {\envfont
%    Lem}.
% \end{Lem}
%
% \begin{Def}[Very impressive Definition]
%    This is a sentence typeset in the theorem environment {\envfont
%    Def}.
% \end{Def}
%
% The last two examples show the effect of the optional argument to
% a theorem environment (it is the text typeset in parentheses).
%
%
% \section{Special Considerations}
% \label{sec:newfonts}
%
% Theoremheader and body are implemented as a unit. This means that
% the |\theoremheaderfont|  will inherit characteristics of the
% |\theorembodyfont| in \LaTeXe. Thus, if for example
% |\theorembodyfont| is |\itshape| and |\theoremheaderfont|
% is |\bfseries| the font selected for the header will have the
% characteristics `bold extended italic'. If this is not desired one
% should set the |\theoremheaderfont| to something like
% \begin{verbatim}
%      \theoremheaderfont{\normalfont\bfseries}
%\end{verbatim}
% i.e.\ supplying all necessary font information explicitly.
%
%
%
% \section{Acknowledgements}
%
% The publication of this set of macros was only possible with the
% help of Christina Busse (translating the manuscript into English),
% Joachim Pense (playing the r\^ole of typist), Chris Rowley
% (looking everything over) and many others providing useful
% suggestions.
%
% \MaybeStop{%
%    \begin{thebibliography}{9}
%       \bibitem{bk:GMS94} \textsc{M.~Goossens}, \textsc{F.~Mittelbach}
%         and \textsc{A.~Samarin}.
%         \newblock The \LaTeX{} Companion.
%         \newblock
%         Addison-Wesley, Reading, Massachusetts, 1994.
%       \bibitem{latex:source} \textsc{Lamport, Leslie.} \newblock
%                               \texttt{latex.tex}, version 2.09, date
%                               Feb. 1990.
%    \end{thebibliography}
%    \PrintIndex
%    \PrintChanges}
%
%
% \section{The documentation driver file}
%
% The next bit of code contains the documentation driver file for
% \TeX{}, i.e., the file that will produce the documentation you are
% currently reading. It will be extracted from this file by the
% \texttt{docstrip} program. Since it is the first code in the file
% one can alternatively process this file directly with \LaTeXe{} to
% obtain the documentation.
%
%    \begin{macrocode}
%<*driver>
\documentclass{ltxdoc}

\usepackage{theorem}

%<+driver>% The next few lines define theorem sets which are used
%<+driver>% in the example section of the documentation.

\theoremstyle{break}       \newtheorem{Cor}{Corollary}
\theoremstyle{plain}       \newtheorem{Exa}{Example}[section]
{\theorembodyfont{\upshape}\newtheorem{Rem}{Remark}}
\theoremstyle{marginbreak} \newtheorem{Lem}[Cor]{Lemma}
\theoremstyle{change}
\theorembodyfont{\itshape} \newtheorem{Def}[Cor]{Definition}

\theoremheaderfont{\scshape}

\RecordChanges

\begin{document}
   \DocInput{theorem.dtx}
\end{document}
%</driver>
%    \end{macrocode}
%
% \section{Definition of the Macros}
%
% If the file has been loaded before, we abort immediately. If not the
% package announces itself (this is actually done at the very top of
% the file---the way it is done isn't good style so don't copy it).
% \changes{v2.0e}{89/07/19}{Spaces removed from `typeout'.}
% \changes{v2.2b}{95/11/19}{Announce moved to top of file}
%    \begin{macrocode}
%<*package>
%\@ifundefined{theorem@style}{}{\endinput}
%\def\FMithmInfo{1995/11/19 v2.2b Theorem extension package (FMi)}
%\ProvidesPackage{theorem}[\FMithmInfo]
%    \end{macrocode}
%
%
% \subsection{Definition of theorem styles and fonts}
%
% All the definitions in this file are done globally to allow
% inputting this file inside a group.
%
% \begin{macro}{\theoremstyle}
%    Before a theorem style can be installed, the chosen style must be
%    known. For that reason, we must test to see that
%    |\th@|\meta{style} is known or, more precisely, that it is
%    different from |\relax|. If the style is not known then
%    |\th@plain| is used.
%    \begin{macrocode}
\gdef\theoremstyle#1{%
   \@ifundefined{th@#1}{\@warning
          {Unknown theoremstyle `#1'. Using `plain'}%
          \theorem@style{plain}}%
%    \end{macrocode}
%    We save the theorem style to be used in the token register
%    |\theorem@style|.
%    \begin{macrocode}
      {\theorem@style{#1}}%
%    \end{macrocode}
%    Now we ``evaluate'' the theorem style: this means, we call the
%    macro |\th@|\meta{style} which will activate the relevant
%    definitions which are contained in a separate file.  This is done
%    in a group to suppress changes to the current font.  This could
%    otherwise pose problems together with the new font selection
%    scheme\footnote{When I printed the original article using the new
%    font selection scheme I ended with a document with slanted
%    typefaces (text headings and all) simply because one of the
%    theorem styles used \texttt{\bslash sl} at toplevel.} if the
%    |\th@|\meta{style} is evaluated a second time.
% \changes{v2.1a}{90/02/15}{Group added around `th@..}
% \changes{v2.1a}{90/02/15}{Reseting of `..@begintheorem removed.}
%    \begin{macrocode}
      \begingroup
        \csname th@\the\theorem@style \endcsname
      \endgroup}
%    \end{macrocode}
% \end{macro}
%
%
% \begin{macro}{\@begintheorem}
% \changes{v2.1a}{90/02/15}{Relaxed at toplevel.}
% \begin{macro}{\@opargbegintheorem}
% \changes{v2.1a}{90/02/15}{Relaxed at toplevel.}
%    We reset |\@begintheorem| and |\@opargbegintheorem| to
%    |\relax| since these commands are no longer necessary at
%    toplevel. This will save a few tokens.
%    \begin{macrocode}
\global\let\@begintheorem\relax
\global\let\@opargbegintheorem\relax
%    \end{macrocode}
% \end{macro}
% \end{macro}
%
% \begin{macro}{\theorem@style}
%    Obviously the \textsf{token} register used above has to be
%    allocated. To assure the utmost compatibility with the
%    original \LaTeX{} definition, we set the default theorem style to
%    \texttt{plain}, which implements the usual \LaTeX{} convention.
% \changes{v2.0f}{89/07/20}{Unnecessary `global' removed.}
%    \begin{macrocode}
\newtoks\theorem@style
\global\theorem@style{plain}
%    \end{macrocode}
% \end{macro}
%
% \begin{macro}{\theorembodyfont}
% \begin{macro}{\theorem@bodyfont}
%    If the |\theorembodyfont| is set by the user then it should not
%    interact with the default font set in the theorem style. When the
%    new font selection is in force this may happen if, for example,
%    the default is |\itshape| and the new |\theorembodyfont| is
%    |\sffamily|. So we add a |\reset@font| command in front of the
%    user definition.
% \changes{v2.0f}{89/07/20}{Unnecessary `global' removed.}
% \changes{v2.0g}{89/09/19}{`theoremfont to `theorembodyfont.}
% \changes{v2.1c}{92/06/22}{`reset@font added and logic changed.}
% \changes{v2.1d}{92/10/28}{Reset to empty if arg is empty}
%    \begin{macrocode}
\gdef\theorembodyfont#1{%
%    \end{macrocode}
%    We check if the argument supplied is empty and if so put nothing
%    into the |\theorem@bodyfont| token register to allow for
%    |\theorembodyfont{}| as a mean of using the default of the
%    current |\theoremstyle|.
%    \begin{macrocode}
   \def\@tempa{#1}%
   \ifx\@tempa\@empty
    \theorem@bodyfont{}%
   \else
    \theorem@bodyfont{\reset@font#1}%
   \fi
   }
\newtoks\theorem@bodyfont
\global\theorem@bodyfont{}
%    \end{macrocode}
% \end{macro}
% \end{macro}
%
%
% \begin{macro}{\theoremheaderfont}
% \changes{v2.0g}{89/09/19}{Macro added.}
%    The font for the theorem headers is handled differently because
%    this definition applies to all theorem styles.
%    \begin{macrocode}
\gdef\theoremheaderfont#1{\gdef\theorem@headerfont{#1}%
%    \end{macrocode}
%    After using the macro once it is redefined to produce an error
%    message.
%    \begin{macrocode}
       \gdef\theoremheaderfont##1{%
        \typeout{\string\theoremheaderfont\space should be used
                 only once.}}}
%    \end{macrocode}
% \end{macro}
%
% \begin{macro}{\theorem@headerfont}
% \changes{v2.0g}{89/09/19}{Macro added.}
%    To set the |\theorem@headerfont| default we first test if
%    the new fontselection scheme is in force.
% \changes{v2.1a}{90/02/15}{Support for new font selection.}
%    \begin{macrocode}
\ifx\upshape\undefined
%    \end{macrocode}
%    If not we define it to expand into |\bfseries|. We don't use
%    |\let| just in case a following style option redefines this
%    macro.
%    \begin{macrocode}
\gdef\theorem@headerfont{\bfseries}
%    \end{macrocode}
%    Otherwise we reset the current shape before calling |\bfseries|.
%    \begin{macrocode}
\else \gdef\theorem@headerfont{\normalfont\bfseries}\fi
%    \end{macrocode}
% \end{macro}
%
% \begin{macro}{\th@plain}
% \begin{macro}{\th@break}
% \begin{macro}{\th@marginbreak}
% \begin{macro}{\th@changebreak}
% \begin{macro}{\th@change}
% \begin{macro}{\th@margin}
%    The different styles are defined in macros such as
%    |\th@plain|.  Since memory space is precious in
%    ``non-Big-versions'', we have to avoid offering too many unused
%    definitions.  Therefore we define these styles in separate files
%    that can be loaded on demand.  Thus the commands themselves only
%    load these files. We use |\@input@|, a \LaTeXe{} internal command
%    that ensures that the file will be listed with |\listfiles|.
% \changes{v2.2a}{94/02/02}{Use `@input@ to load theorem layout files}
%    \begin{macrocode}
\gdef\th@plain{\@input@{thp.sty}}
\gdef\th@break{\@input@{thb.sty}}
\gdef\th@marginbreak{\@input@{thmb.sty}}
\gdef\th@changebreak{\@input@{thcb.sty}}
\gdef\th@change{\@input@{thc.sty}}
\gdef\th@margin{\@input@{thm.sty}}
%    \end{macrocode}
%    This list will be expanded when new styles become
%    available.  For testing,  just append new theorem substyles
%    as document options.
% \end{macro}
% \end{macro}
% \end{macro}
% \end{macro}
% \end{macro}
% \end{macro}
%
%
%
% \subsection{Definition of a new theorem set}
%
% As already pointed out, a new theorem environment can be defined in
% three different ways:
% \begin{verbatim}
%    \newtheorem{Lem}{Lemma}
%    \newtheorem{Lem}{Lemma}[section]
%    \newtheorem{Lem}[Theorem]{Lemma}
%\end{verbatim}
% The function of the macro |\newtheorem| is to recognize these cases
% and then to branch into one of the three macros |\@ynthm|, |\@xnthm|
% or |\@othm|.  This mechanism is adopted unchanged from
% \cite{latex:source}; the essential point here is that, for example,
% in the second case, the arguments \texttt{Lem}, \texttt{Lemma} and
% \texttt{section} are passed over to the macro |\@xnthm|.
%
% We inspect this case first because the others present fewer problems,
% and thus are easily derived from this one.
%
% \begin{macro}{\@xnthm}
%    For our example arguments, the macro |\@xnthm| must fulfill
%    the following:
%    \begin{itemize}
%       \item
%          Define a new \LaTeX-counter `Lem'
%       \item
%          reset this counter within a |\section|
%       \item
%          define the macro |\theLem|
%       \item
%          define the environment macros |\Lem| and
%          |\endLem| using the current
%          |\theoremstyle| and |\theorem@bodyfont|.
%    \end{itemize}
%    Obviously, all this should happen only if the first argument
%    of |\@xnthm| (i.e.\ \texttt{Lem} in our example) is chosen so
%    as not to conflict with any previously defined commands or
%    environments.  This test is performed by the \LaTeX{} macro
%    |\@ifdefinable|.
%    \begin{macrocode}
\gdef\@xnthm#1#2[#3]{\expandafter\@ifdefinable\csname #1\endcsname
%    \end{macrocode}
%    Therefore, the first argument of |\@ifdefinable| is the
%    expansion (in the example, |\Lem|) of
%    |\csname||#1\endcsname|.  The second argument is
%    executed only if the test has been completed successfully.
%    \begin{macrocode}
   {%
%    \end{macrocode}
%    Now we define the new counter. The names of the \LaTeX{}
%    macros employed should speak for themselves:
% \changes{v2.2b}{1995/10/07}{Use \cs{@newctr}}
% \changes{v2.2c}{1995/11/23}{Use \cs{@newctr} properly.}
%    \begin{macrocode}
    \@definecounter{#1}\@newctr{#1}[#3]%
%    \end{macrocode}
%    Using |\@newctr| will give a proper error message if the counter
%    in |#3| is not defined.
%    In defining `|\theLem|' we must generate the desired macro
%    name by use of |\expandafter| and |\csname|.
%    \begin{macrocode}
    \expandafter\xdef\csname the#1\endcsname
%    \end{macrocode}
%    An |\xdef| is used in order to make the definition global,
%    and to ensure that it contains the replacement texts of
%    |\@thmcountersep| and |\@thmcounter|.\footnote{These
%    two macros can be defined by the document style.  Their default
%    values produce a `\texttt{.}' as separation and an arabic
%    representation of the number.} However, not everything should be
%    expanded. For example, it saves space to use |\thesection|
%    instead of its---at times---lengthy expansion.
%    \begin{macrocode}
      {\expandafter \noexpand \csname the#3\endcsname
       \@thmcountersep \@thmcounter{#1}}%
%    \end{macrocode}
%    Thus with the defaults of \LaTeX{}, |\theLem| would be
%    replaced by the command sequence |\thesection.\arabic{Lem}|.
%
%    We will now look at the definition of the macro which is executed
%    at the beginning of the actual environment (in our example this
%    macro is |\Lem|). It should be noted that we use an
%    ``|\expandafter| trick'' to expand only certain parts of the
%    replacement text at the time of the definition.
%    \begin{macrocode}
    \def\@tempa{\global\@namedef{#1}}%
    \expandafter \@tempa \expandafter{%
%    \end{macrocode}
%    First, the macro that contains the current definitions of
%    |\@begintheorem| and |\@opargtheorem| should be called
%    up.  The name of this macro---as is already known---has the form
%    |\th@|\meta{theorem\ style}; therefore, it must be called by
%    \begin{macrocode}
      \csname th@\the \theorem@style
%    \end{macrocode}
%    In addition the default theorem font should be changeable, i.e.\
%    we have to insert the contents of |\theorem@bodyfont|. For
%    that reason, we expand even further, beyond |\endcsname|,
%    and thus insert the contents of the \textsf{token} register
%    |\theorem@bodyfont| in the replacement text.
%    \begin{macrocode}
            \expandafter \endcsname \the \theorem@bodyfont
%    \end{macrocode}
%    Now it is time to call the macro |\@thm| which takes over
%    the further processing.  It has two arguments: the current
%    counter name (in our example, \texttt{Lem}), and the text of the
%    label (in our example, \texttt{Lemma}).
%    \begin{macrocode}
     \@thm{#1}{#2}}%
%    \end{macrocode}
%    With this, the `sub-definition' is complete.  The macro
%    |\@endtheorem| ends a theorem environment and is, so far, nothing
%    but an |\endtrivlist|.  (Hence it is defined globally, and not
%    within the theorem styles.\footnote{This has to be changed as
%    soon as theorem styles that change \texttt{\bslash @endtheorem}
%    exist.  In such a case, all existing styles must be changed as
%    well since they will have to reset the macro.}) Therefore, we can
%    set it equivalent to the macro that ends the theorem set (in our
%    example, |\endLem|).  However, if some day theorem styles exist
%    that do change |\@endtheorem|, we would have to use the
%    commented-out line instead.
%    \begin{macrocode}
    \global \expandafter \let \csname end#1\endcsname \@endtheorem
%   \global\@namedef{end#1}{\@endtheorem}%
%    \end{macrocode}
%    With these commands all the required definitions are employed,
%    unless the test |\@ifdefinable| has failed.  Therefore, we
%    end the second argument of this macro and with it the definition
%    of |\@xnthm|.
%    \begin{macrocode}
   }}
%    \end{macrocode}
% \end{macro}
%
% \begin{macro}{\@ynthm}
%    The definition of |\@ynthm| is completely analogous. In
%    this case the new counter that is defined is not reset within
%    another counter; thus the definition of |\the...|
%    is simplified:
%    \begin{macrocode}
\gdef\@ynthm#1#2{\expandafter\@ifdefinable\csname #1\endcsname
   {\@definecounter{#1}%
    \expandafter\xdef\csname the#1\endcsname{\@thmcounter{#1}}%
%    \end{macrocode}
%    The rest of the definition corresponds literally to that of
%    |\@xnthm|:
%    \begin{macrocode}
    \def\@tempa{\global\@namedef{#1}}\expandafter \@tempa
     \expandafter{\csname th@\the \theorem@style \expandafter
     \endcsname \the\theorem@bodyfont \@thm{#1}{#2}}%
    \global \expandafter \let \csname end#1\endcsname \@endtheorem}}
%    \end{macrocode}
% \end{macro}
%
% \begin{macro}{\@othm}
%    The definition of |\@othm| does not contain anything new.
%    \begin{macrocode}
\gdef\@othm#1[#2]#3{%
%    \end{macrocode}
% \changes{v2.2b}{95/10/07}{Build in check for counter defined}
%    We do not define a new counter but instead use one that has
%    already been defined.  Thus the only definition we need is that
%    of this pseudo-counter (i.e.\ |\the|\meta{env.\ name}).
%    First we check if |#2| corresponds to a known counter name.
%    \begin{macrocode}
  \expandafter\ifx\csname c@#2\endcsname\relax
   \@nocounterr{#2}%
  \else
   \expandafter\@ifdefinable\csname #1\endcsname
   {\expandafter \xdef \csname the#1\endcsname
     {\expandafter \noexpand \csname the#2\endcsname}%
%    \end{macrocode}
%    All other parts of the definition can be adopted from
%    |\@xnthm|. We have to remember, though, that in this case
%    the name of the current counter and the theorem label have moved
%    to the second and third arguments.
%    \begin{macrocode}
    \def\@tempa{\global\@namedef{#1}}\expandafter \@tempa
     \expandafter{\csname th@\the \theorem@style \expandafter
     \endcsname \the\theorem@bodyfont \@thm{#2}{#3}}%
    \global \expandafter \let \csname end#1\endcsname \@endtheorem}%
  \fi}
%    \end{macrocode}
% \end{macro}
%
% \subsection{Macros that are employed in a {\envfont theorem}
%    environment}
%
% \begin{macro}{\@thm}
%    The macro |\@thm| has to increase the current counter.  Then,
%    depending on whether the environment has (or does not have) an
%    optional argument, it has to branch into either |\@begintheorem|
%    or |\@opargtheorem|.
%    \begin{macrocode}
\gdef\@thm#1#2{\refstepcounter{#1}%
%    \end{macrocode}
%    Now we start a {\envfont trivlist} environment, and give
%    |\@topsep| and |\@topsepadd| the values of the \textsf{skip}
%    registers |\theorempreskipamount| and |\theorempostskipamount|.
%    The value in |\@topsep| is the vertical space that is inserted by
%    the first (and only) |\item| in our |\trivlist| whilst
%    |\@topsepadd| is inserted by |\@endparenv| at the end of that
%    {\envfont trivlist} environment.  By using these registers, we
%    obtain the desired space around a {\envfont theorem} environment.
%    \begin{macrocode}
   \trivlist
   \@topsep \theorempreskipamount               % used by first \item
   \@topsepadd \theorempostskipamount           % used by \@endparenv
%    \end{macrocode}
%    Now we have to test whether an optional argument has been given.
% \changes{v2.0d}{89/07/18}{Forgotten percent added.}
%    \begin{macrocode}
   \@ifnextchar [%
%    \end{macrocode}
%    If there is an optional argument, we will call |\@ythm|, and
%    move the arguments read back into the input stream.
%    \begin{macrocode}
   {\@ythm{#1}{#2}}%
%    \end{macrocode}
%    If not, we call |\@begintheorem|. Its first argument is the
%    name of the theorem set (hence the second argument of
%    |\@thm|). Its second argument is the macro that produces the
%    current number.
%    \begin{macrocode}
   {\@begintheorem{#2}{\csname the#1\endcsname}\ignorespaces}}
%    \end{macrocode}
% \end{macro}
%
% \begin{macro}{\@xthm}
% \begin{macro}{\@ythm}
%    Both these macros were originally called by |\@thm|.  We do
%    not need |\@xthm| anymore, hence we reset it to
%    |\relax|. The definition of |\@ythm| has not changed at
%    all from its definition in \LaTeX. In order to make the macros
%    easier to understand, we will
%    nevertheless present it (commented out).
%    \begin{macrocode}
\global\let\@xthm\relax
% \def\@ythm#1#2[#3]{\@opargbegintheorem{#2}{\csname
%           the#1\endcsname}{#3}\ignorespaces}
%</package>
%    \end{macrocode}
%    The primitive |\ignorespaces| in |\@ythm| and
%    |\@thm| is needed to remove the spaces between the
%    |\begin{...}| and the actual text.
% \end{macro}
% \end{macro}
%
%
% \subsection{Definition of the theorem substyles}
%
% As already pointed out, the theorem substyles, defined below, are
% only loaded when necessary.  Note that all these substyles, except
% \texttt{plain}, have |\slshape| as the default body font.
%
% \subsubsection{The \texttt{plain} style}
%
% \label{dok:th@plain} As the following macros use |@|, we have
% to locally set the |\catcode| of this symbol to ``letter''.
% This happens within a group, so that we do not have to worry about
% which |\catcode| that symbol had before.
%    \begin{macrocode}
%<*thp>
\begingroup \makeatletter
%    \end{macrocode}
% Since we are now within a group, we must make all definitions
% globally.  First we make sure that \texttt{theorem.sty} is loaded.
% This will allow us to use this file as a document style option
% without having to call \texttt{theorem} itself as an option.  At the
% same time, we assure that at least version 2 is loaded, since
% |\theorem@style| was not defined in earlier versions.
%    \begin{macrocode}
\@ifundefined{theorem@style}{% \iffalse meta-comment
%
% Copyright (C) 1993-2024
%
% The LaTeX Project and any individual authors listed elsewhere
% in this file.
%
% This file is part of the Standard LaTeX `Tools Bundle'.
% -------------------------------------------------------
%
% It may be distributed and/or modified under the
% conditions of the LaTeX Project Public License, either version 1.3c
% of this license or (at your option) any later version.
% The latest version of this license is in
%    https://www.latex-project.org/lppl.txt
% and version 1.3c or later is part of all distributions of LaTeX
% version 2005/12/01 or later.
%
% The list of all files belonging to the LaTeX `Tools Bundle' is
% given in the file `manifest.txt'.
%
% \fi
%
%%
%
%
\def\FMithmInfo{2023/07/05 v2.2c Theorem extension package (FMi)}
%
% \ProvidesFile{theorem.dtx}[\FMithmInfo]
%\iffalse   % this is a METACOMMENT !
%%
%% Package `theorem' to use with LaTeX2e
%% Copyright (C) 1989-1995 Frank Mittelbach, all rights reserved.
%<+package>\@ifundefined{theorem@style}{}{\endinput}
%<+package>\NeedsTeXFormat{LaTeX2e}[1995/06/01]
%<+package>\ProvidesPackage{theorem}
%<+package>         [\FMithmInfo]
%
% \fi
%
% \changes{v1.0}{87/05/07}{First draft.}
% \changes{v1.0a}{87/05/07}{`thistheoremstyle' added.}
% \changes{v1.1a}{87/07/29}{New concept: redefine the internal
%                          \protect\LaTeX{} commands `@xnthm' etc.}
% \changes{v1.1b}{87/07/31}{`theoremfont' used to specify font.}
% \changes{v1.2a}{88/01/08}{`@endtheorem' now globally defined.}
% \changes{v1.2b}{88/01/11}{`newtheorem' only allowed in the preamble.}
% \changes{v1.3a}{88/02/14}{`theoremfont' now a toks register.}
%
% \changes{v2.0a}{88/11/16}{New implementation and new documentation.}
% \changes{v2.0a}{88/11/16}{No layout changes but internal modifications
%                           old substyles won't work.}
% \changes{v2.0b}{89/01/30}{`global' used for all definitions.}
% \changes{v2.0c}{89/06/15}{Documentation translated to English.
%                            Thanks to LLL and Nina}
% \changes{v2.1a}{90/02/15}{Support for new font selection.}
% \changes{v2.2a}{94/02/03}{Cleanup for LaTeX2e}
%
% \DoNotIndex{\@endparenv,\@ifdefinable,\@ifnextchar,\@ifundefined}
% \DoNotIndex{\@namedef,\@nameuse,\@spaces,\@tempa,\@warning}
% \DoNotIndex{\ }
% \DoNotIndex{\begingroup,\bfseries}
% \DoNotIndex{\csname}
% \DoNotIndex{\def,\do,\docdate}
% \DoNotIndex{\endcsname,\endgroup,\endinput,\endtrivlist,\expandafter}
% \DoNotIndex{\filedate,\fileversion}
% \DoNotIndex{\gdef,\global}
% \DoNotIndex{\hbox,\hskip}
% \DoNotIndex{\ignorespaces,\input,\itshape,\item}
% \DoNotIndex{\labelsep,\let,\llap}
% \DoNotIndex{\makeatletter}
% \DoNotIndex{\noexpand}
% \DoNotIndex{\refstepcounter,\relax,\rlap}
% \DoNotIndex{\setlength,\slshape,\space,\string,\strut}
% \DoNotIndex{\the,\trivlist,\typeout}
% \DoNotIndex{\vbox}
% \DoNotIndex{\xdef}
%
% \def\deflabel#1{\ttfamily #1\hfill}
% \def\deflist#1{\begin{list}{}{\settowidth\labelwidth{\ttfamily #1}%
%                               \setlength\leftmargin\labelwidth
%                               \addtolength\leftmargin\labelsep
%                               \let\makelabel\deflabel}}
% \def\enddeflist{\end{list}}
% \def\envfont{\normalfont\ttfamily}
%
% \MakeShortVerb{\|}
%
% \newcommand{\TUB}{{\sl TUGboat\/}}
%
% \GetFileInfo{theorem.dtx}
% \title{An Extension of the
%        \LaTeX{} theorem environment\thanks{This file
%        has version number \fileversion, last
%        revised \filedate.}}
% \author{Frank Mittelbach}
%
% \MaintainedByLaTeXTeam{tools}
% \maketitle
%
% \begin{abstract}
%    The macros described in this paper yield an extension of the
%    \LaTeX{} theorem mechanism. It is designed to satisfy the
%    different requirements of various journals. Thus, the
%    layout of the ``theorems'' can be manipulated by determining a
%    ``style''. This article describes not only the use, but
%    also the definition, of the necessary macros.
% \end{abstract}
%
% \section*{Preface to version 2.2}
%
% For \LaTeXe{} this package did not need any fundamental changes. I
% only modified the messages generated so that theorem layout styles
% will show up with the |\listfiles| command and cleaned the section
% on the New Font Selection Scheme since this is now included in
% \LaTeX.
%
% \section*{Preface to version 2.1}
%
% This version is identical to 2.0g described in \TUB\ 10\#3 except
% for some internal defaults which are now set depending on the used
% font selection scheme.
%
% This was done to avoid unpleasant surprises if the new font selection
% scheme is in force. For further details see
% section~\ref{sec:newfonts} and \cite{bk:GMS94}.
%
% \section{Introduction}
%
% For our purposes here, ``theorems'' are labelled enunciations,
% often set off from the main text by extra space and a font change.
% Theorems, corollaries, conjectures, definitions, and remarks
% are all instances of ``theorems''.  The ``header'' of these structures
% is composed of a label (such as \textsc{Theorem} or \textsc{Remark})
% and a number which serializes an item in the sequence of items
% with the same label.
%
% Shortly after the introduction of \LaTeX{} at the Fachbereich
% Mathematik in Mainz, the desire to manipulate the layout of
% ``theorems'' arose.  In Mainz, the following two conventions came
% into general use:
% \begin{enumerate}
%   \item The number of the theorem is shown in the margin.
%   \item There is a line break at
%         the end of the theorem header.
% \end{enumerate}
%
% Additionally, some journals require different formats which depend on
% the ``sort of theorem'': e.g.\ often remarks and definitions are set
% in |\upshape|, while |\itshape| is employed for main theorems.
%
% Confronted with these requirements, a theorem environment was
% developed in Mainz which allows separate determination of the
% layout of the ``theorems sets'', comparable to |\pagestyle|.
%
%
% \section{The user interface}
% \subsection{Defining new theorem sets}
%
% \DescribeMacro\newtheorem
% As in the original \LaTeX{} version, the command |\newtheorem|
% defines a new ``theorem set'' or ``theorem-like structure''.
% Two required arguments name the new environment and give the text
% to be typeset with each instance of the new ``set'', while
% an optional argument determines how
% the ``set'' is enumerated:
% \begin{description}
%    \item[\ttfamily \bslash newtheorem\{foo\}\{bar\}]
%       The theorem set {\envfont foo} (whose name is \texttt{bar})
%       uses its own counter.
%    \item[\ttfamily \bslash newtheorem\{foo2\}{[foo]}\{bar2\}]
%       The theorem set {\envfont foo2} (printed name \texttt{bar2})
%       uses the same counter as the theorem set \texttt{foo}.
%    \item[\ttfamily \bslash newtheorem\{foo3\}\{bar3\}{[section]}]
%       The theorem set {\envfont foo3} (printed name \texttt{bar3}) is
%       enumerated within the counter \texttt{section}, i.e.\ with every
%       new |\section| the enumeration begins again with 1, and
%       the enumeration is composed from the section-number and the
%       theorem counter itself.
% \end{description}
%
% \DescribeMacro\theoremstyle
% Additionally, the command |\theoremstyle| can define the layout
% of various, or all, theorem sets.  It should be noted that any
% theorem set defined by |\newtheorem| is typeset in the
% |\theoremstyle| that is current at the time of the definition.
% Thus, the following
% \begin{verbatim}
%    \theoremstyle{break}        \newtheorem{Cor}{Corollary}
%    \theoremstyle{plain}        \newtheorem{Exa}{Example}[section]
%\end{verbatim}                             ^^A No blank at line start
% leads to the result that the set {\envfont Cor} is formatted in the
% style \texttt{break}, while the set {\envfont Exa} and all the
% following ones are formatted in the style \texttt{plain}, unless
% another |\theoremstyle| follows.  Since the definitions installed by
% |\newtheorem| are global, one also can limit |\theoremstyle| locally
% by grouping braces.
%
% \DescribeMacro\theorembodyfont
% The choice of the font for the theorem body is completely
% independent of the chosen |\theoremstyle|; this has proven to
% be very advantageous.  For example,
% \begin{verbatim}
%   {\theorembodyfont{\upshape}          \newtheorem{Rem}{Remark}}
%\end{verbatim}
% defines a theorem set {\envfont Rem}, which will be set in
% |\upshape| in the current layout (which in our example is
% \texttt{plain}).  As with |\theoremstyle|, the |\theorembodyfont|
% chosen is that current at the time of |\newtheorem|.  If
% |\theorembodyfont| is not specified or one defines
% |\theorembodyfont{}|, then the font used will be that defined by the
% |\theoremstyle|.
%
% \DescribeMacro\theoremheaderfont It is also possible to customize
% the font used for the theorem headers.  This is, however, a global
% declaration, and therefore there should be at most one
% |\theoremheaderfont| declaration in the preamble.\footnote{If it is
% actually necessary to have different header fonts, one has to define
% new theorem styles (substituting the desired font) or specify the
% information directly in the \texttt{\bslash newtheorem} declaration
% (the unclean variant).}
%
% \DescribeMacro\theorempreskipamount
% \DescribeMacro\theorempostskipamount Two additional parameters
% affect the vertical space around the theorem environments:
% |\theorempreskipamount| and |\theorempostskipamount| define,
% respectively, the spacing before and after such an environment.
% These parameters apply for all theorem sets and can be manipulated
% with the ordinary length macros.  They are rubber lengths
% (`\textsf{skips}'), and therefore can contain \texttt{plus} and
% \texttt{minus} parts.
%
% Since the definition of theorem sets should---most sensibly---be
% placed in the preamble, we only allow installation there.
% It is therefore possible to release the memory used here after
% |\begin{document}|, in order to make room for other
% applications.
%
%
% \subsection{Existing theorem styles}
%
% The following theorem styles exist to date:
% \begin{deflist}{marginbreak}
%    \item[plain]
%       This theorem style emulates the original \LaTeX{} definition,
%       except that additionally the parameters
%       |\theorem...skipamount| are used.
%    \item[break]
%       In this style, the theorem header is followed by a line
%       break.
%    \item[marginbreak]
%       The theorem number is set in the margin, and there is a line
%       break as in \texttt{break}.
%    \item[changebreak]
%       Like \texttt{break}, but with header number and text
%       interchanged.
%    \item[change]
%       Header number and text are interchanged, without a line break.
%    \item[margin]
%       The number is set in the left margin, without a line break.
% \end{deflist}
% All styles (except \texttt{plain}) select |\slshape| as the default
% |\theorembodyfont|.
%
%
% \subsection{Examples}
%
% Given the above theorem sets {\envfont Cor}, {\envfont Exa}
% and {\envfont Rem},
% suppose that the preamble also contains the declarations:
% \begin{verbatim}
%    \theoremstyle{marginbreak}   \newtheorem{Lem}[Cor]{Lemma}
%    \theoremstyle{change}
%    \theorembodyfont{\itshape}        \newtheorem{Def}[Cor]{Definition}
%
%    \theoremheaderfont{\scshape}
%\end{verbatim}
% Then the following are some typical examples of the typeset output
% resulting from their use.
%
% \begin{Cor}
%    This is a sentence typeset in the theorem environment {\envfont
%    Cor}.
% \end{Cor}
%
% \begin{Exa}
%    This is a sentence typeset in the theorem environment {\envfont
%    Exa}.
% \end{Exa}
%
% \begin{Rem}
%    This is a sentence typeset in the theorem environment {\envfont
%    Rem}.
% \end{Rem}
%
% \begin{Lem}[Ben User]
%    This is a sentence typeset in the theorem environment {\envfont
%    Lem}.
% \end{Lem}
%
% \begin{Def}[Very impressive Definition]
%    This is a sentence typeset in the theorem environment {\envfont
%    Def}.
% \end{Def}
%
% The last two examples show the effect of the optional argument to
% a theorem environment (it is the text typeset in parentheses).
%
%
% \section{Special Considerations}
% \label{sec:newfonts}
%
% Theoremheader and body are implemented as a unit. This means that
% the |\theoremheaderfont|  will inherit characteristics of the
% |\theorembodyfont| in \LaTeXe. Thus, if for example
% |\theorembodyfont| is |\itshape| and |\theoremheaderfont|
% is |\bfseries| the font selected for the header will have the
% characteristics `bold extended italic'. If this is not desired one
% should set the |\theoremheaderfont| to something like
% \begin{verbatim}
%      \theoremheaderfont{\normalfont\bfseries}
%\end{verbatim}
% i.e.\ supplying all necessary font information explicitly.
%
%
%
% \section{Acknowledgements}
%
% The publication of this set of macros was only possible with the
% help of Christina Busse (translating the manuscript into English),
% Joachim Pense (playing the r\^ole of typist), Chris Rowley
% (looking everything over) and many others providing useful
% suggestions.
%
% \MaybeStop{%
%    \begin{thebibliography}{9}
%       \bibitem{bk:GMS94} \textsc{M.~Goossens}, \textsc{F.~Mittelbach}
%         and \textsc{A.~Samarin}.
%         \newblock The \LaTeX{} Companion.
%         \newblock
%         Addison-Wesley, Reading, Massachusetts, 1994.
%       \bibitem{latex:source} \textsc{Lamport, Leslie.} \newblock
%                               \texttt{latex.tex}, version 2.09, date
%                               Feb. 1990.
%    \end{thebibliography}
%    \PrintIndex
%    \PrintChanges}
%
%
% \section{The documentation driver file}
%
% The next bit of code contains the documentation driver file for
% \TeX{}, i.e., the file that will produce the documentation you are
% currently reading. It will be extracted from this file by the
% \texttt{docstrip} program. Since it is the first code in the file
% one can alternatively process this file directly with \LaTeXe{} to
% obtain the documentation.
%
%    \begin{macrocode}
%<*driver>
\documentclass{ltxdoc}

\usepackage{theorem}

%<+driver>% The next few lines define theorem sets which are used
%<+driver>% in the example section of the documentation.

\theoremstyle{break}       \newtheorem{Cor}{Corollary}
\theoremstyle{plain}       \newtheorem{Exa}{Example}[section]
{\theorembodyfont{\upshape}\newtheorem{Rem}{Remark}}
\theoremstyle{marginbreak} \newtheorem{Lem}[Cor]{Lemma}
\theoremstyle{change}
\theorembodyfont{\itshape} \newtheorem{Def}[Cor]{Definition}

\theoremheaderfont{\scshape}

\RecordChanges

\begin{document}
   \DocInput{theorem.dtx}
\end{document}
%</driver>
%    \end{macrocode}
%
% \section{Definition of the Macros}
%
% If the file has been loaded before, we abort immediately. If not the
% package announces itself (this is actually done at the very top of
% the file---the way it is done isn't good style so don't copy it).
% \changes{v2.0e}{89/07/19}{Spaces removed from `typeout'.}
% \changes{v2.2b}{95/11/19}{Announce moved to top of file}
%    \begin{macrocode}
%<*package>
%\@ifundefined{theorem@style}{}{\endinput}
%\def\FMithmInfo{1995/11/19 v2.2b Theorem extension package (FMi)}
%\ProvidesPackage{theorem}[\FMithmInfo]
%    \end{macrocode}
%
%
% \subsection{Definition of theorem styles and fonts}
%
% All the definitions in this file are done globally to allow
% inputting this file inside a group.
%
% \begin{macro}{\theoremstyle}
%    Before a theorem style can be installed, the chosen style must be
%    known. For that reason, we must test to see that
%    |\th@|\meta{style} is known or, more precisely, that it is
%    different from |\relax|. If the style is not known then
%    |\th@plain| is used.
%    \begin{macrocode}
\gdef\theoremstyle#1{%
   \@ifundefined{th@#1}{\@warning
          {Unknown theoremstyle `#1'. Using `plain'}%
          \theorem@style{plain}}%
%    \end{macrocode}
%    We save the theorem style to be used in the token register
%    |\theorem@style|.
%    \begin{macrocode}
      {\theorem@style{#1}}%
%    \end{macrocode}
%    Now we ``evaluate'' the theorem style: this means, we call the
%    macro |\th@|\meta{style} which will activate the relevant
%    definitions which are contained in a separate file.  This is done
%    in a group to suppress changes to the current font.  This could
%    otherwise pose problems together with the new font selection
%    scheme\footnote{When I printed the original article using the new
%    font selection scheme I ended with a document with slanted
%    typefaces (text headings and all) simply because one of the
%    theorem styles used \texttt{\bslash sl} at toplevel.} if the
%    |\th@|\meta{style} is evaluated a second time.
% \changes{v2.1a}{90/02/15}{Group added around `th@..}
% \changes{v2.1a}{90/02/15}{Reseting of `..@begintheorem removed.}
%    \begin{macrocode}
      \begingroup
        \csname th@\the\theorem@style \endcsname
      \endgroup}
%    \end{macrocode}
% \end{macro}
%
%
% \begin{macro}{\@begintheorem}
% \changes{v2.1a}{90/02/15}{Relaxed at toplevel.}
% \begin{macro}{\@opargbegintheorem}
% \changes{v2.1a}{90/02/15}{Relaxed at toplevel.}
%    We reset |\@begintheorem| and |\@opargbegintheorem| to
%    |\relax| since these commands are no longer necessary at
%    toplevel. This will save a few tokens.
%    \begin{macrocode}
\global\let\@begintheorem\relax
\global\let\@opargbegintheorem\relax
%    \end{macrocode}
% \end{macro}
% \end{macro}
%
% \begin{macro}{\theorem@style}
%    Obviously the \textsf{token} register used above has to be
%    allocated. To assure the utmost compatibility with the
%    original \LaTeX{} definition, we set the default theorem style to
%    \texttt{plain}, which implements the usual \LaTeX{} convention.
% \changes{v2.0f}{89/07/20}{Unnecessary `global' removed.}
%    \begin{macrocode}
\newtoks\theorem@style
\global\theorem@style{plain}
%    \end{macrocode}
% \end{macro}
%
% \begin{macro}{\theorembodyfont}
% \begin{macro}{\theorem@bodyfont}
%    If the |\theorembodyfont| is set by the user then it should not
%    interact with the default font set in the theorem style. When the
%    new font selection is in force this may happen if, for example,
%    the default is |\itshape| and the new |\theorembodyfont| is
%    |\sffamily|. So we add a |\reset@font| command in front of the
%    user definition.
% \changes{v2.0f}{89/07/20}{Unnecessary `global' removed.}
% \changes{v2.0g}{89/09/19}{`theoremfont to `theorembodyfont.}
% \changes{v2.1c}{92/06/22}{`reset@font added and logic changed.}
% \changes{v2.1d}{92/10/28}{Reset to empty if arg is empty}
%    \begin{macrocode}
\gdef\theorembodyfont#1{%
%    \end{macrocode}
%    We check if the argument supplied is empty and if so put nothing
%    into the |\theorem@bodyfont| token register to allow for
%    |\theorembodyfont{}| as a mean of using the default of the
%    current |\theoremstyle|.
%    \begin{macrocode}
   \def\@tempa{#1}%
   \ifx\@tempa\@empty
    \theorem@bodyfont{}%
   \else
    \theorem@bodyfont{\reset@font#1}%
   \fi
   }
\newtoks\theorem@bodyfont
\global\theorem@bodyfont{}
%    \end{macrocode}
% \end{macro}
% \end{macro}
%
%
% \begin{macro}{\theoremheaderfont}
% \changes{v2.0g}{89/09/19}{Macro added.}
%    The font for the theorem headers is handled differently because
%    this definition applies to all theorem styles.
%    \begin{macrocode}
\gdef\theoremheaderfont#1{\gdef\theorem@headerfont{#1}%
%    \end{macrocode}
%    After using the macro once it is redefined to produce an error
%    message.
%    \begin{macrocode}
       \gdef\theoremheaderfont##1{%
        \typeout{\string\theoremheaderfont\space should be used
                 only once.}}}
%    \end{macrocode}
% \end{macro}
%
% \begin{macro}{\theorem@headerfont}
% \changes{v2.0g}{89/09/19}{Macro added.}
%    To set the |\theorem@headerfont| default we first test if
%    the new fontselection scheme is in force.
% \changes{v2.1a}{90/02/15}{Support for new font selection.}
%    \begin{macrocode}
\ifx\upshape\undefined
%    \end{macrocode}
%    If not we define it to expand into |\bfseries|. We don't use
%    |\let| just in case a following style option redefines this
%    macro.
%    \begin{macrocode}
\gdef\theorem@headerfont{\bfseries}
%    \end{macrocode}
%    Otherwise we reset the current shape before calling |\bfseries|.
%    \begin{macrocode}
\else \gdef\theorem@headerfont{\normalfont\bfseries}\fi
%    \end{macrocode}
% \end{macro}
%
% \begin{macro}{\th@plain}
% \begin{macro}{\th@break}
% \begin{macro}{\th@marginbreak}
% \begin{macro}{\th@changebreak}
% \begin{macro}{\th@change}
% \begin{macro}{\th@margin}
%    The different styles are defined in macros such as
%    |\th@plain|.  Since memory space is precious in
%    ``non-Big-versions'', we have to avoid offering too many unused
%    definitions.  Therefore we define these styles in separate files
%    that can be loaded on demand.  Thus the commands themselves only
%    load these files. We use |\@input@|, a \LaTeXe{} internal command
%    that ensures that the file will be listed with |\listfiles|.
% \changes{v2.2a}{94/02/02}{Use `@input@ to load theorem layout files}
%    \begin{macrocode}
\gdef\th@plain{\@input@{thp.sty}}
\gdef\th@break{\@input@{thb.sty}}
\gdef\th@marginbreak{\@input@{thmb.sty}}
\gdef\th@changebreak{\@input@{thcb.sty}}
\gdef\th@change{\@input@{thc.sty}}
\gdef\th@margin{\@input@{thm.sty}}
%    \end{macrocode}
%    This list will be expanded when new styles become
%    available.  For testing,  just append new theorem substyles
%    as document options.
% \end{macro}
% \end{macro}
% \end{macro}
% \end{macro}
% \end{macro}
% \end{macro}
%
%
%
% \subsection{Definition of a new theorem set}
%
% As already pointed out, a new theorem environment can be defined in
% three different ways:
% \begin{verbatim}
%    \newtheorem{Lem}{Lemma}
%    \newtheorem{Lem}{Lemma}[section]
%    \newtheorem{Lem}[Theorem]{Lemma}
%\end{verbatim}
% The function of the macro |\newtheorem| is to recognize these cases
% and then to branch into one of the three macros |\@ynthm|, |\@xnthm|
% or |\@othm|.  This mechanism is adopted unchanged from
% \cite{latex:source}; the essential point here is that, for example,
% in the second case, the arguments \texttt{Lem}, \texttt{Lemma} and
% \texttt{section} are passed over to the macro |\@xnthm|.
%
% We inspect this case first because the others present fewer problems,
% and thus are easily derived from this one.
%
% \begin{macro}{\@xnthm}
%    For our example arguments, the macro |\@xnthm| must fulfill
%    the following:
%    \begin{itemize}
%       \item
%          Define a new \LaTeX-counter `Lem'
%       \item
%          reset this counter within a |\section|
%       \item
%          define the macro |\theLem|
%       \item
%          define the environment macros |\Lem| and
%          |\endLem| using the current
%          |\theoremstyle| and |\theorem@bodyfont|.
%    \end{itemize}
%    Obviously, all this should happen only if the first argument
%    of |\@xnthm| (i.e.\ \texttt{Lem} in our example) is chosen so
%    as not to conflict with any previously defined commands or
%    environments.  This test is performed by the \LaTeX{} macro
%    |\@ifdefinable|.
%    \begin{macrocode}
\gdef\@xnthm#1#2[#3]{\expandafter\@ifdefinable\csname #1\endcsname
%    \end{macrocode}
%    Therefore, the first argument of |\@ifdefinable| is the
%    expansion (in the example, |\Lem|) of
%    |\csname||#1\endcsname|.  The second argument is
%    executed only if the test has been completed successfully.
%    \begin{macrocode}
   {%
%    \end{macrocode}
%    Now we define the new counter. The names of the \LaTeX{}
%    macros employed should speak for themselves:
% \changes{v2.2b}{1995/10/07}{Use \cs{@newctr}}
% \changes{v2.2c}{1995/11/23}{Use \cs{@newctr} properly.}
%    \begin{macrocode}
    \@definecounter{#1}\@newctr{#1}[#3]%
%    \end{macrocode}
%    Using |\@newctr| will give a proper error message if the counter
%    in |#3| is not defined.
%    In defining `|\theLem|' we must generate the desired macro
%    name by use of |\expandafter| and |\csname|.
%    \begin{macrocode}
    \expandafter\xdef\csname the#1\endcsname
%    \end{macrocode}
%    An |\xdef| is used in order to make the definition global,
%    and to ensure that it contains the replacement texts of
%    |\@thmcountersep| and |\@thmcounter|.\footnote{These
%    two macros can be defined by the document style.  Their default
%    values produce a `\texttt{.}' as separation and an arabic
%    representation of the number.} However, not everything should be
%    expanded. For example, it saves space to use |\thesection|
%    instead of its---at times---lengthy expansion.
%    \begin{macrocode}
      {\expandafter \noexpand \csname the#3\endcsname
       \@thmcountersep \@thmcounter{#1}}%
%    \end{macrocode}
%    Thus with the defaults of \LaTeX{}, |\theLem| would be
%    replaced by the command sequence |\thesection.\arabic{Lem}|.
%
%    We will now look at the definition of the macro which is executed
%    at the beginning of the actual environment (in our example this
%    macro is |\Lem|). It should be noted that we use an
%    ``|\expandafter| trick'' to expand only certain parts of the
%    replacement text at the time of the definition.
%    \begin{macrocode}
    \def\@tempa{\global\@namedef{#1}}%
    \expandafter \@tempa \expandafter{%
%    \end{macrocode}
%    First, the macro that contains the current definitions of
%    |\@begintheorem| and |\@opargtheorem| should be called
%    up.  The name of this macro---as is already known---has the form
%    |\th@|\meta{theorem\ style}; therefore, it must be called by
%    \begin{macrocode}
      \csname th@\the \theorem@style
%    \end{macrocode}
%    In addition the default theorem font should be changeable, i.e.\
%    we have to insert the contents of |\theorem@bodyfont|. For
%    that reason, we expand even further, beyond |\endcsname|,
%    and thus insert the contents of the \textsf{token} register
%    |\theorem@bodyfont| in the replacement text.
%    \begin{macrocode}
            \expandafter \endcsname \the \theorem@bodyfont
%    \end{macrocode}
%    Now it is time to call the macro |\@thm| which takes over
%    the further processing.  It has two arguments: the current
%    counter name (in our example, \texttt{Lem}), and the text of the
%    label (in our example, \texttt{Lemma}).
%    \begin{macrocode}
     \@thm{#1}{#2}}%
%    \end{macrocode}
%    With this, the `sub-definition' is complete.  The macro
%    |\@endtheorem| ends a theorem environment and is, so far, nothing
%    but an |\endtrivlist|.  (Hence it is defined globally, and not
%    within the theorem styles.\footnote{This has to be changed as
%    soon as theorem styles that change \texttt{\bslash @endtheorem}
%    exist.  In such a case, all existing styles must be changed as
%    well since they will have to reset the macro.}) Therefore, we can
%    set it equivalent to the macro that ends the theorem set (in our
%    example, |\endLem|).  However, if some day theorem styles exist
%    that do change |\@endtheorem|, we would have to use the
%    commented-out line instead.
%    \begin{macrocode}
    \global \expandafter \let \csname end#1\endcsname \@endtheorem
%   \global\@namedef{end#1}{\@endtheorem}%
%    \end{macrocode}
%    With these commands all the required definitions are employed,
%    unless the test |\@ifdefinable| has failed.  Therefore, we
%    end the second argument of this macro and with it the definition
%    of |\@xnthm|.
%    \begin{macrocode}
   }}
%    \end{macrocode}
% \end{macro}
%
% \begin{macro}{\@ynthm}
%    The definition of |\@ynthm| is completely analogous. In
%    this case the new counter that is defined is not reset within
%    another counter; thus the definition of |\the...|
%    is simplified:
%    \begin{macrocode}
\gdef\@ynthm#1#2{\expandafter\@ifdefinable\csname #1\endcsname
   {\@definecounter{#1}%
    \expandafter\xdef\csname the#1\endcsname{\@thmcounter{#1}}%
%    \end{macrocode}
%    The rest of the definition corresponds literally to that of
%    |\@xnthm|:
%    \begin{macrocode}
    \def\@tempa{\global\@namedef{#1}}\expandafter \@tempa
     \expandafter{\csname th@\the \theorem@style \expandafter
     \endcsname \the\theorem@bodyfont \@thm{#1}{#2}}%
    \global \expandafter \let \csname end#1\endcsname \@endtheorem}}
%    \end{macrocode}
% \end{macro}
%
% \begin{macro}{\@othm}
%    The definition of |\@othm| does not contain anything new.
%    \begin{macrocode}
\gdef\@othm#1[#2]#3{%
%    \end{macrocode}
% \changes{v2.2b}{95/10/07}{Build in check for counter defined}
%    We do not define a new counter but instead use one that has
%    already been defined.  Thus the only definition we need is that
%    of this pseudo-counter (i.e.\ |\the|\meta{env.\ name}).
%    First we check if |#2| corresponds to a known counter name.
%    \begin{macrocode}
  \expandafter\ifx\csname c@#2\endcsname\relax
   \@nocounterr{#2}%
  \else
   \expandafter\@ifdefinable\csname #1\endcsname
   {\expandafter \xdef \csname the#1\endcsname
     {\expandafter \noexpand \csname the#2\endcsname}%
%    \end{macrocode}
%    All other parts of the definition can be adopted from
%    |\@xnthm|. We have to remember, though, that in this case
%    the name of the current counter and the theorem label have moved
%    to the second and third arguments.
%    \begin{macrocode}
    \def\@tempa{\global\@namedef{#1}}\expandafter \@tempa
     \expandafter{\csname th@\the \theorem@style \expandafter
     \endcsname \the\theorem@bodyfont \@thm{#2}{#3}}%
    \global \expandafter \let \csname end#1\endcsname \@endtheorem}%
  \fi}
%    \end{macrocode}
% \end{macro}
%
% \subsection{Macros that are employed in a {\envfont theorem}
%    environment}
%
% \begin{macro}{\@thm}
%    The macro |\@thm| has to increase the current counter.  Then,
%    depending on whether the environment has (or does not have) an
%    optional argument, it has to branch into either |\@begintheorem|
%    or |\@opargtheorem|.
%    \begin{macrocode}
\gdef\@thm#1#2{\refstepcounter{#1}%
%    \end{macrocode}
%    Now we start a {\envfont trivlist} environment, and give
%    |\@topsep| and |\@topsepadd| the values of the \textsf{skip}
%    registers |\theorempreskipamount| and |\theorempostskipamount|.
%    The value in |\@topsep| is the vertical space that is inserted by
%    the first (and only) |\item| in our |\trivlist| whilst
%    |\@topsepadd| is inserted by |\@endparenv| at the end of that
%    {\envfont trivlist} environment.  By using these registers, we
%    obtain the desired space around a {\envfont theorem} environment.
%    \begin{macrocode}
   \trivlist
   \@topsep \theorempreskipamount               % used by first \item
   \@topsepadd \theorempostskipamount           % used by \@endparenv
%    \end{macrocode}
%    Now we have to test whether an optional argument has been given.
% \changes{v2.0d}{89/07/18}{Forgotten percent added.}
%    \begin{macrocode}
   \@ifnextchar [%
%    \end{macrocode}
%    If there is an optional argument, we will call |\@ythm|, and
%    move the arguments read back into the input stream.
%    \begin{macrocode}
   {\@ythm{#1}{#2}}%
%    \end{macrocode}
%    If not, we call |\@begintheorem|. Its first argument is the
%    name of the theorem set (hence the second argument of
%    |\@thm|). Its second argument is the macro that produces the
%    current number.
%    \begin{macrocode}
   {\@begintheorem{#2}{\csname the#1\endcsname}\ignorespaces}}
%    \end{macrocode}
% \end{macro}
%
% \begin{macro}{\@xthm}
% \begin{macro}{\@ythm}
%    Both these macros were originally called by |\@thm|.  We do
%    not need |\@xthm| anymore, hence we reset it to
%    |\relax|. The definition of |\@ythm| has not changed at
%    all from its definition in \LaTeX. In order to make the macros
%    easier to understand, we will
%    nevertheless present it (commented out).
%    \begin{macrocode}
\global\let\@xthm\relax
% \def\@ythm#1#2[#3]{\@opargbegintheorem{#2}{\csname
%           the#1\endcsname}{#3}\ignorespaces}
%</package>
%    \end{macrocode}
%    The primitive |\ignorespaces| in |\@ythm| and
%    |\@thm| is needed to remove the spaces between the
%    |\begin{...}| and the actual text.
% \end{macro}
% \end{macro}
%
%
% \subsection{Definition of the theorem substyles}
%
% As already pointed out, the theorem substyles, defined below, are
% only loaded when necessary.  Note that all these substyles, except
% \texttt{plain}, have |\slshape| as the default body font.
%
% \subsubsection{The \texttt{plain} style}
%
% \label{dok:th@plain} As the following macros use |@|, we have
% to locally set the |\catcode| of this symbol to ``letter''.
% This happens within a group, so that we do not have to worry about
% which |\catcode| that symbol had before.
%    \begin{macrocode}
%<*thp>
\begingroup \makeatletter
%    \end{macrocode}
% Since we are now within a group, we must make all definitions
% globally.  First we make sure that \texttt{theorem.sty} is loaded.
% This will allow us to use this file as a document style option
% without having to call \texttt{theorem} itself as an option.  At the
% same time, we assure that at least version 2 is loaded, since
% |\theorem@style| was not defined in earlier versions.
%    \begin{macrocode}
\@ifundefined{theorem@style}{\input{theorem.sty}}{}
\ProvidesFile{thp.sty}
             [\FMithmInfo]
%    \end{macrocode}
%
%
% \begin{macro}{\th@plain}
% \begin{macro}{\@begintheorem}
% \begin{macro}{\@opargbegintheorem}
%    |\theoremstyle{plain}| corresponds to the original
%    definition, except that the distances to the surrounding text are
%    determined by the parameters |\theorempreskipamount| and
%    |\theorempostskipamount|.
%    First we set the default body font.
%    \begin{macrocode}
\gdef\th@plain{\normalfont\itshape
%    \end{macrocode}
%    Then we define |\@begintheorem| and
%    |\@opargbegintheorem|. These two macros define how the
%    header of a theorem is typeset.  |\@opargbegintheorem| will
%    be called if a \textsf{theorem} environment with an optional
%    argument is encountered; otherwise, the header is constructed by
%    calling |\@begintheorem|.  If one of these macros is
%    executed, we are within a \textsf{trivlist} environment started by
%    |\@thm|.  So the theorem header is produced with an
%    |\item| command.
%
%    Instead of specifying the header font directly, all standard
%    theorem styles use the |\theorem@headerfont| macro to allow
%    customization.  The extra space (|\labelsep|) is necessary
%    because of problems in the \textsf{trivlist} environment.
%    \begin{macrocode}
  \def\@begintheorem##1##2{%
        \item[\hskip\labelsep \theorem@headerfont ##1\ ##2]}%
%    \end{macrocode}
%    The definition of |\@opargbegintheorem| is completely
%    analogous. The only difference is the fact that there exists a
%    third argument (which is the optional parameter of the
%    environment and contains additional information about the
%    theorem).  Customarily we enclose it in parentheses.
%    \begin{macrocode}
\def\@opargbegintheorem##1##2##3{%
   \item[\hskip\labelsep \theorem@headerfont ##1\ ##2\ (##3)]}}
%    \end{macrocode}
% \end{macro}
% \end{macro}
% \end{macro}
%
% \noindent
% We conclude with an |\endgroup| to restore the
% |\catcode| of |@|.
%    \begin{macrocode}
\endgroup
%</thp>
%    \end{macrocode}
%
%
% \subsubsection{The \texttt{break} style}
%
% This style option is stored in the file |thb.sty|.
% For the next two lines see the documentation for |\th@plain| on
% page~\pageref{dok:th@plain}.
%    \begin{macrocode}
%<*thb>
\begingroup \makeatletter
\@ifundefined{theorem@style}{\input{theorem.sty}}{}
\ProvidesFile{thb.sty}
             [\FMithmInfo]
%    \end{macrocode}
%
% \begin{macro}{\th@break}
%    |\theoremstyle{break}| produces a line break after the name
%    of the theorem.  The font is |\slshape|.  Hence, we define
%    |\th@break| as follows:
%    \begin{macrocode}
\gdef\th@break{\normalfont\slshape
  \def\@begintheorem##1##2{\item[%
%    \end{macrocode}
%    We run into the following problem: it is not possible to create
%    the header with |\item[|\meta{title}|]| and then start
%    a new line by, for example, |\mbox{}\\|. Such a definition
%    will fail whenever a list environment follows immediately.  With
%    the above construction, the |\mbox{}| causes the switch
%    |@inlabel| (cf.\ definition of |\list| and
%    |\trivlist| in \cite{latex:source}) to be set to \texttt{false}
%    and so the following list will insert additional vertical
%    space (|\topskip|). This is quite annoying. Therefore, we
%    create the line break within the |\item|. In order to ensure
%    that the text will begin at the proper position in the following
%    line, we simply pretend that the label does not take any
%    room.\footnote{This will lead to problems whenever very high
%    symbols occurring in the line tower into the
%    heading. So, something else has to be done here sometime.}
%    \begin{macrocode}
       \rlap{\vbox{\hbox{\hskip \labelsep\theorem@headerfont ##1\ ##2}%
                   \hbox{\strut}}}]}%
%    \end{macrocode}
%    Again, the definition of |\@opargbegintheorem| is
%    completely analogous.
%    \begin{macrocode}
\def\@opargbegintheorem##1##2##3{%
  \item[\rlap{\vbox{\hbox{\hskip \labelsep \theorem@headerfont
                     ##1\ ##2\ (##3)}%
                    \hbox{\strut}}}]}}
\endgroup
%</thb>
%    \end{macrocode}
% \end{macro}
%
%
% \subsubsection{The \texttt{changebreak} style}
%
%    \begin{macrocode}
% This style option is stored in the file |thcb.sty|.
%    \begin{macrocode}
%<*thcb>
\begingroup \makeatletter
\@ifundefined{theorem@style}{\input{theorem.sty}}{}
\ProvidesFile{thcb.sty}
             [\FMithmInfo]
%    \end{macrocode}
%
% \begin{macro}{\th@changebreak}
%    The \texttt{change-break} theorem style is like \texttt{break}
%    but with interchange of theorem name and theorem number.  Thus we
%    define |\th@changebreak| as follows:
%    \begin{macrocode}
\gdef\th@changebreak{\normalfont\slshape
  \def\@begintheorem##1##2{\item
       [\rlap{\vbox{\hbox{\hskip\labelsep \theorem@headerfont ##2\ ##1}%
                    \hbox{\strut}}}]}%
\def\@opargbegintheorem##1##2##3{%
  \item[\rlap{\vbox{\hbox{\hskip\labelsep \theorem@headerfont
                     ##2\ ##1\ (##3)}%
                    \hbox{\strut}}}]}}
\endgroup
%</thcb>
%    \end{macrocode}
% \end{macro}
%
%
% \subsubsection{The \texttt{change} style}
%
% This style option is stored in the file |thc.sty|.
%    \begin{macrocode}
%<*thc>
\begingroup \makeatletter
\@ifundefined{theorem@style}{\input{theorem.sty}}{}
\ProvidesFile{thc.sty}
             [\FMithmInfo]
%    \end{macrocode}
%
% \begin{macro}{\th@change}
%    The \texttt{change} theorem style corresponds to the
%    \texttt{change break} style without a linebreak after the header.
%    To say it in another way, it's the same as the \texttt{plain}
%    style but with number and name interchanged and |\slshape| as the
%    default font.
%    \begin{macrocode}
\gdef\th@change{\normalfont\slshape
  \def\@begintheorem##1##2{\item
       [\hskip\labelsep \theorem@headerfont ##2\ ##1]}%
\def\@opargbegintheorem##1##2##3{%
  \item[\hskip\labelsep \theorem@headerfont ##2\ ##1\ (##3)]}}
\endgroup
%</thc>
%    \end{macrocode}
% \end{macro}
%
%
% \subsubsection{The \texttt{marginbreak} style}
%
% This style option is the one used most often at Mainz.  It is saved
% in the file |thmb.sty|.
%    \begin{macrocode}
%<*thmb>
\begingroup \makeatletter
\@ifundefined{theorem@style}{\input{theorem.sty}}{}
\ProvidesFile{thmb.sty}
             [\FMithmInfo]
%    \end{macrocode}
%
% \begin{macro}{\th@marginbreak}
%    The \texttt{margin break} style is nearly the same as the
%    \texttt{change break} style. The only difference is the placement
%    of the theorem number.  We use |\llap| to place it in the left
%    margin.
%
%    In this style |\labelsep| denotes the separation between the
%    number and the text.
%    \begin{macrocode}
\gdef\th@marginbreak{\normalfont\slshape
  \def\@begintheorem##1##2{\item
       [\rlap{\vbox{\theorem@headerfont
                    \hbox{\llap{##2}\hskip\labelsep ##1}%
                    \hbox{\strut}}}]}%
\def\@opargbegintheorem##1##2##3{%
  \item[\rlap{\vbox{\theorem@headerfont
                    \hbox{\llap{##2}\hskip\labelsep ##1\ (##3)}%
                    \hbox{\strut}}}]}}
\endgroup
%</thmb>
%    \end{macrocode}
% \changes{v2.1c}{90/02/17}{`endgroup outside macrocode env.}
% \end{macro}
%
%
%
% \subsubsection{The \texttt{margin} style}
%
% This style option is stored in the file |thm.sty|.
%    \begin{macrocode}
%<*thm>
\begingroup \makeatletter
\@ifundefined{theorem@style}{\input{theorem.sty}}{}
\ProvidesFile{thm.sty}
             [\FMithmInfo]
%    \end{macrocode}
%
% \begin{macro}{\th@margin}
%    Again this is only a variant of the theorem styles described
%    above without any new ideas.
%    \begin{macrocode}
\gdef\th@margin{\normalfont\slshape
  \def\@begintheorem##1##2{\item
       [\theorem@headerfont \llap{##2}\hskip\labelsep ##1]}%
\def\@opargbegintheorem##1##2##3{%
  \item[\theorem@headerfont \llap{##2}\hskip\labelsep ##1\ (##3)]}}
\endgroup
%</thm>
%    \end{macrocode}
% \end{macro}
%
%
%
%
% \subsection{Final Definitions}
%
% \begin{macro}{\theorempreskipamount}
% \begin{macro}{\theorempostskipamount}
%    The \textsf{skip} parameters that regulate the vertical empty space
%    before and after the {\envfont theorem} environment have
%    to be allocated as well.
% \changes{v2.0f}{89/07/20}{Unnecessary `global' removed.}
%    \begin{macrocode}
%<*package>
\newskip\theorempreskipamount
\newskip\theorempostskipamount
%    \end{macrocode}
%    Since we have used the same values for all theorem sets, we now can
%    assign them.
%    \begin{macrocode}
\global\setlength\theorempreskipamount{12pt plus 5pt minus 3pt}
\global\setlength\theorempostskipamount{8pt plus 3pt minus 1.5pt}
%    \end{macrocode}
% \end{macro}
% \end{macro}
%
%
% \begin{macro}{\@endtheorem}
%    The same holds for the macro |\@endtheorem|, which ends a
%    {\envfont theorem} environment. Since it is the same for all
%    theorem sets, it is removed from the macros |\th@|\meta{style}.
%    It simply ends the {\envfont trivlist} environment, which was
%    begun in |\@thm|.
%    \begin{macrocode}
\global\let\@endtheorem=\endtrivlist
%    \end{macrocode}
% \end{macro}
%
%
%
% \begin{macro}{\@preamblecmds}
%    All macros defined above are to be used only in the preamble.
%    Therefore, we insert them in |\@preamblecmds| which will disable
%    them at begin document. This is done by the internal \LaTeXe{}
%    command |\@onlypreamble|.
% \changes{v2.0g}{89/09/19}{`theoremheaderfont and `theorembodyfont
%    added.}
% \changes{v2.2a}{94/02/03}{Use \LaTeXe{} interface}
%    \begin{macrocode}
\@onlypreamble\@xnthm
\@onlypreamble\@ynthm
\@onlypreamble\@othm
\@onlypreamble\newtheorem
\@onlypreamble\theoremstyle
\@onlypreamble\theorembodyfont
\@onlypreamble\theoremheaderfont
%    \end{macrocode}
% \end{macro}
%
% Finally we declare the \texttt{plain} theorem style to be the default.
% \changes{v2.1b}{90/06/04}{Fixed missing `item by declaring default
%    style.}
%    \begin{macrocode}
\theoremstyle{plain}
%</package>
%    \end{macrocode}
%
% \Finale

\endinput
}{}
\ProvidesFile{thp.sty}
             [\FMithmInfo]
%    \end{macrocode}
%
%
% \begin{macro}{\th@plain}
% \begin{macro}{\@begintheorem}
% \begin{macro}{\@opargbegintheorem}
%    |\theoremstyle{plain}| corresponds to the original
%    definition, except that the distances to the surrounding text are
%    determined by the parameters |\theorempreskipamount| and
%    |\theorempostskipamount|.
%    First we set the default body font.
%    \begin{macrocode}
\gdef\th@plain{\normalfont\itshape
%    \end{macrocode}
%    Then we define |\@begintheorem| and
%    |\@opargbegintheorem|. These two macros define how the
%    header of a theorem is typeset.  |\@opargbegintheorem| will
%    be called if a \textsf{theorem} environment with an optional
%    argument is encountered; otherwise, the header is constructed by
%    calling |\@begintheorem|.  If one of these macros is
%    executed, we are within a \textsf{trivlist} environment started by
%    |\@thm|.  So the theorem header is produced with an
%    |\item| command.
%
%    Instead of specifying the header font directly, all standard
%    theorem styles use the |\theorem@headerfont| macro to allow
%    customization.  The extra space (|\labelsep|) is necessary
%    because of problems in the \textsf{trivlist} environment.
%    \begin{macrocode}
  \def\@begintheorem##1##2{%
        \item[\hskip\labelsep \theorem@headerfont ##1\ ##2]}%
%    \end{macrocode}
%    The definition of |\@opargbegintheorem| is completely
%    analogous. The only difference is the fact that there exists a
%    third argument (which is the optional parameter of the
%    environment and contains additional information about the
%    theorem).  Customarily we enclose it in parentheses.
%    \begin{macrocode}
\def\@opargbegintheorem##1##2##3{%
   \item[\hskip\labelsep \theorem@headerfont ##1\ ##2\ (##3)]}}
%    \end{macrocode}
% \end{macro}
% \end{macro}
% \end{macro}
%
% \noindent
% We conclude with an |\endgroup| to restore the
% |\catcode| of |@|.
%    \begin{macrocode}
\endgroup
%</thp>
%    \end{macrocode}
%
%
% \subsubsection{The \texttt{break} style}
%
% This style option is stored in the file |thb.sty|.
% For the next two lines see the documentation for |\th@plain| on
% page~\pageref{dok:th@plain}.
%    \begin{macrocode}
%<*thb>
\begingroup \makeatletter
\@ifundefined{theorem@style}{% \iffalse meta-comment
%
% Copyright (C) 1993-2024
%
% The LaTeX Project and any individual authors listed elsewhere
% in this file.
%
% This file is part of the Standard LaTeX `Tools Bundle'.
% -------------------------------------------------------
%
% It may be distributed and/or modified under the
% conditions of the LaTeX Project Public License, either version 1.3c
% of this license or (at your option) any later version.
% The latest version of this license is in
%    https://www.latex-project.org/lppl.txt
% and version 1.3c or later is part of all distributions of LaTeX
% version 2005/12/01 or later.
%
% The list of all files belonging to the LaTeX `Tools Bundle' is
% given in the file `manifest.txt'.
%
% \fi
%
%%
%
%
\def\FMithmInfo{2023/07/05 v2.2c Theorem extension package (FMi)}
%
% \ProvidesFile{theorem.dtx}[\FMithmInfo]
%\iffalse   % this is a METACOMMENT !
%%
%% Package `theorem' to use with LaTeX2e
%% Copyright (C) 1989-1995 Frank Mittelbach, all rights reserved.
%<+package>\@ifundefined{theorem@style}{}{\endinput}
%<+package>\NeedsTeXFormat{LaTeX2e}[1995/06/01]
%<+package>\ProvidesPackage{theorem}
%<+package>         [\FMithmInfo]
%
% \fi
%
% \changes{v1.0}{87/05/07}{First draft.}
% \changes{v1.0a}{87/05/07}{`thistheoremstyle' added.}
% \changes{v1.1a}{87/07/29}{New concept: redefine the internal
%                          \protect\LaTeX{} commands `@xnthm' etc.}
% \changes{v1.1b}{87/07/31}{`theoremfont' used to specify font.}
% \changes{v1.2a}{88/01/08}{`@endtheorem' now globally defined.}
% \changes{v1.2b}{88/01/11}{`newtheorem' only allowed in the preamble.}
% \changes{v1.3a}{88/02/14}{`theoremfont' now a toks register.}
%
% \changes{v2.0a}{88/11/16}{New implementation and new documentation.}
% \changes{v2.0a}{88/11/16}{No layout changes but internal modifications
%                           old substyles won't work.}
% \changes{v2.0b}{89/01/30}{`global' used for all definitions.}
% \changes{v2.0c}{89/06/15}{Documentation translated to English.
%                            Thanks to LLL and Nina}
% \changes{v2.1a}{90/02/15}{Support for new font selection.}
% \changes{v2.2a}{94/02/03}{Cleanup for LaTeX2e}
%
% \DoNotIndex{\@endparenv,\@ifdefinable,\@ifnextchar,\@ifundefined}
% \DoNotIndex{\@namedef,\@nameuse,\@spaces,\@tempa,\@warning}
% \DoNotIndex{\ }
% \DoNotIndex{\begingroup,\bfseries}
% \DoNotIndex{\csname}
% \DoNotIndex{\def,\do,\docdate}
% \DoNotIndex{\endcsname,\endgroup,\endinput,\endtrivlist,\expandafter}
% \DoNotIndex{\filedate,\fileversion}
% \DoNotIndex{\gdef,\global}
% \DoNotIndex{\hbox,\hskip}
% \DoNotIndex{\ignorespaces,\input,\itshape,\item}
% \DoNotIndex{\labelsep,\let,\llap}
% \DoNotIndex{\makeatletter}
% \DoNotIndex{\noexpand}
% \DoNotIndex{\refstepcounter,\relax,\rlap}
% \DoNotIndex{\setlength,\slshape,\space,\string,\strut}
% \DoNotIndex{\the,\trivlist,\typeout}
% \DoNotIndex{\vbox}
% \DoNotIndex{\xdef}
%
% \def\deflabel#1{\ttfamily #1\hfill}
% \def\deflist#1{\begin{list}{}{\settowidth\labelwidth{\ttfamily #1}%
%                               \setlength\leftmargin\labelwidth
%                               \addtolength\leftmargin\labelsep
%                               \let\makelabel\deflabel}}
% \def\enddeflist{\end{list}}
% \def\envfont{\normalfont\ttfamily}
%
% \MakeShortVerb{\|}
%
% \newcommand{\TUB}{{\sl TUGboat\/}}
%
% \GetFileInfo{theorem.dtx}
% \title{An Extension of the
%        \LaTeX{} theorem environment\thanks{This file
%        has version number \fileversion, last
%        revised \filedate.}}
% \author{Frank Mittelbach}
%
% \MaintainedByLaTeXTeam{tools}
% \maketitle
%
% \begin{abstract}
%    The macros described in this paper yield an extension of the
%    \LaTeX{} theorem mechanism. It is designed to satisfy the
%    different requirements of various journals. Thus, the
%    layout of the ``theorems'' can be manipulated by determining a
%    ``style''. This article describes not only the use, but
%    also the definition, of the necessary macros.
% \end{abstract}
%
% \section*{Preface to version 2.2}
%
% For \LaTeXe{} this package did not need any fundamental changes. I
% only modified the messages generated so that theorem layout styles
% will show up with the |\listfiles| command and cleaned the section
% on the New Font Selection Scheme since this is now included in
% \LaTeX.
%
% \section*{Preface to version 2.1}
%
% This version is identical to 2.0g described in \TUB\ 10\#3 except
% for some internal defaults which are now set depending on the used
% font selection scheme.
%
% This was done to avoid unpleasant surprises if the new font selection
% scheme is in force. For further details see
% section~\ref{sec:newfonts} and \cite{bk:GMS94}.
%
% \section{Introduction}
%
% For our purposes here, ``theorems'' are labelled enunciations,
% often set off from the main text by extra space and a font change.
% Theorems, corollaries, conjectures, definitions, and remarks
% are all instances of ``theorems''.  The ``header'' of these structures
% is composed of a label (such as \textsc{Theorem} or \textsc{Remark})
% and a number which serializes an item in the sequence of items
% with the same label.
%
% Shortly after the introduction of \LaTeX{} at the Fachbereich
% Mathematik in Mainz, the desire to manipulate the layout of
% ``theorems'' arose.  In Mainz, the following two conventions came
% into general use:
% \begin{enumerate}
%   \item The number of the theorem is shown in the margin.
%   \item There is a line break at
%         the end of the theorem header.
% \end{enumerate}
%
% Additionally, some journals require different formats which depend on
% the ``sort of theorem'': e.g.\ often remarks and definitions are set
% in |\upshape|, while |\itshape| is employed for main theorems.
%
% Confronted with these requirements, a theorem environment was
% developed in Mainz which allows separate determination of the
% layout of the ``theorems sets'', comparable to |\pagestyle|.
%
%
% \section{The user interface}
% \subsection{Defining new theorem sets}
%
% \DescribeMacro\newtheorem
% As in the original \LaTeX{} version, the command |\newtheorem|
% defines a new ``theorem set'' or ``theorem-like structure''.
% Two required arguments name the new environment and give the text
% to be typeset with each instance of the new ``set'', while
% an optional argument determines how
% the ``set'' is enumerated:
% \begin{description}
%    \item[\ttfamily \bslash newtheorem\{foo\}\{bar\}]
%       The theorem set {\envfont foo} (whose name is \texttt{bar})
%       uses its own counter.
%    \item[\ttfamily \bslash newtheorem\{foo2\}{[foo]}\{bar2\}]
%       The theorem set {\envfont foo2} (printed name \texttt{bar2})
%       uses the same counter as the theorem set \texttt{foo}.
%    \item[\ttfamily \bslash newtheorem\{foo3\}\{bar3\}{[section]}]
%       The theorem set {\envfont foo3} (printed name \texttt{bar3}) is
%       enumerated within the counter \texttt{section}, i.e.\ with every
%       new |\section| the enumeration begins again with 1, and
%       the enumeration is composed from the section-number and the
%       theorem counter itself.
% \end{description}
%
% \DescribeMacro\theoremstyle
% Additionally, the command |\theoremstyle| can define the layout
% of various, or all, theorem sets.  It should be noted that any
% theorem set defined by |\newtheorem| is typeset in the
% |\theoremstyle| that is current at the time of the definition.
% Thus, the following
% \begin{verbatim}
%    \theoremstyle{break}        \newtheorem{Cor}{Corollary}
%    \theoremstyle{plain}        \newtheorem{Exa}{Example}[section]
%\end{verbatim}                             ^^A No blank at line start
% leads to the result that the set {\envfont Cor} is formatted in the
% style \texttt{break}, while the set {\envfont Exa} and all the
% following ones are formatted in the style \texttt{plain}, unless
% another |\theoremstyle| follows.  Since the definitions installed by
% |\newtheorem| are global, one also can limit |\theoremstyle| locally
% by grouping braces.
%
% \DescribeMacro\theorembodyfont
% The choice of the font for the theorem body is completely
% independent of the chosen |\theoremstyle|; this has proven to
% be very advantageous.  For example,
% \begin{verbatim}
%   {\theorembodyfont{\upshape}          \newtheorem{Rem}{Remark}}
%\end{verbatim}
% defines a theorem set {\envfont Rem}, which will be set in
% |\upshape| in the current layout (which in our example is
% \texttt{plain}).  As with |\theoremstyle|, the |\theorembodyfont|
% chosen is that current at the time of |\newtheorem|.  If
% |\theorembodyfont| is not specified or one defines
% |\theorembodyfont{}|, then the font used will be that defined by the
% |\theoremstyle|.
%
% \DescribeMacro\theoremheaderfont It is also possible to customize
% the font used for the theorem headers.  This is, however, a global
% declaration, and therefore there should be at most one
% |\theoremheaderfont| declaration in the preamble.\footnote{If it is
% actually necessary to have different header fonts, one has to define
% new theorem styles (substituting the desired font) or specify the
% information directly in the \texttt{\bslash newtheorem} declaration
% (the unclean variant).}
%
% \DescribeMacro\theorempreskipamount
% \DescribeMacro\theorempostskipamount Two additional parameters
% affect the vertical space around the theorem environments:
% |\theorempreskipamount| and |\theorempostskipamount| define,
% respectively, the spacing before and after such an environment.
% These parameters apply for all theorem sets and can be manipulated
% with the ordinary length macros.  They are rubber lengths
% (`\textsf{skips}'), and therefore can contain \texttt{plus} and
% \texttt{minus} parts.
%
% Since the definition of theorem sets should---most sensibly---be
% placed in the preamble, we only allow installation there.
% It is therefore possible to release the memory used here after
% |\begin{document}|, in order to make room for other
% applications.
%
%
% \subsection{Existing theorem styles}
%
% The following theorem styles exist to date:
% \begin{deflist}{marginbreak}
%    \item[plain]
%       This theorem style emulates the original \LaTeX{} definition,
%       except that additionally the parameters
%       |\theorem...skipamount| are used.
%    \item[break]
%       In this style, the theorem header is followed by a line
%       break.
%    \item[marginbreak]
%       The theorem number is set in the margin, and there is a line
%       break as in \texttt{break}.
%    \item[changebreak]
%       Like \texttt{break}, but with header number and text
%       interchanged.
%    \item[change]
%       Header number and text are interchanged, without a line break.
%    \item[margin]
%       The number is set in the left margin, without a line break.
% \end{deflist}
% All styles (except \texttt{plain}) select |\slshape| as the default
% |\theorembodyfont|.
%
%
% \subsection{Examples}
%
% Given the above theorem sets {\envfont Cor}, {\envfont Exa}
% and {\envfont Rem},
% suppose that the preamble also contains the declarations:
% \begin{verbatim}
%    \theoremstyle{marginbreak}   \newtheorem{Lem}[Cor]{Lemma}
%    \theoremstyle{change}
%    \theorembodyfont{\itshape}        \newtheorem{Def}[Cor]{Definition}
%
%    \theoremheaderfont{\scshape}
%\end{verbatim}
% Then the following are some typical examples of the typeset output
% resulting from their use.
%
% \begin{Cor}
%    This is a sentence typeset in the theorem environment {\envfont
%    Cor}.
% \end{Cor}
%
% \begin{Exa}
%    This is a sentence typeset in the theorem environment {\envfont
%    Exa}.
% \end{Exa}
%
% \begin{Rem}
%    This is a sentence typeset in the theorem environment {\envfont
%    Rem}.
% \end{Rem}
%
% \begin{Lem}[Ben User]
%    This is a sentence typeset in the theorem environment {\envfont
%    Lem}.
% \end{Lem}
%
% \begin{Def}[Very impressive Definition]
%    This is a sentence typeset in the theorem environment {\envfont
%    Def}.
% \end{Def}
%
% The last two examples show the effect of the optional argument to
% a theorem environment (it is the text typeset in parentheses).
%
%
% \section{Special Considerations}
% \label{sec:newfonts}
%
% Theoremheader and body are implemented as a unit. This means that
% the |\theoremheaderfont|  will inherit characteristics of the
% |\theorembodyfont| in \LaTeXe. Thus, if for example
% |\theorembodyfont| is |\itshape| and |\theoremheaderfont|
% is |\bfseries| the font selected for the header will have the
% characteristics `bold extended italic'. If this is not desired one
% should set the |\theoremheaderfont| to something like
% \begin{verbatim}
%      \theoremheaderfont{\normalfont\bfseries}
%\end{verbatim}
% i.e.\ supplying all necessary font information explicitly.
%
%
%
% \section{Acknowledgements}
%
% The publication of this set of macros was only possible with the
% help of Christina Busse (translating the manuscript into English),
% Joachim Pense (playing the r\^ole of typist), Chris Rowley
% (looking everything over) and many others providing useful
% suggestions.
%
% \MaybeStop{%
%    \begin{thebibliography}{9}
%       \bibitem{bk:GMS94} \textsc{M.~Goossens}, \textsc{F.~Mittelbach}
%         and \textsc{A.~Samarin}.
%         \newblock The \LaTeX{} Companion.
%         \newblock
%         Addison-Wesley, Reading, Massachusetts, 1994.
%       \bibitem{latex:source} \textsc{Lamport, Leslie.} \newblock
%                               \texttt{latex.tex}, version 2.09, date
%                               Feb. 1990.
%    \end{thebibliography}
%    \PrintIndex
%    \PrintChanges}
%
%
% \section{The documentation driver file}
%
% The next bit of code contains the documentation driver file for
% \TeX{}, i.e., the file that will produce the documentation you are
% currently reading. It will be extracted from this file by the
% \texttt{docstrip} program. Since it is the first code in the file
% one can alternatively process this file directly with \LaTeXe{} to
% obtain the documentation.
%
%    \begin{macrocode}
%<*driver>
\documentclass{ltxdoc}

\usepackage{theorem}

%<+driver>% The next few lines define theorem sets which are used
%<+driver>% in the example section of the documentation.

\theoremstyle{break}       \newtheorem{Cor}{Corollary}
\theoremstyle{plain}       \newtheorem{Exa}{Example}[section]
{\theorembodyfont{\upshape}\newtheorem{Rem}{Remark}}
\theoremstyle{marginbreak} \newtheorem{Lem}[Cor]{Lemma}
\theoremstyle{change}
\theorembodyfont{\itshape} \newtheorem{Def}[Cor]{Definition}

\theoremheaderfont{\scshape}

\RecordChanges

\begin{document}
   \DocInput{theorem.dtx}
\end{document}
%</driver>
%    \end{macrocode}
%
% \section{Definition of the Macros}
%
% If the file has been loaded before, we abort immediately. If not the
% package announces itself (this is actually done at the very top of
% the file---the way it is done isn't good style so don't copy it).
% \changes{v2.0e}{89/07/19}{Spaces removed from `typeout'.}
% \changes{v2.2b}{95/11/19}{Announce moved to top of file}
%    \begin{macrocode}
%<*package>
%\@ifundefined{theorem@style}{}{\endinput}
%\def\FMithmInfo{1995/11/19 v2.2b Theorem extension package (FMi)}
%\ProvidesPackage{theorem}[\FMithmInfo]
%    \end{macrocode}
%
%
% \subsection{Definition of theorem styles and fonts}
%
% All the definitions in this file are done globally to allow
% inputting this file inside a group.
%
% \begin{macro}{\theoremstyle}
%    Before a theorem style can be installed, the chosen style must be
%    known. For that reason, we must test to see that
%    |\th@|\meta{style} is known or, more precisely, that it is
%    different from |\relax|. If the style is not known then
%    |\th@plain| is used.
%    \begin{macrocode}
\gdef\theoremstyle#1{%
   \@ifundefined{th@#1}{\@warning
          {Unknown theoremstyle `#1'. Using `plain'}%
          \theorem@style{plain}}%
%    \end{macrocode}
%    We save the theorem style to be used in the token register
%    |\theorem@style|.
%    \begin{macrocode}
      {\theorem@style{#1}}%
%    \end{macrocode}
%    Now we ``evaluate'' the theorem style: this means, we call the
%    macro |\th@|\meta{style} which will activate the relevant
%    definitions which are contained in a separate file.  This is done
%    in a group to suppress changes to the current font.  This could
%    otherwise pose problems together with the new font selection
%    scheme\footnote{When I printed the original article using the new
%    font selection scheme I ended with a document with slanted
%    typefaces (text headings and all) simply because one of the
%    theorem styles used \texttt{\bslash sl} at toplevel.} if the
%    |\th@|\meta{style} is evaluated a second time.
% \changes{v2.1a}{90/02/15}{Group added around `th@..}
% \changes{v2.1a}{90/02/15}{Reseting of `..@begintheorem removed.}
%    \begin{macrocode}
      \begingroup
        \csname th@\the\theorem@style \endcsname
      \endgroup}
%    \end{macrocode}
% \end{macro}
%
%
% \begin{macro}{\@begintheorem}
% \changes{v2.1a}{90/02/15}{Relaxed at toplevel.}
% \begin{macro}{\@opargbegintheorem}
% \changes{v2.1a}{90/02/15}{Relaxed at toplevel.}
%    We reset |\@begintheorem| and |\@opargbegintheorem| to
%    |\relax| since these commands are no longer necessary at
%    toplevel. This will save a few tokens.
%    \begin{macrocode}
\global\let\@begintheorem\relax
\global\let\@opargbegintheorem\relax
%    \end{macrocode}
% \end{macro}
% \end{macro}
%
% \begin{macro}{\theorem@style}
%    Obviously the \textsf{token} register used above has to be
%    allocated. To assure the utmost compatibility with the
%    original \LaTeX{} definition, we set the default theorem style to
%    \texttt{plain}, which implements the usual \LaTeX{} convention.
% \changes{v2.0f}{89/07/20}{Unnecessary `global' removed.}
%    \begin{macrocode}
\newtoks\theorem@style
\global\theorem@style{plain}
%    \end{macrocode}
% \end{macro}
%
% \begin{macro}{\theorembodyfont}
% \begin{macro}{\theorem@bodyfont}
%    If the |\theorembodyfont| is set by the user then it should not
%    interact with the default font set in the theorem style. When the
%    new font selection is in force this may happen if, for example,
%    the default is |\itshape| and the new |\theorembodyfont| is
%    |\sffamily|. So we add a |\reset@font| command in front of the
%    user definition.
% \changes{v2.0f}{89/07/20}{Unnecessary `global' removed.}
% \changes{v2.0g}{89/09/19}{`theoremfont to `theorembodyfont.}
% \changes{v2.1c}{92/06/22}{`reset@font added and logic changed.}
% \changes{v2.1d}{92/10/28}{Reset to empty if arg is empty}
%    \begin{macrocode}
\gdef\theorembodyfont#1{%
%    \end{macrocode}
%    We check if the argument supplied is empty and if so put nothing
%    into the |\theorem@bodyfont| token register to allow for
%    |\theorembodyfont{}| as a mean of using the default of the
%    current |\theoremstyle|.
%    \begin{macrocode}
   \def\@tempa{#1}%
   \ifx\@tempa\@empty
    \theorem@bodyfont{}%
   \else
    \theorem@bodyfont{\reset@font#1}%
   \fi
   }
\newtoks\theorem@bodyfont
\global\theorem@bodyfont{}
%    \end{macrocode}
% \end{macro}
% \end{macro}
%
%
% \begin{macro}{\theoremheaderfont}
% \changes{v2.0g}{89/09/19}{Macro added.}
%    The font for the theorem headers is handled differently because
%    this definition applies to all theorem styles.
%    \begin{macrocode}
\gdef\theoremheaderfont#1{\gdef\theorem@headerfont{#1}%
%    \end{macrocode}
%    After using the macro once it is redefined to produce an error
%    message.
%    \begin{macrocode}
       \gdef\theoremheaderfont##1{%
        \typeout{\string\theoremheaderfont\space should be used
                 only once.}}}
%    \end{macrocode}
% \end{macro}
%
% \begin{macro}{\theorem@headerfont}
% \changes{v2.0g}{89/09/19}{Macro added.}
%    To set the |\theorem@headerfont| default we first test if
%    the new fontselection scheme is in force.
% \changes{v2.1a}{90/02/15}{Support for new font selection.}
%    \begin{macrocode}
\ifx\upshape\undefined
%    \end{macrocode}
%    If not we define it to expand into |\bfseries|. We don't use
%    |\let| just in case a following style option redefines this
%    macro.
%    \begin{macrocode}
\gdef\theorem@headerfont{\bfseries}
%    \end{macrocode}
%    Otherwise we reset the current shape before calling |\bfseries|.
%    \begin{macrocode}
\else \gdef\theorem@headerfont{\normalfont\bfseries}\fi
%    \end{macrocode}
% \end{macro}
%
% \begin{macro}{\th@plain}
% \begin{macro}{\th@break}
% \begin{macro}{\th@marginbreak}
% \begin{macro}{\th@changebreak}
% \begin{macro}{\th@change}
% \begin{macro}{\th@margin}
%    The different styles are defined in macros such as
%    |\th@plain|.  Since memory space is precious in
%    ``non-Big-versions'', we have to avoid offering too many unused
%    definitions.  Therefore we define these styles in separate files
%    that can be loaded on demand.  Thus the commands themselves only
%    load these files. We use |\@input@|, a \LaTeXe{} internal command
%    that ensures that the file will be listed with |\listfiles|.
% \changes{v2.2a}{94/02/02}{Use `@input@ to load theorem layout files}
%    \begin{macrocode}
\gdef\th@plain{\@input@{thp.sty}}
\gdef\th@break{\@input@{thb.sty}}
\gdef\th@marginbreak{\@input@{thmb.sty}}
\gdef\th@changebreak{\@input@{thcb.sty}}
\gdef\th@change{\@input@{thc.sty}}
\gdef\th@margin{\@input@{thm.sty}}
%    \end{macrocode}
%    This list will be expanded when new styles become
%    available.  For testing,  just append new theorem substyles
%    as document options.
% \end{macro}
% \end{macro}
% \end{macro}
% \end{macro}
% \end{macro}
% \end{macro}
%
%
%
% \subsection{Definition of a new theorem set}
%
% As already pointed out, a new theorem environment can be defined in
% three different ways:
% \begin{verbatim}
%    \newtheorem{Lem}{Lemma}
%    \newtheorem{Lem}{Lemma}[section]
%    \newtheorem{Lem}[Theorem]{Lemma}
%\end{verbatim}
% The function of the macro |\newtheorem| is to recognize these cases
% and then to branch into one of the three macros |\@ynthm|, |\@xnthm|
% or |\@othm|.  This mechanism is adopted unchanged from
% \cite{latex:source}; the essential point here is that, for example,
% in the second case, the arguments \texttt{Lem}, \texttt{Lemma} and
% \texttt{section} are passed over to the macro |\@xnthm|.
%
% We inspect this case first because the others present fewer problems,
% and thus are easily derived from this one.
%
% \begin{macro}{\@xnthm}
%    For our example arguments, the macro |\@xnthm| must fulfill
%    the following:
%    \begin{itemize}
%       \item
%          Define a new \LaTeX-counter `Lem'
%       \item
%          reset this counter within a |\section|
%       \item
%          define the macro |\theLem|
%       \item
%          define the environment macros |\Lem| and
%          |\endLem| using the current
%          |\theoremstyle| and |\theorem@bodyfont|.
%    \end{itemize}
%    Obviously, all this should happen only if the first argument
%    of |\@xnthm| (i.e.\ \texttt{Lem} in our example) is chosen so
%    as not to conflict with any previously defined commands or
%    environments.  This test is performed by the \LaTeX{} macro
%    |\@ifdefinable|.
%    \begin{macrocode}
\gdef\@xnthm#1#2[#3]{\expandafter\@ifdefinable\csname #1\endcsname
%    \end{macrocode}
%    Therefore, the first argument of |\@ifdefinable| is the
%    expansion (in the example, |\Lem|) of
%    |\csname||#1\endcsname|.  The second argument is
%    executed only if the test has been completed successfully.
%    \begin{macrocode}
   {%
%    \end{macrocode}
%    Now we define the new counter. The names of the \LaTeX{}
%    macros employed should speak for themselves:
% \changes{v2.2b}{1995/10/07}{Use \cs{@newctr}}
% \changes{v2.2c}{1995/11/23}{Use \cs{@newctr} properly.}
%    \begin{macrocode}
    \@definecounter{#1}\@newctr{#1}[#3]%
%    \end{macrocode}
%    Using |\@newctr| will give a proper error message if the counter
%    in |#3| is not defined.
%    In defining `|\theLem|' we must generate the desired macro
%    name by use of |\expandafter| and |\csname|.
%    \begin{macrocode}
    \expandafter\xdef\csname the#1\endcsname
%    \end{macrocode}
%    An |\xdef| is used in order to make the definition global,
%    and to ensure that it contains the replacement texts of
%    |\@thmcountersep| and |\@thmcounter|.\footnote{These
%    two macros can be defined by the document style.  Their default
%    values produce a `\texttt{.}' as separation and an arabic
%    representation of the number.} However, not everything should be
%    expanded. For example, it saves space to use |\thesection|
%    instead of its---at times---lengthy expansion.
%    \begin{macrocode}
      {\expandafter \noexpand \csname the#3\endcsname
       \@thmcountersep \@thmcounter{#1}}%
%    \end{macrocode}
%    Thus with the defaults of \LaTeX{}, |\theLem| would be
%    replaced by the command sequence |\thesection.\arabic{Lem}|.
%
%    We will now look at the definition of the macro which is executed
%    at the beginning of the actual environment (in our example this
%    macro is |\Lem|). It should be noted that we use an
%    ``|\expandafter| trick'' to expand only certain parts of the
%    replacement text at the time of the definition.
%    \begin{macrocode}
    \def\@tempa{\global\@namedef{#1}}%
    \expandafter \@tempa \expandafter{%
%    \end{macrocode}
%    First, the macro that contains the current definitions of
%    |\@begintheorem| and |\@opargtheorem| should be called
%    up.  The name of this macro---as is already known---has the form
%    |\th@|\meta{theorem\ style}; therefore, it must be called by
%    \begin{macrocode}
      \csname th@\the \theorem@style
%    \end{macrocode}
%    In addition the default theorem font should be changeable, i.e.\
%    we have to insert the contents of |\theorem@bodyfont|. For
%    that reason, we expand even further, beyond |\endcsname|,
%    and thus insert the contents of the \textsf{token} register
%    |\theorem@bodyfont| in the replacement text.
%    \begin{macrocode}
            \expandafter \endcsname \the \theorem@bodyfont
%    \end{macrocode}
%    Now it is time to call the macro |\@thm| which takes over
%    the further processing.  It has two arguments: the current
%    counter name (in our example, \texttt{Lem}), and the text of the
%    label (in our example, \texttt{Lemma}).
%    \begin{macrocode}
     \@thm{#1}{#2}}%
%    \end{macrocode}
%    With this, the `sub-definition' is complete.  The macro
%    |\@endtheorem| ends a theorem environment and is, so far, nothing
%    but an |\endtrivlist|.  (Hence it is defined globally, and not
%    within the theorem styles.\footnote{This has to be changed as
%    soon as theorem styles that change \texttt{\bslash @endtheorem}
%    exist.  In such a case, all existing styles must be changed as
%    well since they will have to reset the macro.}) Therefore, we can
%    set it equivalent to the macro that ends the theorem set (in our
%    example, |\endLem|).  However, if some day theorem styles exist
%    that do change |\@endtheorem|, we would have to use the
%    commented-out line instead.
%    \begin{macrocode}
    \global \expandafter \let \csname end#1\endcsname \@endtheorem
%   \global\@namedef{end#1}{\@endtheorem}%
%    \end{macrocode}
%    With these commands all the required definitions are employed,
%    unless the test |\@ifdefinable| has failed.  Therefore, we
%    end the second argument of this macro and with it the definition
%    of |\@xnthm|.
%    \begin{macrocode}
   }}
%    \end{macrocode}
% \end{macro}
%
% \begin{macro}{\@ynthm}
%    The definition of |\@ynthm| is completely analogous. In
%    this case the new counter that is defined is not reset within
%    another counter; thus the definition of |\the...|
%    is simplified:
%    \begin{macrocode}
\gdef\@ynthm#1#2{\expandafter\@ifdefinable\csname #1\endcsname
   {\@definecounter{#1}%
    \expandafter\xdef\csname the#1\endcsname{\@thmcounter{#1}}%
%    \end{macrocode}
%    The rest of the definition corresponds literally to that of
%    |\@xnthm|:
%    \begin{macrocode}
    \def\@tempa{\global\@namedef{#1}}\expandafter \@tempa
     \expandafter{\csname th@\the \theorem@style \expandafter
     \endcsname \the\theorem@bodyfont \@thm{#1}{#2}}%
    \global \expandafter \let \csname end#1\endcsname \@endtheorem}}
%    \end{macrocode}
% \end{macro}
%
% \begin{macro}{\@othm}
%    The definition of |\@othm| does not contain anything new.
%    \begin{macrocode}
\gdef\@othm#1[#2]#3{%
%    \end{macrocode}
% \changes{v2.2b}{95/10/07}{Build in check for counter defined}
%    We do not define a new counter but instead use one that has
%    already been defined.  Thus the only definition we need is that
%    of this pseudo-counter (i.e.\ |\the|\meta{env.\ name}).
%    First we check if |#2| corresponds to a known counter name.
%    \begin{macrocode}
  \expandafter\ifx\csname c@#2\endcsname\relax
   \@nocounterr{#2}%
  \else
   \expandafter\@ifdefinable\csname #1\endcsname
   {\expandafter \xdef \csname the#1\endcsname
     {\expandafter \noexpand \csname the#2\endcsname}%
%    \end{macrocode}
%    All other parts of the definition can be adopted from
%    |\@xnthm|. We have to remember, though, that in this case
%    the name of the current counter and the theorem label have moved
%    to the second and third arguments.
%    \begin{macrocode}
    \def\@tempa{\global\@namedef{#1}}\expandafter \@tempa
     \expandafter{\csname th@\the \theorem@style \expandafter
     \endcsname \the\theorem@bodyfont \@thm{#2}{#3}}%
    \global \expandafter \let \csname end#1\endcsname \@endtheorem}%
  \fi}
%    \end{macrocode}
% \end{macro}
%
% \subsection{Macros that are employed in a {\envfont theorem}
%    environment}
%
% \begin{macro}{\@thm}
%    The macro |\@thm| has to increase the current counter.  Then,
%    depending on whether the environment has (or does not have) an
%    optional argument, it has to branch into either |\@begintheorem|
%    or |\@opargtheorem|.
%    \begin{macrocode}
\gdef\@thm#1#2{\refstepcounter{#1}%
%    \end{macrocode}
%    Now we start a {\envfont trivlist} environment, and give
%    |\@topsep| and |\@topsepadd| the values of the \textsf{skip}
%    registers |\theorempreskipamount| and |\theorempostskipamount|.
%    The value in |\@topsep| is the vertical space that is inserted by
%    the first (and only) |\item| in our |\trivlist| whilst
%    |\@topsepadd| is inserted by |\@endparenv| at the end of that
%    {\envfont trivlist} environment.  By using these registers, we
%    obtain the desired space around a {\envfont theorem} environment.
%    \begin{macrocode}
   \trivlist
   \@topsep \theorempreskipamount               % used by first \item
   \@topsepadd \theorempostskipamount           % used by \@endparenv
%    \end{macrocode}
%    Now we have to test whether an optional argument has been given.
% \changes{v2.0d}{89/07/18}{Forgotten percent added.}
%    \begin{macrocode}
   \@ifnextchar [%
%    \end{macrocode}
%    If there is an optional argument, we will call |\@ythm|, and
%    move the arguments read back into the input stream.
%    \begin{macrocode}
   {\@ythm{#1}{#2}}%
%    \end{macrocode}
%    If not, we call |\@begintheorem|. Its first argument is the
%    name of the theorem set (hence the second argument of
%    |\@thm|). Its second argument is the macro that produces the
%    current number.
%    \begin{macrocode}
   {\@begintheorem{#2}{\csname the#1\endcsname}\ignorespaces}}
%    \end{macrocode}
% \end{macro}
%
% \begin{macro}{\@xthm}
% \begin{macro}{\@ythm}
%    Both these macros were originally called by |\@thm|.  We do
%    not need |\@xthm| anymore, hence we reset it to
%    |\relax|. The definition of |\@ythm| has not changed at
%    all from its definition in \LaTeX. In order to make the macros
%    easier to understand, we will
%    nevertheless present it (commented out).
%    \begin{macrocode}
\global\let\@xthm\relax
% \def\@ythm#1#2[#3]{\@opargbegintheorem{#2}{\csname
%           the#1\endcsname}{#3}\ignorespaces}
%</package>
%    \end{macrocode}
%    The primitive |\ignorespaces| in |\@ythm| and
%    |\@thm| is needed to remove the spaces between the
%    |\begin{...}| and the actual text.
% \end{macro}
% \end{macro}
%
%
% \subsection{Definition of the theorem substyles}
%
% As already pointed out, the theorem substyles, defined below, are
% only loaded when necessary.  Note that all these substyles, except
% \texttt{plain}, have |\slshape| as the default body font.
%
% \subsubsection{The \texttt{plain} style}
%
% \label{dok:th@plain} As the following macros use |@|, we have
% to locally set the |\catcode| of this symbol to ``letter''.
% This happens within a group, so that we do not have to worry about
% which |\catcode| that symbol had before.
%    \begin{macrocode}
%<*thp>
\begingroup \makeatletter
%    \end{macrocode}
% Since we are now within a group, we must make all definitions
% globally.  First we make sure that \texttt{theorem.sty} is loaded.
% This will allow us to use this file as a document style option
% without having to call \texttt{theorem} itself as an option.  At the
% same time, we assure that at least version 2 is loaded, since
% |\theorem@style| was not defined in earlier versions.
%    \begin{macrocode}
\@ifundefined{theorem@style}{\input{theorem.sty}}{}
\ProvidesFile{thp.sty}
             [\FMithmInfo]
%    \end{macrocode}
%
%
% \begin{macro}{\th@plain}
% \begin{macro}{\@begintheorem}
% \begin{macro}{\@opargbegintheorem}
%    |\theoremstyle{plain}| corresponds to the original
%    definition, except that the distances to the surrounding text are
%    determined by the parameters |\theorempreskipamount| and
%    |\theorempostskipamount|.
%    First we set the default body font.
%    \begin{macrocode}
\gdef\th@plain{\normalfont\itshape
%    \end{macrocode}
%    Then we define |\@begintheorem| and
%    |\@opargbegintheorem|. These two macros define how the
%    header of a theorem is typeset.  |\@opargbegintheorem| will
%    be called if a \textsf{theorem} environment with an optional
%    argument is encountered; otherwise, the header is constructed by
%    calling |\@begintheorem|.  If one of these macros is
%    executed, we are within a \textsf{trivlist} environment started by
%    |\@thm|.  So the theorem header is produced with an
%    |\item| command.
%
%    Instead of specifying the header font directly, all standard
%    theorem styles use the |\theorem@headerfont| macro to allow
%    customization.  The extra space (|\labelsep|) is necessary
%    because of problems in the \textsf{trivlist} environment.
%    \begin{macrocode}
  \def\@begintheorem##1##2{%
        \item[\hskip\labelsep \theorem@headerfont ##1\ ##2]}%
%    \end{macrocode}
%    The definition of |\@opargbegintheorem| is completely
%    analogous. The only difference is the fact that there exists a
%    third argument (which is the optional parameter of the
%    environment and contains additional information about the
%    theorem).  Customarily we enclose it in parentheses.
%    \begin{macrocode}
\def\@opargbegintheorem##1##2##3{%
   \item[\hskip\labelsep \theorem@headerfont ##1\ ##2\ (##3)]}}
%    \end{macrocode}
% \end{macro}
% \end{macro}
% \end{macro}
%
% \noindent
% We conclude with an |\endgroup| to restore the
% |\catcode| of |@|.
%    \begin{macrocode}
\endgroup
%</thp>
%    \end{macrocode}
%
%
% \subsubsection{The \texttt{break} style}
%
% This style option is stored in the file |thb.sty|.
% For the next two lines see the documentation for |\th@plain| on
% page~\pageref{dok:th@plain}.
%    \begin{macrocode}
%<*thb>
\begingroup \makeatletter
\@ifundefined{theorem@style}{\input{theorem.sty}}{}
\ProvidesFile{thb.sty}
             [\FMithmInfo]
%    \end{macrocode}
%
% \begin{macro}{\th@break}
%    |\theoremstyle{break}| produces a line break after the name
%    of the theorem.  The font is |\slshape|.  Hence, we define
%    |\th@break| as follows:
%    \begin{macrocode}
\gdef\th@break{\normalfont\slshape
  \def\@begintheorem##1##2{\item[%
%    \end{macrocode}
%    We run into the following problem: it is not possible to create
%    the header with |\item[|\meta{title}|]| and then start
%    a new line by, for example, |\mbox{}\\|. Such a definition
%    will fail whenever a list environment follows immediately.  With
%    the above construction, the |\mbox{}| causes the switch
%    |@inlabel| (cf.\ definition of |\list| and
%    |\trivlist| in \cite{latex:source}) to be set to \texttt{false}
%    and so the following list will insert additional vertical
%    space (|\topskip|). This is quite annoying. Therefore, we
%    create the line break within the |\item|. In order to ensure
%    that the text will begin at the proper position in the following
%    line, we simply pretend that the label does not take any
%    room.\footnote{This will lead to problems whenever very high
%    symbols occurring in the line tower into the
%    heading. So, something else has to be done here sometime.}
%    \begin{macrocode}
       \rlap{\vbox{\hbox{\hskip \labelsep\theorem@headerfont ##1\ ##2}%
                   \hbox{\strut}}}]}%
%    \end{macrocode}
%    Again, the definition of |\@opargbegintheorem| is
%    completely analogous.
%    \begin{macrocode}
\def\@opargbegintheorem##1##2##3{%
  \item[\rlap{\vbox{\hbox{\hskip \labelsep \theorem@headerfont
                     ##1\ ##2\ (##3)}%
                    \hbox{\strut}}}]}}
\endgroup
%</thb>
%    \end{macrocode}
% \end{macro}
%
%
% \subsubsection{The \texttt{changebreak} style}
%
%    \begin{macrocode}
% This style option is stored in the file |thcb.sty|.
%    \begin{macrocode}
%<*thcb>
\begingroup \makeatletter
\@ifundefined{theorem@style}{\input{theorem.sty}}{}
\ProvidesFile{thcb.sty}
             [\FMithmInfo]
%    \end{macrocode}
%
% \begin{macro}{\th@changebreak}
%    The \texttt{change-break} theorem style is like \texttt{break}
%    but with interchange of theorem name and theorem number.  Thus we
%    define |\th@changebreak| as follows:
%    \begin{macrocode}
\gdef\th@changebreak{\normalfont\slshape
  \def\@begintheorem##1##2{\item
       [\rlap{\vbox{\hbox{\hskip\labelsep \theorem@headerfont ##2\ ##1}%
                    \hbox{\strut}}}]}%
\def\@opargbegintheorem##1##2##3{%
  \item[\rlap{\vbox{\hbox{\hskip\labelsep \theorem@headerfont
                     ##2\ ##1\ (##3)}%
                    \hbox{\strut}}}]}}
\endgroup
%</thcb>
%    \end{macrocode}
% \end{macro}
%
%
% \subsubsection{The \texttt{change} style}
%
% This style option is stored in the file |thc.sty|.
%    \begin{macrocode}
%<*thc>
\begingroup \makeatletter
\@ifundefined{theorem@style}{\input{theorem.sty}}{}
\ProvidesFile{thc.sty}
             [\FMithmInfo]
%    \end{macrocode}
%
% \begin{macro}{\th@change}
%    The \texttt{change} theorem style corresponds to the
%    \texttt{change break} style without a linebreak after the header.
%    To say it in another way, it's the same as the \texttt{plain}
%    style but with number and name interchanged and |\slshape| as the
%    default font.
%    \begin{macrocode}
\gdef\th@change{\normalfont\slshape
  \def\@begintheorem##1##2{\item
       [\hskip\labelsep \theorem@headerfont ##2\ ##1]}%
\def\@opargbegintheorem##1##2##3{%
  \item[\hskip\labelsep \theorem@headerfont ##2\ ##1\ (##3)]}}
\endgroup
%</thc>
%    \end{macrocode}
% \end{macro}
%
%
% \subsubsection{The \texttt{marginbreak} style}
%
% This style option is the one used most often at Mainz.  It is saved
% in the file |thmb.sty|.
%    \begin{macrocode}
%<*thmb>
\begingroup \makeatletter
\@ifundefined{theorem@style}{\input{theorem.sty}}{}
\ProvidesFile{thmb.sty}
             [\FMithmInfo]
%    \end{macrocode}
%
% \begin{macro}{\th@marginbreak}
%    The \texttt{margin break} style is nearly the same as the
%    \texttt{change break} style. The only difference is the placement
%    of the theorem number.  We use |\llap| to place it in the left
%    margin.
%
%    In this style |\labelsep| denotes the separation between the
%    number and the text.
%    \begin{macrocode}
\gdef\th@marginbreak{\normalfont\slshape
  \def\@begintheorem##1##2{\item
       [\rlap{\vbox{\theorem@headerfont
                    \hbox{\llap{##2}\hskip\labelsep ##1}%
                    \hbox{\strut}}}]}%
\def\@opargbegintheorem##1##2##3{%
  \item[\rlap{\vbox{\theorem@headerfont
                    \hbox{\llap{##2}\hskip\labelsep ##1\ (##3)}%
                    \hbox{\strut}}}]}}
\endgroup
%</thmb>
%    \end{macrocode}
% \changes{v2.1c}{90/02/17}{`endgroup outside macrocode env.}
% \end{macro}
%
%
%
% \subsubsection{The \texttt{margin} style}
%
% This style option is stored in the file |thm.sty|.
%    \begin{macrocode}
%<*thm>
\begingroup \makeatletter
\@ifundefined{theorem@style}{\input{theorem.sty}}{}
\ProvidesFile{thm.sty}
             [\FMithmInfo]
%    \end{macrocode}
%
% \begin{macro}{\th@margin}
%    Again this is only a variant of the theorem styles described
%    above without any new ideas.
%    \begin{macrocode}
\gdef\th@margin{\normalfont\slshape
  \def\@begintheorem##1##2{\item
       [\theorem@headerfont \llap{##2}\hskip\labelsep ##1]}%
\def\@opargbegintheorem##1##2##3{%
  \item[\theorem@headerfont \llap{##2}\hskip\labelsep ##1\ (##3)]}}
\endgroup
%</thm>
%    \end{macrocode}
% \end{macro}
%
%
%
%
% \subsection{Final Definitions}
%
% \begin{macro}{\theorempreskipamount}
% \begin{macro}{\theorempostskipamount}
%    The \textsf{skip} parameters that regulate the vertical empty space
%    before and after the {\envfont theorem} environment have
%    to be allocated as well.
% \changes{v2.0f}{89/07/20}{Unnecessary `global' removed.}
%    \begin{macrocode}
%<*package>
\newskip\theorempreskipamount
\newskip\theorempostskipamount
%    \end{macrocode}
%    Since we have used the same values for all theorem sets, we now can
%    assign them.
%    \begin{macrocode}
\global\setlength\theorempreskipamount{12pt plus 5pt minus 3pt}
\global\setlength\theorempostskipamount{8pt plus 3pt minus 1.5pt}
%    \end{macrocode}
% \end{macro}
% \end{macro}
%
%
% \begin{macro}{\@endtheorem}
%    The same holds for the macro |\@endtheorem|, which ends a
%    {\envfont theorem} environment. Since it is the same for all
%    theorem sets, it is removed from the macros |\th@|\meta{style}.
%    It simply ends the {\envfont trivlist} environment, which was
%    begun in |\@thm|.
%    \begin{macrocode}
\global\let\@endtheorem=\endtrivlist
%    \end{macrocode}
% \end{macro}
%
%
%
% \begin{macro}{\@preamblecmds}
%    All macros defined above are to be used only in the preamble.
%    Therefore, we insert them in |\@preamblecmds| which will disable
%    them at begin document. This is done by the internal \LaTeXe{}
%    command |\@onlypreamble|.
% \changes{v2.0g}{89/09/19}{`theoremheaderfont and `theorembodyfont
%    added.}
% \changes{v2.2a}{94/02/03}{Use \LaTeXe{} interface}
%    \begin{macrocode}
\@onlypreamble\@xnthm
\@onlypreamble\@ynthm
\@onlypreamble\@othm
\@onlypreamble\newtheorem
\@onlypreamble\theoremstyle
\@onlypreamble\theorembodyfont
\@onlypreamble\theoremheaderfont
%    \end{macrocode}
% \end{macro}
%
% Finally we declare the \texttt{plain} theorem style to be the default.
% \changes{v2.1b}{90/06/04}{Fixed missing `item by declaring default
%    style.}
%    \begin{macrocode}
\theoremstyle{plain}
%</package>
%    \end{macrocode}
%
% \Finale

\endinput
}{}
\ProvidesFile{thb.sty}
             [\FMithmInfo]
%    \end{macrocode}
%
% \begin{macro}{\th@break}
%    |\theoremstyle{break}| produces a line break after the name
%    of the theorem.  The font is |\slshape|.  Hence, we define
%    |\th@break| as follows:
%    \begin{macrocode}
\gdef\th@break{\normalfont\slshape
  \def\@begintheorem##1##2{\item[%
%    \end{macrocode}
%    We run into the following problem: it is not possible to create
%    the header with |\item[|\meta{title}|]| and then start
%    a new line by, for example, |\mbox{}\\|. Such a definition
%    will fail whenever a list environment follows immediately.  With
%    the above construction, the |\mbox{}| causes the switch
%    |@inlabel| (cf.\ definition of |\list| and
%    |\trivlist| in \cite{latex:source}) to be set to \texttt{false}
%    and so the following list will insert additional vertical
%    space (|\topskip|). This is quite annoying. Therefore, we
%    create the line break within the |\item|. In order to ensure
%    that the text will begin at the proper position in the following
%    line, we simply pretend that the label does not take any
%    room.\footnote{This will lead to problems whenever very high
%    symbols occurring in the line tower into the
%    heading. So, something else has to be done here sometime.}
%    \begin{macrocode}
       \rlap{\vbox{\hbox{\hskip \labelsep\theorem@headerfont ##1\ ##2}%
                   \hbox{\strut}}}]}%
%    \end{macrocode}
%    Again, the definition of |\@opargbegintheorem| is
%    completely analogous.
%    \begin{macrocode}
\def\@opargbegintheorem##1##2##3{%
  \item[\rlap{\vbox{\hbox{\hskip \labelsep \theorem@headerfont
                     ##1\ ##2\ (##3)}%
                    \hbox{\strut}}}]}}
\endgroup
%</thb>
%    \end{macrocode}
% \end{macro}
%
%
% \subsubsection{The \texttt{changebreak} style}
%
%    \begin{macrocode}
% This style option is stored in the file |thcb.sty|.
%    \begin{macrocode}
%<*thcb>
\begingroup \makeatletter
\@ifundefined{theorem@style}{% \iffalse meta-comment
%
% Copyright (C) 1993-2024
%
% The LaTeX Project and any individual authors listed elsewhere
% in this file.
%
% This file is part of the Standard LaTeX `Tools Bundle'.
% -------------------------------------------------------
%
% It may be distributed and/or modified under the
% conditions of the LaTeX Project Public License, either version 1.3c
% of this license or (at your option) any later version.
% The latest version of this license is in
%    https://www.latex-project.org/lppl.txt
% and version 1.3c or later is part of all distributions of LaTeX
% version 2005/12/01 or later.
%
% The list of all files belonging to the LaTeX `Tools Bundle' is
% given in the file `manifest.txt'.
%
% \fi
%
%%
%
%
\def\FMithmInfo{2023/07/05 v2.2c Theorem extension package (FMi)}
%
% \ProvidesFile{theorem.dtx}[\FMithmInfo]
%\iffalse   % this is a METACOMMENT !
%%
%% Package `theorem' to use with LaTeX2e
%% Copyright (C) 1989-1995 Frank Mittelbach, all rights reserved.
%<+package>\@ifundefined{theorem@style}{}{\endinput}
%<+package>\NeedsTeXFormat{LaTeX2e}[1995/06/01]
%<+package>\ProvidesPackage{theorem}
%<+package>         [\FMithmInfo]
%
% \fi
%
% \changes{v1.0}{87/05/07}{First draft.}
% \changes{v1.0a}{87/05/07}{`thistheoremstyle' added.}
% \changes{v1.1a}{87/07/29}{New concept: redefine the internal
%                          \protect\LaTeX{} commands `@xnthm' etc.}
% \changes{v1.1b}{87/07/31}{`theoremfont' used to specify font.}
% \changes{v1.2a}{88/01/08}{`@endtheorem' now globally defined.}
% \changes{v1.2b}{88/01/11}{`newtheorem' only allowed in the preamble.}
% \changes{v1.3a}{88/02/14}{`theoremfont' now a toks register.}
%
% \changes{v2.0a}{88/11/16}{New implementation and new documentation.}
% \changes{v2.0a}{88/11/16}{No layout changes but internal modifications
%                           old substyles won't work.}
% \changes{v2.0b}{89/01/30}{`global' used for all definitions.}
% \changes{v2.0c}{89/06/15}{Documentation translated to English.
%                            Thanks to LLL and Nina}
% \changes{v2.1a}{90/02/15}{Support for new font selection.}
% \changes{v2.2a}{94/02/03}{Cleanup for LaTeX2e}
%
% \DoNotIndex{\@endparenv,\@ifdefinable,\@ifnextchar,\@ifundefined}
% \DoNotIndex{\@namedef,\@nameuse,\@spaces,\@tempa,\@warning}
% \DoNotIndex{\ }
% \DoNotIndex{\begingroup,\bfseries}
% \DoNotIndex{\csname}
% \DoNotIndex{\def,\do,\docdate}
% \DoNotIndex{\endcsname,\endgroup,\endinput,\endtrivlist,\expandafter}
% \DoNotIndex{\filedate,\fileversion}
% \DoNotIndex{\gdef,\global}
% \DoNotIndex{\hbox,\hskip}
% \DoNotIndex{\ignorespaces,\input,\itshape,\item}
% \DoNotIndex{\labelsep,\let,\llap}
% \DoNotIndex{\makeatletter}
% \DoNotIndex{\noexpand}
% \DoNotIndex{\refstepcounter,\relax,\rlap}
% \DoNotIndex{\setlength,\slshape,\space,\string,\strut}
% \DoNotIndex{\the,\trivlist,\typeout}
% \DoNotIndex{\vbox}
% \DoNotIndex{\xdef}
%
% \def\deflabel#1{\ttfamily #1\hfill}
% \def\deflist#1{\begin{list}{}{\settowidth\labelwidth{\ttfamily #1}%
%                               \setlength\leftmargin\labelwidth
%                               \addtolength\leftmargin\labelsep
%                               \let\makelabel\deflabel}}
% \def\enddeflist{\end{list}}
% \def\envfont{\normalfont\ttfamily}
%
% \MakeShortVerb{\|}
%
% \newcommand{\TUB}{{\sl TUGboat\/}}
%
% \GetFileInfo{theorem.dtx}
% \title{An Extension of the
%        \LaTeX{} theorem environment\thanks{This file
%        has version number \fileversion, last
%        revised \filedate.}}
% \author{Frank Mittelbach}
%
% \MaintainedByLaTeXTeam{tools}
% \maketitle
%
% \begin{abstract}
%    The macros described in this paper yield an extension of the
%    \LaTeX{} theorem mechanism. It is designed to satisfy the
%    different requirements of various journals. Thus, the
%    layout of the ``theorems'' can be manipulated by determining a
%    ``style''. This article describes not only the use, but
%    also the definition, of the necessary macros.
% \end{abstract}
%
% \section*{Preface to version 2.2}
%
% For \LaTeXe{} this package did not need any fundamental changes. I
% only modified the messages generated so that theorem layout styles
% will show up with the |\listfiles| command and cleaned the section
% on the New Font Selection Scheme since this is now included in
% \LaTeX.
%
% \section*{Preface to version 2.1}
%
% This version is identical to 2.0g described in \TUB\ 10\#3 except
% for some internal defaults which are now set depending on the used
% font selection scheme.
%
% This was done to avoid unpleasant surprises if the new font selection
% scheme is in force. For further details see
% section~\ref{sec:newfonts} and \cite{bk:GMS94}.
%
% \section{Introduction}
%
% For our purposes here, ``theorems'' are labelled enunciations,
% often set off from the main text by extra space and a font change.
% Theorems, corollaries, conjectures, definitions, and remarks
% are all instances of ``theorems''.  The ``header'' of these structures
% is composed of a label (such as \textsc{Theorem} or \textsc{Remark})
% and a number which serializes an item in the sequence of items
% with the same label.
%
% Shortly after the introduction of \LaTeX{} at the Fachbereich
% Mathematik in Mainz, the desire to manipulate the layout of
% ``theorems'' arose.  In Mainz, the following two conventions came
% into general use:
% \begin{enumerate}
%   \item The number of the theorem is shown in the margin.
%   \item There is a line break at
%         the end of the theorem header.
% \end{enumerate}
%
% Additionally, some journals require different formats which depend on
% the ``sort of theorem'': e.g.\ often remarks and definitions are set
% in |\upshape|, while |\itshape| is employed for main theorems.
%
% Confronted with these requirements, a theorem environment was
% developed in Mainz which allows separate determination of the
% layout of the ``theorems sets'', comparable to |\pagestyle|.
%
%
% \section{The user interface}
% \subsection{Defining new theorem sets}
%
% \DescribeMacro\newtheorem
% As in the original \LaTeX{} version, the command |\newtheorem|
% defines a new ``theorem set'' or ``theorem-like structure''.
% Two required arguments name the new environment and give the text
% to be typeset with each instance of the new ``set'', while
% an optional argument determines how
% the ``set'' is enumerated:
% \begin{description}
%    \item[\ttfamily \bslash newtheorem\{foo\}\{bar\}]
%       The theorem set {\envfont foo} (whose name is \texttt{bar})
%       uses its own counter.
%    \item[\ttfamily \bslash newtheorem\{foo2\}{[foo]}\{bar2\}]
%       The theorem set {\envfont foo2} (printed name \texttt{bar2})
%       uses the same counter as the theorem set \texttt{foo}.
%    \item[\ttfamily \bslash newtheorem\{foo3\}\{bar3\}{[section]}]
%       The theorem set {\envfont foo3} (printed name \texttt{bar3}) is
%       enumerated within the counter \texttt{section}, i.e.\ with every
%       new |\section| the enumeration begins again with 1, and
%       the enumeration is composed from the section-number and the
%       theorem counter itself.
% \end{description}
%
% \DescribeMacro\theoremstyle
% Additionally, the command |\theoremstyle| can define the layout
% of various, or all, theorem sets.  It should be noted that any
% theorem set defined by |\newtheorem| is typeset in the
% |\theoremstyle| that is current at the time of the definition.
% Thus, the following
% \begin{verbatim}
%    \theoremstyle{break}        \newtheorem{Cor}{Corollary}
%    \theoremstyle{plain}        \newtheorem{Exa}{Example}[section]
%\end{verbatim}                             ^^A No blank at line start
% leads to the result that the set {\envfont Cor} is formatted in the
% style \texttt{break}, while the set {\envfont Exa} and all the
% following ones are formatted in the style \texttt{plain}, unless
% another |\theoremstyle| follows.  Since the definitions installed by
% |\newtheorem| are global, one also can limit |\theoremstyle| locally
% by grouping braces.
%
% \DescribeMacro\theorembodyfont
% The choice of the font for the theorem body is completely
% independent of the chosen |\theoremstyle|; this has proven to
% be very advantageous.  For example,
% \begin{verbatim}
%   {\theorembodyfont{\upshape}          \newtheorem{Rem}{Remark}}
%\end{verbatim}
% defines a theorem set {\envfont Rem}, which will be set in
% |\upshape| in the current layout (which in our example is
% \texttt{plain}).  As with |\theoremstyle|, the |\theorembodyfont|
% chosen is that current at the time of |\newtheorem|.  If
% |\theorembodyfont| is not specified or one defines
% |\theorembodyfont{}|, then the font used will be that defined by the
% |\theoremstyle|.
%
% \DescribeMacro\theoremheaderfont It is also possible to customize
% the font used for the theorem headers.  This is, however, a global
% declaration, and therefore there should be at most one
% |\theoremheaderfont| declaration in the preamble.\footnote{If it is
% actually necessary to have different header fonts, one has to define
% new theorem styles (substituting the desired font) or specify the
% information directly in the \texttt{\bslash newtheorem} declaration
% (the unclean variant).}
%
% \DescribeMacro\theorempreskipamount
% \DescribeMacro\theorempostskipamount Two additional parameters
% affect the vertical space around the theorem environments:
% |\theorempreskipamount| and |\theorempostskipamount| define,
% respectively, the spacing before and after such an environment.
% These parameters apply for all theorem sets and can be manipulated
% with the ordinary length macros.  They are rubber lengths
% (`\textsf{skips}'), and therefore can contain \texttt{plus} and
% \texttt{minus} parts.
%
% Since the definition of theorem sets should---most sensibly---be
% placed in the preamble, we only allow installation there.
% It is therefore possible to release the memory used here after
% |\begin{document}|, in order to make room for other
% applications.
%
%
% \subsection{Existing theorem styles}
%
% The following theorem styles exist to date:
% \begin{deflist}{marginbreak}
%    \item[plain]
%       This theorem style emulates the original \LaTeX{} definition,
%       except that additionally the parameters
%       |\theorem...skipamount| are used.
%    \item[break]
%       In this style, the theorem header is followed by a line
%       break.
%    \item[marginbreak]
%       The theorem number is set in the margin, and there is a line
%       break as in \texttt{break}.
%    \item[changebreak]
%       Like \texttt{break}, but with header number and text
%       interchanged.
%    \item[change]
%       Header number and text are interchanged, without a line break.
%    \item[margin]
%       The number is set in the left margin, without a line break.
% \end{deflist}
% All styles (except \texttt{plain}) select |\slshape| as the default
% |\theorembodyfont|.
%
%
% \subsection{Examples}
%
% Given the above theorem sets {\envfont Cor}, {\envfont Exa}
% and {\envfont Rem},
% suppose that the preamble also contains the declarations:
% \begin{verbatim}
%    \theoremstyle{marginbreak}   \newtheorem{Lem}[Cor]{Lemma}
%    \theoremstyle{change}
%    \theorembodyfont{\itshape}        \newtheorem{Def}[Cor]{Definition}
%
%    \theoremheaderfont{\scshape}
%\end{verbatim}
% Then the following are some typical examples of the typeset output
% resulting from their use.
%
% \begin{Cor}
%    This is a sentence typeset in the theorem environment {\envfont
%    Cor}.
% \end{Cor}
%
% \begin{Exa}
%    This is a sentence typeset in the theorem environment {\envfont
%    Exa}.
% \end{Exa}
%
% \begin{Rem}
%    This is a sentence typeset in the theorem environment {\envfont
%    Rem}.
% \end{Rem}
%
% \begin{Lem}[Ben User]
%    This is a sentence typeset in the theorem environment {\envfont
%    Lem}.
% \end{Lem}
%
% \begin{Def}[Very impressive Definition]
%    This is a sentence typeset in the theorem environment {\envfont
%    Def}.
% \end{Def}
%
% The last two examples show the effect of the optional argument to
% a theorem environment (it is the text typeset in parentheses).
%
%
% \section{Special Considerations}
% \label{sec:newfonts}
%
% Theoremheader and body are implemented as a unit. This means that
% the |\theoremheaderfont|  will inherit characteristics of the
% |\theorembodyfont| in \LaTeXe. Thus, if for example
% |\theorembodyfont| is |\itshape| and |\theoremheaderfont|
% is |\bfseries| the font selected for the header will have the
% characteristics `bold extended italic'. If this is not desired one
% should set the |\theoremheaderfont| to something like
% \begin{verbatim}
%      \theoremheaderfont{\normalfont\bfseries}
%\end{verbatim}
% i.e.\ supplying all necessary font information explicitly.
%
%
%
% \section{Acknowledgements}
%
% The publication of this set of macros was only possible with the
% help of Christina Busse (translating the manuscript into English),
% Joachim Pense (playing the r\^ole of typist), Chris Rowley
% (looking everything over) and many others providing useful
% suggestions.
%
% \MaybeStop{%
%    \begin{thebibliography}{9}
%       \bibitem{bk:GMS94} \textsc{M.~Goossens}, \textsc{F.~Mittelbach}
%         and \textsc{A.~Samarin}.
%         \newblock The \LaTeX{} Companion.
%         \newblock
%         Addison-Wesley, Reading, Massachusetts, 1994.
%       \bibitem{latex:source} \textsc{Lamport, Leslie.} \newblock
%                               \texttt{latex.tex}, version 2.09, date
%                               Feb. 1990.
%    \end{thebibliography}
%    \PrintIndex
%    \PrintChanges}
%
%
% \section{The documentation driver file}
%
% The next bit of code contains the documentation driver file for
% \TeX{}, i.e., the file that will produce the documentation you are
% currently reading. It will be extracted from this file by the
% \texttt{docstrip} program. Since it is the first code in the file
% one can alternatively process this file directly with \LaTeXe{} to
% obtain the documentation.
%
%    \begin{macrocode}
%<*driver>
\documentclass{ltxdoc}

\usepackage{theorem}

%<+driver>% The next few lines define theorem sets which are used
%<+driver>% in the example section of the documentation.

\theoremstyle{break}       \newtheorem{Cor}{Corollary}
\theoremstyle{plain}       \newtheorem{Exa}{Example}[section]
{\theorembodyfont{\upshape}\newtheorem{Rem}{Remark}}
\theoremstyle{marginbreak} \newtheorem{Lem}[Cor]{Lemma}
\theoremstyle{change}
\theorembodyfont{\itshape} \newtheorem{Def}[Cor]{Definition}

\theoremheaderfont{\scshape}

\RecordChanges

\begin{document}
   \DocInput{theorem.dtx}
\end{document}
%</driver>
%    \end{macrocode}
%
% \section{Definition of the Macros}
%
% If the file has been loaded before, we abort immediately. If not the
% package announces itself (this is actually done at the very top of
% the file---the way it is done isn't good style so don't copy it).
% \changes{v2.0e}{89/07/19}{Spaces removed from `typeout'.}
% \changes{v2.2b}{95/11/19}{Announce moved to top of file}
%    \begin{macrocode}
%<*package>
%\@ifundefined{theorem@style}{}{\endinput}
%\def\FMithmInfo{1995/11/19 v2.2b Theorem extension package (FMi)}
%\ProvidesPackage{theorem}[\FMithmInfo]
%    \end{macrocode}
%
%
% \subsection{Definition of theorem styles and fonts}
%
% All the definitions in this file are done globally to allow
% inputting this file inside a group.
%
% \begin{macro}{\theoremstyle}
%    Before a theorem style can be installed, the chosen style must be
%    known. For that reason, we must test to see that
%    |\th@|\meta{style} is known or, more precisely, that it is
%    different from |\relax|. If the style is not known then
%    |\th@plain| is used.
%    \begin{macrocode}
\gdef\theoremstyle#1{%
   \@ifundefined{th@#1}{\@warning
          {Unknown theoremstyle `#1'. Using `plain'}%
          \theorem@style{plain}}%
%    \end{macrocode}
%    We save the theorem style to be used in the token register
%    |\theorem@style|.
%    \begin{macrocode}
      {\theorem@style{#1}}%
%    \end{macrocode}
%    Now we ``evaluate'' the theorem style: this means, we call the
%    macro |\th@|\meta{style} which will activate the relevant
%    definitions which are contained in a separate file.  This is done
%    in a group to suppress changes to the current font.  This could
%    otherwise pose problems together with the new font selection
%    scheme\footnote{When I printed the original article using the new
%    font selection scheme I ended with a document with slanted
%    typefaces (text headings and all) simply because one of the
%    theorem styles used \texttt{\bslash sl} at toplevel.} if the
%    |\th@|\meta{style} is evaluated a second time.
% \changes{v2.1a}{90/02/15}{Group added around `th@..}
% \changes{v2.1a}{90/02/15}{Reseting of `..@begintheorem removed.}
%    \begin{macrocode}
      \begingroup
        \csname th@\the\theorem@style \endcsname
      \endgroup}
%    \end{macrocode}
% \end{macro}
%
%
% \begin{macro}{\@begintheorem}
% \changes{v2.1a}{90/02/15}{Relaxed at toplevel.}
% \begin{macro}{\@opargbegintheorem}
% \changes{v2.1a}{90/02/15}{Relaxed at toplevel.}
%    We reset |\@begintheorem| and |\@opargbegintheorem| to
%    |\relax| since these commands are no longer necessary at
%    toplevel. This will save a few tokens.
%    \begin{macrocode}
\global\let\@begintheorem\relax
\global\let\@opargbegintheorem\relax
%    \end{macrocode}
% \end{macro}
% \end{macro}
%
% \begin{macro}{\theorem@style}
%    Obviously the \textsf{token} register used above has to be
%    allocated. To assure the utmost compatibility with the
%    original \LaTeX{} definition, we set the default theorem style to
%    \texttt{plain}, which implements the usual \LaTeX{} convention.
% \changes{v2.0f}{89/07/20}{Unnecessary `global' removed.}
%    \begin{macrocode}
\newtoks\theorem@style
\global\theorem@style{plain}
%    \end{macrocode}
% \end{macro}
%
% \begin{macro}{\theorembodyfont}
% \begin{macro}{\theorem@bodyfont}
%    If the |\theorembodyfont| is set by the user then it should not
%    interact with the default font set in the theorem style. When the
%    new font selection is in force this may happen if, for example,
%    the default is |\itshape| and the new |\theorembodyfont| is
%    |\sffamily|. So we add a |\reset@font| command in front of the
%    user definition.
% \changes{v2.0f}{89/07/20}{Unnecessary `global' removed.}
% \changes{v2.0g}{89/09/19}{`theoremfont to `theorembodyfont.}
% \changes{v2.1c}{92/06/22}{`reset@font added and logic changed.}
% \changes{v2.1d}{92/10/28}{Reset to empty if arg is empty}
%    \begin{macrocode}
\gdef\theorembodyfont#1{%
%    \end{macrocode}
%    We check if the argument supplied is empty and if so put nothing
%    into the |\theorem@bodyfont| token register to allow for
%    |\theorembodyfont{}| as a mean of using the default of the
%    current |\theoremstyle|.
%    \begin{macrocode}
   \def\@tempa{#1}%
   \ifx\@tempa\@empty
    \theorem@bodyfont{}%
   \else
    \theorem@bodyfont{\reset@font#1}%
   \fi
   }
\newtoks\theorem@bodyfont
\global\theorem@bodyfont{}
%    \end{macrocode}
% \end{macro}
% \end{macro}
%
%
% \begin{macro}{\theoremheaderfont}
% \changes{v2.0g}{89/09/19}{Macro added.}
%    The font for the theorem headers is handled differently because
%    this definition applies to all theorem styles.
%    \begin{macrocode}
\gdef\theoremheaderfont#1{\gdef\theorem@headerfont{#1}%
%    \end{macrocode}
%    After using the macro once it is redefined to produce an error
%    message.
%    \begin{macrocode}
       \gdef\theoremheaderfont##1{%
        \typeout{\string\theoremheaderfont\space should be used
                 only once.}}}
%    \end{macrocode}
% \end{macro}
%
% \begin{macro}{\theorem@headerfont}
% \changes{v2.0g}{89/09/19}{Macro added.}
%    To set the |\theorem@headerfont| default we first test if
%    the new fontselection scheme is in force.
% \changes{v2.1a}{90/02/15}{Support for new font selection.}
%    \begin{macrocode}
\ifx\upshape\undefined
%    \end{macrocode}
%    If not we define it to expand into |\bfseries|. We don't use
%    |\let| just in case a following style option redefines this
%    macro.
%    \begin{macrocode}
\gdef\theorem@headerfont{\bfseries}
%    \end{macrocode}
%    Otherwise we reset the current shape before calling |\bfseries|.
%    \begin{macrocode}
\else \gdef\theorem@headerfont{\normalfont\bfseries}\fi
%    \end{macrocode}
% \end{macro}
%
% \begin{macro}{\th@plain}
% \begin{macro}{\th@break}
% \begin{macro}{\th@marginbreak}
% \begin{macro}{\th@changebreak}
% \begin{macro}{\th@change}
% \begin{macro}{\th@margin}
%    The different styles are defined in macros such as
%    |\th@plain|.  Since memory space is precious in
%    ``non-Big-versions'', we have to avoid offering too many unused
%    definitions.  Therefore we define these styles in separate files
%    that can be loaded on demand.  Thus the commands themselves only
%    load these files. We use |\@input@|, a \LaTeXe{} internal command
%    that ensures that the file will be listed with |\listfiles|.
% \changes{v2.2a}{94/02/02}{Use `@input@ to load theorem layout files}
%    \begin{macrocode}
\gdef\th@plain{\@input@{thp.sty}}
\gdef\th@break{\@input@{thb.sty}}
\gdef\th@marginbreak{\@input@{thmb.sty}}
\gdef\th@changebreak{\@input@{thcb.sty}}
\gdef\th@change{\@input@{thc.sty}}
\gdef\th@margin{\@input@{thm.sty}}
%    \end{macrocode}
%    This list will be expanded when new styles become
%    available.  For testing,  just append new theorem substyles
%    as document options.
% \end{macro}
% \end{macro}
% \end{macro}
% \end{macro}
% \end{macro}
% \end{macro}
%
%
%
% \subsection{Definition of a new theorem set}
%
% As already pointed out, a new theorem environment can be defined in
% three different ways:
% \begin{verbatim}
%    \newtheorem{Lem}{Lemma}
%    \newtheorem{Lem}{Lemma}[section]
%    \newtheorem{Lem}[Theorem]{Lemma}
%\end{verbatim}
% The function of the macro |\newtheorem| is to recognize these cases
% and then to branch into one of the three macros |\@ynthm|, |\@xnthm|
% or |\@othm|.  This mechanism is adopted unchanged from
% \cite{latex:source}; the essential point here is that, for example,
% in the second case, the arguments \texttt{Lem}, \texttt{Lemma} and
% \texttt{section} are passed over to the macro |\@xnthm|.
%
% We inspect this case first because the others present fewer problems,
% and thus are easily derived from this one.
%
% \begin{macro}{\@xnthm}
%    For our example arguments, the macro |\@xnthm| must fulfill
%    the following:
%    \begin{itemize}
%       \item
%          Define a new \LaTeX-counter `Lem'
%       \item
%          reset this counter within a |\section|
%       \item
%          define the macro |\theLem|
%       \item
%          define the environment macros |\Lem| and
%          |\endLem| using the current
%          |\theoremstyle| and |\theorem@bodyfont|.
%    \end{itemize}
%    Obviously, all this should happen only if the first argument
%    of |\@xnthm| (i.e.\ \texttt{Lem} in our example) is chosen so
%    as not to conflict with any previously defined commands or
%    environments.  This test is performed by the \LaTeX{} macro
%    |\@ifdefinable|.
%    \begin{macrocode}
\gdef\@xnthm#1#2[#3]{\expandafter\@ifdefinable\csname #1\endcsname
%    \end{macrocode}
%    Therefore, the first argument of |\@ifdefinable| is the
%    expansion (in the example, |\Lem|) of
%    |\csname||#1\endcsname|.  The second argument is
%    executed only if the test has been completed successfully.
%    \begin{macrocode}
   {%
%    \end{macrocode}
%    Now we define the new counter. The names of the \LaTeX{}
%    macros employed should speak for themselves:
% \changes{v2.2b}{1995/10/07}{Use \cs{@newctr}}
% \changes{v2.2c}{1995/11/23}{Use \cs{@newctr} properly.}
%    \begin{macrocode}
    \@definecounter{#1}\@newctr{#1}[#3]%
%    \end{macrocode}
%    Using |\@newctr| will give a proper error message if the counter
%    in |#3| is not defined.
%    In defining `|\theLem|' we must generate the desired macro
%    name by use of |\expandafter| and |\csname|.
%    \begin{macrocode}
    \expandafter\xdef\csname the#1\endcsname
%    \end{macrocode}
%    An |\xdef| is used in order to make the definition global,
%    and to ensure that it contains the replacement texts of
%    |\@thmcountersep| and |\@thmcounter|.\footnote{These
%    two macros can be defined by the document style.  Their default
%    values produce a `\texttt{.}' as separation and an arabic
%    representation of the number.} However, not everything should be
%    expanded. For example, it saves space to use |\thesection|
%    instead of its---at times---lengthy expansion.
%    \begin{macrocode}
      {\expandafter \noexpand \csname the#3\endcsname
       \@thmcountersep \@thmcounter{#1}}%
%    \end{macrocode}
%    Thus with the defaults of \LaTeX{}, |\theLem| would be
%    replaced by the command sequence |\thesection.\arabic{Lem}|.
%
%    We will now look at the definition of the macro which is executed
%    at the beginning of the actual environment (in our example this
%    macro is |\Lem|). It should be noted that we use an
%    ``|\expandafter| trick'' to expand only certain parts of the
%    replacement text at the time of the definition.
%    \begin{macrocode}
    \def\@tempa{\global\@namedef{#1}}%
    \expandafter \@tempa \expandafter{%
%    \end{macrocode}
%    First, the macro that contains the current definitions of
%    |\@begintheorem| and |\@opargtheorem| should be called
%    up.  The name of this macro---as is already known---has the form
%    |\th@|\meta{theorem\ style}; therefore, it must be called by
%    \begin{macrocode}
      \csname th@\the \theorem@style
%    \end{macrocode}
%    In addition the default theorem font should be changeable, i.e.\
%    we have to insert the contents of |\theorem@bodyfont|. For
%    that reason, we expand even further, beyond |\endcsname|,
%    and thus insert the contents of the \textsf{token} register
%    |\theorem@bodyfont| in the replacement text.
%    \begin{macrocode}
            \expandafter \endcsname \the \theorem@bodyfont
%    \end{macrocode}
%    Now it is time to call the macro |\@thm| which takes over
%    the further processing.  It has two arguments: the current
%    counter name (in our example, \texttt{Lem}), and the text of the
%    label (in our example, \texttt{Lemma}).
%    \begin{macrocode}
     \@thm{#1}{#2}}%
%    \end{macrocode}
%    With this, the `sub-definition' is complete.  The macro
%    |\@endtheorem| ends a theorem environment and is, so far, nothing
%    but an |\endtrivlist|.  (Hence it is defined globally, and not
%    within the theorem styles.\footnote{This has to be changed as
%    soon as theorem styles that change \texttt{\bslash @endtheorem}
%    exist.  In such a case, all existing styles must be changed as
%    well since they will have to reset the macro.}) Therefore, we can
%    set it equivalent to the macro that ends the theorem set (in our
%    example, |\endLem|).  However, if some day theorem styles exist
%    that do change |\@endtheorem|, we would have to use the
%    commented-out line instead.
%    \begin{macrocode}
    \global \expandafter \let \csname end#1\endcsname \@endtheorem
%   \global\@namedef{end#1}{\@endtheorem}%
%    \end{macrocode}
%    With these commands all the required definitions are employed,
%    unless the test |\@ifdefinable| has failed.  Therefore, we
%    end the second argument of this macro and with it the definition
%    of |\@xnthm|.
%    \begin{macrocode}
   }}
%    \end{macrocode}
% \end{macro}
%
% \begin{macro}{\@ynthm}
%    The definition of |\@ynthm| is completely analogous. In
%    this case the new counter that is defined is not reset within
%    another counter; thus the definition of |\the...|
%    is simplified:
%    \begin{macrocode}
\gdef\@ynthm#1#2{\expandafter\@ifdefinable\csname #1\endcsname
   {\@definecounter{#1}%
    \expandafter\xdef\csname the#1\endcsname{\@thmcounter{#1}}%
%    \end{macrocode}
%    The rest of the definition corresponds literally to that of
%    |\@xnthm|:
%    \begin{macrocode}
    \def\@tempa{\global\@namedef{#1}}\expandafter \@tempa
     \expandafter{\csname th@\the \theorem@style \expandafter
     \endcsname \the\theorem@bodyfont \@thm{#1}{#2}}%
    \global \expandafter \let \csname end#1\endcsname \@endtheorem}}
%    \end{macrocode}
% \end{macro}
%
% \begin{macro}{\@othm}
%    The definition of |\@othm| does not contain anything new.
%    \begin{macrocode}
\gdef\@othm#1[#2]#3{%
%    \end{macrocode}
% \changes{v2.2b}{95/10/07}{Build in check for counter defined}
%    We do not define a new counter but instead use one that has
%    already been defined.  Thus the only definition we need is that
%    of this pseudo-counter (i.e.\ |\the|\meta{env.\ name}).
%    First we check if |#2| corresponds to a known counter name.
%    \begin{macrocode}
  \expandafter\ifx\csname c@#2\endcsname\relax
   \@nocounterr{#2}%
  \else
   \expandafter\@ifdefinable\csname #1\endcsname
   {\expandafter \xdef \csname the#1\endcsname
     {\expandafter \noexpand \csname the#2\endcsname}%
%    \end{macrocode}
%    All other parts of the definition can be adopted from
%    |\@xnthm|. We have to remember, though, that in this case
%    the name of the current counter and the theorem label have moved
%    to the second and third arguments.
%    \begin{macrocode}
    \def\@tempa{\global\@namedef{#1}}\expandafter \@tempa
     \expandafter{\csname th@\the \theorem@style \expandafter
     \endcsname \the\theorem@bodyfont \@thm{#2}{#3}}%
    \global \expandafter \let \csname end#1\endcsname \@endtheorem}%
  \fi}
%    \end{macrocode}
% \end{macro}
%
% \subsection{Macros that are employed in a {\envfont theorem}
%    environment}
%
% \begin{macro}{\@thm}
%    The macro |\@thm| has to increase the current counter.  Then,
%    depending on whether the environment has (or does not have) an
%    optional argument, it has to branch into either |\@begintheorem|
%    or |\@opargtheorem|.
%    \begin{macrocode}
\gdef\@thm#1#2{\refstepcounter{#1}%
%    \end{macrocode}
%    Now we start a {\envfont trivlist} environment, and give
%    |\@topsep| and |\@topsepadd| the values of the \textsf{skip}
%    registers |\theorempreskipamount| and |\theorempostskipamount|.
%    The value in |\@topsep| is the vertical space that is inserted by
%    the first (and only) |\item| in our |\trivlist| whilst
%    |\@topsepadd| is inserted by |\@endparenv| at the end of that
%    {\envfont trivlist} environment.  By using these registers, we
%    obtain the desired space around a {\envfont theorem} environment.
%    \begin{macrocode}
   \trivlist
   \@topsep \theorempreskipamount               % used by first \item
   \@topsepadd \theorempostskipamount           % used by \@endparenv
%    \end{macrocode}
%    Now we have to test whether an optional argument has been given.
% \changes{v2.0d}{89/07/18}{Forgotten percent added.}
%    \begin{macrocode}
   \@ifnextchar [%
%    \end{macrocode}
%    If there is an optional argument, we will call |\@ythm|, and
%    move the arguments read back into the input stream.
%    \begin{macrocode}
   {\@ythm{#1}{#2}}%
%    \end{macrocode}
%    If not, we call |\@begintheorem|. Its first argument is the
%    name of the theorem set (hence the second argument of
%    |\@thm|). Its second argument is the macro that produces the
%    current number.
%    \begin{macrocode}
   {\@begintheorem{#2}{\csname the#1\endcsname}\ignorespaces}}
%    \end{macrocode}
% \end{macro}
%
% \begin{macro}{\@xthm}
% \begin{macro}{\@ythm}
%    Both these macros were originally called by |\@thm|.  We do
%    not need |\@xthm| anymore, hence we reset it to
%    |\relax|. The definition of |\@ythm| has not changed at
%    all from its definition in \LaTeX. In order to make the macros
%    easier to understand, we will
%    nevertheless present it (commented out).
%    \begin{macrocode}
\global\let\@xthm\relax
% \def\@ythm#1#2[#3]{\@opargbegintheorem{#2}{\csname
%           the#1\endcsname}{#3}\ignorespaces}
%</package>
%    \end{macrocode}
%    The primitive |\ignorespaces| in |\@ythm| and
%    |\@thm| is needed to remove the spaces between the
%    |\begin{...}| and the actual text.
% \end{macro}
% \end{macro}
%
%
% \subsection{Definition of the theorem substyles}
%
% As already pointed out, the theorem substyles, defined below, are
% only loaded when necessary.  Note that all these substyles, except
% \texttt{plain}, have |\slshape| as the default body font.
%
% \subsubsection{The \texttt{plain} style}
%
% \label{dok:th@plain} As the following macros use |@|, we have
% to locally set the |\catcode| of this symbol to ``letter''.
% This happens within a group, so that we do not have to worry about
% which |\catcode| that symbol had before.
%    \begin{macrocode}
%<*thp>
\begingroup \makeatletter
%    \end{macrocode}
% Since we are now within a group, we must make all definitions
% globally.  First we make sure that \texttt{theorem.sty} is loaded.
% This will allow us to use this file as a document style option
% without having to call \texttt{theorem} itself as an option.  At the
% same time, we assure that at least version 2 is loaded, since
% |\theorem@style| was not defined in earlier versions.
%    \begin{macrocode}
\@ifundefined{theorem@style}{\input{theorem.sty}}{}
\ProvidesFile{thp.sty}
             [\FMithmInfo]
%    \end{macrocode}
%
%
% \begin{macro}{\th@plain}
% \begin{macro}{\@begintheorem}
% \begin{macro}{\@opargbegintheorem}
%    |\theoremstyle{plain}| corresponds to the original
%    definition, except that the distances to the surrounding text are
%    determined by the parameters |\theorempreskipamount| and
%    |\theorempostskipamount|.
%    First we set the default body font.
%    \begin{macrocode}
\gdef\th@plain{\normalfont\itshape
%    \end{macrocode}
%    Then we define |\@begintheorem| and
%    |\@opargbegintheorem|. These two macros define how the
%    header of a theorem is typeset.  |\@opargbegintheorem| will
%    be called if a \textsf{theorem} environment with an optional
%    argument is encountered; otherwise, the header is constructed by
%    calling |\@begintheorem|.  If one of these macros is
%    executed, we are within a \textsf{trivlist} environment started by
%    |\@thm|.  So the theorem header is produced with an
%    |\item| command.
%
%    Instead of specifying the header font directly, all standard
%    theorem styles use the |\theorem@headerfont| macro to allow
%    customization.  The extra space (|\labelsep|) is necessary
%    because of problems in the \textsf{trivlist} environment.
%    \begin{macrocode}
  \def\@begintheorem##1##2{%
        \item[\hskip\labelsep \theorem@headerfont ##1\ ##2]}%
%    \end{macrocode}
%    The definition of |\@opargbegintheorem| is completely
%    analogous. The only difference is the fact that there exists a
%    third argument (which is the optional parameter of the
%    environment and contains additional information about the
%    theorem).  Customarily we enclose it in parentheses.
%    \begin{macrocode}
\def\@opargbegintheorem##1##2##3{%
   \item[\hskip\labelsep \theorem@headerfont ##1\ ##2\ (##3)]}}
%    \end{macrocode}
% \end{macro}
% \end{macro}
% \end{macro}
%
% \noindent
% We conclude with an |\endgroup| to restore the
% |\catcode| of |@|.
%    \begin{macrocode}
\endgroup
%</thp>
%    \end{macrocode}
%
%
% \subsubsection{The \texttt{break} style}
%
% This style option is stored in the file |thb.sty|.
% For the next two lines see the documentation for |\th@plain| on
% page~\pageref{dok:th@plain}.
%    \begin{macrocode}
%<*thb>
\begingroup \makeatletter
\@ifundefined{theorem@style}{\input{theorem.sty}}{}
\ProvidesFile{thb.sty}
             [\FMithmInfo]
%    \end{macrocode}
%
% \begin{macro}{\th@break}
%    |\theoremstyle{break}| produces a line break after the name
%    of the theorem.  The font is |\slshape|.  Hence, we define
%    |\th@break| as follows:
%    \begin{macrocode}
\gdef\th@break{\normalfont\slshape
  \def\@begintheorem##1##2{\item[%
%    \end{macrocode}
%    We run into the following problem: it is not possible to create
%    the header with |\item[|\meta{title}|]| and then start
%    a new line by, for example, |\mbox{}\\|. Such a definition
%    will fail whenever a list environment follows immediately.  With
%    the above construction, the |\mbox{}| causes the switch
%    |@inlabel| (cf.\ definition of |\list| and
%    |\trivlist| in \cite{latex:source}) to be set to \texttt{false}
%    and so the following list will insert additional vertical
%    space (|\topskip|). This is quite annoying. Therefore, we
%    create the line break within the |\item|. In order to ensure
%    that the text will begin at the proper position in the following
%    line, we simply pretend that the label does not take any
%    room.\footnote{This will lead to problems whenever very high
%    symbols occurring in the line tower into the
%    heading. So, something else has to be done here sometime.}
%    \begin{macrocode}
       \rlap{\vbox{\hbox{\hskip \labelsep\theorem@headerfont ##1\ ##2}%
                   \hbox{\strut}}}]}%
%    \end{macrocode}
%    Again, the definition of |\@opargbegintheorem| is
%    completely analogous.
%    \begin{macrocode}
\def\@opargbegintheorem##1##2##3{%
  \item[\rlap{\vbox{\hbox{\hskip \labelsep \theorem@headerfont
                     ##1\ ##2\ (##3)}%
                    \hbox{\strut}}}]}}
\endgroup
%</thb>
%    \end{macrocode}
% \end{macro}
%
%
% \subsubsection{The \texttt{changebreak} style}
%
%    \begin{macrocode}
% This style option is stored in the file |thcb.sty|.
%    \begin{macrocode}
%<*thcb>
\begingroup \makeatletter
\@ifundefined{theorem@style}{\input{theorem.sty}}{}
\ProvidesFile{thcb.sty}
             [\FMithmInfo]
%    \end{macrocode}
%
% \begin{macro}{\th@changebreak}
%    The \texttt{change-break} theorem style is like \texttt{break}
%    but with interchange of theorem name and theorem number.  Thus we
%    define |\th@changebreak| as follows:
%    \begin{macrocode}
\gdef\th@changebreak{\normalfont\slshape
  \def\@begintheorem##1##2{\item
       [\rlap{\vbox{\hbox{\hskip\labelsep \theorem@headerfont ##2\ ##1}%
                    \hbox{\strut}}}]}%
\def\@opargbegintheorem##1##2##3{%
  \item[\rlap{\vbox{\hbox{\hskip\labelsep \theorem@headerfont
                     ##2\ ##1\ (##3)}%
                    \hbox{\strut}}}]}}
\endgroup
%</thcb>
%    \end{macrocode}
% \end{macro}
%
%
% \subsubsection{The \texttt{change} style}
%
% This style option is stored in the file |thc.sty|.
%    \begin{macrocode}
%<*thc>
\begingroup \makeatletter
\@ifundefined{theorem@style}{\input{theorem.sty}}{}
\ProvidesFile{thc.sty}
             [\FMithmInfo]
%    \end{macrocode}
%
% \begin{macro}{\th@change}
%    The \texttt{change} theorem style corresponds to the
%    \texttt{change break} style without a linebreak after the header.
%    To say it in another way, it's the same as the \texttt{plain}
%    style but with number and name interchanged and |\slshape| as the
%    default font.
%    \begin{macrocode}
\gdef\th@change{\normalfont\slshape
  \def\@begintheorem##1##2{\item
       [\hskip\labelsep \theorem@headerfont ##2\ ##1]}%
\def\@opargbegintheorem##1##2##3{%
  \item[\hskip\labelsep \theorem@headerfont ##2\ ##1\ (##3)]}}
\endgroup
%</thc>
%    \end{macrocode}
% \end{macro}
%
%
% \subsubsection{The \texttt{marginbreak} style}
%
% This style option is the one used most often at Mainz.  It is saved
% in the file |thmb.sty|.
%    \begin{macrocode}
%<*thmb>
\begingroup \makeatletter
\@ifundefined{theorem@style}{\input{theorem.sty}}{}
\ProvidesFile{thmb.sty}
             [\FMithmInfo]
%    \end{macrocode}
%
% \begin{macro}{\th@marginbreak}
%    The \texttt{margin break} style is nearly the same as the
%    \texttt{change break} style. The only difference is the placement
%    of the theorem number.  We use |\llap| to place it in the left
%    margin.
%
%    In this style |\labelsep| denotes the separation between the
%    number and the text.
%    \begin{macrocode}
\gdef\th@marginbreak{\normalfont\slshape
  \def\@begintheorem##1##2{\item
       [\rlap{\vbox{\theorem@headerfont
                    \hbox{\llap{##2}\hskip\labelsep ##1}%
                    \hbox{\strut}}}]}%
\def\@opargbegintheorem##1##2##3{%
  \item[\rlap{\vbox{\theorem@headerfont
                    \hbox{\llap{##2}\hskip\labelsep ##1\ (##3)}%
                    \hbox{\strut}}}]}}
\endgroup
%</thmb>
%    \end{macrocode}
% \changes{v2.1c}{90/02/17}{`endgroup outside macrocode env.}
% \end{macro}
%
%
%
% \subsubsection{The \texttt{margin} style}
%
% This style option is stored in the file |thm.sty|.
%    \begin{macrocode}
%<*thm>
\begingroup \makeatletter
\@ifundefined{theorem@style}{\input{theorem.sty}}{}
\ProvidesFile{thm.sty}
             [\FMithmInfo]
%    \end{macrocode}
%
% \begin{macro}{\th@margin}
%    Again this is only a variant of the theorem styles described
%    above without any new ideas.
%    \begin{macrocode}
\gdef\th@margin{\normalfont\slshape
  \def\@begintheorem##1##2{\item
       [\theorem@headerfont \llap{##2}\hskip\labelsep ##1]}%
\def\@opargbegintheorem##1##2##3{%
  \item[\theorem@headerfont \llap{##2}\hskip\labelsep ##1\ (##3)]}}
\endgroup
%</thm>
%    \end{macrocode}
% \end{macro}
%
%
%
%
% \subsection{Final Definitions}
%
% \begin{macro}{\theorempreskipamount}
% \begin{macro}{\theorempostskipamount}
%    The \textsf{skip} parameters that regulate the vertical empty space
%    before and after the {\envfont theorem} environment have
%    to be allocated as well.
% \changes{v2.0f}{89/07/20}{Unnecessary `global' removed.}
%    \begin{macrocode}
%<*package>
\newskip\theorempreskipamount
\newskip\theorempostskipamount
%    \end{macrocode}
%    Since we have used the same values for all theorem sets, we now can
%    assign them.
%    \begin{macrocode}
\global\setlength\theorempreskipamount{12pt plus 5pt minus 3pt}
\global\setlength\theorempostskipamount{8pt plus 3pt minus 1.5pt}
%    \end{macrocode}
% \end{macro}
% \end{macro}
%
%
% \begin{macro}{\@endtheorem}
%    The same holds for the macro |\@endtheorem|, which ends a
%    {\envfont theorem} environment. Since it is the same for all
%    theorem sets, it is removed from the macros |\th@|\meta{style}.
%    It simply ends the {\envfont trivlist} environment, which was
%    begun in |\@thm|.
%    \begin{macrocode}
\global\let\@endtheorem=\endtrivlist
%    \end{macrocode}
% \end{macro}
%
%
%
% \begin{macro}{\@preamblecmds}
%    All macros defined above are to be used only in the preamble.
%    Therefore, we insert them in |\@preamblecmds| which will disable
%    them at begin document. This is done by the internal \LaTeXe{}
%    command |\@onlypreamble|.
% \changes{v2.0g}{89/09/19}{`theoremheaderfont and `theorembodyfont
%    added.}
% \changes{v2.2a}{94/02/03}{Use \LaTeXe{} interface}
%    \begin{macrocode}
\@onlypreamble\@xnthm
\@onlypreamble\@ynthm
\@onlypreamble\@othm
\@onlypreamble\newtheorem
\@onlypreamble\theoremstyle
\@onlypreamble\theorembodyfont
\@onlypreamble\theoremheaderfont
%    \end{macrocode}
% \end{macro}
%
% Finally we declare the \texttt{plain} theorem style to be the default.
% \changes{v2.1b}{90/06/04}{Fixed missing `item by declaring default
%    style.}
%    \begin{macrocode}
\theoremstyle{plain}
%</package>
%    \end{macrocode}
%
% \Finale

\endinput
}{}
\ProvidesFile{thcb.sty}
             [\FMithmInfo]
%    \end{macrocode}
%
% \begin{macro}{\th@changebreak}
%    The \texttt{change-break} theorem style is like \texttt{break}
%    but with interchange of theorem name and theorem number.  Thus we
%    define |\th@changebreak| as follows:
%    \begin{macrocode}
\gdef\th@changebreak{\normalfont\slshape
  \def\@begintheorem##1##2{\item
       [\rlap{\vbox{\hbox{\hskip\labelsep \theorem@headerfont ##2\ ##1}%
                    \hbox{\strut}}}]}%
\def\@opargbegintheorem##1##2##3{%
  \item[\rlap{\vbox{\hbox{\hskip\labelsep \theorem@headerfont
                     ##2\ ##1\ (##3)}%
                    \hbox{\strut}}}]}}
\endgroup
%</thcb>
%    \end{macrocode}
% \end{macro}
%
%
% \subsubsection{The \texttt{change} style}
%
% This style option is stored in the file |thc.sty|.
%    \begin{macrocode}
%<*thc>
\begingroup \makeatletter
\@ifundefined{theorem@style}{% \iffalse meta-comment
%
% Copyright (C) 1993-2024
%
% The LaTeX Project and any individual authors listed elsewhere
% in this file.
%
% This file is part of the Standard LaTeX `Tools Bundle'.
% -------------------------------------------------------
%
% It may be distributed and/or modified under the
% conditions of the LaTeX Project Public License, either version 1.3c
% of this license or (at your option) any later version.
% The latest version of this license is in
%    https://www.latex-project.org/lppl.txt
% and version 1.3c or later is part of all distributions of LaTeX
% version 2005/12/01 or later.
%
% The list of all files belonging to the LaTeX `Tools Bundle' is
% given in the file `manifest.txt'.
%
% \fi
%
%%
%
%
\def\FMithmInfo{2023/07/05 v2.2c Theorem extension package (FMi)}
%
% \ProvidesFile{theorem.dtx}[\FMithmInfo]
%\iffalse   % this is a METACOMMENT !
%%
%% Package `theorem' to use with LaTeX2e
%% Copyright (C) 1989-1995 Frank Mittelbach, all rights reserved.
%<+package>\@ifundefined{theorem@style}{}{\endinput}
%<+package>\NeedsTeXFormat{LaTeX2e}[1995/06/01]
%<+package>\ProvidesPackage{theorem}
%<+package>         [\FMithmInfo]
%
% \fi
%
% \changes{v1.0}{87/05/07}{First draft.}
% \changes{v1.0a}{87/05/07}{`thistheoremstyle' added.}
% \changes{v1.1a}{87/07/29}{New concept: redefine the internal
%                          \protect\LaTeX{} commands `@xnthm' etc.}
% \changes{v1.1b}{87/07/31}{`theoremfont' used to specify font.}
% \changes{v1.2a}{88/01/08}{`@endtheorem' now globally defined.}
% \changes{v1.2b}{88/01/11}{`newtheorem' only allowed in the preamble.}
% \changes{v1.3a}{88/02/14}{`theoremfont' now a toks register.}
%
% \changes{v2.0a}{88/11/16}{New implementation and new documentation.}
% \changes{v2.0a}{88/11/16}{No layout changes but internal modifications
%                           old substyles won't work.}
% \changes{v2.0b}{89/01/30}{`global' used for all definitions.}
% \changes{v2.0c}{89/06/15}{Documentation translated to English.
%                            Thanks to LLL and Nina}
% \changes{v2.1a}{90/02/15}{Support for new font selection.}
% \changes{v2.2a}{94/02/03}{Cleanup for LaTeX2e}
%
% \DoNotIndex{\@endparenv,\@ifdefinable,\@ifnextchar,\@ifundefined}
% \DoNotIndex{\@namedef,\@nameuse,\@spaces,\@tempa,\@warning}
% \DoNotIndex{\ }
% \DoNotIndex{\begingroup,\bfseries}
% \DoNotIndex{\csname}
% \DoNotIndex{\def,\do,\docdate}
% \DoNotIndex{\endcsname,\endgroup,\endinput,\endtrivlist,\expandafter}
% \DoNotIndex{\filedate,\fileversion}
% \DoNotIndex{\gdef,\global}
% \DoNotIndex{\hbox,\hskip}
% \DoNotIndex{\ignorespaces,\input,\itshape,\item}
% \DoNotIndex{\labelsep,\let,\llap}
% \DoNotIndex{\makeatletter}
% \DoNotIndex{\noexpand}
% \DoNotIndex{\refstepcounter,\relax,\rlap}
% \DoNotIndex{\setlength,\slshape,\space,\string,\strut}
% \DoNotIndex{\the,\trivlist,\typeout}
% \DoNotIndex{\vbox}
% \DoNotIndex{\xdef}
%
% \def\deflabel#1{\ttfamily #1\hfill}
% \def\deflist#1{\begin{list}{}{\settowidth\labelwidth{\ttfamily #1}%
%                               \setlength\leftmargin\labelwidth
%                               \addtolength\leftmargin\labelsep
%                               \let\makelabel\deflabel}}
% \def\enddeflist{\end{list}}
% \def\envfont{\normalfont\ttfamily}
%
% \MakeShortVerb{\|}
%
% \newcommand{\TUB}{{\sl TUGboat\/}}
%
% \GetFileInfo{theorem.dtx}
% \title{An Extension of the
%        \LaTeX{} theorem environment\thanks{This file
%        has version number \fileversion, last
%        revised \filedate.}}
% \author{Frank Mittelbach}
%
% \MaintainedByLaTeXTeam{tools}
% \maketitle
%
% \begin{abstract}
%    The macros described in this paper yield an extension of the
%    \LaTeX{} theorem mechanism. It is designed to satisfy the
%    different requirements of various journals. Thus, the
%    layout of the ``theorems'' can be manipulated by determining a
%    ``style''. This article describes not only the use, but
%    also the definition, of the necessary macros.
% \end{abstract}
%
% \section*{Preface to version 2.2}
%
% For \LaTeXe{} this package did not need any fundamental changes. I
% only modified the messages generated so that theorem layout styles
% will show up with the |\listfiles| command and cleaned the section
% on the New Font Selection Scheme since this is now included in
% \LaTeX.
%
% \section*{Preface to version 2.1}
%
% This version is identical to 2.0g described in \TUB\ 10\#3 except
% for some internal defaults which are now set depending on the used
% font selection scheme.
%
% This was done to avoid unpleasant surprises if the new font selection
% scheme is in force. For further details see
% section~\ref{sec:newfonts} and \cite{bk:GMS94}.
%
% \section{Introduction}
%
% For our purposes here, ``theorems'' are labelled enunciations,
% often set off from the main text by extra space and a font change.
% Theorems, corollaries, conjectures, definitions, and remarks
% are all instances of ``theorems''.  The ``header'' of these structures
% is composed of a label (such as \textsc{Theorem} or \textsc{Remark})
% and a number which serializes an item in the sequence of items
% with the same label.
%
% Shortly after the introduction of \LaTeX{} at the Fachbereich
% Mathematik in Mainz, the desire to manipulate the layout of
% ``theorems'' arose.  In Mainz, the following two conventions came
% into general use:
% \begin{enumerate}
%   \item The number of the theorem is shown in the margin.
%   \item There is a line break at
%         the end of the theorem header.
% \end{enumerate}
%
% Additionally, some journals require different formats which depend on
% the ``sort of theorem'': e.g.\ often remarks and definitions are set
% in |\upshape|, while |\itshape| is employed for main theorems.
%
% Confronted with these requirements, a theorem environment was
% developed in Mainz which allows separate determination of the
% layout of the ``theorems sets'', comparable to |\pagestyle|.
%
%
% \section{The user interface}
% \subsection{Defining new theorem sets}
%
% \DescribeMacro\newtheorem
% As in the original \LaTeX{} version, the command |\newtheorem|
% defines a new ``theorem set'' or ``theorem-like structure''.
% Two required arguments name the new environment and give the text
% to be typeset with each instance of the new ``set'', while
% an optional argument determines how
% the ``set'' is enumerated:
% \begin{description}
%    \item[\ttfamily \bslash newtheorem\{foo\}\{bar\}]
%       The theorem set {\envfont foo} (whose name is \texttt{bar})
%       uses its own counter.
%    \item[\ttfamily \bslash newtheorem\{foo2\}{[foo]}\{bar2\}]
%       The theorem set {\envfont foo2} (printed name \texttt{bar2})
%       uses the same counter as the theorem set \texttt{foo}.
%    \item[\ttfamily \bslash newtheorem\{foo3\}\{bar3\}{[section]}]
%       The theorem set {\envfont foo3} (printed name \texttt{bar3}) is
%       enumerated within the counter \texttt{section}, i.e.\ with every
%       new |\section| the enumeration begins again with 1, and
%       the enumeration is composed from the section-number and the
%       theorem counter itself.
% \end{description}
%
% \DescribeMacro\theoremstyle
% Additionally, the command |\theoremstyle| can define the layout
% of various, or all, theorem sets.  It should be noted that any
% theorem set defined by |\newtheorem| is typeset in the
% |\theoremstyle| that is current at the time of the definition.
% Thus, the following
% \begin{verbatim}
%    \theoremstyle{break}        \newtheorem{Cor}{Corollary}
%    \theoremstyle{plain}        \newtheorem{Exa}{Example}[section]
%\end{verbatim}                             ^^A No blank at line start
% leads to the result that the set {\envfont Cor} is formatted in the
% style \texttt{break}, while the set {\envfont Exa} and all the
% following ones are formatted in the style \texttt{plain}, unless
% another |\theoremstyle| follows.  Since the definitions installed by
% |\newtheorem| are global, one also can limit |\theoremstyle| locally
% by grouping braces.
%
% \DescribeMacro\theorembodyfont
% The choice of the font for the theorem body is completely
% independent of the chosen |\theoremstyle|; this has proven to
% be very advantageous.  For example,
% \begin{verbatim}
%   {\theorembodyfont{\upshape}          \newtheorem{Rem}{Remark}}
%\end{verbatim}
% defines a theorem set {\envfont Rem}, which will be set in
% |\upshape| in the current layout (which in our example is
% \texttt{plain}).  As with |\theoremstyle|, the |\theorembodyfont|
% chosen is that current at the time of |\newtheorem|.  If
% |\theorembodyfont| is not specified or one defines
% |\theorembodyfont{}|, then the font used will be that defined by the
% |\theoremstyle|.
%
% \DescribeMacro\theoremheaderfont It is also possible to customize
% the font used for the theorem headers.  This is, however, a global
% declaration, and therefore there should be at most one
% |\theoremheaderfont| declaration in the preamble.\footnote{If it is
% actually necessary to have different header fonts, one has to define
% new theorem styles (substituting the desired font) or specify the
% information directly in the \texttt{\bslash newtheorem} declaration
% (the unclean variant).}
%
% \DescribeMacro\theorempreskipamount
% \DescribeMacro\theorempostskipamount Two additional parameters
% affect the vertical space around the theorem environments:
% |\theorempreskipamount| and |\theorempostskipamount| define,
% respectively, the spacing before and after such an environment.
% These parameters apply for all theorem sets and can be manipulated
% with the ordinary length macros.  They are rubber lengths
% (`\textsf{skips}'), and therefore can contain \texttt{plus} and
% \texttt{minus} parts.
%
% Since the definition of theorem sets should---most sensibly---be
% placed in the preamble, we only allow installation there.
% It is therefore possible to release the memory used here after
% |\begin{document}|, in order to make room for other
% applications.
%
%
% \subsection{Existing theorem styles}
%
% The following theorem styles exist to date:
% \begin{deflist}{marginbreak}
%    \item[plain]
%       This theorem style emulates the original \LaTeX{} definition,
%       except that additionally the parameters
%       |\theorem...skipamount| are used.
%    \item[break]
%       In this style, the theorem header is followed by a line
%       break.
%    \item[marginbreak]
%       The theorem number is set in the margin, and there is a line
%       break as in \texttt{break}.
%    \item[changebreak]
%       Like \texttt{break}, but with header number and text
%       interchanged.
%    \item[change]
%       Header number and text are interchanged, without a line break.
%    \item[margin]
%       The number is set in the left margin, without a line break.
% \end{deflist}
% All styles (except \texttt{plain}) select |\slshape| as the default
% |\theorembodyfont|.
%
%
% \subsection{Examples}
%
% Given the above theorem sets {\envfont Cor}, {\envfont Exa}
% and {\envfont Rem},
% suppose that the preamble also contains the declarations:
% \begin{verbatim}
%    \theoremstyle{marginbreak}   \newtheorem{Lem}[Cor]{Lemma}
%    \theoremstyle{change}
%    \theorembodyfont{\itshape}        \newtheorem{Def}[Cor]{Definition}
%
%    \theoremheaderfont{\scshape}
%\end{verbatim}
% Then the following are some typical examples of the typeset output
% resulting from their use.
%
% \begin{Cor}
%    This is a sentence typeset in the theorem environment {\envfont
%    Cor}.
% \end{Cor}
%
% \begin{Exa}
%    This is a sentence typeset in the theorem environment {\envfont
%    Exa}.
% \end{Exa}
%
% \begin{Rem}
%    This is a sentence typeset in the theorem environment {\envfont
%    Rem}.
% \end{Rem}
%
% \begin{Lem}[Ben User]
%    This is a sentence typeset in the theorem environment {\envfont
%    Lem}.
% \end{Lem}
%
% \begin{Def}[Very impressive Definition]
%    This is a sentence typeset in the theorem environment {\envfont
%    Def}.
% \end{Def}
%
% The last two examples show the effect of the optional argument to
% a theorem environment (it is the text typeset in parentheses).
%
%
% \section{Special Considerations}
% \label{sec:newfonts}
%
% Theoremheader and body are implemented as a unit. This means that
% the |\theoremheaderfont|  will inherit characteristics of the
% |\theorembodyfont| in \LaTeXe. Thus, if for example
% |\theorembodyfont| is |\itshape| and |\theoremheaderfont|
% is |\bfseries| the font selected for the header will have the
% characteristics `bold extended italic'. If this is not desired one
% should set the |\theoremheaderfont| to something like
% \begin{verbatim}
%      \theoremheaderfont{\normalfont\bfseries}
%\end{verbatim}
% i.e.\ supplying all necessary font information explicitly.
%
%
%
% \section{Acknowledgements}
%
% The publication of this set of macros was only possible with the
% help of Christina Busse (translating the manuscript into English),
% Joachim Pense (playing the r\^ole of typist), Chris Rowley
% (looking everything over) and many others providing useful
% suggestions.
%
% \MaybeStop{%
%    \begin{thebibliography}{9}
%       \bibitem{bk:GMS94} \textsc{M.~Goossens}, \textsc{F.~Mittelbach}
%         and \textsc{A.~Samarin}.
%         \newblock The \LaTeX{} Companion.
%         \newblock
%         Addison-Wesley, Reading, Massachusetts, 1994.
%       \bibitem{latex:source} \textsc{Lamport, Leslie.} \newblock
%                               \texttt{latex.tex}, version 2.09, date
%                               Feb. 1990.
%    \end{thebibliography}
%    \PrintIndex
%    \PrintChanges}
%
%
% \section{The documentation driver file}
%
% The next bit of code contains the documentation driver file for
% \TeX{}, i.e., the file that will produce the documentation you are
% currently reading. It will be extracted from this file by the
% \texttt{docstrip} program. Since it is the first code in the file
% one can alternatively process this file directly with \LaTeXe{} to
% obtain the documentation.
%
%    \begin{macrocode}
%<*driver>
\documentclass{ltxdoc}

\usepackage{theorem}

%<+driver>% The next few lines define theorem sets which are used
%<+driver>% in the example section of the documentation.

\theoremstyle{break}       \newtheorem{Cor}{Corollary}
\theoremstyle{plain}       \newtheorem{Exa}{Example}[section]
{\theorembodyfont{\upshape}\newtheorem{Rem}{Remark}}
\theoremstyle{marginbreak} \newtheorem{Lem}[Cor]{Lemma}
\theoremstyle{change}
\theorembodyfont{\itshape} \newtheorem{Def}[Cor]{Definition}

\theoremheaderfont{\scshape}

\RecordChanges

\begin{document}
   \DocInput{theorem.dtx}
\end{document}
%</driver>
%    \end{macrocode}
%
% \section{Definition of the Macros}
%
% If the file has been loaded before, we abort immediately. If not the
% package announces itself (this is actually done at the very top of
% the file---the way it is done isn't good style so don't copy it).
% \changes{v2.0e}{89/07/19}{Spaces removed from `typeout'.}
% \changes{v2.2b}{95/11/19}{Announce moved to top of file}
%    \begin{macrocode}
%<*package>
%\@ifundefined{theorem@style}{}{\endinput}
%\def\FMithmInfo{1995/11/19 v2.2b Theorem extension package (FMi)}
%\ProvidesPackage{theorem}[\FMithmInfo]
%    \end{macrocode}
%
%
% \subsection{Definition of theorem styles and fonts}
%
% All the definitions in this file are done globally to allow
% inputting this file inside a group.
%
% \begin{macro}{\theoremstyle}
%    Before a theorem style can be installed, the chosen style must be
%    known. For that reason, we must test to see that
%    |\th@|\meta{style} is known or, more precisely, that it is
%    different from |\relax|. If the style is not known then
%    |\th@plain| is used.
%    \begin{macrocode}
\gdef\theoremstyle#1{%
   \@ifundefined{th@#1}{\@warning
          {Unknown theoremstyle `#1'. Using `plain'}%
          \theorem@style{plain}}%
%    \end{macrocode}
%    We save the theorem style to be used in the token register
%    |\theorem@style|.
%    \begin{macrocode}
      {\theorem@style{#1}}%
%    \end{macrocode}
%    Now we ``evaluate'' the theorem style: this means, we call the
%    macro |\th@|\meta{style} which will activate the relevant
%    definitions which are contained in a separate file.  This is done
%    in a group to suppress changes to the current font.  This could
%    otherwise pose problems together with the new font selection
%    scheme\footnote{When I printed the original article using the new
%    font selection scheme I ended with a document with slanted
%    typefaces (text headings and all) simply because one of the
%    theorem styles used \texttt{\bslash sl} at toplevel.} if the
%    |\th@|\meta{style} is evaluated a second time.
% \changes{v2.1a}{90/02/15}{Group added around `th@..}
% \changes{v2.1a}{90/02/15}{Reseting of `..@begintheorem removed.}
%    \begin{macrocode}
      \begingroup
        \csname th@\the\theorem@style \endcsname
      \endgroup}
%    \end{macrocode}
% \end{macro}
%
%
% \begin{macro}{\@begintheorem}
% \changes{v2.1a}{90/02/15}{Relaxed at toplevel.}
% \begin{macro}{\@opargbegintheorem}
% \changes{v2.1a}{90/02/15}{Relaxed at toplevel.}
%    We reset |\@begintheorem| and |\@opargbegintheorem| to
%    |\relax| since these commands are no longer necessary at
%    toplevel. This will save a few tokens.
%    \begin{macrocode}
\global\let\@begintheorem\relax
\global\let\@opargbegintheorem\relax
%    \end{macrocode}
% \end{macro}
% \end{macro}
%
% \begin{macro}{\theorem@style}
%    Obviously the \textsf{token} register used above has to be
%    allocated. To assure the utmost compatibility with the
%    original \LaTeX{} definition, we set the default theorem style to
%    \texttt{plain}, which implements the usual \LaTeX{} convention.
% \changes{v2.0f}{89/07/20}{Unnecessary `global' removed.}
%    \begin{macrocode}
\newtoks\theorem@style
\global\theorem@style{plain}
%    \end{macrocode}
% \end{macro}
%
% \begin{macro}{\theorembodyfont}
% \begin{macro}{\theorem@bodyfont}
%    If the |\theorembodyfont| is set by the user then it should not
%    interact with the default font set in the theorem style. When the
%    new font selection is in force this may happen if, for example,
%    the default is |\itshape| and the new |\theorembodyfont| is
%    |\sffamily|. So we add a |\reset@font| command in front of the
%    user definition.
% \changes{v2.0f}{89/07/20}{Unnecessary `global' removed.}
% \changes{v2.0g}{89/09/19}{`theoremfont to `theorembodyfont.}
% \changes{v2.1c}{92/06/22}{`reset@font added and logic changed.}
% \changes{v2.1d}{92/10/28}{Reset to empty if arg is empty}
%    \begin{macrocode}
\gdef\theorembodyfont#1{%
%    \end{macrocode}
%    We check if the argument supplied is empty and if so put nothing
%    into the |\theorem@bodyfont| token register to allow for
%    |\theorembodyfont{}| as a mean of using the default of the
%    current |\theoremstyle|.
%    \begin{macrocode}
   \def\@tempa{#1}%
   \ifx\@tempa\@empty
    \theorem@bodyfont{}%
   \else
    \theorem@bodyfont{\reset@font#1}%
   \fi
   }
\newtoks\theorem@bodyfont
\global\theorem@bodyfont{}
%    \end{macrocode}
% \end{macro}
% \end{macro}
%
%
% \begin{macro}{\theoremheaderfont}
% \changes{v2.0g}{89/09/19}{Macro added.}
%    The font for the theorem headers is handled differently because
%    this definition applies to all theorem styles.
%    \begin{macrocode}
\gdef\theoremheaderfont#1{\gdef\theorem@headerfont{#1}%
%    \end{macrocode}
%    After using the macro once it is redefined to produce an error
%    message.
%    \begin{macrocode}
       \gdef\theoremheaderfont##1{%
        \typeout{\string\theoremheaderfont\space should be used
                 only once.}}}
%    \end{macrocode}
% \end{macro}
%
% \begin{macro}{\theorem@headerfont}
% \changes{v2.0g}{89/09/19}{Macro added.}
%    To set the |\theorem@headerfont| default we first test if
%    the new fontselection scheme is in force.
% \changes{v2.1a}{90/02/15}{Support for new font selection.}
%    \begin{macrocode}
\ifx\upshape\undefined
%    \end{macrocode}
%    If not we define it to expand into |\bfseries|. We don't use
%    |\let| just in case a following style option redefines this
%    macro.
%    \begin{macrocode}
\gdef\theorem@headerfont{\bfseries}
%    \end{macrocode}
%    Otherwise we reset the current shape before calling |\bfseries|.
%    \begin{macrocode}
\else \gdef\theorem@headerfont{\normalfont\bfseries}\fi
%    \end{macrocode}
% \end{macro}
%
% \begin{macro}{\th@plain}
% \begin{macro}{\th@break}
% \begin{macro}{\th@marginbreak}
% \begin{macro}{\th@changebreak}
% \begin{macro}{\th@change}
% \begin{macro}{\th@margin}
%    The different styles are defined in macros such as
%    |\th@plain|.  Since memory space is precious in
%    ``non-Big-versions'', we have to avoid offering too many unused
%    definitions.  Therefore we define these styles in separate files
%    that can be loaded on demand.  Thus the commands themselves only
%    load these files. We use |\@input@|, a \LaTeXe{} internal command
%    that ensures that the file will be listed with |\listfiles|.
% \changes{v2.2a}{94/02/02}{Use `@input@ to load theorem layout files}
%    \begin{macrocode}
\gdef\th@plain{\@input@{thp.sty}}
\gdef\th@break{\@input@{thb.sty}}
\gdef\th@marginbreak{\@input@{thmb.sty}}
\gdef\th@changebreak{\@input@{thcb.sty}}
\gdef\th@change{\@input@{thc.sty}}
\gdef\th@margin{\@input@{thm.sty}}
%    \end{macrocode}
%    This list will be expanded when new styles become
%    available.  For testing,  just append new theorem substyles
%    as document options.
% \end{macro}
% \end{macro}
% \end{macro}
% \end{macro}
% \end{macro}
% \end{macro}
%
%
%
% \subsection{Definition of a new theorem set}
%
% As already pointed out, a new theorem environment can be defined in
% three different ways:
% \begin{verbatim}
%    \newtheorem{Lem}{Lemma}
%    \newtheorem{Lem}{Lemma}[section]
%    \newtheorem{Lem}[Theorem]{Lemma}
%\end{verbatim}
% The function of the macro |\newtheorem| is to recognize these cases
% and then to branch into one of the three macros |\@ynthm|, |\@xnthm|
% or |\@othm|.  This mechanism is adopted unchanged from
% \cite{latex:source}; the essential point here is that, for example,
% in the second case, the arguments \texttt{Lem}, \texttt{Lemma} and
% \texttt{section} are passed over to the macro |\@xnthm|.
%
% We inspect this case first because the others present fewer problems,
% and thus are easily derived from this one.
%
% \begin{macro}{\@xnthm}
%    For our example arguments, the macro |\@xnthm| must fulfill
%    the following:
%    \begin{itemize}
%       \item
%          Define a new \LaTeX-counter `Lem'
%       \item
%          reset this counter within a |\section|
%       \item
%          define the macro |\theLem|
%       \item
%          define the environment macros |\Lem| and
%          |\endLem| using the current
%          |\theoremstyle| and |\theorem@bodyfont|.
%    \end{itemize}
%    Obviously, all this should happen only if the first argument
%    of |\@xnthm| (i.e.\ \texttt{Lem} in our example) is chosen so
%    as not to conflict with any previously defined commands or
%    environments.  This test is performed by the \LaTeX{} macro
%    |\@ifdefinable|.
%    \begin{macrocode}
\gdef\@xnthm#1#2[#3]{\expandafter\@ifdefinable\csname #1\endcsname
%    \end{macrocode}
%    Therefore, the first argument of |\@ifdefinable| is the
%    expansion (in the example, |\Lem|) of
%    |\csname||#1\endcsname|.  The second argument is
%    executed only if the test has been completed successfully.
%    \begin{macrocode}
   {%
%    \end{macrocode}
%    Now we define the new counter. The names of the \LaTeX{}
%    macros employed should speak for themselves:
% \changes{v2.2b}{1995/10/07}{Use \cs{@newctr}}
% \changes{v2.2c}{1995/11/23}{Use \cs{@newctr} properly.}
%    \begin{macrocode}
    \@definecounter{#1}\@newctr{#1}[#3]%
%    \end{macrocode}
%    Using |\@newctr| will give a proper error message if the counter
%    in |#3| is not defined.
%    In defining `|\theLem|' we must generate the desired macro
%    name by use of |\expandafter| and |\csname|.
%    \begin{macrocode}
    \expandafter\xdef\csname the#1\endcsname
%    \end{macrocode}
%    An |\xdef| is used in order to make the definition global,
%    and to ensure that it contains the replacement texts of
%    |\@thmcountersep| and |\@thmcounter|.\footnote{These
%    two macros can be defined by the document style.  Their default
%    values produce a `\texttt{.}' as separation and an arabic
%    representation of the number.} However, not everything should be
%    expanded. For example, it saves space to use |\thesection|
%    instead of its---at times---lengthy expansion.
%    \begin{macrocode}
      {\expandafter \noexpand \csname the#3\endcsname
       \@thmcountersep \@thmcounter{#1}}%
%    \end{macrocode}
%    Thus with the defaults of \LaTeX{}, |\theLem| would be
%    replaced by the command sequence |\thesection.\arabic{Lem}|.
%
%    We will now look at the definition of the macro which is executed
%    at the beginning of the actual environment (in our example this
%    macro is |\Lem|). It should be noted that we use an
%    ``|\expandafter| trick'' to expand only certain parts of the
%    replacement text at the time of the definition.
%    \begin{macrocode}
    \def\@tempa{\global\@namedef{#1}}%
    \expandafter \@tempa \expandafter{%
%    \end{macrocode}
%    First, the macro that contains the current definitions of
%    |\@begintheorem| and |\@opargtheorem| should be called
%    up.  The name of this macro---as is already known---has the form
%    |\th@|\meta{theorem\ style}; therefore, it must be called by
%    \begin{macrocode}
      \csname th@\the \theorem@style
%    \end{macrocode}
%    In addition the default theorem font should be changeable, i.e.\
%    we have to insert the contents of |\theorem@bodyfont|. For
%    that reason, we expand even further, beyond |\endcsname|,
%    and thus insert the contents of the \textsf{token} register
%    |\theorem@bodyfont| in the replacement text.
%    \begin{macrocode}
            \expandafter \endcsname \the \theorem@bodyfont
%    \end{macrocode}
%    Now it is time to call the macro |\@thm| which takes over
%    the further processing.  It has two arguments: the current
%    counter name (in our example, \texttt{Lem}), and the text of the
%    label (in our example, \texttt{Lemma}).
%    \begin{macrocode}
     \@thm{#1}{#2}}%
%    \end{macrocode}
%    With this, the `sub-definition' is complete.  The macro
%    |\@endtheorem| ends a theorem environment and is, so far, nothing
%    but an |\endtrivlist|.  (Hence it is defined globally, and not
%    within the theorem styles.\footnote{This has to be changed as
%    soon as theorem styles that change \texttt{\bslash @endtheorem}
%    exist.  In such a case, all existing styles must be changed as
%    well since they will have to reset the macro.}) Therefore, we can
%    set it equivalent to the macro that ends the theorem set (in our
%    example, |\endLem|).  However, if some day theorem styles exist
%    that do change |\@endtheorem|, we would have to use the
%    commented-out line instead.
%    \begin{macrocode}
    \global \expandafter \let \csname end#1\endcsname \@endtheorem
%   \global\@namedef{end#1}{\@endtheorem}%
%    \end{macrocode}
%    With these commands all the required definitions are employed,
%    unless the test |\@ifdefinable| has failed.  Therefore, we
%    end the second argument of this macro and with it the definition
%    of |\@xnthm|.
%    \begin{macrocode}
   }}
%    \end{macrocode}
% \end{macro}
%
% \begin{macro}{\@ynthm}
%    The definition of |\@ynthm| is completely analogous. In
%    this case the new counter that is defined is not reset within
%    another counter; thus the definition of |\the...|
%    is simplified:
%    \begin{macrocode}
\gdef\@ynthm#1#2{\expandafter\@ifdefinable\csname #1\endcsname
   {\@definecounter{#1}%
    \expandafter\xdef\csname the#1\endcsname{\@thmcounter{#1}}%
%    \end{macrocode}
%    The rest of the definition corresponds literally to that of
%    |\@xnthm|:
%    \begin{macrocode}
    \def\@tempa{\global\@namedef{#1}}\expandafter \@tempa
     \expandafter{\csname th@\the \theorem@style \expandafter
     \endcsname \the\theorem@bodyfont \@thm{#1}{#2}}%
    \global \expandafter \let \csname end#1\endcsname \@endtheorem}}
%    \end{macrocode}
% \end{macro}
%
% \begin{macro}{\@othm}
%    The definition of |\@othm| does not contain anything new.
%    \begin{macrocode}
\gdef\@othm#1[#2]#3{%
%    \end{macrocode}
% \changes{v2.2b}{95/10/07}{Build in check for counter defined}
%    We do not define a new counter but instead use one that has
%    already been defined.  Thus the only definition we need is that
%    of this pseudo-counter (i.e.\ |\the|\meta{env.\ name}).
%    First we check if |#2| corresponds to a known counter name.
%    \begin{macrocode}
  \expandafter\ifx\csname c@#2\endcsname\relax
   \@nocounterr{#2}%
  \else
   \expandafter\@ifdefinable\csname #1\endcsname
   {\expandafter \xdef \csname the#1\endcsname
     {\expandafter \noexpand \csname the#2\endcsname}%
%    \end{macrocode}
%    All other parts of the definition can be adopted from
%    |\@xnthm|. We have to remember, though, that in this case
%    the name of the current counter and the theorem label have moved
%    to the second and third arguments.
%    \begin{macrocode}
    \def\@tempa{\global\@namedef{#1}}\expandafter \@tempa
     \expandafter{\csname th@\the \theorem@style \expandafter
     \endcsname \the\theorem@bodyfont \@thm{#2}{#3}}%
    \global \expandafter \let \csname end#1\endcsname \@endtheorem}%
  \fi}
%    \end{macrocode}
% \end{macro}
%
% \subsection{Macros that are employed in a {\envfont theorem}
%    environment}
%
% \begin{macro}{\@thm}
%    The macro |\@thm| has to increase the current counter.  Then,
%    depending on whether the environment has (or does not have) an
%    optional argument, it has to branch into either |\@begintheorem|
%    or |\@opargtheorem|.
%    \begin{macrocode}
\gdef\@thm#1#2{\refstepcounter{#1}%
%    \end{macrocode}
%    Now we start a {\envfont trivlist} environment, and give
%    |\@topsep| and |\@topsepadd| the values of the \textsf{skip}
%    registers |\theorempreskipamount| and |\theorempostskipamount|.
%    The value in |\@topsep| is the vertical space that is inserted by
%    the first (and only) |\item| in our |\trivlist| whilst
%    |\@topsepadd| is inserted by |\@endparenv| at the end of that
%    {\envfont trivlist} environment.  By using these registers, we
%    obtain the desired space around a {\envfont theorem} environment.
%    \begin{macrocode}
   \trivlist
   \@topsep \theorempreskipamount               % used by first \item
   \@topsepadd \theorempostskipamount           % used by \@endparenv
%    \end{macrocode}
%    Now we have to test whether an optional argument has been given.
% \changes{v2.0d}{89/07/18}{Forgotten percent added.}
%    \begin{macrocode}
   \@ifnextchar [%
%    \end{macrocode}
%    If there is an optional argument, we will call |\@ythm|, and
%    move the arguments read back into the input stream.
%    \begin{macrocode}
   {\@ythm{#1}{#2}}%
%    \end{macrocode}
%    If not, we call |\@begintheorem|. Its first argument is the
%    name of the theorem set (hence the second argument of
%    |\@thm|). Its second argument is the macro that produces the
%    current number.
%    \begin{macrocode}
   {\@begintheorem{#2}{\csname the#1\endcsname}\ignorespaces}}
%    \end{macrocode}
% \end{macro}
%
% \begin{macro}{\@xthm}
% \begin{macro}{\@ythm}
%    Both these macros were originally called by |\@thm|.  We do
%    not need |\@xthm| anymore, hence we reset it to
%    |\relax|. The definition of |\@ythm| has not changed at
%    all from its definition in \LaTeX. In order to make the macros
%    easier to understand, we will
%    nevertheless present it (commented out).
%    \begin{macrocode}
\global\let\@xthm\relax
% \def\@ythm#1#2[#3]{\@opargbegintheorem{#2}{\csname
%           the#1\endcsname}{#3}\ignorespaces}
%</package>
%    \end{macrocode}
%    The primitive |\ignorespaces| in |\@ythm| and
%    |\@thm| is needed to remove the spaces between the
%    |\begin{...}| and the actual text.
% \end{macro}
% \end{macro}
%
%
% \subsection{Definition of the theorem substyles}
%
% As already pointed out, the theorem substyles, defined below, are
% only loaded when necessary.  Note that all these substyles, except
% \texttt{plain}, have |\slshape| as the default body font.
%
% \subsubsection{The \texttt{plain} style}
%
% \label{dok:th@plain} As the following macros use |@|, we have
% to locally set the |\catcode| of this symbol to ``letter''.
% This happens within a group, so that we do not have to worry about
% which |\catcode| that symbol had before.
%    \begin{macrocode}
%<*thp>
\begingroup \makeatletter
%    \end{macrocode}
% Since we are now within a group, we must make all definitions
% globally.  First we make sure that \texttt{theorem.sty} is loaded.
% This will allow us to use this file as a document style option
% without having to call \texttt{theorem} itself as an option.  At the
% same time, we assure that at least version 2 is loaded, since
% |\theorem@style| was not defined in earlier versions.
%    \begin{macrocode}
\@ifundefined{theorem@style}{\input{theorem.sty}}{}
\ProvidesFile{thp.sty}
             [\FMithmInfo]
%    \end{macrocode}
%
%
% \begin{macro}{\th@plain}
% \begin{macro}{\@begintheorem}
% \begin{macro}{\@opargbegintheorem}
%    |\theoremstyle{plain}| corresponds to the original
%    definition, except that the distances to the surrounding text are
%    determined by the parameters |\theorempreskipamount| and
%    |\theorempostskipamount|.
%    First we set the default body font.
%    \begin{macrocode}
\gdef\th@plain{\normalfont\itshape
%    \end{macrocode}
%    Then we define |\@begintheorem| and
%    |\@opargbegintheorem|. These two macros define how the
%    header of a theorem is typeset.  |\@opargbegintheorem| will
%    be called if a \textsf{theorem} environment with an optional
%    argument is encountered; otherwise, the header is constructed by
%    calling |\@begintheorem|.  If one of these macros is
%    executed, we are within a \textsf{trivlist} environment started by
%    |\@thm|.  So the theorem header is produced with an
%    |\item| command.
%
%    Instead of specifying the header font directly, all standard
%    theorem styles use the |\theorem@headerfont| macro to allow
%    customization.  The extra space (|\labelsep|) is necessary
%    because of problems in the \textsf{trivlist} environment.
%    \begin{macrocode}
  \def\@begintheorem##1##2{%
        \item[\hskip\labelsep \theorem@headerfont ##1\ ##2]}%
%    \end{macrocode}
%    The definition of |\@opargbegintheorem| is completely
%    analogous. The only difference is the fact that there exists a
%    third argument (which is the optional parameter of the
%    environment and contains additional information about the
%    theorem).  Customarily we enclose it in parentheses.
%    \begin{macrocode}
\def\@opargbegintheorem##1##2##3{%
   \item[\hskip\labelsep \theorem@headerfont ##1\ ##2\ (##3)]}}
%    \end{macrocode}
% \end{macro}
% \end{macro}
% \end{macro}
%
% \noindent
% We conclude with an |\endgroup| to restore the
% |\catcode| of |@|.
%    \begin{macrocode}
\endgroup
%</thp>
%    \end{macrocode}
%
%
% \subsubsection{The \texttt{break} style}
%
% This style option is stored in the file |thb.sty|.
% For the next two lines see the documentation for |\th@plain| on
% page~\pageref{dok:th@plain}.
%    \begin{macrocode}
%<*thb>
\begingroup \makeatletter
\@ifundefined{theorem@style}{\input{theorem.sty}}{}
\ProvidesFile{thb.sty}
             [\FMithmInfo]
%    \end{macrocode}
%
% \begin{macro}{\th@break}
%    |\theoremstyle{break}| produces a line break after the name
%    of the theorem.  The font is |\slshape|.  Hence, we define
%    |\th@break| as follows:
%    \begin{macrocode}
\gdef\th@break{\normalfont\slshape
  \def\@begintheorem##1##2{\item[%
%    \end{macrocode}
%    We run into the following problem: it is not possible to create
%    the header with |\item[|\meta{title}|]| and then start
%    a new line by, for example, |\mbox{}\\|. Such a definition
%    will fail whenever a list environment follows immediately.  With
%    the above construction, the |\mbox{}| causes the switch
%    |@inlabel| (cf.\ definition of |\list| and
%    |\trivlist| in \cite{latex:source}) to be set to \texttt{false}
%    and so the following list will insert additional vertical
%    space (|\topskip|). This is quite annoying. Therefore, we
%    create the line break within the |\item|. In order to ensure
%    that the text will begin at the proper position in the following
%    line, we simply pretend that the label does not take any
%    room.\footnote{This will lead to problems whenever very high
%    symbols occurring in the line tower into the
%    heading. So, something else has to be done here sometime.}
%    \begin{macrocode}
       \rlap{\vbox{\hbox{\hskip \labelsep\theorem@headerfont ##1\ ##2}%
                   \hbox{\strut}}}]}%
%    \end{macrocode}
%    Again, the definition of |\@opargbegintheorem| is
%    completely analogous.
%    \begin{macrocode}
\def\@opargbegintheorem##1##2##3{%
  \item[\rlap{\vbox{\hbox{\hskip \labelsep \theorem@headerfont
                     ##1\ ##2\ (##3)}%
                    \hbox{\strut}}}]}}
\endgroup
%</thb>
%    \end{macrocode}
% \end{macro}
%
%
% \subsubsection{The \texttt{changebreak} style}
%
%    \begin{macrocode}
% This style option is stored in the file |thcb.sty|.
%    \begin{macrocode}
%<*thcb>
\begingroup \makeatletter
\@ifundefined{theorem@style}{\input{theorem.sty}}{}
\ProvidesFile{thcb.sty}
             [\FMithmInfo]
%    \end{macrocode}
%
% \begin{macro}{\th@changebreak}
%    The \texttt{change-break} theorem style is like \texttt{break}
%    but with interchange of theorem name and theorem number.  Thus we
%    define |\th@changebreak| as follows:
%    \begin{macrocode}
\gdef\th@changebreak{\normalfont\slshape
  \def\@begintheorem##1##2{\item
       [\rlap{\vbox{\hbox{\hskip\labelsep \theorem@headerfont ##2\ ##1}%
                    \hbox{\strut}}}]}%
\def\@opargbegintheorem##1##2##3{%
  \item[\rlap{\vbox{\hbox{\hskip\labelsep \theorem@headerfont
                     ##2\ ##1\ (##3)}%
                    \hbox{\strut}}}]}}
\endgroup
%</thcb>
%    \end{macrocode}
% \end{macro}
%
%
% \subsubsection{The \texttt{change} style}
%
% This style option is stored in the file |thc.sty|.
%    \begin{macrocode}
%<*thc>
\begingroup \makeatletter
\@ifundefined{theorem@style}{\input{theorem.sty}}{}
\ProvidesFile{thc.sty}
             [\FMithmInfo]
%    \end{macrocode}
%
% \begin{macro}{\th@change}
%    The \texttt{change} theorem style corresponds to the
%    \texttt{change break} style without a linebreak after the header.
%    To say it in another way, it's the same as the \texttt{plain}
%    style but with number and name interchanged and |\slshape| as the
%    default font.
%    \begin{macrocode}
\gdef\th@change{\normalfont\slshape
  \def\@begintheorem##1##2{\item
       [\hskip\labelsep \theorem@headerfont ##2\ ##1]}%
\def\@opargbegintheorem##1##2##3{%
  \item[\hskip\labelsep \theorem@headerfont ##2\ ##1\ (##3)]}}
\endgroup
%</thc>
%    \end{macrocode}
% \end{macro}
%
%
% \subsubsection{The \texttt{marginbreak} style}
%
% This style option is the one used most often at Mainz.  It is saved
% in the file |thmb.sty|.
%    \begin{macrocode}
%<*thmb>
\begingroup \makeatletter
\@ifundefined{theorem@style}{\input{theorem.sty}}{}
\ProvidesFile{thmb.sty}
             [\FMithmInfo]
%    \end{macrocode}
%
% \begin{macro}{\th@marginbreak}
%    The \texttt{margin break} style is nearly the same as the
%    \texttt{change break} style. The only difference is the placement
%    of the theorem number.  We use |\llap| to place it in the left
%    margin.
%
%    In this style |\labelsep| denotes the separation between the
%    number and the text.
%    \begin{macrocode}
\gdef\th@marginbreak{\normalfont\slshape
  \def\@begintheorem##1##2{\item
       [\rlap{\vbox{\theorem@headerfont
                    \hbox{\llap{##2}\hskip\labelsep ##1}%
                    \hbox{\strut}}}]}%
\def\@opargbegintheorem##1##2##3{%
  \item[\rlap{\vbox{\theorem@headerfont
                    \hbox{\llap{##2}\hskip\labelsep ##1\ (##3)}%
                    \hbox{\strut}}}]}}
\endgroup
%</thmb>
%    \end{macrocode}
% \changes{v2.1c}{90/02/17}{`endgroup outside macrocode env.}
% \end{macro}
%
%
%
% \subsubsection{The \texttt{margin} style}
%
% This style option is stored in the file |thm.sty|.
%    \begin{macrocode}
%<*thm>
\begingroup \makeatletter
\@ifundefined{theorem@style}{\input{theorem.sty}}{}
\ProvidesFile{thm.sty}
             [\FMithmInfo]
%    \end{macrocode}
%
% \begin{macro}{\th@margin}
%    Again this is only a variant of the theorem styles described
%    above without any new ideas.
%    \begin{macrocode}
\gdef\th@margin{\normalfont\slshape
  \def\@begintheorem##1##2{\item
       [\theorem@headerfont \llap{##2}\hskip\labelsep ##1]}%
\def\@opargbegintheorem##1##2##3{%
  \item[\theorem@headerfont \llap{##2}\hskip\labelsep ##1\ (##3)]}}
\endgroup
%</thm>
%    \end{macrocode}
% \end{macro}
%
%
%
%
% \subsection{Final Definitions}
%
% \begin{macro}{\theorempreskipamount}
% \begin{macro}{\theorempostskipamount}
%    The \textsf{skip} parameters that regulate the vertical empty space
%    before and after the {\envfont theorem} environment have
%    to be allocated as well.
% \changes{v2.0f}{89/07/20}{Unnecessary `global' removed.}
%    \begin{macrocode}
%<*package>
\newskip\theorempreskipamount
\newskip\theorempostskipamount
%    \end{macrocode}
%    Since we have used the same values for all theorem sets, we now can
%    assign them.
%    \begin{macrocode}
\global\setlength\theorempreskipamount{12pt plus 5pt minus 3pt}
\global\setlength\theorempostskipamount{8pt plus 3pt minus 1.5pt}
%    \end{macrocode}
% \end{macro}
% \end{macro}
%
%
% \begin{macro}{\@endtheorem}
%    The same holds for the macro |\@endtheorem|, which ends a
%    {\envfont theorem} environment. Since it is the same for all
%    theorem sets, it is removed from the macros |\th@|\meta{style}.
%    It simply ends the {\envfont trivlist} environment, which was
%    begun in |\@thm|.
%    \begin{macrocode}
\global\let\@endtheorem=\endtrivlist
%    \end{macrocode}
% \end{macro}
%
%
%
% \begin{macro}{\@preamblecmds}
%    All macros defined above are to be used only in the preamble.
%    Therefore, we insert them in |\@preamblecmds| which will disable
%    them at begin document. This is done by the internal \LaTeXe{}
%    command |\@onlypreamble|.
% \changes{v2.0g}{89/09/19}{`theoremheaderfont and `theorembodyfont
%    added.}
% \changes{v2.2a}{94/02/03}{Use \LaTeXe{} interface}
%    \begin{macrocode}
\@onlypreamble\@xnthm
\@onlypreamble\@ynthm
\@onlypreamble\@othm
\@onlypreamble\newtheorem
\@onlypreamble\theoremstyle
\@onlypreamble\theorembodyfont
\@onlypreamble\theoremheaderfont
%    \end{macrocode}
% \end{macro}
%
% Finally we declare the \texttt{plain} theorem style to be the default.
% \changes{v2.1b}{90/06/04}{Fixed missing `item by declaring default
%    style.}
%    \begin{macrocode}
\theoremstyle{plain}
%</package>
%    \end{macrocode}
%
% \Finale

\endinput
}{}
\ProvidesFile{thc.sty}
             [\FMithmInfo]
%    \end{macrocode}
%
% \begin{macro}{\th@change}
%    The \texttt{change} theorem style corresponds to the
%    \texttt{change break} style without a linebreak after the header.
%    To say it in another way, it's the same as the \texttt{plain}
%    style but with number and name interchanged and |\slshape| as the
%    default font.
%    \begin{macrocode}
\gdef\th@change{\normalfont\slshape
  \def\@begintheorem##1##2{\item
       [\hskip\labelsep \theorem@headerfont ##2\ ##1]}%
\def\@opargbegintheorem##1##2##3{%
  \item[\hskip\labelsep \theorem@headerfont ##2\ ##1\ (##3)]}}
\endgroup
%</thc>
%    \end{macrocode}
% \end{macro}
%
%
% \subsubsection{The \texttt{marginbreak} style}
%
% This style option is the one used most often at Mainz.  It is saved
% in the file |thmb.sty|.
%    \begin{macrocode}
%<*thmb>
\begingroup \makeatletter
\@ifundefined{theorem@style}{% \iffalse meta-comment
%
% Copyright (C) 1993-2024
%
% The LaTeX Project and any individual authors listed elsewhere
% in this file.
%
% This file is part of the Standard LaTeX `Tools Bundle'.
% -------------------------------------------------------
%
% It may be distributed and/or modified under the
% conditions of the LaTeX Project Public License, either version 1.3c
% of this license or (at your option) any later version.
% The latest version of this license is in
%    https://www.latex-project.org/lppl.txt
% and version 1.3c or later is part of all distributions of LaTeX
% version 2005/12/01 or later.
%
% The list of all files belonging to the LaTeX `Tools Bundle' is
% given in the file `manifest.txt'.
%
% \fi
%
%%
%
%
\def\FMithmInfo{2023/07/05 v2.2c Theorem extension package (FMi)}
%
% \ProvidesFile{theorem.dtx}[\FMithmInfo]
%\iffalse   % this is a METACOMMENT !
%%
%% Package `theorem' to use with LaTeX2e
%% Copyright (C) 1989-1995 Frank Mittelbach, all rights reserved.
%<+package>\@ifundefined{theorem@style}{}{\endinput}
%<+package>\NeedsTeXFormat{LaTeX2e}[1995/06/01]
%<+package>\ProvidesPackage{theorem}
%<+package>         [\FMithmInfo]
%
% \fi
%
% \changes{v1.0}{87/05/07}{First draft.}
% \changes{v1.0a}{87/05/07}{`thistheoremstyle' added.}
% \changes{v1.1a}{87/07/29}{New concept: redefine the internal
%                          \protect\LaTeX{} commands `@xnthm' etc.}
% \changes{v1.1b}{87/07/31}{`theoremfont' used to specify font.}
% \changes{v1.2a}{88/01/08}{`@endtheorem' now globally defined.}
% \changes{v1.2b}{88/01/11}{`newtheorem' only allowed in the preamble.}
% \changes{v1.3a}{88/02/14}{`theoremfont' now a toks register.}
%
% \changes{v2.0a}{88/11/16}{New implementation and new documentation.}
% \changes{v2.0a}{88/11/16}{No layout changes but internal modifications
%                           old substyles won't work.}
% \changes{v2.0b}{89/01/30}{`global' used for all definitions.}
% \changes{v2.0c}{89/06/15}{Documentation translated to English.
%                            Thanks to LLL and Nina}
% \changes{v2.1a}{90/02/15}{Support for new font selection.}
% \changes{v2.2a}{94/02/03}{Cleanup for LaTeX2e}
%
% \DoNotIndex{\@endparenv,\@ifdefinable,\@ifnextchar,\@ifundefined}
% \DoNotIndex{\@namedef,\@nameuse,\@spaces,\@tempa,\@warning}
% \DoNotIndex{\ }
% \DoNotIndex{\begingroup,\bfseries}
% \DoNotIndex{\csname}
% \DoNotIndex{\def,\do,\docdate}
% \DoNotIndex{\endcsname,\endgroup,\endinput,\endtrivlist,\expandafter}
% \DoNotIndex{\filedate,\fileversion}
% \DoNotIndex{\gdef,\global}
% \DoNotIndex{\hbox,\hskip}
% \DoNotIndex{\ignorespaces,\input,\itshape,\item}
% \DoNotIndex{\labelsep,\let,\llap}
% \DoNotIndex{\makeatletter}
% \DoNotIndex{\noexpand}
% \DoNotIndex{\refstepcounter,\relax,\rlap}
% \DoNotIndex{\setlength,\slshape,\space,\string,\strut}
% \DoNotIndex{\the,\trivlist,\typeout}
% \DoNotIndex{\vbox}
% \DoNotIndex{\xdef}
%
% \def\deflabel#1{\ttfamily #1\hfill}
% \def\deflist#1{\begin{list}{}{\settowidth\labelwidth{\ttfamily #1}%
%                               \setlength\leftmargin\labelwidth
%                               \addtolength\leftmargin\labelsep
%                               \let\makelabel\deflabel}}
% \def\enddeflist{\end{list}}
% \def\envfont{\normalfont\ttfamily}
%
% \MakeShortVerb{\|}
%
% \newcommand{\TUB}{{\sl TUGboat\/}}
%
% \GetFileInfo{theorem.dtx}
% \title{An Extension of the
%        \LaTeX{} theorem environment\thanks{This file
%        has version number \fileversion, last
%        revised \filedate.}}
% \author{Frank Mittelbach}
%
% \MaintainedByLaTeXTeam{tools}
% \maketitle
%
% \begin{abstract}
%    The macros described in this paper yield an extension of the
%    \LaTeX{} theorem mechanism. It is designed to satisfy the
%    different requirements of various journals. Thus, the
%    layout of the ``theorems'' can be manipulated by determining a
%    ``style''. This article describes not only the use, but
%    also the definition, of the necessary macros.
% \end{abstract}
%
% \section*{Preface to version 2.2}
%
% For \LaTeXe{} this package did not need any fundamental changes. I
% only modified the messages generated so that theorem layout styles
% will show up with the |\listfiles| command and cleaned the section
% on the New Font Selection Scheme since this is now included in
% \LaTeX.
%
% \section*{Preface to version 2.1}
%
% This version is identical to 2.0g described in \TUB\ 10\#3 except
% for some internal defaults which are now set depending on the used
% font selection scheme.
%
% This was done to avoid unpleasant surprises if the new font selection
% scheme is in force. For further details see
% section~\ref{sec:newfonts} and \cite{bk:GMS94}.
%
% \section{Introduction}
%
% For our purposes here, ``theorems'' are labelled enunciations,
% often set off from the main text by extra space and a font change.
% Theorems, corollaries, conjectures, definitions, and remarks
% are all instances of ``theorems''.  The ``header'' of these structures
% is composed of a label (such as \textsc{Theorem} or \textsc{Remark})
% and a number which serializes an item in the sequence of items
% with the same label.
%
% Shortly after the introduction of \LaTeX{} at the Fachbereich
% Mathematik in Mainz, the desire to manipulate the layout of
% ``theorems'' arose.  In Mainz, the following two conventions came
% into general use:
% \begin{enumerate}
%   \item The number of the theorem is shown in the margin.
%   \item There is a line break at
%         the end of the theorem header.
% \end{enumerate}
%
% Additionally, some journals require different formats which depend on
% the ``sort of theorem'': e.g.\ often remarks and definitions are set
% in |\upshape|, while |\itshape| is employed for main theorems.
%
% Confronted with these requirements, a theorem environment was
% developed in Mainz which allows separate determination of the
% layout of the ``theorems sets'', comparable to |\pagestyle|.
%
%
% \section{The user interface}
% \subsection{Defining new theorem sets}
%
% \DescribeMacro\newtheorem
% As in the original \LaTeX{} version, the command |\newtheorem|
% defines a new ``theorem set'' or ``theorem-like structure''.
% Two required arguments name the new environment and give the text
% to be typeset with each instance of the new ``set'', while
% an optional argument determines how
% the ``set'' is enumerated:
% \begin{description}
%    \item[\ttfamily \bslash newtheorem\{foo\}\{bar\}]
%       The theorem set {\envfont foo} (whose name is \texttt{bar})
%       uses its own counter.
%    \item[\ttfamily \bslash newtheorem\{foo2\}{[foo]}\{bar2\}]
%       The theorem set {\envfont foo2} (printed name \texttt{bar2})
%       uses the same counter as the theorem set \texttt{foo}.
%    \item[\ttfamily \bslash newtheorem\{foo3\}\{bar3\}{[section]}]
%       The theorem set {\envfont foo3} (printed name \texttt{bar3}) is
%       enumerated within the counter \texttt{section}, i.e.\ with every
%       new |\section| the enumeration begins again with 1, and
%       the enumeration is composed from the section-number and the
%       theorem counter itself.
% \end{description}
%
% \DescribeMacro\theoremstyle
% Additionally, the command |\theoremstyle| can define the layout
% of various, or all, theorem sets.  It should be noted that any
% theorem set defined by |\newtheorem| is typeset in the
% |\theoremstyle| that is current at the time of the definition.
% Thus, the following
% \begin{verbatim}
%    \theoremstyle{break}        \newtheorem{Cor}{Corollary}
%    \theoremstyle{plain}        \newtheorem{Exa}{Example}[section]
%\end{verbatim}                             ^^A No blank at line start
% leads to the result that the set {\envfont Cor} is formatted in the
% style \texttt{break}, while the set {\envfont Exa} and all the
% following ones are formatted in the style \texttt{plain}, unless
% another |\theoremstyle| follows.  Since the definitions installed by
% |\newtheorem| are global, one also can limit |\theoremstyle| locally
% by grouping braces.
%
% \DescribeMacro\theorembodyfont
% The choice of the font for the theorem body is completely
% independent of the chosen |\theoremstyle|; this has proven to
% be very advantageous.  For example,
% \begin{verbatim}
%   {\theorembodyfont{\upshape}          \newtheorem{Rem}{Remark}}
%\end{verbatim}
% defines a theorem set {\envfont Rem}, which will be set in
% |\upshape| in the current layout (which in our example is
% \texttt{plain}).  As with |\theoremstyle|, the |\theorembodyfont|
% chosen is that current at the time of |\newtheorem|.  If
% |\theorembodyfont| is not specified or one defines
% |\theorembodyfont{}|, then the font used will be that defined by the
% |\theoremstyle|.
%
% \DescribeMacro\theoremheaderfont It is also possible to customize
% the font used for the theorem headers.  This is, however, a global
% declaration, and therefore there should be at most one
% |\theoremheaderfont| declaration in the preamble.\footnote{If it is
% actually necessary to have different header fonts, one has to define
% new theorem styles (substituting the desired font) or specify the
% information directly in the \texttt{\bslash newtheorem} declaration
% (the unclean variant).}
%
% \DescribeMacro\theorempreskipamount
% \DescribeMacro\theorempostskipamount Two additional parameters
% affect the vertical space around the theorem environments:
% |\theorempreskipamount| and |\theorempostskipamount| define,
% respectively, the spacing before and after such an environment.
% These parameters apply for all theorem sets and can be manipulated
% with the ordinary length macros.  They are rubber lengths
% (`\textsf{skips}'), and therefore can contain \texttt{plus} and
% \texttt{minus} parts.
%
% Since the definition of theorem sets should---most sensibly---be
% placed in the preamble, we only allow installation there.
% It is therefore possible to release the memory used here after
% |\begin{document}|, in order to make room for other
% applications.
%
%
% \subsection{Existing theorem styles}
%
% The following theorem styles exist to date:
% \begin{deflist}{marginbreak}
%    \item[plain]
%       This theorem style emulates the original \LaTeX{} definition,
%       except that additionally the parameters
%       |\theorem...skipamount| are used.
%    \item[break]
%       In this style, the theorem header is followed by a line
%       break.
%    \item[marginbreak]
%       The theorem number is set in the margin, and there is a line
%       break as in \texttt{break}.
%    \item[changebreak]
%       Like \texttt{break}, but with header number and text
%       interchanged.
%    \item[change]
%       Header number and text are interchanged, without a line break.
%    \item[margin]
%       The number is set in the left margin, without a line break.
% \end{deflist}
% All styles (except \texttt{plain}) select |\slshape| as the default
% |\theorembodyfont|.
%
%
% \subsection{Examples}
%
% Given the above theorem sets {\envfont Cor}, {\envfont Exa}
% and {\envfont Rem},
% suppose that the preamble also contains the declarations:
% \begin{verbatim}
%    \theoremstyle{marginbreak}   \newtheorem{Lem}[Cor]{Lemma}
%    \theoremstyle{change}
%    \theorembodyfont{\itshape}        \newtheorem{Def}[Cor]{Definition}
%
%    \theoremheaderfont{\scshape}
%\end{verbatim}
% Then the following are some typical examples of the typeset output
% resulting from their use.
%
% \begin{Cor}
%    This is a sentence typeset in the theorem environment {\envfont
%    Cor}.
% \end{Cor}
%
% \begin{Exa}
%    This is a sentence typeset in the theorem environment {\envfont
%    Exa}.
% \end{Exa}
%
% \begin{Rem}
%    This is a sentence typeset in the theorem environment {\envfont
%    Rem}.
% \end{Rem}
%
% \begin{Lem}[Ben User]
%    This is a sentence typeset in the theorem environment {\envfont
%    Lem}.
% \end{Lem}
%
% \begin{Def}[Very impressive Definition]
%    This is a sentence typeset in the theorem environment {\envfont
%    Def}.
% \end{Def}
%
% The last two examples show the effect of the optional argument to
% a theorem environment (it is the text typeset in parentheses).
%
%
% \section{Special Considerations}
% \label{sec:newfonts}
%
% Theoremheader and body are implemented as a unit. This means that
% the |\theoremheaderfont|  will inherit characteristics of the
% |\theorembodyfont| in \LaTeXe. Thus, if for example
% |\theorembodyfont| is |\itshape| and |\theoremheaderfont|
% is |\bfseries| the font selected for the header will have the
% characteristics `bold extended italic'. If this is not desired one
% should set the |\theoremheaderfont| to something like
% \begin{verbatim}
%      \theoremheaderfont{\normalfont\bfseries}
%\end{verbatim}
% i.e.\ supplying all necessary font information explicitly.
%
%
%
% \section{Acknowledgements}
%
% The publication of this set of macros was only possible with the
% help of Christina Busse (translating the manuscript into English),
% Joachim Pense (playing the r\^ole of typist), Chris Rowley
% (looking everything over) and many others providing useful
% suggestions.
%
% \MaybeStop{%
%    \begin{thebibliography}{9}
%       \bibitem{bk:GMS94} \textsc{M.~Goossens}, \textsc{F.~Mittelbach}
%         and \textsc{A.~Samarin}.
%         \newblock The \LaTeX{} Companion.
%         \newblock
%         Addison-Wesley, Reading, Massachusetts, 1994.
%       \bibitem{latex:source} \textsc{Lamport, Leslie.} \newblock
%                               \texttt{latex.tex}, version 2.09, date
%                               Feb. 1990.
%    \end{thebibliography}
%    \PrintIndex
%    \PrintChanges}
%
%
% \section{The documentation driver file}
%
% The next bit of code contains the documentation driver file for
% \TeX{}, i.e., the file that will produce the documentation you are
% currently reading. It will be extracted from this file by the
% \texttt{docstrip} program. Since it is the first code in the file
% one can alternatively process this file directly with \LaTeXe{} to
% obtain the documentation.
%
%    \begin{macrocode}
%<*driver>
\documentclass{ltxdoc}

\usepackage{theorem}

%<+driver>% The next few lines define theorem sets which are used
%<+driver>% in the example section of the documentation.

\theoremstyle{break}       \newtheorem{Cor}{Corollary}
\theoremstyle{plain}       \newtheorem{Exa}{Example}[section]
{\theorembodyfont{\upshape}\newtheorem{Rem}{Remark}}
\theoremstyle{marginbreak} \newtheorem{Lem}[Cor]{Lemma}
\theoremstyle{change}
\theorembodyfont{\itshape} \newtheorem{Def}[Cor]{Definition}

\theoremheaderfont{\scshape}

\RecordChanges

\begin{document}
   \DocInput{theorem.dtx}
\end{document}
%</driver>
%    \end{macrocode}
%
% \section{Definition of the Macros}
%
% If the file has been loaded before, we abort immediately. If not the
% package announces itself (this is actually done at the very top of
% the file---the way it is done isn't good style so don't copy it).
% \changes{v2.0e}{89/07/19}{Spaces removed from `typeout'.}
% \changes{v2.2b}{95/11/19}{Announce moved to top of file}
%    \begin{macrocode}
%<*package>
%\@ifundefined{theorem@style}{}{\endinput}
%\def\FMithmInfo{1995/11/19 v2.2b Theorem extension package (FMi)}
%\ProvidesPackage{theorem}[\FMithmInfo]
%    \end{macrocode}
%
%
% \subsection{Definition of theorem styles and fonts}
%
% All the definitions in this file are done globally to allow
% inputting this file inside a group.
%
% \begin{macro}{\theoremstyle}
%    Before a theorem style can be installed, the chosen style must be
%    known. For that reason, we must test to see that
%    |\th@|\meta{style} is known or, more precisely, that it is
%    different from |\relax|. If the style is not known then
%    |\th@plain| is used.
%    \begin{macrocode}
\gdef\theoremstyle#1{%
   \@ifundefined{th@#1}{\@warning
          {Unknown theoremstyle `#1'. Using `plain'}%
          \theorem@style{plain}}%
%    \end{macrocode}
%    We save the theorem style to be used in the token register
%    |\theorem@style|.
%    \begin{macrocode}
      {\theorem@style{#1}}%
%    \end{macrocode}
%    Now we ``evaluate'' the theorem style: this means, we call the
%    macro |\th@|\meta{style} which will activate the relevant
%    definitions which are contained in a separate file.  This is done
%    in a group to suppress changes to the current font.  This could
%    otherwise pose problems together with the new font selection
%    scheme\footnote{When I printed the original article using the new
%    font selection scheme I ended with a document with slanted
%    typefaces (text headings and all) simply because one of the
%    theorem styles used \texttt{\bslash sl} at toplevel.} if the
%    |\th@|\meta{style} is evaluated a second time.
% \changes{v2.1a}{90/02/15}{Group added around `th@..}
% \changes{v2.1a}{90/02/15}{Reseting of `..@begintheorem removed.}
%    \begin{macrocode}
      \begingroup
        \csname th@\the\theorem@style \endcsname
      \endgroup}
%    \end{macrocode}
% \end{macro}
%
%
% \begin{macro}{\@begintheorem}
% \changes{v2.1a}{90/02/15}{Relaxed at toplevel.}
% \begin{macro}{\@opargbegintheorem}
% \changes{v2.1a}{90/02/15}{Relaxed at toplevel.}
%    We reset |\@begintheorem| and |\@opargbegintheorem| to
%    |\relax| since these commands are no longer necessary at
%    toplevel. This will save a few tokens.
%    \begin{macrocode}
\global\let\@begintheorem\relax
\global\let\@opargbegintheorem\relax
%    \end{macrocode}
% \end{macro}
% \end{macro}
%
% \begin{macro}{\theorem@style}
%    Obviously the \textsf{token} register used above has to be
%    allocated. To assure the utmost compatibility with the
%    original \LaTeX{} definition, we set the default theorem style to
%    \texttt{plain}, which implements the usual \LaTeX{} convention.
% \changes{v2.0f}{89/07/20}{Unnecessary `global' removed.}
%    \begin{macrocode}
\newtoks\theorem@style
\global\theorem@style{plain}
%    \end{macrocode}
% \end{macro}
%
% \begin{macro}{\theorembodyfont}
% \begin{macro}{\theorem@bodyfont}
%    If the |\theorembodyfont| is set by the user then it should not
%    interact with the default font set in the theorem style. When the
%    new font selection is in force this may happen if, for example,
%    the default is |\itshape| and the new |\theorembodyfont| is
%    |\sffamily|. So we add a |\reset@font| command in front of the
%    user definition.
% \changes{v2.0f}{89/07/20}{Unnecessary `global' removed.}
% \changes{v2.0g}{89/09/19}{`theoremfont to `theorembodyfont.}
% \changes{v2.1c}{92/06/22}{`reset@font added and logic changed.}
% \changes{v2.1d}{92/10/28}{Reset to empty if arg is empty}
%    \begin{macrocode}
\gdef\theorembodyfont#1{%
%    \end{macrocode}
%    We check if the argument supplied is empty and if so put nothing
%    into the |\theorem@bodyfont| token register to allow for
%    |\theorembodyfont{}| as a mean of using the default of the
%    current |\theoremstyle|.
%    \begin{macrocode}
   \def\@tempa{#1}%
   \ifx\@tempa\@empty
    \theorem@bodyfont{}%
   \else
    \theorem@bodyfont{\reset@font#1}%
   \fi
   }
\newtoks\theorem@bodyfont
\global\theorem@bodyfont{}
%    \end{macrocode}
% \end{macro}
% \end{macro}
%
%
% \begin{macro}{\theoremheaderfont}
% \changes{v2.0g}{89/09/19}{Macro added.}
%    The font for the theorem headers is handled differently because
%    this definition applies to all theorem styles.
%    \begin{macrocode}
\gdef\theoremheaderfont#1{\gdef\theorem@headerfont{#1}%
%    \end{macrocode}
%    After using the macro once it is redefined to produce an error
%    message.
%    \begin{macrocode}
       \gdef\theoremheaderfont##1{%
        \typeout{\string\theoremheaderfont\space should be used
                 only once.}}}
%    \end{macrocode}
% \end{macro}
%
% \begin{macro}{\theorem@headerfont}
% \changes{v2.0g}{89/09/19}{Macro added.}
%    To set the |\theorem@headerfont| default we first test if
%    the new fontselection scheme is in force.
% \changes{v2.1a}{90/02/15}{Support for new font selection.}
%    \begin{macrocode}
\ifx\upshape\undefined
%    \end{macrocode}
%    If not we define it to expand into |\bfseries|. We don't use
%    |\let| just in case a following style option redefines this
%    macro.
%    \begin{macrocode}
\gdef\theorem@headerfont{\bfseries}
%    \end{macrocode}
%    Otherwise we reset the current shape before calling |\bfseries|.
%    \begin{macrocode}
\else \gdef\theorem@headerfont{\normalfont\bfseries}\fi
%    \end{macrocode}
% \end{macro}
%
% \begin{macro}{\th@plain}
% \begin{macro}{\th@break}
% \begin{macro}{\th@marginbreak}
% \begin{macro}{\th@changebreak}
% \begin{macro}{\th@change}
% \begin{macro}{\th@margin}
%    The different styles are defined in macros such as
%    |\th@plain|.  Since memory space is precious in
%    ``non-Big-versions'', we have to avoid offering too many unused
%    definitions.  Therefore we define these styles in separate files
%    that can be loaded on demand.  Thus the commands themselves only
%    load these files. We use |\@input@|, a \LaTeXe{} internal command
%    that ensures that the file will be listed with |\listfiles|.
% \changes{v2.2a}{94/02/02}{Use `@input@ to load theorem layout files}
%    \begin{macrocode}
\gdef\th@plain{\@input@{thp.sty}}
\gdef\th@break{\@input@{thb.sty}}
\gdef\th@marginbreak{\@input@{thmb.sty}}
\gdef\th@changebreak{\@input@{thcb.sty}}
\gdef\th@change{\@input@{thc.sty}}
\gdef\th@margin{\@input@{thm.sty}}
%    \end{macrocode}
%    This list will be expanded when new styles become
%    available.  For testing,  just append new theorem substyles
%    as document options.
% \end{macro}
% \end{macro}
% \end{macro}
% \end{macro}
% \end{macro}
% \end{macro}
%
%
%
% \subsection{Definition of a new theorem set}
%
% As already pointed out, a new theorem environment can be defined in
% three different ways:
% \begin{verbatim}
%    \newtheorem{Lem}{Lemma}
%    \newtheorem{Lem}{Lemma}[section]
%    \newtheorem{Lem}[Theorem]{Lemma}
%\end{verbatim}
% The function of the macro |\newtheorem| is to recognize these cases
% and then to branch into one of the three macros |\@ynthm|, |\@xnthm|
% or |\@othm|.  This mechanism is adopted unchanged from
% \cite{latex:source}; the essential point here is that, for example,
% in the second case, the arguments \texttt{Lem}, \texttt{Lemma} and
% \texttt{section} are passed over to the macro |\@xnthm|.
%
% We inspect this case first because the others present fewer problems,
% and thus are easily derived from this one.
%
% \begin{macro}{\@xnthm}
%    For our example arguments, the macro |\@xnthm| must fulfill
%    the following:
%    \begin{itemize}
%       \item
%          Define a new \LaTeX-counter `Lem'
%       \item
%          reset this counter within a |\section|
%       \item
%          define the macro |\theLem|
%       \item
%          define the environment macros |\Lem| and
%          |\endLem| using the current
%          |\theoremstyle| and |\theorem@bodyfont|.
%    \end{itemize}
%    Obviously, all this should happen only if the first argument
%    of |\@xnthm| (i.e.\ \texttt{Lem} in our example) is chosen so
%    as not to conflict with any previously defined commands or
%    environments.  This test is performed by the \LaTeX{} macro
%    |\@ifdefinable|.
%    \begin{macrocode}
\gdef\@xnthm#1#2[#3]{\expandafter\@ifdefinable\csname #1\endcsname
%    \end{macrocode}
%    Therefore, the first argument of |\@ifdefinable| is the
%    expansion (in the example, |\Lem|) of
%    |\csname||#1\endcsname|.  The second argument is
%    executed only if the test has been completed successfully.
%    \begin{macrocode}
   {%
%    \end{macrocode}
%    Now we define the new counter. The names of the \LaTeX{}
%    macros employed should speak for themselves:
% \changes{v2.2b}{1995/10/07}{Use \cs{@newctr}}
% \changes{v2.2c}{1995/11/23}{Use \cs{@newctr} properly.}
%    \begin{macrocode}
    \@definecounter{#1}\@newctr{#1}[#3]%
%    \end{macrocode}
%    Using |\@newctr| will give a proper error message if the counter
%    in |#3| is not defined.
%    In defining `|\theLem|' we must generate the desired macro
%    name by use of |\expandafter| and |\csname|.
%    \begin{macrocode}
    \expandafter\xdef\csname the#1\endcsname
%    \end{macrocode}
%    An |\xdef| is used in order to make the definition global,
%    and to ensure that it contains the replacement texts of
%    |\@thmcountersep| and |\@thmcounter|.\footnote{These
%    two macros can be defined by the document style.  Their default
%    values produce a `\texttt{.}' as separation and an arabic
%    representation of the number.} However, not everything should be
%    expanded. For example, it saves space to use |\thesection|
%    instead of its---at times---lengthy expansion.
%    \begin{macrocode}
      {\expandafter \noexpand \csname the#3\endcsname
       \@thmcountersep \@thmcounter{#1}}%
%    \end{macrocode}
%    Thus with the defaults of \LaTeX{}, |\theLem| would be
%    replaced by the command sequence |\thesection.\arabic{Lem}|.
%
%    We will now look at the definition of the macro which is executed
%    at the beginning of the actual environment (in our example this
%    macro is |\Lem|). It should be noted that we use an
%    ``|\expandafter| trick'' to expand only certain parts of the
%    replacement text at the time of the definition.
%    \begin{macrocode}
    \def\@tempa{\global\@namedef{#1}}%
    \expandafter \@tempa \expandafter{%
%    \end{macrocode}
%    First, the macro that contains the current definitions of
%    |\@begintheorem| and |\@opargtheorem| should be called
%    up.  The name of this macro---as is already known---has the form
%    |\th@|\meta{theorem\ style}; therefore, it must be called by
%    \begin{macrocode}
      \csname th@\the \theorem@style
%    \end{macrocode}
%    In addition the default theorem font should be changeable, i.e.\
%    we have to insert the contents of |\theorem@bodyfont|. For
%    that reason, we expand even further, beyond |\endcsname|,
%    and thus insert the contents of the \textsf{token} register
%    |\theorem@bodyfont| in the replacement text.
%    \begin{macrocode}
            \expandafter \endcsname \the \theorem@bodyfont
%    \end{macrocode}
%    Now it is time to call the macro |\@thm| which takes over
%    the further processing.  It has two arguments: the current
%    counter name (in our example, \texttt{Lem}), and the text of the
%    label (in our example, \texttt{Lemma}).
%    \begin{macrocode}
     \@thm{#1}{#2}}%
%    \end{macrocode}
%    With this, the `sub-definition' is complete.  The macro
%    |\@endtheorem| ends a theorem environment and is, so far, nothing
%    but an |\endtrivlist|.  (Hence it is defined globally, and not
%    within the theorem styles.\footnote{This has to be changed as
%    soon as theorem styles that change \texttt{\bslash @endtheorem}
%    exist.  In such a case, all existing styles must be changed as
%    well since they will have to reset the macro.}) Therefore, we can
%    set it equivalent to the macro that ends the theorem set (in our
%    example, |\endLem|).  However, if some day theorem styles exist
%    that do change |\@endtheorem|, we would have to use the
%    commented-out line instead.
%    \begin{macrocode}
    \global \expandafter \let \csname end#1\endcsname \@endtheorem
%   \global\@namedef{end#1}{\@endtheorem}%
%    \end{macrocode}
%    With these commands all the required definitions are employed,
%    unless the test |\@ifdefinable| has failed.  Therefore, we
%    end the second argument of this macro and with it the definition
%    of |\@xnthm|.
%    \begin{macrocode}
   }}
%    \end{macrocode}
% \end{macro}
%
% \begin{macro}{\@ynthm}
%    The definition of |\@ynthm| is completely analogous. In
%    this case the new counter that is defined is not reset within
%    another counter; thus the definition of |\the...|
%    is simplified:
%    \begin{macrocode}
\gdef\@ynthm#1#2{\expandafter\@ifdefinable\csname #1\endcsname
   {\@definecounter{#1}%
    \expandafter\xdef\csname the#1\endcsname{\@thmcounter{#1}}%
%    \end{macrocode}
%    The rest of the definition corresponds literally to that of
%    |\@xnthm|:
%    \begin{macrocode}
    \def\@tempa{\global\@namedef{#1}}\expandafter \@tempa
     \expandafter{\csname th@\the \theorem@style \expandafter
     \endcsname \the\theorem@bodyfont \@thm{#1}{#2}}%
    \global \expandafter \let \csname end#1\endcsname \@endtheorem}}
%    \end{macrocode}
% \end{macro}
%
% \begin{macro}{\@othm}
%    The definition of |\@othm| does not contain anything new.
%    \begin{macrocode}
\gdef\@othm#1[#2]#3{%
%    \end{macrocode}
% \changes{v2.2b}{95/10/07}{Build in check for counter defined}
%    We do not define a new counter but instead use one that has
%    already been defined.  Thus the only definition we need is that
%    of this pseudo-counter (i.e.\ |\the|\meta{env.\ name}).
%    First we check if |#2| corresponds to a known counter name.
%    \begin{macrocode}
  \expandafter\ifx\csname c@#2\endcsname\relax
   \@nocounterr{#2}%
  \else
   \expandafter\@ifdefinable\csname #1\endcsname
   {\expandafter \xdef \csname the#1\endcsname
     {\expandafter \noexpand \csname the#2\endcsname}%
%    \end{macrocode}
%    All other parts of the definition can be adopted from
%    |\@xnthm|. We have to remember, though, that in this case
%    the name of the current counter and the theorem label have moved
%    to the second and third arguments.
%    \begin{macrocode}
    \def\@tempa{\global\@namedef{#1}}\expandafter \@tempa
     \expandafter{\csname th@\the \theorem@style \expandafter
     \endcsname \the\theorem@bodyfont \@thm{#2}{#3}}%
    \global \expandafter \let \csname end#1\endcsname \@endtheorem}%
  \fi}
%    \end{macrocode}
% \end{macro}
%
% \subsection{Macros that are employed in a {\envfont theorem}
%    environment}
%
% \begin{macro}{\@thm}
%    The macro |\@thm| has to increase the current counter.  Then,
%    depending on whether the environment has (or does not have) an
%    optional argument, it has to branch into either |\@begintheorem|
%    or |\@opargtheorem|.
%    \begin{macrocode}
\gdef\@thm#1#2{\refstepcounter{#1}%
%    \end{macrocode}
%    Now we start a {\envfont trivlist} environment, and give
%    |\@topsep| and |\@topsepadd| the values of the \textsf{skip}
%    registers |\theorempreskipamount| and |\theorempostskipamount|.
%    The value in |\@topsep| is the vertical space that is inserted by
%    the first (and only) |\item| in our |\trivlist| whilst
%    |\@topsepadd| is inserted by |\@endparenv| at the end of that
%    {\envfont trivlist} environment.  By using these registers, we
%    obtain the desired space around a {\envfont theorem} environment.
%    \begin{macrocode}
   \trivlist
   \@topsep \theorempreskipamount               % used by first \item
   \@topsepadd \theorempostskipamount           % used by \@endparenv
%    \end{macrocode}
%    Now we have to test whether an optional argument has been given.
% \changes{v2.0d}{89/07/18}{Forgotten percent added.}
%    \begin{macrocode}
   \@ifnextchar [%
%    \end{macrocode}
%    If there is an optional argument, we will call |\@ythm|, and
%    move the arguments read back into the input stream.
%    \begin{macrocode}
   {\@ythm{#1}{#2}}%
%    \end{macrocode}
%    If not, we call |\@begintheorem|. Its first argument is the
%    name of the theorem set (hence the second argument of
%    |\@thm|). Its second argument is the macro that produces the
%    current number.
%    \begin{macrocode}
   {\@begintheorem{#2}{\csname the#1\endcsname}\ignorespaces}}
%    \end{macrocode}
% \end{macro}
%
% \begin{macro}{\@xthm}
% \begin{macro}{\@ythm}
%    Both these macros were originally called by |\@thm|.  We do
%    not need |\@xthm| anymore, hence we reset it to
%    |\relax|. The definition of |\@ythm| has not changed at
%    all from its definition in \LaTeX. In order to make the macros
%    easier to understand, we will
%    nevertheless present it (commented out).
%    \begin{macrocode}
\global\let\@xthm\relax
% \def\@ythm#1#2[#3]{\@opargbegintheorem{#2}{\csname
%           the#1\endcsname}{#3}\ignorespaces}
%</package>
%    \end{macrocode}
%    The primitive |\ignorespaces| in |\@ythm| and
%    |\@thm| is needed to remove the spaces between the
%    |\begin{...}| and the actual text.
% \end{macro}
% \end{macro}
%
%
% \subsection{Definition of the theorem substyles}
%
% As already pointed out, the theorem substyles, defined below, are
% only loaded when necessary.  Note that all these substyles, except
% \texttt{plain}, have |\slshape| as the default body font.
%
% \subsubsection{The \texttt{plain} style}
%
% \label{dok:th@plain} As the following macros use |@|, we have
% to locally set the |\catcode| of this symbol to ``letter''.
% This happens within a group, so that we do not have to worry about
% which |\catcode| that symbol had before.
%    \begin{macrocode}
%<*thp>
\begingroup \makeatletter
%    \end{macrocode}
% Since we are now within a group, we must make all definitions
% globally.  First we make sure that \texttt{theorem.sty} is loaded.
% This will allow us to use this file as a document style option
% without having to call \texttt{theorem} itself as an option.  At the
% same time, we assure that at least version 2 is loaded, since
% |\theorem@style| was not defined in earlier versions.
%    \begin{macrocode}
\@ifundefined{theorem@style}{\input{theorem.sty}}{}
\ProvidesFile{thp.sty}
             [\FMithmInfo]
%    \end{macrocode}
%
%
% \begin{macro}{\th@plain}
% \begin{macro}{\@begintheorem}
% \begin{macro}{\@opargbegintheorem}
%    |\theoremstyle{plain}| corresponds to the original
%    definition, except that the distances to the surrounding text are
%    determined by the parameters |\theorempreskipamount| and
%    |\theorempostskipamount|.
%    First we set the default body font.
%    \begin{macrocode}
\gdef\th@plain{\normalfont\itshape
%    \end{macrocode}
%    Then we define |\@begintheorem| and
%    |\@opargbegintheorem|. These two macros define how the
%    header of a theorem is typeset.  |\@opargbegintheorem| will
%    be called if a \textsf{theorem} environment with an optional
%    argument is encountered; otherwise, the header is constructed by
%    calling |\@begintheorem|.  If one of these macros is
%    executed, we are within a \textsf{trivlist} environment started by
%    |\@thm|.  So the theorem header is produced with an
%    |\item| command.
%
%    Instead of specifying the header font directly, all standard
%    theorem styles use the |\theorem@headerfont| macro to allow
%    customization.  The extra space (|\labelsep|) is necessary
%    because of problems in the \textsf{trivlist} environment.
%    \begin{macrocode}
  \def\@begintheorem##1##2{%
        \item[\hskip\labelsep \theorem@headerfont ##1\ ##2]}%
%    \end{macrocode}
%    The definition of |\@opargbegintheorem| is completely
%    analogous. The only difference is the fact that there exists a
%    third argument (which is the optional parameter of the
%    environment and contains additional information about the
%    theorem).  Customarily we enclose it in parentheses.
%    \begin{macrocode}
\def\@opargbegintheorem##1##2##3{%
   \item[\hskip\labelsep \theorem@headerfont ##1\ ##2\ (##3)]}}
%    \end{macrocode}
% \end{macro}
% \end{macro}
% \end{macro}
%
% \noindent
% We conclude with an |\endgroup| to restore the
% |\catcode| of |@|.
%    \begin{macrocode}
\endgroup
%</thp>
%    \end{macrocode}
%
%
% \subsubsection{The \texttt{break} style}
%
% This style option is stored in the file |thb.sty|.
% For the next two lines see the documentation for |\th@plain| on
% page~\pageref{dok:th@plain}.
%    \begin{macrocode}
%<*thb>
\begingroup \makeatletter
\@ifundefined{theorem@style}{\input{theorem.sty}}{}
\ProvidesFile{thb.sty}
             [\FMithmInfo]
%    \end{macrocode}
%
% \begin{macro}{\th@break}
%    |\theoremstyle{break}| produces a line break after the name
%    of the theorem.  The font is |\slshape|.  Hence, we define
%    |\th@break| as follows:
%    \begin{macrocode}
\gdef\th@break{\normalfont\slshape
  \def\@begintheorem##1##2{\item[%
%    \end{macrocode}
%    We run into the following problem: it is not possible to create
%    the header with |\item[|\meta{title}|]| and then start
%    a new line by, for example, |\mbox{}\\|. Such a definition
%    will fail whenever a list environment follows immediately.  With
%    the above construction, the |\mbox{}| causes the switch
%    |@inlabel| (cf.\ definition of |\list| and
%    |\trivlist| in \cite{latex:source}) to be set to \texttt{false}
%    and so the following list will insert additional vertical
%    space (|\topskip|). This is quite annoying. Therefore, we
%    create the line break within the |\item|. In order to ensure
%    that the text will begin at the proper position in the following
%    line, we simply pretend that the label does not take any
%    room.\footnote{This will lead to problems whenever very high
%    symbols occurring in the line tower into the
%    heading. So, something else has to be done here sometime.}
%    \begin{macrocode}
       \rlap{\vbox{\hbox{\hskip \labelsep\theorem@headerfont ##1\ ##2}%
                   \hbox{\strut}}}]}%
%    \end{macrocode}
%    Again, the definition of |\@opargbegintheorem| is
%    completely analogous.
%    \begin{macrocode}
\def\@opargbegintheorem##1##2##3{%
  \item[\rlap{\vbox{\hbox{\hskip \labelsep \theorem@headerfont
                     ##1\ ##2\ (##3)}%
                    \hbox{\strut}}}]}}
\endgroup
%</thb>
%    \end{macrocode}
% \end{macro}
%
%
% \subsubsection{The \texttt{changebreak} style}
%
%    \begin{macrocode}
% This style option is stored in the file |thcb.sty|.
%    \begin{macrocode}
%<*thcb>
\begingroup \makeatletter
\@ifundefined{theorem@style}{\input{theorem.sty}}{}
\ProvidesFile{thcb.sty}
             [\FMithmInfo]
%    \end{macrocode}
%
% \begin{macro}{\th@changebreak}
%    The \texttt{change-break} theorem style is like \texttt{break}
%    but with interchange of theorem name and theorem number.  Thus we
%    define |\th@changebreak| as follows:
%    \begin{macrocode}
\gdef\th@changebreak{\normalfont\slshape
  \def\@begintheorem##1##2{\item
       [\rlap{\vbox{\hbox{\hskip\labelsep \theorem@headerfont ##2\ ##1}%
                    \hbox{\strut}}}]}%
\def\@opargbegintheorem##1##2##3{%
  \item[\rlap{\vbox{\hbox{\hskip\labelsep \theorem@headerfont
                     ##2\ ##1\ (##3)}%
                    \hbox{\strut}}}]}}
\endgroup
%</thcb>
%    \end{macrocode}
% \end{macro}
%
%
% \subsubsection{The \texttt{change} style}
%
% This style option is stored in the file |thc.sty|.
%    \begin{macrocode}
%<*thc>
\begingroup \makeatletter
\@ifundefined{theorem@style}{\input{theorem.sty}}{}
\ProvidesFile{thc.sty}
             [\FMithmInfo]
%    \end{macrocode}
%
% \begin{macro}{\th@change}
%    The \texttt{change} theorem style corresponds to the
%    \texttt{change break} style without a linebreak after the header.
%    To say it in another way, it's the same as the \texttt{plain}
%    style but with number and name interchanged and |\slshape| as the
%    default font.
%    \begin{macrocode}
\gdef\th@change{\normalfont\slshape
  \def\@begintheorem##1##2{\item
       [\hskip\labelsep \theorem@headerfont ##2\ ##1]}%
\def\@opargbegintheorem##1##2##3{%
  \item[\hskip\labelsep \theorem@headerfont ##2\ ##1\ (##3)]}}
\endgroup
%</thc>
%    \end{macrocode}
% \end{macro}
%
%
% \subsubsection{The \texttt{marginbreak} style}
%
% This style option is the one used most often at Mainz.  It is saved
% in the file |thmb.sty|.
%    \begin{macrocode}
%<*thmb>
\begingroup \makeatletter
\@ifundefined{theorem@style}{\input{theorem.sty}}{}
\ProvidesFile{thmb.sty}
             [\FMithmInfo]
%    \end{macrocode}
%
% \begin{macro}{\th@marginbreak}
%    The \texttt{margin break} style is nearly the same as the
%    \texttt{change break} style. The only difference is the placement
%    of the theorem number.  We use |\llap| to place it in the left
%    margin.
%
%    In this style |\labelsep| denotes the separation between the
%    number and the text.
%    \begin{macrocode}
\gdef\th@marginbreak{\normalfont\slshape
  \def\@begintheorem##1##2{\item
       [\rlap{\vbox{\theorem@headerfont
                    \hbox{\llap{##2}\hskip\labelsep ##1}%
                    \hbox{\strut}}}]}%
\def\@opargbegintheorem##1##2##3{%
  \item[\rlap{\vbox{\theorem@headerfont
                    \hbox{\llap{##2}\hskip\labelsep ##1\ (##3)}%
                    \hbox{\strut}}}]}}
\endgroup
%</thmb>
%    \end{macrocode}
% \changes{v2.1c}{90/02/17}{`endgroup outside macrocode env.}
% \end{macro}
%
%
%
% \subsubsection{The \texttt{margin} style}
%
% This style option is stored in the file |thm.sty|.
%    \begin{macrocode}
%<*thm>
\begingroup \makeatletter
\@ifundefined{theorem@style}{\input{theorem.sty}}{}
\ProvidesFile{thm.sty}
             [\FMithmInfo]
%    \end{macrocode}
%
% \begin{macro}{\th@margin}
%    Again this is only a variant of the theorem styles described
%    above without any new ideas.
%    \begin{macrocode}
\gdef\th@margin{\normalfont\slshape
  \def\@begintheorem##1##2{\item
       [\theorem@headerfont \llap{##2}\hskip\labelsep ##1]}%
\def\@opargbegintheorem##1##2##3{%
  \item[\theorem@headerfont \llap{##2}\hskip\labelsep ##1\ (##3)]}}
\endgroup
%</thm>
%    \end{macrocode}
% \end{macro}
%
%
%
%
% \subsection{Final Definitions}
%
% \begin{macro}{\theorempreskipamount}
% \begin{macro}{\theorempostskipamount}
%    The \textsf{skip} parameters that regulate the vertical empty space
%    before and after the {\envfont theorem} environment have
%    to be allocated as well.
% \changes{v2.0f}{89/07/20}{Unnecessary `global' removed.}
%    \begin{macrocode}
%<*package>
\newskip\theorempreskipamount
\newskip\theorempostskipamount
%    \end{macrocode}
%    Since we have used the same values for all theorem sets, we now can
%    assign them.
%    \begin{macrocode}
\global\setlength\theorempreskipamount{12pt plus 5pt minus 3pt}
\global\setlength\theorempostskipamount{8pt plus 3pt minus 1.5pt}
%    \end{macrocode}
% \end{macro}
% \end{macro}
%
%
% \begin{macro}{\@endtheorem}
%    The same holds for the macro |\@endtheorem|, which ends a
%    {\envfont theorem} environment. Since it is the same for all
%    theorem sets, it is removed from the macros |\th@|\meta{style}.
%    It simply ends the {\envfont trivlist} environment, which was
%    begun in |\@thm|.
%    \begin{macrocode}
\global\let\@endtheorem=\endtrivlist
%    \end{macrocode}
% \end{macro}
%
%
%
% \begin{macro}{\@preamblecmds}
%    All macros defined above are to be used only in the preamble.
%    Therefore, we insert them in |\@preamblecmds| which will disable
%    them at begin document. This is done by the internal \LaTeXe{}
%    command |\@onlypreamble|.
% \changes{v2.0g}{89/09/19}{`theoremheaderfont and `theorembodyfont
%    added.}
% \changes{v2.2a}{94/02/03}{Use \LaTeXe{} interface}
%    \begin{macrocode}
\@onlypreamble\@xnthm
\@onlypreamble\@ynthm
\@onlypreamble\@othm
\@onlypreamble\newtheorem
\@onlypreamble\theoremstyle
\@onlypreamble\theorembodyfont
\@onlypreamble\theoremheaderfont
%    \end{macrocode}
% \end{macro}
%
% Finally we declare the \texttt{plain} theorem style to be the default.
% \changes{v2.1b}{90/06/04}{Fixed missing `item by declaring default
%    style.}
%    \begin{macrocode}
\theoremstyle{plain}
%</package>
%    \end{macrocode}
%
% \Finale

\endinput
}{}
\ProvidesFile{thmb.sty}
             [\FMithmInfo]
%    \end{macrocode}
%
% \begin{macro}{\th@marginbreak}
%    The \texttt{margin break} style is nearly the same as the
%    \texttt{change break} style. The only difference is the placement
%    of the theorem number.  We use |\llap| to place it in the left
%    margin.
%
%    In this style |\labelsep| denotes the separation between the
%    number and the text.
%    \begin{macrocode}
\gdef\th@marginbreak{\normalfont\slshape
  \def\@begintheorem##1##2{\item
       [\rlap{\vbox{\theorem@headerfont
                    \hbox{\llap{##2}\hskip\labelsep ##1}%
                    \hbox{\strut}}}]}%
\def\@opargbegintheorem##1##2##3{%
  \item[\rlap{\vbox{\theorem@headerfont
                    \hbox{\llap{##2}\hskip\labelsep ##1\ (##3)}%
                    \hbox{\strut}}}]}}
\endgroup
%</thmb>
%    \end{macrocode}
% \changes{v2.1c}{90/02/17}{`endgroup outside macrocode env.}
% \end{macro}
%
%
%
% \subsubsection{The \texttt{margin} style}
%
% This style option is stored in the file |thm.sty|.
%    \begin{macrocode}
%<*thm>
\begingroup \makeatletter
\@ifundefined{theorem@style}{% \iffalse meta-comment
%
% Copyright (C) 1993-2024
%
% The LaTeX Project and any individual authors listed elsewhere
% in this file.
%
% This file is part of the Standard LaTeX `Tools Bundle'.
% -------------------------------------------------------
%
% It may be distributed and/or modified under the
% conditions of the LaTeX Project Public License, either version 1.3c
% of this license or (at your option) any later version.
% The latest version of this license is in
%    https://www.latex-project.org/lppl.txt
% and version 1.3c or later is part of all distributions of LaTeX
% version 2005/12/01 or later.
%
% The list of all files belonging to the LaTeX `Tools Bundle' is
% given in the file `manifest.txt'.
%
% \fi
%
%%
%
%
\def\FMithmInfo{2023/07/05 v2.2c Theorem extension package (FMi)}
%
% \ProvidesFile{theorem.dtx}[\FMithmInfo]
%\iffalse   % this is a METACOMMENT !
%%
%% Package `theorem' to use with LaTeX2e
%% Copyright (C) 1989-1995 Frank Mittelbach, all rights reserved.
%<+package>\@ifundefined{theorem@style}{}{\endinput}
%<+package>\NeedsTeXFormat{LaTeX2e}[1995/06/01]
%<+package>\ProvidesPackage{theorem}
%<+package>         [\FMithmInfo]
%
% \fi
%
% \changes{v1.0}{87/05/07}{First draft.}
% \changes{v1.0a}{87/05/07}{`thistheoremstyle' added.}
% \changes{v1.1a}{87/07/29}{New concept: redefine the internal
%                          \protect\LaTeX{} commands `@xnthm' etc.}
% \changes{v1.1b}{87/07/31}{`theoremfont' used to specify font.}
% \changes{v1.2a}{88/01/08}{`@endtheorem' now globally defined.}
% \changes{v1.2b}{88/01/11}{`newtheorem' only allowed in the preamble.}
% \changes{v1.3a}{88/02/14}{`theoremfont' now a toks register.}
%
% \changes{v2.0a}{88/11/16}{New implementation and new documentation.}
% \changes{v2.0a}{88/11/16}{No layout changes but internal modifications
%                           old substyles won't work.}
% \changes{v2.0b}{89/01/30}{`global' used for all definitions.}
% \changes{v2.0c}{89/06/15}{Documentation translated to English.
%                            Thanks to LLL and Nina}
% \changes{v2.1a}{90/02/15}{Support for new font selection.}
% \changes{v2.2a}{94/02/03}{Cleanup for LaTeX2e}
%
% \DoNotIndex{\@endparenv,\@ifdefinable,\@ifnextchar,\@ifundefined}
% \DoNotIndex{\@namedef,\@nameuse,\@spaces,\@tempa,\@warning}
% \DoNotIndex{\ }
% \DoNotIndex{\begingroup,\bfseries}
% \DoNotIndex{\csname}
% \DoNotIndex{\def,\do,\docdate}
% \DoNotIndex{\endcsname,\endgroup,\endinput,\endtrivlist,\expandafter}
% \DoNotIndex{\filedate,\fileversion}
% \DoNotIndex{\gdef,\global}
% \DoNotIndex{\hbox,\hskip}
% \DoNotIndex{\ignorespaces,\input,\itshape,\item}
% \DoNotIndex{\labelsep,\let,\llap}
% \DoNotIndex{\makeatletter}
% \DoNotIndex{\noexpand}
% \DoNotIndex{\refstepcounter,\relax,\rlap}
% \DoNotIndex{\setlength,\slshape,\space,\string,\strut}
% \DoNotIndex{\the,\trivlist,\typeout}
% \DoNotIndex{\vbox}
% \DoNotIndex{\xdef}
%
% \def\deflabel#1{\ttfamily #1\hfill}
% \def\deflist#1{\begin{list}{}{\settowidth\labelwidth{\ttfamily #1}%
%                               \setlength\leftmargin\labelwidth
%                               \addtolength\leftmargin\labelsep
%                               \let\makelabel\deflabel}}
% \def\enddeflist{\end{list}}
% \def\envfont{\normalfont\ttfamily}
%
% \MakeShortVerb{\|}
%
% \newcommand{\TUB}{{\sl TUGboat\/}}
%
% \GetFileInfo{theorem.dtx}
% \title{An Extension of the
%        \LaTeX{} theorem environment\thanks{This file
%        has version number \fileversion, last
%        revised \filedate.}}
% \author{Frank Mittelbach}
%
% \MaintainedByLaTeXTeam{tools}
% \maketitle
%
% \begin{abstract}
%    The macros described in this paper yield an extension of the
%    \LaTeX{} theorem mechanism. It is designed to satisfy the
%    different requirements of various journals. Thus, the
%    layout of the ``theorems'' can be manipulated by determining a
%    ``style''. This article describes not only the use, but
%    also the definition, of the necessary macros.
% \end{abstract}
%
% \section*{Preface to version 2.2}
%
% For \LaTeXe{} this package did not need any fundamental changes. I
% only modified the messages generated so that theorem layout styles
% will show up with the |\listfiles| command and cleaned the section
% on the New Font Selection Scheme since this is now included in
% \LaTeX.
%
% \section*{Preface to version 2.1}
%
% This version is identical to 2.0g described in \TUB\ 10\#3 except
% for some internal defaults which are now set depending on the used
% font selection scheme.
%
% This was done to avoid unpleasant surprises if the new font selection
% scheme is in force. For further details see
% section~\ref{sec:newfonts} and \cite{bk:GMS94}.
%
% \section{Introduction}
%
% For our purposes here, ``theorems'' are labelled enunciations,
% often set off from the main text by extra space and a font change.
% Theorems, corollaries, conjectures, definitions, and remarks
% are all instances of ``theorems''.  The ``header'' of these structures
% is composed of a label (such as \textsc{Theorem} or \textsc{Remark})
% and a number which serializes an item in the sequence of items
% with the same label.
%
% Shortly after the introduction of \LaTeX{} at the Fachbereich
% Mathematik in Mainz, the desire to manipulate the layout of
% ``theorems'' arose.  In Mainz, the following two conventions came
% into general use:
% \begin{enumerate}
%   \item The number of the theorem is shown in the margin.
%   \item There is a line break at
%         the end of the theorem header.
% \end{enumerate}
%
% Additionally, some journals require different formats which depend on
% the ``sort of theorem'': e.g.\ often remarks and definitions are set
% in |\upshape|, while |\itshape| is employed for main theorems.
%
% Confronted with these requirements, a theorem environment was
% developed in Mainz which allows separate determination of the
% layout of the ``theorems sets'', comparable to |\pagestyle|.
%
%
% \section{The user interface}
% \subsection{Defining new theorem sets}
%
% \DescribeMacro\newtheorem
% As in the original \LaTeX{} version, the command |\newtheorem|
% defines a new ``theorem set'' or ``theorem-like structure''.
% Two required arguments name the new environment and give the text
% to be typeset with each instance of the new ``set'', while
% an optional argument determines how
% the ``set'' is enumerated:
% \begin{description}
%    \item[\ttfamily \bslash newtheorem\{foo\}\{bar\}]
%       The theorem set {\envfont foo} (whose name is \texttt{bar})
%       uses its own counter.
%    \item[\ttfamily \bslash newtheorem\{foo2\}{[foo]}\{bar2\}]
%       The theorem set {\envfont foo2} (printed name \texttt{bar2})
%       uses the same counter as the theorem set \texttt{foo}.
%    \item[\ttfamily \bslash newtheorem\{foo3\}\{bar3\}{[section]}]
%       The theorem set {\envfont foo3} (printed name \texttt{bar3}) is
%       enumerated within the counter \texttt{section}, i.e.\ with every
%       new |\section| the enumeration begins again with 1, and
%       the enumeration is composed from the section-number and the
%       theorem counter itself.
% \end{description}
%
% \DescribeMacro\theoremstyle
% Additionally, the command |\theoremstyle| can define the layout
% of various, or all, theorem sets.  It should be noted that any
% theorem set defined by |\newtheorem| is typeset in the
% |\theoremstyle| that is current at the time of the definition.
% Thus, the following
% \begin{verbatim}
%    \theoremstyle{break}        \newtheorem{Cor}{Corollary}
%    \theoremstyle{plain}        \newtheorem{Exa}{Example}[section]
%\end{verbatim}                             ^^A No blank at line start
% leads to the result that the set {\envfont Cor} is formatted in the
% style \texttt{break}, while the set {\envfont Exa} and all the
% following ones are formatted in the style \texttt{plain}, unless
% another |\theoremstyle| follows.  Since the definitions installed by
% |\newtheorem| are global, one also can limit |\theoremstyle| locally
% by grouping braces.
%
% \DescribeMacro\theorembodyfont
% The choice of the font for the theorem body is completely
% independent of the chosen |\theoremstyle|; this has proven to
% be very advantageous.  For example,
% \begin{verbatim}
%   {\theorembodyfont{\upshape}          \newtheorem{Rem}{Remark}}
%\end{verbatim}
% defines a theorem set {\envfont Rem}, which will be set in
% |\upshape| in the current layout (which in our example is
% \texttt{plain}).  As with |\theoremstyle|, the |\theorembodyfont|
% chosen is that current at the time of |\newtheorem|.  If
% |\theorembodyfont| is not specified or one defines
% |\theorembodyfont{}|, then the font used will be that defined by the
% |\theoremstyle|.
%
% \DescribeMacro\theoremheaderfont It is also possible to customize
% the font used for the theorem headers.  This is, however, a global
% declaration, and therefore there should be at most one
% |\theoremheaderfont| declaration in the preamble.\footnote{If it is
% actually necessary to have different header fonts, one has to define
% new theorem styles (substituting the desired font) or specify the
% information directly in the \texttt{\bslash newtheorem} declaration
% (the unclean variant).}
%
% \DescribeMacro\theorempreskipamount
% \DescribeMacro\theorempostskipamount Two additional parameters
% affect the vertical space around the theorem environments:
% |\theorempreskipamount| and |\theorempostskipamount| define,
% respectively, the spacing before and after such an environment.
% These parameters apply for all theorem sets and can be manipulated
% with the ordinary length macros.  They are rubber lengths
% (`\textsf{skips}'), and therefore can contain \texttt{plus} and
% \texttt{minus} parts.
%
% Since the definition of theorem sets should---most sensibly---be
% placed in the preamble, we only allow installation there.
% It is therefore possible to release the memory used here after
% |\begin{document}|, in order to make room for other
% applications.
%
%
% \subsection{Existing theorem styles}
%
% The following theorem styles exist to date:
% \begin{deflist}{marginbreak}
%    \item[plain]
%       This theorem style emulates the original \LaTeX{} definition,
%       except that additionally the parameters
%       |\theorem...skipamount| are used.
%    \item[break]
%       In this style, the theorem header is followed by a line
%       break.
%    \item[marginbreak]
%       The theorem number is set in the margin, and there is a line
%       break as in \texttt{break}.
%    \item[changebreak]
%       Like \texttt{break}, but with header number and text
%       interchanged.
%    \item[change]
%       Header number and text are interchanged, without a line break.
%    \item[margin]
%       The number is set in the left margin, without a line break.
% \end{deflist}
% All styles (except \texttt{plain}) select |\slshape| as the default
% |\theorembodyfont|.
%
%
% \subsection{Examples}
%
% Given the above theorem sets {\envfont Cor}, {\envfont Exa}
% and {\envfont Rem},
% suppose that the preamble also contains the declarations:
% \begin{verbatim}
%    \theoremstyle{marginbreak}   \newtheorem{Lem}[Cor]{Lemma}
%    \theoremstyle{change}
%    \theorembodyfont{\itshape}        \newtheorem{Def}[Cor]{Definition}
%
%    \theoremheaderfont{\scshape}
%\end{verbatim}
% Then the following are some typical examples of the typeset output
% resulting from their use.
%
% \begin{Cor}
%    This is a sentence typeset in the theorem environment {\envfont
%    Cor}.
% \end{Cor}
%
% \begin{Exa}
%    This is a sentence typeset in the theorem environment {\envfont
%    Exa}.
% \end{Exa}
%
% \begin{Rem}
%    This is a sentence typeset in the theorem environment {\envfont
%    Rem}.
% \end{Rem}
%
% \begin{Lem}[Ben User]
%    This is a sentence typeset in the theorem environment {\envfont
%    Lem}.
% \end{Lem}
%
% \begin{Def}[Very impressive Definition]
%    This is a sentence typeset in the theorem environment {\envfont
%    Def}.
% \end{Def}
%
% The last two examples show the effect of the optional argument to
% a theorem environment (it is the text typeset in parentheses).
%
%
% \section{Special Considerations}
% \label{sec:newfonts}
%
% Theoremheader and body are implemented as a unit. This means that
% the |\theoremheaderfont|  will inherit characteristics of the
% |\theorembodyfont| in \LaTeXe. Thus, if for example
% |\theorembodyfont| is |\itshape| and |\theoremheaderfont|
% is |\bfseries| the font selected for the header will have the
% characteristics `bold extended italic'. If this is not desired one
% should set the |\theoremheaderfont| to something like
% \begin{verbatim}
%      \theoremheaderfont{\normalfont\bfseries}
%\end{verbatim}
% i.e.\ supplying all necessary font information explicitly.
%
%
%
% \section{Acknowledgements}
%
% The publication of this set of macros was only possible with the
% help of Christina Busse (translating the manuscript into English),
% Joachim Pense (playing the r\^ole of typist), Chris Rowley
% (looking everything over) and many others providing useful
% suggestions.
%
% \MaybeStop{%
%    \begin{thebibliography}{9}
%       \bibitem{bk:GMS94} \textsc{M.~Goossens}, \textsc{F.~Mittelbach}
%         and \textsc{A.~Samarin}.
%         \newblock The \LaTeX{} Companion.
%         \newblock
%         Addison-Wesley, Reading, Massachusetts, 1994.
%       \bibitem{latex:source} \textsc{Lamport, Leslie.} \newblock
%                               \texttt{latex.tex}, version 2.09, date
%                               Feb. 1990.
%    \end{thebibliography}
%    \PrintIndex
%    \PrintChanges}
%
%
% \section{The documentation driver file}
%
% The next bit of code contains the documentation driver file for
% \TeX{}, i.e., the file that will produce the documentation you are
% currently reading. It will be extracted from this file by the
% \texttt{docstrip} program. Since it is the first code in the file
% one can alternatively process this file directly with \LaTeXe{} to
% obtain the documentation.
%
%    \begin{macrocode}
%<*driver>
\documentclass{ltxdoc}

\usepackage{theorem}

%<+driver>% The next few lines define theorem sets which are used
%<+driver>% in the example section of the documentation.

\theoremstyle{break}       \newtheorem{Cor}{Corollary}
\theoremstyle{plain}       \newtheorem{Exa}{Example}[section]
{\theorembodyfont{\upshape}\newtheorem{Rem}{Remark}}
\theoremstyle{marginbreak} \newtheorem{Lem}[Cor]{Lemma}
\theoremstyle{change}
\theorembodyfont{\itshape} \newtheorem{Def}[Cor]{Definition}

\theoremheaderfont{\scshape}

\RecordChanges

\begin{document}
   \DocInput{theorem.dtx}
\end{document}
%</driver>
%    \end{macrocode}
%
% \section{Definition of the Macros}
%
% If the file has been loaded before, we abort immediately. If not the
% package announces itself (this is actually done at the very top of
% the file---the way it is done isn't good style so don't copy it).
% \changes{v2.0e}{89/07/19}{Spaces removed from `typeout'.}
% \changes{v2.2b}{95/11/19}{Announce moved to top of file}
%    \begin{macrocode}
%<*package>
%\@ifundefined{theorem@style}{}{\endinput}
%\def\FMithmInfo{1995/11/19 v2.2b Theorem extension package (FMi)}
%\ProvidesPackage{theorem}[\FMithmInfo]
%    \end{macrocode}
%
%
% \subsection{Definition of theorem styles and fonts}
%
% All the definitions in this file are done globally to allow
% inputting this file inside a group.
%
% \begin{macro}{\theoremstyle}
%    Before a theorem style can be installed, the chosen style must be
%    known. For that reason, we must test to see that
%    |\th@|\meta{style} is known or, more precisely, that it is
%    different from |\relax|. If the style is not known then
%    |\th@plain| is used.
%    \begin{macrocode}
\gdef\theoremstyle#1{%
   \@ifundefined{th@#1}{\@warning
          {Unknown theoremstyle `#1'. Using `plain'}%
          \theorem@style{plain}}%
%    \end{macrocode}
%    We save the theorem style to be used in the token register
%    |\theorem@style|.
%    \begin{macrocode}
      {\theorem@style{#1}}%
%    \end{macrocode}
%    Now we ``evaluate'' the theorem style: this means, we call the
%    macro |\th@|\meta{style} which will activate the relevant
%    definitions which are contained in a separate file.  This is done
%    in a group to suppress changes to the current font.  This could
%    otherwise pose problems together with the new font selection
%    scheme\footnote{When I printed the original article using the new
%    font selection scheme I ended with a document with slanted
%    typefaces (text headings and all) simply because one of the
%    theorem styles used \texttt{\bslash sl} at toplevel.} if the
%    |\th@|\meta{style} is evaluated a second time.
% \changes{v2.1a}{90/02/15}{Group added around `th@..}
% \changes{v2.1a}{90/02/15}{Reseting of `..@begintheorem removed.}
%    \begin{macrocode}
      \begingroup
        \csname th@\the\theorem@style \endcsname
      \endgroup}
%    \end{macrocode}
% \end{macro}
%
%
% \begin{macro}{\@begintheorem}
% \changes{v2.1a}{90/02/15}{Relaxed at toplevel.}
% \begin{macro}{\@opargbegintheorem}
% \changes{v2.1a}{90/02/15}{Relaxed at toplevel.}
%    We reset |\@begintheorem| and |\@opargbegintheorem| to
%    |\relax| since these commands are no longer necessary at
%    toplevel. This will save a few tokens.
%    \begin{macrocode}
\global\let\@begintheorem\relax
\global\let\@opargbegintheorem\relax
%    \end{macrocode}
% \end{macro}
% \end{macro}
%
% \begin{macro}{\theorem@style}
%    Obviously the \textsf{token} register used above has to be
%    allocated. To assure the utmost compatibility with the
%    original \LaTeX{} definition, we set the default theorem style to
%    \texttt{plain}, which implements the usual \LaTeX{} convention.
% \changes{v2.0f}{89/07/20}{Unnecessary `global' removed.}
%    \begin{macrocode}
\newtoks\theorem@style
\global\theorem@style{plain}
%    \end{macrocode}
% \end{macro}
%
% \begin{macro}{\theorembodyfont}
% \begin{macro}{\theorem@bodyfont}
%    If the |\theorembodyfont| is set by the user then it should not
%    interact with the default font set in the theorem style. When the
%    new font selection is in force this may happen if, for example,
%    the default is |\itshape| and the new |\theorembodyfont| is
%    |\sffamily|. So we add a |\reset@font| command in front of the
%    user definition.
% \changes{v2.0f}{89/07/20}{Unnecessary `global' removed.}
% \changes{v2.0g}{89/09/19}{`theoremfont to `theorembodyfont.}
% \changes{v2.1c}{92/06/22}{`reset@font added and logic changed.}
% \changes{v2.1d}{92/10/28}{Reset to empty if arg is empty}
%    \begin{macrocode}
\gdef\theorembodyfont#1{%
%    \end{macrocode}
%    We check if the argument supplied is empty and if so put nothing
%    into the |\theorem@bodyfont| token register to allow for
%    |\theorembodyfont{}| as a mean of using the default of the
%    current |\theoremstyle|.
%    \begin{macrocode}
   \def\@tempa{#1}%
   \ifx\@tempa\@empty
    \theorem@bodyfont{}%
   \else
    \theorem@bodyfont{\reset@font#1}%
   \fi
   }
\newtoks\theorem@bodyfont
\global\theorem@bodyfont{}
%    \end{macrocode}
% \end{macro}
% \end{macro}
%
%
% \begin{macro}{\theoremheaderfont}
% \changes{v2.0g}{89/09/19}{Macro added.}
%    The font for the theorem headers is handled differently because
%    this definition applies to all theorem styles.
%    \begin{macrocode}
\gdef\theoremheaderfont#1{\gdef\theorem@headerfont{#1}%
%    \end{macrocode}
%    After using the macro once it is redefined to produce an error
%    message.
%    \begin{macrocode}
       \gdef\theoremheaderfont##1{%
        \typeout{\string\theoremheaderfont\space should be used
                 only once.}}}
%    \end{macrocode}
% \end{macro}
%
% \begin{macro}{\theorem@headerfont}
% \changes{v2.0g}{89/09/19}{Macro added.}
%    To set the |\theorem@headerfont| default we first test if
%    the new fontselection scheme is in force.
% \changes{v2.1a}{90/02/15}{Support for new font selection.}
%    \begin{macrocode}
\ifx\upshape\undefined
%    \end{macrocode}
%    If not we define it to expand into |\bfseries|. We don't use
%    |\let| just in case a following style option redefines this
%    macro.
%    \begin{macrocode}
\gdef\theorem@headerfont{\bfseries}
%    \end{macrocode}
%    Otherwise we reset the current shape before calling |\bfseries|.
%    \begin{macrocode}
\else \gdef\theorem@headerfont{\normalfont\bfseries}\fi
%    \end{macrocode}
% \end{macro}
%
% \begin{macro}{\th@plain}
% \begin{macro}{\th@break}
% \begin{macro}{\th@marginbreak}
% \begin{macro}{\th@changebreak}
% \begin{macro}{\th@change}
% \begin{macro}{\th@margin}
%    The different styles are defined in macros such as
%    |\th@plain|.  Since memory space is precious in
%    ``non-Big-versions'', we have to avoid offering too many unused
%    definitions.  Therefore we define these styles in separate files
%    that can be loaded on demand.  Thus the commands themselves only
%    load these files. We use |\@input@|, a \LaTeXe{} internal command
%    that ensures that the file will be listed with |\listfiles|.
% \changes{v2.2a}{94/02/02}{Use `@input@ to load theorem layout files}
%    \begin{macrocode}
\gdef\th@plain{\@input@{thp.sty}}
\gdef\th@break{\@input@{thb.sty}}
\gdef\th@marginbreak{\@input@{thmb.sty}}
\gdef\th@changebreak{\@input@{thcb.sty}}
\gdef\th@change{\@input@{thc.sty}}
\gdef\th@margin{\@input@{thm.sty}}
%    \end{macrocode}
%    This list will be expanded when new styles become
%    available.  For testing,  just append new theorem substyles
%    as document options.
% \end{macro}
% \end{macro}
% \end{macro}
% \end{macro}
% \end{macro}
% \end{macro}
%
%
%
% \subsection{Definition of a new theorem set}
%
% As already pointed out, a new theorem environment can be defined in
% three different ways:
% \begin{verbatim}
%    \newtheorem{Lem}{Lemma}
%    \newtheorem{Lem}{Lemma}[section]
%    \newtheorem{Lem}[Theorem]{Lemma}
%\end{verbatim}
% The function of the macro |\newtheorem| is to recognize these cases
% and then to branch into one of the three macros |\@ynthm|, |\@xnthm|
% or |\@othm|.  This mechanism is adopted unchanged from
% \cite{latex:source}; the essential point here is that, for example,
% in the second case, the arguments \texttt{Lem}, \texttt{Lemma} and
% \texttt{section} are passed over to the macro |\@xnthm|.
%
% We inspect this case first because the others present fewer problems,
% and thus are easily derived from this one.
%
% \begin{macro}{\@xnthm}
%    For our example arguments, the macro |\@xnthm| must fulfill
%    the following:
%    \begin{itemize}
%       \item
%          Define a new \LaTeX-counter `Lem'
%       \item
%          reset this counter within a |\section|
%       \item
%          define the macro |\theLem|
%       \item
%          define the environment macros |\Lem| and
%          |\endLem| using the current
%          |\theoremstyle| and |\theorem@bodyfont|.
%    \end{itemize}
%    Obviously, all this should happen only if the first argument
%    of |\@xnthm| (i.e.\ \texttt{Lem} in our example) is chosen so
%    as not to conflict with any previously defined commands or
%    environments.  This test is performed by the \LaTeX{} macro
%    |\@ifdefinable|.
%    \begin{macrocode}
\gdef\@xnthm#1#2[#3]{\expandafter\@ifdefinable\csname #1\endcsname
%    \end{macrocode}
%    Therefore, the first argument of |\@ifdefinable| is the
%    expansion (in the example, |\Lem|) of
%    |\csname||#1\endcsname|.  The second argument is
%    executed only if the test has been completed successfully.
%    \begin{macrocode}
   {%
%    \end{macrocode}
%    Now we define the new counter. The names of the \LaTeX{}
%    macros employed should speak for themselves:
% \changes{v2.2b}{1995/10/07}{Use \cs{@newctr}}
% \changes{v2.2c}{1995/11/23}{Use \cs{@newctr} properly.}
%    \begin{macrocode}
    \@definecounter{#1}\@newctr{#1}[#3]%
%    \end{macrocode}
%    Using |\@newctr| will give a proper error message if the counter
%    in |#3| is not defined.
%    In defining `|\theLem|' we must generate the desired macro
%    name by use of |\expandafter| and |\csname|.
%    \begin{macrocode}
    \expandafter\xdef\csname the#1\endcsname
%    \end{macrocode}
%    An |\xdef| is used in order to make the definition global,
%    and to ensure that it contains the replacement texts of
%    |\@thmcountersep| and |\@thmcounter|.\footnote{These
%    two macros can be defined by the document style.  Their default
%    values produce a `\texttt{.}' as separation and an arabic
%    representation of the number.} However, not everything should be
%    expanded. For example, it saves space to use |\thesection|
%    instead of its---at times---lengthy expansion.
%    \begin{macrocode}
      {\expandafter \noexpand \csname the#3\endcsname
       \@thmcountersep \@thmcounter{#1}}%
%    \end{macrocode}
%    Thus with the defaults of \LaTeX{}, |\theLem| would be
%    replaced by the command sequence |\thesection.\arabic{Lem}|.
%
%    We will now look at the definition of the macro which is executed
%    at the beginning of the actual environment (in our example this
%    macro is |\Lem|). It should be noted that we use an
%    ``|\expandafter| trick'' to expand only certain parts of the
%    replacement text at the time of the definition.
%    \begin{macrocode}
    \def\@tempa{\global\@namedef{#1}}%
    \expandafter \@tempa \expandafter{%
%    \end{macrocode}
%    First, the macro that contains the current definitions of
%    |\@begintheorem| and |\@opargtheorem| should be called
%    up.  The name of this macro---as is already known---has the form
%    |\th@|\meta{theorem\ style}; therefore, it must be called by
%    \begin{macrocode}
      \csname th@\the \theorem@style
%    \end{macrocode}
%    In addition the default theorem font should be changeable, i.e.\
%    we have to insert the contents of |\theorem@bodyfont|. For
%    that reason, we expand even further, beyond |\endcsname|,
%    and thus insert the contents of the \textsf{token} register
%    |\theorem@bodyfont| in the replacement text.
%    \begin{macrocode}
            \expandafter \endcsname \the \theorem@bodyfont
%    \end{macrocode}
%    Now it is time to call the macro |\@thm| which takes over
%    the further processing.  It has two arguments: the current
%    counter name (in our example, \texttt{Lem}), and the text of the
%    label (in our example, \texttt{Lemma}).
%    \begin{macrocode}
     \@thm{#1}{#2}}%
%    \end{macrocode}
%    With this, the `sub-definition' is complete.  The macro
%    |\@endtheorem| ends a theorem environment and is, so far, nothing
%    but an |\endtrivlist|.  (Hence it is defined globally, and not
%    within the theorem styles.\footnote{This has to be changed as
%    soon as theorem styles that change \texttt{\bslash @endtheorem}
%    exist.  In such a case, all existing styles must be changed as
%    well since they will have to reset the macro.}) Therefore, we can
%    set it equivalent to the macro that ends the theorem set (in our
%    example, |\endLem|).  However, if some day theorem styles exist
%    that do change |\@endtheorem|, we would have to use the
%    commented-out line instead.
%    \begin{macrocode}
    \global \expandafter \let \csname end#1\endcsname \@endtheorem
%   \global\@namedef{end#1}{\@endtheorem}%
%    \end{macrocode}
%    With these commands all the required definitions are employed,
%    unless the test |\@ifdefinable| has failed.  Therefore, we
%    end the second argument of this macro and with it the definition
%    of |\@xnthm|.
%    \begin{macrocode}
   }}
%    \end{macrocode}
% \end{macro}
%
% \begin{macro}{\@ynthm}
%    The definition of |\@ynthm| is completely analogous. In
%    this case the new counter that is defined is not reset within
%    another counter; thus the definition of |\the...|
%    is simplified:
%    \begin{macrocode}
\gdef\@ynthm#1#2{\expandafter\@ifdefinable\csname #1\endcsname
   {\@definecounter{#1}%
    \expandafter\xdef\csname the#1\endcsname{\@thmcounter{#1}}%
%    \end{macrocode}
%    The rest of the definition corresponds literally to that of
%    |\@xnthm|:
%    \begin{macrocode}
    \def\@tempa{\global\@namedef{#1}}\expandafter \@tempa
     \expandafter{\csname th@\the \theorem@style \expandafter
     \endcsname \the\theorem@bodyfont \@thm{#1}{#2}}%
    \global \expandafter \let \csname end#1\endcsname \@endtheorem}}
%    \end{macrocode}
% \end{macro}
%
% \begin{macro}{\@othm}
%    The definition of |\@othm| does not contain anything new.
%    \begin{macrocode}
\gdef\@othm#1[#2]#3{%
%    \end{macrocode}
% \changes{v2.2b}{95/10/07}{Build in check for counter defined}
%    We do not define a new counter but instead use one that has
%    already been defined.  Thus the only definition we need is that
%    of this pseudo-counter (i.e.\ |\the|\meta{env.\ name}).
%    First we check if |#2| corresponds to a known counter name.
%    \begin{macrocode}
  \expandafter\ifx\csname c@#2\endcsname\relax
   \@nocounterr{#2}%
  \else
   \expandafter\@ifdefinable\csname #1\endcsname
   {\expandafter \xdef \csname the#1\endcsname
     {\expandafter \noexpand \csname the#2\endcsname}%
%    \end{macrocode}
%    All other parts of the definition can be adopted from
%    |\@xnthm|. We have to remember, though, that in this case
%    the name of the current counter and the theorem label have moved
%    to the second and third arguments.
%    \begin{macrocode}
    \def\@tempa{\global\@namedef{#1}}\expandafter \@tempa
     \expandafter{\csname th@\the \theorem@style \expandafter
     \endcsname \the\theorem@bodyfont \@thm{#2}{#3}}%
    \global \expandafter \let \csname end#1\endcsname \@endtheorem}%
  \fi}
%    \end{macrocode}
% \end{macro}
%
% \subsection{Macros that are employed in a {\envfont theorem}
%    environment}
%
% \begin{macro}{\@thm}
%    The macro |\@thm| has to increase the current counter.  Then,
%    depending on whether the environment has (or does not have) an
%    optional argument, it has to branch into either |\@begintheorem|
%    or |\@opargtheorem|.
%    \begin{macrocode}
\gdef\@thm#1#2{\refstepcounter{#1}%
%    \end{macrocode}
%    Now we start a {\envfont trivlist} environment, and give
%    |\@topsep| and |\@topsepadd| the values of the \textsf{skip}
%    registers |\theorempreskipamount| and |\theorempostskipamount|.
%    The value in |\@topsep| is the vertical space that is inserted by
%    the first (and only) |\item| in our |\trivlist| whilst
%    |\@topsepadd| is inserted by |\@endparenv| at the end of that
%    {\envfont trivlist} environment.  By using these registers, we
%    obtain the desired space around a {\envfont theorem} environment.
%    \begin{macrocode}
   \trivlist
   \@topsep \theorempreskipamount               % used by first \item
   \@topsepadd \theorempostskipamount           % used by \@endparenv
%    \end{macrocode}
%    Now we have to test whether an optional argument has been given.
% \changes{v2.0d}{89/07/18}{Forgotten percent added.}
%    \begin{macrocode}
   \@ifnextchar [%
%    \end{macrocode}
%    If there is an optional argument, we will call |\@ythm|, and
%    move the arguments read back into the input stream.
%    \begin{macrocode}
   {\@ythm{#1}{#2}}%
%    \end{macrocode}
%    If not, we call |\@begintheorem|. Its first argument is the
%    name of the theorem set (hence the second argument of
%    |\@thm|). Its second argument is the macro that produces the
%    current number.
%    \begin{macrocode}
   {\@begintheorem{#2}{\csname the#1\endcsname}\ignorespaces}}
%    \end{macrocode}
% \end{macro}
%
% \begin{macro}{\@xthm}
% \begin{macro}{\@ythm}
%    Both these macros were originally called by |\@thm|.  We do
%    not need |\@xthm| anymore, hence we reset it to
%    |\relax|. The definition of |\@ythm| has not changed at
%    all from its definition in \LaTeX. In order to make the macros
%    easier to understand, we will
%    nevertheless present it (commented out).
%    \begin{macrocode}
\global\let\@xthm\relax
% \def\@ythm#1#2[#3]{\@opargbegintheorem{#2}{\csname
%           the#1\endcsname}{#3}\ignorespaces}
%</package>
%    \end{macrocode}
%    The primitive |\ignorespaces| in |\@ythm| and
%    |\@thm| is needed to remove the spaces between the
%    |\begin{...}| and the actual text.
% \end{macro}
% \end{macro}
%
%
% \subsection{Definition of the theorem substyles}
%
% As already pointed out, the theorem substyles, defined below, are
% only loaded when necessary.  Note that all these substyles, except
% \texttt{plain}, have |\slshape| as the default body font.
%
% \subsubsection{The \texttt{plain} style}
%
% \label{dok:th@plain} As the following macros use |@|, we have
% to locally set the |\catcode| of this symbol to ``letter''.
% This happens within a group, so that we do not have to worry about
% which |\catcode| that symbol had before.
%    \begin{macrocode}
%<*thp>
\begingroup \makeatletter
%    \end{macrocode}
% Since we are now within a group, we must make all definitions
% globally.  First we make sure that \texttt{theorem.sty} is loaded.
% This will allow us to use this file as a document style option
% without having to call \texttt{theorem} itself as an option.  At the
% same time, we assure that at least version 2 is loaded, since
% |\theorem@style| was not defined in earlier versions.
%    \begin{macrocode}
\@ifundefined{theorem@style}{\input{theorem.sty}}{}
\ProvidesFile{thp.sty}
             [\FMithmInfo]
%    \end{macrocode}
%
%
% \begin{macro}{\th@plain}
% \begin{macro}{\@begintheorem}
% \begin{macro}{\@opargbegintheorem}
%    |\theoremstyle{plain}| corresponds to the original
%    definition, except that the distances to the surrounding text are
%    determined by the parameters |\theorempreskipamount| and
%    |\theorempostskipamount|.
%    First we set the default body font.
%    \begin{macrocode}
\gdef\th@plain{\normalfont\itshape
%    \end{macrocode}
%    Then we define |\@begintheorem| and
%    |\@opargbegintheorem|. These two macros define how the
%    header of a theorem is typeset.  |\@opargbegintheorem| will
%    be called if a \textsf{theorem} environment with an optional
%    argument is encountered; otherwise, the header is constructed by
%    calling |\@begintheorem|.  If one of these macros is
%    executed, we are within a \textsf{trivlist} environment started by
%    |\@thm|.  So the theorem header is produced with an
%    |\item| command.
%
%    Instead of specifying the header font directly, all standard
%    theorem styles use the |\theorem@headerfont| macro to allow
%    customization.  The extra space (|\labelsep|) is necessary
%    because of problems in the \textsf{trivlist} environment.
%    \begin{macrocode}
  \def\@begintheorem##1##2{%
        \item[\hskip\labelsep \theorem@headerfont ##1\ ##2]}%
%    \end{macrocode}
%    The definition of |\@opargbegintheorem| is completely
%    analogous. The only difference is the fact that there exists a
%    third argument (which is the optional parameter of the
%    environment and contains additional information about the
%    theorem).  Customarily we enclose it in parentheses.
%    \begin{macrocode}
\def\@opargbegintheorem##1##2##3{%
   \item[\hskip\labelsep \theorem@headerfont ##1\ ##2\ (##3)]}}
%    \end{macrocode}
% \end{macro}
% \end{macro}
% \end{macro}
%
% \noindent
% We conclude with an |\endgroup| to restore the
% |\catcode| of |@|.
%    \begin{macrocode}
\endgroup
%</thp>
%    \end{macrocode}
%
%
% \subsubsection{The \texttt{break} style}
%
% This style option is stored in the file |thb.sty|.
% For the next two lines see the documentation for |\th@plain| on
% page~\pageref{dok:th@plain}.
%    \begin{macrocode}
%<*thb>
\begingroup \makeatletter
\@ifundefined{theorem@style}{\input{theorem.sty}}{}
\ProvidesFile{thb.sty}
             [\FMithmInfo]
%    \end{macrocode}
%
% \begin{macro}{\th@break}
%    |\theoremstyle{break}| produces a line break after the name
%    of the theorem.  The font is |\slshape|.  Hence, we define
%    |\th@break| as follows:
%    \begin{macrocode}
\gdef\th@break{\normalfont\slshape
  \def\@begintheorem##1##2{\item[%
%    \end{macrocode}
%    We run into the following problem: it is not possible to create
%    the header with |\item[|\meta{title}|]| and then start
%    a new line by, for example, |\mbox{}\\|. Such a definition
%    will fail whenever a list environment follows immediately.  With
%    the above construction, the |\mbox{}| causes the switch
%    |@inlabel| (cf.\ definition of |\list| and
%    |\trivlist| in \cite{latex:source}) to be set to \texttt{false}
%    and so the following list will insert additional vertical
%    space (|\topskip|). This is quite annoying. Therefore, we
%    create the line break within the |\item|. In order to ensure
%    that the text will begin at the proper position in the following
%    line, we simply pretend that the label does not take any
%    room.\footnote{This will lead to problems whenever very high
%    symbols occurring in the line tower into the
%    heading. So, something else has to be done here sometime.}
%    \begin{macrocode}
       \rlap{\vbox{\hbox{\hskip \labelsep\theorem@headerfont ##1\ ##2}%
                   \hbox{\strut}}}]}%
%    \end{macrocode}
%    Again, the definition of |\@opargbegintheorem| is
%    completely analogous.
%    \begin{macrocode}
\def\@opargbegintheorem##1##2##3{%
  \item[\rlap{\vbox{\hbox{\hskip \labelsep \theorem@headerfont
                     ##1\ ##2\ (##3)}%
                    \hbox{\strut}}}]}}
\endgroup
%</thb>
%    \end{macrocode}
% \end{macro}
%
%
% \subsubsection{The \texttt{changebreak} style}
%
%    \begin{macrocode}
% This style option is stored in the file |thcb.sty|.
%    \begin{macrocode}
%<*thcb>
\begingroup \makeatletter
\@ifundefined{theorem@style}{\input{theorem.sty}}{}
\ProvidesFile{thcb.sty}
             [\FMithmInfo]
%    \end{macrocode}
%
% \begin{macro}{\th@changebreak}
%    The \texttt{change-break} theorem style is like \texttt{break}
%    but with interchange of theorem name and theorem number.  Thus we
%    define |\th@changebreak| as follows:
%    \begin{macrocode}
\gdef\th@changebreak{\normalfont\slshape
  \def\@begintheorem##1##2{\item
       [\rlap{\vbox{\hbox{\hskip\labelsep \theorem@headerfont ##2\ ##1}%
                    \hbox{\strut}}}]}%
\def\@opargbegintheorem##1##2##3{%
  \item[\rlap{\vbox{\hbox{\hskip\labelsep \theorem@headerfont
                     ##2\ ##1\ (##3)}%
                    \hbox{\strut}}}]}}
\endgroup
%</thcb>
%    \end{macrocode}
% \end{macro}
%
%
% \subsubsection{The \texttt{change} style}
%
% This style option is stored in the file |thc.sty|.
%    \begin{macrocode}
%<*thc>
\begingroup \makeatletter
\@ifundefined{theorem@style}{\input{theorem.sty}}{}
\ProvidesFile{thc.sty}
             [\FMithmInfo]
%    \end{macrocode}
%
% \begin{macro}{\th@change}
%    The \texttt{change} theorem style corresponds to the
%    \texttt{change break} style without a linebreak after the header.
%    To say it in another way, it's the same as the \texttt{plain}
%    style but with number and name interchanged and |\slshape| as the
%    default font.
%    \begin{macrocode}
\gdef\th@change{\normalfont\slshape
  \def\@begintheorem##1##2{\item
       [\hskip\labelsep \theorem@headerfont ##2\ ##1]}%
\def\@opargbegintheorem##1##2##3{%
  \item[\hskip\labelsep \theorem@headerfont ##2\ ##1\ (##3)]}}
\endgroup
%</thc>
%    \end{macrocode}
% \end{macro}
%
%
% \subsubsection{The \texttt{marginbreak} style}
%
% This style option is the one used most often at Mainz.  It is saved
% in the file |thmb.sty|.
%    \begin{macrocode}
%<*thmb>
\begingroup \makeatletter
\@ifundefined{theorem@style}{\input{theorem.sty}}{}
\ProvidesFile{thmb.sty}
             [\FMithmInfo]
%    \end{macrocode}
%
% \begin{macro}{\th@marginbreak}
%    The \texttt{margin break} style is nearly the same as the
%    \texttt{change break} style. The only difference is the placement
%    of the theorem number.  We use |\llap| to place it in the left
%    margin.
%
%    In this style |\labelsep| denotes the separation between the
%    number and the text.
%    \begin{macrocode}
\gdef\th@marginbreak{\normalfont\slshape
  \def\@begintheorem##1##2{\item
       [\rlap{\vbox{\theorem@headerfont
                    \hbox{\llap{##2}\hskip\labelsep ##1}%
                    \hbox{\strut}}}]}%
\def\@opargbegintheorem##1##2##3{%
  \item[\rlap{\vbox{\theorem@headerfont
                    \hbox{\llap{##2}\hskip\labelsep ##1\ (##3)}%
                    \hbox{\strut}}}]}}
\endgroup
%</thmb>
%    \end{macrocode}
% \changes{v2.1c}{90/02/17}{`endgroup outside macrocode env.}
% \end{macro}
%
%
%
% \subsubsection{The \texttt{margin} style}
%
% This style option is stored in the file |thm.sty|.
%    \begin{macrocode}
%<*thm>
\begingroup \makeatletter
\@ifundefined{theorem@style}{\input{theorem.sty}}{}
\ProvidesFile{thm.sty}
             [\FMithmInfo]
%    \end{macrocode}
%
% \begin{macro}{\th@margin}
%    Again this is only a variant of the theorem styles described
%    above without any new ideas.
%    \begin{macrocode}
\gdef\th@margin{\normalfont\slshape
  \def\@begintheorem##1##2{\item
       [\theorem@headerfont \llap{##2}\hskip\labelsep ##1]}%
\def\@opargbegintheorem##1##2##3{%
  \item[\theorem@headerfont \llap{##2}\hskip\labelsep ##1\ (##3)]}}
\endgroup
%</thm>
%    \end{macrocode}
% \end{macro}
%
%
%
%
% \subsection{Final Definitions}
%
% \begin{macro}{\theorempreskipamount}
% \begin{macro}{\theorempostskipamount}
%    The \textsf{skip} parameters that regulate the vertical empty space
%    before and after the {\envfont theorem} environment have
%    to be allocated as well.
% \changes{v2.0f}{89/07/20}{Unnecessary `global' removed.}
%    \begin{macrocode}
%<*package>
\newskip\theorempreskipamount
\newskip\theorempostskipamount
%    \end{macrocode}
%    Since we have used the same values for all theorem sets, we now can
%    assign them.
%    \begin{macrocode}
\global\setlength\theorempreskipamount{12pt plus 5pt minus 3pt}
\global\setlength\theorempostskipamount{8pt plus 3pt minus 1.5pt}
%    \end{macrocode}
% \end{macro}
% \end{macro}
%
%
% \begin{macro}{\@endtheorem}
%    The same holds for the macro |\@endtheorem|, which ends a
%    {\envfont theorem} environment. Since it is the same for all
%    theorem sets, it is removed from the macros |\th@|\meta{style}.
%    It simply ends the {\envfont trivlist} environment, which was
%    begun in |\@thm|.
%    \begin{macrocode}
\global\let\@endtheorem=\endtrivlist
%    \end{macrocode}
% \end{macro}
%
%
%
% \begin{macro}{\@preamblecmds}
%    All macros defined above are to be used only in the preamble.
%    Therefore, we insert them in |\@preamblecmds| which will disable
%    them at begin document. This is done by the internal \LaTeXe{}
%    command |\@onlypreamble|.
% \changes{v2.0g}{89/09/19}{`theoremheaderfont and `theorembodyfont
%    added.}
% \changes{v2.2a}{94/02/03}{Use \LaTeXe{} interface}
%    \begin{macrocode}
\@onlypreamble\@xnthm
\@onlypreamble\@ynthm
\@onlypreamble\@othm
\@onlypreamble\newtheorem
\@onlypreamble\theoremstyle
\@onlypreamble\theorembodyfont
\@onlypreamble\theoremheaderfont
%    \end{macrocode}
% \end{macro}
%
% Finally we declare the \texttt{plain} theorem style to be the default.
% \changes{v2.1b}{90/06/04}{Fixed missing `item by declaring default
%    style.}
%    \begin{macrocode}
\theoremstyle{plain}
%</package>
%    \end{macrocode}
%
% \Finale

\endinput
}{}
\ProvidesFile{thm.sty}
             [\FMithmInfo]
%    \end{macrocode}
%
% \begin{macro}{\th@margin}
%    Again this is only a variant of the theorem styles described
%    above without any new ideas.
%    \begin{macrocode}
\gdef\th@margin{\normalfont\slshape
  \def\@begintheorem##1##2{\item
       [\theorem@headerfont \llap{##2}\hskip\labelsep ##1]}%
\def\@opargbegintheorem##1##2##3{%
  \item[\theorem@headerfont \llap{##2}\hskip\labelsep ##1\ (##3)]}}
\endgroup
%</thm>
%    \end{macrocode}
% \end{macro}
%
%
%
%
% \subsection{Final Definitions}
%
% \begin{macro}{\theorempreskipamount}
% \begin{macro}{\theorempostskipamount}
%    The \textsf{skip} parameters that regulate the vertical empty space
%    before and after the {\envfont theorem} environment have
%    to be allocated as well.
% \changes{v2.0f}{89/07/20}{Unnecessary `global' removed.}
%    \begin{macrocode}
%<*package>
\newskip\theorempreskipamount
\newskip\theorempostskipamount
%    \end{macrocode}
%    Since we have used the same values for all theorem sets, we now can
%    assign them.
%    \begin{macrocode}
\global\setlength\theorempreskipamount{12pt plus 5pt minus 3pt}
\global\setlength\theorempostskipamount{8pt plus 3pt minus 1.5pt}
%    \end{macrocode}
% \end{macro}
% \end{macro}
%
%
% \begin{macro}{\@endtheorem}
%    The same holds for the macro |\@endtheorem|, which ends a
%    {\envfont theorem} environment. Since it is the same for all
%    theorem sets, it is removed from the macros |\th@|\meta{style}.
%    It simply ends the {\envfont trivlist} environment, which was
%    begun in |\@thm|.
%    \begin{macrocode}
\global\let\@endtheorem=\endtrivlist
%    \end{macrocode}
% \end{macro}
%
%
%
% \begin{macro}{\@preamblecmds}
%    All macros defined above are to be used only in the preamble.
%    Therefore, we insert them in |\@preamblecmds| which will disable
%    them at begin document. This is done by the internal \LaTeXe{}
%    command |\@onlypreamble|.
% \changes{v2.0g}{89/09/19}{`theoremheaderfont and `theorembodyfont
%    added.}
% \changes{v2.2a}{94/02/03}{Use \LaTeXe{} interface}
%    \begin{macrocode}
\@onlypreamble\@xnthm
\@onlypreamble\@ynthm
\@onlypreamble\@othm
\@onlypreamble\newtheorem
\@onlypreamble\theoremstyle
\@onlypreamble\theorembodyfont
\@onlypreamble\theoremheaderfont
%    \end{macrocode}
% \end{macro}
%
% Finally we declare the \texttt{plain} theorem style to be the default.
% \changes{v2.1b}{90/06/04}{Fixed missing `item by declaring default
%    style.}
%    \begin{macrocode}
\theoremstyle{plain}
%</package>
%    \end{macrocode}
%
% \Finale

\endinput
}{}
\ProvidesFile{thb.sty}
             [\FMithmInfo]
%    \end{macrocode}
%
% \begin{macro}{\th@break}
%    |\theoremstyle{break}| produces a line break after the name
%    of the theorem.  The font is |\slshape|.  Hence, we define
%    |\th@break| as follows:
%    \begin{macrocode}
\gdef\th@break{\normalfont\slshape
  \def\@begintheorem##1##2{\item[%
%    \end{macrocode}
%    We run into the following problem: it is not possible to create
%    the header with |\item[|\meta{title}|]| and then start
%    a new line by, for example, |\mbox{}\\|. Such a definition
%    will fail whenever a list environment follows immediately.  With
%    the above construction, the |\mbox{}| causes the switch
%    |@inlabel| (cf.\ definition of |\list| and
%    |\trivlist| in \cite{latex:source}) to be set to \texttt{false}
%    and so the following list will insert additional vertical
%    space (|\topskip|). This is quite annoying. Therefore, we
%    create the line break within the |\item|. In order to ensure
%    that the text will begin at the proper position in the following
%    line, we simply pretend that the label does not take any
%    room.\footnote{This will lead to problems whenever very high
%    symbols occurring in the line tower into the
%    heading. So, something else has to be done here sometime.}
%    \begin{macrocode}
       \rlap{\vbox{\hbox{\hskip \labelsep\theorem@headerfont ##1\ ##2}%
                   \hbox{\strut}}}]}%
%    \end{macrocode}
%    Again, the definition of |\@opargbegintheorem| is
%    completely analogous.
%    \begin{macrocode}
\def\@opargbegintheorem##1##2##3{%
  \item[\rlap{\vbox{\hbox{\hskip \labelsep \theorem@headerfont
                     ##1\ ##2\ (##3)}%
                    \hbox{\strut}}}]}}
\endgroup
%</thb>
%    \end{macrocode}
% \end{macro}
%
%
% \subsubsection{The \texttt{changebreak} style}
%
%    \begin{macrocode}
% This style option is stored in the file |thcb.sty|.
%    \begin{macrocode}
%<*thcb>
\begingroup \makeatletter
\@ifundefined{theorem@style}{% \iffalse meta-comment
%
% Copyright (C) 1993-2024
%
% The LaTeX Project and any individual authors listed elsewhere
% in this file.
%
% This file is part of the Standard LaTeX `Tools Bundle'.
% -------------------------------------------------------
%
% It may be distributed and/or modified under the
% conditions of the LaTeX Project Public License, either version 1.3c
% of this license or (at your option) any later version.
% The latest version of this license is in
%    https://www.latex-project.org/lppl.txt
% and version 1.3c or later is part of all distributions of LaTeX
% version 2005/12/01 or later.
%
% The list of all files belonging to the LaTeX `Tools Bundle' is
% given in the file `manifest.txt'.
%
% \fi
%
%%
%
%
\def\FMithmInfo{2023/07/05 v2.2c Theorem extension package (FMi)}
%
% \ProvidesFile{theorem.dtx}[\FMithmInfo]
%\iffalse   % this is a METACOMMENT !
%%
%% Package `theorem' to use with LaTeX2e
%% Copyright (C) 1989-1995 Frank Mittelbach, all rights reserved.
%<+package>\@ifundefined{theorem@style}{}{\endinput}
%<+package>\NeedsTeXFormat{LaTeX2e}[1995/06/01]
%<+package>\ProvidesPackage{theorem}
%<+package>         [\FMithmInfo]
%
% \fi
%
% \changes{v1.0}{87/05/07}{First draft.}
% \changes{v1.0a}{87/05/07}{`thistheoremstyle' added.}
% \changes{v1.1a}{87/07/29}{New concept: redefine the internal
%                          \protect\LaTeX{} commands `@xnthm' etc.}
% \changes{v1.1b}{87/07/31}{`theoremfont' used to specify font.}
% \changes{v1.2a}{88/01/08}{`@endtheorem' now globally defined.}
% \changes{v1.2b}{88/01/11}{`newtheorem' only allowed in the preamble.}
% \changes{v1.3a}{88/02/14}{`theoremfont' now a toks register.}
%
% \changes{v2.0a}{88/11/16}{New implementation and new documentation.}
% \changes{v2.0a}{88/11/16}{No layout changes but internal modifications
%                           old substyles won't work.}
% \changes{v2.0b}{89/01/30}{`global' used for all definitions.}
% \changes{v2.0c}{89/06/15}{Documentation translated to English.
%                            Thanks to LLL and Nina}
% \changes{v2.1a}{90/02/15}{Support for new font selection.}
% \changes{v2.2a}{94/02/03}{Cleanup for LaTeX2e}
%
% \DoNotIndex{\@endparenv,\@ifdefinable,\@ifnextchar,\@ifundefined}
% \DoNotIndex{\@namedef,\@nameuse,\@spaces,\@tempa,\@warning}
% \DoNotIndex{\ }
% \DoNotIndex{\begingroup,\bfseries}
% \DoNotIndex{\csname}
% \DoNotIndex{\def,\do,\docdate}
% \DoNotIndex{\endcsname,\endgroup,\endinput,\endtrivlist,\expandafter}
% \DoNotIndex{\filedate,\fileversion}
% \DoNotIndex{\gdef,\global}
% \DoNotIndex{\hbox,\hskip}
% \DoNotIndex{\ignorespaces,\input,\itshape,\item}
% \DoNotIndex{\labelsep,\let,\llap}
% \DoNotIndex{\makeatletter}
% \DoNotIndex{\noexpand}
% \DoNotIndex{\refstepcounter,\relax,\rlap}
% \DoNotIndex{\setlength,\slshape,\space,\string,\strut}
% \DoNotIndex{\the,\trivlist,\typeout}
% \DoNotIndex{\vbox}
% \DoNotIndex{\xdef}
%
% \def\deflabel#1{\ttfamily #1\hfill}
% \def\deflist#1{\begin{list}{}{\settowidth\labelwidth{\ttfamily #1}%
%                               \setlength\leftmargin\labelwidth
%                               \addtolength\leftmargin\labelsep
%                               \let\makelabel\deflabel}}
% \def\enddeflist{\end{list}}
% \def\envfont{\normalfont\ttfamily}
%
% \MakeShortVerb{\|}
%
% \newcommand{\TUB}{{\sl TUGboat\/}}
%
% \GetFileInfo{theorem.dtx}
% \title{An Extension of the
%        \LaTeX{} theorem environment\thanks{This file
%        has version number \fileversion, last
%        revised \filedate.}}
% \author{Frank Mittelbach}
%
% \MaintainedByLaTeXTeam{tools}
% \maketitle
%
% \begin{abstract}
%    The macros described in this paper yield an extension of the
%    \LaTeX{} theorem mechanism. It is designed to satisfy the
%    different requirements of various journals. Thus, the
%    layout of the ``theorems'' can be manipulated by determining a
%    ``style''. This article describes not only the use, but
%    also the definition, of the necessary macros.
% \end{abstract}
%
% \section*{Preface to version 2.2}
%
% For \LaTeXe{} this package did not need any fundamental changes. I
% only modified the messages generated so that theorem layout styles
% will show up with the |\listfiles| command and cleaned the section
% on the New Font Selection Scheme since this is now included in
% \LaTeX.
%
% \section*{Preface to version 2.1}
%
% This version is identical to 2.0g described in \TUB\ 10\#3 except
% for some internal defaults which are now set depending on the used
% font selection scheme.
%
% This was done to avoid unpleasant surprises if the new font selection
% scheme is in force. For further details see
% section~\ref{sec:newfonts} and \cite{bk:GMS94}.
%
% \section{Introduction}
%
% For our purposes here, ``theorems'' are labelled enunciations,
% often set off from the main text by extra space and a font change.
% Theorems, corollaries, conjectures, definitions, and remarks
% are all instances of ``theorems''.  The ``header'' of these structures
% is composed of a label (such as \textsc{Theorem} or \textsc{Remark})
% and a number which serializes an item in the sequence of items
% with the same label.
%
% Shortly after the introduction of \LaTeX{} at the Fachbereich
% Mathematik in Mainz, the desire to manipulate the layout of
% ``theorems'' arose.  In Mainz, the following two conventions came
% into general use:
% \begin{enumerate}
%   \item The number of the theorem is shown in the margin.
%   \item There is a line break at
%         the end of the theorem header.
% \end{enumerate}
%
% Additionally, some journals require different formats which depend on
% the ``sort of theorem'': e.g.\ often remarks and definitions are set
% in |\upshape|, while |\itshape| is employed for main theorems.
%
% Confronted with these requirements, a theorem environment was
% developed in Mainz which allows separate determination of the
% layout of the ``theorems sets'', comparable to |\pagestyle|.
%
%
% \section{The user interface}
% \subsection{Defining new theorem sets}
%
% \DescribeMacro\newtheorem
% As in the original \LaTeX{} version, the command |\newtheorem|
% defines a new ``theorem set'' or ``theorem-like structure''.
% Two required arguments name the new environment and give the text
% to be typeset with each instance of the new ``set'', while
% an optional argument determines how
% the ``set'' is enumerated:
% \begin{description}
%    \item[\ttfamily \bslash newtheorem\{foo\}\{bar\}]
%       The theorem set {\envfont foo} (whose name is \texttt{bar})
%       uses its own counter.
%    \item[\ttfamily \bslash newtheorem\{foo2\}{[foo]}\{bar2\}]
%       The theorem set {\envfont foo2} (printed name \texttt{bar2})
%       uses the same counter as the theorem set \texttt{foo}.
%    \item[\ttfamily \bslash newtheorem\{foo3\}\{bar3\}{[section]}]
%       The theorem set {\envfont foo3} (printed name \texttt{bar3}) is
%       enumerated within the counter \texttt{section}, i.e.\ with every
%       new |\section| the enumeration begins again with 1, and
%       the enumeration is composed from the section-number and the
%       theorem counter itself.
% \end{description}
%
% \DescribeMacro\theoremstyle
% Additionally, the command |\theoremstyle| can define the layout
% of various, or all, theorem sets.  It should be noted that any
% theorem set defined by |\newtheorem| is typeset in the
% |\theoremstyle| that is current at the time of the definition.
% Thus, the following
% \begin{verbatim}
%    \theoremstyle{break}        \newtheorem{Cor}{Corollary}
%    \theoremstyle{plain}        \newtheorem{Exa}{Example}[section]
%\end{verbatim}                             ^^A No blank at line start
% leads to the result that the set {\envfont Cor} is formatted in the
% style \texttt{break}, while the set {\envfont Exa} and all the
% following ones are formatted in the style \texttt{plain}, unless
% another |\theoremstyle| follows.  Since the definitions installed by
% |\newtheorem| are global, one also can limit |\theoremstyle| locally
% by grouping braces.
%
% \DescribeMacro\theorembodyfont
% The choice of the font for the theorem body is completely
% independent of the chosen |\theoremstyle|; this has proven to
% be very advantageous.  For example,
% \begin{verbatim}
%   {\theorembodyfont{\upshape}          \newtheorem{Rem}{Remark}}
%\end{verbatim}
% defines a theorem set {\envfont Rem}, which will be set in
% |\upshape| in the current layout (which in our example is
% \texttt{plain}).  As with |\theoremstyle|, the |\theorembodyfont|
% chosen is that current at the time of |\newtheorem|.  If
% |\theorembodyfont| is not specified or one defines
% |\theorembodyfont{}|, then the font used will be that defined by the
% |\theoremstyle|.
%
% \DescribeMacro\theoremheaderfont It is also possible to customize
% the font used for the theorem headers.  This is, however, a global
% declaration, and therefore there should be at most one
% |\theoremheaderfont| declaration in the preamble.\footnote{If it is
% actually necessary to have different header fonts, one has to define
% new theorem styles (substituting the desired font) or specify the
% information directly in the \texttt{\bslash newtheorem} declaration
% (the unclean variant).}
%
% \DescribeMacro\theorempreskipamount
% \DescribeMacro\theorempostskipamount Two additional parameters
% affect the vertical space around the theorem environments:
% |\theorempreskipamount| and |\theorempostskipamount| define,
% respectively, the spacing before and after such an environment.
% These parameters apply for all theorem sets and can be manipulated
% with the ordinary length macros.  They are rubber lengths
% (`\textsf{skips}'), and therefore can contain \texttt{plus} and
% \texttt{minus} parts.
%
% Since the definition of theorem sets should---most sensibly---be
% placed in the preamble, we only allow installation there.
% It is therefore possible to release the memory used here after
% |\begin{document}|, in order to make room for other
% applications.
%
%
% \subsection{Existing theorem styles}
%
% The following theorem styles exist to date:
% \begin{deflist}{marginbreak}
%    \item[plain]
%       This theorem style emulates the original \LaTeX{} definition,
%       except that additionally the parameters
%       |\theorem...skipamount| are used.
%    \item[break]
%       In this style, the theorem header is followed by a line
%       break.
%    \item[marginbreak]
%       The theorem number is set in the margin, and there is a line
%       break as in \texttt{break}.
%    \item[changebreak]
%       Like \texttt{break}, but with header number and text
%       interchanged.
%    \item[change]
%       Header number and text are interchanged, without a line break.
%    \item[margin]
%       The number is set in the left margin, without a line break.
% \end{deflist}
% All styles (except \texttt{plain}) select |\slshape| as the default
% |\theorembodyfont|.
%
%
% \subsection{Examples}
%
% Given the above theorem sets {\envfont Cor}, {\envfont Exa}
% and {\envfont Rem},
% suppose that the preamble also contains the declarations:
% \begin{verbatim}
%    \theoremstyle{marginbreak}   \newtheorem{Lem}[Cor]{Lemma}
%    \theoremstyle{change}
%    \theorembodyfont{\itshape}        \newtheorem{Def}[Cor]{Definition}
%
%    \theoremheaderfont{\scshape}
%\end{verbatim}
% Then the following are some typical examples of the typeset output
% resulting from their use.
%
% \begin{Cor}
%    This is a sentence typeset in the theorem environment {\envfont
%    Cor}.
% \end{Cor}
%
% \begin{Exa}
%    This is a sentence typeset in the theorem environment {\envfont
%    Exa}.
% \end{Exa}
%
% \begin{Rem}
%    This is a sentence typeset in the theorem environment {\envfont
%    Rem}.
% \end{Rem}
%
% \begin{Lem}[Ben User]
%    This is a sentence typeset in the theorem environment {\envfont
%    Lem}.
% \end{Lem}
%
% \begin{Def}[Very impressive Definition]
%    This is a sentence typeset in the theorem environment {\envfont
%    Def}.
% \end{Def}
%
% The last two examples show the effect of the optional argument to
% a theorem environment (it is the text typeset in parentheses).
%
%
% \section{Special Considerations}
% \label{sec:newfonts}
%
% Theoremheader and body are implemented as a unit. This means that
% the |\theoremheaderfont|  will inherit characteristics of the
% |\theorembodyfont| in \LaTeXe. Thus, if for example
% |\theorembodyfont| is |\itshape| and |\theoremheaderfont|
% is |\bfseries| the font selected for the header will have the
% characteristics `bold extended italic'. If this is not desired one
% should set the |\theoremheaderfont| to something like
% \begin{verbatim}
%      \theoremheaderfont{\normalfont\bfseries}
%\end{verbatim}
% i.e.\ supplying all necessary font information explicitly.
%
%
%
% \section{Acknowledgements}
%
% The publication of this set of macros was only possible with the
% help of Christina Busse (translating the manuscript into English),
% Joachim Pense (playing the r\^ole of typist), Chris Rowley
% (looking everything over) and many others providing useful
% suggestions.
%
% \MaybeStop{%
%    \begin{thebibliography}{9}
%       \bibitem{bk:GMS94} \textsc{M.~Goossens}, \textsc{F.~Mittelbach}
%         and \textsc{A.~Samarin}.
%         \newblock The \LaTeX{} Companion.
%         \newblock
%         Addison-Wesley, Reading, Massachusetts, 1994.
%       \bibitem{latex:source} \textsc{Lamport, Leslie.} \newblock
%                               \texttt{latex.tex}, version 2.09, date
%                               Feb. 1990.
%    \end{thebibliography}
%    \PrintIndex
%    \PrintChanges}
%
%
% \section{The documentation driver file}
%
% The next bit of code contains the documentation driver file for
% \TeX{}, i.e., the file that will produce the documentation you are
% currently reading. It will be extracted from this file by the
% \texttt{docstrip} program. Since it is the first code in the file
% one can alternatively process this file directly with \LaTeXe{} to
% obtain the documentation.
%
%    \begin{macrocode}
%<*driver>
\documentclass{ltxdoc}

\usepackage{theorem}

%<+driver>% The next few lines define theorem sets which are used
%<+driver>% in the example section of the documentation.

\theoremstyle{break}       \newtheorem{Cor}{Corollary}
\theoremstyle{plain}       \newtheorem{Exa}{Example}[section]
{\theorembodyfont{\upshape}\newtheorem{Rem}{Remark}}
\theoremstyle{marginbreak} \newtheorem{Lem}[Cor]{Lemma}
\theoremstyle{change}
\theorembodyfont{\itshape} \newtheorem{Def}[Cor]{Definition}

\theoremheaderfont{\scshape}

\RecordChanges

\begin{document}
   \DocInput{theorem.dtx}
\end{document}
%</driver>
%    \end{macrocode}
%
% \section{Definition of the Macros}
%
% If the file has been loaded before, we abort immediately. If not the
% package announces itself (this is actually done at the very top of
% the file---the way it is done isn't good style so don't copy it).
% \changes{v2.0e}{89/07/19}{Spaces removed from `typeout'.}
% \changes{v2.2b}{95/11/19}{Announce moved to top of file}
%    \begin{macrocode}
%<*package>
%\@ifundefined{theorem@style}{}{\endinput}
%\def\FMithmInfo{1995/11/19 v2.2b Theorem extension package (FMi)}
%\ProvidesPackage{theorem}[\FMithmInfo]
%    \end{macrocode}
%
%
% \subsection{Definition of theorem styles and fonts}
%
% All the definitions in this file are done globally to allow
% inputting this file inside a group.
%
% \begin{macro}{\theoremstyle}
%    Before a theorem style can be installed, the chosen style must be
%    known. For that reason, we must test to see that
%    |\th@|\meta{style} is known or, more precisely, that it is
%    different from |\relax|. If the style is not known then
%    |\th@plain| is used.
%    \begin{macrocode}
\gdef\theoremstyle#1{%
   \@ifundefined{th@#1}{\@warning
          {Unknown theoremstyle `#1'. Using `plain'}%
          \theorem@style{plain}}%
%    \end{macrocode}
%    We save the theorem style to be used in the token register
%    |\theorem@style|.
%    \begin{macrocode}
      {\theorem@style{#1}}%
%    \end{macrocode}
%    Now we ``evaluate'' the theorem style: this means, we call the
%    macro |\th@|\meta{style} which will activate the relevant
%    definitions which are contained in a separate file.  This is done
%    in a group to suppress changes to the current font.  This could
%    otherwise pose problems together with the new font selection
%    scheme\footnote{When I printed the original article using the new
%    font selection scheme I ended with a document with slanted
%    typefaces (text headings and all) simply because one of the
%    theorem styles used \texttt{\bslash sl} at toplevel.} if the
%    |\th@|\meta{style} is evaluated a second time.
% \changes{v2.1a}{90/02/15}{Group added around `th@..}
% \changes{v2.1a}{90/02/15}{Reseting of `..@begintheorem removed.}
%    \begin{macrocode}
      \begingroup
        \csname th@\the\theorem@style \endcsname
      \endgroup}
%    \end{macrocode}
% \end{macro}
%
%
% \begin{macro}{\@begintheorem}
% \changes{v2.1a}{90/02/15}{Relaxed at toplevel.}
% \begin{macro}{\@opargbegintheorem}
% \changes{v2.1a}{90/02/15}{Relaxed at toplevel.}
%    We reset |\@begintheorem| and |\@opargbegintheorem| to
%    |\relax| since these commands are no longer necessary at
%    toplevel. This will save a few tokens.
%    \begin{macrocode}
\global\let\@begintheorem\relax
\global\let\@opargbegintheorem\relax
%    \end{macrocode}
% \end{macro}
% \end{macro}
%
% \begin{macro}{\theorem@style}
%    Obviously the \textsf{token} register used above has to be
%    allocated. To assure the utmost compatibility with the
%    original \LaTeX{} definition, we set the default theorem style to
%    \texttt{plain}, which implements the usual \LaTeX{} convention.
% \changes{v2.0f}{89/07/20}{Unnecessary `global' removed.}
%    \begin{macrocode}
\newtoks\theorem@style
\global\theorem@style{plain}
%    \end{macrocode}
% \end{macro}
%
% \begin{macro}{\theorembodyfont}
% \begin{macro}{\theorem@bodyfont}
%    If the |\theorembodyfont| is set by the user then it should not
%    interact with the default font set in the theorem style. When the
%    new font selection is in force this may happen if, for example,
%    the default is |\itshape| and the new |\theorembodyfont| is
%    |\sffamily|. So we add a |\reset@font| command in front of the
%    user definition.
% \changes{v2.0f}{89/07/20}{Unnecessary `global' removed.}
% \changes{v2.0g}{89/09/19}{`theoremfont to `theorembodyfont.}
% \changes{v2.1c}{92/06/22}{`reset@font added and logic changed.}
% \changes{v2.1d}{92/10/28}{Reset to empty if arg is empty}
%    \begin{macrocode}
\gdef\theorembodyfont#1{%
%    \end{macrocode}
%    We check if the argument supplied is empty and if so put nothing
%    into the |\theorem@bodyfont| token register to allow for
%    |\theorembodyfont{}| as a mean of using the default of the
%    current |\theoremstyle|.
%    \begin{macrocode}
   \def\@tempa{#1}%
   \ifx\@tempa\@empty
    \theorem@bodyfont{}%
   \else
    \theorem@bodyfont{\reset@font#1}%
   \fi
   }
\newtoks\theorem@bodyfont
\global\theorem@bodyfont{}
%    \end{macrocode}
% \end{macro}
% \end{macro}
%
%
% \begin{macro}{\theoremheaderfont}
% \changes{v2.0g}{89/09/19}{Macro added.}
%    The font for the theorem headers is handled differently because
%    this definition applies to all theorem styles.
%    \begin{macrocode}
\gdef\theoremheaderfont#1{\gdef\theorem@headerfont{#1}%
%    \end{macrocode}
%    After using the macro once it is redefined to produce an error
%    message.
%    \begin{macrocode}
       \gdef\theoremheaderfont##1{%
        \typeout{\string\theoremheaderfont\space should be used
                 only once.}}}
%    \end{macrocode}
% \end{macro}
%
% \begin{macro}{\theorem@headerfont}
% \changes{v2.0g}{89/09/19}{Macro added.}
%    To set the |\theorem@headerfont| default we first test if
%    the new fontselection scheme is in force.
% \changes{v2.1a}{90/02/15}{Support for new font selection.}
%    \begin{macrocode}
\ifx\upshape\undefined
%    \end{macrocode}
%    If not we define it to expand into |\bfseries|. We don't use
%    |\let| just in case a following style option redefines this
%    macro.
%    \begin{macrocode}
\gdef\theorem@headerfont{\bfseries}
%    \end{macrocode}
%    Otherwise we reset the current shape before calling |\bfseries|.
%    \begin{macrocode}
\else \gdef\theorem@headerfont{\normalfont\bfseries}\fi
%    \end{macrocode}
% \end{macro}
%
% \begin{macro}{\th@plain}
% \begin{macro}{\th@break}
% \begin{macro}{\th@marginbreak}
% \begin{macro}{\th@changebreak}
% \begin{macro}{\th@change}
% \begin{macro}{\th@margin}
%    The different styles are defined in macros such as
%    |\th@plain|.  Since memory space is precious in
%    ``non-Big-versions'', we have to avoid offering too many unused
%    definitions.  Therefore we define these styles in separate files
%    that can be loaded on demand.  Thus the commands themselves only
%    load these files. We use |\@input@|, a \LaTeXe{} internal command
%    that ensures that the file will be listed with |\listfiles|.
% \changes{v2.2a}{94/02/02}{Use `@input@ to load theorem layout files}
%    \begin{macrocode}
\gdef\th@plain{\@input@{thp.sty}}
\gdef\th@break{\@input@{thb.sty}}
\gdef\th@marginbreak{\@input@{thmb.sty}}
\gdef\th@changebreak{\@input@{thcb.sty}}
\gdef\th@change{\@input@{thc.sty}}
\gdef\th@margin{\@input@{thm.sty}}
%    \end{macrocode}
%    This list will be expanded when new styles become
%    available.  For testing,  just append new theorem substyles
%    as document options.
% \end{macro}
% \end{macro}
% \end{macro}
% \end{macro}
% \end{macro}
% \end{macro}
%
%
%
% \subsection{Definition of a new theorem set}
%
% As already pointed out, a new theorem environment can be defined in
% three different ways:
% \begin{verbatim}
%    \newtheorem{Lem}{Lemma}
%    \newtheorem{Lem}{Lemma}[section]
%    \newtheorem{Lem}[Theorem]{Lemma}
%\end{verbatim}
% The function of the macro |\newtheorem| is to recognize these cases
% and then to branch into one of the three macros |\@ynthm|, |\@xnthm|
% or |\@othm|.  This mechanism is adopted unchanged from
% \cite{latex:source}; the essential point here is that, for example,
% in the second case, the arguments \texttt{Lem}, \texttt{Lemma} and
% \texttt{section} are passed over to the macro |\@xnthm|.
%
% We inspect this case first because the others present fewer problems,
% and thus are easily derived from this one.
%
% \begin{macro}{\@xnthm}
%    For our example arguments, the macro |\@xnthm| must fulfill
%    the following:
%    \begin{itemize}
%       \item
%          Define a new \LaTeX-counter `Lem'
%       \item
%          reset this counter within a |\section|
%       \item
%          define the macro |\theLem|
%       \item
%          define the environment macros |\Lem| and
%          |\endLem| using the current
%          |\theoremstyle| and |\theorem@bodyfont|.
%    \end{itemize}
%    Obviously, all this should happen only if the first argument
%    of |\@xnthm| (i.e.\ \texttt{Lem} in our example) is chosen so
%    as not to conflict with any previously defined commands or
%    environments.  This test is performed by the \LaTeX{} macro
%    |\@ifdefinable|.
%    \begin{macrocode}
\gdef\@xnthm#1#2[#3]{\expandafter\@ifdefinable\csname #1\endcsname
%    \end{macrocode}
%    Therefore, the first argument of |\@ifdefinable| is the
%    expansion (in the example, |\Lem|) of
%    |\csname||#1\endcsname|.  The second argument is
%    executed only if the test has been completed successfully.
%    \begin{macrocode}
   {%
%    \end{macrocode}
%    Now we define the new counter. The names of the \LaTeX{}
%    macros employed should speak for themselves:
% \changes{v2.2b}{1995/10/07}{Use \cs{@newctr}}
% \changes{v2.2c}{1995/11/23}{Use \cs{@newctr} properly.}
%    \begin{macrocode}
    \@definecounter{#1}\@newctr{#1}[#3]%
%    \end{macrocode}
%    Using |\@newctr| will give a proper error message if the counter
%    in |#3| is not defined.
%    In defining `|\theLem|' we must generate the desired macro
%    name by use of |\expandafter| and |\csname|.
%    \begin{macrocode}
    \expandafter\xdef\csname the#1\endcsname
%    \end{macrocode}
%    An |\xdef| is used in order to make the definition global,
%    and to ensure that it contains the replacement texts of
%    |\@thmcountersep| and |\@thmcounter|.\footnote{These
%    two macros can be defined by the document style.  Their default
%    values produce a `\texttt{.}' as separation and an arabic
%    representation of the number.} However, not everything should be
%    expanded. For example, it saves space to use |\thesection|
%    instead of its---at times---lengthy expansion.
%    \begin{macrocode}
      {\expandafter \noexpand \csname the#3\endcsname
       \@thmcountersep \@thmcounter{#1}}%
%    \end{macrocode}
%    Thus with the defaults of \LaTeX{}, |\theLem| would be
%    replaced by the command sequence |\thesection.\arabic{Lem}|.
%
%    We will now look at the definition of the macro which is executed
%    at the beginning of the actual environment (in our example this
%    macro is |\Lem|). It should be noted that we use an
%    ``|\expandafter| trick'' to expand only certain parts of the
%    replacement text at the time of the definition.
%    \begin{macrocode}
    \def\@tempa{\global\@namedef{#1}}%
    \expandafter \@tempa \expandafter{%
%    \end{macrocode}
%    First, the macro that contains the current definitions of
%    |\@begintheorem| and |\@opargtheorem| should be called
%    up.  The name of this macro---as is already known---has the form
%    |\th@|\meta{theorem\ style}; therefore, it must be called by
%    \begin{macrocode}
      \csname th@\the \theorem@style
%    \end{macrocode}
%    In addition the default theorem font should be changeable, i.e.\
%    we have to insert the contents of |\theorem@bodyfont|. For
%    that reason, we expand even further, beyond |\endcsname|,
%    and thus insert the contents of the \textsf{token} register
%    |\theorem@bodyfont| in the replacement text.
%    \begin{macrocode}
            \expandafter \endcsname \the \theorem@bodyfont
%    \end{macrocode}
%    Now it is time to call the macro |\@thm| which takes over
%    the further processing.  It has two arguments: the current
%    counter name (in our example, \texttt{Lem}), and the text of the
%    label (in our example, \texttt{Lemma}).
%    \begin{macrocode}
     \@thm{#1}{#2}}%
%    \end{macrocode}
%    With this, the `sub-definition' is complete.  The macro
%    |\@endtheorem| ends a theorem environment and is, so far, nothing
%    but an |\endtrivlist|.  (Hence it is defined globally, and not
%    within the theorem styles.\footnote{This has to be changed as
%    soon as theorem styles that change \texttt{\bslash @endtheorem}
%    exist.  In such a case, all existing styles must be changed as
%    well since they will have to reset the macro.}) Therefore, we can
%    set it equivalent to the macro that ends the theorem set (in our
%    example, |\endLem|).  However, if some day theorem styles exist
%    that do change |\@endtheorem|, we would have to use the
%    commented-out line instead.
%    \begin{macrocode}
    \global \expandafter \let \csname end#1\endcsname \@endtheorem
%   \global\@namedef{end#1}{\@endtheorem}%
%    \end{macrocode}
%    With these commands all the required definitions are employed,
%    unless the test |\@ifdefinable| has failed.  Therefore, we
%    end the second argument of this macro and with it the definition
%    of |\@xnthm|.
%    \begin{macrocode}
   }}
%    \end{macrocode}
% \end{macro}
%
% \begin{macro}{\@ynthm}
%    The definition of |\@ynthm| is completely analogous. In
%    this case the new counter that is defined is not reset within
%    another counter; thus the definition of |\the...|
%    is simplified:
%    \begin{macrocode}
\gdef\@ynthm#1#2{\expandafter\@ifdefinable\csname #1\endcsname
   {\@definecounter{#1}%
    \expandafter\xdef\csname the#1\endcsname{\@thmcounter{#1}}%
%    \end{macrocode}
%    The rest of the definition corresponds literally to that of
%    |\@xnthm|:
%    \begin{macrocode}
    \def\@tempa{\global\@namedef{#1}}\expandafter \@tempa
     \expandafter{\csname th@\the \theorem@style \expandafter
     \endcsname \the\theorem@bodyfont \@thm{#1}{#2}}%
    \global \expandafter \let \csname end#1\endcsname \@endtheorem}}
%    \end{macrocode}
% \end{macro}
%
% \begin{macro}{\@othm}
%    The definition of |\@othm| does not contain anything new.
%    \begin{macrocode}
\gdef\@othm#1[#2]#3{%
%    \end{macrocode}
% \changes{v2.2b}{95/10/07}{Build in check for counter defined}
%    We do not define a new counter but instead use one that has
%    already been defined.  Thus the only definition we need is that
%    of this pseudo-counter (i.e.\ |\the|\meta{env.\ name}).
%    First we check if |#2| corresponds to a known counter name.
%    \begin{macrocode}
  \expandafter\ifx\csname c@#2\endcsname\relax
   \@nocounterr{#2}%
  \else
   \expandafter\@ifdefinable\csname #1\endcsname
   {\expandafter \xdef \csname the#1\endcsname
     {\expandafter \noexpand \csname the#2\endcsname}%
%    \end{macrocode}
%    All other parts of the definition can be adopted from
%    |\@xnthm|. We have to remember, though, that in this case
%    the name of the current counter and the theorem label have moved
%    to the second and third arguments.
%    \begin{macrocode}
    \def\@tempa{\global\@namedef{#1}}\expandafter \@tempa
     \expandafter{\csname th@\the \theorem@style \expandafter
     \endcsname \the\theorem@bodyfont \@thm{#2}{#3}}%
    \global \expandafter \let \csname end#1\endcsname \@endtheorem}%
  \fi}
%    \end{macrocode}
% \end{macro}
%
% \subsection{Macros that are employed in a {\envfont theorem}
%    environment}
%
% \begin{macro}{\@thm}
%    The macro |\@thm| has to increase the current counter.  Then,
%    depending on whether the environment has (or does not have) an
%    optional argument, it has to branch into either |\@begintheorem|
%    or |\@opargtheorem|.
%    \begin{macrocode}
\gdef\@thm#1#2{\refstepcounter{#1}%
%    \end{macrocode}
%    Now we start a {\envfont trivlist} environment, and give
%    |\@topsep| and |\@topsepadd| the values of the \textsf{skip}
%    registers |\theorempreskipamount| and |\theorempostskipamount|.
%    The value in |\@topsep| is the vertical space that is inserted by
%    the first (and only) |\item| in our |\trivlist| whilst
%    |\@topsepadd| is inserted by |\@endparenv| at the end of that
%    {\envfont trivlist} environment.  By using these registers, we
%    obtain the desired space around a {\envfont theorem} environment.
%    \begin{macrocode}
   \trivlist
   \@topsep \theorempreskipamount               % used by first \item
   \@topsepadd \theorempostskipamount           % used by \@endparenv
%    \end{macrocode}
%    Now we have to test whether an optional argument has been given.
% \changes{v2.0d}{89/07/18}{Forgotten percent added.}
%    \begin{macrocode}
   \@ifnextchar [%
%    \end{macrocode}
%    If there is an optional argument, we will call |\@ythm|, and
%    move the arguments read back into the input stream.
%    \begin{macrocode}
   {\@ythm{#1}{#2}}%
%    \end{macrocode}
%    If not, we call |\@begintheorem|. Its first argument is the
%    name of the theorem set (hence the second argument of
%    |\@thm|). Its second argument is the macro that produces the
%    current number.
%    \begin{macrocode}
   {\@begintheorem{#2}{\csname the#1\endcsname}\ignorespaces}}
%    \end{macrocode}
% \end{macro}
%
% \begin{macro}{\@xthm}
% \begin{macro}{\@ythm}
%    Both these macros were originally called by |\@thm|.  We do
%    not need |\@xthm| anymore, hence we reset it to
%    |\relax|. The definition of |\@ythm| has not changed at
%    all from its definition in \LaTeX. In order to make the macros
%    easier to understand, we will
%    nevertheless present it (commented out).
%    \begin{macrocode}
\global\let\@xthm\relax
% \def\@ythm#1#2[#3]{\@opargbegintheorem{#2}{\csname
%           the#1\endcsname}{#3}\ignorespaces}
%</package>
%    \end{macrocode}
%    The primitive |\ignorespaces| in |\@ythm| and
%    |\@thm| is needed to remove the spaces between the
%    |\begin{...}| and the actual text.
% \end{macro}
% \end{macro}
%
%
% \subsection{Definition of the theorem substyles}
%
% As already pointed out, the theorem substyles, defined below, are
% only loaded when necessary.  Note that all these substyles, except
% \texttt{plain}, have |\slshape| as the default body font.
%
% \subsubsection{The \texttt{plain} style}
%
% \label{dok:th@plain} As the following macros use |@|, we have
% to locally set the |\catcode| of this symbol to ``letter''.
% This happens within a group, so that we do not have to worry about
% which |\catcode| that symbol had before.
%    \begin{macrocode}
%<*thp>
\begingroup \makeatletter
%    \end{macrocode}
% Since we are now within a group, we must make all definitions
% globally.  First we make sure that \texttt{theorem.sty} is loaded.
% This will allow us to use this file as a document style option
% without having to call \texttt{theorem} itself as an option.  At the
% same time, we assure that at least version 2 is loaded, since
% |\theorem@style| was not defined in earlier versions.
%    \begin{macrocode}
\@ifundefined{theorem@style}{% \iffalse meta-comment
%
% Copyright (C) 1993-2024
%
% The LaTeX Project and any individual authors listed elsewhere
% in this file.
%
% This file is part of the Standard LaTeX `Tools Bundle'.
% -------------------------------------------------------
%
% It may be distributed and/or modified under the
% conditions of the LaTeX Project Public License, either version 1.3c
% of this license or (at your option) any later version.
% The latest version of this license is in
%    https://www.latex-project.org/lppl.txt
% and version 1.3c or later is part of all distributions of LaTeX
% version 2005/12/01 or later.
%
% The list of all files belonging to the LaTeX `Tools Bundle' is
% given in the file `manifest.txt'.
%
% \fi
%
%%
%
%
\def\FMithmInfo{2023/07/05 v2.2c Theorem extension package (FMi)}
%
% \ProvidesFile{theorem.dtx}[\FMithmInfo]
%\iffalse   % this is a METACOMMENT !
%%
%% Package `theorem' to use with LaTeX2e
%% Copyright (C) 1989-1995 Frank Mittelbach, all rights reserved.
%<+package>\@ifundefined{theorem@style}{}{\endinput}
%<+package>\NeedsTeXFormat{LaTeX2e}[1995/06/01]
%<+package>\ProvidesPackage{theorem}
%<+package>         [\FMithmInfo]
%
% \fi
%
% \changes{v1.0}{87/05/07}{First draft.}
% \changes{v1.0a}{87/05/07}{`thistheoremstyle' added.}
% \changes{v1.1a}{87/07/29}{New concept: redefine the internal
%                          \protect\LaTeX{} commands `@xnthm' etc.}
% \changes{v1.1b}{87/07/31}{`theoremfont' used to specify font.}
% \changes{v1.2a}{88/01/08}{`@endtheorem' now globally defined.}
% \changes{v1.2b}{88/01/11}{`newtheorem' only allowed in the preamble.}
% \changes{v1.3a}{88/02/14}{`theoremfont' now a toks register.}
%
% \changes{v2.0a}{88/11/16}{New implementation and new documentation.}
% \changes{v2.0a}{88/11/16}{No layout changes but internal modifications
%                           old substyles won't work.}
% \changes{v2.0b}{89/01/30}{`global' used for all definitions.}
% \changes{v2.0c}{89/06/15}{Documentation translated to English.
%                            Thanks to LLL and Nina}
% \changes{v2.1a}{90/02/15}{Support for new font selection.}
% \changes{v2.2a}{94/02/03}{Cleanup for LaTeX2e}
%
% \DoNotIndex{\@endparenv,\@ifdefinable,\@ifnextchar,\@ifundefined}
% \DoNotIndex{\@namedef,\@nameuse,\@spaces,\@tempa,\@warning}
% \DoNotIndex{\ }
% \DoNotIndex{\begingroup,\bfseries}
% \DoNotIndex{\csname}
% \DoNotIndex{\def,\do,\docdate}
% \DoNotIndex{\endcsname,\endgroup,\endinput,\endtrivlist,\expandafter}
% \DoNotIndex{\filedate,\fileversion}
% \DoNotIndex{\gdef,\global}
% \DoNotIndex{\hbox,\hskip}
% \DoNotIndex{\ignorespaces,\input,\itshape,\item}
% \DoNotIndex{\labelsep,\let,\llap}
% \DoNotIndex{\makeatletter}
% \DoNotIndex{\noexpand}
% \DoNotIndex{\refstepcounter,\relax,\rlap}
% \DoNotIndex{\setlength,\slshape,\space,\string,\strut}
% \DoNotIndex{\the,\trivlist,\typeout}
% \DoNotIndex{\vbox}
% \DoNotIndex{\xdef}
%
% \def\deflabel#1{\ttfamily #1\hfill}
% \def\deflist#1{\begin{list}{}{\settowidth\labelwidth{\ttfamily #1}%
%                               \setlength\leftmargin\labelwidth
%                               \addtolength\leftmargin\labelsep
%                               \let\makelabel\deflabel}}
% \def\enddeflist{\end{list}}
% \def\envfont{\normalfont\ttfamily}
%
% \MakeShortVerb{\|}
%
% \newcommand{\TUB}{{\sl TUGboat\/}}
%
% \GetFileInfo{theorem.dtx}
% \title{An Extension of the
%        \LaTeX{} theorem environment\thanks{This file
%        has version number \fileversion, last
%        revised \filedate.}}
% \author{Frank Mittelbach}
%
% \MaintainedByLaTeXTeam{tools}
% \maketitle
%
% \begin{abstract}
%    The macros described in this paper yield an extension of the
%    \LaTeX{} theorem mechanism. It is designed to satisfy the
%    different requirements of various journals. Thus, the
%    layout of the ``theorems'' can be manipulated by determining a
%    ``style''. This article describes not only the use, but
%    also the definition, of the necessary macros.
% \end{abstract}
%
% \section*{Preface to version 2.2}
%
% For \LaTeXe{} this package did not need any fundamental changes. I
% only modified the messages generated so that theorem layout styles
% will show up with the |\listfiles| command and cleaned the section
% on the New Font Selection Scheme since this is now included in
% \LaTeX.
%
% \section*{Preface to version 2.1}
%
% This version is identical to 2.0g described in \TUB\ 10\#3 except
% for some internal defaults which are now set depending on the used
% font selection scheme.
%
% This was done to avoid unpleasant surprises if the new font selection
% scheme is in force. For further details see
% section~\ref{sec:newfonts} and \cite{bk:GMS94}.
%
% \section{Introduction}
%
% For our purposes here, ``theorems'' are labelled enunciations,
% often set off from the main text by extra space and a font change.
% Theorems, corollaries, conjectures, definitions, and remarks
% are all instances of ``theorems''.  The ``header'' of these structures
% is composed of a label (such as \textsc{Theorem} or \textsc{Remark})
% and a number which serializes an item in the sequence of items
% with the same label.
%
% Shortly after the introduction of \LaTeX{} at the Fachbereich
% Mathematik in Mainz, the desire to manipulate the layout of
% ``theorems'' arose.  In Mainz, the following two conventions came
% into general use:
% \begin{enumerate}
%   \item The number of the theorem is shown in the margin.
%   \item There is a line break at
%         the end of the theorem header.
% \end{enumerate}
%
% Additionally, some journals require different formats which depend on
% the ``sort of theorem'': e.g.\ often remarks and definitions are set
% in |\upshape|, while |\itshape| is employed for main theorems.
%
% Confronted with these requirements, a theorem environment was
% developed in Mainz which allows separate determination of the
% layout of the ``theorems sets'', comparable to |\pagestyle|.
%
%
% \section{The user interface}
% \subsection{Defining new theorem sets}
%
% \DescribeMacro\newtheorem
% As in the original \LaTeX{} version, the command |\newtheorem|
% defines a new ``theorem set'' or ``theorem-like structure''.
% Two required arguments name the new environment and give the text
% to be typeset with each instance of the new ``set'', while
% an optional argument determines how
% the ``set'' is enumerated:
% \begin{description}
%    \item[\ttfamily \bslash newtheorem\{foo\}\{bar\}]
%       The theorem set {\envfont foo} (whose name is \texttt{bar})
%       uses its own counter.
%    \item[\ttfamily \bslash newtheorem\{foo2\}{[foo]}\{bar2\}]
%       The theorem set {\envfont foo2} (printed name \texttt{bar2})
%       uses the same counter as the theorem set \texttt{foo}.
%    \item[\ttfamily \bslash newtheorem\{foo3\}\{bar3\}{[section]}]
%       The theorem set {\envfont foo3} (printed name \texttt{bar3}) is
%       enumerated within the counter \texttt{section}, i.e.\ with every
%       new |\section| the enumeration begins again with 1, and
%       the enumeration is composed from the section-number and the
%       theorem counter itself.
% \end{description}
%
% \DescribeMacro\theoremstyle
% Additionally, the command |\theoremstyle| can define the layout
% of various, or all, theorem sets.  It should be noted that any
% theorem set defined by |\newtheorem| is typeset in the
% |\theoremstyle| that is current at the time of the definition.
% Thus, the following
% \begin{verbatim}
%    \theoremstyle{break}        \newtheorem{Cor}{Corollary}
%    \theoremstyle{plain}        \newtheorem{Exa}{Example}[section]
%\end{verbatim}                             ^^A No blank at line start
% leads to the result that the set {\envfont Cor} is formatted in the
% style \texttt{break}, while the set {\envfont Exa} and all the
% following ones are formatted in the style \texttt{plain}, unless
% another |\theoremstyle| follows.  Since the definitions installed by
% |\newtheorem| are global, one also can limit |\theoremstyle| locally
% by grouping braces.
%
% \DescribeMacro\theorembodyfont
% The choice of the font for the theorem body is completely
% independent of the chosen |\theoremstyle|; this has proven to
% be very advantageous.  For example,
% \begin{verbatim}
%   {\theorembodyfont{\upshape}          \newtheorem{Rem}{Remark}}
%\end{verbatim}
% defines a theorem set {\envfont Rem}, which will be set in
% |\upshape| in the current layout (which in our example is
% \texttt{plain}).  As with |\theoremstyle|, the |\theorembodyfont|
% chosen is that current at the time of |\newtheorem|.  If
% |\theorembodyfont| is not specified or one defines
% |\theorembodyfont{}|, then the font used will be that defined by the
% |\theoremstyle|.
%
% \DescribeMacro\theoremheaderfont It is also possible to customize
% the font used for the theorem headers.  This is, however, a global
% declaration, and therefore there should be at most one
% |\theoremheaderfont| declaration in the preamble.\footnote{If it is
% actually necessary to have different header fonts, one has to define
% new theorem styles (substituting the desired font) or specify the
% information directly in the \texttt{\bslash newtheorem} declaration
% (the unclean variant).}
%
% \DescribeMacro\theorempreskipamount
% \DescribeMacro\theorempostskipamount Two additional parameters
% affect the vertical space around the theorem environments:
% |\theorempreskipamount| and |\theorempostskipamount| define,
% respectively, the spacing before and after such an environment.
% These parameters apply for all theorem sets and can be manipulated
% with the ordinary length macros.  They are rubber lengths
% (`\textsf{skips}'), and therefore can contain \texttt{plus} and
% \texttt{minus} parts.
%
% Since the definition of theorem sets should---most sensibly---be
% placed in the preamble, we only allow installation there.
% It is therefore possible to release the memory used here after
% |\begin{document}|, in order to make room for other
% applications.
%
%
% \subsection{Existing theorem styles}
%
% The following theorem styles exist to date:
% \begin{deflist}{marginbreak}
%    \item[plain]
%       This theorem style emulates the original \LaTeX{} definition,
%       except that additionally the parameters
%       |\theorem...skipamount| are used.
%    \item[break]
%       In this style, the theorem header is followed by a line
%       break.
%    \item[marginbreak]
%       The theorem number is set in the margin, and there is a line
%       break as in \texttt{break}.
%    \item[changebreak]
%       Like \texttt{break}, but with header number and text
%       interchanged.
%    \item[change]
%       Header number and text are interchanged, without a line break.
%    \item[margin]
%       The number is set in the left margin, without a line break.
% \end{deflist}
% All styles (except \texttt{plain}) select |\slshape| as the default
% |\theorembodyfont|.
%
%
% \subsection{Examples}
%
% Given the above theorem sets {\envfont Cor}, {\envfont Exa}
% and {\envfont Rem},
% suppose that the preamble also contains the declarations:
% \begin{verbatim}
%    \theoremstyle{marginbreak}   \newtheorem{Lem}[Cor]{Lemma}
%    \theoremstyle{change}
%    \theorembodyfont{\itshape}        \newtheorem{Def}[Cor]{Definition}
%
%    \theoremheaderfont{\scshape}
%\end{verbatim}
% Then the following are some typical examples of the typeset output
% resulting from their use.
%
% \begin{Cor}
%    This is a sentence typeset in the theorem environment {\envfont
%    Cor}.
% \end{Cor}
%
% \begin{Exa}
%    This is a sentence typeset in the theorem environment {\envfont
%    Exa}.
% \end{Exa}
%
% \begin{Rem}
%    This is a sentence typeset in the theorem environment {\envfont
%    Rem}.
% \end{Rem}
%
% \begin{Lem}[Ben User]
%    This is a sentence typeset in the theorem environment {\envfont
%    Lem}.
% \end{Lem}
%
% \begin{Def}[Very impressive Definition]
%    This is a sentence typeset in the theorem environment {\envfont
%    Def}.
% \end{Def}
%
% The last two examples show the effect of the optional argument to
% a theorem environment (it is the text typeset in parentheses).
%
%
% \section{Special Considerations}
% \label{sec:newfonts}
%
% Theoremheader and body are implemented as a unit. This means that
% the |\theoremheaderfont|  will inherit characteristics of the
% |\theorembodyfont| in \LaTeXe. Thus, if for example
% |\theorembodyfont| is |\itshape| and |\theoremheaderfont|
% is |\bfseries| the font selected for the header will have the
% characteristics `bold extended italic'. If this is not desired one
% should set the |\theoremheaderfont| to something like
% \begin{verbatim}
%      \theoremheaderfont{\normalfont\bfseries}
%\end{verbatim}
% i.e.\ supplying all necessary font information explicitly.
%
%
%
% \section{Acknowledgements}
%
% The publication of this set of macros was only possible with the
% help of Christina Busse (translating the manuscript into English),
% Joachim Pense (playing the r\^ole of typist), Chris Rowley
% (looking everything over) and many others providing useful
% suggestions.
%
% \MaybeStop{%
%    \begin{thebibliography}{9}
%       \bibitem{bk:GMS94} \textsc{M.~Goossens}, \textsc{F.~Mittelbach}
%         and \textsc{A.~Samarin}.
%         \newblock The \LaTeX{} Companion.
%         \newblock
%         Addison-Wesley, Reading, Massachusetts, 1994.
%       \bibitem{latex:source} \textsc{Lamport, Leslie.} \newblock
%                               \texttt{latex.tex}, version 2.09, date
%                               Feb. 1990.
%    \end{thebibliography}
%    \PrintIndex
%    \PrintChanges}
%
%
% \section{The documentation driver file}
%
% The next bit of code contains the documentation driver file for
% \TeX{}, i.e., the file that will produce the documentation you are
% currently reading. It will be extracted from this file by the
% \texttt{docstrip} program. Since it is the first code in the file
% one can alternatively process this file directly with \LaTeXe{} to
% obtain the documentation.
%
%    \begin{macrocode}
%<*driver>
\documentclass{ltxdoc}

\usepackage{theorem}

%<+driver>% The next few lines define theorem sets which are used
%<+driver>% in the example section of the documentation.

\theoremstyle{break}       \newtheorem{Cor}{Corollary}
\theoremstyle{plain}       \newtheorem{Exa}{Example}[section]
{\theorembodyfont{\upshape}\newtheorem{Rem}{Remark}}
\theoremstyle{marginbreak} \newtheorem{Lem}[Cor]{Lemma}
\theoremstyle{change}
\theorembodyfont{\itshape} \newtheorem{Def}[Cor]{Definition}

\theoremheaderfont{\scshape}

\RecordChanges

\begin{document}
   \DocInput{theorem.dtx}
\end{document}
%</driver>
%    \end{macrocode}
%
% \section{Definition of the Macros}
%
% If the file has been loaded before, we abort immediately. If not the
% package announces itself (this is actually done at the very top of
% the file---the way it is done isn't good style so don't copy it).
% \changes{v2.0e}{89/07/19}{Spaces removed from `typeout'.}
% \changes{v2.2b}{95/11/19}{Announce moved to top of file}
%    \begin{macrocode}
%<*package>
%\@ifundefined{theorem@style}{}{\endinput}
%\def\FMithmInfo{1995/11/19 v2.2b Theorem extension package (FMi)}
%\ProvidesPackage{theorem}[\FMithmInfo]
%    \end{macrocode}
%
%
% \subsection{Definition of theorem styles and fonts}
%
% All the definitions in this file are done globally to allow
% inputting this file inside a group.
%
% \begin{macro}{\theoremstyle}
%    Before a theorem style can be installed, the chosen style must be
%    known. For that reason, we must test to see that
%    |\th@|\meta{style} is known or, more precisely, that it is
%    different from |\relax|. If the style is not known then
%    |\th@plain| is used.
%    \begin{macrocode}
\gdef\theoremstyle#1{%
   \@ifundefined{th@#1}{\@warning
          {Unknown theoremstyle `#1'. Using `plain'}%
          \theorem@style{plain}}%
%    \end{macrocode}
%    We save the theorem style to be used in the token register
%    |\theorem@style|.
%    \begin{macrocode}
      {\theorem@style{#1}}%
%    \end{macrocode}
%    Now we ``evaluate'' the theorem style: this means, we call the
%    macro |\th@|\meta{style} which will activate the relevant
%    definitions which are contained in a separate file.  This is done
%    in a group to suppress changes to the current font.  This could
%    otherwise pose problems together with the new font selection
%    scheme\footnote{When I printed the original article using the new
%    font selection scheme I ended with a document with slanted
%    typefaces (text headings and all) simply because one of the
%    theorem styles used \texttt{\bslash sl} at toplevel.} if the
%    |\th@|\meta{style} is evaluated a second time.
% \changes{v2.1a}{90/02/15}{Group added around `th@..}
% \changes{v2.1a}{90/02/15}{Reseting of `..@begintheorem removed.}
%    \begin{macrocode}
      \begingroup
        \csname th@\the\theorem@style \endcsname
      \endgroup}
%    \end{macrocode}
% \end{macro}
%
%
% \begin{macro}{\@begintheorem}
% \changes{v2.1a}{90/02/15}{Relaxed at toplevel.}
% \begin{macro}{\@opargbegintheorem}
% \changes{v2.1a}{90/02/15}{Relaxed at toplevel.}
%    We reset |\@begintheorem| and |\@opargbegintheorem| to
%    |\relax| since these commands are no longer necessary at
%    toplevel. This will save a few tokens.
%    \begin{macrocode}
\global\let\@begintheorem\relax
\global\let\@opargbegintheorem\relax
%    \end{macrocode}
% \end{macro}
% \end{macro}
%
% \begin{macro}{\theorem@style}
%    Obviously the \textsf{token} register used above has to be
%    allocated. To assure the utmost compatibility with the
%    original \LaTeX{} definition, we set the default theorem style to
%    \texttt{plain}, which implements the usual \LaTeX{} convention.
% \changes{v2.0f}{89/07/20}{Unnecessary `global' removed.}
%    \begin{macrocode}
\newtoks\theorem@style
\global\theorem@style{plain}
%    \end{macrocode}
% \end{macro}
%
% \begin{macro}{\theorembodyfont}
% \begin{macro}{\theorem@bodyfont}
%    If the |\theorembodyfont| is set by the user then it should not
%    interact with the default font set in the theorem style. When the
%    new font selection is in force this may happen if, for example,
%    the default is |\itshape| and the new |\theorembodyfont| is
%    |\sffamily|. So we add a |\reset@font| command in front of the
%    user definition.
% \changes{v2.0f}{89/07/20}{Unnecessary `global' removed.}
% \changes{v2.0g}{89/09/19}{`theoremfont to `theorembodyfont.}
% \changes{v2.1c}{92/06/22}{`reset@font added and logic changed.}
% \changes{v2.1d}{92/10/28}{Reset to empty if arg is empty}
%    \begin{macrocode}
\gdef\theorembodyfont#1{%
%    \end{macrocode}
%    We check if the argument supplied is empty and if so put nothing
%    into the |\theorem@bodyfont| token register to allow for
%    |\theorembodyfont{}| as a mean of using the default of the
%    current |\theoremstyle|.
%    \begin{macrocode}
   \def\@tempa{#1}%
   \ifx\@tempa\@empty
    \theorem@bodyfont{}%
   \else
    \theorem@bodyfont{\reset@font#1}%
   \fi
   }
\newtoks\theorem@bodyfont
\global\theorem@bodyfont{}
%    \end{macrocode}
% \end{macro}
% \end{macro}
%
%
% \begin{macro}{\theoremheaderfont}
% \changes{v2.0g}{89/09/19}{Macro added.}
%    The font for the theorem headers is handled differently because
%    this definition applies to all theorem styles.
%    \begin{macrocode}
\gdef\theoremheaderfont#1{\gdef\theorem@headerfont{#1}%
%    \end{macrocode}
%    After using the macro once it is redefined to produce an error
%    message.
%    \begin{macrocode}
       \gdef\theoremheaderfont##1{%
        \typeout{\string\theoremheaderfont\space should be used
                 only once.}}}
%    \end{macrocode}
% \end{macro}
%
% \begin{macro}{\theorem@headerfont}
% \changes{v2.0g}{89/09/19}{Macro added.}
%    To set the |\theorem@headerfont| default we first test if
%    the new fontselection scheme is in force.
% \changes{v2.1a}{90/02/15}{Support for new font selection.}
%    \begin{macrocode}
\ifx\upshape\undefined
%    \end{macrocode}
%    If not we define it to expand into |\bfseries|. We don't use
%    |\let| just in case a following style option redefines this
%    macro.
%    \begin{macrocode}
\gdef\theorem@headerfont{\bfseries}
%    \end{macrocode}
%    Otherwise we reset the current shape before calling |\bfseries|.
%    \begin{macrocode}
\else \gdef\theorem@headerfont{\normalfont\bfseries}\fi
%    \end{macrocode}
% \end{macro}
%
% \begin{macro}{\th@plain}
% \begin{macro}{\th@break}
% \begin{macro}{\th@marginbreak}
% \begin{macro}{\th@changebreak}
% \begin{macro}{\th@change}
% \begin{macro}{\th@margin}
%    The different styles are defined in macros such as
%    |\th@plain|.  Since memory space is precious in
%    ``non-Big-versions'', we have to avoid offering too many unused
%    definitions.  Therefore we define these styles in separate files
%    that can be loaded on demand.  Thus the commands themselves only
%    load these files. We use |\@input@|, a \LaTeXe{} internal command
%    that ensures that the file will be listed with |\listfiles|.
% \changes{v2.2a}{94/02/02}{Use `@input@ to load theorem layout files}
%    \begin{macrocode}
\gdef\th@plain{\@input@{thp.sty}}
\gdef\th@break{\@input@{thb.sty}}
\gdef\th@marginbreak{\@input@{thmb.sty}}
\gdef\th@changebreak{\@input@{thcb.sty}}
\gdef\th@change{\@input@{thc.sty}}
\gdef\th@margin{\@input@{thm.sty}}
%    \end{macrocode}
%    This list will be expanded when new styles become
%    available.  For testing,  just append new theorem substyles
%    as document options.
% \end{macro}
% \end{macro}
% \end{macro}
% \end{macro}
% \end{macro}
% \end{macro}
%
%
%
% \subsection{Definition of a new theorem set}
%
% As already pointed out, a new theorem environment can be defined in
% three different ways:
% \begin{verbatim}
%    \newtheorem{Lem}{Lemma}
%    \newtheorem{Lem}{Lemma}[section]
%    \newtheorem{Lem}[Theorem]{Lemma}
%\end{verbatim}
% The function of the macro |\newtheorem| is to recognize these cases
% and then to branch into one of the three macros |\@ynthm|, |\@xnthm|
% or |\@othm|.  This mechanism is adopted unchanged from
% \cite{latex:source}; the essential point here is that, for example,
% in the second case, the arguments \texttt{Lem}, \texttt{Lemma} and
% \texttt{section} are passed over to the macro |\@xnthm|.
%
% We inspect this case first because the others present fewer problems,
% and thus are easily derived from this one.
%
% \begin{macro}{\@xnthm}
%    For our example arguments, the macro |\@xnthm| must fulfill
%    the following:
%    \begin{itemize}
%       \item
%          Define a new \LaTeX-counter `Lem'
%       \item
%          reset this counter within a |\section|
%       \item
%          define the macro |\theLem|
%       \item
%          define the environment macros |\Lem| and
%          |\endLem| using the current
%          |\theoremstyle| and |\theorem@bodyfont|.
%    \end{itemize}
%    Obviously, all this should happen only if the first argument
%    of |\@xnthm| (i.e.\ \texttt{Lem} in our example) is chosen so
%    as not to conflict with any previously defined commands or
%    environments.  This test is performed by the \LaTeX{} macro
%    |\@ifdefinable|.
%    \begin{macrocode}
\gdef\@xnthm#1#2[#3]{\expandafter\@ifdefinable\csname #1\endcsname
%    \end{macrocode}
%    Therefore, the first argument of |\@ifdefinable| is the
%    expansion (in the example, |\Lem|) of
%    |\csname||#1\endcsname|.  The second argument is
%    executed only if the test has been completed successfully.
%    \begin{macrocode}
   {%
%    \end{macrocode}
%    Now we define the new counter. The names of the \LaTeX{}
%    macros employed should speak for themselves:
% \changes{v2.2b}{1995/10/07}{Use \cs{@newctr}}
% \changes{v2.2c}{1995/11/23}{Use \cs{@newctr} properly.}
%    \begin{macrocode}
    \@definecounter{#1}\@newctr{#1}[#3]%
%    \end{macrocode}
%    Using |\@newctr| will give a proper error message if the counter
%    in |#3| is not defined.
%    In defining `|\theLem|' we must generate the desired macro
%    name by use of |\expandafter| and |\csname|.
%    \begin{macrocode}
    \expandafter\xdef\csname the#1\endcsname
%    \end{macrocode}
%    An |\xdef| is used in order to make the definition global,
%    and to ensure that it contains the replacement texts of
%    |\@thmcountersep| and |\@thmcounter|.\footnote{These
%    two macros can be defined by the document style.  Their default
%    values produce a `\texttt{.}' as separation and an arabic
%    representation of the number.} However, not everything should be
%    expanded. For example, it saves space to use |\thesection|
%    instead of its---at times---lengthy expansion.
%    \begin{macrocode}
      {\expandafter \noexpand \csname the#3\endcsname
       \@thmcountersep \@thmcounter{#1}}%
%    \end{macrocode}
%    Thus with the defaults of \LaTeX{}, |\theLem| would be
%    replaced by the command sequence |\thesection.\arabic{Lem}|.
%
%    We will now look at the definition of the macro which is executed
%    at the beginning of the actual environment (in our example this
%    macro is |\Lem|). It should be noted that we use an
%    ``|\expandafter| trick'' to expand only certain parts of the
%    replacement text at the time of the definition.
%    \begin{macrocode}
    \def\@tempa{\global\@namedef{#1}}%
    \expandafter \@tempa \expandafter{%
%    \end{macrocode}
%    First, the macro that contains the current definitions of
%    |\@begintheorem| and |\@opargtheorem| should be called
%    up.  The name of this macro---as is already known---has the form
%    |\th@|\meta{theorem\ style}; therefore, it must be called by
%    \begin{macrocode}
      \csname th@\the \theorem@style
%    \end{macrocode}
%    In addition the default theorem font should be changeable, i.e.\
%    we have to insert the contents of |\theorem@bodyfont|. For
%    that reason, we expand even further, beyond |\endcsname|,
%    and thus insert the contents of the \textsf{token} register
%    |\theorem@bodyfont| in the replacement text.
%    \begin{macrocode}
            \expandafter \endcsname \the \theorem@bodyfont
%    \end{macrocode}
%    Now it is time to call the macro |\@thm| which takes over
%    the further processing.  It has two arguments: the current
%    counter name (in our example, \texttt{Lem}), and the text of the
%    label (in our example, \texttt{Lemma}).
%    \begin{macrocode}
     \@thm{#1}{#2}}%
%    \end{macrocode}
%    With this, the `sub-definition' is complete.  The macro
%    |\@endtheorem| ends a theorem environment and is, so far, nothing
%    but an |\endtrivlist|.  (Hence it is defined globally, and not
%    within the theorem styles.\footnote{This has to be changed as
%    soon as theorem styles that change \texttt{\bslash @endtheorem}
%    exist.  In such a case, all existing styles must be changed as
%    well since they will have to reset the macro.}) Therefore, we can
%    set it equivalent to the macro that ends the theorem set (in our
%    example, |\endLem|).  However, if some day theorem styles exist
%    that do change |\@endtheorem|, we would have to use the
%    commented-out line instead.
%    \begin{macrocode}
    \global \expandafter \let \csname end#1\endcsname \@endtheorem
%   \global\@namedef{end#1}{\@endtheorem}%
%    \end{macrocode}
%    With these commands all the required definitions are employed,
%    unless the test |\@ifdefinable| has failed.  Therefore, we
%    end the second argument of this macro and with it the definition
%    of |\@xnthm|.
%    \begin{macrocode}
   }}
%    \end{macrocode}
% \end{macro}
%
% \begin{macro}{\@ynthm}
%    The definition of |\@ynthm| is completely analogous. In
%    this case the new counter that is defined is not reset within
%    another counter; thus the definition of |\the...|
%    is simplified:
%    \begin{macrocode}
\gdef\@ynthm#1#2{\expandafter\@ifdefinable\csname #1\endcsname
   {\@definecounter{#1}%
    \expandafter\xdef\csname the#1\endcsname{\@thmcounter{#1}}%
%    \end{macrocode}
%    The rest of the definition corresponds literally to that of
%    |\@xnthm|:
%    \begin{macrocode}
    \def\@tempa{\global\@namedef{#1}}\expandafter \@tempa
     \expandafter{\csname th@\the \theorem@style \expandafter
     \endcsname \the\theorem@bodyfont \@thm{#1}{#2}}%
    \global \expandafter \let \csname end#1\endcsname \@endtheorem}}
%    \end{macrocode}
% \end{macro}
%
% \begin{macro}{\@othm}
%    The definition of |\@othm| does not contain anything new.
%    \begin{macrocode}
\gdef\@othm#1[#2]#3{%
%    \end{macrocode}
% \changes{v2.2b}{95/10/07}{Build in check for counter defined}
%    We do not define a new counter but instead use one that has
%    already been defined.  Thus the only definition we need is that
%    of this pseudo-counter (i.e.\ |\the|\meta{env.\ name}).
%    First we check if |#2| corresponds to a known counter name.
%    \begin{macrocode}
  \expandafter\ifx\csname c@#2\endcsname\relax
   \@nocounterr{#2}%
  \else
   \expandafter\@ifdefinable\csname #1\endcsname
   {\expandafter \xdef \csname the#1\endcsname
     {\expandafter \noexpand \csname the#2\endcsname}%
%    \end{macrocode}
%    All other parts of the definition can be adopted from
%    |\@xnthm|. We have to remember, though, that in this case
%    the name of the current counter and the theorem label have moved
%    to the second and third arguments.
%    \begin{macrocode}
    \def\@tempa{\global\@namedef{#1}}\expandafter \@tempa
     \expandafter{\csname th@\the \theorem@style \expandafter
     \endcsname \the\theorem@bodyfont \@thm{#2}{#3}}%
    \global \expandafter \let \csname end#1\endcsname \@endtheorem}%
  \fi}
%    \end{macrocode}
% \end{macro}
%
% \subsection{Macros that are employed in a {\envfont theorem}
%    environment}
%
% \begin{macro}{\@thm}
%    The macro |\@thm| has to increase the current counter.  Then,
%    depending on whether the environment has (or does not have) an
%    optional argument, it has to branch into either |\@begintheorem|
%    or |\@opargtheorem|.
%    \begin{macrocode}
\gdef\@thm#1#2{\refstepcounter{#1}%
%    \end{macrocode}
%    Now we start a {\envfont trivlist} environment, and give
%    |\@topsep| and |\@topsepadd| the values of the \textsf{skip}
%    registers |\theorempreskipamount| and |\theorempostskipamount|.
%    The value in |\@topsep| is the vertical space that is inserted by
%    the first (and only) |\item| in our |\trivlist| whilst
%    |\@topsepadd| is inserted by |\@endparenv| at the end of that
%    {\envfont trivlist} environment.  By using these registers, we
%    obtain the desired space around a {\envfont theorem} environment.
%    \begin{macrocode}
   \trivlist
   \@topsep \theorempreskipamount               % used by first \item
   \@topsepadd \theorempostskipamount           % used by \@endparenv
%    \end{macrocode}
%    Now we have to test whether an optional argument has been given.
% \changes{v2.0d}{89/07/18}{Forgotten percent added.}
%    \begin{macrocode}
   \@ifnextchar [%
%    \end{macrocode}
%    If there is an optional argument, we will call |\@ythm|, and
%    move the arguments read back into the input stream.
%    \begin{macrocode}
   {\@ythm{#1}{#2}}%
%    \end{macrocode}
%    If not, we call |\@begintheorem|. Its first argument is the
%    name of the theorem set (hence the second argument of
%    |\@thm|). Its second argument is the macro that produces the
%    current number.
%    \begin{macrocode}
   {\@begintheorem{#2}{\csname the#1\endcsname}\ignorespaces}}
%    \end{macrocode}
% \end{macro}
%
% \begin{macro}{\@xthm}
% \begin{macro}{\@ythm}
%    Both these macros were originally called by |\@thm|.  We do
%    not need |\@xthm| anymore, hence we reset it to
%    |\relax|. The definition of |\@ythm| has not changed at
%    all from its definition in \LaTeX. In order to make the macros
%    easier to understand, we will
%    nevertheless present it (commented out).
%    \begin{macrocode}
\global\let\@xthm\relax
% \def\@ythm#1#2[#3]{\@opargbegintheorem{#2}{\csname
%           the#1\endcsname}{#3}\ignorespaces}
%</package>
%    \end{macrocode}
%    The primitive |\ignorespaces| in |\@ythm| and
%    |\@thm| is needed to remove the spaces between the
%    |\begin{...}| and the actual text.
% \end{macro}
% \end{macro}
%
%
% \subsection{Definition of the theorem substyles}
%
% As already pointed out, the theorem substyles, defined below, are
% only loaded when necessary.  Note that all these substyles, except
% \texttt{plain}, have |\slshape| as the default body font.
%
% \subsubsection{The \texttt{plain} style}
%
% \label{dok:th@plain} As the following macros use |@|, we have
% to locally set the |\catcode| of this symbol to ``letter''.
% This happens within a group, so that we do not have to worry about
% which |\catcode| that symbol had before.
%    \begin{macrocode}
%<*thp>
\begingroup \makeatletter
%    \end{macrocode}
% Since we are now within a group, we must make all definitions
% globally.  First we make sure that \texttt{theorem.sty} is loaded.
% This will allow us to use this file as a document style option
% without having to call \texttt{theorem} itself as an option.  At the
% same time, we assure that at least version 2 is loaded, since
% |\theorem@style| was not defined in earlier versions.
%    \begin{macrocode}
\@ifundefined{theorem@style}{\input{theorem.sty}}{}
\ProvidesFile{thp.sty}
             [\FMithmInfo]
%    \end{macrocode}
%
%
% \begin{macro}{\th@plain}
% \begin{macro}{\@begintheorem}
% \begin{macro}{\@opargbegintheorem}
%    |\theoremstyle{plain}| corresponds to the original
%    definition, except that the distances to the surrounding text are
%    determined by the parameters |\theorempreskipamount| and
%    |\theorempostskipamount|.
%    First we set the default body font.
%    \begin{macrocode}
\gdef\th@plain{\normalfont\itshape
%    \end{macrocode}
%    Then we define |\@begintheorem| and
%    |\@opargbegintheorem|. These two macros define how the
%    header of a theorem is typeset.  |\@opargbegintheorem| will
%    be called if a \textsf{theorem} environment with an optional
%    argument is encountered; otherwise, the header is constructed by
%    calling |\@begintheorem|.  If one of these macros is
%    executed, we are within a \textsf{trivlist} environment started by
%    |\@thm|.  So the theorem header is produced with an
%    |\item| command.
%
%    Instead of specifying the header font directly, all standard
%    theorem styles use the |\theorem@headerfont| macro to allow
%    customization.  The extra space (|\labelsep|) is necessary
%    because of problems in the \textsf{trivlist} environment.
%    \begin{macrocode}
  \def\@begintheorem##1##2{%
        \item[\hskip\labelsep \theorem@headerfont ##1\ ##2]}%
%    \end{macrocode}
%    The definition of |\@opargbegintheorem| is completely
%    analogous. The only difference is the fact that there exists a
%    third argument (which is the optional parameter of the
%    environment and contains additional information about the
%    theorem).  Customarily we enclose it in parentheses.
%    \begin{macrocode}
\def\@opargbegintheorem##1##2##3{%
   \item[\hskip\labelsep \theorem@headerfont ##1\ ##2\ (##3)]}}
%    \end{macrocode}
% \end{macro}
% \end{macro}
% \end{macro}
%
% \noindent
% We conclude with an |\endgroup| to restore the
% |\catcode| of |@|.
%    \begin{macrocode}
\endgroup
%</thp>
%    \end{macrocode}
%
%
% \subsubsection{The \texttt{break} style}
%
% This style option is stored in the file |thb.sty|.
% For the next two lines see the documentation for |\th@plain| on
% page~\pageref{dok:th@plain}.
%    \begin{macrocode}
%<*thb>
\begingroup \makeatletter
\@ifundefined{theorem@style}{\input{theorem.sty}}{}
\ProvidesFile{thb.sty}
             [\FMithmInfo]
%    \end{macrocode}
%
% \begin{macro}{\th@break}
%    |\theoremstyle{break}| produces a line break after the name
%    of the theorem.  The font is |\slshape|.  Hence, we define
%    |\th@break| as follows:
%    \begin{macrocode}
\gdef\th@break{\normalfont\slshape
  \def\@begintheorem##1##2{\item[%
%    \end{macrocode}
%    We run into the following problem: it is not possible to create
%    the header with |\item[|\meta{title}|]| and then start
%    a new line by, for example, |\mbox{}\\|. Such a definition
%    will fail whenever a list environment follows immediately.  With
%    the above construction, the |\mbox{}| causes the switch
%    |@inlabel| (cf.\ definition of |\list| and
%    |\trivlist| in \cite{latex:source}) to be set to \texttt{false}
%    and so the following list will insert additional vertical
%    space (|\topskip|). This is quite annoying. Therefore, we
%    create the line break within the |\item|. In order to ensure
%    that the text will begin at the proper position in the following
%    line, we simply pretend that the label does not take any
%    room.\footnote{This will lead to problems whenever very high
%    symbols occurring in the line tower into the
%    heading. So, something else has to be done here sometime.}
%    \begin{macrocode}
       \rlap{\vbox{\hbox{\hskip \labelsep\theorem@headerfont ##1\ ##2}%
                   \hbox{\strut}}}]}%
%    \end{macrocode}
%    Again, the definition of |\@opargbegintheorem| is
%    completely analogous.
%    \begin{macrocode}
\def\@opargbegintheorem##1##2##3{%
  \item[\rlap{\vbox{\hbox{\hskip \labelsep \theorem@headerfont
                     ##1\ ##2\ (##3)}%
                    \hbox{\strut}}}]}}
\endgroup
%</thb>
%    \end{macrocode}
% \end{macro}
%
%
% \subsubsection{The \texttt{changebreak} style}
%
%    \begin{macrocode}
% This style option is stored in the file |thcb.sty|.
%    \begin{macrocode}
%<*thcb>
\begingroup \makeatletter
\@ifundefined{theorem@style}{\input{theorem.sty}}{}
\ProvidesFile{thcb.sty}
             [\FMithmInfo]
%    \end{macrocode}
%
% \begin{macro}{\th@changebreak}
%    The \texttt{change-break} theorem style is like \texttt{break}
%    but with interchange of theorem name and theorem number.  Thus we
%    define |\th@changebreak| as follows:
%    \begin{macrocode}
\gdef\th@changebreak{\normalfont\slshape
  \def\@begintheorem##1##2{\item
       [\rlap{\vbox{\hbox{\hskip\labelsep \theorem@headerfont ##2\ ##1}%
                    \hbox{\strut}}}]}%
\def\@opargbegintheorem##1##2##3{%
  \item[\rlap{\vbox{\hbox{\hskip\labelsep \theorem@headerfont
                     ##2\ ##1\ (##3)}%
                    \hbox{\strut}}}]}}
\endgroup
%</thcb>
%    \end{macrocode}
% \end{macro}
%
%
% \subsubsection{The \texttt{change} style}
%
% This style option is stored in the file |thc.sty|.
%    \begin{macrocode}
%<*thc>
\begingroup \makeatletter
\@ifundefined{theorem@style}{\input{theorem.sty}}{}
\ProvidesFile{thc.sty}
             [\FMithmInfo]
%    \end{macrocode}
%
% \begin{macro}{\th@change}
%    The \texttt{change} theorem style corresponds to the
%    \texttt{change break} style without a linebreak after the header.
%    To say it in another way, it's the same as the \texttt{plain}
%    style but with number and name interchanged and |\slshape| as the
%    default font.
%    \begin{macrocode}
\gdef\th@change{\normalfont\slshape
  \def\@begintheorem##1##2{\item
       [\hskip\labelsep \theorem@headerfont ##2\ ##1]}%
\def\@opargbegintheorem##1##2##3{%
  \item[\hskip\labelsep \theorem@headerfont ##2\ ##1\ (##3)]}}
\endgroup
%</thc>
%    \end{macrocode}
% \end{macro}
%
%
% \subsubsection{The \texttt{marginbreak} style}
%
% This style option is the one used most often at Mainz.  It is saved
% in the file |thmb.sty|.
%    \begin{macrocode}
%<*thmb>
\begingroup \makeatletter
\@ifundefined{theorem@style}{\input{theorem.sty}}{}
\ProvidesFile{thmb.sty}
             [\FMithmInfo]
%    \end{macrocode}
%
% \begin{macro}{\th@marginbreak}
%    The \texttt{margin break} style is nearly the same as the
%    \texttt{change break} style. The only difference is the placement
%    of the theorem number.  We use |\llap| to place it in the left
%    margin.
%
%    In this style |\labelsep| denotes the separation between the
%    number and the text.
%    \begin{macrocode}
\gdef\th@marginbreak{\normalfont\slshape
  \def\@begintheorem##1##2{\item
       [\rlap{\vbox{\theorem@headerfont
                    \hbox{\llap{##2}\hskip\labelsep ##1}%
                    \hbox{\strut}}}]}%
\def\@opargbegintheorem##1##2##3{%
  \item[\rlap{\vbox{\theorem@headerfont
                    \hbox{\llap{##2}\hskip\labelsep ##1\ (##3)}%
                    \hbox{\strut}}}]}}
\endgroup
%</thmb>
%    \end{macrocode}
% \changes{v2.1c}{90/02/17}{`endgroup outside macrocode env.}
% \end{macro}
%
%
%
% \subsubsection{The \texttt{margin} style}
%
% This style option is stored in the file |thm.sty|.
%    \begin{macrocode}
%<*thm>
\begingroup \makeatletter
\@ifundefined{theorem@style}{\input{theorem.sty}}{}
\ProvidesFile{thm.sty}
             [\FMithmInfo]
%    \end{macrocode}
%
% \begin{macro}{\th@margin}
%    Again this is only a variant of the theorem styles described
%    above without any new ideas.
%    \begin{macrocode}
\gdef\th@margin{\normalfont\slshape
  \def\@begintheorem##1##2{\item
       [\theorem@headerfont \llap{##2}\hskip\labelsep ##1]}%
\def\@opargbegintheorem##1##2##3{%
  \item[\theorem@headerfont \llap{##2}\hskip\labelsep ##1\ (##3)]}}
\endgroup
%</thm>
%    \end{macrocode}
% \end{macro}
%
%
%
%
% \subsection{Final Definitions}
%
% \begin{macro}{\theorempreskipamount}
% \begin{macro}{\theorempostskipamount}
%    The \textsf{skip} parameters that regulate the vertical empty space
%    before and after the {\envfont theorem} environment have
%    to be allocated as well.
% \changes{v2.0f}{89/07/20}{Unnecessary `global' removed.}
%    \begin{macrocode}
%<*package>
\newskip\theorempreskipamount
\newskip\theorempostskipamount
%    \end{macrocode}
%    Since we have used the same values for all theorem sets, we now can
%    assign them.
%    \begin{macrocode}
\global\setlength\theorempreskipamount{12pt plus 5pt minus 3pt}
\global\setlength\theorempostskipamount{8pt plus 3pt minus 1.5pt}
%    \end{macrocode}
% \end{macro}
% \end{macro}
%
%
% \begin{macro}{\@endtheorem}
%    The same holds for the macro |\@endtheorem|, which ends a
%    {\envfont theorem} environment. Since it is the same for all
%    theorem sets, it is removed from the macros |\th@|\meta{style}.
%    It simply ends the {\envfont trivlist} environment, which was
%    begun in |\@thm|.
%    \begin{macrocode}
\global\let\@endtheorem=\endtrivlist
%    \end{macrocode}
% \end{macro}
%
%
%
% \begin{macro}{\@preamblecmds}
%    All macros defined above are to be used only in the preamble.
%    Therefore, we insert them in |\@preamblecmds| which will disable
%    them at begin document. This is done by the internal \LaTeXe{}
%    command |\@onlypreamble|.
% \changes{v2.0g}{89/09/19}{`theoremheaderfont and `theorembodyfont
%    added.}
% \changes{v2.2a}{94/02/03}{Use \LaTeXe{} interface}
%    \begin{macrocode}
\@onlypreamble\@xnthm
\@onlypreamble\@ynthm
\@onlypreamble\@othm
\@onlypreamble\newtheorem
\@onlypreamble\theoremstyle
\@onlypreamble\theorembodyfont
\@onlypreamble\theoremheaderfont
%    \end{macrocode}
% \end{macro}
%
% Finally we declare the \texttt{plain} theorem style to be the default.
% \changes{v2.1b}{90/06/04}{Fixed missing `item by declaring default
%    style.}
%    \begin{macrocode}
\theoremstyle{plain}
%</package>
%    \end{macrocode}
%
% \Finale

\endinput
}{}
\ProvidesFile{thp.sty}
             [\FMithmInfo]
%    \end{macrocode}
%
%
% \begin{macro}{\th@plain}
% \begin{macro}{\@begintheorem}
% \begin{macro}{\@opargbegintheorem}
%    |\theoremstyle{plain}| corresponds to the original
%    definition, except that the distances to the surrounding text are
%    determined by the parameters |\theorempreskipamount| and
%    |\theorempostskipamount|.
%    First we set the default body font.
%    \begin{macrocode}
\gdef\th@plain{\normalfont\itshape
%    \end{macrocode}
%    Then we define |\@begintheorem| and
%    |\@opargbegintheorem|. These two macros define how the
%    header of a theorem is typeset.  |\@opargbegintheorem| will
%    be called if a \textsf{theorem} environment with an optional
%    argument is encountered; otherwise, the header is constructed by
%    calling |\@begintheorem|.  If one of these macros is
%    executed, we are within a \textsf{trivlist} environment started by
%    |\@thm|.  So the theorem header is produced with an
%    |\item| command.
%
%    Instead of specifying the header font directly, all standard
%    theorem styles use the |\theorem@headerfont| macro to allow
%    customization.  The extra space (|\labelsep|) is necessary
%    because of problems in the \textsf{trivlist} environment.
%    \begin{macrocode}
  \def\@begintheorem##1##2{%
        \item[\hskip\labelsep \theorem@headerfont ##1\ ##2]}%
%    \end{macrocode}
%    The definition of |\@opargbegintheorem| is completely
%    analogous. The only difference is the fact that there exists a
%    third argument (which is the optional parameter of the
%    environment and contains additional information about the
%    theorem).  Customarily we enclose it in parentheses.
%    \begin{macrocode}
\def\@opargbegintheorem##1##2##3{%
   \item[\hskip\labelsep \theorem@headerfont ##1\ ##2\ (##3)]}}
%    \end{macrocode}
% \end{macro}
% \end{macro}
% \end{macro}
%
% \noindent
% We conclude with an |\endgroup| to restore the
% |\catcode| of |@|.
%    \begin{macrocode}
\endgroup
%</thp>
%    \end{macrocode}
%
%
% \subsubsection{The \texttt{break} style}
%
% This style option is stored in the file |thb.sty|.
% For the next two lines see the documentation for |\th@plain| on
% page~\pageref{dok:th@plain}.
%    \begin{macrocode}
%<*thb>
\begingroup \makeatletter
\@ifundefined{theorem@style}{% \iffalse meta-comment
%
% Copyright (C) 1993-2024
%
% The LaTeX Project and any individual authors listed elsewhere
% in this file.
%
% This file is part of the Standard LaTeX `Tools Bundle'.
% -------------------------------------------------------
%
% It may be distributed and/or modified under the
% conditions of the LaTeX Project Public License, either version 1.3c
% of this license or (at your option) any later version.
% The latest version of this license is in
%    https://www.latex-project.org/lppl.txt
% and version 1.3c or later is part of all distributions of LaTeX
% version 2005/12/01 or later.
%
% The list of all files belonging to the LaTeX `Tools Bundle' is
% given in the file `manifest.txt'.
%
% \fi
%
%%
%
%
\def\FMithmInfo{2023/07/05 v2.2c Theorem extension package (FMi)}
%
% \ProvidesFile{theorem.dtx}[\FMithmInfo]
%\iffalse   % this is a METACOMMENT !
%%
%% Package `theorem' to use with LaTeX2e
%% Copyright (C) 1989-1995 Frank Mittelbach, all rights reserved.
%<+package>\@ifundefined{theorem@style}{}{\endinput}
%<+package>\NeedsTeXFormat{LaTeX2e}[1995/06/01]
%<+package>\ProvidesPackage{theorem}
%<+package>         [\FMithmInfo]
%
% \fi
%
% \changes{v1.0}{87/05/07}{First draft.}
% \changes{v1.0a}{87/05/07}{`thistheoremstyle' added.}
% \changes{v1.1a}{87/07/29}{New concept: redefine the internal
%                          \protect\LaTeX{} commands `@xnthm' etc.}
% \changes{v1.1b}{87/07/31}{`theoremfont' used to specify font.}
% \changes{v1.2a}{88/01/08}{`@endtheorem' now globally defined.}
% \changes{v1.2b}{88/01/11}{`newtheorem' only allowed in the preamble.}
% \changes{v1.3a}{88/02/14}{`theoremfont' now a toks register.}
%
% \changes{v2.0a}{88/11/16}{New implementation and new documentation.}
% \changes{v2.0a}{88/11/16}{No layout changes but internal modifications
%                           old substyles won't work.}
% \changes{v2.0b}{89/01/30}{`global' used for all definitions.}
% \changes{v2.0c}{89/06/15}{Documentation translated to English.
%                            Thanks to LLL and Nina}
% \changes{v2.1a}{90/02/15}{Support for new font selection.}
% \changes{v2.2a}{94/02/03}{Cleanup for LaTeX2e}
%
% \DoNotIndex{\@endparenv,\@ifdefinable,\@ifnextchar,\@ifundefined}
% \DoNotIndex{\@namedef,\@nameuse,\@spaces,\@tempa,\@warning}
% \DoNotIndex{\ }
% \DoNotIndex{\begingroup,\bfseries}
% \DoNotIndex{\csname}
% \DoNotIndex{\def,\do,\docdate}
% \DoNotIndex{\endcsname,\endgroup,\endinput,\endtrivlist,\expandafter}
% \DoNotIndex{\filedate,\fileversion}
% \DoNotIndex{\gdef,\global}
% \DoNotIndex{\hbox,\hskip}
% \DoNotIndex{\ignorespaces,\input,\itshape,\item}
% \DoNotIndex{\labelsep,\let,\llap}
% \DoNotIndex{\makeatletter}
% \DoNotIndex{\noexpand}
% \DoNotIndex{\refstepcounter,\relax,\rlap}
% \DoNotIndex{\setlength,\slshape,\space,\string,\strut}
% \DoNotIndex{\the,\trivlist,\typeout}
% \DoNotIndex{\vbox}
% \DoNotIndex{\xdef}
%
% \def\deflabel#1{\ttfamily #1\hfill}
% \def\deflist#1{\begin{list}{}{\settowidth\labelwidth{\ttfamily #1}%
%                               \setlength\leftmargin\labelwidth
%                               \addtolength\leftmargin\labelsep
%                               \let\makelabel\deflabel}}
% \def\enddeflist{\end{list}}
% \def\envfont{\normalfont\ttfamily}
%
% \MakeShortVerb{\|}
%
% \newcommand{\TUB}{{\sl TUGboat\/}}
%
% \GetFileInfo{theorem.dtx}
% \title{An Extension of the
%        \LaTeX{} theorem environment\thanks{This file
%        has version number \fileversion, last
%        revised \filedate.}}
% \author{Frank Mittelbach}
%
% \MaintainedByLaTeXTeam{tools}
% \maketitle
%
% \begin{abstract}
%    The macros described in this paper yield an extension of the
%    \LaTeX{} theorem mechanism. It is designed to satisfy the
%    different requirements of various journals. Thus, the
%    layout of the ``theorems'' can be manipulated by determining a
%    ``style''. This article describes not only the use, but
%    also the definition, of the necessary macros.
% \end{abstract}
%
% \section*{Preface to version 2.2}
%
% For \LaTeXe{} this package did not need any fundamental changes. I
% only modified the messages generated so that theorem layout styles
% will show up with the |\listfiles| command and cleaned the section
% on the New Font Selection Scheme since this is now included in
% \LaTeX.
%
% \section*{Preface to version 2.1}
%
% This version is identical to 2.0g described in \TUB\ 10\#3 except
% for some internal defaults which are now set depending on the used
% font selection scheme.
%
% This was done to avoid unpleasant surprises if the new font selection
% scheme is in force. For further details see
% section~\ref{sec:newfonts} and \cite{bk:GMS94}.
%
% \section{Introduction}
%
% For our purposes here, ``theorems'' are labelled enunciations,
% often set off from the main text by extra space and a font change.
% Theorems, corollaries, conjectures, definitions, and remarks
% are all instances of ``theorems''.  The ``header'' of these structures
% is composed of a label (such as \textsc{Theorem} or \textsc{Remark})
% and a number which serializes an item in the sequence of items
% with the same label.
%
% Shortly after the introduction of \LaTeX{} at the Fachbereich
% Mathematik in Mainz, the desire to manipulate the layout of
% ``theorems'' arose.  In Mainz, the following two conventions came
% into general use:
% \begin{enumerate}
%   \item The number of the theorem is shown in the margin.
%   \item There is a line break at
%         the end of the theorem header.
% \end{enumerate}
%
% Additionally, some journals require different formats which depend on
% the ``sort of theorem'': e.g.\ often remarks and definitions are set
% in |\upshape|, while |\itshape| is employed for main theorems.
%
% Confronted with these requirements, a theorem environment was
% developed in Mainz which allows separate determination of the
% layout of the ``theorems sets'', comparable to |\pagestyle|.
%
%
% \section{The user interface}
% \subsection{Defining new theorem sets}
%
% \DescribeMacro\newtheorem
% As in the original \LaTeX{} version, the command |\newtheorem|
% defines a new ``theorem set'' or ``theorem-like structure''.
% Two required arguments name the new environment and give the text
% to be typeset with each instance of the new ``set'', while
% an optional argument determines how
% the ``set'' is enumerated:
% \begin{description}
%    \item[\ttfamily \bslash newtheorem\{foo\}\{bar\}]
%       The theorem set {\envfont foo} (whose name is \texttt{bar})
%       uses its own counter.
%    \item[\ttfamily \bslash newtheorem\{foo2\}{[foo]}\{bar2\}]
%       The theorem set {\envfont foo2} (printed name \texttt{bar2})
%       uses the same counter as the theorem set \texttt{foo}.
%    \item[\ttfamily \bslash newtheorem\{foo3\}\{bar3\}{[section]}]
%       The theorem set {\envfont foo3} (printed name \texttt{bar3}) is
%       enumerated within the counter \texttt{section}, i.e.\ with every
%       new |\section| the enumeration begins again with 1, and
%       the enumeration is composed from the section-number and the
%       theorem counter itself.
% \end{description}
%
% \DescribeMacro\theoremstyle
% Additionally, the command |\theoremstyle| can define the layout
% of various, or all, theorem sets.  It should be noted that any
% theorem set defined by |\newtheorem| is typeset in the
% |\theoremstyle| that is current at the time of the definition.
% Thus, the following
% \begin{verbatim}
%    \theoremstyle{break}        \newtheorem{Cor}{Corollary}
%    \theoremstyle{plain}        \newtheorem{Exa}{Example}[section]
%\end{verbatim}                             ^^A No blank at line start
% leads to the result that the set {\envfont Cor} is formatted in the
% style \texttt{break}, while the set {\envfont Exa} and all the
% following ones are formatted in the style \texttt{plain}, unless
% another |\theoremstyle| follows.  Since the definitions installed by
% |\newtheorem| are global, one also can limit |\theoremstyle| locally
% by grouping braces.
%
% \DescribeMacro\theorembodyfont
% The choice of the font for the theorem body is completely
% independent of the chosen |\theoremstyle|; this has proven to
% be very advantageous.  For example,
% \begin{verbatim}
%   {\theorembodyfont{\upshape}          \newtheorem{Rem}{Remark}}
%\end{verbatim}
% defines a theorem set {\envfont Rem}, which will be set in
% |\upshape| in the current layout (which in our example is
% \texttt{plain}).  As with |\theoremstyle|, the |\theorembodyfont|
% chosen is that current at the time of |\newtheorem|.  If
% |\theorembodyfont| is not specified or one defines
% |\theorembodyfont{}|, then the font used will be that defined by the
% |\theoremstyle|.
%
% \DescribeMacro\theoremheaderfont It is also possible to customize
% the font used for the theorem headers.  This is, however, a global
% declaration, and therefore there should be at most one
% |\theoremheaderfont| declaration in the preamble.\footnote{If it is
% actually necessary to have different header fonts, one has to define
% new theorem styles (substituting the desired font) or specify the
% information directly in the \texttt{\bslash newtheorem} declaration
% (the unclean variant).}
%
% \DescribeMacro\theorempreskipamount
% \DescribeMacro\theorempostskipamount Two additional parameters
% affect the vertical space around the theorem environments:
% |\theorempreskipamount| and |\theorempostskipamount| define,
% respectively, the spacing before and after such an environment.
% These parameters apply for all theorem sets and can be manipulated
% with the ordinary length macros.  They are rubber lengths
% (`\textsf{skips}'), and therefore can contain \texttt{plus} and
% \texttt{minus} parts.
%
% Since the definition of theorem sets should---most sensibly---be
% placed in the preamble, we only allow installation there.
% It is therefore possible to release the memory used here after
% |\begin{document}|, in order to make room for other
% applications.
%
%
% \subsection{Existing theorem styles}
%
% The following theorem styles exist to date:
% \begin{deflist}{marginbreak}
%    \item[plain]
%       This theorem style emulates the original \LaTeX{} definition,
%       except that additionally the parameters
%       |\theorem...skipamount| are used.
%    \item[break]
%       In this style, the theorem header is followed by a line
%       break.
%    \item[marginbreak]
%       The theorem number is set in the margin, and there is a line
%       break as in \texttt{break}.
%    \item[changebreak]
%       Like \texttt{break}, but with header number and text
%       interchanged.
%    \item[change]
%       Header number and text are interchanged, without a line break.
%    \item[margin]
%       The number is set in the left margin, without a line break.
% \end{deflist}
% All styles (except \texttt{plain}) select |\slshape| as the default
% |\theorembodyfont|.
%
%
% \subsection{Examples}
%
% Given the above theorem sets {\envfont Cor}, {\envfont Exa}
% and {\envfont Rem},
% suppose that the preamble also contains the declarations:
% \begin{verbatim}
%    \theoremstyle{marginbreak}   \newtheorem{Lem}[Cor]{Lemma}
%    \theoremstyle{change}
%    \theorembodyfont{\itshape}        \newtheorem{Def}[Cor]{Definition}
%
%    \theoremheaderfont{\scshape}
%\end{verbatim}
% Then the following are some typical examples of the typeset output
% resulting from their use.
%
% \begin{Cor}
%    This is a sentence typeset in the theorem environment {\envfont
%    Cor}.
% \end{Cor}
%
% \begin{Exa}
%    This is a sentence typeset in the theorem environment {\envfont
%    Exa}.
% \end{Exa}
%
% \begin{Rem}
%    This is a sentence typeset in the theorem environment {\envfont
%    Rem}.
% \end{Rem}
%
% \begin{Lem}[Ben User]
%    This is a sentence typeset in the theorem environment {\envfont
%    Lem}.
% \end{Lem}
%
% \begin{Def}[Very impressive Definition]
%    This is a sentence typeset in the theorem environment {\envfont
%    Def}.
% \end{Def}
%
% The last two examples show the effect of the optional argument to
% a theorem environment (it is the text typeset in parentheses).
%
%
% \section{Special Considerations}
% \label{sec:newfonts}
%
% Theoremheader and body are implemented as a unit. This means that
% the |\theoremheaderfont|  will inherit characteristics of the
% |\theorembodyfont| in \LaTeXe. Thus, if for example
% |\theorembodyfont| is |\itshape| and |\theoremheaderfont|
% is |\bfseries| the font selected for the header will have the
% characteristics `bold extended italic'. If this is not desired one
% should set the |\theoremheaderfont| to something like
% \begin{verbatim}
%      \theoremheaderfont{\normalfont\bfseries}
%\end{verbatim}
% i.e.\ supplying all necessary font information explicitly.
%
%
%
% \section{Acknowledgements}
%
% The publication of this set of macros was only possible with the
% help of Christina Busse (translating the manuscript into English),
% Joachim Pense (playing the r\^ole of typist), Chris Rowley
% (looking everything over) and many others providing useful
% suggestions.
%
% \MaybeStop{%
%    \begin{thebibliography}{9}
%       \bibitem{bk:GMS94} \textsc{M.~Goossens}, \textsc{F.~Mittelbach}
%         and \textsc{A.~Samarin}.
%         \newblock The \LaTeX{} Companion.
%         \newblock
%         Addison-Wesley, Reading, Massachusetts, 1994.
%       \bibitem{latex:source} \textsc{Lamport, Leslie.} \newblock
%                               \texttt{latex.tex}, version 2.09, date
%                               Feb. 1990.
%    \end{thebibliography}
%    \PrintIndex
%    \PrintChanges}
%
%
% \section{The documentation driver file}
%
% The next bit of code contains the documentation driver file for
% \TeX{}, i.e., the file that will produce the documentation you are
% currently reading. It will be extracted from this file by the
% \texttt{docstrip} program. Since it is the first code in the file
% one can alternatively process this file directly with \LaTeXe{} to
% obtain the documentation.
%
%    \begin{macrocode}
%<*driver>
\documentclass{ltxdoc}

\usepackage{theorem}

%<+driver>% The next few lines define theorem sets which are used
%<+driver>% in the example section of the documentation.

\theoremstyle{break}       \newtheorem{Cor}{Corollary}
\theoremstyle{plain}       \newtheorem{Exa}{Example}[section]
{\theorembodyfont{\upshape}\newtheorem{Rem}{Remark}}
\theoremstyle{marginbreak} \newtheorem{Lem}[Cor]{Lemma}
\theoremstyle{change}
\theorembodyfont{\itshape} \newtheorem{Def}[Cor]{Definition}

\theoremheaderfont{\scshape}

\RecordChanges

\begin{document}
   \DocInput{theorem.dtx}
\end{document}
%</driver>
%    \end{macrocode}
%
% \section{Definition of the Macros}
%
% If the file has been loaded before, we abort immediately. If not the
% package announces itself (this is actually done at the very top of
% the file---the way it is done isn't good style so don't copy it).
% \changes{v2.0e}{89/07/19}{Spaces removed from `typeout'.}
% \changes{v2.2b}{95/11/19}{Announce moved to top of file}
%    \begin{macrocode}
%<*package>
%\@ifundefined{theorem@style}{}{\endinput}
%\def\FMithmInfo{1995/11/19 v2.2b Theorem extension package (FMi)}
%\ProvidesPackage{theorem}[\FMithmInfo]
%    \end{macrocode}
%
%
% \subsection{Definition of theorem styles and fonts}
%
% All the definitions in this file are done globally to allow
% inputting this file inside a group.
%
% \begin{macro}{\theoremstyle}
%    Before a theorem style can be installed, the chosen style must be
%    known. For that reason, we must test to see that
%    |\th@|\meta{style} is known or, more precisely, that it is
%    different from |\relax|. If the style is not known then
%    |\th@plain| is used.
%    \begin{macrocode}
\gdef\theoremstyle#1{%
   \@ifundefined{th@#1}{\@warning
          {Unknown theoremstyle `#1'. Using `plain'}%
          \theorem@style{plain}}%
%    \end{macrocode}
%    We save the theorem style to be used in the token register
%    |\theorem@style|.
%    \begin{macrocode}
      {\theorem@style{#1}}%
%    \end{macrocode}
%    Now we ``evaluate'' the theorem style: this means, we call the
%    macro |\th@|\meta{style} which will activate the relevant
%    definitions which are contained in a separate file.  This is done
%    in a group to suppress changes to the current font.  This could
%    otherwise pose problems together with the new font selection
%    scheme\footnote{When I printed the original article using the new
%    font selection scheme I ended with a document with slanted
%    typefaces (text headings and all) simply because one of the
%    theorem styles used \texttt{\bslash sl} at toplevel.} if the
%    |\th@|\meta{style} is evaluated a second time.
% \changes{v2.1a}{90/02/15}{Group added around `th@..}
% \changes{v2.1a}{90/02/15}{Reseting of `..@begintheorem removed.}
%    \begin{macrocode}
      \begingroup
        \csname th@\the\theorem@style \endcsname
      \endgroup}
%    \end{macrocode}
% \end{macro}
%
%
% \begin{macro}{\@begintheorem}
% \changes{v2.1a}{90/02/15}{Relaxed at toplevel.}
% \begin{macro}{\@opargbegintheorem}
% \changes{v2.1a}{90/02/15}{Relaxed at toplevel.}
%    We reset |\@begintheorem| and |\@opargbegintheorem| to
%    |\relax| since these commands are no longer necessary at
%    toplevel. This will save a few tokens.
%    \begin{macrocode}
\global\let\@begintheorem\relax
\global\let\@opargbegintheorem\relax
%    \end{macrocode}
% \end{macro}
% \end{macro}
%
% \begin{macro}{\theorem@style}
%    Obviously the \textsf{token} register used above has to be
%    allocated. To assure the utmost compatibility with the
%    original \LaTeX{} definition, we set the default theorem style to
%    \texttt{plain}, which implements the usual \LaTeX{} convention.
% \changes{v2.0f}{89/07/20}{Unnecessary `global' removed.}
%    \begin{macrocode}
\newtoks\theorem@style
\global\theorem@style{plain}
%    \end{macrocode}
% \end{macro}
%
% \begin{macro}{\theorembodyfont}
% \begin{macro}{\theorem@bodyfont}
%    If the |\theorembodyfont| is set by the user then it should not
%    interact with the default font set in the theorem style. When the
%    new font selection is in force this may happen if, for example,
%    the default is |\itshape| and the new |\theorembodyfont| is
%    |\sffamily|. So we add a |\reset@font| command in front of the
%    user definition.
% \changes{v2.0f}{89/07/20}{Unnecessary `global' removed.}
% \changes{v2.0g}{89/09/19}{`theoremfont to `theorembodyfont.}
% \changes{v2.1c}{92/06/22}{`reset@font added and logic changed.}
% \changes{v2.1d}{92/10/28}{Reset to empty if arg is empty}
%    \begin{macrocode}
\gdef\theorembodyfont#1{%
%    \end{macrocode}
%    We check if the argument supplied is empty and if so put nothing
%    into the |\theorem@bodyfont| token register to allow for
%    |\theorembodyfont{}| as a mean of using the default of the
%    current |\theoremstyle|.
%    \begin{macrocode}
   \def\@tempa{#1}%
   \ifx\@tempa\@empty
    \theorem@bodyfont{}%
   \else
    \theorem@bodyfont{\reset@font#1}%
   \fi
   }
\newtoks\theorem@bodyfont
\global\theorem@bodyfont{}
%    \end{macrocode}
% \end{macro}
% \end{macro}
%
%
% \begin{macro}{\theoremheaderfont}
% \changes{v2.0g}{89/09/19}{Macro added.}
%    The font for the theorem headers is handled differently because
%    this definition applies to all theorem styles.
%    \begin{macrocode}
\gdef\theoremheaderfont#1{\gdef\theorem@headerfont{#1}%
%    \end{macrocode}
%    After using the macro once it is redefined to produce an error
%    message.
%    \begin{macrocode}
       \gdef\theoremheaderfont##1{%
        \typeout{\string\theoremheaderfont\space should be used
                 only once.}}}
%    \end{macrocode}
% \end{macro}
%
% \begin{macro}{\theorem@headerfont}
% \changes{v2.0g}{89/09/19}{Macro added.}
%    To set the |\theorem@headerfont| default we first test if
%    the new fontselection scheme is in force.
% \changes{v2.1a}{90/02/15}{Support for new font selection.}
%    \begin{macrocode}
\ifx\upshape\undefined
%    \end{macrocode}
%    If not we define it to expand into |\bfseries|. We don't use
%    |\let| just in case a following style option redefines this
%    macro.
%    \begin{macrocode}
\gdef\theorem@headerfont{\bfseries}
%    \end{macrocode}
%    Otherwise we reset the current shape before calling |\bfseries|.
%    \begin{macrocode}
\else \gdef\theorem@headerfont{\normalfont\bfseries}\fi
%    \end{macrocode}
% \end{macro}
%
% \begin{macro}{\th@plain}
% \begin{macro}{\th@break}
% \begin{macro}{\th@marginbreak}
% \begin{macro}{\th@changebreak}
% \begin{macro}{\th@change}
% \begin{macro}{\th@margin}
%    The different styles are defined in macros such as
%    |\th@plain|.  Since memory space is precious in
%    ``non-Big-versions'', we have to avoid offering too many unused
%    definitions.  Therefore we define these styles in separate files
%    that can be loaded on demand.  Thus the commands themselves only
%    load these files. We use |\@input@|, a \LaTeXe{} internal command
%    that ensures that the file will be listed with |\listfiles|.
% \changes{v2.2a}{94/02/02}{Use `@input@ to load theorem layout files}
%    \begin{macrocode}
\gdef\th@plain{\@input@{thp.sty}}
\gdef\th@break{\@input@{thb.sty}}
\gdef\th@marginbreak{\@input@{thmb.sty}}
\gdef\th@changebreak{\@input@{thcb.sty}}
\gdef\th@change{\@input@{thc.sty}}
\gdef\th@margin{\@input@{thm.sty}}
%    \end{macrocode}
%    This list will be expanded when new styles become
%    available.  For testing,  just append new theorem substyles
%    as document options.
% \end{macro}
% \end{macro}
% \end{macro}
% \end{macro}
% \end{macro}
% \end{macro}
%
%
%
% \subsection{Definition of a new theorem set}
%
% As already pointed out, a new theorem environment can be defined in
% three different ways:
% \begin{verbatim}
%    \newtheorem{Lem}{Lemma}
%    \newtheorem{Lem}{Lemma}[section]
%    \newtheorem{Lem}[Theorem]{Lemma}
%\end{verbatim}
% The function of the macro |\newtheorem| is to recognize these cases
% and then to branch into one of the three macros |\@ynthm|, |\@xnthm|
% or |\@othm|.  This mechanism is adopted unchanged from
% \cite{latex:source}; the essential point here is that, for example,
% in the second case, the arguments \texttt{Lem}, \texttt{Lemma} and
% \texttt{section} are passed over to the macro |\@xnthm|.
%
% We inspect this case first because the others present fewer problems,
% and thus are easily derived from this one.
%
% \begin{macro}{\@xnthm}
%    For our example arguments, the macro |\@xnthm| must fulfill
%    the following:
%    \begin{itemize}
%       \item
%          Define a new \LaTeX-counter `Lem'
%       \item
%          reset this counter within a |\section|
%       \item
%          define the macro |\theLem|
%       \item
%          define the environment macros |\Lem| and
%          |\endLem| using the current
%          |\theoremstyle| and |\theorem@bodyfont|.
%    \end{itemize}
%    Obviously, all this should happen only if the first argument
%    of |\@xnthm| (i.e.\ \texttt{Lem} in our example) is chosen so
%    as not to conflict with any previously defined commands or
%    environments.  This test is performed by the \LaTeX{} macro
%    |\@ifdefinable|.
%    \begin{macrocode}
\gdef\@xnthm#1#2[#3]{\expandafter\@ifdefinable\csname #1\endcsname
%    \end{macrocode}
%    Therefore, the first argument of |\@ifdefinable| is the
%    expansion (in the example, |\Lem|) of
%    |\csname||#1\endcsname|.  The second argument is
%    executed only if the test has been completed successfully.
%    \begin{macrocode}
   {%
%    \end{macrocode}
%    Now we define the new counter. The names of the \LaTeX{}
%    macros employed should speak for themselves:
% \changes{v2.2b}{1995/10/07}{Use \cs{@newctr}}
% \changes{v2.2c}{1995/11/23}{Use \cs{@newctr} properly.}
%    \begin{macrocode}
    \@definecounter{#1}\@newctr{#1}[#3]%
%    \end{macrocode}
%    Using |\@newctr| will give a proper error message if the counter
%    in |#3| is not defined.
%    In defining `|\theLem|' we must generate the desired macro
%    name by use of |\expandafter| and |\csname|.
%    \begin{macrocode}
    \expandafter\xdef\csname the#1\endcsname
%    \end{macrocode}
%    An |\xdef| is used in order to make the definition global,
%    and to ensure that it contains the replacement texts of
%    |\@thmcountersep| and |\@thmcounter|.\footnote{These
%    two macros can be defined by the document style.  Their default
%    values produce a `\texttt{.}' as separation and an arabic
%    representation of the number.} However, not everything should be
%    expanded. For example, it saves space to use |\thesection|
%    instead of its---at times---lengthy expansion.
%    \begin{macrocode}
      {\expandafter \noexpand \csname the#3\endcsname
       \@thmcountersep \@thmcounter{#1}}%
%    \end{macrocode}
%    Thus with the defaults of \LaTeX{}, |\theLem| would be
%    replaced by the command sequence |\thesection.\arabic{Lem}|.
%
%    We will now look at the definition of the macro which is executed
%    at the beginning of the actual environment (in our example this
%    macro is |\Lem|). It should be noted that we use an
%    ``|\expandafter| trick'' to expand only certain parts of the
%    replacement text at the time of the definition.
%    \begin{macrocode}
    \def\@tempa{\global\@namedef{#1}}%
    \expandafter \@tempa \expandafter{%
%    \end{macrocode}
%    First, the macro that contains the current definitions of
%    |\@begintheorem| and |\@opargtheorem| should be called
%    up.  The name of this macro---as is already known---has the form
%    |\th@|\meta{theorem\ style}; therefore, it must be called by
%    \begin{macrocode}
      \csname th@\the \theorem@style
%    \end{macrocode}
%    In addition the default theorem font should be changeable, i.e.\
%    we have to insert the contents of |\theorem@bodyfont|. For
%    that reason, we expand even further, beyond |\endcsname|,
%    and thus insert the contents of the \textsf{token} register
%    |\theorem@bodyfont| in the replacement text.
%    \begin{macrocode}
            \expandafter \endcsname \the \theorem@bodyfont
%    \end{macrocode}
%    Now it is time to call the macro |\@thm| which takes over
%    the further processing.  It has two arguments: the current
%    counter name (in our example, \texttt{Lem}), and the text of the
%    label (in our example, \texttt{Lemma}).
%    \begin{macrocode}
     \@thm{#1}{#2}}%
%    \end{macrocode}
%    With this, the `sub-definition' is complete.  The macro
%    |\@endtheorem| ends a theorem environment and is, so far, nothing
%    but an |\endtrivlist|.  (Hence it is defined globally, and not
%    within the theorem styles.\footnote{This has to be changed as
%    soon as theorem styles that change \texttt{\bslash @endtheorem}
%    exist.  In such a case, all existing styles must be changed as
%    well since they will have to reset the macro.}) Therefore, we can
%    set it equivalent to the macro that ends the theorem set (in our
%    example, |\endLem|).  However, if some day theorem styles exist
%    that do change |\@endtheorem|, we would have to use the
%    commented-out line instead.
%    \begin{macrocode}
    \global \expandafter \let \csname end#1\endcsname \@endtheorem
%   \global\@namedef{end#1}{\@endtheorem}%
%    \end{macrocode}
%    With these commands all the required definitions are employed,
%    unless the test |\@ifdefinable| has failed.  Therefore, we
%    end the second argument of this macro and with it the definition
%    of |\@xnthm|.
%    \begin{macrocode}
   }}
%    \end{macrocode}
% \end{macro}
%
% \begin{macro}{\@ynthm}
%    The definition of |\@ynthm| is completely analogous. In
%    this case the new counter that is defined is not reset within
%    another counter; thus the definition of |\the...|
%    is simplified:
%    \begin{macrocode}
\gdef\@ynthm#1#2{\expandafter\@ifdefinable\csname #1\endcsname
   {\@definecounter{#1}%
    \expandafter\xdef\csname the#1\endcsname{\@thmcounter{#1}}%
%    \end{macrocode}
%    The rest of the definition corresponds literally to that of
%    |\@xnthm|:
%    \begin{macrocode}
    \def\@tempa{\global\@namedef{#1}}\expandafter \@tempa
     \expandafter{\csname th@\the \theorem@style \expandafter
     \endcsname \the\theorem@bodyfont \@thm{#1}{#2}}%
    \global \expandafter \let \csname end#1\endcsname \@endtheorem}}
%    \end{macrocode}
% \end{macro}
%
% \begin{macro}{\@othm}
%    The definition of |\@othm| does not contain anything new.
%    \begin{macrocode}
\gdef\@othm#1[#2]#3{%
%    \end{macrocode}
% \changes{v2.2b}{95/10/07}{Build in check for counter defined}
%    We do not define a new counter but instead use one that has
%    already been defined.  Thus the only definition we need is that
%    of this pseudo-counter (i.e.\ |\the|\meta{env.\ name}).
%    First we check if |#2| corresponds to a known counter name.
%    \begin{macrocode}
  \expandafter\ifx\csname c@#2\endcsname\relax
   \@nocounterr{#2}%
  \else
   \expandafter\@ifdefinable\csname #1\endcsname
   {\expandafter \xdef \csname the#1\endcsname
     {\expandafter \noexpand \csname the#2\endcsname}%
%    \end{macrocode}
%    All other parts of the definition can be adopted from
%    |\@xnthm|. We have to remember, though, that in this case
%    the name of the current counter and the theorem label have moved
%    to the second and third arguments.
%    \begin{macrocode}
    \def\@tempa{\global\@namedef{#1}}\expandafter \@tempa
     \expandafter{\csname th@\the \theorem@style \expandafter
     \endcsname \the\theorem@bodyfont \@thm{#2}{#3}}%
    \global \expandafter \let \csname end#1\endcsname \@endtheorem}%
  \fi}
%    \end{macrocode}
% \end{macro}
%
% \subsection{Macros that are employed in a {\envfont theorem}
%    environment}
%
% \begin{macro}{\@thm}
%    The macro |\@thm| has to increase the current counter.  Then,
%    depending on whether the environment has (or does not have) an
%    optional argument, it has to branch into either |\@begintheorem|
%    or |\@opargtheorem|.
%    \begin{macrocode}
\gdef\@thm#1#2{\refstepcounter{#1}%
%    \end{macrocode}
%    Now we start a {\envfont trivlist} environment, and give
%    |\@topsep| and |\@topsepadd| the values of the \textsf{skip}
%    registers |\theorempreskipamount| and |\theorempostskipamount|.
%    The value in |\@topsep| is the vertical space that is inserted by
%    the first (and only) |\item| in our |\trivlist| whilst
%    |\@topsepadd| is inserted by |\@endparenv| at the end of that
%    {\envfont trivlist} environment.  By using these registers, we
%    obtain the desired space around a {\envfont theorem} environment.
%    \begin{macrocode}
   \trivlist
   \@topsep \theorempreskipamount               % used by first \item
   \@topsepadd \theorempostskipamount           % used by \@endparenv
%    \end{macrocode}
%    Now we have to test whether an optional argument has been given.
% \changes{v2.0d}{89/07/18}{Forgotten percent added.}
%    \begin{macrocode}
   \@ifnextchar [%
%    \end{macrocode}
%    If there is an optional argument, we will call |\@ythm|, and
%    move the arguments read back into the input stream.
%    \begin{macrocode}
   {\@ythm{#1}{#2}}%
%    \end{macrocode}
%    If not, we call |\@begintheorem|. Its first argument is the
%    name of the theorem set (hence the second argument of
%    |\@thm|). Its second argument is the macro that produces the
%    current number.
%    \begin{macrocode}
   {\@begintheorem{#2}{\csname the#1\endcsname}\ignorespaces}}
%    \end{macrocode}
% \end{macro}
%
% \begin{macro}{\@xthm}
% \begin{macro}{\@ythm}
%    Both these macros were originally called by |\@thm|.  We do
%    not need |\@xthm| anymore, hence we reset it to
%    |\relax|. The definition of |\@ythm| has not changed at
%    all from its definition in \LaTeX. In order to make the macros
%    easier to understand, we will
%    nevertheless present it (commented out).
%    \begin{macrocode}
\global\let\@xthm\relax
% \def\@ythm#1#2[#3]{\@opargbegintheorem{#2}{\csname
%           the#1\endcsname}{#3}\ignorespaces}
%</package>
%    \end{macrocode}
%    The primitive |\ignorespaces| in |\@ythm| and
%    |\@thm| is needed to remove the spaces between the
%    |\begin{...}| and the actual text.
% \end{macro}
% \end{macro}
%
%
% \subsection{Definition of the theorem substyles}
%
% As already pointed out, the theorem substyles, defined below, are
% only loaded when necessary.  Note that all these substyles, except
% \texttt{plain}, have |\slshape| as the default body font.
%
% \subsubsection{The \texttt{plain} style}
%
% \label{dok:th@plain} As the following macros use |@|, we have
% to locally set the |\catcode| of this symbol to ``letter''.
% This happens within a group, so that we do not have to worry about
% which |\catcode| that symbol had before.
%    \begin{macrocode}
%<*thp>
\begingroup \makeatletter
%    \end{macrocode}
% Since we are now within a group, we must make all definitions
% globally.  First we make sure that \texttt{theorem.sty} is loaded.
% This will allow us to use this file as a document style option
% without having to call \texttt{theorem} itself as an option.  At the
% same time, we assure that at least version 2 is loaded, since
% |\theorem@style| was not defined in earlier versions.
%    \begin{macrocode}
\@ifundefined{theorem@style}{\input{theorem.sty}}{}
\ProvidesFile{thp.sty}
             [\FMithmInfo]
%    \end{macrocode}
%
%
% \begin{macro}{\th@plain}
% \begin{macro}{\@begintheorem}
% \begin{macro}{\@opargbegintheorem}
%    |\theoremstyle{plain}| corresponds to the original
%    definition, except that the distances to the surrounding text are
%    determined by the parameters |\theorempreskipamount| and
%    |\theorempostskipamount|.
%    First we set the default body font.
%    \begin{macrocode}
\gdef\th@plain{\normalfont\itshape
%    \end{macrocode}
%    Then we define |\@begintheorem| and
%    |\@opargbegintheorem|. These two macros define how the
%    header of a theorem is typeset.  |\@opargbegintheorem| will
%    be called if a \textsf{theorem} environment with an optional
%    argument is encountered; otherwise, the header is constructed by
%    calling |\@begintheorem|.  If one of these macros is
%    executed, we are within a \textsf{trivlist} environment started by
%    |\@thm|.  So the theorem header is produced with an
%    |\item| command.
%
%    Instead of specifying the header font directly, all standard
%    theorem styles use the |\theorem@headerfont| macro to allow
%    customization.  The extra space (|\labelsep|) is necessary
%    because of problems in the \textsf{trivlist} environment.
%    \begin{macrocode}
  \def\@begintheorem##1##2{%
        \item[\hskip\labelsep \theorem@headerfont ##1\ ##2]}%
%    \end{macrocode}
%    The definition of |\@opargbegintheorem| is completely
%    analogous. The only difference is the fact that there exists a
%    third argument (which is the optional parameter of the
%    environment and contains additional information about the
%    theorem).  Customarily we enclose it in parentheses.
%    \begin{macrocode}
\def\@opargbegintheorem##1##2##3{%
   \item[\hskip\labelsep \theorem@headerfont ##1\ ##2\ (##3)]}}
%    \end{macrocode}
% \end{macro}
% \end{macro}
% \end{macro}
%
% \noindent
% We conclude with an |\endgroup| to restore the
% |\catcode| of |@|.
%    \begin{macrocode}
\endgroup
%</thp>
%    \end{macrocode}
%
%
% \subsubsection{The \texttt{break} style}
%
% This style option is stored in the file |thb.sty|.
% For the next two lines see the documentation for |\th@plain| on
% page~\pageref{dok:th@plain}.
%    \begin{macrocode}
%<*thb>
\begingroup \makeatletter
\@ifundefined{theorem@style}{\input{theorem.sty}}{}
\ProvidesFile{thb.sty}
             [\FMithmInfo]
%    \end{macrocode}
%
% \begin{macro}{\th@break}
%    |\theoremstyle{break}| produces a line break after the name
%    of the theorem.  The font is |\slshape|.  Hence, we define
%    |\th@break| as follows:
%    \begin{macrocode}
\gdef\th@break{\normalfont\slshape
  \def\@begintheorem##1##2{\item[%
%    \end{macrocode}
%    We run into the following problem: it is not possible to create
%    the header with |\item[|\meta{title}|]| and then start
%    a new line by, for example, |\mbox{}\\|. Such a definition
%    will fail whenever a list environment follows immediately.  With
%    the above construction, the |\mbox{}| causes the switch
%    |@inlabel| (cf.\ definition of |\list| and
%    |\trivlist| in \cite{latex:source}) to be set to \texttt{false}
%    and so the following list will insert additional vertical
%    space (|\topskip|). This is quite annoying. Therefore, we
%    create the line break within the |\item|. In order to ensure
%    that the text will begin at the proper position in the following
%    line, we simply pretend that the label does not take any
%    room.\footnote{This will lead to problems whenever very high
%    symbols occurring in the line tower into the
%    heading. So, something else has to be done here sometime.}
%    \begin{macrocode}
       \rlap{\vbox{\hbox{\hskip \labelsep\theorem@headerfont ##1\ ##2}%
                   \hbox{\strut}}}]}%
%    \end{macrocode}
%    Again, the definition of |\@opargbegintheorem| is
%    completely analogous.
%    \begin{macrocode}
\def\@opargbegintheorem##1##2##3{%
  \item[\rlap{\vbox{\hbox{\hskip \labelsep \theorem@headerfont
                     ##1\ ##2\ (##3)}%
                    \hbox{\strut}}}]}}
\endgroup
%</thb>
%    \end{macrocode}
% \end{macro}
%
%
% \subsubsection{The \texttt{changebreak} style}
%
%    \begin{macrocode}
% This style option is stored in the file |thcb.sty|.
%    \begin{macrocode}
%<*thcb>
\begingroup \makeatletter
\@ifundefined{theorem@style}{\input{theorem.sty}}{}
\ProvidesFile{thcb.sty}
             [\FMithmInfo]
%    \end{macrocode}
%
% \begin{macro}{\th@changebreak}
%    The \texttt{change-break} theorem style is like \texttt{break}
%    but with interchange of theorem name and theorem number.  Thus we
%    define |\th@changebreak| as follows:
%    \begin{macrocode}
\gdef\th@changebreak{\normalfont\slshape
  \def\@begintheorem##1##2{\item
       [\rlap{\vbox{\hbox{\hskip\labelsep \theorem@headerfont ##2\ ##1}%
                    \hbox{\strut}}}]}%
\def\@opargbegintheorem##1##2##3{%
  \item[\rlap{\vbox{\hbox{\hskip\labelsep \theorem@headerfont
                     ##2\ ##1\ (##3)}%
                    \hbox{\strut}}}]}}
\endgroup
%</thcb>
%    \end{macrocode}
% \end{macro}
%
%
% \subsubsection{The \texttt{change} style}
%
% This style option is stored in the file |thc.sty|.
%    \begin{macrocode}
%<*thc>
\begingroup \makeatletter
\@ifundefined{theorem@style}{\input{theorem.sty}}{}
\ProvidesFile{thc.sty}
             [\FMithmInfo]
%    \end{macrocode}
%
% \begin{macro}{\th@change}
%    The \texttt{change} theorem style corresponds to the
%    \texttt{change break} style without a linebreak after the header.
%    To say it in another way, it's the same as the \texttt{plain}
%    style but with number and name interchanged and |\slshape| as the
%    default font.
%    \begin{macrocode}
\gdef\th@change{\normalfont\slshape
  \def\@begintheorem##1##2{\item
       [\hskip\labelsep \theorem@headerfont ##2\ ##1]}%
\def\@opargbegintheorem##1##2##3{%
  \item[\hskip\labelsep \theorem@headerfont ##2\ ##1\ (##3)]}}
\endgroup
%</thc>
%    \end{macrocode}
% \end{macro}
%
%
% \subsubsection{The \texttt{marginbreak} style}
%
% This style option is the one used most often at Mainz.  It is saved
% in the file |thmb.sty|.
%    \begin{macrocode}
%<*thmb>
\begingroup \makeatletter
\@ifundefined{theorem@style}{\input{theorem.sty}}{}
\ProvidesFile{thmb.sty}
             [\FMithmInfo]
%    \end{macrocode}
%
% \begin{macro}{\th@marginbreak}
%    The \texttt{margin break} style is nearly the same as the
%    \texttt{change break} style. The only difference is the placement
%    of the theorem number.  We use |\llap| to place it in the left
%    margin.
%
%    In this style |\labelsep| denotes the separation between the
%    number and the text.
%    \begin{macrocode}
\gdef\th@marginbreak{\normalfont\slshape
  \def\@begintheorem##1##2{\item
       [\rlap{\vbox{\theorem@headerfont
                    \hbox{\llap{##2}\hskip\labelsep ##1}%
                    \hbox{\strut}}}]}%
\def\@opargbegintheorem##1##2##3{%
  \item[\rlap{\vbox{\theorem@headerfont
                    \hbox{\llap{##2}\hskip\labelsep ##1\ (##3)}%
                    \hbox{\strut}}}]}}
\endgroup
%</thmb>
%    \end{macrocode}
% \changes{v2.1c}{90/02/17}{`endgroup outside macrocode env.}
% \end{macro}
%
%
%
% \subsubsection{The \texttt{margin} style}
%
% This style option is stored in the file |thm.sty|.
%    \begin{macrocode}
%<*thm>
\begingroup \makeatletter
\@ifundefined{theorem@style}{\input{theorem.sty}}{}
\ProvidesFile{thm.sty}
             [\FMithmInfo]
%    \end{macrocode}
%
% \begin{macro}{\th@margin}
%    Again this is only a variant of the theorem styles described
%    above without any new ideas.
%    \begin{macrocode}
\gdef\th@margin{\normalfont\slshape
  \def\@begintheorem##1##2{\item
       [\theorem@headerfont \llap{##2}\hskip\labelsep ##1]}%
\def\@opargbegintheorem##1##2##3{%
  \item[\theorem@headerfont \llap{##2}\hskip\labelsep ##1\ (##3)]}}
\endgroup
%</thm>
%    \end{macrocode}
% \end{macro}
%
%
%
%
% \subsection{Final Definitions}
%
% \begin{macro}{\theorempreskipamount}
% \begin{macro}{\theorempostskipamount}
%    The \textsf{skip} parameters that regulate the vertical empty space
%    before and after the {\envfont theorem} environment have
%    to be allocated as well.
% \changes{v2.0f}{89/07/20}{Unnecessary `global' removed.}
%    \begin{macrocode}
%<*package>
\newskip\theorempreskipamount
\newskip\theorempostskipamount
%    \end{macrocode}
%    Since we have used the same values for all theorem sets, we now can
%    assign them.
%    \begin{macrocode}
\global\setlength\theorempreskipamount{12pt plus 5pt minus 3pt}
\global\setlength\theorempostskipamount{8pt plus 3pt minus 1.5pt}
%    \end{macrocode}
% \end{macro}
% \end{macro}
%
%
% \begin{macro}{\@endtheorem}
%    The same holds for the macro |\@endtheorem|, which ends a
%    {\envfont theorem} environment. Since it is the same for all
%    theorem sets, it is removed from the macros |\th@|\meta{style}.
%    It simply ends the {\envfont trivlist} environment, which was
%    begun in |\@thm|.
%    \begin{macrocode}
\global\let\@endtheorem=\endtrivlist
%    \end{macrocode}
% \end{macro}
%
%
%
% \begin{macro}{\@preamblecmds}
%    All macros defined above are to be used only in the preamble.
%    Therefore, we insert them in |\@preamblecmds| which will disable
%    them at begin document. This is done by the internal \LaTeXe{}
%    command |\@onlypreamble|.
% \changes{v2.0g}{89/09/19}{`theoremheaderfont and `theorembodyfont
%    added.}
% \changes{v2.2a}{94/02/03}{Use \LaTeXe{} interface}
%    \begin{macrocode}
\@onlypreamble\@xnthm
\@onlypreamble\@ynthm
\@onlypreamble\@othm
\@onlypreamble\newtheorem
\@onlypreamble\theoremstyle
\@onlypreamble\theorembodyfont
\@onlypreamble\theoremheaderfont
%    \end{macrocode}
% \end{macro}
%
% Finally we declare the \texttt{plain} theorem style to be the default.
% \changes{v2.1b}{90/06/04}{Fixed missing `item by declaring default
%    style.}
%    \begin{macrocode}
\theoremstyle{plain}
%</package>
%    \end{macrocode}
%
% \Finale

\endinput
}{}
\ProvidesFile{thb.sty}
             [\FMithmInfo]
%    \end{macrocode}
%
% \begin{macro}{\th@break}
%    |\theoremstyle{break}| produces a line break after the name
%    of the theorem.  The font is |\slshape|.  Hence, we define
%    |\th@break| as follows:
%    \begin{macrocode}
\gdef\th@break{\normalfont\slshape
  \def\@begintheorem##1##2{\item[%
%    \end{macrocode}
%    We run into the following problem: it is not possible to create
%    the header with |\item[|\meta{title}|]| and then start
%    a new line by, for example, |\mbox{}\\|. Such a definition
%    will fail whenever a list environment follows immediately.  With
%    the above construction, the |\mbox{}| causes the switch
%    |@inlabel| (cf.\ definition of |\list| and
%    |\trivlist| in \cite{latex:source}) to be set to \texttt{false}
%    and so the following list will insert additional vertical
%    space (|\topskip|). This is quite annoying. Therefore, we
%    create the line break within the |\item|. In order to ensure
%    that the text will begin at the proper position in the following
%    line, we simply pretend that the label does not take any
%    room.\footnote{This will lead to problems whenever very high
%    symbols occurring in the line tower into the
%    heading. So, something else has to be done here sometime.}
%    \begin{macrocode}
       \rlap{\vbox{\hbox{\hskip \labelsep\theorem@headerfont ##1\ ##2}%
                   \hbox{\strut}}}]}%
%    \end{macrocode}
%    Again, the definition of |\@opargbegintheorem| is
%    completely analogous.
%    \begin{macrocode}
\def\@opargbegintheorem##1##2##3{%
  \item[\rlap{\vbox{\hbox{\hskip \labelsep \theorem@headerfont
                     ##1\ ##2\ (##3)}%
                    \hbox{\strut}}}]}}
\endgroup
%</thb>
%    \end{macrocode}
% \end{macro}
%
%
% \subsubsection{The \texttt{changebreak} style}
%
%    \begin{macrocode}
% This style option is stored in the file |thcb.sty|.
%    \begin{macrocode}
%<*thcb>
\begingroup \makeatletter
\@ifundefined{theorem@style}{% \iffalse meta-comment
%
% Copyright (C) 1993-2024
%
% The LaTeX Project and any individual authors listed elsewhere
% in this file.
%
% This file is part of the Standard LaTeX `Tools Bundle'.
% -------------------------------------------------------
%
% It may be distributed and/or modified under the
% conditions of the LaTeX Project Public License, either version 1.3c
% of this license or (at your option) any later version.
% The latest version of this license is in
%    https://www.latex-project.org/lppl.txt
% and version 1.3c or later is part of all distributions of LaTeX
% version 2005/12/01 or later.
%
% The list of all files belonging to the LaTeX `Tools Bundle' is
% given in the file `manifest.txt'.
%
% \fi
%
%%
%
%
\def\FMithmInfo{2023/07/05 v2.2c Theorem extension package (FMi)}
%
% \ProvidesFile{theorem.dtx}[\FMithmInfo]
%\iffalse   % this is a METACOMMENT !
%%
%% Package `theorem' to use with LaTeX2e
%% Copyright (C) 1989-1995 Frank Mittelbach, all rights reserved.
%<+package>\@ifundefined{theorem@style}{}{\endinput}
%<+package>\NeedsTeXFormat{LaTeX2e}[1995/06/01]
%<+package>\ProvidesPackage{theorem}
%<+package>         [\FMithmInfo]
%
% \fi
%
% \changes{v1.0}{87/05/07}{First draft.}
% \changes{v1.0a}{87/05/07}{`thistheoremstyle' added.}
% \changes{v1.1a}{87/07/29}{New concept: redefine the internal
%                          \protect\LaTeX{} commands `@xnthm' etc.}
% \changes{v1.1b}{87/07/31}{`theoremfont' used to specify font.}
% \changes{v1.2a}{88/01/08}{`@endtheorem' now globally defined.}
% \changes{v1.2b}{88/01/11}{`newtheorem' only allowed in the preamble.}
% \changes{v1.3a}{88/02/14}{`theoremfont' now a toks register.}
%
% \changes{v2.0a}{88/11/16}{New implementation and new documentation.}
% \changes{v2.0a}{88/11/16}{No layout changes but internal modifications
%                           old substyles won't work.}
% \changes{v2.0b}{89/01/30}{`global' used for all definitions.}
% \changes{v2.0c}{89/06/15}{Documentation translated to English.
%                            Thanks to LLL and Nina}
% \changes{v2.1a}{90/02/15}{Support for new font selection.}
% \changes{v2.2a}{94/02/03}{Cleanup for LaTeX2e}
%
% \DoNotIndex{\@endparenv,\@ifdefinable,\@ifnextchar,\@ifundefined}
% \DoNotIndex{\@namedef,\@nameuse,\@spaces,\@tempa,\@warning}
% \DoNotIndex{\ }
% \DoNotIndex{\begingroup,\bfseries}
% \DoNotIndex{\csname}
% \DoNotIndex{\def,\do,\docdate}
% \DoNotIndex{\endcsname,\endgroup,\endinput,\endtrivlist,\expandafter}
% \DoNotIndex{\filedate,\fileversion}
% \DoNotIndex{\gdef,\global}
% \DoNotIndex{\hbox,\hskip}
% \DoNotIndex{\ignorespaces,\input,\itshape,\item}
% \DoNotIndex{\labelsep,\let,\llap}
% \DoNotIndex{\makeatletter}
% \DoNotIndex{\noexpand}
% \DoNotIndex{\refstepcounter,\relax,\rlap}
% \DoNotIndex{\setlength,\slshape,\space,\string,\strut}
% \DoNotIndex{\the,\trivlist,\typeout}
% \DoNotIndex{\vbox}
% \DoNotIndex{\xdef}
%
% \def\deflabel#1{\ttfamily #1\hfill}
% \def\deflist#1{\begin{list}{}{\settowidth\labelwidth{\ttfamily #1}%
%                               \setlength\leftmargin\labelwidth
%                               \addtolength\leftmargin\labelsep
%                               \let\makelabel\deflabel}}
% \def\enddeflist{\end{list}}
% \def\envfont{\normalfont\ttfamily}
%
% \MakeShortVerb{\|}
%
% \newcommand{\TUB}{{\sl TUGboat\/}}
%
% \GetFileInfo{theorem.dtx}
% \title{An Extension of the
%        \LaTeX{} theorem environment\thanks{This file
%        has version number \fileversion, last
%        revised \filedate.}}
% \author{Frank Mittelbach}
%
% \MaintainedByLaTeXTeam{tools}
% \maketitle
%
% \begin{abstract}
%    The macros described in this paper yield an extension of the
%    \LaTeX{} theorem mechanism. It is designed to satisfy the
%    different requirements of various journals. Thus, the
%    layout of the ``theorems'' can be manipulated by determining a
%    ``style''. This article describes not only the use, but
%    also the definition, of the necessary macros.
% \end{abstract}
%
% \section*{Preface to version 2.2}
%
% For \LaTeXe{} this package did not need any fundamental changes. I
% only modified the messages generated so that theorem layout styles
% will show up with the |\listfiles| command and cleaned the section
% on the New Font Selection Scheme since this is now included in
% \LaTeX.
%
% \section*{Preface to version 2.1}
%
% This version is identical to 2.0g described in \TUB\ 10\#3 except
% for some internal defaults which are now set depending on the used
% font selection scheme.
%
% This was done to avoid unpleasant surprises if the new font selection
% scheme is in force. For further details see
% section~\ref{sec:newfonts} and \cite{bk:GMS94}.
%
% \section{Introduction}
%
% For our purposes here, ``theorems'' are labelled enunciations,
% often set off from the main text by extra space and a font change.
% Theorems, corollaries, conjectures, definitions, and remarks
% are all instances of ``theorems''.  The ``header'' of these structures
% is composed of a label (such as \textsc{Theorem} or \textsc{Remark})
% and a number which serializes an item in the sequence of items
% with the same label.
%
% Shortly after the introduction of \LaTeX{} at the Fachbereich
% Mathematik in Mainz, the desire to manipulate the layout of
% ``theorems'' arose.  In Mainz, the following two conventions came
% into general use:
% \begin{enumerate}
%   \item The number of the theorem is shown in the margin.
%   \item There is a line break at
%         the end of the theorem header.
% \end{enumerate}
%
% Additionally, some journals require different formats which depend on
% the ``sort of theorem'': e.g.\ often remarks and definitions are set
% in |\upshape|, while |\itshape| is employed for main theorems.
%
% Confronted with these requirements, a theorem environment was
% developed in Mainz which allows separate determination of the
% layout of the ``theorems sets'', comparable to |\pagestyle|.
%
%
% \section{The user interface}
% \subsection{Defining new theorem sets}
%
% \DescribeMacro\newtheorem
% As in the original \LaTeX{} version, the command |\newtheorem|
% defines a new ``theorem set'' or ``theorem-like structure''.
% Two required arguments name the new environment and give the text
% to be typeset with each instance of the new ``set'', while
% an optional argument determines how
% the ``set'' is enumerated:
% \begin{description}
%    \item[\ttfamily \bslash newtheorem\{foo\}\{bar\}]
%       The theorem set {\envfont foo} (whose name is \texttt{bar})
%       uses its own counter.
%    \item[\ttfamily \bslash newtheorem\{foo2\}{[foo]}\{bar2\}]
%       The theorem set {\envfont foo2} (printed name \texttt{bar2})
%       uses the same counter as the theorem set \texttt{foo}.
%    \item[\ttfamily \bslash newtheorem\{foo3\}\{bar3\}{[section]}]
%       The theorem set {\envfont foo3} (printed name \texttt{bar3}) is
%       enumerated within the counter \texttt{section}, i.e.\ with every
%       new |\section| the enumeration begins again with 1, and
%       the enumeration is composed from the section-number and the
%       theorem counter itself.
% \end{description}
%
% \DescribeMacro\theoremstyle
% Additionally, the command |\theoremstyle| can define the layout
% of various, or all, theorem sets.  It should be noted that any
% theorem set defined by |\newtheorem| is typeset in the
% |\theoremstyle| that is current at the time of the definition.
% Thus, the following
% \begin{verbatim}
%    \theoremstyle{break}        \newtheorem{Cor}{Corollary}
%    \theoremstyle{plain}        \newtheorem{Exa}{Example}[section]
%\end{verbatim}                             ^^A No blank at line start
% leads to the result that the set {\envfont Cor} is formatted in the
% style \texttt{break}, while the set {\envfont Exa} and all the
% following ones are formatted in the style \texttt{plain}, unless
% another |\theoremstyle| follows.  Since the definitions installed by
% |\newtheorem| are global, one also can limit |\theoremstyle| locally
% by grouping braces.
%
% \DescribeMacro\theorembodyfont
% The choice of the font for the theorem body is completely
% independent of the chosen |\theoremstyle|; this has proven to
% be very advantageous.  For example,
% \begin{verbatim}
%   {\theorembodyfont{\upshape}          \newtheorem{Rem}{Remark}}
%\end{verbatim}
% defines a theorem set {\envfont Rem}, which will be set in
% |\upshape| in the current layout (which in our example is
% \texttt{plain}).  As with |\theoremstyle|, the |\theorembodyfont|
% chosen is that current at the time of |\newtheorem|.  If
% |\theorembodyfont| is not specified or one defines
% |\theorembodyfont{}|, then the font used will be that defined by the
% |\theoremstyle|.
%
% \DescribeMacro\theoremheaderfont It is also possible to customize
% the font used for the theorem headers.  This is, however, a global
% declaration, and therefore there should be at most one
% |\theoremheaderfont| declaration in the preamble.\footnote{If it is
% actually necessary to have different header fonts, one has to define
% new theorem styles (substituting the desired font) or specify the
% information directly in the \texttt{\bslash newtheorem} declaration
% (the unclean variant).}
%
% \DescribeMacro\theorempreskipamount
% \DescribeMacro\theorempostskipamount Two additional parameters
% affect the vertical space around the theorem environments:
% |\theorempreskipamount| and |\theorempostskipamount| define,
% respectively, the spacing before and after such an environment.
% These parameters apply for all theorem sets and can be manipulated
% with the ordinary length macros.  They are rubber lengths
% (`\textsf{skips}'), and therefore can contain \texttt{plus} and
% \texttt{minus} parts.
%
% Since the definition of theorem sets should---most sensibly---be
% placed in the preamble, we only allow installation there.
% It is therefore possible to release the memory used here after
% |\begin{document}|, in order to make room for other
% applications.
%
%
% \subsection{Existing theorem styles}
%
% The following theorem styles exist to date:
% \begin{deflist}{marginbreak}
%    \item[plain]
%       This theorem style emulates the original \LaTeX{} definition,
%       except that additionally the parameters
%       |\theorem...skipamount| are used.
%    \item[break]
%       In this style, the theorem header is followed by a line
%       break.
%    \item[marginbreak]
%       The theorem number is set in the margin, and there is a line
%       break as in \texttt{break}.
%    \item[changebreak]
%       Like \texttt{break}, but with header number and text
%       interchanged.
%    \item[change]
%       Header number and text are interchanged, without a line break.
%    \item[margin]
%       The number is set in the left margin, without a line break.
% \end{deflist}
% All styles (except \texttt{plain}) select |\slshape| as the default
% |\theorembodyfont|.
%
%
% \subsection{Examples}
%
% Given the above theorem sets {\envfont Cor}, {\envfont Exa}
% and {\envfont Rem},
% suppose that the preamble also contains the declarations:
% \begin{verbatim}
%    \theoremstyle{marginbreak}   \newtheorem{Lem}[Cor]{Lemma}
%    \theoremstyle{change}
%    \theorembodyfont{\itshape}        \newtheorem{Def}[Cor]{Definition}
%
%    \theoremheaderfont{\scshape}
%\end{verbatim}
% Then the following are some typical examples of the typeset output
% resulting from their use.
%
% \begin{Cor}
%    This is a sentence typeset in the theorem environment {\envfont
%    Cor}.
% \end{Cor}
%
% \begin{Exa}
%    This is a sentence typeset in the theorem environment {\envfont
%    Exa}.
% \end{Exa}
%
% \begin{Rem}
%    This is a sentence typeset in the theorem environment {\envfont
%    Rem}.
% \end{Rem}
%
% \begin{Lem}[Ben User]
%    This is a sentence typeset in the theorem environment {\envfont
%    Lem}.
% \end{Lem}
%
% \begin{Def}[Very impressive Definition]
%    This is a sentence typeset in the theorem environment {\envfont
%    Def}.
% \end{Def}
%
% The last two examples show the effect of the optional argument to
% a theorem environment (it is the text typeset in parentheses).
%
%
% \section{Special Considerations}
% \label{sec:newfonts}
%
% Theoremheader and body are implemented as a unit. This means that
% the |\theoremheaderfont|  will inherit characteristics of the
% |\theorembodyfont| in \LaTeXe. Thus, if for example
% |\theorembodyfont| is |\itshape| and |\theoremheaderfont|
% is |\bfseries| the font selected for the header will have the
% characteristics `bold extended italic'. If this is not desired one
% should set the |\theoremheaderfont| to something like
% \begin{verbatim}
%      \theoremheaderfont{\normalfont\bfseries}
%\end{verbatim}
% i.e.\ supplying all necessary font information explicitly.
%
%
%
% \section{Acknowledgements}
%
% The publication of this set of macros was only possible with the
% help of Christina Busse (translating the manuscript into English),
% Joachim Pense (playing the r\^ole of typist), Chris Rowley
% (looking everything over) and many others providing useful
% suggestions.
%
% \MaybeStop{%
%    \begin{thebibliography}{9}
%       \bibitem{bk:GMS94} \textsc{M.~Goossens}, \textsc{F.~Mittelbach}
%         and \textsc{A.~Samarin}.
%         \newblock The \LaTeX{} Companion.
%         \newblock
%         Addison-Wesley, Reading, Massachusetts, 1994.
%       \bibitem{latex:source} \textsc{Lamport, Leslie.} \newblock
%                               \texttt{latex.tex}, version 2.09, date
%                               Feb. 1990.
%    \end{thebibliography}
%    \PrintIndex
%    \PrintChanges}
%
%
% \section{The documentation driver file}
%
% The next bit of code contains the documentation driver file for
% \TeX{}, i.e., the file that will produce the documentation you are
% currently reading. It will be extracted from this file by the
% \texttt{docstrip} program. Since it is the first code in the file
% one can alternatively process this file directly with \LaTeXe{} to
% obtain the documentation.
%
%    \begin{macrocode}
%<*driver>
\documentclass{ltxdoc}

\usepackage{theorem}

%<+driver>% The next few lines define theorem sets which are used
%<+driver>% in the example section of the documentation.

\theoremstyle{break}       \newtheorem{Cor}{Corollary}
\theoremstyle{plain}       \newtheorem{Exa}{Example}[section]
{\theorembodyfont{\upshape}\newtheorem{Rem}{Remark}}
\theoremstyle{marginbreak} \newtheorem{Lem}[Cor]{Lemma}
\theoremstyle{change}
\theorembodyfont{\itshape} \newtheorem{Def}[Cor]{Definition}

\theoremheaderfont{\scshape}

\RecordChanges

\begin{document}
   \DocInput{theorem.dtx}
\end{document}
%</driver>
%    \end{macrocode}
%
% \section{Definition of the Macros}
%
% If the file has been loaded before, we abort immediately. If not the
% package announces itself (this is actually done at the very top of
% the file---the way it is done isn't good style so don't copy it).
% \changes{v2.0e}{89/07/19}{Spaces removed from `typeout'.}
% \changes{v2.2b}{95/11/19}{Announce moved to top of file}
%    \begin{macrocode}
%<*package>
%\@ifundefined{theorem@style}{}{\endinput}
%\def\FMithmInfo{1995/11/19 v2.2b Theorem extension package (FMi)}
%\ProvidesPackage{theorem}[\FMithmInfo]
%    \end{macrocode}
%
%
% \subsection{Definition of theorem styles and fonts}
%
% All the definitions in this file are done globally to allow
% inputting this file inside a group.
%
% \begin{macro}{\theoremstyle}
%    Before a theorem style can be installed, the chosen style must be
%    known. For that reason, we must test to see that
%    |\th@|\meta{style} is known or, more precisely, that it is
%    different from |\relax|. If the style is not known then
%    |\th@plain| is used.
%    \begin{macrocode}
\gdef\theoremstyle#1{%
   \@ifundefined{th@#1}{\@warning
          {Unknown theoremstyle `#1'. Using `plain'}%
          \theorem@style{plain}}%
%    \end{macrocode}
%    We save the theorem style to be used in the token register
%    |\theorem@style|.
%    \begin{macrocode}
      {\theorem@style{#1}}%
%    \end{macrocode}
%    Now we ``evaluate'' the theorem style: this means, we call the
%    macro |\th@|\meta{style} which will activate the relevant
%    definitions which are contained in a separate file.  This is done
%    in a group to suppress changes to the current font.  This could
%    otherwise pose problems together with the new font selection
%    scheme\footnote{When I printed the original article using the new
%    font selection scheme I ended with a document with slanted
%    typefaces (text headings and all) simply because one of the
%    theorem styles used \texttt{\bslash sl} at toplevel.} if the
%    |\th@|\meta{style} is evaluated a second time.
% \changes{v2.1a}{90/02/15}{Group added around `th@..}
% \changes{v2.1a}{90/02/15}{Reseting of `..@begintheorem removed.}
%    \begin{macrocode}
      \begingroup
        \csname th@\the\theorem@style \endcsname
      \endgroup}
%    \end{macrocode}
% \end{macro}
%
%
% \begin{macro}{\@begintheorem}
% \changes{v2.1a}{90/02/15}{Relaxed at toplevel.}
% \begin{macro}{\@opargbegintheorem}
% \changes{v2.1a}{90/02/15}{Relaxed at toplevel.}
%    We reset |\@begintheorem| and |\@opargbegintheorem| to
%    |\relax| since these commands are no longer necessary at
%    toplevel. This will save a few tokens.
%    \begin{macrocode}
\global\let\@begintheorem\relax
\global\let\@opargbegintheorem\relax
%    \end{macrocode}
% \end{macro}
% \end{macro}
%
% \begin{macro}{\theorem@style}
%    Obviously the \textsf{token} register used above has to be
%    allocated. To assure the utmost compatibility with the
%    original \LaTeX{} definition, we set the default theorem style to
%    \texttt{plain}, which implements the usual \LaTeX{} convention.
% \changes{v2.0f}{89/07/20}{Unnecessary `global' removed.}
%    \begin{macrocode}
\newtoks\theorem@style
\global\theorem@style{plain}
%    \end{macrocode}
% \end{macro}
%
% \begin{macro}{\theorembodyfont}
% \begin{macro}{\theorem@bodyfont}
%    If the |\theorembodyfont| is set by the user then it should not
%    interact with the default font set in the theorem style. When the
%    new font selection is in force this may happen if, for example,
%    the default is |\itshape| and the new |\theorembodyfont| is
%    |\sffamily|. So we add a |\reset@font| command in front of the
%    user definition.
% \changes{v2.0f}{89/07/20}{Unnecessary `global' removed.}
% \changes{v2.0g}{89/09/19}{`theoremfont to `theorembodyfont.}
% \changes{v2.1c}{92/06/22}{`reset@font added and logic changed.}
% \changes{v2.1d}{92/10/28}{Reset to empty if arg is empty}
%    \begin{macrocode}
\gdef\theorembodyfont#1{%
%    \end{macrocode}
%    We check if the argument supplied is empty and if so put nothing
%    into the |\theorem@bodyfont| token register to allow for
%    |\theorembodyfont{}| as a mean of using the default of the
%    current |\theoremstyle|.
%    \begin{macrocode}
   \def\@tempa{#1}%
   \ifx\@tempa\@empty
    \theorem@bodyfont{}%
   \else
    \theorem@bodyfont{\reset@font#1}%
   \fi
   }
\newtoks\theorem@bodyfont
\global\theorem@bodyfont{}
%    \end{macrocode}
% \end{macro}
% \end{macro}
%
%
% \begin{macro}{\theoremheaderfont}
% \changes{v2.0g}{89/09/19}{Macro added.}
%    The font for the theorem headers is handled differently because
%    this definition applies to all theorem styles.
%    \begin{macrocode}
\gdef\theoremheaderfont#1{\gdef\theorem@headerfont{#1}%
%    \end{macrocode}
%    After using the macro once it is redefined to produce an error
%    message.
%    \begin{macrocode}
       \gdef\theoremheaderfont##1{%
        \typeout{\string\theoremheaderfont\space should be used
                 only once.}}}
%    \end{macrocode}
% \end{macro}
%
% \begin{macro}{\theorem@headerfont}
% \changes{v2.0g}{89/09/19}{Macro added.}
%    To set the |\theorem@headerfont| default we first test if
%    the new fontselection scheme is in force.
% \changes{v2.1a}{90/02/15}{Support for new font selection.}
%    \begin{macrocode}
\ifx\upshape\undefined
%    \end{macrocode}
%    If not we define it to expand into |\bfseries|. We don't use
%    |\let| just in case a following style option redefines this
%    macro.
%    \begin{macrocode}
\gdef\theorem@headerfont{\bfseries}
%    \end{macrocode}
%    Otherwise we reset the current shape before calling |\bfseries|.
%    \begin{macrocode}
\else \gdef\theorem@headerfont{\normalfont\bfseries}\fi
%    \end{macrocode}
% \end{macro}
%
% \begin{macro}{\th@plain}
% \begin{macro}{\th@break}
% \begin{macro}{\th@marginbreak}
% \begin{macro}{\th@changebreak}
% \begin{macro}{\th@change}
% \begin{macro}{\th@margin}
%    The different styles are defined in macros such as
%    |\th@plain|.  Since memory space is precious in
%    ``non-Big-versions'', we have to avoid offering too many unused
%    definitions.  Therefore we define these styles in separate files
%    that can be loaded on demand.  Thus the commands themselves only
%    load these files. We use |\@input@|, a \LaTeXe{} internal command
%    that ensures that the file will be listed with |\listfiles|.
% \changes{v2.2a}{94/02/02}{Use `@input@ to load theorem layout files}
%    \begin{macrocode}
\gdef\th@plain{\@input@{thp.sty}}
\gdef\th@break{\@input@{thb.sty}}
\gdef\th@marginbreak{\@input@{thmb.sty}}
\gdef\th@changebreak{\@input@{thcb.sty}}
\gdef\th@change{\@input@{thc.sty}}
\gdef\th@margin{\@input@{thm.sty}}
%    \end{macrocode}
%    This list will be expanded when new styles become
%    available.  For testing,  just append new theorem substyles
%    as document options.
% \end{macro}
% \end{macro}
% \end{macro}
% \end{macro}
% \end{macro}
% \end{macro}
%
%
%
% \subsection{Definition of a new theorem set}
%
% As already pointed out, a new theorem environment can be defined in
% three different ways:
% \begin{verbatim}
%    \newtheorem{Lem}{Lemma}
%    \newtheorem{Lem}{Lemma}[section]
%    \newtheorem{Lem}[Theorem]{Lemma}
%\end{verbatim}
% The function of the macro |\newtheorem| is to recognize these cases
% and then to branch into one of the three macros |\@ynthm|, |\@xnthm|
% or |\@othm|.  This mechanism is adopted unchanged from
% \cite{latex:source}; the essential point here is that, for example,
% in the second case, the arguments \texttt{Lem}, \texttt{Lemma} and
% \texttt{section} are passed over to the macro |\@xnthm|.
%
% We inspect this case first because the others present fewer problems,
% and thus are easily derived from this one.
%
% \begin{macro}{\@xnthm}
%    For our example arguments, the macro |\@xnthm| must fulfill
%    the following:
%    \begin{itemize}
%       \item
%          Define a new \LaTeX-counter `Lem'
%       \item
%          reset this counter within a |\section|
%       \item
%          define the macro |\theLem|
%       \item
%          define the environment macros |\Lem| and
%          |\endLem| using the current
%          |\theoremstyle| and |\theorem@bodyfont|.
%    \end{itemize}
%    Obviously, all this should happen only if the first argument
%    of |\@xnthm| (i.e.\ \texttt{Lem} in our example) is chosen so
%    as not to conflict with any previously defined commands or
%    environments.  This test is performed by the \LaTeX{} macro
%    |\@ifdefinable|.
%    \begin{macrocode}
\gdef\@xnthm#1#2[#3]{\expandafter\@ifdefinable\csname #1\endcsname
%    \end{macrocode}
%    Therefore, the first argument of |\@ifdefinable| is the
%    expansion (in the example, |\Lem|) of
%    |\csname||#1\endcsname|.  The second argument is
%    executed only if the test has been completed successfully.
%    \begin{macrocode}
   {%
%    \end{macrocode}
%    Now we define the new counter. The names of the \LaTeX{}
%    macros employed should speak for themselves:
% \changes{v2.2b}{1995/10/07}{Use \cs{@newctr}}
% \changes{v2.2c}{1995/11/23}{Use \cs{@newctr} properly.}
%    \begin{macrocode}
    \@definecounter{#1}\@newctr{#1}[#3]%
%    \end{macrocode}
%    Using |\@newctr| will give a proper error message if the counter
%    in |#3| is not defined.
%    In defining `|\theLem|' we must generate the desired macro
%    name by use of |\expandafter| and |\csname|.
%    \begin{macrocode}
    \expandafter\xdef\csname the#1\endcsname
%    \end{macrocode}
%    An |\xdef| is used in order to make the definition global,
%    and to ensure that it contains the replacement texts of
%    |\@thmcountersep| and |\@thmcounter|.\footnote{These
%    two macros can be defined by the document style.  Their default
%    values produce a `\texttt{.}' as separation and an arabic
%    representation of the number.} However, not everything should be
%    expanded. For example, it saves space to use |\thesection|
%    instead of its---at times---lengthy expansion.
%    \begin{macrocode}
      {\expandafter \noexpand \csname the#3\endcsname
       \@thmcountersep \@thmcounter{#1}}%
%    \end{macrocode}
%    Thus with the defaults of \LaTeX{}, |\theLem| would be
%    replaced by the command sequence |\thesection.\arabic{Lem}|.
%
%    We will now look at the definition of the macro which is executed
%    at the beginning of the actual environment (in our example this
%    macro is |\Lem|). It should be noted that we use an
%    ``|\expandafter| trick'' to expand only certain parts of the
%    replacement text at the time of the definition.
%    \begin{macrocode}
    \def\@tempa{\global\@namedef{#1}}%
    \expandafter \@tempa \expandafter{%
%    \end{macrocode}
%    First, the macro that contains the current definitions of
%    |\@begintheorem| and |\@opargtheorem| should be called
%    up.  The name of this macro---as is already known---has the form
%    |\th@|\meta{theorem\ style}; therefore, it must be called by
%    \begin{macrocode}
      \csname th@\the \theorem@style
%    \end{macrocode}
%    In addition the default theorem font should be changeable, i.e.\
%    we have to insert the contents of |\theorem@bodyfont|. For
%    that reason, we expand even further, beyond |\endcsname|,
%    and thus insert the contents of the \textsf{token} register
%    |\theorem@bodyfont| in the replacement text.
%    \begin{macrocode}
            \expandafter \endcsname \the \theorem@bodyfont
%    \end{macrocode}
%    Now it is time to call the macro |\@thm| which takes over
%    the further processing.  It has two arguments: the current
%    counter name (in our example, \texttt{Lem}), and the text of the
%    label (in our example, \texttt{Lemma}).
%    \begin{macrocode}
     \@thm{#1}{#2}}%
%    \end{macrocode}
%    With this, the `sub-definition' is complete.  The macro
%    |\@endtheorem| ends a theorem environment and is, so far, nothing
%    but an |\endtrivlist|.  (Hence it is defined globally, and not
%    within the theorem styles.\footnote{This has to be changed as
%    soon as theorem styles that change \texttt{\bslash @endtheorem}
%    exist.  In such a case, all existing styles must be changed as
%    well since they will have to reset the macro.}) Therefore, we can
%    set it equivalent to the macro that ends the theorem set (in our
%    example, |\endLem|).  However, if some day theorem styles exist
%    that do change |\@endtheorem|, we would have to use the
%    commented-out line instead.
%    \begin{macrocode}
    \global \expandafter \let \csname end#1\endcsname \@endtheorem
%   \global\@namedef{end#1}{\@endtheorem}%
%    \end{macrocode}
%    With these commands all the required definitions are employed,
%    unless the test |\@ifdefinable| has failed.  Therefore, we
%    end the second argument of this macro and with it the definition
%    of |\@xnthm|.
%    \begin{macrocode}
   }}
%    \end{macrocode}
% \end{macro}
%
% \begin{macro}{\@ynthm}
%    The definition of |\@ynthm| is completely analogous. In
%    this case the new counter that is defined is not reset within
%    another counter; thus the definition of |\the...|
%    is simplified:
%    \begin{macrocode}
\gdef\@ynthm#1#2{\expandafter\@ifdefinable\csname #1\endcsname
   {\@definecounter{#1}%
    \expandafter\xdef\csname the#1\endcsname{\@thmcounter{#1}}%
%    \end{macrocode}
%    The rest of the definition corresponds literally to that of
%    |\@xnthm|:
%    \begin{macrocode}
    \def\@tempa{\global\@namedef{#1}}\expandafter \@tempa
     \expandafter{\csname th@\the \theorem@style \expandafter
     \endcsname \the\theorem@bodyfont \@thm{#1}{#2}}%
    \global \expandafter \let \csname end#1\endcsname \@endtheorem}}
%    \end{macrocode}
% \end{macro}
%
% \begin{macro}{\@othm}
%    The definition of |\@othm| does not contain anything new.
%    \begin{macrocode}
\gdef\@othm#1[#2]#3{%
%    \end{macrocode}
% \changes{v2.2b}{95/10/07}{Build in check for counter defined}
%    We do not define a new counter but instead use one that has
%    already been defined.  Thus the only definition we need is that
%    of this pseudo-counter (i.e.\ |\the|\meta{env.\ name}).
%    First we check if |#2| corresponds to a known counter name.
%    \begin{macrocode}
  \expandafter\ifx\csname c@#2\endcsname\relax
   \@nocounterr{#2}%
  \else
   \expandafter\@ifdefinable\csname #1\endcsname
   {\expandafter \xdef \csname the#1\endcsname
     {\expandafter \noexpand \csname the#2\endcsname}%
%    \end{macrocode}
%    All other parts of the definition can be adopted from
%    |\@xnthm|. We have to remember, though, that in this case
%    the name of the current counter and the theorem label have moved
%    to the second and third arguments.
%    \begin{macrocode}
    \def\@tempa{\global\@namedef{#1}}\expandafter \@tempa
     \expandafter{\csname th@\the \theorem@style \expandafter
     \endcsname \the\theorem@bodyfont \@thm{#2}{#3}}%
    \global \expandafter \let \csname end#1\endcsname \@endtheorem}%
  \fi}
%    \end{macrocode}
% \end{macro}
%
% \subsection{Macros that are employed in a {\envfont theorem}
%    environment}
%
% \begin{macro}{\@thm}
%    The macro |\@thm| has to increase the current counter.  Then,
%    depending on whether the environment has (or does not have) an
%    optional argument, it has to branch into either |\@begintheorem|
%    or |\@opargtheorem|.
%    \begin{macrocode}
\gdef\@thm#1#2{\refstepcounter{#1}%
%    \end{macrocode}
%    Now we start a {\envfont trivlist} environment, and give
%    |\@topsep| and |\@topsepadd| the values of the \textsf{skip}
%    registers |\theorempreskipamount| and |\theorempostskipamount|.
%    The value in |\@topsep| is the vertical space that is inserted by
%    the first (and only) |\item| in our |\trivlist| whilst
%    |\@topsepadd| is inserted by |\@endparenv| at the end of that
%    {\envfont trivlist} environment.  By using these registers, we
%    obtain the desired space around a {\envfont theorem} environment.
%    \begin{macrocode}
   \trivlist
   \@topsep \theorempreskipamount               % used by first \item
   \@topsepadd \theorempostskipamount           % used by \@endparenv
%    \end{macrocode}
%    Now we have to test whether an optional argument has been given.
% \changes{v2.0d}{89/07/18}{Forgotten percent added.}
%    \begin{macrocode}
   \@ifnextchar [%
%    \end{macrocode}
%    If there is an optional argument, we will call |\@ythm|, and
%    move the arguments read back into the input stream.
%    \begin{macrocode}
   {\@ythm{#1}{#2}}%
%    \end{macrocode}
%    If not, we call |\@begintheorem|. Its first argument is the
%    name of the theorem set (hence the second argument of
%    |\@thm|). Its second argument is the macro that produces the
%    current number.
%    \begin{macrocode}
   {\@begintheorem{#2}{\csname the#1\endcsname}\ignorespaces}}
%    \end{macrocode}
% \end{macro}
%
% \begin{macro}{\@xthm}
% \begin{macro}{\@ythm}
%    Both these macros were originally called by |\@thm|.  We do
%    not need |\@xthm| anymore, hence we reset it to
%    |\relax|. The definition of |\@ythm| has not changed at
%    all from its definition in \LaTeX. In order to make the macros
%    easier to understand, we will
%    nevertheless present it (commented out).
%    \begin{macrocode}
\global\let\@xthm\relax
% \def\@ythm#1#2[#3]{\@opargbegintheorem{#2}{\csname
%           the#1\endcsname}{#3}\ignorespaces}
%</package>
%    \end{macrocode}
%    The primitive |\ignorespaces| in |\@ythm| and
%    |\@thm| is needed to remove the spaces between the
%    |\begin{...}| and the actual text.
% \end{macro}
% \end{macro}
%
%
% \subsection{Definition of the theorem substyles}
%
% As already pointed out, the theorem substyles, defined below, are
% only loaded when necessary.  Note that all these substyles, except
% \texttt{plain}, have |\slshape| as the default body font.
%
% \subsubsection{The \texttt{plain} style}
%
% \label{dok:th@plain} As the following macros use |@|, we have
% to locally set the |\catcode| of this symbol to ``letter''.
% This happens within a group, so that we do not have to worry about
% which |\catcode| that symbol had before.
%    \begin{macrocode}
%<*thp>
\begingroup \makeatletter
%    \end{macrocode}
% Since we are now within a group, we must make all definitions
% globally.  First we make sure that \texttt{theorem.sty} is loaded.
% This will allow us to use this file as a document style option
% without having to call \texttt{theorem} itself as an option.  At the
% same time, we assure that at least version 2 is loaded, since
% |\theorem@style| was not defined in earlier versions.
%    \begin{macrocode}
\@ifundefined{theorem@style}{\input{theorem.sty}}{}
\ProvidesFile{thp.sty}
             [\FMithmInfo]
%    \end{macrocode}
%
%
% \begin{macro}{\th@plain}
% \begin{macro}{\@begintheorem}
% \begin{macro}{\@opargbegintheorem}
%    |\theoremstyle{plain}| corresponds to the original
%    definition, except that the distances to the surrounding text are
%    determined by the parameters |\theorempreskipamount| and
%    |\theorempostskipamount|.
%    First we set the default body font.
%    \begin{macrocode}
\gdef\th@plain{\normalfont\itshape
%    \end{macrocode}
%    Then we define |\@begintheorem| and
%    |\@opargbegintheorem|. These two macros define how the
%    header of a theorem is typeset.  |\@opargbegintheorem| will
%    be called if a \textsf{theorem} environment with an optional
%    argument is encountered; otherwise, the header is constructed by
%    calling |\@begintheorem|.  If one of these macros is
%    executed, we are within a \textsf{trivlist} environment started by
%    |\@thm|.  So the theorem header is produced with an
%    |\item| command.
%
%    Instead of specifying the header font directly, all standard
%    theorem styles use the |\theorem@headerfont| macro to allow
%    customization.  The extra space (|\labelsep|) is necessary
%    because of problems in the \textsf{trivlist} environment.
%    \begin{macrocode}
  \def\@begintheorem##1##2{%
        \item[\hskip\labelsep \theorem@headerfont ##1\ ##2]}%
%    \end{macrocode}
%    The definition of |\@opargbegintheorem| is completely
%    analogous. The only difference is the fact that there exists a
%    third argument (which is the optional parameter of the
%    environment and contains additional information about the
%    theorem).  Customarily we enclose it in parentheses.
%    \begin{macrocode}
\def\@opargbegintheorem##1##2##3{%
   \item[\hskip\labelsep \theorem@headerfont ##1\ ##2\ (##3)]}}
%    \end{macrocode}
% \end{macro}
% \end{macro}
% \end{macro}
%
% \noindent
% We conclude with an |\endgroup| to restore the
% |\catcode| of |@|.
%    \begin{macrocode}
\endgroup
%</thp>
%    \end{macrocode}
%
%
% \subsubsection{The \texttt{break} style}
%
% This style option is stored in the file |thb.sty|.
% For the next two lines see the documentation for |\th@plain| on
% page~\pageref{dok:th@plain}.
%    \begin{macrocode}
%<*thb>
\begingroup \makeatletter
\@ifundefined{theorem@style}{\input{theorem.sty}}{}
\ProvidesFile{thb.sty}
             [\FMithmInfo]
%    \end{macrocode}
%
% \begin{macro}{\th@break}
%    |\theoremstyle{break}| produces a line break after the name
%    of the theorem.  The font is |\slshape|.  Hence, we define
%    |\th@break| as follows:
%    \begin{macrocode}
\gdef\th@break{\normalfont\slshape
  \def\@begintheorem##1##2{\item[%
%    \end{macrocode}
%    We run into the following problem: it is not possible to create
%    the header with |\item[|\meta{title}|]| and then start
%    a new line by, for example, |\mbox{}\\|. Such a definition
%    will fail whenever a list environment follows immediately.  With
%    the above construction, the |\mbox{}| causes the switch
%    |@inlabel| (cf.\ definition of |\list| and
%    |\trivlist| in \cite{latex:source}) to be set to \texttt{false}
%    and so the following list will insert additional vertical
%    space (|\topskip|). This is quite annoying. Therefore, we
%    create the line break within the |\item|. In order to ensure
%    that the text will begin at the proper position in the following
%    line, we simply pretend that the label does not take any
%    room.\footnote{This will lead to problems whenever very high
%    symbols occurring in the line tower into the
%    heading. So, something else has to be done here sometime.}
%    \begin{macrocode}
       \rlap{\vbox{\hbox{\hskip \labelsep\theorem@headerfont ##1\ ##2}%
                   \hbox{\strut}}}]}%
%    \end{macrocode}
%    Again, the definition of |\@opargbegintheorem| is
%    completely analogous.
%    \begin{macrocode}
\def\@opargbegintheorem##1##2##3{%
  \item[\rlap{\vbox{\hbox{\hskip \labelsep \theorem@headerfont
                     ##1\ ##2\ (##3)}%
                    \hbox{\strut}}}]}}
\endgroup
%</thb>
%    \end{macrocode}
% \end{macro}
%
%
% \subsubsection{The \texttt{changebreak} style}
%
%    \begin{macrocode}
% This style option is stored in the file |thcb.sty|.
%    \begin{macrocode}
%<*thcb>
\begingroup \makeatletter
\@ifundefined{theorem@style}{\input{theorem.sty}}{}
\ProvidesFile{thcb.sty}
             [\FMithmInfo]
%    \end{macrocode}
%
% \begin{macro}{\th@changebreak}
%    The \texttt{change-break} theorem style is like \texttt{break}
%    but with interchange of theorem name and theorem number.  Thus we
%    define |\th@changebreak| as follows:
%    \begin{macrocode}
\gdef\th@changebreak{\normalfont\slshape
  \def\@begintheorem##1##2{\item
       [\rlap{\vbox{\hbox{\hskip\labelsep \theorem@headerfont ##2\ ##1}%
                    \hbox{\strut}}}]}%
\def\@opargbegintheorem##1##2##3{%
  \item[\rlap{\vbox{\hbox{\hskip\labelsep \theorem@headerfont
                     ##2\ ##1\ (##3)}%
                    \hbox{\strut}}}]}}
\endgroup
%</thcb>
%    \end{macrocode}
% \end{macro}
%
%
% \subsubsection{The \texttt{change} style}
%
% This style option is stored in the file |thc.sty|.
%    \begin{macrocode}
%<*thc>
\begingroup \makeatletter
\@ifundefined{theorem@style}{\input{theorem.sty}}{}
\ProvidesFile{thc.sty}
             [\FMithmInfo]
%    \end{macrocode}
%
% \begin{macro}{\th@change}
%    The \texttt{change} theorem style corresponds to the
%    \texttt{change break} style without a linebreak after the header.
%    To say it in another way, it's the same as the \texttt{plain}
%    style but with number and name interchanged and |\slshape| as the
%    default font.
%    \begin{macrocode}
\gdef\th@change{\normalfont\slshape
  \def\@begintheorem##1##2{\item
       [\hskip\labelsep \theorem@headerfont ##2\ ##1]}%
\def\@opargbegintheorem##1##2##3{%
  \item[\hskip\labelsep \theorem@headerfont ##2\ ##1\ (##3)]}}
\endgroup
%</thc>
%    \end{macrocode}
% \end{macro}
%
%
% \subsubsection{The \texttt{marginbreak} style}
%
% This style option is the one used most often at Mainz.  It is saved
% in the file |thmb.sty|.
%    \begin{macrocode}
%<*thmb>
\begingroup \makeatletter
\@ifundefined{theorem@style}{\input{theorem.sty}}{}
\ProvidesFile{thmb.sty}
             [\FMithmInfo]
%    \end{macrocode}
%
% \begin{macro}{\th@marginbreak}
%    The \texttt{margin break} style is nearly the same as the
%    \texttt{change break} style. The only difference is the placement
%    of the theorem number.  We use |\llap| to place it in the left
%    margin.
%
%    In this style |\labelsep| denotes the separation between the
%    number and the text.
%    \begin{macrocode}
\gdef\th@marginbreak{\normalfont\slshape
  \def\@begintheorem##1##2{\item
       [\rlap{\vbox{\theorem@headerfont
                    \hbox{\llap{##2}\hskip\labelsep ##1}%
                    \hbox{\strut}}}]}%
\def\@opargbegintheorem##1##2##3{%
  \item[\rlap{\vbox{\theorem@headerfont
                    \hbox{\llap{##2}\hskip\labelsep ##1\ (##3)}%
                    \hbox{\strut}}}]}}
\endgroup
%</thmb>
%    \end{macrocode}
% \changes{v2.1c}{90/02/17}{`endgroup outside macrocode env.}
% \end{macro}
%
%
%
% \subsubsection{The \texttt{margin} style}
%
% This style option is stored in the file |thm.sty|.
%    \begin{macrocode}
%<*thm>
\begingroup \makeatletter
\@ifundefined{theorem@style}{\input{theorem.sty}}{}
\ProvidesFile{thm.sty}
             [\FMithmInfo]
%    \end{macrocode}
%
% \begin{macro}{\th@margin}
%    Again this is only a variant of the theorem styles described
%    above without any new ideas.
%    \begin{macrocode}
\gdef\th@margin{\normalfont\slshape
  \def\@begintheorem##1##2{\item
       [\theorem@headerfont \llap{##2}\hskip\labelsep ##1]}%
\def\@opargbegintheorem##1##2##3{%
  \item[\theorem@headerfont \llap{##2}\hskip\labelsep ##1\ (##3)]}}
\endgroup
%</thm>
%    \end{macrocode}
% \end{macro}
%
%
%
%
% \subsection{Final Definitions}
%
% \begin{macro}{\theorempreskipamount}
% \begin{macro}{\theorempostskipamount}
%    The \textsf{skip} parameters that regulate the vertical empty space
%    before and after the {\envfont theorem} environment have
%    to be allocated as well.
% \changes{v2.0f}{89/07/20}{Unnecessary `global' removed.}
%    \begin{macrocode}
%<*package>
\newskip\theorempreskipamount
\newskip\theorempostskipamount
%    \end{macrocode}
%    Since we have used the same values for all theorem sets, we now can
%    assign them.
%    \begin{macrocode}
\global\setlength\theorempreskipamount{12pt plus 5pt minus 3pt}
\global\setlength\theorempostskipamount{8pt plus 3pt minus 1.5pt}
%    \end{macrocode}
% \end{macro}
% \end{macro}
%
%
% \begin{macro}{\@endtheorem}
%    The same holds for the macro |\@endtheorem|, which ends a
%    {\envfont theorem} environment. Since it is the same for all
%    theorem sets, it is removed from the macros |\th@|\meta{style}.
%    It simply ends the {\envfont trivlist} environment, which was
%    begun in |\@thm|.
%    \begin{macrocode}
\global\let\@endtheorem=\endtrivlist
%    \end{macrocode}
% \end{macro}
%
%
%
% \begin{macro}{\@preamblecmds}
%    All macros defined above are to be used only in the preamble.
%    Therefore, we insert them in |\@preamblecmds| which will disable
%    them at begin document. This is done by the internal \LaTeXe{}
%    command |\@onlypreamble|.
% \changes{v2.0g}{89/09/19}{`theoremheaderfont and `theorembodyfont
%    added.}
% \changes{v2.2a}{94/02/03}{Use \LaTeXe{} interface}
%    \begin{macrocode}
\@onlypreamble\@xnthm
\@onlypreamble\@ynthm
\@onlypreamble\@othm
\@onlypreamble\newtheorem
\@onlypreamble\theoremstyle
\@onlypreamble\theorembodyfont
\@onlypreamble\theoremheaderfont
%    \end{macrocode}
% \end{macro}
%
% Finally we declare the \texttt{plain} theorem style to be the default.
% \changes{v2.1b}{90/06/04}{Fixed missing `item by declaring default
%    style.}
%    \begin{macrocode}
\theoremstyle{plain}
%</package>
%    \end{macrocode}
%
% \Finale

\endinput
}{}
\ProvidesFile{thcb.sty}
             [\FMithmInfo]
%    \end{macrocode}
%
% \begin{macro}{\th@changebreak}
%    The \texttt{change-break} theorem style is like \texttt{break}
%    but with interchange of theorem name and theorem number.  Thus we
%    define |\th@changebreak| as follows:
%    \begin{macrocode}
\gdef\th@changebreak{\normalfont\slshape
  \def\@begintheorem##1##2{\item
       [\rlap{\vbox{\hbox{\hskip\labelsep \theorem@headerfont ##2\ ##1}%
                    \hbox{\strut}}}]}%
\def\@opargbegintheorem##1##2##3{%
  \item[\rlap{\vbox{\hbox{\hskip\labelsep \theorem@headerfont
                     ##2\ ##1\ (##3)}%
                    \hbox{\strut}}}]}}
\endgroup
%</thcb>
%    \end{macrocode}
% \end{macro}
%
%
% \subsubsection{The \texttt{change} style}
%
% This style option is stored in the file |thc.sty|.
%    \begin{macrocode}
%<*thc>
\begingroup \makeatletter
\@ifundefined{theorem@style}{% \iffalse meta-comment
%
% Copyright (C) 1993-2024
%
% The LaTeX Project and any individual authors listed elsewhere
% in this file.
%
% This file is part of the Standard LaTeX `Tools Bundle'.
% -------------------------------------------------------
%
% It may be distributed and/or modified under the
% conditions of the LaTeX Project Public License, either version 1.3c
% of this license or (at your option) any later version.
% The latest version of this license is in
%    https://www.latex-project.org/lppl.txt
% and version 1.3c or later is part of all distributions of LaTeX
% version 2005/12/01 or later.
%
% The list of all files belonging to the LaTeX `Tools Bundle' is
% given in the file `manifest.txt'.
%
% \fi
%
%%
%
%
\def\FMithmInfo{2023/07/05 v2.2c Theorem extension package (FMi)}
%
% \ProvidesFile{theorem.dtx}[\FMithmInfo]
%\iffalse   % this is a METACOMMENT !
%%
%% Package `theorem' to use with LaTeX2e
%% Copyright (C) 1989-1995 Frank Mittelbach, all rights reserved.
%<+package>\@ifundefined{theorem@style}{}{\endinput}
%<+package>\NeedsTeXFormat{LaTeX2e}[1995/06/01]
%<+package>\ProvidesPackage{theorem}
%<+package>         [\FMithmInfo]
%
% \fi
%
% \changes{v1.0}{87/05/07}{First draft.}
% \changes{v1.0a}{87/05/07}{`thistheoremstyle' added.}
% \changes{v1.1a}{87/07/29}{New concept: redefine the internal
%                          \protect\LaTeX{} commands `@xnthm' etc.}
% \changes{v1.1b}{87/07/31}{`theoremfont' used to specify font.}
% \changes{v1.2a}{88/01/08}{`@endtheorem' now globally defined.}
% \changes{v1.2b}{88/01/11}{`newtheorem' only allowed in the preamble.}
% \changes{v1.3a}{88/02/14}{`theoremfont' now a toks register.}
%
% \changes{v2.0a}{88/11/16}{New implementation and new documentation.}
% \changes{v2.0a}{88/11/16}{No layout changes but internal modifications
%                           old substyles won't work.}
% \changes{v2.0b}{89/01/30}{`global' used for all definitions.}
% \changes{v2.0c}{89/06/15}{Documentation translated to English.
%                            Thanks to LLL and Nina}
% \changes{v2.1a}{90/02/15}{Support for new font selection.}
% \changes{v2.2a}{94/02/03}{Cleanup for LaTeX2e}
%
% \DoNotIndex{\@endparenv,\@ifdefinable,\@ifnextchar,\@ifundefined}
% \DoNotIndex{\@namedef,\@nameuse,\@spaces,\@tempa,\@warning}
% \DoNotIndex{\ }
% \DoNotIndex{\begingroup,\bfseries}
% \DoNotIndex{\csname}
% \DoNotIndex{\def,\do,\docdate}
% \DoNotIndex{\endcsname,\endgroup,\endinput,\endtrivlist,\expandafter}
% \DoNotIndex{\filedate,\fileversion}
% \DoNotIndex{\gdef,\global}
% \DoNotIndex{\hbox,\hskip}
% \DoNotIndex{\ignorespaces,\input,\itshape,\item}
% \DoNotIndex{\labelsep,\let,\llap}
% \DoNotIndex{\makeatletter}
% \DoNotIndex{\noexpand}
% \DoNotIndex{\refstepcounter,\relax,\rlap}
% \DoNotIndex{\setlength,\slshape,\space,\string,\strut}
% \DoNotIndex{\the,\trivlist,\typeout}
% \DoNotIndex{\vbox}
% \DoNotIndex{\xdef}
%
% \def\deflabel#1{\ttfamily #1\hfill}
% \def\deflist#1{\begin{list}{}{\settowidth\labelwidth{\ttfamily #1}%
%                               \setlength\leftmargin\labelwidth
%                               \addtolength\leftmargin\labelsep
%                               \let\makelabel\deflabel}}
% \def\enddeflist{\end{list}}
% \def\envfont{\normalfont\ttfamily}
%
% \MakeShortVerb{\|}
%
% \newcommand{\TUB}{{\sl TUGboat\/}}
%
% \GetFileInfo{theorem.dtx}
% \title{An Extension of the
%        \LaTeX{} theorem environment\thanks{This file
%        has version number \fileversion, last
%        revised \filedate.}}
% \author{Frank Mittelbach}
%
% \MaintainedByLaTeXTeam{tools}
% \maketitle
%
% \begin{abstract}
%    The macros described in this paper yield an extension of the
%    \LaTeX{} theorem mechanism. It is designed to satisfy the
%    different requirements of various journals. Thus, the
%    layout of the ``theorems'' can be manipulated by determining a
%    ``style''. This article describes not only the use, but
%    also the definition, of the necessary macros.
% \end{abstract}
%
% \section*{Preface to version 2.2}
%
% For \LaTeXe{} this package did not need any fundamental changes. I
% only modified the messages generated so that theorem layout styles
% will show up with the |\listfiles| command and cleaned the section
% on the New Font Selection Scheme since this is now included in
% \LaTeX.
%
% \section*{Preface to version 2.1}
%
% This version is identical to 2.0g described in \TUB\ 10\#3 except
% for some internal defaults which are now set depending on the used
% font selection scheme.
%
% This was done to avoid unpleasant surprises if the new font selection
% scheme is in force. For further details see
% section~\ref{sec:newfonts} and \cite{bk:GMS94}.
%
% \section{Introduction}
%
% For our purposes here, ``theorems'' are labelled enunciations,
% often set off from the main text by extra space and a font change.
% Theorems, corollaries, conjectures, definitions, and remarks
% are all instances of ``theorems''.  The ``header'' of these structures
% is composed of a label (such as \textsc{Theorem} or \textsc{Remark})
% and a number which serializes an item in the sequence of items
% with the same label.
%
% Shortly after the introduction of \LaTeX{} at the Fachbereich
% Mathematik in Mainz, the desire to manipulate the layout of
% ``theorems'' arose.  In Mainz, the following two conventions came
% into general use:
% \begin{enumerate}
%   \item The number of the theorem is shown in the margin.
%   \item There is a line break at
%         the end of the theorem header.
% \end{enumerate}
%
% Additionally, some journals require different formats which depend on
% the ``sort of theorem'': e.g.\ often remarks and definitions are set
% in |\upshape|, while |\itshape| is employed for main theorems.
%
% Confronted with these requirements, a theorem environment was
% developed in Mainz which allows separate determination of the
% layout of the ``theorems sets'', comparable to |\pagestyle|.
%
%
% \section{The user interface}
% \subsection{Defining new theorem sets}
%
% \DescribeMacro\newtheorem
% As in the original \LaTeX{} version, the command |\newtheorem|
% defines a new ``theorem set'' or ``theorem-like structure''.
% Two required arguments name the new environment and give the text
% to be typeset with each instance of the new ``set'', while
% an optional argument determines how
% the ``set'' is enumerated:
% \begin{description}
%    \item[\ttfamily \bslash newtheorem\{foo\}\{bar\}]
%       The theorem set {\envfont foo} (whose name is \texttt{bar})
%       uses its own counter.
%    \item[\ttfamily \bslash newtheorem\{foo2\}{[foo]}\{bar2\}]
%       The theorem set {\envfont foo2} (printed name \texttt{bar2})
%       uses the same counter as the theorem set \texttt{foo}.
%    \item[\ttfamily \bslash newtheorem\{foo3\}\{bar3\}{[section]}]
%       The theorem set {\envfont foo3} (printed name \texttt{bar3}) is
%       enumerated within the counter \texttt{section}, i.e.\ with every
%       new |\section| the enumeration begins again with 1, and
%       the enumeration is composed from the section-number and the
%       theorem counter itself.
% \end{description}
%
% \DescribeMacro\theoremstyle
% Additionally, the command |\theoremstyle| can define the layout
% of various, or all, theorem sets.  It should be noted that any
% theorem set defined by |\newtheorem| is typeset in the
% |\theoremstyle| that is current at the time of the definition.
% Thus, the following
% \begin{verbatim}
%    \theoremstyle{break}        \newtheorem{Cor}{Corollary}
%    \theoremstyle{plain}        \newtheorem{Exa}{Example}[section]
%\end{verbatim}                             ^^A No blank at line start
% leads to the result that the set {\envfont Cor} is formatted in the
% style \texttt{break}, while the set {\envfont Exa} and all the
% following ones are formatted in the style \texttt{plain}, unless
% another |\theoremstyle| follows.  Since the definitions installed by
% |\newtheorem| are global, one also can limit |\theoremstyle| locally
% by grouping braces.
%
% \DescribeMacro\theorembodyfont
% The choice of the font for the theorem body is completely
% independent of the chosen |\theoremstyle|; this has proven to
% be very advantageous.  For example,
% \begin{verbatim}
%   {\theorembodyfont{\upshape}          \newtheorem{Rem}{Remark}}
%\end{verbatim}
% defines a theorem set {\envfont Rem}, which will be set in
% |\upshape| in the current layout (which in our example is
% \texttt{plain}).  As with |\theoremstyle|, the |\theorembodyfont|
% chosen is that current at the time of |\newtheorem|.  If
% |\theorembodyfont| is not specified or one defines
% |\theorembodyfont{}|, then the font used will be that defined by the
% |\theoremstyle|.
%
% \DescribeMacro\theoremheaderfont It is also possible to customize
% the font used for the theorem headers.  This is, however, a global
% declaration, and therefore there should be at most one
% |\theoremheaderfont| declaration in the preamble.\footnote{If it is
% actually necessary to have different header fonts, one has to define
% new theorem styles (substituting the desired font) or specify the
% information directly in the \texttt{\bslash newtheorem} declaration
% (the unclean variant).}
%
% \DescribeMacro\theorempreskipamount
% \DescribeMacro\theorempostskipamount Two additional parameters
% affect the vertical space around the theorem environments:
% |\theorempreskipamount| and |\theorempostskipamount| define,
% respectively, the spacing before and after such an environment.
% These parameters apply for all theorem sets and can be manipulated
% with the ordinary length macros.  They are rubber lengths
% (`\textsf{skips}'), and therefore can contain \texttt{plus} and
% \texttt{minus} parts.
%
% Since the definition of theorem sets should---most sensibly---be
% placed in the preamble, we only allow installation there.
% It is therefore possible to release the memory used here after
% |\begin{document}|, in order to make room for other
% applications.
%
%
% \subsection{Existing theorem styles}
%
% The following theorem styles exist to date:
% \begin{deflist}{marginbreak}
%    \item[plain]
%       This theorem style emulates the original \LaTeX{} definition,
%       except that additionally the parameters
%       |\theorem...skipamount| are used.
%    \item[break]
%       In this style, the theorem header is followed by a line
%       break.
%    \item[marginbreak]
%       The theorem number is set in the margin, and there is a line
%       break as in \texttt{break}.
%    \item[changebreak]
%       Like \texttt{break}, but with header number and text
%       interchanged.
%    \item[change]
%       Header number and text are interchanged, without a line break.
%    \item[margin]
%       The number is set in the left margin, without a line break.
% \end{deflist}
% All styles (except \texttt{plain}) select |\slshape| as the default
% |\theorembodyfont|.
%
%
% \subsection{Examples}
%
% Given the above theorem sets {\envfont Cor}, {\envfont Exa}
% and {\envfont Rem},
% suppose that the preamble also contains the declarations:
% \begin{verbatim}
%    \theoremstyle{marginbreak}   \newtheorem{Lem}[Cor]{Lemma}
%    \theoremstyle{change}
%    \theorembodyfont{\itshape}        \newtheorem{Def}[Cor]{Definition}
%
%    \theoremheaderfont{\scshape}
%\end{verbatim}
% Then the following are some typical examples of the typeset output
% resulting from their use.
%
% \begin{Cor}
%    This is a sentence typeset in the theorem environment {\envfont
%    Cor}.
% \end{Cor}
%
% \begin{Exa}
%    This is a sentence typeset in the theorem environment {\envfont
%    Exa}.
% \end{Exa}
%
% \begin{Rem}
%    This is a sentence typeset in the theorem environment {\envfont
%    Rem}.
% \end{Rem}
%
% \begin{Lem}[Ben User]
%    This is a sentence typeset in the theorem environment {\envfont
%    Lem}.
% \end{Lem}
%
% \begin{Def}[Very impressive Definition]
%    This is a sentence typeset in the theorem environment {\envfont
%    Def}.
% \end{Def}
%
% The last two examples show the effect of the optional argument to
% a theorem environment (it is the text typeset in parentheses).
%
%
% \section{Special Considerations}
% \label{sec:newfonts}
%
% Theoremheader and body are implemented as a unit. This means that
% the |\theoremheaderfont|  will inherit characteristics of the
% |\theorembodyfont| in \LaTeXe. Thus, if for example
% |\theorembodyfont| is |\itshape| and |\theoremheaderfont|
% is |\bfseries| the font selected for the header will have the
% characteristics `bold extended italic'. If this is not desired one
% should set the |\theoremheaderfont| to something like
% \begin{verbatim}
%      \theoremheaderfont{\normalfont\bfseries}
%\end{verbatim}
% i.e.\ supplying all necessary font information explicitly.
%
%
%
% \section{Acknowledgements}
%
% The publication of this set of macros was only possible with the
% help of Christina Busse (translating the manuscript into English),
% Joachim Pense (playing the r\^ole of typist), Chris Rowley
% (looking everything over) and many others providing useful
% suggestions.
%
% \MaybeStop{%
%    \begin{thebibliography}{9}
%       \bibitem{bk:GMS94} \textsc{M.~Goossens}, \textsc{F.~Mittelbach}
%         and \textsc{A.~Samarin}.
%         \newblock The \LaTeX{} Companion.
%         \newblock
%         Addison-Wesley, Reading, Massachusetts, 1994.
%       \bibitem{latex:source} \textsc{Lamport, Leslie.} \newblock
%                               \texttt{latex.tex}, version 2.09, date
%                               Feb. 1990.
%    \end{thebibliography}
%    \PrintIndex
%    \PrintChanges}
%
%
% \section{The documentation driver file}
%
% The next bit of code contains the documentation driver file for
% \TeX{}, i.e., the file that will produce the documentation you are
% currently reading. It will be extracted from this file by the
% \texttt{docstrip} program. Since it is the first code in the file
% one can alternatively process this file directly with \LaTeXe{} to
% obtain the documentation.
%
%    \begin{macrocode}
%<*driver>
\documentclass{ltxdoc}

\usepackage{theorem}

%<+driver>% The next few lines define theorem sets which are used
%<+driver>% in the example section of the documentation.

\theoremstyle{break}       \newtheorem{Cor}{Corollary}
\theoremstyle{plain}       \newtheorem{Exa}{Example}[section]
{\theorembodyfont{\upshape}\newtheorem{Rem}{Remark}}
\theoremstyle{marginbreak} \newtheorem{Lem}[Cor]{Lemma}
\theoremstyle{change}
\theorembodyfont{\itshape} \newtheorem{Def}[Cor]{Definition}

\theoremheaderfont{\scshape}

\RecordChanges

\begin{document}
   \DocInput{theorem.dtx}
\end{document}
%</driver>
%    \end{macrocode}
%
% \section{Definition of the Macros}
%
% If the file has been loaded before, we abort immediately. If not the
% package announces itself (this is actually done at the very top of
% the file---the way it is done isn't good style so don't copy it).
% \changes{v2.0e}{89/07/19}{Spaces removed from `typeout'.}
% \changes{v2.2b}{95/11/19}{Announce moved to top of file}
%    \begin{macrocode}
%<*package>
%\@ifundefined{theorem@style}{}{\endinput}
%\def\FMithmInfo{1995/11/19 v2.2b Theorem extension package (FMi)}
%\ProvidesPackage{theorem}[\FMithmInfo]
%    \end{macrocode}
%
%
% \subsection{Definition of theorem styles and fonts}
%
% All the definitions in this file are done globally to allow
% inputting this file inside a group.
%
% \begin{macro}{\theoremstyle}
%    Before a theorem style can be installed, the chosen style must be
%    known. For that reason, we must test to see that
%    |\th@|\meta{style} is known or, more precisely, that it is
%    different from |\relax|. If the style is not known then
%    |\th@plain| is used.
%    \begin{macrocode}
\gdef\theoremstyle#1{%
   \@ifundefined{th@#1}{\@warning
          {Unknown theoremstyle `#1'. Using `plain'}%
          \theorem@style{plain}}%
%    \end{macrocode}
%    We save the theorem style to be used in the token register
%    |\theorem@style|.
%    \begin{macrocode}
      {\theorem@style{#1}}%
%    \end{macrocode}
%    Now we ``evaluate'' the theorem style: this means, we call the
%    macro |\th@|\meta{style} which will activate the relevant
%    definitions which are contained in a separate file.  This is done
%    in a group to suppress changes to the current font.  This could
%    otherwise pose problems together with the new font selection
%    scheme\footnote{When I printed the original article using the new
%    font selection scheme I ended with a document with slanted
%    typefaces (text headings and all) simply because one of the
%    theorem styles used \texttt{\bslash sl} at toplevel.} if the
%    |\th@|\meta{style} is evaluated a second time.
% \changes{v2.1a}{90/02/15}{Group added around `th@..}
% \changes{v2.1a}{90/02/15}{Reseting of `..@begintheorem removed.}
%    \begin{macrocode}
      \begingroup
        \csname th@\the\theorem@style \endcsname
      \endgroup}
%    \end{macrocode}
% \end{macro}
%
%
% \begin{macro}{\@begintheorem}
% \changes{v2.1a}{90/02/15}{Relaxed at toplevel.}
% \begin{macro}{\@opargbegintheorem}
% \changes{v2.1a}{90/02/15}{Relaxed at toplevel.}
%    We reset |\@begintheorem| and |\@opargbegintheorem| to
%    |\relax| since these commands are no longer necessary at
%    toplevel. This will save a few tokens.
%    \begin{macrocode}
\global\let\@begintheorem\relax
\global\let\@opargbegintheorem\relax
%    \end{macrocode}
% \end{macro}
% \end{macro}
%
% \begin{macro}{\theorem@style}
%    Obviously the \textsf{token} register used above has to be
%    allocated. To assure the utmost compatibility with the
%    original \LaTeX{} definition, we set the default theorem style to
%    \texttt{plain}, which implements the usual \LaTeX{} convention.
% \changes{v2.0f}{89/07/20}{Unnecessary `global' removed.}
%    \begin{macrocode}
\newtoks\theorem@style
\global\theorem@style{plain}
%    \end{macrocode}
% \end{macro}
%
% \begin{macro}{\theorembodyfont}
% \begin{macro}{\theorem@bodyfont}
%    If the |\theorembodyfont| is set by the user then it should not
%    interact with the default font set in the theorem style. When the
%    new font selection is in force this may happen if, for example,
%    the default is |\itshape| and the new |\theorembodyfont| is
%    |\sffamily|. So we add a |\reset@font| command in front of the
%    user definition.
% \changes{v2.0f}{89/07/20}{Unnecessary `global' removed.}
% \changes{v2.0g}{89/09/19}{`theoremfont to `theorembodyfont.}
% \changes{v2.1c}{92/06/22}{`reset@font added and logic changed.}
% \changes{v2.1d}{92/10/28}{Reset to empty if arg is empty}
%    \begin{macrocode}
\gdef\theorembodyfont#1{%
%    \end{macrocode}
%    We check if the argument supplied is empty and if so put nothing
%    into the |\theorem@bodyfont| token register to allow for
%    |\theorembodyfont{}| as a mean of using the default of the
%    current |\theoremstyle|.
%    \begin{macrocode}
   \def\@tempa{#1}%
   \ifx\@tempa\@empty
    \theorem@bodyfont{}%
   \else
    \theorem@bodyfont{\reset@font#1}%
   \fi
   }
\newtoks\theorem@bodyfont
\global\theorem@bodyfont{}
%    \end{macrocode}
% \end{macro}
% \end{macro}
%
%
% \begin{macro}{\theoremheaderfont}
% \changes{v2.0g}{89/09/19}{Macro added.}
%    The font for the theorem headers is handled differently because
%    this definition applies to all theorem styles.
%    \begin{macrocode}
\gdef\theoremheaderfont#1{\gdef\theorem@headerfont{#1}%
%    \end{macrocode}
%    After using the macro once it is redefined to produce an error
%    message.
%    \begin{macrocode}
       \gdef\theoremheaderfont##1{%
        \typeout{\string\theoremheaderfont\space should be used
                 only once.}}}
%    \end{macrocode}
% \end{macro}
%
% \begin{macro}{\theorem@headerfont}
% \changes{v2.0g}{89/09/19}{Macro added.}
%    To set the |\theorem@headerfont| default we first test if
%    the new fontselection scheme is in force.
% \changes{v2.1a}{90/02/15}{Support for new font selection.}
%    \begin{macrocode}
\ifx\upshape\undefined
%    \end{macrocode}
%    If not we define it to expand into |\bfseries|. We don't use
%    |\let| just in case a following style option redefines this
%    macro.
%    \begin{macrocode}
\gdef\theorem@headerfont{\bfseries}
%    \end{macrocode}
%    Otherwise we reset the current shape before calling |\bfseries|.
%    \begin{macrocode}
\else \gdef\theorem@headerfont{\normalfont\bfseries}\fi
%    \end{macrocode}
% \end{macro}
%
% \begin{macro}{\th@plain}
% \begin{macro}{\th@break}
% \begin{macro}{\th@marginbreak}
% \begin{macro}{\th@changebreak}
% \begin{macro}{\th@change}
% \begin{macro}{\th@margin}
%    The different styles are defined in macros such as
%    |\th@plain|.  Since memory space is precious in
%    ``non-Big-versions'', we have to avoid offering too many unused
%    definitions.  Therefore we define these styles in separate files
%    that can be loaded on demand.  Thus the commands themselves only
%    load these files. We use |\@input@|, a \LaTeXe{} internal command
%    that ensures that the file will be listed with |\listfiles|.
% \changes{v2.2a}{94/02/02}{Use `@input@ to load theorem layout files}
%    \begin{macrocode}
\gdef\th@plain{\@input@{thp.sty}}
\gdef\th@break{\@input@{thb.sty}}
\gdef\th@marginbreak{\@input@{thmb.sty}}
\gdef\th@changebreak{\@input@{thcb.sty}}
\gdef\th@change{\@input@{thc.sty}}
\gdef\th@margin{\@input@{thm.sty}}
%    \end{macrocode}
%    This list will be expanded when new styles become
%    available.  For testing,  just append new theorem substyles
%    as document options.
% \end{macro}
% \end{macro}
% \end{macro}
% \end{macro}
% \end{macro}
% \end{macro}
%
%
%
% \subsection{Definition of a new theorem set}
%
% As already pointed out, a new theorem environment can be defined in
% three different ways:
% \begin{verbatim}
%    \newtheorem{Lem}{Lemma}
%    \newtheorem{Lem}{Lemma}[section]
%    \newtheorem{Lem}[Theorem]{Lemma}
%\end{verbatim}
% The function of the macro |\newtheorem| is to recognize these cases
% and then to branch into one of the three macros |\@ynthm|, |\@xnthm|
% or |\@othm|.  This mechanism is adopted unchanged from
% \cite{latex:source}; the essential point here is that, for example,
% in the second case, the arguments \texttt{Lem}, \texttt{Lemma} and
% \texttt{section} are passed over to the macro |\@xnthm|.
%
% We inspect this case first because the others present fewer problems,
% and thus are easily derived from this one.
%
% \begin{macro}{\@xnthm}
%    For our example arguments, the macro |\@xnthm| must fulfill
%    the following:
%    \begin{itemize}
%       \item
%          Define a new \LaTeX-counter `Lem'
%       \item
%          reset this counter within a |\section|
%       \item
%          define the macro |\theLem|
%       \item
%          define the environment macros |\Lem| and
%          |\endLem| using the current
%          |\theoremstyle| and |\theorem@bodyfont|.
%    \end{itemize}
%    Obviously, all this should happen only if the first argument
%    of |\@xnthm| (i.e.\ \texttt{Lem} in our example) is chosen so
%    as not to conflict with any previously defined commands or
%    environments.  This test is performed by the \LaTeX{} macro
%    |\@ifdefinable|.
%    \begin{macrocode}
\gdef\@xnthm#1#2[#3]{\expandafter\@ifdefinable\csname #1\endcsname
%    \end{macrocode}
%    Therefore, the first argument of |\@ifdefinable| is the
%    expansion (in the example, |\Lem|) of
%    |\csname||#1\endcsname|.  The second argument is
%    executed only if the test has been completed successfully.
%    \begin{macrocode}
   {%
%    \end{macrocode}
%    Now we define the new counter. The names of the \LaTeX{}
%    macros employed should speak for themselves:
% \changes{v2.2b}{1995/10/07}{Use \cs{@newctr}}
% \changes{v2.2c}{1995/11/23}{Use \cs{@newctr} properly.}
%    \begin{macrocode}
    \@definecounter{#1}\@newctr{#1}[#3]%
%    \end{macrocode}
%    Using |\@newctr| will give a proper error message if the counter
%    in |#3| is not defined.
%    In defining `|\theLem|' we must generate the desired macro
%    name by use of |\expandafter| and |\csname|.
%    \begin{macrocode}
    \expandafter\xdef\csname the#1\endcsname
%    \end{macrocode}
%    An |\xdef| is used in order to make the definition global,
%    and to ensure that it contains the replacement texts of
%    |\@thmcountersep| and |\@thmcounter|.\footnote{These
%    two macros can be defined by the document style.  Their default
%    values produce a `\texttt{.}' as separation and an arabic
%    representation of the number.} However, not everything should be
%    expanded. For example, it saves space to use |\thesection|
%    instead of its---at times---lengthy expansion.
%    \begin{macrocode}
      {\expandafter \noexpand \csname the#3\endcsname
       \@thmcountersep \@thmcounter{#1}}%
%    \end{macrocode}
%    Thus with the defaults of \LaTeX{}, |\theLem| would be
%    replaced by the command sequence |\thesection.\arabic{Lem}|.
%
%    We will now look at the definition of the macro which is executed
%    at the beginning of the actual environment (in our example this
%    macro is |\Lem|). It should be noted that we use an
%    ``|\expandafter| trick'' to expand only certain parts of the
%    replacement text at the time of the definition.
%    \begin{macrocode}
    \def\@tempa{\global\@namedef{#1}}%
    \expandafter \@tempa \expandafter{%
%    \end{macrocode}
%    First, the macro that contains the current definitions of
%    |\@begintheorem| and |\@opargtheorem| should be called
%    up.  The name of this macro---as is already known---has the form
%    |\th@|\meta{theorem\ style}; therefore, it must be called by
%    \begin{macrocode}
      \csname th@\the \theorem@style
%    \end{macrocode}
%    In addition the default theorem font should be changeable, i.e.\
%    we have to insert the contents of |\theorem@bodyfont|. For
%    that reason, we expand even further, beyond |\endcsname|,
%    and thus insert the contents of the \textsf{token} register
%    |\theorem@bodyfont| in the replacement text.
%    \begin{macrocode}
            \expandafter \endcsname \the \theorem@bodyfont
%    \end{macrocode}
%    Now it is time to call the macro |\@thm| which takes over
%    the further processing.  It has two arguments: the current
%    counter name (in our example, \texttt{Lem}), and the text of the
%    label (in our example, \texttt{Lemma}).
%    \begin{macrocode}
     \@thm{#1}{#2}}%
%    \end{macrocode}
%    With this, the `sub-definition' is complete.  The macro
%    |\@endtheorem| ends a theorem environment and is, so far, nothing
%    but an |\endtrivlist|.  (Hence it is defined globally, and not
%    within the theorem styles.\footnote{This has to be changed as
%    soon as theorem styles that change \texttt{\bslash @endtheorem}
%    exist.  In such a case, all existing styles must be changed as
%    well since they will have to reset the macro.}) Therefore, we can
%    set it equivalent to the macro that ends the theorem set (in our
%    example, |\endLem|).  However, if some day theorem styles exist
%    that do change |\@endtheorem|, we would have to use the
%    commented-out line instead.
%    \begin{macrocode}
    \global \expandafter \let \csname end#1\endcsname \@endtheorem
%   \global\@namedef{end#1}{\@endtheorem}%
%    \end{macrocode}
%    With these commands all the required definitions are employed,
%    unless the test |\@ifdefinable| has failed.  Therefore, we
%    end the second argument of this macro and with it the definition
%    of |\@xnthm|.
%    \begin{macrocode}
   }}
%    \end{macrocode}
% \end{macro}
%
% \begin{macro}{\@ynthm}
%    The definition of |\@ynthm| is completely analogous. In
%    this case the new counter that is defined is not reset within
%    another counter; thus the definition of |\the...|
%    is simplified:
%    \begin{macrocode}
\gdef\@ynthm#1#2{\expandafter\@ifdefinable\csname #1\endcsname
   {\@definecounter{#1}%
    \expandafter\xdef\csname the#1\endcsname{\@thmcounter{#1}}%
%    \end{macrocode}
%    The rest of the definition corresponds literally to that of
%    |\@xnthm|:
%    \begin{macrocode}
    \def\@tempa{\global\@namedef{#1}}\expandafter \@tempa
     \expandafter{\csname th@\the \theorem@style \expandafter
     \endcsname \the\theorem@bodyfont \@thm{#1}{#2}}%
    \global \expandafter \let \csname end#1\endcsname \@endtheorem}}
%    \end{macrocode}
% \end{macro}
%
% \begin{macro}{\@othm}
%    The definition of |\@othm| does not contain anything new.
%    \begin{macrocode}
\gdef\@othm#1[#2]#3{%
%    \end{macrocode}
% \changes{v2.2b}{95/10/07}{Build in check for counter defined}
%    We do not define a new counter but instead use one that has
%    already been defined.  Thus the only definition we need is that
%    of this pseudo-counter (i.e.\ |\the|\meta{env.\ name}).
%    First we check if |#2| corresponds to a known counter name.
%    \begin{macrocode}
  \expandafter\ifx\csname c@#2\endcsname\relax
   \@nocounterr{#2}%
  \else
   \expandafter\@ifdefinable\csname #1\endcsname
   {\expandafter \xdef \csname the#1\endcsname
     {\expandafter \noexpand \csname the#2\endcsname}%
%    \end{macrocode}
%    All other parts of the definition can be adopted from
%    |\@xnthm|. We have to remember, though, that in this case
%    the name of the current counter and the theorem label have moved
%    to the second and third arguments.
%    \begin{macrocode}
    \def\@tempa{\global\@namedef{#1}}\expandafter \@tempa
     \expandafter{\csname th@\the \theorem@style \expandafter
     \endcsname \the\theorem@bodyfont \@thm{#2}{#3}}%
    \global \expandafter \let \csname end#1\endcsname \@endtheorem}%
  \fi}
%    \end{macrocode}
% \end{macro}
%
% \subsection{Macros that are employed in a {\envfont theorem}
%    environment}
%
% \begin{macro}{\@thm}
%    The macro |\@thm| has to increase the current counter.  Then,
%    depending on whether the environment has (or does not have) an
%    optional argument, it has to branch into either |\@begintheorem|
%    or |\@opargtheorem|.
%    \begin{macrocode}
\gdef\@thm#1#2{\refstepcounter{#1}%
%    \end{macrocode}
%    Now we start a {\envfont trivlist} environment, and give
%    |\@topsep| and |\@topsepadd| the values of the \textsf{skip}
%    registers |\theorempreskipamount| and |\theorempostskipamount|.
%    The value in |\@topsep| is the vertical space that is inserted by
%    the first (and only) |\item| in our |\trivlist| whilst
%    |\@topsepadd| is inserted by |\@endparenv| at the end of that
%    {\envfont trivlist} environment.  By using these registers, we
%    obtain the desired space around a {\envfont theorem} environment.
%    \begin{macrocode}
   \trivlist
   \@topsep \theorempreskipamount               % used by first \item
   \@topsepadd \theorempostskipamount           % used by \@endparenv
%    \end{macrocode}
%    Now we have to test whether an optional argument has been given.
% \changes{v2.0d}{89/07/18}{Forgotten percent added.}
%    \begin{macrocode}
   \@ifnextchar [%
%    \end{macrocode}
%    If there is an optional argument, we will call |\@ythm|, and
%    move the arguments read back into the input stream.
%    \begin{macrocode}
   {\@ythm{#1}{#2}}%
%    \end{macrocode}
%    If not, we call |\@begintheorem|. Its first argument is the
%    name of the theorem set (hence the second argument of
%    |\@thm|). Its second argument is the macro that produces the
%    current number.
%    \begin{macrocode}
   {\@begintheorem{#2}{\csname the#1\endcsname}\ignorespaces}}
%    \end{macrocode}
% \end{macro}
%
% \begin{macro}{\@xthm}
% \begin{macro}{\@ythm}
%    Both these macros were originally called by |\@thm|.  We do
%    not need |\@xthm| anymore, hence we reset it to
%    |\relax|. The definition of |\@ythm| has not changed at
%    all from its definition in \LaTeX. In order to make the macros
%    easier to understand, we will
%    nevertheless present it (commented out).
%    \begin{macrocode}
\global\let\@xthm\relax
% \def\@ythm#1#2[#3]{\@opargbegintheorem{#2}{\csname
%           the#1\endcsname}{#3}\ignorespaces}
%</package>
%    \end{macrocode}
%    The primitive |\ignorespaces| in |\@ythm| and
%    |\@thm| is needed to remove the spaces between the
%    |\begin{...}| and the actual text.
% \end{macro}
% \end{macro}
%
%
% \subsection{Definition of the theorem substyles}
%
% As already pointed out, the theorem substyles, defined below, are
% only loaded when necessary.  Note that all these substyles, except
% \texttt{plain}, have |\slshape| as the default body font.
%
% \subsubsection{The \texttt{plain} style}
%
% \label{dok:th@plain} As the following macros use |@|, we have
% to locally set the |\catcode| of this symbol to ``letter''.
% This happens within a group, so that we do not have to worry about
% which |\catcode| that symbol had before.
%    \begin{macrocode}
%<*thp>
\begingroup \makeatletter
%    \end{macrocode}
% Since we are now within a group, we must make all definitions
% globally.  First we make sure that \texttt{theorem.sty} is loaded.
% This will allow us to use this file as a document style option
% without having to call \texttt{theorem} itself as an option.  At the
% same time, we assure that at least version 2 is loaded, since
% |\theorem@style| was not defined in earlier versions.
%    \begin{macrocode}
\@ifundefined{theorem@style}{\input{theorem.sty}}{}
\ProvidesFile{thp.sty}
             [\FMithmInfo]
%    \end{macrocode}
%
%
% \begin{macro}{\th@plain}
% \begin{macro}{\@begintheorem}
% \begin{macro}{\@opargbegintheorem}
%    |\theoremstyle{plain}| corresponds to the original
%    definition, except that the distances to the surrounding text are
%    determined by the parameters |\theorempreskipamount| and
%    |\theorempostskipamount|.
%    First we set the default body font.
%    \begin{macrocode}
\gdef\th@plain{\normalfont\itshape
%    \end{macrocode}
%    Then we define |\@begintheorem| and
%    |\@opargbegintheorem|. These two macros define how the
%    header of a theorem is typeset.  |\@opargbegintheorem| will
%    be called if a \textsf{theorem} environment with an optional
%    argument is encountered; otherwise, the header is constructed by
%    calling |\@begintheorem|.  If one of these macros is
%    executed, we are within a \textsf{trivlist} environment started by
%    |\@thm|.  So the theorem header is produced with an
%    |\item| command.
%
%    Instead of specifying the header font directly, all standard
%    theorem styles use the |\theorem@headerfont| macro to allow
%    customization.  The extra space (|\labelsep|) is necessary
%    because of problems in the \textsf{trivlist} environment.
%    \begin{macrocode}
  \def\@begintheorem##1##2{%
        \item[\hskip\labelsep \theorem@headerfont ##1\ ##2]}%
%    \end{macrocode}
%    The definition of |\@opargbegintheorem| is completely
%    analogous. The only difference is the fact that there exists a
%    third argument (which is the optional parameter of the
%    environment and contains additional information about the
%    theorem).  Customarily we enclose it in parentheses.
%    \begin{macrocode}
\def\@opargbegintheorem##1##2##3{%
   \item[\hskip\labelsep \theorem@headerfont ##1\ ##2\ (##3)]}}
%    \end{macrocode}
% \end{macro}
% \end{macro}
% \end{macro}
%
% \noindent
% We conclude with an |\endgroup| to restore the
% |\catcode| of |@|.
%    \begin{macrocode}
\endgroup
%</thp>
%    \end{macrocode}
%
%
% \subsubsection{The \texttt{break} style}
%
% This style option is stored in the file |thb.sty|.
% For the next two lines see the documentation for |\th@plain| on
% page~\pageref{dok:th@plain}.
%    \begin{macrocode}
%<*thb>
\begingroup \makeatletter
\@ifundefined{theorem@style}{\input{theorem.sty}}{}
\ProvidesFile{thb.sty}
             [\FMithmInfo]
%    \end{macrocode}
%
% \begin{macro}{\th@break}
%    |\theoremstyle{break}| produces a line break after the name
%    of the theorem.  The font is |\slshape|.  Hence, we define
%    |\th@break| as follows:
%    \begin{macrocode}
\gdef\th@break{\normalfont\slshape
  \def\@begintheorem##1##2{\item[%
%    \end{macrocode}
%    We run into the following problem: it is not possible to create
%    the header with |\item[|\meta{title}|]| and then start
%    a new line by, for example, |\mbox{}\\|. Such a definition
%    will fail whenever a list environment follows immediately.  With
%    the above construction, the |\mbox{}| causes the switch
%    |@inlabel| (cf.\ definition of |\list| and
%    |\trivlist| in \cite{latex:source}) to be set to \texttt{false}
%    and so the following list will insert additional vertical
%    space (|\topskip|). This is quite annoying. Therefore, we
%    create the line break within the |\item|. In order to ensure
%    that the text will begin at the proper position in the following
%    line, we simply pretend that the label does not take any
%    room.\footnote{This will lead to problems whenever very high
%    symbols occurring in the line tower into the
%    heading. So, something else has to be done here sometime.}
%    \begin{macrocode}
       \rlap{\vbox{\hbox{\hskip \labelsep\theorem@headerfont ##1\ ##2}%
                   \hbox{\strut}}}]}%
%    \end{macrocode}
%    Again, the definition of |\@opargbegintheorem| is
%    completely analogous.
%    \begin{macrocode}
\def\@opargbegintheorem##1##2##3{%
  \item[\rlap{\vbox{\hbox{\hskip \labelsep \theorem@headerfont
                     ##1\ ##2\ (##3)}%
                    \hbox{\strut}}}]}}
\endgroup
%</thb>
%    \end{macrocode}
% \end{macro}
%
%
% \subsubsection{The \texttt{changebreak} style}
%
%    \begin{macrocode}
% This style option is stored in the file |thcb.sty|.
%    \begin{macrocode}
%<*thcb>
\begingroup \makeatletter
\@ifundefined{theorem@style}{\input{theorem.sty}}{}
\ProvidesFile{thcb.sty}
             [\FMithmInfo]
%    \end{macrocode}
%
% \begin{macro}{\th@changebreak}
%    The \texttt{change-break} theorem style is like \texttt{break}
%    but with interchange of theorem name and theorem number.  Thus we
%    define |\th@changebreak| as follows:
%    \begin{macrocode}
\gdef\th@changebreak{\normalfont\slshape
  \def\@begintheorem##1##2{\item
       [\rlap{\vbox{\hbox{\hskip\labelsep \theorem@headerfont ##2\ ##1}%
                    \hbox{\strut}}}]}%
\def\@opargbegintheorem##1##2##3{%
  \item[\rlap{\vbox{\hbox{\hskip\labelsep \theorem@headerfont
                     ##2\ ##1\ (##3)}%
                    \hbox{\strut}}}]}}
\endgroup
%</thcb>
%    \end{macrocode}
% \end{macro}
%
%
% \subsubsection{The \texttt{change} style}
%
% This style option is stored in the file |thc.sty|.
%    \begin{macrocode}
%<*thc>
\begingroup \makeatletter
\@ifundefined{theorem@style}{\input{theorem.sty}}{}
\ProvidesFile{thc.sty}
             [\FMithmInfo]
%    \end{macrocode}
%
% \begin{macro}{\th@change}
%    The \texttt{change} theorem style corresponds to the
%    \texttt{change break} style without a linebreak after the header.
%    To say it in another way, it's the same as the \texttt{plain}
%    style but with number and name interchanged and |\slshape| as the
%    default font.
%    \begin{macrocode}
\gdef\th@change{\normalfont\slshape
  \def\@begintheorem##1##2{\item
       [\hskip\labelsep \theorem@headerfont ##2\ ##1]}%
\def\@opargbegintheorem##1##2##3{%
  \item[\hskip\labelsep \theorem@headerfont ##2\ ##1\ (##3)]}}
\endgroup
%</thc>
%    \end{macrocode}
% \end{macro}
%
%
% \subsubsection{The \texttt{marginbreak} style}
%
% This style option is the one used most often at Mainz.  It is saved
% in the file |thmb.sty|.
%    \begin{macrocode}
%<*thmb>
\begingroup \makeatletter
\@ifundefined{theorem@style}{\input{theorem.sty}}{}
\ProvidesFile{thmb.sty}
             [\FMithmInfo]
%    \end{macrocode}
%
% \begin{macro}{\th@marginbreak}
%    The \texttt{margin break} style is nearly the same as the
%    \texttt{change break} style. The only difference is the placement
%    of the theorem number.  We use |\llap| to place it in the left
%    margin.
%
%    In this style |\labelsep| denotes the separation between the
%    number and the text.
%    \begin{macrocode}
\gdef\th@marginbreak{\normalfont\slshape
  \def\@begintheorem##1##2{\item
       [\rlap{\vbox{\theorem@headerfont
                    \hbox{\llap{##2}\hskip\labelsep ##1}%
                    \hbox{\strut}}}]}%
\def\@opargbegintheorem##1##2##3{%
  \item[\rlap{\vbox{\theorem@headerfont
                    \hbox{\llap{##2}\hskip\labelsep ##1\ (##3)}%
                    \hbox{\strut}}}]}}
\endgroup
%</thmb>
%    \end{macrocode}
% \changes{v2.1c}{90/02/17}{`endgroup outside macrocode env.}
% \end{macro}
%
%
%
% \subsubsection{The \texttt{margin} style}
%
% This style option is stored in the file |thm.sty|.
%    \begin{macrocode}
%<*thm>
\begingroup \makeatletter
\@ifundefined{theorem@style}{\input{theorem.sty}}{}
\ProvidesFile{thm.sty}
             [\FMithmInfo]
%    \end{macrocode}
%
% \begin{macro}{\th@margin}
%    Again this is only a variant of the theorem styles described
%    above without any new ideas.
%    \begin{macrocode}
\gdef\th@margin{\normalfont\slshape
  \def\@begintheorem##1##2{\item
       [\theorem@headerfont \llap{##2}\hskip\labelsep ##1]}%
\def\@opargbegintheorem##1##2##3{%
  \item[\theorem@headerfont \llap{##2}\hskip\labelsep ##1\ (##3)]}}
\endgroup
%</thm>
%    \end{macrocode}
% \end{macro}
%
%
%
%
% \subsection{Final Definitions}
%
% \begin{macro}{\theorempreskipamount}
% \begin{macro}{\theorempostskipamount}
%    The \textsf{skip} parameters that regulate the vertical empty space
%    before and after the {\envfont theorem} environment have
%    to be allocated as well.
% \changes{v2.0f}{89/07/20}{Unnecessary `global' removed.}
%    \begin{macrocode}
%<*package>
\newskip\theorempreskipamount
\newskip\theorempostskipamount
%    \end{macrocode}
%    Since we have used the same values for all theorem sets, we now can
%    assign them.
%    \begin{macrocode}
\global\setlength\theorempreskipamount{12pt plus 5pt minus 3pt}
\global\setlength\theorempostskipamount{8pt plus 3pt minus 1.5pt}
%    \end{macrocode}
% \end{macro}
% \end{macro}
%
%
% \begin{macro}{\@endtheorem}
%    The same holds for the macro |\@endtheorem|, which ends a
%    {\envfont theorem} environment. Since it is the same for all
%    theorem sets, it is removed from the macros |\th@|\meta{style}.
%    It simply ends the {\envfont trivlist} environment, which was
%    begun in |\@thm|.
%    \begin{macrocode}
\global\let\@endtheorem=\endtrivlist
%    \end{macrocode}
% \end{macro}
%
%
%
% \begin{macro}{\@preamblecmds}
%    All macros defined above are to be used only in the preamble.
%    Therefore, we insert them in |\@preamblecmds| which will disable
%    them at begin document. This is done by the internal \LaTeXe{}
%    command |\@onlypreamble|.
% \changes{v2.0g}{89/09/19}{`theoremheaderfont and `theorembodyfont
%    added.}
% \changes{v2.2a}{94/02/03}{Use \LaTeXe{} interface}
%    \begin{macrocode}
\@onlypreamble\@xnthm
\@onlypreamble\@ynthm
\@onlypreamble\@othm
\@onlypreamble\newtheorem
\@onlypreamble\theoremstyle
\@onlypreamble\theorembodyfont
\@onlypreamble\theoremheaderfont
%    \end{macrocode}
% \end{macro}
%
% Finally we declare the \texttt{plain} theorem style to be the default.
% \changes{v2.1b}{90/06/04}{Fixed missing `item by declaring default
%    style.}
%    \begin{macrocode}
\theoremstyle{plain}
%</package>
%    \end{macrocode}
%
% \Finale

\endinput
}{}
\ProvidesFile{thc.sty}
             [\FMithmInfo]
%    \end{macrocode}
%
% \begin{macro}{\th@change}
%    The \texttt{change} theorem style corresponds to the
%    \texttt{change break} style without a linebreak after the header.
%    To say it in another way, it's the same as the \texttt{plain}
%    style but with number and name interchanged and |\slshape| as the
%    default font.
%    \begin{macrocode}
\gdef\th@change{\normalfont\slshape
  \def\@begintheorem##1##2{\item
       [\hskip\labelsep \theorem@headerfont ##2\ ##1]}%
\def\@opargbegintheorem##1##2##3{%
  \item[\hskip\labelsep \theorem@headerfont ##2\ ##1\ (##3)]}}
\endgroup
%</thc>
%    \end{macrocode}
% \end{macro}
%
%
% \subsubsection{The \texttt{marginbreak} style}
%
% This style option is the one used most often at Mainz.  It is saved
% in the file |thmb.sty|.
%    \begin{macrocode}
%<*thmb>
\begingroup \makeatletter
\@ifundefined{theorem@style}{% \iffalse meta-comment
%
% Copyright (C) 1993-2024
%
% The LaTeX Project and any individual authors listed elsewhere
% in this file.
%
% This file is part of the Standard LaTeX `Tools Bundle'.
% -------------------------------------------------------
%
% It may be distributed and/or modified under the
% conditions of the LaTeX Project Public License, either version 1.3c
% of this license or (at your option) any later version.
% The latest version of this license is in
%    https://www.latex-project.org/lppl.txt
% and version 1.3c or later is part of all distributions of LaTeX
% version 2005/12/01 or later.
%
% The list of all files belonging to the LaTeX `Tools Bundle' is
% given in the file `manifest.txt'.
%
% \fi
%
%%
%
%
\def\FMithmInfo{2023/07/05 v2.2c Theorem extension package (FMi)}
%
% \ProvidesFile{theorem.dtx}[\FMithmInfo]
%\iffalse   % this is a METACOMMENT !
%%
%% Package `theorem' to use with LaTeX2e
%% Copyright (C) 1989-1995 Frank Mittelbach, all rights reserved.
%<+package>\@ifundefined{theorem@style}{}{\endinput}
%<+package>\NeedsTeXFormat{LaTeX2e}[1995/06/01]
%<+package>\ProvidesPackage{theorem}
%<+package>         [\FMithmInfo]
%
% \fi
%
% \changes{v1.0}{87/05/07}{First draft.}
% \changes{v1.0a}{87/05/07}{`thistheoremstyle' added.}
% \changes{v1.1a}{87/07/29}{New concept: redefine the internal
%                          \protect\LaTeX{} commands `@xnthm' etc.}
% \changes{v1.1b}{87/07/31}{`theoremfont' used to specify font.}
% \changes{v1.2a}{88/01/08}{`@endtheorem' now globally defined.}
% \changes{v1.2b}{88/01/11}{`newtheorem' only allowed in the preamble.}
% \changes{v1.3a}{88/02/14}{`theoremfont' now a toks register.}
%
% \changes{v2.0a}{88/11/16}{New implementation and new documentation.}
% \changes{v2.0a}{88/11/16}{No layout changes but internal modifications
%                           old substyles won't work.}
% \changes{v2.0b}{89/01/30}{`global' used for all definitions.}
% \changes{v2.0c}{89/06/15}{Documentation translated to English.
%                            Thanks to LLL and Nina}
% \changes{v2.1a}{90/02/15}{Support for new font selection.}
% \changes{v2.2a}{94/02/03}{Cleanup for LaTeX2e}
%
% \DoNotIndex{\@endparenv,\@ifdefinable,\@ifnextchar,\@ifundefined}
% \DoNotIndex{\@namedef,\@nameuse,\@spaces,\@tempa,\@warning}
% \DoNotIndex{\ }
% \DoNotIndex{\begingroup,\bfseries}
% \DoNotIndex{\csname}
% \DoNotIndex{\def,\do,\docdate}
% \DoNotIndex{\endcsname,\endgroup,\endinput,\endtrivlist,\expandafter}
% \DoNotIndex{\filedate,\fileversion}
% \DoNotIndex{\gdef,\global}
% \DoNotIndex{\hbox,\hskip}
% \DoNotIndex{\ignorespaces,\input,\itshape,\item}
% \DoNotIndex{\labelsep,\let,\llap}
% \DoNotIndex{\makeatletter}
% \DoNotIndex{\noexpand}
% \DoNotIndex{\refstepcounter,\relax,\rlap}
% \DoNotIndex{\setlength,\slshape,\space,\string,\strut}
% \DoNotIndex{\the,\trivlist,\typeout}
% \DoNotIndex{\vbox}
% \DoNotIndex{\xdef}
%
% \def\deflabel#1{\ttfamily #1\hfill}
% \def\deflist#1{\begin{list}{}{\settowidth\labelwidth{\ttfamily #1}%
%                               \setlength\leftmargin\labelwidth
%                               \addtolength\leftmargin\labelsep
%                               \let\makelabel\deflabel}}
% \def\enddeflist{\end{list}}
% \def\envfont{\normalfont\ttfamily}
%
% \MakeShortVerb{\|}
%
% \newcommand{\TUB}{{\sl TUGboat\/}}
%
% \GetFileInfo{theorem.dtx}
% \title{An Extension of the
%        \LaTeX{} theorem environment\thanks{This file
%        has version number \fileversion, last
%        revised \filedate.}}
% \author{Frank Mittelbach}
%
% \MaintainedByLaTeXTeam{tools}
% \maketitle
%
% \begin{abstract}
%    The macros described in this paper yield an extension of the
%    \LaTeX{} theorem mechanism. It is designed to satisfy the
%    different requirements of various journals. Thus, the
%    layout of the ``theorems'' can be manipulated by determining a
%    ``style''. This article describes not only the use, but
%    also the definition, of the necessary macros.
% \end{abstract}
%
% \section*{Preface to version 2.2}
%
% For \LaTeXe{} this package did not need any fundamental changes. I
% only modified the messages generated so that theorem layout styles
% will show up with the |\listfiles| command and cleaned the section
% on the New Font Selection Scheme since this is now included in
% \LaTeX.
%
% \section*{Preface to version 2.1}
%
% This version is identical to 2.0g described in \TUB\ 10\#3 except
% for some internal defaults which are now set depending on the used
% font selection scheme.
%
% This was done to avoid unpleasant surprises if the new font selection
% scheme is in force. For further details see
% section~\ref{sec:newfonts} and \cite{bk:GMS94}.
%
% \section{Introduction}
%
% For our purposes here, ``theorems'' are labelled enunciations,
% often set off from the main text by extra space and a font change.
% Theorems, corollaries, conjectures, definitions, and remarks
% are all instances of ``theorems''.  The ``header'' of these structures
% is composed of a label (such as \textsc{Theorem} or \textsc{Remark})
% and a number which serializes an item in the sequence of items
% with the same label.
%
% Shortly after the introduction of \LaTeX{} at the Fachbereich
% Mathematik in Mainz, the desire to manipulate the layout of
% ``theorems'' arose.  In Mainz, the following two conventions came
% into general use:
% \begin{enumerate}
%   \item The number of the theorem is shown in the margin.
%   \item There is a line break at
%         the end of the theorem header.
% \end{enumerate}
%
% Additionally, some journals require different formats which depend on
% the ``sort of theorem'': e.g.\ often remarks and definitions are set
% in |\upshape|, while |\itshape| is employed for main theorems.
%
% Confronted with these requirements, a theorem environment was
% developed in Mainz which allows separate determination of the
% layout of the ``theorems sets'', comparable to |\pagestyle|.
%
%
% \section{The user interface}
% \subsection{Defining new theorem sets}
%
% \DescribeMacro\newtheorem
% As in the original \LaTeX{} version, the command |\newtheorem|
% defines a new ``theorem set'' or ``theorem-like structure''.
% Two required arguments name the new environment and give the text
% to be typeset with each instance of the new ``set'', while
% an optional argument determines how
% the ``set'' is enumerated:
% \begin{description}
%    \item[\ttfamily \bslash newtheorem\{foo\}\{bar\}]
%       The theorem set {\envfont foo} (whose name is \texttt{bar})
%       uses its own counter.
%    \item[\ttfamily \bslash newtheorem\{foo2\}{[foo]}\{bar2\}]
%       The theorem set {\envfont foo2} (printed name \texttt{bar2})
%       uses the same counter as the theorem set \texttt{foo}.
%    \item[\ttfamily \bslash newtheorem\{foo3\}\{bar3\}{[section]}]
%       The theorem set {\envfont foo3} (printed name \texttt{bar3}) is
%       enumerated within the counter \texttt{section}, i.e.\ with every
%       new |\section| the enumeration begins again with 1, and
%       the enumeration is composed from the section-number and the
%       theorem counter itself.
% \end{description}
%
% \DescribeMacro\theoremstyle
% Additionally, the command |\theoremstyle| can define the layout
% of various, or all, theorem sets.  It should be noted that any
% theorem set defined by |\newtheorem| is typeset in the
% |\theoremstyle| that is current at the time of the definition.
% Thus, the following
% \begin{verbatim}
%    \theoremstyle{break}        \newtheorem{Cor}{Corollary}
%    \theoremstyle{plain}        \newtheorem{Exa}{Example}[section]
%\end{verbatim}                             ^^A No blank at line start
% leads to the result that the set {\envfont Cor} is formatted in the
% style \texttt{break}, while the set {\envfont Exa} and all the
% following ones are formatted in the style \texttt{plain}, unless
% another |\theoremstyle| follows.  Since the definitions installed by
% |\newtheorem| are global, one also can limit |\theoremstyle| locally
% by grouping braces.
%
% \DescribeMacro\theorembodyfont
% The choice of the font for the theorem body is completely
% independent of the chosen |\theoremstyle|; this has proven to
% be very advantageous.  For example,
% \begin{verbatim}
%   {\theorembodyfont{\upshape}          \newtheorem{Rem}{Remark}}
%\end{verbatim}
% defines a theorem set {\envfont Rem}, which will be set in
% |\upshape| in the current layout (which in our example is
% \texttt{plain}).  As with |\theoremstyle|, the |\theorembodyfont|
% chosen is that current at the time of |\newtheorem|.  If
% |\theorembodyfont| is not specified or one defines
% |\theorembodyfont{}|, then the font used will be that defined by the
% |\theoremstyle|.
%
% \DescribeMacro\theoremheaderfont It is also possible to customize
% the font used for the theorem headers.  This is, however, a global
% declaration, and therefore there should be at most one
% |\theoremheaderfont| declaration in the preamble.\footnote{If it is
% actually necessary to have different header fonts, one has to define
% new theorem styles (substituting the desired font) or specify the
% information directly in the \texttt{\bslash newtheorem} declaration
% (the unclean variant).}
%
% \DescribeMacro\theorempreskipamount
% \DescribeMacro\theorempostskipamount Two additional parameters
% affect the vertical space around the theorem environments:
% |\theorempreskipamount| and |\theorempostskipamount| define,
% respectively, the spacing before and after such an environment.
% These parameters apply for all theorem sets and can be manipulated
% with the ordinary length macros.  They are rubber lengths
% (`\textsf{skips}'), and therefore can contain \texttt{plus} and
% \texttt{minus} parts.
%
% Since the definition of theorem sets should---most sensibly---be
% placed in the preamble, we only allow installation there.
% It is therefore possible to release the memory used here after
% |\begin{document}|, in order to make room for other
% applications.
%
%
% \subsection{Existing theorem styles}
%
% The following theorem styles exist to date:
% \begin{deflist}{marginbreak}
%    \item[plain]
%       This theorem style emulates the original \LaTeX{} definition,
%       except that additionally the parameters
%       |\theorem...skipamount| are used.
%    \item[break]
%       In this style, the theorem header is followed by a line
%       break.
%    \item[marginbreak]
%       The theorem number is set in the margin, and there is a line
%       break as in \texttt{break}.
%    \item[changebreak]
%       Like \texttt{break}, but with header number and text
%       interchanged.
%    \item[change]
%       Header number and text are interchanged, without a line break.
%    \item[margin]
%       The number is set in the left margin, without a line break.
% \end{deflist}
% All styles (except \texttt{plain}) select |\slshape| as the default
% |\theorembodyfont|.
%
%
% \subsection{Examples}
%
% Given the above theorem sets {\envfont Cor}, {\envfont Exa}
% and {\envfont Rem},
% suppose that the preamble also contains the declarations:
% \begin{verbatim}
%    \theoremstyle{marginbreak}   \newtheorem{Lem}[Cor]{Lemma}
%    \theoremstyle{change}
%    \theorembodyfont{\itshape}        \newtheorem{Def}[Cor]{Definition}
%
%    \theoremheaderfont{\scshape}
%\end{verbatim}
% Then the following are some typical examples of the typeset output
% resulting from their use.
%
% \begin{Cor}
%    This is a sentence typeset in the theorem environment {\envfont
%    Cor}.
% \end{Cor}
%
% \begin{Exa}
%    This is a sentence typeset in the theorem environment {\envfont
%    Exa}.
% \end{Exa}
%
% \begin{Rem}
%    This is a sentence typeset in the theorem environment {\envfont
%    Rem}.
% \end{Rem}
%
% \begin{Lem}[Ben User]
%    This is a sentence typeset in the theorem environment {\envfont
%    Lem}.
% \end{Lem}
%
% \begin{Def}[Very impressive Definition]
%    This is a sentence typeset in the theorem environment {\envfont
%    Def}.
% \end{Def}
%
% The last two examples show the effect of the optional argument to
% a theorem environment (it is the text typeset in parentheses).
%
%
% \section{Special Considerations}
% \label{sec:newfonts}
%
% Theoremheader and body are implemented as a unit. This means that
% the |\theoremheaderfont|  will inherit characteristics of the
% |\theorembodyfont| in \LaTeXe. Thus, if for example
% |\theorembodyfont| is |\itshape| and |\theoremheaderfont|
% is |\bfseries| the font selected for the header will have the
% characteristics `bold extended italic'. If this is not desired one
% should set the |\theoremheaderfont| to something like
% \begin{verbatim}
%      \theoremheaderfont{\normalfont\bfseries}
%\end{verbatim}
% i.e.\ supplying all necessary font information explicitly.
%
%
%
% \section{Acknowledgements}
%
% The publication of this set of macros was only possible with the
% help of Christina Busse (translating the manuscript into English),
% Joachim Pense (playing the r\^ole of typist), Chris Rowley
% (looking everything over) and many others providing useful
% suggestions.
%
% \MaybeStop{%
%    \begin{thebibliography}{9}
%       \bibitem{bk:GMS94} \textsc{M.~Goossens}, \textsc{F.~Mittelbach}
%         and \textsc{A.~Samarin}.
%         \newblock The \LaTeX{} Companion.
%         \newblock
%         Addison-Wesley, Reading, Massachusetts, 1994.
%       \bibitem{latex:source} \textsc{Lamport, Leslie.} \newblock
%                               \texttt{latex.tex}, version 2.09, date
%                               Feb. 1990.
%    \end{thebibliography}
%    \PrintIndex
%    \PrintChanges}
%
%
% \section{The documentation driver file}
%
% The next bit of code contains the documentation driver file for
% \TeX{}, i.e., the file that will produce the documentation you are
% currently reading. It will be extracted from this file by the
% \texttt{docstrip} program. Since it is the first code in the file
% one can alternatively process this file directly with \LaTeXe{} to
% obtain the documentation.
%
%    \begin{macrocode}
%<*driver>
\documentclass{ltxdoc}

\usepackage{theorem}

%<+driver>% The next few lines define theorem sets which are used
%<+driver>% in the example section of the documentation.

\theoremstyle{break}       \newtheorem{Cor}{Corollary}
\theoremstyle{plain}       \newtheorem{Exa}{Example}[section]
{\theorembodyfont{\upshape}\newtheorem{Rem}{Remark}}
\theoremstyle{marginbreak} \newtheorem{Lem}[Cor]{Lemma}
\theoremstyle{change}
\theorembodyfont{\itshape} \newtheorem{Def}[Cor]{Definition}

\theoremheaderfont{\scshape}

\RecordChanges

\begin{document}
   \DocInput{theorem.dtx}
\end{document}
%</driver>
%    \end{macrocode}
%
% \section{Definition of the Macros}
%
% If the file has been loaded before, we abort immediately. If not the
% package announces itself (this is actually done at the very top of
% the file---the way it is done isn't good style so don't copy it).
% \changes{v2.0e}{89/07/19}{Spaces removed from `typeout'.}
% \changes{v2.2b}{95/11/19}{Announce moved to top of file}
%    \begin{macrocode}
%<*package>
%\@ifundefined{theorem@style}{}{\endinput}
%\def\FMithmInfo{1995/11/19 v2.2b Theorem extension package (FMi)}
%\ProvidesPackage{theorem}[\FMithmInfo]
%    \end{macrocode}
%
%
% \subsection{Definition of theorem styles and fonts}
%
% All the definitions in this file are done globally to allow
% inputting this file inside a group.
%
% \begin{macro}{\theoremstyle}
%    Before a theorem style can be installed, the chosen style must be
%    known. For that reason, we must test to see that
%    |\th@|\meta{style} is known or, more precisely, that it is
%    different from |\relax|. If the style is not known then
%    |\th@plain| is used.
%    \begin{macrocode}
\gdef\theoremstyle#1{%
   \@ifundefined{th@#1}{\@warning
          {Unknown theoremstyle `#1'. Using `plain'}%
          \theorem@style{plain}}%
%    \end{macrocode}
%    We save the theorem style to be used in the token register
%    |\theorem@style|.
%    \begin{macrocode}
      {\theorem@style{#1}}%
%    \end{macrocode}
%    Now we ``evaluate'' the theorem style: this means, we call the
%    macro |\th@|\meta{style} which will activate the relevant
%    definitions which are contained in a separate file.  This is done
%    in a group to suppress changes to the current font.  This could
%    otherwise pose problems together with the new font selection
%    scheme\footnote{When I printed the original article using the new
%    font selection scheme I ended with a document with slanted
%    typefaces (text headings and all) simply because one of the
%    theorem styles used \texttt{\bslash sl} at toplevel.} if the
%    |\th@|\meta{style} is evaluated a second time.
% \changes{v2.1a}{90/02/15}{Group added around `th@..}
% \changes{v2.1a}{90/02/15}{Reseting of `..@begintheorem removed.}
%    \begin{macrocode}
      \begingroup
        \csname th@\the\theorem@style \endcsname
      \endgroup}
%    \end{macrocode}
% \end{macro}
%
%
% \begin{macro}{\@begintheorem}
% \changes{v2.1a}{90/02/15}{Relaxed at toplevel.}
% \begin{macro}{\@opargbegintheorem}
% \changes{v2.1a}{90/02/15}{Relaxed at toplevel.}
%    We reset |\@begintheorem| and |\@opargbegintheorem| to
%    |\relax| since these commands are no longer necessary at
%    toplevel. This will save a few tokens.
%    \begin{macrocode}
\global\let\@begintheorem\relax
\global\let\@opargbegintheorem\relax
%    \end{macrocode}
% \end{macro}
% \end{macro}
%
% \begin{macro}{\theorem@style}
%    Obviously the \textsf{token} register used above has to be
%    allocated. To assure the utmost compatibility with the
%    original \LaTeX{} definition, we set the default theorem style to
%    \texttt{plain}, which implements the usual \LaTeX{} convention.
% \changes{v2.0f}{89/07/20}{Unnecessary `global' removed.}
%    \begin{macrocode}
\newtoks\theorem@style
\global\theorem@style{plain}
%    \end{macrocode}
% \end{macro}
%
% \begin{macro}{\theorembodyfont}
% \begin{macro}{\theorem@bodyfont}
%    If the |\theorembodyfont| is set by the user then it should not
%    interact with the default font set in the theorem style. When the
%    new font selection is in force this may happen if, for example,
%    the default is |\itshape| and the new |\theorembodyfont| is
%    |\sffamily|. So we add a |\reset@font| command in front of the
%    user definition.
% \changes{v2.0f}{89/07/20}{Unnecessary `global' removed.}
% \changes{v2.0g}{89/09/19}{`theoremfont to `theorembodyfont.}
% \changes{v2.1c}{92/06/22}{`reset@font added and logic changed.}
% \changes{v2.1d}{92/10/28}{Reset to empty if arg is empty}
%    \begin{macrocode}
\gdef\theorembodyfont#1{%
%    \end{macrocode}
%    We check if the argument supplied is empty and if so put nothing
%    into the |\theorem@bodyfont| token register to allow for
%    |\theorembodyfont{}| as a mean of using the default of the
%    current |\theoremstyle|.
%    \begin{macrocode}
   \def\@tempa{#1}%
   \ifx\@tempa\@empty
    \theorem@bodyfont{}%
   \else
    \theorem@bodyfont{\reset@font#1}%
   \fi
   }
\newtoks\theorem@bodyfont
\global\theorem@bodyfont{}
%    \end{macrocode}
% \end{macro}
% \end{macro}
%
%
% \begin{macro}{\theoremheaderfont}
% \changes{v2.0g}{89/09/19}{Macro added.}
%    The font for the theorem headers is handled differently because
%    this definition applies to all theorem styles.
%    \begin{macrocode}
\gdef\theoremheaderfont#1{\gdef\theorem@headerfont{#1}%
%    \end{macrocode}
%    After using the macro once it is redefined to produce an error
%    message.
%    \begin{macrocode}
       \gdef\theoremheaderfont##1{%
        \typeout{\string\theoremheaderfont\space should be used
                 only once.}}}
%    \end{macrocode}
% \end{macro}
%
% \begin{macro}{\theorem@headerfont}
% \changes{v2.0g}{89/09/19}{Macro added.}
%    To set the |\theorem@headerfont| default we first test if
%    the new fontselection scheme is in force.
% \changes{v2.1a}{90/02/15}{Support for new font selection.}
%    \begin{macrocode}
\ifx\upshape\undefined
%    \end{macrocode}
%    If not we define it to expand into |\bfseries|. We don't use
%    |\let| just in case a following style option redefines this
%    macro.
%    \begin{macrocode}
\gdef\theorem@headerfont{\bfseries}
%    \end{macrocode}
%    Otherwise we reset the current shape before calling |\bfseries|.
%    \begin{macrocode}
\else \gdef\theorem@headerfont{\normalfont\bfseries}\fi
%    \end{macrocode}
% \end{macro}
%
% \begin{macro}{\th@plain}
% \begin{macro}{\th@break}
% \begin{macro}{\th@marginbreak}
% \begin{macro}{\th@changebreak}
% \begin{macro}{\th@change}
% \begin{macro}{\th@margin}
%    The different styles are defined in macros such as
%    |\th@plain|.  Since memory space is precious in
%    ``non-Big-versions'', we have to avoid offering too many unused
%    definitions.  Therefore we define these styles in separate files
%    that can be loaded on demand.  Thus the commands themselves only
%    load these files. We use |\@input@|, a \LaTeXe{} internal command
%    that ensures that the file will be listed with |\listfiles|.
% \changes{v2.2a}{94/02/02}{Use `@input@ to load theorem layout files}
%    \begin{macrocode}
\gdef\th@plain{\@input@{thp.sty}}
\gdef\th@break{\@input@{thb.sty}}
\gdef\th@marginbreak{\@input@{thmb.sty}}
\gdef\th@changebreak{\@input@{thcb.sty}}
\gdef\th@change{\@input@{thc.sty}}
\gdef\th@margin{\@input@{thm.sty}}
%    \end{macrocode}
%    This list will be expanded when new styles become
%    available.  For testing,  just append new theorem substyles
%    as document options.
% \end{macro}
% \end{macro}
% \end{macro}
% \end{macro}
% \end{macro}
% \end{macro}
%
%
%
% \subsection{Definition of a new theorem set}
%
% As already pointed out, a new theorem environment can be defined in
% three different ways:
% \begin{verbatim}
%    \newtheorem{Lem}{Lemma}
%    \newtheorem{Lem}{Lemma}[section]
%    \newtheorem{Lem}[Theorem]{Lemma}
%\end{verbatim}
% The function of the macro |\newtheorem| is to recognize these cases
% and then to branch into one of the three macros |\@ynthm|, |\@xnthm|
% or |\@othm|.  This mechanism is adopted unchanged from
% \cite{latex:source}; the essential point here is that, for example,
% in the second case, the arguments \texttt{Lem}, \texttt{Lemma} and
% \texttt{section} are passed over to the macro |\@xnthm|.
%
% We inspect this case first because the others present fewer problems,
% and thus are easily derived from this one.
%
% \begin{macro}{\@xnthm}
%    For our example arguments, the macro |\@xnthm| must fulfill
%    the following:
%    \begin{itemize}
%       \item
%          Define a new \LaTeX-counter `Lem'
%       \item
%          reset this counter within a |\section|
%       \item
%          define the macro |\theLem|
%       \item
%          define the environment macros |\Lem| and
%          |\endLem| using the current
%          |\theoremstyle| and |\theorem@bodyfont|.
%    \end{itemize}
%    Obviously, all this should happen only if the first argument
%    of |\@xnthm| (i.e.\ \texttt{Lem} in our example) is chosen so
%    as not to conflict with any previously defined commands or
%    environments.  This test is performed by the \LaTeX{} macro
%    |\@ifdefinable|.
%    \begin{macrocode}
\gdef\@xnthm#1#2[#3]{\expandafter\@ifdefinable\csname #1\endcsname
%    \end{macrocode}
%    Therefore, the first argument of |\@ifdefinable| is the
%    expansion (in the example, |\Lem|) of
%    |\csname||#1\endcsname|.  The second argument is
%    executed only if the test has been completed successfully.
%    \begin{macrocode}
   {%
%    \end{macrocode}
%    Now we define the new counter. The names of the \LaTeX{}
%    macros employed should speak for themselves:
% \changes{v2.2b}{1995/10/07}{Use \cs{@newctr}}
% \changes{v2.2c}{1995/11/23}{Use \cs{@newctr} properly.}
%    \begin{macrocode}
    \@definecounter{#1}\@newctr{#1}[#3]%
%    \end{macrocode}
%    Using |\@newctr| will give a proper error message if the counter
%    in |#3| is not defined.
%    In defining `|\theLem|' we must generate the desired macro
%    name by use of |\expandafter| and |\csname|.
%    \begin{macrocode}
    \expandafter\xdef\csname the#1\endcsname
%    \end{macrocode}
%    An |\xdef| is used in order to make the definition global,
%    and to ensure that it contains the replacement texts of
%    |\@thmcountersep| and |\@thmcounter|.\footnote{These
%    two macros can be defined by the document style.  Their default
%    values produce a `\texttt{.}' as separation and an arabic
%    representation of the number.} However, not everything should be
%    expanded. For example, it saves space to use |\thesection|
%    instead of its---at times---lengthy expansion.
%    \begin{macrocode}
      {\expandafter \noexpand \csname the#3\endcsname
       \@thmcountersep \@thmcounter{#1}}%
%    \end{macrocode}
%    Thus with the defaults of \LaTeX{}, |\theLem| would be
%    replaced by the command sequence |\thesection.\arabic{Lem}|.
%
%    We will now look at the definition of the macro which is executed
%    at the beginning of the actual environment (in our example this
%    macro is |\Lem|). It should be noted that we use an
%    ``|\expandafter| trick'' to expand only certain parts of the
%    replacement text at the time of the definition.
%    \begin{macrocode}
    \def\@tempa{\global\@namedef{#1}}%
    \expandafter \@tempa \expandafter{%
%    \end{macrocode}
%    First, the macro that contains the current definitions of
%    |\@begintheorem| and |\@opargtheorem| should be called
%    up.  The name of this macro---as is already known---has the form
%    |\th@|\meta{theorem\ style}; therefore, it must be called by
%    \begin{macrocode}
      \csname th@\the \theorem@style
%    \end{macrocode}
%    In addition the default theorem font should be changeable, i.e.\
%    we have to insert the contents of |\theorem@bodyfont|. For
%    that reason, we expand even further, beyond |\endcsname|,
%    and thus insert the contents of the \textsf{token} register
%    |\theorem@bodyfont| in the replacement text.
%    \begin{macrocode}
            \expandafter \endcsname \the \theorem@bodyfont
%    \end{macrocode}
%    Now it is time to call the macro |\@thm| which takes over
%    the further processing.  It has two arguments: the current
%    counter name (in our example, \texttt{Lem}), and the text of the
%    label (in our example, \texttt{Lemma}).
%    \begin{macrocode}
     \@thm{#1}{#2}}%
%    \end{macrocode}
%    With this, the `sub-definition' is complete.  The macro
%    |\@endtheorem| ends a theorem environment and is, so far, nothing
%    but an |\endtrivlist|.  (Hence it is defined globally, and not
%    within the theorem styles.\footnote{This has to be changed as
%    soon as theorem styles that change \texttt{\bslash @endtheorem}
%    exist.  In such a case, all existing styles must be changed as
%    well since they will have to reset the macro.}) Therefore, we can
%    set it equivalent to the macro that ends the theorem set (in our
%    example, |\endLem|).  However, if some day theorem styles exist
%    that do change |\@endtheorem|, we would have to use the
%    commented-out line instead.
%    \begin{macrocode}
    \global \expandafter \let \csname end#1\endcsname \@endtheorem
%   \global\@namedef{end#1}{\@endtheorem}%
%    \end{macrocode}
%    With these commands all the required definitions are employed,
%    unless the test |\@ifdefinable| has failed.  Therefore, we
%    end the second argument of this macro and with it the definition
%    of |\@xnthm|.
%    \begin{macrocode}
   }}
%    \end{macrocode}
% \end{macro}
%
% \begin{macro}{\@ynthm}
%    The definition of |\@ynthm| is completely analogous. In
%    this case the new counter that is defined is not reset within
%    another counter; thus the definition of |\the...|
%    is simplified:
%    \begin{macrocode}
\gdef\@ynthm#1#2{\expandafter\@ifdefinable\csname #1\endcsname
   {\@definecounter{#1}%
    \expandafter\xdef\csname the#1\endcsname{\@thmcounter{#1}}%
%    \end{macrocode}
%    The rest of the definition corresponds literally to that of
%    |\@xnthm|:
%    \begin{macrocode}
    \def\@tempa{\global\@namedef{#1}}\expandafter \@tempa
     \expandafter{\csname th@\the \theorem@style \expandafter
     \endcsname \the\theorem@bodyfont \@thm{#1}{#2}}%
    \global \expandafter \let \csname end#1\endcsname \@endtheorem}}
%    \end{macrocode}
% \end{macro}
%
% \begin{macro}{\@othm}
%    The definition of |\@othm| does not contain anything new.
%    \begin{macrocode}
\gdef\@othm#1[#2]#3{%
%    \end{macrocode}
% \changes{v2.2b}{95/10/07}{Build in check for counter defined}
%    We do not define a new counter but instead use one that has
%    already been defined.  Thus the only definition we need is that
%    of this pseudo-counter (i.e.\ |\the|\meta{env.\ name}).
%    First we check if |#2| corresponds to a known counter name.
%    \begin{macrocode}
  \expandafter\ifx\csname c@#2\endcsname\relax
   \@nocounterr{#2}%
  \else
   \expandafter\@ifdefinable\csname #1\endcsname
   {\expandafter \xdef \csname the#1\endcsname
     {\expandafter \noexpand \csname the#2\endcsname}%
%    \end{macrocode}
%    All other parts of the definition can be adopted from
%    |\@xnthm|. We have to remember, though, that in this case
%    the name of the current counter and the theorem label have moved
%    to the second and third arguments.
%    \begin{macrocode}
    \def\@tempa{\global\@namedef{#1}}\expandafter \@tempa
     \expandafter{\csname th@\the \theorem@style \expandafter
     \endcsname \the\theorem@bodyfont \@thm{#2}{#3}}%
    \global \expandafter \let \csname end#1\endcsname \@endtheorem}%
  \fi}
%    \end{macrocode}
% \end{macro}
%
% \subsection{Macros that are employed in a {\envfont theorem}
%    environment}
%
% \begin{macro}{\@thm}
%    The macro |\@thm| has to increase the current counter.  Then,
%    depending on whether the environment has (or does not have) an
%    optional argument, it has to branch into either |\@begintheorem|
%    or |\@opargtheorem|.
%    \begin{macrocode}
\gdef\@thm#1#2{\refstepcounter{#1}%
%    \end{macrocode}
%    Now we start a {\envfont trivlist} environment, and give
%    |\@topsep| and |\@topsepadd| the values of the \textsf{skip}
%    registers |\theorempreskipamount| and |\theorempostskipamount|.
%    The value in |\@topsep| is the vertical space that is inserted by
%    the first (and only) |\item| in our |\trivlist| whilst
%    |\@topsepadd| is inserted by |\@endparenv| at the end of that
%    {\envfont trivlist} environment.  By using these registers, we
%    obtain the desired space around a {\envfont theorem} environment.
%    \begin{macrocode}
   \trivlist
   \@topsep \theorempreskipamount               % used by first \item
   \@topsepadd \theorempostskipamount           % used by \@endparenv
%    \end{macrocode}
%    Now we have to test whether an optional argument has been given.
% \changes{v2.0d}{89/07/18}{Forgotten percent added.}
%    \begin{macrocode}
   \@ifnextchar [%
%    \end{macrocode}
%    If there is an optional argument, we will call |\@ythm|, and
%    move the arguments read back into the input stream.
%    \begin{macrocode}
   {\@ythm{#1}{#2}}%
%    \end{macrocode}
%    If not, we call |\@begintheorem|. Its first argument is the
%    name of the theorem set (hence the second argument of
%    |\@thm|). Its second argument is the macro that produces the
%    current number.
%    \begin{macrocode}
   {\@begintheorem{#2}{\csname the#1\endcsname}\ignorespaces}}
%    \end{macrocode}
% \end{macro}
%
% \begin{macro}{\@xthm}
% \begin{macro}{\@ythm}
%    Both these macros were originally called by |\@thm|.  We do
%    not need |\@xthm| anymore, hence we reset it to
%    |\relax|. The definition of |\@ythm| has not changed at
%    all from its definition in \LaTeX. In order to make the macros
%    easier to understand, we will
%    nevertheless present it (commented out).
%    \begin{macrocode}
\global\let\@xthm\relax
% \def\@ythm#1#2[#3]{\@opargbegintheorem{#2}{\csname
%           the#1\endcsname}{#3}\ignorespaces}
%</package>
%    \end{macrocode}
%    The primitive |\ignorespaces| in |\@ythm| and
%    |\@thm| is needed to remove the spaces between the
%    |\begin{...}| and the actual text.
% \end{macro}
% \end{macro}
%
%
% \subsection{Definition of the theorem substyles}
%
% As already pointed out, the theorem substyles, defined below, are
% only loaded when necessary.  Note that all these substyles, except
% \texttt{plain}, have |\slshape| as the default body font.
%
% \subsubsection{The \texttt{plain} style}
%
% \label{dok:th@plain} As the following macros use |@|, we have
% to locally set the |\catcode| of this symbol to ``letter''.
% This happens within a group, so that we do not have to worry about
% which |\catcode| that symbol had before.
%    \begin{macrocode}
%<*thp>
\begingroup \makeatletter
%    \end{macrocode}
% Since we are now within a group, we must make all definitions
% globally.  First we make sure that \texttt{theorem.sty} is loaded.
% This will allow us to use this file as a document style option
% without having to call \texttt{theorem} itself as an option.  At the
% same time, we assure that at least version 2 is loaded, since
% |\theorem@style| was not defined in earlier versions.
%    \begin{macrocode}
\@ifundefined{theorem@style}{\input{theorem.sty}}{}
\ProvidesFile{thp.sty}
             [\FMithmInfo]
%    \end{macrocode}
%
%
% \begin{macro}{\th@plain}
% \begin{macro}{\@begintheorem}
% \begin{macro}{\@opargbegintheorem}
%    |\theoremstyle{plain}| corresponds to the original
%    definition, except that the distances to the surrounding text are
%    determined by the parameters |\theorempreskipamount| and
%    |\theorempostskipamount|.
%    First we set the default body font.
%    \begin{macrocode}
\gdef\th@plain{\normalfont\itshape
%    \end{macrocode}
%    Then we define |\@begintheorem| and
%    |\@opargbegintheorem|. These two macros define how the
%    header of a theorem is typeset.  |\@opargbegintheorem| will
%    be called if a \textsf{theorem} environment with an optional
%    argument is encountered; otherwise, the header is constructed by
%    calling |\@begintheorem|.  If one of these macros is
%    executed, we are within a \textsf{trivlist} environment started by
%    |\@thm|.  So the theorem header is produced with an
%    |\item| command.
%
%    Instead of specifying the header font directly, all standard
%    theorem styles use the |\theorem@headerfont| macro to allow
%    customization.  The extra space (|\labelsep|) is necessary
%    because of problems in the \textsf{trivlist} environment.
%    \begin{macrocode}
  \def\@begintheorem##1##2{%
        \item[\hskip\labelsep \theorem@headerfont ##1\ ##2]}%
%    \end{macrocode}
%    The definition of |\@opargbegintheorem| is completely
%    analogous. The only difference is the fact that there exists a
%    third argument (which is the optional parameter of the
%    environment and contains additional information about the
%    theorem).  Customarily we enclose it in parentheses.
%    \begin{macrocode}
\def\@opargbegintheorem##1##2##3{%
   \item[\hskip\labelsep \theorem@headerfont ##1\ ##2\ (##3)]}}
%    \end{macrocode}
% \end{macro}
% \end{macro}
% \end{macro}
%
% \noindent
% We conclude with an |\endgroup| to restore the
% |\catcode| of |@|.
%    \begin{macrocode}
\endgroup
%</thp>
%    \end{macrocode}
%
%
% \subsubsection{The \texttt{break} style}
%
% This style option is stored in the file |thb.sty|.
% For the next two lines see the documentation for |\th@plain| on
% page~\pageref{dok:th@plain}.
%    \begin{macrocode}
%<*thb>
\begingroup \makeatletter
\@ifundefined{theorem@style}{\input{theorem.sty}}{}
\ProvidesFile{thb.sty}
             [\FMithmInfo]
%    \end{macrocode}
%
% \begin{macro}{\th@break}
%    |\theoremstyle{break}| produces a line break after the name
%    of the theorem.  The font is |\slshape|.  Hence, we define
%    |\th@break| as follows:
%    \begin{macrocode}
\gdef\th@break{\normalfont\slshape
  \def\@begintheorem##1##2{\item[%
%    \end{macrocode}
%    We run into the following problem: it is not possible to create
%    the header with |\item[|\meta{title}|]| and then start
%    a new line by, for example, |\mbox{}\\|. Such a definition
%    will fail whenever a list environment follows immediately.  With
%    the above construction, the |\mbox{}| causes the switch
%    |@inlabel| (cf.\ definition of |\list| and
%    |\trivlist| in \cite{latex:source}) to be set to \texttt{false}
%    and so the following list will insert additional vertical
%    space (|\topskip|). This is quite annoying. Therefore, we
%    create the line break within the |\item|. In order to ensure
%    that the text will begin at the proper position in the following
%    line, we simply pretend that the label does not take any
%    room.\footnote{This will lead to problems whenever very high
%    symbols occurring in the line tower into the
%    heading. So, something else has to be done here sometime.}
%    \begin{macrocode}
       \rlap{\vbox{\hbox{\hskip \labelsep\theorem@headerfont ##1\ ##2}%
                   \hbox{\strut}}}]}%
%    \end{macrocode}
%    Again, the definition of |\@opargbegintheorem| is
%    completely analogous.
%    \begin{macrocode}
\def\@opargbegintheorem##1##2##3{%
  \item[\rlap{\vbox{\hbox{\hskip \labelsep \theorem@headerfont
                     ##1\ ##2\ (##3)}%
                    \hbox{\strut}}}]}}
\endgroup
%</thb>
%    \end{macrocode}
% \end{macro}
%
%
% \subsubsection{The \texttt{changebreak} style}
%
%    \begin{macrocode}
% This style option is stored in the file |thcb.sty|.
%    \begin{macrocode}
%<*thcb>
\begingroup \makeatletter
\@ifundefined{theorem@style}{\input{theorem.sty}}{}
\ProvidesFile{thcb.sty}
             [\FMithmInfo]
%    \end{macrocode}
%
% \begin{macro}{\th@changebreak}
%    The \texttt{change-break} theorem style is like \texttt{break}
%    but with interchange of theorem name and theorem number.  Thus we
%    define |\th@changebreak| as follows:
%    \begin{macrocode}
\gdef\th@changebreak{\normalfont\slshape
  \def\@begintheorem##1##2{\item
       [\rlap{\vbox{\hbox{\hskip\labelsep \theorem@headerfont ##2\ ##1}%
                    \hbox{\strut}}}]}%
\def\@opargbegintheorem##1##2##3{%
  \item[\rlap{\vbox{\hbox{\hskip\labelsep \theorem@headerfont
                     ##2\ ##1\ (##3)}%
                    \hbox{\strut}}}]}}
\endgroup
%</thcb>
%    \end{macrocode}
% \end{macro}
%
%
% \subsubsection{The \texttt{change} style}
%
% This style option is stored in the file |thc.sty|.
%    \begin{macrocode}
%<*thc>
\begingroup \makeatletter
\@ifundefined{theorem@style}{\input{theorem.sty}}{}
\ProvidesFile{thc.sty}
             [\FMithmInfo]
%    \end{macrocode}
%
% \begin{macro}{\th@change}
%    The \texttt{change} theorem style corresponds to the
%    \texttt{change break} style without a linebreak after the header.
%    To say it in another way, it's the same as the \texttt{plain}
%    style but with number and name interchanged and |\slshape| as the
%    default font.
%    \begin{macrocode}
\gdef\th@change{\normalfont\slshape
  \def\@begintheorem##1##2{\item
       [\hskip\labelsep \theorem@headerfont ##2\ ##1]}%
\def\@opargbegintheorem##1##2##3{%
  \item[\hskip\labelsep \theorem@headerfont ##2\ ##1\ (##3)]}}
\endgroup
%</thc>
%    \end{macrocode}
% \end{macro}
%
%
% \subsubsection{The \texttt{marginbreak} style}
%
% This style option is the one used most often at Mainz.  It is saved
% in the file |thmb.sty|.
%    \begin{macrocode}
%<*thmb>
\begingroup \makeatletter
\@ifundefined{theorem@style}{\input{theorem.sty}}{}
\ProvidesFile{thmb.sty}
             [\FMithmInfo]
%    \end{macrocode}
%
% \begin{macro}{\th@marginbreak}
%    The \texttt{margin break} style is nearly the same as the
%    \texttt{change break} style. The only difference is the placement
%    of the theorem number.  We use |\llap| to place it in the left
%    margin.
%
%    In this style |\labelsep| denotes the separation between the
%    number and the text.
%    \begin{macrocode}
\gdef\th@marginbreak{\normalfont\slshape
  \def\@begintheorem##1##2{\item
       [\rlap{\vbox{\theorem@headerfont
                    \hbox{\llap{##2}\hskip\labelsep ##1}%
                    \hbox{\strut}}}]}%
\def\@opargbegintheorem##1##2##3{%
  \item[\rlap{\vbox{\theorem@headerfont
                    \hbox{\llap{##2}\hskip\labelsep ##1\ (##3)}%
                    \hbox{\strut}}}]}}
\endgroup
%</thmb>
%    \end{macrocode}
% \changes{v2.1c}{90/02/17}{`endgroup outside macrocode env.}
% \end{macro}
%
%
%
% \subsubsection{The \texttt{margin} style}
%
% This style option is stored in the file |thm.sty|.
%    \begin{macrocode}
%<*thm>
\begingroup \makeatletter
\@ifundefined{theorem@style}{\input{theorem.sty}}{}
\ProvidesFile{thm.sty}
             [\FMithmInfo]
%    \end{macrocode}
%
% \begin{macro}{\th@margin}
%    Again this is only a variant of the theorem styles described
%    above without any new ideas.
%    \begin{macrocode}
\gdef\th@margin{\normalfont\slshape
  \def\@begintheorem##1##2{\item
       [\theorem@headerfont \llap{##2}\hskip\labelsep ##1]}%
\def\@opargbegintheorem##1##2##3{%
  \item[\theorem@headerfont \llap{##2}\hskip\labelsep ##1\ (##3)]}}
\endgroup
%</thm>
%    \end{macrocode}
% \end{macro}
%
%
%
%
% \subsection{Final Definitions}
%
% \begin{macro}{\theorempreskipamount}
% \begin{macro}{\theorempostskipamount}
%    The \textsf{skip} parameters that regulate the vertical empty space
%    before and after the {\envfont theorem} environment have
%    to be allocated as well.
% \changes{v2.0f}{89/07/20}{Unnecessary `global' removed.}
%    \begin{macrocode}
%<*package>
\newskip\theorempreskipamount
\newskip\theorempostskipamount
%    \end{macrocode}
%    Since we have used the same values for all theorem sets, we now can
%    assign them.
%    \begin{macrocode}
\global\setlength\theorempreskipamount{12pt plus 5pt minus 3pt}
\global\setlength\theorempostskipamount{8pt plus 3pt minus 1.5pt}
%    \end{macrocode}
% \end{macro}
% \end{macro}
%
%
% \begin{macro}{\@endtheorem}
%    The same holds for the macro |\@endtheorem|, which ends a
%    {\envfont theorem} environment. Since it is the same for all
%    theorem sets, it is removed from the macros |\th@|\meta{style}.
%    It simply ends the {\envfont trivlist} environment, which was
%    begun in |\@thm|.
%    \begin{macrocode}
\global\let\@endtheorem=\endtrivlist
%    \end{macrocode}
% \end{macro}
%
%
%
% \begin{macro}{\@preamblecmds}
%    All macros defined above are to be used only in the preamble.
%    Therefore, we insert them in |\@preamblecmds| which will disable
%    them at begin document. This is done by the internal \LaTeXe{}
%    command |\@onlypreamble|.
% \changes{v2.0g}{89/09/19}{`theoremheaderfont and `theorembodyfont
%    added.}
% \changes{v2.2a}{94/02/03}{Use \LaTeXe{} interface}
%    \begin{macrocode}
\@onlypreamble\@xnthm
\@onlypreamble\@ynthm
\@onlypreamble\@othm
\@onlypreamble\newtheorem
\@onlypreamble\theoremstyle
\@onlypreamble\theorembodyfont
\@onlypreamble\theoremheaderfont
%    \end{macrocode}
% \end{macro}
%
% Finally we declare the \texttt{plain} theorem style to be the default.
% \changes{v2.1b}{90/06/04}{Fixed missing `item by declaring default
%    style.}
%    \begin{macrocode}
\theoremstyle{plain}
%</package>
%    \end{macrocode}
%
% \Finale

\endinput
}{}
\ProvidesFile{thmb.sty}
             [\FMithmInfo]
%    \end{macrocode}
%
% \begin{macro}{\th@marginbreak}
%    The \texttt{margin break} style is nearly the same as the
%    \texttt{change break} style. The only difference is the placement
%    of the theorem number.  We use |\llap| to place it in the left
%    margin.
%
%    In this style |\labelsep| denotes the separation between the
%    number and the text.
%    \begin{macrocode}
\gdef\th@marginbreak{\normalfont\slshape
  \def\@begintheorem##1##2{\item
       [\rlap{\vbox{\theorem@headerfont
                    \hbox{\llap{##2}\hskip\labelsep ##1}%
                    \hbox{\strut}}}]}%
\def\@opargbegintheorem##1##2##3{%
  \item[\rlap{\vbox{\theorem@headerfont
                    \hbox{\llap{##2}\hskip\labelsep ##1\ (##3)}%
                    \hbox{\strut}}}]}}
\endgroup
%</thmb>
%    \end{macrocode}
% \changes{v2.1c}{90/02/17}{`endgroup outside macrocode env.}
% \end{macro}
%
%
%
% \subsubsection{The \texttt{margin} style}
%
% This style option is stored in the file |thm.sty|.
%    \begin{macrocode}
%<*thm>
\begingroup \makeatletter
\@ifundefined{theorem@style}{% \iffalse meta-comment
%
% Copyright (C) 1993-2024
%
% The LaTeX Project and any individual authors listed elsewhere
% in this file.
%
% This file is part of the Standard LaTeX `Tools Bundle'.
% -------------------------------------------------------
%
% It may be distributed and/or modified under the
% conditions of the LaTeX Project Public License, either version 1.3c
% of this license or (at your option) any later version.
% The latest version of this license is in
%    https://www.latex-project.org/lppl.txt
% and version 1.3c or later is part of all distributions of LaTeX
% version 2005/12/01 or later.
%
% The list of all files belonging to the LaTeX `Tools Bundle' is
% given in the file `manifest.txt'.
%
% \fi
%
%%
%
%
\def\FMithmInfo{2023/07/05 v2.2c Theorem extension package (FMi)}
%
% \ProvidesFile{theorem.dtx}[\FMithmInfo]
%\iffalse   % this is a METACOMMENT !
%%
%% Package `theorem' to use with LaTeX2e
%% Copyright (C) 1989-1995 Frank Mittelbach, all rights reserved.
%<+package>\@ifundefined{theorem@style}{}{\endinput}
%<+package>\NeedsTeXFormat{LaTeX2e}[1995/06/01]
%<+package>\ProvidesPackage{theorem}
%<+package>         [\FMithmInfo]
%
% \fi
%
% \changes{v1.0}{87/05/07}{First draft.}
% \changes{v1.0a}{87/05/07}{`thistheoremstyle' added.}
% \changes{v1.1a}{87/07/29}{New concept: redefine the internal
%                          \protect\LaTeX{} commands `@xnthm' etc.}
% \changes{v1.1b}{87/07/31}{`theoremfont' used to specify font.}
% \changes{v1.2a}{88/01/08}{`@endtheorem' now globally defined.}
% \changes{v1.2b}{88/01/11}{`newtheorem' only allowed in the preamble.}
% \changes{v1.3a}{88/02/14}{`theoremfont' now a toks register.}
%
% \changes{v2.0a}{88/11/16}{New implementation and new documentation.}
% \changes{v2.0a}{88/11/16}{No layout changes but internal modifications
%                           old substyles won't work.}
% \changes{v2.0b}{89/01/30}{`global' used for all definitions.}
% \changes{v2.0c}{89/06/15}{Documentation translated to English.
%                            Thanks to LLL and Nina}
% \changes{v2.1a}{90/02/15}{Support for new font selection.}
% \changes{v2.2a}{94/02/03}{Cleanup for LaTeX2e}
%
% \DoNotIndex{\@endparenv,\@ifdefinable,\@ifnextchar,\@ifundefined}
% \DoNotIndex{\@namedef,\@nameuse,\@spaces,\@tempa,\@warning}
% \DoNotIndex{\ }
% \DoNotIndex{\begingroup,\bfseries}
% \DoNotIndex{\csname}
% \DoNotIndex{\def,\do,\docdate}
% \DoNotIndex{\endcsname,\endgroup,\endinput,\endtrivlist,\expandafter}
% \DoNotIndex{\filedate,\fileversion}
% \DoNotIndex{\gdef,\global}
% \DoNotIndex{\hbox,\hskip}
% \DoNotIndex{\ignorespaces,\input,\itshape,\item}
% \DoNotIndex{\labelsep,\let,\llap}
% \DoNotIndex{\makeatletter}
% \DoNotIndex{\noexpand}
% \DoNotIndex{\refstepcounter,\relax,\rlap}
% \DoNotIndex{\setlength,\slshape,\space,\string,\strut}
% \DoNotIndex{\the,\trivlist,\typeout}
% \DoNotIndex{\vbox}
% \DoNotIndex{\xdef}
%
% \def\deflabel#1{\ttfamily #1\hfill}
% \def\deflist#1{\begin{list}{}{\settowidth\labelwidth{\ttfamily #1}%
%                               \setlength\leftmargin\labelwidth
%                               \addtolength\leftmargin\labelsep
%                               \let\makelabel\deflabel}}
% \def\enddeflist{\end{list}}
% \def\envfont{\normalfont\ttfamily}
%
% \MakeShortVerb{\|}
%
% \newcommand{\TUB}{{\sl TUGboat\/}}
%
% \GetFileInfo{theorem.dtx}
% \title{An Extension of the
%        \LaTeX{} theorem environment\thanks{This file
%        has version number \fileversion, last
%        revised \filedate.}}
% \author{Frank Mittelbach}
%
% \MaintainedByLaTeXTeam{tools}
% \maketitle
%
% \begin{abstract}
%    The macros described in this paper yield an extension of the
%    \LaTeX{} theorem mechanism. It is designed to satisfy the
%    different requirements of various journals. Thus, the
%    layout of the ``theorems'' can be manipulated by determining a
%    ``style''. This article describes not only the use, but
%    also the definition, of the necessary macros.
% \end{abstract}
%
% \section*{Preface to version 2.2}
%
% For \LaTeXe{} this package did not need any fundamental changes. I
% only modified the messages generated so that theorem layout styles
% will show up with the |\listfiles| command and cleaned the section
% on the New Font Selection Scheme since this is now included in
% \LaTeX.
%
% \section*{Preface to version 2.1}
%
% This version is identical to 2.0g described in \TUB\ 10\#3 except
% for some internal defaults which are now set depending on the used
% font selection scheme.
%
% This was done to avoid unpleasant surprises if the new font selection
% scheme is in force. For further details see
% section~\ref{sec:newfonts} and \cite{bk:GMS94}.
%
% \section{Introduction}
%
% For our purposes here, ``theorems'' are labelled enunciations,
% often set off from the main text by extra space and a font change.
% Theorems, corollaries, conjectures, definitions, and remarks
% are all instances of ``theorems''.  The ``header'' of these structures
% is composed of a label (such as \textsc{Theorem} or \textsc{Remark})
% and a number which serializes an item in the sequence of items
% with the same label.
%
% Shortly after the introduction of \LaTeX{} at the Fachbereich
% Mathematik in Mainz, the desire to manipulate the layout of
% ``theorems'' arose.  In Mainz, the following two conventions came
% into general use:
% \begin{enumerate}
%   \item The number of the theorem is shown in the margin.
%   \item There is a line break at
%         the end of the theorem header.
% \end{enumerate}
%
% Additionally, some journals require different formats which depend on
% the ``sort of theorem'': e.g.\ often remarks and definitions are set
% in |\upshape|, while |\itshape| is employed for main theorems.
%
% Confronted with these requirements, a theorem environment was
% developed in Mainz which allows separate determination of the
% layout of the ``theorems sets'', comparable to |\pagestyle|.
%
%
% \section{The user interface}
% \subsection{Defining new theorem sets}
%
% \DescribeMacro\newtheorem
% As in the original \LaTeX{} version, the command |\newtheorem|
% defines a new ``theorem set'' or ``theorem-like structure''.
% Two required arguments name the new environment and give the text
% to be typeset with each instance of the new ``set'', while
% an optional argument determines how
% the ``set'' is enumerated:
% \begin{description}
%    \item[\ttfamily \bslash newtheorem\{foo\}\{bar\}]
%       The theorem set {\envfont foo} (whose name is \texttt{bar})
%       uses its own counter.
%    \item[\ttfamily \bslash newtheorem\{foo2\}{[foo]}\{bar2\}]
%       The theorem set {\envfont foo2} (printed name \texttt{bar2})
%       uses the same counter as the theorem set \texttt{foo}.
%    \item[\ttfamily \bslash newtheorem\{foo3\}\{bar3\}{[section]}]
%       The theorem set {\envfont foo3} (printed name \texttt{bar3}) is
%       enumerated within the counter \texttt{section}, i.e.\ with every
%       new |\section| the enumeration begins again with 1, and
%       the enumeration is composed from the section-number and the
%       theorem counter itself.
% \end{description}
%
% \DescribeMacro\theoremstyle
% Additionally, the command |\theoremstyle| can define the layout
% of various, or all, theorem sets.  It should be noted that any
% theorem set defined by |\newtheorem| is typeset in the
% |\theoremstyle| that is current at the time of the definition.
% Thus, the following
% \begin{verbatim}
%    \theoremstyle{break}        \newtheorem{Cor}{Corollary}
%    \theoremstyle{plain}        \newtheorem{Exa}{Example}[section]
%\end{verbatim}                             ^^A No blank at line start
% leads to the result that the set {\envfont Cor} is formatted in the
% style \texttt{break}, while the set {\envfont Exa} and all the
% following ones are formatted in the style \texttt{plain}, unless
% another |\theoremstyle| follows.  Since the definitions installed by
% |\newtheorem| are global, one also can limit |\theoremstyle| locally
% by grouping braces.
%
% \DescribeMacro\theorembodyfont
% The choice of the font for the theorem body is completely
% independent of the chosen |\theoremstyle|; this has proven to
% be very advantageous.  For example,
% \begin{verbatim}
%   {\theorembodyfont{\upshape}          \newtheorem{Rem}{Remark}}
%\end{verbatim}
% defines a theorem set {\envfont Rem}, which will be set in
% |\upshape| in the current layout (which in our example is
% \texttt{plain}).  As with |\theoremstyle|, the |\theorembodyfont|
% chosen is that current at the time of |\newtheorem|.  If
% |\theorembodyfont| is not specified or one defines
% |\theorembodyfont{}|, then the font used will be that defined by the
% |\theoremstyle|.
%
% \DescribeMacro\theoremheaderfont It is also possible to customize
% the font used for the theorem headers.  This is, however, a global
% declaration, and therefore there should be at most one
% |\theoremheaderfont| declaration in the preamble.\footnote{If it is
% actually necessary to have different header fonts, one has to define
% new theorem styles (substituting the desired font) or specify the
% information directly in the \texttt{\bslash newtheorem} declaration
% (the unclean variant).}
%
% \DescribeMacro\theorempreskipamount
% \DescribeMacro\theorempostskipamount Two additional parameters
% affect the vertical space around the theorem environments:
% |\theorempreskipamount| and |\theorempostskipamount| define,
% respectively, the spacing before and after such an environment.
% These parameters apply for all theorem sets and can be manipulated
% with the ordinary length macros.  They are rubber lengths
% (`\textsf{skips}'), and therefore can contain \texttt{plus} and
% \texttt{minus} parts.
%
% Since the definition of theorem sets should---most sensibly---be
% placed in the preamble, we only allow installation there.
% It is therefore possible to release the memory used here after
% |\begin{document}|, in order to make room for other
% applications.
%
%
% \subsection{Existing theorem styles}
%
% The following theorem styles exist to date:
% \begin{deflist}{marginbreak}
%    \item[plain]
%       This theorem style emulates the original \LaTeX{} definition,
%       except that additionally the parameters
%       |\theorem...skipamount| are used.
%    \item[break]
%       In this style, the theorem header is followed by a line
%       break.
%    \item[marginbreak]
%       The theorem number is set in the margin, and there is a line
%       break as in \texttt{break}.
%    \item[changebreak]
%       Like \texttt{break}, but with header number and text
%       interchanged.
%    \item[change]
%       Header number and text are interchanged, without a line break.
%    \item[margin]
%       The number is set in the left margin, without a line break.
% \end{deflist}
% All styles (except \texttt{plain}) select |\slshape| as the default
% |\theorembodyfont|.
%
%
% \subsection{Examples}
%
% Given the above theorem sets {\envfont Cor}, {\envfont Exa}
% and {\envfont Rem},
% suppose that the preamble also contains the declarations:
% \begin{verbatim}
%    \theoremstyle{marginbreak}   \newtheorem{Lem}[Cor]{Lemma}
%    \theoremstyle{change}
%    \theorembodyfont{\itshape}        \newtheorem{Def}[Cor]{Definition}
%
%    \theoremheaderfont{\scshape}
%\end{verbatim}
% Then the following are some typical examples of the typeset output
% resulting from their use.
%
% \begin{Cor}
%    This is a sentence typeset in the theorem environment {\envfont
%    Cor}.
% \end{Cor}
%
% \begin{Exa}
%    This is a sentence typeset in the theorem environment {\envfont
%    Exa}.
% \end{Exa}
%
% \begin{Rem}
%    This is a sentence typeset in the theorem environment {\envfont
%    Rem}.
% \end{Rem}
%
% \begin{Lem}[Ben User]
%    This is a sentence typeset in the theorem environment {\envfont
%    Lem}.
% \end{Lem}
%
% \begin{Def}[Very impressive Definition]
%    This is a sentence typeset in the theorem environment {\envfont
%    Def}.
% \end{Def}
%
% The last two examples show the effect of the optional argument to
% a theorem environment (it is the text typeset in parentheses).
%
%
% \section{Special Considerations}
% \label{sec:newfonts}
%
% Theoremheader and body are implemented as a unit. This means that
% the |\theoremheaderfont|  will inherit characteristics of the
% |\theorembodyfont| in \LaTeXe. Thus, if for example
% |\theorembodyfont| is |\itshape| and |\theoremheaderfont|
% is |\bfseries| the font selected for the header will have the
% characteristics `bold extended italic'. If this is not desired one
% should set the |\theoremheaderfont| to something like
% \begin{verbatim}
%      \theoremheaderfont{\normalfont\bfseries}
%\end{verbatim}
% i.e.\ supplying all necessary font information explicitly.
%
%
%
% \section{Acknowledgements}
%
% The publication of this set of macros was only possible with the
% help of Christina Busse (translating the manuscript into English),
% Joachim Pense (playing the r\^ole of typist), Chris Rowley
% (looking everything over) and many others providing useful
% suggestions.
%
% \MaybeStop{%
%    \begin{thebibliography}{9}
%       \bibitem{bk:GMS94} \textsc{M.~Goossens}, \textsc{F.~Mittelbach}
%         and \textsc{A.~Samarin}.
%         \newblock The \LaTeX{} Companion.
%         \newblock
%         Addison-Wesley, Reading, Massachusetts, 1994.
%       \bibitem{latex:source} \textsc{Lamport, Leslie.} \newblock
%                               \texttt{latex.tex}, version 2.09, date
%                               Feb. 1990.
%    \end{thebibliography}
%    \PrintIndex
%    \PrintChanges}
%
%
% \section{The documentation driver file}
%
% The next bit of code contains the documentation driver file for
% \TeX{}, i.e., the file that will produce the documentation you are
% currently reading. It will be extracted from this file by the
% \texttt{docstrip} program. Since it is the first code in the file
% one can alternatively process this file directly with \LaTeXe{} to
% obtain the documentation.
%
%    \begin{macrocode}
%<*driver>
\documentclass{ltxdoc}

\usepackage{theorem}

%<+driver>% The next few lines define theorem sets which are used
%<+driver>% in the example section of the documentation.

\theoremstyle{break}       \newtheorem{Cor}{Corollary}
\theoremstyle{plain}       \newtheorem{Exa}{Example}[section]
{\theorembodyfont{\upshape}\newtheorem{Rem}{Remark}}
\theoremstyle{marginbreak} \newtheorem{Lem}[Cor]{Lemma}
\theoremstyle{change}
\theorembodyfont{\itshape} \newtheorem{Def}[Cor]{Definition}

\theoremheaderfont{\scshape}

\RecordChanges

\begin{document}
   \DocInput{theorem.dtx}
\end{document}
%</driver>
%    \end{macrocode}
%
% \section{Definition of the Macros}
%
% If the file has been loaded before, we abort immediately. If not the
% package announces itself (this is actually done at the very top of
% the file---the way it is done isn't good style so don't copy it).
% \changes{v2.0e}{89/07/19}{Spaces removed from `typeout'.}
% \changes{v2.2b}{95/11/19}{Announce moved to top of file}
%    \begin{macrocode}
%<*package>
%\@ifundefined{theorem@style}{}{\endinput}
%\def\FMithmInfo{1995/11/19 v2.2b Theorem extension package (FMi)}
%\ProvidesPackage{theorem}[\FMithmInfo]
%    \end{macrocode}
%
%
% \subsection{Definition of theorem styles and fonts}
%
% All the definitions in this file are done globally to allow
% inputting this file inside a group.
%
% \begin{macro}{\theoremstyle}
%    Before a theorem style can be installed, the chosen style must be
%    known. For that reason, we must test to see that
%    |\th@|\meta{style} is known or, more precisely, that it is
%    different from |\relax|. If the style is not known then
%    |\th@plain| is used.
%    \begin{macrocode}
\gdef\theoremstyle#1{%
   \@ifundefined{th@#1}{\@warning
          {Unknown theoremstyle `#1'. Using `plain'}%
          \theorem@style{plain}}%
%    \end{macrocode}
%    We save the theorem style to be used in the token register
%    |\theorem@style|.
%    \begin{macrocode}
      {\theorem@style{#1}}%
%    \end{macrocode}
%    Now we ``evaluate'' the theorem style: this means, we call the
%    macro |\th@|\meta{style} which will activate the relevant
%    definitions which are contained in a separate file.  This is done
%    in a group to suppress changes to the current font.  This could
%    otherwise pose problems together with the new font selection
%    scheme\footnote{When I printed the original article using the new
%    font selection scheme I ended with a document with slanted
%    typefaces (text headings and all) simply because one of the
%    theorem styles used \texttt{\bslash sl} at toplevel.} if the
%    |\th@|\meta{style} is evaluated a second time.
% \changes{v2.1a}{90/02/15}{Group added around `th@..}
% \changes{v2.1a}{90/02/15}{Reseting of `..@begintheorem removed.}
%    \begin{macrocode}
      \begingroup
        \csname th@\the\theorem@style \endcsname
      \endgroup}
%    \end{macrocode}
% \end{macro}
%
%
% \begin{macro}{\@begintheorem}
% \changes{v2.1a}{90/02/15}{Relaxed at toplevel.}
% \begin{macro}{\@opargbegintheorem}
% \changes{v2.1a}{90/02/15}{Relaxed at toplevel.}
%    We reset |\@begintheorem| and |\@opargbegintheorem| to
%    |\relax| since these commands are no longer necessary at
%    toplevel. This will save a few tokens.
%    \begin{macrocode}
\global\let\@begintheorem\relax
\global\let\@opargbegintheorem\relax
%    \end{macrocode}
% \end{macro}
% \end{macro}
%
% \begin{macro}{\theorem@style}
%    Obviously the \textsf{token} register used above has to be
%    allocated. To assure the utmost compatibility with the
%    original \LaTeX{} definition, we set the default theorem style to
%    \texttt{plain}, which implements the usual \LaTeX{} convention.
% \changes{v2.0f}{89/07/20}{Unnecessary `global' removed.}
%    \begin{macrocode}
\newtoks\theorem@style
\global\theorem@style{plain}
%    \end{macrocode}
% \end{macro}
%
% \begin{macro}{\theorembodyfont}
% \begin{macro}{\theorem@bodyfont}
%    If the |\theorembodyfont| is set by the user then it should not
%    interact with the default font set in the theorem style. When the
%    new font selection is in force this may happen if, for example,
%    the default is |\itshape| and the new |\theorembodyfont| is
%    |\sffamily|. So we add a |\reset@font| command in front of the
%    user definition.
% \changes{v2.0f}{89/07/20}{Unnecessary `global' removed.}
% \changes{v2.0g}{89/09/19}{`theoremfont to `theorembodyfont.}
% \changes{v2.1c}{92/06/22}{`reset@font added and logic changed.}
% \changes{v2.1d}{92/10/28}{Reset to empty if arg is empty}
%    \begin{macrocode}
\gdef\theorembodyfont#1{%
%    \end{macrocode}
%    We check if the argument supplied is empty and if so put nothing
%    into the |\theorem@bodyfont| token register to allow for
%    |\theorembodyfont{}| as a mean of using the default of the
%    current |\theoremstyle|.
%    \begin{macrocode}
   \def\@tempa{#1}%
   \ifx\@tempa\@empty
    \theorem@bodyfont{}%
   \else
    \theorem@bodyfont{\reset@font#1}%
   \fi
   }
\newtoks\theorem@bodyfont
\global\theorem@bodyfont{}
%    \end{macrocode}
% \end{macro}
% \end{macro}
%
%
% \begin{macro}{\theoremheaderfont}
% \changes{v2.0g}{89/09/19}{Macro added.}
%    The font for the theorem headers is handled differently because
%    this definition applies to all theorem styles.
%    \begin{macrocode}
\gdef\theoremheaderfont#1{\gdef\theorem@headerfont{#1}%
%    \end{macrocode}
%    After using the macro once it is redefined to produce an error
%    message.
%    \begin{macrocode}
       \gdef\theoremheaderfont##1{%
        \typeout{\string\theoremheaderfont\space should be used
                 only once.}}}
%    \end{macrocode}
% \end{macro}
%
% \begin{macro}{\theorem@headerfont}
% \changes{v2.0g}{89/09/19}{Macro added.}
%    To set the |\theorem@headerfont| default we first test if
%    the new fontselection scheme is in force.
% \changes{v2.1a}{90/02/15}{Support for new font selection.}
%    \begin{macrocode}
\ifx\upshape\undefined
%    \end{macrocode}
%    If not we define it to expand into |\bfseries|. We don't use
%    |\let| just in case a following style option redefines this
%    macro.
%    \begin{macrocode}
\gdef\theorem@headerfont{\bfseries}
%    \end{macrocode}
%    Otherwise we reset the current shape before calling |\bfseries|.
%    \begin{macrocode}
\else \gdef\theorem@headerfont{\normalfont\bfseries}\fi
%    \end{macrocode}
% \end{macro}
%
% \begin{macro}{\th@plain}
% \begin{macro}{\th@break}
% \begin{macro}{\th@marginbreak}
% \begin{macro}{\th@changebreak}
% \begin{macro}{\th@change}
% \begin{macro}{\th@margin}
%    The different styles are defined in macros such as
%    |\th@plain|.  Since memory space is precious in
%    ``non-Big-versions'', we have to avoid offering too many unused
%    definitions.  Therefore we define these styles in separate files
%    that can be loaded on demand.  Thus the commands themselves only
%    load these files. We use |\@input@|, a \LaTeXe{} internal command
%    that ensures that the file will be listed with |\listfiles|.
% \changes{v2.2a}{94/02/02}{Use `@input@ to load theorem layout files}
%    \begin{macrocode}
\gdef\th@plain{\@input@{thp.sty}}
\gdef\th@break{\@input@{thb.sty}}
\gdef\th@marginbreak{\@input@{thmb.sty}}
\gdef\th@changebreak{\@input@{thcb.sty}}
\gdef\th@change{\@input@{thc.sty}}
\gdef\th@margin{\@input@{thm.sty}}
%    \end{macrocode}
%    This list will be expanded when new styles become
%    available.  For testing,  just append new theorem substyles
%    as document options.
% \end{macro}
% \end{macro}
% \end{macro}
% \end{macro}
% \end{macro}
% \end{macro}
%
%
%
% \subsection{Definition of a new theorem set}
%
% As already pointed out, a new theorem environment can be defined in
% three different ways:
% \begin{verbatim}
%    \newtheorem{Lem}{Lemma}
%    \newtheorem{Lem}{Lemma}[section]
%    \newtheorem{Lem}[Theorem]{Lemma}
%\end{verbatim}
% The function of the macro |\newtheorem| is to recognize these cases
% and then to branch into one of the three macros |\@ynthm|, |\@xnthm|
% or |\@othm|.  This mechanism is adopted unchanged from
% \cite{latex:source}; the essential point here is that, for example,
% in the second case, the arguments \texttt{Lem}, \texttt{Lemma} and
% \texttt{section} are passed over to the macro |\@xnthm|.
%
% We inspect this case first because the others present fewer problems,
% and thus are easily derived from this one.
%
% \begin{macro}{\@xnthm}
%    For our example arguments, the macro |\@xnthm| must fulfill
%    the following:
%    \begin{itemize}
%       \item
%          Define a new \LaTeX-counter `Lem'
%       \item
%          reset this counter within a |\section|
%       \item
%          define the macro |\theLem|
%       \item
%          define the environment macros |\Lem| and
%          |\endLem| using the current
%          |\theoremstyle| and |\theorem@bodyfont|.
%    \end{itemize}
%    Obviously, all this should happen only if the first argument
%    of |\@xnthm| (i.e.\ \texttt{Lem} in our example) is chosen so
%    as not to conflict with any previously defined commands or
%    environments.  This test is performed by the \LaTeX{} macro
%    |\@ifdefinable|.
%    \begin{macrocode}
\gdef\@xnthm#1#2[#3]{\expandafter\@ifdefinable\csname #1\endcsname
%    \end{macrocode}
%    Therefore, the first argument of |\@ifdefinable| is the
%    expansion (in the example, |\Lem|) of
%    |\csname||#1\endcsname|.  The second argument is
%    executed only if the test has been completed successfully.
%    \begin{macrocode}
   {%
%    \end{macrocode}
%    Now we define the new counter. The names of the \LaTeX{}
%    macros employed should speak for themselves:
% \changes{v2.2b}{1995/10/07}{Use \cs{@newctr}}
% \changes{v2.2c}{1995/11/23}{Use \cs{@newctr} properly.}
%    \begin{macrocode}
    \@definecounter{#1}\@newctr{#1}[#3]%
%    \end{macrocode}
%    Using |\@newctr| will give a proper error message if the counter
%    in |#3| is not defined.
%    In defining `|\theLem|' we must generate the desired macro
%    name by use of |\expandafter| and |\csname|.
%    \begin{macrocode}
    \expandafter\xdef\csname the#1\endcsname
%    \end{macrocode}
%    An |\xdef| is used in order to make the definition global,
%    and to ensure that it contains the replacement texts of
%    |\@thmcountersep| and |\@thmcounter|.\footnote{These
%    two macros can be defined by the document style.  Their default
%    values produce a `\texttt{.}' as separation and an arabic
%    representation of the number.} However, not everything should be
%    expanded. For example, it saves space to use |\thesection|
%    instead of its---at times---lengthy expansion.
%    \begin{macrocode}
      {\expandafter \noexpand \csname the#3\endcsname
       \@thmcountersep \@thmcounter{#1}}%
%    \end{macrocode}
%    Thus with the defaults of \LaTeX{}, |\theLem| would be
%    replaced by the command sequence |\thesection.\arabic{Lem}|.
%
%    We will now look at the definition of the macro which is executed
%    at the beginning of the actual environment (in our example this
%    macro is |\Lem|). It should be noted that we use an
%    ``|\expandafter| trick'' to expand only certain parts of the
%    replacement text at the time of the definition.
%    \begin{macrocode}
    \def\@tempa{\global\@namedef{#1}}%
    \expandafter \@tempa \expandafter{%
%    \end{macrocode}
%    First, the macro that contains the current definitions of
%    |\@begintheorem| and |\@opargtheorem| should be called
%    up.  The name of this macro---as is already known---has the form
%    |\th@|\meta{theorem\ style}; therefore, it must be called by
%    \begin{macrocode}
      \csname th@\the \theorem@style
%    \end{macrocode}
%    In addition the default theorem font should be changeable, i.e.\
%    we have to insert the contents of |\theorem@bodyfont|. For
%    that reason, we expand even further, beyond |\endcsname|,
%    and thus insert the contents of the \textsf{token} register
%    |\theorem@bodyfont| in the replacement text.
%    \begin{macrocode}
            \expandafter \endcsname \the \theorem@bodyfont
%    \end{macrocode}
%    Now it is time to call the macro |\@thm| which takes over
%    the further processing.  It has two arguments: the current
%    counter name (in our example, \texttt{Lem}), and the text of the
%    label (in our example, \texttt{Lemma}).
%    \begin{macrocode}
     \@thm{#1}{#2}}%
%    \end{macrocode}
%    With this, the `sub-definition' is complete.  The macro
%    |\@endtheorem| ends a theorem environment and is, so far, nothing
%    but an |\endtrivlist|.  (Hence it is defined globally, and not
%    within the theorem styles.\footnote{This has to be changed as
%    soon as theorem styles that change \texttt{\bslash @endtheorem}
%    exist.  In such a case, all existing styles must be changed as
%    well since they will have to reset the macro.}) Therefore, we can
%    set it equivalent to the macro that ends the theorem set (in our
%    example, |\endLem|).  However, if some day theorem styles exist
%    that do change |\@endtheorem|, we would have to use the
%    commented-out line instead.
%    \begin{macrocode}
    \global \expandafter \let \csname end#1\endcsname \@endtheorem
%   \global\@namedef{end#1}{\@endtheorem}%
%    \end{macrocode}
%    With these commands all the required definitions are employed,
%    unless the test |\@ifdefinable| has failed.  Therefore, we
%    end the second argument of this macro and with it the definition
%    of |\@xnthm|.
%    \begin{macrocode}
   }}
%    \end{macrocode}
% \end{macro}
%
% \begin{macro}{\@ynthm}
%    The definition of |\@ynthm| is completely analogous. In
%    this case the new counter that is defined is not reset within
%    another counter; thus the definition of |\the...|
%    is simplified:
%    \begin{macrocode}
\gdef\@ynthm#1#2{\expandafter\@ifdefinable\csname #1\endcsname
   {\@definecounter{#1}%
    \expandafter\xdef\csname the#1\endcsname{\@thmcounter{#1}}%
%    \end{macrocode}
%    The rest of the definition corresponds literally to that of
%    |\@xnthm|:
%    \begin{macrocode}
    \def\@tempa{\global\@namedef{#1}}\expandafter \@tempa
     \expandafter{\csname th@\the \theorem@style \expandafter
     \endcsname \the\theorem@bodyfont \@thm{#1}{#2}}%
    \global \expandafter \let \csname end#1\endcsname \@endtheorem}}
%    \end{macrocode}
% \end{macro}
%
% \begin{macro}{\@othm}
%    The definition of |\@othm| does not contain anything new.
%    \begin{macrocode}
\gdef\@othm#1[#2]#3{%
%    \end{macrocode}
% \changes{v2.2b}{95/10/07}{Build in check for counter defined}
%    We do not define a new counter but instead use one that has
%    already been defined.  Thus the only definition we need is that
%    of this pseudo-counter (i.e.\ |\the|\meta{env.\ name}).
%    First we check if |#2| corresponds to a known counter name.
%    \begin{macrocode}
  \expandafter\ifx\csname c@#2\endcsname\relax
   \@nocounterr{#2}%
  \else
   \expandafter\@ifdefinable\csname #1\endcsname
   {\expandafter \xdef \csname the#1\endcsname
     {\expandafter \noexpand \csname the#2\endcsname}%
%    \end{macrocode}
%    All other parts of the definition can be adopted from
%    |\@xnthm|. We have to remember, though, that in this case
%    the name of the current counter and the theorem label have moved
%    to the second and third arguments.
%    \begin{macrocode}
    \def\@tempa{\global\@namedef{#1}}\expandafter \@tempa
     \expandafter{\csname th@\the \theorem@style \expandafter
     \endcsname \the\theorem@bodyfont \@thm{#2}{#3}}%
    \global \expandafter \let \csname end#1\endcsname \@endtheorem}%
  \fi}
%    \end{macrocode}
% \end{macro}
%
% \subsection{Macros that are employed in a {\envfont theorem}
%    environment}
%
% \begin{macro}{\@thm}
%    The macro |\@thm| has to increase the current counter.  Then,
%    depending on whether the environment has (or does not have) an
%    optional argument, it has to branch into either |\@begintheorem|
%    or |\@opargtheorem|.
%    \begin{macrocode}
\gdef\@thm#1#2{\refstepcounter{#1}%
%    \end{macrocode}
%    Now we start a {\envfont trivlist} environment, and give
%    |\@topsep| and |\@topsepadd| the values of the \textsf{skip}
%    registers |\theorempreskipamount| and |\theorempostskipamount|.
%    The value in |\@topsep| is the vertical space that is inserted by
%    the first (and only) |\item| in our |\trivlist| whilst
%    |\@topsepadd| is inserted by |\@endparenv| at the end of that
%    {\envfont trivlist} environment.  By using these registers, we
%    obtain the desired space around a {\envfont theorem} environment.
%    \begin{macrocode}
   \trivlist
   \@topsep \theorempreskipamount               % used by first \item
   \@topsepadd \theorempostskipamount           % used by \@endparenv
%    \end{macrocode}
%    Now we have to test whether an optional argument has been given.
% \changes{v2.0d}{89/07/18}{Forgotten percent added.}
%    \begin{macrocode}
   \@ifnextchar [%
%    \end{macrocode}
%    If there is an optional argument, we will call |\@ythm|, and
%    move the arguments read back into the input stream.
%    \begin{macrocode}
   {\@ythm{#1}{#2}}%
%    \end{macrocode}
%    If not, we call |\@begintheorem|. Its first argument is the
%    name of the theorem set (hence the second argument of
%    |\@thm|). Its second argument is the macro that produces the
%    current number.
%    \begin{macrocode}
   {\@begintheorem{#2}{\csname the#1\endcsname}\ignorespaces}}
%    \end{macrocode}
% \end{macro}
%
% \begin{macro}{\@xthm}
% \begin{macro}{\@ythm}
%    Both these macros were originally called by |\@thm|.  We do
%    not need |\@xthm| anymore, hence we reset it to
%    |\relax|. The definition of |\@ythm| has not changed at
%    all from its definition in \LaTeX. In order to make the macros
%    easier to understand, we will
%    nevertheless present it (commented out).
%    \begin{macrocode}
\global\let\@xthm\relax
% \def\@ythm#1#2[#3]{\@opargbegintheorem{#2}{\csname
%           the#1\endcsname}{#3}\ignorespaces}
%</package>
%    \end{macrocode}
%    The primitive |\ignorespaces| in |\@ythm| and
%    |\@thm| is needed to remove the spaces between the
%    |\begin{...}| and the actual text.
% \end{macro}
% \end{macro}
%
%
% \subsection{Definition of the theorem substyles}
%
% As already pointed out, the theorem substyles, defined below, are
% only loaded when necessary.  Note that all these substyles, except
% \texttt{plain}, have |\slshape| as the default body font.
%
% \subsubsection{The \texttt{plain} style}
%
% \label{dok:th@plain} As the following macros use |@|, we have
% to locally set the |\catcode| of this symbol to ``letter''.
% This happens within a group, so that we do not have to worry about
% which |\catcode| that symbol had before.
%    \begin{macrocode}
%<*thp>
\begingroup \makeatletter
%    \end{macrocode}
% Since we are now within a group, we must make all definitions
% globally.  First we make sure that \texttt{theorem.sty} is loaded.
% This will allow us to use this file as a document style option
% without having to call \texttt{theorem} itself as an option.  At the
% same time, we assure that at least version 2 is loaded, since
% |\theorem@style| was not defined in earlier versions.
%    \begin{macrocode}
\@ifundefined{theorem@style}{\input{theorem.sty}}{}
\ProvidesFile{thp.sty}
             [\FMithmInfo]
%    \end{macrocode}
%
%
% \begin{macro}{\th@plain}
% \begin{macro}{\@begintheorem}
% \begin{macro}{\@opargbegintheorem}
%    |\theoremstyle{plain}| corresponds to the original
%    definition, except that the distances to the surrounding text are
%    determined by the parameters |\theorempreskipamount| and
%    |\theorempostskipamount|.
%    First we set the default body font.
%    \begin{macrocode}
\gdef\th@plain{\normalfont\itshape
%    \end{macrocode}
%    Then we define |\@begintheorem| and
%    |\@opargbegintheorem|. These two macros define how the
%    header of a theorem is typeset.  |\@opargbegintheorem| will
%    be called if a \textsf{theorem} environment with an optional
%    argument is encountered; otherwise, the header is constructed by
%    calling |\@begintheorem|.  If one of these macros is
%    executed, we are within a \textsf{trivlist} environment started by
%    |\@thm|.  So the theorem header is produced with an
%    |\item| command.
%
%    Instead of specifying the header font directly, all standard
%    theorem styles use the |\theorem@headerfont| macro to allow
%    customization.  The extra space (|\labelsep|) is necessary
%    because of problems in the \textsf{trivlist} environment.
%    \begin{macrocode}
  \def\@begintheorem##1##2{%
        \item[\hskip\labelsep \theorem@headerfont ##1\ ##2]}%
%    \end{macrocode}
%    The definition of |\@opargbegintheorem| is completely
%    analogous. The only difference is the fact that there exists a
%    third argument (which is the optional parameter of the
%    environment and contains additional information about the
%    theorem).  Customarily we enclose it in parentheses.
%    \begin{macrocode}
\def\@opargbegintheorem##1##2##3{%
   \item[\hskip\labelsep \theorem@headerfont ##1\ ##2\ (##3)]}}
%    \end{macrocode}
% \end{macro}
% \end{macro}
% \end{macro}
%
% \noindent
% We conclude with an |\endgroup| to restore the
% |\catcode| of |@|.
%    \begin{macrocode}
\endgroup
%</thp>
%    \end{macrocode}
%
%
% \subsubsection{The \texttt{break} style}
%
% This style option is stored in the file |thb.sty|.
% For the next two lines see the documentation for |\th@plain| on
% page~\pageref{dok:th@plain}.
%    \begin{macrocode}
%<*thb>
\begingroup \makeatletter
\@ifundefined{theorem@style}{\input{theorem.sty}}{}
\ProvidesFile{thb.sty}
             [\FMithmInfo]
%    \end{macrocode}
%
% \begin{macro}{\th@break}
%    |\theoremstyle{break}| produces a line break after the name
%    of the theorem.  The font is |\slshape|.  Hence, we define
%    |\th@break| as follows:
%    \begin{macrocode}
\gdef\th@break{\normalfont\slshape
  \def\@begintheorem##1##2{\item[%
%    \end{macrocode}
%    We run into the following problem: it is not possible to create
%    the header with |\item[|\meta{title}|]| and then start
%    a new line by, for example, |\mbox{}\\|. Such a definition
%    will fail whenever a list environment follows immediately.  With
%    the above construction, the |\mbox{}| causes the switch
%    |@inlabel| (cf.\ definition of |\list| and
%    |\trivlist| in \cite{latex:source}) to be set to \texttt{false}
%    and so the following list will insert additional vertical
%    space (|\topskip|). This is quite annoying. Therefore, we
%    create the line break within the |\item|. In order to ensure
%    that the text will begin at the proper position in the following
%    line, we simply pretend that the label does not take any
%    room.\footnote{This will lead to problems whenever very high
%    symbols occurring in the line tower into the
%    heading. So, something else has to be done here sometime.}
%    \begin{macrocode}
       \rlap{\vbox{\hbox{\hskip \labelsep\theorem@headerfont ##1\ ##2}%
                   \hbox{\strut}}}]}%
%    \end{macrocode}
%    Again, the definition of |\@opargbegintheorem| is
%    completely analogous.
%    \begin{macrocode}
\def\@opargbegintheorem##1##2##3{%
  \item[\rlap{\vbox{\hbox{\hskip \labelsep \theorem@headerfont
                     ##1\ ##2\ (##3)}%
                    \hbox{\strut}}}]}}
\endgroup
%</thb>
%    \end{macrocode}
% \end{macro}
%
%
% \subsubsection{The \texttt{changebreak} style}
%
%    \begin{macrocode}
% This style option is stored in the file |thcb.sty|.
%    \begin{macrocode}
%<*thcb>
\begingroup \makeatletter
\@ifundefined{theorem@style}{\input{theorem.sty}}{}
\ProvidesFile{thcb.sty}
             [\FMithmInfo]
%    \end{macrocode}
%
% \begin{macro}{\th@changebreak}
%    The \texttt{change-break} theorem style is like \texttt{break}
%    but with interchange of theorem name and theorem number.  Thus we
%    define |\th@changebreak| as follows:
%    \begin{macrocode}
\gdef\th@changebreak{\normalfont\slshape
  \def\@begintheorem##1##2{\item
       [\rlap{\vbox{\hbox{\hskip\labelsep \theorem@headerfont ##2\ ##1}%
                    \hbox{\strut}}}]}%
\def\@opargbegintheorem##1##2##3{%
  \item[\rlap{\vbox{\hbox{\hskip\labelsep \theorem@headerfont
                     ##2\ ##1\ (##3)}%
                    \hbox{\strut}}}]}}
\endgroup
%</thcb>
%    \end{macrocode}
% \end{macro}
%
%
% \subsubsection{The \texttt{change} style}
%
% This style option is stored in the file |thc.sty|.
%    \begin{macrocode}
%<*thc>
\begingroup \makeatletter
\@ifundefined{theorem@style}{\input{theorem.sty}}{}
\ProvidesFile{thc.sty}
             [\FMithmInfo]
%    \end{macrocode}
%
% \begin{macro}{\th@change}
%    The \texttt{change} theorem style corresponds to the
%    \texttt{change break} style without a linebreak after the header.
%    To say it in another way, it's the same as the \texttt{plain}
%    style but with number and name interchanged and |\slshape| as the
%    default font.
%    \begin{macrocode}
\gdef\th@change{\normalfont\slshape
  \def\@begintheorem##1##2{\item
       [\hskip\labelsep \theorem@headerfont ##2\ ##1]}%
\def\@opargbegintheorem##1##2##3{%
  \item[\hskip\labelsep \theorem@headerfont ##2\ ##1\ (##3)]}}
\endgroup
%</thc>
%    \end{macrocode}
% \end{macro}
%
%
% \subsubsection{The \texttt{marginbreak} style}
%
% This style option is the one used most often at Mainz.  It is saved
% in the file |thmb.sty|.
%    \begin{macrocode}
%<*thmb>
\begingroup \makeatletter
\@ifundefined{theorem@style}{\input{theorem.sty}}{}
\ProvidesFile{thmb.sty}
             [\FMithmInfo]
%    \end{macrocode}
%
% \begin{macro}{\th@marginbreak}
%    The \texttt{margin break} style is nearly the same as the
%    \texttt{change break} style. The only difference is the placement
%    of the theorem number.  We use |\llap| to place it in the left
%    margin.
%
%    In this style |\labelsep| denotes the separation between the
%    number and the text.
%    \begin{macrocode}
\gdef\th@marginbreak{\normalfont\slshape
  \def\@begintheorem##1##2{\item
       [\rlap{\vbox{\theorem@headerfont
                    \hbox{\llap{##2}\hskip\labelsep ##1}%
                    \hbox{\strut}}}]}%
\def\@opargbegintheorem##1##2##3{%
  \item[\rlap{\vbox{\theorem@headerfont
                    \hbox{\llap{##2}\hskip\labelsep ##1\ (##3)}%
                    \hbox{\strut}}}]}}
\endgroup
%</thmb>
%    \end{macrocode}
% \changes{v2.1c}{90/02/17}{`endgroup outside macrocode env.}
% \end{macro}
%
%
%
% \subsubsection{The \texttt{margin} style}
%
% This style option is stored in the file |thm.sty|.
%    \begin{macrocode}
%<*thm>
\begingroup \makeatletter
\@ifundefined{theorem@style}{\input{theorem.sty}}{}
\ProvidesFile{thm.sty}
             [\FMithmInfo]
%    \end{macrocode}
%
% \begin{macro}{\th@margin}
%    Again this is only a variant of the theorem styles described
%    above without any new ideas.
%    \begin{macrocode}
\gdef\th@margin{\normalfont\slshape
  \def\@begintheorem##1##2{\item
       [\theorem@headerfont \llap{##2}\hskip\labelsep ##1]}%
\def\@opargbegintheorem##1##2##3{%
  \item[\theorem@headerfont \llap{##2}\hskip\labelsep ##1\ (##3)]}}
\endgroup
%</thm>
%    \end{macrocode}
% \end{macro}
%
%
%
%
% \subsection{Final Definitions}
%
% \begin{macro}{\theorempreskipamount}
% \begin{macro}{\theorempostskipamount}
%    The \textsf{skip} parameters that regulate the vertical empty space
%    before and after the {\envfont theorem} environment have
%    to be allocated as well.
% \changes{v2.0f}{89/07/20}{Unnecessary `global' removed.}
%    \begin{macrocode}
%<*package>
\newskip\theorempreskipamount
\newskip\theorempostskipamount
%    \end{macrocode}
%    Since we have used the same values for all theorem sets, we now can
%    assign them.
%    \begin{macrocode}
\global\setlength\theorempreskipamount{12pt plus 5pt minus 3pt}
\global\setlength\theorempostskipamount{8pt plus 3pt minus 1.5pt}
%    \end{macrocode}
% \end{macro}
% \end{macro}
%
%
% \begin{macro}{\@endtheorem}
%    The same holds for the macro |\@endtheorem|, which ends a
%    {\envfont theorem} environment. Since it is the same for all
%    theorem sets, it is removed from the macros |\th@|\meta{style}.
%    It simply ends the {\envfont trivlist} environment, which was
%    begun in |\@thm|.
%    \begin{macrocode}
\global\let\@endtheorem=\endtrivlist
%    \end{macrocode}
% \end{macro}
%
%
%
% \begin{macro}{\@preamblecmds}
%    All macros defined above are to be used only in the preamble.
%    Therefore, we insert them in |\@preamblecmds| which will disable
%    them at begin document. This is done by the internal \LaTeXe{}
%    command |\@onlypreamble|.
% \changes{v2.0g}{89/09/19}{`theoremheaderfont and `theorembodyfont
%    added.}
% \changes{v2.2a}{94/02/03}{Use \LaTeXe{} interface}
%    \begin{macrocode}
\@onlypreamble\@xnthm
\@onlypreamble\@ynthm
\@onlypreamble\@othm
\@onlypreamble\newtheorem
\@onlypreamble\theoremstyle
\@onlypreamble\theorembodyfont
\@onlypreamble\theoremheaderfont
%    \end{macrocode}
% \end{macro}
%
% Finally we declare the \texttt{plain} theorem style to be the default.
% \changes{v2.1b}{90/06/04}{Fixed missing `item by declaring default
%    style.}
%    \begin{macrocode}
\theoremstyle{plain}
%</package>
%    \end{macrocode}
%
% \Finale

\endinput
}{}
\ProvidesFile{thm.sty}
             [\FMithmInfo]
%    \end{macrocode}
%
% \begin{macro}{\th@margin}
%    Again this is only a variant of the theorem styles described
%    above without any new ideas.
%    \begin{macrocode}
\gdef\th@margin{\normalfont\slshape
  \def\@begintheorem##1##2{\item
       [\theorem@headerfont \llap{##2}\hskip\labelsep ##1]}%
\def\@opargbegintheorem##1##2##3{%
  \item[\theorem@headerfont \llap{##2}\hskip\labelsep ##1\ (##3)]}}
\endgroup
%</thm>
%    \end{macrocode}
% \end{macro}
%
%
%
%
% \subsection{Final Definitions}
%
% \begin{macro}{\theorempreskipamount}
% \begin{macro}{\theorempostskipamount}
%    The \textsf{skip} parameters that regulate the vertical empty space
%    before and after the {\envfont theorem} environment have
%    to be allocated as well.
% \changes{v2.0f}{89/07/20}{Unnecessary `global' removed.}
%    \begin{macrocode}
%<*package>
\newskip\theorempreskipamount
\newskip\theorempostskipamount
%    \end{macrocode}
%    Since we have used the same values for all theorem sets, we now can
%    assign them.
%    \begin{macrocode}
\global\setlength\theorempreskipamount{12pt plus 5pt minus 3pt}
\global\setlength\theorempostskipamount{8pt plus 3pt minus 1.5pt}
%    \end{macrocode}
% \end{macro}
% \end{macro}
%
%
% \begin{macro}{\@endtheorem}
%    The same holds for the macro |\@endtheorem|, which ends a
%    {\envfont theorem} environment. Since it is the same for all
%    theorem sets, it is removed from the macros |\th@|\meta{style}.
%    It simply ends the {\envfont trivlist} environment, which was
%    begun in |\@thm|.
%    \begin{macrocode}
\global\let\@endtheorem=\endtrivlist
%    \end{macrocode}
% \end{macro}
%
%
%
% \begin{macro}{\@preamblecmds}
%    All macros defined above are to be used only in the preamble.
%    Therefore, we insert them in |\@preamblecmds| which will disable
%    them at begin document. This is done by the internal \LaTeXe{}
%    command |\@onlypreamble|.
% \changes{v2.0g}{89/09/19}{`theoremheaderfont and `theorembodyfont
%    added.}
% \changes{v2.2a}{94/02/03}{Use \LaTeXe{} interface}
%    \begin{macrocode}
\@onlypreamble\@xnthm
\@onlypreamble\@ynthm
\@onlypreamble\@othm
\@onlypreamble\newtheorem
\@onlypreamble\theoremstyle
\@onlypreamble\theorembodyfont
\@onlypreamble\theoremheaderfont
%    \end{macrocode}
% \end{macro}
%
% Finally we declare the \texttt{plain} theorem style to be the default.
% \changes{v2.1b}{90/06/04}{Fixed missing `item by declaring default
%    style.}
%    \begin{macrocode}
\theoremstyle{plain}
%</package>
%    \end{macrocode}
%
% \Finale

\endinput
}{}
\ProvidesFile{thcb.sty}
             [\FMithmInfo]
%    \end{macrocode}
%
% \begin{macro}{\th@changebreak}
%    The \texttt{change-break} theorem style is like \texttt{break}
%    but with interchange of theorem name and theorem number.  Thus we
%    define |\th@changebreak| as follows:
%    \begin{macrocode}
\gdef\th@changebreak{\normalfont\slshape
  \def\@begintheorem##1##2{\item
       [\rlap{\vbox{\hbox{\hskip\labelsep \theorem@headerfont ##2\ ##1}%
                    \hbox{\strut}}}]}%
\def\@opargbegintheorem##1##2##3{%
  \item[\rlap{\vbox{\hbox{\hskip\labelsep \theorem@headerfont
                     ##2\ ##1\ (##3)}%
                    \hbox{\strut}}}]}}
\endgroup
%</thcb>
%    \end{macrocode}
% \end{macro}
%
%
% \subsubsection{The \texttt{change} style}
%
% This style option is stored in the file |thc.sty|.
%    \begin{macrocode}
%<*thc>
\begingroup \makeatletter
\@ifundefined{theorem@style}{% \iffalse meta-comment
%
% Copyright (C) 1993-2024
%
% The LaTeX Project and any individual authors listed elsewhere
% in this file.
%
% This file is part of the Standard LaTeX `Tools Bundle'.
% -------------------------------------------------------
%
% It may be distributed and/or modified under the
% conditions of the LaTeX Project Public License, either version 1.3c
% of this license or (at your option) any later version.
% The latest version of this license is in
%    https://www.latex-project.org/lppl.txt
% and version 1.3c or later is part of all distributions of LaTeX
% version 2005/12/01 or later.
%
% The list of all files belonging to the LaTeX `Tools Bundle' is
% given in the file `manifest.txt'.
%
% \fi
%
%%
%
%
\def\FMithmInfo{2023/07/05 v2.2c Theorem extension package (FMi)}
%
% \ProvidesFile{theorem.dtx}[\FMithmInfo]
%\iffalse   % this is a METACOMMENT !
%%
%% Package `theorem' to use with LaTeX2e
%% Copyright (C) 1989-1995 Frank Mittelbach, all rights reserved.
%<+package>\@ifundefined{theorem@style}{}{\endinput}
%<+package>\NeedsTeXFormat{LaTeX2e}[1995/06/01]
%<+package>\ProvidesPackage{theorem}
%<+package>         [\FMithmInfo]
%
% \fi
%
% \changes{v1.0}{87/05/07}{First draft.}
% \changes{v1.0a}{87/05/07}{`thistheoremstyle' added.}
% \changes{v1.1a}{87/07/29}{New concept: redefine the internal
%                          \protect\LaTeX{} commands `@xnthm' etc.}
% \changes{v1.1b}{87/07/31}{`theoremfont' used to specify font.}
% \changes{v1.2a}{88/01/08}{`@endtheorem' now globally defined.}
% \changes{v1.2b}{88/01/11}{`newtheorem' only allowed in the preamble.}
% \changes{v1.3a}{88/02/14}{`theoremfont' now a toks register.}
%
% \changes{v2.0a}{88/11/16}{New implementation and new documentation.}
% \changes{v2.0a}{88/11/16}{No layout changes but internal modifications
%                           old substyles won't work.}
% \changes{v2.0b}{89/01/30}{`global' used for all definitions.}
% \changes{v2.0c}{89/06/15}{Documentation translated to English.
%                            Thanks to LLL and Nina}
% \changes{v2.1a}{90/02/15}{Support for new font selection.}
% \changes{v2.2a}{94/02/03}{Cleanup for LaTeX2e}
%
% \DoNotIndex{\@endparenv,\@ifdefinable,\@ifnextchar,\@ifundefined}
% \DoNotIndex{\@namedef,\@nameuse,\@spaces,\@tempa,\@warning}
% \DoNotIndex{\ }
% \DoNotIndex{\begingroup,\bfseries}
% \DoNotIndex{\csname}
% \DoNotIndex{\def,\do,\docdate}
% \DoNotIndex{\endcsname,\endgroup,\endinput,\endtrivlist,\expandafter}
% \DoNotIndex{\filedate,\fileversion}
% \DoNotIndex{\gdef,\global}
% \DoNotIndex{\hbox,\hskip}
% \DoNotIndex{\ignorespaces,\input,\itshape,\item}
% \DoNotIndex{\labelsep,\let,\llap}
% \DoNotIndex{\makeatletter}
% \DoNotIndex{\noexpand}
% \DoNotIndex{\refstepcounter,\relax,\rlap}
% \DoNotIndex{\setlength,\slshape,\space,\string,\strut}
% \DoNotIndex{\the,\trivlist,\typeout}
% \DoNotIndex{\vbox}
% \DoNotIndex{\xdef}
%
% \def\deflabel#1{\ttfamily #1\hfill}
% \def\deflist#1{\begin{list}{}{\settowidth\labelwidth{\ttfamily #1}%
%                               \setlength\leftmargin\labelwidth
%                               \addtolength\leftmargin\labelsep
%                               \let\makelabel\deflabel}}
% \def\enddeflist{\end{list}}
% \def\envfont{\normalfont\ttfamily}
%
% \MakeShortVerb{\|}
%
% \newcommand{\TUB}{{\sl TUGboat\/}}
%
% \GetFileInfo{theorem.dtx}
% \title{An Extension of the
%        \LaTeX{} theorem environment\thanks{This file
%        has version number \fileversion, last
%        revised \filedate.}}
% \author{Frank Mittelbach}
%
% \MaintainedByLaTeXTeam{tools}
% \maketitle
%
% \begin{abstract}
%    The macros described in this paper yield an extension of the
%    \LaTeX{} theorem mechanism. It is designed to satisfy the
%    different requirements of various journals. Thus, the
%    layout of the ``theorems'' can be manipulated by determining a
%    ``style''. This article describes not only the use, but
%    also the definition, of the necessary macros.
% \end{abstract}
%
% \section*{Preface to version 2.2}
%
% For \LaTeXe{} this package did not need any fundamental changes. I
% only modified the messages generated so that theorem layout styles
% will show up with the |\listfiles| command and cleaned the section
% on the New Font Selection Scheme since this is now included in
% \LaTeX.
%
% \section*{Preface to version 2.1}
%
% This version is identical to 2.0g described in \TUB\ 10\#3 except
% for some internal defaults which are now set depending on the used
% font selection scheme.
%
% This was done to avoid unpleasant surprises if the new font selection
% scheme is in force. For further details see
% section~\ref{sec:newfonts} and \cite{bk:GMS94}.
%
% \section{Introduction}
%
% For our purposes here, ``theorems'' are labelled enunciations,
% often set off from the main text by extra space and a font change.
% Theorems, corollaries, conjectures, definitions, and remarks
% are all instances of ``theorems''.  The ``header'' of these structures
% is composed of a label (such as \textsc{Theorem} or \textsc{Remark})
% and a number which serializes an item in the sequence of items
% with the same label.
%
% Shortly after the introduction of \LaTeX{} at the Fachbereich
% Mathematik in Mainz, the desire to manipulate the layout of
% ``theorems'' arose.  In Mainz, the following two conventions came
% into general use:
% \begin{enumerate}
%   \item The number of the theorem is shown in the margin.
%   \item There is a line break at
%         the end of the theorem header.
% \end{enumerate}
%
% Additionally, some journals require different formats which depend on
% the ``sort of theorem'': e.g.\ often remarks and definitions are set
% in |\upshape|, while |\itshape| is employed for main theorems.
%
% Confronted with these requirements, a theorem environment was
% developed in Mainz which allows separate determination of the
% layout of the ``theorems sets'', comparable to |\pagestyle|.
%
%
% \section{The user interface}
% \subsection{Defining new theorem sets}
%
% \DescribeMacro\newtheorem
% As in the original \LaTeX{} version, the command |\newtheorem|
% defines a new ``theorem set'' or ``theorem-like structure''.
% Two required arguments name the new environment and give the text
% to be typeset with each instance of the new ``set'', while
% an optional argument determines how
% the ``set'' is enumerated:
% \begin{description}
%    \item[\ttfamily \bslash newtheorem\{foo\}\{bar\}]
%       The theorem set {\envfont foo} (whose name is \texttt{bar})
%       uses its own counter.
%    \item[\ttfamily \bslash newtheorem\{foo2\}{[foo]}\{bar2\}]
%       The theorem set {\envfont foo2} (printed name \texttt{bar2})
%       uses the same counter as the theorem set \texttt{foo}.
%    \item[\ttfamily \bslash newtheorem\{foo3\}\{bar3\}{[section]}]
%       The theorem set {\envfont foo3} (printed name \texttt{bar3}) is
%       enumerated within the counter \texttt{section}, i.e.\ with every
%       new |\section| the enumeration begins again with 1, and
%       the enumeration is composed from the section-number and the
%       theorem counter itself.
% \end{description}
%
% \DescribeMacro\theoremstyle
% Additionally, the command |\theoremstyle| can define the layout
% of various, or all, theorem sets.  It should be noted that any
% theorem set defined by |\newtheorem| is typeset in the
% |\theoremstyle| that is current at the time of the definition.
% Thus, the following
% \begin{verbatim}
%    \theoremstyle{break}        \newtheorem{Cor}{Corollary}
%    \theoremstyle{plain}        \newtheorem{Exa}{Example}[section]
%\end{verbatim}                             ^^A No blank at line start
% leads to the result that the set {\envfont Cor} is formatted in the
% style \texttt{break}, while the set {\envfont Exa} and all the
% following ones are formatted in the style \texttt{plain}, unless
% another |\theoremstyle| follows.  Since the definitions installed by
% |\newtheorem| are global, one also can limit |\theoremstyle| locally
% by grouping braces.
%
% \DescribeMacro\theorembodyfont
% The choice of the font for the theorem body is completely
% independent of the chosen |\theoremstyle|; this has proven to
% be very advantageous.  For example,
% \begin{verbatim}
%   {\theorembodyfont{\upshape}          \newtheorem{Rem}{Remark}}
%\end{verbatim}
% defines a theorem set {\envfont Rem}, which will be set in
% |\upshape| in the current layout (which in our example is
% \texttt{plain}).  As with |\theoremstyle|, the |\theorembodyfont|
% chosen is that current at the time of |\newtheorem|.  If
% |\theorembodyfont| is not specified or one defines
% |\theorembodyfont{}|, then the font used will be that defined by the
% |\theoremstyle|.
%
% \DescribeMacro\theoremheaderfont It is also possible to customize
% the font used for the theorem headers.  This is, however, a global
% declaration, and therefore there should be at most one
% |\theoremheaderfont| declaration in the preamble.\footnote{If it is
% actually necessary to have different header fonts, one has to define
% new theorem styles (substituting the desired font) or specify the
% information directly in the \texttt{\bslash newtheorem} declaration
% (the unclean variant).}
%
% \DescribeMacro\theorempreskipamount
% \DescribeMacro\theorempostskipamount Two additional parameters
% affect the vertical space around the theorem environments:
% |\theorempreskipamount| and |\theorempostskipamount| define,
% respectively, the spacing before and after such an environment.
% These parameters apply for all theorem sets and can be manipulated
% with the ordinary length macros.  They are rubber lengths
% (`\textsf{skips}'), and therefore can contain \texttt{plus} and
% \texttt{minus} parts.
%
% Since the definition of theorem sets should---most sensibly---be
% placed in the preamble, we only allow installation there.
% It is therefore possible to release the memory used here after
% |\begin{document}|, in order to make room for other
% applications.
%
%
% \subsection{Existing theorem styles}
%
% The following theorem styles exist to date:
% \begin{deflist}{marginbreak}
%    \item[plain]
%       This theorem style emulates the original \LaTeX{} definition,
%       except that additionally the parameters
%       |\theorem...skipamount| are used.
%    \item[break]
%       In this style, the theorem header is followed by a line
%       break.
%    \item[marginbreak]
%       The theorem number is set in the margin, and there is a line
%       break as in \texttt{break}.
%    \item[changebreak]
%       Like \texttt{break}, but with header number and text
%       interchanged.
%    \item[change]
%       Header number and text are interchanged, without a line break.
%    \item[margin]
%       The number is set in the left margin, without a line break.
% \end{deflist}
% All styles (except \texttt{plain}) select |\slshape| as the default
% |\theorembodyfont|.
%
%
% \subsection{Examples}
%
% Given the above theorem sets {\envfont Cor}, {\envfont Exa}
% and {\envfont Rem},
% suppose that the preamble also contains the declarations:
% \begin{verbatim}
%    \theoremstyle{marginbreak}   \newtheorem{Lem}[Cor]{Lemma}
%    \theoremstyle{change}
%    \theorembodyfont{\itshape}        \newtheorem{Def}[Cor]{Definition}
%
%    \theoremheaderfont{\scshape}
%\end{verbatim}
% Then the following are some typical examples of the typeset output
% resulting from their use.
%
% \begin{Cor}
%    This is a sentence typeset in the theorem environment {\envfont
%    Cor}.
% \end{Cor}
%
% \begin{Exa}
%    This is a sentence typeset in the theorem environment {\envfont
%    Exa}.
% \end{Exa}
%
% \begin{Rem}
%    This is a sentence typeset in the theorem environment {\envfont
%    Rem}.
% \end{Rem}
%
% \begin{Lem}[Ben User]
%    This is a sentence typeset in the theorem environment {\envfont
%    Lem}.
% \end{Lem}
%
% \begin{Def}[Very impressive Definition]
%    This is a sentence typeset in the theorem environment {\envfont
%    Def}.
% \end{Def}
%
% The last two examples show the effect of the optional argument to
% a theorem environment (it is the text typeset in parentheses).
%
%
% \section{Special Considerations}
% \label{sec:newfonts}
%
% Theoremheader and body are implemented as a unit. This means that
% the |\theoremheaderfont|  will inherit characteristics of the
% |\theorembodyfont| in \LaTeXe. Thus, if for example
% |\theorembodyfont| is |\itshape| and |\theoremheaderfont|
% is |\bfseries| the font selected for the header will have the
% characteristics `bold extended italic'. If this is not desired one
% should set the |\theoremheaderfont| to something like
% \begin{verbatim}
%      \theoremheaderfont{\normalfont\bfseries}
%\end{verbatim}
% i.e.\ supplying all necessary font information explicitly.
%
%
%
% \section{Acknowledgements}
%
% The publication of this set of macros was only possible with the
% help of Christina Busse (translating the manuscript into English),
% Joachim Pense (playing the r\^ole of typist), Chris Rowley
% (looking everything over) and many others providing useful
% suggestions.
%
% \MaybeStop{%
%    \begin{thebibliography}{9}
%       \bibitem{bk:GMS94} \textsc{M.~Goossens}, \textsc{F.~Mittelbach}
%         and \textsc{A.~Samarin}.
%         \newblock The \LaTeX{} Companion.
%         \newblock
%         Addison-Wesley, Reading, Massachusetts, 1994.
%       \bibitem{latex:source} \textsc{Lamport, Leslie.} \newblock
%                               \texttt{latex.tex}, version 2.09, date
%                               Feb. 1990.
%    \end{thebibliography}
%    \PrintIndex
%    \PrintChanges}
%
%
% \section{The documentation driver file}
%
% The next bit of code contains the documentation driver file for
% \TeX{}, i.e., the file that will produce the documentation you are
% currently reading. It will be extracted from this file by the
% \texttt{docstrip} program. Since it is the first code in the file
% one can alternatively process this file directly with \LaTeXe{} to
% obtain the documentation.
%
%    \begin{macrocode}
%<*driver>
\documentclass{ltxdoc}

\usepackage{theorem}

%<+driver>% The next few lines define theorem sets which are used
%<+driver>% in the example section of the documentation.

\theoremstyle{break}       \newtheorem{Cor}{Corollary}
\theoremstyle{plain}       \newtheorem{Exa}{Example}[section]
{\theorembodyfont{\upshape}\newtheorem{Rem}{Remark}}
\theoremstyle{marginbreak} \newtheorem{Lem}[Cor]{Lemma}
\theoremstyle{change}
\theorembodyfont{\itshape} \newtheorem{Def}[Cor]{Definition}

\theoremheaderfont{\scshape}

\RecordChanges

\begin{document}
   \DocInput{theorem.dtx}
\end{document}
%</driver>
%    \end{macrocode}
%
% \section{Definition of the Macros}
%
% If the file has been loaded before, we abort immediately. If not the
% package announces itself (this is actually done at the very top of
% the file---the way it is done isn't good style so don't copy it).
% \changes{v2.0e}{89/07/19}{Spaces removed from `typeout'.}
% \changes{v2.2b}{95/11/19}{Announce moved to top of file}
%    \begin{macrocode}
%<*package>
%\@ifundefined{theorem@style}{}{\endinput}
%\def\FMithmInfo{1995/11/19 v2.2b Theorem extension package (FMi)}
%\ProvidesPackage{theorem}[\FMithmInfo]
%    \end{macrocode}
%
%
% \subsection{Definition of theorem styles and fonts}
%
% All the definitions in this file are done globally to allow
% inputting this file inside a group.
%
% \begin{macro}{\theoremstyle}
%    Before a theorem style can be installed, the chosen style must be
%    known. For that reason, we must test to see that
%    |\th@|\meta{style} is known or, more precisely, that it is
%    different from |\relax|. If the style is not known then
%    |\th@plain| is used.
%    \begin{macrocode}
\gdef\theoremstyle#1{%
   \@ifundefined{th@#1}{\@warning
          {Unknown theoremstyle `#1'. Using `plain'}%
          \theorem@style{plain}}%
%    \end{macrocode}
%    We save the theorem style to be used in the token register
%    |\theorem@style|.
%    \begin{macrocode}
      {\theorem@style{#1}}%
%    \end{macrocode}
%    Now we ``evaluate'' the theorem style: this means, we call the
%    macro |\th@|\meta{style} which will activate the relevant
%    definitions which are contained in a separate file.  This is done
%    in a group to suppress changes to the current font.  This could
%    otherwise pose problems together with the new font selection
%    scheme\footnote{When I printed the original article using the new
%    font selection scheme I ended with a document with slanted
%    typefaces (text headings and all) simply because one of the
%    theorem styles used \texttt{\bslash sl} at toplevel.} if the
%    |\th@|\meta{style} is evaluated a second time.
% \changes{v2.1a}{90/02/15}{Group added around `th@..}
% \changes{v2.1a}{90/02/15}{Reseting of `..@begintheorem removed.}
%    \begin{macrocode}
      \begingroup
        \csname th@\the\theorem@style \endcsname
      \endgroup}
%    \end{macrocode}
% \end{macro}
%
%
% \begin{macro}{\@begintheorem}
% \changes{v2.1a}{90/02/15}{Relaxed at toplevel.}
% \begin{macro}{\@opargbegintheorem}
% \changes{v2.1a}{90/02/15}{Relaxed at toplevel.}
%    We reset |\@begintheorem| and |\@opargbegintheorem| to
%    |\relax| since these commands are no longer necessary at
%    toplevel. This will save a few tokens.
%    \begin{macrocode}
\global\let\@begintheorem\relax
\global\let\@opargbegintheorem\relax
%    \end{macrocode}
% \end{macro}
% \end{macro}
%
% \begin{macro}{\theorem@style}
%    Obviously the \textsf{token} register used above has to be
%    allocated. To assure the utmost compatibility with the
%    original \LaTeX{} definition, we set the default theorem style to
%    \texttt{plain}, which implements the usual \LaTeX{} convention.
% \changes{v2.0f}{89/07/20}{Unnecessary `global' removed.}
%    \begin{macrocode}
\newtoks\theorem@style
\global\theorem@style{plain}
%    \end{macrocode}
% \end{macro}
%
% \begin{macro}{\theorembodyfont}
% \begin{macro}{\theorem@bodyfont}
%    If the |\theorembodyfont| is set by the user then it should not
%    interact with the default font set in the theorem style. When the
%    new font selection is in force this may happen if, for example,
%    the default is |\itshape| and the new |\theorembodyfont| is
%    |\sffamily|. So we add a |\reset@font| command in front of the
%    user definition.
% \changes{v2.0f}{89/07/20}{Unnecessary `global' removed.}
% \changes{v2.0g}{89/09/19}{`theoremfont to `theorembodyfont.}
% \changes{v2.1c}{92/06/22}{`reset@font added and logic changed.}
% \changes{v2.1d}{92/10/28}{Reset to empty if arg is empty}
%    \begin{macrocode}
\gdef\theorembodyfont#1{%
%    \end{macrocode}
%    We check if the argument supplied is empty and if so put nothing
%    into the |\theorem@bodyfont| token register to allow for
%    |\theorembodyfont{}| as a mean of using the default of the
%    current |\theoremstyle|.
%    \begin{macrocode}
   \def\@tempa{#1}%
   \ifx\@tempa\@empty
    \theorem@bodyfont{}%
   \else
    \theorem@bodyfont{\reset@font#1}%
   \fi
   }
\newtoks\theorem@bodyfont
\global\theorem@bodyfont{}
%    \end{macrocode}
% \end{macro}
% \end{macro}
%
%
% \begin{macro}{\theoremheaderfont}
% \changes{v2.0g}{89/09/19}{Macro added.}
%    The font for the theorem headers is handled differently because
%    this definition applies to all theorem styles.
%    \begin{macrocode}
\gdef\theoremheaderfont#1{\gdef\theorem@headerfont{#1}%
%    \end{macrocode}
%    After using the macro once it is redefined to produce an error
%    message.
%    \begin{macrocode}
       \gdef\theoremheaderfont##1{%
        \typeout{\string\theoremheaderfont\space should be used
                 only once.}}}
%    \end{macrocode}
% \end{macro}
%
% \begin{macro}{\theorem@headerfont}
% \changes{v2.0g}{89/09/19}{Macro added.}
%    To set the |\theorem@headerfont| default we first test if
%    the new fontselection scheme is in force.
% \changes{v2.1a}{90/02/15}{Support for new font selection.}
%    \begin{macrocode}
\ifx\upshape\undefined
%    \end{macrocode}
%    If not we define it to expand into |\bfseries|. We don't use
%    |\let| just in case a following style option redefines this
%    macro.
%    \begin{macrocode}
\gdef\theorem@headerfont{\bfseries}
%    \end{macrocode}
%    Otherwise we reset the current shape before calling |\bfseries|.
%    \begin{macrocode}
\else \gdef\theorem@headerfont{\normalfont\bfseries}\fi
%    \end{macrocode}
% \end{macro}
%
% \begin{macro}{\th@plain}
% \begin{macro}{\th@break}
% \begin{macro}{\th@marginbreak}
% \begin{macro}{\th@changebreak}
% \begin{macro}{\th@change}
% \begin{macro}{\th@margin}
%    The different styles are defined in macros such as
%    |\th@plain|.  Since memory space is precious in
%    ``non-Big-versions'', we have to avoid offering too many unused
%    definitions.  Therefore we define these styles in separate files
%    that can be loaded on demand.  Thus the commands themselves only
%    load these files. We use |\@input@|, a \LaTeXe{} internal command
%    that ensures that the file will be listed with |\listfiles|.
% \changes{v2.2a}{94/02/02}{Use `@input@ to load theorem layout files}
%    \begin{macrocode}
\gdef\th@plain{\@input@{thp.sty}}
\gdef\th@break{\@input@{thb.sty}}
\gdef\th@marginbreak{\@input@{thmb.sty}}
\gdef\th@changebreak{\@input@{thcb.sty}}
\gdef\th@change{\@input@{thc.sty}}
\gdef\th@margin{\@input@{thm.sty}}
%    \end{macrocode}
%    This list will be expanded when new styles become
%    available.  For testing,  just append new theorem substyles
%    as document options.
% \end{macro}
% \end{macro}
% \end{macro}
% \end{macro}
% \end{macro}
% \end{macro}
%
%
%
% \subsection{Definition of a new theorem set}
%
% As already pointed out, a new theorem environment can be defined in
% three different ways:
% \begin{verbatim}
%    \newtheorem{Lem}{Lemma}
%    \newtheorem{Lem}{Lemma}[section]
%    \newtheorem{Lem}[Theorem]{Lemma}
%\end{verbatim}
% The function of the macro |\newtheorem| is to recognize these cases
% and then to branch into one of the three macros |\@ynthm|, |\@xnthm|
% or |\@othm|.  This mechanism is adopted unchanged from
% \cite{latex:source}; the essential point here is that, for example,
% in the second case, the arguments \texttt{Lem}, \texttt{Lemma} and
% \texttt{section} are passed over to the macro |\@xnthm|.
%
% We inspect this case first because the others present fewer problems,
% and thus are easily derived from this one.
%
% \begin{macro}{\@xnthm}
%    For our example arguments, the macro |\@xnthm| must fulfill
%    the following:
%    \begin{itemize}
%       \item
%          Define a new \LaTeX-counter `Lem'
%       \item
%          reset this counter within a |\section|
%       \item
%          define the macro |\theLem|
%       \item
%          define the environment macros |\Lem| and
%          |\endLem| using the current
%          |\theoremstyle| and |\theorem@bodyfont|.
%    \end{itemize}
%    Obviously, all this should happen only if the first argument
%    of |\@xnthm| (i.e.\ \texttt{Lem} in our example) is chosen so
%    as not to conflict with any previously defined commands or
%    environments.  This test is performed by the \LaTeX{} macro
%    |\@ifdefinable|.
%    \begin{macrocode}
\gdef\@xnthm#1#2[#3]{\expandafter\@ifdefinable\csname #1\endcsname
%    \end{macrocode}
%    Therefore, the first argument of |\@ifdefinable| is the
%    expansion (in the example, |\Lem|) of
%    |\csname||#1\endcsname|.  The second argument is
%    executed only if the test has been completed successfully.
%    \begin{macrocode}
   {%
%    \end{macrocode}
%    Now we define the new counter. The names of the \LaTeX{}
%    macros employed should speak for themselves:
% \changes{v2.2b}{1995/10/07}{Use \cs{@newctr}}
% \changes{v2.2c}{1995/11/23}{Use \cs{@newctr} properly.}
%    \begin{macrocode}
    \@definecounter{#1}\@newctr{#1}[#3]%
%    \end{macrocode}
%    Using |\@newctr| will give a proper error message if the counter
%    in |#3| is not defined.
%    In defining `|\theLem|' we must generate the desired macro
%    name by use of |\expandafter| and |\csname|.
%    \begin{macrocode}
    \expandafter\xdef\csname the#1\endcsname
%    \end{macrocode}
%    An |\xdef| is used in order to make the definition global,
%    and to ensure that it contains the replacement texts of
%    |\@thmcountersep| and |\@thmcounter|.\footnote{These
%    two macros can be defined by the document style.  Their default
%    values produce a `\texttt{.}' as separation and an arabic
%    representation of the number.} However, not everything should be
%    expanded. For example, it saves space to use |\thesection|
%    instead of its---at times---lengthy expansion.
%    \begin{macrocode}
      {\expandafter \noexpand \csname the#3\endcsname
       \@thmcountersep \@thmcounter{#1}}%
%    \end{macrocode}
%    Thus with the defaults of \LaTeX{}, |\theLem| would be
%    replaced by the command sequence |\thesection.\arabic{Lem}|.
%
%    We will now look at the definition of the macro which is executed
%    at the beginning of the actual environment (in our example this
%    macro is |\Lem|). It should be noted that we use an
%    ``|\expandafter| trick'' to expand only certain parts of the
%    replacement text at the time of the definition.
%    \begin{macrocode}
    \def\@tempa{\global\@namedef{#1}}%
    \expandafter \@tempa \expandafter{%
%    \end{macrocode}
%    First, the macro that contains the current definitions of
%    |\@begintheorem| and |\@opargtheorem| should be called
%    up.  The name of this macro---as is already known---has the form
%    |\th@|\meta{theorem\ style}; therefore, it must be called by
%    \begin{macrocode}
      \csname th@\the \theorem@style
%    \end{macrocode}
%    In addition the default theorem font should be changeable, i.e.\
%    we have to insert the contents of |\theorem@bodyfont|. For
%    that reason, we expand even further, beyond |\endcsname|,
%    and thus insert the contents of the \textsf{token} register
%    |\theorem@bodyfont| in the replacement text.
%    \begin{macrocode}
            \expandafter \endcsname \the \theorem@bodyfont
%    \end{macrocode}
%    Now it is time to call the macro |\@thm| which takes over
%    the further processing.  It has two arguments: the current
%    counter name (in our example, \texttt{Lem}), and the text of the
%    label (in our example, \texttt{Lemma}).
%    \begin{macrocode}
     \@thm{#1}{#2}}%
%    \end{macrocode}
%    With this, the `sub-definition' is complete.  The macro
%    |\@endtheorem| ends a theorem environment and is, so far, nothing
%    but an |\endtrivlist|.  (Hence it is defined globally, and not
%    within the theorem styles.\footnote{This has to be changed as
%    soon as theorem styles that change \texttt{\bslash @endtheorem}
%    exist.  In such a case, all existing styles must be changed as
%    well since they will have to reset the macro.}) Therefore, we can
%    set it equivalent to the macro that ends the theorem set (in our
%    example, |\endLem|).  However, if some day theorem styles exist
%    that do change |\@endtheorem|, we would have to use the
%    commented-out line instead.
%    \begin{macrocode}
    \global \expandafter \let \csname end#1\endcsname \@endtheorem
%   \global\@namedef{end#1}{\@endtheorem}%
%    \end{macrocode}
%    With these commands all the required definitions are employed,
%    unless the test |\@ifdefinable| has failed.  Therefore, we
%    end the second argument of this macro and with it the definition
%    of |\@xnthm|.
%    \begin{macrocode}
   }}
%    \end{macrocode}
% \end{macro}
%
% \begin{macro}{\@ynthm}
%    The definition of |\@ynthm| is completely analogous. In
%    this case the new counter that is defined is not reset within
%    another counter; thus the definition of |\the...|
%    is simplified:
%    \begin{macrocode}
\gdef\@ynthm#1#2{\expandafter\@ifdefinable\csname #1\endcsname
   {\@definecounter{#1}%
    \expandafter\xdef\csname the#1\endcsname{\@thmcounter{#1}}%
%    \end{macrocode}
%    The rest of the definition corresponds literally to that of
%    |\@xnthm|:
%    \begin{macrocode}
    \def\@tempa{\global\@namedef{#1}}\expandafter \@tempa
     \expandafter{\csname th@\the \theorem@style \expandafter
     \endcsname \the\theorem@bodyfont \@thm{#1}{#2}}%
    \global \expandafter \let \csname end#1\endcsname \@endtheorem}}
%    \end{macrocode}
% \end{macro}
%
% \begin{macro}{\@othm}
%    The definition of |\@othm| does not contain anything new.
%    \begin{macrocode}
\gdef\@othm#1[#2]#3{%
%    \end{macrocode}
% \changes{v2.2b}{95/10/07}{Build in check for counter defined}
%    We do not define a new counter but instead use one that has
%    already been defined.  Thus the only definition we need is that
%    of this pseudo-counter (i.e.\ |\the|\meta{env.\ name}).
%    First we check if |#2| corresponds to a known counter name.
%    \begin{macrocode}
  \expandafter\ifx\csname c@#2\endcsname\relax
   \@nocounterr{#2}%
  \else
   \expandafter\@ifdefinable\csname #1\endcsname
   {\expandafter \xdef \csname the#1\endcsname
     {\expandafter \noexpand \csname the#2\endcsname}%
%    \end{macrocode}
%    All other parts of the definition can be adopted from
%    |\@xnthm|. We have to remember, though, that in this case
%    the name of the current counter and the theorem label have moved
%    to the second and third arguments.
%    \begin{macrocode}
    \def\@tempa{\global\@namedef{#1}}\expandafter \@tempa
     \expandafter{\csname th@\the \theorem@style \expandafter
     \endcsname \the\theorem@bodyfont \@thm{#2}{#3}}%
    \global \expandafter \let \csname end#1\endcsname \@endtheorem}%
  \fi}
%    \end{macrocode}
% \end{macro}
%
% \subsection{Macros that are employed in a {\envfont theorem}
%    environment}
%
% \begin{macro}{\@thm}
%    The macro |\@thm| has to increase the current counter.  Then,
%    depending on whether the environment has (or does not have) an
%    optional argument, it has to branch into either |\@begintheorem|
%    or |\@opargtheorem|.
%    \begin{macrocode}
\gdef\@thm#1#2{\refstepcounter{#1}%
%    \end{macrocode}
%    Now we start a {\envfont trivlist} environment, and give
%    |\@topsep| and |\@topsepadd| the values of the \textsf{skip}
%    registers |\theorempreskipamount| and |\theorempostskipamount|.
%    The value in |\@topsep| is the vertical space that is inserted by
%    the first (and only) |\item| in our |\trivlist| whilst
%    |\@topsepadd| is inserted by |\@endparenv| at the end of that
%    {\envfont trivlist} environment.  By using these registers, we
%    obtain the desired space around a {\envfont theorem} environment.
%    \begin{macrocode}
   \trivlist
   \@topsep \theorempreskipamount               % used by first \item
   \@topsepadd \theorempostskipamount           % used by \@endparenv
%    \end{macrocode}
%    Now we have to test whether an optional argument has been given.
% \changes{v2.0d}{89/07/18}{Forgotten percent added.}
%    \begin{macrocode}
   \@ifnextchar [%
%    \end{macrocode}
%    If there is an optional argument, we will call |\@ythm|, and
%    move the arguments read back into the input stream.
%    \begin{macrocode}
   {\@ythm{#1}{#2}}%
%    \end{macrocode}
%    If not, we call |\@begintheorem|. Its first argument is the
%    name of the theorem set (hence the second argument of
%    |\@thm|). Its second argument is the macro that produces the
%    current number.
%    \begin{macrocode}
   {\@begintheorem{#2}{\csname the#1\endcsname}\ignorespaces}}
%    \end{macrocode}
% \end{macro}
%
% \begin{macro}{\@xthm}
% \begin{macro}{\@ythm}
%    Both these macros were originally called by |\@thm|.  We do
%    not need |\@xthm| anymore, hence we reset it to
%    |\relax|. The definition of |\@ythm| has not changed at
%    all from its definition in \LaTeX. In order to make the macros
%    easier to understand, we will
%    nevertheless present it (commented out).
%    \begin{macrocode}
\global\let\@xthm\relax
% \def\@ythm#1#2[#3]{\@opargbegintheorem{#2}{\csname
%           the#1\endcsname}{#3}\ignorespaces}
%</package>
%    \end{macrocode}
%    The primitive |\ignorespaces| in |\@ythm| and
%    |\@thm| is needed to remove the spaces between the
%    |\begin{...}| and the actual text.
% \end{macro}
% \end{macro}
%
%
% \subsection{Definition of the theorem substyles}
%
% As already pointed out, the theorem substyles, defined below, are
% only loaded when necessary.  Note that all these substyles, except
% \texttt{plain}, have |\slshape| as the default body font.
%
% \subsubsection{The \texttt{plain} style}
%
% \label{dok:th@plain} As the following macros use |@|, we have
% to locally set the |\catcode| of this symbol to ``letter''.
% This happens within a group, so that we do not have to worry about
% which |\catcode| that symbol had before.
%    \begin{macrocode}
%<*thp>
\begingroup \makeatletter
%    \end{macrocode}
% Since we are now within a group, we must make all definitions
% globally.  First we make sure that \texttt{theorem.sty} is loaded.
% This will allow us to use this file as a document style option
% without having to call \texttt{theorem} itself as an option.  At the
% same time, we assure that at least version 2 is loaded, since
% |\theorem@style| was not defined in earlier versions.
%    \begin{macrocode}
\@ifundefined{theorem@style}{% \iffalse meta-comment
%
% Copyright (C) 1993-2024
%
% The LaTeX Project and any individual authors listed elsewhere
% in this file.
%
% This file is part of the Standard LaTeX `Tools Bundle'.
% -------------------------------------------------------
%
% It may be distributed and/or modified under the
% conditions of the LaTeX Project Public License, either version 1.3c
% of this license or (at your option) any later version.
% The latest version of this license is in
%    https://www.latex-project.org/lppl.txt
% and version 1.3c or later is part of all distributions of LaTeX
% version 2005/12/01 or later.
%
% The list of all files belonging to the LaTeX `Tools Bundle' is
% given in the file `manifest.txt'.
%
% \fi
%
%%
%
%
\def\FMithmInfo{2023/07/05 v2.2c Theorem extension package (FMi)}
%
% \ProvidesFile{theorem.dtx}[\FMithmInfo]
%\iffalse   % this is a METACOMMENT !
%%
%% Package `theorem' to use with LaTeX2e
%% Copyright (C) 1989-1995 Frank Mittelbach, all rights reserved.
%<+package>\@ifundefined{theorem@style}{}{\endinput}
%<+package>\NeedsTeXFormat{LaTeX2e}[1995/06/01]
%<+package>\ProvidesPackage{theorem}
%<+package>         [\FMithmInfo]
%
% \fi
%
% \changes{v1.0}{87/05/07}{First draft.}
% \changes{v1.0a}{87/05/07}{`thistheoremstyle' added.}
% \changes{v1.1a}{87/07/29}{New concept: redefine the internal
%                          \protect\LaTeX{} commands `@xnthm' etc.}
% \changes{v1.1b}{87/07/31}{`theoremfont' used to specify font.}
% \changes{v1.2a}{88/01/08}{`@endtheorem' now globally defined.}
% \changes{v1.2b}{88/01/11}{`newtheorem' only allowed in the preamble.}
% \changes{v1.3a}{88/02/14}{`theoremfont' now a toks register.}
%
% \changes{v2.0a}{88/11/16}{New implementation and new documentation.}
% \changes{v2.0a}{88/11/16}{No layout changes but internal modifications
%                           old substyles won't work.}
% \changes{v2.0b}{89/01/30}{`global' used for all definitions.}
% \changes{v2.0c}{89/06/15}{Documentation translated to English.
%                            Thanks to LLL and Nina}
% \changes{v2.1a}{90/02/15}{Support for new font selection.}
% \changes{v2.2a}{94/02/03}{Cleanup for LaTeX2e}
%
% \DoNotIndex{\@endparenv,\@ifdefinable,\@ifnextchar,\@ifundefined}
% \DoNotIndex{\@namedef,\@nameuse,\@spaces,\@tempa,\@warning}
% \DoNotIndex{\ }
% \DoNotIndex{\begingroup,\bfseries}
% \DoNotIndex{\csname}
% \DoNotIndex{\def,\do,\docdate}
% \DoNotIndex{\endcsname,\endgroup,\endinput,\endtrivlist,\expandafter}
% \DoNotIndex{\filedate,\fileversion}
% \DoNotIndex{\gdef,\global}
% \DoNotIndex{\hbox,\hskip}
% \DoNotIndex{\ignorespaces,\input,\itshape,\item}
% \DoNotIndex{\labelsep,\let,\llap}
% \DoNotIndex{\makeatletter}
% \DoNotIndex{\noexpand}
% \DoNotIndex{\refstepcounter,\relax,\rlap}
% \DoNotIndex{\setlength,\slshape,\space,\string,\strut}
% \DoNotIndex{\the,\trivlist,\typeout}
% \DoNotIndex{\vbox}
% \DoNotIndex{\xdef}
%
% \def\deflabel#1{\ttfamily #1\hfill}
% \def\deflist#1{\begin{list}{}{\settowidth\labelwidth{\ttfamily #1}%
%                               \setlength\leftmargin\labelwidth
%                               \addtolength\leftmargin\labelsep
%                               \let\makelabel\deflabel}}
% \def\enddeflist{\end{list}}
% \def\envfont{\normalfont\ttfamily}
%
% \MakeShortVerb{\|}
%
% \newcommand{\TUB}{{\sl TUGboat\/}}
%
% \GetFileInfo{theorem.dtx}
% \title{An Extension of the
%        \LaTeX{} theorem environment\thanks{This file
%        has version number \fileversion, last
%        revised \filedate.}}
% \author{Frank Mittelbach}
%
% \MaintainedByLaTeXTeam{tools}
% \maketitle
%
% \begin{abstract}
%    The macros described in this paper yield an extension of the
%    \LaTeX{} theorem mechanism. It is designed to satisfy the
%    different requirements of various journals. Thus, the
%    layout of the ``theorems'' can be manipulated by determining a
%    ``style''. This article describes not only the use, but
%    also the definition, of the necessary macros.
% \end{abstract}
%
% \section*{Preface to version 2.2}
%
% For \LaTeXe{} this package did not need any fundamental changes. I
% only modified the messages generated so that theorem layout styles
% will show up with the |\listfiles| command and cleaned the section
% on the New Font Selection Scheme since this is now included in
% \LaTeX.
%
% \section*{Preface to version 2.1}
%
% This version is identical to 2.0g described in \TUB\ 10\#3 except
% for some internal defaults which are now set depending on the used
% font selection scheme.
%
% This was done to avoid unpleasant surprises if the new font selection
% scheme is in force. For further details see
% section~\ref{sec:newfonts} and \cite{bk:GMS94}.
%
% \section{Introduction}
%
% For our purposes here, ``theorems'' are labelled enunciations,
% often set off from the main text by extra space and a font change.
% Theorems, corollaries, conjectures, definitions, and remarks
% are all instances of ``theorems''.  The ``header'' of these structures
% is composed of a label (such as \textsc{Theorem} or \textsc{Remark})
% and a number which serializes an item in the sequence of items
% with the same label.
%
% Shortly after the introduction of \LaTeX{} at the Fachbereich
% Mathematik in Mainz, the desire to manipulate the layout of
% ``theorems'' arose.  In Mainz, the following two conventions came
% into general use:
% \begin{enumerate}
%   \item The number of the theorem is shown in the margin.
%   \item There is a line break at
%         the end of the theorem header.
% \end{enumerate}
%
% Additionally, some journals require different formats which depend on
% the ``sort of theorem'': e.g.\ often remarks and definitions are set
% in |\upshape|, while |\itshape| is employed for main theorems.
%
% Confronted with these requirements, a theorem environment was
% developed in Mainz which allows separate determination of the
% layout of the ``theorems sets'', comparable to |\pagestyle|.
%
%
% \section{The user interface}
% \subsection{Defining new theorem sets}
%
% \DescribeMacro\newtheorem
% As in the original \LaTeX{} version, the command |\newtheorem|
% defines a new ``theorem set'' or ``theorem-like structure''.
% Two required arguments name the new environment and give the text
% to be typeset with each instance of the new ``set'', while
% an optional argument determines how
% the ``set'' is enumerated:
% \begin{description}
%    \item[\ttfamily \bslash newtheorem\{foo\}\{bar\}]
%       The theorem set {\envfont foo} (whose name is \texttt{bar})
%       uses its own counter.
%    \item[\ttfamily \bslash newtheorem\{foo2\}{[foo]}\{bar2\}]
%       The theorem set {\envfont foo2} (printed name \texttt{bar2})
%       uses the same counter as the theorem set \texttt{foo}.
%    \item[\ttfamily \bslash newtheorem\{foo3\}\{bar3\}{[section]}]
%       The theorem set {\envfont foo3} (printed name \texttt{bar3}) is
%       enumerated within the counter \texttt{section}, i.e.\ with every
%       new |\section| the enumeration begins again with 1, and
%       the enumeration is composed from the section-number and the
%       theorem counter itself.
% \end{description}
%
% \DescribeMacro\theoremstyle
% Additionally, the command |\theoremstyle| can define the layout
% of various, or all, theorem sets.  It should be noted that any
% theorem set defined by |\newtheorem| is typeset in the
% |\theoremstyle| that is current at the time of the definition.
% Thus, the following
% \begin{verbatim}
%    \theoremstyle{break}        \newtheorem{Cor}{Corollary}
%    \theoremstyle{plain}        \newtheorem{Exa}{Example}[section]
%\end{verbatim}                             ^^A No blank at line start
% leads to the result that the set {\envfont Cor} is formatted in the
% style \texttt{break}, while the set {\envfont Exa} and all the
% following ones are formatted in the style \texttt{plain}, unless
% another |\theoremstyle| follows.  Since the definitions installed by
% |\newtheorem| are global, one also can limit |\theoremstyle| locally
% by grouping braces.
%
% \DescribeMacro\theorembodyfont
% The choice of the font for the theorem body is completely
% independent of the chosen |\theoremstyle|; this has proven to
% be very advantageous.  For example,
% \begin{verbatim}
%   {\theorembodyfont{\upshape}          \newtheorem{Rem}{Remark}}
%\end{verbatim}
% defines a theorem set {\envfont Rem}, which will be set in
% |\upshape| in the current layout (which in our example is
% \texttt{plain}).  As with |\theoremstyle|, the |\theorembodyfont|
% chosen is that current at the time of |\newtheorem|.  If
% |\theorembodyfont| is not specified or one defines
% |\theorembodyfont{}|, then the font used will be that defined by the
% |\theoremstyle|.
%
% \DescribeMacro\theoremheaderfont It is also possible to customize
% the font used for the theorem headers.  This is, however, a global
% declaration, and therefore there should be at most one
% |\theoremheaderfont| declaration in the preamble.\footnote{If it is
% actually necessary to have different header fonts, one has to define
% new theorem styles (substituting the desired font) or specify the
% information directly in the \texttt{\bslash newtheorem} declaration
% (the unclean variant).}
%
% \DescribeMacro\theorempreskipamount
% \DescribeMacro\theorempostskipamount Two additional parameters
% affect the vertical space around the theorem environments:
% |\theorempreskipamount| and |\theorempostskipamount| define,
% respectively, the spacing before and after such an environment.
% These parameters apply for all theorem sets and can be manipulated
% with the ordinary length macros.  They are rubber lengths
% (`\textsf{skips}'), and therefore can contain \texttt{plus} and
% \texttt{minus} parts.
%
% Since the definition of theorem sets should---most sensibly---be
% placed in the preamble, we only allow installation there.
% It is therefore possible to release the memory used here after
% |\begin{document}|, in order to make room for other
% applications.
%
%
% \subsection{Existing theorem styles}
%
% The following theorem styles exist to date:
% \begin{deflist}{marginbreak}
%    \item[plain]
%       This theorem style emulates the original \LaTeX{} definition,
%       except that additionally the parameters
%       |\theorem...skipamount| are used.
%    \item[break]
%       In this style, the theorem header is followed by a line
%       break.
%    \item[marginbreak]
%       The theorem number is set in the margin, and there is a line
%       break as in \texttt{break}.
%    \item[changebreak]
%       Like \texttt{break}, but with header number and text
%       interchanged.
%    \item[change]
%       Header number and text are interchanged, without a line break.
%    \item[margin]
%       The number is set in the left margin, without a line break.
% \end{deflist}
% All styles (except \texttt{plain}) select |\slshape| as the default
% |\theorembodyfont|.
%
%
% \subsection{Examples}
%
% Given the above theorem sets {\envfont Cor}, {\envfont Exa}
% and {\envfont Rem},
% suppose that the preamble also contains the declarations:
% \begin{verbatim}
%    \theoremstyle{marginbreak}   \newtheorem{Lem}[Cor]{Lemma}
%    \theoremstyle{change}
%    \theorembodyfont{\itshape}        \newtheorem{Def}[Cor]{Definition}
%
%    \theoremheaderfont{\scshape}
%\end{verbatim}
% Then the following are some typical examples of the typeset output
% resulting from their use.
%
% \begin{Cor}
%    This is a sentence typeset in the theorem environment {\envfont
%    Cor}.
% \end{Cor}
%
% \begin{Exa}
%    This is a sentence typeset in the theorem environment {\envfont
%    Exa}.
% \end{Exa}
%
% \begin{Rem}
%    This is a sentence typeset in the theorem environment {\envfont
%    Rem}.
% \end{Rem}
%
% \begin{Lem}[Ben User]
%    This is a sentence typeset in the theorem environment {\envfont
%    Lem}.
% \end{Lem}
%
% \begin{Def}[Very impressive Definition]
%    This is a sentence typeset in the theorem environment {\envfont
%    Def}.
% \end{Def}
%
% The last two examples show the effect of the optional argument to
% a theorem environment (it is the text typeset in parentheses).
%
%
% \section{Special Considerations}
% \label{sec:newfonts}
%
% Theoremheader and body are implemented as a unit. This means that
% the |\theoremheaderfont|  will inherit characteristics of the
% |\theorembodyfont| in \LaTeXe. Thus, if for example
% |\theorembodyfont| is |\itshape| and |\theoremheaderfont|
% is |\bfseries| the font selected for the header will have the
% characteristics `bold extended italic'. If this is not desired one
% should set the |\theoremheaderfont| to something like
% \begin{verbatim}
%      \theoremheaderfont{\normalfont\bfseries}
%\end{verbatim}
% i.e.\ supplying all necessary font information explicitly.
%
%
%
% \section{Acknowledgements}
%
% The publication of this set of macros was only possible with the
% help of Christina Busse (translating the manuscript into English),
% Joachim Pense (playing the r\^ole of typist), Chris Rowley
% (looking everything over) and many others providing useful
% suggestions.
%
% \MaybeStop{%
%    \begin{thebibliography}{9}
%       \bibitem{bk:GMS94} \textsc{M.~Goossens}, \textsc{F.~Mittelbach}
%         and \textsc{A.~Samarin}.
%         \newblock The \LaTeX{} Companion.
%         \newblock
%         Addison-Wesley, Reading, Massachusetts, 1994.
%       \bibitem{latex:source} \textsc{Lamport, Leslie.} \newblock
%                               \texttt{latex.tex}, version 2.09, date
%                               Feb. 1990.
%    \end{thebibliography}
%    \PrintIndex
%    \PrintChanges}
%
%
% \section{The documentation driver file}
%
% The next bit of code contains the documentation driver file for
% \TeX{}, i.e., the file that will produce the documentation you are
% currently reading. It will be extracted from this file by the
% \texttt{docstrip} program. Since it is the first code in the file
% one can alternatively process this file directly with \LaTeXe{} to
% obtain the documentation.
%
%    \begin{macrocode}
%<*driver>
\documentclass{ltxdoc}

\usepackage{theorem}

%<+driver>% The next few lines define theorem sets which are used
%<+driver>% in the example section of the documentation.

\theoremstyle{break}       \newtheorem{Cor}{Corollary}
\theoremstyle{plain}       \newtheorem{Exa}{Example}[section]
{\theorembodyfont{\upshape}\newtheorem{Rem}{Remark}}
\theoremstyle{marginbreak} \newtheorem{Lem}[Cor]{Lemma}
\theoremstyle{change}
\theorembodyfont{\itshape} \newtheorem{Def}[Cor]{Definition}

\theoremheaderfont{\scshape}

\RecordChanges

\begin{document}
   \DocInput{theorem.dtx}
\end{document}
%</driver>
%    \end{macrocode}
%
% \section{Definition of the Macros}
%
% If the file has been loaded before, we abort immediately. If not the
% package announces itself (this is actually done at the very top of
% the file---the way it is done isn't good style so don't copy it).
% \changes{v2.0e}{89/07/19}{Spaces removed from `typeout'.}
% \changes{v2.2b}{95/11/19}{Announce moved to top of file}
%    \begin{macrocode}
%<*package>
%\@ifundefined{theorem@style}{}{\endinput}
%\def\FMithmInfo{1995/11/19 v2.2b Theorem extension package (FMi)}
%\ProvidesPackage{theorem}[\FMithmInfo]
%    \end{macrocode}
%
%
% \subsection{Definition of theorem styles and fonts}
%
% All the definitions in this file are done globally to allow
% inputting this file inside a group.
%
% \begin{macro}{\theoremstyle}
%    Before a theorem style can be installed, the chosen style must be
%    known. For that reason, we must test to see that
%    |\th@|\meta{style} is known or, more precisely, that it is
%    different from |\relax|. If the style is not known then
%    |\th@plain| is used.
%    \begin{macrocode}
\gdef\theoremstyle#1{%
   \@ifundefined{th@#1}{\@warning
          {Unknown theoremstyle `#1'. Using `plain'}%
          \theorem@style{plain}}%
%    \end{macrocode}
%    We save the theorem style to be used in the token register
%    |\theorem@style|.
%    \begin{macrocode}
      {\theorem@style{#1}}%
%    \end{macrocode}
%    Now we ``evaluate'' the theorem style: this means, we call the
%    macro |\th@|\meta{style} which will activate the relevant
%    definitions which are contained in a separate file.  This is done
%    in a group to suppress changes to the current font.  This could
%    otherwise pose problems together with the new font selection
%    scheme\footnote{When I printed the original article using the new
%    font selection scheme I ended with a document with slanted
%    typefaces (text headings and all) simply because one of the
%    theorem styles used \texttt{\bslash sl} at toplevel.} if the
%    |\th@|\meta{style} is evaluated a second time.
% \changes{v2.1a}{90/02/15}{Group added around `th@..}
% \changes{v2.1a}{90/02/15}{Reseting of `..@begintheorem removed.}
%    \begin{macrocode}
      \begingroup
        \csname th@\the\theorem@style \endcsname
      \endgroup}
%    \end{macrocode}
% \end{macro}
%
%
% \begin{macro}{\@begintheorem}
% \changes{v2.1a}{90/02/15}{Relaxed at toplevel.}
% \begin{macro}{\@opargbegintheorem}
% \changes{v2.1a}{90/02/15}{Relaxed at toplevel.}
%    We reset |\@begintheorem| and |\@opargbegintheorem| to
%    |\relax| since these commands are no longer necessary at
%    toplevel. This will save a few tokens.
%    \begin{macrocode}
\global\let\@begintheorem\relax
\global\let\@opargbegintheorem\relax
%    \end{macrocode}
% \end{macro}
% \end{macro}
%
% \begin{macro}{\theorem@style}
%    Obviously the \textsf{token} register used above has to be
%    allocated. To assure the utmost compatibility with the
%    original \LaTeX{} definition, we set the default theorem style to
%    \texttt{plain}, which implements the usual \LaTeX{} convention.
% \changes{v2.0f}{89/07/20}{Unnecessary `global' removed.}
%    \begin{macrocode}
\newtoks\theorem@style
\global\theorem@style{plain}
%    \end{macrocode}
% \end{macro}
%
% \begin{macro}{\theorembodyfont}
% \begin{macro}{\theorem@bodyfont}
%    If the |\theorembodyfont| is set by the user then it should not
%    interact with the default font set in the theorem style. When the
%    new font selection is in force this may happen if, for example,
%    the default is |\itshape| and the new |\theorembodyfont| is
%    |\sffamily|. So we add a |\reset@font| command in front of the
%    user definition.
% \changes{v2.0f}{89/07/20}{Unnecessary `global' removed.}
% \changes{v2.0g}{89/09/19}{`theoremfont to `theorembodyfont.}
% \changes{v2.1c}{92/06/22}{`reset@font added and logic changed.}
% \changes{v2.1d}{92/10/28}{Reset to empty if arg is empty}
%    \begin{macrocode}
\gdef\theorembodyfont#1{%
%    \end{macrocode}
%    We check if the argument supplied is empty and if so put nothing
%    into the |\theorem@bodyfont| token register to allow for
%    |\theorembodyfont{}| as a mean of using the default of the
%    current |\theoremstyle|.
%    \begin{macrocode}
   \def\@tempa{#1}%
   \ifx\@tempa\@empty
    \theorem@bodyfont{}%
   \else
    \theorem@bodyfont{\reset@font#1}%
   \fi
   }
\newtoks\theorem@bodyfont
\global\theorem@bodyfont{}
%    \end{macrocode}
% \end{macro}
% \end{macro}
%
%
% \begin{macro}{\theoremheaderfont}
% \changes{v2.0g}{89/09/19}{Macro added.}
%    The font for the theorem headers is handled differently because
%    this definition applies to all theorem styles.
%    \begin{macrocode}
\gdef\theoremheaderfont#1{\gdef\theorem@headerfont{#1}%
%    \end{macrocode}
%    After using the macro once it is redefined to produce an error
%    message.
%    \begin{macrocode}
       \gdef\theoremheaderfont##1{%
        \typeout{\string\theoremheaderfont\space should be used
                 only once.}}}
%    \end{macrocode}
% \end{macro}
%
% \begin{macro}{\theorem@headerfont}
% \changes{v2.0g}{89/09/19}{Macro added.}
%    To set the |\theorem@headerfont| default we first test if
%    the new fontselection scheme is in force.
% \changes{v2.1a}{90/02/15}{Support for new font selection.}
%    \begin{macrocode}
\ifx\upshape\undefined
%    \end{macrocode}
%    If not we define it to expand into |\bfseries|. We don't use
%    |\let| just in case a following style option redefines this
%    macro.
%    \begin{macrocode}
\gdef\theorem@headerfont{\bfseries}
%    \end{macrocode}
%    Otherwise we reset the current shape before calling |\bfseries|.
%    \begin{macrocode}
\else \gdef\theorem@headerfont{\normalfont\bfseries}\fi
%    \end{macrocode}
% \end{macro}
%
% \begin{macro}{\th@plain}
% \begin{macro}{\th@break}
% \begin{macro}{\th@marginbreak}
% \begin{macro}{\th@changebreak}
% \begin{macro}{\th@change}
% \begin{macro}{\th@margin}
%    The different styles are defined in macros such as
%    |\th@plain|.  Since memory space is precious in
%    ``non-Big-versions'', we have to avoid offering too many unused
%    definitions.  Therefore we define these styles in separate files
%    that can be loaded on demand.  Thus the commands themselves only
%    load these files. We use |\@input@|, a \LaTeXe{} internal command
%    that ensures that the file will be listed with |\listfiles|.
% \changes{v2.2a}{94/02/02}{Use `@input@ to load theorem layout files}
%    \begin{macrocode}
\gdef\th@plain{\@input@{thp.sty}}
\gdef\th@break{\@input@{thb.sty}}
\gdef\th@marginbreak{\@input@{thmb.sty}}
\gdef\th@changebreak{\@input@{thcb.sty}}
\gdef\th@change{\@input@{thc.sty}}
\gdef\th@margin{\@input@{thm.sty}}
%    \end{macrocode}
%    This list will be expanded when new styles become
%    available.  For testing,  just append new theorem substyles
%    as document options.
% \end{macro}
% \end{macro}
% \end{macro}
% \end{macro}
% \end{macro}
% \end{macro}
%
%
%
% \subsection{Definition of a new theorem set}
%
% As already pointed out, a new theorem environment can be defined in
% three different ways:
% \begin{verbatim}
%    \newtheorem{Lem}{Lemma}
%    \newtheorem{Lem}{Lemma}[section]
%    \newtheorem{Lem}[Theorem]{Lemma}
%\end{verbatim}
% The function of the macro |\newtheorem| is to recognize these cases
% and then to branch into one of the three macros |\@ynthm|, |\@xnthm|
% or |\@othm|.  This mechanism is adopted unchanged from
% \cite{latex:source}; the essential point here is that, for example,
% in the second case, the arguments \texttt{Lem}, \texttt{Lemma} and
% \texttt{section} are passed over to the macro |\@xnthm|.
%
% We inspect this case first because the others present fewer problems,
% and thus are easily derived from this one.
%
% \begin{macro}{\@xnthm}
%    For our example arguments, the macro |\@xnthm| must fulfill
%    the following:
%    \begin{itemize}
%       \item
%          Define a new \LaTeX-counter `Lem'
%       \item
%          reset this counter within a |\section|
%       \item
%          define the macro |\theLem|
%       \item
%          define the environment macros |\Lem| and
%          |\endLem| using the current
%          |\theoremstyle| and |\theorem@bodyfont|.
%    \end{itemize}
%    Obviously, all this should happen only if the first argument
%    of |\@xnthm| (i.e.\ \texttt{Lem} in our example) is chosen so
%    as not to conflict with any previously defined commands or
%    environments.  This test is performed by the \LaTeX{} macro
%    |\@ifdefinable|.
%    \begin{macrocode}
\gdef\@xnthm#1#2[#3]{\expandafter\@ifdefinable\csname #1\endcsname
%    \end{macrocode}
%    Therefore, the first argument of |\@ifdefinable| is the
%    expansion (in the example, |\Lem|) of
%    |\csname||#1\endcsname|.  The second argument is
%    executed only if the test has been completed successfully.
%    \begin{macrocode}
   {%
%    \end{macrocode}
%    Now we define the new counter. The names of the \LaTeX{}
%    macros employed should speak for themselves:
% \changes{v2.2b}{1995/10/07}{Use \cs{@newctr}}
% \changes{v2.2c}{1995/11/23}{Use \cs{@newctr} properly.}
%    \begin{macrocode}
    \@definecounter{#1}\@newctr{#1}[#3]%
%    \end{macrocode}
%    Using |\@newctr| will give a proper error message if the counter
%    in |#3| is not defined.
%    In defining `|\theLem|' we must generate the desired macro
%    name by use of |\expandafter| and |\csname|.
%    \begin{macrocode}
    \expandafter\xdef\csname the#1\endcsname
%    \end{macrocode}
%    An |\xdef| is used in order to make the definition global,
%    and to ensure that it contains the replacement texts of
%    |\@thmcountersep| and |\@thmcounter|.\footnote{These
%    two macros can be defined by the document style.  Their default
%    values produce a `\texttt{.}' as separation and an arabic
%    representation of the number.} However, not everything should be
%    expanded. For example, it saves space to use |\thesection|
%    instead of its---at times---lengthy expansion.
%    \begin{macrocode}
      {\expandafter \noexpand \csname the#3\endcsname
       \@thmcountersep \@thmcounter{#1}}%
%    \end{macrocode}
%    Thus with the defaults of \LaTeX{}, |\theLem| would be
%    replaced by the command sequence |\thesection.\arabic{Lem}|.
%
%    We will now look at the definition of the macro which is executed
%    at the beginning of the actual environment (in our example this
%    macro is |\Lem|). It should be noted that we use an
%    ``|\expandafter| trick'' to expand only certain parts of the
%    replacement text at the time of the definition.
%    \begin{macrocode}
    \def\@tempa{\global\@namedef{#1}}%
    \expandafter \@tempa \expandafter{%
%    \end{macrocode}
%    First, the macro that contains the current definitions of
%    |\@begintheorem| and |\@opargtheorem| should be called
%    up.  The name of this macro---as is already known---has the form
%    |\th@|\meta{theorem\ style}; therefore, it must be called by
%    \begin{macrocode}
      \csname th@\the \theorem@style
%    \end{macrocode}
%    In addition the default theorem font should be changeable, i.e.\
%    we have to insert the contents of |\theorem@bodyfont|. For
%    that reason, we expand even further, beyond |\endcsname|,
%    and thus insert the contents of the \textsf{token} register
%    |\theorem@bodyfont| in the replacement text.
%    \begin{macrocode}
            \expandafter \endcsname \the \theorem@bodyfont
%    \end{macrocode}
%    Now it is time to call the macro |\@thm| which takes over
%    the further processing.  It has two arguments: the current
%    counter name (in our example, \texttt{Lem}), and the text of the
%    label (in our example, \texttt{Lemma}).
%    \begin{macrocode}
     \@thm{#1}{#2}}%
%    \end{macrocode}
%    With this, the `sub-definition' is complete.  The macro
%    |\@endtheorem| ends a theorem environment and is, so far, nothing
%    but an |\endtrivlist|.  (Hence it is defined globally, and not
%    within the theorem styles.\footnote{This has to be changed as
%    soon as theorem styles that change \texttt{\bslash @endtheorem}
%    exist.  In such a case, all existing styles must be changed as
%    well since they will have to reset the macro.}) Therefore, we can
%    set it equivalent to the macro that ends the theorem set (in our
%    example, |\endLem|).  However, if some day theorem styles exist
%    that do change |\@endtheorem|, we would have to use the
%    commented-out line instead.
%    \begin{macrocode}
    \global \expandafter \let \csname end#1\endcsname \@endtheorem
%   \global\@namedef{end#1}{\@endtheorem}%
%    \end{macrocode}
%    With these commands all the required definitions are employed,
%    unless the test |\@ifdefinable| has failed.  Therefore, we
%    end the second argument of this macro and with it the definition
%    of |\@xnthm|.
%    \begin{macrocode}
   }}
%    \end{macrocode}
% \end{macro}
%
% \begin{macro}{\@ynthm}
%    The definition of |\@ynthm| is completely analogous. In
%    this case the new counter that is defined is not reset within
%    another counter; thus the definition of |\the...|
%    is simplified:
%    \begin{macrocode}
\gdef\@ynthm#1#2{\expandafter\@ifdefinable\csname #1\endcsname
   {\@definecounter{#1}%
    \expandafter\xdef\csname the#1\endcsname{\@thmcounter{#1}}%
%    \end{macrocode}
%    The rest of the definition corresponds literally to that of
%    |\@xnthm|:
%    \begin{macrocode}
    \def\@tempa{\global\@namedef{#1}}\expandafter \@tempa
     \expandafter{\csname th@\the \theorem@style \expandafter
     \endcsname \the\theorem@bodyfont \@thm{#1}{#2}}%
    \global \expandafter \let \csname end#1\endcsname \@endtheorem}}
%    \end{macrocode}
% \end{macro}
%
% \begin{macro}{\@othm}
%    The definition of |\@othm| does not contain anything new.
%    \begin{macrocode}
\gdef\@othm#1[#2]#3{%
%    \end{macrocode}
% \changes{v2.2b}{95/10/07}{Build in check for counter defined}
%    We do not define a new counter but instead use one that has
%    already been defined.  Thus the only definition we need is that
%    of this pseudo-counter (i.e.\ |\the|\meta{env.\ name}).
%    First we check if |#2| corresponds to a known counter name.
%    \begin{macrocode}
  \expandafter\ifx\csname c@#2\endcsname\relax
   \@nocounterr{#2}%
  \else
   \expandafter\@ifdefinable\csname #1\endcsname
   {\expandafter \xdef \csname the#1\endcsname
     {\expandafter \noexpand \csname the#2\endcsname}%
%    \end{macrocode}
%    All other parts of the definition can be adopted from
%    |\@xnthm|. We have to remember, though, that in this case
%    the name of the current counter and the theorem label have moved
%    to the second and third arguments.
%    \begin{macrocode}
    \def\@tempa{\global\@namedef{#1}}\expandafter \@tempa
     \expandafter{\csname th@\the \theorem@style \expandafter
     \endcsname \the\theorem@bodyfont \@thm{#2}{#3}}%
    \global \expandafter \let \csname end#1\endcsname \@endtheorem}%
  \fi}
%    \end{macrocode}
% \end{macro}
%
% \subsection{Macros that are employed in a {\envfont theorem}
%    environment}
%
% \begin{macro}{\@thm}
%    The macro |\@thm| has to increase the current counter.  Then,
%    depending on whether the environment has (or does not have) an
%    optional argument, it has to branch into either |\@begintheorem|
%    or |\@opargtheorem|.
%    \begin{macrocode}
\gdef\@thm#1#2{\refstepcounter{#1}%
%    \end{macrocode}
%    Now we start a {\envfont trivlist} environment, and give
%    |\@topsep| and |\@topsepadd| the values of the \textsf{skip}
%    registers |\theorempreskipamount| and |\theorempostskipamount|.
%    The value in |\@topsep| is the vertical space that is inserted by
%    the first (and only) |\item| in our |\trivlist| whilst
%    |\@topsepadd| is inserted by |\@endparenv| at the end of that
%    {\envfont trivlist} environment.  By using these registers, we
%    obtain the desired space around a {\envfont theorem} environment.
%    \begin{macrocode}
   \trivlist
   \@topsep \theorempreskipamount               % used by first \item
   \@topsepadd \theorempostskipamount           % used by \@endparenv
%    \end{macrocode}
%    Now we have to test whether an optional argument has been given.
% \changes{v2.0d}{89/07/18}{Forgotten percent added.}
%    \begin{macrocode}
   \@ifnextchar [%
%    \end{macrocode}
%    If there is an optional argument, we will call |\@ythm|, and
%    move the arguments read back into the input stream.
%    \begin{macrocode}
   {\@ythm{#1}{#2}}%
%    \end{macrocode}
%    If not, we call |\@begintheorem|. Its first argument is the
%    name of the theorem set (hence the second argument of
%    |\@thm|). Its second argument is the macro that produces the
%    current number.
%    \begin{macrocode}
   {\@begintheorem{#2}{\csname the#1\endcsname}\ignorespaces}}
%    \end{macrocode}
% \end{macro}
%
% \begin{macro}{\@xthm}
% \begin{macro}{\@ythm}
%    Both these macros were originally called by |\@thm|.  We do
%    not need |\@xthm| anymore, hence we reset it to
%    |\relax|. The definition of |\@ythm| has not changed at
%    all from its definition in \LaTeX. In order to make the macros
%    easier to understand, we will
%    nevertheless present it (commented out).
%    \begin{macrocode}
\global\let\@xthm\relax
% \def\@ythm#1#2[#3]{\@opargbegintheorem{#2}{\csname
%           the#1\endcsname}{#3}\ignorespaces}
%</package>
%    \end{macrocode}
%    The primitive |\ignorespaces| in |\@ythm| and
%    |\@thm| is needed to remove the spaces between the
%    |\begin{...}| and the actual text.
% \end{macro}
% \end{macro}
%
%
% \subsection{Definition of the theorem substyles}
%
% As already pointed out, the theorem substyles, defined below, are
% only loaded when necessary.  Note that all these substyles, except
% \texttt{plain}, have |\slshape| as the default body font.
%
% \subsubsection{The \texttt{plain} style}
%
% \label{dok:th@plain} As the following macros use |@|, we have
% to locally set the |\catcode| of this symbol to ``letter''.
% This happens within a group, so that we do not have to worry about
% which |\catcode| that symbol had before.
%    \begin{macrocode}
%<*thp>
\begingroup \makeatletter
%    \end{macrocode}
% Since we are now within a group, we must make all definitions
% globally.  First we make sure that \texttt{theorem.sty} is loaded.
% This will allow us to use this file as a document style option
% without having to call \texttt{theorem} itself as an option.  At the
% same time, we assure that at least version 2 is loaded, since
% |\theorem@style| was not defined in earlier versions.
%    \begin{macrocode}
\@ifundefined{theorem@style}{\input{theorem.sty}}{}
\ProvidesFile{thp.sty}
             [\FMithmInfo]
%    \end{macrocode}
%
%
% \begin{macro}{\th@plain}
% \begin{macro}{\@begintheorem}
% \begin{macro}{\@opargbegintheorem}
%    |\theoremstyle{plain}| corresponds to the original
%    definition, except that the distances to the surrounding text are
%    determined by the parameters |\theorempreskipamount| and
%    |\theorempostskipamount|.
%    First we set the default body font.
%    \begin{macrocode}
\gdef\th@plain{\normalfont\itshape
%    \end{macrocode}
%    Then we define |\@begintheorem| and
%    |\@opargbegintheorem|. These two macros define how the
%    header of a theorem is typeset.  |\@opargbegintheorem| will
%    be called if a \textsf{theorem} environment with an optional
%    argument is encountered; otherwise, the header is constructed by
%    calling |\@begintheorem|.  If one of these macros is
%    executed, we are within a \textsf{trivlist} environment started by
%    |\@thm|.  So the theorem header is produced with an
%    |\item| command.
%
%    Instead of specifying the header font directly, all standard
%    theorem styles use the |\theorem@headerfont| macro to allow
%    customization.  The extra space (|\labelsep|) is necessary
%    because of problems in the \textsf{trivlist} environment.
%    \begin{macrocode}
  \def\@begintheorem##1##2{%
        \item[\hskip\labelsep \theorem@headerfont ##1\ ##2]}%
%    \end{macrocode}
%    The definition of |\@opargbegintheorem| is completely
%    analogous. The only difference is the fact that there exists a
%    third argument (which is the optional parameter of the
%    environment and contains additional information about the
%    theorem).  Customarily we enclose it in parentheses.
%    \begin{macrocode}
\def\@opargbegintheorem##1##2##3{%
   \item[\hskip\labelsep \theorem@headerfont ##1\ ##2\ (##3)]}}
%    \end{macrocode}
% \end{macro}
% \end{macro}
% \end{macro}
%
% \noindent
% We conclude with an |\endgroup| to restore the
% |\catcode| of |@|.
%    \begin{macrocode}
\endgroup
%</thp>
%    \end{macrocode}
%
%
% \subsubsection{The \texttt{break} style}
%
% This style option is stored in the file |thb.sty|.
% For the next two lines see the documentation for |\th@plain| on
% page~\pageref{dok:th@plain}.
%    \begin{macrocode}
%<*thb>
\begingroup \makeatletter
\@ifundefined{theorem@style}{\input{theorem.sty}}{}
\ProvidesFile{thb.sty}
             [\FMithmInfo]
%    \end{macrocode}
%
% \begin{macro}{\th@break}
%    |\theoremstyle{break}| produces a line break after the name
%    of the theorem.  The font is |\slshape|.  Hence, we define
%    |\th@break| as follows:
%    \begin{macrocode}
\gdef\th@break{\normalfont\slshape
  \def\@begintheorem##1##2{\item[%
%    \end{macrocode}
%    We run into the following problem: it is not possible to create
%    the header with |\item[|\meta{title}|]| and then start
%    a new line by, for example, |\mbox{}\\|. Such a definition
%    will fail whenever a list environment follows immediately.  With
%    the above construction, the |\mbox{}| causes the switch
%    |@inlabel| (cf.\ definition of |\list| and
%    |\trivlist| in \cite{latex:source}) to be set to \texttt{false}
%    and so the following list will insert additional vertical
%    space (|\topskip|). This is quite annoying. Therefore, we
%    create the line break within the |\item|. In order to ensure
%    that the text will begin at the proper position in the following
%    line, we simply pretend that the label does not take any
%    room.\footnote{This will lead to problems whenever very high
%    symbols occurring in the line tower into the
%    heading. So, something else has to be done here sometime.}
%    \begin{macrocode}
       \rlap{\vbox{\hbox{\hskip \labelsep\theorem@headerfont ##1\ ##2}%
                   \hbox{\strut}}}]}%
%    \end{macrocode}
%    Again, the definition of |\@opargbegintheorem| is
%    completely analogous.
%    \begin{macrocode}
\def\@opargbegintheorem##1##2##3{%
  \item[\rlap{\vbox{\hbox{\hskip \labelsep \theorem@headerfont
                     ##1\ ##2\ (##3)}%
                    \hbox{\strut}}}]}}
\endgroup
%</thb>
%    \end{macrocode}
% \end{macro}
%
%
% \subsubsection{The \texttt{changebreak} style}
%
%    \begin{macrocode}
% This style option is stored in the file |thcb.sty|.
%    \begin{macrocode}
%<*thcb>
\begingroup \makeatletter
\@ifundefined{theorem@style}{\input{theorem.sty}}{}
\ProvidesFile{thcb.sty}
             [\FMithmInfo]
%    \end{macrocode}
%
% \begin{macro}{\th@changebreak}
%    The \texttt{change-break} theorem style is like \texttt{break}
%    but with interchange of theorem name and theorem number.  Thus we
%    define |\th@changebreak| as follows:
%    \begin{macrocode}
\gdef\th@changebreak{\normalfont\slshape
  \def\@begintheorem##1##2{\item
       [\rlap{\vbox{\hbox{\hskip\labelsep \theorem@headerfont ##2\ ##1}%
                    \hbox{\strut}}}]}%
\def\@opargbegintheorem##1##2##3{%
  \item[\rlap{\vbox{\hbox{\hskip\labelsep \theorem@headerfont
                     ##2\ ##1\ (##3)}%
                    \hbox{\strut}}}]}}
\endgroup
%</thcb>
%    \end{macrocode}
% \end{macro}
%
%
% \subsubsection{The \texttt{change} style}
%
% This style option is stored in the file |thc.sty|.
%    \begin{macrocode}
%<*thc>
\begingroup \makeatletter
\@ifundefined{theorem@style}{\input{theorem.sty}}{}
\ProvidesFile{thc.sty}
             [\FMithmInfo]
%    \end{macrocode}
%
% \begin{macro}{\th@change}
%    The \texttt{change} theorem style corresponds to the
%    \texttt{change break} style without a linebreak after the header.
%    To say it in another way, it's the same as the \texttt{plain}
%    style but with number and name interchanged and |\slshape| as the
%    default font.
%    \begin{macrocode}
\gdef\th@change{\normalfont\slshape
  \def\@begintheorem##1##2{\item
       [\hskip\labelsep \theorem@headerfont ##2\ ##1]}%
\def\@opargbegintheorem##1##2##3{%
  \item[\hskip\labelsep \theorem@headerfont ##2\ ##1\ (##3)]}}
\endgroup
%</thc>
%    \end{macrocode}
% \end{macro}
%
%
% \subsubsection{The \texttt{marginbreak} style}
%
% This style option is the one used most often at Mainz.  It is saved
% in the file |thmb.sty|.
%    \begin{macrocode}
%<*thmb>
\begingroup \makeatletter
\@ifundefined{theorem@style}{\input{theorem.sty}}{}
\ProvidesFile{thmb.sty}
             [\FMithmInfo]
%    \end{macrocode}
%
% \begin{macro}{\th@marginbreak}
%    The \texttt{margin break} style is nearly the same as the
%    \texttt{change break} style. The only difference is the placement
%    of the theorem number.  We use |\llap| to place it in the left
%    margin.
%
%    In this style |\labelsep| denotes the separation between the
%    number and the text.
%    \begin{macrocode}
\gdef\th@marginbreak{\normalfont\slshape
  \def\@begintheorem##1##2{\item
       [\rlap{\vbox{\theorem@headerfont
                    \hbox{\llap{##2}\hskip\labelsep ##1}%
                    \hbox{\strut}}}]}%
\def\@opargbegintheorem##1##2##3{%
  \item[\rlap{\vbox{\theorem@headerfont
                    \hbox{\llap{##2}\hskip\labelsep ##1\ (##3)}%
                    \hbox{\strut}}}]}}
\endgroup
%</thmb>
%    \end{macrocode}
% \changes{v2.1c}{90/02/17}{`endgroup outside macrocode env.}
% \end{macro}
%
%
%
% \subsubsection{The \texttt{margin} style}
%
% This style option is stored in the file |thm.sty|.
%    \begin{macrocode}
%<*thm>
\begingroup \makeatletter
\@ifundefined{theorem@style}{\input{theorem.sty}}{}
\ProvidesFile{thm.sty}
             [\FMithmInfo]
%    \end{macrocode}
%
% \begin{macro}{\th@margin}
%    Again this is only a variant of the theorem styles described
%    above without any new ideas.
%    \begin{macrocode}
\gdef\th@margin{\normalfont\slshape
  \def\@begintheorem##1##2{\item
       [\theorem@headerfont \llap{##2}\hskip\labelsep ##1]}%
\def\@opargbegintheorem##1##2##3{%
  \item[\theorem@headerfont \llap{##2}\hskip\labelsep ##1\ (##3)]}}
\endgroup
%</thm>
%    \end{macrocode}
% \end{macro}
%
%
%
%
% \subsection{Final Definitions}
%
% \begin{macro}{\theorempreskipamount}
% \begin{macro}{\theorempostskipamount}
%    The \textsf{skip} parameters that regulate the vertical empty space
%    before and after the {\envfont theorem} environment have
%    to be allocated as well.
% \changes{v2.0f}{89/07/20}{Unnecessary `global' removed.}
%    \begin{macrocode}
%<*package>
\newskip\theorempreskipamount
\newskip\theorempostskipamount
%    \end{macrocode}
%    Since we have used the same values for all theorem sets, we now can
%    assign them.
%    \begin{macrocode}
\global\setlength\theorempreskipamount{12pt plus 5pt minus 3pt}
\global\setlength\theorempostskipamount{8pt plus 3pt minus 1.5pt}
%    \end{macrocode}
% \end{macro}
% \end{macro}
%
%
% \begin{macro}{\@endtheorem}
%    The same holds for the macro |\@endtheorem|, which ends a
%    {\envfont theorem} environment. Since it is the same for all
%    theorem sets, it is removed from the macros |\th@|\meta{style}.
%    It simply ends the {\envfont trivlist} environment, which was
%    begun in |\@thm|.
%    \begin{macrocode}
\global\let\@endtheorem=\endtrivlist
%    \end{macrocode}
% \end{macro}
%
%
%
% \begin{macro}{\@preamblecmds}
%    All macros defined above are to be used only in the preamble.
%    Therefore, we insert them in |\@preamblecmds| which will disable
%    them at begin document. This is done by the internal \LaTeXe{}
%    command |\@onlypreamble|.
% \changes{v2.0g}{89/09/19}{`theoremheaderfont and `theorembodyfont
%    added.}
% \changes{v2.2a}{94/02/03}{Use \LaTeXe{} interface}
%    \begin{macrocode}
\@onlypreamble\@xnthm
\@onlypreamble\@ynthm
\@onlypreamble\@othm
\@onlypreamble\newtheorem
\@onlypreamble\theoremstyle
\@onlypreamble\theorembodyfont
\@onlypreamble\theoremheaderfont
%    \end{macrocode}
% \end{macro}
%
% Finally we declare the \texttt{plain} theorem style to be the default.
% \changes{v2.1b}{90/06/04}{Fixed missing `item by declaring default
%    style.}
%    \begin{macrocode}
\theoremstyle{plain}
%</package>
%    \end{macrocode}
%
% \Finale

\endinput
}{}
\ProvidesFile{thp.sty}
             [\FMithmInfo]
%    \end{macrocode}
%
%
% \begin{macro}{\th@plain}
% \begin{macro}{\@begintheorem}
% \begin{macro}{\@opargbegintheorem}
%    |\theoremstyle{plain}| corresponds to the original
%    definition, except that the distances to the surrounding text are
%    determined by the parameters |\theorempreskipamount| and
%    |\theorempostskipamount|.
%    First we set the default body font.
%    \begin{macrocode}
\gdef\th@plain{\normalfont\itshape
%    \end{macrocode}
%    Then we define |\@begintheorem| and
%    |\@opargbegintheorem|. These two macros define how the
%    header of a theorem is typeset.  |\@opargbegintheorem| will
%    be called if a \textsf{theorem} environment with an optional
%    argument is encountered; otherwise, the header is constructed by
%    calling |\@begintheorem|.  If one of these macros is
%    executed, we are within a \textsf{trivlist} environment started by
%    |\@thm|.  So the theorem header is produced with an
%    |\item| command.
%
%    Instead of specifying the header font directly, all standard
%    theorem styles use the |\theorem@headerfont| macro to allow
%    customization.  The extra space (|\labelsep|) is necessary
%    because of problems in the \textsf{trivlist} environment.
%    \begin{macrocode}
  \def\@begintheorem##1##2{%
        \item[\hskip\labelsep \theorem@headerfont ##1\ ##2]}%
%    \end{macrocode}
%    The definition of |\@opargbegintheorem| is completely
%    analogous. The only difference is the fact that there exists a
%    third argument (which is the optional parameter of the
%    environment and contains additional information about the
%    theorem).  Customarily we enclose it in parentheses.
%    \begin{macrocode}
\def\@opargbegintheorem##1##2##3{%
   \item[\hskip\labelsep \theorem@headerfont ##1\ ##2\ (##3)]}}
%    \end{macrocode}
% \end{macro}
% \end{macro}
% \end{macro}
%
% \noindent
% We conclude with an |\endgroup| to restore the
% |\catcode| of |@|.
%    \begin{macrocode}
\endgroup
%</thp>
%    \end{macrocode}
%
%
% \subsubsection{The \texttt{break} style}
%
% This style option is stored in the file |thb.sty|.
% For the next two lines see the documentation for |\th@plain| on
% page~\pageref{dok:th@plain}.
%    \begin{macrocode}
%<*thb>
\begingroup \makeatletter
\@ifundefined{theorem@style}{% \iffalse meta-comment
%
% Copyright (C) 1993-2024
%
% The LaTeX Project and any individual authors listed elsewhere
% in this file.
%
% This file is part of the Standard LaTeX `Tools Bundle'.
% -------------------------------------------------------
%
% It may be distributed and/or modified under the
% conditions of the LaTeX Project Public License, either version 1.3c
% of this license or (at your option) any later version.
% The latest version of this license is in
%    https://www.latex-project.org/lppl.txt
% and version 1.3c or later is part of all distributions of LaTeX
% version 2005/12/01 or later.
%
% The list of all files belonging to the LaTeX `Tools Bundle' is
% given in the file `manifest.txt'.
%
% \fi
%
%%
%
%
\def\FMithmInfo{2023/07/05 v2.2c Theorem extension package (FMi)}
%
% \ProvidesFile{theorem.dtx}[\FMithmInfo]
%\iffalse   % this is a METACOMMENT !
%%
%% Package `theorem' to use with LaTeX2e
%% Copyright (C) 1989-1995 Frank Mittelbach, all rights reserved.
%<+package>\@ifundefined{theorem@style}{}{\endinput}
%<+package>\NeedsTeXFormat{LaTeX2e}[1995/06/01]
%<+package>\ProvidesPackage{theorem}
%<+package>         [\FMithmInfo]
%
% \fi
%
% \changes{v1.0}{87/05/07}{First draft.}
% \changes{v1.0a}{87/05/07}{`thistheoremstyle' added.}
% \changes{v1.1a}{87/07/29}{New concept: redefine the internal
%                          \protect\LaTeX{} commands `@xnthm' etc.}
% \changes{v1.1b}{87/07/31}{`theoremfont' used to specify font.}
% \changes{v1.2a}{88/01/08}{`@endtheorem' now globally defined.}
% \changes{v1.2b}{88/01/11}{`newtheorem' only allowed in the preamble.}
% \changes{v1.3a}{88/02/14}{`theoremfont' now a toks register.}
%
% \changes{v2.0a}{88/11/16}{New implementation and new documentation.}
% \changes{v2.0a}{88/11/16}{No layout changes but internal modifications
%                           old substyles won't work.}
% \changes{v2.0b}{89/01/30}{`global' used for all definitions.}
% \changes{v2.0c}{89/06/15}{Documentation translated to English.
%                            Thanks to LLL and Nina}
% \changes{v2.1a}{90/02/15}{Support for new font selection.}
% \changes{v2.2a}{94/02/03}{Cleanup for LaTeX2e}
%
% \DoNotIndex{\@endparenv,\@ifdefinable,\@ifnextchar,\@ifundefined}
% \DoNotIndex{\@namedef,\@nameuse,\@spaces,\@tempa,\@warning}
% \DoNotIndex{\ }
% \DoNotIndex{\begingroup,\bfseries}
% \DoNotIndex{\csname}
% \DoNotIndex{\def,\do,\docdate}
% \DoNotIndex{\endcsname,\endgroup,\endinput,\endtrivlist,\expandafter}
% \DoNotIndex{\filedate,\fileversion}
% \DoNotIndex{\gdef,\global}
% \DoNotIndex{\hbox,\hskip}
% \DoNotIndex{\ignorespaces,\input,\itshape,\item}
% \DoNotIndex{\labelsep,\let,\llap}
% \DoNotIndex{\makeatletter}
% \DoNotIndex{\noexpand}
% \DoNotIndex{\refstepcounter,\relax,\rlap}
% \DoNotIndex{\setlength,\slshape,\space,\string,\strut}
% \DoNotIndex{\the,\trivlist,\typeout}
% \DoNotIndex{\vbox}
% \DoNotIndex{\xdef}
%
% \def\deflabel#1{\ttfamily #1\hfill}
% \def\deflist#1{\begin{list}{}{\settowidth\labelwidth{\ttfamily #1}%
%                               \setlength\leftmargin\labelwidth
%                               \addtolength\leftmargin\labelsep
%                               \let\makelabel\deflabel}}
% \def\enddeflist{\end{list}}
% \def\envfont{\normalfont\ttfamily}
%
% \MakeShortVerb{\|}
%
% \newcommand{\TUB}{{\sl TUGboat\/}}
%
% \GetFileInfo{theorem.dtx}
% \title{An Extension of the
%        \LaTeX{} theorem environment\thanks{This file
%        has version number \fileversion, last
%        revised \filedate.}}
% \author{Frank Mittelbach}
%
% \MaintainedByLaTeXTeam{tools}
% \maketitle
%
% \begin{abstract}
%    The macros described in this paper yield an extension of the
%    \LaTeX{} theorem mechanism. It is designed to satisfy the
%    different requirements of various journals. Thus, the
%    layout of the ``theorems'' can be manipulated by determining a
%    ``style''. This article describes not only the use, but
%    also the definition, of the necessary macros.
% \end{abstract}
%
% \section*{Preface to version 2.2}
%
% For \LaTeXe{} this package did not need any fundamental changes. I
% only modified the messages generated so that theorem layout styles
% will show up with the |\listfiles| command and cleaned the section
% on the New Font Selection Scheme since this is now included in
% \LaTeX.
%
% \section*{Preface to version 2.1}
%
% This version is identical to 2.0g described in \TUB\ 10\#3 except
% for some internal defaults which are now set depending on the used
% font selection scheme.
%
% This was done to avoid unpleasant surprises if the new font selection
% scheme is in force. For further details see
% section~\ref{sec:newfonts} and \cite{bk:GMS94}.
%
% \section{Introduction}
%
% For our purposes here, ``theorems'' are labelled enunciations,
% often set off from the main text by extra space and a font change.
% Theorems, corollaries, conjectures, definitions, and remarks
% are all instances of ``theorems''.  The ``header'' of these structures
% is composed of a label (such as \textsc{Theorem} or \textsc{Remark})
% and a number which serializes an item in the sequence of items
% with the same label.
%
% Shortly after the introduction of \LaTeX{} at the Fachbereich
% Mathematik in Mainz, the desire to manipulate the layout of
% ``theorems'' arose.  In Mainz, the following two conventions came
% into general use:
% \begin{enumerate}
%   \item The number of the theorem is shown in the margin.
%   \item There is a line break at
%         the end of the theorem header.
% \end{enumerate}
%
% Additionally, some journals require different formats which depend on
% the ``sort of theorem'': e.g.\ often remarks and definitions are set
% in |\upshape|, while |\itshape| is employed for main theorems.
%
% Confronted with these requirements, a theorem environment was
% developed in Mainz which allows separate determination of the
% layout of the ``theorems sets'', comparable to |\pagestyle|.
%
%
% \section{The user interface}
% \subsection{Defining new theorem sets}
%
% \DescribeMacro\newtheorem
% As in the original \LaTeX{} version, the command |\newtheorem|
% defines a new ``theorem set'' or ``theorem-like structure''.
% Two required arguments name the new environment and give the text
% to be typeset with each instance of the new ``set'', while
% an optional argument determines how
% the ``set'' is enumerated:
% \begin{description}
%    \item[\ttfamily \bslash newtheorem\{foo\}\{bar\}]
%       The theorem set {\envfont foo} (whose name is \texttt{bar})
%       uses its own counter.
%    \item[\ttfamily \bslash newtheorem\{foo2\}{[foo]}\{bar2\}]
%       The theorem set {\envfont foo2} (printed name \texttt{bar2})
%       uses the same counter as the theorem set \texttt{foo}.
%    \item[\ttfamily \bslash newtheorem\{foo3\}\{bar3\}{[section]}]
%       The theorem set {\envfont foo3} (printed name \texttt{bar3}) is
%       enumerated within the counter \texttt{section}, i.e.\ with every
%       new |\section| the enumeration begins again with 1, and
%       the enumeration is composed from the section-number and the
%       theorem counter itself.
% \end{description}
%
% \DescribeMacro\theoremstyle
% Additionally, the command |\theoremstyle| can define the layout
% of various, or all, theorem sets.  It should be noted that any
% theorem set defined by |\newtheorem| is typeset in the
% |\theoremstyle| that is current at the time of the definition.
% Thus, the following
% \begin{verbatim}
%    \theoremstyle{break}        \newtheorem{Cor}{Corollary}
%    \theoremstyle{plain}        \newtheorem{Exa}{Example}[section]
%\end{verbatim}                             ^^A No blank at line start
% leads to the result that the set {\envfont Cor} is formatted in the
% style \texttt{break}, while the set {\envfont Exa} and all the
% following ones are formatted in the style \texttt{plain}, unless
% another |\theoremstyle| follows.  Since the definitions installed by
% |\newtheorem| are global, one also can limit |\theoremstyle| locally
% by grouping braces.
%
% \DescribeMacro\theorembodyfont
% The choice of the font for the theorem body is completely
% independent of the chosen |\theoremstyle|; this has proven to
% be very advantageous.  For example,
% \begin{verbatim}
%   {\theorembodyfont{\upshape}          \newtheorem{Rem}{Remark}}
%\end{verbatim}
% defines a theorem set {\envfont Rem}, which will be set in
% |\upshape| in the current layout (which in our example is
% \texttt{plain}).  As with |\theoremstyle|, the |\theorembodyfont|
% chosen is that current at the time of |\newtheorem|.  If
% |\theorembodyfont| is not specified or one defines
% |\theorembodyfont{}|, then the font used will be that defined by the
% |\theoremstyle|.
%
% \DescribeMacro\theoremheaderfont It is also possible to customize
% the font used for the theorem headers.  This is, however, a global
% declaration, and therefore there should be at most one
% |\theoremheaderfont| declaration in the preamble.\footnote{If it is
% actually necessary to have different header fonts, one has to define
% new theorem styles (substituting the desired font) or specify the
% information directly in the \texttt{\bslash newtheorem} declaration
% (the unclean variant).}
%
% \DescribeMacro\theorempreskipamount
% \DescribeMacro\theorempostskipamount Two additional parameters
% affect the vertical space around the theorem environments:
% |\theorempreskipamount| and |\theorempostskipamount| define,
% respectively, the spacing before and after such an environment.
% These parameters apply for all theorem sets and can be manipulated
% with the ordinary length macros.  They are rubber lengths
% (`\textsf{skips}'), and therefore can contain \texttt{plus} and
% \texttt{minus} parts.
%
% Since the definition of theorem sets should---most sensibly---be
% placed in the preamble, we only allow installation there.
% It is therefore possible to release the memory used here after
% |\begin{document}|, in order to make room for other
% applications.
%
%
% \subsection{Existing theorem styles}
%
% The following theorem styles exist to date:
% \begin{deflist}{marginbreak}
%    \item[plain]
%       This theorem style emulates the original \LaTeX{} definition,
%       except that additionally the parameters
%       |\theorem...skipamount| are used.
%    \item[break]
%       In this style, the theorem header is followed by a line
%       break.
%    \item[marginbreak]
%       The theorem number is set in the margin, and there is a line
%       break as in \texttt{break}.
%    \item[changebreak]
%       Like \texttt{break}, but with header number and text
%       interchanged.
%    \item[change]
%       Header number and text are interchanged, without a line break.
%    \item[margin]
%       The number is set in the left margin, without a line break.
% \end{deflist}
% All styles (except \texttt{plain}) select |\slshape| as the default
% |\theorembodyfont|.
%
%
% \subsection{Examples}
%
% Given the above theorem sets {\envfont Cor}, {\envfont Exa}
% and {\envfont Rem},
% suppose that the preamble also contains the declarations:
% \begin{verbatim}
%    \theoremstyle{marginbreak}   \newtheorem{Lem}[Cor]{Lemma}
%    \theoremstyle{change}
%    \theorembodyfont{\itshape}        \newtheorem{Def}[Cor]{Definition}
%
%    \theoremheaderfont{\scshape}
%\end{verbatim}
% Then the following are some typical examples of the typeset output
% resulting from their use.
%
% \begin{Cor}
%    This is a sentence typeset in the theorem environment {\envfont
%    Cor}.
% \end{Cor}
%
% \begin{Exa}
%    This is a sentence typeset in the theorem environment {\envfont
%    Exa}.
% \end{Exa}
%
% \begin{Rem}
%    This is a sentence typeset in the theorem environment {\envfont
%    Rem}.
% \end{Rem}
%
% \begin{Lem}[Ben User]
%    This is a sentence typeset in the theorem environment {\envfont
%    Lem}.
% \end{Lem}
%
% \begin{Def}[Very impressive Definition]
%    This is a sentence typeset in the theorem environment {\envfont
%    Def}.
% \end{Def}
%
% The last two examples show the effect of the optional argument to
% a theorem environment (it is the text typeset in parentheses).
%
%
% \section{Special Considerations}
% \label{sec:newfonts}
%
% Theoremheader and body are implemented as a unit. This means that
% the |\theoremheaderfont|  will inherit characteristics of the
% |\theorembodyfont| in \LaTeXe. Thus, if for example
% |\theorembodyfont| is |\itshape| and |\theoremheaderfont|
% is |\bfseries| the font selected for the header will have the
% characteristics `bold extended italic'. If this is not desired one
% should set the |\theoremheaderfont| to something like
% \begin{verbatim}
%      \theoremheaderfont{\normalfont\bfseries}
%\end{verbatim}
% i.e.\ supplying all necessary font information explicitly.
%
%
%
% \section{Acknowledgements}
%
% The publication of this set of macros was only possible with the
% help of Christina Busse (translating the manuscript into English),
% Joachim Pense (playing the r\^ole of typist), Chris Rowley
% (looking everything over) and many others providing useful
% suggestions.
%
% \MaybeStop{%
%    \begin{thebibliography}{9}
%       \bibitem{bk:GMS94} \textsc{M.~Goossens}, \textsc{F.~Mittelbach}
%         and \textsc{A.~Samarin}.
%         \newblock The \LaTeX{} Companion.
%         \newblock
%         Addison-Wesley, Reading, Massachusetts, 1994.
%       \bibitem{latex:source} \textsc{Lamport, Leslie.} \newblock
%                               \texttt{latex.tex}, version 2.09, date
%                               Feb. 1990.
%    \end{thebibliography}
%    \PrintIndex
%    \PrintChanges}
%
%
% \section{The documentation driver file}
%
% The next bit of code contains the documentation driver file for
% \TeX{}, i.e., the file that will produce the documentation you are
% currently reading. It will be extracted from this file by the
% \texttt{docstrip} program. Since it is the first code in the file
% one can alternatively process this file directly with \LaTeXe{} to
% obtain the documentation.
%
%    \begin{macrocode}
%<*driver>
\documentclass{ltxdoc}

\usepackage{theorem}

%<+driver>% The next few lines define theorem sets which are used
%<+driver>% in the example section of the documentation.

\theoremstyle{break}       \newtheorem{Cor}{Corollary}
\theoremstyle{plain}       \newtheorem{Exa}{Example}[section]
{\theorembodyfont{\upshape}\newtheorem{Rem}{Remark}}
\theoremstyle{marginbreak} \newtheorem{Lem}[Cor]{Lemma}
\theoremstyle{change}
\theorembodyfont{\itshape} \newtheorem{Def}[Cor]{Definition}

\theoremheaderfont{\scshape}

\RecordChanges

\begin{document}
   \DocInput{theorem.dtx}
\end{document}
%</driver>
%    \end{macrocode}
%
% \section{Definition of the Macros}
%
% If the file has been loaded before, we abort immediately. If not the
% package announces itself (this is actually done at the very top of
% the file---the way it is done isn't good style so don't copy it).
% \changes{v2.0e}{89/07/19}{Spaces removed from `typeout'.}
% \changes{v2.2b}{95/11/19}{Announce moved to top of file}
%    \begin{macrocode}
%<*package>
%\@ifundefined{theorem@style}{}{\endinput}
%\def\FMithmInfo{1995/11/19 v2.2b Theorem extension package (FMi)}
%\ProvidesPackage{theorem}[\FMithmInfo]
%    \end{macrocode}
%
%
% \subsection{Definition of theorem styles and fonts}
%
% All the definitions in this file are done globally to allow
% inputting this file inside a group.
%
% \begin{macro}{\theoremstyle}
%    Before a theorem style can be installed, the chosen style must be
%    known. For that reason, we must test to see that
%    |\th@|\meta{style} is known or, more precisely, that it is
%    different from |\relax|. If the style is not known then
%    |\th@plain| is used.
%    \begin{macrocode}
\gdef\theoremstyle#1{%
   \@ifundefined{th@#1}{\@warning
          {Unknown theoremstyle `#1'. Using `plain'}%
          \theorem@style{plain}}%
%    \end{macrocode}
%    We save the theorem style to be used in the token register
%    |\theorem@style|.
%    \begin{macrocode}
      {\theorem@style{#1}}%
%    \end{macrocode}
%    Now we ``evaluate'' the theorem style: this means, we call the
%    macro |\th@|\meta{style} which will activate the relevant
%    definitions which are contained in a separate file.  This is done
%    in a group to suppress changes to the current font.  This could
%    otherwise pose problems together with the new font selection
%    scheme\footnote{When I printed the original article using the new
%    font selection scheme I ended with a document with slanted
%    typefaces (text headings and all) simply because one of the
%    theorem styles used \texttt{\bslash sl} at toplevel.} if the
%    |\th@|\meta{style} is evaluated a second time.
% \changes{v2.1a}{90/02/15}{Group added around `th@..}
% \changes{v2.1a}{90/02/15}{Reseting of `..@begintheorem removed.}
%    \begin{macrocode}
      \begingroup
        \csname th@\the\theorem@style \endcsname
      \endgroup}
%    \end{macrocode}
% \end{macro}
%
%
% \begin{macro}{\@begintheorem}
% \changes{v2.1a}{90/02/15}{Relaxed at toplevel.}
% \begin{macro}{\@opargbegintheorem}
% \changes{v2.1a}{90/02/15}{Relaxed at toplevel.}
%    We reset |\@begintheorem| and |\@opargbegintheorem| to
%    |\relax| since these commands are no longer necessary at
%    toplevel. This will save a few tokens.
%    \begin{macrocode}
\global\let\@begintheorem\relax
\global\let\@opargbegintheorem\relax
%    \end{macrocode}
% \end{macro}
% \end{macro}
%
% \begin{macro}{\theorem@style}
%    Obviously the \textsf{token} register used above has to be
%    allocated. To assure the utmost compatibility with the
%    original \LaTeX{} definition, we set the default theorem style to
%    \texttt{plain}, which implements the usual \LaTeX{} convention.
% \changes{v2.0f}{89/07/20}{Unnecessary `global' removed.}
%    \begin{macrocode}
\newtoks\theorem@style
\global\theorem@style{plain}
%    \end{macrocode}
% \end{macro}
%
% \begin{macro}{\theorembodyfont}
% \begin{macro}{\theorem@bodyfont}
%    If the |\theorembodyfont| is set by the user then it should not
%    interact with the default font set in the theorem style. When the
%    new font selection is in force this may happen if, for example,
%    the default is |\itshape| and the new |\theorembodyfont| is
%    |\sffamily|. So we add a |\reset@font| command in front of the
%    user definition.
% \changes{v2.0f}{89/07/20}{Unnecessary `global' removed.}
% \changes{v2.0g}{89/09/19}{`theoremfont to `theorembodyfont.}
% \changes{v2.1c}{92/06/22}{`reset@font added and logic changed.}
% \changes{v2.1d}{92/10/28}{Reset to empty if arg is empty}
%    \begin{macrocode}
\gdef\theorembodyfont#1{%
%    \end{macrocode}
%    We check if the argument supplied is empty and if so put nothing
%    into the |\theorem@bodyfont| token register to allow for
%    |\theorembodyfont{}| as a mean of using the default of the
%    current |\theoremstyle|.
%    \begin{macrocode}
   \def\@tempa{#1}%
   \ifx\@tempa\@empty
    \theorem@bodyfont{}%
   \else
    \theorem@bodyfont{\reset@font#1}%
   \fi
   }
\newtoks\theorem@bodyfont
\global\theorem@bodyfont{}
%    \end{macrocode}
% \end{macro}
% \end{macro}
%
%
% \begin{macro}{\theoremheaderfont}
% \changes{v2.0g}{89/09/19}{Macro added.}
%    The font for the theorem headers is handled differently because
%    this definition applies to all theorem styles.
%    \begin{macrocode}
\gdef\theoremheaderfont#1{\gdef\theorem@headerfont{#1}%
%    \end{macrocode}
%    After using the macro once it is redefined to produce an error
%    message.
%    \begin{macrocode}
       \gdef\theoremheaderfont##1{%
        \typeout{\string\theoremheaderfont\space should be used
                 only once.}}}
%    \end{macrocode}
% \end{macro}
%
% \begin{macro}{\theorem@headerfont}
% \changes{v2.0g}{89/09/19}{Macro added.}
%    To set the |\theorem@headerfont| default we first test if
%    the new fontselection scheme is in force.
% \changes{v2.1a}{90/02/15}{Support for new font selection.}
%    \begin{macrocode}
\ifx\upshape\undefined
%    \end{macrocode}
%    If not we define it to expand into |\bfseries|. We don't use
%    |\let| just in case a following style option redefines this
%    macro.
%    \begin{macrocode}
\gdef\theorem@headerfont{\bfseries}
%    \end{macrocode}
%    Otherwise we reset the current shape before calling |\bfseries|.
%    \begin{macrocode}
\else \gdef\theorem@headerfont{\normalfont\bfseries}\fi
%    \end{macrocode}
% \end{macro}
%
% \begin{macro}{\th@plain}
% \begin{macro}{\th@break}
% \begin{macro}{\th@marginbreak}
% \begin{macro}{\th@changebreak}
% \begin{macro}{\th@change}
% \begin{macro}{\th@margin}
%    The different styles are defined in macros such as
%    |\th@plain|.  Since memory space is precious in
%    ``non-Big-versions'', we have to avoid offering too many unused
%    definitions.  Therefore we define these styles in separate files
%    that can be loaded on demand.  Thus the commands themselves only
%    load these files. We use |\@input@|, a \LaTeXe{} internal command
%    that ensures that the file will be listed with |\listfiles|.
% \changes{v2.2a}{94/02/02}{Use `@input@ to load theorem layout files}
%    \begin{macrocode}
\gdef\th@plain{\@input@{thp.sty}}
\gdef\th@break{\@input@{thb.sty}}
\gdef\th@marginbreak{\@input@{thmb.sty}}
\gdef\th@changebreak{\@input@{thcb.sty}}
\gdef\th@change{\@input@{thc.sty}}
\gdef\th@margin{\@input@{thm.sty}}
%    \end{macrocode}
%    This list will be expanded when new styles become
%    available.  For testing,  just append new theorem substyles
%    as document options.
% \end{macro}
% \end{macro}
% \end{macro}
% \end{macro}
% \end{macro}
% \end{macro}
%
%
%
% \subsection{Definition of a new theorem set}
%
% As already pointed out, a new theorem environment can be defined in
% three different ways:
% \begin{verbatim}
%    \newtheorem{Lem}{Lemma}
%    \newtheorem{Lem}{Lemma}[section]
%    \newtheorem{Lem}[Theorem]{Lemma}
%\end{verbatim}
% The function of the macro |\newtheorem| is to recognize these cases
% and then to branch into one of the three macros |\@ynthm|, |\@xnthm|
% or |\@othm|.  This mechanism is adopted unchanged from
% \cite{latex:source}; the essential point here is that, for example,
% in the second case, the arguments \texttt{Lem}, \texttt{Lemma} and
% \texttt{section} are passed over to the macro |\@xnthm|.
%
% We inspect this case first because the others present fewer problems,
% and thus are easily derived from this one.
%
% \begin{macro}{\@xnthm}
%    For our example arguments, the macro |\@xnthm| must fulfill
%    the following:
%    \begin{itemize}
%       \item
%          Define a new \LaTeX-counter `Lem'
%       \item
%          reset this counter within a |\section|
%       \item
%          define the macro |\theLem|
%       \item
%          define the environment macros |\Lem| and
%          |\endLem| using the current
%          |\theoremstyle| and |\theorem@bodyfont|.
%    \end{itemize}
%    Obviously, all this should happen only if the first argument
%    of |\@xnthm| (i.e.\ \texttt{Lem} in our example) is chosen so
%    as not to conflict with any previously defined commands or
%    environments.  This test is performed by the \LaTeX{} macro
%    |\@ifdefinable|.
%    \begin{macrocode}
\gdef\@xnthm#1#2[#3]{\expandafter\@ifdefinable\csname #1\endcsname
%    \end{macrocode}
%    Therefore, the first argument of |\@ifdefinable| is the
%    expansion (in the example, |\Lem|) of
%    |\csname||#1\endcsname|.  The second argument is
%    executed only if the test has been completed successfully.
%    \begin{macrocode}
   {%
%    \end{macrocode}
%    Now we define the new counter. The names of the \LaTeX{}
%    macros employed should speak for themselves:
% \changes{v2.2b}{1995/10/07}{Use \cs{@newctr}}
% \changes{v2.2c}{1995/11/23}{Use \cs{@newctr} properly.}
%    \begin{macrocode}
    \@definecounter{#1}\@newctr{#1}[#3]%
%    \end{macrocode}
%    Using |\@newctr| will give a proper error message if the counter
%    in |#3| is not defined.
%    In defining `|\theLem|' we must generate the desired macro
%    name by use of |\expandafter| and |\csname|.
%    \begin{macrocode}
    \expandafter\xdef\csname the#1\endcsname
%    \end{macrocode}
%    An |\xdef| is used in order to make the definition global,
%    and to ensure that it contains the replacement texts of
%    |\@thmcountersep| and |\@thmcounter|.\footnote{These
%    two macros can be defined by the document style.  Their default
%    values produce a `\texttt{.}' as separation and an arabic
%    representation of the number.} However, not everything should be
%    expanded. For example, it saves space to use |\thesection|
%    instead of its---at times---lengthy expansion.
%    \begin{macrocode}
      {\expandafter \noexpand \csname the#3\endcsname
       \@thmcountersep \@thmcounter{#1}}%
%    \end{macrocode}
%    Thus with the defaults of \LaTeX{}, |\theLem| would be
%    replaced by the command sequence |\thesection.\arabic{Lem}|.
%
%    We will now look at the definition of the macro which is executed
%    at the beginning of the actual environment (in our example this
%    macro is |\Lem|). It should be noted that we use an
%    ``|\expandafter| trick'' to expand only certain parts of the
%    replacement text at the time of the definition.
%    \begin{macrocode}
    \def\@tempa{\global\@namedef{#1}}%
    \expandafter \@tempa \expandafter{%
%    \end{macrocode}
%    First, the macro that contains the current definitions of
%    |\@begintheorem| and |\@opargtheorem| should be called
%    up.  The name of this macro---as is already known---has the form
%    |\th@|\meta{theorem\ style}; therefore, it must be called by
%    \begin{macrocode}
      \csname th@\the \theorem@style
%    \end{macrocode}
%    In addition the default theorem font should be changeable, i.e.\
%    we have to insert the contents of |\theorem@bodyfont|. For
%    that reason, we expand even further, beyond |\endcsname|,
%    and thus insert the contents of the \textsf{token} register
%    |\theorem@bodyfont| in the replacement text.
%    \begin{macrocode}
            \expandafter \endcsname \the \theorem@bodyfont
%    \end{macrocode}
%    Now it is time to call the macro |\@thm| which takes over
%    the further processing.  It has two arguments: the current
%    counter name (in our example, \texttt{Lem}), and the text of the
%    label (in our example, \texttt{Lemma}).
%    \begin{macrocode}
     \@thm{#1}{#2}}%
%    \end{macrocode}
%    With this, the `sub-definition' is complete.  The macro
%    |\@endtheorem| ends a theorem environment and is, so far, nothing
%    but an |\endtrivlist|.  (Hence it is defined globally, and not
%    within the theorem styles.\footnote{This has to be changed as
%    soon as theorem styles that change \texttt{\bslash @endtheorem}
%    exist.  In such a case, all existing styles must be changed as
%    well since they will have to reset the macro.}) Therefore, we can
%    set it equivalent to the macro that ends the theorem set (in our
%    example, |\endLem|).  However, if some day theorem styles exist
%    that do change |\@endtheorem|, we would have to use the
%    commented-out line instead.
%    \begin{macrocode}
    \global \expandafter \let \csname end#1\endcsname \@endtheorem
%   \global\@namedef{end#1}{\@endtheorem}%
%    \end{macrocode}
%    With these commands all the required definitions are employed,
%    unless the test |\@ifdefinable| has failed.  Therefore, we
%    end the second argument of this macro and with it the definition
%    of |\@xnthm|.
%    \begin{macrocode}
   }}
%    \end{macrocode}
% \end{macro}
%
% \begin{macro}{\@ynthm}
%    The definition of |\@ynthm| is completely analogous. In
%    this case the new counter that is defined is not reset within
%    another counter; thus the definition of |\the...|
%    is simplified:
%    \begin{macrocode}
\gdef\@ynthm#1#2{\expandafter\@ifdefinable\csname #1\endcsname
   {\@definecounter{#1}%
    \expandafter\xdef\csname the#1\endcsname{\@thmcounter{#1}}%
%    \end{macrocode}
%    The rest of the definition corresponds literally to that of
%    |\@xnthm|:
%    \begin{macrocode}
    \def\@tempa{\global\@namedef{#1}}\expandafter \@tempa
     \expandafter{\csname th@\the \theorem@style \expandafter
     \endcsname \the\theorem@bodyfont \@thm{#1}{#2}}%
    \global \expandafter \let \csname end#1\endcsname \@endtheorem}}
%    \end{macrocode}
% \end{macro}
%
% \begin{macro}{\@othm}
%    The definition of |\@othm| does not contain anything new.
%    \begin{macrocode}
\gdef\@othm#1[#2]#3{%
%    \end{macrocode}
% \changes{v2.2b}{95/10/07}{Build in check for counter defined}
%    We do not define a new counter but instead use one that has
%    already been defined.  Thus the only definition we need is that
%    of this pseudo-counter (i.e.\ |\the|\meta{env.\ name}).
%    First we check if |#2| corresponds to a known counter name.
%    \begin{macrocode}
  \expandafter\ifx\csname c@#2\endcsname\relax
   \@nocounterr{#2}%
  \else
   \expandafter\@ifdefinable\csname #1\endcsname
   {\expandafter \xdef \csname the#1\endcsname
     {\expandafter \noexpand \csname the#2\endcsname}%
%    \end{macrocode}
%    All other parts of the definition can be adopted from
%    |\@xnthm|. We have to remember, though, that in this case
%    the name of the current counter and the theorem label have moved
%    to the second and third arguments.
%    \begin{macrocode}
    \def\@tempa{\global\@namedef{#1}}\expandafter \@tempa
     \expandafter{\csname th@\the \theorem@style \expandafter
     \endcsname \the\theorem@bodyfont \@thm{#2}{#3}}%
    \global \expandafter \let \csname end#1\endcsname \@endtheorem}%
  \fi}
%    \end{macrocode}
% \end{macro}
%
% \subsection{Macros that are employed in a {\envfont theorem}
%    environment}
%
% \begin{macro}{\@thm}
%    The macro |\@thm| has to increase the current counter.  Then,
%    depending on whether the environment has (or does not have) an
%    optional argument, it has to branch into either |\@begintheorem|
%    or |\@opargtheorem|.
%    \begin{macrocode}
\gdef\@thm#1#2{\refstepcounter{#1}%
%    \end{macrocode}
%    Now we start a {\envfont trivlist} environment, and give
%    |\@topsep| and |\@topsepadd| the values of the \textsf{skip}
%    registers |\theorempreskipamount| and |\theorempostskipamount|.
%    The value in |\@topsep| is the vertical space that is inserted by
%    the first (and only) |\item| in our |\trivlist| whilst
%    |\@topsepadd| is inserted by |\@endparenv| at the end of that
%    {\envfont trivlist} environment.  By using these registers, we
%    obtain the desired space around a {\envfont theorem} environment.
%    \begin{macrocode}
   \trivlist
   \@topsep \theorempreskipamount               % used by first \item
   \@topsepadd \theorempostskipamount           % used by \@endparenv
%    \end{macrocode}
%    Now we have to test whether an optional argument has been given.
% \changes{v2.0d}{89/07/18}{Forgotten percent added.}
%    \begin{macrocode}
   \@ifnextchar [%
%    \end{macrocode}
%    If there is an optional argument, we will call |\@ythm|, and
%    move the arguments read back into the input stream.
%    \begin{macrocode}
   {\@ythm{#1}{#2}}%
%    \end{macrocode}
%    If not, we call |\@begintheorem|. Its first argument is the
%    name of the theorem set (hence the second argument of
%    |\@thm|). Its second argument is the macro that produces the
%    current number.
%    \begin{macrocode}
   {\@begintheorem{#2}{\csname the#1\endcsname}\ignorespaces}}
%    \end{macrocode}
% \end{macro}
%
% \begin{macro}{\@xthm}
% \begin{macro}{\@ythm}
%    Both these macros were originally called by |\@thm|.  We do
%    not need |\@xthm| anymore, hence we reset it to
%    |\relax|. The definition of |\@ythm| has not changed at
%    all from its definition in \LaTeX. In order to make the macros
%    easier to understand, we will
%    nevertheless present it (commented out).
%    \begin{macrocode}
\global\let\@xthm\relax
% \def\@ythm#1#2[#3]{\@opargbegintheorem{#2}{\csname
%           the#1\endcsname}{#3}\ignorespaces}
%</package>
%    \end{macrocode}
%    The primitive |\ignorespaces| in |\@ythm| and
%    |\@thm| is needed to remove the spaces between the
%    |\begin{...}| and the actual text.
% \end{macro}
% \end{macro}
%
%
% \subsection{Definition of the theorem substyles}
%
% As already pointed out, the theorem substyles, defined below, are
% only loaded when necessary.  Note that all these substyles, except
% \texttt{plain}, have |\slshape| as the default body font.
%
% \subsubsection{The \texttt{plain} style}
%
% \label{dok:th@plain} As the following macros use |@|, we have
% to locally set the |\catcode| of this symbol to ``letter''.
% This happens within a group, so that we do not have to worry about
% which |\catcode| that symbol had before.
%    \begin{macrocode}
%<*thp>
\begingroup \makeatletter
%    \end{macrocode}
% Since we are now within a group, we must make all definitions
% globally.  First we make sure that \texttt{theorem.sty} is loaded.
% This will allow us to use this file as a document style option
% without having to call \texttt{theorem} itself as an option.  At the
% same time, we assure that at least version 2 is loaded, since
% |\theorem@style| was not defined in earlier versions.
%    \begin{macrocode}
\@ifundefined{theorem@style}{\input{theorem.sty}}{}
\ProvidesFile{thp.sty}
             [\FMithmInfo]
%    \end{macrocode}
%
%
% \begin{macro}{\th@plain}
% \begin{macro}{\@begintheorem}
% \begin{macro}{\@opargbegintheorem}
%    |\theoremstyle{plain}| corresponds to the original
%    definition, except that the distances to the surrounding text are
%    determined by the parameters |\theorempreskipamount| and
%    |\theorempostskipamount|.
%    First we set the default body font.
%    \begin{macrocode}
\gdef\th@plain{\normalfont\itshape
%    \end{macrocode}
%    Then we define |\@begintheorem| and
%    |\@opargbegintheorem|. These two macros define how the
%    header of a theorem is typeset.  |\@opargbegintheorem| will
%    be called if a \textsf{theorem} environment with an optional
%    argument is encountered; otherwise, the header is constructed by
%    calling |\@begintheorem|.  If one of these macros is
%    executed, we are within a \textsf{trivlist} environment started by
%    |\@thm|.  So the theorem header is produced with an
%    |\item| command.
%
%    Instead of specifying the header font directly, all standard
%    theorem styles use the |\theorem@headerfont| macro to allow
%    customization.  The extra space (|\labelsep|) is necessary
%    because of problems in the \textsf{trivlist} environment.
%    \begin{macrocode}
  \def\@begintheorem##1##2{%
        \item[\hskip\labelsep \theorem@headerfont ##1\ ##2]}%
%    \end{macrocode}
%    The definition of |\@opargbegintheorem| is completely
%    analogous. The only difference is the fact that there exists a
%    third argument (which is the optional parameter of the
%    environment and contains additional information about the
%    theorem).  Customarily we enclose it in parentheses.
%    \begin{macrocode}
\def\@opargbegintheorem##1##2##3{%
   \item[\hskip\labelsep \theorem@headerfont ##1\ ##2\ (##3)]}}
%    \end{macrocode}
% \end{macro}
% \end{macro}
% \end{macro}
%
% \noindent
% We conclude with an |\endgroup| to restore the
% |\catcode| of |@|.
%    \begin{macrocode}
\endgroup
%</thp>
%    \end{macrocode}
%
%
% \subsubsection{The \texttt{break} style}
%
% This style option is stored in the file |thb.sty|.
% For the next two lines see the documentation for |\th@plain| on
% page~\pageref{dok:th@plain}.
%    \begin{macrocode}
%<*thb>
\begingroup \makeatletter
\@ifundefined{theorem@style}{\input{theorem.sty}}{}
\ProvidesFile{thb.sty}
             [\FMithmInfo]
%    \end{macrocode}
%
% \begin{macro}{\th@break}
%    |\theoremstyle{break}| produces a line break after the name
%    of the theorem.  The font is |\slshape|.  Hence, we define
%    |\th@break| as follows:
%    \begin{macrocode}
\gdef\th@break{\normalfont\slshape
  \def\@begintheorem##1##2{\item[%
%    \end{macrocode}
%    We run into the following problem: it is not possible to create
%    the header with |\item[|\meta{title}|]| and then start
%    a new line by, for example, |\mbox{}\\|. Such a definition
%    will fail whenever a list environment follows immediately.  With
%    the above construction, the |\mbox{}| causes the switch
%    |@inlabel| (cf.\ definition of |\list| and
%    |\trivlist| in \cite{latex:source}) to be set to \texttt{false}
%    and so the following list will insert additional vertical
%    space (|\topskip|). This is quite annoying. Therefore, we
%    create the line break within the |\item|. In order to ensure
%    that the text will begin at the proper position in the following
%    line, we simply pretend that the label does not take any
%    room.\footnote{This will lead to problems whenever very high
%    symbols occurring in the line tower into the
%    heading. So, something else has to be done here sometime.}
%    \begin{macrocode}
       \rlap{\vbox{\hbox{\hskip \labelsep\theorem@headerfont ##1\ ##2}%
                   \hbox{\strut}}}]}%
%    \end{macrocode}
%    Again, the definition of |\@opargbegintheorem| is
%    completely analogous.
%    \begin{macrocode}
\def\@opargbegintheorem##1##2##3{%
  \item[\rlap{\vbox{\hbox{\hskip \labelsep \theorem@headerfont
                     ##1\ ##2\ (##3)}%
                    \hbox{\strut}}}]}}
\endgroup
%</thb>
%    \end{macrocode}
% \end{macro}
%
%
% \subsubsection{The \texttt{changebreak} style}
%
%    \begin{macrocode}
% This style option is stored in the file |thcb.sty|.
%    \begin{macrocode}
%<*thcb>
\begingroup \makeatletter
\@ifundefined{theorem@style}{\input{theorem.sty}}{}
\ProvidesFile{thcb.sty}
             [\FMithmInfo]
%    \end{macrocode}
%
% \begin{macro}{\th@changebreak}
%    The \texttt{change-break} theorem style is like \texttt{break}
%    but with interchange of theorem name and theorem number.  Thus we
%    define |\th@changebreak| as follows:
%    \begin{macrocode}
\gdef\th@changebreak{\normalfont\slshape
  \def\@begintheorem##1##2{\item
       [\rlap{\vbox{\hbox{\hskip\labelsep \theorem@headerfont ##2\ ##1}%
                    \hbox{\strut}}}]}%
\def\@opargbegintheorem##1##2##3{%
  \item[\rlap{\vbox{\hbox{\hskip\labelsep \theorem@headerfont
                     ##2\ ##1\ (##3)}%
                    \hbox{\strut}}}]}}
\endgroup
%</thcb>
%    \end{macrocode}
% \end{macro}
%
%
% \subsubsection{The \texttt{change} style}
%
% This style option is stored in the file |thc.sty|.
%    \begin{macrocode}
%<*thc>
\begingroup \makeatletter
\@ifundefined{theorem@style}{\input{theorem.sty}}{}
\ProvidesFile{thc.sty}
             [\FMithmInfo]
%    \end{macrocode}
%
% \begin{macro}{\th@change}
%    The \texttt{change} theorem style corresponds to the
%    \texttt{change break} style without a linebreak after the header.
%    To say it in another way, it's the same as the \texttt{plain}
%    style but with number and name interchanged and |\slshape| as the
%    default font.
%    \begin{macrocode}
\gdef\th@change{\normalfont\slshape
  \def\@begintheorem##1##2{\item
       [\hskip\labelsep \theorem@headerfont ##2\ ##1]}%
\def\@opargbegintheorem##1##2##3{%
  \item[\hskip\labelsep \theorem@headerfont ##2\ ##1\ (##3)]}}
\endgroup
%</thc>
%    \end{macrocode}
% \end{macro}
%
%
% \subsubsection{The \texttt{marginbreak} style}
%
% This style option is the one used most often at Mainz.  It is saved
% in the file |thmb.sty|.
%    \begin{macrocode}
%<*thmb>
\begingroup \makeatletter
\@ifundefined{theorem@style}{\input{theorem.sty}}{}
\ProvidesFile{thmb.sty}
             [\FMithmInfo]
%    \end{macrocode}
%
% \begin{macro}{\th@marginbreak}
%    The \texttt{margin break} style is nearly the same as the
%    \texttt{change break} style. The only difference is the placement
%    of the theorem number.  We use |\llap| to place it in the left
%    margin.
%
%    In this style |\labelsep| denotes the separation between the
%    number and the text.
%    \begin{macrocode}
\gdef\th@marginbreak{\normalfont\slshape
  \def\@begintheorem##1##2{\item
       [\rlap{\vbox{\theorem@headerfont
                    \hbox{\llap{##2}\hskip\labelsep ##1}%
                    \hbox{\strut}}}]}%
\def\@opargbegintheorem##1##2##3{%
  \item[\rlap{\vbox{\theorem@headerfont
                    \hbox{\llap{##2}\hskip\labelsep ##1\ (##3)}%
                    \hbox{\strut}}}]}}
\endgroup
%</thmb>
%    \end{macrocode}
% \changes{v2.1c}{90/02/17}{`endgroup outside macrocode env.}
% \end{macro}
%
%
%
% \subsubsection{The \texttt{margin} style}
%
% This style option is stored in the file |thm.sty|.
%    \begin{macrocode}
%<*thm>
\begingroup \makeatletter
\@ifundefined{theorem@style}{\input{theorem.sty}}{}
\ProvidesFile{thm.sty}
             [\FMithmInfo]
%    \end{macrocode}
%
% \begin{macro}{\th@margin}
%    Again this is only a variant of the theorem styles described
%    above without any new ideas.
%    \begin{macrocode}
\gdef\th@margin{\normalfont\slshape
  \def\@begintheorem##1##2{\item
       [\theorem@headerfont \llap{##2}\hskip\labelsep ##1]}%
\def\@opargbegintheorem##1##2##3{%
  \item[\theorem@headerfont \llap{##2}\hskip\labelsep ##1\ (##3)]}}
\endgroup
%</thm>
%    \end{macrocode}
% \end{macro}
%
%
%
%
% \subsection{Final Definitions}
%
% \begin{macro}{\theorempreskipamount}
% \begin{macro}{\theorempostskipamount}
%    The \textsf{skip} parameters that regulate the vertical empty space
%    before and after the {\envfont theorem} environment have
%    to be allocated as well.
% \changes{v2.0f}{89/07/20}{Unnecessary `global' removed.}
%    \begin{macrocode}
%<*package>
\newskip\theorempreskipamount
\newskip\theorempostskipamount
%    \end{macrocode}
%    Since we have used the same values for all theorem sets, we now can
%    assign them.
%    \begin{macrocode}
\global\setlength\theorempreskipamount{12pt plus 5pt minus 3pt}
\global\setlength\theorempostskipamount{8pt plus 3pt minus 1.5pt}
%    \end{macrocode}
% \end{macro}
% \end{macro}
%
%
% \begin{macro}{\@endtheorem}
%    The same holds for the macro |\@endtheorem|, which ends a
%    {\envfont theorem} environment. Since it is the same for all
%    theorem sets, it is removed from the macros |\th@|\meta{style}.
%    It simply ends the {\envfont trivlist} environment, which was
%    begun in |\@thm|.
%    \begin{macrocode}
\global\let\@endtheorem=\endtrivlist
%    \end{macrocode}
% \end{macro}
%
%
%
% \begin{macro}{\@preamblecmds}
%    All macros defined above are to be used only in the preamble.
%    Therefore, we insert them in |\@preamblecmds| which will disable
%    them at begin document. This is done by the internal \LaTeXe{}
%    command |\@onlypreamble|.
% \changes{v2.0g}{89/09/19}{`theoremheaderfont and `theorembodyfont
%    added.}
% \changes{v2.2a}{94/02/03}{Use \LaTeXe{} interface}
%    \begin{macrocode}
\@onlypreamble\@xnthm
\@onlypreamble\@ynthm
\@onlypreamble\@othm
\@onlypreamble\newtheorem
\@onlypreamble\theoremstyle
\@onlypreamble\theorembodyfont
\@onlypreamble\theoremheaderfont
%    \end{macrocode}
% \end{macro}
%
% Finally we declare the \texttt{plain} theorem style to be the default.
% \changes{v2.1b}{90/06/04}{Fixed missing `item by declaring default
%    style.}
%    \begin{macrocode}
\theoremstyle{plain}
%</package>
%    \end{macrocode}
%
% \Finale

\endinput
}{}
\ProvidesFile{thb.sty}
             [\FMithmInfo]
%    \end{macrocode}
%
% \begin{macro}{\th@break}
%    |\theoremstyle{break}| produces a line break after the name
%    of the theorem.  The font is |\slshape|.  Hence, we define
%    |\th@break| as follows:
%    \begin{macrocode}
\gdef\th@break{\normalfont\slshape
  \def\@begintheorem##1##2{\item[%
%    \end{macrocode}
%    We run into the following problem: it is not possible to create
%    the header with |\item[|\meta{title}|]| and then start
%    a new line by, for example, |\mbox{}\\|. Such a definition
%    will fail whenever a list environment follows immediately.  With
%    the above construction, the |\mbox{}| causes the switch
%    |@inlabel| (cf.\ definition of |\list| and
%    |\trivlist| in \cite{latex:source}) to be set to \texttt{false}
%    and so the following list will insert additional vertical
%    space (|\topskip|). This is quite annoying. Therefore, we
%    create the line break within the |\item|. In order to ensure
%    that the text will begin at the proper position in the following
%    line, we simply pretend that the label does not take any
%    room.\footnote{This will lead to problems whenever very high
%    symbols occurring in the line tower into the
%    heading. So, something else has to be done here sometime.}
%    \begin{macrocode}
       \rlap{\vbox{\hbox{\hskip \labelsep\theorem@headerfont ##1\ ##2}%
                   \hbox{\strut}}}]}%
%    \end{macrocode}
%    Again, the definition of |\@opargbegintheorem| is
%    completely analogous.
%    \begin{macrocode}
\def\@opargbegintheorem##1##2##3{%
  \item[\rlap{\vbox{\hbox{\hskip \labelsep \theorem@headerfont
                     ##1\ ##2\ (##3)}%
                    \hbox{\strut}}}]}}
\endgroup
%</thb>
%    \end{macrocode}
% \end{macro}
%
%
% \subsubsection{The \texttt{changebreak} style}
%
%    \begin{macrocode}
% This style option is stored in the file |thcb.sty|.
%    \begin{macrocode}
%<*thcb>
\begingroup \makeatletter
\@ifundefined{theorem@style}{% \iffalse meta-comment
%
% Copyright (C) 1993-2024
%
% The LaTeX Project and any individual authors listed elsewhere
% in this file.
%
% This file is part of the Standard LaTeX `Tools Bundle'.
% -------------------------------------------------------
%
% It may be distributed and/or modified under the
% conditions of the LaTeX Project Public License, either version 1.3c
% of this license or (at your option) any later version.
% The latest version of this license is in
%    https://www.latex-project.org/lppl.txt
% and version 1.3c or later is part of all distributions of LaTeX
% version 2005/12/01 or later.
%
% The list of all files belonging to the LaTeX `Tools Bundle' is
% given in the file `manifest.txt'.
%
% \fi
%
%%
%
%
\def\FMithmInfo{2023/07/05 v2.2c Theorem extension package (FMi)}
%
% \ProvidesFile{theorem.dtx}[\FMithmInfo]
%\iffalse   % this is a METACOMMENT !
%%
%% Package `theorem' to use with LaTeX2e
%% Copyright (C) 1989-1995 Frank Mittelbach, all rights reserved.
%<+package>\@ifundefined{theorem@style}{}{\endinput}
%<+package>\NeedsTeXFormat{LaTeX2e}[1995/06/01]
%<+package>\ProvidesPackage{theorem}
%<+package>         [\FMithmInfo]
%
% \fi
%
% \changes{v1.0}{87/05/07}{First draft.}
% \changes{v1.0a}{87/05/07}{`thistheoremstyle' added.}
% \changes{v1.1a}{87/07/29}{New concept: redefine the internal
%                          \protect\LaTeX{} commands `@xnthm' etc.}
% \changes{v1.1b}{87/07/31}{`theoremfont' used to specify font.}
% \changes{v1.2a}{88/01/08}{`@endtheorem' now globally defined.}
% \changes{v1.2b}{88/01/11}{`newtheorem' only allowed in the preamble.}
% \changes{v1.3a}{88/02/14}{`theoremfont' now a toks register.}
%
% \changes{v2.0a}{88/11/16}{New implementation and new documentation.}
% \changes{v2.0a}{88/11/16}{No layout changes but internal modifications
%                           old substyles won't work.}
% \changes{v2.0b}{89/01/30}{`global' used for all definitions.}
% \changes{v2.0c}{89/06/15}{Documentation translated to English.
%                            Thanks to LLL and Nina}
% \changes{v2.1a}{90/02/15}{Support for new font selection.}
% \changes{v2.2a}{94/02/03}{Cleanup for LaTeX2e}
%
% \DoNotIndex{\@endparenv,\@ifdefinable,\@ifnextchar,\@ifundefined}
% \DoNotIndex{\@namedef,\@nameuse,\@spaces,\@tempa,\@warning}
% \DoNotIndex{\ }
% \DoNotIndex{\begingroup,\bfseries}
% \DoNotIndex{\csname}
% \DoNotIndex{\def,\do,\docdate}
% \DoNotIndex{\endcsname,\endgroup,\endinput,\endtrivlist,\expandafter}
% \DoNotIndex{\filedate,\fileversion}
% \DoNotIndex{\gdef,\global}
% \DoNotIndex{\hbox,\hskip}
% \DoNotIndex{\ignorespaces,\input,\itshape,\item}
% \DoNotIndex{\labelsep,\let,\llap}
% \DoNotIndex{\makeatletter}
% \DoNotIndex{\noexpand}
% \DoNotIndex{\refstepcounter,\relax,\rlap}
% \DoNotIndex{\setlength,\slshape,\space,\string,\strut}
% \DoNotIndex{\the,\trivlist,\typeout}
% \DoNotIndex{\vbox}
% \DoNotIndex{\xdef}
%
% \def\deflabel#1{\ttfamily #1\hfill}
% \def\deflist#1{\begin{list}{}{\settowidth\labelwidth{\ttfamily #1}%
%                               \setlength\leftmargin\labelwidth
%                               \addtolength\leftmargin\labelsep
%                               \let\makelabel\deflabel}}
% \def\enddeflist{\end{list}}
% \def\envfont{\normalfont\ttfamily}
%
% \MakeShortVerb{\|}
%
% \newcommand{\TUB}{{\sl TUGboat\/}}
%
% \GetFileInfo{theorem.dtx}
% \title{An Extension of the
%        \LaTeX{} theorem environment\thanks{This file
%        has version number \fileversion, last
%        revised \filedate.}}
% \author{Frank Mittelbach}
%
% \MaintainedByLaTeXTeam{tools}
% \maketitle
%
% \begin{abstract}
%    The macros described in this paper yield an extension of the
%    \LaTeX{} theorem mechanism. It is designed to satisfy the
%    different requirements of various journals. Thus, the
%    layout of the ``theorems'' can be manipulated by determining a
%    ``style''. This article describes not only the use, but
%    also the definition, of the necessary macros.
% \end{abstract}
%
% \section*{Preface to version 2.2}
%
% For \LaTeXe{} this package did not need any fundamental changes. I
% only modified the messages generated so that theorem layout styles
% will show up with the |\listfiles| command and cleaned the section
% on the New Font Selection Scheme since this is now included in
% \LaTeX.
%
% \section*{Preface to version 2.1}
%
% This version is identical to 2.0g described in \TUB\ 10\#3 except
% for some internal defaults which are now set depending on the used
% font selection scheme.
%
% This was done to avoid unpleasant surprises if the new font selection
% scheme is in force. For further details see
% section~\ref{sec:newfonts} and \cite{bk:GMS94}.
%
% \section{Introduction}
%
% For our purposes here, ``theorems'' are labelled enunciations,
% often set off from the main text by extra space and a font change.
% Theorems, corollaries, conjectures, definitions, and remarks
% are all instances of ``theorems''.  The ``header'' of these structures
% is composed of a label (such as \textsc{Theorem} or \textsc{Remark})
% and a number which serializes an item in the sequence of items
% with the same label.
%
% Shortly after the introduction of \LaTeX{} at the Fachbereich
% Mathematik in Mainz, the desire to manipulate the layout of
% ``theorems'' arose.  In Mainz, the following two conventions came
% into general use:
% \begin{enumerate}
%   \item The number of the theorem is shown in the margin.
%   \item There is a line break at
%         the end of the theorem header.
% \end{enumerate}
%
% Additionally, some journals require different formats which depend on
% the ``sort of theorem'': e.g.\ often remarks and definitions are set
% in |\upshape|, while |\itshape| is employed for main theorems.
%
% Confronted with these requirements, a theorem environment was
% developed in Mainz which allows separate determination of the
% layout of the ``theorems sets'', comparable to |\pagestyle|.
%
%
% \section{The user interface}
% \subsection{Defining new theorem sets}
%
% \DescribeMacro\newtheorem
% As in the original \LaTeX{} version, the command |\newtheorem|
% defines a new ``theorem set'' or ``theorem-like structure''.
% Two required arguments name the new environment and give the text
% to be typeset with each instance of the new ``set'', while
% an optional argument determines how
% the ``set'' is enumerated:
% \begin{description}
%    \item[\ttfamily \bslash newtheorem\{foo\}\{bar\}]
%       The theorem set {\envfont foo} (whose name is \texttt{bar})
%       uses its own counter.
%    \item[\ttfamily \bslash newtheorem\{foo2\}{[foo]}\{bar2\}]
%       The theorem set {\envfont foo2} (printed name \texttt{bar2})
%       uses the same counter as the theorem set \texttt{foo}.
%    \item[\ttfamily \bslash newtheorem\{foo3\}\{bar3\}{[section]}]
%       The theorem set {\envfont foo3} (printed name \texttt{bar3}) is
%       enumerated within the counter \texttt{section}, i.e.\ with every
%       new |\section| the enumeration begins again with 1, and
%       the enumeration is composed from the section-number and the
%       theorem counter itself.
% \end{description}
%
% \DescribeMacro\theoremstyle
% Additionally, the command |\theoremstyle| can define the layout
% of various, or all, theorem sets.  It should be noted that any
% theorem set defined by |\newtheorem| is typeset in the
% |\theoremstyle| that is current at the time of the definition.
% Thus, the following
% \begin{verbatim}
%    \theoremstyle{break}        \newtheorem{Cor}{Corollary}
%    \theoremstyle{plain}        \newtheorem{Exa}{Example}[section]
%\end{verbatim}                             ^^A No blank at line start
% leads to the result that the set {\envfont Cor} is formatted in the
% style \texttt{break}, while the set {\envfont Exa} and all the
% following ones are formatted in the style \texttt{plain}, unless
% another |\theoremstyle| follows.  Since the definitions installed by
% |\newtheorem| are global, one also can limit |\theoremstyle| locally
% by grouping braces.
%
% \DescribeMacro\theorembodyfont
% The choice of the font for the theorem body is completely
% independent of the chosen |\theoremstyle|; this has proven to
% be very advantageous.  For example,
% \begin{verbatim}
%   {\theorembodyfont{\upshape}          \newtheorem{Rem}{Remark}}
%\end{verbatim}
% defines a theorem set {\envfont Rem}, which will be set in
% |\upshape| in the current layout (which in our example is
% \texttt{plain}).  As with |\theoremstyle|, the |\theorembodyfont|
% chosen is that current at the time of |\newtheorem|.  If
% |\theorembodyfont| is not specified or one defines
% |\theorembodyfont{}|, then the font used will be that defined by the
% |\theoremstyle|.
%
% \DescribeMacro\theoremheaderfont It is also possible to customize
% the font used for the theorem headers.  This is, however, a global
% declaration, and therefore there should be at most one
% |\theoremheaderfont| declaration in the preamble.\footnote{If it is
% actually necessary to have different header fonts, one has to define
% new theorem styles (substituting the desired font) or specify the
% information directly in the \texttt{\bslash newtheorem} declaration
% (the unclean variant).}
%
% \DescribeMacro\theorempreskipamount
% \DescribeMacro\theorempostskipamount Two additional parameters
% affect the vertical space around the theorem environments:
% |\theorempreskipamount| and |\theorempostskipamount| define,
% respectively, the spacing before and after such an environment.
% These parameters apply for all theorem sets and can be manipulated
% with the ordinary length macros.  They are rubber lengths
% (`\textsf{skips}'), and therefore can contain \texttt{plus} and
% \texttt{minus} parts.
%
% Since the definition of theorem sets should---most sensibly---be
% placed in the preamble, we only allow installation there.
% It is therefore possible to release the memory used here after
% |\begin{document}|, in order to make room for other
% applications.
%
%
% \subsection{Existing theorem styles}
%
% The following theorem styles exist to date:
% \begin{deflist}{marginbreak}
%    \item[plain]
%       This theorem style emulates the original \LaTeX{} definition,
%       except that additionally the parameters
%       |\theorem...skipamount| are used.
%    \item[break]
%       In this style, the theorem header is followed by a line
%       break.
%    \item[marginbreak]
%       The theorem number is set in the margin, and there is a line
%       break as in \texttt{break}.
%    \item[changebreak]
%       Like \texttt{break}, but with header number and text
%       interchanged.
%    \item[change]
%       Header number and text are interchanged, without a line break.
%    \item[margin]
%       The number is set in the left margin, without a line break.
% \end{deflist}
% All styles (except \texttt{plain}) select |\slshape| as the default
% |\theorembodyfont|.
%
%
% \subsection{Examples}
%
% Given the above theorem sets {\envfont Cor}, {\envfont Exa}
% and {\envfont Rem},
% suppose that the preamble also contains the declarations:
% \begin{verbatim}
%    \theoremstyle{marginbreak}   \newtheorem{Lem}[Cor]{Lemma}
%    \theoremstyle{change}
%    \theorembodyfont{\itshape}        \newtheorem{Def}[Cor]{Definition}
%
%    \theoremheaderfont{\scshape}
%\end{verbatim}
% Then the following are some typical examples of the typeset output
% resulting from their use.
%
% \begin{Cor}
%    This is a sentence typeset in the theorem environment {\envfont
%    Cor}.
% \end{Cor}
%
% \begin{Exa}
%    This is a sentence typeset in the theorem environment {\envfont
%    Exa}.
% \end{Exa}
%
% \begin{Rem}
%    This is a sentence typeset in the theorem environment {\envfont
%    Rem}.
% \end{Rem}
%
% \begin{Lem}[Ben User]
%    This is a sentence typeset in the theorem environment {\envfont
%    Lem}.
% \end{Lem}
%
% \begin{Def}[Very impressive Definition]
%    This is a sentence typeset in the theorem environment {\envfont
%    Def}.
% \end{Def}
%
% The last two examples show the effect of the optional argument to
% a theorem environment (it is the text typeset in parentheses).
%
%
% \section{Special Considerations}
% \label{sec:newfonts}
%
% Theoremheader and body are implemented as a unit. This means that
% the |\theoremheaderfont|  will inherit characteristics of the
% |\theorembodyfont| in \LaTeXe. Thus, if for example
% |\theorembodyfont| is |\itshape| and |\theoremheaderfont|
% is |\bfseries| the font selected for the header will have the
% characteristics `bold extended italic'. If this is not desired one
% should set the |\theoremheaderfont| to something like
% \begin{verbatim}
%      \theoremheaderfont{\normalfont\bfseries}
%\end{verbatim}
% i.e.\ supplying all necessary font information explicitly.
%
%
%
% \section{Acknowledgements}
%
% The publication of this set of macros was only possible with the
% help of Christina Busse (translating the manuscript into English),
% Joachim Pense (playing the r\^ole of typist), Chris Rowley
% (looking everything over) and many others providing useful
% suggestions.
%
% \MaybeStop{%
%    \begin{thebibliography}{9}
%       \bibitem{bk:GMS94} \textsc{M.~Goossens}, \textsc{F.~Mittelbach}
%         and \textsc{A.~Samarin}.
%         \newblock The \LaTeX{} Companion.
%         \newblock
%         Addison-Wesley, Reading, Massachusetts, 1994.
%       \bibitem{latex:source} \textsc{Lamport, Leslie.} \newblock
%                               \texttt{latex.tex}, version 2.09, date
%                               Feb. 1990.
%    \end{thebibliography}
%    \PrintIndex
%    \PrintChanges}
%
%
% \section{The documentation driver file}
%
% The next bit of code contains the documentation driver file for
% \TeX{}, i.e., the file that will produce the documentation you are
% currently reading. It will be extracted from this file by the
% \texttt{docstrip} program. Since it is the first code in the file
% one can alternatively process this file directly with \LaTeXe{} to
% obtain the documentation.
%
%    \begin{macrocode}
%<*driver>
\documentclass{ltxdoc}

\usepackage{theorem}

%<+driver>% The next few lines define theorem sets which are used
%<+driver>% in the example section of the documentation.

\theoremstyle{break}       \newtheorem{Cor}{Corollary}
\theoremstyle{plain}       \newtheorem{Exa}{Example}[section]
{\theorembodyfont{\upshape}\newtheorem{Rem}{Remark}}
\theoremstyle{marginbreak} \newtheorem{Lem}[Cor]{Lemma}
\theoremstyle{change}
\theorembodyfont{\itshape} \newtheorem{Def}[Cor]{Definition}

\theoremheaderfont{\scshape}

\RecordChanges

\begin{document}
   \DocInput{theorem.dtx}
\end{document}
%</driver>
%    \end{macrocode}
%
% \section{Definition of the Macros}
%
% If the file has been loaded before, we abort immediately. If not the
% package announces itself (this is actually done at the very top of
% the file---the way it is done isn't good style so don't copy it).
% \changes{v2.0e}{89/07/19}{Spaces removed from `typeout'.}
% \changes{v2.2b}{95/11/19}{Announce moved to top of file}
%    \begin{macrocode}
%<*package>
%\@ifundefined{theorem@style}{}{\endinput}
%\def\FMithmInfo{1995/11/19 v2.2b Theorem extension package (FMi)}
%\ProvidesPackage{theorem}[\FMithmInfo]
%    \end{macrocode}
%
%
% \subsection{Definition of theorem styles and fonts}
%
% All the definitions in this file are done globally to allow
% inputting this file inside a group.
%
% \begin{macro}{\theoremstyle}
%    Before a theorem style can be installed, the chosen style must be
%    known. For that reason, we must test to see that
%    |\th@|\meta{style} is known or, more precisely, that it is
%    different from |\relax|. If the style is not known then
%    |\th@plain| is used.
%    \begin{macrocode}
\gdef\theoremstyle#1{%
   \@ifundefined{th@#1}{\@warning
          {Unknown theoremstyle `#1'. Using `plain'}%
          \theorem@style{plain}}%
%    \end{macrocode}
%    We save the theorem style to be used in the token register
%    |\theorem@style|.
%    \begin{macrocode}
      {\theorem@style{#1}}%
%    \end{macrocode}
%    Now we ``evaluate'' the theorem style: this means, we call the
%    macro |\th@|\meta{style} which will activate the relevant
%    definitions which are contained in a separate file.  This is done
%    in a group to suppress changes to the current font.  This could
%    otherwise pose problems together with the new font selection
%    scheme\footnote{When I printed the original article using the new
%    font selection scheme I ended with a document with slanted
%    typefaces (text headings and all) simply because one of the
%    theorem styles used \texttt{\bslash sl} at toplevel.} if the
%    |\th@|\meta{style} is evaluated a second time.
% \changes{v2.1a}{90/02/15}{Group added around `th@..}
% \changes{v2.1a}{90/02/15}{Reseting of `..@begintheorem removed.}
%    \begin{macrocode}
      \begingroup
        \csname th@\the\theorem@style \endcsname
      \endgroup}
%    \end{macrocode}
% \end{macro}
%
%
% \begin{macro}{\@begintheorem}
% \changes{v2.1a}{90/02/15}{Relaxed at toplevel.}
% \begin{macro}{\@opargbegintheorem}
% \changes{v2.1a}{90/02/15}{Relaxed at toplevel.}
%    We reset |\@begintheorem| and |\@opargbegintheorem| to
%    |\relax| since these commands are no longer necessary at
%    toplevel. This will save a few tokens.
%    \begin{macrocode}
\global\let\@begintheorem\relax
\global\let\@opargbegintheorem\relax
%    \end{macrocode}
% \end{macro}
% \end{macro}
%
% \begin{macro}{\theorem@style}
%    Obviously the \textsf{token} register used above has to be
%    allocated. To assure the utmost compatibility with the
%    original \LaTeX{} definition, we set the default theorem style to
%    \texttt{plain}, which implements the usual \LaTeX{} convention.
% \changes{v2.0f}{89/07/20}{Unnecessary `global' removed.}
%    \begin{macrocode}
\newtoks\theorem@style
\global\theorem@style{plain}
%    \end{macrocode}
% \end{macro}
%
% \begin{macro}{\theorembodyfont}
% \begin{macro}{\theorem@bodyfont}
%    If the |\theorembodyfont| is set by the user then it should not
%    interact with the default font set in the theorem style. When the
%    new font selection is in force this may happen if, for example,
%    the default is |\itshape| and the new |\theorembodyfont| is
%    |\sffamily|. So we add a |\reset@font| command in front of the
%    user definition.
% \changes{v2.0f}{89/07/20}{Unnecessary `global' removed.}
% \changes{v2.0g}{89/09/19}{`theoremfont to `theorembodyfont.}
% \changes{v2.1c}{92/06/22}{`reset@font added and logic changed.}
% \changes{v2.1d}{92/10/28}{Reset to empty if arg is empty}
%    \begin{macrocode}
\gdef\theorembodyfont#1{%
%    \end{macrocode}
%    We check if the argument supplied is empty and if so put nothing
%    into the |\theorem@bodyfont| token register to allow for
%    |\theorembodyfont{}| as a mean of using the default of the
%    current |\theoremstyle|.
%    \begin{macrocode}
   \def\@tempa{#1}%
   \ifx\@tempa\@empty
    \theorem@bodyfont{}%
   \else
    \theorem@bodyfont{\reset@font#1}%
   \fi
   }
\newtoks\theorem@bodyfont
\global\theorem@bodyfont{}
%    \end{macrocode}
% \end{macro}
% \end{macro}
%
%
% \begin{macro}{\theoremheaderfont}
% \changes{v2.0g}{89/09/19}{Macro added.}
%    The font for the theorem headers is handled differently because
%    this definition applies to all theorem styles.
%    \begin{macrocode}
\gdef\theoremheaderfont#1{\gdef\theorem@headerfont{#1}%
%    \end{macrocode}
%    After using the macro once it is redefined to produce an error
%    message.
%    \begin{macrocode}
       \gdef\theoremheaderfont##1{%
        \typeout{\string\theoremheaderfont\space should be used
                 only once.}}}
%    \end{macrocode}
% \end{macro}
%
% \begin{macro}{\theorem@headerfont}
% \changes{v2.0g}{89/09/19}{Macro added.}
%    To set the |\theorem@headerfont| default we first test if
%    the new fontselection scheme is in force.
% \changes{v2.1a}{90/02/15}{Support for new font selection.}
%    \begin{macrocode}
\ifx\upshape\undefined
%    \end{macrocode}
%    If not we define it to expand into |\bfseries|. We don't use
%    |\let| just in case a following style option redefines this
%    macro.
%    \begin{macrocode}
\gdef\theorem@headerfont{\bfseries}
%    \end{macrocode}
%    Otherwise we reset the current shape before calling |\bfseries|.
%    \begin{macrocode}
\else \gdef\theorem@headerfont{\normalfont\bfseries}\fi
%    \end{macrocode}
% \end{macro}
%
% \begin{macro}{\th@plain}
% \begin{macro}{\th@break}
% \begin{macro}{\th@marginbreak}
% \begin{macro}{\th@changebreak}
% \begin{macro}{\th@change}
% \begin{macro}{\th@margin}
%    The different styles are defined in macros such as
%    |\th@plain|.  Since memory space is precious in
%    ``non-Big-versions'', we have to avoid offering too many unused
%    definitions.  Therefore we define these styles in separate files
%    that can be loaded on demand.  Thus the commands themselves only
%    load these files. We use |\@input@|, a \LaTeXe{} internal command
%    that ensures that the file will be listed with |\listfiles|.
% \changes{v2.2a}{94/02/02}{Use `@input@ to load theorem layout files}
%    \begin{macrocode}
\gdef\th@plain{\@input@{thp.sty}}
\gdef\th@break{\@input@{thb.sty}}
\gdef\th@marginbreak{\@input@{thmb.sty}}
\gdef\th@changebreak{\@input@{thcb.sty}}
\gdef\th@change{\@input@{thc.sty}}
\gdef\th@margin{\@input@{thm.sty}}
%    \end{macrocode}
%    This list will be expanded when new styles become
%    available.  For testing,  just append new theorem substyles
%    as document options.
% \end{macro}
% \end{macro}
% \end{macro}
% \end{macro}
% \end{macro}
% \end{macro}
%
%
%
% \subsection{Definition of a new theorem set}
%
% As already pointed out, a new theorem environment can be defined in
% three different ways:
% \begin{verbatim}
%    \newtheorem{Lem}{Lemma}
%    \newtheorem{Lem}{Lemma}[section]
%    \newtheorem{Lem}[Theorem]{Lemma}
%\end{verbatim}
% The function of the macro |\newtheorem| is to recognize these cases
% and then to branch into one of the three macros |\@ynthm|, |\@xnthm|
% or |\@othm|.  This mechanism is adopted unchanged from
% \cite{latex:source}; the essential point here is that, for example,
% in the second case, the arguments \texttt{Lem}, \texttt{Lemma} and
% \texttt{section} are passed over to the macro |\@xnthm|.
%
% We inspect this case first because the others present fewer problems,
% and thus are easily derived from this one.
%
% \begin{macro}{\@xnthm}
%    For our example arguments, the macro |\@xnthm| must fulfill
%    the following:
%    \begin{itemize}
%       \item
%          Define a new \LaTeX-counter `Lem'
%       \item
%          reset this counter within a |\section|
%       \item
%          define the macro |\theLem|
%       \item
%          define the environment macros |\Lem| and
%          |\endLem| using the current
%          |\theoremstyle| and |\theorem@bodyfont|.
%    \end{itemize}
%    Obviously, all this should happen only if the first argument
%    of |\@xnthm| (i.e.\ \texttt{Lem} in our example) is chosen so
%    as not to conflict with any previously defined commands or
%    environments.  This test is performed by the \LaTeX{} macro
%    |\@ifdefinable|.
%    \begin{macrocode}
\gdef\@xnthm#1#2[#3]{\expandafter\@ifdefinable\csname #1\endcsname
%    \end{macrocode}
%    Therefore, the first argument of |\@ifdefinable| is the
%    expansion (in the example, |\Lem|) of
%    |\csname||#1\endcsname|.  The second argument is
%    executed only if the test has been completed successfully.
%    \begin{macrocode}
   {%
%    \end{macrocode}
%    Now we define the new counter. The names of the \LaTeX{}
%    macros employed should speak for themselves:
% \changes{v2.2b}{1995/10/07}{Use \cs{@newctr}}
% \changes{v2.2c}{1995/11/23}{Use \cs{@newctr} properly.}
%    \begin{macrocode}
    \@definecounter{#1}\@newctr{#1}[#3]%
%    \end{macrocode}
%    Using |\@newctr| will give a proper error message if the counter
%    in |#3| is not defined.
%    In defining `|\theLem|' we must generate the desired macro
%    name by use of |\expandafter| and |\csname|.
%    \begin{macrocode}
    \expandafter\xdef\csname the#1\endcsname
%    \end{macrocode}
%    An |\xdef| is used in order to make the definition global,
%    and to ensure that it contains the replacement texts of
%    |\@thmcountersep| and |\@thmcounter|.\footnote{These
%    two macros can be defined by the document style.  Their default
%    values produce a `\texttt{.}' as separation and an arabic
%    representation of the number.} However, not everything should be
%    expanded. For example, it saves space to use |\thesection|
%    instead of its---at times---lengthy expansion.
%    \begin{macrocode}
      {\expandafter \noexpand \csname the#3\endcsname
       \@thmcountersep \@thmcounter{#1}}%
%    \end{macrocode}
%    Thus with the defaults of \LaTeX{}, |\theLem| would be
%    replaced by the command sequence |\thesection.\arabic{Lem}|.
%
%    We will now look at the definition of the macro which is executed
%    at the beginning of the actual environment (in our example this
%    macro is |\Lem|). It should be noted that we use an
%    ``|\expandafter| trick'' to expand only certain parts of the
%    replacement text at the time of the definition.
%    \begin{macrocode}
    \def\@tempa{\global\@namedef{#1}}%
    \expandafter \@tempa \expandafter{%
%    \end{macrocode}
%    First, the macro that contains the current definitions of
%    |\@begintheorem| and |\@opargtheorem| should be called
%    up.  The name of this macro---as is already known---has the form
%    |\th@|\meta{theorem\ style}; therefore, it must be called by
%    \begin{macrocode}
      \csname th@\the \theorem@style
%    \end{macrocode}
%    In addition the default theorem font should be changeable, i.e.\
%    we have to insert the contents of |\theorem@bodyfont|. For
%    that reason, we expand even further, beyond |\endcsname|,
%    and thus insert the contents of the \textsf{token} register
%    |\theorem@bodyfont| in the replacement text.
%    \begin{macrocode}
            \expandafter \endcsname \the \theorem@bodyfont
%    \end{macrocode}
%    Now it is time to call the macro |\@thm| which takes over
%    the further processing.  It has two arguments: the current
%    counter name (in our example, \texttt{Lem}), and the text of the
%    label (in our example, \texttt{Lemma}).
%    \begin{macrocode}
     \@thm{#1}{#2}}%
%    \end{macrocode}
%    With this, the `sub-definition' is complete.  The macro
%    |\@endtheorem| ends a theorem environment and is, so far, nothing
%    but an |\endtrivlist|.  (Hence it is defined globally, and not
%    within the theorem styles.\footnote{This has to be changed as
%    soon as theorem styles that change \texttt{\bslash @endtheorem}
%    exist.  In such a case, all existing styles must be changed as
%    well since they will have to reset the macro.}) Therefore, we can
%    set it equivalent to the macro that ends the theorem set (in our
%    example, |\endLem|).  However, if some day theorem styles exist
%    that do change |\@endtheorem|, we would have to use the
%    commented-out line instead.
%    \begin{macrocode}
    \global \expandafter \let \csname end#1\endcsname \@endtheorem
%   \global\@namedef{end#1}{\@endtheorem}%
%    \end{macrocode}
%    With these commands all the required definitions are employed,
%    unless the test |\@ifdefinable| has failed.  Therefore, we
%    end the second argument of this macro and with it the definition
%    of |\@xnthm|.
%    \begin{macrocode}
   }}
%    \end{macrocode}
% \end{macro}
%
% \begin{macro}{\@ynthm}
%    The definition of |\@ynthm| is completely analogous. In
%    this case the new counter that is defined is not reset within
%    another counter; thus the definition of |\the...|
%    is simplified:
%    \begin{macrocode}
\gdef\@ynthm#1#2{\expandafter\@ifdefinable\csname #1\endcsname
   {\@definecounter{#1}%
    \expandafter\xdef\csname the#1\endcsname{\@thmcounter{#1}}%
%    \end{macrocode}
%    The rest of the definition corresponds literally to that of
%    |\@xnthm|:
%    \begin{macrocode}
    \def\@tempa{\global\@namedef{#1}}\expandafter \@tempa
     \expandafter{\csname th@\the \theorem@style \expandafter
     \endcsname \the\theorem@bodyfont \@thm{#1}{#2}}%
    \global \expandafter \let \csname end#1\endcsname \@endtheorem}}
%    \end{macrocode}
% \end{macro}
%
% \begin{macro}{\@othm}
%    The definition of |\@othm| does not contain anything new.
%    \begin{macrocode}
\gdef\@othm#1[#2]#3{%
%    \end{macrocode}
% \changes{v2.2b}{95/10/07}{Build in check for counter defined}
%    We do not define a new counter but instead use one that has
%    already been defined.  Thus the only definition we need is that
%    of this pseudo-counter (i.e.\ |\the|\meta{env.\ name}).
%    First we check if |#2| corresponds to a known counter name.
%    \begin{macrocode}
  \expandafter\ifx\csname c@#2\endcsname\relax
   \@nocounterr{#2}%
  \else
   \expandafter\@ifdefinable\csname #1\endcsname
   {\expandafter \xdef \csname the#1\endcsname
     {\expandafter \noexpand \csname the#2\endcsname}%
%    \end{macrocode}
%    All other parts of the definition can be adopted from
%    |\@xnthm|. We have to remember, though, that in this case
%    the name of the current counter and the theorem label have moved
%    to the second and third arguments.
%    \begin{macrocode}
    \def\@tempa{\global\@namedef{#1}}\expandafter \@tempa
     \expandafter{\csname th@\the \theorem@style \expandafter
     \endcsname \the\theorem@bodyfont \@thm{#2}{#3}}%
    \global \expandafter \let \csname end#1\endcsname \@endtheorem}%
  \fi}
%    \end{macrocode}
% \end{macro}
%
% \subsection{Macros that are employed in a {\envfont theorem}
%    environment}
%
% \begin{macro}{\@thm}
%    The macro |\@thm| has to increase the current counter.  Then,
%    depending on whether the environment has (or does not have) an
%    optional argument, it has to branch into either |\@begintheorem|
%    or |\@opargtheorem|.
%    \begin{macrocode}
\gdef\@thm#1#2{\refstepcounter{#1}%
%    \end{macrocode}
%    Now we start a {\envfont trivlist} environment, and give
%    |\@topsep| and |\@topsepadd| the values of the \textsf{skip}
%    registers |\theorempreskipamount| and |\theorempostskipamount|.
%    The value in |\@topsep| is the vertical space that is inserted by
%    the first (and only) |\item| in our |\trivlist| whilst
%    |\@topsepadd| is inserted by |\@endparenv| at the end of that
%    {\envfont trivlist} environment.  By using these registers, we
%    obtain the desired space around a {\envfont theorem} environment.
%    \begin{macrocode}
   \trivlist
   \@topsep \theorempreskipamount               % used by first \item
   \@topsepadd \theorempostskipamount           % used by \@endparenv
%    \end{macrocode}
%    Now we have to test whether an optional argument has been given.
% \changes{v2.0d}{89/07/18}{Forgotten percent added.}
%    \begin{macrocode}
   \@ifnextchar [%
%    \end{macrocode}
%    If there is an optional argument, we will call |\@ythm|, and
%    move the arguments read back into the input stream.
%    \begin{macrocode}
   {\@ythm{#1}{#2}}%
%    \end{macrocode}
%    If not, we call |\@begintheorem|. Its first argument is the
%    name of the theorem set (hence the second argument of
%    |\@thm|). Its second argument is the macro that produces the
%    current number.
%    \begin{macrocode}
   {\@begintheorem{#2}{\csname the#1\endcsname}\ignorespaces}}
%    \end{macrocode}
% \end{macro}
%
% \begin{macro}{\@xthm}
% \begin{macro}{\@ythm}
%    Both these macros were originally called by |\@thm|.  We do
%    not need |\@xthm| anymore, hence we reset it to
%    |\relax|. The definition of |\@ythm| has not changed at
%    all from its definition in \LaTeX. In order to make the macros
%    easier to understand, we will
%    nevertheless present it (commented out).
%    \begin{macrocode}
\global\let\@xthm\relax
% \def\@ythm#1#2[#3]{\@opargbegintheorem{#2}{\csname
%           the#1\endcsname}{#3}\ignorespaces}
%</package>
%    \end{macrocode}
%    The primitive |\ignorespaces| in |\@ythm| and
%    |\@thm| is needed to remove the spaces between the
%    |\begin{...}| and the actual text.
% \end{macro}
% \end{macro}
%
%
% \subsection{Definition of the theorem substyles}
%
% As already pointed out, the theorem substyles, defined below, are
% only loaded when necessary.  Note that all these substyles, except
% \texttt{plain}, have |\slshape| as the default body font.
%
% \subsubsection{The \texttt{plain} style}
%
% \label{dok:th@plain} As the following macros use |@|, we have
% to locally set the |\catcode| of this symbol to ``letter''.
% This happens within a group, so that we do not have to worry about
% which |\catcode| that symbol had before.
%    \begin{macrocode}
%<*thp>
\begingroup \makeatletter
%    \end{macrocode}
% Since we are now within a group, we must make all definitions
% globally.  First we make sure that \texttt{theorem.sty} is loaded.
% This will allow us to use this file as a document style option
% without having to call \texttt{theorem} itself as an option.  At the
% same time, we assure that at least version 2 is loaded, since
% |\theorem@style| was not defined in earlier versions.
%    \begin{macrocode}
\@ifundefined{theorem@style}{\input{theorem.sty}}{}
\ProvidesFile{thp.sty}
             [\FMithmInfo]
%    \end{macrocode}
%
%
% \begin{macro}{\th@plain}
% \begin{macro}{\@begintheorem}
% \begin{macro}{\@opargbegintheorem}
%    |\theoremstyle{plain}| corresponds to the original
%    definition, except that the distances to the surrounding text are
%    determined by the parameters |\theorempreskipamount| and
%    |\theorempostskipamount|.
%    First we set the default body font.
%    \begin{macrocode}
\gdef\th@plain{\normalfont\itshape
%    \end{macrocode}
%    Then we define |\@begintheorem| and
%    |\@opargbegintheorem|. These two macros define how the
%    header of a theorem is typeset.  |\@opargbegintheorem| will
%    be called if a \textsf{theorem} environment with an optional
%    argument is encountered; otherwise, the header is constructed by
%    calling |\@begintheorem|.  If one of these macros is
%    executed, we are within a \textsf{trivlist} environment started by
%    |\@thm|.  So the theorem header is produced with an
%    |\item| command.
%
%    Instead of specifying the header font directly, all standard
%    theorem styles use the |\theorem@headerfont| macro to allow
%    customization.  The extra space (|\labelsep|) is necessary
%    because of problems in the \textsf{trivlist} environment.
%    \begin{macrocode}
  \def\@begintheorem##1##2{%
        \item[\hskip\labelsep \theorem@headerfont ##1\ ##2]}%
%    \end{macrocode}
%    The definition of |\@opargbegintheorem| is completely
%    analogous. The only difference is the fact that there exists a
%    third argument (which is the optional parameter of the
%    environment and contains additional information about the
%    theorem).  Customarily we enclose it in parentheses.
%    \begin{macrocode}
\def\@opargbegintheorem##1##2##3{%
   \item[\hskip\labelsep \theorem@headerfont ##1\ ##2\ (##3)]}}
%    \end{macrocode}
% \end{macro}
% \end{macro}
% \end{macro}
%
% \noindent
% We conclude with an |\endgroup| to restore the
% |\catcode| of |@|.
%    \begin{macrocode}
\endgroup
%</thp>
%    \end{macrocode}
%
%
% \subsubsection{The \texttt{break} style}
%
% This style option is stored in the file |thb.sty|.
% For the next two lines see the documentation for |\th@plain| on
% page~\pageref{dok:th@plain}.
%    \begin{macrocode}
%<*thb>
\begingroup \makeatletter
\@ifundefined{theorem@style}{\input{theorem.sty}}{}
\ProvidesFile{thb.sty}
             [\FMithmInfo]
%    \end{macrocode}
%
% \begin{macro}{\th@break}
%    |\theoremstyle{break}| produces a line break after the name
%    of the theorem.  The font is |\slshape|.  Hence, we define
%    |\th@break| as follows:
%    \begin{macrocode}
\gdef\th@break{\normalfont\slshape
  \def\@begintheorem##1##2{\item[%
%    \end{macrocode}
%    We run into the following problem: it is not possible to create
%    the header with |\item[|\meta{title}|]| and then start
%    a new line by, for example, |\mbox{}\\|. Such a definition
%    will fail whenever a list environment follows immediately.  With
%    the above construction, the |\mbox{}| causes the switch
%    |@inlabel| (cf.\ definition of |\list| and
%    |\trivlist| in \cite{latex:source}) to be set to \texttt{false}
%    and so the following list will insert additional vertical
%    space (|\topskip|). This is quite annoying. Therefore, we
%    create the line break within the |\item|. In order to ensure
%    that the text will begin at the proper position in the following
%    line, we simply pretend that the label does not take any
%    room.\footnote{This will lead to problems whenever very high
%    symbols occurring in the line tower into the
%    heading. So, something else has to be done here sometime.}
%    \begin{macrocode}
       \rlap{\vbox{\hbox{\hskip \labelsep\theorem@headerfont ##1\ ##2}%
                   \hbox{\strut}}}]}%
%    \end{macrocode}
%    Again, the definition of |\@opargbegintheorem| is
%    completely analogous.
%    \begin{macrocode}
\def\@opargbegintheorem##1##2##3{%
  \item[\rlap{\vbox{\hbox{\hskip \labelsep \theorem@headerfont
                     ##1\ ##2\ (##3)}%
                    \hbox{\strut}}}]}}
\endgroup
%</thb>
%    \end{macrocode}
% \end{macro}
%
%
% \subsubsection{The \texttt{changebreak} style}
%
%    \begin{macrocode}
% This style option is stored in the file |thcb.sty|.
%    \begin{macrocode}
%<*thcb>
\begingroup \makeatletter
\@ifundefined{theorem@style}{\input{theorem.sty}}{}
\ProvidesFile{thcb.sty}
             [\FMithmInfo]
%    \end{macrocode}
%
% \begin{macro}{\th@changebreak}
%    The \texttt{change-break} theorem style is like \texttt{break}
%    but with interchange of theorem name and theorem number.  Thus we
%    define |\th@changebreak| as follows:
%    \begin{macrocode}
\gdef\th@changebreak{\normalfont\slshape
  \def\@begintheorem##1##2{\item
       [\rlap{\vbox{\hbox{\hskip\labelsep \theorem@headerfont ##2\ ##1}%
                    \hbox{\strut}}}]}%
\def\@opargbegintheorem##1##2##3{%
  \item[\rlap{\vbox{\hbox{\hskip\labelsep \theorem@headerfont
                     ##2\ ##1\ (##3)}%
                    \hbox{\strut}}}]}}
\endgroup
%</thcb>
%    \end{macrocode}
% \end{macro}
%
%
% \subsubsection{The \texttt{change} style}
%
% This style option is stored in the file |thc.sty|.
%    \begin{macrocode}
%<*thc>
\begingroup \makeatletter
\@ifundefined{theorem@style}{\input{theorem.sty}}{}
\ProvidesFile{thc.sty}
             [\FMithmInfo]
%    \end{macrocode}
%
% \begin{macro}{\th@change}
%    The \texttt{change} theorem style corresponds to the
%    \texttt{change break} style without a linebreak after the header.
%    To say it in another way, it's the same as the \texttt{plain}
%    style but with number and name interchanged and |\slshape| as the
%    default font.
%    \begin{macrocode}
\gdef\th@change{\normalfont\slshape
  \def\@begintheorem##1##2{\item
       [\hskip\labelsep \theorem@headerfont ##2\ ##1]}%
\def\@opargbegintheorem##1##2##3{%
  \item[\hskip\labelsep \theorem@headerfont ##2\ ##1\ (##3)]}}
\endgroup
%</thc>
%    \end{macrocode}
% \end{macro}
%
%
% \subsubsection{The \texttt{marginbreak} style}
%
% This style option is the one used most often at Mainz.  It is saved
% in the file |thmb.sty|.
%    \begin{macrocode}
%<*thmb>
\begingroup \makeatletter
\@ifundefined{theorem@style}{\input{theorem.sty}}{}
\ProvidesFile{thmb.sty}
             [\FMithmInfo]
%    \end{macrocode}
%
% \begin{macro}{\th@marginbreak}
%    The \texttt{margin break} style is nearly the same as the
%    \texttt{change break} style. The only difference is the placement
%    of the theorem number.  We use |\llap| to place it in the left
%    margin.
%
%    In this style |\labelsep| denotes the separation between the
%    number and the text.
%    \begin{macrocode}
\gdef\th@marginbreak{\normalfont\slshape
  \def\@begintheorem##1##2{\item
       [\rlap{\vbox{\theorem@headerfont
                    \hbox{\llap{##2}\hskip\labelsep ##1}%
                    \hbox{\strut}}}]}%
\def\@opargbegintheorem##1##2##3{%
  \item[\rlap{\vbox{\theorem@headerfont
                    \hbox{\llap{##2}\hskip\labelsep ##1\ (##3)}%
                    \hbox{\strut}}}]}}
\endgroup
%</thmb>
%    \end{macrocode}
% \changes{v2.1c}{90/02/17}{`endgroup outside macrocode env.}
% \end{macro}
%
%
%
% \subsubsection{The \texttt{margin} style}
%
% This style option is stored in the file |thm.sty|.
%    \begin{macrocode}
%<*thm>
\begingroup \makeatletter
\@ifundefined{theorem@style}{\input{theorem.sty}}{}
\ProvidesFile{thm.sty}
             [\FMithmInfo]
%    \end{macrocode}
%
% \begin{macro}{\th@margin}
%    Again this is only a variant of the theorem styles described
%    above without any new ideas.
%    \begin{macrocode}
\gdef\th@margin{\normalfont\slshape
  \def\@begintheorem##1##2{\item
       [\theorem@headerfont \llap{##2}\hskip\labelsep ##1]}%
\def\@opargbegintheorem##1##2##3{%
  \item[\theorem@headerfont \llap{##2}\hskip\labelsep ##1\ (##3)]}}
\endgroup
%</thm>
%    \end{macrocode}
% \end{macro}
%
%
%
%
% \subsection{Final Definitions}
%
% \begin{macro}{\theorempreskipamount}
% \begin{macro}{\theorempostskipamount}
%    The \textsf{skip} parameters that regulate the vertical empty space
%    before and after the {\envfont theorem} environment have
%    to be allocated as well.
% \changes{v2.0f}{89/07/20}{Unnecessary `global' removed.}
%    \begin{macrocode}
%<*package>
\newskip\theorempreskipamount
\newskip\theorempostskipamount
%    \end{macrocode}
%    Since we have used the same values for all theorem sets, we now can
%    assign them.
%    \begin{macrocode}
\global\setlength\theorempreskipamount{12pt plus 5pt minus 3pt}
\global\setlength\theorempostskipamount{8pt plus 3pt minus 1.5pt}
%    \end{macrocode}
% \end{macro}
% \end{macro}
%
%
% \begin{macro}{\@endtheorem}
%    The same holds for the macro |\@endtheorem|, which ends a
%    {\envfont theorem} environment. Since it is the same for all
%    theorem sets, it is removed from the macros |\th@|\meta{style}.
%    It simply ends the {\envfont trivlist} environment, which was
%    begun in |\@thm|.
%    \begin{macrocode}
\global\let\@endtheorem=\endtrivlist
%    \end{macrocode}
% \end{macro}
%
%
%
% \begin{macro}{\@preamblecmds}
%    All macros defined above are to be used only in the preamble.
%    Therefore, we insert them in |\@preamblecmds| which will disable
%    them at begin document. This is done by the internal \LaTeXe{}
%    command |\@onlypreamble|.
% \changes{v2.0g}{89/09/19}{`theoremheaderfont and `theorembodyfont
%    added.}
% \changes{v2.2a}{94/02/03}{Use \LaTeXe{} interface}
%    \begin{macrocode}
\@onlypreamble\@xnthm
\@onlypreamble\@ynthm
\@onlypreamble\@othm
\@onlypreamble\newtheorem
\@onlypreamble\theoremstyle
\@onlypreamble\theorembodyfont
\@onlypreamble\theoremheaderfont
%    \end{macrocode}
% \end{macro}
%
% Finally we declare the \texttt{plain} theorem style to be the default.
% \changes{v2.1b}{90/06/04}{Fixed missing `item by declaring default
%    style.}
%    \begin{macrocode}
\theoremstyle{plain}
%</package>
%    \end{macrocode}
%
% \Finale

\endinput
}{}
\ProvidesFile{thcb.sty}
             [\FMithmInfo]
%    \end{macrocode}
%
% \begin{macro}{\th@changebreak}
%    The \texttt{change-break} theorem style is like \texttt{break}
%    but with interchange of theorem name and theorem number.  Thus we
%    define |\th@changebreak| as follows:
%    \begin{macrocode}
\gdef\th@changebreak{\normalfont\slshape
  \def\@begintheorem##1##2{\item
       [\rlap{\vbox{\hbox{\hskip\labelsep \theorem@headerfont ##2\ ##1}%
                    \hbox{\strut}}}]}%
\def\@opargbegintheorem##1##2##3{%
  \item[\rlap{\vbox{\hbox{\hskip\labelsep \theorem@headerfont
                     ##2\ ##1\ (##3)}%
                    \hbox{\strut}}}]}}
\endgroup
%</thcb>
%    \end{macrocode}
% \end{macro}
%
%
% \subsubsection{The \texttt{change} style}
%
% This style option is stored in the file |thc.sty|.
%    \begin{macrocode}
%<*thc>
\begingroup \makeatletter
\@ifundefined{theorem@style}{% \iffalse meta-comment
%
% Copyright (C) 1993-2024
%
% The LaTeX Project and any individual authors listed elsewhere
% in this file.
%
% This file is part of the Standard LaTeX `Tools Bundle'.
% -------------------------------------------------------
%
% It may be distributed and/or modified under the
% conditions of the LaTeX Project Public License, either version 1.3c
% of this license or (at your option) any later version.
% The latest version of this license is in
%    https://www.latex-project.org/lppl.txt
% and version 1.3c or later is part of all distributions of LaTeX
% version 2005/12/01 or later.
%
% The list of all files belonging to the LaTeX `Tools Bundle' is
% given in the file `manifest.txt'.
%
% \fi
%
%%
%
%
\def\FMithmInfo{2023/07/05 v2.2c Theorem extension package (FMi)}
%
% \ProvidesFile{theorem.dtx}[\FMithmInfo]
%\iffalse   % this is a METACOMMENT !
%%
%% Package `theorem' to use with LaTeX2e
%% Copyright (C) 1989-1995 Frank Mittelbach, all rights reserved.
%<+package>\@ifundefined{theorem@style}{}{\endinput}
%<+package>\NeedsTeXFormat{LaTeX2e}[1995/06/01]
%<+package>\ProvidesPackage{theorem}
%<+package>         [\FMithmInfo]
%
% \fi
%
% \changes{v1.0}{87/05/07}{First draft.}
% \changes{v1.0a}{87/05/07}{`thistheoremstyle' added.}
% \changes{v1.1a}{87/07/29}{New concept: redefine the internal
%                          \protect\LaTeX{} commands `@xnthm' etc.}
% \changes{v1.1b}{87/07/31}{`theoremfont' used to specify font.}
% \changes{v1.2a}{88/01/08}{`@endtheorem' now globally defined.}
% \changes{v1.2b}{88/01/11}{`newtheorem' only allowed in the preamble.}
% \changes{v1.3a}{88/02/14}{`theoremfont' now a toks register.}
%
% \changes{v2.0a}{88/11/16}{New implementation and new documentation.}
% \changes{v2.0a}{88/11/16}{No layout changes but internal modifications
%                           old substyles won't work.}
% \changes{v2.0b}{89/01/30}{`global' used for all definitions.}
% \changes{v2.0c}{89/06/15}{Documentation translated to English.
%                            Thanks to LLL and Nina}
% \changes{v2.1a}{90/02/15}{Support for new font selection.}
% \changes{v2.2a}{94/02/03}{Cleanup for LaTeX2e}
%
% \DoNotIndex{\@endparenv,\@ifdefinable,\@ifnextchar,\@ifundefined}
% \DoNotIndex{\@namedef,\@nameuse,\@spaces,\@tempa,\@warning}
% \DoNotIndex{\ }
% \DoNotIndex{\begingroup,\bfseries}
% \DoNotIndex{\csname}
% \DoNotIndex{\def,\do,\docdate}
% \DoNotIndex{\endcsname,\endgroup,\endinput,\endtrivlist,\expandafter}
% \DoNotIndex{\filedate,\fileversion}
% \DoNotIndex{\gdef,\global}
% \DoNotIndex{\hbox,\hskip}
% \DoNotIndex{\ignorespaces,\input,\itshape,\item}
% \DoNotIndex{\labelsep,\let,\llap}
% \DoNotIndex{\makeatletter}
% \DoNotIndex{\noexpand}
% \DoNotIndex{\refstepcounter,\relax,\rlap}
% \DoNotIndex{\setlength,\slshape,\space,\string,\strut}
% \DoNotIndex{\the,\trivlist,\typeout}
% \DoNotIndex{\vbox}
% \DoNotIndex{\xdef}
%
% \def\deflabel#1{\ttfamily #1\hfill}
% \def\deflist#1{\begin{list}{}{\settowidth\labelwidth{\ttfamily #1}%
%                               \setlength\leftmargin\labelwidth
%                               \addtolength\leftmargin\labelsep
%                               \let\makelabel\deflabel}}
% \def\enddeflist{\end{list}}
% \def\envfont{\normalfont\ttfamily}
%
% \MakeShortVerb{\|}
%
% \newcommand{\TUB}{{\sl TUGboat\/}}
%
% \GetFileInfo{theorem.dtx}
% \title{An Extension of the
%        \LaTeX{} theorem environment\thanks{This file
%        has version number \fileversion, last
%        revised \filedate.}}
% \author{Frank Mittelbach}
%
% \MaintainedByLaTeXTeam{tools}
% \maketitle
%
% \begin{abstract}
%    The macros described in this paper yield an extension of the
%    \LaTeX{} theorem mechanism. It is designed to satisfy the
%    different requirements of various journals. Thus, the
%    layout of the ``theorems'' can be manipulated by determining a
%    ``style''. This article describes not only the use, but
%    also the definition, of the necessary macros.
% \end{abstract}
%
% \section*{Preface to version 2.2}
%
% For \LaTeXe{} this package did not need any fundamental changes. I
% only modified the messages generated so that theorem layout styles
% will show up with the |\listfiles| command and cleaned the section
% on the New Font Selection Scheme since this is now included in
% \LaTeX.
%
% \section*{Preface to version 2.1}
%
% This version is identical to 2.0g described in \TUB\ 10\#3 except
% for some internal defaults which are now set depending on the used
% font selection scheme.
%
% This was done to avoid unpleasant surprises if the new font selection
% scheme is in force. For further details see
% section~\ref{sec:newfonts} and \cite{bk:GMS94}.
%
% \section{Introduction}
%
% For our purposes here, ``theorems'' are labelled enunciations,
% often set off from the main text by extra space and a font change.
% Theorems, corollaries, conjectures, definitions, and remarks
% are all instances of ``theorems''.  The ``header'' of these structures
% is composed of a label (such as \textsc{Theorem} or \textsc{Remark})
% and a number which serializes an item in the sequence of items
% with the same label.
%
% Shortly after the introduction of \LaTeX{} at the Fachbereich
% Mathematik in Mainz, the desire to manipulate the layout of
% ``theorems'' arose.  In Mainz, the following two conventions came
% into general use:
% \begin{enumerate}
%   \item The number of the theorem is shown in the margin.
%   \item There is a line break at
%         the end of the theorem header.
% \end{enumerate}
%
% Additionally, some journals require different formats which depend on
% the ``sort of theorem'': e.g.\ often remarks and definitions are set
% in |\upshape|, while |\itshape| is employed for main theorems.
%
% Confronted with these requirements, a theorem environment was
% developed in Mainz which allows separate determination of the
% layout of the ``theorems sets'', comparable to |\pagestyle|.
%
%
% \section{The user interface}
% \subsection{Defining new theorem sets}
%
% \DescribeMacro\newtheorem
% As in the original \LaTeX{} version, the command |\newtheorem|
% defines a new ``theorem set'' or ``theorem-like structure''.
% Two required arguments name the new environment and give the text
% to be typeset with each instance of the new ``set'', while
% an optional argument determines how
% the ``set'' is enumerated:
% \begin{description}
%    \item[\ttfamily \bslash newtheorem\{foo\}\{bar\}]
%       The theorem set {\envfont foo} (whose name is \texttt{bar})
%       uses its own counter.
%    \item[\ttfamily \bslash newtheorem\{foo2\}{[foo]}\{bar2\}]
%       The theorem set {\envfont foo2} (printed name \texttt{bar2})
%       uses the same counter as the theorem set \texttt{foo}.
%    \item[\ttfamily \bslash newtheorem\{foo3\}\{bar3\}{[section]}]
%       The theorem set {\envfont foo3} (printed name \texttt{bar3}) is
%       enumerated within the counter \texttt{section}, i.e.\ with every
%       new |\section| the enumeration begins again with 1, and
%       the enumeration is composed from the section-number and the
%       theorem counter itself.
% \end{description}
%
% \DescribeMacro\theoremstyle
% Additionally, the command |\theoremstyle| can define the layout
% of various, or all, theorem sets.  It should be noted that any
% theorem set defined by |\newtheorem| is typeset in the
% |\theoremstyle| that is current at the time of the definition.
% Thus, the following
% \begin{verbatim}
%    \theoremstyle{break}        \newtheorem{Cor}{Corollary}
%    \theoremstyle{plain}        \newtheorem{Exa}{Example}[section]
%\end{verbatim}                             ^^A No blank at line start
% leads to the result that the set {\envfont Cor} is formatted in the
% style \texttt{break}, while the set {\envfont Exa} and all the
% following ones are formatted in the style \texttt{plain}, unless
% another |\theoremstyle| follows.  Since the definitions installed by
% |\newtheorem| are global, one also can limit |\theoremstyle| locally
% by grouping braces.
%
% \DescribeMacro\theorembodyfont
% The choice of the font for the theorem body is completely
% independent of the chosen |\theoremstyle|; this has proven to
% be very advantageous.  For example,
% \begin{verbatim}
%   {\theorembodyfont{\upshape}          \newtheorem{Rem}{Remark}}
%\end{verbatim}
% defines a theorem set {\envfont Rem}, which will be set in
% |\upshape| in the current layout (which in our example is
% \texttt{plain}).  As with |\theoremstyle|, the |\theorembodyfont|
% chosen is that current at the time of |\newtheorem|.  If
% |\theorembodyfont| is not specified or one defines
% |\theorembodyfont{}|, then the font used will be that defined by the
% |\theoremstyle|.
%
% \DescribeMacro\theoremheaderfont It is also possible to customize
% the font used for the theorem headers.  This is, however, a global
% declaration, and therefore there should be at most one
% |\theoremheaderfont| declaration in the preamble.\footnote{If it is
% actually necessary to have different header fonts, one has to define
% new theorem styles (substituting the desired font) or specify the
% information directly in the \texttt{\bslash newtheorem} declaration
% (the unclean variant).}
%
% \DescribeMacro\theorempreskipamount
% \DescribeMacro\theorempostskipamount Two additional parameters
% affect the vertical space around the theorem environments:
% |\theorempreskipamount| and |\theorempostskipamount| define,
% respectively, the spacing before and after such an environment.
% These parameters apply for all theorem sets and can be manipulated
% with the ordinary length macros.  They are rubber lengths
% (`\textsf{skips}'), and therefore can contain \texttt{plus} and
% \texttt{minus} parts.
%
% Since the definition of theorem sets should---most sensibly---be
% placed in the preamble, we only allow installation there.
% It is therefore possible to release the memory used here after
% |\begin{document}|, in order to make room for other
% applications.
%
%
% \subsection{Existing theorem styles}
%
% The following theorem styles exist to date:
% \begin{deflist}{marginbreak}
%    \item[plain]
%       This theorem style emulates the original \LaTeX{} definition,
%       except that additionally the parameters
%       |\theorem...skipamount| are used.
%    \item[break]
%       In this style, the theorem header is followed by a line
%       break.
%    \item[marginbreak]
%       The theorem number is set in the margin, and there is a line
%       break as in \texttt{break}.
%    \item[changebreak]
%       Like \texttt{break}, but with header number and text
%       interchanged.
%    \item[change]
%       Header number and text are interchanged, without a line break.
%    \item[margin]
%       The number is set in the left margin, without a line break.
% \end{deflist}
% All styles (except \texttt{plain}) select |\slshape| as the default
% |\theorembodyfont|.
%
%
% \subsection{Examples}
%
% Given the above theorem sets {\envfont Cor}, {\envfont Exa}
% and {\envfont Rem},
% suppose that the preamble also contains the declarations:
% \begin{verbatim}
%    \theoremstyle{marginbreak}   \newtheorem{Lem}[Cor]{Lemma}
%    \theoremstyle{change}
%    \theorembodyfont{\itshape}        \newtheorem{Def}[Cor]{Definition}
%
%    \theoremheaderfont{\scshape}
%\end{verbatim}
% Then the following are some typical examples of the typeset output
% resulting from their use.
%
% \begin{Cor}
%    This is a sentence typeset in the theorem environment {\envfont
%    Cor}.
% \end{Cor}
%
% \begin{Exa}
%    This is a sentence typeset in the theorem environment {\envfont
%    Exa}.
% \end{Exa}
%
% \begin{Rem}
%    This is a sentence typeset in the theorem environment {\envfont
%    Rem}.
% \end{Rem}
%
% \begin{Lem}[Ben User]
%    This is a sentence typeset in the theorem environment {\envfont
%    Lem}.
% \end{Lem}
%
% \begin{Def}[Very impressive Definition]
%    This is a sentence typeset in the theorem environment {\envfont
%    Def}.
% \end{Def}
%
% The last two examples show the effect of the optional argument to
% a theorem environment (it is the text typeset in parentheses).
%
%
% \section{Special Considerations}
% \label{sec:newfonts}
%
% Theoremheader and body are implemented as a unit. This means that
% the |\theoremheaderfont|  will inherit characteristics of the
% |\theorembodyfont| in \LaTeXe. Thus, if for example
% |\theorembodyfont| is |\itshape| and |\theoremheaderfont|
% is |\bfseries| the font selected for the header will have the
% characteristics `bold extended italic'. If this is not desired one
% should set the |\theoremheaderfont| to something like
% \begin{verbatim}
%      \theoremheaderfont{\normalfont\bfseries}
%\end{verbatim}
% i.e.\ supplying all necessary font information explicitly.
%
%
%
% \section{Acknowledgements}
%
% The publication of this set of macros was only possible with the
% help of Christina Busse (translating the manuscript into English),
% Joachim Pense (playing the r\^ole of typist), Chris Rowley
% (looking everything over) and many others providing useful
% suggestions.
%
% \MaybeStop{%
%    \begin{thebibliography}{9}
%       \bibitem{bk:GMS94} \textsc{M.~Goossens}, \textsc{F.~Mittelbach}
%         and \textsc{A.~Samarin}.
%         \newblock The \LaTeX{} Companion.
%         \newblock
%         Addison-Wesley, Reading, Massachusetts, 1994.
%       \bibitem{latex:source} \textsc{Lamport, Leslie.} \newblock
%                               \texttt{latex.tex}, version 2.09, date
%                               Feb. 1990.
%    \end{thebibliography}
%    \PrintIndex
%    \PrintChanges}
%
%
% \section{The documentation driver file}
%
% The next bit of code contains the documentation driver file for
% \TeX{}, i.e., the file that will produce the documentation you are
% currently reading. It will be extracted from this file by the
% \texttt{docstrip} program. Since it is the first code in the file
% one can alternatively process this file directly with \LaTeXe{} to
% obtain the documentation.
%
%    \begin{macrocode}
%<*driver>
\documentclass{ltxdoc}

\usepackage{theorem}

%<+driver>% The next few lines define theorem sets which are used
%<+driver>% in the example section of the documentation.

\theoremstyle{break}       \newtheorem{Cor}{Corollary}
\theoremstyle{plain}       \newtheorem{Exa}{Example}[section]
{\theorembodyfont{\upshape}\newtheorem{Rem}{Remark}}
\theoremstyle{marginbreak} \newtheorem{Lem}[Cor]{Lemma}
\theoremstyle{change}
\theorembodyfont{\itshape} \newtheorem{Def}[Cor]{Definition}

\theoremheaderfont{\scshape}

\RecordChanges

\begin{document}
   \DocInput{theorem.dtx}
\end{document}
%</driver>
%    \end{macrocode}
%
% \section{Definition of the Macros}
%
% If the file has been loaded before, we abort immediately. If not the
% package announces itself (this is actually done at the very top of
% the file---the way it is done isn't good style so don't copy it).
% \changes{v2.0e}{89/07/19}{Spaces removed from `typeout'.}
% \changes{v2.2b}{95/11/19}{Announce moved to top of file}
%    \begin{macrocode}
%<*package>
%\@ifundefined{theorem@style}{}{\endinput}
%\def\FMithmInfo{1995/11/19 v2.2b Theorem extension package (FMi)}
%\ProvidesPackage{theorem}[\FMithmInfo]
%    \end{macrocode}
%
%
% \subsection{Definition of theorem styles and fonts}
%
% All the definitions in this file are done globally to allow
% inputting this file inside a group.
%
% \begin{macro}{\theoremstyle}
%    Before a theorem style can be installed, the chosen style must be
%    known. For that reason, we must test to see that
%    |\th@|\meta{style} is known or, more precisely, that it is
%    different from |\relax|. If the style is not known then
%    |\th@plain| is used.
%    \begin{macrocode}
\gdef\theoremstyle#1{%
   \@ifundefined{th@#1}{\@warning
          {Unknown theoremstyle `#1'. Using `plain'}%
          \theorem@style{plain}}%
%    \end{macrocode}
%    We save the theorem style to be used in the token register
%    |\theorem@style|.
%    \begin{macrocode}
      {\theorem@style{#1}}%
%    \end{macrocode}
%    Now we ``evaluate'' the theorem style: this means, we call the
%    macro |\th@|\meta{style} which will activate the relevant
%    definitions which are contained in a separate file.  This is done
%    in a group to suppress changes to the current font.  This could
%    otherwise pose problems together with the new font selection
%    scheme\footnote{When I printed the original article using the new
%    font selection scheme I ended with a document with slanted
%    typefaces (text headings and all) simply because one of the
%    theorem styles used \texttt{\bslash sl} at toplevel.} if the
%    |\th@|\meta{style} is evaluated a second time.
% \changes{v2.1a}{90/02/15}{Group added around `th@..}
% \changes{v2.1a}{90/02/15}{Reseting of `..@begintheorem removed.}
%    \begin{macrocode}
      \begingroup
        \csname th@\the\theorem@style \endcsname
      \endgroup}
%    \end{macrocode}
% \end{macro}
%
%
% \begin{macro}{\@begintheorem}
% \changes{v2.1a}{90/02/15}{Relaxed at toplevel.}
% \begin{macro}{\@opargbegintheorem}
% \changes{v2.1a}{90/02/15}{Relaxed at toplevel.}
%    We reset |\@begintheorem| and |\@opargbegintheorem| to
%    |\relax| since these commands are no longer necessary at
%    toplevel. This will save a few tokens.
%    \begin{macrocode}
\global\let\@begintheorem\relax
\global\let\@opargbegintheorem\relax
%    \end{macrocode}
% \end{macro}
% \end{macro}
%
% \begin{macro}{\theorem@style}
%    Obviously the \textsf{token} register used above has to be
%    allocated. To assure the utmost compatibility with the
%    original \LaTeX{} definition, we set the default theorem style to
%    \texttt{plain}, which implements the usual \LaTeX{} convention.
% \changes{v2.0f}{89/07/20}{Unnecessary `global' removed.}
%    \begin{macrocode}
\newtoks\theorem@style
\global\theorem@style{plain}
%    \end{macrocode}
% \end{macro}
%
% \begin{macro}{\theorembodyfont}
% \begin{macro}{\theorem@bodyfont}
%    If the |\theorembodyfont| is set by the user then it should not
%    interact with the default font set in the theorem style. When the
%    new font selection is in force this may happen if, for example,
%    the default is |\itshape| and the new |\theorembodyfont| is
%    |\sffamily|. So we add a |\reset@font| command in front of the
%    user definition.
% \changes{v2.0f}{89/07/20}{Unnecessary `global' removed.}
% \changes{v2.0g}{89/09/19}{`theoremfont to `theorembodyfont.}
% \changes{v2.1c}{92/06/22}{`reset@font added and logic changed.}
% \changes{v2.1d}{92/10/28}{Reset to empty if arg is empty}
%    \begin{macrocode}
\gdef\theorembodyfont#1{%
%    \end{macrocode}
%    We check if the argument supplied is empty and if so put nothing
%    into the |\theorem@bodyfont| token register to allow for
%    |\theorembodyfont{}| as a mean of using the default of the
%    current |\theoremstyle|.
%    \begin{macrocode}
   \def\@tempa{#1}%
   \ifx\@tempa\@empty
    \theorem@bodyfont{}%
   \else
    \theorem@bodyfont{\reset@font#1}%
   \fi
   }
\newtoks\theorem@bodyfont
\global\theorem@bodyfont{}
%    \end{macrocode}
% \end{macro}
% \end{macro}
%
%
% \begin{macro}{\theoremheaderfont}
% \changes{v2.0g}{89/09/19}{Macro added.}
%    The font for the theorem headers is handled differently because
%    this definition applies to all theorem styles.
%    \begin{macrocode}
\gdef\theoremheaderfont#1{\gdef\theorem@headerfont{#1}%
%    \end{macrocode}
%    After using the macro once it is redefined to produce an error
%    message.
%    \begin{macrocode}
       \gdef\theoremheaderfont##1{%
        \typeout{\string\theoremheaderfont\space should be used
                 only once.}}}
%    \end{macrocode}
% \end{macro}
%
% \begin{macro}{\theorem@headerfont}
% \changes{v2.0g}{89/09/19}{Macro added.}
%    To set the |\theorem@headerfont| default we first test if
%    the new fontselection scheme is in force.
% \changes{v2.1a}{90/02/15}{Support for new font selection.}
%    \begin{macrocode}
\ifx\upshape\undefined
%    \end{macrocode}
%    If not we define it to expand into |\bfseries|. We don't use
%    |\let| just in case a following style option redefines this
%    macro.
%    \begin{macrocode}
\gdef\theorem@headerfont{\bfseries}
%    \end{macrocode}
%    Otherwise we reset the current shape before calling |\bfseries|.
%    \begin{macrocode}
\else \gdef\theorem@headerfont{\normalfont\bfseries}\fi
%    \end{macrocode}
% \end{macro}
%
% \begin{macro}{\th@plain}
% \begin{macro}{\th@break}
% \begin{macro}{\th@marginbreak}
% \begin{macro}{\th@changebreak}
% \begin{macro}{\th@change}
% \begin{macro}{\th@margin}
%    The different styles are defined in macros such as
%    |\th@plain|.  Since memory space is precious in
%    ``non-Big-versions'', we have to avoid offering too many unused
%    definitions.  Therefore we define these styles in separate files
%    that can be loaded on demand.  Thus the commands themselves only
%    load these files. We use |\@input@|, a \LaTeXe{} internal command
%    that ensures that the file will be listed with |\listfiles|.
% \changes{v2.2a}{94/02/02}{Use `@input@ to load theorem layout files}
%    \begin{macrocode}
\gdef\th@plain{\@input@{thp.sty}}
\gdef\th@break{\@input@{thb.sty}}
\gdef\th@marginbreak{\@input@{thmb.sty}}
\gdef\th@changebreak{\@input@{thcb.sty}}
\gdef\th@change{\@input@{thc.sty}}
\gdef\th@margin{\@input@{thm.sty}}
%    \end{macrocode}
%    This list will be expanded when new styles become
%    available.  For testing,  just append new theorem substyles
%    as document options.
% \end{macro}
% \end{macro}
% \end{macro}
% \end{macro}
% \end{macro}
% \end{macro}
%
%
%
% \subsection{Definition of a new theorem set}
%
% As already pointed out, a new theorem environment can be defined in
% three different ways:
% \begin{verbatim}
%    \newtheorem{Lem}{Lemma}
%    \newtheorem{Lem}{Lemma}[section]
%    \newtheorem{Lem}[Theorem]{Lemma}
%\end{verbatim}
% The function of the macro |\newtheorem| is to recognize these cases
% and then to branch into one of the three macros |\@ynthm|, |\@xnthm|
% or |\@othm|.  This mechanism is adopted unchanged from
% \cite{latex:source}; the essential point here is that, for example,
% in the second case, the arguments \texttt{Lem}, \texttt{Lemma} and
% \texttt{section} are passed over to the macro |\@xnthm|.
%
% We inspect this case first because the others present fewer problems,
% and thus are easily derived from this one.
%
% \begin{macro}{\@xnthm}
%    For our example arguments, the macro |\@xnthm| must fulfill
%    the following:
%    \begin{itemize}
%       \item
%          Define a new \LaTeX-counter `Lem'
%       \item
%          reset this counter within a |\section|
%       \item
%          define the macro |\theLem|
%       \item
%          define the environment macros |\Lem| and
%          |\endLem| using the current
%          |\theoremstyle| and |\theorem@bodyfont|.
%    \end{itemize}
%    Obviously, all this should happen only if the first argument
%    of |\@xnthm| (i.e.\ \texttt{Lem} in our example) is chosen so
%    as not to conflict with any previously defined commands or
%    environments.  This test is performed by the \LaTeX{} macro
%    |\@ifdefinable|.
%    \begin{macrocode}
\gdef\@xnthm#1#2[#3]{\expandafter\@ifdefinable\csname #1\endcsname
%    \end{macrocode}
%    Therefore, the first argument of |\@ifdefinable| is the
%    expansion (in the example, |\Lem|) of
%    |\csname||#1\endcsname|.  The second argument is
%    executed only if the test has been completed successfully.
%    \begin{macrocode}
   {%
%    \end{macrocode}
%    Now we define the new counter. The names of the \LaTeX{}
%    macros employed should speak for themselves:
% \changes{v2.2b}{1995/10/07}{Use \cs{@newctr}}
% \changes{v2.2c}{1995/11/23}{Use \cs{@newctr} properly.}
%    \begin{macrocode}
    \@definecounter{#1}\@newctr{#1}[#3]%
%    \end{macrocode}
%    Using |\@newctr| will give a proper error message if the counter
%    in |#3| is not defined.
%    In defining `|\theLem|' we must generate the desired macro
%    name by use of |\expandafter| and |\csname|.
%    \begin{macrocode}
    \expandafter\xdef\csname the#1\endcsname
%    \end{macrocode}
%    An |\xdef| is used in order to make the definition global,
%    and to ensure that it contains the replacement texts of
%    |\@thmcountersep| and |\@thmcounter|.\footnote{These
%    two macros can be defined by the document style.  Their default
%    values produce a `\texttt{.}' as separation and an arabic
%    representation of the number.} However, not everything should be
%    expanded. For example, it saves space to use |\thesection|
%    instead of its---at times---lengthy expansion.
%    \begin{macrocode}
      {\expandafter \noexpand \csname the#3\endcsname
       \@thmcountersep \@thmcounter{#1}}%
%    \end{macrocode}
%    Thus with the defaults of \LaTeX{}, |\theLem| would be
%    replaced by the command sequence |\thesection.\arabic{Lem}|.
%
%    We will now look at the definition of the macro which is executed
%    at the beginning of the actual environment (in our example this
%    macro is |\Lem|). It should be noted that we use an
%    ``|\expandafter| trick'' to expand only certain parts of the
%    replacement text at the time of the definition.
%    \begin{macrocode}
    \def\@tempa{\global\@namedef{#1}}%
    \expandafter \@tempa \expandafter{%
%    \end{macrocode}
%    First, the macro that contains the current definitions of
%    |\@begintheorem| and |\@opargtheorem| should be called
%    up.  The name of this macro---as is already known---has the form
%    |\th@|\meta{theorem\ style}; therefore, it must be called by
%    \begin{macrocode}
      \csname th@\the \theorem@style
%    \end{macrocode}
%    In addition the default theorem font should be changeable, i.e.\
%    we have to insert the contents of |\theorem@bodyfont|. For
%    that reason, we expand even further, beyond |\endcsname|,
%    and thus insert the contents of the \textsf{token} register
%    |\theorem@bodyfont| in the replacement text.
%    \begin{macrocode}
            \expandafter \endcsname \the \theorem@bodyfont
%    \end{macrocode}
%    Now it is time to call the macro |\@thm| which takes over
%    the further processing.  It has two arguments: the current
%    counter name (in our example, \texttt{Lem}), and the text of the
%    label (in our example, \texttt{Lemma}).
%    \begin{macrocode}
     \@thm{#1}{#2}}%
%    \end{macrocode}
%    With this, the `sub-definition' is complete.  The macro
%    |\@endtheorem| ends a theorem environment and is, so far, nothing
%    but an |\endtrivlist|.  (Hence it is defined globally, and not
%    within the theorem styles.\footnote{This has to be changed as
%    soon as theorem styles that change \texttt{\bslash @endtheorem}
%    exist.  In such a case, all existing styles must be changed as
%    well since they will have to reset the macro.}) Therefore, we can
%    set it equivalent to the macro that ends the theorem set (in our
%    example, |\endLem|).  However, if some day theorem styles exist
%    that do change |\@endtheorem|, we would have to use the
%    commented-out line instead.
%    \begin{macrocode}
    \global \expandafter \let \csname end#1\endcsname \@endtheorem
%   \global\@namedef{end#1}{\@endtheorem}%
%    \end{macrocode}
%    With these commands all the required definitions are employed,
%    unless the test |\@ifdefinable| has failed.  Therefore, we
%    end the second argument of this macro and with it the definition
%    of |\@xnthm|.
%    \begin{macrocode}
   }}
%    \end{macrocode}
% \end{macro}
%
% \begin{macro}{\@ynthm}
%    The definition of |\@ynthm| is completely analogous. In
%    this case the new counter that is defined is not reset within
%    another counter; thus the definition of |\the...|
%    is simplified:
%    \begin{macrocode}
\gdef\@ynthm#1#2{\expandafter\@ifdefinable\csname #1\endcsname
   {\@definecounter{#1}%
    \expandafter\xdef\csname the#1\endcsname{\@thmcounter{#1}}%
%    \end{macrocode}
%    The rest of the definition corresponds literally to that of
%    |\@xnthm|:
%    \begin{macrocode}
    \def\@tempa{\global\@namedef{#1}}\expandafter \@tempa
     \expandafter{\csname th@\the \theorem@style \expandafter
     \endcsname \the\theorem@bodyfont \@thm{#1}{#2}}%
    \global \expandafter \let \csname end#1\endcsname \@endtheorem}}
%    \end{macrocode}
% \end{macro}
%
% \begin{macro}{\@othm}
%    The definition of |\@othm| does not contain anything new.
%    \begin{macrocode}
\gdef\@othm#1[#2]#3{%
%    \end{macrocode}
% \changes{v2.2b}{95/10/07}{Build in check for counter defined}
%    We do not define a new counter but instead use one that has
%    already been defined.  Thus the only definition we need is that
%    of this pseudo-counter (i.e.\ |\the|\meta{env.\ name}).
%    First we check if |#2| corresponds to a known counter name.
%    \begin{macrocode}
  \expandafter\ifx\csname c@#2\endcsname\relax
   \@nocounterr{#2}%
  \else
   \expandafter\@ifdefinable\csname #1\endcsname
   {\expandafter \xdef \csname the#1\endcsname
     {\expandafter \noexpand \csname the#2\endcsname}%
%    \end{macrocode}
%    All other parts of the definition can be adopted from
%    |\@xnthm|. We have to remember, though, that in this case
%    the name of the current counter and the theorem label have moved
%    to the second and third arguments.
%    \begin{macrocode}
    \def\@tempa{\global\@namedef{#1}}\expandafter \@tempa
     \expandafter{\csname th@\the \theorem@style \expandafter
     \endcsname \the\theorem@bodyfont \@thm{#2}{#3}}%
    \global \expandafter \let \csname end#1\endcsname \@endtheorem}%
  \fi}
%    \end{macrocode}
% \end{macro}
%
% \subsection{Macros that are employed in a {\envfont theorem}
%    environment}
%
% \begin{macro}{\@thm}
%    The macro |\@thm| has to increase the current counter.  Then,
%    depending on whether the environment has (or does not have) an
%    optional argument, it has to branch into either |\@begintheorem|
%    or |\@opargtheorem|.
%    \begin{macrocode}
\gdef\@thm#1#2{\refstepcounter{#1}%
%    \end{macrocode}
%    Now we start a {\envfont trivlist} environment, and give
%    |\@topsep| and |\@topsepadd| the values of the \textsf{skip}
%    registers |\theorempreskipamount| and |\theorempostskipamount|.
%    The value in |\@topsep| is the vertical space that is inserted by
%    the first (and only) |\item| in our |\trivlist| whilst
%    |\@topsepadd| is inserted by |\@endparenv| at the end of that
%    {\envfont trivlist} environment.  By using these registers, we
%    obtain the desired space around a {\envfont theorem} environment.
%    \begin{macrocode}
   \trivlist
   \@topsep \theorempreskipamount               % used by first \item
   \@topsepadd \theorempostskipamount           % used by \@endparenv
%    \end{macrocode}
%    Now we have to test whether an optional argument has been given.
% \changes{v2.0d}{89/07/18}{Forgotten percent added.}
%    \begin{macrocode}
   \@ifnextchar [%
%    \end{macrocode}
%    If there is an optional argument, we will call |\@ythm|, and
%    move the arguments read back into the input stream.
%    \begin{macrocode}
   {\@ythm{#1}{#2}}%
%    \end{macrocode}
%    If not, we call |\@begintheorem|. Its first argument is the
%    name of the theorem set (hence the second argument of
%    |\@thm|). Its second argument is the macro that produces the
%    current number.
%    \begin{macrocode}
   {\@begintheorem{#2}{\csname the#1\endcsname}\ignorespaces}}
%    \end{macrocode}
% \end{macro}
%
% \begin{macro}{\@xthm}
% \begin{macro}{\@ythm}
%    Both these macros were originally called by |\@thm|.  We do
%    not need |\@xthm| anymore, hence we reset it to
%    |\relax|. The definition of |\@ythm| has not changed at
%    all from its definition in \LaTeX. In order to make the macros
%    easier to understand, we will
%    nevertheless present it (commented out).
%    \begin{macrocode}
\global\let\@xthm\relax
% \def\@ythm#1#2[#3]{\@opargbegintheorem{#2}{\csname
%           the#1\endcsname}{#3}\ignorespaces}
%</package>
%    \end{macrocode}
%    The primitive |\ignorespaces| in |\@ythm| and
%    |\@thm| is needed to remove the spaces between the
%    |\begin{...}| and the actual text.
% \end{macro}
% \end{macro}
%
%
% \subsection{Definition of the theorem substyles}
%
% As already pointed out, the theorem substyles, defined below, are
% only loaded when necessary.  Note that all these substyles, except
% \texttt{plain}, have |\slshape| as the default body font.
%
% \subsubsection{The \texttt{plain} style}
%
% \label{dok:th@plain} As the following macros use |@|, we have
% to locally set the |\catcode| of this symbol to ``letter''.
% This happens within a group, so that we do not have to worry about
% which |\catcode| that symbol had before.
%    \begin{macrocode}
%<*thp>
\begingroup \makeatletter
%    \end{macrocode}
% Since we are now within a group, we must make all definitions
% globally.  First we make sure that \texttt{theorem.sty} is loaded.
% This will allow us to use this file as a document style option
% without having to call \texttt{theorem} itself as an option.  At the
% same time, we assure that at least version 2 is loaded, since
% |\theorem@style| was not defined in earlier versions.
%    \begin{macrocode}
\@ifundefined{theorem@style}{\input{theorem.sty}}{}
\ProvidesFile{thp.sty}
             [\FMithmInfo]
%    \end{macrocode}
%
%
% \begin{macro}{\th@plain}
% \begin{macro}{\@begintheorem}
% \begin{macro}{\@opargbegintheorem}
%    |\theoremstyle{plain}| corresponds to the original
%    definition, except that the distances to the surrounding text are
%    determined by the parameters |\theorempreskipamount| and
%    |\theorempostskipamount|.
%    First we set the default body font.
%    \begin{macrocode}
\gdef\th@plain{\normalfont\itshape
%    \end{macrocode}
%    Then we define |\@begintheorem| and
%    |\@opargbegintheorem|. These two macros define how the
%    header of a theorem is typeset.  |\@opargbegintheorem| will
%    be called if a \textsf{theorem} environment with an optional
%    argument is encountered; otherwise, the header is constructed by
%    calling |\@begintheorem|.  If one of these macros is
%    executed, we are within a \textsf{trivlist} environment started by
%    |\@thm|.  So the theorem header is produced with an
%    |\item| command.
%
%    Instead of specifying the header font directly, all standard
%    theorem styles use the |\theorem@headerfont| macro to allow
%    customization.  The extra space (|\labelsep|) is necessary
%    because of problems in the \textsf{trivlist} environment.
%    \begin{macrocode}
  \def\@begintheorem##1##2{%
        \item[\hskip\labelsep \theorem@headerfont ##1\ ##2]}%
%    \end{macrocode}
%    The definition of |\@opargbegintheorem| is completely
%    analogous. The only difference is the fact that there exists a
%    third argument (which is the optional parameter of the
%    environment and contains additional information about the
%    theorem).  Customarily we enclose it in parentheses.
%    \begin{macrocode}
\def\@opargbegintheorem##1##2##3{%
   \item[\hskip\labelsep \theorem@headerfont ##1\ ##2\ (##3)]}}
%    \end{macrocode}
% \end{macro}
% \end{macro}
% \end{macro}
%
% \noindent
% We conclude with an |\endgroup| to restore the
% |\catcode| of |@|.
%    \begin{macrocode}
\endgroup
%</thp>
%    \end{macrocode}
%
%
% \subsubsection{The \texttt{break} style}
%
% This style option is stored in the file |thb.sty|.
% For the next two lines see the documentation for |\th@plain| on
% page~\pageref{dok:th@plain}.
%    \begin{macrocode}
%<*thb>
\begingroup \makeatletter
\@ifundefined{theorem@style}{\input{theorem.sty}}{}
\ProvidesFile{thb.sty}
             [\FMithmInfo]
%    \end{macrocode}
%
% \begin{macro}{\th@break}
%    |\theoremstyle{break}| produces a line break after the name
%    of the theorem.  The font is |\slshape|.  Hence, we define
%    |\th@break| as follows:
%    \begin{macrocode}
\gdef\th@break{\normalfont\slshape
  \def\@begintheorem##1##2{\item[%
%    \end{macrocode}
%    We run into the following problem: it is not possible to create
%    the header with |\item[|\meta{title}|]| and then start
%    a new line by, for example, |\mbox{}\\|. Such a definition
%    will fail whenever a list environment follows immediately.  With
%    the above construction, the |\mbox{}| causes the switch
%    |@inlabel| (cf.\ definition of |\list| and
%    |\trivlist| in \cite{latex:source}) to be set to \texttt{false}
%    and so the following list will insert additional vertical
%    space (|\topskip|). This is quite annoying. Therefore, we
%    create the line break within the |\item|. In order to ensure
%    that the text will begin at the proper position in the following
%    line, we simply pretend that the label does not take any
%    room.\footnote{This will lead to problems whenever very high
%    symbols occurring in the line tower into the
%    heading. So, something else has to be done here sometime.}
%    \begin{macrocode}
       \rlap{\vbox{\hbox{\hskip \labelsep\theorem@headerfont ##1\ ##2}%
                   \hbox{\strut}}}]}%
%    \end{macrocode}
%    Again, the definition of |\@opargbegintheorem| is
%    completely analogous.
%    \begin{macrocode}
\def\@opargbegintheorem##1##2##3{%
  \item[\rlap{\vbox{\hbox{\hskip \labelsep \theorem@headerfont
                     ##1\ ##2\ (##3)}%
                    \hbox{\strut}}}]}}
\endgroup
%</thb>
%    \end{macrocode}
% \end{macro}
%
%
% \subsubsection{The \texttt{changebreak} style}
%
%    \begin{macrocode}
% This style option is stored in the file |thcb.sty|.
%    \begin{macrocode}
%<*thcb>
\begingroup \makeatletter
\@ifundefined{theorem@style}{\input{theorem.sty}}{}
\ProvidesFile{thcb.sty}
             [\FMithmInfo]
%    \end{macrocode}
%
% \begin{macro}{\th@changebreak}
%    The \texttt{change-break} theorem style is like \texttt{break}
%    but with interchange of theorem name and theorem number.  Thus we
%    define |\th@changebreak| as follows:
%    \begin{macrocode}
\gdef\th@changebreak{\normalfont\slshape
  \def\@begintheorem##1##2{\item
       [\rlap{\vbox{\hbox{\hskip\labelsep \theorem@headerfont ##2\ ##1}%
                    \hbox{\strut}}}]}%
\def\@opargbegintheorem##1##2##3{%
  \item[\rlap{\vbox{\hbox{\hskip\labelsep \theorem@headerfont
                     ##2\ ##1\ (##3)}%
                    \hbox{\strut}}}]}}
\endgroup
%</thcb>
%    \end{macrocode}
% \end{macro}
%
%
% \subsubsection{The \texttt{change} style}
%
% This style option is stored in the file |thc.sty|.
%    \begin{macrocode}
%<*thc>
\begingroup \makeatletter
\@ifundefined{theorem@style}{\input{theorem.sty}}{}
\ProvidesFile{thc.sty}
             [\FMithmInfo]
%    \end{macrocode}
%
% \begin{macro}{\th@change}
%    The \texttt{change} theorem style corresponds to the
%    \texttt{change break} style without a linebreak after the header.
%    To say it in another way, it's the same as the \texttt{plain}
%    style but with number and name interchanged and |\slshape| as the
%    default font.
%    \begin{macrocode}
\gdef\th@change{\normalfont\slshape
  \def\@begintheorem##1##2{\item
       [\hskip\labelsep \theorem@headerfont ##2\ ##1]}%
\def\@opargbegintheorem##1##2##3{%
  \item[\hskip\labelsep \theorem@headerfont ##2\ ##1\ (##3)]}}
\endgroup
%</thc>
%    \end{macrocode}
% \end{macro}
%
%
% \subsubsection{The \texttt{marginbreak} style}
%
% This style option is the one used most often at Mainz.  It is saved
% in the file |thmb.sty|.
%    \begin{macrocode}
%<*thmb>
\begingroup \makeatletter
\@ifundefined{theorem@style}{\input{theorem.sty}}{}
\ProvidesFile{thmb.sty}
             [\FMithmInfo]
%    \end{macrocode}
%
% \begin{macro}{\th@marginbreak}
%    The \texttt{margin break} style is nearly the same as the
%    \texttt{change break} style. The only difference is the placement
%    of the theorem number.  We use |\llap| to place it in the left
%    margin.
%
%    In this style |\labelsep| denotes the separation between the
%    number and the text.
%    \begin{macrocode}
\gdef\th@marginbreak{\normalfont\slshape
  \def\@begintheorem##1##2{\item
       [\rlap{\vbox{\theorem@headerfont
                    \hbox{\llap{##2}\hskip\labelsep ##1}%
                    \hbox{\strut}}}]}%
\def\@opargbegintheorem##1##2##3{%
  \item[\rlap{\vbox{\theorem@headerfont
                    \hbox{\llap{##2}\hskip\labelsep ##1\ (##3)}%
                    \hbox{\strut}}}]}}
\endgroup
%</thmb>
%    \end{macrocode}
% \changes{v2.1c}{90/02/17}{`endgroup outside macrocode env.}
% \end{macro}
%
%
%
% \subsubsection{The \texttt{margin} style}
%
% This style option is stored in the file |thm.sty|.
%    \begin{macrocode}
%<*thm>
\begingroup \makeatletter
\@ifundefined{theorem@style}{\input{theorem.sty}}{}
\ProvidesFile{thm.sty}
             [\FMithmInfo]
%    \end{macrocode}
%
% \begin{macro}{\th@margin}
%    Again this is only a variant of the theorem styles described
%    above without any new ideas.
%    \begin{macrocode}
\gdef\th@margin{\normalfont\slshape
  \def\@begintheorem##1##2{\item
       [\theorem@headerfont \llap{##2}\hskip\labelsep ##1]}%
\def\@opargbegintheorem##1##2##3{%
  \item[\theorem@headerfont \llap{##2}\hskip\labelsep ##1\ (##3)]}}
\endgroup
%</thm>
%    \end{macrocode}
% \end{macro}
%
%
%
%
% \subsection{Final Definitions}
%
% \begin{macro}{\theorempreskipamount}
% \begin{macro}{\theorempostskipamount}
%    The \textsf{skip} parameters that regulate the vertical empty space
%    before and after the {\envfont theorem} environment have
%    to be allocated as well.
% \changes{v2.0f}{89/07/20}{Unnecessary `global' removed.}
%    \begin{macrocode}
%<*package>
\newskip\theorempreskipamount
\newskip\theorempostskipamount
%    \end{macrocode}
%    Since we have used the same values for all theorem sets, we now can
%    assign them.
%    \begin{macrocode}
\global\setlength\theorempreskipamount{12pt plus 5pt minus 3pt}
\global\setlength\theorempostskipamount{8pt plus 3pt minus 1.5pt}
%    \end{macrocode}
% \end{macro}
% \end{macro}
%
%
% \begin{macro}{\@endtheorem}
%    The same holds for the macro |\@endtheorem|, which ends a
%    {\envfont theorem} environment. Since it is the same for all
%    theorem sets, it is removed from the macros |\th@|\meta{style}.
%    It simply ends the {\envfont trivlist} environment, which was
%    begun in |\@thm|.
%    \begin{macrocode}
\global\let\@endtheorem=\endtrivlist
%    \end{macrocode}
% \end{macro}
%
%
%
% \begin{macro}{\@preamblecmds}
%    All macros defined above are to be used only in the preamble.
%    Therefore, we insert them in |\@preamblecmds| which will disable
%    them at begin document. This is done by the internal \LaTeXe{}
%    command |\@onlypreamble|.
% \changes{v2.0g}{89/09/19}{`theoremheaderfont and `theorembodyfont
%    added.}
% \changes{v2.2a}{94/02/03}{Use \LaTeXe{} interface}
%    \begin{macrocode}
\@onlypreamble\@xnthm
\@onlypreamble\@ynthm
\@onlypreamble\@othm
\@onlypreamble\newtheorem
\@onlypreamble\theoremstyle
\@onlypreamble\theorembodyfont
\@onlypreamble\theoremheaderfont
%    \end{macrocode}
% \end{macro}
%
% Finally we declare the \texttt{plain} theorem style to be the default.
% \changes{v2.1b}{90/06/04}{Fixed missing `item by declaring default
%    style.}
%    \begin{macrocode}
\theoremstyle{plain}
%</package>
%    \end{macrocode}
%
% \Finale

\endinput
}{}
\ProvidesFile{thc.sty}
             [\FMithmInfo]
%    \end{macrocode}
%
% \begin{macro}{\th@change}
%    The \texttt{change} theorem style corresponds to the
%    \texttt{change break} style without a linebreak after the header.
%    To say it in another way, it's the same as the \texttt{plain}
%    style but with number and name interchanged and |\slshape| as the
%    default font.
%    \begin{macrocode}
\gdef\th@change{\normalfont\slshape
  \def\@begintheorem##1##2{\item
       [\hskip\labelsep \theorem@headerfont ##2\ ##1]}%
\def\@opargbegintheorem##1##2##3{%
  \item[\hskip\labelsep \theorem@headerfont ##2\ ##1\ (##3)]}}
\endgroup
%</thc>
%    \end{macrocode}
% \end{macro}
%
%
% \subsubsection{The \texttt{marginbreak} style}
%
% This style option is the one used most often at Mainz.  It is saved
% in the file |thmb.sty|.
%    \begin{macrocode}
%<*thmb>
\begingroup \makeatletter
\@ifundefined{theorem@style}{% \iffalse meta-comment
%
% Copyright (C) 1993-2024
%
% The LaTeX Project and any individual authors listed elsewhere
% in this file.
%
% This file is part of the Standard LaTeX `Tools Bundle'.
% -------------------------------------------------------
%
% It may be distributed and/or modified under the
% conditions of the LaTeX Project Public License, either version 1.3c
% of this license or (at your option) any later version.
% The latest version of this license is in
%    https://www.latex-project.org/lppl.txt
% and version 1.3c or later is part of all distributions of LaTeX
% version 2005/12/01 or later.
%
% The list of all files belonging to the LaTeX `Tools Bundle' is
% given in the file `manifest.txt'.
%
% \fi
%
%%
%
%
\def\FMithmInfo{2023/07/05 v2.2c Theorem extension package (FMi)}
%
% \ProvidesFile{theorem.dtx}[\FMithmInfo]
%\iffalse   % this is a METACOMMENT !
%%
%% Package `theorem' to use with LaTeX2e
%% Copyright (C) 1989-1995 Frank Mittelbach, all rights reserved.
%<+package>\@ifundefined{theorem@style}{}{\endinput}
%<+package>\NeedsTeXFormat{LaTeX2e}[1995/06/01]
%<+package>\ProvidesPackage{theorem}
%<+package>         [\FMithmInfo]
%
% \fi
%
% \changes{v1.0}{87/05/07}{First draft.}
% \changes{v1.0a}{87/05/07}{`thistheoremstyle' added.}
% \changes{v1.1a}{87/07/29}{New concept: redefine the internal
%                          \protect\LaTeX{} commands `@xnthm' etc.}
% \changes{v1.1b}{87/07/31}{`theoremfont' used to specify font.}
% \changes{v1.2a}{88/01/08}{`@endtheorem' now globally defined.}
% \changes{v1.2b}{88/01/11}{`newtheorem' only allowed in the preamble.}
% \changes{v1.3a}{88/02/14}{`theoremfont' now a toks register.}
%
% \changes{v2.0a}{88/11/16}{New implementation and new documentation.}
% \changes{v2.0a}{88/11/16}{No layout changes but internal modifications
%                           old substyles won't work.}
% \changes{v2.0b}{89/01/30}{`global' used for all definitions.}
% \changes{v2.0c}{89/06/15}{Documentation translated to English.
%                            Thanks to LLL and Nina}
% \changes{v2.1a}{90/02/15}{Support for new font selection.}
% \changes{v2.2a}{94/02/03}{Cleanup for LaTeX2e}
%
% \DoNotIndex{\@endparenv,\@ifdefinable,\@ifnextchar,\@ifundefined}
% \DoNotIndex{\@namedef,\@nameuse,\@spaces,\@tempa,\@warning}
% \DoNotIndex{\ }
% \DoNotIndex{\begingroup,\bfseries}
% \DoNotIndex{\csname}
% \DoNotIndex{\def,\do,\docdate}
% \DoNotIndex{\endcsname,\endgroup,\endinput,\endtrivlist,\expandafter}
% \DoNotIndex{\filedate,\fileversion}
% \DoNotIndex{\gdef,\global}
% \DoNotIndex{\hbox,\hskip}
% \DoNotIndex{\ignorespaces,\input,\itshape,\item}
% \DoNotIndex{\labelsep,\let,\llap}
% \DoNotIndex{\makeatletter}
% \DoNotIndex{\noexpand}
% \DoNotIndex{\refstepcounter,\relax,\rlap}
% \DoNotIndex{\setlength,\slshape,\space,\string,\strut}
% \DoNotIndex{\the,\trivlist,\typeout}
% \DoNotIndex{\vbox}
% \DoNotIndex{\xdef}
%
% \def\deflabel#1{\ttfamily #1\hfill}
% \def\deflist#1{\begin{list}{}{\settowidth\labelwidth{\ttfamily #1}%
%                               \setlength\leftmargin\labelwidth
%                               \addtolength\leftmargin\labelsep
%                               \let\makelabel\deflabel}}
% \def\enddeflist{\end{list}}
% \def\envfont{\normalfont\ttfamily}
%
% \MakeShortVerb{\|}
%
% \newcommand{\TUB}{{\sl TUGboat\/}}
%
% \GetFileInfo{theorem.dtx}
% \title{An Extension of the
%        \LaTeX{} theorem environment\thanks{This file
%        has version number \fileversion, last
%        revised \filedate.}}
% \author{Frank Mittelbach}
%
% \MaintainedByLaTeXTeam{tools}
% \maketitle
%
% \begin{abstract}
%    The macros described in this paper yield an extension of the
%    \LaTeX{} theorem mechanism. It is designed to satisfy the
%    different requirements of various journals. Thus, the
%    layout of the ``theorems'' can be manipulated by determining a
%    ``style''. This article describes not only the use, but
%    also the definition, of the necessary macros.
% \end{abstract}
%
% \section*{Preface to version 2.2}
%
% For \LaTeXe{} this package did not need any fundamental changes. I
% only modified the messages generated so that theorem layout styles
% will show up with the |\listfiles| command and cleaned the section
% on the New Font Selection Scheme since this is now included in
% \LaTeX.
%
% \section*{Preface to version 2.1}
%
% This version is identical to 2.0g described in \TUB\ 10\#3 except
% for some internal defaults which are now set depending on the used
% font selection scheme.
%
% This was done to avoid unpleasant surprises if the new font selection
% scheme is in force. For further details see
% section~\ref{sec:newfonts} and \cite{bk:GMS94}.
%
% \section{Introduction}
%
% For our purposes here, ``theorems'' are labelled enunciations,
% often set off from the main text by extra space and a font change.
% Theorems, corollaries, conjectures, definitions, and remarks
% are all instances of ``theorems''.  The ``header'' of these structures
% is composed of a label (such as \textsc{Theorem} or \textsc{Remark})
% and a number which serializes an item in the sequence of items
% with the same label.
%
% Shortly after the introduction of \LaTeX{} at the Fachbereich
% Mathematik in Mainz, the desire to manipulate the layout of
% ``theorems'' arose.  In Mainz, the following two conventions came
% into general use:
% \begin{enumerate}
%   \item The number of the theorem is shown in the margin.
%   \item There is a line break at
%         the end of the theorem header.
% \end{enumerate}
%
% Additionally, some journals require different formats which depend on
% the ``sort of theorem'': e.g.\ often remarks and definitions are set
% in |\upshape|, while |\itshape| is employed for main theorems.
%
% Confronted with these requirements, a theorem environment was
% developed in Mainz which allows separate determination of the
% layout of the ``theorems sets'', comparable to |\pagestyle|.
%
%
% \section{The user interface}
% \subsection{Defining new theorem sets}
%
% \DescribeMacro\newtheorem
% As in the original \LaTeX{} version, the command |\newtheorem|
% defines a new ``theorem set'' or ``theorem-like structure''.
% Two required arguments name the new environment and give the text
% to be typeset with each instance of the new ``set'', while
% an optional argument determines how
% the ``set'' is enumerated:
% \begin{description}
%    \item[\ttfamily \bslash newtheorem\{foo\}\{bar\}]
%       The theorem set {\envfont foo} (whose name is \texttt{bar})
%       uses its own counter.
%    \item[\ttfamily \bslash newtheorem\{foo2\}{[foo]}\{bar2\}]
%       The theorem set {\envfont foo2} (printed name \texttt{bar2})
%       uses the same counter as the theorem set \texttt{foo}.
%    \item[\ttfamily \bslash newtheorem\{foo3\}\{bar3\}{[section]}]
%       The theorem set {\envfont foo3} (printed name \texttt{bar3}) is
%       enumerated within the counter \texttt{section}, i.e.\ with every
%       new |\section| the enumeration begins again with 1, and
%       the enumeration is composed from the section-number and the
%       theorem counter itself.
% \end{description}
%
% \DescribeMacro\theoremstyle
% Additionally, the command |\theoremstyle| can define the layout
% of various, or all, theorem sets.  It should be noted that any
% theorem set defined by |\newtheorem| is typeset in the
% |\theoremstyle| that is current at the time of the definition.
% Thus, the following
% \begin{verbatim}
%    \theoremstyle{break}        \newtheorem{Cor}{Corollary}
%    \theoremstyle{plain}        \newtheorem{Exa}{Example}[section]
%\end{verbatim}                             ^^A No blank at line start
% leads to the result that the set {\envfont Cor} is formatted in the
% style \texttt{break}, while the set {\envfont Exa} and all the
% following ones are formatted in the style \texttt{plain}, unless
% another |\theoremstyle| follows.  Since the definitions installed by
% |\newtheorem| are global, one also can limit |\theoremstyle| locally
% by grouping braces.
%
% \DescribeMacro\theorembodyfont
% The choice of the font for the theorem body is completely
% independent of the chosen |\theoremstyle|; this has proven to
% be very advantageous.  For example,
% \begin{verbatim}
%   {\theorembodyfont{\upshape}          \newtheorem{Rem}{Remark}}
%\end{verbatim}
% defines a theorem set {\envfont Rem}, which will be set in
% |\upshape| in the current layout (which in our example is
% \texttt{plain}).  As with |\theoremstyle|, the |\theorembodyfont|
% chosen is that current at the time of |\newtheorem|.  If
% |\theorembodyfont| is not specified or one defines
% |\theorembodyfont{}|, then the font used will be that defined by the
% |\theoremstyle|.
%
% \DescribeMacro\theoremheaderfont It is also possible to customize
% the font used for the theorem headers.  This is, however, a global
% declaration, and therefore there should be at most one
% |\theoremheaderfont| declaration in the preamble.\footnote{If it is
% actually necessary to have different header fonts, one has to define
% new theorem styles (substituting the desired font) or specify the
% information directly in the \texttt{\bslash newtheorem} declaration
% (the unclean variant).}
%
% \DescribeMacro\theorempreskipamount
% \DescribeMacro\theorempostskipamount Two additional parameters
% affect the vertical space around the theorem environments:
% |\theorempreskipamount| and |\theorempostskipamount| define,
% respectively, the spacing before and after such an environment.
% These parameters apply for all theorem sets and can be manipulated
% with the ordinary length macros.  They are rubber lengths
% (`\textsf{skips}'), and therefore can contain \texttt{plus} and
% \texttt{minus} parts.
%
% Since the definition of theorem sets should---most sensibly---be
% placed in the preamble, we only allow installation there.
% It is therefore possible to release the memory used here after
% |\begin{document}|, in order to make room for other
% applications.
%
%
% \subsection{Existing theorem styles}
%
% The following theorem styles exist to date:
% \begin{deflist}{marginbreak}
%    \item[plain]
%       This theorem style emulates the original \LaTeX{} definition,
%       except that additionally the parameters
%       |\theorem...skipamount| are used.
%    \item[break]
%       In this style, the theorem header is followed by a line
%       break.
%    \item[marginbreak]
%       The theorem number is set in the margin, and there is a line
%       break as in \texttt{break}.
%    \item[changebreak]
%       Like \texttt{break}, but with header number and text
%       interchanged.
%    \item[change]
%       Header number and text are interchanged, without a line break.
%    \item[margin]
%       The number is set in the left margin, without a line break.
% \end{deflist}
% All styles (except \texttt{plain}) select |\slshape| as the default
% |\theorembodyfont|.
%
%
% \subsection{Examples}
%
% Given the above theorem sets {\envfont Cor}, {\envfont Exa}
% and {\envfont Rem},
% suppose that the preamble also contains the declarations:
% \begin{verbatim}
%    \theoremstyle{marginbreak}   \newtheorem{Lem}[Cor]{Lemma}
%    \theoremstyle{change}
%    \theorembodyfont{\itshape}        \newtheorem{Def}[Cor]{Definition}
%
%    \theoremheaderfont{\scshape}
%\end{verbatim}
% Then the following are some typical examples of the typeset output
% resulting from their use.
%
% \begin{Cor}
%    This is a sentence typeset in the theorem environment {\envfont
%    Cor}.
% \end{Cor}
%
% \begin{Exa}
%    This is a sentence typeset in the theorem environment {\envfont
%    Exa}.
% \end{Exa}
%
% \begin{Rem}
%    This is a sentence typeset in the theorem environment {\envfont
%    Rem}.
% \end{Rem}
%
% \begin{Lem}[Ben User]
%    This is a sentence typeset in the theorem environment {\envfont
%    Lem}.
% \end{Lem}
%
% \begin{Def}[Very impressive Definition]
%    This is a sentence typeset in the theorem environment {\envfont
%    Def}.
% \end{Def}
%
% The last two examples show the effect of the optional argument to
% a theorem environment (it is the text typeset in parentheses).
%
%
% \section{Special Considerations}
% \label{sec:newfonts}
%
% Theoremheader and body are implemented as a unit. This means that
% the |\theoremheaderfont|  will inherit characteristics of the
% |\theorembodyfont| in \LaTeXe. Thus, if for example
% |\theorembodyfont| is |\itshape| and |\theoremheaderfont|
% is |\bfseries| the font selected for the header will have the
% characteristics `bold extended italic'. If this is not desired one
% should set the |\theoremheaderfont| to something like
% \begin{verbatim}
%      \theoremheaderfont{\normalfont\bfseries}
%\end{verbatim}
% i.e.\ supplying all necessary font information explicitly.
%
%
%
% \section{Acknowledgements}
%
% The publication of this set of macros was only possible with the
% help of Christina Busse (translating the manuscript into English),
% Joachim Pense (playing the r\^ole of typist), Chris Rowley
% (looking everything over) and many others providing useful
% suggestions.
%
% \MaybeStop{%
%    \begin{thebibliography}{9}
%       \bibitem{bk:GMS94} \textsc{M.~Goossens}, \textsc{F.~Mittelbach}
%         and \textsc{A.~Samarin}.
%         \newblock The \LaTeX{} Companion.
%         \newblock
%         Addison-Wesley, Reading, Massachusetts, 1994.
%       \bibitem{latex:source} \textsc{Lamport, Leslie.} \newblock
%                               \texttt{latex.tex}, version 2.09, date
%                               Feb. 1990.
%    \end{thebibliography}
%    \PrintIndex
%    \PrintChanges}
%
%
% \section{The documentation driver file}
%
% The next bit of code contains the documentation driver file for
% \TeX{}, i.e., the file that will produce the documentation you are
% currently reading. It will be extracted from this file by the
% \texttt{docstrip} program. Since it is the first code in the file
% one can alternatively process this file directly with \LaTeXe{} to
% obtain the documentation.
%
%    \begin{macrocode}
%<*driver>
\documentclass{ltxdoc}

\usepackage{theorem}

%<+driver>% The next few lines define theorem sets which are used
%<+driver>% in the example section of the documentation.

\theoremstyle{break}       \newtheorem{Cor}{Corollary}
\theoremstyle{plain}       \newtheorem{Exa}{Example}[section]
{\theorembodyfont{\upshape}\newtheorem{Rem}{Remark}}
\theoremstyle{marginbreak} \newtheorem{Lem}[Cor]{Lemma}
\theoremstyle{change}
\theorembodyfont{\itshape} \newtheorem{Def}[Cor]{Definition}

\theoremheaderfont{\scshape}

\RecordChanges

\begin{document}
   \DocInput{theorem.dtx}
\end{document}
%</driver>
%    \end{macrocode}
%
% \section{Definition of the Macros}
%
% If the file has been loaded before, we abort immediately. If not the
% package announces itself (this is actually done at the very top of
% the file---the way it is done isn't good style so don't copy it).
% \changes{v2.0e}{89/07/19}{Spaces removed from `typeout'.}
% \changes{v2.2b}{95/11/19}{Announce moved to top of file}
%    \begin{macrocode}
%<*package>
%\@ifundefined{theorem@style}{}{\endinput}
%\def\FMithmInfo{1995/11/19 v2.2b Theorem extension package (FMi)}
%\ProvidesPackage{theorem}[\FMithmInfo]
%    \end{macrocode}
%
%
% \subsection{Definition of theorem styles and fonts}
%
% All the definitions in this file are done globally to allow
% inputting this file inside a group.
%
% \begin{macro}{\theoremstyle}
%    Before a theorem style can be installed, the chosen style must be
%    known. For that reason, we must test to see that
%    |\th@|\meta{style} is known or, more precisely, that it is
%    different from |\relax|. If the style is not known then
%    |\th@plain| is used.
%    \begin{macrocode}
\gdef\theoremstyle#1{%
   \@ifundefined{th@#1}{\@warning
          {Unknown theoremstyle `#1'. Using `plain'}%
          \theorem@style{plain}}%
%    \end{macrocode}
%    We save the theorem style to be used in the token register
%    |\theorem@style|.
%    \begin{macrocode}
      {\theorem@style{#1}}%
%    \end{macrocode}
%    Now we ``evaluate'' the theorem style: this means, we call the
%    macro |\th@|\meta{style} which will activate the relevant
%    definitions which are contained in a separate file.  This is done
%    in a group to suppress changes to the current font.  This could
%    otherwise pose problems together with the new font selection
%    scheme\footnote{When I printed the original article using the new
%    font selection scheme I ended with a document with slanted
%    typefaces (text headings and all) simply because one of the
%    theorem styles used \texttt{\bslash sl} at toplevel.} if the
%    |\th@|\meta{style} is evaluated a second time.
% \changes{v2.1a}{90/02/15}{Group added around `th@..}
% \changes{v2.1a}{90/02/15}{Reseting of `..@begintheorem removed.}
%    \begin{macrocode}
      \begingroup
        \csname th@\the\theorem@style \endcsname
      \endgroup}
%    \end{macrocode}
% \end{macro}
%
%
% \begin{macro}{\@begintheorem}
% \changes{v2.1a}{90/02/15}{Relaxed at toplevel.}
% \begin{macro}{\@opargbegintheorem}
% \changes{v2.1a}{90/02/15}{Relaxed at toplevel.}
%    We reset |\@begintheorem| and |\@opargbegintheorem| to
%    |\relax| since these commands are no longer necessary at
%    toplevel. This will save a few tokens.
%    \begin{macrocode}
\global\let\@begintheorem\relax
\global\let\@opargbegintheorem\relax
%    \end{macrocode}
% \end{macro}
% \end{macro}
%
% \begin{macro}{\theorem@style}
%    Obviously the \textsf{token} register used above has to be
%    allocated. To assure the utmost compatibility with the
%    original \LaTeX{} definition, we set the default theorem style to
%    \texttt{plain}, which implements the usual \LaTeX{} convention.
% \changes{v2.0f}{89/07/20}{Unnecessary `global' removed.}
%    \begin{macrocode}
\newtoks\theorem@style
\global\theorem@style{plain}
%    \end{macrocode}
% \end{macro}
%
% \begin{macro}{\theorembodyfont}
% \begin{macro}{\theorem@bodyfont}
%    If the |\theorembodyfont| is set by the user then it should not
%    interact with the default font set in the theorem style. When the
%    new font selection is in force this may happen if, for example,
%    the default is |\itshape| and the new |\theorembodyfont| is
%    |\sffamily|. So we add a |\reset@font| command in front of the
%    user definition.
% \changes{v2.0f}{89/07/20}{Unnecessary `global' removed.}
% \changes{v2.0g}{89/09/19}{`theoremfont to `theorembodyfont.}
% \changes{v2.1c}{92/06/22}{`reset@font added and logic changed.}
% \changes{v2.1d}{92/10/28}{Reset to empty if arg is empty}
%    \begin{macrocode}
\gdef\theorembodyfont#1{%
%    \end{macrocode}
%    We check if the argument supplied is empty and if so put nothing
%    into the |\theorem@bodyfont| token register to allow for
%    |\theorembodyfont{}| as a mean of using the default of the
%    current |\theoremstyle|.
%    \begin{macrocode}
   \def\@tempa{#1}%
   \ifx\@tempa\@empty
    \theorem@bodyfont{}%
   \else
    \theorem@bodyfont{\reset@font#1}%
   \fi
   }
\newtoks\theorem@bodyfont
\global\theorem@bodyfont{}
%    \end{macrocode}
% \end{macro}
% \end{macro}
%
%
% \begin{macro}{\theoremheaderfont}
% \changes{v2.0g}{89/09/19}{Macro added.}
%    The font for the theorem headers is handled differently because
%    this definition applies to all theorem styles.
%    \begin{macrocode}
\gdef\theoremheaderfont#1{\gdef\theorem@headerfont{#1}%
%    \end{macrocode}
%    After using the macro once it is redefined to produce an error
%    message.
%    \begin{macrocode}
       \gdef\theoremheaderfont##1{%
        \typeout{\string\theoremheaderfont\space should be used
                 only once.}}}
%    \end{macrocode}
% \end{macro}
%
% \begin{macro}{\theorem@headerfont}
% \changes{v2.0g}{89/09/19}{Macro added.}
%    To set the |\theorem@headerfont| default we first test if
%    the new fontselection scheme is in force.
% \changes{v2.1a}{90/02/15}{Support for new font selection.}
%    \begin{macrocode}
\ifx\upshape\undefined
%    \end{macrocode}
%    If not we define it to expand into |\bfseries|. We don't use
%    |\let| just in case a following style option redefines this
%    macro.
%    \begin{macrocode}
\gdef\theorem@headerfont{\bfseries}
%    \end{macrocode}
%    Otherwise we reset the current shape before calling |\bfseries|.
%    \begin{macrocode}
\else \gdef\theorem@headerfont{\normalfont\bfseries}\fi
%    \end{macrocode}
% \end{macro}
%
% \begin{macro}{\th@plain}
% \begin{macro}{\th@break}
% \begin{macro}{\th@marginbreak}
% \begin{macro}{\th@changebreak}
% \begin{macro}{\th@change}
% \begin{macro}{\th@margin}
%    The different styles are defined in macros such as
%    |\th@plain|.  Since memory space is precious in
%    ``non-Big-versions'', we have to avoid offering too many unused
%    definitions.  Therefore we define these styles in separate files
%    that can be loaded on demand.  Thus the commands themselves only
%    load these files. We use |\@input@|, a \LaTeXe{} internal command
%    that ensures that the file will be listed with |\listfiles|.
% \changes{v2.2a}{94/02/02}{Use `@input@ to load theorem layout files}
%    \begin{macrocode}
\gdef\th@plain{\@input@{thp.sty}}
\gdef\th@break{\@input@{thb.sty}}
\gdef\th@marginbreak{\@input@{thmb.sty}}
\gdef\th@changebreak{\@input@{thcb.sty}}
\gdef\th@change{\@input@{thc.sty}}
\gdef\th@margin{\@input@{thm.sty}}
%    \end{macrocode}
%    This list will be expanded when new styles become
%    available.  For testing,  just append new theorem substyles
%    as document options.
% \end{macro}
% \end{macro}
% \end{macro}
% \end{macro}
% \end{macro}
% \end{macro}
%
%
%
% \subsection{Definition of a new theorem set}
%
% As already pointed out, a new theorem environment can be defined in
% three different ways:
% \begin{verbatim}
%    \newtheorem{Lem}{Lemma}
%    \newtheorem{Lem}{Lemma}[section]
%    \newtheorem{Lem}[Theorem]{Lemma}
%\end{verbatim}
% The function of the macro |\newtheorem| is to recognize these cases
% and then to branch into one of the three macros |\@ynthm|, |\@xnthm|
% or |\@othm|.  This mechanism is adopted unchanged from
% \cite{latex:source}; the essential point here is that, for example,
% in the second case, the arguments \texttt{Lem}, \texttt{Lemma} and
% \texttt{section} are passed over to the macro |\@xnthm|.
%
% We inspect this case first because the others present fewer problems,
% and thus are easily derived from this one.
%
% \begin{macro}{\@xnthm}
%    For our example arguments, the macro |\@xnthm| must fulfill
%    the following:
%    \begin{itemize}
%       \item
%          Define a new \LaTeX-counter `Lem'
%       \item
%          reset this counter within a |\section|
%       \item
%          define the macro |\theLem|
%       \item
%          define the environment macros |\Lem| and
%          |\endLem| using the current
%          |\theoremstyle| and |\theorem@bodyfont|.
%    \end{itemize}
%    Obviously, all this should happen only if the first argument
%    of |\@xnthm| (i.e.\ \texttt{Lem} in our example) is chosen so
%    as not to conflict with any previously defined commands or
%    environments.  This test is performed by the \LaTeX{} macro
%    |\@ifdefinable|.
%    \begin{macrocode}
\gdef\@xnthm#1#2[#3]{\expandafter\@ifdefinable\csname #1\endcsname
%    \end{macrocode}
%    Therefore, the first argument of |\@ifdefinable| is the
%    expansion (in the example, |\Lem|) of
%    |\csname||#1\endcsname|.  The second argument is
%    executed only if the test has been completed successfully.
%    \begin{macrocode}
   {%
%    \end{macrocode}
%    Now we define the new counter. The names of the \LaTeX{}
%    macros employed should speak for themselves:
% \changes{v2.2b}{1995/10/07}{Use \cs{@newctr}}
% \changes{v2.2c}{1995/11/23}{Use \cs{@newctr} properly.}
%    \begin{macrocode}
    \@definecounter{#1}\@newctr{#1}[#3]%
%    \end{macrocode}
%    Using |\@newctr| will give a proper error message if the counter
%    in |#3| is not defined.
%    In defining `|\theLem|' we must generate the desired macro
%    name by use of |\expandafter| and |\csname|.
%    \begin{macrocode}
    \expandafter\xdef\csname the#1\endcsname
%    \end{macrocode}
%    An |\xdef| is used in order to make the definition global,
%    and to ensure that it contains the replacement texts of
%    |\@thmcountersep| and |\@thmcounter|.\footnote{These
%    two macros can be defined by the document style.  Their default
%    values produce a `\texttt{.}' as separation and an arabic
%    representation of the number.} However, not everything should be
%    expanded. For example, it saves space to use |\thesection|
%    instead of its---at times---lengthy expansion.
%    \begin{macrocode}
      {\expandafter \noexpand \csname the#3\endcsname
       \@thmcountersep \@thmcounter{#1}}%
%    \end{macrocode}
%    Thus with the defaults of \LaTeX{}, |\theLem| would be
%    replaced by the command sequence |\thesection.\arabic{Lem}|.
%
%    We will now look at the definition of the macro which is executed
%    at the beginning of the actual environment (in our example this
%    macro is |\Lem|). It should be noted that we use an
%    ``|\expandafter| trick'' to expand only certain parts of the
%    replacement text at the time of the definition.
%    \begin{macrocode}
    \def\@tempa{\global\@namedef{#1}}%
    \expandafter \@tempa \expandafter{%
%    \end{macrocode}
%    First, the macro that contains the current definitions of
%    |\@begintheorem| and |\@opargtheorem| should be called
%    up.  The name of this macro---as is already known---has the form
%    |\th@|\meta{theorem\ style}; therefore, it must be called by
%    \begin{macrocode}
      \csname th@\the \theorem@style
%    \end{macrocode}
%    In addition the default theorem font should be changeable, i.e.\
%    we have to insert the contents of |\theorem@bodyfont|. For
%    that reason, we expand even further, beyond |\endcsname|,
%    and thus insert the contents of the \textsf{token} register
%    |\theorem@bodyfont| in the replacement text.
%    \begin{macrocode}
            \expandafter \endcsname \the \theorem@bodyfont
%    \end{macrocode}
%    Now it is time to call the macro |\@thm| which takes over
%    the further processing.  It has two arguments: the current
%    counter name (in our example, \texttt{Lem}), and the text of the
%    label (in our example, \texttt{Lemma}).
%    \begin{macrocode}
     \@thm{#1}{#2}}%
%    \end{macrocode}
%    With this, the `sub-definition' is complete.  The macro
%    |\@endtheorem| ends a theorem environment and is, so far, nothing
%    but an |\endtrivlist|.  (Hence it is defined globally, and not
%    within the theorem styles.\footnote{This has to be changed as
%    soon as theorem styles that change \texttt{\bslash @endtheorem}
%    exist.  In such a case, all existing styles must be changed as
%    well since they will have to reset the macro.}) Therefore, we can
%    set it equivalent to the macro that ends the theorem set (in our
%    example, |\endLem|).  However, if some day theorem styles exist
%    that do change |\@endtheorem|, we would have to use the
%    commented-out line instead.
%    \begin{macrocode}
    \global \expandafter \let \csname end#1\endcsname \@endtheorem
%   \global\@namedef{end#1}{\@endtheorem}%
%    \end{macrocode}
%    With these commands all the required definitions are employed,
%    unless the test |\@ifdefinable| has failed.  Therefore, we
%    end the second argument of this macro and with it the definition
%    of |\@xnthm|.
%    \begin{macrocode}
   }}
%    \end{macrocode}
% \end{macro}
%
% \begin{macro}{\@ynthm}
%    The definition of |\@ynthm| is completely analogous. In
%    this case the new counter that is defined is not reset within
%    another counter; thus the definition of |\the...|
%    is simplified:
%    \begin{macrocode}
\gdef\@ynthm#1#2{\expandafter\@ifdefinable\csname #1\endcsname
   {\@definecounter{#1}%
    \expandafter\xdef\csname the#1\endcsname{\@thmcounter{#1}}%
%    \end{macrocode}
%    The rest of the definition corresponds literally to that of
%    |\@xnthm|:
%    \begin{macrocode}
    \def\@tempa{\global\@namedef{#1}}\expandafter \@tempa
     \expandafter{\csname th@\the \theorem@style \expandafter
     \endcsname \the\theorem@bodyfont \@thm{#1}{#2}}%
    \global \expandafter \let \csname end#1\endcsname \@endtheorem}}
%    \end{macrocode}
% \end{macro}
%
% \begin{macro}{\@othm}
%    The definition of |\@othm| does not contain anything new.
%    \begin{macrocode}
\gdef\@othm#1[#2]#3{%
%    \end{macrocode}
% \changes{v2.2b}{95/10/07}{Build in check for counter defined}
%    We do not define a new counter but instead use one that has
%    already been defined.  Thus the only definition we need is that
%    of this pseudo-counter (i.e.\ |\the|\meta{env.\ name}).
%    First we check if |#2| corresponds to a known counter name.
%    \begin{macrocode}
  \expandafter\ifx\csname c@#2\endcsname\relax
   \@nocounterr{#2}%
  \else
   \expandafter\@ifdefinable\csname #1\endcsname
   {\expandafter \xdef \csname the#1\endcsname
     {\expandafter \noexpand \csname the#2\endcsname}%
%    \end{macrocode}
%    All other parts of the definition can be adopted from
%    |\@xnthm|. We have to remember, though, that in this case
%    the name of the current counter and the theorem label have moved
%    to the second and third arguments.
%    \begin{macrocode}
    \def\@tempa{\global\@namedef{#1}}\expandafter \@tempa
     \expandafter{\csname th@\the \theorem@style \expandafter
     \endcsname \the\theorem@bodyfont \@thm{#2}{#3}}%
    \global \expandafter \let \csname end#1\endcsname \@endtheorem}%
  \fi}
%    \end{macrocode}
% \end{macro}
%
% \subsection{Macros that are employed in a {\envfont theorem}
%    environment}
%
% \begin{macro}{\@thm}
%    The macro |\@thm| has to increase the current counter.  Then,
%    depending on whether the environment has (or does not have) an
%    optional argument, it has to branch into either |\@begintheorem|
%    or |\@opargtheorem|.
%    \begin{macrocode}
\gdef\@thm#1#2{\refstepcounter{#1}%
%    \end{macrocode}
%    Now we start a {\envfont trivlist} environment, and give
%    |\@topsep| and |\@topsepadd| the values of the \textsf{skip}
%    registers |\theorempreskipamount| and |\theorempostskipamount|.
%    The value in |\@topsep| is the vertical space that is inserted by
%    the first (and only) |\item| in our |\trivlist| whilst
%    |\@topsepadd| is inserted by |\@endparenv| at the end of that
%    {\envfont trivlist} environment.  By using these registers, we
%    obtain the desired space around a {\envfont theorem} environment.
%    \begin{macrocode}
   \trivlist
   \@topsep \theorempreskipamount               % used by first \item
   \@topsepadd \theorempostskipamount           % used by \@endparenv
%    \end{macrocode}
%    Now we have to test whether an optional argument has been given.
% \changes{v2.0d}{89/07/18}{Forgotten percent added.}
%    \begin{macrocode}
   \@ifnextchar [%
%    \end{macrocode}
%    If there is an optional argument, we will call |\@ythm|, and
%    move the arguments read back into the input stream.
%    \begin{macrocode}
   {\@ythm{#1}{#2}}%
%    \end{macrocode}
%    If not, we call |\@begintheorem|. Its first argument is the
%    name of the theorem set (hence the second argument of
%    |\@thm|). Its second argument is the macro that produces the
%    current number.
%    \begin{macrocode}
   {\@begintheorem{#2}{\csname the#1\endcsname}\ignorespaces}}
%    \end{macrocode}
% \end{macro}
%
% \begin{macro}{\@xthm}
% \begin{macro}{\@ythm}
%    Both these macros were originally called by |\@thm|.  We do
%    not need |\@xthm| anymore, hence we reset it to
%    |\relax|. The definition of |\@ythm| has not changed at
%    all from its definition in \LaTeX. In order to make the macros
%    easier to understand, we will
%    nevertheless present it (commented out).
%    \begin{macrocode}
\global\let\@xthm\relax
% \def\@ythm#1#2[#3]{\@opargbegintheorem{#2}{\csname
%           the#1\endcsname}{#3}\ignorespaces}
%</package>
%    \end{macrocode}
%    The primitive |\ignorespaces| in |\@ythm| and
%    |\@thm| is needed to remove the spaces between the
%    |\begin{...}| and the actual text.
% \end{macro}
% \end{macro}
%
%
% \subsection{Definition of the theorem substyles}
%
% As already pointed out, the theorem substyles, defined below, are
% only loaded when necessary.  Note that all these substyles, except
% \texttt{plain}, have |\slshape| as the default body font.
%
% \subsubsection{The \texttt{plain} style}
%
% \label{dok:th@plain} As the following macros use |@|, we have
% to locally set the |\catcode| of this symbol to ``letter''.
% This happens within a group, so that we do not have to worry about
% which |\catcode| that symbol had before.
%    \begin{macrocode}
%<*thp>
\begingroup \makeatletter
%    \end{macrocode}
% Since we are now within a group, we must make all definitions
% globally.  First we make sure that \texttt{theorem.sty} is loaded.
% This will allow us to use this file as a document style option
% without having to call \texttt{theorem} itself as an option.  At the
% same time, we assure that at least version 2 is loaded, since
% |\theorem@style| was not defined in earlier versions.
%    \begin{macrocode}
\@ifundefined{theorem@style}{\input{theorem.sty}}{}
\ProvidesFile{thp.sty}
             [\FMithmInfo]
%    \end{macrocode}
%
%
% \begin{macro}{\th@plain}
% \begin{macro}{\@begintheorem}
% \begin{macro}{\@opargbegintheorem}
%    |\theoremstyle{plain}| corresponds to the original
%    definition, except that the distances to the surrounding text are
%    determined by the parameters |\theorempreskipamount| and
%    |\theorempostskipamount|.
%    First we set the default body font.
%    \begin{macrocode}
\gdef\th@plain{\normalfont\itshape
%    \end{macrocode}
%    Then we define |\@begintheorem| and
%    |\@opargbegintheorem|. These two macros define how the
%    header of a theorem is typeset.  |\@opargbegintheorem| will
%    be called if a \textsf{theorem} environment with an optional
%    argument is encountered; otherwise, the header is constructed by
%    calling |\@begintheorem|.  If one of these macros is
%    executed, we are within a \textsf{trivlist} environment started by
%    |\@thm|.  So the theorem header is produced with an
%    |\item| command.
%
%    Instead of specifying the header font directly, all standard
%    theorem styles use the |\theorem@headerfont| macro to allow
%    customization.  The extra space (|\labelsep|) is necessary
%    because of problems in the \textsf{trivlist} environment.
%    \begin{macrocode}
  \def\@begintheorem##1##2{%
        \item[\hskip\labelsep \theorem@headerfont ##1\ ##2]}%
%    \end{macrocode}
%    The definition of |\@opargbegintheorem| is completely
%    analogous. The only difference is the fact that there exists a
%    third argument (which is the optional parameter of the
%    environment and contains additional information about the
%    theorem).  Customarily we enclose it in parentheses.
%    \begin{macrocode}
\def\@opargbegintheorem##1##2##3{%
   \item[\hskip\labelsep \theorem@headerfont ##1\ ##2\ (##3)]}}
%    \end{macrocode}
% \end{macro}
% \end{macro}
% \end{macro}
%
% \noindent
% We conclude with an |\endgroup| to restore the
% |\catcode| of |@|.
%    \begin{macrocode}
\endgroup
%</thp>
%    \end{macrocode}
%
%
% \subsubsection{The \texttt{break} style}
%
% This style option is stored in the file |thb.sty|.
% For the next two lines see the documentation for |\th@plain| on
% page~\pageref{dok:th@plain}.
%    \begin{macrocode}
%<*thb>
\begingroup \makeatletter
\@ifundefined{theorem@style}{\input{theorem.sty}}{}
\ProvidesFile{thb.sty}
             [\FMithmInfo]
%    \end{macrocode}
%
% \begin{macro}{\th@break}
%    |\theoremstyle{break}| produces a line break after the name
%    of the theorem.  The font is |\slshape|.  Hence, we define
%    |\th@break| as follows:
%    \begin{macrocode}
\gdef\th@break{\normalfont\slshape
  \def\@begintheorem##1##2{\item[%
%    \end{macrocode}
%    We run into the following problem: it is not possible to create
%    the header with |\item[|\meta{title}|]| and then start
%    a new line by, for example, |\mbox{}\\|. Such a definition
%    will fail whenever a list environment follows immediately.  With
%    the above construction, the |\mbox{}| causes the switch
%    |@inlabel| (cf.\ definition of |\list| and
%    |\trivlist| in \cite{latex:source}) to be set to \texttt{false}
%    and so the following list will insert additional vertical
%    space (|\topskip|). This is quite annoying. Therefore, we
%    create the line break within the |\item|. In order to ensure
%    that the text will begin at the proper position in the following
%    line, we simply pretend that the label does not take any
%    room.\footnote{This will lead to problems whenever very high
%    symbols occurring in the line tower into the
%    heading. So, something else has to be done here sometime.}
%    \begin{macrocode}
       \rlap{\vbox{\hbox{\hskip \labelsep\theorem@headerfont ##1\ ##2}%
                   \hbox{\strut}}}]}%
%    \end{macrocode}
%    Again, the definition of |\@opargbegintheorem| is
%    completely analogous.
%    \begin{macrocode}
\def\@opargbegintheorem##1##2##3{%
  \item[\rlap{\vbox{\hbox{\hskip \labelsep \theorem@headerfont
                     ##1\ ##2\ (##3)}%
                    \hbox{\strut}}}]}}
\endgroup
%</thb>
%    \end{macrocode}
% \end{macro}
%
%
% \subsubsection{The \texttt{changebreak} style}
%
%    \begin{macrocode}
% This style option is stored in the file |thcb.sty|.
%    \begin{macrocode}
%<*thcb>
\begingroup \makeatletter
\@ifundefined{theorem@style}{\input{theorem.sty}}{}
\ProvidesFile{thcb.sty}
             [\FMithmInfo]
%    \end{macrocode}
%
% \begin{macro}{\th@changebreak}
%    The \texttt{change-break} theorem style is like \texttt{break}
%    but with interchange of theorem name and theorem number.  Thus we
%    define |\th@changebreak| as follows:
%    \begin{macrocode}
\gdef\th@changebreak{\normalfont\slshape
  \def\@begintheorem##1##2{\item
       [\rlap{\vbox{\hbox{\hskip\labelsep \theorem@headerfont ##2\ ##1}%
                    \hbox{\strut}}}]}%
\def\@opargbegintheorem##1##2##3{%
  \item[\rlap{\vbox{\hbox{\hskip\labelsep \theorem@headerfont
                     ##2\ ##1\ (##3)}%
                    \hbox{\strut}}}]}}
\endgroup
%</thcb>
%    \end{macrocode}
% \end{macro}
%
%
% \subsubsection{The \texttt{change} style}
%
% This style option is stored in the file |thc.sty|.
%    \begin{macrocode}
%<*thc>
\begingroup \makeatletter
\@ifundefined{theorem@style}{\input{theorem.sty}}{}
\ProvidesFile{thc.sty}
             [\FMithmInfo]
%    \end{macrocode}
%
% \begin{macro}{\th@change}
%    The \texttt{change} theorem style corresponds to the
%    \texttt{change break} style without a linebreak after the header.
%    To say it in another way, it's the same as the \texttt{plain}
%    style but with number and name interchanged and |\slshape| as the
%    default font.
%    \begin{macrocode}
\gdef\th@change{\normalfont\slshape
  \def\@begintheorem##1##2{\item
       [\hskip\labelsep \theorem@headerfont ##2\ ##1]}%
\def\@opargbegintheorem##1##2##3{%
  \item[\hskip\labelsep \theorem@headerfont ##2\ ##1\ (##3)]}}
\endgroup
%</thc>
%    \end{macrocode}
% \end{macro}
%
%
% \subsubsection{The \texttt{marginbreak} style}
%
% This style option is the one used most often at Mainz.  It is saved
% in the file |thmb.sty|.
%    \begin{macrocode}
%<*thmb>
\begingroup \makeatletter
\@ifundefined{theorem@style}{\input{theorem.sty}}{}
\ProvidesFile{thmb.sty}
             [\FMithmInfo]
%    \end{macrocode}
%
% \begin{macro}{\th@marginbreak}
%    The \texttt{margin break} style is nearly the same as the
%    \texttt{change break} style. The only difference is the placement
%    of the theorem number.  We use |\llap| to place it in the left
%    margin.
%
%    In this style |\labelsep| denotes the separation between the
%    number and the text.
%    \begin{macrocode}
\gdef\th@marginbreak{\normalfont\slshape
  \def\@begintheorem##1##2{\item
       [\rlap{\vbox{\theorem@headerfont
                    \hbox{\llap{##2}\hskip\labelsep ##1}%
                    \hbox{\strut}}}]}%
\def\@opargbegintheorem##1##2##3{%
  \item[\rlap{\vbox{\theorem@headerfont
                    \hbox{\llap{##2}\hskip\labelsep ##1\ (##3)}%
                    \hbox{\strut}}}]}}
\endgroup
%</thmb>
%    \end{macrocode}
% \changes{v2.1c}{90/02/17}{`endgroup outside macrocode env.}
% \end{macro}
%
%
%
% \subsubsection{The \texttt{margin} style}
%
% This style option is stored in the file |thm.sty|.
%    \begin{macrocode}
%<*thm>
\begingroup \makeatletter
\@ifundefined{theorem@style}{\input{theorem.sty}}{}
\ProvidesFile{thm.sty}
             [\FMithmInfo]
%    \end{macrocode}
%
% \begin{macro}{\th@margin}
%    Again this is only a variant of the theorem styles described
%    above without any new ideas.
%    \begin{macrocode}
\gdef\th@margin{\normalfont\slshape
  \def\@begintheorem##1##2{\item
       [\theorem@headerfont \llap{##2}\hskip\labelsep ##1]}%
\def\@opargbegintheorem##1##2##3{%
  \item[\theorem@headerfont \llap{##2}\hskip\labelsep ##1\ (##3)]}}
\endgroup
%</thm>
%    \end{macrocode}
% \end{macro}
%
%
%
%
% \subsection{Final Definitions}
%
% \begin{macro}{\theorempreskipamount}
% \begin{macro}{\theorempostskipamount}
%    The \textsf{skip} parameters that regulate the vertical empty space
%    before and after the {\envfont theorem} environment have
%    to be allocated as well.
% \changes{v2.0f}{89/07/20}{Unnecessary `global' removed.}
%    \begin{macrocode}
%<*package>
\newskip\theorempreskipamount
\newskip\theorempostskipamount
%    \end{macrocode}
%    Since we have used the same values for all theorem sets, we now can
%    assign them.
%    \begin{macrocode}
\global\setlength\theorempreskipamount{12pt plus 5pt minus 3pt}
\global\setlength\theorempostskipamount{8pt plus 3pt minus 1.5pt}
%    \end{macrocode}
% \end{macro}
% \end{macro}
%
%
% \begin{macro}{\@endtheorem}
%    The same holds for the macro |\@endtheorem|, which ends a
%    {\envfont theorem} environment. Since it is the same for all
%    theorem sets, it is removed from the macros |\th@|\meta{style}.
%    It simply ends the {\envfont trivlist} environment, which was
%    begun in |\@thm|.
%    \begin{macrocode}
\global\let\@endtheorem=\endtrivlist
%    \end{macrocode}
% \end{macro}
%
%
%
% \begin{macro}{\@preamblecmds}
%    All macros defined above are to be used only in the preamble.
%    Therefore, we insert them in |\@preamblecmds| which will disable
%    them at begin document. This is done by the internal \LaTeXe{}
%    command |\@onlypreamble|.
% \changes{v2.0g}{89/09/19}{`theoremheaderfont and `theorembodyfont
%    added.}
% \changes{v2.2a}{94/02/03}{Use \LaTeXe{} interface}
%    \begin{macrocode}
\@onlypreamble\@xnthm
\@onlypreamble\@ynthm
\@onlypreamble\@othm
\@onlypreamble\newtheorem
\@onlypreamble\theoremstyle
\@onlypreamble\theorembodyfont
\@onlypreamble\theoremheaderfont
%    \end{macrocode}
% \end{macro}
%
% Finally we declare the \texttt{plain} theorem style to be the default.
% \changes{v2.1b}{90/06/04}{Fixed missing `item by declaring default
%    style.}
%    \begin{macrocode}
\theoremstyle{plain}
%</package>
%    \end{macrocode}
%
% \Finale

\endinput
}{}
\ProvidesFile{thmb.sty}
             [\FMithmInfo]
%    \end{macrocode}
%
% \begin{macro}{\th@marginbreak}
%    The \texttt{margin break} style is nearly the same as the
%    \texttt{change break} style. The only difference is the placement
%    of the theorem number.  We use |\llap| to place it in the left
%    margin.
%
%    In this style |\labelsep| denotes the separation between the
%    number and the text.
%    \begin{macrocode}
\gdef\th@marginbreak{\normalfont\slshape
  \def\@begintheorem##1##2{\item
       [\rlap{\vbox{\theorem@headerfont
                    \hbox{\llap{##2}\hskip\labelsep ##1}%
                    \hbox{\strut}}}]}%
\def\@opargbegintheorem##1##2##3{%
  \item[\rlap{\vbox{\theorem@headerfont
                    \hbox{\llap{##2}\hskip\labelsep ##1\ (##3)}%
                    \hbox{\strut}}}]}}
\endgroup
%</thmb>
%    \end{macrocode}
% \changes{v2.1c}{90/02/17}{`endgroup outside macrocode env.}
% \end{macro}
%
%
%
% \subsubsection{The \texttt{margin} style}
%
% This style option is stored in the file |thm.sty|.
%    \begin{macrocode}
%<*thm>
\begingroup \makeatletter
\@ifundefined{theorem@style}{% \iffalse meta-comment
%
% Copyright (C) 1993-2024
%
% The LaTeX Project and any individual authors listed elsewhere
% in this file.
%
% This file is part of the Standard LaTeX `Tools Bundle'.
% -------------------------------------------------------
%
% It may be distributed and/or modified under the
% conditions of the LaTeX Project Public License, either version 1.3c
% of this license or (at your option) any later version.
% The latest version of this license is in
%    https://www.latex-project.org/lppl.txt
% and version 1.3c or later is part of all distributions of LaTeX
% version 2005/12/01 or later.
%
% The list of all files belonging to the LaTeX `Tools Bundle' is
% given in the file `manifest.txt'.
%
% \fi
%
%%
%
%
\def\FMithmInfo{2023/07/05 v2.2c Theorem extension package (FMi)}
%
% \ProvidesFile{theorem.dtx}[\FMithmInfo]
%\iffalse   % this is a METACOMMENT !
%%
%% Package `theorem' to use with LaTeX2e
%% Copyright (C) 1989-1995 Frank Mittelbach, all rights reserved.
%<+package>\@ifundefined{theorem@style}{}{\endinput}
%<+package>\NeedsTeXFormat{LaTeX2e}[1995/06/01]
%<+package>\ProvidesPackage{theorem}
%<+package>         [\FMithmInfo]
%
% \fi
%
% \changes{v1.0}{87/05/07}{First draft.}
% \changes{v1.0a}{87/05/07}{`thistheoremstyle' added.}
% \changes{v1.1a}{87/07/29}{New concept: redefine the internal
%                          \protect\LaTeX{} commands `@xnthm' etc.}
% \changes{v1.1b}{87/07/31}{`theoremfont' used to specify font.}
% \changes{v1.2a}{88/01/08}{`@endtheorem' now globally defined.}
% \changes{v1.2b}{88/01/11}{`newtheorem' only allowed in the preamble.}
% \changes{v1.3a}{88/02/14}{`theoremfont' now a toks register.}
%
% \changes{v2.0a}{88/11/16}{New implementation and new documentation.}
% \changes{v2.0a}{88/11/16}{No layout changes but internal modifications
%                           old substyles won't work.}
% \changes{v2.0b}{89/01/30}{`global' used for all definitions.}
% \changes{v2.0c}{89/06/15}{Documentation translated to English.
%                            Thanks to LLL and Nina}
% \changes{v2.1a}{90/02/15}{Support for new font selection.}
% \changes{v2.2a}{94/02/03}{Cleanup for LaTeX2e}
%
% \DoNotIndex{\@endparenv,\@ifdefinable,\@ifnextchar,\@ifundefined}
% \DoNotIndex{\@namedef,\@nameuse,\@spaces,\@tempa,\@warning}
% \DoNotIndex{\ }
% \DoNotIndex{\begingroup,\bfseries}
% \DoNotIndex{\csname}
% \DoNotIndex{\def,\do,\docdate}
% \DoNotIndex{\endcsname,\endgroup,\endinput,\endtrivlist,\expandafter}
% \DoNotIndex{\filedate,\fileversion}
% \DoNotIndex{\gdef,\global}
% \DoNotIndex{\hbox,\hskip}
% \DoNotIndex{\ignorespaces,\input,\itshape,\item}
% \DoNotIndex{\labelsep,\let,\llap}
% \DoNotIndex{\makeatletter}
% \DoNotIndex{\noexpand}
% \DoNotIndex{\refstepcounter,\relax,\rlap}
% \DoNotIndex{\setlength,\slshape,\space,\string,\strut}
% \DoNotIndex{\the,\trivlist,\typeout}
% \DoNotIndex{\vbox}
% \DoNotIndex{\xdef}
%
% \def\deflabel#1{\ttfamily #1\hfill}
% \def\deflist#1{\begin{list}{}{\settowidth\labelwidth{\ttfamily #1}%
%                               \setlength\leftmargin\labelwidth
%                               \addtolength\leftmargin\labelsep
%                               \let\makelabel\deflabel}}
% \def\enddeflist{\end{list}}
% \def\envfont{\normalfont\ttfamily}
%
% \MakeShortVerb{\|}
%
% \newcommand{\TUB}{{\sl TUGboat\/}}
%
% \GetFileInfo{theorem.dtx}
% \title{An Extension of the
%        \LaTeX{} theorem environment\thanks{This file
%        has version number \fileversion, last
%        revised \filedate.}}
% \author{Frank Mittelbach}
%
% \MaintainedByLaTeXTeam{tools}
% \maketitle
%
% \begin{abstract}
%    The macros described in this paper yield an extension of the
%    \LaTeX{} theorem mechanism. It is designed to satisfy the
%    different requirements of various journals. Thus, the
%    layout of the ``theorems'' can be manipulated by determining a
%    ``style''. This article describes not only the use, but
%    also the definition, of the necessary macros.
% \end{abstract}
%
% \section*{Preface to version 2.2}
%
% For \LaTeXe{} this package did not need any fundamental changes. I
% only modified the messages generated so that theorem layout styles
% will show up with the |\listfiles| command and cleaned the section
% on the New Font Selection Scheme since this is now included in
% \LaTeX.
%
% \section*{Preface to version 2.1}
%
% This version is identical to 2.0g described in \TUB\ 10\#3 except
% for some internal defaults which are now set depending on the used
% font selection scheme.
%
% This was done to avoid unpleasant surprises if the new font selection
% scheme is in force. For further details see
% section~\ref{sec:newfonts} and \cite{bk:GMS94}.
%
% \section{Introduction}
%
% For our purposes here, ``theorems'' are labelled enunciations,
% often set off from the main text by extra space and a font change.
% Theorems, corollaries, conjectures, definitions, and remarks
% are all instances of ``theorems''.  The ``header'' of these structures
% is composed of a label (such as \textsc{Theorem} or \textsc{Remark})
% and a number which serializes an item in the sequence of items
% with the same label.
%
% Shortly after the introduction of \LaTeX{} at the Fachbereich
% Mathematik in Mainz, the desire to manipulate the layout of
% ``theorems'' arose.  In Mainz, the following two conventions came
% into general use:
% \begin{enumerate}
%   \item The number of the theorem is shown in the margin.
%   \item There is a line break at
%         the end of the theorem header.
% \end{enumerate}
%
% Additionally, some journals require different formats which depend on
% the ``sort of theorem'': e.g.\ often remarks and definitions are set
% in |\upshape|, while |\itshape| is employed for main theorems.
%
% Confronted with these requirements, a theorem environment was
% developed in Mainz which allows separate determination of the
% layout of the ``theorems sets'', comparable to |\pagestyle|.
%
%
% \section{The user interface}
% \subsection{Defining new theorem sets}
%
% \DescribeMacro\newtheorem
% As in the original \LaTeX{} version, the command |\newtheorem|
% defines a new ``theorem set'' or ``theorem-like structure''.
% Two required arguments name the new environment and give the text
% to be typeset with each instance of the new ``set'', while
% an optional argument determines how
% the ``set'' is enumerated:
% \begin{description}
%    \item[\ttfamily \bslash newtheorem\{foo\}\{bar\}]
%       The theorem set {\envfont foo} (whose name is \texttt{bar})
%       uses its own counter.
%    \item[\ttfamily \bslash newtheorem\{foo2\}{[foo]}\{bar2\}]
%       The theorem set {\envfont foo2} (printed name \texttt{bar2})
%       uses the same counter as the theorem set \texttt{foo}.
%    \item[\ttfamily \bslash newtheorem\{foo3\}\{bar3\}{[section]}]
%       The theorem set {\envfont foo3} (printed name \texttt{bar3}) is
%       enumerated within the counter \texttt{section}, i.e.\ with every
%       new |\section| the enumeration begins again with 1, and
%       the enumeration is composed from the section-number and the
%       theorem counter itself.
% \end{description}
%
% \DescribeMacro\theoremstyle
% Additionally, the command |\theoremstyle| can define the layout
% of various, or all, theorem sets.  It should be noted that any
% theorem set defined by |\newtheorem| is typeset in the
% |\theoremstyle| that is current at the time of the definition.
% Thus, the following
% \begin{verbatim}
%    \theoremstyle{break}        \newtheorem{Cor}{Corollary}
%    \theoremstyle{plain}        \newtheorem{Exa}{Example}[section]
%\end{verbatim}                             ^^A No blank at line start
% leads to the result that the set {\envfont Cor} is formatted in the
% style \texttt{break}, while the set {\envfont Exa} and all the
% following ones are formatted in the style \texttt{plain}, unless
% another |\theoremstyle| follows.  Since the definitions installed by
% |\newtheorem| are global, one also can limit |\theoremstyle| locally
% by grouping braces.
%
% \DescribeMacro\theorembodyfont
% The choice of the font for the theorem body is completely
% independent of the chosen |\theoremstyle|; this has proven to
% be very advantageous.  For example,
% \begin{verbatim}
%   {\theorembodyfont{\upshape}          \newtheorem{Rem}{Remark}}
%\end{verbatim}
% defines a theorem set {\envfont Rem}, which will be set in
% |\upshape| in the current layout (which in our example is
% \texttt{plain}).  As with |\theoremstyle|, the |\theorembodyfont|
% chosen is that current at the time of |\newtheorem|.  If
% |\theorembodyfont| is not specified or one defines
% |\theorembodyfont{}|, then the font used will be that defined by the
% |\theoremstyle|.
%
% \DescribeMacro\theoremheaderfont It is also possible to customize
% the font used for the theorem headers.  This is, however, a global
% declaration, and therefore there should be at most one
% |\theoremheaderfont| declaration in the preamble.\footnote{If it is
% actually necessary to have different header fonts, one has to define
% new theorem styles (substituting the desired font) or specify the
% information directly in the \texttt{\bslash newtheorem} declaration
% (the unclean variant).}
%
% \DescribeMacro\theorempreskipamount
% \DescribeMacro\theorempostskipamount Two additional parameters
% affect the vertical space around the theorem environments:
% |\theorempreskipamount| and |\theorempostskipamount| define,
% respectively, the spacing before and after such an environment.
% These parameters apply for all theorem sets and can be manipulated
% with the ordinary length macros.  They are rubber lengths
% (`\textsf{skips}'), and therefore can contain \texttt{plus} and
% \texttt{minus} parts.
%
% Since the definition of theorem sets should---most sensibly---be
% placed in the preamble, we only allow installation there.
% It is therefore possible to release the memory used here after
% |\begin{document}|, in order to make room for other
% applications.
%
%
% \subsection{Existing theorem styles}
%
% The following theorem styles exist to date:
% \begin{deflist}{marginbreak}
%    \item[plain]
%       This theorem style emulates the original \LaTeX{} definition,
%       except that additionally the parameters
%       |\theorem...skipamount| are used.
%    \item[break]
%       In this style, the theorem header is followed by a line
%       break.
%    \item[marginbreak]
%       The theorem number is set in the margin, and there is a line
%       break as in \texttt{break}.
%    \item[changebreak]
%       Like \texttt{break}, but with header number and text
%       interchanged.
%    \item[change]
%       Header number and text are interchanged, without a line break.
%    \item[margin]
%       The number is set in the left margin, without a line break.
% \end{deflist}
% All styles (except \texttt{plain}) select |\slshape| as the default
% |\theorembodyfont|.
%
%
% \subsection{Examples}
%
% Given the above theorem sets {\envfont Cor}, {\envfont Exa}
% and {\envfont Rem},
% suppose that the preamble also contains the declarations:
% \begin{verbatim}
%    \theoremstyle{marginbreak}   \newtheorem{Lem}[Cor]{Lemma}
%    \theoremstyle{change}
%    \theorembodyfont{\itshape}        \newtheorem{Def}[Cor]{Definition}
%
%    \theoremheaderfont{\scshape}
%\end{verbatim}
% Then the following are some typical examples of the typeset output
% resulting from their use.
%
% \begin{Cor}
%    This is a sentence typeset in the theorem environment {\envfont
%    Cor}.
% \end{Cor}
%
% \begin{Exa}
%    This is a sentence typeset in the theorem environment {\envfont
%    Exa}.
% \end{Exa}
%
% \begin{Rem}
%    This is a sentence typeset in the theorem environment {\envfont
%    Rem}.
% \end{Rem}
%
% \begin{Lem}[Ben User]
%    This is a sentence typeset in the theorem environment {\envfont
%    Lem}.
% \end{Lem}
%
% \begin{Def}[Very impressive Definition]
%    This is a sentence typeset in the theorem environment {\envfont
%    Def}.
% \end{Def}
%
% The last two examples show the effect of the optional argument to
% a theorem environment (it is the text typeset in parentheses).
%
%
% \section{Special Considerations}
% \label{sec:newfonts}
%
% Theoremheader and body are implemented as a unit. This means that
% the |\theoremheaderfont|  will inherit characteristics of the
% |\theorembodyfont| in \LaTeXe. Thus, if for example
% |\theorembodyfont| is |\itshape| and |\theoremheaderfont|
% is |\bfseries| the font selected for the header will have the
% characteristics `bold extended italic'. If this is not desired one
% should set the |\theoremheaderfont| to something like
% \begin{verbatim}
%      \theoremheaderfont{\normalfont\bfseries}
%\end{verbatim}
% i.e.\ supplying all necessary font information explicitly.
%
%
%
% \section{Acknowledgements}
%
% The publication of this set of macros was only possible with the
% help of Christina Busse (translating the manuscript into English),
% Joachim Pense (playing the r\^ole of typist), Chris Rowley
% (looking everything over) and many others providing useful
% suggestions.
%
% \MaybeStop{%
%    \begin{thebibliography}{9}
%       \bibitem{bk:GMS94} \textsc{M.~Goossens}, \textsc{F.~Mittelbach}
%         and \textsc{A.~Samarin}.
%         \newblock The \LaTeX{} Companion.
%         \newblock
%         Addison-Wesley, Reading, Massachusetts, 1994.
%       \bibitem{latex:source} \textsc{Lamport, Leslie.} \newblock
%                               \texttt{latex.tex}, version 2.09, date
%                               Feb. 1990.
%    \end{thebibliography}
%    \PrintIndex
%    \PrintChanges}
%
%
% \section{The documentation driver file}
%
% The next bit of code contains the documentation driver file for
% \TeX{}, i.e., the file that will produce the documentation you are
% currently reading. It will be extracted from this file by the
% \texttt{docstrip} program. Since it is the first code in the file
% one can alternatively process this file directly with \LaTeXe{} to
% obtain the documentation.
%
%    \begin{macrocode}
%<*driver>
\documentclass{ltxdoc}

\usepackage{theorem}

%<+driver>% The next few lines define theorem sets which are used
%<+driver>% in the example section of the documentation.

\theoremstyle{break}       \newtheorem{Cor}{Corollary}
\theoremstyle{plain}       \newtheorem{Exa}{Example}[section]
{\theorembodyfont{\upshape}\newtheorem{Rem}{Remark}}
\theoremstyle{marginbreak} \newtheorem{Lem}[Cor]{Lemma}
\theoremstyle{change}
\theorembodyfont{\itshape} \newtheorem{Def}[Cor]{Definition}

\theoremheaderfont{\scshape}

\RecordChanges

\begin{document}
   \DocInput{theorem.dtx}
\end{document}
%</driver>
%    \end{macrocode}
%
% \section{Definition of the Macros}
%
% If the file has been loaded before, we abort immediately. If not the
% package announces itself (this is actually done at the very top of
% the file---the way it is done isn't good style so don't copy it).
% \changes{v2.0e}{89/07/19}{Spaces removed from `typeout'.}
% \changes{v2.2b}{95/11/19}{Announce moved to top of file}
%    \begin{macrocode}
%<*package>
%\@ifundefined{theorem@style}{}{\endinput}
%\def\FMithmInfo{1995/11/19 v2.2b Theorem extension package (FMi)}
%\ProvidesPackage{theorem}[\FMithmInfo]
%    \end{macrocode}
%
%
% \subsection{Definition of theorem styles and fonts}
%
% All the definitions in this file are done globally to allow
% inputting this file inside a group.
%
% \begin{macro}{\theoremstyle}
%    Before a theorem style can be installed, the chosen style must be
%    known. For that reason, we must test to see that
%    |\th@|\meta{style} is known or, more precisely, that it is
%    different from |\relax|. If the style is not known then
%    |\th@plain| is used.
%    \begin{macrocode}
\gdef\theoremstyle#1{%
   \@ifundefined{th@#1}{\@warning
          {Unknown theoremstyle `#1'. Using `plain'}%
          \theorem@style{plain}}%
%    \end{macrocode}
%    We save the theorem style to be used in the token register
%    |\theorem@style|.
%    \begin{macrocode}
      {\theorem@style{#1}}%
%    \end{macrocode}
%    Now we ``evaluate'' the theorem style: this means, we call the
%    macro |\th@|\meta{style} which will activate the relevant
%    definitions which are contained in a separate file.  This is done
%    in a group to suppress changes to the current font.  This could
%    otherwise pose problems together with the new font selection
%    scheme\footnote{When I printed the original article using the new
%    font selection scheme I ended with a document with slanted
%    typefaces (text headings and all) simply because one of the
%    theorem styles used \texttt{\bslash sl} at toplevel.} if the
%    |\th@|\meta{style} is evaluated a second time.
% \changes{v2.1a}{90/02/15}{Group added around `th@..}
% \changes{v2.1a}{90/02/15}{Reseting of `..@begintheorem removed.}
%    \begin{macrocode}
      \begingroup
        \csname th@\the\theorem@style \endcsname
      \endgroup}
%    \end{macrocode}
% \end{macro}
%
%
% \begin{macro}{\@begintheorem}
% \changes{v2.1a}{90/02/15}{Relaxed at toplevel.}
% \begin{macro}{\@opargbegintheorem}
% \changes{v2.1a}{90/02/15}{Relaxed at toplevel.}
%    We reset |\@begintheorem| and |\@opargbegintheorem| to
%    |\relax| since these commands are no longer necessary at
%    toplevel. This will save a few tokens.
%    \begin{macrocode}
\global\let\@begintheorem\relax
\global\let\@opargbegintheorem\relax
%    \end{macrocode}
% \end{macro}
% \end{macro}
%
% \begin{macro}{\theorem@style}
%    Obviously the \textsf{token} register used above has to be
%    allocated. To assure the utmost compatibility with the
%    original \LaTeX{} definition, we set the default theorem style to
%    \texttt{plain}, which implements the usual \LaTeX{} convention.
% \changes{v2.0f}{89/07/20}{Unnecessary `global' removed.}
%    \begin{macrocode}
\newtoks\theorem@style
\global\theorem@style{plain}
%    \end{macrocode}
% \end{macro}
%
% \begin{macro}{\theorembodyfont}
% \begin{macro}{\theorem@bodyfont}
%    If the |\theorembodyfont| is set by the user then it should not
%    interact with the default font set in the theorem style. When the
%    new font selection is in force this may happen if, for example,
%    the default is |\itshape| and the new |\theorembodyfont| is
%    |\sffamily|. So we add a |\reset@font| command in front of the
%    user definition.
% \changes{v2.0f}{89/07/20}{Unnecessary `global' removed.}
% \changes{v2.0g}{89/09/19}{`theoremfont to `theorembodyfont.}
% \changes{v2.1c}{92/06/22}{`reset@font added and logic changed.}
% \changes{v2.1d}{92/10/28}{Reset to empty if arg is empty}
%    \begin{macrocode}
\gdef\theorembodyfont#1{%
%    \end{macrocode}
%    We check if the argument supplied is empty and if so put nothing
%    into the |\theorem@bodyfont| token register to allow for
%    |\theorembodyfont{}| as a mean of using the default of the
%    current |\theoremstyle|.
%    \begin{macrocode}
   \def\@tempa{#1}%
   \ifx\@tempa\@empty
    \theorem@bodyfont{}%
   \else
    \theorem@bodyfont{\reset@font#1}%
   \fi
   }
\newtoks\theorem@bodyfont
\global\theorem@bodyfont{}
%    \end{macrocode}
% \end{macro}
% \end{macro}
%
%
% \begin{macro}{\theoremheaderfont}
% \changes{v2.0g}{89/09/19}{Macro added.}
%    The font for the theorem headers is handled differently because
%    this definition applies to all theorem styles.
%    \begin{macrocode}
\gdef\theoremheaderfont#1{\gdef\theorem@headerfont{#1}%
%    \end{macrocode}
%    After using the macro once it is redefined to produce an error
%    message.
%    \begin{macrocode}
       \gdef\theoremheaderfont##1{%
        \typeout{\string\theoremheaderfont\space should be used
                 only once.}}}
%    \end{macrocode}
% \end{macro}
%
% \begin{macro}{\theorem@headerfont}
% \changes{v2.0g}{89/09/19}{Macro added.}
%    To set the |\theorem@headerfont| default we first test if
%    the new fontselection scheme is in force.
% \changes{v2.1a}{90/02/15}{Support for new font selection.}
%    \begin{macrocode}
\ifx\upshape\undefined
%    \end{macrocode}
%    If not we define it to expand into |\bfseries|. We don't use
%    |\let| just in case a following style option redefines this
%    macro.
%    \begin{macrocode}
\gdef\theorem@headerfont{\bfseries}
%    \end{macrocode}
%    Otherwise we reset the current shape before calling |\bfseries|.
%    \begin{macrocode}
\else \gdef\theorem@headerfont{\normalfont\bfseries}\fi
%    \end{macrocode}
% \end{macro}
%
% \begin{macro}{\th@plain}
% \begin{macro}{\th@break}
% \begin{macro}{\th@marginbreak}
% \begin{macro}{\th@changebreak}
% \begin{macro}{\th@change}
% \begin{macro}{\th@margin}
%    The different styles are defined in macros such as
%    |\th@plain|.  Since memory space is precious in
%    ``non-Big-versions'', we have to avoid offering too many unused
%    definitions.  Therefore we define these styles in separate files
%    that can be loaded on demand.  Thus the commands themselves only
%    load these files. We use |\@input@|, a \LaTeXe{} internal command
%    that ensures that the file will be listed with |\listfiles|.
% \changes{v2.2a}{94/02/02}{Use `@input@ to load theorem layout files}
%    \begin{macrocode}
\gdef\th@plain{\@input@{thp.sty}}
\gdef\th@break{\@input@{thb.sty}}
\gdef\th@marginbreak{\@input@{thmb.sty}}
\gdef\th@changebreak{\@input@{thcb.sty}}
\gdef\th@change{\@input@{thc.sty}}
\gdef\th@margin{\@input@{thm.sty}}
%    \end{macrocode}
%    This list will be expanded when new styles become
%    available.  For testing,  just append new theorem substyles
%    as document options.
% \end{macro}
% \end{macro}
% \end{macro}
% \end{macro}
% \end{macro}
% \end{macro}
%
%
%
% \subsection{Definition of a new theorem set}
%
% As already pointed out, a new theorem environment can be defined in
% three different ways:
% \begin{verbatim}
%    \newtheorem{Lem}{Lemma}
%    \newtheorem{Lem}{Lemma}[section]
%    \newtheorem{Lem}[Theorem]{Lemma}
%\end{verbatim}
% The function of the macro |\newtheorem| is to recognize these cases
% and then to branch into one of the three macros |\@ynthm|, |\@xnthm|
% or |\@othm|.  This mechanism is adopted unchanged from
% \cite{latex:source}; the essential point here is that, for example,
% in the second case, the arguments \texttt{Lem}, \texttt{Lemma} and
% \texttt{section} are passed over to the macro |\@xnthm|.
%
% We inspect this case first because the others present fewer problems,
% and thus are easily derived from this one.
%
% \begin{macro}{\@xnthm}
%    For our example arguments, the macro |\@xnthm| must fulfill
%    the following:
%    \begin{itemize}
%       \item
%          Define a new \LaTeX-counter `Lem'
%       \item
%          reset this counter within a |\section|
%       \item
%          define the macro |\theLem|
%       \item
%          define the environment macros |\Lem| and
%          |\endLem| using the current
%          |\theoremstyle| and |\theorem@bodyfont|.
%    \end{itemize}
%    Obviously, all this should happen only if the first argument
%    of |\@xnthm| (i.e.\ \texttt{Lem} in our example) is chosen so
%    as not to conflict with any previously defined commands or
%    environments.  This test is performed by the \LaTeX{} macro
%    |\@ifdefinable|.
%    \begin{macrocode}
\gdef\@xnthm#1#2[#3]{\expandafter\@ifdefinable\csname #1\endcsname
%    \end{macrocode}
%    Therefore, the first argument of |\@ifdefinable| is the
%    expansion (in the example, |\Lem|) of
%    |\csname||#1\endcsname|.  The second argument is
%    executed only if the test has been completed successfully.
%    \begin{macrocode}
   {%
%    \end{macrocode}
%    Now we define the new counter. The names of the \LaTeX{}
%    macros employed should speak for themselves:
% \changes{v2.2b}{1995/10/07}{Use \cs{@newctr}}
% \changes{v2.2c}{1995/11/23}{Use \cs{@newctr} properly.}
%    \begin{macrocode}
    \@definecounter{#1}\@newctr{#1}[#3]%
%    \end{macrocode}
%    Using |\@newctr| will give a proper error message if the counter
%    in |#3| is not defined.
%    In defining `|\theLem|' we must generate the desired macro
%    name by use of |\expandafter| and |\csname|.
%    \begin{macrocode}
    \expandafter\xdef\csname the#1\endcsname
%    \end{macrocode}
%    An |\xdef| is used in order to make the definition global,
%    and to ensure that it contains the replacement texts of
%    |\@thmcountersep| and |\@thmcounter|.\footnote{These
%    two macros can be defined by the document style.  Their default
%    values produce a `\texttt{.}' as separation and an arabic
%    representation of the number.} However, not everything should be
%    expanded. For example, it saves space to use |\thesection|
%    instead of its---at times---lengthy expansion.
%    \begin{macrocode}
      {\expandafter \noexpand \csname the#3\endcsname
       \@thmcountersep \@thmcounter{#1}}%
%    \end{macrocode}
%    Thus with the defaults of \LaTeX{}, |\theLem| would be
%    replaced by the command sequence |\thesection.\arabic{Lem}|.
%
%    We will now look at the definition of the macro which is executed
%    at the beginning of the actual environment (in our example this
%    macro is |\Lem|). It should be noted that we use an
%    ``|\expandafter| trick'' to expand only certain parts of the
%    replacement text at the time of the definition.
%    \begin{macrocode}
    \def\@tempa{\global\@namedef{#1}}%
    \expandafter \@tempa \expandafter{%
%    \end{macrocode}
%    First, the macro that contains the current definitions of
%    |\@begintheorem| and |\@opargtheorem| should be called
%    up.  The name of this macro---as is already known---has the form
%    |\th@|\meta{theorem\ style}; therefore, it must be called by
%    \begin{macrocode}
      \csname th@\the \theorem@style
%    \end{macrocode}
%    In addition the default theorem font should be changeable, i.e.\
%    we have to insert the contents of |\theorem@bodyfont|. For
%    that reason, we expand even further, beyond |\endcsname|,
%    and thus insert the contents of the \textsf{token} register
%    |\theorem@bodyfont| in the replacement text.
%    \begin{macrocode}
            \expandafter \endcsname \the \theorem@bodyfont
%    \end{macrocode}
%    Now it is time to call the macro |\@thm| which takes over
%    the further processing.  It has two arguments: the current
%    counter name (in our example, \texttt{Lem}), and the text of the
%    label (in our example, \texttt{Lemma}).
%    \begin{macrocode}
     \@thm{#1}{#2}}%
%    \end{macrocode}
%    With this, the `sub-definition' is complete.  The macro
%    |\@endtheorem| ends a theorem environment and is, so far, nothing
%    but an |\endtrivlist|.  (Hence it is defined globally, and not
%    within the theorem styles.\footnote{This has to be changed as
%    soon as theorem styles that change \texttt{\bslash @endtheorem}
%    exist.  In such a case, all existing styles must be changed as
%    well since they will have to reset the macro.}) Therefore, we can
%    set it equivalent to the macro that ends the theorem set (in our
%    example, |\endLem|).  However, if some day theorem styles exist
%    that do change |\@endtheorem|, we would have to use the
%    commented-out line instead.
%    \begin{macrocode}
    \global \expandafter \let \csname end#1\endcsname \@endtheorem
%   \global\@namedef{end#1}{\@endtheorem}%
%    \end{macrocode}
%    With these commands all the required definitions are employed,
%    unless the test |\@ifdefinable| has failed.  Therefore, we
%    end the second argument of this macro and with it the definition
%    of |\@xnthm|.
%    \begin{macrocode}
   }}
%    \end{macrocode}
% \end{macro}
%
% \begin{macro}{\@ynthm}
%    The definition of |\@ynthm| is completely analogous. In
%    this case the new counter that is defined is not reset within
%    another counter; thus the definition of |\the...|
%    is simplified:
%    \begin{macrocode}
\gdef\@ynthm#1#2{\expandafter\@ifdefinable\csname #1\endcsname
   {\@definecounter{#1}%
    \expandafter\xdef\csname the#1\endcsname{\@thmcounter{#1}}%
%    \end{macrocode}
%    The rest of the definition corresponds literally to that of
%    |\@xnthm|:
%    \begin{macrocode}
    \def\@tempa{\global\@namedef{#1}}\expandafter \@tempa
     \expandafter{\csname th@\the \theorem@style \expandafter
     \endcsname \the\theorem@bodyfont \@thm{#1}{#2}}%
    \global \expandafter \let \csname end#1\endcsname \@endtheorem}}
%    \end{macrocode}
% \end{macro}
%
% \begin{macro}{\@othm}
%    The definition of |\@othm| does not contain anything new.
%    \begin{macrocode}
\gdef\@othm#1[#2]#3{%
%    \end{macrocode}
% \changes{v2.2b}{95/10/07}{Build in check for counter defined}
%    We do not define a new counter but instead use one that has
%    already been defined.  Thus the only definition we need is that
%    of this pseudo-counter (i.e.\ |\the|\meta{env.\ name}).
%    First we check if |#2| corresponds to a known counter name.
%    \begin{macrocode}
  \expandafter\ifx\csname c@#2\endcsname\relax
   \@nocounterr{#2}%
  \else
   \expandafter\@ifdefinable\csname #1\endcsname
   {\expandafter \xdef \csname the#1\endcsname
     {\expandafter \noexpand \csname the#2\endcsname}%
%    \end{macrocode}
%    All other parts of the definition can be adopted from
%    |\@xnthm|. We have to remember, though, that in this case
%    the name of the current counter and the theorem label have moved
%    to the second and third arguments.
%    \begin{macrocode}
    \def\@tempa{\global\@namedef{#1}}\expandafter \@tempa
     \expandafter{\csname th@\the \theorem@style \expandafter
     \endcsname \the\theorem@bodyfont \@thm{#2}{#3}}%
    \global \expandafter \let \csname end#1\endcsname \@endtheorem}%
  \fi}
%    \end{macrocode}
% \end{macro}
%
% \subsection{Macros that are employed in a {\envfont theorem}
%    environment}
%
% \begin{macro}{\@thm}
%    The macro |\@thm| has to increase the current counter.  Then,
%    depending on whether the environment has (or does not have) an
%    optional argument, it has to branch into either |\@begintheorem|
%    or |\@opargtheorem|.
%    \begin{macrocode}
\gdef\@thm#1#2{\refstepcounter{#1}%
%    \end{macrocode}
%    Now we start a {\envfont trivlist} environment, and give
%    |\@topsep| and |\@topsepadd| the values of the \textsf{skip}
%    registers |\theorempreskipamount| and |\theorempostskipamount|.
%    The value in |\@topsep| is the vertical space that is inserted by
%    the first (and only) |\item| in our |\trivlist| whilst
%    |\@topsepadd| is inserted by |\@endparenv| at the end of that
%    {\envfont trivlist} environment.  By using these registers, we
%    obtain the desired space around a {\envfont theorem} environment.
%    \begin{macrocode}
   \trivlist
   \@topsep \theorempreskipamount               % used by first \item
   \@topsepadd \theorempostskipamount           % used by \@endparenv
%    \end{macrocode}
%    Now we have to test whether an optional argument has been given.
% \changes{v2.0d}{89/07/18}{Forgotten percent added.}
%    \begin{macrocode}
   \@ifnextchar [%
%    \end{macrocode}
%    If there is an optional argument, we will call |\@ythm|, and
%    move the arguments read back into the input stream.
%    \begin{macrocode}
   {\@ythm{#1}{#2}}%
%    \end{macrocode}
%    If not, we call |\@begintheorem|. Its first argument is the
%    name of the theorem set (hence the second argument of
%    |\@thm|). Its second argument is the macro that produces the
%    current number.
%    \begin{macrocode}
   {\@begintheorem{#2}{\csname the#1\endcsname}\ignorespaces}}
%    \end{macrocode}
% \end{macro}
%
% \begin{macro}{\@xthm}
% \begin{macro}{\@ythm}
%    Both these macros were originally called by |\@thm|.  We do
%    not need |\@xthm| anymore, hence we reset it to
%    |\relax|. The definition of |\@ythm| has not changed at
%    all from its definition in \LaTeX. In order to make the macros
%    easier to understand, we will
%    nevertheless present it (commented out).
%    \begin{macrocode}
\global\let\@xthm\relax
% \def\@ythm#1#2[#3]{\@opargbegintheorem{#2}{\csname
%           the#1\endcsname}{#3}\ignorespaces}
%</package>
%    \end{macrocode}
%    The primitive |\ignorespaces| in |\@ythm| and
%    |\@thm| is needed to remove the spaces between the
%    |\begin{...}| and the actual text.
% \end{macro}
% \end{macro}
%
%
% \subsection{Definition of the theorem substyles}
%
% As already pointed out, the theorem substyles, defined below, are
% only loaded when necessary.  Note that all these substyles, except
% \texttt{plain}, have |\slshape| as the default body font.
%
% \subsubsection{The \texttt{plain} style}
%
% \label{dok:th@plain} As the following macros use |@|, we have
% to locally set the |\catcode| of this symbol to ``letter''.
% This happens within a group, so that we do not have to worry about
% which |\catcode| that symbol had before.
%    \begin{macrocode}
%<*thp>
\begingroup \makeatletter
%    \end{macrocode}
% Since we are now within a group, we must make all definitions
% globally.  First we make sure that \texttt{theorem.sty} is loaded.
% This will allow us to use this file as a document style option
% without having to call \texttt{theorem} itself as an option.  At the
% same time, we assure that at least version 2 is loaded, since
% |\theorem@style| was not defined in earlier versions.
%    \begin{macrocode}
\@ifundefined{theorem@style}{\input{theorem.sty}}{}
\ProvidesFile{thp.sty}
             [\FMithmInfo]
%    \end{macrocode}
%
%
% \begin{macro}{\th@plain}
% \begin{macro}{\@begintheorem}
% \begin{macro}{\@opargbegintheorem}
%    |\theoremstyle{plain}| corresponds to the original
%    definition, except that the distances to the surrounding text are
%    determined by the parameters |\theorempreskipamount| and
%    |\theorempostskipamount|.
%    First we set the default body font.
%    \begin{macrocode}
\gdef\th@plain{\normalfont\itshape
%    \end{macrocode}
%    Then we define |\@begintheorem| and
%    |\@opargbegintheorem|. These two macros define how the
%    header of a theorem is typeset.  |\@opargbegintheorem| will
%    be called if a \textsf{theorem} environment with an optional
%    argument is encountered; otherwise, the header is constructed by
%    calling |\@begintheorem|.  If one of these macros is
%    executed, we are within a \textsf{trivlist} environment started by
%    |\@thm|.  So the theorem header is produced with an
%    |\item| command.
%
%    Instead of specifying the header font directly, all standard
%    theorem styles use the |\theorem@headerfont| macro to allow
%    customization.  The extra space (|\labelsep|) is necessary
%    because of problems in the \textsf{trivlist} environment.
%    \begin{macrocode}
  \def\@begintheorem##1##2{%
        \item[\hskip\labelsep \theorem@headerfont ##1\ ##2]}%
%    \end{macrocode}
%    The definition of |\@opargbegintheorem| is completely
%    analogous. The only difference is the fact that there exists a
%    third argument (which is the optional parameter of the
%    environment and contains additional information about the
%    theorem).  Customarily we enclose it in parentheses.
%    \begin{macrocode}
\def\@opargbegintheorem##1##2##3{%
   \item[\hskip\labelsep \theorem@headerfont ##1\ ##2\ (##3)]}}
%    \end{macrocode}
% \end{macro}
% \end{macro}
% \end{macro}
%
% \noindent
% We conclude with an |\endgroup| to restore the
% |\catcode| of |@|.
%    \begin{macrocode}
\endgroup
%</thp>
%    \end{macrocode}
%
%
% \subsubsection{The \texttt{break} style}
%
% This style option is stored in the file |thb.sty|.
% For the next two lines see the documentation for |\th@plain| on
% page~\pageref{dok:th@plain}.
%    \begin{macrocode}
%<*thb>
\begingroup \makeatletter
\@ifundefined{theorem@style}{\input{theorem.sty}}{}
\ProvidesFile{thb.sty}
             [\FMithmInfo]
%    \end{macrocode}
%
% \begin{macro}{\th@break}
%    |\theoremstyle{break}| produces a line break after the name
%    of the theorem.  The font is |\slshape|.  Hence, we define
%    |\th@break| as follows:
%    \begin{macrocode}
\gdef\th@break{\normalfont\slshape
  \def\@begintheorem##1##2{\item[%
%    \end{macrocode}
%    We run into the following problem: it is not possible to create
%    the header with |\item[|\meta{title}|]| and then start
%    a new line by, for example, |\mbox{}\\|. Such a definition
%    will fail whenever a list environment follows immediately.  With
%    the above construction, the |\mbox{}| causes the switch
%    |@inlabel| (cf.\ definition of |\list| and
%    |\trivlist| in \cite{latex:source}) to be set to \texttt{false}
%    and so the following list will insert additional vertical
%    space (|\topskip|). This is quite annoying. Therefore, we
%    create the line break within the |\item|. In order to ensure
%    that the text will begin at the proper position in the following
%    line, we simply pretend that the label does not take any
%    room.\footnote{This will lead to problems whenever very high
%    symbols occurring in the line tower into the
%    heading. So, something else has to be done here sometime.}
%    \begin{macrocode}
       \rlap{\vbox{\hbox{\hskip \labelsep\theorem@headerfont ##1\ ##2}%
                   \hbox{\strut}}}]}%
%    \end{macrocode}
%    Again, the definition of |\@opargbegintheorem| is
%    completely analogous.
%    \begin{macrocode}
\def\@opargbegintheorem##1##2##3{%
  \item[\rlap{\vbox{\hbox{\hskip \labelsep \theorem@headerfont
                     ##1\ ##2\ (##3)}%
                    \hbox{\strut}}}]}}
\endgroup
%</thb>
%    \end{macrocode}
% \end{macro}
%
%
% \subsubsection{The \texttt{changebreak} style}
%
%    \begin{macrocode}
% This style option is stored in the file |thcb.sty|.
%    \begin{macrocode}
%<*thcb>
\begingroup \makeatletter
\@ifundefined{theorem@style}{\input{theorem.sty}}{}
\ProvidesFile{thcb.sty}
             [\FMithmInfo]
%    \end{macrocode}
%
% \begin{macro}{\th@changebreak}
%    The \texttt{change-break} theorem style is like \texttt{break}
%    but with interchange of theorem name and theorem number.  Thus we
%    define |\th@changebreak| as follows:
%    \begin{macrocode}
\gdef\th@changebreak{\normalfont\slshape
  \def\@begintheorem##1##2{\item
       [\rlap{\vbox{\hbox{\hskip\labelsep \theorem@headerfont ##2\ ##1}%
                    \hbox{\strut}}}]}%
\def\@opargbegintheorem##1##2##3{%
  \item[\rlap{\vbox{\hbox{\hskip\labelsep \theorem@headerfont
                     ##2\ ##1\ (##3)}%
                    \hbox{\strut}}}]}}
\endgroup
%</thcb>
%    \end{macrocode}
% \end{macro}
%
%
% \subsubsection{The \texttt{change} style}
%
% This style option is stored in the file |thc.sty|.
%    \begin{macrocode}
%<*thc>
\begingroup \makeatletter
\@ifundefined{theorem@style}{\input{theorem.sty}}{}
\ProvidesFile{thc.sty}
             [\FMithmInfo]
%    \end{macrocode}
%
% \begin{macro}{\th@change}
%    The \texttt{change} theorem style corresponds to the
%    \texttt{change break} style without a linebreak after the header.
%    To say it in another way, it's the same as the \texttt{plain}
%    style but with number and name interchanged and |\slshape| as the
%    default font.
%    \begin{macrocode}
\gdef\th@change{\normalfont\slshape
  \def\@begintheorem##1##2{\item
       [\hskip\labelsep \theorem@headerfont ##2\ ##1]}%
\def\@opargbegintheorem##1##2##3{%
  \item[\hskip\labelsep \theorem@headerfont ##2\ ##1\ (##3)]}}
\endgroup
%</thc>
%    \end{macrocode}
% \end{macro}
%
%
% \subsubsection{The \texttt{marginbreak} style}
%
% This style option is the one used most often at Mainz.  It is saved
% in the file |thmb.sty|.
%    \begin{macrocode}
%<*thmb>
\begingroup \makeatletter
\@ifundefined{theorem@style}{\input{theorem.sty}}{}
\ProvidesFile{thmb.sty}
             [\FMithmInfo]
%    \end{macrocode}
%
% \begin{macro}{\th@marginbreak}
%    The \texttt{margin break} style is nearly the same as the
%    \texttt{change break} style. The only difference is the placement
%    of the theorem number.  We use |\llap| to place it in the left
%    margin.
%
%    In this style |\labelsep| denotes the separation between the
%    number and the text.
%    \begin{macrocode}
\gdef\th@marginbreak{\normalfont\slshape
  \def\@begintheorem##1##2{\item
       [\rlap{\vbox{\theorem@headerfont
                    \hbox{\llap{##2}\hskip\labelsep ##1}%
                    \hbox{\strut}}}]}%
\def\@opargbegintheorem##1##2##3{%
  \item[\rlap{\vbox{\theorem@headerfont
                    \hbox{\llap{##2}\hskip\labelsep ##1\ (##3)}%
                    \hbox{\strut}}}]}}
\endgroup
%</thmb>
%    \end{macrocode}
% \changes{v2.1c}{90/02/17}{`endgroup outside macrocode env.}
% \end{macro}
%
%
%
% \subsubsection{The \texttt{margin} style}
%
% This style option is stored in the file |thm.sty|.
%    \begin{macrocode}
%<*thm>
\begingroup \makeatletter
\@ifundefined{theorem@style}{\input{theorem.sty}}{}
\ProvidesFile{thm.sty}
             [\FMithmInfo]
%    \end{macrocode}
%
% \begin{macro}{\th@margin}
%    Again this is only a variant of the theorem styles described
%    above without any new ideas.
%    \begin{macrocode}
\gdef\th@margin{\normalfont\slshape
  \def\@begintheorem##1##2{\item
       [\theorem@headerfont \llap{##2}\hskip\labelsep ##1]}%
\def\@opargbegintheorem##1##2##3{%
  \item[\theorem@headerfont \llap{##2}\hskip\labelsep ##1\ (##3)]}}
\endgroup
%</thm>
%    \end{macrocode}
% \end{macro}
%
%
%
%
% \subsection{Final Definitions}
%
% \begin{macro}{\theorempreskipamount}
% \begin{macro}{\theorempostskipamount}
%    The \textsf{skip} parameters that regulate the vertical empty space
%    before and after the {\envfont theorem} environment have
%    to be allocated as well.
% \changes{v2.0f}{89/07/20}{Unnecessary `global' removed.}
%    \begin{macrocode}
%<*package>
\newskip\theorempreskipamount
\newskip\theorempostskipamount
%    \end{macrocode}
%    Since we have used the same values for all theorem sets, we now can
%    assign them.
%    \begin{macrocode}
\global\setlength\theorempreskipamount{12pt plus 5pt minus 3pt}
\global\setlength\theorempostskipamount{8pt plus 3pt minus 1.5pt}
%    \end{macrocode}
% \end{macro}
% \end{macro}
%
%
% \begin{macro}{\@endtheorem}
%    The same holds for the macro |\@endtheorem|, which ends a
%    {\envfont theorem} environment. Since it is the same for all
%    theorem sets, it is removed from the macros |\th@|\meta{style}.
%    It simply ends the {\envfont trivlist} environment, which was
%    begun in |\@thm|.
%    \begin{macrocode}
\global\let\@endtheorem=\endtrivlist
%    \end{macrocode}
% \end{macro}
%
%
%
% \begin{macro}{\@preamblecmds}
%    All macros defined above are to be used only in the preamble.
%    Therefore, we insert them in |\@preamblecmds| which will disable
%    them at begin document. This is done by the internal \LaTeXe{}
%    command |\@onlypreamble|.
% \changes{v2.0g}{89/09/19}{`theoremheaderfont and `theorembodyfont
%    added.}
% \changes{v2.2a}{94/02/03}{Use \LaTeXe{} interface}
%    \begin{macrocode}
\@onlypreamble\@xnthm
\@onlypreamble\@ynthm
\@onlypreamble\@othm
\@onlypreamble\newtheorem
\@onlypreamble\theoremstyle
\@onlypreamble\theorembodyfont
\@onlypreamble\theoremheaderfont
%    \end{macrocode}
% \end{macro}
%
% Finally we declare the \texttt{plain} theorem style to be the default.
% \changes{v2.1b}{90/06/04}{Fixed missing `item by declaring default
%    style.}
%    \begin{macrocode}
\theoremstyle{plain}
%</package>
%    \end{macrocode}
%
% \Finale

\endinput
}{}
\ProvidesFile{thm.sty}
             [\FMithmInfo]
%    \end{macrocode}
%
% \begin{macro}{\th@margin}
%    Again this is only a variant of the theorem styles described
%    above without any new ideas.
%    \begin{macrocode}
\gdef\th@margin{\normalfont\slshape
  \def\@begintheorem##1##2{\item
       [\theorem@headerfont \llap{##2}\hskip\labelsep ##1]}%
\def\@opargbegintheorem##1##2##3{%
  \item[\theorem@headerfont \llap{##2}\hskip\labelsep ##1\ (##3)]}}
\endgroup
%</thm>
%    \end{macrocode}
% \end{macro}
%
%
%
%
% \subsection{Final Definitions}
%
% \begin{macro}{\theorempreskipamount}
% \begin{macro}{\theorempostskipamount}
%    The \textsf{skip} parameters that regulate the vertical empty space
%    before and after the {\envfont theorem} environment have
%    to be allocated as well.
% \changes{v2.0f}{89/07/20}{Unnecessary `global' removed.}
%    \begin{macrocode}
%<*package>
\newskip\theorempreskipamount
\newskip\theorempostskipamount
%    \end{macrocode}
%    Since we have used the same values for all theorem sets, we now can
%    assign them.
%    \begin{macrocode}
\global\setlength\theorempreskipamount{12pt plus 5pt minus 3pt}
\global\setlength\theorempostskipamount{8pt plus 3pt minus 1.5pt}
%    \end{macrocode}
% \end{macro}
% \end{macro}
%
%
% \begin{macro}{\@endtheorem}
%    The same holds for the macro |\@endtheorem|, which ends a
%    {\envfont theorem} environment. Since it is the same for all
%    theorem sets, it is removed from the macros |\th@|\meta{style}.
%    It simply ends the {\envfont trivlist} environment, which was
%    begun in |\@thm|.
%    \begin{macrocode}
\global\let\@endtheorem=\endtrivlist
%    \end{macrocode}
% \end{macro}
%
%
%
% \begin{macro}{\@preamblecmds}
%    All macros defined above are to be used only in the preamble.
%    Therefore, we insert them in |\@preamblecmds| which will disable
%    them at begin document. This is done by the internal \LaTeXe{}
%    command |\@onlypreamble|.
% \changes{v2.0g}{89/09/19}{`theoremheaderfont and `theorembodyfont
%    added.}
% \changes{v2.2a}{94/02/03}{Use \LaTeXe{} interface}
%    \begin{macrocode}
\@onlypreamble\@xnthm
\@onlypreamble\@ynthm
\@onlypreamble\@othm
\@onlypreamble\newtheorem
\@onlypreamble\theoremstyle
\@onlypreamble\theorembodyfont
\@onlypreamble\theoremheaderfont
%    \end{macrocode}
% \end{macro}
%
% Finally we declare the \texttt{plain} theorem style to be the default.
% \changes{v2.1b}{90/06/04}{Fixed missing `item by declaring default
%    style.}
%    \begin{macrocode}
\theoremstyle{plain}
%</package>
%    \end{macrocode}
%
% \Finale

\endinput
}{}
\ProvidesFile{thc.sty}
             [\FMithmInfo]
%    \end{macrocode}
%
% \begin{macro}{\th@change}
%    The \texttt{change} theorem style corresponds to the
%    \texttt{change break} style without a linebreak after the header.
%    To say it in another way, it's the same as the \texttt{plain}
%    style but with number and name interchanged and |\slshape| as the
%    default font.
%    \begin{macrocode}
\gdef\th@change{\normalfont\slshape
  \def\@begintheorem##1##2{\item
       [\hskip\labelsep \theorem@headerfont ##2\ ##1]}%
\def\@opargbegintheorem##1##2##3{%
  \item[\hskip\labelsep \theorem@headerfont ##2\ ##1\ (##3)]}}
\endgroup
%</thc>
%    \end{macrocode}
% \end{macro}
%
%
% \subsubsection{The \texttt{marginbreak} style}
%
% This style option is the one used most often at Mainz.  It is saved
% in the file |thmb.sty|.
%    \begin{macrocode}
%<*thmb>
\begingroup \makeatletter
\@ifundefined{theorem@style}{% \iffalse meta-comment
%
% Copyright (C) 1993-2024
%
% The LaTeX Project and any individual authors listed elsewhere
% in this file.
%
% This file is part of the Standard LaTeX `Tools Bundle'.
% -------------------------------------------------------
%
% It may be distributed and/or modified under the
% conditions of the LaTeX Project Public License, either version 1.3c
% of this license or (at your option) any later version.
% The latest version of this license is in
%    https://www.latex-project.org/lppl.txt
% and version 1.3c or later is part of all distributions of LaTeX
% version 2005/12/01 or later.
%
% The list of all files belonging to the LaTeX `Tools Bundle' is
% given in the file `manifest.txt'.
%
% \fi
%
%%
%
%
\def\FMithmInfo{2023/07/05 v2.2c Theorem extension package (FMi)}
%
% \ProvidesFile{theorem.dtx}[\FMithmInfo]
%\iffalse   % this is a METACOMMENT !
%%
%% Package `theorem' to use with LaTeX2e
%% Copyright (C) 1989-1995 Frank Mittelbach, all rights reserved.
%<+package>\@ifundefined{theorem@style}{}{\endinput}
%<+package>\NeedsTeXFormat{LaTeX2e}[1995/06/01]
%<+package>\ProvidesPackage{theorem}
%<+package>         [\FMithmInfo]
%
% \fi
%
% \changes{v1.0}{87/05/07}{First draft.}
% \changes{v1.0a}{87/05/07}{`thistheoremstyle' added.}
% \changes{v1.1a}{87/07/29}{New concept: redefine the internal
%                          \protect\LaTeX{} commands `@xnthm' etc.}
% \changes{v1.1b}{87/07/31}{`theoremfont' used to specify font.}
% \changes{v1.2a}{88/01/08}{`@endtheorem' now globally defined.}
% \changes{v1.2b}{88/01/11}{`newtheorem' only allowed in the preamble.}
% \changes{v1.3a}{88/02/14}{`theoremfont' now a toks register.}
%
% \changes{v2.0a}{88/11/16}{New implementation and new documentation.}
% \changes{v2.0a}{88/11/16}{No layout changes but internal modifications
%                           old substyles won't work.}
% \changes{v2.0b}{89/01/30}{`global' used for all definitions.}
% \changes{v2.0c}{89/06/15}{Documentation translated to English.
%                            Thanks to LLL and Nina}
% \changes{v2.1a}{90/02/15}{Support for new font selection.}
% \changes{v2.2a}{94/02/03}{Cleanup for LaTeX2e}
%
% \DoNotIndex{\@endparenv,\@ifdefinable,\@ifnextchar,\@ifundefined}
% \DoNotIndex{\@namedef,\@nameuse,\@spaces,\@tempa,\@warning}
% \DoNotIndex{\ }
% \DoNotIndex{\begingroup,\bfseries}
% \DoNotIndex{\csname}
% \DoNotIndex{\def,\do,\docdate}
% \DoNotIndex{\endcsname,\endgroup,\endinput,\endtrivlist,\expandafter}
% \DoNotIndex{\filedate,\fileversion}
% \DoNotIndex{\gdef,\global}
% \DoNotIndex{\hbox,\hskip}
% \DoNotIndex{\ignorespaces,\input,\itshape,\item}
% \DoNotIndex{\labelsep,\let,\llap}
% \DoNotIndex{\makeatletter}
% \DoNotIndex{\noexpand}
% \DoNotIndex{\refstepcounter,\relax,\rlap}
% \DoNotIndex{\setlength,\slshape,\space,\string,\strut}
% \DoNotIndex{\the,\trivlist,\typeout}
% \DoNotIndex{\vbox}
% \DoNotIndex{\xdef}
%
% \def\deflabel#1{\ttfamily #1\hfill}
% \def\deflist#1{\begin{list}{}{\settowidth\labelwidth{\ttfamily #1}%
%                               \setlength\leftmargin\labelwidth
%                               \addtolength\leftmargin\labelsep
%                               \let\makelabel\deflabel}}
% \def\enddeflist{\end{list}}
% \def\envfont{\normalfont\ttfamily}
%
% \MakeShortVerb{\|}
%
% \newcommand{\TUB}{{\sl TUGboat\/}}
%
% \GetFileInfo{theorem.dtx}
% \title{An Extension of the
%        \LaTeX{} theorem environment\thanks{This file
%        has version number \fileversion, last
%        revised \filedate.}}
% \author{Frank Mittelbach}
%
% \MaintainedByLaTeXTeam{tools}
% \maketitle
%
% \begin{abstract}
%    The macros described in this paper yield an extension of the
%    \LaTeX{} theorem mechanism. It is designed to satisfy the
%    different requirements of various journals. Thus, the
%    layout of the ``theorems'' can be manipulated by determining a
%    ``style''. This article describes not only the use, but
%    also the definition, of the necessary macros.
% \end{abstract}
%
% \section*{Preface to version 2.2}
%
% For \LaTeXe{} this package did not need any fundamental changes. I
% only modified the messages generated so that theorem layout styles
% will show up with the |\listfiles| command and cleaned the section
% on the New Font Selection Scheme since this is now included in
% \LaTeX.
%
% \section*{Preface to version 2.1}
%
% This version is identical to 2.0g described in \TUB\ 10\#3 except
% for some internal defaults which are now set depending on the used
% font selection scheme.
%
% This was done to avoid unpleasant surprises if the new font selection
% scheme is in force. For further details see
% section~\ref{sec:newfonts} and \cite{bk:GMS94}.
%
% \section{Introduction}
%
% For our purposes here, ``theorems'' are labelled enunciations,
% often set off from the main text by extra space and a font change.
% Theorems, corollaries, conjectures, definitions, and remarks
% are all instances of ``theorems''.  The ``header'' of these structures
% is composed of a label (such as \textsc{Theorem} or \textsc{Remark})
% and a number which serializes an item in the sequence of items
% with the same label.
%
% Shortly after the introduction of \LaTeX{} at the Fachbereich
% Mathematik in Mainz, the desire to manipulate the layout of
% ``theorems'' arose.  In Mainz, the following two conventions came
% into general use:
% \begin{enumerate}
%   \item The number of the theorem is shown in the margin.
%   \item There is a line break at
%         the end of the theorem header.
% \end{enumerate}
%
% Additionally, some journals require different formats which depend on
% the ``sort of theorem'': e.g.\ often remarks and definitions are set
% in |\upshape|, while |\itshape| is employed for main theorems.
%
% Confronted with these requirements, a theorem environment was
% developed in Mainz which allows separate determination of the
% layout of the ``theorems sets'', comparable to |\pagestyle|.
%
%
% \section{The user interface}
% \subsection{Defining new theorem sets}
%
% \DescribeMacro\newtheorem
% As in the original \LaTeX{} version, the command |\newtheorem|
% defines a new ``theorem set'' or ``theorem-like structure''.
% Two required arguments name the new environment and give the text
% to be typeset with each instance of the new ``set'', while
% an optional argument determines how
% the ``set'' is enumerated:
% \begin{description}
%    \item[\ttfamily \bslash newtheorem\{foo\}\{bar\}]
%       The theorem set {\envfont foo} (whose name is \texttt{bar})
%       uses its own counter.
%    \item[\ttfamily \bslash newtheorem\{foo2\}{[foo]}\{bar2\}]
%       The theorem set {\envfont foo2} (printed name \texttt{bar2})
%       uses the same counter as the theorem set \texttt{foo}.
%    \item[\ttfamily \bslash newtheorem\{foo3\}\{bar3\}{[section]}]
%       The theorem set {\envfont foo3} (printed name \texttt{bar3}) is
%       enumerated within the counter \texttt{section}, i.e.\ with every
%       new |\section| the enumeration begins again with 1, and
%       the enumeration is composed from the section-number and the
%       theorem counter itself.
% \end{description}
%
% \DescribeMacro\theoremstyle
% Additionally, the command |\theoremstyle| can define the layout
% of various, or all, theorem sets.  It should be noted that any
% theorem set defined by |\newtheorem| is typeset in the
% |\theoremstyle| that is current at the time of the definition.
% Thus, the following
% \begin{verbatim}
%    \theoremstyle{break}        \newtheorem{Cor}{Corollary}
%    \theoremstyle{plain}        \newtheorem{Exa}{Example}[section]
%\end{verbatim}                             ^^A No blank at line start
% leads to the result that the set {\envfont Cor} is formatted in the
% style \texttt{break}, while the set {\envfont Exa} and all the
% following ones are formatted in the style \texttt{plain}, unless
% another |\theoremstyle| follows.  Since the definitions installed by
% |\newtheorem| are global, one also can limit |\theoremstyle| locally
% by grouping braces.
%
% \DescribeMacro\theorembodyfont
% The choice of the font for the theorem body is completely
% independent of the chosen |\theoremstyle|; this has proven to
% be very advantageous.  For example,
% \begin{verbatim}
%   {\theorembodyfont{\upshape}          \newtheorem{Rem}{Remark}}
%\end{verbatim}
% defines a theorem set {\envfont Rem}, which will be set in
% |\upshape| in the current layout (which in our example is
% \texttt{plain}).  As with |\theoremstyle|, the |\theorembodyfont|
% chosen is that current at the time of |\newtheorem|.  If
% |\theorembodyfont| is not specified or one defines
% |\theorembodyfont{}|, then the font used will be that defined by the
% |\theoremstyle|.
%
% \DescribeMacro\theoremheaderfont It is also possible to customize
% the font used for the theorem headers.  This is, however, a global
% declaration, and therefore there should be at most one
% |\theoremheaderfont| declaration in the preamble.\footnote{If it is
% actually necessary to have different header fonts, one has to define
% new theorem styles (substituting the desired font) or specify the
% information directly in the \texttt{\bslash newtheorem} declaration
% (the unclean variant).}
%
% \DescribeMacro\theorempreskipamount
% \DescribeMacro\theorempostskipamount Two additional parameters
% affect the vertical space around the theorem environments:
% |\theorempreskipamount| and |\theorempostskipamount| define,
% respectively, the spacing before and after such an environment.
% These parameters apply for all theorem sets and can be manipulated
% with the ordinary length macros.  They are rubber lengths
% (`\textsf{skips}'), and therefore can contain \texttt{plus} and
% \texttt{minus} parts.
%
% Since the definition of theorem sets should---most sensibly---be
% placed in the preamble, we only allow installation there.
% It is therefore possible to release the memory used here after
% |\begin{document}|, in order to make room for other
% applications.
%
%
% \subsection{Existing theorem styles}
%
% The following theorem styles exist to date:
% \begin{deflist}{marginbreak}
%    \item[plain]
%       This theorem style emulates the original \LaTeX{} definition,
%       except that additionally the parameters
%       |\theorem...skipamount| are used.
%    \item[break]
%       In this style, the theorem header is followed by a line
%       break.
%    \item[marginbreak]
%       The theorem number is set in the margin, and there is a line
%       break as in \texttt{break}.
%    \item[changebreak]
%       Like \texttt{break}, but with header number and text
%       interchanged.
%    \item[change]
%       Header number and text are interchanged, without a line break.
%    \item[margin]
%       The number is set in the left margin, without a line break.
% \end{deflist}
% All styles (except \texttt{plain}) select |\slshape| as the default
% |\theorembodyfont|.
%
%
% \subsection{Examples}
%
% Given the above theorem sets {\envfont Cor}, {\envfont Exa}
% and {\envfont Rem},
% suppose that the preamble also contains the declarations:
% \begin{verbatim}
%    \theoremstyle{marginbreak}   \newtheorem{Lem}[Cor]{Lemma}
%    \theoremstyle{change}
%    \theorembodyfont{\itshape}        \newtheorem{Def}[Cor]{Definition}
%
%    \theoremheaderfont{\scshape}
%\end{verbatim}
% Then the following are some typical examples of the typeset output
% resulting from their use.
%
% \begin{Cor}
%    This is a sentence typeset in the theorem environment {\envfont
%    Cor}.
% \end{Cor}
%
% \begin{Exa}
%    This is a sentence typeset in the theorem environment {\envfont
%    Exa}.
% \end{Exa}
%
% \begin{Rem}
%    This is a sentence typeset in the theorem environment {\envfont
%    Rem}.
% \end{Rem}
%
% \begin{Lem}[Ben User]
%    This is a sentence typeset in the theorem environment {\envfont
%    Lem}.
% \end{Lem}
%
% \begin{Def}[Very impressive Definition]
%    This is a sentence typeset in the theorem environment {\envfont
%    Def}.
% \end{Def}
%
% The last two examples show the effect of the optional argument to
% a theorem environment (it is the text typeset in parentheses).
%
%
% \section{Special Considerations}
% \label{sec:newfonts}
%
% Theoremheader and body are implemented as a unit. This means that
% the |\theoremheaderfont|  will inherit characteristics of the
% |\theorembodyfont| in \LaTeXe. Thus, if for example
% |\theorembodyfont| is |\itshape| and |\theoremheaderfont|
% is |\bfseries| the font selected for the header will have the
% characteristics `bold extended italic'. If this is not desired one
% should set the |\theoremheaderfont| to something like
% \begin{verbatim}
%      \theoremheaderfont{\normalfont\bfseries}
%\end{verbatim}
% i.e.\ supplying all necessary font information explicitly.
%
%
%
% \section{Acknowledgements}
%
% The publication of this set of macros was only possible with the
% help of Christina Busse (translating the manuscript into English),
% Joachim Pense (playing the r\^ole of typist), Chris Rowley
% (looking everything over) and many others providing useful
% suggestions.
%
% \MaybeStop{%
%    \begin{thebibliography}{9}
%       \bibitem{bk:GMS94} \textsc{M.~Goossens}, \textsc{F.~Mittelbach}
%         and \textsc{A.~Samarin}.
%         \newblock The \LaTeX{} Companion.
%         \newblock
%         Addison-Wesley, Reading, Massachusetts, 1994.
%       \bibitem{latex:source} \textsc{Lamport, Leslie.} \newblock
%                               \texttt{latex.tex}, version 2.09, date
%                               Feb. 1990.
%    \end{thebibliography}
%    \PrintIndex
%    \PrintChanges}
%
%
% \section{The documentation driver file}
%
% The next bit of code contains the documentation driver file for
% \TeX{}, i.e., the file that will produce the documentation you are
% currently reading. It will be extracted from this file by the
% \texttt{docstrip} program. Since it is the first code in the file
% one can alternatively process this file directly with \LaTeXe{} to
% obtain the documentation.
%
%    \begin{macrocode}
%<*driver>
\documentclass{ltxdoc}

\usepackage{theorem}

%<+driver>% The next few lines define theorem sets which are used
%<+driver>% in the example section of the documentation.

\theoremstyle{break}       \newtheorem{Cor}{Corollary}
\theoremstyle{plain}       \newtheorem{Exa}{Example}[section]
{\theorembodyfont{\upshape}\newtheorem{Rem}{Remark}}
\theoremstyle{marginbreak} \newtheorem{Lem}[Cor]{Lemma}
\theoremstyle{change}
\theorembodyfont{\itshape} \newtheorem{Def}[Cor]{Definition}

\theoremheaderfont{\scshape}

\RecordChanges

\begin{document}
   \DocInput{theorem.dtx}
\end{document}
%</driver>
%    \end{macrocode}
%
% \section{Definition of the Macros}
%
% If the file has been loaded before, we abort immediately. If not the
% package announces itself (this is actually done at the very top of
% the file---the way it is done isn't good style so don't copy it).
% \changes{v2.0e}{89/07/19}{Spaces removed from `typeout'.}
% \changes{v2.2b}{95/11/19}{Announce moved to top of file}
%    \begin{macrocode}
%<*package>
%\@ifundefined{theorem@style}{}{\endinput}
%\def\FMithmInfo{1995/11/19 v2.2b Theorem extension package (FMi)}
%\ProvidesPackage{theorem}[\FMithmInfo]
%    \end{macrocode}
%
%
% \subsection{Definition of theorem styles and fonts}
%
% All the definitions in this file are done globally to allow
% inputting this file inside a group.
%
% \begin{macro}{\theoremstyle}
%    Before a theorem style can be installed, the chosen style must be
%    known. For that reason, we must test to see that
%    |\th@|\meta{style} is known or, more precisely, that it is
%    different from |\relax|. If the style is not known then
%    |\th@plain| is used.
%    \begin{macrocode}
\gdef\theoremstyle#1{%
   \@ifundefined{th@#1}{\@warning
          {Unknown theoremstyle `#1'. Using `plain'}%
          \theorem@style{plain}}%
%    \end{macrocode}
%    We save the theorem style to be used in the token register
%    |\theorem@style|.
%    \begin{macrocode}
      {\theorem@style{#1}}%
%    \end{macrocode}
%    Now we ``evaluate'' the theorem style: this means, we call the
%    macro |\th@|\meta{style} which will activate the relevant
%    definitions which are contained in a separate file.  This is done
%    in a group to suppress changes to the current font.  This could
%    otherwise pose problems together with the new font selection
%    scheme\footnote{When I printed the original article using the new
%    font selection scheme I ended with a document with slanted
%    typefaces (text headings and all) simply because one of the
%    theorem styles used \texttt{\bslash sl} at toplevel.} if the
%    |\th@|\meta{style} is evaluated a second time.
% \changes{v2.1a}{90/02/15}{Group added around `th@..}
% \changes{v2.1a}{90/02/15}{Reseting of `..@begintheorem removed.}
%    \begin{macrocode}
      \begingroup
        \csname th@\the\theorem@style \endcsname
      \endgroup}
%    \end{macrocode}
% \end{macro}
%
%
% \begin{macro}{\@begintheorem}
% \changes{v2.1a}{90/02/15}{Relaxed at toplevel.}
% \begin{macro}{\@opargbegintheorem}
% \changes{v2.1a}{90/02/15}{Relaxed at toplevel.}
%    We reset |\@begintheorem| and |\@opargbegintheorem| to
%    |\relax| since these commands are no longer necessary at
%    toplevel. This will save a few tokens.
%    \begin{macrocode}
\global\let\@begintheorem\relax
\global\let\@opargbegintheorem\relax
%    \end{macrocode}
% \end{macro}
% \end{macro}
%
% \begin{macro}{\theorem@style}
%    Obviously the \textsf{token} register used above has to be
%    allocated. To assure the utmost compatibility with the
%    original \LaTeX{} definition, we set the default theorem style to
%    \texttt{plain}, which implements the usual \LaTeX{} convention.
% \changes{v2.0f}{89/07/20}{Unnecessary `global' removed.}
%    \begin{macrocode}
\newtoks\theorem@style
\global\theorem@style{plain}
%    \end{macrocode}
% \end{macro}
%
% \begin{macro}{\theorembodyfont}
% \begin{macro}{\theorem@bodyfont}
%    If the |\theorembodyfont| is set by the user then it should not
%    interact with the default font set in the theorem style. When the
%    new font selection is in force this may happen if, for example,
%    the default is |\itshape| and the new |\theorembodyfont| is
%    |\sffamily|. So we add a |\reset@font| command in front of the
%    user definition.
% \changes{v2.0f}{89/07/20}{Unnecessary `global' removed.}
% \changes{v2.0g}{89/09/19}{`theoremfont to `theorembodyfont.}
% \changes{v2.1c}{92/06/22}{`reset@font added and logic changed.}
% \changes{v2.1d}{92/10/28}{Reset to empty if arg is empty}
%    \begin{macrocode}
\gdef\theorembodyfont#1{%
%    \end{macrocode}
%    We check if the argument supplied is empty and if so put nothing
%    into the |\theorem@bodyfont| token register to allow for
%    |\theorembodyfont{}| as a mean of using the default of the
%    current |\theoremstyle|.
%    \begin{macrocode}
   \def\@tempa{#1}%
   \ifx\@tempa\@empty
    \theorem@bodyfont{}%
   \else
    \theorem@bodyfont{\reset@font#1}%
   \fi
   }
\newtoks\theorem@bodyfont
\global\theorem@bodyfont{}
%    \end{macrocode}
% \end{macro}
% \end{macro}
%
%
% \begin{macro}{\theoremheaderfont}
% \changes{v2.0g}{89/09/19}{Macro added.}
%    The font for the theorem headers is handled differently because
%    this definition applies to all theorem styles.
%    \begin{macrocode}
\gdef\theoremheaderfont#1{\gdef\theorem@headerfont{#1}%
%    \end{macrocode}
%    After using the macro once it is redefined to produce an error
%    message.
%    \begin{macrocode}
       \gdef\theoremheaderfont##1{%
        \typeout{\string\theoremheaderfont\space should be used
                 only once.}}}
%    \end{macrocode}
% \end{macro}
%
% \begin{macro}{\theorem@headerfont}
% \changes{v2.0g}{89/09/19}{Macro added.}
%    To set the |\theorem@headerfont| default we first test if
%    the new fontselection scheme is in force.
% \changes{v2.1a}{90/02/15}{Support for new font selection.}
%    \begin{macrocode}
\ifx\upshape\undefined
%    \end{macrocode}
%    If not we define it to expand into |\bfseries|. We don't use
%    |\let| just in case a following style option redefines this
%    macro.
%    \begin{macrocode}
\gdef\theorem@headerfont{\bfseries}
%    \end{macrocode}
%    Otherwise we reset the current shape before calling |\bfseries|.
%    \begin{macrocode}
\else \gdef\theorem@headerfont{\normalfont\bfseries}\fi
%    \end{macrocode}
% \end{macro}
%
% \begin{macro}{\th@plain}
% \begin{macro}{\th@break}
% \begin{macro}{\th@marginbreak}
% \begin{macro}{\th@changebreak}
% \begin{macro}{\th@change}
% \begin{macro}{\th@margin}
%    The different styles are defined in macros such as
%    |\th@plain|.  Since memory space is precious in
%    ``non-Big-versions'', we have to avoid offering too many unused
%    definitions.  Therefore we define these styles in separate files
%    that can be loaded on demand.  Thus the commands themselves only
%    load these files. We use |\@input@|, a \LaTeXe{} internal command
%    that ensures that the file will be listed with |\listfiles|.
% \changes{v2.2a}{94/02/02}{Use `@input@ to load theorem layout files}
%    \begin{macrocode}
\gdef\th@plain{\@input@{thp.sty}}
\gdef\th@break{\@input@{thb.sty}}
\gdef\th@marginbreak{\@input@{thmb.sty}}
\gdef\th@changebreak{\@input@{thcb.sty}}
\gdef\th@change{\@input@{thc.sty}}
\gdef\th@margin{\@input@{thm.sty}}
%    \end{macrocode}
%    This list will be expanded when new styles become
%    available.  For testing,  just append new theorem substyles
%    as document options.
% \end{macro}
% \end{macro}
% \end{macro}
% \end{macro}
% \end{macro}
% \end{macro}
%
%
%
% \subsection{Definition of a new theorem set}
%
% As already pointed out, a new theorem environment can be defined in
% three different ways:
% \begin{verbatim}
%    \newtheorem{Lem}{Lemma}
%    \newtheorem{Lem}{Lemma}[section]
%    \newtheorem{Lem}[Theorem]{Lemma}
%\end{verbatim}
% The function of the macro |\newtheorem| is to recognize these cases
% and then to branch into one of the three macros |\@ynthm|, |\@xnthm|
% or |\@othm|.  This mechanism is adopted unchanged from
% \cite{latex:source}; the essential point here is that, for example,
% in the second case, the arguments \texttt{Lem}, \texttt{Lemma} and
% \texttt{section} are passed over to the macro |\@xnthm|.
%
% We inspect this case first because the others present fewer problems,
% and thus are easily derived from this one.
%
% \begin{macro}{\@xnthm}
%    For our example arguments, the macro |\@xnthm| must fulfill
%    the following:
%    \begin{itemize}
%       \item
%          Define a new \LaTeX-counter `Lem'
%       \item
%          reset this counter within a |\section|
%       \item
%          define the macro |\theLem|
%       \item
%          define the environment macros |\Lem| and
%          |\endLem| using the current
%          |\theoremstyle| and |\theorem@bodyfont|.
%    \end{itemize}
%    Obviously, all this should happen only if the first argument
%    of |\@xnthm| (i.e.\ \texttt{Lem} in our example) is chosen so
%    as not to conflict with any previously defined commands or
%    environments.  This test is performed by the \LaTeX{} macro
%    |\@ifdefinable|.
%    \begin{macrocode}
\gdef\@xnthm#1#2[#3]{\expandafter\@ifdefinable\csname #1\endcsname
%    \end{macrocode}
%    Therefore, the first argument of |\@ifdefinable| is the
%    expansion (in the example, |\Lem|) of
%    |\csname||#1\endcsname|.  The second argument is
%    executed only if the test has been completed successfully.
%    \begin{macrocode}
   {%
%    \end{macrocode}
%    Now we define the new counter. The names of the \LaTeX{}
%    macros employed should speak for themselves:
% \changes{v2.2b}{1995/10/07}{Use \cs{@newctr}}
% \changes{v2.2c}{1995/11/23}{Use \cs{@newctr} properly.}
%    \begin{macrocode}
    \@definecounter{#1}\@newctr{#1}[#3]%
%    \end{macrocode}
%    Using |\@newctr| will give a proper error message if the counter
%    in |#3| is not defined.
%    In defining `|\theLem|' we must generate the desired macro
%    name by use of |\expandafter| and |\csname|.
%    \begin{macrocode}
    \expandafter\xdef\csname the#1\endcsname
%    \end{macrocode}
%    An |\xdef| is used in order to make the definition global,
%    and to ensure that it contains the replacement texts of
%    |\@thmcountersep| and |\@thmcounter|.\footnote{These
%    two macros can be defined by the document style.  Their default
%    values produce a `\texttt{.}' as separation and an arabic
%    representation of the number.} However, not everything should be
%    expanded. For example, it saves space to use |\thesection|
%    instead of its---at times---lengthy expansion.
%    \begin{macrocode}
      {\expandafter \noexpand \csname the#3\endcsname
       \@thmcountersep \@thmcounter{#1}}%
%    \end{macrocode}
%    Thus with the defaults of \LaTeX{}, |\theLem| would be
%    replaced by the command sequence |\thesection.\arabic{Lem}|.
%
%    We will now look at the definition of the macro which is executed
%    at the beginning of the actual environment (in our example this
%    macro is |\Lem|). It should be noted that we use an
%    ``|\expandafter| trick'' to expand only certain parts of the
%    replacement text at the time of the definition.
%    \begin{macrocode}
    \def\@tempa{\global\@namedef{#1}}%
    \expandafter \@tempa \expandafter{%
%    \end{macrocode}
%    First, the macro that contains the current definitions of
%    |\@begintheorem| and |\@opargtheorem| should be called
%    up.  The name of this macro---as is already known---has the form
%    |\th@|\meta{theorem\ style}; therefore, it must be called by
%    \begin{macrocode}
      \csname th@\the \theorem@style
%    \end{macrocode}
%    In addition the default theorem font should be changeable, i.e.\
%    we have to insert the contents of |\theorem@bodyfont|. For
%    that reason, we expand even further, beyond |\endcsname|,
%    and thus insert the contents of the \textsf{token} register
%    |\theorem@bodyfont| in the replacement text.
%    \begin{macrocode}
            \expandafter \endcsname \the \theorem@bodyfont
%    \end{macrocode}
%    Now it is time to call the macro |\@thm| which takes over
%    the further processing.  It has two arguments: the current
%    counter name (in our example, \texttt{Lem}), and the text of the
%    label (in our example, \texttt{Lemma}).
%    \begin{macrocode}
     \@thm{#1}{#2}}%
%    \end{macrocode}
%    With this, the `sub-definition' is complete.  The macro
%    |\@endtheorem| ends a theorem environment and is, so far, nothing
%    but an |\endtrivlist|.  (Hence it is defined globally, and not
%    within the theorem styles.\footnote{This has to be changed as
%    soon as theorem styles that change \texttt{\bslash @endtheorem}
%    exist.  In such a case, all existing styles must be changed as
%    well since they will have to reset the macro.}) Therefore, we can
%    set it equivalent to the macro that ends the theorem set (in our
%    example, |\endLem|).  However, if some day theorem styles exist
%    that do change |\@endtheorem|, we would have to use the
%    commented-out line instead.
%    \begin{macrocode}
    \global \expandafter \let \csname end#1\endcsname \@endtheorem
%   \global\@namedef{end#1}{\@endtheorem}%
%    \end{macrocode}
%    With these commands all the required definitions are employed,
%    unless the test |\@ifdefinable| has failed.  Therefore, we
%    end the second argument of this macro and with it the definition
%    of |\@xnthm|.
%    \begin{macrocode}
   }}
%    \end{macrocode}
% \end{macro}
%
% \begin{macro}{\@ynthm}
%    The definition of |\@ynthm| is completely analogous. In
%    this case the new counter that is defined is not reset within
%    another counter; thus the definition of |\the...|
%    is simplified:
%    \begin{macrocode}
\gdef\@ynthm#1#2{\expandafter\@ifdefinable\csname #1\endcsname
   {\@definecounter{#1}%
    \expandafter\xdef\csname the#1\endcsname{\@thmcounter{#1}}%
%    \end{macrocode}
%    The rest of the definition corresponds literally to that of
%    |\@xnthm|:
%    \begin{macrocode}
    \def\@tempa{\global\@namedef{#1}}\expandafter \@tempa
     \expandafter{\csname th@\the \theorem@style \expandafter
     \endcsname \the\theorem@bodyfont \@thm{#1}{#2}}%
    \global \expandafter \let \csname end#1\endcsname \@endtheorem}}
%    \end{macrocode}
% \end{macro}
%
% \begin{macro}{\@othm}
%    The definition of |\@othm| does not contain anything new.
%    \begin{macrocode}
\gdef\@othm#1[#2]#3{%
%    \end{macrocode}
% \changes{v2.2b}{95/10/07}{Build in check for counter defined}
%    We do not define a new counter but instead use one that has
%    already been defined.  Thus the only definition we need is that
%    of this pseudo-counter (i.e.\ |\the|\meta{env.\ name}).
%    First we check if |#2| corresponds to a known counter name.
%    \begin{macrocode}
  \expandafter\ifx\csname c@#2\endcsname\relax
   \@nocounterr{#2}%
  \else
   \expandafter\@ifdefinable\csname #1\endcsname
   {\expandafter \xdef \csname the#1\endcsname
     {\expandafter \noexpand \csname the#2\endcsname}%
%    \end{macrocode}
%    All other parts of the definition can be adopted from
%    |\@xnthm|. We have to remember, though, that in this case
%    the name of the current counter and the theorem label have moved
%    to the second and third arguments.
%    \begin{macrocode}
    \def\@tempa{\global\@namedef{#1}}\expandafter \@tempa
     \expandafter{\csname th@\the \theorem@style \expandafter
     \endcsname \the\theorem@bodyfont \@thm{#2}{#3}}%
    \global \expandafter \let \csname end#1\endcsname \@endtheorem}%
  \fi}
%    \end{macrocode}
% \end{macro}
%
% \subsection{Macros that are employed in a {\envfont theorem}
%    environment}
%
% \begin{macro}{\@thm}
%    The macro |\@thm| has to increase the current counter.  Then,
%    depending on whether the environment has (or does not have) an
%    optional argument, it has to branch into either |\@begintheorem|
%    or |\@opargtheorem|.
%    \begin{macrocode}
\gdef\@thm#1#2{\refstepcounter{#1}%
%    \end{macrocode}
%    Now we start a {\envfont trivlist} environment, and give
%    |\@topsep| and |\@topsepadd| the values of the \textsf{skip}
%    registers |\theorempreskipamount| and |\theorempostskipamount|.
%    The value in |\@topsep| is the vertical space that is inserted by
%    the first (and only) |\item| in our |\trivlist| whilst
%    |\@topsepadd| is inserted by |\@endparenv| at the end of that
%    {\envfont trivlist} environment.  By using these registers, we
%    obtain the desired space around a {\envfont theorem} environment.
%    \begin{macrocode}
   \trivlist
   \@topsep \theorempreskipamount               % used by first \item
   \@topsepadd \theorempostskipamount           % used by \@endparenv
%    \end{macrocode}
%    Now we have to test whether an optional argument has been given.
% \changes{v2.0d}{89/07/18}{Forgotten percent added.}
%    \begin{macrocode}
   \@ifnextchar [%
%    \end{macrocode}
%    If there is an optional argument, we will call |\@ythm|, and
%    move the arguments read back into the input stream.
%    \begin{macrocode}
   {\@ythm{#1}{#2}}%
%    \end{macrocode}
%    If not, we call |\@begintheorem|. Its first argument is the
%    name of the theorem set (hence the second argument of
%    |\@thm|). Its second argument is the macro that produces the
%    current number.
%    \begin{macrocode}
   {\@begintheorem{#2}{\csname the#1\endcsname}\ignorespaces}}
%    \end{macrocode}
% \end{macro}
%
% \begin{macro}{\@xthm}
% \begin{macro}{\@ythm}
%    Both these macros were originally called by |\@thm|.  We do
%    not need |\@xthm| anymore, hence we reset it to
%    |\relax|. The definition of |\@ythm| has not changed at
%    all from its definition in \LaTeX. In order to make the macros
%    easier to understand, we will
%    nevertheless present it (commented out).
%    \begin{macrocode}
\global\let\@xthm\relax
% \def\@ythm#1#2[#3]{\@opargbegintheorem{#2}{\csname
%           the#1\endcsname}{#3}\ignorespaces}
%</package>
%    \end{macrocode}
%    The primitive |\ignorespaces| in |\@ythm| and
%    |\@thm| is needed to remove the spaces between the
%    |\begin{...}| and the actual text.
% \end{macro}
% \end{macro}
%
%
% \subsection{Definition of the theorem substyles}
%
% As already pointed out, the theorem substyles, defined below, are
% only loaded when necessary.  Note that all these substyles, except
% \texttt{plain}, have |\slshape| as the default body font.
%
% \subsubsection{The \texttt{plain} style}
%
% \label{dok:th@plain} As the following macros use |@|, we have
% to locally set the |\catcode| of this symbol to ``letter''.
% This happens within a group, so that we do not have to worry about
% which |\catcode| that symbol had before.
%    \begin{macrocode}
%<*thp>
\begingroup \makeatletter
%    \end{macrocode}
% Since we are now within a group, we must make all definitions
% globally.  First we make sure that \texttt{theorem.sty} is loaded.
% This will allow us to use this file as a document style option
% without having to call \texttt{theorem} itself as an option.  At the
% same time, we assure that at least version 2 is loaded, since
% |\theorem@style| was not defined in earlier versions.
%    \begin{macrocode}
\@ifundefined{theorem@style}{% \iffalse meta-comment
%
% Copyright (C) 1993-2024
%
% The LaTeX Project and any individual authors listed elsewhere
% in this file.
%
% This file is part of the Standard LaTeX `Tools Bundle'.
% -------------------------------------------------------
%
% It may be distributed and/or modified under the
% conditions of the LaTeX Project Public License, either version 1.3c
% of this license or (at your option) any later version.
% The latest version of this license is in
%    https://www.latex-project.org/lppl.txt
% and version 1.3c or later is part of all distributions of LaTeX
% version 2005/12/01 or later.
%
% The list of all files belonging to the LaTeX `Tools Bundle' is
% given in the file `manifest.txt'.
%
% \fi
%
%%
%
%
\def\FMithmInfo{2023/07/05 v2.2c Theorem extension package (FMi)}
%
% \ProvidesFile{theorem.dtx}[\FMithmInfo]
%\iffalse   % this is a METACOMMENT !
%%
%% Package `theorem' to use with LaTeX2e
%% Copyright (C) 1989-1995 Frank Mittelbach, all rights reserved.
%<+package>\@ifundefined{theorem@style}{}{\endinput}
%<+package>\NeedsTeXFormat{LaTeX2e}[1995/06/01]
%<+package>\ProvidesPackage{theorem}
%<+package>         [\FMithmInfo]
%
% \fi
%
% \changes{v1.0}{87/05/07}{First draft.}
% \changes{v1.0a}{87/05/07}{`thistheoremstyle' added.}
% \changes{v1.1a}{87/07/29}{New concept: redefine the internal
%                          \protect\LaTeX{} commands `@xnthm' etc.}
% \changes{v1.1b}{87/07/31}{`theoremfont' used to specify font.}
% \changes{v1.2a}{88/01/08}{`@endtheorem' now globally defined.}
% \changes{v1.2b}{88/01/11}{`newtheorem' only allowed in the preamble.}
% \changes{v1.3a}{88/02/14}{`theoremfont' now a toks register.}
%
% \changes{v2.0a}{88/11/16}{New implementation and new documentation.}
% \changes{v2.0a}{88/11/16}{No layout changes but internal modifications
%                           old substyles won't work.}
% \changes{v2.0b}{89/01/30}{`global' used for all definitions.}
% \changes{v2.0c}{89/06/15}{Documentation translated to English.
%                            Thanks to LLL and Nina}
% \changes{v2.1a}{90/02/15}{Support for new font selection.}
% \changes{v2.2a}{94/02/03}{Cleanup for LaTeX2e}
%
% \DoNotIndex{\@endparenv,\@ifdefinable,\@ifnextchar,\@ifundefined}
% \DoNotIndex{\@namedef,\@nameuse,\@spaces,\@tempa,\@warning}
% \DoNotIndex{\ }
% \DoNotIndex{\begingroup,\bfseries}
% \DoNotIndex{\csname}
% \DoNotIndex{\def,\do,\docdate}
% \DoNotIndex{\endcsname,\endgroup,\endinput,\endtrivlist,\expandafter}
% \DoNotIndex{\filedate,\fileversion}
% \DoNotIndex{\gdef,\global}
% \DoNotIndex{\hbox,\hskip}
% \DoNotIndex{\ignorespaces,\input,\itshape,\item}
% \DoNotIndex{\labelsep,\let,\llap}
% \DoNotIndex{\makeatletter}
% \DoNotIndex{\noexpand}
% \DoNotIndex{\refstepcounter,\relax,\rlap}
% \DoNotIndex{\setlength,\slshape,\space,\string,\strut}
% \DoNotIndex{\the,\trivlist,\typeout}
% \DoNotIndex{\vbox}
% \DoNotIndex{\xdef}
%
% \def\deflabel#1{\ttfamily #1\hfill}
% \def\deflist#1{\begin{list}{}{\settowidth\labelwidth{\ttfamily #1}%
%                               \setlength\leftmargin\labelwidth
%                               \addtolength\leftmargin\labelsep
%                               \let\makelabel\deflabel}}
% \def\enddeflist{\end{list}}
% \def\envfont{\normalfont\ttfamily}
%
% \MakeShortVerb{\|}
%
% \newcommand{\TUB}{{\sl TUGboat\/}}
%
% \GetFileInfo{theorem.dtx}
% \title{An Extension of the
%        \LaTeX{} theorem environment\thanks{This file
%        has version number \fileversion, last
%        revised \filedate.}}
% \author{Frank Mittelbach}
%
% \MaintainedByLaTeXTeam{tools}
% \maketitle
%
% \begin{abstract}
%    The macros described in this paper yield an extension of the
%    \LaTeX{} theorem mechanism. It is designed to satisfy the
%    different requirements of various journals. Thus, the
%    layout of the ``theorems'' can be manipulated by determining a
%    ``style''. This article describes not only the use, but
%    also the definition, of the necessary macros.
% \end{abstract}
%
% \section*{Preface to version 2.2}
%
% For \LaTeXe{} this package did not need any fundamental changes. I
% only modified the messages generated so that theorem layout styles
% will show up with the |\listfiles| command and cleaned the section
% on the New Font Selection Scheme since this is now included in
% \LaTeX.
%
% \section*{Preface to version 2.1}
%
% This version is identical to 2.0g described in \TUB\ 10\#3 except
% for some internal defaults which are now set depending on the used
% font selection scheme.
%
% This was done to avoid unpleasant surprises if the new font selection
% scheme is in force. For further details see
% section~\ref{sec:newfonts} and \cite{bk:GMS94}.
%
% \section{Introduction}
%
% For our purposes here, ``theorems'' are labelled enunciations,
% often set off from the main text by extra space and a font change.
% Theorems, corollaries, conjectures, definitions, and remarks
% are all instances of ``theorems''.  The ``header'' of these structures
% is composed of a label (such as \textsc{Theorem} or \textsc{Remark})
% and a number which serializes an item in the sequence of items
% with the same label.
%
% Shortly after the introduction of \LaTeX{} at the Fachbereich
% Mathematik in Mainz, the desire to manipulate the layout of
% ``theorems'' arose.  In Mainz, the following two conventions came
% into general use:
% \begin{enumerate}
%   \item The number of the theorem is shown in the margin.
%   \item There is a line break at
%         the end of the theorem header.
% \end{enumerate}
%
% Additionally, some journals require different formats which depend on
% the ``sort of theorem'': e.g.\ often remarks and definitions are set
% in |\upshape|, while |\itshape| is employed for main theorems.
%
% Confronted with these requirements, a theorem environment was
% developed in Mainz which allows separate determination of the
% layout of the ``theorems sets'', comparable to |\pagestyle|.
%
%
% \section{The user interface}
% \subsection{Defining new theorem sets}
%
% \DescribeMacro\newtheorem
% As in the original \LaTeX{} version, the command |\newtheorem|
% defines a new ``theorem set'' or ``theorem-like structure''.
% Two required arguments name the new environment and give the text
% to be typeset with each instance of the new ``set'', while
% an optional argument determines how
% the ``set'' is enumerated:
% \begin{description}
%    \item[\ttfamily \bslash newtheorem\{foo\}\{bar\}]
%       The theorem set {\envfont foo} (whose name is \texttt{bar})
%       uses its own counter.
%    \item[\ttfamily \bslash newtheorem\{foo2\}{[foo]}\{bar2\}]
%       The theorem set {\envfont foo2} (printed name \texttt{bar2})
%       uses the same counter as the theorem set \texttt{foo}.
%    \item[\ttfamily \bslash newtheorem\{foo3\}\{bar3\}{[section]}]
%       The theorem set {\envfont foo3} (printed name \texttt{bar3}) is
%       enumerated within the counter \texttt{section}, i.e.\ with every
%       new |\section| the enumeration begins again with 1, and
%       the enumeration is composed from the section-number and the
%       theorem counter itself.
% \end{description}
%
% \DescribeMacro\theoremstyle
% Additionally, the command |\theoremstyle| can define the layout
% of various, or all, theorem sets.  It should be noted that any
% theorem set defined by |\newtheorem| is typeset in the
% |\theoremstyle| that is current at the time of the definition.
% Thus, the following
% \begin{verbatim}
%    \theoremstyle{break}        \newtheorem{Cor}{Corollary}
%    \theoremstyle{plain}        \newtheorem{Exa}{Example}[section]
%\end{verbatim}                             ^^A No blank at line start
% leads to the result that the set {\envfont Cor} is formatted in the
% style \texttt{break}, while the set {\envfont Exa} and all the
% following ones are formatted in the style \texttt{plain}, unless
% another |\theoremstyle| follows.  Since the definitions installed by
% |\newtheorem| are global, one also can limit |\theoremstyle| locally
% by grouping braces.
%
% \DescribeMacro\theorembodyfont
% The choice of the font for the theorem body is completely
% independent of the chosen |\theoremstyle|; this has proven to
% be very advantageous.  For example,
% \begin{verbatim}
%   {\theorembodyfont{\upshape}          \newtheorem{Rem}{Remark}}
%\end{verbatim}
% defines a theorem set {\envfont Rem}, which will be set in
% |\upshape| in the current layout (which in our example is
% \texttt{plain}).  As with |\theoremstyle|, the |\theorembodyfont|
% chosen is that current at the time of |\newtheorem|.  If
% |\theorembodyfont| is not specified or one defines
% |\theorembodyfont{}|, then the font used will be that defined by the
% |\theoremstyle|.
%
% \DescribeMacro\theoremheaderfont It is also possible to customize
% the font used for the theorem headers.  This is, however, a global
% declaration, and therefore there should be at most one
% |\theoremheaderfont| declaration in the preamble.\footnote{If it is
% actually necessary to have different header fonts, one has to define
% new theorem styles (substituting the desired font) or specify the
% information directly in the \texttt{\bslash newtheorem} declaration
% (the unclean variant).}
%
% \DescribeMacro\theorempreskipamount
% \DescribeMacro\theorempostskipamount Two additional parameters
% affect the vertical space around the theorem environments:
% |\theorempreskipamount| and |\theorempostskipamount| define,
% respectively, the spacing before and after such an environment.
% These parameters apply for all theorem sets and can be manipulated
% with the ordinary length macros.  They are rubber lengths
% (`\textsf{skips}'), and therefore can contain \texttt{plus} and
% \texttt{minus} parts.
%
% Since the definition of theorem sets should---most sensibly---be
% placed in the preamble, we only allow installation there.
% It is therefore possible to release the memory used here after
% |\begin{document}|, in order to make room for other
% applications.
%
%
% \subsection{Existing theorem styles}
%
% The following theorem styles exist to date:
% \begin{deflist}{marginbreak}
%    \item[plain]
%       This theorem style emulates the original \LaTeX{} definition,
%       except that additionally the parameters
%       |\theorem...skipamount| are used.
%    \item[break]
%       In this style, the theorem header is followed by a line
%       break.
%    \item[marginbreak]
%       The theorem number is set in the margin, and there is a line
%       break as in \texttt{break}.
%    \item[changebreak]
%       Like \texttt{break}, but with header number and text
%       interchanged.
%    \item[change]
%       Header number and text are interchanged, without a line break.
%    \item[margin]
%       The number is set in the left margin, without a line break.
% \end{deflist}
% All styles (except \texttt{plain}) select |\slshape| as the default
% |\theorembodyfont|.
%
%
% \subsection{Examples}
%
% Given the above theorem sets {\envfont Cor}, {\envfont Exa}
% and {\envfont Rem},
% suppose that the preamble also contains the declarations:
% \begin{verbatim}
%    \theoremstyle{marginbreak}   \newtheorem{Lem}[Cor]{Lemma}
%    \theoremstyle{change}
%    \theorembodyfont{\itshape}        \newtheorem{Def}[Cor]{Definition}
%
%    \theoremheaderfont{\scshape}
%\end{verbatim}
% Then the following are some typical examples of the typeset output
% resulting from their use.
%
% \begin{Cor}
%    This is a sentence typeset in the theorem environment {\envfont
%    Cor}.
% \end{Cor}
%
% \begin{Exa}
%    This is a sentence typeset in the theorem environment {\envfont
%    Exa}.
% \end{Exa}
%
% \begin{Rem}
%    This is a sentence typeset in the theorem environment {\envfont
%    Rem}.
% \end{Rem}
%
% \begin{Lem}[Ben User]
%    This is a sentence typeset in the theorem environment {\envfont
%    Lem}.
% \end{Lem}
%
% \begin{Def}[Very impressive Definition]
%    This is a sentence typeset in the theorem environment {\envfont
%    Def}.
% \end{Def}
%
% The last two examples show the effect of the optional argument to
% a theorem environment (it is the text typeset in parentheses).
%
%
% \section{Special Considerations}
% \label{sec:newfonts}
%
% Theoremheader and body are implemented as a unit. This means that
% the |\theoremheaderfont|  will inherit characteristics of the
% |\theorembodyfont| in \LaTeXe. Thus, if for example
% |\theorembodyfont| is |\itshape| and |\theoremheaderfont|
% is |\bfseries| the font selected for the header will have the
% characteristics `bold extended italic'. If this is not desired one
% should set the |\theoremheaderfont| to something like
% \begin{verbatim}
%      \theoremheaderfont{\normalfont\bfseries}
%\end{verbatim}
% i.e.\ supplying all necessary font information explicitly.
%
%
%
% \section{Acknowledgements}
%
% The publication of this set of macros was only possible with the
% help of Christina Busse (translating the manuscript into English),
% Joachim Pense (playing the r\^ole of typist), Chris Rowley
% (looking everything over) and many others providing useful
% suggestions.
%
% \MaybeStop{%
%    \begin{thebibliography}{9}
%       \bibitem{bk:GMS94} \textsc{M.~Goossens}, \textsc{F.~Mittelbach}
%         and \textsc{A.~Samarin}.
%         \newblock The \LaTeX{} Companion.
%         \newblock
%         Addison-Wesley, Reading, Massachusetts, 1994.
%       \bibitem{latex:source} \textsc{Lamport, Leslie.} \newblock
%                               \texttt{latex.tex}, version 2.09, date
%                               Feb. 1990.
%    \end{thebibliography}
%    \PrintIndex
%    \PrintChanges}
%
%
% \section{The documentation driver file}
%
% The next bit of code contains the documentation driver file for
% \TeX{}, i.e., the file that will produce the documentation you are
% currently reading. It will be extracted from this file by the
% \texttt{docstrip} program. Since it is the first code in the file
% one can alternatively process this file directly with \LaTeXe{} to
% obtain the documentation.
%
%    \begin{macrocode}
%<*driver>
\documentclass{ltxdoc}

\usepackage{theorem}

%<+driver>% The next few lines define theorem sets which are used
%<+driver>% in the example section of the documentation.

\theoremstyle{break}       \newtheorem{Cor}{Corollary}
\theoremstyle{plain}       \newtheorem{Exa}{Example}[section]
{\theorembodyfont{\upshape}\newtheorem{Rem}{Remark}}
\theoremstyle{marginbreak} \newtheorem{Lem}[Cor]{Lemma}
\theoremstyle{change}
\theorembodyfont{\itshape} \newtheorem{Def}[Cor]{Definition}

\theoremheaderfont{\scshape}

\RecordChanges

\begin{document}
   \DocInput{theorem.dtx}
\end{document}
%</driver>
%    \end{macrocode}
%
% \section{Definition of the Macros}
%
% If the file has been loaded before, we abort immediately. If not the
% package announces itself (this is actually done at the very top of
% the file---the way it is done isn't good style so don't copy it).
% \changes{v2.0e}{89/07/19}{Spaces removed from `typeout'.}
% \changes{v2.2b}{95/11/19}{Announce moved to top of file}
%    \begin{macrocode}
%<*package>
%\@ifundefined{theorem@style}{}{\endinput}
%\def\FMithmInfo{1995/11/19 v2.2b Theorem extension package (FMi)}
%\ProvidesPackage{theorem}[\FMithmInfo]
%    \end{macrocode}
%
%
% \subsection{Definition of theorem styles and fonts}
%
% All the definitions in this file are done globally to allow
% inputting this file inside a group.
%
% \begin{macro}{\theoremstyle}
%    Before a theorem style can be installed, the chosen style must be
%    known. For that reason, we must test to see that
%    |\th@|\meta{style} is known or, more precisely, that it is
%    different from |\relax|. If the style is not known then
%    |\th@plain| is used.
%    \begin{macrocode}
\gdef\theoremstyle#1{%
   \@ifundefined{th@#1}{\@warning
          {Unknown theoremstyle `#1'. Using `plain'}%
          \theorem@style{plain}}%
%    \end{macrocode}
%    We save the theorem style to be used in the token register
%    |\theorem@style|.
%    \begin{macrocode}
      {\theorem@style{#1}}%
%    \end{macrocode}
%    Now we ``evaluate'' the theorem style: this means, we call the
%    macro |\th@|\meta{style} which will activate the relevant
%    definitions which are contained in a separate file.  This is done
%    in a group to suppress changes to the current font.  This could
%    otherwise pose problems together with the new font selection
%    scheme\footnote{When I printed the original article using the new
%    font selection scheme I ended with a document with slanted
%    typefaces (text headings and all) simply because one of the
%    theorem styles used \texttt{\bslash sl} at toplevel.} if the
%    |\th@|\meta{style} is evaluated a second time.
% \changes{v2.1a}{90/02/15}{Group added around `th@..}
% \changes{v2.1a}{90/02/15}{Reseting of `..@begintheorem removed.}
%    \begin{macrocode}
      \begingroup
        \csname th@\the\theorem@style \endcsname
      \endgroup}
%    \end{macrocode}
% \end{macro}
%
%
% \begin{macro}{\@begintheorem}
% \changes{v2.1a}{90/02/15}{Relaxed at toplevel.}
% \begin{macro}{\@opargbegintheorem}
% \changes{v2.1a}{90/02/15}{Relaxed at toplevel.}
%    We reset |\@begintheorem| and |\@opargbegintheorem| to
%    |\relax| since these commands are no longer necessary at
%    toplevel. This will save a few tokens.
%    \begin{macrocode}
\global\let\@begintheorem\relax
\global\let\@opargbegintheorem\relax
%    \end{macrocode}
% \end{macro}
% \end{macro}
%
% \begin{macro}{\theorem@style}
%    Obviously the \textsf{token} register used above has to be
%    allocated. To assure the utmost compatibility with the
%    original \LaTeX{} definition, we set the default theorem style to
%    \texttt{plain}, which implements the usual \LaTeX{} convention.
% \changes{v2.0f}{89/07/20}{Unnecessary `global' removed.}
%    \begin{macrocode}
\newtoks\theorem@style
\global\theorem@style{plain}
%    \end{macrocode}
% \end{macro}
%
% \begin{macro}{\theorembodyfont}
% \begin{macro}{\theorem@bodyfont}
%    If the |\theorembodyfont| is set by the user then it should not
%    interact with the default font set in the theorem style. When the
%    new font selection is in force this may happen if, for example,
%    the default is |\itshape| and the new |\theorembodyfont| is
%    |\sffamily|. So we add a |\reset@font| command in front of the
%    user definition.
% \changes{v2.0f}{89/07/20}{Unnecessary `global' removed.}
% \changes{v2.0g}{89/09/19}{`theoremfont to `theorembodyfont.}
% \changes{v2.1c}{92/06/22}{`reset@font added and logic changed.}
% \changes{v2.1d}{92/10/28}{Reset to empty if arg is empty}
%    \begin{macrocode}
\gdef\theorembodyfont#1{%
%    \end{macrocode}
%    We check if the argument supplied is empty and if so put nothing
%    into the |\theorem@bodyfont| token register to allow for
%    |\theorembodyfont{}| as a mean of using the default of the
%    current |\theoremstyle|.
%    \begin{macrocode}
   \def\@tempa{#1}%
   \ifx\@tempa\@empty
    \theorem@bodyfont{}%
   \else
    \theorem@bodyfont{\reset@font#1}%
   \fi
   }
\newtoks\theorem@bodyfont
\global\theorem@bodyfont{}
%    \end{macrocode}
% \end{macro}
% \end{macro}
%
%
% \begin{macro}{\theoremheaderfont}
% \changes{v2.0g}{89/09/19}{Macro added.}
%    The font for the theorem headers is handled differently because
%    this definition applies to all theorem styles.
%    \begin{macrocode}
\gdef\theoremheaderfont#1{\gdef\theorem@headerfont{#1}%
%    \end{macrocode}
%    After using the macro once it is redefined to produce an error
%    message.
%    \begin{macrocode}
       \gdef\theoremheaderfont##1{%
        \typeout{\string\theoremheaderfont\space should be used
                 only once.}}}
%    \end{macrocode}
% \end{macro}
%
% \begin{macro}{\theorem@headerfont}
% \changes{v2.0g}{89/09/19}{Macro added.}
%    To set the |\theorem@headerfont| default we first test if
%    the new fontselection scheme is in force.
% \changes{v2.1a}{90/02/15}{Support for new font selection.}
%    \begin{macrocode}
\ifx\upshape\undefined
%    \end{macrocode}
%    If not we define it to expand into |\bfseries|. We don't use
%    |\let| just in case a following style option redefines this
%    macro.
%    \begin{macrocode}
\gdef\theorem@headerfont{\bfseries}
%    \end{macrocode}
%    Otherwise we reset the current shape before calling |\bfseries|.
%    \begin{macrocode}
\else \gdef\theorem@headerfont{\normalfont\bfseries}\fi
%    \end{macrocode}
% \end{macro}
%
% \begin{macro}{\th@plain}
% \begin{macro}{\th@break}
% \begin{macro}{\th@marginbreak}
% \begin{macro}{\th@changebreak}
% \begin{macro}{\th@change}
% \begin{macro}{\th@margin}
%    The different styles are defined in macros such as
%    |\th@plain|.  Since memory space is precious in
%    ``non-Big-versions'', we have to avoid offering too many unused
%    definitions.  Therefore we define these styles in separate files
%    that can be loaded on demand.  Thus the commands themselves only
%    load these files. We use |\@input@|, a \LaTeXe{} internal command
%    that ensures that the file will be listed with |\listfiles|.
% \changes{v2.2a}{94/02/02}{Use `@input@ to load theorem layout files}
%    \begin{macrocode}
\gdef\th@plain{\@input@{thp.sty}}
\gdef\th@break{\@input@{thb.sty}}
\gdef\th@marginbreak{\@input@{thmb.sty}}
\gdef\th@changebreak{\@input@{thcb.sty}}
\gdef\th@change{\@input@{thc.sty}}
\gdef\th@margin{\@input@{thm.sty}}
%    \end{macrocode}
%    This list will be expanded when new styles become
%    available.  For testing,  just append new theorem substyles
%    as document options.
% \end{macro}
% \end{macro}
% \end{macro}
% \end{macro}
% \end{macro}
% \end{macro}
%
%
%
% \subsection{Definition of a new theorem set}
%
% As already pointed out, a new theorem environment can be defined in
% three different ways:
% \begin{verbatim}
%    \newtheorem{Lem}{Lemma}
%    \newtheorem{Lem}{Lemma}[section]
%    \newtheorem{Lem}[Theorem]{Lemma}
%\end{verbatim}
% The function of the macro |\newtheorem| is to recognize these cases
% and then to branch into one of the three macros |\@ynthm|, |\@xnthm|
% or |\@othm|.  This mechanism is adopted unchanged from
% \cite{latex:source}; the essential point here is that, for example,
% in the second case, the arguments \texttt{Lem}, \texttt{Lemma} and
% \texttt{section} are passed over to the macro |\@xnthm|.
%
% We inspect this case first because the others present fewer problems,
% and thus are easily derived from this one.
%
% \begin{macro}{\@xnthm}
%    For our example arguments, the macro |\@xnthm| must fulfill
%    the following:
%    \begin{itemize}
%       \item
%          Define a new \LaTeX-counter `Lem'
%       \item
%          reset this counter within a |\section|
%       \item
%          define the macro |\theLem|
%       \item
%          define the environment macros |\Lem| and
%          |\endLem| using the current
%          |\theoremstyle| and |\theorem@bodyfont|.
%    \end{itemize}
%    Obviously, all this should happen only if the first argument
%    of |\@xnthm| (i.e.\ \texttt{Lem} in our example) is chosen so
%    as not to conflict with any previously defined commands or
%    environments.  This test is performed by the \LaTeX{} macro
%    |\@ifdefinable|.
%    \begin{macrocode}
\gdef\@xnthm#1#2[#3]{\expandafter\@ifdefinable\csname #1\endcsname
%    \end{macrocode}
%    Therefore, the first argument of |\@ifdefinable| is the
%    expansion (in the example, |\Lem|) of
%    |\csname||#1\endcsname|.  The second argument is
%    executed only if the test has been completed successfully.
%    \begin{macrocode}
   {%
%    \end{macrocode}
%    Now we define the new counter. The names of the \LaTeX{}
%    macros employed should speak for themselves:
% \changes{v2.2b}{1995/10/07}{Use \cs{@newctr}}
% \changes{v2.2c}{1995/11/23}{Use \cs{@newctr} properly.}
%    \begin{macrocode}
    \@definecounter{#1}\@newctr{#1}[#3]%
%    \end{macrocode}
%    Using |\@newctr| will give a proper error message if the counter
%    in |#3| is not defined.
%    In defining `|\theLem|' we must generate the desired macro
%    name by use of |\expandafter| and |\csname|.
%    \begin{macrocode}
    \expandafter\xdef\csname the#1\endcsname
%    \end{macrocode}
%    An |\xdef| is used in order to make the definition global,
%    and to ensure that it contains the replacement texts of
%    |\@thmcountersep| and |\@thmcounter|.\footnote{These
%    two macros can be defined by the document style.  Their default
%    values produce a `\texttt{.}' as separation and an arabic
%    representation of the number.} However, not everything should be
%    expanded. For example, it saves space to use |\thesection|
%    instead of its---at times---lengthy expansion.
%    \begin{macrocode}
      {\expandafter \noexpand \csname the#3\endcsname
       \@thmcountersep \@thmcounter{#1}}%
%    \end{macrocode}
%    Thus with the defaults of \LaTeX{}, |\theLem| would be
%    replaced by the command sequence |\thesection.\arabic{Lem}|.
%
%    We will now look at the definition of the macro which is executed
%    at the beginning of the actual environment (in our example this
%    macro is |\Lem|). It should be noted that we use an
%    ``|\expandafter| trick'' to expand only certain parts of the
%    replacement text at the time of the definition.
%    \begin{macrocode}
    \def\@tempa{\global\@namedef{#1}}%
    \expandafter \@tempa \expandafter{%
%    \end{macrocode}
%    First, the macro that contains the current definitions of
%    |\@begintheorem| and |\@opargtheorem| should be called
%    up.  The name of this macro---as is already known---has the form
%    |\th@|\meta{theorem\ style}; therefore, it must be called by
%    \begin{macrocode}
      \csname th@\the \theorem@style
%    \end{macrocode}
%    In addition the default theorem font should be changeable, i.e.\
%    we have to insert the contents of |\theorem@bodyfont|. For
%    that reason, we expand even further, beyond |\endcsname|,
%    and thus insert the contents of the \textsf{token} register
%    |\theorem@bodyfont| in the replacement text.
%    \begin{macrocode}
            \expandafter \endcsname \the \theorem@bodyfont
%    \end{macrocode}
%    Now it is time to call the macro |\@thm| which takes over
%    the further processing.  It has two arguments: the current
%    counter name (in our example, \texttt{Lem}), and the text of the
%    label (in our example, \texttt{Lemma}).
%    \begin{macrocode}
     \@thm{#1}{#2}}%
%    \end{macrocode}
%    With this, the `sub-definition' is complete.  The macro
%    |\@endtheorem| ends a theorem environment and is, so far, nothing
%    but an |\endtrivlist|.  (Hence it is defined globally, and not
%    within the theorem styles.\footnote{This has to be changed as
%    soon as theorem styles that change \texttt{\bslash @endtheorem}
%    exist.  In such a case, all existing styles must be changed as
%    well since they will have to reset the macro.}) Therefore, we can
%    set it equivalent to the macro that ends the theorem set (in our
%    example, |\endLem|).  However, if some day theorem styles exist
%    that do change |\@endtheorem|, we would have to use the
%    commented-out line instead.
%    \begin{macrocode}
    \global \expandafter \let \csname end#1\endcsname \@endtheorem
%   \global\@namedef{end#1}{\@endtheorem}%
%    \end{macrocode}
%    With these commands all the required definitions are employed,
%    unless the test |\@ifdefinable| has failed.  Therefore, we
%    end the second argument of this macro and with it the definition
%    of |\@xnthm|.
%    \begin{macrocode}
   }}
%    \end{macrocode}
% \end{macro}
%
% \begin{macro}{\@ynthm}
%    The definition of |\@ynthm| is completely analogous. In
%    this case the new counter that is defined is not reset within
%    another counter; thus the definition of |\the...|
%    is simplified:
%    \begin{macrocode}
\gdef\@ynthm#1#2{\expandafter\@ifdefinable\csname #1\endcsname
   {\@definecounter{#1}%
    \expandafter\xdef\csname the#1\endcsname{\@thmcounter{#1}}%
%    \end{macrocode}
%    The rest of the definition corresponds literally to that of
%    |\@xnthm|:
%    \begin{macrocode}
    \def\@tempa{\global\@namedef{#1}}\expandafter \@tempa
     \expandafter{\csname th@\the \theorem@style \expandafter
     \endcsname \the\theorem@bodyfont \@thm{#1}{#2}}%
    \global \expandafter \let \csname end#1\endcsname \@endtheorem}}
%    \end{macrocode}
% \end{macro}
%
% \begin{macro}{\@othm}
%    The definition of |\@othm| does not contain anything new.
%    \begin{macrocode}
\gdef\@othm#1[#2]#3{%
%    \end{macrocode}
% \changes{v2.2b}{95/10/07}{Build in check for counter defined}
%    We do not define a new counter but instead use one that has
%    already been defined.  Thus the only definition we need is that
%    of this pseudo-counter (i.e.\ |\the|\meta{env.\ name}).
%    First we check if |#2| corresponds to a known counter name.
%    \begin{macrocode}
  \expandafter\ifx\csname c@#2\endcsname\relax
   \@nocounterr{#2}%
  \else
   \expandafter\@ifdefinable\csname #1\endcsname
   {\expandafter \xdef \csname the#1\endcsname
     {\expandafter \noexpand \csname the#2\endcsname}%
%    \end{macrocode}
%    All other parts of the definition can be adopted from
%    |\@xnthm|. We have to remember, though, that in this case
%    the name of the current counter and the theorem label have moved
%    to the second and third arguments.
%    \begin{macrocode}
    \def\@tempa{\global\@namedef{#1}}\expandafter \@tempa
     \expandafter{\csname th@\the \theorem@style \expandafter
     \endcsname \the\theorem@bodyfont \@thm{#2}{#3}}%
    \global \expandafter \let \csname end#1\endcsname \@endtheorem}%
  \fi}
%    \end{macrocode}
% \end{macro}
%
% \subsection{Macros that are employed in a {\envfont theorem}
%    environment}
%
% \begin{macro}{\@thm}
%    The macro |\@thm| has to increase the current counter.  Then,
%    depending on whether the environment has (or does not have) an
%    optional argument, it has to branch into either |\@begintheorem|
%    or |\@opargtheorem|.
%    \begin{macrocode}
\gdef\@thm#1#2{\refstepcounter{#1}%
%    \end{macrocode}
%    Now we start a {\envfont trivlist} environment, and give
%    |\@topsep| and |\@topsepadd| the values of the \textsf{skip}
%    registers |\theorempreskipamount| and |\theorempostskipamount|.
%    The value in |\@topsep| is the vertical space that is inserted by
%    the first (and only) |\item| in our |\trivlist| whilst
%    |\@topsepadd| is inserted by |\@endparenv| at the end of that
%    {\envfont trivlist} environment.  By using these registers, we
%    obtain the desired space around a {\envfont theorem} environment.
%    \begin{macrocode}
   \trivlist
   \@topsep \theorempreskipamount               % used by first \item
   \@topsepadd \theorempostskipamount           % used by \@endparenv
%    \end{macrocode}
%    Now we have to test whether an optional argument has been given.
% \changes{v2.0d}{89/07/18}{Forgotten percent added.}
%    \begin{macrocode}
   \@ifnextchar [%
%    \end{macrocode}
%    If there is an optional argument, we will call |\@ythm|, and
%    move the arguments read back into the input stream.
%    \begin{macrocode}
   {\@ythm{#1}{#2}}%
%    \end{macrocode}
%    If not, we call |\@begintheorem|. Its first argument is the
%    name of the theorem set (hence the second argument of
%    |\@thm|). Its second argument is the macro that produces the
%    current number.
%    \begin{macrocode}
   {\@begintheorem{#2}{\csname the#1\endcsname}\ignorespaces}}
%    \end{macrocode}
% \end{macro}
%
% \begin{macro}{\@xthm}
% \begin{macro}{\@ythm}
%    Both these macros were originally called by |\@thm|.  We do
%    not need |\@xthm| anymore, hence we reset it to
%    |\relax|. The definition of |\@ythm| has not changed at
%    all from its definition in \LaTeX. In order to make the macros
%    easier to understand, we will
%    nevertheless present it (commented out).
%    \begin{macrocode}
\global\let\@xthm\relax
% \def\@ythm#1#2[#3]{\@opargbegintheorem{#2}{\csname
%           the#1\endcsname}{#3}\ignorespaces}
%</package>
%    \end{macrocode}
%    The primitive |\ignorespaces| in |\@ythm| and
%    |\@thm| is needed to remove the spaces between the
%    |\begin{...}| and the actual text.
% \end{macro}
% \end{macro}
%
%
% \subsection{Definition of the theorem substyles}
%
% As already pointed out, the theorem substyles, defined below, are
% only loaded when necessary.  Note that all these substyles, except
% \texttt{plain}, have |\slshape| as the default body font.
%
% \subsubsection{The \texttt{plain} style}
%
% \label{dok:th@plain} As the following macros use |@|, we have
% to locally set the |\catcode| of this symbol to ``letter''.
% This happens within a group, so that we do not have to worry about
% which |\catcode| that symbol had before.
%    \begin{macrocode}
%<*thp>
\begingroup \makeatletter
%    \end{macrocode}
% Since we are now within a group, we must make all definitions
% globally.  First we make sure that \texttt{theorem.sty} is loaded.
% This will allow us to use this file as a document style option
% without having to call \texttt{theorem} itself as an option.  At the
% same time, we assure that at least version 2 is loaded, since
% |\theorem@style| was not defined in earlier versions.
%    \begin{macrocode}
\@ifundefined{theorem@style}{\input{theorem.sty}}{}
\ProvidesFile{thp.sty}
             [\FMithmInfo]
%    \end{macrocode}
%
%
% \begin{macro}{\th@plain}
% \begin{macro}{\@begintheorem}
% \begin{macro}{\@opargbegintheorem}
%    |\theoremstyle{plain}| corresponds to the original
%    definition, except that the distances to the surrounding text are
%    determined by the parameters |\theorempreskipamount| and
%    |\theorempostskipamount|.
%    First we set the default body font.
%    \begin{macrocode}
\gdef\th@plain{\normalfont\itshape
%    \end{macrocode}
%    Then we define |\@begintheorem| and
%    |\@opargbegintheorem|. These two macros define how the
%    header of a theorem is typeset.  |\@opargbegintheorem| will
%    be called if a \textsf{theorem} environment with an optional
%    argument is encountered; otherwise, the header is constructed by
%    calling |\@begintheorem|.  If one of these macros is
%    executed, we are within a \textsf{trivlist} environment started by
%    |\@thm|.  So the theorem header is produced with an
%    |\item| command.
%
%    Instead of specifying the header font directly, all standard
%    theorem styles use the |\theorem@headerfont| macro to allow
%    customization.  The extra space (|\labelsep|) is necessary
%    because of problems in the \textsf{trivlist} environment.
%    \begin{macrocode}
  \def\@begintheorem##1##2{%
        \item[\hskip\labelsep \theorem@headerfont ##1\ ##2]}%
%    \end{macrocode}
%    The definition of |\@opargbegintheorem| is completely
%    analogous. The only difference is the fact that there exists a
%    third argument (which is the optional parameter of the
%    environment and contains additional information about the
%    theorem).  Customarily we enclose it in parentheses.
%    \begin{macrocode}
\def\@opargbegintheorem##1##2##3{%
   \item[\hskip\labelsep \theorem@headerfont ##1\ ##2\ (##3)]}}
%    \end{macrocode}
% \end{macro}
% \end{macro}
% \end{macro}
%
% \noindent
% We conclude with an |\endgroup| to restore the
% |\catcode| of |@|.
%    \begin{macrocode}
\endgroup
%</thp>
%    \end{macrocode}
%
%
% \subsubsection{The \texttt{break} style}
%
% This style option is stored in the file |thb.sty|.
% For the next two lines see the documentation for |\th@plain| on
% page~\pageref{dok:th@plain}.
%    \begin{macrocode}
%<*thb>
\begingroup \makeatletter
\@ifundefined{theorem@style}{\input{theorem.sty}}{}
\ProvidesFile{thb.sty}
             [\FMithmInfo]
%    \end{macrocode}
%
% \begin{macro}{\th@break}
%    |\theoremstyle{break}| produces a line break after the name
%    of the theorem.  The font is |\slshape|.  Hence, we define
%    |\th@break| as follows:
%    \begin{macrocode}
\gdef\th@break{\normalfont\slshape
  \def\@begintheorem##1##2{\item[%
%    \end{macrocode}
%    We run into the following problem: it is not possible to create
%    the header with |\item[|\meta{title}|]| and then start
%    a new line by, for example, |\mbox{}\\|. Such a definition
%    will fail whenever a list environment follows immediately.  With
%    the above construction, the |\mbox{}| causes the switch
%    |@inlabel| (cf.\ definition of |\list| and
%    |\trivlist| in \cite{latex:source}) to be set to \texttt{false}
%    and so the following list will insert additional vertical
%    space (|\topskip|). This is quite annoying. Therefore, we
%    create the line break within the |\item|. In order to ensure
%    that the text will begin at the proper position in the following
%    line, we simply pretend that the label does not take any
%    room.\footnote{This will lead to problems whenever very high
%    symbols occurring in the line tower into the
%    heading. So, something else has to be done here sometime.}
%    \begin{macrocode}
       \rlap{\vbox{\hbox{\hskip \labelsep\theorem@headerfont ##1\ ##2}%
                   \hbox{\strut}}}]}%
%    \end{macrocode}
%    Again, the definition of |\@opargbegintheorem| is
%    completely analogous.
%    \begin{macrocode}
\def\@opargbegintheorem##1##2##3{%
  \item[\rlap{\vbox{\hbox{\hskip \labelsep \theorem@headerfont
                     ##1\ ##2\ (##3)}%
                    \hbox{\strut}}}]}}
\endgroup
%</thb>
%    \end{macrocode}
% \end{macro}
%
%
% \subsubsection{The \texttt{changebreak} style}
%
%    \begin{macrocode}
% This style option is stored in the file |thcb.sty|.
%    \begin{macrocode}
%<*thcb>
\begingroup \makeatletter
\@ifundefined{theorem@style}{\input{theorem.sty}}{}
\ProvidesFile{thcb.sty}
             [\FMithmInfo]
%    \end{macrocode}
%
% \begin{macro}{\th@changebreak}
%    The \texttt{change-break} theorem style is like \texttt{break}
%    but with interchange of theorem name and theorem number.  Thus we
%    define |\th@changebreak| as follows:
%    \begin{macrocode}
\gdef\th@changebreak{\normalfont\slshape
  \def\@begintheorem##1##2{\item
       [\rlap{\vbox{\hbox{\hskip\labelsep \theorem@headerfont ##2\ ##1}%
                    \hbox{\strut}}}]}%
\def\@opargbegintheorem##1##2##3{%
  \item[\rlap{\vbox{\hbox{\hskip\labelsep \theorem@headerfont
                     ##2\ ##1\ (##3)}%
                    \hbox{\strut}}}]}}
\endgroup
%</thcb>
%    \end{macrocode}
% \end{macro}
%
%
% \subsubsection{The \texttt{change} style}
%
% This style option is stored in the file |thc.sty|.
%    \begin{macrocode}
%<*thc>
\begingroup \makeatletter
\@ifundefined{theorem@style}{\input{theorem.sty}}{}
\ProvidesFile{thc.sty}
             [\FMithmInfo]
%    \end{macrocode}
%
% \begin{macro}{\th@change}
%    The \texttt{change} theorem style corresponds to the
%    \texttt{change break} style without a linebreak after the header.
%    To say it in another way, it's the same as the \texttt{plain}
%    style but with number and name interchanged and |\slshape| as the
%    default font.
%    \begin{macrocode}
\gdef\th@change{\normalfont\slshape
  \def\@begintheorem##1##2{\item
       [\hskip\labelsep \theorem@headerfont ##2\ ##1]}%
\def\@opargbegintheorem##1##2##3{%
  \item[\hskip\labelsep \theorem@headerfont ##2\ ##1\ (##3)]}}
\endgroup
%</thc>
%    \end{macrocode}
% \end{macro}
%
%
% \subsubsection{The \texttt{marginbreak} style}
%
% This style option is the one used most often at Mainz.  It is saved
% in the file |thmb.sty|.
%    \begin{macrocode}
%<*thmb>
\begingroup \makeatletter
\@ifundefined{theorem@style}{\input{theorem.sty}}{}
\ProvidesFile{thmb.sty}
             [\FMithmInfo]
%    \end{macrocode}
%
% \begin{macro}{\th@marginbreak}
%    The \texttt{margin break} style is nearly the same as the
%    \texttt{change break} style. The only difference is the placement
%    of the theorem number.  We use |\llap| to place it in the left
%    margin.
%
%    In this style |\labelsep| denotes the separation between the
%    number and the text.
%    \begin{macrocode}
\gdef\th@marginbreak{\normalfont\slshape
  \def\@begintheorem##1##2{\item
       [\rlap{\vbox{\theorem@headerfont
                    \hbox{\llap{##2}\hskip\labelsep ##1}%
                    \hbox{\strut}}}]}%
\def\@opargbegintheorem##1##2##3{%
  \item[\rlap{\vbox{\theorem@headerfont
                    \hbox{\llap{##2}\hskip\labelsep ##1\ (##3)}%
                    \hbox{\strut}}}]}}
\endgroup
%</thmb>
%    \end{macrocode}
% \changes{v2.1c}{90/02/17}{`endgroup outside macrocode env.}
% \end{macro}
%
%
%
% \subsubsection{The \texttt{margin} style}
%
% This style option is stored in the file |thm.sty|.
%    \begin{macrocode}
%<*thm>
\begingroup \makeatletter
\@ifundefined{theorem@style}{\input{theorem.sty}}{}
\ProvidesFile{thm.sty}
             [\FMithmInfo]
%    \end{macrocode}
%
% \begin{macro}{\th@margin}
%    Again this is only a variant of the theorem styles described
%    above without any new ideas.
%    \begin{macrocode}
\gdef\th@margin{\normalfont\slshape
  \def\@begintheorem##1##2{\item
       [\theorem@headerfont \llap{##2}\hskip\labelsep ##1]}%
\def\@opargbegintheorem##1##2##3{%
  \item[\theorem@headerfont \llap{##2}\hskip\labelsep ##1\ (##3)]}}
\endgroup
%</thm>
%    \end{macrocode}
% \end{macro}
%
%
%
%
% \subsection{Final Definitions}
%
% \begin{macro}{\theorempreskipamount}
% \begin{macro}{\theorempostskipamount}
%    The \textsf{skip} parameters that regulate the vertical empty space
%    before and after the {\envfont theorem} environment have
%    to be allocated as well.
% \changes{v2.0f}{89/07/20}{Unnecessary `global' removed.}
%    \begin{macrocode}
%<*package>
\newskip\theorempreskipamount
\newskip\theorempostskipamount
%    \end{macrocode}
%    Since we have used the same values for all theorem sets, we now can
%    assign them.
%    \begin{macrocode}
\global\setlength\theorempreskipamount{12pt plus 5pt minus 3pt}
\global\setlength\theorempostskipamount{8pt plus 3pt minus 1.5pt}
%    \end{macrocode}
% \end{macro}
% \end{macro}
%
%
% \begin{macro}{\@endtheorem}
%    The same holds for the macro |\@endtheorem|, which ends a
%    {\envfont theorem} environment. Since it is the same for all
%    theorem sets, it is removed from the macros |\th@|\meta{style}.
%    It simply ends the {\envfont trivlist} environment, which was
%    begun in |\@thm|.
%    \begin{macrocode}
\global\let\@endtheorem=\endtrivlist
%    \end{macrocode}
% \end{macro}
%
%
%
% \begin{macro}{\@preamblecmds}
%    All macros defined above are to be used only in the preamble.
%    Therefore, we insert them in |\@preamblecmds| which will disable
%    them at begin document. This is done by the internal \LaTeXe{}
%    command |\@onlypreamble|.
% \changes{v2.0g}{89/09/19}{`theoremheaderfont and `theorembodyfont
%    added.}
% \changes{v2.2a}{94/02/03}{Use \LaTeXe{} interface}
%    \begin{macrocode}
\@onlypreamble\@xnthm
\@onlypreamble\@ynthm
\@onlypreamble\@othm
\@onlypreamble\newtheorem
\@onlypreamble\theoremstyle
\@onlypreamble\theorembodyfont
\@onlypreamble\theoremheaderfont
%    \end{macrocode}
% \end{macro}
%
% Finally we declare the \texttt{plain} theorem style to be the default.
% \changes{v2.1b}{90/06/04}{Fixed missing `item by declaring default
%    style.}
%    \begin{macrocode}
\theoremstyle{plain}
%</package>
%    \end{macrocode}
%
% \Finale

\endinput
}{}
\ProvidesFile{thp.sty}
             [\FMithmInfo]
%    \end{macrocode}
%
%
% \begin{macro}{\th@plain}
% \begin{macro}{\@begintheorem}
% \begin{macro}{\@opargbegintheorem}
%    |\theoremstyle{plain}| corresponds to the original
%    definition, except that the distances to the surrounding text are
%    determined by the parameters |\theorempreskipamount| and
%    |\theorempostskipamount|.
%    First we set the default body font.
%    \begin{macrocode}
\gdef\th@plain{\normalfont\itshape
%    \end{macrocode}
%    Then we define |\@begintheorem| and
%    |\@opargbegintheorem|. These two macros define how the
%    header of a theorem is typeset.  |\@opargbegintheorem| will
%    be called if a \textsf{theorem} environment with an optional
%    argument is encountered; otherwise, the header is constructed by
%    calling |\@begintheorem|.  If one of these macros is
%    executed, we are within a \textsf{trivlist} environment started by
%    |\@thm|.  So the theorem header is produced with an
%    |\item| command.
%
%    Instead of specifying the header font directly, all standard
%    theorem styles use the |\theorem@headerfont| macro to allow
%    customization.  The extra space (|\labelsep|) is necessary
%    because of problems in the \textsf{trivlist} environment.
%    \begin{macrocode}
  \def\@begintheorem##1##2{%
        \item[\hskip\labelsep \theorem@headerfont ##1\ ##2]}%
%    \end{macrocode}
%    The definition of |\@opargbegintheorem| is completely
%    analogous. The only difference is the fact that there exists a
%    third argument (which is the optional parameter of the
%    environment and contains additional information about the
%    theorem).  Customarily we enclose it in parentheses.
%    \begin{macrocode}
\def\@opargbegintheorem##1##2##3{%
   \item[\hskip\labelsep \theorem@headerfont ##1\ ##2\ (##3)]}}
%    \end{macrocode}
% \end{macro}
% \end{macro}
% \end{macro}
%
% \noindent
% We conclude with an |\endgroup| to restore the
% |\catcode| of |@|.
%    \begin{macrocode}
\endgroup
%</thp>
%    \end{macrocode}
%
%
% \subsubsection{The \texttt{break} style}
%
% This style option is stored in the file |thb.sty|.
% For the next two lines see the documentation for |\th@plain| on
% page~\pageref{dok:th@plain}.
%    \begin{macrocode}
%<*thb>
\begingroup \makeatletter
\@ifundefined{theorem@style}{% \iffalse meta-comment
%
% Copyright (C) 1993-2024
%
% The LaTeX Project and any individual authors listed elsewhere
% in this file.
%
% This file is part of the Standard LaTeX `Tools Bundle'.
% -------------------------------------------------------
%
% It may be distributed and/or modified under the
% conditions of the LaTeX Project Public License, either version 1.3c
% of this license or (at your option) any later version.
% The latest version of this license is in
%    https://www.latex-project.org/lppl.txt
% and version 1.3c or later is part of all distributions of LaTeX
% version 2005/12/01 or later.
%
% The list of all files belonging to the LaTeX `Tools Bundle' is
% given in the file `manifest.txt'.
%
% \fi
%
%%
%
%
\def\FMithmInfo{2023/07/05 v2.2c Theorem extension package (FMi)}
%
% \ProvidesFile{theorem.dtx}[\FMithmInfo]
%\iffalse   % this is a METACOMMENT !
%%
%% Package `theorem' to use with LaTeX2e
%% Copyright (C) 1989-1995 Frank Mittelbach, all rights reserved.
%<+package>\@ifundefined{theorem@style}{}{\endinput}
%<+package>\NeedsTeXFormat{LaTeX2e}[1995/06/01]
%<+package>\ProvidesPackage{theorem}
%<+package>         [\FMithmInfo]
%
% \fi
%
% \changes{v1.0}{87/05/07}{First draft.}
% \changes{v1.0a}{87/05/07}{`thistheoremstyle' added.}
% \changes{v1.1a}{87/07/29}{New concept: redefine the internal
%                          \protect\LaTeX{} commands `@xnthm' etc.}
% \changes{v1.1b}{87/07/31}{`theoremfont' used to specify font.}
% \changes{v1.2a}{88/01/08}{`@endtheorem' now globally defined.}
% \changes{v1.2b}{88/01/11}{`newtheorem' only allowed in the preamble.}
% \changes{v1.3a}{88/02/14}{`theoremfont' now a toks register.}
%
% \changes{v2.0a}{88/11/16}{New implementation and new documentation.}
% \changes{v2.0a}{88/11/16}{No layout changes but internal modifications
%                           old substyles won't work.}
% \changes{v2.0b}{89/01/30}{`global' used for all definitions.}
% \changes{v2.0c}{89/06/15}{Documentation translated to English.
%                            Thanks to LLL and Nina}
% \changes{v2.1a}{90/02/15}{Support for new font selection.}
% \changes{v2.2a}{94/02/03}{Cleanup for LaTeX2e}
%
% \DoNotIndex{\@endparenv,\@ifdefinable,\@ifnextchar,\@ifundefined}
% \DoNotIndex{\@namedef,\@nameuse,\@spaces,\@tempa,\@warning}
% \DoNotIndex{\ }
% \DoNotIndex{\begingroup,\bfseries}
% \DoNotIndex{\csname}
% \DoNotIndex{\def,\do,\docdate}
% \DoNotIndex{\endcsname,\endgroup,\endinput,\endtrivlist,\expandafter}
% \DoNotIndex{\filedate,\fileversion}
% \DoNotIndex{\gdef,\global}
% \DoNotIndex{\hbox,\hskip}
% \DoNotIndex{\ignorespaces,\input,\itshape,\item}
% \DoNotIndex{\labelsep,\let,\llap}
% \DoNotIndex{\makeatletter}
% \DoNotIndex{\noexpand}
% \DoNotIndex{\refstepcounter,\relax,\rlap}
% \DoNotIndex{\setlength,\slshape,\space,\string,\strut}
% \DoNotIndex{\the,\trivlist,\typeout}
% \DoNotIndex{\vbox}
% \DoNotIndex{\xdef}
%
% \def\deflabel#1{\ttfamily #1\hfill}
% \def\deflist#1{\begin{list}{}{\settowidth\labelwidth{\ttfamily #1}%
%                               \setlength\leftmargin\labelwidth
%                               \addtolength\leftmargin\labelsep
%                               \let\makelabel\deflabel}}
% \def\enddeflist{\end{list}}
% \def\envfont{\normalfont\ttfamily}
%
% \MakeShortVerb{\|}
%
% \newcommand{\TUB}{{\sl TUGboat\/}}
%
% \GetFileInfo{theorem.dtx}
% \title{An Extension of the
%        \LaTeX{} theorem environment\thanks{This file
%        has version number \fileversion, last
%        revised \filedate.}}
% \author{Frank Mittelbach}
%
% \MaintainedByLaTeXTeam{tools}
% \maketitle
%
% \begin{abstract}
%    The macros described in this paper yield an extension of the
%    \LaTeX{} theorem mechanism. It is designed to satisfy the
%    different requirements of various journals. Thus, the
%    layout of the ``theorems'' can be manipulated by determining a
%    ``style''. This article describes not only the use, but
%    also the definition, of the necessary macros.
% \end{abstract}
%
% \section*{Preface to version 2.2}
%
% For \LaTeXe{} this package did not need any fundamental changes. I
% only modified the messages generated so that theorem layout styles
% will show up with the |\listfiles| command and cleaned the section
% on the New Font Selection Scheme since this is now included in
% \LaTeX.
%
% \section*{Preface to version 2.1}
%
% This version is identical to 2.0g described in \TUB\ 10\#3 except
% for some internal defaults which are now set depending on the used
% font selection scheme.
%
% This was done to avoid unpleasant surprises if the new font selection
% scheme is in force. For further details see
% section~\ref{sec:newfonts} and \cite{bk:GMS94}.
%
% \section{Introduction}
%
% For our purposes here, ``theorems'' are labelled enunciations,
% often set off from the main text by extra space and a font change.
% Theorems, corollaries, conjectures, definitions, and remarks
% are all instances of ``theorems''.  The ``header'' of these structures
% is composed of a label (such as \textsc{Theorem} or \textsc{Remark})
% and a number which serializes an item in the sequence of items
% with the same label.
%
% Shortly after the introduction of \LaTeX{} at the Fachbereich
% Mathematik in Mainz, the desire to manipulate the layout of
% ``theorems'' arose.  In Mainz, the following two conventions came
% into general use:
% \begin{enumerate}
%   \item The number of the theorem is shown in the margin.
%   \item There is a line break at
%         the end of the theorem header.
% \end{enumerate}
%
% Additionally, some journals require different formats which depend on
% the ``sort of theorem'': e.g.\ often remarks and definitions are set
% in |\upshape|, while |\itshape| is employed for main theorems.
%
% Confronted with these requirements, a theorem environment was
% developed in Mainz which allows separate determination of the
% layout of the ``theorems sets'', comparable to |\pagestyle|.
%
%
% \section{The user interface}
% \subsection{Defining new theorem sets}
%
% \DescribeMacro\newtheorem
% As in the original \LaTeX{} version, the command |\newtheorem|
% defines a new ``theorem set'' or ``theorem-like structure''.
% Two required arguments name the new environment and give the text
% to be typeset with each instance of the new ``set'', while
% an optional argument determines how
% the ``set'' is enumerated:
% \begin{description}
%    \item[\ttfamily \bslash newtheorem\{foo\}\{bar\}]
%       The theorem set {\envfont foo} (whose name is \texttt{bar})
%       uses its own counter.
%    \item[\ttfamily \bslash newtheorem\{foo2\}{[foo]}\{bar2\}]
%       The theorem set {\envfont foo2} (printed name \texttt{bar2})
%       uses the same counter as the theorem set \texttt{foo}.
%    \item[\ttfamily \bslash newtheorem\{foo3\}\{bar3\}{[section]}]
%       The theorem set {\envfont foo3} (printed name \texttt{bar3}) is
%       enumerated within the counter \texttt{section}, i.e.\ with every
%       new |\section| the enumeration begins again with 1, and
%       the enumeration is composed from the section-number and the
%       theorem counter itself.
% \end{description}
%
% \DescribeMacro\theoremstyle
% Additionally, the command |\theoremstyle| can define the layout
% of various, or all, theorem sets.  It should be noted that any
% theorem set defined by |\newtheorem| is typeset in the
% |\theoremstyle| that is current at the time of the definition.
% Thus, the following
% \begin{verbatim}
%    \theoremstyle{break}        \newtheorem{Cor}{Corollary}
%    \theoremstyle{plain}        \newtheorem{Exa}{Example}[section]
%\end{verbatim}                             ^^A No blank at line start
% leads to the result that the set {\envfont Cor} is formatted in the
% style \texttt{break}, while the set {\envfont Exa} and all the
% following ones are formatted in the style \texttt{plain}, unless
% another |\theoremstyle| follows.  Since the definitions installed by
% |\newtheorem| are global, one also can limit |\theoremstyle| locally
% by grouping braces.
%
% \DescribeMacro\theorembodyfont
% The choice of the font for the theorem body is completely
% independent of the chosen |\theoremstyle|; this has proven to
% be very advantageous.  For example,
% \begin{verbatim}
%   {\theorembodyfont{\upshape}          \newtheorem{Rem}{Remark}}
%\end{verbatim}
% defines a theorem set {\envfont Rem}, which will be set in
% |\upshape| in the current layout (which in our example is
% \texttt{plain}).  As with |\theoremstyle|, the |\theorembodyfont|
% chosen is that current at the time of |\newtheorem|.  If
% |\theorembodyfont| is not specified or one defines
% |\theorembodyfont{}|, then the font used will be that defined by the
% |\theoremstyle|.
%
% \DescribeMacro\theoremheaderfont It is also possible to customize
% the font used for the theorem headers.  This is, however, a global
% declaration, and therefore there should be at most one
% |\theoremheaderfont| declaration in the preamble.\footnote{If it is
% actually necessary to have different header fonts, one has to define
% new theorem styles (substituting the desired font) or specify the
% information directly in the \texttt{\bslash newtheorem} declaration
% (the unclean variant).}
%
% \DescribeMacro\theorempreskipamount
% \DescribeMacro\theorempostskipamount Two additional parameters
% affect the vertical space around the theorem environments:
% |\theorempreskipamount| and |\theorempostskipamount| define,
% respectively, the spacing before and after such an environment.
% These parameters apply for all theorem sets and can be manipulated
% with the ordinary length macros.  They are rubber lengths
% (`\textsf{skips}'), and therefore can contain \texttt{plus} and
% \texttt{minus} parts.
%
% Since the definition of theorem sets should---most sensibly---be
% placed in the preamble, we only allow installation there.
% It is therefore possible to release the memory used here after
% |\begin{document}|, in order to make room for other
% applications.
%
%
% \subsection{Existing theorem styles}
%
% The following theorem styles exist to date:
% \begin{deflist}{marginbreak}
%    \item[plain]
%       This theorem style emulates the original \LaTeX{} definition,
%       except that additionally the parameters
%       |\theorem...skipamount| are used.
%    \item[break]
%       In this style, the theorem header is followed by a line
%       break.
%    \item[marginbreak]
%       The theorem number is set in the margin, and there is a line
%       break as in \texttt{break}.
%    \item[changebreak]
%       Like \texttt{break}, but with header number and text
%       interchanged.
%    \item[change]
%       Header number and text are interchanged, without a line break.
%    \item[margin]
%       The number is set in the left margin, without a line break.
% \end{deflist}
% All styles (except \texttt{plain}) select |\slshape| as the default
% |\theorembodyfont|.
%
%
% \subsection{Examples}
%
% Given the above theorem sets {\envfont Cor}, {\envfont Exa}
% and {\envfont Rem},
% suppose that the preamble also contains the declarations:
% \begin{verbatim}
%    \theoremstyle{marginbreak}   \newtheorem{Lem}[Cor]{Lemma}
%    \theoremstyle{change}
%    \theorembodyfont{\itshape}        \newtheorem{Def}[Cor]{Definition}
%
%    \theoremheaderfont{\scshape}
%\end{verbatim}
% Then the following are some typical examples of the typeset output
% resulting from their use.
%
% \begin{Cor}
%    This is a sentence typeset in the theorem environment {\envfont
%    Cor}.
% \end{Cor}
%
% \begin{Exa}
%    This is a sentence typeset in the theorem environment {\envfont
%    Exa}.
% \end{Exa}
%
% \begin{Rem}
%    This is a sentence typeset in the theorem environment {\envfont
%    Rem}.
% \end{Rem}
%
% \begin{Lem}[Ben User]
%    This is a sentence typeset in the theorem environment {\envfont
%    Lem}.
% \end{Lem}
%
% \begin{Def}[Very impressive Definition]
%    This is a sentence typeset in the theorem environment {\envfont
%    Def}.
% \end{Def}
%
% The last two examples show the effect of the optional argument to
% a theorem environment (it is the text typeset in parentheses).
%
%
% \section{Special Considerations}
% \label{sec:newfonts}
%
% Theoremheader and body are implemented as a unit. This means that
% the |\theoremheaderfont|  will inherit characteristics of the
% |\theorembodyfont| in \LaTeXe. Thus, if for example
% |\theorembodyfont| is |\itshape| and |\theoremheaderfont|
% is |\bfseries| the font selected for the header will have the
% characteristics `bold extended italic'. If this is not desired one
% should set the |\theoremheaderfont| to something like
% \begin{verbatim}
%      \theoremheaderfont{\normalfont\bfseries}
%\end{verbatim}
% i.e.\ supplying all necessary font information explicitly.
%
%
%
% \section{Acknowledgements}
%
% The publication of this set of macros was only possible with the
% help of Christina Busse (translating the manuscript into English),
% Joachim Pense (playing the r\^ole of typist), Chris Rowley
% (looking everything over) and many others providing useful
% suggestions.
%
% \MaybeStop{%
%    \begin{thebibliography}{9}
%       \bibitem{bk:GMS94} \textsc{M.~Goossens}, \textsc{F.~Mittelbach}
%         and \textsc{A.~Samarin}.
%         \newblock The \LaTeX{} Companion.
%         \newblock
%         Addison-Wesley, Reading, Massachusetts, 1994.
%       \bibitem{latex:source} \textsc{Lamport, Leslie.} \newblock
%                               \texttt{latex.tex}, version 2.09, date
%                               Feb. 1990.
%    \end{thebibliography}
%    \PrintIndex
%    \PrintChanges}
%
%
% \section{The documentation driver file}
%
% The next bit of code contains the documentation driver file for
% \TeX{}, i.e., the file that will produce the documentation you are
% currently reading. It will be extracted from this file by the
% \texttt{docstrip} program. Since it is the first code in the file
% one can alternatively process this file directly with \LaTeXe{} to
% obtain the documentation.
%
%    \begin{macrocode}
%<*driver>
\documentclass{ltxdoc}

\usepackage{theorem}

%<+driver>% The next few lines define theorem sets which are used
%<+driver>% in the example section of the documentation.

\theoremstyle{break}       \newtheorem{Cor}{Corollary}
\theoremstyle{plain}       \newtheorem{Exa}{Example}[section]
{\theorembodyfont{\upshape}\newtheorem{Rem}{Remark}}
\theoremstyle{marginbreak} \newtheorem{Lem}[Cor]{Lemma}
\theoremstyle{change}
\theorembodyfont{\itshape} \newtheorem{Def}[Cor]{Definition}

\theoremheaderfont{\scshape}

\RecordChanges

\begin{document}
   \DocInput{theorem.dtx}
\end{document}
%</driver>
%    \end{macrocode}
%
% \section{Definition of the Macros}
%
% If the file has been loaded before, we abort immediately. If not the
% package announces itself (this is actually done at the very top of
% the file---the way it is done isn't good style so don't copy it).
% \changes{v2.0e}{89/07/19}{Spaces removed from `typeout'.}
% \changes{v2.2b}{95/11/19}{Announce moved to top of file}
%    \begin{macrocode}
%<*package>
%\@ifundefined{theorem@style}{}{\endinput}
%\def\FMithmInfo{1995/11/19 v2.2b Theorem extension package (FMi)}
%\ProvidesPackage{theorem}[\FMithmInfo]
%    \end{macrocode}
%
%
% \subsection{Definition of theorem styles and fonts}
%
% All the definitions in this file are done globally to allow
% inputting this file inside a group.
%
% \begin{macro}{\theoremstyle}
%    Before a theorem style can be installed, the chosen style must be
%    known. For that reason, we must test to see that
%    |\th@|\meta{style} is known or, more precisely, that it is
%    different from |\relax|. If the style is not known then
%    |\th@plain| is used.
%    \begin{macrocode}
\gdef\theoremstyle#1{%
   \@ifundefined{th@#1}{\@warning
          {Unknown theoremstyle `#1'. Using `plain'}%
          \theorem@style{plain}}%
%    \end{macrocode}
%    We save the theorem style to be used in the token register
%    |\theorem@style|.
%    \begin{macrocode}
      {\theorem@style{#1}}%
%    \end{macrocode}
%    Now we ``evaluate'' the theorem style: this means, we call the
%    macro |\th@|\meta{style} which will activate the relevant
%    definitions which are contained in a separate file.  This is done
%    in a group to suppress changes to the current font.  This could
%    otherwise pose problems together with the new font selection
%    scheme\footnote{When I printed the original article using the new
%    font selection scheme I ended with a document with slanted
%    typefaces (text headings and all) simply because one of the
%    theorem styles used \texttt{\bslash sl} at toplevel.} if the
%    |\th@|\meta{style} is evaluated a second time.
% \changes{v2.1a}{90/02/15}{Group added around `th@..}
% \changes{v2.1a}{90/02/15}{Reseting of `..@begintheorem removed.}
%    \begin{macrocode}
      \begingroup
        \csname th@\the\theorem@style \endcsname
      \endgroup}
%    \end{macrocode}
% \end{macro}
%
%
% \begin{macro}{\@begintheorem}
% \changes{v2.1a}{90/02/15}{Relaxed at toplevel.}
% \begin{macro}{\@opargbegintheorem}
% \changes{v2.1a}{90/02/15}{Relaxed at toplevel.}
%    We reset |\@begintheorem| and |\@opargbegintheorem| to
%    |\relax| since these commands are no longer necessary at
%    toplevel. This will save a few tokens.
%    \begin{macrocode}
\global\let\@begintheorem\relax
\global\let\@opargbegintheorem\relax
%    \end{macrocode}
% \end{macro}
% \end{macro}
%
% \begin{macro}{\theorem@style}
%    Obviously the \textsf{token} register used above has to be
%    allocated. To assure the utmost compatibility with the
%    original \LaTeX{} definition, we set the default theorem style to
%    \texttt{plain}, which implements the usual \LaTeX{} convention.
% \changes{v2.0f}{89/07/20}{Unnecessary `global' removed.}
%    \begin{macrocode}
\newtoks\theorem@style
\global\theorem@style{plain}
%    \end{macrocode}
% \end{macro}
%
% \begin{macro}{\theorembodyfont}
% \begin{macro}{\theorem@bodyfont}
%    If the |\theorembodyfont| is set by the user then it should not
%    interact with the default font set in the theorem style. When the
%    new font selection is in force this may happen if, for example,
%    the default is |\itshape| and the new |\theorembodyfont| is
%    |\sffamily|. So we add a |\reset@font| command in front of the
%    user definition.
% \changes{v2.0f}{89/07/20}{Unnecessary `global' removed.}
% \changes{v2.0g}{89/09/19}{`theoremfont to `theorembodyfont.}
% \changes{v2.1c}{92/06/22}{`reset@font added and logic changed.}
% \changes{v2.1d}{92/10/28}{Reset to empty if arg is empty}
%    \begin{macrocode}
\gdef\theorembodyfont#1{%
%    \end{macrocode}
%    We check if the argument supplied is empty and if so put nothing
%    into the |\theorem@bodyfont| token register to allow for
%    |\theorembodyfont{}| as a mean of using the default of the
%    current |\theoremstyle|.
%    \begin{macrocode}
   \def\@tempa{#1}%
   \ifx\@tempa\@empty
    \theorem@bodyfont{}%
   \else
    \theorem@bodyfont{\reset@font#1}%
   \fi
   }
\newtoks\theorem@bodyfont
\global\theorem@bodyfont{}
%    \end{macrocode}
% \end{macro}
% \end{macro}
%
%
% \begin{macro}{\theoremheaderfont}
% \changes{v2.0g}{89/09/19}{Macro added.}
%    The font for the theorem headers is handled differently because
%    this definition applies to all theorem styles.
%    \begin{macrocode}
\gdef\theoremheaderfont#1{\gdef\theorem@headerfont{#1}%
%    \end{macrocode}
%    After using the macro once it is redefined to produce an error
%    message.
%    \begin{macrocode}
       \gdef\theoremheaderfont##1{%
        \typeout{\string\theoremheaderfont\space should be used
                 only once.}}}
%    \end{macrocode}
% \end{macro}
%
% \begin{macro}{\theorem@headerfont}
% \changes{v2.0g}{89/09/19}{Macro added.}
%    To set the |\theorem@headerfont| default we first test if
%    the new fontselection scheme is in force.
% \changes{v2.1a}{90/02/15}{Support for new font selection.}
%    \begin{macrocode}
\ifx\upshape\undefined
%    \end{macrocode}
%    If not we define it to expand into |\bfseries|. We don't use
%    |\let| just in case a following style option redefines this
%    macro.
%    \begin{macrocode}
\gdef\theorem@headerfont{\bfseries}
%    \end{macrocode}
%    Otherwise we reset the current shape before calling |\bfseries|.
%    \begin{macrocode}
\else \gdef\theorem@headerfont{\normalfont\bfseries}\fi
%    \end{macrocode}
% \end{macro}
%
% \begin{macro}{\th@plain}
% \begin{macro}{\th@break}
% \begin{macro}{\th@marginbreak}
% \begin{macro}{\th@changebreak}
% \begin{macro}{\th@change}
% \begin{macro}{\th@margin}
%    The different styles are defined in macros such as
%    |\th@plain|.  Since memory space is precious in
%    ``non-Big-versions'', we have to avoid offering too many unused
%    definitions.  Therefore we define these styles in separate files
%    that can be loaded on demand.  Thus the commands themselves only
%    load these files. We use |\@input@|, a \LaTeXe{} internal command
%    that ensures that the file will be listed with |\listfiles|.
% \changes{v2.2a}{94/02/02}{Use `@input@ to load theorem layout files}
%    \begin{macrocode}
\gdef\th@plain{\@input@{thp.sty}}
\gdef\th@break{\@input@{thb.sty}}
\gdef\th@marginbreak{\@input@{thmb.sty}}
\gdef\th@changebreak{\@input@{thcb.sty}}
\gdef\th@change{\@input@{thc.sty}}
\gdef\th@margin{\@input@{thm.sty}}
%    \end{macrocode}
%    This list will be expanded when new styles become
%    available.  For testing,  just append new theorem substyles
%    as document options.
% \end{macro}
% \end{macro}
% \end{macro}
% \end{macro}
% \end{macro}
% \end{macro}
%
%
%
% \subsection{Definition of a new theorem set}
%
% As already pointed out, a new theorem environment can be defined in
% three different ways:
% \begin{verbatim}
%    \newtheorem{Lem}{Lemma}
%    \newtheorem{Lem}{Lemma}[section]
%    \newtheorem{Lem}[Theorem]{Lemma}
%\end{verbatim}
% The function of the macro |\newtheorem| is to recognize these cases
% and then to branch into one of the three macros |\@ynthm|, |\@xnthm|
% or |\@othm|.  This mechanism is adopted unchanged from
% \cite{latex:source}; the essential point here is that, for example,
% in the second case, the arguments \texttt{Lem}, \texttt{Lemma} and
% \texttt{section} are passed over to the macro |\@xnthm|.
%
% We inspect this case first because the others present fewer problems,
% and thus are easily derived from this one.
%
% \begin{macro}{\@xnthm}
%    For our example arguments, the macro |\@xnthm| must fulfill
%    the following:
%    \begin{itemize}
%       \item
%          Define a new \LaTeX-counter `Lem'
%       \item
%          reset this counter within a |\section|
%       \item
%          define the macro |\theLem|
%       \item
%          define the environment macros |\Lem| and
%          |\endLem| using the current
%          |\theoremstyle| and |\theorem@bodyfont|.
%    \end{itemize}
%    Obviously, all this should happen only if the first argument
%    of |\@xnthm| (i.e.\ \texttt{Lem} in our example) is chosen so
%    as not to conflict with any previously defined commands or
%    environments.  This test is performed by the \LaTeX{} macro
%    |\@ifdefinable|.
%    \begin{macrocode}
\gdef\@xnthm#1#2[#3]{\expandafter\@ifdefinable\csname #1\endcsname
%    \end{macrocode}
%    Therefore, the first argument of |\@ifdefinable| is the
%    expansion (in the example, |\Lem|) of
%    |\csname||#1\endcsname|.  The second argument is
%    executed only if the test has been completed successfully.
%    \begin{macrocode}
   {%
%    \end{macrocode}
%    Now we define the new counter. The names of the \LaTeX{}
%    macros employed should speak for themselves:
% \changes{v2.2b}{1995/10/07}{Use \cs{@newctr}}
% \changes{v2.2c}{1995/11/23}{Use \cs{@newctr} properly.}
%    \begin{macrocode}
    \@definecounter{#1}\@newctr{#1}[#3]%
%    \end{macrocode}
%    Using |\@newctr| will give a proper error message if the counter
%    in |#3| is not defined.
%    In defining `|\theLem|' we must generate the desired macro
%    name by use of |\expandafter| and |\csname|.
%    \begin{macrocode}
    \expandafter\xdef\csname the#1\endcsname
%    \end{macrocode}
%    An |\xdef| is used in order to make the definition global,
%    and to ensure that it contains the replacement texts of
%    |\@thmcountersep| and |\@thmcounter|.\footnote{These
%    two macros can be defined by the document style.  Their default
%    values produce a `\texttt{.}' as separation and an arabic
%    representation of the number.} However, not everything should be
%    expanded. For example, it saves space to use |\thesection|
%    instead of its---at times---lengthy expansion.
%    \begin{macrocode}
      {\expandafter \noexpand \csname the#3\endcsname
       \@thmcountersep \@thmcounter{#1}}%
%    \end{macrocode}
%    Thus with the defaults of \LaTeX{}, |\theLem| would be
%    replaced by the command sequence |\thesection.\arabic{Lem}|.
%
%    We will now look at the definition of the macro which is executed
%    at the beginning of the actual environment (in our example this
%    macro is |\Lem|). It should be noted that we use an
%    ``|\expandafter| trick'' to expand only certain parts of the
%    replacement text at the time of the definition.
%    \begin{macrocode}
    \def\@tempa{\global\@namedef{#1}}%
    \expandafter \@tempa \expandafter{%
%    \end{macrocode}
%    First, the macro that contains the current definitions of
%    |\@begintheorem| and |\@opargtheorem| should be called
%    up.  The name of this macro---as is already known---has the form
%    |\th@|\meta{theorem\ style}; therefore, it must be called by
%    \begin{macrocode}
      \csname th@\the \theorem@style
%    \end{macrocode}
%    In addition the default theorem font should be changeable, i.e.\
%    we have to insert the contents of |\theorem@bodyfont|. For
%    that reason, we expand even further, beyond |\endcsname|,
%    and thus insert the contents of the \textsf{token} register
%    |\theorem@bodyfont| in the replacement text.
%    \begin{macrocode}
            \expandafter \endcsname \the \theorem@bodyfont
%    \end{macrocode}
%    Now it is time to call the macro |\@thm| which takes over
%    the further processing.  It has two arguments: the current
%    counter name (in our example, \texttt{Lem}), and the text of the
%    label (in our example, \texttt{Lemma}).
%    \begin{macrocode}
     \@thm{#1}{#2}}%
%    \end{macrocode}
%    With this, the `sub-definition' is complete.  The macro
%    |\@endtheorem| ends a theorem environment and is, so far, nothing
%    but an |\endtrivlist|.  (Hence it is defined globally, and not
%    within the theorem styles.\footnote{This has to be changed as
%    soon as theorem styles that change \texttt{\bslash @endtheorem}
%    exist.  In such a case, all existing styles must be changed as
%    well since they will have to reset the macro.}) Therefore, we can
%    set it equivalent to the macro that ends the theorem set (in our
%    example, |\endLem|).  However, if some day theorem styles exist
%    that do change |\@endtheorem|, we would have to use the
%    commented-out line instead.
%    \begin{macrocode}
    \global \expandafter \let \csname end#1\endcsname \@endtheorem
%   \global\@namedef{end#1}{\@endtheorem}%
%    \end{macrocode}
%    With these commands all the required definitions are employed,
%    unless the test |\@ifdefinable| has failed.  Therefore, we
%    end the second argument of this macro and with it the definition
%    of |\@xnthm|.
%    \begin{macrocode}
   }}
%    \end{macrocode}
% \end{macro}
%
% \begin{macro}{\@ynthm}
%    The definition of |\@ynthm| is completely analogous. In
%    this case the new counter that is defined is not reset within
%    another counter; thus the definition of |\the...|
%    is simplified:
%    \begin{macrocode}
\gdef\@ynthm#1#2{\expandafter\@ifdefinable\csname #1\endcsname
   {\@definecounter{#1}%
    \expandafter\xdef\csname the#1\endcsname{\@thmcounter{#1}}%
%    \end{macrocode}
%    The rest of the definition corresponds literally to that of
%    |\@xnthm|:
%    \begin{macrocode}
    \def\@tempa{\global\@namedef{#1}}\expandafter \@tempa
     \expandafter{\csname th@\the \theorem@style \expandafter
     \endcsname \the\theorem@bodyfont \@thm{#1}{#2}}%
    \global \expandafter \let \csname end#1\endcsname \@endtheorem}}
%    \end{macrocode}
% \end{macro}
%
% \begin{macro}{\@othm}
%    The definition of |\@othm| does not contain anything new.
%    \begin{macrocode}
\gdef\@othm#1[#2]#3{%
%    \end{macrocode}
% \changes{v2.2b}{95/10/07}{Build in check for counter defined}
%    We do not define a new counter but instead use one that has
%    already been defined.  Thus the only definition we need is that
%    of this pseudo-counter (i.e.\ |\the|\meta{env.\ name}).
%    First we check if |#2| corresponds to a known counter name.
%    \begin{macrocode}
  \expandafter\ifx\csname c@#2\endcsname\relax
   \@nocounterr{#2}%
  \else
   \expandafter\@ifdefinable\csname #1\endcsname
   {\expandafter \xdef \csname the#1\endcsname
     {\expandafter \noexpand \csname the#2\endcsname}%
%    \end{macrocode}
%    All other parts of the definition can be adopted from
%    |\@xnthm|. We have to remember, though, that in this case
%    the name of the current counter and the theorem label have moved
%    to the second and third arguments.
%    \begin{macrocode}
    \def\@tempa{\global\@namedef{#1}}\expandafter \@tempa
     \expandafter{\csname th@\the \theorem@style \expandafter
     \endcsname \the\theorem@bodyfont \@thm{#2}{#3}}%
    \global \expandafter \let \csname end#1\endcsname \@endtheorem}%
  \fi}
%    \end{macrocode}
% \end{macro}
%
% \subsection{Macros that are employed in a {\envfont theorem}
%    environment}
%
% \begin{macro}{\@thm}
%    The macro |\@thm| has to increase the current counter.  Then,
%    depending on whether the environment has (or does not have) an
%    optional argument, it has to branch into either |\@begintheorem|
%    or |\@opargtheorem|.
%    \begin{macrocode}
\gdef\@thm#1#2{\refstepcounter{#1}%
%    \end{macrocode}
%    Now we start a {\envfont trivlist} environment, and give
%    |\@topsep| and |\@topsepadd| the values of the \textsf{skip}
%    registers |\theorempreskipamount| and |\theorempostskipamount|.
%    The value in |\@topsep| is the vertical space that is inserted by
%    the first (and only) |\item| in our |\trivlist| whilst
%    |\@topsepadd| is inserted by |\@endparenv| at the end of that
%    {\envfont trivlist} environment.  By using these registers, we
%    obtain the desired space around a {\envfont theorem} environment.
%    \begin{macrocode}
   \trivlist
   \@topsep \theorempreskipamount               % used by first \item
   \@topsepadd \theorempostskipamount           % used by \@endparenv
%    \end{macrocode}
%    Now we have to test whether an optional argument has been given.
% \changes{v2.0d}{89/07/18}{Forgotten percent added.}
%    \begin{macrocode}
   \@ifnextchar [%
%    \end{macrocode}
%    If there is an optional argument, we will call |\@ythm|, and
%    move the arguments read back into the input stream.
%    \begin{macrocode}
   {\@ythm{#1}{#2}}%
%    \end{macrocode}
%    If not, we call |\@begintheorem|. Its first argument is the
%    name of the theorem set (hence the second argument of
%    |\@thm|). Its second argument is the macro that produces the
%    current number.
%    \begin{macrocode}
   {\@begintheorem{#2}{\csname the#1\endcsname}\ignorespaces}}
%    \end{macrocode}
% \end{macro}
%
% \begin{macro}{\@xthm}
% \begin{macro}{\@ythm}
%    Both these macros were originally called by |\@thm|.  We do
%    not need |\@xthm| anymore, hence we reset it to
%    |\relax|. The definition of |\@ythm| has not changed at
%    all from its definition in \LaTeX. In order to make the macros
%    easier to understand, we will
%    nevertheless present it (commented out).
%    \begin{macrocode}
\global\let\@xthm\relax
% \def\@ythm#1#2[#3]{\@opargbegintheorem{#2}{\csname
%           the#1\endcsname}{#3}\ignorespaces}
%</package>
%    \end{macrocode}
%    The primitive |\ignorespaces| in |\@ythm| and
%    |\@thm| is needed to remove the spaces between the
%    |\begin{...}| and the actual text.
% \end{macro}
% \end{macro}
%
%
% \subsection{Definition of the theorem substyles}
%
% As already pointed out, the theorem substyles, defined below, are
% only loaded when necessary.  Note that all these substyles, except
% \texttt{plain}, have |\slshape| as the default body font.
%
% \subsubsection{The \texttt{plain} style}
%
% \label{dok:th@plain} As the following macros use |@|, we have
% to locally set the |\catcode| of this symbol to ``letter''.
% This happens within a group, so that we do not have to worry about
% which |\catcode| that symbol had before.
%    \begin{macrocode}
%<*thp>
\begingroup \makeatletter
%    \end{macrocode}
% Since we are now within a group, we must make all definitions
% globally.  First we make sure that \texttt{theorem.sty} is loaded.
% This will allow us to use this file as a document style option
% without having to call \texttt{theorem} itself as an option.  At the
% same time, we assure that at least version 2 is loaded, since
% |\theorem@style| was not defined in earlier versions.
%    \begin{macrocode}
\@ifundefined{theorem@style}{\input{theorem.sty}}{}
\ProvidesFile{thp.sty}
             [\FMithmInfo]
%    \end{macrocode}
%
%
% \begin{macro}{\th@plain}
% \begin{macro}{\@begintheorem}
% \begin{macro}{\@opargbegintheorem}
%    |\theoremstyle{plain}| corresponds to the original
%    definition, except that the distances to the surrounding text are
%    determined by the parameters |\theorempreskipamount| and
%    |\theorempostskipamount|.
%    First we set the default body font.
%    \begin{macrocode}
\gdef\th@plain{\normalfont\itshape
%    \end{macrocode}
%    Then we define |\@begintheorem| and
%    |\@opargbegintheorem|. These two macros define how the
%    header of a theorem is typeset.  |\@opargbegintheorem| will
%    be called if a \textsf{theorem} environment with an optional
%    argument is encountered; otherwise, the header is constructed by
%    calling |\@begintheorem|.  If one of these macros is
%    executed, we are within a \textsf{trivlist} environment started by
%    |\@thm|.  So the theorem header is produced with an
%    |\item| command.
%
%    Instead of specifying the header font directly, all standard
%    theorem styles use the |\theorem@headerfont| macro to allow
%    customization.  The extra space (|\labelsep|) is necessary
%    because of problems in the \textsf{trivlist} environment.
%    \begin{macrocode}
  \def\@begintheorem##1##2{%
        \item[\hskip\labelsep \theorem@headerfont ##1\ ##2]}%
%    \end{macrocode}
%    The definition of |\@opargbegintheorem| is completely
%    analogous. The only difference is the fact that there exists a
%    third argument (which is the optional parameter of the
%    environment and contains additional information about the
%    theorem).  Customarily we enclose it in parentheses.
%    \begin{macrocode}
\def\@opargbegintheorem##1##2##3{%
   \item[\hskip\labelsep \theorem@headerfont ##1\ ##2\ (##3)]}}
%    \end{macrocode}
% \end{macro}
% \end{macro}
% \end{macro}
%
% \noindent
% We conclude with an |\endgroup| to restore the
% |\catcode| of |@|.
%    \begin{macrocode}
\endgroup
%</thp>
%    \end{macrocode}
%
%
% \subsubsection{The \texttt{break} style}
%
% This style option is stored in the file |thb.sty|.
% For the next two lines see the documentation for |\th@plain| on
% page~\pageref{dok:th@plain}.
%    \begin{macrocode}
%<*thb>
\begingroup \makeatletter
\@ifundefined{theorem@style}{\input{theorem.sty}}{}
\ProvidesFile{thb.sty}
             [\FMithmInfo]
%    \end{macrocode}
%
% \begin{macro}{\th@break}
%    |\theoremstyle{break}| produces a line break after the name
%    of the theorem.  The font is |\slshape|.  Hence, we define
%    |\th@break| as follows:
%    \begin{macrocode}
\gdef\th@break{\normalfont\slshape
  \def\@begintheorem##1##2{\item[%
%    \end{macrocode}
%    We run into the following problem: it is not possible to create
%    the header with |\item[|\meta{title}|]| and then start
%    a new line by, for example, |\mbox{}\\|. Such a definition
%    will fail whenever a list environment follows immediately.  With
%    the above construction, the |\mbox{}| causes the switch
%    |@inlabel| (cf.\ definition of |\list| and
%    |\trivlist| in \cite{latex:source}) to be set to \texttt{false}
%    and so the following list will insert additional vertical
%    space (|\topskip|). This is quite annoying. Therefore, we
%    create the line break within the |\item|. In order to ensure
%    that the text will begin at the proper position in the following
%    line, we simply pretend that the label does not take any
%    room.\footnote{This will lead to problems whenever very high
%    symbols occurring in the line tower into the
%    heading. So, something else has to be done here sometime.}
%    \begin{macrocode}
       \rlap{\vbox{\hbox{\hskip \labelsep\theorem@headerfont ##1\ ##2}%
                   \hbox{\strut}}}]}%
%    \end{macrocode}
%    Again, the definition of |\@opargbegintheorem| is
%    completely analogous.
%    \begin{macrocode}
\def\@opargbegintheorem##1##2##3{%
  \item[\rlap{\vbox{\hbox{\hskip \labelsep \theorem@headerfont
                     ##1\ ##2\ (##3)}%
                    \hbox{\strut}}}]}}
\endgroup
%</thb>
%    \end{macrocode}
% \end{macro}
%
%
% \subsubsection{The \texttt{changebreak} style}
%
%    \begin{macrocode}
% This style option is stored in the file |thcb.sty|.
%    \begin{macrocode}
%<*thcb>
\begingroup \makeatletter
\@ifundefined{theorem@style}{\input{theorem.sty}}{}
\ProvidesFile{thcb.sty}
             [\FMithmInfo]
%    \end{macrocode}
%
% \begin{macro}{\th@changebreak}
%    The \texttt{change-break} theorem style is like \texttt{break}
%    but with interchange of theorem name and theorem number.  Thus we
%    define |\th@changebreak| as follows:
%    \begin{macrocode}
\gdef\th@changebreak{\normalfont\slshape
  \def\@begintheorem##1##2{\item
       [\rlap{\vbox{\hbox{\hskip\labelsep \theorem@headerfont ##2\ ##1}%
                    \hbox{\strut}}}]}%
\def\@opargbegintheorem##1##2##3{%
  \item[\rlap{\vbox{\hbox{\hskip\labelsep \theorem@headerfont
                     ##2\ ##1\ (##3)}%
                    \hbox{\strut}}}]}}
\endgroup
%</thcb>
%    \end{macrocode}
% \end{macro}
%
%
% \subsubsection{The \texttt{change} style}
%
% This style option is stored in the file |thc.sty|.
%    \begin{macrocode}
%<*thc>
\begingroup \makeatletter
\@ifundefined{theorem@style}{\input{theorem.sty}}{}
\ProvidesFile{thc.sty}
             [\FMithmInfo]
%    \end{macrocode}
%
% \begin{macro}{\th@change}
%    The \texttt{change} theorem style corresponds to the
%    \texttt{change break} style without a linebreak after the header.
%    To say it in another way, it's the same as the \texttt{plain}
%    style but with number and name interchanged and |\slshape| as the
%    default font.
%    \begin{macrocode}
\gdef\th@change{\normalfont\slshape
  \def\@begintheorem##1##2{\item
       [\hskip\labelsep \theorem@headerfont ##2\ ##1]}%
\def\@opargbegintheorem##1##2##3{%
  \item[\hskip\labelsep \theorem@headerfont ##2\ ##1\ (##3)]}}
\endgroup
%</thc>
%    \end{macrocode}
% \end{macro}
%
%
% \subsubsection{The \texttt{marginbreak} style}
%
% This style option is the one used most often at Mainz.  It is saved
% in the file |thmb.sty|.
%    \begin{macrocode}
%<*thmb>
\begingroup \makeatletter
\@ifundefined{theorem@style}{\input{theorem.sty}}{}
\ProvidesFile{thmb.sty}
             [\FMithmInfo]
%    \end{macrocode}
%
% \begin{macro}{\th@marginbreak}
%    The \texttt{margin break} style is nearly the same as the
%    \texttt{change break} style. The only difference is the placement
%    of the theorem number.  We use |\llap| to place it in the left
%    margin.
%
%    In this style |\labelsep| denotes the separation between the
%    number and the text.
%    \begin{macrocode}
\gdef\th@marginbreak{\normalfont\slshape
  \def\@begintheorem##1##2{\item
       [\rlap{\vbox{\theorem@headerfont
                    \hbox{\llap{##2}\hskip\labelsep ##1}%
                    \hbox{\strut}}}]}%
\def\@opargbegintheorem##1##2##3{%
  \item[\rlap{\vbox{\theorem@headerfont
                    \hbox{\llap{##2}\hskip\labelsep ##1\ (##3)}%
                    \hbox{\strut}}}]}}
\endgroup
%</thmb>
%    \end{macrocode}
% \changes{v2.1c}{90/02/17}{`endgroup outside macrocode env.}
% \end{macro}
%
%
%
% \subsubsection{The \texttt{margin} style}
%
% This style option is stored in the file |thm.sty|.
%    \begin{macrocode}
%<*thm>
\begingroup \makeatletter
\@ifundefined{theorem@style}{\input{theorem.sty}}{}
\ProvidesFile{thm.sty}
             [\FMithmInfo]
%    \end{macrocode}
%
% \begin{macro}{\th@margin}
%    Again this is only a variant of the theorem styles described
%    above without any new ideas.
%    \begin{macrocode}
\gdef\th@margin{\normalfont\slshape
  \def\@begintheorem##1##2{\item
       [\theorem@headerfont \llap{##2}\hskip\labelsep ##1]}%
\def\@opargbegintheorem##1##2##3{%
  \item[\theorem@headerfont \llap{##2}\hskip\labelsep ##1\ (##3)]}}
\endgroup
%</thm>
%    \end{macrocode}
% \end{macro}
%
%
%
%
% \subsection{Final Definitions}
%
% \begin{macro}{\theorempreskipamount}
% \begin{macro}{\theorempostskipamount}
%    The \textsf{skip} parameters that regulate the vertical empty space
%    before and after the {\envfont theorem} environment have
%    to be allocated as well.
% \changes{v2.0f}{89/07/20}{Unnecessary `global' removed.}
%    \begin{macrocode}
%<*package>
\newskip\theorempreskipamount
\newskip\theorempostskipamount
%    \end{macrocode}
%    Since we have used the same values for all theorem sets, we now can
%    assign them.
%    \begin{macrocode}
\global\setlength\theorempreskipamount{12pt plus 5pt minus 3pt}
\global\setlength\theorempostskipamount{8pt plus 3pt minus 1.5pt}
%    \end{macrocode}
% \end{macro}
% \end{macro}
%
%
% \begin{macro}{\@endtheorem}
%    The same holds for the macro |\@endtheorem|, which ends a
%    {\envfont theorem} environment. Since it is the same for all
%    theorem sets, it is removed from the macros |\th@|\meta{style}.
%    It simply ends the {\envfont trivlist} environment, which was
%    begun in |\@thm|.
%    \begin{macrocode}
\global\let\@endtheorem=\endtrivlist
%    \end{macrocode}
% \end{macro}
%
%
%
% \begin{macro}{\@preamblecmds}
%    All macros defined above are to be used only in the preamble.
%    Therefore, we insert them in |\@preamblecmds| which will disable
%    them at begin document. This is done by the internal \LaTeXe{}
%    command |\@onlypreamble|.
% \changes{v2.0g}{89/09/19}{`theoremheaderfont and `theorembodyfont
%    added.}
% \changes{v2.2a}{94/02/03}{Use \LaTeXe{} interface}
%    \begin{macrocode}
\@onlypreamble\@xnthm
\@onlypreamble\@ynthm
\@onlypreamble\@othm
\@onlypreamble\newtheorem
\@onlypreamble\theoremstyle
\@onlypreamble\theorembodyfont
\@onlypreamble\theoremheaderfont
%    \end{macrocode}
% \end{macro}
%
% Finally we declare the \texttt{plain} theorem style to be the default.
% \changes{v2.1b}{90/06/04}{Fixed missing `item by declaring default
%    style.}
%    \begin{macrocode}
\theoremstyle{plain}
%</package>
%    \end{macrocode}
%
% \Finale

\endinput
}{}
\ProvidesFile{thb.sty}
             [\FMithmInfo]
%    \end{macrocode}
%
% \begin{macro}{\th@break}
%    |\theoremstyle{break}| produces a line break after the name
%    of the theorem.  The font is |\slshape|.  Hence, we define
%    |\th@break| as follows:
%    \begin{macrocode}
\gdef\th@break{\normalfont\slshape
  \def\@begintheorem##1##2{\item[%
%    \end{macrocode}
%    We run into the following problem: it is not possible to create
%    the header with |\item[|\meta{title}|]| and then start
%    a new line by, for example, |\mbox{}\\|. Such a definition
%    will fail whenever a list environment follows immediately.  With
%    the above construction, the |\mbox{}| causes the switch
%    |@inlabel| (cf.\ definition of |\list| and
%    |\trivlist| in \cite{latex:source}) to be set to \texttt{false}
%    and so the following list will insert additional vertical
%    space (|\topskip|). This is quite annoying. Therefore, we
%    create the line break within the |\item|. In order to ensure
%    that the text will begin at the proper position in the following
%    line, we simply pretend that the label does not take any
%    room.\footnote{This will lead to problems whenever very high
%    symbols occurring in the line tower into the
%    heading. So, something else has to be done here sometime.}
%    \begin{macrocode}
       \rlap{\vbox{\hbox{\hskip \labelsep\theorem@headerfont ##1\ ##2}%
                   \hbox{\strut}}}]}%
%    \end{macrocode}
%    Again, the definition of |\@opargbegintheorem| is
%    completely analogous.
%    \begin{macrocode}
\def\@opargbegintheorem##1##2##3{%
  \item[\rlap{\vbox{\hbox{\hskip \labelsep \theorem@headerfont
                     ##1\ ##2\ (##3)}%
                    \hbox{\strut}}}]}}
\endgroup
%</thb>
%    \end{macrocode}
% \end{macro}
%
%
% \subsubsection{The \texttt{changebreak} style}
%
%    \begin{macrocode}
% This style option is stored in the file |thcb.sty|.
%    \begin{macrocode}
%<*thcb>
\begingroup \makeatletter
\@ifundefined{theorem@style}{% \iffalse meta-comment
%
% Copyright (C) 1993-2024
%
% The LaTeX Project and any individual authors listed elsewhere
% in this file.
%
% This file is part of the Standard LaTeX `Tools Bundle'.
% -------------------------------------------------------
%
% It may be distributed and/or modified under the
% conditions of the LaTeX Project Public License, either version 1.3c
% of this license or (at your option) any later version.
% The latest version of this license is in
%    https://www.latex-project.org/lppl.txt
% and version 1.3c or later is part of all distributions of LaTeX
% version 2005/12/01 or later.
%
% The list of all files belonging to the LaTeX `Tools Bundle' is
% given in the file `manifest.txt'.
%
% \fi
%
%%
%
%
\def\FMithmInfo{2023/07/05 v2.2c Theorem extension package (FMi)}
%
% \ProvidesFile{theorem.dtx}[\FMithmInfo]
%\iffalse   % this is a METACOMMENT !
%%
%% Package `theorem' to use with LaTeX2e
%% Copyright (C) 1989-1995 Frank Mittelbach, all rights reserved.
%<+package>\@ifundefined{theorem@style}{}{\endinput}
%<+package>\NeedsTeXFormat{LaTeX2e}[1995/06/01]
%<+package>\ProvidesPackage{theorem}
%<+package>         [\FMithmInfo]
%
% \fi
%
% \changes{v1.0}{87/05/07}{First draft.}
% \changes{v1.0a}{87/05/07}{`thistheoremstyle' added.}
% \changes{v1.1a}{87/07/29}{New concept: redefine the internal
%                          \protect\LaTeX{} commands `@xnthm' etc.}
% \changes{v1.1b}{87/07/31}{`theoremfont' used to specify font.}
% \changes{v1.2a}{88/01/08}{`@endtheorem' now globally defined.}
% \changes{v1.2b}{88/01/11}{`newtheorem' only allowed in the preamble.}
% \changes{v1.3a}{88/02/14}{`theoremfont' now a toks register.}
%
% \changes{v2.0a}{88/11/16}{New implementation and new documentation.}
% \changes{v2.0a}{88/11/16}{No layout changes but internal modifications
%                           old substyles won't work.}
% \changes{v2.0b}{89/01/30}{`global' used for all definitions.}
% \changes{v2.0c}{89/06/15}{Documentation translated to English.
%                            Thanks to LLL and Nina}
% \changes{v2.1a}{90/02/15}{Support for new font selection.}
% \changes{v2.2a}{94/02/03}{Cleanup for LaTeX2e}
%
% \DoNotIndex{\@endparenv,\@ifdefinable,\@ifnextchar,\@ifundefined}
% \DoNotIndex{\@namedef,\@nameuse,\@spaces,\@tempa,\@warning}
% \DoNotIndex{\ }
% \DoNotIndex{\begingroup,\bfseries}
% \DoNotIndex{\csname}
% \DoNotIndex{\def,\do,\docdate}
% \DoNotIndex{\endcsname,\endgroup,\endinput,\endtrivlist,\expandafter}
% \DoNotIndex{\filedate,\fileversion}
% \DoNotIndex{\gdef,\global}
% \DoNotIndex{\hbox,\hskip}
% \DoNotIndex{\ignorespaces,\input,\itshape,\item}
% \DoNotIndex{\labelsep,\let,\llap}
% \DoNotIndex{\makeatletter}
% \DoNotIndex{\noexpand}
% \DoNotIndex{\refstepcounter,\relax,\rlap}
% \DoNotIndex{\setlength,\slshape,\space,\string,\strut}
% \DoNotIndex{\the,\trivlist,\typeout}
% \DoNotIndex{\vbox}
% \DoNotIndex{\xdef}
%
% \def\deflabel#1{\ttfamily #1\hfill}
% \def\deflist#1{\begin{list}{}{\settowidth\labelwidth{\ttfamily #1}%
%                               \setlength\leftmargin\labelwidth
%                               \addtolength\leftmargin\labelsep
%                               \let\makelabel\deflabel}}
% \def\enddeflist{\end{list}}
% \def\envfont{\normalfont\ttfamily}
%
% \MakeShortVerb{\|}
%
% \newcommand{\TUB}{{\sl TUGboat\/}}
%
% \GetFileInfo{theorem.dtx}
% \title{An Extension of the
%        \LaTeX{} theorem environment\thanks{This file
%        has version number \fileversion, last
%        revised \filedate.}}
% \author{Frank Mittelbach}
%
% \MaintainedByLaTeXTeam{tools}
% \maketitle
%
% \begin{abstract}
%    The macros described in this paper yield an extension of the
%    \LaTeX{} theorem mechanism. It is designed to satisfy the
%    different requirements of various journals. Thus, the
%    layout of the ``theorems'' can be manipulated by determining a
%    ``style''. This article describes not only the use, but
%    also the definition, of the necessary macros.
% \end{abstract}
%
% \section*{Preface to version 2.2}
%
% For \LaTeXe{} this package did not need any fundamental changes. I
% only modified the messages generated so that theorem layout styles
% will show up with the |\listfiles| command and cleaned the section
% on the New Font Selection Scheme since this is now included in
% \LaTeX.
%
% \section*{Preface to version 2.1}
%
% This version is identical to 2.0g described in \TUB\ 10\#3 except
% for some internal defaults which are now set depending on the used
% font selection scheme.
%
% This was done to avoid unpleasant surprises if the new font selection
% scheme is in force. For further details see
% section~\ref{sec:newfonts} and \cite{bk:GMS94}.
%
% \section{Introduction}
%
% For our purposes here, ``theorems'' are labelled enunciations,
% often set off from the main text by extra space and a font change.
% Theorems, corollaries, conjectures, definitions, and remarks
% are all instances of ``theorems''.  The ``header'' of these structures
% is composed of a label (such as \textsc{Theorem} or \textsc{Remark})
% and a number which serializes an item in the sequence of items
% with the same label.
%
% Shortly after the introduction of \LaTeX{} at the Fachbereich
% Mathematik in Mainz, the desire to manipulate the layout of
% ``theorems'' arose.  In Mainz, the following two conventions came
% into general use:
% \begin{enumerate}
%   \item The number of the theorem is shown in the margin.
%   \item There is a line break at
%         the end of the theorem header.
% \end{enumerate}
%
% Additionally, some journals require different formats which depend on
% the ``sort of theorem'': e.g.\ often remarks and definitions are set
% in |\upshape|, while |\itshape| is employed for main theorems.
%
% Confronted with these requirements, a theorem environment was
% developed in Mainz which allows separate determination of the
% layout of the ``theorems sets'', comparable to |\pagestyle|.
%
%
% \section{The user interface}
% \subsection{Defining new theorem sets}
%
% \DescribeMacro\newtheorem
% As in the original \LaTeX{} version, the command |\newtheorem|
% defines a new ``theorem set'' or ``theorem-like structure''.
% Two required arguments name the new environment and give the text
% to be typeset with each instance of the new ``set'', while
% an optional argument determines how
% the ``set'' is enumerated:
% \begin{description}
%    \item[\ttfamily \bslash newtheorem\{foo\}\{bar\}]
%       The theorem set {\envfont foo} (whose name is \texttt{bar})
%       uses its own counter.
%    \item[\ttfamily \bslash newtheorem\{foo2\}{[foo]}\{bar2\}]
%       The theorem set {\envfont foo2} (printed name \texttt{bar2})
%       uses the same counter as the theorem set \texttt{foo}.
%    \item[\ttfamily \bslash newtheorem\{foo3\}\{bar3\}{[section]}]
%       The theorem set {\envfont foo3} (printed name \texttt{bar3}) is
%       enumerated within the counter \texttt{section}, i.e.\ with every
%       new |\section| the enumeration begins again with 1, and
%       the enumeration is composed from the section-number and the
%       theorem counter itself.
% \end{description}
%
% \DescribeMacro\theoremstyle
% Additionally, the command |\theoremstyle| can define the layout
% of various, or all, theorem sets.  It should be noted that any
% theorem set defined by |\newtheorem| is typeset in the
% |\theoremstyle| that is current at the time of the definition.
% Thus, the following
% \begin{verbatim}
%    \theoremstyle{break}        \newtheorem{Cor}{Corollary}
%    \theoremstyle{plain}        \newtheorem{Exa}{Example}[section]
%\end{verbatim}                             ^^A No blank at line start
% leads to the result that the set {\envfont Cor} is formatted in the
% style \texttt{break}, while the set {\envfont Exa} and all the
% following ones are formatted in the style \texttt{plain}, unless
% another |\theoremstyle| follows.  Since the definitions installed by
% |\newtheorem| are global, one also can limit |\theoremstyle| locally
% by grouping braces.
%
% \DescribeMacro\theorembodyfont
% The choice of the font for the theorem body is completely
% independent of the chosen |\theoremstyle|; this has proven to
% be very advantageous.  For example,
% \begin{verbatim}
%   {\theorembodyfont{\upshape}          \newtheorem{Rem}{Remark}}
%\end{verbatim}
% defines a theorem set {\envfont Rem}, which will be set in
% |\upshape| in the current layout (which in our example is
% \texttt{plain}).  As with |\theoremstyle|, the |\theorembodyfont|
% chosen is that current at the time of |\newtheorem|.  If
% |\theorembodyfont| is not specified or one defines
% |\theorembodyfont{}|, then the font used will be that defined by the
% |\theoremstyle|.
%
% \DescribeMacro\theoremheaderfont It is also possible to customize
% the font used for the theorem headers.  This is, however, a global
% declaration, and therefore there should be at most one
% |\theoremheaderfont| declaration in the preamble.\footnote{If it is
% actually necessary to have different header fonts, one has to define
% new theorem styles (substituting the desired font) or specify the
% information directly in the \texttt{\bslash newtheorem} declaration
% (the unclean variant).}
%
% \DescribeMacro\theorempreskipamount
% \DescribeMacro\theorempostskipamount Two additional parameters
% affect the vertical space around the theorem environments:
% |\theorempreskipamount| and |\theorempostskipamount| define,
% respectively, the spacing before and after such an environment.
% These parameters apply for all theorem sets and can be manipulated
% with the ordinary length macros.  They are rubber lengths
% (`\textsf{skips}'), and therefore can contain \texttt{plus} and
% \texttt{minus} parts.
%
% Since the definition of theorem sets should---most sensibly---be
% placed in the preamble, we only allow installation there.
% It is therefore possible to release the memory used here after
% |\begin{document}|, in order to make room for other
% applications.
%
%
% \subsection{Existing theorem styles}
%
% The following theorem styles exist to date:
% \begin{deflist}{marginbreak}
%    \item[plain]
%       This theorem style emulates the original \LaTeX{} definition,
%       except that additionally the parameters
%       |\theorem...skipamount| are used.
%    \item[break]
%       In this style, the theorem header is followed by a line
%       break.
%    \item[marginbreak]
%       The theorem number is set in the margin, and there is a line
%       break as in \texttt{break}.
%    \item[changebreak]
%       Like \texttt{break}, but with header number and text
%       interchanged.
%    \item[change]
%       Header number and text are interchanged, without a line break.
%    \item[margin]
%       The number is set in the left margin, without a line break.
% \end{deflist}
% All styles (except \texttt{plain}) select |\slshape| as the default
% |\theorembodyfont|.
%
%
% \subsection{Examples}
%
% Given the above theorem sets {\envfont Cor}, {\envfont Exa}
% and {\envfont Rem},
% suppose that the preamble also contains the declarations:
% \begin{verbatim}
%    \theoremstyle{marginbreak}   \newtheorem{Lem}[Cor]{Lemma}
%    \theoremstyle{change}
%    \theorembodyfont{\itshape}        \newtheorem{Def}[Cor]{Definition}
%
%    \theoremheaderfont{\scshape}
%\end{verbatim}
% Then the following are some typical examples of the typeset output
% resulting from their use.
%
% \begin{Cor}
%    This is a sentence typeset in the theorem environment {\envfont
%    Cor}.
% \end{Cor}
%
% \begin{Exa}
%    This is a sentence typeset in the theorem environment {\envfont
%    Exa}.
% \end{Exa}
%
% \begin{Rem}
%    This is a sentence typeset in the theorem environment {\envfont
%    Rem}.
% \end{Rem}
%
% \begin{Lem}[Ben User]
%    This is a sentence typeset in the theorem environment {\envfont
%    Lem}.
% \end{Lem}
%
% \begin{Def}[Very impressive Definition]
%    This is a sentence typeset in the theorem environment {\envfont
%    Def}.
% \end{Def}
%
% The last two examples show the effect of the optional argument to
% a theorem environment (it is the text typeset in parentheses).
%
%
% \section{Special Considerations}
% \label{sec:newfonts}
%
% Theoremheader and body are implemented as a unit. This means that
% the |\theoremheaderfont|  will inherit characteristics of the
% |\theorembodyfont| in \LaTeXe. Thus, if for example
% |\theorembodyfont| is |\itshape| and |\theoremheaderfont|
% is |\bfseries| the font selected for the header will have the
% characteristics `bold extended italic'. If this is not desired one
% should set the |\theoremheaderfont| to something like
% \begin{verbatim}
%      \theoremheaderfont{\normalfont\bfseries}
%\end{verbatim}
% i.e.\ supplying all necessary font information explicitly.
%
%
%
% \section{Acknowledgements}
%
% The publication of this set of macros was only possible with the
% help of Christina Busse (translating the manuscript into English),
% Joachim Pense (playing the r\^ole of typist), Chris Rowley
% (looking everything over) and many others providing useful
% suggestions.
%
% \MaybeStop{%
%    \begin{thebibliography}{9}
%       \bibitem{bk:GMS94} \textsc{M.~Goossens}, \textsc{F.~Mittelbach}
%         and \textsc{A.~Samarin}.
%         \newblock The \LaTeX{} Companion.
%         \newblock
%         Addison-Wesley, Reading, Massachusetts, 1994.
%       \bibitem{latex:source} \textsc{Lamport, Leslie.} \newblock
%                               \texttt{latex.tex}, version 2.09, date
%                               Feb. 1990.
%    \end{thebibliography}
%    \PrintIndex
%    \PrintChanges}
%
%
% \section{The documentation driver file}
%
% The next bit of code contains the documentation driver file for
% \TeX{}, i.e., the file that will produce the documentation you are
% currently reading. It will be extracted from this file by the
% \texttt{docstrip} program. Since it is the first code in the file
% one can alternatively process this file directly with \LaTeXe{} to
% obtain the documentation.
%
%    \begin{macrocode}
%<*driver>
\documentclass{ltxdoc}

\usepackage{theorem}

%<+driver>% The next few lines define theorem sets which are used
%<+driver>% in the example section of the documentation.

\theoremstyle{break}       \newtheorem{Cor}{Corollary}
\theoremstyle{plain}       \newtheorem{Exa}{Example}[section]
{\theorembodyfont{\upshape}\newtheorem{Rem}{Remark}}
\theoremstyle{marginbreak} \newtheorem{Lem}[Cor]{Lemma}
\theoremstyle{change}
\theorembodyfont{\itshape} \newtheorem{Def}[Cor]{Definition}

\theoremheaderfont{\scshape}

\RecordChanges

\begin{document}
   \DocInput{theorem.dtx}
\end{document}
%</driver>
%    \end{macrocode}
%
% \section{Definition of the Macros}
%
% If the file has been loaded before, we abort immediately. If not the
% package announces itself (this is actually done at the very top of
% the file---the way it is done isn't good style so don't copy it).
% \changes{v2.0e}{89/07/19}{Spaces removed from `typeout'.}
% \changes{v2.2b}{95/11/19}{Announce moved to top of file}
%    \begin{macrocode}
%<*package>
%\@ifundefined{theorem@style}{}{\endinput}
%\def\FMithmInfo{1995/11/19 v2.2b Theorem extension package (FMi)}
%\ProvidesPackage{theorem}[\FMithmInfo]
%    \end{macrocode}
%
%
% \subsection{Definition of theorem styles and fonts}
%
% All the definitions in this file are done globally to allow
% inputting this file inside a group.
%
% \begin{macro}{\theoremstyle}
%    Before a theorem style can be installed, the chosen style must be
%    known. For that reason, we must test to see that
%    |\th@|\meta{style} is known or, more precisely, that it is
%    different from |\relax|. If the style is not known then
%    |\th@plain| is used.
%    \begin{macrocode}
\gdef\theoremstyle#1{%
   \@ifundefined{th@#1}{\@warning
          {Unknown theoremstyle `#1'. Using `plain'}%
          \theorem@style{plain}}%
%    \end{macrocode}
%    We save the theorem style to be used in the token register
%    |\theorem@style|.
%    \begin{macrocode}
      {\theorem@style{#1}}%
%    \end{macrocode}
%    Now we ``evaluate'' the theorem style: this means, we call the
%    macro |\th@|\meta{style} which will activate the relevant
%    definitions which are contained in a separate file.  This is done
%    in a group to suppress changes to the current font.  This could
%    otherwise pose problems together with the new font selection
%    scheme\footnote{When I printed the original article using the new
%    font selection scheme I ended with a document with slanted
%    typefaces (text headings and all) simply because one of the
%    theorem styles used \texttt{\bslash sl} at toplevel.} if the
%    |\th@|\meta{style} is evaluated a second time.
% \changes{v2.1a}{90/02/15}{Group added around `th@..}
% \changes{v2.1a}{90/02/15}{Reseting of `..@begintheorem removed.}
%    \begin{macrocode}
      \begingroup
        \csname th@\the\theorem@style \endcsname
      \endgroup}
%    \end{macrocode}
% \end{macro}
%
%
% \begin{macro}{\@begintheorem}
% \changes{v2.1a}{90/02/15}{Relaxed at toplevel.}
% \begin{macro}{\@opargbegintheorem}
% \changes{v2.1a}{90/02/15}{Relaxed at toplevel.}
%    We reset |\@begintheorem| and |\@opargbegintheorem| to
%    |\relax| since these commands are no longer necessary at
%    toplevel. This will save a few tokens.
%    \begin{macrocode}
\global\let\@begintheorem\relax
\global\let\@opargbegintheorem\relax
%    \end{macrocode}
% \end{macro}
% \end{macro}
%
% \begin{macro}{\theorem@style}
%    Obviously the \textsf{token} register used above has to be
%    allocated. To assure the utmost compatibility with the
%    original \LaTeX{} definition, we set the default theorem style to
%    \texttt{plain}, which implements the usual \LaTeX{} convention.
% \changes{v2.0f}{89/07/20}{Unnecessary `global' removed.}
%    \begin{macrocode}
\newtoks\theorem@style
\global\theorem@style{plain}
%    \end{macrocode}
% \end{macro}
%
% \begin{macro}{\theorembodyfont}
% \begin{macro}{\theorem@bodyfont}
%    If the |\theorembodyfont| is set by the user then it should not
%    interact with the default font set in the theorem style. When the
%    new font selection is in force this may happen if, for example,
%    the default is |\itshape| and the new |\theorembodyfont| is
%    |\sffamily|. So we add a |\reset@font| command in front of the
%    user definition.
% \changes{v2.0f}{89/07/20}{Unnecessary `global' removed.}
% \changes{v2.0g}{89/09/19}{`theoremfont to `theorembodyfont.}
% \changes{v2.1c}{92/06/22}{`reset@font added and logic changed.}
% \changes{v2.1d}{92/10/28}{Reset to empty if arg is empty}
%    \begin{macrocode}
\gdef\theorembodyfont#1{%
%    \end{macrocode}
%    We check if the argument supplied is empty and if so put nothing
%    into the |\theorem@bodyfont| token register to allow for
%    |\theorembodyfont{}| as a mean of using the default of the
%    current |\theoremstyle|.
%    \begin{macrocode}
   \def\@tempa{#1}%
   \ifx\@tempa\@empty
    \theorem@bodyfont{}%
   \else
    \theorem@bodyfont{\reset@font#1}%
   \fi
   }
\newtoks\theorem@bodyfont
\global\theorem@bodyfont{}
%    \end{macrocode}
% \end{macro}
% \end{macro}
%
%
% \begin{macro}{\theoremheaderfont}
% \changes{v2.0g}{89/09/19}{Macro added.}
%    The font for the theorem headers is handled differently because
%    this definition applies to all theorem styles.
%    \begin{macrocode}
\gdef\theoremheaderfont#1{\gdef\theorem@headerfont{#1}%
%    \end{macrocode}
%    After using the macro once it is redefined to produce an error
%    message.
%    \begin{macrocode}
       \gdef\theoremheaderfont##1{%
        \typeout{\string\theoremheaderfont\space should be used
                 only once.}}}
%    \end{macrocode}
% \end{macro}
%
% \begin{macro}{\theorem@headerfont}
% \changes{v2.0g}{89/09/19}{Macro added.}
%    To set the |\theorem@headerfont| default we first test if
%    the new fontselection scheme is in force.
% \changes{v2.1a}{90/02/15}{Support for new font selection.}
%    \begin{macrocode}
\ifx\upshape\undefined
%    \end{macrocode}
%    If not we define it to expand into |\bfseries|. We don't use
%    |\let| just in case a following style option redefines this
%    macro.
%    \begin{macrocode}
\gdef\theorem@headerfont{\bfseries}
%    \end{macrocode}
%    Otherwise we reset the current shape before calling |\bfseries|.
%    \begin{macrocode}
\else \gdef\theorem@headerfont{\normalfont\bfseries}\fi
%    \end{macrocode}
% \end{macro}
%
% \begin{macro}{\th@plain}
% \begin{macro}{\th@break}
% \begin{macro}{\th@marginbreak}
% \begin{macro}{\th@changebreak}
% \begin{macro}{\th@change}
% \begin{macro}{\th@margin}
%    The different styles are defined in macros such as
%    |\th@plain|.  Since memory space is precious in
%    ``non-Big-versions'', we have to avoid offering too many unused
%    definitions.  Therefore we define these styles in separate files
%    that can be loaded on demand.  Thus the commands themselves only
%    load these files. We use |\@input@|, a \LaTeXe{} internal command
%    that ensures that the file will be listed with |\listfiles|.
% \changes{v2.2a}{94/02/02}{Use `@input@ to load theorem layout files}
%    \begin{macrocode}
\gdef\th@plain{\@input@{thp.sty}}
\gdef\th@break{\@input@{thb.sty}}
\gdef\th@marginbreak{\@input@{thmb.sty}}
\gdef\th@changebreak{\@input@{thcb.sty}}
\gdef\th@change{\@input@{thc.sty}}
\gdef\th@margin{\@input@{thm.sty}}
%    \end{macrocode}
%    This list will be expanded when new styles become
%    available.  For testing,  just append new theorem substyles
%    as document options.
% \end{macro}
% \end{macro}
% \end{macro}
% \end{macro}
% \end{macro}
% \end{macro}
%
%
%
% \subsection{Definition of a new theorem set}
%
% As already pointed out, a new theorem environment can be defined in
% three different ways:
% \begin{verbatim}
%    \newtheorem{Lem}{Lemma}
%    \newtheorem{Lem}{Lemma}[section]
%    \newtheorem{Lem}[Theorem]{Lemma}
%\end{verbatim}
% The function of the macro |\newtheorem| is to recognize these cases
% and then to branch into one of the three macros |\@ynthm|, |\@xnthm|
% or |\@othm|.  This mechanism is adopted unchanged from
% \cite{latex:source}; the essential point here is that, for example,
% in the second case, the arguments \texttt{Lem}, \texttt{Lemma} and
% \texttt{section} are passed over to the macro |\@xnthm|.
%
% We inspect this case first because the others present fewer problems,
% and thus are easily derived from this one.
%
% \begin{macro}{\@xnthm}
%    For our example arguments, the macro |\@xnthm| must fulfill
%    the following:
%    \begin{itemize}
%       \item
%          Define a new \LaTeX-counter `Lem'
%       \item
%          reset this counter within a |\section|
%       \item
%          define the macro |\theLem|
%       \item
%          define the environment macros |\Lem| and
%          |\endLem| using the current
%          |\theoremstyle| and |\theorem@bodyfont|.
%    \end{itemize}
%    Obviously, all this should happen only if the first argument
%    of |\@xnthm| (i.e.\ \texttt{Lem} in our example) is chosen so
%    as not to conflict with any previously defined commands or
%    environments.  This test is performed by the \LaTeX{} macro
%    |\@ifdefinable|.
%    \begin{macrocode}
\gdef\@xnthm#1#2[#3]{\expandafter\@ifdefinable\csname #1\endcsname
%    \end{macrocode}
%    Therefore, the first argument of |\@ifdefinable| is the
%    expansion (in the example, |\Lem|) of
%    |\csname||#1\endcsname|.  The second argument is
%    executed only if the test has been completed successfully.
%    \begin{macrocode}
   {%
%    \end{macrocode}
%    Now we define the new counter. The names of the \LaTeX{}
%    macros employed should speak for themselves:
% \changes{v2.2b}{1995/10/07}{Use \cs{@newctr}}
% \changes{v2.2c}{1995/11/23}{Use \cs{@newctr} properly.}
%    \begin{macrocode}
    \@definecounter{#1}\@newctr{#1}[#3]%
%    \end{macrocode}
%    Using |\@newctr| will give a proper error message if the counter
%    in |#3| is not defined.
%    In defining `|\theLem|' we must generate the desired macro
%    name by use of |\expandafter| and |\csname|.
%    \begin{macrocode}
    \expandafter\xdef\csname the#1\endcsname
%    \end{macrocode}
%    An |\xdef| is used in order to make the definition global,
%    and to ensure that it contains the replacement texts of
%    |\@thmcountersep| and |\@thmcounter|.\footnote{These
%    two macros can be defined by the document style.  Their default
%    values produce a `\texttt{.}' as separation and an arabic
%    representation of the number.} However, not everything should be
%    expanded. For example, it saves space to use |\thesection|
%    instead of its---at times---lengthy expansion.
%    \begin{macrocode}
      {\expandafter \noexpand \csname the#3\endcsname
       \@thmcountersep \@thmcounter{#1}}%
%    \end{macrocode}
%    Thus with the defaults of \LaTeX{}, |\theLem| would be
%    replaced by the command sequence |\thesection.\arabic{Lem}|.
%
%    We will now look at the definition of the macro which is executed
%    at the beginning of the actual environment (in our example this
%    macro is |\Lem|). It should be noted that we use an
%    ``|\expandafter| trick'' to expand only certain parts of the
%    replacement text at the time of the definition.
%    \begin{macrocode}
    \def\@tempa{\global\@namedef{#1}}%
    \expandafter \@tempa \expandafter{%
%    \end{macrocode}
%    First, the macro that contains the current definitions of
%    |\@begintheorem| and |\@opargtheorem| should be called
%    up.  The name of this macro---as is already known---has the form
%    |\th@|\meta{theorem\ style}; therefore, it must be called by
%    \begin{macrocode}
      \csname th@\the \theorem@style
%    \end{macrocode}
%    In addition the default theorem font should be changeable, i.e.\
%    we have to insert the contents of |\theorem@bodyfont|. For
%    that reason, we expand even further, beyond |\endcsname|,
%    and thus insert the contents of the \textsf{token} register
%    |\theorem@bodyfont| in the replacement text.
%    \begin{macrocode}
            \expandafter \endcsname \the \theorem@bodyfont
%    \end{macrocode}
%    Now it is time to call the macro |\@thm| which takes over
%    the further processing.  It has two arguments: the current
%    counter name (in our example, \texttt{Lem}), and the text of the
%    label (in our example, \texttt{Lemma}).
%    \begin{macrocode}
     \@thm{#1}{#2}}%
%    \end{macrocode}
%    With this, the `sub-definition' is complete.  The macro
%    |\@endtheorem| ends a theorem environment and is, so far, nothing
%    but an |\endtrivlist|.  (Hence it is defined globally, and not
%    within the theorem styles.\footnote{This has to be changed as
%    soon as theorem styles that change \texttt{\bslash @endtheorem}
%    exist.  In such a case, all existing styles must be changed as
%    well since they will have to reset the macro.}) Therefore, we can
%    set it equivalent to the macro that ends the theorem set (in our
%    example, |\endLem|).  However, if some day theorem styles exist
%    that do change |\@endtheorem|, we would have to use the
%    commented-out line instead.
%    \begin{macrocode}
    \global \expandafter \let \csname end#1\endcsname \@endtheorem
%   \global\@namedef{end#1}{\@endtheorem}%
%    \end{macrocode}
%    With these commands all the required definitions are employed,
%    unless the test |\@ifdefinable| has failed.  Therefore, we
%    end the second argument of this macro and with it the definition
%    of |\@xnthm|.
%    \begin{macrocode}
   }}
%    \end{macrocode}
% \end{macro}
%
% \begin{macro}{\@ynthm}
%    The definition of |\@ynthm| is completely analogous. In
%    this case the new counter that is defined is not reset within
%    another counter; thus the definition of |\the...|
%    is simplified:
%    \begin{macrocode}
\gdef\@ynthm#1#2{\expandafter\@ifdefinable\csname #1\endcsname
   {\@definecounter{#1}%
    \expandafter\xdef\csname the#1\endcsname{\@thmcounter{#1}}%
%    \end{macrocode}
%    The rest of the definition corresponds literally to that of
%    |\@xnthm|:
%    \begin{macrocode}
    \def\@tempa{\global\@namedef{#1}}\expandafter \@tempa
     \expandafter{\csname th@\the \theorem@style \expandafter
     \endcsname \the\theorem@bodyfont \@thm{#1}{#2}}%
    \global \expandafter \let \csname end#1\endcsname \@endtheorem}}
%    \end{macrocode}
% \end{macro}
%
% \begin{macro}{\@othm}
%    The definition of |\@othm| does not contain anything new.
%    \begin{macrocode}
\gdef\@othm#1[#2]#3{%
%    \end{macrocode}
% \changes{v2.2b}{95/10/07}{Build in check for counter defined}
%    We do not define a new counter but instead use one that has
%    already been defined.  Thus the only definition we need is that
%    of this pseudo-counter (i.e.\ |\the|\meta{env.\ name}).
%    First we check if |#2| corresponds to a known counter name.
%    \begin{macrocode}
  \expandafter\ifx\csname c@#2\endcsname\relax
   \@nocounterr{#2}%
  \else
   \expandafter\@ifdefinable\csname #1\endcsname
   {\expandafter \xdef \csname the#1\endcsname
     {\expandafter \noexpand \csname the#2\endcsname}%
%    \end{macrocode}
%    All other parts of the definition can be adopted from
%    |\@xnthm|. We have to remember, though, that in this case
%    the name of the current counter and the theorem label have moved
%    to the second and third arguments.
%    \begin{macrocode}
    \def\@tempa{\global\@namedef{#1}}\expandafter \@tempa
     \expandafter{\csname th@\the \theorem@style \expandafter
     \endcsname \the\theorem@bodyfont \@thm{#2}{#3}}%
    \global \expandafter \let \csname end#1\endcsname \@endtheorem}%
  \fi}
%    \end{macrocode}
% \end{macro}
%
% \subsection{Macros that are employed in a {\envfont theorem}
%    environment}
%
% \begin{macro}{\@thm}
%    The macro |\@thm| has to increase the current counter.  Then,
%    depending on whether the environment has (or does not have) an
%    optional argument, it has to branch into either |\@begintheorem|
%    or |\@opargtheorem|.
%    \begin{macrocode}
\gdef\@thm#1#2{\refstepcounter{#1}%
%    \end{macrocode}
%    Now we start a {\envfont trivlist} environment, and give
%    |\@topsep| and |\@topsepadd| the values of the \textsf{skip}
%    registers |\theorempreskipamount| and |\theorempostskipamount|.
%    The value in |\@topsep| is the vertical space that is inserted by
%    the first (and only) |\item| in our |\trivlist| whilst
%    |\@topsepadd| is inserted by |\@endparenv| at the end of that
%    {\envfont trivlist} environment.  By using these registers, we
%    obtain the desired space around a {\envfont theorem} environment.
%    \begin{macrocode}
   \trivlist
   \@topsep \theorempreskipamount               % used by first \item
   \@topsepadd \theorempostskipamount           % used by \@endparenv
%    \end{macrocode}
%    Now we have to test whether an optional argument has been given.
% \changes{v2.0d}{89/07/18}{Forgotten percent added.}
%    \begin{macrocode}
   \@ifnextchar [%
%    \end{macrocode}
%    If there is an optional argument, we will call |\@ythm|, and
%    move the arguments read back into the input stream.
%    \begin{macrocode}
   {\@ythm{#1}{#2}}%
%    \end{macrocode}
%    If not, we call |\@begintheorem|. Its first argument is the
%    name of the theorem set (hence the second argument of
%    |\@thm|). Its second argument is the macro that produces the
%    current number.
%    \begin{macrocode}
   {\@begintheorem{#2}{\csname the#1\endcsname}\ignorespaces}}
%    \end{macrocode}
% \end{macro}
%
% \begin{macro}{\@xthm}
% \begin{macro}{\@ythm}
%    Both these macros were originally called by |\@thm|.  We do
%    not need |\@xthm| anymore, hence we reset it to
%    |\relax|. The definition of |\@ythm| has not changed at
%    all from its definition in \LaTeX. In order to make the macros
%    easier to understand, we will
%    nevertheless present it (commented out).
%    \begin{macrocode}
\global\let\@xthm\relax
% \def\@ythm#1#2[#3]{\@opargbegintheorem{#2}{\csname
%           the#1\endcsname}{#3}\ignorespaces}
%</package>
%    \end{macrocode}
%    The primitive |\ignorespaces| in |\@ythm| and
%    |\@thm| is needed to remove the spaces between the
%    |\begin{...}| and the actual text.
% \end{macro}
% \end{macro}
%
%
% \subsection{Definition of the theorem substyles}
%
% As already pointed out, the theorem substyles, defined below, are
% only loaded when necessary.  Note that all these substyles, except
% \texttt{plain}, have |\slshape| as the default body font.
%
% \subsubsection{The \texttt{plain} style}
%
% \label{dok:th@plain} As the following macros use |@|, we have
% to locally set the |\catcode| of this symbol to ``letter''.
% This happens within a group, so that we do not have to worry about
% which |\catcode| that symbol had before.
%    \begin{macrocode}
%<*thp>
\begingroup \makeatletter
%    \end{macrocode}
% Since we are now within a group, we must make all definitions
% globally.  First we make sure that \texttt{theorem.sty} is loaded.
% This will allow us to use this file as a document style option
% without having to call \texttt{theorem} itself as an option.  At the
% same time, we assure that at least version 2 is loaded, since
% |\theorem@style| was not defined in earlier versions.
%    \begin{macrocode}
\@ifundefined{theorem@style}{\input{theorem.sty}}{}
\ProvidesFile{thp.sty}
             [\FMithmInfo]
%    \end{macrocode}
%
%
% \begin{macro}{\th@plain}
% \begin{macro}{\@begintheorem}
% \begin{macro}{\@opargbegintheorem}
%    |\theoremstyle{plain}| corresponds to the original
%    definition, except that the distances to the surrounding text are
%    determined by the parameters |\theorempreskipamount| and
%    |\theorempostskipamount|.
%    First we set the default body font.
%    \begin{macrocode}
\gdef\th@plain{\normalfont\itshape
%    \end{macrocode}
%    Then we define |\@begintheorem| and
%    |\@opargbegintheorem|. These two macros define how the
%    header of a theorem is typeset.  |\@opargbegintheorem| will
%    be called if a \textsf{theorem} environment with an optional
%    argument is encountered; otherwise, the header is constructed by
%    calling |\@begintheorem|.  If one of these macros is
%    executed, we are within a \textsf{trivlist} environment started by
%    |\@thm|.  So the theorem header is produced with an
%    |\item| command.
%
%    Instead of specifying the header font directly, all standard
%    theorem styles use the |\theorem@headerfont| macro to allow
%    customization.  The extra space (|\labelsep|) is necessary
%    because of problems in the \textsf{trivlist} environment.
%    \begin{macrocode}
  \def\@begintheorem##1##2{%
        \item[\hskip\labelsep \theorem@headerfont ##1\ ##2]}%
%    \end{macrocode}
%    The definition of |\@opargbegintheorem| is completely
%    analogous. The only difference is the fact that there exists a
%    third argument (which is the optional parameter of the
%    environment and contains additional information about the
%    theorem).  Customarily we enclose it in parentheses.
%    \begin{macrocode}
\def\@opargbegintheorem##1##2##3{%
   \item[\hskip\labelsep \theorem@headerfont ##1\ ##2\ (##3)]}}
%    \end{macrocode}
% \end{macro}
% \end{macro}
% \end{macro}
%
% \noindent
% We conclude with an |\endgroup| to restore the
% |\catcode| of |@|.
%    \begin{macrocode}
\endgroup
%</thp>
%    \end{macrocode}
%
%
% \subsubsection{The \texttt{break} style}
%
% This style option is stored in the file |thb.sty|.
% For the next two lines see the documentation for |\th@plain| on
% page~\pageref{dok:th@plain}.
%    \begin{macrocode}
%<*thb>
\begingroup \makeatletter
\@ifundefined{theorem@style}{\input{theorem.sty}}{}
\ProvidesFile{thb.sty}
             [\FMithmInfo]
%    \end{macrocode}
%
% \begin{macro}{\th@break}
%    |\theoremstyle{break}| produces a line break after the name
%    of the theorem.  The font is |\slshape|.  Hence, we define
%    |\th@break| as follows:
%    \begin{macrocode}
\gdef\th@break{\normalfont\slshape
  \def\@begintheorem##1##2{\item[%
%    \end{macrocode}
%    We run into the following problem: it is not possible to create
%    the header with |\item[|\meta{title}|]| and then start
%    a new line by, for example, |\mbox{}\\|. Such a definition
%    will fail whenever a list environment follows immediately.  With
%    the above construction, the |\mbox{}| causes the switch
%    |@inlabel| (cf.\ definition of |\list| and
%    |\trivlist| in \cite{latex:source}) to be set to \texttt{false}
%    and so the following list will insert additional vertical
%    space (|\topskip|). This is quite annoying. Therefore, we
%    create the line break within the |\item|. In order to ensure
%    that the text will begin at the proper position in the following
%    line, we simply pretend that the label does not take any
%    room.\footnote{This will lead to problems whenever very high
%    symbols occurring in the line tower into the
%    heading. So, something else has to be done here sometime.}
%    \begin{macrocode}
       \rlap{\vbox{\hbox{\hskip \labelsep\theorem@headerfont ##1\ ##2}%
                   \hbox{\strut}}}]}%
%    \end{macrocode}
%    Again, the definition of |\@opargbegintheorem| is
%    completely analogous.
%    \begin{macrocode}
\def\@opargbegintheorem##1##2##3{%
  \item[\rlap{\vbox{\hbox{\hskip \labelsep \theorem@headerfont
                     ##1\ ##2\ (##3)}%
                    \hbox{\strut}}}]}}
\endgroup
%</thb>
%    \end{macrocode}
% \end{macro}
%
%
% \subsubsection{The \texttt{changebreak} style}
%
%    \begin{macrocode}
% This style option is stored in the file |thcb.sty|.
%    \begin{macrocode}
%<*thcb>
\begingroup \makeatletter
\@ifundefined{theorem@style}{\input{theorem.sty}}{}
\ProvidesFile{thcb.sty}
             [\FMithmInfo]
%    \end{macrocode}
%
% \begin{macro}{\th@changebreak}
%    The \texttt{change-break} theorem style is like \texttt{break}
%    but with interchange of theorem name and theorem number.  Thus we
%    define |\th@changebreak| as follows:
%    \begin{macrocode}
\gdef\th@changebreak{\normalfont\slshape
  \def\@begintheorem##1##2{\item
       [\rlap{\vbox{\hbox{\hskip\labelsep \theorem@headerfont ##2\ ##1}%
                    \hbox{\strut}}}]}%
\def\@opargbegintheorem##1##2##3{%
  \item[\rlap{\vbox{\hbox{\hskip\labelsep \theorem@headerfont
                     ##2\ ##1\ (##3)}%
                    \hbox{\strut}}}]}}
\endgroup
%</thcb>
%    \end{macrocode}
% \end{macro}
%
%
% \subsubsection{The \texttt{change} style}
%
% This style option is stored in the file |thc.sty|.
%    \begin{macrocode}
%<*thc>
\begingroup \makeatletter
\@ifundefined{theorem@style}{\input{theorem.sty}}{}
\ProvidesFile{thc.sty}
             [\FMithmInfo]
%    \end{macrocode}
%
% \begin{macro}{\th@change}
%    The \texttt{change} theorem style corresponds to the
%    \texttt{change break} style without a linebreak after the header.
%    To say it in another way, it's the same as the \texttt{plain}
%    style but with number and name interchanged and |\slshape| as the
%    default font.
%    \begin{macrocode}
\gdef\th@change{\normalfont\slshape
  \def\@begintheorem##1##2{\item
       [\hskip\labelsep \theorem@headerfont ##2\ ##1]}%
\def\@opargbegintheorem##1##2##3{%
  \item[\hskip\labelsep \theorem@headerfont ##2\ ##1\ (##3)]}}
\endgroup
%</thc>
%    \end{macrocode}
% \end{macro}
%
%
% \subsubsection{The \texttt{marginbreak} style}
%
% This style option is the one used most often at Mainz.  It is saved
% in the file |thmb.sty|.
%    \begin{macrocode}
%<*thmb>
\begingroup \makeatletter
\@ifundefined{theorem@style}{\input{theorem.sty}}{}
\ProvidesFile{thmb.sty}
             [\FMithmInfo]
%    \end{macrocode}
%
% \begin{macro}{\th@marginbreak}
%    The \texttt{margin break} style is nearly the same as the
%    \texttt{change break} style. The only difference is the placement
%    of the theorem number.  We use |\llap| to place it in the left
%    margin.
%
%    In this style |\labelsep| denotes the separation between the
%    number and the text.
%    \begin{macrocode}
\gdef\th@marginbreak{\normalfont\slshape
  \def\@begintheorem##1##2{\item
       [\rlap{\vbox{\theorem@headerfont
                    \hbox{\llap{##2}\hskip\labelsep ##1}%
                    \hbox{\strut}}}]}%
\def\@opargbegintheorem##1##2##3{%
  \item[\rlap{\vbox{\theorem@headerfont
                    \hbox{\llap{##2}\hskip\labelsep ##1\ (##3)}%
                    \hbox{\strut}}}]}}
\endgroup
%</thmb>
%    \end{macrocode}
% \changes{v2.1c}{90/02/17}{`endgroup outside macrocode env.}
% \end{macro}
%
%
%
% \subsubsection{The \texttt{margin} style}
%
% This style option is stored in the file |thm.sty|.
%    \begin{macrocode}
%<*thm>
\begingroup \makeatletter
\@ifundefined{theorem@style}{\input{theorem.sty}}{}
\ProvidesFile{thm.sty}
             [\FMithmInfo]
%    \end{macrocode}
%
% \begin{macro}{\th@margin}
%    Again this is only a variant of the theorem styles described
%    above without any new ideas.
%    \begin{macrocode}
\gdef\th@margin{\normalfont\slshape
  \def\@begintheorem##1##2{\item
       [\theorem@headerfont \llap{##2}\hskip\labelsep ##1]}%
\def\@opargbegintheorem##1##2##3{%
  \item[\theorem@headerfont \llap{##2}\hskip\labelsep ##1\ (##3)]}}
\endgroup
%</thm>
%    \end{macrocode}
% \end{macro}
%
%
%
%
% \subsection{Final Definitions}
%
% \begin{macro}{\theorempreskipamount}
% \begin{macro}{\theorempostskipamount}
%    The \textsf{skip} parameters that regulate the vertical empty space
%    before and after the {\envfont theorem} environment have
%    to be allocated as well.
% \changes{v2.0f}{89/07/20}{Unnecessary `global' removed.}
%    \begin{macrocode}
%<*package>
\newskip\theorempreskipamount
\newskip\theorempostskipamount
%    \end{macrocode}
%    Since we have used the same values for all theorem sets, we now can
%    assign them.
%    \begin{macrocode}
\global\setlength\theorempreskipamount{12pt plus 5pt minus 3pt}
\global\setlength\theorempostskipamount{8pt plus 3pt minus 1.5pt}
%    \end{macrocode}
% \end{macro}
% \end{macro}
%
%
% \begin{macro}{\@endtheorem}
%    The same holds for the macro |\@endtheorem|, which ends a
%    {\envfont theorem} environment. Since it is the same for all
%    theorem sets, it is removed from the macros |\th@|\meta{style}.
%    It simply ends the {\envfont trivlist} environment, which was
%    begun in |\@thm|.
%    \begin{macrocode}
\global\let\@endtheorem=\endtrivlist
%    \end{macrocode}
% \end{macro}
%
%
%
% \begin{macro}{\@preamblecmds}
%    All macros defined above are to be used only in the preamble.
%    Therefore, we insert them in |\@preamblecmds| which will disable
%    them at begin document. This is done by the internal \LaTeXe{}
%    command |\@onlypreamble|.
% \changes{v2.0g}{89/09/19}{`theoremheaderfont and `theorembodyfont
%    added.}
% \changes{v2.2a}{94/02/03}{Use \LaTeXe{} interface}
%    \begin{macrocode}
\@onlypreamble\@xnthm
\@onlypreamble\@ynthm
\@onlypreamble\@othm
\@onlypreamble\newtheorem
\@onlypreamble\theoremstyle
\@onlypreamble\theorembodyfont
\@onlypreamble\theoremheaderfont
%    \end{macrocode}
% \end{macro}
%
% Finally we declare the \texttt{plain} theorem style to be the default.
% \changes{v2.1b}{90/06/04}{Fixed missing `item by declaring default
%    style.}
%    \begin{macrocode}
\theoremstyle{plain}
%</package>
%    \end{macrocode}
%
% \Finale

\endinput
}{}
\ProvidesFile{thcb.sty}
             [\FMithmInfo]
%    \end{macrocode}
%
% \begin{macro}{\th@changebreak}
%    The \texttt{change-break} theorem style is like \texttt{break}
%    but with interchange of theorem name and theorem number.  Thus we
%    define |\th@changebreak| as follows:
%    \begin{macrocode}
\gdef\th@changebreak{\normalfont\slshape
  \def\@begintheorem##1##2{\item
       [\rlap{\vbox{\hbox{\hskip\labelsep \theorem@headerfont ##2\ ##1}%
                    \hbox{\strut}}}]}%
\def\@opargbegintheorem##1##2##3{%
  \item[\rlap{\vbox{\hbox{\hskip\labelsep \theorem@headerfont
                     ##2\ ##1\ (##3)}%
                    \hbox{\strut}}}]}}
\endgroup
%</thcb>
%    \end{macrocode}
% \end{macro}
%
%
% \subsubsection{The \texttt{change} style}
%
% This style option is stored in the file |thc.sty|.
%    \begin{macrocode}
%<*thc>
\begingroup \makeatletter
\@ifundefined{theorem@style}{% \iffalse meta-comment
%
% Copyright (C) 1993-2024
%
% The LaTeX Project and any individual authors listed elsewhere
% in this file.
%
% This file is part of the Standard LaTeX `Tools Bundle'.
% -------------------------------------------------------
%
% It may be distributed and/or modified under the
% conditions of the LaTeX Project Public License, either version 1.3c
% of this license or (at your option) any later version.
% The latest version of this license is in
%    https://www.latex-project.org/lppl.txt
% and version 1.3c or later is part of all distributions of LaTeX
% version 2005/12/01 or later.
%
% The list of all files belonging to the LaTeX `Tools Bundle' is
% given in the file `manifest.txt'.
%
% \fi
%
%%
%
%
\def\FMithmInfo{2023/07/05 v2.2c Theorem extension package (FMi)}
%
% \ProvidesFile{theorem.dtx}[\FMithmInfo]
%\iffalse   % this is a METACOMMENT !
%%
%% Package `theorem' to use with LaTeX2e
%% Copyright (C) 1989-1995 Frank Mittelbach, all rights reserved.
%<+package>\@ifundefined{theorem@style}{}{\endinput}
%<+package>\NeedsTeXFormat{LaTeX2e}[1995/06/01]
%<+package>\ProvidesPackage{theorem}
%<+package>         [\FMithmInfo]
%
% \fi
%
% \changes{v1.0}{87/05/07}{First draft.}
% \changes{v1.0a}{87/05/07}{`thistheoremstyle' added.}
% \changes{v1.1a}{87/07/29}{New concept: redefine the internal
%                          \protect\LaTeX{} commands `@xnthm' etc.}
% \changes{v1.1b}{87/07/31}{`theoremfont' used to specify font.}
% \changes{v1.2a}{88/01/08}{`@endtheorem' now globally defined.}
% \changes{v1.2b}{88/01/11}{`newtheorem' only allowed in the preamble.}
% \changes{v1.3a}{88/02/14}{`theoremfont' now a toks register.}
%
% \changes{v2.0a}{88/11/16}{New implementation and new documentation.}
% \changes{v2.0a}{88/11/16}{No layout changes but internal modifications
%                           old substyles won't work.}
% \changes{v2.0b}{89/01/30}{`global' used for all definitions.}
% \changes{v2.0c}{89/06/15}{Documentation translated to English.
%                            Thanks to LLL and Nina}
% \changes{v2.1a}{90/02/15}{Support for new font selection.}
% \changes{v2.2a}{94/02/03}{Cleanup for LaTeX2e}
%
% \DoNotIndex{\@endparenv,\@ifdefinable,\@ifnextchar,\@ifundefined}
% \DoNotIndex{\@namedef,\@nameuse,\@spaces,\@tempa,\@warning}
% \DoNotIndex{\ }
% \DoNotIndex{\begingroup,\bfseries}
% \DoNotIndex{\csname}
% \DoNotIndex{\def,\do,\docdate}
% \DoNotIndex{\endcsname,\endgroup,\endinput,\endtrivlist,\expandafter}
% \DoNotIndex{\filedate,\fileversion}
% \DoNotIndex{\gdef,\global}
% \DoNotIndex{\hbox,\hskip}
% \DoNotIndex{\ignorespaces,\input,\itshape,\item}
% \DoNotIndex{\labelsep,\let,\llap}
% \DoNotIndex{\makeatletter}
% \DoNotIndex{\noexpand}
% \DoNotIndex{\refstepcounter,\relax,\rlap}
% \DoNotIndex{\setlength,\slshape,\space,\string,\strut}
% \DoNotIndex{\the,\trivlist,\typeout}
% \DoNotIndex{\vbox}
% \DoNotIndex{\xdef}
%
% \def\deflabel#1{\ttfamily #1\hfill}
% \def\deflist#1{\begin{list}{}{\settowidth\labelwidth{\ttfamily #1}%
%                               \setlength\leftmargin\labelwidth
%                               \addtolength\leftmargin\labelsep
%                               \let\makelabel\deflabel}}
% \def\enddeflist{\end{list}}
% \def\envfont{\normalfont\ttfamily}
%
% \MakeShortVerb{\|}
%
% \newcommand{\TUB}{{\sl TUGboat\/}}
%
% \GetFileInfo{theorem.dtx}
% \title{An Extension of the
%        \LaTeX{} theorem environment\thanks{This file
%        has version number \fileversion, last
%        revised \filedate.}}
% \author{Frank Mittelbach}
%
% \MaintainedByLaTeXTeam{tools}
% \maketitle
%
% \begin{abstract}
%    The macros described in this paper yield an extension of the
%    \LaTeX{} theorem mechanism. It is designed to satisfy the
%    different requirements of various journals. Thus, the
%    layout of the ``theorems'' can be manipulated by determining a
%    ``style''. This article describes not only the use, but
%    also the definition, of the necessary macros.
% \end{abstract}
%
% \section*{Preface to version 2.2}
%
% For \LaTeXe{} this package did not need any fundamental changes. I
% only modified the messages generated so that theorem layout styles
% will show up with the |\listfiles| command and cleaned the section
% on the New Font Selection Scheme since this is now included in
% \LaTeX.
%
% \section*{Preface to version 2.1}
%
% This version is identical to 2.0g described in \TUB\ 10\#3 except
% for some internal defaults which are now set depending on the used
% font selection scheme.
%
% This was done to avoid unpleasant surprises if the new font selection
% scheme is in force. For further details see
% section~\ref{sec:newfonts} and \cite{bk:GMS94}.
%
% \section{Introduction}
%
% For our purposes here, ``theorems'' are labelled enunciations,
% often set off from the main text by extra space and a font change.
% Theorems, corollaries, conjectures, definitions, and remarks
% are all instances of ``theorems''.  The ``header'' of these structures
% is composed of a label (such as \textsc{Theorem} or \textsc{Remark})
% and a number which serializes an item in the sequence of items
% with the same label.
%
% Shortly after the introduction of \LaTeX{} at the Fachbereich
% Mathematik in Mainz, the desire to manipulate the layout of
% ``theorems'' arose.  In Mainz, the following two conventions came
% into general use:
% \begin{enumerate}
%   \item The number of the theorem is shown in the margin.
%   \item There is a line break at
%         the end of the theorem header.
% \end{enumerate}
%
% Additionally, some journals require different formats which depend on
% the ``sort of theorem'': e.g.\ often remarks and definitions are set
% in |\upshape|, while |\itshape| is employed for main theorems.
%
% Confronted with these requirements, a theorem environment was
% developed in Mainz which allows separate determination of the
% layout of the ``theorems sets'', comparable to |\pagestyle|.
%
%
% \section{The user interface}
% \subsection{Defining new theorem sets}
%
% \DescribeMacro\newtheorem
% As in the original \LaTeX{} version, the command |\newtheorem|
% defines a new ``theorem set'' or ``theorem-like structure''.
% Two required arguments name the new environment and give the text
% to be typeset with each instance of the new ``set'', while
% an optional argument determines how
% the ``set'' is enumerated:
% \begin{description}
%    \item[\ttfamily \bslash newtheorem\{foo\}\{bar\}]
%       The theorem set {\envfont foo} (whose name is \texttt{bar})
%       uses its own counter.
%    \item[\ttfamily \bslash newtheorem\{foo2\}{[foo]}\{bar2\}]
%       The theorem set {\envfont foo2} (printed name \texttt{bar2})
%       uses the same counter as the theorem set \texttt{foo}.
%    \item[\ttfamily \bslash newtheorem\{foo3\}\{bar3\}{[section]}]
%       The theorem set {\envfont foo3} (printed name \texttt{bar3}) is
%       enumerated within the counter \texttt{section}, i.e.\ with every
%       new |\section| the enumeration begins again with 1, and
%       the enumeration is composed from the section-number and the
%       theorem counter itself.
% \end{description}
%
% \DescribeMacro\theoremstyle
% Additionally, the command |\theoremstyle| can define the layout
% of various, or all, theorem sets.  It should be noted that any
% theorem set defined by |\newtheorem| is typeset in the
% |\theoremstyle| that is current at the time of the definition.
% Thus, the following
% \begin{verbatim}
%    \theoremstyle{break}        \newtheorem{Cor}{Corollary}
%    \theoremstyle{plain}        \newtheorem{Exa}{Example}[section]
%\end{verbatim}                             ^^A No blank at line start
% leads to the result that the set {\envfont Cor} is formatted in the
% style \texttt{break}, while the set {\envfont Exa} and all the
% following ones are formatted in the style \texttt{plain}, unless
% another |\theoremstyle| follows.  Since the definitions installed by
% |\newtheorem| are global, one also can limit |\theoremstyle| locally
% by grouping braces.
%
% \DescribeMacro\theorembodyfont
% The choice of the font for the theorem body is completely
% independent of the chosen |\theoremstyle|; this has proven to
% be very advantageous.  For example,
% \begin{verbatim}
%   {\theorembodyfont{\upshape}          \newtheorem{Rem}{Remark}}
%\end{verbatim}
% defines a theorem set {\envfont Rem}, which will be set in
% |\upshape| in the current layout (which in our example is
% \texttt{plain}).  As with |\theoremstyle|, the |\theorembodyfont|
% chosen is that current at the time of |\newtheorem|.  If
% |\theorembodyfont| is not specified or one defines
% |\theorembodyfont{}|, then the font used will be that defined by the
% |\theoremstyle|.
%
% \DescribeMacro\theoremheaderfont It is also possible to customize
% the font used for the theorem headers.  This is, however, a global
% declaration, and therefore there should be at most one
% |\theoremheaderfont| declaration in the preamble.\footnote{If it is
% actually necessary to have different header fonts, one has to define
% new theorem styles (substituting the desired font) or specify the
% information directly in the \texttt{\bslash newtheorem} declaration
% (the unclean variant).}
%
% \DescribeMacro\theorempreskipamount
% \DescribeMacro\theorempostskipamount Two additional parameters
% affect the vertical space around the theorem environments:
% |\theorempreskipamount| and |\theorempostskipamount| define,
% respectively, the spacing before and after such an environment.
% These parameters apply for all theorem sets and can be manipulated
% with the ordinary length macros.  They are rubber lengths
% (`\textsf{skips}'), and therefore can contain \texttt{plus} and
% \texttt{minus} parts.
%
% Since the definition of theorem sets should---most sensibly---be
% placed in the preamble, we only allow installation there.
% It is therefore possible to release the memory used here after
% |\begin{document}|, in order to make room for other
% applications.
%
%
% \subsection{Existing theorem styles}
%
% The following theorem styles exist to date:
% \begin{deflist}{marginbreak}
%    \item[plain]
%       This theorem style emulates the original \LaTeX{} definition,
%       except that additionally the parameters
%       |\theorem...skipamount| are used.
%    \item[break]
%       In this style, the theorem header is followed by a line
%       break.
%    \item[marginbreak]
%       The theorem number is set in the margin, and there is a line
%       break as in \texttt{break}.
%    \item[changebreak]
%       Like \texttt{break}, but with header number and text
%       interchanged.
%    \item[change]
%       Header number and text are interchanged, without a line break.
%    \item[margin]
%       The number is set in the left margin, without a line break.
% \end{deflist}
% All styles (except \texttt{plain}) select |\slshape| as the default
% |\theorembodyfont|.
%
%
% \subsection{Examples}
%
% Given the above theorem sets {\envfont Cor}, {\envfont Exa}
% and {\envfont Rem},
% suppose that the preamble also contains the declarations:
% \begin{verbatim}
%    \theoremstyle{marginbreak}   \newtheorem{Lem}[Cor]{Lemma}
%    \theoremstyle{change}
%    \theorembodyfont{\itshape}        \newtheorem{Def}[Cor]{Definition}
%
%    \theoremheaderfont{\scshape}
%\end{verbatim}
% Then the following are some typical examples of the typeset output
% resulting from their use.
%
% \begin{Cor}
%    This is a sentence typeset in the theorem environment {\envfont
%    Cor}.
% \end{Cor}
%
% \begin{Exa}
%    This is a sentence typeset in the theorem environment {\envfont
%    Exa}.
% \end{Exa}
%
% \begin{Rem}
%    This is a sentence typeset in the theorem environment {\envfont
%    Rem}.
% \end{Rem}
%
% \begin{Lem}[Ben User]
%    This is a sentence typeset in the theorem environment {\envfont
%    Lem}.
% \end{Lem}
%
% \begin{Def}[Very impressive Definition]
%    This is a sentence typeset in the theorem environment {\envfont
%    Def}.
% \end{Def}
%
% The last two examples show the effect of the optional argument to
% a theorem environment (it is the text typeset in parentheses).
%
%
% \section{Special Considerations}
% \label{sec:newfonts}
%
% Theoremheader and body are implemented as a unit. This means that
% the |\theoremheaderfont|  will inherit characteristics of the
% |\theorembodyfont| in \LaTeXe. Thus, if for example
% |\theorembodyfont| is |\itshape| and |\theoremheaderfont|
% is |\bfseries| the font selected for the header will have the
% characteristics `bold extended italic'. If this is not desired one
% should set the |\theoremheaderfont| to something like
% \begin{verbatim}
%      \theoremheaderfont{\normalfont\bfseries}
%\end{verbatim}
% i.e.\ supplying all necessary font information explicitly.
%
%
%
% \section{Acknowledgements}
%
% The publication of this set of macros was only possible with the
% help of Christina Busse (translating the manuscript into English),
% Joachim Pense (playing the r\^ole of typist), Chris Rowley
% (looking everything over) and many others providing useful
% suggestions.
%
% \MaybeStop{%
%    \begin{thebibliography}{9}
%       \bibitem{bk:GMS94} \textsc{M.~Goossens}, \textsc{F.~Mittelbach}
%         and \textsc{A.~Samarin}.
%         \newblock The \LaTeX{} Companion.
%         \newblock
%         Addison-Wesley, Reading, Massachusetts, 1994.
%       \bibitem{latex:source} \textsc{Lamport, Leslie.} \newblock
%                               \texttt{latex.tex}, version 2.09, date
%                               Feb. 1990.
%    \end{thebibliography}
%    \PrintIndex
%    \PrintChanges}
%
%
% \section{The documentation driver file}
%
% The next bit of code contains the documentation driver file for
% \TeX{}, i.e., the file that will produce the documentation you are
% currently reading. It will be extracted from this file by the
% \texttt{docstrip} program. Since it is the first code in the file
% one can alternatively process this file directly with \LaTeXe{} to
% obtain the documentation.
%
%    \begin{macrocode}
%<*driver>
\documentclass{ltxdoc}

\usepackage{theorem}

%<+driver>% The next few lines define theorem sets which are used
%<+driver>% in the example section of the documentation.

\theoremstyle{break}       \newtheorem{Cor}{Corollary}
\theoremstyle{plain}       \newtheorem{Exa}{Example}[section]
{\theorembodyfont{\upshape}\newtheorem{Rem}{Remark}}
\theoremstyle{marginbreak} \newtheorem{Lem}[Cor]{Lemma}
\theoremstyle{change}
\theorembodyfont{\itshape} \newtheorem{Def}[Cor]{Definition}

\theoremheaderfont{\scshape}

\RecordChanges

\begin{document}
   \DocInput{theorem.dtx}
\end{document}
%</driver>
%    \end{macrocode}
%
% \section{Definition of the Macros}
%
% If the file has been loaded before, we abort immediately. If not the
% package announces itself (this is actually done at the very top of
% the file---the way it is done isn't good style so don't copy it).
% \changes{v2.0e}{89/07/19}{Spaces removed from `typeout'.}
% \changes{v2.2b}{95/11/19}{Announce moved to top of file}
%    \begin{macrocode}
%<*package>
%\@ifundefined{theorem@style}{}{\endinput}
%\def\FMithmInfo{1995/11/19 v2.2b Theorem extension package (FMi)}
%\ProvidesPackage{theorem}[\FMithmInfo]
%    \end{macrocode}
%
%
% \subsection{Definition of theorem styles and fonts}
%
% All the definitions in this file are done globally to allow
% inputting this file inside a group.
%
% \begin{macro}{\theoremstyle}
%    Before a theorem style can be installed, the chosen style must be
%    known. For that reason, we must test to see that
%    |\th@|\meta{style} is known or, more precisely, that it is
%    different from |\relax|. If the style is not known then
%    |\th@plain| is used.
%    \begin{macrocode}
\gdef\theoremstyle#1{%
   \@ifundefined{th@#1}{\@warning
          {Unknown theoremstyle `#1'. Using `plain'}%
          \theorem@style{plain}}%
%    \end{macrocode}
%    We save the theorem style to be used in the token register
%    |\theorem@style|.
%    \begin{macrocode}
      {\theorem@style{#1}}%
%    \end{macrocode}
%    Now we ``evaluate'' the theorem style: this means, we call the
%    macro |\th@|\meta{style} which will activate the relevant
%    definitions which are contained in a separate file.  This is done
%    in a group to suppress changes to the current font.  This could
%    otherwise pose problems together with the new font selection
%    scheme\footnote{When I printed the original article using the new
%    font selection scheme I ended with a document with slanted
%    typefaces (text headings and all) simply because one of the
%    theorem styles used \texttt{\bslash sl} at toplevel.} if the
%    |\th@|\meta{style} is evaluated a second time.
% \changes{v2.1a}{90/02/15}{Group added around `th@..}
% \changes{v2.1a}{90/02/15}{Reseting of `..@begintheorem removed.}
%    \begin{macrocode}
      \begingroup
        \csname th@\the\theorem@style \endcsname
      \endgroup}
%    \end{macrocode}
% \end{macro}
%
%
% \begin{macro}{\@begintheorem}
% \changes{v2.1a}{90/02/15}{Relaxed at toplevel.}
% \begin{macro}{\@opargbegintheorem}
% \changes{v2.1a}{90/02/15}{Relaxed at toplevel.}
%    We reset |\@begintheorem| and |\@opargbegintheorem| to
%    |\relax| since these commands are no longer necessary at
%    toplevel. This will save a few tokens.
%    \begin{macrocode}
\global\let\@begintheorem\relax
\global\let\@opargbegintheorem\relax
%    \end{macrocode}
% \end{macro}
% \end{macro}
%
% \begin{macro}{\theorem@style}
%    Obviously the \textsf{token} register used above has to be
%    allocated. To assure the utmost compatibility with the
%    original \LaTeX{} definition, we set the default theorem style to
%    \texttt{plain}, which implements the usual \LaTeX{} convention.
% \changes{v2.0f}{89/07/20}{Unnecessary `global' removed.}
%    \begin{macrocode}
\newtoks\theorem@style
\global\theorem@style{plain}
%    \end{macrocode}
% \end{macro}
%
% \begin{macro}{\theorembodyfont}
% \begin{macro}{\theorem@bodyfont}
%    If the |\theorembodyfont| is set by the user then it should not
%    interact with the default font set in the theorem style. When the
%    new font selection is in force this may happen if, for example,
%    the default is |\itshape| and the new |\theorembodyfont| is
%    |\sffamily|. So we add a |\reset@font| command in front of the
%    user definition.
% \changes{v2.0f}{89/07/20}{Unnecessary `global' removed.}
% \changes{v2.0g}{89/09/19}{`theoremfont to `theorembodyfont.}
% \changes{v2.1c}{92/06/22}{`reset@font added and logic changed.}
% \changes{v2.1d}{92/10/28}{Reset to empty if arg is empty}
%    \begin{macrocode}
\gdef\theorembodyfont#1{%
%    \end{macrocode}
%    We check if the argument supplied is empty and if so put nothing
%    into the |\theorem@bodyfont| token register to allow for
%    |\theorembodyfont{}| as a mean of using the default of the
%    current |\theoremstyle|.
%    \begin{macrocode}
   \def\@tempa{#1}%
   \ifx\@tempa\@empty
    \theorem@bodyfont{}%
   \else
    \theorem@bodyfont{\reset@font#1}%
   \fi
   }
\newtoks\theorem@bodyfont
\global\theorem@bodyfont{}
%    \end{macrocode}
% \end{macro}
% \end{macro}
%
%
% \begin{macro}{\theoremheaderfont}
% \changes{v2.0g}{89/09/19}{Macro added.}
%    The font for the theorem headers is handled differently because
%    this definition applies to all theorem styles.
%    \begin{macrocode}
\gdef\theoremheaderfont#1{\gdef\theorem@headerfont{#1}%
%    \end{macrocode}
%    After using the macro once it is redefined to produce an error
%    message.
%    \begin{macrocode}
       \gdef\theoremheaderfont##1{%
        \typeout{\string\theoremheaderfont\space should be used
                 only once.}}}
%    \end{macrocode}
% \end{macro}
%
% \begin{macro}{\theorem@headerfont}
% \changes{v2.0g}{89/09/19}{Macro added.}
%    To set the |\theorem@headerfont| default we first test if
%    the new fontselection scheme is in force.
% \changes{v2.1a}{90/02/15}{Support for new font selection.}
%    \begin{macrocode}
\ifx\upshape\undefined
%    \end{macrocode}
%    If not we define it to expand into |\bfseries|. We don't use
%    |\let| just in case a following style option redefines this
%    macro.
%    \begin{macrocode}
\gdef\theorem@headerfont{\bfseries}
%    \end{macrocode}
%    Otherwise we reset the current shape before calling |\bfseries|.
%    \begin{macrocode}
\else \gdef\theorem@headerfont{\normalfont\bfseries}\fi
%    \end{macrocode}
% \end{macro}
%
% \begin{macro}{\th@plain}
% \begin{macro}{\th@break}
% \begin{macro}{\th@marginbreak}
% \begin{macro}{\th@changebreak}
% \begin{macro}{\th@change}
% \begin{macro}{\th@margin}
%    The different styles are defined in macros such as
%    |\th@plain|.  Since memory space is precious in
%    ``non-Big-versions'', we have to avoid offering too many unused
%    definitions.  Therefore we define these styles in separate files
%    that can be loaded on demand.  Thus the commands themselves only
%    load these files. We use |\@input@|, a \LaTeXe{} internal command
%    that ensures that the file will be listed with |\listfiles|.
% \changes{v2.2a}{94/02/02}{Use `@input@ to load theorem layout files}
%    \begin{macrocode}
\gdef\th@plain{\@input@{thp.sty}}
\gdef\th@break{\@input@{thb.sty}}
\gdef\th@marginbreak{\@input@{thmb.sty}}
\gdef\th@changebreak{\@input@{thcb.sty}}
\gdef\th@change{\@input@{thc.sty}}
\gdef\th@margin{\@input@{thm.sty}}
%    \end{macrocode}
%    This list will be expanded when new styles become
%    available.  For testing,  just append new theorem substyles
%    as document options.
% \end{macro}
% \end{macro}
% \end{macro}
% \end{macro}
% \end{macro}
% \end{macro}
%
%
%
% \subsection{Definition of a new theorem set}
%
% As already pointed out, a new theorem environment can be defined in
% three different ways:
% \begin{verbatim}
%    \newtheorem{Lem}{Lemma}
%    \newtheorem{Lem}{Lemma}[section]
%    \newtheorem{Lem}[Theorem]{Lemma}
%\end{verbatim}
% The function of the macro |\newtheorem| is to recognize these cases
% and then to branch into one of the three macros |\@ynthm|, |\@xnthm|
% or |\@othm|.  This mechanism is adopted unchanged from
% \cite{latex:source}; the essential point here is that, for example,
% in the second case, the arguments \texttt{Lem}, \texttt{Lemma} and
% \texttt{section} are passed over to the macro |\@xnthm|.
%
% We inspect this case first because the others present fewer problems,
% and thus are easily derived from this one.
%
% \begin{macro}{\@xnthm}
%    For our example arguments, the macro |\@xnthm| must fulfill
%    the following:
%    \begin{itemize}
%       \item
%          Define a new \LaTeX-counter `Lem'
%       \item
%          reset this counter within a |\section|
%       \item
%          define the macro |\theLem|
%       \item
%          define the environment macros |\Lem| and
%          |\endLem| using the current
%          |\theoremstyle| and |\theorem@bodyfont|.
%    \end{itemize}
%    Obviously, all this should happen only if the first argument
%    of |\@xnthm| (i.e.\ \texttt{Lem} in our example) is chosen so
%    as not to conflict with any previously defined commands or
%    environments.  This test is performed by the \LaTeX{} macro
%    |\@ifdefinable|.
%    \begin{macrocode}
\gdef\@xnthm#1#2[#3]{\expandafter\@ifdefinable\csname #1\endcsname
%    \end{macrocode}
%    Therefore, the first argument of |\@ifdefinable| is the
%    expansion (in the example, |\Lem|) of
%    |\csname||#1\endcsname|.  The second argument is
%    executed only if the test has been completed successfully.
%    \begin{macrocode}
   {%
%    \end{macrocode}
%    Now we define the new counter. The names of the \LaTeX{}
%    macros employed should speak for themselves:
% \changes{v2.2b}{1995/10/07}{Use \cs{@newctr}}
% \changes{v2.2c}{1995/11/23}{Use \cs{@newctr} properly.}
%    \begin{macrocode}
    \@definecounter{#1}\@newctr{#1}[#3]%
%    \end{macrocode}
%    Using |\@newctr| will give a proper error message if the counter
%    in |#3| is not defined.
%    In defining `|\theLem|' we must generate the desired macro
%    name by use of |\expandafter| and |\csname|.
%    \begin{macrocode}
    \expandafter\xdef\csname the#1\endcsname
%    \end{macrocode}
%    An |\xdef| is used in order to make the definition global,
%    and to ensure that it contains the replacement texts of
%    |\@thmcountersep| and |\@thmcounter|.\footnote{These
%    two macros can be defined by the document style.  Their default
%    values produce a `\texttt{.}' as separation and an arabic
%    representation of the number.} However, not everything should be
%    expanded. For example, it saves space to use |\thesection|
%    instead of its---at times---lengthy expansion.
%    \begin{macrocode}
      {\expandafter \noexpand \csname the#3\endcsname
       \@thmcountersep \@thmcounter{#1}}%
%    \end{macrocode}
%    Thus with the defaults of \LaTeX{}, |\theLem| would be
%    replaced by the command sequence |\thesection.\arabic{Lem}|.
%
%    We will now look at the definition of the macro which is executed
%    at the beginning of the actual environment (in our example this
%    macro is |\Lem|). It should be noted that we use an
%    ``|\expandafter| trick'' to expand only certain parts of the
%    replacement text at the time of the definition.
%    \begin{macrocode}
    \def\@tempa{\global\@namedef{#1}}%
    \expandafter \@tempa \expandafter{%
%    \end{macrocode}
%    First, the macro that contains the current definitions of
%    |\@begintheorem| and |\@opargtheorem| should be called
%    up.  The name of this macro---as is already known---has the form
%    |\th@|\meta{theorem\ style}; therefore, it must be called by
%    \begin{macrocode}
      \csname th@\the \theorem@style
%    \end{macrocode}
%    In addition the default theorem font should be changeable, i.e.\
%    we have to insert the contents of |\theorem@bodyfont|. For
%    that reason, we expand even further, beyond |\endcsname|,
%    and thus insert the contents of the \textsf{token} register
%    |\theorem@bodyfont| in the replacement text.
%    \begin{macrocode}
            \expandafter \endcsname \the \theorem@bodyfont
%    \end{macrocode}
%    Now it is time to call the macro |\@thm| which takes over
%    the further processing.  It has two arguments: the current
%    counter name (in our example, \texttt{Lem}), and the text of the
%    label (in our example, \texttt{Lemma}).
%    \begin{macrocode}
     \@thm{#1}{#2}}%
%    \end{macrocode}
%    With this, the `sub-definition' is complete.  The macro
%    |\@endtheorem| ends a theorem environment and is, so far, nothing
%    but an |\endtrivlist|.  (Hence it is defined globally, and not
%    within the theorem styles.\footnote{This has to be changed as
%    soon as theorem styles that change \texttt{\bslash @endtheorem}
%    exist.  In such a case, all existing styles must be changed as
%    well since they will have to reset the macro.}) Therefore, we can
%    set it equivalent to the macro that ends the theorem set (in our
%    example, |\endLem|).  However, if some day theorem styles exist
%    that do change |\@endtheorem|, we would have to use the
%    commented-out line instead.
%    \begin{macrocode}
    \global \expandafter \let \csname end#1\endcsname \@endtheorem
%   \global\@namedef{end#1}{\@endtheorem}%
%    \end{macrocode}
%    With these commands all the required definitions are employed,
%    unless the test |\@ifdefinable| has failed.  Therefore, we
%    end the second argument of this macro and with it the definition
%    of |\@xnthm|.
%    \begin{macrocode}
   }}
%    \end{macrocode}
% \end{macro}
%
% \begin{macro}{\@ynthm}
%    The definition of |\@ynthm| is completely analogous. In
%    this case the new counter that is defined is not reset within
%    another counter; thus the definition of |\the...|
%    is simplified:
%    \begin{macrocode}
\gdef\@ynthm#1#2{\expandafter\@ifdefinable\csname #1\endcsname
   {\@definecounter{#1}%
    \expandafter\xdef\csname the#1\endcsname{\@thmcounter{#1}}%
%    \end{macrocode}
%    The rest of the definition corresponds literally to that of
%    |\@xnthm|:
%    \begin{macrocode}
    \def\@tempa{\global\@namedef{#1}}\expandafter \@tempa
     \expandafter{\csname th@\the \theorem@style \expandafter
     \endcsname \the\theorem@bodyfont \@thm{#1}{#2}}%
    \global \expandafter \let \csname end#1\endcsname \@endtheorem}}
%    \end{macrocode}
% \end{macro}
%
% \begin{macro}{\@othm}
%    The definition of |\@othm| does not contain anything new.
%    \begin{macrocode}
\gdef\@othm#1[#2]#3{%
%    \end{macrocode}
% \changes{v2.2b}{95/10/07}{Build in check for counter defined}
%    We do not define a new counter but instead use one that has
%    already been defined.  Thus the only definition we need is that
%    of this pseudo-counter (i.e.\ |\the|\meta{env.\ name}).
%    First we check if |#2| corresponds to a known counter name.
%    \begin{macrocode}
  \expandafter\ifx\csname c@#2\endcsname\relax
   \@nocounterr{#2}%
  \else
   \expandafter\@ifdefinable\csname #1\endcsname
   {\expandafter \xdef \csname the#1\endcsname
     {\expandafter \noexpand \csname the#2\endcsname}%
%    \end{macrocode}
%    All other parts of the definition can be adopted from
%    |\@xnthm|. We have to remember, though, that in this case
%    the name of the current counter and the theorem label have moved
%    to the second and third arguments.
%    \begin{macrocode}
    \def\@tempa{\global\@namedef{#1}}\expandafter \@tempa
     \expandafter{\csname th@\the \theorem@style \expandafter
     \endcsname \the\theorem@bodyfont \@thm{#2}{#3}}%
    \global \expandafter \let \csname end#1\endcsname \@endtheorem}%
  \fi}
%    \end{macrocode}
% \end{macro}
%
% \subsection{Macros that are employed in a {\envfont theorem}
%    environment}
%
% \begin{macro}{\@thm}
%    The macro |\@thm| has to increase the current counter.  Then,
%    depending on whether the environment has (or does not have) an
%    optional argument, it has to branch into either |\@begintheorem|
%    or |\@opargtheorem|.
%    \begin{macrocode}
\gdef\@thm#1#2{\refstepcounter{#1}%
%    \end{macrocode}
%    Now we start a {\envfont trivlist} environment, and give
%    |\@topsep| and |\@topsepadd| the values of the \textsf{skip}
%    registers |\theorempreskipamount| and |\theorempostskipamount|.
%    The value in |\@topsep| is the vertical space that is inserted by
%    the first (and only) |\item| in our |\trivlist| whilst
%    |\@topsepadd| is inserted by |\@endparenv| at the end of that
%    {\envfont trivlist} environment.  By using these registers, we
%    obtain the desired space around a {\envfont theorem} environment.
%    \begin{macrocode}
   \trivlist
   \@topsep \theorempreskipamount               % used by first \item
   \@topsepadd \theorempostskipamount           % used by \@endparenv
%    \end{macrocode}
%    Now we have to test whether an optional argument has been given.
% \changes{v2.0d}{89/07/18}{Forgotten percent added.}
%    \begin{macrocode}
   \@ifnextchar [%
%    \end{macrocode}
%    If there is an optional argument, we will call |\@ythm|, and
%    move the arguments read back into the input stream.
%    \begin{macrocode}
   {\@ythm{#1}{#2}}%
%    \end{macrocode}
%    If not, we call |\@begintheorem|. Its first argument is the
%    name of the theorem set (hence the second argument of
%    |\@thm|). Its second argument is the macro that produces the
%    current number.
%    \begin{macrocode}
   {\@begintheorem{#2}{\csname the#1\endcsname}\ignorespaces}}
%    \end{macrocode}
% \end{macro}
%
% \begin{macro}{\@xthm}
% \begin{macro}{\@ythm}
%    Both these macros were originally called by |\@thm|.  We do
%    not need |\@xthm| anymore, hence we reset it to
%    |\relax|. The definition of |\@ythm| has not changed at
%    all from its definition in \LaTeX. In order to make the macros
%    easier to understand, we will
%    nevertheless present it (commented out).
%    \begin{macrocode}
\global\let\@xthm\relax
% \def\@ythm#1#2[#3]{\@opargbegintheorem{#2}{\csname
%           the#1\endcsname}{#3}\ignorespaces}
%</package>
%    \end{macrocode}
%    The primitive |\ignorespaces| in |\@ythm| and
%    |\@thm| is needed to remove the spaces between the
%    |\begin{...}| and the actual text.
% \end{macro}
% \end{macro}
%
%
% \subsection{Definition of the theorem substyles}
%
% As already pointed out, the theorem substyles, defined below, are
% only loaded when necessary.  Note that all these substyles, except
% \texttt{plain}, have |\slshape| as the default body font.
%
% \subsubsection{The \texttt{plain} style}
%
% \label{dok:th@plain} As the following macros use |@|, we have
% to locally set the |\catcode| of this symbol to ``letter''.
% This happens within a group, so that we do not have to worry about
% which |\catcode| that symbol had before.
%    \begin{macrocode}
%<*thp>
\begingroup \makeatletter
%    \end{macrocode}
% Since we are now within a group, we must make all definitions
% globally.  First we make sure that \texttt{theorem.sty} is loaded.
% This will allow us to use this file as a document style option
% without having to call \texttt{theorem} itself as an option.  At the
% same time, we assure that at least version 2 is loaded, since
% |\theorem@style| was not defined in earlier versions.
%    \begin{macrocode}
\@ifundefined{theorem@style}{\input{theorem.sty}}{}
\ProvidesFile{thp.sty}
             [\FMithmInfo]
%    \end{macrocode}
%
%
% \begin{macro}{\th@plain}
% \begin{macro}{\@begintheorem}
% \begin{macro}{\@opargbegintheorem}
%    |\theoremstyle{plain}| corresponds to the original
%    definition, except that the distances to the surrounding text are
%    determined by the parameters |\theorempreskipamount| and
%    |\theorempostskipamount|.
%    First we set the default body font.
%    \begin{macrocode}
\gdef\th@plain{\normalfont\itshape
%    \end{macrocode}
%    Then we define |\@begintheorem| and
%    |\@opargbegintheorem|. These two macros define how the
%    header of a theorem is typeset.  |\@opargbegintheorem| will
%    be called if a \textsf{theorem} environment with an optional
%    argument is encountered; otherwise, the header is constructed by
%    calling |\@begintheorem|.  If one of these macros is
%    executed, we are within a \textsf{trivlist} environment started by
%    |\@thm|.  So the theorem header is produced with an
%    |\item| command.
%
%    Instead of specifying the header font directly, all standard
%    theorem styles use the |\theorem@headerfont| macro to allow
%    customization.  The extra space (|\labelsep|) is necessary
%    because of problems in the \textsf{trivlist} environment.
%    \begin{macrocode}
  \def\@begintheorem##1##2{%
        \item[\hskip\labelsep \theorem@headerfont ##1\ ##2]}%
%    \end{macrocode}
%    The definition of |\@opargbegintheorem| is completely
%    analogous. The only difference is the fact that there exists a
%    third argument (which is the optional parameter of the
%    environment and contains additional information about the
%    theorem).  Customarily we enclose it in parentheses.
%    \begin{macrocode}
\def\@opargbegintheorem##1##2##3{%
   \item[\hskip\labelsep \theorem@headerfont ##1\ ##2\ (##3)]}}
%    \end{macrocode}
% \end{macro}
% \end{macro}
% \end{macro}
%
% \noindent
% We conclude with an |\endgroup| to restore the
% |\catcode| of |@|.
%    \begin{macrocode}
\endgroup
%</thp>
%    \end{macrocode}
%
%
% \subsubsection{The \texttt{break} style}
%
% This style option is stored in the file |thb.sty|.
% For the next two lines see the documentation for |\th@plain| on
% page~\pageref{dok:th@plain}.
%    \begin{macrocode}
%<*thb>
\begingroup \makeatletter
\@ifundefined{theorem@style}{\input{theorem.sty}}{}
\ProvidesFile{thb.sty}
             [\FMithmInfo]
%    \end{macrocode}
%
% \begin{macro}{\th@break}
%    |\theoremstyle{break}| produces a line break after the name
%    of the theorem.  The font is |\slshape|.  Hence, we define
%    |\th@break| as follows:
%    \begin{macrocode}
\gdef\th@break{\normalfont\slshape
  \def\@begintheorem##1##2{\item[%
%    \end{macrocode}
%    We run into the following problem: it is not possible to create
%    the header with |\item[|\meta{title}|]| and then start
%    a new line by, for example, |\mbox{}\\|. Such a definition
%    will fail whenever a list environment follows immediately.  With
%    the above construction, the |\mbox{}| causes the switch
%    |@inlabel| (cf.\ definition of |\list| and
%    |\trivlist| in \cite{latex:source}) to be set to \texttt{false}
%    and so the following list will insert additional vertical
%    space (|\topskip|). This is quite annoying. Therefore, we
%    create the line break within the |\item|. In order to ensure
%    that the text will begin at the proper position in the following
%    line, we simply pretend that the label does not take any
%    room.\footnote{This will lead to problems whenever very high
%    symbols occurring in the line tower into the
%    heading. So, something else has to be done here sometime.}
%    \begin{macrocode}
       \rlap{\vbox{\hbox{\hskip \labelsep\theorem@headerfont ##1\ ##2}%
                   \hbox{\strut}}}]}%
%    \end{macrocode}
%    Again, the definition of |\@opargbegintheorem| is
%    completely analogous.
%    \begin{macrocode}
\def\@opargbegintheorem##1##2##3{%
  \item[\rlap{\vbox{\hbox{\hskip \labelsep \theorem@headerfont
                     ##1\ ##2\ (##3)}%
                    \hbox{\strut}}}]}}
\endgroup
%</thb>
%    \end{macrocode}
% \end{macro}
%
%
% \subsubsection{The \texttt{changebreak} style}
%
%    \begin{macrocode}
% This style option is stored in the file |thcb.sty|.
%    \begin{macrocode}
%<*thcb>
\begingroup \makeatletter
\@ifundefined{theorem@style}{\input{theorem.sty}}{}
\ProvidesFile{thcb.sty}
             [\FMithmInfo]
%    \end{macrocode}
%
% \begin{macro}{\th@changebreak}
%    The \texttt{change-break} theorem style is like \texttt{break}
%    but with interchange of theorem name and theorem number.  Thus we
%    define |\th@changebreak| as follows:
%    \begin{macrocode}
\gdef\th@changebreak{\normalfont\slshape
  \def\@begintheorem##1##2{\item
       [\rlap{\vbox{\hbox{\hskip\labelsep \theorem@headerfont ##2\ ##1}%
                    \hbox{\strut}}}]}%
\def\@opargbegintheorem##1##2##3{%
  \item[\rlap{\vbox{\hbox{\hskip\labelsep \theorem@headerfont
                     ##2\ ##1\ (##3)}%
                    \hbox{\strut}}}]}}
\endgroup
%</thcb>
%    \end{macrocode}
% \end{macro}
%
%
% \subsubsection{The \texttt{change} style}
%
% This style option is stored in the file |thc.sty|.
%    \begin{macrocode}
%<*thc>
\begingroup \makeatletter
\@ifundefined{theorem@style}{\input{theorem.sty}}{}
\ProvidesFile{thc.sty}
             [\FMithmInfo]
%    \end{macrocode}
%
% \begin{macro}{\th@change}
%    The \texttt{change} theorem style corresponds to the
%    \texttt{change break} style without a linebreak after the header.
%    To say it in another way, it's the same as the \texttt{plain}
%    style but with number and name interchanged and |\slshape| as the
%    default font.
%    \begin{macrocode}
\gdef\th@change{\normalfont\slshape
  \def\@begintheorem##1##2{\item
       [\hskip\labelsep \theorem@headerfont ##2\ ##1]}%
\def\@opargbegintheorem##1##2##3{%
  \item[\hskip\labelsep \theorem@headerfont ##2\ ##1\ (##3)]}}
\endgroup
%</thc>
%    \end{macrocode}
% \end{macro}
%
%
% \subsubsection{The \texttt{marginbreak} style}
%
% This style option is the one used most often at Mainz.  It is saved
% in the file |thmb.sty|.
%    \begin{macrocode}
%<*thmb>
\begingroup \makeatletter
\@ifundefined{theorem@style}{\input{theorem.sty}}{}
\ProvidesFile{thmb.sty}
             [\FMithmInfo]
%    \end{macrocode}
%
% \begin{macro}{\th@marginbreak}
%    The \texttt{margin break} style is nearly the same as the
%    \texttt{change break} style. The only difference is the placement
%    of the theorem number.  We use |\llap| to place it in the left
%    margin.
%
%    In this style |\labelsep| denotes the separation between the
%    number and the text.
%    \begin{macrocode}
\gdef\th@marginbreak{\normalfont\slshape
  \def\@begintheorem##1##2{\item
       [\rlap{\vbox{\theorem@headerfont
                    \hbox{\llap{##2}\hskip\labelsep ##1}%
                    \hbox{\strut}}}]}%
\def\@opargbegintheorem##1##2##3{%
  \item[\rlap{\vbox{\theorem@headerfont
                    \hbox{\llap{##2}\hskip\labelsep ##1\ (##3)}%
                    \hbox{\strut}}}]}}
\endgroup
%</thmb>
%    \end{macrocode}
% \changes{v2.1c}{90/02/17}{`endgroup outside macrocode env.}
% \end{macro}
%
%
%
% \subsubsection{The \texttt{margin} style}
%
% This style option is stored in the file |thm.sty|.
%    \begin{macrocode}
%<*thm>
\begingroup \makeatletter
\@ifundefined{theorem@style}{\input{theorem.sty}}{}
\ProvidesFile{thm.sty}
             [\FMithmInfo]
%    \end{macrocode}
%
% \begin{macro}{\th@margin}
%    Again this is only a variant of the theorem styles described
%    above without any new ideas.
%    \begin{macrocode}
\gdef\th@margin{\normalfont\slshape
  \def\@begintheorem##1##2{\item
       [\theorem@headerfont \llap{##2}\hskip\labelsep ##1]}%
\def\@opargbegintheorem##1##2##3{%
  \item[\theorem@headerfont \llap{##2}\hskip\labelsep ##1\ (##3)]}}
\endgroup
%</thm>
%    \end{macrocode}
% \end{macro}
%
%
%
%
% \subsection{Final Definitions}
%
% \begin{macro}{\theorempreskipamount}
% \begin{macro}{\theorempostskipamount}
%    The \textsf{skip} parameters that regulate the vertical empty space
%    before and after the {\envfont theorem} environment have
%    to be allocated as well.
% \changes{v2.0f}{89/07/20}{Unnecessary `global' removed.}
%    \begin{macrocode}
%<*package>
\newskip\theorempreskipamount
\newskip\theorempostskipamount
%    \end{macrocode}
%    Since we have used the same values for all theorem sets, we now can
%    assign them.
%    \begin{macrocode}
\global\setlength\theorempreskipamount{12pt plus 5pt minus 3pt}
\global\setlength\theorempostskipamount{8pt plus 3pt minus 1.5pt}
%    \end{macrocode}
% \end{macro}
% \end{macro}
%
%
% \begin{macro}{\@endtheorem}
%    The same holds for the macro |\@endtheorem|, which ends a
%    {\envfont theorem} environment. Since it is the same for all
%    theorem sets, it is removed from the macros |\th@|\meta{style}.
%    It simply ends the {\envfont trivlist} environment, which was
%    begun in |\@thm|.
%    \begin{macrocode}
\global\let\@endtheorem=\endtrivlist
%    \end{macrocode}
% \end{macro}
%
%
%
% \begin{macro}{\@preamblecmds}
%    All macros defined above are to be used only in the preamble.
%    Therefore, we insert them in |\@preamblecmds| which will disable
%    them at begin document. This is done by the internal \LaTeXe{}
%    command |\@onlypreamble|.
% \changes{v2.0g}{89/09/19}{`theoremheaderfont and `theorembodyfont
%    added.}
% \changes{v2.2a}{94/02/03}{Use \LaTeXe{} interface}
%    \begin{macrocode}
\@onlypreamble\@xnthm
\@onlypreamble\@ynthm
\@onlypreamble\@othm
\@onlypreamble\newtheorem
\@onlypreamble\theoremstyle
\@onlypreamble\theorembodyfont
\@onlypreamble\theoremheaderfont
%    \end{macrocode}
% \end{macro}
%
% Finally we declare the \texttt{plain} theorem style to be the default.
% \changes{v2.1b}{90/06/04}{Fixed missing `item by declaring default
%    style.}
%    \begin{macrocode}
\theoremstyle{plain}
%</package>
%    \end{macrocode}
%
% \Finale

\endinput
}{}
\ProvidesFile{thc.sty}
             [\FMithmInfo]
%    \end{macrocode}
%
% \begin{macro}{\th@change}
%    The \texttt{change} theorem style corresponds to the
%    \texttt{change break} style without a linebreak after the header.
%    To say it in another way, it's the same as the \texttt{plain}
%    style but with number and name interchanged and |\slshape| as the
%    default font.
%    \begin{macrocode}
\gdef\th@change{\normalfont\slshape
  \def\@begintheorem##1##2{\item
       [\hskip\labelsep \theorem@headerfont ##2\ ##1]}%
\def\@opargbegintheorem##1##2##3{%
  \item[\hskip\labelsep \theorem@headerfont ##2\ ##1\ (##3)]}}
\endgroup
%</thc>
%    \end{macrocode}
% \end{macro}
%
%
% \subsubsection{The \texttt{marginbreak} style}
%
% This style option is the one used most often at Mainz.  It is saved
% in the file |thmb.sty|.
%    \begin{macrocode}
%<*thmb>
\begingroup \makeatletter
\@ifundefined{theorem@style}{% \iffalse meta-comment
%
% Copyright (C) 1993-2024
%
% The LaTeX Project and any individual authors listed elsewhere
% in this file.
%
% This file is part of the Standard LaTeX `Tools Bundle'.
% -------------------------------------------------------
%
% It may be distributed and/or modified under the
% conditions of the LaTeX Project Public License, either version 1.3c
% of this license or (at your option) any later version.
% The latest version of this license is in
%    https://www.latex-project.org/lppl.txt
% and version 1.3c or later is part of all distributions of LaTeX
% version 2005/12/01 or later.
%
% The list of all files belonging to the LaTeX `Tools Bundle' is
% given in the file `manifest.txt'.
%
% \fi
%
%%
%
%
\def\FMithmInfo{2023/07/05 v2.2c Theorem extension package (FMi)}
%
% \ProvidesFile{theorem.dtx}[\FMithmInfo]
%\iffalse   % this is a METACOMMENT !
%%
%% Package `theorem' to use with LaTeX2e
%% Copyright (C) 1989-1995 Frank Mittelbach, all rights reserved.
%<+package>\@ifundefined{theorem@style}{}{\endinput}
%<+package>\NeedsTeXFormat{LaTeX2e}[1995/06/01]
%<+package>\ProvidesPackage{theorem}
%<+package>         [\FMithmInfo]
%
% \fi
%
% \changes{v1.0}{87/05/07}{First draft.}
% \changes{v1.0a}{87/05/07}{`thistheoremstyle' added.}
% \changes{v1.1a}{87/07/29}{New concept: redefine the internal
%                          \protect\LaTeX{} commands `@xnthm' etc.}
% \changes{v1.1b}{87/07/31}{`theoremfont' used to specify font.}
% \changes{v1.2a}{88/01/08}{`@endtheorem' now globally defined.}
% \changes{v1.2b}{88/01/11}{`newtheorem' only allowed in the preamble.}
% \changes{v1.3a}{88/02/14}{`theoremfont' now a toks register.}
%
% \changes{v2.0a}{88/11/16}{New implementation and new documentation.}
% \changes{v2.0a}{88/11/16}{No layout changes but internal modifications
%                           old substyles won't work.}
% \changes{v2.0b}{89/01/30}{`global' used for all definitions.}
% \changes{v2.0c}{89/06/15}{Documentation translated to English.
%                            Thanks to LLL and Nina}
% \changes{v2.1a}{90/02/15}{Support for new font selection.}
% \changes{v2.2a}{94/02/03}{Cleanup for LaTeX2e}
%
% \DoNotIndex{\@endparenv,\@ifdefinable,\@ifnextchar,\@ifundefined}
% \DoNotIndex{\@namedef,\@nameuse,\@spaces,\@tempa,\@warning}
% \DoNotIndex{\ }
% \DoNotIndex{\begingroup,\bfseries}
% \DoNotIndex{\csname}
% \DoNotIndex{\def,\do,\docdate}
% \DoNotIndex{\endcsname,\endgroup,\endinput,\endtrivlist,\expandafter}
% \DoNotIndex{\filedate,\fileversion}
% \DoNotIndex{\gdef,\global}
% \DoNotIndex{\hbox,\hskip}
% \DoNotIndex{\ignorespaces,\input,\itshape,\item}
% \DoNotIndex{\labelsep,\let,\llap}
% \DoNotIndex{\makeatletter}
% \DoNotIndex{\noexpand}
% \DoNotIndex{\refstepcounter,\relax,\rlap}
% \DoNotIndex{\setlength,\slshape,\space,\string,\strut}
% \DoNotIndex{\the,\trivlist,\typeout}
% \DoNotIndex{\vbox}
% \DoNotIndex{\xdef}
%
% \def\deflabel#1{\ttfamily #1\hfill}
% \def\deflist#1{\begin{list}{}{\settowidth\labelwidth{\ttfamily #1}%
%                               \setlength\leftmargin\labelwidth
%                               \addtolength\leftmargin\labelsep
%                               \let\makelabel\deflabel}}
% \def\enddeflist{\end{list}}
% \def\envfont{\normalfont\ttfamily}
%
% \MakeShortVerb{\|}
%
% \newcommand{\TUB}{{\sl TUGboat\/}}
%
% \GetFileInfo{theorem.dtx}
% \title{An Extension of the
%        \LaTeX{} theorem environment\thanks{This file
%        has version number \fileversion, last
%        revised \filedate.}}
% \author{Frank Mittelbach}
%
% \MaintainedByLaTeXTeam{tools}
% \maketitle
%
% \begin{abstract}
%    The macros described in this paper yield an extension of the
%    \LaTeX{} theorem mechanism. It is designed to satisfy the
%    different requirements of various journals. Thus, the
%    layout of the ``theorems'' can be manipulated by determining a
%    ``style''. This article describes not only the use, but
%    also the definition, of the necessary macros.
% \end{abstract}
%
% \section*{Preface to version 2.2}
%
% For \LaTeXe{} this package did not need any fundamental changes. I
% only modified the messages generated so that theorem layout styles
% will show up with the |\listfiles| command and cleaned the section
% on the New Font Selection Scheme since this is now included in
% \LaTeX.
%
% \section*{Preface to version 2.1}
%
% This version is identical to 2.0g described in \TUB\ 10\#3 except
% for some internal defaults which are now set depending on the used
% font selection scheme.
%
% This was done to avoid unpleasant surprises if the new font selection
% scheme is in force. For further details see
% section~\ref{sec:newfonts} and \cite{bk:GMS94}.
%
% \section{Introduction}
%
% For our purposes here, ``theorems'' are labelled enunciations,
% often set off from the main text by extra space and a font change.
% Theorems, corollaries, conjectures, definitions, and remarks
% are all instances of ``theorems''.  The ``header'' of these structures
% is composed of a label (such as \textsc{Theorem} or \textsc{Remark})
% and a number which serializes an item in the sequence of items
% with the same label.
%
% Shortly after the introduction of \LaTeX{} at the Fachbereich
% Mathematik in Mainz, the desire to manipulate the layout of
% ``theorems'' arose.  In Mainz, the following two conventions came
% into general use:
% \begin{enumerate}
%   \item The number of the theorem is shown in the margin.
%   \item There is a line break at
%         the end of the theorem header.
% \end{enumerate}
%
% Additionally, some journals require different formats which depend on
% the ``sort of theorem'': e.g.\ often remarks and definitions are set
% in |\upshape|, while |\itshape| is employed for main theorems.
%
% Confronted with these requirements, a theorem environment was
% developed in Mainz which allows separate determination of the
% layout of the ``theorems sets'', comparable to |\pagestyle|.
%
%
% \section{The user interface}
% \subsection{Defining new theorem sets}
%
% \DescribeMacro\newtheorem
% As in the original \LaTeX{} version, the command |\newtheorem|
% defines a new ``theorem set'' or ``theorem-like structure''.
% Two required arguments name the new environment and give the text
% to be typeset with each instance of the new ``set'', while
% an optional argument determines how
% the ``set'' is enumerated:
% \begin{description}
%    \item[\ttfamily \bslash newtheorem\{foo\}\{bar\}]
%       The theorem set {\envfont foo} (whose name is \texttt{bar})
%       uses its own counter.
%    \item[\ttfamily \bslash newtheorem\{foo2\}{[foo]}\{bar2\}]
%       The theorem set {\envfont foo2} (printed name \texttt{bar2})
%       uses the same counter as the theorem set \texttt{foo}.
%    \item[\ttfamily \bslash newtheorem\{foo3\}\{bar3\}{[section]}]
%       The theorem set {\envfont foo3} (printed name \texttt{bar3}) is
%       enumerated within the counter \texttt{section}, i.e.\ with every
%       new |\section| the enumeration begins again with 1, and
%       the enumeration is composed from the section-number and the
%       theorem counter itself.
% \end{description}
%
% \DescribeMacro\theoremstyle
% Additionally, the command |\theoremstyle| can define the layout
% of various, or all, theorem sets.  It should be noted that any
% theorem set defined by |\newtheorem| is typeset in the
% |\theoremstyle| that is current at the time of the definition.
% Thus, the following
% \begin{verbatim}
%    \theoremstyle{break}        \newtheorem{Cor}{Corollary}
%    \theoremstyle{plain}        \newtheorem{Exa}{Example}[section]
%\end{verbatim}                             ^^A No blank at line start
% leads to the result that the set {\envfont Cor} is formatted in the
% style \texttt{break}, while the set {\envfont Exa} and all the
% following ones are formatted in the style \texttt{plain}, unless
% another |\theoremstyle| follows.  Since the definitions installed by
% |\newtheorem| are global, one also can limit |\theoremstyle| locally
% by grouping braces.
%
% \DescribeMacro\theorembodyfont
% The choice of the font for the theorem body is completely
% independent of the chosen |\theoremstyle|; this has proven to
% be very advantageous.  For example,
% \begin{verbatim}
%   {\theorembodyfont{\upshape}          \newtheorem{Rem}{Remark}}
%\end{verbatim}
% defines a theorem set {\envfont Rem}, which will be set in
% |\upshape| in the current layout (which in our example is
% \texttt{plain}).  As with |\theoremstyle|, the |\theorembodyfont|
% chosen is that current at the time of |\newtheorem|.  If
% |\theorembodyfont| is not specified or one defines
% |\theorembodyfont{}|, then the font used will be that defined by the
% |\theoremstyle|.
%
% \DescribeMacro\theoremheaderfont It is also possible to customize
% the font used for the theorem headers.  This is, however, a global
% declaration, and therefore there should be at most one
% |\theoremheaderfont| declaration in the preamble.\footnote{If it is
% actually necessary to have different header fonts, one has to define
% new theorem styles (substituting the desired font) or specify the
% information directly in the \texttt{\bslash newtheorem} declaration
% (the unclean variant).}
%
% \DescribeMacro\theorempreskipamount
% \DescribeMacro\theorempostskipamount Two additional parameters
% affect the vertical space around the theorem environments:
% |\theorempreskipamount| and |\theorempostskipamount| define,
% respectively, the spacing before and after such an environment.
% These parameters apply for all theorem sets and can be manipulated
% with the ordinary length macros.  They are rubber lengths
% (`\textsf{skips}'), and therefore can contain \texttt{plus} and
% \texttt{minus} parts.
%
% Since the definition of theorem sets should---most sensibly---be
% placed in the preamble, we only allow installation there.
% It is therefore possible to release the memory used here after
% |\begin{document}|, in order to make room for other
% applications.
%
%
% \subsection{Existing theorem styles}
%
% The following theorem styles exist to date:
% \begin{deflist}{marginbreak}
%    \item[plain]
%       This theorem style emulates the original \LaTeX{} definition,
%       except that additionally the parameters
%       |\theorem...skipamount| are used.
%    \item[break]
%       In this style, the theorem header is followed by a line
%       break.
%    \item[marginbreak]
%       The theorem number is set in the margin, and there is a line
%       break as in \texttt{break}.
%    \item[changebreak]
%       Like \texttt{break}, but with header number and text
%       interchanged.
%    \item[change]
%       Header number and text are interchanged, without a line break.
%    \item[margin]
%       The number is set in the left margin, without a line break.
% \end{deflist}
% All styles (except \texttt{plain}) select |\slshape| as the default
% |\theorembodyfont|.
%
%
% \subsection{Examples}
%
% Given the above theorem sets {\envfont Cor}, {\envfont Exa}
% and {\envfont Rem},
% suppose that the preamble also contains the declarations:
% \begin{verbatim}
%    \theoremstyle{marginbreak}   \newtheorem{Lem}[Cor]{Lemma}
%    \theoremstyle{change}
%    \theorembodyfont{\itshape}        \newtheorem{Def}[Cor]{Definition}
%
%    \theoremheaderfont{\scshape}
%\end{verbatim}
% Then the following are some typical examples of the typeset output
% resulting from their use.
%
% \begin{Cor}
%    This is a sentence typeset in the theorem environment {\envfont
%    Cor}.
% \end{Cor}
%
% \begin{Exa}
%    This is a sentence typeset in the theorem environment {\envfont
%    Exa}.
% \end{Exa}
%
% \begin{Rem}
%    This is a sentence typeset in the theorem environment {\envfont
%    Rem}.
% \end{Rem}
%
% \begin{Lem}[Ben User]
%    This is a sentence typeset in the theorem environment {\envfont
%    Lem}.
% \end{Lem}
%
% \begin{Def}[Very impressive Definition]
%    This is a sentence typeset in the theorem environment {\envfont
%    Def}.
% \end{Def}
%
% The last two examples show the effect of the optional argument to
% a theorem environment (it is the text typeset in parentheses).
%
%
% \section{Special Considerations}
% \label{sec:newfonts}
%
% Theoremheader and body are implemented as a unit. This means that
% the |\theoremheaderfont|  will inherit characteristics of the
% |\theorembodyfont| in \LaTeXe. Thus, if for example
% |\theorembodyfont| is |\itshape| and |\theoremheaderfont|
% is |\bfseries| the font selected for the header will have the
% characteristics `bold extended italic'. If this is not desired one
% should set the |\theoremheaderfont| to something like
% \begin{verbatim}
%      \theoremheaderfont{\normalfont\bfseries}
%\end{verbatim}
% i.e.\ supplying all necessary font information explicitly.
%
%
%
% \section{Acknowledgements}
%
% The publication of this set of macros was only possible with the
% help of Christina Busse (translating the manuscript into English),
% Joachim Pense (playing the r\^ole of typist), Chris Rowley
% (looking everything over) and many others providing useful
% suggestions.
%
% \MaybeStop{%
%    \begin{thebibliography}{9}
%       \bibitem{bk:GMS94} \textsc{M.~Goossens}, \textsc{F.~Mittelbach}
%         and \textsc{A.~Samarin}.
%         \newblock The \LaTeX{} Companion.
%         \newblock
%         Addison-Wesley, Reading, Massachusetts, 1994.
%       \bibitem{latex:source} \textsc{Lamport, Leslie.} \newblock
%                               \texttt{latex.tex}, version 2.09, date
%                               Feb. 1990.
%    \end{thebibliography}
%    \PrintIndex
%    \PrintChanges}
%
%
% \section{The documentation driver file}
%
% The next bit of code contains the documentation driver file for
% \TeX{}, i.e., the file that will produce the documentation you are
% currently reading. It will be extracted from this file by the
% \texttt{docstrip} program. Since it is the first code in the file
% one can alternatively process this file directly with \LaTeXe{} to
% obtain the documentation.
%
%    \begin{macrocode}
%<*driver>
\documentclass{ltxdoc}

\usepackage{theorem}

%<+driver>% The next few lines define theorem sets which are used
%<+driver>% in the example section of the documentation.

\theoremstyle{break}       \newtheorem{Cor}{Corollary}
\theoremstyle{plain}       \newtheorem{Exa}{Example}[section]
{\theorembodyfont{\upshape}\newtheorem{Rem}{Remark}}
\theoremstyle{marginbreak} \newtheorem{Lem}[Cor]{Lemma}
\theoremstyle{change}
\theorembodyfont{\itshape} \newtheorem{Def}[Cor]{Definition}

\theoremheaderfont{\scshape}

\RecordChanges

\begin{document}
   \DocInput{theorem.dtx}
\end{document}
%</driver>
%    \end{macrocode}
%
% \section{Definition of the Macros}
%
% If the file has been loaded before, we abort immediately. If not the
% package announces itself (this is actually done at the very top of
% the file---the way it is done isn't good style so don't copy it).
% \changes{v2.0e}{89/07/19}{Spaces removed from `typeout'.}
% \changes{v2.2b}{95/11/19}{Announce moved to top of file}
%    \begin{macrocode}
%<*package>
%\@ifundefined{theorem@style}{}{\endinput}
%\def\FMithmInfo{1995/11/19 v2.2b Theorem extension package (FMi)}
%\ProvidesPackage{theorem}[\FMithmInfo]
%    \end{macrocode}
%
%
% \subsection{Definition of theorem styles and fonts}
%
% All the definitions in this file are done globally to allow
% inputting this file inside a group.
%
% \begin{macro}{\theoremstyle}
%    Before a theorem style can be installed, the chosen style must be
%    known. For that reason, we must test to see that
%    |\th@|\meta{style} is known or, more precisely, that it is
%    different from |\relax|. If the style is not known then
%    |\th@plain| is used.
%    \begin{macrocode}
\gdef\theoremstyle#1{%
   \@ifundefined{th@#1}{\@warning
          {Unknown theoremstyle `#1'. Using `plain'}%
          \theorem@style{plain}}%
%    \end{macrocode}
%    We save the theorem style to be used in the token register
%    |\theorem@style|.
%    \begin{macrocode}
      {\theorem@style{#1}}%
%    \end{macrocode}
%    Now we ``evaluate'' the theorem style: this means, we call the
%    macro |\th@|\meta{style} which will activate the relevant
%    definitions which are contained in a separate file.  This is done
%    in a group to suppress changes to the current font.  This could
%    otherwise pose problems together with the new font selection
%    scheme\footnote{When I printed the original article using the new
%    font selection scheme I ended with a document with slanted
%    typefaces (text headings and all) simply because one of the
%    theorem styles used \texttt{\bslash sl} at toplevel.} if the
%    |\th@|\meta{style} is evaluated a second time.
% \changes{v2.1a}{90/02/15}{Group added around `th@..}
% \changes{v2.1a}{90/02/15}{Reseting of `..@begintheorem removed.}
%    \begin{macrocode}
      \begingroup
        \csname th@\the\theorem@style \endcsname
      \endgroup}
%    \end{macrocode}
% \end{macro}
%
%
% \begin{macro}{\@begintheorem}
% \changes{v2.1a}{90/02/15}{Relaxed at toplevel.}
% \begin{macro}{\@opargbegintheorem}
% \changes{v2.1a}{90/02/15}{Relaxed at toplevel.}
%    We reset |\@begintheorem| and |\@opargbegintheorem| to
%    |\relax| since these commands are no longer necessary at
%    toplevel. This will save a few tokens.
%    \begin{macrocode}
\global\let\@begintheorem\relax
\global\let\@opargbegintheorem\relax
%    \end{macrocode}
% \end{macro}
% \end{macro}
%
% \begin{macro}{\theorem@style}
%    Obviously the \textsf{token} register used above has to be
%    allocated. To assure the utmost compatibility with the
%    original \LaTeX{} definition, we set the default theorem style to
%    \texttt{plain}, which implements the usual \LaTeX{} convention.
% \changes{v2.0f}{89/07/20}{Unnecessary `global' removed.}
%    \begin{macrocode}
\newtoks\theorem@style
\global\theorem@style{plain}
%    \end{macrocode}
% \end{macro}
%
% \begin{macro}{\theorembodyfont}
% \begin{macro}{\theorem@bodyfont}
%    If the |\theorembodyfont| is set by the user then it should not
%    interact with the default font set in the theorem style. When the
%    new font selection is in force this may happen if, for example,
%    the default is |\itshape| and the new |\theorembodyfont| is
%    |\sffamily|. So we add a |\reset@font| command in front of the
%    user definition.
% \changes{v2.0f}{89/07/20}{Unnecessary `global' removed.}
% \changes{v2.0g}{89/09/19}{`theoremfont to `theorembodyfont.}
% \changes{v2.1c}{92/06/22}{`reset@font added and logic changed.}
% \changes{v2.1d}{92/10/28}{Reset to empty if arg is empty}
%    \begin{macrocode}
\gdef\theorembodyfont#1{%
%    \end{macrocode}
%    We check if the argument supplied is empty and if so put nothing
%    into the |\theorem@bodyfont| token register to allow for
%    |\theorembodyfont{}| as a mean of using the default of the
%    current |\theoremstyle|.
%    \begin{macrocode}
   \def\@tempa{#1}%
   \ifx\@tempa\@empty
    \theorem@bodyfont{}%
   \else
    \theorem@bodyfont{\reset@font#1}%
   \fi
   }
\newtoks\theorem@bodyfont
\global\theorem@bodyfont{}
%    \end{macrocode}
% \end{macro}
% \end{macro}
%
%
% \begin{macro}{\theoremheaderfont}
% \changes{v2.0g}{89/09/19}{Macro added.}
%    The font for the theorem headers is handled differently because
%    this definition applies to all theorem styles.
%    \begin{macrocode}
\gdef\theoremheaderfont#1{\gdef\theorem@headerfont{#1}%
%    \end{macrocode}
%    After using the macro once it is redefined to produce an error
%    message.
%    \begin{macrocode}
       \gdef\theoremheaderfont##1{%
        \typeout{\string\theoremheaderfont\space should be used
                 only once.}}}
%    \end{macrocode}
% \end{macro}
%
% \begin{macro}{\theorem@headerfont}
% \changes{v2.0g}{89/09/19}{Macro added.}
%    To set the |\theorem@headerfont| default we first test if
%    the new fontselection scheme is in force.
% \changes{v2.1a}{90/02/15}{Support for new font selection.}
%    \begin{macrocode}
\ifx\upshape\undefined
%    \end{macrocode}
%    If not we define it to expand into |\bfseries|. We don't use
%    |\let| just in case a following style option redefines this
%    macro.
%    \begin{macrocode}
\gdef\theorem@headerfont{\bfseries}
%    \end{macrocode}
%    Otherwise we reset the current shape before calling |\bfseries|.
%    \begin{macrocode}
\else \gdef\theorem@headerfont{\normalfont\bfseries}\fi
%    \end{macrocode}
% \end{macro}
%
% \begin{macro}{\th@plain}
% \begin{macro}{\th@break}
% \begin{macro}{\th@marginbreak}
% \begin{macro}{\th@changebreak}
% \begin{macro}{\th@change}
% \begin{macro}{\th@margin}
%    The different styles are defined in macros such as
%    |\th@plain|.  Since memory space is precious in
%    ``non-Big-versions'', we have to avoid offering too many unused
%    definitions.  Therefore we define these styles in separate files
%    that can be loaded on demand.  Thus the commands themselves only
%    load these files. We use |\@input@|, a \LaTeXe{} internal command
%    that ensures that the file will be listed with |\listfiles|.
% \changes{v2.2a}{94/02/02}{Use `@input@ to load theorem layout files}
%    \begin{macrocode}
\gdef\th@plain{\@input@{thp.sty}}
\gdef\th@break{\@input@{thb.sty}}
\gdef\th@marginbreak{\@input@{thmb.sty}}
\gdef\th@changebreak{\@input@{thcb.sty}}
\gdef\th@change{\@input@{thc.sty}}
\gdef\th@margin{\@input@{thm.sty}}
%    \end{macrocode}
%    This list will be expanded when new styles become
%    available.  For testing,  just append new theorem substyles
%    as document options.
% \end{macro}
% \end{macro}
% \end{macro}
% \end{macro}
% \end{macro}
% \end{macro}
%
%
%
% \subsection{Definition of a new theorem set}
%
% As already pointed out, a new theorem environment can be defined in
% three different ways:
% \begin{verbatim}
%    \newtheorem{Lem}{Lemma}
%    \newtheorem{Lem}{Lemma}[section]
%    \newtheorem{Lem}[Theorem]{Lemma}
%\end{verbatim}
% The function of the macro |\newtheorem| is to recognize these cases
% and then to branch into one of the three macros |\@ynthm|, |\@xnthm|
% or |\@othm|.  This mechanism is adopted unchanged from
% \cite{latex:source}; the essential point here is that, for example,
% in the second case, the arguments \texttt{Lem}, \texttt{Lemma} and
% \texttt{section} are passed over to the macro |\@xnthm|.
%
% We inspect this case first because the others present fewer problems,
% and thus are easily derived from this one.
%
% \begin{macro}{\@xnthm}
%    For our example arguments, the macro |\@xnthm| must fulfill
%    the following:
%    \begin{itemize}
%       \item
%          Define a new \LaTeX-counter `Lem'
%       \item
%          reset this counter within a |\section|
%       \item
%          define the macro |\theLem|
%       \item
%          define the environment macros |\Lem| and
%          |\endLem| using the current
%          |\theoremstyle| and |\theorem@bodyfont|.
%    \end{itemize}
%    Obviously, all this should happen only if the first argument
%    of |\@xnthm| (i.e.\ \texttt{Lem} in our example) is chosen so
%    as not to conflict with any previously defined commands or
%    environments.  This test is performed by the \LaTeX{} macro
%    |\@ifdefinable|.
%    \begin{macrocode}
\gdef\@xnthm#1#2[#3]{\expandafter\@ifdefinable\csname #1\endcsname
%    \end{macrocode}
%    Therefore, the first argument of |\@ifdefinable| is the
%    expansion (in the example, |\Lem|) of
%    |\csname||#1\endcsname|.  The second argument is
%    executed only if the test has been completed successfully.
%    \begin{macrocode}
   {%
%    \end{macrocode}
%    Now we define the new counter. The names of the \LaTeX{}
%    macros employed should speak for themselves:
% \changes{v2.2b}{1995/10/07}{Use \cs{@newctr}}
% \changes{v2.2c}{1995/11/23}{Use \cs{@newctr} properly.}
%    \begin{macrocode}
    \@definecounter{#1}\@newctr{#1}[#3]%
%    \end{macrocode}
%    Using |\@newctr| will give a proper error message if the counter
%    in |#3| is not defined.
%    In defining `|\theLem|' we must generate the desired macro
%    name by use of |\expandafter| and |\csname|.
%    \begin{macrocode}
    \expandafter\xdef\csname the#1\endcsname
%    \end{macrocode}
%    An |\xdef| is used in order to make the definition global,
%    and to ensure that it contains the replacement texts of
%    |\@thmcountersep| and |\@thmcounter|.\footnote{These
%    two macros can be defined by the document style.  Their default
%    values produce a `\texttt{.}' as separation and an arabic
%    representation of the number.} However, not everything should be
%    expanded. For example, it saves space to use |\thesection|
%    instead of its---at times---lengthy expansion.
%    \begin{macrocode}
      {\expandafter \noexpand \csname the#3\endcsname
       \@thmcountersep \@thmcounter{#1}}%
%    \end{macrocode}
%    Thus with the defaults of \LaTeX{}, |\theLem| would be
%    replaced by the command sequence |\thesection.\arabic{Lem}|.
%
%    We will now look at the definition of the macro which is executed
%    at the beginning of the actual environment (in our example this
%    macro is |\Lem|). It should be noted that we use an
%    ``|\expandafter| trick'' to expand only certain parts of the
%    replacement text at the time of the definition.
%    \begin{macrocode}
    \def\@tempa{\global\@namedef{#1}}%
    \expandafter \@tempa \expandafter{%
%    \end{macrocode}
%    First, the macro that contains the current definitions of
%    |\@begintheorem| and |\@opargtheorem| should be called
%    up.  The name of this macro---as is already known---has the form
%    |\th@|\meta{theorem\ style}; therefore, it must be called by
%    \begin{macrocode}
      \csname th@\the \theorem@style
%    \end{macrocode}
%    In addition the default theorem font should be changeable, i.e.\
%    we have to insert the contents of |\theorem@bodyfont|. For
%    that reason, we expand even further, beyond |\endcsname|,
%    and thus insert the contents of the \textsf{token} register
%    |\theorem@bodyfont| in the replacement text.
%    \begin{macrocode}
            \expandafter \endcsname \the \theorem@bodyfont
%    \end{macrocode}
%    Now it is time to call the macro |\@thm| which takes over
%    the further processing.  It has two arguments: the current
%    counter name (in our example, \texttt{Lem}), and the text of the
%    label (in our example, \texttt{Lemma}).
%    \begin{macrocode}
     \@thm{#1}{#2}}%
%    \end{macrocode}
%    With this, the `sub-definition' is complete.  The macro
%    |\@endtheorem| ends a theorem environment and is, so far, nothing
%    but an |\endtrivlist|.  (Hence it is defined globally, and not
%    within the theorem styles.\footnote{This has to be changed as
%    soon as theorem styles that change \texttt{\bslash @endtheorem}
%    exist.  In such a case, all existing styles must be changed as
%    well since they will have to reset the macro.}) Therefore, we can
%    set it equivalent to the macro that ends the theorem set (in our
%    example, |\endLem|).  However, if some day theorem styles exist
%    that do change |\@endtheorem|, we would have to use the
%    commented-out line instead.
%    \begin{macrocode}
    \global \expandafter \let \csname end#1\endcsname \@endtheorem
%   \global\@namedef{end#1}{\@endtheorem}%
%    \end{macrocode}
%    With these commands all the required definitions are employed,
%    unless the test |\@ifdefinable| has failed.  Therefore, we
%    end the second argument of this macro and with it the definition
%    of |\@xnthm|.
%    \begin{macrocode}
   }}
%    \end{macrocode}
% \end{macro}
%
% \begin{macro}{\@ynthm}
%    The definition of |\@ynthm| is completely analogous. In
%    this case the new counter that is defined is not reset within
%    another counter; thus the definition of |\the...|
%    is simplified:
%    \begin{macrocode}
\gdef\@ynthm#1#2{\expandafter\@ifdefinable\csname #1\endcsname
   {\@definecounter{#1}%
    \expandafter\xdef\csname the#1\endcsname{\@thmcounter{#1}}%
%    \end{macrocode}
%    The rest of the definition corresponds literally to that of
%    |\@xnthm|:
%    \begin{macrocode}
    \def\@tempa{\global\@namedef{#1}}\expandafter \@tempa
     \expandafter{\csname th@\the \theorem@style \expandafter
     \endcsname \the\theorem@bodyfont \@thm{#1}{#2}}%
    \global \expandafter \let \csname end#1\endcsname \@endtheorem}}
%    \end{macrocode}
% \end{macro}
%
% \begin{macro}{\@othm}
%    The definition of |\@othm| does not contain anything new.
%    \begin{macrocode}
\gdef\@othm#1[#2]#3{%
%    \end{macrocode}
% \changes{v2.2b}{95/10/07}{Build in check for counter defined}
%    We do not define a new counter but instead use one that has
%    already been defined.  Thus the only definition we need is that
%    of this pseudo-counter (i.e.\ |\the|\meta{env.\ name}).
%    First we check if |#2| corresponds to a known counter name.
%    \begin{macrocode}
  \expandafter\ifx\csname c@#2\endcsname\relax
   \@nocounterr{#2}%
  \else
   \expandafter\@ifdefinable\csname #1\endcsname
   {\expandafter \xdef \csname the#1\endcsname
     {\expandafter \noexpand \csname the#2\endcsname}%
%    \end{macrocode}
%    All other parts of the definition can be adopted from
%    |\@xnthm|. We have to remember, though, that in this case
%    the name of the current counter and the theorem label have moved
%    to the second and third arguments.
%    \begin{macrocode}
    \def\@tempa{\global\@namedef{#1}}\expandafter \@tempa
     \expandafter{\csname th@\the \theorem@style \expandafter
     \endcsname \the\theorem@bodyfont \@thm{#2}{#3}}%
    \global \expandafter \let \csname end#1\endcsname \@endtheorem}%
  \fi}
%    \end{macrocode}
% \end{macro}
%
% \subsection{Macros that are employed in a {\envfont theorem}
%    environment}
%
% \begin{macro}{\@thm}
%    The macro |\@thm| has to increase the current counter.  Then,
%    depending on whether the environment has (or does not have) an
%    optional argument, it has to branch into either |\@begintheorem|
%    or |\@opargtheorem|.
%    \begin{macrocode}
\gdef\@thm#1#2{\refstepcounter{#1}%
%    \end{macrocode}
%    Now we start a {\envfont trivlist} environment, and give
%    |\@topsep| and |\@topsepadd| the values of the \textsf{skip}
%    registers |\theorempreskipamount| and |\theorempostskipamount|.
%    The value in |\@topsep| is the vertical space that is inserted by
%    the first (and only) |\item| in our |\trivlist| whilst
%    |\@topsepadd| is inserted by |\@endparenv| at the end of that
%    {\envfont trivlist} environment.  By using these registers, we
%    obtain the desired space around a {\envfont theorem} environment.
%    \begin{macrocode}
   \trivlist
   \@topsep \theorempreskipamount               % used by first \item
   \@topsepadd \theorempostskipamount           % used by \@endparenv
%    \end{macrocode}
%    Now we have to test whether an optional argument has been given.
% \changes{v2.0d}{89/07/18}{Forgotten percent added.}
%    \begin{macrocode}
   \@ifnextchar [%
%    \end{macrocode}
%    If there is an optional argument, we will call |\@ythm|, and
%    move the arguments read back into the input stream.
%    \begin{macrocode}
   {\@ythm{#1}{#2}}%
%    \end{macrocode}
%    If not, we call |\@begintheorem|. Its first argument is the
%    name of the theorem set (hence the second argument of
%    |\@thm|). Its second argument is the macro that produces the
%    current number.
%    \begin{macrocode}
   {\@begintheorem{#2}{\csname the#1\endcsname}\ignorespaces}}
%    \end{macrocode}
% \end{macro}
%
% \begin{macro}{\@xthm}
% \begin{macro}{\@ythm}
%    Both these macros were originally called by |\@thm|.  We do
%    not need |\@xthm| anymore, hence we reset it to
%    |\relax|. The definition of |\@ythm| has not changed at
%    all from its definition in \LaTeX. In order to make the macros
%    easier to understand, we will
%    nevertheless present it (commented out).
%    \begin{macrocode}
\global\let\@xthm\relax
% \def\@ythm#1#2[#3]{\@opargbegintheorem{#2}{\csname
%           the#1\endcsname}{#3}\ignorespaces}
%</package>
%    \end{macrocode}
%    The primitive |\ignorespaces| in |\@ythm| and
%    |\@thm| is needed to remove the spaces between the
%    |\begin{...}| and the actual text.
% \end{macro}
% \end{macro}
%
%
% \subsection{Definition of the theorem substyles}
%
% As already pointed out, the theorem substyles, defined below, are
% only loaded when necessary.  Note that all these substyles, except
% \texttt{plain}, have |\slshape| as the default body font.
%
% \subsubsection{The \texttt{plain} style}
%
% \label{dok:th@plain} As the following macros use |@|, we have
% to locally set the |\catcode| of this symbol to ``letter''.
% This happens within a group, so that we do not have to worry about
% which |\catcode| that symbol had before.
%    \begin{macrocode}
%<*thp>
\begingroup \makeatletter
%    \end{macrocode}
% Since we are now within a group, we must make all definitions
% globally.  First we make sure that \texttt{theorem.sty} is loaded.
% This will allow us to use this file as a document style option
% without having to call \texttt{theorem} itself as an option.  At the
% same time, we assure that at least version 2 is loaded, since
% |\theorem@style| was not defined in earlier versions.
%    \begin{macrocode}
\@ifundefined{theorem@style}{\input{theorem.sty}}{}
\ProvidesFile{thp.sty}
             [\FMithmInfo]
%    \end{macrocode}
%
%
% \begin{macro}{\th@plain}
% \begin{macro}{\@begintheorem}
% \begin{macro}{\@opargbegintheorem}
%    |\theoremstyle{plain}| corresponds to the original
%    definition, except that the distances to the surrounding text are
%    determined by the parameters |\theorempreskipamount| and
%    |\theorempostskipamount|.
%    First we set the default body font.
%    \begin{macrocode}
\gdef\th@plain{\normalfont\itshape
%    \end{macrocode}
%    Then we define |\@begintheorem| and
%    |\@opargbegintheorem|. These two macros define how the
%    header of a theorem is typeset.  |\@opargbegintheorem| will
%    be called if a \textsf{theorem} environment with an optional
%    argument is encountered; otherwise, the header is constructed by
%    calling |\@begintheorem|.  If one of these macros is
%    executed, we are within a \textsf{trivlist} environment started by
%    |\@thm|.  So the theorem header is produced with an
%    |\item| command.
%
%    Instead of specifying the header font directly, all standard
%    theorem styles use the |\theorem@headerfont| macro to allow
%    customization.  The extra space (|\labelsep|) is necessary
%    because of problems in the \textsf{trivlist} environment.
%    \begin{macrocode}
  \def\@begintheorem##1##2{%
        \item[\hskip\labelsep \theorem@headerfont ##1\ ##2]}%
%    \end{macrocode}
%    The definition of |\@opargbegintheorem| is completely
%    analogous. The only difference is the fact that there exists a
%    third argument (which is the optional parameter of the
%    environment and contains additional information about the
%    theorem).  Customarily we enclose it in parentheses.
%    \begin{macrocode}
\def\@opargbegintheorem##1##2##3{%
   \item[\hskip\labelsep \theorem@headerfont ##1\ ##2\ (##3)]}}
%    \end{macrocode}
% \end{macro}
% \end{macro}
% \end{macro}
%
% \noindent
% We conclude with an |\endgroup| to restore the
% |\catcode| of |@|.
%    \begin{macrocode}
\endgroup
%</thp>
%    \end{macrocode}
%
%
% \subsubsection{The \texttt{break} style}
%
% This style option is stored in the file |thb.sty|.
% For the next two lines see the documentation for |\th@plain| on
% page~\pageref{dok:th@plain}.
%    \begin{macrocode}
%<*thb>
\begingroup \makeatletter
\@ifundefined{theorem@style}{\input{theorem.sty}}{}
\ProvidesFile{thb.sty}
             [\FMithmInfo]
%    \end{macrocode}
%
% \begin{macro}{\th@break}
%    |\theoremstyle{break}| produces a line break after the name
%    of the theorem.  The font is |\slshape|.  Hence, we define
%    |\th@break| as follows:
%    \begin{macrocode}
\gdef\th@break{\normalfont\slshape
  \def\@begintheorem##1##2{\item[%
%    \end{macrocode}
%    We run into the following problem: it is not possible to create
%    the header with |\item[|\meta{title}|]| and then start
%    a new line by, for example, |\mbox{}\\|. Such a definition
%    will fail whenever a list environment follows immediately.  With
%    the above construction, the |\mbox{}| causes the switch
%    |@inlabel| (cf.\ definition of |\list| and
%    |\trivlist| in \cite{latex:source}) to be set to \texttt{false}
%    and so the following list will insert additional vertical
%    space (|\topskip|). This is quite annoying. Therefore, we
%    create the line break within the |\item|. In order to ensure
%    that the text will begin at the proper position in the following
%    line, we simply pretend that the label does not take any
%    room.\footnote{This will lead to problems whenever very high
%    symbols occurring in the line tower into the
%    heading. So, something else has to be done here sometime.}
%    \begin{macrocode}
       \rlap{\vbox{\hbox{\hskip \labelsep\theorem@headerfont ##1\ ##2}%
                   \hbox{\strut}}}]}%
%    \end{macrocode}
%    Again, the definition of |\@opargbegintheorem| is
%    completely analogous.
%    \begin{macrocode}
\def\@opargbegintheorem##1##2##3{%
  \item[\rlap{\vbox{\hbox{\hskip \labelsep \theorem@headerfont
                     ##1\ ##2\ (##3)}%
                    \hbox{\strut}}}]}}
\endgroup
%</thb>
%    \end{macrocode}
% \end{macro}
%
%
% \subsubsection{The \texttt{changebreak} style}
%
%    \begin{macrocode}
% This style option is stored in the file |thcb.sty|.
%    \begin{macrocode}
%<*thcb>
\begingroup \makeatletter
\@ifundefined{theorem@style}{\input{theorem.sty}}{}
\ProvidesFile{thcb.sty}
             [\FMithmInfo]
%    \end{macrocode}
%
% \begin{macro}{\th@changebreak}
%    The \texttt{change-break} theorem style is like \texttt{break}
%    but with interchange of theorem name and theorem number.  Thus we
%    define |\th@changebreak| as follows:
%    \begin{macrocode}
\gdef\th@changebreak{\normalfont\slshape
  \def\@begintheorem##1##2{\item
       [\rlap{\vbox{\hbox{\hskip\labelsep \theorem@headerfont ##2\ ##1}%
                    \hbox{\strut}}}]}%
\def\@opargbegintheorem##1##2##3{%
  \item[\rlap{\vbox{\hbox{\hskip\labelsep \theorem@headerfont
                     ##2\ ##1\ (##3)}%
                    \hbox{\strut}}}]}}
\endgroup
%</thcb>
%    \end{macrocode}
% \end{macro}
%
%
% \subsubsection{The \texttt{change} style}
%
% This style option is stored in the file |thc.sty|.
%    \begin{macrocode}
%<*thc>
\begingroup \makeatletter
\@ifundefined{theorem@style}{\input{theorem.sty}}{}
\ProvidesFile{thc.sty}
             [\FMithmInfo]
%    \end{macrocode}
%
% \begin{macro}{\th@change}
%    The \texttt{change} theorem style corresponds to the
%    \texttt{change break} style without a linebreak after the header.
%    To say it in another way, it's the same as the \texttt{plain}
%    style but with number and name interchanged and |\slshape| as the
%    default font.
%    \begin{macrocode}
\gdef\th@change{\normalfont\slshape
  \def\@begintheorem##1##2{\item
       [\hskip\labelsep \theorem@headerfont ##2\ ##1]}%
\def\@opargbegintheorem##1##2##3{%
  \item[\hskip\labelsep \theorem@headerfont ##2\ ##1\ (##3)]}}
\endgroup
%</thc>
%    \end{macrocode}
% \end{macro}
%
%
% \subsubsection{The \texttt{marginbreak} style}
%
% This style option is the one used most often at Mainz.  It is saved
% in the file |thmb.sty|.
%    \begin{macrocode}
%<*thmb>
\begingroup \makeatletter
\@ifundefined{theorem@style}{\input{theorem.sty}}{}
\ProvidesFile{thmb.sty}
             [\FMithmInfo]
%    \end{macrocode}
%
% \begin{macro}{\th@marginbreak}
%    The \texttt{margin break} style is nearly the same as the
%    \texttt{change break} style. The only difference is the placement
%    of the theorem number.  We use |\llap| to place it in the left
%    margin.
%
%    In this style |\labelsep| denotes the separation between the
%    number and the text.
%    \begin{macrocode}
\gdef\th@marginbreak{\normalfont\slshape
  \def\@begintheorem##1##2{\item
       [\rlap{\vbox{\theorem@headerfont
                    \hbox{\llap{##2}\hskip\labelsep ##1}%
                    \hbox{\strut}}}]}%
\def\@opargbegintheorem##1##2##3{%
  \item[\rlap{\vbox{\theorem@headerfont
                    \hbox{\llap{##2}\hskip\labelsep ##1\ (##3)}%
                    \hbox{\strut}}}]}}
\endgroup
%</thmb>
%    \end{macrocode}
% \changes{v2.1c}{90/02/17}{`endgroup outside macrocode env.}
% \end{macro}
%
%
%
% \subsubsection{The \texttt{margin} style}
%
% This style option is stored in the file |thm.sty|.
%    \begin{macrocode}
%<*thm>
\begingroup \makeatletter
\@ifundefined{theorem@style}{\input{theorem.sty}}{}
\ProvidesFile{thm.sty}
             [\FMithmInfo]
%    \end{macrocode}
%
% \begin{macro}{\th@margin}
%    Again this is only a variant of the theorem styles described
%    above without any new ideas.
%    \begin{macrocode}
\gdef\th@margin{\normalfont\slshape
  \def\@begintheorem##1##2{\item
       [\theorem@headerfont \llap{##2}\hskip\labelsep ##1]}%
\def\@opargbegintheorem##1##2##3{%
  \item[\theorem@headerfont \llap{##2}\hskip\labelsep ##1\ (##3)]}}
\endgroup
%</thm>
%    \end{macrocode}
% \end{macro}
%
%
%
%
% \subsection{Final Definitions}
%
% \begin{macro}{\theorempreskipamount}
% \begin{macro}{\theorempostskipamount}
%    The \textsf{skip} parameters that regulate the vertical empty space
%    before and after the {\envfont theorem} environment have
%    to be allocated as well.
% \changes{v2.0f}{89/07/20}{Unnecessary `global' removed.}
%    \begin{macrocode}
%<*package>
\newskip\theorempreskipamount
\newskip\theorempostskipamount
%    \end{macrocode}
%    Since we have used the same values for all theorem sets, we now can
%    assign them.
%    \begin{macrocode}
\global\setlength\theorempreskipamount{12pt plus 5pt minus 3pt}
\global\setlength\theorempostskipamount{8pt plus 3pt minus 1.5pt}
%    \end{macrocode}
% \end{macro}
% \end{macro}
%
%
% \begin{macro}{\@endtheorem}
%    The same holds for the macro |\@endtheorem|, which ends a
%    {\envfont theorem} environment. Since it is the same for all
%    theorem sets, it is removed from the macros |\th@|\meta{style}.
%    It simply ends the {\envfont trivlist} environment, which was
%    begun in |\@thm|.
%    \begin{macrocode}
\global\let\@endtheorem=\endtrivlist
%    \end{macrocode}
% \end{macro}
%
%
%
% \begin{macro}{\@preamblecmds}
%    All macros defined above are to be used only in the preamble.
%    Therefore, we insert them in |\@preamblecmds| which will disable
%    them at begin document. This is done by the internal \LaTeXe{}
%    command |\@onlypreamble|.
% \changes{v2.0g}{89/09/19}{`theoremheaderfont and `theorembodyfont
%    added.}
% \changes{v2.2a}{94/02/03}{Use \LaTeXe{} interface}
%    \begin{macrocode}
\@onlypreamble\@xnthm
\@onlypreamble\@ynthm
\@onlypreamble\@othm
\@onlypreamble\newtheorem
\@onlypreamble\theoremstyle
\@onlypreamble\theorembodyfont
\@onlypreamble\theoremheaderfont
%    \end{macrocode}
% \end{macro}
%
% Finally we declare the \texttt{plain} theorem style to be the default.
% \changes{v2.1b}{90/06/04}{Fixed missing `item by declaring default
%    style.}
%    \begin{macrocode}
\theoremstyle{plain}
%</package>
%    \end{macrocode}
%
% \Finale

\endinput
}{}
\ProvidesFile{thmb.sty}
             [\FMithmInfo]
%    \end{macrocode}
%
% \begin{macro}{\th@marginbreak}
%    The \texttt{margin break} style is nearly the same as the
%    \texttt{change break} style. The only difference is the placement
%    of the theorem number.  We use |\llap| to place it in the left
%    margin.
%
%    In this style |\labelsep| denotes the separation between the
%    number and the text.
%    \begin{macrocode}
\gdef\th@marginbreak{\normalfont\slshape
  \def\@begintheorem##1##2{\item
       [\rlap{\vbox{\theorem@headerfont
                    \hbox{\llap{##2}\hskip\labelsep ##1}%
                    \hbox{\strut}}}]}%
\def\@opargbegintheorem##1##2##3{%
  \item[\rlap{\vbox{\theorem@headerfont
                    \hbox{\llap{##2}\hskip\labelsep ##1\ (##3)}%
                    \hbox{\strut}}}]}}
\endgroup
%</thmb>
%    \end{macrocode}
% \changes{v2.1c}{90/02/17}{`endgroup outside macrocode env.}
% \end{macro}
%
%
%
% \subsubsection{The \texttt{margin} style}
%
% This style option is stored in the file |thm.sty|.
%    \begin{macrocode}
%<*thm>
\begingroup \makeatletter
\@ifundefined{theorem@style}{% \iffalse meta-comment
%
% Copyright (C) 1993-2024
%
% The LaTeX Project and any individual authors listed elsewhere
% in this file.
%
% This file is part of the Standard LaTeX `Tools Bundle'.
% -------------------------------------------------------
%
% It may be distributed and/or modified under the
% conditions of the LaTeX Project Public License, either version 1.3c
% of this license or (at your option) any later version.
% The latest version of this license is in
%    https://www.latex-project.org/lppl.txt
% and version 1.3c or later is part of all distributions of LaTeX
% version 2005/12/01 or later.
%
% The list of all files belonging to the LaTeX `Tools Bundle' is
% given in the file `manifest.txt'.
%
% \fi
%
%%
%
%
\def\FMithmInfo{2023/07/05 v2.2c Theorem extension package (FMi)}
%
% \ProvidesFile{theorem.dtx}[\FMithmInfo]
%\iffalse   % this is a METACOMMENT !
%%
%% Package `theorem' to use with LaTeX2e
%% Copyright (C) 1989-1995 Frank Mittelbach, all rights reserved.
%<+package>\@ifundefined{theorem@style}{}{\endinput}
%<+package>\NeedsTeXFormat{LaTeX2e}[1995/06/01]
%<+package>\ProvidesPackage{theorem}
%<+package>         [\FMithmInfo]
%
% \fi
%
% \changes{v1.0}{87/05/07}{First draft.}
% \changes{v1.0a}{87/05/07}{`thistheoremstyle' added.}
% \changes{v1.1a}{87/07/29}{New concept: redefine the internal
%                          \protect\LaTeX{} commands `@xnthm' etc.}
% \changes{v1.1b}{87/07/31}{`theoremfont' used to specify font.}
% \changes{v1.2a}{88/01/08}{`@endtheorem' now globally defined.}
% \changes{v1.2b}{88/01/11}{`newtheorem' only allowed in the preamble.}
% \changes{v1.3a}{88/02/14}{`theoremfont' now a toks register.}
%
% \changes{v2.0a}{88/11/16}{New implementation and new documentation.}
% \changes{v2.0a}{88/11/16}{No layout changes but internal modifications
%                           old substyles won't work.}
% \changes{v2.0b}{89/01/30}{`global' used for all definitions.}
% \changes{v2.0c}{89/06/15}{Documentation translated to English.
%                            Thanks to LLL and Nina}
% \changes{v2.1a}{90/02/15}{Support for new font selection.}
% \changes{v2.2a}{94/02/03}{Cleanup for LaTeX2e}
%
% \DoNotIndex{\@endparenv,\@ifdefinable,\@ifnextchar,\@ifundefined}
% \DoNotIndex{\@namedef,\@nameuse,\@spaces,\@tempa,\@warning}
% \DoNotIndex{\ }
% \DoNotIndex{\begingroup,\bfseries}
% \DoNotIndex{\csname}
% \DoNotIndex{\def,\do,\docdate}
% \DoNotIndex{\endcsname,\endgroup,\endinput,\endtrivlist,\expandafter}
% \DoNotIndex{\filedate,\fileversion}
% \DoNotIndex{\gdef,\global}
% \DoNotIndex{\hbox,\hskip}
% \DoNotIndex{\ignorespaces,\input,\itshape,\item}
% \DoNotIndex{\labelsep,\let,\llap}
% \DoNotIndex{\makeatletter}
% \DoNotIndex{\noexpand}
% \DoNotIndex{\refstepcounter,\relax,\rlap}
% \DoNotIndex{\setlength,\slshape,\space,\string,\strut}
% \DoNotIndex{\the,\trivlist,\typeout}
% \DoNotIndex{\vbox}
% \DoNotIndex{\xdef}
%
% \def\deflabel#1{\ttfamily #1\hfill}
% \def\deflist#1{\begin{list}{}{\settowidth\labelwidth{\ttfamily #1}%
%                               \setlength\leftmargin\labelwidth
%                               \addtolength\leftmargin\labelsep
%                               \let\makelabel\deflabel}}
% \def\enddeflist{\end{list}}
% \def\envfont{\normalfont\ttfamily}
%
% \MakeShortVerb{\|}
%
% \newcommand{\TUB}{{\sl TUGboat\/}}
%
% \GetFileInfo{theorem.dtx}
% \title{An Extension of the
%        \LaTeX{} theorem environment\thanks{This file
%        has version number \fileversion, last
%        revised \filedate.}}
% \author{Frank Mittelbach}
%
% \MaintainedByLaTeXTeam{tools}
% \maketitle
%
% \begin{abstract}
%    The macros described in this paper yield an extension of the
%    \LaTeX{} theorem mechanism. It is designed to satisfy the
%    different requirements of various journals. Thus, the
%    layout of the ``theorems'' can be manipulated by determining a
%    ``style''. This article describes not only the use, but
%    also the definition, of the necessary macros.
% \end{abstract}
%
% \section*{Preface to version 2.2}
%
% For \LaTeXe{} this package did not need any fundamental changes. I
% only modified the messages generated so that theorem layout styles
% will show up with the |\listfiles| command and cleaned the section
% on the New Font Selection Scheme since this is now included in
% \LaTeX.
%
% \section*{Preface to version 2.1}
%
% This version is identical to 2.0g described in \TUB\ 10\#3 except
% for some internal defaults which are now set depending on the used
% font selection scheme.
%
% This was done to avoid unpleasant surprises if the new font selection
% scheme is in force. For further details see
% section~\ref{sec:newfonts} and \cite{bk:GMS94}.
%
% \section{Introduction}
%
% For our purposes here, ``theorems'' are labelled enunciations,
% often set off from the main text by extra space and a font change.
% Theorems, corollaries, conjectures, definitions, and remarks
% are all instances of ``theorems''.  The ``header'' of these structures
% is composed of a label (such as \textsc{Theorem} or \textsc{Remark})
% and a number which serializes an item in the sequence of items
% with the same label.
%
% Shortly after the introduction of \LaTeX{} at the Fachbereich
% Mathematik in Mainz, the desire to manipulate the layout of
% ``theorems'' arose.  In Mainz, the following two conventions came
% into general use:
% \begin{enumerate}
%   \item The number of the theorem is shown in the margin.
%   \item There is a line break at
%         the end of the theorem header.
% \end{enumerate}
%
% Additionally, some journals require different formats which depend on
% the ``sort of theorem'': e.g.\ often remarks and definitions are set
% in |\upshape|, while |\itshape| is employed for main theorems.
%
% Confronted with these requirements, a theorem environment was
% developed in Mainz which allows separate determination of the
% layout of the ``theorems sets'', comparable to |\pagestyle|.
%
%
% \section{The user interface}
% \subsection{Defining new theorem sets}
%
% \DescribeMacro\newtheorem
% As in the original \LaTeX{} version, the command |\newtheorem|
% defines a new ``theorem set'' or ``theorem-like structure''.
% Two required arguments name the new environment and give the text
% to be typeset with each instance of the new ``set'', while
% an optional argument determines how
% the ``set'' is enumerated:
% \begin{description}
%    \item[\ttfamily \bslash newtheorem\{foo\}\{bar\}]
%       The theorem set {\envfont foo} (whose name is \texttt{bar})
%       uses its own counter.
%    \item[\ttfamily \bslash newtheorem\{foo2\}{[foo]}\{bar2\}]
%       The theorem set {\envfont foo2} (printed name \texttt{bar2})
%       uses the same counter as the theorem set \texttt{foo}.
%    \item[\ttfamily \bslash newtheorem\{foo3\}\{bar3\}{[section]}]
%       The theorem set {\envfont foo3} (printed name \texttt{bar3}) is
%       enumerated within the counter \texttt{section}, i.e.\ with every
%       new |\section| the enumeration begins again with 1, and
%       the enumeration is composed from the section-number and the
%       theorem counter itself.
% \end{description}
%
% \DescribeMacro\theoremstyle
% Additionally, the command |\theoremstyle| can define the layout
% of various, or all, theorem sets.  It should be noted that any
% theorem set defined by |\newtheorem| is typeset in the
% |\theoremstyle| that is current at the time of the definition.
% Thus, the following
% \begin{verbatim}
%    \theoremstyle{break}        \newtheorem{Cor}{Corollary}
%    \theoremstyle{plain}        \newtheorem{Exa}{Example}[section]
%\end{verbatim}                             ^^A No blank at line start
% leads to the result that the set {\envfont Cor} is formatted in the
% style \texttt{break}, while the set {\envfont Exa} and all the
% following ones are formatted in the style \texttt{plain}, unless
% another |\theoremstyle| follows.  Since the definitions installed by
% |\newtheorem| are global, one also can limit |\theoremstyle| locally
% by grouping braces.
%
% \DescribeMacro\theorembodyfont
% The choice of the font for the theorem body is completely
% independent of the chosen |\theoremstyle|; this has proven to
% be very advantageous.  For example,
% \begin{verbatim}
%   {\theorembodyfont{\upshape}          \newtheorem{Rem}{Remark}}
%\end{verbatim}
% defines a theorem set {\envfont Rem}, which will be set in
% |\upshape| in the current layout (which in our example is
% \texttt{plain}).  As with |\theoremstyle|, the |\theorembodyfont|
% chosen is that current at the time of |\newtheorem|.  If
% |\theorembodyfont| is not specified or one defines
% |\theorembodyfont{}|, then the font used will be that defined by the
% |\theoremstyle|.
%
% \DescribeMacro\theoremheaderfont It is also possible to customize
% the font used for the theorem headers.  This is, however, a global
% declaration, and therefore there should be at most one
% |\theoremheaderfont| declaration in the preamble.\footnote{If it is
% actually necessary to have different header fonts, one has to define
% new theorem styles (substituting the desired font) or specify the
% information directly in the \texttt{\bslash newtheorem} declaration
% (the unclean variant).}
%
% \DescribeMacro\theorempreskipamount
% \DescribeMacro\theorempostskipamount Two additional parameters
% affect the vertical space around the theorem environments:
% |\theorempreskipamount| and |\theorempostskipamount| define,
% respectively, the spacing before and after such an environment.
% These parameters apply for all theorem sets and can be manipulated
% with the ordinary length macros.  They are rubber lengths
% (`\textsf{skips}'), and therefore can contain \texttt{plus} and
% \texttt{minus} parts.
%
% Since the definition of theorem sets should---most sensibly---be
% placed in the preamble, we only allow installation there.
% It is therefore possible to release the memory used here after
% |\begin{document}|, in order to make room for other
% applications.
%
%
% \subsection{Existing theorem styles}
%
% The following theorem styles exist to date:
% \begin{deflist}{marginbreak}
%    \item[plain]
%       This theorem style emulates the original \LaTeX{} definition,
%       except that additionally the parameters
%       |\theorem...skipamount| are used.
%    \item[break]
%       In this style, the theorem header is followed by a line
%       break.
%    \item[marginbreak]
%       The theorem number is set in the margin, and there is a line
%       break as in \texttt{break}.
%    \item[changebreak]
%       Like \texttt{break}, but with header number and text
%       interchanged.
%    \item[change]
%       Header number and text are interchanged, without a line break.
%    \item[margin]
%       The number is set in the left margin, without a line break.
% \end{deflist}
% All styles (except \texttt{plain}) select |\slshape| as the default
% |\theorembodyfont|.
%
%
% \subsection{Examples}
%
% Given the above theorem sets {\envfont Cor}, {\envfont Exa}
% and {\envfont Rem},
% suppose that the preamble also contains the declarations:
% \begin{verbatim}
%    \theoremstyle{marginbreak}   \newtheorem{Lem}[Cor]{Lemma}
%    \theoremstyle{change}
%    \theorembodyfont{\itshape}        \newtheorem{Def}[Cor]{Definition}
%
%    \theoremheaderfont{\scshape}
%\end{verbatim}
% Then the following are some typical examples of the typeset output
% resulting from their use.
%
% \begin{Cor}
%    This is a sentence typeset in the theorem environment {\envfont
%    Cor}.
% \end{Cor}
%
% \begin{Exa}
%    This is a sentence typeset in the theorem environment {\envfont
%    Exa}.
% \end{Exa}
%
% \begin{Rem}
%    This is a sentence typeset in the theorem environment {\envfont
%    Rem}.
% \end{Rem}
%
% \begin{Lem}[Ben User]
%    This is a sentence typeset in the theorem environment {\envfont
%    Lem}.
% \end{Lem}
%
% \begin{Def}[Very impressive Definition]
%    This is a sentence typeset in the theorem environment {\envfont
%    Def}.
% \end{Def}
%
% The last two examples show the effect of the optional argument to
% a theorem environment (it is the text typeset in parentheses).
%
%
% \section{Special Considerations}
% \label{sec:newfonts}
%
% Theoremheader and body are implemented as a unit. This means that
% the |\theoremheaderfont|  will inherit characteristics of the
% |\theorembodyfont| in \LaTeXe. Thus, if for example
% |\theorembodyfont| is |\itshape| and |\theoremheaderfont|
% is |\bfseries| the font selected for the header will have the
% characteristics `bold extended italic'. If this is not desired one
% should set the |\theoremheaderfont| to something like
% \begin{verbatim}
%      \theoremheaderfont{\normalfont\bfseries}
%\end{verbatim}
% i.e.\ supplying all necessary font information explicitly.
%
%
%
% \section{Acknowledgements}
%
% The publication of this set of macros was only possible with the
% help of Christina Busse (translating the manuscript into English),
% Joachim Pense (playing the r\^ole of typist), Chris Rowley
% (looking everything over) and many others providing useful
% suggestions.
%
% \MaybeStop{%
%    \begin{thebibliography}{9}
%       \bibitem{bk:GMS94} \textsc{M.~Goossens}, \textsc{F.~Mittelbach}
%         and \textsc{A.~Samarin}.
%         \newblock The \LaTeX{} Companion.
%         \newblock
%         Addison-Wesley, Reading, Massachusetts, 1994.
%       \bibitem{latex:source} \textsc{Lamport, Leslie.} \newblock
%                               \texttt{latex.tex}, version 2.09, date
%                               Feb. 1990.
%    \end{thebibliography}
%    \PrintIndex
%    \PrintChanges}
%
%
% \section{The documentation driver file}
%
% The next bit of code contains the documentation driver file for
% \TeX{}, i.e., the file that will produce the documentation you are
% currently reading. It will be extracted from this file by the
% \texttt{docstrip} program. Since it is the first code in the file
% one can alternatively process this file directly with \LaTeXe{} to
% obtain the documentation.
%
%    \begin{macrocode}
%<*driver>
\documentclass{ltxdoc}

\usepackage{theorem}

%<+driver>% The next few lines define theorem sets which are used
%<+driver>% in the example section of the documentation.

\theoremstyle{break}       \newtheorem{Cor}{Corollary}
\theoremstyle{plain}       \newtheorem{Exa}{Example}[section]
{\theorembodyfont{\upshape}\newtheorem{Rem}{Remark}}
\theoremstyle{marginbreak} \newtheorem{Lem}[Cor]{Lemma}
\theoremstyle{change}
\theorembodyfont{\itshape} \newtheorem{Def}[Cor]{Definition}

\theoremheaderfont{\scshape}

\RecordChanges

\begin{document}
   \DocInput{theorem.dtx}
\end{document}
%</driver>
%    \end{macrocode}
%
% \section{Definition of the Macros}
%
% If the file has been loaded before, we abort immediately. If not the
% package announces itself (this is actually done at the very top of
% the file---the way it is done isn't good style so don't copy it).
% \changes{v2.0e}{89/07/19}{Spaces removed from `typeout'.}
% \changes{v2.2b}{95/11/19}{Announce moved to top of file}
%    \begin{macrocode}
%<*package>
%\@ifundefined{theorem@style}{}{\endinput}
%\def\FMithmInfo{1995/11/19 v2.2b Theorem extension package (FMi)}
%\ProvidesPackage{theorem}[\FMithmInfo]
%    \end{macrocode}
%
%
% \subsection{Definition of theorem styles and fonts}
%
% All the definitions in this file are done globally to allow
% inputting this file inside a group.
%
% \begin{macro}{\theoremstyle}
%    Before a theorem style can be installed, the chosen style must be
%    known. For that reason, we must test to see that
%    |\th@|\meta{style} is known or, more precisely, that it is
%    different from |\relax|. If the style is not known then
%    |\th@plain| is used.
%    \begin{macrocode}
\gdef\theoremstyle#1{%
   \@ifundefined{th@#1}{\@warning
          {Unknown theoremstyle `#1'. Using `plain'}%
          \theorem@style{plain}}%
%    \end{macrocode}
%    We save the theorem style to be used in the token register
%    |\theorem@style|.
%    \begin{macrocode}
      {\theorem@style{#1}}%
%    \end{macrocode}
%    Now we ``evaluate'' the theorem style: this means, we call the
%    macro |\th@|\meta{style} which will activate the relevant
%    definitions which are contained in a separate file.  This is done
%    in a group to suppress changes to the current font.  This could
%    otherwise pose problems together with the new font selection
%    scheme\footnote{When I printed the original article using the new
%    font selection scheme I ended with a document with slanted
%    typefaces (text headings and all) simply because one of the
%    theorem styles used \texttt{\bslash sl} at toplevel.} if the
%    |\th@|\meta{style} is evaluated a second time.
% \changes{v2.1a}{90/02/15}{Group added around `th@..}
% \changes{v2.1a}{90/02/15}{Reseting of `..@begintheorem removed.}
%    \begin{macrocode}
      \begingroup
        \csname th@\the\theorem@style \endcsname
      \endgroup}
%    \end{macrocode}
% \end{macro}
%
%
% \begin{macro}{\@begintheorem}
% \changes{v2.1a}{90/02/15}{Relaxed at toplevel.}
% \begin{macro}{\@opargbegintheorem}
% \changes{v2.1a}{90/02/15}{Relaxed at toplevel.}
%    We reset |\@begintheorem| and |\@opargbegintheorem| to
%    |\relax| since these commands are no longer necessary at
%    toplevel. This will save a few tokens.
%    \begin{macrocode}
\global\let\@begintheorem\relax
\global\let\@opargbegintheorem\relax
%    \end{macrocode}
% \end{macro}
% \end{macro}
%
% \begin{macro}{\theorem@style}
%    Obviously the \textsf{token} register used above has to be
%    allocated. To assure the utmost compatibility with the
%    original \LaTeX{} definition, we set the default theorem style to
%    \texttt{plain}, which implements the usual \LaTeX{} convention.
% \changes{v2.0f}{89/07/20}{Unnecessary `global' removed.}
%    \begin{macrocode}
\newtoks\theorem@style
\global\theorem@style{plain}
%    \end{macrocode}
% \end{macro}
%
% \begin{macro}{\theorembodyfont}
% \begin{macro}{\theorem@bodyfont}
%    If the |\theorembodyfont| is set by the user then it should not
%    interact with the default font set in the theorem style. When the
%    new font selection is in force this may happen if, for example,
%    the default is |\itshape| and the new |\theorembodyfont| is
%    |\sffamily|. So we add a |\reset@font| command in front of the
%    user definition.
% \changes{v2.0f}{89/07/20}{Unnecessary `global' removed.}
% \changes{v2.0g}{89/09/19}{`theoremfont to `theorembodyfont.}
% \changes{v2.1c}{92/06/22}{`reset@font added and logic changed.}
% \changes{v2.1d}{92/10/28}{Reset to empty if arg is empty}
%    \begin{macrocode}
\gdef\theorembodyfont#1{%
%    \end{macrocode}
%    We check if the argument supplied is empty and if so put nothing
%    into the |\theorem@bodyfont| token register to allow for
%    |\theorembodyfont{}| as a mean of using the default of the
%    current |\theoremstyle|.
%    \begin{macrocode}
   \def\@tempa{#1}%
   \ifx\@tempa\@empty
    \theorem@bodyfont{}%
   \else
    \theorem@bodyfont{\reset@font#1}%
   \fi
   }
\newtoks\theorem@bodyfont
\global\theorem@bodyfont{}
%    \end{macrocode}
% \end{macro}
% \end{macro}
%
%
% \begin{macro}{\theoremheaderfont}
% \changes{v2.0g}{89/09/19}{Macro added.}
%    The font for the theorem headers is handled differently because
%    this definition applies to all theorem styles.
%    \begin{macrocode}
\gdef\theoremheaderfont#1{\gdef\theorem@headerfont{#1}%
%    \end{macrocode}
%    After using the macro once it is redefined to produce an error
%    message.
%    \begin{macrocode}
       \gdef\theoremheaderfont##1{%
        \typeout{\string\theoremheaderfont\space should be used
                 only once.}}}
%    \end{macrocode}
% \end{macro}
%
% \begin{macro}{\theorem@headerfont}
% \changes{v2.0g}{89/09/19}{Macro added.}
%    To set the |\theorem@headerfont| default we first test if
%    the new fontselection scheme is in force.
% \changes{v2.1a}{90/02/15}{Support for new font selection.}
%    \begin{macrocode}
\ifx\upshape\undefined
%    \end{macrocode}
%    If not we define it to expand into |\bfseries|. We don't use
%    |\let| just in case a following style option redefines this
%    macro.
%    \begin{macrocode}
\gdef\theorem@headerfont{\bfseries}
%    \end{macrocode}
%    Otherwise we reset the current shape before calling |\bfseries|.
%    \begin{macrocode}
\else \gdef\theorem@headerfont{\normalfont\bfseries}\fi
%    \end{macrocode}
% \end{macro}
%
% \begin{macro}{\th@plain}
% \begin{macro}{\th@break}
% \begin{macro}{\th@marginbreak}
% \begin{macro}{\th@changebreak}
% \begin{macro}{\th@change}
% \begin{macro}{\th@margin}
%    The different styles are defined in macros such as
%    |\th@plain|.  Since memory space is precious in
%    ``non-Big-versions'', we have to avoid offering too many unused
%    definitions.  Therefore we define these styles in separate files
%    that can be loaded on demand.  Thus the commands themselves only
%    load these files. We use |\@input@|, a \LaTeXe{} internal command
%    that ensures that the file will be listed with |\listfiles|.
% \changes{v2.2a}{94/02/02}{Use `@input@ to load theorem layout files}
%    \begin{macrocode}
\gdef\th@plain{\@input@{thp.sty}}
\gdef\th@break{\@input@{thb.sty}}
\gdef\th@marginbreak{\@input@{thmb.sty}}
\gdef\th@changebreak{\@input@{thcb.sty}}
\gdef\th@change{\@input@{thc.sty}}
\gdef\th@margin{\@input@{thm.sty}}
%    \end{macrocode}
%    This list will be expanded when new styles become
%    available.  For testing,  just append new theorem substyles
%    as document options.
% \end{macro}
% \end{macro}
% \end{macro}
% \end{macro}
% \end{macro}
% \end{macro}
%
%
%
% \subsection{Definition of a new theorem set}
%
% As already pointed out, a new theorem environment can be defined in
% three different ways:
% \begin{verbatim}
%    \newtheorem{Lem}{Lemma}
%    \newtheorem{Lem}{Lemma}[section]
%    \newtheorem{Lem}[Theorem]{Lemma}
%\end{verbatim}
% The function of the macro |\newtheorem| is to recognize these cases
% and then to branch into one of the three macros |\@ynthm|, |\@xnthm|
% or |\@othm|.  This mechanism is adopted unchanged from
% \cite{latex:source}; the essential point here is that, for example,
% in the second case, the arguments \texttt{Lem}, \texttt{Lemma} and
% \texttt{section} are passed over to the macro |\@xnthm|.
%
% We inspect this case first because the others present fewer problems,
% and thus are easily derived from this one.
%
% \begin{macro}{\@xnthm}
%    For our example arguments, the macro |\@xnthm| must fulfill
%    the following:
%    \begin{itemize}
%       \item
%          Define a new \LaTeX-counter `Lem'
%       \item
%          reset this counter within a |\section|
%       \item
%          define the macro |\theLem|
%       \item
%          define the environment macros |\Lem| and
%          |\endLem| using the current
%          |\theoremstyle| and |\theorem@bodyfont|.
%    \end{itemize}
%    Obviously, all this should happen only if the first argument
%    of |\@xnthm| (i.e.\ \texttt{Lem} in our example) is chosen so
%    as not to conflict with any previously defined commands or
%    environments.  This test is performed by the \LaTeX{} macro
%    |\@ifdefinable|.
%    \begin{macrocode}
\gdef\@xnthm#1#2[#3]{\expandafter\@ifdefinable\csname #1\endcsname
%    \end{macrocode}
%    Therefore, the first argument of |\@ifdefinable| is the
%    expansion (in the example, |\Lem|) of
%    |\csname||#1\endcsname|.  The second argument is
%    executed only if the test has been completed successfully.
%    \begin{macrocode}
   {%
%    \end{macrocode}
%    Now we define the new counter. The names of the \LaTeX{}
%    macros employed should speak for themselves:
% \changes{v2.2b}{1995/10/07}{Use \cs{@newctr}}
% \changes{v2.2c}{1995/11/23}{Use \cs{@newctr} properly.}
%    \begin{macrocode}
    \@definecounter{#1}\@newctr{#1}[#3]%
%    \end{macrocode}
%    Using |\@newctr| will give a proper error message if the counter
%    in |#3| is not defined.
%    In defining `|\theLem|' we must generate the desired macro
%    name by use of |\expandafter| and |\csname|.
%    \begin{macrocode}
    \expandafter\xdef\csname the#1\endcsname
%    \end{macrocode}
%    An |\xdef| is used in order to make the definition global,
%    and to ensure that it contains the replacement texts of
%    |\@thmcountersep| and |\@thmcounter|.\footnote{These
%    two macros can be defined by the document style.  Their default
%    values produce a `\texttt{.}' as separation and an arabic
%    representation of the number.} However, not everything should be
%    expanded. For example, it saves space to use |\thesection|
%    instead of its---at times---lengthy expansion.
%    \begin{macrocode}
      {\expandafter \noexpand \csname the#3\endcsname
       \@thmcountersep \@thmcounter{#1}}%
%    \end{macrocode}
%    Thus with the defaults of \LaTeX{}, |\theLem| would be
%    replaced by the command sequence |\thesection.\arabic{Lem}|.
%
%    We will now look at the definition of the macro which is executed
%    at the beginning of the actual environment (in our example this
%    macro is |\Lem|). It should be noted that we use an
%    ``|\expandafter| trick'' to expand only certain parts of the
%    replacement text at the time of the definition.
%    \begin{macrocode}
    \def\@tempa{\global\@namedef{#1}}%
    \expandafter \@tempa \expandafter{%
%    \end{macrocode}
%    First, the macro that contains the current definitions of
%    |\@begintheorem| and |\@opargtheorem| should be called
%    up.  The name of this macro---as is already known---has the form
%    |\th@|\meta{theorem\ style}; therefore, it must be called by
%    \begin{macrocode}
      \csname th@\the \theorem@style
%    \end{macrocode}
%    In addition the default theorem font should be changeable, i.e.\
%    we have to insert the contents of |\theorem@bodyfont|. For
%    that reason, we expand even further, beyond |\endcsname|,
%    and thus insert the contents of the \textsf{token} register
%    |\theorem@bodyfont| in the replacement text.
%    \begin{macrocode}
            \expandafter \endcsname \the \theorem@bodyfont
%    \end{macrocode}
%    Now it is time to call the macro |\@thm| which takes over
%    the further processing.  It has two arguments: the current
%    counter name (in our example, \texttt{Lem}), and the text of the
%    label (in our example, \texttt{Lemma}).
%    \begin{macrocode}
     \@thm{#1}{#2}}%
%    \end{macrocode}
%    With this, the `sub-definition' is complete.  The macro
%    |\@endtheorem| ends a theorem environment and is, so far, nothing
%    but an |\endtrivlist|.  (Hence it is defined globally, and not
%    within the theorem styles.\footnote{This has to be changed as
%    soon as theorem styles that change \texttt{\bslash @endtheorem}
%    exist.  In such a case, all existing styles must be changed as
%    well since they will have to reset the macro.}) Therefore, we can
%    set it equivalent to the macro that ends the theorem set (in our
%    example, |\endLem|).  However, if some day theorem styles exist
%    that do change |\@endtheorem|, we would have to use the
%    commented-out line instead.
%    \begin{macrocode}
    \global \expandafter \let \csname end#1\endcsname \@endtheorem
%   \global\@namedef{end#1}{\@endtheorem}%
%    \end{macrocode}
%    With these commands all the required definitions are employed,
%    unless the test |\@ifdefinable| has failed.  Therefore, we
%    end the second argument of this macro and with it the definition
%    of |\@xnthm|.
%    \begin{macrocode}
   }}
%    \end{macrocode}
% \end{macro}
%
% \begin{macro}{\@ynthm}
%    The definition of |\@ynthm| is completely analogous. In
%    this case the new counter that is defined is not reset within
%    another counter; thus the definition of |\the...|
%    is simplified:
%    \begin{macrocode}
\gdef\@ynthm#1#2{\expandafter\@ifdefinable\csname #1\endcsname
   {\@definecounter{#1}%
    \expandafter\xdef\csname the#1\endcsname{\@thmcounter{#1}}%
%    \end{macrocode}
%    The rest of the definition corresponds literally to that of
%    |\@xnthm|:
%    \begin{macrocode}
    \def\@tempa{\global\@namedef{#1}}\expandafter \@tempa
     \expandafter{\csname th@\the \theorem@style \expandafter
     \endcsname \the\theorem@bodyfont \@thm{#1}{#2}}%
    \global \expandafter \let \csname end#1\endcsname \@endtheorem}}
%    \end{macrocode}
% \end{macro}
%
% \begin{macro}{\@othm}
%    The definition of |\@othm| does not contain anything new.
%    \begin{macrocode}
\gdef\@othm#1[#2]#3{%
%    \end{macrocode}
% \changes{v2.2b}{95/10/07}{Build in check for counter defined}
%    We do not define a new counter but instead use one that has
%    already been defined.  Thus the only definition we need is that
%    of this pseudo-counter (i.e.\ |\the|\meta{env.\ name}).
%    First we check if |#2| corresponds to a known counter name.
%    \begin{macrocode}
  \expandafter\ifx\csname c@#2\endcsname\relax
   \@nocounterr{#2}%
  \else
   \expandafter\@ifdefinable\csname #1\endcsname
   {\expandafter \xdef \csname the#1\endcsname
     {\expandafter \noexpand \csname the#2\endcsname}%
%    \end{macrocode}
%    All other parts of the definition can be adopted from
%    |\@xnthm|. We have to remember, though, that in this case
%    the name of the current counter and the theorem label have moved
%    to the second and third arguments.
%    \begin{macrocode}
    \def\@tempa{\global\@namedef{#1}}\expandafter \@tempa
     \expandafter{\csname th@\the \theorem@style \expandafter
     \endcsname \the\theorem@bodyfont \@thm{#2}{#3}}%
    \global \expandafter \let \csname end#1\endcsname \@endtheorem}%
  \fi}
%    \end{macrocode}
% \end{macro}
%
% \subsection{Macros that are employed in a {\envfont theorem}
%    environment}
%
% \begin{macro}{\@thm}
%    The macro |\@thm| has to increase the current counter.  Then,
%    depending on whether the environment has (or does not have) an
%    optional argument, it has to branch into either |\@begintheorem|
%    or |\@opargtheorem|.
%    \begin{macrocode}
\gdef\@thm#1#2{\refstepcounter{#1}%
%    \end{macrocode}
%    Now we start a {\envfont trivlist} environment, and give
%    |\@topsep| and |\@topsepadd| the values of the \textsf{skip}
%    registers |\theorempreskipamount| and |\theorempostskipamount|.
%    The value in |\@topsep| is the vertical space that is inserted by
%    the first (and only) |\item| in our |\trivlist| whilst
%    |\@topsepadd| is inserted by |\@endparenv| at the end of that
%    {\envfont trivlist} environment.  By using these registers, we
%    obtain the desired space around a {\envfont theorem} environment.
%    \begin{macrocode}
   \trivlist
   \@topsep \theorempreskipamount               % used by first \item
   \@topsepadd \theorempostskipamount           % used by \@endparenv
%    \end{macrocode}
%    Now we have to test whether an optional argument has been given.
% \changes{v2.0d}{89/07/18}{Forgotten percent added.}
%    \begin{macrocode}
   \@ifnextchar [%
%    \end{macrocode}
%    If there is an optional argument, we will call |\@ythm|, and
%    move the arguments read back into the input stream.
%    \begin{macrocode}
   {\@ythm{#1}{#2}}%
%    \end{macrocode}
%    If not, we call |\@begintheorem|. Its first argument is the
%    name of the theorem set (hence the second argument of
%    |\@thm|). Its second argument is the macro that produces the
%    current number.
%    \begin{macrocode}
   {\@begintheorem{#2}{\csname the#1\endcsname}\ignorespaces}}
%    \end{macrocode}
% \end{macro}
%
% \begin{macro}{\@xthm}
% \begin{macro}{\@ythm}
%    Both these macros were originally called by |\@thm|.  We do
%    not need |\@xthm| anymore, hence we reset it to
%    |\relax|. The definition of |\@ythm| has not changed at
%    all from its definition in \LaTeX. In order to make the macros
%    easier to understand, we will
%    nevertheless present it (commented out).
%    \begin{macrocode}
\global\let\@xthm\relax
% \def\@ythm#1#2[#3]{\@opargbegintheorem{#2}{\csname
%           the#1\endcsname}{#3}\ignorespaces}
%</package>
%    \end{macrocode}
%    The primitive |\ignorespaces| in |\@ythm| and
%    |\@thm| is needed to remove the spaces between the
%    |\begin{...}| and the actual text.
% \end{macro}
% \end{macro}
%
%
% \subsection{Definition of the theorem substyles}
%
% As already pointed out, the theorem substyles, defined below, are
% only loaded when necessary.  Note that all these substyles, except
% \texttt{plain}, have |\slshape| as the default body font.
%
% \subsubsection{The \texttt{plain} style}
%
% \label{dok:th@plain} As the following macros use |@|, we have
% to locally set the |\catcode| of this symbol to ``letter''.
% This happens within a group, so that we do not have to worry about
% which |\catcode| that symbol had before.
%    \begin{macrocode}
%<*thp>
\begingroup \makeatletter
%    \end{macrocode}
% Since we are now within a group, we must make all definitions
% globally.  First we make sure that \texttt{theorem.sty} is loaded.
% This will allow us to use this file as a document style option
% without having to call \texttt{theorem} itself as an option.  At the
% same time, we assure that at least version 2 is loaded, since
% |\theorem@style| was not defined in earlier versions.
%    \begin{macrocode}
\@ifundefined{theorem@style}{\input{theorem.sty}}{}
\ProvidesFile{thp.sty}
             [\FMithmInfo]
%    \end{macrocode}
%
%
% \begin{macro}{\th@plain}
% \begin{macro}{\@begintheorem}
% \begin{macro}{\@opargbegintheorem}
%    |\theoremstyle{plain}| corresponds to the original
%    definition, except that the distances to the surrounding text are
%    determined by the parameters |\theorempreskipamount| and
%    |\theorempostskipamount|.
%    First we set the default body font.
%    \begin{macrocode}
\gdef\th@plain{\normalfont\itshape
%    \end{macrocode}
%    Then we define |\@begintheorem| and
%    |\@opargbegintheorem|. These two macros define how the
%    header of a theorem is typeset.  |\@opargbegintheorem| will
%    be called if a \textsf{theorem} environment with an optional
%    argument is encountered; otherwise, the header is constructed by
%    calling |\@begintheorem|.  If one of these macros is
%    executed, we are within a \textsf{trivlist} environment started by
%    |\@thm|.  So the theorem header is produced with an
%    |\item| command.
%
%    Instead of specifying the header font directly, all standard
%    theorem styles use the |\theorem@headerfont| macro to allow
%    customization.  The extra space (|\labelsep|) is necessary
%    because of problems in the \textsf{trivlist} environment.
%    \begin{macrocode}
  \def\@begintheorem##1##2{%
        \item[\hskip\labelsep \theorem@headerfont ##1\ ##2]}%
%    \end{macrocode}
%    The definition of |\@opargbegintheorem| is completely
%    analogous. The only difference is the fact that there exists a
%    third argument (which is the optional parameter of the
%    environment and contains additional information about the
%    theorem).  Customarily we enclose it in parentheses.
%    \begin{macrocode}
\def\@opargbegintheorem##1##2##3{%
   \item[\hskip\labelsep \theorem@headerfont ##1\ ##2\ (##3)]}}
%    \end{macrocode}
% \end{macro}
% \end{macro}
% \end{macro}
%
% \noindent
% We conclude with an |\endgroup| to restore the
% |\catcode| of |@|.
%    \begin{macrocode}
\endgroup
%</thp>
%    \end{macrocode}
%
%
% \subsubsection{The \texttt{break} style}
%
% This style option is stored in the file |thb.sty|.
% For the next two lines see the documentation for |\th@plain| on
% page~\pageref{dok:th@plain}.
%    \begin{macrocode}
%<*thb>
\begingroup \makeatletter
\@ifundefined{theorem@style}{\input{theorem.sty}}{}
\ProvidesFile{thb.sty}
             [\FMithmInfo]
%    \end{macrocode}
%
% \begin{macro}{\th@break}
%    |\theoremstyle{break}| produces a line break after the name
%    of the theorem.  The font is |\slshape|.  Hence, we define
%    |\th@break| as follows:
%    \begin{macrocode}
\gdef\th@break{\normalfont\slshape
  \def\@begintheorem##1##2{\item[%
%    \end{macrocode}
%    We run into the following problem: it is not possible to create
%    the header with |\item[|\meta{title}|]| and then start
%    a new line by, for example, |\mbox{}\\|. Such a definition
%    will fail whenever a list environment follows immediately.  With
%    the above construction, the |\mbox{}| causes the switch
%    |@inlabel| (cf.\ definition of |\list| and
%    |\trivlist| in \cite{latex:source}) to be set to \texttt{false}
%    and so the following list will insert additional vertical
%    space (|\topskip|). This is quite annoying. Therefore, we
%    create the line break within the |\item|. In order to ensure
%    that the text will begin at the proper position in the following
%    line, we simply pretend that the label does not take any
%    room.\footnote{This will lead to problems whenever very high
%    symbols occurring in the line tower into the
%    heading. So, something else has to be done here sometime.}
%    \begin{macrocode}
       \rlap{\vbox{\hbox{\hskip \labelsep\theorem@headerfont ##1\ ##2}%
                   \hbox{\strut}}}]}%
%    \end{macrocode}
%    Again, the definition of |\@opargbegintheorem| is
%    completely analogous.
%    \begin{macrocode}
\def\@opargbegintheorem##1##2##3{%
  \item[\rlap{\vbox{\hbox{\hskip \labelsep \theorem@headerfont
                     ##1\ ##2\ (##3)}%
                    \hbox{\strut}}}]}}
\endgroup
%</thb>
%    \end{macrocode}
% \end{macro}
%
%
% \subsubsection{The \texttt{changebreak} style}
%
%    \begin{macrocode}
% This style option is stored in the file |thcb.sty|.
%    \begin{macrocode}
%<*thcb>
\begingroup \makeatletter
\@ifundefined{theorem@style}{\input{theorem.sty}}{}
\ProvidesFile{thcb.sty}
             [\FMithmInfo]
%    \end{macrocode}
%
% \begin{macro}{\th@changebreak}
%    The \texttt{change-break} theorem style is like \texttt{break}
%    but with interchange of theorem name and theorem number.  Thus we
%    define |\th@changebreak| as follows:
%    \begin{macrocode}
\gdef\th@changebreak{\normalfont\slshape
  \def\@begintheorem##1##2{\item
       [\rlap{\vbox{\hbox{\hskip\labelsep \theorem@headerfont ##2\ ##1}%
                    \hbox{\strut}}}]}%
\def\@opargbegintheorem##1##2##3{%
  \item[\rlap{\vbox{\hbox{\hskip\labelsep \theorem@headerfont
                     ##2\ ##1\ (##3)}%
                    \hbox{\strut}}}]}}
\endgroup
%</thcb>
%    \end{macrocode}
% \end{macro}
%
%
% \subsubsection{The \texttt{change} style}
%
% This style option is stored in the file |thc.sty|.
%    \begin{macrocode}
%<*thc>
\begingroup \makeatletter
\@ifundefined{theorem@style}{\input{theorem.sty}}{}
\ProvidesFile{thc.sty}
             [\FMithmInfo]
%    \end{macrocode}
%
% \begin{macro}{\th@change}
%    The \texttt{change} theorem style corresponds to the
%    \texttt{change break} style without a linebreak after the header.
%    To say it in another way, it's the same as the \texttt{plain}
%    style but with number and name interchanged and |\slshape| as the
%    default font.
%    \begin{macrocode}
\gdef\th@change{\normalfont\slshape
  \def\@begintheorem##1##2{\item
       [\hskip\labelsep \theorem@headerfont ##2\ ##1]}%
\def\@opargbegintheorem##1##2##3{%
  \item[\hskip\labelsep \theorem@headerfont ##2\ ##1\ (##3)]}}
\endgroup
%</thc>
%    \end{macrocode}
% \end{macro}
%
%
% \subsubsection{The \texttt{marginbreak} style}
%
% This style option is the one used most often at Mainz.  It is saved
% in the file |thmb.sty|.
%    \begin{macrocode}
%<*thmb>
\begingroup \makeatletter
\@ifundefined{theorem@style}{\input{theorem.sty}}{}
\ProvidesFile{thmb.sty}
             [\FMithmInfo]
%    \end{macrocode}
%
% \begin{macro}{\th@marginbreak}
%    The \texttt{margin break} style is nearly the same as the
%    \texttt{change break} style. The only difference is the placement
%    of the theorem number.  We use |\llap| to place it in the left
%    margin.
%
%    In this style |\labelsep| denotes the separation between the
%    number and the text.
%    \begin{macrocode}
\gdef\th@marginbreak{\normalfont\slshape
  \def\@begintheorem##1##2{\item
       [\rlap{\vbox{\theorem@headerfont
                    \hbox{\llap{##2}\hskip\labelsep ##1}%
                    \hbox{\strut}}}]}%
\def\@opargbegintheorem##1##2##3{%
  \item[\rlap{\vbox{\theorem@headerfont
                    \hbox{\llap{##2}\hskip\labelsep ##1\ (##3)}%
                    \hbox{\strut}}}]}}
\endgroup
%</thmb>
%    \end{macrocode}
% \changes{v2.1c}{90/02/17}{`endgroup outside macrocode env.}
% \end{macro}
%
%
%
% \subsubsection{The \texttt{margin} style}
%
% This style option is stored in the file |thm.sty|.
%    \begin{macrocode}
%<*thm>
\begingroup \makeatletter
\@ifundefined{theorem@style}{\input{theorem.sty}}{}
\ProvidesFile{thm.sty}
             [\FMithmInfo]
%    \end{macrocode}
%
% \begin{macro}{\th@margin}
%    Again this is only a variant of the theorem styles described
%    above without any new ideas.
%    \begin{macrocode}
\gdef\th@margin{\normalfont\slshape
  \def\@begintheorem##1##2{\item
       [\theorem@headerfont \llap{##2}\hskip\labelsep ##1]}%
\def\@opargbegintheorem##1##2##3{%
  \item[\theorem@headerfont \llap{##2}\hskip\labelsep ##1\ (##3)]}}
\endgroup
%</thm>
%    \end{macrocode}
% \end{macro}
%
%
%
%
% \subsection{Final Definitions}
%
% \begin{macro}{\theorempreskipamount}
% \begin{macro}{\theorempostskipamount}
%    The \textsf{skip} parameters that regulate the vertical empty space
%    before and after the {\envfont theorem} environment have
%    to be allocated as well.
% \changes{v2.0f}{89/07/20}{Unnecessary `global' removed.}
%    \begin{macrocode}
%<*package>
\newskip\theorempreskipamount
\newskip\theorempostskipamount
%    \end{macrocode}
%    Since we have used the same values for all theorem sets, we now can
%    assign them.
%    \begin{macrocode}
\global\setlength\theorempreskipamount{12pt plus 5pt minus 3pt}
\global\setlength\theorempostskipamount{8pt plus 3pt minus 1.5pt}
%    \end{macrocode}
% \end{macro}
% \end{macro}
%
%
% \begin{macro}{\@endtheorem}
%    The same holds for the macro |\@endtheorem|, which ends a
%    {\envfont theorem} environment. Since it is the same for all
%    theorem sets, it is removed from the macros |\th@|\meta{style}.
%    It simply ends the {\envfont trivlist} environment, which was
%    begun in |\@thm|.
%    \begin{macrocode}
\global\let\@endtheorem=\endtrivlist
%    \end{macrocode}
% \end{macro}
%
%
%
% \begin{macro}{\@preamblecmds}
%    All macros defined above are to be used only in the preamble.
%    Therefore, we insert them in |\@preamblecmds| which will disable
%    them at begin document. This is done by the internal \LaTeXe{}
%    command |\@onlypreamble|.
% \changes{v2.0g}{89/09/19}{`theoremheaderfont and `theorembodyfont
%    added.}
% \changes{v2.2a}{94/02/03}{Use \LaTeXe{} interface}
%    \begin{macrocode}
\@onlypreamble\@xnthm
\@onlypreamble\@ynthm
\@onlypreamble\@othm
\@onlypreamble\newtheorem
\@onlypreamble\theoremstyle
\@onlypreamble\theorembodyfont
\@onlypreamble\theoremheaderfont
%    \end{macrocode}
% \end{macro}
%
% Finally we declare the \texttt{plain} theorem style to be the default.
% \changes{v2.1b}{90/06/04}{Fixed missing `item by declaring default
%    style.}
%    \begin{macrocode}
\theoremstyle{plain}
%</package>
%    \end{macrocode}
%
% \Finale

\endinput
}{}
\ProvidesFile{thm.sty}
             [\FMithmInfo]
%    \end{macrocode}
%
% \begin{macro}{\th@margin}
%    Again this is only a variant of the theorem styles described
%    above without any new ideas.
%    \begin{macrocode}
\gdef\th@margin{\normalfont\slshape
  \def\@begintheorem##1##2{\item
       [\theorem@headerfont \llap{##2}\hskip\labelsep ##1]}%
\def\@opargbegintheorem##1##2##3{%
  \item[\theorem@headerfont \llap{##2}\hskip\labelsep ##1\ (##3)]}}
\endgroup
%</thm>
%    \end{macrocode}
% \end{macro}
%
%
%
%
% \subsection{Final Definitions}
%
% \begin{macro}{\theorempreskipamount}
% \begin{macro}{\theorempostskipamount}
%    The \textsf{skip} parameters that regulate the vertical empty space
%    before and after the {\envfont theorem} environment have
%    to be allocated as well.
% \changes{v2.0f}{89/07/20}{Unnecessary `global' removed.}
%    \begin{macrocode}
%<*package>
\newskip\theorempreskipamount
\newskip\theorempostskipamount
%    \end{macrocode}
%    Since we have used the same values for all theorem sets, we now can
%    assign them.
%    \begin{macrocode}
\global\setlength\theorempreskipamount{12pt plus 5pt minus 3pt}
\global\setlength\theorempostskipamount{8pt plus 3pt minus 1.5pt}
%    \end{macrocode}
% \end{macro}
% \end{macro}
%
%
% \begin{macro}{\@endtheorem}
%    The same holds for the macro |\@endtheorem|, which ends a
%    {\envfont theorem} environment. Since it is the same for all
%    theorem sets, it is removed from the macros |\th@|\meta{style}.
%    It simply ends the {\envfont trivlist} environment, which was
%    begun in |\@thm|.
%    \begin{macrocode}
\global\let\@endtheorem=\endtrivlist
%    \end{macrocode}
% \end{macro}
%
%
%
% \begin{macro}{\@preamblecmds}
%    All macros defined above are to be used only in the preamble.
%    Therefore, we insert them in |\@preamblecmds| which will disable
%    them at begin document. This is done by the internal \LaTeXe{}
%    command |\@onlypreamble|.
% \changes{v2.0g}{89/09/19}{`theoremheaderfont and `theorembodyfont
%    added.}
% \changes{v2.2a}{94/02/03}{Use \LaTeXe{} interface}
%    \begin{macrocode}
\@onlypreamble\@xnthm
\@onlypreamble\@ynthm
\@onlypreamble\@othm
\@onlypreamble\newtheorem
\@onlypreamble\theoremstyle
\@onlypreamble\theorembodyfont
\@onlypreamble\theoremheaderfont
%    \end{macrocode}
% \end{macro}
%
% Finally we declare the \texttt{plain} theorem style to be the default.
% \changes{v2.1b}{90/06/04}{Fixed missing `item by declaring default
%    style.}
%    \begin{macrocode}
\theoremstyle{plain}
%</package>
%    \end{macrocode}
%
% \Finale

\endinput
}{}
\ProvidesFile{thmb.sty}
             [\FMithmInfo]
%    \end{macrocode}
%
% \begin{macro}{\th@marginbreak}
%    The \texttt{margin break} style is nearly the same as the
%    \texttt{change break} style. The only difference is the placement
%    of the theorem number.  We use |\llap| to place it in the left
%    margin.
%
%    In this style |\labelsep| denotes the separation between the
%    number and the text.
%    \begin{macrocode}
\gdef\th@marginbreak{\normalfont\slshape
  \def\@begintheorem##1##2{\item
       [\rlap{\vbox{\theorem@headerfont
                    \hbox{\llap{##2}\hskip\labelsep ##1}%
                    \hbox{\strut}}}]}%
\def\@opargbegintheorem##1##2##3{%
  \item[\rlap{\vbox{\theorem@headerfont
                    \hbox{\llap{##2}\hskip\labelsep ##1\ (##3)}%
                    \hbox{\strut}}}]}}
\endgroup
%</thmb>
%    \end{macrocode}
% \changes{v2.1c}{90/02/17}{`endgroup outside macrocode env.}
% \end{macro}
%
%
%
% \subsubsection{The \texttt{margin} style}
%
% This style option is stored in the file |thm.sty|.
%    \begin{macrocode}
%<*thm>
\begingroup \makeatletter
\@ifundefined{theorem@style}{% \iffalse meta-comment
%
% Copyright (C) 1993-2024
%
% The LaTeX Project and any individual authors listed elsewhere
% in this file.
%
% This file is part of the Standard LaTeX `Tools Bundle'.
% -------------------------------------------------------
%
% It may be distributed and/or modified under the
% conditions of the LaTeX Project Public License, either version 1.3c
% of this license or (at your option) any later version.
% The latest version of this license is in
%    https://www.latex-project.org/lppl.txt
% and version 1.3c or later is part of all distributions of LaTeX
% version 2005/12/01 or later.
%
% The list of all files belonging to the LaTeX `Tools Bundle' is
% given in the file `manifest.txt'.
%
% \fi
%
%%
%
%
\def\FMithmInfo{2023/07/05 v2.2c Theorem extension package (FMi)}
%
% \ProvidesFile{theorem.dtx}[\FMithmInfo]
%\iffalse   % this is a METACOMMENT !
%%
%% Package `theorem' to use with LaTeX2e
%% Copyright (C) 1989-1995 Frank Mittelbach, all rights reserved.
%<+package>\@ifundefined{theorem@style}{}{\endinput}
%<+package>\NeedsTeXFormat{LaTeX2e}[1995/06/01]
%<+package>\ProvidesPackage{theorem}
%<+package>         [\FMithmInfo]
%
% \fi
%
% \changes{v1.0}{87/05/07}{First draft.}
% \changes{v1.0a}{87/05/07}{`thistheoremstyle' added.}
% \changes{v1.1a}{87/07/29}{New concept: redefine the internal
%                          \protect\LaTeX{} commands `@xnthm' etc.}
% \changes{v1.1b}{87/07/31}{`theoremfont' used to specify font.}
% \changes{v1.2a}{88/01/08}{`@endtheorem' now globally defined.}
% \changes{v1.2b}{88/01/11}{`newtheorem' only allowed in the preamble.}
% \changes{v1.3a}{88/02/14}{`theoremfont' now a toks register.}
%
% \changes{v2.0a}{88/11/16}{New implementation and new documentation.}
% \changes{v2.0a}{88/11/16}{No layout changes but internal modifications
%                           old substyles won't work.}
% \changes{v2.0b}{89/01/30}{`global' used for all definitions.}
% \changes{v2.0c}{89/06/15}{Documentation translated to English.
%                            Thanks to LLL and Nina}
% \changes{v2.1a}{90/02/15}{Support for new font selection.}
% \changes{v2.2a}{94/02/03}{Cleanup for LaTeX2e}
%
% \DoNotIndex{\@endparenv,\@ifdefinable,\@ifnextchar,\@ifundefined}
% \DoNotIndex{\@namedef,\@nameuse,\@spaces,\@tempa,\@warning}
% \DoNotIndex{\ }
% \DoNotIndex{\begingroup,\bfseries}
% \DoNotIndex{\csname}
% \DoNotIndex{\def,\do,\docdate}
% \DoNotIndex{\endcsname,\endgroup,\endinput,\endtrivlist,\expandafter}
% \DoNotIndex{\filedate,\fileversion}
% \DoNotIndex{\gdef,\global}
% \DoNotIndex{\hbox,\hskip}
% \DoNotIndex{\ignorespaces,\input,\itshape,\item}
% \DoNotIndex{\labelsep,\let,\llap}
% \DoNotIndex{\makeatletter}
% \DoNotIndex{\noexpand}
% \DoNotIndex{\refstepcounter,\relax,\rlap}
% \DoNotIndex{\setlength,\slshape,\space,\string,\strut}
% \DoNotIndex{\the,\trivlist,\typeout}
% \DoNotIndex{\vbox}
% \DoNotIndex{\xdef}
%
% \def\deflabel#1{\ttfamily #1\hfill}
% \def\deflist#1{\begin{list}{}{\settowidth\labelwidth{\ttfamily #1}%
%                               \setlength\leftmargin\labelwidth
%                               \addtolength\leftmargin\labelsep
%                               \let\makelabel\deflabel}}
% \def\enddeflist{\end{list}}
% \def\envfont{\normalfont\ttfamily}
%
% \MakeShortVerb{\|}
%
% \newcommand{\TUB}{{\sl TUGboat\/}}
%
% \GetFileInfo{theorem.dtx}
% \title{An Extension of the
%        \LaTeX{} theorem environment\thanks{This file
%        has version number \fileversion, last
%        revised \filedate.}}
% \author{Frank Mittelbach}
%
% \MaintainedByLaTeXTeam{tools}
% \maketitle
%
% \begin{abstract}
%    The macros described in this paper yield an extension of the
%    \LaTeX{} theorem mechanism. It is designed to satisfy the
%    different requirements of various journals. Thus, the
%    layout of the ``theorems'' can be manipulated by determining a
%    ``style''. This article describes not only the use, but
%    also the definition, of the necessary macros.
% \end{abstract}
%
% \section*{Preface to version 2.2}
%
% For \LaTeXe{} this package did not need any fundamental changes. I
% only modified the messages generated so that theorem layout styles
% will show up with the |\listfiles| command and cleaned the section
% on the New Font Selection Scheme since this is now included in
% \LaTeX.
%
% \section*{Preface to version 2.1}
%
% This version is identical to 2.0g described in \TUB\ 10\#3 except
% for some internal defaults which are now set depending on the used
% font selection scheme.
%
% This was done to avoid unpleasant surprises if the new font selection
% scheme is in force. For further details see
% section~\ref{sec:newfonts} and \cite{bk:GMS94}.
%
% \section{Introduction}
%
% For our purposes here, ``theorems'' are labelled enunciations,
% often set off from the main text by extra space and a font change.
% Theorems, corollaries, conjectures, definitions, and remarks
% are all instances of ``theorems''.  The ``header'' of these structures
% is composed of a label (such as \textsc{Theorem} or \textsc{Remark})
% and a number which serializes an item in the sequence of items
% with the same label.
%
% Shortly after the introduction of \LaTeX{} at the Fachbereich
% Mathematik in Mainz, the desire to manipulate the layout of
% ``theorems'' arose.  In Mainz, the following two conventions came
% into general use:
% \begin{enumerate}
%   \item The number of the theorem is shown in the margin.
%   \item There is a line break at
%         the end of the theorem header.
% \end{enumerate}
%
% Additionally, some journals require different formats which depend on
% the ``sort of theorem'': e.g.\ often remarks and definitions are set
% in |\upshape|, while |\itshape| is employed for main theorems.
%
% Confronted with these requirements, a theorem environment was
% developed in Mainz which allows separate determination of the
% layout of the ``theorems sets'', comparable to |\pagestyle|.
%
%
% \section{The user interface}
% \subsection{Defining new theorem sets}
%
% \DescribeMacro\newtheorem
% As in the original \LaTeX{} version, the command |\newtheorem|
% defines a new ``theorem set'' or ``theorem-like structure''.
% Two required arguments name the new environment and give the text
% to be typeset with each instance of the new ``set'', while
% an optional argument determines how
% the ``set'' is enumerated:
% \begin{description}
%    \item[\ttfamily \bslash newtheorem\{foo\}\{bar\}]
%       The theorem set {\envfont foo} (whose name is \texttt{bar})
%       uses its own counter.
%    \item[\ttfamily \bslash newtheorem\{foo2\}{[foo]}\{bar2\}]
%       The theorem set {\envfont foo2} (printed name \texttt{bar2})
%       uses the same counter as the theorem set \texttt{foo}.
%    \item[\ttfamily \bslash newtheorem\{foo3\}\{bar3\}{[section]}]
%       The theorem set {\envfont foo3} (printed name \texttt{bar3}) is
%       enumerated within the counter \texttt{section}, i.e.\ with every
%       new |\section| the enumeration begins again with 1, and
%       the enumeration is composed from the section-number and the
%       theorem counter itself.
% \end{description}
%
% \DescribeMacro\theoremstyle
% Additionally, the command |\theoremstyle| can define the layout
% of various, or all, theorem sets.  It should be noted that any
% theorem set defined by |\newtheorem| is typeset in the
% |\theoremstyle| that is current at the time of the definition.
% Thus, the following
% \begin{verbatim}
%    \theoremstyle{break}        \newtheorem{Cor}{Corollary}
%    \theoremstyle{plain}        \newtheorem{Exa}{Example}[section]
%\end{verbatim}                             ^^A No blank at line start
% leads to the result that the set {\envfont Cor} is formatted in the
% style \texttt{break}, while the set {\envfont Exa} and all the
% following ones are formatted in the style \texttt{plain}, unless
% another |\theoremstyle| follows.  Since the definitions installed by
% |\newtheorem| are global, one also can limit |\theoremstyle| locally
% by grouping braces.
%
% \DescribeMacro\theorembodyfont
% The choice of the font for the theorem body is completely
% independent of the chosen |\theoremstyle|; this has proven to
% be very advantageous.  For example,
% \begin{verbatim}
%   {\theorembodyfont{\upshape}          \newtheorem{Rem}{Remark}}
%\end{verbatim}
% defines a theorem set {\envfont Rem}, which will be set in
% |\upshape| in the current layout (which in our example is
% \texttt{plain}).  As with |\theoremstyle|, the |\theorembodyfont|
% chosen is that current at the time of |\newtheorem|.  If
% |\theorembodyfont| is not specified or one defines
% |\theorembodyfont{}|, then the font used will be that defined by the
% |\theoremstyle|.
%
% \DescribeMacro\theoremheaderfont It is also possible to customize
% the font used for the theorem headers.  This is, however, a global
% declaration, and therefore there should be at most one
% |\theoremheaderfont| declaration in the preamble.\footnote{If it is
% actually necessary to have different header fonts, one has to define
% new theorem styles (substituting the desired font) or specify the
% information directly in the \texttt{\bslash newtheorem} declaration
% (the unclean variant).}
%
% \DescribeMacro\theorempreskipamount
% \DescribeMacro\theorempostskipamount Two additional parameters
% affect the vertical space around the theorem environments:
% |\theorempreskipamount| and |\theorempostskipamount| define,
% respectively, the spacing before and after such an environment.
% These parameters apply for all theorem sets and can be manipulated
% with the ordinary length macros.  They are rubber lengths
% (`\textsf{skips}'), and therefore can contain \texttt{plus} and
% \texttt{minus} parts.
%
% Since the definition of theorem sets should---most sensibly---be
% placed in the preamble, we only allow installation there.
% It is therefore possible to release the memory used here after
% |\begin{document}|, in order to make room for other
% applications.
%
%
% \subsection{Existing theorem styles}
%
% The following theorem styles exist to date:
% \begin{deflist}{marginbreak}
%    \item[plain]
%       This theorem style emulates the original \LaTeX{} definition,
%       except that additionally the parameters
%       |\theorem...skipamount| are used.
%    \item[break]
%       In this style, the theorem header is followed by a line
%       break.
%    \item[marginbreak]
%       The theorem number is set in the margin, and there is a line
%       break as in \texttt{break}.
%    \item[changebreak]
%       Like \texttt{break}, but with header number and text
%       interchanged.
%    \item[change]
%       Header number and text are interchanged, without a line break.
%    \item[margin]
%       The number is set in the left margin, without a line break.
% \end{deflist}
% All styles (except \texttt{plain}) select |\slshape| as the default
% |\theorembodyfont|.
%
%
% \subsection{Examples}
%
% Given the above theorem sets {\envfont Cor}, {\envfont Exa}
% and {\envfont Rem},
% suppose that the preamble also contains the declarations:
% \begin{verbatim}
%    \theoremstyle{marginbreak}   \newtheorem{Lem}[Cor]{Lemma}
%    \theoremstyle{change}
%    \theorembodyfont{\itshape}        \newtheorem{Def}[Cor]{Definition}
%
%    \theoremheaderfont{\scshape}
%\end{verbatim}
% Then the following are some typical examples of the typeset output
% resulting from their use.
%
% \begin{Cor}
%    This is a sentence typeset in the theorem environment {\envfont
%    Cor}.
% \end{Cor}
%
% \begin{Exa}
%    This is a sentence typeset in the theorem environment {\envfont
%    Exa}.
% \end{Exa}
%
% \begin{Rem}
%    This is a sentence typeset in the theorem environment {\envfont
%    Rem}.
% \end{Rem}
%
% \begin{Lem}[Ben User]
%    This is a sentence typeset in the theorem environment {\envfont
%    Lem}.
% \end{Lem}
%
% \begin{Def}[Very impressive Definition]
%    This is a sentence typeset in the theorem environment {\envfont
%    Def}.
% \end{Def}
%
% The last two examples show the effect of the optional argument to
% a theorem environment (it is the text typeset in parentheses).
%
%
% \section{Special Considerations}
% \label{sec:newfonts}
%
% Theoremheader and body are implemented as a unit. This means that
% the |\theoremheaderfont|  will inherit characteristics of the
% |\theorembodyfont| in \LaTeXe. Thus, if for example
% |\theorembodyfont| is |\itshape| and |\theoremheaderfont|
% is |\bfseries| the font selected for the header will have the
% characteristics `bold extended italic'. If this is not desired one
% should set the |\theoremheaderfont| to something like
% \begin{verbatim}
%      \theoremheaderfont{\normalfont\bfseries}
%\end{verbatim}
% i.e.\ supplying all necessary font information explicitly.
%
%
%
% \section{Acknowledgements}
%
% The publication of this set of macros was only possible with the
% help of Christina Busse (translating the manuscript into English),
% Joachim Pense (playing the r\^ole of typist), Chris Rowley
% (looking everything over) and many others providing useful
% suggestions.
%
% \MaybeStop{%
%    \begin{thebibliography}{9}
%       \bibitem{bk:GMS94} \textsc{M.~Goossens}, \textsc{F.~Mittelbach}
%         and \textsc{A.~Samarin}.
%         \newblock The \LaTeX{} Companion.
%         \newblock
%         Addison-Wesley, Reading, Massachusetts, 1994.
%       \bibitem{latex:source} \textsc{Lamport, Leslie.} \newblock
%                               \texttt{latex.tex}, version 2.09, date
%                               Feb. 1990.
%    \end{thebibliography}
%    \PrintIndex
%    \PrintChanges}
%
%
% \section{The documentation driver file}
%
% The next bit of code contains the documentation driver file for
% \TeX{}, i.e., the file that will produce the documentation you are
% currently reading. It will be extracted from this file by the
% \texttt{docstrip} program. Since it is the first code in the file
% one can alternatively process this file directly with \LaTeXe{} to
% obtain the documentation.
%
%    \begin{macrocode}
%<*driver>
\documentclass{ltxdoc}

\usepackage{theorem}

%<+driver>% The next few lines define theorem sets which are used
%<+driver>% in the example section of the documentation.

\theoremstyle{break}       \newtheorem{Cor}{Corollary}
\theoremstyle{plain}       \newtheorem{Exa}{Example}[section]
{\theorembodyfont{\upshape}\newtheorem{Rem}{Remark}}
\theoremstyle{marginbreak} \newtheorem{Lem}[Cor]{Lemma}
\theoremstyle{change}
\theorembodyfont{\itshape} \newtheorem{Def}[Cor]{Definition}

\theoremheaderfont{\scshape}

\RecordChanges

\begin{document}
   \DocInput{theorem.dtx}
\end{document}
%</driver>
%    \end{macrocode}
%
% \section{Definition of the Macros}
%
% If the file has been loaded before, we abort immediately. If not the
% package announces itself (this is actually done at the very top of
% the file---the way it is done isn't good style so don't copy it).
% \changes{v2.0e}{89/07/19}{Spaces removed from `typeout'.}
% \changes{v2.2b}{95/11/19}{Announce moved to top of file}
%    \begin{macrocode}
%<*package>
%\@ifundefined{theorem@style}{}{\endinput}
%\def\FMithmInfo{1995/11/19 v2.2b Theorem extension package (FMi)}
%\ProvidesPackage{theorem}[\FMithmInfo]
%    \end{macrocode}
%
%
% \subsection{Definition of theorem styles and fonts}
%
% All the definitions in this file are done globally to allow
% inputting this file inside a group.
%
% \begin{macro}{\theoremstyle}
%    Before a theorem style can be installed, the chosen style must be
%    known. For that reason, we must test to see that
%    |\th@|\meta{style} is known or, more precisely, that it is
%    different from |\relax|. If the style is not known then
%    |\th@plain| is used.
%    \begin{macrocode}
\gdef\theoremstyle#1{%
   \@ifundefined{th@#1}{\@warning
          {Unknown theoremstyle `#1'. Using `plain'}%
          \theorem@style{plain}}%
%    \end{macrocode}
%    We save the theorem style to be used in the token register
%    |\theorem@style|.
%    \begin{macrocode}
      {\theorem@style{#1}}%
%    \end{macrocode}
%    Now we ``evaluate'' the theorem style: this means, we call the
%    macro |\th@|\meta{style} which will activate the relevant
%    definitions which are contained in a separate file.  This is done
%    in a group to suppress changes to the current font.  This could
%    otherwise pose problems together with the new font selection
%    scheme\footnote{When I printed the original article using the new
%    font selection scheme I ended with a document with slanted
%    typefaces (text headings and all) simply because one of the
%    theorem styles used \texttt{\bslash sl} at toplevel.} if the
%    |\th@|\meta{style} is evaluated a second time.
% \changes{v2.1a}{90/02/15}{Group added around `th@..}
% \changes{v2.1a}{90/02/15}{Reseting of `..@begintheorem removed.}
%    \begin{macrocode}
      \begingroup
        \csname th@\the\theorem@style \endcsname
      \endgroup}
%    \end{macrocode}
% \end{macro}
%
%
% \begin{macro}{\@begintheorem}
% \changes{v2.1a}{90/02/15}{Relaxed at toplevel.}
% \begin{macro}{\@opargbegintheorem}
% \changes{v2.1a}{90/02/15}{Relaxed at toplevel.}
%    We reset |\@begintheorem| and |\@opargbegintheorem| to
%    |\relax| since these commands are no longer necessary at
%    toplevel. This will save a few tokens.
%    \begin{macrocode}
\global\let\@begintheorem\relax
\global\let\@opargbegintheorem\relax
%    \end{macrocode}
% \end{macro}
% \end{macro}
%
% \begin{macro}{\theorem@style}
%    Obviously the \textsf{token} register used above has to be
%    allocated. To assure the utmost compatibility with the
%    original \LaTeX{} definition, we set the default theorem style to
%    \texttt{plain}, which implements the usual \LaTeX{} convention.
% \changes{v2.0f}{89/07/20}{Unnecessary `global' removed.}
%    \begin{macrocode}
\newtoks\theorem@style
\global\theorem@style{plain}
%    \end{macrocode}
% \end{macro}
%
% \begin{macro}{\theorembodyfont}
% \begin{macro}{\theorem@bodyfont}
%    If the |\theorembodyfont| is set by the user then it should not
%    interact with the default font set in the theorem style. When the
%    new font selection is in force this may happen if, for example,
%    the default is |\itshape| and the new |\theorembodyfont| is
%    |\sffamily|. So we add a |\reset@font| command in front of the
%    user definition.
% \changes{v2.0f}{89/07/20}{Unnecessary `global' removed.}
% \changes{v2.0g}{89/09/19}{`theoremfont to `theorembodyfont.}
% \changes{v2.1c}{92/06/22}{`reset@font added and logic changed.}
% \changes{v2.1d}{92/10/28}{Reset to empty if arg is empty}
%    \begin{macrocode}
\gdef\theorembodyfont#1{%
%    \end{macrocode}
%    We check if the argument supplied is empty and if so put nothing
%    into the |\theorem@bodyfont| token register to allow for
%    |\theorembodyfont{}| as a mean of using the default of the
%    current |\theoremstyle|.
%    \begin{macrocode}
   \def\@tempa{#1}%
   \ifx\@tempa\@empty
    \theorem@bodyfont{}%
   \else
    \theorem@bodyfont{\reset@font#1}%
   \fi
   }
\newtoks\theorem@bodyfont
\global\theorem@bodyfont{}
%    \end{macrocode}
% \end{macro}
% \end{macro}
%
%
% \begin{macro}{\theoremheaderfont}
% \changes{v2.0g}{89/09/19}{Macro added.}
%    The font for the theorem headers is handled differently because
%    this definition applies to all theorem styles.
%    \begin{macrocode}
\gdef\theoremheaderfont#1{\gdef\theorem@headerfont{#1}%
%    \end{macrocode}
%    After using the macro once it is redefined to produce an error
%    message.
%    \begin{macrocode}
       \gdef\theoremheaderfont##1{%
        \typeout{\string\theoremheaderfont\space should be used
                 only once.}}}
%    \end{macrocode}
% \end{macro}
%
% \begin{macro}{\theorem@headerfont}
% \changes{v2.0g}{89/09/19}{Macro added.}
%    To set the |\theorem@headerfont| default we first test if
%    the new fontselection scheme is in force.
% \changes{v2.1a}{90/02/15}{Support for new font selection.}
%    \begin{macrocode}
\ifx\upshape\undefined
%    \end{macrocode}
%    If not we define it to expand into |\bfseries|. We don't use
%    |\let| just in case a following style option redefines this
%    macro.
%    \begin{macrocode}
\gdef\theorem@headerfont{\bfseries}
%    \end{macrocode}
%    Otherwise we reset the current shape before calling |\bfseries|.
%    \begin{macrocode}
\else \gdef\theorem@headerfont{\normalfont\bfseries}\fi
%    \end{macrocode}
% \end{macro}
%
% \begin{macro}{\th@plain}
% \begin{macro}{\th@break}
% \begin{macro}{\th@marginbreak}
% \begin{macro}{\th@changebreak}
% \begin{macro}{\th@change}
% \begin{macro}{\th@margin}
%    The different styles are defined in macros such as
%    |\th@plain|.  Since memory space is precious in
%    ``non-Big-versions'', we have to avoid offering too many unused
%    definitions.  Therefore we define these styles in separate files
%    that can be loaded on demand.  Thus the commands themselves only
%    load these files. We use |\@input@|, a \LaTeXe{} internal command
%    that ensures that the file will be listed with |\listfiles|.
% \changes{v2.2a}{94/02/02}{Use `@input@ to load theorem layout files}
%    \begin{macrocode}
\gdef\th@plain{\@input@{thp.sty}}
\gdef\th@break{\@input@{thb.sty}}
\gdef\th@marginbreak{\@input@{thmb.sty}}
\gdef\th@changebreak{\@input@{thcb.sty}}
\gdef\th@change{\@input@{thc.sty}}
\gdef\th@margin{\@input@{thm.sty}}
%    \end{macrocode}
%    This list will be expanded when new styles become
%    available.  For testing,  just append new theorem substyles
%    as document options.
% \end{macro}
% \end{macro}
% \end{macro}
% \end{macro}
% \end{macro}
% \end{macro}
%
%
%
% \subsection{Definition of a new theorem set}
%
% As already pointed out, a new theorem environment can be defined in
% three different ways:
% \begin{verbatim}
%    \newtheorem{Lem}{Lemma}
%    \newtheorem{Lem}{Lemma}[section]
%    \newtheorem{Lem}[Theorem]{Lemma}
%\end{verbatim}
% The function of the macro |\newtheorem| is to recognize these cases
% and then to branch into one of the three macros |\@ynthm|, |\@xnthm|
% or |\@othm|.  This mechanism is adopted unchanged from
% \cite{latex:source}; the essential point here is that, for example,
% in the second case, the arguments \texttt{Lem}, \texttt{Lemma} and
% \texttt{section} are passed over to the macro |\@xnthm|.
%
% We inspect this case first because the others present fewer problems,
% and thus are easily derived from this one.
%
% \begin{macro}{\@xnthm}
%    For our example arguments, the macro |\@xnthm| must fulfill
%    the following:
%    \begin{itemize}
%       \item
%          Define a new \LaTeX-counter `Lem'
%       \item
%          reset this counter within a |\section|
%       \item
%          define the macro |\theLem|
%       \item
%          define the environment macros |\Lem| and
%          |\endLem| using the current
%          |\theoremstyle| and |\theorem@bodyfont|.
%    \end{itemize}
%    Obviously, all this should happen only if the first argument
%    of |\@xnthm| (i.e.\ \texttt{Lem} in our example) is chosen so
%    as not to conflict with any previously defined commands or
%    environments.  This test is performed by the \LaTeX{} macro
%    |\@ifdefinable|.
%    \begin{macrocode}
\gdef\@xnthm#1#2[#3]{\expandafter\@ifdefinable\csname #1\endcsname
%    \end{macrocode}
%    Therefore, the first argument of |\@ifdefinable| is the
%    expansion (in the example, |\Lem|) of
%    |\csname||#1\endcsname|.  The second argument is
%    executed only if the test has been completed successfully.
%    \begin{macrocode}
   {%
%    \end{macrocode}
%    Now we define the new counter. The names of the \LaTeX{}
%    macros employed should speak for themselves:
% \changes{v2.2b}{1995/10/07}{Use \cs{@newctr}}
% \changes{v2.2c}{1995/11/23}{Use \cs{@newctr} properly.}
%    \begin{macrocode}
    \@definecounter{#1}\@newctr{#1}[#3]%
%    \end{macrocode}
%    Using |\@newctr| will give a proper error message if the counter
%    in |#3| is not defined.
%    In defining `|\theLem|' we must generate the desired macro
%    name by use of |\expandafter| and |\csname|.
%    \begin{macrocode}
    \expandafter\xdef\csname the#1\endcsname
%    \end{macrocode}
%    An |\xdef| is used in order to make the definition global,
%    and to ensure that it contains the replacement texts of
%    |\@thmcountersep| and |\@thmcounter|.\footnote{These
%    two macros can be defined by the document style.  Their default
%    values produce a `\texttt{.}' as separation and an arabic
%    representation of the number.} However, not everything should be
%    expanded. For example, it saves space to use |\thesection|
%    instead of its---at times---lengthy expansion.
%    \begin{macrocode}
      {\expandafter \noexpand \csname the#3\endcsname
       \@thmcountersep \@thmcounter{#1}}%
%    \end{macrocode}
%    Thus with the defaults of \LaTeX{}, |\theLem| would be
%    replaced by the command sequence |\thesection.\arabic{Lem}|.
%
%    We will now look at the definition of the macro which is executed
%    at the beginning of the actual environment (in our example this
%    macro is |\Lem|). It should be noted that we use an
%    ``|\expandafter| trick'' to expand only certain parts of the
%    replacement text at the time of the definition.
%    \begin{macrocode}
    \def\@tempa{\global\@namedef{#1}}%
    \expandafter \@tempa \expandafter{%
%    \end{macrocode}
%    First, the macro that contains the current definitions of
%    |\@begintheorem| and |\@opargtheorem| should be called
%    up.  The name of this macro---as is already known---has the form
%    |\th@|\meta{theorem\ style}; therefore, it must be called by
%    \begin{macrocode}
      \csname th@\the \theorem@style
%    \end{macrocode}
%    In addition the default theorem font should be changeable, i.e.\
%    we have to insert the contents of |\theorem@bodyfont|. For
%    that reason, we expand even further, beyond |\endcsname|,
%    and thus insert the contents of the \textsf{token} register
%    |\theorem@bodyfont| in the replacement text.
%    \begin{macrocode}
            \expandafter \endcsname \the \theorem@bodyfont
%    \end{macrocode}
%    Now it is time to call the macro |\@thm| which takes over
%    the further processing.  It has two arguments: the current
%    counter name (in our example, \texttt{Lem}), and the text of the
%    label (in our example, \texttt{Lemma}).
%    \begin{macrocode}
     \@thm{#1}{#2}}%
%    \end{macrocode}
%    With this, the `sub-definition' is complete.  The macro
%    |\@endtheorem| ends a theorem environment and is, so far, nothing
%    but an |\endtrivlist|.  (Hence it is defined globally, and not
%    within the theorem styles.\footnote{This has to be changed as
%    soon as theorem styles that change \texttt{\bslash @endtheorem}
%    exist.  In such a case, all existing styles must be changed as
%    well since they will have to reset the macro.}) Therefore, we can
%    set it equivalent to the macro that ends the theorem set (in our
%    example, |\endLem|).  However, if some day theorem styles exist
%    that do change |\@endtheorem|, we would have to use the
%    commented-out line instead.
%    \begin{macrocode}
    \global \expandafter \let \csname end#1\endcsname \@endtheorem
%   \global\@namedef{end#1}{\@endtheorem}%
%    \end{macrocode}
%    With these commands all the required definitions are employed,
%    unless the test |\@ifdefinable| has failed.  Therefore, we
%    end the second argument of this macro and with it the definition
%    of |\@xnthm|.
%    \begin{macrocode}
   }}
%    \end{macrocode}
% \end{macro}
%
% \begin{macro}{\@ynthm}
%    The definition of |\@ynthm| is completely analogous. In
%    this case the new counter that is defined is not reset within
%    another counter; thus the definition of |\the...|
%    is simplified:
%    \begin{macrocode}
\gdef\@ynthm#1#2{\expandafter\@ifdefinable\csname #1\endcsname
   {\@definecounter{#1}%
    \expandafter\xdef\csname the#1\endcsname{\@thmcounter{#1}}%
%    \end{macrocode}
%    The rest of the definition corresponds literally to that of
%    |\@xnthm|:
%    \begin{macrocode}
    \def\@tempa{\global\@namedef{#1}}\expandafter \@tempa
     \expandafter{\csname th@\the \theorem@style \expandafter
     \endcsname \the\theorem@bodyfont \@thm{#1}{#2}}%
    \global \expandafter \let \csname end#1\endcsname \@endtheorem}}
%    \end{macrocode}
% \end{macro}
%
% \begin{macro}{\@othm}
%    The definition of |\@othm| does not contain anything new.
%    \begin{macrocode}
\gdef\@othm#1[#2]#3{%
%    \end{macrocode}
% \changes{v2.2b}{95/10/07}{Build in check for counter defined}
%    We do not define a new counter but instead use one that has
%    already been defined.  Thus the only definition we need is that
%    of this pseudo-counter (i.e.\ |\the|\meta{env.\ name}).
%    First we check if |#2| corresponds to a known counter name.
%    \begin{macrocode}
  \expandafter\ifx\csname c@#2\endcsname\relax
   \@nocounterr{#2}%
  \else
   \expandafter\@ifdefinable\csname #1\endcsname
   {\expandafter \xdef \csname the#1\endcsname
     {\expandafter \noexpand \csname the#2\endcsname}%
%    \end{macrocode}
%    All other parts of the definition can be adopted from
%    |\@xnthm|. We have to remember, though, that in this case
%    the name of the current counter and the theorem label have moved
%    to the second and third arguments.
%    \begin{macrocode}
    \def\@tempa{\global\@namedef{#1}}\expandafter \@tempa
     \expandafter{\csname th@\the \theorem@style \expandafter
     \endcsname \the\theorem@bodyfont \@thm{#2}{#3}}%
    \global \expandafter \let \csname end#1\endcsname \@endtheorem}%
  \fi}
%    \end{macrocode}
% \end{macro}
%
% \subsection{Macros that are employed in a {\envfont theorem}
%    environment}
%
% \begin{macro}{\@thm}
%    The macro |\@thm| has to increase the current counter.  Then,
%    depending on whether the environment has (or does not have) an
%    optional argument, it has to branch into either |\@begintheorem|
%    or |\@opargtheorem|.
%    \begin{macrocode}
\gdef\@thm#1#2{\refstepcounter{#1}%
%    \end{macrocode}
%    Now we start a {\envfont trivlist} environment, and give
%    |\@topsep| and |\@topsepadd| the values of the \textsf{skip}
%    registers |\theorempreskipamount| and |\theorempostskipamount|.
%    The value in |\@topsep| is the vertical space that is inserted by
%    the first (and only) |\item| in our |\trivlist| whilst
%    |\@topsepadd| is inserted by |\@endparenv| at the end of that
%    {\envfont trivlist} environment.  By using these registers, we
%    obtain the desired space around a {\envfont theorem} environment.
%    \begin{macrocode}
   \trivlist
   \@topsep \theorempreskipamount               % used by first \item
   \@topsepadd \theorempostskipamount           % used by \@endparenv
%    \end{macrocode}
%    Now we have to test whether an optional argument has been given.
% \changes{v2.0d}{89/07/18}{Forgotten percent added.}
%    \begin{macrocode}
   \@ifnextchar [%
%    \end{macrocode}
%    If there is an optional argument, we will call |\@ythm|, and
%    move the arguments read back into the input stream.
%    \begin{macrocode}
   {\@ythm{#1}{#2}}%
%    \end{macrocode}
%    If not, we call |\@begintheorem|. Its first argument is the
%    name of the theorem set (hence the second argument of
%    |\@thm|). Its second argument is the macro that produces the
%    current number.
%    \begin{macrocode}
   {\@begintheorem{#2}{\csname the#1\endcsname}\ignorespaces}}
%    \end{macrocode}
% \end{macro}
%
% \begin{macro}{\@xthm}
% \begin{macro}{\@ythm}
%    Both these macros were originally called by |\@thm|.  We do
%    not need |\@xthm| anymore, hence we reset it to
%    |\relax|. The definition of |\@ythm| has not changed at
%    all from its definition in \LaTeX. In order to make the macros
%    easier to understand, we will
%    nevertheless present it (commented out).
%    \begin{macrocode}
\global\let\@xthm\relax
% \def\@ythm#1#2[#3]{\@opargbegintheorem{#2}{\csname
%           the#1\endcsname}{#3}\ignorespaces}
%</package>
%    \end{macrocode}
%    The primitive |\ignorespaces| in |\@ythm| and
%    |\@thm| is needed to remove the spaces between the
%    |\begin{...}| and the actual text.
% \end{macro}
% \end{macro}
%
%
% \subsection{Definition of the theorem substyles}
%
% As already pointed out, the theorem substyles, defined below, are
% only loaded when necessary.  Note that all these substyles, except
% \texttt{plain}, have |\slshape| as the default body font.
%
% \subsubsection{The \texttt{plain} style}
%
% \label{dok:th@plain} As the following macros use |@|, we have
% to locally set the |\catcode| of this symbol to ``letter''.
% This happens within a group, so that we do not have to worry about
% which |\catcode| that symbol had before.
%    \begin{macrocode}
%<*thp>
\begingroup \makeatletter
%    \end{macrocode}
% Since we are now within a group, we must make all definitions
% globally.  First we make sure that \texttt{theorem.sty} is loaded.
% This will allow us to use this file as a document style option
% without having to call \texttt{theorem} itself as an option.  At the
% same time, we assure that at least version 2 is loaded, since
% |\theorem@style| was not defined in earlier versions.
%    \begin{macrocode}
\@ifundefined{theorem@style}{% \iffalse meta-comment
%
% Copyright (C) 1993-2024
%
% The LaTeX Project and any individual authors listed elsewhere
% in this file.
%
% This file is part of the Standard LaTeX `Tools Bundle'.
% -------------------------------------------------------
%
% It may be distributed and/or modified under the
% conditions of the LaTeX Project Public License, either version 1.3c
% of this license or (at your option) any later version.
% The latest version of this license is in
%    https://www.latex-project.org/lppl.txt
% and version 1.3c or later is part of all distributions of LaTeX
% version 2005/12/01 or later.
%
% The list of all files belonging to the LaTeX `Tools Bundle' is
% given in the file `manifest.txt'.
%
% \fi
%
%%
%
%
\def\FMithmInfo{2023/07/05 v2.2c Theorem extension package (FMi)}
%
% \ProvidesFile{theorem.dtx}[\FMithmInfo]
%\iffalse   % this is a METACOMMENT !
%%
%% Package `theorem' to use with LaTeX2e
%% Copyright (C) 1989-1995 Frank Mittelbach, all rights reserved.
%<+package>\@ifundefined{theorem@style}{}{\endinput}
%<+package>\NeedsTeXFormat{LaTeX2e}[1995/06/01]
%<+package>\ProvidesPackage{theorem}
%<+package>         [\FMithmInfo]
%
% \fi
%
% \changes{v1.0}{87/05/07}{First draft.}
% \changes{v1.0a}{87/05/07}{`thistheoremstyle' added.}
% \changes{v1.1a}{87/07/29}{New concept: redefine the internal
%                          \protect\LaTeX{} commands `@xnthm' etc.}
% \changes{v1.1b}{87/07/31}{`theoremfont' used to specify font.}
% \changes{v1.2a}{88/01/08}{`@endtheorem' now globally defined.}
% \changes{v1.2b}{88/01/11}{`newtheorem' only allowed in the preamble.}
% \changes{v1.3a}{88/02/14}{`theoremfont' now a toks register.}
%
% \changes{v2.0a}{88/11/16}{New implementation and new documentation.}
% \changes{v2.0a}{88/11/16}{No layout changes but internal modifications
%                           old substyles won't work.}
% \changes{v2.0b}{89/01/30}{`global' used for all definitions.}
% \changes{v2.0c}{89/06/15}{Documentation translated to English.
%                            Thanks to LLL and Nina}
% \changes{v2.1a}{90/02/15}{Support for new font selection.}
% \changes{v2.2a}{94/02/03}{Cleanup for LaTeX2e}
%
% \DoNotIndex{\@endparenv,\@ifdefinable,\@ifnextchar,\@ifundefined}
% \DoNotIndex{\@namedef,\@nameuse,\@spaces,\@tempa,\@warning}
% \DoNotIndex{\ }
% \DoNotIndex{\begingroup,\bfseries}
% \DoNotIndex{\csname}
% \DoNotIndex{\def,\do,\docdate}
% \DoNotIndex{\endcsname,\endgroup,\endinput,\endtrivlist,\expandafter}
% \DoNotIndex{\filedate,\fileversion}
% \DoNotIndex{\gdef,\global}
% \DoNotIndex{\hbox,\hskip}
% \DoNotIndex{\ignorespaces,\input,\itshape,\item}
% \DoNotIndex{\labelsep,\let,\llap}
% \DoNotIndex{\makeatletter}
% \DoNotIndex{\noexpand}
% \DoNotIndex{\refstepcounter,\relax,\rlap}
% \DoNotIndex{\setlength,\slshape,\space,\string,\strut}
% \DoNotIndex{\the,\trivlist,\typeout}
% \DoNotIndex{\vbox}
% \DoNotIndex{\xdef}
%
% \def\deflabel#1{\ttfamily #1\hfill}
% \def\deflist#1{\begin{list}{}{\settowidth\labelwidth{\ttfamily #1}%
%                               \setlength\leftmargin\labelwidth
%                               \addtolength\leftmargin\labelsep
%                               \let\makelabel\deflabel}}
% \def\enddeflist{\end{list}}
% \def\envfont{\normalfont\ttfamily}
%
% \MakeShortVerb{\|}
%
% \newcommand{\TUB}{{\sl TUGboat\/}}
%
% \GetFileInfo{theorem.dtx}
% \title{An Extension of the
%        \LaTeX{} theorem environment\thanks{This file
%        has version number \fileversion, last
%        revised \filedate.}}
% \author{Frank Mittelbach}
%
% \MaintainedByLaTeXTeam{tools}
% \maketitle
%
% \begin{abstract}
%    The macros described in this paper yield an extension of the
%    \LaTeX{} theorem mechanism. It is designed to satisfy the
%    different requirements of various journals. Thus, the
%    layout of the ``theorems'' can be manipulated by determining a
%    ``style''. This article describes not only the use, but
%    also the definition, of the necessary macros.
% \end{abstract}
%
% \section*{Preface to version 2.2}
%
% For \LaTeXe{} this package did not need any fundamental changes. I
% only modified the messages generated so that theorem layout styles
% will show up with the |\listfiles| command and cleaned the section
% on the New Font Selection Scheme since this is now included in
% \LaTeX.
%
% \section*{Preface to version 2.1}
%
% This version is identical to 2.0g described in \TUB\ 10\#3 except
% for some internal defaults which are now set depending on the used
% font selection scheme.
%
% This was done to avoid unpleasant surprises if the new font selection
% scheme is in force. For further details see
% section~\ref{sec:newfonts} and \cite{bk:GMS94}.
%
% \section{Introduction}
%
% For our purposes here, ``theorems'' are labelled enunciations,
% often set off from the main text by extra space and a font change.
% Theorems, corollaries, conjectures, definitions, and remarks
% are all instances of ``theorems''.  The ``header'' of these structures
% is composed of a label (such as \textsc{Theorem} or \textsc{Remark})
% and a number which serializes an item in the sequence of items
% with the same label.
%
% Shortly after the introduction of \LaTeX{} at the Fachbereich
% Mathematik in Mainz, the desire to manipulate the layout of
% ``theorems'' arose.  In Mainz, the following two conventions came
% into general use:
% \begin{enumerate}
%   \item The number of the theorem is shown in the margin.
%   \item There is a line break at
%         the end of the theorem header.
% \end{enumerate}
%
% Additionally, some journals require different formats which depend on
% the ``sort of theorem'': e.g.\ often remarks and definitions are set
% in |\upshape|, while |\itshape| is employed for main theorems.
%
% Confronted with these requirements, a theorem environment was
% developed in Mainz which allows separate determination of the
% layout of the ``theorems sets'', comparable to |\pagestyle|.
%
%
% \section{The user interface}
% \subsection{Defining new theorem sets}
%
% \DescribeMacro\newtheorem
% As in the original \LaTeX{} version, the command |\newtheorem|
% defines a new ``theorem set'' or ``theorem-like structure''.
% Two required arguments name the new environment and give the text
% to be typeset with each instance of the new ``set'', while
% an optional argument determines how
% the ``set'' is enumerated:
% \begin{description}
%    \item[\ttfamily \bslash newtheorem\{foo\}\{bar\}]
%       The theorem set {\envfont foo} (whose name is \texttt{bar})
%       uses its own counter.
%    \item[\ttfamily \bslash newtheorem\{foo2\}{[foo]}\{bar2\}]
%       The theorem set {\envfont foo2} (printed name \texttt{bar2})
%       uses the same counter as the theorem set \texttt{foo}.
%    \item[\ttfamily \bslash newtheorem\{foo3\}\{bar3\}{[section]}]
%       The theorem set {\envfont foo3} (printed name \texttt{bar3}) is
%       enumerated within the counter \texttt{section}, i.e.\ with every
%       new |\section| the enumeration begins again with 1, and
%       the enumeration is composed from the section-number and the
%       theorem counter itself.
% \end{description}
%
% \DescribeMacro\theoremstyle
% Additionally, the command |\theoremstyle| can define the layout
% of various, or all, theorem sets.  It should be noted that any
% theorem set defined by |\newtheorem| is typeset in the
% |\theoremstyle| that is current at the time of the definition.
% Thus, the following
% \begin{verbatim}
%    \theoremstyle{break}        \newtheorem{Cor}{Corollary}
%    \theoremstyle{plain}        \newtheorem{Exa}{Example}[section]
%\end{verbatim}                             ^^A No blank at line start
% leads to the result that the set {\envfont Cor} is formatted in the
% style \texttt{break}, while the set {\envfont Exa} and all the
% following ones are formatted in the style \texttt{plain}, unless
% another |\theoremstyle| follows.  Since the definitions installed by
% |\newtheorem| are global, one also can limit |\theoremstyle| locally
% by grouping braces.
%
% \DescribeMacro\theorembodyfont
% The choice of the font for the theorem body is completely
% independent of the chosen |\theoremstyle|; this has proven to
% be very advantageous.  For example,
% \begin{verbatim}
%   {\theorembodyfont{\upshape}          \newtheorem{Rem}{Remark}}
%\end{verbatim}
% defines a theorem set {\envfont Rem}, which will be set in
% |\upshape| in the current layout (which in our example is
% \texttt{plain}).  As with |\theoremstyle|, the |\theorembodyfont|
% chosen is that current at the time of |\newtheorem|.  If
% |\theorembodyfont| is not specified or one defines
% |\theorembodyfont{}|, then the font used will be that defined by the
% |\theoremstyle|.
%
% \DescribeMacro\theoremheaderfont It is also possible to customize
% the font used for the theorem headers.  This is, however, a global
% declaration, and therefore there should be at most one
% |\theoremheaderfont| declaration in the preamble.\footnote{If it is
% actually necessary to have different header fonts, one has to define
% new theorem styles (substituting the desired font) or specify the
% information directly in the \texttt{\bslash newtheorem} declaration
% (the unclean variant).}
%
% \DescribeMacro\theorempreskipamount
% \DescribeMacro\theorempostskipamount Two additional parameters
% affect the vertical space around the theorem environments:
% |\theorempreskipamount| and |\theorempostskipamount| define,
% respectively, the spacing before and after such an environment.
% These parameters apply for all theorem sets and can be manipulated
% with the ordinary length macros.  They are rubber lengths
% (`\textsf{skips}'), and therefore can contain \texttt{plus} and
% \texttt{minus} parts.
%
% Since the definition of theorem sets should---most sensibly---be
% placed in the preamble, we only allow installation there.
% It is therefore possible to release the memory used here after
% |\begin{document}|, in order to make room for other
% applications.
%
%
% \subsection{Existing theorem styles}
%
% The following theorem styles exist to date:
% \begin{deflist}{marginbreak}
%    \item[plain]
%       This theorem style emulates the original \LaTeX{} definition,
%       except that additionally the parameters
%       |\theorem...skipamount| are used.
%    \item[break]
%       In this style, the theorem header is followed by a line
%       break.
%    \item[marginbreak]
%       The theorem number is set in the margin, and there is a line
%       break as in \texttt{break}.
%    \item[changebreak]
%       Like \texttt{break}, but with header number and text
%       interchanged.
%    \item[change]
%       Header number and text are interchanged, without a line break.
%    \item[margin]
%       The number is set in the left margin, without a line break.
% \end{deflist}
% All styles (except \texttt{plain}) select |\slshape| as the default
% |\theorembodyfont|.
%
%
% \subsection{Examples}
%
% Given the above theorem sets {\envfont Cor}, {\envfont Exa}
% and {\envfont Rem},
% suppose that the preamble also contains the declarations:
% \begin{verbatim}
%    \theoremstyle{marginbreak}   \newtheorem{Lem}[Cor]{Lemma}
%    \theoremstyle{change}
%    \theorembodyfont{\itshape}        \newtheorem{Def}[Cor]{Definition}
%
%    \theoremheaderfont{\scshape}
%\end{verbatim}
% Then the following are some typical examples of the typeset output
% resulting from their use.
%
% \begin{Cor}
%    This is a sentence typeset in the theorem environment {\envfont
%    Cor}.
% \end{Cor}
%
% \begin{Exa}
%    This is a sentence typeset in the theorem environment {\envfont
%    Exa}.
% \end{Exa}
%
% \begin{Rem}
%    This is a sentence typeset in the theorem environment {\envfont
%    Rem}.
% \end{Rem}
%
% \begin{Lem}[Ben User]
%    This is a sentence typeset in the theorem environment {\envfont
%    Lem}.
% \end{Lem}
%
% \begin{Def}[Very impressive Definition]
%    This is a sentence typeset in the theorem environment {\envfont
%    Def}.
% \end{Def}
%
% The last two examples show the effect of the optional argument to
% a theorem environment (it is the text typeset in parentheses).
%
%
% \section{Special Considerations}
% \label{sec:newfonts}
%
% Theoremheader and body are implemented as a unit. This means that
% the |\theoremheaderfont|  will inherit characteristics of the
% |\theorembodyfont| in \LaTeXe. Thus, if for example
% |\theorembodyfont| is |\itshape| and |\theoremheaderfont|
% is |\bfseries| the font selected for the header will have the
% characteristics `bold extended italic'. If this is not desired one
% should set the |\theoremheaderfont| to something like
% \begin{verbatim}
%      \theoremheaderfont{\normalfont\bfseries}
%\end{verbatim}
% i.e.\ supplying all necessary font information explicitly.
%
%
%
% \section{Acknowledgements}
%
% The publication of this set of macros was only possible with the
% help of Christina Busse (translating the manuscript into English),
% Joachim Pense (playing the r\^ole of typist), Chris Rowley
% (looking everything over) and many others providing useful
% suggestions.
%
% \MaybeStop{%
%    \begin{thebibliography}{9}
%       \bibitem{bk:GMS94} \textsc{M.~Goossens}, \textsc{F.~Mittelbach}
%         and \textsc{A.~Samarin}.
%         \newblock The \LaTeX{} Companion.
%         \newblock
%         Addison-Wesley, Reading, Massachusetts, 1994.
%       \bibitem{latex:source} \textsc{Lamport, Leslie.} \newblock
%                               \texttt{latex.tex}, version 2.09, date
%                               Feb. 1990.
%    \end{thebibliography}
%    \PrintIndex
%    \PrintChanges}
%
%
% \section{The documentation driver file}
%
% The next bit of code contains the documentation driver file for
% \TeX{}, i.e., the file that will produce the documentation you are
% currently reading. It will be extracted from this file by the
% \texttt{docstrip} program. Since it is the first code in the file
% one can alternatively process this file directly with \LaTeXe{} to
% obtain the documentation.
%
%    \begin{macrocode}
%<*driver>
\documentclass{ltxdoc}

\usepackage{theorem}

%<+driver>% The next few lines define theorem sets which are used
%<+driver>% in the example section of the documentation.

\theoremstyle{break}       \newtheorem{Cor}{Corollary}
\theoremstyle{plain}       \newtheorem{Exa}{Example}[section]
{\theorembodyfont{\upshape}\newtheorem{Rem}{Remark}}
\theoremstyle{marginbreak} \newtheorem{Lem}[Cor]{Lemma}
\theoremstyle{change}
\theorembodyfont{\itshape} \newtheorem{Def}[Cor]{Definition}

\theoremheaderfont{\scshape}

\RecordChanges

\begin{document}
   \DocInput{theorem.dtx}
\end{document}
%</driver>
%    \end{macrocode}
%
% \section{Definition of the Macros}
%
% If the file has been loaded before, we abort immediately. If not the
% package announces itself (this is actually done at the very top of
% the file---the way it is done isn't good style so don't copy it).
% \changes{v2.0e}{89/07/19}{Spaces removed from `typeout'.}
% \changes{v2.2b}{95/11/19}{Announce moved to top of file}
%    \begin{macrocode}
%<*package>
%\@ifundefined{theorem@style}{}{\endinput}
%\def\FMithmInfo{1995/11/19 v2.2b Theorem extension package (FMi)}
%\ProvidesPackage{theorem}[\FMithmInfo]
%    \end{macrocode}
%
%
% \subsection{Definition of theorem styles and fonts}
%
% All the definitions in this file are done globally to allow
% inputting this file inside a group.
%
% \begin{macro}{\theoremstyle}
%    Before a theorem style can be installed, the chosen style must be
%    known. For that reason, we must test to see that
%    |\th@|\meta{style} is known or, more precisely, that it is
%    different from |\relax|. If the style is not known then
%    |\th@plain| is used.
%    \begin{macrocode}
\gdef\theoremstyle#1{%
   \@ifundefined{th@#1}{\@warning
          {Unknown theoremstyle `#1'. Using `plain'}%
          \theorem@style{plain}}%
%    \end{macrocode}
%    We save the theorem style to be used in the token register
%    |\theorem@style|.
%    \begin{macrocode}
      {\theorem@style{#1}}%
%    \end{macrocode}
%    Now we ``evaluate'' the theorem style: this means, we call the
%    macro |\th@|\meta{style} which will activate the relevant
%    definitions which are contained in a separate file.  This is done
%    in a group to suppress changes to the current font.  This could
%    otherwise pose problems together with the new font selection
%    scheme\footnote{When I printed the original article using the new
%    font selection scheme I ended with a document with slanted
%    typefaces (text headings and all) simply because one of the
%    theorem styles used \texttt{\bslash sl} at toplevel.} if the
%    |\th@|\meta{style} is evaluated a second time.
% \changes{v2.1a}{90/02/15}{Group added around `th@..}
% \changes{v2.1a}{90/02/15}{Reseting of `..@begintheorem removed.}
%    \begin{macrocode}
      \begingroup
        \csname th@\the\theorem@style \endcsname
      \endgroup}
%    \end{macrocode}
% \end{macro}
%
%
% \begin{macro}{\@begintheorem}
% \changes{v2.1a}{90/02/15}{Relaxed at toplevel.}
% \begin{macro}{\@opargbegintheorem}
% \changes{v2.1a}{90/02/15}{Relaxed at toplevel.}
%    We reset |\@begintheorem| and |\@opargbegintheorem| to
%    |\relax| since these commands are no longer necessary at
%    toplevel. This will save a few tokens.
%    \begin{macrocode}
\global\let\@begintheorem\relax
\global\let\@opargbegintheorem\relax
%    \end{macrocode}
% \end{macro}
% \end{macro}
%
% \begin{macro}{\theorem@style}
%    Obviously the \textsf{token} register used above has to be
%    allocated. To assure the utmost compatibility with the
%    original \LaTeX{} definition, we set the default theorem style to
%    \texttt{plain}, which implements the usual \LaTeX{} convention.
% \changes{v2.0f}{89/07/20}{Unnecessary `global' removed.}
%    \begin{macrocode}
\newtoks\theorem@style
\global\theorem@style{plain}
%    \end{macrocode}
% \end{macro}
%
% \begin{macro}{\theorembodyfont}
% \begin{macro}{\theorem@bodyfont}
%    If the |\theorembodyfont| is set by the user then it should not
%    interact with the default font set in the theorem style. When the
%    new font selection is in force this may happen if, for example,
%    the default is |\itshape| and the new |\theorembodyfont| is
%    |\sffamily|. So we add a |\reset@font| command in front of the
%    user definition.
% \changes{v2.0f}{89/07/20}{Unnecessary `global' removed.}
% \changes{v2.0g}{89/09/19}{`theoremfont to `theorembodyfont.}
% \changes{v2.1c}{92/06/22}{`reset@font added and logic changed.}
% \changes{v2.1d}{92/10/28}{Reset to empty if arg is empty}
%    \begin{macrocode}
\gdef\theorembodyfont#1{%
%    \end{macrocode}
%    We check if the argument supplied is empty and if so put nothing
%    into the |\theorem@bodyfont| token register to allow for
%    |\theorembodyfont{}| as a mean of using the default of the
%    current |\theoremstyle|.
%    \begin{macrocode}
   \def\@tempa{#1}%
   \ifx\@tempa\@empty
    \theorem@bodyfont{}%
   \else
    \theorem@bodyfont{\reset@font#1}%
   \fi
   }
\newtoks\theorem@bodyfont
\global\theorem@bodyfont{}
%    \end{macrocode}
% \end{macro}
% \end{macro}
%
%
% \begin{macro}{\theoremheaderfont}
% \changes{v2.0g}{89/09/19}{Macro added.}
%    The font for the theorem headers is handled differently because
%    this definition applies to all theorem styles.
%    \begin{macrocode}
\gdef\theoremheaderfont#1{\gdef\theorem@headerfont{#1}%
%    \end{macrocode}
%    After using the macro once it is redefined to produce an error
%    message.
%    \begin{macrocode}
       \gdef\theoremheaderfont##1{%
        \typeout{\string\theoremheaderfont\space should be used
                 only once.}}}
%    \end{macrocode}
% \end{macro}
%
% \begin{macro}{\theorem@headerfont}
% \changes{v2.0g}{89/09/19}{Macro added.}
%    To set the |\theorem@headerfont| default we first test if
%    the new fontselection scheme is in force.
% \changes{v2.1a}{90/02/15}{Support for new font selection.}
%    \begin{macrocode}
\ifx\upshape\undefined
%    \end{macrocode}
%    If not we define it to expand into |\bfseries|. We don't use
%    |\let| just in case a following style option redefines this
%    macro.
%    \begin{macrocode}
\gdef\theorem@headerfont{\bfseries}
%    \end{macrocode}
%    Otherwise we reset the current shape before calling |\bfseries|.
%    \begin{macrocode}
\else \gdef\theorem@headerfont{\normalfont\bfseries}\fi
%    \end{macrocode}
% \end{macro}
%
% \begin{macro}{\th@plain}
% \begin{macro}{\th@break}
% \begin{macro}{\th@marginbreak}
% \begin{macro}{\th@changebreak}
% \begin{macro}{\th@change}
% \begin{macro}{\th@margin}
%    The different styles are defined in macros such as
%    |\th@plain|.  Since memory space is precious in
%    ``non-Big-versions'', we have to avoid offering too many unused
%    definitions.  Therefore we define these styles in separate files
%    that can be loaded on demand.  Thus the commands themselves only
%    load these files. We use |\@input@|, a \LaTeXe{} internal command
%    that ensures that the file will be listed with |\listfiles|.
% \changes{v2.2a}{94/02/02}{Use `@input@ to load theorem layout files}
%    \begin{macrocode}
\gdef\th@plain{\@input@{thp.sty}}
\gdef\th@break{\@input@{thb.sty}}
\gdef\th@marginbreak{\@input@{thmb.sty}}
\gdef\th@changebreak{\@input@{thcb.sty}}
\gdef\th@change{\@input@{thc.sty}}
\gdef\th@margin{\@input@{thm.sty}}
%    \end{macrocode}
%    This list will be expanded when new styles become
%    available.  For testing,  just append new theorem substyles
%    as document options.
% \end{macro}
% \end{macro}
% \end{macro}
% \end{macro}
% \end{macro}
% \end{macro}
%
%
%
% \subsection{Definition of a new theorem set}
%
% As already pointed out, a new theorem environment can be defined in
% three different ways:
% \begin{verbatim}
%    \newtheorem{Lem}{Lemma}
%    \newtheorem{Lem}{Lemma}[section]
%    \newtheorem{Lem}[Theorem]{Lemma}
%\end{verbatim}
% The function of the macro |\newtheorem| is to recognize these cases
% and then to branch into one of the three macros |\@ynthm|, |\@xnthm|
% or |\@othm|.  This mechanism is adopted unchanged from
% \cite{latex:source}; the essential point here is that, for example,
% in the second case, the arguments \texttt{Lem}, \texttt{Lemma} and
% \texttt{section} are passed over to the macro |\@xnthm|.
%
% We inspect this case first because the others present fewer problems,
% and thus are easily derived from this one.
%
% \begin{macro}{\@xnthm}
%    For our example arguments, the macro |\@xnthm| must fulfill
%    the following:
%    \begin{itemize}
%       \item
%          Define a new \LaTeX-counter `Lem'
%       \item
%          reset this counter within a |\section|
%       \item
%          define the macro |\theLem|
%       \item
%          define the environment macros |\Lem| and
%          |\endLem| using the current
%          |\theoremstyle| and |\theorem@bodyfont|.
%    \end{itemize}
%    Obviously, all this should happen only if the first argument
%    of |\@xnthm| (i.e.\ \texttt{Lem} in our example) is chosen so
%    as not to conflict with any previously defined commands or
%    environments.  This test is performed by the \LaTeX{} macro
%    |\@ifdefinable|.
%    \begin{macrocode}
\gdef\@xnthm#1#2[#3]{\expandafter\@ifdefinable\csname #1\endcsname
%    \end{macrocode}
%    Therefore, the first argument of |\@ifdefinable| is the
%    expansion (in the example, |\Lem|) of
%    |\csname||#1\endcsname|.  The second argument is
%    executed only if the test has been completed successfully.
%    \begin{macrocode}
   {%
%    \end{macrocode}
%    Now we define the new counter. The names of the \LaTeX{}
%    macros employed should speak for themselves:
% \changes{v2.2b}{1995/10/07}{Use \cs{@newctr}}
% \changes{v2.2c}{1995/11/23}{Use \cs{@newctr} properly.}
%    \begin{macrocode}
    \@definecounter{#1}\@newctr{#1}[#3]%
%    \end{macrocode}
%    Using |\@newctr| will give a proper error message if the counter
%    in |#3| is not defined.
%    In defining `|\theLem|' we must generate the desired macro
%    name by use of |\expandafter| and |\csname|.
%    \begin{macrocode}
    \expandafter\xdef\csname the#1\endcsname
%    \end{macrocode}
%    An |\xdef| is used in order to make the definition global,
%    and to ensure that it contains the replacement texts of
%    |\@thmcountersep| and |\@thmcounter|.\footnote{These
%    two macros can be defined by the document style.  Their default
%    values produce a `\texttt{.}' as separation and an arabic
%    representation of the number.} However, not everything should be
%    expanded. For example, it saves space to use |\thesection|
%    instead of its---at times---lengthy expansion.
%    \begin{macrocode}
      {\expandafter \noexpand \csname the#3\endcsname
       \@thmcountersep \@thmcounter{#1}}%
%    \end{macrocode}
%    Thus with the defaults of \LaTeX{}, |\theLem| would be
%    replaced by the command sequence |\thesection.\arabic{Lem}|.
%
%    We will now look at the definition of the macro which is executed
%    at the beginning of the actual environment (in our example this
%    macro is |\Lem|). It should be noted that we use an
%    ``|\expandafter| trick'' to expand only certain parts of the
%    replacement text at the time of the definition.
%    \begin{macrocode}
    \def\@tempa{\global\@namedef{#1}}%
    \expandafter \@tempa \expandafter{%
%    \end{macrocode}
%    First, the macro that contains the current definitions of
%    |\@begintheorem| and |\@opargtheorem| should be called
%    up.  The name of this macro---as is already known---has the form
%    |\th@|\meta{theorem\ style}; therefore, it must be called by
%    \begin{macrocode}
      \csname th@\the \theorem@style
%    \end{macrocode}
%    In addition the default theorem font should be changeable, i.e.\
%    we have to insert the contents of |\theorem@bodyfont|. For
%    that reason, we expand even further, beyond |\endcsname|,
%    and thus insert the contents of the \textsf{token} register
%    |\theorem@bodyfont| in the replacement text.
%    \begin{macrocode}
            \expandafter \endcsname \the \theorem@bodyfont
%    \end{macrocode}
%    Now it is time to call the macro |\@thm| which takes over
%    the further processing.  It has two arguments: the current
%    counter name (in our example, \texttt{Lem}), and the text of the
%    label (in our example, \texttt{Lemma}).
%    \begin{macrocode}
     \@thm{#1}{#2}}%
%    \end{macrocode}
%    With this, the `sub-definition' is complete.  The macro
%    |\@endtheorem| ends a theorem environment and is, so far, nothing
%    but an |\endtrivlist|.  (Hence it is defined globally, and not
%    within the theorem styles.\footnote{This has to be changed as
%    soon as theorem styles that change \texttt{\bslash @endtheorem}
%    exist.  In such a case, all existing styles must be changed as
%    well since they will have to reset the macro.}) Therefore, we can
%    set it equivalent to the macro that ends the theorem set (in our
%    example, |\endLem|).  However, if some day theorem styles exist
%    that do change |\@endtheorem|, we would have to use the
%    commented-out line instead.
%    \begin{macrocode}
    \global \expandafter \let \csname end#1\endcsname \@endtheorem
%   \global\@namedef{end#1}{\@endtheorem}%
%    \end{macrocode}
%    With these commands all the required definitions are employed,
%    unless the test |\@ifdefinable| has failed.  Therefore, we
%    end the second argument of this macro and with it the definition
%    of |\@xnthm|.
%    \begin{macrocode}
   }}
%    \end{macrocode}
% \end{macro}
%
% \begin{macro}{\@ynthm}
%    The definition of |\@ynthm| is completely analogous. In
%    this case the new counter that is defined is not reset within
%    another counter; thus the definition of |\the...|
%    is simplified:
%    \begin{macrocode}
\gdef\@ynthm#1#2{\expandafter\@ifdefinable\csname #1\endcsname
   {\@definecounter{#1}%
    \expandafter\xdef\csname the#1\endcsname{\@thmcounter{#1}}%
%    \end{macrocode}
%    The rest of the definition corresponds literally to that of
%    |\@xnthm|:
%    \begin{macrocode}
    \def\@tempa{\global\@namedef{#1}}\expandafter \@tempa
     \expandafter{\csname th@\the \theorem@style \expandafter
     \endcsname \the\theorem@bodyfont \@thm{#1}{#2}}%
    \global \expandafter \let \csname end#1\endcsname \@endtheorem}}
%    \end{macrocode}
% \end{macro}
%
% \begin{macro}{\@othm}
%    The definition of |\@othm| does not contain anything new.
%    \begin{macrocode}
\gdef\@othm#1[#2]#3{%
%    \end{macrocode}
% \changes{v2.2b}{95/10/07}{Build in check for counter defined}
%    We do not define a new counter but instead use one that has
%    already been defined.  Thus the only definition we need is that
%    of this pseudo-counter (i.e.\ |\the|\meta{env.\ name}).
%    First we check if |#2| corresponds to a known counter name.
%    \begin{macrocode}
  \expandafter\ifx\csname c@#2\endcsname\relax
   \@nocounterr{#2}%
  \else
   \expandafter\@ifdefinable\csname #1\endcsname
   {\expandafter \xdef \csname the#1\endcsname
     {\expandafter \noexpand \csname the#2\endcsname}%
%    \end{macrocode}
%    All other parts of the definition can be adopted from
%    |\@xnthm|. We have to remember, though, that in this case
%    the name of the current counter and the theorem label have moved
%    to the second and third arguments.
%    \begin{macrocode}
    \def\@tempa{\global\@namedef{#1}}\expandafter \@tempa
     \expandafter{\csname th@\the \theorem@style \expandafter
     \endcsname \the\theorem@bodyfont \@thm{#2}{#3}}%
    \global \expandafter \let \csname end#1\endcsname \@endtheorem}%
  \fi}
%    \end{macrocode}
% \end{macro}
%
% \subsection{Macros that are employed in a {\envfont theorem}
%    environment}
%
% \begin{macro}{\@thm}
%    The macro |\@thm| has to increase the current counter.  Then,
%    depending on whether the environment has (or does not have) an
%    optional argument, it has to branch into either |\@begintheorem|
%    or |\@opargtheorem|.
%    \begin{macrocode}
\gdef\@thm#1#2{\refstepcounter{#1}%
%    \end{macrocode}
%    Now we start a {\envfont trivlist} environment, and give
%    |\@topsep| and |\@topsepadd| the values of the \textsf{skip}
%    registers |\theorempreskipamount| and |\theorempostskipamount|.
%    The value in |\@topsep| is the vertical space that is inserted by
%    the first (and only) |\item| in our |\trivlist| whilst
%    |\@topsepadd| is inserted by |\@endparenv| at the end of that
%    {\envfont trivlist} environment.  By using these registers, we
%    obtain the desired space around a {\envfont theorem} environment.
%    \begin{macrocode}
   \trivlist
   \@topsep \theorempreskipamount               % used by first \item
   \@topsepadd \theorempostskipamount           % used by \@endparenv
%    \end{macrocode}
%    Now we have to test whether an optional argument has been given.
% \changes{v2.0d}{89/07/18}{Forgotten percent added.}
%    \begin{macrocode}
   \@ifnextchar [%
%    \end{macrocode}
%    If there is an optional argument, we will call |\@ythm|, and
%    move the arguments read back into the input stream.
%    \begin{macrocode}
   {\@ythm{#1}{#2}}%
%    \end{macrocode}
%    If not, we call |\@begintheorem|. Its first argument is the
%    name of the theorem set (hence the second argument of
%    |\@thm|). Its second argument is the macro that produces the
%    current number.
%    \begin{macrocode}
   {\@begintheorem{#2}{\csname the#1\endcsname}\ignorespaces}}
%    \end{macrocode}
% \end{macro}
%
% \begin{macro}{\@xthm}
% \begin{macro}{\@ythm}
%    Both these macros were originally called by |\@thm|.  We do
%    not need |\@xthm| anymore, hence we reset it to
%    |\relax|. The definition of |\@ythm| has not changed at
%    all from its definition in \LaTeX. In order to make the macros
%    easier to understand, we will
%    nevertheless present it (commented out).
%    \begin{macrocode}
\global\let\@xthm\relax
% \def\@ythm#1#2[#3]{\@opargbegintheorem{#2}{\csname
%           the#1\endcsname}{#3}\ignorespaces}
%</package>
%    \end{macrocode}
%    The primitive |\ignorespaces| in |\@ythm| and
%    |\@thm| is needed to remove the spaces between the
%    |\begin{...}| and the actual text.
% \end{macro}
% \end{macro}
%
%
% \subsection{Definition of the theorem substyles}
%
% As already pointed out, the theorem substyles, defined below, are
% only loaded when necessary.  Note that all these substyles, except
% \texttt{plain}, have |\slshape| as the default body font.
%
% \subsubsection{The \texttt{plain} style}
%
% \label{dok:th@plain} As the following macros use |@|, we have
% to locally set the |\catcode| of this symbol to ``letter''.
% This happens within a group, so that we do not have to worry about
% which |\catcode| that symbol had before.
%    \begin{macrocode}
%<*thp>
\begingroup \makeatletter
%    \end{macrocode}
% Since we are now within a group, we must make all definitions
% globally.  First we make sure that \texttt{theorem.sty} is loaded.
% This will allow us to use this file as a document style option
% without having to call \texttt{theorem} itself as an option.  At the
% same time, we assure that at least version 2 is loaded, since
% |\theorem@style| was not defined in earlier versions.
%    \begin{macrocode}
\@ifundefined{theorem@style}{\input{theorem.sty}}{}
\ProvidesFile{thp.sty}
             [\FMithmInfo]
%    \end{macrocode}
%
%
% \begin{macro}{\th@plain}
% \begin{macro}{\@begintheorem}
% \begin{macro}{\@opargbegintheorem}
%    |\theoremstyle{plain}| corresponds to the original
%    definition, except that the distances to the surrounding text are
%    determined by the parameters |\theorempreskipamount| and
%    |\theorempostskipamount|.
%    First we set the default body font.
%    \begin{macrocode}
\gdef\th@plain{\normalfont\itshape
%    \end{macrocode}
%    Then we define |\@begintheorem| and
%    |\@opargbegintheorem|. These two macros define how the
%    header of a theorem is typeset.  |\@opargbegintheorem| will
%    be called if a \textsf{theorem} environment with an optional
%    argument is encountered; otherwise, the header is constructed by
%    calling |\@begintheorem|.  If one of these macros is
%    executed, we are within a \textsf{trivlist} environment started by
%    |\@thm|.  So the theorem header is produced with an
%    |\item| command.
%
%    Instead of specifying the header font directly, all standard
%    theorem styles use the |\theorem@headerfont| macro to allow
%    customization.  The extra space (|\labelsep|) is necessary
%    because of problems in the \textsf{trivlist} environment.
%    \begin{macrocode}
  \def\@begintheorem##1##2{%
        \item[\hskip\labelsep \theorem@headerfont ##1\ ##2]}%
%    \end{macrocode}
%    The definition of |\@opargbegintheorem| is completely
%    analogous. The only difference is the fact that there exists a
%    third argument (which is the optional parameter of the
%    environment and contains additional information about the
%    theorem).  Customarily we enclose it in parentheses.
%    \begin{macrocode}
\def\@opargbegintheorem##1##2##3{%
   \item[\hskip\labelsep \theorem@headerfont ##1\ ##2\ (##3)]}}
%    \end{macrocode}
% \end{macro}
% \end{macro}
% \end{macro}
%
% \noindent
% We conclude with an |\endgroup| to restore the
% |\catcode| of |@|.
%    \begin{macrocode}
\endgroup
%</thp>
%    \end{macrocode}
%
%
% \subsubsection{The \texttt{break} style}
%
% This style option is stored in the file |thb.sty|.
% For the next two lines see the documentation for |\th@plain| on
% page~\pageref{dok:th@plain}.
%    \begin{macrocode}
%<*thb>
\begingroup \makeatletter
\@ifundefined{theorem@style}{\input{theorem.sty}}{}
\ProvidesFile{thb.sty}
             [\FMithmInfo]
%    \end{macrocode}
%
% \begin{macro}{\th@break}
%    |\theoremstyle{break}| produces a line break after the name
%    of the theorem.  The font is |\slshape|.  Hence, we define
%    |\th@break| as follows:
%    \begin{macrocode}
\gdef\th@break{\normalfont\slshape
  \def\@begintheorem##1##2{\item[%
%    \end{macrocode}
%    We run into the following problem: it is not possible to create
%    the header with |\item[|\meta{title}|]| and then start
%    a new line by, for example, |\mbox{}\\|. Such a definition
%    will fail whenever a list environment follows immediately.  With
%    the above construction, the |\mbox{}| causes the switch
%    |@inlabel| (cf.\ definition of |\list| and
%    |\trivlist| in \cite{latex:source}) to be set to \texttt{false}
%    and so the following list will insert additional vertical
%    space (|\topskip|). This is quite annoying. Therefore, we
%    create the line break within the |\item|. In order to ensure
%    that the text will begin at the proper position in the following
%    line, we simply pretend that the label does not take any
%    room.\footnote{This will lead to problems whenever very high
%    symbols occurring in the line tower into the
%    heading. So, something else has to be done here sometime.}
%    \begin{macrocode}
       \rlap{\vbox{\hbox{\hskip \labelsep\theorem@headerfont ##1\ ##2}%
                   \hbox{\strut}}}]}%
%    \end{macrocode}
%    Again, the definition of |\@opargbegintheorem| is
%    completely analogous.
%    \begin{macrocode}
\def\@opargbegintheorem##1##2##3{%
  \item[\rlap{\vbox{\hbox{\hskip \labelsep \theorem@headerfont
                     ##1\ ##2\ (##3)}%
                    \hbox{\strut}}}]}}
\endgroup
%</thb>
%    \end{macrocode}
% \end{macro}
%
%
% \subsubsection{The \texttt{changebreak} style}
%
%    \begin{macrocode}
% This style option is stored in the file |thcb.sty|.
%    \begin{macrocode}
%<*thcb>
\begingroup \makeatletter
\@ifundefined{theorem@style}{\input{theorem.sty}}{}
\ProvidesFile{thcb.sty}
             [\FMithmInfo]
%    \end{macrocode}
%
% \begin{macro}{\th@changebreak}
%    The \texttt{change-break} theorem style is like \texttt{break}
%    but with interchange of theorem name and theorem number.  Thus we
%    define |\th@changebreak| as follows:
%    \begin{macrocode}
\gdef\th@changebreak{\normalfont\slshape
  \def\@begintheorem##1##2{\item
       [\rlap{\vbox{\hbox{\hskip\labelsep \theorem@headerfont ##2\ ##1}%
                    \hbox{\strut}}}]}%
\def\@opargbegintheorem##1##2##3{%
  \item[\rlap{\vbox{\hbox{\hskip\labelsep \theorem@headerfont
                     ##2\ ##1\ (##3)}%
                    \hbox{\strut}}}]}}
\endgroup
%</thcb>
%    \end{macrocode}
% \end{macro}
%
%
% \subsubsection{The \texttt{change} style}
%
% This style option is stored in the file |thc.sty|.
%    \begin{macrocode}
%<*thc>
\begingroup \makeatletter
\@ifundefined{theorem@style}{\input{theorem.sty}}{}
\ProvidesFile{thc.sty}
             [\FMithmInfo]
%    \end{macrocode}
%
% \begin{macro}{\th@change}
%    The \texttt{change} theorem style corresponds to the
%    \texttt{change break} style without a linebreak after the header.
%    To say it in another way, it's the same as the \texttt{plain}
%    style but with number and name interchanged and |\slshape| as the
%    default font.
%    \begin{macrocode}
\gdef\th@change{\normalfont\slshape
  \def\@begintheorem##1##2{\item
       [\hskip\labelsep \theorem@headerfont ##2\ ##1]}%
\def\@opargbegintheorem##1##2##3{%
  \item[\hskip\labelsep \theorem@headerfont ##2\ ##1\ (##3)]}}
\endgroup
%</thc>
%    \end{macrocode}
% \end{macro}
%
%
% \subsubsection{The \texttt{marginbreak} style}
%
% This style option is the one used most often at Mainz.  It is saved
% in the file |thmb.sty|.
%    \begin{macrocode}
%<*thmb>
\begingroup \makeatletter
\@ifundefined{theorem@style}{\input{theorem.sty}}{}
\ProvidesFile{thmb.sty}
             [\FMithmInfo]
%    \end{macrocode}
%
% \begin{macro}{\th@marginbreak}
%    The \texttt{margin break} style is nearly the same as the
%    \texttt{change break} style. The only difference is the placement
%    of the theorem number.  We use |\llap| to place it in the left
%    margin.
%
%    In this style |\labelsep| denotes the separation between the
%    number and the text.
%    \begin{macrocode}
\gdef\th@marginbreak{\normalfont\slshape
  \def\@begintheorem##1##2{\item
       [\rlap{\vbox{\theorem@headerfont
                    \hbox{\llap{##2}\hskip\labelsep ##1}%
                    \hbox{\strut}}}]}%
\def\@opargbegintheorem##1##2##3{%
  \item[\rlap{\vbox{\theorem@headerfont
                    \hbox{\llap{##2}\hskip\labelsep ##1\ (##3)}%
                    \hbox{\strut}}}]}}
\endgroup
%</thmb>
%    \end{macrocode}
% \changes{v2.1c}{90/02/17}{`endgroup outside macrocode env.}
% \end{macro}
%
%
%
% \subsubsection{The \texttt{margin} style}
%
% This style option is stored in the file |thm.sty|.
%    \begin{macrocode}
%<*thm>
\begingroup \makeatletter
\@ifundefined{theorem@style}{\input{theorem.sty}}{}
\ProvidesFile{thm.sty}
             [\FMithmInfo]
%    \end{macrocode}
%
% \begin{macro}{\th@margin}
%    Again this is only a variant of the theorem styles described
%    above without any new ideas.
%    \begin{macrocode}
\gdef\th@margin{\normalfont\slshape
  \def\@begintheorem##1##2{\item
       [\theorem@headerfont \llap{##2}\hskip\labelsep ##1]}%
\def\@opargbegintheorem##1##2##3{%
  \item[\theorem@headerfont \llap{##2}\hskip\labelsep ##1\ (##3)]}}
\endgroup
%</thm>
%    \end{macrocode}
% \end{macro}
%
%
%
%
% \subsection{Final Definitions}
%
% \begin{macro}{\theorempreskipamount}
% \begin{macro}{\theorempostskipamount}
%    The \textsf{skip} parameters that regulate the vertical empty space
%    before and after the {\envfont theorem} environment have
%    to be allocated as well.
% \changes{v2.0f}{89/07/20}{Unnecessary `global' removed.}
%    \begin{macrocode}
%<*package>
\newskip\theorempreskipamount
\newskip\theorempostskipamount
%    \end{macrocode}
%    Since we have used the same values for all theorem sets, we now can
%    assign them.
%    \begin{macrocode}
\global\setlength\theorempreskipamount{12pt plus 5pt minus 3pt}
\global\setlength\theorempostskipamount{8pt plus 3pt minus 1.5pt}
%    \end{macrocode}
% \end{macro}
% \end{macro}
%
%
% \begin{macro}{\@endtheorem}
%    The same holds for the macro |\@endtheorem|, which ends a
%    {\envfont theorem} environment. Since it is the same for all
%    theorem sets, it is removed from the macros |\th@|\meta{style}.
%    It simply ends the {\envfont trivlist} environment, which was
%    begun in |\@thm|.
%    \begin{macrocode}
\global\let\@endtheorem=\endtrivlist
%    \end{macrocode}
% \end{macro}
%
%
%
% \begin{macro}{\@preamblecmds}
%    All macros defined above are to be used only in the preamble.
%    Therefore, we insert them in |\@preamblecmds| which will disable
%    them at begin document. This is done by the internal \LaTeXe{}
%    command |\@onlypreamble|.
% \changes{v2.0g}{89/09/19}{`theoremheaderfont and `theorembodyfont
%    added.}
% \changes{v2.2a}{94/02/03}{Use \LaTeXe{} interface}
%    \begin{macrocode}
\@onlypreamble\@xnthm
\@onlypreamble\@ynthm
\@onlypreamble\@othm
\@onlypreamble\newtheorem
\@onlypreamble\theoremstyle
\@onlypreamble\theorembodyfont
\@onlypreamble\theoremheaderfont
%    \end{macrocode}
% \end{macro}
%
% Finally we declare the \texttt{plain} theorem style to be the default.
% \changes{v2.1b}{90/06/04}{Fixed missing `item by declaring default
%    style.}
%    \begin{macrocode}
\theoremstyle{plain}
%</package>
%    \end{macrocode}
%
% \Finale

\endinput
}{}
\ProvidesFile{thp.sty}
             [\FMithmInfo]
%    \end{macrocode}
%
%
% \begin{macro}{\th@plain}
% \begin{macro}{\@begintheorem}
% \begin{macro}{\@opargbegintheorem}
%    |\theoremstyle{plain}| corresponds to the original
%    definition, except that the distances to the surrounding text are
%    determined by the parameters |\theorempreskipamount| and
%    |\theorempostskipamount|.
%    First we set the default body font.
%    \begin{macrocode}
\gdef\th@plain{\normalfont\itshape
%    \end{macrocode}
%    Then we define |\@begintheorem| and
%    |\@opargbegintheorem|. These two macros define how the
%    header of a theorem is typeset.  |\@opargbegintheorem| will
%    be called if a \textsf{theorem} environment with an optional
%    argument is encountered; otherwise, the header is constructed by
%    calling |\@begintheorem|.  If one of these macros is
%    executed, we are within a \textsf{trivlist} environment started by
%    |\@thm|.  So the theorem header is produced with an
%    |\item| command.
%
%    Instead of specifying the header font directly, all standard
%    theorem styles use the |\theorem@headerfont| macro to allow
%    customization.  The extra space (|\labelsep|) is necessary
%    because of problems in the \textsf{trivlist} environment.
%    \begin{macrocode}
  \def\@begintheorem##1##2{%
        \item[\hskip\labelsep \theorem@headerfont ##1\ ##2]}%
%    \end{macrocode}
%    The definition of |\@opargbegintheorem| is completely
%    analogous. The only difference is the fact that there exists a
%    third argument (which is the optional parameter of the
%    environment and contains additional information about the
%    theorem).  Customarily we enclose it in parentheses.
%    \begin{macrocode}
\def\@opargbegintheorem##1##2##3{%
   \item[\hskip\labelsep \theorem@headerfont ##1\ ##2\ (##3)]}}
%    \end{macrocode}
% \end{macro}
% \end{macro}
% \end{macro}
%
% \noindent
% We conclude with an |\endgroup| to restore the
% |\catcode| of |@|.
%    \begin{macrocode}
\endgroup
%</thp>
%    \end{macrocode}
%
%
% \subsubsection{The \texttt{break} style}
%
% This style option is stored in the file |thb.sty|.
% For the next two lines see the documentation for |\th@plain| on
% page~\pageref{dok:th@plain}.
%    \begin{macrocode}
%<*thb>
\begingroup \makeatletter
\@ifundefined{theorem@style}{% \iffalse meta-comment
%
% Copyright (C) 1993-2024
%
% The LaTeX Project and any individual authors listed elsewhere
% in this file.
%
% This file is part of the Standard LaTeX `Tools Bundle'.
% -------------------------------------------------------
%
% It may be distributed and/or modified under the
% conditions of the LaTeX Project Public License, either version 1.3c
% of this license or (at your option) any later version.
% The latest version of this license is in
%    https://www.latex-project.org/lppl.txt
% and version 1.3c or later is part of all distributions of LaTeX
% version 2005/12/01 or later.
%
% The list of all files belonging to the LaTeX `Tools Bundle' is
% given in the file `manifest.txt'.
%
% \fi
%
%%
%
%
\def\FMithmInfo{2023/07/05 v2.2c Theorem extension package (FMi)}
%
% \ProvidesFile{theorem.dtx}[\FMithmInfo]
%\iffalse   % this is a METACOMMENT !
%%
%% Package `theorem' to use with LaTeX2e
%% Copyright (C) 1989-1995 Frank Mittelbach, all rights reserved.
%<+package>\@ifundefined{theorem@style}{}{\endinput}
%<+package>\NeedsTeXFormat{LaTeX2e}[1995/06/01]
%<+package>\ProvidesPackage{theorem}
%<+package>         [\FMithmInfo]
%
% \fi
%
% \changes{v1.0}{87/05/07}{First draft.}
% \changes{v1.0a}{87/05/07}{`thistheoremstyle' added.}
% \changes{v1.1a}{87/07/29}{New concept: redefine the internal
%                          \protect\LaTeX{} commands `@xnthm' etc.}
% \changes{v1.1b}{87/07/31}{`theoremfont' used to specify font.}
% \changes{v1.2a}{88/01/08}{`@endtheorem' now globally defined.}
% \changes{v1.2b}{88/01/11}{`newtheorem' only allowed in the preamble.}
% \changes{v1.3a}{88/02/14}{`theoremfont' now a toks register.}
%
% \changes{v2.0a}{88/11/16}{New implementation and new documentation.}
% \changes{v2.0a}{88/11/16}{No layout changes but internal modifications
%                           old substyles won't work.}
% \changes{v2.0b}{89/01/30}{`global' used for all definitions.}
% \changes{v2.0c}{89/06/15}{Documentation translated to English.
%                            Thanks to LLL and Nina}
% \changes{v2.1a}{90/02/15}{Support for new font selection.}
% \changes{v2.2a}{94/02/03}{Cleanup for LaTeX2e}
%
% \DoNotIndex{\@endparenv,\@ifdefinable,\@ifnextchar,\@ifundefined}
% \DoNotIndex{\@namedef,\@nameuse,\@spaces,\@tempa,\@warning}
% \DoNotIndex{\ }
% \DoNotIndex{\begingroup,\bfseries}
% \DoNotIndex{\csname}
% \DoNotIndex{\def,\do,\docdate}
% \DoNotIndex{\endcsname,\endgroup,\endinput,\endtrivlist,\expandafter}
% \DoNotIndex{\filedate,\fileversion}
% \DoNotIndex{\gdef,\global}
% \DoNotIndex{\hbox,\hskip}
% \DoNotIndex{\ignorespaces,\input,\itshape,\item}
% \DoNotIndex{\labelsep,\let,\llap}
% \DoNotIndex{\makeatletter}
% \DoNotIndex{\noexpand}
% \DoNotIndex{\refstepcounter,\relax,\rlap}
% \DoNotIndex{\setlength,\slshape,\space,\string,\strut}
% \DoNotIndex{\the,\trivlist,\typeout}
% \DoNotIndex{\vbox}
% \DoNotIndex{\xdef}
%
% \def\deflabel#1{\ttfamily #1\hfill}
% \def\deflist#1{\begin{list}{}{\settowidth\labelwidth{\ttfamily #1}%
%                               \setlength\leftmargin\labelwidth
%                               \addtolength\leftmargin\labelsep
%                               \let\makelabel\deflabel}}
% \def\enddeflist{\end{list}}
% \def\envfont{\normalfont\ttfamily}
%
% \MakeShortVerb{\|}
%
% \newcommand{\TUB}{{\sl TUGboat\/}}
%
% \GetFileInfo{theorem.dtx}
% \title{An Extension of the
%        \LaTeX{} theorem environment\thanks{This file
%        has version number \fileversion, last
%        revised \filedate.}}
% \author{Frank Mittelbach}
%
% \MaintainedByLaTeXTeam{tools}
% \maketitle
%
% \begin{abstract}
%    The macros described in this paper yield an extension of the
%    \LaTeX{} theorem mechanism. It is designed to satisfy the
%    different requirements of various journals. Thus, the
%    layout of the ``theorems'' can be manipulated by determining a
%    ``style''. This article describes not only the use, but
%    also the definition, of the necessary macros.
% \end{abstract}
%
% \section*{Preface to version 2.2}
%
% For \LaTeXe{} this package did not need any fundamental changes. I
% only modified the messages generated so that theorem layout styles
% will show up with the |\listfiles| command and cleaned the section
% on the New Font Selection Scheme since this is now included in
% \LaTeX.
%
% \section*{Preface to version 2.1}
%
% This version is identical to 2.0g described in \TUB\ 10\#3 except
% for some internal defaults which are now set depending on the used
% font selection scheme.
%
% This was done to avoid unpleasant surprises if the new font selection
% scheme is in force. For further details see
% section~\ref{sec:newfonts} and \cite{bk:GMS94}.
%
% \section{Introduction}
%
% For our purposes here, ``theorems'' are labelled enunciations,
% often set off from the main text by extra space and a font change.
% Theorems, corollaries, conjectures, definitions, and remarks
% are all instances of ``theorems''.  The ``header'' of these structures
% is composed of a label (such as \textsc{Theorem} or \textsc{Remark})
% and a number which serializes an item in the sequence of items
% with the same label.
%
% Shortly after the introduction of \LaTeX{} at the Fachbereich
% Mathematik in Mainz, the desire to manipulate the layout of
% ``theorems'' arose.  In Mainz, the following two conventions came
% into general use:
% \begin{enumerate}
%   \item The number of the theorem is shown in the margin.
%   \item There is a line break at
%         the end of the theorem header.
% \end{enumerate}
%
% Additionally, some journals require different formats which depend on
% the ``sort of theorem'': e.g.\ often remarks and definitions are set
% in |\upshape|, while |\itshape| is employed for main theorems.
%
% Confronted with these requirements, a theorem environment was
% developed in Mainz which allows separate determination of the
% layout of the ``theorems sets'', comparable to |\pagestyle|.
%
%
% \section{The user interface}
% \subsection{Defining new theorem sets}
%
% \DescribeMacro\newtheorem
% As in the original \LaTeX{} version, the command |\newtheorem|
% defines a new ``theorem set'' or ``theorem-like structure''.
% Two required arguments name the new environment and give the text
% to be typeset with each instance of the new ``set'', while
% an optional argument determines how
% the ``set'' is enumerated:
% \begin{description}
%    \item[\ttfamily \bslash newtheorem\{foo\}\{bar\}]
%       The theorem set {\envfont foo} (whose name is \texttt{bar})
%       uses its own counter.
%    \item[\ttfamily \bslash newtheorem\{foo2\}{[foo]}\{bar2\}]
%       The theorem set {\envfont foo2} (printed name \texttt{bar2})
%       uses the same counter as the theorem set \texttt{foo}.
%    \item[\ttfamily \bslash newtheorem\{foo3\}\{bar3\}{[section]}]
%       The theorem set {\envfont foo3} (printed name \texttt{bar3}) is
%       enumerated within the counter \texttt{section}, i.e.\ with every
%       new |\section| the enumeration begins again with 1, and
%       the enumeration is composed from the section-number and the
%       theorem counter itself.
% \end{description}
%
% \DescribeMacro\theoremstyle
% Additionally, the command |\theoremstyle| can define the layout
% of various, or all, theorem sets.  It should be noted that any
% theorem set defined by |\newtheorem| is typeset in the
% |\theoremstyle| that is current at the time of the definition.
% Thus, the following
% \begin{verbatim}
%    \theoremstyle{break}        \newtheorem{Cor}{Corollary}
%    \theoremstyle{plain}        \newtheorem{Exa}{Example}[section]
%\end{verbatim}                             ^^A No blank at line start
% leads to the result that the set {\envfont Cor} is formatted in the
% style \texttt{break}, while the set {\envfont Exa} and all the
% following ones are formatted in the style \texttt{plain}, unless
% another |\theoremstyle| follows.  Since the definitions installed by
% |\newtheorem| are global, one also can limit |\theoremstyle| locally
% by grouping braces.
%
% \DescribeMacro\theorembodyfont
% The choice of the font for the theorem body is completely
% independent of the chosen |\theoremstyle|; this has proven to
% be very advantageous.  For example,
% \begin{verbatim}
%   {\theorembodyfont{\upshape}          \newtheorem{Rem}{Remark}}
%\end{verbatim}
% defines a theorem set {\envfont Rem}, which will be set in
% |\upshape| in the current layout (which in our example is
% \texttt{plain}).  As with |\theoremstyle|, the |\theorembodyfont|
% chosen is that current at the time of |\newtheorem|.  If
% |\theorembodyfont| is not specified or one defines
% |\theorembodyfont{}|, then the font used will be that defined by the
% |\theoremstyle|.
%
% \DescribeMacro\theoremheaderfont It is also possible to customize
% the font used for the theorem headers.  This is, however, a global
% declaration, and therefore there should be at most one
% |\theoremheaderfont| declaration in the preamble.\footnote{If it is
% actually necessary to have different header fonts, one has to define
% new theorem styles (substituting the desired font) or specify the
% information directly in the \texttt{\bslash newtheorem} declaration
% (the unclean variant).}
%
% \DescribeMacro\theorempreskipamount
% \DescribeMacro\theorempostskipamount Two additional parameters
% affect the vertical space around the theorem environments:
% |\theorempreskipamount| and |\theorempostskipamount| define,
% respectively, the spacing before and after such an environment.
% These parameters apply for all theorem sets and can be manipulated
% with the ordinary length macros.  They are rubber lengths
% (`\textsf{skips}'), and therefore can contain \texttt{plus} and
% \texttt{minus} parts.
%
% Since the definition of theorem sets should---most sensibly---be
% placed in the preamble, we only allow installation there.
% It is therefore possible to release the memory used here after
% |\begin{document}|, in order to make room for other
% applications.
%
%
% \subsection{Existing theorem styles}
%
% The following theorem styles exist to date:
% \begin{deflist}{marginbreak}
%    \item[plain]
%       This theorem style emulates the original \LaTeX{} definition,
%       except that additionally the parameters
%       |\theorem...skipamount| are used.
%    \item[break]
%       In this style, the theorem header is followed by a line
%       break.
%    \item[marginbreak]
%       The theorem number is set in the margin, and there is a line
%       break as in \texttt{break}.
%    \item[changebreak]
%       Like \texttt{break}, but with header number and text
%       interchanged.
%    \item[change]
%       Header number and text are interchanged, without a line break.
%    \item[margin]
%       The number is set in the left margin, without a line break.
% \end{deflist}
% All styles (except \texttt{plain}) select |\slshape| as the default
% |\theorembodyfont|.
%
%
% \subsection{Examples}
%
% Given the above theorem sets {\envfont Cor}, {\envfont Exa}
% and {\envfont Rem},
% suppose that the preamble also contains the declarations:
% \begin{verbatim}
%    \theoremstyle{marginbreak}   \newtheorem{Lem}[Cor]{Lemma}
%    \theoremstyle{change}
%    \theorembodyfont{\itshape}        \newtheorem{Def}[Cor]{Definition}
%
%    \theoremheaderfont{\scshape}
%\end{verbatim}
% Then the following are some typical examples of the typeset output
% resulting from their use.
%
% \begin{Cor}
%    This is a sentence typeset in the theorem environment {\envfont
%    Cor}.
% \end{Cor}
%
% \begin{Exa}
%    This is a sentence typeset in the theorem environment {\envfont
%    Exa}.
% \end{Exa}
%
% \begin{Rem}
%    This is a sentence typeset in the theorem environment {\envfont
%    Rem}.
% \end{Rem}
%
% \begin{Lem}[Ben User]
%    This is a sentence typeset in the theorem environment {\envfont
%    Lem}.
% \end{Lem}
%
% \begin{Def}[Very impressive Definition]
%    This is a sentence typeset in the theorem environment {\envfont
%    Def}.
% \end{Def}
%
% The last two examples show the effect of the optional argument to
% a theorem environment (it is the text typeset in parentheses).
%
%
% \section{Special Considerations}
% \label{sec:newfonts}
%
% Theoremheader and body are implemented as a unit. This means that
% the |\theoremheaderfont|  will inherit characteristics of the
% |\theorembodyfont| in \LaTeXe. Thus, if for example
% |\theorembodyfont| is |\itshape| and |\theoremheaderfont|
% is |\bfseries| the font selected for the header will have the
% characteristics `bold extended italic'. If this is not desired one
% should set the |\theoremheaderfont| to something like
% \begin{verbatim}
%      \theoremheaderfont{\normalfont\bfseries}
%\end{verbatim}
% i.e.\ supplying all necessary font information explicitly.
%
%
%
% \section{Acknowledgements}
%
% The publication of this set of macros was only possible with the
% help of Christina Busse (translating the manuscript into English),
% Joachim Pense (playing the r\^ole of typist), Chris Rowley
% (looking everything over) and many others providing useful
% suggestions.
%
% \MaybeStop{%
%    \begin{thebibliography}{9}
%       \bibitem{bk:GMS94} \textsc{M.~Goossens}, \textsc{F.~Mittelbach}
%         and \textsc{A.~Samarin}.
%         \newblock The \LaTeX{} Companion.
%         \newblock
%         Addison-Wesley, Reading, Massachusetts, 1994.
%       \bibitem{latex:source} \textsc{Lamport, Leslie.} \newblock
%                               \texttt{latex.tex}, version 2.09, date
%                               Feb. 1990.
%    \end{thebibliography}
%    \PrintIndex
%    \PrintChanges}
%
%
% \section{The documentation driver file}
%
% The next bit of code contains the documentation driver file for
% \TeX{}, i.e., the file that will produce the documentation you are
% currently reading. It will be extracted from this file by the
% \texttt{docstrip} program. Since it is the first code in the file
% one can alternatively process this file directly with \LaTeXe{} to
% obtain the documentation.
%
%    \begin{macrocode}
%<*driver>
\documentclass{ltxdoc}

\usepackage{theorem}

%<+driver>% The next few lines define theorem sets which are used
%<+driver>% in the example section of the documentation.

\theoremstyle{break}       \newtheorem{Cor}{Corollary}
\theoremstyle{plain}       \newtheorem{Exa}{Example}[section]
{\theorembodyfont{\upshape}\newtheorem{Rem}{Remark}}
\theoremstyle{marginbreak} \newtheorem{Lem}[Cor]{Lemma}
\theoremstyle{change}
\theorembodyfont{\itshape} \newtheorem{Def}[Cor]{Definition}

\theoremheaderfont{\scshape}

\RecordChanges

\begin{document}
   \DocInput{theorem.dtx}
\end{document}
%</driver>
%    \end{macrocode}
%
% \section{Definition of the Macros}
%
% If the file has been loaded before, we abort immediately. If not the
% package announces itself (this is actually done at the very top of
% the file---the way it is done isn't good style so don't copy it).
% \changes{v2.0e}{89/07/19}{Spaces removed from `typeout'.}
% \changes{v2.2b}{95/11/19}{Announce moved to top of file}
%    \begin{macrocode}
%<*package>
%\@ifundefined{theorem@style}{}{\endinput}
%\def\FMithmInfo{1995/11/19 v2.2b Theorem extension package (FMi)}
%\ProvidesPackage{theorem}[\FMithmInfo]
%    \end{macrocode}
%
%
% \subsection{Definition of theorem styles and fonts}
%
% All the definitions in this file are done globally to allow
% inputting this file inside a group.
%
% \begin{macro}{\theoremstyle}
%    Before a theorem style can be installed, the chosen style must be
%    known. For that reason, we must test to see that
%    |\th@|\meta{style} is known or, more precisely, that it is
%    different from |\relax|. If the style is not known then
%    |\th@plain| is used.
%    \begin{macrocode}
\gdef\theoremstyle#1{%
   \@ifundefined{th@#1}{\@warning
          {Unknown theoremstyle `#1'. Using `plain'}%
          \theorem@style{plain}}%
%    \end{macrocode}
%    We save the theorem style to be used in the token register
%    |\theorem@style|.
%    \begin{macrocode}
      {\theorem@style{#1}}%
%    \end{macrocode}
%    Now we ``evaluate'' the theorem style: this means, we call the
%    macro |\th@|\meta{style} which will activate the relevant
%    definitions which are contained in a separate file.  This is done
%    in a group to suppress changes to the current font.  This could
%    otherwise pose problems together with the new font selection
%    scheme\footnote{When I printed the original article using the new
%    font selection scheme I ended with a document with slanted
%    typefaces (text headings and all) simply because one of the
%    theorem styles used \texttt{\bslash sl} at toplevel.} if the
%    |\th@|\meta{style} is evaluated a second time.
% \changes{v2.1a}{90/02/15}{Group added around `th@..}
% \changes{v2.1a}{90/02/15}{Reseting of `..@begintheorem removed.}
%    \begin{macrocode}
      \begingroup
        \csname th@\the\theorem@style \endcsname
      \endgroup}
%    \end{macrocode}
% \end{macro}
%
%
% \begin{macro}{\@begintheorem}
% \changes{v2.1a}{90/02/15}{Relaxed at toplevel.}
% \begin{macro}{\@opargbegintheorem}
% \changes{v2.1a}{90/02/15}{Relaxed at toplevel.}
%    We reset |\@begintheorem| and |\@opargbegintheorem| to
%    |\relax| since these commands are no longer necessary at
%    toplevel. This will save a few tokens.
%    \begin{macrocode}
\global\let\@begintheorem\relax
\global\let\@opargbegintheorem\relax
%    \end{macrocode}
% \end{macro}
% \end{macro}
%
% \begin{macro}{\theorem@style}
%    Obviously the \textsf{token} register used above has to be
%    allocated. To assure the utmost compatibility with the
%    original \LaTeX{} definition, we set the default theorem style to
%    \texttt{plain}, which implements the usual \LaTeX{} convention.
% \changes{v2.0f}{89/07/20}{Unnecessary `global' removed.}
%    \begin{macrocode}
\newtoks\theorem@style
\global\theorem@style{plain}
%    \end{macrocode}
% \end{macro}
%
% \begin{macro}{\theorembodyfont}
% \begin{macro}{\theorem@bodyfont}
%    If the |\theorembodyfont| is set by the user then it should not
%    interact with the default font set in the theorem style. When the
%    new font selection is in force this may happen if, for example,
%    the default is |\itshape| and the new |\theorembodyfont| is
%    |\sffamily|. So we add a |\reset@font| command in front of the
%    user definition.
% \changes{v2.0f}{89/07/20}{Unnecessary `global' removed.}
% \changes{v2.0g}{89/09/19}{`theoremfont to `theorembodyfont.}
% \changes{v2.1c}{92/06/22}{`reset@font added and logic changed.}
% \changes{v2.1d}{92/10/28}{Reset to empty if arg is empty}
%    \begin{macrocode}
\gdef\theorembodyfont#1{%
%    \end{macrocode}
%    We check if the argument supplied is empty and if so put nothing
%    into the |\theorem@bodyfont| token register to allow for
%    |\theorembodyfont{}| as a mean of using the default of the
%    current |\theoremstyle|.
%    \begin{macrocode}
   \def\@tempa{#1}%
   \ifx\@tempa\@empty
    \theorem@bodyfont{}%
   \else
    \theorem@bodyfont{\reset@font#1}%
   \fi
   }
\newtoks\theorem@bodyfont
\global\theorem@bodyfont{}
%    \end{macrocode}
% \end{macro}
% \end{macro}
%
%
% \begin{macro}{\theoremheaderfont}
% \changes{v2.0g}{89/09/19}{Macro added.}
%    The font for the theorem headers is handled differently because
%    this definition applies to all theorem styles.
%    \begin{macrocode}
\gdef\theoremheaderfont#1{\gdef\theorem@headerfont{#1}%
%    \end{macrocode}
%    After using the macro once it is redefined to produce an error
%    message.
%    \begin{macrocode}
       \gdef\theoremheaderfont##1{%
        \typeout{\string\theoremheaderfont\space should be used
                 only once.}}}
%    \end{macrocode}
% \end{macro}
%
% \begin{macro}{\theorem@headerfont}
% \changes{v2.0g}{89/09/19}{Macro added.}
%    To set the |\theorem@headerfont| default we first test if
%    the new fontselection scheme is in force.
% \changes{v2.1a}{90/02/15}{Support for new font selection.}
%    \begin{macrocode}
\ifx\upshape\undefined
%    \end{macrocode}
%    If not we define it to expand into |\bfseries|. We don't use
%    |\let| just in case a following style option redefines this
%    macro.
%    \begin{macrocode}
\gdef\theorem@headerfont{\bfseries}
%    \end{macrocode}
%    Otherwise we reset the current shape before calling |\bfseries|.
%    \begin{macrocode}
\else \gdef\theorem@headerfont{\normalfont\bfseries}\fi
%    \end{macrocode}
% \end{macro}
%
% \begin{macro}{\th@plain}
% \begin{macro}{\th@break}
% \begin{macro}{\th@marginbreak}
% \begin{macro}{\th@changebreak}
% \begin{macro}{\th@change}
% \begin{macro}{\th@margin}
%    The different styles are defined in macros such as
%    |\th@plain|.  Since memory space is precious in
%    ``non-Big-versions'', we have to avoid offering too many unused
%    definitions.  Therefore we define these styles in separate files
%    that can be loaded on demand.  Thus the commands themselves only
%    load these files. We use |\@input@|, a \LaTeXe{} internal command
%    that ensures that the file will be listed with |\listfiles|.
% \changes{v2.2a}{94/02/02}{Use `@input@ to load theorem layout files}
%    \begin{macrocode}
\gdef\th@plain{\@input@{thp.sty}}
\gdef\th@break{\@input@{thb.sty}}
\gdef\th@marginbreak{\@input@{thmb.sty}}
\gdef\th@changebreak{\@input@{thcb.sty}}
\gdef\th@change{\@input@{thc.sty}}
\gdef\th@margin{\@input@{thm.sty}}
%    \end{macrocode}
%    This list will be expanded when new styles become
%    available.  For testing,  just append new theorem substyles
%    as document options.
% \end{macro}
% \end{macro}
% \end{macro}
% \end{macro}
% \end{macro}
% \end{macro}
%
%
%
% \subsection{Definition of a new theorem set}
%
% As already pointed out, a new theorem environment can be defined in
% three different ways:
% \begin{verbatim}
%    \newtheorem{Lem}{Lemma}
%    \newtheorem{Lem}{Lemma}[section]
%    \newtheorem{Lem}[Theorem]{Lemma}
%\end{verbatim}
% The function of the macro |\newtheorem| is to recognize these cases
% and then to branch into one of the three macros |\@ynthm|, |\@xnthm|
% or |\@othm|.  This mechanism is adopted unchanged from
% \cite{latex:source}; the essential point here is that, for example,
% in the second case, the arguments \texttt{Lem}, \texttt{Lemma} and
% \texttt{section} are passed over to the macro |\@xnthm|.
%
% We inspect this case first because the others present fewer problems,
% and thus are easily derived from this one.
%
% \begin{macro}{\@xnthm}
%    For our example arguments, the macro |\@xnthm| must fulfill
%    the following:
%    \begin{itemize}
%       \item
%          Define a new \LaTeX-counter `Lem'
%       \item
%          reset this counter within a |\section|
%       \item
%          define the macro |\theLem|
%       \item
%          define the environment macros |\Lem| and
%          |\endLem| using the current
%          |\theoremstyle| and |\theorem@bodyfont|.
%    \end{itemize}
%    Obviously, all this should happen only if the first argument
%    of |\@xnthm| (i.e.\ \texttt{Lem} in our example) is chosen so
%    as not to conflict with any previously defined commands or
%    environments.  This test is performed by the \LaTeX{} macro
%    |\@ifdefinable|.
%    \begin{macrocode}
\gdef\@xnthm#1#2[#3]{\expandafter\@ifdefinable\csname #1\endcsname
%    \end{macrocode}
%    Therefore, the first argument of |\@ifdefinable| is the
%    expansion (in the example, |\Lem|) of
%    |\csname||#1\endcsname|.  The second argument is
%    executed only if the test has been completed successfully.
%    \begin{macrocode}
   {%
%    \end{macrocode}
%    Now we define the new counter. The names of the \LaTeX{}
%    macros employed should speak for themselves:
% \changes{v2.2b}{1995/10/07}{Use \cs{@newctr}}
% \changes{v2.2c}{1995/11/23}{Use \cs{@newctr} properly.}
%    \begin{macrocode}
    \@definecounter{#1}\@newctr{#1}[#3]%
%    \end{macrocode}
%    Using |\@newctr| will give a proper error message if the counter
%    in |#3| is not defined.
%    In defining `|\theLem|' we must generate the desired macro
%    name by use of |\expandafter| and |\csname|.
%    \begin{macrocode}
    \expandafter\xdef\csname the#1\endcsname
%    \end{macrocode}
%    An |\xdef| is used in order to make the definition global,
%    and to ensure that it contains the replacement texts of
%    |\@thmcountersep| and |\@thmcounter|.\footnote{These
%    two macros can be defined by the document style.  Their default
%    values produce a `\texttt{.}' as separation and an arabic
%    representation of the number.} However, not everything should be
%    expanded. For example, it saves space to use |\thesection|
%    instead of its---at times---lengthy expansion.
%    \begin{macrocode}
      {\expandafter \noexpand \csname the#3\endcsname
       \@thmcountersep \@thmcounter{#1}}%
%    \end{macrocode}
%    Thus with the defaults of \LaTeX{}, |\theLem| would be
%    replaced by the command sequence |\thesection.\arabic{Lem}|.
%
%    We will now look at the definition of the macro which is executed
%    at the beginning of the actual environment (in our example this
%    macro is |\Lem|). It should be noted that we use an
%    ``|\expandafter| trick'' to expand only certain parts of the
%    replacement text at the time of the definition.
%    \begin{macrocode}
    \def\@tempa{\global\@namedef{#1}}%
    \expandafter \@tempa \expandafter{%
%    \end{macrocode}
%    First, the macro that contains the current definitions of
%    |\@begintheorem| and |\@opargtheorem| should be called
%    up.  The name of this macro---as is already known---has the form
%    |\th@|\meta{theorem\ style}; therefore, it must be called by
%    \begin{macrocode}
      \csname th@\the \theorem@style
%    \end{macrocode}
%    In addition the default theorem font should be changeable, i.e.\
%    we have to insert the contents of |\theorem@bodyfont|. For
%    that reason, we expand even further, beyond |\endcsname|,
%    and thus insert the contents of the \textsf{token} register
%    |\theorem@bodyfont| in the replacement text.
%    \begin{macrocode}
            \expandafter \endcsname \the \theorem@bodyfont
%    \end{macrocode}
%    Now it is time to call the macro |\@thm| which takes over
%    the further processing.  It has two arguments: the current
%    counter name (in our example, \texttt{Lem}), and the text of the
%    label (in our example, \texttt{Lemma}).
%    \begin{macrocode}
     \@thm{#1}{#2}}%
%    \end{macrocode}
%    With this, the `sub-definition' is complete.  The macro
%    |\@endtheorem| ends a theorem environment and is, so far, nothing
%    but an |\endtrivlist|.  (Hence it is defined globally, and not
%    within the theorem styles.\footnote{This has to be changed as
%    soon as theorem styles that change \texttt{\bslash @endtheorem}
%    exist.  In such a case, all existing styles must be changed as
%    well since they will have to reset the macro.}) Therefore, we can
%    set it equivalent to the macro that ends the theorem set (in our
%    example, |\endLem|).  However, if some day theorem styles exist
%    that do change |\@endtheorem|, we would have to use the
%    commented-out line instead.
%    \begin{macrocode}
    \global \expandafter \let \csname end#1\endcsname \@endtheorem
%   \global\@namedef{end#1}{\@endtheorem}%
%    \end{macrocode}
%    With these commands all the required definitions are employed,
%    unless the test |\@ifdefinable| has failed.  Therefore, we
%    end the second argument of this macro and with it the definition
%    of |\@xnthm|.
%    \begin{macrocode}
   }}
%    \end{macrocode}
% \end{macro}
%
% \begin{macro}{\@ynthm}
%    The definition of |\@ynthm| is completely analogous. In
%    this case the new counter that is defined is not reset within
%    another counter; thus the definition of |\the...|
%    is simplified:
%    \begin{macrocode}
\gdef\@ynthm#1#2{\expandafter\@ifdefinable\csname #1\endcsname
   {\@definecounter{#1}%
    \expandafter\xdef\csname the#1\endcsname{\@thmcounter{#1}}%
%    \end{macrocode}
%    The rest of the definition corresponds literally to that of
%    |\@xnthm|:
%    \begin{macrocode}
    \def\@tempa{\global\@namedef{#1}}\expandafter \@tempa
     \expandafter{\csname th@\the \theorem@style \expandafter
     \endcsname \the\theorem@bodyfont \@thm{#1}{#2}}%
    \global \expandafter \let \csname end#1\endcsname \@endtheorem}}
%    \end{macrocode}
% \end{macro}
%
% \begin{macro}{\@othm}
%    The definition of |\@othm| does not contain anything new.
%    \begin{macrocode}
\gdef\@othm#1[#2]#3{%
%    \end{macrocode}
% \changes{v2.2b}{95/10/07}{Build in check for counter defined}
%    We do not define a new counter but instead use one that has
%    already been defined.  Thus the only definition we need is that
%    of this pseudo-counter (i.e.\ |\the|\meta{env.\ name}).
%    First we check if |#2| corresponds to a known counter name.
%    \begin{macrocode}
  \expandafter\ifx\csname c@#2\endcsname\relax
   \@nocounterr{#2}%
  \else
   \expandafter\@ifdefinable\csname #1\endcsname
   {\expandafter \xdef \csname the#1\endcsname
     {\expandafter \noexpand \csname the#2\endcsname}%
%    \end{macrocode}
%    All other parts of the definition can be adopted from
%    |\@xnthm|. We have to remember, though, that in this case
%    the name of the current counter and the theorem label have moved
%    to the second and third arguments.
%    \begin{macrocode}
    \def\@tempa{\global\@namedef{#1}}\expandafter \@tempa
     \expandafter{\csname th@\the \theorem@style \expandafter
     \endcsname \the\theorem@bodyfont \@thm{#2}{#3}}%
    \global \expandafter \let \csname end#1\endcsname \@endtheorem}%
  \fi}
%    \end{macrocode}
% \end{macro}
%
% \subsection{Macros that are employed in a {\envfont theorem}
%    environment}
%
% \begin{macro}{\@thm}
%    The macro |\@thm| has to increase the current counter.  Then,
%    depending on whether the environment has (or does not have) an
%    optional argument, it has to branch into either |\@begintheorem|
%    or |\@opargtheorem|.
%    \begin{macrocode}
\gdef\@thm#1#2{\refstepcounter{#1}%
%    \end{macrocode}
%    Now we start a {\envfont trivlist} environment, and give
%    |\@topsep| and |\@topsepadd| the values of the \textsf{skip}
%    registers |\theorempreskipamount| and |\theorempostskipamount|.
%    The value in |\@topsep| is the vertical space that is inserted by
%    the first (and only) |\item| in our |\trivlist| whilst
%    |\@topsepadd| is inserted by |\@endparenv| at the end of that
%    {\envfont trivlist} environment.  By using these registers, we
%    obtain the desired space around a {\envfont theorem} environment.
%    \begin{macrocode}
   \trivlist
   \@topsep \theorempreskipamount               % used by first \item
   \@topsepadd \theorempostskipamount           % used by \@endparenv
%    \end{macrocode}
%    Now we have to test whether an optional argument has been given.
% \changes{v2.0d}{89/07/18}{Forgotten percent added.}
%    \begin{macrocode}
   \@ifnextchar [%
%    \end{macrocode}
%    If there is an optional argument, we will call |\@ythm|, and
%    move the arguments read back into the input stream.
%    \begin{macrocode}
   {\@ythm{#1}{#2}}%
%    \end{macrocode}
%    If not, we call |\@begintheorem|. Its first argument is the
%    name of the theorem set (hence the second argument of
%    |\@thm|). Its second argument is the macro that produces the
%    current number.
%    \begin{macrocode}
   {\@begintheorem{#2}{\csname the#1\endcsname}\ignorespaces}}
%    \end{macrocode}
% \end{macro}
%
% \begin{macro}{\@xthm}
% \begin{macro}{\@ythm}
%    Both these macros were originally called by |\@thm|.  We do
%    not need |\@xthm| anymore, hence we reset it to
%    |\relax|. The definition of |\@ythm| has not changed at
%    all from its definition in \LaTeX. In order to make the macros
%    easier to understand, we will
%    nevertheless present it (commented out).
%    \begin{macrocode}
\global\let\@xthm\relax
% \def\@ythm#1#2[#3]{\@opargbegintheorem{#2}{\csname
%           the#1\endcsname}{#3}\ignorespaces}
%</package>
%    \end{macrocode}
%    The primitive |\ignorespaces| in |\@ythm| and
%    |\@thm| is needed to remove the spaces between the
%    |\begin{...}| and the actual text.
% \end{macro}
% \end{macro}
%
%
% \subsection{Definition of the theorem substyles}
%
% As already pointed out, the theorem substyles, defined below, are
% only loaded when necessary.  Note that all these substyles, except
% \texttt{plain}, have |\slshape| as the default body font.
%
% \subsubsection{The \texttt{plain} style}
%
% \label{dok:th@plain} As the following macros use |@|, we have
% to locally set the |\catcode| of this symbol to ``letter''.
% This happens within a group, so that we do not have to worry about
% which |\catcode| that symbol had before.
%    \begin{macrocode}
%<*thp>
\begingroup \makeatletter
%    \end{macrocode}
% Since we are now within a group, we must make all definitions
% globally.  First we make sure that \texttt{theorem.sty} is loaded.
% This will allow us to use this file as a document style option
% without having to call \texttt{theorem} itself as an option.  At the
% same time, we assure that at least version 2 is loaded, since
% |\theorem@style| was not defined in earlier versions.
%    \begin{macrocode}
\@ifundefined{theorem@style}{\input{theorem.sty}}{}
\ProvidesFile{thp.sty}
             [\FMithmInfo]
%    \end{macrocode}
%
%
% \begin{macro}{\th@plain}
% \begin{macro}{\@begintheorem}
% \begin{macro}{\@opargbegintheorem}
%    |\theoremstyle{plain}| corresponds to the original
%    definition, except that the distances to the surrounding text are
%    determined by the parameters |\theorempreskipamount| and
%    |\theorempostskipamount|.
%    First we set the default body font.
%    \begin{macrocode}
\gdef\th@plain{\normalfont\itshape
%    \end{macrocode}
%    Then we define |\@begintheorem| and
%    |\@opargbegintheorem|. These two macros define how the
%    header of a theorem is typeset.  |\@opargbegintheorem| will
%    be called if a \textsf{theorem} environment with an optional
%    argument is encountered; otherwise, the header is constructed by
%    calling |\@begintheorem|.  If one of these macros is
%    executed, we are within a \textsf{trivlist} environment started by
%    |\@thm|.  So the theorem header is produced with an
%    |\item| command.
%
%    Instead of specifying the header font directly, all standard
%    theorem styles use the |\theorem@headerfont| macro to allow
%    customization.  The extra space (|\labelsep|) is necessary
%    because of problems in the \textsf{trivlist} environment.
%    \begin{macrocode}
  \def\@begintheorem##1##2{%
        \item[\hskip\labelsep \theorem@headerfont ##1\ ##2]}%
%    \end{macrocode}
%    The definition of |\@opargbegintheorem| is completely
%    analogous. The only difference is the fact that there exists a
%    third argument (which is the optional parameter of the
%    environment and contains additional information about the
%    theorem).  Customarily we enclose it in parentheses.
%    \begin{macrocode}
\def\@opargbegintheorem##1##2##3{%
   \item[\hskip\labelsep \theorem@headerfont ##1\ ##2\ (##3)]}}
%    \end{macrocode}
% \end{macro}
% \end{macro}
% \end{macro}
%
% \noindent
% We conclude with an |\endgroup| to restore the
% |\catcode| of |@|.
%    \begin{macrocode}
\endgroup
%</thp>
%    \end{macrocode}
%
%
% \subsubsection{The \texttt{break} style}
%
% This style option is stored in the file |thb.sty|.
% For the next two lines see the documentation for |\th@plain| on
% page~\pageref{dok:th@plain}.
%    \begin{macrocode}
%<*thb>
\begingroup \makeatletter
\@ifundefined{theorem@style}{\input{theorem.sty}}{}
\ProvidesFile{thb.sty}
             [\FMithmInfo]
%    \end{macrocode}
%
% \begin{macro}{\th@break}
%    |\theoremstyle{break}| produces a line break after the name
%    of the theorem.  The font is |\slshape|.  Hence, we define
%    |\th@break| as follows:
%    \begin{macrocode}
\gdef\th@break{\normalfont\slshape
  \def\@begintheorem##1##2{\item[%
%    \end{macrocode}
%    We run into the following problem: it is not possible to create
%    the header with |\item[|\meta{title}|]| and then start
%    a new line by, for example, |\mbox{}\\|. Such a definition
%    will fail whenever a list environment follows immediately.  With
%    the above construction, the |\mbox{}| causes the switch
%    |@inlabel| (cf.\ definition of |\list| and
%    |\trivlist| in \cite{latex:source}) to be set to \texttt{false}
%    and so the following list will insert additional vertical
%    space (|\topskip|). This is quite annoying. Therefore, we
%    create the line break within the |\item|. In order to ensure
%    that the text will begin at the proper position in the following
%    line, we simply pretend that the label does not take any
%    room.\footnote{This will lead to problems whenever very high
%    symbols occurring in the line tower into the
%    heading. So, something else has to be done here sometime.}
%    \begin{macrocode}
       \rlap{\vbox{\hbox{\hskip \labelsep\theorem@headerfont ##1\ ##2}%
                   \hbox{\strut}}}]}%
%    \end{macrocode}
%    Again, the definition of |\@opargbegintheorem| is
%    completely analogous.
%    \begin{macrocode}
\def\@opargbegintheorem##1##2##3{%
  \item[\rlap{\vbox{\hbox{\hskip \labelsep \theorem@headerfont
                     ##1\ ##2\ (##3)}%
                    \hbox{\strut}}}]}}
\endgroup
%</thb>
%    \end{macrocode}
% \end{macro}
%
%
% \subsubsection{The \texttt{changebreak} style}
%
%    \begin{macrocode}
% This style option is stored in the file |thcb.sty|.
%    \begin{macrocode}
%<*thcb>
\begingroup \makeatletter
\@ifundefined{theorem@style}{\input{theorem.sty}}{}
\ProvidesFile{thcb.sty}
             [\FMithmInfo]
%    \end{macrocode}
%
% \begin{macro}{\th@changebreak}
%    The \texttt{change-break} theorem style is like \texttt{break}
%    but with interchange of theorem name and theorem number.  Thus we
%    define |\th@changebreak| as follows:
%    \begin{macrocode}
\gdef\th@changebreak{\normalfont\slshape
  \def\@begintheorem##1##2{\item
       [\rlap{\vbox{\hbox{\hskip\labelsep \theorem@headerfont ##2\ ##1}%
                    \hbox{\strut}}}]}%
\def\@opargbegintheorem##1##2##3{%
  \item[\rlap{\vbox{\hbox{\hskip\labelsep \theorem@headerfont
                     ##2\ ##1\ (##3)}%
                    \hbox{\strut}}}]}}
\endgroup
%</thcb>
%    \end{macrocode}
% \end{macro}
%
%
% \subsubsection{The \texttt{change} style}
%
% This style option is stored in the file |thc.sty|.
%    \begin{macrocode}
%<*thc>
\begingroup \makeatletter
\@ifundefined{theorem@style}{\input{theorem.sty}}{}
\ProvidesFile{thc.sty}
             [\FMithmInfo]
%    \end{macrocode}
%
% \begin{macro}{\th@change}
%    The \texttt{change} theorem style corresponds to the
%    \texttt{change break} style without a linebreak after the header.
%    To say it in another way, it's the same as the \texttt{plain}
%    style but with number and name interchanged and |\slshape| as the
%    default font.
%    \begin{macrocode}
\gdef\th@change{\normalfont\slshape
  \def\@begintheorem##1##2{\item
       [\hskip\labelsep \theorem@headerfont ##2\ ##1]}%
\def\@opargbegintheorem##1##2##3{%
  \item[\hskip\labelsep \theorem@headerfont ##2\ ##1\ (##3)]}}
\endgroup
%</thc>
%    \end{macrocode}
% \end{macro}
%
%
% \subsubsection{The \texttt{marginbreak} style}
%
% This style option is the one used most often at Mainz.  It is saved
% in the file |thmb.sty|.
%    \begin{macrocode}
%<*thmb>
\begingroup \makeatletter
\@ifundefined{theorem@style}{\input{theorem.sty}}{}
\ProvidesFile{thmb.sty}
             [\FMithmInfo]
%    \end{macrocode}
%
% \begin{macro}{\th@marginbreak}
%    The \texttt{margin break} style is nearly the same as the
%    \texttt{change break} style. The only difference is the placement
%    of the theorem number.  We use |\llap| to place it in the left
%    margin.
%
%    In this style |\labelsep| denotes the separation between the
%    number and the text.
%    \begin{macrocode}
\gdef\th@marginbreak{\normalfont\slshape
  \def\@begintheorem##1##2{\item
       [\rlap{\vbox{\theorem@headerfont
                    \hbox{\llap{##2}\hskip\labelsep ##1}%
                    \hbox{\strut}}}]}%
\def\@opargbegintheorem##1##2##3{%
  \item[\rlap{\vbox{\theorem@headerfont
                    \hbox{\llap{##2}\hskip\labelsep ##1\ (##3)}%
                    \hbox{\strut}}}]}}
\endgroup
%</thmb>
%    \end{macrocode}
% \changes{v2.1c}{90/02/17}{`endgroup outside macrocode env.}
% \end{macro}
%
%
%
% \subsubsection{The \texttt{margin} style}
%
% This style option is stored in the file |thm.sty|.
%    \begin{macrocode}
%<*thm>
\begingroup \makeatletter
\@ifundefined{theorem@style}{\input{theorem.sty}}{}
\ProvidesFile{thm.sty}
             [\FMithmInfo]
%    \end{macrocode}
%
% \begin{macro}{\th@margin}
%    Again this is only a variant of the theorem styles described
%    above without any new ideas.
%    \begin{macrocode}
\gdef\th@margin{\normalfont\slshape
  \def\@begintheorem##1##2{\item
       [\theorem@headerfont \llap{##2}\hskip\labelsep ##1]}%
\def\@opargbegintheorem##1##2##3{%
  \item[\theorem@headerfont \llap{##2}\hskip\labelsep ##1\ (##3)]}}
\endgroup
%</thm>
%    \end{macrocode}
% \end{macro}
%
%
%
%
% \subsection{Final Definitions}
%
% \begin{macro}{\theorempreskipamount}
% \begin{macro}{\theorempostskipamount}
%    The \textsf{skip} parameters that regulate the vertical empty space
%    before and after the {\envfont theorem} environment have
%    to be allocated as well.
% \changes{v2.0f}{89/07/20}{Unnecessary `global' removed.}
%    \begin{macrocode}
%<*package>
\newskip\theorempreskipamount
\newskip\theorempostskipamount
%    \end{macrocode}
%    Since we have used the same values for all theorem sets, we now can
%    assign them.
%    \begin{macrocode}
\global\setlength\theorempreskipamount{12pt plus 5pt minus 3pt}
\global\setlength\theorempostskipamount{8pt plus 3pt minus 1.5pt}
%    \end{macrocode}
% \end{macro}
% \end{macro}
%
%
% \begin{macro}{\@endtheorem}
%    The same holds for the macro |\@endtheorem|, which ends a
%    {\envfont theorem} environment. Since it is the same for all
%    theorem sets, it is removed from the macros |\th@|\meta{style}.
%    It simply ends the {\envfont trivlist} environment, which was
%    begun in |\@thm|.
%    \begin{macrocode}
\global\let\@endtheorem=\endtrivlist
%    \end{macrocode}
% \end{macro}
%
%
%
% \begin{macro}{\@preamblecmds}
%    All macros defined above are to be used only in the preamble.
%    Therefore, we insert them in |\@preamblecmds| which will disable
%    them at begin document. This is done by the internal \LaTeXe{}
%    command |\@onlypreamble|.
% \changes{v2.0g}{89/09/19}{`theoremheaderfont and `theorembodyfont
%    added.}
% \changes{v2.2a}{94/02/03}{Use \LaTeXe{} interface}
%    \begin{macrocode}
\@onlypreamble\@xnthm
\@onlypreamble\@ynthm
\@onlypreamble\@othm
\@onlypreamble\newtheorem
\@onlypreamble\theoremstyle
\@onlypreamble\theorembodyfont
\@onlypreamble\theoremheaderfont
%    \end{macrocode}
% \end{macro}
%
% Finally we declare the \texttt{plain} theorem style to be the default.
% \changes{v2.1b}{90/06/04}{Fixed missing `item by declaring default
%    style.}
%    \begin{macrocode}
\theoremstyle{plain}
%</package>
%    \end{macrocode}
%
% \Finale

\endinput
}{}
\ProvidesFile{thb.sty}
             [\FMithmInfo]
%    \end{macrocode}
%
% \begin{macro}{\th@break}
%    |\theoremstyle{break}| produces a line break after the name
%    of the theorem.  The font is |\slshape|.  Hence, we define
%    |\th@break| as follows:
%    \begin{macrocode}
\gdef\th@break{\normalfont\slshape
  \def\@begintheorem##1##2{\item[%
%    \end{macrocode}
%    We run into the following problem: it is not possible to create
%    the header with |\item[|\meta{title}|]| and then start
%    a new line by, for example, |\mbox{}\\|. Such a definition
%    will fail whenever a list environment follows immediately.  With
%    the above construction, the |\mbox{}| causes the switch
%    |@inlabel| (cf.\ definition of |\list| and
%    |\trivlist| in \cite{latex:source}) to be set to \texttt{false}
%    and so the following list will insert additional vertical
%    space (|\topskip|). This is quite annoying. Therefore, we
%    create the line break within the |\item|. In order to ensure
%    that the text will begin at the proper position in the following
%    line, we simply pretend that the label does not take any
%    room.\footnote{This will lead to problems whenever very high
%    symbols occurring in the line tower into the
%    heading. So, something else has to be done here sometime.}
%    \begin{macrocode}
       \rlap{\vbox{\hbox{\hskip \labelsep\theorem@headerfont ##1\ ##2}%
                   \hbox{\strut}}}]}%
%    \end{macrocode}
%    Again, the definition of |\@opargbegintheorem| is
%    completely analogous.
%    \begin{macrocode}
\def\@opargbegintheorem##1##2##3{%
  \item[\rlap{\vbox{\hbox{\hskip \labelsep \theorem@headerfont
                     ##1\ ##2\ (##3)}%
                    \hbox{\strut}}}]}}
\endgroup
%</thb>
%    \end{macrocode}
% \end{macro}
%
%
% \subsubsection{The \texttt{changebreak} style}
%
%    \begin{macrocode}
% This style option is stored in the file |thcb.sty|.
%    \begin{macrocode}
%<*thcb>
\begingroup \makeatletter
\@ifundefined{theorem@style}{% \iffalse meta-comment
%
% Copyright (C) 1993-2024
%
% The LaTeX Project and any individual authors listed elsewhere
% in this file.
%
% This file is part of the Standard LaTeX `Tools Bundle'.
% -------------------------------------------------------
%
% It may be distributed and/or modified under the
% conditions of the LaTeX Project Public License, either version 1.3c
% of this license or (at your option) any later version.
% The latest version of this license is in
%    https://www.latex-project.org/lppl.txt
% and version 1.3c or later is part of all distributions of LaTeX
% version 2005/12/01 or later.
%
% The list of all files belonging to the LaTeX `Tools Bundle' is
% given in the file `manifest.txt'.
%
% \fi
%
%%
%
%
\def\FMithmInfo{2023/07/05 v2.2c Theorem extension package (FMi)}
%
% \ProvidesFile{theorem.dtx}[\FMithmInfo]
%\iffalse   % this is a METACOMMENT !
%%
%% Package `theorem' to use with LaTeX2e
%% Copyright (C) 1989-1995 Frank Mittelbach, all rights reserved.
%<+package>\@ifundefined{theorem@style}{}{\endinput}
%<+package>\NeedsTeXFormat{LaTeX2e}[1995/06/01]
%<+package>\ProvidesPackage{theorem}
%<+package>         [\FMithmInfo]
%
% \fi
%
% \changes{v1.0}{87/05/07}{First draft.}
% \changes{v1.0a}{87/05/07}{`thistheoremstyle' added.}
% \changes{v1.1a}{87/07/29}{New concept: redefine the internal
%                          \protect\LaTeX{} commands `@xnthm' etc.}
% \changes{v1.1b}{87/07/31}{`theoremfont' used to specify font.}
% \changes{v1.2a}{88/01/08}{`@endtheorem' now globally defined.}
% \changes{v1.2b}{88/01/11}{`newtheorem' only allowed in the preamble.}
% \changes{v1.3a}{88/02/14}{`theoremfont' now a toks register.}
%
% \changes{v2.0a}{88/11/16}{New implementation and new documentation.}
% \changes{v2.0a}{88/11/16}{No layout changes but internal modifications
%                           old substyles won't work.}
% \changes{v2.0b}{89/01/30}{`global' used for all definitions.}
% \changes{v2.0c}{89/06/15}{Documentation translated to English.
%                            Thanks to LLL and Nina}
% \changes{v2.1a}{90/02/15}{Support for new font selection.}
% \changes{v2.2a}{94/02/03}{Cleanup for LaTeX2e}
%
% \DoNotIndex{\@endparenv,\@ifdefinable,\@ifnextchar,\@ifundefined}
% \DoNotIndex{\@namedef,\@nameuse,\@spaces,\@tempa,\@warning}
% \DoNotIndex{\ }
% \DoNotIndex{\begingroup,\bfseries}
% \DoNotIndex{\csname}
% \DoNotIndex{\def,\do,\docdate}
% \DoNotIndex{\endcsname,\endgroup,\endinput,\endtrivlist,\expandafter}
% \DoNotIndex{\filedate,\fileversion}
% \DoNotIndex{\gdef,\global}
% \DoNotIndex{\hbox,\hskip}
% \DoNotIndex{\ignorespaces,\input,\itshape,\item}
% \DoNotIndex{\labelsep,\let,\llap}
% \DoNotIndex{\makeatletter}
% \DoNotIndex{\noexpand}
% \DoNotIndex{\refstepcounter,\relax,\rlap}
% \DoNotIndex{\setlength,\slshape,\space,\string,\strut}
% \DoNotIndex{\the,\trivlist,\typeout}
% \DoNotIndex{\vbox}
% \DoNotIndex{\xdef}
%
% \def\deflabel#1{\ttfamily #1\hfill}
% \def\deflist#1{\begin{list}{}{\settowidth\labelwidth{\ttfamily #1}%
%                               \setlength\leftmargin\labelwidth
%                               \addtolength\leftmargin\labelsep
%                               \let\makelabel\deflabel}}
% \def\enddeflist{\end{list}}
% \def\envfont{\normalfont\ttfamily}
%
% \MakeShortVerb{\|}
%
% \newcommand{\TUB}{{\sl TUGboat\/}}
%
% \GetFileInfo{theorem.dtx}
% \title{An Extension of the
%        \LaTeX{} theorem environment\thanks{This file
%        has version number \fileversion, last
%        revised \filedate.}}
% \author{Frank Mittelbach}
%
% \MaintainedByLaTeXTeam{tools}
% \maketitle
%
% \begin{abstract}
%    The macros described in this paper yield an extension of the
%    \LaTeX{} theorem mechanism. It is designed to satisfy the
%    different requirements of various journals. Thus, the
%    layout of the ``theorems'' can be manipulated by determining a
%    ``style''. This article describes not only the use, but
%    also the definition, of the necessary macros.
% \end{abstract}
%
% \section*{Preface to version 2.2}
%
% For \LaTeXe{} this package did not need any fundamental changes. I
% only modified the messages generated so that theorem layout styles
% will show up with the |\listfiles| command and cleaned the section
% on the New Font Selection Scheme since this is now included in
% \LaTeX.
%
% \section*{Preface to version 2.1}
%
% This version is identical to 2.0g described in \TUB\ 10\#3 except
% for some internal defaults which are now set depending on the used
% font selection scheme.
%
% This was done to avoid unpleasant surprises if the new font selection
% scheme is in force. For further details see
% section~\ref{sec:newfonts} and \cite{bk:GMS94}.
%
% \section{Introduction}
%
% For our purposes here, ``theorems'' are labelled enunciations,
% often set off from the main text by extra space and a font change.
% Theorems, corollaries, conjectures, definitions, and remarks
% are all instances of ``theorems''.  The ``header'' of these structures
% is composed of a label (such as \textsc{Theorem} or \textsc{Remark})
% and a number which serializes an item in the sequence of items
% with the same label.
%
% Shortly after the introduction of \LaTeX{} at the Fachbereich
% Mathematik in Mainz, the desire to manipulate the layout of
% ``theorems'' arose.  In Mainz, the following two conventions came
% into general use:
% \begin{enumerate}
%   \item The number of the theorem is shown in the margin.
%   \item There is a line break at
%         the end of the theorem header.
% \end{enumerate}
%
% Additionally, some journals require different formats which depend on
% the ``sort of theorem'': e.g.\ often remarks and definitions are set
% in |\upshape|, while |\itshape| is employed for main theorems.
%
% Confronted with these requirements, a theorem environment was
% developed in Mainz which allows separate determination of the
% layout of the ``theorems sets'', comparable to |\pagestyle|.
%
%
% \section{The user interface}
% \subsection{Defining new theorem sets}
%
% \DescribeMacro\newtheorem
% As in the original \LaTeX{} version, the command |\newtheorem|
% defines a new ``theorem set'' or ``theorem-like structure''.
% Two required arguments name the new environment and give the text
% to be typeset with each instance of the new ``set'', while
% an optional argument determines how
% the ``set'' is enumerated:
% \begin{description}
%    \item[\ttfamily \bslash newtheorem\{foo\}\{bar\}]
%       The theorem set {\envfont foo} (whose name is \texttt{bar})
%       uses its own counter.
%    \item[\ttfamily \bslash newtheorem\{foo2\}{[foo]}\{bar2\}]
%       The theorem set {\envfont foo2} (printed name \texttt{bar2})
%       uses the same counter as the theorem set \texttt{foo}.
%    \item[\ttfamily \bslash newtheorem\{foo3\}\{bar3\}{[section]}]
%       The theorem set {\envfont foo3} (printed name \texttt{bar3}) is
%       enumerated within the counter \texttt{section}, i.e.\ with every
%       new |\section| the enumeration begins again with 1, and
%       the enumeration is composed from the section-number and the
%       theorem counter itself.
% \end{description}
%
% \DescribeMacro\theoremstyle
% Additionally, the command |\theoremstyle| can define the layout
% of various, or all, theorem sets.  It should be noted that any
% theorem set defined by |\newtheorem| is typeset in the
% |\theoremstyle| that is current at the time of the definition.
% Thus, the following
% \begin{verbatim}
%    \theoremstyle{break}        \newtheorem{Cor}{Corollary}
%    \theoremstyle{plain}        \newtheorem{Exa}{Example}[section]
%\end{verbatim}                             ^^A No blank at line start
% leads to the result that the set {\envfont Cor} is formatted in the
% style \texttt{break}, while the set {\envfont Exa} and all the
% following ones are formatted in the style \texttt{plain}, unless
% another |\theoremstyle| follows.  Since the definitions installed by
% |\newtheorem| are global, one also can limit |\theoremstyle| locally
% by grouping braces.
%
% \DescribeMacro\theorembodyfont
% The choice of the font for the theorem body is completely
% independent of the chosen |\theoremstyle|; this has proven to
% be very advantageous.  For example,
% \begin{verbatim}
%   {\theorembodyfont{\upshape}          \newtheorem{Rem}{Remark}}
%\end{verbatim}
% defines a theorem set {\envfont Rem}, which will be set in
% |\upshape| in the current layout (which in our example is
% \texttt{plain}).  As with |\theoremstyle|, the |\theorembodyfont|
% chosen is that current at the time of |\newtheorem|.  If
% |\theorembodyfont| is not specified or one defines
% |\theorembodyfont{}|, then the font used will be that defined by the
% |\theoremstyle|.
%
% \DescribeMacro\theoremheaderfont It is also possible to customize
% the font used for the theorem headers.  This is, however, a global
% declaration, and therefore there should be at most one
% |\theoremheaderfont| declaration in the preamble.\footnote{If it is
% actually necessary to have different header fonts, one has to define
% new theorem styles (substituting the desired font) or specify the
% information directly in the \texttt{\bslash newtheorem} declaration
% (the unclean variant).}
%
% \DescribeMacro\theorempreskipamount
% \DescribeMacro\theorempostskipamount Two additional parameters
% affect the vertical space around the theorem environments:
% |\theorempreskipamount| and |\theorempostskipamount| define,
% respectively, the spacing before and after such an environment.
% These parameters apply for all theorem sets and can be manipulated
% with the ordinary length macros.  They are rubber lengths
% (`\textsf{skips}'), and therefore can contain \texttt{plus} and
% \texttt{minus} parts.
%
% Since the definition of theorem sets should---most sensibly---be
% placed in the preamble, we only allow installation there.
% It is therefore possible to release the memory used here after
% |\begin{document}|, in order to make room for other
% applications.
%
%
% \subsection{Existing theorem styles}
%
% The following theorem styles exist to date:
% \begin{deflist}{marginbreak}
%    \item[plain]
%       This theorem style emulates the original \LaTeX{} definition,
%       except that additionally the parameters
%       |\theorem...skipamount| are used.
%    \item[break]
%       In this style, the theorem header is followed by a line
%       break.
%    \item[marginbreak]
%       The theorem number is set in the margin, and there is a line
%       break as in \texttt{break}.
%    \item[changebreak]
%       Like \texttt{break}, but with header number and text
%       interchanged.
%    \item[change]
%       Header number and text are interchanged, without a line break.
%    \item[margin]
%       The number is set in the left margin, without a line break.
% \end{deflist}
% All styles (except \texttt{plain}) select |\slshape| as the default
% |\theorembodyfont|.
%
%
% \subsection{Examples}
%
% Given the above theorem sets {\envfont Cor}, {\envfont Exa}
% and {\envfont Rem},
% suppose that the preamble also contains the declarations:
% \begin{verbatim}
%    \theoremstyle{marginbreak}   \newtheorem{Lem}[Cor]{Lemma}
%    \theoremstyle{change}
%    \theorembodyfont{\itshape}        \newtheorem{Def}[Cor]{Definition}
%
%    \theoremheaderfont{\scshape}
%\end{verbatim}
% Then the following are some typical examples of the typeset output
% resulting from their use.
%
% \begin{Cor}
%    This is a sentence typeset in the theorem environment {\envfont
%    Cor}.
% \end{Cor}
%
% \begin{Exa}
%    This is a sentence typeset in the theorem environment {\envfont
%    Exa}.
% \end{Exa}
%
% \begin{Rem}
%    This is a sentence typeset in the theorem environment {\envfont
%    Rem}.
% \end{Rem}
%
% \begin{Lem}[Ben User]
%    This is a sentence typeset in the theorem environment {\envfont
%    Lem}.
% \end{Lem}
%
% \begin{Def}[Very impressive Definition]
%    This is a sentence typeset in the theorem environment {\envfont
%    Def}.
% \end{Def}
%
% The last two examples show the effect of the optional argument to
% a theorem environment (it is the text typeset in parentheses).
%
%
% \section{Special Considerations}
% \label{sec:newfonts}
%
% Theoremheader and body are implemented as a unit. This means that
% the |\theoremheaderfont|  will inherit characteristics of the
% |\theorembodyfont| in \LaTeXe. Thus, if for example
% |\theorembodyfont| is |\itshape| and |\theoremheaderfont|
% is |\bfseries| the font selected for the header will have the
% characteristics `bold extended italic'. If this is not desired one
% should set the |\theoremheaderfont| to something like
% \begin{verbatim}
%      \theoremheaderfont{\normalfont\bfseries}
%\end{verbatim}
% i.e.\ supplying all necessary font information explicitly.
%
%
%
% \section{Acknowledgements}
%
% The publication of this set of macros was only possible with the
% help of Christina Busse (translating the manuscript into English),
% Joachim Pense (playing the r\^ole of typist), Chris Rowley
% (looking everything over) and many others providing useful
% suggestions.
%
% \MaybeStop{%
%    \begin{thebibliography}{9}
%       \bibitem{bk:GMS94} \textsc{M.~Goossens}, \textsc{F.~Mittelbach}
%         and \textsc{A.~Samarin}.
%         \newblock The \LaTeX{} Companion.
%         \newblock
%         Addison-Wesley, Reading, Massachusetts, 1994.
%       \bibitem{latex:source} \textsc{Lamport, Leslie.} \newblock
%                               \texttt{latex.tex}, version 2.09, date
%                               Feb. 1990.
%    \end{thebibliography}
%    \PrintIndex
%    \PrintChanges}
%
%
% \section{The documentation driver file}
%
% The next bit of code contains the documentation driver file for
% \TeX{}, i.e., the file that will produce the documentation you are
% currently reading. It will be extracted from this file by the
% \texttt{docstrip} program. Since it is the first code in the file
% one can alternatively process this file directly with \LaTeXe{} to
% obtain the documentation.
%
%    \begin{macrocode}
%<*driver>
\documentclass{ltxdoc}

\usepackage{theorem}

%<+driver>% The next few lines define theorem sets which are used
%<+driver>% in the example section of the documentation.

\theoremstyle{break}       \newtheorem{Cor}{Corollary}
\theoremstyle{plain}       \newtheorem{Exa}{Example}[section]
{\theorembodyfont{\upshape}\newtheorem{Rem}{Remark}}
\theoremstyle{marginbreak} \newtheorem{Lem}[Cor]{Lemma}
\theoremstyle{change}
\theorembodyfont{\itshape} \newtheorem{Def}[Cor]{Definition}

\theoremheaderfont{\scshape}

\RecordChanges

\begin{document}
   \DocInput{theorem.dtx}
\end{document}
%</driver>
%    \end{macrocode}
%
% \section{Definition of the Macros}
%
% If the file has been loaded before, we abort immediately. If not the
% package announces itself (this is actually done at the very top of
% the file---the way it is done isn't good style so don't copy it).
% \changes{v2.0e}{89/07/19}{Spaces removed from `typeout'.}
% \changes{v2.2b}{95/11/19}{Announce moved to top of file}
%    \begin{macrocode}
%<*package>
%\@ifundefined{theorem@style}{}{\endinput}
%\def\FMithmInfo{1995/11/19 v2.2b Theorem extension package (FMi)}
%\ProvidesPackage{theorem}[\FMithmInfo]
%    \end{macrocode}
%
%
% \subsection{Definition of theorem styles and fonts}
%
% All the definitions in this file are done globally to allow
% inputting this file inside a group.
%
% \begin{macro}{\theoremstyle}
%    Before a theorem style can be installed, the chosen style must be
%    known. For that reason, we must test to see that
%    |\th@|\meta{style} is known or, more precisely, that it is
%    different from |\relax|. If the style is not known then
%    |\th@plain| is used.
%    \begin{macrocode}
\gdef\theoremstyle#1{%
   \@ifundefined{th@#1}{\@warning
          {Unknown theoremstyle `#1'. Using `plain'}%
          \theorem@style{plain}}%
%    \end{macrocode}
%    We save the theorem style to be used in the token register
%    |\theorem@style|.
%    \begin{macrocode}
      {\theorem@style{#1}}%
%    \end{macrocode}
%    Now we ``evaluate'' the theorem style: this means, we call the
%    macro |\th@|\meta{style} which will activate the relevant
%    definitions which are contained in a separate file.  This is done
%    in a group to suppress changes to the current font.  This could
%    otherwise pose problems together with the new font selection
%    scheme\footnote{When I printed the original article using the new
%    font selection scheme I ended with a document with slanted
%    typefaces (text headings and all) simply because one of the
%    theorem styles used \texttt{\bslash sl} at toplevel.} if the
%    |\th@|\meta{style} is evaluated a second time.
% \changes{v2.1a}{90/02/15}{Group added around `th@..}
% \changes{v2.1a}{90/02/15}{Reseting of `..@begintheorem removed.}
%    \begin{macrocode}
      \begingroup
        \csname th@\the\theorem@style \endcsname
      \endgroup}
%    \end{macrocode}
% \end{macro}
%
%
% \begin{macro}{\@begintheorem}
% \changes{v2.1a}{90/02/15}{Relaxed at toplevel.}
% \begin{macro}{\@opargbegintheorem}
% \changes{v2.1a}{90/02/15}{Relaxed at toplevel.}
%    We reset |\@begintheorem| and |\@opargbegintheorem| to
%    |\relax| since these commands are no longer necessary at
%    toplevel. This will save a few tokens.
%    \begin{macrocode}
\global\let\@begintheorem\relax
\global\let\@opargbegintheorem\relax
%    \end{macrocode}
% \end{macro}
% \end{macro}
%
% \begin{macro}{\theorem@style}
%    Obviously the \textsf{token} register used above has to be
%    allocated. To assure the utmost compatibility with the
%    original \LaTeX{} definition, we set the default theorem style to
%    \texttt{plain}, which implements the usual \LaTeX{} convention.
% \changes{v2.0f}{89/07/20}{Unnecessary `global' removed.}
%    \begin{macrocode}
\newtoks\theorem@style
\global\theorem@style{plain}
%    \end{macrocode}
% \end{macro}
%
% \begin{macro}{\theorembodyfont}
% \begin{macro}{\theorem@bodyfont}
%    If the |\theorembodyfont| is set by the user then it should not
%    interact with the default font set in the theorem style. When the
%    new font selection is in force this may happen if, for example,
%    the default is |\itshape| and the new |\theorembodyfont| is
%    |\sffamily|. So we add a |\reset@font| command in front of the
%    user definition.
% \changes{v2.0f}{89/07/20}{Unnecessary `global' removed.}
% \changes{v2.0g}{89/09/19}{`theoremfont to `theorembodyfont.}
% \changes{v2.1c}{92/06/22}{`reset@font added and logic changed.}
% \changes{v2.1d}{92/10/28}{Reset to empty if arg is empty}
%    \begin{macrocode}
\gdef\theorembodyfont#1{%
%    \end{macrocode}
%    We check if the argument supplied is empty and if so put nothing
%    into the |\theorem@bodyfont| token register to allow for
%    |\theorembodyfont{}| as a mean of using the default of the
%    current |\theoremstyle|.
%    \begin{macrocode}
   \def\@tempa{#1}%
   \ifx\@tempa\@empty
    \theorem@bodyfont{}%
   \else
    \theorem@bodyfont{\reset@font#1}%
   \fi
   }
\newtoks\theorem@bodyfont
\global\theorem@bodyfont{}
%    \end{macrocode}
% \end{macro}
% \end{macro}
%
%
% \begin{macro}{\theoremheaderfont}
% \changes{v2.0g}{89/09/19}{Macro added.}
%    The font for the theorem headers is handled differently because
%    this definition applies to all theorem styles.
%    \begin{macrocode}
\gdef\theoremheaderfont#1{\gdef\theorem@headerfont{#1}%
%    \end{macrocode}
%    After using the macro once it is redefined to produce an error
%    message.
%    \begin{macrocode}
       \gdef\theoremheaderfont##1{%
        \typeout{\string\theoremheaderfont\space should be used
                 only once.}}}
%    \end{macrocode}
% \end{macro}
%
% \begin{macro}{\theorem@headerfont}
% \changes{v2.0g}{89/09/19}{Macro added.}
%    To set the |\theorem@headerfont| default we first test if
%    the new fontselection scheme is in force.
% \changes{v2.1a}{90/02/15}{Support for new font selection.}
%    \begin{macrocode}
\ifx\upshape\undefined
%    \end{macrocode}
%    If not we define it to expand into |\bfseries|. We don't use
%    |\let| just in case a following style option redefines this
%    macro.
%    \begin{macrocode}
\gdef\theorem@headerfont{\bfseries}
%    \end{macrocode}
%    Otherwise we reset the current shape before calling |\bfseries|.
%    \begin{macrocode}
\else \gdef\theorem@headerfont{\normalfont\bfseries}\fi
%    \end{macrocode}
% \end{macro}
%
% \begin{macro}{\th@plain}
% \begin{macro}{\th@break}
% \begin{macro}{\th@marginbreak}
% \begin{macro}{\th@changebreak}
% \begin{macro}{\th@change}
% \begin{macro}{\th@margin}
%    The different styles are defined in macros such as
%    |\th@plain|.  Since memory space is precious in
%    ``non-Big-versions'', we have to avoid offering too many unused
%    definitions.  Therefore we define these styles in separate files
%    that can be loaded on demand.  Thus the commands themselves only
%    load these files. We use |\@input@|, a \LaTeXe{} internal command
%    that ensures that the file will be listed with |\listfiles|.
% \changes{v2.2a}{94/02/02}{Use `@input@ to load theorem layout files}
%    \begin{macrocode}
\gdef\th@plain{\@input@{thp.sty}}
\gdef\th@break{\@input@{thb.sty}}
\gdef\th@marginbreak{\@input@{thmb.sty}}
\gdef\th@changebreak{\@input@{thcb.sty}}
\gdef\th@change{\@input@{thc.sty}}
\gdef\th@margin{\@input@{thm.sty}}
%    \end{macrocode}
%    This list will be expanded when new styles become
%    available.  For testing,  just append new theorem substyles
%    as document options.
% \end{macro}
% \end{macro}
% \end{macro}
% \end{macro}
% \end{macro}
% \end{macro}
%
%
%
% \subsection{Definition of a new theorem set}
%
% As already pointed out, a new theorem environment can be defined in
% three different ways:
% \begin{verbatim}
%    \newtheorem{Lem}{Lemma}
%    \newtheorem{Lem}{Lemma}[section]
%    \newtheorem{Lem}[Theorem]{Lemma}
%\end{verbatim}
% The function of the macro |\newtheorem| is to recognize these cases
% and then to branch into one of the three macros |\@ynthm|, |\@xnthm|
% or |\@othm|.  This mechanism is adopted unchanged from
% \cite{latex:source}; the essential point here is that, for example,
% in the second case, the arguments \texttt{Lem}, \texttt{Lemma} and
% \texttt{section} are passed over to the macro |\@xnthm|.
%
% We inspect this case first because the others present fewer problems,
% and thus are easily derived from this one.
%
% \begin{macro}{\@xnthm}
%    For our example arguments, the macro |\@xnthm| must fulfill
%    the following:
%    \begin{itemize}
%       \item
%          Define a new \LaTeX-counter `Lem'
%       \item
%          reset this counter within a |\section|
%       \item
%          define the macro |\theLem|
%       \item
%          define the environment macros |\Lem| and
%          |\endLem| using the current
%          |\theoremstyle| and |\theorem@bodyfont|.
%    \end{itemize}
%    Obviously, all this should happen only if the first argument
%    of |\@xnthm| (i.e.\ \texttt{Lem} in our example) is chosen so
%    as not to conflict with any previously defined commands or
%    environments.  This test is performed by the \LaTeX{} macro
%    |\@ifdefinable|.
%    \begin{macrocode}
\gdef\@xnthm#1#2[#3]{\expandafter\@ifdefinable\csname #1\endcsname
%    \end{macrocode}
%    Therefore, the first argument of |\@ifdefinable| is the
%    expansion (in the example, |\Lem|) of
%    |\csname||#1\endcsname|.  The second argument is
%    executed only if the test has been completed successfully.
%    \begin{macrocode}
   {%
%    \end{macrocode}
%    Now we define the new counter. The names of the \LaTeX{}
%    macros employed should speak for themselves:
% \changes{v2.2b}{1995/10/07}{Use \cs{@newctr}}
% \changes{v2.2c}{1995/11/23}{Use \cs{@newctr} properly.}
%    \begin{macrocode}
    \@definecounter{#1}\@newctr{#1}[#3]%
%    \end{macrocode}
%    Using |\@newctr| will give a proper error message if the counter
%    in |#3| is not defined.
%    In defining `|\theLem|' we must generate the desired macro
%    name by use of |\expandafter| and |\csname|.
%    \begin{macrocode}
    \expandafter\xdef\csname the#1\endcsname
%    \end{macrocode}
%    An |\xdef| is used in order to make the definition global,
%    and to ensure that it contains the replacement texts of
%    |\@thmcountersep| and |\@thmcounter|.\footnote{These
%    two macros can be defined by the document style.  Their default
%    values produce a `\texttt{.}' as separation and an arabic
%    representation of the number.} However, not everything should be
%    expanded. For example, it saves space to use |\thesection|
%    instead of its---at times---lengthy expansion.
%    \begin{macrocode}
      {\expandafter \noexpand \csname the#3\endcsname
       \@thmcountersep \@thmcounter{#1}}%
%    \end{macrocode}
%    Thus with the defaults of \LaTeX{}, |\theLem| would be
%    replaced by the command sequence |\thesection.\arabic{Lem}|.
%
%    We will now look at the definition of the macro which is executed
%    at the beginning of the actual environment (in our example this
%    macro is |\Lem|). It should be noted that we use an
%    ``|\expandafter| trick'' to expand only certain parts of the
%    replacement text at the time of the definition.
%    \begin{macrocode}
    \def\@tempa{\global\@namedef{#1}}%
    \expandafter \@tempa \expandafter{%
%    \end{macrocode}
%    First, the macro that contains the current definitions of
%    |\@begintheorem| and |\@opargtheorem| should be called
%    up.  The name of this macro---as is already known---has the form
%    |\th@|\meta{theorem\ style}; therefore, it must be called by
%    \begin{macrocode}
      \csname th@\the \theorem@style
%    \end{macrocode}
%    In addition the default theorem font should be changeable, i.e.\
%    we have to insert the contents of |\theorem@bodyfont|. For
%    that reason, we expand even further, beyond |\endcsname|,
%    and thus insert the contents of the \textsf{token} register
%    |\theorem@bodyfont| in the replacement text.
%    \begin{macrocode}
            \expandafter \endcsname \the \theorem@bodyfont
%    \end{macrocode}
%    Now it is time to call the macro |\@thm| which takes over
%    the further processing.  It has two arguments: the current
%    counter name (in our example, \texttt{Lem}), and the text of the
%    label (in our example, \texttt{Lemma}).
%    \begin{macrocode}
     \@thm{#1}{#2}}%
%    \end{macrocode}
%    With this, the `sub-definition' is complete.  The macro
%    |\@endtheorem| ends a theorem environment and is, so far, nothing
%    but an |\endtrivlist|.  (Hence it is defined globally, and not
%    within the theorem styles.\footnote{This has to be changed as
%    soon as theorem styles that change \texttt{\bslash @endtheorem}
%    exist.  In such a case, all existing styles must be changed as
%    well since they will have to reset the macro.}) Therefore, we can
%    set it equivalent to the macro that ends the theorem set (in our
%    example, |\endLem|).  However, if some day theorem styles exist
%    that do change |\@endtheorem|, we would have to use the
%    commented-out line instead.
%    \begin{macrocode}
    \global \expandafter \let \csname end#1\endcsname \@endtheorem
%   \global\@namedef{end#1}{\@endtheorem}%
%    \end{macrocode}
%    With these commands all the required definitions are employed,
%    unless the test |\@ifdefinable| has failed.  Therefore, we
%    end the second argument of this macro and with it the definition
%    of |\@xnthm|.
%    \begin{macrocode}
   }}
%    \end{macrocode}
% \end{macro}
%
% \begin{macro}{\@ynthm}
%    The definition of |\@ynthm| is completely analogous. In
%    this case the new counter that is defined is not reset within
%    another counter; thus the definition of |\the...|
%    is simplified:
%    \begin{macrocode}
\gdef\@ynthm#1#2{\expandafter\@ifdefinable\csname #1\endcsname
   {\@definecounter{#1}%
    \expandafter\xdef\csname the#1\endcsname{\@thmcounter{#1}}%
%    \end{macrocode}
%    The rest of the definition corresponds literally to that of
%    |\@xnthm|:
%    \begin{macrocode}
    \def\@tempa{\global\@namedef{#1}}\expandafter \@tempa
     \expandafter{\csname th@\the \theorem@style \expandafter
     \endcsname \the\theorem@bodyfont \@thm{#1}{#2}}%
    \global \expandafter \let \csname end#1\endcsname \@endtheorem}}
%    \end{macrocode}
% \end{macro}
%
% \begin{macro}{\@othm}
%    The definition of |\@othm| does not contain anything new.
%    \begin{macrocode}
\gdef\@othm#1[#2]#3{%
%    \end{macrocode}
% \changes{v2.2b}{95/10/07}{Build in check for counter defined}
%    We do not define a new counter but instead use one that has
%    already been defined.  Thus the only definition we need is that
%    of this pseudo-counter (i.e.\ |\the|\meta{env.\ name}).
%    First we check if |#2| corresponds to a known counter name.
%    \begin{macrocode}
  \expandafter\ifx\csname c@#2\endcsname\relax
   \@nocounterr{#2}%
  \else
   \expandafter\@ifdefinable\csname #1\endcsname
   {\expandafter \xdef \csname the#1\endcsname
     {\expandafter \noexpand \csname the#2\endcsname}%
%    \end{macrocode}
%    All other parts of the definition can be adopted from
%    |\@xnthm|. We have to remember, though, that in this case
%    the name of the current counter and the theorem label have moved
%    to the second and third arguments.
%    \begin{macrocode}
    \def\@tempa{\global\@namedef{#1}}\expandafter \@tempa
     \expandafter{\csname th@\the \theorem@style \expandafter
     \endcsname \the\theorem@bodyfont \@thm{#2}{#3}}%
    \global \expandafter \let \csname end#1\endcsname \@endtheorem}%
  \fi}
%    \end{macrocode}
% \end{macro}
%
% \subsection{Macros that are employed in a {\envfont theorem}
%    environment}
%
% \begin{macro}{\@thm}
%    The macro |\@thm| has to increase the current counter.  Then,
%    depending on whether the environment has (or does not have) an
%    optional argument, it has to branch into either |\@begintheorem|
%    or |\@opargtheorem|.
%    \begin{macrocode}
\gdef\@thm#1#2{\refstepcounter{#1}%
%    \end{macrocode}
%    Now we start a {\envfont trivlist} environment, and give
%    |\@topsep| and |\@topsepadd| the values of the \textsf{skip}
%    registers |\theorempreskipamount| and |\theorempostskipamount|.
%    The value in |\@topsep| is the vertical space that is inserted by
%    the first (and only) |\item| in our |\trivlist| whilst
%    |\@topsepadd| is inserted by |\@endparenv| at the end of that
%    {\envfont trivlist} environment.  By using these registers, we
%    obtain the desired space around a {\envfont theorem} environment.
%    \begin{macrocode}
   \trivlist
   \@topsep \theorempreskipamount               % used by first \item
   \@topsepadd \theorempostskipamount           % used by \@endparenv
%    \end{macrocode}
%    Now we have to test whether an optional argument has been given.
% \changes{v2.0d}{89/07/18}{Forgotten percent added.}
%    \begin{macrocode}
   \@ifnextchar [%
%    \end{macrocode}
%    If there is an optional argument, we will call |\@ythm|, and
%    move the arguments read back into the input stream.
%    \begin{macrocode}
   {\@ythm{#1}{#2}}%
%    \end{macrocode}
%    If not, we call |\@begintheorem|. Its first argument is the
%    name of the theorem set (hence the second argument of
%    |\@thm|). Its second argument is the macro that produces the
%    current number.
%    \begin{macrocode}
   {\@begintheorem{#2}{\csname the#1\endcsname}\ignorespaces}}
%    \end{macrocode}
% \end{macro}
%
% \begin{macro}{\@xthm}
% \begin{macro}{\@ythm}
%    Both these macros were originally called by |\@thm|.  We do
%    not need |\@xthm| anymore, hence we reset it to
%    |\relax|. The definition of |\@ythm| has not changed at
%    all from its definition in \LaTeX. In order to make the macros
%    easier to understand, we will
%    nevertheless present it (commented out).
%    \begin{macrocode}
\global\let\@xthm\relax
% \def\@ythm#1#2[#3]{\@opargbegintheorem{#2}{\csname
%           the#1\endcsname}{#3}\ignorespaces}
%</package>
%    \end{macrocode}
%    The primitive |\ignorespaces| in |\@ythm| and
%    |\@thm| is needed to remove the spaces between the
%    |\begin{...}| and the actual text.
% \end{macro}
% \end{macro}
%
%
% \subsection{Definition of the theorem substyles}
%
% As already pointed out, the theorem substyles, defined below, are
% only loaded when necessary.  Note that all these substyles, except
% \texttt{plain}, have |\slshape| as the default body font.
%
% \subsubsection{The \texttt{plain} style}
%
% \label{dok:th@plain} As the following macros use |@|, we have
% to locally set the |\catcode| of this symbol to ``letter''.
% This happens within a group, so that we do not have to worry about
% which |\catcode| that symbol had before.
%    \begin{macrocode}
%<*thp>
\begingroup \makeatletter
%    \end{macrocode}
% Since we are now within a group, we must make all definitions
% globally.  First we make sure that \texttt{theorem.sty} is loaded.
% This will allow us to use this file as a document style option
% without having to call \texttt{theorem} itself as an option.  At the
% same time, we assure that at least version 2 is loaded, since
% |\theorem@style| was not defined in earlier versions.
%    \begin{macrocode}
\@ifundefined{theorem@style}{\input{theorem.sty}}{}
\ProvidesFile{thp.sty}
             [\FMithmInfo]
%    \end{macrocode}
%
%
% \begin{macro}{\th@plain}
% \begin{macro}{\@begintheorem}
% \begin{macro}{\@opargbegintheorem}
%    |\theoremstyle{plain}| corresponds to the original
%    definition, except that the distances to the surrounding text are
%    determined by the parameters |\theorempreskipamount| and
%    |\theorempostskipamount|.
%    First we set the default body font.
%    \begin{macrocode}
\gdef\th@plain{\normalfont\itshape
%    \end{macrocode}
%    Then we define |\@begintheorem| and
%    |\@opargbegintheorem|. These two macros define how the
%    header of a theorem is typeset.  |\@opargbegintheorem| will
%    be called if a \textsf{theorem} environment with an optional
%    argument is encountered; otherwise, the header is constructed by
%    calling |\@begintheorem|.  If one of these macros is
%    executed, we are within a \textsf{trivlist} environment started by
%    |\@thm|.  So the theorem header is produced with an
%    |\item| command.
%
%    Instead of specifying the header font directly, all standard
%    theorem styles use the |\theorem@headerfont| macro to allow
%    customization.  The extra space (|\labelsep|) is necessary
%    because of problems in the \textsf{trivlist} environment.
%    \begin{macrocode}
  \def\@begintheorem##1##2{%
        \item[\hskip\labelsep \theorem@headerfont ##1\ ##2]}%
%    \end{macrocode}
%    The definition of |\@opargbegintheorem| is completely
%    analogous. The only difference is the fact that there exists a
%    third argument (which is the optional parameter of the
%    environment and contains additional information about the
%    theorem).  Customarily we enclose it in parentheses.
%    \begin{macrocode}
\def\@opargbegintheorem##1##2##3{%
   \item[\hskip\labelsep \theorem@headerfont ##1\ ##2\ (##3)]}}
%    \end{macrocode}
% \end{macro}
% \end{macro}
% \end{macro}
%
% \noindent
% We conclude with an |\endgroup| to restore the
% |\catcode| of |@|.
%    \begin{macrocode}
\endgroup
%</thp>
%    \end{macrocode}
%
%
% \subsubsection{The \texttt{break} style}
%
% This style option is stored in the file |thb.sty|.
% For the next two lines see the documentation for |\th@plain| on
% page~\pageref{dok:th@plain}.
%    \begin{macrocode}
%<*thb>
\begingroup \makeatletter
\@ifundefined{theorem@style}{\input{theorem.sty}}{}
\ProvidesFile{thb.sty}
             [\FMithmInfo]
%    \end{macrocode}
%
% \begin{macro}{\th@break}
%    |\theoremstyle{break}| produces a line break after the name
%    of the theorem.  The font is |\slshape|.  Hence, we define
%    |\th@break| as follows:
%    \begin{macrocode}
\gdef\th@break{\normalfont\slshape
  \def\@begintheorem##1##2{\item[%
%    \end{macrocode}
%    We run into the following problem: it is not possible to create
%    the header with |\item[|\meta{title}|]| and then start
%    a new line by, for example, |\mbox{}\\|. Such a definition
%    will fail whenever a list environment follows immediately.  With
%    the above construction, the |\mbox{}| causes the switch
%    |@inlabel| (cf.\ definition of |\list| and
%    |\trivlist| in \cite{latex:source}) to be set to \texttt{false}
%    and so the following list will insert additional vertical
%    space (|\topskip|). This is quite annoying. Therefore, we
%    create the line break within the |\item|. In order to ensure
%    that the text will begin at the proper position in the following
%    line, we simply pretend that the label does not take any
%    room.\footnote{This will lead to problems whenever very high
%    symbols occurring in the line tower into the
%    heading. So, something else has to be done here sometime.}
%    \begin{macrocode}
       \rlap{\vbox{\hbox{\hskip \labelsep\theorem@headerfont ##1\ ##2}%
                   \hbox{\strut}}}]}%
%    \end{macrocode}
%    Again, the definition of |\@opargbegintheorem| is
%    completely analogous.
%    \begin{macrocode}
\def\@opargbegintheorem##1##2##3{%
  \item[\rlap{\vbox{\hbox{\hskip \labelsep \theorem@headerfont
                     ##1\ ##2\ (##3)}%
                    \hbox{\strut}}}]}}
\endgroup
%</thb>
%    \end{macrocode}
% \end{macro}
%
%
% \subsubsection{The \texttt{changebreak} style}
%
%    \begin{macrocode}
% This style option is stored in the file |thcb.sty|.
%    \begin{macrocode}
%<*thcb>
\begingroup \makeatletter
\@ifundefined{theorem@style}{\input{theorem.sty}}{}
\ProvidesFile{thcb.sty}
             [\FMithmInfo]
%    \end{macrocode}
%
% \begin{macro}{\th@changebreak}
%    The \texttt{change-break} theorem style is like \texttt{break}
%    but with interchange of theorem name and theorem number.  Thus we
%    define |\th@changebreak| as follows:
%    \begin{macrocode}
\gdef\th@changebreak{\normalfont\slshape
  \def\@begintheorem##1##2{\item
       [\rlap{\vbox{\hbox{\hskip\labelsep \theorem@headerfont ##2\ ##1}%
                    \hbox{\strut}}}]}%
\def\@opargbegintheorem##1##2##3{%
  \item[\rlap{\vbox{\hbox{\hskip\labelsep \theorem@headerfont
                     ##2\ ##1\ (##3)}%
                    \hbox{\strut}}}]}}
\endgroup
%</thcb>
%    \end{macrocode}
% \end{macro}
%
%
% \subsubsection{The \texttt{change} style}
%
% This style option is stored in the file |thc.sty|.
%    \begin{macrocode}
%<*thc>
\begingroup \makeatletter
\@ifundefined{theorem@style}{\input{theorem.sty}}{}
\ProvidesFile{thc.sty}
             [\FMithmInfo]
%    \end{macrocode}
%
% \begin{macro}{\th@change}
%    The \texttt{change} theorem style corresponds to the
%    \texttt{change break} style without a linebreak after the header.
%    To say it in another way, it's the same as the \texttt{plain}
%    style but with number and name interchanged and |\slshape| as the
%    default font.
%    \begin{macrocode}
\gdef\th@change{\normalfont\slshape
  \def\@begintheorem##1##2{\item
       [\hskip\labelsep \theorem@headerfont ##2\ ##1]}%
\def\@opargbegintheorem##1##2##3{%
  \item[\hskip\labelsep \theorem@headerfont ##2\ ##1\ (##3)]}}
\endgroup
%</thc>
%    \end{macrocode}
% \end{macro}
%
%
% \subsubsection{The \texttt{marginbreak} style}
%
% This style option is the one used most often at Mainz.  It is saved
% in the file |thmb.sty|.
%    \begin{macrocode}
%<*thmb>
\begingroup \makeatletter
\@ifundefined{theorem@style}{\input{theorem.sty}}{}
\ProvidesFile{thmb.sty}
             [\FMithmInfo]
%    \end{macrocode}
%
% \begin{macro}{\th@marginbreak}
%    The \texttt{margin break} style is nearly the same as the
%    \texttt{change break} style. The only difference is the placement
%    of the theorem number.  We use |\llap| to place it in the left
%    margin.
%
%    In this style |\labelsep| denotes the separation between the
%    number and the text.
%    \begin{macrocode}
\gdef\th@marginbreak{\normalfont\slshape
  \def\@begintheorem##1##2{\item
       [\rlap{\vbox{\theorem@headerfont
                    \hbox{\llap{##2}\hskip\labelsep ##1}%
                    \hbox{\strut}}}]}%
\def\@opargbegintheorem##1##2##3{%
  \item[\rlap{\vbox{\theorem@headerfont
                    \hbox{\llap{##2}\hskip\labelsep ##1\ (##3)}%
                    \hbox{\strut}}}]}}
\endgroup
%</thmb>
%    \end{macrocode}
% \changes{v2.1c}{90/02/17}{`endgroup outside macrocode env.}
% \end{macro}
%
%
%
% \subsubsection{The \texttt{margin} style}
%
% This style option is stored in the file |thm.sty|.
%    \begin{macrocode}
%<*thm>
\begingroup \makeatletter
\@ifundefined{theorem@style}{\input{theorem.sty}}{}
\ProvidesFile{thm.sty}
             [\FMithmInfo]
%    \end{macrocode}
%
% \begin{macro}{\th@margin}
%    Again this is only a variant of the theorem styles described
%    above without any new ideas.
%    \begin{macrocode}
\gdef\th@margin{\normalfont\slshape
  \def\@begintheorem##1##2{\item
       [\theorem@headerfont \llap{##2}\hskip\labelsep ##1]}%
\def\@opargbegintheorem##1##2##3{%
  \item[\theorem@headerfont \llap{##2}\hskip\labelsep ##1\ (##3)]}}
\endgroup
%</thm>
%    \end{macrocode}
% \end{macro}
%
%
%
%
% \subsection{Final Definitions}
%
% \begin{macro}{\theorempreskipamount}
% \begin{macro}{\theorempostskipamount}
%    The \textsf{skip} parameters that regulate the vertical empty space
%    before and after the {\envfont theorem} environment have
%    to be allocated as well.
% \changes{v2.0f}{89/07/20}{Unnecessary `global' removed.}
%    \begin{macrocode}
%<*package>
\newskip\theorempreskipamount
\newskip\theorempostskipamount
%    \end{macrocode}
%    Since we have used the same values for all theorem sets, we now can
%    assign them.
%    \begin{macrocode}
\global\setlength\theorempreskipamount{12pt plus 5pt minus 3pt}
\global\setlength\theorempostskipamount{8pt plus 3pt minus 1.5pt}
%    \end{macrocode}
% \end{macro}
% \end{macro}
%
%
% \begin{macro}{\@endtheorem}
%    The same holds for the macro |\@endtheorem|, which ends a
%    {\envfont theorem} environment. Since it is the same for all
%    theorem sets, it is removed from the macros |\th@|\meta{style}.
%    It simply ends the {\envfont trivlist} environment, which was
%    begun in |\@thm|.
%    \begin{macrocode}
\global\let\@endtheorem=\endtrivlist
%    \end{macrocode}
% \end{macro}
%
%
%
% \begin{macro}{\@preamblecmds}
%    All macros defined above are to be used only in the preamble.
%    Therefore, we insert them in |\@preamblecmds| which will disable
%    them at begin document. This is done by the internal \LaTeXe{}
%    command |\@onlypreamble|.
% \changes{v2.0g}{89/09/19}{`theoremheaderfont and `theorembodyfont
%    added.}
% \changes{v2.2a}{94/02/03}{Use \LaTeXe{} interface}
%    \begin{macrocode}
\@onlypreamble\@xnthm
\@onlypreamble\@ynthm
\@onlypreamble\@othm
\@onlypreamble\newtheorem
\@onlypreamble\theoremstyle
\@onlypreamble\theorembodyfont
\@onlypreamble\theoremheaderfont
%    \end{macrocode}
% \end{macro}
%
% Finally we declare the \texttt{plain} theorem style to be the default.
% \changes{v2.1b}{90/06/04}{Fixed missing `item by declaring default
%    style.}
%    \begin{macrocode}
\theoremstyle{plain}
%</package>
%    \end{macrocode}
%
% \Finale

\endinput
}{}
\ProvidesFile{thcb.sty}
             [\FMithmInfo]
%    \end{macrocode}
%
% \begin{macro}{\th@changebreak}
%    The \texttt{change-break} theorem style is like \texttt{break}
%    but with interchange of theorem name and theorem number.  Thus we
%    define |\th@changebreak| as follows:
%    \begin{macrocode}
\gdef\th@changebreak{\normalfont\slshape
  \def\@begintheorem##1##2{\item
       [\rlap{\vbox{\hbox{\hskip\labelsep \theorem@headerfont ##2\ ##1}%
                    \hbox{\strut}}}]}%
\def\@opargbegintheorem##1##2##3{%
  \item[\rlap{\vbox{\hbox{\hskip\labelsep \theorem@headerfont
                     ##2\ ##1\ (##3)}%
                    \hbox{\strut}}}]}}
\endgroup
%</thcb>
%    \end{macrocode}
% \end{macro}
%
%
% \subsubsection{The \texttt{change} style}
%
% This style option is stored in the file |thc.sty|.
%    \begin{macrocode}
%<*thc>
\begingroup \makeatletter
\@ifundefined{theorem@style}{% \iffalse meta-comment
%
% Copyright (C) 1993-2024
%
% The LaTeX Project and any individual authors listed elsewhere
% in this file.
%
% This file is part of the Standard LaTeX `Tools Bundle'.
% -------------------------------------------------------
%
% It may be distributed and/or modified under the
% conditions of the LaTeX Project Public License, either version 1.3c
% of this license or (at your option) any later version.
% The latest version of this license is in
%    https://www.latex-project.org/lppl.txt
% and version 1.3c or later is part of all distributions of LaTeX
% version 2005/12/01 or later.
%
% The list of all files belonging to the LaTeX `Tools Bundle' is
% given in the file `manifest.txt'.
%
% \fi
%
%%
%
%
\def\FMithmInfo{2023/07/05 v2.2c Theorem extension package (FMi)}
%
% \ProvidesFile{theorem.dtx}[\FMithmInfo]
%\iffalse   % this is a METACOMMENT !
%%
%% Package `theorem' to use with LaTeX2e
%% Copyright (C) 1989-1995 Frank Mittelbach, all rights reserved.
%<+package>\@ifundefined{theorem@style}{}{\endinput}
%<+package>\NeedsTeXFormat{LaTeX2e}[1995/06/01]
%<+package>\ProvidesPackage{theorem}
%<+package>         [\FMithmInfo]
%
% \fi
%
% \changes{v1.0}{87/05/07}{First draft.}
% \changes{v1.0a}{87/05/07}{`thistheoremstyle' added.}
% \changes{v1.1a}{87/07/29}{New concept: redefine the internal
%                          \protect\LaTeX{} commands `@xnthm' etc.}
% \changes{v1.1b}{87/07/31}{`theoremfont' used to specify font.}
% \changes{v1.2a}{88/01/08}{`@endtheorem' now globally defined.}
% \changes{v1.2b}{88/01/11}{`newtheorem' only allowed in the preamble.}
% \changes{v1.3a}{88/02/14}{`theoremfont' now a toks register.}
%
% \changes{v2.0a}{88/11/16}{New implementation and new documentation.}
% \changes{v2.0a}{88/11/16}{No layout changes but internal modifications
%                           old substyles won't work.}
% \changes{v2.0b}{89/01/30}{`global' used for all definitions.}
% \changes{v2.0c}{89/06/15}{Documentation translated to English.
%                            Thanks to LLL and Nina}
% \changes{v2.1a}{90/02/15}{Support for new font selection.}
% \changes{v2.2a}{94/02/03}{Cleanup for LaTeX2e}
%
% \DoNotIndex{\@endparenv,\@ifdefinable,\@ifnextchar,\@ifundefined}
% \DoNotIndex{\@namedef,\@nameuse,\@spaces,\@tempa,\@warning}
% \DoNotIndex{\ }
% \DoNotIndex{\begingroup,\bfseries}
% \DoNotIndex{\csname}
% \DoNotIndex{\def,\do,\docdate}
% \DoNotIndex{\endcsname,\endgroup,\endinput,\endtrivlist,\expandafter}
% \DoNotIndex{\filedate,\fileversion}
% \DoNotIndex{\gdef,\global}
% \DoNotIndex{\hbox,\hskip}
% \DoNotIndex{\ignorespaces,\input,\itshape,\item}
% \DoNotIndex{\labelsep,\let,\llap}
% \DoNotIndex{\makeatletter}
% \DoNotIndex{\noexpand}
% \DoNotIndex{\refstepcounter,\relax,\rlap}
% \DoNotIndex{\setlength,\slshape,\space,\string,\strut}
% \DoNotIndex{\the,\trivlist,\typeout}
% \DoNotIndex{\vbox}
% \DoNotIndex{\xdef}
%
% \def\deflabel#1{\ttfamily #1\hfill}
% \def\deflist#1{\begin{list}{}{\settowidth\labelwidth{\ttfamily #1}%
%                               \setlength\leftmargin\labelwidth
%                               \addtolength\leftmargin\labelsep
%                               \let\makelabel\deflabel}}
% \def\enddeflist{\end{list}}
% \def\envfont{\normalfont\ttfamily}
%
% \MakeShortVerb{\|}
%
% \newcommand{\TUB}{{\sl TUGboat\/}}
%
% \GetFileInfo{theorem.dtx}
% \title{An Extension of the
%        \LaTeX{} theorem environment\thanks{This file
%        has version number \fileversion, last
%        revised \filedate.}}
% \author{Frank Mittelbach}
%
% \MaintainedByLaTeXTeam{tools}
% \maketitle
%
% \begin{abstract}
%    The macros described in this paper yield an extension of the
%    \LaTeX{} theorem mechanism. It is designed to satisfy the
%    different requirements of various journals. Thus, the
%    layout of the ``theorems'' can be manipulated by determining a
%    ``style''. This article describes not only the use, but
%    also the definition, of the necessary macros.
% \end{abstract}
%
% \section*{Preface to version 2.2}
%
% For \LaTeXe{} this package did not need any fundamental changes. I
% only modified the messages generated so that theorem layout styles
% will show up with the |\listfiles| command and cleaned the section
% on the New Font Selection Scheme since this is now included in
% \LaTeX.
%
% \section*{Preface to version 2.1}
%
% This version is identical to 2.0g described in \TUB\ 10\#3 except
% for some internal defaults which are now set depending on the used
% font selection scheme.
%
% This was done to avoid unpleasant surprises if the new font selection
% scheme is in force. For further details see
% section~\ref{sec:newfonts} and \cite{bk:GMS94}.
%
% \section{Introduction}
%
% For our purposes here, ``theorems'' are labelled enunciations,
% often set off from the main text by extra space and a font change.
% Theorems, corollaries, conjectures, definitions, and remarks
% are all instances of ``theorems''.  The ``header'' of these structures
% is composed of a label (such as \textsc{Theorem} or \textsc{Remark})
% and a number which serializes an item in the sequence of items
% with the same label.
%
% Shortly after the introduction of \LaTeX{} at the Fachbereich
% Mathematik in Mainz, the desire to manipulate the layout of
% ``theorems'' arose.  In Mainz, the following two conventions came
% into general use:
% \begin{enumerate}
%   \item The number of the theorem is shown in the margin.
%   \item There is a line break at
%         the end of the theorem header.
% \end{enumerate}
%
% Additionally, some journals require different formats which depend on
% the ``sort of theorem'': e.g.\ often remarks and definitions are set
% in |\upshape|, while |\itshape| is employed for main theorems.
%
% Confronted with these requirements, a theorem environment was
% developed in Mainz which allows separate determination of the
% layout of the ``theorems sets'', comparable to |\pagestyle|.
%
%
% \section{The user interface}
% \subsection{Defining new theorem sets}
%
% \DescribeMacro\newtheorem
% As in the original \LaTeX{} version, the command |\newtheorem|
% defines a new ``theorem set'' or ``theorem-like structure''.
% Two required arguments name the new environment and give the text
% to be typeset with each instance of the new ``set'', while
% an optional argument determines how
% the ``set'' is enumerated:
% \begin{description}
%    \item[\ttfamily \bslash newtheorem\{foo\}\{bar\}]
%       The theorem set {\envfont foo} (whose name is \texttt{bar})
%       uses its own counter.
%    \item[\ttfamily \bslash newtheorem\{foo2\}{[foo]}\{bar2\}]
%       The theorem set {\envfont foo2} (printed name \texttt{bar2})
%       uses the same counter as the theorem set \texttt{foo}.
%    \item[\ttfamily \bslash newtheorem\{foo3\}\{bar3\}{[section]}]
%       The theorem set {\envfont foo3} (printed name \texttt{bar3}) is
%       enumerated within the counter \texttt{section}, i.e.\ with every
%       new |\section| the enumeration begins again with 1, and
%       the enumeration is composed from the section-number and the
%       theorem counter itself.
% \end{description}
%
% \DescribeMacro\theoremstyle
% Additionally, the command |\theoremstyle| can define the layout
% of various, or all, theorem sets.  It should be noted that any
% theorem set defined by |\newtheorem| is typeset in the
% |\theoremstyle| that is current at the time of the definition.
% Thus, the following
% \begin{verbatim}
%    \theoremstyle{break}        \newtheorem{Cor}{Corollary}
%    \theoremstyle{plain}        \newtheorem{Exa}{Example}[section]
%\end{verbatim}                             ^^A No blank at line start
% leads to the result that the set {\envfont Cor} is formatted in the
% style \texttt{break}, while the set {\envfont Exa} and all the
% following ones are formatted in the style \texttt{plain}, unless
% another |\theoremstyle| follows.  Since the definitions installed by
% |\newtheorem| are global, one also can limit |\theoremstyle| locally
% by grouping braces.
%
% \DescribeMacro\theorembodyfont
% The choice of the font for the theorem body is completely
% independent of the chosen |\theoremstyle|; this has proven to
% be very advantageous.  For example,
% \begin{verbatim}
%   {\theorembodyfont{\upshape}          \newtheorem{Rem}{Remark}}
%\end{verbatim}
% defines a theorem set {\envfont Rem}, which will be set in
% |\upshape| in the current layout (which in our example is
% \texttt{plain}).  As with |\theoremstyle|, the |\theorembodyfont|
% chosen is that current at the time of |\newtheorem|.  If
% |\theorembodyfont| is not specified or one defines
% |\theorembodyfont{}|, then the font used will be that defined by the
% |\theoremstyle|.
%
% \DescribeMacro\theoremheaderfont It is also possible to customize
% the font used for the theorem headers.  This is, however, a global
% declaration, and therefore there should be at most one
% |\theoremheaderfont| declaration in the preamble.\footnote{If it is
% actually necessary to have different header fonts, one has to define
% new theorem styles (substituting the desired font) or specify the
% information directly in the \texttt{\bslash newtheorem} declaration
% (the unclean variant).}
%
% \DescribeMacro\theorempreskipamount
% \DescribeMacro\theorempostskipamount Two additional parameters
% affect the vertical space around the theorem environments:
% |\theorempreskipamount| and |\theorempostskipamount| define,
% respectively, the spacing before and after such an environment.
% These parameters apply for all theorem sets and can be manipulated
% with the ordinary length macros.  They are rubber lengths
% (`\textsf{skips}'), and therefore can contain \texttt{plus} and
% \texttt{minus} parts.
%
% Since the definition of theorem sets should---most sensibly---be
% placed in the preamble, we only allow installation there.
% It is therefore possible to release the memory used here after
% |\begin{document}|, in order to make room for other
% applications.
%
%
% \subsection{Existing theorem styles}
%
% The following theorem styles exist to date:
% \begin{deflist}{marginbreak}
%    \item[plain]
%       This theorem style emulates the original \LaTeX{} definition,
%       except that additionally the parameters
%       |\theorem...skipamount| are used.
%    \item[break]
%       In this style, the theorem header is followed by a line
%       break.
%    \item[marginbreak]
%       The theorem number is set in the margin, and there is a line
%       break as in \texttt{break}.
%    \item[changebreak]
%       Like \texttt{break}, but with header number and text
%       interchanged.
%    \item[change]
%       Header number and text are interchanged, without a line break.
%    \item[margin]
%       The number is set in the left margin, without a line break.
% \end{deflist}
% All styles (except \texttt{plain}) select |\slshape| as the default
% |\theorembodyfont|.
%
%
% \subsection{Examples}
%
% Given the above theorem sets {\envfont Cor}, {\envfont Exa}
% and {\envfont Rem},
% suppose that the preamble also contains the declarations:
% \begin{verbatim}
%    \theoremstyle{marginbreak}   \newtheorem{Lem}[Cor]{Lemma}
%    \theoremstyle{change}
%    \theorembodyfont{\itshape}        \newtheorem{Def}[Cor]{Definition}
%
%    \theoremheaderfont{\scshape}
%\end{verbatim}
% Then the following are some typical examples of the typeset output
% resulting from their use.
%
% \begin{Cor}
%    This is a sentence typeset in the theorem environment {\envfont
%    Cor}.
% \end{Cor}
%
% \begin{Exa}
%    This is a sentence typeset in the theorem environment {\envfont
%    Exa}.
% \end{Exa}
%
% \begin{Rem}
%    This is a sentence typeset in the theorem environment {\envfont
%    Rem}.
% \end{Rem}
%
% \begin{Lem}[Ben User]
%    This is a sentence typeset in the theorem environment {\envfont
%    Lem}.
% \end{Lem}
%
% \begin{Def}[Very impressive Definition]
%    This is a sentence typeset in the theorem environment {\envfont
%    Def}.
% \end{Def}
%
% The last two examples show the effect of the optional argument to
% a theorem environment (it is the text typeset in parentheses).
%
%
% \section{Special Considerations}
% \label{sec:newfonts}
%
% Theoremheader and body are implemented as a unit. This means that
% the |\theoremheaderfont|  will inherit characteristics of the
% |\theorembodyfont| in \LaTeXe. Thus, if for example
% |\theorembodyfont| is |\itshape| and |\theoremheaderfont|
% is |\bfseries| the font selected for the header will have the
% characteristics `bold extended italic'. If this is not desired one
% should set the |\theoremheaderfont| to something like
% \begin{verbatim}
%      \theoremheaderfont{\normalfont\bfseries}
%\end{verbatim}
% i.e.\ supplying all necessary font information explicitly.
%
%
%
% \section{Acknowledgements}
%
% The publication of this set of macros was only possible with the
% help of Christina Busse (translating the manuscript into English),
% Joachim Pense (playing the r\^ole of typist), Chris Rowley
% (looking everything over) and many others providing useful
% suggestions.
%
% \MaybeStop{%
%    \begin{thebibliography}{9}
%       \bibitem{bk:GMS94} \textsc{M.~Goossens}, \textsc{F.~Mittelbach}
%         and \textsc{A.~Samarin}.
%         \newblock The \LaTeX{} Companion.
%         \newblock
%         Addison-Wesley, Reading, Massachusetts, 1994.
%       \bibitem{latex:source} \textsc{Lamport, Leslie.} \newblock
%                               \texttt{latex.tex}, version 2.09, date
%                               Feb. 1990.
%    \end{thebibliography}
%    \PrintIndex
%    \PrintChanges}
%
%
% \section{The documentation driver file}
%
% The next bit of code contains the documentation driver file for
% \TeX{}, i.e., the file that will produce the documentation you are
% currently reading. It will be extracted from this file by the
% \texttt{docstrip} program. Since it is the first code in the file
% one can alternatively process this file directly with \LaTeXe{} to
% obtain the documentation.
%
%    \begin{macrocode}
%<*driver>
\documentclass{ltxdoc}

\usepackage{theorem}

%<+driver>% The next few lines define theorem sets which are used
%<+driver>% in the example section of the documentation.

\theoremstyle{break}       \newtheorem{Cor}{Corollary}
\theoremstyle{plain}       \newtheorem{Exa}{Example}[section]
{\theorembodyfont{\upshape}\newtheorem{Rem}{Remark}}
\theoremstyle{marginbreak} \newtheorem{Lem}[Cor]{Lemma}
\theoremstyle{change}
\theorembodyfont{\itshape} \newtheorem{Def}[Cor]{Definition}

\theoremheaderfont{\scshape}

\RecordChanges

\begin{document}
   \DocInput{theorem.dtx}
\end{document}
%</driver>
%    \end{macrocode}
%
% \section{Definition of the Macros}
%
% If the file has been loaded before, we abort immediately. If not the
% package announces itself (this is actually done at the very top of
% the file---the way it is done isn't good style so don't copy it).
% \changes{v2.0e}{89/07/19}{Spaces removed from `typeout'.}
% \changes{v2.2b}{95/11/19}{Announce moved to top of file}
%    \begin{macrocode}
%<*package>
%\@ifundefined{theorem@style}{}{\endinput}
%\def\FMithmInfo{1995/11/19 v2.2b Theorem extension package (FMi)}
%\ProvidesPackage{theorem}[\FMithmInfo]
%    \end{macrocode}
%
%
% \subsection{Definition of theorem styles and fonts}
%
% All the definitions in this file are done globally to allow
% inputting this file inside a group.
%
% \begin{macro}{\theoremstyle}
%    Before a theorem style can be installed, the chosen style must be
%    known. For that reason, we must test to see that
%    |\th@|\meta{style} is known or, more precisely, that it is
%    different from |\relax|. If the style is not known then
%    |\th@plain| is used.
%    \begin{macrocode}
\gdef\theoremstyle#1{%
   \@ifundefined{th@#1}{\@warning
          {Unknown theoremstyle `#1'. Using `plain'}%
          \theorem@style{plain}}%
%    \end{macrocode}
%    We save the theorem style to be used in the token register
%    |\theorem@style|.
%    \begin{macrocode}
      {\theorem@style{#1}}%
%    \end{macrocode}
%    Now we ``evaluate'' the theorem style: this means, we call the
%    macro |\th@|\meta{style} which will activate the relevant
%    definitions which are contained in a separate file.  This is done
%    in a group to suppress changes to the current font.  This could
%    otherwise pose problems together with the new font selection
%    scheme\footnote{When I printed the original article using the new
%    font selection scheme I ended with a document with slanted
%    typefaces (text headings and all) simply because one of the
%    theorem styles used \texttt{\bslash sl} at toplevel.} if the
%    |\th@|\meta{style} is evaluated a second time.
% \changes{v2.1a}{90/02/15}{Group added around `th@..}
% \changes{v2.1a}{90/02/15}{Reseting of `..@begintheorem removed.}
%    \begin{macrocode}
      \begingroup
        \csname th@\the\theorem@style \endcsname
      \endgroup}
%    \end{macrocode}
% \end{macro}
%
%
% \begin{macro}{\@begintheorem}
% \changes{v2.1a}{90/02/15}{Relaxed at toplevel.}
% \begin{macro}{\@opargbegintheorem}
% \changes{v2.1a}{90/02/15}{Relaxed at toplevel.}
%    We reset |\@begintheorem| and |\@opargbegintheorem| to
%    |\relax| since these commands are no longer necessary at
%    toplevel. This will save a few tokens.
%    \begin{macrocode}
\global\let\@begintheorem\relax
\global\let\@opargbegintheorem\relax
%    \end{macrocode}
% \end{macro}
% \end{macro}
%
% \begin{macro}{\theorem@style}
%    Obviously the \textsf{token} register used above has to be
%    allocated. To assure the utmost compatibility with the
%    original \LaTeX{} definition, we set the default theorem style to
%    \texttt{plain}, which implements the usual \LaTeX{} convention.
% \changes{v2.0f}{89/07/20}{Unnecessary `global' removed.}
%    \begin{macrocode}
\newtoks\theorem@style
\global\theorem@style{plain}
%    \end{macrocode}
% \end{macro}
%
% \begin{macro}{\theorembodyfont}
% \begin{macro}{\theorem@bodyfont}
%    If the |\theorembodyfont| is set by the user then it should not
%    interact with the default font set in the theorem style. When the
%    new font selection is in force this may happen if, for example,
%    the default is |\itshape| and the new |\theorembodyfont| is
%    |\sffamily|. So we add a |\reset@font| command in front of the
%    user definition.
% \changes{v2.0f}{89/07/20}{Unnecessary `global' removed.}
% \changes{v2.0g}{89/09/19}{`theoremfont to `theorembodyfont.}
% \changes{v2.1c}{92/06/22}{`reset@font added and logic changed.}
% \changes{v2.1d}{92/10/28}{Reset to empty if arg is empty}
%    \begin{macrocode}
\gdef\theorembodyfont#1{%
%    \end{macrocode}
%    We check if the argument supplied is empty and if so put nothing
%    into the |\theorem@bodyfont| token register to allow for
%    |\theorembodyfont{}| as a mean of using the default of the
%    current |\theoremstyle|.
%    \begin{macrocode}
   \def\@tempa{#1}%
   \ifx\@tempa\@empty
    \theorem@bodyfont{}%
   \else
    \theorem@bodyfont{\reset@font#1}%
   \fi
   }
\newtoks\theorem@bodyfont
\global\theorem@bodyfont{}
%    \end{macrocode}
% \end{macro}
% \end{macro}
%
%
% \begin{macro}{\theoremheaderfont}
% \changes{v2.0g}{89/09/19}{Macro added.}
%    The font for the theorem headers is handled differently because
%    this definition applies to all theorem styles.
%    \begin{macrocode}
\gdef\theoremheaderfont#1{\gdef\theorem@headerfont{#1}%
%    \end{macrocode}
%    After using the macro once it is redefined to produce an error
%    message.
%    \begin{macrocode}
       \gdef\theoremheaderfont##1{%
        \typeout{\string\theoremheaderfont\space should be used
                 only once.}}}
%    \end{macrocode}
% \end{macro}
%
% \begin{macro}{\theorem@headerfont}
% \changes{v2.0g}{89/09/19}{Macro added.}
%    To set the |\theorem@headerfont| default we first test if
%    the new fontselection scheme is in force.
% \changes{v2.1a}{90/02/15}{Support for new font selection.}
%    \begin{macrocode}
\ifx\upshape\undefined
%    \end{macrocode}
%    If not we define it to expand into |\bfseries|. We don't use
%    |\let| just in case a following style option redefines this
%    macro.
%    \begin{macrocode}
\gdef\theorem@headerfont{\bfseries}
%    \end{macrocode}
%    Otherwise we reset the current shape before calling |\bfseries|.
%    \begin{macrocode}
\else \gdef\theorem@headerfont{\normalfont\bfseries}\fi
%    \end{macrocode}
% \end{macro}
%
% \begin{macro}{\th@plain}
% \begin{macro}{\th@break}
% \begin{macro}{\th@marginbreak}
% \begin{macro}{\th@changebreak}
% \begin{macro}{\th@change}
% \begin{macro}{\th@margin}
%    The different styles are defined in macros such as
%    |\th@plain|.  Since memory space is precious in
%    ``non-Big-versions'', we have to avoid offering too many unused
%    definitions.  Therefore we define these styles in separate files
%    that can be loaded on demand.  Thus the commands themselves only
%    load these files. We use |\@input@|, a \LaTeXe{} internal command
%    that ensures that the file will be listed with |\listfiles|.
% \changes{v2.2a}{94/02/02}{Use `@input@ to load theorem layout files}
%    \begin{macrocode}
\gdef\th@plain{\@input@{thp.sty}}
\gdef\th@break{\@input@{thb.sty}}
\gdef\th@marginbreak{\@input@{thmb.sty}}
\gdef\th@changebreak{\@input@{thcb.sty}}
\gdef\th@change{\@input@{thc.sty}}
\gdef\th@margin{\@input@{thm.sty}}
%    \end{macrocode}
%    This list will be expanded when new styles become
%    available.  For testing,  just append new theorem substyles
%    as document options.
% \end{macro}
% \end{macro}
% \end{macro}
% \end{macro}
% \end{macro}
% \end{macro}
%
%
%
% \subsection{Definition of a new theorem set}
%
% As already pointed out, a new theorem environment can be defined in
% three different ways:
% \begin{verbatim}
%    \newtheorem{Lem}{Lemma}
%    \newtheorem{Lem}{Lemma}[section]
%    \newtheorem{Lem}[Theorem]{Lemma}
%\end{verbatim}
% The function of the macro |\newtheorem| is to recognize these cases
% and then to branch into one of the three macros |\@ynthm|, |\@xnthm|
% or |\@othm|.  This mechanism is adopted unchanged from
% \cite{latex:source}; the essential point here is that, for example,
% in the second case, the arguments \texttt{Lem}, \texttt{Lemma} and
% \texttt{section} are passed over to the macro |\@xnthm|.
%
% We inspect this case first because the others present fewer problems,
% and thus are easily derived from this one.
%
% \begin{macro}{\@xnthm}
%    For our example arguments, the macro |\@xnthm| must fulfill
%    the following:
%    \begin{itemize}
%       \item
%          Define a new \LaTeX-counter `Lem'
%       \item
%          reset this counter within a |\section|
%       \item
%          define the macro |\theLem|
%       \item
%          define the environment macros |\Lem| and
%          |\endLem| using the current
%          |\theoremstyle| and |\theorem@bodyfont|.
%    \end{itemize}
%    Obviously, all this should happen only if the first argument
%    of |\@xnthm| (i.e.\ \texttt{Lem} in our example) is chosen so
%    as not to conflict with any previously defined commands or
%    environments.  This test is performed by the \LaTeX{} macro
%    |\@ifdefinable|.
%    \begin{macrocode}
\gdef\@xnthm#1#2[#3]{\expandafter\@ifdefinable\csname #1\endcsname
%    \end{macrocode}
%    Therefore, the first argument of |\@ifdefinable| is the
%    expansion (in the example, |\Lem|) of
%    |\csname||#1\endcsname|.  The second argument is
%    executed only if the test has been completed successfully.
%    \begin{macrocode}
   {%
%    \end{macrocode}
%    Now we define the new counter. The names of the \LaTeX{}
%    macros employed should speak for themselves:
% \changes{v2.2b}{1995/10/07}{Use \cs{@newctr}}
% \changes{v2.2c}{1995/11/23}{Use \cs{@newctr} properly.}
%    \begin{macrocode}
    \@definecounter{#1}\@newctr{#1}[#3]%
%    \end{macrocode}
%    Using |\@newctr| will give a proper error message if the counter
%    in |#3| is not defined.
%    In defining `|\theLem|' we must generate the desired macro
%    name by use of |\expandafter| and |\csname|.
%    \begin{macrocode}
    \expandafter\xdef\csname the#1\endcsname
%    \end{macrocode}
%    An |\xdef| is used in order to make the definition global,
%    and to ensure that it contains the replacement texts of
%    |\@thmcountersep| and |\@thmcounter|.\footnote{These
%    two macros can be defined by the document style.  Their default
%    values produce a `\texttt{.}' as separation and an arabic
%    representation of the number.} However, not everything should be
%    expanded. For example, it saves space to use |\thesection|
%    instead of its---at times---lengthy expansion.
%    \begin{macrocode}
      {\expandafter \noexpand \csname the#3\endcsname
       \@thmcountersep \@thmcounter{#1}}%
%    \end{macrocode}
%    Thus with the defaults of \LaTeX{}, |\theLem| would be
%    replaced by the command sequence |\thesection.\arabic{Lem}|.
%
%    We will now look at the definition of the macro which is executed
%    at the beginning of the actual environment (in our example this
%    macro is |\Lem|). It should be noted that we use an
%    ``|\expandafter| trick'' to expand only certain parts of the
%    replacement text at the time of the definition.
%    \begin{macrocode}
    \def\@tempa{\global\@namedef{#1}}%
    \expandafter \@tempa \expandafter{%
%    \end{macrocode}
%    First, the macro that contains the current definitions of
%    |\@begintheorem| and |\@opargtheorem| should be called
%    up.  The name of this macro---as is already known---has the form
%    |\th@|\meta{theorem\ style}; therefore, it must be called by
%    \begin{macrocode}
      \csname th@\the \theorem@style
%    \end{macrocode}
%    In addition the default theorem font should be changeable, i.e.\
%    we have to insert the contents of |\theorem@bodyfont|. For
%    that reason, we expand even further, beyond |\endcsname|,
%    and thus insert the contents of the \textsf{token} register
%    |\theorem@bodyfont| in the replacement text.
%    \begin{macrocode}
            \expandafter \endcsname \the \theorem@bodyfont
%    \end{macrocode}
%    Now it is time to call the macro |\@thm| which takes over
%    the further processing.  It has two arguments: the current
%    counter name (in our example, \texttt{Lem}), and the text of the
%    label (in our example, \texttt{Lemma}).
%    \begin{macrocode}
     \@thm{#1}{#2}}%
%    \end{macrocode}
%    With this, the `sub-definition' is complete.  The macro
%    |\@endtheorem| ends a theorem environment and is, so far, nothing
%    but an |\endtrivlist|.  (Hence it is defined globally, and not
%    within the theorem styles.\footnote{This has to be changed as
%    soon as theorem styles that change \texttt{\bslash @endtheorem}
%    exist.  In such a case, all existing styles must be changed as
%    well since they will have to reset the macro.}) Therefore, we can
%    set it equivalent to the macro that ends the theorem set (in our
%    example, |\endLem|).  However, if some day theorem styles exist
%    that do change |\@endtheorem|, we would have to use the
%    commented-out line instead.
%    \begin{macrocode}
    \global \expandafter \let \csname end#1\endcsname \@endtheorem
%   \global\@namedef{end#1}{\@endtheorem}%
%    \end{macrocode}
%    With these commands all the required definitions are employed,
%    unless the test |\@ifdefinable| has failed.  Therefore, we
%    end the second argument of this macro and with it the definition
%    of |\@xnthm|.
%    \begin{macrocode}
   }}
%    \end{macrocode}
% \end{macro}
%
% \begin{macro}{\@ynthm}
%    The definition of |\@ynthm| is completely analogous. In
%    this case the new counter that is defined is not reset within
%    another counter; thus the definition of |\the...|
%    is simplified:
%    \begin{macrocode}
\gdef\@ynthm#1#2{\expandafter\@ifdefinable\csname #1\endcsname
   {\@definecounter{#1}%
    \expandafter\xdef\csname the#1\endcsname{\@thmcounter{#1}}%
%    \end{macrocode}
%    The rest of the definition corresponds literally to that of
%    |\@xnthm|:
%    \begin{macrocode}
    \def\@tempa{\global\@namedef{#1}}\expandafter \@tempa
     \expandafter{\csname th@\the \theorem@style \expandafter
     \endcsname \the\theorem@bodyfont \@thm{#1}{#2}}%
    \global \expandafter \let \csname end#1\endcsname \@endtheorem}}
%    \end{macrocode}
% \end{macro}
%
% \begin{macro}{\@othm}
%    The definition of |\@othm| does not contain anything new.
%    \begin{macrocode}
\gdef\@othm#1[#2]#3{%
%    \end{macrocode}
% \changes{v2.2b}{95/10/07}{Build in check for counter defined}
%    We do not define a new counter but instead use one that has
%    already been defined.  Thus the only definition we need is that
%    of this pseudo-counter (i.e.\ |\the|\meta{env.\ name}).
%    First we check if |#2| corresponds to a known counter name.
%    \begin{macrocode}
  \expandafter\ifx\csname c@#2\endcsname\relax
   \@nocounterr{#2}%
  \else
   \expandafter\@ifdefinable\csname #1\endcsname
   {\expandafter \xdef \csname the#1\endcsname
     {\expandafter \noexpand \csname the#2\endcsname}%
%    \end{macrocode}
%    All other parts of the definition can be adopted from
%    |\@xnthm|. We have to remember, though, that in this case
%    the name of the current counter and the theorem label have moved
%    to the second and third arguments.
%    \begin{macrocode}
    \def\@tempa{\global\@namedef{#1}}\expandafter \@tempa
     \expandafter{\csname th@\the \theorem@style \expandafter
     \endcsname \the\theorem@bodyfont \@thm{#2}{#3}}%
    \global \expandafter \let \csname end#1\endcsname \@endtheorem}%
  \fi}
%    \end{macrocode}
% \end{macro}
%
% \subsection{Macros that are employed in a {\envfont theorem}
%    environment}
%
% \begin{macro}{\@thm}
%    The macro |\@thm| has to increase the current counter.  Then,
%    depending on whether the environment has (or does not have) an
%    optional argument, it has to branch into either |\@begintheorem|
%    or |\@opargtheorem|.
%    \begin{macrocode}
\gdef\@thm#1#2{\refstepcounter{#1}%
%    \end{macrocode}
%    Now we start a {\envfont trivlist} environment, and give
%    |\@topsep| and |\@topsepadd| the values of the \textsf{skip}
%    registers |\theorempreskipamount| and |\theorempostskipamount|.
%    The value in |\@topsep| is the vertical space that is inserted by
%    the first (and only) |\item| in our |\trivlist| whilst
%    |\@topsepadd| is inserted by |\@endparenv| at the end of that
%    {\envfont trivlist} environment.  By using these registers, we
%    obtain the desired space around a {\envfont theorem} environment.
%    \begin{macrocode}
   \trivlist
   \@topsep \theorempreskipamount               % used by first \item
   \@topsepadd \theorempostskipamount           % used by \@endparenv
%    \end{macrocode}
%    Now we have to test whether an optional argument has been given.
% \changes{v2.0d}{89/07/18}{Forgotten percent added.}
%    \begin{macrocode}
   \@ifnextchar [%
%    \end{macrocode}
%    If there is an optional argument, we will call |\@ythm|, and
%    move the arguments read back into the input stream.
%    \begin{macrocode}
   {\@ythm{#1}{#2}}%
%    \end{macrocode}
%    If not, we call |\@begintheorem|. Its first argument is the
%    name of the theorem set (hence the second argument of
%    |\@thm|). Its second argument is the macro that produces the
%    current number.
%    \begin{macrocode}
   {\@begintheorem{#2}{\csname the#1\endcsname}\ignorespaces}}
%    \end{macrocode}
% \end{macro}
%
% \begin{macro}{\@xthm}
% \begin{macro}{\@ythm}
%    Both these macros were originally called by |\@thm|.  We do
%    not need |\@xthm| anymore, hence we reset it to
%    |\relax|. The definition of |\@ythm| has not changed at
%    all from its definition in \LaTeX. In order to make the macros
%    easier to understand, we will
%    nevertheless present it (commented out).
%    \begin{macrocode}
\global\let\@xthm\relax
% \def\@ythm#1#2[#3]{\@opargbegintheorem{#2}{\csname
%           the#1\endcsname}{#3}\ignorespaces}
%</package>
%    \end{macrocode}
%    The primitive |\ignorespaces| in |\@ythm| and
%    |\@thm| is needed to remove the spaces between the
%    |\begin{...}| and the actual text.
% \end{macro}
% \end{macro}
%
%
% \subsection{Definition of the theorem substyles}
%
% As already pointed out, the theorem substyles, defined below, are
% only loaded when necessary.  Note that all these substyles, except
% \texttt{plain}, have |\slshape| as the default body font.
%
% \subsubsection{The \texttt{plain} style}
%
% \label{dok:th@plain} As the following macros use |@|, we have
% to locally set the |\catcode| of this symbol to ``letter''.
% This happens within a group, so that we do not have to worry about
% which |\catcode| that symbol had before.
%    \begin{macrocode}
%<*thp>
\begingroup \makeatletter
%    \end{macrocode}
% Since we are now within a group, we must make all definitions
% globally.  First we make sure that \texttt{theorem.sty} is loaded.
% This will allow us to use this file as a document style option
% without having to call \texttt{theorem} itself as an option.  At the
% same time, we assure that at least version 2 is loaded, since
% |\theorem@style| was not defined in earlier versions.
%    \begin{macrocode}
\@ifundefined{theorem@style}{\input{theorem.sty}}{}
\ProvidesFile{thp.sty}
             [\FMithmInfo]
%    \end{macrocode}
%
%
% \begin{macro}{\th@plain}
% \begin{macro}{\@begintheorem}
% \begin{macro}{\@opargbegintheorem}
%    |\theoremstyle{plain}| corresponds to the original
%    definition, except that the distances to the surrounding text are
%    determined by the parameters |\theorempreskipamount| and
%    |\theorempostskipamount|.
%    First we set the default body font.
%    \begin{macrocode}
\gdef\th@plain{\normalfont\itshape
%    \end{macrocode}
%    Then we define |\@begintheorem| and
%    |\@opargbegintheorem|. These two macros define how the
%    header of a theorem is typeset.  |\@opargbegintheorem| will
%    be called if a \textsf{theorem} environment with an optional
%    argument is encountered; otherwise, the header is constructed by
%    calling |\@begintheorem|.  If one of these macros is
%    executed, we are within a \textsf{trivlist} environment started by
%    |\@thm|.  So the theorem header is produced with an
%    |\item| command.
%
%    Instead of specifying the header font directly, all standard
%    theorem styles use the |\theorem@headerfont| macro to allow
%    customization.  The extra space (|\labelsep|) is necessary
%    because of problems in the \textsf{trivlist} environment.
%    \begin{macrocode}
  \def\@begintheorem##1##2{%
        \item[\hskip\labelsep \theorem@headerfont ##1\ ##2]}%
%    \end{macrocode}
%    The definition of |\@opargbegintheorem| is completely
%    analogous. The only difference is the fact that there exists a
%    third argument (which is the optional parameter of the
%    environment and contains additional information about the
%    theorem).  Customarily we enclose it in parentheses.
%    \begin{macrocode}
\def\@opargbegintheorem##1##2##3{%
   \item[\hskip\labelsep \theorem@headerfont ##1\ ##2\ (##3)]}}
%    \end{macrocode}
% \end{macro}
% \end{macro}
% \end{macro}
%
% \noindent
% We conclude with an |\endgroup| to restore the
% |\catcode| of |@|.
%    \begin{macrocode}
\endgroup
%</thp>
%    \end{macrocode}
%
%
% \subsubsection{The \texttt{break} style}
%
% This style option is stored in the file |thb.sty|.
% For the next two lines see the documentation for |\th@plain| on
% page~\pageref{dok:th@plain}.
%    \begin{macrocode}
%<*thb>
\begingroup \makeatletter
\@ifundefined{theorem@style}{\input{theorem.sty}}{}
\ProvidesFile{thb.sty}
             [\FMithmInfo]
%    \end{macrocode}
%
% \begin{macro}{\th@break}
%    |\theoremstyle{break}| produces a line break after the name
%    of the theorem.  The font is |\slshape|.  Hence, we define
%    |\th@break| as follows:
%    \begin{macrocode}
\gdef\th@break{\normalfont\slshape
  \def\@begintheorem##1##2{\item[%
%    \end{macrocode}
%    We run into the following problem: it is not possible to create
%    the header with |\item[|\meta{title}|]| and then start
%    a new line by, for example, |\mbox{}\\|. Such a definition
%    will fail whenever a list environment follows immediately.  With
%    the above construction, the |\mbox{}| causes the switch
%    |@inlabel| (cf.\ definition of |\list| and
%    |\trivlist| in \cite{latex:source}) to be set to \texttt{false}
%    and so the following list will insert additional vertical
%    space (|\topskip|). This is quite annoying. Therefore, we
%    create the line break within the |\item|. In order to ensure
%    that the text will begin at the proper position in the following
%    line, we simply pretend that the label does not take any
%    room.\footnote{This will lead to problems whenever very high
%    symbols occurring in the line tower into the
%    heading. So, something else has to be done here sometime.}
%    \begin{macrocode}
       \rlap{\vbox{\hbox{\hskip \labelsep\theorem@headerfont ##1\ ##2}%
                   \hbox{\strut}}}]}%
%    \end{macrocode}
%    Again, the definition of |\@opargbegintheorem| is
%    completely analogous.
%    \begin{macrocode}
\def\@opargbegintheorem##1##2##3{%
  \item[\rlap{\vbox{\hbox{\hskip \labelsep \theorem@headerfont
                     ##1\ ##2\ (##3)}%
                    \hbox{\strut}}}]}}
\endgroup
%</thb>
%    \end{macrocode}
% \end{macro}
%
%
% \subsubsection{The \texttt{changebreak} style}
%
%    \begin{macrocode}
% This style option is stored in the file |thcb.sty|.
%    \begin{macrocode}
%<*thcb>
\begingroup \makeatletter
\@ifundefined{theorem@style}{\input{theorem.sty}}{}
\ProvidesFile{thcb.sty}
             [\FMithmInfo]
%    \end{macrocode}
%
% \begin{macro}{\th@changebreak}
%    The \texttt{change-break} theorem style is like \texttt{break}
%    but with interchange of theorem name and theorem number.  Thus we
%    define |\th@changebreak| as follows:
%    \begin{macrocode}
\gdef\th@changebreak{\normalfont\slshape
  \def\@begintheorem##1##2{\item
       [\rlap{\vbox{\hbox{\hskip\labelsep \theorem@headerfont ##2\ ##1}%
                    \hbox{\strut}}}]}%
\def\@opargbegintheorem##1##2##3{%
  \item[\rlap{\vbox{\hbox{\hskip\labelsep \theorem@headerfont
                     ##2\ ##1\ (##3)}%
                    \hbox{\strut}}}]}}
\endgroup
%</thcb>
%    \end{macrocode}
% \end{macro}
%
%
% \subsubsection{The \texttt{change} style}
%
% This style option is stored in the file |thc.sty|.
%    \begin{macrocode}
%<*thc>
\begingroup \makeatletter
\@ifundefined{theorem@style}{\input{theorem.sty}}{}
\ProvidesFile{thc.sty}
             [\FMithmInfo]
%    \end{macrocode}
%
% \begin{macro}{\th@change}
%    The \texttt{change} theorem style corresponds to the
%    \texttt{change break} style without a linebreak after the header.
%    To say it in another way, it's the same as the \texttt{plain}
%    style but with number and name interchanged and |\slshape| as the
%    default font.
%    \begin{macrocode}
\gdef\th@change{\normalfont\slshape
  \def\@begintheorem##1##2{\item
       [\hskip\labelsep \theorem@headerfont ##2\ ##1]}%
\def\@opargbegintheorem##1##2##3{%
  \item[\hskip\labelsep \theorem@headerfont ##2\ ##1\ (##3)]}}
\endgroup
%</thc>
%    \end{macrocode}
% \end{macro}
%
%
% \subsubsection{The \texttt{marginbreak} style}
%
% This style option is the one used most often at Mainz.  It is saved
% in the file |thmb.sty|.
%    \begin{macrocode}
%<*thmb>
\begingroup \makeatletter
\@ifundefined{theorem@style}{\input{theorem.sty}}{}
\ProvidesFile{thmb.sty}
             [\FMithmInfo]
%    \end{macrocode}
%
% \begin{macro}{\th@marginbreak}
%    The \texttt{margin break} style is nearly the same as the
%    \texttt{change break} style. The only difference is the placement
%    of the theorem number.  We use |\llap| to place it in the left
%    margin.
%
%    In this style |\labelsep| denotes the separation between the
%    number and the text.
%    \begin{macrocode}
\gdef\th@marginbreak{\normalfont\slshape
  \def\@begintheorem##1##2{\item
       [\rlap{\vbox{\theorem@headerfont
                    \hbox{\llap{##2}\hskip\labelsep ##1}%
                    \hbox{\strut}}}]}%
\def\@opargbegintheorem##1##2##3{%
  \item[\rlap{\vbox{\theorem@headerfont
                    \hbox{\llap{##2}\hskip\labelsep ##1\ (##3)}%
                    \hbox{\strut}}}]}}
\endgroup
%</thmb>
%    \end{macrocode}
% \changes{v2.1c}{90/02/17}{`endgroup outside macrocode env.}
% \end{macro}
%
%
%
% \subsubsection{The \texttt{margin} style}
%
% This style option is stored in the file |thm.sty|.
%    \begin{macrocode}
%<*thm>
\begingroup \makeatletter
\@ifundefined{theorem@style}{\input{theorem.sty}}{}
\ProvidesFile{thm.sty}
             [\FMithmInfo]
%    \end{macrocode}
%
% \begin{macro}{\th@margin}
%    Again this is only a variant of the theorem styles described
%    above without any new ideas.
%    \begin{macrocode}
\gdef\th@margin{\normalfont\slshape
  \def\@begintheorem##1##2{\item
       [\theorem@headerfont \llap{##2}\hskip\labelsep ##1]}%
\def\@opargbegintheorem##1##2##3{%
  \item[\theorem@headerfont \llap{##2}\hskip\labelsep ##1\ (##3)]}}
\endgroup
%</thm>
%    \end{macrocode}
% \end{macro}
%
%
%
%
% \subsection{Final Definitions}
%
% \begin{macro}{\theorempreskipamount}
% \begin{macro}{\theorempostskipamount}
%    The \textsf{skip} parameters that regulate the vertical empty space
%    before and after the {\envfont theorem} environment have
%    to be allocated as well.
% \changes{v2.0f}{89/07/20}{Unnecessary `global' removed.}
%    \begin{macrocode}
%<*package>
\newskip\theorempreskipamount
\newskip\theorempostskipamount
%    \end{macrocode}
%    Since we have used the same values for all theorem sets, we now can
%    assign them.
%    \begin{macrocode}
\global\setlength\theorempreskipamount{12pt plus 5pt minus 3pt}
\global\setlength\theorempostskipamount{8pt plus 3pt minus 1.5pt}
%    \end{macrocode}
% \end{macro}
% \end{macro}
%
%
% \begin{macro}{\@endtheorem}
%    The same holds for the macro |\@endtheorem|, which ends a
%    {\envfont theorem} environment. Since it is the same for all
%    theorem sets, it is removed from the macros |\th@|\meta{style}.
%    It simply ends the {\envfont trivlist} environment, which was
%    begun in |\@thm|.
%    \begin{macrocode}
\global\let\@endtheorem=\endtrivlist
%    \end{macrocode}
% \end{macro}
%
%
%
% \begin{macro}{\@preamblecmds}
%    All macros defined above are to be used only in the preamble.
%    Therefore, we insert them in |\@preamblecmds| which will disable
%    them at begin document. This is done by the internal \LaTeXe{}
%    command |\@onlypreamble|.
% \changes{v2.0g}{89/09/19}{`theoremheaderfont and `theorembodyfont
%    added.}
% \changes{v2.2a}{94/02/03}{Use \LaTeXe{} interface}
%    \begin{macrocode}
\@onlypreamble\@xnthm
\@onlypreamble\@ynthm
\@onlypreamble\@othm
\@onlypreamble\newtheorem
\@onlypreamble\theoremstyle
\@onlypreamble\theorembodyfont
\@onlypreamble\theoremheaderfont
%    \end{macrocode}
% \end{macro}
%
% Finally we declare the \texttt{plain} theorem style to be the default.
% \changes{v2.1b}{90/06/04}{Fixed missing `item by declaring default
%    style.}
%    \begin{macrocode}
\theoremstyle{plain}
%</package>
%    \end{macrocode}
%
% \Finale

\endinput
}{}
\ProvidesFile{thc.sty}
             [\FMithmInfo]
%    \end{macrocode}
%
% \begin{macro}{\th@change}
%    The \texttt{change} theorem style corresponds to the
%    \texttt{change break} style without a linebreak after the header.
%    To say it in another way, it's the same as the \texttt{plain}
%    style but with number and name interchanged and |\slshape| as the
%    default font.
%    \begin{macrocode}
\gdef\th@change{\normalfont\slshape
  \def\@begintheorem##1##2{\item
       [\hskip\labelsep \theorem@headerfont ##2\ ##1]}%
\def\@opargbegintheorem##1##2##3{%
  \item[\hskip\labelsep \theorem@headerfont ##2\ ##1\ (##3)]}}
\endgroup
%</thc>
%    \end{macrocode}
% \end{macro}
%
%
% \subsubsection{The \texttt{marginbreak} style}
%
% This style option is the one used most often at Mainz.  It is saved
% in the file |thmb.sty|.
%    \begin{macrocode}
%<*thmb>
\begingroup \makeatletter
\@ifundefined{theorem@style}{% \iffalse meta-comment
%
% Copyright (C) 1993-2024
%
% The LaTeX Project and any individual authors listed elsewhere
% in this file.
%
% This file is part of the Standard LaTeX `Tools Bundle'.
% -------------------------------------------------------
%
% It may be distributed and/or modified under the
% conditions of the LaTeX Project Public License, either version 1.3c
% of this license or (at your option) any later version.
% The latest version of this license is in
%    https://www.latex-project.org/lppl.txt
% and version 1.3c or later is part of all distributions of LaTeX
% version 2005/12/01 or later.
%
% The list of all files belonging to the LaTeX `Tools Bundle' is
% given in the file `manifest.txt'.
%
% \fi
%
%%
%
%
\def\FMithmInfo{2023/07/05 v2.2c Theorem extension package (FMi)}
%
% \ProvidesFile{theorem.dtx}[\FMithmInfo]
%\iffalse   % this is a METACOMMENT !
%%
%% Package `theorem' to use with LaTeX2e
%% Copyright (C) 1989-1995 Frank Mittelbach, all rights reserved.
%<+package>\@ifundefined{theorem@style}{}{\endinput}
%<+package>\NeedsTeXFormat{LaTeX2e}[1995/06/01]
%<+package>\ProvidesPackage{theorem}
%<+package>         [\FMithmInfo]
%
% \fi
%
% \changes{v1.0}{87/05/07}{First draft.}
% \changes{v1.0a}{87/05/07}{`thistheoremstyle' added.}
% \changes{v1.1a}{87/07/29}{New concept: redefine the internal
%                          \protect\LaTeX{} commands `@xnthm' etc.}
% \changes{v1.1b}{87/07/31}{`theoremfont' used to specify font.}
% \changes{v1.2a}{88/01/08}{`@endtheorem' now globally defined.}
% \changes{v1.2b}{88/01/11}{`newtheorem' only allowed in the preamble.}
% \changes{v1.3a}{88/02/14}{`theoremfont' now a toks register.}
%
% \changes{v2.0a}{88/11/16}{New implementation and new documentation.}
% \changes{v2.0a}{88/11/16}{No layout changes but internal modifications
%                           old substyles won't work.}
% \changes{v2.0b}{89/01/30}{`global' used for all definitions.}
% \changes{v2.0c}{89/06/15}{Documentation translated to English.
%                            Thanks to LLL and Nina}
% \changes{v2.1a}{90/02/15}{Support for new font selection.}
% \changes{v2.2a}{94/02/03}{Cleanup for LaTeX2e}
%
% \DoNotIndex{\@endparenv,\@ifdefinable,\@ifnextchar,\@ifundefined}
% \DoNotIndex{\@namedef,\@nameuse,\@spaces,\@tempa,\@warning}
% \DoNotIndex{\ }
% \DoNotIndex{\begingroup,\bfseries}
% \DoNotIndex{\csname}
% \DoNotIndex{\def,\do,\docdate}
% \DoNotIndex{\endcsname,\endgroup,\endinput,\endtrivlist,\expandafter}
% \DoNotIndex{\filedate,\fileversion}
% \DoNotIndex{\gdef,\global}
% \DoNotIndex{\hbox,\hskip}
% \DoNotIndex{\ignorespaces,\input,\itshape,\item}
% \DoNotIndex{\labelsep,\let,\llap}
% \DoNotIndex{\makeatletter}
% \DoNotIndex{\noexpand}
% \DoNotIndex{\refstepcounter,\relax,\rlap}
% \DoNotIndex{\setlength,\slshape,\space,\string,\strut}
% \DoNotIndex{\the,\trivlist,\typeout}
% \DoNotIndex{\vbox}
% \DoNotIndex{\xdef}
%
% \def\deflabel#1{\ttfamily #1\hfill}
% \def\deflist#1{\begin{list}{}{\settowidth\labelwidth{\ttfamily #1}%
%                               \setlength\leftmargin\labelwidth
%                               \addtolength\leftmargin\labelsep
%                               \let\makelabel\deflabel}}
% \def\enddeflist{\end{list}}
% \def\envfont{\normalfont\ttfamily}
%
% \MakeShortVerb{\|}
%
% \newcommand{\TUB}{{\sl TUGboat\/}}
%
% \GetFileInfo{theorem.dtx}
% \title{An Extension of the
%        \LaTeX{} theorem environment\thanks{This file
%        has version number \fileversion, last
%        revised \filedate.}}
% \author{Frank Mittelbach}
%
% \MaintainedByLaTeXTeam{tools}
% \maketitle
%
% \begin{abstract}
%    The macros described in this paper yield an extension of the
%    \LaTeX{} theorem mechanism. It is designed to satisfy the
%    different requirements of various journals. Thus, the
%    layout of the ``theorems'' can be manipulated by determining a
%    ``style''. This article describes not only the use, but
%    also the definition, of the necessary macros.
% \end{abstract}
%
% \section*{Preface to version 2.2}
%
% For \LaTeXe{} this package did not need any fundamental changes. I
% only modified the messages generated so that theorem layout styles
% will show up with the |\listfiles| command and cleaned the section
% on the New Font Selection Scheme since this is now included in
% \LaTeX.
%
% \section*{Preface to version 2.1}
%
% This version is identical to 2.0g described in \TUB\ 10\#3 except
% for some internal defaults which are now set depending on the used
% font selection scheme.
%
% This was done to avoid unpleasant surprises if the new font selection
% scheme is in force. For further details see
% section~\ref{sec:newfonts} and \cite{bk:GMS94}.
%
% \section{Introduction}
%
% For our purposes here, ``theorems'' are labelled enunciations,
% often set off from the main text by extra space and a font change.
% Theorems, corollaries, conjectures, definitions, and remarks
% are all instances of ``theorems''.  The ``header'' of these structures
% is composed of a label (such as \textsc{Theorem} or \textsc{Remark})
% and a number which serializes an item in the sequence of items
% with the same label.
%
% Shortly after the introduction of \LaTeX{} at the Fachbereich
% Mathematik in Mainz, the desire to manipulate the layout of
% ``theorems'' arose.  In Mainz, the following two conventions came
% into general use:
% \begin{enumerate}
%   \item The number of the theorem is shown in the margin.
%   \item There is a line break at
%         the end of the theorem header.
% \end{enumerate}
%
% Additionally, some journals require different formats which depend on
% the ``sort of theorem'': e.g.\ often remarks and definitions are set
% in |\upshape|, while |\itshape| is employed for main theorems.
%
% Confronted with these requirements, a theorem environment was
% developed in Mainz which allows separate determination of the
% layout of the ``theorems sets'', comparable to |\pagestyle|.
%
%
% \section{The user interface}
% \subsection{Defining new theorem sets}
%
% \DescribeMacro\newtheorem
% As in the original \LaTeX{} version, the command |\newtheorem|
% defines a new ``theorem set'' or ``theorem-like structure''.
% Two required arguments name the new environment and give the text
% to be typeset with each instance of the new ``set'', while
% an optional argument determines how
% the ``set'' is enumerated:
% \begin{description}
%    \item[\ttfamily \bslash newtheorem\{foo\}\{bar\}]
%       The theorem set {\envfont foo} (whose name is \texttt{bar})
%       uses its own counter.
%    \item[\ttfamily \bslash newtheorem\{foo2\}{[foo]}\{bar2\}]
%       The theorem set {\envfont foo2} (printed name \texttt{bar2})
%       uses the same counter as the theorem set \texttt{foo}.
%    \item[\ttfamily \bslash newtheorem\{foo3\}\{bar3\}{[section]}]
%       The theorem set {\envfont foo3} (printed name \texttt{bar3}) is
%       enumerated within the counter \texttt{section}, i.e.\ with every
%       new |\section| the enumeration begins again with 1, and
%       the enumeration is composed from the section-number and the
%       theorem counter itself.
% \end{description}
%
% \DescribeMacro\theoremstyle
% Additionally, the command |\theoremstyle| can define the layout
% of various, or all, theorem sets.  It should be noted that any
% theorem set defined by |\newtheorem| is typeset in the
% |\theoremstyle| that is current at the time of the definition.
% Thus, the following
% \begin{verbatim}
%    \theoremstyle{break}        \newtheorem{Cor}{Corollary}
%    \theoremstyle{plain}        \newtheorem{Exa}{Example}[section]
%\end{verbatim}                             ^^A No blank at line start
% leads to the result that the set {\envfont Cor} is formatted in the
% style \texttt{break}, while the set {\envfont Exa} and all the
% following ones are formatted in the style \texttt{plain}, unless
% another |\theoremstyle| follows.  Since the definitions installed by
% |\newtheorem| are global, one also can limit |\theoremstyle| locally
% by grouping braces.
%
% \DescribeMacro\theorembodyfont
% The choice of the font for the theorem body is completely
% independent of the chosen |\theoremstyle|; this has proven to
% be very advantageous.  For example,
% \begin{verbatim}
%   {\theorembodyfont{\upshape}          \newtheorem{Rem}{Remark}}
%\end{verbatim}
% defines a theorem set {\envfont Rem}, which will be set in
% |\upshape| in the current layout (which in our example is
% \texttt{plain}).  As with |\theoremstyle|, the |\theorembodyfont|
% chosen is that current at the time of |\newtheorem|.  If
% |\theorembodyfont| is not specified or one defines
% |\theorembodyfont{}|, then the font used will be that defined by the
% |\theoremstyle|.
%
% \DescribeMacro\theoremheaderfont It is also possible to customize
% the font used for the theorem headers.  This is, however, a global
% declaration, and therefore there should be at most one
% |\theoremheaderfont| declaration in the preamble.\footnote{If it is
% actually necessary to have different header fonts, one has to define
% new theorem styles (substituting the desired font) or specify the
% information directly in the \texttt{\bslash newtheorem} declaration
% (the unclean variant).}
%
% \DescribeMacro\theorempreskipamount
% \DescribeMacro\theorempostskipamount Two additional parameters
% affect the vertical space around the theorem environments:
% |\theorempreskipamount| and |\theorempostskipamount| define,
% respectively, the spacing before and after such an environment.
% These parameters apply for all theorem sets and can be manipulated
% with the ordinary length macros.  They are rubber lengths
% (`\textsf{skips}'), and therefore can contain \texttt{plus} and
% \texttt{minus} parts.
%
% Since the definition of theorem sets should---most sensibly---be
% placed in the preamble, we only allow installation there.
% It is therefore possible to release the memory used here after
% |\begin{document}|, in order to make room for other
% applications.
%
%
% \subsection{Existing theorem styles}
%
% The following theorem styles exist to date:
% \begin{deflist}{marginbreak}
%    \item[plain]
%       This theorem style emulates the original \LaTeX{} definition,
%       except that additionally the parameters
%       |\theorem...skipamount| are used.
%    \item[break]
%       In this style, the theorem header is followed by a line
%       break.
%    \item[marginbreak]
%       The theorem number is set in the margin, and there is a line
%       break as in \texttt{break}.
%    \item[changebreak]
%       Like \texttt{break}, but with header number and text
%       interchanged.
%    \item[change]
%       Header number and text are interchanged, without a line break.
%    \item[margin]
%       The number is set in the left margin, without a line break.
% \end{deflist}
% All styles (except \texttt{plain}) select |\slshape| as the default
% |\theorembodyfont|.
%
%
% \subsection{Examples}
%
% Given the above theorem sets {\envfont Cor}, {\envfont Exa}
% and {\envfont Rem},
% suppose that the preamble also contains the declarations:
% \begin{verbatim}
%    \theoremstyle{marginbreak}   \newtheorem{Lem}[Cor]{Lemma}
%    \theoremstyle{change}
%    \theorembodyfont{\itshape}        \newtheorem{Def}[Cor]{Definition}
%
%    \theoremheaderfont{\scshape}
%\end{verbatim}
% Then the following are some typical examples of the typeset output
% resulting from their use.
%
% \begin{Cor}
%    This is a sentence typeset in the theorem environment {\envfont
%    Cor}.
% \end{Cor}
%
% \begin{Exa}
%    This is a sentence typeset in the theorem environment {\envfont
%    Exa}.
% \end{Exa}
%
% \begin{Rem}
%    This is a sentence typeset in the theorem environment {\envfont
%    Rem}.
% \end{Rem}
%
% \begin{Lem}[Ben User]
%    This is a sentence typeset in the theorem environment {\envfont
%    Lem}.
% \end{Lem}
%
% \begin{Def}[Very impressive Definition]
%    This is a sentence typeset in the theorem environment {\envfont
%    Def}.
% \end{Def}
%
% The last two examples show the effect of the optional argument to
% a theorem environment (it is the text typeset in parentheses).
%
%
% \section{Special Considerations}
% \label{sec:newfonts}
%
% Theoremheader and body are implemented as a unit. This means that
% the |\theoremheaderfont|  will inherit characteristics of the
% |\theorembodyfont| in \LaTeXe. Thus, if for example
% |\theorembodyfont| is |\itshape| and |\theoremheaderfont|
% is |\bfseries| the font selected for the header will have the
% characteristics `bold extended italic'. If this is not desired one
% should set the |\theoremheaderfont| to something like
% \begin{verbatim}
%      \theoremheaderfont{\normalfont\bfseries}
%\end{verbatim}
% i.e.\ supplying all necessary font information explicitly.
%
%
%
% \section{Acknowledgements}
%
% The publication of this set of macros was only possible with the
% help of Christina Busse (translating the manuscript into English),
% Joachim Pense (playing the r\^ole of typist), Chris Rowley
% (looking everything over) and many others providing useful
% suggestions.
%
% \MaybeStop{%
%    \begin{thebibliography}{9}
%       \bibitem{bk:GMS94} \textsc{M.~Goossens}, \textsc{F.~Mittelbach}
%         and \textsc{A.~Samarin}.
%         \newblock The \LaTeX{} Companion.
%         \newblock
%         Addison-Wesley, Reading, Massachusetts, 1994.
%       \bibitem{latex:source} \textsc{Lamport, Leslie.} \newblock
%                               \texttt{latex.tex}, version 2.09, date
%                               Feb. 1990.
%    \end{thebibliography}
%    \PrintIndex
%    \PrintChanges}
%
%
% \section{The documentation driver file}
%
% The next bit of code contains the documentation driver file for
% \TeX{}, i.e., the file that will produce the documentation you are
% currently reading. It will be extracted from this file by the
% \texttt{docstrip} program. Since it is the first code in the file
% one can alternatively process this file directly with \LaTeXe{} to
% obtain the documentation.
%
%    \begin{macrocode}
%<*driver>
\documentclass{ltxdoc}

\usepackage{theorem}

%<+driver>% The next few lines define theorem sets which are used
%<+driver>% in the example section of the documentation.

\theoremstyle{break}       \newtheorem{Cor}{Corollary}
\theoremstyle{plain}       \newtheorem{Exa}{Example}[section]
{\theorembodyfont{\upshape}\newtheorem{Rem}{Remark}}
\theoremstyle{marginbreak} \newtheorem{Lem}[Cor]{Lemma}
\theoremstyle{change}
\theorembodyfont{\itshape} \newtheorem{Def}[Cor]{Definition}

\theoremheaderfont{\scshape}

\RecordChanges

\begin{document}
   \DocInput{theorem.dtx}
\end{document}
%</driver>
%    \end{macrocode}
%
% \section{Definition of the Macros}
%
% If the file has been loaded before, we abort immediately. If not the
% package announces itself (this is actually done at the very top of
% the file---the way it is done isn't good style so don't copy it).
% \changes{v2.0e}{89/07/19}{Spaces removed from `typeout'.}
% \changes{v2.2b}{95/11/19}{Announce moved to top of file}
%    \begin{macrocode}
%<*package>
%\@ifundefined{theorem@style}{}{\endinput}
%\def\FMithmInfo{1995/11/19 v2.2b Theorem extension package (FMi)}
%\ProvidesPackage{theorem}[\FMithmInfo]
%    \end{macrocode}
%
%
% \subsection{Definition of theorem styles and fonts}
%
% All the definitions in this file are done globally to allow
% inputting this file inside a group.
%
% \begin{macro}{\theoremstyle}
%    Before a theorem style can be installed, the chosen style must be
%    known. For that reason, we must test to see that
%    |\th@|\meta{style} is known or, more precisely, that it is
%    different from |\relax|. If the style is not known then
%    |\th@plain| is used.
%    \begin{macrocode}
\gdef\theoremstyle#1{%
   \@ifundefined{th@#1}{\@warning
          {Unknown theoremstyle `#1'. Using `plain'}%
          \theorem@style{plain}}%
%    \end{macrocode}
%    We save the theorem style to be used in the token register
%    |\theorem@style|.
%    \begin{macrocode}
      {\theorem@style{#1}}%
%    \end{macrocode}
%    Now we ``evaluate'' the theorem style: this means, we call the
%    macro |\th@|\meta{style} which will activate the relevant
%    definitions which are contained in a separate file.  This is done
%    in a group to suppress changes to the current font.  This could
%    otherwise pose problems together with the new font selection
%    scheme\footnote{When I printed the original article using the new
%    font selection scheme I ended with a document with slanted
%    typefaces (text headings and all) simply because one of the
%    theorem styles used \texttt{\bslash sl} at toplevel.} if the
%    |\th@|\meta{style} is evaluated a second time.
% \changes{v2.1a}{90/02/15}{Group added around `th@..}
% \changes{v2.1a}{90/02/15}{Reseting of `..@begintheorem removed.}
%    \begin{macrocode}
      \begingroup
        \csname th@\the\theorem@style \endcsname
      \endgroup}
%    \end{macrocode}
% \end{macro}
%
%
% \begin{macro}{\@begintheorem}
% \changes{v2.1a}{90/02/15}{Relaxed at toplevel.}
% \begin{macro}{\@opargbegintheorem}
% \changes{v2.1a}{90/02/15}{Relaxed at toplevel.}
%    We reset |\@begintheorem| and |\@opargbegintheorem| to
%    |\relax| since these commands are no longer necessary at
%    toplevel. This will save a few tokens.
%    \begin{macrocode}
\global\let\@begintheorem\relax
\global\let\@opargbegintheorem\relax
%    \end{macrocode}
% \end{macro}
% \end{macro}
%
% \begin{macro}{\theorem@style}
%    Obviously the \textsf{token} register used above has to be
%    allocated. To assure the utmost compatibility with the
%    original \LaTeX{} definition, we set the default theorem style to
%    \texttt{plain}, which implements the usual \LaTeX{} convention.
% \changes{v2.0f}{89/07/20}{Unnecessary `global' removed.}
%    \begin{macrocode}
\newtoks\theorem@style
\global\theorem@style{plain}
%    \end{macrocode}
% \end{macro}
%
% \begin{macro}{\theorembodyfont}
% \begin{macro}{\theorem@bodyfont}
%    If the |\theorembodyfont| is set by the user then it should not
%    interact with the default font set in the theorem style. When the
%    new font selection is in force this may happen if, for example,
%    the default is |\itshape| and the new |\theorembodyfont| is
%    |\sffamily|. So we add a |\reset@font| command in front of the
%    user definition.
% \changes{v2.0f}{89/07/20}{Unnecessary `global' removed.}
% \changes{v2.0g}{89/09/19}{`theoremfont to `theorembodyfont.}
% \changes{v2.1c}{92/06/22}{`reset@font added and logic changed.}
% \changes{v2.1d}{92/10/28}{Reset to empty if arg is empty}
%    \begin{macrocode}
\gdef\theorembodyfont#1{%
%    \end{macrocode}
%    We check if the argument supplied is empty and if so put nothing
%    into the |\theorem@bodyfont| token register to allow for
%    |\theorembodyfont{}| as a mean of using the default of the
%    current |\theoremstyle|.
%    \begin{macrocode}
   \def\@tempa{#1}%
   \ifx\@tempa\@empty
    \theorem@bodyfont{}%
   \else
    \theorem@bodyfont{\reset@font#1}%
   \fi
   }
\newtoks\theorem@bodyfont
\global\theorem@bodyfont{}
%    \end{macrocode}
% \end{macro}
% \end{macro}
%
%
% \begin{macro}{\theoremheaderfont}
% \changes{v2.0g}{89/09/19}{Macro added.}
%    The font for the theorem headers is handled differently because
%    this definition applies to all theorem styles.
%    \begin{macrocode}
\gdef\theoremheaderfont#1{\gdef\theorem@headerfont{#1}%
%    \end{macrocode}
%    After using the macro once it is redefined to produce an error
%    message.
%    \begin{macrocode}
       \gdef\theoremheaderfont##1{%
        \typeout{\string\theoremheaderfont\space should be used
                 only once.}}}
%    \end{macrocode}
% \end{macro}
%
% \begin{macro}{\theorem@headerfont}
% \changes{v2.0g}{89/09/19}{Macro added.}
%    To set the |\theorem@headerfont| default we first test if
%    the new fontselection scheme is in force.
% \changes{v2.1a}{90/02/15}{Support for new font selection.}
%    \begin{macrocode}
\ifx\upshape\undefined
%    \end{macrocode}
%    If not we define it to expand into |\bfseries|. We don't use
%    |\let| just in case a following style option redefines this
%    macro.
%    \begin{macrocode}
\gdef\theorem@headerfont{\bfseries}
%    \end{macrocode}
%    Otherwise we reset the current shape before calling |\bfseries|.
%    \begin{macrocode}
\else \gdef\theorem@headerfont{\normalfont\bfseries}\fi
%    \end{macrocode}
% \end{macro}
%
% \begin{macro}{\th@plain}
% \begin{macro}{\th@break}
% \begin{macro}{\th@marginbreak}
% \begin{macro}{\th@changebreak}
% \begin{macro}{\th@change}
% \begin{macro}{\th@margin}
%    The different styles are defined in macros such as
%    |\th@plain|.  Since memory space is precious in
%    ``non-Big-versions'', we have to avoid offering too many unused
%    definitions.  Therefore we define these styles in separate files
%    that can be loaded on demand.  Thus the commands themselves only
%    load these files. We use |\@input@|, a \LaTeXe{} internal command
%    that ensures that the file will be listed with |\listfiles|.
% \changes{v2.2a}{94/02/02}{Use `@input@ to load theorem layout files}
%    \begin{macrocode}
\gdef\th@plain{\@input@{thp.sty}}
\gdef\th@break{\@input@{thb.sty}}
\gdef\th@marginbreak{\@input@{thmb.sty}}
\gdef\th@changebreak{\@input@{thcb.sty}}
\gdef\th@change{\@input@{thc.sty}}
\gdef\th@margin{\@input@{thm.sty}}
%    \end{macrocode}
%    This list will be expanded when new styles become
%    available.  For testing,  just append new theorem substyles
%    as document options.
% \end{macro}
% \end{macro}
% \end{macro}
% \end{macro}
% \end{macro}
% \end{macro}
%
%
%
% \subsection{Definition of a new theorem set}
%
% As already pointed out, a new theorem environment can be defined in
% three different ways:
% \begin{verbatim}
%    \newtheorem{Lem}{Lemma}
%    \newtheorem{Lem}{Lemma}[section]
%    \newtheorem{Lem}[Theorem]{Lemma}
%\end{verbatim}
% The function of the macro |\newtheorem| is to recognize these cases
% and then to branch into one of the three macros |\@ynthm|, |\@xnthm|
% or |\@othm|.  This mechanism is adopted unchanged from
% \cite{latex:source}; the essential point here is that, for example,
% in the second case, the arguments \texttt{Lem}, \texttt{Lemma} and
% \texttt{section} are passed over to the macro |\@xnthm|.
%
% We inspect this case first because the others present fewer problems,
% and thus are easily derived from this one.
%
% \begin{macro}{\@xnthm}
%    For our example arguments, the macro |\@xnthm| must fulfill
%    the following:
%    \begin{itemize}
%       \item
%          Define a new \LaTeX-counter `Lem'
%       \item
%          reset this counter within a |\section|
%       \item
%          define the macro |\theLem|
%       \item
%          define the environment macros |\Lem| and
%          |\endLem| using the current
%          |\theoremstyle| and |\theorem@bodyfont|.
%    \end{itemize}
%    Obviously, all this should happen only if the first argument
%    of |\@xnthm| (i.e.\ \texttt{Lem} in our example) is chosen so
%    as not to conflict with any previously defined commands or
%    environments.  This test is performed by the \LaTeX{} macro
%    |\@ifdefinable|.
%    \begin{macrocode}
\gdef\@xnthm#1#2[#3]{\expandafter\@ifdefinable\csname #1\endcsname
%    \end{macrocode}
%    Therefore, the first argument of |\@ifdefinable| is the
%    expansion (in the example, |\Lem|) of
%    |\csname||#1\endcsname|.  The second argument is
%    executed only if the test has been completed successfully.
%    \begin{macrocode}
   {%
%    \end{macrocode}
%    Now we define the new counter. The names of the \LaTeX{}
%    macros employed should speak for themselves:
% \changes{v2.2b}{1995/10/07}{Use \cs{@newctr}}
% \changes{v2.2c}{1995/11/23}{Use \cs{@newctr} properly.}
%    \begin{macrocode}
    \@definecounter{#1}\@newctr{#1}[#3]%
%    \end{macrocode}
%    Using |\@newctr| will give a proper error message if the counter
%    in |#3| is not defined.
%    In defining `|\theLem|' we must generate the desired macro
%    name by use of |\expandafter| and |\csname|.
%    \begin{macrocode}
    \expandafter\xdef\csname the#1\endcsname
%    \end{macrocode}
%    An |\xdef| is used in order to make the definition global,
%    and to ensure that it contains the replacement texts of
%    |\@thmcountersep| and |\@thmcounter|.\footnote{These
%    two macros can be defined by the document style.  Their default
%    values produce a `\texttt{.}' as separation and an arabic
%    representation of the number.} However, not everything should be
%    expanded. For example, it saves space to use |\thesection|
%    instead of its---at times---lengthy expansion.
%    \begin{macrocode}
      {\expandafter \noexpand \csname the#3\endcsname
       \@thmcountersep \@thmcounter{#1}}%
%    \end{macrocode}
%    Thus with the defaults of \LaTeX{}, |\theLem| would be
%    replaced by the command sequence |\thesection.\arabic{Lem}|.
%
%    We will now look at the definition of the macro which is executed
%    at the beginning of the actual environment (in our example this
%    macro is |\Lem|). It should be noted that we use an
%    ``|\expandafter| trick'' to expand only certain parts of the
%    replacement text at the time of the definition.
%    \begin{macrocode}
    \def\@tempa{\global\@namedef{#1}}%
    \expandafter \@tempa \expandafter{%
%    \end{macrocode}
%    First, the macro that contains the current definitions of
%    |\@begintheorem| and |\@opargtheorem| should be called
%    up.  The name of this macro---as is already known---has the form
%    |\th@|\meta{theorem\ style}; therefore, it must be called by
%    \begin{macrocode}
      \csname th@\the \theorem@style
%    \end{macrocode}
%    In addition the default theorem font should be changeable, i.e.\
%    we have to insert the contents of |\theorem@bodyfont|. For
%    that reason, we expand even further, beyond |\endcsname|,
%    and thus insert the contents of the \textsf{token} register
%    |\theorem@bodyfont| in the replacement text.
%    \begin{macrocode}
            \expandafter \endcsname \the \theorem@bodyfont
%    \end{macrocode}
%    Now it is time to call the macro |\@thm| which takes over
%    the further processing.  It has two arguments: the current
%    counter name (in our example, \texttt{Lem}), and the text of the
%    label (in our example, \texttt{Lemma}).
%    \begin{macrocode}
     \@thm{#1}{#2}}%
%    \end{macrocode}
%    With this, the `sub-definition' is complete.  The macro
%    |\@endtheorem| ends a theorem environment and is, so far, nothing
%    but an |\endtrivlist|.  (Hence it is defined globally, and not
%    within the theorem styles.\footnote{This has to be changed as
%    soon as theorem styles that change \texttt{\bslash @endtheorem}
%    exist.  In such a case, all existing styles must be changed as
%    well since they will have to reset the macro.}) Therefore, we can
%    set it equivalent to the macro that ends the theorem set (in our
%    example, |\endLem|).  However, if some day theorem styles exist
%    that do change |\@endtheorem|, we would have to use the
%    commented-out line instead.
%    \begin{macrocode}
    \global \expandafter \let \csname end#1\endcsname \@endtheorem
%   \global\@namedef{end#1}{\@endtheorem}%
%    \end{macrocode}
%    With these commands all the required definitions are employed,
%    unless the test |\@ifdefinable| has failed.  Therefore, we
%    end the second argument of this macro and with it the definition
%    of |\@xnthm|.
%    \begin{macrocode}
   }}
%    \end{macrocode}
% \end{macro}
%
% \begin{macro}{\@ynthm}
%    The definition of |\@ynthm| is completely analogous. In
%    this case the new counter that is defined is not reset within
%    another counter; thus the definition of |\the...|
%    is simplified:
%    \begin{macrocode}
\gdef\@ynthm#1#2{\expandafter\@ifdefinable\csname #1\endcsname
   {\@definecounter{#1}%
    \expandafter\xdef\csname the#1\endcsname{\@thmcounter{#1}}%
%    \end{macrocode}
%    The rest of the definition corresponds literally to that of
%    |\@xnthm|:
%    \begin{macrocode}
    \def\@tempa{\global\@namedef{#1}}\expandafter \@tempa
     \expandafter{\csname th@\the \theorem@style \expandafter
     \endcsname \the\theorem@bodyfont \@thm{#1}{#2}}%
    \global \expandafter \let \csname end#1\endcsname \@endtheorem}}
%    \end{macrocode}
% \end{macro}
%
% \begin{macro}{\@othm}
%    The definition of |\@othm| does not contain anything new.
%    \begin{macrocode}
\gdef\@othm#1[#2]#3{%
%    \end{macrocode}
% \changes{v2.2b}{95/10/07}{Build in check for counter defined}
%    We do not define a new counter but instead use one that has
%    already been defined.  Thus the only definition we need is that
%    of this pseudo-counter (i.e.\ |\the|\meta{env.\ name}).
%    First we check if |#2| corresponds to a known counter name.
%    \begin{macrocode}
  \expandafter\ifx\csname c@#2\endcsname\relax
   \@nocounterr{#2}%
  \else
   \expandafter\@ifdefinable\csname #1\endcsname
   {\expandafter \xdef \csname the#1\endcsname
     {\expandafter \noexpand \csname the#2\endcsname}%
%    \end{macrocode}
%    All other parts of the definition can be adopted from
%    |\@xnthm|. We have to remember, though, that in this case
%    the name of the current counter and the theorem label have moved
%    to the second and third arguments.
%    \begin{macrocode}
    \def\@tempa{\global\@namedef{#1}}\expandafter \@tempa
     \expandafter{\csname th@\the \theorem@style \expandafter
     \endcsname \the\theorem@bodyfont \@thm{#2}{#3}}%
    \global \expandafter \let \csname end#1\endcsname \@endtheorem}%
  \fi}
%    \end{macrocode}
% \end{macro}
%
% \subsection{Macros that are employed in a {\envfont theorem}
%    environment}
%
% \begin{macro}{\@thm}
%    The macro |\@thm| has to increase the current counter.  Then,
%    depending on whether the environment has (or does not have) an
%    optional argument, it has to branch into either |\@begintheorem|
%    or |\@opargtheorem|.
%    \begin{macrocode}
\gdef\@thm#1#2{\refstepcounter{#1}%
%    \end{macrocode}
%    Now we start a {\envfont trivlist} environment, and give
%    |\@topsep| and |\@topsepadd| the values of the \textsf{skip}
%    registers |\theorempreskipamount| and |\theorempostskipamount|.
%    The value in |\@topsep| is the vertical space that is inserted by
%    the first (and only) |\item| in our |\trivlist| whilst
%    |\@topsepadd| is inserted by |\@endparenv| at the end of that
%    {\envfont trivlist} environment.  By using these registers, we
%    obtain the desired space around a {\envfont theorem} environment.
%    \begin{macrocode}
   \trivlist
   \@topsep \theorempreskipamount               % used by first \item
   \@topsepadd \theorempostskipamount           % used by \@endparenv
%    \end{macrocode}
%    Now we have to test whether an optional argument has been given.
% \changes{v2.0d}{89/07/18}{Forgotten percent added.}
%    \begin{macrocode}
   \@ifnextchar [%
%    \end{macrocode}
%    If there is an optional argument, we will call |\@ythm|, and
%    move the arguments read back into the input stream.
%    \begin{macrocode}
   {\@ythm{#1}{#2}}%
%    \end{macrocode}
%    If not, we call |\@begintheorem|. Its first argument is the
%    name of the theorem set (hence the second argument of
%    |\@thm|). Its second argument is the macro that produces the
%    current number.
%    \begin{macrocode}
   {\@begintheorem{#2}{\csname the#1\endcsname}\ignorespaces}}
%    \end{macrocode}
% \end{macro}
%
% \begin{macro}{\@xthm}
% \begin{macro}{\@ythm}
%    Both these macros were originally called by |\@thm|.  We do
%    not need |\@xthm| anymore, hence we reset it to
%    |\relax|. The definition of |\@ythm| has not changed at
%    all from its definition in \LaTeX. In order to make the macros
%    easier to understand, we will
%    nevertheless present it (commented out).
%    \begin{macrocode}
\global\let\@xthm\relax
% \def\@ythm#1#2[#3]{\@opargbegintheorem{#2}{\csname
%           the#1\endcsname}{#3}\ignorespaces}
%</package>
%    \end{macrocode}
%    The primitive |\ignorespaces| in |\@ythm| and
%    |\@thm| is needed to remove the spaces between the
%    |\begin{...}| and the actual text.
% \end{macro}
% \end{macro}
%
%
% \subsection{Definition of the theorem substyles}
%
% As already pointed out, the theorem substyles, defined below, are
% only loaded when necessary.  Note that all these substyles, except
% \texttt{plain}, have |\slshape| as the default body font.
%
% \subsubsection{The \texttt{plain} style}
%
% \label{dok:th@plain} As the following macros use |@|, we have
% to locally set the |\catcode| of this symbol to ``letter''.
% This happens within a group, so that we do not have to worry about
% which |\catcode| that symbol had before.
%    \begin{macrocode}
%<*thp>
\begingroup \makeatletter
%    \end{macrocode}
% Since we are now within a group, we must make all definitions
% globally.  First we make sure that \texttt{theorem.sty} is loaded.
% This will allow us to use this file as a document style option
% without having to call \texttt{theorem} itself as an option.  At the
% same time, we assure that at least version 2 is loaded, since
% |\theorem@style| was not defined in earlier versions.
%    \begin{macrocode}
\@ifundefined{theorem@style}{\input{theorem.sty}}{}
\ProvidesFile{thp.sty}
             [\FMithmInfo]
%    \end{macrocode}
%
%
% \begin{macro}{\th@plain}
% \begin{macro}{\@begintheorem}
% \begin{macro}{\@opargbegintheorem}
%    |\theoremstyle{plain}| corresponds to the original
%    definition, except that the distances to the surrounding text are
%    determined by the parameters |\theorempreskipamount| and
%    |\theorempostskipamount|.
%    First we set the default body font.
%    \begin{macrocode}
\gdef\th@plain{\normalfont\itshape
%    \end{macrocode}
%    Then we define |\@begintheorem| and
%    |\@opargbegintheorem|. These two macros define how the
%    header of a theorem is typeset.  |\@opargbegintheorem| will
%    be called if a \textsf{theorem} environment with an optional
%    argument is encountered; otherwise, the header is constructed by
%    calling |\@begintheorem|.  If one of these macros is
%    executed, we are within a \textsf{trivlist} environment started by
%    |\@thm|.  So the theorem header is produced with an
%    |\item| command.
%
%    Instead of specifying the header font directly, all standard
%    theorem styles use the |\theorem@headerfont| macro to allow
%    customization.  The extra space (|\labelsep|) is necessary
%    because of problems in the \textsf{trivlist} environment.
%    \begin{macrocode}
  \def\@begintheorem##1##2{%
        \item[\hskip\labelsep \theorem@headerfont ##1\ ##2]}%
%    \end{macrocode}
%    The definition of |\@opargbegintheorem| is completely
%    analogous. The only difference is the fact that there exists a
%    third argument (which is the optional parameter of the
%    environment and contains additional information about the
%    theorem).  Customarily we enclose it in parentheses.
%    \begin{macrocode}
\def\@opargbegintheorem##1##2##3{%
   \item[\hskip\labelsep \theorem@headerfont ##1\ ##2\ (##3)]}}
%    \end{macrocode}
% \end{macro}
% \end{macro}
% \end{macro}
%
% \noindent
% We conclude with an |\endgroup| to restore the
% |\catcode| of |@|.
%    \begin{macrocode}
\endgroup
%</thp>
%    \end{macrocode}
%
%
% \subsubsection{The \texttt{break} style}
%
% This style option is stored in the file |thb.sty|.
% For the next two lines see the documentation for |\th@plain| on
% page~\pageref{dok:th@plain}.
%    \begin{macrocode}
%<*thb>
\begingroup \makeatletter
\@ifundefined{theorem@style}{\input{theorem.sty}}{}
\ProvidesFile{thb.sty}
             [\FMithmInfo]
%    \end{macrocode}
%
% \begin{macro}{\th@break}
%    |\theoremstyle{break}| produces a line break after the name
%    of the theorem.  The font is |\slshape|.  Hence, we define
%    |\th@break| as follows:
%    \begin{macrocode}
\gdef\th@break{\normalfont\slshape
  \def\@begintheorem##1##2{\item[%
%    \end{macrocode}
%    We run into the following problem: it is not possible to create
%    the header with |\item[|\meta{title}|]| and then start
%    a new line by, for example, |\mbox{}\\|. Such a definition
%    will fail whenever a list environment follows immediately.  With
%    the above construction, the |\mbox{}| causes the switch
%    |@inlabel| (cf.\ definition of |\list| and
%    |\trivlist| in \cite{latex:source}) to be set to \texttt{false}
%    and so the following list will insert additional vertical
%    space (|\topskip|). This is quite annoying. Therefore, we
%    create the line break within the |\item|. In order to ensure
%    that the text will begin at the proper position in the following
%    line, we simply pretend that the label does not take any
%    room.\footnote{This will lead to problems whenever very high
%    symbols occurring in the line tower into the
%    heading. So, something else has to be done here sometime.}
%    \begin{macrocode}
       \rlap{\vbox{\hbox{\hskip \labelsep\theorem@headerfont ##1\ ##2}%
                   \hbox{\strut}}}]}%
%    \end{macrocode}
%    Again, the definition of |\@opargbegintheorem| is
%    completely analogous.
%    \begin{macrocode}
\def\@opargbegintheorem##1##2##3{%
  \item[\rlap{\vbox{\hbox{\hskip \labelsep \theorem@headerfont
                     ##1\ ##2\ (##3)}%
                    \hbox{\strut}}}]}}
\endgroup
%</thb>
%    \end{macrocode}
% \end{macro}
%
%
% \subsubsection{The \texttt{changebreak} style}
%
%    \begin{macrocode}
% This style option is stored in the file |thcb.sty|.
%    \begin{macrocode}
%<*thcb>
\begingroup \makeatletter
\@ifundefined{theorem@style}{\input{theorem.sty}}{}
\ProvidesFile{thcb.sty}
             [\FMithmInfo]
%    \end{macrocode}
%
% \begin{macro}{\th@changebreak}
%    The \texttt{change-break} theorem style is like \texttt{break}
%    but with interchange of theorem name and theorem number.  Thus we
%    define |\th@changebreak| as follows:
%    \begin{macrocode}
\gdef\th@changebreak{\normalfont\slshape
  \def\@begintheorem##1##2{\item
       [\rlap{\vbox{\hbox{\hskip\labelsep \theorem@headerfont ##2\ ##1}%
                    \hbox{\strut}}}]}%
\def\@opargbegintheorem##1##2##3{%
  \item[\rlap{\vbox{\hbox{\hskip\labelsep \theorem@headerfont
                     ##2\ ##1\ (##3)}%
                    \hbox{\strut}}}]}}
\endgroup
%</thcb>
%    \end{macrocode}
% \end{macro}
%
%
% \subsubsection{The \texttt{change} style}
%
% This style option is stored in the file |thc.sty|.
%    \begin{macrocode}
%<*thc>
\begingroup \makeatletter
\@ifundefined{theorem@style}{\input{theorem.sty}}{}
\ProvidesFile{thc.sty}
             [\FMithmInfo]
%    \end{macrocode}
%
% \begin{macro}{\th@change}
%    The \texttt{change} theorem style corresponds to the
%    \texttt{change break} style without a linebreak after the header.
%    To say it in another way, it's the same as the \texttt{plain}
%    style but with number and name interchanged and |\slshape| as the
%    default font.
%    \begin{macrocode}
\gdef\th@change{\normalfont\slshape
  \def\@begintheorem##1##2{\item
       [\hskip\labelsep \theorem@headerfont ##2\ ##1]}%
\def\@opargbegintheorem##1##2##3{%
  \item[\hskip\labelsep \theorem@headerfont ##2\ ##1\ (##3)]}}
\endgroup
%</thc>
%    \end{macrocode}
% \end{macro}
%
%
% \subsubsection{The \texttt{marginbreak} style}
%
% This style option is the one used most often at Mainz.  It is saved
% in the file |thmb.sty|.
%    \begin{macrocode}
%<*thmb>
\begingroup \makeatletter
\@ifundefined{theorem@style}{\input{theorem.sty}}{}
\ProvidesFile{thmb.sty}
             [\FMithmInfo]
%    \end{macrocode}
%
% \begin{macro}{\th@marginbreak}
%    The \texttt{margin break} style is nearly the same as the
%    \texttt{change break} style. The only difference is the placement
%    of the theorem number.  We use |\llap| to place it in the left
%    margin.
%
%    In this style |\labelsep| denotes the separation between the
%    number and the text.
%    \begin{macrocode}
\gdef\th@marginbreak{\normalfont\slshape
  \def\@begintheorem##1##2{\item
       [\rlap{\vbox{\theorem@headerfont
                    \hbox{\llap{##2}\hskip\labelsep ##1}%
                    \hbox{\strut}}}]}%
\def\@opargbegintheorem##1##2##3{%
  \item[\rlap{\vbox{\theorem@headerfont
                    \hbox{\llap{##2}\hskip\labelsep ##1\ (##3)}%
                    \hbox{\strut}}}]}}
\endgroup
%</thmb>
%    \end{macrocode}
% \changes{v2.1c}{90/02/17}{`endgroup outside macrocode env.}
% \end{macro}
%
%
%
% \subsubsection{The \texttt{margin} style}
%
% This style option is stored in the file |thm.sty|.
%    \begin{macrocode}
%<*thm>
\begingroup \makeatletter
\@ifundefined{theorem@style}{\input{theorem.sty}}{}
\ProvidesFile{thm.sty}
             [\FMithmInfo]
%    \end{macrocode}
%
% \begin{macro}{\th@margin}
%    Again this is only a variant of the theorem styles described
%    above without any new ideas.
%    \begin{macrocode}
\gdef\th@margin{\normalfont\slshape
  \def\@begintheorem##1##2{\item
       [\theorem@headerfont \llap{##2}\hskip\labelsep ##1]}%
\def\@opargbegintheorem##1##2##3{%
  \item[\theorem@headerfont \llap{##2}\hskip\labelsep ##1\ (##3)]}}
\endgroup
%</thm>
%    \end{macrocode}
% \end{macro}
%
%
%
%
% \subsection{Final Definitions}
%
% \begin{macro}{\theorempreskipamount}
% \begin{macro}{\theorempostskipamount}
%    The \textsf{skip} parameters that regulate the vertical empty space
%    before and after the {\envfont theorem} environment have
%    to be allocated as well.
% \changes{v2.0f}{89/07/20}{Unnecessary `global' removed.}
%    \begin{macrocode}
%<*package>
\newskip\theorempreskipamount
\newskip\theorempostskipamount
%    \end{macrocode}
%    Since we have used the same values for all theorem sets, we now can
%    assign them.
%    \begin{macrocode}
\global\setlength\theorempreskipamount{12pt plus 5pt minus 3pt}
\global\setlength\theorempostskipamount{8pt plus 3pt minus 1.5pt}
%    \end{macrocode}
% \end{macro}
% \end{macro}
%
%
% \begin{macro}{\@endtheorem}
%    The same holds for the macro |\@endtheorem|, which ends a
%    {\envfont theorem} environment. Since it is the same for all
%    theorem sets, it is removed from the macros |\th@|\meta{style}.
%    It simply ends the {\envfont trivlist} environment, which was
%    begun in |\@thm|.
%    \begin{macrocode}
\global\let\@endtheorem=\endtrivlist
%    \end{macrocode}
% \end{macro}
%
%
%
% \begin{macro}{\@preamblecmds}
%    All macros defined above are to be used only in the preamble.
%    Therefore, we insert them in |\@preamblecmds| which will disable
%    them at begin document. This is done by the internal \LaTeXe{}
%    command |\@onlypreamble|.
% \changes{v2.0g}{89/09/19}{`theoremheaderfont and `theorembodyfont
%    added.}
% \changes{v2.2a}{94/02/03}{Use \LaTeXe{} interface}
%    \begin{macrocode}
\@onlypreamble\@xnthm
\@onlypreamble\@ynthm
\@onlypreamble\@othm
\@onlypreamble\newtheorem
\@onlypreamble\theoremstyle
\@onlypreamble\theorembodyfont
\@onlypreamble\theoremheaderfont
%    \end{macrocode}
% \end{macro}
%
% Finally we declare the \texttt{plain} theorem style to be the default.
% \changes{v2.1b}{90/06/04}{Fixed missing `item by declaring default
%    style.}
%    \begin{macrocode}
\theoremstyle{plain}
%</package>
%    \end{macrocode}
%
% \Finale

\endinput
}{}
\ProvidesFile{thmb.sty}
             [\FMithmInfo]
%    \end{macrocode}
%
% \begin{macro}{\th@marginbreak}
%    The \texttt{margin break} style is nearly the same as the
%    \texttt{change break} style. The only difference is the placement
%    of the theorem number.  We use |\llap| to place it in the left
%    margin.
%
%    In this style |\labelsep| denotes the separation between the
%    number and the text.
%    \begin{macrocode}
\gdef\th@marginbreak{\normalfont\slshape
  \def\@begintheorem##1##2{\item
       [\rlap{\vbox{\theorem@headerfont
                    \hbox{\llap{##2}\hskip\labelsep ##1}%
                    \hbox{\strut}}}]}%
\def\@opargbegintheorem##1##2##3{%
  \item[\rlap{\vbox{\theorem@headerfont
                    \hbox{\llap{##2}\hskip\labelsep ##1\ (##3)}%
                    \hbox{\strut}}}]}}
\endgroup
%</thmb>
%    \end{macrocode}
% \changes{v2.1c}{90/02/17}{`endgroup outside macrocode env.}
% \end{macro}
%
%
%
% \subsubsection{The \texttt{margin} style}
%
% This style option is stored in the file |thm.sty|.
%    \begin{macrocode}
%<*thm>
\begingroup \makeatletter
\@ifundefined{theorem@style}{% \iffalse meta-comment
%
% Copyright (C) 1993-2024
%
% The LaTeX Project and any individual authors listed elsewhere
% in this file.
%
% This file is part of the Standard LaTeX `Tools Bundle'.
% -------------------------------------------------------
%
% It may be distributed and/or modified under the
% conditions of the LaTeX Project Public License, either version 1.3c
% of this license or (at your option) any later version.
% The latest version of this license is in
%    https://www.latex-project.org/lppl.txt
% and version 1.3c or later is part of all distributions of LaTeX
% version 2005/12/01 or later.
%
% The list of all files belonging to the LaTeX `Tools Bundle' is
% given in the file `manifest.txt'.
%
% \fi
%
%%
%
%
\def\FMithmInfo{2023/07/05 v2.2c Theorem extension package (FMi)}
%
% \ProvidesFile{theorem.dtx}[\FMithmInfo]
%\iffalse   % this is a METACOMMENT !
%%
%% Package `theorem' to use with LaTeX2e
%% Copyright (C) 1989-1995 Frank Mittelbach, all rights reserved.
%<+package>\@ifundefined{theorem@style}{}{\endinput}
%<+package>\NeedsTeXFormat{LaTeX2e}[1995/06/01]
%<+package>\ProvidesPackage{theorem}
%<+package>         [\FMithmInfo]
%
% \fi
%
% \changes{v1.0}{87/05/07}{First draft.}
% \changes{v1.0a}{87/05/07}{`thistheoremstyle' added.}
% \changes{v1.1a}{87/07/29}{New concept: redefine the internal
%                          \protect\LaTeX{} commands `@xnthm' etc.}
% \changes{v1.1b}{87/07/31}{`theoremfont' used to specify font.}
% \changes{v1.2a}{88/01/08}{`@endtheorem' now globally defined.}
% \changes{v1.2b}{88/01/11}{`newtheorem' only allowed in the preamble.}
% \changes{v1.3a}{88/02/14}{`theoremfont' now a toks register.}
%
% \changes{v2.0a}{88/11/16}{New implementation and new documentation.}
% \changes{v2.0a}{88/11/16}{No layout changes but internal modifications
%                           old substyles won't work.}
% \changes{v2.0b}{89/01/30}{`global' used for all definitions.}
% \changes{v2.0c}{89/06/15}{Documentation translated to English.
%                            Thanks to LLL and Nina}
% \changes{v2.1a}{90/02/15}{Support for new font selection.}
% \changes{v2.2a}{94/02/03}{Cleanup for LaTeX2e}
%
% \DoNotIndex{\@endparenv,\@ifdefinable,\@ifnextchar,\@ifundefined}
% \DoNotIndex{\@namedef,\@nameuse,\@spaces,\@tempa,\@warning}
% \DoNotIndex{\ }
% \DoNotIndex{\begingroup,\bfseries}
% \DoNotIndex{\csname}
% \DoNotIndex{\def,\do,\docdate}
% \DoNotIndex{\endcsname,\endgroup,\endinput,\endtrivlist,\expandafter}
% \DoNotIndex{\filedate,\fileversion}
% \DoNotIndex{\gdef,\global}
% \DoNotIndex{\hbox,\hskip}
% \DoNotIndex{\ignorespaces,\input,\itshape,\item}
% \DoNotIndex{\labelsep,\let,\llap}
% \DoNotIndex{\makeatletter}
% \DoNotIndex{\noexpand}
% \DoNotIndex{\refstepcounter,\relax,\rlap}
% \DoNotIndex{\setlength,\slshape,\space,\string,\strut}
% \DoNotIndex{\the,\trivlist,\typeout}
% \DoNotIndex{\vbox}
% \DoNotIndex{\xdef}
%
% \def\deflabel#1{\ttfamily #1\hfill}
% \def\deflist#1{\begin{list}{}{\settowidth\labelwidth{\ttfamily #1}%
%                               \setlength\leftmargin\labelwidth
%                               \addtolength\leftmargin\labelsep
%                               \let\makelabel\deflabel}}
% \def\enddeflist{\end{list}}
% \def\envfont{\normalfont\ttfamily}
%
% \MakeShortVerb{\|}
%
% \newcommand{\TUB}{{\sl TUGboat\/}}
%
% \GetFileInfo{theorem.dtx}
% \title{An Extension of the
%        \LaTeX{} theorem environment\thanks{This file
%        has version number \fileversion, last
%        revised \filedate.}}
% \author{Frank Mittelbach}
%
% \MaintainedByLaTeXTeam{tools}
% \maketitle
%
% \begin{abstract}
%    The macros described in this paper yield an extension of the
%    \LaTeX{} theorem mechanism. It is designed to satisfy the
%    different requirements of various journals. Thus, the
%    layout of the ``theorems'' can be manipulated by determining a
%    ``style''. This article describes not only the use, but
%    also the definition, of the necessary macros.
% \end{abstract}
%
% \section*{Preface to version 2.2}
%
% For \LaTeXe{} this package did not need any fundamental changes. I
% only modified the messages generated so that theorem layout styles
% will show up with the |\listfiles| command and cleaned the section
% on the New Font Selection Scheme since this is now included in
% \LaTeX.
%
% \section*{Preface to version 2.1}
%
% This version is identical to 2.0g described in \TUB\ 10\#3 except
% for some internal defaults which are now set depending on the used
% font selection scheme.
%
% This was done to avoid unpleasant surprises if the new font selection
% scheme is in force. For further details see
% section~\ref{sec:newfonts} and \cite{bk:GMS94}.
%
% \section{Introduction}
%
% For our purposes here, ``theorems'' are labelled enunciations,
% often set off from the main text by extra space and a font change.
% Theorems, corollaries, conjectures, definitions, and remarks
% are all instances of ``theorems''.  The ``header'' of these structures
% is composed of a label (such as \textsc{Theorem} or \textsc{Remark})
% and a number which serializes an item in the sequence of items
% with the same label.
%
% Shortly after the introduction of \LaTeX{} at the Fachbereich
% Mathematik in Mainz, the desire to manipulate the layout of
% ``theorems'' arose.  In Mainz, the following two conventions came
% into general use:
% \begin{enumerate}
%   \item The number of the theorem is shown in the margin.
%   \item There is a line break at
%         the end of the theorem header.
% \end{enumerate}
%
% Additionally, some journals require different formats which depend on
% the ``sort of theorem'': e.g.\ often remarks and definitions are set
% in |\upshape|, while |\itshape| is employed for main theorems.
%
% Confronted with these requirements, a theorem environment was
% developed in Mainz which allows separate determination of the
% layout of the ``theorems sets'', comparable to |\pagestyle|.
%
%
% \section{The user interface}
% \subsection{Defining new theorem sets}
%
% \DescribeMacro\newtheorem
% As in the original \LaTeX{} version, the command |\newtheorem|
% defines a new ``theorem set'' or ``theorem-like structure''.
% Two required arguments name the new environment and give the text
% to be typeset with each instance of the new ``set'', while
% an optional argument determines how
% the ``set'' is enumerated:
% \begin{description}
%    \item[\ttfamily \bslash newtheorem\{foo\}\{bar\}]
%       The theorem set {\envfont foo} (whose name is \texttt{bar})
%       uses its own counter.
%    \item[\ttfamily \bslash newtheorem\{foo2\}{[foo]}\{bar2\}]
%       The theorem set {\envfont foo2} (printed name \texttt{bar2})
%       uses the same counter as the theorem set \texttt{foo}.
%    \item[\ttfamily \bslash newtheorem\{foo3\}\{bar3\}{[section]}]
%       The theorem set {\envfont foo3} (printed name \texttt{bar3}) is
%       enumerated within the counter \texttt{section}, i.e.\ with every
%       new |\section| the enumeration begins again with 1, and
%       the enumeration is composed from the section-number and the
%       theorem counter itself.
% \end{description}
%
% \DescribeMacro\theoremstyle
% Additionally, the command |\theoremstyle| can define the layout
% of various, or all, theorem sets.  It should be noted that any
% theorem set defined by |\newtheorem| is typeset in the
% |\theoremstyle| that is current at the time of the definition.
% Thus, the following
% \begin{verbatim}
%    \theoremstyle{break}        \newtheorem{Cor}{Corollary}
%    \theoremstyle{plain}        \newtheorem{Exa}{Example}[section]
%\end{verbatim}                             ^^A No blank at line start
% leads to the result that the set {\envfont Cor} is formatted in the
% style \texttt{break}, while the set {\envfont Exa} and all the
% following ones are formatted in the style \texttt{plain}, unless
% another |\theoremstyle| follows.  Since the definitions installed by
% |\newtheorem| are global, one also can limit |\theoremstyle| locally
% by grouping braces.
%
% \DescribeMacro\theorembodyfont
% The choice of the font for the theorem body is completely
% independent of the chosen |\theoremstyle|; this has proven to
% be very advantageous.  For example,
% \begin{verbatim}
%   {\theorembodyfont{\upshape}          \newtheorem{Rem}{Remark}}
%\end{verbatim}
% defines a theorem set {\envfont Rem}, which will be set in
% |\upshape| in the current layout (which in our example is
% \texttt{plain}).  As with |\theoremstyle|, the |\theorembodyfont|
% chosen is that current at the time of |\newtheorem|.  If
% |\theorembodyfont| is not specified or one defines
% |\theorembodyfont{}|, then the font used will be that defined by the
% |\theoremstyle|.
%
% \DescribeMacro\theoremheaderfont It is also possible to customize
% the font used for the theorem headers.  This is, however, a global
% declaration, and therefore there should be at most one
% |\theoremheaderfont| declaration in the preamble.\footnote{If it is
% actually necessary to have different header fonts, one has to define
% new theorem styles (substituting the desired font) or specify the
% information directly in the \texttt{\bslash newtheorem} declaration
% (the unclean variant).}
%
% \DescribeMacro\theorempreskipamount
% \DescribeMacro\theorempostskipamount Two additional parameters
% affect the vertical space around the theorem environments:
% |\theorempreskipamount| and |\theorempostskipamount| define,
% respectively, the spacing before and after such an environment.
% These parameters apply for all theorem sets and can be manipulated
% with the ordinary length macros.  They are rubber lengths
% (`\textsf{skips}'), and therefore can contain \texttt{plus} and
% \texttt{minus} parts.
%
% Since the definition of theorem sets should---most sensibly---be
% placed in the preamble, we only allow installation there.
% It is therefore possible to release the memory used here after
% |\begin{document}|, in order to make room for other
% applications.
%
%
% \subsection{Existing theorem styles}
%
% The following theorem styles exist to date:
% \begin{deflist}{marginbreak}
%    \item[plain]
%       This theorem style emulates the original \LaTeX{} definition,
%       except that additionally the parameters
%       |\theorem...skipamount| are used.
%    \item[break]
%       In this style, the theorem header is followed by a line
%       break.
%    \item[marginbreak]
%       The theorem number is set in the margin, and there is a line
%       break as in \texttt{break}.
%    \item[changebreak]
%       Like \texttt{break}, but with header number and text
%       interchanged.
%    \item[change]
%       Header number and text are interchanged, without a line break.
%    \item[margin]
%       The number is set in the left margin, without a line break.
% \end{deflist}
% All styles (except \texttt{plain}) select |\slshape| as the default
% |\theorembodyfont|.
%
%
% \subsection{Examples}
%
% Given the above theorem sets {\envfont Cor}, {\envfont Exa}
% and {\envfont Rem},
% suppose that the preamble also contains the declarations:
% \begin{verbatim}
%    \theoremstyle{marginbreak}   \newtheorem{Lem}[Cor]{Lemma}
%    \theoremstyle{change}
%    \theorembodyfont{\itshape}        \newtheorem{Def}[Cor]{Definition}
%
%    \theoremheaderfont{\scshape}
%\end{verbatim}
% Then the following are some typical examples of the typeset output
% resulting from their use.
%
% \begin{Cor}
%    This is a sentence typeset in the theorem environment {\envfont
%    Cor}.
% \end{Cor}
%
% \begin{Exa}
%    This is a sentence typeset in the theorem environment {\envfont
%    Exa}.
% \end{Exa}
%
% \begin{Rem}
%    This is a sentence typeset in the theorem environment {\envfont
%    Rem}.
% \end{Rem}
%
% \begin{Lem}[Ben User]
%    This is a sentence typeset in the theorem environment {\envfont
%    Lem}.
% \end{Lem}
%
% \begin{Def}[Very impressive Definition]
%    This is a sentence typeset in the theorem environment {\envfont
%    Def}.
% \end{Def}
%
% The last two examples show the effect of the optional argument to
% a theorem environment (it is the text typeset in parentheses).
%
%
% \section{Special Considerations}
% \label{sec:newfonts}
%
% Theoremheader and body are implemented as a unit. This means that
% the |\theoremheaderfont|  will inherit characteristics of the
% |\theorembodyfont| in \LaTeXe. Thus, if for example
% |\theorembodyfont| is |\itshape| and |\theoremheaderfont|
% is |\bfseries| the font selected for the header will have the
% characteristics `bold extended italic'. If this is not desired one
% should set the |\theoremheaderfont| to something like
% \begin{verbatim}
%      \theoremheaderfont{\normalfont\bfseries}
%\end{verbatim}
% i.e.\ supplying all necessary font information explicitly.
%
%
%
% \section{Acknowledgements}
%
% The publication of this set of macros was only possible with the
% help of Christina Busse (translating the manuscript into English),
% Joachim Pense (playing the r\^ole of typist), Chris Rowley
% (looking everything over) and many others providing useful
% suggestions.
%
% \MaybeStop{%
%    \begin{thebibliography}{9}
%       \bibitem{bk:GMS94} \textsc{M.~Goossens}, \textsc{F.~Mittelbach}
%         and \textsc{A.~Samarin}.
%         \newblock The \LaTeX{} Companion.
%         \newblock
%         Addison-Wesley, Reading, Massachusetts, 1994.
%       \bibitem{latex:source} \textsc{Lamport, Leslie.} \newblock
%                               \texttt{latex.tex}, version 2.09, date
%                               Feb. 1990.
%    \end{thebibliography}
%    \PrintIndex
%    \PrintChanges}
%
%
% \section{The documentation driver file}
%
% The next bit of code contains the documentation driver file for
% \TeX{}, i.e., the file that will produce the documentation you are
% currently reading. It will be extracted from this file by the
% \texttt{docstrip} program. Since it is the first code in the file
% one can alternatively process this file directly with \LaTeXe{} to
% obtain the documentation.
%
%    \begin{macrocode}
%<*driver>
\documentclass{ltxdoc}

\usepackage{theorem}

%<+driver>% The next few lines define theorem sets which are used
%<+driver>% in the example section of the documentation.

\theoremstyle{break}       \newtheorem{Cor}{Corollary}
\theoremstyle{plain}       \newtheorem{Exa}{Example}[section]
{\theorembodyfont{\upshape}\newtheorem{Rem}{Remark}}
\theoremstyle{marginbreak} \newtheorem{Lem}[Cor]{Lemma}
\theoremstyle{change}
\theorembodyfont{\itshape} \newtheorem{Def}[Cor]{Definition}

\theoremheaderfont{\scshape}

\RecordChanges

\begin{document}
   \DocInput{theorem.dtx}
\end{document}
%</driver>
%    \end{macrocode}
%
% \section{Definition of the Macros}
%
% If the file has been loaded before, we abort immediately. If not the
% package announces itself (this is actually done at the very top of
% the file---the way it is done isn't good style so don't copy it).
% \changes{v2.0e}{89/07/19}{Spaces removed from `typeout'.}
% \changes{v2.2b}{95/11/19}{Announce moved to top of file}
%    \begin{macrocode}
%<*package>
%\@ifundefined{theorem@style}{}{\endinput}
%\def\FMithmInfo{1995/11/19 v2.2b Theorem extension package (FMi)}
%\ProvidesPackage{theorem}[\FMithmInfo]
%    \end{macrocode}
%
%
% \subsection{Definition of theorem styles and fonts}
%
% All the definitions in this file are done globally to allow
% inputting this file inside a group.
%
% \begin{macro}{\theoremstyle}
%    Before a theorem style can be installed, the chosen style must be
%    known. For that reason, we must test to see that
%    |\th@|\meta{style} is known or, more precisely, that it is
%    different from |\relax|. If the style is not known then
%    |\th@plain| is used.
%    \begin{macrocode}
\gdef\theoremstyle#1{%
   \@ifundefined{th@#1}{\@warning
          {Unknown theoremstyle `#1'. Using `plain'}%
          \theorem@style{plain}}%
%    \end{macrocode}
%    We save the theorem style to be used in the token register
%    |\theorem@style|.
%    \begin{macrocode}
      {\theorem@style{#1}}%
%    \end{macrocode}
%    Now we ``evaluate'' the theorem style: this means, we call the
%    macro |\th@|\meta{style} which will activate the relevant
%    definitions which are contained in a separate file.  This is done
%    in a group to suppress changes to the current font.  This could
%    otherwise pose problems together with the new font selection
%    scheme\footnote{When I printed the original article using the new
%    font selection scheme I ended with a document with slanted
%    typefaces (text headings and all) simply because one of the
%    theorem styles used \texttt{\bslash sl} at toplevel.} if the
%    |\th@|\meta{style} is evaluated a second time.
% \changes{v2.1a}{90/02/15}{Group added around `th@..}
% \changes{v2.1a}{90/02/15}{Reseting of `..@begintheorem removed.}
%    \begin{macrocode}
      \begingroup
        \csname th@\the\theorem@style \endcsname
      \endgroup}
%    \end{macrocode}
% \end{macro}
%
%
% \begin{macro}{\@begintheorem}
% \changes{v2.1a}{90/02/15}{Relaxed at toplevel.}
% \begin{macro}{\@opargbegintheorem}
% \changes{v2.1a}{90/02/15}{Relaxed at toplevel.}
%    We reset |\@begintheorem| and |\@opargbegintheorem| to
%    |\relax| since these commands are no longer necessary at
%    toplevel. This will save a few tokens.
%    \begin{macrocode}
\global\let\@begintheorem\relax
\global\let\@opargbegintheorem\relax
%    \end{macrocode}
% \end{macro}
% \end{macro}
%
% \begin{macro}{\theorem@style}
%    Obviously the \textsf{token} register used above has to be
%    allocated. To assure the utmost compatibility with the
%    original \LaTeX{} definition, we set the default theorem style to
%    \texttt{plain}, which implements the usual \LaTeX{} convention.
% \changes{v2.0f}{89/07/20}{Unnecessary `global' removed.}
%    \begin{macrocode}
\newtoks\theorem@style
\global\theorem@style{plain}
%    \end{macrocode}
% \end{macro}
%
% \begin{macro}{\theorembodyfont}
% \begin{macro}{\theorem@bodyfont}
%    If the |\theorembodyfont| is set by the user then it should not
%    interact with the default font set in the theorem style. When the
%    new font selection is in force this may happen if, for example,
%    the default is |\itshape| and the new |\theorembodyfont| is
%    |\sffamily|. So we add a |\reset@font| command in front of the
%    user definition.
% \changes{v2.0f}{89/07/20}{Unnecessary `global' removed.}
% \changes{v2.0g}{89/09/19}{`theoremfont to `theorembodyfont.}
% \changes{v2.1c}{92/06/22}{`reset@font added and logic changed.}
% \changes{v2.1d}{92/10/28}{Reset to empty if arg is empty}
%    \begin{macrocode}
\gdef\theorembodyfont#1{%
%    \end{macrocode}
%    We check if the argument supplied is empty and if so put nothing
%    into the |\theorem@bodyfont| token register to allow for
%    |\theorembodyfont{}| as a mean of using the default of the
%    current |\theoremstyle|.
%    \begin{macrocode}
   \def\@tempa{#1}%
   \ifx\@tempa\@empty
    \theorem@bodyfont{}%
   \else
    \theorem@bodyfont{\reset@font#1}%
   \fi
   }
\newtoks\theorem@bodyfont
\global\theorem@bodyfont{}
%    \end{macrocode}
% \end{macro}
% \end{macro}
%
%
% \begin{macro}{\theoremheaderfont}
% \changes{v2.0g}{89/09/19}{Macro added.}
%    The font for the theorem headers is handled differently because
%    this definition applies to all theorem styles.
%    \begin{macrocode}
\gdef\theoremheaderfont#1{\gdef\theorem@headerfont{#1}%
%    \end{macrocode}
%    After using the macro once it is redefined to produce an error
%    message.
%    \begin{macrocode}
       \gdef\theoremheaderfont##1{%
        \typeout{\string\theoremheaderfont\space should be used
                 only once.}}}
%    \end{macrocode}
% \end{macro}
%
% \begin{macro}{\theorem@headerfont}
% \changes{v2.0g}{89/09/19}{Macro added.}
%    To set the |\theorem@headerfont| default we first test if
%    the new fontselection scheme is in force.
% \changes{v2.1a}{90/02/15}{Support for new font selection.}
%    \begin{macrocode}
\ifx\upshape\undefined
%    \end{macrocode}
%    If not we define it to expand into |\bfseries|. We don't use
%    |\let| just in case a following style option redefines this
%    macro.
%    \begin{macrocode}
\gdef\theorem@headerfont{\bfseries}
%    \end{macrocode}
%    Otherwise we reset the current shape before calling |\bfseries|.
%    \begin{macrocode}
\else \gdef\theorem@headerfont{\normalfont\bfseries}\fi
%    \end{macrocode}
% \end{macro}
%
% \begin{macro}{\th@plain}
% \begin{macro}{\th@break}
% \begin{macro}{\th@marginbreak}
% \begin{macro}{\th@changebreak}
% \begin{macro}{\th@change}
% \begin{macro}{\th@margin}
%    The different styles are defined in macros such as
%    |\th@plain|.  Since memory space is precious in
%    ``non-Big-versions'', we have to avoid offering too many unused
%    definitions.  Therefore we define these styles in separate files
%    that can be loaded on demand.  Thus the commands themselves only
%    load these files. We use |\@input@|, a \LaTeXe{} internal command
%    that ensures that the file will be listed with |\listfiles|.
% \changes{v2.2a}{94/02/02}{Use `@input@ to load theorem layout files}
%    \begin{macrocode}
\gdef\th@plain{\@input@{thp.sty}}
\gdef\th@break{\@input@{thb.sty}}
\gdef\th@marginbreak{\@input@{thmb.sty}}
\gdef\th@changebreak{\@input@{thcb.sty}}
\gdef\th@change{\@input@{thc.sty}}
\gdef\th@margin{\@input@{thm.sty}}
%    \end{macrocode}
%    This list will be expanded when new styles become
%    available.  For testing,  just append new theorem substyles
%    as document options.
% \end{macro}
% \end{macro}
% \end{macro}
% \end{macro}
% \end{macro}
% \end{macro}
%
%
%
% \subsection{Definition of a new theorem set}
%
% As already pointed out, a new theorem environment can be defined in
% three different ways:
% \begin{verbatim}
%    \newtheorem{Lem}{Lemma}
%    \newtheorem{Lem}{Lemma}[section]
%    \newtheorem{Lem}[Theorem]{Lemma}
%\end{verbatim}
% The function of the macro |\newtheorem| is to recognize these cases
% and then to branch into one of the three macros |\@ynthm|, |\@xnthm|
% or |\@othm|.  This mechanism is adopted unchanged from
% \cite{latex:source}; the essential point here is that, for example,
% in the second case, the arguments \texttt{Lem}, \texttt{Lemma} and
% \texttt{section} are passed over to the macro |\@xnthm|.
%
% We inspect this case first because the others present fewer problems,
% and thus are easily derived from this one.
%
% \begin{macro}{\@xnthm}
%    For our example arguments, the macro |\@xnthm| must fulfill
%    the following:
%    \begin{itemize}
%       \item
%          Define a new \LaTeX-counter `Lem'
%       \item
%          reset this counter within a |\section|
%       \item
%          define the macro |\theLem|
%       \item
%          define the environment macros |\Lem| and
%          |\endLem| using the current
%          |\theoremstyle| and |\theorem@bodyfont|.
%    \end{itemize}
%    Obviously, all this should happen only if the first argument
%    of |\@xnthm| (i.e.\ \texttt{Lem} in our example) is chosen so
%    as not to conflict with any previously defined commands or
%    environments.  This test is performed by the \LaTeX{} macro
%    |\@ifdefinable|.
%    \begin{macrocode}
\gdef\@xnthm#1#2[#3]{\expandafter\@ifdefinable\csname #1\endcsname
%    \end{macrocode}
%    Therefore, the first argument of |\@ifdefinable| is the
%    expansion (in the example, |\Lem|) of
%    |\csname||#1\endcsname|.  The second argument is
%    executed only if the test has been completed successfully.
%    \begin{macrocode}
   {%
%    \end{macrocode}
%    Now we define the new counter. The names of the \LaTeX{}
%    macros employed should speak for themselves:
% \changes{v2.2b}{1995/10/07}{Use \cs{@newctr}}
% \changes{v2.2c}{1995/11/23}{Use \cs{@newctr} properly.}
%    \begin{macrocode}
    \@definecounter{#1}\@newctr{#1}[#3]%
%    \end{macrocode}
%    Using |\@newctr| will give a proper error message if the counter
%    in |#3| is not defined.
%    In defining `|\theLem|' we must generate the desired macro
%    name by use of |\expandafter| and |\csname|.
%    \begin{macrocode}
    \expandafter\xdef\csname the#1\endcsname
%    \end{macrocode}
%    An |\xdef| is used in order to make the definition global,
%    and to ensure that it contains the replacement texts of
%    |\@thmcountersep| and |\@thmcounter|.\footnote{These
%    two macros can be defined by the document style.  Their default
%    values produce a `\texttt{.}' as separation and an arabic
%    representation of the number.} However, not everything should be
%    expanded. For example, it saves space to use |\thesection|
%    instead of its---at times---lengthy expansion.
%    \begin{macrocode}
      {\expandafter \noexpand \csname the#3\endcsname
       \@thmcountersep \@thmcounter{#1}}%
%    \end{macrocode}
%    Thus with the defaults of \LaTeX{}, |\theLem| would be
%    replaced by the command sequence |\thesection.\arabic{Lem}|.
%
%    We will now look at the definition of the macro which is executed
%    at the beginning of the actual environment (in our example this
%    macro is |\Lem|). It should be noted that we use an
%    ``|\expandafter| trick'' to expand only certain parts of the
%    replacement text at the time of the definition.
%    \begin{macrocode}
    \def\@tempa{\global\@namedef{#1}}%
    \expandafter \@tempa \expandafter{%
%    \end{macrocode}
%    First, the macro that contains the current definitions of
%    |\@begintheorem| and |\@opargtheorem| should be called
%    up.  The name of this macro---as is already known---has the form
%    |\th@|\meta{theorem\ style}; therefore, it must be called by
%    \begin{macrocode}
      \csname th@\the \theorem@style
%    \end{macrocode}
%    In addition the default theorem font should be changeable, i.e.\
%    we have to insert the contents of |\theorem@bodyfont|. For
%    that reason, we expand even further, beyond |\endcsname|,
%    and thus insert the contents of the \textsf{token} register
%    |\theorem@bodyfont| in the replacement text.
%    \begin{macrocode}
            \expandafter \endcsname \the \theorem@bodyfont
%    \end{macrocode}
%    Now it is time to call the macro |\@thm| which takes over
%    the further processing.  It has two arguments: the current
%    counter name (in our example, \texttt{Lem}), and the text of the
%    label (in our example, \texttt{Lemma}).
%    \begin{macrocode}
     \@thm{#1}{#2}}%
%    \end{macrocode}
%    With this, the `sub-definition' is complete.  The macro
%    |\@endtheorem| ends a theorem environment and is, so far, nothing
%    but an |\endtrivlist|.  (Hence it is defined globally, and not
%    within the theorem styles.\footnote{This has to be changed as
%    soon as theorem styles that change \texttt{\bslash @endtheorem}
%    exist.  In such a case, all existing styles must be changed as
%    well since they will have to reset the macro.}) Therefore, we can
%    set it equivalent to the macro that ends the theorem set (in our
%    example, |\endLem|).  However, if some day theorem styles exist
%    that do change |\@endtheorem|, we would have to use the
%    commented-out line instead.
%    \begin{macrocode}
    \global \expandafter \let \csname end#1\endcsname \@endtheorem
%   \global\@namedef{end#1}{\@endtheorem}%
%    \end{macrocode}
%    With these commands all the required definitions are employed,
%    unless the test |\@ifdefinable| has failed.  Therefore, we
%    end the second argument of this macro and with it the definition
%    of |\@xnthm|.
%    \begin{macrocode}
   }}
%    \end{macrocode}
% \end{macro}
%
% \begin{macro}{\@ynthm}
%    The definition of |\@ynthm| is completely analogous. In
%    this case the new counter that is defined is not reset within
%    another counter; thus the definition of |\the...|
%    is simplified:
%    \begin{macrocode}
\gdef\@ynthm#1#2{\expandafter\@ifdefinable\csname #1\endcsname
   {\@definecounter{#1}%
    \expandafter\xdef\csname the#1\endcsname{\@thmcounter{#1}}%
%    \end{macrocode}
%    The rest of the definition corresponds literally to that of
%    |\@xnthm|:
%    \begin{macrocode}
    \def\@tempa{\global\@namedef{#1}}\expandafter \@tempa
     \expandafter{\csname th@\the \theorem@style \expandafter
     \endcsname \the\theorem@bodyfont \@thm{#1}{#2}}%
    \global \expandafter \let \csname end#1\endcsname \@endtheorem}}
%    \end{macrocode}
% \end{macro}
%
% \begin{macro}{\@othm}
%    The definition of |\@othm| does not contain anything new.
%    \begin{macrocode}
\gdef\@othm#1[#2]#3{%
%    \end{macrocode}
% \changes{v2.2b}{95/10/07}{Build in check for counter defined}
%    We do not define a new counter but instead use one that has
%    already been defined.  Thus the only definition we need is that
%    of this pseudo-counter (i.e.\ |\the|\meta{env.\ name}).
%    First we check if |#2| corresponds to a known counter name.
%    \begin{macrocode}
  \expandafter\ifx\csname c@#2\endcsname\relax
   \@nocounterr{#2}%
  \else
   \expandafter\@ifdefinable\csname #1\endcsname
   {\expandafter \xdef \csname the#1\endcsname
     {\expandafter \noexpand \csname the#2\endcsname}%
%    \end{macrocode}
%    All other parts of the definition can be adopted from
%    |\@xnthm|. We have to remember, though, that in this case
%    the name of the current counter and the theorem label have moved
%    to the second and third arguments.
%    \begin{macrocode}
    \def\@tempa{\global\@namedef{#1}}\expandafter \@tempa
     \expandafter{\csname th@\the \theorem@style \expandafter
     \endcsname \the\theorem@bodyfont \@thm{#2}{#3}}%
    \global \expandafter \let \csname end#1\endcsname \@endtheorem}%
  \fi}
%    \end{macrocode}
% \end{macro}
%
% \subsection{Macros that are employed in a {\envfont theorem}
%    environment}
%
% \begin{macro}{\@thm}
%    The macro |\@thm| has to increase the current counter.  Then,
%    depending on whether the environment has (or does not have) an
%    optional argument, it has to branch into either |\@begintheorem|
%    or |\@opargtheorem|.
%    \begin{macrocode}
\gdef\@thm#1#2{\refstepcounter{#1}%
%    \end{macrocode}
%    Now we start a {\envfont trivlist} environment, and give
%    |\@topsep| and |\@topsepadd| the values of the \textsf{skip}
%    registers |\theorempreskipamount| and |\theorempostskipamount|.
%    The value in |\@topsep| is the vertical space that is inserted by
%    the first (and only) |\item| in our |\trivlist| whilst
%    |\@topsepadd| is inserted by |\@endparenv| at the end of that
%    {\envfont trivlist} environment.  By using these registers, we
%    obtain the desired space around a {\envfont theorem} environment.
%    \begin{macrocode}
   \trivlist
   \@topsep \theorempreskipamount               % used by first \item
   \@topsepadd \theorempostskipamount           % used by \@endparenv
%    \end{macrocode}
%    Now we have to test whether an optional argument has been given.
% \changes{v2.0d}{89/07/18}{Forgotten percent added.}
%    \begin{macrocode}
   \@ifnextchar [%
%    \end{macrocode}
%    If there is an optional argument, we will call |\@ythm|, and
%    move the arguments read back into the input stream.
%    \begin{macrocode}
   {\@ythm{#1}{#2}}%
%    \end{macrocode}
%    If not, we call |\@begintheorem|. Its first argument is the
%    name of the theorem set (hence the second argument of
%    |\@thm|). Its second argument is the macro that produces the
%    current number.
%    \begin{macrocode}
   {\@begintheorem{#2}{\csname the#1\endcsname}\ignorespaces}}
%    \end{macrocode}
% \end{macro}
%
% \begin{macro}{\@xthm}
% \begin{macro}{\@ythm}
%    Both these macros were originally called by |\@thm|.  We do
%    not need |\@xthm| anymore, hence we reset it to
%    |\relax|. The definition of |\@ythm| has not changed at
%    all from its definition in \LaTeX. In order to make the macros
%    easier to understand, we will
%    nevertheless present it (commented out).
%    \begin{macrocode}
\global\let\@xthm\relax
% \def\@ythm#1#2[#3]{\@opargbegintheorem{#2}{\csname
%           the#1\endcsname}{#3}\ignorespaces}
%</package>
%    \end{macrocode}
%    The primitive |\ignorespaces| in |\@ythm| and
%    |\@thm| is needed to remove the spaces between the
%    |\begin{...}| and the actual text.
% \end{macro}
% \end{macro}
%
%
% \subsection{Definition of the theorem substyles}
%
% As already pointed out, the theorem substyles, defined below, are
% only loaded when necessary.  Note that all these substyles, except
% \texttt{plain}, have |\slshape| as the default body font.
%
% \subsubsection{The \texttt{plain} style}
%
% \label{dok:th@plain} As the following macros use |@|, we have
% to locally set the |\catcode| of this symbol to ``letter''.
% This happens within a group, so that we do not have to worry about
% which |\catcode| that symbol had before.
%    \begin{macrocode}
%<*thp>
\begingroup \makeatletter
%    \end{macrocode}
% Since we are now within a group, we must make all definitions
% globally.  First we make sure that \texttt{theorem.sty} is loaded.
% This will allow us to use this file as a document style option
% without having to call \texttt{theorem} itself as an option.  At the
% same time, we assure that at least version 2 is loaded, since
% |\theorem@style| was not defined in earlier versions.
%    \begin{macrocode}
\@ifundefined{theorem@style}{\input{theorem.sty}}{}
\ProvidesFile{thp.sty}
             [\FMithmInfo]
%    \end{macrocode}
%
%
% \begin{macro}{\th@plain}
% \begin{macro}{\@begintheorem}
% \begin{macro}{\@opargbegintheorem}
%    |\theoremstyle{plain}| corresponds to the original
%    definition, except that the distances to the surrounding text are
%    determined by the parameters |\theorempreskipamount| and
%    |\theorempostskipamount|.
%    First we set the default body font.
%    \begin{macrocode}
\gdef\th@plain{\normalfont\itshape
%    \end{macrocode}
%    Then we define |\@begintheorem| and
%    |\@opargbegintheorem|. These two macros define how the
%    header of a theorem is typeset.  |\@opargbegintheorem| will
%    be called if a \textsf{theorem} environment with an optional
%    argument is encountered; otherwise, the header is constructed by
%    calling |\@begintheorem|.  If one of these macros is
%    executed, we are within a \textsf{trivlist} environment started by
%    |\@thm|.  So the theorem header is produced with an
%    |\item| command.
%
%    Instead of specifying the header font directly, all standard
%    theorem styles use the |\theorem@headerfont| macro to allow
%    customization.  The extra space (|\labelsep|) is necessary
%    because of problems in the \textsf{trivlist} environment.
%    \begin{macrocode}
  \def\@begintheorem##1##2{%
        \item[\hskip\labelsep \theorem@headerfont ##1\ ##2]}%
%    \end{macrocode}
%    The definition of |\@opargbegintheorem| is completely
%    analogous. The only difference is the fact that there exists a
%    third argument (which is the optional parameter of the
%    environment and contains additional information about the
%    theorem).  Customarily we enclose it in parentheses.
%    \begin{macrocode}
\def\@opargbegintheorem##1##2##3{%
   \item[\hskip\labelsep \theorem@headerfont ##1\ ##2\ (##3)]}}
%    \end{macrocode}
% \end{macro}
% \end{macro}
% \end{macro}
%
% \noindent
% We conclude with an |\endgroup| to restore the
% |\catcode| of |@|.
%    \begin{macrocode}
\endgroup
%</thp>
%    \end{macrocode}
%
%
% \subsubsection{The \texttt{break} style}
%
% This style option is stored in the file |thb.sty|.
% For the next two lines see the documentation for |\th@plain| on
% page~\pageref{dok:th@plain}.
%    \begin{macrocode}
%<*thb>
\begingroup \makeatletter
\@ifundefined{theorem@style}{\input{theorem.sty}}{}
\ProvidesFile{thb.sty}
             [\FMithmInfo]
%    \end{macrocode}
%
% \begin{macro}{\th@break}
%    |\theoremstyle{break}| produces a line break after the name
%    of the theorem.  The font is |\slshape|.  Hence, we define
%    |\th@break| as follows:
%    \begin{macrocode}
\gdef\th@break{\normalfont\slshape
  \def\@begintheorem##1##2{\item[%
%    \end{macrocode}
%    We run into the following problem: it is not possible to create
%    the header with |\item[|\meta{title}|]| and then start
%    a new line by, for example, |\mbox{}\\|. Such a definition
%    will fail whenever a list environment follows immediately.  With
%    the above construction, the |\mbox{}| causes the switch
%    |@inlabel| (cf.\ definition of |\list| and
%    |\trivlist| in \cite{latex:source}) to be set to \texttt{false}
%    and so the following list will insert additional vertical
%    space (|\topskip|). This is quite annoying. Therefore, we
%    create the line break within the |\item|. In order to ensure
%    that the text will begin at the proper position in the following
%    line, we simply pretend that the label does not take any
%    room.\footnote{This will lead to problems whenever very high
%    symbols occurring in the line tower into the
%    heading. So, something else has to be done here sometime.}
%    \begin{macrocode}
       \rlap{\vbox{\hbox{\hskip \labelsep\theorem@headerfont ##1\ ##2}%
                   \hbox{\strut}}}]}%
%    \end{macrocode}
%    Again, the definition of |\@opargbegintheorem| is
%    completely analogous.
%    \begin{macrocode}
\def\@opargbegintheorem##1##2##3{%
  \item[\rlap{\vbox{\hbox{\hskip \labelsep \theorem@headerfont
                     ##1\ ##2\ (##3)}%
                    \hbox{\strut}}}]}}
\endgroup
%</thb>
%    \end{macrocode}
% \end{macro}
%
%
% \subsubsection{The \texttt{changebreak} style}
%
%    \begin{macrocode}
% This style option is stored in the file |thcb.sty|.
%    \begin{macrocode}
%<*thcb>
\begingroup \makeatletter
\@ifundefined{theorem@style}{\input{theorem.sty}}{}
\ProvidesFile{thcb.sty}
             [\FMithmInfo]
%    \end{macrocode}
%
% \begin{macro}{\th@changebreak}
%    The \texttt{change-break} theorem style is like \texttt{break}
%    but with interchange of theorem name and theorem number.  Thus we
%    define |\th@changebreak| as follows:
%    \begin{macrocode}
\gdef\th@changebreak{\normalfont\slshape
  \def\@begintheorem##1##2{\item
       [\rlap{\vbox{\hbox{\hskip\labelsep \theorem@headerfont ##2\ ##1}%
                    \hbox{\strut}}}]}%
\def\@opargbegintheorem##1##2##3{%
  \item[\rlap{\vbox{\hbox{\hskip\labelsep \theorem@headerfont
                     ##2\ ##1\ (##3)}%
                    \hbox{\strut}}}]}}
\endgroup
%</thcb>
%    \end{macrocode}
% \end{macro}
%
%
% \subsubsection{The \texttt{change} style}
%
% This style option is stored in the file |thc.sty|.
%    \begin{macrocode}
%<*thc>
\begingroup \makeatletter
\@ifundefined{theorem@style}{\input{theorem.sty}}{}
\ProvidesFile{thc.sty}
             [\FMithmInfo]
%    \end{macrocode}
%
% \begin{macro}{\th@change}
%    The \texttt{change} theorem style corresponds to the
%    \texttt{change break} style without a linebreak after the header.
%    To say it in another way, it's the same as the \texttt{plain}
%    style but with number and name interchanged and |\slshape| as the
%    default font.
%    \begin{macrocode}
\gdef\th@change{\normalfont\slshape
  \def\@begintheorem##1##2{\item
       [\hskip\labelsep \theorem@headerfont ##2\ ##1]}%
\def\@opargbegintheorem##1##2##3{%
  \item[\hskip\labelsep \theorem@headerfont ##2\ ##1\ (##3)]}}
\endgroup
%</thc>
%    \end{macrocode}
% \end{macro}
%
%
% \subsubsection{The \texttt{marginbreak} style}
%
% This style option is the one used most often at Mainz.  It is saved
% in the file |thmb.sty|.
%    \begin{macrocode}
%<*thmb>
\begingroup \makeatletter
\@ifundefined{theorem@style}{\input{theorem.sty}}{}
\ProvidesFile{thmb.sty}
             [\FMithmInfo]
%    \end{macrocode}
%
% \begin{macro}{\th@marginbreak}
%    The \texttt{margin break} style is nearly the same as the
%    \texttt{change break} style. The only difference is the placement
%    of the theorem number.  We use |\llap| to place it in the left
%    margin.
%
%    In this style |\labelsep| denotes the separation between the
%    number and the text.
%    \begin{macrocode}
\gdef\th@marginbreak{\normalfont\slshape
  \def\@begintheorem##1##2{\item
       [\rlap{\vbox{\theorem@headerfont
                    \hbox{\llap{##2}\hskip\labelsep ##1}%
                    \hbox{\strut}}}]}%
\def\@opargbegintheorem##1##2##3{%
  \item[\rlap{\vbox{\theorem@headerfont
                    \hbox{\llap{##2}\hskip\labelsep ##1\ (##3)}%
                    \hbox{\strut}}}]}}
\endgroup
%</thmb>
%    \end{macrocode}
% \changes{v2.1c}{90/02/17}{`endgroup outside macrocode env.}
% \end{macro}
%
%
%
% \subsubsection{The \texttt{margin} style}
%
% This style option is stored in the file |thm.sty|.
%    \begin{macrocode}
%<*thm>
\begingroup \makeatletter
\@ifundefined{theorem@style}{\input{theorem.sty}}{}
\ProvidesFile{thm.sty}
             [\FMithmInfo]
%    \end{macrocode}
%
% \begin{macro}{\th@margin}
%    Again this is only a variant of the theorem styles described
%    above without any new ideas.
%    \begin{macrocode}
\gdef\th@margin{\normalfont\slshape
  \def\@begintheorem##1##2{\item
       [\theorem@headerfont \llap{##2}\hskip\labelsep ##1]}%
\def\@opargbegintheorem##1##2##3{%
  \item[\theorem@headerfont \llap{##2}\hskip\labelsep ##1\ (##3)]}}
\endgroup
%</thm>
%    \end{macrocode}
% \end{macro}
%
%
%
%
% \subsection{Final Definitions}
%
% \begin{macro}{\theorempreskipamount}
% \begin{macro}{\theorempostskipamount}
%    The \textsf{skip} parameters that regulate the vertical empty space
%    before and after the {\envfont theorem} environment have
%    to be allocated as well.
% \changes{v2.0f}{89/07/20}{Unnecessary `global' removed.}
%    \begin{macrocode}
%<*package>
\newskip\theorempreskipamount
\newskip\theorempostskipamount
%    \end{macrocode}
%    Since we have used the same values for all theorem sets, we now can
%    assign them.
%    \begin{macrocode}
\global\setlength\theorempreskipamount{12pt plus 5pt minus 3pt}
\global\setlength\theorempostskipamount{8pt plus 3pt minus 1.5pt}
%    \end{macrocode}
% \end{macro}
% \end{macro}
%
%
% \begin{macro}{\@endtheorem}
%    The same holds for the macro |\@endtheorem|, which ends a
%    {\envfont theorem} environment. Since it is the same for all
%    theorem sets, it is removed from the macros |\th@|\meta{style}.
%    It simply ends the {\envfont trivlist} environment, which was
%    begun in |\@thm|.
%    \begin{macrocode}
\global\let\@endtheorem=\endtrivlist
%    \end{macrocode}
% \end{macro}
%
%
%
% \begin{macro}{\@preamblecmds}
%    All macros defined above are to be used only in the preamble.
%    Therefore, we insert them in |\@preamblecmds| which will disable
%    them at begin document. This is done by the internal \LaTeXe{}
%    command |\@onlypreamble|.
% \changes{v2.0g}{89/09/19}{`theoremheaderfont and `theorembodyfont
%    added.}
% \changes{v2.2a}{94/02/03}{Use \LaTeXe{} interface}
%    \begin{macrocode}
\@onlypreamble\@xnthm
\@onlypreamble\@ynthm
\@onlypreamble\@othm
\@onlypreamble\newtheorem
\@onlypreamble\theoremstyle
\@onlypreamble\theorembodyfont
\@onlypreamble\theoremheaderfont
%    \end{macrocode}
% \end{macro}
%
% Finally we declare the \texttt{plain} theorem style to be the default.
% \changes{v2.1b}{90/06/04}{Fixed missing `item by declaring default
%    style.}
%    \begin{macrocode}
\theoremstyle{plain}
%</package>
%    \end{macrocode}
%
% \Finale

\endinput
}{}
\ProvidesFile{thm.sty}
             [\FMithmInfo]
%    \end{macrocode}
%
% \begin{macro}{\th@margin}
%    Again this is only a variant of the theorem styles described
%    above without any new ideas.
%    \begin{macrocode}
\gdef\th@margin{\normalfont\slshape
  \def\@begintheorem##1##2{\item
       [\theorem@headerfont \llap{##2}\hskip\labelsep ##1]}%
\def\@opargbegintheorem##1##2##3{%
  \item[\theorem@headerfont \llap{##2}\hskip\labelsep ##1\ (##3)]}}
\endgroup
%</thm>
%    \end{macrocode}
% \end{macro}
%
%
%
%
% \subsection{Final Definitions}
%
% \begin{macro}{\theorempreskipamount}
% \begin{macro}{\theorempostskipamount}
%    The \textsf{skip} parameters that regulate the vertical empty space
%    before and after the {\envfont theorem} environment have
%    to be allocated as well.
% \changes{v2.0f}{89/07/20}{Unnecessary `global' removed.}
%    \begin{macrocode}
%<*package>
\newskip\theorempreskipamount
\newskip\theorempostskipamount
%    \end{macrocode}
%    Since we have used the same values for all theorem sets, we now can
%    assign them.
%    \begin{macrocode}
\global\setlength\theorempreskipamount{12pt plus 5pt minus 3pt}
\global\setlength\theorempostskipamount{8pt plus 3pt minus 1.5pt}
%    \end{macrocode}
% \end{macro}
% \end{macro}
%
%
% \begin{macro}{\@endtheorem}
%    The same holds for the macro |\@endtheorem|, which ends a
%    {\envfont theorem} environment. Since it is the same for all
%    theorem sets, it is removed from the macros |\th@|\meta{style}.
%    It simply ends the {\envfont trivlist} environment, which was
%    begun in |\@thm|.
%    \begin{macrocode}
\global\let\@endtheorem=\endtrivlist
%    \end{macrocode}
% \end{macro}
%
%
%
% \begin{macro}{\@preamblecmds}
%    All macros defined above are to be used only in the preamble.
%    Therefore, we insert them in |\@preamblecmds| which will disable
%    them at begin document. This is done by the internal \LaTeXe{}
%    command |\@onlypreamble|.
% \changes{v2.0g}{89/09/19}{`theoremheaderfont and `theorembodyfont
%    added.}
% \changes{v2.2a}{94/02/03}{Use \LaTeXe{} interface}
%    \begin{macrocode}
\@onlypreamble\@xnthm
\@onlypreamble\@ynthm
\@onlypreamble\@othm
\@onlypreamble\newtheorem
\@onlypreamble\theoremstyle
\@onlypreamble\theorembodyfont
\@onlypreamble\theoremheaderfont
%    \end{macrocode}
% \end{macro}
%
% Finally we declare the \texttt{plain} theorem style to be the default.
% \changes{v2.1b}{90/06/04}{Fixed missing `item by declaring default
%    style.}
%    \begin{macrocode}
\theoremstyle{plain}
%</package>
%    \end{macrocode}
%
% \Finale

\endinput
}{}
\ProvidesFile{thm.sty}
             [\FMithmInfo]
%    \end{macrocode}
%
% \begin{macro}{\th@margin}
%    Again this is only a variant of the theorem styles described
%    above without any new ideas.
%    \begin{macrocode}
\gdef\th@margin{\normalfont\slshape
  \def\@begintheorem##1##2{\item
       [\theorem@headerfont \llap{##2}\hskip\labelsep ##1]}%
\def\@opargbegintheorem##1##2##3{%
  \item[\theorem@headerfont \llap{##2}\hskip\labelsep ##1\ (##3)]}}
\endgroup
%</thm>
%    \end{macrocode}
% \end{macro}
%
%
%
%
% \subsection{Final Definitions}
%
% \begin{macro}{\theorempreskipamount}
% \begin{macro}{\theorempostskipamount}
%    The \textsf{skip} parameters that regulate the vertical empty space
%    before and after the {\envfont theorem} environment have
%    to be allocated as well.
% \changes{v2.0f}{89/07/20}{Unnecessary `global' removed.}
%    \begin{macrocode}
%<*package>
\newskip\theorempreskipamount
\newskip\theorempostskipamount
%    \end{macrocode}
%    Since we have used the same values for all theorem sets, we now can
%    assign them.
%    \begin{macrocode}
\global\setlength\theorempreskipamount{12pt plus 5pt minus 3pt}
\global\setlength\theorempostskipamount{8pt plus 3pt minus 1.5pt}
%    \end{macrocode}
% \end{macro}
% \end{macro}
%
%
% \begin{macro}{\@endtheorem}
%    The same holds for the macro |\@endtheorem|, which ends a
%    {\envfont theorem} environment. Since it is the same for all
%    theorem sets, it is removed from the macros |\th@|\meta{style}.
%    It simply ends the {\envfont trivlist} environment, which was
%    begun in |\@thm|.
%    \begin{macrocode}
\global\let\@endtheorem=\endtrivlist
%    \end{macrocode}
% \end{macro}
%
%
%
% \begin{macro}{\@preamblecmds}
%    All macros defined above are to be used only in the preamble.
%    Therefore, we insert them in |\@preamblecmds| which will disable
%    them at begin document. This is done by the internal \LaTeXe{}
%    command |\@onlypreamble|.
% \changes{v2.0g}{89/09/19}{`theoremheaderfont and `theorembodyfont
%    added.}
% \changes{v2.2a}{94/02/03}{Use \LaTeXe{} interface}
%    \begin{macrocode}
\@onlypreamble\@xnthm
\@onlypreamble\@ynthm
\@onlypreamble\@othm
\@onlypreamble\newtheorem
\@onlypreamble\theoremstyle
\@onlypreamble\theorembodyfont
\@onlypreamble\theoremheaderfont
%    \end{macrocode}
% \end{macro}
%
% Finally we declare the \texttt{plain} theorem style to be the default.
% \changes{v2.1b}{90/06/04}{Fixed missing `item by declaring default
%    style.}
%    \begin{macrocode}
\theoremstyle{plain}
%</package>
%    \end{macrocode}
%
% \Finale

\endinput
}{}
\ProvidesFile{thm.sty}
             [\FMithmInfo]
%    \end{macrocode}
%
% \begin{macro}{\th@margin}
%    Again this is only a variant of the theorem styles described
%    above without any new ideas.
%    \begin{macrocode}
\gdef\th@margin{\normalfont\slshape
  \def\@begintheorem##1##2{\item
       [\theorem@headerfont \llap{##2}\hskip\labelsep ##1]}%
\def\@opargbegintheorem##1##2##3{%
  \item[\theorem@headerfont \llap{##2}\hskip\labelsep ##1\ (##3)]}}
\endgroup
%</thm>
%    \end{macrocode}
% \end{macro}
%
%
%
%
% \subsection{Final Definitions}
%
% \begin{macro}{\theorempreskipamount}
% \begin{macro}{\theorempostskipamount}
%    The \textsf{skip} parameters that regulate the vertical empty space
%    before and after the {\envfont theorem} environment have
%    to be allocated as well.
% \changes{v2.0f}{89/07/20}{Unnecessary `global' removed.}
%    \begin{macrocode}
%<*package>
\newskip\theorempreskipamount
\newskip\theorempostskipamount
%    \end{macrocode}
%    Since we have used the same values for all theorem sets, we now can
%    assign them.
%    \begin{macrocode}
\global\setlength\theorempreskipamount{12pt plus 5pt minus 3pt}
\global\setlength\theorempostskipamount{8pt plus 3pt minus 1.5pt}
%    \end{macrocode}
% \end{macro}
% \end{macro}
%
%
% \begin{macro}{\@endtheorem}
%    The same holds for the macro |\@endtheorem|, which ends a
%    {\envfont theorem} environment. Since it is the same for all
%    theorem sets, it is removed from the macros |\th@|\meta{style}.
%    It simply ends the {\envfont trivlist} environment, which was
%    begun in |\@thm|.
%    \begin{macrocode}
\global\let\@endtheorem=\endtrivlist
%    \end{macrocode}
% \end{macro}
%
%
%
% \begin{macro}{\@preamblecmds}
%    All macros defined above are to be used only in the preamble.
%    Therefore, we insert them in |\@preamblecmds| which will disable
%    them at begin document. This is done by the internal \LaTeXe{}
%    command |\@onlypreamble|.
% \changes{v2.0g}{89/09/19}{`theoremheaderfont and `theorembodyfont
%    added.}
% \changes{v2.2a}{94/02/03}{Use \LaTeXe{} interface}
%    \begin{macrocode}
\@onlypreamble\@xnthm
\@onlypreamble\@ynthm
\@onlypreamble\@othm
\@onlypreamble\newtheorem
\@onlypreamble\theoremstyle
\@onlypreamble\theorembodyfont
\@onlypreamble\theoremheaderfont
%    \end{macrocode}
% \end{macro}
%
% Finally we declare the \texttt{plain} theorem style to be the default.
% \changes{v2.1b}{90/06/04}{Fixed missing `item by declaring default
%    style.}
%    \begin{macrocode}
\theoremstyle{plain}
%</package>
%    \end{macrocode}
%
% \Finale

\endinput
